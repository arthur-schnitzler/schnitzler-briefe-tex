%% latex-korrekturansicht-vorspann.tex
%% Vorspann für die Korrekturansicht.
%% Lädt die gemeinsame Datei latex-vorspann.tex mit gesetztem Schalter.

\newif\ifkorrekturansicht
\korrekturansichttrue

\input{../tex-inputs/latex-vorspann}


               \section[Hugo von Hofmannsthal und Louise Schnitzler an Arthur Schnitzler, 5. 9. 1907]{ Hugo von Hofmannsthal und Louise Schnitzler an Arthur Schnitzler,
               5. 9. 1907}\nopagebreak\mylabel{v}\rehead{ }\normalsize\beginnumbering\briefempfaengerindex{Schnitzler, Arthur@\textsc{Schnitzler, Arthur}!zzzSchnitzler, Louise@\emph{von Louise Schnitzler}!1907-09-051@{5. 9. 1907}|(be}\briefempfaengerindex{Schnitzler, Arthur@\textsc{Schnitzler, Arthur}!zzzHofmannsthal, Hugo von@\emph{von Hugo von Hofmannsthal}!1907-09-051@{5. 9. 1907}|(be} \toendnotes[C]{\smallbreak\pagebreak[2]} \Standort{CUL, Schnitzler, B 43.}
\physDesc{Bildpostkarte
\newline{}Handschrift: Bleistift, deutsche Kurrent\newline{}Versand: Stempel: »\nobreak{}\oindex{Semmering@\textbf{Semmering}, \emph{Besiedelter Ort (A.BSO)}|pwk}Semmering 1, 5. IX. 07, 6\nobreak{}«.  \newline{}Ordnung: 1) mit Bleistift von unbekannter Hand nummeriert: »\strikeout{302}« 2) mit Bleistift von unbekannter Hand  nummeriert: »284«}\buchAbdrucke{\weitereDrucke{Hugo von Hofmannsthal, Arthur Schnitzler: \emph{Briefwechsel}. Hg. Therese Nickl und Heinrich Schnitzler. Frankfurt am Main: \emph{S. Fischer} 1964, S. 230.} }\toendnotes[C]{\smallbreak}\pstart{}{\pb}\textsc{Herrn D\textsuperscript{r}}\pend{}\pstart{}\textsc{Arthur Schnitzler}\pend{}\pstart{}\textsc{\textcolor{pink}{Meran}{}\ledrightnote{\textcolor{pink}{Meran}}}\pend{}\pstart{}\textsc{\textcolor{pink}{Palast Hôtel}{}\ledrightnote{\textcolor{pink}{Palasthotel Meran}}}\pend{}{\bigskip}\pstart
           \noindent{}\centering{}\textcolor{gray}{\textbf{{\pb}\textcolor{pink}{Semmering}{}\ledrightnote{\textcolor{pink}{Semmering}}bahn. Poleruswand mit Adlitzgraben.}}\pend
           \pstart
           {\pb}5 IX.\pend
           \pstart
           Schreibe bei Ihrer \textcolor{blue}{Mama}{}\ledrightnote{→\textcolor{blue}{Louise Schnitzler}} – d. h.
               dieſe Karte nicht mein \textcolor{green}{Stück}{}\ledrightnote{→\textcolor{green}{Silvia im »Stern«}}.
               Letzteres ſchreib ich in meinem Zimmer. Es iſt ſehr ſchön. Ich finde es iſt wie von
                  \textcolor{blue}{Neſtroy}{}\ledrightnote{\textcolor{blue}{Johann Nepomuk Nestroy}}, wenn er \strikeout{viel} Schnitzler geleſen hätte und \textcolor{blue}{Goldoni}{}\ledrightnote{\textcolor{blue}{Carlo Goldoni}} copieren wollen hätte. Nun im Ernſt, es iſt viel beſſer wie \textcolor{blue}{Nestroy}{}\ledrightnote{\textcolor{blue}{Johann Nepomuk Nestroy}}{ }\textcolor{blue}{Goldoni}{}\ledrightnote{\textcolor{blue}{Carlo Goldoni}} und (natürlich) Schnitzler und ſchlechter
               wie nichts.\pend
           \pstart
           Ende Septemb. bin ich in \textcolor{pink}{Wien}{}\ledrightnote{\textcolor{pink}{Wien}}.\pend
           \pstart \spacefill\mbox{Hugo.}\pend{}\pstart
           \noindent{}{[}hs. Schnitzler:{]} Einen ſchönen Gruß und viele Küſſe von{\\}\spacefill\mbox{Mama}\pend
           \endnumbering\briefempfaengerindex{Schnitzler, Arthur@\textsc{Schnitzler, Arthur}!zzzSchnitzler, Louise@\emph{von Louise Schnitzler}!1907-09-051@{5. 9. 1907}|)be}\briefempfaengerindex{Schnitzler, Arthur@\textsc{Schnitzler, Arthur}!zzzHofmannsthal, Hugo von@\emph{von Hugo von Hofmannsthal}!1907-09-051@{5. 9. 1907}|)be}\mylabel{h}  \normalsize

\doendnotes{C}
\bigskip
\vfill

\clearpage

\footnotesize

\lohead{\textsc{register}}

% Definiere theindex-Environment komplett neu ohne reledmac
\makeatletter
\renewenvironment{theindex}{%
  \section*{\indexname}%
  \setlength{\parindent}{0pt}%
  \setlength{\parskip}{0pt plus 0.3pt}%
  \let\item\@idxitem
}{%
  \clearpage
}
\makeatother

\IfFileExists{\jobname-pw.ind}{\input{\jobname-pw.ind}}{}

\end{document}

      