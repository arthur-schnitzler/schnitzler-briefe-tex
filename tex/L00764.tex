%% latex-korrekturansicht-vorspann.tex
%% Vorspann für die Korrekturansicht.
%% Lädt die gemeinsame Datei latex-vorspann.tex mit gesetztem Schalter.

\newif\ifkorrekturansicht
\korrekturansichttrue

\input{../tex-inputs/latex-vorspann}


               \section[Max Burckhard an Arthur Schnitzler, nach dem 18. 1. 1898]{ Max Burckhard an Arthur Schnitzler, nach dem 18. 1. 1898}\nopagebreak\mylabel{v}\rehead{ }\normalsize\beginnumbering\briefempfaengerindex{Schnitzler, Arthur@\textsc{Schnitzler, Arthur}!zzzBurckhard, Max Eugen@\emph{von Max Eugen Burckhard}!1898-01-181@{nach dem 18. 1. 1898}|(be} \toendnotes[C]{\smallbreak\pagebreak[2]} \Standort{CUL, Schnitzler, B 20.}
\physDesc{Visitenkarte
\newline{}Handschrift: Bleistift, deutsche Kurrent}\toendnotes[C]{\smallbreak}\pstart
           \noindent{}\centering{}{\pb}\textcolor{gray}{\textbf{\textsc{D\textsuperscript{r.} Max Eugen
                            Burckhard}}}\pend
           \pstart
           \noindent{}\centering{}\textcolor{gray}{\textbf{\textsc{\strikeout{K. u. K. \label{K_L00764_1v}\edtext{Director}{\lemma{\textnormal{\emph{Director}}}\Cendnote{\textnormal{\textcolor{blue}{Burckhard} legte am
                                        18. 1. 1898 die Leitung des \textcolor{pink}{Burgtheaters} nieder. Die handschriftliche
                                    Streichung dürfte diese Karte also unmittelbar in zeitliche Nähe
                                    verorten.}}}\label{K_L00764_1h} des K. K. \textcolor{pink}{Hofburgtheaters}{}\ledrightnote{\textcolor{pink}{Burgtheater}}}}}}\pend
           \pstart
           \noindent{}mit herzlichſten Grüßen und bestem Dank\pend
           \endnumbering\briefempfaengerindex{Schnitzler, Arthur@\textsc{Schnitzler, Arthur}!zzzBurckhard, Max Eugen@\emph{von Max Eugen Burckhard}!1898-01-181@{nach dem 18. 1. 1898}|)be}\mylabel{h}  \normalsize

\doendnotes{C}
\bigskip
\vfill

\clearpage

\footnotesize

\lohead{\textsc{register}}

% Definiere theindex-Environment komplett neu ohne reledmac
\makeatletter
\renewenvironment{theindex}{%
  \section*{\indexname}%
  \setlength{\parindent}{0pt}%
  \setlength{\parskip}{0pt plus 0.3pt}%
  \let\item\@idxitem
}{%
  \clearpage
}
\makeatother

\IfFileExists{\jobname-pw.ind}{\input{\jobname-pw.ind}}{}

\end{document}

      