%% latex-korrekturansicht-vorspann.tex
%% Vorspann für die Korrekturansicht.
%% Lädt die gemeinsame Datei latex-vorspann.tex mit gesetztem Schalter.

\newif\ifkorrekturansicht
\korrekturansichttrue

\input{../tex-inputs/latex-vorspann}


               \section[Arthur Schnitzler an Hugo von Hofmannsthal, 11. 8. 1893]{ Arthur Schnitzler an Hugo von Hofmannsthal, 11. 8. 1893}\nopagebreak\mylabel{v}\rehead{ }\normalsize\beginnumbering\briefempfaengerindex{Hofmannsthal, Hugo von@\textsc{Hofmannsthal, Hugo von}!zzzSchnitzler, Arthur@\emph{von Arthur Schnitzler}!1893-08-112@{11. 8. 1893}|(be} \toendnotes[C]{\smallbreak\pagebreak[2]} \Standort{FDH, Hs-30885,38.}
\physDesc{Brief, 2 Blätter (Briefpapier mit Trauerrand), 5 Seiten
\newline{}Handschrift: schwarze Tinte, deutsche Kurrent\newline{}Ordnung: von Schnitzler mutmaßlich bei der Durchsicht der Korrespondenz
                                    1929 mit Bleistift datiert: »11. 8. 93« }\buchAbdrucke{\weitereDrucke{Hugo von Hofmannsthal, Arthur Schnitzler: \emph{Briefwechsel}. Hg. Therese Nickl und Heinrich Schnitzler. Frankfurt am Main: \emph{S. Fischer} 1964, S. 43–44.} }\toendnotes[C]{\smallbreak}\pstart{}{\pb}Lieber Hugo,\pend\pstart
           Ihr \textcolor{green}{Feu{[}i{]}lleton}{}\ledrightnote{→\textcolor{green}{Gabriele d’Annunzio}} über \textcolor{blue}{\textsc{Annunzio}}{}\ledrightnote{\textcolor{blue}{Gabriele D’Annunzio}} hab ich mit großer Freude geleſen; es iſt eins Ihrer ſchönſten, mit weiten
               Ausblicken. – Iſt von dem Mann was ins Deutſche überſetzt? –\pend
           \pstart
           – Denken Sie, mir iſt man endlich draufgeko{\geminationm}en, daſs ich
                  \strikeout{i}auf die ſexuellen Inſtincte der Menge ſpeculire und
                  {\pb}meine »\textcolor{green}{cyniſchen}{}\ledrightnote{→\textcolor{green}{Literatur. »Bunte Reihe.« Ein Geschichtenbuch von Moritz Goldschmidt. »Anatol« von Arthur Schnitzler}}«, »\textcolor{green}{plumpen}{}\ledrightnote{→\textcolor{green}{Literatur. »Bunte Reihe.« Ein Geschichtenbuch von Moritz Goldschmidt. »Anatol« von Arthur Schnitzler}}«
               Sachen mit verletzender Abſichtlichkeit ſchreibe – (offenbar um mittelſt meiner
               Trivialität viel Geld zu machen.) – Der Ruhm dieſer Entdeckung gebührt der \textcolor{brown}{Wiener Abendpoſt}{}\ledrightnote{\textcolor{brown}{Wiener Abendpost}}, welche im übrigen zugleich
               Geſchmack genug hat, die Leichtbeſchwingtheit Ihrer \textcolor{green}{Verſe}{}\ledrightnote{→\textcolor{green}{Einleitung}} zu loben. (Referent \textcolor{blue}{Bruno Walden}{}\ledrightnote{\textcolor{blue}{Florentine Galliny}}.) –\pend
           \pstart
           Meine Abſicht geht vorläufig dahin {\pb}Ende nächſter Woche
               ins \textcolor{pink}{Puſterthal}{}\ledrightnote{\textcolor{pink}{Pustertal}} zu reiſen, und vielleicht von dort
               per \textsc{Bic.} nach \textcolor{pink}{Wien}{}\ledrightnote{\textcolor{pink}{Wien}} zurück.
                  (\textcolor{blue}{\textsc{Salten}}{}\ledrightnote{\textcolor{blue}{Felix Salten}} iſt bereits unten.) – \textcolor{blue}{\textsc{Paul Goldma{\geminationn}}}{}\ledrightnote{\textcolor{blue}{Paul Goldmann}} will im September nach \textcolor{pink}{Salzburg}{}\ledrightnote{\textcolor{pink}{Salzburg}}
                  ko{\geminationm}en; vielleicht läßt ſich eine Zuſa{\geminationm}enkunft Ende Auguſt arrangiren?\pend
           \pstart
           Wie ſind Ihre Pläne? Schreiben Sie doch was darüber. Arbeiten Sie was? Meine kleine
                  \textcolor{green}{Novelle}{}\ledrightnote{\textcolor{green}{Die kleine Komödie}} iſt bis auf wenige Zeilen fertig. {\pb}Das hab ich Ihnen ſchon geſchrieben. – Jetzt ſchreib ich
               ab und zu ein paar Verſe an dem »allegoriſchen« \textcolor{green}{Gedicht}{}\ledrightnote{\textcolor{green}{Artifex}}; bedauere aber ſehr, nicht die ausreichende Befähigung dazu zu
               haben. –\pend
           \pstart
           Den Mut zu was größerem, das wird Sie nach alledem nicht wundern, hab ich noch nicht
               erlangt. – Unter vier Augen: das \textcolor{pink}{Volkstheater}{}\ledrightnote{\textcolor{pink}{Volkstheater}} beginnt
               mit mir \introOben{}(wegen »\textcolor{green}{Märchen}{}\ledrightnote{\textcolor{green}{Das Märchen. Schauspiel in drei Aufzügen}}«)\introOben{}
               zu unterhandeln; ich ſage Ihnen – Zuſtände!! – Weiteres darüber mündlich.\pend
           \pstart
           {\pb}– Wie gehts dem \textcolor{pink}{aegyptiſchen}{}\ledrightnote{\textcolor{pink}{Ägypten}} unanſtändigen \textcolor{green}{Stück}{}\ledrightnote{→\textcolor{green}{Alexanderzug}}? – Wenn es \uline{nur}{ }\textcolor{pink}{aegyptiſch}{}\ledrightnote{\textcolor{pink}{Ägypten}} wäre, läge es der Allgemeinheit zu fern!
               – Der Tod \textsc{\textcolor{blue}{Kafka}{}\ledrightnote{\textcolor{blue}{Eduard Michael Kafka}}}’s iſt Ihnen wohl bekannt worden? –\pend
           \pstart
           – Hören Sie was von \textsc{\textcolor{blue}{Fels}{}\ledrightnote{\textcolor{blue}{Friedrich Michael Fels}}}? – Schreibt Ihnen \textcolor{blue}{Richard}{}\ledrightnote{\textcolor{blue}{Richard Beer-Hofmann}}? –\pend
           \pstart
           Sind Sie vergnügt? –\pend
           \pstart
           Herzlich der Ihre{\\[\baselineskip]}\spacefill\mbox{Arthur}\pend
           \leftskip=0em{}\pstart
           \textcolor{pink}{Wien}{}\ledrightnote{\textcolor{pink}{Wien}}, 11. 8. 93\pend
           \pstart
           \label{T_L00251_1v}\edtext{\uline{Sie müssen \textsc{Bicycle} fahren
                     lernen!}}{\lemma{\textnormal{\emph{Sie … lernen!}}}\Cendnote{\textnormal{quer am linken Rand}}}\label{T_L00251_1h}\pend
           \endnumbering\briefempfaengerindex{Hofmannsthal, Hugo von@\textsc{Hofmannsthal, Hugo von}!zzzSchnitzler, Arthur@\emph{von Arthur Schnitzler}!1893-08-112@{11. 8. 1893}|)be}\mylabel{h}  \normalsize

\doendnotes{C}
\bigskip
\vfill

\clearpage

\footnotesize

\lohead{\textsc{register}}

% Definiere theindex-Environment komplett neu ohne reledmac
\makeatletter
\renewenvironment{theindex}{%
  \section*{\indexname}%
  \setlength{\parindent}{0pt}%
  \setlength{\parskip}{0pt plus 0.3pt}%
  \let\item\@idxitem
}{%
  \clearpage
}
\makeatother

\IfFileExists{\jobname-pw.ind}{\input{\jobname-pw.ind}}{}

\end{document}

      