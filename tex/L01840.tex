%% latex-korrekturansicht-vorspann.tex
%% Vorspann für die Korrekturansicht.
%% Lädt die gemeinsame Datei latex-vorspann.tex mit gesetztem Schalter.

\newif\ifkorrekturansicht
\korrekturansichttrue

\input{../tex-inputs/latex-vorspann}


               \section[Albert Ehrenstein an Arthur Schnitzler, 6. 5. 1909]{ Albert Ehrenstein an Arthur Schnitzler, 6. 5. 1909}\nopagebreak\mylabel{v}\rehead{ }\normalsize\beginnumbering\briefempfaengerindex{Schnitzler, Arthur@\textsc{Schnitzler, Arthur}!zzzEhrenstein, Albert@\emph{von Albert Ehrenstein}!1909-05-061@{6. 5. 1909}|(be} \toendnotes[C]{\smallbreak\pagebreak[2]} \Standort{CUL, Schnitzler, B 30.}
\physDesc{Brief, 1 Blatt, 4 Seiten
\newline{}Handschrift: schwarze Tinte, deutsche Kurrent
\newline{}Schnitzler: mit Bleistift beschriftet: »\textsc{Ehrenstein}« }\buchAbdrucke{\weitereDrucke{Albert Ehrenstein: \emph{Briefe}. Hg. Hanni Mittelmann. München: \emph{Boer} 1989, S. 29–30 (Werke, 1).} }\toendnotes[C]{\smallbreak}\pstart
           {\pb}\textsc{\textcolor{pink}{Wien, XVI. Ottakringerstr. 114}{}\ledrightnote{\textcolor{pink}{Ottakringerstraße}}}.\hfill \textsc{6. Mai. 09}.\pend
           \pstart{}\textsc{Sehr geehrter Herr Doktor!}\pend\pstart
           Wenn ich keinen Zwicker trage (und aus Eitelkeit trage ich meiſtens keinen), ſo
                    bin ich recht kurzſichtig; überdies und auch dann iſt mein Perſonengedächtnis
                    ein ziemlich mangelhaftes und geſtörtes, warum? Darüber möchte ich gerne etwas
                    näheres erfahren. Jedenfalls haben ſich meine Augen ſchon manchen Ulk mit mir
                    erlaubt, die ärgerlichſten und gröbſten Verwechslungen ſind mir zugestoßen. Die
                    anfänglich vorhanden geweſene Geneigtheit, jede Agnoszierung ohne weiteres für
                    wichtig anzuſehen, iſt infolgedeſſen einem ſo zweifelſüchtigen Mißtrauen gegen
                    alle Wahrnehmung gewichen, daß es mir nur ſehr ſelten gelingt, einen Begegnenden
                    richtig zu identifizieren oder gar ſtets davon überzeigt zu ſein. Wie ich
                    glaube, iſt mir ein derartiges Malheur ſchon einmal Ihnen gegenüber, ſehr
                    geehrter \introOben{}Herr\introOben{} Doktor, paſſiert, in einer Tramway nach
                    der \label{K_L01840_1v}\edtext{Premiere}{\lemma{\textnormal{\emph{Premiere}}}\Cendnote{\textnormal{Am 29. 12. 1906 im \textcolor{pink}{Lustspieltheater} in \textcolor{pink}{Wien}, \textcolor{blue}{Schnitzler} war nicht bei
                        der Premiere.}}}\label{K_L01840_1h} der \textcolor{blue}{Donnay}{}\ledrightnote{\textcolor{blue}{Maurice Donnay}}’ſchen
                    {\pb}\textcolor{green}{Lyſiſtrata}{}\ledrightnote{\textcolor{green}{Lysistrata}}. Ein anderesmal nach einer \label{K_L01840_2v}\edtext{Vorleſung im \textcolor{pink}{Mariahilf}{}\ledrightnote{\textcolor{pink}{VI., Mariahilf}}er Arbeiterheim}{\lemma{\textnormal{\emph{Vorleſung … Arbeiterheim}}}\Cendnote{\textnormal{Gemeint ist die Vorlesung am 16. 10. 1907 für die \emph{\textcolor{brown}{Wiener Freie Volksbühne}} im
                        sozialdemokratischen \textcolor{pink}{Verbandsheim} in der
                            \textcolor{pink}{Königseggasse 10}.}}}\label{K_L01840_2h} verſchlug mir
                    die Befangenheit jeden Gruß. Ein gewiſſer kindlicher und doch dämoniſcher Trotz
                    und Eigenſinn verbietet es, wenn man ſich von der ersten Lähmung des Willens
                    erholt hat, baldmöglichſt den Fehler gutzumachen. Nach dem Geſetz der Trägheit
                    geht man den einmal genommenen Weg verdroſſen oder ratlos weiter, und bevor man
                    ſich von der Überrumplung durch die ſelbſtverſchuldeten Ereigniſſe freigemacht
                    hat, ſagt man ſicher »Jetzt iſt ſchon alles gleichgültig.« Ich würde derartige
                    Erlebniſſe trotz ihrer Wiederkehr gewiß nicht ſo tragiſch nehmen, wenn ich nicht
                    wüßte, wie ſehr derartige Unterlaſſungsſünden dem Selbſtvernichtungstriebe
                    entſprechen, krankhaftes Benehmen und davon Betroffenen nicht gerade das Leben
                    erleichtert. Das ſchlechter werdende Gehör trägt auch nicht dazu bei, {\pb}die Lage angenehmer zu machen, verſäumte Grüße
                    ſummierten ſich mit oft wider Willen emporgefahrenen biſſigen Antworten auf
                    falsch verſtandenen Bemerkungen, und entrißen mir die wenigen Freunde. Es iſt
                    eben ſelbſt der Teilnahmsvollſte nicht immer in der Stimmung, kurzſichtigen
                    Unverſtand von Hochmut, Eigentümlichkeit und Schrullen von Überhebung zu
                    ſondern. Sollte Mittwoch, den 5. Mai um 9\textsuperscript{h} früh meinerſeits Ihnen gegenüber eine
                    Kette neuerlicher Verſtöße oder Sinnestäuſchungen vorgefallen ſein, ſo wäre es
                    mir ſehr lieb, wenn ich von allerhand quälenden Betrachtungen befreit würde.
                    Faſt ſcheint es ſo, als ſtellte ich die unmöglichſten Dinge bloß zu dem Zwecke
                    an, auch nachträglich entſchuldigen zu können. Nie tat ich das Plauſible, ſeit
                    jeher ſchon war ich mir ziemlich wehrlos ausgeſetzt, und wenn es irgend anginge,
                    zöge ich {\pb}wahrhaftig mit größtem Vergnügen aus mir
                    aus.\pend
           \pstart
           Hochachtungsvoll{\\[\baselineskip]}Ihr ergebenſter{\\[\baselineskip]}\spacefill\mbox{Albert Ehrenstein.}\pend
           \leftskip=0em{}\endnumbering\briefempfaengerindex{Schnitzler, Arthur@\textsc{Schnitzler, Arthur}!zzzEhrenstein, Albert@\emph{von Albert Ehrenstein}!1909-05-061@{6. 5. 1909}|)be}\mylabel{h}  \normalsize

\doendnotes{C}
\bigskip
\vfill

\clearpage

\footnotesize

\lohead{\textsc{register}}

% Definiere theindex-Environment komplett neu ohne reledmac
\makeatletter
\renewenvironment{theindex}{%
  \section*{\indexname}%
  \setlength{\parindent}{0pt}%
  \setlength{\parskip}{0pt plus 0.3pt}%
  \let\item\@idxitem
}{%
  \clearpage
}
\makeatother

\IfFileExists{\jobname-pw.ind}{\input{\jobname-pw.ind}}{}

\end{document}

      