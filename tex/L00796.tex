%% latex-korrekturansicht-vorspann.tex
%% Vorspann für die Korrekturansicht.
%% Lädt die gemeinsame Datei latex-vorspann.tex mit gesetztem Schalter.

\newif\ifkorrekturansicht
\korrekturansichttrue

\input{../tex-inputs/latex-vorspann}


               \section[Hugo von Hofmannsthal an Arthur Schnitzler, {[}18. 5. 1898{]}]{ Hugo von Hofmannsthal an Arthur Schnitzler, {[}18. 5. 1898{]}}\nopagebreak\mylabel{v}\rehead{ }\normalsize\beginnumbering\briefempfaengerindex{Schnitzler, Arthur@\textsc{Schnitzler, Arthur}!zzzHofmannsthal, Hugo von@\emph{von Hugo von Hofmannsthal}!1898-05-181@{{[}18. 5. 1898{]}}|(be} \toendnotes[C]{\smallbreak\pagebreak[2]} \Standort{CUL, Schnitzler, B 43.}
\physDesc{Brief, 1 Blatt, 2 Seiten
\newline{}Handschrift: Bleistift, deutsche Kurrent
\newline{}Schnitzler: mit Bleistift datiert: »Mai 98« \newline{}Ordnung: 1) mit Bleistift von unbekannter Hand nummeriert:
                                 »114« 2) mit Bleistift von unbekannter Hand nummeriert:
                                    »117«}\buchAbdrucke{\weitereDrucke{Hugo von Hofmannsthal, Arthur Schnitzler: \emph{Briefwechsel}. Hg. Therese Nickl und Heinrich Schnitzler. Frankfurt am Main: \emph{S. Fischer} 1964, S. 101–102.} }\toendnotes[C]{\smallbreak}\pstart{}{\pb}lieber Arthur!\pend\pstart
           ich hätt Sie ſo gern geſehen.\pend
           \pstart
           Ich hab ſchrecklich wenig Zeit wegen der Prüfung. \label{K_L00796_1v}\edtext{Morgen}{\lemma{\textnormal{\emph{Morgen}}}\Cendnote{\textnormal{Dieser
                  Hinweis lässt den Brief am Mittwoch nach der Premiere von \emph{\textcolor{green}{Madonna Dianora}} zeitlich einordnen.}}}\label{K_L00796_1h}{ }Do{\geminationn}erstag abend werd ich beſtimmt um
                  ¾ 11 im \textcolor{pink}{Arkadencafé}{}\ledrightnote{\textcolor{pink}{Café Arkaden}}{ }ſein, ich hoff Sie ſind dort. Über die \label{K_L00796_2v}\edtext{\textcolor{green}{Première}{}\ledrightnote{→\textcolor{green}{Die Frau im Fenster}}}{\lemma{\textnormal{\emph{Première}}}\Cendnote{\textnormal{Als \emph{\textcolor{green}{Madonna
                     Dianora}} hatte \textcolor{blue}{Hofmannsthal}s \emph{\textcolor{green}{Die Frau im Fenster}} am 15. 5. 1898
                  als öffentliche Matinée der \textcolor{pink}{Berlin}er \emph{\textcolor{brown}{Freien Bühne}} am \textcolor{pink}{Deutschen Theater} die Uraufführung erlebt.}}}\label{K_L00796_2h} iſt natürlich nur
               mündlich zu reden.\pend
           \pstart
           Es iſt mir ein biſſel zuwider, daſs die \textcolor{pink}{W\textsuperscript{r}}{}\ledrightnote{\textcolor{pink}{Wien}} Zeitungen gar keine Telegra{\geminationm}e haben. \textcolor{blue}{Schiff}{}\ledrightnote{\textcolor{blue}{Emil Schiff}} wird zudem nicht {\pb}ſehr freundlich ſein.\pend
           \pstart
           Könnte nicht \textcolor{blue}{Salten}{}\ledrightnote{\textcolor{blue}{Felix Salten}} etwas bringen, etwa einen
                  \label{K_L00796_3v}\edtext{Auszug}{\lemma{\textnormal{\emph{Auszug}}}\Cendnote{\textnormal{Im \emph{\textcolor{brown}{Berliner Börsen-Courier}}
                  erschien keine Besprechung, sehr wohl aber im \emph{\textcolor{green}{Berliner Tageblatt}}: \textcolor{blue}{F. E.} (=\textcolor{blue}{Fritz
                        Engel}): \emph{\textcolor{green}{»Freie Bühne«}}. In: \emph{\textcolor{green}{Berliner Tageblatt}}, Jg. 27, Nr. 245,
                     Montags-Ausgabe, 16. 5. 1898, S. 2.}}}\label{K_L00796_3h} aus dem \textcolor{brown}{\textsc{Börsencourier}}{}\ledrightnote{\textcolor{brown}{Berliner Börsen-Courier}} oder ſonſt woher, ich würde ihm die Ausſchnitte natürlich auch ſchicken.
               Vielleicht fragen Sie ihn telephoniſch oder ſonſt.\pend
           \pstart
           Herzlich Ihr{\\[\baselineskip]}\spacefill\mbox{Hugo}\pend
           \leftskip=0em{}\endnumbering\briefempfaengerindex{Schnitzler, Arthur@\textsc{Schnitzler, Arthur}!zzzHofmannsthal, Hugo von@\emph{von Hugo von Hofmannsthal}!1898-05-181@{{[}18. 5. 1898{]}}|)be}\mylabel{h}  \normalsize

\doendnotes{C}
\bigskip
\vfill

\clearpage

\footnotesize

\lohead{\textsc{register}}

% Definiere theindex-Environment komplett neu ohne reledmac
\makeatletter
\renewenvironment{theindex}{%
  \section*{\indexname}%
  \setlength{\parindent}{0pt}%
  \setlength{\parskip}{0pt plus 0.3pt}%
  \let\item\@idxitem
}{%
  \clearpage
}
\makeatother

\IfFileExists{\jobname-pw.ind}{\input{\jobname-pw.ind}}{}

\end{document}

      