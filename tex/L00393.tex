%% latex-korrekturansicht-vorspann.tex
%% Vorspann für die Korrekturansicht.
%% Lädt die gemeinsame Datei latex-vorspann.tex mit gesetztem Schalter.

\newif\ifkorrekturansicht
\korrekturansichttrue

\input{../tex-inputs/latex-vorspann}


               \section[Friedrich M. Fels an Arthur Schnitzler, 26. 10. 1894]{ Friedrich M. Fels an Arthur Schnitzler, 26. 10. 1894}\nopagebreak\mylabel{v}\rehead{ }\normalsize\beginnumbering\briefempfaengerindex{Schnitzler, Arthur@\textsc{Schnitzler, Arthur}!zzzFels, Friedrich Michael@\emph{von Friedrich Michael Fels}!1894-10-262@{26. 10. 1894}|(be} \toendnotes[C]{\smallbreak\pagebreak[2]} \Standort{DLA, A:Schnitzler, HS.NZ85.1.2956.}
\physDesc{Brief, 1 Blatt, 2 Seiten
\newline{}Handschrift: schwarze Tinte, lateinische Kurrent
\newline{}Schnitzler: mit Bleistift nummeriert: »17« }\toendnotes[C]{\smallbreak}\pstart
           \raggedleft{}{\pb}\textcolor{pink}{Wien}{}\ledrightnote{\textcolor{pink}{Wien}}{ }26. Okt. 94\pend
           \pstart{}Lieber Dr Schnitzler!\pend\pstart
           Danke für Ihre frdl. Bemühungen wegen \textcolor{brown}{Extrapost}{}\ledrightnote{\textcolor{brown}{Extrapost}};
                    sie sind gegenstandslos geworden. Ich soeben, mit Empfehlung von Dr. \textcolor{blue}{Brüll-Neuda}{}\ledrightnote{\textcolor{blue}{Wilhelm Brüll-Neuda}}, bei dem Besitzer, Konsul \textcolor{blue}{Thalberg}{}\ledrightnote{\textcolor{blue}{Josef Thalberg}}, der mir sagte, mit Theater- und
                    Kunstreferat sei er versorgt, dagegen möge ich ihm Feuilletons geben: er habe
                    gestern den \label{K_L00393_1v}\edtext{\textcolor{green}{\textcolor{blue}{Nietzsche}{}\ledrightnote{\textcolor{blue}{Friedrich Nietzsche}}artikel}{}\ledrightnote{→\textcolor{green}{Friedrich Nietzsche}}}{\lemma{\textnormal{\emph{Nietzscheartikel}}}\Cendnote{\textnormal{\textcolor{blue}{Friedr. M. Fels}: \emph{\textcolor{green}{Friedrich Nietzsche}}. In: \emph{\textcolor{green}{Wiener Allgemeine Zeitung}}, Nr. 4988,
                                26. 10. 1894, S. 2–3.}}}\label{K_L00393_1h} in der \textcolor{green}{Allg.}{}\ledrightnote{\textcolor{green}{Wiener Allgemeine Zeitung}} gelesen.\pend
           \pstart
           Das Folgende bitte ich geheim zu halten: Dr. \textcolor{blue}{Ludassy}{}\ledrightnote{\textcolor{blue}{Julius von Gans-Ludassy}} hat vor ein paar Tagen den \textcolor{blue}{Kraus}{}\ledrightnote{\textcolor{blue}{Karl Kraus}} ko{\geminationm}en laſsen; er möge versuchen,
                    Theaterreferate zu schreiben; er, \textcolor{blue}{Ludassy}{}\ledrightnote{\textcolor{blue}{Julius von Gans-Ludassy}},
                    werde suchen, sie unterzubringen, nachdem er mit \textcolor{blue}{Glücksma{\geminationn}}{}\ledrightnote{\textcolor{blue}{Heinrich Glücksmann}}s Berichten nicht zufrieden sei. So steht also die Sache diesmal so: ich
                    bin nicht etwa, wie schon mehrmals zu spät geko{\geminationm}en,
                    sondern einfach übergangen worden wegen – \textcolor{blue}{Kraus}{}\ledrightnote{\textcolor{blue}{Karl Kraus}}, den Sie zwar schätzen, der aber nichts weiſs und nichts ka{\geminationn}.\pend
           \pstart
           {\pb}An sich geht mir die Sache nicht nahe;
                    dazu schätze ich mich viel zu sehr und weiſs, daſs, wer \textcolor{blue}{Kraus}{}\ledrightnote{\textcolor{blue}{Karl Kraus}} mir vorzieht, um seinen Geschmack nicht zu beneiden
                    ist; auch \textcolor{blue}{Neuma{\geminationn}-Hofer}{}\ledrightnote{\textcolor{blue}{Gilbert Otto Neumann-Hofer}} hat den \introOben{}\textcolor{blue}{Kraus}{}\ledrightnote{\textcolor{blue}{Karl Kraus}}\introOben{} ja wegen »Unwiſsenheit, die durch einen schneidigen Ton allein nicht gut
                    zu machen sei«, hinausgeschmiſsen. Aber daſs ich wieder einmal kein ständiges
                    Referat beko{\geminationm}en habe, das schmerzt mich, we{\geminationn} ich bedenke, daſs nun wieder mehr Aussicht für
                    mich vorhanden ist, das nicht zu erreichen, was ich anstrebe. Mögen also die
                    Dinge ihren Lauf nehmen: ich hadere mit niemanden.\pend
           \pstart
           Herzlichen Gruſs{\\[\baselineskip]}von Ihrem \spacefill\mbox{Fels}\pend
           \leftskip=0em{}\endnumbering\briefempfaengerindex{Schnitzler, Arthur@\textsc{Schnitzler, Arthur}!zzzFels, Friedrich Michael@\emph{von Friedrich Michael Fels}!1894-10-262@{26. 10. 1894}|)be}\mylabel{h}  \normalsize

\doendnotes{C}
\bigskip
\vfill

\clearpage

\footnotesize

\lohead{\textsc{register}}

% Definiere theindex-Environment komplett neu ohne reledmac
\makeatletter
\renewenvironment{theindex}{%
  \section*{\indexname}%
  \setlength{\parindent}{0pt}%
  \setlength{\parskip}{0pt plus 0.3pt}%
  \let\item\@idxitem
}{%
  \clearpage
}
\makeatother

\IfFileExists{\jobname-pw.ind}{\input{\jobname-pw.ind}}{}

\end{document}

      