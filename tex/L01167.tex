%% latex-korrekturansicht-vorspann.tex
%% Vorspann für die Korrekturansicht.
%% Lädt die gemeinsame Datei latex-vorspann.tex mit gesetztem Schalter.

\newif\ifkorrekturansicht
\korrekturansichttrue

\input{../tex-inputs/latex-vorspann}


               \section[Arthur Schnitzler an Richard Beer-Hofmann, 6. 9. 1901]{ Arthur Schnitzler an Richard Beer-Hofmann, 6. 9. 1901}\nopagebreak\mylabel{v}\rehead{ }\normalsize\beginnumbering\briefempfaengerindex{Beer-Hofmann, Richard@\textsc{Beer-Hofmann, Richard}!zzzSchnitzler, Arthur@\emph{von Arthur Schnitzler}!1901-09-061@{6. 9. 1901}|(be} \toendnotes[C]{\smallbreak\pagebreak[2]} \Standort{YCGL, MSS 31.}
\physDesc{Postkarte
\newline{}Handschrift: schwarze Tinte, deutsche Kurrent\newline{}Versand: 1) Stempel: »\nobreak{}\oindex{IX., Alsergrund@\textbf{IX., Alsergrund}, \emph{Bezirk (A.BZK)}|pwk}Wien 9/1 66, 6. 9. \textcolor{gray}{0}1\nobreak{}«.  2) Stempel: »\nobreak{}\oindex{Poertschach@\textbf{Pörtschach}, \emph{https://www.geonames.org/ontologyP.PPL}|pwk}\textcolor{gray}{Pörtsc}hach am See , 7 \textcolor{gray}{9} 0{[}0{]}1\nobreak{}«. \newline{}Ordnung: mit Bleistift von unbekannter Hand datiert: »6. 9.« }\toendnotes[C]{\smallbreak}\pstart{}{\pb}Herrn \textsc{Dr. Rich. Beer-Hofmann
                  }\pend{}\pstart{}\textsc{\textcolor{pink}{Pörtschach}{}\ledrightnote{\textcolor{pink}{Pörtschach}}}\pend{}\pstart{}\textsc{\textcolor{pink}{Villa Arnstein}{}\ledrightnote{\textcolor{pink}{Villa Arnstein}}.}\pend{}{\bigskip}\pstart
           \noindent{}{\pb}lieber Richard, heute nur die kurze Frage, ob Sie in den \textcolor{brown}{Club}{}\ledrightnote{→\textcolor{brown}{Wiener Schachclub}} eintreten werden? hat wohl
               nicht mehr viel Sinn. Ich bin nicht ſehr dafür. Wie gehts? Ich habe dictirt.\pend
           \pstart
           Herzlichſt Ihr{\\[\baselineskip]}\spacefill\mbox{Arthur}\pend
           \leftskip=0em{}\pstart
           6. 9. 901.\pend
           \endnumbering\briefempfaengerindex{Beer-Hofmann, Richard@\textsc{Beer-Hofmann, Richard}!zzzSchnitzler, Arthur@\emph{von Arthur Schnitzler}!1901-09-061@{6. 9. 1901}|)be}\mylabel{h}  \normalsize

\doendnotes{C}
\bigskip
\vfill

\clearpage

\footnotesize

\lohead{\textsc{register}}

% Definiere theindex-Environment komplett neu ohne reledmac
\makeatletter
\renewenvironment{theindex}{%
  \section*{\indexname}%
  \setlength{\parindent}{0pt}%
  \setlength{\parskip}{0pt plus 0.3pt}%
  \let\item\@idxitem
}{%
  \clearpage
}
\makeatother

\IfFileExists{\jobname-pw.ind}{\input{\jobname-pw.ind}}{}

\end{document}

      