%% latex-korrekturansicht-vorspann.tex
%% Vorspann für die Korrekturansicht.
%% Lädt die gemeinsame Datei latex-vorspann.tex mit gesetztem Schalter.

\newif\ifkorrekturansicht
\korrekturansichttrue

\input{../tex-inputs/latex-vorspann}


               \section[Friedrich M. Fels an Arthur Schnitzler, 1{[}7{]}. 2. 1893]{ Friedrich M. Fels an Arthur Schnitzler, 1{[}7{]}. 2. 1893}\nopagebreak\mylabel{v}\rehead{ }\normalsize\beginnumbering\briefempfaengerindex{Schnitzler, Arthur@\textsc{Schnitzler, Arthur}!zzzFels, Friedrich Michael@\emph{von Friedrich Michael Fels}!1893-02-171@{1{[}7{]}. 2. 1893}|(be} \toendnotes[C]{\smallbreak\pagebreak[2]} \Standort{DLA, A:Schnitzler, HS.NZ85.1.2956.}
\physDesc{Brief, 1 Blatt, 2 Seiten
\newline{}Handschrift: schwarze Tinte, lateinische Kurrent
\newline{}Schnitzler: mit Bleistift nummeriert: »9.« und unterhalb der
            Datumsangabe klein »17« vermerkt }\toendnotes[C]{\smallbreak}\pstart
           \raggedleft{}{\pb}\textcolor{pink}{Meran-Obermais, Hotel Erzherz. Rainer}{}\ledrightnote{\textcolor{pink}{Erzherzog Rainer}}{\\}18. II. 1893\pend
           \pstart{}Lieber Doktor!\pend\pstart
           Zu meinem gesterigen Brief trage ich noch einiges nach, was ich dort vergeſsen
                    habe.\pend
           \pstart
           Ihre Medizin, die \textcolor{blue}{\uline{Schreiber}}{}\ledrightnote{\textcolor{blue}{Joseph Schreiber}} für sehr gut erklärt, nehme ich weiter; später soll da{\geminationn} ein Eisenpräparat folgen.\pend
           \pstart
           Hier im Hotel habe ich einen Beka{\geminationn}ten aus \textcolor{pink}{Wien}{}\ledrightnote{\textcolor{pink}{Wien}} getroffen, den Sie auch ke{\geminationn}en, den Schwager von \textcolor{blue}{Moriz Rosenthal}{}\ledrightnote{\textcolor{blue}{Moritz Rosenthal}}, Dr. med. \textcolor{blue}{Schrager}{}\ledrightnote{\textcolor{blue}{Sigmund Schraga}}. Er kam hierher, sich von einer Lungenentzündung zu erholen,
                    ist schon zwei Monate hier und bleibt bis Ende Februar. Auſserdem
                    verkehre ich mit dem \textcolor{blue}{Erzieher}{}\ledrightnote{→\textcolor{blue}{?? [Erzieher von Max von Fürstenberg]}} des \textcolor{blue}{Erbprinzen von Fürstenberg}{}\ledrightnote{→\textcolor{blue}{Maximilian Egon von Fürstenberg}}, einem Philologen, der kürzlich sein Examen
                    gemacht hat und mich durch Gestalt, Benehmen usw sehr an meine \textcolor{pink}{München}{}\ledrightnote{\textcolor{pink}{München}}er Studierzeit eri{\geminationn}ert. Übrigens ist er ein wütender Naturalist.\pend
           \pstart
           Am Tag, da ich hier ankam, als wir mit dem Bu{\geminationm}elzug
                    von \textcolor{pink}{Bozen}{}\ledrightnote{\textcolor{pink}{Bozen}} herüber fuhren, hatte es 28° in der
                        So{\geminationn}e; gestern ebenso. Sonst circa 24°. {\pb}Trotzdem ka{\geminationn} ich es
                    absolut zu keinem Gefühl der Wärme bringen. Ich trage wollene Unterkleider,
                    warme Oberkleider, Mantel, Plaid – und mir ist, we{\geminationn}
                    ich mir die So{\geminationn}e direkt in den Magen scheinen
                    laſse, als hätte es 14°.\pend
           \pstart
           Sie wiſsen, daſs ich angeschwollene Füſse habe, die auch schmerzen. Ich dachte
                        i{\geminationm}er, es sei vom vielen Gehen; aber \textcolor{blue}{Schreiber}{}\ledrightnote{\textcolor{blue}{Joseph Schreiber}}{ }ſagt: Anämie! alles Anämie!\pend
           \pstart
           Herzl. {\\[\baselineskip]}\spacefill\mbox{Fels}\pend
           \leftskip=0em{}\endnumbering\briefempfaengerindex{Schnitzler, Arthur@\textsc{Schnitzler, Arthur}!zzzFels, Friedrich Michael@\emph{von Friedrich Michael Fels}!1893-02-171@{1{[}7{]}. 2. 1893}|)be}\mylabel{h}  \normalsize

\doendnotes{C}
\bigskip
\vfill

\clearpage

\footnotesize

\lohead{\textsc{register}}

% Definiere theindex-Environment komplett neu ohne reledmac
\makeatletter
\renewenvironment{theindex}{%
  \section*{\indexname}%
  \setlength{\parindent}{0pt}%
  \setlength{\parskip}{0pt plus 0.3pt}%
  \let\item\@idxitem
}{%
  \clearpage
}
\makeatother

\IfFileExists{\jobname-pw.ind}{\input{\jobname-pw.ind}}{}

\end{document}

      