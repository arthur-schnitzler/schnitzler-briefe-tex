%% latex-korrekturansicht-vorspann.tex
%% Vorspann für die Korrekturansicht.
%% Lädt die gemeinsame Datei latex-vorspann.tex mit gesetztem Schalter.

\newif\ifkorrekturansicht
\korrekturansichttrue

\input{../tex-inputs/latex-vorspann}


               \section[Joseph Victor Widmann an Arthur Schnitzler, 28. 5. 1900]{ Joseph Victor Widmann an Arthur Schnitzler, 28. 5. 1900}\nopagebreak\mylabel{v}\rehead{ }\normalsize\beginnumbering\briefempfaengerindex{Schnitzler, Arthur@\textsc{Schnitzler, Arthur}!zzzWidmann, Joseph Victor@\emph{von Joseph Victor Widmann}!1900-05-283@{28. 5. 1900}|(be} \toendnotes[C]{\smallbreak\pagebreak[2]} \Standort{TMW, HS Schn 4/104/1.}
\physDesc{Brief, 1 Blatt, 3 Seiten
\newline{}Handschrift: schwarze Tinte, deutsche Kurrent
\newline{}Schnitzler: 1) mit Bleistift beschriftet: »Widmann« 2) mit rotem Buntstift eine
            Unterstreichung}\toendnotes[C]{\smallbreak}\pstart
           \raggedleft{}{\pb}\textsc{\textcolor{pink}{Bern}{}\ledrightnote{\textcolor{pink}{Bern}}, 28. Mai 1900}.\pend
           \pstart{}Verehrter Herr!\pend\pstart
           Erſt geſtern habe ich über allerlei Rezenſionsbüchervolk Ihren »\textcolor{green}{Reigen}{}\ledrightnote{\textcolor{green}{Reigen. Zehn Dialoge}}« und in dem kleinen Buche die große
                    Liebenswürdigkeit entdeckt, die in einer ſo auszeichnenden perſönlichen {\pb}Sendung und Widmung eines als \textsc{Manuscript} gedruckten Werkes liegt.\pend
           \pstart
           Und wie gut ich mich dann nachher mit dem aus ſo echter Menſchenkenntniß
                    geſchöpften, feinen Buche unterhalten habe, das wird Ihnen der \textcolor{green}{Bewunderer}{}\ledrightnote{→\textcolor{green}{Kunst und Litteratur}} Ihrer \textsc{\textcolor{green}{Anatole}{}\ledrightnote{\textcolor{green}{Anatol}}}-Dialoge nicht erſt zu verſichern brauchen.\pend
           \pstart
           Ich beglückwünſche Sie zu dem poetiſchen Einfall eines solchen Venusreigens, bei
                    dem der komiſche Plumpsack, den wir alle ke{\geminationn}en, von
                    einer Hand in die andere gleitet. Wir ſind da wieder bei der freien Kunſt
                    angelangt, wie wir ſie aus fröhlichen Bildern des alten \textcolor{pink}{Pompeji}{}\ledrightnote{\textcolor{pink}{Pompei}} kennen. Und wie Ihr \textcolor{green}{Soldat}{}\ledrightnote{→\textcolor{green}{Reigen. Zehn Dialoge}} zum \textcolor{green}{Stubenmädchen}{}\ledrightnote{→\textcolor{green}{Reigen. Zehn Dialoge}} dürfen Sie in dieſen
                    manchmal {\pb}etwas dunkeln Zeiten zu Ihrer Mühe
                    ſagen: »Gott ſei Dank! Mir sein mir!«\pend
           \pstart
           Seien Sie alſo ſchönſtens bedankt für Ihr \textcolor{green}{Buch}{}\ledrightnote{→\textcolor{green}{Reigen. Zehn Dialoge}} u. für die Ehre, die Sie mir mit der Zuſendung
                    erwieſen haben.\pend
           \pstart
           In herzlicher Verehrung{\\[\baselineskip]}Ihr{\\[\baselineskip]}\spacefill\mbox{J. V. Widmann}\pend
           \leftskip=0em{}\endnumbering\briefempfaengerindex{Schnitzler, Arthur@\textsc{Schnitzler, Arthur}!zzzWidmann, Joseph Victor@\emph{von Joseph Victor Widmann}!1900-05-283@{28. 5. 1900}|)be}\mylabel{h}  \normalsize

\doendnotes{C}
\bigskip
\vfill

\clearpage

\footnotesize

\lohead{\textsc{register}}

% Definiere theindex-Environment komplett neu ohne reledmac
\makeatletter
\renewenvironment{theindex}{%
  \section*{\indexname}%
  \setlength{\parindent}{0pt}%
  \setlength{\parskip}{0pt plus 0.3pt}%
  \let\item\@idxitem
}{%
  \clearpage
}
\makeatother

\IfFileExists{\jobname-pw.ind}{\input{\jobname-pw.ind}}{}

\end{document}

      