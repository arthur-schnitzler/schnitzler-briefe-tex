%% latex-korrekturansicht-vorspann.tex
%% Vorspann für die Korrekturansicht.
%% Lädt die gemeinsame Datei latex-vorspann.tex mit gesetztem Schalter.

\newif\ifkorrekturansicht
\korrekturansichttrue

\input{../tex-inputs/latex-vorspann}


               \section[Wilhelm Bölsche an Arthur Schnitzler, 1. 7. 1893]{ Wilhelm Bölsche an Arthur Schnitzler, 1. 7. 1893}\nopagebreak\mylabel{v}\rehead{ }\normalsize\beginnumbering\briefempfaengerindex{Schnitzler, Arthur@\textsc{Schnitzler, Arthur}!zzzBoelsche, Wilhelm@\emph{von Wilhelm Bölsche}!1893-07-011@{1. 7. 1893}|(be} \toendnotes[C]{\smallbreak\pagebreak[2]} \Standort{DLA, A:Schnitzler, HS.NZ85.1.2577,8.}
\physDesc{Brief, 1 Blatt, 1 Seite
\newline{}Handschrift: schwarze Tinte, deutsche Kurrent
\newline{}Schnitzler: mit rotem Buntstift nummeriert: »9« }\buchAbdrucke{\weitereDrucke{Wilhelm Bölsche: \emph{Briefwechsel. Mit Autoren der Freien Bühne}. Hg. Gerd-Hermann Susen. Berlin: \emph{Weidler} 2010, S. 690 (Werke und Briefe. Wissenschaftliche Ausgabe, Briefe I).} }\toendnotes[C]{\smallbreak}\pstart
           \noindent{}\raggedleft{}{\pb}\textcolor{gray}{\textbf{\textit{Wilhelm Bölsche}}}\pend
           \pstart
           \noindent{}\raggedleft{}\textcolor{gray}{\textbf{\textit{\textcolor{pink}{Friedrichshagen}{}\ledrightnote{\textcolor{pink}{Friedrichshagen}}}}}\pend
           \pstart
           \raggedleft{}1. VII. 93.\pend
           \pstart\center{}Hochgeehrter Herr Dr.!\pend\pstart
           Ihre erſte, frühere Anfrage muß, zu meinem lebhaften Bedauern, wohl von mir
                    überſehen worden ſein. Auf Ihre neuere kann ich jetzt definitiv antworten, daß
                    in dieſem Sommer eine Möglichkeit, \substVorne{}\textsuperscript{für die}{\allowbreak}\substDazwischen{}in der\substHinten{}{ }\textcolor{green}{Fr. B.}{}\ledrightnote{\textcolor{green}{Freie Bühne für den Entwickelungskampf der Zeit}} noch ein Drama zu veröffentlichen,
                    leider nicht beſteht. \textcolor{blue}{Rosmer}{}\ledrightnote{\textcolor{blue}{Elsa Bernstein}}’s »\textcolor{green}{Dämmerung}{}\ledrightnote{\textcolor{green}{Dämmerung}}« füllt noch Juli und
                        Auguſt, dann kommt \label{K_L00229_1v}\edtext{\textcolor{blue}{Halbe}{}\ledrightnote{\textcolor{blue}{Max Halbe}}’s neues \textcolor{green}{Stück}{}\ledrightnote{→\textcolor{green}{Der Amerikafahrer}}}{\lemma{\textnormal{\emph{Halbe’s neues Stück}}}\Cendnote{\textnormal{\emph{\textcolor{green}{Der Amerikafahrer}} erschien nicht in der
                            \emph{\textcolor{green}{Freien Bühne}}.}}}\label{K_L00229_1h}. Zwei
                    Theaterſtücke nebeneinander aber geht nicht gut!\pend
           \pstart
           Mit vorzüglichſter Hochachtung und der nochmaligen Bitte, Verzögerungen nicht als
                    Wertungen perſönlicher Art aufzufaſſen\pend
           \pstart Ihr \spacefill\mbox{W. Bölsche}\pend{}\endnumbering\briefempfaengerindex{Schnitzler, Arthur@\textsc{Schnitzler, Arthur}!zzzBoelsche, Wilhelm@\emph{von Wilhelm Bölsche}!1893-07-011@{1. 7. 1893}|)be}\mylabel{h}  \normalsize

\doendnotes{C}
\bigskip
\vfill

\clearpage

\footnotesize

\lohead{\textsc{register}}

% Definiere theindex-Environment komplett neu ohne reledmac
\makeatletter
\renewenvironment{theindex}{%
  \section*{\indexname}%
  \setlength{\parindent}{0pt}%
  \setlength{\parskip}{0pt plus 0.3pt}%
  \let\item\@idxitem
}{%
  \clearpage
}
\makeatother

\IfFileExists{\jobname-pw.ind}{\input{\jobname-pw.ind}}{}

\end{document}

      