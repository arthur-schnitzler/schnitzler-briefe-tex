%% latex-korrekturansicht-vorspann.tex
%% Vorspann für die Korrekturansicht.
%% Lädt die gemeinsame Datei latex-vorspann.tex mit gesetztem Schalter.

\newif\ifkorrekturansicht
\korrekturansichttrue

\input{../tex-inputs/latex-vorspann}


               \section[Arthur Schnitzler an Hugo von Hofmannsthal, 9. 11. 1892]{ Arthur Schnitzler an Hugo von Hofmannsthal, 9. 11. 1892}\nopagebreak\mylabel{v}\rehead{ }\normalsize\beginnumbering\briefempfaengerindex{Hofmannsthal, Hugo von@\textsc{Hofmannsthal, Hugo von}!zzzSchnitzler, Arthur@\emph{von Arthur Schnitzler}!1892-11-091@{9. 11. 1892}|(be} \toendnotes[C]{\smallbreak\pagebreak[2]} \Standort{FDH, Hs-30885,26.}
\physDesc{Brief, 1 Blatt, 2 Seiten
\newline{}Handschrift: schwarze Tinte, deutsche Kurrent\newline{}Ordnung: auf der ersten Seite von
                                  Schnitzler mutmaßlich bei der Durchsicht der Korrespondenz 1929 mit Bleistift datiert: »9/11 92« }\buchAbdrucke{\weitereDrucke{1) Hugo von Hofmannsthal, Arthur Schnitzler: \emph{Briefwechsel}. Hg. Therese Nickl und Heinrich Schnitzler. Frankfurt am Main: \emph{S. Fischer} 1964, S. 30–31.} \weitereDrucke{2) Hermann Bahr, Arthur Schnitzler: \emph{Briefwechsel, Aufzeichnungen, Dokumente
                                (1891–1931)}. Hg. Kurt Ifkovits und Martin Anton Müller. Göttingen: \emph{Wallstein} 2018.} }\pstart{}{\pb}Liebſter Hugo,\pend\pstart
           zu \textcolor{green}{\textsc{Musotte}}{}\ledrightnote{\textcolor{green}{Musotte}} geh ich beinahe ſicher. –\pend
           \pstart
           Wir ſoupiren alſo miteinander. –\pend
           \pstart
           Rendezvous einfach im Parterre Foyer. –\pend
           \pstart
           Herrn von \textcolor{blue}{\textsc{Ehrhardt}}{}\ledrightnote{\textcolor{blue}{Robert Ehrhart von Ehrhartstein}} hab ich alles ausgerichtet. – Wiſſen Sie ſchon? Dienſtag \textcolor{gray}{{\kaufmannsund}}{ }Samſtag{ }\textcolor{pink}{\textsc{Cafe Pfob}}{}\ledrightnote{\textcolor{pink}{Café Pfob}}. – Die andern Abende \textcolor{pink}{\textsc{Café Union}}{}\ledrightnote{\textcolor{pink}{Café Union}} – \introOben{}lies \textsc{\uline{Union}}\introOben{} (\textcolor{pink}{\textsc{Grillparzerstraße}}{}\ledrightnote{\textcolor{pink}{Grillparzerstraße}}.) –\pend
           \pstart
           {\pb}Hat Ihnen \textcolor{blue}{Bölſche}{}\ledrightnote{\textcolor{blue}{Wilhelm Bölsche}} geantwortet? –\pend
           \pstart
           Was treiben Sie überhaupt? –\pend
           \pstart
           Eigentlich habe ich gehofft, Sie heuer öfters zu ſehen. Ich arbeite; bin aber
                    leider ſehr talentlos.\pend
           \pstart
           Herzlichſt der Ihre{\\[\baselineskip]}\spacefill\mbox{Arthur}\pend
           \leftskip=0em{}\pstart
           9/XI. 92\pend
           \pstart
           Grüßen Sie \textcolor{blue}{Bahr}{}\ledrightnote{\textcolor{blue}{Hermann Bahr}}!\pend
           \endnumbering\briefempfaengerindex{Hofmannsthal, Hugo von@\textsc{Hofmannsthal, Hugo von}!zzzSchnitzler, Arthur@\emph{von Arthur Schnitzler}!1892-11-091@{9. 11. 1892}|)be}\mylabel{h}  \normalsize

\doendnotes{C}
\bigskip
\vfill

\clearpage

\footnotesize

\lohead{\textsc{register}}

% Definiere theindex-Environment komplett neu ohne reledmac
\makeatletter
\renewenvironment{theindex}{%
  \section*{\indexname}%
  \setlength{\parindent}{0pt}%
  \setlength{\parskip}{0pt plus 0.3pt}%
  \let\item\@idxitem
}{%
  \clearpage
}
\makeatother

\IfFileExists{\jobname-pw.ind}{\input{\jobname-pw.ind}}{}

\end{document}

      