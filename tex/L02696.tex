%% latex-korrekturansicht-vorspann.tex
%% Vorspann für die Korrekturansicht.
%% Lädt die gemeinsame Datei latex-vorspann.tex mit gesetztem Schalter.

\newif\ifkorrekturansicht
\korrekturansichttrue

\input{../tex-inputs/latex-vorspann}


               \section[Paul Goldmann an Arthur Schnitzler, 10. 8. 1892]{ Paul Goldmann an Arthur Schnitzler, 10. 8. 1892}\nopagebreak\mylabel{v}\rehead{ }\normalsize\beginnumbering\briefempfaengerindex{Schnitzler, Arthur@\textsc{Schnitzler, Arthur}!zzzGoldmann, Paul@\emph{von Paul Goldmann}!1892-08-101@{10. 8. 1892}|(be} \toendnotes[C]{\smallbreak\pagebreak[2]} \Standort{DLA, A:Schnitzler, HS.NZ85.1.3163.}
\physDesc{Postkarte
\newline{}Handschrift: schwarze Tinte, lateinische Kurrent\newline{}Versand: 1) Stempel: »\nobreak{}10 Ago. 92, Amb. Des\textcolor{gray}{c}.\nobreak{}«.  2) Stempel: »\nobreak{}Wien 1/1, 13{[}.{]} 8. 92, 9–10½V., Bestellt\nobreak{}«. 
\newline{}Schnitzler: mit Bleistift das Datum »10/8/92« vermerkt sowie die Jahresangabe »92« der Datumsangabe ergänzt }\toendnotes[C]{\smallbreak}\pstart{}{\pb}\textcolor{pink}{Autriche}{}\ledrightnote{\textcolor{pink}{Österreich}}! \pend{}\pstart{}\textcolor{gray}{\textbf{\begin{otherlanguage}{french}A\end{otherlanguage}}} Herrn Dr. Arthur Schnitzler\pend{}\pstart{}\textcolor{pink}{I. Giselastraße 11}{}\ledrightnote{\textcolor{pink}{Bösendorferstraße}}\pend{}\pstart{}\textcolor{pink}{Wien}{}\ledrightnote{\textcolor{pink}{Wien}}. \pend{}{\bigskip}\pstart
           \raggedleft{}{\pb}\textcolor{pink}{San Sebastian}{}\ledrightnote{\textcolor{pink}{San Sebastian}}, 10 août\pend
           \pstart
           \label{K_L02696-1v}\edtext{Me voilà donc en \textcolor{pink}{Espagne}{}\ledrightnote{\textcolor{pink}{Spanien}}, mon bien cher ami. J’ai passé trois jours dans ce
               petit \textcolor{pink}{paradis}{}\ledrightnote{→\textcolor{pink}{San Sebastian}} au \textcolor{pink}{Golfe de Biscaya}{}\ledrightnote{\textcolor{pink}{Biskaya}}. J’ai vu des choses on ne peut
               plus espagnoles. J’ai assisté aux grandes courses de taureaux, j’ai regardé la \textcolor{blue}{reine}{}\ledrightnote{→\textcolor{blue}{Maria Christina von Österreich}} prendre son bain et le
                  \textcolor{blue}{petit roi}{}\ledrightnote{→\textcolor{blue}{Alfons XIII.}} jouant dans le
                  \strikeout{sabl} sable, j’ai fumé des cigares de \textcolor{pink}{Havanah}{}\ledrightnote{\textcolor{pink}{Havana}} et j’ai bu du vin d’\textcolor{pink}{Andalousie}{}\ledrightnote{\textcolor{pink}{Andalusia}}. Mais je t’assure, que, le premièr moment de
               curiosité passé, mon cœur était rongé de soucis et d’inquiétude nerveuse comme
               avant. Peut-être que tant cela sara beau dans le souvenir, mais dans la présesence,
               ça ne c’est point. Meilleures amitiés. Bien à toi.}{\lemma{\textnormal{\emph{Me … toi.}}}\Cendnote{\textnormal{französisch: »Nun also in Spanien, mein lieber Freund: Ich
                  habe drei Tage in diesem kleinen Paradies am Golf von Biscaya verbracht. Ich habe
                  Dinge gesehen, die spanischer nicht sein könnten. Ich war bei den großen
                  Stierrennen dabei, habe der Königin beim Baden und dem kleinen König beim
                  Sandspielen zugesehen, ich habe Havanna-Zigarren geraucht und Wein aus Andalusien
                  getrunken. Aber sei versichert, dass mein Herz, nachdem der erste Eindruck der
                  Neugierde vorüber war, von Sorgen und nervöser Unruhe zerfressen war wie zuvor.
                  Vielleicht ist es in der Erinnerung schön, aber in der Gegenwart ist es das nicht.
                  Mit besten Grüßen. Alles Gute für dich.«}}}\label{K_L02696-1h}\pend
           \pstart \label{T_L02696-1v}\edtext{Ton \spacefill\mbox{Paul Goldmann.}}{\lemma{\textnormal{\emph{Ton Paul Goldmann.}}}\Cendnote{\textnormal{seitlich am rechten Rand}}}\label{T_L02696-1h}\pend{}\endnumbering\briefempfaengerindex{Schnitzler, Arthur@\textsc{Schnitzler, Arthur}!zzzGoldmann, Paul@\emph{von Paul Goldmann}!1892-08-101@{10. 8. 1892}|)be}\mylabel{h}  \normalsize

\doendnotes{C}
\bigskip
\vfill

\clearpage

\footnotesize

\lohead{\textsc{register}}

% Definiere theindex-Environment komplett neu ohne reledmac
\makeatletter
\renewenvironment{theindex}{%
  \section*{\indexname}%
  \setlength{\parindent}{0pt}%
  \setlength{\parskip}{0pt plus 0.3pt}%
  \let\item\@idxitem
}{%
  \clearpage
}
\makeatother

\IfFileExists{\jobname-pw.ind}{\input{\jobname-pw.ind}}{}

\end{document}

      