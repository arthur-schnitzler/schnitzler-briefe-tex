%% latex-korrekturansicht-vorspann.tex
%% Vorspann für die Korrekturansicht.
%% Lädt die gemeinsame Datei latex-vorspann.tex mit gesetztem Schalter.

\newif\ifkorrekturansicht
\korrekturansichttrue

\input{../tex-inputs/latex-vorspann}


               \section[Robert Adam an Arthur Schnitzler, Briefentwurf, 15. 4. 1913]{ Robert Adam an Arthur Schnitzler, Briefentwurf,
                    15. 4. 1913}\nopagebreak\mylabel{v}\rehead{ }\normalsize\beginnumbering\briefempfaengerindex{Schnitzler, Arthur@\textsc{Schnitzler, Arthur}!zzzAdam, Robert@\emph{von Robert Adam}!1913-04-151@{15. 4. 1913}|(be} \toendnotes[C]{\smallbreak\pagebreak[2]} \Standort{Wien, Österreichische Nationalbibliothek, Cod. ser. 52.266, 161.}
\physDesc{Brief, , 2 Seiten, Entwurf
\newline{}Handschrift: schwarze Tinte, deutsche Kurrent}\toendnotes[C]{\smallbreak}\pstart
           \raggedleft{}{\pb}\textcolor{pink}{Ziſtersdorf}{}\ledrightnote{\textcolor{pink}{Zistersdorf}}, am 1\substVorne{}\textsuperscript{4}\substDazwischen{}5\substHinten{}. April 1913\pend
           \pstart{}Hochverehrter Herr Doktor!\pend\pstart
           Ich mache von Ihrer liebenswürdigen \label{K_L02119_1v}\edtext{Erlaubnis Gebrauch}{\lemma{\textnormal{\emph{Erlaubnis Gebrauch}}}\Cendnote{\textnormal{Eine Fassung des
                        Briefes wurde am 15. 4. 1913 abgesandt, wie aus dem
                        unmittelbar auf den Entwurf folgenden Tagebucheintrag hervorgeht.}}}\label{K_L02119_1h} und
                    überſende Ihnen das Manuſkript \substVorne{}\textsuperscript{der}\substDazwischen{}von\substHinten{} »\textcolor{green}{Fatme}{}\ledrightnote{\textcolor{green}{Fatme}}«.\pend
           \pstart
           Hiebei muß ich Sie vor allem deshalb um Nachſicht bitten, weil die
                    Schreibmaſchinenabſchrift \substVorne{}\textsuperscript{keineswegs ſo}{\allowbreak}\substDazwischen{}verſchiedener leidiger Umſtände halber nicht recht\substHinten{} preſentabel ausgefallen iſt \strikeout{wie ich ſie
                        erwarte. Beſonders der blaue Druck der erſten Hälfte iſt mir herzlich
                        unangenehm. Trotzdem ſende ich Ihnen dies und und nicht das
                        Durchſchlagsexemplar, da letzteres doch weniger deutlich iſt.}\pend
           \pstart
           Und dann bitte ich Sie \introOben{}betreffs\introOben{} der »\textcolor{green}{Fatme}{}\ledrightnote{\textcolor{green}{Fatme}}« ſelbſt \strikeout{wegen} um
                    Duldung. Ich nenne ſie eine »Studie«; ich wage es nicht, ſie eine dramatiſche
                    Studie zu nennen. Die beſte Bezeichnung wäre vielleicht: ein Konglomerat. Wenn
                    ich \introOben{}mir\introOben{} die Frage \substVorne{}\textsuperscript{erwäge}{\allowbreak}\substDazwischen{}ſtelle\substHinten{}, ob dies \substVorne{}\textsuperscript{Konglomerat}{\allowbreak}\substDazwischen{}\strikeout{Sammelſurium} Gemengſel\substHinten{} von \strikeout{Phantaſie,} Phantaſterei, \introOben{}Theorie, \strikeout{Ökonomie,}\introOben{}{ }Satire, \introOben{}Erlebnis\introOben{},
                    Roſinen, \introOben{}Geſellſchafts\introOben{}Kritik-\introOben{}Charakteririſierungs-\introOben{} und Dramenanſätzen Sie intereſſieren werde –
                        \strikeout{ſo} zweifle ich \strikeout{über}{ }\strikeout{die Antwort}; ja ich verzweifle geradezu. Ich
                    möchte faſt wünſchen, ich hätte mich \introOben{}wegen\introOben{} dieſes \introOben{}höchſt undramatiſchen\introOben{} Miſchlings von Ernſt und Spott
                        \introOben{}der betr. \strikeout{d\textcolor{gray}{och}} jedem Akt, ja jeder Szene \strikeout{nicht} einer
                        Spezialexpoſition \strikeout{eröffnen muß} bedarf\introOben{}{ }\strikeout{wegen} nicht an Sie gewendet, da ich ſehr
                    befürchte, eine etwa gute Meinung, die Sie von meinem Geſchmack \introOben{}u. techniſchen Geſchick\introOben{} haben könnten, \strikeout{da}durch \introOben{}ihn\introOben{} zu \substVorne{}\textsuperscript{töten}\substDazwischen{}vernichten\substHinten{}, und ich wünſchte, ich hätte die Vollendung einer \introOben{}weniger exotiſchen u. ſtrafferen\introOben{} Komödie »\textcolor{green}{Geſellſchaft}{}\ledrightnote{\textcolor{green}{Gesellschaft [Eine Gaunerkomödie]}}«, an der ich jetzt arbeite, abgewartet,
                    anſtatt mich »\textcolor{green}{Fatme}{}\ledrightnote{\textcolor{green}{Fatme}}« \introOben{}gewiſſermaßen\introOben{} zu würfeln.\pend
           \pstart
           Was dieſe betrifft, möchte ich zur Aufklärung nur \substVorne{}\textsuperscript{ſagen}\substDazwischen{}beifügen\substHinten{}, daß ich urſprünglich die \introOben{}einfache\introOben{}
                    Dramatiſierung einer Erzählung \textcolor{blue}{\textsc{Wells}}{}\ledrightnote{\textcolor{blue}{H. G. Wells}}{ }\introOben{}(»\textcolor{green}{\textsc{A story of the Days to come}}{}\ledrightnote{\textcolor{green}{A Story of the Days to Come}}{[}«{]} in \textcolor{green}{\textsc{Tales of Space and Time}}{}\ledrightnote{\textcolor{green}{Tales of Space and Time}}{ }\strikeout{\textsc{and Space}})\introOben{}{ }\substVorne{}\textsuperscript{beabſichtigte}{\allowbreak}\substDazwischen{}im Auge hatte\substHinten{}, dann aber, \introOben{}beim Überdenken\introOben{} des Stoffes \strikeout{überdenkend}{ }\strikeout{zur Anſicht}{ }\strikeout{gelangte}{ }\introOben{}mich vor dem \textcolor{gray}{×}\-\textcolor{gray}{×}\-\textcolor{gray}{×}\-\textcolor{gray}{×}weg {\kaufmannsund} die Notwendigkeit
                        geſtellt ſah\introOben{}, \strikeout{ich möchte}{ }\substVorne{}\textsuperscript{den}\substDazwischen{}\strikeout{einen ganzen}\substHinten{}{ }\strikeout{Zukunftsſtaat,}{ }\introOben{}an\introOben{}ſtatt den \textcolor{blue}{\textsc{Wells}}{}\ledrightnote{\textcolor{blue}{H. G. Wells}}’ſchen \introOben{}Zukunftsſta\textcolor{gray}{at}\introOben{} einfach \substVorne{}\textsuperscript{anzunehmen}{\allowbreak}\substDazwischen{}\strikeout{als gegeben}\substHinten{}, \strikeout{nach}{ }\introOben{}\strikeout{gänzl} zu akzeptieren, in einen Staat zu
                        verlegen, der\introOben{} meinen \strikeout{eigenen} Anſichten
                        \strikeout{raus}{ }\introOben{}zu\introOben{}{ }\introOben{}von einer möglichen Entwicklung der ſozialen Verhältniſſe
                        beſſer entſpräche. So mußte ich für den gegebenen Stoff einen eigenen
                            Zukunftsſta\textcolor{gray}{at}\introOben{} konſtruieren; und kaum {\pb}war \substVorne{}\textsuperscript{damit begonnen}{\allowbreak}\substDazwischen{}dies geſchehen\substHinten{}, ſo \substVorne{}\textsuperscript{ſah ich auch}{\allowbreak}\substDazwischen{}ergab ſich\substHinten{} die \introOben{}weitere\introOben{} Notwendigkeit \strikeout{vor mir}, \introOben{}auch\introOben{} mit dem
                        \textcolor{blue}{\textsc{Wells}}{}\ledrightnote{\textcolor{blue}{H. G. Wells}}’ſchen Stoff zu brechen \substVorne{}\textsuperscript{und formte meinen eigenen, wie er meinem Staat
                            entſprach.}{\allowbreak}\substDazwischen{}und die Fabel meinem Staate anzupaſſen. So iſt \textcolor{green}{Fatme}{}\ledrightnote{\textcolor{green}{Fatme}} die \textcolor{green}{\textsc{Story of the Days to come}}{}\ledrightnote{\textcolor{green}{Gesellschaft [Eine Gaunerkomödie]}};\substHinten{}{ }\substVorne{}\textsuperscript{Alſo wurde zuerſt das Feſt, dann die}{\allowbreak}\substDazwischen{}dasſelbe Meſſer, doch mit anderem und andrer\substHinten{} Klinge \strikeout{des Meſſers geändert}\pend
           \pstart
           Sollten Sie, hochverehrter Herr Doktor, der Studie kein Intereſſe ab\substVorne{}\textsuperscript{nötigen}{\allowbreak}\substDazwischen{}gewinnen\substHinten{} können, ſo bitte ich Sie, mir wegen ihrer Ueberſendung nicht zu grollen
                    und mir zu erlauben, ſie \introOben{}ſpäter\introOben{} gegen die »\textcolor{green}{Geſellſchaft}{}\ledrightnote{\textcolor{green}{Gesellschaft [Eine Gaunerkomödie]}}«, \strikeout{die
                        jedenfalls weniger Sammelſurium werden wird,} umzutauſchen.\pend
           \pstart Ich verbleibe mit den ergebenſten Grüßen\hspace*{1.5em}Ihr\hspace*{1.5em}\spacefill\mbox{RA}\pend{}\endnumbering\briefempfaengerindex{Schnitzler, Arthur@\textsc{Schnitzler, Arthur}!zzzAdam, Robert@\emph{von Robert Adam}!1913-04-151@{15. 4. 1913}|)be}\mylabel{h}  \normalsize

\doendnotes{C}
\bigskip
\vfill

\clearpage

\footnotesize

\lohead{\textsc{register}}

% Definiere theindex-Environment komplett neu ohne reledmac
\makeatletter
\renewenvironment{theindex}{%
  \section*{\indexname}%
  \setlength{\parindent}{0pt}%
  \setlength{\parskip}{0pt plus 0.3pt}%
  \let\item\@idxitem
}{%
  \clearpage
}
\makeatother

\IfFileExists{\jobname-pw.ind}{\input{\jobname-pw.ind}}{}

\end{document}

      