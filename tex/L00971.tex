%% latex-korrekturansicht-vorspann.tex
%% Vorspann für die Korrekturansicht.
%% Lädt die gemeinsame Datei latex-vorspann.tex mit gesetztem Schalter.

\newif\ifkorrekturansicht
\korrekturansichttrue

\input{../tex-inputs/latex-vorspann}


               \section[Arthur Schnitzler an Richard Beer-Hofmann, 9. 9. 1899]{ Arthur Schnitzler an Richard Beer-Hofmann, 9. 9. 1899}\nopagebreak\mylabel{v}\rehead{ }\normalsize\beginnumbering\briefempfaengerindex{Beer-Hofmann, Richard@\textsc{Beer-Hofmann, Richard}!zzzSchnitzler, Arthur@\emph{von Arthur Schnitzler}!1899-09-091@{9. 9. 1899}|(be} \toendnotes[C]{\smallbreak\pagebreak[2]} \Standort{YCGL, MSS 31.}
\physDesc{Brief, 1 Blatt, 4 Seiten, Umschlag
\newline{}Handschrift: Bleistift, deutsche Kurrent\newline{}Versand: 1) Stempel: »\nobreak{}\oindex{Bad Ischl@\textbf{Bad Ischl}, \emph{Besiedelter Ort (A.BSO)}|pwk}Isc\textcolor{gray}{hl}, 9. {[}9. 1899{]}, 5–6{[}N{]}\nobreak{}«.  2) Stempel: »\nobreak{}\oindex{Sachsenburg@\textbf{Sachsenburg}, \emph{http://www.geonames.org/ontologyA.ADM3}|pwk}Sachsenburg, 10 9 \textcolor{gray}{99}\nobreak{}«. 3) Stempel: »\nobreak{}\oindex{Vahrn@\textbf{Vahrn}, \emph{Besiedelter Ort (A.BSO)}|pwk}Vahrn, 12 9 99\nobreak{}«. 4) mit schwarzer Tinte von unbekannter Hand nachgesandt nach »\textsc{\textcolor{pink}{Vahrn} bei \textcolor{pink}{Brixen}}«}\buchAbdrucke{\weitereDrucke{Arthur Schnitzler, Richard Beer-Hofmann: \emph{Briefwechsel 1891–1931}. Hg. Konstanze Fliedl. Wien, Zürich: \emph{Europaverlag} 1992, S. 134.} }\toendnotes[C]{\smallbreak}\pstart{}{\pb}\textsc{Dr Richard Beer-Hofmann}\pend{}\pstart{}\textcolor{pink}{\textsc{Sachsenburg}}{}\ledrightnote{\textcolor{pink}{Sachsenburg}}\pend{}\pstart{}\textcolor{pink}{Gaſthof Fritz}{}\ledrightnote{\textcolor{pink}{Gasthof Fritz}}\pend{}\pstart{}\textcolor{pink}{\textsc{Kärnthen}}{}\ledrightnote{\textcolor{pink}{Kärnten}}\pend{}{\bigskip}\pstart
           {\pb}\textcolor{pink}{\textsc{Ischl}}{}\ledrightnote{\textcolor{pink}{Bad Ischl}}.\hfill 9. 9. 99.\pend
           \pstart{}Mein lieber Richard,\pend\pstart
           Dinſtag verlaſſe ich \textcolor{pink}{Iſchl}{}\ledrightnote{\textcolor{pink}{Bad Ischl}} und fahre
               vorerſt nach \textcolor{pink}{München}{}\ledrightnote{\textcolor{pink}{München}}. Ich möchte dort gern \introOben{}Mittwoch o Donnerſtg\introOben{} eine Nachricht von Ihnen \textsc{\uline{post. rest.}} finden.\pend
           \pstart
           {\pb}Mir iſt’s mit mein\textcolor{gray}{em}{ }\textcolor{green}{Stück}{}\ledrightnote{→\textcolor{green}{Der Schleier der Beatrice. Schauspiel in fünf Akten}} momentweiſe gut, öfters
               mäßig gegangen, u ich habe es heute mit einem vorläufigen durchaus undefinitiven
               Abschluſs bei Seite gelegt; – auf 1–2\introOben{}–3\introOben{} Tage.\pend
           \pstart
           {\pb}Ich hoffe, Sie fühlen ſich mit mehr Kraft Ihrem \textcolor{green}{Stoff}{}\ledrightnote{→\textcolor{green}{Der Graf von Charolais. Ein Trauerspiel}} gegenüber als ich.\pend
           \pstart
           – \textcolor{blue}{Hugo}{}\ledrightnote{\textcolor{blue}{Hugo von Hofmannsthal}} iſt ſchon wieder fort; ich bin ſehr froh
               geweſen, \substVorne{}\textsuperscript{als}\substDazwischen{}dſs\substHinten{} er da war, Sie werden ihn wohl bald ſehen. – Ich bin {\pb}recht ſehr gequält, durch allerlei; – durch das Ohr
               wohl am meiſten u tiefſten augenblicklich.\pend
           \pstart
           Grüßen Sie \textcolor{blue}{Frau}{}\ledrightnote{→\textcolor{blue}{Paula Beer-Hofmann}} und \textcolor{blue}{Kinder}{}\ledrightnote{→\textcolor{blue}{Naëmah Beer-Hofmann}{\newline}→\textcolor{blue}{Mirjam Beer-Hofmann}}\pend
           \pstart
           Von Herzen Ihr{\\[\baselineskip]}\spacefill\mbox{Arthur}\pend
           \leftskip=0em{}\endnumbering\briefempfaengerindex{Beer-Hofmann, Richard@\textsc{Beer-Hofmann, Richard}!zzzSchnitzler, Arthur@\emph{von Arthur Schnitzler}!1899-09-091@{9. 9. 1899}|)be}\mylabel{h}  \normalsize

\doendnotes{C}
\bigskip
\vfill

\clearpage

\footnotesize

\lohead{\textsc{register}}

% Definiere theindex-Environment komplett neu ohne reledmac
\makeatletter
\renewenvironment{theindex}{%
  \section*{\indexname}%
  \setlength{\parindent}{0pt}%
  \setlength{\parskip}{0pt plus 0.3pt}%
  \let\item\@idxitem
}{%
  \clearpage
}
\makeatother

\IfFileExists{\jobname-pw.ind}{\input{\jobname-pw.ind}}{}

\end{document}

      