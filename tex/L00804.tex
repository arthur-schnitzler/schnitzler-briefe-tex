%% latex-korrekturansicht-vorspann.tex
%% Vorspann für die Korrekturansicht.
%% Lädt die gemeinsame Datei latex-vorspann.tex mit gesetztem Schalter.

\newif\ifkorrekturansicht
\korrekturansichttrue

\input{../tex-inputs/latex-vorspann}


               \section[Arthur Schnitzler und Richard Beer-Hofmann an Hugo von Hofmannsthal, 10. 6. 1898]{ Arthur Schnitzler und Richard Beer-Hofmann an Hugo von
                    Hofmannsthal, 10. 6. 1898}\nopagebreak\mylabel{v}\rehead{ }\normalsize\beginnumbering\briefempfaengerindex{Hofmannsthal, Hugo von@\textsc{Hofmannsthal, Hugo von}!zzzBeer-Hofmann, Richard@\emph{von Richard Beer-Hofmann}!1898-06-101@{10. 6. 1898}|(be}\briefempfaengerindex{Hofmannsthal, Hugo von@\textsc{Hofmannsthal, Hugo von}!zzzSchnitzler, Arthur@\emph{von Arthur Schnitzler}!1898-06-101@{10. 6. 1898}|(be} \toendnotes[C]{\smallbreak\pagebreak[2]} \Standort{FDH, Hs-30885,67.}
\physDesc{Brief, 1 Blatt, 1 Seite
\newline{}Handschrift Arthur Schnitzler: Bleistift, deutsche Kurrent\newline{}Handschrift Richard Beer-Hofmann: Bleistift}\buchAbdrucke{\weitereDrucke{Hugo von Hofmannsthal, Arthur Schnitzler: \emph{Briefwechsel}. Hg. Therese Nickl und Heinrich Schnitzler. Frankfurt am Main: \emph{S. Fischer} 1964, S. 102.} }\pstart
           \raggedleft{}{\pb}10/6 98\pend
           \pstart
           Lieber Hugo, die Radpartie war ſehr ſchön; ſeit
                        Dinſtag bin ich in \textcolor{pink}{\textsc{Steindorf}}{}\ledrightnote{\textcolor{pink}{Steindorf am Ossiacher See}}, wo es viel regnet; So{\geminationn}tag möcht
                    ich gerne in \textcolor{pink}{Wien}{}\ledrightnote{\textcolor{pink}{Wien}} eine Nachricht von Ihnen
                    finden, ob man Sie vielleicht abends ſehen kann.\pend
           \pstart
           Herzlichen Gruß Ihr \spacefill\mbox{Arthur}{\\[\baselineskip]}\spacefill\mbox{{[}hs. Beer-Hofmann:{]} Richard}\pend
           \leftskip=0em{}\endnumbering\briefempfaengerindex{Hofmannsthal, Hugo von@\textsc{Hofmannsthal, Hugo von}!zzzBeer-Hofmann, Richard@\emph{von Richard Beer-Hofmann}!1898-06-101@{10. 6. 1898}|)be}\briefempfaengerindex{Hofmannsthal, Hugo von@\textsc{Hofmannsthal, Hugo von}!zzzSchnitzler, Arthur@\emph{von Arthur Schnitzler}!1898-06-101@{10. 6. 1898}|)be}\mylabel{h}  \normalsize

\doendnotes{C}
\bigskip
\vfill

\clearpage

\footnotesize

\lohead{\textsc{register}}

% Definiere theindex-Environment komplett neu ohne reledmac
\makeatletter
\renewenvironment{theindex}{%
  \section*{\indexname}%
  \setlength{\parindent}{0pt}%
  \setlength{\parskip}{0pt plus 0.3pt}%
  \let\item\@idxitem
}{%
  \clearpage
}
\makeatother

\IfFileExists{\jobname-pw.ind}{\input{\jobname-pw.ind}}{}

\end{document}

      