%% latex-korrekturansicht-vorspann.tex
%% Vorspann für die Korrekturansicht.
%% Lädt die gemeinsame Datei latex-vorspann.tex mit gesetztem Schalter.

\newif\ifkorrekturansicht
\korrekturansichttrue

\input{../tex-inputs/latex-vorspann}


               \section[Arthur Schnitzler an Georg Brandes, 23. 6. 1924]{ Arthur Schnitzler an Georg Brandes, 23. 6. 1924}\nopagebreak\mylabel{v}\rehead{ }\normalsize\beginnumbering\briefempfaengerindex{Brandes, Georg@\textsc{Brandes, Georg}!zzzSchnitzler, Arthur@\emph{von Arthur Schnitzler}!1924-06-231@{23. 6. 1924}|(be} \toendnotes[C]{\smallbreak\pagebreak[2]} \Standort{Kopenhagen, Det Kongelige Bibliotek, Georg Brandes Arkiv, box 125.}
\physDesc{Postkarte
\newline{}Handschrift: Bleistift, lateinische Kurrent\newline{}Versand: 1) Stempel: »\nobreak{}\oindex{XVIII., Waehring@\textbf{XVIII., Währing}, \emph{Bezirk (A.BZK)}|pwk}18 W\textcolor{gray}{i}en
                                          \textcolor{gray}{110}, 23. VI. 24, 17\nobreak{}«.  2) mit blauer Tinte von unbekannter Hand die Ortsangabe in der
                                 Adresse geändert zu: »\noindent{}\textcolor{pink}{Villa Iris}{ / }\textcolor{pink}{\uline{Hornbæk}}«\newline{}Ordnung: mit Bleistift von unbekannter Hand nummeriert:
                                    »Schnitzler 48.« }\buchAbdrucke{\weitereDrucke{Georg Brandes, Arthur Schnitzler: \emph{Ein Briefwechsel}. Hg. Kurt Bergel. Bern: \emph{Francke} 1956, S. 139–140.} }\toendnotes[C]{\smallbreak}\pstart{}{\pb}\label{T_L02413-1v}\edtext{\textcolor{gray}{\textbf{A. S.}}}{\lemma{\textnormal{\emph{A. S.}}}\Cendnote{\textnormal{ovaler Absenderkleber}}}\label{T_L02413-1h}\pend{}\pstart{}\textcolor{pink}{\textcolor{gray}{\textbf{WIEN, XVIII.}}}{}\ledrightnote{\textcolor{pink}{XVIII., Währing}}\pend{}\pstart{}\textcolor{pink}{\textcolor{gray}{\textbf{STERNWARTESTR. 71}}}{}\ledrightnote{\textcolor{pink}{Sternwartestraße}}\pend{}{\bigskip}\pstart{}Hr\pend{}\pstart{}Georg Brandes\pend{}\pstart{}\textcolor{pink}{Kopenhagen}{}\ledrightnote{\textcolor{pink}{Kopenhagen}}\pend{}{\bigskip}\pstart
           \raggedleft{}{\pb}\textcolor{pink}{Wien}{}\ledrightnote{\textcolor{pink}{Wien}}, 23. 6. 24\pend
           \pstart
           Mein lieber und verehrter Freund, vor kurzem erst hab ich Ihren
               wunderbaren \textcolor{green}{Voltaire}{}\ledrightnote{\textcolor{green}{Voltaire und sein Jahrhundert}} mit wahrem Entzücken gelesen
               und wieder erfreuen Sie mich durch die gütige Übersendg der zwei Bände Ihrer \textcolor{green}{Hauptströmungen}{}\ledrightnote{\textcolor{green}{Hauptströmungen der Literatur des neunzehnten Jahrhunderts}}, – die, eine theure
               Jugenderinnerung, mich nun in ihrer \label{K_L02413_1v}\edtext{neuen Form}{\lemma{\textnormal{\emph{neuen Form}}}\Cendnote{\textnormal{\textcolor{blue}{Georg Brandes}: \emph{\textcolor{green}{Hauptströmungen der Literatur des 19. Jahrhunderts}}. Vom Verfasser neu
                     bearbeitete endgültige Ausgabe. Berlin: \emph{\textcolor{brown}{Erich
                        Reiss}}{ }1924.}}}\label{K_L02413_1h} in den Sommer begleiten sollen, wie der \textcolor{green}{Michel Angelo}{}\ledrightnote{\textcolor{green}{Michelangelo Buonarotti}}. Wie werd ich Ihnen immer von neuem, – und wie gern
               immer wieder Dank schuldig. – Ich bin in den letzten Monaten nicht ganz unthätig
               gewesen, und hoffe mich für Ihre kostbaren {\pb}Gaben,
               in recht bescheidener \textcolor{green}{Weise}{}\ledrightnote{→\textcolor{green}{Fräulein Else}{\newline}→\textcolor{green}{Komödie der Verführung. In drei Akten}}, bald revanchiren zu dürfen. Ich hoffe liebster un\textcolor{gray}{d}
               verehrtester Georg Brandes, Sie befin\textcolor{gray}{d}en sich wohl. Lassen Sie
               mich auch darüber ein Wort vernehmen; ich schreibe demnächst mehr. In Freundschaft u.
               Bewunderung stets der Ihre \spacefill\mbox{Arthur Schnitzler}\pend
           \endnumbering\briefempfaengerindex{Brandes, Georg@\textsc{Brandes, Georg}!zzzSchnitzler, Arthur@\emph{von Arthur Schnitzler}!1924-06-231@{23. 6. 1924}|)be}\mylabel{h}  \normalsize

\doendnotes{C}
\bigskip
\vfill

\clearpage

\footnotesize

\lohead{\textsc{register}}

% Definiere theindex-Environment komplett neu ohne reledmac
\makeatletter
\renewenvironment{theindex}{%
  \section*{\indexname}%
  \setlength{\parindent}{0pt}%
  \setlength{\parskip}{0pt plus 0.3pt}%
  \let\item\@idxitem
}{%
  \clearpage
}
\makeatother

\IfFileExists{\jobname-pw.ind}{\input{\jobname-pw.ind}}{}

\end{document}

      