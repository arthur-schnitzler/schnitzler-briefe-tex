%% latex-korrekturansicht-vorspann.tex
%% Vorspann für die Korrekturansicht.
%% Lädt die gemeinsame Datei latex-vorspann.tex mit gesetztem Schalter.

\newif\ifkorrekturansicht
\korrekturansichttrue

\input{../tex-inputs/latex-vorspann}


               \section[Hermann Bahr an Arthur Schnitzler, 12. 2. 1904]{ Hermann Bahr an Arthur Schnitzler, 12. 2. 1904}\nopagebreak\mylabel{v}\rehead{ }\normalsize\beginnumbering\briefempfaengerindex{Schnitzler, Arthur@\textsc{Schnitzler, Arthur}!zzzBahr, Hermann@\emph{von Hermann Bahr}!1904-02-121@{12. 2. 1904}|(be} \toendnotes[C]{\smallbreak\pagebreak[2]} \Standort{CUL, Schnitzler, B 5b.}
\physDesc{Bildpostkarte
\newline{}Handschrift: Bleistift, deutsche Kurrent\newline{}Versand: 1) Stempel: »\nobreak{}\oindex{Konstanz@\textbf{Konstanz}, \emph{Besiedelter Ort (A.BSO)}|pwk}Konstanz–Basel Bahnpost, Zug 1627, 13/2 04\nobreak{}«.  2) Stempel: »\nobreak{}14/2 {[}04{]}\nobreak{}«. 3) Stempel: »\nobreak{}\textcolor{gray}{×}\-\textcolor{gray}{×}{[}/2{]} 04, 7–8N, Bestellt vom Postamte 7\nobreak{}«. 4) von unbekannten Händen Adresse teilweise
                           gestrichen und ergänzt: »7«, »\textsc{\textcolor{pink}{NW 7 Continentalhotel}}« sowie »\textcolor{gray}{×}\-\textcolor{gray}{×}\-\textcolor{gray}{×}\-\textcolor{gray}{×}\-\textcolor{gray}{×}\-\textcolor{gray}{×}\-\textcolor{gray}{×}\-\textcolor{gray}{×}\-\textcolor{gray}{×}\-\textcolor{gray}{×}\-\textcolor{gray}{×}\-\textcolor{gray}{×} 8/6.«}\buchAbdrucke{\weitereDrucke{Hermann Bahr, Arthur Schnitzler: \emph{Briefwechsel, Aufzeichnungen, Dokumente (1891–1931)}. Hg. Kurt Ifkovits und Martin Anton Müller. Göttingen: \emph{Wallstein} 2018, S. 300.} }\toendnotes[C]{\smallbreak}\pstart{}{\pb}Herrn \textsc{D\textsuperscript{r} Arthur Schnitzler}\pend{}\pstart{}\textcolor{pink}{\textsc{Deutsches Theater}}{}\ledrightnote{\textcolor{pink}{Deutsches Theater Berlin}}\pend{}\pstart{}\textcolor{pink}{\textsc{Berlin NW}}{}\ledrightnote{\textcolor{pink}{Berlin}}\pend{}\pstart{}\textcolor{pink}{\textsc{Schumannstr.}}{}\ledrightnote{\textcolor{pink}{Schumannstraße}}\pend{}{\bigskip}\pstart
           \noindent{}\centering{}\textcolor{gray}{\textbf{{\pb}\textcolor{pink}{Konstanz Conciliumsgebäude}{}\ledrightnote{\textcolor{pink}{Konzilgebäude}}}}\pend
           \pstart
           12/2\pend
           \pstart
           Ich denk viel an morgen Abend.\pend
           \pstart
           Das Beſte an Deine \textcolor{blue}{Frau}{}\ledrightnote{→\textcolor{blue}{Olga Schnitzler}}, an \textcolor{blue}{Brahm}{}\ledrightnote{\textcolor{blue}{Otto Brahm}} u. an \textcolor{blue}{Rittner}{}\ledrightnote{\textcolor{blue}{Rudolf Rittner}}.\pend
           \pstart Herzlichſt grüßt Dein \spacefill\mbox{Herma{\geminationn}}\pend{}\pstart
           \noindent{}Auf der Flucht nach dem Süden!\pend
           \endnumbering\briefempfaengerindex{Schnitzler, Arthur@\textsc{Schnitzler, Arthur}!zzzBahr, Hermann@\emph{von Hermann Bahr}!1904-02-121@{12. 2. 1904}|)be}\mylabel{h}  \normalsize

\doendnotes{C}
\bigskip
\vfill

\clearpage

\footnotesize

\lohead{\textsc{register}}

% Definiere theindex-Environment komplett neu ohne reledmac
\makeatletter
\renewenvironment{theindex}{%
  \section*{\indexname}%
  \setlength{\parindent}{0pt}%
  \setlength{\parskip}{0pt plus 0.3pt}%
  \let\item\@idxitem
}{%
  \clearpage
}
\makeatother

\IfFileExists{\jobname-pw.ind}{\input{\jobname-pw.ind}}{}

\end{document}

      