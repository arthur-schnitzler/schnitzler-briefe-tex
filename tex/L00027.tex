%% latex-korrekturansicht-vorspann.tex
%% Vorspann für die Korrekturansicht.
%% Lädt die gemeinsame Datei latex-vorspann.tex mit gesetztem Schalter.

\newif\ifkorrekturansicht
\korrekturansichttrue

\input{../tex-inputs/latex-vorspann}


               \section[Arthur Schnitzler an Wilhelm Bölsche, 3. 8. 1891]{ Arthur Schnitzler an Wilhelm Bölsche, 3. 8. 1891}\nopagebreak\mylabel{v}\rehead{ }\normalsize\beginnumbering\briefempfaengerindex{Boelsche, Wilhelm@\textsc{Bölsche, Wilhelm}!zzzSchnitzler, Arthur@\emph{von Arthur Schnitzler}!1891-08-031@{3. 8. 1891}|(be} \toendnotes[C]{\smallbreak\pagebreak[2]} \Standort{Wrocław, Biblioteka Uniwersytecka, Böl.Pis 1760.}
\physDesc{Brief, 1 Blatt, 2 Seiten
\newline{}Handschrift: schwarze Tinte, deutsche Kurrent
\newline{}Bölsche: mit schwarzer Tinte mit dem Vermerk »Angen{[}ommen{]}« beschriftet }\buchAbdrucke{\weitereDrucke{1) Alois Woldan: \emph{Arthur Schnitzler – Briefe an Wilhelm Bölsche.} In: \emph{Germanica Wratislaviensia} (1987) Nr. 77, S. 458.} \weitereDrucke{2) Wilhelm Bölsche: \emph{Briefwechsel. Mit Autoren der Freien Bühne}. Hg. Gerd-Hermann Susen. Berlin: \emph{Weidler} 2010, S. 671–672 (Werke und Briefe. Wissenschaftliche Ausgabe, Briefe I).} }\toendnotes[C]{\smallbreak}\pstart{}{\pb}Sehr geehrter Herr Redacteur!\pend\pstart
           Vor einigen Monaten war ich ſo frei, Ihnen eine Skizze, »\textcolor{green}{Der Sohn}{}\ledrightnote{\textcolor{green}{Der Sohn. Aus den Papieren eines Arztes}}« betitelt, einzuſenden, mit dem Erſuchen, mich
                    davon zu verſtändigen, ob Sie dieſelbe in Ihrer geſchätzten \textcolor{green}{Zeitſchrift}{}\ledrightnote{→\textcolor{green}{Freie Bühne für den Entwickelungskampf der Zeit}} zur Veröffentlichung bringen
                    wollen. Da mir bis heute keine Nachricht zugeko{\geminationm}en,
                        {\pb}wiederhole ich hiermit meine Anfrage.\pend
           \pstart
           Mit ausgezeichneter Hochachtung{\\[\baselineskip]}\spacefill\mbox{Dr Arthur Schnitzler}\pend
           \leftskip=0em{}\pstart
           \textcolor{pink}{\textsc{Wien I Giselastraße 11}}{}\ledrightnote{\textcolor{pink}{Bösendorferstraße}}{\\}\textsc{3. August 1891}.\pend
           \endnumbering\briefempfaengerindex{Boelsche, Wilhelm@\textsc{Bölsche, Wilhelm}!zzzSchnitzler, Arthur@\emph{von Arthur Schnitzler}!1891-08-031@{3. 8. 1891}|)be}\mylabel{h}  \normalsize

\doendnotes{C}
\bigskip
\vfill

\clearpage

\footnotesize

\lohead{\textsc{register}}

% Definiere theindex-Environment komplett neu ohne reledmac
\makeatletter
\renewenvironment{theindex}{%
  \section*{\indexname}%
  \setlength{\parindent}{0pt}%
  \setlength{\parskip}{0pt plus 0.3pt}%
  \let\item\@idxitem
}{%
  \clearpage
}
\makeatother

\IfFileExists{\jobname-pw.ind}{\input{\jobname-pw.ind}}{}

\end{document}

      