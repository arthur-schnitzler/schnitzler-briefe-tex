%% latex-korrekturansicht-vorspann.tex
%% Vorspann für die Korrekturansicht.
%% Lädt die gemeinsame Datei latex-vorspann.tex mit gesetztem Schalter.

\newif\ifkorrekturansicht
\korrekturansichttrue

\input{../tex-inputs/latex-vorspann}


               \section[Arthur Schnitzler an Gertrud Rung, 9. 3. 1925]{ Arthur Schnitzler an Gertrud Rung, 9. 3. 1925}\nopagebreak\mylabel{v}\rehead{ }\normalsize\beginnumbering\briefempfaengerindex{Rung, Gertrud@\textsc{Rung, Gertrud}!zzzSchnitzler, Arthur@\emph{von Arthur Schnitzler}!1925-03-091@{9. 3. 1925}|(be} \toendnotes[C]{\smallbreak\pagebreak[2]} \Standort{Kopenhagen, Det Kongelige Bibliotek, Georg Brandes Arkiv, box 125.}
\physDesc{Postkarte
\newline{}Handschrift: schwarze Tinte, lateinische Kurrent\newline{}Versand: Stempel: »\nobreak{}\oindex{XVIII., Waehring@\textbf{XVIII., Währing}, \emph{Bezirk (A.BZK)}|pwk}18/1 Wien 110, 11. III. 25, 9\nobreak{}«.  \newline{}Ordnung: 1) mit Bleistift von unbekannter Hand nummeriert: »\strikeout{48a}« 2) mit Bleistift von unbekannter Hand nummeriert: »51.a«}\buchAbdrucke{\weitereDrucke{Georg Brandes, Arthur Schnitzler: \emph{Ein Briefwechsel}. Hg. Kurt Bergel. Bern: \emph{Francke} 1956, S. 145.} }\toendnotes[C]{\smallbreak}\pstart{}{\pb}\label{T_L02438-1v}\edtext{\textcolor{gray}{\textbf{A. S.}}}{\lemma{\textnormal{\emph{A. S.}}}\Cendnote{\textnormal{ovaler Absenderkleber}}}\label{T_L02438-1h}\pend{}\pstart{}\textcolor{pink}{\textcolor{gray}{\textbf{WIEN, XVIII.}}}{}\ledrightnote{\textcolor{pink}{XVIII., Währing}}\pend{}\pstart{}\textcolor{pink}{\textcolor{gray}{\textbf{STERNWARTESTR. 71}}}{}\ledrightnote{\textcolor{pink}{Sternwartestraße}}\pend{}{\bigskip}\pstart{}An Frau Rung\pend{}\pstart{}per Adr. Georg Brandes\pend{}\pstart{}\textcolor{pink}{Kopenhagen}{}\ledrightnote{\textcolor{pink}{Kopenhagen}}. \pend{}{\bigskip}\pstart
           \raggedleft{}{\pb}\textcolor{pink}{Wien}{}\ledrightnote{\textcolor{pink}{Wien}}, 9. 3. 25\pend
           \pstart{}Verehrte Frau Rung,\pend\pstart
           schönen Dank für Ihre freundl Nachricht; – da ich schon früher nach \textcolor{pink}{Berlin}{}\ledrightnote{\textcolor{pink}{Berlin}} fahren muſs, ist es unsicher ob ich
                    Professor \textcolor{blue}{Brandes}{}\ledrightnote{\textcolor{blue}{Georg Brandes}} Ankunft werde abwarten
                    können. Doch lese ich in der Zeitung, dſs \textcolor{blue}{G. B.}{}\ledrightnote{\textcolor{blue}{Georg Brandes}} auch nach \textcolor{pink}{\uline{Wien}}{}\ledrightnote{\textcolor{pink}{Wien}} reisen wird – \uline{bewahrheitet sich das}? Wie
                    froh wäre ich. Ich bitte um Nachricht nach \textcolor{pink}{Berlin}{}\ledrightnote{\textcolor{pink}{Berlin}}, an die Adresse meines Sohnes \textcolor{blue}{Heinrich Schnitzler}{}\ledrightnote{\textcolor{blue}{Heinrich Schnitzler}}{ }\textcolor{pink}{Matthäikirchstraße 4}{}\ledrightnote{\textcolor{pink}{Herbert-von-Karajan-Straße}}, bei \textcolor{blue}{Dernburg}{}\ledrightnote{\textcolor{blue}{Ilse Dernburg}}. Meine herzlichsten Grüße an \textcolor{blue}{Georg {\pb}Brandes}{}\ledrightnote{\textcolor{blue}{Georg Brandes}},\pend
           \pstart
           mit vielen Empfehlungen{\\[\baselineskip]}Ihr ergebner \spacefill\mbox{Arthur Schnitzler}\pend
           \leftskip=0em{}\endnumbering\briefempfaengerindex{Rung, Gertrud@\textsc{Rung, Gertrud}!zzzSchnitzler, Arthur@\emph{von Arthur Schnitzler}!1925-03-091@{9. 3. 1925}|)be}\mylabel{h}  \normalsize

\doendnotes{C}
\bigskip
\vfill

\clearpage

\footnotesize

\lohead{\textsc{register}}

% Definiere theindex-Environment komplett neu ohne reledmac
\makeatletter
\renewenvironment{theindex}{%
  \section*{\indexname}%
  \setlength{\parindent}{0pt}%
  \setlength{\parskip}{0pt plus 0.3pt}%
  \let\item\@idxitem
}{%
  \clearpage
}
\makeatother

\IfFileExists{\jobname-pw.ind}{\input{\jobname-pw.ind}}{}

\end{document}

      