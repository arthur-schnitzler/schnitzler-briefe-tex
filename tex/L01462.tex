%% latex-korrekturansicht-vorspann.tex
%% Vorspann für die Korrekturansicht.
%% Lädt die gemeinsame Datei latex-vorspann.tex mit gesetztem Schalter.

\newif\ifkorrekturansicht
\korrekturansichttrue

\input{../tex-inputs/latex-vorspann}


               \section[Richard Beer-Hofmann an Arthur Schnitzler, 30. 10. 1904]{ Richard Beer-Hofmann an Arthur Schnitzler,
               30. 10. 1904}\nopagebreak\mylabel{v}\rehead{ }\normalsize\beginnumbering\briefempfaengerindex{Schnitzler, Arthur@\textsc{Schnitzler, Arthur}!zzzBeer-Hofmann, Richard@\emph{von Richard Beer-Hofmann}!1904-10-301@{30. 10. 1904}|(be} \toendnotes[C]{\smallbreak\pagebreak[2]} \Standort{CUL, Schnitzler, B 8.}
\physDesc{Brief, 1 Blatt, 1 Seite
\newline{}Handschrift: schwarze Tinte, lateinische Kurrent\newline{}Ordnung: mit Bleistift von unbekannter Hand nummeriert: »193« }\buchAbdrucke{\weitereDrucke{Arthur Schnitzler, Richard Beer-Hofmann: \emph{Briefwechsel 1891–1931}. Hg. Konstanze Fliedl. Wien, Zürich: \emph{Europaverlag} 1992, S. 167.} }\toendnotes[C]{\smallbreak}\pstart
           \raggedleft{}{\pb}\textcolor{pink}{Rodaun}{}\ledrightnote{\textcolor{pink}{Rodaun}}{ }30/X 04\pend
           \pstart
           Lieber Arthur! Ich bin seit gestern zurück. Können wir uns morgen –
                  \label{K_L01462_1v}\edtext{Montag}{\lemma{\textnormal{\emph{Montag}}}\Cendnote{\textnormal{siehe A. S.: \emph{Tagebuch}, 31. 10. 1904}}}\label{K_L01462_1h} um 7.43 in \textcolor{pink}{Hietzing}{}\ledrightnote{\textcolor{pink}{XIII., Hietzing}} – \textcolor{pink}{Ottakringer}{}\ledrightnote{\textcolor{pink}{Ottakringer Bräu}} –
               treffen? Nicht später, da ich nicht zu spät nach Hause ko{\geminationm}en möchte. Es wäre mir lieb morgen – schon um Ihnen ausführlich über den \textcolor{green}{grünen Kakadu}{}\ledrightnote{\textcolor{green}{Der grüne Kakadu. Groteske in einem Akt}} berichten zu können der in guten
               Händen ist und an dem man schon arbeitet.\pend
           \pstart
           Herzlichst Ihr{\\[\baselineskip]}\spacefill\mbox{Richard}\pend
           \leftskip=0em{}\pstart
           \noindent{}Bitte telegraphiren Sie oder telephoniren Sie!\pend
           \endnumbering\briefempfaengerindex{Schnitzler, Arthur@\textsc{Schnitzler, Arthur}!zzzBeer-Hofmann, Richard@\emph{von Richard Beer-Hofmann}!1904-10-301@{30. 10. 1904}|)be}\mylabel{h}  \normalsize

\doendnotes{C}
\bigskip
\vfill

\clearpage

\footnotesize

\lohead{\textsc{register}}

% Definiere theindex-Environment komplett neu ohne reledmac
\makeatletter
\renewenvironment{theindex}{%
  \section*{\indexname}%
  \setlength{\parindent}{0pt}%
  \setlength{\parskip}{0pt plus 0.3pt}%
  \let\item\@idxitem
}{%
  \clearpage
}
\makeatother

\IfFileExists{\jobname-pw.ind}{\input{\jobname-pw.ind}}{}

\end{document}

      