%% latex-korrekturansicht-vorspann.tex
%% Vorspann für die Korrekturansicht.
%% Lädt die gemeinsame Datei latex-vorspann.tex mit gesetztem Schalter.

\newif\ifkorrekturansicht
\korrekturansichttrue

\input{../tex-inputs/latex-vorspann}


               \section[Arthur Schnitzler an Hermann Bahr, 18. 9. 1905]{ Arthur Schnitzler an Hermann Bahr, 18. 9. 1905}\nopagebreak\mylabel{v}\rehead{ }\normalsize\beginnumbering\briefempfaengerindex{Bahr, Hermann@\textsc{Bahr, Hermann}!zzzSchnitzler, Arthur@\emph{von Arthur Schnitzler}!1905-09-181@{18. 9. 1905}|(be} \toendnotes[C]{\smallbreak\pagebreak[2]} \Standort{TMW, HS AM 23377 Ba.}
\physDesc{Kartenbrief
\newline{}Handschrift: schwarze Tinte, deutsche Kurrent\newline{}Versand: 1) Stempel: »\nobreak{}Wien, 19. IX. 05\nobreak{}«.  2) Stempel: »\nobreak{}\oindex{XIII., Hietzing@\textbf{XIII., Hietzing}, \emph{Bezirk (A.BZK)}|pwk}Wien 13/7, 19. 9. 05\nobreak{}«. \newline{}Ordnung: Lochung }\buchAbdrucke{\weitereDrucke{1) \emph{18. 9. 1905.} In: Arthur Schnitzler: \emph{The Letters of Arthur Schnitzler to Hermann Bahr}. Edited, annotated, and with an introduction, by Donald G.
                        Daviau. Chapel Hill: \emph{The University of North Carolina Press} 1978, S. 91 (University of North Carolina studies in the Germanic languages
                        and literatures, 89).} \weitereDrucke{2) Hermann Bahr, Arthur Schnitzler: \emph{Briefwechsel, Aufzeichnungen, Dokumente (1891–1931)}. Hg. Kurt Ifkovits und Martin Anton Müller. Göttingen: \emph{Wallstein} 2018, S. 353.} }\toendnotes[C]{\smallbreak}\pstart{}{\pb}\textcolor{gray}{\textbf{Dr. Arthur Schnitzler}}\pend{}\pstart{}\textcolor{gray}{\textbf{\textcolor{pink}{Wien XVIII. Spoettelgasse 7}{}\ledrightnote{\textcolor{pink}{Edmund-Weiß-Gasse}}.}}\pend{}{\bigskip}\pstart{}\textsc{Herrn Hermann Bahr}\pend{}\pstart{}\textsc{\textcolor{pink}{Wien Ober St Veit}{}\ledrightnote{\textcolor{pink}{Ober Sankt Veit}}}\pend{}\pstart{}\textsc{\textcolor{pink}{Veitlissengasse}{}\ledrightnote{\textcolor{pink}{Veitlissengasse}}}\pend{}{\bigskip}\pstart
           \raggedleft{}{\pb}18/9 905\pend
           \pstart
           lieber Hermann, herzlichen Dank für deinen Brief. Es iſt mir ſehr
               wahrſcheinlich, daſs du in deinem Bedenken gegen den \textcolor{green}{2. Akt}{}\ledrightnote{→\textcolor{green}{Der Ruf des Lebens. Schauspiel in drei Akten}} recht haſt – vielleicht ſpricht sogar \uline{dafür}, dſs er beim Vorleſen i{\geminationm}er am ſtärkſten wirkte. Ob es aber in der Oekonomie
               gerade dieſes Stückes (ſo wie es mir eben eingefallen ist) \introOben{}möglich \substVorne{}\textsuperscript{iſt}\substDazwischen{}u\substHinten{} geſta\textcolor{gray}{t}tet iſt\introOben{} die Figuren dieſes Aktes, deren
                  (we{\geminationn} ich den Ausdruck erfinden darf) Fernhaftigkeit
               nicht allein im Unvermögen des Autors begründet liegt, realer zu machen, das iſt die
               Frage. (Bisher hat von allen Figuren immer der \damage{Ob}erſt am stärkſten gewirkt. Nun ja, gewirkt.)\pend
           \pstart
           \label{K_L01549_1v}\edtext{Freitag fahr ich vielleicht auf 3–6 Tage
                  fort}{\lemma{\textnormal{\emph{Freitag … fort}}}\Cendnote{\textnormal{Schnitzler fuhr tatsächlich am
                     Freitag, den 22., auf den \textcolor{pink}{Semmering} und kehrte am Donnerstag, den 26. 9. 1905,
                  zurück.}}}\label{K_L01549_1h}; aber da{\geminationn} muſs man ſich doch wirklich
               endlich, endlich ſehn. Das \textcolor{green}{\textsc{Mscrpt}}{}\ledrightnote{→\textcolor{green}{Der Ruf des Lebens. Schauspiel in drei Akten}}{ }ſchicke mir gelegentlich, da ich nur 1 Ex.
               daheim habe, u das wieder fortſchicken muſs. – \pend
           \pstart
           Herzlichſt dein{\\[\baselineskip]}\spacefill\mbox{A.}\pend
           \leftskip=0em{}\endnumbering\briefempfaengerindex{Bahr, Hermann@\textsc{Bahr, Hermann}!zzzSchnitzler, Arthur@\emph{von Arthur Schnitzler}!1905-09-181@{18. 9. 1905}|)be}\mylabel{h}  \normalsize

\doendnotes{C}
\bigskip
\vfill

\clearpage

\footnotesize

\lohead{\textsc{register}}

% Definiere theindex-Environment komplett neu ohne reledmac
\makeatletter
\renewenvironment{theindex}{%
  \section*{\indexname}%
  \setlength{\parindent}{0pt}%
  \setlength{\parskip}{0pt plus 0.3pt}%
  \let\item\@idxitem
}{%
  \clearpage
}
\makeatother

\IfFileExists{\jobname-pw.ind}{\input{\jobname-pw.ind}}{}

\end{document}

      