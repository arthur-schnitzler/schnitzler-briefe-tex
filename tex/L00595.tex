%% latex-korrekturansicht-vorspann.tex
%% Vorspann für die Korrekturansicht.
%% Lädt die gemeinsame Datei latex-vorspann.tex mit gesetztem Schalter.

\newif\ifkorrekturansicht
\korrekturansichttrue

\input{../tex-inputs/latex-vorspann}


               \section[Arthur Schnitzler an Hermann Bahr, {[}20. 9. 1896?{]}]{ Arthur Schnitzler an Hermann Bahr, {[}20. 9. 1896?{]}}\nopagebreak\mylabel{v}\rehead{ }\normalsize\beginnumbering\briefempfaengerindex{Bahr, Hermann@\textsc{Bahr, Hermann}!zzzSchnitzler, Arthur@\emph{von Arthur Schnitzler}!1896-09-201@{{[}20. 9. 1896?{]}}|(be} \toendnotes[C]{\smallbreak\pagebreak[2]} \Standort{TMW, HS AM 60153 Ba.}
\physDesc{Briefkarte
\newline{}Handschrift: Bleistift, deutsche Kurrent\newline{}Ordnung: Lochung }\buchAbdrucke{\weitereDrucke{1) \emph{[20. 9. 1896?], Abschrift.} In: Arthur Schnitzler: \emph{The Letters of Arthur Schnitzler to Hermann Bahr}. Edited, annotated, and with an introduction, by Donald G.
                        Daviau. Chapel Hill: \emph{The University of North Carolina Press} 1978, S. 59 (University of North Carolina studies in the Germanic languages
                        and literatures, 89).} \weitereDrucke{2) Hermann Bahr, Arthur Schnitzler: \emph{Briefwechsel, Aufzeichnungen, Dokumente (1891–1931)}. Hg. Kurt Ifkovits und Martin Anton Müller. Göttingen: \emph{Wallstein} 2018, S. 126.} }\toendnotes[C]{\smallbreak}\pstart
           \raggedleft{}{\pb}\label{K_L00595_1v}\edtext{So{\geminationn}tag abd}{\lemma{\textnormal{\emph{Sotag abd}}}\Cendnote{\textnormal{undatierte Briefkarte; am 14. 9. 1896 traf \textcolor{blue}{Schnitzler}{ }\textcolor{blue}{Beer-Hofmann} nicht in \textcolor{pink}{Baden} an, worauf ihm dieser mitteilte, er werde »am
                           24. in \textcolor{pink}{Wien}{ }sein« (Richard Beer-Hofmann an Arthur Schnitzler,
               15. 9. 1896). Der 20. 9. 1896 ist ein
                     Sonntag.}}}\label{K_L00595_1h}\pend
           \pstart
           Lieber Hermann, als ich geſtern Abend fragte, wußte man noch nichts
               von deiner Sendung, jetzt eben beim Nachhauſegehen übergab mir die \textcolor{blue}{Hausmeiſterin}{}\ledrightnote{→\textcolor{blue}{?? [Hausmeisterin von Arthur Schnitzler]}} das Paket; da dein Brief mit der
               Adreſſe mit eingeschloſſen war, hatte sie nicht gewußt, daſs es für mich gehörte. – \substVorne{}\textsuperscript{h}\substDazwischen{}H\substHinten{}erzlichen Dank! {\pb}\textcolor{blue}{Richard}{}\ledrightnote{\textcolor{blue}{Richard Beer-Hofmann}} wohnt \textsc{\textcolor{pink}{Baden, \label{K_L00595_2v}\edtext{Franzensgasse}{\lemma{\textnormal{\emph{Franzensgasse}}}\Cendnote{\textnormal{Ein Irrtum \textcolor{blue}{Schnitzler}s, \textcolor{blue}{Beer-Hofmann} wohnte in der \textcolor{pink}{Franzensstraße}.}}}\label{K_L00595_2h} 54}{}\ledrightnote{\textcolor{pink}{Kaiser-Franz-Ring}}}, ko{\geminationm}t am 24. herein. – \pend
           \pstart
           Herzlichen Gruſs dein{\\[\baselineskip]}\spacefill\mbox{Arth}\pend
           \leftskip=0em{}\endnumbering\briefempfaengerindex{Bahr, Hermann@\textsc{Bahr, Hermann}!zzzSchnitzler, Arthur@\emph{von Arthur Schnitzler}!1896-09-201@{{[}20. 9. 1896?{]}}|)be}\mylabel{h}  \normalsize

\doendnotes{C}
\bigskip
\vfill

\clearpage

\footnotesize

\lohead{\textsc{register}}

% Definiere theindex-Environment komplett neu ohne reledmac
\makeatletter
\renewenvironment{theindex}{%
  \section*{\indexname}%
  \setlength{\parindent}{0pt}%
  \setlength{\parskip}{0pt plus 0.3pt}%
  \let\item\@idxitem
}{%
  \clearpage
}
\makeatother

\IfFileExists{\jobname-pw.ind}{\input{\jobname-pw.ind}}{}

\end{document}

      