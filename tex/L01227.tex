%% latex-korrekturansicht-vorspann.tex
%% Vorspann für die Korrekturansicht.
%% Lädt die gemeinsame Datei latex-vorspann.tex mit gesetztem Schalter.

\newif\ifkorrekturansicht
\korrekturansichttrue

\input{../tex-inputs/latex-vorspann}


               \section[Hugo von Hofmannsthal und Arthur Schnitzler an Hermann Bahr, 3. 7. 1902]{ Hugo von Hofmannsthal und Arthur Schnitzler an Hermann Bahr,
               3. 7. 1902}\nopagebreak\mylabel{v}\rehead{ }\normalsize\beginnumbering\briefempfaengerindex{Bahr, Hermann@\textsc{Bahr, Hermann}!zzzSchnitzler, Arthur@\emph{von Arthur Schnitzler}!1902-07-031@{3. 7. 1902}|(be}\briefempfaengerindex{Bahr, Hermann@\textsc{Bahr, Hermann}!zzzHofmannsthal, Hugo von@\emph{von Hugo von Hofmannsthal}!1902-07-031@{3. 7. 1902}|(be} \toendnotes[C]{\smallbreak\pagebreak[2]} \Standort{TMW, HS AM 49090 Ba.}
\physDesc{Bildpostkarte
\newline{}Handschrift Hugo von Hofmannsthal: Bleistift, deutsche Kurrent\newline{}Handschrift Arthur Schnitzler: Bleistift, deutsche Kurrent\newline{}Versand: 1) Stempel: »\nobreak{}\oindex{Matrei am Brenner@\textbf{Matrei am Brenner}, \emph{Besiedelter Ort (A.BSO)}|pwk}Deutsch-Matrei, 3/7 {[}1902{]}\nobreak{}«.  2) Stempel: »\nobreak{}\oindex{XIII., Hietzing@\textbf{XIII., Hietzing}, \emph{Bezirk (A.BZK)}|pwk}Wien 13, 4. 7. 02, 11. V, Bestellt\nobreak{}«. \newline{}Ordnung: Lochung }\buchAbdrucke{\weitereDrucke{Hermann Bahr, Arthur Schnitzler: \emph{Briefwechsel, Aufzeichnungen, Dokumente (1891–1931)}. Hg. Kurt Ifkovits und Martin Anton Müller. Göttingen: \emph{Wallstein} 2018, S. 240.} }\toendnotes[C]{\smallbreak}\pstart{}{\pb}\textsc{Herrn Hermann Bahr}\pend{}\pstart{}\textsc{Redacteur}\pend{}\pstart{}\textsc{\textcolor{pink}{Wien}{}\ledrightnote{\textcolor{pink}{Wien}}}\pend{}\pstart{}\textcolor{pink}{\textsc{XIII\textsubscript{7} Veitlissengasse}}{}\ledrightnote{\textcolor{pink}{Veitlissengasse}}\pend{}\pstart{}\textsc{in \textcolor{pink}{Ober St Veit}{}\ledrightnote{\textcolor{pink}{Ober Sankt Veit}}.}\pend{}{\bigskip}\pstart
           \noindent{}\centering{}\textcolor{gray}{\textbf{{\pb}\textcolor{pink}{MATREI}{}\ledrightnote{\textcolor{pink}{Matrei am Brenner}}.}}\pend
           \pstart
           \label{T_L01227_1v}\edtext{Das was Sie}{\lemma{\textnormal{\emph{Das was Sie}}}\Cendnote{\textnormal{quer am rechten Rand}}}\label{T_L01227_1h} über’n \label{K_L01227_1v}\edtext{\textcolor{green}{Automobil}{}\ledrightnote{→\textcolor{green}{Entgegen}}}{\lemma{\textnormal{\emph{Automobil}}}\Cendnote{\textnormal{In \emph{\textcolor{green}{Entgegen}} (\emph{\textcolor{brown}{Neues Wiener Tagblatt}}, Jg. 36, Nr. 179,
                        1. 7. 1902, S. 1–2) schildert \textcolor{blue}{Bahr} ein Automobilrennen.}}}\label{K_L01227_1h} geſchrieben haben, war \uline{ſehr} gut.\pend
           \pstart
           3 VII 02.\pend
           \pstart Viele Grüße\hspace*{1.5em}\spacefill\mbox{Hugo}\pend{}\pstart
           \noindent{}{[}hs. Schnitzler:{]} \label{T_L01227_2v}\edtext{Ich auch}{\lemma{\textnormal{\emph{Ich auch}}}\Cendnote{\textnormal{am rechten Rand auf dem Kopf}}}\label{T_L01227_2h}\pend
           \pstart \spacefill\mbox{Arthur.}\pend{}\endnumbering\briefempfaengerindex{Bahr, Hermann@\textsc{Bahr, Hermann}!zzzSchnitzler, Arthur@\emph{von Arthur Schnitzler}!1902-07-031@{3. 7. 1902}|)be}\briefempfaengerindex{Bahr, Hermann@\textsc{Bahr, Hermann}!zzzHofmannsthal, Hugo von@\emph{von Hugo von Hofmannsthal}!1902-07-031@{3. 7. 1902}|)be}\mylabel{h}  \normalsize

\doendnotes{C}
\bigskip
\vfill

\clearpage

\footnotesize

\lohead{\textsc{register}}

% Definiere theindex-Environment komplett neu ohne reledmac
\makeatletter
\renewenvironment{theindex}{%
  \section*{\indexname}%
  \setlength{\parindent}{0pt}%
  \setlength{\parskip}{0pt plus 0.3pt}%
  \let\item\@idxitem
}{%
  \clearpage
}
\makeatother

\IfFileExists{\jobname-pw.ind}{\input{\jobname-pw.ind}}{}

\end{document}

      