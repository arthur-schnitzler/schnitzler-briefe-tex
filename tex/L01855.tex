%% latex-korrekturansicht-vorspann.tex
%% Vorspann für die Korrekturansicht.
%% Lädt die gemeinsame Datei latex-vorspann.tex mit gesetztem Schalter.

\newif\ifkorrekturansicht
\korrekturansichttrue

\input{../tex-inputs/latex-vorspann}


               \section[Richard Beer-Hofmann an Arthur Schnitzler, 12. 7. 1909]{ Richard Beer-Hofmann an Arthur Schnitzler, 12. 7. 1909}\nopagebreak\mylabel{v}\rehead{ }\normalsize\beginnumbering\briefempfaengerindex{Schnitzler, Arthur@\textsc{Schnitzler, Arthur}!zzzBeer-Hofmann, Richard@\emph{von Richard Beer-Hofmann}!1909-07-121@{12. 7. 1909}|(be} \toendnotes[C]{\smallbreak\pagebreak[2]} \Standort{CUL, Schnitzler, B 8.}
\physDesc{Kartenbrief
\newline{}Handschrift: schwarze Tinte, lateinische Kurrent\newline{}Versand: 1) Stempel: »\nobreak{}\oindex{Pichl am See@\textbf{Pichl am See}, \emph{http://www.geonames.org/ontologyP.PPL}|pwk}Pichl am Mondsee, 13 7 09\nobreak{}«.  2) Stempel: »\nobreak{}\oindex{Edlach@\textbf{Edlach}, \emph{Besiedelter Ort (A.BSO)}|pwk}Edlach b. Reichenau in N.OE, 14 7 09, 8–12V\nobreak{}«. 
\newline{}Schnitzler: mit Bleistift beschriftet: »\textsc{Beerhof}« \newline{}Ordnung: mit Bleistift von unbekannter Hand nummeriert:
                              »220« }\buchAbdrucke{\weitereDrucke{Arthur Schnitzler, Richard Beer-Hofmann: \emph{Briefwechsel 1891–1931}. Hg. Konstanze Fliedl. Wien, Zürich: \emph{Europaverlag} 1992, S. 194.} }\toendnotes[C]{\smallbreak}\pstart{}{\pb}Herrn\pend{}\pstart{}D\textsuperscript{r} Arthur Schnitzler\pend{}\pstart{}\textcolor{pink}{Edlach}{}\ledrightnote{\textcolor{pink}{Edlach}}\pend{}\pstart{}bei \textcolor{pink}{Reichenau}{}\ledrightnote{\textcolor{pink}{Reichenau an der Rax}}\pend{}{\bigskip}\pstart
           \raggedleft{}{\pb}12./VII 09\pend
           \pstart
           Lieber Arthur! In 12 Tagen, 9 Regentage. Der Regen hält an, 5° \strikeout{Kälte} Wärme (?) am Nachmittag. Wir wollen schon am
                  15. an den \textcolor{pink}{Lido}{}\ledrightnote{\textcolor{pink}{Lido}}, u. sehnen uns nach
               Hitze. Vielleicht sind wir von 1–15 Aug. in \textcolor{pink}{Wien}{}\ledrightnote{\textcolor{pink}{Wien}}. Herzliche Grüsse Ihnen und Ihrer \textcolor{blue}{Frau}{}\ledrightnote{→\textcolor{blue}{Olga Schnitzler}}.\pend
           \pstart \label{T_L01855-1v}\edtext{Ihr \spacefill\mbox{Richard.}}{\lemma{\textnormal{\emph{Ihr Richard.}}}\Cendnote{\textnormal{quer am rechten Rand}}}\label{T_L01855-1h}\pend{}\endnumbering\briefempfaengerindex{Schnitzler, Arthur@\textsc{Schnitzler, Arthur}!zzzBeer-Hofmann, Richard@\emph{von Richard Beer-Hofmann}!1909-07-121@{12. 7. 1909}|)be}\mylabel{h}  \normalsize

\doendnotes{C}
\bigskip
\vfill

\clearpage

\footnotesize

\lohead{\textsc{register}}

% Definiere theindex-Environment komplett neu ohne reledmac
\makeatletter
\renewenvironment{theindex}{%
  \section*{\indexname}%
  \setlength{\parindent}{0pt}%
  \setlength{\parskip}{0pt plus 0.3pt}%
  \let\item\@idxitem
}{%
  \clearpage
}
\makeatother

\IfFileExists{\jobname-pw.ind}{\input{\jobname-pw.ind}}{}

\end{document}

      