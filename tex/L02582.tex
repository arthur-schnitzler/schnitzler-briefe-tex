%% latex-korrekturansicht-vorspann.tex
%% Vorspann für die Korrekturansicht.
%% Lädt die gemeinsame Datei latex-vorspann.tex mit gesetztem Schalter.

\newif\ifkorrekturansicht
\korrekturansichttrue

\input{../tex-inputs/latex-vorspann}


               \section[Karin Michaëlis an Arthur Schnitzler, 21. 10. 1912]{ Karin Michaëlis an Arthur Schnitzler, 21. 10. 1912}\nopagebreak\mylabel{v}\rehead{ }\normalsize\beginnumbering\briefempfaengerindex{Schnitzler, Arthur@\textsc{Schnitzler, Arthur}!zzzMichaelis, Karin@\emph{von Karin Michaëlis}!1912-10-211@{21. 10. 1912}|(be} \toendnotes[C]{\smallbreak\pagebreak[2]} \Standort{DLA, A:Schnitzler, HS.1985.1.4092.}
\physDesc{Brief, 1 Blatt, 2 Seiten
\newline{}Handschrift: schwarze Tinte, lateinische Kurrent
\newline{}Schnitzler: mit Bleistift »\textsc{Michaeli\textcolor{gray}{s}}« }\toendnotes[C]{\smallbreak}\pstart
           \centering{}{\pb}21 Oct. 1912.\pend
           \pstart\center{}Sehr geehrter Hrr Dr.!\pend\pstart
           Mein Freund \textcolor{blue}{Peter Nansen}{}\ledrightnote{\textcolor{blue}{Peter Nansen}} aus \textcolor{pink}{Kopenhagen}{}\ledrightnote{\textcolor{pink}{Kopenhagen}} ist \textcolor{pink}{hier}{}\ledrightnote{→\textcolor{pink}{Wien}} und hat den Wunsch (den ich also auch habe) Sie bald zu sehen.\pend
           \pstart
           Wollen Sie mir die Freude machen, \label{K_L02582-1v}\edtext{Morgen}{\lemma{\textnormal{\emph{Morgen}}}\Cendnote{\textnormal{Ein
                     Treffen mit \textcolor{blue}{Karin Michaëlis}, \textcolor{blue}{Peter Nansen} und anderen fand jedenfalls am
                        25. 10. 1912{ }abends statt.}}}\label{K_L02582-1h}, \uline{Dienstag} um \uline{2 Uhr} mit uns im Hause der \textcolor{blue}{Freundin}{}\ledrightnote{→\textcolor{blue}{Eugenie Schwarzwald}}, bei der ich wohne (Frau Dr. \textcolor{blue}{Schwarzwald}{}\ledrightnote{\textcolor{blue}{Eugenie Schwarzwald}}{ }\textcolor{pink}{VIII Josefstädterstrasse 68}{}\ledrightnote{\textcolor{pink}{Josefstädter Straße}}) zu
               frühstücken?\pend
           \pstart
           {\pb}Ich hoffe von Herzen, dass Sie noch nicht vergeben sind
               und bitte um freundliche telefonische (N. 21237) \strikeout{be\textcolor{gray}{n}} Nachricht, ob wir die Freude haben werden, Sie zu
               begrüssen.\pend
           \pstart
           Ihre verehrungsvoll ergebene{\\[\baselineskip]}\spacefill\mbox{Karin Michaëlis
               Stangeland}\pend
           \leftskip=0em{}\endnumbering\briefempfaengerindex{Schnitzler, Arthur@\textsc{Schnitzler, Arthur}!zzzMichaelis, Karin@\emph{von Karin Michaëlis}!1912-10-211@{21. 10. 1912}|)be}\mylabel{h}  \normalsize

\doendnotes{C}
\bigskip
\vfill

\clearpage

\footnotesize

\lohead{\textsc{register}}

% Definiere theindex-Environment komplett neu ohne reledmac
\makeatletter
\renewenvironment{theindex}{%
  \section*{\indexname}%
  \setlength{\parindent}{0pt}%
  \setlength{\parskip}{0pt plus 0.3pt}%
  \let\item\@idxitem
}{%
  \clearpage
}
\makeatother

\IfFileExists{\jobname-pw.ind}{\input{\jobname-pw.ind}}{}

\end{document}

      