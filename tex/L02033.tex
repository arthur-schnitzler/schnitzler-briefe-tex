%% latex-korrekturansicht-vorspann.tex
%% Vorspann für die Korrekturansicht.
%% Lädt die gemeinsame Datei latex-vorspann.tex mit gesetztem Schalter.

\newif\ifkorrekturansicht
\korrekturansichttrue

\input{../tex-inputs/latex-vorspann}


               \section[Georg Brandes an Arthur Schnitzler, 6. 10. 1911]{ Georg Brandes an Arthur Schnitzler, 6. 10. 1911}\nopagebreak\mylabel{v}\rehead{ }\normalsize\beginnumbering\briefempfaengerindex{Schnitzler, Arthur@\textsc{Schnitzler, Arthur}!zzzBrandes, Georg@\emph{von Georg Brandes}!1911-10-061@{6. 10. 1911}|(be} \toendnotes[C]{\smallbreak\pagebreak[2]} \Standort{CUL, Schnitzler, B 17.}
\physDesc{Brief, 1 Blatt, 3 Seiten
\newline{}Handschrift: schwarze Tinte, lateinische Kurrent
\newline{}Schnitzler: 1) mit Bleistift beschriftet: »\textsc{Brandes}« 2) mit rotem Buntstift eine Unterstreichung\newline{}Ordnung: von unbekannter Hand nummeriert: »36« }\buchAbdrucke{\weitereDrucke{Georg Brandes, Arthur Schnitzler: \emph{Ein Briefwechsel}. Hg. Kurt Bergel. Bern: \emph{Francke} 1956, S. 101.} }\toendnotes[C]{\smallbreak}\pstart
           \raggedleft{}{\pb}\textcolor{pink}{Kopenhagen}{}\ledrightnote{\textcolor{pink}{Kopenhagen}} (genügend Adresse){\\}6 October 11\pend
           \pstart{}Verehrter und lieber Freund\pend\pstart
           Graf \textcolor{blue}{Prozor}{}\ledrightnote{\textcolor{blue}{Moritz Prozor}}, russischer Diplomat, vieljähriger
               Uebersetzer \textcolor{blue}{Ibsens}{}\ledrightnote{\textcolor{blue}{Henrik Ibsen}} ins \textcolor{pink}{Französische}{}\ledrightnote{\textcolor{pink}{Frankreich}} – er \strikeout{hatte} hat zur
                  \textcolor{blue}{Frau}{}\ledrightnote{→\textcolor{blue}{Märta Margareta Prozor}} eine schwedische Gräfin
               und kennt unsere Sprachen – hat eine \textcolor{blue}{Tochter}{}\ledrightnote{→\textcolor{blue}{Grete Prozor}}, die durch die Wirksamkeit des Vaters \textcolor{blue}{Ibsen}{}\ledrightnote{\textcolor{blue}{Henrik Ibsen}}-Enthusiastin und \textcolor{blue}{\uline{Ibsen}}{}\ledrightnote{\textcolor{blue}{Henrik Ibsen}}\uline{-Darstellerin} geworden ist.\pend
           \pstart
           Fräulein \textcolor{blue}{Prozor}{}\ledrightnote{\textcolor{blue}{Grete Prozor}}{ }\label{K_L02033_1v}\edtext{soll am \uline{12\textsuperscript{ten}} in \textcolor{pink}{Wien}{}\ledrightnote{\textcolor{pink}{Wien}}{ }\textcolor{green}{Hedda}{}\ledrightnote{→\textcolor{green}{Hedda Gabler}} spielen}{\lemma{\textnormal{\emph{soll … spielen}}}\Cendnote{\textnormal{Obwohl \emph{\textcolor{green}{Hedda
                     Gabler}} in der Presse als Matinée-Veranstaltung im \textcolor{pink}{Carl-Theater} im Rahmen des Gastspiels von \textcolor{blue}{Suzanne Desprès} für den 12. 10. 1911
                  angekündigt wurde, ließen sich keine Kritiken zu dieser Inszenierung auffinden. Am
                  gleichen Abend spielte \textcolor{blue}{Greta Prozor} in \emph{\textcolor{green}{La Vie de Bohême}} von \textcolor{blue}{Théodor Barrière} und \textcolor{blue}{Henri
                     Murger} die Rolle der \textcolor{green}{Madame
                     de Rouvres}. In \textcolor{blue}{Ibsen}s \emph{\textcolor{green}{Nora}} hatte sie am 8. 10. 1911 die Rolle
                  der \textcolor{green}{Frau Linden}
                  gespielt.}}}\label{K_L02033_1h}. Der \textcolor{blue}{Vater}{}\ledrightnote{→\textcolor{blue}{Moritz Prozor}}
               hat mich wiederholt gebeten, ihr die Bahn zu ebnen durch einen Artikel in der \textcolor{brown}{N. fr. Presse}{}\ledrightnote{\textcolor{brown}{Neue Freie Presse}}. Ich antworte ihm 1) dass ich in
               keinerlei Verbindung mit der \textcolor{brown}{N. fr. Presse}{}\ledrightnote{\textcolor{brown}{Neue Freie Presse}} stehe
               2) dass ich seine \textcolor{blue}{Tochter}{}\ledrightnote{→\textcolor{blue}{Grete Prozor}} nie
               gesehen habe.\pend
           \pstart
           Er giebt nicht nach, fleht immer als alter Freund, ich möge jemand in \textcolor{pink}{Wien}{}\ledrightnote{\textcolor{pink}{Wien}} seinet{\pb}halber plagen.\pend
           \pstart
           Ich kenne Niemand, der mit Theatersachen irgendwie in Berührung steht, als Sie
               allein.\pend
           \pstart
           Meine Bitte ist also: fordern Sie, lieber Freund und in \textcolor{pink}{Wien}{}\ledrightnote{\textcolor{pink}{Wien}} gewiss nicht ohnmächtiger Meister, irgend einen Journalisten auf, das
               Frl. \textcolor{blue}{Prozor}{}\ledrightnote{\textcolor{blue}{Grete Prozor}} (in der Truppe von \textcolor{blue}{\uline{Suzanne Desprès}}{}\ledrightnote{\textcolor{blue}{Suzanne Desprès}}) zu interviewen und für Sie ein wenig Stimmung zu machen.\pend
           \leftskip=3em{}\pstart
           \noindent{}Dies \label{K_L02033_2v}\edtext{ma corvée}{\lemma{\textnormal{\emph{ma corvée}}}\Cendnote{\textnormal{französisch: meine lästige
               Pflicht}}}\label{K_L02033_2h}.\pend
           \leftskip=0em{}\pstart
           \noindent{}Aber ich mag nicht dies langweilige Zeug abschicken ohne Ihnen aufs Neue zu sagen,
               wie lieb ich Sie trotz der Entfernung und meines Alters habe, und wie gerne ich Sie
               wiedersähe.\pend
           \pstart
           Ich habe in \textcolor{pink}{Italien}{}\ledrightnote{\textcolor{pink}{Italien}}, \textcolor{pink}{Frankreich}{}\ledrightnote{\textcolor{pink}{Frankreich}} und \textcolor{pink}{Dänemark}{}\ledrightnote{\textcolor{pink}{Dänemark}} in diesem
               Frühjahr 3 Monate durch Venenentzündung verloren. Ich war jetzt in {\pb}\textcolor{pink}{Schottland}{}\ledrightnote{\textcolor{pink}{Schottland}}, weil die Universität \textcolor{pink}{St. Andrews}{}\ledrightnote{\textcolor{pink}{University of St. Andrews}} mich \label{K_L02033_3v}\edtext{à l’occasion}{\lemma{\textnormal{\emph{à l’occasion}}}\Cendnote{\textnormal{französisch: bei
                  Gelegenheit}}}\label{K_L02033_3h} seines 500 jährigen Bestehens zum Ehrendoktor ernannt hatte. So
               sah ich allerlei Malerisches in \textcolor{pink}{Schottland}{}\ledrightnote{\textcolor{pink}{Schottland}}.\pend
           \pstart
           Ich weiss jedoch, dass mehr Geist in \textcolor{pink}{Wien}{}\ledrightnote{\textcolor{pink}{Wien}} als in \textcolor{pink}{Edinburgh}{}\ledrightnote{\textcolor{pink}{Edinburgh}} ist, und Sie sind mir der eigentliche
               Vertreter dieses Geistes.\pend
           \pstart
           Ihr in alter Freundschaft ergebener{\\[\baselineskip]}\spacefill\mbox{Georg Brandes}\pend
           \leftskip=0em{}\pstart
           \noindent{}Ich habe leider Ihre Adresse vergessen, was den Brief verspäten wird\pend
           \endnumbering\briefempfaengerindex{Schnitzler, Arthur@\textsc{Schnitzler, Arthur}!zzzBrandes, Georg@\emph{von Georg Brandes}!1911-10-061@{6. 10. 1911}|)be}\mylabel{h}  \normalsize

\doendnotes{C}
\bigskip
\vfill

\clearpage

\footnotesize

\lohead{\textsc{register}}

% Definiere theindex-Environment komplett neu ohne reledmac
\makeatletter
\renewenvironment{theindex}{%
  \section*{\indexname}%
  \setlength{\parindent}{0pt}%
  \setlength{\parskip}{0pt plus 0.3pt}%
  \let\item\@idxitem
}{%
  \clearpage
}
\makeatother

\IfFileExists{\jobname-pw.ind}{\input{\jobname-pw.ind}}{}

\end{document}

      