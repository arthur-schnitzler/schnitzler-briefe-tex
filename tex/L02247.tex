%% latex-korrekturansicht-vorspann.tex
%% Vorspann für die Korrekturansicht.
%% Lädt die gemeinsame Datei latex-vorspann.tex mit gesetztem Schalter.

\newif\ifkorrekturansicht
\korrekturansichttrue

\input{../tex-inputs/latex-vorspann}


               \section[Arthur Schnitzler an Robert Adam, 23. 11. 1916]{ Arthur Schnitzler an Robert Adam, 23. 11. 1916}\nopagebreak\mylabel{v}\rehead{ }\normalsize\beginnumbering\briefempfaengerindex{Adam, Robert@\textsc{Adam, Robert}!zzzSchnitzler, Arthur@\emph{von Arthur Schnitzler}!1916-11-231@{23. 11. 1916}|(be} \toendnotes[C]{\smallbreak\pagebreak[2]} \Standort{DLA, 96.34.1/23.}
\physDesc{Briefkarte, Umschlag
\newline{}Schreibmaschine
\newline{}Handschrift: schwarze Tinte (\noindent{}Unterschrift)}\toendnotes[C]{\smallbreak}\pstart{}{\pb}\textcolor{gray}{\textbf{Dr. Arthur Schnitzler}}\pend{}\pstart{}\textcolor{gray}{\textbf{\textcolor{pink}{Wien XVIII. Sternwartestrasse 71}{}\ledrightnote{\textcolor{pink}{Sternwartestraße}}}}\pend{}{\bigskip}\pstart{}{\pb}Herrn Dr. Robert Adam Pollak\pend{}\pstart{}\textcolor{pink}{Wien XII}{}\ledrightnote{\textcolor{pink}{XII., Meidling}}.\pend{}\pstart{}\textcolor{pink}{Meidlinger Hauptstrasse 58}{}\ledrightnote{\textcolor{pink}{Meidlinger Hauptstraße}}.\pend{}{\bigskip}\pstart
           \noindent{}{\pb}\textcolor{gray}{\textbf{Dr. Arthur Schnitzler}}\hfill 23. 11. 1916.\pend
           \pstart
           \textcolor{gray}{\textbf{\textcolor{pink}{Wien XVIII. Sternwartestrasse 71}{}\ledrightnote{\textcolor{pink}{Sternwartestraße}}}}\pend
           \pstart\center{}Verehrter Herr Doktor.\pend\pstart
           Ohne mich gerade in prosodischen Fragen sonderlich kompetent zu fühlen, bin ich
                    selbstverständlich mit Vergnügen bereit Ihnen unverbindlich den gewünschten
                    Ratschlag zu erteilen. Vielleicht haben Sie die Freundlichkeit mir das \textcolor{green}{Manuscript}{}\ledrightnote{→\textcolor{green}{Wundervogel}} sowohl die
                    Alexandriner wie die Knittelverse zu übersenden und bald nachdem ich sie gelesen
                    wollen wir uns persönlich darüber weiter unterhalten. Ueber alles andere, \textcolor{green}{Jean Christophe}{}\ledrightnote{\textcolor{green}{Jean Christophe}} und \textcolor{blue}{\textcolor{green}{Dumas}{}\ledrightnote{→\textcolor{green}{Meine Memoiren}}}{}\ledrightnote{\textcolor{blue}{Alexandre père Dumas}} mündlich.\pend
           \pstart
           Herzlichst grüs{[}s{]}end{\\[\baselineskip]}Ihr{\\[\baselineskip]}\spacefill\mbox{{[}hs.:{]} Arthur Schnitzler}\pend
           \leftskip=0em{}\endnumbering\briefempfaengerindex{Adam, Robert@\textsc{Adam, Robert}!zzzSchnitzler, Arthur@\emph{von Arthur Schnitzler}!1916-11-231@{23. 11. 1916}|)be}\mylabel{h}  \normalsize

\doendnotes{C}
\bigskip
\vfill

\clearpage

\footnotesize

\lohead{\textsc{register}}

% Definiere theindex-Environment komplett neu ohne reledmac
\makeatletter
\renewenvironment{theindex}{%
  \section*{\indexname}%
  \setlength{\parindent}{0pt}%
  \setlength{\parskip}{0pt plus 0.3pt}%
  \let\item\@idxitem
}{%
  \clearpage
}
\makeatother

\IfFileExists{\jobname-pw.ind}{\input{\jobname-pw.ind}}{}

\end{document}

      