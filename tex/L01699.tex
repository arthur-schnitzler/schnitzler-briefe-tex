%% latex-korrekturansicht-vorspann.tex
%% Vorspann für die Korrekturansicht.
%% Lädt die gemeinsame Datei latex-vorspann.tex mit gesetztem Schalter.

\newif\ifkorrekturansicht
\korrekturansichttrue

\input{../tex-inputs/latex-vorspann}


               \section[Arthur und Olga Schnitzler an Richard Beer-Hofmann, 13. 8. 1907]{ Arthur und Olga Schnitzler an Richard Beer-Hofmann, 13. 8. 1907}\nopagebreak\mylabel{v}\rehead{ }\normalsize\beginnumbering\briefempfaengerindex{Beer-Hofmann, Richard@\textsc{Beer-Hofmann, Richard}!zzzSchnitzler, Olga@\emph{von Olga Schnitzler}!1907-08-131@{13. 8. 1907}|(be}\briefempfaengerindex{Beer-Hofmann, Richard@\textsc{Beer-Hofmann, Richard}!zzzSchnitzler, Arthur@\emph{von Arthur Schnitzler}!1907-08-131@{13. 8. 1907}|(be} \toendnotes[C]{\smallbreak\pagebreak[2]} \Standort{YCGL, MSS 31.}
\physDesc{Bildpostkarte
\newline{}Handschrift Arthur Schnitzler: Bleistift, deutsche Kurrent\newline{}Handschrift Olga Schnitzler: Bleistift, lateinische Kurrent\newline{}Versand: Stempel: »\nobreak{}\oindex{Misurina@\textbf{Misurina}, \emph{http://www.geonames.org/ontologyP.PPL}|pwk}Misurina, 13. 8. 1907\nobreak{}«.  \newline{}Ordnung: mit Bleistift von unbekannter Hand datiert: »12. 8.« }\toendnotes[C]{\smallbreak}\pstart{}{\pb}\textsc{Dr. Richard Beerhofmann}\pend{}\pstart{}\textcolor{pink}{Wien}{}\ledrightnote{\textcolor{pink}{Wien}}\pend{}\pstart{}\textsc{\textcolor{pink}{Hasenauerstr 59}{}\ledrightnote{\textcolor{pink}{Hasenauerstraße}}.}\pend{}\pstart{}\textsc{\textcolor{pink}{Austria}{}\ledrightnote{\textcolor{pink}{Österreich}}}\pend{}{\bigskip}\pstart
           \noindent{}\centering{}{\pb}\textcolor{gray}{\textbf{\textcolor{pink}{Lago di Misurina}{}\ledrightnote{\textcolor{pink}{Misurinasee}} (1755 m), \textcolor{pink}{Grand Hôtel}{}\ledrightnote{→\textcolor{pink}{Grand Hotel Misurina}}.}}\pend
           \pstart
           \raggedleft{}{\pb}\label{K_L01699_1v}\edtext{12. 8. 907}{\lemma{\textnormal{\emph{12. 8. 907}}}\Cendnote{\textnormal{Das \emph{\textcolor{green}{Tagebuch}} erwähnt den Ausflug erst für den 13. 8. 1907, weswegen dieses Datum
                        falsch sein dürfte.}}}\label{K_L01699_1h}\pend
           \pstart
           In \label{K_L01699_2v}\edtext{Erinnerung}{\lemma{\textnormal{\emph{Erinnerung}}}\Cendnote{\textnormal{siehe A. S.: \emph{Tagebuch}, 6. 8. 1898}}}\label{K_L01699_2h} und herzlichſt,\pend
           \pstart
           Ihr{\\[\baselineskip]}\spacefill\mbox{Arthur}\pend
           \leftskip=0em{}\pstart
           \noindent{}{[}hs. O. Schnitzler:{]} Die herzlichsten Grüsse!\pend
           \pstart \spacefill\mbox{Olga.}\pend{}\pstart
           \noindent{}{[}hs. Schnitzler:{]} Auf einer Fußwanderung \textcolor{pink}{\textsc{Schluderbach}}{}\ledrightnote{\textcolor{pink}{Carbonin}}{ }\textcolor{pink}{\textsc{Cortina}}{}\ledrightnote{\textcolor{pink}{Cortina d'Ampezzo}}\pend
           \pstart
           Morgen wieder in \textcolor{pink}{\textsc{Welsberg}}{}\ledrightnote{\textcolor{pink}{Welsberg-Taisten}}. –\pend
           \endnumbering\briefempfaengerindex{Beer-Hofmann, Richard@\textsc{Beer-Hofmann, Richard}!zzzSchnitzler, Olga@\emph{von Olga Schnitzler}!1907-08-131@{13. 8. 1907}|)be}\briefempfaengerindex{Beer-Hofmann, Richard@\textsc{Beer-Hofmann, Richard}!zzzSchnitzler, Arthur@\emph{von Arthur Schnitzler}!1907-08-131@{13. 8. 1907}|)be}\mylabel{h}  \normalsize

\doendnotes{C}
\bigskip
\vfill

\clearpage

\footnotesize

\lohead{\textsc{register}}

% Definiere theindex-Environment komplett neu ohne reledmac
\makeatletter
\renewenvironment{theindex}{%
  \section*{\indexname}%
  \setlength{\parindent}{0pt}%
  \setlength{\parskip}{0pt plus 0.3pt}%
  \let\item\@idxitem
}{%
  \clearpage
}
\makeatother

\IfFileExists{\jobname-pw.ind}{\input{\jobname-pw.ind}}{}

\end{document}

      