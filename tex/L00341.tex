%% latex-korrekturansicht-vorspann.tex
%% Vorspann für die Korrekturansicht.
%% Lädt die gemeinsame Datei latex-vorspann.tex mit gesetztem Schalter.

\newif\ifkorrekturansicht
\korrekturansichttrue

\input{../tex-inputs/latex-vorspann}


               \section[Arthur Schnitzler an Richard Beer-Hofmann, {[}zwischen 26.–29. 6. 1894?{]}]{ Arthur Schnitzler an Richard Beer-Hofmann, {[}zwischen
               26.–29. 6. 1894?{]}}\nopagebreak\mylabel{v}\rehead{ }\normalsize\beginnumbering\briefempfaengerindex{Beer-Hofmann, Richard@\textsc{Beer-Hofmann, Richard}!zzzSchnitzler, Arthur@\emph{von Arthur Schnitzler}!1894-06-261@{{[}zwischen
                  26.–29. 6. 1894?{]}}|(be} \toendnotes[C]{\smallbreak\pagebreak[2]} \Standort{YCGL, MSS 31.}
\physDesc{Brief, 1 Blatt, 3 Seiten, Umschlag
\newline{}Handschrift: Bleistift, deutsche Kurrent\newline{}Versand: ohne postalischen Übermittlungsvermerk \newline{}Ordnung: mit Bleistift von unbekannter Hand datiert:
                                 »Mai 1894« }\pstart{}{\pb}Herrn Dr. \textsc{Rich. Beer
                     Hofmann}\pend{}\pstart{}\textcolor{pink}{Wien}{}\ledrightnote{\textcolor{pink}{Wien}}\pend{}\pstart{}\textcolor{pink}{I. \textsc{Wollzeile} 15}{}\ledrightnote{\textcolor{pink}{Wollzeile}}\pend{}{\bigskip}\pstart{}{\pb}Lieber Richard,\pend\pstart
           \textcolor{blue}{Fels}{}\ledrightnote{\textcolor{blue}{Friedrich Michael Fels}} iſt eben bei mir, ſagt, hat das von Ihnen
               geſchickte noch nicht erhalten.\pend
           \pstart
           {\pb}\uline{Unbegreiflich! –}\pend
           \pstart
           Adreſſe \textcolor{pink}{\textsc{XVIII Exnergasse 3}}{}\ledrightnote{\textcolor{pink}{Krütznergasse}}, 3. Stock, Thür 22. –\pend
           \pstart
           – Bitte ſehr, ſenden Sie ſofort ab, we{\geminationn}
               Sie zufällig
               vergeſſen {\pb}haben.\pend
           \pstart
           Herzlich grüßt{\\[\baselineskip]}Ihr{\\[\baselineskip]}\spacefill\mbox{Arthur}\pend
           \leftskip=0em{}\endnumbering\briefempfaengerindex{Beer-Hofmann, Richard@\textsc{Beer-Hofmann, Richard}!zzzSchnitzler, Arthur@\emph{von Arthur Schnitzler}!1894-06-261@{{[}zwischen
                  26.–29. 6. 1894?{]}}|)be}\mylabel{h}  \normalsize

\doendnotes{C}
\bigskip
\vfill

\clearpage

\footnotesize

\lohead{\textsc{register}}

% Definiere theindex-Environment komplett neu ohne reledmac
\makeatletter
\renewenvironment{theindex}{%
  \section*{\indexname}%
  \setlength{\parindent}{0pt}%
  \setlength{\parskip}{0pt plus 0.3pt}%
  \let\item\@idxitem
}{%
  \clearpage
}
\makeatother

\IfFileExists{\jobname-pw.ind}{\input{\jobname-pw.ind}}{}

\end{document}

      