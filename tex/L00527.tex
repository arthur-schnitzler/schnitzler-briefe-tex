%% latex-korrekturansicht-vorspann.tex
%% Vorspann für die Korrekturansicht.
%% Lädt die gemeinsame Datei latex-vorspann.tex mit gesetztem Schalter.

\newif\ifkorrekturansicht
\korrekturansichttrue

\input{../tex-inputs/latex-vorspann}


               \section[Lou Andreas-Salomé an Arthur Schnitzler, {[}9. 1. 1896{]}]{ Lou Andreas-Salomé an Arthur Schnitzler, {[}9. 1. 1896{]}}\nopagebreak\mylabel{v}\rehead{ }\normalsize\beginnumbering\briefempfaengerindex{Schnitzler, Arthur@\textsc{Schnitzler, Arthur}!zzzAndreas-Salome, Lou@\emph{von Lou Andreas-Salomé}!1896-01-091@{{[}9. 1. 1896{]}}|(be} \toendnotes[C]{\smallbreak\pagebreak[2]} \Standort{CUL, Schnitzler, B 3.}
\physDesc{Kartenbrief
\newline{}Handschrift: schwarze Tinte, deutsche Kurrent\newline{}Versand: ohne postalischen Übermittlungsvermerk 
\newline{}Schnitzler: 1) mit Bleistift datiert: »9/1 96« 2) mit rotem Buntstift zwei Unterstreichungen\newline{}Ordnung: mit Bleistift von unbekannter Hand nummeriert:
                                    »15« }\toendnotes[C]{\smallbreak}\pstart{}{\pb}Herrn \textsc{D\textsuperscript{r}}\pend{}\pstart{}\textsc{Arthur Schnitzler.}\pend{}{\bigskip}\pstart
           \noindent{}{\pb}Lieber Herr \textsc{D\textsuperscript{r}}, glückliche \label{K_L00527_1v}\edtext{Reiſe}{\lemma{\textnormal{\emph{Reiſe}}}\Cendnote{\textnormal{Die Reise nach \textcolor{pink}{Frankfurt} fand von 10. 1. bis zum
                     15. 1. 1896 statt und führte auch nach \textcolor{pink}{Köln}.}}}\label{K_L00527_1h} und heiteres Wiederſehn! Für den \textcolor{pink}{\textsc{Griensteidl}}{}\ledrightnote{\textcolor{pink}{Café Griensteidl}} bin ich zu müde, ich ſchlafe ſo ſehr wenig und muß oft früh heraus. Ganz
               niedergeſchlagen hat mich in dieſen Tagen \textcolor{blue}{Hauptmann}{}\ledrightnote{\textcolor{blue}{Gerhart Hauptmann}}’s \label{K_L00527_2v}\edtext{\textcolor{green}{Mißerfolg}{}\ledrightnote{→\textcolor{green}{Florian Geyer. Die Tragödie des Bauernkrieges}}}{\lemma{\textnormal{\emph{Mißerfolg}}}\Cendnote{\textnormal{Die Uraufführung von \emph{\textcolor{green}{Florian Geyer}} fand am 4. 1. 1896 im \textcolor{pink}{Deutschen Theater in Berlin} statt.}}}\label{K_L00527_2h}, er ſelbſt iſt
               total herunter, nach den \textcolor{pink}{Berlin}{}\ledrightnote{\textcolor{pink}{Berlin}}er Briefen zu
               urtheilen. Und gerade jetzt hatte er einen großen Sieg ſo nöthig. Da \label{K_L00527_3v}\edtext{\textcolor{green}{\textcolor{blue}{\textsc{Halbe}}{}\ledrightnote{\textcolor{blue}{Max Halbe}}}{}\ledrightnote{→\textcolor{green}{Lebenswende}}}{\lemma{\textnormal{\emph{Halbe}}}\Cendnote{\textnormal{\emph{\textcolor{green}{Lebenswende}} hatte am 21. 1. 1896 im
                     \textcolor{pink}{Deutschen Theater} Uraufführung.}}}\label{K_L00527_3h} ihm
               zunächſt folgt, wird die \textcolor{green}{\textsc{Liebelei}}{}\ledrightnote{\textcolor{green}{Liebelei. Schauspiel in drei Akten}} alſo in den \label{K_L00527_4v}\edtext{Februar}{\lemma{\textnormal{\emph{Februar}}}\Cendnote{\textnormal{Die \textcolor{pink}{Berlin}er Premiere fand am 4. 2. 1896 im \textcolor{pink}{Deutschen Theater} statt.}}}\label{K_L00527_4h} fallen, ſolange kann ich wohl
               nicht hier bleiben, obſchon ich gern bliebe.\pend
           \pstart
           Grüßen Sie in \textcolor{pink}{Frankfurt}{}\ledrightnote{\textcolor{pink}{Frankfurt am Main}}{ }\textcolor{blue}{\textsc{Goldmann}}{}\ledrightnote{\textcolor{blue}{Paul Goldmann}}’s \label{K_L00527_5v}\edtext{\textcolor{blue}{Schwager}{}\ledrightnote{→\textcolor{blue}{Fedor Mamroth}}}{\lemma{\textnormal{\emph{Schwager}}}\Cendnote{\textnormal{Der Mediziner \textcolor{blue}{Josef Rosengart}, der Mann der Schwester \textcolor{blue}{Vally}}}}\label{K_L00527_5h}.\pend
           \pstart \spacefill\mbox{LouAS.}\pend{}\endnumbering\briefempfaengerindex{Schnitzler, Arthur@\textsc{Schnitzler, Arthur}!zzzAndreas-Salome, Lou@\emph{von Lou Andreas-Salomé}!1896-01-091@{{[}9. 1. 1896{]}}|)be}\mylabel{h}  \normalsize

\doendnotes{C}
\bigskip
\vfill

\clearpage

\footnotesize

\lohead{\textsc{register}}

% Definiere theindex-Environment komplett neu ohne reledmac
\makeatletter
\renewenvironment{theindex}{%
  \section*{\indexname}%
  \setlength{\parindent}{0pt}%
  \setlength{\parskip}{0pt plus 0.3pt}%
  \let\item\@idxitem
}{%
  \clearpage
}
\makeatother

\IfFileExists{\jobname-pw.ind}{\input{\jobname-pw.ind}}{}

\end{document}

      