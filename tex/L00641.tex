%% latex-korrekturansicht-vorspann.tex
%% Vorspann für die Korrekturansicht.
%% Lädt die gemeinsame Datei latex-vorspann.tex mit gesetztem Schalter.

\newif\ifkorrekturansicht
\korrekturansichttrue

\input{../tex-inputs/latex-vorspann}


               \section[Max Burckhard an Arthur Schnitzler, {[}18. 1. 1897{]}]{ Max Burckhard an Arthur Schnitzler, {[}18. 1. 1897{]}}\nopagebreak\mylabel{v}\rehead{ }\normalsize\beginnumbering\briefempfaengerindex{Schnitzler, Arthur@\textsc{Schnitzler, Arthur}!zzzBurckhard, Max Eugen@\emph{von Max Eugen Burckhard}!1897-01-181@{{[}18. 1. 1897{]}}|(be} \toendnotes[C]{\smallbreak\pagebreak[2]} \Standort{CUL, Schnitzler, B 20.}
\physDesc{Visitenkarte
\newline{}Handschrift: schwarze Tinte, deutsche Kurrent
\newline{}Schnitzler: mit Bleistift datiert: »18/1 9\textcolor{gray}{7}« \newline{}Ordnung: mit Bleistift von unbekannter Hand nummeriert:
                                        »1« }\toendnotes[C]{\smallbreak}\pstart
           \noindent{}\centering{}{\pb}\textcolor{gray}{\textbf{\textsc{D\textsuperscript{r.} Max Eugen
                                Burckhard}}}\pend
           \pstart
           \noindent{}\centering{}\textcolor{gray}{\textbf{\textsc{k. u. k. Director des k. k. \textcolor{pink}{Hofburgtheaters}{}\ledrightnote{\textcolor{pink}{Burgtheater}}}}}\pend
           \pstart
           erlaubt ſich ergebenſt zu ſeiner \label{K_L00641_1v}\edtext{Vorleſung}{\lemma{\textnormal{\emph{Vorleſung}}}\Cendnote{\textnormal{Eine Vorlesung fand am
                            Dienstag, den 19. 1. 1897 statt. \textcolor{blue}{Schnitzler} nahm teil.}}}\label{K_L00641_1h} in der \textcolor{brown}{Grillparzergeſellschaft}{}\ledrightnote{\textcolor{brown}{Grillparzer-Gesellschaft}}{ }Dienſtag Abend einzuladen. Bitte aber dieſe Einladung nur als
                    Zeichen meiner Verehrung nicht aber als zudringliche Zumuthung anzuſehen.\pend
           \endnumbering\briefempfaengerindex{Schnitzler, Arthur@\textsc{Schnitzler, Arthur}!zzzBurckhard, Max Eugen@\emph{von Max Eugen Burckhard}!1897-01-181@{{[}18. 1. 1897{]}}|)be}\mylabel{h}  \normalsize

\doendnotes{C}
\bigskip
\vfill

\clearpage

\footnotesize

\lohead{\textsc{register}}

% Definiere theindex-Environment komplett neu ohne reledmac
\makeatletter
\renewenvironment{theindex}{%
  \section*{\indexname}%
  \setlength{\parindent}{0pt}%
  \setlength{\parskip}{0pt plus 0.3pt}%
  \let\item\@idxitem
}{%
  \clearpage
}
\makeatother

\IfFileExists{\jobname-pw.ind}{\input{\jobname-pw.ind}}{}

\end{document}

      