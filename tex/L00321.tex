%% latex-korrekturansicht-vorspann.tex
%% Vorspann für die Korrekturansicht.
%% Lädt die gemeinsame Datei latex-vorspann.tex mit gesetztem Schalter.

\newif\ifkorrekturansicht
\korrekturansichttrue

\input{../tex-inputs/latex-vorspann}


               \section[Detlev von Liliencron an Arthur Schnitzler, 7. 5. 1894]{ Detlev von Liliencron an Arthur Schnitzler, 7. 5. 1894}\nopagebreak\mylabel{v}\rehead{ }\normalsize\beginnumbering\briefempfaengerindex{Schnitzler, Arthur@\textsc{Schnitzler, Arthur}!zzzLiliencron, Detlev von@\emph{von Detlev von Liliencron}!1894-05-071@{7. 5. 1894}|(be} \toendnotes[C]{\smallbreak\pagebreak[2]} \Standort{DLA, A:Schnitzler, HS.NZ85.1.3896, S. 1.}
\physDesc{maschinelle Abschrift}\toendnotes[C]{\smallbreak}\pstart
           \raggedleft{}{\pb}\textcolor{pink}{Altona (Elbe), Palmaille 5}{}\ledrightnote{\textcolor{pink}{Palmaille}},{\\}Den
                            7. 5. 94.\pend
           \pstart{}Sehr geehrter Herr Doctor,\pend\pstart
           Sie hatten die Güte mir Ihr Schauspiel: \textcolor{green}{Das
                        Märchen}{}\ledrightnote{\textcolor{green}{Das Märchen. Schauspiel in drei Aufzügen}} zu übersenden.\pend
           \pstart
           Ich habs jetzt in einem Zuge durchgelesen. Ich habe keine Ahnung von Dramatik.
                    Ich kann also nur das aussprechen, was ich beim Lesen gefühlt habe. Und das ist
                    in erster Reihe: dass ich bis zur letzten Zeile gefesselt war von Ihrem \textcolor{green}{Stück}{}\ledrightnote{→\textcolor{green}{Das Märchen. Schauspiel in drei Aufzügen}}, mit allen Fibern! Es
                    ist ein \textcolor{green}{Stück}{}\ledrightnote{→\textcolor{green}{Das Märchen. Schauspiel in drei Aufzügen}} aus \uline{unserm} Leben und aus dem Leben der \uline{Zukunft}. Ungemein fein haben Sie die Frauenfrage
                    gestreift. Ich \uline{sah} beim Lesen alle Ihre Menschen
                    ganz leibhaftig vor mir. Und ich hoffe sehr, dass das \textcolor{green}{Märchen}{}\ledrightnote{\textcolor{green}{Das Märchen. Schauspiel in drei Aufzügen}} nicht nur die Freien Bühnen beschäftigen wird,
                    sondern erst recht unsere grossen Theater, wenn diesen noch ein letzter Ernst
                    geblieben ist.\pend
           \pstart
           Ihr hochachtungsvoll ergebener{\\[\baselineskip]}\spacefill\mbox{Baron Detlev Liliencron.}\pend
           \leftskip=0em{}\endnumbering\briefempfaengerindex{Schnitzler, Arthur@\textsc{Schnitzler, Arthur}!zzzLiliencron, Detlev von@\emph{von Detlev von Liliencron}!1894-05-071@{7. 5. 1894}|)be}\mylabel{h}  \normalsize

\doendnotes{C}
\bigskip
\vfill

\clearpage

\footnotesize

\lohead{\textsc{register}}

% Definiere theindex-Environment komplett neu ohne reledmac
\makeatletter
\renewenvironment{theindex}{%
  \section*{\indexname}%
  \setlength{\parindent}{0pt}%
  \setlength{\parskip}{0pt plus 0.3pt}%
  \let\item\@idxitem
}{%
  \clearpage
}
\makeatother

\IfFileExists{\jobname-pw.ind}{\input{\jobname-pw.ind}}{}

\end{document}

      