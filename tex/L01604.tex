%% latex-korrekturansicht-vorspann.tex
%% Vorspann für die Korrekturansicht.
%% Lädt die gemeinsame Datei latex-vorspann.tex mit gesetztem Schalter.

\newif\ifkorrekturansicht
\korrekturansichttrue

\input{../tex-inputs/latex-vorspann}


               \section[Arthur Schnitzler an Hermann Bahr, 24.–25. 6. 1906]{ Arthur Schnitzler an Hermann Bahr, 24.–25. 6. 1906}\nopagebreak\mylabel{v}\rehead{ }\normalsize\beginnumbering\briefempfaengerindex{Bahr, Hermann@\textsc{Bahr, Hermann}!zzzSchnitzler, Arthur@\emph{von Arthur Schnitzler}!1906-06-242@{24. –25. 6. 1906}|(be} \toendnotes[C]{\smallbreak\pagebreak[2]} \Standort{TMW, HS AM 23379 Ba.}
\physDesc{Brief, 2 Blätter, 7 Seiten
\newline{}Handschrift: schwarze Tinte, deutsche Kurrent\newline{}Ordnung: Lochung }\buchAbdrucke{\weitereDrucke{1) Arthur Schnitzler: \emph{Briefe 1875–1912}. Hg. Therese Nickl und Heinrich Schnitzler. Frankfurt am Main: \emph{S. Fischer} 1981, S. 537–538.} \weitereDrucke{2) \emph{24. 6. 1906.} In: Arthur Schnitzler: \emph{The Letters of Arthur Schnitzler to Hermann Bahr}. Edited, annotated, and with an introduction, by Donald G.
                        Daviau. Chapel Hill: \emph{The University of North Carolina Press} 1978, S. 94–95 (University of North Carolina studies in the Germanic languages
                        and literatures, 89).} \weitereDrucke{3) Hermann Bahr, Arthur Schnitzler: \emph{Briefwechsel, Aufzeichnungen, Dokumente (1891–1931)}. Hg. Kurt Ifkovits und Martin Anton Müller. Göttingen: \emph{Wallstein} 2018, S. 379–380.} }\toendnotes[C]{\smallbreak}\pstart
           \raggedleft{}{\pb}\textcolor{pink}{Wien}{}\ledrightnote{\textcolor{pink}{Wien}}, 24. 6. 90\textcolor{gray}{6}\pend
           \pstart{}lieber Hermann, \pend\pstart
           ich finde deinen neuen \textcolor{green}{Einakter}{}\ledrightnote{→\textcolor{green}{Der Faun}}{ }ſehr intereſſant; feſſelnd vom e\damage{rſt}en bis zum letzten Wort, und halte (we{\geminationn} es
               nicht zu einem Skandal kommt, was man bei Bahren und Faunen nie wiſſen kann) auch
               eine ſtarke Bühnenwirkung für wahrſcheinlich. (Deine 3 Einakter müſſten zuſammen
               gegeben werden; \textcolor{green}{Faun}{}\ledrightnote{\textcolor{green}{Der Faun}} zum Schluſs, \textcolor{green}{Narr}{}\ledrightnote{\textcolor{green}{Der arme Narr}} zu Anfang, das »du kannst ja mitkommen«, {\pb}der Helmine am Schluſs
               bekäme dann ſeine beſondre Bedeutung.)\pend
           \pstart
           Man denkt natürlich ſo ein Stück weiter, wie man wirkliche Erlebniſſe weiter
               phantaſirt, und ſo habe ich auch einen zwei\damage{ten} u dritten Akt geſehen, die man vorläufig nicht wird ſpielen können. Der
               zweite Akt auf der ſteilen Bergwieſe. Falls du ihn ſchreiben ſollteſt, rathe ich dir,
               ihn nicht von \textcolor{blue}{Leſſing}{}\ledrightnote{\textcolor{blue}{Emil Lessing}} inſzeniren zu laſſen, der
               Orgien nur ein mäßiges Verſtändnis entgegenbringt, was {\pb}ſich im 4. Akt der \textcolor{green}{\textsc{Beatrice}}{}\ledrightnote{\textcolor{green}{Der Schleier der Beatrice. Schauspiel in fünf Akten}}{ }\label{K_L01604_1v}\edtext{ja{\geminationm}ervoll
                  erwieſen}{\lemma{\textnormal{\emph{jaervoll
                  erwieſen}}}\Cendnote{\textnormal{Die Anmerkung bezieht sich
                  auf die Inszenierung am \textcolor{pink}{Deutschen Theater in
                  Berlin}, die am 7. 3. 1903 Premiere hatte.}}}\label{K_L01604_1h}. Dieſer zweite
               Akt, der verſchiedentlich geführt werden könnte bekäme ſeinen ganzen Sinn natürlich
               nur durch die vollendeſte Rückſichtsloſigkeit. Alſo Bedingung: Unaufführbarkeit. Da
               für mich (wenigſtens wie ich das Stück weitergedacht habe) \textsc{\textcolor{green}{Helmine}{}\ledrightnote{→\textcolor{green}{Der Faun}}} die Heldin iſt, brächte der 3. Akt den ſeelischen Untergang oder Sieg der \textsc{\textcolor{green}{Helmine}{}\ledrightnote{→\textcolor{green}{Der Faun}}}. Man wird zu irgend etwas wahrſcheinlich nur reif, wenn man eigentlich dazu
               geboren war. Man kann ein Faun ſein; man ka{\geminationn}{ }{\pb}aber kein Faun werden.
               Man kann ein Hexchen und eine Nymphe ſein, aber man ka{\geminationn}
               es nicht werden. Ich bin nicht klar darüber, ob \textcolor{green}{Helmine}{}\ledrightnote{→\textcolor{green}{Der Faun}} das Recht auf die Welt gebracht hat, auf die
                  ſtei\damage{le} Bergwieſe zu wandern. Jedenfalls ſie eher als Edgar, wie ja die Frauen
               überhaupt mit den Urelementen verwandter ſind als die Männer. Es wäre auch zu
               bedenken, ob \textsc{Helmine} nicht irgend was, das man nur aus {\pb}ſeiner Natur heraus
               thun darf, \label{K_L01604_2v}\edtext{\textsc{par dépit}}{\lemma{\textnormal{\emph{par dépit}}}\Cendnote{\textnormal{französisch: aus Neid}}}\label{K_L01604_2h} thut – was
               vielleicht eine der häufigſten tragiſchen Verſchuldungen bedeutet. Eine andere, eher
               komoediſche Verſchuldung hinwiederum: jemand denkt auf dem Wege der \introOben{}Höher-\introOben{}Entwicklung irgendwohin gelangt \strikeout{ſei} zu ſein – und iſt nur \label{K_L01604_3v}\edtext{ataviſtiſch}{\lemma{\textnormal{\emph{ataviſtiſch}}}\Cendnote{\textnormal{neuerlich auftretende Eigenschaften früherer Generationen, die durch die
                  Entwicklung unnötig geworden sind und für überwunden gelten}}}\label{K_L01604_3h} hingerathen.
               Auch auf den ſteilen Bergwieſen tanzen zumeiſt Leute, die nicht hin gehören. Dahin
               ungefähr führte mich dein fauniſch-tiefſinnig-burleskes \textcolor{green}{Stückchen}{}\ledrightnote{→\textcolor{green}{Der Faun}}, und ſo möchte es wahrſcheinlich damit {\pb}enden, daſs irgend
               welche nicht bergwieſenwürdige Geſchöpfe vom wahren Faun zu Thale geprügelt
               würden. –\pend
           \pstart
           \noindent{}– Heute, \introOben{}den 25.\introOben{} mein lieber Hermann, reiſen wir ab. Nach
                  \textcolor{pink}{Berlin}{}\ledrightnote{\textcolor{pink}{Berlin}}. (1, 2 Tage) \textcolor{pink}{Kopenhagen}{}\ledrightnote{\textcolor{pink}{Kopenhagen}} (3, 4 Tage.) \textcolor{pink}{\textsc{Marienlyst}}{}\ledrightnote{\textcolor{pink}{Marienlyst}}. Ein paar Wochen. Dann, Auguſt vielleicht noch irgendwohin an die Nordſee. (\textcolor{pink}{Nordyk}{}\ledrightnote{\textcolor{pink}{Graz}}?). Laſs
               uns jedenfalls in brieflich-anſichtskartlicher Verbindung bleiben. – \pend
           \pstart
           Mit guten Sommerwünſchen und {\pb}Grüßen von \textcolor{blue}{Olga}{}\ledrightnote{\textcolor{blue}{Olga Schnitzler}} u mir{\\[\baselineskip]}herzlichſt der Deine{\\[\baselineskip]}\spacefill\mbox{Arthur }\pend
           \leftskip=0em{}\pstart
           \noindent{}Das \textsc{\textcolor{green}{Mscrpt}{}\ledrightnote{→\textcolor{green}{Der Faun}}} ist an \textsc{\textcolor{blue}{Salten}{}\ledrightnote{\textcolor{blue}{Felix Salten}}} abgesandt.\pend
           \endnumbering\briefempfaengerindex{Bahr, Hermann@\textsc{Bahr, Hermann}!zzzSchnitzler, Arthur@\emph{von Arthur Schnitzler}!1906-06-242@{24. –25. 6. 1906}|)be}\mylabel{h}  \normalsize

\doendnotes{C}
\bigskip
\vfill

\clearpage

\footnotesize

\lohead{\textsc{register}}

% Definiere theindex-Environment komplett neu ohne reledmac
\makeatletter
\renewenvironment{theindex}{%
  \section*{\indexname}%
  \setlength{\parindent}{0pt}%
  \setlength{\parskip}{0pt plus 0.3pt}%
  \let\item\@idxitem
}{%
  \clearpage
}
\makeatother

\IfFileExists{\jobname-pw.ind}{\input{\jobname-pw.ind}}{}

\end{document}

      