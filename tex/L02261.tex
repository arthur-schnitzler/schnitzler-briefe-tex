%% latex-korrekturansicht-vorspann.tex
%% Vorspann für die Korrekturansicht.
%% Lädt die gemeinsame Datei latex-vorspann.tex mit gesetztem Schalter.

\newif\ifkorrekturansicht
\korrekturansichttrue

\input{../tex-inputs/latex-vorspann}


               \section[Arthur Schnitzler an Robert Adam, 28. 5. 1917]{ Arthur Schnitzler an Robert Adam, 28. 5. 1917}\nopagebreak\mylabel{v}\rehead{ }\normalsize\beginnumbering\briefempfaengerindex{Adam, Robert@\textsc{Adam, Robert}!zzzSchnitzler, Arthur@\emph{von Arthur Schnitzler}!1917-05-281@{28. 5. 1917}|(be} \toendnotes[C]{\smallbreak\pagebreak[2]} \Standort{DLA, 96.34.2/2.}
\physDesc{Kartenbrief
\newline{}Handschrift: schwarze Tinte, lateinische Kurrent\newline{}Versand: Stempel: »\nobreak{}Wien, 29. V. 17, 7\nobreak{}«.  }\toendnotes[C]{\smallbreak}\pstart{}{\pb}Hrn Dr. Robert Adam Pollak\pend{}\pstart{}\textcolor{pink}{Wien XII}{}\ledrightnote{\textcolor{pink}{XII., Meidling}}\pend{}\pstart{}\textcolor{pink}{Meidlinger Hptstr 56}{}\ledrightnote{\textcolor{pink}{Meidlinger Hauptstraße}}.\pend{}{\bigskip}\pstart
           \raggedleft{}{\pb}28. 5. 1917\pend
           \pstart{}verehrter Herr Doktor,\pend\pstart
           es thut mir sehr leid, daß Sie schon wieder eine theatralische Enttäuschung
                    erleben mußten; – da gibts nun einmal nichts andres, als weiter arbeiten –
                    vielleicht glückt es mit dem nächsten besser, und da{\geminationn} rücken die Vorgänger nach.\pend
           \pstart
           Ich sehe Sie hoffentlich bald wieder, nicht wahr? Ende dieser Woche wollen wir
                    auf circa 14 Tage nach \textcolor{pink}{Gastein}{}\ledrightnote{\textcolor{pink}{Bad Gastein}} (wir waren schon
                    in \textcolor{pink}{Salzburg}{}\ledrightnote{\textcolor{pink}{Salzburg}} – auf dem Weg – und wurden durch die
                    Nachricht vom \label{K_L02261_1v}\edtext{Tode}{\lemma{\textnormal{\emph{Tode}}}\Cendnote{\textnormal{\textcolor{blue}{Stefanie Bachrach} nahm sich am
                            16. 5. 1917 das Leben.}}}\label{K_L02261_1h} einer sehr lieben \textcolor{blue}{Freundin}{}\ledrightnote{→\textcolor{blue}{Stefanie Bachrach}} zurückgerufen) –
                        Mitte Juni aber dürften wir wieder zu Hause sein. Ich schicke
                    Ihnen den sehr amüsanten \textcolor{blue}{\textcolor{green}{Dumas}{}\ledrightnote{→\textcolor{green}{Meine Memoiren}}}{}\ledrightnote{\textcolor{blue}{Alexandre père Dumas}} mit vielem Dank zurück.\pend
           \pstart Herzlichſt grüßend Ihr \spacefill\mbox{Arthur Schnitzler}\pend{}\endnumbering\briefempfaengerindex{Adam, Robert@\textsc{Adam, Robert}!zzzSchnitzler, Arthur@\emph{von Arthur Schnitzler}!1917-05-281@{28. 5. 1917}|)be}\mylabel{h}  \normalsize

\doendnotes{C}
\bigskip
\vfill

\clearpage

\footnotesize

\lohead{\textsc{register}}

% Definiere theindex-Environment komplett neu ohne reledmac
\makeatletter
\renewenvironment{theindex}{%
  \section*{\indexname}%
  \setlength{\parindent}{0pt}%
  \setlength{\parskip}{0pt plus 0.3pt}%
  \let\item\@idxitem
}{%
  \clearpage
}
\makeatother

\IfFileExists{\jobname-pw.ind}{\input{\jobname-pw.ind}}{}

\end{document}

      