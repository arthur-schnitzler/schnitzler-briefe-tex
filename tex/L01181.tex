%% latex-korrekturansicht-vorspann.tex
%% Vorspann für die Korrekturansicht.
%% Lädt die gemeinsame Datei latex-vorspann.tex mit gesetztem Schalter.

\newif\ifkorrekturansicht
\korrekturansichttrue

\input{../tex-inputs/latex-vorspann}


               \section[Arthur Schnitzler an Hermann Bahr, 18. 10. 1901]{ Arthur Schnitzler an Hermann Bahr, 18. 10. 1901}\nopagebreak\mylabel{v}\rehead{ }\normalsize\beginnumbering\briefempfaengerindex{Bahr, Hermann@\textsc{Bahr, Hermann}!zzzSchnitzler, Arthur@\emph{von Arthur Schnitzler}!1901-10-181@{18. 10. 1901}|(be} \toendnotes[C]{\smallbreak\pagebreak[2]} \Standort{TMW, HS AM 23345 Ba.}
\physDesc{Brief, 1 Blatt, 3 Seiten
\newline{}Handschrift: schwarze Tinte, deutsche Kurrent\newline{}Ordnung: Lochung }\buchAbdrucke{\weitereDrucke{1) \emph{18. 10. 1901.} In: Arthur Schnitzler: \emph{The Letters of Arthur Schnitzler to Hermann Bahr}. Edited, annotated, and with an introduction, by Donald G.
                        Daviau. Chapel Hill: \emph{The University of North Carolina Press} 1978, S. 71 (University of North Carolina studies in the Germanic languages
                        and literatures, 89).} \weitereDrucke{2) Hermann Bahr, Arthur Schnitzler: \emph{Briefwechsel, Aufzeichnungen, Dokumente (1891–1931)}. Hg. Kurt Ifkovits und Martin Anton Müller. Göttingen: \emph{Wallstein} 2018, S. 215.} }\toendnotes[C]{\smallbreak}\pstart
           \noindent{}{\pb}lieber Hermann, ich habe nach reiflicher Erwägung den »\uline{\textcolor{green}{Puppenſpieler}{}\ledrightnote{\textcolor{green}{Der Puppenspieler}}}« aus meinem Einaktercyklus ausgeſchieden, so daſs der Cyclus jetzt nur mehr aus
               den \textcolor{green}{4 andern Einaktern}{}\ledrightnote{→\textcolor{green}{Lebendige Stunden. Vier Einakter}} beſteht.
               Ich habe die Abſicht, den \textcolor{green}{Puppenſpieler}{}\ledrightnote{\textcolor{green}{Der Puppenspieler}}{ }{\pb}der mir dramatiſch zu
               ſchwach ſcheint, gelegentlich neu zu bearbeiten.\pend
           \pstart
           Da du die Güte hattest, meine \textcolor{green}{2
                  neuen Stücke}{}\ledrightnote{→\textcolor{green}{Die Frau mit dem Dolche}{\newline}→\textcolor{green}{Lebendige Stunden}} zu übernehmen, theile ich diese Thatſache vor allem dir mit und
               ſtelle dir anheim, dem Direktor des {\pb}\textcolor{pink}{Deutſchen Volkstheater}{}\ledrightnote{\textcolor{pink}{Volkstheater}}s gelegentlich Mittheilung
               hievon zu machen –\pend
           \pstart
           Mit herzlichem Gruſs{\\[\baselineskip]}dein{\\[\baselineskip]}\spacefill\mbox{Arthur}\pend
           \leftskip=0em{}\pstart
           \textcolor{pink}{Wien}{}\ledrightnote{\textcolor{pink}{Wien}}{ }18. 10. 901\pend
           \endnumbering\briefempfaengerindex{Bahr, Hermann@\textsc{Bahr, Hermann}!zzzSchnitzler, Arthur@\emph{von Arthur Schnitzler}!1901-10-181@{18. 10. 1901}|)be}\mylabel{h}  \normalsize

\doendnotes{C}
\bigskip
\vfill

\clearpage

\footnotesize

\lohead{\textsc{register}}

% Definiere theindex-Environment komplett neu ohne reledmac
\makeatletter
\renewenvironment{theindex}{%
  \section*{\indexname}%
  \setlength{\parindent}{0pt}%
  \setlength{\parskip}{0pt plus 0.3pt}%
  \let\item\@idxitem
}{%
  \clearpage
}
\makeatother

\IfFileExists{\jobname-pw.ind}{\input{\jobname-pw.ind}}{}

\end{document}

      