%% latex-korrekturansicht-vorspann.tex
%% Vorspann für die Korrekturansicht.
%% Lädt die gemeinsame Datei latex-vorspann.tex mit gesetztem Schalter.

\newif\ifkorrekturansicht
\korrekturansichttrue

\input{../tex-inputs/latex-vorspann}


               \section[Robert Adam an Arthur Schnitzler, 19. 6. 1917]{ Robert Adam an Arthur Schnitzler, 19. 6. 1917}\nopagebreak\mylabel{v}\rehead{ }\normalsize\beginnumbering\briefempfaengerindex{Schnitzler, Arthur@\textsc{Schnitzler, Arthur}!zzzAdam, Robert@\emph{von Robert Adam}!1917-06-191@{19. 6. 1917}|(be} \toendnotes[C]{\smallbreak\pagebreak[2]} \Standort{DLA, A:Schnitzler, HS.NZ85.1.4230,19.}
\physDesc{Brief, 1 Blatt, 4 Seiten
\newline{}Handschrift: schwarze Tinte, deutsche Kurrent
\newline{}Schnitzler: 1) mit Bleistift beschriftet: »\textsc{Adam}« 2) mit rotem Buntstift mehrere Unterstreichungen}\Standort{Wien, Österreichische Nationalbibliothek, Cod.ser. 52.263, 197.}
\physDesc{Brief, maschinelle Abschrift
\newline{}Schreibmaschine}\toendnotes[C]{\smallbreak}\pstart
           \raggedleft{}{\pb}\textcolor{pink}{Wien}{}\ledrightnote{\textcolor{pink}{Wien}}, am 19. Juni 1917. \pend
           \pstart{}Hochverehrter Herr Doktor!\pend\pstart
           Ich danke Ihnen herzlich für Ihren Glückwunſch. Die Verſetzung von \textcolor{pink}{Floridsdorf}{}\ledrightnote{\textcolor{pink}{XXI., Floridsdorf}} zum \textcolor{brown}{Bezirksgericht \textcolor{pink}{Joſefſtadt}{}\ledrightnote{\textcolor{pink}{VIII., Josefstadt}}}{}\ledrightnote{→\textcolor{brown}{Bezirksgericht Wien Josefstadt}} empfand und empfinde ich noch als eine Befreiung aus dem unleidlichſten
                    Zuſtande, dem Zwang zur Zeitvergeudung. Denn mochte ich mich auch bemühen, die
                    endloſen täglichen Tramwayfahrten zu irgendeinem Studium auszunützen, es gelang
                    höchſtens bei der Morgenfahrt, während mir die Rückreiſe, die ich ermüdet und
                    hungrig zurücklegen mußte, nur gerade noch eine Zeitungslektüre {\pb}verſtattete. Auch die Amtsbeſchäftigung – die
                    Säuberung einer von meinem verſtorbenen \textcolor{blue}{Vorgänger}{}\ledrightnote{→\textcolor{blue}{Aemilius Hacker}} arg verwahrloſten außerſtreitigen Abteilung
                    – bot nur wenig Befriedigung.\pend
           \pstart
           Durch die Verſetzung bin ich allerdings wieder, und zwar aller Wahrſcheinlichkeit
                    nach auf längere Zeit, in die Nachrichtertätigkeit zurückgeworfen; da ich aber
                    nur in Preistreibereiſachen zu judizieren habe, bleibt mir das Peinliche fern,
                    das in jeder andern Nachjudikatur in Zeiten allgemeiner Not liegt. Ich brauche
                    nicht Leute zu verurteilen, deren Vergehen durch die Hungersnot kauſal begründet
                    iſt, ſondern habe vor allem gegen ſolche einzuſchreiten, deren Vergehen {\pb}eben die Mitverurſachung der Hungersnot bildet.
                    Und ſo arbeite ich ohne böſes Gewiſſen.\pend
           \pstart
           Auch literariſch bin ich nicht ganz untätig. Von einer ſeltſamen \textcolor{green}{Urchriſtenkomödie}{}\ledrightnote{→\textcolor{green}{Das Ende des Judas}} (oder Tragödie?) habe
                    ich faſt drei Akte im Rohen fertig entworfen und hoffe, die reſtlichen zwei
                    Akte, die mir beſonders am Herzen liegen, während des Urlaubs zu Papier zu
                    bringen. Diesen trete ich Ende Juni an und will ihn zur Hälfte bei
                        \textcolor{blue}{Frau}{}\ledrightnote{→\textcolor{blue}{Maria Pollak}} und \textcolor{blue}{Kind}{}\ledrightnote{→\textcolor{blue}{Viktor Franz Patzner}} verbringen, die ich
                    günſtigerer Ernährungsverhältniſſe wegen in meinem früheren Dienſtorte, in \textcolor{pink}{Ziſtersdorf}{}\ledrightnote{\textcolor{pink}{Zistersdorf}}, angesiedelt habe; während der
                    reſtlichen Zeit gedenke ich mit D\textsuperscript{r}{ }\textcolor{blue}{\textsc{Beer}}{}\ledrightnote{\textcolor{blue}{Richard Beer}} irgendwo in \textcolor{pink}{Steiermark}{}\ledrightnote{\textcolor{pink}{Steiermark}}, bewaffnet mit
                    einer Salami, das dazu gehörige tägliche {\pb}Brot zu
                    ſuchen.\pend
           \pstart
           Da ich nicht weiß, wann Sie, hochverehrter Herr Doktor, nach \textcolor{pink}{Wien}{}\ledrightnote{\textcolor{pink}{Wien}} zurückkehren – das herrliche Wetter dürfte Ihre
                    Rückkehr wohl verzögern –, will ich im Laufe der nächſten Woche bei Ihnen
                    anklopfen, auf die Gefahr hin, Sie nicht anzutreffen.\pend
           \pstart
           Indem ich ſchließlich den Rückerhalt des \textcolor{green}{\textcolor{blue}{\textsc{Dumas}}{}\ledrightnote{\textcolor{blue}{Alexandre père Dumas}}}{}\ledrightnote{→\textcolor{green}{Meine Memoiren}} mit beſtem Dank beſtätige, verbleibe ich mit beſten Grüßen und
                    Empfehlungen Ihr\pend
           \pstart
           ſehr ergebener{\\[\baselineskip]}\spacefill\mbox{Robert Adam}\pend
           \leftskip=0em{}\endnumbering\briefempfaengerindex{Schnitzler, Arthur@\textsc{Schnitzler, Arthur}!zzzAdam, Robert@\emph{von Robert Adam}!1917-06-191@{19. 6. 1917}|)be}\mylabel{h}  \normalsize

\doendnotes{C}
\bigskip
\vfill

\clearpage

\footnotesize

\lohead{\textsc{register}}

% Definiere theindex-Environment komplett neu ohne reledmac
\makeatletter
\renewenvironment{theindex}{%
  \section*{\indexname}%
  \setlength{\parindent}{0pt}%
  \setlength{\parskip}{0pt plus 0.3pt}%
  \let\item\@idxitem
}{%
  \clearpage
}
\makeatother

\IfFileExists{\jobname-pw.ind}{\input{\jobname-pw.ind}}{}

\end{document}

      