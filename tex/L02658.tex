%% latex-korrekturansicht-vorspann.tex
%% Vorspann für die Korrekturansicht.
%% Lädt die gemeinsame Datei latex-vorspann.tex mit gesetztem Schalter.

\newif\ifkorrekturansicht
\korrekturansichttrue

\input{../tex-inputs/latex-vorspann}


               \section[Paul Goldmann an Arthur Schnitzler, 7. 1. 1891]{ Paul Goldmann an Arthur Schnitzler, 7. 1. 1891}\nopagebreak\mylabel{v}\rehead{ }\normalsize\beginnumbering\briefempfaengerindex{Schnitzler, Arthur@\textsc{Schnitzler, Arthur}!zzzGoldmann, Paul@\emph{von Paul Goldmann}!1891-01-071@{7. 1. 1891}|(be} \toendnotes[C]{\smallbreak\pagebreak[2]} \Standort{DLA, A:Schnitzler, HS.NZ85.1.3162.}
\physDesc{Brief, 1 Blatt, 2 Seiten
\newline{}Handschrift: Bleistift, deutsche Kurrent
\newline{}Schnitzler: mit Bleistift das Datum »Jän 91« vermerkt }\toendnotes[C]{\smallbreak}\pstart\center{}{\pb}Lieber Arthur!\pend\pstart
           Eine große Gefälligkeit, bitte! Geh’ heut{ }Abend in’s \textcolor{brown}{Burgtheater}{}\ledrightnote{\textcolor{brown}{Burgtheater}} u »ſchreib«
               mir ein \label{K_L02658-11v}\edtext{\textcolor{green}{Referat}{}\ledrightnote{→\textcolor{green}{?? [Rezension des Gastspiels von Anna Hochenburger, 7.1.1891]}}}{\lemma{\textnormal{\emph{Referat}}}\Cendnote{\textnormal{Im letzten Heft des Jahres
                     1890 stand letztmalig \textcolor{blue}{Goldmann}s Name als »\textcolor{blue}{Mit-Redakteur}« im Impressum von \emph{\textcolor{brown}{An der
                     schönen blauen Donau}}. Anzunehmen ist, dass er danach gemeinsam mit dem \textcolor{blue}{Herausgeber} und \textcolor{blue}{Onkel}{ }\textcolor{blue}{Fedor Mamroth} die Mitarbeit an der \textcolor{brown}{Zeitschrift} beendet hatte.
                  Nachdem er die Stelle bei der \emph{\textcolor{brown}{Frankfurter
                     Zeitung}} erst mit April 1891 antrat und erst
                  kurz vorher davon erfahren haben dürfte (vgl. A. S.: \emph{Tagebuch}, 29. 3. 1891), bleibt offen, für welche Publikation er in den
                  ersten drei Monaten des Jahres 1891 tätig war. Weil er die \textcolor{green}{Rezension} erst für den übernächsten Tag erbittet, dürfte es sich um ein
                  Wochen- oder Monatsblatt handeln. Oder er benötigte das \textcolor{green}{Referat} als Stilprobe für eine
                  Stellenbewerbung, wogegen aber zu sprechen scheint, dass er über ein Büro
                  verfügte.}}}\label{K_L02658-11h} über die \label{K_L02658-1v}\edtext{\textsc{\textcolor{blue}{Hochenburger}{}\ledrightnote{\textcolor{blue}{Anna Hochenburger}}}}{\lemma{\textnormal{\emph{Hochenburger}}}\Cendnote{\textnormal{Die \textcolor{pink}{Berlin}er \textcolor{blue}{Schauspielerin}{ }\textcolor{blue}{Anna Hochenburger} hatte im Januar 1891 ein Gastspiel am \emph{\textcolor{brown}{Burgtheater}}. Es begann am 7. 1. 1891, sie gab \textcolor{green}{Julia} in \emph{\textcolor{green}{Romeo und Julia}}. \textcolor{blue}{Schnitzler} nahm an
                  der Premiere am 7. 1. 1891 teil. Das und der Folgebrief (Paul Goldmann an Arthur Schnitzler, 7. 1. 1891) ermöglichen die verlässliche Datierung des undatierten
                  Korrespondenzstücks.}}}\label{K_L02658-1h}! Aus Gründen, die ich Dir für mich
                  ent\textcolor{gray}{wickeln} kann, bin ich verhindert ſelbſt zu gehen. Es darf
               aber Niemand wiſſen, daß du \uline{für mich} gehſt! Sollteſt
               Du aus irgend einem Grunde {\pb}verhindert ſein, \strikeout{m\textcolor{gray}{e}i} meine Bitte zu erfüllen, ſo
               ſchicke mir, bitte, \uuline{umgehend} die Karte in’s Bureau
               zurück. Das \textcolor{green}{Referat}{}\ledrightnote{\textcolor{green}{?? [Rezension des Gastspiels von Anna Hochenburger, 7.1.1891]}} müßte ich bis übermorgen{ }früh in Händen haben.\pend
           \pstart
           Herzl. Gruß! {\\[\baselineskip]}Dein {\\[\baselineskip]}\spacefill\mbox{Paul Goldm}\pend
           \leftskip=0em{}\endnumbering\briefempfaengerindex{Schnitzler, Arthur@\textsc{Schnitzler, Arthur}!zzzGoldmann, Paul@\emph{von Paul Goldmann}!1891-01-071@{7. 1. 1891}|)be}\mylabel{h}  \normalsize

\doendnotes{C}
\bigskip
\vfill

\clearpage

\footnotesize

\lohead{\textsc{register}}

% Definiere theindex-Environment komplett neu ohne reledmac
\makeatletter
\renewenvironment{theindex}{%
  \section*{\indexname}%
  \setlength{\parindent}{0pt}%
  \setlength{\parskip}{0pt plus 0.3pt}%
  \let\item\@idxitem
}{%
  \clearpage
}
\makeatother

\IfFileExists{\jobname-pw.ind}{\input{\jobname-pw.ind}}{}

\end{document}

      