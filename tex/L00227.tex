%% latex-korrekturansicht-vorspann.tex
%% Vorspann für die Korrekturansicht.
%% Lädt die gemeinsame Datei latex-vorspann.tex mit gesetztem Schalter.

\newif\ifkorrekturansicht
\korrekturansichttrue

\input{../tex-inputs/latex-vorspann}


               \section[Richard Beer-Hofmann an Arthur Schnitzler, 23. 6. 1893]{ Richard Beer-Hofmann an Arthur Schnitzler, 23. 6. 1893}\nopagebreak\mylabel{v}\rehead{ }\normalsize\beginnumbering\briefempfaengerindex{Schnitzler, Arthur@\textsc{Schnitzler, Arthur}!zzzBeer-Hofmann, Richard@\emph{von Richard Beer-Hofmann}!1893-06-231@{23. 6. 1893}|(be} \toendnotes[C]{\smallbreak\pagebreak[2]} \Standort{CUL, Schnitzler, B 8.}
\physDesc{Brief, 2 Blätter, 3 Seiten
\newline{}Handschrift: Bleistift, lateinische Kurrent
\newline{}Schnitzler: mit Bleistift nummeriert: »18« bzw.
               »18a« }\buchAbdrucke{\weitereDrucke{Arthur Schnitzler, Richard Beer-Hofmann: \emph{Briefwechsel 1891–1931}. Hg. Konstanze Fliedl. Wien, Zürich: \emph{Europaverlag} 1992, S. 45.} }\pstart
           \noindent{}{\pb}Lieber Arthur! Bisher
               hat sich \textcolor{blue}{Jarno}{}\ledrightnote{\textcolor{blue}{Josef Jarno}} noch nicht sehen lassen; übrigens
                  ko{\geminationm}en Sie ja hoffentlich in einigen Tagen selbst.
               Bitte, wenn Sie ko{\geminationm}en bringen Sie mir ein Flaccon Parfüm
               mit; es ist bei »\uline{\textcolor{brown}{Weisse}{}\ledrightnote{\textcolor{brown}{Theodor Weisse}}}« am \textcolor{pink}{Mehlmarkt}{}\ledrightnote{\textcolor{pink}{Neuer Markt}} Ecke der \textcolor{pink}{Plankengasse}{}\ledrightnote{\textcolor{pink}{Plankengasse}} erhältlich, der Name ist, \uline{glaube} ich: »\uline{Neomir du
               Phare}« oder sonst irgendwie aehnlich; auch bringen – oder wenn\strikeout{s} es Sie genirt, – schicken Sie mir 100 Stück \textcolor{pink}{egyptische}{}\ledrightnote{\textcolor{pink}{Ägypten}} echte Cigaretten irgendwelche Marke zu
               5-6 fl. höchstens (\textcolor{pink}{Riedhof}{}\ledrightnote{\textcolor{pink}{Riedhof}}, \textcolor{pink}{Central}{}\ledrightnote{\textcolor{pink}{Café Central}}, \textcolor{pink}{Sacher}{}\ledrightnote{\textcolor{pink}{Hotel Sacher}}, \textcolor{pink}{Caffée Impérial}{}\ledrightnote{\textcolor{pink}{Café Imperial}}). Vielleicht ni{\geminationm}t
                  \textcolor{blue}{Salten}{}\ledrightnote{\textcolor{blue}{Felix Salten}} seinen Urlaub auch um dieselbe Zeit? Ich
               sehe ein daß mir – da ich Euch \textcolor{gray}{d}och nicht nachlaufen kann – nichts
               anderes {\pb}übrig bleiben wird, als im
               Herbste gleichfalls Bycicle oder Bicycle fahren zu lernen; ich traure bereits jetzt
               bei dem Gedanken wieviel Ersparnisse an Fiakern und Omnibus-Fahrten mich das wieder
               kosten wird!\pend
           \pstart
           Herzlichst{\\[\baselineskip]}\spacefill\mbox{Richard}\pend
           \leftskip=0em{}\pstart
           \noindent{}Grüßen Sie nach Ermessen, und wenn Sie die Comissionen irgendwie geniren, geben
                  Sie sich keine Mühe, – es ist nicht wichtig.\pend
           \pstart
           \raggedleft{}R.\pend
           \pstart
           \noindent{}23 Juni 93 \textcolor{pink}{Ischl}{}\ledrightnote{\textcolor{pink}{Bad Ischl}}\pend
           \pstart
           {\pb}Soeben fällt mir ein\substVorne{}\textsuperscript{,}\substDazwischen{}:\substHinten{} Gestern saß in der Theater-Loge ein Fräulein »\textcolor{blue}{Wreden}{}\ledrightnote{\textcolor{blue}{Grethe Wreden}}«, mir »wolbekannt«, eine der
                  3 Schlafwagenconducteurstöchter wenn ich nicht irre, und \textcolor{blue}{P. H.}{}\ledrightnote{\textcolor{blue}{Paul Horn}}{[}s{]} gewesene Herrin? Was ist mit ihr? Soll man sie besuchen, –
                  ansprechen – ignoriren, weiß \textcolor{blue}{P. H.}{}\ledrightnote{\textcolor{blue}{Paul Horn}} von ihrem
                  hiesigen Aufenthalte, ko{\geminationm}t er her?\pend
           \endnumbering\briefempfaengerindex{Schnitzler, Arthur@\textsc{Schnitzler, Arthur}!zzzBeer-Hofmann, Richard@\emph{von Richard Beer-Hofmann}!1893-06-231@{23. 6. 1893}|)be}\mylabel{h}  \normalsize

\doendnotes{C}
\bigskip
\vfill

\clearpage

\footnotesize

\lohead{\textsc{register}}

% Definiere theindex-Environment komplett neu ohne reledmac
\makeatletter
\renewenvironment{theindex}{%
  \section*{\indexname}%
  \setlength{\parindent}{0pt}%
  \setlength{\parskip}{0pt plus 0.3pt}%
  \let\item\@idxitem
}{%
  \clearpage
}
\makeatother

\IfFileExists{\jobname-pw.ind}{\input{\jobname-pw.ind}}{}

\end{document}

      