%% latex-korrekturansicht-vorspann.tex
%% Vorspann für die Korrekturansicht.
%% Lädt die gemeinsame Datei latex-vorspann.tex mit gesetztem Schalter.

\newif\ifkorrekturansicht
\korrekturansichttrue

\input{../tex-inputs/latex-vorspann}


               \section[Arthur und Olga Schnitzler an Richard Beer-Hofmann, 11. 6. 1917]{ Arthur und Olga Schnitzler an Richard Beer-Hofmann,
                    11. 6. 1917}\nopagebreak\mylabel{v}\rehead{ }\normalsize\beginnumbering\briefempfaengerindex{Beer-Hofmann, Richard@\textsc{Beer-Hofmann, Richard}!zzzSchnitzler, Olga@\emph{von Olga Schnitzler}!1917-06-111@{11. 6. 1917}|(be}\briefempfaengerindex{Beer-Hofmann, Richard@\textsc{Beer-Hofmann, Richard}!zzzSchnitzler, Arthur@\emph{von Arthur Schnitzler}!1917-06-111@{11. 6. 1917}|(be} \toendnotes[C]{\smallbreak\pagebreak[2]} \Standort{YCGL, MSS 31.}
\physDesc{Bildpostkarte
\newline{}Handschrift Arthur Schnitzler: Bleistift, deutsche Kurrent\newline{}Handschrift Olga Schnitzler: Bleistift, lateinische Kurrent\newline{}Versand: Stempel: »\nobreak{}\oindex{Bad Gastein@\textbf{Bad Gastein}, \emph{Besiedelter Ort (A.BSO)}|pwk}\textcolor{gray}{Badgastein}, \textcolor{gray}{11. VI. 17}, \textcolor{gray}{4}\nobreak{}«.  }\toendnotes[C]{\smallbreak}\pstart{}{\pb}Herrn Dr.\pend{}\pstart{}\textsc{Richard Beer-Hofmann}\pend{}\pstart{}\textcolor{pink}{Wien XVIII}{}\ledrightnote{\textcolor{pink}{Wien}}\pend{}\pstart{}\textsc{\textcolor{pink}{Hasenauerstr 54}{}\ledrightnote{\textcolor{pink}{Hasenauerstraße}}.}\pend{}{\bigskip}\pstart
           \noindent{}\centering{}{\pb}\textcolor{gray}{\textbf{\textcolor{pink}{BADGASTEIN}{}\ledrightnote{\textcolor{pink}{Bad Gastein}} von der \textcolor{pink}{Schwarzenbergpromenade}{}\ledrightnote{\textcolor{pink}{Kaiser-Franz-Josefstraße}}}}\pend
           \pstart
           {\pb}Herzliche Grüße Ihnen \textcolor{blue}{Allen}{}\ledrightnote{→\textcolor{blue}{Gabriel Beer-Hofmann}{\newline}→\textcolor{blue}{Paula Beer-Hofmann}{\newline}→\textcolor{blue}{Mirjam Beer-Hofmann}{\newline}→\textcolor{blue}{Naëmah Beer-Hofmann}}. Es ſcheint ja
                    wir treffen Sie noch in \textcolor{pink}{Wien}{}\ledrightnote{\textcolor{pink}{Wien}} an? So hab auch ich
                    einmal Grund \textcolor{blue}{Reinhardt}{}\ledrightnote{\textcolor{blue}{Max Reinhardt}} dankbar zu ſein!
                    Rutſchen Sie vielleicht auch auf ein paar Tage her? {\pb}Ganz abgeſehen von uns – es wäre der Mühe werth.
                    Wir fühlen uns wohl. Es iſt noch ſo leer!\pend
           \pstart Herzlichſt \spacefill\mbox{Arth}\pend{}\pstart
           \noindent{}{[}hs. O. Schnitzler:{]} Herzlichst Ihre\pend
           \pstart \spacefill\mbox{O. S.}\pend{}\endnumbering\briefempfaengerindex{Beer-Hofmann, Richard@\textsc{Beer-Hofmann, Richard}!zzzSchnitzler, Olga@\emph{von Olga Schnitzler}!1917-06-111@{11. 6. 1917}|)be}\briefempfaengerindex{Beer-Hofmann, Richard@\textsc{Beer-Hofmann, Richard}!zzzSchnitzler, Arthur@\emph{von Arthur Schnitzler}!1917-06-111@{11. 6. 1917}|)be}\mylabel{h}  \normalsize

\doendnotes{C}
\bigskip
\vfill

\clearpage

\footnotesize

\lohead{\textsc{register}}

% Definiere theindex-Environment komplett neu ohne reledmac
\makeatletter
\renewenvironment{theindex}{%
  \section*{\indexname}%
  \setlength{\parindent}{0pt}%
  \setlength{\parskip}{0pt plus 0.3pt}%
  \let\item\@idxitem
}{%
  \clearpage
}
\makeatother

\IfFileExists{\jobname-pw.ind}{\input{\jobname-pw.ind}}{}

\end{document}

      