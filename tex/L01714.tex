%% latex-korrekturansicht-vorspann.tex
%% Vorspann für die Korrekturansicht.
%% Lädt die gemeinsame Datei latex-vorspann.tex mit gesetztem Schalter.

\newif\ifkorrekturansicht
\korrekturansichttrue

\input{../tex-inputs/latex-vorspann}


               \section[Stefan Großmann an Arthur Schnitzler, 29. 9. 1907]{ Stefan Großmann an Arthur Schnitzler, 29. 9. 1907}\nopagebreak\mylabel{v}\rehead{ }\normalsize\beginnumbering\briefempfaengerindex{Schnitzler, Arthur@\textsc{Schnitzler, Arthur}!zzzGrossmann, Stefan@\emph{von Stefan Großmann}!1907-09-292@{29. 9. 1907}|(be} \toendnotes[C]{\smallbreak\pagebreak[2]} \Standort{CUL, Schnitzler, B 34.}
\physDesc{Brief, 1 Blatt (Briefpapier mit Trauerrand), 1 Seite
\newline{}Handschrift: schwarze Tinte, deutsche Kurrent\newline{}Ordnung: mit Bleistift von unbekannter Hand nummeriert:
                                 »5« }\toendnotes[C]{\smallbreak}\pstart
           \noindent{}{\pb}\textcolor{gray}{\textbf{\textcolor{brown}{Freie Volksbühne}{}\ledrightnote{\textcolor{brown}{Wiener Freie Volksbühne}}}}\pend
           \pstart
           \textcolor{gray}{\textbf{\textcolor{pink}{Wien VI/\textsubscript{1}}{}\ledrightnote{\textcolor{pink}{VI., Mariahilf}}.}}\pend
           \pstart
           \textcolor{gray}{\textbf{\textcolor{pink}{Mariahilferſtraße Nr. 89.}{}\ledrightnote{\textcolor{pink}{Mariahilferstraße}}}}\hfill \textcolor{gray}{\textbf{\textcolor{pink}{Wien}{}\ledrightnote{\textcolor{pink}{Wien}}, am}}{ }29. \damage{\textcolor{gray}{Se}}pt. \textcolor{gray}{\textbf{190}}7\pend
           \pstart
           \textcolor{gray}{\textbf{Poſtſparkaſſen-Konto Nr. 87.544.}}\pend
           \pstart\center{}Verehrter Herr,\pend\pstart
           Danke für Ihre Bereitwilligkeit.\pend
           \pstart
           Wir werden alle Ihre Wünſche berückſichtigen u den Abend am \uline{17. Oktober} in einem kleinen (500 Leute faſſenden) Saal veranſtalten. Herr Abg \uline{\textsc{\textcolor{blue}{Pernerstorfer}{}\ledrightnote{\textcolor{blue}{Engelbert Pernerstorfer}}}} wird den Abend mit einem kleinen Vortrag eröffnen. Dann leſen Sie, verehrter
               Herr, vielleicht den »\textcolor{green}{\label{T_L01714_1v}\edtext{\textsc{\uline{Lieutenant}}}{\lemma{\textnormal{\emph{Lieutenant}}}\Cendnote{\textnormal{Er schreibt:
                        »Leuitenant«.}}}\label{T_L01714_1h}\textsc{\uline{{ }Gustl}}}{}\ledrightnote{\textcolor{green}{Lieutenant Gustl. Novelle}}« und irgendeine kleine \textcolor{green}{Arbeit}{}\ledrightnote{→\textcolor{green}{Das neue Lied}}.\pend
           \pstart
           Wir bitten uns \uline{recht bald} Ihre Zust\textcolor{gray}{i{\geminationm}un}g definitiv zu übermitteln, da wir
               14 Tage vorher die Ankündigung in unſeren Vereinsmittheilungen loslaſſen müſſen.\pend
           \pstart
           Aufrichtig dankend{\\[\baselineskip]}ſehr ergeben{\\[\baselineskip]}f. d. \textcolor{brown}{Fr. V.}{}\ledrightnote{\textcolor{brown}{Wiener Freie Volksbühne}}{\\[\baselineskip]}\spacefill\mbox{GroßSt}\pend
           \leftskip=0em{}\endnumbering\briefempfaengerindex{Schnitzler, Arthur@\textsc{Schnitzler, Arthur}!zzzGrossmann, Stefan@\emph{von Stefan Großmann}!1907-09-292@{29. 9. 1907}|)be}\mylabel{h}  \normalsize

\doendnotes{C}
\bigskip
\vfill

\clearpage

\footnotesize

\lohead{\textsc{register}}

% Definiere theindex-Environment komplett neu ohne reledmac
\makeatletter
\renewenvironment{theindex}{%
  \section*{\indexname}%
  \setlength{\parindent}{0pt}%
  \setlength{\parskip}{0pt plus 0.3pt}%
  \let\item\@idxitem
}{%
  \clearpage
}
\makeatother

\IfFileExists{\jobname-pw.ind}{\input{\jobname-pw.ind}}{}

\end{document}

      