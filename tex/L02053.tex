%% latex-korrekturansicht-vorspann.tex
%% Vorspann für die Korrekturansicht.
%% Lädt die gemeinsame Datei latex-vorspann.tex mit gesetztem Schalter.

\newif\ifkorrekturansicht
\korrekturansichttrue

\input{../tex-inputs/latex-vorspann}


               \section[Hugo von Hofmannsthal an Arthur Schnitzler, {[}7. 2. 1912{]}]{ Hugo von Hofmannsthal an Arthur Schnitzler, {[}7. 2. 1912{]}}\nopagebreak\mylabel{v}\rehead{ }\normalsize\beginnumbering\briefempfaengerindex{Schnitzler, Arthur@\textsc{Schnitzler, Arthur}!zzzHofmannsthal, Hugo von@\emph{von Hugo von Hofmannsthal}!1912-02-071@{{[}7. 2. 1912{]}}|(be} \toendnotes[C]{\smallbreak\pagebreak[2]} \Standort{CUL, Schnitzler, B 43.}
\physDesc{Brief, 1 Blatt, 2 Seiten
\newline{}Handschrift: schwarze Tinte, deutsche Kurrent
\newline{}Schnitzler: mit Bleistift datiert: »7/2 912« und beschriftet: »\textsc{Hugo}« \newline{}Ordnung: 1) mit Bleistift von unbekannter Hand nummeriert: »\strikeout{325}« 2) mit Bleistift von unbekannter Hand nummeriert:
                                    »335«}\buchAbdrucke{\weitereDrucke{Hugo von Hofmannsthal, Arthur Schnitzler: \emph{Briefwechsel}. Hg. Therese Nickl und Heinrich Schnitzler. Frankfurt am Main: \emph{S. Fischer} 1964, S. 264.} }\toendnotes[C]{\smallbreak}\pstart
           \noindent{}\centering{}{\pb}\textcolor{gray}{\textbf{\textcolor{pink}{SÜDBAHN-HOTEL SEMMERING BEI WIEN}{}\ledrightnote{\textcolor{pink}{Südbahnhotel}}}}\pend
           \pstart
           \noindent{}\textcolor{gray}{\textbf{ERSTES HOTEL M. 350 ZIMMERN, GESCHÜTZTE, SCHÖNSTE U.
                        KLIMATISCH GÜNSTIGSTE LAGE AM \textcolor{pink}{SEMMERING}{}\ledrightnote{\textcolor{pink}{Semmering}} MIT
                        AUSSICHT AUF \textcolor{pink}{RAX}{}\ledrightnote{\textcolor{pink}{Rax}}, \textcolor{pink}{SCHNEEBERG}{}\ledrightnote{\textcolor{pink}{Schneeberg}}, \textcolor{brown}{EISENBAHNLINIE}{}\ledrightnote{→\textcolor{brown}{Südbahnstrecke}} ETC. K.K. HAUPTPOST, TELEGRAPHEN-
                        U. TELEPHONAMT IM HOTEL}}\pend
           \pstart
           \textcolor{gray}{\textbf{1000 M ÜBER DEM MEERE. SOMMER- UND WINTERKURORT ERSTEN
                        RANGES}}{[}.{]}{ }\textcolor{gray}{\textbf{GRÖSSTER UND VORNEHMSTER WINTERSPORTPLATZ \textcolor{pink}{ÖSTERREICHS}{}\ledrightnote{\textcolor{pink}{Österreich}}. 2 STUNDEN EISENBAHNFAHRT VON \textcolor{pink}{WIEN}{}\ledrightnote{\textcolor{pink}{Wien}} UND \textcolor{pink}{GRAZ}{}\ledrightnote{\textcolor{pink}{Graz}}.}}\pend
           \pstart
           \centering{}\textcolor{gray}{\textbf{TELEGR.- U BRIEF-ADR: \textcolor{pink}{SÜDBAHNHOTEL SEMMERING}{}\ledrightnote{\textcolor{pink}{Südbahnhotel}}, TELEPHON \textcolor{pink}{SEMMERING 5}{}\ledrightnote{\textcolor{pink}{Semmering}}.}}\pend
           \pstart
           \noindent{}\raggedleft{}\textcolor{gray}{\textbf{\textcolor{pink}{Semmering}{}\ledrightnote{\textcolor{pink}{Semmering}}, am ..........}}\pend
           \pstart{}mein lieber Arthur \pend\pstart
           Ihre Zeilen waren lieb und woltuend wie immer, ich danke Ihnen ſehr.\pend
           \pstart
           Leſe in der Zeitung daſs es die »\label{K_L02053_1v}\edtext{\textcolor{green}{Marionetten}{}\ledrightnote{\textcolor{green}{Marionetten. Drei Einakter}}}{\lemma{\textnormal{\emph{Marionetten}}}\Cendnote{\textnormal{am 10. 2. 1912 im \textcolor{pink}{Deutschen Volkstheater}}}}\label{K_L02053_1h}« ſind, die man ſpielt, würde ich wohl für
               den 1\textsuperscript{ten}{ }{\pb}oder 2\textsuperscript{ten} Abend \uline{2 Balconsitze} durch Sie beko{\geminationm}en können? würde mich ſehr freuen; vielleicht iſt es am
               einfachſten, Sie bezahlen ſie für mich und ſchicken mir ſie an die Adreſſe \textcolor{pink}{Eliſabethſtraße 6}{}\ledrightnote{\textcolor{pink}{Elisabethstraße}}.\pend
           \pstart
           Vielleicht ko{\geminationm}t die Bitte ſchon zu ſpät, dann gehe ich
               halt in eine ſpätere Vorstellung.\pend
           \pstart
           Herzlich{\\[\baselineskip]}\spacefill\mbox{Hugo.}\pend
           \leftskip=0em{}\endnumbering\briefempfaengerindex{Schnitzler, Arthur@\textsc{Schnitzler, Arthur}!zzzHofmannsthal, Hugo von@\emph{von Hugo von Hofmannsthal}!1912-02-071@{{[}7. 2. 1912{]}}|)be}\mylabel{h}  \normalsize

\doendnotes{C}
\bigskip
\vfill

\clearpage

\footnotesize

\lohead{\textsc{register}}

% Definiere theindex-Environment komplett neu ohne reledmac
\makeatletter
\renewenvironment{theindex}{%
  \section*{\indexname}%
  \setlength{\parindent}{0pt}%
  \setlength{\parskip}{0pt plus 0.3pt}%
  \let\item\@idxitem
}{%
  \clearpage
}
\makeatother

\IfFileExists{\jobname-pw.ind}{\input{\jobname-pw.ind}}{}

\end{document}

      