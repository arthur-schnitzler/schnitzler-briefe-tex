%% latex-korrekturansicht-vorspann.tex
%% Vorspann für die Korrekturansicht.
%% Lädt die gemeinsame Datei latex-vorspann.tex mit gesetztem Schalter.

\newif\ifkorrekturansicht
\korrekturansichttrue

\input{../tex-inputs/latex-vorspann}


               \section[Arthur Schnitzler an Georg Brandes, 30. 1. 1922]{ Arthur Schnitzler an Georg Brandes, 30. 1. 1922}\nopagebreak\mylabel{v}\rehead{ }\normalsize\beginnumbering\briefempfaengerindex{Brandes, Georg@\textsc{Brandes, Georg}!zzzSchnitzler, Arthur@\emph{von Arthur Schnitzler}!1922-01-301@{30. 1. 1922}|(be} \toendnotes[C]{\smallbreak\pagebreak[2]} \Standort{Kopenhagen, Det Kongelige Bibliotek, Georg Brandes Arkiv, box 125.}
\physDesc{Brief, 3 Blätter, 6 Seiten
\newline{}Handschrift: schwarze Tinte, lateinische Kurrent\newline{}Ordnung: mit Bleistift von unbekannter Hand beschriftet:
                                    »Schnitzler« und nummeriert:
                                 »44.«, die Blätter durchgezählt
                                    »1«–»3«, wobei bei den letzten
                                 beiden Blättern auch zusätzlich das Datum ergänzt ist: »30/1 22« }\buchAbdrucke{\weitereDrucke{1) Georg Brandes, Arthur Schnitzler: \emph{Ein Briefwechsel}. Hg. Kurt Bergel. Bern: \emph{Francke} 1956, S. 133–135.} \weitereDrucke{2) Arthur Schnitzler: \emph{Briefe 1913–1931}. Hg. Peter Michael Braunwarth, Richard Miklin, Susanne Pertlik und Heinrich Schnitzler. Frankfurt am Main: \emph{S. Fischer} 1984, S. 263–266.} }\toendnotes[C]{\smallbreak}\pstart
           \raggedleft{}{\pb}\textcolor{pink}{Wien}{}\ledrightnote{\textcolor{pink}{Wien}}, 30. 1. 1922\pend
           \pstart
           Mein lieber und verehrter Freund, es trifft sich gut, daß ich Ihnen
               auf Ihren letzten Brief noch zu antworten habe, so darf ich, ganz nebenbei und
               gewissermaßen unabsichtlich die Gelegenheit benutzen und Ihnen zu Ihrem
               80. Geburtstag Glück wünschen, von dem Sie natürlich nichts hören wollen. Aber we{\geminationn} solche Daten auch nicht viel Sinn haben, – man darf zu
               einem solchen Tag rückhaltloser \strikeout{derart} allerlei
               aussprechen, was sonst vielleicht pathetisch oder sentimental klänge, und so erlauben
               Sie mir nur ganz einfach hier niederzuschreiben, daß unter den Menschen, die älter
               sind als ich – und denen ich nicht eben durch die engsten verwandtschaftlichen Bande
               verknüpft war, kaum Einer ist, dem ich so von Herzen und von Geiste zugethan war und
               bin als Ihnen, Georg Brandes – und der mir – nicht nur durch seine Werke, sondern
               durch sein Sein, sein Dasein, – \uline{mein} Bewußtsein von
               seiner Gegenwart {\pb}in der Welt so viel gegeben hat
               als Sie! Möchten Sie doch allen die Sie lieben und bewundern, noch lange erhalten
               bleiben, – und möchte es das Schicksal fügen, daß wir einander wieder einmal
               persönlich begegnen.\pend
           \pstart
           Was in jenem »\textcolor{green}{Interview}{}\ledrightnote{→\textcolor{green}{?? [nicht ermitteltes dänisches Interview]}}«
               gestanden, weiß ich natürlich nicht; – mir war es bisher ganz unbekannt, daß mich ein
                  \textcolor{pink}{daenischer}{}\ledrightnote{\textcolor{pink}{Dänemark}}{ }\textcolor{blue}{Journalist}{}\ledrightnote{→\textcolor{blue}{Julius Bangert}} interviewt hat; – es
               waren 2 oder 3 Herren aus \textcolor{pink}{Daenemark}{}\ledrightnote{\textcolor{pink}{Dänemark}} im Lauf der
               letzten Jahre bei mir, und ich habe \introOben{}mich\introOben{} mit ihnen \introOben{}über allerlei\introOben{} unterhalten, – hoffentlich war das, was diesen
               Besuchern in Erinnerung verblieben, nicht so confus wie das \label{K_L02376_1v}\edtext{\textcolor{green}{Zeug}{}\ledrightnote{→\textcolor{green}{Dr. Arthur Schnitzler on the Vienna of To-day}}}{\lemma{\textnormal{\emph{Zeug}}}\Cendnote{\textnormal{Unklar; womöglich meint er \textcolor{blue}{Joseph Gollomb}: \emph{\textcolor{green}{Dr. Arthur Schnitzler on the Vienna of To-day}}. In: \emph{\textcolor{green}{New York Evening Post}},
                     5. 6. 1920, Sec. 3, S. [1] und S. 12, vgl. A. S.: \emph{Tagebuch}, 2. 7. 1902.}}}\label{K_L02376_1h}, was ich
               gleichfalls als »Interview« mit mir, vor einem Jahr in einer \textcolor{pink}{amerikanischen}{}\ledrightnote{\textcolor{pink}{Amerika}} Zeitung zu lesen bekam – Nun Sie haben wohl
               ähnliche Erfahrungen gemacht. Es freut mich schon aus Ihrem Brief zu entnehmen, daß
               ich immerhin über {\pb}Sie, lieber Freund, nichts
               böses geäußert zu haben scheine.\pend
           \pstart
           Mit dem »\textcolor{green}{Reigen}{}\ledrightnote{\textcolor{green}{Reigen. Zehn Dialoge}}« hab ich freilich allerlei dummes
               erlebt; – was mir aber kaum nah gegangen ist. Das schli{\geminationm}ste erfährt man ja immer (auch das wird Ihnen nicht neu sein) nicht von den
               Gegnern, – sondern von den Freunden, – die den bessern Theil der Tapferkeit, die
               Vorsicht wählen. Aber es ist schon wahr, – unter den zahlreichen Affairen meines
               Lebens, ist es wohl diese letzte, in de\substVorne{}\textsuperscript{en}\substDazwischen{}r\substHinten{} Verlogenheit, Unverstand und Feigheit sich selbst übertroffen haben. (Dabei
               gesteh ich ohne weiteres zu, daß gegen die \uline{Aufführung}
               des »\textcolor{green}{Reigen}{}\ledrightnote{\textcolor{green}{Reigen. Zehn Dialoge}}« immerhin auch ehrliche Einwendungen
               möglich sind – aber \substVorne{}\textsuperscript{diese}\substDazwischen{}solche\substHinten{} ehrlichen und discutabeln Einwendungen sind eben in hundert Fällen, wo sie
               auch und besser am Platze gewesen wären, \uline{nicht}
               erhoben worden.) Ich lege hier übrigens einen \label{K_L02376_2v}\edtext{\textcolor{green}{Artikel}{}\ledrightnote{→\textcolor{green}{Berichtigung. Ein paar Worte zum Gutachten Maximilian Hardens über den »Reigen«}}}{\lemma{\textnormal{\emph{Artikel}}}\Cendnote{\textnormal{\textcolor{blue}{Artur Schnitzler}: \emph{\textcolor{green}{Berichtigung. Ein paar Worte zum Gutachten Maximilian Hardens
                        über den »Reigen«}}. In: \emph{\textcolor{green}{Neues Wiener
                        Journal}}, Jg. 29, Nr. 9782, 30. 1. 1921, S. 6}}}\label{K_L02376_2h} bei – das einzige Document, in dem ich \strikeout{selbst}
               mich \introOben{}persönlich\introOben{} zu Worte habe kommen lassen; – er erklärt
               sich selbst.\pend
           \pstart
           {\pb}Meine beiden \textcolor{blue}{Casanova}{}\ledrightnote{\textcolor{blue}{Giacomo Girolamo Casanova}}-Sachen, das Lustspiel »\textcolor{green}{die
                  Schwestern}{}\ledrightnote{\textcolor{green}{Die Schwestern oder Casanova in Spa. Lustspiel in Versen}}« und die Novelle »\textcolor{green}{Casan.
                  Heimfahrt}{}\ledrightnote{\textcolor{green}{Casanovas Heimfahrt}}« sind so entstanden, daß mir zwei Stoffe, die schon geraume Zeit
               unter meinen Papieren lagen, durch die Lectüre der \textcolor{green}{Casanova Memoiren}{}\ledrightnote{→\textcolor{green}{Aus meinem Leben}} plötzlich lebendig geworden sind. Die
               Beschäftigung damit bedeutete keine bewußte Abkehr von der Zeit. Zu den Ereignissen
               selbst hätt ich natürlich geschwiegen – gelegentlich mußte man sich nur melden, um
               gegen eine Verläumdung oder gar gegen \label{K_L02376_3v}\edtext{Mißbrauch seiner Unterschrift zu protestiren}{\lemma{\textnormal{\emph{Mißbrauch … protestiren}}}\Cendnote{\textnormal{\textcolor{blue}{Schnitzler} spielt auf seine angebliche
                  Unterschrift auf einer \textcolor{green}{Protestnote} gegen die Hinrichtung \textcolor{blue}{Ernst
                     Toller}s an, die am 11. 6. 1919 durch die Presse ging. \textcolor{blue}{Schnitzler} hatte diese nicht unterschrieben und
                  verfasste in der Folge einen Leserbrief, indem er sich gegen die ungefragte
                  Verwendung seines Namens verwehrte.}}}\label{K_L02376_3h} – Überraschungen hab ich eigentlich
               nicht erlebt, – die existiren für Unser Einen doch wohl nur in quantitativer
               Hinsicht.\pend
           \pstart
           Die Zustände in \textcolor{pink}{Wien}{}\ledrightnote{\textcolor{pink}{Wien}} sind übel genug, – die
               Preissteigerungen phantastisch 1000–2000 fach; – dabei ungeheuer viel Luxus; – und
               mehr stilles Elend als sichtbares. Die denen es am schlechtesten geht, halten weder
               Umzüge noch plündern sie. Wie es weiter gehen soll, weiß niemand. Wirkliche {\pb}Hilfe ka{\geminationn} natürlich
               von außen – auch durch die berühmten Credite, nie und ni{\geminationm}er kommen; – es müßten die außerordentlichen inneren \introOben{}national-\introOben{}oekonomischen Möglichkeiten unseres Landes mit Energie und ohne
               jede Rücksicht auf \introOben{}partei\introOben{}politische \strikeout{und} Interessen ausgenutzt werden; – aber vielleicht ist
               es heute schon zu spät dazu. An ein Zugrundegehen von \textcolor{pink}{Wien}{}\ledrightnote{\textcolor{pink}{Wien}} glaub ich nicht (etwa im Sinne von \textcolor{pink}{Venedig}{}\ledrightnote{\textcolor{pink}{Venedig}} –), aber als was es sich erheben und wieder emporblühen soll – und
               wann, das vermag ich nicht vorauszusehen. –\pend
           \pstart
           In meinen äußeren Verhältnissen – \uline{da wo sie schon die
                  innern sind} hat sich manches verändert. Von meiner \textcolor{blue}{Frau}{}\ledrightnote{→\textcolor{blue}{Olga Schnitzler}} bin ich geschieden, – aber wir sind gute
               Freunde geblieben, – ja in der letzten Zeit wieder geworden, könnte man besser sagen.
               Sie lebt vorläufig in \textcolor{pink}{Salzburg}{}\ledrightnote{\textcolor{pink}{Salzburg}}, war aber in den
               letzten Tagen in \textcolor{pink}{Wien}{}\ledrightnote{\textcolor{pink}{Wien}}, und Sie können kaum glauben,
               wie viel wir gerade von Ihnen gesprochen haben. Mein \textcolor{blue}{Sohn}{}\ledrightnote{→\textcolor{blue}{Heinrich Schnitzler}}, der {\pb}heuer zwanzig
               wird, zeigt sich in \uline{theatralibus} theoretisch und
               praktisch recht begabt, – auch musikalisch leistet er etwas. Dabei fehlt aber jede
                  \uline{falsche} Tendenz ins selbstschöpferische, – d. h.
               er dilettirt weder als Dichter noch als Componist. Ich glaube er ist der geborene
               Regisseur – und andre glauben es auch. Seine Hauptbeschäftigung ist jetzt \textcolor{blue}{Shakespeare}{}\ledrightnote{\textcolor{blue}{William Shakespeare}}; eben hat er eine Inszenierung von \textcolor{green}{Maß für Maß}{}\ledrightnote{\textcolor{green}{Maß für Maß}} gemacht – er arbeitet in der \textcolor{pink}{Hofbibliothek}{}\ledrightnote{\textcolor{pink}{Nationalbibliothek}} – jetzt \textcolor{pink}{Nationalbibliothek}{}\ledrightnote{\textcolor{pink}{Nationalbibliothek}}, – und hat auch an der \textcolor{brown}{Wanderbühne}{}\ledrightnote{\textcolor{brown}{Wanderbühne des österreichischen Volksbildungsamtes}} schon kleinere Rollen gespielt. – Meine Tochter \textcolor{blue}{Lili}{}\ledrightnote{\textcolor{blue}{Lili Schnitzler}}, zwölf vorbei geht ins Gymnasium; – declamirt die \textcolor{green}{Jungfrau von Orleans}{}\ledrightnote{\textcolor{green}{Die Jungfrau von Orleans}}, schreibt »Geschichten«, – und
               verwickelt mich jeden Morgen in die schwierigsten Gespräche über Gott und \introOben{}den\introOben{} freien Willen. Aber Landschaft, Schwimmen und
               Milchchocolade ist ihr glücklicherweise doch noch wichtiger.\pend
           \pstart
           Und von mir selber we{\geminationn} Sie erlauben schreib ich Ihnen
               nächstens. Freundschaftlich treu\pend
           \pstart Der Ihrige wie immer \spacefill\mbox{Arthur Schnitzler}\pend{}\endnumbering\briefempfaengerindex{Brandes, Georg@\textsc{Brandes, Georg}!zzzSchnitzler, Arthur@\emph{von Arthur Schnitzler}!1922-01-301@{30. 1. 1922}|)be}\mylabel{h}  \normalsize

\doendnotes{C}
\bigskip
\vfill

\clearpage

\footnotesize

\lohead{\textsc{register}}

% Definiere theindex-Environment komplett neu ohne reledmac
\makeatletter
\renewenvironment{theindex}{%
  \section*{\indexname}%
  \setlength{\parindent}{0pt}%
  \setlength{\parskip}{0pt plus 0.3pt}%
  \let\item\@idxitem
}{%
  \clearpage
}
\makeatother

\IfFileExists{\jobname-pw.ind}{\input{\jobname-pw.ind}}{}

\end{document}

      