%% latex-korrekturansicht-vorspann.tex
%% Vorspann für die Korrekturansicht.
%% Lädt die gemeinsame Datei latex-vorspann.tex mit gesetztem Schalter.

\newif\ifkorrekturansicht
\korrekturansichttrue

\input{../tex-inputs/latex-vorspann}


               \section[Arthur Schnitzler an Richard Beer-Hofmann, 20. 2. 1903]{ Arthur Schnitzler an Richard Beer-Hofmann, 20. 2. 1903}\nopagebreak\mylabel{v}\rehead{ }\normalsize\beginnumbering\briefempfaengerindex{Beer-Hofmann, Richard@\textsc{Beer-Hofmann, Richard}!zzzSchnitzler, Arthur@\emph{von Arthur Schnitzler}!1903-02-202@{20. 2. 1903}|(be} \toendnotes[C]{\smallbreak\pagebreak[2]} \Standort{YCGL, MSS 31.}
\physDesc{Brief, 1 Blatt (Briefpapier mit Trauerrand), 4 Seiten, Umschlag
\newline{}Handschrift: schwarze Tinte, deutsche Kurrent\newline{}Versand: 1) Stempel: »\nobreak{}\oindex{IX., Alsergrund@\textbf{IX., Alsergrund}, \emph{Bezirk (A.BZK)}|pwk}9/3 Wien, 20. 2. 03, 5–6N\nobreak{}«.  2) Stempel: »\nobreak{}\oindex{Rodaun@\textbf{Rodaun}, \emph{Teil eines besiedelten Ortes (A.BSOX)}|pwk}{\pb}R{[}odau{]}n, 21. 2. {[}03{]}, 7–\textcolor{gray}{9}V\nobreak{}«. }\buchAbdrucke{\weitereDrucke{Arthur Schnitzler, Richard Beer-Hofmann: \emph{Briefwechsel 1891–1931}. Hg. Konstanze Fliedl. Wien, Zürich: \emph{Europaverlag} 1992, S. 161–162.} }\toendnotes[C]{\smallbreak}\pstart{}{\pb}Herrn \textsc{Dr Richard
                     Beer-Hofmann}\pend{}\pstart{}\textcolor{pink}{Rodaun}{}\ledrightnote{\textcolor{pink}{Rodaun}}\pend{}\pstart{}\textcolor{pink}{Lieſinger Hauptstraße 2}{}\ledrightnote{\textcolor{pink}{Liesingerstraße}}\pend{}{\bigskip}\pstart
           \raggedleft{}{\pb}20. 2. 903\pend
           \pstart{}Lieber Richard,\pend\pstart
           Ihnen und \textcolor{blue}{Hugo}{}\ledrightnote{\textcolor{blue}{Hugo von Hofmannsthal}} danke ich für das Gutachten und
               theile Ihnen mit, dſs ich heute gegen vorherige Honorirung von \textcolor{green}{\introOben{}3\introOben{} Auflagen}{}\ledrightnote{→\textcolor{green}{Reigen. Zehn Dialoge}} mit dem \textcolor{brown}{Wiener Verlag}{}\ledrightnote{\textcolor{brown}{Wiener Verlag}}{ }\label{K_L01271_1v}\edtext{abgeſchloſſen}{\lemma{\textnormal{\emph{abgeſchloſſen}}}\Cendnote{\textnormal{für die Veröffentlichung des \emph{\textcolor{green}{Reigen}}, der im April erscheinen
                  sollte}}}\label{K_L01271_1h} habe. Auch die Ausſtattung wird Ihren Wünſchen entſprechend
               ausfallen. –\pend
           \pstart
           Im übrigen reiſe ich morgen {\pb}nach \textcolor{pink}{Berlin, \uline{Palaſthotel}}{}\ledrightnote{\textcolor{pink}{Palasthotel Berlin}} woſelbſt ich alſo bis etwa 8. März zu bleiben denke.\pend
           \pstart
           Mein neues \textcolor{green}{Stück}{}\ledrightnote{\textcolor{green}{Der einsame Weg. Schauspiel in fünf Akten}{\newline}\textcolor{green}{Professor Bernhardi. Komödie in fünf Akten}} in jetziger Faſſung iſt,
               nach theilweiſer Mittheilung an \textcolor{blue}{Olga}{}\ledrightnote{\textcolor{blue}{Olga Schnitzler}} und \textcolor{blue}{Schwarzkopf}{}\ledrightnote{\textcolor{blue}{Gustav Schwarzkopf}}, meinem eigenen Antrag entſprechend,
               misbilligt und damit erledigt worden. Es iſt ein ſiameſiſches {\pb}\label{K_L01271_2v}\edtext{Zwilling}{\lemma{\textnormal{\emph{Zwilling}}}\Cendnote{\textnormal{Gemeint ist die Trennung der Stoffe in \emph{\textcolor{green}{Der
                     einsame Weg}} und dem späteren \emph{\textcolor{green}{Professor
                     Bernhardi}}.}}}\label{K_L01271_2h}; vielleicht hilft eine Operation, und Sie ſehen, zur
               rechten und zur linken je einen Siam herunterſinken.\pend
           \pstart
           – Immerhin, – es iſt eine »fertige Sach« – und ſomit bin ich beſſer gelaunt als alle
               dieſe letzten Tage{\dots}\pend
           \pstart
           Überdies, Frühling!. Soll man daran glauben?{\dots} Nun,
               genug.\pend
           \pstart
           {\pb}Ich hoffe, wir ſehen uns alle, in 3 Wochen etwa,
               geſund wieder.\pend
           \pstart
           Grüßen Sie allerorten.\pend
           \pstart
           Herzlichſt Ihr{\\[\baselineskip]}\spacefill\mbox{A.}\pend
           \leftskip=0em{}\endnumbering\briefempfaengerindex{Beer-Hofmann, Richard@\textsc{Beer-Hofmann, Richard}!zzzSchnitzler, Arthur@\emph{von Arthur Schnitzler}!1903-02-202@{20. 2. 1903}|)be}\mylabel{h}  \normalsize

\doendnotes{C}
\bigskip
\vfill

\clearpage

\footnotesize

\lohead{\textsc{register}}

% Definiere theindex-Environment komplett neu ohne reledmac
\makeatletter
\renewenvironment{theindex}{%
  \section*{\indexname}%
  \setlength{\parindent}{0pt}%
  \setlength{\parskip}{0pt plus 0.3pt}%
  \let\item\@idxitem
}{%
  \clearpage
}
\makeatother

\IfFileExists{\jobname-pw.ind}{\input{\jobname-pw.ind}}{}

\end{document}

      