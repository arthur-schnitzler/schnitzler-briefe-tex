%% latex-korrekturansicht-vorspann.tex
%% Vorspann für die Korrekturansicht.
%% Lädt die gemeinsame Datei latex-vorspann.tex mit gesetztem Schalter.

\newif\ifkorrekturansicht
\korrekturansichttrue

\input{../tex-inputs/latex-vorspann}


               \section[Albert Ehrenstein an Arthur Schnitzler, 13. 7. 1909]{ Albert Ehrenstein an Arthur Schnitzler, 13. 7. 1909}\nopagebreak\mylabel{v}\rehead{ }\normalsize\beginnumbering\briefempfaengerindex{Schnitzler, Arthur@\textsc{Schnitzler, Arthur}!zzzEhrenstein, Albert@\emph{von Albert Ehrenstein}!1909-07-131@{13. 7. 1909}|(be} \toendnotes[C]{\smallbreak\pagebreak[2]} \Standort{CUL, Schnitzler, B 30.}
\physDesc{Brief, 4 Blätter, 4 Seiten (Paginierung)
\newline{}Handschrift: schwarze Tinte, deutsche Kurrent
\newline{}Schnitzler: mit Bleistift beschriftet: »\textsc{Ehrenstein}« }\buchAbdrucke{\weitereDrucke{Albert Ehrenstein: \emph{Briefe}. Hg. Hanni Mittelmann. München: \emph{Boer} 1989, S. 29–31 (Werke, 1).} }\toendnotes[C]{\smallbreak}\pstart
           {\pb}\textcolor{pink}{\textsc{Wien, XVI. Ottakringerstr.} 114}{}\ledrightnote{\textcolor{pink}{Ottakringerstraße}}.\hfill 13. Juli 09.\pend
           \pstart{}\textsc{Sehr geehrter Herr Doktor!}\pend\pstart
           Ihr freundlicher Brief gab mir gerade jetzt einigen Troſt. Mein \textcolor{blue}{Geſchichtsprofeſſor}{}\ledrightnote{→\textcolor{blue}{August Fournier}} nämlich, mit einem
                    ewigen Bronchialkatarrh behaftet und daher außerordentlich ſekant, hat mir die
                    Ehre erwieſen, mir meine \textcolor{green}{Diſſertation}{}\ledrightnote{→\textcolor{green}{Die Lage in Ungarn (Siebenbürgen und Serbien ausgenommen) im Jahre 1790}} zur gänzlichen Umarbeitung zurückzugeben. Hätte der gute
                        \textcolor{blue}{Mann}{}\ledrightnote{→\textcolor{blue}{August Fournier}} bei dieſer
                    Abweiſung imponierendes Sachverſtändnis dokumentiert, ſo wäre dawider wohl
                    nichts einzuwenden geweſen. Aber das war nicht allzuſehr der Fall. Eine
                    übergroße und malitiöſe Empfindlichkeit modernerem und zugreifenderem Ausdruck
                    und Satzbau gegenüber verführte ihn ſogar dazu, mir faſt auf jeder Seite Mängel
                    ſtiliſtiſcher Natur nachweiſen zu wollen. Wozu erſtens der \textcolor{blue}{Verfaſſer}{}\ledrightnote{→\textcolor{blue}{August Fournier}} des langweiligsten \textcolor{green}{\textcolor{blue}{Napoleon}{}\ledrightnote{\textcolor{blue}{Napoleon Bonaparte}}buches}{}\ledrightnote{→\textcolor{green}{Napoleon I. Eine Biographie}} nicht das Recht
                    hatte, zweitens – und das iſt die komiſche Seite der Affaire – habe ich einem
                        \textcolor{pink}{galiziſchen}{}\ledrightnote{\textcolor{pink}{Galizien}}{ }\textcolor{blue}{Kollegen}{}\ledrightnote{→\textcolor{blue}{?? [Studienkollege von Albert Ehrenstein]}}, der nicht gut
                    Deutſch kann, ſeine Arbeit durchgeſehen und die gröbſten Verſtöße darin
                    korrigiert. Bei dem hat der \textcolor{blue}{Hofrat}{}\ledrightnote{→\textcolor{blue}{August Fournier}} merkwürdigerweiſe wenig Stilwidrigkeiten zu regiſtrieren
                    gehabt. Warum? Weil ich dem \textcolor{pink}{Polen}{}\ledrightnote{\textcolor{pink}{Polen}} den Tric
                    angeraten hatte, dem \textcolor{blue}{Profeſſor}{}\ledrightnote{→\textcolor{blue}{August Fournier}} von vornherein weiszumachen, er werde ſeine Diſſertation \textcolor{pink}{polniſch}{}\ledrightnote{\textcolor{pink}{Polen}} drucken laſſen. Da begann des \textcolor{blue}{Profeſſor}{}\ledrightnote{→\textcolor{blue}{August Fournier}}s Eigenliebe und
                    Nationalgefühl zu funktionieren. Eine aus ſeinem, einem Deutſchen Seminar
                    hervorgegangene Abhandlung ſollte anderswo, in einer slawiſchen Sprache
                    erſcheinen? Lieber veranlaßte er – was beabſichtigt war – die Drucklegung des
                    Manuſkriptes in Deutſcher Sprache, {\pb}hatte an dem von ihm empfohlenen \textcolor{green}{Werke}{}\ledrightnote{→\textcolor{green}{?? [Dissertation]}} (von dem er übrigens auch nicht viel verſteht)
                    wenig zu bekritteln und prüfte den \textcolor{blue}{Polen}{}\ledrightnote{→\textcolor{blue}{?? [Studienkollege von Albert Ehrenstein]}} nicht, ſondern plauſchte mit ihm beim
                    Rigoroſum. Unglücklicherweiſe kann ich nicht \textcolor{pink}{magyariſch}{}\ledrightnote{\textcolor{pink}{Ungarn}} und daher nicht mit dem \textcolor{pink}{magyariſchen}{}\ledrightnote{\textcolor{pink}{Ungarn}} Erſcheinen meines \textcolor{pink}{ungariſche}{}\ledrightnote{\textcolor{pink}{Ungarn}} Verhältniſſe gloſſierenden \textcolor{green}{Elaborates}{}\ledrightnote{→\textcolor{green}{Die Lage in Ungarn (Siebenbürgen und Serbien ausgenommen) im Jahre 1790}} dienen.\pend
           \pstart
           Obgleich die Umarbeitung nur 3 Wochen in Anſspruch nahm, wurde ich, da es nur
                    3 Lehramtsprüfungstermine im Jahr gibt und ich einen durch die Nichtannahme
                    meiner Diſſertation verſäumen mußte, aus meiner Bahn geworfen, ich kann meinen
                    urſprünglichen Plan nicht ausführen, werde um ein halbes Jahr ſpäter mit dem
                    lächerlichen Namen- und Zahlenkram fertig werden, und außerdem – ich hatte ſchon
                        1908 keine Ferien – gibt es auch heuer keine Erholung für mich.
                    Im Oktober wird meine \textcolor{green}{Abhandlung}{}\ledrightnote{→\textcolor{green}{Helena}} in ihrer neuen Form zenſiert. Mich noch
                    weiterhin von dem \textcolor{blue}{Profeſſor}{}\ledrightnote{→\textcolor{blue}{August Fournier}} wie einen Schuldigen behandeln zu laſſen, habe ich keine
                    Luſt. Es iſt kaum ein Verbrechen, wenn man ſich einen biſſigen Hofrat mit einem
                    Stückchen Wurſt vom Leibe hält, ebenſowenig halte ich es für korrupt, im Regen
                    einen Schirm aufzuſpannen. Aus dieſer Weltanſchauung heraus muß ich es mit
                    Freude begrüßen, wenn Sie, ſehr geehrter Herr Doktor, die Liebenswürdigkeit
                    beſäßen, Herrn \textcolor{blue}{Auernheimer}{}\ledrightnote{\textcolor{blue}{Raoul Auernheimer}} gegenüber ein
                    paar Worte über mich fallen zu laſſen. Ich möchte nämlich dann gern Ende
                        Juli Herrn \textcolor{blue}{Auernheimer}{}\ledrightnote{\textcolor{blue}{Raoul Auernheimer}} eine Notiz
                    über die im Erſcheinen begriffene \textcolor{green}{Diſſertation}{}\ledrightnote{→\textcolor{green}{?? [Dissertation]}} jenes \textcolor{pink}{galiziſchen}{}\ledrightnote{\textcolor{pink}{Galizien}}{ }\textcolor{blue}{Kollegen}{}\ledrightnote{→\textcolor{blue}{?? [Studienkollege von Albert Ehrenstein]}}{ }{\pb}ſowie meinen \textcolor{green}{Baber}{}\ledrightnote{\textcolor{green}{Tod des Zehir eddin Muhammed Baber}} einſenden. Kurze Kritiken über Belletriſtiker
                    einſchicken, was mir \textcolor{blue}{Auernheimer}{}\ledrightnote{\textcolor{blue}{Raoul Auernheimer}} geſtattete,
                    mag ich nicht; ich ſehne mich nicht danach, mich mit irgendwelchen Literaten
                    durch Tauſchhandel zu verfreunden, in meiner gegenwärtigen Stimmung würde ich
                    übrigens ſelbſt den Herrgott zu diskreditieren verſuchen, und das eine wie das
                    andere darf doch eigentlich nur einer, der durch eigene Schöpfungen öffentlich
                    einen gewiſſen Befähigungsnachweis erbracht hat. Die Notiz über die von ihm
                    empfohlene \textcolor{green}{Diſſertation}{}\ledrightnote{→\textcolor{green}{?? [Dissertation]}}
                    würde den Hiſtoriker umgänglicher machen, der \textcolor{green}{Baber}{}\ledrightnote{\textcolor{green}{Tod des Zehir eddin Muhammed Baber}} – den ich ſonſt in aller Eile anderweitig unterzubringen das
                    gefährliche und bei meinem Mangel an Beziehungen auch ausſichtsloſe Wagnis
                    unternehmen müßte – würde ihm imponieren, den \textcolor{blue}{Geographieprofeſſor}{}\ledrightnote{→\textcolor{blue}{Eugen Oberhummer}}, der uns die \textcolor{green}{Memoiren}{}\ledrightnote{→\textcolor{green}{Baburnama}} dieſes \textcolor{blue}{Regenten}{}\ledrightnote{→\textcolor{blue}{Zahir ad-Din Muhammad Babur}} namhaft machte, freuen. Daher,
                    um ſozuſagen als Reſpektsperſon wenigſtens Chikanen zu entgehen, wäre es mir
                    wirklich ſehr angenehm, wenn Herr \textcolor{blue}{Auernheimer}{}\ledrightnote{\textcolor{blue}{Raoul Auernheimer}} nicht (wie im Feber) ſich ausſchließlich
                    darauf beſchränkte, in meinen Manuſkripten hin und wieder einen Beiſtrich
                    anzubringen, was mich beluſtigte, oder ab und zu ein »Sehr ſchön«
                    hinzuſchreiben, was mich ärgerte. Heute noch würde es mich freuen und mir in
                    vieler Beziehung helfen, wenn die \textcolor{brown}{Preſſe}{}\ledrightnote{\textcolor{brown}{Neue Freie Presse}} oder
                    ſonſt ein Blatt mich lancierte, in ein bis zwei Jahren, wenn ich einen Poſten
                    habe, wird es mir ſehr gleichgültig ſein, ob mein Name in einer Zeitung ſteht,
                    oder ob ich ihn mit dem Spazierſtock auf einen in der Sonne zerrinnenden
                    Schneehaufen ſchreibe. {\pb}Die Ehre iſt
                    ſchließlich ſchon jetzt nicht gar ſo überwältigend. Und ſpäter, wenn ich einmal
                    bekannt ſein werde – ich bin ſchrecklich rachſüchtig – würden die Zeitungen
                    zunächſt doch nichts anderes von mir bekommen als die von ihnen ſelbſt
                    abgelehnten Sachen. Den Luxus, derartige Prinzipien \introOben{}zu\introOben{}
                    beſitzen zu glauben, kann ich mir ja jetzt noch getroſt geſtatten.\pend
           \pstart
           Indem ich zwar auf eine gnädige Erfüllung meiner \introOben{}unbeſcheidenen\introOben{} Wünſche hoffe, nichtsdeſtoweniger auch auf eine
                    ſtrenge Kritik meiner novelliſtiſchen Taſtverſuche und moraliſchen Grundsätze
                    gefaßt mache, verbleibe ich hochachtungsvoll\pend
           \pstart
           Ihr ergebenſter{\\[\baselineskip]}\spacefill\mbox{Albert Ehrenstein.}\pend
           \leftskip=0em{}\endnumbering\briefempfaengerindex{Schnitzler, Arthur@\textsc{Schnitzler, Arthur}!zzzEhrenstein, Albert@\emph{von Albert Ehrenstein}!1909-07-131@{13. 7. 1909}|)be}\mylabel{h}  \normalsize

\doendnotes{C}
\bigskip
\vfill

\clearpage

\footnotesize

\lohead{\textsc{register}}

% Definiere theindex-Environment komplett neu ohne reledmac
\makeatletter
\renewenvironment{theindex}{%
  \section*{\indexname}%
  \setlength{\parindent}{0pt}%
  \setlength{\parskip}{0pt plus 0.3pt}%
  \let\item\@idxitem
}{%
  \clearpage
}
\makeatother

\IfFileExists{\jobname-pw.ind}{\input{\jobname-pw.ind}}{}

\end{document}

      