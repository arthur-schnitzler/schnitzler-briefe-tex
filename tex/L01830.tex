%% latex-korrekturansicht-vorspann.tex
%% Vorspann für die Korrekturansicht.
%% Lädt die gemeinsame Datei latex-vorspann.tex mit gesetztem Schalter.

\newif\ifkorrekturansicht
\korrekturansichttrue

\input{../tex-inputs/latex-vorspann}


               \section[Robert Adam an Arthur Schnitzler, 25. 2. 1909]{ Robert Adam an Arthur Schnitzler, 25. 2. 1909}\nopagebreak\mylabel{v}\rehead{ }\normalsize\beginnumbering\briefempfaengerindex{Schnitzler, Arthur@\textsc{Schnitzler, Arthur}!zzzAdam, Robert@\emph{von Robert Adam}!1909-02-251@{25. 2. 1909}|(be} \toendnotes[C]{\smallbreak\pagebreak[2]} \Standort{DLA, A:Schnitzler, HS.NZ85.1.4230,1.}
\physDesc{Brief, 1 Blatt, 1 Seite
\newline{}Handschrift: schwarze Tinte, deutsche Kurrent
\newline{}Schnitzler: 1) mit Bleistift beschriftet: »\textsc{Ada\textcolor{gray}{m}}« 2) mit rotem Buntstift zwei Unterstreichungen}\toendnotes[C]{\smallbreak}\pstart
           \raggedleft{}{\pb}\textcolor{pink}{Wien}{}\ledrightnote{\textcolor{pink}{Wien}}, am 25. Febr. 1909\pend
           \pstart{}Hochverehrter Herr Doktor!\pend\pstart
           Ich bin ſo frei, Ihnen als Zeichen meiner Hochſchätzung meine Komödie: »\label{K_L01830_1v}\edtext{\textcolor{green}{Die Geſchichte des Alî ibn Bekkâr mit Schams
                        an-Nahâr}{}\ledrightnote{\textcolor{green}{Die Geschichte des Alî ibn Bekkâr mit Schams an-Nahâr}}}{\lemma{\textnormal{\emph{Die … an-Nahâr}}}\Cendnote{\textnormal{\emph{\textcolor{green}{Die Geschichte des Alî ibn Bekkâr mit
                                Schams an-Nahâr}}. Eine Komödie von \textcolor{blue}{Robert Adam}. Wien und Leipzig: \emph{\textcolor{brown}{Hugo Heller {\kaufmannsund} Cie.}}{ }1909 (erschienen im Februar).}}}\label{K_L01830_1h}« zu überſenden und
                    bitte Sie, mein Buch Ihrer Aufmerkſamkeit für wert zu erachten.\pend
           \pstart
           Ihr ergebener{\\[\baselineskip]}\spacefill\mbox{Robert Adam}\pend
           \leftskip=0em{}\pstart
           \noindent{}\textcolor{pink}{Wien XII/\textsubscript{1} Meidlinger
                            Hauptſtr. 56}{}\ledrightnote{\textcolor{pink}{Meidlinger Hauptstraße}}\pend
           \endnumbering\briefempfaengerindex{Schnitzler, Arthur@\textsc{Schnitzler, Arthur}!zzzAdam, Robert@\emph{von Robert Adam}!1909-02-251@{25. 2. 1909}|)be}\mylabel{h}  \normalsize

\doendnotes{C}
\bigskip
\vfill

\clearpage

\footnotesize

\lohead{\textsc{register}}

% Definiere theindex-Environment komplett neu ohne reledmac
\makeatletter
\renewenvironment{theindex}{%
  \section*{\indexname}%
  \setlength{\parindent}{0pt}%
  \setlength{\parskip}{0pt plus 0.3pt}%
  \let\item\@idxitem
}{%
  \clearpage
}
\makeatother

\IfFileExists{\jobname-pw.ind}{\input{\jobname-pw.ind}}{}

\end{document}

      