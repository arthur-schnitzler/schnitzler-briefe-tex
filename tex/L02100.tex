%% latex-korrekturansicht-vorspann.tex
%% Vorspann für die Korrekturansicht.
%% Lädt die gemeinsame Datei latex-vorspann.tex mit gesetztem Schalter.

\newif\ifkorrekturansicht
\korrekturansichttrue

\input{../tex-inputs/latex-vorspann}


               \section[Georg Brandes: Widmungsexemplar Armand Carrel für Arthur Schnitzler, {[}nach dem 16. 11. 1912{]}]{ Georg Brandes: Widmungsexemplar Armand Carrel für Arthur
                    Schnitzler, {[}nach dem 16. 11. 1912{]}}\nopagebreak\mylabel{v}\rehead{ }\normalsize\beginnumbering\briefempfaengerindex{Schnitzler, Arthur@\textsc{Schnitzler, Arthur}!zzzBrandes, Georg@\emph{von Georg Brandes}!1912-11-162@{{[}nach dem 16. 11. 1912{]}}|(be} \toendnotes[C]{\smallbreak\pagebreak[2]} \Standort{DLA, G:Schnitzler, Arthur (Sammlung Heinrich Schnitzler).}
\physDesc{Widmung am Vorsatzblatt
\newline{}Handschrift: schwarze Tinte, lateinische Kurrent\newline{}Ordnung: mit Bleistift von unbekannter Hand das Pseudonym der
                                            Übersetzerin aufgelöst: »\textcolor{blue}{Prager Mathilde}« }\toendnotes[C]{\smallbreak}\pstart
           \noindent{}{\pb}An Arthur Schnitzler\pend
           \pstart
           Diese Bagatelle, \label{K_L02100_1v}\edtext{Diomedes’
                    Geschenk an Glaukos}{\lemma{\textnormal{\emph{Diomedes’ … Glaukos}}}\Cendnote{\textnormal{Glaukos erneuert
                        den Freundschaftsbund, er gibt Diomedes eine goldene, dieser ihm eine eherne
                        Rüstung.}}}\label{K_L02100_1h}, (\textcolor{green}{Ilias}{}\ledrightnote{\textcolor{green}{Ilias}} IV 235) soll nur
                    ein Zeichen treuer Freundschaft\pend
           \pstart \spacefill\mbox{G.B.}\pend{}{\bigskip}\pstart
           \noindent{}\centering{}\textcolor{gray}{\textbf{\textcolor{green}{Armand Carrel}{}\ledrightnote{\textcolor{green}{Armand Carrel}}}}\pend
           {\bigskip}\pstart
           \noindent{}\centering{}{\pb}\textcolor{gray}{\textbf{\textcolor{green}{Armand Carrel}{}\ledrightnote{\textcolor{green}{Armand Carrel}}}}\pend
           \pstart
           \noindent{}\centering{}\textcolor{gray}{\textbf{Von}}\pend
           \pstart
           \noindent{}\centering{}\textcolor{gray}{\textbf{Georg Brandes}}\pend
           \pstart
           \noindent{}\centering{}\textcolor{gray}{\textbf{Autoriſierte Überſetzung von \textcolor{blue}{\so{Erich Holm}}{}\ledrightnote{\textcolor{blue}{Erich Holm}}}}\pend
           {\bigskip}\pstart
           \noindent{}\centering{}\textcolor{gray}{\textbf{\textcolor{pink}{Stuttgart}{}\ledrightnote{\textcolor{pink}{Karlsruhe}} und \textcolor{pink}{Berlin}{}\ledrightnote{\textcolor{pink}{Berlin}}{ }\label{K_L02100_2v}\edtext{1913}{\lemma{\textnormal{\emph{1913}}}\Cendnote{\textnormal{am 16. 11. 1912 vom \emph{\textcolor{green}{Börsenblatt für den deutschen
                     Buchhandel}} als Neuerscheinung gemeldet}}}\label{K_L02100_2h}}}\pend
           \pstart
           \noindent{}\centering{}\textcolor{gray}{\textbf{\textcolor{brown}{J. G. Cotta’ſche Buchhandlung Nachfolger}{}\ledrightnote{\textcolor{brown}{J.G. Cotta’sche Buchhandlung Nachfolger}}}}\pend
           \endnumbering\briefempfaengerindex{Schnitzler, Arthur@\textsc{Schnitzler, Arthur}!zzzBrandes, Georg@\emph{von Georg Brandes}!1912-11-162@{{[}nach dem 16. 11. 1912{]}}|)be}\mylabel{h}  \normalsize

\doendnotes{C}
\bigskip
\vfill

\clearpage

\footnotesize

\lohead{\textsc{register}}

% Definiere theindex-Environment komplett neu ohne reledmac
\makeatletter
\renewenvironment{theindex}{%
  \section*{\indexname}%
  \setlength{\parindent}{0pt}%
  \setlength{\parskip}{0pt plus 0.3pt}%
  \let\item\@idxitem
}{%
  \clearpage
}
\makeatother

\IfFileExists{\jobname-pw.ind}{\input{\jobname-pw.ind}}{}

\end{document}

      