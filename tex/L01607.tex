%% latex-korrekturansicht-vorspann.tex
%% Vorspann für die Korrekturansicht.
%% Lädt die gemeinsame Datei latex-vorspann.tex mit gesetztem Schalter.

\newif\ifkorrekturansicht
\korrekturansichttrue

\input{../tex-inputs/latex-vorspann}


               \section[Hermann Bahr an Arthur Schnitzler, 4. 7. 1906]{ Hermann Bahr an Arthur Schnitzler, 4. 7. 1906}\nopagebreak\mylabel{v}\rehead{ }\normalsize\beginnumbering\briefempfaengerindex{Schnitzler, Arthur@\textsc{Schnitzler, Arthur}!zzzBahr, Hermann@\emph{von Hermann Bahr}!1906-07-041@{4. 7. 1906}|(be} \toendnotes[C]{\smallbreak\pagebreak[2]} \Standort{CUL, Schnitzler, B 5b.}
\physDesc{Postkarte
\newline{}Handschrift: Bleistift, deutsche Kurrent\newline{}Versand: 1) Stempel: »\nobreak{}\oindex{Bahnhof@\textbf{Bahnhof}, \emph{Bahnhofsgebäude (K.BHF)}|pwk}Venezia Ferrovia, {[}4. 7.{]} 06, 2S\nobreak{}«.  2) Stempel: »\nobreak{}\oindex{XVIII., Waehring@\textbf{XVIII., Währing}, \emph{Bezirk (A.BZK)}|pwk}18/1 Wien, 6. VII. 06, Bestellt\nobreak{}«. \newline{}Ordnung: mit Bleistift von unbekannter Hand nummeriert:
                              »140« }\buchAbdrucke{\weitereDrucke{Hermann Bahr, Arthur Schnitzler: \emph{Briefwechsel, Aufzeichnungen, Dokumente (1891–1931)}. Hg. Kurt Ifkovits und Martin Anton Müller. Göttingen: \emph{Wallstein} 2018, S. 380.} }\toendnotes[C]{\smallbreak}\pstart{}{\pb}\textsc{D\textsuperscript{r} Artur Schnitzler}\pend{}\pstart{}\textcolor{pink}{\textsc{XVIII Spöttelgasse 7}}{}\ledrightnote{\textcolor{pink}{XVIII., Währing}}\pend{}\pstart{}\textcolor{pink}{\textsc{Wien}}{}\ledrightnote{\textcolor{pink}{Wien}}\pend{}\pstart{}\textsc{\textcolor{pink}{Austria}{}\ledrightnote{\textcolor{pink}{Österreich}}}\pend{}{\bigskip}\pstart
           \raggedleft{}{\pb}\textcolor{pink}{\textsc{Venezia}}{}\ledrightnote{\textcolor{pink}{Venedig}}{ }4. 7. 06{\\}\textsc{\textcolor{pink}{Casa}{}\ledrightnote{\textcolor{pink}{Casa Petrarca}}{ }\textcolor{blue}{Petrarca}{}\ledrightnote{→\textcolor{blue}{Francesco Petrarca}}}\pend
           \pstart
           Dank ſchön, lieber Artur. Dein Brief hat mir eine große Freude
               gemacht, und Luſt, ſolchen zweiten und dritten \textcolor{green}{Akt}{}\ledrightnote{→\textcolor{green}{Der Faun}} wirklich zu ſchreiben. Neugierig, was \textcolor{blue}{Brahm}{}\ledrightnote{\textcolor{blue}{Otto Brahm}}{ }ſagen wird. – Hier herrlichſt, obwol mir die Sonne die
               Beine ſo verbrannt hat, daß ſie zwei Tage in Bleiwaſſer gelegt werden mußten. – Grüß
               Frau \textcolor{blue}{Olga}{}\ledrightnote{\textcolor{blue}{Olga Schnitzler}} herzlichſt und laßt es Euch gut gehen
               und ſchreib Deine Adreſſe\pend
           \pstart
           Deinem alten{\\[\baselineskip]}\spacefill\mbox{Hermann{ }\textcolor{green}{Faun}{}\ledrightnote{→\textcolor{green}{Der Faun}}}\pend
           \leftskip=0em{}\endnumbering\briefempfaengerindex{Schnitzler, Arthur@\textsc{Schnitzler, Arthur}!zzzBahr, Hermann@\emph{von Hermann Bahr}!1906-07-041@{4. 7. 1906}|)be}\mylabel{h}  \normalsize

\doendnotes{C}
\bigskip
\vfill

\clearpage

\footnotesize

\lohead{\textsc{register}}

% Definiere theindex-Environment komplett neu ohne reledmac
\makeatletter
\renewenvironment{theindex}{%
  \section*{\indexname}%
  \setlength{\parindent}{0pt}%
  \setlength{\parskip}{0pt plus 0.3pt}%
  \let\item\@idxitem
}{%
  \clearpage
}
\makeatother

\IfFileExists{\jobname-pw.ind}{\input{\jobname-pw.ind}}{}

\end{document}

      