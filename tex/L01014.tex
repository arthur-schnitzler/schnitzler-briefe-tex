%% latex-korrekturansicht-vorspann.tex
%% Vorspann für die Korrekturansicht.
%% Lädt die gemeinsame Datei latex-vorspann.tex mit gesetztem Schalter.

\newif\ifkorrekturansicht
\korrekturansichttrue

\input{../tex-inputs/latex-vorspann}


               \section[Arthur Schnitzler an Richard Beer-Hofmann, 17. 2. 1900]{ Arthur Schnitzler an Richard Beer-Hofmann, 17. 2. 1900}\nopagebreak\mylabel{v}\rehead{ }\normalsize\beginnumbering\briefempfaengerindex{Beer-Hofmann, Richard@\textsc{Beer-Hofmann, Richard}!zzzSchnitzler, Arthur@\emph{von Arthur Schnitzler}!1900-02-171@{17. 2. 1900}|(be} \toendnotes[C]{\smallbreak\pagebreak[2]} \Standort{YCGL, MSS 31.}
\physDesc{Brief, 2 Blätter, 5 Seiten, Umschlag
\newline{}Handschrift: 1) schwarze Tinte, deutsche Kurrent (\noindent{}Umschlag)\hspace{1em}2) Bleistift, deutsche Kurrent\hspace{1em}\newline{}Versand: 1) nachgesandt nach »\textsc{poste restante \textcolor{pink}{Sanremo}}«  2) Stempel: »\nobreak{}\oindex{I., Innere Stadt@\textbf{I., Innere Stadt}, \emph{Bezirk (A.BZK)}|pwk}Wien 1, 17. 2. 00, 11–12N\nobreak{}«. 3) Stempel: »\nobreak{}\oindex{Pegli@\textbf{Pegli}, \emph{http://www.geonames.org/ontologyP.PPLX}|pwk}{\pb}Pegli
                                 (G\textcolor{gray}{eno}va), \textcolor{gray}{19}{[} 2. 1900{]}\nobreak{}«. 4) Stempel: »\nobreak{}\oindex{Sanremo@\textbf{Sanremo}, \emph{https://www.geonames.org/ontologyP.PPLA3}|pwk}\textcolor{gray}{Sanremo} (Porto
                              Maurizio), 20 2 {[}0{]}0, 7M\nobreak{}«. }\buchAbdrucke{\weitereDrucke{Arthur Schnitzler, Richard Beer-Hofmann: \emph{Briefwechsel 1891–1931}. Hg. Konstanze Fliedl. Wien, Zürich: \emph{Europaverlag} 1992, S. 141–142.} }\toendnotes[C]{\smallbreak}\pstart{}{\pb}\textcolor{pink}{\textsc{Italia}}{}\ledrightnote{\textcolor{pink}{Italien}}\pend{}\pstart{}Herrn \textsc{Dr. Richard Beer-Hofmann}\pend{}\pstart{}\textcolor{pink}{\textsc{Pegli} bei \textsc{Genua}}{}\ledrightnote{\textcolor{pink}{Pegli}}\pend{}\pstart{}\textcolor{pink}{\textsc{Grand Hotel Mediterranée}}{}\ledrightnote{\textcolor{pink}{Grand Hotel Mediterranée}}\pend{}{\bigskip}\pstart
           \raggedleft{}{\pb}17. 2. 1900.\pend
           \pstart
           Mein lieber Richard, \textcolor{blue}{Paul}{}\ledrightnote{\textcolor{blue}{Paul Goldmann}} wohnt \textcolor{pink}{Berlin}{}\ledrightnote{\textcolor{pink}{Berlin}}, \textcolor{pink}{Hotel Saxonia}{}\ledrightnote{\textcolor{pink}{Hotel Saxonia}}, in der \textcolor{pink}{Königgrätzer Straße}{}\ledrightnote{\textcolor{pink}{Stresemannstraße}}; ſein Onkel heißt \textcolor{blue}{Fedor}{}\ledrightnote{\textcolor{blue}{Fedor Mamroth}}, und ich komme nicht nach \textcolor{pink}{Italien}{}\ledrightnote{\textcolor{pink}{Italien}}. Was ich mache? – eine \textcolor{green}{Novelle}{}\ledrightnote{→\textcolor{green}{Frau Bertha Garlan. Roman}} ſchreiben, an der ich zeitweilig Freude habe,
               meinem Ohrenſauſen zuhören und dem was es bedeutet, – mich meiſtens einſam, oder
               beſſer vereinſamt, oder noch beſſer – {\pb}vereinſamend
               fühlen – Ihnen heut eine \textcolor{green}{\textsc{Beatrice}}{}\ledrightnote{\textcolor{green}{Der Schleier der Beatrice. Schauspiel in fünf Akten}} geſchickt haben – und Sie – ohne Neid – beneiden. –\pend
           \pstart
           Ich möchte aber auch wiſſen, was Sie machen, ob Sie ſich wohl fühlen, ob ſich Ihre
                  \textcolor{blue}{Frau}{}\ledrightnote{→\textcolor{blue}{Paula Beer-Hofmann}} erholt hat, ob Sie was
               arbeiten, ob Sie Menſchen kennen gelernt haben, ob Sie ſchon eine Nachricht von \textcolor{blue}{Hugo}{}\ledrightnote{\textcolor{blue}{Hugo von Hofmannsthal}} haben. –\pend
           \pstart
           Seit Sie und \textcolor{blue}{Hugo}{}\ledrightnote{\textcolor{blue}{Hugo von Hofmannsthal}} weg ſind, bin {\pb}ich faſt nie im \textcolor{brown}{Club}{}\ledrightnote{→\textcolor{brown}{Wiener Schachclub}}. \textsc{\textcolor{blue}{Wasserma{\geminationn}}{}\ledrightnote{\textcolor{blue}{Jakob Wassermann}}}, auch \textcolor{blue}{\textsc{Leo}}{}\ledrightnote{\textcolor{blue}{Leo Van-Jung}} ſind
               beinah allabendlich bei dem aſthmatiſchen \textcolor{blue}{Naſchauer}{}\ledrightnote{\textcolor{blue}{Paul Naschauer}}; ich war \label{K_L01014_1v}\edtext{2mal dort}{\lemma{\textnormal{\emph{2mal dort}}}\Cendnote{\textnormal{siehe A. S.: \emph{Tagebuch}, 4. 2. 1900 und A. S.: \emph{Tagebuch}, 12. 2. 1900}}}\label{K_L01014_1h} und habe bei dieſer Gelegenheit einmal 21,
               einmal Poker mit \textcolor{blue}{\textsc{Herzl}}{}\ledrightnote{\textcolor{blue}{Theodor Herzl}}
               und den \textcolor{blue}{\textsc{Naſchaueri{\geminationn}en}}{}\ledrightnote{\textcolor{blue}{Julie Herzl}{\newline}\textcolor{blue}{Therese Czopp}{\newline}\textcolor{blue}{Ella Naschauer}{\newline}\textcolor{blue}{Helene Eisner}} geſpielt. –\pend
           \pstart
           Ein neues \textcolor{green}{Buch}{}\ledrightnote{→\textcolor{green}{Wiener Bummelgeschichten}}, von dem
               dampfenden Jüngling \textcolor{blue}{\textsc{Messer}}{}\ledrightnote{\textcolor{blue}{Max Messer}} verfaſſt, werd ich Ihnen ſchicken, damit Ihnen auch in \textcolor{pink}{\textsc{Pegli}}{}\ledrightnote{\textcolor{pink}{Pegli}} ein{\pb}mal übel wird. – Der
                  \textcolor{green}{Roman}{}\ledrightnote{→\textcolor{green}{Im toten Wasser. Ein Wiener Roman}} von \textcolor{blue}{Wolff}{}\ledrightnote{\textcolor{blue}{Ludwig Wolff}} iſt ſehr anſtändig intentionirt und ohne
               Geſchmackloſigkeiten\pend
           \pstart
           Mit Vergnügen les’ ich die \textcolor{blue}{\textsc{Kuh}}{}\ledrightnote{\textcolor{blue}{Emil Kuh}}{ }\textcolor{green}{\textcolor{blue}{\textsc{Hebb}}{}\ledrightnote{\textcolor{blue}{Friedrich Hebbel}}{[}el{]} Biographie}{}\ledrightnote{→\textcolor{green}{Biographie Friedrich Hebbels}}. Den \textcolor{green}{Götterliebling}{}\ledrightnote{\textcolor{green}{Der Tod Georgs}} heb
               ich mir auf einen Frühlingstag auf dem Land auf. Denken Sie, dſs Ihr \textcolor{green}{Buch}{}\ledrightnote{→\textcolor{green}{Der Tod Georgs}} erſt vor 2 Tagen hier in den Buchhdlg
                  angeko{\geminationm}en iſt. Frau \textcolor{blue}{Elly Hirſchfeld}{}\ledrightnote{\textcolor{blue}{Elly Petersen}} – um Ihnen nichts zu verſchweigen – iſt ſchon ganz, beinah
               ganz geſund, und \textcolor{blue}{Georg H.}{}\ledrightnote{\textcolor{blue}{Georg Hirschfeld}} iſt mir wieder viel {\pb}ſympathiſcher geworden. Frau \textcolor{blue}{Fulda}{}\ledrightnote{\textcolor{blue}{Ida d’Albert}} iſt ſeit ein paar Tagen in \textcolor{pink}{Wien}{}\ledrightnote{\textcolor{pink}{Wien}}, \textsc{resp}. \textcolor{pink}{Hietzing}{}\ledrightnote{\textcolor{pink}{XIII., Hietzing}}. – \textcolor{blue}{\textsc{Schlenther}}{}\ledrightnote{\textcolor{blue}{Paul Schlenther}} hat die \textcolor{green}{\textsc{Bea}.}{}\ledrightnote{\textcolor{green}{Der Schleier der Beatrice. Schauspiel in fünf Akten}} in im ganzen recht vernünftiger Weiſe zuſa{\geminationm}engeſtrichen u. iſt jetzt auch für \textcolor{blue}{Kainz}{}\ledrightnote{\textcolor{blue}{Josef Kainz}}{ }\textcolor{green}{Dichter}{}\ledrightnote{→\textcolor{green}{Der Schleier der Beatrice. Schauspiel in fünf Akten}}, \textcolor{blue}{Reimers}{}\ledrightnote{\textcolor{blue}{Georg Reimers}}{ }\textcolor{green}{Herzog}{}\ledrightnote{→\textcolor{green}{Der Schleier der Beatrice. Schauspiel in fünf Akten}}. Aber ich bin wieder
               ſchwankend geworden. – Über die \textcolor{green}{\textsc{Beatrice}}{}\ledrightnote{\textcolor{green}{Der Schleier der Beatrice. Schauspiel in fünf Akten}}{ }ſchreiben Sie mir nichts; vielleicht ſagen Sie mir
               noch einiges, we{\geminationn} Sie wieder zurück ſind. –\pend
           \pstart
           Leben Sie wohl. Von Herzen{\\[\baselineskip]}Ihr \spacefill\mbox{Arthur}\pend
           \leftskip=0em{}\endnumbering\briefempfaengerindex{Beer-Hofmann, Richard@\textsc{Beer-Hofmann, Richard}!zzzSchnitzler, Arthur@\emph{von Arthur Schnitzler}!1900-02-171@{17. 2. 1900}|)be}\mylabel{h}  \normalsize

\doendnotes{C}
\bigskip
\vfill

\clearpage

\footnotesize

\lohead{\textsc{register}}

% Definiere theindex-Environment komplett neu ohne reledmac
\makeatletter
\renewenvironment{theindex}{%
  \section*{\indexname}%
  \setlength{\parindent}{0pt}%
  \setlength{\parskip}{0pt plus 0.3pt}%
  \let\item\@idxitem
}{%
  \clearpage
}
\makeatother

\IfFileExists{\jobname-pw.ind}{\input{\jobname-pw.ind}}{}

\end{document}

      