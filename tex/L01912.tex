%% latex-korrekturansicht-vorspann.tex
%% Vorspann für die Korrekturansicht.
%% Lädt die gemeinsame Datei latex-vorspann.tex mit gesetztem Schalter.

\newif\ifkorrekturansicht
\korrekturansichttrue

\input{../tex-inputs/latex-vorspann}


               \section[Albert Ehrenstein an Arthur Schnitzler, 10. 2. 1910]{ Albert Ehrenstein an Arthur Schnitzler, 10. 2. 1910}\nopagebreak\mylabel{v}\rehead{ }\normalsize\beginnumbering\briefempfaengerindex{Schnitzler, Arthur@\textsc{Schnitzler, Arthur}!zzzEhrenstein, Albert@\emph{von Albert Ehrenstein}!1910-02-101@{10. 2. 1910}|(be} \toendnotes[C]{\smallbreak\pagebreak[2]} \Standort{CUL, Schnitzler, B 30.}
\physDesc{Brief, 1 Blatt, 2 Seiten
\newline{}Handschrift: schwarze Tinte, deutsche Kurrent
\newline{}Schnitzler: mit Bleistift beschriftet: »\textsc{Ehrenstein}« }\buchAbdrucke{\weitereDrucke{Albert Ehrenstein: \emph{Briefe}. Hg. Hanni Mittelmann. München: \emph{Boer} 1989, S. 37 (Werke, 1).} }\toendnotes[C]{\smallbreak}\pstart
           {\pb}\textcolor{pink}{XVI. \textsc{Ottakringerstr.} 114}{}\ledrightnote{\textcolor{pink}{Ottakringerstraße}}.\hfill 10{\\}II{\\}1910\pend
           \pstart{}Sehr geehrter Herr Doktor,\pend\pstart
           geſtern endlich erhielt ich Antwort von Herrn \textcolor{blue}{Bie}{}\ledrightnote{\textcolor{blue}{Oskar Bie}}, die ich beilege, da ich mich in deren Interpretation nicht ſicher
                    fühle. Ich weiß vor allem nicht, ob ich dem Schreiben entnehmen darf, »\textcolor{green}{Tubutſch}{}\ledrightnote{\textcolor{green}{Tubutsch}}« werde – was mir den Fang eines
                    Verlegers erleichtern würde – nach einer Umarbeitung \textcolor{green}{rundſchau}{}\ledrightnote{\textcolor{green}{Die neue Rundschau}}möglich ſein. Das wäre mir am liebſten, Denn
                    eſſayiſtiſch habe ich mich noch nicht recht verſucht, das \textcolor{pink}{Wien}{}\ledrightnote{\textcolor{pink}{Wien}}er Leben iſt mir unbekannt und was Herr \textcolor{blue}{Bie}{}\ledrightnote{\textcolor{blue}{Oskar Bie}} unter einem netten Thema verſteht (er
                    meint wohl ſo etwas wie die \label{K_L01912_1v}\edtext{\textcolor{blue}{Hofrichter}{}\ledrightnote{\textcolor{blue}{Adolf Hofrichter}}}{\lemma{\textnormal{\emph{Hofrichter}}}\Cendnote{\textnormal{\textcolor{blue}{Adolf Hofrichter} wurde im Frühjahr der
                        Prozess gemacht. Ihm wurde vorgeworfen, als Aphrodisiakum getarnte
                        Zyankalikapseln an höherrangige Militärs geschickt zu haben, um für seine
                        Beförderung Platz zu machen. Nachdem es bis zum Geständnis ein
                        Indizienverfahren war, fand der Prozess unter reger Anteilnahme der
                        Öffentlichkeit statt.}}}\label{K_L01912_1h}- oder \label{K_L01912_2v}\edtext{\textcolor{blue}{Borowska}{}\ledrightnote{\textcolor{blue}{Janina Borowska}}affaire}{\lemma{\textnormal{\emph{Borowskaaffaire}}}\Cendnote{\textnormal{\textcolor{blue}{Janina Borowska} wurde 1909
                        von dem Vorwurf freigesprochen, eine Spionin zu sein. Während des Prozesses
                        begannen sie und ihr Anwalt eine Affäre, die dieser nach einiger Zeit lösen
                        wollte. Am 5. 6. 1909 wurde er tot in seinem Bett gefunden,
                        neben ihm \textcolor{blue}{Borowska}. Im folgenden Prozess
                        gelang es nicht, den von ihr behaupteten Suizid zu wiederlegen und sie wurde
                        am 10. 10. 1910 in \textcolor{pink}{Krakau}
                        freigesprochen.}}}\label{K_L01912_2h}) hat auf mich bei meiner Gefühlsſtumpfheit kaum je
                    einen zu druckfähiger Meinungsäußerung {\pb}drängenden
                    Eindruck gemacht. Gern aber würde ich mich z. B. \textcolor{blue}{Schroeder}{}\ledrightnote{\textcolor{blue}{Rudolf Alexander Schröder}}’s \label{K_L01912_3v}\edtext{\textcolor{green}{\textcolor{blue}{Homer}{}\ledrightnote{\textcolor{blue}{Homer}}überſetzung}{}\ledrightnote{→\textcolor{green}{Odyssee}}}{\lemma{\textnormal{\emph{Homerüberſetzung}}}\Cendnote{\textnormal{\emph{\textcolor{green}{Die Odyssee}}. Neu ins Deutsche
                            übertragen von \textcolor{blue}{Rudolf Alexander
                                Schröder}. Gedruckt in 425 Exemplaren. Leipzig: \emph{\textcolor{brown}{Insel}}{ }1910.}}}\label{K_L01912_3h} befaſſen, wenn mir das Buch dieſes exkluſiven Autors
                    zugänglich wäre. Vielleicht können Sie, hochverehrter Herr Doktor, mir raten und
                    zugleich mir eine zweite Frage beantworten, die mich ſehr intereſſiert. Wann
                    nämlich der \textcolor{green}{junge Herr Medardus}{}\ledrightnote{\textcolor{green}{Der junge Medardus. Dramatische Historie in einem Vorspiel und fünf Aufzügen}} urſprünglich
                    im Buchhandel hätte erſcheinen ſollen, wenn er nicht (um die Zeit Ihrer \label{K_L01912_4v}\edtext{\textcolor{pink}{Volkstheater}{}\ledrightnote{\textcolor{pink}{Volkstheater}}premiere}{\lemma{\textnormal{\emph{Volkstheaterpremiere}}}\Cendnote{\textnormal{Verwechslung \textcolor{blue}{Ehrenstein}s, diese war immer für das \textcolor{pink}{Burgtheater} geplant und fand am 24. 11. 1910
                   statt.}}}\label{K_L01912_4h}?) zurückgezogen worden wäre?\pend
           \pstart
           Indem ich herzlichſt für Ihre Empfehlung danke, die, ſcheint es, diesmal doch zu
                    einem für das deutſche Schrifttum erfreulichen Resultate\strikeout{n} führen dürfte, bin ich mit den beſten
                    Grüßen\pend
           \pstart
           Hochachtungsvoll{\\[\baselineskip]}Ihr ergebenſter{\\[\baselineskip]}\spacefill\mbox{Albert Ehrenstein.}\pend
           \leftskip=0em{}\endnumbering\briefempfaengerindex{Schnitzler, Arthur@\textsc{Schnitzler, Arthur}!zzzEhrenstein, Albert@\emph{von Albert Ehrenstein}!1910-02-101@{10. 2. 1910}|)be}\mylabel{h}  \normalsize

\doendnotes{C}
\bigskip
\vfill

\clearpage

\footnotesize

\lohead{\textsc{register}}

% Definiere theindex-Environment komplett neu ohne reledmac
\makeatletter
\renewenvironment{theindex}{%
  \section*{\indexname}%
  \setlength{\parindent}{0pt}%
  \setlength{\parskip}{0pt plus 0.3pt}%
  \let\item\@idxitem
}{%
  \clearpage
}
\makeatother

\IfFileExists{\jobname-pw.ind}{\input{\jobname-pw.ind}}{}

\end{document}

      