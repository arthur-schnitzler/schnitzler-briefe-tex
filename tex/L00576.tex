%% latex-korrekturansicht-vorspann.tex
%% Vorspann für die Korrekturansicht.
%% Lädt die gemeinsame Datei latex-vorspann.tex mit gesetztem Schalter.

\newif\ifkorrekturansicht
\korrekturansichttrue

\input{../tex-inputs/latex-vorspann}


               \section[Felix Salten und Hugo von Hofmannsthal an Arthur Schnitzler und Richard Beer-Hofmann, 1. 8. 1896]{ Felix Salten und Hugo von Hofmannsthal an Arthur Schnitzler und
                    Richard Beer-Hofmann, 1. 8. 1896}\nopagebreak\mylabel{v}\rehead{ }\normalsize\beginnumbering\briefempfaengerindex{Beer-Hofmann, Richard@\textsc{Beer-Hofmann, Richard}!zzzSalten, Felix@\emph{von Felix Salten}!1896-08-011@{1. 8. 1896}|(be}\briefempfaengerindex{Beer-Hofmann, Richard@\textsc{Beer-Hofmann, Richard}!zzzHofmannsthal, Hugo von@\emph{von Hugo von Hofmannsthal}!1896-08-011@{1. 8. 1896}|(be}\briefempfaengerindex{Schnitzler, Arthur@\textsc{Schnitzler, Arthur}!zzzSalten, Felix@\emph{von Felix Salten}!1896-08-011@{1. 8. 1896}|(be}\briefempfaengerindex{Schnitzler, Arthur@\textsc{Schnitzler, Arthur}!zzzHofmannsthal, Hugo von@\emph{von Hugo von Hofmannsthal}!1896-08-011@{1. 8. 1896}|(be} \toendnotes[C]{\smallbreak\pagebreak[2]} \Standort{CUL, Schnitzler, B 89.}
\physDesc{Postkarte
\newline{}Handschrift Felix Salten: Bleistift, lateinische Kurrent\newline{}Handschrift Hugo von Hofmannsthal: Bleistift, lateinische Kurrent\newline{}Versand: 1) Stempel: »\nobreak{}\oindex{Bad Ischl@\textbf{Bad Ischl}, \emph{Besiedelter Ort (A.BSO)}|pwk}Ischl, 1 8 {[}96{]}, A\nobreak{}«.  2) Stempel: »\nobreak{}\oindex{Kopenhagen@\textbf{Kopenhagen}, \emph{Besiedelter Ort (A.BSO)}|pwk}Kjøbenhavn, 20MB\textcolor{gray}{3}–8\textcolor{gray}{8}6\nobreak{}«. \newline{}Ordnung: 1) mit Bleistift von unbekannter Hand die Jahreszahl
                                                »1896« bei der geschriebenen
                                            Datumsangabe ergänzt 2) mit Bleistift von unbekannter Hand nummeriert:
                                                »75«}\toendnotes[C]{\smallbreak}\pstart{}{\pb}Herrn D\textsuperscript{r} Arthur Schnitzler\pend{}\pstart{}\textcolor{pink}{Kopenhagen}{}\ledrightnote{\textcolor{pink}{Kopenhagen}}\pend{}\pstart{}\textcolor{pink}{Dänemark}{}\ledrightnote{\textcolor{pink}{Dänemark}}\pend{}\pstart{}poste restante\pend{}{\bigskip}\pstart
           \noindent{}{\pb}Für \uline{Arthur {\kaufmannsund}
                            Richard}\pend
           \pstart
           \raggedleft{}\textcolor{pink}{Ischl}{}\ledrightnote{\textcolor{pink}{Bad Ischl}}, 1. August\pend
           \pstart
           Wir haben uns zufällig getroffen, und da hat er mir (ich ihm) natürlich gleich
                    eine \textcolor{green}{Novelle}{}\ledrightnote{→\textcolor{green}{Geschichte der beiden Liebespaare}} vorgelesen. Sie
                    hat ihm (mir) recht gut (sehr gut! das »recht gut« ist nur meine ((seine))
                    Bescheidenheit) gefallen. Natürlich ist er (ich) \uline{sofort} wieder \label{K_L00576_1v}\edtext{abgereist}{\lemma{\textnormal{\emph{abgereist}}}\Cendnote{\textnormal{\textcolor{blue}{Hofmannsthal} urlaubte im gut 25 km
                        entfernten \textcolor{pink}{Aussee}.}}}\label{K_L00576_1h}. Das hat er (habe
                    ich) seit sechs Wochen vorher gewusst. Dies wünscht Euch\pend
           \pstart
           \spacefill\mbox{Salten}{\\[\baselineskip]}\spacefill\mbox{{[}hs. Hofmannsthal:{]} Hugo}\pend
           \leftskip=0em{}\endnumbering\briefempfaengerindex{Beer-Hofmann, Richard@\textsc{Beer-Hofmann, Richard}!zzzSalten, Felix@\emph{von Felix Salten}!1896-08-011@{1. 8. 1896}|)be}\briefempfaengerindex{Beer-Hofmann, Richard@\textsc{Beer-Hofmann, Richard}!zzzHofmannsthal, Hugo von@\emph{von Hugo von Hofmannsthal}!1896-08-011@{1. 8. 1896}|)be}\briefempfaengerindex{Schnitzler, Arthur@\textsc{Schnitzler, Arthur}!zzzSalten, Felix@\emph{von Felix Salten}!1896-08-011@{1. 8. 1896}|)be}\briefempfaengerindex{Schnitzler, Arthur@\textsc{Schnitzler, Arthur}!zzzHofmannsthal, Hugo von@\emph{von Hugo von Hofmannsthal}!1896-08-011@{1. 8. 1896}|)be}\mylabel{h}  \normalsize

\doendnotes{C}
\bigskip
\vfill

\clearpage

\footnotesize

\lohead{\textsc{register}}

% Definiere theindex-Environment komplett neu ohne reledmac
\makeatletter
\renewenvironment{theindex}{%
  \section*{\indexname}%
  \setlength{\parindent}{0pt}%
  \setlength{\parskip}{0pt plus 0.3pt}%
  \let\item\@idxitem
}{%
  \clearpage
}
\makeatother

\IfFileExists{\jobname-pw.ind}{\input{\jobname-pw.ind}}{}

\end{document}

      