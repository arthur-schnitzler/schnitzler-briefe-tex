%% latex-korrekturansicht-vorspann.tex
%% Vorspann für die Korrekturansicht.
%% Lädt die gemeinsame Datei latex-vorspann.tex mit gesetztem Schalter.

\newif\ifkorrekturansicht
\korrekturansichttrue

\input{../tex-inputs/latex-vorspann}


               \section[Arthur Schnitzler an Felix Braun, 28. 5. 1927]{ Arthur Schnitzler an Felix Braun, 28. 5. 1927}\nopagebreak\mylabel{v}\rehead{ }\normalsize\beginnumbering\briefempfaengerindex{Braun, Felix@\textsc{Braun, Felix}!zzzSchnitzler, Arthur@\emph{von Arthur Schnitzler}!1927-05-281@{28. 5. 1927}|(be} \toendnotes[C]{\smallbreak\pagebreak[2]} \Standort{Wienbibliothek im Rathaus, H.I.N.-198050.}
\physDesc{Brief, 1 Blatt, 1 Seite, Umschlag
\newline{}Handschrift: schwarze Tinte, lateinische Kurrent\newline{}Versand: 1) Stempel: »\nobreak{}\oindex{XVIII., Waehring@\textbf{XVIII., Währing}, \emph{Bezirk (A.BZK)}|pwk}18/1 Wien 110, \textcolor{gray}{3}0. V. 27, \textcolor{gray}{8}\nobreak{}«.  2) mit blauem Buntstift der Bezirk
                                        »XIX« nochmals auf das
                                    Kuvert geschrieben, womöglich wegen der falschen Hausnummer in
                                    Schnitzlers Adressierung}\toendnotes[C]{\smallbreak}\pstart{}{\pb}\label{T_L02487-1v}\edtext{\textcolor{gray}{\textbf{A. S.}}}{\lemma{\textnormal{\emph{A. S.}}}\Cendnote{\textnormal{ovaler Absenderkleber}}}\label{T_L02487-1h}\pend{}\pstart{}\textcolor{pink}{\textcolor{gray}{\textbf{WIEN, XVIII.}}}{}\ledrightnote{\textcolor{pink}{XVIII., Währing}}\pend{}\pstart{}\textcolor{pink}{\textcolor{gray}{\textbf{STERNWARTESTR. 71}}}{}\ledrightnote{\textcolor{pink}{Sternwartestraße}}\pend{}{\bigskip}\pstart{}Herrn Felix Braun\pend{}\pstart{}Schriftsteller\pend{}\pstart{}\textcolor{pink}{Wien XIX}{}\ledrightnote{\textcolor{pink}{XIX., Döbling}}\pend{}\pstart{}\textcolor{pink}{Sievringerstraße 99}{}\ledrightnote{\textcolor{pink}{Sieveringer Straße}}.\pend{}{\bigskip}\pstart
           \raggedleft{}{\pb}\textcolor{pink}{Wien}{}\ledrightnote{\textcolor{pink}{Wien}}. 28. 5. 927\pend
           \pstart
           lieber und verehrter Herr Braun,  Sie wissen wohl schon wie
                    sehr mich Ihr Brief gefreut hat; Herr \textcolor{blue}{von
                        Guenther}{}\ledrightnote{\textcolor{blue}{Johannes von Guenther}} hats Ihnen erzählt, – ich will doch nicht versäumen es
                    schriftlich zu wiederholen. Ihre Bedenken gegenüber dem Schluſs versteh ich wohl
                    – nach einem halben Dutzend ganz mislungener hat sich dieser endlich gemeldete
                    als der beste herausgestellt. Freilich ermangelt es allzusehr der Bedeutung,
                    aber jeder andre (der mir einfiel) hatte praetentiös gewirkt.\pend
           \pstart
           Schönen Dank auch für den \label{K_L02487_1v}\edtext{\textcolor{green}{Heraklesroman}{}\ledrightnote{→\textcolor{green}{Die Taten des Herakles}}}{\lemma{\textnormal{\emph{Heraklesroman}}}\Cendnote{\textnormal{\textcolor{blue}{Felix Braun}: \emph{\textcolor{green}{Die Taten des Herakles}}. Roman. 4.–6., neu
                            durchgesehene Auflage. Leipzig, Wien: \emph{F. G.
                                Speidel}{ }1927.}}}\label{K_L02487_1h} – ich freu mich sehr, ihn in der nächsten Zeit, vermutlich
                    auf einer Reise, zu lesen. Erhalten Sie mir lieber Felix Braun Ihre Sympathie –
                        {\pb}sie ist mir ein werthvoller
                    Gewinn und ich erwidere sie aufs Freundschaftlichste.\pend
           \pstart
           Herzlich grüße ich Sie als Ihr ergebner{\\[\baselineskip]}\spacefill\mbox{ArthurSchnitzler}\pend
           \leftskip=0em{}\endnumbering\briefempfaengerindex{Braun, Felix@\textsc{Braun, Felix}!zzzSchnitzler, Arthur@\emph{von Arthur Schnitzler}!1927-05-281@{28. 5. 1927}|)be}\mylabel{h}  \normalsize

\doendnotes{C}
\bigskip
\vfill

\clearpage

\footnotesize

\lohead{\textsc{register}}

% Definiere theindex-Environment komplett neu ohne reledmac
\makeatletter
\renewenvironment{theindex}{%
  \section*{\indexname}%
  \setlength{\parindent}{0pt}%
  \setlength{\parskip}{0pt plus 0.3pt}%
  \let\item\@idxitem
}{%
  \clearpage
}
\makeatother

\IfFileExists{\jobname-pw.ind}{\input{\jobname-pw.ind}}{}

\end{document}

      