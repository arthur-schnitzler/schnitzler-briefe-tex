%% latex-korrekturansicht-vorspann.tex
%% Vorspann für die Korrekturansicht.
%% Lädt die gemeinsame Datei latex-vorspann.tex mit gesetztem Schalter.

\newif\ifkorrekturansicht
\korrekturansichttrue

\input{../tex-inputs/latex-vorspann}


               \section[Arthur Schnitzler an Robert Adam, 14. 6. 1929]{ Arthur Schnitzler an Robert Adam, 14. 6. 1929}\nopagebreak\mylabel{v}\rehead{ }\normalsize\beginnumbering\briefempfaengerindex{Adam, Robert@\textsc{Adam, Robert}!zzzSchnitzler, Arthur@\emph{von Arthur Schnitzler}!1929-06-141@{14. 6. 1929}|(be} \toendnotes[C]{\smallbreak\pagebreak[2]} \Standort{DLA, 96.34.2/34.}
\physDesc{Brief, 1 Blatt (Briefpapier mit Trauerrand), 1 Seite, Umschlag mit Trauerrand
\newline{}Handschrift: schwarze Tinte, lateinische Kurrent\newline{}Versand: Stempel: »\nobreak{}\oindex{XVIII., Waehring@\textbf{XVIII., Währing}, \emph{Bezirk (A.BZK)}|pwk}18/\textsubscript{1}Wien
                                        110, 15. \textcolor{gray}{XI}. 29, 7\nobreak{}«.  }\toendnotes[C]{\smallbreak}\pstart{}{\pb}\label{T_L02512-1v}\edtext{\textcolor{gray}{\textbf{A. S.}}}{\lemma{\textnormal{\emph{A. S.}}}\Cendnote{\textnormal{ovaler Absenderkleber}}}\label{T_L02512-1h}\pend{}\pstart{}\textcolor{pink}{\textcolor{gray}{\textbf{WIEN, XVIII.}}}{}\ledrightnote{\textcolor{pink}{XVIII., Währing}}\pend{}\pstart{}\textcolor{pink}{\textcolor{gray}{\textbf{STERNWARTESTR. 71}}}{}\ledrightnote{\textcolor{pink}{Sternwartestraße}}\pend{}{\bigskip}\pstart{}{\pb}Herrn Ob.Landesger-Rath\pend{}\pstart{}Dr. Rob. Adam Pollak\pend{}\pstart{}\textcolor{pink}{Wien XII}{}\ledrightnote{\textcolor{pink}{XII., Meidling}}\pend{}\pstart{}\textcolor{pink}{Meidlinger Hauptstr 58}{}\ledrightnote{\textcolor{pink}{Meidlinger Hauptstraße}}.\pend{}{\bigskip}\pstart
           \raggedleft{}{\pb}\textcolor{pink}{Wien}{}\ledrightnote{\textcolor{pink}{Wien}}, 14/6 929\pend
           \pstart{}Verehrter Herr Oberlandesgerichtsrath,\pend\pstart
           ich fahre dieser Tage auf den \textcolor{pink}{Semmering}{}\ledrightnote{\textcolor{pink}{Semmering}}; nach
                    meiner Rückkehr Anfang Juli wird es mir ein besondres Vergnügen
                    sein, Sie nach so langer Zeit wieder einmal bei mir zu sehen. Ob eine Bühne sich
                    entschließen wird, Ihre \textcolor{green}{Margot}{}\ledrightnote{\textcolor{green}{Margot und das Jugendgericht}} zur Aufführung
                    zu bringen, läßt sich schwer voraussagen; die \textcolor{gray}{Galerie}, so
                    lustig sie ist – und selbst angeno{\geminationm}en, es stecke
                    mehr bittre Wahrheit drin als heitre Erfindung, scheint mir stellenweise in
                    künstlerischem Sinne so grob, als daſs ein Theaterpublikum die rechte Freude
                    daran haben sollte.\pend
           \pstart Aber unfehlbar bin ich nicht. Also auf bald, und herzliche Grüße\hspace*{1.5em}Ihr sehr ergebner\spacefill\mbox{ArthSchnitzler}\pend{}\endnumbering\briefempfaengerindex{Adam, Robert@\textsc{Adam, Robert}!zzzSchnitzler, Arthur@\emph{von Arthur Schnitzler}!1929-06-141@{14. 6. 1929}|)be}\mylabel{h}  \normalsize

\doendnotes{C}
\bigskip
\vfill

\clearpage

\footnotesize

\lohead{\textsc{register}}

% Definiere theindex-Environment komplett neu ohne reledmac
\makeatletter
\renewenvironment{theindex}{%
  \section*{\indexname}%
  \setlength{\parindent}{0pt}%
  \setlength{\parskip}{0pt plus 0.3pt}%
  \let\item\@idxitem
}{%
  \clearpage
}
\makeatother

\IfFileExists{\jobname-pw.ind}{\input{\jobname-pw.ind}}{}

\end{document}

      