%% latex-korrekturansicht-vorspann.tex
%% Vorspann für die Korrekturansicht.
%% Lädt die gemeinsame Datei latex-vorspann.tex mit gesetztem Schalter.

\newif\ifkorrekturansicht
\korrekturansichttrue

\input{../tex-inputs/latex-vorspann}


               \section[Olga und Arthur Schnitzler an Richard und Paula Beer-Hofmann, 18. 5. 1910]{ Olga und Arthur Schnitzler an Richard und Paula Beer-Hofmann,
               18. 5. 1910}\nopagebreak\mylabel{v}\rehead{ }\normalsize\beginnumbering\briefempfaengerindex{Beer-Hofmann, Paula@\textsc{Beer-Hofmann, Paula}!zzzSchnitzler, Arthur@\emph{von Arthur Schnitzler}!1910-05-181@{18. 5. 1910}|(be}\briefempfaengerindex{Beer-Hofmann, Paula@\textsc{Beer-Hofmann, Paula}!zzzSchnitzler, Olga@\emph{von Olga Schnitzler}!1910-05-181@{18. 5. 1910}|(be}\briefempfaengerindex{Beer-Hofmann, Richard@\textsc{Beer-Hofmann, Richard}!zzzSchnitzler, Arthur@\emph{von Arthur Schnitzler}!1910-05-181@{18. 5. 1910}|(be}\briefempfaengerindex{Beer-Hofmann, Richard@\textsc{Beer-Hofmann, Richard}!zzzSchnitzler, Olga@\emph{von Olga Schnitzler}!1910-05-181@{18. 5. 1910}|(be} \toendnotes[C]{\smallbreak\pagebreak[2]} \Standort{YCGL, MSS 31.}
\physDesc{Bildpostkarte
\newline{}Handschrift Arthur Schnitzler: Bleistift\newline{}Handschrift Olga Schnitzler: Bleistift, lateinische Kurrent\newline{}Versand: 1) Stempel: »\nobreak{}\oindex{Zuerich@\textbf{Zürich}, \emph{Besiedelter Ort (A.BSO)}|pwk}Zürich, 1\textcolor{gray}{9}. V. 10, X\nobreak{}«.  2) Stempel: »\nobreak{}\oindex{Grand Hotel Bellevue au lac@\textbf{Grand Hotel Bellevue au lac}, \emph{Hotel (K.HTL)}|pwk}Bellevue au lac Zürich, 19 5. 1920\nobreak{}«. \newline{}Ordnung: mit Bleistift von unbekannter Hand datiert: »18. 5.« }\toendnotes[C]{\smallbreak}\pstart{}{\pb}Herrn u. Frau\pend{}\pstart{}D\textsuperscript{r} Richard Beer-Hofmann\pend{}\pstart{}\textcolor{pink}{Wien XIX}{}\ledrightnote{\textcolor{pink}{XIX., Döbling}}\pend{}\pstart{}\textcolor{pink}{Hasenauerstrasse 59}{}\ledrightnote{\textcolor{pink}{Hasenauerstraße}}.\pend{}{\bigskip}\pstart
           \noindent{}\centering{}{\pb}\textcolor{gray}{\textbf{\textcolor{pink}{Zürich}{}\ledrightnote{\textcolor{pink}{Zürich}}: Blick vom See}}\pend
           \pstart
           {\pb}Heut war die Welt, vom Speisewagen aus gesehen, so
               unglaublich schön, dass wir Euch sehr herbeigewünscht haben – leider vergeblich und
               es bleibt nichts übrig, als Euch und den \textcolor{blue}{Kindern}{}\ledrightnote{→\textcolor{blue}{Naëmah Beer-Hofmann}{\newline}→\textcolor{blue}{Mirjam Beer-Hofmann}{\newline}→\textcolor{blue}{Gabriel Beer-Hofmann}} viele herzliche Grüsse zu schicken.\pend
           \pstart \spacefill\mbox{Olga}\pend{}\pstart
           \raggedleft{}1\substVorne{}\textsuperscript{6}\substDazwischen{}8\substHinten{}. Mai 10.\pend
           \pstart
           {[}hs. Schnitzler:{]} \spacefill\mbox{Arthur.}\pend
           \endnumbering\briefempfaengerindex{Beer-Hofmann, Paula@\textsc{Beer-Hofmann, Paula}!zzzSchnitzler, Arthur@\emph{von Arthur Schnitzler}!1910-05-181@{18. 5. 1910}|)be}\briefempfaengerindex{Beer-Hofmann, Paula@\textsc{Beer-Hofmann, Paula}!zzzSchnitzler, Olga@\emph{von Olga Schnitzler}!1910-05-181@{18. 5. 1910}|)be}\briefempfaengerindex{Beer-Hofmann, Richard@\textsc{Beer-Hofmann, Richard}!zzzSchnitzler, Arthur@\emph{von Arthur Schnitzler}!1910-05-181@{18. 5. 1910}|)be}\briefempfaengerindex{Beer-Hofmann, Richard@\textsc{Beer-Hofmann, Richard}!zzzSchnitzler, Olga@\emph{von Olga Schnitzler}!1910-05-181@{18. 5. 1910}|)be}\mylabel{h}  \normalsize

\doendnotes{C}
\bigskip
\vfill

\clearpage

\footnotesize

\lohead{\textsc{register}}

% Definiere theindex-Environment komplett neu ohne reledmac
\makeatletter
\renewenvironment{theindex}{%
  \section*{\indexname}%
  \setlength{\parindent}{0pt}%
  \setlength{\parskip}{0pt plus 0.3pt}%
  \let\item\@idxitem
}{%
  \clearpage
}
\makeatother

\IfFileExists{\jobname-pw.ind}{\input{\jobname-pw.ind}}{}

\end{document}

      