%% latex-korrekturansicht-vorspann.tex
%% Vorspann für die Korrekturansicht.
%% Lädt die gemeinsame Datei latex-vorspann.tex mit gesetztem Schalter.

\newif\ifkorrekturansicht
\korrekturansichttrue

\input{../tex-inputs/latex-vorspann}


               \section[Robert Adam an Arthur Schnitzler, 1. 5. 1927]{ Robert Adam an Arthur Schnitzler, 1. 5. 1927}\nopagebreak\mylabel{v}\rehead{ }\normalsize\beginnumbering\briefempfaengerindex{Schnitzler, Arthur@\textsc{Schnitzler, Arthur}!zzzAdam, Robert@\emph{von Robert Adam}!1927-05-011@{1. 5. 1927}|(be} \toendnotes[C]{\smallbreak\pagebreak[2]} \Standort{CUL, Schnitzler, B 1.}
\physDesc{Brief, 1 Blatt, 4 Seiten
\newline{}Handschrift: schwarze Tinte, deutsche Kurrent
\newline{}Schnitzler: 1) mit Bleistift beschriftet: »\textsc{Adam}« 2) mit rotem Buntstift vereinzelte Unterstreichungen\newline{}Ordnung: mit Bleistift von unbekannter Hand nummeriert:
                                        »18« }\Standort{Wien, Österreichische Nationalbibliothek, Cod.ser. 52.268, 335 und 330.}
\physDesc{handschriftliche Abschrift
\newline{}Handschrift: schwarze Tinte, Gabelsberger Kurzschrift}\Standort{Wien, Österreichische Nationalbibliothek, Cod.ser. 52.268, 335 und 330.}
\physDesc{maschinelle Abschrift
\newline{}Schreibmaschine}\toendnotes[C]{\smallbreak}\pstart
           \raggedleft{}{\pb}\textcolor{pink}{Wien}{}\ledrightnote{\textcolor{pink}{Wien}}, am 1. Mai
                        1927.\pend
           \pstart\center{}Hochverehrter Herr Doktor!\pend\pstart
           Ich darf Ihnen neuerlich für eine Gabe danken, für Ihr »\textcolor{green}{Spiel im Morgengrauen}{}\ledrightnote{\textcolor{green}{Spiel im Morgengrauen. Novelle}}«, das mir durch den \textcolor{brown}{Verleger}{}\ledrightnote{→\textcolor{brown}{Ullstein Verlag}} zugegangen iſt. Welche Luſt
                    künſtleriſchen Genießens es mir bereitete, kann ich nicht ausdrücken. Es kam mir
                    vor, als hätten Sie ſich aus unſerer ſtaubigen Verfallszeit in ein altes \textcolor{pink}{Wien}{}\ledrightnote{\textcolor{pink}{Wien}}er \textcolor{pink}{Griechenland}{}\ledrightnote{\textcolor{pink}{Griechenland}} geflüchtet, in dem auch über den düſterſten Ereigniſſen,
                    über dem Kampf und Vergehen \strikeout{klei}{ }{\pb}der kleinen Menſchen ein ewigblauer
                    Himmel bei kühlen Frühlingslüften lacht, in ein Land, das wir alle gekannt
                    haben, ein Orplid ohne die Nebelhaftigkeit romantiſchen Ahnens, in dem vielmehr
                    alle Konturen und alle Geſtalten klar umriſſen und hell beſchienen ſind. Solange
                    Sie dies \textcolor{pink}{Wien}{}\ledrightnote{\textcolor{pink}{Wien}}er \textcolor{pink}{Alt-Hellas}{}\ledrightnote{\textcolor{pink}{Griechenland}} mit Ihren Geſtalten, Gefühlen und Gedanken beleben, iſt es
                    nicht untergegangen und wir dürfen uns hineinflüchten wie in die Erinnerung
                    froher Jugendtage. Wie harmoniſch iſt dort alles, wie harmoniſch ſelbſt die
                    Disharmonie! Und wie froh macht mich die Klarheit Ihrer {\pb}Sprache, voll und funkelnd wie
                    reifer Wein! Sie wirkte auf mich doppelt mächtig, da ich vom Ärger über den
                    neologiſchen Nachwuchs herkam, der ſich entrüſtet gegen die Zumutung wehrt, die
                    »Sprache des 19. Jahrhunderts« zu ſprechen, und infolgedeſſen kühnlich die des
                    21. vorwegnimmt, feierlich um den Gral der Unverſtändlichkeit bemüht, die
                    Ritterſchar von Wortſalvat. –\pend
           \pstart
           Ihrer freundlichen Einladung, Sie einmal aufzuſuchen, werde ich natürlich mit
                    größter Freude nachkommen. Vielleicht könnten Sie mir den Ihnen genehmen Tag
                    durch Dr \textcolor{blue}{Karl Pollak}{}\ledrightnote{\textcolor{blue}{Karl Pollak}} im kurzen Wege
                    mitteilen laſſen.\pend
           \pstart
           {\pb}Unter Wiederholung meines Dankes mit
                    den beſten Empfehlungen\pend
           \pstart
           Ihr ergebener{\\[\baselineskip]}\spacefill\mbox{D\textsuperscript{r}RAdam.}\pend
           \leftskip=0em{}\endnumbering\briefempfaengerindex{Schnitzler, Arthur@\textsc{Schnitzler, Arthur}!zzzAdam, Robert@\emph{von Robert Adam}!1927-05-011@{1. 5. 1927}|)be}\mylabel{h}  \normalsize

\doendnotes{C}
\bigskip
\vfill

\clearpage

\footnotesize

\lohead{\textsc{register}}

% Definiere theindex-Environment komplett neu ohne reledmac
\makeatletter
\renewenvironment{theindex}{%
  \section*{\indexname}%
  \setlength{\parindent}{0pt}%
  \setlength{\parskip}{0pt plus 0.3pt}%
  \let\item\@idxitem
}{%
  \clearpage
}
\makeatother

\IfFileExists{\jobname-pw.ind}{\input{\jobname-pw.ind}}{}

\end{document}

      