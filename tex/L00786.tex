%% latex-korrekturansicht-vorspann.tex
%% Vorspann für die Korrekturansicht.
%% Lädt die gemeinsame Datei latex-vorspann.tex mit gesetztem Schalter.

\newif\ifkorrekturansicht
\korrekturansichttrue

\input{../tex-inputs/latex-vorspann}


               \section[Hugo von Hofmannsthal an Arthur Schnitzler, {[}23. 3. 1898{]}]{ Hugo von Hofmannsthal an Arthur Schnitzler, {[}23. 3. 1898{]}}\nopagebreak\mylabel{v}\rehead{ }\normalsize\beginnumbering\briefempfaengerindex{Schnitzler, Arthur@\textsc{Schnitzler, Arthur}!zzzHofmannsthal, Hugo von@\emph{von Hugo von Hofmannsthal}!1898-03-231@{{[}23. 3. 1898{]}}|(be} \toendnotes[C]{\smallbreak\pagebreak[2]} \Standort{CUL, Schnitzler, B 43b/1.}
\physDesc{Brief, 1 Blatt, 2 Seiten
\newline{}Handschrift: schwarze Tinte, deutsche Kurrent
\newline{}Schnitzler: mit Bleistift datiert: »c 20 März 98« \newline{}Ordnung: 1) mit Bleistift von unbekannter Hand nummeriert: »\strikeout{107}« 2) mit Bleistift von unbekannter Hand nummeriert: »108«}\buchAbdrucke{\weitereDrucke{Hugo von Hofmannsthal, Arthur Schnitzler: \emph{Briefwechsel}. Hg. Therese Nickl und Heinrich Schnitzler. Frankfurt am Main: \emph{S. Fischer} 1964, S. 99.} }\toendnotes[C]{\smallbreak}\pstart{}{\pb}lieber Arthur\pend\pstart
           alſo \label{K_L00786_1v}\edtext{morgen}{\lemma{\textnormal{\emph{morgen}}}\Cendnote{\textnormal{Am 24. 3. 1898 war \textcolor{blue}{Schnitzler} in der Uraufführung von \emph{\textcolor{green}{Neigung}} von \textcolor{blue}{J. J.
                            David} im \textcolor{pink}{Burgtheater}.}}}\label{K_L00786_1h} nach
                    der \textcolor{green}{Neigung}{}\ledrightnote{\textcolor{green}{Neigung}} im \textcolor{pink}{\textsc{Pucher}}{}\ledrightnote{\textcolor{pink}{Café Pucher}}.\pend
           \pstart
           \textcolor{blue}{\textsc{Clemens Franckenstein}}{}\ledrightnote{\textcolor{blue}{Clemens von Franckenstein}}{ }\textcolor{pink}{\textsc{I. Am Hof} 13.}{}\ledrightnote{\textcolor{pink}{Am Hof}} Ich möcht erſt dann aufs Land
                    fahren, wenn ein biſſel grün und ein biſſel wirkliche Frühlingsluft iſt, ich
                    find wenn {\pb}man es anders
                    thut, hat man dann Ungeduld und Ärger. Mit unſern Landpartien wars immer ſo.\pend
           \pstart
           Herzlich Ihr{\\[\baselineskip]}\spacefill\mbox{Hugo.}\pend
           \leftskip=0em{}\endnumbering\briefempfaengerindex{Schnitzler, Arthur@\textsc{Schnitzler, Arthur}!zzzHofmannsthal, Hugo von@\emph{von Hugo von Hofmannsthal}!1898-03-231@{{[}23. 3. 1898{]}}|)be}\mylabel{h}  \normalsize

\doendnotes{C}
\bigskip
\vfill

\clearpage

\footnotesize

\lohead{\textsc{register}}

% Definiere theindex-Environment komplett neu ohne reledmac
\makeatletter
\renewenvironment{theindex}{%
  \section*{\indexname}%
  \setlength{\parindent}{0pt}%
  \setlength{\parskip}{0pt plus 0.3pt}%
  \let\item\@idxitem
}{%
  \clearpage
}
\makeatother

\IfFileExists{\jobname-pw.ind}{\input{\jobname-pw.ind}}{}

\end{document}

      