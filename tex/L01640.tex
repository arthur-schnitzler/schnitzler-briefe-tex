%% latex-korrekturansicht-vorspann.tex
%% Vorspann für die Korrekturansicht.
%% Lädt die gemeinsame Datei latex-vorspann.tex mit gesetztem Schalter.

\newif\ifkorrekturansicht
\korrekturansichttrue

\input{../tex-inputs/latex-vorspann}


               \section[Albert Ehrenstein an Arthur Schnitzler, 30. 11. 1906]{ Albert Ehrenstein an Arthur Schnitzler, 30. 11. 1906}\nopagebreak\mylabel{v}\rehead{ }\normalsize\beginnumbering\briefempfaengerindex{Schnitzler, Arthur@\textsc{Schnitzler, Arthur}!zzzEhrenstein, Albert@\emph{von Albert Ehrenstein}!1906-11-301@{30. 11. 1906}|(be} \toendnotes[C]{\smallbreak\pagebreak[2]} \Standort{CUL, Schnitzler, B 30.}
\physDesc{Brief, 1 Blatt, 2 Seiten
\newline{}Handschrift: schwarze Tinte, deutsche Kurrent
\newline{}Schnitzler: Beschriftung »Ehrenste\textcolor{gray}{in}« }\buchAbdrucke{\weitereDrucke{Albert Ehrenstein: \emph{Briefe}. Hg. Hanni Mittelmann. München: \emph{Boer} 1989, S. 20 (Werke, 1).} }\toendnotes[C]{\smallbreak}\pstart
           \raggedleft{}{\pb}\textcolor{pink}{Wien}{}\ledrightnote{\textcolor{pink}{Wien}}, den 30. Nov. 06\pend
           \pstart{}Sehr geehrter Herr Doktor.\pend\pstart
           Ihre außerordentliche Geduld, ſehr geehrter Herr Doktor, hoffe ich nicht auf eine
                    allzuharte Probe geſtellt zu haben, wenn ich höflichſt bitte, meine etwas
                    dilettantiſche Übertragung des \textcolor{blue}{euripideiſchen}{}\ledrightnote{\textcolor{blue}{Euripides}}{ }\label{K_L01640_1v}\edtext{\textcolor{green}{Librettos}{}\ledrightnote{→\textcolor{green}{Helena}}}{\lemma{\textnormal{\emph{Librettos}}}\Cendnote{\textnormal{Die Bearbeitung von \emph{\textcolor{green}{Helena}} ist nicht erhalten.}}}\label{K_L01640_1h} einiger Lektüre zu
                    unterziehen. Sollte dies aber doch der Fall ſein, ſo möchte ich Euer
                    Hochwohlgeboren ergebenſt erſuchen, beachten zu wollen, daß ich nicht daran
                    denke, die Arbeit etwa in dieſer Form {\pb}irgendwie bekannt zu machen, ſondern falls ſich überhaupt das Sujet zu einer
                    Veröffentlichung eignen ſollte, würde ich von den 2000 Verſen des \textcolor{blue}{Euripides}{}\ledrightnote{\textcolor{blue}{Euripides}} und meiner Überſetzung etwa 1000
                    weglaſſen, die vier Akte in zwei oder einen zuſammenziehen, was mir bei der
                    Fülle entbehrlicher Chorlieder, bei dem Überfluſſe an Wiederholungen und
                    unnützen Längen des Dialoges nicht ſchwer fiele. Indem ich Sie, ſehr verehrter
                    Herr Doktor, bitte, mir dieſe Arbeit nicht übelzunehmen, verbleibe ich
                    hochachtungsvoll\pend
           \pstart
           Ihr Sie verehrender{\\[\baselineskip]}\spacefill\mbox{Albert Ehrenstein.}\pend
           \leftskip=0em{}\endnumbering\briefempfaengerindex{Schnitzler, Arthur@\textsc{Schnitzler, Arthur}!zzzEhrenstein, Albert@\emph{von Albert Ehrenstein}!1906-11-301@{30. 11. 1906}|)be}\mylabel{h}  \normalsize

\doendnotes{C}
\bigskip
\vfill

\clearpage

\footnotesize

\lohead{\textsc{register}}

% Definiere theindex-Environment komplett neu ohne reledmac
\makeatletter
\renewenvironment{theindex}{%
  \section*{\indexname}%
  \setlength{\parindent}{0pt}%
  \setlength{\parskip}{0pt plus 0.3pt}%
  \let\item\@idxitem
}{%
  \clearpage
}
\makeatother

\IfFileExists{\jobname-pw.ind}{\input{\jobname-pw.ind}}{}

\end{document}

      