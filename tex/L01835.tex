%% latex-korrekturansicht-vorspann.tex
%% Vorspann für die Korrekturansicht.
%% Lädt die gemeinsame Datei latex-vorspann.tex mit gesetztem Schalter.

\newif\ifkorrekturansicht
\korrekturansichttrue

\input{../tex-inputs/latex-vorspann}


               \section[Albert Ehrenstein an Arthur Schnitzler, 27. 3. 1909]{ Albert Ehrenstein an Arthur Schnitzler, 27. 3. 1909}\nopagebreak\mylabel{v}\rehead{ }\normalsize\beginnumbering\briefempfaengerindex{Schnitzler, Arthur@\textsc{Schnitzler, Arthur}!zzzEhrenstein, Albert@\emph{von Albert Ehrenstein}!1909-03-271@{27. 3. 1909}|(be} \toendnotes[C]{\smallbreak\pagebreak[2]} \Standort{CUL, Schnitzler, B 30.}
\physDesc{Brief, 1 Blatt, 3 Seiten
\newline{}Handschrift: schwarze Tinte, deutsche Kurrent
\newline{}Schnitzler: mit Bleistift beschriftet: »\textsc{Ehrenst\textcolor{gray}{ein}}« }\buchAbdrucke{\weitereDrucke{Albert Ehrenstein: \emph{Briefe}. Hg. Hanni Mittelmann. München: \emph{Boer} 1989, S. 27 (Werke, 1).} }\toendnotes[C]{\smallbreak}\pstart
           {\pb}\textcolor{pink}{XVI \textsc{Ottakringerstr}
                        114}{}\ledrightnote{\textcolor{pink}{Ottakringerstraße}}.\hfill 27 III. 09.\pend
           \pstart{}Sehr geehrter Herr Doktor,\pend\pstart
           gerne möchte ich pflichtſchuldigſt einen ausführlichen Bericht erſtatten über meine
               »Besuche« bei den Herren \textcolor{blue}{Geld-
                  und Schreibheimers}{}\ledrightnote{→\textcolor{blue}{Raoul Auernheimer}{\newline}→\textcolor{blue}{Felix von Oppenheimer}}. Es liegen bei mir aus verſchiedenen Jahren Briefe an Sie,
               ſehr geehrter Herr Doktor, die ich nicht abſchickte, fröhlich-ergebene und
               verärgerte, Geſchäftsbriefe und ſolche vornehmeren Charakters. Auch diesmal verfaßte
               ich eine Menge mehr, minder gewundener Schreiben. Sie gerieten aber wie jene anderen
               im Format zu groß, und (ich ſage es \label{K_L01835_1v}\edtext{\textsc{pro privata Augustissimi notitia}}{\lemma{\textnormal{\emph{pro … notitia}}}\Cendnote{\textnormal{lateinisch: zur persönlichen Kenntnisnahme
                  des Herrschers}}}\label{K_L01835_1h}) {\pb}inhaltlich
               bargen ſie Dinge, die weder für die genannten Herren noch für mich beſonders
               schmeichelhaft waren. Wenn eine getreue Schilderung des mir Widerfahrenen für Sie,
               ſehr geehrter Herr Doktor, Intereſſe haben ſollte, würden Sie mich aufs Neue
               verbinden, indem Sie mir geſtatten, Ihnen einmal mündlich über meine Erfahrungen im
               Lande der Ariſtokratoiden und Zeitungsleute Rede zu ſtehen. Starke pſychiſche
               Depreſſionen, hervorgerufen durch das empfangsfeindliche Benehmen der Herren \textcolor{blue}{Gloſſy}{}\ledrightnote{\textcolor{blue}{Karl Glossy}}, \textcolor{blue}{Auern-}{}\ledrightnote{\textcolor{blue}{Raoul Auernheimer}}
               und \textcolor{blue}{Oppenheimer}{}\ledrightnote{\textcolor{blue}{Felix von Oppenheimer}}, und {\pb}nicht zumindeſt durch meine altbewährten
               Ungeſchicklichkeiten, die leider auch auf Sie, ſehr geehrter Herr Doktor, Bezug
               haben, Bitterkeit und Rachſucht, wie Demut und übertriebene Sucht gerecht zu ſein,
               machen die Abfaſſung eines vernünftigen Briefes zur Unmöglichkeit Ihrem Ihnen, ſehr
               geehrter Herr Doktor, nun auch noch für recht merkwürdige tragikomiſche Erlebniſſe
               dankbaren, ergebenſten\pend
           \pstart \spacefill\mbox{Albert Ehrenstein.}\pend{}\endnumbering\briefempfaengerindex{Schnitzler, Arthur@\textsc{Schnitzler, Arthur}!zzzEhrenstein, Albert@\emph{von Albert Ehrenstein}!1909-03-271@{27. 3. 1909}|)be}\mylabel{h}  \normalsize

\doendnotes{C}
\bigskip
\vfill

\clearpage

\footnotesize

\lohead{\textsc{register}}

% Definiere theindex-Environment komplett neu ohne reledmac
\makeatletter
\renewenvironment{theindex}{%
  \section*{\indexname}%
  \setlength{\parindent}{0pt}%
  \setlength{\parskip}{0pt plus 0.3pt}%
  \let\item\@idxitem
}{%
  \clearpage
}
\makeatother

\IfFileExists{\jobname-pw.ind}{\input{\jobname-pw.ind}}{}

\end{document}

      