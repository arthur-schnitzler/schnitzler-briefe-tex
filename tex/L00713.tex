%% latex-korrekturansicht-vorspann.tex
%% Vorspann für die Korrekturansicht.
%% Lädt die gemeinsame Datei latex-vorspann.tex mit gesetztem Schalter.

\newif\ifkorrekturansicht
\korrekturansichttrue

\input{../tex-inputs/latex-vorspann}


               \section[Arthur Schnitzler an Richard Beer-Hofmann, 5. 8. 1897]{ Arthur Schnitzler an Richard Beer-Hofmann, 5. 8. 1897}\nopagebreak\mylabel{v}\rehead{ }\normalsize\beginnumbering\briefempfaengerindex{Beer-Hofmann, Richard@\textsc{Beer-Hofmann, Richard}!zzzSchnitzler, Arthur@\emph{von Arthur Schnitzler}!1897-08-051@{5. 8. 1897}|(be} \toendnotes[C]{\smallbreak\pagebreak[2]} \Standort{YCGL, MSS 31.}
\physDesc{Telegramm
\newline{}maschinell\newline{}Versand: Stempel des Telegrafenbeamten, der Telegrafenbeamtin: »W. 106 \textcolor{gray}{Graser}« und wohl von derselben Schreibkraft mit Bleistift: »5/8{ }11\textsuperscript{10}« }\buchAbdrucke{\weitereDrucke{Arthur Schnitzler, Richard Beer-Hofmann: \emph{Briefwechsel 1891–1931}. Hg. Konstanze Fliedl. Wien, Zürich: \emph{Europaverlag} 1992, S. 112.} }\toendnotes[C]{\smallbreak}\pstart{}{\pb}richard beer-hofman \textcolor{pink}{ischl}{}\ledrightnote{\textcolor{pink}{Bad Ischl}}{ }\textcolor{pink}{eglmoos 22}{}\ledrightnote{\textcolor{pink}{Eglmoosgasse}}\pend{}{\bigskip}\pstart
           \noindent{}fr. \textcolor{pink}{wien}{}\ledrightnote{\textcolor{pink}{Wien}} 62+ 330 30 5/8{ }9 35\pend
           \pstart
           bitte \textcolor{blue}{sagen}{}\ledrightnote{→\textcolor{blue}{Rosa Freudenthal}} sie lieber dass ein
               derartiger wunsch fuer den absolut kein grund ersichtlich widersinnig und zerstoerend
               scheint. brief bringt ihnen aufklaerung herzlich \spacefill\mbox{arthur}\pend
           \endnumbering\briefempfaengerindex{Beer-Hofmann, Richard@\textsc{Beer-Hofmann, Richard}!zzzSchnitzler, Arthur@\emph{von Arthur Schnitzler}!1897-08-051@{5. 8. 1897}|)be}\mylabel{h}  \normalsize

\doendnotes{C}
\bigskip
\vfill

\clearpage

\footnotesize

\lohead{\textsc{register}}

% Definiere theindex-Environment komplett neu ohne reledmac
\makeatletter
\renewenvironment{theindex}{%
  \section*{\indexname}%
  \setlength{\parindent}{0pt}%
  \setlength{\parskip}{0pt plus 0.3pt}%
  \let\item\@idxitem
}{%
  \clearpage
}
\makeatother

\IfFileExists{\jobname-pw.ind}{\input{\jobname-pw.ind}}{}

\end{document}

      