%% latex-korrekturansicht-vorspann.tex
%% Vorspann für die Korrekturansicht.
%% Lädt die gemeinsame Datei latex-vorspann.tex mit gesetztem Schalter.

\newif\ifkorrekturansicht
\korrekturansichttrue

\input{../tex-inputs/latex-vorspann}


               \section[Hugo von Hofmannsthal an Arthur Schnitzler, 26. 3. {[}1902{]}]{ Hugo von Hofmannsthal an Arthur Schnitzler, 26. 3. {[}1902{]}}\nopagebreak\mylabel{v}\rehead{ }\normalsize\beginnumbering\briefempfaengerindex{Schnitzler, Arthur@\textsc{Schnitzler, Arthur}!zzzHofmannsthal, Hugo von@\emph{von Hugo von Hofmannsthal}!1902-03-261@{26. 3. 1902}|(be} \toendnotes[C]{\smallbreak\pagebreak[2]} \Standort{CUL, Schnitzler, B 43.}
\physDesc{Brief, 1 Blatt, 3 Seiten
\newline{}Handschrift: schwarze Tinte, deutsche Kurrent\newline{}Ordnung: 1) mit Bleistift von unbekannter Hand nummeriert: »\strikeout{192}« 2) mit Bleistift von unbekannter Hand nummeriert: »185«}\buchAbdrucke{\weitereDrucke{Hugo von Hofmannsthal, Arthur Schnitzler: \emph{Briefwechsel}. Hg. Therese Nickl und Heinrich Schnitzler. Frankfurt am Main: \emph{S. Fischer} 1964, S. 153.} }\pstart
           \raggedleft{}{\pb}26. III{ }abends.\pend
           \pstart
           lieber, wollen Sie nächſten Dinstag, Mittwoch oder Donnerstag mit
               mir, der Gräfin \textcolor{blue}{Christiane Thun}{}\ledrightnote{\textcolor{blue}{Christiane von Thun-Hohenstein-Salm-Reifferscheidt}} und \textcolor{blue}{Kaſſner}{}\ledrightnote{\textcolor{blue}{Rudolf Kassner}} (ſonſt niemand) um 1 Uhr
               frühſtücken, und zwar nicht bei mir, sondern im \textcolor{pink}{\textsc{Palais Thun-Salm}}{}\ledrightnote{\textcolor{pink}{Palais Thun-Salm}}, {\pb}\textcolor{pink}{\textsc{Kärntnerstrasse 41}.}{}\ledrightnote{\textcolor{pink}{Kärntner Straße}}?\pend
           \pstart
           Bitte wählen Sie den Tag, der Ihnen am beſten paſst (\uline{mir} wäre Mittwoch der liebſte) und ſchreiben mir \uuline{gleich} eine Zeile.\pend
           \pstart
           Von Herzen{\\[\baselineskip]}Ihr{\\[\baselineskip]}\spacefill\mbox{Hugo}\pend
           \leftskip=0em{}\pstart
           \noindent{}P. S. Die 50 fl. für den Hund ſchicken Sie {\pb}am beſten direct per Poſt an
                  Frau Hofräthin \textcolor{blue}{von Pollanetz}{}\ledrightnote{\textcolor{blue}{Malvine von Pollanetz}}, \textcolor{pink}{Wien I. Domgaſſe 6}{}\ledrightnote{\textcolor{pink}{Domgasse}}.\pend
           \endnumbering\briefempfaengerindex{Schnitzler, Arthur@\textsc{Schnitzler, Arthur}!zzzHofmannsthal, Hugo von@\emph{von Hugo von Hofmannsthal}!1902-03-261@{26. 3. 1902}|)be}\mylabel{h}  \normalsize

\doendnotes{C}
\bigskip
\vfill

\clearpage

\footnotesize

\lohead{\textsc{register}}

% Definiere theindex-Environment komplett neu ohne reledmac
\makeatletter
\renewenvironment{theindex}{%
  \section*{\indexname}%
  \setlength{\parindent}{0pt}%
  \setlength{\parskip}{0pt plus 0.3pt}%
  \let\item\@idxitem
}{%
  \clearpage
}
\makeatother

\IfFileExists{\jobname-pw.ind}{\input{\jobname-pw.ind}}{}

\end{document}

      