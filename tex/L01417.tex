%% latex-korrekturansicht-vorspann.tex
%% Vorspann für die Korrekturansicht.
%% Lädt die gemeinsame Datei latex-vorspann.tex mit gesetztem Schalter.

\newif\ifkorrekturansicht
\korrekturansichttrue

\input{../tex-inputs/latex-vorspann}


               \section[Richard Beer-Hofmann und Arthur Kaufmann an Arthur Schnitzler, {[}23.? 7. 1904{]}]{ Richard Beer-Hofmann und Arthur Kaufmann an Arthur Schnitzler,
               {[}23.? 7. 1904{]}}\nopagebreak\mylabel{v}\rehead{ }\normalsize\beginnumbering\briefempfaengerindex{Schnitzler, Arthur@\textsc{Schnitzler, Arthur}!zzzKaufmann, Arthur@\emph{von Arthur Kaufmann}!1904-07-231@{{[}23.? 7. 1904{]}}|(be}\briefempfaengerindex{Schnitzler, Arthur@\textsc{Schnitzler, Arthur}!zzzBeer-Hofmann, Richard@\emph{von Richard Beer-Hofmann}!1904-07-231@{{[}23.? 7. 1904{]}}|(be} \toendnotes[C]{\smallbreak\pagebreak[2]} \Standort{CUL, Schnitzler, B 8.}
\physDesc{Brief, 1 Blatt, 1 Seite
\newline{}Gedruckter Theaterzettel
\newline{}Handschrift Richard Beer-Hofmann: Bleistift, lateinische Kurrent\newline{}Handschrift Arthur Kaufmann: Bleistift, lateinische Kurrent\newline{}Ordnung: mit Bleistift von unbekannter Hand nummeriert:
                                    »185a« }\toendnotes[C]{\smallbreak}\pstart
           \noindent{}\centering{}{\pb}\textcolor{gray}{\textbf{\textcolor{brown}{Kurtheater}{}\ledrightnote{\textcolor{brown}{Kurtheater in Aussee}} in \textcolor{pink}{Aussee}{}\ledrightnote{\textcolor{pink}{Bad Aussee}}.}}\pend
           \pstart
           \noindent{}\centering{}\textcolor{gray}{\textbf{Direktion: \textcolor{blue}{Gustav Charlé}{}\ledrightnote{\textcolor{blue}{Gustav Charlé}} und
                     \textcolor{blue}{Gustav Müller}{}\ledrightnote{\textcolor{blue}{Gustav Müller}}.}}\pend
           \pstart
           \noindent{}\centering{}\textcolor{gray}{\textbf{Samstag, den 23. Juli 1904}}\pend
           {\bigskip}\pstart
           \noindent{}\centering{}\textcolor{gray}{\textbf{Bunter Abend}}\pend
           \pstart
           \noindent{}\centering{}\textcolor{gray}{\textbf{Gastspiel der Frau \textcolor{blue}{Emmy
                     Förster}{}\ledrightnote{\textcolor{blue}{Emmy Förster}}}}\pend
           {\bigskip}\pstart
           \noindent{}\centering{}\textcolor{gray}{\textbf{Den Anfang macht}}\pend
           \pstart
           \noindent{}\centering{}\textcolor{gray}{\textbf{\textcolor{green}{Kollegen}{}\ledrightnote{\textcolor{green}{Kollegen!}}}}\pend
           \pstart
           \noindent{}\centering{}\textcolor{gray}{\textbf{Komödie in 1 Akt von \textcolor{blue}{Annie
                     Neumann}{}\ledrightnote{\textcolor{blue}{Annie Neumann-Hofer}}.}}\pend
           \pstart
           \noindent{}\centering{}\textcolor{gray}{\textbf{(Regisseur Direktor \textcolor{blue}{Müller}{}\ledrightnote{\textcolor{blue}{Gustav Müller}}).}}\pend
           \pstart
           \noindent{}\centering{}\textcolor{gray}{\textbf{PERSONEN:}}\pend
           \pstart
           \noindent{}\textcolor{gray}{\textbf{Stella v. Balakow-Hartmann, Geigen-Virtuosin }}\hfill \textcolor{gray}{\textbf{\textsuperscript{*} \textsubscript{*} \textsuperscript{*}}}\pend
           \pstart
           \textcolor{gray}{\textbf{Werner Hartmann, ihr Gatte, Klavier-Virtuose }}\hfill \textcolor{gray}{\textbf{\textcolor{blue}{Oskar Beraun}{}\ledrightnote{\textcolor{blue}{Oskar Beraun}}}}\pend
           \pstart
           \textcolor{gray}{\textbf{Arthur v. Bront, Klavier-Virtuose }}\hfill \textcolor{gray}{\textbf{\textcolor{blue}{Theodor Robert}{}\ledrightnote{\textcolor{blue}{Theodor Robert}}}}\pend
           \pstart
           \textcolor{gray}{\textbf{Schwarz, Impresario }}\hfill \textcolor{gray}{\textbf{Dir. \textcolor{blue}{Gustav Müller}{}\ledrightnote{\textcolor{blue}{Gustav Müller}}}}\pend
           \pstart
           \textcolor{gray}{\textbf{Minna, Kammermädchen bei Hartmann}}\hfill \textcolor{gray}{\textbf{\textcolor{blue}{Rosa Vennyer}{}\ledrightnote{\textcolor{blue}{Rosa Vennyer}}}}.\pend
           \pstart
           \textcolor{gray}{\textbf{Franz{[},{]} Diener {[}bei
                        Hartmann{]}}}\hfill \textcolor{gray}{\textbf{\textcolor{blue}{Fritz Schönhof}{}\ledrightnote{\textcolor{blue}{Friedrich Schönhof}}}}\pend
           \pstart
           \centering{}\textcolor{gray}{\textbf{Zeit: die Gegenwart. Ein Winter{[}-{]}Nachmittag
                  von 4 bis halb 8 Uhr. Ort: \textcolor{pink}{Berlin}{}\ledrightnote{\textcolor{pink}{Berlin}}.}}\pend
           \pstart
           \noindent{}\textcolor{gray}{\textbf{\textsuperscript{*} \textsubscript{*} \textsuperscript{*} Stella }}\hfill \textcolor{gray}{\textbf{Frau \textcolor{blue}{Emmy Förster}{}\ledrightnote{\textcolor{blue}{Emmy Förster}} als
                     Gast.}}\pend
           {\bigskip}\pstart
           \noindent{}\centering{}\textcolor{gray}{\textbf{Hierauf:}}\pend
           \pstart
           \noindent{}\textcolor{gray}{\textbf{Vorträge}}\hfill \textcolor{gray}{\textbf{Frl. \textcolor{blue}{Hel. Robert}{}\ledrightnote{\textcolor{blue}{Helene Robert}}}}\pend
           \pstart
           \centering{}\textcolor{gray}{\textbf{»\textcolor{green}{Frauentypen}{}\ledrightnote{\textcolor{green}{Frauentypen}}« von \textcolor{blue}{Arthur Pserhofer}{}\ledrightnote{\textcolor{blue}{Arthur Pserhofer}}}}\pend
           \pstart
           \noindent{}\centering{}\textcolor{gray}{\textbf{»\textcolor{green}{Capricio}{}\ledrightnote{\textcolor{green}{Auf Kypros}}« aus dem Tagebuche
                  einer Demi-Vierge von \textcolor{blue}{Marie Madeleine}{}\ledrightnote{\textcolor{blue}{Marie Madeleine}}}}\pend
           \pstart
           \noindent{}\centering{}\textcolor{gray}{\textbf{»\textcolor{green}{Das Mädel ohne Bräutigam}{}\ledrightnote{\textcolor{green}{Das Mädchen ohne Bräutigam}}« v.
                     \textcolor{blue}{Otto Jul. Bierbaum}{}\ledrightnote{\textcolor{blue}{Otto Julius Bierbaum}}}}\pend
           {\bigskip}\pstart
           \noindent{}\centering{}\textcolor{gray}{\textbf{\textcolor{green}{Am Theater und im Leben}{}\ledrightnote{\textcolor{green}{Am Theater und im Leben}}.}}\pend
           \pstart
           \noindent{}\centering{}\textcolor{gray}{\textbf{Tanz-Duett, vorgetragen von Frln \textcolor{blue}{\textbf{Jenik}}{}\ledrightnote{\textcolor{blue}{Hilda Jenik}} und Direktor \textcolor{blue}{\textbf{Müller}}{}\ledrightnote{\textcolor{blue}{Gustav Müller}}.}}\pend
           {\bigskip}\pstart
           \noindent{}\centering{}\textcolor{gray}{\textbf{Zum Schlusse:}}\pend
           \pstart
           \noindent{}\centering{}\textcolor{gray}{\textbf{\textcolor{green}{Literatur}{}\ledrightnote{\textcolor{green}{Literatur}}}}\pend
           \pstart
           \noindent{}\centering{}\textcolor{gray}{\textbf{Lustspiel in 1 Akt von Arthur Schnitzler.}}\pend
           \pstart
           \noindent{}\centering{}\textcolor{gray}{\textbf{(Regisseur Direktor \textcolor{blue}{Müller}{}\ledrightnote{\textcolor{blue}{Gustav Müller}}).}}\pend
           \pstart
           \noindent{}\centering{}\textcolor{gray}{\textbf{Personen:}}\pend
           \pstart
           \noindent{}\textcolor{gray}{\textbf{Margarethe }}\hfill \textcolor{gray}{\textbf{\textsuperscript{*} \textsubscript{*} \textsuperscript{*}}}\pend
           \pstart
           \textcolor{gray}{\textbf{Clemens }}\hfill \textcolor{gray}{\textbf{Dir. \textcolor{blue}{Gustav Müller}{}\ledrightnote{\textcolor{blue}{Gustav Müller}}}}\pend
           \pstart
           \textcolor{gray}{\textbf{Gilbert }}\hfill \textcolor{gray}{\textbf{\textcolor{blue}{Fritz Digruber}{}\ledrightnote{\textcolor{blue}{Friedrich Digruber}}}}\pend
           \pstart
           \textcolor{gray}{\textbf{\textsuperscript{*} \textsubscript{*} \textsuperscript{*} Margarethe }}\hfill \textcolor{gray}{\textbf{Frau \textcolor{blue}{Emmy Förster}{}\ledrightnote{\textcolor{blue}{Emmy Förster}} als
                     Gast.}}\pend
           {\bigskip}\pstart
           \noindent{}\textcolor{gray}{\textbf{\label{T_L01417_1v}\edtext{Preise}{\lemma{\textnormal{\emph{Preise}}}\Cendnote{\textnormal{Druckfehler, korrigiert aus »Priese«}}}\label{T_L01417_1h} der
                  Plätze:}}\pend
           \settowidth{\longeste}{Ein Parterresitz, 8.–15. Reihe}\settowidth{\longestz}{„}\settowidth{\longestd}{–.80}\settowidth{\longestv}{}\settowidth{\longestf}{}\addtolength\longeste{1em}
        \addtolength\longestz{1em}
        \addtolength\longestd{1em}
      \pstart\noindent\makebox[\the\longeste][l]{\textcolor{gray}{\textbf{Eine Loge für 4 Personen}}}\makebox[\the\longestz][l]{\textcolor{gray}{\textbf{K}}}
                  \makebox[\the\longestd][l]{\textcolor{gray}{\textbf{16.–}}}\pend\pstart\noindent\makebox[\the\longeste][l]{\textcolor{gray}{\textbf{Eine Logensitz}}}\makebox[\the\longestz][l]{\textcolor{gray}{\textbf{„}}}
                  \makebox[\the\longestd][l]{\textcolor{gray}{\textbf{5.–}}}\pend\pstart\noindent\makebox[\the\longeste][l]{\textcolor{gray}{\textbf{Ein Orchestersitz, 1.–3. Reihe}}}\makebox[\the\longestz][l]{\textcolor{gray}{\textbf{„}}}
                  \makebox[\the\longestd][l]{\textcolor{gray}{\textbf{4.–}}}\pend\pstart\noindent\makebox[\the\longeste][l]{\textcolor{gray}{\textbf{Ein Parkettsitz, 4.–7. Reihe }}}\makebox[\the\longestz][l]{\textcolor{gray}{\textbf{„}}}
                  \makebox[\the\longestd][l]{\textcolor{gray}{\textbf{3.–}}}\pend\pstart\noindent\makebox[\the\longeste][l]{\textcolor{gray}{\textbf{Ein Parterresitz, 8.–15. Reihe }}}\makebox[\the\longestz][l]{\textcolor{gray}{\textbf{„}}}
                  \makebox[\the\longestd][l]{\textcolor{gray}{\textbf{2.–}}}\pend\pstart\noindent\makebox[\the\longeste][l]{\textcolor{gray}{\textbf{Ein Parterrestehplatz }}}\makebox[\the\longestz][l]{\textcolor{gray}{\textbf{„}}}
                  \makebox[\the\longestd][l]{\textcolor{gray}{\textbf{–.80}}}\pend\pstart
           \centering{}\textcolor{gray}{\textbf{Die Tageskasse befindet sich \textcolor{pink}{Ischlerstrasse 72}{}\ledrightnote{\textcolor{pink}{Ischler Straße}} und ist geöffnet von 9 Uhr vormittags bis 1 Uhr mittags
                  und von 3–5 nachmittags.}}\pend
           \pstart
           \noindent{}Nicht umzubringen! »\label{K_L01417_1v}\edtext{In Zeit und
                  Ewigkeit}{\lemma{\textnormal{\emph{In Zeit und
                  Ewigkeit}}}\Cendnote{\textnormal{kein Zitat, sondern stehende
                  Wendung in der katholischen Fachsprache}}}\label{K_L01417_1h}«!\pend
           \pstart
           Seien Sie \textcolor{blue}{Beide}{}\ledrightnote{→\textcolor{blue}{Olga Schnitzler}} herzlich
               gegrüsst von \spacefill\mbox{Richard und \textcolor{blue}{Paula}{}\ledrightnote{\textcolor{blue}{Paula Beer-Hofmann}}.}\pend
           \pstart
           {[}hs. Kaufmann:{]} Es grüsst Sie herzlich{\\}\spacefill\mbox{AKaufmann}\pend
           \endnumbering\briefempfaengerindex{Schnitzler, Arthur@\textsc{Schnitzler, Arthur}!zzzKaufmann, Arthur@\emph{von Arthur Kaufmann}!1904-07-231@{{[}23.? 7. 1904{]}}|)be}\briefempfaengerindex{Schnitzler, Arthur@\textsc{Schnitzler, Arthur}!zzzBeer-Hofmann, Richard@\emph{von Richard Beer-Hofmann}!1904-07-231@{{[}23.? 7. 1904{]}}|)be}\mylabel{h}  \normalsize

\doendnotes{C}
\bigskip
\vfill

\clearpage

\footnotesize

\lohead{\textsc{register}}

% Definiere theindex-Environment komplett neu ohne reledmac
\makeatletter
\renewenvironment{theindex}{%
  \section*{\indexname}%
  \setlength{\parindent}{0pt}%
  \setlength{\parskip}{0pt plus 0.3pt}%
  \let\item\@idxitem
}{%
  \clearpage
}
\makeatother

\IfFileExists{\jobname-pw.ind}{\input{\jobname-pw.ind}}{}

\end{document}

      