%% latex-korrekturansicht-vorspann.tex
%% Vorspann für die Korrekturansicht.
%% Lädt die gemeinsame Datei latex-vorspann.tex mit gesetztem Schalter.

\newif\ifkorrekturansicht
\korrekturansichttrue

\input{../tex-inputs/latex-vorspann}


               \section[Arthur Schnitzler an Hugo von Hofmannsthal, {[}12. 3. 1892{]}]{ Arthur Schnitzler an Hugo von Hofmannsthal, {[}12. 3. 1892{]}}\nopagebreak\mylabel{v}\rehead{ }\normalsize\beginnumbering\briefempfaengerindex{Hofmannsthal, Hugo von@\textsc{Hofmannsthal, Hugo von}!zzzSchnitzler, Arthur@\emph{von Arthur Schnitzler}!1892-03-121@{{[}12. 3. 1892{]}}|(be} \toendnotes[C]{\smallbreak\pagebreak[2]} \Standort{FDH, Hs-30885,16.}
\physDesc{Briefkarte
\newline{}Handschrift: Bleistift, deutsche Kurrent\newline{}Ordnung: von Schnitzler mutmaßlich während der Durchsicht der Briefe 1929 mit Bleistift mit
                                    einer Jahreszahl versehen: »9\substVorne{}\textsuperscript{2}\substDazwischen{}1\substHinten{}« }\buchAbdrucke{\weitereDrucke{Hugo von Hofmannsthal, Arthur Schnitzler: \emph{Briefwechsel}. Hg. Therese Nickl und Heinrich Schnitzler. Frankfurt am Main: \emph{S. Fischer} 1964, S. 30.} }\toendnotes[C]{\smallbreak}\pstart
           \noindent{}{\pb}Lieber Hugo, morgen So{\geminationn}tag bin ich Nachmittags
                    in einem \label{K_L00080_1v}\edtext{Concert}{\lemma{\textnormal{\emph{Concert}}}\Cendnote{\textnormal{Das einzige Konzert, das an einem
                        Sonntagnachmittag Werke von \textcolor{blue}{Anton
                            Rückauf} auf dem Programm hatte und sich nachweisen lässt,
                        fand am 13. 3. 1892
                   statt. In \textcolor{blue}{Schnitzler}s Aufzeichnungen gibt es keinen Hinweis, dass er
                        es besuchte, sondern es ist für den Tag nur ein Besuch in der \emph{\textcolor{brown}{Schauspielschule Otto}} vermerkt. (\emph{Cambridge University Library} A 179a).}}}\label{K_L00080_1h}, wo \textcolor{blue}{Rückauf}{}\ledrightnote{\textcolor{blue}{Anton Rückauf}} (mein einſtiger Lehrer, der mich ſehr
                    intereſſirt) aufgeführt wird. Alſo nicht {\pb}zu
                    Hauſe. Ko{\geminationm}en Sie möglichſt bald, damit wir noch
                    einen Abend dieſer Woche verabreden können.\pend
           \pstart
           Herzlichst{\\[\baselineskip]}Ihr\spacefill\mbox{Arth Sch}\pend
           \leftskip=0em{}\endnumbering\briefempfaengerindex{Hofmannsthal, Hugo von@\textsc{Hofmannsthal, Hugo von}!zzzSchnitzler, Arthur@\emph{von Arthur Schnitzler}!1892-03-121@{{[}12. 3. 1892{]}}|)be}\mylabel{h}  \normalsize

\doendnotes{C}
\bigskip
\vfill

\clearpage

\footnotesize

\lohead{\textsc{register}}

% Definiere theindex-Environment komplett neu ohne reledmac
\makeatletter
\renewenvironment{theindex}{%
  \section*{\indexname}%
  \setlength{\parindent}{0pt}%
  \setlength{\parskip}{0pt plus 0.3pt}%
  \let\item\@idxitem
}{%
  \clearpage
}
\makeatother

\IfFileExists{\jobname-pw.ind}{\input{\jobname-pw.ind}}{}

\end{document}

      