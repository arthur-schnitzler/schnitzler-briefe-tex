%% latex-korrekturansicht-vorspann.tex
%% Vorspann für die Korrekturansicht.
%% Lädt die gemeinsame Datei latex-vorspann.tex mit gesetztem Schalter.

\newif\ifkorrekturansicht
\korrekturansichttrue

\input{../tex-inputs/latex-vorspann}


               \section[Robert Adam an Arthur Schnitzler, 2. 4. 1913]{ Robert Adam an Arthur Schnitzler, 2. 4. 1913}\nopagebreak\mylabel{v}\rehead{ }\normalsize\beginnumbering\briefempfaengerindex{Schnitzler, Arthur@\textsc{Schnitzler, Arthur}!zzzAdam, Robert@\emph{von Robert Adam}!1913-04-021@{2. 4. 1913}|(be} \toendnotes[C]{\smallbreak\pagebreak[2]} \Standort{DLA, A:Schnitzler, HS.NZ85.1.4230,5.}
\physDesc{Brief, 1 Blatt, 2 Seiten
\newline{}Handschrift: schwarze Tinte, deutsche Kurrent
\newline{}Schnitzler: 1) mit Bleistift beschriftet: »\textsc{Adam}« 2) mit rotem Buntstift eine Unterstreichung}\Standort{Wien, Österreichische Nationalbibliothek, Cod.ser. 52.266, 155.}
\physDesc{handschriftliche Abschrift
\newline{}Handschrift: schwarze Tinte, Gabelsberger Kurzschrift}\Standort{Wien, Österreichische Nationalbibliothek, Cod.ser. 52.266, 155.}
\physDesc{maschinelle Abschrift
\newline{}Schreibmaschine}\pstart
           \raggedleft{}{\pb}\textcolor{pink}{Ziſtersdorf}{}\ledrightnote{\textcolor{pink}{Zistersdorf}}, am 2. April
                            1913.\pend
           \pstart{}Hochverehrter Herr Doktor!\pend\pstart
           Das freundliche Intereſſe, das Sie seinerzeit meiner Komödie \textcolor{green}{Die Geſchichte des Alî ibn Bekkâr mit Schams an-Nahâr}{}\ledrightnote{\textcolor{green}{Die Geschichte des Alî ibn Bekkâr mit Schams an-Nahâr}} und
                    vor zwei Jahren dem Manuſkript der Komödie: \textcolor{green}{Neidhard}{}\ledrightnote{\textcolor{green}{Neidhard}} entgegenbrachten, ermutigt mich, hochverehrter Herr Doktor,
                    neuerlich mit einer Bitte an Sie heranzutreten.\pend
           \pstart
           Ich habe in meiner ländlichen Abgeſchiedenheit kürzlich eine dramatiſche Studie
                    zum Abſchluß gebracht, die ich \textcolor{green}{\textsc{Fatme}}{}\ledrightnote{\textcolor{green}{Fatme}} nennen will. Es ſind vier Proſa-Akte von nicht allzu großem Umfange.\pend
           \pstart
           {\pb}Darf ich mir erlauben, hochverehrter Herr
                    Doktor, Ihnen das Manuſkript, ſobald die Schreibmaſchinenabſchrift
                    fertiggeſtellt iſt, einzuſenden?\pend
           \pstart
           Ich weiß, daß ich Ihre Güte und Zeit in unbilligem Maße in Anſpruch nehme; aber
                    Sie waren bisher der Einzige, der ſich meiner annahm, und ich ſetze meine ganze
                    Hoffnung in Ihre Güte.\pend
           \pstart
           Mit den ergebenſten Grüßen\pend
           \pstart
           Ihr{\\[\baselineskip]}\spacefill\mbox{Robert Adam}\pend
           \leftskip=0em{}\pstart
           \noindent{}\raggedleft{}(Bezirksrichter Dr Robert Adam{\\}Pollak, \textcolor{pink}{Ziſtersdorf{ }\textsc{N. Ö.}}{}\ledrightnote{\textcolor{pink}{Zistersdorf}})\pend
           \endnumbering\briefempfaengerindex{Schnitzler, Arthur@\textsc{Schnitzler, Arthur}!zzzAdam, Robert@\emph{von Robert Adam}!1913-04-021@{2. 4. 1913}|)be}\mylabel{h}  \normalsize

\doendnotes{C}
\bigskip
\vfill

\clearpage

\footnotesize

\lohead{\textsc{register}}

% Definiere theindex-Environment komplett neu ohne reledmac
\makeatletter
\renewenvironment{theindex}{%
  \section*{\indexname}%
  \setlength{\parindent}{0pt}%
  \setlength{\parskip}{0pt plus 0.3pt}%
  \let\item\@idxitem
}{%
  \clearpage
}
\makeatother

\IfFileExists{\jobname-pw.ind}{\input{\jobname-pw.ind}}{}

\end{document}

      