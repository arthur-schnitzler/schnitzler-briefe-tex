%% latex-korrekturansicht-vorspann.tex
%% Vorspann für die Korrekturansicht.
%% Lädt die gemeinsame Datei latex-vorspann.tex mit gesetztem Schalter.

\newif\ifkorrekturansicht
\korrekturansichttrue

\input{../tex-inputs/latex-vorspann}


               \section[Marie Herzfeld an Arthur Schnitzler, 5. 3. 1931]{ Marie Herzfeld an Arthur Schnitzler, 5. 3. 1931}\nopagebreak\mylabel{v}\rehead{ }\normalsize\beginnumbering\briefempfaengerindex{Schnitzler, Arthur@\textsc{Schnitzler, Arthur}!zzzHerzfeld, Marie@\emph{von Marie Herzfeld}!1931-03-052@{5. 3. 1931}|(be} \toendnotes[C]{\smallbreak\pagebreak[2]} \Standort{Privatbesitz, Reinhard Urbach, \emph{ohne Signatur}.}
\physDesc{Brief, 1 Blatt, 4 Seiten, fotografische Vervielfältigung
\newline{}Handschrift: schwarze Tinte, lateinische Kurrent
\newline{}Schnitzler: mutmaßlich mit rotem Buntstift drei Unterstreichungen \newline{}Zusatz: Das Original des Briefes ist verschollen. Eine Kopie des Briefes
                                 wurde am 20. 10. 1972 von \textcolor{blue}{Heinrich Schnitzler} an \textcolor{blue}{Reinhard Urbach}
                                 übermittelt. }\toendnotes[C]{\smallbreak}\pstart
           \noindent{}\centering{}{\pb}\textcolor{pink}{Wien III/\textsubscript{3}, Oetzeltgasse 1}{}\ledrightnote{\textcolor{pink}{Ölzeltgasse}}\pend
           \pstart
           \raggedleft{}den 5. März 1931\pend
           \pstart
           Sehr geehrter Herr Doktor, verzeihen Sie, wenn ich Ihre Muße – Arbeitsmuße – störe und mit einer Frage in Ihre
               Einsamkeit breche. Auf Wunsch der Zeitschrift »\textcolor{green}{Corona}{}\ledrightnote{\textcolor{green}{Corona. Zweimonatsschrift}}« habe ich aus meinen \textcolor{blue}{Loris}{}\ledrightnote{\textcolor{blue}{Hugo von Hofmannsthal}}-Erinnerungen und \textcolor{blue}{Loris}{}\ledrightnote{\textcolor{blue}{Hugo von Hofmannsthal}}-Briefen einen
                  \label{K_L02589-1v}\edtext{\textcolor{green}{Aufsatz}{}\ledrightnote{\textcolor{green}{Loris. Blätter der Erinnerung}}}{\lemma{\textnormal{\emph{Aufsatz}}}\Cendnote{\textnormal{Trotz der im Brief
                  vorgebrachten Eile verzögerte sich die Publikation: \textcolor{blue}{Marie Herzfeld}: \emph{\textcolor{green}{Loris.
                        Blätter der Erinnerungen}}. In: \emph{\textcolor{green}{Corona.
                        Zweimonatsschrift}}, Jg. 2, Nr. 6, Mai 1932,
                     S. 715–732.}}}\label{K_L02589-1h} zusammengestellt, {\pb}in dem ich auch aus den schönen
                  \label{K_L02589-2v}\edtext{\textcolor{green}{Briefen}{}\ledrightnote{\textcolor{green}{Briefe an Freunde}} schöpfe, die Sie im Aprilheft der \textcolor{green}{N. R.}{}\ledrightnote{\textcolor{green}{Die neue Rundschau}} v. 1930}{\lemma{\textnormal{\emph{Briefen … 1930}}}\Cendnote{\textnormal{\textcolor{blue}{Hugo von Hofmannsthal}: \emph{\textcolor{green}{Briefe an Freunde}}. In: \emph{\textcolor{green}{Die
                        neue Rundschau}}, Jg. 41, Nr. 4,
                        1. 4. 1930, S. 512–519. Siehe Hugo von Hofmannsthal an Arthur Schnitzler, 19. 7. [1892]}}}\label{K_L02589-2h} hatten. Am
                  19. Juli 92 spricht \textcolor{blue}{Hofmannsthal}{}\ledrightnote{\textcolor{blue}{Hugo von Hofmannsthal}} von dem \label{K_L02589-3v}\edtext{\textcolor{green}{Renaissancedrama}{}\ledrightnote{→\textcolor{green}{Ascanio und Gioconda}}, an dem er
                  arbeite}{\lemma{\textnormal{\emph{Renaissancedrama, … arbeite}}}\Cendnote{\textnormal{\textcolor{green}{Ascanio und Gioconda} blieb zu Lebzeiten
                  unveröffentlicht, heute in \emph{Sämtliche Werke. Kritische
                        Ausgabe}, Bd. 18.}}}\label{K_L02589-3h}: mir erzählte er davon nichts,
               obwohl er um diese Zeit mit mir lebhaft korrespon{\pb}dierte, und ich wagte, trotz einiger
               innerer Einwände, die Hypothese, dass es sich um eine Beschäftigung mit d. \textcolor{green}{geretteten Venedig}{}\ledrightnote{\textcolor{green}{Das gerettete Venedig}} handelte, die er dann später, wie
               Sie wissen, \uline{mehrmals} neu aufnahm und erst \label{K_L02589-4v}\edtext{nach Jahren zu Ende brachte.}{\lemma{\textnormal{\emph{nach … brachte.}}}\Cendnote{\textnormal{\textcolor{blue}{Hofmannsthal} arbeitete von August 1902 bis
                     Juli 1904 an seinem Trauerspiel \emph{\textcolor{green}{Das gerettete Venedig}}, das am
                     21. 1. 1905 in \textcolor{pink}{Berlin}
                  uraufgeführt wurde und im gleichen Jahr gedruckt erschien.}}}\label{K_L02589-4h}{ }{\pb}Wollen Sie, aus Ihrem besseren Wissen,
               mich aufklären? Ich wäre Ihnen sehr dankbar! Aber die Sache drängt! In großer
               Schätzung,\pend
           \pstart \spacefill\mbox{Marie Herzfeld}\pend{}\endnumbering\briefempfaengerindex{Schnitzler, Arthur@\textsc{Schnitzler, Arthur}!zzzHerzfeld, Marie@\emph{von Marie Herzfeld}!1931-03-052@{5. 3. 1931}|)be}\mylabel{h}  \normalsize

\doendnotes{C}
\bigskip
\vfill

\clearpage

\footnotesize

\lohead{\textsc{register}}

% Definiere theindex-Environment komplett neu ohne reledmac
\makeatletter
\renewenvironment{theindex}{%
  \section*{\indexname}%
  \setlength{\parindent}{0pt}%
  \setlength{\parskip}{0pt plus 0.3pt}%
  \let\item\@idxitem
}{%
  \clearpage
}
\makeatother

\IfFileExists{\jobname-pw.ind}{\input{\jobname-pw.ind}}{}

\end{document}

      