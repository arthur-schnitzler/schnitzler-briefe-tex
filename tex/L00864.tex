%% latex-korrekturansicht-vorspann.tex
%% Vorspann für die Korrekturansicht.
%% Lädt die gemeinsame Datei latex-vorspann.tex mit gesetztem Schalter.

\newif\ifkorrekturansicht
\korrekturansichttrue

\input{../tex-inputs/latex-vorspann}


               \section[Arthur Schnitzler an Hermann Bahr, 1. 12. 1898]{ Arthur Schnitzler an Hermann Bahr, 1. 12. 1898}\nopagebreak\mylabel{v}\rehead{ }\normalsize\beginnumbering\briefempfaengerindex{Bahr, Hermann@\textsc{Bahr, Hermann}!zzzSchnitzler, Arthur@\emph{von Arthur Schnitzler}!1898-12-012@{1. 12. 1898}|(be} \toendnotes[C]{\smallbreak\pagebreak[2]} \Standort{TMW, HS AM 60159 Ba.}
\physDesc{Briefkarte
\newline{}Handschrift: schwarze Tinte, deutsche Kurrent\newline{}Ordnung: Lochung }\buchAbdrucke{\weitereDrucke{1) \emph{9. 12. 1898, Abschrift.} In: Arthur Schnitzler: \emph{The Letters of Arthur Schnitzler to Hermann Bahr}. Edited, annotated, and with an introduction, by Donald G.
                        Daviau. Chapel Hill: \emph{The University of North Carolina Press} 1978, S. 64 (University of North Carolina studies in the Germanic languages
                        and literatures, 89).} \weitereDrucke{2) Hermann Bahr, Arthur Schnitzler: \emph{Briefwechsel, Aufzeichnungen, Dokumente (1891–1931)}. Hg. Kurt Ifkovits und Martin Anton Müller. Göttingen: \emph{Wallstein} 2018, S. 165.} }\toendnotes[C]{\smallbreak}\pstart
           \noindent{}{\pb}Lieber Hermann, ich danke dir herzlich für deine freundlichen \textcolor{green}{Glückw\damage{ün}ſche}{}\ledrightnote{→\textcolor{green}{Das Vermächtnis. Schauspiel in drei Akten}}. Den »\textcolor{green}{Kakadu}{}\ledrightnote{\textcolor{green}{Der grüne Kakadu. Groteske in einem Akt}}« hat die \textcolor{green}{F\damage{rei}e Bühne}{}\ledrightnote{\textcolor{green}{Neue Deutsche Rundschau}}{ }ſchon (»\textcolor{green}{Die \textsc{Neue Deutsche Rundschau}}{}\ledrightnote{\textcolor{green}{Neue Deutsche Rundschau}}« mein’ ich); er ſoll, während der Recurs wegen der \label{K_L00864_1v}\edtext{Freigabe}{\lemma{\textnormal{\emph{Freigabe}}}\Cendnote{\textnormal{Nachdem
                  das Stück am \textcolor{pink}{Burgtheater} am
                     1. 3. 1899 zum ersten Mal gegeben worden war, wurde es in der \textcolor{pink}{Wien}er Einrichtung (Umbenennung einer Figur,
                  Kürzung von Freiheitsrufen) in \textcolor{pink}{Berlin} erneut der
                  Zensur eingereicht und diese »hat soeben das Stück in dieser Form zur
                     Aufführung freigegeben« (\emph{\textcolor{brown}{Berliner Tageblatt}}, Jg. 28, Nr. 136,
                        15. 3. 1899, Morgen-Ausgabe, S. 3).}}}\label{K_L00864_1h} im Gang iſt,
               an der »\textcolor{brown}{Freien Bühne}{}\ledrightnote{\textcolor{brown}{Freie Bühne}}« in \textcolor{pink}{Berlin}{}\ledrightnote{\textcolor{pink}{Berlin}} aufgeführt werden. Jedenfalls iſt nun mein ganzer \textcolor{green}{Einakter Abend}{}\ledrightnote{→\textcolor{green}{Der grüne Kakadu – Paracelsus – Die Gefährtin. Drei Einakter}} hinausgeſchoben. So iſt es
               vorläufig noch verfrüht, dir von der »\textcolor{green}{Gefährtin}{}\ledrightnote{\textcolor{green}{Die Gefährtin. Schauspiel in einem Akt}}«,
               einem dieſer Einakter, zu reden, den ich {\pb}keineswegs \label{LL023-1v}\uline{vor}\label{LL023-1h} der Aufführg erſcheinen laſſen möchte, den ich aber bi\damage{sh}er noch nicht vergeben habe. – Du \damage{hof}fſt meine \textcolor{brown}{\textsc{Kosmopolis}-Honorarforderungen}{}\ledrightnote{\textcolor{brown}{Cosmopolis}} durchzuſetzen – das
               wäre ſehr ſchön – denn die \textcolor{brown}{\textsc{Kosmopolis}}{}\ledrightnote{\textcolor{brown}{Cosmopolis}} iſt \label{K_L00864_2v}\edtext{verkracht und ſchuldet mir
               ungezählte Mark}{\lemma{\textnormal{\emph{verkracht … Mark}}}\Cendnote{\textnormal{\emph{\textcolor{brown}{Cosmopolis}} erschien mehrsprachig und monatlich,
                  zum ersten Mal im Januar 1896, zum letzten Mal im November
                     1898. Zum finalen Heft hat \textcolor{blue}{Schnitzler}{ }\emph{\textcolor{green}{Paracelsus}} (Bd. 12, H. 35,
                     S. 489–527) beigesteuert.}}}\label{K_L00864_2h}. Also verſuch’s\substVorne{}\textsuperscript{, –}\substDazwischen{}.\substHinten{}\pend
           \pstart
           – Auf baldige \label{K_L00864_3v}\edtext{Gratulationsrevanche}{\lemma{\textnormal{\emph{Gratulationsrevanche}}}\Cendnote{\textnormal{Premiere der
                  ersten \textcolor{pink}{Wiener} Inszenierung von \emph{\textcolor{green}{Der Star}} am 10. 12. 1897}}}\label{K_L00864_3h} im \textcolor{pink}{Volkstheater}{}\ledrightnote{\textcolor{pink}{Volkstheater}}.\pend
           \pstart Herzlichen Gruſs. Dein \spacefill\mbox{Arthur Sch.}\pend{}\pstart
           \textcolor{pink}{Wien}{}\ledrightnote{\textcolor{pink}{Wien}}{ }1. 12. 98\pend
           \endnumbering\briefempfaengerindex{Bahr, Hermann@\textsc{Bahr, Hermann}!zzzSchnitzler, Arthur@\emph{von Arthur Schnitzler}!1898-12-012@{1. 12. 1898}|)be}\mylabel{h}  \normalsize

\doendnotes{C}
\bigskip
\vfill

\clearpage

\footnotesize

\lohead{\textsc{register}}

% Definiere theindex-Environment komplett neu ohne reledmac
\makeatletter
\renewenvironment{theindex}{%
  \section*{\indexname}%
  \setlength{\parindent}{0pt}%
  \setlength{\parskip}{0pt plus 0.3pt}%
  \let\item\@idxitem
}{%
  \clearpage
}
\makeatother

\IfFileExists{\jobname-pw.ind}{\input{\jobname-pw.ind}}{}

\end{document}

      