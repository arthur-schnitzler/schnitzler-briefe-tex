%% latex-korrekturansicht-vorspann.tex
%% Vorspann für die Korrekturansicht.
%% Lädt die gemeinsame Datei latex-vorspann.tex mit gesetztem Schalter.

\newif\ifkorrekturansicht
\korrekturansichttrue

\input{../tex-inputs/latex-vorspann}


               \section[Arthur Schnitzler an Richard Beer-Hofmann, 24. 8. 1895]{ Arthur Schnitzler an Richard Beer-Hofmann, 24. 8. 1895}\nopagebreak\mylabel{v}\rehead{ }\normalsize\beginnumbering\briefempfaengerindex{Beer-Hofmann, Richard@\textsc{Beer-Hofmann, Richard}!zzzSchnitzler, Arthur@\emph{von Arthur Schnitzler}!1895-08-241@{24. 8. 1895}|(be} \toendnotes[C]{\smallbreak\pagebreak[2]} \Standort{YCGL, MSS 31.}
\physDesc{Brief, 1 Blatt, 2 Seiten
\newline{}Handschrift: Bleistift, deutsche Kurrent}\buchAbdrucke{\weitereDrucke{Arthur Schnitzler, Richard Beer-Hofmann: \emph{Briefwechsel 1891–1931}. Hg. Konstanze Fliedl. Wien, Zürich: \emph{Europaverlag} 1992, S. 78–79.} }\toendnotes[C]{\smallbreak}\pstart
           \raggedleft{}{\pb}\textcolor{pink}{\textsc{St Johann in Tirol}}{}\ledrightnote{\textcolor{pink}{St. Johann in Tirol}}{\\}24. 8. 95\pend
           \pstart{}Lieber Richard.\pend\pstart
           Genau auf der \uline{Grenze} von \textcolor{pink}{\textsc{Baiern}}{}\ledrightnote{\textcolor{pink}{Bayern}} u \textcolor{pink}{\textsc{Tirol}}{}\ledrightnote{\textcolor{pink}{Tirol}}{ }ſauſte uns ein unheimlich gekleideter \textsc{Bicyclist} mit einem Dolch, Lederhoſen, Zugſchuhen, nackten
               Knieen, weißem Flanellhemd, keiner Cravate, Lodenhut entgegen, und war der \textcolor{blue}{Burckhard}{}\ledrightnote{\textcolor{blue}{Max Eugen Burckhard}}. –\pend
           \pstart
           Jetzt hat es angefangen zu gießen, zu blitzen, zu donnern. Vielleicht ſchlägt es ein;
                  da{\geminationn}{ }ſind wir extra von \textcolor{pink}{Salzburg}{}\ledrightnote{\textcolor{pink}{Salzburg}} nach {\pb}\textcolor{pink}{Johann in Tirol}{}\ledrightnote{\textcolor{pink}{St. Johann in Tirol}} gefahren u. ſ. w. (Siehe \textcolor{green}{Märchen}{}\ledrightnote{\textcolor{green}{Das Märchen der 672. Nacht}} von \textcolor{blue}{\textsc{Loris}}{}\ledrightnote{\textcolor{blue}{Hugo von Hofmannsthal}}.)\pend
           \pstart
           Wir warten auf einen Zug. Die Partie war wunderbar. \label{K_L00477_1v}\edtext{\textsc{Le canif} das Federmeſſer}{\lemma{\textnormal{\emph{Le canif das Federmeſſer}}}\Cendnote{\textnormal{Die französische Vokabel »canif« richtig übersetzt, unklare
                  Anspielung.}}}\label{K_L00477_1h}.\pend
           \pstart
           Herzliche Grüße{\\[\baselineskip]}Ihr \spacefill\mbox{Arthur}\pend
           \leftskip=0em{}\pstart
           \noindent{}Wenn Sie jenes kleine \textcolor{blue}{Weſen}{}\ledrightnote{→\textcolor{blue}{Irma Fabiani}}{ }ſehen, dem \label{K_L00477_2v}\edtext{Wehmut und Verachtung bevorſteht}{\lemma{\textnormal{\emph{Wehmut … bevorſteht}}}\Cendnote{\textnormal{vgl. A. S.: \emph{Tagebuch}, 9. 8. 1895}}}\label{K_L00477_2h}, grüßen
                  Sie ſie von mir.\pend
           \endnumbering\briefempfaengerindex{Beer-Hofmann, Richard@\textsc{Beer-Hofmann, Richard}!zzzSchnitzler, Arthur@\emph{von Arthur Schnitzler}!1895-08-241@{24. 8. 1895}|)be}\mylabel{h}  \normalsize

\doendnotes{C}
\bigskip
\vfill

\clearpage

\footnotesize

\lohead{\textsc{register}}

% Definiere theindex-Environment komplett neu ohne reledmac
\makeatletter
\renewenvironment{theindex}{%
  \section*{\indexname}%
  \setlength{\parindent}{0pt}%
  \setlength{\parskip}{0pt plus 0.3pt}%
  \let\item\@idxitem
}{%
  \clearpage
}
\makeatother

\IfFileExists{\jobname-pw.ind}{\input{\jobname-pw.ind}}{}

\end{document}

      