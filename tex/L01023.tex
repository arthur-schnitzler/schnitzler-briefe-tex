%% latex-korrekturansicht-vorspann.tex
%% Vorspann für die Korrekturansicht.
%% Lädt die gemeinsame Datei latex-vorspann.tex mit gesetztem Schalter.

\newif\ifkorrekturansicht
\korrekturansichttrue

\input{../tex-inputs/latex-vorspann}


               \section[Hugo August von Hofmannsthal an Arthur Schnitzler, 22. 3. 1900]{ Hugo August von Hofmannsthal an Arthur Schnitzler,
                    22. 3. 1900}\nopagebreak\mylabel{v}\rehead{ }\normalsize\beginnumbering\briefempfaengerindex{Schnitzler, Arthur@\textsc{Schnitzler, Arthur}!zzzHofmannsthal, Hugo August von@\emph{von Hugo August von Hofmannsthal}!1900-03-221@{22. 3. 1900}|(be} \toendnotes[C]{\smallbreak\pagebreak[2]} \Standort{DLA, A:Schnitzler, HS.NZ85.1.3483.}
\physDesc{Briefkarte
\newline{}Handschrift: schwarze Tinte, lateinische Kurrent}\toendnotes[C]{\smallbreak}\pstart
           \noindent{}\label{T_L01023_1v}\edtext{{\pb}\textcolor{gray}{\textbf{\textcolor{brown}{Oesterreichische Central-Boden-Credit-Bank
                                Wien}{}\ledrightnote{\textcolor{brown}{Allgemeine Bodencreditanstalt}}.}}}{\lemma{\textnormal{\emph{Oesterreichische … Wien.}}}\Cendnote{\textnormal{quer
                            am linken Rand}}}\label{T_L01023_1h}\pend
           \pstart
           \textcolor{blue}{Hugo}{}\ledrightnote{\textcolor{blue}{Hugo von Hofmannsthal}} bittet Sie die \textcolor{green}{Beilage}{}\ledrightnote{→\textcolor{green}{Vorspiel zur Antigone des Sophokles}} ehethunlichst an D\textsuperscript{r}{ }\textcolor{blue}{Paul Goldmann}{}\ledrightnote{\textcolor{blue}{Paul Goldmann}} gelangen zu laſſen.\pend
           \pstart
           Mit besten Grüßen Ihr ergebenster{\\[\baselineskip]}\spacefill\mbox{D\textsuperscript{r}Hofmannsthal}\pend
           \leftskip=0em{}\pstart
           22/3 1900.\pend
           \endnumbering\briefempfaengerindex{Schnitzler, Arthur@\textsc{Schnitzler, Arthur}!zzzHofmannsthal, Hugo August von@\emph{von Hugo August von Hofmannsthal}!1900-03-221@{22. 3. 1900}|)be}\mylabel{h}  \normalsize

\doendnotes{C}
\bigskip
\vfill

\clearpage

\footnotesize

\lohead{\textsc{register}}

% Definiere theindex-Environment komplett neu ohne reledmac
\makeatletter
\renewenvironment{theindex}{%
  \section*{\indexname}%
  \setlength{\parindent}{0pt}%
  \setlength{\parskip}{0pt plus 0.3pt}%
  \let\item\@idxitem
}{%
  \clearpage
}
\makeatother

\IfFileExists{\jobname-pw.ind}{\input{\jobname-pw.ind}}{}

\end{document}

      