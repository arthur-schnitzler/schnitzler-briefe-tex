\documentclass[twoside=false,titlepage=false,open=any, parskip=never, fontsize=12pt, headings=small, chapterprefix=false, appendixprefix=false]{scrbook}
\addtolength{\oddsidemargin}{\evensidemargin}
\setlength{\oddsidemargin}{.5\oddsidemargin}
\setlength{\evensidemargin}{\oddsidemargin}

\usepackage[{textwidth=13cm,textheight=23cm,marginpar=3cm, left=2cm}]{geometry}
%\usepackage[textwidth=80mm, layoutwidth=170mm, paperheight =297mm, paperwidth  =210mm, layoutvoffset= 20mm,layouthoffset= 20mm]{geometry}
%\usepackage[paperheight =297mm, paperwidth  =210mm, layoutheight=230mm, layoutwidth=158mm, layoutvoffset= 20mm, layouthoffset= 20mm, textwidth=150mm, textheight=185mm, showcrop=false]{geometry}
%sepackage[paperheight=230mm, paperwidth=138mm, textwidth=100mm, textheight=185mm]{geometry}
 \usepackage[usenames, dvipsnames]{xcolor}
\usepackage{scrlayer-scrpage}
\usepackage{hyphenat}
\usepackage{fontspec}
\usepackage{moresize}
\usepackage[english, french, greek, ngerman]{babel}
%\usepackage{ipa}  für das Seitenwechselzeichens
\usepackage[babel]{microtype}
\usepackage[dash, dot]{dashundergaps}
\usepackage{soul}
\usepackage{ragged2e}
\usepackage[makeindex, protected]{splitidx}
\usepackage[itemlayout=abshang,hangindent=0.85em, subindent=0em, subsubindent=1em, justific=RaggedRight, columns=1, columnsep=0pt, indentunit=1em, totoc=false]{idxlayout}
\usepackage{scrhack}
\usepackage{xpatch}
\usepackage{reledmac}
\usepackage{refcount} % Für die Seitenverweise 1–3 etc. 
\usepackage{etoolbox}
\usepackage{framed}
\usepackage[export]{adjustbox} % loads also graphicx, für Bildgröße autom. maximal
\usepackage{float} %ermöglicht exakte Bildpositionierung
\usepackage{mdframed}
\usepackage{enumitem}
\usepackage{relsize}
\usepackage{longtable}
\usepackage{chngcntr} % Sectionnummern durchgehend
\usepackage{hanging} % Für hängende Absätze
\usepackage[rightmargin=0em, leftmargin=1em, indentfirst=false]{quoting} % Für die geänderte quote-Umgebung in den Hrsg-Texten
%\usepackage{fontawesome}
\usepackage{ellipsis}
\RequirePackage{hyphsubst}%
\HyphSubstIfExists{ngerman-x-latest}{\HyphSubstLet{ngerman}{ngerman-x-latest}}{} 
\listfiles
\usepackage[noadjust]{marginnote}

\KOMAoptions{toc=chapterentrydotfill, toc=flat}
\addtokomafont{chapterentrypagenumber}{\mdseries}
\setkomafont{chapterentry}{\normalfont\mdseries}
\setkomafont{partentry}{\normalfont\mdseries}
\RedeclareSectionCommand[tocbeforeskip=0pt]{chapter}

\setlength{\skip\footins}{4mm plus 2mm} % Abstand Fussnote Text
\interfootnotelinepenalty=10000 % Kein Seitenwechsel in Fuss

%\DeclareTextFontCommand{\emph}{\textit}

% Der Befehl erlaubt rechtsbündig bei Unterschriften, die nicht mehr in die Zeile passen
\def\spacefill{\hspace{\fill}\mbox{}\linebreak[0]\hspace*{\fill}}
\usepackage{atbegshi}
\usepackage{zref-abspage}
\usepackage{perpage}
\usepackage{zref-user}
\usepackage{tikz}
\usepackage{ulem}
\usetikzlibrary{calc,decorations.pathmorphing}

\PassOptionsToPackage{gray}{xcolor}
\definecolor{gray}{gray}{0.6}

\doublehyphendemerits=1000000 % das hier verhindert zu viele aufeinanderfolgende Trennstriche am Zeilenende


\usepackage{zref-abspage}
\usepackage{zref-user}
\usepackage{tikz}
\usepackage{atbegshi}
\usepackage{ulem}
\usetikzlibrary{calc,decorations.pathmorphing}

\PassOptionsToPackage{gray}{xcolor}
\definecolor{gray}{gray}{0.6}

\doublehyphendemerits=1000000 % das hier verhindert zu viele aufeinanderfolgende Trennstriche am Zeilenende

\makeatletter
\newcommand{\currentsidemargin}{%
  \ifodd\zref@extract{textarea-\thetextarea}{abspage}%
    \oddsidemargin%
  \else%
    \evensidemargin%
  \fi%
}

\newcounter{textarea}
\newcommand{\settextarea}{%
   \stepcounter{textarea}%
   \zlabel{textarea-\thetextarea}%
   \begin{tikzpicture}[overlay,remember picture]
    % Helper nodes
    \path (current page.north west) ++(\hoffset, -\voffset)
        node[anchor=north west, shape=rectangle, inner sep=0, minimum width=\paperwidth, minimum height=\paperheight]
        (pagearea) {};
    \path (pagearea.north west) ++(1in+\currentsidemargin,-1in-\topmargin-\headheight-\headsep)
        node[anchor=north west, shape=rectangle, inner sep=0, minimum width=\textwidth, minimum height=7pt]
        (textarea) {};
  \end{tikzpicture}%
}

\tikzset{tikzul/.style={yshift=-.75\dp\strutbox}}

\newcounter{tikzul}%
\newcommand\tikzul[1][]{%
    \begingroup
    \global\tikzullinewidth\linewidth
    \def\tikzulsetting{[#1]}%
    \stepcounter{tikzul}%
    \settextarea
    \zlabel{tikzul-begin-\thetikzul}%
    \tikz[overlay,remember picture,tikzul] \coordinate (tikzul-\thetikzul) at (0,0);% Modified \tikzmark macro
    \ifnum\zref@extract{tikzul-begin-\thetikzul}{abspage}=\zref@extract{tikzul-end-\thetikzul}{abspage}
    \else
        \AtBeginShipoutNext{\tikzul@endpage{#1}}%
    \fi
    \bgroup
    \def\par{\ifhmode\unskip\fi\egroup\par\@ifnextchar\noindent{\noindent\tikzul[#1]}{\tikzul[#1]\bgroup}}%
    \aftergroup\endtikzul
    \let\@let@token=%
}
\newlength\tikzullinewidth


\def\tikzul@endpage#1{%
\setbox\AtBeginShipoutBox\hbox{%
\box\AtBeginShipoutBox
\hbox{%
\begin{tikzpicture}[overlay,remember picture,tikzul]
\draw[#1]
    let \p1 = (tikzul-\thetikzul), \p2 = ([xshift=\tikzullinewidth+\@totalleftmargin]textarea.south west) in
    \ifdim\dimexpr\y1-\y2<.5\baselineskip
        (\x1,\y1) -- (\x2,\y1)
    \else
        let \p3 = ([xshift=\@totalleftmargin]textarea.west) in
        (\x1,\y1) -- +(\tikzullinewidth-\x1+\x3,0)
        % (\x3,\y2) -- (\x2,\y2)
        (\x3,\y1)
       \myloop{\y1-\y2+.5\baselineskip}{%
           ++(0,-\baselineskip) -- +(\tikzullinewidth,0)
       }%
    \fi
;
\end{tikzpicture}%
}}%
}%


\def\endtikzul{%
    \zlabel{tikzul-end-\thetikzul}%
    \ifnum\zref@extract{tikzul-begin-\thetikzul}{abspage}=\zref@extract{tikzul-end-\thetikzul}{abspage}
    \begin{tikzpicture}[overlay,remember picture,tikzul]
        \expandafter\draw\tikzulsetting
            let \p1 = (tikzul-\thetikzul), \p2 = (0,0) in
            \ifdim\y1=\y2
                (\x1,\y1) -- (\x2,\y2)
            \else
                let \p3 = ([xshift=\@totalleftmargin]textarea.west), \p4 = ([xshift=-\rightmargin]textarea.east) in
                (\x1,\y1) -- +(\tikzullinewidth-\x1+\x3,0)
                (\x3,\y2) -- (\x2,\y2)
                (\x3,\y1)
                \myloop{\y1-\y2}{%
                    ++(0,-\baselineskip) -- +(\tikzullinewidth,0)
                }%
            \fi
        ;
    \end{tikzpicture}%
    \else
    \settextarea
    \begin{tikzpicture}[overlay,remember picture,tikzul]
        \expandafter\draw\tikzulsetting
            let \p1 = ([xshift=\@totalleftmargin,yshift=-.5\baselineskip]textarea.north west), \p2 = (0,0) in
            \ifdim\dimexpr\y1-\y2<.5\baselineskip
                (\x1,\y2) -- (\x2,\y2)
            \else
                let \p3 = ([xshift=\@totalleftmargin]textarea.west), \p4 = ([xshift=-\rightmargin]textarea.east) in
                (\x3,\y2) -- (\x2,\y2)
                (\x3,\y2)
                \myloop{\y1-\y2}{%
                    ++(0,+\baselineskip) -- +(\tikzullinewidth,0)
                }
            \fi
        ;
    \end{tikzpicture}%
    \fi
    \endgroup
}

% -------------------------------------------------------------- Additions by Peter Grill

\tikzset{tikzst/.style={yshift=0.5\dp\strutbox}}

\newcounter{tikzst}%
\newcommand\tikzst[1][]{%
    \begingroup
    \global\tikzstlinewidth\linewidth
    \def\tikzstsetting{[#1]}%
    \stepcounter{tikzst}%
    \settextarea
    \zlabel{tikzst-begin-\thetikzst}%
    \tikz[overlay,remember picture,tikzst] \coordinate (tikzst-\thetikzst) at (0,0);% Modified \tikzmark macro
    \ifnum\zref@extract{tikzst-begin-\thetikzst}{abspage}=\zref@extract{tikzst-end-\thetikzst}{abspage}
    \else
        \AtBeginShipoutNext{\tikzst@endpage{#1}}%
    \fi
    \bgroup
    \def\par{\ifhmode\unskip\fi\egroup\par\@ifnextchar\noindent{\noindent\tikzst[#1]}{\tikzst[#1]\bgroup}}%
    \aftergroup\endtikzst
    \let\@let@token=%
}
\newlength\tikzstlinewidth


\def\tikzst@endpage#1{%
\setbox\AtBeginShipoutBox\hbox{%
\box\AtBeginShipoutBox
\hbox{%
\begin{tikzpicture}[overlay,remember picture,tikzst]
\draw[#1]
    let \p1 = (tikzst-\thetikzst), \p2 = ([xshift=\tikzstlinewidth+\@totalleftmargin]textarea.south west) in
    \ifdim\dimexpr\y1-\y2<.5\baselineskip
        (\x1,\y1) -- (\x2,\y1)
    \else
        let \p3 = ([xshift=\@totalleftmargin]textarea.west) in
        (\x1,\y1) -- +(\tikzstlinewidth-\x1+\x3,0)
        % (\x3,\y2) -- (\x2,\y2)
        (\x3,\y1)
       \myloop{\y1-\y2+.5\baselineskip}{%
           ++(0,-\baselineskip) -- +(\tikzstlinewidth,0)
       }%
    \fi
;
\end{tikzpicture}%
}}%
}%


\def\endtikzst{%
    \zlabel{tikzst-end-\thetikzst}%
    \ifnum\zref@extract{tikzst-begin-\thetikzst}{abspage}=\zref@extract{tikzst-end-\thetikzst}{abspage}
    \begin{tikzpicture}[overlay,remember picture,tikzst]
        \expandafter\draw\tikzstsetting
            let \p1 = (tikzst-\thetikzst), \p2 = (0,0) in
            \ifdim\y1=\y2
                (\x1,\y1) -- (\x2,\y2)
            \else
                let \p3 = ([xshift=\@totalleftmargin]textarea.west), \p4 = ([xshift=-\rightmargin]textarea.east) in
                (\x1,\y1) -- +(\tikzstlinewidth-\x1+\x3,0)
                (\x3,\y2) -- (\x2,\y2)
                (\x3,\y1)
                \myloop{\y1-\y2}{%
                    ++(0,-\baselineskip) -- +(\tikzstlinewidth,0)
                }%
            \fi
        ;
    \end{tikzpicture}%
    \else
    \settextarea
    \begin{tikzpicture}[overlay,remember picture,tikzst]
        \expandafter\draw\tikzstsetting
            let \p1 = ([xshift=\@totalleftmargin,yshift=-.5\baselineskip]textarea.north west), \p2 = (0,0) in
            \ifdim\dimexpr\y1-\y2<.5\baselineskip
                (\x1,\y2) -- (\x2,\y2)
            \else
                let \p3 = ([xshift=\@totalleftmargin]textarea.west), \p4 = ([xshift=-\rightmargin]textarea.east) in
                (\x3,\y2) -- (\x2,\y2)
                (\x3,\y2)
                \myloop{\y1-\y2}{%
                    ++(0,+\baselineskip) -- +(\tikzstlinewidth,0)
                }
            \fi
        ;
    \end{tikzpicture}%
    \fi
    \endgroup
}
% --------------------------------------------------------------

\def\myloop#1#2#3{%
    #3%
    \ifdim\dimexpr#1>1.1\baselineskip
        #2%
        \expandafter\myloop\expandafter{\the\dimexpr#1-\baselineskip\relax}{#2}%
    \fi
}

\makeatother






\def\myloop#1#2#3{%
    #3%
    \ifdim\dimexpr#1>1.1\baselineskip
        #2%
        \expandafter\myloop\expandafter{\the\dimexpr#1-\baselineskip\relax}{#2}%
    \fi
}

\makeatother
%\newcommand{\damage}[1]{\tikzul[gray,line width=0.15\ht\strutbox,semitransparent]{#1}}
%\newcommand{\strikeout}[1]{\tikzst[black]{#1}}

\newcommand{\damage}[1]{\textcolor{orange}{#1}}
\newcommand{\strikeout}[1]{\sout{#1}}


\setlength{\parindent}{1em}

% Mehr als drei Auslassungspunkte 

\newcommand{\dotsseven}{%
.\kern\ellipsisgap 
.\kern\ellipsisgap
.\kern\ellipsisgap 
.\kern\ellipsisgap
.\kern\ellipsisgap
.\kern\ellipsisgap 
.\kern\ellipsisgap 	
\relax}

\newcommand{\dotssix}{%
.\kern\ellipsisgap 
.\kern\ellipsisgap
.\kern\ellipsisgap
.\kern\ellipsisgap
.\kern\ellipsisgap 
.\kern\ellipsisgap 
\relax}

\newcommand{\dotsfive}{%
.\kern\ellipsisgap 
.\kern\ellipsisgap
.\kern\ellipsisgap
.\kern\ellipsisgap 
.\kern\ellipsisgap 
\relax}

\newcommand{\dotsfour}{%
.\kern\ellipsisgap 
.\kern\ellipsisgap
.\kern\ellipsisgap
.\kern\ellipsisgap 
\relax}

\newcommand{\dotstwo}{%
.\kern\ellipsisgap 
.\kern\ellipsisgap
\relax}


% Silbentrennung
\selectlanguage{ngerman}
\hyphenation{Re-kours EP-STEIN Her-vay-vor-les-ung Steu-er-sa-chen Öst-reich Burck-hard Keuch-hus-ten Oedi-pus-auf-führ-un-gen Hi-obs-post Kärnt-ner-ring Vei-tlis-sen-gas-se Franck-gas-se Rath-hau-se Sechs-schg Stu-bai-thal Tha-deusz Volks-th Halb-mo-nats-schrift JAHR-ES-ZEI-TEN Te-le-phon mit-ge-theilt Ge-schäfts-ver-bin-dung hoch-müth-ig Ueber-zeu-gung bis-chen Au-tor-rech-te Hof-manns-thal Nor-deijk Irre-seins Tschap-perl mit-zu-thei-len Aeu-ße-rung be-thö-ren Kü-ni-gel Be-ur-thei-lung Kuenst-lern ko-moe-di-sche hae-mor-rha-gi-scher Doer-mann Wash-burn flei-ssig haute Buddh-ist Preu-ssen Lin-den-café Mit-theil-un-gen An-theil Lieu-te-nant oes-terr Rieg-ner Oes-ter-reich gro-ssem Fran-zo-sen-thum Roche Lili Ent-schlie-ssun-gen äu-ssert wuen-sche Trans-ac-tio-nen Ue-ber-win-dung Eu-gene Stra-ssen-dir-ne qua-tre Deutsch-öst-er-reich Deutsch-öst-er-reichs Bjørn-stjer-ne noth-ing Edit-ed Olga Ar-naud Mer-gent-heim Léon-tine Polla-czek Brion Barre Hoch-sin-ger Ka-tha-rina Arouet Va-len-ci-ennes Ueber-win-dung Type-writer-in Tolstoi-buch Schnitzler Copier-buche Schiller Intel-lek-tuell-en-as-so-zi-a-tion Salten Devrient Grien-steidl Ge-sell-ſchaft ein-ge-ſchloſ-ſen Fort-ſetz-un-gen Bor-dell-ſtück fort-ſchrei-ten wirk-ſam-es ſchrift-ſtel-ler-i-ſchen hin-weg-ſe-hen Gerichts-saal-be-richt-er-ſtat-ter}



% Sonderbefehl für .–
\def\dotdash{\nobreak\hspace{0pt}.–}  %ACHTUNG BEIM ERSETZEN: LEERZEICHEN DANACH 
\def\commadash{\nobreak\hspace{0pt},–}
\def\excdash{\nobreak\hspace{0pt}!–}
\def\semicolondash{\nobreak\hspace{0pt};–}
\def\parentdotdash{\nobreak\hspace{0pt}).–}
\def\slashislash{\,\slash\,\allowbreak\hspace{0pt}}

\newcommand{\strich}{\makebox[1em][l]{– }}


% Seite einrichten

% Farbe definieren
%\setmainfont[RawFeature={-liga}, 
%SmallCapsFont=WSVgara-Caps, 
%ItalicFont=WSVgara-Italic, 
%BoldFont=WSVgara-Bold,
%BoldItalicFont=WSVgara-BoldItalic
%]{WSVgara}
%\setsansfont[RawFeature={-liga}, 
%SmallCapsFont=WSVgara-Caps, 
%ItalicFont=WSVgara-Italic, 
%BoldFont=WSVgara-Bold,
%BoldItalicFont=WSVgara-BoldItalic
%]{WSVgara}

%\setmainfont{Brill}
%\setsansfont{Brill}

%\setmainfont[ItalicFont=SinaNova-Italic, 
%BoldFont=SinaNova-Bold,
%BoldItalicFont=SinaNova-BoldItalic
%]{SinaNova-Regular}
%\setsansfont[ItalicFont=SinaNova-Italic, 
%BoldFont=SinaNova-Bold,
%BoldItalicFont=SinaNova-BoldItalic
%]{SinaNova-Regular}



\def\labelitemi{--}

% Geminationsstrich, U-Strich

 \newcommand{\overbar}[1]{$\overline{\hbox{#1}}$}


% Ausrufezeichen in den Index kriegen
\newcommand{\rufezeichen}{"!}

% Griechisch
	
%\newfontfamily\greekfont{GaramondPremrPro}
%\newcommand\griechisch[1]{\greekfont{}#1{}\normalfont}

%\newfontfamily\sansseriffont[HyphenChar=None, RawFeature={-liga}, Scale=1.03]{TheSans-Regular}
%\newfontfamily\sansseriffont{uarial}


%\newfontfamily\sansseriffont[HyphenChar=None, LetterSpace=1.0, RawFeature={-liga}]{TheSans-SemiBold}
%\newcommand\sansseriff[1]{\sffamily{}#1{}\normalfont}

\newcommand{\mini}{\,}


\newcommand{\key}{\textsuperscript{\textcolor{red}{KEY}}}


%% Sperrung (Package Soul)
%% Hier ist die Sperrung definiert. Sperrung erreicht man mit \so{gesperrtes Wort}
\sodef\so{}{.14em}{.4em plus.1em minus .1em}{.4em plus.1em minus .1em}

% SCHRIFTEN
\setkomafont{disposition}{}
\addtokomafont{caption}{\small}
\addtokomafont{captionlabel}{\small}

%% Schrift der Kopf und Fußzeile
\renewcommand*{\headfont}{\normalfont}
\setkomafont{pagehead}{\footnotesize\addfontfeature{LetterSpace=10.0}}
\setkomafont{pagenumber}{\normalfont\normalsize}
\ohead[]{\pagemark}% Seitenzahl (c = centered) 
\ofoot[]{}


 
% Flatterndes Seitenende
\raggedbottom

% Fussnoten neu Anfangen

\makeatletter
\pretocmd{\@schapter}{\setcounter{footnote}{0}}{}{}
\pretocmd{\@chapter}{\setcounter{footnote}{0}}{}{}
\pretocmd{\@section}{\setcounter{footnote}{0}}{}{}
\makeatother


% Section Nummern durchgehend

\RedeclareSectionCommand[
  counterwithout=chapter
]{section}

% Section Punkt

\renewcommand*{\sectionformat}{}
\renewcommand*{\partformat}{}


% Marginpar Schrift

\newkomafont{margin}{\footnotesize} 
\makeatletter 
\let\MarginParOriginal\marginpar 
\renewcommand*{\marginpar}{\@dblarg\@marginpar} 
\newcommand{\@marginpar}[2][]{% 
  \MarginParOriginal[\usekomafont{margin}{#1\par}]{\usekomafont{margin}{#2\par}} 
} 
\makeatother 



\let\oldbeginnumbering\beginnumbering

\def\beginnumbering{\oldbeginnumbering\par\nopagebreak}


% Fußnoten linksbündig
\deffootnote{1.5em}{1em}{% 
\makebox[1.5em][l]{\thefootnotemark}%
}


% Fussnotenlineal (wobei für reledmac wohl was anderes gilt)
\let\normalfootnoterule\footnoterule
\setfootnoterule{0pt}
\let\normalfootnoterule\footnoterule


\setlength{\skip\footins}{8mm plus 2mm} % Abstand Fussnote Text
\interfootnotelinepenalty=10000 % Kein Seitenwechsel in Fuss

%% Kapitelüberschriften
\renewcommand*{\raggedchapter}{\centering} 
\renewcommand*{\raggedsection}{%
 \CenteringLeftskip=1cm plus 1em\relax 
 \CenteringRightskip=1cm plus 1em\relax 
 \Centering\footnotesize\thesection{}.\ }
\setkomafont{section}{\footnotesize}
\setkomafont{chapter}{\normalfont\Large}
\renewcommand{\chapterpagestyle}{empty}%The first page in each chapter won't have any heading or footer, especially no page number

% section ohne führende Kapitelnummer
\renewcommand*\thesection{\arabic{section}}

% Bildunterschrift ohne Nummer
\renewcommand*{\figureformat}{}
\renewcommand*{\captionformat}{}

% Abstand Bild
\setlength{\textfloatsep}{\baselineskip}

%% Zeilennummern
\firstlinenum{0} \linenumincrement{5}
\lineation{section} %Jeder Abschnitt wird durchnummeriert
\renewcommand{\numlabfont}{\ssmall} %Schriftgröße Zeilennummern

%\AtBeginEnvironment{multicols}{\RaggedRight} % Linksbündig in Spalten


% SEITENUMBRÜCHE IM TEXT MARKIEREN

%% Seitenumbrüche


\newcommand{\Theight}{\dimexpr\fontcharht\font`W}
\newcommand{\pbposition}{\depth}
\newcommand{\pb}{\nobreak\hspace{0pt}\raisebox{-0.1em}{\raisebox{\pbposition}{\textnormal{|}}}\nobreak\hspace{0pt}}

% EINFÜGUNGEN IM TEXT MARKIEREN

\renewcaptionname{ngerman}{\contentsname}{Inhalt}           %Table of contents


\newcommand{\introOben}{\textnormal{\raisebox{\Theight}{\raisebox{-\height}{\small\downp\normalsize}}}}
\newcommand{\introUnten}{\textnormal{\raisebox{\Theight}{\raisebox{-\height}{\small\downp\normalsize}}}}
\newcommand{\introMitteVorne}{\textnormal{\raisebox{\Theight}{\raisebox{-\height}{\small\downp\normalsize}}}}
\newcommand{\introMitteHinten}{\textnormal{\raisebox{\Theight}{\raisebox{-\height}{\small\downp\normalsize}}}}
\newcommand{\substVorne}{\textnormal{\raisebox{\Theight}{\raisebox{-\height}{\small\upp\normalsize}}}}
\newcommand{\substDazwischen}{}
\newcommand{\substHinten}{\textnormal{\raisebox{\Theight}{\raisebox{-\height}{\small\downp\normalsize}}}}


% MARGINALSPALTE
\setlength\ledrsnotewidth{1.5cm}


% FUSSNOTE
%% Im Apparat f. und ff.
\Xtwolines{f.}
\Xtwolinesbutnotmore

%% Sperrungen bei Lemmas im Apparat
%\pretocmd{\so}{\null}{}{}
% Hab ich auskommentiert: Hat einen Fehler ergeben, denn plötzlich war ein Abstand vor Absätzen, die mit einer Sperrung beginnen

%% Zeilennummerierung Abstand zum Lemma
\Xboxlinenum{5mm}

%% Bei zwei Apparateinträgen in einer Zeile wird nur beim ersten Mal die Zeile gezählt
\Xnumberonlyfirstinline
\Xnumberonlyfirstintwolines
\Xinplaceofnumber{1em}
\Xhangindent{1em}

% ENDNOTEN
\Xendlemmadisablefontselection[A]
\renewcommand*{\printnpnum}[1]{{\noindent}\tiny}
\Xendparagraph[A] % Endnoten in einem Absatz
%\Xendtwolines{\tiny{f.}}
\Xendbeforepagenumber{} 
\Xendnotenumfont[A]{\tiny}
\Xendboxlinenum[A]{0em}
\Xendlemmaseparator{$\rbracket$}
\Xendnotefontsize[A]{\footnotesize}
\Xendhangindent[A]{1em}
\Xendlemmafont[A]{\itshape}
\Xendlemmafont[B]{\bfseries}
\Xendnotefontsize[B]{\footnotesize}
\Xendnotenumfont{\footnotesize}
\Xendlineprefixsingle[C]{\tiny}
\Xendlineprefixmore[C]{\tiny}
\Xendlemmadisablefontselection
\Xendlemmafont{\itshape}
\Xendlinerangeseparator{\tiny{--}}
\Xendhangindent{4em}
\Xendboxlinenum{3.6em}
\Xendafternumber{0.4em}
\Xendboxlinenumalign{R}

%\Xendboxstartlinenum{3.5em}
%\Xendboxendlinenum{1em}


%% Kaufmanns-Und (=)
            
            

\newcommand{\kaufmannsund}{\&} 

%% Tabelle Zellensprung
% Ein weiterer Anlass, das Kaufmannsund in der Übergabe zu vermeiden:

\newcommand{\zellensprung}{ \& }

%% INDEX
    
    \makeindex 
    \newcommand*\lettergroup[1]{}
    
        \newcommand{\pw}[1]{#1}
        \newcommand{\pwt}[1]{\textbf{#1}}
        \newcommand{\pws}[1]{\upshape{\textbf{#1}}}
            
        \newcommand{\pwe}[1]{\textbf{\emph{#1}}}
             
    \newcommand{\pwk}[1]{#1\textsuperscript{\tiny{K}}}
    \newcommand{\pwv}[1]{\emph{#1}}
     \newcommand{\pwkv}[1]{\emph{#1}\textsuperscript{\tiny{K}}}
               \newcommand{\pwuv}[1]{\emph{#1}?}
               \newcommand{\pwu}[1]{#1?}
 \newcommand{\range}[2]{{\def\pw##1{##1}#1}--#2}

\newcommand{\buch}[1]{#1}


%% MEHRERE INDIZES

\newindex[Register]{pw}
%\newindex[Institutionen Organisationen Periodika und Unternehmen]{org}
%\newindex[Institutionen und Orte]{o}
\newindex[Korrespondenzpartner]{briefe-out}
\newindex[Gedruckte Quellen]{buch-abdruck}

\newcommand\briefsenderindex[1]{\sindex[briefe-out]{#1}}
\newcommand\briefempfaengerindex[1]{\sindex[briefe-out]{#1}}

\newcommand\buchabdruck[1]{\sindex[buch-abdruck]{#1}}
\renewcommand\buchabdruck[1]{}



%% Symbole

%\newcommand{\symaddr}{\includegraphics[height=6pt]{symbol/noun_637366.png}}
%\newcommand{\symweiteredrucke}{\includegraphics[height=6pt]{symbol/noun_634729.png}}
%\newcommand{\symdruckvorlage}{\includegraphics[height=6pt]{symbol/noun_637409.png}}
%\newcommand{\symstandort}{\includegraphics[height=6pt]{symbol/noun_634216.png}}
%\newcommand{\symhead}{\includegraphics[height=6pt]{symbol/noun_1162030_cc.png}}


\newcommand{\symaddr}{A}
\newcommand{\symweiteredrucke}{D}
\newcommand{\symdruckvorlage}{V}
\newcommand{\symstandort}{O}
\newcommand{\symhead}{H}



\newcommand\anhangTitel[2]{\toendnotes[C]{\hangpara{4em}{1}{\makebox[4em][l]{\textbf{#1}}\textbf{#2}}\endgraf}}
\newcommand\Adresse[1]{\toendnotes[C]{\hangpara{4em}{1}{\makebox[4em][l]{\makebox[3.6em][r]{\symaddr}}}#1\endgraf}}

\newcommand\buchAlsQuelle[1]{\toendnotes[C]{\footnotesize\par\hangpara{4em}{1}{\makebox[4em][l]{\makebox[3.6em][r]{\symdruckvorlage}}}#1\endgraf}}
\newcommand\buchAbdrucke[1]{\toendnotes[C]{\footnotesize\par\hangpara{4em}{1}{\makebox[4em][l]{\makebox[3.6em][r]{\symweiteredrucke}}}#1\endgraf}}
\newcommand\Standort[1]{\toendnotes[C]{\footnotesize\hangpara{4em}{1}{\makebox[4em][l]{\makebox[3.6em][r]{\symstandort}}}#1\endgraf}}
\newcommand\biographical[1]{\toendnotes[C]{\footnotesize\hangpara{4em}{1}{\makebox[4em][l]{\makebox[3.6em][r]{\symhead}}}#1\endgraf}}
\newcommand\biographicalOhne[1]{\toendnotes[C]{\footnotesize\hangpara{4em}{1}{\makebox[4em][l]{\makebox[3.6em][r]{}}}#1\endgraf}}



\newcommand\datumImAnhang[1]{\toendnotes[C]{#1}}

\let\newcell&

\newcommand\physDesc[1]{\toendnotes[C]{\hangpara{4em}{0}#1\endgraf}}
\newcommand\weitereDrucke[1]{#1}


% Schnitzler Tagebuch Auszüge
\newcommand{\prgrph}[1]{\endgraf\medskip\noindent\textbf{#1}\newline}


%% VERWEISE
% Dieser Befehl vom Typ
% \verweis{FW_V_schwn_A}{FW_V_schwn_E} 
% dient den Verweisen auf den Text von Kommentar und Herausgebereingriffen. Ihm werden die Namen der beiden Labels – Anfang und Ende – übergeben und er setzt den Anfang und entscheidet ob f. oder ff. folgt 


\newcounter{mystart}
\newcounter{mystop}
\newcounter{phantom}

\newcommand*\myrangeref[2]{%
  \setcounterpageref{mystart}{#1}%
  \setcounterpageref{mystop}{#2}%
  \ifnum\value{mystop}<\value{mystart}%
    \typeout{[myrangeref] Strange...stop (#2) before start (#1).}%
    \pageref{#2}--\pageref{#1}%
  \else
    \pageref{#1}%
    \ifnum\value{mystart}<\value{mystop}%
      \addtocounter{mystop}{-1}%
      \ifnum\value{mystart}<\value{mystop}%
        \,ff.
        %--\pageref{#2}%%
      \else
        \,f.
         %%--\pageref{#2}%
              \fi
    \fi
  \fi
}
            
\newcommand*\myrangerefkasten[2]{%
  \setcounterpageref{mystart}{#1}%
  \setcounterpageref{mystop}{#2}%
  \ifnum\value{mystop}<\value{mystart}%
    \typeout{[myrangeref] Strange...stop (#2) before start (#1).}%
    \pageref{#2}--\pageref{#1}%
  \else
    \makebox[12pt][r]{\pageref{#1}}%
    \ifnum\value{mystart}<\value{mystop}%
      \addtocounter{mystop}{-1}%
      \ifnum\value{mystart}<\value{mystop}%
        --\pageref{#2}%%
      \else
         --\pageref{#2}%
         % alternativ hierher: f.
      \fi
    \fi
  \fi
}


\newcommand*\mylabel[1]{%
  \refstepcounter{phantom}%
  \label{#1}%
}

\newenvironment{anhang}{\vspace{1cm}
}{}

\emfontdeclare{\itshape}

%% RAHMEN SEITLICH

\newlength{\leftbarwidth}
\setlength{\leftbarwidth}{3pt}
\newlength{\leftbarsep}
\setlength{\leftbarsep}{10pt}

\renewenvironment{leftbar}[1][\hsize]
{% 
\def\FrameCommand 
{%
{\hspace{-7pt} \color{black} \vrule width 0.5pt}%
\hspace{0pt}%must no space.
\fboxsep=\FrameSep\colorbox{white}%
}%
\MakeFramed{\hsize#1\advance\hsize-\width\FrameRestore}%
}
{\endMakeFramed}
\setlength{\FrameSep}{5pt}

\newmdenv[topline=false, leftline=true, rightline=true, bottomline=false,%
  linewidth=0.5pt, leftmargin=30pt, rightmargin=30pt, %
  skipabove=8pt, skipbelow=8pt]{mdbar}

% Überstreichung (OVERLINE)

\makeatletter
\newcommand*{\textoverline}[1]{$\overline{\hbox{#1}}\m@th$}
\makeatother

% Rahmen für Hintergrundfarbe
\fboxsep0mm

% Befehl für gekürzte Texte

\newcommand{\kuerzung}{, Auszug}

% Verse 

\setlength{\stanzaindentbase}{20pt} %Play with it later.
\setstanzaindents{5,1,1}
\setcounter{stanzaindentsrepetition}{2}
\newcommand{\stanzaend}{\&}
\sethangingsymbol{\protect\hfill}
\AtEveryStopStanza{\vspace{0.25\baselineskip}} %Abstand zwischen Strophen


% Versuch eines Grid

\RedeclareSectionCommand[
  beforeskip=3\baselineskip,
  afterskip=\baselineskip
]{chapter}
\RedeclareSectionCommand[
  beforeskip=2\baselineskip,
  afterskip=\baselineskip
]{section}

\newcommand\adjacent[2][]{%
  \bgroup
  \RedeclareSectionCommand[
    beforeskip=2\baselineskip,
    afterskip=\baselineskip,
  ]{chapter}%
  \if\relax\detokenize{#1}\relax
    \addchap{#2}%
  \else
    \addchap[#1]{#2}%
  \fi
  \egroup
  \section
}


%change the part format in table of contents
\renewcaptionname{ngerman}{\contentsname}{Inhalt} 


% Inhaltsverzeichnis

\AtBeginDocument{%
  \addtocontents{toc}{\protect\label{toc}}%
}

\renewcaptionname{ngerman}{\contentsname}{Verzeichnis der Dokumente} 
 
 
   \DeclareTOCStyleEntry[
  beforeskip=15pt,
  entryformat=\normalsize\normalfont\centering,
  pagenumberformat=\nullfont,
  linefill={},
  raggedentrytext=true
]{part}{part}

  \DeclareTOCStyleEntry[
  beforeskip=5pt,
  entryformat=\normalsize\normalfont\centering,
  pagenumberformat=\nullfont,
  linefill={},
  raggedentrytext=true
]{chapter}{chapter}

\DeclareTOCStyleEntry[
  onstarthigherlevel=\vspace*{0.5\baselineskip}\nobreak,
  indent=0pt,
  entryformat=\normalsize\def\autodot{.},
  pagenumberformat=\normalsize,
  raggedentrytext=true
]{section}{section}



 
% Das folgende auskommentiert, funktionierte nicht mehr, ging aber in Bahr/Schnitzler. Sollte eigentlich dazu dienen, beim Inhaltsverzeichnis die Nummern rechtsbündig zu setzen

 \iffalse
 
  \DeclareTOCStyleEntry[
  beforeskip=5pt,
  entryformat=\normalsize\normalfont\centering,
  pagenumberformat=\nullfont,
  linefill={},
  raggedentrytext=true
]{chapter}{chapter}

\DeclareTOCStyleEntry[
  onstarthigherlevel=\vspace*{0.5\baselineskip}\nobreak,
  indent=0pt,
  entryformat=\normalsize\def\autodot{.},
  pagenumberformat=\normalsize,
  raggedentrytext=true
]{section}{section}
 
 
  \newcommand*\sectionnumberbox[1]{\hfill #1\hspace{.6em}}

\newlength{\zweiziffern}
\newlength{\dreiziffern}
\newlength{\vierziffern}
\settowidth{\zweiziffern}{9999}
\settowidth{\dreiziffern}{99999}
\settowidth{\vierziffern}{99999999}
 
\BeforeStartingTOC[toc]{\value{tocdepth}=\sectiontocdepth}


\DeclareTOCStyleEntry[
  onstarthigherlevel=\vspace*{0.5\baselineskip}\nobreak,
  indent=0pt,
  entryformat=\normalsize\def\autodot{.},
  entrynumberformat=\sectionnumberbox,
  pagenumberformat=\normalsize,
  numwidth=\zweiziffern,
  raggedentrytext=true
]{section}{section}

\newcommand{\toccheck}{\ifnum \value{section}=76 \addtocontents{toc}{\protect\DeclareTOCStyleEntry[numwidth=\dreiziffern]{section}{section}} \else \ifnum \value{section}=990 \addtocontents{toc}{\protect\DeclareTOCStyleEntry[numwidth=\vierziffern]{section}{section}} \fi \fi}
\fi



% Längen für Tabellen
\newlength{\longeste}
\newlength{\longestz}
\newlength{\longestd}
\newlength{\longestv}
\newlength{\longestf}

\newcommand\halbtextwidth{0.9\textwidth}

\newcommand\pwindex[1]{{\sindex[pw]{#1}}}
\newcommand\oindex[1]{{\sindex[pw]{#1}}}
\newcommand\orgindex[1]{{\sindex[pw]{#1}}}

\renewcommand\oindex[1]{{{\sindex[pw]{#1}}}}
\renewcommand\orgindex[1]{{{\sindex[pw]{#1}}}}



% INDEX

%\renewcommand\pwindex[1]{}
%\renewcommand\oindex[1]{}
%\renewcommand\orgindex[1]{}
%\renewcommand\buchabdruck[1]{}


\newcommand\url[1]{\mbox{#1}}
\renewcommand\ngermanhyphenmins{33}

\makeatletter
\newcommand*{\geminationm}{$\overline{\hbox{m}}\m@th$}
\newcommand*{\geminationn}{$\overline{\hbox{n}}\m@th$}
\makeatother

%part
\renewcommand{\partmarkformat}{}
\renewcommand{\partheadmidvskip}{\enskip}
\renewcommand{\partformat}{}
\setkomafont{partnumber}{\usekomafont{part}}


%\geometry{headsep=8pt} % Abstand Kopfzeile - Text
%% DOKUMENT

\begin{document}

% Section ohne Nummer
\renewcommand*{\raggedsection}{%
 \CenteringLeftskip=1cm plus 1em\relax 
 \CenteringRightskip=1cm plus 1em\relax 
 \Centering\normalsize}



\widowpenalty=10000         % avoid widows
\clubpenalty=10000          % avoid orphans

\sloppy
\setlength{\parindent}{0em}

\setlength{\ledlsnotewidth}{4cm}
\setlength{\ledrsnotewidth}{4cm}
\renewcommand*{\ledlsnotefontsetup}{\scriptsize\sffamily}% left
\renewcommand*{\ledrsnotefontsetup}{\scriptsize\sffamily}% left
\thispagestyle{empty} 

               \section[Paul Goldmann an Arthur Schnitzler, 24. 12. {[}1892{]}]{ Paul Goldmann an Arthur Schnitzler, 24. 12. {[}1892{]}}\nopagebreak\mylabel{v}\rehead{ }\normalsize\beginnumbering\briefempfaengerindex{Schnitzler, Arthur@\textsc{Schnitzler, Arthur}!zzzGoldmann, Paul@\emph{von Paul Goldmann}!1892-12-241@{24. 12. {[}1892{]}}|(be} \toendnotes[C]{\smallbreak\pagebreak[2]} \Standort{DLA, A:Schnitzler, HS.NZ85.1.3163.}
\physDesc{Brief, 6 Blätter, 22 Seiten
\newline{}Handschrift: schwarze Tinte, deutsche Kurrent
\newline{}Schnitzler: mit Bleistift das Jahr »92« vermerkt }\toendnotes[C]{\smallbreak}\pstart
           \centering{}{\pb}\textsc{\textcolor{pink}{Paris}{}\ledrightnote{\textcolor{pink}{Paris}}}, 24. December.\pend
           \pstart
           Alſo Weihnachtsabend. Aber nicht ſentimental,
               beileibe. Das thun wir hier nicht, das hält auf, das iſt reacſionär. Wir wollen
               vorwärts. Und darum müſſen wir ſtark werden. Was für einen ſchwachen Menſchen wohl
               nur ſoviel bedeutet, daß er daran vergißt, daß er eigentlich ſchwach iſt.\pend
           \pstart
           Mein theurer Freund! Es iſt Weihnachtsabend, und ich
               hätte \strikeout{\textcolor{gray}{g}} unter keinen Umſtänden Zeit, Dir zu ſchreiben, wenn ich nicht die \textsc{Chance} gehabt hätte, {\pb}vorgeſtern beim Herunterſteigen von der Tramway zu
               ſtürzen und mir die linke Schulter auszurenken. Man nennt das hier eine \textsc{\begin{otherlanguage}{french}luxation de l’épaule\end{otherlanguage}}, renkt das gewohnheitsmäßig falſch ein, renkt das dann wieder aus – \textsc{\begin{otherlanguage}{french}remettre\end{otherlanguage} und \begin{otherlanguage}{french}démettre\end{otherlanguage}} – und conſtatirt jedesmal, daß eine neue Gelenkkapſel oder Gelenkband – ich
               weiß nicht, wie das Zeug auf deutſch heißt – zerriſſen iſt. Der Tag geht für den
               Patienten unter dieſen Umſtänden nicht ohne heitere Zerſtreuungen vor{\pb}über. \label{K_L02704-1v}\edtext{\textsc{\begin{otherlanguage}{french}Mais, enfin\end{otherlanguage}}}{\lemma{\textnormal{\emph{Mais, enfin}}}\Cendnote{\textnormal{französisch: aber letztendlich}}}\label{K_L02704-1h}
               ich bin genöthigt, für einige Tage meinen Dienſt einzuſtellen – wenn nicht die
               Kurpfuſcher, in deren Händen ich hier bin, einige Wochen daraus machen – und vor
               Allem, ich ſitze heut{ }Abends müßig zuhauſe. Habe ich alſo geſucht, an der Sache eine gute
               Seite zu finden, habe eine ſehr künſtliche Inſtallation auf meinem Schreibtiſch
               gemacht, um das Papier ſeſthalten zu können, und habe mich dann niedergeſetzt, um {\pb}endlich einmal wieder mit Dir, Liebſter, zu plaudern.
               Und ſiehe da, es geht.\pend
           \pstart
           Ich ſehe zu meiner großen Herzenserleichterung – habe mir wirklich viel Sorge darüber
               gemacht – daß Du mir nicht bös biſt, weil ich Dir nicht antworte. Aber, weiß Gott, es
               geht nicht! Das Leben, das wir in dieſer böſen Zeit zu führen gezwungen ſind, iſt
               einfach unmenſchlich. Der Dienſt verſchlingt Alles, Eſſenszeit, Schlafenszeit, und
               nun gar erſt die {\pb}Zeit zum freundſchaftlichen
               Briefwechſel. An Dich gedacht? Oh, mein lieber Freund, wie oft, wie oft! Mitten im
               Sturm der \label{T_L02704-1v}\edtext{Eindrücke}{\lemma{\textnormal{\emph{Eindrücke}}}\Cendnote{\textnormal{Goldmann schreibt
                  »Eindrücken«}}}\label{T_L02704-1h}, mitten im feinem Kunſtgenuß, wo ich immer gar
               ſo gern mit Dir getheilt hätte. Und beſonders auch in dieſen Stunden der
               verzweifelten Verlaſſenheit und Lebensmüdigkeit, wo ich mich nach Dir geſehnt, als
               nach einem \uline{Menſchen}! Denn das gibt es hier nun wohl
               gar nicht. Ich habe immer den gleich ſtarken Wunſch, Dich {\pb}wiederzuſehen. Aber ich würde mich anderſeits doch
               davor fürchten; denn einmal habe ich Sorge davor, du würdeſt mich in Vielem verändert
               und nicht mehr ſo mit Dir zuſammenſtimmend finden; und dann fürchte ich, ich würde
               die Verlaſſenheit wieder ſchwerer ertragen und würde wieder arg mit meiner \textcolor{pink}{Wien}{}\ledrightnote{\textcolor{pink}{Paris}}-Sehnſucht zu ringen haben, die eine Form
               meiner Sentimentalität iſt, will ſagen meines Nichtvorwärtskommens, will ſagen \textsc{etc.} ſiehe oben. {\pb}Aber
               Eines begreiſe ich doch nicht: Ganz abgeſehen von dem zwiſchen mir und Dir. Sag’ mir:
               warum kommſt Du nicht nach \textsc{\textcolor{pink}{Paris}{}\ledrightnote{\textcolor{pink}{Paris}}}? Und zwar auf lange? Um jeden Preis? Glaub’ mir – ich ſehe es jetzt ſo
               deutlich, wie nur irgend etwas auf der Welt – es iſt für Deine ganze Entwickelung
               einfach unentbehrlich. Es wird Dir ekelhaft, abſcheulich, unerträglich ſein. Aber Du
               weißt ja, daß das die {\pb}Formen ſind, in denen die
               Entwickelungs-Kriſis aufzutreten pflegt. Und Du würdeſt hier eine ſolche Fülle neuer
               Ideen, – würdeſt ſo gewaltige \textsc{\begin{otherlanguage}{french}Chocs\end{otherlanguage}} bekommen – daß Du \strikeout{\textcolor{gray}{von}} am Ende wie ein neuer Menſch daſtehen und mit ganz anderen Augen ſehen
               würdeſt. Specieller: Das Leben in \textsc{\textcolor{pink}{Paris}{}\ledrightnote{\textcolor{pink}{Paris}}} entſubjectivirt, es objectivirt – und Du biſt unter allen Umſtänden
               verpflichtet, es auch damit zu verſuchen{[}.{]}{\pb}Alſo komm’ her, mein lieber \textsc{Arthur}, – nicht meinetwegen. Ich würde Dich vielleicht alle drei Wochen
               einmal ſehen können, um Dich zu bitten, daß Du mir ein Nachtmahl zahlſt. Aber
               Deinetwegen! Folge mir! Du wirſt es nicht zu bereuen haben. Das heißt, Du wirſt es
               furchtbar bereuen. Aber es wird Dir ganz enorm geſund ſein.\pend
           \pstart
           Woraus Du nicht etwa ſchließen darfſt, daß ich mich hier wohl fühle. {\pb}Im Gegentheil! Entſetzlich elend. Heimathlos,
               verſtoßen, zuſchanden gearbeitet, angewidert, unbefriedigt \textsc{etc}. Aber eine große Compenſation dafür iſt da: Ich fühle, daß ich \uline{lerne}. Und ſolange das Gefühl anhält, will ich es
               muthig hier aushalten. Vom eigentlichen Lebensziel freilich ferner als je. Keine
               Selbſtändigkeit zu erblicken – kein Erwerb, kein Vermögen. Tagelohn und Schulden. {\pb}Keinen Weg zu den 12000 \textsc{frcs} Rente, die ich brauche. Weißt Du mir vielleicht einen? Dann komme ich
               gleich wieder, und dann bleiben und ſchaffen wir mitſammen. Oder irgend eine ſicher
               nicht-journaliſtiſche Stellung? Wenn Dir ſo etwas unter die Augen komm, denk’ bitte
               an mich! {\dotsfour}\pend
           \pstart
           Und nun Du. Vielen Dank für die Kritiken. Werth hat nur \label{K_L02704-3v}\edtext{die von Dr. \textsc{\textcolor{blue}{Meyer}{}\ledrightnote{\textcolor{blue}{F. Meyer}}}}{\lemma{\textnormal{\emph{die von Dr. Meyer}}}\Cendnote{\textnormal{\textcolor{blue}{f. m.} [=\textcolor{blue}{F. Meyer}]: \textcolor{green}{[Mit unſerer öſterreichiſchen Literatur]}. In: \emph{\textcolor{green}{Berliner Neueste Nachrichten}}, Jg. 12,
                     Nr. 563, 6. 11. 1892, S. [3].}}}\label{K_L02704-3h}. Es
               erhöht meinen {\pb}Reſpekt vor dem \textcolor{blue}{Mann}{}\ledrightnote{→\textcolor{blue}{F. Meyer}} beträchtlich, daß er einem \textcolor{blue}{Freunde}{}\ledrightnote{→\textcolor{blue}{Jakob Julius David}} ſo derb ſeine \label{K_L02704-4v}\edtext{Meinung}{\lemma{\textnormal{\emph{Meinung}}}\Cendnote{\textnormal{\textcolor{blue}{Meyer} kritisierte in seinem kurzen \textcolor{green}{Absatz} zur österreichischen
                  Literatur auch \textcolor{blue}{Jakob Julius David}
                  beziehungsweise dessen Stil (»Seine Probleme und Charaktere ſind einfach,
                     ſeine Sprache iſt knapp und alterthümelnd.«) vor der besprochenen \textcolor{green}{Erzählsammlung}{ }\emph{\textcolor{green}{Probleme}}.}}}\label{K_L02704-4h} ſagt. Er hat zwar in der
               Sache meiner Anſicht nach Unrecht, aber als Offenheit iſt es werthzuſchätzen. Alle
               übrigen verſtehen Dich nicht, außer etwa 
               \textsc{\label{K_L02704-5v}\edtext{\textcolor{blue}{Ludassy}{}\ledrightnote{\textcolor{blue}{Julius von Gans-Ludassy}}}{\lemma{\textnormal{\emph{Ludassy}}}\Cendnote{\textnormal{\textcolor{blue}{Julius von Gans-Ludassy}: \emph{\textcolor{green}{Bücher}}. In: \emph{\textcolor{green}{Fremden-Blatt}}, Jg. 46, Nr. 351, 19. 12. 1892, S. XXXX.}}}\label{K_L02704-5h}}. \label{K_L02704-6v}\edtext{\textsc{\textcolor{blue}{Bauer}{}\ledrightnote{\textcolor{blue}{Julius Bauer}}}}{\lemma{\textnormal{\emph{Bauer}}}\Cendnote{\textnormal{[\textcolor{blue}{Julius Bauer}]: \emph{\textcolor{green}{Oeſterreichiſche Autoren}}. In: \emph{\textcolor{green}{Neues Wiener Abendblatt}}, Jg. 26, Nr. 351, 19. 12. 1892, S. 3–4, hier: S. 3.}}}\label{K_L02704-6h}:
               eine lobende \textcolor{green}{Notiz}{}\ledrightnote{→\textcolor{green}{Oeſterreichiſche Autoren}} mit
               Rückſicht darauf, daß man in dem Hauſe dinirt und ſich die Beziehung zu Deinem \textcolor{blue}{Papa-\label{K_L02704-7v}\edtext{Regierungsrath}{\lemma{\textnormal{\emph{Regierungsrath}}}\Cendnote{\textnormal{\textcolor{blue}{Johann Schnitzler} wurde
                        1883 zum \textcolor{blue}{Regierungsrat} ernannt. \textcolor{blue}{Julius
                        Bauer} nannte Arthur Schnitzler in seiner \textcolor{green}{Rezension} den »Sohn des
                        bekannten \textcolor{blue}{Profeſſors} Dr. \textcolor{blue}{Schnitzler}«.}}}\label{K_L02704-7h}}{}\ledrightnote{→\textcolor{blue}{Johann Schnitzler}} erhalten will. \label{K_L02704-8v}\edtext{\textsc{\textcolor{blue}{Nossig}{}\ledrightnote{\textcolor{blue}{Alfred Nossig}}}}{\lemma{\textnormal{\emph{Nossig}}}\Cendnote{\textnormal{\textcolor{green}{Rezension} nicht ermittelt
                  XXXX evtl. könnte es die Rezension vom 3. 12. 1892 im Extrablatt sein, aber das
                  lässt sich nicht verifizieren; andere Möglichkeit wäre z. B. auch Wiener
                  Allgemeine Zeitung XXXX}}}\label{K_L02704-8h}: {\pb}einer, der auf
               Beides – die \strikeout{Dine} Diners und die Beziehung –
               candidirt. Macht aber nichts; ſie ſollen nur von dir ſprechen. Der Ruf wird ja nicht
               dadurch zunächſt gemacht, daß man verſtanden, ſondern dadurch, daß überſaupt von
               Einem geſprochen wird. Ich ſelbſt hätte längſt über Dich ſchreiben ſollen. Aber wann?
               Pure phyſiſche Unmöglichkeit, das ich Dich doch nicht damit {\pb}beſchimpfen will, daß ich eine Reklamnotiz für Dich
               zuſammenſchmiere. Die Sache mußte künſtleriſch verarbeitet werden. Aber ich habe
               nicht eine Stunde dafür gehabt. Soll alſo inzwiſchen der \label{K_L02704-12v}\edtext{Andere}{\lemma{\textnormal{\emph{Andere}}}\Cendnote{\textnormal{nicht
                  identifiziert}}}\label{K_L02704-12h} ſchreiben – der \textcolor{pink}{Berlin}{}\ledrightnote{\textcolor{pink}{Berlin}}er
               – ein ganz braver Menſch, \strikeout{b\textcolor{gray}{o}} bornirt, aber nach der guten Richtung bornirt, d.h. mit einem dummen
               Vorurtheil für das Moderne be{\pb}haftet, was Dir
               zuſtatten kommen wird. Er wird wohl bald \label{K_L02704-13v}\edtext{losſchießen}{\lemma{\textnormal{\emph{losſchießen}}}\Cendnote{\textnormal{XXXX Anatol-Rezension des Berliners erschienen? XXXX}}}\label{K_L02704-13h}. Und dann kann ich ja
               immer noch das Wort nehmen, wie es mein ſehnlicher Wunſch und feſter Vorſatz iſt. \textsc{\textcolor{blue}{Herzl}{}\ledrightnote{\textcolor{blue}{Theodor Herzl}}} aber wird nicht ſchreiben. Ich habe mein Möglichſtes gethan – ich bin ſoweit
               gegangen, als ich gehen konnte, – aber, ein ſo braver \textcolor{blue}{Menſch}{}\ledrightnote{→\textcolor{blue}{Theodor Herzl}} er iſt, ſo kennſt Du doch auch
               ſeinen {\pb}Größenwahn. Und er hat mir auf meine
               Andeutungen in einer Weiſe geantwortet, daß ich nicht mehr darauf zu rückkommen
               konnte, ohne Dich bloßzuſtellen. (»Wenn er mir ſein \textcolor{green}{Buch}{}\ledrightnote{→\textcolor{green}{Anatol}} deshalb geſchickt hat, damit ich darüber ſchreibe \textsc{etc}« {\dotsfour})\pend
           \pstart
           Und nun Dein \textcolor{green}{Stück}{}\ledrightnote{→\textcolor{green}{Das Märchen. Schauspiel in drei Aufzügen}}? Auf wann
               die \label{K_L02704-9v}\edtext{Aufführung}{\lemma{\textnormal{\emph{Aufführung}}}\Cendnote{\textnormal{Erst ein knappes Jahr später, am 1. 12. 1893, kam es zur Uraufführung des \emph{\textcolor{green}{Märchen}}s am \emph{\textcolor{brown}{Deutschen
                     Volkstheater}} in \textcolor{pink}{Wien}. Zuvor lehnte das
                     \emph{\textcolor{brown}{Burgtheater}} das \emph{\textcolor{green}{Märchen}} ab, wie Schnitzler am 19. 11. 1892 im \emph{\textcolor{green}{Tagebuch}} notierte. Außerdem war eine Aufführung in der zweiten Hälfte des
                     Januars 1893 am \emph{\textcolor{brown}{Neuen Deutschen Theater}} in \textcolor{pink}{Prag}
                  geplant, die jedoch ebenso nicht stattfand Siehe Paul Goldmann an Arthur Schnitzler, 27. 6. [1892] wie Bemühungen um eine Aufführung am \textcolor{pink}{Berlin}er \emph{\textcolor{brown}{Lessing-Theater}}
                  gelingen wollten Siehe A. S.: \emph{Tagebuch}, 18. 3. 1893.}}}\label{K_L02704-9h}? Und das neue \textcolor{green}{Stück}{}\ledrightnote{→\textcolor{green}{Abschiedssouper}}? Und Deine \label{K_L02704-14v}\edtext{\textcolor{green}{Novellen}{}\ledrightnote{→\textcolor{green}{Sterben. Novelle}}}{\lemma{\textnormal{\emph{Novellen}}}\Cendnote{\textnormal{Goldmann bezog sich hier womöglich auf
                  den \textcolor{green}{\emph{\textcolor{green}{Anatol}}-Einakter}{ }\emph{\textcolor{green}{Abschiedssouper}}, der am 14. 7. 1893 im \textcolor{pink}{Bad Ischl}er \textcolor{pink}{Stadttheater} uraufgeführt wurde. Eine der erwähnten
                     »Novellen« könnte \emph{\textcolor{green}{Sterben}}
                  sein.}}}\label{K_L02704-14h}? Und, ſag mir nur, warum {\pb}biſt Du
               ein ſo elender Menſch und ich ſchreibſt mir nichts PerſönIiches mehr? Weißt Du, daß
               Du mich glücklich aus Deinem Leben herausgeworfen haſt? Und daß Du mich auf
               literariſche Diät geſetzt haſt? Literariſcher Beirath! Aber Arthur! Pfui Teufel!
               Schämſt Du Dich denn gar nicht? {\dots}\pend
           \pstart
           Ich habe \textcolor{blue}{Jemanden}{}\ledrightnote{→\textcolor{blue}{Heinrich Kanner}} für Euren
               lieben Kreis. {\pb}Das ſympathiſcheſte \textcolor{blue}{Mitglied}{}\ledrightnote{→\textcolor{blue}{Heinrich Kanner}} hat ſich aus unſerer \textcolor{brown}{Redaction}{}\ledrightnote{→\textcolor{brown}{Frankfurter Zeitung}} losgelöſt, weil es von
                  \textsc{\textcolor{blue}{Sonnemann}{}\ledrightnote{\textcolor{blue}{Leopold Sonnemann}}} denn doch gar zu ſehr chicanirt wurde, und iſt – \textcolor{pink}{Wien}{}\ledrightnote{\textcolor{pink}{Wien}}er von \label{K_L02704-2v}\edtext{Geburt}{\lemma{\textnormal{\emph{Geburt}}}\Cendnote{\textnormal{\textcolor{blue}{Heinrich Kanner} wurde in \textcolor{pink}{Galatz} (\textcolor{pink}{Rumänien})
                  geboren, zog aber als Kleinkind im Jahr 1866 mit seiner Familie nach
                     \textcolor{pink}{Wien}.}}}\label{K_L02704-2h} und Erziehung
               – unſer \textcolor{pink}{Wien}{}\ledrightnote{\textcolor{pink}{Wien}}er \textcolor{blue}{Correſpondent}{}\ledrightnote{→\textcolor{blue}{Heinrich Kanner}} geworden. \textsc{Dr. \textcolor{blue}{Heinrich Kanner}{}\ledrightnote{\textcolor{blue}{Heinrich Kanner}}} – Adreſſe wird Dir Dr. \textsc{\textcolor{blue}{Joachim}{}\ledrightnote{\textcolor{blue}{Jaques Joachim}}} ſagen, oder ich ſchreib’ ſie Dir auf – \textcolor{blue}{einer}{}\ledrightnote{→\textcolor{blue}{Heinrich Kanner}} der liebſten Leute, die mir überhaupt be{\pb}gegnet ſind. Kein Künſtler ſondern \textcolor{blue}{Volkswirth}{}\ledrightnote{→\textcolor{blue}{Heinrich Kanner}} und \textcolor{blue}{Politiker}{}\ledrightnote{→\textcolor{blue}{Heinrich Kanner}}. Aber doch vielleicht \textcolor{blue}{Künſtlernatur}{}\ledrightnote{→\textcolor{blue}{Heinrich Kanner}}, vor Allem aber
               ein wahres \textcolor{blue}{Ideal}{}\ledrightnote{→\textcolor{blue}{Heinrich Kanner}} an
               Geſcheitheit, Feinſinn und \textsc{Noblesse}. Geh’, ſetz Dich mit
               ihm in \label{K_L02704-10v}\edtext{Verbindung}{\lemma{\textnormal{\emph{Verbindung}}}\Cendnote{\textnormal{Es sind keine Briefe zwischen Schnitzler
                  und \textcolor{blue}{Heinrich Kanner}, der außerdem erst am
                     24. 9. 1896 im \emph{\textcolor{green}{Tagebuch}} erwähnt wurde, bekannt.}}}\label{K_L02704-10h}. Wirſt
               Deine Freude daran haben{\dotsfive}\pend
           \pstart
           Von ganzem Herzen ein frohes neues Jahr, mein theurer Freund! {\pb}Arbeitsluſt! Erfolg! Und vorwärts! Die allerwärmſten
               Grüße an \textsc{\textcolor{blue}{Loris}{}\ledrightnote{\textcolor{blue}{Hugo von Hofmannsthal}{\newline}\textcolor{red}{KEY PROBLEM}{\newline}\textcolor{red}{KEY PROBLEM}{\newline}\textcolor{red}{KEY PROBLEM}{\newline}\textcolor{red}{KEY PROBLEM}{\newline}\textcolor{red}{KEY PROBLEM}{\newline}\textcolor{red}{KEY PROBLEM}{\newline}\textcolor{red}{KEY PROBLEM}{\newline}\textcolor{red}{KEY PROBLEM}{\newline}\textcolor{red}{KEY PROBLEM}{\newline}\textcolor{red}{KEY PROBLEM}{\newline}\textcolor{red}{KEY PROBLEM}{\newline}\textcolor{red}{KEY PROBLEM}{\newline}\textcolor{red}{KEY PROBLEM}{\newline}\textcolor{red}{KEY PROBLEM}{\newline}\textcolor{red}{KEY PROBLEM}{\newline}\textcolor{red}{KEY PROBLEM}{\newline}\textcolor{red}{KEY PROBLEM}{\newline}\textcolor{red}{KEY PROBLEM}{\newline}\textcolor{red}{KEY PROBLEM}{\newline}\textcolor{red}{KEY PROBLEM}{\newline}\textcolor{red}{KEY PROBLEM}}} und \textsc{\textcolor{blue}{Richard}{}\ledrightnote{\textcolor{blue}{Richard Beer-Hofmann}}} (\textsc{\textcolor{blue}{Richard}{}\ledrightnote{\textcolor{blue}{Richard Beer-Hofmann}}} ſoll mir ſchreiben!!!). Ergebene Empfehlungen und Neujahrswünſche an Deine \textcolor{blue}{Eltern}{}\ledrightnote{→\textcolor{blue}{Louise Schnitzler}{\newline}→\textcolor{blue}{Johann Schnitzler}}. Grüße an
               Deinen \textcolor{blue}{Bruder}{}\ledrightnote{→\textcolor{blue}{Julius Schnitzler}}, \textsc{\textcolor{blue}{Kapper}{}\ledrightnote{\textcolor{blue}{Friedrich Kapper}}} und wen ich ſonſt noch in \textcolor{pink}{Wien}{}\ledrightnote{\textcolor{pink}{Wien}} lieb habe,
               was Du ja ebenſo wohl weißt wie ich.\pend
           \pstart
           Und ich umarme Dich von ganzem Herzen, {\pb}in alter,
               unwandelbarer, treuer Freundſchaft.\pend
           \pstart
           Dein {\\[\baselineskip]}\spacefill\mbox{Paul Goldm}\pend
           \leftskip=0em{}\pstart
           \noindent{}Der kleinen \textcolor{blue}{Elſe}{}\ledrightnote{\textcolor{blue}{Else Singer}}: Handkuß, und ich hab’ die
                  Sachen leider ſelbſt nicht mehr. Liegt auch ſo weit hinter mir. Will mich auch gar
                  nicht mehr daran erinnern, daß ich einmal Künſtler werden wollte und daß es kleine
                     \textcolor{blue}{Elſe}{}\ledrightnote{\textcolor{blue}{Else Singer}}n in der {\pb}Welt gibt. Das thut ſo weh!\pend
           \pstart
           Und ſag’ einmal: Könnteſt Du nicht unter der Hand einmal und ganz zufällig
                  erfahren, was \label{K_L02704-15v}\edtext{\textsc{\textcolor{blue}{Hilda}{}\ledrightnote{\textcolor{blue}{Hilda von Mitis}}}}{\lemma{\textnormal{\emph{Hilda}}}\Cendnote{\textnormal{Siehe Paul Goldmann an Arthur Schnitzler, 27. 4. 1891}}}\label{K_L02704-15h} macht? Ich glaube, ich habe mich da doch wie ein Schaf benommen. Dieſes
                  aber unter uns.\pend
           \pstart
           Bald einen Brief, nicht wahr? Theils literariſch, theils perſönlich!\pend
           \endnumbering\briefempfaengerindex{Schnitzler, Arthur@\textsc{Schnitzler, Arthur}!zzzGoldmann, Paul@\emph{von Paul Goldmann}!1892-12-241@{24. 12. {[}1892{]}}|)be}\mylabel{h}\begin{anhang}\end{anhang}
         \normalsize

\newenvironment{esempio}[3]%
{
    \vspace{1.5ex}
    \rlap{\underline{#1}}
    \par
    \setlength{\parindent}{0cm}
    \nopagebreak
    \leftskip=#2cm
    \rightskip=#3cm
}
{
    \par
}

\doendnotes{C}
\bigskip

\printindex[pw]


\end{document}
      