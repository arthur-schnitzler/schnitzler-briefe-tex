%% latex-korrekturansicht-vorspann.tex
%% Vorspann für die Korrekturansicht.
%% Lädt die gemeinsame Datei latex-vorspann.tex mit gesetztem Schalter.

\newif\ifkorrekturansicht
\korrekturansichttrue

\input{../tex-inputs/latex-vorspann}


               \section[Arthur Schnitzler an Richard Beer-Hofmann, 23. 7. 1917]{ Arthur Schnitzler an Richard Beer-Hofmann, 23. 7. 1917}\nopagebreak\mylabel{v}\rehead{ }\normalsize\beginnumbering\briefempfaengerindex{Beer-Hofmann, Richard@\textsc{Beer-Hofmann, Richard}!zzzSchnitzler, Arthur@\emph{von Arthur Schnitzler}!1917-07-231@{23. 7. 1917}|(be} \toendnotes[C]{\smallbreak\pagebreak[2]} \Standort{YCGL, MSS 31.}
\physDesc{Brief, 1 Blatt, 2 Seiten, Umschlag
\newline{}Handschrift: Bleistift, lateinische Kurrent\newline{}Versand: Stempel: »\nobreak{}\oindex{Bad Ischl@\textbf{Bad Ischl}, \emph{Besiedelter Ort (A.BSO)}|pwk}Bad Ischl, 25. VII. 17, IX\nobreak{}«.  }\buchAbdrucke{\weitereDrucke{1) Arthur Schnitzler: \emph{Briefe 1913–1931}. Hg. Peter Michael Braunwarth, Richard Miklin, Susanne Pertlik und Heinrich Schnitzler. Frankfurt am Main: \emph{S. Fischer} 1984, S. 138–140.} \weitereDrucke{2) Arthur Schnitzler, Richard Beer-Hofmann: \emph{Briefwechsel 1891–1931}. Hg. Konstanze Fliedl. Wien, Zürich: \emph{Europaverlag} 1992, S. 224–225.} }\toendnotes[C]{\smallbreak}\pstart{}{\pb}Dr Arthur Schnitzler \textcolor{pink}{XVIII Sternwartestr 71}{}\ledrightnote{\textcolor{pink}{Sternwartestraße}}.\pend{}\pstart{}\textcolor{pink}{Wien}{}\ledrightnote{\textcolor{pink}{Wien}}\pend{}{\bigskip}\pstart{}{\pb}Dr. Richard Beer-Hofmann\pend{}\pstart{}\textcolor{pink}{Bad Ischl}{}\ledrightnote{\textcolor{pink}{Bad Ischl}}\pend{}\pstart{}\textcolor{pink}{Grazerstr. 56}{}\ledrightnote{\textcolor{pink}{Grazer Straße}}.\pend{}{\bigskip}\pstart
           \raggedleft{}{\pb}\textcolor{pink}{Wien}{}\ledrightnote{\textcolor{pink}{Wien}}, 23. 7. 1917\pend
           \pstart
           lieber Richard – man wird so leicht unbescheiden! Da Sie mir einen
               Brief geschrieben haben, so wär es mir natürlich sehr erfreulich gewesen, darin auch
               etwas über Sie, die Ihren, Ihr Leben, Ihr Arbeiten, und was es eben so von \textcolor{pink}{Ischl}{}\ledrightnote{\textcolor{pink}{Bad Ischl}} nach \textcolor{pink}{Wien}{}\ledrightnote{\textcolor{pink}{Wien}} zu
               berichten gibt vorzufinden, und ich hoffe, daß Sie in der Antwort auf diesen hier
               einiges nachtragen werden. Ich will Ihnen heute nur sagen, daſs es \textcolor{blue}{Arthur K.}{}\ledrightnote{\textcolor{blue}{Arthur Kaufmann}} völlig gut geht und daß er Mittwoch in seine \textcolor{pink}{Wien}{}\ledrightnote{\textcolor{pink}{Wien}}er (übrigens definitiv gekündigte) Wohnung
               wiederkehrt. Vorgestern fügte es sich, daß er mir seine Ideen (über die er mir schon
               manches vorher andeutungsweise mitgetheilt) – in einer Art von Zusa{\geminationm}enhang vortrug. Meine \strikeout{\textcolor{gray}{×}\-\textcolor{gray}{×}\-\textcolor{gray}{×}\-\textcolor{gray}{×}\-\textcolor{gray}{×}} Vorbildg in der Philosophie \introOben{}ist\introOben{} zu wenig exact und
               ausgreifend, als daß ich mir ein Urtheil zu bilden vermöchte, ob die merkwürdigen
               Dinge, die \textcolor{blue}{K.}{}\ledrightnote{\textcolor{blue}{Arthur Kaufmann}} eingefallen sind einen Schritt
               vorwärts bedeuten in der Geschichte des menschlichen Denkens: für mich handelt es
               sich hier um wunderschöne Gedankenspiele (nicht -spielereien), in einer
               beträchtlichen und sehr reinen Höhe, an denen ich ein Wohlgefallen empfinde, in dem
                  \strikeout{v} intellectuelle, aesthetische und auch moralische
               Elemente vorhanden sind. Mir wär es wahrscheinlich nicht anders gegangen, we{\geminationn} mir \textcolor{blue}{Kant}{}\ledrightnote{\textcolor{blue}{Immanuel Kant}} oder \textcolor{blue}{Schopenhauer}{}\ledrightnote{\textcolor{blue}{Arthur Schopenhauer}} ihre geistigen Entdeckungen zum ersten
               Mal vorgetragen hätten; – meine Ansichten über Philosophie als Wissenschaft {\pb}sind überhaupt etwas ketzerisch; nicht daß ich die
               Philosophie »unterschätzte« – ich rangire sie nur anderswo ein, als ihre Adepten es
               im allgemeinen zu thun pflegen. Und mir scheint als we{\geminationn}
               mir gerade aus manchem was \textcolor{blue}{K.}{}\ledrightnote{\textcolor{blue}{Arthur Kaufmann}} ausspricht,
               Bestätigungen für meine Auffassung – oder sagen wir Empfindung – entgegenkämen. Über
               die Krankheit selbst, und über die Aerzte wollen wir uns mündlich unterhalten. Wann?
                  \textcolor{pink}{Salzka{\geminationm}ergut}{}\ledrightnote{\textcolor{pink}{Salzkammergut}} nicht
               sehr wahrscheinlich. Ende August gedenken wir (we{\geminationn}s nicht gar zu unbequem) nach \textcolor{pink}{Partenkirchen}{}\ledrightnote{\textcolor{pink}{Partenkirchen}} zu meiner \textcolor{blue}{Schwägerin}{}\ledrightnote{→\textcolor{blue}{Elisabeth Steinrück}}, ev. halten wir uns in \textcolor{pink}{Salzburg}{}\ledrightnote{\textcolor{pink}{Salzburg}} auf. – Hier ist es ganz erträglich, ich mache (fast
               immer allein) schöne Spaziergänge im \textcolor{pink}{Wien}{}\ledrightnote{\textcolor{pink}{Wien}}er Wald,
               (den Sie kennen lernen sollten) – entdecke immer neue Gegenden, mit neuen
               Schönheiten. Im übrigen arbeite ich – es ist, neben dem Spazierengehen, die einzige
               Art, über das Grauen, die Si{\geminationn}losigkeit und die
               Abgeschmacktheit dieser Zeit gelegentlich wegzuko{\geminationm}en.
                  Si{\geminationn}losigkeit? – Oder sollte es doch einen Sinn haben?
                  Da{\geminationn} müßte man erst recht verrückt werden. – Nehmen
               Sie unser Beileid zu Schufterls Hinscheiden; bei uns \substVorne{}\textsuperscript{\textcolor{gray}{quartiert}}{\allowbreak}\substDazwischen{}hat\substHinten{} sich \introOben{}nun\introOben{} auch so ein kleines Thierchen
               einquartiert, das eigentlich der \textcolor{blue}{Wucki}{}\ledrightnote{\textcolor{blue}{Hermine Simandt}} gehört,
               die jetzt mit ihm auf Urlaub ist – in \textcolor{pink}{Oberhollabrunn}{}\ledrightnote{\textcolor{pink}{Hollabrunn}}. Die Rückkehr beider erwarte ich mit Fassung.\pend
           \pstart
           Wir grüßen Sie \textcolor{blue}{Alle}{}\ledrightnote{→\textcolor{blue}{Gabriel Beer-Hofmann}{\newline}→\textcolor{blue}{Paula Beer-Hofmann}{\newline}→\textcolor{blue}{Mirjam Beer-Hofmann}{\newline}→\textcolor{blue}{Naëmah Beer-Hofmann}} herzlichst.\pend
           \pstart Ihr \spacefill\mbox{Arthur}\pend{}\endnumbering\briefempfaengerindex{Beer-Hofmann, Richard@\textsc{Beer-Hofmann, Richard}!zzzSchnitzler, Arthur@\emph{von Arthur Schnitzler}!1917-07-231@{23. 7. 1917}|)be}\mylabel{h}  \normalsize

\doendnotes{C}
\bigskip
\vfill

\clearpage

\footnotesize

\lohead{\textsc{register}}

% Definiere theindex-Environment komplett neu ohne reledmac
\makeatletter
\renewenvironment{theindex}{%
  \section*{\indexname}%
  \setlength{\parindent}{0pt}%
  \setlength{\parskip}{0pt plus 0.3pt}%
  \let\item\@idxitem
}{%
  \clearpage
}
\makeatother

\IfFileExists{\jobname-pw.ind}{\input{\jobname-pw.ind}}{}

\end{document}

      