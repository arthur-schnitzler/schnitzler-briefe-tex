%% latex-korrekturansicht-vorspann.tex
%% Vorspann für die Korrekturansicht.
%% Lädt die gemeinsame Datei latex-vorspann.tex mit gesetztem Schalter.

\newif\ifkorrekturansicht
\korrekturansichttrue

\input{../tex-inputs/latex-vorspann}


               \section[Arthur Schnitzler an Richard Beer-Hofmann, 9. 9. 1894]{ Arthur Schnitzler an Richard Beer-Hofmann, 9. 9. 1894}\nopagebreak\mylabel{v}\rehead{ }\normalsize\beginnumbering\briefempfaengerindex{Beer-Hofmann, Richard@\textsc{Beer-Hofmann, Richard}!zzzSchnitzler, Arthur@\emph{von Arthur Schnitzler}!1894-09-091@{9. 9. 1894}|(be} \toendnotes[C]{\smallbreak\pagebreak[2]} \Standort{YCGL, MSS 31.}
\physDesc{Brief, 1 Blatt, 2 Seiten, Umschlag
\newline{}Handschrift: Bleistift, deutsche Kurrent\newline{}Versand: 1) Stempel: »\nobreak{}\oindex{IX., Alsergrund@\textbf{IX., Alsergrund}, \emph{Bezirk (A.BZK)}|pwk}Wien 9/3, 9. 9. 94, 3–4 N\nobreak{}«.  2) Stempel: »\nobreak{}\oindex{Bad Ischl@\textbf{Bad Ischl}, \emph{Besiedelter Ort (A.BSO)}|pwk}Ischl, 10/9 9{[}4{]}, 7 F\nobreak{}«. }\buchAbdrucke{\weitereDrucke{Arthur Schnitzler, Richard Beer-Hofmann: \emph{Briefwechsel 1891–1931}. Hg. Konstanze Fliedl. Wien, Zürich: \emph{Europaverlag} 1992, S. 59.} }\toendnotes[C]{\smallbreak}\pstart{}{\pb}Herrn Dr. \textsc{Richard
                     Beer-Hofmann}\pend{}\pstart{}\textsc{\textcolor{pink}{Ischl}{}\ledrightnote{\textcolor{pink}{Bad Ischl}}}\pend{}\pstart{}\textsc{\textcolor{pink}{Egelmoos 22}{}\ledrightnote{\textcolor{pink}{Eglmoosgasse}}}\pend{}{\bigskip}\pstart{}{\pb}Lieber Richard,\pend\pstart
           1) \textcolor{blue}{\textcolor{green}{Bolgar}{}\ledrightnote{→\textcolor{green}{Die Regeln des Duells}}}{}\ledrightnote{\textcolor{blue}{Franz von Bolgár}} geht eben unter Kreuzband ab.\pend
           \pstart
           2.) an \textcolor{blue}{P. Horn}{}\ledrightnote{\textcolor{blue}{Paul Horn}}{ }ſchrieb ich, weil \textcolor{brown}{Schenker}{}\ledrightnote{\textcolor{brown}{Schenker {\kaufmannsund} Co.}} immer beſetzt iſt und das telefoniren mich nervös macht. Ich bat
               ihn, Ihnen direct ſofort zu antworten.\pend
           \pstart
           3.) \textcolor{blue}{Bahr}{}\ledrightnote{\textcolor{blue}{Hermann Bahr}} werde ich morgen ſprechen.\pend
           \pstart
           4.) \textcolor{blue}{Adele S.}{}\ledrightnote{\textcolor{blue}{Adele Sandrock}} wohnt \textcolor{pink}{Opernring 19}{}\ledrightnote{\textcolor{pink}{Opernring}}.\pend
           \pstart
           5.) Der \label{K_L00368-4v}\edtext{\textcolor{green}{Artikel}{}\ledrightnote{→\textcolor{green}{Ein Märchen}}}{\lemma{\textnormal{\emph{Artikel}}}\Cendnote{\textnormal{\textcolor{blue}{Laura Marholm}: \emph{\textcolor{green}{Ein Märchen}}. In: \emph{\textcolor{green}{Die Zukunft}}, Jg. 8, 25. 8. 1894, S. 368–371.}}}\label{K_L00368-4h} der \textcolor{blue}{Marholm}{}\ledrightnote{\textcolor{blue}{Laura Marholm}} iſt  ſehr ſchön, ſehr werthvoll
               beſonders. – Hieſs »\textcolor{green}{Ein Märchen}{}\ledrightnote{\textcolor{green}{Ein Märchen}}« und beſchäftigt
               ſich nach 1 ½ Seiten allg. Einleitung auf 2 ½ Seiten {\pb}mit mir. – (Beſtellt; Sie kriegen ihn da{\geminationn})\pend
           \pstart
           6.) Vergeſſen Sie nicht mir den Stock, welcher in Ihrer Hand ſo elegant wird, nach
                  \textcolor{pink}{Wien}{}\ledrightnote{\textcolor{pink}{Wien}} zu ſchicken.\pend
           \pstart
           7.) Glücklicher! –\pend
           \pstart
           Herzliche Grüße Ihr{\\[\baselineskip]}\spacefill\mbox{Arthur}\pend
           \leftskip=0em{}\pstart
           9. Sept. 94{ }\textcolor{pink}{Wien}{}\ledrightnote{\textcolor{pink}{Wien}}.\pend
           \endnumbering\briefempfaengerindex{Beer-Hofmann, Richard@\textsc{Beer-Hofmann, Richard}!zzzSchnitzler, Arthur@\emph{von Arthur Schnitzler}!1894-09-091@{9. 9. 1894}|)be}\mylabel{h}  \normalsize

\doendnotes{C}
\bigskip
\vfill

\clearpage

\footnotesize

\lohead{\textsc{register}}

% Definiere theindex-Environment komplett neu ohne reledmac
\makeatletter
\renewenvironment{theindex}{%
  \section*{\indexname}%
  \setlength{\parindent}{0pt}%
  \setlength{\parskip}{0pt plus 0.3pt}%
  \let\item\@idxitem
}{%
  \clearpage
}
\makeatother

\IfFileExists{\jobname-pw.ind}{\input{\jobname-pw.ind}}{}

\end{document}

      