%% latex-korrekturansicht-vorspann.tex
%% Vorspann für die Korrekturansicht.
%% Lädt die gemeinsame Datei latex-vorspann.tex mit gesetztem Schalter.

\newif\ifkorrekturansicht
\korrekturansichttrue

\input{../tex-inputs/latex-vorspann}


               \section[Marie von Ebner-Eschenbach an Arthur Schnitzler, 26. 9. 1901]{ Marie von Ebner-Eschenbach an Arthur Schnitzler, 26. 9. 1901}\nopagebreak\mylabel{v}\rehead{ }\normalsize\beginnumbering\briefempfaengerindex{Schnitzler, Arthur@\textsc{Schnitzler, Arthur}!zzzEbner-Eschenbach, Marie von@\emph{von Marie von Ebner-Eschenbach}!1901-09-261@{26. 9. 1901}|(be} \toendnotes[C]{\smallbreak\pagebreak[2]} \Standort{DLA, A:Schnitzler, HS.1985.1.5718.}
\physDesc{Brief, 1 Blatt, 2 Seiten, fotografische Vervielfältigung
\newline{}Handschrift: schwarze Tinte, lateinische Kurrent
\newline{}Schnitzler: vermutlich mit rotem Buntstift »\textsc{\textcolor{gray}{\textcolor{green}{Leutnt}}}«, »\textsc{Ebner Eschenbach}« und eine Unterstreichung }\toendnotes[C]{\smallbreak}\pstart
           \noindent{}\raggedleft{}{\pb}\textcolor{pink}{\textcolor{gray}{\textbf{SCHLOSS ZDISSLAWITZ}}}{}\ledrightnote{\textcolor{pink}{Schloss Zdislavice}}{\\}\textcolor{gray}{\textbf{POST \textcolor{pink}{ZDOUNEK. MÄHREN}{}\ledrightnote{\textcolor{pink}{Zdounky}}}}\pend
           \pstart
           \raggedleft{}26. Sept. 1901.\pend
           \pstart\center{}Verehrter Herr Doctor!\pend\pstart
           Viel zu spät danke ich Ihnen, verzeihen Sie es mir. So manche Entschuldigung hätte
               ich vorzubringen, will Sie aber nicht damit langweilen, sondern gleich anfangen das
               allzu lang Versäumte nachzuholen. Sie haben mir mit Ihrer großmütigen \label{K_L02581-1v}\edtext{Spende}{\lemma{\textnormal{\emph{Spende}}}\Cendnote{\textnormal{Vermutlich hat ihr \textcolor{blue}{Schnitzler} seine beiden im April erschienenen \emph{\textcolor{green}{Lieutenant Gustl}} und \emph{\textcolor{green}{Frau Bertha Garlan}} geschenkt.}}}\label{K_L02581-1h} Ehre erwiesen und
               Freude gemacht, Ihre beiden letzten \textcolor{green}{Werke}{}\ledrightnote{→\textcolor{green}{Lieutenant Gustl. Novelle}{\newline}→\textcolor{green}{Frau Bertha Garlan. Roman}} sind mir – wie deren Vorgänger – lieb und
               wert geworden und ich bewundere sie. Mit wärmster Zustimmung {\pb}las ich eben im \textcolor{green}{Westermannschen Octoberheft}{}\ledrightnote{\textcolor{green}{Westermanns Monatshefte}}
               die \label{K_L02581-2v}\edtext{\textcolor{green}{Besprechung}{}\ledrightnote{→\textcolor{green}{Romane und Novellen}}}{\lemma{\textnormal{\emph{Besprechung}}}\Cendnote{\textnormal{\textcolor{blue}{F. D. [=Friedrich Düsel]}: \emph{\textcolor{green}{Romane und Novellen}}. In: \emph{\textcolor{green}{Westermanns Monatshefte}}, Jg. 46, Nr. 541, Oktober
                        1901, S. 157–160.}}}\label{K_L02581-2h} des »\textcolor{green}{Leutnant Gustl}{}\ledrightnote{\textcolor{green}{Lieutenant Gustl. Novelle}}«\textcolor{gray}{.}\pend
           \pstart
           Mir uralten Erzählerin ist das Zeichen des Wohlwollens das eines der glänzendsten
               Vertreter der neuen Richtung unserer Litteratur mir gegeben hat, eine Quelle
                  \textcolor{gray}{ewigster} Befriedigung.\pend
           \pstart
           Dankbarst, verehrter Herr Doctor, {\\[\baselineskip]}Ihre ergebene {\\[\baselineskip]}\spacefill\mbox{Marie Ebner-Eschenbach.}\pend
           \leftskip=0em{}\endnumbering\briefempfaengerindex{Schnitzler, Arthur@\textsc{Schnitzler, Arthur}!zzzEbner-Eschenbach, Marie von@\emph{von Marie von Ebner-Eschenbach}!1901-09-261@{26. 9. 1901}|)be}\mylabel{h}  \normalsize

\doendnotes{C}
\bigskip
\vfill

\clearpage

\footnotesize

\lohead{\textsc{register}}

% Definiere theindex-Environment komplett neu ohne reledmac
\makeatletter
\renewenvironment{theindex}{%
  \section*{\indexname}%
  \setlength{\parindent}{0pt}%
  \setlength{\parskip}{0pt plus 0.3pt}%
  \let\item\@idxitem
}{%
  \clearpage
}
\makeatother

\IfFileExists{\jobname-pw.ind}{\input{\jobname-pw.ind}}{}

\end{document}

      