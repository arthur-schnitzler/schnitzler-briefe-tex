%% latex-korrekturansicht-vorspann.tex
%% Vorspann für die Korrekturansicht.
%% Lädt die gemeinsame Datei latex-vorspann.tex mit gesetztem Schalter.

\newif\ifkorrekturansicht
\korrekturansichttrue

\input{../tex-inputs/latex-vorspann}


               \section[Hugo von Hofmannsthal an Arthur Schnitzler, {[}15. 3. 1904{]}]{ Hugo von Hofmannsthal an Arthur Schnitzler, {[}15. 3. 1904{]}}\nopagebreak\mylabel{v}\rehead{ }\normalsize\beginnumbering\briefempfaengerindex{Schnitzler, Arthur@\textsc{Schnitzler, Arthur}!zzzHofmannsthal, Hugo von@\emph{von Hugo von Hofmannsthal}!1904-03-152@{15. 3. 1904}|(be} \toendnotes[C]{\smallbreak\pagebreak[2]} \Standort{CUL, Schnitzler, B 43.}
\physDesc{Brief, 1 Blatt, 4 Seiten
\newline{}Handschrift: schwarze Tinte, deutsche Kurrent
\newline{}Schnitzler: mit Bleistift datiert: »15/3 904.« \newline{}Ordnung: 1) mit Bleistift von unbekannter Hand nummeriert: »\strikeout{241}« 2) mit Bleistift von unbekannter Hand nummeriert:
                                    »217«}\buchAbdrucke{\weitereDrucke{Hugo von Hofmannsthal, Arthur Schnitzler: \emph{Briefwechsel}. Hg. Therese Nickl und Heinrich Schnitzler. Frankfurt am Main: \emph{S. Fischer} 1964, S. 184.} }\toendnotes[C]{\smallbreak}\pstart
           \noindent{}{\pb}Mein lieber Arthur,\hspace*{1.5em}meiner \textcolor{blue}{Mama}{}\ledrightnote{→\textcolor{blue}{Anna von Hofmannsthal}} Zuſtand iſt – wie ja nicht anders zu erwarten, – genau
               ſo elend wie vor ein paar Tagen. Geprüft durch jahrelangen Anblick eines ſolchen
               complicierten \label{K_L01384_1v}\edtext{pſychaſtheniſchen}{\lemma{\textnormal{\emph{pſychaſtheniſchen}}}\Cendnote{\textnormal{1903 von \textcolor{blue}{Pierre Janet}
                  eingeführter Ausdruck für jemanden, der aufgrund einer neurotischen Störung eine
                  nur geringe körperliche und psychische Belastbarkeit aufweist.}}}\label{K_L01384_1h} Leidens ſind
               wir ja auch nicht ungeduldig.\hspace*{1.5em}Nicht wahr aber, Sie
               ſind nicht bös, daſs das Leben es mit {\pb}ſich gebracht hat, daſs zwei ſo
               verſchiedene Dinge, wie Ihre zufällige Arzt-eigenſchaft und unſere Freundſchaft mich
               jetzt ermuthigen, Sie um Hilfe anzubetteln. Es erſcheint halt alles ringsum, alles
               was man verſuchen kann, alles was man herbeirufen kann, ſo erſchöpft.\pend
           \pstart
           Das iſt der Gegenſtand von meiner und meines \textcolor{blue}{Vaters}{}\ledrightnote{→\textcolor{blue}{Hugo August von Hofmannsthal}} hauptſächlicher Bitte: daſs Sie {\pb}Ihr Verſtändnis der \uline{Geſamterſcheinung} dieſer kranken \textcolor{blue}{Frau}{}\ledrightnote{→\textcolor{blue}{Anna von Hofmannsthal}} in einem Geſpräch Ihrem \textcolor{blue}{Bruder}{}\ledrightnote{→\textcolor{blue}{Julius Schnitzler}} nahebringen, ſo daſs er von ſeinem
               nächſten Beſuch an – und bei öfteren Beſuchen, die man erbitten wird – neben dem \textcolor{blue}{Hausarzt}{}\ledrightnote{→\textcolor{blue}{Hans Schandlbauer}} oder über
               dem \textcolor{blue}{Hausarzt}{}\ledrightnote{→\textcolor{blue}{Hans Schandlbauer}} der
               leitende Arzt im Ganzen wird, derjenige gute Arzt der die Einwirkungen {\pb}auf einen Theil (hier die
               Narbungen im Darm) ſo weit als möglich dem Einblick in das Ganze unterordnet.\pend
           \pstart
           Wir bilden uns nicht ein, daſs ein ſolcher Patient zu \uline{curieren} iſt. Aber von einer ſolchen Krise des Elends wieder in das relativ
               normale zurückzuführen iſt ſie doch vielleicht\textcolor{gray}{?} Sie werden mir
                  Freitag vielleicht ſagen, wann Sie mit Ihrem \textcolor{blue}{Bruder}{}\ledrightnote{→\textcolor{blue}{Julius Schnitzler}}{ }ſprechen können, nachher ruft man ihn dann wieder.
               Ihr \spacefill\mbox{Hugo}\pend
           \endnumbering\briefempfaengerindex{Schnitzler, Arthur@\textsc{Schnitzler, Arthur}!zzzHofmannsthal, Hugo von@\emph{von Hugo von Hofmannsthal}!1904-03-152@{15. 3. 1904}|)be}\mylabel{h}  \normalsize

\doendnotes{C}
\bigskip
\vfill

\clearpage

\footnotesize

\lohead{\textsc{register}}

% Definiere theindex-Environment komplett neu ohne reledmac
\makeatletter
\renewenvironment{theindex}{%
  \section*{\indexname}%
  \setlength{\parindent}{0pt}%
  \setlength{\parskip}{0pt plus 0.3pt}%
  \let\item\@idxitem
}{%
  \clearpage
}
\makeatother

\IfFileExists{\jobname-pw.ind}{\input{\jobname-pw.ind}}{}

\end{document}

      