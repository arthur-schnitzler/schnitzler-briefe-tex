%% latex-korrekturansicht-vorspann.tex
%% Vorspann für die Korrekturansicht.
%% Lädt die gemeinsame Datei latex-vorspann.tex mit gesetztem Schalter.

\newif\ifkorrekturansicht
\korrekturansichttrue

\input{../tex-inputs/latex-vorspann}


               \section[Arthur Schnitzler an Richard Beer-Hofmann, 21. 9. 1901]{ Arthur Schnitzler an Richard Beer-Hofmann, 21. 9. 1901}\nopagebreak\mylabel{v}\rehead{ }\normalsize\beginnumbering\briefempfaengerindex{Beer-Hofmann, Richard@\textsc{Beer-Hofmann, Richard}!zzzSchnitzler, Arthur@\emph{von Arthur Schnitzler}!1901-09-211@{21. 9. 1901}|(be} \toendnotes[C]{\smallbreak\pagebreak[2]} \Standort{YCGL, MSS 31.}
\physDesc{Brief, 1 Blatt, 1 Seite, Umschlag
\newline{}Handschrift: Bleistift, deutsche Kurrent\newline{}Versand: 1) Rohrpost 2) Stempel: »\nobreak{}\oindex{I., Innere Stadt@\textbf{I., Innere Stadt}, \emph{Bezirk (A.BZK)}|pwk}Wien 1/1, 21 IX 01, 10 40\nobreak{}«. 3) Stempel: »\nobreak{}\oindex{I., Innere Stadt@\textbf{I., Innere Stadt}, \emph{Bezirk (A.BZK)}|pwk}Wien 1/1, 21 IX 01, 10 50N\nobreak{}«. \newline{}Ordnung: mit Bleistift von unbekannter Hand datiert: »{\pb}21. 9.« }\toendnotes[C]{\smallbreak}\pstart{}{\pb}Herrn Dr. \textsc{Rich.
                            Beer-Hofmann}\pend{}\pstart{}\textcolor{pink}{Wien}{}\ledrightnote{\textcolor{pink}{Wien}}\pend{}\pstart{}\textsc{\textcolor{pink}{I. Wollzeile 15}{}\ledrightnote{\textcolor{pink}{Wollzeile}}.}\pend{}{\bigskip}\pstart
           \raggedleft{}{\pb}Samſtag\pend
           \pstart
           lieber Richard, ohne \textcolor{gray}{Praejudiz} für event.
                    Eintritt werde ich heute Abend, am Ende ſchon zum Nachtmahl, jedenfalls aber um
                        10, im \textcolor{brown}{Club}{}\ledrightnote{→\textcolor{brown}{Wiener Schachclub}}
                    ſein.\pend
           \pstart
           Herzlichſt{\\[\baselineskip]}Ihr{\\[\baselineskip]}\spacefill\mbox{Arthur}\pend
           \leftskip=0em{}\pstart
           \noindent{}\textcolor{blue}{Guſtav}{}\ledrightnote{\textcolor{blue}{Gustav Schwarzkopf}} wohnt nach wie vor \textcolor{pink}{Tief. Gr. 23}{}\ledrightnote{\textcolor{pink}{Tiefer Graben}}; vielleicht iſt er aber in
                        dieſen Tagen in der \textcolor{pink}{Brühl}{}\ledrightnote{\textcolor{pink}{Brühl}}\pend
           \endnumbering\briefempfaengerindex{Beer-Hofmann, Richard@\textsc{Beer-Hofmann, Richard}!zzzSchnitzler, Arthur@\emph{von Arthur Schnitzler}!1901-09-211@{21. 9. 1901}|)be}\mylabel{h}  \normalsize

\doendnotes{C}
\bigskip
\vfill

\clearpage

\footnotesize

\lohead{\textsc{register}}

% Definiere theindex-Environment komplett neu ohne reledmac
\makeatletter
\renewenvironment{theindex}{%
  \section*{\indexname}%
  \setlength{\parindent}{0pt}%
  \setlength{\parskip}{0pt plus 0.3pt}%
  \let\item\@idxitem
}{%
  \clearpage
}
\makeatother

\IfFileExists{\jobname-pw.ind}{\input{\jobname-pw.ind}}{}

\end{document}

      