%% latex-korrekturansicht-vorspann.tex
%% Vorspann für die Korrekturansicht.
%% Lädt die gemeinsame Datei latex-vorspann.tex mit gesetztem Schalter.

\newif\ifkorrekturansicht
\korrekturansichttrue

\input{../tex-inputs/latex-vorspann}


               \section[Arthur Schnitzler an Georg Brandes, 12. 10. 1911]{ Arthur Schnitzler an Georg Brandes, 12. 10. 1911}\nopagebreak\mylabel{v}\rehead{ }\normalsize\beginnumbering\briefempfaengerindex{Brandes, Georg@\textsc{Brandes, Georg}!zzzSchnitzler, Arthur@\emph{von Arthur Schnitzler}!1911-10-121@{12. 10. 1911}|(be} \toendnotes[C]{\smallbreak\pagebreak[2]} \Standort{Kopenhagen, Det Kongelige Bibliotek, Georg Brandes Arkiv, box 125.}
\physDesc{Brief, 1 Blatt (Briefpapier mit Trauerrand), 4 Seiten
\newline{}Handschrift: schwarze Tinte, lateinische Kurrent\newline{}Ordnung: mit Bleistift von unbekannter Hand beschriftet:
                                        »Schnitzler« und »Arthur
                                        Schnitzler«, nummeriert: »32.« und
                                 mehrere Unterstreichungen }\buchAbdrucke{\weitereDrucke{Georg Brandes, Arthur Schnitzler: \emph{Ein Briefwechsel}. Hg. Kurt Bergel. Bern: \emph{Francke} 1956, S. 102.} }\toendnotes[C]{\smallbreak}\pstart
           \raggedleft{}{\pb}\textcolor{pink}{Wien, XVIII.}{}\ledrightnote{\textcolor{pink}{Sternwartestraße}}{\\}Sternwartestr. 71{\\}12. X. 911\pend
           \pstart{}Lieber und verehrter Herr Brandes,\pend\pstart
           ich habe mich in der Angelegenheit des Frl. \textcolor{blue}{Prozor}{}\ledrightnote{\textcolor{blue}{Grete Prozor}} gleich an die \textcolor{brown}{Neue Freie
                        Presse}{}\ledrightnote{\textcolor{brown}{Neue Freie Presse}} gewendet; hier das \label{K_L02035_1v}\edtext{\textcolor{green}{Resultat}{}\ledrightnote{→\textcolor{green}{Ibsen in Frankreich}}}{\lemma{\textnormal{\emph{Resultat}}}\Cendnote{\textnormal{Ein Interview mit \textcolor{blue}{Grete Prozor} enthält: [O. V.:]: \emph{\textcolor{green}{Ibsen in Frankreich}}. In: \emph{\textcolor{green}{Neue Freie Presse}}, Nr. 16933,
                                12. 10. 1911, Morgenblatt, S. 10.}}}\label{K_L02035_1h}.\pend
           \pstart
           Sie reisen überall hin – nur nach \textcolor{pink}{Wien}{}\ledrightnote{\textcolor{pink}{Wien}} wollen Sie
                    niemals kommen! Nun, vielleicht führt uns der nächste Sommer wieder nordwärts,
                    und man sieht einander wieder. Es freut mich immer so sehr in Ihren Briefen zu
                    lesen, daß Sie meiner {\pb}in Sympathie
                    gedenken;– was Sie, mein verehrter und lieber Freund mir bedeuten – mir schon
                    bedeutet haben, lang eh Sie von meiner Existenz wußten, das fühlen Sie wohl! Nur
                    schade, daß man sich meist an diesem Wissen u Fühlen muß genügen lassen – und in
                    so vielen vielen Jahren innerer Zusa{\geminationm}engehörigkeit
                    keine fünfzig Stunden miteinander verbracht hat!\pend
           \pstart
           – Ich bin nun mit den Proben meiner {\pb}neuen
                    Tragikomödie »\textcolor{green}{das weite Land}{}\ledrightnote{\textcolor{green}{Das weite Land. Tragikomödie in fünf Akten}}« beschäftigt – am
                    Sonntag ist die Première zugleich am \textcolor{pink}{Burgtheater}{}\ledrightnote{\textcolor{pink}{Burgtheater}}, in \textcolor{pink}{Berlin}{}\ledrightnote{\textcolor{pink}{Berlin}}, \textcolor{pink}{München}{}\ledrightnote{\textcolor{pink}{München}}, \textcolor{pink}{Hamburg}{}\ledrightnote{\textcolor{pink}{Hamburg}}, \textcolor{pink}{Frankfurt}{}\ledrightnote{\textcolor{pink}{Frankfurt am Main}} und noch etlichen andern Städten. Sie
                    werden das \textcolor{green}{Buch}{}\ledrightnote{→\textcolor{green}{Das weite Land. Tragikomödie in fünf Akten}} in diesen
                    Tagen \substVorne{}\textsuperscript{haben}\substDazwischen{}beko{\geminationm}en\substHinten{}; hoffentlich werden Sie einige Freude daran haben.\pend
           \pstart
           – Der schwarze Rand dieses Blattes besagt, daß meine \textcolor{blue}{Mutter}{}\ledrightnote{→\textcolor{blue}{Louise Schnitzler}} gestorben ist. Es sind nun fünf
                    Wochen her – nach einer {\pb}Lungenentzündung,
                    von der sie gar nichts verspürte (sie glaubte im Sanatorium eine Mastkur zu
                    gebrauchen,) ist sie ruhig eingeschlafen für ewige Zeit. –\pend
           \pstart
           Leben Sie wohl, erhalten Sie mir Ihre Freundschaft, und lassen Sie uns ein
                    Wiedersehen in guter Gesundheit erhoffen.\pend
           \pstart
           Herzlichst der{\\[\baselineskip]}Ihre{\\[\baselineskip]}\spacefill\mbox{ArthurSchnitzler}\pend
           \leftskip=0em{}\pstart
           \noindent{}Meine \textcolor{blue}{Frau}{}\ledrightnote{→\textcolor{blue}{Olga Schnitzler}} grüßt Sie.
                        Auch sie möchte so gern wieder einmal Georg Brandes sehen!\pend
           \endnumbering\briefempfaengerindex{Brandes, Georg@\textsc{Brandes, Georg}!zzzSchnitzler, Arthur@\emph{von Arthur Schnitzler}!1911-10-121@{12. 10. 1911}|)be}\mylabel{h}  \normalsize

\doendnotes{C}
\bigskip
\vfill

\clearpage

\footnotesize

\lohead{\textsc{register}}

% Definiere theindex-Environment komplett neu ohne reledmac
\makeatletter
\renewenvironment{theindex}{%
  \section*{\indexname}%
  \setlength{\parindent}{0pt}%
  \setlength{\parskip}{0pt plus 0.3pt}%
  \let\item\@idxitem
}{%
  \clearpage
}
\makeatother

\IfFileExists{\jobname-pw.ind}{\input{\jobname-pw.ind}}{}

\end{document}

      