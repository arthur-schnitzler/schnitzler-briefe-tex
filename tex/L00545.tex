%% latex-korrekturansicht-vorspann.tex
%% Vorspann für die Korrekturansicht.
%% Lädt die gemeinsame Datei latex-vorspann.tex mit gesetztem Schalter.

\newif\ifkorrekturansicht
\korrekturansichttrue

\input{../tex-inputs/latex-vorspann}


               \section[Hugo von Hofmannsthal an Arthur Schnitzler, 17. 5. {[}1896{]}]{ Hugo von Hofmannsthal an Arthur Schnitzler, 17. 5. {[}1896{]}}\nopagebreak\mylabel{v}\rehead{ }\normalsize\beginnumbering\briefempfaengerindex{Schnitzler, Arthur@\textsc{Schnitzler, Arthur}!zzzHofmannsthal, Hugo von@\emph{von Hugo von Hofmannsthal}!1896-05-171@{17. 5. {[}1896{]}}|(be} \toendnotes[C]{\smallbreak\pagebreak[2]} \Standort{CUL, Schnitzler, B 43.}
\physDesc{Brief, 1 Blatt (Briefpapier mit aufgeprägtem Wappen), 4 Seiten
\newline{}Handschrift: schwarze Tinte, deutsche Kurrent\newline{}Ordnung: von unbekannter Hand nummeriert:
                                            »1« }\buchAbdrucke{\weitereDrucke{1) Hugo von Hofmannsthal: \emph{Briefe. 1890–1901}. Berlin: \emph{S. Fischer} 1935, S. 192–193.} \weitereDrucke{2) Hugo von Hofmannsthal, Arthur Schnitzler: \emph{Briefwechsel}. Hg. Therese Nickl und Heinrich Schnitzler. Frankfurt am Main: \emph{S. Fischer} 1964, S. 65–66.} \weitereDrucke{3) Hermann Bahr, Arthur Schnitzler: \emph{Briefwechsel, Aufzeichnungen, Dokumente
                                (1891–1931)}. Hg. Kurt Ifkovits und Martin Anton Müller. Göttingen: \emph{Wallstein} 2018, S. 121.} }\toendnotes[C]{\smallbreak}\pstart
           \raggedleft{}{\pb}\textcolor{pink}{\label{K_L00545_1v}\edtext{\textsc{Tłumacz}}{\lemma{\textnormal{\emph{Tłumacz}}}\Cendnote{\textnormal{Hugo von Hofmannsthal leistete im
                                    Mai 1896 seinen Militärdienst in Tłumacz ab.}}}\label{K_L00545_1h}
                            bei \textsc{Stanislau}}{}\ledrightnote{\textcolor{pink}{Tlumatsch}}{ }\textsc{(\textcolor{pink}{Galizien}{}\ledrightnote{\textcolor{pink}{Galizien}})}{\\}\textsc{K. u. K. 8\textsuperscript{tes}
                            Uhlanenregiment}{\\}Sonntag 17\textsuperscript{ten} Mai.\pend
           \pstart{}lieber Arthur!\pend\pstart
           vor einer Woche hat mir meine \textcolor{blue}{Mutter}{}\ledrightnote{→\textcolor{blue}{Anna von Hofmannsthal}} geſchrieben, Sie hätten mit ihr geſprochen und ihr erzählt,
                    daſs im Herbſt wieder ein \strikeout{ein}{ }\textcolor{green}{Stück}{}\ledrightnote{→\textcolor{green}{Freiwild. Schauspiel in 3 Akten}} von Ihnen aufgeführt
                    werden wird. Das hat mich, wie es der Zufall manchmal bringt, ſo »hiſtoriſch«
                    berührt. Die ganze Zeit, ſeit wir uns kennen, iſt mir als ein ganzes
                    eingefallen, wie eine Landſchaft, {\pb}aber viel merkwürdiger: als
                    wenn man in einem Thal ſtünde und durch die Wände der Berge hindurch die andern
                    Thäler gleichzeitig ſehen würde.\pend
           \pstart
           Auch der gute \textcolor{blue}{Goldmann}{}\ledrightnote{\textcolor{blue}{Paul Goldmann}} ist mir ſehr ſtark
                    eingefallen und ſein ſonderbares ſchmerzliches Leben. Es iſt merkwürdig, wie
                    ſtark man an Vergangenes denken kann, wenn man ſo allein und abgeſchnitten lebt,
                    wie ich hier. Mir iſt eingefallen, wie mir der \textcolor{blue}{Goldmann}{}\ledrightnote{\textcolor{blue}{Paul Goldmann}} zum erſten Mal von \textcolor{blue}{Nietzſche}{}\ledrightnote{\textcolor{blue}{Friedrich Nietzsche}} und von \textcolor{blue}{Bahr}{}\ledrightnote{\textcolor{blue}{Hermann Bahr}} erzählt
                    hat, das ganze kleine \label{K_L00545_2v}\edtext{Redactionszimmer}{\lemma{\textnormal{\emph{Redactionszimmer}}}\Cendnote{\textnormal{\textcolor{blue}{Goldmann} war bis 1890
                        Redakteur der Zeitschrift \emph{\textcolor{green}{An der schönen blauen
                            Donau}}, in der \textcolor{blue}{Schnitzler} einige
                        frühe Texte publizierte.}}}\label{K_L00545_2h} und unſre {\pb}erſten Begegnungen, und
                    alles kommt mir ſo unglaublich vergangen vor und ſo nett und altmodiſch wie eine
                    Geſchichte aus der \textcolor{blue}{Jean Paul}{}\ledrightnote{\textcolor{blue}{Jean Paul}}-Zeit. \uline{Wir haben doch in dieſen paar Jahren ſehr viele
                        ſchöne Stunden gehabt.} Wir haben ſehr oft das Leben reich und groß
                    geſehen und waren im Stande, viele Dinge auf einander zu beziehen, und immer hat
                    ſichs wieder verändert, das war das ſchönſte. \label{LL435-1v}Auch daſs wir voneinander nicht gar zu viel wiſſen und
                        immer \strikeout{ein} jeder {\pb}wie ein Neuer aus ſeinem
                        Leben hervortritt und wieder hinein geht, ist ſehr ſchön.\label{LL435-1h}\pend
           \pstart
           Über meinen augenblicklichen Zuſtand will ich lieber nichts erzählen: die Station
                    iſt von einer teufliſchen Häſslichkeit, die Menſchen nicht recht erfreulich, das
                    Wetter fortwährend elend. Ich habe einige Bändchen \textcolor{blue}{Platon}{}\ledrightnote{\textcolor{blue}{Platon}} mit, auch den \textcolor{blue}{Pindar}{}\ledrightnote{\textcolor{blue}{Pindaros}} und den
                    unerſchöpflichen erſten Band von \textcolor{blue}{Goethe}{}\ledrightnote{\textcolor{blue}{Johann Wolfgang von Goethe}}: die
                    Lieder, die Elegien, und die Sprüche. Ich freue mich im ſtillen (wenn auch mit
                    Zweifeln) Ihr neues \textcolor{green}{Stück}{}\ledrightnote{→\textcolor{green}{Freiwild. Schauspiel in 3 Akten}}
                    noch im Juni bei der \textcolor{blue}{Tini}{}\ledrightnote{\textcolor{blue}{Christine Schönberger}} zu
                    hören.\pend
           \pstart
           Herzlich Ihr{\\[\baselineskip]}\spacefill\mbox{Hugo.}\pend
           \leftskip=0em{}\endnumbering\briefempfaengerindex{Schnitzler, Arthur@\textsc{Schnitzler, Arthur}!zzzHofmannsthal, Hugo von@\emph{von Hugo von Hofmannsthal}!1896-05-171@{17. 5. {[}1896{]}}|)be}\mylabel{h}  \normalsize

\doendnotes{C}
\bigskip
\vfill

\clearpage

\footnotesize

\lohead{\textsc{register}}

% Definiere theindex-Environment komplett neu ohne reledmac
\makeatletter
\renewenvironment{theindex}{%
  \section*{\indexname}%
  \setlength{\parindent}{0pt}%
  \setlength{\parskip}{0pt plus 0.3pt}%
  \let\item\@idxitem
}{%
  \clearpage
}
\makeatother

\IfFileExists{\jobname-pw.ind}{\input{\jobname-pw.ind}}{}

\end{document}

      