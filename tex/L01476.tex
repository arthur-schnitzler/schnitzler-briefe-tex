%% latex-korrekturansicht-vorspann.tex
%% Vorspann für die Korrekturansicht.
%% Lädt die gemeinsame Datei latex-vorspann.tex mit gesetztem Schalter.

\newif\ifkorrekturansicht
\korrekturansichttrue

\input{../tex-inputs/latex-vorspann}


               \section[Hugo August von Hofmannsthal an Arthur Schnitzler, 5. 12. 1904]{ Hugo August von Hofmannsthal an Arthur Schnitzler, 5. 12. 1904}\nopagebreak\mylabel{v}\rehead{ }\normalsize\beginnumbering\briefempfaengerindex{Schnitzler, Arthur@\textsc{Schnitzler, Arthur}!zzzHofmannsthal, Hugo August von@\emph{von Hugo August von Hofmannsthal}!1904-12-052@{5. 12. 1904}|(be} \toendnotes[C]{\smallbreak\pagebreak[2]} \Standort{DLA, A:Schnitzler, HS.NZ85.1.3483.}
\physDesc{Brief, 1 Blatt (Briefpapier mit Trauerrand), 2 Seiten
\newline{}Handschrift: schwarze Tinte, deutsche Kurrent
\newline{}Schnitzler: mit rotem Buntstift beschriftet: »\textsc{(\textcolor{blue}{Hugos}
                                       Vater)}« }\toendnotes[C]{\smallbreak}\pstart
           \raggedleft{}{\pb}\textcolor{pink}{Wien}{}\ledrightnote{\textcolor{pink}{Wien}} den 5 December.{\\}1904\pend
           \pstart{}Geehrter Freund!\pend\pstart
           Ich beeile mich Ihnen mitzuteilen, dß ich mich meiner \textsc{diplomatischen Miſsionen} betreffs der \textsc{Tantième} von
               der \label{K_L01476_1v}\edtext{Woltätigkeitsvorſtellung}{\lemma{\textnormal{\emph{Woltätigkeitsvorſtellung}}}\Cendnote{\textnormal{Es handelt sich um den am 12. 12. 1904 stattfindenden
                  »Arthur-Schnitzler-Abend« im \textcolor{pink}{Carl-Theater}. Dieser
                  wurde für das seit 1787 bestehende \emph{\textcolor{brown}{Erste
                     öffentliche Kinderkrankeninstitut}} abgehalten, dessen Leitung \textcolor{blue}{Carl Hochsinger} inne hatte.}}}\label{K_L01476_1h}, geſtern
               pflichtgemäß entledigt habe. Die \textsc{Arrangeure} waren ſehr
               erſchüttert, weil ſie natürlich an den Dichter, der ja bekanntlich von der Luft zu
               leben verpflichtet iſt, nicht gedacht hatten, aber ich habe \strikeout{pf} ihnen den \textcolor{gray}{Stand}pönal klar gemacht. Baron \textcolor{blue}{Haas}{}\ledrightnote{\textcolor{blue}{Philipp von Haas-Teichen}} hat wegen des Ablebens ſeines {\pb}Schwagers \textsc{Grfen \textcolor{blue}{Castell}{}\ledrightnote{\textcolor{blue}{Wilhelm zu Castell-Rüdenhausen}}} abſagen müſſen u D\textsuperscript{r}{ }\textcolor{blue}{\textsc{Hochsinger}}{}\ledrightnote{\textcolor{blue}{Carl Hochsinger}} iſt bemüht mit Hilfe \textsc{\textcolor{blue}{Heine}{}\ledrightnote{\textcolor{blue}{Albert Heine}}s} u \textcolor{blue}{\textsc{Treßler}}{}\ledrightnote{\textcolor{blue}{Otto Tressler}} einen paſſenden Erſatz zu finden.\pend
           \pstart
           Empfehlen Sie mich gütigſt Ihrer \textcolor{blue}{Gnädigen}{}\ledrightnote{→\textcolor{blue}{Olga Schnitzler}} und ſein Sie beſtens gegrüßt von Ihrem\pend
           \pstart
           ergebenſten{\\[\baselineskip]}\spacefill\mbox{D\textsuperscript{r} Hofmannsthal}\pend
           \leftskip=0em{}\endnumbering\briefempfaengerindex{Schnitzler, Arthur@\textsc{Schnitzler, Arthur}!zzzHofmannsthal, Hugo August von@\emph{von Hugo August von Hofmannsthal}!1904-12-052@{5. 12. 1904}|)be}\mylabel{h}  \normalsize

\doendnotes{C}
\bigskip
\vfill

\clearpage

\footnotesize

\lohead{\textsc{register}}

% Definiere theindex-Environment komplett neu ohne reledmac
\makeatletter
\renewenvironment{theindex}{%
  \section*{\indexname}%
  \setlength{\parindent}{0pt}%
  \setlength{\parskip}{0pt plus 0.3pt}%
  \let\item\@idxitem
}{%
  \clearpage
}
\makeatother

\IfFileExists{\jobname-pw.ind}{\input{\jobname-pw.ind}}{}

\end{document}

      