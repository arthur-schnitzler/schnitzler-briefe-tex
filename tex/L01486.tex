%% latex-korrekturansicht-vorspann.tex
%% Vorspann für die Korrekturansicht.
%% Lädt die gemeinsame Datei latex-vorspann.tex mit gesetztem Schalter.

\newif\ifkorrekturansicht
\korrekturansichttrue

\input{../tex-inputs/latex-vorspann}


               \section[Hugo von Hofmannsthal an Arthur Schnitzler, 29. 12. 1904]{ Hugo von Hofmannsthal an Arthur Schnitzler,
               29. 12. 1904}\nopagebreak\mylabel{v}\rehead{ }\normalsize\beginnumbering\briefempfaengerindex{Schnitzler, Arthur@\textsc{Schnitzler, Arthur}!zzzHofmannsthal, Hugo von@\emph{von Hugo von Hofmannsthal}!1904-12-291@{29. 12. 1904}|(be} \toendnotes[C]{\smallbreak\pagebreak[2]} \Standort{CUL, Schnitzler, B 43.}
\physDesc{Postkarte
\newline{}Handschrift: schwarze Tinte, deutsche Kurrent\newline{}Versand: 1) Stempel: »\nobreak{}\oindex{Rodaun@\textbf{Rodaun}, \emph{Teil eines besiedelten Ortes (A.BSOX)}|pwk}Rodaun, 29. 12. 04, \textcolor{gray}{7}–9N\nobreak{}«.  2) Stempel: »\nobreak{}\oindex{XVIII., Waehring@\textbf{XVIII., Währing}, \emph{Bezirk (A.BZK)}|pwk}18/1 Wien 110, 30. 12. 04, 12.V, Bestellt\nobreak{}«. 
\newline{}Schnitzler: mit Bleistift die Jahreszahl ergänzt: »04« \newline{}Ordnung: 1) mit Bleistift von unbekannter Hand nummeriert: »\strikeout{220}« 2) mit Bleistift von unbekannter Hand nummeriert: »245«}\buchAbdrucke{\weitereDrucke{Hugo von Hofmannsthal, Arthur Schnitzler: \emph{Briefwechsel}. Hg. Therese Nickl und Heinrich Schnitzler. Frankfurt am Main: \emph{S. Fischer} 1964, S. 208.} }\toendnotes[C]{\smallbreak}\pstart{}{\pb}\textsc{Herrn D\textsuperscript{r} Arthur Schnitzler}\pend{}\pstart{}\textcolor{pink}{\textsc{Wien}}{}\ledrightnote{\textcolor{pink}{Wien}}\pend{}\pstart{}\textcolor{pink}{\textsc{XVIII. Spöttelgasse 7}}{}\ledrightnote{\textcolor{pink}{Edmund-Weiß-Gasse}}\pend{}{\bigskip}\pstart
           \raggedleft{}{\pb}29 XII.\pend
           \pstart
           lieber, bitte doch gleich um ein Wort wann Sie \label{K_L01486_1v}\edtext{zurück}{\lemma{\textnormal{\emph{zurück}}}\Cendnote{\textnormal{Er war seit 26. 12. 1904 und noch bis
                  zum 30. 12. 1904 in \textcolor{pink}{Lueg am Wolfgangsee}.}}}\label{K_L01486_1h} ſind, damit man ſich
               noch einmal \label{K_L01486_2v}\edtext{ſieht}{\lemma{\textnormal{\emph{ſieht}}}\Cendnote{\textnormal{Er
                  reiste am 8. 1. 1905 nach \textcolor{pink}{Berlin}.}}}\label{K_L01486_2h}. \textcolor{blue}{Richard}{}\ledrightnote{\textcolor{blue}{Richard Beer-Hofmann}} noch nicht
               zurück. – \textcolor{blue}{\textsc{Bassermann}}{}\ledrightnote{\textcolor{blue}{Albert Bassermann}} widerſtrebt der \textcolor{green}{\textsc{Jaffier}}{}\ledrightnote{→\textcolor{green}{Das gerettete Venedig. Trauerspiel in fünf Aufzügen}}{ }ſo ſehr, daſs man ihm die Rolle abnehmen muſs. \textcolor{blue}{Brahm}{}\ledrightnote{\textcolor{blue}{Otto Brahm}} wünſcht ſie \textcolor{blue}{\uline{Grunwald}}{}\ledrightnote{\textcolor{blue}{Willy Grunwald}} zu geben, der ſich heftig darum bewirbt. \textcolor{blue}{Brahm}{}\ledrightnote{\textcolor{blue}{Otto Brahm}} depeſchierte mir, ich ſollte mit Ihnen über \textcolor{blue}{G.}{}\ledrightnote{\textcolor{blue}{Willy Grunwald}} reden.\pend
           \pstart
           Ihr{\\[\baselineskip]}\spacefill\mbox{Hugo.}\pend
           \leftskip=0em{}\endnumbering\briefempfaengerindex{Schnitzler, Arthur@\textsc{Schnitzler, Arthur}!zzzHofmannsthal, Hugo von@\emph{von Hugo von Hofmannsthal}!1904-12-291@{29. 12. 1904}|)be}\mylabel{h}  \normalsize

\doendnotes{C}
\bigskip
\vfill

\clearpage

\footnotesize

\lohead{\textsc{register}}

% Definiere theindex-Environment komplett neu ohne reledmac
\makeatletter
\renewenvironment{theindex}{%
  \section*{\indexname}%
  \setlength{\parindent}{0pt}%
  \setlength{\parskip}{0pt plus 0.3pt}%
  \let\item\@idxitem
}{%
  \clearpage
}
\makeatother

\IfFileExists{\jobname-pw.ind}{\input{\jobname-pw.ind}}{}

\end{document}

      