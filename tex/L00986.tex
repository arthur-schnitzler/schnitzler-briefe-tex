%% latex-korrekturansicht-vorspann.tex
%% Vorspann für die Korrekturansicht.
%% Lädt die gemeinsame Datei latex-vorspann.tex mit gesetztem Schalter.

\newif\ifkorrekturansicht
\korrekturansichttrue

\input{../tex-inputs/latex-vorspann}


               \section[Hugo von Hofmannsthal an Arthur Schnitzler, 2. 10. {[}1899{]}]{ Hugo von Hofmannsthal an Arthur Schnitzler, 2. 10. {[}1899{]}}\nopagebreak\mylabel{v}\rehead{ }\normalsize\beginnumbering\briefempfaengerindex{Schnitzler, Arthur@\textsc{Schnitzler, Arthur}!zzzHofmannsthal, Hugo von@\emph{von Hugo von Hofmannsthal}!1899-10-021@{2. 10. {[}1899{]}}|(be} \toendnotes[C]{\smallbreak\pagebreak[2]} \Standort{CUL, Schnitzler, B 43.}
\physDesc{Brief, 1 Blatt, 4 Seiten
\newline{}Handschrift: schwarze Tinte, deutsche Kurrent\newline{}Ordnung: mit Bleistift von unbekannter Hand nummeriert: »236« }\buchAbdrucke{\weitereDrucke{Hugo von Hofmannsthal, Arthur Schnitzler: \emph{Briefwechsel}. Hg. Therese Nickl und Heinrich Schnitzler. Frankfurt am Main: \emph{S. Fischer} 1964, S. 131–132.} }\toendnotes[C]{\smallbreak}\pstart
           \noindent{}\raggedleft{}{\pb}\textcolor{gray}{\textbf{\textcolor{pink}{Venice}{}\ledrightnote{\textcolor{pink}{Venedig}}}}\pend
           \pstart
           \noindent{}\centering{}\textcolor{gray}{\textbf{\textcolor{pink}{Grand Hôtel
                     Britannia}{}\ledrightnote{\textcolor{pink}{Grand Hotel Britannia}}}}\pend
           \pstart
           \noindent{}\textcolor{gray}{\textbf{Charles Walther}}\hfill \textcolor{gray}{\textbf{Electric light and steamheat in all rooms}}\pend
           \pstart
           \textcolor{gray}{\textbf{Propr.}}\hfill \textcolor{gray}{\textbf{Hydraulic Lifts}}\pend
           \pstart
           \centering{}\textcolor{gray}{\textbf{Mêmes Maisons}}\pend
           \pstart
           \noindent{}\textcolor{gray}{\textbf{\textcolor{pink}{Hôtel
                        Victoria}{}\ledrightnote{\textcolor{pink}{Hotel Victoria}}}}\hfill \textcolor{gray}{\textbf{\textcolor{pink}{Hôtel de la Ville}{}\ledrightnote{\textcolor{pink}{Hotel de la Ville}}}}\pend
           \pstart
           \textcolor{gray}{\textbf{\textcolor{pink}{Bozen}{}\ledrightnote{\textcolor{pink}{Bozen}} (\textcolor{pink}{Tyrol}{}\ledrightnote{\textcolor{pink}{Tirol}})}}\hfill \textcolor{gray}{\textbf{\textcolor{pink}{Genoa}{}\ledrightnote{\textcolor{pink}{Genua}} – \textcolor{pink}{Gênes}{}\ledrightnote{\textcolor{pink}{Genua}} –
                     \textcolor{pink}{Genúa}{}\ledrightnote{\textcolor{pink}{Genua}}}}\pend
           \pstart
           \raggedleft{}\textcolor{gray}{\textbf{Venice}}, den 2\textsuperscript{ten} X.\pend
           \pstart{}mein lieber Arthur\pend\pstart
           was Sie mir ſchreiben, iſt ſo wahr: für die Momente dankbar ſein, in denen man eine
               gewiſſe innere Fülle empfindet. Daſs aber das alles unter ſo furchtbar dunklen
               Geſetzen ſteht und daſs die Starrheit manchmal alles ergreifen {\pb}kann, ſogar die Empfindung für die
               Exiſtenz aller andern Menschen!\pend
           \pstart
           Mit meinem \textcolor{green}{Stück}{}\ledrightnote{→\textcolor{green}{Das Bergwerk zu Falun}} geht es
               ſonderbar. Ich hab in \textcolor{pink}{Vahrn}{}\ledrightnote{\textcolor{pink}{Vahrn}} nochmals einen ganz
               unbrauchbaren 3\textsuperscript{ten} Act gemacht, recht verſchieden von
               dem, den Sie in \textcolor{pink}{Iſchl}{}\ledrightnote{\textcolor{pink}{Bad Ischl}} geſehen haben, und doch
               falsch. Eine ſchlechte Art, die Menſchen und ihr Schickſal anzuſehen. Der Grundfehler
               war, wie ich jetzt weiß, schon im \substVorne{}\textsuperscript{erſten}{\allowbreak}\substDazwischen{}zweiten\substHinten{}{ }\textcolor{green}{Act}{}\ledrightnote{→\textcolor{green}{Das Bergwerk zu Falun}} gelegen. Bin dann hier her gefahren. Wollte ganz
               aufhören, mich abſolut von dem Stoff losmachen. Das war ich aber auch nicht im
               Stande. Habe wieder den 2\textsuperscript{ten}{ }\textcolor{green}{Act}{}\ledrightnote{→\textcolor{green}{Das Bergwerk zu Falun}} vorgeno{\geminationm}en. In dieſer weichen helleren Luft hier {\pb}nimmt alles weichere Formen an;
               ich arbeite wieder mit Freude, die Bekanntſchaft mit den umgeſchmolzenen Figuren
               kommt mir zu Hilfe und ich hoffe hier ſehr raſch weit zu kommen.\pend
           \pstart
           \textcolor{blue}{Brahm}{}\ledrightnote{\textcolor{blue}{Otto Brahm}} will ich in dieſen Tagen ſchreiben. Es
               liegt mir aus weitläufigen Gründen ſehr viel daran, daſs das \textcolor{green}{Stück}{}\ledrightnote{→\textcolor{green}{Das Bergwerk zu Falun}} wenigſtens in einem der Theater noch in
               dieſem Spieljahr drankommt.\pend
           \pstart
           \textcolor{blue}{Richard}{}\ledrightnote{\textcolor{blue}{Richard Beer-Hofmann}}s \textcolor{green}{Stück}{}\ledrightnote{→\textcolor{green}{Der Graf von Charolais. Ein Trauerspiel}} iſt in der Anlage wunderſchön und er arbeitet gar
               nicht langſam, etwa 30–40 Verſe {\pb}im Tag.\hspace*{1.5em}Wie froh bin ich, ſolche Menſchen zu haben
               wie Sie und \textcolor{blue}{Richard}{}\ledrightnote{\textcolor{blue}{Richard Beer-Hofmann}}. Daſs man trotzdem ſo \strikeout{vielfach} oft ſo traurig, oed und ſtarr ſein kann.\pend
           \pstart
           Ich bin vielleicht noch 14 Tage hier. Ko{\geminationm}en Sie nicht
               vorbei und leſen mir zur Ermuthigung was vor?\pend
           \pstart
           Von Herzen Ihr{\\[\baselineskip]}\spacefill\mbox{Hugo.}\pend
           \leftskip=0em{}\endnumbering\briefempfaengerindex{Schnitzler, Arthur@\textsc{Schnitzler, Arthur}!zzzHofmannsthal, Hugo von@\emph{von Hugo von Hofmannsthal}!1899-10-021@{2. 10. {[}1899{]}}|)be}\mylabel{h}  \normalsize

\doendnotes{C}
\bigskip
\vfill

\clearpage

\footnotesize

\lohead{\textsc{register}}

% Definiere theindex-Environment komplett neu ohne reledmac
\makeatletter
\renewenvironment{theindex}{%
  \section*{\indexname}%
  \setlength{\parindent}{0pt}%
  \setlength{\parskip}{0pt plus 0.3pt}%
  \let\item\@idxitem
}{%
  \clearpage
}
\makeatother

\IfFileExists{\jobname-pw.ind}{\input{\jobname-pw.ind}}{}

\end{document}

      