%% latex-korrekturansicht-vorspann.tex
%% Vorspann für die Korrekturansicht.
%% Lädt die gemeinsame Datei latex-vorspann.tex mit gesetztem Schalter.

\newif\ifkorrekturansicht
\korrekturansichttrue

\input{../tex-inputs/latex-vorspann}


               \section[Arthur Schnitzler an Hermann Bahr, 5. 12. 1904]{ Arthur Schnitzler an Hermann Bahr, 5. 12. 1904}\nopagebreak\mylabel{v}\rehead{ }\normalsize\beginnumbering\briefempfaengerindex{Bahr, Hermann@\textsc{Bahr, Hermann}!zzzSchnitzler, Arthur@\emph{von Arthur Schnitzler}!1904-12-051@{5. 12. 1904}|(be} \toendnotes[C]{\smallbreak\pagebreak[2]} \Standort{TMW, HS AM 23369 Ba.}
\physDesc{Brief, 1 Blatt, 4 Seiten
\newline{}Handschrift: schwarze Tinte, deutsche Kurrent\newline{}Ordnung: Lochung }\buchAbdrucke{\weitereDrucke{1) Arthur Schnitzler: \emph{Briefe 1875–1912}. Hg. Therese Nickl und Heinrich Schnitzler. Frankfurt am Main: \emph{S. Fischer} 1981, S. 499.} \weitereDrucke{2) \emph{5. 12. 1904.} In: Arthur Schnitzler: \emph{The Letters of Arthur Schnitzler to Hermann Bahr}. Edited, annotated, and with an introduction, by Donald G.
                        Daviau. Chapel Hill: \emph{The University of North Carolina Press} 1978, S. 86 (University of North Carolina studies in the Germanic languages
                        and literatures, 89).} \weitereDrucke{3) Hermann Bahr, Arthur Schnitzler: \emph{Briefwechsel, Aufzeichnungen, Dokumente (1891–1931)}. Hg. Kurt Ifkovits und Martin Anton Müller. Göttingen: \emph{Wallstein} 2018, S. 327.} }\toendnotes[C]{\smallbreak}\pstart
           \noindent{}\raggedleft{}{\pb}\textcolor{pink}{XVIII \textsc{Spoettelg.} 7}{}\ledrightnote{\textcolor{pink}{Edmund-Weiß-Gasse}}\pend
           \pstart
           \raggedleft{}\textsc{\textcolor{pink}{Wien}{}\ledrightnote{\textcolor{pink}{Wien}}}, 5. 12. 904\pend
           \pstart{}lieber Hermann,\pend\pstart
           dictiren u ſitzen (\label{K_L01475_1v}\edtext{\textcolor{green}{Relief}{}\ledrightnote{→\textcolor{green}{Arthur Schnitzler}}}{\lemma{\textnormal{\emph{Relief}}}\Cendnote{\textnormal{bei \textcolor{blue}{Gustav
                     Gurschner}}}}\label{K_L01475_1h}) und allerlei andres haben mich abgehalten, dich aufzuſuchen und dir die
               vielen Grüße persönlich zu überbringen, die mir, am heftigſten von Frau \textcolor{blue}{\textsc{Eysoldt}}{}\ledrightnote{\textcolor{blue}{Gertrud Eysoldt}}, an dich aufgetragen worden ſind. Hoffentlich können wir dich an einem Abend zu
               Beginn nächſter Woche bei uns ſehen {\pb}und bei dieſer
               Gelegenheit auch über den Weihnachtsausflug reden, zu dem große Luſt vorhanden iſt.
               (Wahrſcheinlich aber würden wir erſt nach dem in jüdischen Kreiſen ſo heiligen Abend
               abfahren.) Da wir ſchon bei den frommen Feſten halten, theile ich dir auch mit, daſs
               ich zum Nicolo den \textcolor{green}{Triſtan-Auszug}{}\ledrightnote{\textcolor{green}{Tristan und Isolde}} beko{\geminationm}en habe, ihn aber {\pb}noch ſpiele wie ein
               Krampus. –\pend
           \pstart
           Laß es dir weiter wohl ſein im neu errungenen Glück der Töne – warum ſuchſt du irgend
               ein Vorgefühl darin? Eine Seligkeit hat genug \strikeout{damit}
               zu thun, wenn ſie ſich ſelbſt bedeutet. – \pend
           \pstart
           Beigeſchloſſen der »\textcolor{green}{Puppenſpieler}{}\ledrightnote{\textcolor{green}{Der Puppenspieler}}«, den \label{K_L01475_2v}\edtext{\textcolor{blue}{Baſſermann}{}\ledrightnote{\textcolor{blue}{Albert Bassermann}} in \textcolor{pink}{Berlin}{}\ledrightnote{\textcolor{pink}{Berlin}}}{\lemma{\textnormal{\emph{Baſſermann in Berlin}}}\Cendnote{\textnormal{\textcolor{blue}{Bassermann} hatte in der Uraufführung am
                     14. 9. 1903 im \textcolor{pink}{Deutschen Theater}
                  die Hauptrolle.}}}\label{K_L01475_2h} wundervoll gegeben haben soll. –\pend
           \pstart
           Auf Wiederſehen und herzliche Grüße {\pb}auch von meiner \textcolor{blue}{Frau}{}\ledrightnote{→\textcolor{blue}{Olga Schnitzler}}.{\\[\baselineskip]}Dein{\\[\baselineskip]}\spacefill\mbox{A.}\pend
           \leftskip=0em{}\endnumbering\briefempfaengerindex{Bahr, Hermann@\textsc{Bahr, Hermann}!zzzSchnitzler, Arthur@\emph{von Arthur Schnitzler}!1904-12-051@{5. 12. 1904}|)be}\mylabel{h}  \normalsize

\doendnotes{C}
\bigskip
\vfill

\clearpage

\footnotesize

\lohead{\textsc{register}}

% Definiere theindex-Environment komplett neu ohne reledmac
\makeatletter
\renewenvironment{theindex}{%
  \section*{\indexname}%
  \setlength{\parindent}{0pt}%
  \setlength{\parskip}{0pt plus 0.3pt}%
  \let\item\@idxitem
}{%
  \clearpage
}
\makeatother

\IfFileExists{\jobname-pw.ind}{\input{\jobname-pw.ind}}{}

\end{document}

      