%% latex-korrekturansicht-vorspann.tex
%% Vorspann für die Korrekturansicht.
%% Lädt die gemeinsame Datei latex-vorspann.tex mit gesetztem Schalter.

\newif\ifkorrekturansicht
\korrekturansichttrue

\input{../tex-inputs/latex-vorspann}


               \section[Hermann Bahr an Arthur Schnitzler, {[}13. 3.? 1901{]}]{ Hermann Bahr an Arthur Schnitzler, {[}13. 3.? 1901{]}}\nopagebreak\mylabel{v}\rehead{ }\normalsize\beginnumbering\briefempfaengerindex{Schnitzler, Arthur@\textsc{Schnitzler, Arthur}!zzzBahr, Hermann@\emph{von Hermann Bahr}!1901-03-131@{{[}13. 3.? 1901{]}}|(be} \toendnotes[C]{\smallbreak\pagebreak[2]} \Standort{CUL, Schnitzler, B 5b.}
\physDesc{Brief, 1 Blatt, 1 Seite
\newline{}Handschrift: schwarze Tinte, deutsche Kurrent
\newline{}Schnitzler: mit Bleistift ergänztes Datum: »Feber? 901« \newline{}Ordnung: mit Bleistift von unbekannter Hand nummeriert:
                                    »74« }\buchAbdrucke{\weitereDrucke{Hermann Bahr, Arthur Schnitzler: \emph{Briefwechsel, Aufzeichnungen, Dokumente (1891–1931)}. Hg. Kurt Ifkovits und Martin Anton Müller. Göttingen: \emph{Wallstein} 2018, S. 201.} }\toendnotes[C]{\smallbreak}\pstart
           \noindent{}\centering{}{\pb}\textcolor{gray}{\textbf{\textcolor{brown}{Redaktion des Neuen Wiener Tagblatt}{}\ledrightnote{\textcolor{brown}{Neues Wiener Tagblatt}}}}\pend
           \pstart
           \noindent{}\centering{}\textcolor{gray}{\textbf{\textsc{\textcolor{pink}{Wien, I., Rothenturmstrasse,
                        Steyrerhof}{}\ledrightnote{\textcolor{pink}{Steyrerhof}}.}}}\pend
           \pstart
           \noindent{}\centering{}\textcolor{gray}{\textbf{Telegramm-Adresse: \textcolor{brown}{Tagblatt}{}\ledrightnote{\textcolor{brown}{Neues Wiener Tagblatt}},
                        \textcolor{pink}{Steyrerhof, Wien}{}\ledrightnote{\textcolor{pink}{Steyrerhof}}. – Telephon Nr. 384.
                     Staats-Telephon Nr. 36.}}\pend
           \pstart
           Mittwoch\pend
           \pstart\center{}Lieber Arthur!\pend\pstart
           Ich bin morgen Vormittag heraußen, doch nur bis 12, wo ich ins \label{K_L01102_1v}\edtext{\textcolor{pink}{Künſtlerhaus}{}\ledrightnote{\textcolor{pink}{Künstlerhaus}}}{\lemma{\textnormal{\emph{Künſtlerhaus}}}\Cendnote{\textnormal{Am Samstag, 16. 3. 1901,
                  eröffnete die 23. Jahresausstellung.}}}\label{K_L01102_1h} muß. Freitag,
                  Samſtag unbeſtimmt, wegen \label{K_L01102_2v}\edtext{\textcolor{pink}{Seceſſion}{}\ledrightnote{\textcolor{pink}{Secession}}}{\lemma{\textnormal{\emph{Seceſſion}}}\Cendnote{\textnormal{Am Freitag, 15. 3. 1901,
                  eröffnete die 10. Jahresausstellung.}}}\label{K_L01102_2h}. Ganz sicher Dienſtag,
               den ganzen Vormittag. Haſt Du was Dringendes, ſo morgen oder Samſtag um
               6 in meiner \textcolor{pink}{Redaction}{}\ledrightnote{\textcolor{pink}{Steyrerhof}}.\pend
           \pstart
           Herzlichſt{\\[\baselineskip]}Dein{\\[\baselineskip]}\spacefill\mbox{Hermann}\pend
           \leftskip=0em{}\endnumbering\briefempfaengerindex{Schnitzler, Arthur@\textsc{Schnitzler, Arthur}!zzzBahr, Hermann@\emph{von Hermann Bahr}!1901-03-131@{{[}13. 3.? 1901{]}}|)be}\mylabel{h}  \normalsize

\doendnotes{C}
\bigskip
\vfill

\clearpage

\footnotesize

\lohead{\textsc{register}}

% Definiere theindex-Environment komplett neu ohne reledmac
\makeatletter
\renewenvironment{theindex}{%
  \section*{\indexname}%
  \setlength{\parindent}{0pt}%
  \setlength{\parskip}{0pt plus 0.3pt}%
  \let\item\@idxitem
}{%
  \clearpage
}
\makeatother

\IfFileExists{\jobname-pw.ind}{\input{\jobname-pw.ind}}{}

\end{document}

      