%% latex-korrekturansicht-vorspann.tex
%% Vorspann für die Korrekturansicht.
%% Lädt die gemeinsame Datei latex-vorspann.tex mit gesetztem Schalter.

\newif\ifkorrekturansicht
\korrekturansichttrue

\input{../tex-inputs/latex-vorspann}


               \section[Arthur Schnitzler an Robert Adam, 4. 11. 1918]{ Arthur Schnitzler an Robert Adam, 4. 11. 1918}\nopagebreak\mylabel{v}\rehead{ }\normalsize\beginnumbering\briefempfaengerindex{Adam, Robert@\textsc{Adam, Robert}!zzzSchnitzler, Arthur@\emph{von Arthur Schnitzler}!1918-11-041@{4. 11. 1918}|(be} \toendnotes[C]{\smallbreak\pagebreak[2]} \Standort{DLA, 96.34.2/14.}
\physDesc{Postkarte
\newline{}Handschrift: Bleistift, lateinische Kurrent\newline{}Versand: Stempel: »\nobreak{}{[}4. 11. 19{]}18, 4\nobreak{}«.  }\pstart{}{\pb}Dr. Arth Schnitzler\pend{}\pstart{}\textcolor{pink}{Wien XVIII.}{}\ledrightnote{\textcolor{pink}{XVIII., Währing}}\pend{}\pstart{}\textcolor{pink}{Sternwartestr 71}{}\ledrightnote{\textcolor{pink}{Sternwartestraße}}.\pend{}{\bigskip}\pstart{}Hrn Dr. Robert Adam\pend{}\pstart{}Pollak\pend{}\pstart{}\textcolor{pink}{Wien XII}{}\ledrightnote{\textcolor{pink}{XII., Meidling}}\pend{}\pstart{}\textcolor{pink}{Meidlinger Hptstr 58}{}\ledrightnote{\textcolor{pink}{Meidlinger Hauptstraße}}\pend{}{\bigskip}\pstart
           {\pb}\textcolor{gray}{\textbf{A. S.}}\hfill 4. 11. 18\pend
           \pstart{}Verehrter Herr Doktor,\pend\pstart
           passt es Ihnen so ist mir Ihr freundlicher Besuch am Freitag{ }Abend, gegen ½ 7 sehr willkommen.\pend
           \pstart
           Herzlichst grüßt Sie{\\[\baselineskip]}Ihr erg\spacefill\mbox{A. S.}\pend
           \leftskip=0em{}\endnumbering\briefempfaengerindex{Adam, Robert@\textsc{Adam, Robert}!zzzSchnitzler, Arthur@\emph{von Arthur Schnitzler}!1918-11-041@{4. 11. 1918}|)be}\mylabel{h}  \normalsize

\doendnotes{C}
\bigskip
\vfill

\clearpage

\footnotesize

\lohead{\textsc{register}}

% Definiere theindex-Environment komplett neu ohne reledmac
\makeatletter
\renewenvironment{theindex}{%
  \section*{\indexname}%
  \setlength{\parindent}{0pt}%
  \setlength{\parskip}{0pt plus 0.3pt}%
  \let\item\@idxitem
}{%
  \clearpage
}
\makeatother

\IfFileExists{\jobname-pw.ind}{\input{\jobname-pw.ind}}{}

\end{document}

      