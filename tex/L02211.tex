%% latex-korrekturansicht-vorspann.tex
%% Vorspann für die Korrekturansicht.
%% Lädt die gemeinsame Datei latex-vorspann.tex mit gesetztem Schalter.

\newif\ifkorrekturansicht
\korrekturansichttrue

\input{../tex-inputs/latex-vorspann}


               \section[Richard Beer-Hofmann an Arthur Schnitzler, 29. 6. 1915]{ Richard Beer-Hofmann an Arthur Schnitzler, 29. 6. 1915}\nopagebreak\mylabel{v}\rehead{ }\normalsize\beginnumbering\briefempfaengerindex{Schnitzler, Arthur@\textsc{Schnitzler, Arthur}!zzzBeer-Hofmann, Richard@\emph{von Richard Beer-Hofmann}!1915-06-292@{29. 6. 1915}|(be} \toendnotes[C]{\smallbreak\pagebreak[2]} \Standort{CUL, Schnitzler, B 8.}
\physDesc{Bildpostkarte
\newline{}Handschrift: Bleistift, lateinische Kurrent\newline{}Versand: Stempel: »\nobreak{}\oindex{Bad Ischl@\textbf{Bad Ischl}, \emph{Besiedelter Ort (A.BSO)}|pwk}Bad Ischl 1, 29 VI 15, 2\nobreak{}«.  \newline{}Ordnung: mit Bleistift von unbekannter Hand nummeriert: »260« }\buchAbdrucke{\weitereDrucke{Arthur Schnitzler, Richard Beer-Hofmann: \emph{Briefwechsel 1891–1931}. Hg. Konstanze Fliedl. Wien, Zürich: \emph{Europaverlag} 1992, S. 221.} }\toendnotes[C]{\smallbreak}\pstart{}{\pb}S. H.\pend{}\pstart{}Herrn\pend{}\pstart{}D\textsuperscript{r} Arthur Schnitzler\pend{}\pstart{}\textcolor{pink}{Wien XVIII}{}\ledrightnote{\textcolor{pink}{XVIII., Währing}}\pend{}\pstart{}\textcolor{pink}{Sternwartestrasse 71}{}\ledrightnote{\textcolor{pink}{Sternwartestraße}}\pend{}{\bigskip}\pstart
           \noindent{}\centering{}{\pb}\textcolor{gray}{\textbf{\textcolor{pink}{Salzkammergut}{}\ledrightnote{\textcolor{pink}{Salzkammergut}}.\hspace*{1.5em}\textcolor{pink}{Bad Ischl}{}\ledrightnote{\textcolor{pink}{Bad Ischl}}.}}\pend
           \pstart
           \raggedleft{}{\pb}29/VI 15\pend
           \pstart
           Lieber Arthur! Ich wollte zu Ihnen, aber \textcolor{blue}{Kaufmann}{}\ledrightnote{\textcolor{blue}{Arthur Kaufmann}} sagte mir, Sie wären auf dem \textcolor{pink}{Se{\geminationm}ering}{}\ledrightnote{\textcolor{pink}{Semmering}}. So wünsche ich Ihnen und
                  \textcolor{blue}{Olga}{}\ledrightnote{\textcolor{blue}{Olga Schnitzler}} den schönsten So{\geminationm}er – bringt er Sie nicht doch noch hieher? Bitte
               schreiben Sie mir gelegentlich D\textsuperscript{r} \textcolor{blue}{Reiks}{}\ledrightnote{\textcolor{blue}{Theodor Reik}} Adresse ich muss ihm noch für einen zugesandten \label{K_L02211_1v}\edtext{Aufsatz}{\lemma{\textnormal{\emph{Aufsatz}}}\Cendnote{\textnormal{nicht ermittelt}}}\label{K_L02211_1h} danken\pend
           \endnumbering\briefempfaengerindex{Schnitzler, Arthur@\textsc{Schnitzler, Arthur}!zzzBeer-Hofmann, Richard@\emph{von Richard Beer-Hofmann}!1915-06-292@{29. 6. 1915}|)be}\mylabel{h}  \normalsize

\doendnotes{C}
\bigskip
\vfill

\clearpage

\footnotesize

\lohead{\textsc{register}}

% Definiere theindex-Environment komplett neu ohne reledmac
\makeatletter
\renewenvironment{theindex}{%
  \section*{\indexname}%
  \setlength{\parindent}{0pt}%
  \setlength{\parskip}{0pt plus 0.3pt}%
  \let\item\@idxitem
}{%
  \clearpage
}
\makeatother

\IfFileExists{\jobname-pw.ind}{\input{\jobname-pw.ind}}{}

\end{document}

      