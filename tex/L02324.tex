%% latex-korrekturansicht-vorspann.tex
%% Vorspann für die Korrekturansicht.
%% Lädt die gemeinsame Datei latex-vorspann.tex mit gesetztem Schalter.

\newif\ifkorrekturansicht
\korrekturansichttrue

\input{../tex-inputs/latex-vorspann}


               \section[Arthur Schnitzler an Robert Adam, 5. 8. 1919]{ Arthur Schnitzler an Robert Adam, 5. 8. 1919}\nopagebreak\mylabel{v}\rehead{ }\normalsize\beginnumbering\briefempfaengerindex{Adam, Robert@\textsc{Adam, Robert}!zzzSchnitzler, Arthur@\emph{von Arthur Schnitzler}!1919-08-051@{5. 8. 1919}|(be} \toendnotes[C]{\smallbreak\pagebreak[2]} \Standort{DLA, 96.34.2/18.}
\physDesc{Postkarte
\newline{}Handschrift: schwarze Tinte, deutsche Kurrent\newline{}Versand: 1) zuerst nachgesandt nach \textcolor{pink}{Karlsbad}, \textcolor{pink}{Beamtenkurhaus}, dann zurück nach \textcolor{pink}{Wien} in die \textcolor{pink}{Meidlinger Hauptstraße 58} 2) Stempel: »\nobreak{}\oindex{XVIII., Waehring@\textbf{XVIII., Währing}, \emph{Bezirk (A.BZK)}|pwk}18\textsubscript{1} Wien
                                        110, 5. VIII. 19, 7\nobreak{}«. }\toendnotes[C]{\smallbreak}\pstart{}{\pb}A. S. \textcolor{pink}{Wien XVIII, \textsc{Sternwartestr} 71}{}\ledrightnote{\textcolor{pink}{Sternwartestraße}}\pend{}{\bigskip}\pstart{}Herrn \textsc{Dr. Robert Adam}\pend{}\pstart{}\textsc{Pollak}\pend{}\pstart{}Landes\textcolor{gray}{gerichtsrat}h\pend{}\pstart{}\textcolor{pink}{\textsc{Wien} XII}{}\ledrightnote{\textcolor{pink}{XII., Meidling}}.\pend{}\pstart{}\textcolor{pink}{\textsc{Meidlinger Hptstr} 52}{}\ledrightnote{\textcolor{pink}{Meidlinger Hauptstraße}}. \pend{}{\bigskip}\pstart
           \raggedleft{}{\pb}5. 8. 1919\pend
           \pstart
           Verehrter Herr Doktor, vielen Dank für Ihre liebe Karte aus \textcolor{pink}{Karlsbad}{}\ledrightnote{\textcolor{pink}{Karlsbad}}. Wie lange hab ich ſchon nichts von
                    Ihnen gehört! Morgen fahr ich auf ein paar Tage oder Wochen (je nachdem ob ich
                    mich dort wohl fühle) nach \textcolor{pink}{Reichenau}{}\ledrightnote{\textcolor{pink}{Reichenau an der Rax}}, wo ſich
                        \textcolor{blue}{Frau}{}\ledrightnote{→\textcolor{blue}{Olga Schnitzler}} u \textcolor{blue}{Tochter}{}\ledrightnote{→\textcolor{blue}{Lili Schnitzler}} ſeit 14 Tagen
                    befinden. Mein \textcolor{blue}{Sohn}{}\ledrightnote{→\textcolor{blue}{Heinrich Schnitzler}}
                    begleitet mich. Bitte laſſen Sie michs wiſſen, ſobald Sie {\pb}wieder in
                        \textcolor{pink}{Wien}{}\ledrightnote{\textcolor{pink}{Wien}} ſind. Haben Sie aus dem \textcolor{pink}{Volkstheater}{}\ledrightnote{\textcolor{pink}{Volkstheater}} was neues erfahren? Intereſſe iſt vorhanden,
                    beſonders bei \textcolor{blue}{Roſenthal}{}\ledrightnote{\textcolor{blue}{Friedrich Rosenthal}}. Auf recht bald
                    alſo.\pend
           \pstart
           Herzlichſt grüßt Sie Ihr{\\[\baselineskip]}\spacefill\mbox{Arthur Schnitzler}\pend
           \leftskip=0em{}\endnumbering\briefempfaengerindex{Adam, Robert@\textsc{Adam, Robert}!zzzSchnitzler, Arthur@\emph{von Arthur Schnitzler}!1919-08-051@{5. 8. 1919}|)be}\mylabel{h}  \normalsize

\doendnotes{C}
\bigskip
\vfill

\clearpage

\footnotesize

\lohead{\textsc{register}}

% Definiere theindex-Environment komplett neu ohne reledmac
\makeatletter
\renewenvironment{theindex}{%
  \section*{\indexname}%
  \setlength{\parindent}{0pt}%
  \setlength{\parskip}{0pt plus 0.3pt}%
  \let\item\@idxitem
}{%
  \clearpage
}
\makeatother

\IfFileExists{\jobname-pw.ind}{\input{\jobname-pw.ind}}{}

\end{document}

      