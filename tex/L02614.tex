%% latex-korrekturansicht-vorspann.tex
%% Vorspann für die Korrekturansicht.
%% Lädt die gemeinsame Datei latex-vorspann.tex mit gesetztem Schalter.

\newif\ifkorrekturansicht
\korrekturansichttrue

\input{../tex-inputs/latex-vorspann}


               \section[Paul Goldmann an Arthur Schnitzler, 21. 9. {[}1894{]}]{ Paul Goldmann an Arthur Schnitzler, 21. 9. {[}1894{]}}\nopagebreak\mylabel{v}\rehead{ }\normalsize\beginnumbering\briefempfaengerindex{Schnitzler, Arthur@\textsc{Schnitzler, Arthur}!zzzGoldmann, Paul@\emph{von Paul Goldmann}!1894-09-211@{21. 9. {[}1894{]}}|(be} \toendnotes[C]{\smallbreak\pagebreak[2]} \Standort{DLA, A:Schnitzler, HS.NZ85.1.3164.}
\physDesc{Brief, 2 Blätter, 7 Seiten
\newline{}Handschrift: schwarze Tinte, deutsche Kurrent
\newline{}Schnitzler: 1) mit Bleistift auf dem ersten Blatt die Jahreszahl
                                       »94« vermerkt 2) mit rotem Buntstift drei Unterstreichungen}\toendnotes[C]{\smallbreak}\pstart
           \noindent{}{\pb}\textcolor{gray}{\textbf{\textcolor{brown}{Frankfurter Zeitung}{}\ledrightnote{\textcolor{brown}{Frankfurter Zeitung}}.}}\hfill \textsc{\textcolor{pink}{Paris}{}\ledrightnote{\textcolor{pink}{Paris}}}, 21. September.\pend
           \pstart
           \textcolor{gray}{\textbf{(\textcolor{brown}{Gazette de
                  Francfort}{}\ledrightnote{\textcolor{brown}{Frankfurter Zeitung}}.)}}\pend
           \pstart
           \textcolor{gray}{\textbf{\begin{otherlanguage}{french}Fondateur\end{otherlanguage}{ }\textbf{M. \textcolor{blue}{L.
                  Sonnemann}{}\ledrightnote{\textcolor{blue}{Leopold Sonnemann}}}.}}\pend
           \pstart
           \textcolor{gray}{\textbf{\begin{otherlanguage}{french}Journal politique,
                        financier,\end{otherlanguage}}}\pend
           \pstart
           \textcolor{gray}{\textbf{\begin{otherlanguage}{french}commercial et
                     littéraire.\end{otherlanguage}}}\pend
           \pstart
           \textcolor{gray}{\textbf{\begin{otherlanguage}{french}\textbf{Paraissant trois fois
                           par jour}\end{otherlanguage}}}.\pend
           \pstart
           \textcolor{gray}{\textbf{–}}\pend
           \pstart
           \textcolor{gray}{\textbf{\begin{otherlanguage}{french}\textbf{Bureaux à \textcolor{pink}{Paris}{}\ledrightnote{\textcolor{pink}{Paris}}:}\end{otherlanguage}}}\pend
           \pstart
           \textcolor{gray}{\textbf{\begin{otherlanguage}{french}\textcolor{pink}{\textbf{24. Rue Feydeau}}{}\ledrightnote{\textcolor{pink}{rue Feydeau}}.\end{otherlanguage}}}\pend
           \pstart\center{}Mein lieber Freund,\pend\pstart
           Ich bin dieſer Tage nach \textsc{\textcolor{pink}{Paris}{}\ledrightnote{\textcolor{pink}{Paris}}} zurückgekehrt. Die \textcolor{pink}{Frankfurt}{}\ledrightnote{\textcolor{pink}{Frankfurt am Main}}er
               Zeit war auch recht ſchön. Die Meinigen haben gewetteifert, mir den Aufenthalt
               angenehm zu machen\strikeout{,} und \strikeout{mich} mir das Heimathsgefühl zu geben. Sie laſſen Dich Alle vielmals grüßen.
               Mein \textcolor{blue}{Onkel}{}\ledrightnote{→\textcolor{blue}{Fedor Mamroth}} iſt dieſer Tage auf
               Urlaub gegangen. Wenn er zurückkommt, wirſt Du die erſten \label{K_L02614-2v}\edtext{Bücher zur Beſprechung}{\lemma{\textnormal{\emph{Bücher zur Beſprechung}}}\Cendnote{\textnormal{XXXX}}}\label{K_L02614-2h} erhalten. Thu mir den einzigen Gefallen und
               ſtell’ Dir die Sache nicht {\pb}ſo ſchwer vor. Was Dich
               erſchreckt, iſt lediglich eine mechaniſche Schwierigkeit. Man trainirt ſich zum
               Bücherbeſprechen, wie zu jedem andern Ding. Es handelt sich nur darum, ſich mit der
               nöthigen Sicherheit zum Schreibtiſch zu ſetzen und anzufangen. Der Stoff erſcheint
               Anfangs nicht zu bewältigen, aber im Schreiben tritt das Weſentliche \substVorne{}\textsuperscript{klar}\substDazwischen{}klar\substHinten{} hervor, und das übrige ſällt ab. Du ſollſt ja auch mir \strikeout{d} über die Bücher referiren und nicht ein
               gerichtsordnungsmäßiges Protocoll {\pb}davon geben.
               Deine \label{K_L02614-3v}\edtext{Pſeudonymitäts-Wünſche}{\lemma{\textnormal{\emph{Pſeudonymitäts-Wünſche}}}\Cendnote{\textnormal{XXXX}}}\label{K_L02614-3h} wirſt Du meinem \textcolor{blue}{Onkel}{}\ledrightnote{→\textcolor{blue}{Fedor Mamroth}} bei Überſendung des erſten Feuilletons\textcolor{red}{\textsuperscript{\textbf{KEY}}} mittheilen. Ich habe ſie ihm
               bisher \strikeout{\textcolor{gray}{m}} verſchwiegen, weil ich
               nicht wollte, daß er Dich jetzt ſchon zögern ſehe.\pend
           \pstart
           Die \textsc{20 fl.} haben bei der Einwechſelung \textsc{40 fr. 40 ct} ergeben. Das Abonnement auf das »\textcolor{brown}{Journal}{}\ledrightnote{\textcolor{brown}{Le Journal}}« hat \textsc{10 fr.}
               gekoſtet. Du haſt alſo \textsc{30 fr. 40 ct.} bei mir gut,
               und ich ſehe Deinen Aufträgen entgegen. Dein Abonnement läuft vom 1. \textsc{Oct}. Ich habe aber gebeten, daß {\pb}Du das \textcolor{green}{Blatt}{}\ledrightnote{→\textcolor{green}{Le Journal}} bereits von Montag ab
               erhältſt. \strikeout{Theile} Theile mir mit, ob die Zuſendung
               regelmäßig erfolgt.\pend
           \pstart
           Gestern ist \textsc{\textcolor{blue}{Herzl}{}\ledrightnote{\textcolor{blue}{Theodor Herzl}}}{ }\label{K_L02614-1v}\edtext{zurückgekommen}{\lemma{\textnormal{\emph{zurückgekommen}}}\Cendnote{\textnormal{Dieser war auch in \textcolor{pink}{Ischl},
                     siehe A. S.: \emph{Tagebuch}, 31. 8. 1894}}}\label{K_L02614-1h}. Er war
               bei mir und hat mir erzählt, er habe ſich insbeſondern mit \textsc{\textcolor{blue}{Burckhardt}{}\ledrightnote{\textcolor{blue}{Max Eugen Burckhard}}} angefreundet. Dieſen habe er
               vor Allem auf Dich aufmerkſam gemacht. \textsc{\textcolor{blue}{B.}{}\ledrightnote{\textcolor{blue}{Max Eugen Burckhard}}} ſcheine ſehr geneigt, Dich zu \label{K_L02614-4v}\edtext{ſpielen}{\lemma{\textnormal{\emph{ſpielen}}}\Cendnote{\textnormal{XXXX}}}\label{K_L02614-4h}, ſobald Du nur irgend etwas \textcolor{brown}{Burgtheater}{}\ledrightnote{\textcolor{brown}{Burgtheater}}mäßiges hätteſt. Inzwiſchen habe \textsc{\textcolor{blue}{Herzl}{}\ledrightnote{\textcolor{blue}{Theodor Herzl}}} gerathen, Dir \label{K_L02614-5v}\edtext{Bearbeitungen {\pb}aus dem Franzöſiſchen}{\lemma{\textnormal{\emph{Bearbeitungen … Franzöſiſchen}}}\Cendnote{\textnormal{XXXX}}}\label{K_L02614-5h} zu übertragen. \textsc{\textcolor{blue}{B.}{}\ledrightnote{\textcolor{blue}{Max Eugen Burckhard}}} werde Dich vielleicht den \label{K_L02614-6v}\edtext{\textsc{\textcolor{blue}{Marivaux}{}\ledrightnote{\textcolor{blue}{Pierre Carlet de Marivaux}}} überſetzen}{\lemma{\textnormal{\emph{Marivaux überſetzen}}}\Cendnote{\textnormal{XXXX}}}\label{K_L02614-6h} laſſen \textsc{etc.}{ }\textsc{\textcolor{blue}{Herzl}{}\ledrightnote{\textcolor{blue}{Theodor Herzl}}} ſelbſt will ein \label{K_L02614-22v}\edtext{dreiaktiges \textcolor{green}{Luſtſpiel}{}\ledrightnote{→\textcolor{green}{Unser Käthchen. Lustspiel in 4 Acten}}}{\lemma{\textnormal{\emph{dreiaktiges Luſtſpiel}}}\Cendnote{\textnormal{nicht identifiziert.
                  Eventuell könnte das 1898 fertiggestellte Lustspiel \emph{\textcolor{green}{Unser Käthchen}} gemeint sein, an dem \textcolor{blue}{Herzl}{ }1891 zu
                  arbeiten begonnen hatte.}}}\label{K_L02614-22h} ſchreiben, von dem er bereits zwei Akte liegen
               hat.\pend
           \pstart
           Und was machſt Du? Geht das \textcolor{green}{Stück}{}\ledrightnote{→\textcolor{green}{Liebelei. Schauspiel in drei Akten}}
               vorwärts? Fühlſt Du Dich wohl in \textcolor{pink}{Wien}{}\ledrightnote{\textcolor{pink}{Wien}}? Iſt \textsc{\textcolor{blue}{Richard}{}\ledrightnote{\textcolor{blue}{Richard Beer-Hofmann}}} abgereiſt und
               wohin? Was hört man von der neuen \textsc{\textcolor{brown}{Revue}{}\ledrightnote{→\textcolor{brown}{Die Zeit. Wiener Wochenschrift}}}?\pend
           \pstart
           {\pb}Ich freue mich darauf, bald einen Brief von Dir zu
               erhalten. Bin ſonſt recht lebensmüde. Ich ſehe, daß ich auf einem falſchen Wege bin,
               daß ich nicht mehr hierher zurückkehren durſte. Die Arbeit iſt mir zuwider. Ich
               möchte gern nachkommen und kann keinen Schritt thun. So ſühle ich mich zurückbeiben.
               Und da mir dies das Herz zereißt, ſo glaube ich, daß das unmöglich ein normales Ende
               nehmen kann.\pend
           \pstart
           {\pb}Sei von Herzen gegrüßt, mein lieber Arthur. Es war
               ſo ſchön bei \label{K_L02614-7v}\edtext{Euch}{\lemma{\textnormal{\emph{Euch}}}\Cendnote{\textnormal{im Urlaub in \textcolor{pink}{Bad Ischl}}}}\label{K_L02614-7h}, und es iſt gar ſchwer, nach alledem wieder in \textsc{\textcolor{pink}{Paris}{}\ledrightnote{\textcolor{pink}{Paris}}} zu leben.\pend
           \pstart
           In Treue{\\[\baselineskip]} Dein{\\[\baselineskip]}\spacefill\mbox{Paul Goldmann.}\pend
           \leftskip=0em{}\pstart
           \noindent{}Bitte, empfiehl mich dem Fräulein \textsc{\textcolor{blue}{Sandrock}{}\ledrightnote{\textcolor{blue}{Adele Sandrock}}}, wenn Du dazu einmal Gelegenheit haſt, und
                     \strikeout{\textcolor{gray}{zwarr}} zwar recht
                  herzlich.\pend
           \endnumbering\briefempfaengerindex{Schnitzler, Arthur@\textsc{Schnitzler, Arthur}!zzzGoldmann, Paul@\emph{von Paul Goldmann}!1894-09-211@{21. 9. {[}1894{]}}|)be}\mylabel{h}\begin{anhang}\end{anhang}\normalsize

\doendnotes{C}
\bigskip
\vfill

\clearpage

\footnotesize

\lohead{\textsc{register}}

% Definiere theindex-Environment komplett neu ohne reledmac
\makeatletter
\renewenvironment{theindex}{%
  \section*{\indexname}%
  \setlength{\parindent}{0pt}%
  \setlength{\parskip}{0pt plus 0.3pt}%
  \let\item\@idxitem
}{%
  \clearpage
}
\makeatother

\IfFileExists{\jobname-pw.ind}{\input{\jobname-pw.ind}}{}

\end{document}

      