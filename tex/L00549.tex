%% latex-korrekturansicht-vorspann.tex
%% Vorspann für die Korrekturansicht.
%% Lädt die gemeinsame Datei latex-vorspann.tex mit gesetztem Schalter.

\newif\ifkorrekturansicht
\korrekturansichttrue

\input{../tex-inputs/latex-vorspann}


               \section[Arthur Schnitzler an Richard Beer-Hofmann, {[}4. 6. 1896?{]}]{ Arthur Schnitzler an Richard Beer-Hofmann, {[}4. 6. 1896?{]}}\nopagebreak\mylabel{v}\rehead{ }\normalsize\beginnumbering\briefempfaengerindex{Beer-Hofmann, Richard@\textsc{Beer-Hofmann, Richard}!zzzSchnitzler, Arthur@\emph{von Arthur Schnitzler}!1896-06-041@{{[}4. 6. 1896?{]}}|(be} \toendnotes[C]{\smallbreak\pagebreak[2]} \Standort{YCGL, MSS 31.}
\physDesc{Brief, 1 Blatt, 4 Seiten, Umschlag
\newline{}Handschrift: Bleistift, deutsche Kurrent\newline{}Versand: ohne postalischen Übermittlungsvermerk }\buchAbdrucke{\weitereDrucke{Arthur Schnitzler, Richard Beer-Hofmann: \emph{Briefwechsel 1891–1931}. Hg. Konstanze Fliedl. Wien, Zürich: \emph{Europaverlag} 1992, S. 91.} }\toendnotes[C]{\smallbreak}\pstart{}{\pb}\textsc{Herrn Dr Rich. Beer-Hofmann}\pend{}\pstart{}\textsc{\textcolor{pink}{Wien}{}\ledrightnote{\textcolor{pink}{Wien}}.}\pend{}\pstart{}\textsc{\textcolor{pink}{I. Wollzeile 15}{}\ledrightnote{\textcolor{pink}{Wollzeile}}}.\pend{}{\bigskip}\pstart
           \noindent{}{\pb}\textcolor{gray}{\textbf{\label{T_L00549-v}\edtext{A S}{\lemma{\textnormal{\emph{A S}}}\Cendnote{\textnormal{Prägedruck}}}\label{T_L00549-h}}}\hfill Do{\geminationn}erſ\textcolor{gray}{tg}\pend
           \pstart{}Lieber Richard,\pend\pstart
           alſo wo nachtmahl ich heute – warten Sie –\pend
           \pstart
           Ich werde vielleicht um, resp nach 7 bei Ihnen anläuten, ja? Weiter als
               bis in den \label{K_L00549_1v}\edtext{\textcolor{pink}{Prater}{}\ledrightnote{\textcolor{pink}{Prater}}}{\lemma{\textnormal{\emph{Prater}}}\Cendnote{\textnormal{undatiert. Als ›wahrscheinlichster‹ Tag bietet sich der 4. 6. 1896 an, da an diesem Tag \textcolor{blue}{Schnitzler} und \textcolor{blue}{Beer-Hofmann} im \textcolor{pink}{Prater} essen. Ein
                  Aufenthalt \textcolor{blue}{Hofmannsthal}s bei \textcolor{blue}{Christine Schönberger} lässt sich für diesen Tag nicht
                  belegen.}}}\label{K_L00549_1h} wird man ſich ja doch nicht {\pb}wagen
               können, ſelbſt we{\geminationn} es ganz ſchön wird. Aber richten
               Sie’s ſo ein, daſs ich nicht die 5 Stöcke zu ſteigen brauche, ſondern daſs Sie bereit
               ſind herunter zu ko{\geminationm}en. Haben Sie keine Luſt zu warten
               ſo gehen Sie ruhig fort, ich verpflichte Sie zu {\pb}nichts. Ich bin \uline{jedenfalls} bis nahezu 7 zu Haus,
               werde arbeiten.\pend
           \pstart
           Danke vielmals für die Bücher\pend
           \pstart
           Sein Sie engliſch gegrüßt{\\[\baselineskip]}Ihr \spacefill\mbox{Arthur}\pend
           \leftskip=0em{}\pstart
           Sollten Sie zu einem ſehr feſten Entschluſs gelangen, wo {\pb}wir heute Abend ſein werden, so telegrafiren Sie
               vielleicht gleich an die \textcolor{blue}{Tini}{}\ledrightnote{\textcolor{blue}{Christine Schönberger}} fürn \textcolor{blue}{Hugo}{}\ledrightnote{\textcolor{blue}{Hugo von Hofmannsthal}}. (\textcolor{pink}{Südbahn}{}\ledrightnote{\textcolor{pink}{Südbahnhof}},
                  \label{K_L00549_2v}\edtext{z. E.}{\lemma{\textnormal{\emph{z. E.}}}\Cendnote{\textnormal{zum Exempel}}}\label{K_L00549_2h})\pend
           \endnumbering\briefempfaengerindex{Beer-Hofmann, Richard@\textsc{Beer-Hofmann, Richard}!zzzSchnitzler, Arthur@\emph{von Arthur Schnitzler}!1896-06-041@{{[}4. 6. 1896?{]}}|)be}\mylabel{h}  \normalsize

\doendnotes{C}
\bigskip
\vfill

\clearpage

\footnotesize

\lohead{\textsc{register}}

% Definiere theindex-Environment komplett neu ohne reledmac
\makeatletter
\renewenvironment{theindex}{%
  \section*{\indexname}%
  \setlength{\parindent}{0pt}%
  \setlength{\parskip}{0pt plus 0.3pt}%
  \let\item\@idxitem
}{%
  \clearpage
}
\makeatother

\IfFileExists{\jobname-pw.ind}{\input{\jobname-pw.ind}}{}

\end{document}

      