%% latex-korrekturansicht-vorspann.tex
%% Vorspann für die Korrekturansicht.
%% Lädt die gemeinsame Datei latex-vorspann.tex mit gesetztem Schalter.

\newif\ifkorrekturansicht
\korrekturansichttrue

\input{../tex-inputs/latex-vorspann}


               \section[Hermann Bahr an Arthur Schnitzler, 18. 3. 1892]{ Hermann Bahr an Arthur Schnitzler, 18. 3. 1892}\nopagebreak\mylabel{v}\rehead{ }\normalsize\beginnumbering\briefempfaengerindex{Schnitzler, Arthur@\textsc{Schnitzler, Arthur}!zzzBahr, Hermann@\emph{von Hermann Bahr}!1892-03-181@{18. 3. 1892}|(be} \toendnotes[C]{\smallbreak\pagebreak[2]} \Standort{CUL, Schnitzler, B 5b.}
\physDesc{Brief, 1 Blatt, 2 Seiten
\newline{}Handschrift einer Schreibkraft: schwarze Tinte, deutsche Kurrent\newline{}Ordnung: mit rotem Buntstift von unbekannter Hand nummeriert: »7« }\buchAbdrucke{\weitereDrucke{Hermann Bahr, Arthur Schnitzler: \emph{Briefwechsel, Aufzeichnungen, Dokumente (1891–1931)}. Hg. Kurt Ifkovits und Martin Anton Müller. Göttingen: \emph{Wallstein} 2018, S. 23.} }\pstart
           \raggedleft{}{\pb}\textcolor{pink}{Wien}{}\ledrightnote{\textcolor{pink}{Wien}}, 18./3. 1892{\\}\textcolor{pink}{III. Heumarkt 9}{}\ledrightnote{\textcolor{pink}{Am Heumarkt}}\pend
           \pstart{}Lieber Freund!\pend\pstart
           Man erzählt mir ſoeben, daß es für meine Augen ein unfehlbares Mittel gibt: das iſt
               Jod, innerlich genommen. Ich habe leider in den nächſten Tagen keine Minute frei und
               kann unmöglich zu Ihnen kommen. Bitte, ſeien Sie doch nett und ſchicken Sie mir
               ſofort ein entſprechendes Recept, aber \uline{eine}{ }{\pb}gehörige Doſis, \substVorne{}\textsuperscript{ſ}\substDazwischen{}S\substHinten{}ie kennen doch meine \label{LL153-1v}Ochſennatur\label{LL153-1h} die nur auf die ſtärkſten Effecte reagirt. Nehmen Sie im
               Voraus meinen herzlichſten Dank Ihres treu ergebenen\pend
           \pstart \spacefill\mbox{Hermann Bahr}\pend{}\endnumbering\briefempfaengerindex{Schnitzler, Arthur@\textsc{Schnitzler, Arthur}!zzzBahr, Hermann@\emph{von Hermann Bahr}!1892-03-181@{18. 3. 1892}|)be}\mylabel{h}  \normalsize

\doendnotes{C}
\bigskip
\vfill

\clearpage

\footnotesize

\lohead{\textsc{register}}

% Definiere theindex-Environment komplett neu ohne reledmac
\makeatletter
\renewenvironment{theindex}{%
  \section*{\indexname}%
  \setlength{\parindent}{0pt}%
  \setlength{\parskip}{0pt plus 0.3pt}%
  \let\item\@idxitem
}{%
  \clearpage
}
\makeatother

\IfFileExists{\jobname-pw.ind}{\input{\jobname-pw.ind}}{}

\end{document}

      