%% latex-korrekturansicht-vorspann.tex
%% Vorspann für die Korrekturansicht.
%% Lädt die gemeinsame Datei latex-vorspann.tex mit gesetztem Schalter.

\newif\ifkorrekturansicht
\korrekturansichttrue

\input{../tex-inputs/latex-vorspann}


               \section[Hermann Bahr an Arthur Schnitzler, 10. 11. 1908]{ Hermann Bahr an Arthur Schnitzler, 10. 11. 1908}\nopagebreak\mylabel{v}\rehead{ }\normalsize\beginnumbering\briefempfaengerindex{Schnitzler, Arthur@\textsc{Schnitzler, Arthur}!zzzBahr, Hermann@\emph{von Hermann Bahr}!1908-11-101@{10. 11. 1908}|(be} \toendnotes[C]{\smallbreak\pagebreak[2]} \Standort{CUL, Schnitzler, B 5b.}
\physDesc{Bildpostkarte
\newline{}Handschrift: Bleistift, deutsche Kurrent\newline{}Versand: Stempel: »\nobreak{}\oindex{Bahnhof@\textbf{Bahnhof}, \emph{Bahnhofsgebäude (K.BHF)}|pwk}Lind. K. B. Bahnhof, 10 Nov. 08\nobreak{}«.  
\newline{}Schnitzler: mit Bleistift ergänzt »Bahr« \newline{}Ordnung: mit Bleistift von unbekannter Hand nummeriert:
                                    »161« }\buchAbdrucke{\weitereDrucke{Hermann Bahr, Arthur Schnitzler: \emph{Briefwechsel, Aufzeichnungen, Dokumente (1891–1931)}. Hg. Kurt Ifkovits und Martin Anton Müller. Göttingen: \emph{Wallstein} 2018, S. 406.} }\toendnotes[C]{\smallbreak}\pstart{}{\pb}\textsc{Artur Schnitzler}\pend{}\pstart{}\textcolor{pink}{\textsc{Wien XVIII}}{}\ledrightnote{\textcolor{pink}{XVIII., Währing}}\pend{}\pstart{}\textsc{\textcolor{pink}{Spöttelgasse 7}{}\ledrightnote{\textcolor{pink}{Edmund-Weiß-Gasse}}}\pend{}{\bigskip}\pstart
           \noindent{}\centering{}\textcolor{gray}{\textbf{{\pb}\textcolor{pink}{Lindau i. B.}{}\ledrightnote{\textcolor{pink}{Lindau}}}}\pend
           \pstart
           \noindent{}\centering{}\textcolor{gray}{\textbf{Partie im Hafen mit \textcolor{pink}{Bayrischen
                        Hof}{}\ledrightnote{\textcolor{pink}{Bayerischer Hof}} und \textcolor{pink}{alten Leuchtturm}{}\ledrightnote{\textcolor{pink}{Mangturm}}}}\pend
           \pstart
           \raggedleft{}{\pb}10. 11.\pend
           \pstart
           Ich habe Dich am 5. in \textcolor{pink}{Frankfurt}{}\ledrightnote{\textcolor{pink}{Frankfurt am Main}} und
               geſtern \label{K_L01800_1v}\edtext{in \textcolor{pink}{Zürich}{}\ledrightnote{\textcolor{pink}{Zürich}}}{\lemma{\textnormal{\emph{in Zürich}}}\Cendnote{\textnormal{Zur Lesung am 9. 11. 1908
                  im \emph{\textcolor{brown}{Lesezirkel Hottingen}} ist sowohl in \textcolor{blue}{Bahrs} wie auch in \textcolor{blue}{Schnitzlers} Papieren (University of Exeter, \emph{The
                        Schnitzler Press-Cuttings Archive}, Box 1/6) das
                  Programmheft überliefert. Als Ablauf wird angegeben: »1. Über \textcolor{blue}{Schnitzler}. 2. \textcolor{blue}{Schnitzlers} Novelle: ›\textcolor{green}{Die Toten
                        schweigen}‹«.}}}\label{K_L01800_1h} beſungen, \introOben{}über\introOben{}morgen wirſt Dus auch noch nicht in \textcolor{pink}{Mannheim}{}\ledrightnote{\textcolor{pink}{Mannheim}}. Verſchaff Dir das letzte Heft des »\textcolor{green}{Morgen}{}\ledrightnote{\textcolor{green}{Morgen. Wochenschrift für deutsche Kultur}}«, wo ich \textcolor{green}{einiges}{}\ledrightnote{→\textcolor{green}{Tagebuch. 10. Juni [1908]}}
               zum »\textcolor{green}{Weg ins Freie}{}\ledrightnote{\textcolor{green}{Der Weg ins Freie. Roman}}« geſagt habe.\pend
           \pstart
           Mit vielen Grüßen an Deine liebe \textcolor{blue}{Frau}{}\ledrightnote{→\textcolor{blue}{Olga Schnitzler}}{\\[\baselineskip]}herzlichſt\hspace*{1.5em}\spacefill\mbox{Hermann}\pend
           \leftskip=0em{}\endnumbering\briefempfaengerindex{Schnitzler, Arthur@\textsc{Schnitzler, Arthur}!zzzBahr, Hermann@\emph{von Hermann Bahr}!1908-11-101@{10. 11. 1908}|)be}\mylabel{h}  \normalsize

\doendnotes{C}
\bigskip
\vfill

\clearpage

\footnotesize

\lohead{\textsc{register}}

% Definiere theindex-Environment komplett neu ohne reledmac
\makeatletter
\renewenvironment{theindex}{%
  \section*{\indexname}%
  \setlength{\parindent}{0pt}%
  \setlength{\parskip}{0pt plus 0.3pt}%
  \let\item\@idxitem
}{%
  \clearpage
}
\makeatother

\IfFileExists{\jobname-pw.ind}{\input{\jobname-pw.ind}}{}

\end{document}

      