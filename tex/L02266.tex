%% latex-korrekturansicht-vorspann.tex
%% Vorspann für die Korrekturansicht.
%% Lädt die gemeinsame Datei latex-vorspann.tex mit gesetztem Schalter.

\newif\ifkorrekturansicht
\korrekturansichttrue

\input{../tex-inputs/latex-vorspann}


               \section[Richard Beer-Hofmann an Arthur Schnitzler, 18. 7. 1917]{ Richard Beer-Hofmann an Arthur Schnitzler, 18. 7. 1917}\nopagebreak\mylabel{v}\rehead{ }\normalsize\beginnumbering\briefempfaengerindex{Schnitzler, Arthur@\textsc{Schnitzler, Arthur}!zzzBeer-Hofmann, Richard@\emph{von Richard Beer-Hofmann}!1917-07-181@{18. 7. 1917}|(be} \toendnotes[C]{\smallbreak\pagebreak[2]} \Standort{CUL, Schnitzler, B 8.}
\physDesc{Brief, 1 Blatt, 2 Seiten
\newline{}Handschrift: Bleistift, lateinische Kurrent
\newline{}Schnitzler: 1) mit Bleistift beschriftet: »\textsc{Richar}« 2) mit rotem Buntstift zwei Unterstreichungen\newline{}Ordnung: mit Bleistift von unbekannter Hand nummeriert:
                                    »263« }\buchAbdrucke{\weitereDrucke{Arthur Schnitzler, Richard Beer-Hofmann: \emph{Briefwechsel 1891–1931}. Hg. Konstanze Fliedl. Wien, Zürich: \emph{Europaverlag} 1992, S. 223–224.} }\toendnotes[C]{\smallbreak}\pstart
           \raggedleft{}{\pb}\textcolor{pink}{Ischl}{}\ledrightnote{\textcolor{pink}{Bad Ischl}}{ }18/VII 17\pend
           \pstart
           Lieber Arthur! Ich habe Ihren Brief erwartet. Ich hatte mit Absicht
               Ihnen nicht geschrieben, ich wollte wissen, wie Sie – unbeeinflusst durch meinen
               Bericht – die Sache ansehen. Ich war durch den akuten Anfall, den ich ja durch
               3 Stunden mit ansah (\textcolor{blue}{K.}{}\ledrightnote{\textcolor{blue}{Arthur Kaufmann}} hatte nach mir verlangt)
               sehr erschreckt. Sie selbst sahen ja nur einen Zustand, der vom Normalen nicht so
               weit abzuliegen schien. Ich aber verbrachte auch die dem Anfall folgenden Tage, bis
               zu seiner Abreise ins \textcolor{pink}{Sanatorium}{}\ledrightnote{→\textcolor{pink}{Sanatorium Purkersdorf}}
               in einer unaufhörlichen Anspannung, da ich mich – es war ja niemand, als seine \textcolor{blue}{Schwester}{}\ledrightnote{→\textcolor{blue}{Malvine Kaufmann}} da – irgendwie
               verantwortlich fühlte. Auch betonte Dozent \textcolor{blue}{K.}{}\ledrightnote{\textcolor{blue}{Rudolf Kaufmann}} ja
                  i{\geminationm}er sein Laiesein in derartigen Dingen, sah aber
               recht schwarz {\pb}und ich mit ihm. Was
               mich bestürzte, war, dass es nicht eine Steigerung oder Über-Spannung seiner
               sonstigen Art zu denken war, sondern ein vollständiges Anders-sein, Reden,
               »Philosophiren«, wie es ihm sein Lebtag verhasst und lächerlich erschienen war.
               Niederschreiben mag und kann ich das Alles nicht, und nun – da es ja wieder gutgeht,
               hätte es ja auch nicht viel Sinn, es festzuhalten.\pend
           \pstart
           Ich bin von Herzen froh, dass es so – und nicht anders – ausgieng.\pend
           \pstart
           Von uns ist nichts zu berichten, als dass wir eine schlechte Woche mit \label{KLL02266_AS-1v}\edtext{Schufterl}{\lemma{\textnormal{\emph{Schufterl}}}\Cendnote{\textnormal{ein weißer Spitz, erworben im Dezember 1905}}}\label{KLL02266_AS-1h}
               verbrachten, der fast zwölf Jahre mit uns lebte, und nun im Garten der Villa begraben
               wurde. –\pend
           \pstart
           Werden wir Sie im So{\geminationm}er im \textcolor{pink}{Salzka{\geminationm}ergut}{}\ledrightnote{\textcolor{pink}{Salzkammergut}} sehen?\pend
           \pstart
           Alles Herzliche Ihnen, Frau \textcolor{blue}{Olga}{}\ledrightnote{\textcolor{blue}{Olga Schnitzler}} und den \textcolor{blue}{Kindern}{}\ledrightnote{→\textcolor{blue}{Heinrich Schnitzler}{\newline}→\textcolor{blue}{Lili Schnitzler}}! Ihr\pend
           \pstart \spacefill\mbox{Richard}\pend{}\endnumbering\briefempfaengerindex{Schnitzler, Arthur@\textsc{Schnitzler, Arthur}!zzzBeer-Hofmann, Richard@\emph{von Richard Beer-Hofmann}!1917-07-181@{18. 7. 1917}|)be}\mylabel{h}  \normalsize

\doendnotes{C}
\bigskip
\vfill

\clearpage

\footnotesize

\lohead{\textsc{register}}

% Definiere theindex-Environment komplett neu ohne reledmac
\makeatletter
\renewenvironment{theindex}{%
  \section*{\indexname}%
  \setlength{\parindent}{0pt}%
  \setlength{\parskip}{0pt plus 0.3pt}%
  \let\item\@idxitem
}{%
  \clearpage
}
\makeatother

\IfFileExists{\jobname-pw.ind}{\input{\jobname-pw.ind}}{}

\end{document}

      