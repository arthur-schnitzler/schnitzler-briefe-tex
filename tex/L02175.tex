%% latex-korrekturansicht-vorspann.tex
%% Vorspann für die Korrekturansicht.
%% Lädt die gemeinsame Datei latex-vorspann.tex mit gesetztem Schalter.

\newif\ifkorrekturansicht
\korrekturansichttrue

\input{../tex-inputs/latex-vorspann}


               \section[Richard Beer-Hofmann an Arthur Schnitzler, 16. 4. 1914]{ Richard Beer-Hofmann an Arthur Schnitzler,
               16. 4. 1914}\nopagebreak\mylabel{v}\rehead{ }\normalsize\beginnumbering\briefempfaengerindex{Schnitzler, Arthur@\textsc{Schnitzler, Arthur}!zzzBeer-Hofmann, Richard@\emph{von Richard Beer-Hofmann}!1914-04-161@{16. 4. 1914}|(be} \toendnotes[C]{\smallbreak\pagebreak[2]} \Standort{CUL, Schnitzler, B 8.}
\physDesc{Bildpostkarte
\newline{}Handschrift: blaue Tinte, lateinische Kurrent\newline{}Versand: Stempel: »\nobreak{}\oindex{Menton@\textbf{Menton}, \emph{http://www.geonames.org/ontologyP.PPL}|pwk}Menton Alpes Maritimes, 16 – 4. 14, 15\nobreak{}«.  \newline{}Ordnung: mit Bleistift von unbekannter Hand nummeriert: »257« }\pstart{}{\pb}Herrn\pend{}\pstart{}Arthur Schnitzler\pend{}\pstart{}\textcolor{pink}{Wien XVIII}{}\ledrightnote{\textcolor{pink}{XVIII., Währing}}\pend{}\pstart{}\strikeout{Has}{ }\textcolor{pink}{Sternwartestr}{}\ledrightnote{\textcolor{pink}{Sternwartestraße}}\pend{}\pstart{}\textcolor{pink}{Autriche}{}\ledrightnote{\textcolor{pink}{Österreich}}.\pend{}{\bigskip}\pstart
           \noindent{}\centering{}{\pb}\textcolor{gray}{\textbf{\textcolor{pink}{MENTON}{}\ledrightnote{\textcolor{pink}{Menton}}. La vielle ville}}\pend
           \pstart
           \raggedleft{}{\pb}16./IV 14\pend
           \pstart
           Vielen Dank für Ihre Karte! Hoffentlich \strikeout{se} treffen
               wir Sie noch in \textcolor{pink}{Wien}{}\ledrightnote{\textcolor{pink}{Wien}} an. Wir wollen gegen 25.
               zurück sein. Herzlichst\pend
           \pstart Ihr\spacefill\mbox{Richard}\pend{}\endnumbering\briefempfaengerindex{Schnitzler, Arthur@\textsc{Schnitzler, Arthur}!zzzBeer-Hofmann, Richard@\emph{von Richard Beer-Hofmann}!1914-04-161@{16. 4. 1914}|)be}\mylabel{h}  \normalsize

\doendnotes{C}
\bigskip
\vfill

\clearpage

\footnotesize

\lohead{\textsc{register}}

% Definiere theindex-Environment komplett neu ohne reledmac
\makeatletter
\renewenvironment{theindex}{%
  \section*{\indexname}%
  \setlength{\parindent}{0pt}%
  \setlength{\parskip}{0pt plus 0.3pt}%
  \let\item\@idxitem
}{%
  \clearpage
}
\makeatother

\IfFileExists{\jobname-pw.ind}{\input{\jobname-pw.ind}}{}

\end{document}

      