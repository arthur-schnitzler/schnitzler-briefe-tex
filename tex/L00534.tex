%% latex-korrekturansicht-vorspann.tex
%% Vorspann für die Korrekturansicht.
%% Lädt die gemeinsame Datei latex-vorspann.tex mit gesetztem Schalter.

\newif\ifkorrekturansicht
\korrekturansichttrue

\input{../tex-inputs/latex-vorspann}


               \section[Arthur Schnitzler an Richard Beer-Hofmann, {[}19. 2. 1896?{]}]{ Arthur Schnitzler an Richard Beer-Hofmann, {[}19. 2. 1896?{]}}\nopagebreak\mylabel{v}\rehead{ }\normalsize\beginnumbering\briefempfaengerindex{Beer-Hofmann, Richard@\textsc{Beer-Hofmann, Richard}!zzzSchnitzler, Arthur@\emph{von Arthur Schnitzler}!1896-02-191@{{[}19. 2. 1896?{]}}|(be} \toendnotes[C]{\smallbreak\pagebreak[2]} \Standort{YCGL, MSS 31.}
\physDesc{Briefkarte, Umschlag
\newline{}Handschrift: 1) schwarze Tinte, deutsche Kurrent\hspace{1em}2) Bleistift, deutsche Kurrent (\noindent{}Umschlag)\hspace{1em}\newline{}Versand: ohne postalischen Übermittlungsvermerk }\toendnotes[C]{\smallbreak}\pstart{}{\pb}Herrn \textsc{Dr. Rich Beer
                     Hofmann}\pend{}\pstart{}\textcolor{pink}{Wien}{}\ledrightnote{\textcolor{pink}{Wien}}\pend{}\pstart{}\textsc{\textcolor{pink}{I Wollzeile 15}{}\ledrightnote{\textcolor{pink}{Wollzeile}}}\pend{}\pstart{}4. Stock\pend{}{\bigskip}\pstart
           \noindent{}{\pb}Lieber Richard, we{\geminationn} Sie nichts beſſeres
               vorhaben, nachmahlen Sie \label{K_L00534_1v}\edtext{Freitag}{\lemma{\textnormal{\emph{Freitag}}}\Cendnote{\textnormal{Unter den Annahmen, dass das
                  Korrespondenzstück zum Jahr 1896 gehört (es wird zusammen mit
                  diesen aufbewahrt) und dass das Essen stattfand und auch im \emph{\textcolor{green}{Tagebuch}}{ }\textcolor{blue}{Schnitzler}s erwähnt wird – lassen sich zwei
                  Freitage eingrenzen: 21. 2. 1896 und 22. 5. 1896. Bei ersterem Datum kommt es zu einer größeren
                  Gesellschaft, während bei zweiterem bereits am Vortag ein Essen mit \textcolor{blue}{Brahm} bei \textcolor{blue}{Beer-Hofmann} stattfand, so dass die Kommunikation eher zu knapp ausfällt.
                  Hier wird der Annahme gefolgt, dass es um das erste Datum geht und in Entsprechung
                  zur Reaktion Hofmannsthals vom [20. 2. 1896] auf eine mutmaßlich ähnlich lautende Einladung
                  datiert.}}}\label{K_L00534_1h}{ }Abend bei uns, ja?\pend
           \pstart Herzlich Ihr \spacefill\mbox{Arthur}\pend{}\endnumbering\briefempfaengerindex{Beer-Hofmann, Richard@\textsc{Beer-Hofmann, Richard}!zzzSchnitzler, Arthur@\emph{von Arthur Schnitzler}!1896-02-191@{{[}19. 2. 1896?{]}}|)be}\mylabel{h}  \normalsize

\doendnotes{C}
\bigskip
\vfill

\clearpage

\footnotesize

\lohead{\textsc{register}}

% Definiere theindex-Environment komplett neu ohne reledmac
\makeatletter
\renewenvironment{theindex}{%
  \section*{\indexname}%
  \setlength{\parindent}{0pt}%
  \setlength{\parskip}{0pt plus 0.3pt}%
  \let\item\@idxitem
}{%
  \clearpage
}
\makeatother

\IfFileExists{\jobname-pw.ind}{\input{\jobname-pw.ind}}{}

\end{document}

      