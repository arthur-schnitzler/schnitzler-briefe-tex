%% latex-korrekturansicht-vorspann.tex
%% Vorspann für die Korrekturansicht.
%% Lädt die gemeinsame Datei latex-vorspann.tex mit gesetztem Schalter.

\newif\ifkorrekturansicht
\korrekturansichttrue

\input{../tex-inputs/latex-vorspann}


               \section[Arthur Schnitzler an Hugo von Hofmannsthal, 5. 7. 1893]{ Arthur Schnitzler an Hugo von Hofmannsthal, 5. 7. 1893}\nopagebreak\mylabel{v}\rehead{ }\normalsize\beginnumbering\briefempfaengerindex{Hofmannsthal, Hugo von@\textsc{Hofmannsthal, Hugo von}!zzzSchnitzler, Arthur@\emph{von Arthur Schnitzler}!1893-07-051@{5. 7. 1893}|(be} \toendnotes[C]{\smallbreak\pagebreak[2]} \Standort{FDH, Hs-30885,35.}
\physDesc{Brief, 1 Blatt (Briefpapier mit Trauerrand), 3 Seiten
\newline{}Handschrift: schwarze Tinte, deutsche Kurrent}\buchAbdrucke{\weitereDrucke{Hugo von Hofmannsthal, Arthur Schnitzler: \emph{Briefwechsel}. Hg. Therese Nickl und Heinrich Schnitzler. Frankfurt am Main: \emph{S. Fischer} 1964, S. 39–40.} }\toendnotes[C]{\smallbreak}\pstart
           \raggedleft{}{\pb}\textcolor{pink}{\textsc{Ischl, Pens. Leopold}}{}\ledrightnote{\textcolor{pink}{Hotel und Pension Rudolfshöhe (Leopold Petter)}}{\\}5/7. 93. \pend
           \pstart{}Lieber Loris,\pend\pstart
           bin in \textcolor{pink}{Iſchl}{}\ledrightnote{\textcolor{pink}{Hotel und Pension Rudolfshöhe (Leopold Petter)}}, war \textsc{per
                        Bic}. u. mit \textcolor{blue}{\textsc{Richard}}{}\ledrightnote{\textcolor{blue}{Richard Beer-Hofmann}} in \textcolor{pink}{\textsc{Strobl}}{}\ledrightnote{\textcolor{pink}{Strobl}}, wo Sie von der \textcolor{blue}{Badekabinenvermietherin}{}\ledrightnote{→\textcolor{blue}{?? [Badekabinenvermieterin in Strobl]}}{ }{\pb}gekannt werden u Ihr Name unorthographiſch auf den
                    Brettern ſteht. –\pend
           \pstart
           Ich bleibe etwa bis zum 14. da, wünſchte was von Ihnen zu hören und
                    ſchätze Sie ſowohl als Poeten wie als Menſchen {\pb}ſehr
                    hoch. –\pend
           \pstart
           Geſchrieben hab ich wenig oder nichts oder gar nichts oder doch \label{K_L00233_1v}\edtext{etwas}{\lemma{\textnormal{\emph{etwas}}}\Cendnote{\textnormal{nicht identifiziert}}}\label{K_L00233_1h}, und meine Laune iſt kühl,
                    dumpf und grau mit grünen Tupfen. –\pend
           \pstart
           Ihr entarteter{\\[\baselineskip]}\spacefill\mbox{ArthSch}\pend
           \leftskip=0em{}\endnumbering\briefempfaengerindex{Hofmannsthal, Hugo von@\textsc{Hofmannsthal, Hugo von}!zzzSchnitzler, Arthur@\emph{von Arthur Schnitzler}!1893-07-051@{5. 7. 1893}|)be}\mylabel{h}  \normalsize

\doendnotes{C}
\bigskip
\vfill

\clearpage

\footnotesize

\lohead{\textsc{register}}

% Definiere theindex-Environment komplett neu ohne reledmac
\makeatletter
\renewenvironment{theindex}{%
  \section*{\indexname}%
  \setlength{\parindent}{0pt}%
  \setlength{\parskip}{0pt plus 0.3pt}%
  \let\item\@idxitem
}{%
  \clearpage
}
\makeatother

\IfFileExists{\jobname-pw.ind}{\input{\jobname-pw.ind}}{}

\end{document}

      