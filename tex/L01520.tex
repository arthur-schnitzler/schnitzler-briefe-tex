%% latex-korrekturansicht-vorspann.tex
%% Vorspann für die Korrekturansicht.
%% Lädt die gemeinsame Datei latex-vorspann.tex mit gesetztem Schalter.

\newif\ifkorrekturansicht
\korrekturansichttrue

\input{../tex-inputs/latex-vorspann}


               \section[Arthur Schnitzler an Richard Beer-Hofmann, 26. 5. 1905]{ Arthur Schnitzler an Richard Beer-Hofmann, 26. 5. 1905}\nopagebreak\mylabel{v}\rehead{ }\normalsize\beginnumbering\briefempfaengerindex{Beer-Hofmann, Richard@\textsc{Beer-Hofmann, Richard}!zzzSchnitzler, Arthur@\emph{von Arthur Schnitzler}!1905-05-262@{26. 5. 1905}|(be} \toendnotes[C]{\smallbreak\pagebreak[2]} \Standort{YCGL, MSS 31.}
\physDesc{Brief, 2 Blätter, 5 Seiten, Umschlag
\newline{}Handschrift: Bleistift, deutsche Kurrent\newline{}Versand: 1) Stempel: »\nobreak{}\oindex{XVIII., Waehring@\textbf{XVIII., Währing}, \emph{Bezirk (A.BZK)}|pwk}Wien 18/1 110, 26. 5. 05, 8–9N\nobreak{}«.  2) Stempel: »\nobreak{}\oindex{Rodaun@\textbf{Rodaun}, \emph{Teil eines besiedelten Ortes (A.BSOX)}|pwk}{\pb}Rodaun, 27 {[}5. 1905{]}, 8–9V\nobreak{}«. }\buchAbdrucke{\weitereDrucke{Arthur Schnitzler, Richard Beer-Hofmann: \emph{Briefwechsel 1891–1931}. Hg. Konstanze Fliedl. Wien, Zürich: \emph{Europaverlag} 1992, S. 173–174.} }\toendnotes[C]{\smallbreak}\pstart{}{\pb}\textcolor{gray}{\textbf{Dr. Arthur Schnitzler}}\pend{}\pstart{}\textcolor{gray}{\textbf{\textcolor{pink}{Wien XVIII. Spoettelgasse 7}{}\ledrightnote{\textcolor{pink}{Edmund-Weiß-Gasse}}.}}\pend{}{\bigskip}\pstart{}{\pb}\textsc{Herrn Dr Richard Beer-Hofmann}\pend{}\pstart{}\textcolor{pink}{Rodaun}{}\ledrightnote{\textcolor{pink}{Rodaun}}\pend{}\pstart{}\textcolor{pink}{\textsc{Liesingerstraße 2}}{}\ledrightnote{\textcolor{pink}{Liesingerstraße}}.\pend{}\pstart{}\textsc{bei \textcolor{pink}{Wien}{}\ledrightnote{\textcolor{pink}{Wien}}.}\pend{}{\bigskip}\pstart
           \raggedleft{}{\pb}\textcolor{pink}{Wien}{}\ledrightnote{\textcolor{pink}{Wien}}{ }26. 5. 905\pend
           \pstart
           lieber Richard, eigentlich hab ich mir gedacht, daſs das viele
               unverſtändige u perfide, das Sie nun leſen mußten (mußten?), Sie kühler gelaſſen
               hätte – aber es ſcheint wirklich: auf etwas gefaſſt ſein hilft uns i{\geminationm}er nur ſo lange als es nicht da iſt. Mir war am
               zuwiderſten \textcolor{blue}{Polgar}{}\ledrightnote{\textcolor{blue}{Alfred Polgar}}, der mir nebſtbei Talent zu
               haben ſcheint und gut ſchreibt, – und der ſich zum Schluſs, in ſeiner Sehnſucht {\pb}nach dem gemeinen Kerl, ſo anmutig verräth. Er hat
               doch bisher ſo ſelten vergeblich gelechzt; – man dürfte ihm ſagen: Warum in die Ferne
               ſchweifen? Ach das gemeine liegt ſo nah. Auch er gehört übrigens zu denjenigen, denen
               man doch einmal Zeit gö{\geminationn}en ſollte – meinetwegen 12
               Jahre, damit ſie ungeſtört ihren \textcolor{green}{Grafen von \textsc{Charolais}}{}\ledrightnote{\textcolor{green}{Der Graf von Charolais. Ein Trauerspiel}} oder auch nur die 10 ſchönen Verſe dichten können – da{\geminationn} würde man doch {\pb}ſehen,
               was herausko{\geminationm}t {\dots} mit Bildung
               und Fleiß und Willen {\dotsfour}\pend
           \pstart
           – Was mich nicht hindert, mich dem Wunſche mancher andrer anzuſchließen, daſs Sie
               bald was neues anfangen –; wohl aus andern Motiven wünſch ich das, als die manchen
               andern; aber ich wünſch es ſehr. Vor allem darum weil Sie da{\geminationn} die Empfindung hätten, daſs die Leute, die über den
               Dichter des \textcolor{green}{\textsc{Charolais}}{}\ledrightnote{\textcolor{green}{Der Graf von Charolais. Ein Trauerspiel}} ſchreiben, eigentlich nicht mehr über Sie, ſondern über {\pb}einen andern ſchreiben, und \substVorne{}\textsuperscript{das}\substDazwischen{}es\substHinten{} iſt Einem, ich
               verſichre Sie, \strikeout{da{\geminationn}} ziemlich gleichgiltig, – was die
               Leute über einen andern ſchreiben.\pend
           \pstart
           – Heute erſt hab ich wieder Ihren Grund bewundert. Frl. \textcolor{blue}{\textsc{Erl}}{}\ledrightnote{\textcolor{blue}{Dora Erl}}, die mit uns war, ſagte: Wieſo ist er ihm noch nicht weg gekauft worden? –\pend
           \pstart
           Ko{\geminationm}en Sie bald, vielleicht zu Tiſch? Ich dictire jetzt
               manchmal Nachmittag alſo wärs mir lieb, we{\geminationn} ich früher
               von Ihrem Ko{\geminationm}en unterrichtet {\pb}wäre. – Vormittag ſpielen wir 3mal \textsc{Tennis}, was mir enorm viel Vergnügen macht. Müſſen Sie auch,
               ſobald Sie \textcolor{pink}{Währinger}{}\ledrightnote{\textcolor{pink}{XVIII., Währing}} geworden ſind.\pend
           \pstart
           Wir grüßen Sie \textcolor{blue}{beide}{}\ledrightnote{→\textcolor{blue}{Paula Beer-Hofmann}} und die
                  \textcolor{blue}{Kinder}{}\ledrightnote{→\textcolor{blue}{Naëmah Beer-Hofmann}{\newline}→\textcolor{blue}{Gabriel Beer-Hofmann}{\newline}→\textcolor{blue}{Mirjam Beer-Hofmann}}. \textcolor{blue}{Olga}{}\ledrightnote{\textcolor{blue}{Olga Schnitzler}} war von Ihrem Brief ſo ergriffen, daſs ſie
               eine Thräne im Augenwinkel hatte. Ich ſage nichts als: dos is e Dichter. Aber ich
               hab mich ſehr gefreut. Warum »aber«?\pend
           \pstart
           Herzlichſt{\\}Ihr{\\}\spacefill\mbox{A.}\pend
           \endnumbering\briefempfaengerindex{Beer-Hofmann, Richard@\textsc{Beer-Hofmann, Richard}!zzzSchnitzler, Arthur@\emph{von Arthur Schnitzler}!1905-05-262@{26. 5. 1905}|)be}\mylabel{h}  \normalsize

\doendnotes{C}
\bigskip
\vfill

\clearpage

\footnotesize

\lohead{\textsc{register}}

% Definiere theindex-Environment komplett neu ohne reledmac
\makeatletter
\renewenvironment{theindex}{%
  \section*{\indexname}%
  \setlength{\parindent}{0pt}%
  \setlength{\parskip}{0pt plus 0.3pt}%
  \let\item\@idxitem
}{%
  \clearpage
}
\makeatother

\IfFileExists{\jobname-pw.ind}{\input{\jobname-pw.ind}}{}

\end{document}

      