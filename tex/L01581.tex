%% latex-korrekturansicht-vorspann.tex
%% Vorspann für die Korrekturansicht.
%% Lädt die gemeinsame Datei latex-vorspann.tex mit gesetztem Schalter.

\newif\ifkorrekturansicht
\korrekturansichttrue

\input{../tex-inputs/latex-vorspann}


               \section[Arthur Schnitzler an Hermann Bahr, 3. 2. 1906]{ Arthur Schnitzler an Hermann Bahr, 3. 2. 1906}\nopagebreak\mylabel{v}\rehead{ }\normalsize\beginnumbering\briefempfaengerindex{Bahr, Hermann@\textsc{Bahr, Hermann}!zzzSchnitzler, Arthur@\emph{von Arthur Schnitzler}!1906-02-031@{3. 2. 1906}|(be} \toendnotes[C]{\smallbreak\pagebreak[2]} \Standort{TMW, HS AM 60176 Ba.}
\physDesc{Briefkarte
\newline{}Handschrift: schwarze Tinte, deutsche Kurrent\newline{}Ordnung: Lochung }\buchAbdrucke{\weitereDrucke{1) \emph{3. 2. 1906, Abschrift.} In: Arthur Schnitzler: \emph{The Letters of Arthur Schnitzler to Hermann Bahr}. Edited, annotated, and with an introduction, by Donald G.
                        Daviau. Chapel Hill: \emph{The University of North Carolina Press} 1978, S. 93–94 (University of North Carolina studies in the Germanic languages
                        and literatures, 89).} \weitereDrucke{2) Hermann Bahr, Arthur Schnitzler: \emph{Briefwechsel, Aufzeichnungen, Dokumente (1891–1931)}. Hg. Kurt Ifkovits und Martin Anton Müller. Göttingen: \emph{Wallstein} 2018, S. 373.} }\toendnotes[C]{\smallbreak}\pstart
           \noindent{}{\pb}\textcolor{gray}{\textbf{Dr. Arthur Schnitzler}}\hfill 3. 2. 906.\pend
           \pstart
           \textcolor{gray}{\textbf{\textcolor{pink}{Wien, XVIII. Spoettelgasse 7}{}\ledrightnote{\textcolor{pink}{Edmund-Weiß-Gasse}}.}}\pend
           \pstart
           mein lieber Hermann, ich fahre heute auf ein paar Tage nach \textcolor{pink}{Berlin}{}\ledrightnote{\textcolor{pink}{Berlin}}. (\textsc{\textcolor{pink}{Hotel Continental}{}\ledrightnote{\textcolor{pink}{Hotel Continental}}}) Iſt der »\textcolor{green}{Ruf}{}\ledrightnote{\textcolor{green}{Der Ruf des Lebens. Schauspiel in drei Akten}}« als definitiv von der \textcolor{brown}{Münchner Hofbühne}{}\ledrightnote{\textcolor{brown}{Königliche Hof- und Nationaltheater München}} abgelehnt zu betrachten? Oder hältſt
               du es für möglich, daſs ein eventueller ſtarker E\damage{rf}olg in \textcolor{pink}{Berlin}{}\ledrightnote{\textcolor{pink}{Berlin}} doch noch den \textcolor{blue}{Intendanten}{}\ledrightnote{→\textcolor{blue}{Albert von Speidel}}{ }{\pb}anders beſtimmen könnte? In
               diesem Fa\damage{lle} möchte ich einen Antrag des \textcolor{pink}{Münchner
                  Schauſpielhauſes}{}\ledrightnote{\textcolor{pink}{Münchner Schauspielhaus}} (der \textcolor{blue}{Fiſcher}{}\ledrightnote{\textcolor{blue}{Samuel Fischer}}{ }ſchon ſeit
               Wochen vorliegt) vorläufg \label{K_L01581_1v}\edtext{dilatoriſch}{\lemma{\textnormal{\emph{dilatoriſch}}}\Cendnote{\textnormal{verzögernd}}}\label{K_L01581_1h}
               behandeln.\pend
           \pstart
           Herzlichſt{\\[\baselineskip]}dein{\\[\baselineskip]}\spacefill\mbox{A.}\pend
           \leftskip=0em{}\endnumbering\briefempfaengerindex{Bahr, Hermann@\textsc{Bahr, Hermann}!zzzSchnitzler, Arthur@\emph{von Arthur Schnitzler}!1906-02-031@{3. 2. 1906}|)be}\mylabel{h}  \normalsize

\doendnotes{C}
\bigskip
\vfill

\clearpage

\footnotesize

\lohead{\textsc{register}}

% Definiere theindex-Environment komplett neu ohne reledmac
\makeatletter
\renewenvironment{theindex}{%
  \section*{\indexname}%
  \setlength{\parindent}{0pt}%
  \setlength{\parskip}{0pt plus 0.3pt}%
  \let\item\@idxitem
}{%
  \clearpage
}
\makeatother

\IfFileExists{\jobname-pw.ind}{\input{\jobname-pw.ind}}{}

\end{document}

      