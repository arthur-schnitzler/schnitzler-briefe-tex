%% latex-korrekturansicht-vorspann.tex
%% Vorspann für die Korrekturansicht.
%% Lädt die gemeinsame Datei latex-vorspann.tex mit gesetztem Schalter.

\newif\ifkorrekturansicht
\korrekturansichttrue

\input{../tex-inputs/latex-vorspann}


               \section[Olga Schnitzler an Richard und Paula Beer-Hofmann, {[}9. 6. 1909?{]}]{ Olga Schnitzler an Richard und Paula Beer-Hofmann,
               {[}9. 6. 1909?{]}}\nopagebreak\mylabel{v}\rehead{ }\normalsize\beginnumbering\briefempfaengerindex{Beer-Hofmann, Paula@\textsc{Beer-Hofmann, Paula}!zzzSchnitzler, Olga@\emph{von Olga Schnitzler}!1909-06-091@{{[}9. 6. 1909?{]}}|(be}\briefempfaengerindex{Beer-Hofmann, Richard@\textsc{Beer-Hofmann, Richard}!zzzSchnitzler, Olga@\emph{von Olga Schnitzler}!1909-06-091@{{[}9. 6. 1909?{]}}|(be} \toendnotes[C]{\smallbreak\pagebreak[2]} \Standort{YCGL, MSS 31.}
\physDesc{Brief, 1 Blatt, 1 Seite, Umschlag
\newline{}Handschrift: schwarze Tinte, lateinische Kurrent\newline{}Versand: ohne postalischen Übermittlungsvermerk }\toendnotes[C]{\smallbreak}\pstart{}{\pb}\textcolor{gray}{\textbf{O. S.}}\pend{}{\bigskip}\pstart{}{\pb}Herrn u. Frau D\textsuperscript{r}
                  Richard Beer-Hofmann\pend{}{\bigskip}\pstart
           \noindent{}{\pb}\textcolor{gray}{\textbf{O. S.}}\pend
           \pstart
           Meine Lieben, der Kapellmeister \textcolor{blue}{Walter}{}\ledrightnote{\textcolor{blue}{Bruno Walter}} hat sich für \label{K_L02549-1v}\edtext{morgen
                  Donnerstag}{\lemma{\textnormal{\emph{morgen
                  Donnerstag}}}\Cendnote{\textnormal{Die Datierung ergibt sich
                  aus dem Tagebucheintrag vom 10. 6. 1909.}}}\label{K_L02549-1h}{ }½ 8 Uhr Abend bei uns angesagt, wir bitten Euch, ebenfalls zu
               kommen.\pend
           \pstart
           Ihr werdet einen wirklich aussergewöhnlichen und sehr lieben Menschen kennen
               lernen\pend
           \pstart
           Seine \textcolor{blue}{Frau}{}\ledrightnote{→\textcolor{blue}{Elsa Walter}} ist schon verreist,
                  \label{KLmmL02549-2v}\edtext{tan mieux}{\lemma{\textnormal{\emph{tan mieux}}}\Cendnote{\textnormal{korrekt wäre: »tant mieux«, französisch für: »so viel
                  besser«}}}\label{KLmmL02549-2h}. Wir rechnen bestimmt auf Euch.\pend
           \pstart Herzliche Grüsse \spacefill\mbox{OlgaS.}\pend{}\endnumbering\briefempfaengerindex{Beer-Hofmann, Paula@\textsc{Beer-Hofmann, Paula}!zzzSchnitzler, Olga@\emph{von Olga Schnitzler}!1909-06-091@{{[}9. 6. 1909?{]}}|)be}\briefempfaengerindex{Beer-Hofmann, Richard@\textsc{Beer-Hofmann, Richard}!zzzSchnitzler, Olga@\emph{von Olga Schnitzler}!1909-06-091@{{[}9. 6. 1909?{]}}|)be}\mylabel{h}  \normalsize

\doendnotes{C}
\bigskip
\vfill

\clearpage

\footnotesize

\lohead{\textsc{register}}

% Definiere theindex-Environment komplett neu ohne reledmac
\makeatletter
\renewenvironment{theindex}{%
  \section*{\indexname}%
  \setlength{\parindent}{0pt}%
  \setlength{\parskip}{0pt plus 0.3pt}%
  \let\item\@idxitem
}{%
  \clearpage
}
\makeatother

\IfFileExists{\jobname-pw.ind}{\input{\jobname-pw.ind}}{}

\end{document}

      