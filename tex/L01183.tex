%% latex-korrekturansicht-vorspann.tex
%% Vorspann für die Korrekturansicht.
%% Lädt die gemeinsame Datei latex-vorspann.tex mit gesetztem Schalter.

\newif\ifkorrekturansicht
\korrekturansichttrue

\input{../tex-inputs/latex-vorspann}


               \section[Arthur Schnitzler an Hermann Bahr, 26. 10. 1901]{ Arthur Schnitzler an Hermann Bahr, 26. 10. 1901}\nopagebreak\mylabel{v}\rehead{ }\normalsize\beginnumbering\briefempfaengerindex{Bahr, Hermann@\textsc{Bahr, Hermann}!zzzSchnitzler, Arthur@\emph{von Arthur Schnitzler}!1901-10-261@{26. 10. 1901}|(be} \toendnotes[C]{\smallbreak\pagebreak[2]} \Standort{TMW, HS AM 37430 Ba.}
\physDesc{Brief, 1 Blatt, 3 Seiten
\newline{}Handschrift: schwarze Tinte, deutsche Kurrent\newline{}Ordnung: Lochung }\buchAbdrucke{\weitereDrucke{1) \emph{26. 10. 1901.} In: Arthur Schnitzler: \emph{The Letters of Arthur Schnitzler to Hermann Bahr}. Edited, annotated, and with an introduction, by Donald G.
                        Daviau. Chapel Hill: \emph{The University of North Carolina Press} 1978, S. 72 (University of North Carolina studies in the Germanic languages
                        and literatures, 89).} \weitereDrucke{2) Hermann Bahr, Arthur Schnitzler: \emph{Briefwechsel, Aufzeichnungen, Dokumente (1891–1931)}. Hg. Kurt Ifkovits und Martin Anton Müller. Göttingen: \emph{Wallstein} 2018, S. 216.} }\toendnotes[C]{\smallbreak}\pstart{}{\pb}lieber
                  Hermann,\pend\pstart
           ich danke dir ſehr für dein neues \label{K_L01183_1v}\edtext{\textcolor{green}{Buch}{}\ledrightnote{→\textcolor{green}{Wirkung in die Ferne und Anderes}}}{\lemma{\textnormal{\emph{Buch}}}\Cendnote{\textnormal{\textcolor{blue}{Hermann Bahr}: \emph{\textcolor{green}{Wirkung in die Ferne und Anderes}}. Wien: \emph{\textcolor{brown}{Wiener Verlag}}{ }1902.}}}\label{K_L01183_1h}. Die \label{K_L01183_2v}\edtext{\textcolor{green}{Titelnovelle}{}\ledrightnote{→\textcolor{green}{Wirkung in die Ferne}}}{\lemma{\textnormal{\emph{Titelnovelle}}}\Cendnote{\textnormal{\emph{\textcolor{green}{Wirkung in die Ferne}}, zuerst erschienen in:
                        \emph{\textcolor{brown}{Neues Wiener Tagblatt}}, Jg. 34, Nr. 103,
                        15. 4. 1900, S. 79–85.}}}\label{K_L01183_2h} hat mich beſonders
               intereſſirt; du haſt vielleicht bemerkt, daſs in der Erzählg des \textcolor{green}{Puppenſpielers}{}\ledrightnote{→\textcolor{green}{Der Puppenspieler}} von dem \label{K_L01183_3v}\edtext{Mann in der Eisſnbahn}{\lemma{\textnormal{\emph{Mann in der Eisſnbahn}}}\Cendnote{\textnormal{\textcolor{blue}{Arthur Schnitzler}: \emph{\textcolor{green}{Marionetten. Drei Einakter}}. Berlin: \emph{\textcolor{brown}{S. Fischer}}{ }1906, S. 18–19.}}}\label{K_L01183_3h} ein ähnliches Thema leicht angerührt
               iſt. In {\pb}dem Geſpräch
                  »\label{K_L01183_4v}\edtext{\textcolor{green}{Räuber u \damage{M}örder}{}\ledrightnote{\textcolor{green}{Räuber und Mörder}}}{\lemma{\textnormal{\emph{Räuber u Mörder}}}\Cendnote{\textnormal{\emph{\textcolor{green}{Räuber und Mörder}}, zuerst erschienen in: \emph{\textcolor{brown}{Neues Wiener Tagblatt}}, Jg. 34, Nr. 151,
                        3. 6. 1900, S. 2–3.}}}\label{K_L01183_4h}« erzählſt du ganz flüchtig
               eine Geſchichte, die mir ein geborner Schwank ſcheint: von dem Hofrath, der dem Dieb
               bietet, ihn nicht anzuzeigen. Wäre ich der \label{K_L01183_5v}\edtext{\textcolor{pink}{liebe Auguſtin}{}\ledrightnote{\textcolor{pink}{Jung-Wiener Theater zum Lieben Augustin}}}{\lemma{\textnormal{\emph{liebe Auguſtin}}}\Cendnote{\textnormal{von \textcolor{blue}{Salten} geleitetes Kabarett}}}\label{K_L01183_5h}, ſo redete ich dir zu, die Scene zu
               ſchreiben. –\pend
           \pstart
           Manches hab ich ſchon gekannt, und mit Vergnügen wieder {\pb}geleſen. Lieb iſt die
                  \label{K_L01183_6v}\edtext{\textcolor{green}{Pantomime}{}\ledrightnote{\textcolor{green}{Die Pantomime vom braven Manne}}}{\lemma{\textnormal{\emph{Pantomime}}}\Cendnote{\textnormal{\emph{\textcolor{green}{Die Pantomime vom braven Manne}}, zuerst
                     erschienen in: \emph{\textcolor{green}{Das Magazin für Litteratur}},
                     Jg. 62, Nr. 6, 11. 2. 1893, Sp. 93–95.}}}\label{K_L01183_6h}. Wird ſie wer
                  \label{K_L01183_7v}\edtext{componiren}{\lemma{\textnormal{\emph{componiren}}}\Cendnote{\textnormal{vgl. Arthur Schnitzler an Hermann Bahr, 24. 8. 1918}}}\label{K_L01183_7h}? \pend
           \pstart
           Ich grüß dich herzlich{\\[\baselineskip]}dein{\\[\baselineskip]}\spacefill\mbox{Arthur}\pend
           \leftskip=0em{}\pstart
           26. X. 901\pend
           \endnumbering\briefempfaengerindex{Bahr, Hermann@\textsc{Bahr, Hermann}!zzzSchnitzler, Arthur@\emph{von Arthur Schnitzler}!1901-10-261@{26. 10. 1901}|)be}\mylabel{h}  \normalsize

\doendnotes{C}
\bigskip
\vfill

\clearpage

\footnotesize

\lohead{\textsc{register}}

% Definiere theindex-Environment komplett neu ohne reledmac
\makeatletter
\renewenvironment{theindex}{%
  \section*{\indexname}%
  \setlength{\parindent}{0pt}%
  \setlength{\parskip}{0pt plus 0.3pt}%
  \let\item\@idxitem
}{%
  \clearpage
}
\makeatother

\IfFileExists{\jobname-pw.ind}{\input{\jobname-pw.ind}}{}

\end{document}

      