%% latex-korrekturansicht-vorspann.tex
%% Vorspann für die Korrekturansicht.
%% Lädt die gemeinsame Datei latex-vorspann.tex mit gesetztem Schalter.

\newif\ifkorrekturansicht
\korrekturansichttrue

\input{../tex-inputs/latex-vorspann}


               \section[Arthur Schnitzler an Auguste Hauschner, 12. 10. 1908]{ Arthur Schnitzler an Auguste Hauschner,
               12. 10. 1908}\nopagebreak\mylabel{v}\rehead{ }\normalsize\beginnumbering\briefempfaengerindex{Hauschner, Auguste@\textsc{Hauschner, Auguste}!zzzSchnitzler, Arthur@\emph{von Arthur Schnitzler}!1908-10-121@{12. 10. 1908}|(be} \toendnotes[C]{\smallbreak\pagebreak[2]} \Standort{DLA, A:Schnitzler, HS.1985.1.955.}
\physDesc{Brief, 2 Blätter, 2 Seiten, maschineller Durchschlag
\newline{}Schreibmaschine
\newline{}Handschrift: 1) Bleistift, lateinische Kurrent (\noindent{}»Hauschner«, dasselbe neuerlich am 2. Blatt
                              und dort auch Datierung:
                              »12/10 08«)\hspace{1em}2) roter Buntstift (\noindent{}vier Unterstreichungen)\hspace{1em}}\toendnotes[C]{\smallbreak}\pstart
           \raggedleft{}{\pb}12. Okt. 08.\pend
           \pstart{}Verehrte Frau,\pend\pstart
           Ich weiss natürlich nicht mit Bestimmtheit zu sagen, in welchen Zeitungen
               Besprechungen meines \textcolor{green}{Roman}{}\ledrightnote{→\textcolor{green}{Der Weg ins Freie. Roman}}s noch
               nicht erschienen sind, da ich ja wahrscheinlich nicht alle Blätter zu Gesicht
               bekommen habe, in denen Kritiken veröffentlicht waren. Nur aufs gerate Wohl kann ich
               einige Zeitungen nennen, von denen ich nicht weiss, ob sie schon etwas gebracht
               haben, zum Beispiel: »\textcolor{brown}{Tag}{}\ledrightnote{\textcolor{brown}{Der Tag}}«, »\textcolor{brown}{Nord und Süd}{}\ledrightnote{\textcolor{brown}{Nord und Süd}}«, »\textcolor{brown}{Westermann}{}\ledrightnote{\textcolor{brown}{Westermanns Monatshefte}}«,
                  »\textcolor{brown}{deutsche Revue}{}\ledrightnote{\textcolor{brown}{Deutsche Revue. Eine Monatsschrift}}«, »\textcolor{brown}{Neue Revue}{}\ledrightnote{\textcolor{brown}{Neue Revue. Wochenschrift für das öffentliche Leben}}« u. s. w. Gewiss haben die meisten dieser Blätter
               ständige Berichterstatter und so kann ich Ihnen beim besten Willen keinen Rat
               erteilen. Dass Sie aber irgendwo vergeblich anklopfen könnten, wo die Besprechung
               über meinen \textcolor{green}{Roman}{}\ledrightnote{→\textcolor{green}{Der Weg ins Freie. Roman}} noch nicht
               vergeben wäre, kann ich mir kaum denken und ich möchte gewiss nicht gern darauf
               verzichten Sie irgendwo gedruckt zu lesen, umsoweniger als mir ebenso wie Ihnen nicht
               wenige vollkommen verständnislose zu Gesicht gekommen sind. Ich darf Sie wohl darum
               bitten, mir Ihre \textcolor{green}{Kritik}{}\ledrightnote{→\textcolor{green}{Der Weg ins Freie}} nach
               Erscheinen zuzusenden, danke Ihnen sehr für Ihr Interesse und jetzt da ich ihn
               gelesen habe {\pb}nochmals und herzlich für Ihren \textcolor{green}{Roman}{}\ledrightnote{→\textcolor{green}{Die Familie Lowositz. Roman}}.\pend
           \pstart
           In aufrichtiger Hochschätzung{\\[\baselineskip]}Ihr sehr ergebener\pend
           \leftskip=0em{}{\bigskip}\pstart
           \noindent{}Frau Auguste Hauschner, \textcolor{pink}{Berlin}{}\ledrightnote{\textcolor{pink}{Berlin}}.\pend
           \endnumbering\briefempfaengerindex{Hauschner, Auguste@\textsc{Hauschner, Auguste}!zzzSchnitzler, Arthur@\emph{von Arthur Schnitzler}!1908-10-121@{12. 10. 1908}|)be}\mylabel{h}  \normalsize

\doendnotes{C}
\bigskip
\vfill

\clearpage

\footnotesize

\lohead{\textsc{register}}

% Definiere theindex-Environment komplett neu ohne reledmac
\makeatletter
\renewenvironment{theindex}{%
  \section*{\indexname}%
  \setlength{\parindent}{0pt}%
  \setlength{\parskip}{0pt plus 0.3pt}%
  \let\item\@idxitem
}{%
  \clearpage
}
\makeatother

\IfFileExists{\jobname-pw.ind}{\input{\jobname-pw.ind}}{}

\end{document}

      