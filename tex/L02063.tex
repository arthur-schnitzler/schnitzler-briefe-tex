%% latex-korrekturansicht-vorspann.tex
%% Vorspann für die Korrekturansicht.
%% Lädt die gemeinsame Datei latex-vorspann.tex mit gesetztem Schalter.

\newif\ifkorrekturansicht
\korrekturansichttrue

\input{../tex-inputs/latex-vorspann}


               \section[Arthur Schnitzler an Richard Beer-Hofmann, 13. 5. 1912]{ Arthur Schnitzler an Richard Beer-Hofmann, 13. 5. 1912}\nopagebreak\mylabel{v}\rehead{ }\normalsize\beginnumbering\briefempfaengerindex{Beer-Hofmann, Richard@\textsc{Beer-Hofmann, Richard}!zzzSchnitzler, Arthur@\emph{von Arthur Schnitzler}!1912-05-131@{13. 5. 1912}|(be} \toendnotes[C]{\smallbreak\pagebreak[2]} \Standort{YCGL, MSS 31.}
\physDesc{Bildpostkarte
\newline{}Handschrift: Bleistift, deutsche Kurrent\newline{}Versand: 1) Stempel: »\nobreak{}\oindex{Brijuni@\textbf{Brijuni}, \emph{https://www.geonames.org/ontologyP.PPL}|pwk}Insel Brioni i. d. Adria, Alle Rechte vorbehalten.\nobreak{}«.  2) Stempel: »\nobreak{}\oindex{Brijuni@\textbf{Brijuni}, \emph{https://www.geonames.org/ontologyP.PPL}|pwk}Brioni, 13. 5. 12\nobreak{}«. }\toendnotes[C]{\smallbreak}\pstart{}{\pb}\textsc{Dr. Richard BeerHofmann aus \textcolor{pink}{Wien}{}\ledrightnote{\textcolor{pink}{Wien}}}\pend{}\pstart{}\textsc{\textcolor{pink}{Hotel Bauer u Grünwald}{}\ledrightnote{\textcolor{pink}{Grand Hotel Bauer-Grünwald}}}\pend{}\pstart{}\textsc{\textcolor{pink}{Venedig}{}\ledrightnote{\textcolor{pink}{Venedig}}}\pend{}\pstart{}nachſenden: \textsc{(nach \textcolor{pink}{Wien
                        XVIII Hasenauerstr 59}{}\ledrightnote{\textcolor{pink}{Hasenauerstraße}})}\pend{}{\bigskip}\pstart
           \noindent{}\centering{}{\pb}\textcolor{gray}{\textbf{{[}Römische Ausgrabungen{]}}}\pend
           \pstart
           \centering{}13. 5. 912.\pend
           \pstart
           {\pb}\textcolor{pink}{Brioni}{}\ledrightnote{\textcolor{pink}{Brijuni}} iſt \uline{bezaubernd}. Hotel vorzüglich. Wir haben gemiethet. Rathen Ihnen von Herzen
               das gleiche! Meer, Wald, Wieſen, Vergangenheit.\pend
           \pstart Herzlichſt Ihr \spacefill\mbox{A.}\pend{}\pstart
           \noindent{}Hoffen \uline{\label{KLL02063_Beer-Hofmann-1v}\edtext{Mittwoch}{\lemma{\textnormal{\emph{Mittwoch}}}\Cendnote{\textnormal{siehe A. S.: \emph{Tagebuch}, 15. 5. 1912}}}\label{KLL02063_Beer-Hofmann-1h} Abend} in \textcolor{pink}{Vened}{}\ledrightnote{\textcolor{pink}{Venedig}}{ }\label{T_L02063-1v}\edtext{\textcolor{pink}{\textsc{Europe}}{}\ledrightnote{\textcolor{pink}{Hotel de l’Europe}} zu ſein.}{\lemma{\textnormal{\emph{Europe zu ſein.}}}\Cendnote{\textnormal{quer zum Text entlang
                     des oberen Kartenrandes}}}\label{T_L02063-1h}\pend
           \endnumbering\briefempfaengerindex{Beer-Hofmann, Richard@\textsc{Beer-Hofmann, Richard}!zzzSchnitzler, Arthur@\emph{von Arthur Schnitzler}!1912-05-131@{13. 5. 1912}|)be}\mylabel{h}  \normalsize

\doendnotes{C}
\bigskip
\vfill

\clearpage

\footnotesize

\lohead{\textsc{register}}

% Definiere theindex-Environment komplett neu ohne reledmac
\makeatletter
\renewenvironment{theindex}{%
  \section*{\indexname}%
  \setlength{\parindent}{0pt}%
  \setlength{\parskip}{0pt plus 0.3pt}%
  \let\item\@idxitem
}{%
  \clearpage
}
\makeatother

\IfFileExists{\jobname-pw.ind}{\input{\jobname-pw.ind}}{}

\end{document}

      