%% latex-korrekturansicht-vorspann.tex
%% Vorspann für die Korrekturansicht.
%% Lädt die gemeinsame Datei latex-vorspann.tex mit gesetztem Schalter.

\newif\ifkorrekturansicht
\korrekturansichttrue

\input{../tex-inputs/latex-vorspann}


               \section[Arthur Schnitzler an Robert Adam, 20. 12. {[}1918{]}]{ Arthur Schnitzler an Robert Adam, 20. 12. {[}1918{]}}\nopagebreak\mylabel{v}\rehead{ }\normalsize\beginnumbering\briefempfaengerindex{Adam, Robert@\textsc{Adam, Robert}!zzzSchnitzler, Arthur@\emph{von Arthur Schnitzler}!1918-12-201@{20. 12. {[}1918{]}}|(be} \toendnotes[C]{\smallbreak\pagebreak[2]} \Standort{DLA, 96.34.2/16.}
\physDesc{Postkarte
\newline{}Handschrift: Bleistift, deutsche Kurrent\newline{}Versand: 1) Stempel: »\nobreak{}\textcolor{gray}{W}ien, 20. {[}XII. 18{]}\nobreak{}«.  2) Stempel: »\nobreak{}20. XII. 1\textcolor{gray}{8}, 5\nobreak{}«. }\pstart{}{\pb}\textcolor{gray}{\textbf{Dr. Arthur Schnitzler}}\pend{}\pstart{}\textcolor{gray}{\textbf{\textcolor{pink}{Wien XVIII. Sternwartestrasse 71}{}\ledrightnote{\textcolor{pink}{Sternwartestraße}}}}\pend{}{\bigskip}\pstart{}\textsc{Herrn Dr. Rob. Adam}\pend{}\pstart{}\textsc{Pollak},\pend{}\pstart{}\textcolor{pink}{\textsc{Wien} XII}{}\ledrightnote{\textcolor{pink}{XII., Meidling}}\pend{}\pstart{}\textcolor{pink}{\textsc{Meidling. Hptstr} 58}{}\ledrightnote{\textcolor{pink}{Meidlinger Hauptstraße}}\pend{}{\bigskip}\pstart
           {\pb}\textcolor{gray}{\textbf{A. S.}}\hfill 4. 11. 18\pend
           \pstart{}lieber Herr Doktor,\pend\pstart
           haben Sie am Freitag gegen 7 Uhr Abend Zeit, ſo ſind
                    Sie willko{\geminationm}en\pend
           \pstart
           Ihrem ſehr ergebn\textcolor{gray}{en}{\\[\baselineskip]}\spacefill\mbox{Arthur Schnitzler}\pend
           \leftskip=0em{}\endnumbering\briefempfaengerindex{Adam, Robert@\textsc{Adam, Robert}!zzzSchnitzler, Arthur@\emph{von Arthur Schnitzler}!1918-12-201@{20. 12. {[}1918{]}}|)be}\mylabel{h}  \normalsize

\doendnotes{C}
\bigskip
\vfill

\clearpage

\footnotesize

\lohead{\textsc{register}}

% Definiere theindex-Environment komplett neu ohne reledmac
\makeatletter
\renewenvironment{theindex}{%
  \section*{\indexname}%
  \setlength{\parindent}{0pt}%
  \setlength{\parskip}{0pt plus 0.3pt}%
  \let\item\@idxitem
}{%
  \clearpage
}
\makeatother

\IfFileExists{\jobname-pw.ind}{\input{\jobname-pw.ind}}{}

\end{document}

      