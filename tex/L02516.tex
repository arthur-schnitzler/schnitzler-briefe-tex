%% latex-korrekturansicht-vorspann.tex
%% Vorspann für die Korrekturansicht.
%% Lädt die gemeinsame Datei latex-vorspann.tex mit gesetztem Schalter.

\newif\ifkorrekturansicht
\korrekturansichttrue

\input{../tex-inputs/latex-vorspann}


               \section[Arthur Schnitzler an Gerty von Hofmannsthal, 2. 8. 1929]{ Arthur Schnitzler an Gerty von Hofmannsthal, 2. 8. 1929}\nopagebreak\mylabel{v}\rehead{ }\normalsize\beginnumbering\briefempfaengerindex{Hofmannsthal, Gertrude von@\textsc{Hofmannsthal, Gertrude von}!zzzSchnitzler, Arthur@\emph{von Arthur Schnitzler}!1929-08-021@{2. 8. 1929}|(be} \toendnotes[C]{\smallbreak\pagebreak[2]} \Standort{FDH, Hs-31346,2.}
\physDesc{Brief, 1 Blatt (Briefpapier mit Trauerrand), 2 Seiten
\newline{}Handschrift: schwarze Tinte, lateinische Kurrent
\newline{}Hofmannsthal: mit schwarzer Tinte beschriftet: »\textsc{erledigt}« }\pstart
           \raggedleft{}{\pb}\textcolor{pink}{Wien}{}\ledrightnote{\textcolor{pink}{Wien}}, 2/8 929\pend
           \pstart
           liebe Gerty, die Briefe sind angelangt, es sind auch einige wenige
               von \textcolor{blue}{Gustav Schwarzkopf}{}\ledrightnote{\textcolor{blue}{Gustav Schwarzkopf}} und \textcolor{blue}{Felix Salten}{}\ledrightnote{\textcolor{blue}{Felix Salten}} aus der gleichen Zeit dabei. \textcolor{gray}{Indß}
               habe ich mir die \uline{Briefe \textcolor{blue}{Hugo}{}\ledrightnote{\textcolor{blue}{Hugo von Hofmannsthal}}s an \textcolor{blue}{G. Schw.}{}\ledrightnote{\textcolor{blue}{Gustav Schwarzkopf}}} von diesem geben lassen, dabei waren auch etliche ungedruckte Gedichte – ich
               habe, speciell in die Briefe vorläufg nur flüchtig hineingeblickt – es sind besondere
               Briefe aus der \textcolor{gray}{früherlieg} Zeit, – ganz wunderbares. Vor allem würd
               ich \introOben{}an Ihrer Stelle\introOben{} dies alles (es ist nicht übermäßg viel)
               abschreiben lassen, eventuell gleich in 2 Exemplaren – Soll ich dieses Paket (gleich
               mit den Briefen \textcolor{blue}{Hugo}{}\ledrightnote{\textcolor{blue}{Hugo von Hofmannsthal}}s an mich) {\pb}(vielfach undatiert) nach \textcolor{pink}{Aussee}{}\ledrightnote{\textcolor{pink}{Bad Aussee}} schicken, oder möchten Sie, dſs \introOben{}ich\introOben{} die
               Abschriften \substVorne{}\textsuperscript{aus}\substDazwischen{}der Briefe von\substHinten{}{ }\textcolor{blue}{Schwarzkopf}{}\ledrightnote{\textcolor{blue}{Gustav Schwarzkopf}}{ }\uline{hier} besorgen lasse, (was erst im
                  September möglich wäre.)\pend
           \pstart
           Ich hoffe liebe Gerty die Tage in \textcolor{pink}{Aussee}{}\ledrightnote{\textcolor{pink}{Bad Aussee}} sind für
               Sie und die \textcolor{gray}{I}hren so gut und ruhig wie sie eben sein können. In
               Freundschaft mit Grüßen an Alle\pend
           \pstart
           Ihr{\\[\baselineskip]}\spacefill\mbox{Arthur}\pend
           \leftskip=0em{}\endnumbering\briefempfaengerindex{Hofmannsthal, Gertrude von@\textsc{Hofmannsthal, Gertrude von}!zzzSchnitzler, Arthur@\emph{von Arthur Schnitzler}!1929-08-021@{2. 8. 1929}|)be}\mylabel{h}  \normalsize

\doendnotes{C}
\bigskip
\vfill

\clearpage

\footnotesize

\lohead{\textsc{register}}

% Definiere theindex-Environment komplett neu ohne reledmac
\makeatletter
\renewenvironment{theindex}{%
  \section*{\indexname}%
  \setlength{\parindent}{0pt}%
  \setlength{\parskip}{0pt plus 0.3pt}%
  \let\item\@idxitem
}{%
  \clearpage
}
\makeatother

\IfFileExists{\jobname-pw.ind}{\input{\jobname-pw.ind}}{}

\end{document}

      