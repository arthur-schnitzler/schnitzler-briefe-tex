%% latex-korrekturansicht-vorspann.tex
%% Vorspann für die Korrekturansicht.
%% Lädt die gemeinsame Datei latex-vorspann.tex mit gesetztem Schalter.

\newif\ifkorrekturansicht
\korrekturansichttrue

\input{../tex-inputs/latex-vorspann}


               \section[Hermann Bahr an Arthur Schnitzler, 16. 3. 1907]{ Hermann Bahr an Arthur Schnitzler, 16. 3. 1907}\nopagebreak\mylabel{v}\rehead{ }\normalsize\beginnumbering\briefempfaengerindex{Schnitzler, Arthur@\textsc{Schnitzler, Arthur}!zzzBahr, Hermann@\emph{von Hermann Bahr}!1907-03-161@{16. 3. 1907}|(be} \toendnotes[C]{\smallbreak\pagebreak[2]} \Standort{CUL, Schnitzler, B 5b.}
\physDesc{Kartenbrief
\newline{}Handschrift: schwarze Tinte, deutsche Kurrent\newline{}Versand: 1) Stempel: »\nobreak{}\oindex{Berlin@\textbf{Berlin}, \emph{https://www.geonames.org/ontologyP.PPLC}|pwk}Berlin. N.W., 16. 3. 07, 8–9N\nobreak{}«.  2) Stempel: »\nobreak{}Bestellt, \oindex{XVIII., Waehring@\textbf{XVIII., Währing}, \emph{Bezirk (A.BZK)}|pwk}18/1 Wien, 18. 3. 07, 9\nobreak{}«. \newline{}Ordnung: mit Bleistift von unbekannter Hand
                           nummeriert: »145« }\buchAbdrucke{\weitereDrucke{Hermann Bahr, Arthur Schnitzler: \emph{Briefwechsel, Aufzeichnungen, Dokumente (1891–1931)}. Hg. Kurt Ifkovits und Martin Anton Müller. Göttingen: \emph{Wallstein} 2018, S. 390.} }\toendnotes[C]{\smallbreak}\pstart{}{\pb}Herrn \textsc{D\textsuperscript{r} Artur Schnitzler}\pend{}\pstart{}\textsc{\textcolor{pink}{Wien XVIII}{}\ledrightnote{\textcolor{pink}{XVIII., Währing}}}\pend{}\pstart{}\textsc{\textcolor{pink}{Spöttelgasse 7}{}\ledrightnote{\textcolor{pink}{Edmund-Weiß-Gasse}}}\pend{}{\bigskip}\pstart
           \raggedleft{}{\pb}16. 3. 07\pend
           \pstart{}Lieber Artur!\pend\pstart
           »\textcolor{green}{Liebelei}{}\ledrightnote{\textcolor{green}{Liebelei. Schauspiel in drei Akten}}« ging im letzten Moment nicht, weil wir
               abſolut keine \textcolor{green}{Mizzi Schlager}{}\ledrightnote{→\textcolor{green}{Liebelei. Schauspiel in drei Akten}}
               hatten (da \textcolor{blue}{\textsc{Durieux}}{}\ledrightnote{\textcolor{blue}{Tilla Durieux}}
               gleichzeitg im \textcolor{pink}{Deutſchen}{}\ledrightnote{\textcolor{pink}{Deutsches Theater Berlin}} unentbehrlich). Dafür mache
               ich jetzt »\label{K_L01664_1v}\edtext{\textcolor{green}{Comödie der Liebe}{}\ledrightnote{\textcolor{green}{Komödie der Liebe}}}{\lemma{\textnormal{\emph{Comödie der Liebe}}}\Cendnote{\textnormal{Die Premiere der \emph{\textcolor{green}{Komödie der Liebe}} von \textcolor{blue}{Ibsen} am 25. 3. 1907 in den \textcolor{pink}{Kammerspielen des Deutschen Theaters}. Das Regiebuch findet sich in \textcolor{blue}{Bahr}s Nachlass (\emph{Theatermuseum Wien}, VM 3684 Ba).}}}\label{K_L01664_1h}«. Hoffentlich kommts im
               Herbſt zur \textcolor{green}{L.}{}\ledrightnote{→\textcolor{green}{Liebelei. Schauspiel in drei Akten}}, was ich ſchon
               wegen der \textcolor{blue}{Höflich}{}\ledrightnote{\textcolor{blue}{Lucie Höflich}}{ }ſehr möchte. Wegen »\textcolor{green}{Märchen}{}\ledrightnote{\textcolor{green}{Das Märchen. Schauspiel in drei Aufzügen}}« ſprach ich mit \textcolor{blue}{Reinhardt}{}\ledrightnote{\textcolor{blue}{Max Reinhardt}}, aber da wird man lang und viel bohren müſſen.\pend
           \pstart
           Anfang April bin ich wieder in \textcolor{pink}{Wien}{}\ledrightnote{\textcolor{pink}{Wien}} und hab \textcolor{blue}{Euch}{}\ledrightnote{→\textcolor{blue}{Olga Schnitzler}} viel von hier zu erzälen, wo
               doch \label{LL306-1v}alles, faſt alles ganz famos\label{LL306-1h}
               ist.\pend
           \pstart
           Herzlichſt{\\[\baselineskip]}mit vielen Grüßen an Deine \textcolor{blue}{Frau}{}\ledrightnote{→\textcolor{blue}{Olga Schnitzler}}{ }\spacefill\mbox{Hermann}\pend
           \leftskip=0em{}\endnumbering\briefempfaengerindex{Schnitzler, Arthur@\textsc{Schnitzler, Arthur}!zzzBahr, Hermann@\emph{von Hermann Bahr}!1907-03-161@{16. 3. 1907}|)be}\mylabel{h}  \normalsize

\doendnotes{C}
\bigskip
\vfill

\clearpage

\footnotesize

\lohead{\textsc{register}}

% Definiere theindex-Environment komplett neu ohne reledmac
\makeatletter
\renewenvironment{theindex}{%
  \section*{\indexname}%
  \setlength{\parindent}{0pt}%
  \setlength{\parskip}{0pt plus 0.3pt}%
  \let\item\@idxitem
}{%
  \clearpage
}
\makeatother

\IfFileExists{\jobname-pw.ind}{\input{\jobname-pw.ind}}{}

\end{document}

      