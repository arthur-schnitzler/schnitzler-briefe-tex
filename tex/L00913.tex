%% latex-korrekturansicht-vorspann.tex
%% Vorspann für die Korrekturansicht.
%% Lädt die gemeinsame Datei latex-vorspann.tex mit gesetztem Schalter.

\newif\ifkorrekturansicht
\korrekturansichttrue

\input{../tex-inputs/latex-vorspann}


               \section[Richard Beer-Hofmann an Arthur Schnitzler, 28. 4. 1899]{ Richard Beer-Hofmann an Arthur Schnitzler, 28. 4. 1899}\nopagebreak\mylabel{v}\rehead{ }\normalsize\beginnumbering\briefempfaengerindex{Schnitzler, Arthur@\textsc{Schnitzler, Arthur}!zzzBeer-Hofmann, Richard@\emph{von Richard Beer-Hofmann}!1899-04-281@{28. 4. 1899}|(be} \toendnotes[C]{\smallbreak\pagebreak[2]} \Standort{CUL, Schnitzler, B 8.}
\physDesc{Brief, 2 Blätter, 6 Seiten
\newline{}Handschrift: Bleistift, lateinische Kurrent\newline{}Ordnung: mit Bleistift von unbekannter Hand nummeriert:
                                    »127« }\buchAbdrucke{\weitereDrucke{Arthur Schnitzler, Richard Beer-Hofmann: \emph{Briefwechsel 1891–1931}. Hg. Konstanze Fliedl. Wien, Zürich: \emph{Europaverlag} 1992, S. 127.} }\toendnotes[C]{\smallbreak}\pstart
           \raggedleft{}{\pb}\textcolor{pink}{Spittal a. d. Drau}{}\ledrightnote{\textcolor{pink}{Spittal an der Drau}}{\\}28/IV 99\pend
           \pstart
           Lieber Arthur, ich bin hier um Wohnung zu suchen, und lese soeben
               daß eine junge \textcolor{blue}{Dame}{}\ledrightnote{→\textcolor{blue}{Juliane Déry}} zum Theil
               auch deshalb weil man ihr die Rolle der \textcolor{green}{Christine}{}\ledrightnote{→\textcolor{green}{Liebelei. Schauspiel in drei Akten}} weggeno{\geminationm}en hat, sich
               vergiften wollte. Es steht das in einer \label{K_L00913_1v}\edtext{Kärntner Zeitung, in einer Skizze}{\lemma{\textnormal{\emph{Kärntner … Skizze}}}\Cendnote{\textnormal{nicht
                  nachweisbar; inhaltliche Bedenken an der Angabe bestehen, wenn man die zwei
                  Äußerungen der in \textcolor{pink}{Berlin} lebenden \textcolor{blue}{Meyer-Förster} über ihre Freundin \textcolor{blue}{Juliane Déry} als Orientierung nimmt. In einem
                  Leserbrief unmittelbar nach dem Suizid spricht sie deutlich von »\so{tieferem} menschlichem Leiden« als Motiv (\emph{\textcolor{green}{Zu dem tragischen Hingang von Juliane Dery}}.
                     In: \emph{\textcolor{green}{Berliner Tageblatt}}, Jg. 28, Nr. 168,
                        2. 4. 1899, S. 3). In einem längeren Beitrag (\emph{\textcolor{green}{Juliane Déry. Ein Nachruf}}. In: \emph{\textcolor{green}{Wiener Rundschau}}, Jg. 3, Nr. 11,
                        15. 4. 1899, S. 265–267) erwähnt sie ebenfalls
                  neuerliche Theaterambitionen der Toten.}}}\label{K_L00913_1h} von \textcolor{blue}{Elsbeth {\pb}Meyer-Förster}{}\ledrightnote{\textcolor{blue}{Elsbeth Meyer-Förster}}.
               Sie werden also auch hier durch Litteratur in der Litteratur – man könnte dies mit
               dem Quadratzeichen ausdrücken – berühmt. \label{K_L00913_2v}\edtext{Morgen wenn man Ihre \textcolor{green}{Stücke}{}\ledrightnote{\textcolor{green}{Der grüne Kakadu – Paracelsus – Die Gefährtin. Drei Einakter}}}{\lemma{\textnormal{\emph{Morgen … Stücke}}}\Cendnote{\textnormal{\textcolor{pink}{Berlin}er Premiere von \emph{\textcolor{green}{Der grüne Kakadu – Paracelsus – Die Gefährtin}}.}}}\label{K_L00913_2h} gibt,
               werde ich hier in der Wirtsstube sitzen und so wie heute die Glocken sieben {\pb}läuten hören. Wenn ich bis dahin
               nicht todt bin; man soll überhaupt nicht »ich werde« sagen, es ist i{\geminationm}er eine Provokation des Schicksals, und wenn ich morgen
               todt bin meint dann das du{\geminationm}e Schicksal es habe einen
               glänzenden Witz gemacht.\pend
           \pstart
           {\pb}Ich wohne Zimmer Nr\textsuperscript{o} II. So steht über der Thür, das Schlüsselbrett und das
               Stubenmädchen haben mir verrathen daß II früher 13 hieß – Freitag ist auch noch
               gerade heute. Jetzt weiß ich nicht: Bleib ich auf Nr\textsuperscript{o} 13,
               so wird das vielleicht als Provocation aufgefasst; {\pb}wechsle ich das Zi{\geminationm}er, so heißt es: Damit entko{\geminationm}t man mir nicht. Auch daß ich das so niederschreibe,
               wird vielleicht als fauler Ausweg durchschaut. Finden Sie nicht, daß es schwer ist
               sich zu benehmen? Grüßen Sie mir \textcolor{blue}{Brahm}{}\ledrightnote{\textcolor{blue}{Otto Brahm}}, und wenn
               Sie ihn {\pb}sehen auch \textcolor{blue}{Kerr}{}\ledrightnote{\textcolor{blue}{Alfred Kerr}}; den letzteren kenne ich zwar nur flüchtig
               aber ich laß ihn grüßen wegen des schönen \label{K_L00913_3v}\edtext{\textcolor{green}{Artikels}{}\ledrightnote{→\textcolor{green}{Hirschfeld, Halbe, Sudermann}}}{\lemma{\textnormal{\emph{Artikels}}}\Cendnote{\textnormal{\textcolor{blue}{Alfred Kerr}: \emph{\textcolor{green}{Hirschfeld, Halbe, Sudermann}}. In: \emph{\textcolor{green}{Neue
                        Deutsche Rundschau}}, Jg. 10, H. 4, April 1899,
                     S. 439–446.}}}\label{K_L00913_3h} über \textcolor{blue}{Sudermann}{}\ledrightnote{\textcolor{blue}{Hermann Sudermann}}
               etc.\pend
           \pstart
           Längstens Mittwoch bin ich wieder in \textcolor{pink}{Wien}{}\ledrightnote{\textcolor{pink}{Wien}}, – womit ich aber nichts unbescheidenes gesagt haben will –.\pend
           \pstart
           Herzlichst {\\[\baselineskip]}Ihr \spacefill\mbox{Richard}\pend
           \leftskip=0em{}\endnumbering\briefempfaengerindex{Schnitzler, Arthur@\textsc{Schnitzler, Arthur}!zzzBeer-Hofmann, Richard@\emph{von Richard Beer-Hofmann}!1899-04-281@{28. 4. 1899}|)be}\mylabel{h}  \normalsize

\doendnotes{C}
\bigskip
\vfill

\clearpage

\footnotesize

\lohead{\textsc{register}}

% Definiere theindex-Environment komplett neu ohne reledmac
\makeatletter
\renewenvironment{theindex}{%
  \section*{\indexname}%
  \setlength{\parindent}{0pt}%
  \setlength{\parskip}{0pt plus 0.3pt}%
  \let\item\@idxitem
}{%
  \clearpage
}
\makeatother

\IfFileExists{\jobname-pw.ind}{\input{\jobname-pw.ind}}{}

\end{document}

      