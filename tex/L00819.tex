%% latex-korrekturansicht-vorspann.tex
%% Vorspann für die Korrekturansicht.
%% Lädt die gemeinsame Datei latex-vorspann.tex mit gesetztem Schalter.

\newif\ifkorrekturansicht
\korrekturansichttrue

\input{../tex-inputs/latex-vorspann}


               \section[Arthur Schnitzler an Richard Beer-Hofmann, 15. 7. 1898]{ Arthur Schnitzler an Richard Beer-Hofmann, 15. 7. 1898}\nopagebreak\mylabel{v}\rehead{ }\normalsize\beginnumbering\briefempfaengerindex{Beer-Hofmann, Richard@\textsc{Beer-Hofmann, Richard}!zzzSchnitzler, Arthur@\emph{von Arthur Schnitzler}!1898-07-151@{15. 7. 1898}|(be} \toendnotes[C]{\smallbreak\pagebreak[2]} \Standort{YCGL, MSS 31.}
\physDesc{Brief, 1 Blatt, 2 Seiten, Umschlag
\newline{}Handschrift: 1) Bleistift, deutsche Kurrent\hspace{1em}2) schwarze Tinte, deutsche Kurrent (\noindent{}Umschlag)\hspace{1em}\newline{}Versand: 1) Stempel: »\nobreak{}\oindex{Graz@\textbf{Graz}, \emph{Besiedelter Ort (A.BSO)}|pwk}Graz, 15/7 98, 7.A\nobreak{}«.  2) Stempel: »\nobreak{}\oindex{Steindorf am Ossiacher See@\textbf{Steindorf am Ossiacher See}, \emph{http://www.geonames.org/ontologyA.ADM3}|pwk}\textcolor{gray}{Steindorf} am Ossiacher
                              See, 16{[} 7 98{]}\nobreak{}«. }\buchAbdrucke{\weitereDrucke{Arthur Schnitzler, Richard Beer-Hofmann: \emph{Briefwechsel 1891–1931}. Hg. Konstanze Fliedl. Wien, Zürich: \emph{Europaverlag} 1992, S. 123.} }\toendnotes[C]{\smallbreak}\pstart{}{\pb}Dr. \textsc{Arthur \damage{\textcolor{gray}{Schnitz}}ler}, \textcolor{pink}{Wien IX.
                  Frankgaſſe 1}{}\ledrightnote{\textcolor{pink}{Frankgasse}}.\pend{}{\bigskip}\pstart{}{\pb}Herrn \textsc{Dr. Richard
                     Beer-Hofmann}\pend{}\pstart{}\textsc{\textcolor{pink}{Steindorf}{}\ledrightnote{\textcolor{pink}{Steindorf am Ossiacher See}}}\pend{}\pstart{}\textsc{am \textcolor{pink}{Ossiacher}{}\ledrightnote{\textcolor{pink}{Ossiacher See}}-See}\pend{}\pstart{}\textcolor{pink}{Kärnthen}{}\ledrightnote{\textcolor{pink}{Kärnten}}.\pend{}{\bigskip}\pstart
           \raggedleft{}{\pb}\textcolor{pink}{Graz}{}\ledrightnote{\textcolor{pink}{Graz}}{ }15/7 98\pend
           \pstart
           Mein lieber Richard, So{\geminationn}tag den
                  17. verlaſſe ich \textcolor{pink}{Graz}{}\ledrightnote{\textcolor{pink}{Graz}}, komme auf mancherlei
               Art am 21. nach \textcolor{pink}{\textsc{\uline{Bad Gastein, Villa Wassing}}}{}\ledrightnote{\textcolor{pink}{Villa Dr. Wassing}}, zu meiner \textcolor{blue}{Mama}{}\ledrightnote{→\textcolor{blue}{Louise Schnitzler}}, wo ich
               bis 23. bleibe und ein Wort von Ihnen erwarte. Radle dann nach \textcolor{pink}{Salzburg}{}\ledrightnote{\textcolor{pink}{Salzburg}}, bin ſpäteſtens Dinſtag 26. dort
               und bleibe bis 28; radle da{\geminationn} (in Geſellſchaft) {\pb}nach \textcolor{pink}{Tegernſee}{}\ledrightnote{\textcolor{pink}{Tegernsee}}. \textcolor{blue}{Hugo}{}\ledrightnote{\textcolor{blue}{Hugo von Hofmannsthal}} hat Ihnen geſchrieben – werden wir uns alſo
               am 9. Auguſt circa irgendwo treffen, um \substVorne{}\textsuperscript{\textcolor{gray}{b}}\substDazwischen{}a\substHinten{}uf 10 Tage mindeſtens zuſa{\geminationm}en zu bleiben? Machen Sie’s doch möglich. Können Sie
               zwiſchen 23 u 26. d. nach \textcolor{pink}{Salzburg}{}\ledrightnote{\textcolor{pink}{Salzburg}} kommen? – Arbeiten Sie was?\pend
           \pstart
           Grüßen Sie \textcolor{blue}{Paula}{}\ledrightnote{\textcolor{blue}{Paula Beer-Hofmann}} und \textcolor{blue}{Mirjam}{}\ledrightnote{\textcolor{blue}{Mirjam Beer-Hofmann}}.\pend
           \pstart Herzlichſt Ihr \spacefill\mbox{Arthur}\pend{}\endnumbering\briefempfaengerindex{Beer-Hofmann, Richard@\textsc{Beer-Hofmann, Richard}!zzzSchnitzler, Arthur@\emph{von Arthur Schnitzler}!1898-07-151@{15. 7. 1898}|)be}\mylabel{h}  \normalsize

\doendnotes{C}
\bigskip
\vfill

\clearpage

\footnotesize

\lohead{\textsc{register}}

% Definiere theindex-Environment komplett neu ohne reledmac
\makeatletter
\renewenvironment{theindex}{%
  \section*{\indexname}%
  \setlength{\parindent}{0pt}%
  \setlength{\parskip}{0pt plus 0.3pt}%
  \let\item\@idxitem
}{%
  \clearpage
}
\makeatother

\IfFileExists{\jobname-pw.ind}{\input{\jobname-pw.ind}}{}

\end{document}

      