%% latex-korrekturansicht-vorspann.tex
%% Vorspann für die Korrekturansicht.
%% Lädt die gemeinsame Datei latex-vorspann.tex mit gesetztem Schalter.

\newif\ifkorrekturansicht
\korrekturansichttrue

\input{../tex-inputs/latex-vorspann}


               \section[Georg Brandes an Arthur Schnitzler, 21. 4. 1926]{ Georg Brandes an Arthur Schnitzler, 21. 4. 1926}\nopagebreak\mylabel{v}\rehead{ }\normalsize\beginnumbering\briefempfaengerindex{Schnitzler, Arthur@\textsc{Schnitzler, Arthur}!zzzBrandes, Georg@\emph{von Georg Brandes}!1926-04-211@{21. 4. 1926}|(be} \toendnotes[C]{\smallbreak\pagebreak[2]} \Standort{CUL, Schnitzler, B 17.}
\physDesc{Briefkarte
\newline{}Handschrift: schwarze Tinte, lateinische Kurrent
\newline{}Schnitzler: mit rotem Buntstift eine Unterstreichung \newline{}Ordnung: mit Bleistift von unbekannter Hand nummeriert: »62« }\buchAbdrucke{\weitereDrucke{Georg Brandes, Arthur Schnitzler: \emph{Ein Briefwechsel}. Hg. Kurt Bergel. Bern: \emph{Francke} 1956, S. 152–153.} }\toendnotes[C]{\smallbreak}\pstart
           \raggedleft{}{\pb}\textcolor{pink}{Kopenhagen}{}\ledrightnote{\textcolor{pink}{Kopenhagen}}{ }21 April 26\pend
           \pstart
           Mein liebster Freund Sie sind einer der wenigen Menschen, dem
                    ich nur Gutes verdanke, einen wahren geistigen Reichtum. Heute las ich zum
                    zweiten Male – nach Monaten – Ihr tiefsinniges Drama über den \textcolor{green}{\uline{Weiher}}{}\ledrightnote{\textcolor{green}{Der Gang zum Weiher. Dramatische Dichtung}}, und verstand es inniger als das erste Mal, hatte meine Freude daran. Sie
                    haben dort eine Saite angeschlagen, die in der Gegenwart selten \substVorne{}\textsuperscript{geworden ist}{\allowbreak}\substDazwischen{}gehört wird\substHinten{}; Verse klingen heutzutage selten von der Bühne, und Sie sind zu den
                    ausführlicheren Repliken älterer Zeiten zurückgekehrt. Aber Sie meistern diesen
                    Stil, und Sie \substVorne{}\textsuperscript{\textcolor{gray}{fesslen}}{\allowbreak}\substDazwischen{}fesseln\substHinten{}. Das \textcolor{green}{Stück}{}\ledrightnote{→\textcolor{green}{Der Gang zum Weiher. Dramatische Dichtung}} ist ein
                    schönes Ganzes.\pend
           \pstart
           {\pb}Ich habe keine Zeitungen in
                    deutscher Sprache, weiss deshalb nicht, ob das \textcolor{green}{Stück}{}\ledrightnote{→\textcolor{green}{Der Gang zum Weiher. Dramatische Dichtung}} aufgeführt worden noch ob es Erfolg hatte. Sie
                    wissen, dass ich Ihnen jeglichen Erfolg wünsche. – Ich denke mir, dass ich
                        Anfang Mai um meiner Gesundheit willen nach \textcolor{pink}{Karlsbad}{}\ledrightnote{\textcolor{pink}{Karlsbad}} reise. Ich bin wol mehr als ein Dutzend Mal vor
                    dem Kriege dort gewesen. Jetzt wird es wol dort, wie überall, \strikeout{dort } ärmer sein. Die Sprache trennt mich leider
                    von Ihnen. Mein deutscher Verleger, \textcolor{blue}{Erich
                        Reiss}{}\ledrightnote{\textcolor{blue}{Erich Reiss}}, hat Fallissement gemacht. Alles was er mir schuldig war, seit
                    Jahren, ist in Rauch aufgegangen.\pend
           \pstart
           Ich hoffe, dass es Ihnen und den \textcolor{blue}{Kindern}{}\ledrightnote{→\textcolor{blue}{Heinrich Schnitzler}{\newline}→\textcolor{blue}{Lili Schnitzler}} gut geht. – Frau \textcolor{blue}{Gertrud Rung}{}\ledrightnote{\textcolor{blue}{Gertrud Rung}}, die Sie freundlich empfingen, liebt Sie
                    sehr. Ihr Freund \spacefill\mbox{Georg Brandes}\pend
           \endnumbering\briefempfaengerindex{Schnitzler, Arthur@\textsc{Schnitzler, Arthur}!zzzBrandes, Georg@\emph{von Georg Brandes}!1926-04-211@{21. 4. 1926}|)be}\mylabel{h}  \normalsize

\doendnotes{C}
\bigskip
\vfill

\clearpage

\footnotesize

\lohead{\textsc{register}}

% Definiere theindex-Environment komplett neu ohne reledmac
\makeatletter
\renewenvironment{theindex}{%
  \section*{\indexname}%
  \setlength{\parindent}{0pt}%
  \setlength{\parskip}{0pt plus 0.3pt}%
  \let\item\@idxitem
}{%
  \clearpage
}
\makeatother

\IfFileExists{\jobname-pw.ind}{\input{\jobname-pw.ind}}{}

\end{document}

      