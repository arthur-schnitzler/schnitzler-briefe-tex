%% latex-korrekturansicht-vorspann.tex
%% Vorspann für die Korrekturansicht.
%% Lädt die gemeinsame Datei latex-vorspann.tex mit gesetztem Schalter.

\newif\ifkorrekturansicht
\korrekturansichttrue

\input{../tex-inputs/latex-vorspann}


               \section[Hermann Bahr an Arthur Schnitzler, 23. 4. {[}1904{]}]{ Hermann Bahr an Arthur Schnitzler, 23. 4. {[}1904{]}}\nopagebreak\mylabel{v}\rehead{ }\normalsize\beginnumbering\briefempfaengerindex{Schnitzler, Arthur@\textsc{Schnitzler, Arthur}!zzzBahr, Hermann@\emph{von Hermann Bahr}!1904-04-231@{23. 4. {[}1904{]}}|(be} \toendnotes[C]{\smallbreak\pagebreak[2]} \Standort{CUL, Schnitzler, B 5b.}
\physDesc{Brief, 1 Blatt, 1 Seite
\newline{}Handschrift: schwarze Tinte, deutsche Kurrent
\newline{}Schnitzler: mit Bleistift Jahreszahl ergänzt: »904« \newline{}Ordnung: mit Bleistift von unbekannter Hand nummeriert: »115« }\buchAbdrucke{\weitereDrucke{Hermann Bahr, Arthur Schnitzler: \emph{Briefwechsel, Aufzeichnungen, Dokumente (1891–1931)}. Hg. Kurt Ifkovits und Martin Anton Müller. Göttingen: \emph{Wallstein} 2018, S. 306.} }\toendnotes[C]{\smallbreak}\pstart
           \raggedleft{}{\pb}23. 4.\pend
           \pstart\center{}Lieber Arthur!\pend\pstart
           Ich bin zurück, möchte Dich bald ſehen, höre leider, daß man nicht zu Dir darf, hoffe
               den \textcolor{blue}{Jüngling}{}\ledrightnote{→\textcolor{blue}{Heinrich Schnitzler}} jedoch bald
                  \label{K_L01392_1v}\edtext{geneſen}{\lemma{\textnormal{\emph{geneſen}}}\Cendnote{\textnormal{\textcolor{blue}{Heinrich} hatte die Masern (A. S. \emph{Briefe} I,481).}}}\label{K_L01392_1h} und bitte Dich dann um ein Wort,
               wann ich Dich treffe.\pend
           \pstart
           Mit vielen Grüßen an Deine \textcolor{blue}{Frau}{}\ledrightnote{→\textcolor{blue}{Olga Schnitzler}}{\\[\baselineskip]}herzlichſt{\\[\baselineskip]}Dein{\\[\baselineskip]}\spacefill\mbox{Hermann}\pend
           \leftskip=0em{}\pstart
           \noindent{}Über Deinen \textcolor{pink}{Pariſ}{}\ledrightnote{\textcolor{pink}{Paris}}er \label{K_L01392_2v}\edtext{\textcolor{green}{Rieſenerfolg}{}\ledrightnote{→\textcolor{green}{Abschiedssouper}}}{\lemma{\textnormal{\emph{Rieſenerfolg}}}\Cendnote{\textnormal{Vgl. \textcolor{blue}{Stephan Epstein} an \textcolor{blue}{Bahr},
                           15. 2. 1904, in: \emph{Briefwechsel}
                        Bahr/Schnitzler 302.}}}\label{K_L01392_2h}, von dem D\textsuperscript{r}{ }\textcolor{blue}{Epſtein}{}\ledrightnote{\textcolor{blue}{Stephan Epstein}} erzälte, hab ich mich ſo ſehr
                  gefreut.\pend
           \endnumbering\briefempfaengerindex{Schnitzler, Arthur@\textsc{Schnitzler, Arthur}!zzzBahr, Hermann@\emph{von Hermann Bahr}!1904-04-231@{23. 4. {[}1904{]}}|)be}\mylabel{h}  \normalsize

\doendnotes{C}
\bigskip
\vfill

\clearpage

\footnotesize

\lohead{\textsc{register}}

% Definiere theindex-Environment komplett neu ohne reledmac
\makeatletter
\renewenvironment{theindex}{%
  \section*{\indexname}%
  \setlength{\parindent}{0pt}%
  \setlength{\parskip}{0pt plus 0.3pt}%
  \let\item\@idxitem
}{%
  \clearpage
}
\makeatother

\IfFileExists{\jobname-pw.ind}{\input{\jobname-pw.ind}}{}

\end{document}

      