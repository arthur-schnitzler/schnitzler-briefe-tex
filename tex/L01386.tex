%% latex-korrekturansicht-vorspann.tex
%% Vorspann für die Korrekturansicht.
%% Lädt die gemeinsame Datei latex-vorspann.tex mit gesetztem Schalter.

\newif\ifkorrekturansicht
\korrekturansichttrue

\input{../tex-inputs/latex-vorspann}


               \section[Hugo von Hofmannsthal an Arthur Schnitzler, {[}18. 3. 1904{]}]{ Hugo von Hofmannsthal an Arthur Schnitzler, {[}18. 3. 1904{]}}\nopagebreak\mylabel{v}\rehead{ }\normalsize\beginnumbering\briefempfaengerindex{Schnitzler, Arthur@\textsc{Schnitzler, Arthur}!zzzHofmannsthal, Hugo von@\emph{von Hugo von Hofmannsthal}!1904-03-181@{{[}18. 3. 1904{]}}|(be} \toendnotes[C]{\smallbreak\pagebreak[2]} \Standort{CUL, Schnitzler, B 43.}
\physDesc{Telegramm
\newline{}Handschrift einer Schreibkraft: Bleistift, lateinische Kurrent
\newline{}Schnitzler: mit Bleistift datiert: »18. 3. 904.« \newline{}Ordnung: 1) beschnitten 2) mit Bleistift von unbekannter Hand nummeriert: »218«}\buchAbdrucke{\weitereDrucke{Hugo von Hofmannsthal, Arthur Schnitzler: \emph{Briefwechsel}. Hg. Therese Nickl und Heinrich Schnitzler. Frankfurt am Main: \emph{S. Fischer} 1964, S. 185.} }\toendnotes[C]{\smallbreak}\pstart
           \noindent{}{\pb}\textcolor{blue}{Mama}{}\ledrightnote{→\textcolor{blue}{Anna von Hofmannsthal}}s Befinden wirklich
               verzweifelt bitte Sie innigst morgen vormittag{ }\textcolor{pink}{Salesianergasse}{}\ledrightnote{\textcolor{pink}{Salesianergasse}} kommen wo ich auch bin eventuelle
               Absage bitte depeschieren an \textcolor{blue}{Schlesinger}{}\ledrightnote{\textcolor{blue}{Franziska Schlesinger}}{ }\textcolor{pink}{Elisabethstrasse 6}{}\ledrightnote{\textcolor{pink}{Elisabethstraße}}\pend
           \pstart \spacefill\mbox{Hugo}\pend{}\endnumbering\briefempfaengerindex{Schnitzler, Arthur@\textsc{Schnitzler, Arthur}!zzzHofmannsthal, Hugo von@\emph{von Hugo von Hofmannsthal}!1904-03-181@{{[}18. 3. 1904{]}}|)be}\mylabel{h}  \normalsize

\doendnotes{C}
\bigskip
\vfill

\clearpage

\footnotesize

\lohead{\textsc{register}}

% Definiere theindex-Environment komplett neu ohne reledmac
\makeatletter
\renewenvironment{theindex}{%
  \section*{\indexname}%
  \setlength{\parindent}{0pt}%
  \setlength{\parskip}{0pt plus 0.3pt}%
  \let\item\@idxitem
}{%
  \clearpage
}
\makeatother

\IfFileExists{\jobname-pw.ind}{\input{\jobname-pw.ind}}{}

\end{document}

      