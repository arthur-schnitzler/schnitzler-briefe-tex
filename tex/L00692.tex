%% latex-korrekturansicht-vorspann.tex
%% Vorspann für die Korrekturansicht.
%% Lädt die gemeinsame Datei latex-vorspann.tex mit gesetztem Schalter.

\newif\ifkorrekturansicht
\korrekturansichttrue

\input{../tex-inputs/latex-vorspann}


               \section[Richard Beer-Hofmann an Arthur Schnitzler, 25. 6. 1897]{ Richard Beer-Hofmann an Arthur Schnitzler,
               25. 6. 1897}\nopagebreak\mylabel{v}\rehead{ }\normalsize\beginnumbering\briefempfaengerindex{Schnitzler, Arthur@\textsc{Schnitzler, Arthur}!zzzBeer-Hofmann, Richard@\emph{von Richard Beer-Hofmann}!1897-06-251@{25. 6. 1897}|(be} \toendnotes[C]{\smallbreak\pagebreak[2]} \Standort{CUL, Schnitzler, B 8.}
\physDesc{Briefkarte
\newline{}Handschrift: Bleistift, lateinische Kurrent\newline{}Ordnung: mit Bleistift von unbekannter Hand nummeriert: »101« }\buchAbdrucke{\weitereDrucke{Arthur Schnitzler, Richard Beer-Hofmann: \emph{Briefwechsel 1891–1931}. Hg. Konstanze Fliedl. Wien, Zürich: \emph{Europaverlag} 1992, S. 111.} }\toendnotes[C]{\smallbreak}\pstart
           \raggedleft{}{\pb}\textcolor{pink}{Ischl}{}\ledrightnote{\textcolor{pink}{Bad Ischl}}{ }25/VI 97\pend
           \pstart
           Lieber Arthur ich habe kein Zimmer für Sie gewählt weil Herr \textcolor{blue}{Petter}{}\ledrightnote{\textcolor{blue}{Leopold Petter}} mir sagt er hätte 30 zu Ihrer Verfügung.
               Ich selbst mache \uline{morgen} – \uline{Samstag} – mit \textcolor{blue}{Papa}{}\ledrightnote{→\textcolor{blue}{Hermann Beer}}, \textcolor{blue}{Onkel}{}\ledrightnote{→\textcolor{blue}{Sigmund Beer}}, \textcolor{blue}{Tante}{}\ledrightnote{→\textcolor{blue}{Agnes Beer}} einen Ausflug nach \textcolor{pink}{Gmunden}{}\ledrightnote{\textcolor{pink}{Gmunden}}{ }{\pb}und bin um 6 oder
                  8 Abends wieder in \textcolor{pink}{Ischl}{}\ledrightnote{\textcolor{pink}{Bad Ischl}}. Um
                  8 nachtmalen wir und um ½ 9 gehe ich weg – Wollen
               Sie mich also noch Samstag
                sehn, dann sind Sie zwischen 8 u
                  ½ 9 bei mir. Von Herzen Ihr \spacefill\mbox{R}\pend
           \endnumbering\briefempfaengerindex{Schnitzler, Arthur@\textsc{Schnitzler, Arthur}!zzzBeer-Hofmann, Richard@\emph{von Richard Beer-Hofmann}!1897-06-251@{25. 6. 1897}|)be}\mylabel{h}  \normalsize

\doendnotes{C}
\bigskip
\vfill

\clearpage

\footnotesize

\lohead{\textsc{register}}

% Definiere theindex-Environment komplett neu ohne reledmac
\makeatletter
\renewenvironment{theindex}{%
  \section*{\indexname}%
  \setlength{\parindent}{0pt}%
  \setlength{\parskip}{0pt plus 0.3pt}%
  \let\item\@idxitem
}{%
  \clearpage
}
\makeatother

\IfFileExists{\jobname-pw.ind}{\input{\jobname-pw.ind}}{}

\end{document}

      