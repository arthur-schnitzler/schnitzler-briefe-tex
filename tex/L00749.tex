%% latex-korrekturansicht-vorspann.tex
%% Vorspann für die Korrekturansicht.
%% Lädt die gemeinsame Datei latex-vorspann.tex mit gesetztem Schalter.

\newif\ifkorrekturansicht
\korrekturansichttrue

\input{../tex-inputs/latex-vorspann}


               \section[Julius Rodenberg an Arthur Schnitzler, 13. 12. 1897]{ Julius Rodenberg an Arthur Schnitzler, 13. 12. 1897}\nopagebreak\mylabel{v}\rehead{ }\normalsize\beginnumbering\briefempfaengerindex{Schnitzler, Arthur@\textsc{Schnitzler, Arthur}!zzzRodenberg, Julius@\emph{von Julius Rodenberg}!1897-12-131@{13. 12. 1897}|(be} \toendnotes[C]{\smallbreak\pagebreak[2]} \Standort{CUL, Schnitzler, B 85.}
\physDesc{Brief, 1 Blatt, 2 Seiten
\newline{}Handschrift: schwarze Tinte, deutsche Kurrent
\newline{}Schnitzler: 1) mit rotem Buntstift vereinzelte Unterstreichungen 2) mit Bleistift beschriftet: »\textsc{Rodenberg}«}\pstart
           \noindent{}\centering{}{\pb}\textcolor{gray}{\textbf{\textcolor{brown}{Deutsche Rundschau}{}\ledrightnote{\textcolor{brown}{Deutsche Rundschau}}}}\pend
           \pstart
           \noindent{}\textcolor{gray}{\textbf{Expedition u. Redaction:}}\hfill \textcolor{gray}{\textbf{Herausgeber:}}\pend
           \pstart
           \textcolor{gray}{\textbf{\textcolor{brown}{Gebrüder Paetel}{}\ledrightnote{\textcolor{brown}{Gebrüder Paetel Verlag}} in \textcolor{pink}{Berlin}{}\ledrightnote{\textcolor{pink}{Berlin}}}}\hfill \textcolor{gray}{\textbf{Julius Rodenberg in \textcolor{pink}{Berlin}{}\ledrightnote{\textcolor{pink}{Berlin}}}}\pend
           \pstart
           \textcolor{gray}{\textbf{(\textcolor{blue}{Elwin
                                Paetel}{}\ledrightnote{\textcolor{blue}{Elwin Paetel}})}}\hfill \textcolor{gray}{\textbf{\textcolor{pink}{W., Margarethenstr. 1}{}\ledrightnote{\textcolor{pink}{Margaretenstraße}}.}}\pend
           \pstart
           \textcolor{gray}{\textbf{\textcolor{pink}{W., Lützowstr. 7}{}\ledrightnote{\textcolor{pink}{Lützowstraße}}.}}\pend
           \pstart
           \raggedleft{}\textbf{\textcolor{gray}{\textbf{\textcolor{pink}{Berlin W.}{}\ledrightnote{\textcolor{pink}{Berlin}},}} den}{ }13. Dec. \textcolor{gray}{\textbf{189}}7.\pend
           \pstart{}Hochgeehrter Herr Doctor!\pend\pstart
           Durch meinen Schwager Dr. \textcolor{blue}{Ed. Schiff}{}\ledrightnote{\textcolor{blue}{Eduard Liberius Schiff}} iſt mir
                    die höchſt erfreuliche Kunde geworden, daß die »\textcolor{brown}{\textsc{Rundschau}}{}\ledrightnote{\textcolor{brown}{Deutsche Rundschau}}« ſich Hoffnung machen darf, in nicht allzuferner Zeit einen
                    novelliſtiſchen Beitrag von Ihnen zu erhalten. Längſt ſchon iſt dieß mein Wunſch
                    geweſen u. wenn ich ihn nicht eher ausſprach, ſo werden Sie ſich das daraus
                    erklären können, daß ich mich nicht gern einem Refus ausgeſetzt haben würde. Nun
                    iſt aber bei Ihnen freundliches Entgegenko{\geminationm}en
                    gefunden, will ich nicht zögern, Ihnen dafür zu danken u. meine Bitte direct zu
                    wiederholen. Daß Sie dieſer im Augenblick nicht zu willfahren vermöchten, hab’
                    ich vorausgeſetzt, u. darauf ko{\geminationm}t es mir auch nicht
                    an; es genügt mir, zu wißen, daß Sie bei nächſter Gelegenheit unſerer
                    Zeitſchrift gedenken wollen, u. {\pb}ich
                    bitte nur, mich eintretenden Falls zu benachrichtigen, um Sie nicht unnöthig
                    lang mit dem Abdruck warten laßen zu müßen.\pend
           \pstart
           Mit dem Ausdruck beſonderer Hochachtung{\\[\baselineskip]}Ihr ergebener{\\[\baselineskip]}\spacefill\mbox{Dr Julius Rodenberg.}\pend
           \leftskip=0em{}\endnumbering\briefempfaengerindex{Schnitzler, Arthur@\textsc{Schnitzler, Arthur}!zzzRodenberg, Julius@\emph{von Julius Rodenberg}!1897-12-131@{13. 12. 1897}|)be}\mylabel{h}  \normalsize

\doendnotes{C}
\bigskip
\vfill

\clearpage

\footnotesize

\lohead{\textsc{register}}

% Definiere theindex-Environment komplett neu ohne reledmac
\makeatletter
\renewenvironment{theindex}{%
  \section*{\indexname}%
  \setlength{\parindent}{0pt}%
  \setlength{\parskip}{0pt plus 0.3pt}%
  \let\item\@idxitem
}{%
  \clearpage
}
\makeatother

\IfFileExists{\jobname-pw.ind}{\input{\jobname-pw.ind}}{}

\end{document}

      