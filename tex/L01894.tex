%% latex-korrekturansicht-vorspann.tex
%% Vorspann für die Korrekturansicht.
%% Lädt die gemeinsame Datei latex-vorspann.tex mit gesetztem Schalter.

\newif\ifkorrekturansicht
\korrekturansichttrue

\input{../tex-inputs/latex-vorspann}


               \section[Max Mell an Arthur Schnitzler, 8. 12. 1909]{ Max Mell an Arthur Schnitzler, 8. 12. 1909}\nopagebreak\mylabel{v}\rehead{ }\normalsize\beginnumbering\briefempfaengerindex{Schnitzler, Arthur@\textsc{Schnitzler, Arthur}!zzzMell, Max@\emph{von Max Mell}!1909-12-081@{8. 12. 1909}|(be} \toendnotes[C]{\smallbreak\pagebreak[2]} \Standort{DLA, A:Schnitzler, HS.NZ85.1.4055, S. [7].}
\physDesc{maschinelle Abschrift}\toendnotes[C]{\smallbreak}\pstart
           \raggedleft{}{\pb}8. Dez. 1909.\pend
           \pstart{}Sehr verehrter Herr Doktor,\pend\pstart
           Kann ich Ihnen ohne allzu unbescheiden zu sein, mit einer Bitte kommen? Ich habe,
                    obwohl ich von \textcolor{blue}{Schlenther}{}\ledrightnote{\textcolor{blue}{Paul Schlenther}} natürlich noch keine
                    Entscheidung habe, mein \textcolor{green}{Stück}{}\ledrightnote{→\textcolor{green}{Die Kinder des Hauses}}
                    jetzt an das \textcolor{brown}{Deutsche Volkstheater}{}\ledrightnote{\textcolor{brown}{Volkstheater}} geschickt – würden
                    Sie die Güte haben mit einem Wort bei der Direktion nur dahin zu wirken, dass es
                    überhaupt angesehen wird und nicht in dem notwendig \label{T_L01894_1v}\edtext{ungelesenen}{\lemma{\textnormal{\emph{ungelesenen}}}\Cendnote{\textnormal{die Abschrift hat: »ungelesenem«}}}\label{T_L01894_1h} Wust des Einlaufs
                    verschwindet? Es handelt sich mir nur darum überhaupt eine Erledigung zu
                    bekommen und Sie würden mich sehr verpflichten, wenn Sie mir dazu verhelfen
                    wollten.\pend
           \pstart
           Mit den besten Empfehlungen{\\[\baselineskip]}Ihres{\\[\baselineskip]}Max Mell\pend
           \leftskip=0em{}\endnumbering\briefempfaengerindex{Schnitzler, Arthur@\textsc{Schnitzler, Arthur}!zzzMell, Max@\emph{von Max Mell}!1909-12-081@{8. 12. 1909}|)be}\mylabel{h}  \normalsize

\doendnotes{C}
\bigskip
\vfill

\clearpage

\footnotesize

\lohead{\textsc{register}}

% Definiere theindex-Environment komplett neu ohne reledmac
\makeatletter
\renewenvironment{theindex}{%
  \section*{\indexname}%
  \setlength{\parindent}{0pt}%
  \setlength{\parskip}{0pt plus 0.3pt}%
  \let\item\@idxitem
}{%
  \clearpage
}
\makeatother

\IfFileExists{\jobname-pw.ind}{\input{\jobname-pw.ind}}{}

\end{document}

      