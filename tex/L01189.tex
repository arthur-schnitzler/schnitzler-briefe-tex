%% latex-korrekturansicht-vorspann.tex
%% Vorspann für die Korrekturansicht.
%% Lädt die gemeinsame Datei latex-vorspann.tex mit gesetztem Schalter.

\newif\ifkorrekturansicht
\korrekturansichttrue

\input{../tex-inputs/latex-vorspann}


               \section[Arthur Schnitzler an Hermann Bahr, 11. 12. 1901]{ Arthur Schnitzler an Hermann Bahr, 11. 12. 1901}\nopagebreak\mylabel{v}\rehead{ }\normalsize\beginnumbering\briefempfaengerindex{Bahr, Hermann@\textsc{Bahr, Hermann}!zzzSchnitzler, Arthur@\emph{von Arthur Schnitzler}!1901-12-112@{11. 12. 1901}|(be} \toendnotes[C]{\smallbreak\pagebreak[2]} \Standort{TMW, HS AM 23346 Ba.}
\physDesc{Brief, 1 Blatt, 3 Seiten
\newline{}Handschrift: Bleistift, deutsche Kurrent\newline{}Ordnung: Lochung }\buchAbdrucke{\weitereDrucke{1) \emph{11. 12. 1901.} In: Arthur Schnitzler: \emph{The Letters of Arthur Schnitzler to Hermann Bahr}. Edited, annotated, and with an introduction, by Donald G.
                        Daviau. Chapel Hill: \emph{The University of North Carolina Press} 1978, S. 73 (University of North Carolina studies in the Germanic languages
                        and literatures, 89).} \weitereDrucke{2) Hermann Bahr, Arthur Schnitzler: \emph{Briefwechsel, Aufzeichnungen, Dokumente (1891–1931)}. Hg. Kurt Ifkovits und Martin Anton Müller. Göttingen: \emph{Wallstein} 2018, S. 220.} }\pstart
           \raggedleft{}{\pb}11. 12. 901\pend
           \pstart{}mein lieber Hermann,\pend\pstart
           ich nehme an, Direktor \textsc{\textcolor{blue}{Bukovics}{}\ledrightnote{\textcolor{blue}{Emerich von Bukovics}}} wird dir den Brief zeigen, den ich heute an ihn geſchrieben, um die Sache
               endgiltig abzuſchließen und etliche ſonderbare Auffaſſungen ſeinerſeits
               richtigzuſtellen. We{\geminationn} nicht, ſteht dir gelegentlich {\pb}eine Abſchrift zur
               Verfügung.\pend
           \pstart
           – Jedenfalls habe ich dir für deine wiederholten Verſuche, \textsc{\textcolor{blue}{Bukovics}{}\ledrightnote{\textcolor{blue}{Emerich von Bukovics}}} auf ſeine Höflichkeitsverpflichtungen (ich ſehe von den andern ab, die
               vielleicht ein Theaterdirektor gegen einen Autor haben kö{\geminationn}te) aufmerkſam zu machen, herzlichſt zu danken.{\pb}\pend
           \pstart
           {\pb}Auf baldgs
               Wiederſehen{\\[\baselineskip]}dein treuer{\\[\baselineskip]}\spacefill\mbox{Arth Sch}\pend
           \leftskip=0em{}\endnumbering\briefempfaengerindex{Bahr, Hermann@\textsc{Bahr, Hermann}!zzzSchnitzler, Arthur@\emph{von Arthur Schnitzler}!1901-12-112@{11. 12. 1901}|)be}\mylabel{h}  \normalsize

\doendnotes{C}
\bigskip
\vfill

\clearpage

\footnotesize

\lohead{\textsc{register}}

% Definiere theindex-Environment komplett neu ohne reledmac
\makeatletter
\renewenvironment{theindex}{%
  \section*{\indexname}%
  \setlength{\parindent}{0pt}%
  \setlength{\parskip}{0pt plus 0.3pt}%
  \let\item\@idxitem
}{%
  \clearpage
}
\makeatother

\IfFileExists{\jobname-pw.ind}{\input{\jobname-pw.ind}}{}

\end{document}

      