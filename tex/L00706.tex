%% latex-korrekturansicht-vorspann.tex
%% Vorspann für die Korrekturansicht.
%% Lädt die gemeinsame Datei latex-vorspann.tex mit gesetztem Schalter.

\newif\ifkorrekturansicht
\korrekturansichttrue

\input{../tex-inputs/latex-vorspann}


               \section[Hugo von Hofmannsthal an Arthur Schnitzler, {[}19. 7. 1897{]}]{ Hugo von Hofmannsthal an Arthur Schnitzler, {[}19. 7. 1897{]}}\nopagebreak\mylabel{v}\rehead{ }\normalsize\beginnumbering\briefempfaengerindex{Schnitzler, Arthur@\textsc{Schnitzler, Arthur}!zzzHofmannsthal, Hugo von@\emph{von Hugo von Hofmannsthal}!1897-07-191@{{[}19. 7. 1897{]}}|(be} \toendnotes[C]{\smallbreak\pagebreak[2]} \Standort{CUL, Schnitzler, B 43.}
\physDesc{Brief, 1 Blatt, 3 Seiten
\newline{}Handschrift: Bleistift, deutsche Kurrent
\newline{}Schnitzler: mit Bleistift falsch datiert: »1\substVorne{}\textsuperscript{8}\substDazwischen{}9\substHinten{}/7 96« \newline{}Ordnung: 1) mit Bleistift von unbekannter Hand nummeriert: »\strikeout{95}« 2) mit Bleistift von unbekannter Hand nummeriert: »78a«}\buchAbdrucke{\weitereDrucke{Hugo von Hofmannsthal, Arthur Schnitzler: \emph{Briefwechsel}. Hg. Therese Nickl und Heinrich Schnitzler. Frankfurt am Main: \emph{S. Fischer} 1964, S. 93.} }\toendnotes[C]{\smallbreak}\pstart
           \raggedleft{}{\pb}Montag.\pend
           \pstart
           \strikeout{Herr}{ }mein lieber Arthur!ich habe erſt heute erfahren, daſs \textcolor{blue}{Papa}{}\ledrightnote{→\textcolor{blue}{Hugo August von Hofmannsthal}} nächſten
                        Montag von \textcolor{pink}{hier}{}\ledrightnote{→\textcolor{pink}{Bad Fusch}} abreiſt; ſo möchte ich nicht gern den letzten Tag von hier fort
                    und wir laſſen alſo lieber das \textsc{rendez vous}. Es thut
                    mir ſehr leid, aber wenn wir beide etwas gearbeitet haben werden, wird es eine
                    große Freude ſein, uns im Spätherbſt wieder{\pb}zuſehen. Sie ſchreiben mir
                    wohl hie und da eine Zeile nach \textcolor{pink}{Italien}{}\ledrightnote{\textcolor{pink}{Italien}}, ich
                    werde Ihnen immer meine Adreſſe zuko{\geminationm}en laſſen.\pend
           \pstart
           Die \textcolor{green}{\textcolor{blue}{Mozart}{}\ledrightnote{\textcolor{blue}{Wolfgang Amadeus Mozart}}-biographie}{}\ledrightnote{→\textcolor{green}{W. A. Mozart}} iſt ein entzückendes Buch von einer unglaublichen Ausführlichkeit
                    und Intimität. Man gewinnt \textcolor{blue}{ihn}{}\ledrightnote{→\textcolor{blue}{Wolfgang Amadeus Mozart}}{ }ſehr lieb. Ich ſchicke Ihnen die
                    beiden Bände im Auguſt nach \textcolor{pink}{Wien}{}\ledrightnote{\textcolor{pink}{Wien}}.\pend
           \pstart
           {\pb}Werd ich von \textcolor{blue}{Richard}{}\ledrightnote{\textcolor{blue}{Richard Beer-Hofmann}} nie auch nur eine Zeile
                    bekommen?\pend
           \pstart
           Es ärgert mich ſehr.\pend
           \pstart
           Ich wünſche Ihnen für die nächſten 2 Monate alles Gute.\pend
           \pstart
           Von Herzen Ihr{\\[\baselineskip]}\spacefill\mbox{Hugo.}\pend
           \leftskip=0em{}\endnumbering\briefempfaengerindex{Schnitzler, Arthur@\textsc{Schnitzler, Arthur}!zzzHofmannsthal, Hugo von@\emph{von Hugo von Hofmannsthal}!1897-07-191@{{[}19. 7. 1897{]}}|)be}\mylabel{h}  \normalsize

\doendnotes{C}
\bigskip
\vfill

\clearpage

\footnotesize

\lohead{\textsc{register}}

% Definiere theindex-Environment komplett neu ohne reledmac
\makeatletter
\renewenvironment{theindex}{%
  \section*{\indexname}%
  \setlength{\parindent}{0pt}%
  \setlength{\parskip}{0pt plus 0.3pt}%
  \let\item\@idxitem
}{%
  \clearpage
}
\makeatother

\IfFileExists{\jobname-pw.ind}{\input{\jobname-pw.ind}}{}

\end{document}

      