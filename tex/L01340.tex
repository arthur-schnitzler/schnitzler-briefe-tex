%% latex-korrekturansicht-vorspann.tex
%% Vorspann für die Korrekturansicht.
%% Lädt die gemeinsame Datei latex-vorspann.tex mit gesetztem Schalter.

\newif\ifkorrekturansicht
\korrekturansichttrue

\input{../tex-inputs/latex-vorspann}


               \section[Hermann Bahr an Arthur Schnitzler, 10. 11. 1903]{ Hermann Bahr an Arthur Schnitzler, 10. 11. 1903}\nopagebreak\mylabel{v}\rehead{ }\normalsize\beginnumbering\briefempfaengerindex{Schnitzler, Arthur@\textsc{Schnitzler, Arthur}!zzzBahr, Hermann@\emph{von Hermann Bahr}!1903-11-103@{10. 11. 1903}|(be} \toendnotes[C]{\smallbreak\pagebreak[2]} \Standort{CUL, Schnitzler, B 5b.}
\physDesc{Brief, 1 Blatt, 1 Seite
\newline{}Handschrift: schwarze Tinte, deutsche Kurrent\newline{}Ordnung: mit Bleistift von unbekannter Hand nummeriert: »102a« }\buchAbdrucke{\weitereDrucke{Hermann Bahr, Arthur Schnitzler: \emph{Briefwechsel, Aufzeichnungen, Dokumente (1891–1931)}. Hg. Kurt Ifkovits und Martin Anton Müller. Göttingen: \emph{Wallstein} 2018, S. 279.} }\toendnotes[C]{\smallbreak}\pstart
           \raggedleft{}{\pb}10. 11. 03\pend
           \pstart\center{}Lieber Arthur!\pend\pstart
           Kannſt Du mir, auf einer Correspondenz Karte, Auskunft geben, ob der Titel »\label{K_L01340_1v}\edtext{Primarius}{\lemma{\textnormal{\emph{Primarius}}}\Cendnote{\textnormal{Hintergrund der Anfrage \textcolor{blue}{Brahm}s bildet die Premierenvorbereitung von \emph{\textcolor{green}{Der Meister}}.}}}\label{K_L01340_1h}« in \textcolor{pink}{Süddeutſchland}{}\ledrightnote{\textcolor{pink}{Deutschland}}
               üblich iſt und wie jemand, der bei uns Primarius heißt, in \textcolor{pink}{Norddeutſchland}{}\ledrightnote{\textcolor{pink}{Deutschland}} genannt wird? \textcolor{blue}{Brahm}{}\ledrightnote{\textcolor{blue}{Otto Brahm}} weiß es nicht und gibt \substVorne{}\textsuperscript{vor}\substDazwischen{}an\substHinten{}, den Titel überhaupt nie gehört zu haben.\pend
           \pstart
           \textcolor{blue}{Brahm}{}\ledrightnote{\textcolor{blue}{Otto Brahm}} telegrafiert mir eben um die Änderungen,
               die ich in meinem \textcolor{green}{Stück}{}\ledrightnote{→\textcolor{green}{Der Meister}} noch
               machen will. Was geht da vor? Ich denke doch, daß Du zunächſt daran kommst. Es wäre
               mir wichtig, das Datum Deiner \label{K_L01340_2v}\edtext{\textcolor{green}{Première}{}\ledrightnote{→\textcolor{green}{Der einsame Weg. Schauspiel in fünf Akten}}}{\lemma{\textnormal{\emph{Première}}}\Cendnote{\textnormal{von \emph{\textcolor{green}{Der
                     einsame Weg}}}}}\label{K_L01340_2h} zu erfahren, ſo bald Du es weißt.\pend
           \pstart
           Verzeih die Haſt dieſer Zeilen{\\[\baselineskip]}Deinem abgehetzten{\\[\baselineskip]}\spacefill\mbox{Hermann}\pend
           \leftskip=0em{}\endnumbering\briefempfaengerindex{Schnitzler, Arthur@\textsc{Schnitzler, Arthur}!zzzBahr, Hermann@\emph{von Hermann Bahr}!1903-11-103@{10. 11. 1903}|)be}\mylabel{h}  \normalsize

\doendnotes{C}
\bigskip
\vfill

\clearpage

\footnotesize

\lohead{\textsc{register}}

% Definiere theindex-Environment komplett neu ohne reledmac
\makeatletter
\renewenvironment{theindex}{%
  \section*{\indexname}%
  \setlength{\parindent}{0pt}%
  \setlength{\parskip}{0pt plus 0.3pt}%
  \let\item\@idxitem
}{%
  \clearpage
}
\makeatother

\IfFileExists{\jobname-pw.ind}{\input{\jobname-pw.ind}}{}

\end{document}

      