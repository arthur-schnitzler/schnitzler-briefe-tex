%% latex-korrekturansicht-vorspann.tex
%% Vorspann für die Korrekturansicht.
%% Lädt die gemeinsame Datei latex-vorspann.tex mit gesetztem Schalter.

\newif\ifkorrekturansicht
\korrekturansichttrue

\input{../tex-inputs/latex-vorspann}


               \section[Arthur Schnitzler an Richard Beer-Hofmann, 16. 9. 1925]{ Arthur Schnitzler an Richard Beer-Hofmann, 16. 9. 1925}\nopagebreak\mylabel{v}\rehead{ }\normalsize\beginnumbering\briefempfaengerindex{Beer-Hofmann, Richard@\textsc{Beer-Hofmann, Richard}!zzzSchnitzler, Arthur@\emph{von Arthur Schnitzler}!1925-09-161@{16. 9. 1925}|(be} \toendnotes[C]{\smallbreak\pagebreak[2]} \Standort{YCGL, MSS 31.}
\physDesc{Bildpostkarte
\newline{}Handschrift: Bleistift, lateinische Kurrent\newline{}Versand: Stempel: »\nobreak{}\oindex{Santa Maria Novella@\textbf{Santa Maria Novella}, \emph{Bahnhofsgebäude (K.BHF)}|pwk}Firenze Ferrovia, 17–IX 1925, 1–2\nobreak{}«.  \newline{}Ordnung: 1) mit Bleistift von unbekannter Hand datiert: »16. 9.« 2) mit Bleistift von unbekannter Hand die Jahreszahl vermerkt:
                                    »25«}\toendnotes[C]{\smallbreak}\pstart{}{\pb}Dr Richard Beerhofma{\geminationn}\pend{}\pstart{}\textcolor{pink}{Wien XVIII}{}\ledrightnote{\textcolor{pink}{XVIII., Währing}}\pend{}\pstart{}\textcolor{pink}{Hasenauerstr 59}{}\ledrightnote{\textcolor{pink}{Hasenauerstraße}}.\pend{}\pstart{}\textcolor{pink}{Austria}{}\ledrightnote{\textcolor{pink}{Österreich}}\pend{}{\bigskip}\pstart
           \noindent{}\centering{}{\pb}\textcolor{gray}{\textbf{\textcolor{pink}{Firenze}{}\ledrightnote{\textcolor{pink}{Florenz}}}}\pend
           \pstart
           \noindent{}\centering{}\textcolor{gray}{\textbf{Panorama della \textcolor{pink}{Città}{}\ledrightnote{→\textcolor{pink}{Florenz}} visto da \textcolor{pink}{S. Miniato al Monte}{}\ledrightnote{\textcolor{pink}{San Miniato al Monte}}}}\pend
           \pstart
           16. 9. 25\pend
           \pstart
           Herzliche Grüße!\pend
           \pstart
           Ihr{\\[\baselineskip]}\spacefill\mbox{Arthur}\pend
           \leftskip=0em{}\endnumbering\briefempfaengerindex{Beer-Hofmann, Richard@\textsc{Beer-Hofmann, Richard}!zzzSchnitzler, Arthur@\emph{von Arthur Schnitzler}!1925-09-161@{16. 9. 1925}|)be}\mylabel{h}  \normalsize

\doendnotes{C}
\bigskip
\vfill

\clearpage

\footnotesize

\lohead{\textsc{register}}

% Definiere theindex-Environment komplett neu ohne reledmac
\makeatletter
\renewenvironment{theindex}{%
  \section*{\indexname}%
  \setlength{\parindent}{0pt}%
  \setlength{\parskip}{0pt plus 0.3pt}%
  \let\item\@idxitem
}{%
  \clearpage
}
\makeatother

\IfFileExists{\jobname-pw.ind}{\input{\jobname-pw.ind}}{}

\end{document}

      