%% latex-korrekturansicht-vorspann.tex
%% Vorspann für die Korrekturansicht.
%% Lädt die gemeinsame Datei latex-vorspann.tex mit gesetztem Schalter.

\newif\ifkorrekturansicht
\korrekturansichttrue

\input{../tex-inputs/latex-vorspann}


               \section[Arthur Schnitzler an Hugo von Hofmannsthal, {[}9. 3. 1894{]}]{ Arthur Schnitzler an Hugo von Hofmannsthal, {[}9. 3. 1894{]}}\nopagebreak\mylabel{v}\rehead{ }\normalsize\beginnumbering\briefempfaengerindex{Hofmannsthal, Hugo von@\textsc{Hofmannsthal, Hugo von}!zzzSchnitzler, Arthur@\emph{von Arthur Schnitzler}!1894-03-092@{{[}9. 3. 1894{]}}|(be} \toendnotes[C]{\smallbreak\pagebreak[2]} \Standort{FDH, Hs-30885,42.}
\physDesc{Brief, 1 Blatt (Briefpapier mit Trauerrand), 4 Seiten
\newline{}Handschrift: Bleistift, deutsche Kurrent\newline{}Ordnung: von Schnitzler mutmaßlich bei der Durchsicht der Briefe
                                    1929 mit Bleistift datiert: »93« }\buchAbdrucke{\weitereDrucke{Hugo von Hofmannsthal, Arthur Schnitzler: \emph{Briefwechsel}. Hg. Therese Nickl und Heinrich Schnitzler. Frankfurt am Main: \emph{S. Fischer} 1964, S. 51.} }\toendnotes[C]{\smallbreak}\pstart
           \raggedleft{}{\pb}\uline{Freitag.}\pend
           \pstart
           Liebſter Hugo, So{\geminationn}tag iſt
               nichts bei mir. Vielleicht ko{\geminationm}’ ich um 8,
                  ½ 9 zu \textsc{\textcolor{blue}{Karlweis}{}\ledrightnote{\textcolor{blue}{Carl Karlweis}}}; Sie auch? –\pend
           \pstart
           Bitte ſehr \label{K_L00305-2v}\edtext{ſchicken Sie doch an \textcolor{blue}{Goldmann}{}\ledrightnote{\textcolor{blue}{Paul Goldmann}}}{\lemma{\textnormal{\emph{ſchicken … Goldmann}}}\Cendnote{\textnormal{siehe Paul Goldmann an Arthur Schnitzler, 28. 2. [1894], der diesen
                     Brief motiviert haben dürfte; Vgl. A. S.: \emph{Tagebuch}, 5. 3. 1894}}}\label{K_L00305-2h}{ }\textsc{\textcolor{pink}{75 rue Richelieu}{}\ledrightnote{\textcolor{pink}{rue Richelieu}}} Ihre Sachen. Er ſchreibt mir ſo oft drum. »\textcolor{green}{Tizian}{}\ledrightnote{\textcolor{green}{Der Tod des Tizian}}« und »\textcolor{green}{Thor u Tod}{}\ledrightnote{\textcolor{green}{Der Thor und der Tod}}«
               wenigſtens.\pend
           \pstart
           {\pb}Von \textsc{\textcolor{blue}{Albert}{}\ledrightnote{\textcolor{blue}{Henri Albert}}} iſt in der \textcolor{brown}{\textsc{Nou\textcolor{gray}{v} Revue}}{}\ledrightnote{\textcolor{brown}{Nouvelle Revue}} eine \label{K_L00305_1v}\edtext{\textcolor{green}{Beſprechg}{}\ledrightnote{→\textcolor{green}{Le nouvel almanach de M. Bierbaum}}}{\lemma{\textnormal{\emph{Beſprechg}}}\Cendnote{\textnormal{Die Besprechung \emph{\textcolor{green}{Le nouvel almanach de M. Bierbaum}} erschien am
                     1. 3. 1894 im \emph{\textcolor{green}{Mercure de
                     France}} (S. 243–246).}}}\label{K_L00305_1h} des \textsc{\textcolor{green}{Musenalmanach}{}\ledrightnote{\textcolor{green}{Moderner Musen-Almanach auf das Jahr 1894}}s}, in dem Sie u ich mit
               ſehr viel Liebe behandelt ſind. (\label{K_L00305_2v}\edtext{\textcolor{green}{\textsc{Le génial Loris etc.}}{}\ledrightnote{→\textcolor{green}{Le nouvel almanach de M. Bierbaum}}}{\lemma{\textnormal{\emph{Le génial Loris etc.}}}\Cendnote{\textnormal{auf S. 245}}}\label{K_L00305_2h}). Vielleicht ſchreiben Sie dem Mann auch 2 Zeilen (\textsc{\textcolor{blue}{Henri Albert}{}\ledrightnote{\textcolor{blue}{Henri Albert}}, \textcolor{pink}{25 rue Jacob}{}\ledrightnote{\textcolor{pink}{rue Jacob}}.})\pend
           \pstart
           {\pb}– Bei dieſer Gelegenheit eri{\geminationn}er’ ich Sie an Ihre Verſprechung mir Ihre Gedichte zu überſenden.\pend
           \pstart
           – Haben Sie Nachricht von \textcolor{blue}{Richard}{}\ledrightnote{\textcolor{blue}{Richard Beer-Hofmann}}? Ich nur
               eine Correſp-Karte mit Adreſſe. –\pend
           \pstart
           Sind Sie vielleicht Samſtag{ }Abend im {\pb}\textsc{\textcolor{pink}{Central}{}\ledrightnote{\textcolor{pink}{Café Central}}}, ich meine, nach zehn? –\pend
           \pstart
           Wann gehn wir ins \textcolor{pink}{Arſenal}{}\ledrightnote{\textcolor{pink}{Arsenal}}? –\pend
           \pstart
           Und, überhaupt, wann ſehn wir uns wieder? Daſs uns nur \textsc{Trio}’s zuſa{\geminationm}enführen, iſt eigentlich komiſch.\pend
           \pstart
           Herzlich der Ihre{\\[\baselineskip]}\spacefill\mbox{Arthur.}\pend
           \leftskip=0em{}\endnumbering\briefempfaengerindex{Hofmannsthal, Hugo von@\textsc{Hofmannsthal, Hugo von}!zzzSchnitzler, Arthur@\emph{von Arthur Schnitzler}!1894-03-092@{{[}9. 3. 1894{]}}|)be}\mylabel{h}  \normalsize

\doendnotes{C}
\bigskip
\vfill

\clearpage

\footnotesize

\lohead{\textsc{register}}

% Definiere theindex-Environment komplett neu ohne reledmac
\makeatletter
\renewenvironment{theindex}{%
  \section*{\indexname}%
  \setlength{\parindent}{0pt}%
  \setlength{\parskip}{0pt plus 0.3pt}%
  \let\item\@idxitem
}{%
  \clearpage
}
\makeatother

\IfFileExists{\jobname-pw.ind}{\input{\jobname-pw.ind}}{}

\end{document}

      