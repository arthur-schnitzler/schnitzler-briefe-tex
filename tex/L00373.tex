%% latex-korrekturansicht-vorspann.tex
%% Vorspann für die Korrekturansicht.
%% Lädt die gemeinsame Datei latex-vorspann.tex mit gesetztem Schalter.

\newif\ifkorrekturansicht
\korrekturansichttrue

\input{../tex-inputs/latex-vorspann}


               \section[Richard Beer-Hofmann an Arthur Schnitzler, 2{[}6?{]}. 9. 1894]{ Richard Beer-Hofmann an Arthur Schnitzler, 2{[}6?{]}. 9. 1894}\nopagebreak\mylabel{v}\rehead{ }\normalsize\beginnumbering\briefempfaengerindex{Schnitzler, Arthur@\textsc{Schnitzler, Arthur}!zzzBeer-Hofmann, Richard@\emph{von Richard Beer-Hofmann}!1894-09-261@{2{[}6?{]}. 9. 189}|(be} \toendnotes[C]{\smallbreak\pagebreak[2]} \Standort{CUL, Schnitzler, B 8.}
\physDesc{Postkarte (mit aufgedrucktem Hotelwappen)
\newline{}Handschrift: Bleistift, lateinische Kurrent\newline{}Versand: Stempel: »\nobreak{}\oindex{Mailand@\textbf{Mailand}, \emph{Besiedelter Ort (A.BSO)}|pwk}Milano, 2\textcolor{gray}{×} 9-{[}94{]}\nobreak{}«.  
\newline{}Schnitzler: mit Bleistift nummeriert: »39« }\buchAbdrucke{\weitereDrucke{Arthur Schnitzler, Richard Beer-Hofmann: \emph{Briefwechsel 1891–1931}. Hg. Konstanze Fliedl. Wien, Zürich: \emph{Europaverlag} 1992, S. 60.} }\toendnotes[C]{\smallbreak}\pstart{}{\pb}\textcolor{gray}{\textbf{A}}n Herrn D\textsuperscript{r} Arthur
                  Schnitzler \pend{}\pstart{}\textcolor{pink}{Wien}{}\ledrightnote{\textcolor{pink}{Wien}}\pend{}\pstart{}\textcolor{pink}{IX Frankgasse 1}{}\ledrightnote{\textcolor{pink}{Frankgasse}}\pend{}\pstart{}\textcolor{pink}{Austria}{}\ledrightnote{\textcolor{pink}{Österreich}}\pend{}{\bigskip}\pstart
           \noindent{}{\pb}\label{K_L00373_1v}\edtext{Mau}{\lemma{\textnormal{\emph{Mau}}}\Cendnote{\textnormal{Die Datierung dieses Korrespondenzstücks auf einen bestimmten Tag ist
                  problematisch. Der Poststempel gibt den sicheren Hinweis »2«,
                  doch war \textcolor{blue}{Beer-Hofmann} Anfang des Monats noch
                  nicht auf seiner Reise. Nachdem \textcolor{blue}{Schnitzler} am
                     29. 9. 1894 das »Mau« aufnimmt, ist es auf die
                  Woche davor anzusetzen.}}}\label{K_L00373_1h}! hätt’ ich wenigstens gesagt wenn ich schon zum
               Schreiben zu faul bin.\pend
           \pstart
           Bitte senden (lassen Sie) Sie mir die »\textcolor{brown}{Zeit}{}\ledrightnote{\textcolor{brown}{Die Zeit. Wiener Wochenschrift}}« a \uline{posta ferma}{ }\textcolor{pink}{Florenz}{}\ledrightnote{\textcolor{pink}{Florenz}} wo ich bis incl. 3\textsuperscript{ten} bin. Vielleicht ist dort auch eine Karte von Ihnen an mich.\pend
           \pstart  Herzlichst Ihr\spacefill\mbox{Richard}\pend{}\endnumbering\briefempfaengerindex{Schnitzler, Arthur@\textsc{Schnitzler, Arthur}!zzzBeer-Hofmann, Richard@\emph{von Richard Beer-Hofmann}!1894-09-261@{2{[}6?{]}. 9. 189}|)be}\mylabel{h}  \normalsize

\doendnotes{C}
\bigskip
\vfill

\clearpage

\footnotesize

\lohead{\textsc{register}}

% Definiere theindex-Environment komplett neu ohne reledmac
\makeatletter
\renewenvironment{theindex}{%
  \section*{\indexname}%
  \setlength{\parindent}{0pt}%
  \setlength{\parskip}{0pt plus 0.3pt}%
  \let\item\@idxitem
}{%
  \clearpage
}
\makeatother

\IfFileExists{\jobname-pw.ind}{\input{\jobname-pw.ind}}{}

\end{document}

      