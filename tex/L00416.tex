%% latex-korrekturansicht-vorspann.tex
%% Vorspann für die Korrekturansicht.
%% Lädt die gemeinsame Datei latex-vorspann.tex mit gesetztem Schalter.

\newif\ifkorrekturansicht
\korrekturansichttrue

\input{../tex-inputs/latex-vorspann}


               \section[Richard Beer-Hofmann an Arthur Schnitzler, 28. 1. 1895]{ Richard Beer-Hofmann an Arthur Schnitzler,
               28. 1. 1895}\nopagebreak\mylabel{v}\rehead{ }\normalsize\beginnumbering\briefempfaengerindex{Schnitzler, Arthur@\textsc{Schnitzler, Arthur}!zzzBeer-Hofmann, Richard@\emph{von Richard Beer-Hofmann}!1895-01-281@{28. 1. 1895}|(be} \toendnotes[C]{\smallbreak\pagebreak[2]} \Standort{CUL, Schnitzler, B 8.}
\physDesc{Postkarte
\newline{}Handschrift: Bleistift, lateinische Kurrent\newline{}Versand: 1) Rohrpost 2) Stempel: »\nobreak{}\oindex{I., Innere Stadt@\textbf{I., Innere Stadt}, \emph{Bezirk (A.BZK)}|pwk}Wien 1/1, 28 I 95, 3 40 N\nobreak{}«. 3) Stempel: »\nobreak{}\oindex{IX., Alsergrund@\textbf{IX., Alsergrund}, \emph{Bezirk (A.BZK)}|pwk}Wien 9/2, 28 I 95, 4 10 N\nobreak{}«. 
\newline{}Schnitzler: mit Bleistift nummeriert: »53« }\buchAbdrucke{\weitereDrucke{Arthur Schnitzler, Richard Beer-Hofmann: \emph{Briefwechsel 1891–1931}. Hg. Konstanze Fliedl. Wien, Zürich: \emph{Europaverlag} 1992, S. 70.} }\toendnotes[C]{\smallbreak}\pstart{}{\pb}Herrn\pend{}\pstart{}D\textsuperscript{r} Arthur Schnitzler\pend{}\pstart{}\textcolor{pink}{IX Frankgasse}{}\ledrightnote{\textcolor{pink}{Frankgasse}} 1\pend{}{\bigskip}\pstart
           \noindent{}{\pb}Lieber Arthur! Wo haben Sie Ihren schwarzen So{\geminationm}erstrohhut gekauft? Morgen ist nämlich \label{K_L00416_1v}\edtext{\textcolor{pink}{Raimundtheater}{}\ledrightnote{\textcolor{pink}{Raimund-Theater}}abend}{\lemma{\textnormal{\emph{Raimundtheaterabend}}}\Cendnote{\textnormal{Am 29. 1. 1895
                  wurde im \textcolor{pink}{Raimund-Theater} das Volksstück \emph{\textcolor{green}{Bruder Martin}} von \textcolor{blue}{Karl Costa} mit der Musik von \textcolor{blue}{Max von
                     Weinzierl} gegeben.}}}\label{K_L00416_1h}. –\pend
           \pstart
           Ich gehe vielleicht, – fast sicher wenn Sie gehen. Herzlichst\pend
           \pstart \spacefill\mbox{Richard}\pend{}\endnumbering\briefempfaengerindex{Schnitzler, Arthur@\textsc{Schnitzler, Arthur}!zzzBeer-Hofmann, Richard@\emph{von Richard Beer-Hofmann}!1895-01-281@{28. 1. 1895}|)be}\mylabel{h}  \normalsize

\doendnotes{C}
\bigskip
\vfill

\clearpage

\footnotesize

\lohead{\textsc{register}}

% Definiere theindex-Environment komplett neu ohne reledmac
\makeatletter
\renewenvironment{theindex}{%
  \section*{\indexname}%
  \setlength{\parindent}{0pt}%
  \setlength{\parskip}{0pt plus 0.3pt}%
  \let\item\@idxitem
}{%
  \clearpage
}
\makeatother

\IfFileExists{\jobname-pw.ind}{\input{\jobname-pw.ind}}{}

\end{document}

      