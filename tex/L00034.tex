%% latex-korrekturansicht-vorspann.tex
%% Vorspann für die Korrekturansicht.
%% Lädt die gemeinsame Datei latex-vorspann.tex mit gesetztem Schalter.

\newif\ifkorrekturansicht
\korrekturansichttrue

\input{../tex-inputs/latex-vorspann}


               \section[Hugo von Hofmannsthal an Arthur Schnitzler, {[}16. 8. 1891{]}]{ Hugo von Hofmannsthal an Arthur Schnitzler,
                    {[}16. 8. 1891{]}}\nopagebreak\mylabel{v}\rehead{ }\normalsize\beginnumbering\briefempfaengerindex{Schnitzler, Arthur@\textsc{Schnitzler, Arthur}!zzzHofmannsthal, Hugo von@\emph{von Hugo von Hofmannsthal}!1891-08-161@{{[}16. 8. 1891{]}}|(be} \toendnotes[C]{\smallbreak\pagebreak[2]} \Standort{CUL, Schnitzler, B 43.}
\physDesc{Visitenkarte
\newline{}Handschrift: Bleistift, deutsche Kurrent
\newline{}Schnitzler: mit Bleistift auf der Namensseite datiert »16/8 91« \newline{}Ordnung: mit Bleistift von unbekannter Hand auf der Rückseite
                                    nummeriert: »6« }\buchAbdrucke{\weitereDrucke{Hugo von Hofmannsthal, Arthur Schnitzler: \emph{Briefwechsel}. Hg. Therese Nickl und Heinrich Schnitzler. Frankfurt am Main: \emph{S. Fischer} 1964, S. 12.} }\pstart{}{\pb}Liebſter
                        Freund!\pend\pstart
           Heute nacht vielleicht infolge ſchlechter Champignons ſehr unwohl kann heute kaum
                    ſtehen. Seien Sie und \textcolor{blue}{Richard}{}\ledrightnote{\textcolor{blue}{Richard Beer-Hofmann}} nicht bös
                    und behandeln Sie meine Unarten als Object der Analyſe.\pend
           \pstart
           {\pb}Herzlichst{\\[\baselineskip]}\spacefill\mbox{Loris.}\pend
           \leftskip=0em{}\pstart
           \noindent{}\centering{}\textcolor{gray}{\textbf{D\textsuperscript{r.}{ }\textcolor{blue}{Hugo von Hofmannsthal}{}\ledrightnote{\textcolor{blue}{Hugo August von Hofmannsthal}}}}\pend
           \endnumbering\briefempfaengerindex{Schnitzler, Arthur@\textsc{Schnitzler, Arthur}!zzzHofmannsthal, Hugo von@\emph{von Hugo von Hofmannsthal}!1891-08-161@{{[}16. 8. 1891{]}}|)be}\mylabel{h}  \normalsize

\doendnotes{C}
\bigskip
\vfill

\clearpage

\footnotesize

\lohead{\textsc{register}}

% Definiere theindex-Environment komplett neu ohne reledmac
\makeatletter
\renewenvironment{theindex}{%
  \section*{\indexname}%
  \setlength{\parindent}{0pt}%
  \setlength{\parskip}{0pt plus 0.3pt}%
  \let\item\@idxitem
}{%
  \clearpage
}
\makeatother

\IfFileExists{\jobname-pw.ind}{\input{\jobname-pw.ind}}{}

\end{document}

      