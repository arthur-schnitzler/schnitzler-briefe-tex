%% latex-korrekturansicht-vorspann.tex
%% Vorspann für die Korrekturansicht.
%% Lädt die gemeinsame Datei latex-vorspann.tex mit gesetztem Schalter.

\newif\ifkorrekturansicht
\korrekturansichttrue

\input{../tex-inputs/latex-vorspann}


               \section[Hugo von Hofmannsthal an Arthur Schnitzler, 14. 10. 1904]{ Hugo von Hofmannsthal an Arthur Schnitzler, 14. 10. 1904}\nopagebreak\mylabel{v}\rehead{ }\normalsize\beginnumbering\briefempfaengerindex{Schnitzler, Arthur@\textsc{Schnitzler, Arthur}!zzzHofmannsthal, Hugo von@\emph{von Hugo von Hofmannsthal}!1904-10-141@{14. 10. 1904}|(be} \toendnotes[C]{\smallbreak\pagebreak[2]} \Standort{CUL, Schnitzler, B 43.}
\physDesc{Brief, 1 Blatt, 3 Seiten
\newline{}Handschrift Gertrude von Hofmannsthal: schwarze Tinte, lateinische Kurrent
\newline{}Schnitzler: mit Bleistift beschriftet: »\textsc{Hugo}« \newline{}Ordnung: 1) mit Bleistift von unbekannter Hand nummeriert: »\strikeout{229}« 2) mit Bleistift von unbekannter Hand nummeriert:
                                    »238«}\buchAbdrucke{\weitereDrucke{Hugo von Hofmannsthal, Arthur Schnitzler: \emph{Briefwechsel}. Hg. Therese Nickl und Heinrich Schnitzler. Frankfurt am Main: \emph{S. Fischer} 1964, S. 204.} }\toendnotes[C]{\smallbreak}\pstart
           {\pb}(dictiert)\hfill \textcolor{pink}{Rodaun}{}\ledrightnote{\textcolor{pink}{Rodaun}}, d. 14. X. 1904.\pend
           \pstart
           Mein lieber Arthur, ich muss Sie bitten den inliegenden leider sehr
               unleserlichen \label{K_L01454_1v}\edtext{Brief}{\lemma{\textnormal{\emph{Brief}}}\Cendnote{\textnormal{Dieser befindet sich heute im Nachlass \textcolor{blue}{Hofmannsthal}s (Hs-30605,8). Der
                  Brief abgedruckt in: \emph{Louise Dumont. Eine Kulturgeschichte in Briefen und
                        Dokumenten.} Bd. 1: 1879–1904. Hg. Gertrude Cepl-Kaufmann, Michael
                     Matzigkeit Winrich Meiszies. Bearbeitet von Jasmin Grande, Nina Heidrich,
                     Karoline Riener. Essen: \emph{Klartext}{ }2013, S. 354–356.}}}\label{K_L01454_1h} der \textcolor{blue}{Dumont}{}\ledrightnote{\textcolor{blue}{Louise Dumont}} zu lesen und mir über diese Sache umgehend Ihren Rat zu
               geben. Es ist gewissermassen eine gemeinsame Angelegenheit. Die Unternehmung \textcolor{blue}{Dumont}{}\ledrightnote{\textcolor{blue}{Louise Dumont}}–\textcolor{blue}{Lindemann}{}\ledrightnote{\textcolor{blue}{Gustav Lindemann}} bewirbt sich um fast sämmtliche meiner dram. Arbeiten, was für
               mich immerhin nicht unwichtig. Nun war ich {\pb}durch \textcolor{blue}{S. Fischer}{}\ledrightnote{\textcolor{blue}{Samuel Fischer}} davon unterrichtet, dass sich die gleiche
               Unternehmung gegen Sie (\textcolor{green}{Einsamer Weg}{}\ledrightnote{\textcolor{green}{Der einsame Weg. Schauspiel in fünf Akten}}) uncorrect
               oder direct unanständig benommen habe. Ich that daher das Selbstverständliche d. h.
               ich verweigerte meine sämmtlichen Stücke »bis ich erfahren hätte, dass diese
               Angelegenheit zu Ihrer Befriedigung beigelegt sei«. Nun stellt der inliegende {\pb}Brief der \textcolor{blue}{Dumont}{}\ledrightnote{\textcolor{blue}{Louise Dumont}} die Sache ganz anders dar und ich bitte daher Sie mir
               mit zwei Worten zu sagen wo die Wahrheit liegt und ob vielleicht wirklich eine
               Ungeschicklichkeit \textcolor{blue}{Fischers}{}\ledrightnote{\textcolor{blue}{Samuel Fischer}} die Sache auf diesen
               bösen Punkt getrieben hat, in welchem Falle ich mich natürlich nicht verpflichtet
               hielte die Stücke zu verweigern.\pend
           \pstart
           Herzlich{\\[\baselineskip]}Ihr{\\[\baselineskip]}\spacefill\mbox{Hugo.}\pend
           \leftskip=0em{}\endnumbering\briefempfaengerindex{Schnitzler, Arthur@\textsc{Schnitzler, Arthur}!zzzHofmannsthal, Hugo von@\emph{von Hugo von Hofmannsthal}!1904-10-141@{14. 10. 1904}|)be}\mylabel{h}  \normalsize

\doendnotes{C}
\bigskip
\vfill

\clearpage

\footnotesize

\lohead{\textsc{register}}

% Definiere theindex-Environment komplett neu ohne reledmac
\makeatletter
\renewenvironment{theindex}{%
  \section*{\indexname}%
  \setlength{\parindent}{0pt}%
  \setlength{\parskip}{0pt plus 0.3pt}%
  \let\item\@idxitem
}{%
  \clearpage
}
\makeatother

\IfFileExists{\jobname-pw.ind}{\input{\jobname-pw.ind}}{}

\end{document}

      