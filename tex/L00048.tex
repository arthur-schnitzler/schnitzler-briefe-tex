%% latex-korrekturansicht-vorspann.tex
%% Vorspann für die Korrekturansicht.
%% Lädt die gemeinsame Datei latex-vorspann.tex mit gesetztem Schalter.

\newif\ifkorrekturansicht
\korrekturansichttrue

\input{../tex-inputs/latex-vorspann}


               \section[Arthur Schnitzler: Widmungsexemplar Das Märchen für Hugo August von Hofmannsthal, {[}5.{]} 12. 1891]{ Arthur Schnitzler: Widmungsexemplar Das Märchen für Hugo August von
               Hofmannsthal, {[}5.{]} 12. 1891}\nopagebreak\mylabel{v}\rehead{ }\normalsize\beginnumbering\briefempfaengerindex{Hofmannsthal, Hugo August von@\textsc{Hofmannsthal, Hugo August von}!zzzSchnitzler, Arthur@\emph{von Arthur Schnitzler}!1891-12-052@{{[}5.{]} 12. 1891}|(be} \toendnotes[C]{\smallbreak\pagebreak[2]} \Standort{FDH, FDH 3227.}
\physDesc{Widmung am Umschlag
\newline{}Handschrift: schwarze Tinte, deutsche Kurrent}\buchAbdrucke{\weitereDrucke{Hugo von Hofmannsthal: \emph{Bibliothek}. Hg. Ellen Ritter † in Zusammenarbeit mit Dalia Bukauskaité und
                        Konrad Heumann. Frankfurt am Main: \emph{S. Fischer} 2011, S. 604 (Sämtliche Werke. Kritische Ausgabe, XL).} }\toendnotes[C]{\smallbreak}\pstart
           \noindent{}\raggedleft{}{\pb}Herrn Dr. v. \textsc{Hofma{\geminationn}sthal}\pend
           \pstart
           \noindent{}\raggedleft{}verehrungsvoll\pend
           \pstart \spacefill\mbox{ArthSch.}\pend{}{\bigskip}\pstart
           \noindent{}\textcolor{gray}{\textbf{\uline{Manuſkript.}}}\pend
           \pstart
           \centering{}\textcolor{gray}{\textbf{\textcolor{green}{\so{Das Märchen}}{}\ledrightnote{\textcolor{green}{Das Märchen. Schauspiel in drei Aufzügen}}.}}\pend
           \pstart
           \noindent{}\centering{}\textcolor{gray}{\textbf{Schauſpiel in drei Aufzügen}}{\\}\textcolor{gray}{\textbf{von}}{\\}\textcolor{gray}{\textbf{Arthur Schnitzler.}}\pend
           {\bigskip}\pstart
           \noindent{}\centering{}\textcolor{gray}{\textbf{\textcolor{pink}{Wien}{}\ledrightnote{\textcolor{pink}{Wien}}{ }\label{K_L00048_1v}\edtext{1891}{\lemma{\textnormal{\emph{1891}}}\Cendnote{\textnormal{\textcolor{blue}{Schnitzler} bekam die Drucke des Stücks am
                           5. 12. 1891.
                        Zwei Tage später hatte es \textcolor{blue}{Hugo August
                           Hofmannsthal} gelesen.}}}\label{K_L00048_1h}.}}\pend
           \pstart
           \noindent{}\centering{}\textcolor{gray}{\textbf{Druck von \textcolor{brown}{Carl Steinhardt {\kaufmannsund} Cie.}{}\ledrightnote{\textcolor{brown}{Carl Steinhardt {\kaufmannsund} Co.}} (verantw. Leiter \textcolor{blue}{Guſtav Röttig}{}\ledrightnote{\textcolor{blue}{Gustav Röttig}}).}}\pend
           \endnumbering\briefempfaengerindex{Hofmannsthal, Hugo August von@\textsc{Hofmannsthal, Hugo August von}!zzzSchnitzler, Arthur@\emph{von Arthur Schnitzler}!1891-12-052@{{[}5.{]} 12. 1891}|)be}\mylabel{h}  \normalsize

\doendnotes{C}
\bigskip
\vfill

\clearpage

\footnotesize

\lohead{\textsc{register}}

% Definiere theindex-Environment komplett neu ohne reledmac
\makeatletter
\renewenvironment{theindex}{%
  \section*{\indexname}%
  \setlength{\parindent}{0pt}%
  \setlength{\parskip}{0pt plus 0.3pt}%
  \let\item\@idxitem
}{%
  \clearpage
}
\makeatother

\IfFileExists{\jobname-pw.ind}{\input{\jobname-pw.ind}}{}

\end{document}

      