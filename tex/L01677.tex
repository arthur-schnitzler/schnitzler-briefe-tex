%% latex-korrekturansicht-vorspann.tex
%% Vorspann für die Korrekturansicht.
%% Lädt die gemeinsame Datei latex-vorspann.tex mit gesetztem Schalter.

\newif\ifkorrekturansicht
\korrekturansichttrue

\input{../tex-inputs/latex-vorspann}


               \section[Arthur Schnitzler an Hermann Bahr, 20. 5. 1907]{ Arthur Schnitzler an Hermann Bahr, 20. 5. 1907}\nopagebreak\mylabel{v}\rehead{ }\normalsize\beginnumbering\briefempfaengerindex{Bahr, Hermann@\textsc{Bahr, Hermann}!zzzSchnitzler, Arthur@\emph{von Arthur Schnitzler}!1907-05-201@{20. 5. 1907}|(be} \toendnotes[C]{\smallbreak\pagebreak[2]} \Standort{TMW, HS AM 23385 Ba.}
\physDesc{Kartenbrief
\newline{}Handschrift: schwarze Tinte, deutsche Kurrent\newline{}Versand: 1) Stempel: »\nobreak{}Wien, 20.V{[}.07{]}, 7–8\nobreak{}«.  2) Stempel: »\nobreak{}Wien, 21. V. 07\nobreak{}«. \newline{}Ordnung: Lochung }\buchAbdrucke{\weitereDrucke{1) \emph{20. 5. 1907.} In: Arthur Schnitzler: \emph{The Letters of Arthur Schnitzler to Hermann Bahr}. Edited, annotated, and with an introduction, by Donald G.
                        Daviau. Chapel Hill: \emph{The University of North Carolina Press} 1978, S. 98 (University of North Carolina studies in the Germanic languages
                        and literatures, 89).} \weitereDrucke{2) Hermann Bahr, Arthur Schnitzler: \emph{Briefwechsel, Aufzeichnungen, Dokumente (1891–1931)}. Hg. Kurt Ifkovits und Martin Anton Müller. Göttingen: \emph{Wallstein} 2018, S. 393.} }\toendnotes[C]{\smallbreak}\pstart{}{\pb}Herrn \textsc{Hermann Bahr,}\pend{}\pstart{}\textsc{\textcolor{pink}{Wien Ob St Veit}{}\ledrightnote{\textcolor{pink}{Ober Sankt Veit}}}\pend{}\pstart{}\textsc{\textcolor{pink}{Veitlissengasse}{}\ledrightnote{\textcolor{pink}{Veitlissengasse}}.}\pend{}{\bigskip}\pstart
           \raggedleft{}{\pb}20/5 907\pend
           \pstart{}lieber Hermann, \pend\pstart
           gar nichts wichtiges. Wollte dich nur wieder einmal ſehn. Schreib mir, wann du
               wieder aus deiner Welt emportauchſt. Vielleicht fahren wir Ende \substVorne{}\textsuperscript{nächſter}{\allowbreak}\substDazwischen{}der\substHinten{} Woche auf ein paar Tage in die \textcolor{pink}{Brühl}{}\ledrightnote{\textcolor{pink}{Brühl}}. Du
               haſt hoffentlich deine \label{K_L01677_1v}\edtext{Meeresvilla}{\lemma{\textnormal{\emph{Meeresvilla}}}\Cendnote{\textnormal{Den Sommer verbrachten \textcolor{blue}{Bahr} und \textcolor{blue}{Mildenburg}
                  jedoch in einem Hotel am \textcolor{pink}{Lido}.}}}\label{K_L01677_1h} gefunden.
                  \textcolor{green}{Brehm}{}\ledrightnote{\textcolor{green}{Brehms Tierleben}} behalte natürlich ſo lang du willst.\pend
           \pstart
           Von Herzen dein{\\[\baselineskip]}\spacefill\mbox{Arthur.}\pend
           \leftskip=0em{}\endnumbering\briefempfaengerindex{Bahr, Hermann@\textsc{Bahr, Hermann}!zzzSchnitzler, Arthur@\emph{von Arthur Schnitzler}!1907-05-201@{20. 5. 1907}|)be}\mylabel{h}  \normalsize

\doendnotes{C}
\bigskip
\vfill

\clearpage

\footnotesize

\lohead{\textsc{register}}

% Definiere theindex-Environment komplett neu ohne reledmac
\makeatletter
\renewenvironment{theindex}{%
  \section*{\indexname}%
  \setlength{\parindent}{0pt}%
  \setlength{\parskip}{0pt plus 0.3pt}%
  \let\item\@idxitem
}{%
  \clearpage
}
\makeatother

\IfFileExists{\jobname-pw.ind}{\input{\jobname-pw.ind}}{}

\end{document}

      