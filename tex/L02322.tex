%% latex-korrekturansicht-vorspann.tex
%% Vorspann für die Korrekturansicht.
%% Lädt die gemeinsame Datei latex-vorspann.tex mit gesetztem Schalter.

\newif\ifkorrekturansicht
\korrekturansichttrue

\input{../tex-inputs/latex-vorspann}


               \section[Arthur Schnitzler an Georg Engländer, 3. 3. 1919]{ Arthur Schnitzler an Georg Engländer, 3. 3. 1919}\nopagebreak\mylabel{v}\rehead{ }\normalsize\beginnumbering\briefempfaengerindex{Englaender, Georg@\textsc{Engländer, Georg}!zzzSchnitzler, Arthur@\emph{von Arthur Schnitzler}!1919-03-031@{3. 3. 1919}|(be} \toendnotes[C]{\smallbreak\pagebreak[2]} \Standort{Wien, Österreichische Nationalbibliothek, 228/B8/1-3 LIT MAG.}
\physDesc{Briefkarte, Umschlag
\newline{}Schreibmaschine
\newline{}Handschrift: schwarze Tinte, deutsche Kurrent (\noindent{}Ergänzung und Unterschrift)\newline{}Versand: Stempel: »\nobreak{}\oindex{IX., Alsergrund@\textbf{IX., Alsergrund}, \emph{Bezirk (A.BZK)}|pwk}9/1 Wien 66, 4. III. 19, 5\nobreak{}«.  }\toendnotes[C]{\smallbreak}\pstart{}{\pb}\textcolor{gray}{\textbf{D\textsuperscript{R} ARTHUR SCHNITZLER}}\pend{}\pstart{}\textcolor{gray}{\textbf{\textcolor{pink}{WIEN, XVIII. STERNWARTESTRASSE 71}{}\ledrightnote{\textcolor{pink}{Sternwartestraße}}.}}\pend{}{\bigskip}\pstart{}{\pb}Herrn Georg Engländer\pend{}\pstart{}\textcolor{pink}{Wien IX.}{}\ledrightnote{\textcolor{pink}{IX., Alsergrund}}\pend{}\pstart{}\textcolor{pink}{Nussdorferstrasse 10}{}\ledrightnote{\textcolor{pink}{Nussdorfer Straße}}.\pend{}{\bigskip}\pstart
           \noindent{}{\pb}\textcolor{gray}{\textbf{D\textsuperscript{R} ARTHUR SCHNITZLER}}\hfill 3. 3. 1919.\pend
           \pstart
           \textcolor{gray}{\textbf{\textcolor{pink}{WIEN, XVIII. STERNWARTESTRASSE 71}{}\ledrightnote{\textcolor{pink}{Sternwartestraße}}.}}\pend
           \pstart\center{}Sehr verehrter Herr Engländer.\pend\pstart
           Vielen Dank für Ihr freundliches Schreiben. Zu meinem grössten Bedauern kann ich dem
                  \label{K_L02322_1v}\edtext{Vortragsabend}{\lemma{\textnormal{\emph{Vortragsabend}}}\Cendnote{\textnormal{der »\textcolor{blue}{Altenberg}-Abend« am 5. 3. 1919 im \textcolor{pink}{Kleinen Konzerthaussaal}}}}\label{K_L02322_1h} nicht beiwohnen, da ich für
               den Mittwoch{ }Abend schon vor längerer Zeit eine andere \introOben{}unverſchiebbare\introOben{} Verpflichtung übernommen habe\introOben{};\introOben{}
               und zwar die einer Vorlesung in privatem Kreise beizuwohnen.\pend
           \pstart
           {\pb}Mit bestem Danke retourniere ich den freundlichst
               an mich gesandten Sitz (es war nur einer, nicht wie in Ihrem Brief vermerkt steht,
               zwei).\pend
           \pstart
           Mit verbindlichen Grüssen{\\[\baselineskip]}Ihr sehr ergebener{\\[\baselineskip]}\spacefill\mbox{{[}hs.:{]} Arthur Schnitzler}\pend
           \leftskip=0em{}\endnumbering\briefempfaengerindex{Englaender, Georg@\textsc{Engländer, Georg}!zzzSchnitzler, Arthur@\emph{von Arthur Schnitzler}!1919-03-031@{3. 3. 1919}|)be}\mylabel{h}  \normalsize

\doendnotes{C}
\bigskip
\vfill

\clearpage

\footnotesize

\lohead{\textsc{register}}

% Definiere theindex-Environment komplett neu ohne reledmac
\makeatletter
\renewenvironment{theindex}{%
  \section*{\indexname}%
  \setlength{\parindent}{0pt}%
  \setlength{\parskip}{0pt plus 0.3pt}%
  \let\item\@idxitem
}{%
  \clearpage
}
\makeatother

\IfFileExists{\jobname-pw.ind}{\input{\jobname-pw.ind}}{}

\end{document}

      