%% latex-korrekturansicht-vorspann.tex
%% Vorspann für die Korrekturansicht.
%% Lädt die gemeinsame Datei latex-vorspann.tex mit gesetztem Schalter.

\newif\ifkorrekturansicht
\korrekturansichttrue

\input{../tex-inputs/latex-vorspann}


               \section[Hugo von Hofmannsthal an Arthur Schnitzler, 1. 4. {[}1909{]}]{ Hugo von Hofmannsthal an Arthur Schnitzler, 1. 4. {[}1909{]}}\nopagebreak\mylabel{v}\rehead{ }\normalsize\beginnumbering\briefempfaengerindex{Schnitzler, Arthur@\textsc{Schnitzler, Arthur}!zzzHofmannsthal, Hugo von@\emph{von Hugo von Hofmannsthal}!1909-04-011@{1. 4. {[}1909{]}}|(be} \toendnotes[C]{\smallbreak\pagebreak[2]} \Standort{CUL, Schnitzler, B 43.}
\physDesc{Brief, 2 Blätter, 6 Seiten
\newline{}Handschrift: schwarze Tinte, deutsche Kurrent
\newline{}Schnitzler: mit Bleistift die Jahreszahl ergänzt: »09« und beschriftet: »Hofmannsthal« \newline{}Ordnung: 1) mit Bleistift von unbekannter Hand nummeriert: »\strikeout{300}« 2) mit Bleistift von unbekannter Hand nummeriert:
                                    »304«}\buchAbdrucke{\weitereDrucke{Hugo von Hofmannsthal, Arthur Schnitzler: \emph{Briefwechsel}. Hg. Therese Nickl und Heinrich Schnitzler. Frankfurt am Main: \emph{S. Fischer} 1964, S. 244.} }\toendnotes[C]{\smallbreak}\pstart
           \raggedleft{}{\pb}1 IV.{ }\textcolor{pink}{Rodaun}{}\ledrightnote{\textcolor{pink}{Rodaun}}\pend
           \pstart{}mein lieber Arthur\pend\pstart
           ich danke Ihnen ſehr für Ihre guten Worte über \textcolor{green}{Elektra}{}\ledrightnote{\textcolor{green}{Elektra. Tragödie in einem Aufzug}}. Dies iſt die reinſte Freude, von einem Menſchen, den man ſo gern
               hat. Ich habe Ihre Arbeiten immer gern gehabt, aber erſt in den letzten 4–5 Jahren
               iſt mir eigentlich der Knopf für ihren ganzen Wert aufgegangen {\pb}und ſeitdem habe ich mir
               angewöhnt, ſie mit ſo großer Freude wiederholt zu leſen.\pend
           \pstart
           Es iſt mir ſehr hart, Sie ſo gar ſelten zu ſehen. Nie habe ich eine Stunde mit Ihnen
               verbracht, die nicht von einem ganz beſti{\geminationm}ten poſitiven
               Wohlgefühl, mehr noch des Gemütes als des Geiſtes begleitet geweſen wäre.\hspace*{1.5em}Ich denke daran, {\pb}wenn Sie Ende Mai
               nach \textcolor{pink}{Tirol}{}\ledrightnote{\textcolor{pink}{Tirol}} fahren, um Wohnung zu ſuchen,
               mitzufahren, auch ohne dieſen Zweck. – Es iſt nun bald zwanzig Jahre, daſs wir uns
               kennen.\pend
           \pstart
           \numberlinefalse{}\centering{}–\numberlinetrue{}\pend
           \pstart
           \noindent{}Die \textcolor{green}{Gedichte}{}\ledrightnote{→\textcolor{green}{[Gedichte]}} von \textcolor{blue}{Winterſtein}{}\ledrightnote{\textcolor{blue}{Alfred von Winterstein}} haben mir zum Teil ſehr gut gefallen. Ohne allen
               Zweifel habe ich ſie damals (vor Monaten) an Sie zurückgeſchickt, denn ich bin in
               dieſem Punkt ſehr {\pb}genau und an
               dem einzigen Platz, wo ſie liegen könnten, liegen ſie nicht mehr. – Es ſchien mir
               eine feine, aber ſchwache Perſönlichkeit ſich zu äußern. –\pend
           \pstart
           Betreffs \textcolor{green}{Elektra}{}\ledrightnote{\textcolor{green}{Elektra. Tragödie in einem Aufzug}}, ſo habe ich \textcolor{blue}{Fiſcher}{}\ledrightnote{\textcolor{blue}{Samuel Fischer}} nicht ohne Mühe veranlaßt, ſeine \uline{Verlagsrechte} an \textcolor{blue}{Fürſtner}{}\ledrightnote{\textcolor{blue}{Otto Fürstner}}
               abzutreten. Hiefür bezahle ich an \textcolor{blue}{Fiſcher}{}\ledrightnote{\textcolor{blue}{Samuel Fischer}} die
               Hälfte der von \textcolor{blue}{Fürſtner}{}\ledrightnote{\textcolor{blue}{Otto Fürstner}} mir zufließenden 25{\%}. D. h. von 10000 Exemplaren bekomme ich 1250\strikeout{0} Mark, \textcolor{blue}{Fiſcher}{}\ledrightnote{\textcolor{blue}{Samuel Fischer}} das
               gleiche.\pend
           \pstart
           Ihr{\\[\baselineskip]}\spacefill\mbox{Hugo}\pend
           \leftskip=0em{}\pstart
           \noindent{}{\pb}\textsc{P. S.}\label{K_L01837_1v}\edtext{In 14 Tagen}{\lemma{\textnormal{\emph{In 14 Tagen}}}\Cendnote{\textnormal{vgl. A. S.: \emph{Tagebuch}, 16. 10. 1909}}}\label{K_L01837_1h} ſpielt die \textcolor{blue}{\textsc{Després}}{}\ledrightnote{\textcolor{blue}{Suzanne Desprès}} hier die \textcolor{green}{\textsc{Elektra}}{}\ledrightnote{\textcolor{green}{Elektra (op. 58)}}. Referent über ſolche \label{K_L01837_2v}\edtext{Dinge}{\lemma{\textnormal{\emph{Dinge}}}\Cendnote{\textnormal{Die Besprechung des
                     Gastspiels erschien nicht gezeichnet und äußert sich nicht explizit zu \emph{\textcolor{green}{Elektra}}, nennt aber den Auftritt von \textcolor{blue}{Desprès} im Stück das »künstlerische
                        Ereignis des Abends« (\emph{\textcolor{green}{Gastspiel der Suzanne Després}}. In: \emph{\textcolor{green}{Neue Freie Presse}}, Nr. 16040,
                           17. 4. 1909, S. 12).}}}\label{K_L01837_2h} iſt \textcolor{blue}{Auernheimer}{}\ledrightnote{\textcolor{blue}{Raoul Auernheimer}}. Nun iſt das ein anſtändiger und nicht
                  übelwollender Menſch und ich wäre wahrhaftig froh nicht durch eine unangenehme
                  Haltung ſeinerſeits wiederum auch gegen dieſe Figur in die gewiſſe defenſive {\pb}Haltung gerathen zu müſſen. Ich
                  glaube daſs ein Geſpräch von 10 Minuten mit Ihnen hinreichen würde, ihm verſtehen
                  zu machen worin die Qualität des \textcolor{green}{Stückes}{}\ledrightnote{→\textcolor{green}{Elektra. Tragödie in einem Aufzug}} liegt, – glaube aber auch daſs er ohne dieſes Geſpräch \uline{nicht} auf dem \textsc{niveau} iſt,
                  ſich zu dem Stück in ein loyales Verhältnis zu ſetzen, beſonders in ſeine
                  Atmoſphäre. Vielleicht finden Sie die Gelegenheit. – \pend
           \endnumbering\briefempfaengerindex{Schnitzler, Arthur@\textsc{Schnitzler, Arthur}!zzzHofmannsthal, Hugo von@\emph{von Hugo von Hofmannsthal}!1909-04-011@{1. 4. {[}1909{]}}|)be}\mylabel{h}  \normalsize

\doendnotes{C}
\bigskip
\vfill

\clearpage

\footnotesize

\lohead{\textsc{register}}

% Definiere theindex-Environment komplett neu ohne reledmac
\makeatletter
\renewenvironment{theindex}{%
  \section*{\indexname}%
  \setlength{\parindent}{0pt}%
  \setlength{\parskip}{0pt plus 0.3pt}%
  \let\item\@idxitem
}{%
  \clearpage
}
\makeatother

\IfFileExists{\jobname-pw.ind}{\input{\jobname-pw.ind}}{}

\end{document}

      