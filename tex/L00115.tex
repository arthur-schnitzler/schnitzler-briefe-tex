%% latex-korrekturansicht-vorspann.tex
%% Vorspann für die Korrekturansicht.
%% Lädt die gemeinsame Datei latex-vorspann.tex mit gesetztem Schalter.

\newif\ifkorrekturansicht
\korrekturansichttrue

\input{../tex-inputs/latex-vorspann}


               \section[Richard Beer-Hofmann an Arthur Schnitzler, 19. 8. 1892]{ Richard Beer-Hofmann an Arthur Schnitzler, 19. 8. 1892}\nopagebreak\mylabel{v}\rehead{ }\normalsize\beginnumbering\briefempfaengerindex{Schnitzler, Arthur@\textsc{Schnitzler, Arthur}!zzzBeer-Hofmann, Richard@\emph{von Richard Beer-Hofmann}!1892-08-191@{19. 8. 1892}|(be} \toendnotes[C]{\smallbreak\pagebreak[2]} \Standort{CUL, Schnitzler, B 8.}
\physDesc{Brief, 2 Blätter, 8 Seiten
\newline{}Handschrift: Bleistift, lateinische Kurrent
\newline{}Schnitzler: mit Bleistift datiert: »19/8 92« und nummeriert: »9.« }\buchAbdrucke{\weitereDrucke{Arthur Schnitzler, Richard Beer-Hofmann: \emph{Briefwechsel 1891–1931}. Hg. Konstanze Fliedl. Wien, Zürich: \emph{Europaverlag} 1992, S. 36–37.} }\toendnotes[C]{\smallbreak}\pstart
           \noindent{}{\pb}Lieber Arthur! Sie wissen ja, wie schreibfaul ich bin, und wie sehr
               ich mir immer Zeit lasse.\pend
           \pstart
           Also vor Allem: Ich freue mich sehr, \uline{sehr} sie auf ein
               paar Tage hier zu haben; mit Ihnen {\pb}werde ich freilich kaum gehen können; im Allgemeinen habe ich einen verdorbenen
                  So{\geminationm}er, schlechte Laune in xter Potenz, die erst jetzt
               etwas, nachlässt; gearbeitet {\pb}hab
               ich circa \uuline{15} (!) Druckzeilen – also – nichts. Ausser
               ein paar Gedanken, deren Wert äußerst p\substVorne{}\textsuperscript{o}\substDazwischen{}ro\substHinten{}blematisch ist, also ein verlorener So{\geminationm}er. In
               den nächsten {\pb}Tagen werde ich
               voraussichtlich meine \textcolor{green}{Pantomime}{}\ledrightnote{→\textcolor{green}{Pierrot Hypnotiseur}} an Sie senden, und Sie bitten \strikeout{Sie},
               dieselbe durch Ihren \textcolor{blue}{Abschreiber}{}\ledrightnote{→\textcolor{blue}{?? [Schreibkraft für Arthur Schnitzler]}} copiren zu lassen, da ich sie möglicherweise in der nächsten
               Zeit an irgend einen \label{K_L00115_1v}\edtext{Verleger}{\lemma{\textnormal{\emph{Verleger}}}\Cendnote{\textnormal{\emph{\textcolor{green}{Pierrot hypnotiseur}}, Pantomine von \textcolor{blue}{Richard Beer-Hofmann}, blieb zu Lebzeiten
                  ungedruckt.}}}\label{K_L00115_1h}{\pb}\strikeout{u} schicken werde.\pend
           \pstart
           Ihr »\textcolor{green}{Märchen}{}\ledrightnote{\textcolor{green}{Das Märchen. Schauspiel in drei Aufzügen}}« und Ihre »\textcolor{green}{Episode}{}\ledrightnote{\textcolor{green}{Episode}}« habe ich bereits mehrfach verborgt; könnten Sie mir noch
               vor Ihrer Ankunft – denn die sich dafür Interessirenden reisen bald ab – \pend
           \pstart
           {\pb}»\textcolor{green}{Anatols Hochzeitsmorgen}{}\ledrightnote{\textcolor{green}{Anatols Hochzeitsmorgen}}«\pend
           \pstart
           »\textcolor{green}{Abschiedsouper}{}\ledrightnote{\textcolor{green}{Abschiedssouper}}«\pend
           \pstart
           »\textcolor{green}{Frage an das Schicksal}{}\ledrightnote{\textcolor{green}{Die Frage an das Schicksal}}«\pend
           \pstart
           senden?\pend
           \pstart
           Frau \textcolor{blue}{Flegmann}{}\ledrightnote{\textcolor{blue}{Bertha Flegmann}}, die wie Sie wissen ein klein
               wenig litterarischen Salon treibt interessirt sich dafür; {\pb}ich würde die Sachen fall\substVorne{}\textsuperscript{ls}\substDazwischen{}s\substHinten{} es nur \uline{Abschriften} sind nicht verborgen,
               sondern vorlesen. »\label{K_L00115-2v}\edtext{\textcolor{green}{\uline{Das} Gedicht}{}\ledrightnote{→\textcolor{green}{Anfang vom Ende}}}{\lemma{\textnormal{\emph{Das Gedicht}}}\Cendnote{\textnormal{\textcolor{blue}{Arthur Schnitzler}: \emph{\textcolor{green}{Anfang vom Ende}}. In: \emph{\textcolor{green}{Deutsche Dichtung}}, Bd. 12, Nr. 8, 15. 7. 1892,
                     S. 192.}}}\label{K_L00115-2h}« ist wie ich vom Kleinen \textcolor{blue}{Kraus}{}\ledrightnote{\textcolor{blue}{Karl Kraus}} (vide \textcolor{blue}{Salten}{}\ledrightnote{\textcolor{blue}{Felix Salten}}) höre
               in der »\textcolor{green}{Deutschen Dichtung}{}\ledrightnote{\textcolor{green}{Deutsche Dichtung}}« erschienen. \textcolor{blue}{Loris}{}\ledrightnote{\textcolor{blue}{Hugo von Hofmannsthal}}, der {\pb}wie es scheint gesellschaftlich
               zerrissen wird ist öfters hier, bei mir.\pend
           \pstart
           Bitte schreiben Sie mir wieder ein paar Zeilen, – und vor allem annonciren Sie Ihr
                  Ko{\geminationm}en. Bitte was macht \textcolor{blue}{Schwarzkopf}{}\ledrightnote{\textcolor{blue}{Gustav Schwarzkopf}}, ich hörte traurige Nachrichten? Herzlichst Ihr\pend
           \pstart \spacefill\mbox{Richard}\pend{}\pstart
           \textcolor{pink}{Ischl}{}\ledrightnote{\textcolor{pink}{Bad Ischl}}{ }19 Aug. 92\pend
           \endnumbering\briefempfaengerindex{Schnitzler, Arthur@\textsc{Schnitzler, Arthur}!zzzBeer-Hofmann, Richard@\emph{von Richard Beer-Hofmann}!1892-08-191@{19. 8. 1892}|)be}\mylabel{h}  \normalsize

\doendnotes{C}
\bigskip
\vfill

\clearpage

\footnotesize

\lohead{\textsc{register}}

% Definiere theindex-Environment komplett neu ohne reledmac
\makeatletter
\renewenvironment{theindex}{%
  \section*{\indexname}%
  \setlength{\parindent}{0pt}%
  \setlength{\parskip}{0pt plus 0.3pt}%
  \let\item\@idxitem
}{%
  \clearpage
}
\makeatother

\IfFileExists{\jobname-pw.ind}{\input{\jobname-pw.ind}}{}

\end{document}

      