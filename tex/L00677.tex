%% latex-korrekturansicht-vorspann.tex
%% Vorspann für die Korrekturansicht.
%% Lädt die gemeinsame Datei latex-vorspann.tex mit gesetztem Schalter.

\newif\ifkorrekturansicht
\korrekturansichttrue

\input{../tex-inputs/latex-vorspann}


               \section[Hugo von Hofmannsthal an Arthur Schnitzler, 17. 5. {[}1897{]}]{ Hugo von Hofmannsthal an Arthur Schnitzler, 17. 5. {[}1897{]}}\nopagebreak\mylabel{v}\rehead{ }\normalsize\beginnumbering\briefempfaengerindex{Schnitzler, Arthur@\textsc{Schnitzler, Arthur}!zzzHofmannsthal, Hugo von@\emph{von Hugo von Hofmannsthal}!1897-05-171@{17. 5. {[}1897{]}}|(be} \toendnotes[C]{\smallbreak\pagebreak[2]} \Standort{CUL, Schnitzler, B 43.}
\physDesc{Brief, 1 Blatt, 4 Seiten
\newline{}Handschrift: schwarze Tinte, deutsche Kurrent
\newline{}Schnitzler: mit Bleistift die Jahreszahl ergänzt: »97« \newline{}Ordnung: mit Bleistift von unbekannter Hand nummeriert:
                                        »90a« }\buchAbdrucke{\weitereDrucke{Hugo von Hofmannsthal, Arthur Schnitzler: \emph{Briefwechsel}. Hg. Therese Nickl und Heinrich Schnitzler. Frankfurt am Main: \emph{S. Fischer} 1964, S. 86.} }\pstart
           \raggedleft{}{\pb}\textcolor{pink}{Wien}{}\ledrightnote{\textcolor{pink}{Wien}}{ }17\textsuperscript{ten} Mai.\pend
           \pstart{}Mein lieber Arthur\pend\pstart
           ich höre mit großer Freude von verſchiedenen, daſs es Ihnen ſehr gut \strikeout{G} geht und hoffe, dieſer Brief trifft Sie noch
                    vor der Abreiſe nach \textcolor{pink}{London}{}\ledrightnote{\textcolor{pink}{London}}. Mir ginge es auch
                    recht gut (beſſer als lange) wenn nicht dieſes unglaubliche Wetter wäre. Man muß
                    das Wetter erwähnen, es iſt {\pb}zu wichtig. Seit den erſten Tagen Mai iſt ein finſterer Himmel
                    wie im Februar, ſtundenlange Regengüſſe, 3–5 Grad, manchmal in einer Woche kein
                    Stück blauer Himmel. Und da ſchon vorher ein paar ſehr ſchöne Tage waren, ſo
                    ſehnt man ſich umſomehr, wie nach einem unterbrochenen Traum. Ich war die ganze
                    Zeit faſt nur zuhaus und habe meine Grammatika gelernt {\pb}und alte Texte geleſen. Ich
                    freue mich mehr als ich ſagen kann, darauf wieder aufs Land zu können, das
                    drängt alles andere zurück.\pend
           \pstart
           Vom Sommer weiß ich noch nicht viel beſtimmtes. Jedenfalls bin ich bis zum
                            20\textsuperscript{ten} Juni in \textcolor{pink}{Wien}{}\ledrightnote{\textcolor{pink}{Wien}}. Einen Abend, dann noch einen und einen kalten
                    unfreundlichen Tag am Land (\textcolor{pink}{Dornbach}{}\ledrightnote{\textcolor{pink}{Dornbach}}, \textcolor{pink}{Neuwaldegg}{}\ledrightnote{\textcolor{pink}{Neuwaldegg}}) hab ich mit \textcolor{blue}{Brahm}{}\ledrightnote{\textcolor{blue}{Otto Brahm}} verbracht, jedesmal {\pb}nur mit ihm und \textcolor{blue}{Hirſchfeld}{}\ledrightnote{\textcolor{blue}{Georg Hirschfeld}}. \textcolor{blue}{Brahm}{}\ledrightnote{\textcolor{blue}{Otto Brahm}} iſt ein überaus guter und angenehmer Menſch; es muſs von ſolchen
                    Menſchen wohl gar nicht ſo wenige geben und wir ſind manchmal zu ſehr geneigt,
                    diejenigen, die wir zufällig nicht kennen, abzuleugnen. Wir ſind überhaupt ſehr
                    vorlaut. Wir haben aber vielleicht doch ein bischen Talent.\pend
           \pstart
           Leben Sie weiter wohl und erfreuen uns bald durch merkwürdige Erzählungen.\pend
           \pstart Ihr\spacefill\mbox{Hugo.}\pend{}\endnumbering\briefempfaengerindex{Schnitzler, Arthur@\textsc{Schnitzler, Arthur}!zzzHofmannsthal, Hugo von@\emph{von Hugo von Hofmannsthal}!1897-05-171@{17. 5. {[}1897{]}}|)be}\mylabel{h}  \normalsize

\doendnotes{C}
\bigskip
\vfill

\clearpage

\footnotesize

\lohead{\textsc{register}}

% Definiere theindex-Environment komplett neu ohne reledmac
\makeatletter
\renewenvironment{theindex}{%
  \section*{\indexname}%
  \setlength{\parindent}{0pt}%
  \setlength{\parskip}{0pt plus 0.3pt}%
  \let\item\@idxitem
}{%
  \clearpage
}
\makeatother

\IfFileExists{\jobname-pw.ind}{\input{\jobname-pw.ind}}{}

\end{document}

      