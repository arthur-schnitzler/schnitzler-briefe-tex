%% latex-korrekturansicht-vorspann.tex
%% Vorspann für die Korrekturansicht.
%% Lädt die gemeinsame Datei latex-vorspann.tex mit gesetztem Schalter.

\newif\ifkorrekturansicht
\korrekturansichttrue

\input{../tex-inputs/latex-vorspann}


               \section[Hugo von Hofmannsthal an Arthur Schnitzler, 28. 12. 1906]{ Hugo von Hofmannsthal an Arthur Schnitzler, 28. 12. 1906}\nopagebreak\mylabel{v}\rehead{ }\normalsize\beginnumbering\briefempfaengerindex{Schnitzler, Arthur@\textsc{Schnitzler, Arthur}!zzzHofmannsthal, Hugo von@\emph{von Hugo von Hofmannsthal}!1906-12-281@{28. 12. 1906}|(be} \toendnotes[C]{\smallbreak\pagebreak[2]} \Standort{CUL, Schnitzler, B 43.}
\physDesc{Postkarte
\newline{}Handschrift: schwarze Tinte, deutsche Kurrent\newline{}Versand: 1) Rohrpost 2) Stempel: »\nobreak{}\oindex{I., Innere Stadt@\textbf{I., Innere Stadt}, \emph{Bezirk (A.BZK)}|pwk}1/1 Wien, 29 XII 06, 10 20V\nobreak{}«. 3) Stempel: »\nobreak{}\oindex{XVIII., Waehring@\textbf{XVIII., Währing}, \emph{Bezirk (A.BZK)}|pwk}18/1 Wien 110, 229 XII 06, 11–V\nobreak{}«. 4) Stempel: »\nobreak{}\oindex{XVIII., Waehring@\textbf{XVIII., Währing}, \emph{Bezirk (A.BZK)}|pwk}18/\textsubscript{1}
                                       Wien, 29 XII 06, XI\textsuperscript{50}\nobreak{}«. \newline{}Ordnung: 1) mit Bleistift von unbekannter Hand nummeriert: »\strikeout{272}« 2) mit Bleistift von unbekannter Hand nummeriert:
                                    »269«}\buchAbdrucke{\weitereDrucke{Hugo von Hofmannsthal, Arthur Schnitzler: \emph{Briefwechsel}. Hg. Therese Nickl und Heinrich Schnitzler. Frankfurt am Main: \emph{S. Fischer} 1964, S. 225.} }\toendnotes[C]{\smallbreak}\pstart{}{\pb}Herrn D\textsuperscript{r} Arthur Schnitzler\pend{}\pstart{}\textcolor{pink}{Wien}{}\ledrightnote{\textcolor{pink}{Wien}}\pend{}\pstart{}\textcolor{pink}{XVII Spöttelgasse 7}{}\ledrightnote{\textcolor{pink}{Edmund-Weiß-Gasse}}.\pend{}{\bigskip}\pstart
           \raggedleft{}{\pb}28 XII.\pend
           \pstart
           lieber, ſehr lieb und gut daſs Sie ko{\geminationm}en wollen, aber unter dieſen Umſtänden erwarten wir Sie \uuline{nicht}, denn gerade Allein-herüber-fahren iſt das Langweilige und
               Unerfreuliche, beſonders in der Dunkelheit, das wollen wir nicht, alſo bald ein
               andres Mal Ihr beide.\pend
           \pstart
           Bitte den Abend des \uline{17}\textsuperscript{ten} Jänner freihalten für meinen
               (nicht-öffentlichen) \label{K_L01646_1v}\edtext{\textcolor{green}{Vortrag}{}\ledrightnote{\textcolor{green}{Der Dichter und diese Zeit}}}{\lemma{\textnormal{\emph{Vortrag}}}\Cendnote{\textnormal{Am 17. 1. 1907 hielt \textcolor{blue}{Hofmannsthal} den Vortrag \emph{\textcolor{green}{Der Dichter und diese Zeit}} im \emph{\textcolor{brown}{Kunstsalon Miethke}} vor geladenen, zehn Kronen zahlenden Gästen.}}}\label{K_L01646_1h}. Ihr
                  beko{\geminationm}t Eure Plätze direct von mir.\pend
           \pstart Ihr \spacefill\mbox{Hugo}\pend{}\pstart
           \noindent{}Wir ko{\geminationm}en baldmöglichſt zu Euch.\pend
           \endnumbering\briefempfaengerindex{Schnitzler, Arthur@\textsc{Schnitzler, Arthur}!zzzHofmannsthal, Hugo von@\emph{von Hugo von Hofmannsthal}!1906-12-281@{28. 12. 1906}|)be}\mylabel{h}  \normalsize

\doendnotes{C}
\bigskip
\vfill

\clearpage

\footnotesize

\lohead{\textsc{register}}

% Definiere theindex-Environment komplett neu ohne reledmac
\makeatletter
\renewenvironment{theindex}{%
  \section*{\indexname}%
  \setlength{\parindent}{0pt}%
  \setlength{\parskip}{0pt plus 0.3pt}%
  \let\item\@idxitem
}{%
  \clearpage
}
\makeatother

\IfFileExists{\jobname-pw.ind}{\input{\jobname-pw.ind}}{}

\end{document}

      