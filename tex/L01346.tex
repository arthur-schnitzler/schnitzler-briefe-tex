%% latex-korrekturansicht-vorspann.tex
%% Vorspann für die Korrekturansicht.
%% Lädt die gemeinsame Datei latex-vorspann.tex mit gesetztem Schalter.

\newif\ifkorrekturansicht
\korrekturansichttrue

\input{../tex-inputs/latex-vorspann}


               \section[Franz Blei an Arthur Schnitzler, 25. 11. 1903]{ Franz Blei an Arthur Schnitzler, 25. 11. 1903}\nopagebreak\mylabel{v}\rehead{ }\normalsize\beginnumbering\briefempfaengerindex{Schnitzler, Arthur@\textsc{Schnitzler, Arthur}!zzzBlei, Franz@\emph{von Franz Blei}!1903-11-251@{25. 11. 1903}|(be} \toendnotes[C]{\smallbreak\pagebreak[2]} \Standort{CUL, Schnitzler, B 14.}
\physDesc{Brief, 1 Blatt, 1 Seite
\newline{}Handschrift: schwarze Tinte, lateinische Kurrent
\newline{}Schnitzler: mit rotem Buntstift eine Unterstreichung \newline{}Ordnung: von Schnitzler mit Bleistift beschriftet: »\textsc{Blei}«, von unbekannter Hand mit Bleistift nummeriert:
                                        »1« }\toendnotes[C]{\smallbreak}\pstart
           \raggedleft{}{\pb}\textcolor{pink}{München, Arcisstrasse 19}{}\ledrightnote{\textcolor{pink}{Arcisstraße}}\pend
           \pstart\raggedleft{}Sehr geehrter Herr Doktor,\pend\pstart
           auf meine Anfrage theilt mit die Direktion der \textcolor{brown}{11 Scharfrichter}{}\ledrightnote{\textcolor{brown}{Die elf Scharfrichter}} mit, dass die Tantièmen für den vierzehnmal
                    gespielten \textcolor{green}{Dialog}{}\ledrightnote{→\textcolor{green}{Reigen. Zehn Dialoge}} 82 Mark
                    betragen. Direktor \textcolor{blue}{Henry}{}\ledrightnote{\textcolor{blue}{Marc Henry}} wird Ihnen den
                    Betrag \label{K_L01346_1v}\edtext{am 10. December}{\lemma{\textnormal{\emph{am 10. December}}}\Cendnote{\textnormal{An dem Tag sollten die \emph{\textcolor{brown}{11 Scharfrichter}} in \textcolor{pink}{Wien} auftreten.}}}\label{K_L01346_1h} in \textcolor{pink}{Wien}{}\ledrightnote{\textcolor{pink}{Wien}} zustellen.\pend
           \pstart
           Den beiliegenden \textcolor{green}{Ausschnitt}{}\ledrightnote{→\textcolor{green}{Ein Reigen}}
                    finde ich in der heutigen »\textcolor{brown}{Münchner Poſt}{}\ledrightnote{\textcolor{brown}{Münchener Post}}«, er
                    wird Sie interessieren.\pend
           \pstart
           Mit besten Grüssen{\\[\baselineskip]}Ihr ergebener{\\[\baselineskip]}\spacefill\mbox{Franz Blei}\pend
           \leftskip=0em{}\pstart
           25. 11. 1903\pend
           \endnumbering\briefempfaengerindex{Schnitzler, Arthur@\textsc{Schnitzler, Arthur}!zzzBlei, Franz@\emph{von Franz Blei}!1903-11-251@{25. 11. 1903}|)be}\mylabel{h}  \normalsize

\doendnotes{C}
\bigskip
\vfill

\clearpage

\footnotesize

\lohead{\textsc{register}}

% Definiere theindex-Environment komplett neu ohne reledmac
\makeatletter
\renewenvironment{theindex}{%
  \section*{\indexname}%
  \setlength{\parindent}{0pt}%
  \setlength{\parskip}{0pt plus 0.3pt}%
  \let\item\@idxitem
}{%
  \clearpage
}
\makeatother

\IfFileExists{\jobname-pw.ind}{\input{\jobname-pw.ind}}{}

\end{document}

      