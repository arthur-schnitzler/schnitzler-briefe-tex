%% latex-korrekturansicht-vorspann.tex
%% Vorspann für die Korrekturansicht.
%% Lädt die gemeinsame Datei latex-vorspann.tex mit gesetztem Schalter.

\newif\ifkorrekturansicht
\korrekturansichttrue

\input{../tex-inputs/latex-vorspann}


               \section[Arthur Schnitzler an Richard Beer-Hofmann, 5. 3. 1893]{ Arthur Schnitzler an Richard Beer-Hofmann, 5. 3. 1893}\nopagebreak\mylabel{v}\rehead{ }\normalsize\beginnumbering\briefempfaengerindex{Beer-Hofmann, Richard@\textsc{Beer-Hofmann, Richard}!zzzSchnitzler, Arthur@\emph{von Arthur Schnitzler}!1893-03-051@{5. 3. 1893}|(be} \toendnotes[C]{\smallbreak\pagebreak[2]} \Standort{YCGL, MSS 31.}
\physDesc{Brief, 1 Blatt, 4 Seiten, Umschlag
\newline{}Handschrift: blaue Tinte, deutsche Kurrent\newline{}Versand: 1) Stempel: »\nobreak{}\oindex{Pension Quisisana@\textbf{Pension Quisisana}, \emph{Hotel (K.HTL)}|pwk}Pension »Quisisana« Abbazia\nobreak{}«.  2) Stempel: »\nobreak{}\oindex{Opatija@\textbf{Opatija}, \emph{http://www.geonames.org/ontologyP.PPLA2}|pwk}Abbazia, 5/3 9\textcolor{gray}{3}\nobreak{}«. 3) Stempel: »\nobreak{}\oindex{I., Innere Stadt@\textbf{I., Innere Stadt}, \emph{Bezirk (A.BZK)}|pwk}Wien 1/1, 6/3. 93, 11½V–1N, Bestellt\nobreak{}«. }\buchAbdrucke{\weitereDrucke{Arthur Schnitzler, Richard Beer-Hofmann: \emph{Briefwechsel 1891–1931}. Hg. Konstanze Fliedl. Wien, Zürich: \emph{Europaverlag} 1992, S. 42.} }\toendnotes[C]{\smallbreak}\pstart{}{\pb}\textsc{Herrn Doctor Richard Beer-Hofmann}\pend{}\pstart{}\textsc{\textcolor{pink}{Wien}{}\ledrightnote{\textcolor{pink}{Wien}}}\pend{}\pstart{}\textsc{\textcolor{pink}{I Wollzeile 15}{}\ledrightnote{\textcolor{pink}{Wollzeile}}}.\pend{}{\bigskip}\pstart{}{\pb}Lieber Richard,\pend\pstart
           für die Anempfehlung von \textcolor{pink}{\textsc{Quisisana}}{}\ledrightnote{\textcolor{pink}{Pension Quisisana}} meinen beſten Dank! Ich fühle mich hier ſehr wohl, und habe überdies ein sehr
               hübſches Parterrezi{\geminationm}er mit Ausblick aufs weite Meer, das
               herrlichſte Wetter (ke{\geminationn}e keinen Ueberzieher mehr) und
               ſehr ſympathiſche Geſellschaft (die malende \textcolor{blue}{Schweſter}{}\ledrightnote{→\textcolor{blue}{Marie Rosenthal}}{ }\textcolor{blue}{\textsc{Rosenthal}}{}\ledrightnote{\textcolor{blue}{Moritz Rosenthal}}’s und die \textcolor{blue}{\textsc{Sophie Link}}{}\ledrightnote{\textcolor{blue}{Sophie Link}}, ſeit 6 Wochen in \textcolor{pink}{Berlin}{}\ledrightnote{\textcolor{pink}{Berlin}}{ }\textcolor{blue}{verheiratet}{}\ledrightnote{→\textcolor{blue}{Harry Löwenstein}}). – Ich bin meiſt
               im Freien, und pendle zwiſchen \textcolor{pink}{\textsc{Lovrana}}{}\ledrightnote{\textcolor{pink}{Lovran}} und \textcolor{pink}{\textsc{Voloska}}{}\ledrightnote{\textcolor{pink}{Volosko}}{ }{\pb}hin u her. – Gearbeitet – wenig; i{\geminationm}erhin ein Stück der \textcolor{green}{Novellette}{}\ledrightnote{→\textcolor{green}{Die kleine Komödie}}. – Die »\textcolor{green}{Familie}{}\ledrightnote{\textcolor{green}{Familie}}« durchgeleſen, merke, daſs was fehlt, und bin nicht recht klar was.
               Ich werde es auch jedenfalls in 2–3 Wochen vorleſen, aber um Rathschläge erſuchen
               müſſen. Keineswegs leſe ich, bevor wir Ihre \textcolor{green}{Novelle}{}\ledrightnote{→\textcolor{green}{Das Kind}} zu hören beko{\geminationm}en, was
               hoffentlich kurz nach meiner Ankunft möglich ſein wird! –\pend
           \pstart
           – Ich denke nicht gern ans Fortreiſen; die Ruhe hier thut mir ganz unbeſchreiblich
               wohl; wäre ich mein eigner Herr, ſo blieb’ ich zwei Monate da. We{\geminationn} man auch nicht {\pb}arbeitet, – man hat die Empfindung, daſs man es jeden Augenblick könnte, was faſt
               noch mehr werth ist. – Hübſch wär’s, we{\geminationn} wir das nächſte
               Frühjahr die ganze \textcolor{pink}{\textsc{Quisisana}}{}\ledrightnote{\textcolor{pink}{Pension Quisisana}} miethen könnten! – Ah, diese Luft – einfach entzückend! – Es iſt doch recht
               traurig zu den »Müſſenden« zu gehören! –\pend
           \pstart
           Grüßen Sie \textcolor{blue}{\textsc{Loris}}{}\ledrightnote{\textcolor{blue}{Hugo von Hofmannsthal}} und \textcolor{blue}{\textsc{Salten}}{}\ledrightnote{\textcolor{blue}{Felix Salten}} aufs allerherzlichſte, desgleichen \textcolor{blue}{\textsc{Schwarzkopf}}{}\ledrightnote{\textcolor{blue}{Gustav Schwarzkopf}}, der mir doch zwei Zeilen über das Befinden seines \textcolor{blue}{Bruders}{}\ledrightnote{→\textcolor{blue}{Rudolf Schwarzkopf}}{ }ſchreiben möchte; und grüßen Sie nebſtbei
               jedermann, der die Freundlichkeit hat nach mir zu fragen. – Schade, daſs {\pb}Sie nicht auch da ſind! Hoffentlich find ich Sie aber
               in geſegneterer Sti{\geminationm}ung als ich Sie verlaſſen!\pend
           \pstart
           Stets der Ihre{\\[\baselineskip]}\spacefill\mbox{Arthur.}\pend
           \leftskip=0em{}\pstart
           \textcolor{pink}{\textsc{Abbazia}}{}\ledrightnote{\textcolor{pink}{Opatija}}5. 3. 9\textcolor{gray}{3}. So{\geminationn}tag. –\pend
           \endnumbering\briefempfaengerindex{Beer-Hofmann, Richard@\textsc{Beer-Hofmann, Richard}!zzzSchnitzler, Arthur@\emph{von Arthur Schnitzler}!1893-03-051@{5. 3. 1893}|)be}\mylabel{h}  \normalsize

\doendnotes{C}
\bigskip
\vfill

\clearpage

\footnotesize

\lohead{\textsc{register}}

% Definiere theindex-Environment komplett neu ohne reledmac
\makeatletter
\renewenvironment{theindex}{%
  \section*{\indexname}%
  \setlength{\parindent}{0pt}%
  \setlength{\parskip}{0pt plus 0.3pt}%
  \let\item\@idxitem
}{%
  \clearpage
}
\makeatother

\IfFileExists{\jobname-pw.ind}{\input{\jobname-pw.ind}}{}

\end{document}

      