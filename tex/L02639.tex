%% latex-korrekturansicht-vorspann.tex
%% Vorspann für die Korrekturansicht.
%% Lädt die gemeinsame Datei latex-vorspann.tex mit gesetztem Schalter.

\newif\ifkorrekturansicht
\korrekturansichttrue

\input{../tex-inputs/latex-vorspann}


               \section[Paul Goldmann an Arthur Schnitzler, 14. 6. 1889]{ Paul Goldmann an Arthur Schnitzler, 14. 6. 1889}\nopagebreak\mylabel{v}\rehead{ }\normalsize\beginnumbering\briefempfaengerindex{Schnitzler, Arthur@\textsc{Schnitzler, Arthur}!zzzGoldmann, Paul@\emph{von Paul Goldmann}!1889-06-141@{14. 6. 1889}|(be} \toendnotes[C]{\smallbreak\pagebreak[2]} \Standort{DLA, A:Schnitzler, HS.NZ85.1.3162.}
\physDesc{Brief, 1 Blatt, 3 Seiten
\newline{}Handschrift: blaue Tinte, deutsche Kurrent
\newline{}Schnitzler: mit rotem Buntstift zwei Unterstreichungen }\toendnotes[C]{\smallbreak}\pstart
           \noindent{}\centering{}{\pb}\textcolor{gray}{\textbf{\textbf{Adminiſtration: \textcolor{pink}{VII.
                           Seidengaſſe 7}{}\ledrightnote{\textcolor{pink}{Seidengasse}}} (\textcolor{brown}{Jos. Eberle {\kaufmannsund} Co.}{}\ledrightnote{\textcolor{brown}{Josef Eberle  Stein-, Buch und Musikaliendruckerei}})}}\pend
           \pstart
           \noindent{}\centering{}\textcolor{gray}{\textbf{\textcolor{brown}{An der Schönen Blauen Donau}{}\ledrightnote{\textcolor{brown}{An der schönen blauen Donau}}}}\pend
           \pstart
           \noindent{}\centering{}\textcolor{gray}{\textbf{Chef-Redacteur: Dr. \textcolor{blue}{F.
                        Mamroth}{}\ledrightnote{\textcolor{blue}{Fedor Mamroth}}. – Redaction: \textcolor{pink}{IX.,
                        Berggaſſe 31}{}\ledrightnote{\textcolor{pink}{Berggasse}}.}}\pend
           \pstart
           \raggedleft{}\textcolor{gray}{\textbf{\textcolor{pink}{Wien}{}\ledrightnote{\textcolor{pink}{Wien}}, den}}{ }14. Juni \textcolor{gray}{\textbf{18}}89.\pend
           \pstart\center{}Sehr geehrter Herr Doctor!\pend\pstart
           Soeben erhalte ich von Herrn \textsc{\textcolor{blue}{Boxer}{}\ledrightnote{\textcolor{blue}{Oswald Boxer}}} die gewünſchte \label{K_L02639-1v}\edtext{Empfehlung}{\lemma{\textnormal{\emph{Empfehlung}}}\Cendnote{\textnormal{Es handelt sich um ein Empfehlungsschreiben
                  für die im Folgenden angesprochene Kontaktaufnahme mit \textcolor{blue}{Paul Lindau}. Die erhaltene Korrespondenz \textcolor{blue}{Schnitzler}s mit \textcolor{blue}{Lindau} beginnt 1895.}}}\label{K_L02639-1h}. Ich halte es für ſehr günftig,
               daß er ſelbſt es übernommen hat, Ihnen dieſe Empfehlung zu geben, da College \textsc{\textcolor{blue}{Boxer}{}\ledrightnote{\textcolor{blue}{Oswald Boxer}}}, wie ich weiß, zu all den Herren der \textcolor{pink}{Berlin}{}\ledrightnote{\textcolor{pink}{Berlin}}er Schriftſteller-Welt infolge ſeiner einflußreichen Stellung als \textcolor{blue}{Correſpondent}{}\ledrightnote{→\textcolor{blue}{Oswald Boxer}} dreier großer
                  \textcolor{pink}{Wien}{}\ledrightnote{\textcolor{pink}{Wien}}er \label{K_L02639-4v}\edtext{\textcolor{brown}{Blätter}{}\ledrightnote{→\textcolor{brown}{Die Presse}}}{\lemma{\textnormal{\emph{Blätter}}}\Cendnote{\textnormal{\textcolor{blue}{Oswald Boxer} arbeitete jedenfalls als \textcolor{pink}{Berlin}er Korrespondent der \emph{\textcolor{brown}{Presse}}.}}}\label{K_L02639-4h} ſehr gute Beziehungen hat.\pend
           \pstart
           Wenn ich mir nun erlauben {\pb}darf,
               Ihnen noch weiterhin einen Rath zu geben, ſo geht derſelbe dahin: Überſenden Sie das
                  \label{K_L02639-2v}\edtext{Manuſcript}{\lemma{\textnormal{\emph{Manuſcript}}}\Cendnote{\textnormal{nicht identifiziert}}}\label{K_L02639-2h} dem \textsc{\textcolor{blue}{Paul Lindau}{}\ledrightnote{\textcolor{blue}{Paul Lindau}}}{ }\uline{bald}, damit er die Sendung erhält, bevor er in’s Bad
               fährt; adreſſiren Sie ferner an ihn direct, \uline{nicht} an
               die \textcolor{brown}{Redaction}{}\ledrightnote{→\textcolor{brown}{Nord und Süd}};
               nun legen Sie in Ihrem Begleitſchreiben ganz offen den Grund des Empfehlungs-Briefes
               dar: daß \strikeout{es} Ihnen nichts ferner gelegen, als dadurch
               ſein Urtheil beeinfluſſen zu wollen, daß Sie im Gegentheil – was Ihnen, als
               unbekannten jüngern Litteraten ſonſt vielleicht unmöglich geweſen wäre – dadurch nur
               erreichen wollten, daß Ihr \label{K_L02639-6v}\edtext{Manuscript}{\lemma{\textnormal{\emph{Manuscript}}}\Cendnote{\textnormal{nicht
                  identifiziert}}}\label{K_L02639-6h} von ihm \uline{geleſen} werde.\pend
           \pstart
           Die \label{K_L02639-5v}\edtext{\textcolor{green}{Wärterin}{}\ledrightnote{\textcolor{green}{[Die Wärterin]}}}{\lemma{\textnormal{\emph{Wärterin}}}\Cendnote{\textnormal{unklar; eventuell handelt es sich um
                  eine Ausarbeitung der folgenden Notiz: »Die junge Frau bei dem Assistenzarzt des Spitals. Er hat Dienst, Eine
                        Wärterin ruft ihn ab. Ein Selbstmörder ist gebracht worden, sterbend. Sie
                        ist fortgegangen, findet ihren Mann nicht zuhause. Bringt die Photographie
                        ihres Manns ins Spital, frägt den Geliebten: ›Ist’s der?‹ - Ja, es ist der
                        Selbstmörder.{ / }Einakter: Gespräch der Bedienerin mit der Frau. Zurückkehren des
                        Sekundararztes. Er schickt die Frau nach Hause. Der Freund kommt. Oder eine
                        Wärterin kommt: Die Identität ist festgestellt.« (\emph{Entworfenes und Verworfenes} 27)}}}\label{K_L02639-5h} haben
               Sie hoffentlich ſchon herausgeputzt; einen fübſchen, markanten Titel werden Sie wohl
               noch finden; und dann {\pb}– Glückauf
               zur \label{K_L02639-3v}\edtext{Fahrt}{\lemma{\textnormal{\emph{Fahrt}}}\Cendnote{\textnormal{nicht ermittelt}}}\label{K_L02639-3h}! {\dots}\pend
           \pstart
           Ich empfehle mich Ihnen hochachtungsvoll {\\[\baselineskip]}Ihr ergebener {\\[\baselineskip]}\spacefill\mbox{Dr. Paul Goldmann}\pend
           \leftskip=0em{}\endnumbering\briefempfaengerindex{Schnitzler, Arthur@\textsc{Schnitzler, Arthur}!zzzGoldmann, Paul@\emph{von Paul Goldmann}!1889-06-141@{14. 6. 1889}|)be}\mylabel{h}  \normalsize

\doendnotes{C}
\bigskip
\vfill

\clearpage

\footnotesize

\lohead{\textsc{register}}

% Definiere theindex-Environment komplett neu ohne reledmac
\makeatletter
\renewenvironment{theindex}{%
  \section*{\indexname}%
  \setlength{\parindent}{0pt}%
  \setlength{\parskip}{0pt plus 0.3pt}%
  \let\item\@idxitem
}{%
  \clearpage
}
\makeatother

\IfFileExists{\jobname-pw.ind}{\input{\jobname-pw.ind}}{}

\end{document}

      