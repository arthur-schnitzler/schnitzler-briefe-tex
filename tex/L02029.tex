%% latex-korrekturansicht-vorspann.tex
%% Vorspann für die Korrekturansicht.
%% Lädt die gemeinsame Datei latex-vorspann.tex mit gesetztem Schalter.

\newif\ifkorrekturansicht
\korrekturansichttrue

\input{../tex-inputs/latex-vorspann}


               \section[Hermann Bahr an Arthur Schnitzler, 17. 9. 1911]{ Hermann Bahr an Arthur Schnitzler, 17. 9. 1911}\nopagebreak\mylabel{v}\rehead{ }\normalsize\beginnumbering\briefempfaengerindex{Schnitzler, Arthur@\textsc{Schnitzler, Arthur}!zzzBahr, Hermann@\emph{von Hermann Bahr}!1911-09-171@{17. 9. 1911}|(be} \toendnotes[C]{\smallbreak\pagebreak[2]} \Standort{CUL, Schnitzler, B 5b.}
\physDesc{Brief, 1 Blatt, 1 Seite
\newline{}Handschrift Lisa Clarus: schwarze Tinte, lateinische Kurrent\newline{}Handschrift Hermann Bahr: schwarze Tinte (\noindent{}Unterschrift)\newline{}Ordnung: mit Bleistift von unbekannter Hand nummeriert:
                                    »171« }\buchAbdrucke{\weitereDrucke{Hermann Bahr, Arthur Schnitzler: \emph{Briefwechsel, Aufzeichnungen, Dokumente (1891–1931)}. Hg. Kurt Ifkovits und Martin Anton Müller. Göttingen: \emph{Wallstein} 2018, S. 452.} }\toendnotes[C]{\smallbreak}\pstart
           \raggedleft{}{\pb}\textcolor{pink}{Wien XIII/\textsubscript{7}}{}\ledrightnote{\textcolor{pink}{Ober Sankt Veit}}\hspace*{2.5em}17. 9. 11.\pend
           \pstart\center{}Lieber Freund!\pend\pstart
           Möchtest Du so lieb sein, mir Dein \textcolor{green}{neues Stück}{}\ledrightnote{→\textcolor{green}{Das weite Land. Tragikomödie in fünf Akten}} zu schicken? \textcolor{blue}{Burckhard}{}\ledrightnote{\textcolor{blue}{Max Eugen Burckhard}} hat
               mir neulich so viel davon so schön erzählt, dass ich noch neugieriger geworden bin.
               Und ich könnte es jetzt halbwegs in Ruhe lesen!\pend
           \pstart
           Mit den herzlichsten Grüssen von uns beiden, auch an Deine liebe \textcolor{blue}{Frau}{}\ledrightnote{→\textcolor{blue}{Olga Schnitzler}},\pend
           \pstart
           Dein alter{\\[\baselineskip]}\spacefill\mbox{{[}hs. Bahr:{]} HermannBahr}\pend
           \leftskip=0em{}\endnumbering\briefempfaengerindex{Schnitzler, Arthur@\textsc{Schnitzler, Arthur}!zzzBahr, Hermann@\emph{von Hermann Bahr}!1911-09-171@{17. 9. 1911}|)be}\mylabel{h}  \normalsize

\doendnotes{C}
\bigskip
\vfill

\clearpage

\footnotesize

\lohead{\textsc{register}}

% Definiere theindex-Environment komplett neu ohne reledmac
\makeatletter
\renewenvironment{theindex}{%
  \section*{\indexname}%
  \setlength{\parindent}{0pt}%
  \setlength{\parskip}{0pt plus 0.3pt}%
  \let\item\@idxitem
}{%
  \clearpage
}
\makeatother

\IfFileExists{\jobname-pw.ind}{\input{\jobname-pw.ind}}{}

\end{document}

      