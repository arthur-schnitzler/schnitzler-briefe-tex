%% latex-korrekturansicht-vorspann.tex
%% Vorspann für die Korrekturansicht.
%% Lädt die gemeinsame Datei latex-vorspann.tex mit gesetztem Schalter.

\newif\ifkorrekturansicht
\korrekturansichttrue

\input{../tex-inputs/latex-vorspann}


               \section[Richard Beer-Hofmann an Arthur Schnitzler, 29. 5. 1899]{ Richard Beer-Hofmann an Arthur Schnitzler, 29. 5. 1899}\nopagebreak\mylabel{v}\rehead{ }\normalsize\beginnumbering\briefempfaengerindex{Schnitzler, Arthur@\textsc{Schnitzler, Arthur}!zzzBeer-Hofmann, Richard@\emph{von Richard Beer-Hofmann}!1899-05-291@{29. 5. 1899}|(be} \toendnotes[C]{\smallbreak\pagebreak[2]} \Standort{CUL, Schnitzler, B 8.}
\physDesc{Faltkarte
\newline{}Handschrift: Bleistift, lateinische Kurrent}\buchAbdrucke{\weitereDrucke{Arthur Schnitzler, Richard Beer-Hofmann: \emph{Briefwechsel 1891–1931}. Hg. Konstanze Fliedl. Wien, Zürich: \emph{Europaverlag} 1992, S. 127–128.} }\toendnotes[C]{\smallbreak}\pstart
           \noindent{}\centering{}{\pb}\textcolor{gray}{\textbf{Platz}}.\pend
           \pstart
           \noindent{}\centering{}\textcolor{gray}{\textbf{{\pb}Besten Gruss aus \textcolor{pink}{Villach}{}\ledrightnote{\textcolor{pink}{Villach}} sendet}}\pend
           \pstart\center{}Lieber Arthur!\pend\pstart
           In \label{T_L00919_1v}\edtext{diesem
                  Hause}{\lemma{\textnormal{\emph{diesem
                  Hause}}}\Cendnote{\textnormal{Ein Pfeil
                  mit Bleistift markiert das Gebäude auf der gedruckten Abbildung.}}}\label{T_L00919_1h} lebte von
                  1502 bis zu seinem Tode 8 Sept 1534 als Stadtarzt von
                  \textcolor{pink}{Villach}{}\ledrightnote{\textcolor{pink}{Villach}}, \textcolor{blue}{Wilhelm
                  Bombast von Hohenheim}{}\ledrightnote{\textcolor{blue}{Wilhelm Bombast von Hohenheim}}; sein Sohn, der durch Sie — so \label{K_L00919_1v}\edtext{\textcolor{green}{berühmte}{}\ledrightnote{\textcolor{green}{Der grüne Kakadu. Groteske in einem Akt}}}{\lemma{\textnormal{\emph{berühmte}}}\Cendnote{\textnormal{Anspielung auf \textcolor{blue}{Schnitzler}s Einakter \emph{\textcolor{green}{Paracelsus}}.}}}\label{K_L00919_1h}{ }\textcolor{blue}{Paracelsus}{}\ledrightnote{\textcolor{blue}{Theofrastus Bombastus Paracelsus}} lebte hier von
                  1502–1516, und Richard Beer-Hofmann trank am
                  29/V 1899 hier schwarzen Kaffee; das letzte kann natürlich heute noch
               nicht auf der Gedenktafel stehen.\pend
           \pstart
           Herzlichst{\\[\baselineskip]}\spacefill\mbox{Richard}\pend
           \leftskip=0em{}\endnumbering\briefempfaengerindex{Schnitzler, Arthur@\textsc{Schnitzler, Arthur}!zzzBeer-Hofmann, Richard@\emph{von Richard Beer-Hofmann}!1899-05-291@{29. 5. 1899}|)be}\mylabel{h}  \normalsize

\doendnotes{C}
\bigskip
\vfill

\clearpage

\footnotesize

\lohead{\textsc{register}}

% Definiere theindex-Environment komplett neu ohne reledmac
\makeatletter
\renewenvironment{theindex}{%
  \section*{\indexname}%
  \setlength{\parindent}{0pt}%
  \setlength{\parskip}{0pt plus 0.3pt}%
  \let\item\@idxitem
}{%
  \clearpage
}
\makeatother

\IfFileExists{\jobname-pw.ind}{\input{\jobname-pw.ind}}{}

\end{document}

      