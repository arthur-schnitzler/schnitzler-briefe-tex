%% latex-korrekturansicht-vorspann.tex
%% Vorspann für die Korrekturansicht.
%% Lädt die gemeinsame Datei latex-vorspann.tex mit gesetztem Schalter.

\newif\ifkorrekturansicht
\korrekturansichttrue

\input{../tex-inputs/latex-vorspann}


               \section[Arthur Schnitzler an Richard Beer-Hofmann, 24. 9. 1899]{ Arthur Schnitzler an Richard Beer-Hofmann, 24. 9. 1899}\nopagebreak\mylabel{v}\rehead{ }\normalsize\beginnumbering\briefempfaengerindex{Beer-Hofmann, Richard@\textsc{Beer-Hofmann, Richard}!zzzSchnitzler, Arthur@\emph{von Arthur Schnitzler}!1899-09-241@{24. 9. 1899}|(be} \toendnotes[C]{\smallbreak\pagebreak[2]} \Standort{CUL, Schnitzler, B 8.}
\physDesc{Bildpostkarte
\newline{}Handschrift: Bleistift, deutsche Kurrent\newline{}Versand: 1) Stempel: »\nobreak{}\oindex{Wiesbaden@\textbf{Wiesbaden}, \emph{Besiedelter Ort (A.BSO)}|pwk}Wiesbaden, 24. 9. 99, 6–7N\nobreak{}«.  2) Stempel: »\nobreak{}\oindex{Vahrn@\textbf{Vahrn}, \emph{Besiedelter Ort (A.BSO)}|pwk}V{[}ahrn{]}, 2\textcolor{gray}{6.} 9. 99\nobreak{}«. \newline{}Ordnung: mit Bleistift von unbekannter Hand datiert: »24. 9.« }\toendnotes[C]{\smallbreak}\pstart{}{\pb}\textsc{Dr. Richard Beer-Hofmann}\pend{}\pstart{}\textcolor{pink}{\textsc{Vahrn}}{}\ledrightnote{\textcolor{pink}{Vahrn}}\pend{}\pstart{}bei \textsc{\textcolor{pink}{Brixen}{}\ledrightnote{\textcolor{pink}{Brixen}}}\pend{}\pstart{}\textcolor{pink}{\textsc{Tirol}}{}\ledrightnote{\textcolor{pink}{Tirol}}\pend{}{\bigskip}\pstart
           \noindent{}\centering{}\textcolor{gray}{\textbf{{\pb}\textcolor{pink}{Nerobergbahn}{}\ledrightnote{\textcolor{pink}{Nerobergbahn}}{ }\textcolor{pink}{Wiesbaden}{}\ledrightnote{\textcolor{pink}{Wiesbaden}}}}\pend
           \pstart
           \raggedleft{}{\pb}24. 9. 99.\pend
           \pstart
           Will hier 8 Tage bleiben, arbeiten\pend
           \pstart
           Bitte ſchreiben Sie mir, auch \textcolor{blue}{Hugo}{}\ledrightnote{\textcolor{blue}{Hugo von Hofmannsthal}}, wie’s Ihnen
               geht, und was die Arbeit macht. Ich wohne \textcolor{pink}{Parkhotel}{}\ledrightnote{\textcolor{pink}{Hôtel du Parc {\kaufmannsund} Bristol}}.\pend
           \pstart
           Die \label{K_L00980_1v}\edtext{Ovation}{\lemma{\textnormal{\emph{Ovation}}}\Cendnote{\textnormal{vgl. Richard Beer-Hofmann und Hugo von Hofmannsthal an Arthur Schnitzler,
               19. 9. 1899}}}\label{K_L00980_1h} hab ich erhalten. \textcolor{blue}{Paul}{}\ledrightnote{\textcolor{blue}{Paul Goldmann}} ist heut nach \textcolor{pink}{Florenz}{}\ledrightnote{\textcolor{pink}{Florenz}}.\pend
           \endnumbering\briefempfaengerindex{Beer-Hofmann, Richard@\textsc{Beer-Hofmann, Richard}!zzzSchnitzler, Arthur@\emph{von Arthur Schnitzler}!1899-09-241@{24. 9. 1899}|)be}\mylabel{h}  \normalsize

\doendnotes{C}
\bigskip
\vfill

\clearpage

\footnotesize

\lohead{\textsc{register}}

% Definiere theindex-Environment komplett neu ohne reledmac
\makeatletter
\renewenvironment{theindex}{%
  \section*{\indexname}%
  \setlength{\parindent}{0pt}%
  \setlength{\parskip}{0pt plus 0.3pt}%
  \let\item\@idxitem
}{%
  \clearpage
}
\makeatother

\IfFileExists{\jobname-pw.ind}{\input{\jobname-pw.ind}}{}

\end{document}

      