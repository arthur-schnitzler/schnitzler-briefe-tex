%% latex-korrekturansicht-vorspann.tex
%% Vorspann für die Korrekturansicht.
%% Lädt die gemeinsame Datei latex-vorspann.tex mit gesetztem Schalter.

\newif\ifkorrekturansicht
\korrekturansichttrue

\input{../tex-inputs/latex-vorspann}


               \section[Hugo von Hofmannsthal an Arthur Schnitzler, 30. 11. {[}1911{]}]{ Hugo von Hofmannsthal an Arthur Schnitzler,
               30. 11. {[}1911{]}}\nopagebreak\mylabel{v}\rehead{ }\normalsize\beginnumbering\briefempfaengerindex{Schnitzler, Arthur@\textsc{Schnitzler, Arthur}!zzzHofmannsthal, Hugo von@\emph{von Hugo von Hofmannsthal}!1911-11-302@{30. 11. {[}1911{]}}|(be} \toendnotes[C]{\smallbreak\pagebreak[2]} \Standort{CUL, Schnitzler, B 43.}
\physDesc{Telegramm
\newline{}maschinell\newline{}Versand: Stempel des Telegrafenbeamten: »\textcolor{gray}{\textbf{\textit{\textcolor{blue}{J. S.
                                 Steimetzer}}}}« 
\newline{}Schnitzler: mit Bleistift datiert: »30/12 911« \newline{}Ordnung: mit Bleistift von unbekannter Hand nummeriert: »334« }\buchAbdrucke{\weitereDrucke{Hugo von Hofmannsthal, Arthur Schnitzler: \emph{Briefwechsel}. Hg. Therese Nickl und Heinrich Schnitzler. Frankfurt am Main: \emph{S. Fischer} 1964, S. 264.} }\toendnotes[C]{\smallbreak}\pstart
           {\pb}\textcolor{pink}{berlin}{}\ledrightnote{\textcolor{pink}{Berlin}} fd 954 17 30/11{ }7,38 s =\pend
           \pstart
           danke herzlichst fuer so gute liebe worte in etwas baenglichem \textcolor{green}{moment}{}\ledrightnote{→\textcolor{green}{Jedermann. Das Spiel vom Sterben des reichen Mannes}}\pend
           \pstart \spacefill\mbox{= hugo .+}\pend{}\endnumbering\briefempfaengerindex{Schnitzler, Arthur@\textsc{Schnitzler, Arthur}!zzzHofmannsthal, Hugo von@\emph{von Hugo von Hofmannsthal}!1911-11-302@{30. 11. {[}1911{]}}|)be}\mylabel{h}  \normalsize

\doendnotes{C}
\bigskip
\vfill

\clearpage

\footnotesize

\lohead{\textsc{register}}

% Definiere theindex-Environment komplett neu ohne reledmac
\makeatletter
\renewenvironment{theindex}{%
  \section*{\indexname}%
  \setlength{\parindent}{0pt}%
  \setlength{\parskip}{0pt plus 0.3pt}%
  \let\item\@idxitem
}{%
  \clearpage
}
\makeatother

\IfFileExists{\jobname-pw.ind}{\input{\jobname-pw.ind}}{}

\end{document}

      