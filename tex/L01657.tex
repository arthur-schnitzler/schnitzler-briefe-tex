%% latex-korrekturansicht-vorspann.tex
%% Vorspann für die Korrekturansicht.
%% Lädt die gemeinsame Datei latex-vorspann.tex mit gesetztem Schalter.

\newif\ifkorrekturansicht
\korrekturansichttrue

\input{../tex-inputs/latex-vorspann}


               \section[Arthur Schnitzler an Hermann Bahr, 15. 2. 1907]{ Arthur Schnitzler an Hermann Bahr, 15. 2. 1907}\nopagebreak\mylabel{v}\rehead{ }\normalsize\beginnumbering\briefempfaengerindex{Bahr, Hermann@\textsc{Bahr, Hermann}!zzzSchnitzler, Arthur@\emph{von Arthur Schnitzler}!1907-02-151@{15. 2. 1907}|(be} \toendnotes[C]{\smallbreak\pagebreak[2]} \Standort{TMW, HS AM 23382 Ba.}
\physDesc{Brief, 1 Blatt, 3 Seiten
\newline{}Handschrift: schwarze Tinte, deutsche Kurrent\newline{}Ordnung: Lochung }\buchAbdrucke{\weitereDrucke{1) \emph{15. 2. 1907.} In: Arthur Schnitzler: \emph{The Letters of Arthur Schnitzler to Hermann Bahr}. Edited, annotated, and with an introduction, by Donald G.
                        Daviau. Chapel Hill: \emph{The University of North Carolina Press} 1978, S. 96–97 (University of North Carolina studies in the Germanic languages
                        and literatures, 89).} \weitereDrucke{2) Hermann Bahr, Arthur Schnitzler: \emph{Briefwechsel, Aufzeichnungen, Dokumente (1891–1931)}. Hg. Kurt Ifkovits und Martin Anton Müller. Göttingen: \emph{Wallstein} 2018, S. 389.} }\toendnotes[C]{\smallbreak}\pstart
           \raggedleft{}{\pb}\textcolor{pink}{Wien}{}\ledrightnote{\textcolor{pink}{Wien}}, 15. 2. 907\pend
           \pstart{}lieber Hermann,\pend\pstart
           vielen Dank. \textsc{\textcolor{green}{Lbl}{}\ledrightnote{\textcolor{green}{Liebelei. Schauspiel in drei Akten}}} ein Exemplar geſtern an dich geſandt. Ich bitte dich nur recht ſehr, dir
               keinerlei Ungelegenheiten zu machen. Wenn \textcolor{blue}{\textsc{R}.}{}\ledrightnote{\textcolor{blue}{Max Reinhardt}} gern daran geht, ja. Aber wenns ihm nicht von
               Herzen iſt, da{\geminationn} lieber nicht. Wie denkſt du dir die
               ſonſtigen Beſetzungsmöglichkeiten? Iſt \textcolor{blue}{Pagay}{}\ledrightnote{\textcolor{blue}{Hans Pagay}} für
               den Alten nicht zu trocken?\pend
           \pstart
           {\pb}\textcolor{blue}{\textsc{Valentin}}{}\ledrightnote{\textcolor{blue}{Richard Vallentin}} hat mir neuerdings wegen der \textcolor{green}{\textsc{Bea}}{}\ledrightnote{\textcolor{green}{Der Schleier der Beatrice. Schauspiel in fünf Akten}}. geſchrieben; ich hab mich noch nicht endgiltig ausgeſprochen.\pend
           \pstart
           Bin im übrigen ziemlich fleißig und hoffe zu nächſtem Herbst mit etlichem bereit zu
               ſein.\pend
           \pstart
           Famos dein »\label{K_L01657_1v}\edtext{\textcolor{green}{\textcolor{blue}{Grillparzer}{}\ledrightnote{\textcolor{blue}{Franz Grillparzer}}}{}\ledrightnote{\textcolor{green}{Grillparzer}}}{\lemma{\textnormal{\emph{Grillparzer}}}\Cendnote{\textnormal{\textcolor{blue}{Hermann Bahr}: \emph{\textcolor{green}{Grillparzer}}. In: \emph{\textcolor{green}{Die Schaubühne}},
                     Jg. 3, H. 7, 14. 2. 1907, S. 163–170, als Vorabdruck aus
                     \emph{\textcolor{green}{Wien}} gekennzeichnet.}}}\label{K_L01657_1h}« in der \textcolor{brown}{Schaubühne}{}\ledrightnote{\textcolor{brown}{Die Schaubühne / Die Weltbühne}}. Freu\damage{e} mich auf das ganze \label{K_L01657_2v}\edtext{\textcolor{green}{Buch}{}\ledrightnote{→\textcolor{green}{Wien}}}{\lemma{\textnormal{\emph{Buch}}}\Cendnote{\textnormal{\textcolor{blue}{Hermann Bahr}: \emph{\textcolor{green}{Wien}}. Stuttgart: \emph{Karl Krabbe}{ }1907 (erschienen in der
               zweiten Mai-Hälfte).}}}\label{K_L01657_2h}.\pend
           \pstart
           Was machſt du nach \textcolor{pink}{Berlin}{}\ledrightnote{\textcolor{pink}{Berlin}}? Sollte die \label{K_L01657_3v}\edtext{\textcolor{green}{\textcolor{brown}{\textsc{Neue Freie}}{}\ledrightnote{\textcolor{brown}{Neue Freie Presse}}}{}\ledrightnote{→\textcolor{green}{Laiengedanken über die Wahlen in Österreich}} den {\pb}Beginn
               deiner Wiederkehr}{\lemma{\textnormal{\emph{Neue … Wiederkehr}}}\Cendnote{\textnormal{Das Feuilleton \emph{\textcolor{green}{Laiengedanken über die Wahlen in Österreich}} am
                     2. 2. 1907 (Nr. 15249, Morgenblatt, S. 3–4) eröffnete
                  die bis zum Tod anhaltende Mitarbeit an der \emph{\textcolor{brown}{Neuen
                     Freien Presse}}.}}}\label{K_L01657_3h} bedeuten? \pend
           \pstart
           Meine \textcolor{blue}{Frau}{}\ledrightnote{→\textcolor{blue}{Olga Schnitzler}} grüßt dich
               vielmals. Von Herzen{\\[\baselineskip]}Dein{\\[\baselineskip]}\spacefill\mbox{Arthur}\pend
           \leftskip=0em{}\endnumbering\briefempfaengerindex{Bahr, Hermann@\textsc{Bahr, Hermann}!zzzSchnitzler, Arthur@\emph{von Arthur Schnitzler}!1907-02-151@{15. 2. 1907}|)be}\mylabel{h}  \normalsize

\doendnotes{C}
\bigskip
\vfill

\clearpage

\footnotesize

\lohead{\textsc{register}}

% Definiere theindex-Environment komplett neu ohne reledmac
\makeatletter
\renewenvironment{theindex}{%
  \section*{\indexname}%
  \setlength{\parindent}{0pt}%
  \setlength{\parskip}{0pt plus 0.3pt}%
  \let\item\@idxitem
}{%
  \clearpage
}
\makeatother

\IfFileExists{\jobname-pw.ind}{\input{\jobname-pw.ind}}{}

\end{document}

      