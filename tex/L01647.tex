%% latex-korrekturansicht-vorspann.tex
%% Vorspann für die Korrekturansicht.
%% Lädt die gemeinsame Datei latex-vorspann.tex mit gesetztem Schalter.

\newif\ifkorrekturansicht
\korrekturansichttrue

\input{../tex-inputs/latex-vorspann}


               \section[Richard Dehmel an Arthur Schnitzler, {[}1907{]}]{ Richard Dehmel an Arthur Schnitzler, {[}1907{]}}\nopagebreak\mylabel{v}\rehead{ }\normalsize\beginnumbering\briefempfaengerindex{Schnitzler, Arthur@\textsc{Schnitzler, Arthur}!zzzDehmel, Richard@\emph{von Richard Dehmel}!1907-01-012@{{[}1907{]}}|(be} \toendnotes[C]{\smallbreak\pagebreak[2]} \Standort{CUL, Schnitzler, B 26.}
\physDesc{Brief, 1 Blatt, 1 Seite
\newline{}Druck}\pstart\center{}{\pb}EURER WOHLGEBOREN\pend\pstart
           erhalten anbei ein Exemplar meiner »\textcolor{green}{Verwandlungen
                        der Venus}{}\ledrightnote{\textcolor{green}{Die Verwandlungen der Venus}}« im \uline{vollständigen} Wortlaut.
                    Ich sende es Ihnen, weil ich annehmen darf, daß Sie der genannten Dichtung,
                    deren öffentliche Ausgabe an einer wichtigen Stelle (Venus Consolatrix) auf
                    gerichtlichen Befehl verstümmelt werden mußte, ein rein ästhetisches oder
                    ideelles Interesse entgegenbringen. Deshalb darf ich auch glauben, daß Sie
                    dieses private Exemplar, welches ich Ihnen als \uline{vertrauliche} Gabe überreiche, nicht in falsche Hände geraten lassen
                    werden. Meine Absicht dabei ist lediglich die, einige vollständige Exemplare des
                    Textes dem Urteil der Nachlebenden zuzuführen.\pend
           \pstart
           Mit besonderer Hochachtung{\\[\baselineskip]}\spacefill\mbox{R. DEHMEL.}\pend
           \leftskip=0em{}\endnumbering\briefempfaengerindex{Schnitzler, Arthur@\textsc{Schnitzler, Arthur}!zzzDehmel, Richard@\emph{von Richard Dehmel}!1907-01-012@{{[}1907{]}}|)be}\mylabel{h}  \normalsize

\doendnotes{C}
\bigskip
\vfill

\clearpage

\footnotesize

\lohead{\textsc{register}}

% Definiere theindex-Environment komplett neu ohne reledmac
\makeatletter
\renewenvironment{theindex}{%
  \section*{\indexname}%
  \setlength{\parindent}{0pt}%
  \setlength{\parskip}{0pt plus 0.3pt}%
  \let\item\@idxitem
}{%
  \clearpage
}
\makeatother

\IfFileExists{\jobname-pw.ind}{\input{\jobname-pw.ind}}{}

\end{document}

      