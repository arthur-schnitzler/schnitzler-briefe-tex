%% latex-korrekturansicht-vorspann.tex
%% Vorspann für die Korrekturansicht.
%% Lädt die gemeinsame Datei latex-vorspann.tex mit gesetztem Schalter.

\newif\ifkorrekturansicht
\korrekturansichttrue

\input{../tex-inputs/latex-vorspann}


               \section[Richard Beer-Hofmann an Arthur Schnitzler, 13. 6. 1930]{ Richard Beer-Hofmann an Arthur Schnitzler,
               13. 6. 1930}\nopagebreak\mylabel{v}\rehead{ }\normalsize\beginnumbering\briefempfaengerindex{Schnitzler, Arthur@\textsc{Schnitzler, Arthur}!zzzBeer-Hofmann, Richard@\emph{von Richard Beer-Hofmann}!1930-06-131@{13. 6. 1930}|(be} \toendnotes[C]{\smallbreak\pagebreak[2]} \Standort{CUL, Schnitzler, B 8.}
\physDesc{Bildpostkarte
\newline{}Handschrift: blauer Buntstift, lateinische Kurrent\newline{}Versand: Stempel: »\nobreak{}\oindex{Mariazell@\textbf{Mariazell}, \emph{Besiedelter Ort (A.BSO)}|pwk}Mariazell, 13. VI. 30\nobreak{}«.  \newline{}Ordnung: mit Bleistift von unbekannter Hand nummeriert: »276« }\pstart{}{\pb}Herrn\pend{}\pstart{}Arthur Schnitzler\pend{}\pstart{}\textcolor{pink}{Wien XVIII}{}\ledrightnote{\textcolor{pink}{XVIII., Währing}}\pend{}\pstart{}\textcolor{pink}{Sternwartstrasse 71}{}\ledrightnote{\textcolor{pink}{Sternwartestraße}}\pend{}{\bigskip}\pstart
           \noindent{}\centering{}{\pb}\textcolor{gray}{\textbf{\textcolor{pink}{Mariazell}{}\ledrightnote{\textcolor{pink}{Mariazell}}, 862 m Seehöhe, \textcolor{pink}{Steiermark}{}\ledrightnote{\textcolor{pink}{Steiermark}}\hspace*{1.5em}\textcolor{pink}{Gnadenkirche}{}\ledrightnote{\textcolor{pink}{Basilika Mariä Geburt}}}}\pend
           \pstart
           \raggedleft{}{\pb}13.
                     VI. 30\pend
           \pstart
           Herzliche Grüsse von \textcolor{blue}{Paula}{}\ledrightnote{\textcolor{blue}{Paula Beer-Hofmann}} und
               mir!\pend
           \pstart Ihr\spacefill\mbox{Richard}\pend{}\endnumbering\briefempfaengerindex{Schnitzler, Arthur@\textsc{Schnitzler, Arthur}!zzzBeer-Hofmann, Richard@\emph{von Richard Beer-Hofmann}!1930-06-131@{13. 6. 1930}|)be}\mylabel{h}  \normalsize

\doendnotes{C}
\bigskip
\vfill

\clearpage

\footnotesize

\lohead{\textsc{register}}

% Definiere theindex-Environment komplett neu ohne reledmac
\makeatletter
\renewenvironment{theindex}{%
  \section*{\indexname}%
  \setlength{\parindent}{0pt}%
  \setlength{\parskip}{0pt plus 0.3pt}%
  \let\item\@idxitem
}{%
  \clearpage
}
\makeatother

\IfFileExists{\jobname-pw.ind}{\input{\jobname-pw.ind}}{}

\end{document}

      