%% latex-korrekturansicht-vorspann.tex
%% Vorspann für die Korrekturansicht.
%% Lädt die gemeinsame Datei latex-vorspann.tex mit gesetztem Schalter.

\newif\ifkorrekturansicht
\korrekturansichttrue

\input{../tex-inputs/latex-vorspann}


               \section[Hermann Bahr an Arthur Schnitzler, 27. 9. 1906]{ Hermann Bahr an Arthur Schnitzler, 27. 9. 1906}\nopagebreak\mylabel{v}\rehead{ }\normalsize\beginnumbering\briefempfaengerindex{Schnitzler, Arthur@\textsc{Schnitzler, Arthur}!zzzBahr, Hermann@\emph{von Hermann Bahr}!1906-09-271@{27. 9. 1906}|(be} \toendnotes[C]{\smallbreak\pagebreak[2]} \Standort{CUL, Schnitzler, B 5b.}
\physDesc{Brief, 1 Blatt, 2 Seiten
\newline{}Handschrift Lisa Clarus: blaue Tinte, lateinische Kurrent\newline{}Handschrift Hermann Bahr: blaue Tinte\newline{}Ordnung: mit Bleistift von unbekannter Hand nummeriert:
                                    »141« }\buchAbdrucke{\weitereDrucke{Hermann Bahr, Arthur Schnitzler: \emph{Briefwechsel, Aufzeichnungen, Dokumente (1891–1931)}. Hg. Kurt Ifkovits und Martin Anton Müller. Göttingen: \emph{Wallstein} 2018, S. 381–382.} }\toendnotes[C]{\smallbreak}\pstart
           \raggedleft{}{\pb}27. 9. 06.\pend
           \pstart\center{}Lieber Arthur!\pend\pstart
           Verzeihe, dass ich dictiere, aber mich macht das Mechanische des Schreibens
               schrecklich nervös.\pend
           \pstart
           Ich bleibe bis zum 1. November noch in \textcolor{pink}{Wien}{}\ledrightnote{\textcolor{pink}{Wien}} und
               möchte nun sehr gern Ende der nächsten Woche, oder Anfang der übernächsten Woche
               einmal Vormittag zu Dir kommen. Vielleicht bestimmst Du mir einen Tag, der Dir
               passt.\pend
           \pstart
           Und noch etwas: Du hast einen \textcolor{pink}{russischen}{}\ledrightnote{\textcolor{pink}{Russland}}{ }{\pb}\textcolor{blue}{Uebersetzer}{}\ledrightnote{→\textcolor{blue}{Peter Rotenstern}}, der sich auch
               einmal an mich gewendet hat, ich habe aber seinen Namen und seine Adresse vergessen.
               Kannst Du mir diese schreiben?\pend
           \pstart
           Mit vielen Grüssen an Frau \textcolor{blue}{Olga}{}\ledrightnote{\textcolor{blue}{Olga Schnitzler}}{\\[\baselineskip]}herzlichst{\\[\baselineskip]}Dein{\\[\baselineskip]}\spacefill\mbox{{[}hs. Bahr:{]} HermannBahr}\pend
           \leftskip=0em{}\endnumbering\briefempfaengerindex{Schnitzler, Arthur@\textsc{Schnitzler, Arthur}!zzzBahr, Hermann@\emph{von Hermann Bahr}!1906-09-271@{27. 9. 1906}|)be}\mylabel{h}  \normalsize

\doendnotes{C}
\bigskip
\vfill

\clearpage

\footnotesize

\lohead{\textsc{register}}

% Definiere theindex-Environment komplett neu ohne reledmac
\makeatletter
\renewenvironment{theindex}{%
  \section*{\indexname}%
  \setlength{\parindent}{0pt}%
  \setlength{\parskip}{0pt plus 0.3pt}%
  \let\item\@idxitem
}{%
  \clearpage
}
\makeatother

\IfFileExists{\jobname-pw.ind}{\input{\jobname-pw.ind}}{}

\end{document}

      