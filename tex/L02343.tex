%% latex-korrekturansicht-vorspann.tex
%% Vorspann für die Korrekturansicht.
%% Lädt die gemeinsame Datei latex-vorspann.tex mit gesetztem Schalter.

\newif\ifkorrekturansicht
\korrekturansichttrue

\input{../tex-inputs/latex-vorspann}


               \section[Robert Adam an Arthur Schnitzler, 17. 6. 1920]{ Robert Adam an Arthur Schnitzler, 17. 6. 1920}\nopagebreak\mylabel{v}\rehead{ }\normalsize\beginnumbering\briefempfaengerindex{Schnitzler, Arthur@\textsc{Schnitzler, Arthur}!zzzAdam, Robert@\emph{von Robert Adam}!1920-06-171@{17. 6. 1920}|(be} \toendnotes[C]{\smallbreak\pagebreak[2]} \Standort{CUL, Schnitzler, B 1.}
\physDesc{Brief, 1 Blatt, 3 Seiten
\newline{}Handschrift: blaue Tinte, deutsche Kurrent
\newline{}Schnitzler: 1) mit Bleistift beschriftet: »\textsc{Adam}« 2) mit rotem Buntstift mehrere Unterstreichungen\newline{}Ordnung: mit Bleistift von unbekannter Hand nummeriert:
                                        »15« }\Standort{Wien, Österreichische Nationalbibliothek, Cod.ser. 52.268, 74 recto und 73 recto.}
\physDesc{Brief, maschinelle Abschrift
\newline{}Schreibmaschine}\toendnotes[C]{\smallbreak}\pstart
           \raggedleft{}{\pb}\textcolor{pink}{Wien}{}\ledrightnote{\textcolor{pink}{Wien}}, am 17. Juni 1920\pend
           \pstart\center{}Hochverehrter Herr Doktor!\pend\pstart
           Beſten Dank für Ihre Karte! Daß Sie ſich mit der Lektüre meines \textcolor{green}{Aufſatzes}{}\ledrightnote{→\textcolor{green}{Über Rechtsprinzipien. Eine analytische Untersuchung}} plagen, darf ich gar nicht
                    verlangen!\pend
           \pstart
           Ich habe meinem Magenleiden, das mich ſeit mehr als einem Jahre quälte und faſt
                    arbeitsunfähig, jedenfalls aber lebensunluſtig machte, endlich dadurch ein Ende
                    gemacht, daß ich mich – Mitte Mai – operieren ließ. Ich bin noch
                    immer ſehr ſchwach, gehe aber doch ſchon aus und würde ſehr gerne {\pb}im Laufe der nächſten Woche – den
                        26. muß ich ausnehmen – zu Ihnen kommen; bitte mir einen Tag zu
                    beſtimmen.\pend
           \pstart
           Am 3. Juli fahre ich mit \textcolor{blue}{Frau}{}\ledrightnote{→\textcolor{blue}{Maria Pollak}} und \textcolor{blue}{Kind}{}\ledrightnote{→\textcolor{blue}{Viktor Franz Patzner}} nach \textcolor{pink}{Gutenſtein}{}\ledrightnote{\textcolor{pink}{Gutenstein}}, wo uns ein von den \textcolor{pink}{Schweden}{}\ledrightnote{\textcolor{pink}{Schweden}} beliefertes Richtererholungsheim, das den verſprechenden
                    Namen: »\textcolor{pink}{Heim der Ruhe}{}\ledrightnote{\textcolor{pink}{Erholungsheim der Bundesbeamten}}« führt, für wenig Geld
                    durch 4 Wochen verpflegen ſoll. Was dann geſchieht, hängt davon ab, ob ich mich
                        anfangs Auguſt bereits zur Wiederaufnahme des Dienſtes ſtark
                    genug fühlen werde oder noch irgendwo Erholungsmöglichkeit ſuchen muß.\pend
           \pstart
           Gearbeitet habe ich \strikeout{die}{ }ſeit dem Herbſt gar nichts, aber viel Lehrreiches
                    geleſen, vor allem vieles Lateiniſche.\pend
           \pstart
           {\pb}Mit den ergebenſten Grüßen\pend
           \pstart
           Ihr{\\[\baselineskip]}\spacefill\mbox{D\textsuperscript{r}RAdam}\pend
           \leftskip=0em{}\endnumbering\briefempfaengerindex{Schnitzler, Arthur@\textsc{Schnitzler, Arthur}!zzzAdam, Robert@\emph{von Robert Adam}!1920-06-171@{17. 6. 1920}|)be}\mylabel{h}  \normalsize

\doendnotes{C}
\bigskip
\vfill

\clearpage

\footnotesize

\lohead{\textsc{register}}

% Definiere theindex-Environment komplett neu ohne reledmac
\makeatletter
\renewenvironment{theindex}{%
  \section*{\indexname}%
  \setlength{\parindent}{0pt}%
  \setlength{\parskip}{0pt plus 0.3pt}%
  \let\item\@idxitem
}{%
  \clearpage
}
\makeatother

\IfFileExists{\jobname-pw.ind}{\input{\jobname-pw.ind}}{}

\end{document}

      