%% latex-korrekturansicht-vorspann.tex
%% Vorspann für die Korrekturansicht.
%% Lädt die gemeinsame Datei latex-vorspann.tex mit gesetztem Schalter.

\newif\ifkorrekturansicht
\korrekturansichttrue

\input{../tex-inputs/latex-vorspann}


               \section[Robert Adam an Arthur Schnitzler, 2. 12. 1918]{ Robert Adam an Arthur Schnitzler, 2. 12. 1918}\nopagebreak\mylabel{v}\rehead{ }\normalsize\beginnumbering\briefempfaengerindex{Schnitzler, Arthur@\textsc{Schnitzler, Arthur}!zzzAdam, Robert@\emph{von Robert Adam}!1918-12-021@{2. 12. 1918}|(be} \toendnotes[C]{\smallbreak\pagebreak[2]} \Standort{CUL, Schnitzler, B 1.}
\physDesc{Brief, 1 Blatt, 3 Seiten
\newline{}Handschrift: schwarze Tinte, deutsche Kurrent
\newline{}Schnitzler: 1) mit Bleistift beschriftet: »\textsc{Adam}« 2) mit rotem Buntstift zwei Unterstreichungen\newline{}Ordnung: von unbekannter Hand nummeriert: »10« }\Standort{Wien, Österreichische Nationalbibliothek, Cod. ser. 52.263.}
\physDesc{Brief, 1 Blatt, 4 Seiten, Entwurf
\newline{}Handschrift: schwarze Tinte, deutsche Kurrent\newline{}Zusatz: Entwurf des Briefes, datiert auf den 1. 12. 1918
                                 und mit leichten sprachlichen Variationen }\Standort{Wien, Österreichische Nationalbibliothek, Cod.ser. 52.269, 225 verso.}
\physDesc{Brief, maschinelle Abschrift
\newline{}Schreibmaschine}\toendnotes[C]{\smallbreak}\pstart
           \raggedleft{}{\pb}\textcolor{pink}{Wien}{}\ledrightnote{\textcolor{pink}{Wien}}, am 2. Dezember 1918\pend
           \pstart\center{}Hochverehrter Herr Doktor!\pend\pstart
           Verzeihen Sie es meiner bangen Ungeduld, daß ich, obwohl nicht viel mehr als zwei
               Wochen verſtrichen ſind, ſeit ich dem \textcolor{pink}{Deutſchen
                  Volkstheater}{}\ledrightnote{\textcolor{pink}{Volkstheater}} meine zwei \textcolor{green}{Stücke}{}\ledrightnote{→\textcolor{green}{Yppl. Idylle in fünf Akten}{\newline}→\textcolor{green}{Der Fremde}} überreichte, bei Ihnen anfrage, ob Ihnen von dem Schickſal, das ihrer
               harrt, ſchon etwas bekannt geworden iſt? Ich bin ohne jede Nachricht und weiß nicht
               recht, ob ich wieder im \textcolor{pink}{Theater}{}\ledrightnote{→\textcolor{pink}{Volkstheater}}
               vorſprechen ſoll und an wen ich mich am beſten wenden ſollte; ich beſorge, mir durch
               Zudringlichkeit und Zurſchautragen von Ungeduld Chancen, die ich etwa hätte, zu
               verderben, anderſeits aber wieder, ſtilles Zuwarten möchte auch nicht das {\pb}richtige Vorgehen ſein. Könnten Sie mir,
               bitte, hierin einen Rat geben?\pend
           \pstart
           Mir hilft jetzt über viele Unannehmlichkeiten der \textcolor{pink}{deutſchöſterreichiſchen}{}\ledrightnote{\textcolor{pink}{Österreich}} Epoche – Amtsarbeit, Verkühlung, Fett- und
               Fleiſchhunger, kühle Zimmer – die Lektüre eines wundervollen Buches hinweg, das ich
               neulich in der \textcolor{brown}{Bibliothek der Juſtizbeamten}{}\ledrightnote{\textcolor{brown}{Privatbibliothek der Wiener Justizbeamten}}
               aufſtöberte und das mir bis jetzt vollkommen unbekannt war (obwohl es in den
                     80\textsuperscript{er} Jahren einiges Aufſehen erregt
               haben muß). Es heißt: »\textcolor{green}{Briefe eines Unbekannten}{}\ledrightnote{\textcolor{green}{Briefe eines Unbekannten}}«
               und wurde von dem Grafen \textcolor{blue}{Rudolf \textsc{Hoyos}}{}\ledrightnote{\textcolor{blue}{Rudolf von Hoyos}} bei \textcolor{brown}{Gerold}{}\ledrightnote{\textcolor{brown}{Carl Gerold’s Sohn}} in \textcolor{pink}{Wien}{}\ledrightnote{\textcolor{pink}{Wien}} herausgegeben, 1887 in zweiter Auflage. Der Briefſchreiber
               war ein Herr \textcolor{blue}{von \textsc{Villers}}{}\ledrightnote{\textcolor{blue}{Alexander von Villers}}, penſionierter \textcolor{pink}{ſächſiſcher}{}\ledrightnote{\textcolor{pink}{Sachsen}} Legationsrat, ein
               Mann von höchſter Kultur. Wie konnte es kommen, daß ich von dieſem Buch nie etwas las
               oder hörte? Es gehört, will mich dünken, nicht nur zu den vornehmſten, ſondern zu den
               geiſtvollſten und liebenswürdigſten Büchern der deutſchen {\pb}Literatur. Ich muß mich zurückhalten,
               Ihnen nicht Stücke auszuſchreiben, um Ihnen davon – falls Sie dieſe Briefe nicht
               ohnehin kennen ſollten – Proben zu geben; aber vielleicht kennen Sie, was ich
               entdeckt oder wiederentdeckt zu haben glaubte, ohnehin und meine Begeiſterung ſcheint
               Ihnen zwar nicht lächerlich – denn ich glaube kaum, daß ein für Literatur
               Empfänglicher dieſen Briefen gegenüber kalt bleiben könnte –, aber doch unnütz. –\pend
           \pstart
           Zu ſchriftſtelleriſcher Betätigung komme ich jetzt gar nicht; mir iſt, als müßte ich
               alle mir nach viereinhalb Kriegsjahren verbliebene Energie dazu aufbrauchen, nicht
               allzuſehr zu frieren, und als bliebe für’s Denken keine mehr übrig.\pend
           \pstart
           Mit den ergebenſten Grüßen{\\[\baselineskip]}Ihr{\\[\baselineskip]}\spacefill\mbox{D\textsuperscript{r}RAdam}\pend
           \leftskip=0em{}\endnumbering\briefempfaengerindex{Schnitzler, Arthur@\textsc{Schnitzler, Arthur}!zzzAdam, Robert@\emph{von Robert Adam}!1918-12-021@{2. 12. 1918}|)be}\mylabel{h}  \normalsize

\doendnotes{C}
\bigskip
\vfill

\clearpage

\footnotesize

\lohead{\textsc{register}}

% Definiere theindex-Environment komplett neu ohne reledmac
\makeatletter
\renewenvironment{theindex}{%
  \section*{\indexname}%
  \setlength{\parindent}{0pt}%
  \setlength{\parskip}{0pt plus 0.3pt}%
  \let\item\@idxitem
}{%
  \clearpage
}
\makeatother

\IfFileExists{\jobname-pw.ind}{\input{\jobname-pw.ind}}{}

\end{document}

      