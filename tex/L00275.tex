%% latex-korrekturansicht-vorspann.tex
%% Vorspann für die Korrekturansicht.
%% Lädt die gemeinsame Datei latex-vorspann.tex mit gesetztem Schalter.

\newif\ifkorrekturansicht
\korrekturansichttrue

\input{../tex-inputs/latex-vorspann}


               \section[Richard Beer-Hofmann an Arthur Schnitzler, {[}24?. 10. 1893{]}]{ Richard Beer-Hofmann an Arthur Schnitzler, {[}24?. 10. 1893{]}}\nopagebreak\mylabel{v}\rehead{ }\normalsize\beginnumbering\briefempfaengerindex{Schnitzler, Arthur@\textsc{Schnitzler, Arthur}!zzzBeer-Hofmann, Richard@\emph{von Richard Beer-Hofmann}!1893-10-242@{{[}24?. 10. 1893{]}}|(be} \toendnotes[C]{\smallbreak\pagebreak[2]} \Standort{CUL, Schnitzler, B 8.}
\physDesc{Briefkarte
\newline{}Handschrift: Bleistift, lateinische Kurrent
\newline{}Schnitzler: mit Bleistift datiert: »2\substVorne{}\textsuperscript{\textcolor{gray}{7}}\substDazwischen{}5\substHinten{}/X 93« \newline{}Ordnung: mit Bleistift von unbekannter Hand nummeriert: »24« }\buchAbdrucke{\weitereDrucke{Arthur Schnitzler, Richard Beer-Hofmann: \emph{Briefwechsel 1891–1931}. Hg. Konstanze Fliedl. Wien, Zürich: \emph{Europaverlag} 1992, S. 53.} }\toendnotes[C]{\smallbreak}\pstart
           \noindent{}{\pb}Lieber! \textcolor{blue}{Meixner}{}\ledrightnote{\textcolor{blue}{Julius Meixner}} nahm mich heute beiseite, hat Bedenken ob er den
                  \textcolor{green}{\uline{Wandel}}{}\ledrightnote{→\textcolor{green}{Das Märchen. Schauspiel in drei Aufzügen}} treffen
               wird; habe ihn ihm erklärt; kennt das Stück nicht; bringen Sie bitte morgen \label{K_L00275_1v}\edtext{Mittwoch}{\lemma{\textnormal{\emph{Mittwoch}}}\Cendnote{\textnormal{Es ist anzunehmen, dass \textcolor{blue}{Schnitzler}s Datierung
                  den Empfangstag bezeichnet, da der 25. 10. 1893 ein Mittwoch war. Das Korrespondenzstück
                  stammt demgemäß vom Vortag.}}}\label{K_L00275_1h} ins Caffée ein \uline{gekürztes} Exemplar {\pb}des \textcolor{green}{Märchen}{}\ledrightnote{\textcolor{green}{Das Märchen. Schauspiel in drei Aufzügen}} mit. Aber vor 7 Uhr.\pend
           \pstart
           Das \textcolor{green}{Märchen}{}\ledrightnote{\textcolor{green}{Das Märchen. Schauspiel in drei Aufzügen}} ist \uuline{sehr} gut; ich habe es wieder gelesen – ich glaube jetzt sogar an einen
               Bühnenerfolg. Herzlichst\pend
           \pstart \spacefill\mbox{Richard}\pend{}\endnumbering\briefempfaengerindex{Schnitzler, Arthur@\textsc{Schnitzler, Arthur}!zzzBeer-Hofmann, Richard@\emph{von Richard Beer-Hofmann}!1893-10-242@{{[}24?. 10. 1893{]}}|)be}\mylabel{h}  \normalsize

\doendnotes{C}
\bigskip
\vfill

\clearpage

\footnotesize

\lohead{\textsc{register}}

% Definiere theindex-Environment komplett neu ohne reledmac
\makeatletter
\renewenvironment{theindex}{%
  \section*{\indexname}%
  \setlength{\parindent}{0pt}%
  \setlength{\parskip}{0pt plus 0.3pt}%
  \let\item\@idxitem
}{%
  \clearpage
}
\makeatother

\IfFileExists{\jobname-pw.ind}{\input{\jobname-pw.ind}}{}

\end{document}

      