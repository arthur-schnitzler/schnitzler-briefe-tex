%% latex-korrekturansicht-vorspann.tex
%% Vorspann für die Korrekturansicht.
%% Lädt die gemeinsame Datei latex-vorspann.tex mit gesetztem Schalter.

\newif\ifkorrekturansicht
\korrekturansichttrue

\input{../tex-inputs/latex-vorspann}


               \section[Richard Dehmel an Arthur Schnitzler, 14. 2. 1902]{ Richard Dehmel an Arthur Schnitzler, 14. 2. 1902}\nopagebreak\mylabel{v}\rehead{ }\normalsize\beginnumbering\briefempfaengerindex{Schnitzler, Arthur@\textsc{Schnitzler, Arthur}!zzzDehmel, Richard@\emph{von Richard Dehmel}!1902-02-141@{14. 2. 1902}|(be} \toendnotes[C]{\smallbreak\pagebreak[2]} \Standort{CUL, Schnitzler, B 26.}
\physDesc{Postkarte
\newline{}Handschrift: schwarze Tinte, lateinische Kurrent\newline{}Versand: 1) Stempel: »\nobreak{}\oindex{Blankenese@\textbf{Blankenese}, \emph{Bezirk (A.BZK)}|pwk}Blankenese, 14. 2. 02, 4–5N\nobreak{}«.  2) Stempel: »\nobreak{}\oindex{IX., Alsergrund@\textbf{IX., Alsergrund}, \emph{Bezirk (A.BZK)}|pwk}9/3 Wien 72, 16. 2. 02, 9.V, Bestellt\nobreak{}«. 
\newline{}Schnitzler: mit rotem Buntstift eine Unterstreichung }\pstart{}{\pb}Herrn Dr. Arthur
                  Schnitzler.\pend{}\pstart{}\textcolor{pink}{Wien IX.}{}\ledrightnote{\textcolor{pink}{Wien}}\pend{}\pstart{}\textcolor{pink}{Frankgasse 1}{}\ledrightnote{\textcolor{pink}{Frankgasse}}.\pend{}{\bigskip}\pstart{}{\pb}Verehrter Herr Schnitzler!
               \pend\pstart
           An meinem »\textcolor{green}{Schleier der Beatrice}{}\ledrightnote{\textcolor{green}{Der Schleier der Beatrice. Schauspiel in fünf Akten}}« fehlt ein
               Stückchen. Grade die letzten Worte der beiden Schlußzeilen, also je das letzte Wort,
               sind im Druck nicht gekommen (»abgesprungen«). Möchten Sie wol die Güte haben, sie
               mir schriftlich mitzuteilen! Im übrigen brauche ich Ihnen wol kaum zu sagen, daß ich
               die Dichtung mit größter Freude gelesen habe.\pend
           \pstart
           Dankbar grüßend{\\[\baselineskip]}\spacefill\mbox{R. Dehmel.}\pend
           \leftskip=0em{}\pstart
           \noindent{}\textcolor{pink}{Blankenese b/Hamburg}{}\ledrightnote{\textcolor{pink}{Blankenese}}.\pend
           \endnumbering\briefempfaengerindex{Schnitzler, Arthur@\textsc{Schnitzler, Arthur}!zzzDehmel, Richard@\emph{von Richard Dehmel}!1902-02-141@{14. 2. 1902}|)be}\mylabel{h}  \normalsize

\doendnotes{C}
\bigskip
\vfill

\clearpage

\footnotesize

\lohead{\textsc{register}}

% Definiere theindex-Environment komplett neu ohne reledmac
\makeatletter
\renewenvironment{theindex}{%
  \section*{\indexname}%
  \setlength{\parindent}{0pt}%
  \setlength{\parskip}{0pt plus 0.3pt}%
  \let\item\@idxitem
}{%
  \clearpage
}
\makeatother

\IfFileExists{\jobname-pw.ind}{\input{\jobname-pw.ind}}{}

\end{document}

      