%% latex-korrekturansicht-vorspann.tex
%% Vorspann für die Korrekturansicht.
%% Lädt die gemeinsame Datei latex-vorspann.tex mit gesetztem Schalter.

\newif\ifkorrekturansicht
\korrekturansichttrue

\input{../tex-inputs/latex-vorspann}


               \section[Arthur und Olga Schnitzler an Richard Beer-Hofmann, 6. 7. 1909]{ Arthur und Olga Schnitzler an Richard Beer-Hofmann, 6. 7. 1909}\nopagebreak\mylabel{v}\rehead{ }\normalsize\beginnumbering\briefempfaengerindex{Beer-Hofmann, Richard@\textsc{Beer-Hofmann, Richard}!zzzSchnitzler, Olga@\emph{von Olga Schnitzler}!1909-07-061@{6. 7. 1909}|(be}\briefempfaengerindex{Beer-Hofmann, Richard@\textsc{Beer-Hofmann, Richard}!zzzSchnitzler, Arthur@\emph{von Arthur Schnitzler}!1909-07-061@{6. 7. 1909}|(be} \toendnotes[C]{\smallbreak\pagebreak[2]} \Standort{YCGL, MSS 31.}
\physDesc{Bildpostkarte
\newline{}Handschrift Arthur Schnitzler: Bleistift, deutsche Kurrent\newline{}Handschrift Olga Schnitzler: Bleistift, lateinische Kurrent\newline{}Versand: Stempel: »\nobreak{}7 7 09, 8–12V\nobreak{}«.  
\newline{}Beer-Hofmann: mit blauem Bunstift das Datum der Beantwortung
                                    festgehalten: »B 12/VII 09« }\buchAbdrucke{\weitereDrucke{Arthur Schnitzler, Richard Beer-Hofmann: \emph{Briefwechsel 1891–1931}. Hg. Konstanze Fliedl. Wien, Zürich: \emph{Europaverlag} 1992, S. 193.} }\toendnotes[C]{\smallbreak}\pstart{}{\pb}\textsc{Herrn Dr. Richard}\pend{}\pstart{}\textsc{Beer-Hofmann}\pend{}\pstart{}\textsc{\textcolor{pink}{Pichl Auhof{\\}am Mondsee}{}\ledrightnote{\textcolor{pink}{Hotel Pichl-Auhof}}}\pend{}{\bigskip}\pstart
           \noindent{}\centering{}{\pb}\textcolor{gray}{\textbf{\textcolor{pink}{Edlach bei Reichenau in N.-Oe.}{}\ledrightnote{\textcolor{pink}{Edlach}}, 593 m Seehöhe
                     mit \textcolor{pink}{Schneeberg}{}\ledrightnote{\textcolor{pink}{Schneeberg}}.}}\pend
           \pstart
           {\pb}lieber Richard, ſehr ſchön hier – und wir würden uns ganz wohl
               fühlen, we{\geminationn} wir den \textcolor{blue}{Buben}{}\ledrightnote{→\textcolor{blue}{Heinrich Schnitzler}}{ }ſchon heraußen hätten (den ich heute u geſtern in
                  \textcolor{pink}{Wien}{}\ledrightnote{\textcolor{pink}{Wien}} beſucht habe).\pend
           \pstart
           Wie behagen Sie ſich? Ko{\geminationm}en Sie doch nachher auch hier
               her, ich glaube es gefiel Ihnen Allen.\pend
           \pstart Herzlichſt Ihr \spacefill\mbox{Arthur}\pend{}\pstart
           6/7 09\pend
           \pstart
           \noindent{}{[}hs. O. Schnitzler:{]} Ja, das wäre wunderschön! An die \textcolor{blue}{Paula}{}\ledrightnote{\textcolor{blue}{Paula Beer-Hofmann}} will ich nächstens einen längern Brief schreiben.
               Hoffentlich geht es Euch allen sehr gut.\pend
           \pstart Herzliche Grüsse! \spacefill\mbox{Olga.}\pend{}\endnumbering\briefempfaengerindex{Beer-Hofmann, Richard@\textsc{Beer-Hofmann, Richard}!zzzSchnitzler, Olga@\emph{von Olga Schnitzler}!1909-07-061@{6. 7. 1909}|)be}\briefempfaengerindex{Beer-Hofmann, Richard@\textsc{Beer-Hofmann, Richard}!zzzSchnitzler, Arthur@\emph{von Arthur Schnitzler}!1909-07-061@{6. 7. 1909}|)be}\mylabel{h}  \normalsize

\doendnotes{C}
\bigskip
\vfill

\clearpage

\footnotesize

\lohead{\textsc{register}}

% Definiere theindex-Environment komplett neu ohne reledmac
\makeatletter
\renewenvironment{theindex}{%
  \section*{\indexname}%
  \setlength{\parindent}{0pt}%
  \setlength{\parskip}{0pt plus 0.3pt}%
  \let\item\@idxitem
}{%
  \clearpage
}
\makeatother

\IfFileExists{\jobname-pw.ind}{\input{\jobname-pw.ind}}{}

\end{document}

      