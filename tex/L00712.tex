%% latex-korrekturansicht-vorspann.tex
%% Vorspann für die Korrekturansicht.
%% Lädt die gemeinsame Datei latex-vorspann.tex mit gesetztem Schalter.

\newif\ifkorrekturansicht
\korrekturansichttrue

\input{../tex-inputs/latex-vorspann}


               \section[Arthur Schnitzler an Richard Beer-Hofmann, 4. 8. 1897]{ Arthur Schnitzler an Richard Beer-Hofmann, 4. 8. 1897}\nopagebreak\mylabel{v}\rehead{ }\normalsize\beginnumbering\briefempfaengerindex{Beer-Hofmann, Richard@\textsc{Beer-Hofmann, Richard}!zzzSchnitzler, Arthur@\emph{von Arthur Schnitzler}!1897-08-041@{4. 8. 1897}|(be} \toendnotes[C]{\smallbreak\pagebreak[2]} \Standort{YCGL, MSS 31.}
\physDesc{Brief, 1 Blatt, 3 Seiten, Umschlag
\newline{}Handschrift: Bleistift, deutsche Kurrent\newline{}Versand: 1) Stempel: »\nobreak{}\oindex{IX., Alsergrund@\textbf{IX., Alsergrund}, \emph{Bezirk (A.BZK)}|pwk}Wien 9/3, 4. 8. 97, 5–6N\nobreak{}«.  2) Stempel: »\nobreak{}\oindex{Bad Ischl@\textbf{Bad Ischl}, \emph{Besiedelter Ort (A.BSO)}|pwk}Ischl, 6. 8. 97, 1–2N\nobreak{}«. }\buchAbdrucke{\weitereDrucke{Arthur Schnitzler, Richard Beer-Hofmann: \emph{Briefwechsel 1891–1931}. Hg. Konstanze Fliedl. Wien, Zürich: \emph{Europaverlag} 1992, S. 112.} }\toendnotes[C]{\smallbreak}\pstart{}{\pb}Herrn \textsc{Dr. Richard
                     Beer-Hofmann }\pend{}\pstart{}\textsc{\textcolor{pink}{Ischl}{}\ledrightnote{\textcolor{pink}{Bad Ischl}}}\pend{}\pstart{}\textcolor{pink}{\textsc{Egelmoos 22}}{}\ledrightnote{\textcolor{pink}{Eglmoosgasse}}.\pend{}{\bigskip}\pstart{}{\pb}Lieber Richard.\pend\pstart
           Thun Sie mir einen großen Gefallen.\pend
           \pstart
           Frau \textcolor{blue}{F.}{}\ledrightnote{\textcolor{blue}{Rosa Freudenthal}} iſt wieder in \textcolor{pink}{Iſchl}{}\ledrightnote{\textcolor{pink}{Bad Ischl}}; heute erhielt ich einen Brief von ihr, ich möge ihr \uline{durch Sie} Briefe u Bilder zurückſchicken, in \textcolor{pink}{Wien}{}\ledrightnote{\textcolor{pink}{Wien}} erhalte ich die Erklärung. – Gehn Sie zu {\pb}\textcolor{pink}{Petter}{}\ledrightnote{\textcolor{pink}{Hotel und Pension Rudolfshöhe (Leopold Petter)}}, ſie ist
                  \label{K_L00712_1v}\edtext{\textsc{en
                  fam.}}{\lemma{\textnormal{\emph{en
                  fam.}}}\Cendnote{\textnormal{französisch en famille: mit ihrer
                  Familie}}}\label{K_L00712_1h} dort, Sie werden ſie aber leicht allein ſprechen können. Sagen Sie
               ihr, ich käme bald ſelbſt nach \textcolor{pink}{Iſchl}{}\ledrightnote{\textcolor{pink}{Bad Ischl}} und erfülle
               lieber perſönlich ihren Wunſch, ſie kö{\geminationn}e ſicher darauf
               rechnen. {\pb}Bringen Sie aber heraus was dahinter
               ſteckt, ich ärgere mich mehr als die Geſchichte werth iſt. Antworten Sie mir gleich,
               am liebſten telegrafiſch.\pend
           \pstart
           Herzlich Ihr{\\[\baselineskip]}\spacefill\mbox{Arthur}\pend
           \leftskip=0em{}\endnumbering\briefempfaengerindex{Beer-Hofmann, Richard@\textsc{Beer-Hofmann, Richard}!zzzSchnitzler, Arthur@\emph{von Arthur Schnitzler}!1897-08-041@{4. 8. 1897}|)be}\mylabel{h}  \normalsize

\doendnotes{C}
\bigskip
\vfill

\clearpage

\footnotesize

\lohead{\textsc{register}}

% Definiere theindex-Environment komplett neu ohne reledmac
\makeatletter
\renewenvironment{theindex}{%
  \section*{\indexname}%
  \setlength{\parindent}{0pt}%
  \setlength{\parskip}{0pt plus 0.3pt}%
  \let\item\@idxitem
}{%
  \clearpage
}
\makeatother

\IfFileExists{\jobname-pw.ind}{\input{\jobname-pw.ind}}{}

\end{document}

      