%% latex-korrekturansicht-vorspann.tex
%% Vorspann für die Korrekturansicht.
%% Lädt die gemeinsame Datei latex-vorspann.tex mit gesetztem Schalter.

\newif\ifkorrekturansicht
\korrekturansichttrue

\input{../tex-inputs/latex-vorspann}


               \section[Friedrich M. Fels an Arthur Schnitzler, {[}29. 4. 1893?{]}]{ Friedrich M. Fels an Arthur Schnitzler, {[}29. 4. 1893?{]}}\nopagebreak\mylabel{v}\rehead{ }\normalsize\beginnumbering\briefempfaengerindex{Schnitzler, Arthur@\textsc{Schnitzler, Arthur}!zzzFels, Friedrich Michael@\emph{von Friedrich Michael Fels}!1893-04-291@{{[}29. 4. 1893?{]}}|(be} \toendnotes[C]{\smallbreak\pagebreak[2]} \Standort{DLA, A:Schnitzler, HS.NZ85.1.2956.}
\physDesc{Telegramm
\newline{}maschinell
\newline{}Schnitzler: mit Bleistift nummeriert: »11« \newline{}Ordnung: beschnitten }\toendnotes[C]{\smallbreak}\pstart
           {\pb}\textcolor{pink}{win}{}\ledrightnote{\textcolor{pink}{Wien}}{ }\textcolor{pink}{obermais}{}\ledrightnote{\textcolor{pink}{Obermais}}{ }658{ }20{ }11 20\pend
           \pstart
           bitte um 25 gulden damit ich wenigstens abreisen kann mit \textcolor{blue}{wirt}{}\ledrightnote{→\textcolor{blue}{Josef Drassl}} ist verglejch geschlossen \pend
           \pstart \spacefill\mbox{= fels =}\pend{}\endnumbering\briefempfaengerindex{Schnitzler, Arthur@\textsc{Schnitzler, Arthur}!zzzFels, Friedrich Michael@\emph{von Friedrich Michael Fels}!1893-04-291@{{[}29. 4. 1893?{]}}|)be}\mylabel{h}  \normalsize

\doendnotes{C}
\bigskip
\vfill

\clearpage

\footnotesize

\lohead{\textsc{register}}

% Definiere theindex-Environment komplett neu ohne reledmac
\makeatletter
\renewenvironment{theindex}{%
  \section*{\indexname}%
  \setlength{\parindent}{0pt}%
  \setlength{\parskip}{0pt plus 0.3pt}%
  \let\item\@idxitem
}{%
  \clearpage
}
\makeatother

\IfFileExists{\jobname-pw.ind}{\input{\jobname-pw.ind}}{}

\end{document}

      