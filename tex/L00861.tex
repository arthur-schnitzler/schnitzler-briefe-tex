%% latex-korrekturansicht-vorspann.tex
%% Vorspann für die Korrekturansicht.
%% Lädt die gemeinsame Datei latex-vorspann.tex mit gesetztem Schalter.

\newif\ifkorrekturansicht
\korrekturansichttrue

\input{../tex-inputs/latex-vorspann}


               \section[Peter Altenberg an Arthur Schnitzler, 30. 11. 1898]{ Peter Altenberg an Arthur Schnitzler, 30. 11. 1898}\nopagebreak\mylabel{v}\rehead{ }\normalsize\beginnumbering\briefempfaengerindex{Schnitzler, Arthur@\textsc{Schnitzler, Arthur}!zzzAltenberg, Peter@\emph{von Peter Altenberg}!1898-11-301@{30. 11. 1898}|(be} \toendnotes[C]{\smallbreak\pagebreak[2]} \Standort{CUL, Schnitzler, B 2.}
\physDesc{Brief, 1 Blatt, 2 Seiten
\newline{}Handschrift: schwarze Tinte, deutsche Kurrent
\newline{}Schnitzler: mit rotem Buntstift eine Unterstreichung \newline{}Ordnung: mit Bleistift von unbekannter Hand nummeriert:
                                 »7« }\toendnotes[C]{\smallbreak}\pstart{}{\pb}Lieber \textsc{D\textsuperscript{r.}} Arthur Schnitzler:\pend\pstart
           Mit beſonderem Vergnügen ergreife ich die Gelegenheit, Ihnen etwas Angenehmes,
               Freundliches zu ſagen. Ihr \textcolor{green}{Stück}{}\ledrightnote{→\textcolor{green}{Das Vermächtnis. Schauspiel in drei Akten}}
               hat mir ganz \label{K_L00861_1v}\edtext{außerordentlich
                  gefallen}{\lemma{\textnormal{\emph{außerordentlich
                  gefallen}}}\Cendnote{\textnormal{\emph{\textcolor{green}{Das Vermächtnis}} wurde am
                     30. 11. 1898 zum ersten Mal am \textcolor{pink}{Burgtheater} gegeben, das Schreiben \textcolor{blue}{Altenberg}s dürfte also nach Ende der Vorstellung (gegen 21 Uhr 30) verfasst
                  sein.}}}\label{K_L00861_1h} und habe ich im \textcolor{pink}{Theater}{}\ledrightnote{→\textcolor{pink}{Burgtheater}}{ }ſelbſt dieſer Empfindung in zügelloſer Weiſe
               Ausdruck gegeben. Dieſe Geſtalt des Profeſſors \textcolor{green}{Loſati}{}\ledrightnote{→\textcolor{green}{Das Vermächtnis. Schauspiel in drei Akten}}, noch dazu von \textcolor{blue}{Hartmann}{}\ledrightnote{\textcolor{blue}{Ernst Hartmann}} in dieſer letzten Vollkommenheit lebendig gemacht, iſt wirklich
               wunderbar ausgeführt.\pend
           \pstart
           {\pb}Ich hätte entſchieden dieſes Stück
               betitelt: »\textcolor{green}{\uline{Profeſſor Loſati}}{}\ledrightnote{→\textcolor{green}{Das Vermächtnis. Schauspiel in drei Akten}}«. Der \textcolor{green}{3. Akt}{}\ledrightnote{→\textcolor{green}{Das Vermächtnis. Schauspiel in drei Akten}} mit den
               Karakteren des Profeſſors u. ſeiner Tochter iſt meiſterhaft.\pend
           \pstart
           Ich war ganz hingeriſſen.\pend
           \pstart
           Es iſt entſchieden Ihre kraftvollſte \textcolor{green}{Arbeit}{}\ledrightnote{→\textcolor{green}{Das Vermächtnis. Schauspiel in drei Akten}}. Einfach vorzüglich.\pend
           \pstart
           Ich ſpreche Ihnen meine allerherzlichſte Gratulation aus.\pend
           \pstart \spacefill\mbox{Peter Altenberg}\pend{}\pstart
           30. November 98.\pend
           \endnumbering\briefempfaengerindex{Schnitzler, Arthur@\textsc{Schnitzler, Arthur}!zzzAltenberg, Peter@\emph{von Peter Altenberg}!1898-11-301@{30. 11. 1898}|)be}\mylabel{h}  \normalsize

\doendnotes{C}
\bigskip
\vfill

\clearpage

\footnotesize

\lohead{\textsc{register}}

% Definiere theindex-Environment komplett neu ohne reledmac
\makeatletter
\renewenvironment{theindex}{%
  \section*{\indexname}%
  \setlength{\parindent}{0pt}%
  \setlength{\parskip}{0pt plus 0.3pt}%
  \let\item\@idxitem
}{%
  \clearpage
}
\makeatother

\IfFileExists{\jobname-pw.ind}{\input{\jobname-pw.ind}}{}

\end{document}

      