%% latex-korrekturansicht-vorspann.tex
%% Vorspann für die Korrekturansicht.
%% Lädt die gemeinsame Datei latex-vorspann.tex mit gesetztem Schalter.

\newif\ifkorrekturansicht
\korrekturansichttrue

\input{../tex-inputs/latex-vorspann}


               \section[Max Burckhard an Arthur Schnitzler, {[}zwischen 27. 10. und 1. 11. 1896{]}]{ Max Burckhard an Arthur Schnitzler, {[}zwischen 27. 10. und
                    1. 11. 1896{]}}\nopagebreak\mylabel{v}\rehead{ }\normalsize\beginnumbering\briefempfaengerindex{Schnitzler, Arthur@\textsc{Schnitzler, Arthur}!zzzBurckhard, Max Eugen@\emph{von Max Eugen Burckhard}!1896-10-271@{{[}zwischen 27. 10. und
                        1. 11. 1896{]}}|(be} \toendnotes[C]{\smallbreak\pagebreak[2]} \Standort{CUL, Schnitzler, B 20.}
\physDesc{Telegramm
\newline{}maschinell
\newline{}Schnitzler: mit Bleistift datiert: »96?« \newline{}Ordnung: beschnitten }\toendnotes[C]{\smallbreak}\pstart
           \noindent{}{\pb}herzlichsten dank fuer mittheilung.
                    leider habe ich dienstag{ }vormittag generalprobe in \textcolor{pink}{schoenbrunn}{}\ledrightnote{\textcolor{pink}{Schloß Schönbrunn}} fuer die \label{K_L00610_1v}\edtext{festvorstellung}{\lemma{\textnormal{\emph{festvorstellung}}}\Cendnote{\textnormal{Diese fand am
                            4. 11. 1896, dem Vorabend der Hochzeit von Erzherzogin \textcolor{blue}{Marie Dorothea} mit \textcolor{blue}{Louis Philippe d’Orleans}, statt.}}}\label{K_L00610_1h} die ich
                    unmoeglich stuerzen kann. – ich hatte mich \label{K_L00610_2v}\edtext{so gefreut}{\lemma{\textnormal{\emph{so gefreut}}}\Cendnote{\textnormal{Am
                            3. 11. 1896 fand in \textcolor{pink}{Berlin}
                        am \textcolor{pink}{Deutschen Theater} die Uraufführung von \emph{\textcolor{green}{Freiwild}} statt.}}}\label{K_L00610_2h}. – so geht es
                    einem. – herzlichste gruesse und die besten wuensche. \spacefill\mbox{= doctor burckhard
                        +}\pend
           \endnumbering\briefempfaengerindex{Schnitzler, Arthur@\textsc{Schnitzler, Arthur}!zzzBurckhard, Max Eugen@\emph{von Max Eugen Burckhard}!1896-10-271@{{[}zwischen 27. 10. und
                        1. 11. 1896{]}}|)be}\mylabel{h}  \normalsize

\doendnotes{C}
\bigskip
\vfill

\clearpage

\footnotesize

\lohead{\textsc{register}}

% Definiere theindex-Environment komplett neu ohne reledmac
\makeatletter
\renewenvironment{theindex}{%
  \section*{\indexname}%
  \setlength{\parindent}{0pt}%
  \setlength{\parskip}{0pt plus 0.3pt}%
  \let\item\@idxitem
}{%
  \clearpage
}
\makeatother

\IfFileExists{\jobname-pw.ind}{\input{\jobname-pw.ind}}{}

\end{document}

      