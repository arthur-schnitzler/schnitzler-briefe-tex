%% latex-korrekturansicht-vorspann.tex
%% Vorspann für die Korrekturansicht.
%% Lädt die gemeinsame Datei latex-vorspann.tex mit gesetztem Schalter.

\newif\ifkorrekturansicht
\korrekturansichttrue

\input{../tex-inputs/latex-vorspann}


               \section[Paul Goldmann an Arthur Schnitzler, 4. 8. 1889]{ Paul Goldmann an Arthur Schnitzler, 4. 8. 1889}\nopagebreak\mylabel{v}\rehead{ }\normalsize\beginnumbering\briefempfaengerindex{Schnitzler, Arthur@\textsc{Schnitzler, Arthur}!zzzGoldmann, Paul@\emph{von Paul Goldmann}!1889-08-041@{4. 8. 1889}|(be} \toendnotes[C]{\smallbreak\pagebreak[2]} \Standort{DLA, A:Schnitzler, HS.NZ85.1.3162.}
\physDesc{Brief, 1 Blatt, 3 Seiten
\newline{}Handschrift: schwarze Tinte, deutsche Kurrent
\newline{}Schnitzler: mit rotem Buntstift eine Unterstreichung }\toendnotes[C]{\smallbreak}\pstart
           \noindent{}\centering{}{\pb}\textcolor{gray}{\textbf{\textbf{Adminiſtration: \textcolor{pink}{VII.
                           Seidengaſſe 7}{}\ledrightnote{\textcolor{pink}{Seidengasse}}} (\textcolor{brown}{Jos. Eberle {\kaufmannsund} Co.}{}\ledrightnote{\textcolor{brown}{Josef Eberle  Stein-, Buch und Musikaliendruckerei}})}}\pend
           \pstart
           \noindent{}\centering{}\textcolor{gray}{\textbf{\textcolor{brown}{An der Schönen Blauen Donau}{}\ledrightnote{\textcolor{brown}{An der schönen blauen Donau}}}}\pend
           \pstart
           \noindent{}\centering{}\textcolor{gray}{\textbf{Chef-Redacteur: Dr. \textcolor{blue}{F.
                        Mamroth}{}\ledrightnote{\textcolor{blue}{Fedor Mamroth}}. – Redaction: \textcolor{pink}{IX.,
                        Berggaſſe 31}{}\ledrightnote{\textcolor{pink}{Berggasse}}.}}\pend
           \pstart
           \raggedleft{}\textcolor{gray}{\textbf{\textcolor{pink}{Wien}{}\ledrightnote{\textcolor{pink}{Wien}}, den}}{ }4. Auguſt \textcolor{gray}{\textbf{18}}89.\pend
           \pstart{}Verehrter Herr Doctor!\pend\pstart
           Mein \textcolor{blue}{Onkel}{}\ledrightnote{\textcolor{blue}{Fedor Mamroth}}, mit dem ich geſtern beiſammen war, theilt mir mit, daß er ſich aus denſelben Gründen,
               wie ich, nämlich wegen der Düſterkeit des Süjets, ſcheut, Ihr \textcolor{green}{Feuilleton}{}\ledrightnote{→\textcolor{green}{Der Sohn. Aus den Papieren eines Arztes}} zu veröffentlichen. Im Übrigen
               hat es ihm ſehr gut gefallen und er möchte etwas \label{K_L02642-1v}\edtext{Anderes}{\lemma{\textnormal{\emph{Anderes}}}\Cendnote{\textnormal{siehe Fedor Mamroth an Arthur Schnitzler, 2. 8. 1889}}}\label{K_L02642-1h} von Ihnen haben. Eine Ablehnung alſo, die Sie abſolut {\pb}nicht tragiſch nehmen dürfen. Das
               Nähere mündlich.\pend
           \pstart
           Ich habe mich nämlich entſchloſſen, Ihre freundliche Aufforderung anzunehmen und mit
               Ihnen die \label{K_L02642-2v}\edtext{Parthie}{\lemma{\textnormal{\emph{Parthie}}}\Cendnote{\textnormal{Vom 10. 8. 1889 bis zum 18. 8. 1889 wanderten \textcolor{blue}{Goldmann}, \textcolor{blue}{Schnitzler}
                  und dessen Bruder \textcolor{blue}{Julius Schnitzler} von \textcolor{pink}{Traunkirchen} nach \textcolor{pink}{Reichenau}.}}}\label{K_L02642-2h} zu machen. Es fragt ſich freilich noch,
               ob ich die Fahrkarte bekomme, zur Zeit mit den redactionellen Arbeiten fertig werde
                  \textsc{etc}. Prinzipiell aber bin ich entſchloſſen, Donnerſtag{ }Abend von \textcolor{pink}{hier}{}\ledrightnote{→\textcolor{pink}{Wien}}
               abzureiſen und Sie Freitag{ }früh, wenn Sie inzwiſchen Ihre Entſchließungen nicht geändert haben
               ſollten, \label{K_L02642-4v}\edtext{irgendwo in der Welt}{\lemma{\textnormal{\emph{irgendwo in der Welt}}}\Cendnote{\textnormal{Sie trafen am 9. 8. 1889 auf dem Weg nach \textcolor{pink}{Traunkirchen} zusammen.}}}\label{K_L02642-4h} zu treffen. Ich bitte Sie
               alſo, mir umgehend mitzutheilen, wo Sie am Freitag
               ſind. {\pb}Vielleicht können Sie mich
               noch in \textcolor{pink}{\textsc{Ischl}}{}\ledrightnote{\textcolor{pink}{Bad Ischl}} erwarten. Ich ſelbſt werde Ihnen am Donnerſtag
               meine mir zu beſtimmende Adreſſe \label{K_L02642-3v}\edtext{telegraphiren}{\lemma{\textnormal{\emph{telegraphiren}}}\Cendnote{\textnormal{Ein entsprechendes Telegramm ist nicht
                  überliefert.}}}\label{K_L02642-3h}, ob ich mit meinen Angelegenheiten in Ordnung bin und kommen
               kann.\pend
           \pstart
           Herzlichſten Gruß und Dank im Voraus! {\\[\baselineskip]}Ihr {\\[\baselineskip]}\spacefill\mbox{Dr. Paul Goldma{\geminationn}}\pend
           \leftskip=0em{}\endnumbering\briefempfaengerindex{Schnitzler, Arthur@\textsc{Schnitzler, Arthur}!zzzGoldmann, Paul@\emph{von Paul Goldmann}!1889-08-041@{4. 8. 1889}|)be}\mylabel{h}  \normalsize

\doendnotes{C}
\bigskip
\vfill

\clearpage

\footnotesize

\lohead{\textsc{register}}

% Definiere theindex-Environment komplett neu ohne reledmac
\makeatletter
\renewenvironment{theindex}{%
  \section*{\indexname}%
  \setlength{\parindent}{0pt}%
  \setlength{\parskip}{0pt plus 0.3pt}%
  \let\item\@idxitem
}{%
  \clearpage
}
\makeatother

\IfFileExists{\jobname-pw.ind}{\input{\jobname-pw.ind}}{}

\end{document}

      