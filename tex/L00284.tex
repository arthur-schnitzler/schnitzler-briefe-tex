%% latex-korrekturansicht-vorspann.tex
%% Vorspann für die Korrekturansicht.
%% Lädt die gemeinsame Datei latex-vorspann.tex mit gesetztem Schalter.

\newif\ifkorrekturansicht
\korrekturansichttrue

\input{../tex-inputs/latex-vorspann}


               \section[Wilhelm Bölsche an Arthur Schnitzler, 16. 11. 1893]{ Wilhelm Bölsche an Arthur Schnitzler, 16. 11. 1893}\nopagebreak\mylabel{v}\rehead{ }\normalsize\beginnumbering\briefempfaengerindex{Schnitzler, Arthur@\textsc{Schnitzler, Arthur}!zzzBoelsche, Wilhelm@\emph{von Wilhelm Bölsche}!1893-11-161@{16. 11. 1893}|(be} \toendnotes[C]{\smallbreak\pagebreak[2]} \Standort{DLA, A:Schnitzler, HS.NZ85.1.2577,9.}
\physDesc{Postkarte
\newline{}Handschrift: schwarze Tinte, deutsche Kurrent\newline{}Versand: 1) Stempel: »\nobreak{}\oindex{Enge@\textbf{Enge}, \emph{Teil eines besiedelten Ortes (A.BSOX)}|pwk}Zürich 7 Enge, 16. XI. 93., 6\nobreak{}«.  2) Stempel: »\nobreak{}Wien 9/3 72, 18. 11. 93, 8.V, Bestellt\nobreak{}«. 
\newline{}Schnitzler: mit rotem Buntstift nummeriert: »10« }\buchAbdrucke{\weitereDrucke{Wilhelm Bölsche: \emph{Briefwechsel. Mit Autoren der Freien Bühne}. Hg. Gerd-Hermann Susen. Berlin: \emph{Weidler} 2010, S. 695 (Werke und Briefe. Wissenschaftliche Ausgabe, Briefe I).} }\pstart{}{\pb}Herrn Dr. Schnitzler\pend{}\pstart{}\textcolor{pink}{Wien IX}{}\ledrightnote{\textcolor{pink}{IX., Alsergrund}}\pend{}\pstart{}\textcolor{pink}{Frankgaſſe 1}{}\ledrightnote{\textcolor{pink}{Frankgasse}}. \pend{}{\bigskip}\pstart{}{\pb}Hochgeehrter Herr Dr.!\pend\pstart
           Die Redaktion der »\textcolor{green}{Freien Bühne}{}\ledrightnote{\textcolor{green}{Freie Bühne für den Entwickelungskampf der Zeit}}« hat Hr. \textcolor{blue}{Otto Julius Bierbaum}{}\ledrightnote{\textcolor{blue}{Otto Julius Bierbaum}}, \textcolor{pink}{Berlin, Köthener Str. 44}{}\ledrightnote{\textcolor{pink}{Köthenerstraße}} übernommen, ich bitte Sie, bei
                    dieſem nachzufragen. Ich bin ſeit 1. Okt. zurückgetreten, – in
                    einer allgemeinen »Redaktionsmüdigkeit,« die Sie vielleicht verſtehen
                    werden.\pend
           \pstart
           Mit herzlichem Gruß{\\[\baselineskip]}Ihr\spacefill\mbox{W. Bölsche}\pend
           \leftskip=0em{}\pstart
           \noindent{}\textcolor{pink}{Zürich-Enge}{}\ledrightnote{\textcolor{pink}{Enge}}.{\\}\textcolor{pink}{Seewartstr. 12\textsubscript{I}}{}\ledrightnote{\textcolor{pink}{Seewartstraße}}.\pend
           \endnumbering\briefempfaengerindex{Schnitzler, Arthur@\textsc{Schnitzler, Arthur}!zzzBoelsche, Wilhelm@\emph{von Wilhelm Bölsche}!1893-11-161@{16. 11. 1893}|)be}\mylabel{h}  \normalsize

\doendnotes{C}
\bigskip
\vfill

\clearpage

\footnotesize

\lohead{\textsc{register}}

% Definiere theindex-Environment komplett neu ohne reledmac
\makeatletter
\renewenvironment{theindex}{%
  \section*{\indexname}%
  \setlength{\parindent}{0pt}%
  \setlength{\parskip}{0pt plus 0.3pt}%
  \let\item\@idxitem
}{%
  \clearpage
}
\makeatother

\IfFileExists{\jobname-pw.ind}{\input{\jobname-pw.ind}}{}

\end{document}

      