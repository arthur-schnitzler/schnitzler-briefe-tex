%% latex-korrekturansicht-vorspann.tex
%% Vorspann für die Korrekturansicht.
%% Lädt die gemeinsame Datei latex-vorspann.tex mit gesetztem Schalter.

\newif\ifkorrekturansicht
\korrekturansichttrue

\input{../tex-inputs/latex-vorspann}


               \section[Richard Beer-Hofmann an Arthur Schnitzler, {[}23. 3. 1897{]}]{ Richard Beer-Hofmann an Arthur Schnitzler, {[}23. 3. 1897{]}}\nopagebreak\mylabel{v}\rehead{ }\normalsize\beginnumbering\briefempfaengerindex{Schnitzler, Arthur@\textsc{Schnitzler, Arthur}!zzzBeer-Hofmann, Richard@\emph{von Richard Beer-Hofmann}!1897-03-233@{23. 3. 1897}|(be} \toendnotes[C]{\smallbreak\pagebreak[2]} \Standort{CUL, Schnitzler, B 8.}
\physDesc{Brief, 1 Blatt, 1 Seite
\newline{}Handschrift: Bleistift, lateinische Kurrent
\newline{}Schnitzler: mit Bleistift datiert: »23/III 97« \newline{}Ordnung: mit Bleistift von unbekannter Hand nummeriert:
                                    »102« }\pstart
           \noindent{}{\pb}Lieber Arthur! Ein Frl. \textcolor{blue}{Wengeroff}{}\ledrightnote{\textcolor{blue}{Isabella Vengerova}} (\textcolor{pink}{Russin}{}\ledrightnote{\textcolor{pink}{Russland}}) möchte Sie und \textcolor{blue}{Hugo}{}\ledrightnote{\textcolor{blue}{Hugo von Hofmannsthal}} heut nach 10 im Caffee sehn.
               Wenn Sie können ko{\geminationm}en Sie doch. Führer = Herr \textcolor{blue}{A. Brauner}{}\ledrightnote{\textcolor{blue}{Alexander Brauner}}. Gefasst sein!\pend
           \pstart
           Herzlichst{\\[\baselineskip]}Richard\pend
           \leftskip=0em{}\endnumbering\briefempfaengerindex{Schnitzler, Arthur@\textsc{Schnitzler, Arthur}!zzzBeer-Hofmann, Richard@\emph{von Richard Beer-Hofmann}!1897-03-233@{23. 3. 1897}|)be}\mylabel{h}  \normalsize

\doendnotes{C}
\bigskip
\vfill

\clearpage

\footnotesize

\lohead{\textsc{register}}

% Definiere theindex-Environment komplett neu ohne reledmac
\makeatletter
\renewenvironment{theindex}{%
  \section*{\indexname}%
  \setlength{\parindent}{0pt}%
  \setlength{\parskip}{0pt plus 0.3pt}%
  \let\item\@idxitem
}{%
  \clearpage
}
\makeatother

\IfFileExists{\jobname-pw.ind}{\input{\jobname-pw.ind}}{}

\end{document}

      