%% latex-korrekturansicht-vorspann.tex
%% Vorspann für die Korrekturansicht.
%% Lädt die gemeinsame Datei latex-vorspann.tex mit gesetztem Schalter.

\newif\ifkorrekturansicht
\korrekturansichttrue

\input{../tex-inputs/latex-vorspann}


               \section[Arthur Schnitzler an Richard Beer-Hofmann, 11. 8. 1893]{ Arthur Schnitzler an Richard Beer-Hofmann, 11. 8. 1893}\nopagebreak\mylabel{v}\rehead{ }\normalsize\beginnumbering\briefempfaengerindex{Beer-Hofmann, Richard@\textsc{Beer-Hofmann, Richard}!zzzSchnitzler, Arthur@\emph{von Arthur Schnitzler}!1893-08-111@{11. 8. 1893}|(be} \toendnotes[C]{\smallbreak\pagebreak[2]} \Standort{YCGL, MSS 31.}
\physDesc{Brief, 1 Blatt (Briefpapier mit Trauerrand), 1 Seite
\newline{}Handschrift: schwarze Tinte, deutsche Kurrent}\buchAbdrucke{\weitereDrucke{Arthur Schnitzler, Richard Beer-Hofmann: \emph{Briefwechsel 1891–1931}. Hg. Konstanze Fliedl. Wien, Zürich: \emph{Europaverlag} 1992, S. 50.} }\toendnotes[C]{\smallbreak}\pstart
           \noindent{}{\pb}Lieber Richard, warum ſchreiben Sie
               mir nicht? – – Haben Sie Ihre \textcolor{green}{Novelle}{}\ledrightnote{→\textcolor{green}{Camelias}}
               vorgeleſen? – Was macht der \textcolor{green}{Götterliebling}{}\ledrightnote{\textcolor{green}{Der Tod Georgs}}? – Erfuhren Sie was über \textcolor{brown}{Freund u \textsc{Jäckel}}{}\ledrightnote{\textcolor{brown}{Freund {\kaufmannsund} Jeckel}}? – Sehen Sie \textcolor{blue}{Benedikt’s}{}\ledrightnote{\textcolor{blue}{Marianne Benedict}{\newline}\textcolor{blue}{Markus Benedict}}? – Haben Sie gehört, wie ſchauerlich und wie du{\geminationm} die \textcolor{green}{Abendpoſt}{}\ledrightnote{\textcolor{green}{Wiener Abendpost}} den
                  \textcolor{green}{Anatol}{}\ledrightnote{\textcolor{green}{Anatol}}{ }\textcolor{green}{verriſs}{}\ledrightnote{→\textcolor{green}{Literatur. »Bunte Reihe.« Ein Geschichtenbuch von Moritz Goldschmidt. »Anatol« von Arthur Schnitzler}}? – Wa{\geminationn}
               rücken Sie ein? Wann sind Sie in \textcolor{pink}{Wien}{}\ledrightnote{\textcolor{pink}{Wien}}? – Ich
               reiſe vielleicht am 19. oder 20. ab. – Sind Sie
               glücklich? – Sind Sie arrogant? – Wiſſen Sie, daſs Sie noch im Herbſt \textsc{Bic}. fahren lernen werden? Was macht Frau \textcolor{blue}{\textsc{Flegm}.}{}\ledrightnote{\textcolor{blue}{Bertha Flegmann}}? Was das Theater? – Sprachen Sie
                  \textcolor{blue}{\textsc{Jarno}}{}\ledrightnote{\textcolor{blue}{Josef Jarno}}? – Die \textcolor{blue}{\textsc{Wreden}}{}\ledrightnote{\textcolor{blue}{Grethe Wreden}}? – Stand was in der \textcolor{green}{Iſchler
                  Ztg.}{}\ledrightnote{→\textcolor{green}{Ischler Wochenblatt}} über mein \textcolor{green}{Stück}{}\ledrightnote{→\textcolor{green}{Anatol}}? – Senden Sie
               – ich vertrage alles\substVorne{}\textsuperscript{?}\substDazwischen{}. –\substHinten{}{ }\textcolor{blue}{Goldmann}{}\ledrightnote{\textcolor{blue}{Paul Goldmann}} ko{\geminationm}t im September
               nach \textcolor{pink}{Salzburg}{}\ledrightnote{\textcolor{pink}{Salzburg}}. –\pend
           \pstart Herzlich der Ihre \spacefill\mbox{Arthur}\pend{}\endnumbering\briefempfaengerindex{Beer-Hofmann, Richard@\textsc{Beer-Hofmann, Richard}!zzzSchnitzler, Arthur@\emph{von Arthur Schnitzler}!1893-08-111@{11. 8. 1893}|)be}\mylabel{h}  \normalsize

\doendnotes{C}
\bigskip
\vfill

\clearpage

\footnotesize

\lohead{\textsc{register}}

% Definiere theindex-Environment komplett neu ohne reledmac
\makeatletter
\renewenvironment{theindex}{%
  \section*{\indexname}%
  \setlength{\parindent}{0pt}%
  \setlength{\parskip}{0pt plus 0.3pt}%
  \let\item\@idxitem
}{%
  \clearpage
}
\makeatother

\IfFileExists{\jobname-pw.ind}{\input{\jobname-pw.ind}}{}

\end{document}

      