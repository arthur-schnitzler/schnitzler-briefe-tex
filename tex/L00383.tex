%% latex-korrekturansicht-vorspann.tex
%% Vorspann für die Korrekturansicht.
%% Lädt die gemeinsame Datei latex-vorspann.tex mit gesetztem Schalter.

\newif\ifkorrekturansicht
\korrekturansichttrue

\input{../tex-inputs/latex-vorspann}


               \section[Richard Beer-Hofmann an Arthur Schnitzler, 15. 10. 1894]{ Richard Beer-Hofmann an Arthur Schnitzler, 15. 10. 1894}\nopagebreak\mylabel{v}\rehead{ }\normalsize\beginnumbering\briefempfaengerindex{Schnitzler, Arthur@\textsc{Schnitzler, Arthur}!zzzBeer-Hofmann, Richard@\emph{von Richard Beer-Hofmann}!1894-10-152@{15. 10. 1894}|(be} \toendnotes[C]{\smallbreak\pagebreak[2]} \Standort{CUL, Schnitzler, B 8.}
\physDesc{Brief, 1 Blatt, 4 Seiten
\newline{}Handschrift: Bleistift, deutsche Kurrent
\newline{}Schnitzler: mit Bleistift datiert: »15/10 94« und nummeriert: »40« \newline{}Ordnung: mit Bleistift von unbekannter Hand nummeriert: »40« }\buchAbdrucke{\weitereDrucke{Arthur Schnitzler, Richard Beer-Hofmann: \emph{Briefwechsel 1891–1931}. Hg. Konstanze Fliedl. Wien, Zürich: \emph{Europaverlag} 1992, S. 63.} }\toendnotes[C]{\smallbreak}\pstart
           \raggedleft{}{\pb}\textcolor{pink}{Fraskati}{}\ledrightnote{\textcolor{pink}{Frascati}}{ }Sonntag{ }½ 8\pend
           \pstart
           {\pb}Lieber Arthur, diesen Brief schreibe ich au\substVorne{}\textsuperscript{s}\substDazwischen{}f\substHinten{}{ }\substVorne{}\textsuperscript{a}\substDazwischen{}e\substHinten{}iner Terrasse \strikeout{b} in \textcolor{pink}{Fraskati}{}\ledrightnote{\textcolor{pink}{Frascati}}, stehend, im Mondlicht; ich habe nämlich noch eine halbe
               Stunde Zeit bis zum Abgang des Zuges nach \textcolor{pink}{Rom}{}\ledrightnote{\textcolor{pink}{Rom}}.
                  {\pb}Ich bin sehr »\textcolor{green}{des Gottes voll}{}\ledrightnote{→\textcolor{green}{Die Kraniche des Ibykus}}« aber \uline{arbeite}
               gar nichts, und notire mittelmäßig viel. Ich sehe vieles anders und verstehe Einiges
               was mir fremd war. Arroganter werd ich {\pb}sein als je, wenn ich zurückko{\geminationm}e. Wenn man tagsüber mit schönen Bildern, einer Natur
               die hier Künstlerin ist, und mit – seinen Gedanken – verkehrt {\pb}findet man die Gesellschaft die um
               uns (– wie heißt das analoge Wort zu\pend
           \settowidth{\longeste}{crepiren!}\settowidth{\longestz}{–}\settowidth{\longestd}{sterben}\settowidth{\longestv}{}\settowidth{\longestf}{}\addtolength\longeste{1em}
        \addtolength\longestz{1em}
        \addtolength\longestd{1em}
      \pstart\noindent\makebox[\the\longeste][l]{crepiren\strikeout{!}}\makebox[\the\longestz][l]{–}
                  \makebox[\the\longestd][l]{sterben}\pend\pstart\noindent\makebox[\the\longeste][l]{×}\makebox[\the\longestz][l]{–}
                  \makebox[\the\longestd][l]{leben)}\pend\pstart
           unmöglich; ich bin am 4. od. 5. voraussichtlich in
                  \textcolor{pink}{Wien}{}\ledrightnote{\textcolor{pink}{Wien}}; von morgen an \uline{\textcolor{pink}{Neapel}{}\ledrightnote{\textcolor{pink}{Neapel}}} a posta ferma.\pend
           \pstart Herzlichst Ihr \spacefill\mbox{R}\pend{}\endnumbering\briefempfaengerindex{Schnitzler, Arthur@\textsc{Schnitzler, Arthur}!zzzBeer-Hofmann, Richard@\emph{von Richard Beer-Hofmann}!1894-10-152@{15. 10. 1894}|)be}\mylabel{h}  \normalsize

\doendnotes{C}
\bigskip
\vfill

\clearpage

\footnotesize

\lohead{\textsc{register}}

% Definiere theindex-Environment komplett neu ohne reledmac
\makeatletter
\renewenvironment{theindex}{%
  \section*{\indexname}%
  \setlength{\parindent}{0pt}%
  \setlength{\parskip}{0pt plus 0.3pt}%
  \let\item\@idxitem
}{%
  \clearpage
}
\makeatother

\IfFileExists{\jobname-pw.ind}{\input{\jobname-pw.ind}}{}

\end{document}

      