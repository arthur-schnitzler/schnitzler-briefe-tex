%% latex-korrekturansicht-vorspann.tex
%% Vorspann für die Korrekturansicht.
%% Lädt die gemeinsame Datei latex-vorspann.tex mit gesetztem Schalter.

\newif\ifkorrekturansicht
\korrekturansichttrue

\input{../tex-inputs/latex-vorspann}


               \section[Robert Adam an Arthur Schnitzler, 8. 12. 1918]{ Robert Adam an Arthur Schnitzler, 8. 12. 1918}\nopagebreak\mylabel{v}\rehead{ }\normalsize\beginnumbering\briefempfaengerindex{Schnitzler, Arthur@\textsc{Schnitzler, Arthur}!zzzAdam, Robert@\emph{von Robert Adam}!1918-12-081@{8. 12. 1918}|(be} \toendnotes[C]{\smallbreak\pagebreak[2]} \Standort{CUL, Schnitzler, B 1.}
\physDesc{Brief, 1 Blatt, 4 Seiten
\newline{}Handschrift: schwarze Tinte, deutsche Kurrent
\newline{}Schnitzler: 1) mit Bleistift beschriftet: »\textsc{Adam}« 2) mit rotem Buntstift zwei Unterstreichungen\newline{}Ordnung: von unbekannter Hand nummeriert: »11« }\Standort{Wien, Österreichische Nationalbibliothek, Cod.ser. 52.263, 227.}
\physDesc{Brief, maschinelle Abschrift
\newline{}Schreibmaschine}\toendnotes[C]{\smallbreak}\pstart
           \raggedleft{}{\pb}\textcolor{pink}{Wien}{}\ledrightnote{\textcolor{pink}{Wien}}, 8. Dezember 1918\pend
           \pstart\center{}Hochverehrter Herr Doktor!\pend\pstart
           Sie haben mir durch die Zuſendung von »\textcolor{green}{\textsc{Casanovas Heimfahrt}}{}\ledrightnote{\textcolor{green}{Casanovas Heimfahrt}}« eine große Freude bereitet, und ich ſage Ihnen herzlichen Dank. Wie ſehr
                    ich dieſe \textcolor{green}{Novelle}{}\ledrightnote{→\textcolor{green}{Casanovas Heimfahrt}}, die ich
                    zum erſtenmal während des Erſcheinens in der \textcolor{green}{Neuen
                        Rundschau}{}\ledrightnote{\textcolor{green}{Die neue Rundschau}} las, als die wundervoll-weiſe und ſüße Frucht einer
                    Erzählermeiſterſchaft ſchätze, habe ich Ihnen bereits geſagt. Wenn ich mich
                    geneigt fühle, ſie allen Ihren früheren epiſchen Arbeiten voranzuſtellen, mag
                    mich vielleicht meine Vorliebe für den Helden, mit deſſen Memoiren ich mich
                    längere Zeit beſchäftigt habe, beeinfluſſen; aber daß hier alle Geſtalten, nicht
                    nur der Held, ein eigenes Leben lebten, ſodaß es iſt, als ſchüfe der Dichter
                    nicht, wie eine \textsc{laterna magica}, ſondern als
                    beleuchtete er bloß, wie ein ſcharfer Scheinwerfer ſchon Exiſtierendes; daß jede
                    Geberde der handelnden Perſonen, alles {\pb}Lebende und Lebloſe, das ſie umgibt,
                    mit gewaltiger Plaſtik, die doch nie aufhört, das einfachſte und
                    ſelbſtverſtändlichſte Ding der Welt zu ſcheinen, hingeſtellt und umriſſen iſt;
                    daß auf allen der 181 Seiten des \textcolor{green}{Buchs}{}\ledrightnote{→\textcolor{green}{Casanovas Heimfahrt}} kein Wort zuviel und daher unnütz zu ſein ſcheint, was mir als
                    Merkzeichen einer klaſſiſchen Arbeit gilt – das muß und wird jeder
                    Kunſtverſtändige, wenn er auch meine Spezialliebe zum Helden nicht teilt, aus
                    vollem Herzen bezeugen. Ich bin ſchon außerordentlich auf Ihren jungen \textcolor{green}{\textcolor{blue}{\textsc{Casanova}}{}\ledrightnote{\textcolor{blue}{Giacomo Girolamo Casanova}} in \textcolor{pink}{Spaa}{}\ledrightnote{\textcolor{pink}{Spa}}}{}\ledrightnote{→\textcolor{green}{Die Schwestern oder Casanova in Spa. Lustspiel in Versen}} begierig, den wir wohl ſchon längſt kennen gelernt hätten, wenn die
                    politiſche Umwälzung nicht gekommen wäre. Bis er erſcheint, will ich mir noch
                    einmal, und nun mit Muße und unabhängig von Fortſetzungen, den gealterten Sünder
                    vornehmen und an Ihrem Werke lernen, wie man klar und farbig und ſpannend und
                    einfach und doch geiſtreich erzählen kann: daß ich dies nicht kann und niemals
                    können werde, iſt etwas, was mich manchmal niedergeſchlagen, immer aber vor dem,
                    der es kann, ehrfürchtig und beſcheiden {\pb}macht. –\pend
           \pstart
           Die Bitte, die ich in meinem letzten Briefe an Sie ſtellte – Sie möchten ſich
                    über das Geſchick meiner zwei \textcolor{green}{Stücke}{}\ledrightnote{→\textcolor{green}{Yppl. Idylle in fünf Akten}{\newline}→\textcolor{green}{Der Fremde}} gelegentlich erkundigen – iſt durch die
                    traurigen Ereigniſſe der letzten Woche gegenſtandslos geworden; Sie werden
                    einſehen, daß mich wirklich das Pech verfolgt – ich glaube ſogar, daß das
                    Theater, das wirklich einmal eines meiner Stücke zur Aufführung bringen wollte,
                    zumindeſt am Tage der Erſtaufführung in Flammen aufgehen oder Konkurs anſagen
                    würde. Wenn ich alſo Trübſal blaſe – das einzige Inſtrument, für das meine
                    muſikaliſche Anlage zureicht –, ſo iſt dieſe Beſchäftigung nicht ſo ganz
                    unberechtigt, zumal es, trotz mancher hübſchen neuen Geſetze, nicht viel
                    Erquickliches ringsum gibt, das aufheitern oder tröſten könnte – die
                    Verhältniſſe haben es mit ſich gebracht, daß ich, der noch vor kurzem aus dem
                    Staatsdienſt mich wegſehnte, um die mir noch etwa verbliebene Kraft frei
                    verwerten zu können, nunmehr, beim Anblick ſo vieler \label{K_L02315_1v}\edtext{\textcolor{blue}{Beliſare}{}\ledrightnote{→\textcolor{blue}{Flavius Belisar}}}{\lemma{\textnormal{\emph{Beliſare}}}\Cendnote{\textnormal{Hier wohl im Sinne der apokrpyhen
                        Überlieferung, \textcolor{blue}{Belisar} hätte, nach seiner
                        Zeit als Feldherr, die Augen ausgestochen bekommen und als Bettler auf der
                        Straße gelebt.}}}\label{K_L02315_1h}, froh ſein muß, ein feſtes Amt zu bekleiden, und
                    nicht, wie ſo mancher meines Alters, auf Stel{\pb}lungsſuche gehen zu müſſen. Daß ich
                    aber in der hungernden und frierenden \textcolor{pink}{Republik}{}\ledrightnote{→\textcolor{pink}{Österreich}} gerade ſo wie im Kaiſerſtaat Tag für Tag
                    über Preistreibereien zu Gericht ſitze, als wäre gar nichts geſchehen, als
                    beſtünde noch der außerordentliche Kriegszuſtand, das kommt mir manchmal ſo
                    grauenhaft vor wie das Weiterwachſen der Haare einer Leiche, die verfault und
                    zerfällt. –\pend
           \pstart
           Nochmals beſten Dank! Und die herzlichsten Grüße von Ihrem{\\[\baselineskip]}ergebenen{\\[\baselineskip]}\spacefill\mbox{D\textsuperscript{r}RAdam}\pend
           \leftskip=0em{}\endnumbering\briefempfaengerindex{Schnitzler, Arthur@\textsc{Schnitzler, Arthur}!zzzAdam, Robert@\emph{von Robert Adam}!1918-12-081@{8. 12. 1918}|)be}\mylabel{h}  \normalsize

\doendnotes{C}
\bigskip
\vfill

\clearpage

\footnotesize

\lohead{\textsc{register}}

% Definiere theindex-Environment komplett neu ohne reledmac
\makeatletter
\renewenvironment{theindex}{%
  \section*{\indexname}%
  \setlength{\parindent}{0pt}%
  \setlength{\parskip}{0pt plus 0.3pt}%
  \let\item\@idxitem
}{%
  \clearpage
}
\makeatother

\IfFileExists{\jobname-pw.ind}{\input{\jobname-pw.ind}}{}

\end{document}

      