%% latex-korrekturansicht-vorspann.tex
%% Vorspann für die Korrekturansicht.
%% Lädt die gemeinsame Datei latex-vorspann.tex mit gesetztem Schalter.

\newif\ifkorrekturansicht
\korrekturansichttrue

\input{../tex-inputs/latex-vorspann}


               \section[Arthur Schnitzler an Hermann Bahr, {[}2. – 6.?{]} 3. 1899]{ Arthur Schnitzler an Hermann Bahr, {[}2. – 6.?{]} 3. 1899}\nopagebreak\mylabel{v}\rehead{ }\normalsize\beginnumbering\briefempfaengerindex{Bahr, Hermann@\textsc{Bahr, Hermann}!zzzSchnitzler, Arthur@\emph{von Arthur Schnitzler}!1899-03-021@{{[}2. – 6.?{]} 3. 1899}|(be} \toendnotes[C]{\smallbreak\pagebreak[2]} \Standort{TMW, HS AM 60155 Ba.}
\physDesc{Briefkarte
\newline{}Handschrift: schwarze Tinte, deutsche Kurrent\newline{}Ordnung: Lochung }\buchAbdrucke{\weitereDrucke{1) \emph{[5. 3. 1899?], Abschrift.} In: Arthur Schnitzler: \emph{The Letters of Arthur Schnitzler to Hermann Bahr}. Edited, annotated, and with an introduction, by Donald G.
                        Daviau. Chapel Hill: \emph{The University of North Carolina Press} 1978, S. 65 (University of North Carolina studies in the Germanic languages
                        and literatures, 89).} \weitereDrucke{2) Hermann Bahr, Arthur Schnitzler: \emph{Briefwechsel, Aufzeichnungen, Dokumente (1891–1931)}. Hg. Kurt Ifkovits und Martin Anton Müller. Göttingen: \emph{Wallstein} 2018, S. 168.} }\toendnotes[C]{\smallbreak}\pstart
           \noindent{}{\pb}Lieber Hermann, beſten Dank für deine freundl \label{K_L00896_1v}\edtext{Gratulation}{\lemma{\textnormal{\emph{Gratulation}}}\Cendnote{\textnormal{nicht überliefert; am 1. 3. 1900 Uraufführung der drei Einakter \emph{\textcolor{green}{Der grüne Kakadu}}, \emph{\textcolor{green}{Paracelsus}}, \emph{\textcolor{green}{Die Gefährtin}}  am \emph{\textcolor{brown}{Burgtheater}}}}}\label{K_L00896_1h}. Bei dieſer Gelegenheit:\pend
           \pstart
           1) kannſt du die »\uline{\textcolor{green}{Gefährtin}{}\ledrightnote{\textcolor{green}{Die Gefährtin. Schauspiel in einem Akt}}}«, da \textcolor{blue}{Hofmannsthal}{}\ledrightnote{\textcolor{blue}{Hugo von Hofmannsthal}}’s \textcolor{green}{Sobeïde}{}\ledrightnote{\textcolor{green}{Die Hochzeit der Sobeide}} wegfällt, gleich nach \textcolor{blue}{Salten}{}\ledrightnote{\textcolor{blue}{Felix Salten}} bringen?\pend
           \pstart
           2) biſt du \textsc{resp}{ }ſeid Ihr mit dem Honorar von 200 Gulden
               einverſtanden? \pend
           \pstart
           {\pb}Herzlichen Gruſs. Dein ergebner{\\[\baselineskip]}\spacefill\mbox{Arth Schnitzler}\pend
           \leftskip=0em{}\endnumbering\briefempfaengerindex{Bahr, Hermann@\textsc{Bahr, Hermann}!zzzSchnitzler, Arthur@\emph{von Arthur Schnitzler}!1899-03-021@{{[}2. – 6.?{]} 3. 1899}|)be}\mylabel{h}  \normalsize

\doendnotes{C}
\bigskip
\vfill

\clearpage

\footnotesize

\lohead{\textsc{register}}

% Definiere theindex-Environment komplett neu ohne reledmac
\makeatletter
\renewenvironment{theindex}{%
  \section*{\indexname}%
  \setlength{\parindent}{0pt}%
  \setlength{\parskip}{0pt plus 0.3pt}%
  \let\item\@idxitem
}{%
  \clearpage
}
\makeatother

\IfFileExists{\jobname-pw.ind}{\input{\jobname-pw.ind}}{}

\end{document}

      