%% latex-korrekturansicht-vorspann.tex
%% Vorspann für die Korrekturansicht.
%% Lädt die gemeinsame Datei latex-vorspann.tex mit gesetztem Schalter.

\newif\ifkorrekturansicht
\korrekturansichttrue

\input{../tex-inputs/latex-vorspann}


               \section[Hermann Bahr an Arthur Schnitzler, 4. 4. {[}1903{]}]{ Hermann Bahr an Arthur Schnitzler, 4. 4. {[}1903{]}}\nopagebreak\mylabel{v}\rehead{ }\normalsize\beginnumbering\briefempfaengerindex{Schnitzler, Arthur@\textsc{Schnitzler, Arthur}!zzzBahr, Hermann@\emph{von Hermann Bahr}!1903-04-041@{4. 4. 1903}|(be} \toendnotes[C]{\smallbreak\pagebreak[2]} \Standort{CUL, Schnitzler, B 5b.}
\physDesc{Brief, 2 Blätter, 3 Seiten
\newline{}Handschrift: schwarze Tinte, deutsche Kurrent\newline{}Ordnung: mit Bleistift von unbekannter Hand nummeriert:
                                    »98« }\buchAbdrucke{\weitereDrucke{Hermann Bahr, Arthur Schnitzler: \emph{Briefwechsel, Aufzeichnungen, Dokumente (1891–1931)}. Hg. Kurt Ifkovits und Martin Anton Müller. Göttingen: \emph{Wallstein} 2018, S. 258–259.} }\toendnotes[C]{\smallbreak}\pstart
           \raggedleft{}{\pb}4. 4.\pend
           \pstart\center{}Lieber Arthur!\pend\pstart
           Nächſtens erſcheint von mir bei \textcolor{blue}{Fiſcher}{}\ledrightnote{\textcolor{blue}{Samuel Fischer}} ein Band
                  »\textcolor{green}{Rezenſionen}{}\ledrightnote{\textcolor{green}{Rezensionen. Wiener Theater 1901 bis 1903}}«, Kritiken von 1901–1903. Mir wäre
               lieb, ihn Dir widmen zu dürfen. Macht Dir das aber keinen Spaß oder iſt es Dir aus
               irgend einem Grunde, den Du mir gar nicht zu nennen brauchſt, (vielleicht, weil man
               wieder Clique sagen wird), zuwider oder auch nur unbequem, kurz wenn Du irgend das
               Gefühl haſst: Lieber nicht, ſo werde ich weder beleidigt noch gekränkt noch
               verſchnupft noch irgend unangenehm berührt oder gegen Dich verändert ſein, ſo weit
               kennſt Du mich doch!{\pb}\pend
           \pstart
           Im \textcolor{green}{Neuen Wiener Journal}{}\ledrightnote{\textcolor{green}{Neues Wiener Journal}}{ }ſteht, daß Du \label{K_L01286_1v}\edtext{geheiratet haſt}{\lemma{\textnormal{\emph{geheiratet haſt}}}\Cendnote{\textnormal{\emph{\textcolor{green}{Neues Wiener Journal}}, Jg. 11, Nr. 3389,
                        3. 4. 1903, S. 6: »Wie uns
                        mitgethei{[}l{]}t wird, hat sich Dr. Arthur \so{Schnitzler} dieser Tage in aller Stille \so{vermählt}. Seine \textcolor{blue}{Gattin} ist eine junge Dame, die noch vor Kurzem das
                        \textcolor{pink}{Conservatorium} besucht hat.« Am
                  Folgetag stand auf S. 8: »Herr Dr. Arthur \so{Schnitzler} theilt uns mit, daß er noch immer
                     unvermählt ist.«}}}\label{K_L01286_1h}. Vielleicht iſt es aber nicht wahr. Nach meinen
               Erfahrungen einer Ehe von acht Jahren kann man Dir in beiden Fällen herzlich
               gratulieren, was hiemit geſchieht.\pend
           \pstart
           Damit Du aber ſiehſt, wie man in dieſer Inſtitution herabkommt, wiſſe, daß ich Deinem
                  \label{K_L01286_2v}\edtext{Bernhardiner}{\lemma{\textnormal{\emph{Bernhardiner}}}\Cendnote{\textnormal{\textcolor{blue}{Schnitzler} besaß für kurze Zeit, vermutlich ab
                  dem 23. 3. 1902, einen
                  Bernhardiner namens »Bern«. Im Oktober wurde er in dem im gleichen
                  Monat eröffneten \textcolor{pink}{Tierschutzhaus} des \emph{\textcolor{brown}{Wiener Tierschutz-Vereins}} behandelt; Mitte
                     Dezember erneut. Ab Januar 1903 versucht er ihn zu
                  vermitteln, da wohnt er aber bereits nicht mehr bei ihnen (siehe Arthur Schnitzler an Richard Beer-Hofmann, 14. 1. 1903). In diesem Jahr finden sich noch drei
                  Erwähnungen im \emph{\textcolor{green}{Tagebuch}} (23. 5. 1903, 18. 6. 1903 und 6. 8. 1903). Vgl.
                        \emph{Briefe} II,118.}}}\label{K_L01286_2h} leider entſagen muß,
               vorläufig wenigſtens, da meine \textcolor{blue}{Frau}{}\ledrightnote{→\textcolor{blue}{Rosa Bahr}} gerade wieder die Laune hat, alle Hunde zu haßen.\pend
           \pstart
           Herzlichſt{\\[\baselineskip]}Dein{\\[\baselineskip]}\spacefill\mbox{Hermann }\pend
           \leftskip=0em{}\pstart
           \noindent{}{\pb}Die Widmung soll lauten:\pend
           \begin{mdbar}\pstart
           \noindent{}\centering{}Meinem lieben Arthur Schnitzler{\\}nach zwölf Jahren.\pend
           \end{mdbar}\endnumbering\briefempfaengerindex{Schnitzler, Arthur@\textsc{Schnitzler, Arthur}!zzzBahr, Hermann@\emph{von Hermann Bahr}!1903-04-041@{4. 4. 1903}|)be}\mylabel{h}  \normalsize

\doendnotes{C}
\bigskip
\vfill

\clearpage

\footnotesize

\lohead{\textsc{register}}

% Definiere theindex-Environment komplett neu ohne reledmac
\makeatletter
\renewenvironment{theindex}{%
  \section*{\indexname}%
  \setlength{\parindent}{0pt}%
  \setlength{\parskip}{0pt plus 0.3pt}%
  \let\item\@idxitem
}{%
  \clearpage
}
\makeatother

\IfFileExists{\jobname-pw.ind}{\input{\jobname-pw.ind}}{}

\end{document}

      