%% latex-korrekturansicht-vorspann.tex
%% Vorspann für die Korrekturansicht.
%% Lädt die gemeinsame Datei latex-vorspann.tex mit gesetztem Schalter.

\newif\ifkorrekturansicht
\korrekturansichttrue

\input{../tex-inputs/latex-vorspann}


               \section[Olga und Arthur Schnitzler an Richard und Paula Beer-Hofmann, 26. 6. 1908]{ Olga und Arthur Schnitzler an Richard und Paula Beer-Hofmann,
               26. 6. 1908}\nopagebreak\mylabel{v}\rehead{ }\normalsize\beginnumbering\briefempfaengerindex{Beer-Hofmann, Paula@\textsc{Beer-Hofmann, Paula}!zzzSchnitzler, Arthur@\emph{von Arthur Schnitzler}!1908-06-261@{26. 6. 1908}|(be}\briefempfaengerindex{Beer-Hofmann, Paula@\textsc{Beer-Hofmann, Paula}!zzzSchnitzler, Olga@\emph{von Olga Schnitzler}!1908-06-261@{26. 6. 1908}|(be}\briefempfaengerindex{Beer-Hofmann, Richard@\textsc{Beer-Hofmann, Richard}!zzzSchnitzler, Arthur@\emph{von Arthur Schnitzler}!1908-06-261@{26. 6. 1908}|(be}\briefempfaengerindex{Beer-Hofmann, Richard@\textsc{Beer-Hofmann, Richard}!zzzSchnitzler, Olga@\emph{von Olga Schnitzler}!1908-06-261@{26. 6. 1908}|(be} \toendnotes[C]{\smallbreak\pagebreak[2]} \Standort{YCGL, MSS 31.}
\physDesc{Bildpostkarte
\newline{}Handschrift Arthur Schnitzler: schwarze Tinte, deutsche Kurrent\newline{}Handschrift Olga Schnitzler: schwarze Tinte, lateinische Kurrent\newline{}Versand: Stempel: »\nobreak{}\oindex{Seis am Schlern@\textbf{Seis am Schlern}, \emph{Besiedelter Ort (A.BSO)}|pwk}Seis, 26. 6. {[}1906{]}\nobreak{}«.  }\toendnotes[C]{\smallbreak}\pstart{}{\pb}Herrn u. Frau\pend{}\pstart{}D\textsuperscript{r} Richard Beer-Hofmann\pend{}\pstart{}\textcolor{pink}{Wien XVIII}{}\ledrightnote{\textcolor{pink}{XVIII., Währing}}\pend{}\pstart{}\textcolor{pink}{Hasenauserstrasse 59}{}\ledrightnote{\textcolor{pink}{Hasenauerstraße}}.\pend{}{\bigskip}\pstart
           \noindent{}\centering{}{\pb}\textcolor{gray}{\textbf{\textcolor{pink}{Tirol}{}\ledrightnote{\textcolor{pink}{Südtirol}}: \textcolor{pink}{\label{T_L01778_1v}\edtext{\uline{Villa Heufler, Seis am Schlern}}{\lemma{\textnormal{\emph{Villa … Schlern}}}\Cendnote{\textnormal{Unterstreichung mit schwarzer
                           Tinte.}}}\label{T_L01778_1h}}{}\ledrightnote{\textcolor{pink}{Villa Heufler}}, 1000m. Nach dem \textcolor{green}{Aquarell}{}\ledrightnote{→\textcolor{green}{Partie in Seis am Schlern}} von \textcolor{blue}{F. A. C. M.
                     Reisch}{}\ledrightnote{\textcolor{blue}{Franz August Carl Maria Reisch}}, \textcolor{pink}{Meran}{}\ledrightnote{\textcolor{pink}{Meran}}.}}\pend
           \pstart
           \raggedleft{}26. Juni 08.\pend
           \pstart
           \label{T_L01778_2v}\edtext{Mein Fenster}{\lemma{\textnormal{\emph{Mein Fenster}}}\Cendnote{\textnormal{Ein Pfeil weist auf das Fenster links des Balkons im zweiten
                  Stock.}}}\label{T_L01778_2h}\pend
           \pstart
           \label{T_L01778_3v}\edtext{\textcolor{blue}{Heini}{}\ledrightnote{\textcolor{blue}{Heinrich Schnitzler}}}{\lemma{\textnormal{\emph{Heini}}}\Cendnote{\textnormal{Ein Pfeil weist auf das zweite Fenster
                  von links beim Balkon im zweiten Stock.}}}\label{T_L01778_3h}\pend
           \pstart
           Um’s Eck hab ich auch noch ein Fenster, daneben ist auch Arthurs Balcon-Zimmer.\pend
           \pstart
           \textcolor{pink}{Salegg}{}\ledrightnote{\textcolor{pink}{Hotel Salegg}} war rasch erledigt, da der schlaue {\pb}\textcolor{blue}{Wirt}{}\ledrightnote{→\textcolor{blue}{Michael Honeck}} die Mängel seines
               Hauses durch wohlwollendes Schulterklopfen zu ersetzen suchte.\pend
           \pstart
           Wir speisen im gegenüberliegenden \textcolor{pink}{Seiserhof}{}\ledrightnote{\textcolor{pink}{Seiserhof}},
               wollen lange bleiben. Heut ist \textcolor{blue}{Heini}{}\ledrightnote{\textcolor{blue}{Heinrich Schnitzler}}
                  einge\textcolor{gray}{t}roffen.\pend
           \pstart
           Die Wälder sind unglaublich schön. Hoffentlich sind Sie ebenso zufrieden, aber wo???
               Die allerherzlichsten Wünsche und Grüsse, auch den \textcolor{blue}{Kindern}{}\ledrightnote{→\textcolor{blue}{Naëmah Beer-Hofmann}{\newline}→\textcolor{blue}{Mirjam Beer-Hofmann}{\newline}→\textcolor{blue}{Gabriel Beer-Hofmann}}.\pend
           \pstart \spacefill\mbox{O. S.}\pend{}{\bigskip}\pstart
           \noindent{}{\pb}{[}hs. Schnitzler:{]} Herzlichſst\pend
           \pstart \spacefill\mbox{Arthur}\pend{}\endnumbering\briefempfaengerindex{Beer-Hofmann, Paula@\textsc{Beer-Hofmann, Paula}!zzzSchnitzler, Arthur@\emph{von Arthur Schnitzler}!1908-06-261@{26. 6. 1908}|)be}\briefempfaengerindex{Beer-Hofmann, Paula@\textsc{Beer-Hofmann, Paula}!zzzSchnitzler, Olga@\emph{von Olga Schnitzler}!1908-06-261@{26. 6. 1908}|)be}\briefempfaengerindex{Beer-Hofmann, Richard@\textsc{Beer-Hofmann, Richard}!zzzSchnitzler, Arthur@\emph{von Arthur Schnitzler}!1908-06-261@{26. 6. 1908}|)be}\briefempfaengerindex{Beer-Hofmann, Richard@\textsc{Beer-Hofmann, Richard}!zzzSchnitzler, Olga@\emph{von Olga Schnitzler}!1908-06-261@{26. 6. 1908}|)be}\mylabel{h}  \normalsize

\doendnotes{C}
\bigskip
\vfill

\clearpage

\footnotesize

\lohead{\textsc{register}}

% Definiere theindex-Environment komplett neu ohne reledmac
\makeatletter
\renewenvironment{theindex}{%
  \section*{\indexname}%
  \setlength{\parindent}{0pt}%
  \setlength{\parskip}{0pt plus 0.3pt}%
  \let\item\@idxitem
}{%
  \clearpage
}
\makeatother

\IfFileExists{\jobname-pw.ind}{\input{\jobname-pw.ind}}{}

\end{document}

      