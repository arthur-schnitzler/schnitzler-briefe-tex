%% latex-korrekturansicht-vorspann.tex
%% Vorspann für die Korrekturansicht.
%% Lädt die gemeinsame Datei latex-vorspann.tex mit gesetztem Schalter.

\newif\ifkorrekturansicht
\korrekturansichttrue

\input{../tex-inputs/latex-vorspann}


               \section[Hugo von Hofmannsthal an Arthur Schnitzler, 8. 12. {[}1903{]}]{ Hugo von Hofmannsthal an Arthur Schnitzler, 8. 12. {[}1903{]}}\nopagebreak\mylabel{v}\rehead{ }\normalsize\beginnumbering\briefempfaengerindex{Schnitzler, Arthur@\textsc{Schnitzler, Arthur}!zzzHofmannsthal, Hugo von@\emph{von Hugo von Hofmannsthal}!1903-12-081@{8. 12. {[}1903{]}}|(be} \toendnotes[C]{\smallbreak\pagebreak[2]} \Standort{CUL, Schnitzler, B 43.}
\physDesc{Brief, 1 Blatt, 4 Seiten
\newline{}Handschrift: schwarze Tinte, deutsche Kurrent
\newline{}Schnitzler: mit Bleistift die Jahreszahl ergänzt: »903.« \newline{}Ordnung: 1) mit Bleistift von unbekannter Hand nummeriert: »\strikeout{222}« 2) mit Bleistift von unbekannter Hand nummeriert:
                                    »206«}\buchAbdrucke{\weitereDrucke{Hugo von Hofmannsthal, Arthur Schnitzler: \emph{Briefwechsel}. Hg. Therese Nickl und Heinrich Schnitzler. Frankfurt am Main: \emph{S. Fischer} 1964, S. 178–179.} }\toendnotes[C]{\smallbreak}\pstart
           \raggedleft{}{\pb}8. XII.\pend
           \pstart{}lieber,\pend\pstart
           nun ſind es wieder vielleicht 4 Wochen, daſs man ſich nicht geſehen hat!\hspace*{1em}Iſt das nicht ſchad?\hspace*{1.5em}Und ich konnte diesmal abſolut nichts machen als warten, da Sie beim letzten Mal
               beſtimmt geſagt hatten, Sie würden herüberkommen. Wenn Ihnen aber das in der ganzen
                  {\pb}Zeit niemals paſste, warum
               dann kein \textsc{rendez-vous} in \textcolor{pink}{Hietzing}{}\ledrightnote{\textcolor{pink}{XIII., Hietzing}}? –\pend
           \pstart
           Dieſe Woche bin ich Mittwoch Samstag Sonntag beſtimmt nicht frei.\hspace*{1.5em}Daſs Sie auch nie eine Zeile ſchreiben! \pend
           \pstart
           Ich habe in der Zwiſchenzeit »\textcolor{green}{Frau Bertha \textsc{Garlan}}{}\ledrightnote{\textcolor{green}{Frau Bertha Garlan. Roman}}« wieder geleſen, mit noch viel {\pb}intenſiverem Vergnügen als das
               erſte mal, ja mit ungetrübtem Genuſs. Dieſes Buch und das neue \textcolor{green}{Stück}{}\ledrightnote{→\textcolor{green}{Der einsame Weg. Schauspiel in fünf Akten}} ſind wohl Ihre ſchönſten Arbeiten. Kaum
               zu glauben daſs das von einer Hand iſt, mit einem ſo dürren quälenden Buch wie »\textcolor{green}{Sterben}{}\ledrightnote{\textcolor{green}{Sterben. Novelle}}« einem Buch, wie es deren eigentlich keine
               geben dürfte. {\pb}So viel Kraft und
               Wärme, Überſicht, Tact, Weltgefühl und Herzenskenntnis ſteckt in dieſer »\textcolor{green}{Bertha \textsc{Garlan}}{}\ledrightnote{\textcolor{green}{Frau Bertha Garlan. Roman}}«, ſo ſchön zuſammengehalten iſt es und ſo gut und geſcheidt dabei.\pend
           \pstart
           Wenn Sie einmal ein überflüſſiges Exemplar der »\textcolor{green}{Frau des
                  Weiſen}{}\ledrightnote{\textcolor{green}{Die Frau des Weisen. Novelletten}}« haben, meins iſt geſtohlen.\pend
           \pstart
           Haben Sie nun ſchon die »\textcolor{green}{Elektra}{}\ledrightnote{\textcolor{green}{Elektra. Tragödie in einem Aufzug}}« oder nicht? –
               bekommen übrigens nächſtens auch noch etwas \label{K_L01347_1v}\edtext{\textcolor{green}{andres}{}\ledrightnote{→\textcolor{green}{Das gerettete Venedig. Trauerspiel in fünf Aufzügen}}}{\lemma{\textnormal{\emph{andres}}}\Cendnote{\textnormal{\textcolor{blue}{Schnitzler} rechnet damit, \emph{\textcolor{green}{Das gerettete Venedig}} zu bekommen; siehe Arthur Schnitzler an Hugo von Hofmannsthal, 10. 12. 1903}}}\label{K_L01347_1h}.\pend
           \pstart Von Herzen \spacefill\mbox{Hugo.}\pend{}\endnumbering\briefempfaengerindex{Schnitzler, Arthur@\textsc{Schnitzler, Arthur}!zzzHofmannsthal, Hugo von@\emph{von Hugo von Hofmannsthal}!1903-12-081@{8. 12. {[}1903{]}}|)be}\mylabel{h}  \normalsize

\doendnotes{C}
\bigskip
\vfill

\clearpage

\footnotesize

\lohead{\textsc{register}}

% Definiere theindex-Environment komplett neu ohne reledmac
\makeatletter
\renewenvironment{theindex}{%
  \section*{\indexname}%
  \setlength{\parindent}{0pt}%
  \setlength{\parskip}{0pt plus 0.3pt}%
  \let\item\@idxitem
}{%
  \clearpage
}
\makeatother

\IfFileExists{\jobname-pw.ind}{\input{\jobname-pw.ind}}{}

\end{document}

      