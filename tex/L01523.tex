%% latex-korrekturansicht-vorspann.tex
%% Vorspann für die Korrekturansicht.
%% Lädt die gemeinsame Datei latex-vorspann.tex mit gesetztem Schalter.

\newif\ifkorrekturansicht
\korrekturansichttrue

\input{../tex-inputs/latex-vorspann}


               \section[Arthur Schnitzler an Hermann Bahr, 6. 6. 1905]{ Arthur Schnitzler an Hermann Bahr, 6. 6. 1905}\nopagebreak\mylabel{v}\rehead{ }\normalsize\beginnumbering\briefempfaengerindex{Bahr, Hermann@\textsc{Bahr, Hermann}!zzzSchnitzler, Arthur@\emph{von Arthur Schnitzler}!1905-06-061@{6. 6. 1905}|(be} \toendnotes[C]{\smallbreak\pagebreak[2]} \Standort{TMW, HS AM 23374 Ba.}
\physDesc{Brief, 1 Blatt, 2 Seiten
\newline{}Handschrift: schwarze Tinte, deutsche Kurrent\newline{}Ordnung: Lochung }\buchAbdrucke{\weitereDrucke{1) \emph{6. 6. 1905.} In: Arthur Schnitzler: \emph{The Letters of Arthur Schnitzler to Hermann Bahr}. Edited, annotated, and with an introduction, by Donald G.
                        Daviau. Chapel Hill: \emph{The University of North Carolina Press} 1978, S. 89 (University of North Carolina studies in the Germanic languages
                        and literatures, 89).} \weitereDrucke{2) Hermann Bahr, Arthur Schnitzler: \emph{Briefwechsel, Aufzeichnungen, Dokumente (1891–1931)}. Hg. Kurt Ifkovits und Martin Anton Müller. Göttingen: \emph{Wallstein} 2018, S. 345.} }\toendnotes[C]{\smallbreak}\pstart
           \raggedleft{}{\pb}\textsc{\textcolor{pink}{Wien}{}\ledrightnote{\textcolor{pink}{Wien}}}{ }6. Juni 905\pend
           \pstart{}lieber Hermann\pend\pstart
           ich gratulire dir herzlich zum geſtrigen Erfolg von \textsc{\textcolor{green}{Sanna}{}\ledrightnote{\textcolor{green}{Sanna. Schauspiel in fünf Aufzügen}}}. Einiges was mir nach der erſten Lectüre des Stücks nicht ganz eingeleuchtet,
               iſt mir geſtern, ſchön und ergreifend aufgegangen. Die Aufführung war etwas ganz
               einziges, und die \textcolor{blue}{Höflich}{}\ledrightnote{\textcolor{blue}{Lucie Höflich}}{ }{\pb}iſt – vielleicht nicht
               das echte Genie, aber, nach ihren Entwicklungsmöglichkeiten in alles tragiſche und
               heitre Gebiet, der größte Glücksfall, den die Deutſche Bühne ſeit der \textcolor{blue}{Sorma}{}\ledrightnote{\textcolor{blue}{Agnes Sorma}} erlebt hat.\pend
           \pstart
           Ich habe mich ſehr gefreut, auch meine \textcolor{blue}{Frau}{}\ledrightnote{→\textcolor{blue}{Olga Schnitzler}} läßt dir von Herzen glückwünſchen.\pend
           \pstart
           Hoffentlich ſeh ich dich bald; ich habe ein rechtes Bedürfnis, dir zu danken.\pend
           \pstart
           Dein{\\[\baselineskip]}\spacefill\mbox{Arthur}\pend
           \leftskip=0em{}\endnumbering\briefempfaengerindex{Bahr, Hermann@\textsc{Bahr, Hermann}!zzzSchnitzler, Arthur@\emph{von Arthur Schnitzler}!1905-06-061@{6. 6. 1905}|)be}\mylabel{h}  \normalsize

\doendnotes{C}
\bigskip
\vfill

\clearpage

\footnotesize

\lohead{\textsc{register}}

% Definiere theindex-Environment komplett neu ohne reledmac
\makeatletter
\renewenvironment{theindex}{%
  \section*{\indexname}%
  \setlength{\parindent}{0pt}%
  \setlength{\parskip}{0pt plus 0.3pt}%
  \let\item\@idxitem
}{%
  \clearpage
}
\makeatother

\IfFileExists{\jobname-pw.ind}{\input{\jobname-pw.ind}}{}

\end{document}

      