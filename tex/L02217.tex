%% latex-korrekturansicht-vorspann.tex
%% Vorspann für die Korrekturansicht.
%% Lädt die gemeinsame Datei latex-vorspann.tex mit gesetztem Schalter.

\newif\ifkorrekturansicht
\korrekturansichttrue

\input{../tex-inputs/latex-vorspann}


               \section[Arthur Schnitzler an Richard Beer-Hofmann, {[}3. 8. 1915{]}]{ Arthur Schnitzler an Richard Beer-Hofmann, {[}3. 8. 1915{]}}\nopagebreak\mylabel{v}\rehead{ }\normalsize\beginnumbering\briefempfaengerindex{Beer-Hofmann, Richard@\textsc{Beer-Hofmann, Richard}!zzzSchnitzler, Arthur@\emph{von Arthur Schnitzler}!1915-08-031@{{[}3. 8. 1915{]}}|(be} \toendnotes[C]{\smallbreak\pagebreak[2]} \Standort{YCGL, MSS 31.}
\physDesc{Visitenkarte, Umschlag
\newline{}Handschrift: Bleistift, deutsche Kurrent\newline{}Versand: ohne postalischen Übermittlungsvermerk 
\newline{}Beer-Hofmann: mit Bleistift datiert: »{\pb}3/8 15« }\pstart{}{\pb}\textsc{Herrn Dr Richard Beer-Hofma{\geminationn}}\pend{}\pstart{}\textsc{\textcolor{pink}{Ischl}{}\ledrightnote{\textcolor{pink}{Bad Ischl}}}\pend{}{\bigskip}\pstart
           \noindent{}{\pb}lieber Richard, wir ſind zum Nachtmahl bei \textcolor{pink}{\textsc{Sonnenschein}}{}\ledrightnote{\textcolor{pink}{Restaurant Sonnenschein}}, auch \textcolor{blue}{Kaufma{\geminationn}}{}\ledrightnote{\textcolor{blue}{Arthur Kaufmann}}\pend
           \pstart {\pb}Herzlichſt\pend{}\pstart
           \centering{}\textcolor{gray}{\textbf{\strikeout{D\textsuperscript{r}} Arthur \strikeout{Schnitzler}}}\pend
           \endnumbering\briefempfaengerindex{Beer-Hofmann, Richard@\textsc{Beer-Hofmann, Richard}!zzzSchnitzler, Arthur@\emph{von Arthur Schnitzler}!1915-08-031@{{[}3. 8. 1915{]}}|)be}\mylabel{h}  \normalsize

\doendnotes{C}
\bigskip
\vfill

\clearpage

\footnotesize

\lohead{\textsc{register}}

% Definiere theindex-Environment komplett neu ohne reledmac
\makeatletter
\renewenvironment{theindex}{%
  \section*{\indexname}%
  \setlength{\parindent}{0pt}%
  \setlength{\parskip}{0pt plus 0.3pt}%
  \let\item\@idxitem
}{%
  \clearpage
}
\makeatother

\IfFileExists{\jobname-pw.ind}{\input{\jobname-pw.ind}}{}

\end{document}

      