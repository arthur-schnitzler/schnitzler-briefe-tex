%% latex-korrekturansicht-vorspann.tex
%% Vorspann für die Korrekturansicht.
%% Lädt die gemeinsame Datei latex-vorspann.tex mit gesetztem Schalter.

\newif\ifkorrekturansicht
\korrekturansichttrue

\input{../tex-inputs/latex-vorspann}


               \section[Hugo von Hofmannsthal an Arthur Schnitzler, 21. 9. 1912]{ Hugo von Hofmannsthal an Arthur Schnitzler, 21. 9. 1912}\nopagebreak\mylabel{v}\rehead{ }\normalsize\beginnumbering\briefempfaengerindex{Schnitzler, Arthur@\textsc{Schnitzler, Arthur}!zzzHofmannsthal, Hugo von@\emph{von Hugo von Hofmannsthal}!1912-09-211@{21. 9. 1912}|(be} \toendnotes[C]{\smallbreak\pagebreak[2]} \Standort{CUL, Schnitzler, B 43.}
\physDesc{Bildpostkarte
\newline{}Handschrift: schwarze Tinte, deutsche Kurrent\newline{}Versand: Stempel: »\nobreak{}\oindex{Sankt Michael@\textbf{Sankt Michael}, \emph{Bezirk (A.BZK)}|pwk}St. Michael in Eppan, 22. IX. 12\nobreak{}«.  
\newline{}Schnitzler: mit Bleistift die Jahreszahl ergänzt: »912« \newline{}Ordnung: 1) mit Bleistift von unbekannter Hand nummeriert: »\strikeout{330}« 2) mit Bleistift von unbekannter Hand nummeriert: »340«}\buchAbdrucke{\weitereDrucke{Hugo von Hofmannsthal, Arthur Schnitzler: \emph{Briefwechsel}. Hg. Therese Nickl und Heinrich Schnitzler. Frankfurt am Main: \emph{S. Fischer} 1964, S. 269.} }\toendnotes[C]{\smallbreak}\pstart{}{\pb}\textsc{Herrn
                        D\textsuperscript{r} Arthur Schnitzler}\pend{}\pstart{}\textcolor{pink}{\textsc{Wien}}{}\ledrightnote{\textcolor{pink}{Wien}}\pend{}\pstart{}\textsc{\textcolor{pink}{XVIII Sternwartestrasse 71}{}\ledrightnote{\textcolor{pink}{Sternwartestraße}}}.\pend{}{\bigskip}\pstart
           \noindent{}\centering{}\textcolor{gray}{\textbf{{\pb}\textcolor{pink}{Schloss Gandegg}{}\ledrightnote{\textcolor{pink}{Schloss Gandegg}} in \textcolor{pink}{Eppan (Überetsch)}{}\ledrightnote{\textcolor{pink}{Eppan an der Weinstraße}}. \textcolor{pink}{Tirol}{}\ledrightnote{\textcolor{pink}{Südtirol}}.}}\pend
           \pstart
           {\pb}\textcolor{pink}{Gandegg}{}\ledrightnote{\textcolor{pink}{Schloss Gandegg}}{ }21. IX.\pend
           \pstart
           Dies \textcolor{pink}{Schloſs}{}\ledrightnote{→\textcolor{pink}{Schloss Gandegg}}{ }ſteht leer, wir habens gemiethet und genießen ein
               letztes oder erſtes Stück So{\geminationm}er. Ich verſuche – was Sie
               beim letzten Mal als Wunſch ausgeſprochen haben, mein lieber Arthur: – zu erzählen.
               Der \textcolor{green}{Stoff}{}\ledrightnote{→\textcolor{green}{Andreas oder Die Vereinigten}} iſt ſchön, ich will mir
               viel Mühe geben. Von Herzen\pend
           \pstart Ihr \spacefill\mbox{Hugo.}\pend{}\endnumbering\briefempfaengerindex{Schnitzler, Arthur@\textsc{Schnitzler, Arthur}!zzzHofmannsthal, Hugo von@\emph{von Hugo von Hofmannsthal}!1912-09-211@{21. 9. 1912}|)be}\mylabel{h}  \normalsize

\doendnotes{C}
\bigskip
\vfill

\clearpage

\footnotesize

\lohead{\textsc{register}}

% Definiere theindex-Environment komplett neu ohne reledmac
\makeatletter
\renewenvironment{theindex}{%
  \section*{\indexname}%
  \setlength{\parindent}{0pt}%
  \setlength{\parskip}{0pt plus 0.3pt}%
  \let\item\@idxitem
}{%
  \clearpage
}
\makeatother

\IfFileExists{\jobname-pw.ind}{\input{\jobname-pw.ind}}{}

\end{document}

      