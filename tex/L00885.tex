%% latex-korrekturansicht-vorspann.tex
%% Vorspann für die Korrekturansicht.
%% Lädt die gemeinsame Datei latex-vorspann.tex mit gesetztem Schalter.

\newif\ifkorrekturansicht
\korrekturansichttrue

\input{../tex-inputs/latex-vorspann}


               \section[Arthur Schnitzler an Richard Beer-Hofmann, 7. 2. 1899]{ Arthur Schnitzler an Richard Beer-Hofmann, 7. 2. 1899}\nopagebreak\mylabel{v}\rehead{ }\normalsize\beginnumbering\briefempfaengerindex{Beer-Hofmann, Richard@\textsc{Beer-Hofmann, Richard}!zzzSchnitzler, Arthur@\emph{von Arthur Schnitzler}!1899-02-071@{7. 2. 1899}|(be} \toendnotes[C]{\smallbreak\pagebreak[2]} \Standort{YCGL, MSS 31.}
\physDesc{Briefkarte, Umschlag
\newline{}Handschrift: Bleistift, deutsche Kurrent\newline{}Versand: Stempel: »\nobreak{}\oindex{I., Innere Stadt@\textbf{I., Innere Stadt}, \emph{Bezirk (A.BZK)}|pwk}Wien 1/1, {[}7.{]} 2. 99, 10–11 N\nobreak{}«.  }\buchAbdrucke{\weitereDrucke{Arthur Schnitzler, Richard Beer-Hofmann: \emph{Briefwechsel 1891–1931}. Hg. Konstanze Fliedl. Wien, Zürich: \emph{Europaverlag} 1992, S. 126–127.} }\toendnotes[C]{\smallbreak}\pstart{}{\pb}Herrn \textsc{Dr. Rich
                     Beer-Hofmann}\pend{}\pstart{}\textcolor{pink}{Wien}{}\ledrightnote{\textcolor{pink}{Wien}}\pend{}\pstart{}\textsc{\textcolor{pink}{I. Wollzeile 15}{}\ledrightnote{\textcolor{pink}{Wollzeile}}}.\pend{}{\bigskip}\pstart
           \noindent{}{\pb}Lieber Richard, für \label{K_L00885_1v}\edtext{\textcolor{green}{Freitag}{}\ledrightnote{→\textcolor{green}{Unser Käthchen. Lustspiel in 4 Acten}}}{\lemma{\textnormal{\emph{Freitag}}}\Cendnote{\textnormal{Aufführung von \emph{\textcolor{green}{Unser Käthchen}} im \textcolor{pink}{Deutschen Volkstheater}.}}}\label{K_L00885_1h}{ }ſind keine ordentlichen Nebeneinander-Sitze mehr zu
               haben. Sie kö{\geminationn}en alſo \label{K_L00885_2v}\edtext{nix ä hin kommen ſtuppen}{\lemma{\textnormal{\emph{nix ä hin kommen ſtuppen}}}\Cendnote{\textnormal{ugs. für: nicht einfach kommen, um durch Anstuppsen der
                  richtigen Person das Gewünschte erhalten.}}}\label{K_L00885_2h}. Werden wir noch die Erfindung
               des Teleſtupp erleben?\pend
           \pstart Herzlich Ihr \spacefill\mbox{Arthur}\pend{}\pstart
           7/2 99\pend
           \endnumbering\briefempfaengerindex{Beer-Hofmann, Richard@\textsc{Beer-Hofmann, Richard}!zzzSchnitzler, Arthur@\emph{von Arthur Schnitzler}!1899-02-071@{7. 2. 1899}|)be}\mylabel{h}  \normalsize

\doendnotes{C}
\bigskip
\vfill

\clearpage

\footnotesize

\lohead{\textsc{register}}

% Definiere theindex-Environment komplett neu ohne reledmac
\makeatletter
\renewenvironment{theindex}{%
  \section*{\indexname}%
  \setlength{\parindent}{0pt}%
  \setlength{\parskip}{0pt plus 0.3pt}%
  \let\item\@idxitem
}{%
  \clearpage
}
\makeatother

\IfFileExists{\jobname-pw.ind}{\input{\jobname-pw.ind}}{}

\end{document}

      