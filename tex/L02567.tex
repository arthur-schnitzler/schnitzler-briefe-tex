%% latex-korrekturansicht-vorspann.tex
%% Vorspann für die Korrekturansicht.
%% Lädt die gemeinsame Datei latex-vorspann.tex mit gesetztem Schalter.

\newif\ifkorrekturansicht
\korrekturansichttrue

\input{../tex-inputs/latex-vorspann}


               \section[Therese Rie-Andro an Arthur Schnitzler, 22. 12. 1929]{ Therese Rie-Andro an Arthur Schnitzler, 22. 12. 1929}\nopagebreak\mylabel{v}\rehead{ }\normalsize\beginnumbering\briefempfaengerindex{Schnitzler, Arthur@\textsc{Schnitzler, Arthur}!zzzRie, Therese@\emph{von Therese Rie}!1929-12-222@{22. 12. 1929}|(be} \toendnotes[C]{\smallbreak\pagebreak[2]} \Standort{CUL, Schnitzler, B 82.}
\physDesc{Brief, 1 Blatt, 2 Seiten
\newline{}Schreibmaschine
\newline{}Handschrift: schwarze Tinte, lateinische Kurrent (\noindent{}marginale Korrekturen, Schlussformel und Unterschrift)
\newline{}Schnitzler: mit rotem Buntstift beschriftet: »\textsc{\textcolor{green}{So{\geminationm}erlüfte}}« und mehrere Unterstreichungen }\toendnotes[C]{\smallbreak}\pstart
           \noindent{}{\pb}\textcolor{gray}{\textbf{THERESE RIE-ANDRO}}\hfill \textcolor{gray}{\textbf{\textcolor{pink}{WIEN, IV.}{}\ledrightnote{\textcolor{pink}{IV., Wieden}}}}\pend
           \pstart
           \raggedleft{}\textcolor{gray}{\textbf{\textcolor{pink}{SCHÖNBURGSTRASSE 48}{}\ledrightnote{\textcolor{pink}{Schönburgstraße}}}}\pend
           \pstart
           \raggedleft{}22/12/29\pend
           \pstart{}Verehrter Herr Doktor,\pend\pstart
           Ich möchte Ihnen nur danken für das bezaubernde \textcolor{green}{Stück}{}\ledrightnote{→\textcolor{green}{Im Spiel der Sommerlüfte. In drei Aufzügen}} Leben und Athmosphäre, das Sie \label{K_L02567-1v}\edtext{gestern}{\lemma{\textnormal{\emph{gestern}}}\Cendnote{\textnormal{siehe A. S.: \emph{Tagebuch}, 21. 12. 1929}}}\label{K_L02567-1h} vor uns haben erstehen lassen. Es ist wol in jedem Ihrer Stücke so, dass sich
               einem Selbsterlebtes zur Allgemeingiltigkeit sublimiert. Oh, wie gut kannte man sie,
               diese stillen Villen, eine Stunde und doch weltenweit von der Stadt, in deren weichem
               und etwas feuchtem Grün Frauen und Kinder von Mai bis September spielten und
               träumten. Es war nicht immer ein ganz gutes Träumen, das zeigt ihr \textcolor{green}{Stück}{}\ledrightnote{→\textcolor{green}{Im Spiel der Sommerlüfte. In drei Aufzügen}}, das mit so zarter Hand einen Schleier
               von Gesichtern entfernt, die uns \strikeout{so} vertraut und im
               Grunde doch fremd waren: von denen unserer Mütter. Man war schon rebellischer, man
               wollte nicht mehr so pflanzenhaft passiv dahinleben, man hielt für unlebendig, wo nur
               tiefstes Verbergen war; man begriff urplötzlich hervorquellende Bitterkeiten nicht.
               Das und noch so viel anderes lehrt Ihr \textcolor{green}{Stück}{}\ledrightnote{→\textcolor{green}{Im Spiel der Sommerlüfte. In drei Aufzügen}} verstehen – wann hätte ein Werk von Ihnen einen nicht das Leben besser
               verstehen gelehrt!\pend
           \pstart
           Und \textcolor{green}{Gusti}{}\ledrightnote{→\textcolor{green}{Im Spiel der Sommerlüfte. In drei Aufzügen}}! Ich kannte \textcolor{green}{Gusti}{}\ledrightnote{→\textcolor{green}{Im Spiel der Sommerlüfte. In drei Aufzügen}} persönlich; immer war man
               Freund Ihrer Gestalten. \textcolor{green}{Gusti}{}\ledrightnote{→\textcolor{green}{Im Spiel der Sommerlüfte. In drei Aufzügen}} war
               eine heissverehrte Freundin (bei den Eltern weniger beliebt!), die man still
               bewunderte, weil sie so gut konnte, was man selbst nicht fertig brachte, weil ihre
               Unternehmungslust nicht von den Gedanken gehemmt war, dass der Mensch in einem
               gewissen Alter doch eigentlich nur aus Ellenbogen und linken Füssen besteht. (Das
               Wort »sex-appeal« war noch nicht erfunden.) Ich hoffe, Sie haben das junge Fräulein
                  \textcolor{blue}{Ullrich}{}\ledrightnote{\textcolor{blue}{Luise Ullrich}}, die ich bisher noch garnicht kannte,
               ebenso entzückend gefunden wie ich: so ganz echt und am meisten, wo sie lügt!\pend
           \pstart
           Ueberhaupt eine Aufführung, der man anmerkte, dass nicht nur gewöhnliche Regiearbeit
               geleistet worden war. Ueber \textcolor{blue}{Moissi}{}\ledrightnote{\textcolor{blue}{Alexander Moissi}}{ }{\pb}freilich möchte ich lieber nicht
               sprechen; er ist Ihnen gewiss lieb und auch persönlich ein anziehender Mensch. Aber
               er ist immer aus \textcolor{pink}{Neapel}{}\ledrightnote{\textcolor{pink}{Neapel}} an der \textcolor{pink}{Newa}{}\ledrightnote{\textcolor{pink}{Newa}} – nie aus \textcolor{pink}{Österreich}{}\ledrightnote{\textcolor{pink}{Österreich}}{\dots}\pend
           \pstart
           Ich habe noch keine Kritiken gelesen und ich denke mir, es wird einen Ueberfluss an
               schönen Worten von Seiten der Herren geben, die ja alles besser wissen. Ich möchte
               Ihnen, verehrter Herr Doktor, nur ganz einfach und persönlich sagen, wie ganz
               mitgenommen ich von jeder Szene war, und wie ganz mir Ihr \textcolor{green}{Stück}{}\ledrightnote{→\textcolor{green}{Im Spiel der Sommerlüfte. In drei Aufzügen}} das \textcolor{blue}{Shakespeare}{}\ledrightnote{\textcolor{blue}{William Shakespeare}}’sche Wort zu erfüllen schien: »\label{K_L02567-2v}\edtext{Sind wir ein Spiel von jedem Druck der Luft}{\lemma{\textnormal{\emph{Sind … Luft}}}\Cendnote{\textnormal{richtig: \textcolor{blue}{Goethe}, \emph{\textcolor{green}{Faust I}}}}}\label{K_L02567-2h}«. Denn immer
               noch sind es die Abenteuer der Seele, die uns am tiefsten ans Herz rühren!\pend
           \pstart
           In Dankbarkeit und Verehrung{\\[\baselineskip]}{[}hs.:{]} Ihre{\\[\baselineskip]}\spacefill\mbox{ThereseRie-Andro.}\pend
           \leftskip=0em{}\endnumbering\briefempfaengerindex{Schnitzler, Arthur@\textsc{Schnitzler, Arthur}!zzzRie, Therese@\emph{von Therese Rie}!1929-12-222@{22. 12. 1929}|)be}\mylabel{h}  \normalsize

\doendnotes{C}
\bigskip
\vfill

\clearpage

\footnotesize

\lohead{\textsc{register}}

% Definiere theindex-Environment komplett neu ohne reledmac
\makeatletter
\renewenvironment{theindex}{%
  \section*{\indexname}%
  \setlength{\parindent}{0pt}%
  \setlength{\parskip}{0pt plus 0.3pt}%
  \let\item\@idxitem
}{%
  \clearpage
}
\makeatother

\IfFileExists{\jobname-pw.ind}{\input{\jobname-pw.ind}}{}

\end{document}

      