%% latex-korrekturansicht-vorspann.tex
%% Vorspann für die Korrekturansicht.
%% Lädt die gemeinsame Datei latex-vorspann.tex mit gesetztem Schalter.

\newif\ifkorrekturansicht
\korrekturansichttrue

\input{../tex-inputs/latex-vorspann}


               \section[Arthur Schnitzler an Hugo von Hofmannsthal, 2. 10. 1904]{ Arthur Schnitzler an Hugo von Hofmannsthal, 2. 10. 1904}\nopagebreak\mylabel{v}\rehead{ }\normalsize\beginnumbering\briefempfaengerindex{Hofmannsthal, Hugo von@\textsc{Hofmannsthal, Hugo von}!zzzSchnitzler, Arthur@\emph{von Arthur Schnitzler}!1904-10-021@{2. 10. 1904}|(be} \toendnotes[C]{\smallbreak\pagebreak[2]} \Standort{FDH, Hs-30885,116.}
\physDesc{Kartenbrief
\newline{}Handschrift: schwarze Tinte, deutsche Kurrent\newline{}Versand: 1) Stempel: »\nobreak{}\oindex{XVIII., Waehring@\textbf{XVIII., Währing}, \emph{Bezirk (A.BZK)}|pwk}{[}Wi{]}en 110, 3. X. 04, IX\nobreak{}«.  2) Stempel: »\nobreak{}\oindex{Rodaun@\textbf{Rodaun}, \emph{Teil eines besiedelten Ortes (A.BSOX)}|pwk}Rodaun, 3. {[}10.{]} 04\nobreak{}«. }\buchAbdrucke{\weitereDrucke{Hugo von Hofmannsthal, Arthur Schnitzler: \emph{Briefwechsel}. Hg. Therese Nickl und Heinrich Schnitzler. Frankfurt am Main: \emph{S. Fischer} 1964, S. 203.} }\toendnotes[C]{\smallbreak}\pstart{}{\pb}\damage{Herr}n \textsc{Dr Hugo v Hofmannsthal}\pend{}\pstart{}\textcolor{pink}{\textsc{Rodaun \textsuperscript{b}/Liesing}}{}\ledrightnote{\textcolor{pink}{Rodaun}}\pend{}\pstart{}\textcolor{pink}{\textsc{Badgasse 5}}{}\ledrightnote{\textcolor{pink}{Badgasse}}. \pend{}{\bigskip}\pstart
           \raggedleft{}{\pb}\textcolor{pink}{Wien}{}\ledrightnote{\textcolor{pink}{Wien}}, 2. 10. 904\pend
           \pstart
           lieber, in d\substVorne{}\textsuperscript{er}\substDazwischen{}ieſer\substHinten{} Woche werden wir uns kaum ſehen können; – es fügt ſich gerade, daſs allerlei
                  zuſa{\geminationm}enko{\geminationm}t: \label{K_L01451_1v}\edtext{\textcolor{blue}{\textsc{Duse}}{}\ledrightnote{\textcolor{blue}{Eleonora Duse}}}{\lemma{\textnormal{\emph{Duse}}}\Cendnote{\textnormal{Er besuchte am 6. 10. 1904 das Gastspiel
                  von \textcolor{blue}{Eleonora Duse} am \emph{\textcolor{brown}{Theater an der Wien}} in der Hauptrolle von \emph{\textcolor{green}{Die Kameliendame}}.}}}\label{K_L01451_1h}, \textcolor{pink}{Burgtheater}{}\ledrightnote{\textcolor{pink}{Burgtheater}} (\label{K_L01451_2v}\edtext{\textcolor{green}{Heinrich}{}\ledrightnote{\textcolor{green}{Heinrich V.}}}{\lemma{\textnormal{\emph{Heinrich}}}\Cendnote{\textnormal{am 8. 10. 1904}}}\label{K_L01451_2h}), \label{K_L01451_3v}\edtext{\textcolor{pink}{Josefſtadt}{}\ledrightnote{\textcolor{pink}{Theater in der Josefstadt}}}{\lemma{\textnormal{\emph{Josefſtadt}}}\Cendnote{\textnormal{Am 5. 10. 1904 besuchte er \emph{\textcolor{green}{Herzogin Crevette. Schauspiel in fünf Acten}} von \textcolor{blue}{Georges Feydeau}.}}}\label{K_L01451_3h}, Familie, und ſo müſſen
               wir das abendliche \textcolor{pink}{Hietzing}{}\ledrightnote{\textcolor{pink}{XIII., Hietzing}} auf Beginn nächſter Woche
               verſchieben. Nachmittags arbeite ich ſo viel als möglich. Wie iſt Ihre Eintheilung?
               Wenn man einmal in den Vormittagsſtunden nach \textcolor{pink}{Rodaun}{}\ledrightnote{\textcolor{pink}{Rodaun}}
               käme, (wofür ich freilich nicht garantiren kann) würde man Sie ſtören?\pend
           \pstart
           Die \textcolor{green}{Bücher}{}\ledrightnote{→\textcolor{green}{Entweder – Oder}{\newline}→\textcolor{green}{Kunst und Künstler}} haben Sie
               bekommen? \pend
           \pstart
           Von Herzen Ihr{\\[\baselineskip]}\spacefill\mbox{Arthur}\pend
           \leftskip=0em{}\endnumbering\briefempfaengerindex{Hofmannsthal, Hugo von@\textsc{Hofmannsthal, Hugo von}!zzzSchnitzler, Arthur@\emph{von Arthur Schnitzler}!1904-10-021@{2. 10. 1904}|)be}\mylabel{h}  \normalsize

\doendnotes{C}
\bigskip
\vfill

\clearpage

\footnotesize

\lohead{\textsc{register}}

% Definiere theindex-Environment komplett neu ohne reledmac
\makeatletter
\renewenvironment{theindex}{%
  \section*{\indexname}%
  \setlength{\parindent}{0pt}%
  \setlength{\parskip}{0pt plus 0.3pt}%
  \let\item\@idxitem
}{%
  \clearpage
}
\makeatother

\IfFileExists{\jobname-pw.ind}{\input{\jobname-pw.ind}}{}

\end{document}

      