%% latex-korrekturansicht-vorspann.tex
%% Vorspann für die Korrekturansicht.
%% Lädt die gemeinsame Datei latex-vorspann.tex mit gesetztem Schalter.

\newif\ifkorrekturansicht
\korrekturansichttrue

\input{../tex-inputs/latex-vorspann}


               \section[Karl Kraus an Arthur Schnitzler, 27. 7. 1893]{ Karl Kraus an Arthur Schnitzler, 27. 7. 1893}\nopagebreak\mylabel{v}\rehead{ }\normalsize\beginnumbering\briefempfaengerindex{Schnitzler, Arthur@\textsc{Schnitzler, Arthur}!zzzKraus, Karl@\emph{von Karl Kraus}!1893-07-271@{27. 7. 1893}|(be} \toendnotes[C]{\smallbreak\pagebreak[2]} \Standort{CUL, Schnitzler, B 55.}
\physDesc{Postkarte
\newline{}Handschrift: schwarze Tinte, deutsche Kurrent\newline{}Versand: Stempel: »\nobreak{}\oindex{Bad Ischl@\textbf{Bad Ischl}, \emph{Besiedelter Ort (A.BSO)}|pwk}Ischl, 27/7 93, 1–N\nobreak{}«.  
\newline{}Schnitzler: mit Bleistift seitlich des Textes neben die »\textcolor{brown}{Fr. Bühne}«: »|| \textsc{\textcolor{blue}{Hirschfeld}–\textcolor{blue}{Wengraf} –
                                          \textcolor{gray}{frei}? ||}« }\buchAbdrucke{\weitereDrucke{\emph{Karl Kraus und Arthur Schnitzler. Eine Dokumentation.} Hg. Reinhard Urbach. In: \emph{Literatur und Kritik}, Bd. 49, Oktober 1970, S. 519–520.} }\toendnotes[C]{\smallbreak}\pstart{}{\pb}Herrn Doktor Arthur
                  Schnitzler,\pend{}\pstart{}Schriftſteller\pend{}\pstart{}\textcolor{pink}{I. Grillparzerstr. 7}{}\ledrightnote{\textcolor{pink}{Grillparzerstraße}}\pend{}\pstart{}\textcolor{pink}{Wien}{}\ledrightnote{\textcolor{pink}{Wien}}\pend{}{\bigskip}\pstart
           \noindent{}{\pb}\uline{Innigſten Dank}, liebſter Doktor,  für den
               lieben Brief! Beifolgend das letzte \uline{\textcolor{green}{Magazin}{}\ledrightnote{\textcolor{green}{Magazin für die Literatur des Auslandes}}}, das ich erſt heute bekam; es ſteht eine \label{K_L00244_1v}\edtext{\textcolor{green}{Nachricht}{}\ledrightnote{→\textcolor{green}{Wiener Theater. – Luise Sigert. Auferstanden!}}}{\lemma{\textnormal{\emph{Nachricht}}}\Cendnote{\textnormal{Diese Karte bezieht sich auf ein
                  Gerichtsverfahren, das am 24. und 25. 7. 1893 in \textcolor{pink}{Wien} wegen sexuell zu expliziter Veröffentlichungen
                  in einer Wochenschrift namens \emph{\textcolor{brown}{Gesellschaft}}
                  verhandelt wurde. Dabei wurden \textcolor{blue}{Moriz Ehrenfeld},
                     \textcolor{blue}{Ferdinand Mautner} und \textcolor{blue}{Alfred Brehmer} zu mehrmonatigen Haftstrafen verurteilt.
                  Verteidigt wurden die letzteren beiden von \textcolor{blue}{Friedrich
                     Elbogen}. Mit \textcolor{blue}{Brehmer} gibt es dabei eine
                  Überlappung zu einer weiteren Zeitschrift, \emph{\textcolor{brown}{Wiener
                     Kunst}}, wobei beide Zeitschriften nicht erhalten sind. Der Konnex, den \textcolor{blue}{Kraus} herstellt, bezieht sich auf den letzten
                  Absatz seines Theaterbriefs, erschienen am 22. 7. 1893; in \emph{\textcolor{green}{Wiener Theater. – \textcolor{blue}{Luise
                        Sigert}. \emph{\textcolor{green}{Auferstanden!}}}} (\emph{\textcolor{green}{Das Magazin für Litteratur}}, Jg. 62, Nr. 29,
                     S. 466–467.) endet \textcolor{blue}{Kraus} mit einer
                  Kritik an der Zeitschrift \emph{\textcolor{brown}{Wiener Kunst}} und
                  erwähnt eine geplante Musteraufführung von \emph{\textcolor{green}{Die
                     Weber}} von \textcolor{blue}{Gerhart Hauptmann}. Die \textcolor{pink}{Wien}er \emph{\textcolor{brown}{Freie
                  Bühne}}, bei der unter anderem auch \textcolor{blue}{Robert
                     Hirschfeld} und \textcolor{blue}{Edmund Wengraf}
                  federführend waren, sollte nunmehr unter der Führung von dem Verteidiger \textcolor{blue}{Elbogen} umgesetzt werden. Im nächsten Heft
                  erschien eine ungezeichnete Meldung, die auch von \textcolor{blue}{Kraus} stammen dürfte und ausführlicher auf das (nicht verwirklichte)
                  Theatervorhaben eingeht (\emph{\textcolor{green}{[Eine Freie Bühne]}}, Nr. 30, S. 484).}}}\label{K_L00244_1h}, wie ich eben erſt
               vor 1 Min. entdeckte, drin, die Sie als von einem in dieſen Mittheil. ſehr
               competenten Blatte{ }\introOben{}aus\introOben{} gewiss \uline{freuen} wird.
               Glückauf! – Hauptmacher der \textcolor{brown}{Fr. Bühne}{}\ledrightnote{\textcolor{brown}{»Freie Bühne« Verein für moderne Literatur}} iſt ja doch
               die »\textcolor{brown}{Wiener Kunst}{}\ledrightnote{\textcolor{brown}{Wiener Kunst}}« – Revolverblatt!!!! Redacteur \textcolor{blue}{Brehmer}{}\ledrightnote{\textcolor{blue}{Arthur Brehmer}} hat ſich \strikeout{ja}
               jezt auf \label{K_L00244_2v}\edtext{\uline{4 Monate} zurückgezogen}{\lemma{\textnormal{\emph{4 Monate zurückgezogen}}}\Cendnote{\textnormal{D. h. er wurde zu vier Monaten Arrest verurteilt ([O. V.]:
                        \emph{\textcolor{green}{Vergehen gegen die Sittlichkeit – Schluß}}.
                     In: \emph{\textcolor{green}{Neue Freie Presse}}, Nr. 10.388,
                        25. 7. 1893, S. 6).}}}\label{K_L00244_2h}.\pend
           \pstart
           Was ſagen Sie zu dem Proceſſe, der genialen Rede \textcolor{blue}{Elbogens}{}\ledrightnote{\textcolor{blue}{Friedrich Elbogen}} von der \label{K_L00244_3v}\edtext{Hemmung d.
                  \uline{Naturalismus} (!) i. der Kunſt übhpt.}{\lemma{\textnormal{\emph{Hemmung … übhpt.}}}\Cendnote{\textnormal{In seiner Verteidigung hatte \textcolor{blue}{Elbogen} den größeren Zusammenhang hergestellt:
                     »Es handle sich vielmehr um die Hemmung einer neuen Kunstrichtung, des
                     Naturalismus. \textsc{Principiis obsta}. Wenn Sie diesen
                     Anfängen nicht widerstehen, meine Herren Geschworenen, dann ist es mit aller
                     Kunst und Literatur für alle Zeiten aus und vorbei.« (Vgl. [O. V.:]
                        \emph{\textcolor{green}{Vergehen gegen die Sittlichkeit}}. In: \emph{\textcolor{green}{Neue Freie Presse}}, Nr. 10.387,
                        24. 7. 1893, S. 3–4, hier S. 4).}}}\label{K_L00244_3h}{ }\uline{für alle} Zeiten durch Verbot der »\textcolor{brown}{Geſellſchaft}{}\ledrightnote{\textcolor{brown}{Die Gesellschaft [Wien]}}«ſchweinigel.\pend
           \pstart
           \textcolor{green}{Einakter}{}\ledrightnote{→\textcolor{green}{Abschiedssouper}} geht flott weiter. Heut
               las ich im \textcolor{brown}{B. Börſ.courier}{}\ledrightnote{\textcolor{brown}{Berliner Börsen-Courier}} circa \label{K_L00244_4v}\edtext{\textcolor{green}{40 Zeilen}{}\ledrightnote{→\textcolor{green}{[Man schreibt uns aus Ischl]}}}{\lemma{\textnormal{\emph{40 Zeilen}}}\Cendnote{\textnormal{[O. V.]: \emph{\textcolor{green}{[Man schreibt uns aus Ischl]}}.
                     In: \emph{\textcolor{green}{Berliner Börsen-Courier}}, Nr. 343,
                        25. 7. 1893, Morgen-Ausgabe, S. 4.}}}\label{K_L00244_4h} über \textcolor{green}{Abſchiedssouper}{}\ledrightnote{\textcolor{green}{Abschiedssouper}}{ }\uline{gelesen}? Darf ich, daſs \uline{\textcolor{green}{Abschiedss.}{}\ledrightnote{\textcolor{green}{Abschiedssouper}}} im \textcolor{pink}{Residenz}{}\ledrightnote{\textcolor{pink}{Residenztheater München}} angenommen ist, im \textcolor{brown}{Magazin}{}\ledrightnote{\textcolor{brown}{Magazin für die Literatur des Auslandes}}{ }\label{K_L00244_5v}\edtext{publicieren}{\lemma{\textnormal{\emph{publicieren}}}\Cendnote{\textnormal{nicht erschienen}}}\label{K_L00244_5h}? 1000 Grüße Ihr \spacefill\mbox{Kraus}\pend
           \pstart
           \noindent{}Schicken Sie Ihr \textcolor{green}{Drama}{}\ledrightnote{→\textcolor{green}{Das Märchen. Schauspiel in drei Aufzügen}}
                  hin!!\pend
           \endnumbering\briefempfaengerindex{Schnitzler, Arthur@\textsc{Schnitzler, Arthur}!zzzKraus, Karl@\emph{von Karl Kraus}!1893-07-271@{27. 7. 1893}|)be}\mylabel{h}  \normalsize

\doendnotes{C}
\bigskip
\vfill

\clearpage

\footnotesize

\lohead{\textsc{register}}

% Definiere theindex-Environment komplett neu ohne reledmac
\makeatletter
\renewenvironment{theindex}{%
  \section*{\indexname}%
  \setlength{\parindent}{0pt}%
  \setlength{\parskip}{0pt plus 0.3pt}%
  \let\item\@idxitem
}{%
  \clearpage
}
\makeatother

\IfFileExists{\jobname-pw.ind}{\input{\jobname-pw.ind}}{}

\end{document}

      