%% latex-korrekturansicht-vorspann.tex
%% Vorspann für die Korrekturansicht.
%% Lädt die gemeinsame Datei latex-vorspann.tex mit gesetztem Schalter.

\newif\ifkorrekturansicht
\korrekturansichttrue

\input{../tex-inputs/latex-vorspann}


               \section[Richard Beer-Hofmann an Arthur Schnitzler, 18. 1. 1903]{ Richard Beer-Hofmann an Arthur Schnitzler,
               18. 1. 1903}\nopagebreak\mylabel{v}\rehead{ }\normalsize\beginnumbering\briefempfaengerindex{Schnitzler, Arthur@\textsc{Schnitzler, Arthur}!zzzBeer-Hofmann, Richard@\emph{von Richard Beer-Hofmann}!1903-01-181@{18. 1. 1903}|(be} \toendnotes[C]{\smallbreak\pagebreak[2]} \Standort{CUL, Schnitzler, B 8.}
\physDesc{Brief, 1 Blatt (Briefpapier mit Trauerrand), 2 Seiten
\newline{}Handschrift: blauer Buntstift, lateinische Kurrent\newline{}Ordnung: mit Bleistift von unbekannter Hand nummeriert:
                              »177« }\buchAbdrucke{\weitereDrucke{Arthur Schnitzler, Richard Beer-Hofmann: \emph{Briefwechsel 1891–1931}. Hg. Konstanze Fliedl. Wien, Zürich: \emph{Europaverlag} 1992, S. 160.} }\toendnotes[C]{\smallbreak}\pstart
           \raggedleft{}{\pb}\textcolor{pink}{Rodaun}{}\ledrightnote{\textcolor{pink}{Rodaun}}{ }18/1 1903\pend
           \pstart
           Lieber Arthur! Vielen Dank für Ihren Antrag. Ich kann mich aber
               nicht entschließen »\label{K_L01267_1v}\edtext{Bern}{\lemma{\textnormal{\emph{Bern}}}\Cendnote{\textnormal{den Bernhardinerhund}}}\label{K_L01267_1h}« in unsere
               Familie aufzunehmen. Abgesehen vom Großfolio-Format würde ich – wenn – nur einen \uline{ganz jungen} Hund wieder nehmen damit er an die Kinder,
               und sie an ihn sich gewöhnen, und ich sein Inneres von seinen ersten Lebenswochen an
               bilden kann. Jedenfalls werde ich ihm aber demnächst einen Besuch abstatten. Um \textcolor{pink}{Salzburg}{}\ledrightnote{\textcolor{pink}{Salzburg}} beneide ich Sie na{\pb}türlich. Ich arbeite (ja!) und \textcolor{blue}{Hugo}{}\ledrightnote{\textcolor{blue}{Hugo von Hofmannsthal}} ist mit dem Flohtheater beschäftigt –
               bestehend aus Ihren – \textcolor{blue}{Schwarzkopfs}{}\ledrightnote{\textcolor{blue}{Gustav Schwarzkopf}} etc. Flöhen
               die ihm ins Ohr gesetzt wurden. Vielleicht sehe ich Sie \label{K_L01267_2v}\edtext{Samstag (24){ }Nachm. (\textcolor{brown}{Akad. Verein}{}\ledrightnote{\textcolor{brown}{Akademischer Verein für Kunst und Literatur}})}{\lemma{\textnormal{\emph{Samstag … Verein)}}}\Cendnote{\textnormal{Der \emph{\textcolor{brown}{Akademische Verein für Kunst und Literatur}} veranstaltete im
                     \textcolor{pink}{Theater an der Wien} die erste \textcolor{pink}{Wien}er Inszenierung von \emph{\textcolor{green}{Elpenor}}. \textcolor{blue}{Schnitzler} dürfte nicht
                  teilgenommen haben. Im Original steht der »Akad. Verein« in eckigen
                  Klammern.}}}\label{K_L01267_2h}. Nochmals Dank und herzliche Grüße – auch an \textcolor{blue}{Mutter}{}\ledrightnote{→\textcolor{blue}{Olga Schnitzler}} und \textcolor{blue}{Kind}{}\ledrightnote{→\textcolor{blue}{Heinrich Schnitzler}}.\pend
           \pstart
           Ihr{\\[\baselineskip]}\spacefill\mbox{Richard}\pend
           \leftskip=0em{}\pstart
           \noindent{}\label{K_L01267_3v}\edtext{\textcolor{pink}{Hallein}{}\ledrightnote{\textcolor{pink}{Hallein}} erhalten}{\lemma{\textnormal{\emph{Hallein erhalten}}}\Cendnote{\textnormal{Arthur Schnitzler und Olga Gussmann an Richard Beer-Hofmann,
               16. 1. 1903}}}\label{K_L01267_3h}\pend
           \endnumbering\briefempfaengerindex{Schnitzler, Arthur@\textsc{Schnitzler, Arthur}!zzzBeer-Hofmann, Richard@\emph{von Richard Beer-Hofmann}!1903-01-181@{18. 1. 1903}|)be}\mylabel{h}  \normalsize

\doendnotes{C}
\bigskip
\vfill

\clearpage

\footnotesize

\lohead{\textsc{register}}

% Definiere theindex-Environment komplett neu ohne reledmac
\makeatletter
\renewenvironment{theindex}{%
  \section*{\indexname}%
  \setlength{\parindent}{0pt}%
  \setlength{\parskip}{0pt plus 0.3pt}%
  \let\item\@idxitem
}{%
  \clearpage
}
\makeatother

\IfFileExists{\jobname-pw.ind}{\input{\jobname-pw.ind}}{}

\end{document}

      