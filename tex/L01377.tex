%% latex-korrekturansicht-vorspann.tex
%% Vorspann für die Korrekturansicht.
%% Lädt die gemeinsame Datei latex-vorspann.tex mit gesetztem Schalter.

\newif\ifkorrekturansicht
\korrekturansichttrue

\input{../tex-inputs/latex-vorspann}


               \section[Hermann Bahr an Arthur Schnitzler, {[}5.{]} 3. 1904]{ Hermann Bahr an Arthur Schnitzler, {[}5.{]} 3. 1904}\nopagebreak\mylabel{v}\rehead{ }\normalsize\beginnumbering\briefempfaengerindex{Schnitzler, Arthur@\textsc{Schnitzler, Arthur}!zzzBahr, Hermann@\emph{von Hermann Bahr}!1904-03-051@{{[}5.{]} 3. 1904}|(be} \toendnotes[C]{\smallbreak\pagebreak[2]} \Standort{CUL, Schnitzler, B 5b.}
\physDesc{Bildpostkarte
\newline{}Handschrift: Bleistift, deutsche Kurrent\newline{}Versand: 1) Stempel: »\nobreak{}\oindex{Solin@\textbf{Solin}, \emph{Besiedelter Ort (A.BSO)}|pwk}Solin Salona\nobreak{}«.  2) Stempel: »\nobreak{}\oindex{XVIII., Waehring@\textbf{XVIII., Währing}, \emph{Bezirk (A.BZK)}|pwk}18/1 Wien, 8. 3. 04, 12. V, Bestellt\nobreak{}«. 
\newline{}Schnitzler: mit Bleistift datiert: »8. 3« \newline{}Ordnung: mit Bleistift von unbekannter Hand nummeriert:
                              »112« }\buchAbdrucke{\weitereDrucke{Hermann Bahr, Arthur Schnitzler: \emph{Briefwechsel, Aufzeichnungen, Dokumente (1891–1931)}. Hg. Kurt Ifkovits und Martin Anton Müller. Göttingen: \emph{Wallstein} 2018, S. 305.} }\pstart{}{\pb}Herrn \textsc{D\textsuperscript{r} Arthur Schnitzler}\pend{}\pstart{}\textcolor{pink}{\textsc{Wien XVIII}}{}\ledrightnote{\textcolor{pink}{XVIII., Währing}}\pend{}\pstart{}\textcolor{pink}{\textsc{Spöttelgasse} 7}{}\ledrightnote{\textcolor{pink}{Edmund-Weiß-Gasse}}\pend{}{\bigskip}\pstart
           \noindent{}\centering{}\textcolor{gray}{\textbf{{\pb}Salutem Pluriman E \textcolor{pink}{Ruinis Salonae / Basilica Martyrum
                     Salonitanorum in Manastirine}{}\ledrightnote{\textcolor{pink}{Solin}}}}\pend
           \pstart
           {\pb}hier mußt Du einmal her! Das ist mehr als alles
                  \textcolor{pink}{Italien}{}\ledrightnote{\textcolor{pink}{Italien}}.\pend
           \pstart Herzlichſt\hspace*{1.5em}Dein\spacefill\mbox{H.}\pend{}\endnumbering\briefempfaengerindex{Schnitzler, Arthur@\textsc{Schnitzler, Arthur}!zzzBahr, Hermann@\emph{von Hermann Bahr}!1904-03-051@{{[}5.{]} 3. 1904}|)be}\mylabel{h}  \normalsize

\doendnotes{C}
\bigskip
\vfill

\clearpage

\footnotesize

\lohead{\textsc{register}}

% Definiere theindex-Environment komplett neu ohne reledmac
\makeatletter
\renewenvironment{theindex}{%
  \section*{\indexname}%
  \setlength{\parindent}{0pt}%
  \setlength{\parskip}{0pt plus 0.3pt}%
  \let\item\@idxitem
}{%
  \clearpage
}
\makeatother

\IfFileExists{\jobname-pw.ind}{\input{\jobname-pw.ind}}{}

\end{document}

      