%% latex-korrekturansicht-vorspann.tex
%% Vorspann für die Korrekturansicht.
%% Lädt die gemeinsame Datei latex-vorspann.tex mit gesetztem Schalter.

\newif\ifkorrekturansicht
\korrekturansichttrue

\input{../tex-inputs/latex-vorspann}


               \section[Arthur Schnitzler an Hugo von Hofmannsthal, {[}10. 1. 1899{]}]{ Arthur Schnitzler an Hugo von Hofmannsthal, {[}10. 1. 1899{]}}\nopagebreak\mylabel{v}\rehead{ }\normalsize\beginnumbering\briefempfaengerindex{Hofmannsthal, Hugo von@\textsc{Hofmannsthal, Hugo von}!zzzSchnitzler, Arthur@\emph{von Arthur Schnitzler}!1899-01-102@{{[}10. 1. 1899{]}}|(be} \toendnotes[C]{\smallbreak\pagebreak[2]} \Standort{FDH, Hs-30885,79.}
\physDesc{Brief, 1 Blatt, 3 Seiten
\newline{}Handschrift: Bleistift, deutsche Kurrent\newline{}Ordnung: mit Bleistift von unbekannter Hand datiert: »Anf. 99, 98?« }\buchAbdrucke{\weitereDrucke{Hugo von Hofmannsthal, Arthur Schnitzler: \emph{Briefwechsel}. Hg. Therese Nickl und Heinrich Schnitzler. Frankfurt am Main: \emph{S. Fischer} 1964, S. 116–117.} }\toendnotes[C]{\smallbreak}\pstart
           \raggedleft{}{\pb}\uline{Dinſtg.}\pend
           \pstart
           Mein lieber Hugo, ich wußte gar nicht, dſs Sie ſchon da ſind.
                    Morgen ko{\geminationm} ich jedenfalls ins \textsc{\textcolor{pink}{Pfob}{}\ledrightnote{\textcolor{pink}{Café Pfob}}} u freu mich Sie endlich wiederzuſehn. \textsc{\textcolor{pink}{Pfob}{}\ledrightnote{\textcolor{pink}{Café Pfob}}} iſt allerdgs wenig. Vor \textsc{\textcolor{pink}{Pfob}{}\ledrightnote{\textcolor{pink}{Café Pfob}}} will ich morgen komiſcherweiſe ins \textcolor{pink}{Jantſchtheater}{}\ledrightnote{\textcolor{pink}{Jantsch-Theater}} zu \textcolor{green}{Theodora}{}\ledrightnote{\textcolor{green}{Theodora}}, u zw hab
                    ich mit \textcolor{blue}{\textsc{Wassermann}}{}\ledrightnote{\textcolor{blue}{Jakob Wassermann}} vor ½ 8 im Vorraum des Theaters Rendezvous. Vielleicht hat er
                    eine {\pb}geſchenkte Loge; ev. kaufen wir uns Billetts.
                    Vielleicht ſind Sie auch vor ½ 8 im Vorraum. Eine gute \textcolor{blue}{Schauſpielerin}{}\ledrightnote{→\textcolor{blue}{Elisabeth Anders}}{ }ſoll die \textcolor{green}{Theodora}{}\ledrightnote{→\textcolor{green}{Theodora}}{ }ſpielen.\pend
           \pstart
           Mir iſt es wieder innerlich recht miſerabel gegangen; aber mit dem Arbeiten
                    beſſer. Im übrigen muſs ich über \textcolor{brown}{Burg}{}\ledrightnote{\textcolor{brown}{Burgtheater}} mit Ihnen
                    reden. Denken Sie, dſs der \textcolor{green}{Kakadu}{}\ledrightnote{\textcolor{green}{Der grüne Kakadu. Groteske in einem Akt}}{ }{\pb}nicht unbeträchtliche Chancen hat! – Aber das alles
                    mündlich –\pend
           \pstart Von Herzen Ihr \spacefill\mbox{Arthur}\pend{}\endnumbering\briefempfaengerindex{Hofmannsthal, Hugo von@\textsc{Hofmannsthal, Hugo von}!zzzSchnitzler, Arthur@\emph{von Arthur Schnitzler}!1899-01-102@{{[}10. 1. 1899{]}}|)be}\mylabel{h}  \normalsize

\doendnotes{C}
\bigskip
\vfill

\clearpage

\footnotesize

\lohead{\textsc{register}}

% Definiere theindex-Environment komplett neu ohne reledmac
\makeatletter
\renewenvironment{theindex}{%
  \section*{\indexname}%
  \setlength{\parindent}{0pt}%
  \setlength{\parskip}{0pt plus 0.3pt}%
  \let\item\@idxitem
}{%
  \clearpage
}
\makeatother

\IfFileExists{\jobname-pw.ind}{\input{\jobname-pw.ind}}{}

\end{document}

      