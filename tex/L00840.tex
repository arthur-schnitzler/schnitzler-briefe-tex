%% latex-korrekturansicht-vorspann.tex
%% Vorspann für die Korrekturansicht.
%% Lädt die gemeinsame Datei latex-vorspann.tex mit gesetztem Schalter.

\newif\ifkorrekturansicht
\korrekturansichttrue

\input{../tex-inputs/latex-vorspann}


               \section[Arthur Schnitzler an Hugo von Hofmannsthal, 27. 8. 1898]{ Arthur Schnitzler an Hugo von Hofmannsthal, 27. 8. 1898}\nopagebreak\mylabel{v}\rehead{ }\normalsize\beginnumbering\briefempfaengerindex{Hofmannsthal, Hugo von@\textsc{Hofmannsthal, Hugo von}!zzzSchnitzler, Arthur@\emph{von Arthur Schnitzler}!1898-08-271@{27. 8. 1898}|(be} \toendnotes[C]{\smallbreak\pagebreak[2]} \Standort{FDH, Hs-30885,75.}
\physDesc{Postkarte
\newline{}Handschrift: Bleistift, deutsche Kurrent\newline{}Versand: 1) Stempel: »\nobreak{}\oindex{Alpnachstad@\textbf{Alpnachstad}, \emph{Besiedelter Ort (A.BSO)}|pwk}Alpnach Stad, 27. VIII. 98\nobreak{}«.  2) Stempel: »\nobreak{}\oindex{Lugano@\textbf{Lugano}, \emph{Besiedelter Ort (A.BSO)}|pwk}L{[}ugano{]}
                                        Lettere, 27. VIII. 98\nobreak{}«. \newline{}Ordnung: von Schnitzler mumaßlich bei der Durchsicht der Briefe 1929 mit Bleistift datiert: »27/8 98« }\buchAbdrucke{\weitereDrucke{Hugo von Hofmannsthal, Arthur Schnitzler: \emph{Briefwechsel}. Hg. Therese Nickl und Heinrich Schnitzler. Frankfurt am Main: \emph{S. Fischer} 1964, S. 111.} }\toendnotes[C]{\smallbreak}\pstart{}{\pb}\textsc{Herrn Hugo von Hofmannsthal}, \pend{}\pstart{}\textcolor{pink}{\textsc{Lugano}}{}\ledrightnote{\textcolor{pink}{Lugano}}\pend{}\pstart{}\textsc{\textcolor{pink}{Hotel du parc}{}\ledrightnote{\textcolor{pink}{Hôtel du Parc & Bristol}}}.\pend{}{\bigskip}\pstart
           \noindent{}{\pb}Mein lieber Hugo, ich bin hieher \textsc{per}
                    Rad gefahren; will \textsc{per} Bahn auf den \textcolor{pink}{Pilatus}{}\ledrightnote{\textcolor{pink}{Pilatus}}. Morgen denk ich \textcolor{pink}{Luzern}{}\ledrightnote{\textcolor{pink}{Luzern}} zu verlaſſen, in dem ich mich ganz wohl behagt, nur phyſiſch
                    nicht ſo beiſa{\geminationm}en war als ich gewünſcht. Ich will
                    die Route \textsc{\textcolor{pink}{Mailand}{}\ledrightnote{\textcolor{pink}{Mailand}} – (\textcolor{pink}{Pavia}{}\ledrightnote{\textcolor{pink}{Pavia}} –)}{ }\textcolor{pink}{Piazenza}{}\ledrightnote{\textcolor{pink}{Piacenza}} – (\textsc{\textcolor{pink}{Parma}{}\ledrightnote{\textcolor{pink}{Parma}}) \textcolor{pink}{Modena}{}\ledrightnote{\textcolor{pink}{Modena}} – \textcolor{pink}{Bologna}{}\ledrightnote{\textcolor{pink}{Bologna}} – \textcolor{pink}{Ferrara}{}\ledrightnote{\textcolor{pink}{Ferrara}} – \textcolor{pink}{Padua}{}\ledrightnote{\textcolor{pink}{Padua}} – \textcolor{pink}{Vicenza}{}\ledrightnote{\textcolor{pink}{Vicenza}} – \strikeout{(Ve}{ }\textcolor{pink}{Verona}{}\ledrightnote{\textcolor{pink}{Verona}}} – \textcolor{pink}{Wien}{}\ledrightnote{\textcolor{pink}{Wien}} einſchlagen. – Geſtern hab ich eine
                    kleine \textcolor{green}{Geſchichte}{}\ledrightnote{→\textcolor{green}{Excentric}} zu
                    ſchreiben angefangen. Schreiben Sie mir ein Wort nach \textsc{\textcolor{pink}{Bologna}{}\ledrightnote{\textcolor{pink}{Bologna}} post rest}. Grüßen Sie \textcolor{blue}{Richard}{}\ledrightnote{\textcolor{blue}{Richard Beer-Hofmann}} von mir, we{\geminationn} er ko{\geminationm}t. Ich hoffe
                    Sie gut gelaunt und heiter und bin von Herzen \pend
           \pstart Ihr \spacefill\mbox{Arthur}\pend{}\pstart
           \textcolor{pink}{\textsc{Alpnach Stad}}{}\ledrightnote{\textcolor{pink}{Alpnachstad}}, \substVorne{}\textsuperscript{Frei}\substDazwischen{}Samſ\substHinten{}tag{ }früh.\pend
           \endnumbering\briefempfaengerindex{Hofmannsthal, Hugo von@\textsc{Hofmannsthal, Hugo von}!zzzSchnitzler, Arthur@\emph{von Arthur Schnitzler}!1898-08-271@{27. 8. 1898}|)be}\mylabel{h}  \normalsize

\doendnotes{C}
\bigskip
\vfill

\clearpage

\footnotesize

\lohead{\textsc{register}}

% Definiere theindex-Environment komplett neu ohne reledmac
\makeatletter
\renewenvironment{theindex}{%
  \section*{\indexname}%
  \setlength{\parindent}{0pt}%
  \setlength{\parskip}{0pt plus 0.3pt}%
  \let\item\@idxitem
}{%
  \clearpage
}
\makeatother

\IfFileExists{\jobname-pw.ind}{\input{\jobname-pw.ind}}{}

\end{document}

      