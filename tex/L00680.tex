%% latex-korrekturansicht-vorspann.tex
%% Vorspann für die Korrekturansicht.
%% Lädt die gemeinsame Datei latex-vorspann.tex mit gesetztem Schalter.

\newif\ifkorrekturansicht
\korrekturansichttrue

\input{../tex-inputs/latex-vorspann}


               \section[Richard Beer-Hofmann an Arthur Schnitzler, 20. 5. 1897]{ Richard Beer-Hofmann an Arthur Schnitzler,
                    20. 5. 1897}\nopagebreak\mylabel{v}\rehead{ }\normalsize\beginnumbering\briefempfaengerindex{Schnitzler, Arthur@\textsc{Schnitzler, Arthur}!zzzBeer-Hofmann, Richard@\emph{von Richard Beer-Hofmann}!1897-05-203@{20. 5. 1897}|(be} \toendnotes[C]{\smallbreak\pagebreak[2]} \Standort{CUL, Schnitzler, B 8.}
\physDesc{Brief, 3 Blätter, 9 Seiten
\newline{}Handschrift: blauer Buntstift, lateinische Kurrent\newline{}Ordnung: mit Bleistift von unbekannter Hand nummeriert:
                                        »96« }\buchAbdrucke{\weitereDrucke{Arthur Schnitzler, Richard Beer-Hofmann: \emph{Briefwechsel 1891–1931}. Hg. Konstanze Fliedl. Wien, Zürich: \emph{Europaverlag} 1992, S. 105–106.} }\toendnotes[C]{\smallbreak}\pstart
           \raggedleft{}{\pb}20/V 97{ }\textcolor{pink}{Wien}{}\ledrightnote{\textcolor{pink}{Wien}}\pend
           \pstart
           Lieber Arthur, ich hab Ihren Brief vor einer Viertelstunde
                    erhalten und antworte schon damit Sie bei Ihrer Ankunft in \textcolor{pink}{London}{}\ledrightnote{\textcolor{pink}{London}} ihn vorfinden. Ich reise am
                        3. Juni Früh nach \textcolor{pink}{Ischl}{}\ledrightnote{\textcolor{pink}{Bad Ischl}}.
                    Länger kann ich nicht hier bleiben. Ich bin {\pb}recht verdrießlich: Mein
                    Husten, kein Geld, Wohnung in Ordnung bringen – ich beko{\geminationm}e Wutanfälle wenn ich hausfrauliche Pflichten
                    erfüllen soll. Ko{\geminationm}en Sie nicht {\pb}im Juni mit
                    Ihrer \textcolor{blue}{Mama}{}\ledrightnote{→\textcolor{blue}{Louise Schnitzler}} nach \textcolor{pink}{Ischl}{}\ledrightnote{\textcolor{pink}{Bad Ischl}}? \textcolor{pink}{Wien}{}\ledrightnote{\textcolor{pink}{Wien}}
                    dürfte Ihnen ja unerträglich sein.\pend
           \pstart
           Dem \textcolor{blue}{Paul}{}\ledrightnote{\textcolor{blue}{Paul Goldmann}}
                sagen Sie: »Ein \uline{guter} Mensch in seinem – – – –« und betonen Sie
                    das »gut«. Er ha\substVorne{}\textsuperscript{tt}\substDazwischen{}t\substHinten{}{ }{\pb}tausendmal
                    recht gehabt mit Allem was er von der Verlogenheit und Niedrigkeit dieses Packs
                    sagte.\pend
           \pstart
           \textcolor{blue}{Altenberg}{}\ledrightnote{\textcolor{blue}{Peter Altenberg}} hat mir – ich bat ihn nicht darum –
                        {\pb}im \textcolor{pink}{Tiergarten}{}\ledrightnote{\textcolor{pink}{Tiergarten Schönbrunn}} durch einige Stunden Gesellschaft geleistet{[}.{]}
                    Von dem plumpem Comödiespielen dieses armseeligen Schmierencomödianten können
                    Sie sich kaum einen Begriff machen. {\pb}Er lehnt verzückt an irgend
                    einer Umfriedung und starrt auf irgend einen Schwarzen oder Schwarze und wartet
                    daß ihn ein zufällig Vorübergehender (– er ist natürlich nur am Nachmittag
                    in den Besuchsstunden dort wo er gesehen wird –) {\pb}aus seiner Verzückung reiße.
                    Dabei ist er blind für den wirklichen Reiz dieser dunkeln Menschen\pend
           \pstart
           Er kann \uline{nur} lügen.\pend
           \pstart
           Von \textcolor{blue}{Bahr}{}\ledrightnote{\textcolor{blue}{Hermann Bahr}} mag ich {\pb}nicht mehr reden. Er »sinkt«
                    i{\geminationm}er tiefer würde ich sagen, wenn er jemals hoch gestanden wäre. –\pend
           \pstart
           \textcolor{blue}{P.}{}\ledrightnote{\textcolor{blue}{Paula Beer-Hofmann}}
                schreibt mir täglich und ist geduldig und
                    brav. Da fällt {\pb}mir ein daß
                    Sie ja – da ich nach \textcolor{pink}{London}{}\ledrightnote{\textcolor{pink}{London}} adressire – \textcolor{blue}{Paul}{}\ledrightnote{\textcolor{blue}{Paul Goldmann}} nicht mehr sprechen; also schreiben Sie
                    ihm viel Herzliches von mir, und seine neue Adresse möcht ich wissen. Bicycle?
                    Noch nicht!\pend
           \pstart Ihr \spacefill\mbox{Richard}\pend{}\endnumbering\briefempfaengerindex{Schnitzler, Arthur@\textsc{Schnitzler, Arthur}!zzzBeer-Hofmann, Richard@\emph{von Richard Beer-Hofmann}!1897-05-203@{20. 5. 1897}|)be}\mylabel{h}  \normalsize

\doendnotes{C}
\bigskip
\vfill

\clearpage

\footnotesize

\lohead{\textsc{register}}

% Definiere theindex-Environment komplett neu ohne reledmac
\makeatletter
\renewenvironment{theindex}{%
  \section*{\indexname}%
  \setlength{\parindent}{0pt}%
  \setlength{\parskip}{0pt plus 0.3pt}%
  \let\item\@idxitem
}{%
  \clearpage
}
\makeatother

\IfFileExists{\jobname-pw.ind}{\input{\jobname-pw.ind}}{}

\end{document}

      