%% latex-korrekturansicht-vorspann.tex
%% Vorspann für die Korrekturansicht.
%% Lädt die gemeinsame Datei latex-vorspann.tex mit gesetztem Schalter.

\newif\ifkorrekturansicht
\korrekturansichttrue

\input{../tex-inputs/latex-vorspann}


               \section[Arthur Schnitzler an Richard Beer-Hofmann, 30. 12. 1897]{ Arthur Schnitzler an Richard Beer-Hofmann, 30. 12. 1897}\nopagebreak\mylabel{v}\rehead{ }\normalsize\beginnumbering\briefempfaengerindex{Beer-Hofmann, Richard@\textsc{Beer-Hofmann, Richard}!zzzSchnitzler, Arthur@\emph{von Arthur Schnitzler}!1897-12-301@{30. 12. 1897}|(be} \toendnotes[C]{\smallbreak\pagebreak[2]} \Standort{YCGL, MSS 31.}
\physDesc{Briefkarte, Umschlag
\newline{}Handschrift: schwarze Tinte, deutsche Kurrent\newline{}Versand: 1) Stempel: »\nobreak{}\oindex{IX., Alsergrund@\textbf{IX., Alsergrund}, \emph{Bezirk (A.BZK)}|pwk}Wien 9/3, 30. 12. 9\textcolor{gray}{7}, 3–4N\nobreak{}«.  2) Stempel: »\nobreak{}\oindex{I., Innere Stadt@\textbf{I., Innere Stadt}, \emph{Bezirk (A.BZK)}|pwk}{\pb}Wien 1/1, 30/12 97, 62½–8N, Bestellt\nobreak{}«. }\buchAbdrucke{\weitereDrucke{Arthur Schnitzler, Richard Beer-Hofmann: \emph{Briefwechsel 1891–1931}. Hg. Konstanze Fliedl. Wien, Zürich: \emph{Europaverlag} 1992, S. 114.} }\toendnotes[C]{\smallbreak}\pstart{}{\pb}Herrn \textsc{Dr. Richard
                     Beer-Hofmann}\pend{}\pstart{}\textcolor{pink}{Wien}{}\ledrightnote{\textcolor{pink}{Wien}}\pend{}\pstart{}\textcolor{pink}{\textsc{I. Wollzeile 15}}{}\ledrightnote{\textcolor{pink}{Wollzeile}}.\pend{}{\bigskip}\pstart
           \raggedleft{}{\pb}30/12 97\pend
           \pstart
           Lieber Richard, die verſchiedenen Anregungen von
                  Dinſtag hab ich, für den \textcolor{green}{2 Akt}{}\ledrightnote{→\textcolor{green}{Das Vermächtnis. Schauspiel in drei Akten}} vorläufig, nicht unglücklich benützt – er ſieht
               jetzt, ich muſs es ſelber ſagen, etwas beſſer aus. Ich möcht Ihnen das bald einmal
               zeigen. Sagen Sie das auch \textcolor{blue}{Hugo}{}\ledrightnote{\textcolor{blue}{Hugo von Hofmannsthal}}, den Sie
               wahrſcheinlich früher ſehn werden als ich. Wenn ich besti{\geminationm}t weiſs, daſs {\pb}Sie in der
                  Sylveſternacht im \textcolor{pink}{\textsc{Pucher}}{}\ledrightnote{\textcolor{pink}{Café Pucher}}{ }ſein werden, ſo ko{\geminationm}
               ich hin.\pend
           \pstart Herzlichſt Ihr \spacefill\mbox{Arthur.}\pend{}\endnumbering\briefempfaengerindex{Beer-Hofmann, Richard@\textsc{Beer-Hofmann, Richard}!zzzSchnitzler, Arthur@\emph{von Arthur Schnitzler}!1897-12-301@{30. 12. 1897}|)be}\mylabel{h}  \normalsize

\doendnotes{C}
\bigskip
\vfill

\clearpage

\footnotesize

\lohead{\textsc{register}}

% Definiere theindex-Environment komplett neu ohne reledmac
\makeatletter
\renewenvironment{theindex}{%
  \section*{\indexname}%
  \setlength{\parindent}{0pt}%
  \setlength{\parskip}{0pt plus 0.3pt}%
  \let\item\@idxitem
}{%
  \clearpage
}
\makeatother

\IfFileExists{\jobname-pw.ind}{\input{\jobname-pw.ind}}{}

\end{document}

      