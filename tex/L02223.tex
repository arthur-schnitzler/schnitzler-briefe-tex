%% latex-korrekturansicht-vorspann.tex
%% Vorspann für die Korrekturansicht.
%% Lädt die gemeinsame Datei latex-vorspann.tex mit gesetztem Schalter.

\newif\ifkorrekturansicht
\korrekturansichttrue

\input{../tex-inputs/latex-vorspann}


               \section[Georg Brandes an Arthur Schnitzler, 13. 12. 1915]{ Georg Brandes an Arthur Schnitzler, 13. 12. 1915}\nopagebreak\mylabel{v}\rehead{ }\normalsize\beginnumbering\briefempfaengerindex{Schnitzler, Arthur@\textsc{Schnitzler, Arthur}!zzzBrandes, Georg@\emph{von Georg Brandes}!1915-12-131@{13. 12. 1915}|(be} \toendnotes[C]{\smallbreak\pagebreak[2]} \Standort{CUL, Schnitzler, B 17.}
\physDesc{Brief, 1 Blatt, 4 Seiten
\newline{}Handschrift: schwarze Tinte, lateinische Kurrent
\newline{}Schnitzler: 1) mit Bleistift beschriftet: »\textsc{Brandes}« 2) mit rotem Buntstift vereinzelte Unterstreichungen\newline{}Ordnung: mit Bleistift von unbekannter Hand nummeriert:
                                    »46« }\buchAbdrucke{\weitereDrucke{Georg Brandes, Arthur Schnitzler: \emph{Ein Briefwechsel}. Hg. Kurt Bergel. Bern: \emph{Francke} 1956, S. 119–120.} }\toendnotes[C]{\smallbreak}\pstart
           \raggedleft{}{\pb}\textcolor{pink}{Kopenhagen}{}\ledrightnote{\textcolor{pink}{Kopenhagen}} d. 13 December 15\pend
           \pstart
           Verehrter Freund \hspace*{1.5em}Es war mir eine angenehme Ueberraschung, so bald
               von Ihnen zu hören. Ich erwartete das nicht. Ich bin leider bettlägerig. Sie wissen
               als Artzt, wie langwierig dies verdammte Uebel ist, gegen welches das Serum erst zehn
               Jahre nach meinem Tode gefunden wird. Warme Umschläge imponiren den Bacillen nicht,
               und ich kann ihnen das nicht verdenken.\pend
           \pstart
           \textcolor{blue}{Peter Nansen}{}\ledrightnote{\textcolor{blue}{Peter Nansen}} soll besser sein. Es ist nur eine
               Bronchitis, die ein schwaches Fieber verursacht.\pend
           \pstart
           Sie haben ja völlig und unbestrittenes Recht, wenn Sie behaupten, als Dramatiker
               nicht mit irgend einer Ihrer Persönlichkeiten identisch zu sein. Aber die \textcolor{green}{Schlussscene}{}\ledrightnote{→\textcolor{green}{Professor Bernhardi. Komödie in fünf Akten}} scheint mir jedoch
               den Totaleindruck zusammenfassen zu sollen. {\pb}Ueber die Feierlichen denke ich
               natürlich wie Sie. Da ich die 20 Jahre älter bin gewiss mit noch grösserem Widerwille
               als Sie.\pend
           \pstart
           Was Sie über die Kritiker sagen erstaunt mich nicht; ich kenne nichts widerlicheres
               und dümmeres.\pend
           \pstart
           \textcolor{blue}{Lassalle}{}\ledrightnote{\textcolor{blue}{Ferdinand Lassalle}} sagte »\label{K_L02223_1v}\edtext{Zwei Arten von Menschen sind mir vor Allem verhasst
               Journalisten und Juden – und ich bin beides.}{\lemma{\textnormal{\emph{Zwei … beides.}}}\Cendnote{\textnormal{Nicht genauer nachgewiesenes Zitat in: \emph{\textcolor{green}{\textcolor{blue}{Ferdinand Lassalle}’s Briefe an \textcolor{blue}{Georg Herwegh}. Nebst Briefen der Gräfin
                           \textcolor{blue}{Sophie Hatzfeldt} an Frau \textcolor{blue}{Emma Herwegh}}}. Herausgegeben von \textcolor{blue}{Marcel Herwegh}. Mit einem Bild und Brief \textcolor{blue}{Lassalle’s}. Zürich: \emph{Albert
                        Müller’s Verlag}{ }1896, S. 4–5: »›Zwei Dinge in der Welt‹ – pflegte
                        \textcolor{blue}{Lassalle} zu sagen – ›sind mir vor Allem
                     verhaßt: Journalisten und Juden; und Beides bin ich!‹«.}}}\label{K_L02223_1h}«– Ich
               hasse die Kritiker und verachte sie, besonders die moralisierenden.\pend
           \pstart
           Einen Punkt muss ich beantworten, eine schwache \substVorne{}\textsuperscript{Andeutung}{\allowbreak}\substDazwischen{}Anspielung\substHinten{}. Sie sagen, ich wisse wohl jetzt mehr über den Krieg als im Anfang. Einst
               schrieben Sie mir ebenfalls, ich solle doch nicht glauben, in \textcolor{pink}{Wien}{}\ledrightnote{\textcolor{pink}{Wien}} herrsche \label{K_L02223_2v}\edtext{Hungersnoth}{\lemma{\textnormal{\emph{Hungersnoth}}}\Cendnote{\textnormal{Arthur Schnitzler an Georg Brandes, 20. 10. 1914}}}\label{K_L02223_2h}. {\pb}Ich vergass damals zu
               antworten.\pend
           \pstart
           Irgend ein erbärmlicher Wicht von \textcolor{blue}{Journalist}{}\ledrightnote{→\textcolor{blue}{?? [Dänischer Journalist]}}, der in einem \textcolor{pink}{dänischen}{}\ledrightnote{\textcolor{pink}{Dänemark}}{ }Blatt irgend einen der gewöhnlichen idiotischen \textcolor{green}{Artikel}{}\ledrightnote{→\textcolor{green}{?? [Russischer Korrespondentenbericht]}} von einem sogenannt \textcolor{pink}{russischen}{}\ledrightnote{\textcolor{pink}{Russland}}{ }\textcolor{blue}{Correspondenten}{}\ledrightnote{→\textcolor{blue}{?? [Korrespondent aus Russland]}} gelesen hatte,
               bekam den Einfall, im Anfang des Krieges, \uline{mich}
               deshalb anzugreifen, \uline{mich} dafür verantwortlich zu
               machen. Darin soll gestanden haben, in \textcolor{pink}{Wien}{}\ledrightnote{\textcolor{pink}{Wien}} hungere
               man.\pend
           \pstart
           Ich hatte den \textcolor{green}{Artikel}{}\ledrightnote{→\textcolor{green}{?? [Russischer Korrespondentenbericht]}}{ }\uline{nie gelesen}, nie gesehen, viel weniger geschrieben
               oder aufgenommen. Nun ging diese \label{K_L02223_3v}\edtext{Idiotie}{\lemma{\textnormal{\emph{Idiotie}}}\Cendnote{\textnormal{nicht nachgewiesen}}}\label{K_L02223_3h}
               wie ein Lauffeuer durch die \textcolor{pink}{deutsche}{}\ledrightnote{\textcolor{pink}{Deutschland}} und \textcolor{pink}{österreichische}{}\ledrightnote{\textcolor{pink}{Österreich}} Presse, mit imbecilen Schimpfworten
               gegen mich.\pend
           \pstart
           Sie scheinen daran geglaubt zu haben. So sind wir alle. Wie viele tausend Mal wir
               erfahren haben, dass das Gedruckte {\pb}nur Lüge war, immer wieder glauben wir an etwas.\pend
           \pstart
           Mein junger \textcolor{blue}{Schwiegersohn}{}\ledrightnote{→\textcolor{blue}{Reinhold Philipp}} ist
               noch nicht verwundet, aber leidet grässlich an \uline{Leere},
               fühlt es, als verliere er den Verstand, werde alt und grau. Es ist immer besser als
               Wunden und Tod.\pend
           \pstart
           Ich bin von ganzem Herzen Ihr Freund{\\[\baselineskip]}\spacefill\mbox{G B}\pend
           \leftskip=0em{}\endnumbering\briefempfaengerindex{Schnitzler, Arthur@\textsc{Schnitzler, Arthur}!zzzBrandes, Georg@\emph{von Georg Brandes}!1915-12-131@{13. 12. 1915}|)be}\mylabel{h}  \normalsize

\doendnotes{C}
\bigskip
\vfill

\clearpage

\footnotesize

\lohead{\textsc{register}}

% Definiere theindex-Environment komplett neu ohne reledmac
\makeatletter
\renewenvironment{theindex}{%
  \section*{\indexname}%
  \setlength{\parindent}{0pt}%
  \setlength{\parskip}{0pt plus 0.3pt}%
  \let\item\@idxitem
}{%
  \clearpage
}
\makeatother

\IfFileExists{\jobname-pw.ind}{\input{\jobname-pw.ind}}{}

\end{document}

      