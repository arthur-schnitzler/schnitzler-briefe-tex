%% latex-korrekturansicht-vorspann.tex
%% Vorspann für die Korrekturansicht.
%% Lädt die gemeinsame Datei latex-vorspann.tex mit gesetztem Schalter.

\newif\ifkorrekturansicht
\korrekturansichttrue

\input{../tex-inputs/latex-vorspann}


               \section[Hugo von Hofmannsthal an Arthur Schnitzler, 14. 1. 1898]{ Hugo von Hofmannsthal an Arthur Schnitzler, 14. 1. 1898}\nopagebreak\mylabel{v}\rehead{ }\normalsize\beginnumbering\briefempfaengerindex{Schnitzler, Arthur@\textsc{Schnitzler, Arthur}!zzzHofmannsthal, Hugo von@\emph{von Hugo von Hofmannsthal}!1898-01-141@{14. 1. 1898}|(be} \toendnotes[C]{\smallbreak\pagebreak[2]} \Standort{CUL, Schnitzler, B 43.}
\physDesc{Kartenbrief
\newline{}Handschrift: schwarze Tinte, deutsche Kurrent\newline{}Versand: 1) Stempel: »\nobreak{}Wien 3/3, 14. 1. 98, 12 1 N\nobreak{}«.  2) Stempel: »\nobreak{}Wien 9/3, 14. 1. 98, 5.N\nobreak{}«. 
\newline{}Schnitzler: mit Bleistift datiert: »14/1 98« \newline{}Ordnung: 1) mit Bleistift von unbekannter Hand nummeriert: »\strikeout{106}« 2) mit Bleistift von unbekannter Hand nummeriert: »105«}\buchAbdrucke{\weitereDrucke{Hugo von Hofmannsthal, Arthur Schnitzler: \emph{Briefwechsel}. Hg. Therese Nickl und Heinrich Schnitzler. Frankfurt am Main: \emph{S. Fischer} 1964, S. 98.} }\toendnotes[C]{\smallbreak}\pstart{}{\pb}\textsc{Herrn D\textsuperscript{r} Arthur
                            Schnitzler}\pend{}\pstart{}\textcolor{pink}{\textsc{Wien}}{}\ledrightnote{\textcolor{pink}{Wien}}\pend{}\pstart{}\textcolor{pink}{\textsc{IX Franckgasse} 1}{}\ledrightnote{\textcolor{pink}{Frankgasse}}\pend{}{\bigskip}\pstart{}{\pb}mein lieber
                        Arthur\pend\pstart
           wenn Sie zufällig ein oder gar 2 \textsc{entrées} für \label{K_L00762_1v}\edtext{Sonntag}{\lemma{\textnormal{\emph{Sonntag}}}\Cendnote{\textnormal{Am 16. 1. 1898 wurden
                            \emph{\textcolor{green}{Weihnachts-Einkäufe}} und \emph{\textcolor{green}{Abschiedssouper}} neben anderen Stücken im
                        Rahmen einer Wohltätigkeitsveranstaltung für den Verein \emph{\textcolor{brown}{Ferienheim}} gegeben.}}}\label{K_L00762_1h} übrig hätten und dem \textcolor{blue}{\textsc{Poldy}}{}\ledrightnote{\textcolor{blue}{Leopold von Andrian-Werburg}}{ }ſchicken wollten (d. h. nur wenn Sie ſie nicht anders verwenden wollen)
                    würde es ihm ſehr viel Vergnügen machen.\pend
           \pstart
           Ihr{\\[\baselineskip]}\spacefill\mbox{Hugo.}\pend
           \leftskip=0em{}\endnumbering\briefempfaengerindex{Schnitzler, Arthur@\textsc{Schnitzler, Arthur}!zzzHofmannsthal, Hugo von@\emph{von Hugo von Hofmannsthal}!1898-01-141@{14. 1. 1898}|)be}\mylabel{h}  \normalsize

\doendnotes{C}
\bigskip
\vfill

\clearpage

\footnotesize

\lohead{\textsc{register}}

% Definiere theindex-Environment komplett neu ohne reledmac
\makeatletter
\renewenvironment{theindex}{%
  \section*{\indexname}%
  \setlength{\parindent}{0pt}%
  \setlength{\parskip}{0pt plus 0.3pt}%
  \let\item\@idxitem
}{%
  \clearpage
}
\makeatother

\IfFileExists{\jobname-pw.ind}{\input{\jobname-pw.ind}}{}

\end{document}

      