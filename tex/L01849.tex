%% latex-korrekturansicht-vorspann.tex
%% Vorspann für die Korrekturansicht.
%% Lädt die gemeinsame Datei latex-vorspann.tex mit gesetztem Schalter.

\newif\ifkorrekturansicht
\korrekturansichttrue

\input{../tex-inputs/latex-vorspann}


               \section[Richard Beer-Hofmann an Arthur Schnitzler, 26. 6. 1909]{ Richard Beer-Hofmann an Arthur Schnitzler, 26. 6. 1909}\nopagebreak\mylabel{v}\rehead{ }\normalsize\beginnumbering\briefempfaengerindex{Schnitzler, Arthur@\textsc{Schnitzler, Arthur}!zzzBeer-Hofmann, Richard@\emph{von Richard Beer-Hofmann}!1909-06-261@{26. 6. 1909}|(be} \toendnotes[C]{\smallbreak\pagebreak[2]} \Standort{CUL, Schnitzler, B 8.}
\physDesc{Brief, 1 Blatt, 2 Seiten
\newline{}Handschrift: Bleistift, lateinische Kurrent
\newline{}Schnitzler: mit Bleistift beschriftet: »\textsc{Beerhofm.}« \newline{}Ordnung: mit Bleistift von unbekannter Hand nummeriert:
                              »219« }\buchAbdrucke{\weitereDrucke{Arthur Schnitzler, Richard Beer-Hofmann: \emph{Briefwechsel 1891–1931}. Hg. Konstanze Fliedl. Wien, Zürich: \emph{Europaverlag} 1992, S. 193.} }\toendnotes[C]{\smallbreak}\pstart
           \raggedleft{}{\pb}\substVorne{}\textsuperscript{22}\substDazwischen{}26\substHinten{}/VI 09\pend
           \pstart
           Lieber Arthur! Sie waren \label{K_L01849-1v}\edtext{vorgestern}{\lemma{\textnormal{\emph{vorgestern}}}\Cendnote{\textnormal{vgl. A. S.: \emph{Tagebuch}, 24. 6. 1909}}}\label{K_L01849-1h}{ }Abends bei uns als wir schon im \textcolor{pink}{Türkenschanzpark}{}\ledrightnote{\textcolor{pink}{Türkenschanzpark}} waren. Wir waren in bewusster, Ihnen odioser, Gesellschaft.
               Wir gehen heute wieder hinüber, haben dort Rendezvous mit \textcolor{blue}{Leo}{}\ledrightnote{\textcolor{blue}{Leo Van-Jung}}, \textcolor{blue}{Bella Wengeroff}{}\ledrightnote{\textcolor{blue}{Isabella Vengerova}}, \textcolor{blue}{Kaufmann}{}\ledrightnote{\textcolor{blue}{Arthur Kaufmann}}; vielleicht ko{\geminationm}en Sie doch? (Ich bemerke soeben dass {\pb}»doch« keinen Sinn hat.) Also
               »auch«! Wir \label{K_L01849-2v}\edtext{reisen}{\lemma{\textnormal{\emph{reisen}}}\Cendnote{\textnormal{nach \textcolor{pink}{Pichl am
                     See}}}}\label{K_L01849-2h} (– nein, \uline{wollen} reisen – nein
               reisen \uline{sicher}, nein – Schicksal mach \uuline{\edtext{D}{\Cendnote{dreifach unterstrichen}}}ir selber den »Dreh« der \uuline{\edtext{D}{\Cendnote{dreifach unterstrichen}}}ir passt) Dienstag{ }Früh.\pend
           \pstart
           Herzlichst{\\[\baselineskip]}\spacefill\mbox{Richard}\pend
           \leftskip=0em{}\endnumbering\briefempfaengerindex{Schnitzler, Arthur@\textsc{Schnitzler, Arthur}!zzzBeer-Hofmann, Richard@\emph{von Richard Beer-Hofmann}!1909-06-261@{26. 6. 1909}|)be}\mylabel{h}  \normalsize

\doendnotes{C}
\bigskip
\vfill

\clearpage

\footnotesize

\lohead{\textsc{register}}

% Definiere theindex-Environment komplett neu ohne reledmac
\makeatletter
\renewenvironment{theindex}{%
  \section*{\indexname}%
  \setlength{\parindent}{0pt}%
  \setlength{\parskip}{0pt plus 0.3pt}%
  \let\item\@idxitem
}{%
  \clearpage
}
\makeatother

\IfFileExists{\jobname-pw.ind}{\input{\jobname-pw.ind}}{}

\end{document}

      