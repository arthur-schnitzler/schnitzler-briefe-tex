%% latex-korrekturansicht-vorspann.tex
%% Vorspann für die Korrekturansicht.
%% Lädt die gemeinsame Datei latex-vorspann.tex mit gesetztem Schalter.

\newif\ifkorrekturansicht
\korrekturansichttrue

\input{../tex-inputs/latex-vorspann}


               \section[Hugo Hofmannsthal an Arthur Schnitzler, 8. 12. {[}1919{]}]{ Hugo Hofmannsthal an Arthur Schnitzler,
               8. 12. {[}1919{]}}\nopagebreak\mylabel{v}\rehead{ }\normalsize\beginnumbering\briefempfaengerindex{Schnitzler, Arthur@\textsc{Schnitzler, Arthur}!zzzHofmannsthal, Hugo von@\emph{von Hugo von Hofmannsthal}!1919-12-082@{8. 12. {[}1919{]}}|(be} \toendnotes[C]{\smallbreak\pagebreak[2]} \Standort{CUL, Schnitzler, B 43.}
\physDesc{Brief, 1 Blatt, 2 Seiten
\newline{}Handschrift: schwarze Tinte, deutsche Kurrent
\newline{}Schnitzler: mit Bleistift die Jahreszahl ein zweites Mal ergänzt:
               »19« \newline{}Ordnung: 1) mit Bleistift von \textcolor{blue}{Frieda Pollak} (?) mit dem Buchstaben »A« (Abgeschrieben/Abschrift) gekennzeichnet 2) mit Bleistift von unbekannter Hand nummeriert: »\strikeout{353}«3) mit Bleistift von unbekannter Hand nummeriert: »384«}\buchAbdrucke{\weitereDrucke{Hugo von Hofmannsthal, Arthur Schnitzler: \emph{Briefwechsel}. Hg. Therese Nickl und Heinrich Schnitzler. Frankfurt am Main: \emph{S. Fischer} 1964, S. 289.} }\toendnotes[C]{\smallbreak}\pstart
           \raggedleft{}{\pb}\textcolor{pink}{R.}{}\ledrightnote{\textcolor{pink}{Rodaun}}{ }8 XII \textsubscript{19}.\pend
           \pstart{}mein lieber Arthur\pend\pstart
           ich dank Ihnen ſchön für den Brief den Sie mir nach \textcolor{pink}{Auſſee}{}\ledrightnote{\textcolor{pink}{Bad Aussee}} geſchrieben haben.\hspace*{1.5em}Ich bin nun
               zurück und wünſche mir, wie herzlich, Sie zu ſehen.\hspace*{1.5em}Aber ich bin ſelten in der Stadt – \textcolor{blue}{Gerty}{}\ledrightnote{\textcolor{blue}{Gertrude von Hofmannsthal}} und die
                  \textcolor{blue}{Kinder}{}\ledrightnote{→\textcolor{blue}{Christiane von Hofmannsthal}{\newline}→\textcolor{blue}{Raimund von Hofmannsthal}{\newline}→\textcolor{blue}{Franz von Hofmannsthal}} weit
               öfter, ich aber hab mir hier ein ganz kleines Zimmer bei \textcolor{pink}{Rodaun}{}\ledrightnote{\textcolor{pink}{Rodaun}}er \textcolor{blue}{Leuten}{}\ledrightnote{→\textcolor{blue}{?? [Vermieter von Hugo von Hofmannsthal]}}
               gemiethet das ſich mit Holz erträglich heizen läſst und ſo bleib ich ſo viel als
               möglich heraußen, eine leidliche Productivität im Fluſs zu halten, denn ich kenne
               mich vor angefangenen Dingen, Plänen u. \textsc{Scenarien} wirklich
                  {\pb}nicht aus und muſs sehen, daſs
               alles weiter \label{T_L02332_1v}\edtext{k\textcolor{gray}{o{\geminationm}t}\strikeout{e}}{\lemma{\textnormal{\emph{kote}}}\Cendnote{\textnormal{unsichere Lesart; von unbekannter Hand mit Bleistift unterstrichen und am Rand
                  mit einem Fragezeichen markiert.}}}\label{T_L02332_1h}. (Von Ihrem \textcolor{green}{\textsc{Casanova}ſtück}{}\ledrightnote{→\textcolor{green}{Die Schwestern oder Casanova in Spa. Lustspiel in Versen}} höre ich übrigens daſs es beſonders
               reizend fröhlich u. erfreuend iſt, und daſs es bald geſpielt wird, melde mich alſo
               hiemit für die \label{K_L02332_1v}\edtext{Première}{\lemma{\textnormal{\emph{Première}}}\Cendnote{\textnormal{siehe A. S.: \emph{Tagebuch}, 26. 3. 1920}}}\label{K_L02332_1h}.)\pend
           \pstart
           Wie ſehe ich Sie aber mit alledem? Welche Stunde, mit \textcolor{blue}{Olga}{}\ledrightnote{\textcolor{blue}{Olga Schnitzler}} in die Stadt zu uns zu ko{\geminationm}en iſt denn
               Ihnen u. ihr halbwegs convenierend?\pend
           \pstart
           Sie ſind der Mann der ſtrengen Einteilung, ich bin, \uline{wenn} ich in der Stadt bin, alle Wochen 1 ½ – 2 Tage, dann ganz
               frei! Also ſchreiben Sie mir ein Wort, wie Sie’s beide wollen, ob Sie zu einem ſehr
               beſcheidenen Nachtmahl \label{T_L02332_2v}\edtext{ko{\geminationm}en
               wollen, das wäre das Gemütlichſte – oder wie immer! Ihr \spacefill\mbox{Hugo.}}{\lemma{\textnormal{\emph{koen … Hugo.}}}\Cendnote{\textnormal{quer am linken
                  Rand}}}\label{T_L02332_2h}\pend
           \endnumbering\briefempfaengerindex{Schnitzler, Arthur@\textsc{Schnitzler, Arthur}!zzzHofmannsthal, Hugo von@\emph{von Hugo von Hofmannsthal}!1919-12-082@{8. 12. {[}1919{]}}|)be}\mylabel{h}  \normalsize

\doendnotes{C}
\bigskip
\vfill

\clearpage

\footnotesize

\lohead{\textsc{register}}

% Definiere theindex-Environment komplett neu ohne reledmac
\makeatletter
\renewenvironment{theindex}{%
  \section*{\indexname}%
  \setlength{\parindent}{0pt}%
  \setlength{\parskip}{0pt plus 0.3pt}%
  \let\item\@idxitem
}{%
  \clearpage
}
\makeatother

\IfFileExists{\jobname-pw.ind}{\input{\jobname-pw.ind}}{}

\end{document}

      