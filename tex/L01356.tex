%% latex-korrekturansicht-vorspann.tex
%% Vorspann für die Korrekturansicht.
%% Lädt die gemeinsame Datei latex-vorspann.tex mit gesetztem Schalter.

\newif\ifkorrekturansicht
\korrekturansichttrue

\input{../tex-inputs/latex-vorspann}


               \section[Franz Blei an Arthur Schnitzler, 4. 1. 1904]{ Franz Blei an Arthur Schnitzler, 4. 1. 1904}\nopagebreak\mylabel{v}\rehead{ }\normalsize\beginnumbering\briefempfaengerindex{Schnitzler, Arthur@\textsc{Schnitzler, Arthur}!zzzBlei, Franz@\emph{von Franz Blei}!1904-01-041@{4. 1. 1904}|(be} \toendnotes[C]{\smallbreak\pagebreak[2]} \Standort{CUL, Schnitzler, B 14.}
\physDesc{Brief, 1 Blatt, 3 Seiten
\newline{}Handschrift: schwarze Tinte, lateinische Kurrent
\newline{}Schnitzler: 1) mit Bleistift beschriftet: »\textsc{Blei}« und datiert: »4. 1. 904« 2) mit rotem Buntstift mehrere Unterstreichungen\newline{}Ordnung: 1) mit Bleistift von unbekannter Hand nummeriert: »\strikeout{2}« 2) mit Bleistift von unbekannter Hand nummeriert: »3«}\toendnotes[C]{\smallbreak}\pstart
           \raggedleft{}{\pb}\textcolor{pink}{München, Arcisstrasse 19}{}\ledrightnote{\textcolor{pink}{Arcisstraße}}\pend
           \pstart{}Verehrter Herr Doktor,\pend\pstart
           meine unvorhergesehene frühe Abreise von \textcolor{pink}{Wien}{}\ledrightnote{\textcolor{pink}{Wien}}
                    liess es nicht dazu kommen, dass ich Sie, wie ich so gern gethan hätte,
                    besuchte. Was mir sehr leid thut.\pend
           \pstart
           Heute schreibt mir Miss \textcolor{blue}{Johnson}{}\ledrightnote{\textcolor{blue}{Fanny Johnson}}, \textcolor{pink}{Oxford}{}\ledrightnote{\textcolor{pink}{Oxford}}, dass Ihr \textcolor{pink}{englischer}{}\ledrightnote{\textcolor{pink}{England}}{ }\textcolor{blue}{Verleger}{}\ledrightnote{→\textcolor{blue}{Alfred Bates}} »\begin{otherlanguage}{english}distinctly shady\end{otherlanguage}« sei, was sie in Ihrem wie in ihrem
                    Interesse bedauert. Doch lässt sie sich dadurch nicht abhalten, die angefangene
                    Übertragung des »\textcolor{green}{Grünen Kakadu}{}\ledrightnote{\textcolor{green}{Der grüne Kakadu. Groteske in einem Akt}}« {\pb}zu beenden, aus Freude an der Sache,
                    denn ihr Honorar sei eine leere Versprechung. Ich berichte damit nur was die \textcolor{blue}{Dame}{}\ledrightnote{→\textcolor{blue}{Fanny Johnson}} schreibt und kann
                    meinerseits nur sagen, dass sie soweit ich sie kenne, recht haben wird wenn sie
                    den \textcolor{blue}{Verleger}{}\ledrightnote{→\textcolor{blue}{Alfred Bates}} nicht
                        \begin{otherlanguage}{english}reputable\end{otherlanguage} findet. Für die Zukunft möchte
                    ich Ihnen \textcolor{brown}{Heinemann}{}\ledrightnote{\textcolor{brown}{William Heinemann Ltd}}, den ich persönlich und als
                    einen sehr noblen Geschäftsmann {\pb}kenne{[}, anempfehlen{]}. Wenn Sie Miss \textcolor{blue}{Johnson}{}\ledrightnote{\textcolor{blue}{Fanny Johnson}} die Übersetzung Ihrer Novellen anvertrauen,
                    werden Sie dazu in \textcolor{blue}{W. Heinemann}{}\ledrightnote{\textcolor{blue}{William Heinemann}} einen in
                    jeder Beziehung vortrefflichen Verleger haben, sowohl was Reputation als
                    Ausstattung als besonders Honorar betrifft.\pend
           \pstart
           Mit bestem Gruss{\\[\baselineskip]}Ihr ganz ergebener{\\[\baselineskip]}\spacefill\mbox{Dr Franz Blei}\pend
           \leftskip=0em{}\pstart
           4. 1. 1904\pend
           \endnumbering\briefempfaengerindex{Schnitzler, Arthur@\textsc{Schnitzler, Arthur}!zzzBlei, Franz@\emph{von Franz Blei}!1904-01-041@{4. 1. 1904}|)be}\mylabel{h}  \normalsize

\doendnotes{C}
\bigskip
\vfill

\clearpage

\footnotesize

\lohead{\textsc{register}}

% Definiere theindex-Environment komplett neu ohne reledmac
\makeatletter
\renewenvironment{theindex}{%
  \section*{\indexname}%
  \setlength{\parindent}{0pt}%
  \setlength{\parskip}{0pt plus 0.3pt}%
  \let\item\@idxitem
}{%
  \clearpage
}
\makeatother

\IfFileExists{\jobname-pw.ind}{\input{\jobname-pw.ind}}{}

\end{document}

      