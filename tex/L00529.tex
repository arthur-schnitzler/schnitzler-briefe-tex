%% latex-korrekturansicht-vorspann.tex
%% Vorspann für die Korrekturansicht.
%% Lädt die gemeinsame Datei latex-vorspann.tex mit gesetztem Schalter.

\newif\ifkorrekturansicht
\korrekturansichttrue

\input{../tex-inputs/latex-vorspann}


               \section[Lou Andreas-Salomé an Arthur Schnitzler, 18. 1. 1896]{ Lou Andreas-Salomé an Arthur Schnitzler, 18. 1. 1896}\nopagebreak\mylabel{v}\rehead{ }\normalsize\beginnumbering\briefempfaengerindex{Schnitzler, Arthur@\textsc{Schnitzler, Arthur}!zzzAndreas-Salome, Lou@\emph{von Lou Andreas-Salomé}!1896-01-181@{18. 1. 1896}|(be} \toendnotes[C]{\smallbreak\pagebreak[2]} \Standort{CUL, Schnitzler, B 3.}
\physDesc{Kartenbrief
\newline{}Handschrift: schwarze Tinte, deutsche Kurrent\newline{}Versand: 1) Stempel: »\nobreak{}\oindex{I., Innere Stadt@\textbf{I., Innere Stadt}, \emph{Bezirk (A.BZK)}|pwk}Wien 1/1, 18. 1. 96, 2–3V\nobreak{}«.  2) Stempel: »\nobreak{}\oindex{IX., Alsergrund@\textbf{IX., Alsergrund}, \emph{Bezirk (A.BZK)}|pwk}Wien 9/3, 18. 1. 96, 5 N\nobreak{}«. 
\newline{}Schnitzler: mit Bleistift datiert: »18/1 96« \newline{}Ordnung: mit Bleistift von unbekannter Hand nummeriert:
                                        »16« }\pstart{}{\pb}Herrn \textsc{D\textsuperscript{r}}\pend{}\pstart{}\textsc{Arthur Schnitzler}\pend{}\pstart{}\textsc{\textcolor{pink}{Wien}{}\ledrightnote{\textcolor{pink}{Wien}}}\pend{}\pstart{}\textcolor{pink}{Frankgasse 1}{}\ledrightnote{\textcolor{pink}{Frankgasse}}.
                    \pend{}{\bigskip}\pstart
           \noindent{}{\pb}Lieber Herr \textsc{D\textsuperscript{r}}! es thut mir ſchrecklich leid, daß Sie heute Morgen
                    vergeblich kamen. ich hatte die Nacht gelumpt und befand mich nicht ganz gut,
                    blieb wegen dieſer beiden Dinge zu Bett. Morgen bin ich von früh bis Abends am
                    Land, aber Montag frei, und freue mich darauf, Sie zu ſprechen. Es iſt Ihnen
                    ſicher bequemer, wenn ich zu Ihnen in die Sprechſtunde komme, was ich dann
                    Montag zwiſchen 3–4 Uhr thun würde, falls Sie nicht weiter
                    antworten. Zum \textcolor{pink}{\textsc{Griensteidl}}{}\ledrightnote{\textcolor{pink}{Café Griensteidl}} kann ich mich nicht mehr recht entſchließen, aber vielleicht ſind
                    wir noch einmal im Theater oder ſonſtwo zuſammen?\pend
           \pstart
           Mit herzlichem Gruß{\\[\baselineskip]}Ihre \spacefill\mbox{LouAS.}\pend
           \leftskip=0em{}\endnumbering\briefempfaengerindex{Schnitzler, Arthur@\textsc{Schnitzler, Arthur}!zzzAndreas-Salome, Lou@\emph{von Lou Andreas-Salomé}!1896-01-181@{18. 1. 1896}|)be}\mylabel{h}  \normalsize

\doendnotes{C}
\bigskip
\vfill

\clearpage

\footnotesize

\lohead{\textsc{register}}

% Definiere theindex-Environment komplett neu ohne reledmac
\makeatletter
\renewenvironment{theindex}{%
  \section*{\indexname}%
  \setlength{\parindent}{0pt}%
  \setlength{\parskip}{0pt plus 0.3pt}%
  \let\item\@idxitem
}{%
  \clearpage
}
\makeatother

\IfFileExists{\jobname-pw.ind}{\input{\jobname-pw.ind}}{}

\end{document}

      