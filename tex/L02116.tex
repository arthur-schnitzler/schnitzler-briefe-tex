%% latex-korrekturansicht-vorspann.tex
%% Vorspann für die Korrekturansicht.
%% Lädt die gemeinsame Datei latex-vorspann.tex mit gesetztem Schalter.

\newif\ifkorrekturansicht
\korrekturansichttrue

\input{../tex-inputs/latex-vorspann}


               \section[Georg Brandes an Arthur Schnitzler, 5. 3. 1913]{ Georg Brandes an Arthur Schnitzler, 5. 3. 1913}\nopagebreak\mylabel{v}\rehead{ }\normalsize\beginnumbering\briefempfaengerindex{Schnitzler, Arthur@\textsc{Schnitzler, Arthur}!zzzBrandes, Georg@\emph{von Georg Brandes}!1913-03-051@{5. 3. 1913}|(be} \toendnotes[C]{\smallbreak\pagebreak[2]} \Standort{CUL, Schnitzler, B 17.}
\physDesc{Postkarte
\newline{}Handschrift: schwarze Tinte, lateinische Kurrent\newline{}Versand: Stempel: »\nobreak{}\oindex{Taormina@\textbf{Taormina}, \emph{Besiedelter Ort (A.BSO)}|pwk}Taormina Messina, 6 3 13\nobreak{}«.  \newline{}Ordnung: mit Bleistift von unbekannter Hand nummeriert: »41« }\buchAbdrucke{\weitereDrucke{Georg Brandes, Arthur Schnitzler: \emph{Ein Briefwechsel}. Hg. Kurt Bergel. Bern: \emph{Francke} 1956, S. 107.} }\toendnotes[C]{\smallbreak}\pstart{}{\pb}Herrn Dr. Arthur
                        Schnitzler\pend{}\pstart{}\textcolor{pink}{71 Sternwartestrasse Wien XVIII}{}\ledrightnote{\textcolor{pink}{Sternwartestraße}}\pend{}\pstart{}\textcolor{pink}{Vienna}{}\ledrightnote{\textcolor{pink}{Wien}}\pend{}\pstart{}\textcolor{pink}{Austria}{}\ledrightnote{\textcolor{pink}{Österreich}}\pend{}{\bigskip}\pstart
           \raggedleft{}{\pb}5 März 13\pend
           \pstart{}Mein verehrtester Freund\pend\pstart
           Ich erhalte hier (\textcolor{pink}{Hotel Métropole, Taormina}{}\ledrightnote{\textcolor{pink}{Grand Hotel Metropol}})
                    Ihren liebenswürdigen Brief, der mir zeigt, dass ich Unrecht hatte zu glauben,
                    was die Professorin \textcolor{blue}{Zuckerkandl}{}\ledrightnote{\textcolor{blue}{Berta Zuckerkandl}} mir in \textcolor{pink}{Wien}{}\ledrightnote{\textcolor{pink}{Wien}} über den Anlass Ihres \textcolor{green}{Schauspiels}{}\ledrightnote{→\textcolor{green}{Professor Bernhardi. Komödie in fünf Akten}} erzählte. Ich bitte Sie
                    meinen Irrthum zu entschuldigen. Man sollte nie Vertrauen an dergleichen
                    Mittheilungen haben.\pend
           \pstart
           Ich habe nie die Uebersetzung jenes vor Monaten geschriebenen \textcolor{green}{Artikels}{}\ledrightnote{→\textcolor{green}{Theater und Schauspiele in Deutschland}} gesehen, und ich hatte sogar
                    ganz vergessen, dass ich vor Monaten den \textcolor{blue}{Photographen}{}\ledrightnote{→\textcolor{blue}{?? [Fotograf in Paris]}} in \textcolor{pink}{Paris}{}\ledrightnote{\textcolor{pink}{Paris}} bat, Ihnen mein \textcolor{green}{Bild}{}\ledrightnote{→\textcolor{green}{[Georg Brandes]}} zu senden.\pend
           \pstart
           Es geht mir mit Ihnen heute, wie es mir wöchentlich mit meiner liebsten \textcolor{blue}{Freundin}{}\ledrightnote{→\textcolor{blue}{Bertha Knudtzon}} geht, die
                    augenblicklich, auf einer Seereise begriffen, sich in \textcolor{pink}{Hongkong}{}\ledrightnote{\textcolor{pink}{Hong Kong}} befindet. Wenn Ihre Antworten kommen, verstehe
                    ich sie kaum, weil ich meine alten Briefe ganz vergessen habe.\pend
           \pstart
           Ich war nach \textcolor{pink}{Paris}{}\ledrightnote{\textcolor{pink}{Paris}} in \textcolor{pink}{Pallanza}{}\ledrightnote{\textcolor{pink}{Pallanza}}, \textcolor{pink}{Rom}{}\ledrightnote{\textcolor{pink}{Rom}}, \textcolor{pink}{Neapel}{}\ledrightnote{\textcolor{pink}{Neapel}}, \textcolor{pink}{Palermo}{}\ledrightnote{\textcolor{pink}{Palermo}} und längere Zeit in \textcolor{pink}{Tunis}{}\ledrightnote{\textcolor{pink}{Tunis}},
                    das mir sehr gefiel trotz des ungünstigsten Wetters.\pend
           \pstart
           Ich soll im April in \textcolor{pink}{Neapel}{}\ledrightnote{\textcolor{pink}{Neapel}} und \textcolor{pink}{Rom}{}\ledrightnote{\textcolor{pink}{Rom}} reden, denke etwa am 1 Mai in
                        \textcolor{pink}{Kopenhagen}{}\ledrightnote{\textcolor{pink}{Kopenhagen}} zurück zu sein. Hier bleibe
                    ich ungefähr {\pb}drei Wochen.
                    Hier hab ich endlich Sonne gefunden.\pend
           \pstart
           Habe ich mich auch unrichtig ausgedrückt, können Sie wenigstens nicht meine
                    freundschaftliche Gesinnung bezweifeln.\pend
           \pstart
           Ihre werthe und liebe Frau \textcolor{blue}{Gemahlin}{}\ledrightnote{→\textcolor{blue}{Olga Schnitzler}} und die beiden mir so lieben \textcolor{blue}{Beer-Hofmanns}{}\ledrightnote{\textcolor{blue}{Richard Beer-Hofmann}{\newline}\textcolor{blue}{Paula Beer-Hofmann}} bitte ich an mich zu erinnern.\pend
           \pstart Ihr ergebener \spacefill\mbox{Georg Brandes}\pend{}\endnumbering\briefempfaengerindex{Schnitzler, Arthur@\textsc{Schnitzler, Arthur}!zzzBrandes, Georg@\emph{von Georg Brandes}!1913-03-051@{5. 3. 1913}|)be}\mylabel{h}  \normalsize

\doendnotes{C}
\bigskip
\vfill

\clearpage

\footnotesize

\lohead{\textsc{register}}

% Definiere theindex-Environment komplett neu ohne reledmac
\makeatletter
\renewenvironment{theindex}{%
  \section*{\indexname}%
  \setlength{\parindent}{0pt}%
  \setlength{\parskip}{0pt plus 0.3pt}%
  \let\item\@idxitem
}{%
  \clearpage
}
\makeatother

\IfFileExists{\jobname-pw.ind}{\input{\jobname-pw.ind}}{}

\end{document}

      