%% latex-korrekturansicht-vorspann.tex
%% Vorspann für die Korrekturansicht.
%% Lädt die gemeinsame Datei latex-vorspann.tex mit gesetztem Schalter.

\newif\ifkorrekturansicht
\korrekturansichttrue

\input{../tex-inputs/latex-vorspann}


               \section[Arthur Schnitzler an Georg Brandes, 14. 12. 1924]{ Arthur Schnitzler an Georg Brandes, 14. 12. 1924}\nopagebreak\mylabel{v}\rehead{ }\normalsize\beginnumbering\briefempfaengerindex{Brandes, Georg@\textsc{Brandes, Georg}!zzzSchnitzler, Arthur@\emph{von Arthur Schnitzler}!1924-12-141@{14. 12. 1924}|(be} \toendnotes[C]{\smallbreak\pagebreak[2]} \Standort{Kopenhagen, Det Kongelige Bibliotek, Georg Brandes Arkiv, box 125.}
\physDesc{Brief, 4 Blätter (die weiteren Blätter von Schnitzler nummeriert: »II«–»IV«, die Rückseite des zweiten Blattes blieb unbeschrieben), 7 Seiten
\newline{}Handschrift: schwarze Tinte, lateinische Kurrent\newline{}Ordnung: 1) mit Bleistift von unbekannter Hand nummeriert:
                                    »49.« 2) mit Bleistift von unbekannter Hand die weiteren Blätter datiert:
                                       »14/12 24«}\buchAbdrucke{\weitereDrucke{Georg Brandes, Arthur Schnitzler: \emph{Ein Briefwechsel}. Hg. Kurt Bergel. Bern: \emph{Francke} 1956, S. 141–143.} }\toendnotes[C]{\smallbreak}\pstart
           \raggedleft{}{\pb}\textcolor{pink}{Wien}{}\ledrightnote{\textcolor{pink}{Wien}}{ }14. 12. 924\pend
           \pstart
           mein lieber und verehrter Freund, den Empfang Ihres Briefes vom
                  10. Dezember will ich gleich mit dem herzlichsten Dank bestätigen.
               Denken Sie, \uline{mit der gleichen Post} kam Ihr \textcolor{green}{Julius Caesar}{}\ledrightnote{\textcolor{green}{Gaius Julius Cæsar}} – vom Verleger \introOben{}(\textcolor{brown}{Reiss}{}\ledrightnote{\textcolor{brown}{Erich Reiß}})\introOben{} übersandt, zugleich mit dem dritten
               Band der neuen Ausgabe der \textcolor{green}{Hauptströmungen}{}\ledrightnote{\textcolor{green}{Hauptströmungen der Literatur des neunzehnten Jahrhunderts}}. Also –
               dieser \textcolor{green}{Caesar}{}\ledrightnote{\textcolor{green}{Gaius Julius Cæsar}} ist ohne Ihre Autorisation in \textcolor{pink}{Deutschland}{}\ledrightnote{\textcolor{pink}{Deutschland}} erschienen? Aber \textcolor{green}{Voltaire}{}\ledrightnote{\textcolor{green}{Voltaire und sein Jahrhundert}}, \textcolor{green}{Michel Angelo}{}\ledrightnote{\textcolor{green}{Michelangelo Buonarotti}}, \textcolor{green}{Goethe}{}\ledrightnote{\textcolor{green}{Wolfgang Goethe}} – das sind doch autorisirte deutsche
               Ausgaben? Bitte sagen Sie mir ein Wort darüber. Ich erkundigte mich im vergangenen
               Frühjahr – anläßlich meiner Bestätigung der eingetroffenen anderen Brandes Bände, –
                  \introOben{}bei \textcolor{brown}{Reiss}{}\ledrightnote{\textcolor{brown}{Erich Reiß}}\introOben{} für wann der \textcolor{green}{Caesar}{}\ledrightnote{\textcolor{green}{Gaius Julius Cæsar}} zu erwarten sei – er
               erwiderte, dſs er ihn gleich nach Erscheinen an mich senden werde – das hat er {\pb}nun gethan – und \uline{Sie} sollten erst durch mich authentisches von diesem deutschen \textcolor{green}{Caesar}{}\ledrightnote{\textcolor{green}{Gaius Julius Cæsar}} erfahren – u hatten nicht einmal Honorar
               erhalten??\pend
           \pstart
           – Die Angelegenheit irritirt mich vielleicht darum ein bißchen mehr, weil ich immer
               wieder so arge und ärgerliche Dinge mit meinen Büchern im Ausland erlebe. Noch nie
               ist der Diebstahl, jeder Raub am geistigen Eigenthum so schamlos betrieben worden als
               jetzt! Man muſs Mahnbriefe schreiben, Prozeſſe führen – oh nicht nur ins Ausland; –
               auch in nächste Nähe, – verschwendet Zeit und Geisteskraft an geschäftliche
               Correspondenzen – und erreicht so wenig! – Aber genug davon. –\pend
           \pstart
           Es freut mich, dſs Ihnen die \textcolor{green}{Kom. der Verführung}{}\ledrightnote{\textcolor{green}{Komödie der Verführung. In drei Akten}}
               einigen Spaſs gemacht hat und dſs Sie mir die Palmen, die ich in \textcolor{pink}{Gilleleje}{}\ledrightnote{\textcolor{pink}{Gilleleje}} wachsen {\pb}lieſs,
               nicht übel genommen haben – \label{T_L02423_1v}\edtext{(}{\lemma{\textnormal{\emph{(}}}\Cendnote{\textnormal{die öffnende Klammer doppelt platziert,
                  vermutlich eine versuchte Verdeutlichung, da bei einer Tinte fehlt}}}\label{T_L02423_1h}im
               Gegensatz zu einer Landsmännin (und entfernten Verwandten) von Ihnen glaub ich), der
               Frau \textcolor{blue}{Karen Stampe Bendix}{}\ledrightnote{\textcolor{blue}{Karen Stampe Bendix}}, die ich manchmal sehe –
               und die eine reizende kleine \textcolor{blue}{Tochter}{}\ledrightnote{→\textcolor{blue}{Lillian Ellis}} – Tänzerin hat.) Das \textcolor{green}{Stück}{}\ledrightnote{→\textcolor{green}{Komödie der Verführung. In drei Akten}} hat es ziemlich schwer und wird sich – wie es mit meinen meisten
               Stücken geht – von meinen allerersten abgesehen, – nur allmälig durchsetzen. Die
               Verlogenheit der Kritik in »moralischen« Dingen ist seltsamerweise – je freier die
               Existenz gerade in dieser Hinsicht sich gestaltet hat – ungeheuerlicher als je. Für
               mich hat jetzt das Völkchen eine neue Formel gefunden: dſs ich nemlich eine
               »versunkene Welt« gestalte, für die sich kein Mensch mehr interessire. (Man darf nur
               Dramen von 1924 schreiben – haben Sie das gewußt?) Auch sind Tod und
               Liebe unwürdige Sujets; – nur Grenzregulirungen, Valutenaenderungen, Steuerfragen,
               Diebstähle und Hungerrevolten interessiren den {\pb}ernsten (insbesondere ernsten deutschen) Mann. –\pend
           \pstart
           Hab ich Ihnen schon einmal geschrieben, dſs mein Sohn \textcolor{blue}{Heinrich}{}\ledrightnote{\textcolor{blue}{Heinrich Schnitzler}} in \textcolor{pink}{Berlin Staatstheater}{}\ledrightnote{\textcolor{pink}{Schauspielhaus}} engagirt
               ist? Er fühlt sich dort sehr wohl; er wird wohl allmälig nach dem Regisseur und
               Theaterdirector zu sich entwickeln. Anfangs sah's aus, als würd er Kapellmeister
               werden.\pend
           \pstart
           Meine \textcolor{blue}{Frau}{}\ledrightnote{→\textcolor{blue}{Olga Schnitzler}} lebt in \textcolor{pink}{Baden-Baden}{}\ledrightnote{\textcolor{pink}{Baden-Baden}}; – so bin ich jetzt hier mit meiner
               fünfzehnjährigen aber sehr erwachsenen Tochter \textcolor{blue}{Lili}{}\ledrightnote{\textcolor{blue}{Lili Schnitzler}} (Interesse: Sprachen, – Theater, – Geschichte (vor allem \textcolor{blue}{Friedrich II}{}\ledrightnote{\textcolor{blue}{Friedrich II. von Preußen}} und \textcolor{blue}{Napoleon}{}\ledrightnote{\textcolor{blue}{Napoleon Bonaparte}}) – Eisläufen und Tanzen) allein, sehe aber ziemlich viele Menschen
               – die Hälfte davon \substVorne{}\textsuperscript{selten}{\allowbreak}\substDazwischen{}kaum\substHinten{} öfter als 1–2 Mal. Auch so liebe Freunde wie \textcolor{blue}{Richard Beer Hofmann}{}\ledrightnote{\textcolor{blue}{Richard Beer-Hofmann}} seh ich eigentlich selten; – und \textcolor{blue}{Hofmannsthal}{}\ledrightnote{\textcolor{blue}{Hugo von Hofmannsthal}} – da gibt es Pausen bis zu einem Jahr! \textcolor{blue}{B.-H}{}\ledrightnote{\textcolor{blue}{Richard Beer-Hofmann}} hat jetzt einen erheblichen Erfolg als
               Regisseur gehabt; er hat ein \textcolor{pink}{englisches}{}\ledrightnote{\textcolor{pink}{England}}{ }\textcolor{green}{Stück}{}\ledrightnote{→\textcolor{green}{Überfahrt. Schauspiel in drei Akten}} umgearbeitet {\pb}und \label{K_L02423_1v}\edtext{inszenirt}{\lemma{\textnormal{\emph{inszenirt}}}\Cendnote{\textnormal{\textcolor{blue}{Sutton Vane}s \emph{\textcolor{green}{Outward Bound}} wurde am 14. 11. 1924 im \textcolor{pink}{Theater in der Josefstadt} in der Übersetzung von \textcolor{blue}{Otto Klement} – also unter Pseudonym – und in
                  Regie von \textcolor{blue}{Beer-Hofmann} gegeben.}}}\label{K_L02423_1h}. Seine
               Tochter \textcolor{blue}{Mirjam}{}\ledrightnote{\textcolor{blue}{Mirjam Beer-Hofmann}} hat geheiratet, und wird mit ihrem
                  \textcolor{blue}{Gatten}{}\ledrightnote{→\textcolor{blue}{Ernst Lens}} wahrscheinlich bald
               nach \textcolor{pink}{Kopenhagen}{}\ledrightnote{\textcolor{pink}{Kopenhagen}} übersiedeln. –\pend
           \pstart
           Es erscheinen bald wieder \textcolor{green}{Novellen}{}\ledrightnote{→\textcolor{green}{Die Frau des Richters. Novelle}{\newline}→\textcolor{green}{Traumnovelle}} von mir, – und ein \textcolor{green}{Versstück}{}\ledrightnote{→\textcolor{green}{Der Gang zum Weiher. Dramatische Dichtung}} wird vielleicht auch bald fertig sein; – besonders viel aber feil
               ich an aphoristisch-fragmentistischem herum – mein Bedürfnis, in möglichst \strikeout{kurz} praeciser u conciser Form gewisse Lebenswahrheiten
               auszusprechen – die natürlich an sich nicht neu sind – zu denen ich aber meinen
               eigenen Weg gegangen bin – dieses Bedürfnis wird mit den Jahren immer stärker. Es ist
               auch etwas Pedanterie und etwas Verspieltheit dabei.\pend
           \pstart
           Ich bin sehr glücklich, dſs Sie immer in gleicher Herzlichkeit meiner gedenken – was
               Sie mir bedeuten, – muſs ich Ihnen das noch sagen? Ich hoffe Sie sind schon ganz wohl
               und der \textcolor{green}{\textcolor{blue}{Jesus}{}\ledrightnote{\textcolor{blue}{Jesus}}}{}\ledrightnote{→\textcolor{green}{Urkristendom}} ist bald vollendet. Was Sie, Georg Brandes, {\pb}in diesem letzten Jahrzehnt gemacht haben – und \uline{wie}
               Sie es gemacht haben –; gibt es dafür in der Geschichte menschlicher Geistesarbeit
               ein Analogon? Und wie menschlich nah sind Sie einem \introOben{}in\introOben{} jedem
               Ihrer Bücher, wie liebt man Sie in jedem! – Und ob \textcolor{blue}{Jesus}{}\ledrightnote{\textcolor{blue}{Jesus}} ein Lebendiger oder ein Mythos war; – in Ihrem \textcolor{green}{Buch}{}\ledrightnote{→\textcolor{green}{Urkristendom}} wird er beides zu sein verstehn. –\pend
           \pstart
           Im Januar werd ich wahrscheinlich eine Vortragsreise in der \textcolor{pink}{Schweiz}{}\ledrightnote{\textcolor{pink}{Schweiz}} machen. Und wann sieht man einander wieder?
               Sie haben's ja in der \textcolor{green}{Komoedie der Verf.}{}\ledrightnote{\textcolor{green}{Komödie der Verführung. In drei Akten}} gelesen:
               das Alter ist nur eine Intrigue, die die Jugend gegen uns einfädelt. In meinem
               nächsten \textcolor{green}{Stück}{}\ledrightnote{→\textcolor{green}{Der Gang zum Weiher. Dramatische Dichtung}} soll der
               Neunzigjährige als Sieger übrig bleiben.\pend
           \pstart
           {\pb}– Schreiben Sie mir bald wieder einen Brief, mein
               verehrter Freund – oder we{\geminationn}s Ihnen leichter von der Hand
               gehen sollte, ein Buch. Es darf ja auch eins über Brandes sein.\pend
           \pstart
           Seien Sie herzlichst gegrüßt von{\\[\baselineskip]}Ihrem getreuen{\\[\baselineskip]}\spacefill\mbox{Arthur Schnitzler}\pend
           \leftskip=0em{}\pstart
           \noindent{}Verzeihen Sie die Klexe! Fließende Tinte – neue \textcolor{pink}{englische}{}\ledrightnote{\textcolor{pink}{England}} Feder, – Ungeschicklichkeit. –\pend
           \endnumbering\briefempfaengerindex{Brandes, Georg@\textsc{Brandes, Georg}!zzzSchnitzler, Arthur@\emph{von Arthur Schnitzler}!1924-12-141@{14. 12. 1924}|)be}\mylabel{h}  \normalsize

\doendnotes{C}
\bigskip
\vfill

\clearpage

\footnotesize

\lohead{\textsc{register}}

% Definiere theindex-Environment komplett neu ohne reledmac
\makeatletter
\renewenvironment{theindex}{%
  \section*{\indexname}%
  \setlength{\parindent}{0pt}%
  \setlength{\parskip}{0pt plus 0.3pt}%
  \let\item\@idxitem
}{%
  \clearpage
}
\makeatother

\IfFileExists{\jobname-pw.ind}{\input{\jobname-pw.ind}}{}

\end{document}

      