%% latex-korrekturansicht-vorspann.tex
%% Vorspann für die Korrekturansicht.
%% Lädt die gemeinsame Datei latex-vorspann.tex mit gesetztem Schalter.

\newif\ifkorrekturansicht
\korrekturansichttrue

\input{../tex-inputs/latex-vorspann}


               \section[Richard Beer-Hofmann an Arthur Schnitzler, 19. 6. 1910]{ Richard Beer-Hofmann an Arthur Schnitzler, 19. 6. 1910}\nopagebreak\mylabel{v}\rehead{ }\normalsize\beginnumbering\briefempfaengerindex{Schnitzler, Arthur@\textsc{Schnitzler, Arthur}!zzzBeer-Hofmann, Richard@\emph{von Richard Beer-Hofmann}!1910-06-191@{19. 6. 1910}|(be} \toendnotes[C]{\smallbreak\pagebreak[2]} \Standort{CUL, Schnitzler, B 8.}
\physDesc{Kartenbrief, 1 Blatt, 4 Seiten
\newline{}Handschrift: Bleistift, lateinische Kurrent\newline{}Versand: ohne postalischen Übermittlungsvermerk 
\newline{}Schnitzler: mit Bleistift beschriftet: »\textsc{BH}« \newline{}Ordnung: mit Bleistift von unbekannter Hand nummeriert:
                              »232« }\buchAbdrucke{\weitereDrucke{Arthur Schnitzler, Richard Beer-Hofmann: \emph{Briefwechsel 1891–1931}. Hg. Konstanze Fliedl. Wien, Zürich: \emph{Europaverlag} 1992, S. 208.} }\toendnotes[C]{\smallbreak}\pstart{}{\pb}Herrn\pend{}\pstart{}D\textsuperscript{r} Arthur Schnitzler\pend{}{\bigskip}\pstart
           \raggedleft{}{\pb}19/VI 10\pend
           \pstart
           Lieber Arthur! \textcolor{blue}{Naëmah}{}\ledrightnote{\textcolor{blue}{Naëmah Beer-Hofmann}} ist heute Früh gefallen und
               hat sich am Kinn verletzt – die Wunde reicht bis auf den Knochen – so dass wir bei
               Ihrem \textcolor{blue}{Bruder}{}\ledrightnote{→\textcolor{blue}{Julius Schnitzler}} im Spital waren
               der die Wunde vernähte. Es ist hoffentlich nichts Bedeutendes trotzdem möchte aber
                  \textcolor{blue}{Paula}{}\ledrightnote{\textcolor{blue}{Paula Beer-Hofmann}} bei dem \textcolor{blue}{Kind}{}\ledrightnote{→\textcolor{blue}{Naëmah Beer-Hofmann}}{ }{\pb}bleiben. Ist auch unruhig wenn ich
               zu Ihnen hinübergehe, da sie die Association: Scharlach – Wunde – septisch etc. nicht
               los wird. Verzeihen Sie also wenn wir heute nicht {\pb}ko{\geminationm}en,
               und so spät absagen. Herzliche Grüße Ihnen und Ihrer \textcolor{blue}{Frau}{}\ledrightnote{→\textcolor{blue}{Olga Schnitzler}}.\pend
           \pstart \spacefill\mbox{Richard}\pend{}\endnumbering\briefempfaengerindex{Schnitzler, Arthur@\textsc{Schnitzler, Arthur}!zzzBeer-Hofmann, Richard@\emph{von Richard Beer-Hofmann}!1910-06-191@{19. 6. 1910}|)be}\mylabel{h}  \normalsize

\doendnotes{C}
\bigskip
\vfill

\clearpage

\footnotesize

\lohead{\textsc{register}}

% Definiere theindex-Environment komplett neu ohne reledmac
\makeatletter
\renewenvironment{theindex}{%
  \section*{\indexname}%
  \setlength{\parindent}{0pt}%
  \setlength{\parskip}{0pt plus 0.3pt}%
  \let\item\@idxitem
}{%
  \clearpage
}
\makeatother

\IfFileExists{\jobname-pw.ind}{\input{\jobname-pw.ind}}{}

\end{document}

      