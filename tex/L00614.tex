%% latex-korrekturansicht-vorspann.tex
%% Vorspann für die Korrekturansicht.
%% Lädt die gemeinsame Datei latex-vorspann.tex mit gesetztem Schalter.

\newif\ifkorrekturansicht
\korrekturansichttrue

\input{../tex-inputs/latex-vorspann}


               \section[Arthur Schnitzler an Hugo von Hofmannsthal, 31. 10. 1896]{ Arthur Schnitzler an Hugo von Hofmannsthal, 31. 10. 1896}\nopagebreak\mylabel{v}\rehead{ }\normalsize\beginnumbering\briefempfaengerindex{Hofmannsthal, Hugo von@\textsc{Hofmannsthal, Hugo von}!zzzSchnitzler, Arthur@\emph{von Arthur Schnitzler}!1896-10-312@{31. 10. 1896}|(be} \toendnotes[C]{\smallbreak\pagebreak[2]} \Standort{FDH, Hs-30885,53.}
\physDesc{Brief, 1 Blatt, 3 Seiten
\newline{}Handschrift: Bleistift, deutsche Kurrent}\buchAbdrucke{\weitereDrucke{Hugo von Hofmannsthal, Arthur Schnitzler: \emph{Briefwechsel}. Hg. Therese Nickl und Heinrich Schnitzler. Frankfurt am Main: \emph{S. Fischer} 1964, S. 76.} }\toendnotes[C]{\smallbreak}\pstart
           \raggedleft{}{\pb}31. X. 96.\pend
           \pstart
           Lieber Hugo, iſt das liebe \label{K_L00614_1v}\edtext{Telegramm}{\lemma{\textnormal{\emph{Telegramm}}}\Cendnote{\textnormal{vgl.
                        das Telegramm von Richard \textcolor{blue}{Beer-Hofmann}
                        vom 31. 10. 1891, das keinen Absender nennt.}}}\label{K_L00614_1h} von dem »Halbwahren aus
                        \textcolor{pink}{\textsc{Upsala}}{}\ledrightnote{\textcolor{pink}{Uppsala}}« von Ihnen –?\pend
           \pstart
           Wie i{\geminationm}er; ich grüße Sie herzlich. Den \textcolor{green}{Thor u Tod}{}\ledrightnote{\textcolor{green}{Der Thor und der Tod}} hat \textcolor{blue}{Brahm}{}\ledrightnote{\textcolor{blue}{Otto Brahm}} geſtern durchgeflogen u will ihn morgen \uline{leſen}. {\pb}Die Beſetzung hab
                    ich ihm ſchon mitgetheilt. –\pend
           \pstart
           Heute war Generalprobe von \textcolor{green}{Freiwild}{}\ledrightnote{\textcolor{green}{Freiwild. Schauspiel in 3 Akten}}; \textcolor{blue}{\textsc{Gerhart Hauptmann}}{}\ledrightnote{\textcolor{blue}{Gerhart Hauptmann}} u \textcolor{blue}{\textsc{Georg Hirſchfeld}}{}\ledrightnote{\textcolor{blue}{Georg Hirschfeld}} waren dabei, und es hat offenbar auf ſie gewirkt. Mit \textsc{Hauptmann} bin ich ſchon {\pb}ein paar Mal
                        zuſa{\geminationm}en geweſen; er iſt mir außerordentlich
                    ſympathiſch; ſchon ſeine Art zu ſchauen hat mich für ihn eingenommen. –\pend
           \pstart
           Grüßen Sie \textcolor{blue}{Richard}{}\ledrightnote{\textcolor{blue}{Richard Beer-Hofmann}} vielmals!\pend
           \pstart Ihr \spacefill\mbox{Arthur}\pend{}\pstart
           \noindent{}Wie gehts der \textcolor{green}{Novelle}{}\ledrightnote{→\textcolor{green}{Geschichte der beiden Liebespaare}}?\pend
           \endnumbering\briefempfaengerindex{Hofmannsthal, Hugo von@\textsc{Hofmannsthal, Hugo von}!zzzSchnitzler, Arthur@\emph{von Arthur Schnitzler}!1896-10-312@{31. 10. 1896}|)be}\mylabel{h}  \normalsize

\doendnotes{C}
\bigskip
\vfill

\clearpage

\footnotesize

\lohead{\textsc{register}}

% Definiere theindex-Environment komplett neu ohne reledmac
\makeatletter
\renewenvironment{theindex}{%
  \section*{\indexname}%
  \setlength{\parindent}{0pt}%
  \setlength{\parskip}{0pt plus 0.3pt}%
  \let\item\@idxitem
}{%
  \clearpage
}
\makeatother

\IfFileExists{\jobname-pw.ind}{\input{\jobname-pw.ind}}{}

\end{document}

      