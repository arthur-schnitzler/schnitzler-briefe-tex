%% latex-korrekturansicht-vorspann.tex
%% Vorspann für die Korrekturansicht.
%% Lädt die gemeinsame Datei latex-vorspann.tex mit gesetztem Schalter.

\newif\ifkorrekturansicht
\korrekturansichttrue

\input{../tex-inputs/latex-vorspann}


               \section[Hugo von Hofmannsthal an Arthur Schnitzler, {[}24. 4. 1893{]}]{ Hugo von Hofmannsthal an Arthur Schnitzler, {[}24. 4. 1893{]}}\nopagebreak\mylabel{v}\rehead{ }\normalsize\beginnumbering\briefempfaengerindex{Schnitzler, Arthur@\textsc{Schnitzler, Arthur}!zzzHofmannsthal, Hugo von@\emph{von Hugo von Hofmannsthal}!1893-04-241@{{[}24. 4. 1893{]}}|(be} \toendnotes[C]{\smallbreak\pagebreak[2]} \Standort{CUL, Schnitzler, B 43.}
\physDesc{Briefkarte mit aufgeprägtem Wappen
\newline{}Handschrift: schwarze Tinte, deutsche Kurrent
\newline{}Schnitzler: mit Bleistift nummeriert: »46« und datiert: »24/4 93« \newline{}Ordnung: mit Bleistift von unbekannter Hand die frühere Zählung
                                            gestrichen und neu nummeriert:
                                            »47« }\buchAbdrucke{\weitereDrucke{Hugo von Hofmannsthal, Arthur Schnitzler: \emph{Briefwechsel}. Hg. Therese Nickl und Heinrich Schnitzler. Frankfurt am Main: \emph{S. Fischer} 1964, S. 38–39.} }\pstart
           \raggedleft{}{\pb}Montag\pend
           \pstart{}Lieber Arthur.\pend\pstart
           Ich kann Mittwoch, Donnerstag, Freitag von
                        ¼ 6 Uhr, eventuell von 4 Uhr an aufs Land, nur
                    muſs ich es 24 Stunden früher wiſſen. Bitte ſchauen Sie daſs es zuſtande
                    kommt.\pend
           \pstart
           Es wäre mir ſehr angenehm, wenn Sie die Güte hätten, \textcolor{blue}{Robert Ehrhardt}{}\ledrightnote{\textcolor{blue}{Robert Ehrhart von Ehrhartstein}} (\textcolor{pink}{\textsc{V. Siebenbrunnengasse 29}}{}\ledrightnote{\textcolor{pink}{Siebenbrunnengasse}}) durch eine Karte vom Aufhören unſerer officiellen Sonntage zu
                    verſtändigen, außer {\pb}Sie
                    wollten ihm die Freude machen ihn zu einer der bevorſtehenden Vorleſungen, wo
                    wir auch einige fremdere einladen, gleichfalls aufzufordern. Das wäre mir ſehr
                    angenehm iſt aber natürlich Sache der ſubjectiven Empfindung.\pend
           \pstart
           Auf Nachricht freut ſich\pend
           \pstart
           Ihr herzlich ergebener{\\[\baselineskip]}\spacefill\mbox{Loris.}\pend
           \leftskip=0em{}\endnumbering\briefempfaengerindex{Schnitzler, Arthur@\textsc{Schnitzler, Arthur}!zzzHofmannsthal, Hugo von@\emph{von Hugo von Hofmannsthal}!1893-04-241@{{[}24. 4. 1893{]}}|)be}\mylabel{h}  \normalsize

\doendnotes{C}
\bigskip
\vfill

\clearpage

\footnotesize

\lohead{\textsc{register}}

% Definiere theindex-Environment komplett neu ohne reledmac
\makeatletter
\renewenvironment{theindex}{%
  \section*{\indexname}%
  \setlength{\parindent}{0pt}%
  \setlength{\parskip}{0pt plus 0.3pt}%
  \let\item\@idxitem
}{%
  \clearpage
}
\makeatother

\IfFileExists{\jobname-pw.ind}{\input{\jobname-pw.ind}}{}

\end{document}

      