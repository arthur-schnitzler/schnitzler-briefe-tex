%% latex-korrekturansicht-vorspann.tex
%% Vorspann für die Korrekturansicht.
%% Lädt die gemeinsame Datei latex-vorspann.tex mit gesetztem Schalter.

\newif\ifkorrekturansicht
\korrekturansichttrue

\input{../tex-inputs/latex-vorspann}


               \section[Arthur Schnitzler an Hugo von Hofmannsthal, 9. 8. 1904]{ Arthur Schnitzler an Hugo von Hofmannsthal, 9. 8. 1904}\nopagebreak\mylabel{v}\rehead{ }\normalsize\beginnumbering\briefempfaengerindex{Hofmannsthal, Hugo von@\textsc{Hofmannsthal, Hugo von}!zzzSchnitzler, Arthur@\emph{von Arthur Schnitzler}!1904-08-091@{9. 8. 1904}|(be} \toendnotes[C]{\smallbreak\pagebreak[2]} \Standort{FDH, Hs-30885,111.}
\physDesc{Brief, 2 Blätter, 7 Seiten
\newline{}Handschrift: schwarze Tinte, deutsche Kurrent\newline{}Ordnung: mit Bleistift von Schnitzler mutmaßlich bei der
                                 Durchsicht der Korrespondenz 1929 das zweite Blatt datiert: »9/8 904 II« }\buchAbdrucke{\weitereDrucke{1) Hugo von Hofmannsthal, Arthur Schnitzler: \emph{Briefwechsel}. Hg. Therese Nickl und Heinrich Schnitzler. Frankfurt am Main: \emph{S. Fischer} 1964, S. 194–195.} \weitereDrucke{2) Arthur Schnitzler: \emph{Briefe 1875–1912}. Hg. Therese Nickl und Heinrich Schnitzler. Frankfurt am Main: \emph{S. Fischer} 1981, S. 484–485.} \weitereDrucke{3) Hermann Bahr, Arthur Schnitzler: \emph{Briefwechsel, Aufzeichnungen, Dokumente (1891–1931)}. Hg. Kurt Ifkovits und Martin Anton Müller. Göttingen: \emph{Wallstein} 2018, S. 313–314.} }\toendnotes[C]{\smallbreak}\pstart
           \raggedleft{}{\pb}\textcolor{pink}{Wien}{}\ledrightnote{\textcolor{pink}{Wien}}, 9. 8. 904,\pend
           \pstart
           \label{T_L01424_1v}\edtext{l\damage{ie}ber}{\lemma{\textnormal{\emph{lieber}}}\Cendnote{\textnormal{Den Tintenfleck
                     kommentiert Schnitzler mit Bleistift verkehrt zum Text: »\textsc{neue Tinte, pardon!}«.}}}\label{T_L01424_1h} Hugo, über \textcolor{blue}{Bahr}{}\ledrightnote{\textcolor{blue}{Hermann Bahr}}
               glaube ich Sie beruhigen zu können. Er war So{\geminationn}tag bei
               uns, da{\geminationn} haben wir zuſa{\geminationm}en
               im \textcolor{pink}{Türkenſchanzpark}{}\ledrightnote{\textcolor{pink}{Türkenschanzpark}} genachtmahlt und er war in der
               beſten Sti{\geminationm}ung. Morgen holen wir ihn Abends ab und
               fahren ins grüne. Die Hitze thut ihm im ganzen wohl; und wie er ſagt, fühlt er ſich
               durch allmäliges Steigen eher angenehm erleichtert als daſs er Beſchwerden davon
               hätte. {\pb}Seeliſche Depreſſionen wirken auf ſeinen phyſ.
               Zuſtand am heftigſten: ſo war er nach dem Tod \textcolor{blue}{Herzls}{}\ledrightnote{\textcolor{blue}{Theodor Herzl}} kränker als ſeit lang, und nach irgend einem Aerger neulich hat er
               wieder dieſes Würgen ein paar Mal gehabt, das aber nun ganz verſchwunden ſcheint. –
               Könnte man ihn doch nur dazu bringen, daſs er heuer die verſchiedenen Erregungen des
               Winters \introOben{}u den Winter ſelbſt\introOben{} nicht zu Hauſe abwartet und zu
               guter \label{T_L01424_2v}\edtext{Zeit}{\lemma{\textnormal{\emph{Zeit}}}\Cendnote{\textnormal{durch Tintenfleck ab dem zweiten Buchstaben unlesbar, von
                  Schnitzler unter der Zeile mit Bleistift wiederholt}}}\label{T_L01424_2h} und mit ruhigem Gemüth
                  {\pb}nach dem Meere, dem Süden abreiſt! –\pend
           \pstart
           Meinen Brief von neulich haben Sie wohl bekommen? Ich wünſche Ihnen ſehr, daſs eine
               günſtige Erledigung vom Militär eintrifft! –\pend
           \pstart
           Mit dem Arbeiten gehts weiter leidlich, ja gut. Mit der ſtärkſten Antheilnahme, die
               auf irgend ein\introOben{}en\introOben{} tiefere\substVorne{}\textsuperscript{s}\substDazwischen{}n\substHinten{} Grund ſchließen läßt, in den ich noch nicht ganz hinabblicken ka{\geminationn}, leſe ich im \textsc{\textcolor{blue}{Vehse}{}\ledrightnote{\textcolor{blue}{Karl Eduard Vehse}}}{ }{\pb}\textcolor{green}{Die Zeit des \textcolor{blue}{fünften \textsc{Carl}}{}\ledrightnote{\textcolor{blue}{Karl V.}}}{}\ledrightnote{\textcolor{green}{Geschichte der deutschen Höfe seit der Reformation}}. Seite für Seite hat man die Empfindung: Undramatiſirter \textsc{\textcolor{blue}{Shakespeare}{}\ledrightnote{\textcolor{blue}{William Shakespeare}}}. –\pend
           \pstart
           – Die \textcolor{blue}{Hebbel}{}\ledrightnote{\textcolor{blue}{Friedrich Hebbel}}{ }\textcolor{green}{Tagebücher}{}\ledrightnote{\textcolor{green}{Tagebücher}} habe ich nun zum zweiten Male geleſen;
               meine Bewunderung ist womöglich noch geſtiegen – aber menſchlich hab ich mich von ihm
               diesmal entfernt. Es iſt ein prachtvoller Geiſt, in beinah ununterbrochener Arbeit;
               aber \substVorne{}\textsuperscript{\textcolor{gray}{×}\-\textcolor{gray}{×}}\substDazwischen{}man\substHinten{} dürf\substVorne{}\textsuperscript{en}\substDazwischen{}te\substHinten{} das ganze auch von 1863 nach rückwärts leſen – ohne daſs
               Verſtändnis {\pb}oder Genuſs darunter litte. Was mir die
               Geſellschaft von weit geringern \introOben{}manchmal\introOben{} werther macht als
               die ſeine iſt daſs es mir erlaubt iſt einer Entwicklung zuzuſchauen, und das iſt doch
               immer das ſchönſte und packendſte, was wir erleben können. Es iſt unheimlich in einem
               Menſchen auch blättern zu können wie in einem Aphorismenbuch. We{\geminationn} mir ein Band aus einer Exiſtenz fehlt, möchte ich vor
                  {\pb}dem nächſten wie vor einem Wunder ſtehen müſſen u
               fragen: Wie biſt du dahin gekommen –?\pend
           \pstart
           Leben Sie wohl und ſchreiben Sie mir.\pend
           \pstart
           Sagen Sie auch \textcolor{blue}{Waſſermann}{}\ledrightnote{\textcolor{blue}{Jakob Wassermann}}, falls Sie ihn ſehen,
               daſs wir hier das \textcolor{green}{Los der Juden}{}\ledrightnote{\textcolor{green}{Das Los der Juden}} mit großem
               Vergnügen geleſen haben. Es iſt ein ſchönes Vorwort zu einem Buch das heute glaub ich
               keiner ſchreiben kann, weder Chriſt noch Jude. –\pend
           \pstart
           – Und wird \textcolor{blue}{Richard}{}\ledrightnote{\textcolor{blue}{Richard Beer-Hofmann}} bald {\pb}fertig mit dem \textcolor{green}{Stück}{}\ledrightnote{→\textcolor{green}{Der Graf von Charolais. Ein Trauerspiel}}? Wie gehts ihm?\pend
           \pstart
            Grüßen Sie Alle.\pend
           \pstart Herzlichſt Ihr\spacefill\mbox{A.}\pend{}\endnumbering\briefempfaengerindex{Hofmannsthal, Hugo von@\textsc{Hofmannsthal, Hugo von}!zzzSchnitzler, Arthur@\emph{von Arthur Schnitzler}!1904-08-091@{9. 8. 1904}|)be}\mylabel{h}  \normalsize

\doendnotes{C}
\bigskip
\vfill

\clearpage

\footnotesize

\lohead{\textsc{register}}

% Definiere theindex-Environment komplett neu ohne reledmac
\makeatletter
\renewenvironment{theindex}{%
  \section*{\indexname}%
  \setlength{\parindent}{0pt}%
  \setlength{\parskip}{0pt plus 0.3pt}%
  \let\item\@idxitem
}{%
  \clearpage
}
\makeatother

\IfFileExists{\jobname-pw.ind}{\input{\jobname-pw.ind}}{}

\end{document}

      