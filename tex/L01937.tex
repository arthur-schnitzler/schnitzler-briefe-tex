%% latex-korrekturansicht-vorspann.tex
%% Vorspann für die Korrekturansicht.
%% Lädt die gemeinsame Datei latex-vorspann.tex mit gesetztem Schalter.

\newif\ifkorrekturansicht
\korrekturansichttrue

\input{../tex-inputs/latex-vorspann}


               \section[Frank Wedekind an Arthur Schnitzler, 19. 6. 1910]{ Frank Wedekind an Arthur Schnitzler, 19. 6. 1910}\nopagebreak\mylabel{v}\rehead{ }\normalsize\beginnumbering\briefempfaengerindex{Schnitzler, Arthur@\textsc{Schnitzler, Arthur}!zzzWedekind, Frank@\emph{von Frank Wedekind}!1910-06-192@{19. 6. 1910}|(be} \toendnotes[C]{\smallbreak\pagebreak[2]} \Standort{CUL, Schnitzler, B 111.}
\physDesc{Brief, 1 Blatt, 3 Seiten
\newline{}Handschrift: schwarze Tinte, deutsche Kurrent
\newline{}Schnitzler: 1) mit Bleistift beschriftet: »\textsc{Wedekind}« 2) mit rotem Buntstift eine Unterstreichung}\toendnotes[C]{\smallbreak}\pstart{}{\pb}Sehr verehrter Herr
                        Doctor!\pend\pstart
           \label{K_L01937_1v}\edtext{Neulich}{\lemma{\textnormal{\emph{Neulich}}}\Cendnote{\textnormal{Am 11. 5. 1910 wurde im \textcolor{pink}{Schauspielhaus} zum ersten Mal \emph{\textcolor{green}{Komtesse Mizzi}} (gemeinsam mit \emph{\textcolor{green}{Die letzten Masken}} und \emph{\textcolor{green}{Literatur}}) gegeben.}}}\label{K_L01937_1h} hatte ich die große Freude \textcolor{green}{Conteſſe Mizzi}{}\ledrightnote{\textcolor{green}{Komtesse Mizzi oder Der Familientag}} auf der Bühne zu ſehen und bin
                    noch voll vom Genuß der Schönheit dieſes vornehmen ſcharfgeſchliffenen
                    Kuntwerks. \textcolor{green}{Conteſſe Mizzi}{}\ledrightnote{\textcolor{green}{Komtesse Mizzi oder Der Familientag}} erſcheint mir als
                    eine Meiſterſchöpfung, als der Urtypus der Komödie im beſten Sinne des Wortes.
                        {\pb}Als Kunſtwerk ſcheint mir das
                    Stück ebenſo ein Unicum zu ſein wie es mir vor \label{K_L01937_2v}\edtext{7 Jahren}{\lemma{\textnormal{\emph{7 Jahren}}}\Cendnote{\textnormal{\emph{\textcolor{green}{Lieutenant Gustl}} lag bereits
                            1902 in Buchform vor.}}}\label{K_L01937_2h}{ }\textcolor{green}{Leutnant Guſtl}{}\ledrightnote{\textcolor{green}{Lieutenant Gustl. Novelle}} erſchien. Ich kann es mir nicht
                    verſagen, Ihnen, dem ich ſchon ſo viele verſchiedenartige Genüſſe verdanke,
                    meiner hellen Freude Ausdruck zu geben.\pend
           \pstart
           {\pb}Seien Sie herzlichſt gegrüßt. An
                    unſern zufälligen Abenden ist ſehr viel von Ihnen die Rede.\pend
           \pstart
           Mit verbindlichſten Empfehlungen auch von meiner \textcolor{blue}{Frau}{}\ledrightnote{→\textcolor{blue}{Tilly Wedekind}}\pend
           \pstart
           Ihr ergebener{\\[\baselineskip]}\spacefill\mbox{FrankWedekind.}\pend
           \leftskip=0em{}\pstart
           \textcolor{pink}{München}{}\ledrightnote{\textcolor{pink}{München}}, 19. Juni
                        1910.\pend
           \endnumbering\briefempfaengerindex{Schnitzler, Arthur@\textsc{Schnitzler, Arthur}!zzzWedekind, Frank@\emph{von Frank Wedekind}!1910-06-192@{19. 6. 1910}|)be}\mylabel{h}  \normalsize

\doendnotes{C}
\bigskip
\vfill

\clearpage

\footnotesize

\lohead{\textsc{register}}

% Definiere theindex-Environment komplett neu ohne reledmac
\makeatletter
\renewenvironment{theindex}{%
  \section*{\indexname}%
  \setlength{\parindent}{0pt}%
  \setlength{\parskip}{0pt plus 0.3pt}%
  \let\item\@idxitem
}{%
  \clearpage
}
\makeatother

\IfFileExists{\jobname-pw.ind}{\input{\jobname-pw.ind}}{}

\end{document}

      