%% latex-korrekturansicht-vorspann.tex
%% Vorspann für die Korrekturansicht.
%% Lädt die gemeinsame Datei latex-vorspann.tex mit gesetztem Schalter.

\newif\ifkorrekturansicht
\korrekturansichttrue

\input{../tex-inputs/latex-vorspann}


               \section[Georg Brandes an Arthur Schnitzler, 19. 10. 1911]{ Georg Brandes an Arthur Schnitzler, 19. 10. 1911}\nopagebreak\mylabel{v}\rehead{ }\normalsize\beginnumbering\briefempfaengerindex{Schnitzler, Arthur@\textsc{Schnitzler, Arthur}!zzzBrandes, Georg@\emph{von Georg Brandes}!1911-10-191@{19. 10. 1911}|(be} \toendnotes[C]{\smallbreak\pagebreak[2]} \Standort{CUL, Schnitzler, B 17.}
\physDesc{Brief, 1 Blatt, 4 Seiten
\newline{}Handschrift: schwarze Tinte, lateinische Kurrent
\newline{}Schnitzler: mit Bleistift beschriftet: »\textsc{Brandes}« \newline{}Ordnung: mit Bleistift von unbekannter Hand nummeriert:
                                        »37« }\buchAbdrucke{\weitereDrucke{Georg Brandes, Arthur Schnitzler: \emph{Ein Briefwechsel}. Hg. Kurt Bergel. Bern: \emph{Francke} 1956, S. 102–104.} }\toendnotes[C]{\smallbreak}\pstart
           \raggedleft{}{\pb}\textcolor{pink}{\uline{Kopenhagen}}{}\ledrightnote{\textcolor{pink}{Dänemark}} (\uline{nicht}{ }\textcolor{pink}{Havnegade}{}\ledrightnote{\textcolor{pink}{Havnegade}}){\\}19 Oct 11\pend
           \pstart{}Mein verehrter Freund\pend\pstart
           Ihr \textcolor{green}{Schauspiel}{}\ledrightnote{→\textcolor{green}{Das weite Land. Tragikomödie in fünf Akten}} und Ihr Brief
                    haben mir beide tief bewegt. Der Brief, weil er so herzlich war und weil ich,
                    seit lange von allerlei Unglück und Missgeschick verfolgt, für Herzlichkeit sehr
                    empfänglich bin, das \textcolor{green}{Schauspiel}{}\ledrightnote{→\textcolor{green}{Das weite Land. Tragikomödie in fünf Akten}}, weil es mir das Werk eines Meisters scheint, vollreif.\pend
           \pstart
           Diese Menschen, die Sie dort darstellen, stehen uns vor Augen als wirkliche
                    Individualitäten, voll und rund und originell, mit Eigenschaften und
                    Eigenheiten, die ein Ensemble ausmachen. Die Nebenfiguren wie \textcolor{green}{Natter}{}\ledrightnote{→\textcolor{green}{Das weite Land. Tragikomödie in fünf Akten}}, oder die amüsant Karikierten,
                    wie \textcolor{green}{Rhon}{}\ledrightnote{→\textcolor{green}{Das weite Land. Tragikomödie in fünf Akten}} und \textcolor{green}{Serknitz}{}\ledrightnote{→\textcolor{green}{Das weite Land. Tragikomödie in fünf Akten}}, sind nicht weniger
                    unvergesslich als die tiefsinnig studierten und räthselvollen wie \textcolor{green}{Friedrich}{}\ledrightnote{→\textcolor{green}{Das weite Land. Tragikomödie in fünf Akten}}, \textcolor{green}{Genia}{}\ledrightnote{→\textcolor{green}{Das weite Land. Tragikomödie in fünf Akten}} und die eine \uline{ganze} Seele, \textcolor{green}{Erna}{}\ledrightnote{→\textcolor{green}{Das weite Land. Tragikomödie in fünf Akten}}. {\pb}Ich würde nichts darüber
                    schreiben können, das etwas hinzufügte an die Wirkung, und nichts, das irgend
                    etwas erklärte, denn alles erklärt sich von selbst.\pend
           \pstart
           Sie lieben es, die Nebentriebe und Nebenpassionen zu verfolgen, die Sprünge und
                    Seitensprünge des Gefühlslebens, alles Getheilte, das von dem Hauptstamm sich
                    ablöst, auszubreiten. Die Welt, so gesehen, ist auf eine specielle Weise
                    traurig. Meiner Gefühlart nach wäre, um das Bild zu supplieren, auch das
                    Erhebende, das ab und zu, wenn auch sehr selten, uns begegnet, ich meine: das,
                    was das Leben erträglich macht, \strikeout{auch} mit in
                    Rechenschaft zu ziehen.\hspace*{2em}Ich bin, glaub ich, im
                    Ganzen pessimistischer als Sie, aber dennoch empfind ich einige Ruhepunkte, und
                    man muss das, soll man sich nicht tödten. Man muss z. B. Jemand vertrauen
                    können; {\pb}in der hier
                    vorgeführten, sehr reichen und schillernden Welt, ist aber jedes Vertrauen
                    unmöglich; alle arbeiten sich von ihren Neigungen und Bänden los.\pend
           \pstart
           Haben Sie Dank, dass Sie sich um das mir unbekannte Frl. \textcolor{blue}{Prozor}{}\ledrightnote{\textcolor{blue}{Grete Prozor}} bemühten, und dass Sie ihr so nützlich waren.\pend
           \pstart
           Sie irren sich wenn Sie glauben, ich möchte nicht gern nach \textcolor{pink}{Wien}{}\ledrightnote{\textcolor{pink}{Wien}} kommen. Im Gegentheil \textcolor{pink}{Wien}{}\ledrightnote{\textcolor{pink}{Wien}} hat immer für mich eine grosse Anziehungskraft gehabt; ich habe
                    dort sehr angenehme Stunden verlebt, besonders – es ist lange her – in
                        1885, als ich den alten \textcolor{blue}{Gompertz}{}\ledrightnote{\textcolor{blue}{Theodor Gomperz}} kennen lernte. Später einmal, ich weiss nicht wann, es ist
                    wohl 20 Jahre her, luden \uline{Sie} sich zu mir ein,
                    und es war bei Ihnen eine \label{K_L02040_1v}\edtext{Herrengesellschaft}{\lemma{\textnormal{\emph{Herrengesellschaft}}}\Cendnote{\textnormal{Brandes dürfte
                        auf den 22. 3. 1900 anspielen,
                        wenngleich \textcolor{blue}{Hofmannsthal} im \emph{\textcolor{green}{Tagebuch}} nicht explizit genannt
                        ist.}}}\label{K_L02040_1h} spät Abends, wo viele, die später berühmt \strikeout{\textcolor{gray}{×}\-\textcolor{gray}{×}\-\textcolor{gray}{×}} geworden, zusammen waren: \textcolor{blue}{Hoffmannsthal}{}\ledrightnote{\textcolor{blue}{Hugo von Hofmannsthal}}, \textcolor{blue}{Wassermann}{}\ledrightnote{\textcolor{blue}{Jakob Wassermann}}, und
                    andere. Sonst habe ich in \textcolor{pink}{Wien}{}\ledrightnote{\textcolor{pink}{Wien}} nur bei \textcolor{blue}{Gompertz}{}\ledrightnote{\textcolor{blue}{Theodor Gomperz}} Menschen gesehen. Ich kenne {\pb}ja Niemand dort.\hspace*{2em}Aber ich bin in der Regel wie in einem
                    Schraubenstock; ich kann nicht fort, wenn ich wollte, was zu weitläufig zu
                    erklären ist. Leichter zu erklären ist, dass ich eigentlich nie Geld zu meinen
                    Reisen habe. Aus \textcolor{pink}{Deutschland}{}\ledrightnote{\textcolor{pink}{Deutschland}} bekomme ich nie
                    einen Pfennig, habe dort seit lange nicht einmal mehr einmal einen \uline{Verleger} und stehe mit keiner Zeitung in Verbindung.
                    In \textcolor{pink}{Dänemark}{}\ledrightnote{\textcolor{pink}{Dänemark}} verdiente ich durch ein Buch im
                    Jahr 10375 Kronen, aus \textcolor{pink}{England}{}\ledrightnote{\textcolor{pink}{England}} bekomme ich als
                    Royalty für ein Dutzend Bände jährlich 400 Kronen. – \uline{Ihre} Einnahmen werden sich glücklicherweise anders gestalten.\pend
           \pstart
           Ich habe das Glück gehabt, meine \textcolor{blue}{Mutter}{}\ledrightnote{→\textcolor{blue}{Emilie Brandes}} etwas länger zu behalten als es Ihnen gestattet wurde. Die
                    Mutter ist ja vielleicht das einzige unbedingt sichere, das wir zum Vertrauen
                    haben, um so unersetzlicher. Sie müssen jetzt 50 Jahre alt sein, ich bin in
                    wenigen Monaten 70, deshalb einigermassen isolirt, obwohl mein Temperament
                    dasselbe geblieben.\pend
           \pstart
           Ich drücke Ihre Hand in alter Ergebenheit{\\[\baselineskip]}\spacefill\mbox{Georg Brandes}\pend
           \leftskip=0em{}\endnumbering\briefempfaengerindex{Schnitzler, Arthur@\textsc{Schnitzler, Arthur}!zzzBrandes, Georg@\emph{von Georg Brandes}!1911-10-191@{19. 10. 1911}|)be}\mylabel{h}  \normalsize

\doendnotes{C}
\bigskip
\vfill

\clearpage

\footnotesize

\lohead{\textsc{register}}

% Definiere theindex-Environment komplett neu ohne reledmac
\makeatletter
\renewenvironment{theindex}{%
  \section*{\indexname}%
  \setlength{\parindent}{0pt}%
  \setlength{\parskip}{0pt plus 0.3pt}%
  \let\item\@idxitem
}{%
  \clearpage
}
\makeatother

\IfFileExists{\jobname-pw.ind}{\input{\jobname-pw.ind}}{}

\end{document}

      