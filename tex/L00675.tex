%% latex-korrekturansicht-vorspann.tex
%% Vorspann für die Korrekturansicht.
%% Lädt die gemeinsame Datei latex-vorspann.tex mit gesetztem Schalter.

\newif\ifkorrekturansicht
\korrekturansichttrue

\input{../tex-inputs/latex-vorspann}


               \section[Richard Beer-Hofmann an Arthur Schnitzler, 12. 5. 1897]{ Richard Beer-Hofmann an Arthur Schnitzler,
               12. 5. 1897}\nopagebreak\mylabel{v}\rehead{ }\normalsize\beginnumbering\briefempfaengerindex{Schnitzler, Arthur@\textsc{Schnitzler, Arthur}!zzzBeer-Hofmann, Richard@\emph{von Richard Beer-Hofmann}!1897-05-121@{12. 5. 1897}|(be} \toendnotes[C]{\smallbreak\pagebreak[2]} \Standort{CUL, Schnitzler, B 8.}
\physDesc{Brief, 2 Blätter, 7 Seiten
\newline{}Handschrift: Bleistift, lateinische Kurrent\newline{}Ordnung: mit Bleistift von unbekannter Hand nummeriert: »95« }\buchAbdrucke{\weitereDrucke{Arthur Schnitzler, Richard Beer-Hofmann: \emph{Briefwechsel 1891–1931}. Hg. Konstanze Fliedl. Wien, Zürich: \emph{Europaverlag} 1992, S. 103–104.} }\toendnotes[C]{\smallbreak}\pstart
           \centering{}{\pb}\textcolor{pink}{Ischl}{}\ledrightnote{\textcolor{pink}{Bad Ischl}}. 12/V 97\pend
           \pstart
           Lieber Arthur! Ich habe einen recht starken Luftröhrenkatarrh gehabt
               (war auch bei Ihrem \textcolor{blue}{Schwager}{}\ledrightnote{\textcolor{blue}{Markus Hajek}}) und bin deshalb,
               (Luftveränderung) und auch um für \textcolor{blue}{P.}{}\ledrightnote{\textcolor{blue}{Paula Beer-Hofmann}} Wohnung zu
               suchen am 7/V hieher gereist; übermorgen fahre ich wieder nach \textcolor{pink}{Wien}{}\ledrightnote{\textcolor{pink}{Wien}} zurück. Anfangs Juni ko{\geminationm}e {\pb}ich dann wieder mit \textcolor{blue}{Papa}{}\ledrightnote{→\textcolor{blue}{Alois Hofmann}}
               hieher – in unsere alte Wohnung im \textcolor{pink}{Egelmoos}{}\ledrightnote{\textcolor{pink}{Eglmoosgasse}}. \textcolor{blue}{P.}{}\ledrightnote{\textcolor{blue}{Paula Beer-Hofmann}} wohnt schon hier in einem kleinen Zi{\geminationm}er, in einem kleinen Haus und ist recht lieb und gut.
               – (Sie werden jetzt lächeln und dieselbe Zärtlichkeit bei sich suchen und finden
               – außer Sie sind ein gottverlassenes {\pb}Scheusaal)\footnote{\noindent{}die 2 a im letzten Worte sind ein orthographischer Irrthum – keine Feinheit} Über Ihr und \textcolor{blue}{Goldmann}{}\ledrightnote{\textcolor{blue}{Paul Goldmann}}s Schicksa\strikeout{a}l \strikeout{B} bei dem Brandunglück hab ich mir keine Sorgen
               gemacht. Von \textcolor{blue}{Goldmann}{}\ledrightnote{\textcolor{blue}{Paul Goldmann}} wußte ich daß er noch nicht
               in \textcolor{pink}{Paris}{}\ledrightnote{\textcolor{pink}{Paris}} war, – ich sprach am selben {\pb}Tag telefonisch mit Ihrer \textcolor{blue}{Mama}{}\ledrightnote{→\textcolor{blue}{Louise Schnitzler}}, und daß Sie nicht zu
               dergleichen Dingen gehen war mir bekannt.\pend
           \pstart
           – Wahrscheinlich sind Ihnen aber bei diesem Anlasse alte (»Ihrige«) oder auch neue
               Novellenstoffe von Hinterbliebenen eingefallen; auch {\pb}die Notwendigkeit des Testaments
               machen wird sehr deutlich. –\pend
           \pstart
           \textcolor{blue}{Paul Goldmann}{}\ledrightnote{\textcolor{blue}{Paul Goldmann}} wird – da er ja immer aus allen
               Ereignissen wie die Biene den Honig saugt – aus der Tatsache daß ich \uline{Ihnen}
                schreibe, irgendwelche Schlüße auf mein Verhältniß zu
               ihm ziehen, und erklären {\pb}»Siehst
               Du, \uline{Dir}
                schreibt er«! Dann folgt Ihr
               Beruhigungsversuch; dann sagt \textcolor{blue}{Paul}{}\ledrightnote{\textcolor{blue}{Paul Goldmann}}
                sehr großartig
               resignirt: »Laß das Kinderl – ich weiß ja– –! Ja – ja!« Sollte er aber die
               Gemeinheit der Gesinnung soweit treiben, daß er sich vor Aufregung {\pb}auf den eigenen Fuß tritt, –
               »Pardon« ruft und ein Erdbeben markirt, – dann schimpfen Sie ihn gehörig in
               meinem Namen zusa{\geminationm}en. –\pend
           \pstart
           Wann kommen Sie? –\pend
           \pstart
           Was macht \textcolor{blue}{Paul}{}\ledrightnote{\textcolor{blue}{Paul Goldmann}} im So{\geminationm}er?\pend
           \pstart
           Herzlichst{\\[\baselineskip]}\spacefill\mbox{Richard}\pend
           \leftskip=0em{}\pstart
           »\uline{Deutlicher schreiben!}«\pend
           \endnumbering\briefempfaengerindex{Schnitzler, Arthur@\textsc{Schnitzler, Arthur}!zzzBeer-Hofmann, Richard@\emph{von Richard Beer-Hofmann}!1897-05-121@{12. 5. 1897}|)be}\mylabel{h}  \normalsize

\doendnotes{C}
\bigskip
\vfill

\clearpage

\footnotesize

\lohead{\textsc{register}}

% Definiere theindex-Environment komplett neu ohne reledmac
\makeatletter
\renewenvironment{theindex}{%
  \section*{\indexname}%
  \setlength{\parindent}{0pt}%
  \setlength{\parskip}{0pt plus 0.3pt}%
  \let\item\@idxitem
}{%
  \clearpage
}
\makeatother

\IfFileExists{\jobname-pw.ind}{\input{\jobname-pw.ind}}{}

\end{document}

      