%% latex-korrekturansicht-vorspann.tex
%% Vorspann für die Korrekturansicht.
%% Lädt die gemeinsame Datei latex-vorspann.tex mit gesetztem Schalter.

\newif\ifkorrekturansicht
\korrekturansichttrue

\input{../tex-inputs/latex-vorspann}


               \section[Bertha von Suttner an Arthur und Olga Schnitzler, Dezember 1910]{ Bertha von Suttner an Arthur und Olga Schnitzler, Dezember 1910}\nopagebreak\mylabel{v}\rehead{ }\normalsize\beginnumbering\briefempfaengerindex{Schnitzler, Arthur@\textsc{Schnitzler, Arthur}!zzzSuttner, Bertha von@\emph{von Bertha von Suttner}!1910-12-011@{Dezember 1910}|(be} \toendnotes[C]{\smallbreak\pagebreak[2]} \Standort{DLA, A:Schnitzler, HS.NZ66.198.}
\physDesc{Briefkarte mit Krone in Golddruck
\newline{}Handschrift: schwarze Tinte, deutsche Kurrent
\newline{}Schnitzler: mit Bleistift die Absenderadresse vermerkt: »\textcolor{pink}{\textsc{Zedlitzg. 7}}.« }\pstart
           \raggedleft{}{\pb}Dezember 1910\pend
           \pstart{}Geehrter Herr D\textsuperscript{r} Schnitzler\pend\pstart
           Ich kann Ihnen Frl. \textcolor{blue}{\textsc{\textcolor{gray}{Taussig}}}{}\ledrightnote{\textcolor{blue}{Taussig}} – die durch lange Zeiten mir vortreffliche und beſonders \uline{intelligente} Sekretärdienſte {\pb}geleiſtet
               hat, auf das wärmſte empfehlen.\pend
           \pstart
           Es grüßt Sie eine Ihrer alten Bewunderinnen{\\[\baselineskip]}\spacefill\mbox{Bertha v. Suttner}\pend
           \leftskip=0em{}\endnumbering\briefempfaengerindex{Schnitzler, Arthur@\textsc{Schnitzler, Arthur}!zzzSuttner, Bertha von@\emph{von Bertha von Suttner}!1910-12-011@{Dezember 1910}|)be}\mylabel{h}  \normalsize

\doendnotes{C}
\bigskip
\vfill

\clearpage

\footnotesize

\lohead{\textsc{register}}

% Definiere theindex-Environment komplett neu ohne reledmac
\makeatletter
\renewenvironment{theindex}{%
  \section*{\indexname}%
  \setlength{\parindent}{0pt}%
  \setlength{\parskip}{0pt plus 0.3pt}%
  \let\item\@idxitem
}{%
  \clearpage
}
\makeatother

\IfFileExists{\jobname-pw.ind}{\input{\jobname-pw.ind}}{}

\end{document}

      