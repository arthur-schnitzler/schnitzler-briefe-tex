%% latex-korrekturansicht-vorspann.tex
%% Vorspann für die Korrekturansicht.
%% Lädt die gemeinsame Datei latex-vorspann.tex mit gesetztem Schalter.

\newif\ifkorrekturansicht
\korrekturansichttrue

\input{../tex-inputs/latex-vorspann}


               \section[Arthur Schnitzler an Richard Beer-Hofmann, {[}17. oder 18. 7. 1897?{]}]{ Arthur Schnitzler an Richard Beer-Hofmann, {[}17. oder 18. 7. 1897?{]}}\nopagebreak\mylabel{v}\rehead{ }\normalsize\beginnumbering\briefempfaengerindex{Beer-Hofmann, Richard@\textsc{Beer-Hofmann, Richard}!zzzSchnitzler, Arthur@\emph{von Arthur Schnitzler}!1897-07-172@{{[}17. oder 18. 7. 1897?{]}}|(be} \toendnotes[C]{\smallbreak\pagebreak[2]} \Standort{YCGL, MSS 31.}
\physDesc{Visitenkarte
\newline{}Handschrift: Bleistift, deutsche Kurrent\newline{}Zusatz: die Adresse quer am linken Rand der
                           bedruckten Seite }\pstart{}{\pb}\textsc{Dr Beer Hofma{\geminationn}}\pend{}\pstart{}\textsc{\textcolor{pink}{Egelmoos 22}{}\ledrightnote{\textcolor{pink}{Eglmoosgasse}}}\pend{}{\bigskip}\pstart
           \noindent{}\centering{}{\pb}\textcolor{gray}{\textbf{D\textsuperscript{r} Arthur Schnitzler}}\pend
           \pstart
           \noindent{}\raggedleft{}\textcolor{gray}{\textbf{\textcolor{pink}{Wien}{}\ledrightnote{\textcolor{pink}{Wien}}.}}\pend
           \pstart
           \noindent{}{\pb}Lieber Richard, eben Telegra{\geminationm} von \textcolor{blue}{Hugo}{}\ledrightnote{\textcolor{blue}{Hugo von Hofmannsthal}}, er ist
                     So{\geminationn}tag früh in \textcolor{pink}{Zell}{}\ledrightnote{\textcolor{pink}{Zell am See}}.\pend
           \pstart Herzlich Ihr \spacefill\mbox{A}\pend{}\endnumbering\briefempfaengerindex{Beer-Hofmann, Richard@\textsc{Beer-Hofmann, Richard}!zzzSchnitzler, Arthur@\emph{von Arthur Schnitzler}!1897-07-172@{{[}17. oder 18. 7. 1897?{]}}|)be}\mylabel{h}  \normalsize

\doendnotes{C}
\bigskip
\vfill

\clearpage

\footnotesize

\lohead{\textsc{register}}

% Definiere theindex-Environment komplett neu ohne reledmac
\makeatletter
\renewenvironment{theindex}{%
  \section*{\indexname}%
  \setlength{\parindent}{0pt}%
  \setlength{\parskip}{0pt plus 0.3pt}%
  \let\item\@idxitem
}{%
  \clearpage
}
\makeatother

\IfFileExists{\jobname-pw.ind}{\input{\jobname-pw.ind}}{}

\end{document}

      