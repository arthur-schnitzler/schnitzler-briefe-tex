%% latex-korrekturansicht-vorspann.tex
%% Vorspann für die Korrekturansicht.
%% Lädt die gemeinsame Datei latex-vorspann.tex mit gesetztem Schalter.

\newif\ifkorrekturansicht
\korrekturansichttrue

\input{../tex-inputs/latex-vorspann}


               \section[Max Burckhard an Arthur Schnitzler, 7. 6. 1908]{ Max Burckhard an Arthur Schnitzler, 7. 6. 1908}\nopagebreak\mylabel{v}\rehead{ }\normalsize\beginnumbering\briefempfaengerindex{Schnitzler, Arthur@\textsc{Schnitzler, Arthur}!zzzBurckhard, Max Eugen@\emph{von Max Eugen Burckhard}!1908-06-071@{7. 6. 1908}|(be} \toendnotes[C]{\smallbreak\pagebreak[2]} \Standort{CUL, Schnitzler, B 20.}
\physDesc{Brief, 1 Blatt, 1 Seite
\newline{}Handschrift: schwarze Tinte, deutsche Kurrent\newline{}Ordnung: mit Bleistift von unbekannter Hand nummeriert: »22« }\toendnotes[C]{\smallbreak}\pstart
           \noindent{}{\pb}\textcolor{gray}{\textbf{D\textsuperscript{r.} Max Burckhard}}\hfill \textcolor{gray}{\textbf{\textcolor{pink}{Wien, IX. Porzellangasse 48}{}\ledrightnote{\textcolor{pink}{Porzellangasse}}{ }..........}}\pend
           \pstart
           \raggedleft{}\textcolor{gray}{\textbf{\textcolor{pink}{St. Gilgen}{}\ledrightnote{\textcolor{pink}{St. Gilgen}}}}{ }7. 6. 08.\pend
           \pstart{}Lieber, ſehr verehrter Herr Doctor!\pend\pstart
           Ich ſage Ihnen herzlichſten Dank für die freundliche Zuſendung Ihres eben
                    erſchienenen \textcolor{green}{Roman}{}\ledrightnote{→\textcolor{green}{Der Weg ins Freie. Roman}}s. Gegen
                    meine Principien hatte ich die »Fortſetzungen« bereits in der \textcolor{green}{Rundſchau}{}\ledrightnote{\textcolor{green}{Die neue Rundschau}} geleſen, da mich ſchon die erſte Nu{\geminationm}er hiezu verleitet hatte: den Schluß aber hatte
                    ich noch nicht erhalten, denn die Entfernung von \textcolor{pink}{Wien}{}\ledrightnote{\textcolor{pink}{Wien}} nach \textcolor{pink}{Gilgen}{}\ledrightnote{\textcolor{pink}{St. Gilgen}} iſt lang und mein
                    Buchhändler und die Poſt ſind langſam. Mich hat ſo Vieles in dem \textcolor{green}{Buche}{}\ledrightnote{→\textcolor{green}{Der Weg ins Freie. Roman}} tief bewegt, daſs ich es nicht mit
                    ein paar Zeilen zum Ausdruck bringen könnte.\pend
           \pstart
           Ko{\geminationm}en Sie nicht heuer nach Jahrhunderten wieder nach
                        \textcolor{pink}{St Gilgen}{}\ledrightnote{\textcolor{pink}{St. Gilgen}}? Ich war leider, da ich im Herbſt
                    und nach Weihnachten in \textcolor{pink}{Wien}{}\ledrightnote{\textcolor{pink}{Wien}} war, beidemal
                    unwohl und konnte daher meinen Vorſatz, Sie aufzuſuchen nicht ausführen.\pend
           \pstart
           Herzlichſt mit Handkuſs an die verehrte gnädige \textcolor{blue}{Frau}{}\ledrightnote{→\textcolor{blue}{Olga Schnitzler}}\pend
           \pstart
           Ihr{\\[\baselineskip]}\spacefill\mbox{D\textsuperscript{r}Burckhard}\pend
           \leftskip=0em{}\endnumbering\briefempfaengerindex{Schnitzler, Arthur@\textsc{Schnitzler, Arthur}!zzzBurckhard, Max Eugen@\emph{von Max Eugen Burckhard}!1908-06-071@{7. 6. 1908}|)be}\mylabel{h}  \normalsize

\doendnotes{C}
\bigskip
\vfill

\clearpage

\footnotesize

\lohead{\textsc{register}}

% Definiere theindex-Environment komplett neu ohne reledmac
\makeatletter
\renewenvironment{theindex}{%
  \section*{\indexname}%
  \setlength{\parindent}{0pt}%
  \setlength{\parskip}{0pt plus 0.3pt}%
  \let\item\@idxitem
}{%
  \clearpage
}
\makeatother

\IfFileExists{\jobname-pw.ind}{\input{\jobname-pw.ind}}{}

\end{document}

      