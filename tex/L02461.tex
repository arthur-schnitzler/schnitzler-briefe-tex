%% latex-korrekturansicht-vorspann.tex
%% Vorspann für die Korrekturansicht.
%% Lädt die gemeinsame Datei latex-vorspann.tex mit gesetztem Schalter.

\newif\ifkorrekturansicht
\korrekturansichttrue

\input{../tex-inputs/latex-vorspann}


               \section[Arthur Schnitzler an Hugo Hofmannsthal, 26. 12. 1925]{ Arthur Schnitzler an Hugo Hofmannsthal, 26. 12. 1925}\nopagebreak\mylabel{v}\rehead{ }\normalsize\beginnumbering\briefempfaengerindex{Hofmannsthal, Hugo von@\textsc{Hofmannsthal, Hugo von}!zzzSchnitzler, Arthur@\emph{von Arthur Schnitzler}!1925-12-262@{26. 12. 1925}|(be} \toendnotes[C]{\smallbreak\pagebreak[2]} \Standort{FDH, Hs-30885,155.}
\physDesc{Brief, 1 Blatt, 2 Seiten
\newline{}Handschrift: Bleistift, lateinische Kurrent}\buchAbdrucke{\weitereDrucke{Hugo von Hofmannsthal, Arthur Schnitzler: \emph{Briefwechsel}. Hg. Therese Nickl und Heinrich Schnitzler. Frankfurt am Main: \emph{S. Fischer} 1964, S. 304.} }\toendnotes[C]{\smallbreak}\pstart
           \raggedleft{}{\pb}\textcolor{pink}{Wien}{}\ledrightnote{\textcolor{pink}{Wien}}, 26/12 925\pend
           \pstart
           mein lieber Hugo, viel Dank für den \label{K_L02461_1v}\edtext{\textcolor{green}{Briefwechsel}{}\ledrightnote{→\textcolor{green}{Briefwechsel mit Hugo von Hofmannsthal}}}{\lemma{\textnormal{\emph{Briefwechsel}}}\Cendnote{\textnormal{\textcolor{blue}{Richard Strauss}: \emph{\textcolor{green}{Briefwechsel mit Hugo von Hofmannsthal}}. Berlin,
                            Wien, Leipzig: \emph{\textcolor{brown}{Paul Zsolnay}}{ }1926.}}}\label{K_L02461_1h}. Ich find ihn ganz besonders interessant, aufschließend,
                    anregend und – nebstbei, unglaublich amüsant. Ein wahres Feiertagsvergnügen{\dots}\pend
           \pstart
           Ihr \label{K_L02461_2v}\edtext{\textcolor{green}{\textcolor{blue}{Schiller}{}\ledrightnote{\textcolor{blue}{Friedrich von Schiller}}-Artikel}{}\ledrightnote{\textcolor{green}{Schiller}}}{\lemma{\textnormal{\emph{Schiller-Artikel}}}\Cendnote{\textnormal{\textcolor{blue}{Hugo v. Hofmannstal}: \emph{\textcolor{green}{Schiller}}. In: \emph{\textcolor{green}{Neue
                                Freie Presse}}, Nr. 22013,
                                25. 12. 1925, Weihnachtsbeilage,
                            S. 29–33.}}}\label{K_L02461_2h} in d \textcolor{green}{N. Fr. Pr}{}\ledrightnote{\textcolor{green}{Neue Freie Presse}} war ganz außerordentlich. Ich glaube nicht, daſs es heute in
                    Deutschland neben Ihnen einen Schriftsteller gibt, der im »Essayistischen« (im
                    höchsten Sinn) an dieses Niveau heranreicht. In jedem Absatz, jedem Satz – spürt
                    man den Dichter, – oder vielmehr beide, \textcolor{blue}{Schiller}{}\ledrightnote{\textcolor{blue}{Friedrich von Schiller}} un\textcolor{gray}{d} Sie; – (ohne dſs Sie je
                    »poetisch« werden, was übrigens den Feuilletonisten eher passirt); – es {\pb}ist mir ein rechtes Bedürfnis, Ihnen bei dieser
                    Gelegenheit wieder einmal – ach man unterläßt es so oft –! meine liebende
                    Bewunderung auszudrücken.\pend
           \pstart
           Alles beste zum neuen Jahr{\\[\baselineskip]}Von Herzen Ihr{\\[\baselineskip]}\spacefill\mbox{Arthur}\pend
           \leftskip=0em{}\endnumbering\briefempfaengerindex{Hofmannsthal, Hugo von@\textsc{Hofmannsthal, Hugo von}!zzzSchnitzler, Arthur@\emph{von Arthur Schnitzler}!1925-12-262@{26. 12. 1925}|)be}\mylabel{h}  \normalsize

\doendnotes{C}
\bigskip
\vfill

\clearpage

\footnotesize

\lohead{\textsc{register}}

% Definiere theindex-Environment komplett neu ohne reledmac
\makeatletter
\renewenvironment{theindex}{%
  \section*{\indexname}%
  \setlength{\parindent}{0pt}%
  \setlength{\parskip}{0pt plus 0.3pt}%
  \let\item\@idxitem
}{%
  \clearpage
}
\makeatother

\IfFileExists{\jobname-pw.ind}{\input{\jobname-pw.ind}}{}

\end{document}

      