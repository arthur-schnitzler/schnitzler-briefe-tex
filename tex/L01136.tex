%% latex-korrekturansicht-vorspann.tex
%% Vorspann für die Korrekturansicht.
%% Lädt die gemeinsame Datei latex-vorspann.tex mit gesetztem Schalter.

\newif\ifkorrekturansicht
\korrekturansichttrue

\input{../tex-inputs/latex-vorspann}


               \section[Arthur Schnitzler an Richard Beer-Hofmann, 27. 6. 1901]{ Arthur Schnitzler an Richard Beer-Hofmann, 27. 6. 1901}\nopagebreak\mylabel{v}\rehead{ }\normalsize\beginnumbering\briefempfaengerindex{Beer-Hofmann, Richard@\textsc{Beer-Hofmann, Richard}!zzzSchnitzler, Arthur@\emph{von Arthur Schnitzler}!1901-06-271@{27. 6. 1901}|(be} \toendnotes[C]{\smallbreak\pagebreak[2]} \Standort{YCGL, MSS 31.}
\physDesc{Bildpostkarte
\newline{}Handschrift: Bleistift, deutsche Kurrent\newline{}Versand: 1) Stempel: »\nobreak{}\oindex{Schoenberg im Stubaital@\textbf{Schönberg im Stubaital}, \emph{Besiedelter Ort (A.BSO)}|pwk}Schönberg in Tirol, 27 6 01\nobreak{}«.  2) Stempel: »\nobreak{}\oindex{Poertschach@\textbf{Pörtschach}, \emph{https://www.geonames.org/ontologyP.PPL}|pwk}Pörtschach am See, 29 6 01\nobreak{}«. }\buchAbdrucke{\weitereDrucke{Arthur Schnitzler, Richard Beer-Hofmann: \emph{Briefwechsel 1891–1931}. Hg. Konstanze Fliedl. Wien, Zürich: \emph{Europaverlag} 1992, S. 151.} }\toendnotes[C]{\smallbreak}\pstart{}{\pb}Herrn \textsc{Dr. Richard
                     Beer-Hofmann}\pend{}\pstart{}\textsc{\textcolor{pink}{Pörtschach}{}\ledrightnote{\textcolor{pink}{Pörtschach}}}\pend{}\pstart{}\textsc{am \textcolor{pink}{Wörthersee}{}\ledrightnote{\textcolor{pink}{Wörthersee}}}\pend{}\pstart{}\textsc{\textcolor{pink}{Villa Arnstein}{}\ledrightnote{\textcolor{pink}{Villa Arnstein}}.}\pend{}{\bigskip}\pstart
           \noindent{}\centering{}\textcolor{gray}{\textbf{Gruss aus dem \textcolor{pink}{Stubaithal}{}\ledrightnote{\textcolor{pink}{Stubaital}}
                     und}}\pend
           \pstart
           \noindent{}\centering{}{\pb}\textcolor{gray}{\textbf{\textcolor{pink}{Jagerhof–Schönberg}{}\ledrightnote{\textcolor{pink}{Gasthaus Jagerhof}}}}\pend
           \pstart
           27. 6. 901.\pend
           \pstart
           \label{K_L01136_1v}\edtext{Wallfahrt}{\lemma{\textnormal{\emph{Wallfahrt}}}\Cendnote{\textnormal{In \textcolor{pink}{Schönberg} hielt sich \textcolor{blue}{Beer-Hofmann} im Sommer 1895 auf.}}}\label{K_L01136_1h}.\pend
           \pstart Ihr \spacefill\mbox{A.}\pend{}\endnumbering\briefempfaengerindex{Beer-Hofmann, Richard@\textsc{Beer-Hofmann, Richard}!zzzSchnitzler, Arthur@\emph{von Arthur Schnitzler}!1901-06-271@{27. 6. 1901}|)be}\mylabel{h}  \normalsize

\doendnotes{C}
\bigskip
\vfill

\clearpage

\footnotesize

\lohead{\textsc{register}}

% Definiere theindex-Environment komplett neu ohne reledmac
\makeatletter
\renewenvironment{theindex}{%
  \section*{\indexname}%
  \setlength{\parindent}{0pt}%
  \setlength{\parskip}{0pt plus 0.3pt}%
  \let\item\@idxitem
}{%
  \clearpage
}
\makeatother

\IfFileExists{\jobname-pw.ind}{\input{\jobname-pw.ind}}{}

\end{document}

      