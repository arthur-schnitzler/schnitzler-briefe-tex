%% latex-korrekturansicht-vorspann.tex
%% Vorspann für die Korrekturansicht.
%% Lädt die gemeinsame Datei latex-vorspann.tex mit gesetztem Schalter.

\newif\ifkorrekturansicht
\korrekturansichttrue

\input{../tex-inputs/latex-vorspann}


               \section[Arthur Schnitzler an Richard Beer-Hofmann, 11. 5. 1905]{ Arthur Schnitzler an Richard Beer-Hofmann, 11. 5. 1905}\nopagebreak\mylabel{v}\rehead{ }\normalsize\beginnumbering\briefempfaengerindex{Beer-Hofmann, Richard@\textsc{Beer-Hofmann, Richard}!zzzSchnitzler, Arthur@\emph{von Arthur Schnitzler}!1905-05-111@{11. 5. 1905}|(be} \toendnotes[C]{\smallbreak\pagebreak[2]} \Standort{YCGL, MSS 31.}
\physDesc{Brief, 1 Blatt, 2 Seiten, Umschlag
\newline{}Handschrift: Bleistift, deutsche Kurrent\newline{}Versand: 1) Stempel: »\nobreak{}\oindex{XVIII., Waehring@\textbf{XVIII., Währing}, \emph{Bezirk (A.BZK)}|pwk}Wien 1\textcolor{gray}{8}/1, 11. V. 05\nobreak{}«.  2) Stempel: »\nobreak{}\oindex{Rodaun@\textbf{Rodaun}, \emph{Teil eines besiedelten Ortes (A.BSOX)}|pwk}{\pb}Ro{[}da{]}un, 11. 5. 05, 12–2N\nobreak{}«. }\toendnotes[C]{\smallbreak}\pstart{}{\pb}\textcolor{gray}{\textbf{Dr. Arthur Schnitzler}}\pend{}\pstart{}\textcolor{gray}{\textbf{\textcolor{pink}{Wien, XVIII Spoettelgasse 7}{}\ledrightnote{\textcolor{pink}{Edmund-Weiß-Gasse}}.}}\pend{}{\bigskip}\pstart{}{\pb}\textsc{Herrn Dr Rich Beer-Hofma{\geminationn}}\pend{}\pstart{}\textcolor{pink}{Rodaun}{}\ledrightnote{\textcolor{pink}{Rodaun}}\pend{}\pstart{}\textcolor{pink}{\textsc{Liesingerstr 2}}{}\ledrightnote{\textcolor{pink}{Liesingerstraße}}\pend{}{\bigskip}\pstart
           \raggedleft{}{\pb}11/5 905\pend
           \pstart{}lieber Richard, \pend\pstart
           ich erfahre eben von den wahnwitzigen Preiſen bei \textcolor{blue}{\textsc{Reinhardt}}{}\ledrightnote{\textcolor{blue}{Max Reinhardt}}. Alſo bitte (we{\geminationn} Sie ſo gütig ſind mir zu beſtellen) nicht
               1. Reihe Orcheſter ſondern Parket vorn ſehr vorn. Ecke unbedingt. Iſt die \textcolor{green}{Beſtellung}{}\ledrightnote{→\textcolor{green}{Der Graf von Charolais. Ein Trauerspiel}} ſchon \substVorne{}\textsuperscript{verfügt}{\allowbreak}\substDazwischen{}erfolgt\substHinten{}, ſo bitte \uline{nichts}{ }{\pb}mehr zu verfügen. –\pend
           \pstart
           Herzlichſt{\\[\baselineskip]}Ihr{\\[\baselineskip]}\spacefill\mbox{A.}\pend
           \leftskip=0em{}\endnumbering\briefempfaengerindex{Beer-Hofmann, Richard@\textsc{Beer-Hofmann, Richard}!zzzSchnitzler, Arthur@\emph{von Arthur Schnitzler}!1905-05-111@{11. 5. 1905}|)be}\mylabel{h}  \normalsize

\doendnotes{C}
\bigskip
\vfill

\clearpage

\footnotesize

\lohead{\textsc{register}}

% Definiere theindex-Environment komplett neu ohne reledmac
\makeatletter
\renewenvironment{theindex}{%
  \section*{\indexname}%
  \setlength{\parindent}{0pt}%
  \setlength{\parskip}{0pt plus 0.3pt}%
  \let\item\@idxitem
}{%
  \clearpage
}
\makeatother

\IfFileExists{\jobname-pw.ind}{\input{\jobname-pw.ind}}{}

\end{document}

      