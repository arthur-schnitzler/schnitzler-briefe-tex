%% latex-korrekturansicht-vorspann.tex
%% Vorspann für die Korrekturansicht.
%% Lädt die gemeinsame Datei latex-vorspann.tex mit gesetztem Schalter.

\newif\ifkorrekturansicht
\korrekturansichttrue

\input{../tex-inputs/latex-vorspann}


               \section[Arthur Schnitzler an Hugo von Hofmannsthal, 2. 8. 1893]{ Arthur Schnitzler an Hugo von Hofmannsthal, 2. 8. 1893}\nopagebreak\mylabel{v}\rehead{ }\normalsize\beginnumbering\briefempfaengerindex{Hofmannsthal, Hugo von@\textsc{Hofmannsthal, Hugo von}!zzzSchnitzler, Arthur@\emph{von Arthur Schnitzler}!1893-08-021@{2. 8. 1893}|(be} \toendnotes[C]{\smallbreak\pagebreak[2]} \Standort{FDH, Hs-30885,37.}
\physDesc{Brief, 2 Blätter (Briefpapier mit Trauerrand), 6 Seiten
\newline{}Handschrift: schwarze Tinte, deutsche Kurrent\newline{}Ordnung: 1) mit rotem Buntstift das erste
                                    Blatt nummeriert: »IX« 2) mit Bleistift datiert von Schnitzler das zweite Blatt mutmaßlich bei der Durchsicht der Briefe 1929  »2. 8. 92«}\buchAbdrucke{\weitereDrucke{Hugo von Hofmannsthal, Arthur Schnitzler: \emph{Briefwechsel}. Hg. Therese Nickl und Heinrich Schnitzler. Frankfurt am Main: \emph{S. Fischer} 1964, S. 42–43.} }\toendnotes[C]{\smallbreak}\pstart
           \raggedleft{}{\pb}\textcolor{pink}{Wien}{}\ledrightnote{\textcolor{pink}{Wien}}, \uline{2. 8. 93}\pend
           \pstart{}Mein lieber Hugo,\pend\pstart
           ich las Ihren Brief an \textcolor{blue}{\textsc{Salten}}{}\ledrightnote{\textcolor{blue}{Felix Salten}}. Daſs Sie nicht in \textcolor{pink}{München}{}\ledrightnote{\textcolor{pink}{München}}, wußt’ ich, da ich \textcolor{blue}{\textsc{Bahr}}{}\ledrightnote{\textcolor{blue}{Hermann Bahr}}{ }ſprach. Sie wollen im September hin? Nicht unmöglich, daſs ich mich
                    anſchließe; de{\geminationn} ich habe zur Waffenübung keine
                    Einberufung beko{\geminationm}en, u dürfte auch vorausſichtlich
                    keine mehr erhalten.\pend
           \pstart
           Vorläufig bleibe ich in \textcolor{pink}{Wien}{}\ledrightnote{\textcolor{pink}{Wien}}; Mitte
                        Auguſt fahre ich vielleicht mit \textcolor{blue}{Mama}{}\ledrightnote{→\textcolor{blue}{Louise Schnitzler}} weg, {\pb}mache auch event. eine \textsc{Bicycle}tour mit \textcolor{blue}{\textsc{Salten}}{}\ledrightnote{\textcolor{blue}{Felix Salten}}. Sie müſſen \textsc{Bic.} fahren lernen; ebenſo wie
                        \textcolor{blue}{Richard}{}\ledrightnote{\textcolor{blue}{Richard Beer-Hofmann}}; es iſt wirklich ein großes
                    Vergnügen. –\pend
           \pstart
           \textcolor{pink}{Wien}{}\ledrightnote{\textcolor{pink}{Wien}} bietet mir jetzt einiges zu thun; eine
                    kleine \textcolor{blue}{Couſine}{}\ledrightnote{→\textcolor{blue}{Adele von Suppé}} von mir iſt ſchwer krank;
                    die beſuch’ ich 1, 2, 3 mal im Tag; da{\geminationn} ab u zu
                    irgend was andres ärztliches, ſo daſs die Zeit zerſplittert iſt.
                        Aben\textcolor{gray}{d}s zuweilen auf dem Kahlenberg, wo \textcolor{blue}{Mama}{}\ledrightnote{→\textcolor{blue}{Louise Schnitzler}} u \textcolor{blue}{Schweſter}{}\ledrightnote{→\textcolor{blue}{Gisela Hajek}}
                    wohnen oder mit dem \textsc{Bic.} da oder dorthin.\pend
           \pstart
           {\pb}– Die »luſtige« \textcolor{green}{Novelle}{}\ledrightnote{→\textcolor{green}{Die kleine Komödie}} hab ich bis auf wenige Zeilen beendet, die ich erſt ſchreiben
                    kann, wenn ich Luſt beko{\geminationm}e, das ganze Zeug wieder
                    durchzuleſen. Was ich zunächſt ſchreiben werde, iſt unklar – am liebſten eins
                    meiner im Umriſs fertigen 3aktigen Stücke; aber ich ſtehe der dramatiſchen Kunſt
                    unglaublich muthlos gegenüber; ja ich hatte in der letzten Zeit oft die
                    Empfindung, daſs ich überhaupt nie {\pb}ein gutes
                    Stück werde ſchreiben können. Geſtalten u Scenen, einzelne, wären da; aber mir
                    iſt, als hätt’ ich jedes ſtrategiſche Talent verloren. Vielleicht hatt’ ichs
                    auch nie – und hab nur aus meinen kleinen Schmerzen die großen \substVorne{}\textsuperscript{S}\substDazwischen{}D\substHinten{}reiakter machen können; und ſeit meinen großen Schmerzen \strikeout{hab} werden mir nur die kleinen Novellettchen
                    gelingen. Wie leicht, wie mühelos hab ich vor – zehn, zwölf Jahren
                    geſchrieben, – {\pb}es kam zwar nie was gutes heraus;
                    aber ich war damals vielleicht ein echterer »Poet« als heut. Denn heut nagen an
                    meiner Poeſie viele Würmer, z. B. das Leben. –\pend
           \pstart
           – Wollen Sie mir nicht Ihre Pläne für den Reſt des So{\geminationm}ers mittheilen. Es iſt nicht unmöglich, daſs wir uns begegnen können.
                    Jedenfalls ſchreiben Sie mir einige Zeilen – oder Seiten, was mir lieber wäre.
                    Beleuchten {\pb}Sie mit einem »Flähmchen« die ganze
                    Umgebung!\pend
           \pstart
           Herzlich der Ihre{\\[\baselineskip]}\spacefill\mbox{Arthur}\pend
           \leftskip=0em{}\endnumbering\briefempfaengerindex{Hofmannsthal, Hugo von@\textsc{Hofmannsthal, Hugo von}!zzzSchnitzler, Arthur@\emph{von Arthur Schnitzler}!1893-08-021@{2. 8. 1893}|)be}\mylabel{h}  \normalsize

\doendnotes{C}
\bigskip
\vfill

\clearpage

\footnotesize

\lohead{\textsc{register}}

% Definiere theindex-Environment komplett neu ohne reledmac
\makeatletter
\renewenvironment{theindex}{%
  \section*{\indexname}%
  \setlength{\parindent}{0pt}%
  \setlength{\parskip}{0pt plus 0.3pt}%
  \let\item\@idxitem
}{%
  \clearpage
}
\makeatother

\IfFileExists{\jobname-pw.ind}{\input{\jobname-pw.ind}}{}

\end{document}

      