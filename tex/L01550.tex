%% latex-korrekturansicht-vorspann.tex
%% Vorspann für die Korrekturansicht.
%% Lädt die gemeinsame Datei latex-vorspann.tex mit gesetztem Schalter.

\newif\ifkorrekturansicht
\korrekturansichttrue

\input{../tex-inputs/latex-vorspann}


               \section[Hermann Bahr an Arthur Schnitzler, 20. 9. 1905]{ Hermann Bahr an Arthur Schnitzler, 20. 9. 1905}\nopagebreak\mylabel{v}\rehead{ }\normalsize\beginnumbering\briefempfaengerindex{Schnitzler, Arthur@\textsc{Schnitzler, Arthur}!zzzBahr, Hermann@\emph{von Hermann Bahr}!1905-09-201@{20. 9. 1905}|(be} \toendnotes[C]{\smallbreak\pagebreak[2]} \Standort{CUL, Schnitzler, B 5b.}
\physDesc{Kartenbrief
\newline{}Handschrift: schwarze Tinte, deutsche Kurrent\newline{}Versand: 1) Stempel: »\nobreak{}\oindex{XIII., Hietzing@\textbf{XIII., Hietzing}, \emph{Bezirk (A.BZK)}|pwk}Wien 13/5, 20. IX. 05\nobreak{}«.  2) Stempel: »\nobreak{}Bestellt, \oindex{XVIII., Waehring@\textbf{XVIII., Währing}, \emph{Bezirk (A.BZK)}|pwk}18/1 Wien, 20 IX 05\nobreak{}«. 
\newline{}Schnitzler: mit Bleistift die Jahreszahl ergänzt: »905« \newline{}Ordnung: mit Bleistift von unbekannter Hand nummeriert: »133« }\buchAbdrucke{\weitereDrucke{Hermann Bahr, Arthur Schnitzler: \emph{Briefwechsel, Aufzeichnungen, Dokumente (1891–1931)}. Hg. Kurt Ifkovits und Martin Anton Müller. Göttingen: \emph{Wallstein} 2018, S. 354.} }\toendnotes[C]{\smallbreak}\pstart{}{\pb}Herrn \textsc{Dr Arthur
                     Schnitzler}\pend{}\pstart{}\textcolor{pink}{\textsc{Wien XIX}}{}\ledrightnote{\textcolor{pink}{XIX., Döbling}}\pend{}\pstart{}\textcolor{pink}{Spöttelgaſſe 7}{}\ledrightnote{\textcolor{pink}{Edmund-Weiß-Gasse}}\pend{}{\bigskip}\pstart
           \raggedleft{}{\pb}20. 9.\pend
           \pstart{}Lieber Arthur!\pend\pstart
           Ich hab nun auch das \textcolor{green}{Zwischenſpiel}{}\ledrightnote{\textcolor{green}{Zwischenspiel. Komödie in drei Akten}} geleſen, mit
               einem ſehr großen artiſtiſchen Vergnügen. Es iſt eine reizende Comödie und ich finde
               es wunderbar, wie Du in die Form des alten \textcolor{pink}{Burgtheater}{}\ledrightnote{\textcolor{pink}{Burgtheater}}ſtücks die feinſte \textsc{Psychologi}e und
               unſere neueſten Probleme gebracht haſt. Mich ſtört nur manchmal der (gewiß
               beabſichtigte) Cafehauston zwiſchen den beiden Freunden, eine Art von \textsc{philosoph}isch \textcolor{pink}{wieneriſch}{}\ledrightnote{\textcolor{pink}{Wien}}
               jüdiſcher Schnoddrigkeit, die in früheren Jahren mir vielleicht noch geläufiger als
               Dir war, aber ſeien wir froh, daß es vorbei iſt! Mehr noch ſtört mich Dein \textcolor{green}{Fürſt}{}\ledrightnote{→\textcolor{green}{Zwischenspiel. Komödie in drei Akten}}. Warum mußt Du einen ſich in
               einer heiklen Situation ſehr nett benehmenden Menſchen in eine Kaſte verſetzen, in
               welcher Roheit die Regel, ſittlicher Takt unbekannt ist? Und wie unangenehm wird
               einem die Frau, die ſich von ſo einem hofieren läßt! Aber dies alles mündlich. Könnte
               ich nicht nächſte Woche einmal Vormittag auf ein paar Stunden zu Dir kommen? An
               Abenden macht ſichs zu ſchwer. Grüß Deine \textcolor{blue}{Frau}{}\ledrightnote{→\textcolor{blue}{Olga Schnitzler}} herzlichſt! \spacefill\mbox{H.}\pend
           \endnumbering\briefempfaengerindex{Schnitzler, Arthur@\textsc{Schnitzler, Arthur}!zzzBahr, Hermann@\emph{von Hermann Bahr}!1905-09-201@{20. 9. 1905}|)be}\mylabel{h}  \normalsize

\doendnotes{C}
\bigskip
\vfill

\clearpage

\footnotesize

\lohead{\textsc{register}}

% Definiere theindex-Environment komplett neu ohne reledmac
\makeatletter
\renewenvironment{theindex}{%
  \section*{\indexname}%
  \setlength{\parindent}{0pt}%
  \setlength{\parskip}{0pt plus 0.3pt}%
  \let\item\@idxitem
}{%
  \clearpage
}
\makeatother

\IfFileExists{\jobname-pw.ind}{\input{\jobname-pw.ind}}{}

\end{document}

      