%% latex-korrekturansicht-vorspann.tex
%% Vorspann für die Korrekturansicht.
%% Lädt die gemeinsame Datei latex-vorspann.tex mit gesetztem Schalter.

\newif\ifkorrekturansicht
\korrekturansichttrue

\input{../tex-inputs/latex-vorspann}


               \section[Georg Brandes an Arthur Schnitzler, 23. 8. 1914]{ Georg Brandes an Arthur Schnitzler, 23. 8. 1914}\nopagebreak\mylabel{v}\rehead{ }\normalsize\beginnumbering\briefempfaengerindex{Schnitzler, Arthur@\textsc{Schnitzler, Arthur}!zzzBrandes, Georg@\emph{von Georg Brandes}!1914-08-231@{23. 8. 1914}|(be} \toendnotes[C]{\smallbreak\pagebreak[2]} \Standort{CUL, Schnitzler, B 17.}
\physDesc{Brief, 1 Blatt, 4 Seiten
\newline{}Handschrift: schwarze Tinte, lateinische Kurrent
\newline{}Schnitzler: mit Bleistift unterhalb des Datums wohl der Tag der Zustellung
            ergänzt: »am 10. 9. 14« \newline{}Ordnung: mit Bleistift von unbekannter Hand nummeriert:
                                        »=42?« }\buchAbdrucke{\weitereDrucke{Georg Brandes, Arthur Schnitzler: \emph{Ein Briefwechsel}. Hg. Kurt Bergel. Bern: \emph{Francke} 1956, S. 109–110.} }\toendnotes[C]{\smallbreak}\pstart
           \raggedleft{}{\pb}\textcolor{pink}{Kopenhagen}{}\ledrightnote{\textcolor{pink}{Kopenhagen}}{\\}23 August 14\pend
           \pstart{}Verehrter und lieber Freund\pend\pstart
           Erst jetzt erhalte ich Ihren \textcolor{pink}{Schweiz}{}\ledrightnote{\textcolor{pink}{Schweiz}}erbrief
                    vom 3 August. Er war 20 Tage unterwegs.\pend
           \pstart
           Ich brauche kaum zu sagen, wie gerne ich etwas für Sie thun möchte. Sie wissen,
                    wie lieb ich Sie habe und wie sehr ich Sie schätze.\pend
           \pstart
           Leider bin ich nicht der rechte Mann. Ich bin in der \textcolor{brown}{schwedischen Akademie}{}\ledrightnote{\textcolor{brown}{Kungliga Vetenskapsakademien}} ganz unbeliebt.\pend
           \pstart
           \uline{Erstens}: Ich glaube nicht, dass
                    der \textcolor{blue}{Schwede}{}\ledrightnote{→\textcolor{blue}{?? [Schwede, mit dem Arthur Schnitzler über den Nobelpreis spricht]}} der Ihnen von
                        \textcolor{pink}{\uline{Oesterreich}}{}\ledrightnote{\textcolor{pink}{Österreich}} sprach, wirklich etwas \uline{wusste}. Jedes Jahr
                    werden völlig unrichtige Gerüchte in Umlauf gesetzt. Die Eingeweihten \uline{dürfen} nichts sagen. Der Preis wird 1914
                    gar nicht vertheilt, erst Frühling 1915. Man hat November
                    abgeschafft, Juni eingeführt.\pend
           \pstart
           {\pb}\uline{Zweitens}. Man fragt nicht speciell im \textcolor{pink}{Ministerium}{}\ledrightnote{→\textcolor{pink}{Ministerium für Unterricht}} oder in der \textcolor{brown}{Akademie}{}\ledrightnote{\textcolor{brown}{Österreichische Akademie der Wissenschaften}}. Jedes Jahr haben alle Mitglieder
                    einer \uline{Universität} und alle Mitglieder der \uline{Akademien} des Landes eine Stimme. So haben hier
                    Universitätsprofessoren und Akademiemitglieder jeder eine Stimme.\pend
           \pstart
           \uline{Ich} habe keine. Denn obwohl Ehrendoctor an \textcolor{pink}{schottischen}{}\ledrightnote{\textcolor{pink}{Schottland}} Universitäten und Ehrenmitglied
                    der \textcolor{brown}{amerikanischen Akademie der Wissenschaften und
                        Künste}{}\ledrightnote{\textcolor{brown}{American Academy of Arts and Sciences}}, der \textcolor{brown}{italiänischen}{}\ledrightnote{→\textcolor{brown}{Società Italiana delle Scienze detta dei XL}}, der \textcolor{brown}{norwegischen}{}\ledrightnote{→\textcolor{brown}{Det Kongelige Norske Videnskabers Selskab}}, der \textcolor{brown}{Royal Society}{}\ledrightnote{\textcolor{brown}{Royal Society}} usw.
                    bin ich nicht einmal ordinäres Mitglied der \textcolor{brown}{\uline{dänischen} Akademie}{}\ledrightnote{\textcolor{brown}{Kongelige Danske Videnskabernes Selskab}}, noch angestellt an
                    der \textcolor{brown}{\uline{dänischen} Universität}{}\ledrightnote{\textcolor{brown}{Københavns Universitet}}.\pend
           \pstart
           Bin also \uline{nie} gefragt worden.\pend
           \pstart
           \uline{Drittens}. Schon vor zehn Jahren schlugen viele
                    fremde Schriftsteller (u. a. \textcolor{blue}{Anatole France}{}\ledrightnote{\textcolor{blue}{Anatole France}})
                    mich zum \textcolor{brown}{Nobelpreis}{}\ledrightnote{\textcolor{brown}{Nobelpreis}} vor; schon vor 9 Jahren
                    schlug {\pb}die \textcolor{brown}{dänische Akademie der Wissenschaften}{}\ledrightnote{\textcolor{brown}{Kongelige Danske Videnskabernes Selskab}} mich einstimmig zum \textcolor{brown}{Nobelpreis}{}\ledrightnote{\textcolor{brown}{Nobelpreis}} vor und hat nie später einen anderen
                    Vorschlag machen wollen. Die \textcolor{pink}{Schweden}{}\ledrightnote{\textcolor{pink}{Schweden}} aber,
                    die mich hassen, weil ich einen \textcolor{pink}{russischen}{}\ledrightnote{\textcolor{pink}{Russland}}{ }\textcolor{blue}{Flüchtling}{}\ledrightnote{→\textcolor{blue}{?? [Russischer Flüchtling in Stockholm]}}, der in \textcolor{pink}{Stockholm}{}\ledrightnote{\textcolor{pink}{Stockholm}} gefesselt war, gegen Auslieferung
                    schützte, haben erklärt, dass von mir \uline{nie} die
                    Rede sein konnte. So unpopulär bin ich dort. Sie sehen also, dass ich ganz
                    ausser Lage bin, jemand offiziell zu empfehlen.\pend
           \pstart
           \uline{Viertens}. Ich kenne indessen privat einige
                    einflussreiche Mitglieder der \textcolor{brown}{Akademie}{}\ledrightnote{→\textcolor{brown}{Kungliga Vetenskapsakademien}} und ich werde Ihnen schreiben.\pend
           \pstart
           Nur ist dies nicht der Moment. Kein Mensch in \textcolor{pink}{Schweden}{}\ledrightnote{\textcolor{pink}{Schweden}} denkt an anderes als an den Krieg; das ganze Land ist zur
                    Vertheidigung gegen \textcolor{pink}{Rusland}{}\ledrightnote{\textcolor{pink}{Russland}} gerüstet.\pend
           \pstart
           {\pb}Ich lernte im vergangenen
                    Sommer einigermassen englisch reden, hielt im
                        November–December mit viel Erfolg Vorlesungen in
                    allen Städten \textcolor{pink}{Englands}{}\ledrightnote{\textcolor{pink}{England}} und \textcolor{pink}{Schottlands}{}\ledrightnote{\textcolor{pink}{Schottland}}.\hspace*{2em}Mai und Juni redete ich in \textcolor{pink}{Nordamerika}{}\ledrightnote{\textcolor{pink}{Amerika}}, in \textcolor{pink}{New Haven}{}\ledrightnote{\textcolor{pink}{New Haven}}, \textcolor{pink}{Chicago}{}\ledrightnote{\textcolor{pink}{Chicago}}, \textcolor{pink}{Minneapolis}{}\ledrightnote{\textcolor{pink}{Minneapolis}} und \textcolor{pink}{New York}{}\ledrightnote{\textcolor{pink}{New York City}}. An meinem
                    letzten Abend in \textcolor{pink}{New York}{}\ledrightnote{\textcolor{pink}{New York City}} im
                        Juni (93 {\%} Fahrenheit) hatte ich das
                        \textcolor{pink}{Comedy Theatre}{}\ledrightnote{\textcolor{pink}{Comedy Theatre}} so voll dass über tausend
                    Personen mit unverrichteteter Sache weggehen müssten.\pend
           \pstart
           Und nun haben wir den schrecklichen Weltkrieg. Ich möchte Untergang für \textcolor{pink}{Rusland}{}\ledrightnote{\textcolor{pink}{Russland}}, Rettung für \textcolor{pink}{Frankreich}{}\ledrightnote{\textcolor{pink}{Frankreich}}. Aber wer fragt nach unsern Wünschen! Meine \textcolor{blue}{Tochter}{}\ledrightnote{→\textcolor{blue}{Edith Philipp}} hat einen jungen
                    deutschen \textcolor{blue}{Artillerieofficier}{}\ledrightnote{→\textcolor{blue}{Reinhold Philipp}} von 32 Jahren zum Gatten. Sie ist hier mit einem
                    kl. \textcolor{blue}{Mädchen}{}\ledrightnote{→\textcolor{blue}{Gerda Philipp}} von 6 Jahren
                    und einem kl. \textcolor{blue}{Jungen}{}\ledrightnote{→\textcolor{blue}{Georg Philipp}} von
                    2 Jahren in grosser Angst für ihren \textcolor{blue}{Mann}{}\ledrightnote{→\textcolor{blue}{Reinhold Philipp}}, den sie leidenschaftlich liebt.\pend
           \pstart
           Mein ehrerbietiger Gruss an Ihre liebe Frau \textcolor{blue}{Gemahlin}{}\ledrightnote{→\textcolor{blue}{Olga Schnitzler}}.\hspace*{2em}Ich bin
                    Ihr treuer Freund{\\[\baselineskip]}\spacefill\mbox{Georg Brandes}\pend
           \leftskip=0em{}\endnumbering\briefempfaengerindex{Schnitzler, Arthur@\textsc{Schnitzler, Arthur}!zzzBrandes, Georg@\emph{von Georg Brandes}!1914-08-231@{23. 8. 1914}|)be}\mylabel{h}  \normalsize

\doendnotes{C}
\bigskip
\vfill

\clearpage

\footnotesize

\lohead{\textsc{register}}

% Definiere theindex-Environment komplett neu ohne reledmac
\makeatletter
\renewenvironment{theindex}{%
  \section*{\indexname}%
  \setlength{\parindent}{0pt}%
  \setlength{\parskip}{0pt plus 0.3pt}%
  \let\item\@idxitem
}{%
  \clearpage
}
\makeatother

\IfFileExists{\jobname-pw.ind}{\input{\jobname-pw.ind}}{}

\end{document}

      