%% latex-korrekturansicht-vorspann.tex
%% Vorspann für die Korrekturansicht.
%% Lädt die gemeinsame Datei latex-vorspann.tex mit gesetztem Schalter.

\newif\ifkorrekturansicht
\korrekturansichttrue

\input{../tex-inputs/latex-vorspann}


               \section[Hugo von Hofmannsthal an Arthur Schnitzler, 24. 5. 1903]{ Hugo von Hofmannsthal an Arthur Schnitzler, 24. 5. 1903}\nopagebreak\mylabel{v}\rehead{ }\normalsize\beginnumbering\briefempfaengerindex{Schnitzler, Arthur@\textsc{Schnitzler, Arthur}!zzzHofmannsthal, Hugo von@\emph{von Hugo von Hofmannsthal}!1903-05-241@{24. 5. 1903}|(be} \toendnotes[C]{\smallbreak\pagebreak[2]} \Standort{CUL, Schnitzler, B 43.}
\physDesc{Postkarte
\newline{}Handschrift: schwarze Tinte, deutsche Kurrent\newline{}Versand: 1) Stempel: »\nobreak{}\oindex{Rodaun@\textbf{Rodaun}, \emph{Teil eines besiedelten Ortes (A.BSOX)}|pwk}Rodaun, 25. 5. 03, 9|V\nobreak{}«.  2) Stempel: »\nobreak{}\oindex{IX., Alsergrund@\textbf{IX., Alsergrund}, \emph{Bezirk (A.BZK)}|pwk}Wien 9/3, 25. 5. 03, 5.N, Bestellt\nobreak{}«. 
\newline{}Schnitzler: mit Bleistift datiert: »25. 5. 903.« \newline{}Ordnung: 1) mit Bleistift von unbekannter Hand nummeriert: »\strikeout{213}« 2) mit Bleistift von unbekannter Hand nummeriert:
                                    »196«}\buchAbdrucke{\weitereDrucke{Hugo von Hofmannsthal, Arthur Schnitzler: \emph{Briefwechsel}. Hg. Therese Nickl und Heinrich Schnitzler. Frankfurt am Main: \emph{S. Fischer} 1964, S. 169.} }\toendnotes[C]{\smallbreak}\pstart{}{\pb}\textsc{Herrn D\textsuperscript{r} Arthur Schnitzler}\pend{}\pstart{}\textcolor{pink}{\textsc{Wien}}{}\ledrightnote{\textcolor{pink}{Wien}}\pend{}\pstart{}\textcolor{pink}{\textsc{IX. Franckgasse 1}.}{}\ledrightnote{\textcolor{pink}{Frankgasse}}\pend{}{\bigskip}\pstart
           \noindent{}{\pb}lieber Arthur, ich
               ſtelle dem lieben \textcolor{blue}{Weſen}{}\ledrightnote{→\textcolor{blue}{Marie Luggin}} alles
               beliebige von mir zur Verfügung. Sie ſoll nur ſeinerzeit an mich ſchreiben, was ſie
               haben will.\pend
           \pstart
           Glückliche Reise! \pend
           \pstart
           Von Herzen{\\[\baselineskip]}\spacefill\mbox{Hugo}\pend
           \leftskip=0em{}\pstart
           \noindent{}Sonntag.\pend
           \pstart
           \label{T_L01293_1v}\edtext{\textsc{Bitte vielmals um ein Exemplar »\textcolor{green}{Reigen}{}\ledrightnote{\textcolor{green}{Reigen. Zehn Dialoge}}«}{\\}und der \textcolor{blue}{Richard}{}\ledrightnote{\textcolor{blue}{Richard Beer-Hofmann}} auch.}{\lemma{\textnormal{\emph{Bitte … auch.}}}\Cendnote{\textnormal{quer am rechten Rand}}}\label{T_L01293_1h}\pend
           \endnumbering\briefempfaengerindex{Schnitzler, Arthur@\textsc{Schnitzler, Arthur}!zzzHofmannsthal, Hugo von@\emph{von Hugo von Hofmannsthal}!1903-05-241@{24. 5. 1903}|)be}\mylabel{h}  \normalsize

\doendnotes{C}
\bigskip
\vfill

\clearpage

\footnotesize

\lohead{\textsc{register}}

% Definiere theindex-Environment komplett neu ohne reledmac
\makeatletter
\renewenvironment{theindex}{%
  \section*{\indexname}%
  \setlength{\parindent}{0pt}%
  \setlength{\parskip}{0pt plus 0.3pt}%
  \let\item\@idxitem
}{%
  \clearpage
}
\makeatother

\IfFileExists{\jobname-pw.ind}{\input{\jobname-pw.ind}}{}

\end{document}

      