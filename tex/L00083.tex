%% latex-korrekturansicht-vorspann.tex
%% Vorspann für die Korrekturansicht.
%% Lädt die gemeinsame Datei latex-vorspann.tex mit gesetztem Schalter.

\newif\ifkorrekturansicht
\korrekturansichttrue

\input{../tex-inputs/latex-vorspann}


               \section[Hugo von Hofmannsthal an Arthur Schnitzler, {[}17. 3. 1892{]}]{ Hugo von Hofmannsthal an Arthur Schnitzler, {[}17. 3. 1892{]}}\nopagebreak\mylabel{v}\rehead{ }\normalsize\beginnumbering\briefempfaengerindex{Schnitzler, Arthur@\textsc{Schnitzler, Arthur}!zzzHofmannsthal, Hugo von@\emph{von Hugo von Hofmannsthal}!1892-03-171@{{[}17. 3. 1892{]}}|(be} \toendnotes[C]{\smallbreak\pagebreak[2]} \Standort{CUL, Schnitzler, B 43.}
\physDesc{Briefkarte mit aufgeprägtem Wappen
\newline{}Handschrift: schwarze Tinte, deutsche Kurrent
\newline{}Schnitzler: mit Bleistift das Datum ergänzt: »Mitte März 92« und nummeriert: »19« }\buchAbdrucke{\weitereDrucke{Hugo von Hofmannsthal, Arthur Schnitzler: \emph{Briefwechsel}. Hg. Therese Nickl und Heinrich Schnitzler. Frankfurt am Main: \emph{S. Fischer} 1964, S. 17.} }\toendnotes[C]{\smallbreak}\pstart
           \raggedleft{}{\pb}Donnerstag.\pend
           \pstart
           Thatſachen: 1.) Frl. \textcolor{blue}{Herzfeld}{}\ledrightnote{\textcolor{blue}{Marie Herzfeld}}{ }ſagt mir, daſs die \textcolor{brown}{\textsc{Revue}}{}\ledrightnote{→\textcolor{brown}{Allgemeine Theater-Revue für Bühne und Welt}} von \textcolor{blue}{Fried}{}\ledrightnote{\textcolor{blue}{Alfred Hermann Fried}} in jeder Beziehung ernſt zu
                    nehmen iſt. 2.) Wegen \textcolor{blue}{Schwarzkopf}{}\ledrightnote{\textcolor{blue}{Gustav Schwarzkopf}}s
                    Empfehlung an \textcolor{brown}{Bonz}{}\ledrightnote{\textcolor{brown}{Adolf Bonz {\kaufmannsund} Comp.}} müſſen wir noch
                    ſprechen.\pend
           \pstart
           3.) Dem \textcolor{blue}{Bératon}{}\ledrightnote{\textcolor{blue}{Ferry Bératon}} werde ich ſo bald als möglich
                    10 fl ſchicken.\pend
           \pstart
           4.) Wäre es nicht möglich, daſs ich Sonntag um \damage{\textcolor{gray}{4}} zu Ihnen komme, daſs auch \textcolor{blue}{Salten}{}\ledrightnote{\textcolor{blue}{Felix Salten}}
                    beſtimmt kommt und daſs ich Euch \textcolor{green}{etwas}{}\ledrightnote{→\textcolor{green}{Der Tod des Tizian}} vorle\substVorne{}\textsuperscript{ſen}\substDazwischen{}ſe\substHinten{}, was ich zum Druck verſprochen habe, aber nicht gern ohne Euch
                    fortſchicken möchte?, wenn nicht Sonntag, ſo machen Sie einen anderen
                    Vorſchlag.\pend
           \pstart
           Herzlichſt{\\[\baselineskip]}\spacefill\mbox{Loris.}\pend
           \leftskip=0em{}\pstart
           \noindent{}Beiliegend, danke, \textcolor{blue}{Nietzſche}{}\ledrightnote{\textcolor{blue}{Friedrich Nietzsche}}.\pend
           \endnumbering\briefempfaengerindex{Schnitzler, Arthur@\textsc{Schnitzler, Arthur}!zzzHofmannsthal, Hugo von@\emph{von Hugo von Hofmannsthal}!1892-03-171@{{[}17. 3. 1892{]}}|)be}\mylabel{h}  \normalsize

\doendnotes{C}
\bigskip
\vfill

\clearpage

\footnotesize

\lohead{\textsc{register}}

% Definiere theindex-Environment komplett neu ohne reledmac
\makeatletter
\renewenvironment{theindex}{%
  \section*{\indexname}%
  \setlength{\parindent}{0pt}%
  \setlength{\parskip}{0pt plus 0.3pt}%
  \let\item\@idxitem
}{%
  \clearpage
}
\makeatother

\IfFileExists{\jobname-pw.ind}{\input{\jobname-pw.ind}}{}

\end{document}

      