%% latex-korrekturansicht-vorspann.tex
%% Vorspann für die Korrekturansicht.
%% Lädt die gemeinsame Datei latex-vorspann.tex mit gesetztem Schalter.

\newif\ifkorrekturansicht
\korrekturansichttrue

\input{../tex-inputs/latex-vorspann}


               \section[Arthur Schnitzler, Karl Kraus und Friedrich Schik an Richard Beer-Hofmann, {[}31. 12. 1893?{]}]{ Arthur Schnitzler, Karl Kraus und Friedrich Schik an Richard
               Beer-Hofmann, {[}31. 12. 1893?{]}}\nopagebreak\mylabel{v}\rehead{ }\normalsize\beginnumbering\briefempfaengerindex{Beer-Hofmann, Richard@\textsc{Beer-Hofmann, Richard}!zzzSchik, Friedrich@\emph{von Friedrich Schik}!1893-12-314@{{[}31. 12. 1893?{]}}|(be}\briefempfaengerindex{Beer-Hofmann, Richard@\textsc{Beer-Hofmann, Richard}!zzzKraus, Karl@\emph{von Karl Kraus}!1893-12-314@{{[}31. 12. 1893?{]}}|(be}\briefempfaengerindex{Beer-Hofmann, Richard@\textsc{Beer-Hofmann, Richard}!zzzSchnitzler, Arthur@\emph{von Arthur Schnitzler}!1893-12-314@{{[}31. 12. 1893?{]}}|(be} \toendnotes[C]{\smallbreak\pagebreak[2]} \Standort{YCGL, MSS 31.}
\physDesc{Visitenkarte mit Trauerrand
\newline{}Handschrift Arthur Schnitzler: Bleistift, deutsche Kurrent\newline{}Handschrift Karl Kraus: Bleistift, deutsche Kurrent\newline{}Handschrift Friedrich Schik: Bleistift, deutsche Kurrent}\buchAbdrucke{\weitereDrucke{Arthur Schnitzler, Richard Beer-Hofmann: \emph{Briefwechsel 1891–1931}. Hg. Konstanze Fliedl. Wien, Zürich: \emph{Europaverlag} 1992, S. 54.} }\toendnotes[C]{\smallbreak}\pstart
           \noindent{}{\pb}An den Verfaſſer des »\textcolor{green}{Kinds}{}\ledrightnote{\textcolor{green}{Das Kind}}«. –\pend
           \pstart
           Wir haben ½ Stunde ununterbrochen über Sie \label{K_L00288_1v}\edtext{geſprochen}{\lemma{\textnormal{\emph{geſprochen}}}\Cendnote{\textnormal{Die drei
                  Unterzeichner sind laut \emph{\textcolor{green}{Tagebuch}} am 31. 12. 1893 gemeinsam im
                  Kaffeehaus.}}}\label{K_L00288_1h}. Auch der \textcolor{blue}{Autor}{}\ledrightnote{→\textcolor{blue}{Felix Salten}} des »\label{K_L00288_2v}\edtext{\textcolor{green}{Begräbniſſes}{}\ledrightnote{\textcolor{green}{Begräbnis}}}{\lemma{\textnormal{\emph{Begräbniſſes}}}\Cendnote{\textnormal{\textcolor{blue}{Felix Salten}: \emph{\textcolor{green}{Begräbnis}}. In: \emph{\textcolor{green}{Mährisches Tagblatt}},
                     Jg. 14, Nr. 160, 17. 7. 1893, S. 1–2. }}}\label{K_L00288_2h}« blieb nicht unerwähnt. – Es iſt bedauerlich, daß ſolche Männer ihre Nächte
               in Dominoorgien hinbringen. –\pend
           \pstart {\pb}In Hochachtung\pend{}\pstart
           \centering{}\textcolor{gray}{\textbf{D\textsuperscript{r}Arthur Schnitzler}}\pend
           \pstart
           \noindent{}{[}hs. Kraus:{]} in aufrichtiger Bewunderung u. Wertschätzung\pend
           \pstart \spacefill\mbox{KarlKraus}\pend{}\pstart
           \noindent{}{[}hs. Schik:{]} ergebenſt\pend
           \pstart \spacefill\mbox{FSchik}\pend{}\endnumbering\briefempfaengerindex{Beer-Hofmann, Richard@\textsc{Beer-Hofmann, Richard}!zzzSchik, Friedrich@\emph{von Friedrich Schik}!1893-12-314@{{[}31. 12. 1893?{]}}|)be}\briefempfaengerindex{Beer-Hofmann, Richard@\textsc{Beer-Hofmann, Richard}!zzzKraus, Karl@\emph{von Karl Kraus}!1893-12-314@{{[}31. 12. 1893?{]}}|)be}\briefempfaengerindex{Beer-Hofmann, Richard@\textsc{Beer-Hofmann, Richard}!zzzSchnitzler, Arthur@\emph{von Arthur Schnitzler}!1893-12-314@{{[}31. 12. 1893?{]}}|)be}\mylabel{h}  \normalsize

\doendnotes{C}
\bigskip
\vfill

\clearpage

\footnotesize

\lohead{\textsc{register}}

% Definiere theindex-Environment komplett neu ohne reledmac
\makeatletter
\renewenvironment{theindex}{%
  \section*{\indexname}%
  \setlength{\parindent}{0pt}%
  \setlength{\parskip}{0pt plus 0.3pt}%
  \let\item\@idxitem
}{%
  \clearpage
}
\makeatother

\IfFileExists{\jobname-pw.ind}{\input{\jobname-pw.ind}}{}

\end{document}

      