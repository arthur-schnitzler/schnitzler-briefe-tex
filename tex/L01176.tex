%% latex-korrekturansicht-vorspann.tex
%% Vorspann für die Korrekturansicht.
%% Lädt die gemeinsame Datei latex-vorspann.tex mit gesetztem Schalter.

\newif\ifkorrekturansicht
\korrekturansichttrue

\input{../tex-inputs/latex-vorspann}


               \section[Arthur Schnitzler an Richard Beer-Hofmann, 30. 9. 1901]{ Arthur Schnitzler an Richard Beer-Hofmann, 30. 9. 1901}\nopagebreak\mylabel{v}\rehead{ }\normalsize\beginnumbering\briefempfaengerindex{Beer-Hofmann, Richard@\textsc{Beer-Hofmann, Richard}!zzzSchnitzler, Arthur@\emph{von Arthur Schnitzler}!1901-09-301@{30. 9. 1901}|(be} \toendnotes[C]{\smallbreak\pagebreak[2]} \Standort{YCGL, MSS 31.}
\physDesc{Brief, 1 Blatt (Briefpapier mit Trauerrand), 1 Seite, Umschlag
\newline{}Handschrift: Bleistift, deutsche Kurrent\newline{}Versand: 1) Rohrpost 2) Stempel: »\nobreak{}Wien, 30 IX 01, 11 10\nobreak{}«. 3) Stempel: »\nobreak{}\oindex{I., Innere Stadt@\textbf{I., Innere Stadt}, \emph{Bezirk (A.BZK)}|pwk}{\pb}Wien 1/1, 30 IX 01, 11 30\nobreak{}«. \newline{}Ordnung: mit Bleistift von unbekannter Hand datiert:
                           »30. 9.« }\buchAbdrucke{\weitereDrucke{Arthur Schnitzler, Richard Beer-Hofmann: \emph{Briefwechsel 1891–1931}. Hg. Konstanze Fliedl. Wien, Zürich: \emph{Europaverlag} 1992, S. 156.} }\pstart{}{\pb}Herrn \textsc{Dr Rich
                     Beer-Hofmann}\pend{}\pstart{}\textcolor{pink}{Wien}{}\ledrightnote{\textcolor{pink}{Wien}}\pend{}\pstart{}\textsc{\textcolor{pink}{I. Wollzeile 15}{}\ledrightnote{\textcolor{pink}{Wollzeile}}}.\pend{}{\bigskip}\pstart
           \noindent{}{\pb}lieber Richard, wieſo hört u ſieht man nichts von Ihnen? Neulich war
               ich bei Ihnen, Sie waren in \textcolor{pink}{Rodaun}{}\ledrightnote{\textcolor{pink}{Rodaun}}. Wollen Sie heut
               mit mir zu \uline{\textcolor{green}{Hannah
                  Jagert}{}\ledrightnote{\textcolor{green}{Hanna Jagert. Komödie}}? Ich habe 2 Sitze}.
               Telephoniren Sie mir etwa 2, oder pneumatisiren Sie. \introOben{}(Rendezvous wäre im Foyer unten.)\introOben{}\pend
           \pstart Herzlichſt Ihr\spacefill\mbox{Arthur}\pend{}\endnumbering\briefempfaengerindex{Beer-Hofmann, Richard@\textsc{Beer-Hofmann, Richard}!zzzSchnitzler, Arthur@\emph{von Arthur Schnitzler}!1901-09-301@{30. 9. 1901}|)be}\mylabel{h}  \normalsize

\doendnotes{C}
\bigskip
\vfill

\clearpage

\footnotesize

\lohead{\textsc{register}}

% Definiere theindex-Environment komplett neu ohne reledmac
\makeatletter
\renewenvironment{theindex}{%
  \section*{\indexname}%
  \setlength{\parindent}{0pt}%
  \setlength{\parskip}{0pt plus 0.3pt}%
  \let\item\@idxitem
}{%
  \clearpage
}
\makeatother

\IfFileExists{\jobname-pw.ind}{\input{\jobname-pw.ind}}{}

\end{document}

      