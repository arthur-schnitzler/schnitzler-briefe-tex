%% latex-korrekturansicht-vorspann.tex
%% Vorspann für die Korrekturansicht.
%% Lädt die gemeinsame Datei latex-vorspann.tex mit gesetztem Schalter.

\newif\ifkorrekturansicht
\korrekturansichttrue

\input{../tex-inputs/latex-vorspann}


               \section[Hugo von Hofmannsthal an Arthur Schnitzler, 31. 7. {[}1899{]}]{ Hugo von Hofmannsthal an Arthur Schnitzler, 31. 7. {[}1899{]}}\nopagebreak\mylabel{v}\rehead{ }\normalsize\beginnumbering\briefempfaengerindex{Schnitzler, Arthur@\textsc{Schnitzler, Arthur}!zzzHofmannsthal, Hugo von@\emph{von Hugo von Hofmannsthal}!1899-07-314@{31. 7. {[}1899{]}}|(be} \toendnotes[C]{\smallbreak\pagebreak[2]} \Standort{CUL, Schnitzler, B 43.}
\physDesc{Brief, 1 Blatt, 3 Seiten
\newline{}Handschrift: schwarze Tinte, deutsche Kurrent
\newline{}Schnitzler: mit Bleistift die Jahreszahl ergänzt: »99« \newline{}Ordnung: mit Bleistift von unbekannter Hand eine frühere Zählung
                           überarbeitet: »15\substVorne{}\textsuperscript{6}\substDazwischen{}3\substHinten{}« }\buchAbdrucke{\weitereDrucke{Hugo von Hofmannsthal, Arthur Schnitzler: \emph{Briefwechsel}. Hg. Therese Nickl und Heinrich Schnitzler. Frankfurt am Main: \emph{S. Fischer} 1964, S. 128.} }\toendnotes[C]{\smallbreak}\pstart
           \raggedleft{}{\pb}\textcolor{pink}{Alt-Aussee}{}\ledrightnote{\textcolor{pink}{Altaussee}}{ }31. VII.\pend
           \pstart{}mein lieber Arthur\pend\pstart
           denken Sie doch was uns ein neues \textcolor{green}{Stück}{}\ledrightnote{→\textcolor{green}{Der Schleier der Beatrice. Schauspiel in fünf Akten}} von Ihnen für eine Freude iſt, dem \textcolor{blue}{Richard}{}\ledrightnote{\textcolor{blue}{Richard Beer-Hofmann}} und mir. Ich war ſo froh, daſs Sie mir über Ihre \textcolor{green}{Arbeit}{}\ledrightnote{→\textcolor{green}{Der Schleier der Beatrice. Schauspiel in fünf Akten}} und über eine Beſſerung in \textcolor{blue}{Richard}{}\ledrightnote{\textcolor{blue}{Richard Beer-Hofmann}}s Sti{\geminationm}ung
               ſchreiben. Ich lebe jetzt hier ein gedankenloſes Leben mit \textsc{tennys} und {\pb}\textsc{bycicle-polo}, nach einer Zeit werde ich an den 3\textsuperscript{ten}{ }\textcolor{green}{Act}{}\ledrightnote{→\textcolor{green}{Das Bergwerk zu Falun}} gehen. Vielleicht, wenn Sie
               nach \textcolor{pink}{Iſchl}{}\ledrightnote{\textcolor{pink}{Bad Ischl}} gehen, in \textcolor{pink}{Iſchl}{}\ledrightnote{\textcolor{pink}{Bad Ischl}}! oder beide in \textcolor{pink}{Salzburg}{}\ledrightnote{\textcolor{pink}{Salzburg}}?\pend
           \pstart
           Ich wünſche Ihnen und den andern möglichſt viel Freude von der Fußpartie.\pend
           \pstart
           \textcolor{blue}{Clemens Franckenstein}{}\ledrightnote{\textcolor{blue}{Clemens von Franckenstein}}{ }{\pb}läſst den \textcolor{blue}{Waſſermann}{}\ledrightnote{\textcolor{blue}{Jakob Wassermann}} fragen, was mit dem \label{K_L00955_1v}\edtext{Operntext}{\lemma{\textnormal{\emph{Operntext}}}\Cendnote{\textnormal{unklar}}}\label{K_L00955_1h} iſt.\pend
           \pstart
           Herzlich Ihr{\\[\baselineskip]}\spacefill\mbox{Hugo.}\pend
           \leftskip=0em{}\endnumbering\briefempfaengerindex{Schnitzler, Arthur@\textsc{Schnitzler, Arthur}!zzzHofmannsthal, Hugo von@\emph{von Hugo von Hofmannsthal}!1899-07-314@{31. 7. {[}1899{]}}|)be}\mylabel{h}  \normalsize

\doendnotes{C}
\bigskip
\vfill

\clearpage

\footnotesize

\lohead{\textsc{register}}

% Definiere theindex-Environment komplett neu ohne reledmac
\makeatletter
\renewenvironment{theindex}{%
  \section*{\indexname}%
  \setlength{\parindent}{0pt}%
  \setlength{\parskip}{0pt plus 0.3pt}%
  \let\item\@idxitem
}{%
  \clearpage
}
\makeatother

\IfFileExists{\jobname-pw.ind}{\input{\jobname-pw.ind}}{}

\end{document}

      