%% latex-korrekturansicht-vorspann.tex
%% Vorspann für die Korrekturansicht.
%% Lädt die gemeinsame Datei latex-vorspann.tex mit gesetztem Schalter.

\newif\ifkorrekturansicht
\korrekturansichttrue

\input{../tex-inputs/latex-vorspann}


               \section[Arthur Schnitzler an Marie Herzfeld, 7. 3. 1931]{ Arthur Schnitzler an Marie Herzfeld, 7. 3. 1931}\nopagebreak\mylabel{v}\rehead{ }\normalsize\beginnumbering\briefempfaengerindex{Herzfeld, Marie@\textsc{Herzfeld, Marie}!zzzSchnitzler, Arthur@\emph{von Arthur Schnitzler}!1931-03-071@{7. 3. 1931}|(be} \toendnotes[C]{\smallbreak\pagebreak[2]} \Standort{DLA, A:Schnitzler, HS.1985.1.993.}
\physDesc{Brief, 1 Blatt, 1 Seite, maschineller Durchschlag
\newline{}Schreibmaschine
\newline{}Handschrift: roter Buntstift, lateinische Kurrent (\noindent{}mit rotem Buntstift Vermerk »\textsc{\uline{Herzfeld}}« und
                                 sieben Unterstreichungen)}\toendnotes[C]{\smallbreak}\pstart
           \raggedleft{}{\pb}7. 3. 1931\pend
           \pstart{}Verehrtes Fräulein.\pend\pstart
           Dass es sich bei dem in \textcolor{blue}{Hofmannsthal}{}\ledrightnote{\textcolor{blue}{Hugo von Hofmannsthal}}s \label{K_L02598-1v}\edtext{Brief vom
                  19. Juli 92}{\lemma{\textnormal{\emph{Brief vom
                  19. Juli 92}}}\Cendnote{\textnormal{siehe Marie Herzfeld an Arthur Schnitzler, 5. 3. 1931}}}\label{K_L02598-1h} und am \label{K_L02598-2v}\edtext{4. August}{\lemma{\textnormal{\emph{4. August}}}\Cendnote{\textnormal{siehe Hugo von Hofmannsthal an Arthur Schnitzler, 4. 8. [1892]}}}\label{K_L02598-2h} erwähnten \label{K_L02598-3v}\edtext{\textcolor{green}{Renaissancedrama}{}\ledrightnote{→\textcolor{green}{Ascanio und Gioconda}}}{\lemma{\textnormal{\emph{Renaissancedrama}}}\Cendnote{\textnormal{Gemeint ist das zu Lebzeiten
                  unveröffentlicht gebliebene Drama \emph{\textcolor{green}{Ascanio und
                  Gioconda}}.}}}\label{K_L02598-3h} schon um die Vorarbeiten zum »\textcolor{green}{Geretteten Venedig}{}\ledrightnote{\textcolor{green}{Das gerettete Venedig. Trauerspiel in fünf Aufzügen}}« handeln könnte, halte ich für durchaus unwahrscheinlich;
               Positives kann ich freilich nicht behaupten. Ich vermag mich auch nicht zu erinnern,
               dass \textcolor{blue}{Hofmannsthal}{}\ledrightnote{\textcolor{blue}{Hugo von Hofmannsthal}} mir später von dieser
               fünfaktigen \textcolor{green}{Renaissancetragödie}{}\ledrightnote{→\textcolor{green}{Ascanio und Gioconda}}{ }\label{T_L02598-1v}\edtext{»dramatisierter
                  Novelle{[}«{]}}{\lemma{\textnormal{\emph{»dramatisierter
                  Novelle«}}}\Cendnote{\textnormal{Das eine
                  Anführungszeichen ist mit Schreibmaschine genau in den Leerraum zwischen
                     »Renaissancetragödie« und »dramatisierter«
                  gesetzt, so dass das Anführungszeichen alternativ auch das schließende der
                     »Renaissancetragödie« sein könnte.}}}\label{T_L02598-1h}, äusserlich im Stil
               von »\textcolor{green}{Romeo und Julie}{}\ledrightnote{\textcolor{green}{Romeo und Julia}}« später wieder gesprochen oder
               mir Verse daraus vorgelesen hätte. Immerhin wäre es denkbar, dass Stellen aus dem
               Entwurf in andere Werke von ihm übergegangen sind, vielleicht sogar ins »\textcolor{green}{Gerettete Venedig}{}\ledrightnote{\textcolor{green}{Das gerettete Venedig. Trauerspiel in fünf Aufzügen}}«.\pend
           \pstart
           Möglich auch, dass er mir seinerzeit mehr von jener \textcolor{green}{Tragödie}{}\ledrightnote{→\textcolor{green}{Ascanio und Gioconda}} erzählt oder mir manchmal auch daraus vorgelesen
               hätte; – das wäre ja bald 40 Jahre her und man hat ja leider mancherlei
               vergessen.\pend
           \pstart
           Ich freue mich, nach so langer Zeit wieder einmal direkt von
                  {[}Ihnen{]} etwas gehört zu haben und bin mit herzlichen
               Grüssen\pend
           \pstart Ihr aufrichtig ergebener\pend{}{\bigskip}\pstart
           \noindent{}Fräulein Marie Herzfeld,{\\}\textcolor{pink}{Wien III.}{}\ledrightnote{\textcolor{pink}{III., Landstraße}}{\\}\textcolor{pink}{Oetzeltg. 1}{}\ledrightnote{\textcolor{pink}{Ölzeltgasse}}.\pend
           \endnumbering\briefempfaengerindex{Herzfeld, Marie@\textsc{Herzfeld, Marie}!zzzSchnitzler, Arthur@\emph{von Arthur Schnitzler}!1931-03-071@{7. 3. 1931}|)be}\mylabel{h}  \normalsize

\doendnotes{C}
\bigskip
\vfill

\clearpage

\footnotesize

\lohead{\textsc{register}}

% Definiere theindex-Environment komplett neu ohne reledmac
\makeatletter
\renewenvironment{theindex}{%
  \section*{\indexname}%
  \setlength{\parindent}{0pt}%
  \setlength{\parskip}{0pt plus 0.3pt}%
  \let\item\@idxitem
}{%
  \clearpage
}
\makeatother

\IfFileExists{\jobname-pw.ind}{\input{\jobname-pw.ind}}{}

\end{document}

      