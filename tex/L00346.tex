%% latex-korrekturansicht-vorspann.tex
%% Vorspann für die Korrekturansicht.
%% Lädt die gemeinsame Datei latex-vorspann.tex mit gesetztem Schalter.

\newif\ifkorrekturansicht
\korrekturansichttrue

\input{../tex-inputs/latex-vorspann}


               \section[Peter Altenberg an Arthur Schnitzler, {[}5.? 7. 1894{]}]{ Peter Altenberg an Arthur Schnitzler, {[}5.? 7. 1894{]}}\nopagebreak\mylabel{v}\rehead{ }\normalsize\beginnumbering\briefempfaengerindex{Schnitzler, Arthur@\textsc{Schnitzler, Arthur}!zzzAltenberg, Peter@\emph{von Peter Altenberg}!1894-07-051@{{[}5.? 7. 1894{]}}|(be} \toendnotes[C]{\smallbreak\pagebreak[2]} \Standort{CUL, Schnitzler, B 2.}
\physDesc{Brief, 1 Blatt, 1 Seite
\newline{}Handschrift: schwarze Tinte, deutsche Kurrent
\newline{}Schnitzler: 1) mit rotem Buntstift eine Unterstreichung 2) mit Bleistift datiert: »Anf Juli 94.« und nummeriert: »2«\newline{}Ordnung: mit Bleistift von unbekannter Hand nummeriert: »2« }\buchAbdrucke{\weitereDrucke{Peter Altenberg: \emph{Die Selbsterfindung eines Dichters. Briefe und
                                Dokumente 1892–1896}. Hg. und mit einem Nachwort von Leo A. Lensing. Göttingen: \emph{Wallstein} 2009, S. 23.} }\toendnotes[C]{\smallbreak}\pstart{}{\pb}Lieber \textsc{D\textsuperscript{r.}} Arthur Schnitzler.\pend\pstart
           Auf ihren Wunſch ſende ich Ihnen eine Skizze »\label{K_L00346_1v}\edtext{\textcolor{green}{See-Ufer}{}\ledrightnote{\textcolor{green}{See-Ufer}}}{\lemma{\textnormal{\emph{See-Ufer}}}\Cendnote{\textnormal{Die Skizze ist nicht überliefert, sehr
                        wohl aber verwendete \textcolor{blue}{Altenberg} ihn als
                        Sammeltitel für eine Skizzenreihe in seiner ersten Buchveröffentlichung
                            \emph{\textcolor{green}{Wie ich es sehe}} (Berlin: \emph{\textcolor{brown}{S. Fischer}}{ }1896).}}}\label{K_L00346_1h}« u. hoffe, daß dieſelbe Ihnen nicht zu ſehr miſſfallen
                    wird.\pend
           \pstart
           Ihr{\\[\baselineskip]}\spacefill\mbox{Richard Engländer.}\pend
           \leftskip=0em{}\endnumbering\briefempfaengerindex{Schnitzler, Arthur@\textsc{Schnitzler, Arthur}!zzzAltenberg, Peter@\emph{von Peter Altenberg}!1894-07-051@{{[}5.? 7. 1894{]}}|)be}\mylabel{h}  \normalsize

\doendnotes{C}
\bigskip
\vfill

\clearpage

\footnotesize

\lohead{\textsc{register}}

% Definiere theindex-Environment komplett neu ohne reledmac
\makeatletter
\renewenvironment{theindex}{%
  \section*{\indexname}%
  \setlength{\parindent}{0pt}%
  \setlength{\parskip}{0pt plus 0.3pt}%
  \let\item\@idxitem
}{%
  \clearpage
}
\makeatother

\IfFileExists{\jobname-pw.ind}{\input{\jobname-pw.ind}}{}

\end{document}

      