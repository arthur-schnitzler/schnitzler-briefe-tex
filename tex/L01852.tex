%% latex-korrekturansicht-vorspann.tex
%% Vorspann für die Korrekturansicht.
%% Lädt die gemeinsame Datei latex-vorspann.tex mit gesetztem Schalter.

\newif\ifkorrekturansicht
\korrekturansichttrue

\input{../tex-inputs/latex-vorspann}


               \section[Albert Ehrenstein an Arthur Schnitzler, 1. 7. 1909]{ Albert Ehrenstein an Arthur Schnitzler, 1. 7. 1909}\nopagebreak\mylabel{v}\rehead{ }\normalsize\beginnumbering\briefempfaengerindex{Schnitzler, Arthur@\textsc{Schnitzler, Arthur}!zzzEhrenstein, Albert@\emph{von Albert Ehrenstein}!1909-07-011@{1. 7. 1909}|(be} \toendnotes[C]{\smallbreak\pagebreak[2]} \Standort{CUL, Schnitzler, B 30.}
\physDesc{Brief, 1 Blatt, 3 Seiten
\newline{}Handschrift: schwarze Tinte, deutsche Kurrent
\newline{}Schnitzler: mit Bleistift beschriftet: »\textsc{Ehrenste\textcolor{gray}{in}}« }\toendnotes[C]{\smallbreak}\pstart
           {\pb}\textcolor{pink}{Wien, XVI. \textsc{Ottakringerstr 114}}{}\ledrightnote{\textcolor{pink}{Ottakringerstraße}}\hfill \textsc{1. Juli 09}.\pend
           \pstart{}\textsc{Sehr geehrter Herr Doktor,}\pend\pstart
           ohne läſtig fallen zu wollen, wäre es mir ſehr angenehm, wenn Sie, ſehr geehrter
                    Herr Doktor, meinen \textcolor{green}{drei}{}\ledrightnote{→\textcolor{green}{Apaturien}{\newline}→\textcolor{green}{Tubutsch}{\newline}→\textcolor{green}{Tod des Zehir eddin Muhammed Baber}} ebenſo länglichen als mißlungenen novelliſtiſchen Verſuchen, im
                    Laufe der nächſten Wochen auf die eine oder die andere Art nahe zu treten die
                    Güte haben möchten. Nach den Betrachtungen, die über \textcolor{blue}{H. Mann}{}\ledrightnote{\textcolor{blue}{Heinrich Mann}} anzuſtellen ich unvorſichtig genug war, ſehne ich
                    mich keineswegs. Da {\pb}der \textcolor{brown}{Erdgeiſt}{}\ledrightnote{\textcolor{brown}{Erdgeist}} eingegangen iſt und mir dabei mein noch nicht abgedrucktes
                    und abſchriftloſes Manuſkript einer Skizze verloren ging, meine \textcolor{green}{Diſſertation}{}\ledrightnote{→\textcolor{green}{Die Lage in Ungarn (Siebenbürgen und Serbien ausgenommen) im Jahre 1790}}, ſo konſervativ wie meine
                    andern Arbeiten gehalten war, begegnete ich bei dem betreffenden \textcolor{blue}{Hofrat}{}\ledrightnote{→\textcolor{blue}{August Fournier}} namenloſen
                    Chikanen. Ich werde allen möglichen Namen- und Zahlenkram lernen müſſen und doch
                    nicht viel Chancen bei der Prüfung haben, wenn nicht irgend was augenfälliges
                    von mir in der \textcolor{green}{Zeit}{}\ledrightnote{\textcolor{green}{Die Zeit}} oder \textcolor{brown}{Preſſe}{}\ledrightnote{\textcolor{brown}{Neue Freie Presse}} oder ſonſt einer reſpektabeln Zeitung erſcheint.
                    Sollten Sie, {\pb}ſehr geehrter Herr Doktor mir in dieſer
                    unverſchuldeten Zwangslage im mindeſten Beihilfe leiſten können, wäre ich ſo
                    glücklich wie nur ein Menſch ſein kann, der die Namen ſämtlicher Erzbiſchöfe von
                        \textcolor{pink}{Köln}{}\ledrightnote{\textcolor{pink}{Köln}} und dergleichen Ungeheuerlichkeiten
                    ſeinem Gedächtniſſe einzuverleiben das Vergnügen hat.\pend
           \pstart
           Indem ich um Entſchuldigung dieſes in der Eile hingeworfenen Briefes bitte,
                    verbleibe ich\pend
           \pstart
           Ihr ergebenſter{\\[\baselineskip]}\spacefill\mbox{Albert Ehrenstein.}\pend
           \leftskip=0em{}\endnumbering\briefempfaengerindex{Schnitzler, Arthur@\textsc{Schnitzler, Arthur}!zzzEhrenstein, Albert@\emph{von Albert Ehrenstein}!1909-07-011@{1. 7. 1909}|)be}\mylabel{h}  \normalsize

\doendnotes{C}
\bigskip
\vfill

\clearpage

\footnotesize

\lohead{\textsc{register}}

% Definiere theindex-Environment komplett neu ohne reledmac
\makeatletter
\renewenvironment{theindex}{%
  \section*{\indexname}%
  \setlength{\parindent}{0pt}%
  \setlength{\parskip}{0pt plus 0.3pt}%
  \let\item\@idxitem
}{%
  \clearpage
}
\makeatother

\IfFileExists{\jobname-pw.ind}{\input{\jobname-pw.ind}}{}

\end{document}

      