%% latex-korrekturansicht-vorspann.tex
%% Vorspann für die Korrekturansicht.
%% Lädt die gemeinsame Datei latex-vorspann.tex mit gesetztem Schalter.

\newif\ifkorrekturansicht
\korrekturansichttrue

\input{../tex-inputs/latex-vorspann}


               \section[Georg Brandes an Arthur Schnitzler, 26. 9. 1912]{ Georg Brandes an Arthur Schnitzler, 26. 9. 1912}\nopagebreak\mylabel{v}\rehead{ }\normalsize\beginnumbering\briefempfaengerindex{Schnitzler, Arthur@\textsc{Schnitzler, Arthur}!zzzBrandes, Georg@\emph{von Georg Brandes}!1912-09-261@{26. 9. 1912}|(be} \toendnotes[C]{\smallbreak\pagebreak[2]} \Standort{CUL, Schnitzler, B 17.}
\physDesc{Postkarte
\newline{}Handschrift: blaue Tinte, lateinische Kurrent\newline{}Versand: Stempel: »\nobreak{}\oindex{Skagen@\textbf{Skagen}, \emph{https://www.geonames.org/ontologyP.PPL}|pwk}Skagen, 26.9. 12, 12–3E\nobreak{}«.  \newline{}Ordnung: mit Bleistift von unbekannter Hand nummeriert:
                                    »39« }\buchAbdrucke{\weitereDrucke{Georg Brandes, Arthur Schnitzler: \emph{Ein Briefwechsel}. Hg. Kurt Bergel. Bern: \emph{Francke} 1956, S. 104–105.} }\toendnotes[C]{\smallbreak}\pstart{}{\pb}Herrn Dr. Arthur
                  Schnitzler\pend{}\pstart{}\textcolor{pink}{Sternwartestrasse 71}{}\ledrightnote{\textcolor{pink}{Sternwartestraße}}\pend{}\pstart{}\textcolor{pink}{Wien XVIII}{}\ledrightnote{\textcolor{pink}{XVIII., Währing}}\pend{}{\bigskip}\pstart
           \raggedleft{}{\pb}\label{K_L02091_1v}\edtext{p. T.}{\lemma{\textnormal{\emph{p. T.}}}\Cendnote{\textnormal{pro tempore, lateinisch: zur Zeit}}}\label{K_L02091_1h}{ }\textcolor{pink}{Skagen}{}\ledrightnote{\textcolor{pink}{Skagen}}{\\}\textcolor{pink}{Dänemark}{}\ledrightnote{\textcolor{pink}{Dänemark}}{\\}Adresse: \textcolor{pink}{Kopenhagen}{}\ledrightnote{\textcolor{pink}{Kopenhagen}}\pend
           \pstart{}Verehrter Freund\pend\pstart
           Da es zehn oder elf Jahre her ist, dass ich in \textcolor{pink}{Wien}{}\ledrightnote{\textcolor{pink}{Wien}} war, habe ich um einen Vorwand zu haben, es wiederzusehen, mich
               engagiren lassen, ein Paar erbärmliche Vorträge zwischen 20{ }\strikeout{\textcolor{gray}{×}} und 23 November dort zu halten (\textcolor{pink}{\uline{Urania}}{}\ledrightnote{\textcolor{pink}{Urania}}). Da nun Sie ein Hauptstück von meinem \textcolor{pink}{Wien}{}\ledrightnote{\textcolor{pink}{Wien}}
               sind, möchte ich gerne wissen, ob Sie wol zu der Zeit sich in \textcolor{pink}{Wien}{}\ledrightnote{\textcolor{pink}{Wien}} befinden (nur auf einer Karte antworten).\pend
           \pstart
           Ich möchte noch gerne \textcolor{blue}{Beer-Hoffmann}{}\ledrightnote{\textcolor{blue}{Richard Beer-Hofmann}} und \textcolor{blue}{Wassermann}{}\ledrightnote{\textcolor{blue}{Jakob Wassermann}} und \textcolor{blue}{Hofmannsthal}{}\ledrightnote{\textcolor{blue}{Hugo von Hofmannsthal}} sehen, die ja alle \textcolor{pink}{Wien}{}\ledrightnote{\textcolor{pink}{Wien}}er sind; Sie sind aber für mich die Hauptperson.\pend
           \pstart
           Ihr sehr ergebener{\\[\baselineskip]}\spacefill\mbox{Georg Brandes}\pend
           \leftskip=0em{}\endnumbering\briefempfaengerindex{Schnitzler, Arthur@\textsc{Schnitzler, Arthur}!zzzBrandes, Georg@\emph{von Georg Brandes}!1912-09-261@{26. 9. 1912}|)be}\mylabel{h}  \normalsize

\doendnotes{C}
\bigskip
\vfill

\clearpage

\footnotesize

\lohead{\textsc{register}}

% Definiere theindex-Environment komplett neu ohne reledmac
\makeatletter
\renewenvironment{theindex}{%
  \section*{\indexname}%
  \setlength{\parindent}{0pt}%
  \setlength{\parskip}{0pt plus 0.3pt}%
  \let\item\@idxitem
}{%
  \clearpage
}
\makeatother

\IfFileExists{\jobname-pw.ind}{\input{\jobname-pw.ind}}{}

\end{document}

      