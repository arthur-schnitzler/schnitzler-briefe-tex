%% latex-korrekturansicht-vorspann.tex
%% Vorspann für die Korrekturansicht.
%% Lädt die gemeinsame Datei latex-vorspann.tex mit gesetztem Schalter.

\newif\ifkorrekturansicht
\korrekturansichttrue

\input{../tex-inputs/latex-vorspann}


               \section[Franz Blei an Arthur Schnitzler, {[}17. 12. 1909{]}]{ Franz Blei an Arthur Schnitzler, {[}17. 12. 1909{]}}\nopagebreak\mylabel{v}\rehead{ }\normalsize\beginnumbering\briefempfaengerindex{Schnitzler, Arthur@\textsc{Schnitzler, Arthur}!zzzBlei, Franz@\emph{von Franz Blei}!1909-12-171@{{[}17. 12. 1909{]}}|(be} \toendnotes[C]{\smallbreak\pagebreak[2]} \Standort{CUL, Schnitzler, B 14.}
\physDesc{Brief, 1 Blatt, 1 Seite
\newline{}Handschrift: schwarze Tinte, lateinische Kurrent
\newline{}Schnitzler: 1) mit Bleistift beschriftet: »\textsc{Blei}« und datiert: »17/12 09« 2) mit rotem Buntstift zwei Unterstreichungen\newline{}Ordnung: mit Bleistift von unbekannter Hand nummeriert:
                                        »4« }\toendnotes[C]{\smallbreak}\pstart{}{\pb}Sehr verehrter Herr
                        Schnitzler,\pend\pstart
           ich danke Ihnen sehr, dass Sie an den \textcolor{brown}{Hyper.}{}\ledrightnote{\textcolor{brown}{Hyperion}}
                    gedacht haben und würde – wie Sie sich denken können – mit Freuden das \textcolor{green}{Vorspiel}{}\ledrightnote{→\textcolor{green}{Der junge Medardus. Dramatische Historie in einem Vorspiel und fünf Aufzügen}} drucken, wenn der
                        \textcolor{blue}{Verleger}{}\ledrightnote{→\textcolor{blue}{Hans von Weber}} nicht anderer
                    Meinung wäre damit, dass er mir die Unmöglichkeit beweist, dass die Zeitschrift
                    das verlangte Honorar zahlen kann, auch nicht zahlen könnte, wenn sie mehr als
                    540 Abonnenten hätte und die ganze Auflage von 1000 Ex. abonniert wäre. Der \textcolor{brown}{Hyper}{}\ledrightnote{\textcolor{brown}{Hyperion}} ist für keinen der Betheiligten
                    irgendwann einmal ein Geschäft. – So kann ich also nur traurig für Ihre
                    Freundlichkeit danken.\pend
           \pstart
           Ich bin Ihr immer ergebner{\\[\baselineskip]}\spacefill\mbox{Frz Blei}\pend
           \leftskip=0em{}\endnumbering\briefempfaengerindex{Schnitzler, Arthur@\textsc{Schnitzler, Arthur}!zzzBlei, Franz@\emph{von Franz Blei}!1909-12-171@{{[}17. 12. 1909{]}}|)be}\mylabel{h}  \normalsize

\doendnotes{C}
\bigskip
\vfill

\clearpage

\footnotesize

\lohead{\textsc{register}}

% Definiere theindex-Environment komplett neu ohne reledmac
\makeatletter
\renewenvironment{theindex}{%
  \section*{\indexname}%
  \setlength{\parindent}{0pt}%
  \setlength{\parskip}{0pt plus 0.3pt}%
  \let\item\@idxitem
}{%
  \clearpage
}
\makeatother

\IfFileExists{\jobname-pw.ind}{\input{\jobname-pw.ind}}{}

\end{document}

      