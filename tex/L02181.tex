%% latex-korrekturansicht-vorspann.tex
%% Vorspann für die Korrekturansicht.
%% Lädt die gemeinsame Datei latex-vorspann.tex mit gesetztem Schalter.

\newif\ifkorrekturansicht
\korrekturansichttrue

\input{../tex-inputs/latex-vorspann}


               \section[Arthur Schnitzler an Hermann Bahr, 12. 6. 1914]{ Arthur Schnitzler an Hermann Bahr, 12. 6. 1914}\nopagebreak\mylabel{v}\rehead{ }\normalsize\beginnumbering\briefempfaengerindex{Bahr, Hermann@\textsc{Bahr, Hermann}!zzzSchnitzler, Arthur@\emph{von Arthur Schnitzler}!1914-06-121@{12. 6. 1914}|(be} \toendnotes[C]{\smallbreak\pagebreak[2]} \Standort{TMW, HS AM 23395 Ba.}
\physDesc{Brief, 1 Blatt, 2 Seiten
\newline{}Schreibmaschine
\newline{}Handschrift: schwarze Tinte (\noindent{}Korrekturen, Unterschrift)}\Standort{DLA, A:Schnitzler, 85.1.294/5.}
\physDesc{Brief, maschineller Durchschlag
\newline{}Schreibmaschine}\buchAbdrucke{\weitereDrucke{1) Arthur Schnitzler: \emph{Briefe 1913–1931}. Hg. Peter Michael Braunwarth, Richard Miklin, Susanne Pertlik und Heinrich Schnitzler. Frankfurt am Main: \emph{S. Fischer} 1984, S. 43.} \weitereDrucke{2) \emph{12. 6. 1914.} In: Arthur Schnitzler: \emph{The Letters of Arthur Schnitzler to Hermann Bahr}. Edited, annotated, and with an introduction, by Donald G.
                        Daviau. Chapel Hill: \emph{The University of North Carolina Press} 1978, S. 113 (University of North Carolina studies in the Germanic languages
                        and literatures, 89).} \weitereDrucke{3) Hermann Bahr, Arthur Schnitzler: \emph{Briefwechsel, Aufzeichnungen, Dokumente (1891–1931)}. Hg. Kurt Ifkovits und Martin Anton Müller. Göttingen: \emph{Wallstein} 2018, S. 494.} }\toendnotes[C]{\smallbreak}\pstart
           \noindent{}{\pb}\textcolor{gray}{\textbf{Dr. Arthur Schnitzler}}\hfill 12. 6. 1914. \pend
           \pstart
           \textcolor{gray}{\textbf{\textcolor{pink}{Wien XVIII. Sternwartestrasse 71}{}\ledrightnote{\textcolor{pink}{Sternwartestraße}}}}\pend
           \pstart{}Lieber Hermann.\pend\pstart
           Wie Dir ja bekannt ist war der »\textcolor{green}{Reigen}{}\ledrightnote{\textcolor{green}{Reigen. Zehn Dialoge}}« bisher in
                  \textcolor{pink}{Deutschland}{}\ledrightnote{\textcolor{pink}{Deutschland}} ein verbotenes Buch. Nun soll von dem
               Verlag \textcolor{brown}{J. Singer {\kaufmannsund} Co.}{}\ledrightnote{\textcolor{brown}{J. Singer {\kaufmannsund} Co.}},
                  \textcolor{pink}{Berlin}{}\ledrightnote{\textcolor{pink}{Berlin}}, eine \label{K_L02181_1v}\edtext{Neuauflage}{\lemma{\textnormal{\emph{Neuauflage}}}\Cendnote{\textnormal{Eine Titelauflage der Erstausgabe im \emph{\textcolor{brown}{Wiener Verlag}},
                  erschienen ohne Jahresangabe. Das heißt, dass Seiten des ursprünglichen Druckes verwendet werden, aber mit einem
                  neuen Titel und Umschlag versehen sind.
                  Selbst die Verlagswerbung deutet auf das
                  ursprüngliche Erscheinen (»Im gleichen Verlag erscheint von \textcolor{blue}{Arthur Schnitzler}«), ebenso die Hinweise auf die Auflage: »44.–46.
                  Tausend«.}}}\label{K_L02181_1h} veröffentlicht werden, deren Beschlagnahme vorauszusehen
               ist, und es kommt dem Verlag darauf an bei einem eventuell bevorstehenden Prozess
               etliche \label{K_L02181_2v}\edtext{Gutachten}{\lemma{\textnormal{\emph{Gutachten}}}\Cendnote{\textnormal{Die Briefe der Genannten und ein weiterer
                  von \textcolor{blue}{Maximilian Harden} finden sich in der Mappe
                     B 128 in der \emph{Cambridge University Library}
                     (»Opinions on Reigen«).}}}\label{K_L02181_2h} zur Verfügung zu haben. Solche von \textcolor{blue}{Liszt}{}\ledrightnote{\textcolor{blue}{Franz von Liszt}}, \textcolor{blue}{Lilienthal}{}\ledrightnote{\textcolor{blue}{Karl von Lilienthal}}, \textcolor{blue}{Eulenburg}{}\ledrightnote{\textcolor{blue}{Herbert Eulenberg}}, \textcolor{blue}{Simmel}{}\ledrightnote{\textcolor{blue}{Georg Simmel}}, \textcolor{blue}{Liebermann}{}\ledrightnote{\textcolor{blue}{Max Liebermann}}, \textcolor{blue}{Fulda}{}\ledrightnote{\textcolor{blue}{Ludwig Fulda}} liegen schon vor (in zum Teil ganz
               überraschend günstigem Sinne, muss ich sagen); und da der Verlag doch gern auch aus
                  \textcolor{pink}{Oesterreich}{}\ledrightnote{\textcolor{pink}{Österreich}} etwas in der Art möchte vorweisen
               können, so fiel mir ein, dass vor Jahren, als dir einmal die öffentliche Vorlesung
               des »\textcolor{green}{Reigen}{}\ledrightnote{\textcolor{green}{Reigen. Zehn Dialoge}}« untersagt wurde, \textcolor{blue}{Burckhardt}{}\ledrightnote{\textcolor{blue}{Max Eugen Burckhard}} einen Rekurs eingebracht hat, der sich vielleicht
               noch in Deinem Besitze finden mag. Ich frage Dich nun, ob Du dem Verlag \textcolor{brown}{J. Singer}{}\ledrightnote{\textcolor{brown}{J. Singer {\kaufmannsund} Co.}}, wenn er sich {\pb}mit entsprechender
               Bitte an Dich wenden sollte, jenes Schriftstück zu eventueller Benützung vor Gericht
               auszufolgen geneigt wärest? \pend
           \pstart
           Mit herzlichem Gruss{\\[\baselineskip]}Dein{\\[\baselineskip]}\spacefill\mbox{{[}hs.:{]} Arthur}\pend
           \leftskip=0em{}\endnumbering\briefempfaengerindex{Bahr, Hermann@\textsc{Bahr, Hermann}!zzzSchnitzler, Arthur@\emph{von Arthur Schnitzler}!1914-06-121@{12. 6. 1914}|)be}\mylabel{h}  \normalsize

\doendnotes{C}
\bigskip
\vfill

\clearpage

\footnotesize

\lohead{\textsc{register}}

% Definiere theindex-Environment komplett neu ohne reledmac
\makeatletter
\renewenvironment{theindex}{%
  \section*{\indexname}%
  \setlength{\parindent}{0pt}%
  \setlength{\parskip}{0pt plus 0.3pt}%
  \let\item\@idxitem
}{%
  \clearpage
}
\makeatother

\IfFileExists{\jobname-pw.ind}{\input{\jobname-pw.ind}}{}

\end{document}

      