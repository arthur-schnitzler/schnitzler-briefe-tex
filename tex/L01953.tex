%% latex-korrekturansicht-vorspann.tex
%% Vorspann für die Korrekturansicht.
%% Lädt die gemeinsame Datei latex-vorspann.tex mit gesetztem Schalter.

\newif\ifkorrekturansicht
\korrekturansichttrue

\input{../tex-inputs/latex-vorspann}


               \section[Arthur Schnitzler an Richard Beer-Hofmann, 11. 8. 1910]{ Arthur Schnitzler an Richard Beer-Hofmann, 11. 8. 1910}\nopagebreak\mylabel{v}\rehead{ }\normalsize\beginnumbering\briefempfaengerindex{Beer-Hofmann, Richard@\textsc{Beer-Hofmann, Richard}!zzzSchnitzler, Arthur@\emph{von Arthur Schnitzler}!1910-08-111@{23. 7. 1910}|(be} \toendnotes[C]{\smallbreak\pagebreak[2]} \Standort{YCGL, MSS 31.}
\physDesc{Postkarte
\newline{}Handschrift: Bleistift, deutsche Kurrent\newline{}Versand: Stempel: »\nobreak{}\oindex{XVIII., Waehring@\textbf{XVIII., Währing}, \emph{Bezirk (A.BZK)}|pwk}18/\textcolor{gray}{×} Wien 110, 11. VIII. 10, XII\nobreak{}«.  \newline{}Ordnung: mit Bleistift von unbekannter Hand datiert: »11. 8.« }\toendnotes[C]{\smallbreak}\pstart{}{\pb}\textsc{\textcolor{pink}{XVIII. Sternwartestraße 71}{}\ledrightnote{\textcolor{pink}{Sternwartestraße}}.}\pend{}{\bigskip}\pstart{}Herrn Doctor\pend{}\pstart{}\textsc{Richard Beer-Hofma{\geminationn}}\pend{}\pstart{}\textsc{\textcolor{pink}{Ischl}{}\ledrightnote{\textcolor{pink}{Bad Ischl}}}\pend{}\pstart{}\textsc{\textcolor{pink}{Steinfeld 6}{}\ledrightnote{\textcolor{pink}{Steinfeld}}}\pend{}{\bigskip}\pstart
           \noindent{}{\pb}lieber Richard, wollen Sie ſo gut ſein mir eine Zeile (womöglich
               gleich) ſchreiben wo Sie Ihre Gartenmöbel beſtellt haben?\pend
           \pstart \label{T_L01953_1v}\edtext{Herzlichſt Ihr \spacefill\mbox{A.}}{\lemma{\textnormal{\emph{Herzlichſt Ihr A.}}}\Cendnote{\textnormal{am rechten Seitenrand, quer zum
                  Text}}}\label{T_L01953_1h}\pend{}\endnumbering\briefempfaengerindex{Beer-Hofmann, Richard@\textsc{Beer-Hofmann, Richard}!zzzSchnitzler, Arthur@\emph{von Arthur Schnitzler}!1910-08-111@{23. 7. 1910}|)be}\mylabel{h}  \normalsize

\doendnotes{C}
\bigskip
\vfill

\clearpage

\footnotesize

\lohead{\textsc{register}}

% Definiere theindex-Environment komplett neu ohne reledmac
\makeatletter
\renewenvironment{theindex}{%
  \section*{\indexname}%
  \setlength{\parindent}{0pt}%
  \setlength{\parskip}{0pt plus 0.3pt}%
  \let\item\@idxitem
}{%
  \clearpage
}
\makeatother

\IfFileExists{\jobname-pw.ind}{\input{\jobname-pw.ind}}{}

\end{document}

      