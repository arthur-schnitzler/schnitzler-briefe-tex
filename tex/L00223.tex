%% latex-korrekturansicht-vorspann.tex
%% Vorspann für die Korrekturansicht.
%% Lädt die gemeinsame Datei latex-vorspann.tex mit gesetztem Schalter.

\newif\ifkorrekturansicht
\korrekturansichttrue

\input{../tex-inputs/latex-vorspann}


               \section[Richard Beer-Hofmann an Arthur Schnitzler, {[}vor dem 22. 6. 1893?{]}]{ Richard Beer-Hofmann an Arthur Schnitzler, {[}vor dem
               22. 6. 1893?{]}}\nopagebreak\mylabel{v}\rehead{ }\normalsize\beginnumbering\briefempfaengerindex{Schnitzler, Arthur@\textsc{Schnitzler, Arthur}!zzzBeer-Hofmann, Richard@\emph{von Richard Beer-Hofmann}!1893-06-201@{{[}vor dem 22. 6. 1893?{]}}|(be} \toendnotes[C]{\smallbreak\pagebreak[2]} \Standort{CUL, Schnitzler, B 8.}
\physDesc{Briefkarte
\newline{}Handschrift: Bleistift, lateinische Kurrent
\newline{}Schnitzler: mit Bleistift nummeriert: »16« }\buchAbdrucke{\weitereDrucke{Arthur Schnitzler, Richard Beer-Hofmann: \emph{Briefwechsel 1891–1931}. Hg. Konstanze Fliedl. Wien, Zürich: \emph{Europaverlag} 1992, S. 44.} }\toendnotes[C]{\smallbreak}\pstart
           \noindent{}\textcolor{gray}{\textbf{\label{T_L00223-1v}\edtext{RB}{\lemma{\textnormal{\emph{RB}}}\Cendnote{\textnormal{Monogramm in Golddruck}}}\label{T_L00223-1h}}}\pend
           \pstart{}{\pb}Lieber Arthur!\pend\pstart
           Wie ich aus den Theaterzetteln entnehme ist \textcolor{blue}{Jarno}{}\ledrightnote{\textcolor{blue}{Josef Jarno}}
               hier a. G. und aber auch als Regisseur (also offenbar für die Saison). Schreiben Sie
               ihm also \uline{er} möge \uline{mich}
               aufsuchen (motiviren Sie das irgendwie, da es mir nicht passt zu ihm zu gehen) sagen
                  {\pb}Sie was von Bewunderung für
               ihn; in \textcolor{pink}{Wien}{}\ledrightnote{\textcolor{pink}{Wien}} gesehen etc, – ich Ihre Intentionen
               kennen u. s. w. Vielleicht geht es für \uline{Juli} einen Abend mit Ihren Sachen zu geben z. B.\pend
           \leftskip=3em{}\pstart
           \noindent{}\textcolor{green}{Episode}{}\ledrightnote{\textcolor{green}{Episode}}\pend
           \leftskip=0em{}\leftskip=3em{}\pstart
           \textcolor{green}{Abschiedssouper}{}\ledrightnote{\textcolor{green}{Abschiedssouper}}\pend
           \leftskip=0em{}\leftskip=3em{}\pstart
           \textcolor{green}{Hochzeitsmorgen}{}\ledrightnote{\textcolor{green}{Anatols Hochzeitsmorgen}}\pend
           \leftskip=0em{}\pstart
           \noindent{}Ko{\geminationm}en Sie bald, Grüße an alle.\pend
           \pstart
           Herzlichst{\\[\baselineskip]}\spacefill\mbox{Richard}\pend
           \leftskip=0em{}\pstart
           \noindent{}\label{T_L00223_1v}\edtext{Ich bin i{\geminationm}er gegen 2 Uhr zu Hause (wegen \textcolor{blue}{Jarno}{}\ledrightnote{\textcolor{blue}{Josef Jarno}})}{\lemma{\textnormal{\emph{Ich … Jarno)}}}\Cendnote{\textnormal{zwischen den Zeilen}}}\label{T_L00223_1h}\pend
           \pstart
           \label{K_L00223_1v}\edtext{\textcolor{blue}{Tartaglia}{}\ledrightnote{→\textcolor{blue}{Benedikt Felix}}}{\lemma{\textnormal{\emph{Tartaglia}}}\Cendnote{\textnormal{womöglich \textcolor{blue}{Benedikt Felix}, der in der abgelaufenen Theatersaison in
                        \emph{\textcolor{green}{Signor Formica}} in der Rolle des Tartaglia
                     aufgetreten war.}}}\label{K_L00223_1h} schrieb ich gestern.\pend
           \endnumbering\briefempfaengerindex{Schnitzler, Arthur@\textsc{Schnitzler, Arthur}!zzzBeer-Hofmann, Richard@\emph{von Richard Beer-Hofmann}!1893-06-201@{{[}vor dem 22. 6. 1893?{]}}|)be}\mylabel{h}  \normalsize

\doendnotes{C}
\bigskip
\vfill

\clearpage

\footnotesize

\lohead{\textsc{register}}

% Definiere theindex-Environment komplett neu ohne reledmac
\makeatletter
\renewenvironment{theindex}{%
  \section*{\indexname}%
  \setlength{\parindent}{0pt}%
  \setlength{\parskip}{0pt plus 0.3pt}%
  \let\item\@idxitem
}{%
  \clearpage
}
\makeatother

\IfFileExists{\jobname-pw.ind}{\input{\jobname-pw.ind}}{}

\end{document}

      