%% latex-korrekturansicht-vorspann.tex
%% Vorspann für die Korrekturansicht.
%% Lädt die gemeinsame Datei latex-vorspann.tex mit gesetztem Schalter.

\newif\ifkorrekturansicht
\korrekturansichttrue

\input{../tex-inputs/latex-vorspann}


               \section[Eduard Michael Kafka an Arthur Schnitzler, 24. 2. 1893]{ Eduard Michael Kafka an Arthur Schnitzler, 24. 2. 1893}\nopagebreak\mylabel{v}\rehead{ }\normalsize\beginnumbering\briefempfaengerindex{Schnitzler, Arthur@\textsc{Schnitzler, Arthur}!zzzKafka, Eduard Michael@\emph{von Eduard Michael Kafka}!1893-02-241@{24. 2. 1893}|(be} \toendnotes[C]{\smallbreak\pagebreak[2]} \Standort{DLA, A:Schnitzler, HS.NZ85.1.3604.}
\physDesc{Brief, 2 Blätter, 6 Seiten
\newline{}Handschrift: schwarze Tinte, deutsche Kurrent
\newline{}Schnitzler: mit rotem Buntstift mehrere Unterstreichungen }\toendnotes[C]{\smallbreak}\pstart
           \raggedleft{}{\pb}24/II 93.{\\}\textcolor{pink}{\textsc{Breslau}}{}\ledrightnote{\textcolor{pink}{Breslau}},{\\}\textcolor{pink}{\textsc{Hotel Galisch}}{}\ledrightnote{\textcolor{pink}{Hotel Galisch}}.\pend
           \pstart{}Lieber Schnitzler,\pend\pstart
           bitte, ſchreiben Sie mir freundlichſt, was \textcolor{blue}{Fels}{}\ledrightnote{\textcolor{blue}{Friedrich Michael Fels}}
               macht. Iſt er wirklich in \textcolor{pink}{Meran}{}\ledrightnote{\textcolor{pink}{Meran}}, wie \textcolor{blue}{\textsc{Bahr}}{}\ledrightnote{\textcolor{blue}{Hermann Bahr}} mir erzählte. Ich möchte \substVorne{}\textsuperscript{I}\substDazwischen{}i\substHinten{}hn gerne, wenn’s geht, in den nächſten Tagen beſuchen.\pend
           \pstart
           Ich traf \textcolor{blue}{\textsc{Bahr}}{}\ledrightnote{\textcolor{blue}{Hermann Bahr}} in \textcolor{pink}{\textsc{Berlin}}{}\ledrightnote{\textcolor{pink}{Berlin}}, vor einigen Tagen bei der \textsc{»\textcolor{green}{Gaea}{}\ledrightnote{\textcolor{green}{Gaea}}«vorlesung}. \textsc{\textcolor{blue}{Berti Goldschmidt}{}\ledrightnote{\textcolor{blue}{Adalbert von Goldschmidt}}} hat dort einen ganz koloſſalen Erfolg damit gehabt. \textcolor{blue}{\textsc{Reicher}}{}\ledrightnote{\textcolor{blue}{Emanuel Reicher}} las aber auch mit einer Meiſterſchaft, die sich in Worten nicht aus{\pb}drücken läßt: er bot eine unglaubliche,
               unübertreffliche Leiſtung, die ihm auf der ganzen Welt keiner nachmachen kann.\pend
           \pstart
           Ich sprach in \textcolor{pink}{\textsc{Berlin}}{}\ledrightnote{\textcolor{pink}{Berlin}} mit \textcolor{blue}{\textsc{Rittner}}{}\ledrightnote{\textcolor{blue}{Rudolf Rittner}} über die \textsc{\textcolor{green}{Anatol}{}\ledrightnote{\textcolor{green}{Anatol}}}ſachen. Bitte, ſenden Sie ein Ex. an ihn, \textcolor{pink}{O.
                     Schillingſtr. 14\textsubscript{II.}}{}\ledrightnote{\textcolor{pink}{Schillingstraße}}, – er wird \label{T_L00182_1v}\edtext{ſich ſicher}{\lemma{\textnormal{\emph{ſich ſicher}}}\Cendnote{\textnormal{durch Linien umgestellt von »ſicher
                     ſich«}}}\label{T_L00182_1h} für die Sachen einſetzen, wenn Sie ihn in einem lieben
               Brief überdies noch recht ſchön darum bitten.\pend
           \pstart
           Auch an \textcolor{blue}{\textsc{Jarno}}{}\ledrightnote{\textcolor{blue}{Josef Jarno}}, bitte, ſchreiben Sie; die beiden jungen Leute können Ihnen {\pb}ganz außerordentlich viel nutzen.\pend
           \pstart
           Ich bin jetzt mit \textcolor{blue}{\textsc{Reicher}}{}\ledrightnote{\textcolor{blue}{Emanuel Reicher}} für ein paar Tage nach \textcolor{pink}{\textsc{Breslau}}{}\ledrightnote{\textcolor{pink}{Breslau}} gefahren: er ſpielt morgen hier den \introOben{}König im\introOben{}{ }\textcolor{green}{\textsc{Talisman}}{}\ledrightnote{\textcolor{green}{Der Talisman. Dramatisches Märchen}} zum erſtenmale: ich bin ſehr geſpannt, was er damit machen wird.\pend
           \pstart
           An’s \textcolor{brown}{Magazin}{}\ledrightnote{\textcolor{brown}{Magazin für die Literatur des Auslandes}} würde ich Ihnen raten, doch einmal ein
                  \textsc{Manuscript} zu ſenden: ich höre doch von verſchiedenen
               Seiten, Sie hätten eine ſo hübſche \textcolor{green}{Novelle}{}\ledrightnote{→\textcolor{green}{Sterben. Novelle}} geſchrieben. Auch dem {\pb}\textcolor{brown}{\textsc{Berliner Tagblatt}}{}\ledrightnote{\textcolor{brown}{Berliner Tageblatt}}, wo Sie viele Freunde haben, in erſter Linie \textsc{D\textsuperscript{r}{ }\textcolor{blue}{Levysohn}{}\ledrightnote{\textcolor{blue}{Arthur Levysohn}}}{ }ſelbſt, u \textsc{\textcolor{blue}{Neumann Hofer}{}\ledrightnote{\textcolor{blue}{Gilbert Otto Neumann-Hofer}}}, der Sie ſehr ſchätzt, möchte ich doch an Ihrer Stelle einmal eine kleine
               Skizze ſenden.\pend
           \pstart
           Was iſt denn mit Ihrem neuen \textcolor{green}{Stück}{}\ledrightnote{→\textcolor{green}{Familie}}? Bitte, ſchreiben Sie mir ausführlich über
               dasſelbe. – Sie wiſſen, Sie haben einen aufrichtigen, guten Freund in mir: vielleicht
               kann ich Ihnen irgendwie behilflich ſein: ich bin ja jetzt \textsc{Weltvagabund} im großen Stil, heut da, morgen dort, u. überall doch nur
               gerade in \uline{den} Kreiſen, die Sie brauchen. Alſo!\pend
           \pstart
           Herzlichſt Ihr{\\[\baselineskip]}\spacefill\mbox{Kafka}\pend
           \leftskip=0em{}\pstart
           \noindent{}{\pb}\textsc{P.S.}\pend
           \pstart
           Jetzt habe ich richtig gerade an das vergeſſen, \substVorne{}\textsuperscript{warum}\substDazwischen{}deſſentwegen\substHinten{} ich Ihnen eigentlich ſchreiben wollte.\pend
           \pstart
           \textcolor{blue}{\textsc{Reicher}}{}\ledrightnote{\textcolor{blue}{Emanuel Reicher}} las geſtern bei einer \textsc{Soiree} hier, welcher ich
                  gleichfalls beiwohnte, Ihre \textcolor{green}{Frage an das
                     Schickſal}{}\ledrightnote{\textcolor{green}{Die Frage an das Schicksal}}. Mit richtigem Beifall. Und natürlich in brillanter Weiſe. \textcolor{blue}{\textsc{Reicher}}{}\ledrightnote{\textcolor{blue}{Emanuel Reicher}} iſt unermüdlich für Ihren Ruhm thätig. Sie ſollten ihm doch wieder mal
                  ſchreiben. {\pb}Daſs er Ihnen nicht i{\geminationm}er antwortet, daraus dürfen Sie sich nichts machen:
                  er hat ja wirklich ſo haarſträubend viel zu thun.\pend
           \pstart
           Grüßen Sie mir doch freundlichſt unſren lieben \textsc{\textcolor{blue}{Loris}{}\ledrightnote{\textcolor{blue}{Hugo von Hofmannsthal}}} u. die »anderen«. Hat noch i{\geminationm}er keiner Luſt,
                  ſein Bündel zu ſchnüren u. nach \textcolor{pink}{Berlin}{}\ledrightnote{\textcolor{pink}{Berlin}} zu
                  wandern?\pend
           \pstart
           Wenn ich nur ſchon wüßte, wohin ich von hier hinreiſen ſoll! Nach \textcolor{pink}{Hamburg}{}\ledrightnote{\textcolor{pink}{Hamburg}} oder nach \textcolor{pink}{München}{}\ledrightnote{\textcolor{pink}{München}}? Oder ſoll ich zu \textcolor{blue}{Holländer}{}\ledrightnote{\textcolor{blue}{Felix Hollaender}},
                  der Sie beſtens \label{T_L00182_2v}\edtext{grüßen läßt}{\lemma{\textnormal{\emph{grüßen läßt}}}\Cendnote{\textnormal{weiter am linken Rand}}}\label{T_L00182_2h}, nach \textcolor{pink}{Schreiberhau}{}\ledrightnote{\textcolor{pink}{Szklarska Poręba}}? Bis zum 15. März darf
                  ich mich goldener Freiheit freuen!\pend
           \pstart
           \spacefill\mbox{EMKafka.}\pend
           \pstart
           \label{T_L00182_3v}\edtext{Briefe treffen mich am beſten
                  jeweilig durch das \textsc{literarische} Auskunftsbureau \textsc{\textcolor{brown}{Clemens Freyer}{}\ledrightnote{\textcolor{brown}{Literarisches Bureau Clemens Freyer}}, \textcolor{pink}{Berlin, Wilhelmſtr 94/96}{}\ledrightnote{\textcolor{pink}{Wilhelmstraße}}}, das mir alles nachſendet.}{\lemma{\textnormal{\emph{Briefe … nachſendet.}}}\Cendnote{\textnormal{auf
                     dem ersten Blatt über Anrede und Datum eingefügt}}}\label{T_L00182_3h}\pend
           \endnumbering\briefempfaengerindex{Schnitzler, Arthur@\textsc{Schnitzler, Arthur}!zzzKafka, Eduard Michael@\emph{von Eduard Michael Kafka}!1893-02-241@{24. 2. 1893}|)be}\mylabel{h}  \normalsize

\doendnotes{C}
\bigskip
\vfill

\clearpage

\footnotesize

\lohead{\textsc{register}}

% Definiere theindex-Environment komplett neu ohne reledmac
\makeatletter
\renewenvironment{theindex}{%
  \section*{\indexname}%
  \setlength{\parindent}{0pt}%
  \setlength{\parskip}{0pt plus 0.3pt}%
  \let\item\@idxitem
}{%
  \clearpage
}
\makeatother

\IfFileExists{\jobname-pw.ind}{\input{\jobname-pw.ind}}{}

\end{document}

      