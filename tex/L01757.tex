%% latex-korrekturansicht-vorspann.tex
%% Vorspann für die Korrekturansicht.
%% Lädt die gemeinsame Datei latex-vorspann.tex mit gesetztem Schalter.

\newif\ifkorrekturansicht
\korrekturansichttrue

\input{../tex-inputs/latex-vorspann}


               \section[Arthur Schnitzler an Albert Ehrenstein, 24. 1. 1908]{ Arthur Schnitzler an Albert Ehrenstein, 24. 1. 1908}\nopagebreak\mylabel{v}\rehead{ }\normalsize\beginnumbering\briefempfaengerindex{Ehrenstein, Albert@\textsc{Ehrenstein, Albert}!zzzSchnitzler, Arthur@\emph{von Arthur Schnitzler}!1908-01-241@{24. 1. 1908}|(be} \toendnotes[C]{\smallbreak\pagebreak[2]} \Standort{Jerusalem, The National Library of Israel, ARC. Ms. Var. 306 1 118.}
\physDesc{Briefkarte
\newline{}Handschrift: blaue Tinte, deutsche Kurrent\newline{}Ordnung: mit Bleistift von unbekannter Hand nummeriert:
                                 »5« }\toendnotes[C]{\smallbreak}\pstart
           \noindent{}{\pb}\textcolor{gray}{\textbf{Dr. Arthur Schnitzler}}\hfill 24/1 908\pend
           \pstart
           \textcolor{gray}{\textbf{\textcolor{pink}{Wien XVIII. Spoettelgasse 7}{}\ledrightnote{\textcolor{pink}{Edmund-Weiß-Gasse}}.}}\pend
           \pstart
           lieber Herr Ehrenſtein, Ihr Brief hat mich ſehr erfreut u ich danke
               Ihnen herzlich für Ihren Glückwunſch. Vielleicht haben Sie in den nächſten Wochen
               einmal Zeit, mir perſönlich von Ihren Lüſten und \label{T_L01757_1v}\edtext{Trachten}{\lemma{\textnormal{\emph{Trachten}}}\Cendnote{\textnormal{das Wort
                  gestrichen und darüber erneut geschrieben}}}\label{T_L01757_1h} Kunde zu geben, ich würde mir
               gern beſtätigen {\pb}laſſen, was Ihr Brief mich wünſchen läßt, daſs
               Sie auf gutem Wege ſind (nicht nur weil Ihnen mein erſtes \textcolor{green}{Capitel}{}\ledrightnote{→\textcolor{green}{Der Weg ins Freie. Roman}} gut gefällt.)\pend
           \pstart
           Verbindlichſten Gruß von{\\[\baselineskip]} Ihrem{\\[\baselineskip]}\spacefill\mbox{A. S.}\pend
           \leftskip=0em{}\endnumbering\briefempfaengerindex{Ehrenstein, Albert@\textsc{Ehrenstein, Albert}!zzzSchnitzler, Arthur@\emph{von Arthur Schnitzler}!1908-01-241@{24. 1. 1908}|)be}\mylabel{h}  \normalsize

\doendnotes{C}
\bigskip
\vfill

\clearpage

\footnotesize

\lohead{\textsc{register}}

% Definiere theindex-Environment komplett neu ohne reledmac
\makeatletter
\renewenvironment{theindex}{%
  \section*{\indexname}%
  \setlength{\parindent}{0pt}%
  \setlength{\parskip}{0pt plus 0.3pt}%
  \let\item\@idxitem
}{%
  \clearpage
}
\makeatother

\IfFileExists{\jobname-pw.ind}{\input{\jobname-pw.ind}}{}

\end{document}

      