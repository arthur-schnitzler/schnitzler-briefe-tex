%% latex-korrekturansicht-vorspann.tex
%% Vorspann für die Korrekturansicht.
%% Lädt die gemeinsame Datei latex-vorspann.tex mit gesetztem Schalter.

\newif\ifkorrekturansicht
\korrekturansichttrue

\input{../tex-inputs/latex-vorspann}


               \section[Richard Beer-Hofmann an Arthur Schnitzler, 16. 8. 1909]{ Richard Beer-Hofmann an Arthur Schnitzler, 16. 8. 1909}\nopagebreak\mylabel{v}\rehead{ }\normalsize\beginnumbering\briefempfaengerindex{Schnitzler, Arthur@\textsc{Schnitzler, Arthur}!zzzBeer-Hofmann, Richard@\emph{von Richard Beer-Hofmann}!1909-08-161@{16. 8. 1909}|(be} \toendnotes[C]{\smallbreak\pagebreak[2]} \Standort{CUL, Schnitzler, B 8.}
\physDesc{Bildpostkarte
\newline{}Handschrift: schwarze Tinte, lateinische Kurrent\newline{}Versand: Stempel: »\nobreak{}\oindex{Venedig@\textbf{Venedig}, \emph{Besiedelter Ort (A.BSO)}|pwk}Venezia Ferrovia, 16 8–09, 2S\nobreak{}«.  
\newline{}Schnitzler: mit Bleistift beschriftet: »\textsc{Beerhof}« \newline{}Ordnung: mit Bleistift von unbekannter Hand nummeriert:
                              »222« }\toendnotes[C]{\smallbreak}\pstart{}{\pb}Herrn\pend{}\pstart{}D\textsuperscript{r} Arthur Schnitzler\pend{}\pstart{}\textcolor{pink}{Edlach}{}\ledrightnote{\textcolor{pink}{Edlach}}\pend{}\pstart{} bei \textcolor{pink}{Reichenau}{}\ledrightnote{\textcolor{pink}{Reichenau an der Rax}}\pend{}\pstart{}\textcolor{pink}{N. Ö.}{}\ledrightnote{\textcolor{pink}{Niederösterreich}}\pend{}\pstart{}\textcolor{pink}{Austria}{}\ledrightnote{\textcolor{pink}{Österreich}}\pend{}{\bigskip}\pstart
           \noindent{}\centering{}{\pb}\textcolor{gray}{\textbf{\textcolor{pink}{VENEZIA}{}\ledrightnote{\textcolor{pink}{Venedig}} – \textcolor{pink}{LIDO}{}\ledrightnote{\textcolor{pink}{Lido}} – \textcolor{pink}{Villa Elena}{}\ledrightnote{\textcolor{pink}{Villa Elena}}}}\pend
           \pstart
           \noindent{}\label{TLL01865_AS-1v}\edtext{mein Zi{\geminationm}er}{\lemma{\textnormal{\emph{mein Zier}}}\Cendnote{\textnormal{Ein Pfeil weist von einem
            Fenster links von der Eingangstüre auf den Text.}}}\label{TLL01865_AS-1h}\pend
           \pstart Herzliche Grüsse von \spacefill\mbox{Richard u. \textcolor{blue}{Paula}{}\ledrightnote{\textcolor{blue}{Paula Beer-Hofmann}}}\pend{}\pstart
           \noindent{}{\pb}Heiss aber schön.\pend
           \endnumbering\briefempfaengerindex{Schnitzler, Arthur@\textsc{Schnitzler, Arthur}!zzzBeer-Hofmann, Richard@\emph{von Richard Beer-Hofmann}!1909-08-161@{16. 8. 1909}|)be}\mylabel{h}  \normalsize

\doendnotes{C}
\bigskip
\vfill

\clearpage

\footnotesize

\lohead{\textsc{register}}

% Definiere theindex-Environment komplett neu ohne reledmac
\makeatletter
\renewenvironment{theindex}{%
  \section*{\indexname}%
  \setlength{\parindent}{0pt}%
  \setlength{\parskip}{0pt plus 0.3pt}%
  \let\item\@idxitem
}{%
  \clearpage
}
\makeatother

\IfFileExists{\jobname-pw.ind}{\input{\jobname-pw.ind}}{}

\end{document}

      