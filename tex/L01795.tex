%% latex-korrekturansicht-vorspann.tex
%% Vorspann für die Korrekturansicht.
%% Lädt die gemeinsame Datei latex-vorspann.tex mit gesetztem Schalter.

\newif\ifkorrekturansicht
\korrekturansichttrue

\input{../tex-inputs/latex-vorspann}


               \section[Olga Schnitzler an Paula Beer-Hofmann, {[}23. 10. 1908?{]}]{ Olga Schnitzler an Paula Beer-Hofmann, {[}23. 10. 1908?{]}}\nopagebreak\mylabel{v}\rehead{ }\normalsize\beginnumbering\briefempfaengerindex{Beer-Hofmann, Paula@\textsc{Beer-Hofmann, Paula}!zzzSchnitzler, Olga@\emph{von Olga Schnitzler}!1908-10-231@{{[}23. 10. 1908?{]}}|(be} \toendnotes[C]{\smallbreak\pagebreak[2]} \Standort{YCGL, MSS 31.}
\physDesc{Brief, 1 Blatt, 2 Seiten (die zweite Seite über den Mittelfalz geschrieben), Umschlag, mit rotem Wachssiegel verschlossen
\newline{}Handschrift: schwarze Tinte, lateinische Kurrent\newline{}Versand: ohne postalischen Übermittlungsvermerk }\toendnotes[C]{\smallbreak}\pstart{}{\pb}Herrn D\textsuperscript{r} Richard
                  Beer-Hofmann\pend{}{\bigskip}\pstart
           \noindent{}{\pb}\textcolor{gray}{\textbf{O. S.}}\pend
           \pstart
           Liebe Paula, eine grosse Bitte! ich glaube Sie haben mehrere
               Pelzjacken, ich soll morgen bis Sonntag auf den \textcolor{pink}{Semmering}{}\ledrightnote{\textcolor{pink}{Semmering}} – \textcolor{blue}{Brahm}{}\ledrightnote{\textcolor{blue}{Otto Brahm}} ist oben –
               mein \textcolor{blue}{Schneider}{}\ledrightnote{→\textcolor{blue}{?? [Schneider von Olga Schnitzler]}} hat meine
               Pelzjacke nicht fertig, würden Sie mir eine der Ihren auf \label{K_L01795_1v}\edtext{2 Tage}{\lemma{\textnormal{\emph{2 Tage}}}\Cendnote{\textnormal{Ein solcher
                  Kurzaufenthalt lässt sich nicht nachweisen. Mutmaßlich war er für den Aufenthalt
                     \textcolor{blue}{Brahm}s vom 22. 10. 1908 bis zum
                     27. 10. 1908 geplant? Eine alternative Datierung wäre der 9. 11. 1906, wenngleich es
                  damit das erste überlieferte Dokument nachbarschaftlicher Korrespondenz direkt
                  nach dem Einzug wäre.}}}\label{K_L01795_1h} leihen? nur wenn es Ihnen gar keine Umstände
               verursacht.\pend
           \pstart
           Seien Sie nicht bös, lassen Sie {\pb}von sich hören und
               seien Sie alle herzlich gegrüsst von Ihrer\pend
           \pstart \spacefill\mbox{Olga.}\pend{}\pstart
           Freitag.\pend
           \endnumbering\briefempfaengerindex{Beer-Hofmann, Paula@\textsc{Beer-Hofmann, Paula}!zzzSchnitzler, Olga@\emph{von Olga Schnitzler}!1908-10-231@{{[}23. 10. 1908?{]}}|)be}\mylabel{h}  \normalsize

\doendnotes{C}
\bigskip
\vfill

\clearpage

\footnotesize

\lohead{\textsc{register}}

% Definiere theindex-Environment komplett neu ohne reledmac
\makeatletter
\renewenvironment{theindex}{%
  \section*{\indexname}%
  \setlength{\parindent}{0pt}%
  \setlength{\parskip}{0pt plus 0.3pt}%
  \let\item\@idxitem
}{%
  \clearpage
}
\makeatother

\IfFileExists{\jobname-pw.ind}{\input{\jobname-pw.ind}}{}

\end{document}

      