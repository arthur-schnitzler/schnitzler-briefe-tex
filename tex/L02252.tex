%% latex-korrekturansicht-vorspann.tex
%% Vorspann für die Korrekturansicht.
%% Lädt die gemeinsame Datei latex-vorspann.tex mit gesetztem Schalter.

\newif\ifkorrekturansicht
\korrekturansichttrue

\input{../tex-inputs/latex-vorspann}


               \section[Arthur Schnitzler an Robert Adam, 21. 12. 1916]{ Arthur Schnitzler an Robert Adam, 21. 12. 1916}\nopagebreak\mylabel{v}\rehead{ }\normalsize\beginnumbering\briefempfaengerindex{Adam, Robert@\textsc{Adam, Robert}!zzzSchnitzler, Arthur@\emph{von Arthur Schnitzler}!1916-12-211@{21. 12. 1916}|(be} \toendnotes[C]{\smallbreak\pagebreak[2]} \Standort{DLA, 96.34.1/26.}
\physDesc{Postkarte
\newline{}Handschrift: Bleistift, deutsche Kurrent\newline{}Versand: 1) Stempel: »\nobreak{}\oindex{XVIII., Waehring@\textbf{XVIII., Währing}, \emph{Bezirk (A.BZK)}|pwk}18/1 Wien 111, 21. XII. 16, 7\textsuperscript{30}\nobreak{}«.  2) Stempel: »\nobreak{}Wien 82\nobreak{}«. 3) Stempel: »\nobreak{}\oindex{XII., Meidling@\textbf{XII., Meidling}, \emph{Bezirk (A.BZK)}|pwk}12/1 Wien 82, 21. XII. 16, 7\textsuperscript{30}\nobreak{}«. }\pstart{}{\pb}\textcolor{gray}{\textbf{Dr. Art}}{[}hur Schnitzler{]}\pend{}\pstart{}\textcolor{pink}{\textcolor{gray}{\textbf{Wien XVIII. S}}{[}ternwartestrasse 71{]}}{}\ledrightnote{\textcolor{pink}{Sternwartestraße}}\pend{}{\bigskip}\pstart{}\textsc{Herrn Dr. Robert}\pend{}\pstart{}\textsc{Adam Pollak}\pend{}\pstart{}\textcolor{pink}{Wien XII}{}\ledrightnote{\textcolor{pink}{XII., Meidling}}\pend{}\pstart{}\textcolor{pink}{\textsc{Meidlinger Hptstr}
                            58}{}\ledrightnote{\textcolor{pink}{Meidlinger Hauptstraße}}\pend{}{\bigskip}\pstart
           \raggedleft{}{\pb}21. 12. 16\pend
           \pstart
           verehrter Herr Doktor, nun werd ich leider wieder für morgen
                    Abend verhindert; – wär Ihnen Mittwoch der 27., abends gegen
                        7 recht?\pend
           \pstart
           Herzlichſt grüßen\textcolor{gray}{d} Ihr ergeb\textcolor{gray}{ner}{\\[\baselineskip]}\spacefill\mbox{A. S.}\pend
           \leftskip=0em{}\endnumbering\briefempfaengerindex{Adam, Robert@\textsc{Adam, Robert}!zzzSchnitzler, Arthur@\emph{von Arthur Schnitzler}!1916-12-211@{21. 12. 1916}|)be}\mylabel{h}  \normalsize

\doendnotes{C}
\bigskip
\vfill

\clearpage

\footnotesize

\lohead{\textsc{register}}

% Definiere theindex-Environment komplett neu ohne reledmac
\makeatletter
\renewenvironment{theindex}{%
  \section*{\indexname}%
  \setlength{\parindent}{0pt}%
  \setlength{\parskip}{0pt plus 0.3pt}%
  \let\item\@idxitem
}{%
  \clearpage
}
\makeatother

\IfFileExists{\jobname-pw.ind}{\input{\jobname-pw.ind}}{}

\end{document}

      