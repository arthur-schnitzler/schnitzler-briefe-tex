%% latex-korrekturansicht-vorspann.tex
%% Vorspann für die Korrekturansicht.
%% Lädt die gemeinsame Datei latex-vorspann.tex mit gesetztem Schalter.

\newif\ifkorrekturansicht
\korrekturansichttrue

\input{../tex-inputs/latex-vorspann}


               \section[Hermann Bahr an Arthur Schnitzler, 25. 8. 1918]{ Hermann Bahr an Arthur Schnitzler, 25. 8. 1918}\nopagebreak\mylabel{v}\rehead{ }\normalsize\beginnumbering\briefempfaengerindex{Schnitzler, Arthur@\textsc{Schnitzler, Arthur}!zzzBahr, Hermann@\emph{von Hermann Bahr}!1918-08-251@{25. 8. 1918}|(be} \toendnotes[C]{\smallbreak\pagebreak[2]} \Standort{CUL, Schnitzler, B 5b.}
\physDesc{Postkarte
\newline{}Handschrift: schwarze Tinte, deutsche Kurrent\newline{}Versand: Stempel: »\nobreak{}\oindex{Salzburg@\textbf{Salzburg}, \emph{Besiedelter Ort (A.BSO)}|pwk}Salzburg 2, 25. VIII. {[}1{]}8, 2\nobreak{}«.  
\newline{}Schnitzler: mit Bleistift Vermerk  »\uline{A}«, vermutlich für
            »Abzuschreiben«/»Abschrift« \newline{}Ordnung: mit Bleistift von unbekannter Hand nummeriert: »182« }\buchAbdrucke{\weitereDrucke{Hermann Bahr, Arthur Schnitzler: \emph{Briefwechsel, Aufzeichnungen, Dokumente (1891–1931)}. Hg. Kurt Ifkovits und Martin Anton Müller. Göttingen: \emph{Wallstein} 2018, S. 512.} }\toendnotes[C]{\smallbreak}\pstart{}{\pb}Abs. Hermann Bahr\pend{}{\bigskip}\pstart{}Herrn\pend{}\pstart{}D\textsuperscript{r} Arthur Schnitzler\pend{}\pstart{}\textcolor{pink}{Wien XVIII}{}\ledrightnote{\textcolor{pink}{XVIII., Währing}}\pend{}\pstart{}\textcolor{pink}{Sternwarteſtr 71}{}\ledrightnote{\textcolor{pink}{Sternwartestraße}}\pend{}{\bigskip}\pstart
           \raggedleft{}{\pb}25. 8. 18\pend
           \pstart
           Herzlichſten Dank, lieber Arthur, für Deinen lieben Brief – Frau \textcolor{blue}{Kainz}{}\ledrightnote{\textcolor{blue}{Margarethe Kainz}} verhieß \textcolor{blue}{uns}{}\ledrightnote{→\textcolor{blue}{Anna Bahr-Mildenburg}} immer \textcolor{blue}{Euren}{}\ledrightnote{→\textcolor{blue}{Olga Schnitzler}} erſehnten Beſuch und \textcolor{blue}{wir}{}\ledrightnote{→\textcolor{blue}{Anna Bahr-Mildenburg}} warteten den ganzen Sommer auf \textcolor{blue}{Euch}{}\ledrightnote{→\textcolor{blue}{Olga Schnitzler}}, leider vergeblich. So bald ich in \textcolor{pink}{Wien}{}\ledrightnote{\textcolor{pink}{Wien}} bin, melde ich mich bei Dir, um gleich in den erſten Tagen
               einmal zu Dir zu kommen. Bis dahin (wo wir dann auch über Deinen \textcolor{blue}{Muſiker}{}\ledrightnote{→\textcolor{blue}{Arthur Johannes Scholz}}{ }ſprechen) mit den herzlichſten Grüßen von \textcolor{blue}{uns Beiden}{}\ledrightnote{→\textcolor{blue}{Anna Bahr-Mildenburg}} an Dich und Deine
               liebe \textcolor{blue}{Frau}{}\ledrightnote{→\textcolor{blue}{Olga Schnitzler}}\pend
           \pstart
           Dein{\\[\baselineskip]}alter{\\[\baselineskip]}\spacefill\mbox{H.}\pend
           \leftskip=0em{}\endnumbering\briefempfaengerindex{Schnitzler, Arthur@\textsc{Schnitzler, Arthur}!zzzBahr, Hermann@\emph{von Hermann Bahr}!1918-08-251@{25. 8. 1918}|)be}\mylabel{h}  \normalsize

\doendnotes{C}
\bigskip
\vfill

\clearpage

\footnotesize

\lohead{\textsc{register}}

% Definiere theindex-Environment komplett neu ohne reledmac
\makeatletter
\renewenvironment{theindex}{%
  \section*{\indexname}%
  \setlength{\parindent}{0pt}%
  \setlength{\parskip}{0pt plus 0.3pt}%
  \let\item\@idxitem
}{%
  \clearpage
}
\makeatother

\IfFileExists{\jobname-pw.ind}{\input{\jobname-pw.ind}}{}

\end{document}

      