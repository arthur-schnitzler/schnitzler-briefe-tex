%% latex-korrekturansicht-vorspann.tex
%% Vorspann für die Korrekturansicht.
%% Lädt die gemeinsame Datei latex-vorspann.tex mit gesetztem Schalter.

\newif\ifkorrekturansicht
\korrekturansichttrue

\input{../tex-inputs/latex-vorspann}


               \section[Paul Goldmann an Arthur Schnitzler, 12. 12. {[}1891{]}]{ Paul Goldmann an Arthur Schnitzler, 12. 12. {[}1891{]}}\nopagebreak\mylabel{v}\rehead{ }\normalsize\beginnumbering\briefempfaengerindex{Schnitzler, Arthur@\textsc{Schnitzler, Arthur}!zzzGoldmann, Paul@\emph{von Paul Goldmann}!1891-12-122@{12. 12. {[}1891{]}}|(be} \toendnotes[C]{\smallbreak\pagebreak[2]} \Standort{DLA, A:Schnitzler, HS.NZ85.1.3162.}
\physDesc{Brief, 2 Blätter, 8 Seiten
\newline{}Handschrift: schwarze Tinte, deutsche Kurrent
\newline{}Schnitzler: 1) mit rotem Buntstift Vermerk »\textsc{(über \uline{\textcolor{green}{Märchen}}}« 2) mit Bleistift die Jahreszahl ergänzt »91«}\toendnotes[C]{\smallbreak}\pstart
           \raggedleft{}{\pb}\textsc{\textcolor{pink}{Paris}{}\ledrightnote{\textcolor{pink}{Paris}}}, 12. December.\pend
           \pstart\center{}Mein lieber Arthur!\pend\pstart
           Bei der ungeheuren Überbürdung, die gleich noch ehe ich den eigentlichen Dienſt
               übernommen, auf mich gefallen iſt, muß ich kurz ſein und kann keine Form für meine
               Anſicht ſuchen. Alſo folgendes: Der erſte \textcolor{green}{Act}{}\ledrightnote{→\textcolor{green}{Das Märchen. Schauspiel in drei Aufzügen}} iſt ſchlankweg entzückend, gehört zu den beſten erſten
               Acten, die ich kenne, ſprüht von Geiſt und Leben, enthält prachtvolle dramatiſche
               Steigerungen und einen \strikeout{E} erbeben machenden Schluß,
               iſt meiſterhaft in der Bewältigung der Perſonenmehrheiten, vergnüglich in der
               Entwerfung der Phyſiognomien, edel und neu in den Gedanken. Ich ſtelle ihn ruhig
               einem \textsc{\textcolor{blue}{Augier}{}\ledrightnote{\textcolor{blue}{Émile Augier}}} zur Seite. Äußerlich habe ich einzuwenden, daß während der Hauptdialoge auf der
               Bühne Clavier geſpielt wird, was ich für einen Mangel an ſceniſcher Geſchicklichkeit
               halte. Zweiter \textcolor{green}{Act}{}\ledrightnote{→\textcolor{green}{Das Märchen. Schauspiel in drei Aufzügen}}: Beginn
               gut; erſtes Geſpräch zwiſchen \textcolor{green}{Fedor}{}\ledrightnote{→\textcolor{green}{Das Märchen. Schauspiel in drei Aufzügen}} und \textcolor{green}{Leo}{}\ledrightnote{→\textcolor{green}{Das Märchen. Schauspiel in drei Aufzügen}} gut,
               desgleichen erſtes Geſpräch zwiſchen \textcolor{green}{Fedor}{}\ledrightnote{→\textcolor{green}{Das Märchen. Schauspiel in drei Aufzügen}} und \textcolor{green}{Fanny}{}\ledrightnote{→\textcolor{green}{Das Märchen. Schauspiel in drei Aufzügen}}, {\pb}Auftreten \textsc{Fr.
                  Wittes} guter dramatiſcher \label{K_L02674-6v}\edtext{\textsc{Truc}}{\lemma{\textnormal{\emph{Truc}}}\Cendnote{\textnormal{französisch: Kniff, Trick}}}\label{K_L02674-6h}. \textsc{Fr. \textcolor{green}{Witte}{}\ledrightnote{→\textcolor{green}{Das Märchen. Schauspiel in drei Aufzügen}}}{ }ſelbſt{[},{]} verſtändlich für Dich,
               mich und die gewiſſen drei oder vier Andern; für das große Publicum zu ſehr im
               Viertelprofil; der Durchschnittszuſchauer weiß nicht, was er daraus \strikeout{\textcolor{gray}{×}\-\textcolor{gray}{×}} machen ſoll. Aber bei den ſchönen geiſtreichen Sachen, die der Dialog enthält,
               geht die Scene vielleicht durch; nur kommen mir die Pointen zu gehäuft vor. \textsc{\textcolor{blue}{Zola}{}\ledrightnote{\textcolor{blue}{Émile Zola}}} ſprach mir in \textcolor{pink}{Brüſſel}{}\ledrightnote{\textcolor{pink}{Brüssel}} von dieſen mit
               Pointen vollgeſtopften Scenen, deren dramatiſche Wirkung er bezweifelt: »\label{K_L02674-1v}\edtext{\textsc{\begin{otherlanguage}{french}On doit avoir le temps de se moucher\end{otherlanguage}}}{\lemma{\textnormal{\emph{On … moucher}}}\Cendnote{\textnormal{französisch: man muss Zeit haben, um
                  sich die Nase zu putzen}}}\label{K_L02674-1h}«, ſagte er. Letzte Scene zwiſchen \textcolor{green}{Fedor}{}\ledrightnote{→\textcolor{green}{Das Märchen. Schauspiel in drei Aufzügen}} und \textcolor{green}{Fanny}{}\ledrightnote{→\textcolor{green}{Das Märchen. Schauspiel in drei Aufzügen}}. Da beginnt das \label{K_L02674-2v}\edtext{\textsc{\begin{otherlanguage}{french}embrouillement\end{otherlanguage}}}{\lemma{\textnormal{\emph{embrouillement}}}\Cendnote{\textnormal{französisch: Verwirrung,
                  Verworrenheit}}}\label{K_L02674-2h}. Der Zuſchauer kennt ſich nicht mehr aus. Das Geſicht des \textcolor{green}{Stückes}{}\ledrightnote{→\textcolor{green}{Das Märchen. Schauspiel in drei Aufzügen}} wechſelt plötzlich;
               ſtatt der \label{K_L02674-11v}\edtext{Gefallenen}{\lemma{\textnormal{\emph{Gefallenen}}}\Cendnote{\textnormal{Gemeint ist damit die Figur der \textcolor{green}{Fanny}, die bereits vor ihrer
                  Beziehung zu \textcolor{green}{Fedor} sexuell
                  aktiv war.}}}\label{K_L02674-11h} tritt auf einmal der \label{K_L02674-66v}\edtext{junge Mann, die Analyſe, die Seelenzerfleiſchung}{\lemma{\textnormal{\emph{junge … Seelenzerfleiſchung}}}\Cendnote{\textnormal{\textcolor{green}{Fedor} gelingt es nicht, das
                  sexuelle Vorleben von \textcolor{green}{Fanny}
                  zu akzeptieren, trotzdem er mit dem Verstand die Idealisierung der
                  Jungfräulichkeit als »Märchen« abtut.}}}\label{K_L02674-66h} in den {\pb}Vordergrund. Es kommen Motive in’s Spiel, mit einem
               Ruck, unvermittelt, welche zu fein und zu atomiſch zerfaſert ſind, als daß das
               Publicum mit ſeinen groben Werktagshänden ihnen nachtaſten könnte. Das iſt
               pſychologiſch, aber nicht mehr dramatiſch. Und wenn die Scene doch einen Erfolg hat,
               ſo kann es nur dadurch geſchehen, daß Meiſter Publicus ſich das auf ſeine Weiſe
               zurechtlegt und, von all’ \strikeout{de\textcolor{gray}{m}} den pſychologiſchen \strikeout{h\textcolor{gray}{oc}hf\textcolor{gray}{×}\-\textcolor{gray}{×}\-\textcolor{gray}{×}\-\textcolor{gray}{×}\-\textcolor{gray}{×}} Tendenzen abſtrahierend, nur den rohen Kern herausnimmt, der darin ſteckt: er
               will das Mädel nicht, aber das Mädel läßt nicht nach, und am End’ fallen ſie ſich
               doch in die Arme. Dritter \textcolor{green}{Act}{}\ledrightnote{→\textcolor{green}{Das Märchen. Schauspiel in drei Aufzügen}}.
               Der hätte ſein ſollen wie der erſte: Perſonenmehrheiten, feſtes Zuſammenhalten der
               Handlung und Steigerung \strikeout{der H} auf einen Punkt hin, wo
               die Entladung mit mächtigem Ruck erfolgt; und dann Vorhang. Der \label{K_L02674-88v}\edtext{Contract}{\lemma{\textnormal{\emph{Contract}}}\Cendnote{\textnormal{Ein Arbeitsvertrag, der \textcolor{green}{Fanny}, wenn sie ihn unterzeichnet, an ein Theater in \emph{\textcolor{green}{St. Petersburg}} engagiert und damit auch einen
                  Ausweg aus der Beziehung zu \textcolor{green}{Fedor} ermöglicht.}}}\label{K_L02674-88h}{ }{\pb}vortreffliche Idee. Aber am Schluß, nachdem man den
               ganzen \textcolor{green}{Act}{}\ledrightnote{→\textcolor{green}{Das Märchen. Schauspiel in drei Aufzügen}} mit all’ ſeinen
               Fäden auf den Contract hat hinlaufen geſehen. Der Aufzug fällt aber in lauter Dialoge
               auseinander, und die Handlungen ſind ſchichtenweis nebeneinander aufgeſtellt, ſtatt
               in einem Körper zuſammengeſchmolzen zu ſein. Dialog zwiſchen \textsc{\textcolor{green}{Wandel}{}\ledrightnote{→\textcolor{green}{Das Märchen. Schauspiel in drei Aufzügen}}} und \textsc{\textcolor{green}{Klara}{}\ledrightnote{→\textcolor{green}{Das Märchen. Schauspiel in drei Aufzügen}}} – ſehr ſchön an ſich, aber bringt aus der Stimmung, iſt zu lang und verläuft,
               ohne in der Haupthandlung ſeine Fortſetzung zu finden. Und ſo weiter. Stell’ Dir das
               auf der Scene vor: einen Act, einen Hauptact eines Dramas, wo Alles Wichtige, was
               vorgeht, in lauter »Beiſeite« ſtattfindet! Stell’ Dir vor, wie ein Act ſich ausnimmt,
               wo \strikeout{i\textcolor{gray}{m}} die Haupt\strikeout{\textcolor{gray}{h}}zahl der Perſonen immer im ſtummen Spiel im Hintergrunde oder auf der Seite
               ſteht, während vorn immer zwei paarweis {\pb}die
               Handlung machen. Und welche Aufgabe für den Hauptdarſteller, ſeine größten Scenen,
               ſeine Leidenſchaftsausbrüche »gedämpft« vorzubringen! Welch’ ungünſtiger Abgang!
               Statt nach einer starken Scene mit einem ſtarken Wort hinauszugehen, ſchleicht er
               ſich von hinnen, nachdem all’ ſeine dramatiſchen Feuer verloſchen! Starke und
               gewaltſame Mittel waren nöthig. Kein beiſeite, aus Furcht zu compromittiren, ſondern
               eben dieſes Compromittiren ſelbſt, ein wuchtiger Fauſtſchlag \strikeout{\textcolor{gray}{×}\-\textcolor{gray}{×}\-\textcolor{gray}{×}} in dieſes falſche \textsc{Milieu}, in dieſes Philiſtertum \textsc{à la \textcolor{green}{Wandel}{}\ledrightnote{→\textcolor{green}{Das Märchen. Schauspiel in drei Aufzügen}}} hinein. Mit Aufſchrei muß die ſchreckliche Wahrheit aus der Bruſt \strikeout{des}{ }\textcolor{green}{Fedor}{}\ledrightnote{→\textcolor{green}{Das Märchen. Schauspiel in drei Aufzügen}}s heraus, mit Aufſchrei
               muß das Mädchen die Vernichtung beantworten, Leidenſchaft gegen Leidenſchaft, zwei
               Flammen, die über dem Haupte des \textcolor{green}{Stückes}{}\ledrightnote{→\textcolor{green}{Das Märchen. Schauspiel in drei Aufzügen}} zuſammenſchlagen. Schwung und Kunſt im dritten \textcolor{green}{Acte}{}\ledrightnote{→\textcolor{green}{Das Märchen. Schauspiel in drei Aufzügen}}, aber {\pb}um Gotteswillen nur hier kein Grübeln, Quälen und Vertuſchen.\pend
           \pstart
           Mit einem Wort: ein fertiges \textcolor{green}{Stück}{}\ledrightnote{→\textcolor{green}{Das Märchen. Schauspiel in drei Aufzügen}} ist das nicht. Aber ich meine, Du haſt auch kein Recht, zu
               beanſpruchen, daß Dir ein fertiges Stück jetzt ſchon gelingt. Als Weg zum Ziele iſt
               es jedoch ein gewaltiger Schritt, als Talentbeweis ein glänzendes Ergebniß. Wer
               dieſen erſten \textcolor{green}{Act}{}\ledrightnote{→\textcolor{green}{Das Märchen. Schauspiel in drei Aufzügen}} geſchrieben,
               iſt ein Dramatiker von Gottes Gnaden; und wer \textsc{\textcolor{green}{Robert}{}\ledrightnote{→\textcolor{green}{Das Märchen. Schauspiel in drei Aufzügen}}} und \textsc{\textcolor{green}{Ninetten}{}\ledrightnote{→\textcolor{green}{Das Märchen. Schauspiel in drei Aufzügen}}} erdacht, iſt ein Dichter von goldenem Herzen. Als litterariſche \textcolor{green}{Arbeit}{}\ledrightnote{→\textcolor{green}{Das Märchen. Schauspiel in drei Aufzügen}} iſt »\textcolor{green}{Das Märchen}{}\ledrightnote{\textcolor{green}{Das Märchen. Schauspiel in drei Aufzügen}}« eine \textcolor{green}{Erſcheinung}{}\ledrightnote{→\textcolor{green}{Das Märchen. Schauspiel in drei Aufzügen}}, wie ſie in dem letzten Jahrzehnt in der deutſchen Litteratur ſo
               bemerkenswerth kaum noch da war und iſt mit \textsc{\textcolor{blue}{Sudermann}{}\ledrightnote{\textcolor{blue}{Hermann Sudermann}}} und \textsc{\textcolor{blue}{Hauptmann}{}\ledrightnote{\textcolor{blue}{Gerhart Hauptmann}}} zu nennen. Dramatiſch, unter dem {\pb}Geſichtspunkte der Aufführbarkeit ein \textcolor{green}{Unvollendetes}{}\ledrightnote{→\textcolor{green}{Das Märchen. Schauspiel in drei Aufzügen}}, das in Kürze Vollendetes verſpricht. Ich rathe
               Dir entſchieden ab, das »\textcolor{green}{Märchen}{}\ledrightnote{\textcolor{green}{Das Märchen. Schauspiel in drei Aufzügen}}« \label{K_L02674-3v}\edtext{aufführen}{\lemma{\textnormal{\emph{aufführen}}}\Cendnote{\textnormal{\emph{\textcolor{green}{Das Märchen}} wurde am 1. 12. 1893 am \emph{\textcolor{brown}{Deutschen Volkstheater}} in \textcolor{pink}{Wien}
                  uraufgeführt, mit einem von \textcolor{blue}{Schnitzler}
                  modifizierten Schluss.}}}\label{K_L02674-3h} zu laſſen; es gibt nur einen Weg für Dich:
               weiterſchreiben. Das thut weh; aber Du haſt noch keine Berechtigung, Dich auszuruhen;
               denke, ſeit wie kurzer Zeit Du erſt auf dem Wege biſt. Und der Erfolg beſteht für
               Leute wie Dich, deren Berufung außer Zweifel ſteht, nur in der Frage, ob ſie nicht zu
               früh bequem werden. Ein neues \textcolor{green}{Stück}{}\ledrightnote{→\textcolor{green}{Das Märchen. Schauspiel in drei Aufzügen}} alſo; in einem halben Jahre arbeiteſt Du vielleicht dann den dritten
                  \textcolor{green}{Akt}{}\ledrightnote{→\textcolor{green}{Das Märchen. Schauspiel in drei Aufzügen}} des »\textcolor{green}{Märchen}{}\ledrightnote{\textcolor{green}{Das Märchen. Schauspiel in drei Aufzügen}}s« um, und da haſt Du auch \strikeout{da\textcolor{gray}{rin}} damit einen dramatiſchen Erfolg \textsc{in petto}. Daß der
               Dialog von \textsc{A} bis \textsc{Z} voll iſt der
               entzückendſten Sachen habe ich \strikeout{\textcolor{gray}{×}} wohl ſchon geſagt. Kein einziger unter den \label{K_L02674-69v}\edtext{Jungdeutschen}{\lemma{\textnormal{\emph{Jungdeutschen}}}\Cendnote{\textnormal{hier als Synonym für deutschsprachige Autorinnen und Autoren am Beginn ihrer
                  Karriere}}}\label{K_L02674-69h} in \textcolor{pink}{Berlin}{}\ledrightnote{\textcolor{pink}{Berlin}} oder \textcolor{pink}{Wien}{}\ledrightnote{\textcolor{pink}{Wien}} iſt Dir das {\pb}nachzuthun imſtande. Wie hoch ſteht das »\textcolor{green}{Märchen}{}\ledrightnote{\textcolor{green}{Das Märchen. Schauspiel in drei Aufzügen}}« mit allen ſeinen Fehlern z. B. über \textsc{\textcolor{blue}{Herzl}{}\ledrightnote{\textcolor{blue}{Theodor Herzl}}}’s Sachen!{\dotsfour}\pend
           \pstart
           Im Vertrauen auf Deine Freundſchaft, mein lieber Arthur, habe ich Dir geſagt, was ich
               denke, ohne ein \label{K_L02674-4v}\edtext{\textsc{Jota}}{\lemma{\textnormal{\emph{Jota}}}\Cendnote{\textnormal{Redewendung: Ohne die kleinste
                  Abänderung. (»Jota« bezeichnet den kleinsten Buchstaben im
                  griechischen Alphabet.)}}}\label{K_L02674-4h} zu ändern. Es war unklug von mir, denn eine
               Bitterkeit wird bei Dir doch zurückbleiben. Ich habe Dir vielleicht noch nie ſo weh
               gethan. Aber ich mußte wohl. Freundespflicht! Wenn \uline{ich} Dir nicht die Wahrheit ſagen ſollte – wer \strikeout{da\textcolor{gray}{n}} denn ſonſt? Und ſo bin ich wieder einmal das Opfer meiner Pflicht geworden,
               umſomehr als ich ja, wie Du weißt, nicht zu den Leuten gehöre, welche über allen
               Nachtheilen der Pflichterfüllung ſich mit dem Bewußtſein begnügen, daß es eben doch
               die Pflicht war.\pend
           \pstart
           Grüß’ Dich Gott!{\\[\baselineskip]}Dein{\\[\baselineskip]}\spacefill\mbox{Paul Goldmann}\pend
           \leftskip=0em{}\pstart
           \noindent{}Bitte, ſchick’ mir ein paar Empfehlungen für \textcolor{pink}{Paris}{}\ledrightnote{\textcolor{pink}{Paris}}! – Grüße an \textsc{\textcolor{blue}{Richard}{}\ledrightnote{\textcolor{blue}{Richard Beer-Hofmann}}}, \textsc{\textcolor{blue}{Loris}{}\ledrightnote{\textcolor{blue}{Hugo von Hofmannsthal}}} und \textsc{\textcolor{blue}{Kapper}{}\ledrightnote{\textcolor{blue}{Friedrich Kapper}}}.\pend
           \endnumbering\briefempfaengerindex{Schnitzler, Arthur@\textsc{Schnitzler, Arthur}!zzzGoldmann, Paul@\emph{von Paul Goldmann}!1891-12-122@{12. 12. {[}1891{]}}|)be}\mylabel{h}\begin{anhang}\end{anhang}\normalsize

\doendnotes{C}
\bigskip
\vfill

\clearpage

\footnotesize

\lohead{\textsc{register}}

% Definiere theindex-Environment komplett neu ohne reledmac
\makeatletter
\renewenvironment{theindex}{%
  \section*{\indexname}%
  \setlength{\parindent}{0pt}%
  \setlength{\parskip}{0pt plus 0.3pt}%
  \let\item\@idxitem
}{%
  \clearpage
}
\makeatother

\IfFileExists{\jobname-pw.ind}{\input{\jobname-pw.ind}}{}

\end{document}

      