%% latex-korrekturansicht-vorspann.tex
%% Vorspann für die Korrekturansicht.
%% Lädt die gemeinsame Datei latex-vorspann.tex mit gesetztem Schalter.

\newif\ifkorrekturansicht
\korrekturansichttrue

\input{../tex-inputs/latex-vorspann}


               \section[Arthur Schnitzler an Stefan Großmann, 31. 5. 1926]{ Arthur Schnitzler an Stefan Großmann, 31. 5. 1926}\nopagebreak\mylabel{v}\rehead{ }\normalsize\beginnumbering\briefempfaengerindex{Grossmann, Stefan@\textsc{Großmann, Stefan}!zzzSchnitzler, Arthur@\emph{von Arthur Schnitzler}!1926-05-311@{31. 5. 1926}|(be} \toendnotes[C]{\smallbreak\pagebreak[2]} \Standort{DLA, A:Schnitzler, HS.NZ85.1.896.}
\physDesc{Brief, 1 Blatt, 1 Seite, maschineller Durchschlag
\newline{}Schreibmaschine
\newline{}Handschrift: roter Buntstift, lateinische Kurrent (\noindent{}»Grossmann« und »\textcolor{pink}{Berlin}« sowie »Urteiln\textcolor{gray}{ahme}«. Zahlreiche Unterstreichungen)}\toendnotes[C]{\smallbreak}\pstart
           \raggedleft{}{\pb}31. 5. 1926. \pend
           \pstart{}Verehrter Herr Grossmann.\pend\pstart
           Sie haben meine Zustimmung zu dem \textcolor{green}{Nachdruck}{}\ledrightnote{→\textcolor{green}{Bemerkungen}} der in der \textcolor{green}{Neuen Freien
                        Presse}{}\ledrightnote{\textcolor{green}{Neue Freie Presse}} zu Pfingsten veröffentlichten »\textcolor{green}{Bemerkungen}{}\ledrightnote{\textcolor{green}{Bemerkungen. (Aus dem noch unveröffentlichten „Buch der Sprüche und Bedenken“.)}}« nicht abgewartet, doch da ich in jedem Fall
                    bereit gewesen wäre Ihnen diese Zustimmung zu erteilen, so habe ich auch
                    nachträglich nichts einzuwenden. Höchst ärgerlich aber ist mir, dass das
                    vorletzte Aphorisma nur zur Hälfte abgedruckt und dadurch zu einer pretentiösen
                    Plattheit geworden ist. Offenbar ist die 3. Spalte des \textcolor{green}{Originaldruckes}{}\ledrightnote{→\textcolor{green}{Bemerkungen. (Aus dem noch unveröffentlichten „Buch der Sprüche und Bedenken“.)}} der \textcolor{green}{Neuen Freien Presse}{}\ledrightnote{\textcolor{green}{Neue Freie Presse}} dem Setzer in Verlust geraten und er
                    hat meine »Bemerkung« aus eigener Machtvollkommenheit durch Hinzufügung eines
                    Wortes zu Ende gedichtet. Sie lautet daher im »\textcolor{green}{Tagebuch}{}\ledrightnote{\textcolor{green}{Das Tage-Buch}}«: »Ob ein Mensch dich bestohlen, betrogen, verleumdet habe –
                    es könnte immer noch die Möglichkeit einer Versöhnung, ja selbst eines späteren
                    reinen Verhältnisses zwischen dir und ihm bestehen. Ja, wenn es sich praktisch
                    durchführen lässt –« (!!!)\pend
           \pstart
           In Wirklichkaut lautet die »Bemerkung{[}«{]} wie folgt:\pend
           \pstart
           »Ob ein Mensch dich betrogen, bestohlen, \strikeout{verlejde}
                    verleumdet habe, es könnte immer noch die Möglichkeit einer Versöhnung, ja
                    selbst eines späteren reinen Verhältnisses zwischen dir und ihm bestehen. Ja,
                    wenn es sich praktisch durchführen liesse –\strikeout{ä}:
                    selbst mit deinem Mörder könntest du dich nach geschehener Tat vielleicht
                    trefflich verstehen, am ehesten vielleicht mit ihm! Nur {\pb}zu einem Menschen, der nicht weiss, was er dir
                    getan hat, führt, selbst wenn du dieses Tun persönlich längst verschmerztest, in
                    aller Ewigkeit kein Weg zurück.«\pend
           \pstart
           (Es folgt dann noch ein Aphorisma, das dem Setzer selbstverständlich völlig
                    entgehen musste, da es auf der 3. Spalte stand.)\pend
           \pstart
           Ich bitte Sie sehr das Versehen \label{K_L02476_1v}\edtext{richtig zu stellen}{\lemma{\textnormal{\emph{richtig zu stellen}}}\Cendnote{\textnormal{Die
                        Richtigstellung erschien am 5. 6. 1926 (Jg. 7, H. 23,
                            S. 915) einschließlich des zusätzlichen von \textcolor{blue}{Schnitzler} vorgeschlagenen Aphorismus.}}}\label{K_L02476_1h} und meine
                        »\textcolor{green}{Bemerkung}{}\ledrightnote{\textcolor{green}{Bemerkungen. (Aus dem noch unveröffentlichten „Buch der Sprüche und Bedenken“.)}}« in Gänze dem Original gemäss
                    abdrucken zu wollen\pend
           \pstart
           Den Empfang des Nachdruckshonorars im Betrage von S. 85.– bestätige ich mit
                    bestem Dank und bin mit den verbindlichsten Grüssen\pend
           \pstart Ihr sehr ergebener\pend{}{\bigskip}\pstart
           \noindent{}Das letzte Aphorisma, wenn Sie es vielleicht noch nachträglich drucken
                        wollen, lautet:\pend
           \pstart
           »Es ist schon oft genug vorgekommen, dass ein Bösewicht aus Klugheit etwas
                        Gutes, aber noch nie, dass ein Dummkopf aus Güte etwas Kluges getan
                        hat.«\pend
           {\bigskip}\pstart
           \noindent{}Herrn Stefan Grossmann,{\\}Herausgeber des »\textcolor{brown}{Tagebuch}{}\ledrightnote{\textcolor{brown}{Das Tage-Buch}}«,{\\}\textcolor{pink}{Berlin}{}\ledrightnote{\textcolor{pink}{Berlin}}.\pend
           \endnumbering\briefempfaengerindex{Grossmann, Stefan@\textsc{Großmann, Stefan}!zzzSchnitzler, Arthur@\emph{von Arthur Schnitzler}!1926-05-311@{31. 5. 1926}|)be}\mylabel{h}  \normalsize

\doendnotes{C}
\bigskip
\vfill

\clearpage

\footnotesize

\lohead{\textsc{register}}

% Definiere theindex-Environment komplett neu ohne reledmac
\makeatletter
\renewenvironment{theindex}{%
  \section*{\indexname}%
  \setlength{\parindent}{0pt}%
  \setlength{\parskip}{0pt plus 0.3pt}%
  \let\item\@idxitem
}{%
  \clearpage
}
\makeatother

\IfFileExists{\jobname-pw.ind}{\input{\jobname-pw.ind}}{}

\end{document}

      