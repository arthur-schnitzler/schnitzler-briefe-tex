%% latex-korrekturansicht-vorspann.tex
%% Vorspann für die Korrekturansicht.
%% Lädt die gemeinsame Datei latex-vorspann.tex mit gesetztem Schalter.

\newif\ifkorrekturansicht
\korrekturansichttrue

\input{../tex-inputs/latex-vorspann}


               \section[Arthur Schnitzler an Richard Beer-Hofmann, 10. 8. 1901]{ Arthur Schnitzler an Richard Beer-Hofmann, 10. 8. 1901}\nopagebreak\mylabel{v}\rehead{ }\normalsize\beginnumbering\briefempfaengerindex{Beer-Hofmann, Richard@\textsc{Beer-Hofmann, Richard}!zzzSchnitzler, Arthur@\emph{von Arthur Schnitzler}!1901-08-101@{10. 8. 1901}|(be} \toendnotes[C]{\smallbreak\pagebreak[2]} \Standort{YCGL, MSS 31.}
\physDesc{Brief, 1 Blatt, 3 Seiten, Umschlag
\newline{}Handschrift: 1) Bleistift, deutsche Kurrent\hspace{1em}2) schwarze Tinte, deutsche Kurrent (\noindent{}Umschlag)\hspace{1em}\newline{}Versand: 1) Stempel: »\nobreak{}\oindex{Vahrn@\textbf{Vahrn}, \emph{Besiedelter Ort (A.BSO)}|pwk}Vahr\textcolor{gray}{n}, 10. 8. \textcolor{gray}{01}\nobreak{}«.  2) Stempel: »\nobreak{}\oindex{Poertschach@\textbf{Pörtschach}, \emph{https://www.geonames.org/ontologyP.PPL}|pwk}{\pb}Pörtschach am
                              See, 11 8 01\nobreak{}«. \newline{}Ordnung: mit Bleistift von unbekannter Hand datiert:
                                 »10. 8.« }\buchAbdrucke{\weitereDrucke{Arthur Schnitzler, Richard Beer-Hofmann: \emph{Briefwechsel 1891–1931}. Hg. Konstanze Fliedl. Wien, Zürich: \emph{Europaverlag} 1992, S. 154.} }\toendnotes[C]{\smallbreak}\pstart{}{\pb}Herrn \textsc{Dr. Richard
                     Beer-Hofmann}\pend{}\pstart{}\textsc{\textcolor{pink}{Pörtschach}{}\ledrightnote{\textcolor{pink}{Pörtschach}}}\pend{}\pstart{}am \textcolor{pink}{\textsc{Wörther}ſee}{}\ledrightnote{\textcolor{pink}{Wörthersee}}\pend{}\pstart{}\textsc{\textcolor{pink}{Villa Arnstein}{}\ledrightnote{\textcolor{pink}{Villa Arnstein}}.}\pend{}{\bigskip}\pstart
           \noindent{}{\pb}mein lieber Richard, am Montag fahren wir
               nach \textcolor{pink}{Bozen}{}\ledrightnote{\textcolor{pink}{Bozen}}, wo Rendez vous mit Paul, den ich
               neulich in \textsc{\textcolor{pink}{Welsberg}{}\ledrightnote{\textcolor{pink}{Welsberg-Taisten}}} ſprach. Da{\geminationn}{ }\textcolor{pink}{Trient}{}\ledrightnote{\textcolor{pink}{Trient}}. Freitag wollen wir in \textsc{\textcolor{pink}{Welsberg}{}\ledrightnote{\textcolor{pink}{Welsberg-Taisten}}} (\textcolor{pink}{Puſterthal}{}\ledrightnote{\textcolor{pink}{Pustertal}}) ſein, haben dort Zimmer genommen
                  (\textsc{Pens. \textcolor{pink}{Waldbrunn}{}\ledrightnote{\textcolor{pink}{Wildbad Waldbrunn}}}, entzückend gelegen, 1150 \textsc{meter}, 20 Minuten von der
               Bahn) wollen dort die letzten 14 Tage {\dots} ich mei{\pb}ne die letzten 14 Tage vor \textcolor{pink}{Wien}{}\ledrightnote{\textcolor{pink}{Wien}} (kümmert man ſich dort oben um meine »Meinung?«) verbringen. Es wäre
               wirklich hübſch von Ihnen, we{\geminationn} Sie auch dorthin kämen.
               Ich will auch arbeiten. Nebſtbei haben Sie’s ſo nah. – \pend
           \pstart
           We{\geminationn} Sie mir \uline{gleich}
               antworten, bitte Trient \textsc{postrest} (thun Sie das!).\pend
           \pstart
           {\pb}Leben Sie wohl und grüßen Sie Ihre \textcolor{blue}{Frau}{}\ledrightnote{→\textcolor{blue}{Paula Beer-Hofmann}} und Ihre
               \textcolor{blue}{Kinder}{}\ledrightnote{→\textcolor{blue}{Gabriel Beer-Hofmann}{\newline}→\textcolor{blue}{Gabriel Beer-Hofmann}{\newline}→\textcolor{blue}{Mirjam Beer-Hofmann}}, ſoweit sie es verſtehen.\pend
           \pstart
           Herzlichſt Ihr{\\[\baselineskip]}\spacefill\mbox{Arth.}\pend
           \leftskip=0em{}\pstart
           \textcolor{pink}{\textsc{Vahrn}}{}\ledrightnote{\textcolor{pink}{Vahrn}}, 10. 8. 901.\pend
           \endnumbering\briefempfaengerindex{Beer-Hofmann, Richard@\textsc{Beer-Hofmann, Richard}!zzzSchnitzler, Arthur@\emph{von Arthur Schnitzler}!1901-08-101@{10. 8. 1901}|)be}\mylabel{h}  \normalsize

\doendnotes{C}
\bigskip
\vfill

\clearpage

\footnotesize

\lohead{\textsc{register}}

% Definiere theindex-Environment komplett neu ohne reledmac
\makeatletter
\renewenvironment{theindex}{%
  \section*{\indexname}%
  \setlength{\parindent}{0pt}%
  \setlength{\parskip}{0pt plus 0.3pt}%
  \let\item\@idxitem
}{%
  \clearpage
}
\makeatother

\IfFileExists{\jobname-pw.ind}{\input{\jobname-pw.ind}}{}

\end{document}

      