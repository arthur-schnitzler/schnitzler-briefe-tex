%% latex-korrekturansicht-vorspann.tex
%% Vorspann für die Korrekturansicht.
%% Lädt die gemeinsame Datei latex-vorspann.tex mit gesetztem Schalter.

\newif\ifkorrekturansicht
\korrekturansichttrue

\input{../tex-inputs/latex-vorspann}


               \section[Paul Goldmann an Arthur und Olga Schnitzler, 12. 8. {[}1902?{]}]{ Paul Goldmann an Arthur und Olga Schnitzler,
               12. 8. {[}1902?{]}}\nopagebreak\mylabel{v}\rehead{ }\normalsize\beginnumbering\briefempfaengerindex{Schnitzler, Olga@\textsc{Schnitzler, Olga}!zzzGoldmann, Paul@\emph{von Paul Goldmann}!1902-08-121@{12. 8. {[}1902?{]}}|(be}\briefempfaengerindex{Schnitzler, Arthur@\textsc{Schnitzler, Arthur}!zzzGoldmann, Paul@\emph{von Paul Goldmann}!1902-08-121@{12. 8. {[}1902?{]}}|(be} \toendnotes[C]{\smallbreak\pagebreak[2]} \Standort{DLA, A:Schnitzler, HS.NZ85.1.3172.}
\physDesc{Telegramm
\newline{}maschinell\newline{}Ordnung: beschnitten }\toendnotes[C]{\smallbreak}\pstart
           \noindent{}\centering{}{\pb}\textcolor{pink}{wien}{}\ledrightnote{\textcolor{pink}{Wien}} v
                  \textcolor{pink}{muerren}{}\ledrightnote{\textcolor{pink}{Mürren}} 177 10 12/8{ }4-50n =\pend
           \pstart
           \noindent{}innigste \label{K_L02636-1v}\edtext{glueckwuensche}{\lemma{\textnormal{\emph{glueckwuensche}}}\Cendnote{\textnormal{zur Geburt von \textcolor{blue}{Heinrich Schnitzler} am 9. 8. 1902}}}\label{K_L02636-1h} euch beiden \spacefill\mbox{=
                  goldmann. + 1.+}\pend
           \endnumbering\briefempfaengerindex{Schnitzler, Olga@\textsc{Schnitzler, Olga}!zzzGoldmann, Paul@\emph{von Paul Goldmann}!1902-08-121@{12. 8. {[}1902?{]}}|)be}\briefempfaengerindex{Schnitzler, Arthur@\textsc{Schnitzler, Arthur}!zzzGoldmann, Paul@\emph{von Paul Goldmann}!1902-08-121@{12. 8. {[}1902?{]}}|)be}\mylabel{h}  \normalsize

\doendnotes{C}
\bigskip
\vfill

\clearpage

\footnotesize

\lohead{\textsc{register}}

% Definiere theindex-Environment komplett neu ohne reledmac
\makeatletter
\renewenvironment{theindex}{%
  \section*{\indexname}%
  \setlength{\parindent}{0pt}%
  \setlength{\parskip}{0pt plus 0.3pt}%
  \let\item\@idxitem
}{%
  \clearpage
}
\makeatother

\IfFileExists{\jobname-pw.ind}{\input{\jobname-pw.ind}}{}

\end{document}

      