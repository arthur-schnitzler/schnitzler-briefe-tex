%% latex-korrekturansicht-vorspann.tex
%% Vorspann für die Korrekturansicht.
%% Lädt die gemeinsame Datei latex-vorspann.tex mit gesetztem Schalter.

\newif\ifkorrekturansicht
\korrekturansichttrue

\input{../tex-inputs/latex-vorspann}


               \section[Hugo von Hofmannsthal an Arthur Schnitzler, {[}11. 6. 1909{]}]{ Hugo von Hofmannsthal an Arthur Schnitzler, {[}11. 6. 1909{]}}\nopagebreak\mylabel{v}\rehead{ }\normalsize\beginnumbering\briefempfaengerindex{Schnitzler, Arthur@\textsc{Schnitzler, Arthur}!zzzHofmannsthal, Hugo von@\emph{von Hugo von Hofmannsthal}!1909-06-112@{{[}11. 6. 1909{]}}|(be} \toendnotes[C]{\smallbreak\pagebreak[2]} \Standort{CUL, Schnitzler, B 43.}
\physDesc{Brief, 1 Blatt, 2 Seiten
\newline{}Handschrift: schwarze Tinte, deutsche Kurrent
\newline{}Schnitzler: mit Bleistift datiert: »11/6 09« und beschriftet: »Hofma{\geminationn}sthal« \newline{}Ordnung: 1) mit Bleistift von unbekannter Hand nummeriert: »\strikeout{30\textcolor{gray}{×}}« 2) mit Bleistift von unbekannter Hand nummeriert: »\strikeout{305}«}\buchAbdrucke{\weitereDrucke{Hugo von Hofmannsthal, Arthur Schnitzler: \emph{Briefwechsel}. Hg. Therese Nickl und Heinrich Schnitzler. Frankfurt am Main: \emph{S. Fischer} 1964, S. 245.} }\pstart
           \noindent{}\centering{}{\pb}\textcolor{gray}{\textbf{\textsc{Hotel}}}\pend
           \pstart
           \noindent{}\centering{}\textcolor{gray}{\textbf{\textcolor{pink}{\textsc{Vier Jahreszeiten}}{}\ledrightnote{\textcolor{pink}{Hotel Vier Jahreszeiten}}}}\pend
           \pstart
           \noindent{}\centering{}\textcolor{gray}{\textbf{TELEGRAMM-ADRESSE: JAHRESZEITENTYP, \textcolor{pink}{MÜNCHEN}{}\ledrightnote{\textcolor{pink}{München}}.}}\pend
           \pstart
           \noindent{}\centering{}\textcolor{gray}{\textbf{Lieber’s Code – International Hôtel-Code.}}\pend
           \pstart
           \noindent{}\centering{}\textcolor{gray}{\textbf{Telefon 23073–23076.}}\pend
           \pstart
           \noindent{}\raggedleft{}\textcolor{gray}{\textbf{\textcolor{pink}{\textsc{München}}{}\ledrightnote{\textcolor{pink}{München}},}}\pend
           \pstart{}lieber\pend\pstart
           wir treiben uns Automobilfahrend im ſüdlichen und weſtlichen \textcolor{pink}{Baiern}{}\ledrightnote{\textcolor{pink}{Bayern}} herum (\textcolor{pink}{Lech}{}\ledrightnote{\textcolor{pink}{Lech}}, \textcolor{pink}{Augsburg}{}\ledrightnote{\textcolor{pink}{Augsburg}}{ }\textsc{etc}.) ſchreibt doch ein Wort wann Ihr etwa nach \textcolor{pink}{München}{}\ledrightnote{\textcolor{pink}{München}} ko{\geminationm}t, an {\pb}dieſe Adreſſe: \textsc{\textcolor{pink}{Villa Cantacuzene Starnberg}{}\ledrightnote{\textcolor{pink}{Villa Cantacuzène}}}, es wird uns nachgeſchickt.\pend
           \pstart
           Alles Liebe an \textcolor{blue}{Olga}{}\ledrightnote{\textcolor{blue}{Olga Schnitzler}}.\pend
           \pstart
           Von Herzen Ihr{\\[\baselineskip]}\spacefill\mbox{Hugo.}\pend
           \leftskip=0em{}\endnumbering\briefempfaengerindex{Schnitzler, Arthur@\textsc{Schnitzler, Arthur}!zzzHofmannsthal, Hugo von@\emph{von Hugo von Hofmannsthal}!1909-06-112@{{[}11. 6. 1909{]}}|)be}\mylabel{h}  \normalsize

\doendnotes{C}
\bigskip
\vfill

\clearpage

\footnotesize

\lohead{\textsc{register}}

% Definiere theindex-Environment komplett neu ohne reledmac
\makeatletter
\renewenvironment{theindex}{%
  \section*{\indexname}%
  \setlength{\parindent}{0pt}%
  \setlength{\parskip}{0pt plus 0.3pt}%
  \let\item\@idxitem
}{%
  \clearpage
}
\makeatother

\IfFileExists{\jobname-pw.ind}{\input{\jobname-pw.ind}}{}

\end{document}

      