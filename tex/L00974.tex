%% latex-korrekturansicht-vorspann.tex
%% Vorspann für die Korrekturansicht.
%% Lädt die gemeinsame Datei latex-vorspann.tex mit gesetztem Schalter.

\newif\ifkorrekturansicht
\korrekturansichttrue

\input{../tex-inputs/latex-vorspann}


               \section[Arthur Schnitzler an Richard Beer-Hofmann, 15. 9. 1899]{ Arthur Schnitzler an Richard Beer-Hofmann, 15. 9. 1899}\nopagebreak\mylabel{v}\rehead{ }\normalsize\beginnumbering\briefempfaengerindex{Beer-Hofmann, Richard@\textsc{Beer-Hofmann, Richard}!zzzSchnitzler, Arthur@\emph{von Arthur Schnitzler}!1899-09-151@{15. 9. 1899}|(be} \toendnotes[C]{\smallbreak\pagebreak[2]} \Standort{YCGL, MSS 31.}
\physDesc{Bildpostkarte
\newline{}Handschrift: Bleistift, deutsche Kurrent\newline{}Versand: 1) Stempel: »\nobreak{}\oindex{Muenchen@\textbf{München}, \emph{https://www.geonames.org/ontologyP.PPLA}|pwk}München, 15 Sep 99, 6–7Nm\nobreak{}«.  2) Stempel: »\nobreak{}17. 9. 99\nobreak{}«. \newline{}Zusatz: Postkartenmotiv von \textcolor{blue}{Otto
                                    Strützel} }\toendnotes[C]{\smallbreak}\pstart{}{\pb}Hrn \textsc{Dr. Richard
                     Beer-Hofmann}\pend{}\pstart{}\textcolor{pink}{\textsc{Vahrn}}{}\ledrightnote{\textcolor{pink}{Vahrn}} bei \textsc{\textcolor{pink}{Brixen}{}\ledrightnote{\textcolor{pink}{Brixen}}}\pend{}\pstart{}\textcolor{pink}{\textsc{Tirol}}{}\ledrightnote{\textcolor{pink}{Südtirol}}\pend{}{\bigskip}\pstart
           \noindent{}\centering{}\textcolor{gray}{\textbf{{\pb}\textcolor{pink}{München}{}\ledrightnote{\textcolor{pink}{München}}, \textcolor{pink}{Frauenkirche}{}\ledrightnote{\textcolor{pink}{Frauenkirche}}}}\pend
           \pstart
           lieber Richard, ich fahre von hier (nicht ganz direct)
               wahrſcheinlich \textcolor{pink}{Frankfurt}{}\ledrightnote{\textcolor{pink}{Frankfurt am Main}} zu \textcolor{blue}{Goldmann}{}\ledrightnote{\textcolor{blue}{Paul Goldmann}}; nehme an, eventuell Mittwoch dort zu
               ſein. Iſt \textcolor{blue}{Hugo}{}\ledrightnote{\textcolor{blue}{Hugo von Hofmannsthal}} bei Ihnen? Von \textcolor{pink}{Fr.}{}\ledrightnote{\textcolor{pink}{Frankfurt am Main}} fahr ich nach – pardon – will ich nach \textcolor{pink}{Berlin}{}\ledrightnote{\textcolor{pink}{Berlin}} fahren. Bitte Nachricht \textcolor{pink}{Frkf
                  a M}{}\ledrightnote{\textcolor{pink}{Frankfurt am Main}}{ }\textsc{post rest}.\pend
           \pstart
           \label{T_L00974_1v}\edtext{Herzlich}{\lemma{\textnormal{\emph{Herzlich}}}\Cendnote{\textnormal{Grußformel über dem Text am rechten Rand.}}}\label{T_L00974_1h}{\\[\baselineskip]}Ihr\spacefill\mbox{A.}\pend
           \leftskip=0em{}\pstart
           \noindent{}\label{T_L00974_2v}\edtext{Danke für Ihren l. Brief.}{\lemma{\textnormal{\emph{Danke … Brief.}}}\Cendnote{\textnormal{am linken Rand neben dem Bild}}}\label{T_L00974_2h}\pend
           \endnumbering\briefempfaengerindex{Beer-Hofmann, Richard@\textsc{Beer-Hofmann, Richard}!zzzSchnitzler, Arthur@\emph{von Arthur Schnitzler}!1899-09-151@{15. 9. 1899}|)be}\mylabel{h}  \normalsize

\doendnotes{C}
\bigskip
\vfill

\clearpage

\footnotesize

\lohead{\textsc{register}}

% Definiere theindex-Environment komplett neu ohne reledmac
\makeatletter
\renewenvironment{theindex}{%
  \section*{\indexname}%
  \setlength{\parindent}{0pt}%
  \setlength{\parskip}{0pt plus 0.3pt}%
  \let\item\@idxitem
}{%
  \clearpage
}
\makeatother

\IfFileExists{\jobname-pw.ind}{\input{\jobname-pw.ind}}{}

\end{document}

      