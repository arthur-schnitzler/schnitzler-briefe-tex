%% latex-korrekturansicht-vorspann.tex
%% Vorspann für die Korrekturansicht.
%% Lädt die gemeinsame Datei latex-vorspann.tex mit gesetztem Schalter.

\newif\ifkorrekturansicht
\korrekturansichttrue

\input{../tex-inputs/latex-vorspann}


               \section[Arthur Schnitzler an Hugo von Hofmannsthal, 6. 7. 1899]{ Arthur Schnitzler an Hugo von Hofmannsthal, 6. 7. 1899}\nopagebreak\mylabel{v}\rehead{ }\normalsize\beginnumbering\briefempfaengerindex{Hofmannsthal, Hugo von@\textsc{Hofmannsthal, Hugo von}!zzzSchnitzler, Arthur@\emph{von Arthur Schnitzler}!1899-07-062@{6. 7. 1899}|(be} \toendnotes[C]{\smallbreak\pagebreak[2]} \Standort{FDH, Hs-30885,82.}
\physDesc{Brief, 1 Blatt, 4 Seiten
\newline{}Handschrift: schwarze Tinte, deutsche Kurrent\newline{}Ordnung: mit Bleistift von Schnitzler mutmaßlich während der
                                            Durchsicht der Briefe 1929 am oberen
                                            Blattrand zusätzlich datiert: »6/7 99« }\buchAbdrucke{\weitereDrucke{1) Hugo von Hofmannsthal, Arthur Schnitzler: \emph{Briefwechsel}. Hg. Therese Nickl und Heinrich Schnitzler. Frankfurt am Main: \emph{S. Fischer} 1964, S. 123.} \weitereDrucke{2) Hermann Bahr, Arthur Schnitzler: \emph{Briefwechsel, Aufzeichnungen, Dokumente
                                (1891–1931)}. Hg. Kurt Ifkovits und Martin Anton Müller. Göttingen: \emph{Wallstein} 2018, S. 170.} }\toendnotes[C]{\smallbreak}\pstart{}{\pb}lieber Hugo,\pend\pstart
           folgendes iſt mit \uuline{\edtext{vollkommener Discretion}{\Cendnote{dreifach unterstrichen}}} zu
                    behandeln: \textcolor{blue}{\uline{Bahr}}{}\ledrightnote{\textcolor{blue}{Hermann Bahr}}\uline{ verläßt die }\textcolor{brown}{\uline{Zeit}}{}\ledrightnote{\textcolor{brown}{Die Zeit. Wiener Wochenschrift}}. \textcolor{blue}{Singer}{}\ledrightnote{\textcolor{blue}{Isidor Singer}} und \textcolor{blue}{Kanner}{}\ledrightnote{\textcolor{blue}{Heinrich Kanner}} waren bei mir. Lange Unterredung ohne Intereſſe
                    für Sie (nur mich.) Das weſentliche: ſie möchten auf das Blatt ſtellen: unter
                    Mitwirkung von – \textsc{etc etc} nur erſte Namen, ich möchte
                    Sie fragen, ob Sie im Princip damit {\pb}einverſtanden
                    wären, auch als »Mitwirkender{[}«{]} oder »ſtändg Mitwirkender«
                    aufs Blatt zu ko{\geminationm}en, neben \textsc{\textcolor{blue}{Burckhard}{}\ledrightnote{\textcolor{blue}{Max Eugen Burckhard}}}, mich, – event. \textcolor{blue}{\textsc{Hauptmann}}{}\ledrightnote{\textcolor{blue}{Gerhart Hauptmann}} (\label{K_L00934_1v}\edtext{an den ich mich über \textcolor{blue}{Brahm}{}\ledrightnote{\textcolor{blue}{Otto Brahm}} wende}{\lemma{\textnormal{\emph{an … wende}}}\Cendnote{\textnormal{siehe Arthur Schnitzler an Gerhart Hauptmann,
                    15. 7. 1899}}}\label{K_L00934_1h}.) Sie können natürlich ohne weiters zuſagen. Für die Herausgeber ſcheint
                    mir die Sache allerdings überflüſſig: ſie brauchten Arbeitskräfte, nicht
                    Namen. –\pend
           \pstart
           Ich bin noch hier; und will über meine {\pb}Sti{\geminationm}ung nichts ſagen, da nichts neues u nicht
                    erfreuliches vorliegt. Gerade dſs ſich das Leben da und dort wieder zu melden
                    anfängt, iſt das traurige; es iſt ein Leben dritter Ordnung, das beſte iſt
                    vorbei.\pend
           \pstart
           Das Wetter ist ſchändlich. Mitte Juli reiſe ich nach \textcolor{pink}{Kärnthen}{}\ledrightnote{\textcolor{pink}{Kärnten}}; zuerſt \textcolor{pink}{\textsc{Velden}}{}\ledrightnote{\textcolor{pink}{Velden}}, dann zu \textcolor{blue}{Richard}{}\ledrightnote{\textcolor{blue}{Richard Beer-Hofmann}}, von dem ich eine
                    kurze Karte habe. – Hat ſich in den Chancen für Mitte Auguſt (\textcolor{pink}{Thü{\pb}ringen}{}\ledrightnote{\textcolor{pink}{Thüringen}}{ }\textsc{etc}) was geändert? – Arbeiten Sie? Sehn Sie \textcolor{blue}{Minnie}{}\ledrightnote{\textcolor{blue}{Hermine von Schaffgotsch}}? –\pend
           \pstart Leben Sie wohl. Von Herzen Ihr \spacefill\mbox{Arthur Sch}\pend{}\pstart
           \textcolor{pink}{Wien}{}\ledrightnote{\textcolor{pink}{Wien}}{ }6. 7. 99.\pend
           \endnumbering\briefempfaengerindex{Hofmannsthal, Hugo von@\textsc{Hofmannsthal, Hugo von}!zzzSchnitzler, Arthur@\emph{von Arthur Schnitzler}!1899-07-062@{6. 7. 1899}|)be}\mylabel{h}  \normalsize

\doendnotes{C}
\bigskip
\vfill

\clearpage

\footnotesize

\lohead{\textsc{register}}

% Definiere theindex-Environment komplett neu ohne reledmac
\makeatletter
\renewenvironment{theindex}{%
  \section*{\indexname}%
  \setlength{\parindent}{0pt}%
  \setlength{\parskip}{0pt plus 0.3pt}%
  \let\item\@idxitem
}{%
  \clearpage
}
\makeatother

\IfFileExists{\jobname-pw.ind}{\input{\jobname-pw.ind}}{}

\end{document}

      