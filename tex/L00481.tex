%% latex-korrekturansicht-vorspann.tex
%% Vorspann für die Korrekturansicht.
%% Lädt die gemeinsame Datei latex-vorspann.tex mit gesetztem Schalter.

\newif\ifkorrekturansicht
\korrekturansichttrue

\input{../tex-inputs/latex-vorspann}


               \section[Arthur Schnitzler an Richard Beer-Hofmann, 12. 9. 1895]{ Arthur Schnitzler an Richard Beer-Hofmann,
               12. 9. 1895}\nopagebreak\mylabel{v}\rehead{ }\normalsize\beginnumbering\briefempfaengerindex{Beer-Hofmann, Richard@\textsc{Beer-Hofmann, Richard}!zzzSchnitzler, Arthur@\emph{von Arthur Schnitzler}!1895-09-121@{12. 9. 1895}|(be} \toendnotes[C]{\smallbreak\pagebreak[2]} \Standort{YCGL, MSS 31.}
\physDesc{Brief, 1 Blatt, 4 Seiten, Umschlag
\newline{}Handschrift: Bleistift, deutsche Kurrent\newline{}Versand: 1) Stempel: »\nobreak{}\oindex{IX., Alsergrund@\textbf{IX., Alsergrund}, \emph{Bezirk (A.BZK)}|pwk}Wien 9/3, 12. 9. 95, 2–3V\nobreak{}«.  2) Stempel: »\nobreak{}\oindex{Schoenberg im Stubaital@\textbf{Schönberg im Stubaital}, \emph{Besiedelter Ort (A.BSO)}|pwk}\textcolor{gray}{Schön}{[}berg{]} in Tirol, \textcolor{gray}{13} {[}9{]} \textcolor{gray}{95}\nobreak{}«. }\buchAbdrucke{\weitereDrucke{Arthur Schnitzler, Richard Beer-Hofmann: \emph{Briefwechsel 1891–1931}. Hg. Konstanze Fliedl. Wien, Zürich: \emph{Europaverlag} 1992, S. 79–80.} }\toendnotes[C]{\smallbreak}\pstart{}{\pb}\textsc{Herrn Dr Rich Beer-Hofmann}\pend{}\pstart{}\textsc{\textcolor{pink}{Tirol}{}\ledrightnote{\textcolor{pink}{Tirol}}}\pend{}\pstart{}\textsc{\textcolor{pink}{Schönberg im Stubaithal}{}\ledrightnote{\textcolor{pink}{Schönberg im Stubaital}}}\pend{}{\bigskip}\pstart
           \noindent{}{\pb}Lieber Richard, Sie werden ſich hoffentlich \substVorne{}\textsuperscript{hier}\substDazwischen{}dort\substHinten{}{ }ſehr wohl fühlen. We{\geminationn} es nur ſchön
               bleibt – hier iſt der Umſchlag ſchon, regnet, iſt kalt. Was werden Sie da thun bis
               Ende October? Ich glaube, Sie werden vom 16. an plötzlich in irgend
               einer Stadt {\pb}ſein und früher als Sie ahnten in
                  \textcolor{pink}{Wien}{}\ledrightnote{\textcolor{pink}{Wien}}. –\pend
           \pstart
           Viel neues gibts nicht. \textcolor{green}{\textsc{Liebelei}}{}\ledrightnote{\textcolor{green}{Liebelei. Schauspiel in drei Akten}}{ }ſoll wirklich die 1. \label{K_L00481_1v}\edtext{Nov.}{\lemma{\textnormal{\emph{Nov.}}}\Cendnote{\textnormal{Novität}}}\label{K_L00481_1h}{ }ſein,
                  Anfang October. – Die \textcolor{blue}{\textsc{Trag}}{}\ledrightnote{→\textcolor{blue}{Adele Sandrock}} hat ſchon wieder ihre Feindſeligkeiten eröffnet in kindiſcher u hilfloſer
               Weise. – Kleine Aergerlichkeiten durch das »Zu Hauſe« – die Schlüſſel {\pb}klappern zu viel. (\textsc{Symbol}.)\pend
           \pstart
           – Aerztlich zu thun. Ja! – Zufall natürlich. –\pend
           \pstart
           Geschrieben noch nichts. –\pend
           \pstart
           Bitte grüßen Sie Frau \textcolor{blue}{Lou}{}\ledrightnote{\textcolor{blue}{Lou Andreas-Salomé}} recht herzlich, wenn
               ſie noch da iſt; we{\geminationn}
               Sie mir ein Wort gleich ſchreiben, {\pb}hören Sie ſofort wieder, etwas ausführlicher, von
               mir\pend
           \pstart Ihr \spacefill\mbox{Arth}\pend{}\pstart
           12. 9. 95. \textcolor{pink}{Wien}{}\ledrightnote{\textcolor{pink}{Wien}}\pend
           \endnumbering\briefempfaengerindex{Beer-Hofmann, Richard@\textsc{Beer-Hofmann, Richard}!zzzSchnitzler, Arthur@\emph{von Arthur Schnitzler}!1895-09-121@{12. 9. 1895}|)be}\mylabel{h}  \normalsize

\doendnotes{C}
\bigskip
\vfill

\clearpage

\footnotesize

\lohead{\textsc{register}}

% Definiere theindex-Environment komplett neu ohne reledmac
\makeatletter
\renewenvironment{theindex}{%
  \section*{\indexname}%
  \setlength{\parindent}{0pt}%
  \setlength{\parskip}{0pt plus 0.3pt}%
  \let\item\@idxitem
}{%
  \clearpage
}
\makeatother

\IfFileExists{\jobname-pw.ind}{\input{\jobname-pw.ind}}{}

\end{document}

      