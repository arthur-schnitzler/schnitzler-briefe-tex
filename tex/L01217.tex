%% latex-korrekturansicht-vorspann.tex
%% Vorspann für die Korrekturansicht.
%% Lädt die gemeinsame Datei latex-vorspann.tex mit gesetztem Schalter.

\newif\ifkorrekturansicht
\korrekturansichttrue

\input{../tex-inputs/latex-vorspann}


               \section[Arthur Schnitzler an Richard Beer-Hofmann, 2. {[}5.?{]} 1902]{ Arthur Schnitzler an Richard Beer-Hofmann, 2. {[}5.?{]} 1902}\nopagebreak\mylabel{v}\rehead{ }\normalsize\beginnumbering\briefempfaengerindex{Beer-Hofmann, Richard@\textsc{Beer-Hofmann, Richard}!zzzSchnitzler, Arthur@\emph{von Arthur Schnitzler}!1902-05-021@{2. {[}5.?{]} 1902}|(be} \toendnotes[C]{\smallbreak\pagebreak[2]} \Standort{YCGL, MSS 31.}
\physDesc{Briefkarte, Umschlag
\newline{}Handschrift: Bleistift, deutsche Kurrent\newline{}Versand: 1) Stempel: »\nobreak{}\textcolor{gray}{Wien}, \textcolor{gray}{2 5} 02, 5–6N\nobreak{}«.  2) Stempel: »\nobreak{}\oindex{Rodaun@\textbf{Rodaun}, \emph{Teil eines besiedelten Ortes (A.BSOX)}|pwk}{\pb}Rodaun, 3. \textcolor{gray}{5}. 02, \textcolor{gray}{7–9}V\nobreak{}«. \newline{}Ordnung: mit Bleistift von unbekannter Hand falsch datiert: »3. 3.« }\buchAbdrucke{\weitereDrucke{Arthur Schnitzler, Richard Beer-Hofmann: \emph{Briefwechsel 1891–1931}. Hg. Konstanze Fliedl. Wien, Zürich: \emph{Europaverlag} 1992, S. 157.} }\toendnotes[C]{\smallbreak}\pstart{}{\pb}Herrn \textsc{Dr. Richard
                     Beer-Hofmann}\pend{}\pstart{}\textcolor{pink}{\textsc{Rodaun}}{}\ledrightnote{\textcolor{pink}{Rodaun}}\pend{}\pstart{}\textsc{\textcolor{pink}{Liesinger Straße 2}{}\ledrightnote{\textcolor{pink}{Liesingerstraße}}}\pend{}{\bigskip}\pstart
           \noindent{}{\pb} lieber Richard, ich weiſs nicht, ob Sie Sitze haben, jedenfalls
               laſſe ich Ihnen bis \label{K_L01217_1v}\edtext{Dinſtag}{\lemma{\textnormal{\emph{Dinſtag}}}\Cendnote{\textnormal{Die Poststempel dieses
                  Korrespondenzstücks sind, mit Ausnahme der Jahresangabe, nur unzuverlässig zu
                  entziffern, weswegen es bislang auch mit 2. 3. 1902 datiert wurde. Da
                  es sich aber um einen Zeitraum handeln muss, in dem \textcolor{blue}{Brahm} für das Gastspiel im \textcolor{pink}{Carltheater}
                  in \textcolor{pink}{Wien} weilt, ist die Monatsangabe mit Mai
                  anzusetzen und mit »Dienstag« der 6. 5. 1902 gemeint, der erste Tag des Gastspiels. Dazu passt auch
                  das Telegramm \textcolor{blue}{Brahm}s vom
                     2. 5. 1902 (\emph{Der Briefwechsel Arthur Schnitzler — Otto Brahm}.
                     Vollständige Ausgabe. Herausgegeben, eingeleitet und erläutert von Oskar
                     Seidlin. Tübingen: \emph{Niemeyer}{ }1975, S. 122), in dem er die hier in Folge an \textcolor{blue}{Hofmannsthal} weiterzugebende Antwort
                  kommuniziert.}}}\label{K_L01217_1h}{ }Mittag an der \textcolor{pink}{Carltheater}{}\ledrightnote{\textcolor{pink}{Carl-Theater}} Caſſe
               2 Parkets reſerviren. Holen Sie ſie nicht, ſo werden ſie anderweitig {\pb}verkauft. – Sie haben ſich alſo nicht weiter zu
               kümmern. –\pend
           \pstart
           Dem \textcolor{blue}{Hugo}{}\ledrightnote{\textcolor{blue}{Hugo von Hofmannsthal}} ſagen Sie bitte, \uline{aber sicher}, dſs \textcolor{blue}{Brahm}{}\ledrightnote{\textcolor{blue}{Otto Brahm}}{ }Dinſtag{ }\uline{nicht} zu mir kommt.\pend
           \pstart
           Ich hoffe übrigens So{\geminationn}tag{ }Vormittag{ }\textcolor{pink}{Rodaun}{}\ledrightnote{\textcolor{pink}{Rodaun}} zu durchradeln.\pend
           \pstart
           Herzlichſt Ihr{\\[\baselineskip]}\spacefill\mbox{A.}\pend
           \leftskip=0em{}\endnumbering\briefempfaengerindex{Beer-Hofmann, Richard@\textsc{Beer-Hofmann, Richard}!zzzSchnitzler, Arthur@\emph{von Arthur Schnitzler}!1902-05-021@{2. {[}5.?{]} 1902}|)be}\mylabel{h}  \normalsize

\doendnotes{C}
\bigskip
\vfill

\clearpage

\footnotesize

\lohead{\textsc{register}}

% Definiere theindex-Environment komplett neu ohne reledmac
\makeatletter
\renewenvironment{theindex}{%
  \section*{\indexname}%
  \setlength{\parindent}{0pt}%
  \setlength{\parskip}{0pt plus 0.3pt}%
  \let\item\@idxitem
}{%
  \clearpage
}
\makeatother

\IfFileExists{\jobname-pw.ind}{\input{\jobname-pw.ind}}{}

\end{document}

      