%% latex-korrekturansicht-vorspann.tex
%% Vorspann für die Korrekturansicht.
%% Lädt die gemeinsame Datei latex-vorspann.tex mit gesetztem Schalter.

\newif\ifkorrekturansicht
\korrekturansichttrue

\input{../tex-inputs/latex-vorspann}


               \section[Charlotte Ehrenstein an Arthur Schnitzler, {[}16. 3.? 1906{]}]{ Charlotte Ehrenstein an Arthur Schnitzler, {[}16. 3.? 1906{]}}\nopagebreak\mylabel{v}\rehead{ }\normalsize\beginnumbering\briefempfaengerindex{Schnitzler, Arthur@\textsc{Schnitzler, Arthur}!zzzEhrenstein, Charlotte@\emph{von Charlotte Ehrenstein}!1906-03-161@{{[}16. 3.? 1906{]}}|(be} \toendnotes[C]{\smallbreak\pagebreak[2]} \Standort{DLA, A:Schnitzler, HS.NZ85.1.2837,1.}
\physDesc{Brief, 1 Blatt, 2 Seiten
\newline{}Handschrift: Bleistift, deutsche Kurrent
\newline{}Schnitzler: mit Bleistift beschriftet: »\textsc{Ehrenstein}« und datiert: »ca 16/3 906« }\toendnotes[C]{\smallbreak}\pstart
           \noindent{}{\pb}\textsc{Hochwohlgeb. Herrn Dr. Arthur Schnitzler}! \pend
           \pstart\center{}\textsc{Sehr geehrter Herr Doctor!}\pend\pstart
           Um in dieſen vielbeſchäftigten Tagen nicht zu beläſtigen, und da Dr \textcolor{blue}{Kornfeld}{}\ledrightnote{\textcolor{blue}{Sigmund Kornfeld}} nach vielwöchentlicher Pauſe, meinen
                    l. \textcolor{blue}{Albert}{}\ledrightnote{\textcolor{blue}{Albert Ehrenstein}} vor einigen Tagen beſuchte,
                    geſtatte ich mir heute über \textcolor{blue}{Albert}{}\ledrightnote{\textcolor{blue}{Albert Ehrenstein}}s Befinden
                    zu berichten. Dr. \textcolor{blue}{Kornfeld}{}\ledrightnote{\textcolor{blue}{Sigmund Kornfeld}} iſt mit der
                    Beſſerung in \textcolor{blue}{Albert}{}\ledrightnote{\textcolor{blue}{Albert Ehrenstein}}s Zuſtand in jeder
                    Hinſicht zufrieden, er beſchäftigt ſich fleißig mit Büchern ernſten Inhalts,
                    Eſſays und Lebensbeſchreibungen. Bei weiterer ernſter Beherrſchung dürfte \textcolor{blue}{Albert}{}\ledrightnote{\textcolor{blue}{Albert Ehrenstein}}, ſo meint Dr. \textcolor{blue}{Kornfeld}{}\ledrightnote{\textcolor{blue}{Sigmund Kornfeld}}, im zweiten {\pb}Semeſter noch die Univerſität beſuchen. Einſtweilen muſs er aber, obwohl er
                    recht gut ausſieht noch immer im Bette bleiben. Mich Ihrer verehrten Frau \textcolor{blue}{Gemahlin}{}\ledrightnote{→\textcolor{blue}{Olga Schnitzler}} empfehlend, von
                    meinem \textcolor{blue}{Mann}{}\ledrightnote{→\textcolor{blue}{Alexander Ehrenstein}} u. \textcolor{blue}{Albert}{}\ledrightnote{\textcolor{blue}{Albert Ehrenstein}} die höflichſten Grüße, bin ich
                    Ihre\pend
           \pstart
           dankbare u. Sie verehrende{\\[\baselineskip]}\spacefill\mbox{Charlotte Ehrenſtein}\pend
           \leftskip=0em{}\endnumbering\briefempfaengerindex{Schnitzler, Arthur@\textsc{Schnitzler, Arthur}!zzzEhrenstein, Charlotte@\emph{von Charlotte Ehrenstein}!1906-03-161@{{[}16. 3.? 1906{]}}|)be}\mylabel{h}  \normalsize

\doendnotes{C}
\bigskip
\vfill

\clearpage

\footnotesize

\lohead{\textsc{register}}

% Definiere theindex-Environment komplett neu ohne reledmac
\makeatletter
\renewenvironment{theindex}{%
  \section*{\indexname}%
  \setlength{\parindent}{0pt}%
  \setlength{\parskip}{0pt plus 0.3pt}%
  \let\item\@idxitem
}{%
  \clearpage
}
\makeatother

\IfFileExists{\jobname-pw.ind}{\input{\jobname-pw.ind}}{}

\end{document}

      