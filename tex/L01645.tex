%% latex-korrekturansicht-vorspann.tex
%% Vorspann für die Korrekturansicht.
%% Lädt die gemeinsame Datei latex-vorspann.tex mit gesetztem Schalter.

\newif\ifkorrekturansicht
\korrekturansichttrue

\input{../tex-inputs/latex-vorspann}


               \section[Max Burckhard: Widmungsexemplar Im Paradiese für Arthur Schnitzler, 2{[}2?{]}. 12. 1906]{ Max Burckhard: Widmungsexemplar Im Paradiese für Arthur Schnitzler,
               2{[}2?{]}. 12. 1906}\nopagebreak\mylabel{v}\rehead{ }\normalsize\beginnumbering\briefempfaengerindex{Schnitzler, Arthur@\textsc{Schnitzler, Arthur}!zzzBurckhard, Max Eugen@\emph{von Max Eugen Burckhard}!1906-12-221@{2{[}2?{]}. 12. 1906}|(be} \toendnotes[C]{\smallbreak\pagebreak[2]} \Standort{DLA, G:Schnitzler, Arthur (Sammlung Heinrich Schnitzler).}
\physDesc{Widmung am Vortitel
\newline{}Handschrift: schwarze Tinte, deutsche Kurrent\newline{}Ordnung: bei der Enteignung des Exemplars 1938 von
                                 unbekannter Hand mit Bleistift ergänzte Kenntlichmachung als
                                 Dublette: »= 452.154-B« }\toendnotes[C]{\smallbreak}\pstart
           \noindent{}{\pb}Arthur Schnitzler{\\}in herzlicher Verehrung \pend
           \pstart \spacefill\mbox{Max Burckhard}\pend{}{\bigskip}\pstart
           \noindent{}\centering{}\textcolor{gray}{\textbf{\textcolor{green}{Im Paradiese}{}\ledrightnote{\textcolor{green}{Im Paradiese. Komödie in 4 Akten}}.}}\pend
           {\bigskip}\pstart
           \noindent{}\centering{}{\pb}\textcolor{gray}{\textbf{Max Burckhard.}}\pend
           \pstart
           \noindent{}\centering{}\textcolor{gray}{\textbf{\textcolor{green}{Im Paradiese}{}\ledrightnote{\textcolor{green}{Im Paradiese. Komödie in 4 Akten}}.}}\pend
           \pstart
           \noindent{}\centering{}\textcolor{gray}{\textbf{Komödie in 4 Akten.}}\pend
           {\bigskip}\pstart
           \noindent{}\centering{}\textcolor{gray}{\textbf{\textcolor{brown}{Wiener Verlag, G. m. b. H.}{}\ledrightnote{\textcolor{brown}{Wiener Verlag}}}}\pend
           \pstart
           \noindent{}\centering{}\textcolor{gray}{\textbf{\textcolor{pink}{WIEN}{}\ledrightnote{\textcolor{pink}{Wien}}{ }UND{ }\textcolor{pink}{LEIPZIG}{}\ledrightnote{\textcolor{pink}{Leipzig}}}}\pend
           \pstart
           \noindent{}\centering{}\textcolor{gray}{\textbf{\label{K_L01645_1v}\edtext{1907}{\lemma{\textnormal{\emph{1907}}}\Cendnote{\textnormal{vordatiert, vgl. A. S.: \emph{Tagebuch}, 22. 12. 1906}}}\label{K_L01645_1h}.}}\pend
           \endnumbering\briefempfaengerindex{Schnitzler, Arthur@\textsc{Schnitzler, Arthur}!zzzBurckhard, Max Eugen@\emph{von Max Eugen Burckhard}!1906-12-221@{2{[}2?{]}. 12. 1906}|)be}\mylabel{h}  \normalsize

\doendnotes{C}
\bigskip
\vfill

\clearpage

\footnotesize

\lohead{\textsc{register}}

% Definiere theindex-Environment komplett neu ohne reledmac
\makeatletter
\renewenvironment{theindex}{%
  \section*{\indexname}%
  \setlength{\parindent}{0pt}%
  \setlength{\parskip}{0pt plus 0.3pt}%
  \let\item\@idxitem
}{%
  \clearpage
}
\makeatother

\IfFileExists{\jobname-pw.ind}{\input{\jobname-pw.ind}}{}

\end{document}

      