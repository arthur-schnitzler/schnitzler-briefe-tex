%% latex-korrekturansicht-vorspann.tex
%% Vorspann für die Korrekturansicht.
%% Lädt die gemeinsame Datei latex-vorspann.tex mit gesetztem Schalter.

\newif\ifkorrekturansicht
\korrekturansichttrue

\input{../tex-inputs/latex-vorspann}


               \section[Richard Beer-Hofmann an Arthur Schnitzler, {[}20. 10. 1897{]}]{ Richard Beer-Hofmann an Arthur Schnitzler, {[}20. 10. 1897{]}}\nopagebreak\mylabel{v}\rehead{ }\normalsize\beginnumbering\briefempfaengerindex{Schnitzler, Arthur@\textsc{Schnitzler, Arthur}!zzzBeer-Hofmann, Richard@\emph{von Richard Beer-Hofmann}!1897-10-201@{{[}20. 10. 1897{]}}|(be} \toendnotes[C]{\smallbreak\pagebreak[2]} \Standort{CUL, Schnitzler, B 8.}
\physDesc{Briefkarte
\newline{}Handschrift: Bleistift, lateinische Kurrent
\newline{}Schnitzler: mit Bleistift datiert: »20/10 97« \newline{}Ordnung: mit Bleistift von unbekannter Hand nummeriert:
                                    »106« }\buchAbdrucke{\weitereDrucke{Arthur Schnitzler, Richard Beer-Hofmann: \emph{Briefwechsel 1891–1931}. Hg. Konstanze Fliedl. Wien, Zürich: \emph{Europaverlag} 1992, S. 113.} }\toendnotes[C]{\smallbreak}\pstart
           \noindent{}{\pb}Lieber Arthur! Ich bin
                  Freitag{ }2 ¾ mit \textcolor{blue}{Leo}{}\ledrightnote{\textcolor{blue}{Leo Van-Jung}} bei Ihnen, wir gehen
               dann zum \label{K_L00732_1v}\edtext{\textcolor{blue}{Notar}{}\ledrightnote{→\textcolor{blue}{Emil Wolf}}}{\lemma{\textnormal{\emph{Notar}}}\Cendnote{\textnormal{Der genaue Vorgang ist unklar. Am
                     4. 9. 1897 war die Tochter \textcolor{blue}{Mirjam} auf die Welt gekommen. Obwohl die Kindeseltern nicht verheiratet
                  waren, wurde der Name des Vaters eingetragen. Die Legitimierung erfolgte wenige
                  Tage nach der Hochzeit (14. 5. 1898) am 25. 5. 1898. \textcolor{blue}{Schnitzler} fungierte sowohl für die Geburt, wie für die Hochzeit als Zeuge.}}}\label{K_L00732_1h} unterschreiben. {\pb}Ja?\pend
           \pstart
           Herzlichst{\\[\baselineskip]}\spacefill\mbox{Richard}\pend
           \leftskip=0em{}\endnumbering\briefempfaengerindex{Schnitzler, Arthur@\textsc{Schnitzler, Arthur}!zzzBeer-Hofmann, Richard@\emph{von Richard Beer-Hofmann}!1897-10-201@{{[}20. 10. 1897{]}}|)be}\mylabel{h}  \normalsize

\doendnotes{C}
\bigskip
\vfill

\clearpage

\footnotesize

\lohead{\textsc{register}}

% Definiere theindex-Environment komplett neu ohne reledmac
\makeatletter
\renewenvironment{theindex}{%
  \section*{\indexname}%
  \setlength{\parindent}{0pt}%
  \setlength{\parskip}{0pt plus 0.3pt}%
  \let\item\@idxitem
}{%
  \clearpage
}
\makeatother

\IfFileExists{\jobname-pw.ind}{\input{\jobname-pw.ind}}{}

\end{document}

      