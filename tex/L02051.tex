%% latex-korrekturansicht-vorspann.tex
%% Vorspann für die Korrekturansicht.
%% Lädt die gemeinsame Datei latex-vorspann.tex mit gesetztem Schalter.

\newif\ifkorrekturansicht
\korrekturansichttrue

\input{../tex-inputs/latex-vorspann}


               \section[Georg Brandes an Arthur und Olga Schnitzler, 3. 2. 1912]{ Georg Brandes an Arthur und Olga Schnitzler, 3. 2. 1912}\nopagebreak\mylabel{v}\rehead{ }\normalsize\beginnumbering\briefempfaengerindex{Schnitzler, Olga@\textsc{Schnitzler, Olga}!zzzBrandes, Georg@\emph{von Georg Brandes}!1912-02-031@{3. 2. 1912}|(be}\briefempfaengerindex{Schnitzler, Arthur@\textsc{Schnitzler, Arthur}!zzzBrandes, Georg@\emph{von Georg Brandes}!1912-02-031@{3. 2. 1912}|(be} \toendnotes[C]{\smallbreak\pagebreak[2]} \Standort{CUL, Schnitzler, B 17.}
\physDesc{Postkarte
\newline{}Handschrift: schwarze Tinte, lateinische Kurrent\newline{}Versand: Stempel: »\nobreak{}\oindex{Paris@\textbf{Paris}, \emph{Besiedelter Ort (A.BSO)}|pwk}Paris, 3-2 12\nobreak{}«.  \newline{}Ordnung: mit Bleistift von unbekannter Hand nummeriert: »38« }\buchAbdrucke{\weitereDrucke{Georg Brandes, Arthur Schnitzler: \emph{Ein Briefwechsel}. Hg. Kurt Bergel. Bern: \emph{Francke} 1956, S. 104.} }\toendnotes[C]{\smallbreak}\pstart{}{\pb}\textcolor{gray}{\textbf{* Expedié par}}\pend{}\pstart{}\textcolor{gray}{\textbf{M}} Brandes\pend{}\pstart{}\textcolor{gray}{\textbf{Dem\textsuperscript{t} à}}{ }\textcolor{pink}{Hotel d’Jéna}{}\ledrightnote{\textcolor{pink}{Hotel d’Jéna}}\pend{}\pstart{}\textcolor{pink}{Paris}{}\ledrightnote{\textcolor{pink}{Paris}}\pend{}{\bigskip}\pstart{}\textcolor{gray}{\textbf{M}}onsieur Arthur Schnitzler\pend{}\pstart{}\textcolor{pink}{Sternwartestrasse 71}{}\ledrightnote{\textcolor{pink}{Sternwartestraße}}\pend{}\pstart{}Vienne\hspace*{2.5em}Autriche\pend{}{\bigskip}\pstart
           {\pb}\textcolor{pink}{Paris. Hotel d’Jéna}{}\ledrightnote{\textcolor{pink}{Hotel d’Jéna}}\hfill 3 Febr. 12\pend
           \pstart{}Verehrter Freund, verehrte Freundin\pend\pstart
           Ihre lieben und schönen Portraits haben mich hier eingeholt, wohin ich geflohen
                    bin um verschiedenen Festlichkeiten in \textcolor{pink}{Kopenhagen}{}\ledrightnote{\textcolor{pink}{Kopenhagen}} zu vermeiden. Ich bin Ihnen sehr dankbar, dass auch Sie,
                    die ich so sehr schätze, an mich (bei dieser schmählichen tragikomischen
                    Gelegenheit) gedacht haben.\pend
           \pstart
           Ihnen gegenüber ist mein Herz voll. \label{K_L02051_1v}\edtext{On a eu l’idée saugrenue}{\lemma{\textnormal{\emph{On a eu l’idée saugrenue}}}\Cendnote{\textnormal{französisch: man hat
                        eine groteske Idee gehabt}}}\label{K_L02051_1h} – da ich sowohl das Rathausfest wie einem
                    von der Universität und den Schriftstellern veranstalteten ausschlug – einen
                    Saal der \textcolor{brown}{Kgl. Bibliotek}{}\ledrightnote{\textcolor{brown}{Det Kongelige Bibliotek}} zu einem \textcolor{brown}{G. B.-Archiv}{}\ledrightnote{\textcolor{brown}{Georg Brandes-arkiv}} zu verwandeln und mit meiner Büste
                    zu versehen.\pend
           \pstart
           Da sollen idiotische Literaturhistoriker der Zukunft in meinen alten
                    Liebesbriefen schnüffeln. Das soll mir Freude machen.\pend
           \pstart
           Glücklicherweise für Arthur S. halten wir noch immer dieselbe Distanz von
                    20 Jahren.\pend
           \pstart Ihr ergebenster \spacefill\mbox{Georg Brandes}\pend{}\endnumbering\briefempfaengerindex{Schnitzler, Olga@\textsc{Schnitzler, Olga}!zzzBrandes, Georg@\emph{von Georg Brandes}!1912-02-031@{3. 2. 1912}|)be}\briefempfaengerindex{Schnitzler, Arthur@\textsc{Schnitzler, Arthur}!zzzBrandes, Georg@\emph{von Georg Brandes}!1912-02-031@{3. 2. 1912}|)be}\mylabel{h}  \normalsize

\doendnotes{C}
\bigskip
\vfill

\clearpage

\footnotesize

\lohead{\textsc{register}}

% Definiere theindex-Environment komplett neu ohne reledmac
\makeatletter
\renewenvironment{theindex}{%
  \section*{\indexname}%
  \setlength{\parindent}{0pt}%
  \setlength{\parskip}{0pt plus 0.3pt}%
  \let\item\@idxitem
}{%
  \clearpage
}
\makeatother

\IfFileExists{\jobname-pw.ind}{\input{\jobname-pw.ind}}{}

\end{document}

      