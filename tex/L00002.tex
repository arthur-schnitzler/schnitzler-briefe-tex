%% latex-korrekturansicht-vorspann.tex
%% Vorspann für die Korrekturansicht.
%% Lädt die gemeinsame Datei latex-vorspann.tex mit gesetztem Schalter.

\newif\ifkorrekturansicht
\korrekturansichttrue

\input{../tex-inputs/latex-vorspann}


               \section[Jaques Joachim an Arthur Schnitzler, 31. 1. 1890]{ Jaques Joachim an Arthur Schnitzler, 31. 1. 1890}\nopagebreak\mylabel{v}\rehead{ }\normalsize\beginnumbering\briefempfaengerindex{Schnitzler, Arthur@\textsc{Schnitzler, Arthur}!zzzJoachim, Jaques@\emph{von Jaques Joachim}!1890-01-311@{31. 1. 1890}|(be} \toendnotes[C]{\smallbreak\pagebreak[2]} \Standort{DLA, A:Schnitzler, HS.NZ85.1.3571,1.}
\physDesc{Brief, 2 Blätter, 2 Seiten
\newline{}Handschrift: schwarze Tinte, lateinische Kurrent
\newline{}Schnitzler: 1) mit Bleistift beschriftet: »\textsc{DrJoachim}« 2) mit rotem Buntstift »\textcolor{brown}{Mod Dicht}« und zwei Unterstreichungen}\toendnotes[C]{\smallbreak}\pstart
           \raggedleft{}{\pb}\textcolor{pink}{Wien I Fr. J. Quai 31}{}\ledrightnote{\textcolor{pink}{Franz-Josefs-Kai}}{\\}31. Januar 1890\pend
           \pstart{}Sehr geehrter Herr Doctor!\pend\pstart
           Unter Berufung auf Herrn D\textsuperscript{r}{ }\textcolor{blue}{Goldmann}{}\ledrightnote{\textcolor{blue}{Paul Goldmann}} erlaube ich mir als
               Redactions-Mitglied der in \textcolor{pink}{Brünn}{}\ledrightnote{\textcolor{pink}{Brünn}} erscheinenden
                  \label{K_L00002_1v}\edtext{neuen Zeitschrift}{\lemma{\textnormal{\emph{neuen Zeitschrift}}}\Cendnote{\textnormal{Das erste Heft der \emph{\textcolor{green}{Modernen Dichtung}} war am 1. 1. 1891
                  erschienen.}}}\label{K_L00002_1h} »\textcolor{green}{Moderne Dichtung}{}\ledrightnote{\textcolor{green}{Moderne Dichtung. Monatsschrift für Literatur und Kritik}}« zur
               Mitarbeiterschaft an derselben aufzufordern. Herr D\textsuperscript{r}{ }\textcolor{blue}{Goldmann}{}\ledrightnote{\textcolor{blue}{Paul Goldmann}}{ }{\pb}theilte mir freundlichst mit, daß Sie eine Novelle
                  \label{K_L00002_2v}\edtext{»\textcolor{green}{Belastet}{}\ledrightnote{\textcolor{green}{Belastet}}«}{\lemma{\textnormal{\emph{»Belastet«}}}\Cendnote{\textnormal{Die Novelle blieb zu
                  Lebzeiten Schnitzlers ungedruckt. Eine Inhaltsangabe findet sich in \emph{\textcolor{green}{Jugend in Wien}}.}}}\label{K_L00002_2h} und einen \textcolor{green}{Cyclus}{}\ledrightnote{→\textcolor{green}{Anatol}} von Einaktern
               geschrieben haben – ich wäre sehr erfreut, wenn Sie sich entschliessen würden mir
               selbe bald zu übersenden.\pend
           \pstart
           Hochachtungsvoll{\\[\baselineskip]}\spacefill\mbox{D\textsuperscript{r}JJoachim}\pend
           \leftskip=0em{}\endnumbering\briefempfaengerindex{Schnitzler, Arthur@\textsc{Schnitzler, Arthur}!zzzJoachim, Jaques@\emph{von Jaques Joachim}!1890-01-311@{31. 1. 1890}|)be}\mylabel{h}  \normalsize

\doendnotes{C}
\bigskip
\vfill

\clearpage

\footnotesize

\lohead{\textsc{register}}

% Definiere theindex-Environment komplett neu ohne reledmac
\makeatletter
\renewenvironment{theindex}{%
  \section*{\indexname}%
  \setlength{\parindent}{0pt}%
  \setlength{\parskip}{0pt plus 0.3pt}%
  \let\item\@idxitem
}{%
  \clearpage
}
\makeatother

\IfFileExists{\jobname-pw.ind}{\input{\jobname-pw.ind}}{}

\end{document}

      