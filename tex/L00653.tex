%% latex-korrekturansicht-vorspann.tex
%% Vorspann für die Korrekturansicht.
%% Lädt die gemeinsame Datei latex-vorspann.tex mit gesetztem Schalter.

\newif\ifkorrekturansicht
\korrekturansichttrue

\input{../tex-inputs/latex-vorspann}


               \section[Arthur Schnitzler an Richard Beer-Hofmann, 17. 3. 1897]{ Arthur Schnitzler an Richard Beer-Hofmann, 17. 3. 1897}\nopagebreak\mylabel{v}\rehead{ }\normalsize\beginnumbering\briefempfaengerindex{Beer-Hofmann, Richard@\textsc{Beer-Hofmann, Richard}!zzzSchnitzler, Arthur@\emph{von Arthur Schnitzler}!1897-03-171@{17. 3. 1897}|(be} \toendnotes[C]{\smallbreak\pagebreak[2]} \Standort{YCGL, MSS 31.}
\physDesc{Briefkarte, Umschlag
\newline{}Handschrift: Bleistift, deutsche Kurrent\newline{}Versand: 1) Rohrpost 2) Stempel: »\nobreak{}\oindex{IX., Alsergrund@\textbf{IX., Alsergrund}, \emph{Bezirk (A.BZK)}|pwk}Wien {[}9/3{]}, \textcolor{gray}{17}{[}. 3. 1897{]}, 4 10 N\nobreak{}«. 3) Stempel: »\nobreak{}\oindex{I., Innere Stadt@\textbf{I., Innere Stadt}, \emph{Bezirk (A.BZK)}|pwk}Wien 1/1, 17 III \textcolor{gray}{9}7, 4 40 N\nobreak{}«. }\toendnotes[C]{\smallbreak}\pstart{}{\pb}Dr. \textsc{Rich.
                            Beer-Hofmann}\pend{}\pstart{}\textcolor{pink}{Wien}{}\ledrightnote{\textcolor{pink}{Wien}}\pend{}\pstart{}\textsc{\textcolor{pink}{I. Wollzeile 15}{}\ledrightnote{\textcolor{pink}{Wollzeile}}.}\pend{}{\bigskip}\pstart
           \noindent{}{\pb}lieber Richard, ſehr ſchön wär es, we{\geminationn} Sie ſchon \uline{zum}
                    Nachtmahl kämen, aber pünktlich \label{K_L00653_1v}\edtext{\textcolor{gray}{¼} 9}{\lemma{\textnormal{\emph{¼ 9}}}\Cendnote{\textnormal{20 Uhr 15}}}\label{K_L00653_1h}. Nur \textcolor{blue}{Hugo}{}\ledrightnote{\textcolor{blue}{Hugo von Hofmannsthal}} u der \textcolor{blue}{Georg {\pb}Hirſchf.}{}\ledrightnote{\textcolor{blue}{Georg Hirschfeld}}{ }ſind zum Nachtmahl da. Antworten müſſen Sie
                    nicht, aber ko{\geminationm}en.\pend
           \pstart Herzlich Ihr \spacefill\mbox{Arthur}\pend{}\endnumbering\briefempfaengerindex{Beer-Hofmann, Richard@\textsc{Beer-Hofmann, Richard}!zzzSchnitzler, Arthur@\emph{von Arthur Schnitzler}!1897-03-171@{17. 3. 1897}|)be}\mylabel{h}  \normalsize

\doendnotes{C}
\bigskip
\vfill

\clearpage

\footnotesize

\lohead{\textsc{register}}

% Definiere theindex-Environment komplett neu ohne reledmac
\makeatletter
\renewenvironment{theindex}{%
  \section*{\indexname}%
  \setlength{\parindent}{0pt}%
  \setlength{\parskip}{0pt plus 0.3pt}%
  \let\item\@idxitem
}{%
  \clearpage
}
\makeatother

\IfFileExists{\jobname-pw.ind}{\input{\jobname-pw.ind}}{}

\end{document}

      