%% latex-korrekturansicht-vorspann.tex
%% Vorspann für die Korrekturansicht.
%% Lädt die gemeinsame Datei latex-vorspann.tex mit gesetztem Schalter.

\newif\ifkorrekturansicht
\korrekturansichttrue

\input{../tex-inputs/latex-vorspann}


               \section[Arthur Schnitzler an Hugo Hofmannsthal, 9. 10. 1925]{ Arthur Schnitzler an Hugo Hofmannsthal, 9. 10. 1925}\nopagebreak\mylabel{v}\rehead{ }\normalsize\beginnumbering\briefempfaengerindex{Hofmannsthal, Hugo von@\textsc{Hofmannsthal, Hugo von}!zzzSchnitzler, Arthur@\emph{von Arthur Schnitzler}!1925-10-091@{9. 10. 1925}|(be} \toendnotes[C]{\smallbreak\pagebreak[2]} \Standort{FDH, Hs-30885,153.}
\physDesc{Postkarte
\newline{}Handschrift: Bleistift, lateinische Kurrent\newline{}Versand: Stempel: »\nobreak{}\oindex{XVIII., Waehring@\textbf{XVIII., Währing}, \emph{Bezirk (A.BZK)}|pwk}18/1 Wien, 10. X. 25, 1\textcolor{gray}{8}\nobreak{}«.  }\buchAbdrucke{\weitereDrucke{Hugo von Hofmannsthal, Arthur Schnitzler: \emph{Briefwechsel}. Hg. Therese Nickl und Heinrich Schnitzler. Frankfurt am Main: \emph{S. Fischer} 1964, S. 302.} }\toendnotes[C]{\smallbreak}\pstart{}{\pb}\label{T_L02453-1v}\edtext{\textcolor{gray}{\textbf{A. S.}}}{\lemma{\textnormal{\emph{A. S.}}}\Cendnote{\textnormal{ovaler Absenderkleber}}}\label{T_L02453-1h}\pend{}\pstart{}\textcolor{pink}{\textcolor{gray}{\textbf{WIEN, XVIII.}}}{}\ledrightnote{\textcolor{pink}{XVIII., Währing}}\pend{}\pstart{}\textcolor{pink}{\textcolor{gray}{\textbf{STERNWARTESTR. 71}}}{}\ledrightnote{\textcolor{pink}{Sternwartestraße}}\pend{}{\bigskip}\pstart{}Hrn Hugo v Hofmannsthal\pend{}\pstart{}\textcolor{pink}{Bad Aussee}{}\ledrightnote{\textcolor{pink}{Bad Aussee}}\pend{}\pstart{}\textcolor{pink}{Ramgut}{}\ledrightnote{\textcolor{pink}{Ramgut}}.\pend{}{\bigskip}\pstart
           \raggedleft{}{\pb}\textcolor{pink}{Wien}{}\ledrightnote{\textcolor{pink}{Wien}}, 9. X. 1925\pend
           \pstart
           mein lieber Hugo, So{\geminationn}tag fahre ich nach \textcolor{pink}{Berlin}{}\ledrightnote{\textcolor{pink}{Berlin}}, (\textcolor{pink}{Hotel Esplanade}{}\ledrightnote{\textcolor{pink}{Hotel Esplanade}}) –
               schicken Sie den \textcolor{green}{Thurm}{}\ledrightnote{\textcolor{green}{Der Turm. Ein Trauerspiel}} gleich ab, so findet er mich
               dort, da ich wohl mindestens 8 Tage dort bleibe. Unter anderm werd ich dort
                  \label{K_L02453_1v}\edtext{\textcolor{blue}{Heini}{}\ledrightnote{\textcolor{blue}{Heinrich Schnitzler}} als \textcolor{green}{Theodor}{}\ledrightnote{→\textcolor{green}{Liebelei. Schauspiel in drei Akten}}}{\lemma{\textnormal{\emph{Heini als Theodor}}}\Cendnote{\textnormal{siehe A. S.: \emph{Tagebuch}, 13. 10. 1925}}}\label{K_L02453_1h} in der \textcolor{green}{Liebelei}{}\ledrightnote{\textcolor{green}{Liebelei. Schauspiel in drei Akten}}{ }sehen (die \uline{heute}
               vor 30 Jahren in \textcolor{pink}{Wien}{}\ledrightnote{\textcolor{pink}{Wien}} zum »überhaupt« ersten Mal
               aufgeführt wurde.) Auch ein neues Stück nehm ich nach \textcolor{pink}{Berlin}{}\ledrightnote{\textcolor{pink}{Berlin}} mit, in Versen, und heißt: {[}»{]}\textcolor{green}{Der Gang zum Weiher}{}\ledrightnote{\textcolor{green}{Der Gang zum Weiher. Dramatische Dichtung}}«{[}.{]} Gegen
               die Aufführg von \textcolor{green}{Kom. d. Verf.}{}\ledrightnote{\textcolor{green}{Komödie der Verführung. In drei Akten}} bei \textcolor{blue}{Barnowsky}{}\ledrightnote{\textcolor{blue}{Victor Barnowsky}}{ }setze ich mich zur Wehre – (die Hauptrollen
               scheinen nemlich noch nicht besetzt zu sein.) Auch eine »\textcolor{green}{Traumnovelle}{}\ledrightnote{\textcolor{green}{Traumnovelle}}« (so heißt sie) erscheint nächstens. – Von \textcolor{pink}{Forte dei Marmi}{}\ledrightnote{\textcolor{pink}{Forte dei Marmi}} bin ich nach \textcolor{pink}{Florenz}{}\ledrightnote{\textcolor{pink}{Florenz}}, nach \textcolor{pink}{Venedig}{}\ledrightnote{\textcolor{pink}{Venedig}}; und vor 3 Wochen
               nach \textcolor{pink}{Wien}{}\ledrightnote{\textcolor{pink}{Wien}}. Hoffentlich sieht man {\pb}sich einmal wieder – und bald. (Es wird immer später.)
                  \textcolor{blue}{Christiane}{}\ledrightnote{\textcolor{blue}{Christiane von Hofmannsthal}}{ }sah ich in \textcolor{pink}{Venedig}{}\ledrightnote{\textcolor{pink}{Venedig}}; ich glaube, \textcolor{blue}{Lili}{}\ledrightnote{\textcolor{blue}{Lili Schnitzler}} u \textcolor{blue}{Olga}{}\ledrightnote{\textcolor{blue}{Olga Schnitzler}} haben sie nach meiner Abreise auch gesprochen. –\pend
           \pstart
           Nichts von alldem ahnten wir heute vor 30 Jahren. Und eigentlich war es gestern.\pend
           \pstart
           Leben Sie wohl.\pend
           \pstart
           In Herzlichkeit Ihr{\\[\baselineskip]}\spacefill\mbox{A.}\pend
           \leftskip=0em{}\endnumbering\briefempfaengerindex{Hofmannsthal, Hugo von@\textsc{Hofmannsthal, Hugo von}!zzzSchnitzler, Arthur@\emph{von Arthur Schnitzler}!1925-10-091@{9. 10. 1925}|)be}\mylabel{h}  \normalsize

\doendnotes{C}
\bigskip
\vfill

\clearpage

\footnotesize

\lohead{\textsc{register}}

% Definiere theindex-Environment komplett neu ohne reledmac
\makeatletter
\renewenvironment{theindex}{%
  \section*{\indexname}%
  \setlength{\parindent}{0pt}%
  \setlength{\parskip}{0pt plus 0.3pt}%
  \let\item\@idxitem
}{%
  \clearpage
}
\makeatother

\IfFileExists{\jobname-pw.ind}{\input{\jobname-pw.ind}}{}

\end{document}

      