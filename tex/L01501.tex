%% latex-korrekturansicht-vorspann.tex
%% Vorspann für die Korrekturansicht.
%% Lädt die gemeinsame Datei latex-vorspann.tex mit gesetztem Schalter.

\newif\ifkorrekturansicht
\korrekturansichttrue

\input{../tex-inputs/latex-vorspann}


               \section[Hugo von Hofmannsthal an Arthur Schnitzler, 16. 2. 1905]{ Hugo von Hofmannsthal an Arthur Schnitzler, 16. 2. 1905}\nopagebreak\mylabel{v}\rehead{ }\normalsize\beginnumbering\briefempfaengerindex{Schnitzler, Arthur@\textsc{Schnitzler, Arthur}!zzzHofmannsthal, Hugo von@\emph{von Hugo von Hofmannsthal}!1905-02-161@{16. 2. 1905}|(be} \toendnotes[C]{\smallbreak\pagebreak[2]} \Standort{CUL, Schnitzler, B 43.}
\physDesc{Postkarte
\newline{}Handschrift: schwarze Tinte, deutsche Kurrent\newline{}Versand: 1) Stempel: »\nobreak{}\oindex{Rodaun@\textbf{Rodaun}, \emph{Teil eines besiedelten Ortes (A.BSOX)}|pwk}Rodaun, 17. 2. 05, 9–12V\nobreak{}«.  2) Stempel: »\nobreak{}\oindex{XVIII., Waehring@\textbf{XVIII., Währing}, \emph{Bezirk (A.BZK)}|pwk}18/1 Wien 110, 17. 2. 05, 3.N, Bestellt\nobreak{}«. \newline{}Ordnung: 1) mit Bleistift von unbekannter Hand nummeriert: »\strikeout{223}« 2) mit Bleistift von unbekannter Hand  nummeriert: »249«}\buchAbdrucke{\weitereDrucke{Hugo von Hofmannsthal, Arthur Schnitzler: \emph{Briefwechsel}. Hg. Therese Nickl und Heinrich Schnitzler. Frankfurt am Main: \emph{S. Fischer} 1964, S. 210.} }\toendnotes[C]{\smallbreak}\pstart{}{\pb}\textsc{Herrn D\textsuperscript{r} Arthur Schnitzler}\pend{}\pstart{}\textcolor{pink}{\textsc{Wien}}{}\ledrightnote{\textcolor{pink}{Wien}}\pend{}\pstart{}\textcolor{pink}{\textsc{XVIII Spöttelgasse 7}}{}\ledrightnote{\textcolor{pink}{Edmund-Weiß-Gasse}}\pend{}{\bigskip}\pstart
           \raggedleft{}{\pb}16 II.\pend
           \pstart
           Höre, Ihr ko{\geminationm}t \label{K_L01501_1v}\edtext{Sonntag zu \textcolor{blue}{Waſſermanns}{}\ledrightnote{\textcolor{blue}{Jakob Wassermann}{\newline}\textcolor{blue}{Julie Wassermann}}}{\lemma{\textnormal{\emph{Sonntag zu Waſſermanns}}}\Cendnote{\textnormal{Das Treffen fand nicht statt.}}}\label{K_L01501_1h}. Wie ko{\geminationm}ts, daſs Ihr dort schon öfter wart und nie nach \textcolor{pink}{Rodaun}{}\ledrightnote{\textcolor{pink}{Rodaun}} ko{\geminationm}t.\pend
           \pstart
           Vielleicht ko{\geminationm}e ich Sonntag auf eine Stunde
               vor dem Nachtmahl hin. (Zu \textcolor{blue}{W.}{}\ledrightnote{\textcolor{blue}{Jakob Wassermann}{\newline}\textcolor{blue}{Julie Wassermann}})\hspace*{1.5em}Welcher Tag nächſter Woche würde Euch paſſen, daſs wir
               zu Euch ko{\geminationm}en?\pend
           \pstart
           Herzlich{\\[\baselineskip]}\spacefill\mbox{Hugo.}\pend
           \leftskip=0em{}\endnumbering\briefempfaengerindex{Schnitzler, Arthur@\textsc{Schnitzler, Arthur}!zzzHofmannsthal, Hugo von@\emph{von Hugo von Hofmannsthal}!1905-02-161@{16. 2. 1905}|)be}\mylabel{h}  \normalsize

\doendnotes{C}
\bigskip
\vfill

\clearpage

\footnotesize

\lohead{\textsc{register}}

% Definiere theindex-Environment komplett neu ohne reledmac
\makeatletter
\renewenvironment{theindex}{%
  \section*{\indexname}%
  \setlength{\parindent}{0pt}%
  \setlength{\parskip}{0pt plus 0.3pt}%
  \let\item\@idxitem
}{%
  \clearpage
}
\makeatother

\IfFileExists{\jobname-pw.ind}{\input{\jobname-pw.ind}}{}

\end{document}

      