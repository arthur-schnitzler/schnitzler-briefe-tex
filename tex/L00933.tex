%% latex-korrekturansicht-vorspann.tex
%% Vorspann für die Korrekturansicht.
%% Lädt die gemeinsame Datei latex-vorspann.tex mit gesetztem Schalter.

\newif\ifkorrekturansicht
\korrekturansichttrue

\input{../tex-inputs/latex-vorspann}


               \section[Arthur Schnitzler an Richard Beer-Hofmann, 6. 7. 1899]{ Arthur Schnitzler an Richard Beer-Hofmann, 6. 7. 1899}\nopagebreak\mylabel{v}\rehead{ }\normalsize\beginnumbering\briefempfaengerindex{Beer-Hofmann, Richard@\textsc{Beer-Hofmann, Richard}!zzzSchnitzler, Arthur@\emph{von Arthur Schnitzler}!1899-07-061@{6. 7. 1899}|(be} \toendnotes[C]{\smallbreak\pagebreak[2]} \Standort{YCGL, MSS 31.}
\physDesc{Brief, 1 Blatt, 2 Seiten, Umschlag
\newline{}Handschrift: Bleistift, deutsche Kurrent\newline{}Versand: 1) Stempel: »\nobreak{}\oindex{I., Innere Stadt@\textbf{I., Innere Stadt}, \emph{Bezirk (A.BZK)}|pwk}Wien 1/1, 6. 7. 99, 2–3N\nobreak{}«.  2) Stempel: »\nobreak{}\oindex{Seeboden@\textbf{Seeboden}, \emph{http://www.geonames.org/ontologyA.ADM3}|pwk}{\pb}Seeboden, \textcolor{gray}{7}. 7. 99\nobreak{}«. }\buchAbdrucke{\weitereDrucke{Arthur Schnitzler, Richard Beer-Hofmann: \emph{Briefwechsel 1891–1931}. Hg. Konstanze Fliedl. Wien, Zürich: \emph{Europaverlag} 1992, S. 131.} }\toendnotes[C]{\smallbreak}\pstart{}{\pb}\textcolor{pink}{\textsc{Kärnthen}}{}\ledrightnote{\textcolor{pink}{Kärnten}}\pend{}\pstart{}\textsc{Herrn Dr. Rich. Beer-Hofmann}\pend{}\pstart{}\textcolor{pink}{\textsc{Villa Platzer}}{}\ledrightnote{\textcolor{pink}{Villa Platzer}}\pend{}\pstart{}\textcolor{pink}{\textsc{Seeboden am Millstätter}ſee}{}\ledrightnote{\textcolor{pink}{Seeboden}}\pend{}{\bigskip}\pstart
           \raggedleft{}{\pb}6/7 99\pend
           \pstart
           lieber, \textcolor{blue}{Mayer}{}\ledrightnote{\textcolor{blue}{Oskar Mayer}} ko{\geminationm}t ja
               keineswegs mit; hat ers Ihnen noch nicht geſchrieben?\pend
           \pstart
           – Ich ko{\geminationm}e Mitte Juli nach \textcolor{pink}{\textsc{Velden}}{}\ledrightnote{\textcolor{pink}{Velden}} zu meiner \textcolor{blue}{Mama}{}\ledrightnote{→\textcolor{blue}{Louise Schnitzler}}, beſuch
               Sie dann gleich (oder Sie mich?) wir beſprechen dann näheres.\pend
           \pstart
           Eigentlich möchte ich am {\pb}31. Juli in \textcolor{pink}{\textsc{Bayreuth}}{}\ledrightnote{\textcolor{pink}{Bayreuth}} zu \textcolor{green}{\textsc{Parsifal}}{}\ledrightnote{\textcolor{green}{Parsifal}}{ }ſein.\pend
           \pstart
           Es ärgert mich dſs Sie mir mit keinem Wort ſchreiben was Sie thun oder nicht
               thun.\pend
           \pstart
           – Den Todten muſs es ſehr komiſch vorkommen, was wir »Erleben« nennen. – \pend
           \pstart Herzlichſt Ihr \spacefill\mbox{Arthur}\pend{}\endnumbering\briefempfaengerindex{Beer-Hofmann, Richard@\textsc{Beer-Hofmann, Richard}!zzzSchnitzler, Arthur@\emph{von Arthur Schnitzler}!1899-07-061@{6. 7. 1899}|)be}\mylabel{h}  \normalsize

\doendnotes{C}
\bigskip
\vfill

\clearpage

\footnotesize

\lohead{\textsc{register}}

% Definiere theindex-Environment komplett neu ohne reledmac
\makeatletter
\renewenvironment{theindex}{%
  \section*{\indexname}%
  \setlength{\parindent}{0pt}%
  \setlength{\parskip}{0pt plus 0.3pt}%
  \let\item\@idxitem
}{%
  \clearpage
}
\makeatother

\IfFileExists{\jobname-pw.ind}{\input{\jobname-pw.ind}}{}

\end{document}

      