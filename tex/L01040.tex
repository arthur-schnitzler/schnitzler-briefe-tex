%% latex-korrekturansicht-vorspann.tex
%% Vorspann für die Korrekturansicht.
%% Lädt die gemeinsame Datei latex-vorspann.tex mit gesetztem Schalter.

\newif\ifkorrekturansicht
\korrekturansichttrue

\input{../tex-inputs/latex-vorspann}


               \section[Adalbert Seligmann an Arthur Schnitzler, 25. 5. 1900]{ Adalbert Seligmann an Arthur Schnitzler, 25. 5. 1900}\nopagebreak\mylabel{v}\rehead{ }\normalsize\beginnumbering\briefempfaengerindex{Schnitzler, Arthur@\textsc{Schnitzler, Arthur}!zzzSeligmann, Adalbert Franz@\emph{von Adalbert Franz Seligmann}!1900-05-251@{25. 5. 1900}|(be} \toendnotes[C]{\smallbreak\pagebreak[2]} \Standort{TMW, HS Schn 4/61/1.}
\physDesc{Brief, 1 Blatt, 1 Seite
\newline{}Handschrift: schwarze Tinte, deutsche Kurrent
\newline{}Schnitzler: mit Bleistift beschriftet: »\textsc{Seligma{\geminationn}}« und nummeriert: »2« }\toendnotes[C]{\smallbreak}\pstart
           \raggedleft{}{\pb}25/5 1900\pend
           \pstart
           Verehrter Freund! Meinen beſten Dank für die \textcolor{green}{Sendung}{}\ledrightnote{→\textcolor{green}{Reigen. Zehn Dialoge}}! Die Sachen ſind wirklich
                    reizend, und eine Fülle der verſchiedenſten Beobachtungen iſt darin, die aus
                    Mittheilungen nicht geſchöpft ſein können. Das muß man ſelber mitgemacht haben –
                    ich gratulire Ihnen noch nachträglich dazu!\pend
           \pstart
           Nochmals herzlich dankend{\\[\baselineskip]}Ihr{\\[\baselineskip]}\spacefill\mbox{Seligmann}\pend
           \leftskip=0em{}\endnumbering\briefempfaengerindex{Schnitzler, Arthur@\textsc{Schnitzler, Arthur}!zzzSeligmann, Adalbert Franz@\emph{von Adalbert Franz Seligmann}!1900-05-251@{25. 5. 1900}|)be}\mylabel{h}  \normalsize

\doendnotes{C}
\bigskip
\vfill

\clearpage

\footnotesize

\lohead{\textsc{register}}

% Definiere theindex-Environment komplett neu ohne reledmac
\makeatletter
\renewenvironment{theindex}{%
  \section*{\indexname}%
  \setlength{\parindent}{0pt}%
  \setlength{\parskip}{0pt plus 0.3pt}%
  \let\item\@idxitem
}{%
  \clearpage
}
\makeatother

\IfFileExists{\jobname-pw.ind}{\input{\jobname-pw.ind}}{}

\end{document}

      