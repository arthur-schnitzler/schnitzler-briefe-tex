%% latex-korrekturansicht-vorspann.tex
%% Vorspann für die Korrekturansicht.
%% Lädt die gemeinsame Datei latex-vorspann.tex mit gesetztem Schalter.

\newif\ifkorrekturansicht
\korrekturansichttrue

\input{../tex-inputs/latex-vorspann}


               \section[Therese Rie-Andro an Arthur Schnitzler, 3. 5. 1923]{ Therese Rie-Andro an Arthur Schnitzler, 3. 5. 1923}\nopagebreak\mylabel{v}\rehead{ }\normalsize\beginnumbering\briefempfaengerindex{Schnitzler, Arthur@\textsc{Schnitzler, Arthur}!zzzRie, Therese@\emph{von Therese Rie}!1923-05-031@{3. 5. 1923}|(be} \toendnotes[C]{\smallbreak\pagebreak[2]} \Standort{DLA, A:Schnitzler, 85.1.4310.}
\physDesc{Brief, 2 Blätter (das zweite Blatt mit »II.« paginiert), 4 Seiten
\newline{}Handschrift: blaue Tinte, lateinische Kurrent
\newline{}Schnitzler: 1) mit Bleistift beschriftet: »\textsc{Andro}« 2) mit rotem Buntstift vier Unterstreichungen\newline{}Ordnung: mit Bleistift von unbekannter Hand den Artikel »eine« vor
                                    »Schokoladebonbon« durch Streichung des Schluss-e
                                 angepasst }\toendnotes[C]{\smallbreak}\pstart
           \raggedleft{}{\pb}\textcolor{pink}{Wien}{}\ledrightnote{\textcolor{pink}{Wien}}, 3. Mai 1923.\pend
           \pstart
           \raggedleft{}\textcolor{pink}{IV, Schönburgſtr. 48}{}\ledrightnote{\textcolor{pink}{Schönburgstraße}}.\pend
           \pstart{}Verehrter Herr Doktor,\pend\pstart
           wie sehr mich Ihre guten und lieben \label{K_L02572-2v}\edtext{Worte}{\lemma{\textnormal{\emph{Worte}}}\Cendnote{\textnormal{Sie reagiert hier auf eine
                  nicht überlieferte Karte \textcolor{blue}{Schnitzler}s, in der
                  dieser ihr zu einer Arbeit gratuliert haben dürfte. Es bietet sich unmittelbar
                  keine Buchausgabe an. Eventuell hat er ihre Besprechung von \textcolor{blue}{Stefan Zweig}s \emph{\textcolor{green}{Amok}} gelesen,
                  in der auch von »fernen \textcolor{green}{Anatol}-Tagen« die Rede ist. (L. Andro: \emph{\textcolor{green}{Von neuen Büchern. Amok}}. In: \emph{\textcolor{green}{Neues Wiener Abendblatt}}, Jg. 56, Nr. 325, 5. 12. 1922,
                     S. 4.)}}}\label{K_L02572-2h} erfreut haben, kann ich Ihnen schwer schildern; denn Sie
               sind es ja gewesen, der mich und meine ganze Generation künstlerisch gesäugt hat –
               die Kühnheit dieses Bildes bedrückt Sie hoffentlich nicht! – und es iſt kaum
               vorstellbar, was aus uns geworden wäre, wenn wie Sie, \textcolor{blue}{Gustav Mahler}{}\ledrightnote{\textcolor{blue}{Gustav Mahler}} und \textcolor{blue}{Hugo Wolf}{}\ledrightnote{\textcolor{blue}{Hugo Wolf}} nicht gehabt
               hätten, zu denen ich als Reſpondizierenden auch noch \textcolor{blue}{Kainz}{}\ledrightnote{\textcolor{blue}{Josef Kainz}} rechnen möchte. Ich bin mein ganzes Leben lang \strikeout{\textcolor{gray}{viel}} mit Ihren Gestalten umgeben gewesen und namentlich \textcolor{green}{Herr v. Sala}{}\ledrightnote{→\textcolor{green}{Der einsame Weg. Schauspiel in fünf Akten}} war es, der mich oft und oft auf
               meinen \textcolor{pink}{Wienerwald}{}\ledrightnote{\textcolor{pink}{Wienerwald}}-Spaziergängen begleitet hat. Es
               gibt kaum eine Frage meines Lebens, die ich nicht mit ihm durchgesprochen habe und
               oft habe ich mich auch über ihn ärgern müſſen, weil er gar nicht meiner Ansicht war
               und sich zuweilen in der nichtsnutzigsten Art über mich luſtig gemacht hat. Aber das
               war heilsam. Und das meiſtzitierte Werk in meinem Hause iſt jedenfalls »\textcolor{green}{Literatur}{}\ledrightnote{\textcolor{green}{Literatur}}« gewesen, das mich, so hoffe ich
               wenigſtens, vor mancher kleinen Geschmacksentgleisung bewahrt hat. So haben Sie also
               auch noch ungemein pädagogiſch gewirkt!\pend
           \pstart
           {\pb}Manches Jahr habe ich mir gewünscht, Ihnen das einmal
               persönlich zu sagen, dann aber davon absehen gelernt. Denn es wäre nur auf Grund
               gemeinsamer gesellschaftlicher Beziehungen möglich gewesen und davon halte ich nicht
               sehr viel. Es ko{\geminationm}t dabei kaum jemals etwas Menschliches
               heraus und wird schließlich nur zu einer Serie von Verlegenheiten. Und am Ende iſt es
               einem Künſtler wol lieber, wenn die Saat, die er in andern gesät hat, zu einer, wenn
               auch noch so bescheidenen Frucht reift, als wenn ihm \uline{noch} eine Dame versichert, wie sehr sie seine Werke bewundere! – –\pend
           \pstart
           Nur der freundliche Passus in Ihrer Karte: Sie wollten auch meine andern Arbeiten
               kennnen lernen, veranlaßt mich, Ihnen mein kleines Buch »\textcolor{green}{Die Komödiantin Dora X.}{}\ledrightnote{\textcolor{green}{Die Komödiantin Dora X. Roman}}« zu schicken; sonſt bin ich nicht so, daß ich die
               Menschen mit meiner Literatur überschütte. Das Büchlein bitte ich Sie\strikeout{,} aber nur als \label{K_L02572-1v}\edtext{Eisenbahnlektüre}{\lemma{\textnormal{\emph{Eisenbahnlektüre}}}\Cendnote{\textnormal{siehe A. S.: \emph{Tagebuch}, 7. 5. 1923}}}\label{K_L02572-1h} zu verwenden; zu viel mehr taugt es nicht. Es iſt ein nicht sehr tiefes
               Problem, nicht sehr tief gefaßt und für mich höchstens dadurch bemerkenswert, daß es
               Jahre später in meiner Umgebung ziemlich wahr geworden iſt. Wie es denn offenbar den
               meiſten Schreibenden, den Kleinen wie den Großen, so ergeht, {\pb}daß sie meinen, das Leben abzuschreiben, während es schließlich das Leben iſt, daß
                  \uline{sie} ganz munter plagiiert. – –\pend
           \pstart
           Wenn ich aber vorhin von gemeinsamen Beziehungen sprach, die ich nicht für so wichtig
               halte, so möchte ich doch einer gedenken, die mir lieb und teuer iſt und an die ich
               denken muſs, so oft ich Ihren Namen höre: der Erinnerung an Ihre \textcolor{blue}{Eltern}{}\ledrightnote{→\textcolor{blue}{Louise Schnitzler}{\newline}→\textcolor{blue}{Johann Schnitzler}}, die ich beide noch gekannt habe
               und namentlich an Ihren \textcolor{blue}{Vater}{}\ledrightnote{→\textcolor{blue}{Johann Schnitzler}},
               der meine früheſte Kindheitserinnerung bildet. Man sagte mir, daſs er mich als
               3jähriges Kind von einer schweren Diphteritis errettet habe und es iſt meine erſte
               Erinnerung überhaupt, wie er mir i{\geminationm}er eine
               Schokolodebonbon auf einen Löffel Chinin tat, daſs ich das bittere Zeug nehmen
               sollte. Wieviel iſt seither vorbeigegangen und vergessen worden, aber das Bild iſt
               mir geblieben! – – Im Nachlaß meiner \textcolor{blue}{Eltern}{}\ledrightnote{→\textcolor{blue}{Maximilian Herz}{\newline}→\textcolor{blue}{Marie Herz}} fand ich später ein Tagebuch meines \textcolor{blue}{Vaters}{}\ledrightnote{→\textcolor{blue}{Maximilian Herz}} aus dem Jahre 1863, in
               welchem viel von \introOben{}einem Briefwechsel mit\introOben{} dem Ihren die Rede
               iſt – sie waren ja Kollegen, wie ich weiß, schon vom \label{K_L02572-4v}\edtext{\textcolor{brown}{Schottengymnasium}{}\ledrightnote{\textcolor{brown}{Schottengymnasium}}}{\lemma{\textnormal{\emph{Schottengymnasium}}}\Cendnote{\textnormal{\textcolor{blue}{Johann Schnitzler} kam erst zum Studium nach
                     \textcolor{pink}{Wien}.}}}\label{K_L02572-4h} her oder mindeſtens vom erſten
               Jahre Medizin. Ich habe oft nach Briefen gesucht, aber nichts gefunden – nur diese
               Karte fand ich einmal und schicke sie Ihnen. Trotz {\pb}des
               belanglosen Inhalts grüßt Sie vielleicht eine liebe und vertraute Schrift! –\pend
           \pstart
           Bitte, lächeln Sie nicht über diesen langen Brief als Antwort auf Ihre Karte – \textcolor{green}{Herr v. Sala}{}\ledrightnote{→\textcolor{green}{Der einsame Weg. Schauspiel in fünf Akten}} täte es, sein
               Schöpfer iſt hoffentlich milder – aber ich habe ihn jahrelang »verdrängt«, um mich
               ganz modern auszudrücken, und einmal mußte er doch geschrieben werden. Ihre
               freundlichen Worte sind ein Anlaſs dazu. Möchte Ihnen das silberschi{\geminationm}ernde \textcolor{pink}{Dänemark}{}\ledrightnote{\textcolor{pink}{Dänemark}} viel
               Liebes und Freundliches geben! Seien Sie nochmals bedankt und begrüßt von Ihrer\pend
           \pstart \spacefill\mbox{Therese Rie.}\pend{}\endnumbering\briefempfaengerindex{Schnitzler, Arthur@\textsc{Schnitzler, Arthur}!zzzRie, Therese@\emph{von Therese Rie}!1923-05-031@{3. 5. 1923}|)be}\mylabel{h}  \normalsize

\doendnotes{C}
\bigskip
\vfill

\clearpage

\footnotesize

\lohead{\textsc{register}}

% Definiere theindex-Environment komplett neu ohne reledmac
\makeatletter
\renewenvironment{theindex}{%
  \section*{\indexname}%
  \setlength{\parindent}{0pt}%
  \setlength{\parskip}{0pt plus 0.3pt}%
  \let\item\@idxitem
}{%
  \clearpage
}
\makeatother

\IfFileExists{\jobname-pw.ind}{\input{\jobname-pw.ind}}{}

\end{document}

      