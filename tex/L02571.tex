%% latex-korrekturansicht-vorspann.tex
%% Vorspann für die Korrekturansicht.
%% Lädt die gemeinsame Datei latex-vorspann.tex mit gesetztem Schalter.

\newif\ifkorrekturansicht
\korrekturansichttrue

\input{../tex-inputs/latex-vorspann}


               \section[Therese Rie-Andro an Arthur Schnitzler, 27. 1. 1913]{ Therese Rie-Andro an Arthur Schnitzler, 27. 1. 1913}\nopagebreak\mylabel{v}\rehead{ }\normalsize\beginnumbering\briefempfaengerindex{Schnitzler, Arthur@\textsc{Schnitzler, Arthur}!zzzRie, Therese@\emph{von Therese Rie}!1913-01-271@{27. 1. 1913}|(be} \toendnotes[C]{\smallbreak\pagebreak[2]} \Standort{DLA, A:Schnitzler, 85.1.4310.}
\physDesc{Brief, 1 Blatt, 2 Seiten
\newline{}Handschrift: blaue Tinte, lateinische Kurrent
\newline{}Schnitzler: 1) mit Bleistift beschriftet: »\textsc{Andro}« 2) mit rotem Buntstift zwei Unterstreichungen}\toendnotes[C]{\smallbreak}\pstart
           \raggedleft{}{\pb}\textcolor{pink}{Wien}{}\ledrightnote{\textcolor{pink}{Wien}}, d. 27. Januar 13.\pend
           \pstart
           \raggedleft{}\textcolor{pink}{IV, Schönburgſtr. 48}{}\ledrightnote{\textcolor{pink}{Schönburgstraße}}\pend
           \pstart{}Verehrter Herr Doktor,\pend\pstart
           Auf der Rückreise von \textcolor{pink}{Berlin}{}\ledrightnote{\textcolor{pink}{Berlin}} las ich den »\textcolor{green}{Weg ins Freie}{}\ledrightnote{\textcolor{green}{Der Weg ins Freie. Roman}}« so ungefähr zum sechsten Mal und wie
               jedesmal bei diesem merkwürdig reichen Buche fielen mir eine Menge neue, nicht
               erfaßte Dinge auf, diesmal besonders im letzten Teil. Dabei stieß ich auch auf eine
               kleine Bemerkung über \textcolor{green}{Melot}{}\ledrightnote{→\textcolor{green}{Tristan und Isolde}}, den
               von einem zweiten Sänger \substVorne{}\textsuperscript{D}\substDazwischen{}d\substHinten{}argestellt zu sehen \label{K_L02571-2v}\edtext{\textcolor{green}{Georg}{}\ledrightnote{→\textcolor{green}{Der Weg ins Freie. Roman}} sich ärgert}{\lemma{\textnormal{\emph{Georg sich ärgert}}}\Cendnote{\textnormal{»\textcolor{green}{und gar nicht einverstanden war
                        er damit, daß Melot, durch dessen Hand Tristan sterben mußte, hier von einem
                        Sänger zweiten Ranges dargestellt wurde, wie übrigens beinahe überall.}« (neuntes Kapitel).}}}\label{K_L02571-2h}. Da fiel mir ein, daß Sie sich
               für \textcolor{blue}{Pfitzner}{}\ledrightnote{\textcolor{blue}{Hans Pfitzner}} interessieren und daß von ihm ein
               feiner geistvoller \textcolor{green}{Aufsatz}{}\ledrightnote{→\textcolor{green}{Bühnentradition}}
               existiert, der ausführlich das begründet, was Sie \introOben{}in ganz
                  ähnlicher Auffassung\introOben{} in einem Satze andeuten. Ich grabe ihn also aus meinem
               Bücherschrank aus und schicke ihn an Sie – vielleicht kennen Sie ihn nicht und es
               macht {\pb}Ihnen Vergnügen, ihn zu lesen.\pend
           \pstart
           Vom \textcolor{green}{Palestrina}{}\ledrightnote{\textcolor{green}{Palestrina. Musikalische Legende in drei Akten}} weiß ich seit diesem So{\geminationm}er, wo ich \textcolor{blue}{Pf.}{}\ledrightnote{\textcolor{blue}{Hans Pfitzner}} in
                  \textcolor{pink}{Leipzig}{}\ledrightnote{\textcolor{pink}{Leipzig}} traf, nicht mehr viel, außer daß der
               1. Akt auch musikalisch fertig iſt. Weiter wird er wol inzwischen auch nicht geko{\geminationm}en sein, da er ja leider als \textcolor{brown}{Operndirektor}{}\ledrightnote{→\textcolor{brown}{Oper Straßburg}} tätig iſt – leider, da wir ja
               nichts davon haben; für die \textcolor{pink}{Straßburg}{}\ledrightnote{\textcolor{pink}{Straßburg}}er mag’s ja
               ganz hübsch sein.\pend
           \pstart
           Noch will ich Sie von zweien Ihrer Werke grüßen: vom »\textcolor{green}{Professor Bernhardi}{}\ledrightnote{\textcolor{green}{Professor Bernhardi. Komödie in fünf Akten}}«, von dem ich durch einen Zufall aber nur die erſten
               zwei Akte hörte; und vom »\textcolor{green}{Schleier der Pierrette}{}\ledrightnote{\textcolor{green}{Der Schleier der Pierrette}}«,
               den ich in \textcolor{pink}{Dresden}{}\ledrightnote{\textcolor{pink}{Dresden}}, bei der \label{K_L02571-1v}\edtext{Generalprobe}{\lemma{\textnormal{\emph{Generalprobe}}}\Cendnote{\textnormal{\emph{\textcolor{green}{Tante Simona}} hatte am 22. 1. 1913
                  Uraufführung und wurde gemeinsam mit \emph{\textcolor{green}{Schleier der
                     Pierrette}} gegeben. Entsprechend ist die Generalprobe einen oder zwei Tage
                  davor anzusetzen.}}}\label{K_L02571-1h} von \textcolor{blue}{Dóhnanyi}{}\ledrightnote{\textcolor{blue}{Ernst von Dohnányi}}s neuer
                  \textcolor{green}{Oper}{}\ledrightnote{→\textcolor{green}{Tante Simona. Komische Oper in einem Akt}} zu sehen bekam.\pend
           \pstart
           In alter herzlicher Bewunderung{\\[\baselineskip]}\spacefill\mbox{L. Andro.}{\\[\baselineskip]}(Therese Rie.) \pend
           \leftskip=0em{}\endnumbering\briefempfaengerindex{Schnitzler, Arthur@\textsc{Schnitzler, Arthur}!zzzRie, Therese@\emph{von Therese Rie}!1913-01-271@{27. 1. 1913}|)be}\mylabel{h}  \normalsize

\doendnotes{C}
\bigskip
\vfill

\clearpage

\footnotesize

\lohead{\textsc{register}}

% Definiere theindex-Environment komplett neu ohne reledmac
\makeatletter
\renewenvironment{theindex}{%
  \section*{\indexname}%
  \setlength{\parindent}{0pt}%
  \setlength{\parskip}{0pt plus 0.3pt}%
  \let\item\@idxitem
}{%
  \clearpage
}
\makeatother

\IfFileExists{\jobname-pw.ind}{\input{\jobname-pw.ind}}{}

\end{document}

      