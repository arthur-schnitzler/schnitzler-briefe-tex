%% latex-korrekturansicht-vorspann.tex
%% Vorspann für die Korrekturansicht.
%% Lädt die gemeinsame Datei latex-vorspann.tex mit gesetztem Schalter.

\newif\ifkorrekturansicht
\korrekturansichttrue

\input{../tex-inputs/latex-vorspann}


               \section[Richard Dehmel an Arthur Schnitzler, {[}18. 11. 1913?{]}]{ Richard Dehmel an Arthur Schnitzler, {[}18. 11. 1913?{]}}\nopagebreak\mylabel{v}\rehead{ }\normalsize\beginnumbering\briefempfaengerindex{Schnitzler, Arthur@\textsc{Schnitzler, Arthur}!zzzDehmel, Richard@\emph{von Richard Dehmel}!1913-11-181@{{[}18. 11. 1913?{]}}|(be} \toendnotes[C]{\smallbreak\pagebreak[2]} \Standort{CUL, Schnitzler, B 26.}
\physDesc{Brief, 1 Blatt, 3 Seiten
\newline{}Druck
\newline{}Schnitzler: 1) mit Bleistift beschrieben: »\textsc{Dehmel}« 2) mit rotem Buntstift: »\textsc{\uline{(nicht abschr!)}}«\newline{}Ordnung: mit Bleistift von unbekannter Hand datiert: »1913« \newline{}Zusatz: Im Nachlass von \textcolor{blue}{Martin
                              Sturm} (Heinrich-Heine-Institut, Düsseldorf,
                              HHI.94.5036.281) findet sich der gleiche Druck einschließlich
                           des Briefumschlags, der genau am Tag des 50. Geburtstages, am
                              18. 11. 1913 in \textcolor{pink}{Blankenese} gestempelt ist. }\toendnotes[C]{\smallbreak}\pstart
           \noindent{}\centering{}{\pb}\textcolor{green}{\textbf{Das Haus des Dichters}}{}\ledrightnote{\textcolor{green}{Das Haus des Dichters}}\pend
           \pstart
           \noindent{}\centering{}*\pend
           \pstart
           \noindent{}\centering{}Allen Freunden zur Erinnerung{\\}an meinen 50. Geburtstag\pend
           \pstart
           \noindent{}\centering{}⋅ Richard Dehmel ⋅\pend
           \pstart
           \noindent{}\centering{}*\pend
           \stanza{}\textcolor{green}{O bleib, Phönix, verlaß mich nicht,}{}\ledrightnote{\textcolor{green}{Das Haus des Dichters}}\newverse{}\textcolor{green}{Traumfeuervogel, mein göttlicher,}{}\ledrightnote{\textcolor{green}{Das Haus des Dichters}}\newverse{}\textcolor{green}{wie ſchweiften wir frei von Herd zu Herd!}{}\ledrightnote{\textcolor{green}{Das Haus des Dichters}}\newverse{}\textcolor{green}{Wenn ich ſcheu, ich ſtaubgeborener Wicht,}{}\ledrightnote{\textcolor{green}{Das Haus des Dichters}}\newverse{}\textcolor{green}{in die Aſche blies mit finſteren Geſicht,}{}\ledrightnote{\textcolor{green}{Das Haus des Dichters}}\newverse{}\textcolor{green}{flogſt du goldrot auf, immer neu hellauf,}{}\ledrightnote{\textcolor{green}{Das Haus des Dichters}}\newverse{}\textcolor{green}{unbeſchwert,}{}\ledrightnote{\textcolor{green}{Das Haus des Dichters}}\newverse{}\textcolor{green}{und Sternbilder ſprühten von deinen Schwingen.}{}\ledrightnote{\textcolor{green}{Das Haus des Dichters}}\newverse{}\textcolor{green}{Bis ein Abend kam, wo ich müd dir grollte,}{}\ledrightnote{\textcolor{green}{Das Haus des Dichters}}\newverse{}\textcolor{green}{unter fremden Fichten, in
                     Menſchenſehnſuchtsqual,}{}\ledrightnote{\textcolor{green}{Das Haus des Dichters}}\newverse{}\textcolor{green}{nicht mehr von dir träumen wollte,}{}\ledrightnote{\textcolor{green}{Das Haus des Dichters}}\newverse{}\textcolor{green}{von deinem ewigen Zauberſtrahl}{}\ledrightnote{\textcolor{green}{Das Haus des Dichters}}\newverse{}\textcolor{green}{und nie erlebten Wunderdingen,}{}\ledrightnote{\textcolor{green}{Das Haus des Dichters}}\newverse{}\textcolor{green}{nur von Heimat, Heimat endlich einmal –}{}\ledrightnote{\textcolor{green}{Das Haus des Dichters}}\newverse{}\textcolor{green}{da huben die Sterne an zu klingen:}{}\ledrightnote{\textcolor{green}{Das Haus des Dichters}}\newverse{}\textcolor{green}{Ja, die ganze Welt kannſt du wild durchſchweifen}{}\ledrightnote{\textcolor{green}{Das Haus des Dichters}}\newverse{}\textcolor{green}{in deinem freiheitstrunknen Flug,}{}\ledrightnote{\textcolor{green}{Das Haus des Dichters}}\newverse{}\textcolor{green}{kannſt Kometen begleiten durch Urnebelſtreifen,}{}\ledrightnote{\textcolor{green}{Das Haus des Dichters}}\newverse{}\textcolor{green}{Stürme, Wolken, Blitz dir zum Spielzeug greifen,}{}\ledrightnote{\textcolor{green}{Das Haus des Dichters}}\newverse{}\textcolor{green}{{\pb}ach, und haſt nicht Kraft
                     genug,}{}\ledrightnote{\textcolor{green}{Das Haus des Dichters}}\newverse{}\textcolor{green}{ein Haus auf der feſten Erde zu bauen,}{}\ledrightnote{\textcolor{green}{Das Haus des Dichters}}\newverse{}\textcolor{green}{für dich und die Deinen ein ſichres Bett,}{}\ledrightnote{\textcolor{green}{Das Haus des Dichters}}\newverse{}\textcolor{green}{kannſt dir nicht einen Balken ſelber hauen,}{}\ledrightnote{\textcolor{green}{Das Haus des Dichters}}\newverse{}\textcolor{green}{nicht ein Tiſchlein zu zimmern dich getrauen,}{}\ledrightnote{\textcolor{green}{Das Haus des Dichters}}\newverse{}\textcolor{green}{nicht ein Brett,}{}\ledrightnote{\textcolor{green}{Das Haus des Dichters}}\newverse{}\textcolor{green}{hockſt wie ein unbeholfnes Tier}{}\ledrightnote{\textcolor{green}{Das Haus des Dichters}}\newverse{}\textcolor{green}{unter den fremden Fichten hier –}{}\ledrightnote{\textcolor{green}{Das Haus des Dichters}}\newverse{}\textcolor{green}{ſo erklangen die Sterne – da flucht’ ich dir.}{}\ledrightnote{\textcolor{green}{Das Haus des Dichters}}\newverse{}\textcolor{green}{Bis der Morgen graute, bis Menſchen kamen,}{}\ledrightnote{\textcolor{green}{Das Haus des Dichters}}\newverse{}\textcolor{green}{hilfreich kamen, Mann für Mann,}{}\ledrightnote{\textcolor{green}{Das Haus des Dichters}}\newverse{}\textcolor{green}{mich herzlich bei den Händen nahmen,}{}\ledrightnote{\textcolor{green}{Das Haus des Dichters}}\newverse{}\textcolor{green}{und holde Frauen lachten mich an:}{}\ledrightnote{\textcolor{green}{Das Haus des Dichters}}\newverse{}\textcolor{green}{Sieh doch, da ſteht das Haus ſchon errichtet;}{}\ledrightnote{\textcolor{green}{Das Haus des Dichters}}\newverse{}\textcolor{green}{während du ſchweifteſt von Traum zu Traum,}{}\ledrightnote{\textcolor{green}{Das Haus des Dichters}}\newverse{}\textcolor{green}{ward Stein auf Stein zur Mauer geſchichtet,}{}\ledrightnote{\textcolor{green}{Das Haus des Dichters}}\newverse{}\textcolor{green}{der dunkle Hain zum Garten gelichtet,}{}\ledrightnote{\textcolor{green}{Das Haus des Dichters}}\newverse{}\textcolor{green}{dir zum heimatlichen Raum.}{}\ledrightnote{\textcolor{green}{Das Haus des Dichters}}\newverse{}\textcolor{green}{Nach freudiger Menſchheit ging dein Trachten;}{}\ledrightnote{\textcolor{green}{Das Haus des Dichters}}\newverse{}\textcolor{green}{weil du ſie träumteſt, lebt ſie nun;}{}\ledrightnote{\textcolor{green}{Das Haus des Dichters}}\newverse{}\textcolor{green}{du halfeſt ihr ſich göttlich achten,}{}\ledrightnote{\textcolor{green}{Das Haus des Dichters}}\newverse{}\textcolor{green}{empfang als Schöpferlohn ihr Tun;}{}\ledrightnote{\textcolor{green}{Das Haus des Dichters}}\newverse{}\textcolor{green}{laß dir aus unſern ſchwachen Händen}{}\ledrightnote{\textcolor{green}{Das Haus des Dichters}}\newverse{}\textcolor{green}{den Segen vieler ſtarken ſpenden!}{}\ledrightnote{\textcolor{green}{Das Haus des Dichters}}\newverse{}\textcolor{green}{{\pb}So ſprachen ſtrahlend zwei der
                     Frauen,}{}\ledrightnote{\textcolor{green}{Das Haus des Dichters}}\newverse{}\textcolor{green}{mich aber wehte ein Bangen an:}{}\ledrightnote{\textcolor{green}{Das Haus des Dichters}}\newverse{}\textcolor{green}{verflogen war das Morgengrauen,}{}\ledrightnote{\textcolor{green}{Das Haus des Dichters}}\newverse{}\textcolor{green}{und über dem ſonneblanken Tann}{}\ledrightnote{\textcolor{green}{Das Haus des Dichters}}\newverse{}\textcolor{green}{fern im Blauen}{}\ledrightnote{\textcolor{green}{Das Haus des Dichters}}\newverse{}\textcolor{green}{ſah ich ſtarr dich mit zitternden Klauen}{}\ledrightnote{\textcolor{green}{Das Haus des Dichters}}\newverse{}\textcolor{green}{ſchreckbeſchwert}{}\ledrightnote{\textcolor{green}{Das Haus des Dichters}}\newverse{}\textcolor{green}{– Phönix – ſprühend niederſchauen}{}\ledrightnote{\textcolor{green}{Das Haus des Dichters}}\newverse{}\textcolor{green}{auf meinen Herd.}{}\ledrightnote{\textcolor{green}{Das Haus des Dichters}}\newverse{}\textcolor{green}{Wie Sankt Johannes zwiſchen den ſieben Leuchtern}{}\ledrightnote{\textcolor{green}{Das Haus des Dichters}}\newverse{}\textcolor{green}{mit gen Boden gebeugtem Geſicht}{}\ledrightnote{\textcolor{green}{Das Haus des Dichters}}\newverse{}\textcolor{green}{barg ich unter den hohen Bäumen}{}\ledrightnote{\textcolor{green}{Das Haus des Dichters}}\newverse{}\textcolor{green}{meinen Blick vor all dem Gnadenlicht;}{}\ledrightnote{\textcolor{green}{Das Haus des Dichters}}\newverse{}\textcolor{green}{in meinen Tränen ſtoſſen zu taumelnden Flammen}{}\ledrightnote{\textcolor{green}{Das Haus des Dichters}}\newverse{}\textcolor{green}{die Menſchen rings mit euch zuſammen,}{}\ledrightnote{\textcolor{green}{Das Haus des Dichters}}\newverse{}\textcolor{green}{ihr alten Fichten um dies neue Dach –}{}\ledrightnote{\textcolor{green}{Das Haus des Dichters}}\newverse{}\textcolor{green}{was rauſcht ihr mir Erinnrung, ach!}{}\ledrightnote{\textcolor{green}{Das Haus des Dichters}}\newverse{}\textcolor{green}{Ich fühl’s noch heute beim Schwanken eurer
                     Zweige,}{}\ledrightnote{\textcolor{green}{Das Haus des Dichters}}\newverse{}\textcolor{green}{wie ich erſchüttert den Nacken neige,}{}\ledrightnote{\textcolor{green}{Das Haus des Dichters}}\newverse{}\textcolor{green}{weil mir zum Dank die Kraft gebricht.}{}\ledrightnote{\textcolor{green}{Das Haus des Dichters}}\newverse{}\textcolor{green}{Ich kann ja nichts als immer wieder träumen}{}\ledrightnote{\textcolor{green}{Das Haus des Dichters}}\newverse{}\textcolor{green}{von ſeligem Aufflug zu den freien Räumen –}{}\ledrightnote{\textcolor{green}{Das Haus des Dichters}}\newverse{}\textcolor{green}{O Phönix, Phönix, verlaß mich nicht! –}{}\ledrightnote{\textcolor{green}{Das Haus des Dichters}}\stanzaend{}\pstart
           \centering{}* * *\pend
           \pstart
           \noindent{}\centering{}{\pb}\label{T_L02157-1v}\edtext{WD}{\lemma{\textnormal{\emph{WD}}}\Cendnote{\textnormal{in Form eines Adlers, die nächste Zeile als Wappenspruch}}}\label{T_L02157-1h}\pend
           \pstart
           \noindent{}\centering{}Force m’est trop\pend
           \endnumbering\briefempfaengerindex{Schnitzler, Arthur@\textsc{Schnitzler, Arthur}!zzzDehmel, Richard@\emph{von Richard Dehmel}!1913-11-181@{{[}18. 11. 1913?{]}}|)be}\mylabel{h}  \normalsize

\doendnotes{C}
\bigskip
\vfill

\clearpage

\footnotesize

\lohead{\textsc{register}}

% Definiere theindex-Environment komplett neu ohne reledmac
\makeatletter
\renewenvironment{theindex}{%
  \section*{\indexname}%
  \setlength{\parindent}{0pt}%
  \setlength{\parskip}{0pt plus 0.3pt}%
  \let\item\@idxitem
}{%
  \clearpage
}
\makeatother

\IfFileExists{\jobname-pw.ind}{\input{\jobname-pw.ind}}{}

\end{document}

      