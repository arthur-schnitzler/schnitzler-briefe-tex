%% latex-korrekturansicht-vorspann.tex
%% Vorspann für die Korrekturansicht.
%% Lädt die gemeinsame Datei latex-vorspann.tex mit gesetztem Schalter.

\newif\ifkorrekturansicht
\korrekturansichttrue

\input{../tex-inputs/latex-vorspann}


               \section[Arthur und Olga Schnitzler an Richard Beer-Hofmann, 3. 7. 1906]{ Arthur und Olga Schnitzler an Richard Beer-Hofmann,
               3. 7. 1906}\nopagebreak\mylabel{v}\rehead{ }\normalsize\beginnumbering\briefempfaengerindex{Beer-Hofmann, Richard@\textsc{Beer-Hofmann, Richard}!zzzSchnitzler, Olga@\emph{von Olga Schnitzler}!1906-07-031@{3. 7. 1906}|(be}\briefempfaengerindex{Beer-Hofmann, Richard@\textsc{Beer-Hofmann, Richard}!zzzSchnitzler, Arthur@\emph{von Arthur Schnitzler}!1906-07-031@{3. 7. 1906}|(be} \toendnotes[C]{\smallbreak\pagebreak[2]} \Standort{YCGL, MSS 31.}
\physDesc{Bildpostkarte
\newline{}Handschrift Arthur Schnitzler: Bleistift, deutsche Kurrent\newline{}Handschrift Olga Schnitzler: Bleistift\newline{}Versand: Stempel: »\nobreak{}3. 7. 06\nobreak{}«.  }\buchAbdrucke{\weitereDrucke{Arthur Schnitzler, Richard Beer-Hofmann: \emph{Briefwechsel 1891–1931}. Hg. Konstanze Fliedl. Wien, Zürich: \emph{Europaverlag} 1992, S. 178.} }\pstart{}{\pb}\textsc{Dr. Richard Beer Hofmann}\pend{}\pstart{}\textsc{\textcolor{pink}{Rodaun}{}\ledrightnote{\textcolor{pink}{Rodaun}} b/\textcolor{pink}{Wien}{}\ledrightnote{\textcolor{pink}{Wien}}}\pend{}\pstart{}\textcolor{pink}{\textsc{Liesingerstr}. 2}{}\ledrightnote{\textcolor{pink}{Liesingerstraße}}.\pend{}\pstart{}\textsc{\textcolor{pink}{Austria}{}\ledrightnote{\textcolor{pink}{Österreich}}}\pend{}{\bigskip}\pstart
           \noindent{}\centering{}{\pb}\textcolor{gray}{\textbf{Hilsen fra \textcolor{pink}{Marienlyst}{}\ledrightnote{\textcolor{pink}{Marienlyst}}.}}\pend
           \pstart
           \raggedleft{}\textcolor{pink}{\textsc{Marienlyst, Kurhaus}}{}\ledrightnote{\textcolor{pink}{Kurhotellet}},
                     3/7. 9\textcolor{gray}{06}\pend
           \pstart
           Wunderſchön hier. Dürften längre Zeit bleiben. Habe geſtern \textcolor{blue}{Brandes}{}\ledrightnote{\textcolor{blue}{Georg Brandes}} beſucht, der im \textcolor{pink}{\textsc{Co{\geminationm}une Hospital}}{}\ledrightnote{\textcolor{pink}{Kommunehospitalet}} wieder
               mit ſeiner Venenentzündung krank liegt aber friſch iſt wie je.\pend
           \pstart
           Laſſen Sie von ſich und den Ihren hören. Herzlichſt,\pend
           \pstart
           Ihr \spacefill\mbox{A.}{\\[\baselineskip]}\spacefill\mbox{{[}hs. O. Schnitzler:{]} O. S.}\pend
           \leftskip=0em{}\pstart
           {[}hs. Schnitzler:{]} \textcolor{green}{\textsc{Charolais}}{}\ledrightnote{\textcolor{green}{Der Graf von Charolais. Ein Trauerspiel}} wird im Herbſt in \textcolor{pink}{\textsc{Kopenhagen}}{}\ledrightnote{\textcolor{pink}{Kopenhagen}} aufgeführt.\pend
           \endnumbering\briefempfaengerindex{Beer-Hofmann, Richard@\textsc{Beer-Hofmann, Richard}!zzzSchnitzler, Olga@\emph{von Olga Schnitzler}!1906-07-031@{3. 7. 1906}|)be}\briefempfaengerindex{Beer-Hofmann, Richard@\textsc{Beer-Hofmann, Richard}!zzzSchnitzler, Arthur@\emph{von Arthur Schnitzler}!1906-07-031@{3. 7. 1906}|)be}\mylabel{h}  \normalsize

\doendnotes{C}
\bigskip
\vfill

\clearpage

\footnotesize

\lohead{\textsc{register}}

% Definiere theindex-Environment komplett neu ohne reledmac
\makeatletter
\renewenvironment{theindex}{%
  \section*{\indexname}%
  \setlength{\parindent}{0pt}%
  \setlength{\parskip}{0pt plus 0.3pt}%
  \let\item\@idxitem
}{%
  \clearpage
}
\makeatother

\IfFileExists{\jobname-pw.ind}{\input{\jobname-pw.ind}}{}

\end{document}

      