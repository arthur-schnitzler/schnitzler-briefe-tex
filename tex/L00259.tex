%% latex-korrekturansicht-vorspann.tex
%% Vorspann für die Korrekturansicht.
%% Lädt die gemeinsame Datei latex-vorspann.tex mit gesetztem Schalter.

\newif\ifkorrekturansicht
\korrekturansichttrue

\input{../tex-inputs/latex-vorspann}


               \section[Arthur Schnitzler und Felix Salten an Hugo von Hofmannsthal, 24. 8. 1893]{ Arthur Schnitzler und Felix Salten an Hugo von Hofmannsthal,
               24. 8. 1893}\nopagebreak\mylabel{v}\rehead{ }\normalsize\beginnumbering\briefempfaengerindex{Hofmannsthal, Hugo von@\textsc{Hofmannsthal, Hugo von}!zzzSalten, Felix@\emph{von Felix Salten}!1893-08-241@{24. 8. 1893}|(be}\briefempfaengerindex{Hofmannsthal, Hugo von@\textsc{Hofmannsthal, Hugo von}!zzzSchnitzler, Arthur@\emph{von Arthur Schnitzler}!1893-08-241@{24. 8. 1893}|(be} \toendnotes[C]{\smallbreak\pagebreak[2]} \Standort{FDH, Hs-30885,11.}
\physDesc{Brief, 1 Blatt (Briefpapier mit Trauerrand), 4 Seiten
\newline{}Handschrift Arthur Schnitzler: Bleistift, deutsche Kurrent\newline{}Handschrift Felix Salten: Bleistift, deutsche Kurrent
\newline{}Hofmannsthal: mit Bleistift Vermerk: »\uline{Launiger Brief}« und Ergänzung: »›Des \textcolor{blue}{Meiſters von Cadore}
                                    reiche Farben‹ – Th. Morren. –« \newline{}Ordnung: von Schnitzler mutmaßlich bei der Durchsicht der Briefe
                                    1929 auf der ersten Seite mit Bleistift datiert: »24/8 93« }\buchAbdrucke{\weitereDrucke{Hugo von Hofmannsthal, Arthur Schnitzler: \emph{Briefwechsel}. Hg. Therese Nickl und Heinrich Schnitzler. Frankfurt am Main: \emph{S. Fischer} 1964, S. 44–45.} }\toendnotes[C]{\smallbreak}\pstart
           \noindent{}{\pb}{[}hs. Salten:{]} \introOben{}\uuline{Launiger Brief}\introOben{}\pend
           \pstart
           {[}hs. Schnitzler:{]} Mein lieber Hugo, Sie haben allerdings \textcolor{green}{Tizians Tod}{}\ledrightnote{\textcolor{green}{Der Tod des Tizian}} geſchrieben, wir aber haben ſoeben das Zi{\geminationm}er betreten, in welchem \textcolor{blue}{Tizian}{}\ledrightnote{\textcolor{blue}{Tizian}} geboren ward. Wir ſind nemlich in \textcolor{pink}{\textsc{Pieve di Cadore}}{}\ledrightnote{\textcolor{pink}{Pieve di Cadore}}; heute früh von \textcolor{pink}{\textsc{Toblach}}{}\ledrightnote{\textcolor{pink}{Toblach}} mit unſeren Rädern abgefahren, und über \textcolor{pink}{\textsc{Cortina}}{}\ledrightnote{\textcolor{pink}{Cortina d'Ampezzo}} hieher – manchmal {\pb}unter Hagel und Regen, und
               keineswegs ohne daſs uns die Zollbehörden anhielten.  – Hier haben wir in den paar
               Stunden unſres Aufenthaltes viel Schönheit und Leben geſehen: blonde Kinder\footnote{\noindent{}Schönheit}, die auf ſteinernen Löwen\footnote{\noindent{}Leben}{ }ſpielten, andre wieder, die »Muſikbande« ſpielten
               und wo der Kapellmeiſter ſeine ſämtlichen auf {\pb}Holzſtäben
               und Löffeln muſicirenden Untergebenen jä{\geminationm}erlich
                  prügelte.\footnote{\noindent{}Schönheit} Ein altes Weib,\footnote{\noindent{}Leben} das von Haus zu Haus ging und die kleinen Kinder küſſte, ein Kerl, der
               zum Fenſter hinausſchaute und dem Strümpfe\footnote{\noindent{}Schönheit} zum Mund heraushingen, mit welchen ich, wie \textcolor{blue}{\textsc{Salten}}{}\ledrightnote{\textcolor{blue}{Felix Salten}} meint, verbleiben ſoll\pend
           \pstart Ihr hoch- u rad-fahrender \spacefill\mbox{ArthSch.}\pend{}\pstart
           \noindent{}{\pb}{[}hs. Salten:{]} lieber Freund! Die Fahrt durch die Pracht des \textcolor{pink}{Ampezzo}{}\ledrightnote{\textcolor{pink}{Valle d’Ampezzo}} u \textcolor{pink}{Cadore Thales}{}\ledrightnote{\textcolor{pink}{Valle di Cadore}} und
               der Aufenthalt hier haben gelehrt: Es genügt nicht, dass der Mensch den \textcolor{green}{Tod des Tizian}{}\ledrightnote{\textcolor{green}{Der Tod des Tizian}} schreibe, er muss auch Bicycle fahren
               können. Ersteres haben Sie gethan, das Zweite bleibt Ihnen noch. Wir allerdings haben
               beim zweiten angefangen, und das Schwierigere steht uns noch bevor, was wir, wie
               Arthur meint, heute ’mal versuchen wollen.\pend
           \pstart
           Herzlichst{\\}Ihr{\\}\spacefill\mbox{Salten}\pend
           \pstart
           \noindent{}{[}hs. Schnitzler:{]} \label{T_L00259_1v}\edtext{\textcolor{pink}{\textsc{Pieve di Cadore}}{}\ledrightnote{\textcolor{pink}{Pieve di Cadore}}}{\lemma{\textnormal{\emph{Pieve di Cadore}}}\Cendnote{\textnormal{dies und das Folgende am unteren
                  Blattrand auf dem Kopf. Möglicherweise handelt es sich um den ursprünglichen
                  Briefkopf?}}}\label{T_L00259_1h}\pend
           \pstart
           {[}hs. Salten:{]} den 24. August 93\pend
           \pstart
           Ein \label{K_L00259_1v}\edtext{Jahr}{\lemma{\textnormal{\emph{Jahr}}}\Cendnote{\textnormal{siehe A. S.: \emph{Tagebuch}, 31. 8. 1892}}}\label{K_L00259_1h}, nach dem Loris in \textcolor{pink}{Strobl}{}\ledrightnote{\textcolor{pink}{Strobl}} seinen Freunden
                  »\textcolor{green}{Tizians Tod}{}\ledrightnote{\textcolor{green}{Der Tod des Tizian}}« las.\pend
           \endnumbering\briefempfaengerindex{Hofmannsthal, Hugo von@\textsc{Hofmannsthal, Hugo von}!zzzSalten, Felix@\emph{von Felix Salten}!1893-08-241@{24. 8. 1893}|)be}\briefempfaengerindex{Hofmannsthal, Hugo von@\textsc{Hofmannsthal, Hugo von}!zzzSchnitzler, Arthur@\emph{von Arthur Schnitzler}!1893-08-241@{24. 8. 1893}|)be}\mylabel{h}  \normalsize

\doendnotes{C}
\bigskip
\vfill

\clearpage

\footnotesize

\lohead{\textsc{register}}

% Definiere theindex-Environment komplett neu ohne reledmac
\makeatletter
\renewenvironment{theindex}{%
  \section*{\indexname}%
  \setlength{\parindent}{0pt}%
  \setlength{\parskip}{0pt plus 0.3pt}%
  \let\item\@idxitem
}{%
  \clearpage
}
\makeatother

\IfFileExists{\jobname-pw.ind}{\input{\jobname-pw.ind}}{}

\end{document}

      