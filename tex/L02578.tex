%% latex-korrekturansicht-vorspann.tex
%% Vorspann für die Korrekturansicht.
%% Lädt die gemeinsame Datei latex-vorspann.tex mit gesetztem Schalter.

\newif\ifkorrekturansicht
\korrekturansichttrue

\input{../tex-inputs/latex-vorspann}


               \section[Felix Salten, Jakob Wassermann, Otto Brahm, Ludwig Brahm an Arthur Schnitzler, 21. 07. {[}1907?{]}]{ Felix Salten, Jakob Wassermann, Otto Brahm, Ludwig Brahm an Arthur
               Schnitzler, 21. 07. {[}1907?{]}}\nopagebreak\mylabel{v}\rehead{ }\normalsize\beginnumbering\briefempfaengerindex{Schnitzler, Arthur@\textsc{Schnitzler, Arthur}!zzzWassermann, Jakob@\emph{von Jakob Wassermann}!1907-12-211@{21. 07. {[}1907?{]}}|(be}\briefempfaengerindex{Schnitzler, Arthur@\textsc{Schnitzler, Arthur}!zzzBrahm, Otto@\emph{von Otto Brahm}!1907-12-211@{21. 07. {[}1907?{]}}|(be}\briefempfaengerindex{Schnitzler, Arthur@\textsc{Schnitzler, Arthur}!zzzBrahm, Ludwig@\emph{von Ludwig Brahm}!1907-12-211@{21. 07. {[}1907?{]}}|(be}\briefempfaengerindex{Schnitzler, Arthur@\textsc{Schnitzler, Arthur}!zzzSalten, Felix@\emph{von Felix Salten}!1907-12-211@{21. 07. {[}1907?{]}}|(be} \toendnotes[C]{\smallbreak\pagebreak[2]} \Standort{CUL, Schnitzler, B 113.}
\physDesc{Bildpostkarte
\newline{}Handschrift Felix Salten: Bleistift, lateinische Kurrent\newline{}Handschrift Ludwig Brahm: Bleistift, deutsche Kurrent\newline{}Handschrift Jakob Wassermann: Bleistift, deutsche Kurrent\newline{}Handschrift Otto Brahm: Bleistift, lateinische Kurrent\newline{}Versand: 1) mit rotem Buntstift Adresse gestrichen und ursprüngliche
                                 Adresszeile durch »Bahnhofstraße« ersetzt 2) Stempel: »\nobreak{}\oindex{Semmering@\textbf{Semmering}, \emph{Besiedelter Ort (A.BSO)}|pwk}Semmer\textcolor{gray}{ing}, 21. XII. \textcolor{gray}{07}, 9\nobreak{}«. 
\newline{}Schnitzler: mit Bleistift eine Unterstreichung  }\pstart{}{\pb}Herrn D\textsuperscript{r}
                  Arthur Schnitzler\pend{}\pstart{}\textcolor{pink}{Wien XVIII.}{}\ledrightnote{\textcolor{pink}{XVIII., Währing}}\pend{}\pstart{}\textcolor{pink}{Spoettelgasse 7}{}\ledrightnote{\textcolor{pink}{Edmund-Weiß-Gasse}}\pend{}{\bigskip}\pstart
           \noindent{}\centering{}{\pb}\textcolor{gray}{\textbf{Winter-Idylle}}\pend
           \pstart
           {\pb}{[}hs. Wassermann:{]} Lieber Arthur! Wie ſehr leid tut uns allen Ihr Nichtdaſein! Wir
               denken und sprechen viel von Ihnen.\pend
           \pstart
           \introOben{}Für \textcolor{blue}{Olga}{}\ledrightnote{\textcolor{blue}{Olga Schnitzler}} das Herzlichste an
                  Wünschen\introOben{}\pend
           \pstart Der Ihre \spacefill\mbox{Wassermann}\pend{}\pstart
           \noindent{}{[}hs. Salten:{]} Hoffentlich geht es Frau \textcolor{blue}{Olga}{}\ledrightnote{\textcolor{blue}{Olga Schnitzler}} täglich besser und besser. Viele herzliche Grüße an Sie Beide!\pend
           \pstart Ihr \spacefill\mbox{Salten.}\pend{}\pstart
           \noindent{}Die Bücher sende ich Montag.\pend
           \pstart
           \noindent{}{[}hs. Brahm:{]} Lieber Freund, da wir Fr. \textcolor{blue}{O.}{}\ledrightnote{\textcolor{blue}{Olga Schnitzler}} und
               Sie leider, leider nicht hier haben, huldigten wir Ihnen und verspürten Ihres Geistes
               ein Hauch auf dem Wasserleitungswege. Alles Gute wünschet von Herzen\pend
           \pstart Ihr \spacefill\mbox{Otto Brahm}\pend{}\pstart
           \noindent{}{[}hs. Brahm:{]} Den herzlichsten Wünſchen für die schnelle Geneſung Ihrer
                  \textcolor{blue}{Gattin}{}\ledrightnote{\textcolor{blue}{Olga Schnitzler}} schließt ſich mit den besten Grüßen für
               Sie an\pend
           \pstart
           Ihr{\\[\baselineskip]}\spacefill\mbox{Ludwig Brahm.}\pend
           \leftskip=0em{}\endnumbering\briefempfaengerindex{Schnitzler, Arthur@\textsc{Schnitzler, Arthur}!zzzWassermann, Jakob@\emph{von Jakob Wassermann}!1907-12-211@{21. 07. {[}1907?{]}}|)be}\briefempfaengerindex{Schnitzler, Arthur@\textsc{Schnitzler, Arthur}!zzzBrahm, Otto@\emph{von Otto Brahm}!1907-12-211@{21. 07. {[}1907?{]}}|)be}\briefempfaengerindex{Schnitzler, Arthur@\textsc{Schnitzler, Arthur}!zzzBrahm, Ludwig@\emph{von Ludwig Brahm}!1907-12-211@{21. 07. {[}1907?{]}}|)be}\briefempfaengerindex{Schnitzler, Arthur@\textsc{Schnitzler, Arthur}!zzzSalten, Felix@\emph{von Felix Salten}!1907-12-211@{21. 07. {[}1907?{]}}|)be}\mylabel{h}  \normalsize

\doendnotes{C}
\bigskip
\vfill

\clearpage

\footnotesize

\lohead{\textsc{register}}

% Definiere theindex-Environment komplett neu ohne reledmac
\makeatletter
\renewenvironment{theindex}{%
  \section*{\indexname}%
  \setlength{\parindent}{0pt}%
  \setlength{\parskip}{0pt plus 0.3pt}%
  \let\item\@idxitem
}{%
  \clearpage
}
\makeatother

\IfFileExists{\jobname-pw.ind}{\input{\jobname-pw.ind}}{}

\end{document}

      