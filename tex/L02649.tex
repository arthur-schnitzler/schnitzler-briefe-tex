\documentclass[twoside=false,titlepage=false,open=any, parskip=never, fontsize=12pt, headings=small, chapterprefix=false, appendixprefix=false]{scrbook}
\addtolength{\oddsidemargin}{\evensidemargin}
\setlength{\oddsidemargin}{.5\oddsidemargin}
\setlength{\evensidemargin}{\oddsidemargin}

\usepackage[{textwidth=13cm,textheight=23cm,marginpar=3cm, left=2cm}]{geometry}
%\usepackage[textwidth=80mm, layoutwidth=170mm, paperheight =297mm, paperwidth  =210mm, layoutvoffset= 20mm,layouthoffset= 20mm]{geometry}
%\usepackage[paperheight =297mm, paperwidth  =210mm, layoutheight=230mm, layoutwidth=158mm, layoutvoffset= 20mm, layouthoffset= 20mm, textwidth=150mm, textheight=185mm, showcrop=false]{geometry}
%sepackage[paperheight=230mm, paperwidth=138mm, textwidth=100mm, textheight=185mm]{geometry}
 \usepackage[usenames, dvipsnames]{xcolor}
\usepackage{scrlayer-scrpage}
\usepackage{hyphenat}
\usepackage{fontspec}
\usepackage{moresize}
\usepackage[english, french, greek, ngerman]{babel}
%\usepackage{ipa}  für das Seitenwechselzeichens
\usepackage[babel]{microtype}
\usepackage[dash, dot]{dashundergaps}
\usepackage{soul}
\usepackage{ragged2e}
\usepackage[makeindex, protected]{splitidx}
\usepackage[itemlayout=abshang,hangindent=0.85em, subindent=0em, subsubindent=1em, justific=RaggedRight, columns=1, columnsep=0pt, indentunit=1em, totoc=false]{idxlayout}
\usepackage{scrhack}
\usepackage{xpatch}
\usepackage{reledmac}
\usepackage{refcount} % Für die Seitenverweise 1–3 etc. 
\usepackage{etoolbox}
\usepackage{framed}
\usepackage[export]{adjustbox} % loads also graphicx, für Bildgröße autom. maximal
\usepackage{float} %ermöglicht exakte Bildpositionierung
\usepackage{mdframed}
\usepackage{enumitem}
\usepackage{relsize}
\usepackage{longtable}
\usepackage{chngcntr} % Sectionnummern durchgehend
\usepackage{hanging} % Für hängende Absätze
\usepackage[rightmargin=0em, leftmargin=1em, indentfirst=false]{quoting} % Für die geänderte quote-Umgebung in den Hrsg-Texten
%\usepackage{fontawesome}
\usepackage{ellipsis}
\RequirePackage{hyphsubst}%
\HyphSubstIfExists{ngerman-x-latest}{\HyphSubstLet{ngerman}{ngerman-x-latest}}{} 
\listfiles
\usepackage[noadjust]{marginnote}

\KOMAoptions{toc=chapterentrydotfill, toc=flat}
\addtokomafont{chapterentrypagenumber}{\mdseries}
\setkomafont{chapterentry}{\normalfont\mdseries}
\setkomafont{partentry}{\normalfont\mdseries}
\RedeclareSectionCommand[tocbeforeskip=0pt]{chapter}

\setlength{\skip\footins}{4mm plus 2mm} % Abstand Fussnote Text
\interfootnotelinepenalty=10000 % Kein Seitenwechsel in Fuss

%\DeclareTextFontCommand{\emph}{\textit}

% Der Befehl erlaubt rechtsbündig bei Unterschriften, die nicht mehr in die Zeile passen
\def\spacefill{\hspace{\fill}\mbox{}\linebreak[0]\hspace*{\fill}}
\usepackage{atbegshi}
\usepackage{zref-abspage}
\usepackage{perpage}
\usepackage{zref-user}
\usepackage{tikz}
\usepackage{ulem}
\usetikzlibrary{calc,decorations.pathmorphing}

\PassOptionsToPackage{gray}{xcolor}
\definecolor{gray}{gray}{0.6}

\doublehyphendemerits=1000000 % das hier verhindert zu viele aufeinanderfolgende Trennstriche am Zeilenende


\usepackage{zref-abspage}
\usepackage{zref-user}
\usepackage{tikz}
\usepackage{atbegshi}
\usepackage{ulem}
\usetikzlibrary{calc,decorations.pathmorphing}

\PassOptionsToPackage{gray}{xcolor}
\definecolor{gray}{gray}{0.6}

\doublehyphendemerits=1000000 % das hier verhindert zu viele aufeinanderfolgende Trennstriche am Zeilenende

\makeatletter
\newcommand{\currentsidemargin}{%
  \ifodd\zref@extract{textarea-\thetextarea}{abspage}%
    \oddsidemargin%
  \else%
    \evensidemargin%
  \fi%
}

\newcounter{textarea}
\newcommand{\settextarea}{%
   \stepcounter{textarea}%
   \zlabel{textarea-\thetextarea}%
   \begin{tikzpicture}[overlay,remember picture]
    % Helper nodes
    \path (current page.north west) ++(\hoffset, -\voffset)
        node[anchor=north west, shape=rectangle, inner sep=0, minimum width=\paperwidth, minimum height=\paperheight]
        (pagearea) {};
    \path (pagearea.north west) ++(1in+\currentsidemargin,-1in-\topmargin-\headheight-\headsep)
        node[anchor=north west, shape=rectangle, inner sep=0, minimum width=\textwidth, minimum height=7pt]
        (textarea) {};
  \end{tikzpicture}%
}

\tikzset{tikzul/.style={yshift=-.75\dp\strutbox}}

\newcounter{tikzul}%
\newcommand\tikzul[1][]{%
    \begingroup
    \global\tikzullinewidth\linewidth
    \def\tikzulsetting{[#1]}%
    \stepcounter{tikzul}%
    \settextarea
    \zlabel{tikzul-begin-\thetikzul}%
    \tikz[overlay,remember picture,tikzul] \coordinate (tikzul-\thetikzul) at (0,0);% Modified \tikzmark macro
    \ifnum\zref@extract{tikzul-begin-\thetikzul}{abspage}=\zref@extract{tikzul-end-\thetikzul}{abspage}
    \else
        \AtBeginShipoutNext{\tikzul@endpage{#1}}%
    \fi
    \bgroup
    \def\par{\ifhmode\unskip\fi\egroup\par\@ifnextchar\noindent{\noindent\tikzul[#1]}{\tikzul[#1]\bgroup}}%
    \aftergroup\endtikzul
    \let\@let@token=%
}
\newlength\tikzullinewidth


\def\tikzul@endpage#1{%
\setbox\AtBeginShipoutBox\hbox{%
\box\AtBeginShipoutBox
\hbox{%
\begin{tikzpicture}[overlay,remember picture,tikzul]
\draw[#1]
    let \p1 = (tikzul-\thetikzul), \p2 = ([xshift=\tikzullinewidth+\@totalleftmargin]textarea.south west) in
    \ifdim\dimexpr\y1-\y2<.5\baselineskip
        (\x1,\y1) -- (\x2,\y1)
    \else
        let \p3 = ([xshift=\@totalleftmargin]textarea.west) in
        (\x1,\y1) -- +(\tikzullinewidth-\x1+\x3,0)
        % (\x3,\y2) -- (\x2,\y2)
        (\x3,\y1)
       \myloop{\y1-\y2+.5\baselineskip}{%
           ++(0,-\baselineskip) -- +(\tikzullinewidth,0)
       }%
    \fi
;
\end{tikzpicture}%
}}%
}%


\def\endtikzul{%
    \zlabel{tikzul-end-\thetikzul}%
    \ifnum\zref@extract{tikzul-begin-\thetikzul}{abspage}=\zref@extract{tikzul-end-\thetikzul}{abspage}
    \begin{tikzpicture}[overlay,remember picture,tikzul]
        \expandafter\draw\tikzulsetting
            let \p1 = (tikzul-\thetikzul), \p2 = (0,0) in
            \ifdim\y1=\y2
                (\x1,\y1) -- (\x2,\y2)
            \else
                let \p3 = ([xshift=\@totalleftmargin]textarea.west), \p4 = ([xshift=-\rightmargin]textarea.east) in
                (\x1,\y1) -- +(\tikzullinewidth-\x1+\x3,0)
                (\x3,\y2) -- (\x2,\y2)
                (\x3,\y1)
                \myloop{\y1-\y2}{%
                    ++(0,-\baselineskip) -- +(\tikzullinewidth,0)
                }%
            \fi
        ;
    \end{tikzpicture}%
    \else
    \settextarea
    \begin{tikzpicture}[overlay,remember picture,tikzul]
        \expandafter\draw\tikzulsetting
            let \p1 = ([xshift=\@totalleftmargin,yshift=-.5\baselineskip]textarea.north west), \p2 = (0,0) in
            \ifdim\dimexpr\y1-\y2<.5\baselineskip
                (\x1,\y2) -- (\x2,\y2)
            \else
                let \p3 = ([xshift=\@totalleftmargin]textarea.west), \p4 = ([xshift=-\rightmargin]textarea.east) in
                (\x3,\y2) -- (\x2,\y2)
                (\x3,\y2)
                \myloop{\y1-\y2}{%
                    ++(0,+\baselineskip) -- +(\tikzullinewidth,0)
                }
            \fi
        ;
    \end{tikzpicture}%
    \fi
    \endgroup
}

% -------------------------------------------------------------- Additions by Peter Grill

\tikzset{tikzst/.style={yshift=0.5\dp\strutbox}}

\newcounter{tikzst}%
\newcommand\tikzst[1][]{%
    \begingroup
    \global\tikzstlinewidth\linewidth
    \def\tikzstsetting{[#1]}%
    \stepcounter{tikzst}%
    \settextarea
    \zlabel{tikzst-begin-\thetikzst}%
    \tikz[overlay,remember picture,tikzst] \coordinate (tikzst-\thetikzst) at (0,0);% Modified \tikzmark macro
    \ifnum\zref@extract{tikzst-begin-\thetikzst}{abspage}=\zref@extract{tikzst-end-\thetikzst}{abspage}
    \else
        \AtBeginShipoutNext{\tikzst@endpage{#1}}%
    \fi
    \bgroup
    \def\par{\ifhmode\unskip\fi\egroup\par\@ifnextchar\noindent{\noindent\tikzst[#1]}{\tikzst[#1]\bgroup}}%
    \aftergroup\endtikzst
    \let\@let@token=%
}
\newlength\tikzstlinewidth


\def\tikzst@endpage#1{%
\setbox\AtBeginShipoutBox\hbox{%
\box\AtBeginShipoutBox
\hbox{%
\begin{tikzpicture}[overlay,remember picture,tikzst]
\draw[#1]
    let \p1 = (tikzst-\thetikzst), \p2 = ([xshift=\tikzstlinewidth+\@totalleftmargin]textarea.south west) in
    \ifdim\dimexpr\y1-\y2<.5\baselineskip
        (\x1,\y1) -- (\x2,\y1)
    \else
        let \p3 = ([xshift=\@totalleftmargin]textarea.west) in
        (\x1,\y1) -- +(\tikzstlinewidth-\x1+\x3,0)
        % (\x3,\y2) -- (\x2,\y2)
        (\x3,\y1)
       \myloop{\y1-\y2+.5\baselineskip}{%
           ++(0,-\baselineskip) -- +(\tikzstlinewidth,0)
       }%
    \fi
;
\end{tikzpicture}%
}}%
}%


\def\endtikzst{%
    \zlabel{tikzst-end-\thetikzst}%
    \ifnum\zref@extract{tikzst-begin-\thetikzst}{abspage}=\zref@extract{tikzst-end-\thetikzst}{abspage}
    \begin{tikzpicture}[overlay,remember picture,tikzst]
        \expandafter\draw\tikzstsetting
            let \p1 = (tikzst-\thetikzst), \p2 = (0,0) in
            \ifdim\y1=\y2
                (\x1,\y1) -- (\x2,\y2)
            \else
                let \p3 = ([xshift=\@totalleftmargin]textarea.west), \p4 = ([xshift=-\rightmargin]textarea.east) in
                (\x1,\y1) -- +(\tikzstlinewidth-\x1+\x3,0)
                (\x3,\y2) -- (\x2,\y2)
                (\x3,\y1)
                \myloop{\y1-\y2}{%
                    ++(0,-\baselineskip) -- +(\tikzstlinewidth,0)
                }%
            \fi
        ;
    \end{tikzpicture}%
    \else
    \settextarea
    \begin{tikzpicture}[overlay,remember picture,tikzst]
        \expandafter\draw\tikzstsetting
            let \p1 = ([xshift=\@totalleftmargin,yshift=-.5\baselineskip]textarea.north west), \p2 = (0,0) in
            \ifdim\dimexpr\y1-\y2<.5\baselineskip
                (\x1,\y2) -- (\x2,\y2)
            \else
                let \p3 = ([xshift=\@totalleftmargin]textarea.west), \p4 = ([xshift=-\rightmargin]textarea.east) in
                (\x3,\y2) -- (\x2,\y2)
                (\x3,\y2)
                \myloop{\y1-\y2}{%
                    ++(0,+\baselineskip) -- +(\tikzstlinewidth,0)
                }
            \fi
        ;
    \end{tikzpicture}%
    \fi
    \endgroup
}
% --------------------------------------------------------------

\def\myloop#1#2#3{%
    #3%
    \ifdim\dimexpr#1>1.1\baselineskip
        #2%
        \expandafter\myloop\expandafter{\the\dimexpr#1-\baselineskip\relax}{#2}%
    \fi
}

\makeatother






\def\myloop#1#2#3{%
    #3%
    \ifdim\dimexpr#1>1.1\baselineskip
        #2%
        \expandafter\myloop\expandafter{\the\dimexpr#1-\baselineskip\relax}{#2}%
    \fi
}

\makeatother
%\newcommand{\damage}[1]{\tikzul[gray,line width=0.15\ht\strutbox,semitransparent]{#1}}
%\newcommand{\strikeout}[1]{\tikzst[black]{#1}}

\newcommand{\damage}[1]{\textcolor{orange}{#1}}
\newcommand{\strikeout}[1]{\sout{#1}}


\setlength{\parindent}{1em}

% Mehr als drei Auslassungspunkte 

\newcommand{\dotsseven}{%
.\kern\ellipsisgap 
.\kern\ellipsisgap
.\kern\ellipsisgap 
.\kern\ellipsisgap
.\kern\ellipsisgap
.\kern\ellipsisgap 
.\kern\ellipsisgap 	
\relax}

\newcommand{\dotssix}{%
.\kern\ellipsisgap 
.\kern\ellipsisgap
.\kern\ellipsisgap
.\kern\ellipsisgap
.\kern\ellipsisgap 
.\kern\ellipsisgap 
\relax}

\newcommand{\dotsfive}{%
.\kern\ellipsisgap 
.\kern\ellipsisgap
.\kern\ellipsisgap
.\kern\ellipsisgap 
.\kern\ellipsisgap 
\relax}

\newcommand{\dotsfour}{%
.\kern\ellipsisgap 
.\kern\ellipsisgap
.\kern\ellipsisgap
.\kern\ellipsisgap 
\relax}

\newcommand{\dotstwo}{%
.\kern\ellipsisgap 
.\kern\ellipsisgap
\relax}


% Silbentrennung
\selectlanguage{ngerman}
\hyphenation{Re-kours EP-STEIN Her-vay-vor-les-ung Steu-er-sa-chen Öst-reich Burck-hard Keuch-hus-ten Oedi-pus-auf-führ-un-gen Hi-obs-post Kärnt-ner-ring Vei-tlis-sen-gas-se Franck-gas-se Rath-hau-se Sechs-schg Stu-bai-thal Tha-deusz Volks-th Halb-mo-nats-schrift JAHR-ES-ZEI-TEN Te-le-phon mit-ge-theilt Ge-schäfts-ver-bin-dung hoch-müth-ig Ueber-zeu-gung bis-chen Au-tor-rech-te Hof-manns-thal Nor-deijk Irre-seins Tschap-perl mit-zu-thei-len Aeu-ße-rung be-thö-ren Kü-ni-gel Be-ur-thei-lung Kuenst-lern ko-moe-di-sche hae-mor-rha-gi-scher Doer-mann Wash-burn flei-ssig haute Buddh-ist Preu-ssen Lin-den-café Mit-theil-un-gen An-theil Lieu-te-nant oes-terr Rieg-ner Oes-ter-reich gro-ssem Fran-zo-sen-thum Roche Lili Ent-schlie-ssun-gen äu-ssert wuen-sche Trans-ac-tio-nen Ue-ber-win-dung Eu-gene Stra-ssen-dir-ne qua-tre Deutsch-öst-er-reich Deutsch-öst-er-reichs Bjørn-stjer-ne noth-ing Edit-ed Olga Ar-naud Mer-gent-heim Léon-tine Polla-czek Brion Barre Hoch-sin-ger Ka-tha-rina Arouet Va-len-ci-ennes Ueber-win-dung Type-writer-in Tolstoi-buch Schnitzler Copier-buche Schiller Intel-lek-tuell-en-as-so-zi-a-tion Salten Devrient Grien-steidl Ge-sell-ſchaft ein-ge-ſchloſ-ſen Fort-ſetz-un-gen Bor-dell-ſtück fort-ſchrei-ten wirk-ſam-es ſchrift-ſtel-ler-i-ſchen hin-weg-ſe-hen Gerichts-saal-be-richt-er-ſtat-ter}



% Sonderbefehl für .–
\def\dotdash{\nobreak\hspace{0pt}.–}  %ACHTUNG BEIM ERSETZEN: LEERZEICHEN DANACH 
\def\commadash{\nobreak\hspace{0pt},–}
\def\excdash{\nobreak\hspace{0pt}!–}
\def\semicolondash{\nobreak\hspace{0pt};–}
\def\parentdotdash{\nobreak\hspace{0pt}).–}
\def\slashislash{\,\slash\,\allowbreak\hspace{0pt}}

\newcommand{\strich}{\makebox[1em][l]{– }}


% Seite einrichten

% Farbe definieren
%\setmainfont[RawFeature={-liga}, 
%SmallCapsFont=WSVgara-Caps, 
%ItalicFont=WSVgara-Italic, 
%BoldFont=WSVgara-Bold,
%BoldItalicFont=WSVgara-BoldItalic
%]{WSVgara}
%\setsansfont[RawFeature={-liga}, 
%SmallCapsFont=WSVgara-Caps, 
%ItalicFont=WSVgara-Italic, 
%BoldFont=WSVgara-Bold,
%BoldItalicFont=WSVgara-BoldItalic
%]{WSVgara}

%\setmainfont{Brill}
%\setsansfont{Brill}

%\setmainfont[ItalicFont=SinaNova-Italic, 
%BoldFont=SinaNova-Bold,
%BoldItalicFont=SinaNova-BoldItalic
%]{SinaNova-Regular}
%\setsansfont[ItalicFont=SinaNova-Italic, 
%BoldFont=SinaNova-Bold,
%BoldItalicFont=SinaNova-BoldItalic
%]{SinaNova-Regular}



\def\labelitemi{--}

% Geminationsstrich, U-Strich

 \newcommand{\overbar}[1]{$\overline{\hbox{#1}}$}


% Ausrufezeichen in den Index kriegen
\newcommand{\rufezeichen}{"!}

% Griechisch
	
%\newfontfamily\greekfont{GaramondPremrPro}
%\newcommand\griechisch[1]{\greekfont{}#1{}\normalfont}

%\newfontfamily\sansseriffont[HyphenChar=None, RawFeature={-liga}, Scale=1.03]{TheSans-Regular}
%\newfontfamily\sansseriffont{uarial}


%\newfontfamily\sansseriffont[HyphenChar=None, LetterSpace=1.0, RawFeature={-liga}]{TheSans-SemiBold}
%\newcommand\sansseriff[1]{\sffamily{}#1{}\normalfont}

\newcommand{\mini}{\,}


\newcommand{\key}{\textsuperscript{\textcolor{red}{KEY}}}


%% Sperrung (Package Soul)
%% Hier ist die Sperrung definiert. Sperrung erreicht man mit \so{gesperrtes Wort}
\sodef\so{}{.14em}{.4em plus.1em minus .1em}{.4em plus.1em minus .1em}

% SCHRIFTEN
\setkomafont{disposition}{}
\addtokomafont{caption}{\small}
\addtokomafont{captionlabel}{\small}

%% Schrift der Kopf und Fußzeile
\renewcommand*{\headfont}{\normalfont}
\setkomafont{pagehead}{\footnotesize\addfontfeature{LetterSpace=10.0}}
\setkomafont{pagenumber}{\normalfont\normalsize}
\ohead[]{\pagemark}% Seitenzahl (c = centered) 
\ofoot[]{}


 
% Flatterndes Seitenende
\raggedbottom

% Fussnoten neu Anfangen

\makeatletter
\pretocmd{\@schapter}{\setcounter{footnote}{0}}{}{}
\pretocmd{\@chapter}{\setcounter{footnote}{0}}{}{}
\pretocmd{\@section}{\setcounter{footnote}{0}}{}{}
\makeatother


% Section Nummern durchgehend

\RedeclareSectionCommand[
  counterwithout=chapter
]{section}

% Section Punkt

\renewcommand*{\sectionformat}{}
\renewcommand*{\partformat}{}


% Marginpar Schrift

\newkomafont{margin}{\footnotesize} 
\makeatletter 
\let\MarginParOriginal\marginpar 
\renewcommand*{\marginpar}{\@dblarg\@marginpar} 
\newcommand{\@marginpar}[2][]{% 
  \MarginParOriginal[\usekomafont{margin}{#1\par}]{\usekomafont{margin}{#2\par}} 
} 
\makeatother 



\let\oldbeginnumbering\beginnumbering

\def\beginnumbering{\oldbeginnumbering\par\nopagebreak}


% Fußnoten linksbündig
\deffootnote{1.5em}{1em}{% 
\makebox[1.5em][l]{\thefootnotemark}%
}


% Fussnotenlineal (wobei für reledmac wohl was anderes gilt)
\let\normalfootnoterule\footnoterule
\setfootnoterule{0pt}
\let\normalfootnoterule\footnoterule


\setlength{\skip\footins}{8mm plus 2mm} % Abstand Fussnote Text
\interfootnotelinepenalty=10000 % Kein Seitenwechsel in Fuss

%% Kapitelüberschriften
\renewcommand*{\raggedchapter}{\centering} 
\renewcommand*{\raggedsection}{%
 \CenteringLeftskip=1cm plus 1em\relax 
 \CenteringRightskip=1cm plus 1em\relax 
 \Centering\footnotesize\thesection{}.\ }
\setkomafont{section}{\footnotesize}
\setkomafont{chapter}{\normalfont\Large}
\renewcommand{\chapterpagestyle}{empty}%The first page in each chapter won't have any heading or footer, especially no page number

% section ohne führende Kapitelnummer
\renewcommand*\thesection{\arabic{section}}

% Bildunterschrift ohne Nummer
\renewcommand*{\figureformat}{}
\renewcommand*{\captionformat}{}

% Abstand Bild
\setlength{\textfloatsep}{\baselineskip}

%% Zeilennummern
\firstlinenum{0} \linenumincrement{5}
\lineation{section} %Jeder Abschnitt wird durchnummeriert
\renewcommand{\numlabfont}{\ssmall} %Schriftgröße Zeilennummern

%\AtBeginEnvironment{multicols}{\RaggedRight} % Linksbündig in Spalten


% SEITENUMBRÜCHE IM TEXT MARKIEREN

%% Seitenumbrüche


\newcommand{\Theight}{\dimexpr\fontcharht\font`W}
\newcommand{\pbposition}{\depth}
\newcommand{\pb}{\nobreak\hspace{0pt}\raisebox{-0.1em}{\raisebox{\pbposition}{\textnormal{|}}}\nobreak\hspace{0pt}}

% EINFÜGUNGEN IM TEXT MARKIEREN

\renewcaptionname{ngerman}{\contentsname}{Inhalt}           %Table of contents


\newcommand{\introOben}{\textnormal{\raisebox{\Theight}{\raisebox{-\height}{\small{v}\normalsize}}}}
\newcommand{\introUnten}{\textnormal{\raisebox{\Theight}{\raisebox{-\height}{\small{v}\normalsize}}}}
\newcommand{\introMitteVorne}{\textnormal{\raisebox{\Theight}{\raisebox{-\height}{\small{v}\normalsize}}}}
\newcommand{\introMitteHinten}{\textnormal{\raisebox{\Theight}{\raisebox{-\height}{\small{v}\normalsize}}}}
\newcommand{\substVorne}{\textnormal{\raisebox{\Theight}{\raisebox{-\height}{\rotatebox[origin=c]{180}{v}\normalsize}}}}
\newcommand{\substDazwischen}{}
\newcommand{\substHinten}{\textnormal{\raisebox{\Theight}{\raisebox{-\height}{\small{v}\normalsize}}}}


% MARGINALSPALTE
\setlength\ledrsnotewidth{1.5cm}


% FUSSNOTE
%% Im Apparat f. und ff.
\Xtwolines{f.}
\Xtwolinesbutnotmore

%% Sperrungen bei Lemmas im Apparat
%\pretocmd{\so}{\null}{}{}
% Hab ich auskommentiert: Hat einen Fehler ergeben, denn plötzlich war ein Abstand vor Absätzen, die mit einer Sperrung beginnen

%% Zeilennummerierung Abstand zum Lemma
\Xboxlinenum{5mm}

%% Bei zwei Apparateinträgen in einer Zeile wird nur beim ersten Mal die Zeile gezählt
\Xnumberonlyfirstinline
\Xnumberonlyfirstintwolines
\Xinplaceofnumber{1em}
\Xhangindent{1em}

% ENDNOTEN
\Xendlemmadisablefontselection[A]
\renewcommand*{\printnpnum}[1]{{\noindent}\tiny}
\Xendparagraph[A] % Endnoten in einem Absatz
%\Xendtwolines{\tiny{f.}}
\Xendbeforepagenumber{} 
\Xendnotenumfont[A]{\tiny}
\Xendboxlinenum[A]{0em}
\Xendlemmaseparator{$\rbracket$}
\Xendnotefontsize[A]{\footnotesize}
\Xendhangindent[A]{1em}
\Xendlemmafont[A]{\itshape}
\Xendlemmafont[B]{\bfseries}
\Xendnotefontsize[B]{\footnotesize}
\Xendnotenumfont{\footnotesize}
\Xendlineprefixsingle[C]{\tiny}
\Xendlineprefixmore[C]{\tiny}
\Xendlemmadisablefontselection
\Xendlemmafont{\itshape}
\Xendlinerangeseparator{\tiny{--}}
\Xendhangindent{4em}
\Xendboxlinenum{3.6em}
\Xendafternumber{0.4em}
\Xendboxlinenumalign{R}

%\Xendboxstartlinenum{3.5em}
%\Xendboxendlinenum{1em}


%% Kaufmanns-Und (=)
            
            

\newcommand{\kaufmannsund}{\&} 

%% Tabelle Zellensprung
% Ein weiterer Anlass, das Kaufmannsund in der Übergabe zu vermeiden:

\newcommand{\zellensprung}{ \& }

%% INDEX
    
    \makeindex 
    \newcommand*\lettergroup[1]{}
    
        \newcommand{\pw}[1]{#1}
        \newcommand{\pwt}[1]{\textbf{#1}}
        \newcommand{\pws}[1]{\upshape{\textbf{#1}}}
            
        \newcommand{\pwe}[1]{\textbf{\emph{#1}}}
             
    \newcommand{\pwk}[1]{#1\textsuperscript{\tiny{K}}}
    \newcommand{\pwv}[1]{\emph{#1}}
     \newcommand{\pwkv}[1]{\emph{#1}\textsuperscript{\tiny{K}}}
               \newcommand{\pwuv}[1]{\emph{#1}?}
               \newcommand{\pwu}[1]{#1?}
 \newcommand{\range}[2]{{\def\pw##1{##1}#1}--#2}

\newcommand{\buch}[1]{#1}


%% MEHRERE INDIZES

\newindex[Register]{pw}
%\newindex[Institutionen Organisationen Periodika und Unternehmen]{org}
%\newindex[Institutionen und Orte]{o}
\newindex[Korrespondenzpartner]{briefe-out}
\newindex[Gedruckte Quellen]{buch-abdruck}

\newcommand\briefsenderindex[1]{\sindex[briefe-out]{#1}}
\newcommand\briefempfaengerindex[1]{\sindex[briefe-out]{#1}}

\newcommand\buchabdruck[1]{\sindex[buch-abdruck]{#1}}
\renewcommand\buchabdruck[1]{}



%% Symbole

%\newcommand{\symaddr}{\includegraphics[height=6pt]{symbol/noun_637366.png}}
%\newcommand{\symweiteredrucke}{\includegraphics[height=6pt]{symbol/noun_634729.png}}
%\newcommand{\symdruckvorlage}{\includegraphics[height=6pt]{symbol/noun_637409.png}}
%\newcommand{\symstandort}{\includegraphics[height=6pt]{symbol/noun_634216.png}}
%\newcommand{\symhead}{\includegraphics[height=6pt]{symbol/noun_1162030_cc.png}}


\newcommand{\symaddr}{A}
\newcommand{\symweiteredrucke}{D}
\newcommand{\symdruckvorlage}{V}
\newcommand{\symstandort}{O}
\newcommand{\symhead}{H}



\newcommand\anhangTitel[2]{\toendnotes[C]{\hangpara{4em}{1}{\makebox[4em][l]{\textbf{#1}}\textbf{#2}}\endgraf}}
\newcommand\Adresse[1]{\toendnotes[C]{\hangpara{4em}{1}{\makebox[4em][l]{\makebox[3.6em][r]{\symaddr}}}#1\endgraf}}

\newcommand\buchAlsQuelle[1]{\toendnotes[C]{\footnotesize\par\hangpara{4em}{1}{\makebox[4em][l]{\makebox[3.6em][r]{\symdruckvorlage}}}#1\endgraf}}
\newcommand\buchAbdrucke[1]{\toendnotes[C]{\footnotesize\par\hangpara{4em}{1}{\makebox[4em][l]{\makebox[3.6em][r]{\symweiteredrucke}}}#1\endgraf}}
\newcommand\Standort[1]{\toendnotes[C]{\footnotesize\hangpara{4em}{1}{\makebox[4em][l]{\makebox[3.6em][r]{\symstandort}}}#1\endgraf}}
\newcommand\biographical[1]{\toendnotes[C]{\footnotesize\hangpara{4em}{1}{\makebox[4em][l]{\makebox[3.6em][r]{\symhead}}}#1\endgraf}}
\newcommand\biographicalOhne[1]{\toendnotes[C]{\footnotesize\hangpara{4em}{1}{\makebox[4em][l]{\makebox[3.6em][r]{}}}#1\endgraf}}



\newcommand\datumImAnhang[1]{\toendnotes[C]{#1}}

\let\newcell&

\newcommand\physDesc[1]{\toendnotes[C]{\hangpara{4em}{0}#1\endgraf}}
\newcommand\weitereDrucke[1]{#1}


% Schnitzler Tagebuch Auszüge
\newcommand{\prgrph}[1]{\endgraf\medskip\noindent\textbf{#1}\newline}


%% VERWEISE
% Dieser Befehl vom Typ
% \verweis{FW_V_schwn_A}{FW_V_schwn_E} 
% dient den Verweisen auf den Text von Kommentar und Herausgebereingriffen. Ihm werden die Namen der beiden Labels – Anfang und Ende – übergeben und er setzt den Anfang und entscheidet ob f. oder ff. folgt 


\newcounter{mystart}
\newcounter{mystop}
\newcounter{phantom}

\newcommand*\myrangeref[2]{%
  \setcounterpageref{mystart}{#1}%
  \setcounterpageref{mystop}{#2}%
  \ifnum\value{mystop}<\value{mystart}%
    \typeout{[myrangeref] Strange...stop (#2) before start (#1).}%
    \pageref{#2}--\pageref{#1}%
  \else
    \pageref{#1}%
    \ifnum\value{mystart}<\value{mystop}%
      \addtocounter{mystop}{-1}%
      \ifnum\value{mystart}<\value{mystop}%
        \,ff.
        %--\pageref{#2}%%
      \else
        \,f.
         %%--\pageref{#2}%
              \fi
    \fi
  \fi
}
            
\newcommand*\myrangerefkasten[2]{%
  \setcounterpageref{mystart}{#1}%
  \setcounterpageref{mystop}{#2}%
  \ifnum\value{mystop}<\value{mystart}%
    \typeout{[myrangeref] Strange...stop (#2) before start (#1).}%
    \pageref{#2}--\pageref{#1}%
  \else
    \makebox[12pt][r]{\pageref{#1}}%
    \ifnum\value{mystart}<\value{mystop}%
      \addtocounter{mystop}{-1}%
      \ifnum\value{mystart}<\value{mystop}%
        --\pageref{#2}%%
      \else
         --\pageref{#2}%
         % alternativ hierher: f.
      \fi
    \fi
  \fi
}


\newcommand*\mylabel[1]{%
  \refstepcounter{phantom}%
  \label{#1}%
}

\newenvironment{anhang}{\vspace{1cm}
}{}

\emfontdeclare{\itshape}

%% RAHMEN SEITLICH

\newlength{\leftbarwidth}
\setlength{\leftbarwidth}{3pt}
\newlength{\leftbarsep}
\setlength{\leftbarsep}{10pt}

\renewenvironment{leftbar}[1][\hsize]
{% 
\def\FrameCommand 
{%
{\hspace{-7pt} \color{black} \vrule width 0.5pt}%
\hspace{0pt}%must no space.
\fboxsep=\FrameSep\colorbox{white}%
}%
\MakeFramed{\hsize#1\advance\hsize-\width\FrameRestore}%
}
{\endMakeFramed}
\setlength{\FrameSep}{5pt}

\newmdenv[topline=false, leftline=true, rightline=true, bottomline=false,%
  linewidth=0.5pt, leftmargin=30pt, rightmargin=30pt, %
  skipabove=8pt, skipbelow=8pt]{mdbar}

% Überstreichung (OVERLINE)

\makeatletter
\newcommand*{\textoverline}[1]{$\overline{\hbox{#1}}\m@th$}
\makeatother

% Rahmen für Hintergrundfarbe
\fboxsep0mm

% Befehl für gekürzte Texte

\newcommand{\kuerzung}{, Auszug}

% Verse 

\setlength{\stanzaindentbase}{20pt} %Play with it later.
\setstanzaindents{5,1,1}
\setcounter{stanzaindentsrepetition}{2}
\newcommand{\stanzaend}{\&}
\sethangingsymbol{\protect\hfill}
\AtEveryStopStanza{\vspace{0.25\baselineskip}} %Abstand zwischen Strophen


% Versuch eines Grid

\RedeclareSectionCommand[
  beforeskip=3\baselineskip,
  afterskip=\baselineskip
]{chapter}
\RedeclareSectionCommand[
  beforeskip=2\baselineskip,
  afterskip=\baselineskip
]{section}

\newcommand\adjacent[2][]{%
  \bgroup
  \RedeclareSectionCommand[
    beforeskip=2\baselineskip,
    afterskip=\baselineskip,
  ]{chapter}%
  \if\relax\detokenize{#1}\relax
    \addchap{#2}%
  \else
    \addchap[#1]{#2}%
  \fi
  \egroup
  \section
}


%change the part format in table of contents
\renewcaptionname{ngerman}{\contentsname}{Inhalt} 


% Inhaltsverzeichnis

\AtBeginDocument{%
  \addtocontents{toc}{\protect\label{toc}}%
}

\renewcaptionname{ngerman}{\contentsname}{Verzeichnis der Dokumente} 
 
 
   \DeclareTOCStyleEntry[
  beforeskip=15pt,
  entryformat=\normalsize\normalfont\centering,
  pagenumberformat=\nullfont,
  linefill={},
  raggedentrytext=true
]{part}{part}

  \DeclareTOCStyleEntry[
  beforeskip=5pt,
  entryformat=\normalsize\normalfont\centering,
  pagenumberformat=\nullfont,
  linefill={},
  raggedentrytext=true
]{chapter}{chapter}

\DeclareTOCStyleEntry[
  onstarthigherlevel=\vspace*{0.5\baselineskip}\nobreak,
  indent=0pt,
  entryformat=\normalsize\def\autodot{.},
  pagenumberformat=\normalsize,
  raggedentrytext=true
]{section}{section}



 
% Das folgende auskommentiert, funktionierte nicht mehr, ging aber in Bahr/Schnitzler. Sollte eigentlich dazu dienen, beim Inhaltsverzeichnis die Nummern rechtsbündig zu setzen

 \iffalse
 
  \DeclareTOCStyleEntry[
  beforeskip=5pt,
  entryformat=\normalsize\normalfont\centering,
  pagenumberformat=\nullfont,
  linefill={},
  raggedentrytext=true
]{chapter}{chapter}

\DeclareTOCStyleEntry[
  onstarthigherlevel=\vspace*{0.5\baselineskip}\nobreak,
  indent=0pt,
  entryformat=\normalsize\def\autodot{.},
  pagenumberformat=\normalsize,
  raggedentrytext=true
]{section}{section}
 
 
  \newcommand*\sectionnumberbox[1]{\hfill #1\hspace{.6em}}

\newlength{\zweiziffern}
\newlength{\dreiziffern}
\newlength{\vierziffern}
\settowidth{\zweiziffern}{9999}
\settowidth{\dreiziffern}{99999}
\settowidth{\vierziffern}{99999999}
 
\BeforeStartingTOC[toc]{\value{tocdepth}=\sectiontocdepth}


\DeclareTOCStyleEntry[
  onstarthigherlevel=\vspace*{0.5\baselineskip}\nobreak,
  indent=0pt,
  entryformat=\normalsize\def\autodot{.},
  entrynumberformat=\sectionnumberbox,
  pagenumberformat=\normalsize,
  numwidth=\zweiziffern,
  raggedentrytext=true
]{section}{section}

\newcommand{\toccheck}{\ifnum \value{section}=76 \addtocontents{toc}{\protect\DeclareTOCStyleEntry[numwidth=\dreiziffern]{section}{section}} \else \ifnum \value{section}=990 \addtocontents{toc}{\protect\DeclareTOCStyleEntry[numwidth=\vierziffern]{section}{section}} \fi \fi}
\fi



% Längen für Tabellen
\newlength{\longeste}
\newlength{\longestz}
\newlength{\longestd}
\newlength{\longestv}
\newlength{\longestf}

\newcommand\halbtextwidth{0.9\textwidth}

\newcommand\pwindex[1]{{\sindex[pw]{#1}}}
\newcommand\oindex[1]{{\sindex[pw]{#1}}}
\newcommand\orgindex[1]{{\sindex[pw]{#1}}}

\renewcommand\oindex[1]{{{\sindex[pw]{#1}}}}
\renewcommand\orgindex[1]{{{\sindex[pw]{#1}}}}



% INDEX

%\renewcommand\pwindex[1]{}
%\renewcommand\oindex[1]{}
%\renewcommand\orgindex[1]{}
%\renewcommand\buchabdruck[1]{}


\newcommand\url[1]{\mbox{#1}}
\renewcommand\ngermanhyphenmins{33}

\makeatletter
\newcommand*{\geminationm}{$\overline{\hbox{m}}\m@th$}
\newcommand*{\geminationn}{$\overline{\hbox{n}}\m@th$}
\makeatother

%part
\renewcommand{\partmarkformat}{}
\renewcommand{\partheadmidvskip}{\enskip}
\renewcommand{\partformat}{}
\setkomafont{partnumber}{\usekomafont{part}}


%\geometry{headsep=8pt} % Abstand Kopfzeile - Text
%% DOKUMENT

\begin{document}

% Section ohne Nummer
\renewcommand*{\raggedsection}{%
 \CenteringLeftskip=1cm plus 1em\relax 
 \CenteringRightskip=1cm plus 1em\relax 
 \Centering\normalsize}



\widowpenalty=10000         % avoid widows
\clubpenalty=10000          % avoid orphans

\sloppy
\setlength{\parindent}{0em}

\setlength{\ledlsnotewidth}{4cm}
\setlength{\ledrsnotewidth}{4cm}
\renewcommand*{\ledlsnotefontsetup}{\scriptsize\sffamily}% left
\renewcommand*{\ledrsnotefontsetup}{\scriptsize\sffamily}% left
\thispagestyle{empty} 

               \section[Paul Goldmann an Arthur Schnitzler, 18. 8. 1890]{ Paul Goldmann an Arthur Schnitzler, 18. 8. 1890}\nopagebreak\mylabel{v}\rehead{ }\normalsize\beginnumbering\briefempfaengerindex{Schnitzler, Arthur@\textsc{Schnitzler, Arthur}!zzzGoldmann, Paul@\emph{von Paul Goldmann}!1890-08-181@{18. 8. 1890}|(be} \toendnotes[C]{\smallbreak\pagebreak[2]} \Standort{DLA, A:Schnitzler, HS.NZ85.1.3162.}
\physDesc{Brief, 2 Blätter, 7 Seiten
\newline{}Handschrift: schwarze Tinte, deutsche Kurrent
\newline{}Schnitzler: mit rotem Buntstift eine Unterstreichung }\toendnotes[C]{\smallbreak}\pstart
           \noindent{}\centering{}{\pb}\textcolor{gray}{\textbf{\textbf{Adminiſtration: \textcolor{pink}{VII.
                           Seidengaſſe 7}{}\ledrightnote{\textcolor{pink}{Seidengasse}}} (\textcolor{brown}{Jos. Eberle {\kaufmannsund} Co.}{}\ledrightnote{\textcolor{brown}{Josef Eberle  Stein-, Buch und Musikaliendruckerei}})}}\pend
           \pstart
           \noindent{}\centering{}\textcolor{gray}{\textbf{\textcolor{brown}{An der Schönen Blauen Donau}{}\ledrightnote{\textcolor{brown}{An der schönen blauen Donau}}}}\pend
           \pstart
           \noindent{}\centering{}\textcolor{gray}{\textbf{Chef-Redacteur: Dr. \textcolor{blue}{F.
                        Mamroth}{}\ledrightnote{\textcolor{blue}{Fedor Mamroth}}. – Redaction: \textcolor{pink}{IX.,
                        Berggaſſe 31}{}\ledrightnote{\textcolor{pink}{Berggasse}}.}}\pend
           \pstart
           \raggedleft{}\textsc{\textcolor{pink}{Pörtschach}{}\ledrightnote{\textcolor{pink}{Pörtschach}}}{ }\textcolor{gray}{\textbf{\strikeout{\textcolor{pink}{Wien}{}\ledrightnote{\textcolor{pink}{Wien}}}, den}}{ }18. August \textcolor{gray}{\textbf{18}}90.\pend
           \pstart\center{}Mein lieber Arthur!\pend\pstart
           Viel Dank für Deinen lieben Brief! Ich habe mich ehrlich damit gefreut, wenigſtens
               inſoweit, als ich ſehe, daß Du meiner in Treuen gedenkſt. Was Dich angeht, freilich –
               die Nachrichten über Deine Perſon, die die Epiſtel bringt, – bin ich wenig zufrieden.
               Wenig – nein, gar nicht! Kind, Kind – ſei geſcheit! Laß’ Dich nicht ſo willenlos
               untergehen in der \label{K_L02649-123v}\edtext{Geſchichte}{\lemma{\textnormal{\emph{Geſchichte}}}\Cendnote{\textnormal{Er spielt auf die Beziehung \textcolor{blue}{Schnitzler}s mit \textcolor{blue}{Marie Glümer} an, mit der dieser seit Juni
                     1889 eine Liebesbeziehung hatte. Am 13. 7. 1889 nennt er sie im \emph{\textcolor{green}{Tagebuch}} »das Ideal des ›süßen Mädels‹, wie ichs
                     geträumt«.}}}\label{K_L02649-123h}! Fühlen, Stimmung empfinden iſt gut; aber ein wenig
               Denken und Wollen iſt auch vonnöthen. Du brauchſt kein raſches Ende – \begin{otherlanguage}{french}pardon\end{otherlanguage}! – zu machen; aber \textcolor{gray}{da} das Ende
               von ſelbst kommen wird, {\pb}wäre es
               Wahnſinn, ſich nicht bei Zeiten damit abzufinden. Jetzt haſt Du das \textcolor{blue}{Mädel}{}\ledrightnote{→\textcolor{blue}{Marie Glümer}} – \textsc{\begin{otherlanguage}{french}bon\end{otherlanguage}}! – aber wenn Du das \textcolor{blue}{Mädel}{}\ledrightnote{→\textcolor{blue}{Marie Glümer}} nicht mehr haſt, wirſt Du etwas viel Beſſeres wieder haben – Dich
               ſelbſt. Der Tauſch iſt, weiß Gott, kein ſchlechter. Überleg’ Dir das! Und denk’ nur
               an meine Spießbürger-Philoſophie, die aber doch die einzig geſcheite iſt: der Menſch
               iſt nicht zum Lieben allein da. Dieſes Taumeln von Rauſch zu Rauſch, dieſes
               Selbſtzerquälen um ein Nichts iſt verderblich und zerrüttend. Beſonders dieſe
               Quälereien. Ich ſehe das ſo klar: in Dir iſt eine große Kunſt vorhanden, und da Du
               ſie nirgends hin ableiteſt, kehrt ſie ſich gegen Dich ſelbſt. Dieſe \label{K_L02649-22v}\edtext{Eiferſucht auf die Vergangenheit}{\lemma{\textnormal{\emph{Eiferſucht … Vergangenheit}}}\Cendnote{\textnormal{\textcolor{blue}{Schnitzler} war nicht der erste Liebhaber von
                     \textcolor{blue}{Marie Glümer} gewesen: »Ich bin
                     nie völlig glücklich mit ihr; weil ich eben das gewesene nie los werde. Sie
                     sagt, sie liebe mich unendlich mehr, ganz anders u. s. w.― Natürlich sagt sies.
                     Ja, natürlich glaubt sie’s. Es ist sonderbar, daß ich absolut nicht darüber weg
                     kann.« (A. S.: \emph{Tagebuch}, 10. 8. 1890)}}}\label{K_L02649-22h} iſt
               vielleicht nichts, als die Eiferſucht \uline{der}
               Vergangenheit, \uline{Deiner} Vergangenheit, jener Stunden,
               in denen Du geſchafft und geſtrebt haſt, jener hohen Ziele, denen Du zugeſtaunt, und
               die Dich jetzt wieder haben wollen. Nun, ſie \uline{werden}
               Dich wieder haben; und ich, der ich Dein Beſtes ſehe und will, kann das »Ende« nicht
               erwarten. Übrigens, glaube ich, es wird Dir nicht gar ſo weh thun. Dieſe tollen
               Schmerzen, die Du vorausempfindeſt, {\pb}ſtumpfen das Empfindungsvermögen ab, ſo daß es ſicherlich gegenüber dem großen
               Schmerze, wenn er wirklich eintritt, verſagen wird. Alſo, nochmals, ſei geſcheit: Du
               lebſt in \label{K_L02649-7v}\edtext{\textsc{\textcolor{pink}{Capua}{}\ledrightnote{\textcolor{pink}{Capua}}}}{\lemma{\textnormal{\emph{Capua}}}\Cendnote{\textnormal{Synonym für Luxus, Komfort etc.}}}\label{K_L02649-7h},
               und mußt ſroh ſein, wenn Du herauskommſt. Oder, wenn Du willſt, Du biſt im Paradieſe;
               aber, als ſrommer \textcolor{green}{Bibel}{}\ledrightnote{\textcolor{green}{Bibel}}leſer, \strikeout{ist \textcolor{gray}{d}} weißt Du, daß wir Alle da nicht hineingehören; und Du wirſt Dich doch wieder
               mit der Erde beſreunden müſſen, auf der zu leben ſchließlich auch nicht ohne Reiz
               iſt.\pend
           \pstart
           Dies die Moralpredigt eines Menſchen, der ſelbst nichts dringender brauchte, als eine
               ſolche. In Kurzem: auch mich hat’s wieder, mein Sohn! Das \label{K_L02649-2v}\edtext{ſüße \textcolor{blue}{Mädel}{}\ledrightnote{→\textcolor{blue}{Elise Pserhofer}}}{\lemma{\textnormal{\emph{ſüße Mädel}}}\Cendnote{\textnormal{Es handelt sich hierbei um eine frühe
                  Verwendung des von \textcolor{blue}{Schnitzler} populär
                  gemachten Begriffs. Im \emph{\textcolor{green}{Tagebuch}} findet sich
                  der Begriff bereits am 19. 10. 1887. In einem veröffentlichten literarischen Text gebrauchte
                  Schnitzler »süßes Mädel« erstmals im \emph{\textcolor{green}{Anatol}}-Einakter \emph{\textcolor{green}{Weihnachts-Einkäufe}}
                  (erschienen 24. 12. 1891).}}}\label{K_L02649-2h} – geſcheit,
               wahrhaftig und nicht \begin{otherlanguage}{french}coquett\end{otherlanguage}, das ich ſo lange mit
               der Laterne geſucht – mir ſcheint, ich hab’s gefunden. Seit geſtern ſind in mir wieder alle Teufel los. Und ich ſehe, es wird wieder
               genau die alte Geſchichte. Eine wahnſinnige Sehnſucht, das erblickte Glück zu faſſen,
               ein toller Geſühlsüberſchwang, ein Mich-Unwürdig-Fühlen gegenüber der \textcolor{blue}{Auserwählten}{}\ledrightnote{→\textcolor{blue}{Elise Pserhofer}} – dieſe drei
               Sachen, die es mir ſchon einmal verdorben haben, werden es mir wieder verderben. Da
               ſteh’ ich {\pb}nun mit meinem
               weltumfaſſenden Geiſte, und kann das praktiſche Problem nicht löſen, wie ich ein
               kleines \textcolor{blue}{Mädchen}{}\ledrightnote{→\textcolor{blue}{Elise Pserhofer}}herz lehren
               ſoll, mich gern zu haben. Dich quält das bevorſtehende Ende des Glücks, mich bringt
               es zur Verzweiflung, daß ich ſeinen Anfang nicht herbeiführen kann. So bin ich geſtern{ }Abend geſeſſen, den Kopf in beide Hände geſtützt und die Stirne heiß von
               Rauſch und Sehnſucht, und es hat in mir gewühlt und gewühlt und ich habe geſehen, daß
               ich ein hoffnungslos unglücklicher Menſch bin. Hab’ ich’s alſo wieder einmal mit dem
               Beten verſucht – Du weißt, ich gedenke gern des lieben Gottes, wenn ich ihn brauche –
               und warte nun ab, ob mir das vielleicht nutzen wird. Ich habe mir bei alledem ſo heiß
               gewünſcht, Du zu ſein, mit all' Deinen Reizen und \strikeout{\textcolor{gray}{Liſten}} Liſten, Du, der Du die große Kunſt verſtehſt: geliebt zu werden. Vielleicht
               theilſt Du mir ein oder das andere \label{K_L02649-3v}\edtext{\textsc{arcanum}}{\lemma{\textnormal{\emph{arcanum}}}\Cendnote{\textnormal{lateinisch: Geheimnis}}}\label{K_L02649-3h} mit. Wie
               geſagt: mir ſcheint, ich habe das Richtige gefunden, und ich wäre außer mir vor
               Schmerz, wenn ich es wieder nicht faſſen könnte.\pend
           \pstart
           Thatſächliches – unter Discretion, würde \textsc{\textcolor{blue}{Fritz Kapper}{}\ledrightnote{\textcolor{blue}{Friedrich Kapper}}} ſagen. Das Richtige heißt: {\pb}\textsc{\textcolor{blue}{Lisi Pserhofer}{}\ledrightnote{\textcolor{blue}{Elise Pserhofer}}}, \textcolor{blue}{Tochter}{}\ledrightnote{→\textcolor{blue}{Elise Pserhofer}} des bekannten
                  \textsc{\textcolor{blue}{Apothekers}{}\ledrightnote{→\textcolor{blue}{Ignaz Pserhofer}}}, \label{K_L02649-11v}\edtext{Familie \textsc{Mautner}, \textsc{Ernst}}{\lemma{\textnormal{\emph{Familie Mautner, Ernst}}}\Cendnote{\textnormal{Die drei genannten Familien Pserhofer,
                  von Mauthner und Ernst werden durch drei Schwestern verbunden, alle geborene
                  Benedikt: \textcolor{blue}{Emma}, die Mutter von \textcolor{blue}{Elise Pserhofer} und Ehefrau von \textcolor{blue}{Ignaz Pserhofer}; \textcolor{blue}{Betty Ernst} und \textcolor{blue}{Hermine
                     von Mauthner}, die Mutter der beiden in Folge genannten Söhne.}}}\label{K_L02649-11h}{ }\textsc{etc}. Noch iſt es mir nicht gelungen, in den intimen Kreis
               dieſer Leute einzudringen, die ſich hier vollkommen reſervirt verhalten\strikeout{,} und den einzig erſtrebenswerthen Verkehr
               repräſentiren. Kennſt du nicht die beiden \textsc{\textcolor{blue}{Mautner}{}\ledrightnote{→\textcolor{blue}{Hans Johann von Mauthner}{\newline}→\textcolor{blue}{Stephan von Mauthner}}}’s \strikeout{,}{ }\textsc{\textcolor{blue}{Hans}{}\ledrightnote{\textcolor{blue}{Hans Johann von Mauthner}}} und \textsc{\textcolor{blue}{Stephan}{}\ledrightnote{\textcolor{blue}{Stephan von Mauthner}}}? Und kannſt Du mir nicht ein wenig helfen? Den Leuten ein Wort ſchreiben, daß
               ich ein anſtändiger Menſch bin ober ſo was? \textsc{\textcolor{blue}{Max Rosenberg}{}\ledrightnote{\textcolor{blue}{Max von Rosenberg}}} kennt ſie, wie mir ſcheint, ſehr gut; aber der iſt wohl nicht in \textcolor{pink}{Wien}{}\ledrightnote{\textcolor{pink}{Wien}}. Das ſind nur ſo akademiſche Fragen. Ich ſehne
               mich nach irgend einer Hilfe von Außen, da ich mich ſelbſt ſo unendlich ſchwach
               fühle. Oder kennſt Du das {\pb}\textcolor{blue}{Mädel}{}\ledrightnote{→\textcolor{blue}{Elise Pserhofer}} ſelber und weißt etwas
               von ihr? Vielleicht etwas Ungünſtiges? Noch wäre es Zeit, ſich die Geſchichte aus dem
               Herzen zu reißen.\pend
           \pstart
           Sonſt \label{K_L02649-4v}\edtext{wimmelt der \textcolor{pink}{Ort}{}\ledrightnote{→\textcolor{pink}{Pörtschach}} wohl
               von Menſchen}{\lemma{\textnormal{\emph{wimmelt … Menſchen}}}\Cendnote{\textnormal{\textcolor{blue}{Beer-Hofmann} war in diesem Sommer ebenfalls in \textcolor{pink}{Pörtschach}
                  und lernte hier \textcolor{blue}{Goldmann} und \textcolor{blue}{Leo Van-Jung} kennen, so
               dass auch eine Bekanntschaft zwischen den letzteren beiden anzunehmen ist.}}}\label{K_L02649-4h}, aber es iſt Alles das gewöhnliche Börſenjuden-Niveau, blöd, frech,
               unſympathiſch, die Landſchaſt iſt großartig, aber Du weißt, wie ſehr ich auf die
               Landſchaft pfeife, wenn ich nicht bei ihrem Anblick am Abend eine weiche Hand drücken
               kann und dabei ſagen: »Süßes Mädel!«\pend
           \pstart
           Geleſen: die 
                  \textcolor{green}{Kreutzer-Sonate}{}\ledrightnote{\textcolor{green}{Die Kreutzersonate}}. Kritiſch großartig\strikeout{\textcolor{gray}{e}}, das Poſitive aber wahnſinnig und pervers. Aber Alles in Allem ein echter \textsc{\textcolor{blue}{Tolstoi}{}\ledrightnote{\textcolor{blue}{Leo N. von Tolstoi}}} und höchſt leſenswerth. Sonſt nichts. Geſchrieben auch nichts. Von der »\textcolor{brown}{Preſſe}{}\ledrightnote{\textcolor{brown}{Die Presse}}« höre ich allerlei Sorgenvolles. \textsc{\textcolor{blue}{Granichstaedten}{}\ledrightnote{\textcolor{blue}{Emil Granichstaedten}}} ſoll fortgehen, und man ſucht einen Erſatz, aber nicht mich. Hierbleiben werde
               ich ſo lange als möglich, zumindeſt eine Woche. Könnteſt du nicht auch einen Sprung
                  \label{K_L02649-8v}\edtext{herkommmen}{\lemma{\textnormal{\emph{herkommmen}}}\Cendnote{\textnormal{\textcolor{blue}{Schnitzler} kam
                     1890 nicht nach \textcolor{pink}{Pörtschach}.}}}\label{K_L02649-8h}? Jedenfalls \strikeout{ſch} ſchreib’
               mir bald über all’ das Wichtige, das ich Dich gefragt. Wieder \textsc{Poste restante}.\pend
           \pstart
           {\pb}Viele herzliche Grüße an \textcolor{blue}{Herrn}{}\ledrightnote{→\textcolor{blue}{Friedrich Kapper}} und \textcolor{blue}{Frau}{}\ledrightnote{\textcolor{blue}{Adele Kapper}}{ }\textsc{Fritz}. Ebenſo an Dich!\pend
           \pstart
           Dein {\\[\baselineskip]}\spacefill\mbox{Paul Goldmann.}\pend
           \leftskip=0em{}\pstart
           \noindent{}Empfehlungen an Deine\strikeout{\textcolor{gray}{n}}{ }\textcolor{blue}{Schweſter}{}\ledrightnote{→\textcolor{blue}{Gisela Hajek}} und deinen \textcolor{blue}{Schwager}{}\ledrightnote{→\textcolor{blue}{Markus Hajek}}, die ſich wie
                  befinden?\pend
           \pstart
           Bitte, antworte raſch! Mir ſcheint übrigens, ich hab’ das ſchon oben irgendwo
                  geſagt.\pend
           \pstart
           Unter Discretion: \textsc{\textcolor{pink}{Pörtschach}{}\ledrightnote{\textcolor{pink}{Pörtschach}}} liegt in \textsc{\textcolor{pink}{\uline{Kärnthen}}{}\ledrightnote{\textcolor{pink}{Kärnten}}}.\pend
           \endnumbering\briefempfaengerindex{Schnitzler, Arthur@\textsc{Schnitzler, Arthur}!zzzGoldmann, Paul@\emph{von Paul Goldmann}!1890-08-181@{18. 8. 1890}|)be}\mylabel{h}  
         \normalsize

\newenvironment{esempio}[3]%
{
    \vspace{1.5ex}
    \rlap{\underline{#1}}
    \par
    \setlength{\parindent}{0cm}
    \nopagebreak
    \leftskip=#2cm
    \rightskip=#3cm
}
{
    \par
}

\doendnotes{C}
\bigskip

\printindex[pw]


\end{document}
      