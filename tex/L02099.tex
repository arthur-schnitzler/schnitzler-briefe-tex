%% latex-korrekturansicht-vorspann.tex
%% Vorspann für die Korrekturansicht.
%% Lädt die gemeinsame Datei latex-vorspann.tex mit gesetztem Schalter.

\newif\ifkorrekturansicht
\korrekturansichttrue

\input{../tex-inputs/latex-vorspann}


               \section[Arthur Schnitzler an Hermann Bahr, 16. 11. 1912]{ Arthur Schnitzler an Hermann Bahr, 16. 11. 1912}\nopagebreak\mylabel{v}\rehead{ }\normalsize\beginnumbering\briefempfaengerindex{Bahr, Hermann@\textsc{Bahr, Hermann}!zzzSchnitzler, Arthur@\emph{von Arthur Schnitzler}!1912-11-161@{16. 11. 1912}|(be} \toendnotes[C]{\smallbreak\pagebreak[2]} \Standort{TMW, HS AM 60161 Ba.}
\physDesc{Briefkarte
\newline{}Schreibmaschine
\newline{}Handschrift: schwarze Tinte, deutsche Kurrent (\noindent{}Korrektur und Grußformel)\newline{}Ordnung: Lochung }\buchAbdrucke{\weitereDrucke{1) \emph{16. 11. 1912, Abschrift.} In: Arthur Schnitzler: \emph{The Letters of Arthur Schnitzler to Hermann Bahr}. Edited, annotated, and with an introduction, by Donald G.
                        Daviau. Chapel Hill: \emph{The University of North Carolina Press} 1978, S. 109–110 (University of North Carolina studies in the Germanic languages
                        and literatures, 89).} \weitereDrucke{2) Hermann Bahr, Arthur Schnitzler: \emph{Briefwechsel, Aufzeichnungen, Dokumente (1891–1931)}. Hg. Kurt Ifkovits und Martin Anton Müller. Göttingen: \emph{Wallstein} 2018, S. 479.} }\pstart
           \noindent{}{\pb}\textcolor{gray}{\textbf{Dr. Arthur Schnitzler}}\hfill 16. 11. 1912. \pend
           \pstart
           \textcolor{gray}{\textbf{\textcolor{pink}{Wien XVIII. Sternwartestrasse 71}{}\ledrightnote{\textcolor{pink}{Sternwartestraße}}}}\pend
           \pstart{}Lieber Hermann.\pend\pstart
           Neulich schrieb mir \textcolor{blue}{Peter Altenberg}{}\ledrightnote{\textcolor{blue}{Peter Altenberg}}, dass eine
               Anzahl derjenigen Leute, die ihn im Laufe der letzten Jahre regelmässig
               unterstützten, allmählich ausgesprungen seien und frägt mich zugleich, ob ich bereit
               wäre an Stelle dieser Leute einzutreten und andere in gleichem Sinn zu gewinnen.
               Unter diesen nennt er Dich und so frage ich an, ob Du bereit wärst ihm monatlich bis
               auf Weiteres einen von Dir zu bestimmenden Betrag anzuweisen, wie es vorläufig \textcolor{blue}{Hugo}{}\ledrightnote{\textcolor{blue}{Hugo von Hofmannsthal}} und ich zu tun gedenken. Bist Du
               einverstanden, so teile es mir freundlichst mit und schreibe zugleich an \textcolor{blue}{S. Fischer}{}\ledrightnote{\textcolor{blue}{Samuel Fischer}}, mit welchem Betrag Du Dich zu
               beteiligen ge{\pb}denkst. \substVorne{}\textsuperscript{Dieser}{\allowbreak}\substDazwischen{}\textcolor{blue}{\textsc{\damage{Fis}cher}}{}\ledrightnote{\textcolor{blue}{Samuel Fischer}}\substHinten{} will es nämlich übernehmen das Geld allmonatlich an \textcolor{blue}{P. A.}{}\ledrightnote{\textcolor{blue}{Peter Altenberg}} zu expedieren.\pend
           \pstart
           Ich schreibe Dir noch an Deine \textcolor{pink}{St. Veiter}{}\ledrightnote{\textcolor{pink}{Sankt Veit}}-Adresse,
               obwohl ich ja annehmen muss, dass Du schon in der Uebersiedelung nach \textcolor{pink}{Salzburg}{}\ledrightnote{\textcolor{pink}{Salzburg}} begriffen bist.\pend
           \pstart
           Auf baldiges Wiedersehen und herzliche Grüsse\pend
           \pstart
           {[}hs.:{]} Dein{\\[\baselineskip]}\spacefill\mbox{Arthur}\pend
           \leftskip=0em{}\endnumbering\briefempfaengerindex{Bahr, Hermann@\textsc{Bahr, Hermann}!zzzSchnitzler, Arthur@\emph{von Arthur Schnitzler}!1912-11-161@{16. 11. 1912}|)be}\mylabel{h}  \normalsize

\doendnotes{C}
\bigskip
\vfill

\clearpage

\footnotesize

\lohead{\textsc{register}}

% Definiere theindex-Environment komplett neu ohne reledmac
\makeatletter
\renewenvironment{theindex}{%
  \section*{\indexname}%
  \setlength{\parindent}{0pt}%
  \setlength{\parskip}{0pt plus 0.3pt}%
  \let\item\@idxitem
}{%
  \clearpage
}
\makeatother

\IfFileExists{\jobname-pw.ind}{\input{\jobname-pw.ind}}{}

\end{document}

      