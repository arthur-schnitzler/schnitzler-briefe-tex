%% latex-korrekturansicht-vorspann.tex
%% Vorspann für die Korrekturansicht.
%% Lädt die gemeinsame Datei latex-vorspann.tex mit gesetztem Schalter.

\newif\ifkorrekturansicht
\korrekturansichttrue

\input{../tex-inputs/latex-vorspann}


               \section[Friedrich M. Fels an Arthur Schnitzler, {[}20. 1. 1893{]}]{ Friedrich M. Fels an Arthur Schnitzler, {[}20. 1. 1893{]}}\nopagebreak\mylabel{v}\rehead{ }\normalsize\beginnumbering\briefempfaengerindex{Schnitzler, Arthur@\textsc{Schnitzler, Arthur}!zzzFels, Friedrich Michael@\emph{von Friedrich Michael Fels}!1893-01-201@{{[}20. 1. 1893{]}}|(be} \toendnotes[C]{\smallbreak\pagebreak[2]} \Standort{DLA, A:Schnitzler, HS.NZ85.1.2956.}
\physDesc{Brief, 1 Blatt, 1 Seite
\newline{}Handschrift: schwarze Tinte, lateinische Kurrent
\newline{}Schnitzler: mit Bleistift datiert: »20/1 93« und nummeriert: »2« }\toendnotes[C]{\smallbreak}\pstart
           \noindent{}{\pb}Lieber Dr Schnitzler! Heute früh beschloß, die Apathie fahren
                    zu laßen und selbst energisch mich zum Fleischfreßer auszubilden. Wolan!
                        Progra{\geminationm}: Bureau, Eßen, Café. Allerdings die
                    Kälte hat mich scheußlich niedergesti{\geminationm}t; das ist
                    ja abscheulich. Im Bureau habe ich mir vom Diener aus dem Ihnen beka{\geminationn}ten Lokal genau unsere Speisekarte von neulich
                        wi{[}e{]}derholen laßen und habe \uline{das Ganze aufgefreßen}, was genügt. Nun werde wahrscheinlich
                        \textcolor{pink}{Central}{}\ledrightnote{\textcolor{pink}{Café Central}} gehen und mit Rücksicht auf
                    Zeitung, Beka{\geminationn}ten u. v. a. Abort.\pend
           \pstart
           Ob Sie mit meinem heutigen Tag zufrieden sind, weiß \label{T_L00160_1v}\edtext{ich}{\lemma{\textnormal{\emph{ich}}}\Cendnote{\textnormal{Er schreibt: »ich ich«.}}}\label{T_L00160_1h} nicht, obwol es
                    eigentlich \introOben{}gut\introOben{} angebracht ist, aber, ich glaube, mit
                    der Instruktion, die Sie mir gegeben, sti{\geminationm}t es
                    wenig.\pend
           \pstart
           Jedenfalls, damit ich nicht ganz in dieser Selbstverständlichkeit bleibe, ersuche
                    ich Sie, mich morgen in meinen Bureaustunden zu besuchen, zu strafen, zu
                    kasteien,\pend
           \pstart \spacefill\mbox{Fels}\pend{}\pstart
           \noindent{}Herzl. Gruß!\pend
           \endnumbering\briefempfaengerindex{Schnitzler, Arthur@\textsc{Schnitzler, Arthur}!zzzFels, Friedrich Michael@\emph{von Friedrich Michael Fels}!1893-01-201@{{[}20. 1. 1893{]}}|)be}\mylabel{h}  \normalsize

\doendnotes{C}
\bigskip
\vfill

\clearpage

\footnotesize

\lohead{\textsc{register}}

% Definiere theindex-Environment komplett neu ohne reledmac
\makeatletter
\renewenvironment{theindex}{%
  \section*{\indexname}%
  \setlength{\parindent}{0pt}%
  \setlength{\parskip}{0pt plus 0.3pt}%
  \let\item\@idxitem
}{%
  \clearpage
}
\makeatother

\IfFileExists{\jobname-pw.ind}{\input{\jobname-pw.ind}}{}

\end{document}

      