%% latex-korrekturansicht-vorspann.tex
%% Vorspann für die Korrekturansicht.
%% Lädt die gemeinsame Datei latex-vorspann.tex mit gesetztem Schalter.

\newif\ifkorrekturansicht
\korrekturansichttrue

\input{../tex-inputs/latex-vorspann}


               \section[Arthur Schnitzler an Richard Dehmel, 22. 3. 1903]{ Arthur Schnitzler an Richard Dehmel, 22. 3. 1903}\nopagebreak\mylabel{v}\rehead{ }\normalsize\beginnumbering\briefempfaengerindex{Dehmel, Richard@\textsc{Dehmel, Richard}!zzzSchnitzler, Arthur@\emph{von Arthur Schnitzler}!1903-03-221@{22. 3. 1903}|(be} \toendnotes[C]{\smallbreak\pagebreak[2]} \Standort{Hamburg, Staats- und Universitätsbibliothek, DA:Br:S:618.}
\physDesc{Brief, 1 Blatt, 2 Seiten
\newline{}Handschrift: schwarze Tinte, deutsche Kurrent}\toendnotes[C]{\smallbreak}\pstart{}{\pb}Verehrteſter Herr Dehmel,\pend\pstart
           für die freundliche Überſendung Ihres neuen \textcolor{green}{Buches}{}\ledrightnote{→\textcolor{green}{Zwei Menschen. Roman in Romanzen}} danke ich Ihnen herzlich. In der \textcolor{green}{N. D. R.}{}\ledrightnote{\textcolor{green}{Neue Deutsche Rundschau}} war wohl ein \label{K_L01278_1v}\edtext{\textcolor{green}{Theil}{}\ledrightnote{→\textcolor{green}{Zwei Menschen. Roman in Romanzen}}}{\lemma{\textnormal{\emph{Theil}}}\Cendnote{\textnormal{Im Januar-Heft
                        erschienen mehrere Romanzen (\textcolor{blue}{Richard Dehmel}: \emph{\textcolor{green}{Zwei Menschen. Romanzen}}. In: \emph{\textcolor{green}{Neue Deutsche Rundschau}}, Jg. 14, H. 1,
                                15. 1. 1903, S. 49–76).}}}\label{K_L01278_1h} davon abgedruckt;
                    was ich dort las, hat mich außerordentlich ergriffen und ich hab es dem
                    allerſchönſten zugerechnet, was ich von Ihnen {\pb}kenne.
                    Nun freue ich mich ſehr, liebgewonnenes bekanntes \substVorne{}\textsuperscript{\textcolor{gray}{neu}}\substDazwischen{}in\substHinten{} ein\substVorne{}\textsuperscript{e}\substDazwischen{}em\substHinten{} herbeigewünſchte\substVorne{}\textsuperscript{s}\substDazwischen{}n\substHinten{} ganze\substVorne{}\textsuperscript{s}\substDazwischen{}n\substHinten{} aufzunehmen.\pend
           \pstart
           Ihr Sie aufrichtig hochſchätzender{\\[\baselineskip]}\spacefill\mbox{Arthur Schnitzler}\pend
           \leftskip=0em{}\pstart
           \textcolor{pink}{Wien}{}\ledrightnote{\textcolor{pink}{Wien}}{ }22/3 903\pend
           \endnumbering\briefempfaengerindex{Dehmel, Richard@\textsc{Dehmel, Richard}!zzzSchnitzler, Arthur@\emph{von Arthur Schnitzler}!1903-03-221@{22. 3. 1903}|)be}\mylabel{h}  \normalsize

\doendnotes{C}
\bigskip
\vfill

\clearpage

\footnotesize

\lohead{\textsc{register}}

% Definiere theindex-Environment komplett neu ohne reledmac
\makeatletter
\renewenvironment{theindex}{%
  \section*{\indexname}%
  \setlength{\parindent}{0pt}%
  \setlength{\parskip}{0pt plus 0.3pt}%
  \let\item\@idxitem
}{%
  \clearpage
}
\makeatother

\IfFileExists{\jobname-pw.ind}{\input{\jobname-pw.ind}}{}

\end{document}

      