%% latex-korrekturansicht-vorspann.tex
%% Vorspann für die Korrekturansicht.
%% Lädt die gemeinsame Datei latex-vorspann.tex mit gesetztem Schalter.

\newif\ifkorrekturansicht
\korrekturansichttrue

\input{../tex-inputs/latex-vorspann}


               \section[Arthur Schnitzler an Hugo von Hofmannsthal, 24. 11. 1892]{ Arthur Schnitzler an Hugo von Hofmannsthal, 24. 11. 1892}\nopagebreak\mylabel{v}\rehead{ }\normalsize\beginnumbering\briefempfaengerindex{Hofmannsthal, Hugo von@\textsc{Hofmannsthal, Hugo von}!zzzSchnitzler, Arthur@\emph{von Arthur Schnitzler}!1892-11-241@{24. 11. 1892}|(be} \toendnotes[C]{\smallbreak\pagebreak[2]} \Standort{FDH, Hs-30885,27.}
\physDesc{Brief, 1 Blatt, 3 Seiten
\newline{}Handschrift: Bleistift, deutsche Kurrent\newline{}Ordnung: mit Bleistift von Schnitzler mutmaßlich während der Durchsicht
                                 der Briefe 1929 am oberen Rand der ersten Seite
                                 datiert: »24/11 92« }\buchAbdrucke{\weitereDrucke{1) Hugo von Hofmannsthal, Arthur Schnitzler: \emph{Briefwechsel}. Hg. Therese Nickl und Heinrich Schnitzler. Frankfurt am Main: \emph{S. Fischer} 1964, S. 31–32.} \weitereDrucke{2) Hermann Bahr, Arthur Schnitzler: \emph{Briefwechsel, Aufzeichnungen, Dokumente (1891–1931)}. Hg. Kurt Ifkovits und Martin Anton Müller. Göttingen: \emph{Wallstein} 2018.} }\toendnotes[C]{\smallbreak}\pstart{}{\pb}Lieber Loris,\pend\pstart
           ſehr wahr! – Und wie denken Sie z. B. darüber, für einen Abend der Woche ſtatt des
                  \textcolor{pink}{Pfob}{}\ledrightnote{\textcolor{pink}{Café Pfob}} ein anderes Café zu beſti{\geminationm}en, in dem \uline{nur}{ }\uline{wir} zuſa{\geminationm}en ko{\geminationm}en? – Und eventuell \textcolor{blue}{Bahr}{}\ledrightnote{\textcolor{blue}{Hermann Bahr}}. Ich wiederhole übrigens, was ich Ihnen ſchon \label{K_L00139_1v}\edtext{neulich geſchrieben}{\lemma{\textnormal{\emph{neulich geſchrieben}}}\Cendnote{\textnormal{am 9. 11. 1892 (\emph{Briefwechsel} Hofmannsthal/Schnitzler
                  31).}}}\label{K_L00139_1h}, daſs ich nämlich ſehr {\pb}unangenehm
               enttäuſcht bin, auch heuer ſo wenig mit Ihnen zuſa{\geminationm}en zu
                  ko{\geminationm}en.\pend
           \pstart
           Beſti{\geminationm}en Sie Abend, beſti{\geminationm}en Sie Caféhaus – und beſti{\geminationm}en Sie \substVorne{}\textsuperscript{und}\substDazwischen{}vielleicht\substHinten{} auch \textcolor{blue}{Bahr}{}\ledrightnote{\textcolor{blue}{Hermann Bahr}}, einmal hinzuko{\geminationm}en.\pend
           \pstart
           \label{K_L00139_2v}\edtext{So{\geminationn}tag
               alſo bei mir}{\lemma{\textnormal{\emph{Sotag
               alſo bei mir}}}\Cendnote{\textnormal{Am 27. 11. 1892 ist lediglich der Besuch \textcolor{blue}{Hofmannsthals} in \textcolor{blue}{Schnitzler}s \emph{\textcolor{green}{Tagebuch}} erwähnt.}}}\label{K_L00139_2h},
               für alle Fälle? – Ich möchte mir den Vorſchlag erlauben, daſs Sie {\pb}Ihre \textsc{psychol.}{ }\label{K_L00139_3v}\edtext{\textcolor{green}{Novellette}{}\ledrightnote{→\textcolor{green}{Age of Innocence}}}{\lemma{\textnormal{\emph{Novellette}}}\Cendnote{\textnormal{\emph{\textcolor{green}{Age of Innocence}} (postum veröffentlicht
                     1930).}}}\label{K_L00139_3h} (die von der \textcolor{green}{\textsc{Freien Bühne}}{}\ledrightnote{\textcolor{green}{Freie Bühne für den Entwickelungskampf der Zeit}} refüſirt wurde) vorleſen. Ich glaube, daſs weder \textcolor{blue}{\textsc{Richard}}{}\ledrightnote{\textcolor{blue}{Richard Beer-Hofmann}} noch \textcolor{blue}{\textsc{Salten}}{}\ledrightnote{\textcolor{blue}{Felix Salten}} dieſelbe kennen. –\pend
           \pstart
           Herzlich der Ihre{\\[\baselineskip]}\spacefill\mbox{Arthur}\pend
           \leftskip=0em{}\pstart
           \textcolor{pink}{Wien}{}\ledrightnote{\textcolor{pink}{Wien}}{ }24. XI. 92.\pend
           \endnumbering\briefempfaengerindex{Hofmannsthal, Hugo von@\textsc{Hofmannsthal, Hugo von}!zzzSchnitzler, Arthur@\emph{von Arthur Schnitzler}!1892-11-241@{24. 11. 1892}|)be}\mylabel{h}  \normalsize

\doendnotes{C}
\bigskip
\vfill

\clearpage

\footnotesize

\lohead{\textsc{register}}

% Definiere theindex-Environment komplett neu ohne reledmac
\makeatletter
\renewenvironment{theindex}{%
  \section*{\indexname}%
  \setlength{\parindent}{0pt}%
  \setlength{\parskip}{0pt plus 0.3pt}%
  \let\item\@idxitem
}{%
  \clearpage
}
\makeatother

\IfFileExists{\jobname-pw.ind}{\input{\jobname-pw.ind}}{}

\end{document}

      