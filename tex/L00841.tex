%% latex-korrekturansicht-vorspann.tex
%% Vorspann für die Korrekturansicht.
%% Lädt die gemeinsame Datei latex-vorspann.tex mit gesetztem Schalter.

\newif\ifkorrekturansicht
\korrekturansichttrue

\input{../tex-inputs/latex-vorspann}


               \section[Hugo von Hofmannsthal an Arthur Schnitzler, 30. 8. 1898]{ Hugo von Hofmannsthal an Arthur Schnitzler, 30. 8. 1898}\nopagebreak\mylabel{v}\rehead{ }\normalsize\beginnumbering\briefempfaengerindex{Schnitzler, Arthur@\textsc{Schnitzler, Arthur}!zzzHofmannsthal, Hugo von@\emph{von Hugo von Hofmannsthal}!1898-08-301@{30. 8. 1898}|(be} \toendnotes[C]{\smallbreak\pagebreak[2]} \Standort{CUL, Schnitzler, B 43.}
\physDesc{Postkarte
\newline{}Handschrift: Bleistift, deutsche Kurrent\newline{}Versand: 1) Stempel: »\nobreak{}\oindex{Lugano@\textbf{Lugano}, \emph{Besiedelter Ort (A.BSO)}|pwk}Lugano, 30. VIII. 98, 8\nobreak{}«.  2) Stempel: »\nobreak{}\oindex{Bologna@\textbf{Bologna}, \emph{Besiedelter Ort (A.BSO)}|pwk}Bologna, 30. VIII. 98, 8\nobreak{}«. 3) Stempel: »\nobreak{}\oindex{Bologna@\textbf{Bologna}, \emph{Besiedelter Ort (A.BSO)}|pwk}{[}Bo{]}logna, {[}31.{]}  8. 1898, 8H\nobreak{}«. \newline{}Ordnung: 1) mit Bleistift von unbekannter Hand nummeriert: »\strikeout{134}« 2) mit Bleistift von unbekannter Hand nummeriert: »123«}\buchAbdrucke{\weitereDrucke{Hugo von Hofmannsthal, Arthur Schnitzler: \emph{Briefwechsel}. Hg. Therese Nickl und Heinrich Schnitzler. Frankfurt am Main: \emph{S. Fischer} 1964, S. 111.} }\pstart{}{\pb}\textsc{Herrn D\textsuperscript{r} Arthur
                            Schnitzler}\pend{}\pstart{}\textcolor{pink}{\textsc{Italia}}{}\ledrightnote{\textcolor{pink}{Italien}}\pend{}\pstart{}\textcolor{pink}{\textsc{Bologna}}{}\ledrightnote{\textcolor{pink}{Bologna}}\pend{}\pstart{}\textsc{ferma in posta}\pend{}{\bigskip}\pstart
           \raggedleft{}{\pb}\textcolor{pink}{Lugano}{}\ledrightnote{\textcolor{pink}{Hôtel du Parc}}{ }30. XIII.\pend
           \pstart
           lieber, ich lebe nun ganz ruhig und zufrieden, ſchreibe etwas
                    Proſa, erwarte \textcolor{blue}{Richard}{}\ledrightnote{\textcolor{blue}{Richard Beer-Hofmann}} und genieße die nun
                    ſehr ſchöngefärbte reine Luft.\pend
           \pstart
           Mit Briefen oder Karten machen Sie mir eine große Freude,{\\}und hierher!\pend
           \pstart
           Von Herzen Ihr{\\[\baselineskip]}\spacefill\mbox{Hugo.}\pend
           \leftskip=0em{}\endnumbering\briefempfaengerindex{Schnitzler, Arthur@\textsc{Schnitzler, Arthur}!zzzHofmannsthal, Hugo von@\emph{von Hugo von Hofmannsthal}!1898-08-301@{30. 8. 1898}|)be}\mylabel{h}  \normalsize

\doendnotes{C}
\bigskip
\vfill

\clearpage

\footnotesize

\lohead{\textsc{register}}

% Definiere theindex-Environment komplett neu ohne reledmac
\makeatletter
\renewenvironment{theindex}{%
  \section*{\indexname}%
  \setlength{\parindent}{0pt}%
  \setlength{\parskip}{0pt plus 0.3pt}%
  \let\item\@idxitem
}{%
  \clearpage
}
\makeatother

\IfFileExists{\jobname-pw.ind}{\input{\jobname-pw.ind}}{}

\end{document}

      