%% latex-korrekturansicht-vorspann.tex
%% Vorspann für die Korrekturansicht.
%% Lädt die gemeinsame Datei latex-vorspann.tex mit gesetztem Schalter.

\newif\ifkorrekturansicht
\korrekturansichttrue

\input{../tex-inputs/latex-vorspann}


               \section[Gerhart Hauptmann an Arthur Schnitzler, 24. 1. 1905]{ Gerhart Hauptmann an Arthur Schnitzler, 24. 1. 1905}\nopagebreak\mylabel{v}\rehead{ }\normalsize\beginnumbering\briefempfaengerindex{Schnitzler, Arthur@\textsc{Schnitzler, Arthur}!zzzHauptmann, Gerhart@\emph{von Gerhart Hauptmann}!1905-01-241@{24. 1. 1905}|(be} \toendnotes[C]{\smallbreak\pagebreak[2]} \Standort{CUL, Schnitzler, B 36.}
\physDesc{Brief, 1 Blatt, 3 Seiten
\newline{}Handschrift: schwarze Tinte, lateinische Kurrent}\toendnotes[C]{\smallbreak}\pstart{}{\pb}Lieber Herr Schnitzler.\pend\pstart
           Ich war in den \textcolor{pink}{Berlin}{}\ledrightnote{\textcolor{pink}{Berlin}}er Trubel gerathen, sonst
                    hätte ich Ihnen gleich geantwortet und gedankt, für das Gute und Herzliche, was
                    Sie mir erwiesen haben, durch Ihren Brief. So sehr wir geneigt sein mögen, eine
                    erfahrene Auszeichnung nicht als unverdient zu erachten, so sehr bin ich mir
                    doch auch der Verdienste bewusst, die Sie, verehrter Herr Schnitzler, und andere
                    gleichstrebende deutsche Dichter in \textcolor{pink}{Oesterreich}{}\ledrightnote{\textcolor{pink}{Österreich}}, haben: und es fällt mir nicht ein, sie geringer
                    anzuschlagen, als die Meinen.\pend
           \pstart
           Ich sage es, obgleich ich {\pb}annehme, Sie wissen das ungesagt.
                    Und ich wünschte auch nichts sehnlicher, als fortan eine schöne Reihe von
                    Gratulationen nach \textcolor{pink}{Wien}{}\ledrightnote{\textcolor{pink}{Wien}} richten zu können.
                    Wahrhaftig! Wenn ich an \label{K_L01494_1v}\edtext{\textcolor{brown}{Preise}{}\ledrightnote{\textcolor{brown}{Franz-Grillparzer-Preis}}}{\lemma{\textnormal{\emph{Preise}}}\Cendnote{\textnormal{Die Zuerkennung des \emph{\textcolor{brown}{Grillparzer-Preises}} für \emph{\textcolor{green}{Der
                            arme Heinrich}} wurde Mitte Januar 1905 bekannt
                        gegeben.}}}\label{K_L01494_1h} überhaupt gedacht hätte, so würde ich es schon früher
                    gewusst haben. Seien Sie vielmals gegrüsst! Alles Glück für Leben und Wirken und
                    auf gesundes Wiedersehen!\pend
           \pstart
           Herzlich{\\[\baselineskip]} Ihr{\\[\baselineskip]}\spacefill\mbox{Gerhart Hauptmann}\pend
           \leftskip=0em{}\pstart
           \noindent{}{\pb}\textcolor{pink}{Agnetendorf}{}\ledrightnote{\textcolor{pink}{Agnetendorf}}\pend
           \pstart
           d 24.{\\}Januar{\\}1905.\pend
           \endnumbering\briefempfaengerindex{Schnitzler, Arthur@\textsc{Schnitzler, Arthur}!zzzHauptmann, Gerhart@\emph{von Gerhart Hauptmann}!1905-01-241@{24. 1. 1905}|)be}\mylabel{h}  \normalsize

\doendnotes{C}
\bigskip
\vfill

\clearpage

\footnotesize

\lohead{\textsc{register}}

% Definiere theindex-Environment komplett neu ohne reledmac
\makeatletter
\renewenvironment{theindex}{%
  \section*{\indexname}%
  \setlength{\parindent}{0pt}%
  \setlength{\parskip}{0pt plus 0.3pt}%
  \let\item\@idxitem
}{%
  \clearpage
}
\makeatother

\IfFileExists{\jobname-pw.ind}{\input{\jobname-pw.ind}}{}

\end{document}

      