%% latex-korrekturansicht-vorspann.tex
%% Vorspann für die Korrekturansicht.
%% Lädt die gemeinsame Datei latex-vorspann.tex mit gesetztem Schalter.

\newif\ifkorrekturansicht
\korrekturansichttrue

\input{../tex-inputs/latex-vorspann}


               \section[Georg Engländer an Arthur Schnitzler, 27. 2. 1919]{ Georg Engländer an Arthur Schnitzler, 27. 2. 1919}\nopagebreak\mylabel{v}\rehead{ }\normalsize\beginnumbering\briefempfaengerindex{Schnitzler, Arthur@\textsc{Schnitzler, Arthur}!zzzEnglaender, Georg@\emph{von Georg Engländer}!1919-02-271@{27. 2. 1919}|(be} \toendnotes[C]{\smallbreak\pagebreak[2]} \Standort{DLA, A:Schnitzler, HS.NZ85.1.2889.}
\physDesc{Brief, 1 Blatt, 2 Seiten
\newline{}Handschrift: schwarze Tinte, lateinische Kurrent
\newline{}Schnitzler: mit rotem Buntstift zwei Unterstreichungen }\toendnotes[C]{\smallbreak}\pstart
           \noindent{}{\pb}\textcolor{gray}{\textbf{Georg
                                Engländer}}\hfill \textcolor{gray}{\textbf{\textcolor{pink}{Wien}{}\ledrightnote{\textcolor{pink}{Wien}},}} den 27/2 19\pend
           \pstart
           \textcolor{gray}{\textbf{\textcolor{pink}{IX. Nußdorferſtraße
                                Nr. 10}{}\ledrightnote{\textcolor{pink}{Nussdorfer Straße}}.}}\pend
           \pstart
           \textcolor{gray}{\textbf{Betrifft: Nachlaß \textbf{\textcolor{blue}{Peter Altenberg}{}\ledrightnote{\textcolor{blue}{Peter Altenberg}}}.}}\pend
           \pstart{}Geehrter Meister!\pend\pstart
           Erst heute kann ich meinen tiefinnigsten Dank für die so schönen {\kaufmannsund} ehrenvollen Worte abstatten, die Sie werther
                    Meister anlässlich Ihrer Condolenz meinem \textcolor{blue}{Bruder}{}\ledrightnote{→\textcolor{blue}{Peter Altenberg}} geſpendet; lt. innliegendem Kouvert dessen
                    letzter sichtbarer Stempel d. 22/II trägt, hat der Brief eine
                    beinahe 8wöchentliche Wanderung durchgemacht bevor er gestern an mich gelangte;
                    so kann ich den Scheine löschen, als hätte ich, so werthvolle Freunde {\kaufmannsund} Gönner \textcolor{blue}{Peter}{}\ledrightnote{\textcolor{blue}{Peter Altenberg}}\textsuperscript{s} nicht, sofort u. zu allererst
                    berücksichtigend, \substVorne{}\textsuperscript{mit}\substDazwischen{}in\substHinten{} ergebenster {\kaufmannsund} dankbarster Art, mit
                    Erdwiederung bedacht.\pend
           \pstart
           Ich wünschte Meister, Ihre prognostische Werthung, möge in Erfüllung gehen, ich
                    will selbst Alles, als Nachlasserbe, auch dazu thun {\kaufmannsund} denke noch in den folgenden Jahren noch 2 oder 3 {\pb}Bände mit Hinterlassenem, ausführlicher
                    Biographie, Briefen an Freunde {\kaufmannsund} Freundinnen in
                    seinem Sinne erscheinen zu lassen; auch will ich durch Vorträge den Kreis der
                    ihn Verstehenden erweitern.\pend
           \pstart
           Mittwoch, d. 5 März{ }\uline{½} 6\footnote{\noindent{}\uline{\textcolor{pink}{Kl.
                                Konzerthaus-Saal}{}\ledrightnote{\textcolor{pink}{Konzerthaus}}.}{\\}½ 6 Uhr\hspace*{1em}5/III 19.} findet der erste Abend statt, dem ich ein selbst gewähltes Programm
                    mehr lyrischen Charakters {\kaufmannsund} doch sehr
                    abwechslungsreich besti{\geminationm}t habe; ich habe mir
                    erlaubt Ihnen werther Meister 2 Sitze zugehen zu lassen, wäre besonders
                    geehrt wenn Sie davon Gebrauch machen, um Ihr mir besonders maassgebendes
                    Urtheil für diese Form der beabsichtigten litterarischen Popularisirung des \textcolor{blue}{Verewigten}{}\ledrightnote{→\textcolor{blue}{Peter Altenberg}}, erfahren zu
                    können.\pend
           \pstart
           In grösster Hochachtung{\\[\baselineskip]}Ihr ganz ergebenster{\\[\baselineskip]}\spacefill\mbox{G.
                        Engländer}\pend
           \leftskip=0em{}\endnumbering\briefempfaengerindex{Schnitzler, Arthur@\textsc{Schnitzler, Arthur}!zzzEnglaender, Georg@\emph{von Georg Engländer}!1919-02-271@{27. 2. 1919}|)be}\mylabel{h}  \normalsize

\doendnotes{C}
\bigskip
\vfill

\clearpage

\footnotesize

\lohead{\textsc{register}}

% Definiere theindex-Environment komplett neu ohne reledmac
\makeatletter
\renewenvironment{theindex}{%
  \section*{\indexname}%
  \setlength{\parindent}{0pt}%
  \setlength{\parskip}{0pt plus 0.3pt}%
  \let\item\@idxitem
}{%
  \clearpage
}
\makeatother

\IfFileExists{\jobname-pw.ind}{\input{\jobname-pw.ind}}{}

\end{document}

      