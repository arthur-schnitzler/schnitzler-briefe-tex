%% latex-korrekturansicht-vorspann.tex
%% Vorspann für die Korrekturansicht.
%% Lädt die gemeinsame Datei latex-vorspann.tex mit gesetztem Schalter.

\newif\ifkorrekturansicht
\korrekturansichttrue

\input{../tex-inputs/latex-vorspann}


               \section[Robert Adam an Arthur Schnitzler, 21. 3. 1927]{ Robert Adam an Arthur Schnitzler, 21. 3. 1927}\nopagebreak\mylabel{v}\rehead{ }\normalsize\beginnumbering\briefempfaengerindex{Schnitzler, Arthur@\textsc{Schnitzler, Arthur}!zzzAdam, Robert@\emph{von Robert Adam}!1927-03-211@{21. 3. 1927}|(be} \toendnotes[C]{\smallbreak\pagebreak[2]} \Standort{CUL, Schnitzler, B 1.}
\physDesc{Brief, 2 Blätter, 8 Seiten
\newline{}Handschrift: schwarze Tinte, deutsche Kurrent
\newline{}Schnitzler: 1) mit rotem Buntstift beschriftet: »\textsc{Adam}«, »\textcolor{green}{\textsc{(Diagr}{[}amm){]}}« und mehrere Unterstreichungen 2) mit anderem rotem Buntstift weitere Unterstreichungen\newline{}Ordnung: mit Bleistift von unbekannter Hand nummeriert:
                                    »17« }\Standort{Wien, Österreichische Nationalbibliothek, Cod.ser. 52.268, 328–329.}
\physDesc{handschriftliche Abschrift
\newline{}Handschrift: schwarze Tinte, Gabelsberger Kurzschrift}\Standort{Wien, Österreichische Nationalbibliothek, Cod.ser. 52.268, 328–329.}
\physDesc{maschinelle Abschrift
\newline{}Schreibmaschine}\toendnotes[C]{\smallbreak}\pstart
           \raggedleft{}{\pb}\textcolor{pink}{Wien}{}\ledrightnote{\textcolor{pink}{Wien}}, am 21. März 1927.\pend
           \pstart\center{}Hochverehrter Herr Doktor!\pend\pstart
           Die liebenswürdige Überſendung Ihres \textcolor{green}{Werkes}{}\ledrightnote{→\textcolor{green}{Der Geist im Wort und der Geist in der Tat}} hat mir die größte Freude bereitet, nicht nur die an Ihrem \textcolor{green}{Werke}{}\ledrightnote{→\textcolor{green}{Der Geist im Wort und der Geist in der Tat}}{ }ſelbſt, ſondern auch durch die Erkenntnis, daß Sie,
               den ich von allen lebenden deutſchen Dichtern am höchſten ſchätze, meine kleine und
               nun im Aktenſtaub ſchon ganz und gar vertrocknete Exiſtenz noch nicht ganz vergeſſen
               haben. Ich weiß alſo gar nicht, wie ich Ihnen danken soll.\pend
           \pstart
           Ich habe Ihr \textcolor{green}{Werk}{}\ledrightnote{→\textcolor{green}{Der Geist im Wort und der Geist in der Tat}}, ſobald ich
               nach Überwindung eines aufgetürmten Ak{\pb}tenbergs zu ihm gelangen konnte, mit Eifer und Luſt ſtudiert (nicht bloß geleſen)
               und möchte, wenn Sie es geſtatten, einige Bemerkungen, die ſich mir aufdrängten, kurz
               ſkizzieren.\pend
           \pstart
           Der von Ihnen unternommene Verſuch, die alten \textcolor{blue}{theophraſtiſch}{}\ledrightnote{\textcolor{blue}{Theophrast von Eresos}}-\textcolor{blue}{\textsc{La-Bruyère}}{}\ledrightnote{\textcolor{blue}{Jean de La Bruyère}}ſchen Bemühungen von einem höheren Geſichtspunkte aus wiederaufzunehmen und in
               das Wirrſal der uns umdrängenden (und ſchließlich auch in uns ſelbſt hauſenden)
               menſchlichen Charaktere durch Auszeichnung und vergleichende Gegenüberſtellung von
               Urtypen reinliche Ordnung zu bringen, den Beſtand gewiſſer Geiſtesverfaſſungen,
               geſondert von Begabung und Seelenzuſtänden hervorzuheben und dadurch der
               Charakteriſierung von Einzelindividuen die ſichere Grundlage des feſtſtehenden
               Vergleichstypus {\pb}zu ſchaffen, iſt
               ungeheuer intereſſant und, wie ich meine, wertvoll; er ſcheint mir geeignet, eine
               noch fehlende Disziplin der Charakterologie einzuleiten, und ich bin ſicher, daß
               nunmehr, da Sie den Weg gezeigt haben, das Volk der philoſophiſchen Kärrner, an dem
               kein Land ſo reich iſt wie \textcolor{pink}{Deutſchland}{}\ledrightnote{\textcolor{pink}{Deutschland}}, mit
               Schotterzufuhren und bequemer Ausharkung, mit Anlage von Abzugsgräben und ſeitlicher
               Raſenverbrämung nicht kargen wird. Es bedarf oft nur des Manifeſtes \introOben{}\label{T_L02483_1v}\edtext{(aber es bedarf ſeiner)}{\lemma{\textnormal{\emph{(aber es bedarf ſeiner)}}}\Cendnote{\textnormal{ursprünglich nach »eines großen
                        Geiſtes«, durch Verschiebezeichen im Satz umgereiht}}}\label{T_L02483_1h}\introOben{} eines großen Geiſtes, damit eine ganze große Welt entſtehe. Mir kommen hiebei
               die wenigen Seiten des kommuniſtiſchen in den Sinn und \strikeout{auf} die neue Art von Geſchichtswiſſenſchaft, die ſich über ihnen aufgebaut
               hat.\pend
           \pstart
           Wenn ich, der Skeptiker, einen {\pb}kritiſierenden Kärrnerbeitrag liefern darf, ſo würde er der »\label{K_L02483_1v}\edtext{\textcolor{green}{ideellen unüberſchreitbaren
                  Grenzlinie}{}\ledrightnote{→\textcolor{green}{Der Geist im Wort und der Geist in der Tat}}}{\lemma{\textnormal{\emph{ideellen … Grenzlinie}}}\Cendnote{\textnormal{vgl. S. 9 der Erstausgabe (= Abschnitt 2)}}}\label{K_L02483_1h}« gelten, die Ihre
               Diagramme zwiſchen den poſitiven und negativen Typen ziehen. Es iſt mir klar, daß die
               Urtypen nicht empiriſch konſtatierte Haupterſcheinungsformen menſchlicher
               Geiſtesverfaſſungen ſind, ſondern Abſtraktionen beſtimmter derartiger Geſtaltungen
                  (\label{K_L02483_2v}\edtext{\textcolor{green}{nicht eine Erfahrung, ſondern eine
                  Idee}{}\ledrightnote{→\textcolor{green}{Glückliches Ereignis}}}{\lemma{\textnormal{\emph{nicht … Idee}}}\Cendnote{\textnormal{Nach \textcolor{blue}{Goethe}s Schilderung habe \textcolor{blue}{Schiller}{ }seine Vorstellung einer Urpflanze mit der
                  Argument »Das ist keine Erfahrung, das ist eine Idee« abgelehnt
                     (\emph{\textcolor{green}{Glückliches Ereignis}}).}}}\label{K_L02483_2h}, um ein
               bekanntes Wort zu zitieren). Lägen empiriſch gefundene Haupttypen vor, dann wäre es
               ohne weiteres evident, daß eine ſtrikte Scheidewand zwiſchen ihnen nicht errichtet
               werden könnte: da die unendliche Mannigfaltigkeit der wirklich gegebenen Charaktere
               die Gewißheit gäbe, daß es zwiſchen allen ſolchen Typen, die nur als Grenztypen
               gelten {\pb}könnten, Übergangsformen in
               ununterbrochener Reihe geben müſſe. Aber auch bei Aufſtellung von Urtypen als Ideen
               (Gebilden des Sollens, nicht des Seins, wie \textcolor{blue}{Kelſsen}{}\ledrightnote{\textcolor{blue}{Hans Kelsen}}{ }ſagen möchte) handelt es ſich nicht um
               kontradiktoriſche, ſondern um konträre Gegenſätze, die die Möglichkeit einer
               unendlichen Reihe ſie verbindender Varietäten nicht ausſchlöſſen. Auch die Urtypen
               als Ideen ſind Grenztypen.\pend
           \pstart
           Sie bezeichnen zwar die Typen der oberen Vierecke als die poſitiven, die der unteren
               als die negativen, und poſitiv\strikeout{e}–negativ oder plus und
               minus (\textcolor{green}{S. 9}{}\ledrightnote{→\textcolor{green}{Der Geist im Wort und der Geist in der Tat}}) ſind allerdings
               kontradiktoriſche Gegenſätze: nicht aber werden es die Typen durch dieſe
               Bezeichnung.\pend
           \pstart
           Zu demſelben Ergebnis kommt {\pb}man, wenn man
               auf die Grundidee zurückgeht, die der Unterſcheidung der Seite »Gottes« und der Seite
               »des Teufels« zugrundeliegt (welche poetiſchen Termini, wie ich beſorge, in Menſchen
               das Mißverſtändnis erwecken können, es ſei auf eine Diſtinktion im Sinne chriſtlicher
               Moral abgezielt). Sie liegt wohl darin, daß den einen das Werk Zweck, den andern
               Mittel zum Zweck iſt, woran ſich der Gegenſatz zwiſchen Idealismus (im landläufigen
               Sinne) und realiſtiſcher Lebenseinſtellung und zwiſchen Altruismus und Egoismus
               anſchließt (obwohl man vielleicht ſagen könnte, es ſei ein Egoismus im höchſten
               Sinne, wenn der Poſitive nur für ſein Werk lebe, da es dem Schöpfer nur eine andere
               Form seines Ich ſei). {\pb}Alle dieſe
               Gegenſätze nun sind konträre, und daraus folgt, daß die auf ihrer Baſis einander
               gegenübergeſtellten Typen ebenfalls einander konträr gegenüberſtehen, das heißt
               Endglieder von Reihen ſind, deren Elemente \substVorne{}\textsuperscript{mit}\substDazwischen{}in\substHinten{} unendlich kleinen Unterſchieden ſich ſteigernd gedacht werden können. –\pend
           \pstart
           Ich muß es, um Ihre Geduld nicht zu erſchöpfen, \strikeout{\textcolor{gray}{an}} bei dieſen Anmerkungen bewenden laſſen: obwohl ich Luſt hätte, noch ſo Manches
               feſtzuhalten, was mir bei der Durchſtudierung Ihres Werks – eines der anregendſten,
               die ich kenne – an klugen und unklugen Gedanken gekommen iſt.\pend
           \pstart
           Nehmen Sie nochmals, hochverehrter Herr Doktor, meinen beſten {\pb}Dank!\pend
           \pstart
           Mit vielen Empfehlungen Ihr{\\[\baselineskip]}ergebener{\\[\baselineskip]}\spacefill\mbox{D\textsuperscript{r}RAdam}\pend
           \leftskip=0em{}\endnumbering\briefempfaengerindex{Schnitzler, Arthur@\textsc{Schnitzler, Arthur}!zzzAdam, Robert@\emph{von Robert Adam}!1927-03-211@{21. 3. 1927}|)be}\mylabel{h}  \normalsize

\doendnotes{C}
\bigskip
\vfill

\clearpage

\footnotesize

\lohead{\textsc{register}}

% Definiere theindex-Environment komplett neu ohne reledmac
\makeatletter
\renewenvironment{theindex}{%
  \section*{\indexname}%
  \setlength{\parindent}{0pt}%
  \setlength{\parskip}{0pt plus 0.3pt}%
  \let\item\@idxitem
}{%
  \clearpage
}
\makeatother

\IfFileExists{\jobname-pw.ind}{\input{\jobname-pw.ind}}{}

\end{document}

      