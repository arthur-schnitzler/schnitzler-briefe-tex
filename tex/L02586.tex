%% latex-korrekturansicht-vorspann.tex
%% Vorspann für die Korrekturansicht.
%% Lädt die gemeinsame Datei latex-vorspann.tex mit gesetztem Schalter.

\newif\ifkorrekturansicht
\korrekturansichttrue

\input{../tex-inputs/latex-vorspann}


               \section[Auguste Hauschner an Arthur Schnitzler, 17. 6. 1908]{ Auguste Hauschner an Arthur Schnitzler, 17. 6. 1908}\nopagebreak\mylabel{v}\rehead{ }\normalsize\beginnumbering\briefempfaengerindex{Schnitzler, Arthur@\textsc{Schnitzler, Arthur}!zzzHauschner, Auguste@\emph{von Auguste Hauschner}!1908-06-171@{17. 6. 1908}|(be} \toendnotes[C]{\smallbreak\pagebreak[2]} \Standort{DLA, A:Schnitzler, HS1985.1.3363.}
\physDesc{Brief, 1 Blatt, 3 Seiten
\newline{}Handschrift: schwarze Tinte, lateinische Kurrent
\newline{}Schnitzler: mit Bleistift Vermerk »\textsc{Hauschner}« }\toendnotes[C]{\smallbreak}\pstart
           \raggedleft{}{\pb}\textcolor{pink}{Berlin}{}\ledrightnote{\textcolor{pink}{Berlin}} d. 17. 6. 08\pend
           \pstart
           Sehr geehrter Herr Doctor – ich wünschte sehr, ich dürfte meine
               Bewunderung Ihres \textcolor{green}{Roman}{}\ledrightnote{→\textcolor{green}{Der Weg ins Freie. Roman}}s
               öffentlich aussprechen. Aber auf dem Weg zur Buchbesprechung ist für mich leider gar
               kein Plätzchen frei. So möchte ich Ihnen wenigstens, als ein Zeichen meiner Verehrung
               mein eigenes, so eben erschienenes, \textcolor{green}{Buch}{}\ledrightnote{→\textcolor{green}{Die Familie Lowositz. Roman}}{ }{\pb}senden.
               Leider hat es mit dem Ihren nichts gemein, als eine Stimmung. In einem \label{K_L02586-1v}\edtext{zweiten Band}{\lemma{\textnormal{\emph{zweiten Band}}}\Cendnote{\textnormal{Die Fortführung erschien 1910 mit dem Titel \emph{\textcolor{green}{Rudolf und Camilla}}.}}}\label{K_L02586-1h} soll diese noch
               vertiefter werden. –\pend
           \pstart
           Hätte ich mich an Ihrem \textcolor{green}{Werk}{}\ledrightnote{→\textcolor{green}{Der Weg ins Freie. Roman}}
               nicht so entzückt, so könnte ich Sie darum beneiden. Wie kann man so viel können!
               Einen solchen Reichthum in sich haben und solche Kraft ihn auszumünzen. Ich liebe \textcolor{blue}{Maupassant}{}\ledrightnote{\textcolor{blue}{Guy de Maupassant}}, aber ich suche nicht den billigen
               Vergleich mit Ihnen. Der Sie so persönlich sind, so ganz ein Eigener. {\pb}Ganz traurig wird man doch, dass es so eine restlose
               Fähigkeit des Ausdrucks giebt, so eine Seelenkunde, so ein Verstehen des
               Menschlichen. Und Unsereins wagt sich daneben auch Schriftsteller zu nennen.
               Verzeihen Sie mir Beides. Diesen Herzensschrei und das Senden meines \textcolor{green}{Buchs}{}\ledrightnote{→\textcolor{green}{Die Familie Lowositz. Roman}}.\pend
           \pstart
           In aufrichtiger Ergebenheit{\\[\baselineskip]}\spacefill\mbox{Frau Auguste Hauschner}\pend
           \leftskip=0em{}\endnumbering\briefempfaengerindex{Schnitzler, Arthur@\textsc{Schnitzler, Arthur}!zzzHauschner, Auguste@\emph{von Auguste Hauschner}!1908-06-171@{17. 6. 1908}|)be}\mylabel{h}  \normalsize

\doendnotes{C}
\bigskip
\vfill

\clearpage

\footnotesize

\lohead{\textsc{register}}

% Definiere theindex-Environment komplett neu ohne reledmac
\makeatletter
\renewenvironment{theindex}{%
  \section*{\indexname}%
  \setlength{\parindent}{0pt}%
  \setlength{\parskip}{0pt plus 0.3pt}%
  \let\item\@idxitem
}{%
  \clearpage
}
\makeatother

\IfFileExists{\jobname-pw.ind}{\input{\jobname-pw.ind}}{}

\end{document}

      