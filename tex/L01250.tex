%% latex-korrekturansicht-vorspann.tex
%% Vorspann für die Korrekturansicht.
%% Lädt die gemeinsame Datei latex-vorspann.tex mit gesetztem Schalter.

\newif\ifkorrekturansicht
\korrekturansichttrue

\input{../tex-inputs/latex-vorspann}


               \section[Adalbert Seligmann an Arthur Schnitzler, 21. 11. 1902]{ Adalbert Seligmann an Arthur Schnitzler, 21. 11. 1902}\nopagebreak\mylabel{v}\rehead{ }\normalsize\beginnumbering\briefempfaengerindex{Schnitzler, Arthur@\textsc{Schnitzler, Arthur}!zzzSeligmann, Adalbert Franz@\emph{von Adalbert Franz Seligmann}!1902-11-211@{21. 11. 1902}|(be} \toendnotes[C]{\smallbreak\pagebreak[2]} \Standort{CUL, Schnitzler, B 97.}
\physDesc{Brief, 1 Blatt, 4 Seiten
\newline{}Handschrift: schwarze Tinte, deutsche Kurrent
\newline{}Schnitzler: 1) mit Bleistift beschriftet: »\textsc{Seligmann}« und nummeriert: »4«  2) mit rotem Buntstift eine Unterstreichung}\toendnotes[C]{\smallbreak}\pstart
           \noindent{}{\pb}Verehrter Freund! Vor allem Verzeihung, daſs ich Ihnen bis jetzt
               nicht für die Ueberſendung Ihrer beiden \label{K_L01250_1v}\edtext{\textcolor{green}{Werke}{}\ledrightnote{→\textcolor{green}{Der Schleier der Beatrice. Schauspiel in fünf Akten}{\newline}→\textcolor{green}{Lebendige Stunden. Vier Einakter}}}{\lemma{\textnormal{\emph{Werke}}}\Cendnote{\textnormal{Obzwar im Folgenden nicht genannt,
                  dürfte es sich um \textcolor{blue}{Schnitzler}s einzige
                  Neuerscheinung in Buchform des Jahres 1902 handeln, die vier Einakter
                     \emph{\textcolor{green}{Lebendige Stunden}}.}}}\label{K_L01250_1h} gedankt habe. Aber
               ich wollte nicht früher ſchreiben, als bis ich den »\textcolor{green}{Schleier der Beatrice}{}\ledrightnote{\textcolor{green}{Der Schleier der Beatrice. Schauspiel in fünf Akten}}{[}«{]}, über den ich mancherlei gehört, auch geleſen hätte; und ich
               bin in diesen Tagen durch mannigfache Arbeit und ſonſtige Scherereien nicht gleich
               dazu gekommen. – Ich weiß, daſs nichts lächerlicher iſt, als wenn man einem Künſtler
               über sein {\pb}Werk\strikeout{e} Dinge ſagt, die er ſelber viel beſſer weiß. Darum nur ſo viel: Ich halte
               dieſe Arbeit für Ihre dichteriſch bedeutendſte. Die Idee, eine Handlung unter dem
               Hochdruck, den das Vorgefühl \substVorne{}\textsuperscript{eines}\substDazwischen{}des\substHinten{} unentrinnbaren Untergangs erzeugt, ſpielen zu laſſen, und dadurch alle
               Hemmungen fortzuſchaffen, die ſich den immerhin etwas wunderlichen Begebenheiten
               ſonſt hindernd in den Weg ſtellen möchten, finde ich genial! Die Geſtalt der \textcolor{green}{Beatrice}{}\ledrightnote{→\textcolor{green}{Der Schleier der Beatrice. Schauspiel in fünf Akten}}{ }{\pb}unglaublich rührend und – wahr! Dabei
               alles trotz der ſchwülen Atmosphäre keinen Augenblick verletzend oder unfein!
               Allerdings geſteh’ ich, begreife ich ganz gut daſs ein Theaterdirector das Werk ſich
               nicht aufzuführen getraut. Unſer Publicum, das täglich gemeiner wird – beachten Sie,
               bei welchen Stellen in einem \textcolor{blue}{Shakeſpeare}{}\ledrightnote{\textcolor{blue}{William Shakespeare}}ſtück
               gelacht wird – würde die Subtilität der pſychologiſchen Vorgänge gewiß nicht
               verſtehen – da es ſich um das Werk eines Zeitgenoſſen handelt. Wenn Sie {\pb}\textcolor{blue}{Kleiſt}{}\ledrightnote{\textcolor{blue}{Heinrich von Kleist}} oder ſo jemand wären – \textsc{\label{K_L01250_2v}\edtext{à la bonheur}{\lemma{\textnormal{\emph{à la bonheur}}}\Cendnote{\textnormal{französisch: auf gut Glück}}}\label{K_L01250_2h}}! Aber für einen Kreis verſtändiger und dichteriſch empfindender Menſchen wird
               Ihr Werk ein wahrer Genuß ſein und bleiben. Ich danke Ihnen noch \uline{ſehr} für Ihre Liebenswürdigkeit und\pend
           \pstart
           bin Ihr{\\[\baselineskip]}ſtets ergebener{\\[\baselineskip]}\spacefill\mbox{Seligmann}\pend
           \leftskip=0em{}\pstart
           \textcolor{pink}{Wien}{}\ledrightnote{\textcolor{pink}{Wien}}{ }21 Nov. 1902.\pend
           \endnumbering\briefempfaengerindex{Schnitzler, Arthur@\textsc{Schnitzler, Arthur}!zzzSeligmann, Adalbert Franz@\emph{von Adalbert Franz Seligmann}!1902-11-211@{21. 11. 1902}|)be}\mylabel{h}  \normalsize

\doendnotes{C}
\bigskip
\vfill

\clearpage

\footnotesize

\lohead{\textsc{register}}

% Definiere theindex-Environment komplett neu ohne reledmac
\makeatletter
\renewenvironment{theindex}{%
  \section*{\indexname}%
  \setlength{\parindent}{0pt}%
  \setlength{\parskip}{0pt plus 0.3pt}%
  \let\item\@idxitem
}{%
  \clearpage
}
\makeatother

\IfFileExists{\jobname-pw.ind}{\input{\jobname-pw.ind}}{}

\end{document}

      