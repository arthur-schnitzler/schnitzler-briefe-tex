%% latex-korrekturansicht-vorspann.tex
%% Vorspann für die Korrekturansicht.
%% Lädt die gemeinsame Datei latex-vorspann.tex mit gesetztem Schalter.

\newif\ifkorrekturansicht
\korrekturansichttrue

\input{../tex-inputs/latex-vorspann}


               \section[Arthur Schnitzler an Hugo von Hofmannsthal, 4. 10. 1898]{ Arthur Schnitzler an Hugo von Hofmannsthal, 4. 10. 1898}\nopagebreak\mylabel{v}\rehead{ }\normalsize\beginnumbering\briefempfaengerindex{Hofmannsthal, Hugo von@\textsc{Hofmannsthal, Hugo von}!zzzSchnitzler, Arthur@\emph{von Arthur Schnitzler}!1898-10-041@{4. 10. 1898}|(be} \toendnotes[C]{\smallbreak\pagebreak[2]} \Standort{FDH, Hs-30885,77.}
\physDesc{Brief, 1 Blatt, 4 Seiten
\newline{}Handschrift: Bleistift, deutsche Kurrent}\buchAbdrucke{\weitereDrucke{Hugo von Hofmannsthal, Arthur Schnitzler: \emph{Briefwechsel}. Hg. Therese Nickl und Heinrich Schnitzler. Frankfurt am Main: \emph{S. Fischer} 1964, S. 112–113.} }\toendnotes[C]{\smallbreak}\pstart
           {\pb}Dinſtag 4. X. 98.\pend
           \pstart
           Mein lieber Hugo, heut vor der Probe hat mir \textcolor{blue}{Brahm}{}\ledrightnote{\textcolor{blue}{Otto Brahm}} Ihren Brief gegeben; er hat mir große Freude
                    gemacht. Von dem \textcolor{green}{Vermächtnis}{}\ledrightnote{\textcolor{green}{Das Vermächtnis. Schauspiel in drei Akten}} hab ich nicht
                    viel Spaſs; die Sache iſt die: Das Stück iſt nur solang gut, als die »Heldin«
                    nicht auf der Bühne iſt. Erster Akt – und der dritte wieder, ſobald ſich das
                        Frauenzi{\geminationm}er ins {\pb}Waſſer ſtürzt. Da ſind alle übrigen Figuren wie von einem Bann befreit,
                    nachdem dieſes Geſpenſt angebracht iſt, und reden vernünftige, lebendige,
                    menſchliche, nahezu ſchöne Sachen. – Dabei iſt mir heute paſſirt, während d\textcolor{gray}{er}
                    Probe, dſs mir das \textcolor{green}{Stück}{}\ledrightnote{→\textcolor{green}{Das Vermächtnis. Schauspiel in drei Akten}}
                    ganz neu, in 5 {\pb}Akten, dramatiſch eingefallen iſt.
                    Wär ich anſtändg, ſo zög ichs zurück, wie es jetzt iſt.\pend
           \pstart
           Ich freu mich auf Ihre \textcolor{green}{venez.
                        Comödie}{}\ledrightnote{→\textcolor{green}{Der Abenteurer und die Sängerin oder Die Geschenke des Lebens}}; ſo wäre ja der Theaterabend fertig. In \textcolor{pink}{Wien}{}\ledrightnote{\textcolor{pink}{Wien}} find ich Sie ſchon; ich ko{\geminationm}e wohl Mitte nächſter Woche.\pend
           \pstart
           – Mein Ohr ſtört mich wieder mehr als je. Solch ſchleichende, {\pb}i{\geminationm}er gegenwärtige u unaufhaltſame Dinge in uns ſind
                    doch die perfideſte Art, wie Alter und Vernichtung ſich ankündigen.\pend
           \pstart
           Leben Sie wohl. Das mit dem \textcolor{pink}{Thurm}{}\ledrightnote{\textcolor{pink}{Le due Torri: Garisenda e degli Asinelli}} war ja nur
                    ein Spaſs. Ich hab ja gar kein Recht, Ihnen einen \textcolor{pink}{Thurm}{}\ledrightnote{→\textcolor{pink}{Bologna}} zu ſchenken, der in \textcolor{pink}{Bologna}{}\ledrightnote{\textcolor{pink}{Bologna}}{ }ſteht. Und was für Scherereien hätten Sie an
                    der Grenze!\pend
           \pstart Von Herzen Ihr \spacefill\mbox{Arthur}\pend{}\endnumbering\briefempfaengerindex{Hofmannsthal, Hugo von@\textsc{Hofmannsthal, Hugo von}!zzzSchnitzler, Arthur@\emph{von Arthur Schnitzler}!1898-10-041@{4. 10. 1898}|)be}\mylabel{h}  \normalsize

\doendnotes{C}
\bigskip
\vfill

\clearpage

\footnotesize

\lohead{\textsc{register}}

% Definiere theindex-Environment komplett neu ohne reledmac
\makeatletter
\renewenvironment{theindex}{%
  \section*{\indexname}%
  \setlength{\parindent}{0pt}%
  \setlength{\parskip}{0pt plus 0.3pt}%
  \let\item\@idxitem
}{%
  \clearpage
}
\makeatother

\IfFileExists{\jobname-pw.ind}{\input{\jobname-pw.ind}}{}

\end{document}

      