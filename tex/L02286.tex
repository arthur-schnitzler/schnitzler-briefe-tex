%% latex-korrekturansicht-vorspann.tex
%% Vorspann für die Korrekturansicht.
%% Lädt die gemeinsame Datei latex-vorspann.tex mit gesetztem Schalter.

\newif\ifkorrekturansicht
\korrekturansichttrue

\input{../tex-inputs/latex-vorspann}


               \section[Arthur Schnitzler an Robert Adam, 11. 6. 1918]{ Arthur Schnitzler an Robert Adam, 11. 6. 1918}\nopagebreak\mylabel{v}\rehead{ }\normalsize\beginnumbering\briefempfaengerindex{Adam, Robert@\textsc{Adam, Robert}!zzzSchnitzler, Arthur@\emph{von Arthur Schnitzler}!1918-06-111@{11. 6. 1918}|(be} \toendnotes[C]{\smallbreak\pagebreak[2]} \Standort{DLA, 96.34.2/8.}
\physDesc{Briefkarte, Umschlag
\newline{}Handschrift: schwarze Tinte, deutsche Kurrent\newline{}Versand: Stempel: »\nobreak{}Wien, 12. VI. 18\nobreak{}«.  }\pstart{}{\pb}\textsc{Dr Arthur Schnitzler}. \textcolor{pink}{Wien XVIII. \textsc{Sternwartestr} 71}{}\ledrightnote{\textcolor{pink}{VIII., Josefstadt}}.\pend{}{\bigskip}\pstart{}{\pb}Hrn \textsc{Dr. Robert Adam
                            Pollak}\pend{}\pstart{}\textcolor{pink}{Wien XII}{}\ledrightnote{\textcolor{pink}{XII., Meidling}}.\pend{}\pstart{}\textcolor{pink}{\textsc{Meidlinger Hauptstr} 56}{}\ledrightnote{\textcolor{pink}{Meidlinger Hauptstraße}}.\pend{}{\bigskip}\pstart
           \noindent{}{\pb}\textcolor{gray}{\textbf{Dr. Arthur Schnitzler}}\hfill 11. 6. 1918.\pend
           \pstart
           \textcolor{gray}{\textbf{\textcolor{pink}{Wien XVIII. Sternwartestrasse 71}{}\ledrightnote{\textcolor{pink}{Sternwartestraße}}}}\pend
           \pstart
           Verehrter Herr Doktor, den ganzen Winter u Frühling hab ich
                    nichts von Ihnen verno{\geminationm}en. Wie gehts Ihnen denn?
                        Ka{\geminationn} man Sie wieder einmal ſehen? Ich würde mich
                    freuen, von Ihnen richterliches und dichteriſches Leben \introOben{}das
                        neueſte –\introOben{} zu erfahren. Seien Sie {\pb}mir bis dahin herzlichſt
                    gegrüßt\pend
           \pstart
           Ihr ſehr ergebner{\\[\baselineskip]}\spacefill\mbox{Arthur Schnitzler}\pend
           \leftskip=0em{}\endnumbering\briefempfaengerindex{Adam, Robert@\textsc{Adam, Robert}!zzzSchnitzler, Arthur@\emph{von Arthur Schnitzler}!1918-06-111@{11. 6. 1918}|)be}\mylabel{h}  \normalsize

\doendnotes{C}
\bigskip
\vfill

\clearpage

\footnotesize

\lohead{\textsc{register}}

% Definiere theindex-Environment komplett neu ohne reledmac
\makeatletter
\renewenvironment{theindex}{%
  \section*{\indexname}%
  \setlength{\parindent}{0pt}%
  \setlength{\parskip}{0pt plus 0.3pt}%
  \let\item\@idxitem
}{%
  \clearpage
}
\makeatother

\IfFileExists{\jobname-pw.ind}{\input{\jobname-pw.ind}}{}

\end{document}

      