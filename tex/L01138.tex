%% latex-korrekturansicht-vorspann.tex
%% Vorspann für die Korrekturansicht.
%% Lädt die gemeinsame Datei latex-vorspann.tex mit gesetztem Schalter.

\newif\ifkorrekturansicht
\korrekturansichttrue

\input{../tex-inputs/latex-vorspann}


               \section[Arthur Schnitzler an Hermann Bahr, 1. 7. 1901]{ Arthur Schnitzler an Hermann Bahr, 1. 7. 1901}\nopagebreak\mylabel{v}\rehead{ }\normalsize\beginnumbering\briefempfaengerindex{Bahr, Hermann@\textsc{Bahr, Hermann}!zzzSchnitzler, Arthur@\emph{von Arthur Schnitzler}!1901-07-011@{1. 7. 1901}|(be} \toendnotes[C]{\smallbreak\pagebreak[2]} \Standort{TMW, HS AM 23390 Ba.}
\physDesc{Brief, 1 Blatt, 3 Seiten
\newline{}Handschrift: Bleistift, deutsche Kurrent\newline{}Ordnung: Lochung }\buchAbdrucke{\weitereDrucke{1) \emph{1. 7. 1909.} In: Arthur Schnitzler: \emph{The Letters of Arthur Schnitzler to Hermann Bahr}. Edited, annotated, and with an introduction, by Donald G.
                        Daviau. Chapel Hill: \emph{The University of North Carolina Press} 1978, S. 103 (University of North Carolina studies in the Germanic languages
                        and literatures, 89).} \weitereDrucke{2) Hermann Bahr, Arthur Schnitzler: \emph{Briefwechsel, Aufzeichnungen, Dokumente (1891–1931)}. Hg. Kurt Ifkovits und Martin Anton Müller. Göttingen: \emph{Wallstein} 2018, S. 212.} }\toendnotes[C]{\smallbreak}\pstart{}{\pb}lieber
                  Hermann\pend\pstart
           es drängt mich, dir zu deinem Collegen \textcolor{blue}{Poetzl}{}\ledrightnote{\textcolor{blue}{Eduard Pötzl}}
               wärmſtens zu gratuliren. Das ſind einmal mannhafte, echt \label{K_L01138_1v}\edtext{\textcolor{green}{t\damage{e}utſche Worte}{}\ledrightnote{→\textcolor{green}{Lüsternheit (Predigt in der Wüste)}}}{\lemma{\textnormal{\emph{teutſche Worte}}}\Cendnote{\textnormal{\textcolor{blue}{Ed. Pötzl}: \emph{\textcolor{green}{Lüsternheit. (Predigt in der Wüste).}} In: \emph{\textcolor{green}{Neues Wiener Tagblatt}}, Jg. 35, Nr. 176, 29. 6. 1901,
                     S. 1–2, ist eine schon im Titel erkennbare Replik auf \textcolor{blue}{Bahrs}{ }\emph{\textcolor{green}{Erotisch}}.}}}\label{K_L01138_1h}! Das Herz geht einem auf, wenn
               man ſie lieſt. »\textcolor{green}{Es iſt beſſer, das
                  gute zu heucheln als es durch offenkundige Frevel {\pb}aller Art von der
                  Tagesordnung gänzlich abſetzen}{}\ledrightnote{→\textcolor{green}{Lüsternheit (Predigt in der Wüste)}}.« – »\textcolor{green}{Es iſt immer noch moraliſcher im Geheimen zu ſündigen als auf
                  oeffentlichem Markte mit dem Laſter Arm in Arm zu gehen –}{}\ledrightnote{→\textcolor{green}{Lüsternheit (Predigt in der Wüste)}}« »\textcolor{green}{Die Geſa{\geminationm}theit darf
                  die Tugend nicht verachten, ſondern muſs ſie heilig halten und auf ihren Schild
                  erheben}{}\ledrightnote{→\textcolor{green}{Lüsternheit (Predigt in der Wüste)}}« –\pend
           \pstart
           {\pb}– So ehrlich iſt die
               Heuchelei ſelten geweſen!\pend
           \pstart
           Leb wohl und ſei herzlich gegrüßt.{\\[\baselineskip]}Dein{\\[\baselineskip]}\spacefill\mbox{Arth Sch}\pend
           \leftskip=0em{}\pstart
           \textcolor{pink}{St Anton}{}\ledrightnote{\textcolor{pink}{St. Anton am Arlberg}}{ }1. 7. 109.\pend
           \endnumbering\briefempfaengerindex{Bahr, Hermann@\textsc{Bahr, Hermann}!zzzSchnitzler, Arthur@\emph{von Arthur Schnitzler}!1901-07-011@{1. 7. 1901}|)be}\mylabel{h}  \normalsize

\doendnotes{C}
\bigskip
\vfill

\clearpage

\footnotesize

\lohead{\textsc{register}}

% Definiere theindex-Environment komplett neu ohne reledmac
\makeatletter
\renewenvironment{theindex}{%
  \section*{\indexname}%
  \setlength{\parindent}{0pt}%
  \setlength{\parskip}{0pt plus 0.3pt}%
  \let\item\@idxitem
}{%
  \clearpage
}
\makeatother

\IfFileExists{\jobname-pw.ind}{\input{\jobname-pw.ind}}{}

\end{document}

      