%% latex-korrekturansicht-vorspann.tex
%% Vorspann für die Korrekturansicht.
%% Lädt die gemeinsame Datei latex-vorspann.tex mit gesetztem Schalter.

\newif\ifkorrekturansicht
\korrekturansichttrue

\input{../tex-inputs/latex-vorspann}


               \section[Lou Andreas-Salomé an Arthur Schnitzler, {[}25. 11. 1895{]}]{ Lou Andreas-Salomé an Arthur Schnitzler, {[}25. 11. 1895{]}}\nopagebreak\mylabel{v}\rehead{ }\normalsize\beginnumbering\briefempfaengerindex{Schnitzler, Arthur@\textsc{Schnitzler, Arthur}!zzzAndreas-Salome, Lou@\emph{von Lou Andreas-Salomé}!1895-11-251@{{[}25. 11. 1895{]}}|(be} \toendnotes[C]{\smallbreak\pagebreak[2]} \Standort{CUL, Schnitzler, B 3.}
\physDesc{Brief, 1 Blatt, 2 Seiten
\newline{}Handschrift: schwarze Tinte, deutsche Kurrent
\newline{}Schnitzler: 1) mit Bleistift datiert »25/11 95« 2) mit rotem Buntstift eine Unterstreichung\newline{}Ordnung: mit rotem Buntstift von unbekannter Hand nummeriert
                                    »10« }\toendnotes[C]{\smallbreak}\pstart
           {\pb}Montag Abend.\pend
           \pstart{}Lieber Herr \textsc{D\textsuperscript{r}},\pend\pstart
           danke für die »\textcolor{green}{Liebelei}{}\ledrightnote{\textcolor{green}{Liebelei. Schauspiel in drei Akten}}«, die ich heute Nachmittag
               erhalten und ſeitdem geleſen und wieder geleſen habe. Hätte ich ſie ſchon vorher
               gekannt, – den erſten Eindruck von Ihnen ſelbſt anſtatt von den \textcolor{pink}{Burgſchauſpielern}{}\ledrightnote{\textcolor{pink}{Burgtheater}} empfangen, ſo würde die (an ſich vielleicht
               nicht ſo großen) Schwächen des Spiels, beſonders des Spiels der \textcolor{green}{Chriſtine}{}\ledrightnote{→\textcolor{green}{Liebelei. Schauspiel in drei Akten}}, mir nicht ſo viel vom Beſten
               verwiſcht haben. Ich kam ganz gedrückt aus dem Theater, ich konnte unter dem Spiel
               Ihre Eigenart nicht überall herauserkennen. Es geht ja mit dem »\textcolor{green}{\textsc{Hannele}}{}\ledrightnote{\textcolor{green}{Hanneles Himmelfahrt}}« {\pb}auch ſo: erſt dadurch, daß man das
               Werk ſelbſt kennt, ergänzt und unterſtützt man den Theatereindruck, der ſonſt
               unzulänglich bleibt, und wahrſcheinlich wird es allen intimen und lebensfeinen, \uline{lebenseinfachen} Kunſtwerken ſo ergehen, auch bei guter
               Darſtellung. Das Theater iſt eben nothwendig ein grobes Ding, was ein Dichter aber
               mit ſeiner groben Hülfe in uns hervorrufen will, iſt etwas ſo zartes.\pend
           \pstart
           Die »\textcolor{green}{Liebelei}{}\ledrightnote{\textcolor{green}{Liebelei. Schauspiel in drei Akten}}« iſt wunderſchön. Von Ihnen \textcolor{blue}{Dreien}{}\ledrightnote{→\textcolor{blue}{Richard Beer-Hofmann}{\newline}→\textcolor{blue}{Hugo von Hofmannsthal}}, – von Ihnen drei
               glücklichen \textcolor{blue}{Freunden}{}\ledrightnote{→\textcolor{blue}{Richard Beer-Hofmann}{\newline}→\textcolor{blue}{Hugo von Hofmannsthal}}, –
               ſind doch Sie der \label{K_L00516_1v}\edtext{Glücklichſte}{\lemma{\textnormal{\emph{Glücklichſte}}}\Cendnote{\textnormal{vgl. A. S.: \emph{Tagebuch}, 19. 5. 1895}}}\label{K_L00516_1h}.\pend
           \pstart
           Mit herzlichem Gruß Ihre{\\[\baselineskip]}\spacefill\mbox{LouAS.}\pend
           \leftskip=0em{}\endnumbering\briefempfaengerindex{Schnitzler, Arthur@\textsc{Schnitzler, Arthur}!zzzAndreas-Salome, Lou@\emph{von Lou Andreas-Salomé}!1895-11-251@{{[}25. 11. 1895{]}}|)be}\mylabel{h}  \normalsize

\doendnotes{C}
\bigskip
\vfill

\clearpage

\footnotesize

\lohead{\textsc{register}}

% Definiere theindex-Environment komplett neu ohne reledmac
\makeatletter
\renewenvironment{theindex}{%
  \section*{\indexname}%
  \setlength{\parindent}{0pt}%
  \setlength{\parskip}{0pt plus 0.3pt}%
  \let\item\@idxitem
}{%
  \clearpage
}
\makeatother

\IfFileExists{\jobname-pw.ind}{\input{\jobname-pw.ind}}{}

\end{document}

      