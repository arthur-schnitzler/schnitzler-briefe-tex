%% latex-korrekturansicht-vorspann.tex
%% Vorspann für die Korrekturansicht.
%% Lädt die gemeinsame Datei latex-vorspann.tex mit gesetztem Schalter.

\newif\ifkorrekturansicht
\korrekturansichttrue

\input{../tex-inputs/latex-vorspann}


               \section[Hugo von Hofmannsthal an Arthur Schnitzler, 1. 8. 1902]{ Hugo von Hofmannsthal an Arthur Schnitzler, 1. 8. 1902}\nopagebreak\mylabel{v}\rehead{ }\normalsize\beginnumbering\briefempfaengerindex{Schnitzler, Arthur@\textsc{Schnitzler, Arthur}!zzzHofmannsthal, Hugo von@\emph{von Hugo von Hofmannsthal}!1902-08-011@{1. 8. 1902}|(be} \toendnotes[C]{\smallbreak\pagebreak[2]} \Standort{CUL, Schnitzler, B 43.}
\physDesc{Postkarte
\newline{}Handschrift: Bleistift, deutsche Kurrent\newline{}Versand: 1) Stempel: »\nobreak{}\oindex{Rodaun@\textbf{Rodaun}, \emph{Teil eines besiedelten Ortes (A.BSOX)}|pwk}Rodaun, 1 8 02\nobreak{}«.  2) Stempel: »\nobreak{}\oindex{Hinterbruehl@\textbf{Hinterbrühl}, \emph{Besiedelter Ort (A.BSO)}|pwk}Hinterbrühl, 2. 8. 02, 2–5 N, Bestellt\nobreak{}«. 
\newline{}Schnitzler: mit Bleistift datiert: »1/8 902« \newline{}Ordnung: 1) mit Bleistift von unbekannter Hand nummeriert: »\strikeout{200}« 2) mit Bleistift von unbekannter Hand nummeriert: »183«}\buchAbdrucke{\weitereDrucke{Hugo von Hofmannsthal, Arthur Schnitzler: \emph{Briefwechsel}. Hg. Therese Nickl und Heinrich Schnitzler. Frankfurt am Main: \emph{S. Fischer} 1964, S. 160.} }\pstart{}{\pb}\textsc{Herrn D\textsuperscript{r} Arthur Šnitzler}\pend{}\pstart{}\textsc{\textcolor{pink}{Hinterbrühl bei Mödling}{}\ledrightnote{\textcolor{pink}{Hinterbrühl}}}\pend{}\pstart{}\textsc{\textcolor{pink}{Hauptstrasse 56}{}\ledrightnote{\textcolor{pink}{Hauptstraße}}.}\pend{}{\bigskip}\pstart
           \noindent{}{\pb}lieber, ich muſs
               morgen früh zum Zahnarzt.\pend
           \pstart
           Also auf Wiederſehen Dienstag früh, vielleicht ſchau ich aber ſchon
               früher bei Ihnen hinein.\pend
           \pstart
           Von Herzen{\\[\baselineskip]}\spacefill\mbox{Hugo.}\pend
           \leftskip=0em{}\pstart
           Freitag.\pend
           \endnumbering\briefempfaengerindex{Schnitzler, Arthur@\textsc{Schnitzler, Arthur}!zzzHofmannsthal, Hugo von@\emph{von Hugo von Hofmannsthal}!1902-08-011@{1. 8. 1902}|)be}\mylabel{h}  \normalsize

\doendnotes{C}
\bigskip
\vfill

\clearpage

\footnotesize

\lohead{\textsc{register}}

% Definiere theindex-Environment komplett neu ohne reledmac
\makeatletter
\renewenvironment{theindex}{%
  \section*{\indexname}%
  \setlength{\parindent}{0pt}%
  \setlength{\parskip}{0pt plus 0.3pt}%
  \let\item\@idxitem
}{%
  \clearpage
}
\makeatother

\IfFileExists{\jobname-pw.ind}{\input{\jobname-pw.ind}}{}

\end{document}

      