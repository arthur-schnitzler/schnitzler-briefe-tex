%% latex-korrekturansicht-vorspann.tex
%% Vorspann für die Korrekturansicht.
%% Lädt die gemeinsame Datei latex-vorspann.tex mit gesetztem Schalter.

\newif\ifkorrekturansicht
\korrekturansichttrue

\input{../tex-inputs/latex-vorspann}


               \section[Arthur Schnitzler an Wilhelm Bölsche, {[}12.? 11. 1893{]}]{ Arthur Schnitzler an Wilhelm Bölsche, {[}12.? 11. 1893{]}}\nopagebreak\mylabel{v}\rehead{ }\normalsize\beginnumbering\briefempfaengerindex{Boelsche, Wilhelm@\textsc{Bölsche, Wilhelm}!zzzSchnitzler, Arthur@\emph{von Arthur Schnitzler}!1893-11-141@{{[}12.? 11. 1893{]}}|(be} \toendnotes[C]{\smallbreak\pagebreak[2]} \Standort{Wrocław, Biblioteka Uniwersytecka, Böl.Pis 1771.}
\physDesc{Brief, 1 Blatt (Briefpapier mit Trauerrand), 2 Seiten
\newline{}Handschrift: schwarze Tinte, deutsche Kurrent}\buchAbdrucke{\weitereDrucke{1) Alois Woldan: \emph{Arthur Schnitzler – Briefe an Wilhelm Bölsche.} In: \emph{Germanica Wratislaviensia} (1987) Nr. 77, S. 465.} \weitereDrucke{2) Wilhelm Bölsche: \emph{Briefwechsel. Mit Autoren der Freien Bühne}. Hg. Gerd-Hermann Susen. Berlin: \emph{Weidler} 2010, S. 694 (Werke und Briefe. Wissenschaftliche Ausgabe, Briefe I).} }\toendnotes[C]{\smallbreak}\pstart
           \raggedleft{}{\pb}\textcolor{pink}{\textsc{IX. \label{K_L00282_1v}\edtext{Frankgasse}{\lemma{\textnormal{\emph{Frankgasse}}}\Cendnote{\textnormal{Die Übersiedlung
                           in sein neues Zuhause fand am 14. 11. 1893 statt. Die Antwort \textcolor{blue}{Bölsches}, der den Brief aus \textcolor{pink}{Friedrichshagen} nach \textcolor{pink}{Zürich} nachgesandt bekam, stammt vom
                              16. 11. 1893. Aufgrund der Verzögerung durch
                           die Post ist der 12. 11. 1893 als Absendetag
                           plausibel.}}}\label{K_L00282_1h}}}{}\ledrightnote{\textcolor{pink}{Frankgasse}}\pend
           \pstart{}Sehr geehrter Herr Doktor,\pend\pstart
           ich habe das \textcolor{green}{Das Märchen}{}\ledrightnote{\textcolor{green}{Das Märchen. Schauspiel in drei Aufzügen}} vor \label{K_L00282_2v}\edtext{etwa 3 Monaten}{\lemma{\textnormal{\emph{etwa 3 Monaten}}}\Cendnote{\textnormal{am 25. 7. 1893, Arthur Schnitzler an Samuel Fischer, 25. 7. 1893}}}\label{K_L00282_2h} Ihrer Aufforderung nach an den Verleger \textsc{Hrn \textcolor{blue}{Fischer}{}\ledrightnote{\textcolor{blue}{Samuel Fischer}}} geſandt. Seither habe ich 3mal verſucht, von dieſem Herrn eine Antwort zu
               erhalten – leider vergebens.\pend
           \pstart
           Ich muſs mich doch weiter an den \textcolor{brown}{Redakteur}{}\ledrightnote{→\textcolor{brown}{Neue Rundschau, Neue Deutsche Rundschau, Freie Bühne}} wenden, {\pb}und erſuche Sie, die
               Beantwortung meiner Fragen oder die Rückſendung meines Manuscripts umſo ſchleuniger
               veranlaſſen zu wollen, als die Aufführung des Stückes \label{K_L00282_3v}\edtext{in etwa 14 Tagen}{\lemma{\textnormal{\emph{in etwa 14 Tagen}}}\Cendnote{\textnormal{am
                     1. 12. 1893}}}\label{K_L00282_3h} im
                  \textcolor{pink}{Dtſch. Volkstheater}{}\ledrightnote{\textcolor{pink}{Volkstheater}}{ }ſtattfindet.\pend
           \pstart
           Mit ausgezeichneter Hochachtung{\\[\baselineskip]}\spacefill\mbox{Dr Arthur Schnitzler}\pend
           \leftskip=0em{}\endnumbering\briefempfaengerindex{Boelsche, Wilhelm@\textsc{Bölsche, Wilhelm}!zzzSchnitzler, Arthur@\emph{von Arthur Schnitzler}!1893-11-141@{{[}12.? 11. 1893{]}}|)be}\mylabel{h}  \normalsize

\doendnotes{C}
\bigskip
\vfill

\clearpage

\footnotesize

\lohead{\textsc{register}}

% Definiere theindex-Environment komplett neu ohne reledmac
\makeatletter
\renewenvironment{theindex}{%
  \section*{\indexname}%
  \setlength{\parindent}{0pt}%
  \setlength{\parskip}{0pt plus 0.3pt}%
  \let\item\@idxitem
}{%
  \clearpage
}
\makeatother

\IfFileExists{\jobname-pw.ind}{\input{\jobname-pw.ind}}{}

\end{document}

      