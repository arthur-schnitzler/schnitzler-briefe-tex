%% latex-korrekturansicht-vorspann.tex
%% Vorspann für die Korrekturansicht.
%% Lädt die gemeinsame Datei latex-vorspann.tex mit gesetztem Schalter.

\newif\ifkorrekturansicht
\korrekturansichttrue

\input{../tex-inputs/latex-vorspann}


               \section[Hugo von Hofmannsthal an Arthur Schnitzler, 15. 11. {[}1911{]}]{ Hugo von Hofmannsthal an Arthur Schnitzler, 15. 11. {[}1911{]}}\nopagebreak\mylabel{v}\rehead{ }\normalsize\beginnumbering\briefempfaengerindex{Schnitzler, Arthur@\textsc{Schnitzler, Arthur}!zzzHofmannsthal, Hugo von@\emph{von Hugo von Hofmannsthal}!1911-11-151@{15. 11. 1911}|(be} \toendnotes[C]{\smallbreak\pagebreak[2]} \Standort{CUL, Schnitzler, B 43.}
\physDesc{Briefkarte
\newline{}Handschrift: schwarze Tinte, deutsche Kurrent
\newline{}Schnitzler: mit Bleistift die Jahreszahl ergänzt: »911« und beschriftet: »Hugo« \newline{}Ordnung: 1) mit Bleistift von unbekannter Hand nummeriert:
                              »324« 2) mit Bleistift von unbekannter Hand nummeriert: »333«}\buchAbdrucke{\weitereDrucke{Hugo von Hofmannsthal, Arthur Schnitzler: \emph{Briefwechsel}. Hg. Therese Nickl und Heinrich Schnitzler. Frankfurt am Main: \emph{S. Fischer} 1964, S. 264.} }\pstart
           \raggedleft{}{\pb}15 XI.\pend
           \pstart{}mein lieber Arthur\pend\pstart
           wenn Sie da ſind, ſo geben Sie mir doch \uline{gleich} ein
               Zeichen\pend
           \pstart
           ich muſs nun baldigſt wieder nach \textcolor{pink}{Berlin}{}\ledrightnote{\textcolor{pink}{Berlin}},
               wegen \textcolor{green}{\textsc{Jedermann}}{}\ledrightnote{\textcolor{green}{Jedermann. Das Spiel vom Sterben des reichen Mannes}}, wie {\pb}viele Monate des Lebens ſollen
               denn noch vergehen ohne daſs man etwas davon hat, vorläufig noch gleichzeitig am
               Leben zu ſein.\hspace*{1.5em}Alles Liebe an \textcolor{blue}{Olga}{}\ledrightnote{\textcolor{blue}{Olga Schnitzler}}.\pend
           \pstart Ihr \spacefill\mbox{Hugo.}\pend{}\endnumbering\briefempfaengerindex{Schnitzler, Arthur@\textsc{Schnitzler, Arthur}!zzzHofmannsthal, Hugo von@\emph{von Hugo von Hofmannsthal}!1911-11-151@{15. 11. 1911}|)be}\mylabel{h}  \normalsize

\doendnotes{C}
\bigskip
\vfill

\clearpage

\footnotesize

\lohead{\textsc{register}}

% Definiere theindex-Environment komplett neu ohne reledmac
\makeatletter
\renewenvironment{theindex}{%
  \section*{\indexname}%
  \setlength{\parindent}{0pt}%
  \setlength{\parskip}{0pt plus 0.3pt}%
  \let\item\@idxitem
}{%
  \clearpage
}
\makeatother

\IfFileExists{\jobname-pw.ind}{\input{\jobname-pw.ind}}{}

\end{document}

      