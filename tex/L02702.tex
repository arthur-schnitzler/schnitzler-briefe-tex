%% latex-korrekturansicht-vorspann.tex
%% Vorspann für die Korrekturansicht.
%% Lädt die gemeinsame Datei latex-vorspann.tex mit gesetztem Schalter.

\newif\ifkorrekturansicht
\korrekturansichttrue

\input{../tex-inputs/latex-vorspann}


               \section[Paul Goldmann an Arthur Schnitzler, 9. 10. {[}1892{]}]{ Paul Goldmann an Arthur Schnitzler, 9. 10. {[}1892{]}}\nopagebreak\mylabel{v}\rehead{ }\normalsize\beginnumbering\briefempfaengerindex{Schnitzler, Arthur@\textsc{Schnitzler, Arthur}!zzzGoldmann, Paul@\emph{von Paul Goldmann}!1892-10-091@{9. 10. {[}1892{]}}|(be} \toendnotes[C]{\smallbreak\pagebreak[2]} \Standort{DLA, A:Schnitzler, HS.NZ85.1.3163.}
\physDesc{Brief, 1 Blatt, 3 Seiten
\newline{}Handschrift: schwarze Tinte, deutsche Kurrent
\newline{}Schnitzler: mit Bleistift das Jahr »92« vermerkt }\toendnotes[C]{\smallbreak}\pstart
           \noindent{}{\pb}\textcolor{gray}{\textbf{\textcolor{brown}{Frankfurter Zeitung}{}\ledrightnote{\textcolor{brown}{Frankfurter Zeitung}}.}}\pend
           \pstart
           \textcolor{gray}{\textbf{(\textcolor{brown}{Gazette de
                     Francfort}{}\ledrightnote{\textcolor{brown}{Frankfurter Zeitung}}.)}}\hfill \textcolor{pink}{Paris}{}\ledrightnote{\textcolor{pink}{Paris}}, 9. October.\pend
           \pstart
           \textcolor{gray}{\textbf{\begin{otherlanguage}{french}Directeur\end{otherlanguage}: \textbf{M. \textcolor{blue}{L. Sonnemann}{}\ledrightnote{\textcolor{blue}{Leopold Sonnemann}}}.}}\pend
           \pstart
           \textcolor{gray}{\textbf{\begin{otherlanguage}{french}Journal politique, financier,\end{otherlanguage}}}\pend
           \pstart
           \textcolor{gray}{\textbf{\begin{otherlanguage}{french}commercial et littéraire.\end{otherlanguage}}}\pend
           \pstart
           \textcolor{gray}{\textbf{\begin{otherlanguage}{french}\textbf{Paraissant trois fois par jour}\end{otherlanguage}}}.\pend
           \pstart
           \textcolor{gray}{\textbf{–}}\pend
           \pstart
           \textcolor{gray}{\textbf{\begin{otherlanguage}{french}\textbf{Bureaux à \textcolor{pink}{Paris}{}\ledrightnote{\textcolor{pink}{Paris}}:}\end{otherlanguage}}}\pend
           \pstart
           \textcolor{gray}{\textbf{\begin{otherlanguage}{french}\textbf{\textcolor{pink}{rue Richelieu 75.}{}\ledrightnote{\textcolor{pink}{rue Richelieu}}.}\end{otherlanguage}}}\pend
           \pstart
           \centering{}Mein lieber Arthur!\pend
           \pstart
           \noindent{}Ich brauche Dir nicht erſt zu ſchreiben, daß du in Allem auf mich zählen kannſt. Den
               Brief hebe ich auf. Aber bitte, ſchreibe mir bald. Ich ſehne mich ſchon ſehr nach
               einem Worte von Dir. Genauer Bericht, bitte! Mein \textcolor{blue}{Onkel}{}\ledrightnote{→\textcolor{blue}{Fedor Mamroth}} kann Dir keine Empfehlung an den \textcolor{pink}{Frankfurt}{}\ledrightnote{\textcolor{pink}{Frankfurt am Main}}er \textcolor{blue}{Director}{}\ledrightnote{→\textcolor{blue}{Leopold Sonnemann}} geben, weil er ſchlechter mit ihm ſteht als je. Infolge ſeiner
               letzten ſcharfen \label{K_L02702-1v}\edtext{Kritiken}{\lemma{\textnormal{\emph{Kritiken}}}\Cendnote{\textnormal{XXXX}}}\label{K_L02702-1h} iſt es ſogar zu bedrohlichen
               Auftritten zwiſchen meinem \textcolor{blue}{Onkel}{}\ledrightnote{→\textcolor{blue}{Fedor Mamroth}} u. Herrn \textsc{\textcolor{blue}{Sonnemann}{}\ledrightnote{→\textcolor{blue}{Leopold Sonnemann}}} gekommen. {\pb}Ob ich hier werde etwas thun
               können, weiß ich nicht. Jedenfalls arbeite ich daran. Läge Dir aber etwas daran, in
                  \label{K_L02702-2v}\edtext{\textsc{\textcolor{pink}{Breslau}{}\ledrightnote{\textcolor{pink}{Breslau}}}}{\lemma{\textnormal{\emph{Breslau}}}\Cendnote{\textnormal{Aus 1892 sind keine
                  Bemühungen um Aufführungen in \textcolor{pink}{Breslau} bekannt,
                  sehr wohl jedoch aus 1890 und 1891, als Schnitzler mit
                     \textcolor{blue}{Theodor Loewe} wegen einer möglichen
                  Aufführung von \emph{\textcolor{green}{Alkandi’s Lied}} in Kontakt war.
                     Siehe A. S.: \emph{Tagebuch}, 23. 6. 1891}}}\label{K_L02702-2h} aufgeführt zu werden, ſo könnte ich vielleicht etwas richten. Kommſt Du alſo
               doch zuerſt in \label{K_L02702-3v}\edtext{\textsc{\textcolor{pink}{Prag}{}\ledrightnote{\textcolor{pink}{Prag}}}}{\lemma{\textnormal{\emph{Prag}}}\Cendnote{\textnormal{Siehe Paul Goldmann an Arthur Schnitzler, 27. 6. [1892]}}}\label{K_L02702-3h} daran? Und wann und bei wem das \label{K_L02702-6v}\edtext{\textcolor{green}{Buch}{}\ledrightnote{→\textcolor{green}{Anatol}}}{\lemma{\textnormal{\emph{Buch}}}\Cendnote{\textnormal{Arthur Schnitzler: \emph{\textcolor{green}{Anatol}}. \textcolor{pink}{Berlin}: \emph{\textcolor{brown}{Bibliographisches Bureau}}{ }1892, vordatiert auf 1893.}}}\label{K_L02702-6h}? Ich weiß leider ſo gar nichts mehr. Und \label{K_L02702-4v}\edtext{mit wem warſt Du in \textcolor{pink}{Venedig}{}\ledrightnote{\textcolor{pink}{Venedig}}}{\lemma{\textnormal{\emph{mit … Venedig}}}\Cendnote{\textnormal{Schnitzler war von 17. 8. 1892 bis 22. 9. 1892 mit seinem \textcolor{blue}{Bruder}{ }\textcolor{blue}{Julius} in \textcolor{pink}{Venedig}. \textcolor{blue}{Julius} reiste jedoch
                  bereits am 20. 9. 1892
                  ab.}}}\label{K_L02702-4h}? Hätteſt du mir ein Wort geſagt, ſo würde ich meinen Urlaub verſchoben
               haben und mitgekommen ſein.\pend
           \pstart
           Bitte lies: 1.) \label{K_L02702-5v}\edtext{\textsc{\textcolor{blue}{Renan}{}\ledrightnote{\textcolor{blue}{Ernest Renan}}}: \textcolor{green}{Leben Jeſu}{}\ledrightnote{\textcolor{green}{Das Leben Jesu. Vollständige Volks-Ausgabe}}}{\lemma{\textnormal{\emph{Renan: Leben Jeſu}}}\Cendnote{\textnormal{Lektüre keiner der genannten \textcolor{green}{Werke}
                  bekannt}}}\label{K_L02702-5h} (Kleine Volks{\pb}ausgabe) 2. \textsc{\textcolor{blue}{Chamfort}{}\ledrightnote{\textcolor{blue}{Sébastien Roch Nicolas Chamfort}}}: \textsc{\textcolor{green}{Maximes}{}\ledrightnote{\textcolor{green}{Maximes et Pensées, Caractères et Anecdotes}}} (\textsc{\textcolor{green}{Collection des auteurs célèbres}{}\ledrightnote{\textcolor{green}{Œuvres choises de Chamfort, tome 2}}}) 3.) In der \textcolor{green}{Sammlung}{}\ledrightnote{→\textcolor{green}{Œuvres de Sully Prudhomme, tome 2}} der Gedichte von \textsc{\textcolor{blue}{Sully Prud’homme}{}\ledrightnote{\textcolor{blue}{Sully Prudhomme}}} dasjenige, das den Titel trägt »\textsc{\textcolor{green}{Les caresses}{}\ledrightnote{\textcolor{green}{Les caresses}}}«. Beſonders das \textcolor{green}{letztere}{}\ledrightnote{→\textcolor{green}{Les caresses}}
               wird Dir vielleicht ein wenig eine brennende Herzenswunde kühlen.\pend
           \pstart
           Grüß’ Dich Gott, liebſter Freund!\pend
           \pstart
           Ich umarme Dich und \textsc{\textcolor{blue}{Richard}{}\ledrightnote{\textcolor{blue}{Richard Beer-Hofmann}}}.\pend
           \pstart
           Dein {\\[\baselineskip]}\spacefill\mbox{Paul Goldmann.}\pend
           \leftskip=0em{}\endnumbering\briefempfaengerindex{Schnitzler, Arthur@\textsc{Schnitzler, Arthur}!zzzGoldmann, Paul@\emph{von Paul Goldmann}!1892-10-091@{9. 10. {[}1892{]}}|)be}\mylabel{h}\begin{anhang}\end{anhang}\normalsize

\doendnotes{C}
\bigskip
\vfill

\clearpage

\footnotesize

\lohead{\textsc{register}}

% Definiere theindex-Environment komplett neu ohne reledmac
\makeatletter
\renewenvironment{theindex}{%
  \section*{\indexname}%
  \setlength{\parindent}{0pt}%
  \setlength{\parskip}{0pt plus 0.3pt}%
  \let\item\@idxitem
}{%
  \clearpage
}
\makeatother

\IfFileExists{\jobname-pw.ind}{\input{\jobname-pw.ind}}{}

\end{document}

      