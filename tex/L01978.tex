%% latex-korrekturansicht-vorspann.tex
%% Vorspann für die Korrekturansicht.
%% Lädt die gemeinsame Datei latex-vorspann.tex mit gesetztem Schalter.

\newif\ifkorrekturansicht
\korrekturansichttrue

\input{../tex-inputs/latex-vorspann}


               \section[Hugo von Hofmannsthal an Arthur Schnitzler, 7. 11. 1910]{ Hugo von Hofmannsthal an Arthur Schnitzler, 7. 11. 1910}\nopagebreak\mylabel{v}\rehead{ }\normalsize\beginnumbering\briefempfaengerindex{Schnitzler, Arthur@\textsc{Schnitzler, Arthur}!zzzHofmannsthal, Hugo von@\emph{von Hugo von Hofmannsthal}!1910-11-071@{7. 11. 1910}|(be} \toendnotes[C]{\smallbreak\pagebreak[2]} \Standort{CUL, Schnitzler, B 43.}
\physDesc{Brief, 3 Blätter (die Blätter 2 und 3 sind nummeriert), 12 Seiten
\newline{}Handschrift: schwarze Tinte, deutsche Kurrent
\newline{}Schnitzler: mit Bleistift datiert: »Nov. 910.« und beschriftet:
                                    »Hugo« \newline{}Ordnung: 1) mit Bleistift von unbekannter Hand nummeriert: »\strikeout{308}« 2) mit Bleistift von unbekannter Hand nummeriert:
                                    »325«}\buchAbdrucke{\weitereDrucke{Hugo von Hofmannsthal, Arthur Schnitzler: \emph{Briefwechsel}. Hg. Therese Nickl und Heinrich Schnitzler. Frankfurt am Main: \emph{S. Fischer} 1964, S. 257.} }\toendnotes[C]{\smallbreak}\pstart
           \raggedleft{}{\pb}Montag{ }früh.\pend
           \pstart{}mein guter lieber Arthur \pend\pstart
           es tut mir ſo tief ſchmerzlich leid Ihnen weh getan und Sie geärgert zu haben – und
               wenn ſich das Ganze auch (wie Sie ſehen werden) gar nicht in der Wirklichkeit
               abgeſpielt hat – ſo haben Sie darum nicht minder eine unangenehme Stunde durch mich
               erfahren, haben ſich, {\pb}müde und
               enerviert nach einer langen \textcolor{green}{Probe}{}\ledrightnote{\textcolor{green}{Der junge Medardus. Dramatische Historie in einem Vorspiel und fünf Aufzügen}}, hinſetzen und
               mir dieſen begreiflichen und berechtigten Brief ſchreiben müſſen – dies alles tut mir
               ſo furchtbar leid, geſtern und heute nacht, gegen Morgen, jedesmal zur gleichen
               Stunde, wache ich auf und denke an Sie und Ihre Verſtimmung gegen mich mit einem ſo
                  {\pb}gräſslichen Gefühl – geſtern
               nachmittag wollte ich zu Ihnen, hatte aber wirklich zu ſehr Angſt, daſs wir uns, wenn
               auch nur für einen Augenblick, verdüſtert gegenüberſtehen ſollten – ſo ſchreibe ich
               lieber und bitte Sie vor allem herzlich, mir dieſe unglückliche Sache zu verzeihen
                  und{ }{\pb}\substVorne{}\textsuperscript{S}\substDazwischen{}ſ\substHinten{}ie ſoweit als möglich aus Ihrem Gedächtnis zu verbannen.\pend
           \pstart
           Meine unglückliche \strikeout{St} Feder hat etwas ſehr
               Ungeſchicktes hingemalt aber die häſsliche Härte und Rohheit, die Sie herausgeleſen
               haben, war es nicht –: \uline{das} hatte ich weder gethan
               noch vermeinte ich, Ihnen auch extra noch nach {\pb}Jahren mitzuteilen, daſs ich es
               getan hätte. Nein! ſondern: wenn ich ſchrieb »halb abſichtlich, halb unabſichtlich«
               ſo meinte ich einen jener Schwebezuſtände des Willens, zwiſchen Bewuſst und
               Unbewuſst, aber doch ziemlich tief im Unbewuſsten, dem \textcolor{blue}{Freud}{}\ledrightnote{\textcolor{blue}{Sigmund Freud}} in der \textcolor{green}{Pſychopathologie des \uline{Alltagslebens}}{}\ledrightnote{\textcolor{green}{Zur Psychopathologie des Alltagslebens}} ganze Neſter und {\pb}Ketten
               ſehr geiſtreich nachgewieſen hat, jenes ſcheinbar völlig \label{K_L01978_1v}\edtext{unbewuſste fallen laſſen eines Bildes}{\lemma{\textnormal{\emph{unbewuſste … Bildes}}}\Cendnote{\textnormal{vgl. das 8. Kapitel (»\textcolor{green}{Das Vergreifen}«) von \emph{\textcolor{green}{Zur
                     Psychopathologie des Alltagslebens}} (1904)}}}\label{K_L01978_1h}, weil man
               gegen die Perſon, die das Bild darſtellt, etwas verborgenes Böſes auf dem Herzen
               hat, – kurz eine Tat, die vor keinerlei Forum gezogen werden kann, kaum vor das des
               allerzarteſten eigenen Gewiſſens, ſo ſehr verbirgt ſie{ }{\pb}ſich ins Dunkel des Unbewuſsten –
               und wenn ich das heute ausſpreche, ſo nehme ich jenen intim erregten Zuſtand gegen
               das \textcolor{green}{Buch}{}\ledrightnote{→\textcolor{green}{Der Weg ins Freie. Roman}} eben heute hiſtoriſch,
               fühle mich frei davon und darf darum gerade aus Ihrer Hand mit allem, auch dem
               zarteſten Recht, ein neues \textcolor{green}{Exemplar}{}\ledrightnote{→\textcolor{green}{Der Weg ins Freie. Roman}} erbitten.\hspace*{1.5em}Daſs ich ein \textcolor{green}{Exemplar}{}\ledrightnote{→\textcolor{green}{Der Weg ins Freie. Roman}}{ }{\pb}mit einer Zueignung \uline{im bürgerlichen Sinn} ebenſo wenig in der Eiſenbahn liegen
               laſſen \uline{wollte} als meinen Regenſchirm oder
               Spazierſtock, das lieber Arthur, bitte ich Sie, zu glauben.\pend
           \pstart
           \uline{So}. Ich habe dies ausgeſprochen, weil ich finde, daſs
               man in ſo zarten Dingen, wie Freundſchaft und Liebe, auch das auf ſich nehmen muſs,
               was man hätte begehen können. Und {\pb}daſs ich ein ſolches ſymboliſches Liegenlaſſen des \textcolor{green}{Buches}{}\ledrightnote{→\textcolor{green}{Der Weg ins Freie. Roman}} damals hätte vollbringen können, glaube
               ich darum, weil ich mir eben eingebildet hatte, ich hätte es wirklich in der
               Eiſenbahn verloren.\hspace*{1.5em}Nun weiß ich ſeit geſtern, daſs
               gar nicht ich das \textcolor{green}{Buch}{}\ledrightnote{→\textcolor{green}{Der Weg ins Freie. Roman}} verloren
                  habe,{ }{\pb}ſondern \textcolor{blue}{Gerty}{}\ledrightnote{\textcolor{blue}{Gertrude von Hofmannsthal}}, die darüber natürlich ſehr unglücklich war, eben der
               Widmung wegen, vergeblich bei Conducteuren und Stationschefs ſich bemühte es
                  wiederzubeko{\geminationm}en und es aber nicht wiedererlangen
               konnte.\pend
           \pstart
           Es war alſo eine Gedächtnis-täuſchung {\pb}meinerſeits, und die unglücklichen
               Worte jener Nachſchrift aus \textcolor{pink}{Graetz}{}\ledrightnote{\textcolor{pink}{Graz}} haben ſich auf
               ein Doppelt-nichtgeſchehenes bezogen, auf den Schatten eines Schattens oder noch
               weniger.\pend
           \pstart
           Alſo ſeien Sie mir wieder gut, mein lieber Arthur, und glauben Sie weiter, was Sie
                  {\pb}zu glauben, denke ich, nicht
               aufgehört haben, daſs es ſehr wenige Menſchen auf der Welt geben wird, die das Ganze
               Ihres menſchlichen und künſtleriſchen Daſeins mit ſo großer Freude und Liebe, und ſo
               viel Dankbarkeit für das unbegreifliche Phänomen der »Gleichzeitigkeit« erfaſſen, als
               Ihr\pend
           \pstart \spacefill\mbox{Hugo.}\pend{}\endnumbering\briefempfaengerindex{Schnitzler, Arthur@\textsc{Schnitzler, Arthur}!zzzHofmannsthal, Hugo von@\emph{von Hugo von Hofmannsthal}!1910-11-071@{7. 11. 1910}|)be}\mylabel{h}  \normalsize

\doendnotes{C}
\bigskip
\vfill

\clearpage

\footnotesize

\lohead{\textsc{register}}

% Definiere theindex-Environment komplett neu ohne reledmac
\makeatletter
\renewenvironment{theindex}{%
  \section*{\indexname}%
  \setlength{\parindent}{0pt}%
  \setlength{\parskip}{0pt plus 0.3pt}%
  \let\item\@idxitem
}{%
  \clearpage
}
\makeatother

\IfFileExists{\jobname-pw.ind}{\input{\jobname-pw.ind}}{}

\end{document}

      