%% latex-korrekturansicht-vorspann.tex
%% Vorspann für die Korrekturansicht.
%% Lädt die gemeinsame Datei latex-vorspann.tex mit gesetztem Schalter.

\newif\ifkorrekturansicht
\korrekturansichttrue

\input{../tex-inputs/latex-vorspann}


               \section[Hugo von Hofmannsthal an Arthur Schnitzler, 13. 11. 1912]{ Hugo von Hofmannsthal an Arthur Schnitzler, 13. 11. 1912}\nopagebreak\mylabel{v}\rehead{ }\normalsize\beginnumbering\briefempfaengerindex{Schnitzler, Arthur@\textsc{Schnitzler, Arthur}!zzzHofmannsthal, Hugo von@\emph{von Hugo von Hofmannsthal}!1912-11-131@{13. 11. 1912}|(be} \toendnotes[C]{\smallbreak\pagebreak[2]} \Standort{CUL, Schnitzler, B 43.}
\physDesc{Postkarte
\newline{}Handschrift: schwarze Tinte, deutsche Kurrent\newline{}Versand: Stempel: »\nobreak{}\oindex{Rodaun@\textbf{Rodaun}, \emph{Teil eines besiedelten Ortes (A.BSOX)}|pwk}Rodaun, 14 11 12, 3N\nobreak{}«.  \newline{}Ordnung: 1) mit Bleistift von unbekannter Hand nummeriert:
                                    »381« 2) mit Bleistift von unbekannter Hand nummeriert:
                                    »342«}\buchAbdrucke{\weitereDrucke{Hugo von Hofmannsthal, Arthur Schnitzler: \emph{Briefwechsel}. Hg. Therese Nickl und Heinrich Schnitzler. Frankfurt am Main: \emph{S. Fischer} 1964, S. 269.} }\toendnotes[C]{\smallbreak}\pstart{}{\pb}\textsc{Herrn D\textsuperscript{r} Arthur Schnitzler}\pend{}\pstart{}\textcolor{pink}{\textsc{Wien}}{}\ledrightnote{\textcolor{pink}{Wien}}\pend{}\pstart{}\textsc{\textcolor{pink}{XVIII. Sternwartestrasse 71}{}\ledrightnote{\textcolor{pink}{Sternwartestraße}}.}\pend{}{\bigskip}\pstart
           \centering{}{\pb}13 XI.\pend
           \pstart
           Retourniere gleicher Poſt im So{\geminationm}er entliehene Bücher.
                  \textcolor{green}{\textcolor{blue}{Varnhagen}{}\ledrightnote{\textcolor{blue}{Karl August von Varnhagen-Ense}} Band III.}{}\ledrightnote{→\textcolor{green}{Tagebücher}} hat \textcolor{blue}{Waſſermann}{}\ledrightnote{\textcolor{blue}{Jakob Wassermann}} trotz meines Widerſtrebens an ſich geno{\geminationm}en, auf \uline{eigene
                  Verantwortung}, {\pb}und Ihnen
               in \textcolor{pink}{Wien}{}\ledrightnote{\textcolor{pink}{Wien}}{ }ſofort zurückzuſtellen geſchworen.\pend
           \pstart
           Ich gehe, nach Überlegung, Sonntag{ }abends zu dem \textcolor{blue}{Hauptmann}{}\ledrightnote{\textcolor{blue}{Gerhart Hauptmann}}-banquett der
                  \textcolor{brown}{\textsc{Concordia}}{}\ledrightnote{\textcolor{brown}{Concordia}} weil ich es abſurd finde, daſs einem Menſchen wie \textcolor{blue}{H.}{}\ledrightnote{\textcolor{blue}{Gerhart Hauptmann}} gegenüber, nicht ein anſtändiger Menſch an dem ganzen Tisch
               ſitzt.\pend
           \pstart
           Wäre ſehr froh, wenn Sie allenfalls ſchon zurück wären und ſich gleichfalls \label{K_L02096_1v}\edtext{hinzugehen entſchlöſſen}{\lemma{\textnormal{\emph{hinzugehen entſchlöſſen}}}\Cendnote{\textnormal{\textcolor{blue}{Schnitzler} ging hin (Vgl. A. S.: \emph{Tagebuch}, 17. 11. 1912), \textcolor{blue}{Hofmannsthal} wegen eines Streits mit \textcolor{blue}{Salten} nicht (Vgl. A. S.: \emph{Tagebuch}, 15. 11. 1912).}}}\label{K_L02096_1h}.\pend
           \pstart
           Herzlich{\\[\baselineskip]}\spacefill\mbox{Hugo.}\pend
           \leftskip=0em{}\endnumbering\briefempfaengerindex{Schnitzler, Arthur@\textsc{Schnitzler, Arthur}!zzzHofmannsthal, Hugo von@\emph{von Hugo von Hofmannsthal}!1912-11-131@{13. 11. 1912}|)be}\mylabel{h}  \normalsize

\doendnotes{C}
\bigskip
\vfill

\clearpage

\footnotesize

\lohead{\textsc{register}}

% Definiere theindex-Environment komplett neu ohne reledmac
\makeatletter
\renewenvironment{theindex}{%
  \section*{\indexname}%
  \setlength{\parindent}{0pt}%
  \setlength{\parskip}{0pt plus 0.3pt}%
  \let\item\@idxitem
}{%
  \clearpage
}
\makeatother

\IfFileExists{\jobname-pw.ind}{\input{\jobname-pw.ind}}{}

\end{document}

      