%% latex-korrekturansicht-vorspann.tex
%% Vorspann für die Korrekturansicht.
%% Lädt die gemeinsame Datei latex-vorspann.tex mit gesetztem Schalter.

\newif\ifkorrekturansicht
\korrekturansichttrue

\input{../tex-inputs/latex-vorspann}


               \section[Arthur Schnitzler an Hugo von Hofmannsthal, {[}1. 2. 1893{]}]{ Arthur Schnitzler an Hugo von Hofmannsthal, {[}1. 2. 1893{]}}\nopagebreak\mylabel{v}\rehead{ }\normalsize\beginnumbering\briefempfaengerindex{Hofmannsthal, Hugo von@\textsc{Hofmannsthal, Hugo von}!zzzSchnitzler, Arthur@\emph{von Arthur Schnitzler}!1893-02-012@{{[}1. 2. 1893{]}}|(be} \toendnotes[C]{\smallbreak\pagebreak[2]} \Standort{FDH, Hs-30885,33.}
\physDesc{Briefkarte
\newline{}Handschrift: schwarze Tinte, deutsche Kurrent\newline{}Ordnung: von Schnitzler mutmaßlich bei der Durchsicht der Korrespondenz
                                    1929 mit Bleistift datiert: »\substVorne{}\textsuperscript{91}\substDazwischen{}Anfang 93\substHinten{}« }\buchAbdrucke{\weitereDrucke{Hugo von Hofmannsthal, Arthur Schnitzler: \emph{Briefwechsel}. Hg. Therese Nickl und Heinrich Schnitzler. Frankfurt am Main: \emph{S. Fischer} 1964, S. 34.} }\toendnotes[C]{\smallbreak}\pstart{}{\pb}Mein lieber Hugo,\pend\pstart
           \textcolor{blue}{Fels}{}\ledrightnote{\textcolor{blue}{Friedrich Michael Fels}} befindet ſich bereits beſſer; ernſtere
               Beſorgniſſe ſind nun wohl auszuſchließen. Hingegen wäre nunmehr Ihre ſ. Z.
               beſprochene Liebenswürdigkeit ſehr erwünſcht, u die Idee mit den Freunden ohne
                  Namensne{\geminationn}ung iſt ſehr gut, und raſcher Durchführung
               zu empfehlen. –\pend
           \pstart
           Die Arbeit \textcolor{blue}{Engländer}{}\ledrightnote{\textcolor{blue}{Peter Altenberg}}s iſt über \textcolor{green}{Sölneß}{}\ledrightnote{\textcolor{green}{Baumeister Solness}}; \textcolor{blue}{Schick}{}\ledrightnote{\textcolor{blue}{Friedrich Schik}} richtete das
               Ihnen übermittelte Erſuchen an mich. –\pend
           \pstart
           Was ſoll ich der \textcolor{brown}{akad. Vereinigung}{}\ledrightnote{\textcolor{brown}{Wiener Akademische Vereinigung}} ins \textcolor{green}{Exemplar}{}\ledrightnote{→\textcolor{green}{Anatol}}{ }ſchreiben, ich ke{\geminationn}
               mich da gar nicht aus? – \textcolor{blue}{Teltſch}{}\ledrightnote{\textcolor{blue}{Ede Telcs}} erhält eins, {\pb}ſobald ich wieder welche von \textcolor{brown}{Berlin}{}\ledrightnote{\textcolor{brown}{Bibliographisches Bureau}} beko{\geminationm}e, in ein paar Tagen; ich grüſs ihn
               herzlich. – Sah heute im \textcolor{pink}{Gewerbemuſeum}{}\ledrightnote{\textcolor{pink}{Österreichisches Museum für Kunst und Industrie}} Ihr \label{K_L00170_1v}\edtext{\textcolor{green}{Relief}{}\ledrightnote{→\textcolor{green}{Hugo von Hofmannsthal}}}{\lemma{\textnormal{\emph{Relief}}}\Cendnote{\textnormal{Das \textcolor{green}{Relief} befindet sich heute in der Sammlung Richard
                     und Hilda Mises, \emph{\textcolor{brown}{Houghton Library}},
                     Harvard.}}}\label{K_L00170_1h}. Plötzlich lag es da, zwiſchen einem \textcolor{pink}{pompej}{}\ledrightnote{\textcolor{pink}{Pompei}}aniſchen Tiſchfuſs und einem \textcolor{pink}{Nürnberg}{}\ledrightnote{\textcolor{pink}{Nürnberg}}er Hanswurſt. – Ich glaube, es iſt ſehr gut, hab’ aber kein gutes
               Licht gehabt. –\pend
           \pstart
           \textcolor{blue}{\textsc{Salten}}{}\ledrightnote{\textcolor{blue}{Felix Salten}}{ }ſoll Mitte März fort. – \label{K_L00170_2v}\edtext{\textcolor{green}{Familie}{}\ledrightnote{\textcolor{green}{Familie}} beendet}{\lemma{\textnormal{\emph{Familie beendet}}}\Cendnote{\textnormal{Das erlaubt die Datierung des Briefes nach dem 24. 1. 1893, da dieser Tag
                  sowohl im \emph{\textcolor{green}{Tagebuch}} als auch am Manuskript (vgl.
                        \emph{Entworfenes und Verworfenes} 508) als Datum des Abschlusses genannt
                  wird.}}}\label{K_L00170_2h}, traue mich nicht \strikeout{zu}{ }ſie durchzuleſen; fürchte mich vor der grauſamen
               Gewißheit. Abſicht: Ende Feber auf 10–14 Tage in die Wärme, von der
               Klinik und dem grauen Leben weg, das \textcolor{green}{Stück}{}\ledrightnote{→\textcolor{green}{Familie}} im Koffer. \label{K_L00170_3v}\edtext{Schreibe
               jetzt »\textcolor{green}{Verwandlungen}{}\ledrightnote{\textcolor{green}{Die kleine Komödie}}«}{\lemma{\textnormal{\emph{Schreibe jetzt »Verwandlungen«}}}\Cendnote{\textnormal{Am 1. 2. 1893 nahm \textcolor{blue}{Schnitzler} die Arbeit an \emph{\textcolor{green}{Verwandlungen}}
                  wieder auf, was, gemeinsam mit den Datierungen der vorangehenden zwei
                  Korrespondenzstücke, auf die hier geantwortet wird, nach vorne hin beschränkt.}}}\label{K_L00170_3h}, Novellette in Briefen, u gehe heut Abend auf die \label{K_L00170_4v}\edtext{Redoute}{\lemma{\textnormal{\emph{Redoute}}}\Cendnote{\textnormal{Finaler
                  Hinweis zur Datierung: Am 1. 2. 1893 besuchte \textcolor{blue}{Schnitzler} die Redoute der \textcolor{pink}{Hofoper}.}}}\label{K_L00170_4h}, weil ich ein Lebemann bin. – Ihr herzlich ergebener
               Arthur, welcher Sie bald zu ſehen und zu hören verlangt. –\pend
           \endnumbering\briefempfaengerindex{Hofmannsthal, Hugo von@\textsc{Hofmannsthal, Hugo von}!zzzSchnitzler, Arthur@\emph{von Arthur Schnitzler}!1893-02-012@{{[}1. 2. 1893{]}}|)be}\mylabel{h}  \normalsize

\doendnotes{C}
\bigskip
\vfill

\clearpage

\footnotesize

\lohead{\textsc{register}}

% Definiere theindex-Environment komplett neu ohne reledmac
\makeatletter
\renewenvironment{theindex}{%
  \section*{\indexname}%
  \setlength{\parindent}{0pt}%
  \setlength{\parskip}{0pt plus 0.3pt}%
  \let\item\@idxitem
}{%
  \clearpage
}
\makeatother

\IfFileExists{\jobname-pw.ind}{\input{\jobname-pw.ind}}{}

\end{document}

      