%% latex-korrekturansicht-vorspann.tex
%% Vorspann für die Korrekturansicht.
%% Lädt die gemeinsame Datei latex-vorspann.tex mit gesetztem Schalter.

\newif\ifkorrekturansicht
\korrekturansichttrue

\input{../tex-inputs/latex-vorspann}


               \section[Arthur Schnitzler: Widmungsexemplar Episode für Hugo von Hofmannsthal, {[}2. 2. 1892?{]}]{ Arthur Schnitzler: Widmungsexemplar Episode für Hugo von
                    Hofmannsthal, {[}2. 2. 1892?{]}}\nopagebreak\mylabel{v}\rehead{ }\normalsize\beginnumbering\briefempfaengerindex{Hofmannsthal, Hugo von@\textsc{Hofmannsthal, Hugo von}!zzzSchnitzler, Arthur@\emph{von Arthur Schnitzler}!1892-02-021@{{[}2. 2. 1892?{]}}|(be} \toendnotes[C]{\smallbreak\pagebreak[2]} \Standort{FDH, FDH 3223.}
\physDesc{Widmung am Titelblatt
\newline{}Handschrift: schwarze Tinte, deutsche Kurrent}\buchAbdrucke{\weitereDrucke{Hugo von Hofmannsthal: \emph{Bibliothek}. Hg. Ellen Ritter † in Zusammenarbeit mit Dalia Bukauskaité und
                                Konrad Heumann. Frankfurt am Main: \emph{S. Fischer} 2011, S. 603 (Sämtliche Werke. Kritische Ausgabe, XL).} }\toendnotes[C]{\smallbreak}\pstart
           \noindent{}{\pb}Meinem ſehr verehrten Freunde \textsc{Loris}\pend
           \pstart
           herzlichſt{\\[\baselineskip]}\spacefill\mbox{Arth}\pend
           \leftskip=0em{}{\bigskip}\pstart
           \noindent{}\centering{}\textcolor{gray}{\textbf{\textbf{\textcolor{green}{Epiſode}{}\ledrightnote{\textcolor{green}{Episode}}.}}}\pend
           \pstart
           \noindent{}\centering{}\textcolor{gray}{\textbf{Von}}{\\}\textcolor{gray}{\textbf{Arthur Schnitzler.}}\pend
           {\bigskip}\pstart
           \noindent{}\centering{}\textcolor{gray}{\textbf{Den Bühnen gegenüber als Manuſcript.}}\pend
           \pstart
           \noindent{}\centering{}\textcolor{gray}{\textbf{\textcolor{pink}{Wien}{}\ledrightnote{\textcolor{pink}{Wien}}, \label{K_L00070_1v}\edtext{1889}{\lemma{\textnormal{\emph{1889}}}\Cendnote{\textnormal{Die Widmung ist nicht datiert.
                            Der Separatdruck aus Heft 18 der Zeitschrift \emph{\textcolor{green}{An der schönen blauen Donau}} erschien am 8. 9. 1889,
                            als sich die beiden noch nicht kannten. Da auch handschriftliche
                            Änderungen enthalten sind, die in der Druckfassung berücksichtigt
                            wurden, ist eine Übergabe kurz vor Ablieferung des Verlagsmanuskripts
                            der \textcolor{green}{Buchausgabe} am
                                5. 2. 1892 ein wahrscheinlicher Termin. Und da bietet sich wiederum der 2. 2. 1892 an,
                            da an diesem Tag \textcolor{blue}{Hofmannsthal} seinen
                                \emph{\textcolor{green}{Prolog}} verfasste.}}}\label{K_L00070_1h}.}}\pend
           \endnumbering\briefempfaengerindex{Hofmannsthal, Hugo von@\textsc{Hofmannsthal, Hugo von}!zzzSchnitzler, Arthur@\emph{von Arthur Schnitzler}!1892-02-021@{{[}2. 2. 1892?{]}}|)be}\mylabel{h}  \normalsize

\doendnotes{C}
\bigskip
\vfill

\clearpage

\footnotesize

\lohead{\textsc{register}}

% Definiere theindex-Environment komplett neu ohne reledmac
\makeatletter
\renewenvironment{theindex}{%
  \section*{\indexname}%
  \setlength{\parindent}{0pt}%
  \setlength{\parskip}{0pt plus 0.3pt}%
  \let\item\@idxitem
}{%
  \clearpage
}
\makeatother

\IfFileExists{\jobname-pw.ind}{\input{\jobname-pw.ind}}{}

\end{document}

      