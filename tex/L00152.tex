%% latex-korrekturansicht-vorspann.tex
%% Vorspann für die Korrekturansicht.
%% Lädt die gemeinsame Datei latex-vorspann.tex mit gesetztem Schalter.

\newif\ifkorrekturansicht
\korrekturansichttrue

\input{../tex-inputs/latex-vorspann}


               \section[Friedrich M. Fels an Arthur Schnitzler, {[}vor dem 1. 1. 1893?{]}]{ Friedrich M. Fels an Arthur Schnitzler, {[}vor dem 1. 1. 1893?{]}}\nopagebreak\mylabel{v}\rehead{ }\normalsize\beginnumbering\briefempfaengerindex{Schnitzler, Arthur@\textsc{Schnitzler, Arthur}!zzzFels, Friedrich Michael@\emph{von Friedrich Michael Fels}!1892-12-313@{{[}vor dem 1. 1. 1893?{]}}|(be} \toendnotes[C]{\smallbreak\pagebreak[2]} \Standort{DLA, A:Schnitzler, HS.NZ85.1.2956.}
\physDesc{Brief, 1 Blatt, 1 Seite
\newline{}Handschrift: schwarze Tinte, lateinische Kurrent
\newline{}Schnitzler: mit Bleistift datiert: »93« und nummeriert: »4« }\toendnotes[C]{\smallbreak}\pstart
           \noindent{}{\pb}Lieber Dr \hspace*{1.5em}Wie man sich bisweilen irren ka{\geminationn}: Gestern kam ich gar nicht ins Café, sondern um
                  5 Uhr lag ich im Bett. – Warum sah ich sie heute Frühe nicht? Und es
               wäre grade so dringend gewesen! Ich muſs vielleicht heute noch \label{K_L00152_1v}\edtext{ausziehen}{\lemma{\textnormal{\emph{ausziehen}}}\Cendnote{\textnormal{Da \textcolor{blue}{Fels} am
                     1. 1. 1893 sich für eine neue Wohnung entscheidet, dürfte es sich
                  bei dieser um die vorhergehende Adresse in der \textcolor{pink}{Leopoldstadt} handeln.}}}\label{K_L00152_1h}: das hätte mit Ihnen gesprochen.\pend
           \pstart
           – Bitte, \uline{nach 5 Uhr} auf \uline{einen Augenblick ins \textcolor{pink}{Central}{}\ledrightnote{\textcolor{pink}{Café Central}}}, nicht ins groſse Lokal, sondern ins erste der langen Reihe. Ich bitte Sie so
               dringend wie herzlich darum.\pend
           \pstart \spacefill\mbox{Fels}\pend{}\endnumbering\briefempfaengerindex{Schnitzler, Arthur@\textsc{Schnitzler, Arthur}!zzzFels, Friedrich Michael@\emph{von Friedrich Michael Fels}!1892-12-313@{{[}vor dem 1. 1. 1893?{]}}|)be}\mylabel{h}  \normalsize

\doendnotes{C}
\bigskip
\vfill

\clearpage

\footnotesize

\lohead{\textsc{register}}

% Definiere theindex-Environment komplett neu ohne reledmac
\makeatletter
\renewenvironment{theindex}{%
  \section*{\indexname}%
  \setlength{\parindent}{0pt}%
  \setlength{\parskip}{0pt plus 0.3pt}%
  \let\item\@idxitem
}{%
  \clearpage
}
\makeatother

\IfFileExists{\jobname-pw.ind}{\input{\jobname-pw.ind}}{}

\end{document}

      