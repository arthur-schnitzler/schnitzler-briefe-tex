%% latex-korrekturansicht-vorspann.tex
%% Vorspann für die Korrekturansicht.
%% Lädt die gemeinsame Datei latex-vorspann.tex mit gesetztem Schalter.

\newif\ifkorrekturansicht
\korrekturansichttrue

\input{../tex-inputs/latex-vorspann}


               \section[Richard Beer-Hofmann an Arthur Schnitzler, 3. 7. 1898]{ Richard Beer-Hofmann an Arthur Schnitzler, 3. 7. 1898}\nopagebreak\mylabel{v}\rehead{ }\normalsize\beginnumbering\briefempfaengerindex{Schnitzler, Arthur@\textsc{Schnitzler, Arthur}!zzzBeer-Hofmann, Richard@\emph{von Richard Beer-Hofmann}!1898-07-031@{3. 7. 1898}|(be} \toendnotes[C]{\smallbreak\pagebreak[2]} \Standort{CUL, Schnitzler, B 8.}
\physDesc{Brief, 1 Blatt, 4 Seiten
\newline{}Handschrift: Bleistift, lateinische Kurrent\newline{}Ordnung: mit Bleistift von unbekannter Hand nummeriert:
                                    »118« }\buchAbdrucke{\weitereDrucke{Arthur Schnitzler, Richard Beer-Hofmann: \emph{Briefwechsel 1891–1931}. Hg. Konstanze Fliedl. Wien, Zürich: \emph{Europaverlag} 1992, S. 121–122.} }\toendnotes[C]{\smallbreak}\pstart
           \raggedleft{}{\pb}3/7 98\pend
           \pstart
           Lieber Arthur! Brief Cigaretten, Tasche, erhalten, – danke sehr.\pend
           \pstart
           Im August werden wir uns hoffentlich treffen nur wird sich das Nähere
               voraussichtlich erst im August feststellen lassen. \textcolor{blue}{Mirjam}{}\ledrightnote{\textcolor{blue}{Mirjam Beer-Hofmann}} und \textcolor{blue}{Paula}{}\ledrightnote{\textcolor{blue}{Paula Beer-Hofmann}} hab ich
               Ihren Traum erzählt; man {\pb}dankt.
               Der zudringliche \textcolor{blue}{Mime}{}\ledrightnote{→\textcolor{blue}{?? [Schauspieler]}} hat \uline{mir} richtig von \textcolor{pink}{Ebensee}{}\ledrightnote{\textcolor{pink}{Ebensee}} aus eine Ansichtskarte mit Grüßen gesandt – Ein Viech! – Ich
               arbeite, aber nicht genug – leider schlaf ich auch nur täglich von ½ 11
               bis 2–3 Uhr nachts. Zu wenig. Ich erhalte {\pb}soeben die \textcolor{green}{N. Fr. Presse}{}\ledrightnote{\textcolor{green}{Neue Freie Presse}} von heute – (Sonntag
                  3/VII){[}.{]} Lese darin die Inhaltsangabe der »\textcolor{green}{Wiener Rundschau}{}\ledrightnote{\textcolor{green}{Wiener Rundschau}}« und werde nervös. Wenn Sie die
                  \label{K_L00811_1v}\edtext{Inhaltsangabe}{\lemma{\textnormal{\emph{Inhaltsangabe}}}\Cendnote{\textnormal{›— ›\textcolor{green}{\so{Wiener Rundschau}}.‹ (Herausgeber \textcolor{blue}{Gustav \so{Schoenaich}}, \textcolor{blue}{Felix \so{Rappaport}}.) Nr. 16 (II. Jahrgang) vom 1. Juli 1898 hat folgenden
                        Inhalt: \textcolor{green}{Die Maiwiese}. Von \textcolor{blue}{Ricarda \so{Huch}}. — \textcolor{green}{Burne-Jones}. Von \textcolor{blue}{Wilhelm \so{Schölermann}}. — \textcolor{green}{Riesengebirge}. \textcolor{green}{Dichter}. Von \textcolor{blue}{Georg \so{Hirschfeld}}. — \textcolor{green}{Der botanische Poet. (Anton Kerner v.
                           Marilaun †.)} Von \textcolor{blue}{M. \so{Kronfeld}}. — \textcolor{green}{Diese ist sein.} Von \textcolor{blue}{Peter \so{Altenberg}}. — \textcolor{green}{Die Engländer und die Franzosen in der
                           Jubiläums-Ausstellung.} Von \textcolor{blue}{Paul
                           Ritter v. \so{Rittinger}}. — Notizen. — Preis per Quartal 2 fl. Redaction und Administration:
                           \textcolor{pink}{Wien, 1/1, Spiegelgasse Nr. 11}.‹ (\emph{\textcolor{green}{Neue Freie Presse}}, Nr. 12162,
                        3. 7. 1898, S. 9.) Vermutlich dürfte er irrtümlicherweise den Text \textcolor{blue}{Altenberg}s auf sich bezogen haben.}}}\label{K_L00811_1h} lesen werden Sie ahnen warum: Verfolgungswahn? – Schicken Sie
               mir jedenfalls gleich – bitte – die betreffende Nu{\geminationm}er
                  (N\textsuperscript{r.} 16).\pend
           \pstart
           {\pb}Ich habe eben nur die Empfindung
               daß von dieser Seite etwas gegen mich vorbereitet wird. Wenn möglich lachen Sie mich
               aus – hoffentlich ist Grund dazu – zum Auslachen\pend
           \pstart
           Ihre \textcolor{green}{Stücke}{}\ledrightnote{→\textcolor{green}{Der grüne Kakadu – Paracelsus – Die Gefährtin. Drei Einakter}}? Wie heißen sie? \textcolor{green}{Kakadu}{}\ledrightnote{\textcolor{green}{Der grüne Kakadu. Groteske in einem Akt}} und – –?\pend
           \pstart
           Herzlichst Ihr {\\[\baselineskip]}\spacefill\mbox{Richard}\pend
           \leftskip=0em{}\endnumbering\briefempfaengerindex{Schnitzler, Arthur@\textsc{Schnitzler, Arthur}!zzzBeer-Hofmann, Richard@\emph{von Richard Beer-Hofmann}!1898-07-031@{3. 7. 1898}|)be}\mylabel{h}  \normalsize

\doendnotes{C}
\bigskip
\vfill

\clearpage

\footnotesize

\lohead{\textsc{register}}

% Definiere theindex-Environment komplett neu ohne reledmac
\makeatletter
\renewenvironment{theindex}{%
  \section*{\indexname}%
  \setlength{\parindent}{0pt}%
  \setlength{\parskip}{0pt plus 0.3pt}%
  \let\item\@idxitem
}{%
  \clearpage
}
\makeatother

\IfFileExists{\jobname-pw.ind}{\input{\jobname-pw.ind}}{}

\end{document}

      