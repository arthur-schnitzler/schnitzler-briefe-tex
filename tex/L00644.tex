%% latex-korrekturansicht-vorspann.tex
%% Vorspann für die Korrekturansicht.
%% Lädt die gemeinsame Datei latex-vorspann.tex mit gesetztem Schalter.

\newif\ifkorrekturansicht
\korrekturansichttrue

\input{../tex-inputs/latex-vorspann}


               \section[Hugo von Hofmannsthal an Arthur Schnitzler, 9. 2. 1897]{ Hugo von Hofmannsthal an Arthur Schnitzler, 9. 2. 1897}\nopagebreak\mylabel{v}\rehead{ }\normalsize\beginnumbering\briefempfaengerindex{Schnitzler, Arthur@\textsc{Schnitzler, Arthur}!zzzHofmannsthal, Hugo von@\emph{von Hugo von Hofmannsthal}!1897-02-091@{9. 2. 1897}|(be} \toendnotes[C]{\smallbreak\pagebreak[2]} \Standort{CUL, Schnitzler, B 43.}
\physDesc{Kartenbrief
\newline{}Handschrift: schwarze Tinte, deutsche Kurrent\newline{}Versand: 1) Rohrpost 2) Stempel: »\nobreak{}\oindex{III., Landstrasse@\textbf{III., Landstraße}, \emph{Bezirk (A.BZK)}|pwk}Wien 3/3, 9 II 97, 12–N\nobreak{}«. 3) Stempel: »\nobreak{}\oindex{IX., Alsergrund@\textbf{IX., Alsergrund}, \emph{Bezirk (A.BZK)}|pwk}Wien 9/2, 9 II 97, 12 50N\nobreak{}«. 
\newline{}Schnitzler: mit Bleistift datiert: »9/2 97« \newline{}Ordnung: mit Bleistift von unbekannter Hand nummeriert:
                                    »86« }\buchAbdrucke{\weitereDrucke{Hugo von Hofmannsthal, Arthur Schnitzler: \emph{Briefwechsel}. Hg. Therese Nickl und Heinrich Schnitzler. Frankfurt am Main: \emph{S. Fischer} 1964, S. 77.} }\toendnotes[C]{\smallbreak}\pstart{}{\pb}\textcolor{gray}{\textbf{An}}\pend{}\pstart{}Herrn D\textsuperscript{r} Arthur Schnitzler\pend{}\pstart{}\textcolor{gray}{\textbf{in}}{ }\textcolor{pink}{Wien}{}\ledrightnote{\textcolor{pink}{Wien}}\pend{}\pstart{}\textcolor{pink}{IX Franckgasse 1}{}\ledrightnote{\textcolor{pink}{Frankgasse}}\pend{}{\bigskip}\pstart
           \raggedleft{}{\pb}Dienstag.\pend
           \pstart{}lieber Arthur\pend\pstart
           wollen Sie mir einen großen \label{K_L00644_1v}\edtext{Gefallen}{\lemma{\textnormal{\emph{Gefallen}}}\Cendnote{\textnormal{\textcolor{blue}{Hofmannsthal} glaubte zu diesem Zeitpunkt, \textcolor{blue}{Hermine Benedict} wäre in ihn verliebt. Die
                  Klärung der Sache, die auch \textcolor{blue}{Schnitzler} als
                  dritten, nicht amourös Interessierten involviert, zieht sich bis in den
                     März.}}}\label{K_L00644_1h} thuen? telephonieren Sie zwiſchen 2 und
                  4 der \textcolor{blue}{Minnie}{}\ledrightnote{\textcolor{blue}{Hermine von Schaffgotsch}} 12140 und fragen Sie
               irgend etwas gleichgiltiges z. B. Sie hätten gehört, daſs Sonntag die
                  \label{K_L00644_2v}\edtext{2\textsuperscript{te}
                  Vorſtellung}{\lemma{\textnormal{\emph{2te
                  Vorſtellung}}}\Cendnote{\textnormal{Privatinszenierung
                  von \textcolor{blue}{Hofmannsthal}s \emph{\textcolor{green}{Was die Braut geträumt hat. Ein Gelegenheitsgedicht}}, die zweite
                  Vorstellung fand am Donnerstag, den 18. 2. 1897 statt.}}}\label{K_L00644_2h}{ }ſein ſoll, ob es wahr iſt?\pend
           \pstart
           und wenn Sie mit ihr ſelbſt ſprechen können und es unauffällig ſich anknüpfen läſst
               (an das Hereinfahren Freitag{ }abend) fragen Sie ſie, wie es ihr geht und ſchreiben mir das \uline{pneumatiſch}, bitte! Wenn Sie aber nur für \uline{möglich} halten, daſs es auffallen oder daſs man den Zuſa{\geminationm}enhang errathen könnte, ſo iſt natürlich beſſer Sie
               laſſen es und ich thue es ſelber. Aber bitte antworten Sie jedenfalls!\hspace*{3.5em}Ihr\spacefill\mbox{Hugo.}\pend
           \endnumbering\briefempfaengerindex{Schnitzler, Arthur@\textsc{Schnitzler, Arthur}!zzzHofmannsthal, Hugo von@\emph{von Hugo von Hofmannsthal}!1897-02-091@{9. 2. 1897}|)be}\mylabel{h}  \normalsize

\doendnotes{C}
\bigskip
\vfill

\clearpage

\footnotesize

\lohead{\textsc{register}}

% Definiere theindex-Environment komplett neu ohne reledmac
\makeatletter
\renewenvironment{theindex}{%
  \section*{\indexname}%
  \setlength{\parindent}{0pt}%
  \setlength{\parskip}{0pt plus 0.3pt}%
  \let\item\@idxitem
}{%
  \clearpage
}
\makeatother

\IfFileExists{\jobname-pw.ind}{\input{\jobname-pw.ind}}{}

\end{document}

      