%% latex-korrekturansicht-vorspann.tex
%% Vorspann für die Korrekturansicht.
%% Lädt die gemeinsame Datei latex-vorspann.tex mit gesetztem Schalter.

\newif\ifkorrekturansicht
\korrekturansichttrue

\input{../tex-inputs/latex-vorspann}


               \section[Arthur Schnitzler an Richard Beer-Hofmann, 24. 11. 1899]{ Arthur Schnitzler an Richard Beer-Hofmann, 24. 11. 1899}\nopagebreak\mylabel{v}\rehead{ }\normalsize\beginnumbering\briefempfaengerindex{Beer-Hofmann, Richard@\textsc{Beer-Hofmann, Richard}!zzzSchnitzler, Arthur@\emph{von Arthur Schnitzler}!1899-11-241@{24. 11. 1899}|(be} \toendnotes[C]{\smallbreak\pagebreak[2]} \Standort{YCGL, MSS 31.}
\physDesc{Postkarte
\newline{}Handschrift: schwarze Tinte, deutsche Kurrent\newline{}Versand: Stempel: »\nobreak{}\oindex{I., Innere Stadt@\textbf{I., Innere Stadt}, \emph{Bezirk (A.BZK)}|pwk}Wien 1/1, 24 11 99, 12–1N\nobreak{}«.  \newline{}Ordnung: mit Bleistift von unbekannter Hand datiert: »24. 11.« }\buchAbdrucke{\weitereDrucke{Arthur Schnitzler, Richard Beer-Hofmann: \emph{Briefwechsel 1891–1931}. Hg. Konstanze Fliedl. Wien, Zürich: \emph{Europaverlag} 1992, S. 139.} }\pstart{}{\pb}Herrn \textsc{Dr. Richard
                     Beer-Hofmann}\pend{}\pstart{}\textcolor{pink}{Wien}{}\ledrightnote{\textcolor{pink}{Wien}}\pend{}\pstart{}\textsc{\textcolor{pink}{I. Wollzeile 15}{}\ledrightnote{\textcolor{pink}{Wollzeile}}}.\pend{}{\bigskip}\pstart
           \noindent{}{\pb}Alſo es bleibt bei Sonntag um
                  5. Sollten Sie etwa Luſt haben ſich den \textcolor{blue}{Korff}{}\ledrightnote{\textcolor{blue}{Arnold Korff}} in der \textcolor{green}{Lbl.}{}\ledrightnote{\textcolor{green}{Liebelei. Schauspiel in drei Akten}} anzuſehn?\pend
           \pstart
           Herzlichſt Ihr{\\[\baselineskip]}\spacefill\mbox{Arthur.}\pend
           \leftskip=0em{}\endnumbering\briefempfaengerindex{Beer-Hofmann, Richard@\textsc{Beer-Hofmann, Richard}!zzzSchnitzler, Arthur@\emph{von Arthur Schnitzler}!1899-11-241@{24. 11. 1899}|)be}\mylabel{h}  \normalsize

\doendnotes{C}
\bigskip
\vfill

\clearpage

\footnotesize

\lohead{\textsc{register}}

% Definiere theindex-Environment komplett neu ohne reledmac
\makeatletter
\renewenvironment{theindex}{%
  \section*{\indexname}%
  \setlength{\parindent}{0pt}%
  \setlength{\parskip}{0pt plus 0.3pt}%
  \let\item\@idxitem
}{%
  \clearpage
}
\makeatother

\IfFileExists{\jobname-pw.ind}{\input{\jobname-pw.ind}}{}

\end{document}

      