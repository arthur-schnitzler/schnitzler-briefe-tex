%% latex-korrekturansicht-vorspann.tex
%% Vorspann für die Korrekturansicht.
%% Lädt die gemeinsame Datei latex-vorspann.tex mit gesetztem Schalter.

\newif\ifkorrekturansicht
\korrekturansichttrue

\input{../tex-inputs/latex-vorspann}


               \section[Arthur Schnitzler an Hugo von Hofmannsthal, {[}4. 6. 1898{]}]{ Arthur Schnitzler an Hugo von Hofmannsthal, {[}4. 6. 1898{]}}\nopagebreak\mylabel{v}\rehead{ }\normalsize\beginnumbering\briefempfaengerindex{Hofmannsthal, Hugo von@\textsc{Hofmannsthal, Hugo von}!zzzSchnitzler, Arthur@\emph{von Arthur Schnitzler}!1898-06-042@{{[}4. 6. 1898{]}}|(be} \toendnotes[C]{\smallbreak\pagebreak[2]} \Standort{FDH, Hs-30885,66.}
\physDesc{Brief, 1 Blatt, 4 Seiten
\newline{}Handschrift: Bleistift, deutsche Kurrent\newline{}Ordnung: von Schnitzler mutmaßlich bei der Durchsicht der Korrespondenz 1929 mit Bleistift
                                    datiert: »Anf? 98« }\buchAbdrucke{\weitereDrucke{Hugo von Hofmannsthal, Arthur Schnitzler: \emph{Briefwechsel}. Hg. Therese Nickl und Heinrich Schnitzler. Frankfurt am Main: \emph{S. Fischer} 1964, S. 102.} }\pstart
           \raggedleft{}{\pb}Samſtag.\pend
           \pstart
           Lieber Hugo, morgen früh will ich auf den \textcolor{pink}{Semmering}{}\ledrightnote{\textcolor{pink}{Semmering}} fahren, dann \textsc{per} Rad zum \textcolor{blue}{Richard}{}\ledrightnote{\textcolor{blue}{Richard Beer-Hofmann}}, wo ich wohl
                        Dinſtag{ }ſein werde. Wahrſcheinlich fahr ich allein;
                        \textcolor{blue}{\textsc{Kramer}}{}\ledrightnote{\textcolor{blue}{Leopold Kramer}}{ }ſcheint {\pb}unverläßlich. Daſs Sie \textcolor{blue}{\textsc{Kerr}}{}\ledrightnote{\textcolor{blue}{Alfred Kerr}} nicht kennen gelernt haben, iſt ſchade; im Anfang befangen und etwas
                    unſicher findet er ſich bald bei einigem Entgegenko{\geminationm}en und wirkt durch ſeinen Verſtand, ſeine Sympathie und mannigfache {\pb}günſtige Intentionen höchſt erfreulich. –\pend
           \pstart
           Es geht mir mit der Sti{\geminationm}ung nun etwas beſſer; es iſt
                    doch ſehr ſonderbar, wie auch \strikeout{\textcolor{gray}{ganz feſtſtehende}} ihrem Weſen nach unveränderliche ſeeliſche Laſten an Schwere gewinnen und
                    verlieren können. – Ich möchte auch in \textcolor{pink}{Kärnthen}{}\ledrightnote{\textcolor{pink}{Kärnten}}{ }{\pb}ein bischen arbeiten. Sie können mir jedenfalls nach
                        \textcolor{pink}{\textsc{Steindorf}}{}\ledrightnote{\textcolor{pink}{Steindorf am Ossiacher See}} zu \textcolor{blue}{R.}{}\ledrightnote{\textcolor{blue}{Richard Beer-Hofmann}}{ }ſchreiben; obzwar ich nicht glaube, dſs ich
                    dort bleibe.\pend
           \pstart
           \textcolor{blue}{Brahm}{}\ledrightnote{\textcolor{blue}{Otto Brahm}} läßt Sie vielmals grüßen; er hofft Sie
                    werden noch oft Gelegenhei\textcolor{gray}{t} haben ſich am \textcolor{brown}{Dtſch
                        Theater}{}\ledrightnote{\textcolor{brown}{Deutsches Theater Berlin}} wohl zu fühlen.\pend
           \pstart Herzlichſte Grüße Ihr \spacefill\mbox{A.}\pend{}\endnumbering\briefempfaengerindex{Hofmannsthal, Hugo von@\textsc{Hofmannsthal, Hugo von}!zzzSchnitzler, Arthur@\emph{von Arthur Schnitzler}!1898-06-042@{{[}4. 6. 1898{]}}|)be}\mylabel{h}  \normalsize

\doendnotes{C}
\bigskip
\vfill

\clearpage

\footnotesize

\lohead{\textsc{register}}

% Definiere theindex-Environment komplett neu ohne reledmac
\makeatletter
\renewenvironment{theindex}{%
  \section*{\indexname}%
  \setlength{\parindent}{0pt}%
  \setlength{\parskip}{0pt plus 0.3pt}%
  \let\item\@idxitem
}{%
  \clearpage
}
\makeatother

\IfFileExists{\jobname-pw.ind}{\input{\jobname-pw.ind}}{}

\end{document}

      