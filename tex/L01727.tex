%% latex-korrekturansicht-vorspann.tex
%% Vorspann für die Korrekturansicht.
%% Lädt die gemeinsame Datei latex-vorspann.tex mit gesetztem Schalter.

\newif\ifkorrekturansicht
\korrekturansichttrue

\input{../tex-inputs/latex-vorspann}


               \section[Hugo von Hofmannsthal an Arthur Schnitzler, 1. 11. {[}1907{]}]{ Hugo von Hofmannsthal an Arthur Schnitzler, 1. 11. {[}1907{]}}\nopagebreak\mylabel{v}\rehead{ }\normalsize\beginnumbering\briefempfaengerindex{Schnitzler, Arthur@\textsc{Schnitzler, Arthur}!zzzHofmannsthal, Hugo von@\emph{von Hugo von Hofmannsthal}!1907-11-011@{1. 11. {[}1907{]}}|(be} \toendnotes[C]{\smallbreak\pagebreak[2]} \Standort{CUL, Schnitzler, B 43.}
\physDesc{Brief, 1 Blatt, 4 Seiten
\newline{}Handschrift: schwarze Tinte, deutsche Kurrent
\newline{}Schnitzler: mit Bleistift die Jahreszahl ergänzt: »90\textcolor{gray}{7}« \newline{}Ordnung: mit Bleistift von unbekannter Hand nummeriert:
                              »187a« und beschriftet: »?Date?« }\buchAbdrucke{\weitereDrucke{Hugo von Hofmannsthal, Arthur Schnitzler: \emph{Briefwechsel}. Hg. Therese Nickl und Heinrich Schnitzler. Frankfurt am Main: \emph{S. Fischer} 1964, S. 232–233.} }\toendnotes[C]{\smallbreak}\pstart
           \raggedleft{}{\pb}\textcolor{pink}{R.}{}\ledrightnote{\textcolor{pink}{Rodaun}}, 1. XI.\pend
           \pstart{}mein guter Arthur,\pend\pstart
           wir kämen ja ſehr gern – aber ich arbeite jetzt (ungefähr ſeit 2 Wochen) jeden
               Vormittag jeden Abend. Durch einen Abend bei Euch verlöre ich einen Abend und den
               nächſten Vormittag (und vielleicht durch Nervoſität mehr als das) alſo muſs ich
               leider verzichten.\pend
           \pstart
           Nicht wahr Sie bringen das Geſpräch dann mit \textcolor{blue}{Auernheimer}{}\ledrightnote{\textcolor{blue}{Raoul Auernheimer}} auf mich und ſpeciell darauf, daſs er den »\label{K_L01727_1v}\edtext{\textcolor{green}{Rodauner Aeſtheten}{}\ledrightnote{→\textcolor{green}{Der Herr von Balthesser}}}{\lemma{\textnormal{\emph{Rodauner Aeſtheten}}}\Cendnote{\textnormal{\textcolor{blue}{Auernheimer} schreibt in seiner Rezension von \textcolor{blue}{Richard von Schaukal}s \emph{\textcolor{green}{Leben und Meinungen
                     des Herrn Andreas von Balthesser}}: »Der Rodauner Ästhet geht ihm
                     sogar entgegen und macht dem neu Angekommenen ein Kompliment über sein jüngstes
                     Buch.« (\textcolor{blue}{Rauoul Auernheimer}:
                        \emph{\textcolor{green}{Der Herr von Balthesser}}. In: \emph{\textcolor{green}{Neue Freie Presse}}, Nr. 15462,
                        8. 9. 1907, Morgenblatt, S. 1–3, hier
                     S. 1). }}}\label{K_L01727_1h}« {\pb}anführte als eine Figur die von \textcolor{blue}{Schaukal}{}\ledrightnote{\textcolor{blue}{Richard von Schaukal}}
               entzückt iſt und der \textcolor{blue}{Schaukal}{}\ledrightnote{\textcolor{blue}{Richard von Schaukal}} für ſeinen Dreck (um
               den ſich das \textcolor{green}{Feuilleton}{}\ledrightnote{→\textcolor{green}{Der Herr von Balthesser}} dreht)
               becomplimentiert. Fragen Sie ihn bitte welche meiner Arbeiten einer ähnlichen
               Characterisierung die Handhabe bietet.\pend
           \pstart
           Ich habe es ſo ſatt, nach 17 Jahren ziemlich ernſthaften Arbeitens in dieſer Weiſe
               »ironiſiert« zu werden – und in dieſem Fall iſt es ja kein \textcolor{blue}{Lausbub}{}\ledrightnote{→\textcolor{blue}{Raoul Auernheimer}}, ſondern jemand
               anſcheinend Anſtändiger. Also wozu?\pend
           \pstart
           {\pb}Mein \textcolor{green}{Stück}{}\ledrightnote{→\textcolor{green}{Silvia im »Stern«}} iſt ein recht ſonderbares Ding. Wenns
               nicht miſslingt – iſt es viel wert, für mich meine ich. Jedenfalls gehen mir hie und
               da einige Ahnungen auf darüber wie das was man die Leute reden läſst wieder
               zurückwirkt auf die ſogenannte Handlung (das Scenarium) u. ſ. f.
                  u. ſ. f.\hspace*{1.5em}Sehr einſam iſt man in ſolchen
               Momenten, wie tief in einem Bergwerk nur im Finſtern {\pb}irgendwo neben ſich, aber weit,
               glaubt man einen andern hämmern zu hören.\hspace*{1.5em}Sie
                  z. B.\hspace*{1.5em}So habe ich neulich den erſten Act vom »\textcolor{green}{Ruf des Lebens}{}\ledrightnote{\textcolor{green}{Der Ruf des Lebens. Schauspiel in drei Akten}}« ſehr aufmerkſam geleſen, mit viel
               Gewinn (vielleicht auch für Sie.) Ich glaube das notwendige organiſche Stück ſteckt
               hier (wie natürlich){[}.{]} Sie ſind aber wie mit geſchloſſenen Augen
               darüber hinweggegangen. (In der Scene \textcolor{green}{Marie}{}\ledrightnote{→\textcolor{green}{Der Ruf des Lebens. Schauspiel in drei Akten}}–\textcolor{green}{Adjunct}{}\ledrightnote{→\textcolor{green}{Der Ruf des Lebens. Schauspiel in drei Akten}}{ }ſteckt die \uline{Idee} des
               Stückes.) Davon nächſtens.\pend
           \pstart
           Ich glaube ich werde Sie plötzlich \uline{brauchen}, zu
               Hilfe.\pend
           \pstart
           Adieu.{\\[\baselineskip]}Ihr\spacefill\mbox{Hugo.}\pend
           \leftskip=0em{}\pstart
           \noindent{}\label{T_L01727_1v}\edtext{Ich wüſste
                  gern, wie denn überhaupt \textcolor{blue}{A.}{}\ledrightnote{\textcolor{blue}{Raoul Auernheimer}} zu meinen Arbeiten
                  ſteht, z. B. den proſaiſchen.}{\lemma{\textnormal{\emph{Ich … proſaiſchen.}}}\Cendnote{\textnormal{quer am linken Rand der zweiten
                     Seite}}}\label{T_L01727_1h}\pend
           \endnumbering\briefempfaengerindex{Schnitzler, Arthur@\textsc{Schnitzler, Arthur}!zzzHofmannsthal, Hugo von@\emph{von Hugo von Hofmannsthal}!1907-11-011@{1. 11. {[}1907{]}}|)be}\mylabel{h}  \normalsize

\doendnotes{C}
\bigskip
\vfill

\clearpage

\footnotesize

\lohead{\textsc{register}}

% Definiere theindex-Environment komplett neu ohne reledmac
\makeatletter
\renewenvironment{theindex}{%
  \section*{\indexname}%
  \setlength{\parindent}{0pt}%
  \setlength{\parskip}{0pt plus 0.3pt}%
  \let\item\@idxitem
}{%
  \clearpage
}
\makeatother

\IfFileExists{\jobname-pw.ind}{\input{\jobname-pw.ind}}{}

\end{document}

      