%% latex-korrekturansicht-vorspann.tex
%% Vorspann für die Korrekturansicht.
%% Lädt die gemeinsame Datei latex-vorspann.tex mit gesetztem Schalter.

\newif\ifkorrekturansicht
\korrekturansichttrue

\input{../tex-inputs/latex-vorspann}


               \section[Richard Beer-Hofmann an Arthur Schnitzler, Mitte August 1905]{ Richard Beer-Hofmann an Arthur Schnitzler, Mitte August 1905}\nopagebreak\mylabel{v}\rehead{ }\normalsize\beginnumbering\briefempfaengerindex{Schnitzler, Arthur@\textsc{Schnitzler, Arthur}!zzzBeer-Hofmann, Richard@\emph{von Richard Beer-Hofmann}!1905-08-151@{Mitte August 1905}|(be} \toendnotes[C]{\smallbreak\pagebreak[2]} \Standort{CUL, Schnitzler, B 8.}
\physDesc{Brief, 1 Blatt, 4 Seiten
\newline{}Handschrift: schwarze Tinte, lateinische Kurrent
\newline{}Schnitzler: mit Bleistift datiert: »Mitte Auguſt 905« \newline{}Ordnung: mit Bleistift von unbekannter Hand nummeriert:
                                    »204« }\buchAbdrucke{\weitereDrucke{Arthur Schnitzler, Richard Beer-Hofmann: \emph{Briefwechsel 1891–1931}. Hg. Konstanze Fliedl. Wien, Zürich: \emph{Europaverlag} 1992, S. 174–175.} }\toendnotes[C]{\smallbreak}\pstart
           \noindent{}\centering{}{\pb}\textcolor{pink}{\textcolor{gray}{\textbf{GRAND HOTEL STUBAI}}}{}\ledrightnote{\textcolor{pink}{Grand Hotel Stubai}}\pend
           \pstart
           \noindent{}\centering{}\textcolor{gray}{\textbf{\textcolor{pink}{FULPMES}{}\ledrightnote{\textcolor{pink}{Fulpmes}}}}{ }\textsc{\textcolor{gray}{\textbf{bei \textcolor{pink}{innsbruck}{}\ledrightnote{\textcolor{pink}{Innsbruck}} (\textcolor{pink}{tirol}{}\ledrightnote{\textcolor{pink}{Tirol}})}}}\pend
           \pstart
           \noindent{}\textcolor{gray}{\textbf{TELEGRAMM-ADRESSE:}}\pend
           \pstart
           \textcolor{gray}{\textbf{\textcolor{pink}{STUBAIHOTEL FULPMES-INNSBRUCK}{}\ledrightnote{\textcolor{pink}{Grand Hotel Stubai}}}}\pend
           \pstart
           \textcolor{gray}{\textbf{ENDSTATION DER ELEKTRISCHEN BERGBAHN}}\pend
           \pstart
           \textcolor{gray}{\textbf{INNSBRUCK-FULPMES}}\pend
           \pstart
           Lieber Arthur! Wir sind da \label{T_L01542_1v}\edtext{oben}{\lemma{\textnormal{\emph{oben}}}\Cendnote{\textnormal{Ein Pfeil weist
                  auf die Hoteladresse.}}}\label{T_L01542_1h}. In \textcolor{pink}{Kärnten}{}\ledrightnote{\textcolor{pink}{Kärnten}} fand ich
               keine Unterkunft. Dort – wie im \textcolor{pink}{Pusterthal}{}\ledrightnote{\textcolor{pink}{Pustertal}} alles
               furchtbar überfüllt, so dass ich froh war hier unterzukommen. Das Hôtel ist \strikeout{es} erst voriges Jahr eröffnet worden, noch nicht sehr
               bekannt und daher halbleer.\pend
           \pstart
           \textcolor{blue}{Wassermanns}{}\ledrightnote{\textcolor{blue}{Jakob Wassermann}{\newline}\textcolor{blue}{Julie Wassermann}}, \textcolor{blue}{S. Fischer}{}\ledrightnote{\textcolor{blue}{Samuel Fischer}}, \textcolor{blue}{Bella Wengerow}{}\ledrightnote{\textcolor{blue}{Isabella Vengerova}},
                  \textcolor{blue}{Schwester}{}\ledrightnote{→\textcolor{blue}{Zinaida A. Vengerova}}, \textcolor{blue}{Mutter}{}\ledrightnote{→\textcolor{blue}{Pauline Wengeroff}}, u. D\textsuperscript{r}{ }\textcolor{blue}{Kaufmann}{}\ledrightnote{\textcolor{blue}{Arthur Kaufmann}}{ }{\pb}sind hier. \textcolor{blue}{Paula}{}\ledrightnote{\textcolor{blue}{Paula Beer-Hofmann}} hat kein Behagen an den kühlen Abenden und auch sonst an
               der Gegend – der Wald ist für sie – augenblicklich – zu weit vom Hôtel. Ich will also
               am 21 oder 22 von hier weg, und über \textcolor{pink}{Bozen}{}\ledrightnote{\textcolor{pink}{Bozen}}, eventuell \textcolor{pink}{Gardasee}{}\ledrightnote{\textcolor{pink}{Lago di Garda}}, an
               den \textcolor{pink}{Lido}{}\ledrightnote{\textcolor{pink}{Lido}}. Hoffentlich tut ihr der Aufenthalt dort
               gut. Sie ist \uline{sehr} blutleer, und hat recht miserable
               Nerven. Das \textcolor{green}{Stück}{}\ledrightnote{→\textcolor{green}{Die Andere}} von \textcolor{blue}{Bahr}{}\ledrightnote{\textcolor{blue}{Hermann Bahr}} blieb in \textcolor{pink}{Rodaun}{}\ledrightnote{\textcolor{pink}{Rodaun}} liegen, weil in folge der Aufschrift {\pb}»\textcolor{brown}{Eisenstein}{}\ledrightnote{\textcolor{brown}{J. Eisenstein {\kaufmannsund} Co.}}« nur Bücher darin vermutet wurden, mit denen es nicht eilig sei;
               ich lasse es mir heute nachschicken.\pend
           \pstart
           Bitte sind Sie so gut und fügen Sie auf beiliegendem Brief die Adresse hinzu. Wer »A«
               sagt – –!\pend
           \pstart
           Hier hat sich das Gerücht verbreitet, Sie hätten dem \textcolor{blue}{Hugo}{}\ledrightnote{\textcolor{blue}{Hugo von Hofmannsthal}} zwei wunderschöne \textcolor{green}{Stücke}{}\ledrightnote{→\textcolor{green}{Zwischenspiel. Komödie in drei Akten}{\newline}→\textcolor{green}{Der Ruf des Lebens. Schauspiel in drei Akten}}{ }\label{K_L01542_2v}\edtext{vorgelesen}{\lemma{\textnormal{\emph{vorgelesen}}}\Cendnote{\textnormal{vgl. A. S.: \emph{Tagebuch}, 12. 8. 1905}}}\label{K_L01542_2h}. Ich freue mich sehr im
                  Oktober mehr davon zu erfahren.\pend
           \pstart
           Von mir will ich nichts schreiben, ich ziehe es vor Ihnen mündlich vorzujammern –
               obgleich {\pb}Sie mir dann bei
               physischen Dingen versichern werden Sie hätten dies Alles seit Jahren.\pend
           \pstart
           Schreiben Sie mir, bitte, i{\geminationm}er wo Sie sind – ich will es
               auch tun. Die Möglichkeit soll uns doch bleiben, uns etwas zu sagen.\pend
           \pstart
           Viele Grüsse an Ihre \textcolor{blue}{Frau}{}\ledrightnote{→\textcolor{blue}{Olga Schnitzler}} von
               mir und \textcolor{blue}{Paula}{}\ledrightnote{\textcolor{blue}{Paula Beer-Hofmann}}.\pend
           \pstart
           Von Herzen Ihr{\\[\baselineskip]}\spacefill\mbox{Richard}\pend
           \leftskip=0em{}\pstart
           \noindent{}Bitte entschuldigen Sie mich gelegentlich bei Ihrer \textcolor{blue}{Schwägerin}{}\ledrightnote{→\textcolor{blue}{Elisabeth Steinrück}}, u. \textcolor{blue}{Steinrück}{}\ledrightnote{\textcolor{blue}{Albert Steinrück}}. Ich hatte vor der Abreise zuviel zu besorgen.\hspace*{1.5em}R.\pend
           \endnumbering\briefempfaengerindex{Schnitzler, Arthur@\textsc{Schnitzler, Arthur}!zzzBeer-Hofmann, Richard@\emph{von Richard Beer-Hofmann}!1905-08-151@{Mitte August 1905}|)be}\mylabel{h}  \normalsize

\doendnotes{C}
\bigskip
\vfill

\clearpage

\footnotesize

\lohead{\textsc{register}}

% Definiere theindex-Environment komplett neu ohne reledmac
\makeatletter
\renewenvironment{theindex}{%
  \section*{\indexname}%
  \setlength{\parindent}{0pt}%
  \setlength{\parskip}{0pt plus 0.3pt}%
  \let\item\@idxitem
}{%
  \clearpage
}
\makeatother

\IfFileExists{\jobname-pw.ind}{\input{\jobname-pw.ind}}{}

\end{document}

      