%% latex-korrekturansicht-vorspann.tex
%% Vorspann für die Korrekturansicht.
%% Lädt die gemeinsame Datei latex-vorspann.tex mit gesetztem Schalter.

\newif\ifkorrekturansicht
\korrekturansichttrue

\input{../tex-inputs/latex-vorspann}


               \section[Arthur Schnitzler an Hugo von Hofmannsthal, {[}21. 4. 1893?{]}]{ Arthur Schnitzler an Hugo von Hofmannsthal, {[}21. 4. 1893?{]}}\nopagebreak\mylabel{v}\rehead{ }\normalsize\beginnumbering\briefempfaengerindex{Hofmannsthal, Hugo von@\textsc{Hofmannsthal, Hugo von}!zzzSchnitzler, Arthur@\emph{von Arthur Schnitzler}!1893-04-211@{{[}21. 4. 1893?{]}}|(be} \toendnotes[C]{\smallbreak\pagebreak[2]} \Standort{FDH, Hs-30885,39.}
\physDesc{Brief, 1 Blatt, 2 Seiten
\newline{}Handschrift: schwarze Tinte, deutsche Kurrent\newline{}Ordnung: mit Bleistift von unbekannter Hand datiert: »91?« }\buchAbdrucke{\weitereDrucke{1) Hugo von Hofmannsthal, Arthur Schnitzler: \emph{Briefwechsel}. Hg. Therese Nickl und Heinrich Schnitzler. Frankfurt am Main: \emph{S. Fischer} 1964, S. 47–48.} \weitereDrucke{2) Hermann Bahr, Arthur Schnitzler: \emph{Briefwechsel, Aufzeichnungen, Dokumente
                                (1891–1931)}. Hg. Kurt Ifkovits und Martin Anton Müller. Göttingen: \emph{Wallstein} 2018.} }\toendnotes[C]{\smallbreak}\pstart{}{\pb}Lieber Hugo,\pend\pstart
           beifolgende Briefe, erſter \label{K_L00199_1v}\edtext{von
                        \textcolor{blue}{\textsc{Fels}}{}\ledrightnote{\textcolor{blue}{Friedrich Michael Fels}}}{\lemma{\textnormal{\emph{von
                        Fels}}}\Cendnote{\textnormal{In einem Brief vom
                            20. 4. 1893 (\emph{Deutsches Literaturarchiv}, A:Schnitzler, 85.1.2956) schreibt
                            \textcolor{blue}{Fels}, dass er zum Monatsende nach
                            \textcolor{pink}{Wien} und mit 1. 5. bei
                        der \emph{\textcolor{brown}{Deutschen Zeitung}} beginnen könne. Er
                        würde dann ein Drittel oder Viertel des Einkommens dazu verwenden, seine
                        Schulden in \textcolor{pink}{Meran} zu begleichen.}}}\label{K_L00199_1h}
                    zweiter \label{K_L00199_2v}\edtext{von Frau \textcolor{blue}{\textsc{Clara Schreiber}}{}\ledrightnote{\textcolor{blue}{Clara Schreiber}}}{\lemma{\textnormal{\emph{von Frau Clara Schreiber}}}\Cendnote{\textnormal{Sie bittet um Hilfe, \textcolor{blue}{Fels} habe nun seit acht Wochen sein Logis nicht
                        bezahlt und er würde behaupten, kein Geld zu haben (\emph{Cambridge University Library}, Schnitzler, B 385).}}}\label{K_L00199_2h}, an die ich
                    unſern Freund empfohlen habe, die Gattin des Dr. \textcolor{blue}{\textsc{Schreiber}}{}\ledrightnote{\textcolor{blue}{Joseph Schreiber}}, Curarzt in \textcolor{pink}{Meran}{}\ledrightnote{\textcolor{pink}{Meran}}, – ſind auch
                    für Sie von Intereſſe. Ich bitte Sie, ſich vielleicht an \textcolor{blue}{Bahr}{}\ledrightnote{\textcolor{blue}{Hermann Bahr}} zu wenden, was Sie ja von uns dreien am
                        \label{K_L00199_3v}\edtext{leichteſten}{\lemma{\textnormal{\emph{leichteſten}}}\Cendnote{\textnormal{Sie wohnten beide in der \textcolor{pink}{Salesianergasse 12}.}}}\label{K_L00199_3h} u beſten können, {\pb}und mich ſo raſch als möglich von dem Ausfall
                    Ihrer Bemühungen zu unterrichten, ſowie die beiden Briefe mir
                    zurückzuſchicken.\pend
           \pstart
           Ich bin mit herzlichen Grüßen{\\[\baselineskip]}Ihr\spacefill\mbox{Arthur}\pend
           \leftskip=0em{}\endnumbering\briefempfaengerindex{Hofmannsthal, Hugo von@\textsc{Hofmannsthal, Hugo von}!zzzSchnitzler, Arthur@\emph{von Arthur Schnitzler}!1893-04-211@{{[}21. 4. 1893?{]}}|)be}\mylabel{h}  \normalsize

\doendnotes{C}
\bigskip
\vfill

\clearpage

\footnotesize

\lohead{\textsc{register}}

% Definiere theindex-Environment komplett neu ohne reledmac
\makeatletter
\renewenvironment{theindex}{%
  \section*{\indexname}%
  \setlength{\parindent}{0pt}%
  \setlength{\parskip}{0pt plus 0.3pt}%
  \let\item\@idxitem
}{%
  \clearpage
}
\makeatother

\IfFileExists{\jobname-pw.ind}{\input{\jobname-pw.ind}}{}

\end{document}

      