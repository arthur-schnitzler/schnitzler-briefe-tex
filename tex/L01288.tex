%% latex-korrekturansicht-vorspann.tex
%% Vorspann für die Korrekturansicht.
%% Lädt die gemeinsame Datei latex-vorspann.tex mit gesetztem Schalter.

\newif\ifkorrekturansicht
\korrekturansichttrue

\input{../tex-inputs/latex-vorspann}


               \section[Joseph Victor Widmann an Arthur Schnitzler, 9. 5. 1903]{ Joseph Victor Widmann an Arthur Schnitzler, 9. 5. 1903}\nopagebreak\mylabel{v}\rehead{ }\normalsize\beginnumbering\briefempfaengerindex{Schnitzler, Arthur@\textsc{Schnitzler, Arthur}!zzzWidmann, Joseph Victor@\emph{von Joseph Victor Widmann}!1903-05-091@{9. 5. 1903}|(be} \toendnotes[C]{\smallbreak\pagebreak[2]} \Standort{CUL, Schnitzler, B 113.}
\physDesc{Bildpostkarte
\newline{}Handschrift: schwarze Tinte, deutsche Kurrent\newline{}Versand: 1) Stempel: »\nobreak{}\oindex{Bern@\textbf{Bern}, \emph{Besiedelter Ort (A.BSO)}|pwk}Bern Brf. Exp., 10. V. 03., 11\nobreak{}«.  2) Stempel: »\nobreak{}\oindex{IX., Alsergrund@\textbf{IX., Alsergrund}, \emph{Bezirk (A.BZK)}|pwk}Wien 9/3, 11. 5. 03, 5N, Bestellt\nobreak{}«. 3) mit Bleistift von unbekannter Hand zur Adresse
                                            ergänzt: »IX/3«\newline{}Zusatz: auf dem Motiv im Vordergrund eine Illustration von \textcolor{blue}{Rudolf Münger} zu Schnitzlers \textcolor{green}{Der grüne Kakadu}, der
                                            Vogel und der Bildrahmen grün koloriert }\toendnotes[C]{\smallbreak}\pstart{}{\pb}\textsc{Herrn D\textsuperscript{r} Arthur
                            Schnitzler}\pend{}\pstart{}Dichter in\pend{}\pstart{}\textcolor{pink}{\textsc{Wien}}{}\ledrightnote{\textcolor{pink}{Wien}}.\pend{}\pstart{}(\textsc{\textcolor{pink}{Österreich}{}\ledrightnote{\textcolor{pink}{Österreich}}}.)\pend{}{\bigskip}\pstart
           \noindent{}\centering{}\textcolor{gray}{\textbf{{\pb}⋅1903⋅
                            Theater-Bazar}}\pend
           \pstart
           Da ſehn Sie, verehrteſter Herr, wie man in \textcolor{pink}{Bern}{}\ledrightnote{\textcolor{pink}{Bern}}{ }Sie liebt u. ke{\geminationn}t
                    und ſchätzt. Und natürlich aufführt, ſobald das neue \textcolor{brown}{Theater}{}\ledrightnote{→\textcolor{brown}{Stadttheater}} in dieſem Herbſt ſeine Hallen
                    öffnet.\pend
           \pstart
           Mit höflichem Gruß{\\[\baselineskip]}\spacefill\mbox{J. V. Widmann}\pend
           \leftskip=0em{}\pstart
           \textcolor{gray}{\textbf{\textcolor{pink}{BERN}{}\ledrightnote{\textcolor{pink}{Bern}}, den}}{ }9. Mai 1903.\pend
           \endnumbering\briefempfaengerindex{Schnitzler, Arthur@\textsc{Schnitzler, Arthur}!zzzWidmann, Joseph Victor@\emph{von Joseph Victor Widmann}!1903-05-091@{9. 5. 1903}|)be}\mylabel{h}  \normalsize

\doendnotes{C}
\bigskip
\vfill

\clearpage

\footnotesize

\lohead{\textsc{register}}

% Definiere theindex-Environment komplett neu ohne reledmac
\makeatletter
\renewenvironment{theindex}{%
  \section*{\indexname}%
  \setlength{\parindent}{0pt}%
  \setlength{\parskip}{0pt plus 0.3pt}%
  \let\item\@idxitem
}{%
  \clearpage
}
\makeatother

\IfFileExists{\jobname-pw.ind}{\input{\jobname-pw.ind}}{}

\end{document}

      