%% latex-korrekturansicht-vorspann.tex
%% Vorspann für die Korrekturansicht.
%% Lädt die gemeinsame Datei latex-vorspann.tex mit gesetztem Schalter.

\newif\ifkorrekturansicht
\korrekturansichttrue

\input{../tex-inputs/latex-vorspann}


               \section[Arthur Schnitzler an Thomas Mann, 7. 6. 1925]{ Arthur Schnitzler an Thomas Mann, 7. 6. 1925}\nopagebreak\mylabel{v}\rehead{ }\normalsize\beginnumbering\briefempfaengerindex{Mann, Thomas@\textsc{Mann, Thomas}!zzzSchnitzler, Arthur@\emph{von Arthur Schnitzler}!1925-06-071@{7. 6. 1925}|(be} \toendnotes[C]{\smallbreak\pagebreak[2]} \buchAlsQuelle{\pwindex{Festgruesse an Thomas Mann@\emph{Festgrüße an Thomas Mann}|pwk}\pwindex{Neue Freie Presse@\emph{Neue Freie Presse}|pwk}\emph{Festgrüße an Thomas Mann. Zum fünfzigsten Geburtstage.} In: \emph{Neue Freie Presse}, Nr. 21814, 7. 6. 1925, S. 29.}\buchAbdrucke{\weitereDrucke{Hertha Krotkoff: \emph{Arthur Schnitzler – Thomas Mann: Briefe.} In: \emph{Modern Austrian Literature}, Jg. 7 (1974) Nr. 1/2, S. 4.} }\pstart\center{}{\pb}Lieber und verehrter Thomas Mann!\pend\pstart
           Erlauben Sie mir, daß ich ſtatt eines Glückwunſches ein paar anſpruchsloſe
               Bemerkungen hieher ſetze, die ich anläßlich der Lektüre Ihres wundervollen »\textcolor{green}{Zauberberg}{}\ledrightnote{\textcolor{green}{Der Zauberberg. Roman}}« in mein Notizbuch geſchrieben habe
               und die ich daher in aller Beſcheidenheit als Ihnen gewidmet bezeichnen darf. Im
               übrigen wiſſen Sie ſeit lange, wie ſehr ich Sie liebe und bewundere.\pend
           \pstart
           Ihr{\\[\baselineskip]}\spacefill\mbox{\so{Arthur Schnitzler.}}\pend
           \leftskip=0em{}\pstart
           \noindent{}Dem Humoriſten – und nur ihm unter allen Schriftſtellern – iſt Weitſchweifigkeit
                  erlaubt; ja, ſie iſt unter Umſtänden ein Kunſtmittel mehr, deſſen er nicht
                  entraten darf und kann.\pend
           \pstart
           Behagen iſt die eigentliche Grundbedingung des Humors ſowohl in ſubjektivem als in
                  objektivem Sinn. Und der Begriff des Behagens verträgt ſich nicht mit
                  Beſchränkungen irgendwelcher Art. In gewiſſem Sinne kann der Humoriſt niemals ein
                  Ende machen – kaum einen Anfang. Nur techniſche Notwendigkeiten nötigen ihn
                  dazu.\pend
           \pstart
           Der Humoriſt luſtwandelt innerhalb der Unendlichkeit.\pend
           \pstart
           In der Tragik gerät der menſchliche Geiſt, ſo tief er auch hinabſteigen mag,
                  irgendeinmal auf Grund – im Humor niemals.\pend
           \pstart
           Die tragiſche Weltanſchauung, von den Höhen des Humors aus betrachtet, wirkt in
                  jedem Falle irgendwie beſchränkt, wenn nicht lächerlich oder gar unſinnig.\pend
           \pstart
           Dem Humor, dem göttlichen Kind, iſt nichts verwehrt; auch nicht mit dem Schmerz,
                  dem Elend, dem Tod zu ſpielen. Wenn die Ironie, der Witz, die Satire das Gleiche
                  verſuchen, empfinden wir das als geſchmacklos, roh, wenn nicht gar als
                  Blasphemie.\pend
           \pstart
           Ironie, Witz, Satire können nur als gelegentliche Ausdrucksformen des Humors
                  künſtleriſch beſtehen. Auf ſich ſelbſt geſtellt mögen ſie allerlei Wirkung tun –
                  Wirkungen politiſcher, moraliſcher, ſchriftſtelleriſcher Art, aber mit Kunſt in
                  höherem Sinne haben dieſe Wirkungen nichts zu ſchaffen.\pend
           \pstart
           Humor iſt immer dämoniſcher Natur; das Reich von Witz, Ironie, Satire, dieſer
                  gefallenen Engel des Geiſtes, iſt innerhalb des Sataniſchen geſchlossen.\pend
           \pstart
           Nicht jeder Künſtler von Genie – ſo ſchrieb ich vor kurzem \textcolor{blue}{Hugo Thimig}{}\ledrightnote{\textcolor{blue}{Hugo Thimig}} ins Stammbuch – hat Humor, aber jeder Künſtler
                  von Humor (nicht jeder Spaßmacher) hat Genie. Humor iſt der weitere und höhere
                  Begriff. Er iſt das eigentliche Genie des Herzens, da Güte wohl ohne Humor, aber
                  Humor niemals ohne Güte beſtehen kann.\pend
           \endnumbering\briefempfaengerindex{Mann, Thomas@\textsc{Mann, Thomas}!zzzSchnitzler, Arthur@\emph{von Arthur Schnitzler}!1925-06-071@{7. 6. 1925}|)be}\mylabel{h}  \normalsize

\doendnotes{C}
\bigskip
\vfill

\clearpage

\footnotesize

\lohead{\textsc{register}}

% Definiere theindex-Environment komplett neu ohne reledmac
\makeatletter
\renewenvironment{theindex}{%
  \section*{\indexname}%
  \setlength{\parindent}{0pt}%
  \setlength{\parskip}{0pt plus 0.3pt}%
  \let\item\@idxitem
}{%
  \clearpage
}
\makeatother

\IfFileExists{\jobname-pw.ind}{\input{\jobname-pw.ind}}{}

\end{document}

      