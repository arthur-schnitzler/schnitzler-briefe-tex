%% latex-korrekturansicht-vorspann.tex
%% Vorspann für die Korrekturansicht.
%% Lädt die gemeinsame Datei latex-vorspann.tex mit gesetztem Schalter.

\newif\ifkorrekturansicht
\korrekturansichttrue

\input{../tex-inputs/latex-vorspann}


               \section[Arthur Schnitzler an Hugo von Hofmannsthal, 28. 4. 1897]{ Arthur Schnitzler an Hugo von Hofmannsthal, 28. 4. 1897}\nopagebreak\mylabel{v}\rehead{ }\normalsize\beginnumbering\briefempfaengerindex{Hofmannsthal, Hugo von@\textsc{Hofmannsthal, Hugo von}!zzzSchnitzler, Arthur@\emph{von Arthur Schnitzler}!1897-04-281@{28. 4. 1897}|(be} \toendnotes[C]{\smallbreak\pagebreak[2]} \Standort{FDH, Hs-30885,57.}
\physDesc{Brief, 1 Blatt, 4 Seiten
\newline{}Handschrift: schwarze Tinte, deutsche Kurrent}\buchAbdrucke{\weitereDrucke{Hugo von Hofmannsthal, Arthur Schnitzler: \emph{Briefwechsel}. Hg. Therese Nickl und Heinrich Schnitzler. Frankfurt am Main: \emph{S. Fischer} 1964, S. 82–83.} }\toendnotes[C]{\smallbreak}\pstart
           {\pb}\textcolor{pink}{5  \textsc{rue de Maubeuge}}{}\ledrightnote{\textcolor{pink}{rue de Maubeuge}}{\\}\textcolor{pink}{\textsc{Paris}}{}\ledrightnote{\textcolor{pink}{Paris}}{ }28. 4. 97\pend
           \pstart{}Lieber Hugo, \pend\pstart
           an \textcolor{blue}{Fiſcher}{}\ledrightnote{\textcolor{blue}{Samuel Fischer}} hab ich geſchrieben, ich zweifle
                    nicht, dſs er ohne weiters einverſtanden iſt. Warum aber glauben Sie, daſs alle
                    dieſe Sachen ſich nur von \textcolor{pink}{Paris}{}\ledrightnote{\textcolor{pink}{Paris}} aus komiſch
                    anhören. Sie ſind übrigens mehr ekelhaft als komiſch. We{\geminationn}{ }{\pb}ſich \textcolor{blue}{Clara}{}\ledrightnote{\textcolor{blue}{Clara Katharina Pollaczek}} nur
                    nicht viel draus macht und ſich nicht gar zu viel \label{K_L00672_1v}\edtext{ſekiren}{\lemma{\textnormal{\emph{ſekiren}}}\Cendnote{\textnormal{österreichisch sekkieren: ärgern}}}\label{K_L00672_1h}
                    laſſen muſs. Grüßen Sie ſie u \textcolor{blue}{Anna}{}\ledrightnote{\textcolor{blue}{Anna Epstein}} von mir
                    herzlich.\pend
           \pstart
           – Iſt es möglich, dſs \textcolor{blue}{Minnie}{}\ledrightnote{\textcolor{blue}{Hermine von Schaffgotsch}} an dem Tratſch
                    zum Theil ſchuld iſt? (Da wird ſie mir ja auch was ähnliches anrichten!)
                    Sonderbarer Weiſe das einzige literariſche, worüber ich hier ein biſſel
                    nachgedacht, iſt das \textcolor{green}{Stück}{}\ledrightnote{→\textcolor{green}{Der Weg ins Freie. Roman}},
                    wo \strikeout{ſich} ſie mich {\pb}rettet. Aber ſie ändert ſich mir im Kopf, ſie ist ſchon beinah blond.\pend
           \pstart
           Meinen Brief von geſtern oder vorgeſtern haben Sie doch? –\pend
           \pstart
           Arbeiten Sie was?\pend
           \pstart
           Eben komme ich von \textcolor{pink}{\textsc{Versailles}}{}\ledrightnote{\textcolor{pink}{Versailles}} zurück und habe eine unbeſchreibliche Luſt nach Grün und Luft und Stille
                        heimge{\pb}bracht; eine ſo heftige Ungeduld, daſs ich
                    gleich wieder aus Paris wegmöchte, we{\geminationn}’s ſo ohne
                    weiteres ginge.\pend
           \pstart
           Das gibt ſich wieder.\pend
           \pstart
           Seien Sie herzlich gegrüßt.{\\[\baselineskip]}Ihr\spacefill\mbox{Arthur.}\pend
           \leftskip=0em{}\pstart
           \noindent{}Statt gemiſchten Hausbrodes eſſe ich gemiſchtes Hausbrod. –\pend
           \endnumbering\briefempfaengerindex{Hofmannsthal, Hugo von@\textsc{Hofmannsthal, Hugo von}!zzzSchnitzler, Arthur@\emph{von Arthur Schnitzler}!1897-04-281@{28. 4. 1897}|)be}\mylabel{h}  \normalsize

\doendnotes{C}
\bigskip
\vfill

\clearpage

\footnotesize

\lohead{\textsc{register}}

% Definiere theindex-Environment komplett neu ohne reledmac
\makeatletter
\renewenvironment{theindex}{%
  \section*{\indexname}%
  \setlength{\parindent}{0pt}%
  \setlength{\parskip}{0pt plus 0.3pt}%
  \let\item\@idxitem
}{%
  \clearpage
}
\makeatother

\IfFileExists{\jobname-pw.ind}{\input{\jobname-pw.ind}}{}

\end{document}

      