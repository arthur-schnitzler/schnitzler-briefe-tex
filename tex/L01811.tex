%% latex-korrekturansicht-vorspann.tex
%% Vorspann für die Korrekturansicht.
%% Lädt die gemeinsame Datei latex-vorspann.tex mit gesetztem Schalter.

\newif\ifkorrekturansicht
\korrekturansichttrue

\input{../tex-inputs/latex-vorspann}


               \section[Arthur Schnitzler an Richard Beer-Hofmann, 28. 11. 1908]{ Arthur Schnitzler an Richard Beer-Hofmann, 28. 11. 1908}\nopagebreak\mylabel{v}\rehead{ }\normalsize\beginnumbering\briefempfaengerindex{Beer-Hofmann, Richard@\textsc{Beer-Hofmann, Richard}!zzzSchnitzler, Arthur@\emph{von Arthur Schnitzler}!1908-11-281@{28. 11. 1908}|(be} \toendnotes[C]{\smallbreak\pagebreak[2]} \Standort{YCGL, MSS 31.}
\physDesc{Brief, 1 Blatt, 4 Seiten, Umschlag
\newline{}Handschrift: Bleistift, deutsche Kurrent\newline{}Versand: ohne postalischen Übermittlungsvermerk }\buchAbdrucke{\weitereDrucke{Arthur Schnitzler, Richard Beer-Hofmann: \emph{Briefwechsel 1891–1931}. Hg. Konstanze Fliedl. Wien, Zürich: \emph{Europaverlag} 1992, S. 191–192.} }\toendnotes[C]{\smallbreak}\pstart{}{\pb}\textcolor{gray}{\textbf{Dr. Arthur Schnitzler}}\pend{}\pstart{}\textcolor{gray}{\textbf{\textcolor{pink}{Wien XVIII. Spoettelgasse 7}{}\ledrightnote{\textcolor{pink}{Edmund-Weiß-Gasse}}.}}\pend{}{\bigskip}\pstart{}{\pb}\textsc{Dr. Richard Beer Hofma{\geminationn}}\pend{}\pstart{}\textcolor{pink}{Wien}{}\ledrightnote{\textcolor{pink}{Wien}}\pend{}{\bigskip}\pstart
           \noindent{}{\pb}\textcolor{gray}{\textbf{Dr. Arthur Schnitzler}}\hfill 28/11 08\pend
           \pstart
           \textcolor{gray}{\textbf{\textcolor{pink}{Wien XVIII. Spoettelgasse 7}{}\ledrightnote{\textcolor{pink}{Edmund-Weiß-Gasse}}.}}\pend
           \pstart{}lieber Richard,\pend\pstart
           we{\geminationn}{ }\textcolor{blue}{\textsc{Kerr}}{}\ledrightnote{\textcolor{blue}{Alfred Kerr}} jetzt bei Ihnen iſt (er war gegen 1 bei mir ohne mich zu treffen)
               ſo fragen Sie ihn bitte, wie lang er hier bleibt und arrangiren Sie es {\pb}womöglich daſs wir morgen nach der \label{K_L01811_1v}\edtext{\textcolor{blue}{Heine}{}\ledrightnote{\textcolor{blue}{Heinrich Heine}}{ }Sache}{\lemma{\textnormal{\emph{Heine Sache}}}\Cendnote{\textnormal{Am 29. 11. 1908 fand im \textcolor{pink}{Bösendorfer-Saal} die \textcolor{blue}{Heine}-Feier des
                     \emph{\textcolor{brown}{Vereins für Kunst und Kultur}} statt. \textcolor{blue}{Alfred Kerr} hielt zu Beginn der Veranstaltung
                  einen Vortrag über \textcolor{blue}{Heine}. \textcolor{blue}{Schnitzler} war anwesend, anschließend speisten sie im \textcolor{pink}{Meissl {\kaufmannsund} Schadn}. (vgl. A. S.: \emph{Tagebuch}, 29. 11. 1908)}}}\label{K_L01811_1h} mit ihm
               allein (bei \textcolor{pink}{\textsc{Meissl}}{}\ledrightnote{\textcolor{pink}{Meissl & Schadn}}) nachtmahlen. Und we{\geminationn} Sie ev. heute Abends mit ihm
               ſind, ſchreiben {\pb}Sie mir ein unverbindl Wort (wir ſind
               im Concert{ }\textcolor{blue}{\textsc{Dohnanyi}}{}\ledrightnote{\textcolor{blue}{Ernst von Dohnányi}})\pend
           \pstart
           Montag fahren wir aller Wahrſcheinlichke\textcolor{gray}{it} nach \textcolor{pink}{\textsc{Semmering}}{}\ledrightnote{\textcolor{pink}{Semmering}} – auf 2–3 Tage, vielleicht {\pb}ko{\geminationm}t \textcolor{blue}{\textsc{Kerr}}{}\ledrightnote{\textcolor{blue}{Alfred Kerr}} hinauf\textcolor{gray}{?}\pend
           \pstart
           – All dies an Sie, verzeihen Sie, weil \textcolor{blue}{\textsc{Kerr}}{}\ledrightnote{\textcolor{blue}{Alfred Kerr}} behauptet hat, noch keine Adreſſe zu haben.\pend
           \pstart
           Herzlichſt Ihr{\\[\baselineskip]}\spacefill\mbox{A.}\pend
           \leftskip=0em{}\pstart
           \noindent{}Auch heute nach 5 bin ich zu Hauſe.\pend
           \endnumbering\briefempfaengerindex{Beer-Hofmann, Richard@\textsc{Beer-Hofmann, Richard}!zzzSchnitzler, Arthur@\emph{von Arthur Schnitzler}!1908-11-281@{28. 11. 1908}|)be}\mylabel{h}  \normalsize

\doendnotes{C}
\bigskip
\vfill

\clearpage

\footnotesize

\lohead{\textsc{register}}

% Definiere theindex-Environment komplett neu ohne reledmac
\makeatletter
\renewenvironment{theindex}{%
  \section*{\indexname}%
  \setlength{\parindent}{0pt}%
  \setlength{\parskip}{0pt plus 0.3pt}%
  \let\item\@idxitem
}{%
  \clearpage
}
\makeatother

\IfFileExists{\jobname-pw.ind}{\input{\jobname-pw.ind}}{}

\end{document}

      