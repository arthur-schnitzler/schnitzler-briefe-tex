%% latex-korrekturansicht-vorspann.tex
%% Vorspann für die Korrekturansicht.
%% Lädt die gemeinsame Datei latex-vorspann.tex mit gesetztem Schalter.

\newif\ifkorrekturansicht
\korrekturansichttrue

\input{../tex-inputs/latex-vorspann}


               \section[Arthur Schnitzler an Hugo von Hofmannsthal, {[}7. 5. 1892{]}]{ Arthur Schnitzler an Hugo von Hofmannsthal, {[}7. 5. 1892{]}}\nopagebreak\mylabel{v}\rehead{ }\normalsize\beginnumbering\briefempfaengerindex{Hofmannsthal, Hugo von@\textsc{Hofmannsthal, Hugo von}!zzzSchnitzler, Arthur@\emph{von Arthur Schnitzler}!1892-05-071@{{[}7. 5. 1892{]}}|(be} \toendnotes[C]{\smallbreak\pagebreak[2]} \Standort{FDH, Hs-30885,28.}
\physDesc{Briefkarte
\newline{}Handschrift: Bleistift, deutsche Kurrent\newline{}Ordnung: mit Bleistift von unbekannter Hand datiert: »\strikeout{91?}{ }92« }\buchAbdrucke{\weitereDrucke{1) Hugo von Hofmannsthal, Arthur Schnitzler: \emph{Briefwechsel}. Hg. Therese Nickl und Heinrich Schnitzler. Frankfurt am Main: \emph{S. Fischer} 1964, S. 21.} \weitereDrucke{2) Hermann Bahr, Arthur Schnitzler: \emph{Briefwechsel, Aufzeichnungen, Dokumente (1891–1931)}. Hg. Kurt Ifkovits und Martin Anton Müller. Göttingen: \emph{Wallstein} 2018, S. 24.} }\toendnotes[C]{\smallbreak}\pstart
           \noindent{}{\pb}Lieber Loris, eben erhalte ich einen Brief von
                  \textcolor{blue}{Bahr}{}\ledrightnote{\textcolor{blue}{Hermann Bahr}}; er käme heute Nachmittag um 3 Uhr mit
               Ihnen zu mir. Da aber mein \textcolor{blue}{Papa}{}\ledrightnote{→\textcolor{blue}{Johann Schnitzler}}
               noch \label{K_L00101_1v}\edtext{krank}{\lemma{\textnormal{\emph{krank}}}\Cendnote{\textnormal{\textcolor{blue}{Johann Schnitzler} hatte eine Rippen- oder
                  Brustfellentzündung (vgl. A. S.: \emph{Tagebuch}, 24. 4. 1892, 27. 4. 1892).}}}\label{K_L00101_1h} iſt, ordinire ich für ihn
                  \textcolor{pink}{Burgring 1}{}\ledrightnote{\textcolor{pink}{Burgring}}, und kann erſt um ½ 5{ }\textcolor{pink}{Giſelastraße}{}\ledrightnote{\textcolor{pink}{Bösendorferstraße}}{ }ſein. Abends bin ich im \textcolor{pink}{Ausſtellungs{\pb}theater}{}\ledrightnote{\textcolor{pink}{Internationales Ausstellungstheater im k.k. Prater}}; können wir nicht
               auch nachher beiſa{\geminationm}en sein? Können Sie um
                  ½ 5 nicht auf mich warten, so laſſen Sie mir entweder eine Poſt
               zurück oder ko{\geminationm}en Sie vielleicht mit \textcolor{blue}{Bahr}{}\ledrightnote{\textcolor{blue}{Hermann Bahr}} zu mir auf den \textcolor{pink}{Burgring}{}\ledrightnote{\textcolor{pink}{Burgring}}
               um 3 Uhr. Grüßen Sie \textcolor{blue}{Bahr}{}\ledrightnote{\textcolor{blue}{Hermann Bahr}} und seien
               Sie ſelbſt, Unſichtbarer, vielmals gegrüßt, \spacefill\mbox{Arth}\pend
           \endnumbering\briefempfaengerindex{Hofmannsthal, Hugo von@\textsc{Hofmannsthal, Hugo von}!zzzSchnitzler, Arthur@\emph{von Arthur Schnitzler}!1892-05-071@{{[}7. 5. 1892{]}}|)be}\mylabel{h}  \normalsize

\doendnotes{C}
\bigskip
\vfill

\clearpage

\footnotesize

\lohead{\textsc{register}}

% Definiere theindex-Environment komplett neu ohne reledmac
\makeatletter
\renewenvironment{theindex}{%
  \section*{\indexname}%
  \setlength{\parindent}{0pt}%
  \setlength{\parskip}{0pt plus 0.3pt}%
  \let\item\@idxitem
}{%
  \clearpage
}
\makeatother

\IfFileExists{\jobname-pw.ind}{\input{\jobname-pw.ind}}{}

\end{document}

      