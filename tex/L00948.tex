%% latex-korrekturansicht-vorspann.tex
%% Vorspann für die Korrekturansicht.
%% Lädt die gemeinsame Datei latex-vorspann.tex mit gesetztem Schalter.

\newif\ifkorrekturansicht
\korrekturansichttrue

\input{../tex-inputs/latex-vorspann}


               \section[Arthur Schnitzler an Richard Beer-Hofmann, 20. 7. 1899]{ Arthur Schnitzler an Richard Beer-Hofmann, 20. 7. 1899}\nopagebreak\mylabel{v}\rehead{ }\normalsize\beginnumbering\briefempfaengerindex{Beer-Hofmann, Richard@\textsc{Beer-Hofmann, Richard}!zzzSchnitzler, Arthur@\emph{von Arthur Schnitzler}!1899-07-201@{20. 7. 1899}|(be} \toendnotes[C]{\smallbreak\pagebreak[2]} \Standort{YCGL, MSS 31.}
\physDesc{Briefkarte, Umschlag
\newline{}Handschrift: Bleistift, deutsche Kurrent\newline{}Versand: 1) Stempel: »\nobreak{}\oindex{Velden@\textbf{Velden}, \emph{Besiedelter Ort (A.BSO)}|pwk}Velden \textcolor{gray}{am}
                                       Wörthersee, 20 {[}7.{]} \textcolor{gray}{9}9 , 9N\nobreak{}«.  2) Stempel: »\nobreak{}\oindex{Seeboden@\textbf{Seeboden}, \emph{http://www.geonames.org/ontologyA.ADM3}|pwk}{\pb}Seeboden, 21. 7. {[}189{]}9\nobreak{}«. 
\newline{}Beer-Hofmann: eventuell vom Empfänger mit Bleistift am Umschlag datiert: »20. 7.« }\buchAbdrucke{\weitereDrucke{Arthur Schnitzler, Richard Beer-Hofmann: \emph{Briefwechsel 1891–1931}. Hg. Konstanze Fliedl. Wien, Zürich: \emph{Europaverlag} 1992, S. 133.} }\toendnotes[C]{\smallbreak}\pstart{}{\pb}\textsc{Dr. Rich. Beer-Hofmann}\pend{}\pstart{}\textsc{\textcolor{pink}{Seeboden}{}\ledrightnote{\textcolor{pink}{Seeboden}}}\pend{}\pstart{}\textsc{\textcolor{pink}{Villa Platzer}{}\ledrightnote{\textcolor{pink}{Villa Platzer}}}\pend{}\pstart{} am \textcolor{pink}{Millstätterſee}{}\ledrightnote{\textcolor{pink}{Millstätter See}}\pend{}{\bigskip}\pstart
           \raggedleft{}{\pb}20. 7. 99\pend
           \pstart
           lieber Richard, telegr. Sie mir jedenfalls einen Tag früher, bevor
               Sie ko{\geminationm}en. Bleiben Sie da{\geminationn}
               über Nacht hier? – Event. aviſiren Sie auch \textcolor{blue}{Robert
                  Hirſchfeld}{}\ledrightnote{\textcolor{blue}{Robert Hirschfeld}} (\textcolor{pink}{\textsc{Krumpendorf}}{}\ledrightnote{\textcolor{pink}{Krumpendorf am Wörthersee}}) wann Sie hier ſind? – An die \textcolor{pink}{Tauern}{}\ledrightnote{\textcolor{pink}{Hohe Tauern}} glaub
               ich nicht, ſind mir auch nicht ſehr ſympathiſch. Meinen Sie den Übergang vom \textcolor{pink}{Millſtätterſee}{}\ledrightnote{\textcolor{pink}{Millstätter See}}{ }\textsc{resp.}{ }\textcolor{pink}{Spital}{}\ledrightnote{\textcolor{pink}{Spittal an der Drau}} aus? – Ich habe andre Vorſchläge zu unterbreiten. We{\geminationn} ich nur ahnte, {\pb}ob wir
               1 oder 2 oder 14 Tage zuſa{\geminationm}en bleiben? – \pend
           \pstart
           \textcolor{blue}{Waſſerm.}{}\ledrightnote{\textcolor{blue}{Jakob Wassermann}} ko{\geminationm}t erſt
               heut Abend an. –\pend
           \pstart
           – Geſtern hab ich eine Radtour gemacht, \textcolor{pink}{Faakerſee}{}\ledrightnote{\textcolor{pink}{Faakersee}},
               mit \textcolor{blue}{Ihrer}{}\ledrightnote{→\textcolor{blue}{Marie Reinhard}}{ }\textcolor{blue}{Schweſter}{}\ledrightnote{→\textcolor{blue}{Caroline Burger}} und Ihrem \textcolor{blue}{Schwager}{}\ledrightnote{→\textcolor{blue}{Rudolf Burger}} – es war beinah ganz
               wie im \label{K_L00948_1v}\edtext{vorigen Jahre}{\lemma{\textnormal{\emph{vorigen Jahre}}}\Cendnote{\textnormal{Im vorigen Jahr war er mit \textcolor{blue}{Marie Reinhard} und ihrer Schwester \textcolor{blue}{Lola Burger} im Sommerurlaub. Siehe A. S.: \emph{Tagebuch}, 29. 7. 1898}}}\label{K_L00948_1h} – und –\pend
           \pstart
           – Es iſt vergeblich ein \label{K_L00948_2v}\edtext{Wort zu
                  ſuchen}{\lemma{\textnormal{\emph{Wort zu
                  ſuchen}}}\Cendnote{\textnormal{Er trauerte um \textcolor{blue}{Marie Reinhard}, die am 18. 3. 1899 verstorben
                  war.}}}\label{K_L00948_2h}.\pend
           \pstart
           Leben Sie wohl.{\\[\baselineskip]}Ihr \spacefill\mbox{Arthur.}\pend
           \leftskip=0em{}\endnumbering\briefempfaengerindex{Beer-Hofmann, Richard@\textsc{Beer-Hofmann, Richard}!zzzSchnitzler, Arthur@\emph{von Arthur Schnitzler}!1899-07-201@{20. 7. 1899}|)be}\mylabel{h}  \normalsize

\doendnotes{C}
\bigskip
\vfill

\clearpage

\footnotesize

\lohead{\textsc{register}}

% Definiere theindex-Environment komplett neu ohne reledmac
\makeatletter
\renewenvironment{theindex}{%
  \section*{\indexname}%
  \setlength{\parindent}{0pt}%
  \setlength{\parskip}{0pt plus 0.3pt}%
  \let\item\@idxitem
}{%
  \clearpage
}
\makeatother

\IfFileExists{\jobname-pw.ind}{\input{\jobname-pw.ind}}{}

\end{document}

      