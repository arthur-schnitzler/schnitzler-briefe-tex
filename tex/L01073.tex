%% latex-korrekturansicht-vorspann.tex
%% Vorspann für die Korrekturansicht.
%% Lädt die gemeinsame Datei latex-vorspann.tex mit gesetztem Schalter.

\newif\ifkorrekturansicht
\korrekturansichttrue

\input{../tex-inputs/latex-vorspann}


               \section[Richard Beer-Hofmann an Arthur Schnitzler, 14. 9. 1900]{ Richard Beer-Hofmann an Arthur Schnitzler, 14. 9. 1900}\nopagebreak\mylabel{v}\rehead{ }\normalsize\beginnumbering\briefempfaengerindex{Schnitzler, Arthur@\textsc{Schnitzler, Arthur}!zzzBeer-Hofmann, Richard@\emph{von Richard Beer-Hofmann}!1900-09-141@{14. 9. 1900}|(be} \toendnotes[C]{\smallbreak\pagebreak[2]} \Standort{CUL, Schnitzler, B 8.}
\physDesc{Telegramm
\newline{}maschinell\newline{}Versand: »\textcolor{gray}{\textbf{{[}Aufgenom{]}men durch}}{ }\textcolor{gray}{\textbf{\textit{/9 \textcolor{blue}{F. Spehar}}}}« 
\newline{}Schnitzler: mit Bleistift datiert: »14/9 90« \newline{}Ordnung: 1) beschnitten 2) mit Bleistift von unbekannter Hand nummeriert:
                              »159«}\buchAbdrucke{\weitereDrucke{Arthur Schnitzler, Richard Beer-Hofmann: \emph{Briefwechsel 1891–1931}. Hg. Konstanze Fliedl. Wien, Zürich: \emph{Europaverlag} 1992, S. 151.} }\toendnotes[C]{\smallbreak}\pstart
           \noindent{}{\pb}+ fr \textcolor{pink}{altauszee}{}\ledrightnote{\textcolor{pink}{Altaussee}} 478 30 14{ }7 15 m.–\pend
           \pstart
           komme hoffentlich heute vier uhr nachmittag an moechte dasz sye und \textcolor{blue}{paul}{}\ledrightnote{\textcolor{blue}{Paul Goldmann}} mich um halb sechs abholen.
               erfahre soeben die \label{K_L01073_1v}\edtext{\textcolor{blue}{mercier}{}\ledrightnote{\textcolor{blue}{Auguste Mercier}}tat des \textcolor{blue}{seehundes}{}\ledrightnote{→\textcolor{blue}{Paul Schlenther}}}{\lemma{\textnormal{\emph{merciertat des seehundes}}}\Cendnote{\textnormal{\textcolor{blue}{Paul Schlenther} hatte nach anfänglichen
                  Zusagen die Aufführung von \emph{\textcolor{green}{Der Schleier der
                     Beatrice}} doch abgelehnt. Am 14. 9. 1900 druckten mehrere
                  Zeitungen eine \emph{\textcolor{green}{Erklärung}} – ein heftiger Protest
                  von \textcolor{blue}{Hermann Bahr}, \textcolor{blue}{Julius Bauer}, \textcolor{blue}{Jakob Julius
                     David}, \textcolor{blue}{Robert Hirschfeld}, \textcolor{blue}{Felix Salten} und \textcolor{blue}{Ludwig Speidel} gegen die Vorgehensweise. \textcolor{blue}{Beer-Hofmann} stellt mit der Bezugnahme auf den Kriegsminister \textcolor{blue}{Auguste Mercier} eine Verbindung zum
                  antisemitisch motivierten \textcolor{blue}{Dreyfus}prozess
                  her.}}}\label{K_L01073_1h} herzlychst \spacefill\mbox{= richard .+}\pend
           \endnumbering\briefempfaengerindex{Schnitzler, Arthur@\textsc{Schnitzler, Arthur}!zzzBeer-Hofmann, Richard@\emph{von Richard Beer-Hofmann}!1900-09-141@{14. 9. 1900}|)be}\mylabel{h}  \normalsize

\doendnotes{C}
\bigskip
\vfill

\clearpage

\footnotesize

\lohead{\textsc{register}}

% Definiere theindex-Environment komplett neu ohne reledmac
\makeatletter
\renewenvironment{theindex}{%
  \section*{\indexname}%
  \setlength{\parindent}{0pt}%
  \setlength{\parskip}{0pt plus 0.3pt}%
  \let\item\@idxitem
}{%
  \clearpage
}
\makeatother

\IfFileExists{\jobname-pw.ind}{\input{\jobname-pw.ind}}{}

\end{document}

      