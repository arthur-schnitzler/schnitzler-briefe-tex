%% latex-korrekturansicht-vorspann.tex
%% Vorspann für die Korrekturansicht.
%% Lädt die gemeinsame Datei latex-vorspann.tex mit gesetztem Schalter.

\newif\ifkorrekturansicht
\korrekturansichttrue

\input{../tex-inputs/latex-vorspann}


               \section[Richard Beer-Hofmann an Arthur Schnitzler, {[}17. 9. 1904{]}]{ Richard Beer-Hofmann an Arthur Schnitzler,
               {[}17. 9. 1904{]}}\nopagebreak\mylabel{v}\rehead{ }\normalsize\beginnumbering\briefempfaengerindex{Schnitzler, Arthur@\textsc{Schnitzler, Arthur}!zzzBeer-Hofmann, Richard@\emph{von Richard Beer-Hofmann}!1904-09-171@{17. 9. 1904}|(be} \toendnotes[C]{\smallbreak\pagebreak[2]} \Standort{CUL, Schnitzler, B 8.}
\physDesc{Brief, 1 Blatt, 1 Seite
\newline{}Handschrift: Bleistift, lateinische Kurrent
\newline{}Schnitzler: mit Bleistift beschriftet: »\textcolor{pink}{Salzburg}{ }17/9 904« \newline{}Ordnung: mit Bleistift von unbekannter Hand nummeriert: »191« }\toendnotes[C]{\smallbreak}\pstart
           \noindent{}{\pb}Gehe zuerst Buchhandlung \textcolor{brown}{Höllriegl}{}\ledrightnote{\textcolor{brown}{Buchhandlung und Verlag Eduard Höllrigel}} (\textcolor{brown}{Kerber}{}\ledrightnote{\textcolor{brown}{Buchhandlung und Verlag Eduard Höllrigel}})
               beim \textcolor{pink}{Durchhaus}{}\ledrightnote{→\textcolor{pink}{Ritzerhaus}} auf den \textcolor{pink}{Marktplatz}{}\ledrightnote{\textcolor{pink}{Alter Markt}}, dann zu \label{K_L01447_1v}\edtext{\textcolor{brown}{Svatek}{}\ledrightnote{\textcolor{brown}{Wenzel Swatek}}}{\lemma{\textnormal{\emph{Svatek}}}\Cendnote{\textnormal{Sowohl bei \emph{\textcolor{brown}{Swatek}} wie \emph{\textcolor{brown}{Schwarz}} handelt es sich um Antiquitätenhändler.}}}\label{K_L01447_1h}
               (Parterre oder I Stock) dann zu \textcolor{brown}{Schwarz}{}\ledrightnote{\textcolor{brown}{Karl Schwarz Antiquitäten}}{ }\textcolor{pink}{Kaigasse}{}\ledrightnote{\textcolor{pink}{Kaigasse}}.\pend
           \pstart \spacefill\mbox{Richard}\pend{}\endnumbering\briefempfaengerindex{Schnitzler, Arthur@\textsc{Schnitzler, Arthur}!zzzBeer-Hofmann, Richard@\emph{von Richard Beer-Hofmann}!1904-09-171@{17. 9. 1904}|)be}\mylabel{h}  \normalsize

\doendnotes{C}
\bigskip
\vfill

\clearpage

\footnotesize

\lohead{\textsc{register}}

% Definiere theindex-Environment komplett neu ohne reledmac
\makeatletter
\renewenvironment{theindex}{%
  \section*{\indexname}%
  \setlength{\parindent}{0pt}%
  \setlength{\parskip}{0pt plus 0.3pt}%
  \let\item\@idxitem
}{%
  \clearpage
}
\makeatother

\IfFileExists{\jobname-pw.ind}{\input{\jobname-pw.ind}}{}

\end{document}

      