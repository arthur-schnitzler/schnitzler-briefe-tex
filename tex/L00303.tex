%% latex-korrekturansicht-vorspann.tex
%% Vorspann für die Korrekturansicht.
%% Lädt die gemeinsame Datei latex-vorspann.tex mit gesetztem Schalter.

\newif\ifkorrekturansicht
\korrekturansichttrue

\input{../tex-inputs/latex-vorspann}


               \section[Arthur Schnitzler an Richard Beer-Hofmann, 2. 3. 1894]{ Arthur Schnitzler an Richard Beer-Hofmann,
               2. 3. 1894}\nopagebreak\mylabel{v}\rehead{ }\normalsize\beginnumbering\briefempfaengerindex{Beer-Hofmann, Richard@\textsc{Beer-Hofmann, Richard}!zzzSchnitzler, Arthur@\emph{von Arthur Schnitzler}!1894-03-022@{2. 3. 1894}|(be} \toendnotes[C]{\smallbreak\pagebreak[2]} \Standort{YCGL, MSS 31.}
\physDesc{Postkarte
\newline{}Handschrift: Bleistift, deutsche Kurrent\newline{}Versand: 1) Stempel: »\nobreak{}\oindex{IX., Alsergrund@\textbf{IX., Alsergrund}, \emph{Bezirk (A.BZK)}|pwk}Wien 9/3, 2. 3. 94, 3–4 N\nobreak{}«.  2) Stempel: »\nobreak{}\oindex{Berlin@\textbf{Berlin}, \emph{https://www.geonames.org/ontologyP.PPLC}|pwk}Berlin, 3|8. 94, 3–3½N, Bestellt vom Postamte 64\nobreak{}«. }\pstart{}{\pb}\textsc{Herrn Dr. Richard
                     Beer Hofmann}\pend{}\pstart{}\textcolor{pink}{\textsc{Berlin}}{}\ledrightnote{\textcolor{pink}{Berlin}}\pend{}\pstart{}\textsc{\textcolor{pink}{Hotel
                        Westminster}{}\ledrightnote{\textcolor{pink}{Hotel Westminster}}}\pend{}{\bigskip}\pstart
           \noindent{}{\pb}Lieber Richard, ſollten Sie
                  \textcolor{green}{Anatol}{}\ledrightnote{\textcolor{green}{Anatol}} brauchen, ſo kaufen Sie gef. auf meine
               Koſten ein Exemplar; ich müßte das gebundene, das ich habe, als Paket aufgeben, was
               Umſtände macht. Auch ka{\geminationn} ich das ungebundene ſehr gut
               brauchen. Schade, daſs Sie nicht ſchreiben.\pend
           \pstart Herzl Ihr\spacefill\mbox{Arthur}\pend{}\endnumbering\briefempfaengerindex{Beer-Hofmann, Richard@\textsc{Beer-Hofmann, Richard}!zzzSchnitzler, Arthur@\emph{von Arthur Schnitzler}!1894-03-022@{2. 3. 1894}|)be}\mylabel{h}  \normalsize

\doendnotes{C}
\bigskip
\vfill

\clearpage

\footnotesize

\lohead{\textsc{register}}

% Definiere theindex-Environment komplett neu ohne reledmac
\makeatletter
\renewenvironment{theindex}{%
  \section*{\indexname}%
  \setlength{\parindent}{0pt}%
  \setlength{\parskip}{0pt plus 0.3pt}%
  \let\item\@idxitem
}{%
  \clearpage
}
\makeatother

\IfFileExists{\jobname-pw.ind}{\input{\jobname-pw.ind}}{}

\end{document}

      