%% latex-korrekturansicht-vorspann.tex
%% Vorspann für die Korrekturansicht.
%% Lädt die gemeinsame Datei latex-vorspann.tex mit gesetztem Schalter.

\newif\ifkorrekturansicht
\korrekturansichttrue

\input{../tex-inputs/latex-vorspann}


               \section[Friedrich M. Fels an Arthur Schnitzler, 19. 9. 1895]{ Friedrich M. Fels an Arthur Schnitzler, 19. 9. 1895}\nopagebreak\mylabel{v}\rehead{ }\normalsize\beginnumbering\briefempfaengerindex{Schnitzler, Arthur@\textsc{Schnitzler, Arthur}!zzzFels, Friedrich Michael@\emph{von Friedrich Michael Fels}!1895-09-192@{19. 9. 1895}|(be} \toendnotes[C]{\smallbreak\pagebreak[2]} \Standort{DLA, A:Schnitzler, HS.NZ85.1.2956.}
\physDesc{Brief, 1 Blatt, 4 Seiten
\newline{}Handschrift: schwarze Tinte, lateinische Kurrent
\newline{}Schnitzler: 1) mit Bleistift nummeriert: »24« 2) mit rotem
            Buntstift eine Unterstreichung}\toendnotes[C]{\smallbreak}\pstart
           \raggedleft{}{\pb}\textcolor{pink}{Zürich}{}\ledrightnote{\textcolor{pink}{Zürich}}, am 19. September 1895\pend
           \pstart\center{}Lieber Doktor Schnitzler!\pend\pstart
           Verzeihen Sie, daſs ich Ihnen auf Ihren \textcolor{pink}{Ischl}{}\ledrightnote{\textcolor{pink}{Bad Ischl}}er
                    Brief erst heute antworte. Ich hätte Ihnen gern Gutes von mir berichtet, doch es
                    ist mir unmöglich. Es will scheinen, als ob ich gar nie zur Ruhe ko{\geminationm}en kö{\geminationn}e. Die
                    hiesigen Zeitungsverhältniſse sind traurig, sehr traurig, und es ist
                    unglaublich, wie viel Mühe es kostet, etwas unterzubringen. Fast so viel oder
                    vielleicht mehr als in \substVorne{}\textsuperscript{\textcolor{pink}{Zürich}{}\ledrightnote{\textcolor{pink}{Zürich}}}{\allowbreak}\substDazwischen{}\textcolor{pink}{Wien}{}\ledrightnote{\textcolor{pink}{Wien}}\substHinten{}. Die \textcolor{green}{Neue Zürcher Zeitung}{}\ledrightnote{\textcolor{green}{Neue Zürcher Zeitung}} hat ein \textcolor{green}{Doppelfeuilleton}{}\ledrightnote{→\textcolor{green}{Bulgarische Volksdichtungen}} von mir
                    gedruckt und mir auf einen zweiten Artikel einen Vorschuſs von 50 francs
                    gewährt; jetzt allerdings hat sie eine größere Bestellung bei mir gemacht, eine
                    Reihe von Aufsätzen, jeder 500–600 Druckzeilen, in denen ich die Entwickelung
                    der modernen deutschen Literatur darlegen soll. Das Honorar freilich ist
                    schlecht genug: pro Druckzeile 8 cent 4 Kr. Andere Blätter zahlen bloß 5 cent.
                    So habe ich einen ganzen Monat Theaterreferate geschrieben und am Ende 10 francs
                    eingeheimst – hübsch, na?!\pend
           \pstart
           {\pb}Gegenwärtig bin ich von einer neuen Kalamität
                    heimgesucht worden. Ich bin nämlich zur Abwechslung von meiner \textcolor{pink}{Schweiz}{}\ledrightnote{\textcolor{pink}{Schweiz}}er \textcolor{blue}{Wirtin}{}\ledrightnote{→\textcolor{blue}{?? [Vermieterin von F. M. Fels in Zürich]}} (– weil ich ihr die Miete 5 Tage, nachdem sie fällig war, noch
                    nicht entrichten ko{\geminationn}te –) unter Zurückbehaltung
                    meiner Sachen auf die Straſse gesetzt worden, und hause nun wieder so bei Beka{\geminationn}ten. Ich bin Ihnen, so dreckig mir’s auch ging,
                    in diesen letzten 3 Monaten gewiſs nicht mit Bitten zur Last gefallen; ich habe
                    gedacht, überhaupt nicht mehr in eine solche Lage ko{\geminationm}en zu kö{\geminationn}en. Nun ist es doch eingetreten, und ich
                    muſs wieder an Ihre Güte und Freundschaft appellieren. Wären Sie imstande,
                        zusa{\geminationm}en mit andern mir noch einmal 25 fl zu
                    senden; seien Sie überzeugt, ich würde mich nicht an Sie wenden, we{\geminationn} ich irgend einen Ausweg wüſste. Die Beka{\geminationn}ten, die ich hier habe, sind alle entweder selbst
                    vollständig auf dem Hund, oder sie sind z.Zt. in Ferien. We{\geminationn} es in Ihrer Macht steht, meine Bitte zu
                    erfüllen, wollen Sie freundlichst einen reko{\geminationm}andierten Brief senden an\pend
           \pstart
           \centering{}\uline{Dr. Friedr. M. Fels}\pend
           \pstart
           \noindent{}\centering{}\uline{per Adreſse Herrn \textcolor{blue}{Hugo
                            Bettauer}{}\ledrightnote{\textcolor{blue}{Hugo Bettauer}}}\pend
           \pstart
           \noindent{}\raggedleft{}\uline{\textcolor{pink}{Zürich I, Rämistraſse 2}{}\ledrightnote{\textcolor{pink}{Rämistrasse}}}\pend
           \pstart
           \noindent{}{\pb}Sie haben wohl \textcolor{blue}{J. H. Mackay}{}\ledrightnote{\textcolor{blue}{John Henry Mackay}}{ }ſchon gesprochen. Er ist vor ein paar Tagen
                    nach \textcolor{pink}{Wien}{}\ledrightnote{\textcolor{pink}{Wien}} abgereist, um dort eine Woche zu
                    verweilen, und ich habe ihm viele, viele Grüſse an Sie aufgetragen. \textcolor{blue}{Pollandt}{}\ledrightnote{\textcolor{blue}{Max Pollandt}} wird diesen Winter ans hiesige \textcolor{pink}{Stadttheater}{}\ledrightnote{\textcolor{pink}{Stadttheater}} ko{\geminationm}en, dürfte wohl auch schon hier sein; doch hab ich ihn noch nicht gesehen. Am
                        \textcolor{pink}{Volkstheater}{}\ledrightnote{\textcolor{pink}{Volkstheater (Zürich)}}{ }ſind auch \textcolor{pink}{Wien}{}\ledrightnote{\textcolor{pink}{Wien}}er: die \textcolor{blue}{Je{\geminationn}y Neuhut}{}\ledrightnote{\textcolor{blue}{Jenny Neuhut}}, die Sie wohl noch aus dem \textcolor{pink}{Griensteidl}{}\ledrightnote{\textcolor{pink}{Café Griensteidl}} ke{\geminationn}en (\textcolor{blue}{Salten}{}\ledrightnote{\textcolor{blue}{Felix Salten}} ke{\geminationn}t sie jedenfalls)
                    und ein Frl. \textcolor{blue}{Josephine Sorger}{}\ledrightnote{\textcolor{blue}{Josefine Sorger}}, ein ganz
                    allerliebster Käfer.\pend
           \pstart
           Haben Sie in \textcolor{pink}{Wien}{}\ledrightnote{\textcolor{pink}{Wien}} auch so abscheuliches Wetter
                    gehabt? Hier hatten wir 5 Wochen keinen Regen und im Schatten 37°, in der So{\geminationn}e 47° Celsius. Es war zum aus der Haut fahren.
                    Gottlob, es ists etwas kühler.\pend
           \pstart
           Was Sie vielleicht intereſsieren wird, ich werde jetzt anfangen, Stunden zu
                    geben: Literaturgeschischte u. dgl. In ein paar Tagen werde ich meine ersten
                        Schüleri{\geminationn}en erhalten: 2 \textcolor{blue}{Amerikaneri{\geminationn}en}{}\ledrightnote{→\textcolor{blue}{?? [Amerikanische Studentin in Zürich 1]}{\newline}→\textcolor{blue}{?? [Amerikanische Studentin in Zürich 2]}}, denen ich Deutsch beibringen soll,
                    damit sie den Vorlesungen beſser folgen kö{\geminationn}en.\pend
           \pstart
           {\pb}Ihre \textcolor{green}{Novelle in Briefen}{}\ledrightnote{→\textcolor{green}{Die kleine Komödie}} in der \textcolor{green}{N. D. R.}{}\ledrightnote{\textcolor{green}{Neue Deutsche Rundschau}} habe ich gelesen. Sie ist sehr hübsch, aber – Sie
                    verzeihen mir – meines Erachtens auch nicht mehr. Illustrationen kö{\geminationn}en ihr nicht schaden.\pend
           \pstart
           Also leben Sie wohl! verzeihen Sie meine Bitte und erfüllen Sie sie, falls Sie
                        kö{\geminationn}en! und auf jedenfall laſsen Sie wieder
                    einmal etwas von Sich hören! \textcolor{blue}{Beer-Hofma{\geminationn}}{}\ledrightnote{\textcolor{blue}{Richard Beer-Hofmann}}, \textcolor{blue}{Hofma{\geminationn}sthal}{}\ledrightnote{\textcolor{blue}{Hugo von Hofmannsthal}}, \textcolor{blue}{Salten}{}\ledrightnote{\textcolor{blue}{Felix Salten}} etc. bitte ich zu
                    grüſsen; vor allen aber seien Sie gegrüſst \pend
           \pstart
           von{\\[\baselineskip]}Ihrem{\\[\baselineskip]}dankbar ergebenen{\\[\baselineskip]}\spacefill\mbox{Fels}\pend
           \leftskip=0em{}\endnumbering\briefempfaengerindex{Schnitzler, Arthur@\textsc{Schnitzler, Arthur}!zzzFels, Friedrich Michael@\emph{von Friedrich Michael Fels}!1895-09-192@{19. 9. 1895}|)be}\mylabel{h}  \normalsize

\doendnotes{C}
\bigskip
\vfill

\clearpage

\footnotesize

\lohead{\textsc{register}}

% Definiere theindex-Environment komplett neu ohne reledmac
\makeatletter
\renewenvironment{theindex}{%
  \section*{\indexname}%
  \setlength{\parindent}{0pt}%
  \setlength{\parskip}{0pt plus 0.3pt}%
  \let\item\@idxitem
}{%
  \clearpage
}
\makeatother

\IfFileExists{\jobname-pw.ind}{\input{\jobname-pw.ind}}{}

\end{document}

      