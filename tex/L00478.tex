%% latex-korrekturansicht-vorspann.tex
%% Vorspann für die Korrekturansicht.
%% Lädt die gemeinsame Datei latex-vorspann.tex mit gesetztem Schalter.

\newif\ifkorrekturansicht
\korrekturansichttrue

\input{../tex-inputs/latex-vorspann}


               \section[Arthur Schnitzler an Richard Beer-Hofmann, 27. 8. 1895]{ Arthur Schnitzler an Richard Beer-Hofmann, 27. 8. 1895}\nopagebreak\mylabel{v}\rehead{ }\normalsize\beginnumbering\briefempfaengerindex{Beer-Hofmann, Richard@\textsc{Beer-Hofmann, Richard}!zzzSchnitzler, Arthur@\emph{von Arthur Schnitzler}!1895-08-271@{27. 8. 1895}|(be} \toendnotes[C]{\smallbreak\pagebreak[2]} \Standort{YCGL, MSS 31.}
\physDesc{Telegramm
\newline{}Handschrift einer Schreibkraft: blaue Tinte, deutsche Kurrent\newline{}Versand: »\noindent{}\textcolor{gray}{\textbf{Aufgegeben am {\dots} 18{\dots} um}}{ }4 \textcolor{gray}{\textbf{Uhr}}{ }45 \textcolor{gray}{\textbf{Min.}} N\textcolor{gray}{\textbf{Mittag}}{ / }\textcolor{gray}{\textbf{Eingelangt von}} S \textcolor{gray}{\textbf{auf Leitung Nr.}} 1050 \textcolor{gray}{\textbf{am}}{ }27/8\textcolor{gray}{\textbf{189}}5{ }\textcolor{gray}{\textbf{um}}{ }5 \textcolor{gray}{\textbf{Uhr}} 50 \textcolor{gray}{\textbf{Min. {\dots} Mittag}}{ / }\textcolor{gray}{\textbf{Aufgenommen durch}}{ }\textcolor{gray}{JF.}{ / }\textcolor{gray}{\textbf{Von}}{ }\textcolor{pink}{München}{ }\textcolor{gray}{\textbf{mit}} 7.232p{ }\textcolor{gray}{\textbf{Taxworten (}}17 \textcolor{gray}{\textbf{Worten {\dots} Chiffern)}}« }\toendnotes[C]{\smallbreak}\pstart{}{\pb}Richard Beer\pend{}\pstart{}Hoffmann\pend{}\pstart{}\textcolor{pink}{Egelmos 22}{}\ledrightnote{\textcolor{pink}{Eglmoosgasse}}\pend{}\pstart{}\textcolor{pink}{\textcolor{gray}{\textbf{\textit{Ischl}}}}{}\ledrightnote{\textcolor{pink}{Bad Ischl}}\pend{}{\bigskip}\pstart
           \noindent{}{\pb}Wohne ſchön \textcolor{pink}{Hotel
                  Continental}{}\ledrightnote{\textcolor{pink}{Hotel Continental}}{ }\label{K_L00478_1v}\edtext{ſitze beſorgt}{\lemma{\textnormal{\emph{ſitze beſorgt}}}\Cendnote{\textnormal{Möglicherweise ist dieses Telegramm der
                  Ursprung eines beliebten Witzes, den Zeitungen mehrfach abdrucken und der zumeist
                     \textcolor{blue}{Hofmannsthal} und \textcolor{blue}{Schnitzler} als Protagonisten hat: »In \textcolor{pink}{Wien}er Literatenkreisen wird über folgende angeblich wahre
                     Geschichte herzlich gelacht: \textcolor{blue}{Artur
                        Schnitzler} ersuchte in \textcolor{pink}{Aussee}{ }seinen Freund \textcolor{blue}{Hugo Hoffmannsthal}, er möge ihm, wenn er nach \textcolor{pink}{Salzburg} fahre, Karten für die \textcolor{green}{Jedermann}-Aufführung besorgen. Nach einigen Wochen, als \textcolor{blue}{Schnitzler} längst diese Bitte vergessen
                     hatte, erhielt er aus \textcolor{pink}{Salzburg} folgendes
                     Telegramm: \so{Sitze besorgt }\textcolor{pink}{\so{Hotel Europe}}. \textcolor{blue}{Hoffmannsthal}. Worauf \textcolor{blue}{Schnitzler} bestürzt zurückdrahtete: \so{Warum sitzt du besorgt im }\textcolor{pink}{\so{Hotel Europe}}\so{? }\textcolor{blue}{\so{Schnitzler}}\so{.}« (\emph{\textcolor{green}{Der Morgen}}, Jg. 12, Nr. 42,
                        17. 10. 1921, S. 8.) Vgl. Arthur Schnitzler an Richard Beer-Hofmann, 5. 8. 1912, 28. 7. 1922}}}\label{K_L00478_1h}{ }\textcolor{blue}{Paul}{}\ledrightnote{\textcolor{blue}{Paul Goldmann}} kommt morgen herzlichſt\pend
           \pstart \spacefill\mbox{Arthur}\pend{}\endnumbering\briefempfaengerindex{Beer-Hofmann, Richard@\textsc{Beer-Hofmann, Richard}!zzzSchnitzler, Arthur@\emph{von Arthur Schnitzler}!1895-08-271@{27. 8. 1895}|)be}\mylabel{h}  \normalsize

\doendnotes{C}
\bigskip
\vfill

\clearpage

\footnotesize

\lohead{\textsc{register}}

% Definiere theindex-Environment komplett neu ohne reledmac
\makeatletter
\renewenvironment{theindex}{%
  \section*{\indexname}%
  \setlength{\parindent}{0pt}%
  \setlength{\parskip}{0pt plus 0.3pt}%
  \let\item\@idxitem
}{%
  \clearpage
}
\makeatother

\IfFileExists{\jobname-pw.ind}{\input{\jobname-pw.ind}}{}

\end{document}

      