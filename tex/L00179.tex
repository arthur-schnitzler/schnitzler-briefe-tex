%% latex-korrekturansicht-vorspann.tex
%% Vorspann für die Korrekturansicht.
%% Lädt die gemeinsame Datei latex-vorspann.tex mit gesetztem Schalter.

\newif\ifkorrekturansicht
\korrekturansichttrue

\input{../tex-inputs/latex-vorspann}


               \section[Hugo von Hofmannsthal an Arthur Schnitzler, {[}18. 2. 1893{]}]{ Hugo von Hofmannsthal an Arthur Schnitzler, {[}18. 2. 1893{]}}\nopagebreak\mylabel{v}\rehead{ }\normalsize\beginnumbering\briefempfaengerindex{Schnitzler, Arthur@\textsc{Schnitzler, Arthur}!zzzHofmannsthal, Hugo von@\emph{von Hugo von Hofmannsthal}!1893-02-182@{{[}18. 2. 1893{]}}|(be} \toendnotes[C]{\smallbreak\pagebreak[2]} \Standort{CUL, Schnitzler, B 43.}
\physDesc{Brief, 1 Blatt (Briefpapier mit aufgeprägtem Wappen), 3 Seiten
\newline{}Handschrift: schwarze Tinte, deutsche Kurrent
\newline{}Schnitzler: mit Bleistift nummeriert: »38« }\buchAbdrucke{\weitereDrucke{1) Hugo von Hofmannsthal, Arthur Schnitzler: \emph{Briefwechsel}. Hg. Therese Nickl und Heinrich Schnitzler. Frankfurt am Main: \emph{S. Fischer} 1964, S. 48.} \weitereDrucke{2) Hermann Bahr, Arthur Schnitzler: \emph{Briefwechsel, Aufzeichnungen, Dokumente
                                (1891–1931)}. Hg. Kurt Ifkovits und Martin Anton Müller. Göttingen: \emph{Wallstein} 2018, S. 33.} }\pstart
           \raggedleft{}{\pb}Samstag abends.\pend
           \pstart\center{}Lieber Arthur.\pend\pstart
           Ich komme möglicherweiſe nach einer Geſellschaft ins \textcolor{pink}{Central}{}\ledrightnote{\textcolor{pink}{Café Central}}, antworte aber lieber ſo. Der Brief von \textcolor{blue}{Fels}{}\ledrightnote{\textcolor{blue}{Friedrich Michael Fels}} hat mich ſehr ſchmerzlich berührt. Er muſs
                    jedenfalls unten erhalten werden; ich werde ihm dazu auch ſelbſt ſchriftlich
                    zureden und hoffe Ihnen in den allernächſten Tagen wenigſtens circa 25 fl
                    ſchicken zu können. \textcolor{blue}{Bahr}{}\ledrightnote{\textcolor{blue}{Hermann Bahr}} ist momentan in \textcolor{pink}{Berlin}{}\ledrightnote{\textcolor{pink}{Berlin}}, {\pb}er kommt \substVorne{}\textsuperscript{Dienstag}{\allowbreak}\substDazwischen{}Montag\substHinten{} abends zurück; Dienstag, ſpäteſtens Mittwoch werde ich ernſthaft und
                    endgiltig mit ihm reden. Er hat allen möglichen guten Willen, nur nicht die
                    Energie, um die mit ſolchen Dingen verbundenen ekelhaften kleinlichen Anſtände
                    zu überwinden. Er muſs ſie aber eben haben. Also \uline{ich} für meinen Theil fürchte mich vor gar nichts als vor der muthloſen
                        {\pb}Traurigkeit des \textcolor{blue}{Fels}{}\ledrightnote{\textcolor{blue}{Friedrich Michael Fels}}, die ja hoffentlich vor guter Luft und
                    Ernährung weichen wird. Das übrige wird ſich und werden wir finden.\pend
           \pstart
           Herzlichſt{\\[\baselineskip]}\spacefill\mbox{Loris}\pend
           \leftskip=0em{}\endnumbering\briefempfaengerindex{Schnitzler, Arthur@\textsc{Schnitzler, Arthur}!zzzHofmannsthal, Hugo von@\emph{von Hugo von Hofmannsthal}!1893-02-182@{{[}18. 2. 1893{]}}|)be}\mylabel{h}  \normalsize

\doendnotes{C}
\bigskip
\vfill

\clearpage

\footnotesize

\lohead{\textsc{register}}

% Definiere theindex-Environment komplett neu ohne reledmac
\makeatletter
\renewenvironment{theindex}{%
  \section*{\indexname}%
  \setlength{\parindent}{0pt}%
  \setlength{\parskip}{0pt plus 0.3pt}%
  \let\item\@idxitem
}{%
  \clearpage
}
\makeatother

\IfFileExists{\jobname-pw.ind}{\input{\jobname-pw.ind}}{}

\end{document}

      