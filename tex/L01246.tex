%% latex-korrekturansicht-vorspann.tex
%% Vorspann für die Korrekturansicht.
%% Lädt die gemeinsame Datei latex-vorspann.tex mit gesetztem Schalter.

\newif\ifkorrekturansicht
\korrekturansichttrue

\input{../tex-inputs/latex-vorspann}


               \section[Arthur Schnitzler an Richard Beer-Hofmann, 29. 10. 1902]{ Arthur Schnitzler an Richard Beer-Hofmann, 29. 10. 1902}\nopagebreak\mylabel{v}\rehead{ }\normalsize\beginnumbering\briefempfaengerindex{Beer-Hofmann, Richard@\textsc{Beer-Hofmann, Richard}!zzzSchnitzler, Arthur@\emph{von Arthur Schnitzler}!1902-10-291@{29. 10. 1902}|(be} \toendnotes[C]{\smallbreak\pagebreak[2]} \Standort{YCGL, MSS 31.}
\physDesc{Brief, 1 Blatt, 2 Seiten, Umschlag
\newline{}Handschrift: Bleistift, deutsche Kurrent\newline{}Versand: 1) Stempel: »\nobreak{}\oindex{I., Innere Stadt@\textbf{I., Innere Stadt}, \emph{Bezirk (A.BZK)}|pwk}\textcolor{gray}{Wie}n 1/1, \textcolor{gray}{29}. 10. 02, 12–1N\nobreak{}«.  2) Stempel: »\nobreak{}\oindex{Rodaun@\textbf{Rodaun}, \emph{Teil eines besiedelten Ortes (A.BSOX)}|pwk}{\pb}Rodaun, 30. \textcolor{gray}{10. 02}, 7–9V\nobreak{}«. }\buchAbdrucke{\weitereDrucke{Arthur Schnitzler, Richard Beer-Hofmann: \emph{Briefwechsel 1891–1931}. Hg. Konstanze Fliedl. Wien, Zürich: \emph{Europaverlag} 1992, S. 158.} }\pstart{}{\pb}Hrn \textsc{Dr Richard Beer-Hofmann}\pend{}\pstart{}\textcolor{pink}{\textsc{Rodaun}}{}\ledrightnote{\textcolor{pink}{Rodaun}}\pend{}\pstart{}\textsc{\textcolor{pink}{Liesinger Straße 2}{}\ledrightnote{\textcolor{pink}{Liesingerstraße}}.}\pend{}{\bigskip}\pstart
           \raggedleft{}{\pb}29. X. 902\pend
           \pstart
           lieber Richard, Sie ſind beinah täglich in \textcolor{pink}{Wien}{}\ledrightnote{\textcolor{pink}{Wien}} und – von andern Möglichkeiten ganz abgeſehen – ſagen Sie
               einem nicht einmal wo man Sie gelegentlich auf eine {\pb}Viertelſtunde ſehen kann. Es iſt wirklich nicht beſonders ſchön von Ihnen.\pend
           \pstart
           Herzlichſt Ihr{\\[\baselineskip]}\spacefill\mbox{A.}\pend
           \leftskip=0em{}\endnumbering\briefempfaengerindex{Beer-Hofmann, Richard@\textsc{Beer-Hofmann, Richard}!zzzSchnitzler, Arthur@\emph{von Arthur Schnitzler}!1902-10-291@{29. 10. 1902}|)be}\mylabel{h}  \normalsize

\doendnotes{C}
\bigskip
\vfill

\clearpage

\footnotesize

\lohead{\textsc{register}}

% Definiere theindex-Environment komplett neu ohne reledmac
\makeatletter
\renewenvironment{theindex}{%
  \section*{\indexname}%
  \setlength{\parindent}{0pt}%
  \setlength{\parskip}{0pt plus 0.3pt}%
  \let\item\@idxitem
}{%
  \clearpage
}
\makeatother

\IfFileExists{\jobname-pw.ind}{\input{\jobname-pw.ind}}{}

\end{document}

      