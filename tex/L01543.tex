%% latex-korrekturansicht-vorspann.tex
%% Vorspann für die Korrekturansicht.
%% Lädt die gemeinsame Datei latex-vorspann.tex mit gesetztem Schalter.

\newif\ifkorrekturansicht
\korrekturansichttrue

\input{../tex-inputs/latex-vorspann}


               \section[Hermann Bahr an Arthur Schnitzler, {[}18.?{]} 8. 1905]{ Hermann Bahr an Arthur Schnitzler, {[}18.?{]} 8. 1905}\nopagebreak\mylabel{v}\rehead{ }\normalsize\beginnumbering\briefempfaengerindex{Schnitzler, Arthur@\textsc{Schnitzler, Arthur}!zzzBahr, Hermann@\emph{von Hermann Bahr}!1905-08-181@{{[}18.?{]} 8. 1905}|(be} \toendnotes[C]{\smallbreak\pagebreak[2]} \Standort{CUL, Schnitzler, B 5b.}
\physDesc{Bildpostkarte
\newline{}Handschrift: Bleistift, deutsche Kurrent\newline{}Versand: Stempel: »\nobreak{}\oindex{Salzburg@\textbf{Salzburg}, \emph{Besiedelter Ort (A.BSO)}|pwk}Salzburg, \textcolor{gray}{18}. VIII. 05\nobreak{}«.  \newline{}Ordnung: mit Bleistift von unbekannter Hand nummeriert:
                                    »130« }\buchAbdrucke{\weitereDrucke{Hermann Bahr, Arthur Schnitzler: \emph{Briefwechsel, Aufzeichnungen, Dokumente (1891–1931)}. Hg. Kurt Ifkovits und Martin Anton Müller. Göttingen: \emph{Wallstein} 2018, S. 350.} }\toendnotes[C]{\smallbreak}\pstart{}{\pb}\textsc{Artur Schnitzler}\pend{}\pstart{}\textcolor{pink}{\textsc{Wien XVIII}}{}\ledrightnote{\textcolor{pink}{XVIII., Währing}}\pend{}\pstart{}\textcolor{pink}{\textsc{Spöttelgasse 7}}{}\ledrightnote{\textcolor{pink}{Edmund-Weiß-Gasse}}\pend{}{\bigskip}\pstart
           \noindent{}\centering{}\textcolor{gray}{\textbf{{\pb}Das \textcolor{blue}{Weiser}{}\ledrightnote{→\textcolor{blue}{Ignatz Anton von Weiser}}haus am alten Markt in \textcolor{pink}{Salzburg}{}\ledrightnote{\textcolor{pink}{Salzburg}} um 1800 (jetzt Salzburger Sparkasse,
                        \textcolor{pink}{Ludwig-Viktorplatz}{}\ledrightnote{\textcolor{pink}{Alter Markt}})}}\pend
           \pstart
           \label{K_L01543_1v}\edtext{\textcolor{blue}{Vanjung}{}\ledrightnote{\textcolor{blue}{Leo Van-Jung}} erzält mir eben}{\lemma{\textnormal{\emph{Vanjung erzält mir eben}}}\Cendnote{\textnormal{Am 12. 8. 1905 hatte \textcolor{blue}{Van
                     Jung} bei \textcolor{blue}{Schnitzler}{ }\emph{\textcolor{green}{Zwischenspiel}} und \emph{\textcolor{green}{Ruf des Lebens}} vorgelesen bekommen. Vom 18. bis zum
                     20. 8. 1905 war \textcolor{blue}{Bahr} in \textcolor{pink}{Salzburg} (Bahr: \emph{Tagebücher, Skizzenhefte, Notizbücher}
                     IV,424).}}}\label{K_L01543_1h} von Deinen beiden \textcolor{green}{Stücken}{}\ledrightnote{→\textcolor{green}{Zwischenspiel. Komödie in drei Akten}{\newline}→\textcolor{green}{Der Ruf des Lebens. Schauspiel in drei Akten}}, ich freu mich ſehr und bin ungeheuer
                  neugierig.\hspace*{1.5em}Herzlichſt \spacefill\mbox{Hermann}\pend
           \endnumbering\briefempfaengerindex{Schnitzler, Arthur@\textsc{Schnitzler, Arthur}!zzzBahr, Hermann@\emph{von Hermann Bahr}!1905-08-181@{{[}18.?{]} 8. 1905}|)be}\mylabel{h}  \normalsize

\doendnotes{C}
\bigskip
\vfill

\clearpage

\footnotesize

\lohead{\textsc{register}}

% Definiere theindex-Environment komplett neu ohne reledmac
\makeatletter
\renewenvironment{theindex}{%
  \section*{\indexname}%
  \setlength{\parindent}{0pt}%
  \setlength{\parskip}{0pt plus 0.3pt}%
  \let\item\@idxitem
}{%
  \clearpage
}
\makeatother

\IfFileExists{\jobname-pw.ind}{\input{\jobname-pw.ind}}{}

\end{document}

      