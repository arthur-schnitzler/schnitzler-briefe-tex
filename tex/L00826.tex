%% latex-korrekturansicht-vorspann.tex
%% Vorspann für die Korrekturansicht.
%% Lädt die gemeinsame Datei latex-vorspann.tex mit gesetztem Schalter.

\newif\ifkorrekturansicht
\korrekturansichttrue

\input{../tex-inputs/latex-vorspann}


               \section[Arthur Schnitzler an Hugo von Hofmannsthal, 23. 7. 1898]{ Arthur Schnitzler an Hugo von Hofmannsthal, 23. 7. 1898}\nopagebreak\mylabel{v}\rehead{ }\normalsize\beginnumbering\briefempfaengerindex{Hofmannsthal, Hugo von@\textsc{Hofmannsthal, Hugo von}!zzzSchnitzler, Arthur@\emph{von Arthur Schnitzler}!1898-07-231@{23. 7. 1898}|(be} \toendnotes[C]{\smallbreak\pagebreak[2]} \Standort{FDH, Hs-30885,72.}
\physDesc{Brief, 1 Blatt, 4 Seiten
\newline{}Handschrift: Bleistift, deutsche Kurrent}\buchAbdrucke{\weitereDrucke{Hugo von Hofmannsthal, Arthur Schnitzler: \emph{Briefwechsel}. Hg. Therese Nickl und Heinrich Schnitzler. Frankfurt am Main: \emph{S. Fischer} 1964, S. 107.} }\toendnotes[C]{\smallbreak}\pstart
           \raggedleft{}{\pb}\textcolor{pink}{\textsc{Bad Gastein}}{}\ledrightnote{\textcolor{pink}{Bad Gastein}}{ }23. 7. 98\pend
           \pstart
           Mein lieber Hugo, ich riskir noch ein paar Zeilen nach \textcolor{pink}{\textsc{Czortków}}{}\ledrightnote{\textcolor{pink}{Tschortkiw}} – Sie wiſſen ſchon, dſs ich bei Ihren \textcolor{blue}{Eltern}{}\ledrightnote{→\textcolor{blue}{Hugo August von Hofmannsthal}{\newline}→\textcolor{blue}{Anna von Hofmannsthal}} war, die von viel
                    Herzlichkeit gegen mich waren. Ich hab mich ſehr gefreut. Die \textcolor{blue}{Sp. Mädeln}{}\ledrightnote{\textcolor{blue}{Paula Schmidl}{\newline}\textcolor{blue}{Julie Wassermann}{\newline}\textcolor{blue}{Agnes Ulmann}{\newline}\textcolor{blue}{Emilie Sgal}{\newline}\textcolor{blue}{Dora Michaelis}{\newline}\textcolor{blue}{Sophie Knepler}}
                    haben mich herumgeführt und \introOben{}mir\introOben{} die Stätten gezeigt, wo
                    Sie gedichtet haben – es war nur wenig Zeit, die \textcolor{pink}{\textsc{Weil{\pb}guni}}{}\ledrightnote{\textcolor{pink}{Hotel Weilguni}}ſche \textsc{table d’hôte} drohte – und ſo kam eine
                    rührende Haſt über die Geſchöpfe. Es iſt was hübſches um dieſe kleinen
                    Unſterblichkeiten – über die großen werden wir nicht ſo gemütlich plaudern
                    können; fürcht ich; es wird zu ſpät ſein. –\pend
           \pstart
           Herrliches Wetter hab ich überall; hier ganz beſonders. Montag fahr
                    ich nach \textcolor{pink}{Salzburg}{}\ledrightnote{\textcolor{pink}{Salzburg}}. Warten Sie {\pb}jedenfalls eine neue Nachricht ab, bevor Sie mir
                    ſchreiben. Auf \textcolor{blue}{Richard}{}\ledrightnote{\textcolor{blue}{Richard Beer-Hofmann}}{ }ſcheints werden wir verzichten müſſen – doch
                        \uline{Sie}{ }\introOben{}allein\introOben{} werden ihn ſpäter haben, geht aus einem eiligen
                    Brief von ihm hervor. –\pend
           \pstart
           Gearbeitet hab ich nichts; doch iſt trotz allem, was bedrückt, eine gewiſſe Fülle
                    in mir, ja ſogar die Neigung dieſer Fülle, ſich zu {\pb}ordnen.\pend
           \pstart
           Ich hoffe Sie kö{\geminationn}en mir bald ſagen, wie es Ihnen
                        \introOben{}oder vielmehr\introOben{} daſs es Ihnen beſſer geht. Was werden
                    Sie ſchreiben. In mir iſt der Streit zwiſchen dem \textcolor{green}{Stück}{}\ledrightnote{→\textcolor{green}{Der Schleier der Beatrice. Schauspiel in fünf Akten}} und dem \textcolor{green}{Roman}{}\ledrightnote{→\textcolor{green}{Der Weg ins Freie. Roman}} noch nicht entſchieden.\pend
           \pstart
           Leben Sie wohl – ich ſende den Brief doch lieber nach \textcolor{pink}{Mödling}{}\ledrightnote{\textcolor{pink}{Mödling}}; möge er Sie heiter u. herzlich begrüßen.\pend
           \pstart Ihr \spacefill\mbox{Arthur.}\pend{}\endnumbering\briefempfaengerindex{Hofmannsthal, Hugo von@\textsc{Hofmannsthal, Hugo von}!zzzSchnitzler, Arthur@\emph{von Arthur Schnitzler}!1898-07-231@{23. 7. 1898}|)be}\mylabel{h}  \normalsize

\doendnotes{C}
\bigskip
\vfill

\clearpage

\footnotesize

\lohead{\textsc{register}}

% Definiere theindex-Environment komplett neu ohne reledmac
\makeatletter
\renewenvironment{theindex}{%
  \section*{\indexname}%
  \setlength{\parindent}{0pt}%
  \setlength{\parskip}{0pt plus 0.3pt}%
  \let\item\@idxitem
}{%
  \clearpage
}
\makeatother

\IfFileExists{\jobname-pw.ind}{\input{\jobname-pw.ind}}{}

\end{document}

      