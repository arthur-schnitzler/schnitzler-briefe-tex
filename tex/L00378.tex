%% latex-korrekturansicht-vorspann.tex
%% Vorspann für die Korrekturansicht.
%% Lädt die gemeinsame Datei latex-vorspann.tex mit gesetztem Schalter.

\newif\ifkorrekturansicht
\korrekturansichttrue

\input{../tex-inputs/latex-vorspann}


               \section[Richard Beer-Hofmann an Arthur Schnitzler, 7. 10. 1894]{ Richard Beer-Hofmann an Arthur Schnitzler, 7. 10. 1894}\nopagebreak\mylabel{v}\rehead{ }\normalsize\beginnumbering\briefempfaengerindex{Schnitzler, Arthur@\textsc{Schnitzler, Arthur}!zzzBeer-Hofmann, Richard@\emph{von Richard Beer-Hofmann}!1894-10-071@{7. 10. 1894}|(be} \toendnotes[C]{\smallbreak\pagebreak[2]} \Standort{CUL, Schnitzler, B 8.}
\physDesc{Bildpostkarte
\newline{}Handschrift: Bleistift, lateinische Kurrent\newline{}Versand: 1) Stempel: »\nobreak{}\oindex{Hotel Quirinale@\textbf{Hotel Quirinale}, \emph{Hotel (K.HTL)}|pwk}Grand Hôtel du Quirinal ROME, 7–OTT.–94, Tenu par Alessandro Marroni\nobreak{}«.  2) Stempel: »\nobreak{}\oindex{IX., Alsergrund@\textbf{IX., Alsergrund}, \emph{Bezirk (A.BZK)}|pwk}Wien 9/3, 9. 10. 94, 8.V, Bestellt\nobreak{}«. 
\newline{}Schnitzler: mit Bleistift nummeriert: »38« \newline{}Zusatz: Postkartenmotiv ist ein Lichtdruck mit \textcolor{pink}{Engelsburg} und \textcolor{pink}{Petersdom} }\pstart{}{\pb}Herrn D\textsuperscript{r}\pend{}\pstart{}Arthur Schnitzler\pend{}\pstart{}\textcolor{pink}{Austria}{}\ledrightnote{\textcolor{pink}{Österreich}}\pend{}\pstart{}\textcolor{pink}{Wien}{}\ledrightnote{\textcolor{pink}{Wien}}\pend{}\pstart{}\textcolor{pink}{Frankgasse 1}{}\ledrightnote{\textcolor{pink}{Frankgasse}}\pend{}{\bigskip}\pstart
           \noindent{}\centering{}\textcolor{gray}{\textbf{{\pb}Ricordo di \textcolor{pink}{Roma}{}\ledrightnote{\textcolor{pink}{Rom}}}}\pend
           \pstart
           \raggedleft{}Sonntag 7/X{ }\textcolor{pink}{\uline{Rom}}{}\ledrightnote{\textcolor{pink}{Rom}}\pend
           \pstart
           Lieber Arthur! Warum schreiben Sie nicht? bis incl. nächsten
                  Sonntag bin ich hier – »\textcolor{pink}{Hôtel
                  Quirinal}{}\ledrightnote{\textcolor{pink}{Hotel Quirinale}}.« Sehe aber auch auf Post nach ob nichts »posta ferma« von Ihnen.
                  \textcolor{green}{Zeit}{}\ledrightnote{\textcolor{green}{Die Zeit. Wiener Wochenschrift}}? \textcolor{green}{Schmetterlingsschlacht}{}\ledrightnote{\textcolor{green}{Die Schmetterlingsschlacht}}? \textcolor{blue}{Bahr}{}\ledrightnote{\textcolor{blue}{Hermann Bahr}}s’
               Privatadresse habe ich in unsäglicher Du{\geminationm}heit vergessen.
               In \textcolor{pink}{\uline{Rom}}{}\ledrightnote{\textcolor{pink}{Rom}} bin ich.\pend
           \pstart
           Herzlichst{\\[\baselineskip]}Ihr \spacefill\mbox{Richard}\pend
           \leftskip=0em{}\endnumbering\briefempfaengerindex{Schnitzler, Arthur@\textsc{Schnitzler, Arthur}!zzzBeer-Hofmann, Richard@\emph{von Richard Beer-Hofmann}!1894-10-071@{7. 10. 1894}|)be}\mylabel{h}  \normalsize

\doendnotes{C}
\bigskip
\vfill

\clearpage

\footnotesize

\lohead{\textsc{register}}

% Definiere theindex-Environment komplett neu ohne reledmac
\makeatletter
\renewenvironment{theindex}{%
  \section*{\indexname}%
  \setlength{\parindent}{0pt}%
  \setlength{\parskip}{0pt plus 0.3pt}%
  \let\item\@idxitem
}{%
  \clearpage
}
\makeatother

\IfFileExists{\jobname-pw.ind}{\input{\jobname-pw.ind}}{}

\end{document}

      