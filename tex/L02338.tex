%% latex-korrekturansicht-vorspann.tex
%% Vorspann für die Korrekturansicht.
%% Lädt die gemeinsame Datei latex-vorspann.tex mit gesetztem Schalter.

\newif\ifkorrekturansicht
\korrekturansichttrue

\input{../tex-inputs/latex-vorspann}


               \section[Hugo Hofmannsthal an Arthur Schnitzler, 13. 3. 1920]{ Hugo Hofmannsthal an Arthur Schnitzler, 13. 3. 1920}\nopagebreak\mylabel{v}\rehead{ }\normalsize\beginnumbering\briefempfaengerindex{Schnitzler, Arthur@\textsc{Schnitzler, Arthur}!zzzHofmannsthal, Hugo von@\emph{von Hugo von Hofmannsthal}!1920-03-131@{13. 3. 1920}|(be} \toendnotes[C]{\smallbreak\pagebreak[2]} \Standort{CUL, Schnitzler, B 43.}
\physDesc{Postkarte
\newline{}Handschrift: schwarze Tinte, deutsche Kurrent\newline{}Versand: Stempel: »\nobreak{}\oindex{Rodaun@\textbf{Rodaun}, \emph{Teil eines besiedelten Ortes (A.BSOX)}|pwk}Rodaun\nobreak{}«.  \newline{}Ordnung: 1) mit Bleistift von \textcolor{blue}{Frieda Pollak} (?) mit dem Buchstaben »A« (Abgeschrieben/Abschrift) gekennzeichnet 2) mit Bleistift von unbekannter Hand nummeriert: »\strikeout{260}«3) mit Bleistift von unbekannter Hand nummeriert: »364«}\buchAbdrucke{\weitereDrucke{Hugo von Hofmannsthal, Arthur Schnitzler: \emph{Briefwechsel}. Hg. Therese Nickl und Heinrich Schnitzler. Frankfurt am Main: \emph{S. Fischer} 1964, S. 291.} }\toendnotes[C]{\smallbreak}\pstart{}{\pb}\textsc{Herrn D\textsuperscript{r} Arthur Schnitzler}\pend{}\pstart{}\textsc{\textcolor{pink}{Wien}{}\ledrightnote{\textcolor{pink}{Wien}}}\pend{}\pstart{}\textsc{\textcolor{pink}{XVIII. Sternwartestrasse 71}{}\ledrightnote{\textcolor{pink}{Sternwartestraße}}}\pend{}{\bigskip}\pstart
           \raggedleft{}{\pb}\textcolor{pink}{Rodaun}{}\ledrightnote{\textcolor{pink}{Rodaun}}{ }13 III 20\pend
           \pstart{}mein lieber Arthur, \pend\pstart
           seit 5 Wochen vegetiere ich hier zwiſchen Bett u. Fauteuil (mehr Bett als Fauteuil)
               mit Grippe in Form von Rheumatismen vom Genick bis in die Fußzehen. \pend
           \pstart
           {\pb}Hab ſeit 5 Wochen \textcolor{blue}{Gerty}{}\ledrightnote{\textcolor{blue}{Gertrude von Hofmannsthal}} nicht geſehen, die drinnen, aber indeſſen hergeſtellt. –
               Hab ich, um mein Vergnügen an dem \textcolor{green}{Luſtſpiel}{}\ledrightnote{→\textcolor{green}{Die Schwestern oder Casanova in Spa. Lustspiel in Versen}} zu bezeichnen, das Wort »unterhaltend« gebraucht? u. war Ihnen das
               Wort unlieb? (faſt ſcheint’s mir ſo.) Ich gebrauchte es, um etwas Seltenes
               auszudrücken, den freien leichten Silberglanz des Geiſtes, den zu empfangen woltuend
               iſt. Natürlich hat ein Dichterwerk noch viele andere Eigenſchaften!\pend
           \pstart
           Alles Gute Ihnen für die \textcolor{green}{Proben}{}\ledrightnote{→\textcolor{green}{Die Schwestern oder Casanova in Spa. Lustspiel in Versen}}
               u. überhaupt! Von Herzen Ihr{\\}\spacefill\mbox{Hugo.}\pend
           \endnumbering\briefempfaengerindex{Schnitzler, Arthur@\textsc{Schnitzler, Arthur}!zzzHofmannsthal, Hugo von@\emph{von Hugo von Hofmannsthal}!1920-03-131@{13. 3. 1920}|)be}\mylabel{h}  \normalsize

\doendnotes{C}
\bigskip
\vfill

\clearpage

\footnotesize

\lohead{\textsc{register}}

% Definiere theindex-Environment komplett neu ohne reledmac
\makeatletter
\renewenvironment{theindex}{%
  \section*{\indexname}%
  \setlength{\parindent}{0pt}%
  \setlength{\parskip}{0pt plus 0.3pt}%
  \let\item\@idxitem
}{%
  \clearpage
}
\makeatother

\IfFileExists{\jobname-pw.ind}{\input{\jobname-pw.ind}}{}

\end{document}

      