%% latex-korrekturansicht-vorspann.tex
%% Vorspann für die Korrekturansicht.
%% Lädt die gemeinsame Datei latex-vorspann.tex mit gesetztem Schalter.

\newif\ifkorrekturansicht
\korrekturansichttrue

\input{../tex-inputs/latex-vorspann}


               \section[Arthur Schnitzler an Richard Beer-Hofmann, 28. 6. 1898]{ Arthur Schnitzler an Richard Beer-Hofmann, 28. 6. 1898}\nopagebreak\mylabel{v}\rehead{ }\normalsize\beginnumbering\briefempfaengerindex{Beer-Hofmann, Richard@\textsc{Beer-Hofmann, Richard}!zzzSchnitzler, Arthur@\emph{von Arthur Schnitzler}!1898-06-281@{28. 6. 1898}|(be} \toendnotes[C]{\smallbreak\pagebreak[2]} \Standort{YCGL, MSS 31.}
\physDesc{Brief, 1 Blatt, 4 Seiten, Umschlag
\newline{}Handschrift: Bleistift, deutsche Kurrent\newline{}Versand: 1) Stempel: »\nobreak{}\oindex{IX., Alsergrund@\textbf{IX., Alsergrund}, \emph{Bezirk (A.BZK)}|pwk}Wien 9/3 72, 28. 6. 98, 2–\textcolor{gray}{3N}\nobreak{}«.  2) Stempel: »\nobreak{}\oindex{Steindorf am Ossiacher See@\textbf{Steindorf am Ossiacher See}, \emph{http://www.geonames.org/ontologyA.ADM3}|pwk}{\pb}{[}Stein{]}dorf am Ossiacher See, 29 6 98\nobreak{}«. }\buchAbdrucke{\weitereDrucke{Arthur Schnitzler, Richard Beer-Hofmann: \emph{Briefwechsel 1891–1931}. Hg. Konstanze Fliedl. Wien, Zürich: \emph{Europaverlag} 1992, S. 120–121.} }\toendnotes[C]{\smallbreak}\pstart{}\textcolor{pink}{{\pb}\textsc{Kärnthen}}{}\ledrightnote{\textcolor{pink}{Kärnten}}.\pend{}\pstart{}
                  Herrn \textsc{Dr. Richard
                     Beer-Hofmann}\pend{}\pstart{}\textsc{\textcolor{pink}{Steindorf}{}\ledrightnote{\textcolor{pink}{Steindorf am Ossiacher See}}}\pend{}\pstart{}\textsc{am \textcolor{pink}{Ossiacher}{}\ledrightnote{\textcolor{pink}{Ossiacher See}}}ſee \pend{}{\bigskip}\pstart
           \raggedleft{}{\pb}28. 6. 98.\pend
           \pstart
           Mein lieber Richard, ich bin die letzten Tage wirklich ſehr fleißig
               geweſen. Habe \textcolor{green}{Vermächtnis}{}\ledrightnote{\textcolor{green}{Das Vermächtnis. Schauspiel in drei Akten}} insbeſondre 2. u 3. Akt
               ziemlich gründlich hergeno{\geminationm}en und glaube, dſs ich mit
               dieſem Stück heute kaum viel weiter ko{\geminationm}en könnte als es
               iſt. Morgen gebe ich \textcolor{blue}{Schlenther}{}\ledrightnote{\textcolor{blue}{Paul Schlenther}} die Aenderungen.
               Auch die \textcolor{green}{Einakter}{}\ledrightnote{→\textcolor{green}{Der grüne Kakadu – Paracelsus – Die Gefährtin. Drei Einakter}}{ }ſind ſo gut wie fertig – »und wie geht es
               Ihnen?«\pend
           \pstart
           Ich ke{\geminationn} mich heuer mit dem So{\geminationm}er gar nicht ordentlich aus. Hoffentlich können wir uns
               im Auguſt, erſte Hälfte treffen – doch ſowohl \introOben{}ich\introOben{} als \textcolor{blue}{Hugo}{}\ledrightnote{\textcolor{blue}{Hugo von Hofmannsthal}} wären ſehr für was {\pb}andres als \textcolor{pink}{Salzburg}{}\ledrightnote{\textcolor{pink}{Salzburg}}
                  eingeno{\geminationm}en \introOben{}(\introOben{}(wo ich im Lauf
               des Juli (20–27 herum) jedenfalls ſein
               werde.)) – \textcolor{pink}{Schweiz}{}\ledrightnote{\textcolor{pink}{Schweiz}} – \textcolor{pink}{Luzern}{}\ledrightnote{\textcolor{pink}{Luzern}} – mit Rad gemiſcht –\pend
           \pstart
           Es ist nemlich auch ſehr möglich, daſs meine \textcolor{blue}{Mama}{}\ledrightnote{→\textcolor{blue}{Louise Schnitzler}} nach \textcolor{pink}{Luzern}{}\ledrightnote{\textcolor{pink}{Luzern}} geht, in
               welchem Fall ich mich beinah verpflichtet habe hinzugehn. \uline{Hier} bleib ich noch bis 12, 13, 14, 15 Juli. –\pend
           \pstart
           – Heut hab ich von \textcolor{blue}{Mirjam}{}\ledrightnote{\textcolor{blue}{Mirjam Beer-Hofmann}} geträumt, aber es war
               eigentlich ein kleines Kind, das ich behandelt habe, und ich {\pb}war rieſig ſtolz, daſs eine Patientin von mir ſo gut
               ausſieht – und ich hab ſie Ihnen gezeigt, wir ſind vor dem Haus, das an der Donau war, zuſa{\geminationm}en geſtanden,
               und \textcolor{blue}{Mirjam}{}\ledrightnote{\textcolor{blue}{Mirjam Beer-Hofmann}} war am Fenſter, 2. Stock, in den Armen
               einer \textcolor{blue}{\label{K_L00809_1v}\edtext{\textsc{sage femme}}{\lemma{\textnormal{\emph{sage femme}}}\Cendnote{\textnormal{französisch: Hebamme}}}\label{K_L00809_1h}}{}\ledrightnote{→\textcolor{blue}{Leopoldine Kirchrath}} (\introOben{}der\introOben{}{ }\label{K_L00809_2v}\edtext{mir
                  bekannten}{\lemma{\textnormal{\emph{mir
                  bekannten}}}\Cendnote{\textnormal{Gemeint dürfte \textcolor{blue}{Leopoldine Kirchrath}
                  sein.}}}\label{K_L00809_2h}) – und war ſo dick und glücklich, daſs ſie halb beim Fenſter draußen
               war. (Dieſer Traum iſt ein Geſchenk für \textcolor{blue}{Paula}{}\ledrightnote{\textcolor{blue}{Paula Beer-Hofmann}}. –)\pend
           \pstart
           – Wir machen gelegentlich kleine Aus{\pb}flüge per Rad,
                  \textcolor{pink}{Rohrerhütte}{}\ledrightnote{\textcolor{pink}{Rohrerhütte}}, \textcolor{pink}{Weidlingau}{}\ledrightnote{\textcolor{pink}{Weidlingau}}.\pend
           \pstart
           Wie iſt Ihre Sti{\geminationm}ung? Verſuchen Sie zu radeln? Arbeiten
               Sie?\pend
           \pstart
           Leben Sie wohl. Herzlicher Gruſs. Ihr \spacefill\mbox{Arth}\pend
           \endnumbering\briefempfaengerindex{Beer-Hofmann, Richard@\textsc{Beer-Hofmann, Richard}!zzzSchnitzler, Arthur@\emph{von Arthur Schnitzler}!1898-06-281@{28. 6. 1898}|)be}\mylabel{h}  \normalsize

\doendnotes{C}
\bigskip
\vfill

\clearpage

\footnotesize

\lohead{\textsc{register}}

% Definiere theindex-Environment komplett neu ohne reledmac
\makeatletter
\renewenvironment{theindex}{%
  \section*{\indexname}%
  \setlength{\parindent}{0pt}%
  \setlength{\parskip}{0pt plus 0.3pt}%
  \let\item\@idxitem
}{%
  \clearpage
}
\makeatother

\IfFileExists{\jobname-pw.ind}{\input{\jobname-pw.ind}}{}

\end{document}

      