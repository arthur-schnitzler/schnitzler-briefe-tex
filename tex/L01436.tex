%% latex-korrekturansicht-vorspann.tex
%% Vorspann für die Korrekturansicht.
%% Lädt die gemeinsame Datei latex-vorspann.tex mit gesetztem Schalter.

\newif\ifkorrekturansicht
\korrekturansichttrue

\input{../tex-inputs/latex-vorspann}


               \section[Richard Beer-Hofmann an Arthur Schnitzler, 5. 9. 1904]{ Richard Beer-Hofmann an Arthur Schnitzler, 5. 9. 1904}\nopagebreak\mylabel{v}\rehead{ }\normalsize\beginnumbering\briefempfaengerindex{Schnitzler, Arthur@\textsc{Schnitzler, Arthur}!zzzBeer-Hofmann, Richard@\emph{von Richard Beer-Hofmann}!1904-09-051@{5. 9. 1904}|(be} \toendnotes[C]{\smallbreak\pagebreak[2]} \Standort{CUL, Schnitzler, B 8.}
\physDesc{Brief, 1 Blatt, 1 Seite
\newline{}Handschrift: schwarze Tinte, lateinische Kurrent\newline{}Ordnung: mit Bleistift von unbekannter Hand nummeriert:
                                    »187« }\toendnotes[C]{\smallbreak}\pstart
           \noindent{}\centering{}{\pb}\textcolor{pink}{Aussee}{}\ledrightnote{\textcolor{pink}{Bad Aussee}}{ }5./IX 04\pend
           \pstart
           \noindent{}Lieber Arthur! Wenn das Wetter nicht zu scheusslich ist, bin ich
                  Mittwoch{ }\uline{11.46} in \textcolor{pink}{Lueg}{}\ledrightnote{\textcolor{pink}{Lueg am Wolfgangsee}}. Um \uline{4.02} fahre ich von \textcolor{pink}{Lueg}{}\ledrightnote{\textcolor{pink}{Lueg am Wolfgangsee}} nach \textcolor{pink}{Ischl}{}\ledrightnote{\textcolor{pink}{Bad Ischl}} (\uline{5.02}), und von dort (\substVorne{}\textsuperscript{\uline{6.56}}\substDazwischen{}6.05\substHinten{}) nach \textcolor{pink}{Aussee}{}\ledrightnote{\textcolor{pink}{Bad Aussee}} (\uline{7.15}). Bei späterer Abfahrt von \textcolor{pink}{Lueg}{}\ledrightnote{\textcolor{pink}{Lueg am Wolfgangsee}} hätte ich
               keinen guten Anschluss nach \textcolor{pink}{Aussee}{}\ledrightnote{\textcolor{pink}{Bad Aussee}}. Vielleicht
               fahren Sie dann \strikeout{d} statt Donnerstag früh,
                  Mittwoch Nachmittag mit mir. In \textcolor{pink}{Aussee}{}\ledrightnote{\textcolor{pink}{Bad Aussee}} wohnen Sie nicht \textcolor{pink}{Elisabeth}{}\ledrightnote{\textcolor{pink}{Bade-Hotel Elisabeth}}, das um
               diese Zeit im Veröden sein dürfte. Vielleicht »\textcolor{pink}{Post}{}\ledrightnote{\textcolor{pink}{Gasthaus Post}}« (Ich glaube jetzt »\textcolor{pink}{Franz Carl}{}\ledrightnote{\textcolor{pink}{Gasthaus Post}}«) wo
               Sie schon einmal \label{K_L01436_1v}\edtext{wohnten}{\lemma{\textnormal{\emph{wohnten}}}\Cendnote{\textnormal{vom 28. 7. 1900 bis zum 1. 8. 1900}}}\label{K_L01436_1h}. Oder »\textcolor{pink}{Hackinger}{}\ledrightnote{\textcolor{pink}{Hackinger’s Hotel zum Kaiser von Österreich}}«, »\textcolor{pink}{Erzherz.
               Johann}{}\ledrightnote{\textcolor{pink}{Erzherzog Johann}}«?\pend
           \pstart
           Herzlichst Ihr{\\[\baselineskip]}\spacefill\mbox{Richard}\pend
           \leftskip=0em{}\endnumbering\briefempfaengerindex{Schnitzler, Arthur@\textsc{Schnitzler, Arthur}!zzzBeer-Hofmann, Richard@\emph{von Richard Beer-Hofmann}!1904-09-051@{5. 9. 1904}|)be}\mylabel{h}  \normalsize

\doendnotes{C}
\bigskip
\vfill

\clearpage

\footnotesize

\lohead{\textsc{register}}

% Definiere theindex-Environment komplett neu ohne reledmac
\makeatletter
\renewenvironment{theindex}{%
  \section*{\indexname}%
  \setlength{\parindent}{0pt}%
  \setlength{\parskip}{0pt plus 0.3pt}%
  \let\item\@idxitem
}{%
  \clearpage
}
\makeatother

\IfFileExists{\jobname-pw.ind}{\input{\jobname-pw.ind}}{}

\end{document}

      