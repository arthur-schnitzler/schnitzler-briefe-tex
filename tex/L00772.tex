%% latex-korrekturansicht-vorspann.tex
%% Vorspann für die Korrekturansicht.
%% Lädt die gemeinsame Datei latex-vorspann.tex mit gesetztem Schalter.

\newif\ifkorrekturansicht
\korrekturansichttrue

\input{../tex-inputs/latex-vorspann}


               \section[Richard Beer-Hofmann an Arthur Schnitzler, 4. 2. 1898]{ Richard Beer-Hofmann an Arthur Schnitzler, 4. 2. 1898}\nopagebreak\mylabel{v}\rehead{ }\normalsize\beginnumbering\briefempfaengerindex{Schnitzler, Arthur@\textsc{Schnitzler, Arthur}!zzzBeer-Hofmann, Richard@\emph{von Richard Beer-Hofmann}!1898-02-041@{4. 2. 1898}|(be} \toendnotes[C]{\smallbreak\pagebreak[2]} \Standort{CUL, Schnitzler, B 8.}
\physDesc{Briefkarte
\newline{}Handschrift: blauer Buntstift, lateinische Kurrent\newline{}Ordnung: mit Bleistift von unbekannter Hand nummeriert:
                                    »109« }\toendnotes[C]{\smallbreak}\pstart
           \raggedleft{}{\pb}4/II 1898\pend
           \pstart
           Lieber Arthur,  also heute Abends im \textcolor{pink}{Caffee Royal}{}\ledrightnote{\textcolor{pink}{Café Scheuchenstuel}} (\label{K_L00772_1v}\edtext{\textcolor{pink}{Scheuchen{\pb}stuhl}{}\ledrightnote{\textcolor{pink}{Café Scheuchenstuel}}}{\lemma{\textnormal{\emph{Scheuchenstuhl}}}\Cendnote{\textnormal{richtig: \textcolor{pink}{Café Scheuchenstuel}. Ein Namenswechsel zu \textcolor{pink}{Café Royal} lässt sich nicht verifizieren.}}}\label{K_L00772_1h}) Ecke der \textcolor{pink}{Schuler}{}\ledrightnote{\textcolor{pink}{Schulerstraße}} u. \textcolor{pink}{Stroblgasse}{}\ledrightnote{\textcolor{pink}{Strobelgasse}}.\pend
           \pstart
           \uline{Von Herzen Ihr}{\\[\baselineskip]}\spacefill\mbox{Richard}\pend
           \leftskip=0em{}\endnumbering\briefempfaengerindex{Schnitzler, Arthur@\textsc{Schnitzler, Arthur}!zzzBeer-Hofmann, Richard@\emph{von Richard Beer-Hofmann}!1898-02-041@{4. 2. 1898}|)be}\mylabel{h}  \normalsize

\doendnotes{C}
\bigskip
\vfill

\clearpage

\footnotesize

\lohead{\textsc{register}}

% Definiere theindex-Environment komplett neu ohne reledmac
\makeatletter
\renewenvironment{theindex}{%
  \section*{\indexname}%
  \setlength{\parindent}{0pt}%
  \setlength{\parskip}{0pt plus 0.3pt}%
  \let\item\@idxitem
}{%
  \clearpage
}
\makeatother

\IfFileExists{\jobname-pw.ind}{\input{\jobname-pw.ind}}{}

\end{document}

      