%% latex-korrekturansicht-vorspann.tex
%% Vorspann für die Korrekturansicht.
%% Lädt die gemeinsame Datei latex-vorspann.tex mit gesetztem Schalter.

\newif\ifkorrekturansicht
\korrekturansichttrue

\input{../tex-inputs/latex-vorspann}


               \section[Arthur Schnitzler an Hugo von Hofmannsthal, 26. 4. 1897]{ Arthur Schnitzler an Hugo von Hofmannsthal, 26. 4. 1897}\nopagebreak\mylabel{v}\rehead{ }\normalsize\beginnumbering\briefempfaengerindex{Hofmannsthal, Hugo von@\textsc{Hofmannsthal, Hugo von}!zzzSchnitzler, Arthur@\emph{von Arthur Schnitzler}!1897-04-262@{26. 4. 1897}|(be} \toendnotes[C]{\smallbreak\pagebreak[2]} \Standort{FDH, Hs-30885,56.}
\physDesc{Brief, 1 Blatt, 4 Seiten
\newline{}Handschrift: schwarze Tinte, deutsche Kurrent}\buchAbdrucke{\weitereDrucke{1) Hugo von Hofmannsthal, Arthur Schnitzler: \emph{Briefwechsel}. Hg. Therese Nickl und Heinrich Schnitzler. Frankfurt am Main: \emph{S. Fischer} 1964, S. 81–82.} \weitereDrucke{2) Arthur Schnitzler: \emph{Briefe 1875–1912}. Hg. Therese Nickl und Heinrich Schnitzler. Frankfurt am Main: \emph{S. Fischer} 1981, S. 319–320.} }\toendnotes[C]{\smallbreak}\pstart
           \raggedleft{}{\pb}\textcolor{pink}{5 \textsc{rue \introOben{}de\introOben{} Maubeuge}}{}\ledrightnote{\textcolor{pink}{rue de Maubeuge}}{\\}\textcolor{pink}{\textsc{Paris}}{}\ledrightnote{\textcolor{pink}{Paris}}. 2\substVorne{}\textsuperscript{7}\substDazwischen{}6\substHinten{}. 4. 97.\pend
           \pstart
           Mein lieber Hugo. Seien Sie mir herzlich gegrüßt. Ich lebe im
                        I{\geminationn}erſten der Stadt, wie ich in \textcolor{pink}{Wien}{}\ledrightnote{\textcolor{pink}{Wien}} um keinen Preis leben möchte; an der Kreuzung vieler
                    Straßen, mitten im Lärm der Geſchäfte u des Verkehrs. Der Zufall hat es gefügt,
                    daſs ich gerade hier die Wohnung gefunden habe, wie ich ſie brauche, und
                    günſtige Verbindungen von \textcolor{blue}{Goldmann}{}\ledrightnote{\textcolor{blue}{Paul Goldmann}} haben ſie
                    mir verſchafft. Ich ſage \uline{mir}, obwohl das nicht
                    ganz \textcolor{blue}{richtig}{}\ledrightnote{→\textcolor{blue}{Marie Reinhard}} iſt. Aber ich habe mein Zi{\geminationm}er allein u ſo viel Freiheit, als unter
                    den bekannten Umſtänden möglich iſt. Manchmal möcht ich wohl lieber ganz allein
                    ſein; aber vielleicht iſt {\pb}es nur die Sehnſucht nach
                    der ich mich ſehne. Ich bin nemlich bisher wirklich noch nie von \textcolor{pink}{Wien}{}\ledrightnote{\textcolor{pink}{Wien}} fortgeweſen, ohne dort irgendwen zurück zu laſſen, um
                    den ich mehr oder weniger »zittern« mußte; das geht mir vielleicht ab. Im ganzen
                    aber fühl ich mich, wie Sie ſagen würden »eher« wohl; insbeſondere tritt das
                    ſonderbare ein, was ſich i{\geminationm}er beinah einſtellt,
                    we{\geminationn} ich auf Reiſen, beſſer: we{\geminationn} ich nicht daheim
                    bin; ich bin beinah gänzlich erlöſt von den Bangigkeiten und Hypochondrien, die
                    mir das Leben zu Hauſe oft ſo heftig ſtören. Aber \introOben{}auch\introOben{}
                    daſs ich gerade \uline{hier} bin, freut mich. Es iſt mir
                    oft, als we{\geminationn} ich hier lieber leben möchte als in \textcolor{pink}{Wien}{}\ledrightnote{\textcolor{pink}{Wien}}; aber das iſt wahrſchein{\pb}lich ein
                    Irrtum. Von allem, was ich hier ſchon geſehn, möchte ich Ihnen lieber erſt in
                        \textcolor{pink}{Wien}{}\ledrightnote{\textcolor{pink}{Wien}} erzählen; denn ich frage mich
                    vergeblich, was ich herausſuchen ſollte. Das ſchönſte hat mir bisher die
                    Schauſpielerei geboten; es iſt einfach was andres als die Deutſchen haben; nicht
                    immer was beſſres vielleicht – aber dem Weſen der Stücke, die ſie ſpielen,
                    wunderbar verwandt, was ja ſchließlich doch das wichtigſte iſt. Dramen ſcheinen
                    sie ja hier (wo denn???) auch nicht mehr zu ſchreiben; ich habe \textsc{\textcolor{green}{loi de l’homme}{}\ledrightnote{\textcolor{green}{Männerrecht}}, (\textcolor{blue}{Hervieu}{}\ledrightnote{\textcolor{blue}{Paul Ernest Hervieu}}); \textcolor{green}{Douloureuse}{}\ledrightnote{\textcolor{green}{La Douloureuse}} (\textcolor{blue}{Donnay}{}\ledrightnote{\textcolor{blue}{Maurice Donnay}}), – \textcolor{green}{Carrière}{}\ledrightnote{\textcolor{green}{La Carrière}} (\textcolor{blue}{Hermant}{}\ledrightnote{\textcolor{blue}{Abel Hermant}}); – \textcolor{green}{Snob}{}\ledrightnote{\textcolor{green}{Snob}} (\textcolor{blue}{Guiche}{}\ledrightnote{\textcolor{blue}{Gustave Guiches}})} – geſehen – es iſt ein
                    vollko{\geminationm}ener Sieg des Feuilletons auf dem Theater. Ich habe {\pb}wohl auch ein bischen das Gefühl des »\textcolor{green}{Menſchenfreunds}{}\ledrightnote{→\textcolor{green}{Der Alpenkönig und der Menschenfeind}}« aus dem \textcolor{blue}{Raimund}{}\ledrightnote{\textcolor{blue}{Ferdinand Raimund}}’ſchen Märchen gehabt, – aber können
                    wir wirklichen Menſchen uns auch »beſſern«? Mit Bewußtſein entwickeln – das müßte
                    wohl möglich ſein! –\pend
           \pstart
           – Sagen Sie mir ein Wort, wie es Ihnen und andren Leuten, von denen Sie gerade
                    erzählen wollen (was mir jedenfalls erwünſcht wäre) geht. – Ich werde Ende
                        Mai, ſpäteſtens Anfang Juni wieder in \textcolor{pink}{Wien}{}\ledrightnote{\textcolor{pink}{Wien}}{ }ſein. Das Wetter iſt nicht ſchön; noch ke{\geminationn} ich eigentlich den \textcolor{pink}{Pariſ}{}\ledrightnote{\textcolor{pink}{Paris}}er Frühling nicht.\pend
           \pstart
           Grüßen Sie alle, die wir beide gern haben.\pend
           \pstart Herzlich grüßt Sie Ihr \spacefill\mbox{Arthur.}\pend{}\pstart
           \noindent{}Auch Ihren \textcolor{blue}{Eltern}{}\ledrightnote{→\textcolor{blue}{Hugo August von Hofmannsthal}{\newline}→\textcolor{blue}{Anna von Hofmannsthal}}, bitte, empfehlen Sie mich freundlich.\pend
           \endnumbering\briefempfaengerindex{Hofmannsthal, Hugo von@\textsc{Hofmannsthal, Hugo von}!zzzSchnitzler, Arthur@\emph{von Arthur Schnitzler}!1897-04-262@{26. 4. 1897}|)be}\mylabel{h}  \normalsize

\doendnotes{C}
\bigskip
\vfill

\clearpage

\footnotesize

\lohead{\textsc{register}}

% Definiere theindex-Environment komplett neu ohne reledmac
\makeatletter
\renewenvironment{theindex}{%
  \section*{\indexname}%
  \setlength{\parindent}{0pt}%
  \setlength{\parskip}{0pt plus 0.3pt}%
  \let\item\@idxitem
}{%
  \clearpage
}
\makeatother

\IfFileExists{\jobname-pw.ind}{\input{\jobname-pw.ind}}{}

\end{document}

      