%% latex-korrekturansicht-vorspann.tex
%% Vorspann für die Korrekturansicht.
%% Lädt die gemeinsame Datei latex-vorspann.tex mit gesetztem Schalter.

\newif\ifkorrekturansicht
\korrekturansichttrue

\input{../tex-inputs/latex-vorspann}


               \section[Hugo von Hofmannsthal an Arthur Schnitzler, {[}9. 9. 1893{]}]{ Hugo von Hofmannsthal an Arthur Schnitzler, {[}9. 9. 1893{]}}\nopagebreak\mylabel{v}\rehead{ }\normalsize\beginnumbering\briefempfaengerindex{Schnitzler, Arthur@\textsc{Schnitzler, Arthur}!zzzHofmannsthal, Hugo von@\emph{von Hugo von Hofmannsthal}!1893-09-091@{{[}9. 9. 1893{]}}|(be} \toendnotes[C]{\smallbreak\pagebreak[2]} \Standort{CUL, Schnitzler, B 43.}
\physDesc{Brief, 1 Blatt, 4 Seiten
\newline{}Handschrift: schwarze Tinte, deutsche Kurrent
\newline{}Schnitzler: mit Bleistift datiert: »9/9 93« und nummeriert: »57« }\buchAbdrucke{\weitereDrucke{1) Hugo von Hofmannsthal: \emph{Briefe. 1890–1901}. Berlin: \emph{S. Fischer} 1935, S. 88–89.} \weitereDrucke{2) Hugo von Hofmannsthal, Arthur Schnitzler: \emph{Briefwechsel}. Hg. Therese Nickl und Heinrich Schnitzler. Frankfurt am Main: \emph{S. Fischer} 1964, S. 45–46.} }\toendnotes[C]{\smallbreak}\pstart
           \raggedleft{}{\pb}\textcolor{pink}{\textsc{Strobl}}{}\ledrightnote{\textcolor{pink}{Strobl}}\pend
           \pstart{}mein lieber Arthur!\pend\pstart
           Schönheit und Leben! Iſt Ihnen das nicht aufgefallen, daſs einem das Leben ſo
                    ganz beſonders gut gefällt und man ganz genau weiß, wie es ausſchaut und
                    ſchmeckt, wenn man eben momentan innerlich müſſig iſt und eigentlich nicht lebt?
                    Wie \textcolor{blue}{Euer}{}\ledrightnote{→\textcolor{blue}{Felix Salten}} Brief gekommen
                    iſt, der »launige« Brief mit dieſen 2 großen Worten, iſt es mir ein bischen
                    vorgekommen, wie wenn ich an einem Tiſch ſäße und wirklich gegeſſen hätte und
                    vor mir lägen in unappetitlicher Realität {\pb}Krebsſchalen, Hühnerknochen
                    und Pfirſichkerne{\dots} Ihr aber ſitzt vor einem
                    wunderſchönen Stilleben mit roten Languſten, goldrothen Weintrauben und bunten
                    Truthühnern. Um es zu eſſen, muſs man es rupfen und ſieden und ſchälen und
                    ſchneiden und kauen und dann iſt es gar nicht mehr ſchön!\pend
           \pstart
           Und doch gehört’s zum Eſſen und nicht zum Anſchauen. Es – ich meine das
                    Leben.\pend
           \pstart
           Ich bleibe alſo hier bis zum 11\textsuperscript{ten}; dann mit den \textcolor{blue}{Eltern}{}\ledrightnote{\textcolor{blue}{Hugo August von Hofmannsthal}{\newline}\textcolor{blue}{Anna von Hofmannsthal}} nach {\pb}\textcolor{pink}{München}{}\ledrightnote{\textcolor{pink}{München}} u. \textcolor{pink}{Nürnberg}{}\ledrightnote{\textcolor{pink}{Nürnberg}}; dann vielleicht zur Jagd nach \textcolor{pink}{Böhmen}{}\ledrightnote{\textcolor{pink}{Böhmen}}.\pend
           \pstart
           Jedenfalls bin ich Ende September bei Euch.\pend
           \pstart
           Dieſer Tage iſt die 8\textsuperscript{te}, letzte Rate von 12 fl. an
                        \textcolor{blue}{Fels}{}\ledrightnote{\textcolor{blue}{Friedrich Michael Fels}} (\textcolor{pink}{III
                            \textsc{Strohgasse 3}}{}\ledrightnote{\textcolor{pink}{Strohgasse}}) fällig; ich weiß nicht, ob Sie dazu nur 5 fl oder mehr ſchulden; da ich
                    aber momentan kein Geld habe und \textcolor{blue}{Richard}{}\ledrightnote{\textcolor{blue}{Richard Beer-Hofmann}}
                    nicht da iſt, ſo bitte ſchicken Sie ihm \uuline{12} fl.
                    mit dem Vermerk »letzte Rate.«\pend
           \pstart
           {\pb}Wiſſen Sie die Nummer von
                        \textcolor{blue}{Richard}{}\ledrightnote{\textcolor{blue}{Richard Beer-Hofmann}}’s Regiment (\textcolor{pink}{Znaim}{}\ledrightnote{\textcolor{pink}{Znaim}})?\pend
           \pstart
           Servus{\\[\baselineskip]}\spacefill\mbox{Loris.}\pend
           \leftskip=0em{}\pstart
           \noindent{}Bitte bald ſchreiben! Wo iſt \textcolor{blue}{\textsc{Salten}}{}\ledrightnote{\textcolor{blue}{Felix Salten}}?\pend
           \endnumbering\briefempfaengerindex{Schnitzler, Arthur@\textsc{Schnitzler, Arthur}!zzzHofmannsthal, Hugo von@\emph{von Hugo von Hofmannsthal}!1893-09-091@{{[}9. 9. 1893{]}}|)be}\mylabel{h}  \normalsize

\doendnotes{C}
\bigskip
\vfill

\clearpage

\footnotesize

\lohead{\textsc{register}}

% Definiere theindex-Environment komplett neu ohne reledmac
\makeatletter
\renewenvironment{theindex}{%
  \section*{\indexname}%
  \setlength{\parindent}{0pt}%
  \setlength{\parskip}{0pt plus 0.3pt}%
  \let\item\@idxitem
}{%
  \clearpage
}
\makeatother

\IfFileExists{\jobname-pw.ind}{\input{\jobname-pw.ind}}{}

\end{document}

      