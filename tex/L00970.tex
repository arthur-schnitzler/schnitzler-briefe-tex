%% latex-korrekturansicht-vorspann.tex
%% Vorspann für die Korrekturansicht.
%% Lädt die gemeinsame Datei latex-vorspann.tex mit gesetztem Schalter.

\newif\ifkorrekturansicht
\korrekturansichttrue

\input{../tex-inputs/latex-vorspann}


               \section[Arthur Schnitzler an Hugo von Hofmannsthal, 8. 9. 1899]{ Arthur Schnitzler an Hugo von Hofmannsthal, 8. 9. 1899}\nopagebreak\mylabel{v}\rehead{ }\normalsize\beginnumbering\briefempfaengerindex{Hofmannsthal, Hugo von@\textsc{Hofmannsthal, Hugo von}!zzzSchnitzler, Arthur@\emph{von Arthur Schnitzler}!1899-09-082@{8. 9. 1899}|(be} \toendnotes[C]{\smallbreak\pagebreak[2]} \Standort{FDH, Hs-30885,86.}
\physDesc{Postkarte
\newline{}Handschrift: Bleistift, deutsche Kurrent\newline{}Versand: 1) Stempel: »\nobreak{}\oindex{Bad Ischl@\textbf{Bad Ischl}, \emph{Besiedelter Ort (A.BSO)}|pwk}Ischl, 8. 9. 99, 8–9N\nobreak{}«.  2) Stempel: »\nobreak{}\oindex{Gasthaus Brunnthaler@\textbf{Gasthaus Brunnthaler}, \emph{Hotel (K.HTL)}|pwk}Alt-Aussee, 9 9 99\nobreak{}«. \newline{}Ordnung: von Schnitzler mit Bleistift mutmaßlich bei der Durchsicht der
                                 Briefe 1929 datiert: »9/9 99« }\buchAbdrucke{\weitereDrucke{Hugo von Hofmannsthal, Arthur Schnitzler: \emph{Briefwechsel}. Hg. Therese Nickl und Heinrich Schnitzler. Frankfurt am Main: \emph{S. Fischer} 1964, S. 130.} }\toendnotes[C]{\smallbreak}\pstart{}{\pb}Herrn \textsc{Hugo v Hofmannsthal}\pend{}\pstart{}\textsc{\textcolor{pink}{Altaussee}{}\ledrightnote{\textcolor{pink}{Gasthaus Brunnthaler}}}\pend{}\pstart{}\textcolor{pink}{\textsc{Brunthaler}s Gaſthaus}{}\ledrightnote{\textcolor{pink}{Gasthaus Brunnthaler}}\pend{}{\bigskip}\pstart
           \noindent{}{\pb}lieber, bin eben auf der Bahn, habe \textcolor{blue}{Stationschef}{}\ledrightnote{→\textcolor{blue}{Ferdinand Miliczek}} geſprochen, der ſofort Träger 1 rufen lieſs,
               welch letzterer sich \uline{abſolut}{ }\uline{nicht} an Ihr Futteral erinnern will. Auch \uline{gefunden} wurde es nicht. – Wohl in ein fremdes \textsc{Coupé} gerathen? –\pend
           \pstart
           Ich werde wahrſcheinlich So{\geminationn}tag{ }Mittag bei \textcolor{pink}{\textsc{Brunthaler}}{}\ledrightnote{\textcolor{pink}{Gasthaus Brunnthaler}}{ }ſein. Herzlich Ihr \spacefill\mbox{A. S.}\pend
           \endnumbering\briefempfaengerindex{Hofmannsthal, Hugo von@\textsc{Hofmannsthal, Hugo von}!zzzSchnitzler, Arthur@\emph{von Arthur Schnitzler}!1899-09-082@{8. 9. 1899}|)be}\mylabel{h}  \normalsize

\doendnotes{C}
\bigskip
\vfill

\clearpage

\footnotesize

\lohead{\textsc{register}}

% Definiere theindex-Environment komplett neu ohne reledmac
\makeatletter
\renewenvironment{theindex}{%
  \section*{\indexname}%
  \setlength{\parindent}{0pt}%
  \setlength{\parskip}{0pt plus 0.3pt}%
  \let\item\@idxitem
}{%
  \clearpage
}
\makeatother

\IfFileExists{\jobname-pw.ind}{\input{\jobname-pw.ind}}{}

\end{document}

      