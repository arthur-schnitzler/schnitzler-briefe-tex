%% latex-korrekturansicht-vorspann.tex
%% Vorspann für die Korrekturansicht.
%% Lädt die gemeinsame Datei latex-vorspann.tex mit gesetztem Schalter.

\newif\ifkorrekturansicht
\korrekturansichttrue

\input{../tex-inputs/latex-vorspann}


               \section[Hugo von Hofmannsthal an Arthur Schnitzler, 21. {[}8. 1895{]}]{ Hugo von Hofmannsthal an Arthur Schnitzler, 21. {[}8. 1895{]}}\nopagebreak\mylabel{v}\rehead{ }\normalsize\beginnumbering\briefempfaengerindex{Schnitzler, Arthur@\textsc{Schnitzler, Arthur}!zzzHofmannsthal, Hugo von@\emph{von Hugo von Hofmannsthal}!1895-08-211@{21. {[}8. 1895{]}}|(be} \toendnotes[C]{\smallbreak\pagebreak[2]} \Standort{CUL, Schnitzler, B 43.}
\physDesc{Brief, 1 Blatt, 4 Seiten
\newline{}Handschrift: Bleistift, deutsche Kurrent
\newline{}Schnitzler: mit Bleistift das Datum vervollständigt: »8. 95« und nummeriert: »75« }\buchAbdrucke{\weitereDrucke{1) Hugo von Hofmannsthal: \emph{Briefe. 1890–1901}. Berlin: \emph{S. Fischer} 1935, S. 174–175.} \weitereDrucke{2) Hugo von Hofmannsthal, Arthur Schnitzler: \emph{Briefwechsel}. Hg. Therese Nickl und Heinrich Schnitzler. Frankfurt am Main: \emph{S. Fischer} 1964, S. 60–61.} }\toendnotes[C]{\smallbreak}\pstart
           \raggedleft{}{\pb}Quartier zu \textcolor{pink}{Klein Teſſwitz}{}\ledrightnote{\textcolor{pink}{Dobšice u Znojma}} bei \textcolor{pink}{Znaim}{}\ledrightnote{\textcolor{pink}{Znaim}},{\\}Mittwoch 21\textsuperscript{ten}\pend
           \pstart
           Es freut mich herzlich, Sie zufrieden zu wiſſen und von guten und geſcheiten
                    Menſchen umgeben zu denken. Unſer \textcolor{blue}{Goldmann}{}\ledrightnote{\textcolor{blue}{Paul Goldmann}},
                    der im Journalismus lebt und ſich ſo völlig vor \textsc{\label{K_L00476_1v}\edtext{mesquinerie}{\lemma{\textnormal{\emph{mesquinerie}}}\Cendnote{\textnormal{Knausrigkeit}}}\label{K_L00476_1h}} bewahrt hat, und Frau \textsc{D\textsuperscript{r}{ }\textcolor{blue}{Salomé}{}\ledrightnote{\textcolor{blue}{Lou Andreas-Salomé}}}{ }ſind ganz die Atmoſphäre, worin einem die Vermuthung von der Jugend der
                    Seele glaubhaft wird. Ich bin, in gewiſſem Sinn, mutterſeelenallein, und {\pb}doch ſo montiert, daſs ich
                    mich manchmal gewaltſam zwingen muſs, an die Realität zu glauben. Mir iſt, wie
                    einem der in der tiefen ſtillen Kajüte eines Schiffes dem ſchönſten Land langſam
                    zufährt.\pend
           \pstart
           Es ſind wundervolle Sommertage. Ich wohne in einem kühlen niedrigen Bauernzimmer,
                    hinter einem großen Birnbaum. Gegenüber iſt ein zehnjähriges Mädel, die doch
                    eine Frau iſt, und ihr eigenes Kind, ihre eigene Mutter iſt. Ich habe den
                        »\textcolor{green}{Faust}{}\ledrightnote{\textcolor{green}{Faust}}« mit und die \textcolor{green}{Wanderjahre}{}\ledrightnote{\textcolor{green}{Wilhelm Meisters Wanderjahre}}. Ich weiß von meinem {\pb}wirklichen Leben und bin
                    doch unendlich weit davon.\pend
           \pstart
           Die friſchen Birnen ſind ganz warm von der gedämpften Sonne, die im Wipfel des
                    Birnbaums iſt. Von der \textcolor{green}{Helena}{}\ledrightnote{\textcolor{green}{Faust}} les’ ich
                    dieſen Vers: »\label{K_L00476_2v}\edtext{\textcolor{green}{Wer ſie verſteht, der darf ſie nicht
                        entbehren!}{}\ledrightnote{→\textcolor{green}{Faust}}}{\lemma{\textnormal{\emph{Wer … entbehren!}}}\Cendnote{\textnormal{richtig: »Wer sie erkennt der
                            darf sie nicht entbehren.« (II. Teil, Ende des 1.
                            Akts).}}}\label{K_L00476_2h}« Heute abend werd ich nach \textcolor{pink}{Znaim}{}\ledrightnote{\textcolor{pink}{Znaim}} hineinfahren, wo Muſik von den \textcolor{brown}{Deutſchmeiſtern}{}\ledrightnote{\textcolor{brown}{Hoch- und Deutschmeisterkapelle}} iſt und in der kühlen ſternhellen
                    Nacht zurückfahren, ein biſſel vom weißen Wein montiert, auf einem hohen Wagen,
                    der ſehr {\pb}unſicher fährt, mit
                    meinem Rittmeiſter und meinem hübſchen und indolent-graciöſen Lieutenant, die in
                    der Nacht ſehr wenig und ſehr lieb reden werden. Begreifen Sie daſs ich
                    zufrieden bin?\pend
           \pstart
           Leben Sie wohl und denken mit Ihren Freunden freundlich an mich. Adieu.\pend
           \pstart
           Der Ihre{\\[\baselineskip]}\spacefill\mbox{Hugo.}\pend
           \leftskip=0em{}\endnumbering\briefempfaengerindex{Schnitzler, Arthur@\textsc{Schnitzler, Arthur}!zzzHofmannsthal, Hugo von@\emph{von Hugo von Hofmannsthal}!1895-08-211@{21. {[}8. 1895{]}}|)be}\mylabel{h}  \normalsize

\doendnotes{C}
\bigskip
\vfill

\clearpage

\footnotesize

\lohead{\textsc{register}}

% Definiere theindex-Environment komplett neu ohne reledmac
\makeatletter
\renewenvironment{theindex}{%
  \section*{\indexname}%
  \setlength{\parindent}{0pt}%
  \setlength{\parskip}{0pt plus 0.3pt}%
  \let\item\@idxitem
}{%
  \clearpage
}
\makeatother

\IfFileExists{\jobname-pw.ind}{\input{\jobname-pw.ind}}{}

\end{document}

      