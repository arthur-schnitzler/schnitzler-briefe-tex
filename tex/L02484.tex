%% latex-korrekturansicht-vorspann.tex
%% Vorspann für die Korrekturansicht.
%% Lädt die gemeinsame Datei latex-vorspann.tex mit gesetztem Schalter.

\newif\ifkorrekturansicht
\korrekturansichttrue

\input{../tex-inputs/latex-vorspann}


               \section[Arthur Schnitzler an Robert Adam, 9. 4. 1927]{ Arthur Schnitzler an Robert Adam, 9. 4. 1927}\nopagebreak\mylabel{v}\rehead{ }\normalsize\beginnumbering\briefempfaengerindex{Adam, Robert@\textsc{Adam, Robert}!zzzSchnitzler, Arthur@\emph{von Arthur Schnitzler}!1927-04-091@{9. 4. 1927}|(be} \toendnotes[C]{\smallbreak\pagebreak[2]} \Standort{DLA, 96.34.2/29.}
\physDesc{Postkarte
\newline{}Handschrift: schwarze Tinte, lateinische Kurrent\newline{}Versand: Stempel: »\nobreak{}9. IV. \textcolor{gray}{27}\nobreak{}«.  }\toendnotes[C]{\smallbreak}\pstart{}{\pb}\label{T_L02484-1v}\edtext{\textcolor{gray}{\textbf{A. S.}}}{\lemma{\textnormal{\emph{A. S.}}}\Cendnote{\textnormal{ovaler Absenderkleber}}}\label{T_L02484-1h}\pend{}\pstart{}\textcolor{pink}{\textcolor{gray}{\textbf{WIEN, XVIII.}}}{}\ledrightnote{\textcolor{pink}{XVIII., Währing}}\pend{}\pstart{}\textcolor{pink}{\textcolor{gray}{\textbf{STERNWARTESTR. 71}}}{}\ledrightnote{\textcolor{pink}{Sternwartestraße}}\pend{}{\bigskip}\pstart{}H. Dr. Robert Adam Pollak\pend{}\pstart{}Ob.-Landesger-Rath\pend{}\pstart{}\textcolor{pink}{XII Wien Meidling}{}\ledrightnote{\textcolor{pink}{XII., Meidling}}\pend{}\pstart{}\textcolor{pink}{Meidlinger Hptstr 54}{}\ledrightnote{\textcolor{pink}{Meidlinger Hauptstraße}}.\pend{}{\bigskip}\pstart
           \raggedleft{}{\pb}\textcolor{pink}{Wien}{}\ledrightnote{\textcolor{pink}{Wien}}, 9. 4. 927\pend
           \pstart
           lieber und verehrter Herr Doctor, entschuldigen Sie dſs ich
                    erst heute, u überdies auch mit ein paar flachligen Worten nur den Empfang Ihres
                    interessanten u liebenswürdigen Briefes bestätige, der mit seinen Bedenken, wie
                    nicht anders zu erwarten, gleich das Zentrum meiner kleinen \textcolor{green}{Arbeit}{}\ledrightnote{→\textcolor{green}{Der Geist im Wort und der Geist in der Tat}} trifft. Sie haben gewiſs recht,
                    daſs es sich nie um eine \uline{Idee} handelt – aber ob
                    nicht zugleich um etwas, das mit \textcolor{gray}{Recht} persönlicher \uline{Erfahrung} schon nah verwandt ist, wäre
                    vielleicht zu erwägen. Ohne Erfahrung – gäbe es da{\geminationn}
                    überhaupt eine Idee? – Doch das läßt {\pb}sich nicht auf dem
                    Correspondenzwege (und überhaupt nicht endgiltig) erläutern. Vielleicht haben
                    Sie, bei schönem Wetter, im späten Frühjahr einmal ein Stündchen Zeit für mich,
                    ich denke an unsere Gespräche und an Sie selbst verehrter Herr Doktor in
                    herzlicher Sympathie zurück.\pend
           \pstart Viele Grüſſe Ihr \spacefill\mbox{ArthSchnitzler}\pend{}\endnumbering\briefempfaengerindex{Adam, Robert@\textsc{Adam, Robert}!zzzSchnitzler, Arthur@\emph{von Arthur Schnitzler}!1927-04-091@{9. 4. 1927}|)be}\mylabel{h}  \normalsize

\doendnotes{C}
\bigskip
\vfill

\clearpage

\footnotesize

\lohead{\textsc{register}}

% Definiere theindex-Environment komplett neu ohne reledmac
\makeatletter
\renewenvironment{theindex}{%
  \section*{\indexname}%
  \setlength{\parindent}{0pt}%
  \setlength{\parskip}{0pt plus 0.3pt}%
  \let\item\@idxitem
}{%
  \clearpage
}
\makeatother

\IfFileExists{\jobname-pw.ind}{\input{\jobname-pw.ind}}{}

\end{document}

      