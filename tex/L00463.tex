%% latex-korrekturansicht-vorspann.tex
%% Vorspann für die Korrekturansicht.
%% Lädt die gemeinsame Datei latex-vorspann.tex mit gesetztem Schalter.

\newif\ifkorrekturansicht
\korrekturansichttrue

\input{../tex-inputs/latex-vorspann}


               \section[Arthur Schnitzler an Hermann Bahr, 17. 7. 1895]{ Arthur Schnitzler an Hermann Bahr, 17. 7. 1895}\nopagebreak\mylabel{v}\rehead{ }\normalsize\beginnumbering\briefempfaengerindex{Bahr, Hermann@\textsc{Bahr, Hermann}!zzzSchnitzler, Arthur@\emph{von Arthur Schnitzler}!1895-07-171@{17. 7. 1895}|(be} \toendnotes[C]{\smallbreak\pagebreak[2]} \Standort{TMW, HS AM 23324 Ba.}
\physDesc{Brief, 1 Blatt, 3 Seiten
\newline{}Handschrift: schwarze Tinte, deutsche Kurrent\newline{}Ordnung: Lochung }\buchAbdrucke{\weitereDrucke{1) \emph{17. 7. 1895.} In: Arthur Schnitzler: \emph{The Letters of Arthur Schnitzler to Hermann Bahr}. Edited, annotated, and with an introduction, by Donald G.
                        Daviau. Chapel Hill: \emph{The University of North Carolina Press} 1978, S. 58 (University of North Carolina studies in the Germanic languages
                        and literatures, 89).} \weitereDrucke{2) Hermann Bahr, Arthur Schnitzler: \emph{Briefwechsel, Aufzeichnungen, Dokumente (1891–1931)}. Hg. Kurt Ifkovits und Martin Anton Müller. Göttingen: \emph{Wallstein} 2018, S. 103.} }\toendnotes[C]{\smallbreak}\pstart{}{\pb}Lieber
                  Hermann, \pend\pstart
           hier iſt alſo die \textcolor{green}{Novelle}{}\ledrightnote{→\textcolor{green}{Später Ruhm}}. Ich
               habe viel geſtrichen, fürchte aber noch i{\geminationm}er dß ſie zu
               lang iſt. In dieſem Falle hätte ich nichts dagegen, daſs ſie in kleinerm Drucke
               erſcheint. (Wie ſ. Z. \label{K_L00463_1v}\edtext{\textcolor{green}{\textcolor{blue}{\textsc{Saar}}{}\ledrightnote{\textcolor{blue}{Ferdinand von Saar}}}{}\ledrightnote{→\textcolor{green}{Herr Fridolin und sein Glück}}}{\lemma{\textnormal{\emph{Saar}}}\Cendnote{\textnormal{\textcolor{blue}{Ferdinand von Saar}: \emph{\textcolor{green}{Herr Fridolin und sein Glück}}. In: \emph{\textcolor{green}{Die Zeit}}, Bd. 1, Nr. 1, 6. 10. 1894 – Nr. 5,
                        3. 11. 1894 (5 Teile).}}}\label{K_L00463_1h}.) Findeſt Du noch Stellen,
               die Du für entbehrlich hältſt, ſo gib ſie mir vielleicht an, ſtreiche aber
               keinesfalls ſelbſt. {\pb}Auch wenn dir ein wirkſamerer Titel einfiele, ſo wäre mir das ſehr willko{\geminationm}en. –\pend
           \pstart
           Kannſt Du die Geſchichte nicht brauchen, ſo behalte das \textsc{Manuscr}. jedenfalls freundlichſt bei Dir, bis ich nach \textcolor{pink}{Wien}{}\ledrightnote{\textcolor{pink}{Wien}} zurückkehre. Nachrichten erbitte ich mir an untenſtehende
               Adreſſe. \textcolor{blue}{Richard}{}\ledrightnote{\textcolor{blue}{Richard Beer-Hofmann}}{ }ſagt mir übrigens, dß Du bald
                  {\pb}wieder her ko{\geminationm}ſt, da ſprechen wir uns wohl, was mich ſehr freuen
               wird.\pend
           \pstart Herzliche Grüße von Deinem ergeb \spacefill\mbox{ArthSch}\pend{}\pstart
           1\substVorne{}\textsuperscript{6}\substDazwischen{}7\substHinten{}/7. 95{\\}\textsc{\textcolor{pink}{Ischl, Rudolfshöhe}{}\ledrightnote{\textcolor{pink}{Hotel und Pension Rudolfshöhe (Leopold Petter)}}.}\pend
           \endnumbering\briefempfaengerindex{Bahr, Hermann@\textsc{Bahr, Hermann}!zzzSchnitzler, Arthur@\emph{von Arthur Schnitzler}!1895-07-171@{17. 7. 1895}|)be}\mylabel{h}  \normalsize

\doendnotes{C}
\bigskip
\vfill

\clearpage

\footnotesize

\lohead{\textsc{register}}

% Definiere theindex-Environment komplett neu ohne reledmac
\makeatletter
\renewenvironment{theindex}{%
  \section*{\indexname}%
  \setlength{\parindent}{0pt}%
  \setlength{\parskip}{0pt plus 0.3pt}%
  \let\item\@idxitem
}{%
  \clearpage
}
\makeatother

\IfFileExists{\jobname-pw.ind}{\input{\jobname-pw.ind}}{}

\end{document}

      