%% latex-korrekturansicht-vorspann.tex
%% Vorspann für die Korrekturansicht.
%% Lädt die gemeinsame Datei latex-vorspann.tex mit gesetztem Schalter.

\newif\ifkorrekturansicht
\korrekturansichttrue

\input{../tex-inputs/latex-vorspann}


               \section[Richard Beer-Hofmann an Arthur Schnitzler, 14. 10. 1892]{ Richard Beer-Hofmann an Arthur Schnitzler, 14. 10. 1892}\nopagebreak\mylabel{v}\rehead{ }\normalsize\beginnumbering\briefempfaengerindex{Schnitzler, Arthur@\textsc{Schnitzler, Arthur}!zzzBeer-Hofmann, Richard@\emph{von Richard Beer-Hofmann}!1892-10-141@{14. 10. 1892}|(be} \toendnotes[C]{\smallbreak\pagebreak[2]} \Standort{CUL, Schnitzler, B 8.}
\physDesc{Brief, 1 Blatt, 2 Seiten
\newline{}Handschrift: blauer Buntstift, lateinische Kurrent
\newline{}Schnitzler: mit Bleistift datiert: »14/10 92« \newline{}Ordnung: mit Bleistift von unbekannter Hand nummeriert: »11« }\buchAbdrucke{\weitereDrucke{Arthur Schnitzler, Richard Beer-Hofmann: \emph{Briefwechsel 1891–1931}. Hg. Konstanze Fliedl. Wien, Zürich: \emph{Europaverlag} 1992, S. 39–40.} }\toendnotes[C]{\smallbreak}\pstart{}{\pb}Lieber Arthur!\pend\pstart
           Ich bin seit gestern hier; Ich möchte heute zur »\label{K_L00127_1v}\edtext{\textcolor{green}{Cameliendame}{}\ledrightnote{\textcolor{green}{Die Kameliendame. Drama in  fünf Akten}}}{\lemma{\textnormal{\emph{Cameliendame}}}\Cendnote{\textnormal{Dass \textcolor{blue}{Schnitzler}
                  sich in das Gastspiel von \textcolor{blue}{Sarah
                     Bernhardt} am \textcolor{pink}{Theater an der Wien}
                  begab, ist eher auszuschließen.}}}\label{K_L00127_1h}« gehen; wenn
               es Ihnen möglich ist ko{\geminationm}en Sie so um \label{K_L00127_2v}\edtext{¼ 6}{\lemma{\textnormal{\emph{¼ 6}}}\Cendnote{\textnormal{17 Uhr 15}}}\label{K_L00127_2h} zu mir und
               bringen mir dabei auch mein Opernglas mit.\pend
           \pstart
           {\pb}Sie waren doch noch nicht
               dabei?\pend
           \pstart
           Ich warte also bis ¼ 6.\pend
           \pstart
           Herzlichst{\\[\baselineskip]}\spacefill\mbox{Richard}\pend
           \leftskip=0em{}\pstart
           14/X 92\pend
           \pstart
           Pardon für die zwei »dabei«.\pend
           \endnumbering\briefempfaengerindex{Schnitzler, Arthur@\textsc{Schnitzler, Arthur}!zzzBeer-Hofmann, Richard@\emph{von Richard Beer-Hofmann}!1892-10-141@{14. 10. 1892}|)be}\mylabel{h}  \normalsize

\doendnotes{C}
\bigskip
\vfill

\clearpage

\footnotesize

\lohead{\textsc{register}}

% Definiere theindex-Environment komplett neu ohne reledmac
\makeatletter
\renewenvironment{theindex}{%
  \section*{\indexname}%
  \setlength{\parindent}{0pt}%
  \setlength{\parskip}{0pt plus 0.3pt}%
  \let\item\@idxitem
}{%
  \clearpage
}
\makeatother

\IfFileExists{\jobname-pw.ind}{\input{\jobname-pw.ind}}{}

\end{document}

      