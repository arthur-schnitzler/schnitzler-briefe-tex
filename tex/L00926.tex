%% latex-korrekturansicht-vorspann.tex
%% Vorspann für die Korrekturansicht.
%% Lädt die gemeinsame Datei latex-vorspann.tex mit gesetztem Schalter.

\newif\ifkorrekturansicht
\korrekturansichttrue

\input{../tex-inputs/latex-vorspann}


               \section[Richard Beer-Hofmann an Arthur Schnitzler, 16. {[}6.{]} 1899]{ Richard Beer-Hofmann an Arthur Schnitzler,
               16. {[}6.{]} 1899}\nopagebreak\mylabel{v}\rehead{ }\normalsize\beginnumbering\briefempfaengerindex{Schnitzler, Arthur@\textsc{Schnitzler, Arthur}!zzzBeer-Hofmann, Richard@\emph{von Richard Beer-Hofmann}!1899-06-161@{16. {[}6.{]} 1899}|(be} \toendnotes[C]{\smallbreak\pagebreak[2]} \Standort{CUL, Schnitzler, B 8.}
\physDesc{Brief, 2 Blätter, 8 Seiten
\newline{}Handschrift: Bleistift, lateinische Kurrent
\newline{}Schnitzler: mit rotem Buntstift die Monatszahl »VII« zu
               »6« korrigiert \newline{}Ordnung: mit Bleistift von unbekannter Hand
                           nummeriert: »129« }\buchAbdrucke{\weitereDrucke{Arthur Schnitzler, Richard Beer-Hofmann: \emph{Briefwechsel 1891–1931}. Hg. Konstanze Fliedl. Wien, Zürich: \emph{Europaverlag} 1992, S. 129–130.} }\toendnotes[C]{\smallbreak}\pstart
           \raggedleft{}{\pb}\textcolor{pink}{Seeboden}{}\ledrightnote{\textcolor{pink}{Seeboden}}{ }16/VII 1899.\pend
           \pstart
           Lieber Arthur! ich schreibe Ihnen an einem jener »Abende am Wasser«
               die Sie so fürchten, und die ich nicht sehr liebe. Auf den Bergen liegt neuer Schnee,
               tagsüber hat’s geregnet und in der Villa nebenan spielen 4 junge Mäd{\pb}chen bei offenem Fenster Clavier,
               singen »\textcolor{green}{ich bin eine Wittwe}{}\ledrightnote{\textcolor{green}{Die kleine Witwe}}« und tollen mit einer
               empörenden Lustigkeit umher die alles nur nicht jung und unbefangen ist.\pend
           \pstart
           Ich wollte mit meiner Antwort warten bis ich in besserer Sti{\geminationm}ung wäre; aber wann {\pb}wird das sein? Ich bin recht
                  versti{\geminationm}t und traurig; aus vielen Gründen; aus solchen
                  \strikeout{ke} die ich kenne und aus vielen anderen die ich
               nicht kenne, die aber sicher vorhanden sind und gegen die man noch machtloser {\pb}ist als gegen die anderen. Von \textcolor{blue}{Mayer}{}\ledrightnote{\textcolor{blue}{Oskar Mayer}} hatte ich dieser Tage Brief; er wollte
               näheres von mir hören wann wir unsere Fußpartie machen würden.\pend
           \pstart
           Am selben Tag habe ich einen Brief aus \textcolor{pink}{Wien}{}\ledrightnote{\textcolor{pink}{Wien}} erhalten
               daß Professor \textcolor{blue}{Fuchs}{}\ledrightnote{\textcolor{blue}{Ernst Fuchs}}{ }{\pb}bei meinem \textcolor{blue}{Vater}{}\ledrightnote{→\textcolor{blue}{Hermann Beer}} (– D\textsuperscript{r}{ }\textcolor{blue}{Beer}{}\ledrightnote{\textcolor{blue}{Hermann Beer}} –) grauen Staar diagnosticirte. Ich erhielt
               die Nachricht indirekt und wußte daher absolut nicht wie oder wo ich meinen So{\geminationm}er verbringen würde. Habe daher an \textcolor{blue}{Mayer}{}\ledrightnote{\textcolor{blue}{Oskar Mayer}} nur kurz geschrieben {\pb}daß ich momentan nicht über meine
               Zeit disponiren könne.\pend
           \pstart
           Inzwischen habe ich bessere Nachrichten von meinem \textcolor{blue}{Vater}{}\ledrightnote{→\textcolor{blue}{Hermann Beer}}; es hat noch 1–2 Jahre eventuell Zeit mit einer
               Operation u sein moralischer Zustand ist kein schlechter. {\pb}Sollten Sie \textcolor{blue}{Mayer}{}\ledrightnote{\textcolor{blue}{Oskar Mayer}} sehen so besprechen Sie mit ihm das Nötige wegen einer
               Fußtour; ich schließe mich an.\pend
           \pstart
           Wann wollen Sie hieher ko{\geminationm}en? Schreiben Sie mir früher
               damit ich Zi{\geminationm}er etc. versorge. Vielleicht hole ich Sie
               an irgend einer Bahnstation ab.\pend
           \pstart
           {\pb}Bitte wie ist \textcolor{blue}{Paul}{}\ledrightnote{\textcolor{blue}{Paul Goldmann}}s Adresse in \textcolor{pink}{\uline{Frankfurt}}{}\ledrightnote{\textcolor{pink}{Frankfurt am Main}}? Grüßen Sie \textcolor{blue}{Schwarzkopf}{}\ledrightnote{\textcolor{blue}{Gustav Schwarzkopf}} und \textcolor{blue}{Hugo}{}\ledrightnote{\textcolor{blue}{Hugo von Hofmannsthal}}. Von Herzen\pend
           \pstart Ihr \spacefill\mbox{Richard}\pend{}\endnumbering\briefempfaengerindex{Schnitzler, Arthur@\textsc{Schnitzler, Arthur}!zzzBeer-Hofmann, Richard@\emph{von Richard Beer-Hofmann}!1899-06-161@{16. {[}6.{]} 1899}|)be}\mylabel{h}  \normalsize

\doendnotes{C}
\bigskip
\vfill

\clearpage

\footnotesize

\lohead{\textsc{register}}

% Definiere theindex-Environment komplett neu ohne reledmac
\makeatletter
\renewenvironment{theindex}{%
  \section*{\indexname}%
  \setlength{\parindent}{0pt}%
  \setlength{\parskip}{0pt plus 0.3pt}%
  \let\item\@idxitem
}{%
  \clearpage
}
\makeatother

\IfFileExists{\jobname-pw.ind}{\input{\jobname-pw.ind}}{}

\end{document}

      