%% latex-korrekturansicht-vorspann.tex
%% Vorspann für die Korrekturansicht.
%% Lädt die gemeinsame Datei latex-vorspann.tex mit gesetztem Schalter.

\newif\ifkorrekturansicht
\korrekturansichttrue

\input{../tex-inputs/latex-vorspann}


               \section[Karl Kraus an Arthur Schnitzler, 8. 7. 1894]{ Karl Kraus an Arthur Schnitzler, 8. 7. 1894}\nopagebreak\mylabel{v}\rehead{ }\normalsize\beginnumbering\briefempfaengerindex{Schnitzler, Arthur@\textsc{Schnitzler, Arthur}!zzzKraus, Karl@\emph{von Karl Kraus}!1894-07-081@{9. 7. 1894}|(be} \toendnotes[C]{\smallbreak\pagebreak[2]} \Standort{CUL, Schnitzler, B 55.}
\physDesc{Postkarte
\newline{}Handschrift: Bleistift, deutsche Kurrent\newline{}Versand: 1) Stempel: »\nobreak{}\oindex{Bad Ischl@\textbf{Bad Ischl}, \emph{Besiedelter Ort (A.BSO)}|pwk}Ischl, 9/7 94, 7–F\nobreak{}«.  2) Stempel: »\nobreak{}\oindex{IX., Alsergrund@\textbf{IX., Alsergrund}, \emph{Bezirk (A.BZK)}|pwk}Wien 9/\textcolor{gray}{3}, 10. 7. 94, 8.V, Beste{[}llt{]}\nobreak{}«. 
\newline{}Schnitzler: mit Bleistift datiert: »9/7 94« }\buchAbdrucke{\weitereDrucke{\emph{Karl Kraus und Arthur Schnitzler. Eine Dokumentation.} Hg. Reinhard Urbach. In: \emph{Literatur und Kritik}, Bd. 49, Oktober 1970, S. 521.} }\toendnotes[C]{\smallbreak}\pstart{}{\pb}Herrn\pend{}\pstart{}D\textsuperscript{r} Arthur Schnitzler\pend{}\pstart{}\textcolor{pink}{Wien IX.}{}\ledrightnote{\textcolor{pink}{IX., Alsergrund}}\pend{}\pstart{}\textcolor{pink}{Frankgasse 1}{}\ledrightnote{\textcolor{pink}{Frankgasse}}\pend{}{\bigskip}\pstart
           \noindent{}{\pb}Lieber Schnitzler, im »\textcolor{green}{Prager
                        Tagblatt}{}\ledrightnote{\textcolor{green}{Prager Tagblatt}}« vom \uline{Samstag}, 7.{ }ſteht eine (halb günſtige) \label{K_L00348_1v}\edtext{\textcolor{green}{Kritik}{}\ledrightnote{→\textcolor{green}{Das Märchen}}}{\lemma{\textnormal{\emph{Kritik}}}\Cendnote{\textnormal{[O. V.:] \emph{\textcolor{green}{Das Märchen}}. In: \emph{\textcolor{green}{Prager Tagblatt}}, Jg. 18, Nr. 185,
                                7. 7. 1894, S. 8.}}}\label{K_L00348_1h} Ihres »\textcolor{green}{Märchen}{}\ledrightnote{\textcolor{green}{Das Märchen. Schauspiel in drei Aufzügen}}«. Ich wollt’ Ihnen den Ausschnitt ſchicken, erfahre
                    aber eben, daſs das Blatt hier subabonniert ist. Seien Sie mir herzlichst
                    gegrüßt! Hoffentlich ſehen wir uns bald. Ihr \spacefill\mbox{Kraus,}\pend
           \pstart
           {[}({]}\textcolor{pink}{Ischl, Grazerſtr 133}{}\ledrightnote{\textcolor{pink}{Grazer Straße}}, \textcolor{pink}{Café Walter}{}\ledrightnote{\textcolor{pink}{Café Walther}}, 8. VII.)\pend
           \pstart
           \label{T_L00348_1v}\edtext{Der kl. \textcolor{blue}{\uline{Rosner}}{}\ledrightnote{\textcolor{blue}{Karl Peter Rosner}} fragt mich heute nach Ihrer Adreſſe; er will Ihnen ſeine »\textcolor{green}{Gefühle}{}\ledrightnote{\textcolor{green}{Gefühle}}« ſchicken.}{\lemma{\textnormal{\emph{Der … ſchicken.}}}\Cendnote{\textnormal{quer am rechten Rand}}}\label{T_L00348_1h}\pend
           \endnumbering\briefempfaengerindex{Schnitzler, Arthur@\textsc{Schnitzler, Arthur}!zzzKraus, Karl@\emph{von Karl Kraus}!1894-07-081@{9. 7. 1894}|)be}\mylabel{h}  \normalsize

\doendnotes{C}
\bigskip
\vfill

\clearpage

\footnotesize

\lohead{\textsc{register}}

% Definiere theindex-Environment komplett neu ohne reledmac
\makeatletter
\renewenvironment{theindex}{%
  \section*{\indexname}%
  \setlength{\parindent}{0pt}%
  \setlength{\parskip}{0pt plus 0.3pt}%
  \let\item\@idxitem
}{%
  \clearpage
}
\makeatother

\IfFileExists{\jobname-pw.ind}{\input{\jobname-pw.ind}}{}

\end{document}

      