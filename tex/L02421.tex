%% latex-korrekturansicht-vorspann.tex
%% Vorspann für die Korrekturansicht.
%% Lädt die gemeinsame Datei latex-vorspann.tex mit gesetztem Schalter.

\newif\ifkorrekturansicht
\korrekturansichttrue

\input{../tex-inputs/latex-vorspann}


               \section[Felix Braun an Arthur Schnitzler, 9. 12. 1924]{ Felix Braun an Arthur Schnitzler, 9. 12. 1924}\nopagebreak\mylabel{v}\rehead{ }\normalsize\beginnumbering\briefempfaengerindex{Schnitzler, Arthur@\textsc{Schnitzler, Arthur}!zzzBraun, Felix@\emph{von Felix Braun}!1924-12-091@{9. 12. 1924}|(be} \toendnotes[C]{\smallbreak\pagebreak[2]} \Standort{DLA, A:Schnitzler, HS.NZ85.1.2604,4.}
\physDesc{Brief, 1 Blatt, 2 Seiten
\newline{}Handschrift: schwarze Tinte, deutsche Kurrent
\newline{}Schnitzler: 1) mit Bleistift beschriftet: »\textsc{Braun}« 2) mit rotem Buntstift mehrere Unterstreichungen}\toendnotes[C]{\smallbreak}\pstart
           \centering{}{\pb}\textcolor{pink}{Salzburg}{}\ledrightnote{\textcolor{pink}{Salzburg}} / 9. XII. 1924\pend
           \pstart{}Verehrter Herr Doktor!\pend\pstart
           Statt Ihnen für die liebe Gabe Ihres neuen \textcolor{green}{Buches}{}\ledrightnote{→\textcolor{green}{Fräulein Else}} zu danken, komme ich mit einer Bitte, die nun
                    wohl eben dieſes \textcolor{green}{Buch}{}\ledrightnote{→\textcolor{green}{Fräulein Else}}
                    betrifft. Ich habe es nämlich – nicht erhalten, man hat es mir von \textcolor{pink}{Wien}{}\ledrightnote{\textcolor{pink}{Wien}} hieher, wo ich für einige Tage \textcolor{blue}{Stefan Zweigs}{}\ledrightnote{\textcolor{blue}{Stefan Zweig}} Stellvertreter war, nachgeſandt und da hat
                    es ein autographenſammelnder Poſtbeamter an ſich genommen – ich hoffe leider
                    nicht mehr auf den Wiedergewinn des mir durch Ihre Inſchrift doppelt wertvollen
                        \textcolor{green}{Buchs}{}\ledrightnote{→\textcolor{green}{Fräulein Else}}. Darf ich Ihnen
                    nun die Bitte unterbreiten, in das Exemplar, das ich Ihnen ſenden werde, mir
                    wieder eine Widmung einzuſchreiben? Ich wäre Ihnen ſehr, ſehr dankbar dafür. In
                    einer Woche etwa bin ich wieder zu Hauſe. {\pb}So
                    käme mir das erneute Geſchenk gerade als Weihnachtsgabe recht.\pend
           \pstart
           Für die ehrenvolle Freude, die Sie mir zugedacht haben, ſage ich Ihnen, verehrter
                    Herr Doktor, beſten Dank und ſo bleibe ich Ihr herzlich ergebener\pend
           \pstart \spacefill\mbox{Felix Braun.}\pend{}\endnumbering\briefempfaengerindex{Schnitzler, Arthur@\textsc{Schnitzler, Arthur}!zzzBraun, Felix@\emph{von Felix Braun}!1924-12-091@{9. 12. 1924}|)be}\mylabel{h}  \normalsize

\doendnotes{C}
\bigskip
\vfill

\clearpage

\footnotesize

\lohead{\textsc{register}}

% Definiere theindex-Environment komplett neu ohne reledmac
\makeatletter
\renewenvironment{theindex}{%
  \section*{\indexname}%
  \setlength{\parindent}{0pt}%
  \setlength{\parskip}{0pt plus 0.3pt}%
  \let\item\@idxitem
}{%
  \clearpage
}
\makeatother

\IfFileExists{\jobname-pw.ind}{\input{\jobname-pw.ind}}{}

\end{document}

      