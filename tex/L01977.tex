%% latex-korrekturansicht-vorspann.tex
%% Vorspann für die Korrekturansicht.
%% Lädt die gemeinsame Datei latex-vorspann.tex mit gesetztem Schalter.

\newif\ifkorrekturansicht
\korrekturansichttrue

\input{../tex-inputs/latex-vorspann}


               \section[Arthur Schnitzler an Richard Beer-Hofmann, 3. 11. 1910]{ Arthur Schnitzler an Richard Beer-Hofmann, 3. 11. 1910}\nopagebreak\mylabel{v}\rehead{ }\normalsize\beginnumbering\briefempfaengerindex{Beer-Hofmann, Richard@\textsc{Beer-Hofmann, Richard}!zzzSchnitzler, Arthur@\emph{von Arthur Schnitzler}!1910-11-032@{3. 11. 1910}|(be} \toendnotes[C]{\smallbreak\pagebreak[2]} \Standort{YCGL, MSS 31.}
\physDesc{Brief, 1 Blatt, 3 Seiten, Umschlag
\newline{}Handschrift: Bleistift, deutsche Kurrent\newline{}Versand: ohne postalischen Übermittlungsvermerk }\buchAbdrucke{\weitereDrucke{Arthur Schnitzler, Richard Beer-Hofmann: \emph{Briefwechsel 1891–1931}. Hg. Konstanze Fliedl. Wien, Zürich: \emph{Europaverlag} 1992, S. 213.} }\toendnotes[C]{\smallbreak}\pstart{}{\pb}\textcolor{gray}{\textbf{Dr. Arthur Schnitzler}}\pend{}\pstart{}\textcolor{gray}{\textbf{\textcolor{pink}{Wien XVIII. Sternwartestrasse 71}{}\ledrightnote{\textcolor{pink}{Sternwartestraße}}}}\pend{}{\bigskip}\pstart{}{\pb}\textsc{Herrn Dr. Richard Beer Hofmann}\pend{}\pstart{}\textcolor{pink}{Wien XVIII}{}\ledrightnote{\textcolor{pink}{XVIII., Währing}}\pend{}\pstart{}\textsc{\textcolor{pink}{Hasenauerstr 59}{}\ledrightnote{\textcolor{pink}{Hasenauerstraße}}}\pend{}{\bigskip}\pstart
           \noindent{}{\pb}\textcolor{gray}{\textbf{Dr. Arthur Schnitzler}}\hfill 3/11 910\pend
           \pstart
           \textcolor{gray}{\textbf{\textcolor{pink}{Wien XVIII. \label{K_L01977-1v}\edtext{Spoettelgasse 7}{\lemma{\textnormal{\emph{Spoettelgasse 7}}}\Cendnote{\textnormal{Schnitzler verwendet das nicht mehr aktuelle
                           Briefpapier.}}}\label{K_L01977-1h}}{}\ledrightnote{\textcolor{pink}{Edmund-Weiß-Gasse}}.}}\pend
           \pstart{}lieber Richard, \pend\pstart
           für den »\textcolor{brown}{Mutterſchutz}{}\ledrightnote{\textcolor{brown}{Bund für Mutterschutz}}« ka{\geminationn} ich nicht leſen – weil ich ſonſt auch für ſo u ſoviel
               andre Vereine leſen müßte, {\pb}die mich ſchon
               aufgefordert haben u noch auffordern werden; u. weil ich überhaupt in \textcolor{pink}{Wien}{}\ledrightnote{\textcolor{pink}{Wien}} nicht gern leſe.\pend
           \pstart
           \textcolor{blue}{Leo}{}\ledrightnote{\textcolor{blue}{Leo Van-Jung}} ni{\geminationm}ts mir gewiſs
               nicht übel.\pend
           \pstart
           {\pb}Herzlichſt Ihr{\\[\baselineskip]}\spacefill\mbox{A.}\pend
           \leftskip=0em{}\endnumbering\briefempfaengerindex{Beer-Hofmann, Richard@\textsc{Beer-Hofmann, Richard}!zzzSchnitzler, Arthur@\emph{von Arthur Schnitzler}!1910-11-032@{3. 11. 1910}|)be}\mylabel{h}  \normalsize

\doendnotes{C}
\bigskip
\vfill

\clearpage

\footnotesize

\lohead{\textsc{register}}

% Definiere theindex-Environment komplett neu ohne reledmac
\makeatletter
\renewenvironment{theindex}{%
  \section*{\indexname}%
  \setlength{\parindent}{0pt}%
  \setlength{\parskip}{0pt plus 0.3pt}%
  \let\item\@idxitem
}{%
  \clearpage
}
\makeatother

\IfFileExists{\jobname-pw.ind}{\input{\jobname-pw.ind}}{}

\end{document}

      