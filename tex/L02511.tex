%% latex-korrekturansicht-vorspann.tex
%% Vorspann für die Korrekturansicht.
%% Lädt die gemeinsame Datei latex-vorspann.tex mit gesetztem Schalter.

\newif\ifkorrekturansicht
\korrekturansichttrue

\input{../tex-inputs/latex-vorspann}


               \section[Arthur Schnitzler: Widmungsexemplar Spiel im Morgengrauen für Hugo von Hofmannsthal, 11. 6. 1929]{ Arthur Schnitzler: Widmungsexemplar Spiel im Morgengrauen für Hugo von
               Hofmannsthal, 11. 6. 1929}\nopagebreak\mylabel{v}\rehead{ }\normalsize\beginnumbering\briefempfaengerindex{Hofmannsthal, Hugo von@\textsc{Hofmannsthal, Hugo von}!zzzSchnitzler, Arthur@\emph{von Arthur Schnitzler}!1929-06-111@{11. 6. 1929}|(be} \toendnotes[C]{\smallbreak\pagebreak[2]} \Standort{FDH, FDH 1938.}
\physDesc{Widmung am Vorsatzblatt
\newline{}Handschrift: schwarze Tinte, lateinische Kurrent}\buchAbdrucke{\weitereDrucke{Hugo von Hofmannsthal: \emph{Bibliothek}. Hg. Ellen Ritter † in Zusammenarbeit mit Dalia Bukauskaité und
                        Konrad Heumann. Frankfurt am Main: \emph{S. Fischer} 2011, S. 606 (Sämtliche Werke. Kritische Ausgabe, XL).} }\toendnotes[C]{\smallbreak}\pstart
           \noindent{}{\pb}Meinem lieben Hugo\pend
           \pstart
           herzlich{\\[\baselineskip]}\introOben{}wie immer\introOben{}{\\[\baselineskip]}\spacefill\mbox{Arthur}\pend
           \leftskip=0em{}\pstart
           \textcolor{pink}{Wien}{}\ledrightnote{\textcolor{pink}{Wien}}{ }\label{K_L02511_1v}\edtext{11. 6. 929}{\lemma{\textnormal{\emph{11. 6. 929}}}\Cendnote{\textnormal{am 9. 3. 1927 vom \emph{\textcolor{green}{Börsenblatt für den deutschen Buchhandel}}
                        als Neuerscheinung gemeldet}}}\label{K_L02511_1h}.\pend
           {\bigskip}\pstart
           \noindent{}\centering{}{\pb}\textcolor{gray}{\textbf{\textcolor{green}{SPIEL IM MORGENGRAUEN}{}\ledrightnote{\textcolor{green}{Spiel im Morgengrauen. Novelle}}}}\pend
           \pstart
           \noindent{}\centering{}\textcolor{gray}{\textbf{NOVELLE}}{\\}\textcolor{gray}{\textbf{VON}}{\\}\textcolor{gray}{\textbf{ARTHUR SCHNITZLER}}\pend
           {\bigskip}\pstart
           \noindent{}\centering{}\textcolor{gray}{\textbf{\textcolor{brown}{S. FISCHER / VERLAG}{}\ledrightnote{\textcolor{brown}{S. Fischer Verlag}} / \textcolor{pink}{BERLIN}{}\ledrightnote{\textcolor{pink}{Berlin}}}}\pend
           \endnumbering\briefempfaengerindex{Hofmannsthal, Hugo von@\textsc{Hofmannsthal, Hugo von}!zzzSchnitzler, Arthur@\emph{von Arthur Schnitzler}!1929-06-111@{11. 6. 1929}|)be}\mylabel{h}  \normalsize

\doendnotes{C}
\bigskip
\vfill

\clearpage

\footnotesize

\lohead{\textsc{register}}

% Definiere theindex-Environment komplett neu ohne reledmac
\makeatletter
\renewenvironment{theindex}{%
  \section*{\indexname}%
  \setlength{\parindent}{0pt}%
  \setlength{\parskip}{0pt plus 0.3pt}%
  \let\item\@idxitem
}{%
  \clearpage
}
\makeatother

\IfFileExists{\jobname-pw.ind}{\input{\jobname-pw.ind}}{}

\end{document}

      