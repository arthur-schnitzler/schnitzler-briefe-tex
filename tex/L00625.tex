%% latex-korrekturansicht-vorspann.tex
%% Vorspann für die Korrekturansicht.
%% Lädt die gemeinsame Datei latex-vorspann.tex mit gesetztem Schalter.

\newif\ifkorrekturansicht
\korrekturansichttrue

\input{../tex-inputs/latex-vorspann}


               \section[Arthur Schnitzler an Hermann Bahr, 28. 11. 1896]{ Arthur Schnitzler an Hermann Bahr, 28. 11. 1896}\nopagebreak\mylabel{v}\rehead{ }\normalsize\beginnumbering\briefempfaengerindex{Bahr, Hermann@\textsc{Bahr, Hermann}!zzzSchnitzler, Arthur@\emph{von Arthur Schnitzler}!1896-11-281@{28. 11. 1896}|(be} \toendnotes[C]{\smallbreak\pagebreak[2]} \Standort{TMW, HS AM 23327 Ba.}
\physDesc{Brief, 1 Blatt, 4 Seiten
\newline{}Handschrift: Bleistift, deutsche Kurrent}\buchAbdrucke{\weitereDrucke{1) \emph{28. 11. 1896.} In: Arthur Schnitzler: \emph{The Letters of Arthur Schnitzler to Hermann Bahr}. Edited, annotated, and with an introduction, by Donald G.
                        Daviau. Chapel Hill: \emph{The University of North Carolina Press} 1978, S. 59 (University of North Carolina studies in the Germanic languages
                        and literatures, 89).} \weitereDrucke{2) Hermann Bahr, Arthur Schnitzler: \emph{Briefwechsel, Aufzeichnungen, Dokumente (1891–1931)}. Hg. Kurt Ifkovits und Martin Anton Müller. Göttingen: \emph{Wallstein} 2018, S. 131.} }\toendnotes[C]{\smallbreak}\pstart
           \raggedleft{}{\pb}Samſtag \damage{2}8. 11. 96.\pend
           \pstart{}Lieber Hermann,\pend\pstart
           als ich neulich bei dir war, hab ich vergeſſen, Dir von \textcolor{blue}{Reicher}{}\ledrightnote{\textcolor{blue}{Emanuel Reicher}} etwas auszurichten, um was er mich in \textcolor{pink}{Berlin}{}\ledrightnote{\textcolor{pink}{Berlin}} gebeten hat. Er hat nemlich die {\pb}Abſicht, im Frühjahr mit einem Schauspielenſemble herzuko{\geminationm}en und einige hier noch nicht geſpielte Stücke
               aufzuführen, von denen er noch nicht weiſs, ob, \textsc{resp}. unter
               welchen Bedingungen die {\pb}Cenſur
               ſie freigeben wird. Er ſcheint auf deinen Rath, vielleicht auch auf deinen Beiſtand
               zu rechnen. Es handelt ſich vor allem um die \label{K_L00625_1v}\edtext{\textcolor{green}{Jugend}{}\ledrightnote{\textcolor{green}{Jugend. Ein Liebesdrama}}, ich glaube auch um die \textcolor{green}{Weber}{}\ledrightnote{\textcolor{green}{Die Weber. Schauspiel aus den vierziger Jahren}}}{\lemma{\textnormal{\emph{Jugend, … Weber}}}\Cendnote{\textnormal{\emph{\textcolor{green}{Jugend}} von \textcolor{blue}{Max
                     Halbe} konnte erst 1901, \emph{\textcolor{green}{Die
                     Weber}} von \textcolor{blue}{Gerhart Hauptmann} erst
                     1904 in \textcolor{pink}{Österreich} aufgeführt
                  werden.}}}\label{K_L00625_1h}. Näheres hat {\pb}er mir ſelbſt noch nicht geſagt; ich nehme an er \label{K_L00625_2v}\edtext{wird dir ſchreiben}{\lemma{\textnormal{\emph{wird dir ſchreiben}}}\Cendnote{\textnormal{Kein
                  in Frage kommender Brief liegt im Nachlass \textcolor{blue}{Bahrs}.}}}\label{K_L00625_2h}, und dieſe Zeilen bereiten dich nur darauf vor.\pend
           \pstart
           Herzlich grüßt dich{\\[\baselineskip]}dein \spacefill\mbox{Arthur Sch}\pend
           \leftskip=0em{}\endnumbering\briefempfaengerindex{Bahr, Hermann@\textsc{Bahr, Hermann}!zzzSchnitzler, Arthur@\emph{von Arthur Schnitzler}!1896-11-281@{28. 11. 1896}|)be}\mylabel{h}  \normalsize

\doendnotes{C}
\bigskip
\vfill

\clearpage

\footnotesize

\lohead{\textsc{register}}

% Definiere theindex-Environment komplett neu ohne reledmac
\makeatletter
\renewenvironment{theindex}{%
  \section*{\indexname}%
  \setlength{\parindent}{0pt}%
  \setlength{\parskip}{0pt plus 0.3pt}%
  \let\item\@idxitem
}{%
  \clearpage
}
\makeatother

\IfFileExists{\jobname-pw.ind}{\input{\jobname-pw.ind}}{}

\end{document}

      