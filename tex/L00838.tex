%% latex-korrekturansicht-vorspann.tex
%% Vorspann für die Korrekturansicht.
%% Lädt die gemeinsame Datei latex-vorspann.tex mit gesetztem Schalter.

\newif\ifkorrekturansicht
\korrekturansichttrue

\input{../tex-inputs/latex-vorspann}


               \section[Hugo von Hofmannsthal an Arthur Schnitzler, 25. 8. 1898]{ Hugo von Hofmannsthal an Arthur Schnitzler, 25. 8. 1898}\nopagebreak\mylabel{v}\rehead{ }\normalsize\beginnumbering\briefempfaengerindex{Schnitzler, Arthur@\textsc{Schnitzler, Arthur}!zzzHofmannsthal, Hugo von@\emph{von Hugo von Hofmannsthal}!1898-08-251@{25. 8. 1898}|(be} \toendnotes[C]{\smallbreak\pagebreak[2]} \Standort{CUL, Schnitzler, B 43.}
\physDesc{Postkarte
\newline{}Handschrift: Bleistift, deutsche Kurrent\newline{}Versand: 1) Stempel: »\nobreak{}\oindex{Lugano@\textbf{Lugano}, \emph{Besiedelter Ort (A.BSO)}|pwk}Lugano, 25. VIII. 98, XII\nobreak{}«.  2) Stempel: »\nobreak{}\oindex{Luzern@\textbf{Luzern}, \emph{Besiedelter Ort (A.BSO)}|pwk}Luzern Brf. Dist, 25. VIII. 98, 7\nobreak{}«. 
\newline{}Schnitzler: mit Bleistift datiert: »25/8 98« \newline{}Ordnung: 1) mit Bleistift von unbekannter Hand nummeriert: »\strikeout{121}« 2) mit Bleistift von unbekannter Hand nummeriert:
                                    »122«}\buchAbdrucke{\weitereDrucke{Hugo von Hofmannsthal, Arthur Schnitzler: \emph{Briefwechsel}. Hg. Therese Nickl und Heinrich Schnitzler. Frankfurt am Main: \emph{S. Fischer} 1964, S. 110–111.} }\toendnotes[C]{\smallbreak}\pstart{}{\pb}\textsc{Herrn D\textsuperscript{r} Arthur Schnitzler}\pend{}\pstart{}\textcolor{pink}{\textsc{Luzerne}}{}\ledrightnote{\textcolor{pink}{Luzern}}\pend{}\pstart{}\textsc{post. rest.}\pend{}{\bigskip}\pstart
           \raggedleft{}{\pb}\textcolor{pink}{Lugano}{}\ledrightnote{\textcolor{pink}{Hôtel du Parc}}, Do{\geminationn}erstg.\pend
           \pstart
           Ich arbeite nicht, war darüber in den erſten Tagen unſinnig verſti{\geminationm}t und niedergeſchlagen, jetzt hab ich mich
               dreingefunden und leb ſtill und angenehm, beſonders ſeit die furchtbare Schwüle
               aufgehört hat.\pend
           \pstart
           \textcolor{blue}{Richard}{}\ledrightnote{\textcolor{blue}{Richard Beer-Hofmann}} arbeitet »\label{K_L00838_1v}\edtext{mehr und leichter als je}{\lemma{\textnormal{\emph{mehr und leichter als je}}}\Cendnote{\textnormal{Im Brief vom 22. 8. 1898{ }schreibt \textcolor{blue}{Beer-Hofmann} an \textcolor{blue}{Hofmannsthal}:
                     »ich bin mitten in der Arbeit, arbeite leicht, und mehr als
                     sonst.« (Hugo von Hofmannsthal, Richard Beer-Hofmann: \emph{Briefwechsel}. Hg. Eugene Weber. Frankfurt am Main:
                        \emph{S. Fischer} 1972, S. 83)}}}\label{K_L00838_1h}« und dürfte den
                     31\textsuperscript{ten} hierher zu mir ko{\geminationm}en. Bitte \uline{bald} wieder Nachricht. Von Herzen Ihr \spacefill\mbox{Hugo.}\pend
           \endnumbering\briefempfaengerindex{Schnitzler, Arthur@\textsc{Schnitzler, Arthur}!zzzHofmannsthal, Hugo von@\emph{von Hugo von Hofmannsthal}!1898-08-251@{25. 8. 1898}|)be}\mylabel{h}  \normalsize

\doendnotes{C}
\bigskip
\vfill

\clearpage

\footnotesize

\lohead{\textsc{register}}

% Definiere theindex-Environment komplett neu ohne reledmac
\makeatletter
\renewenvironment{theindex}{%
  \section*{\indexname}%
  \setlength{\parindent}{0pt}%
  \setlength{\parskip}{0pt plus 0.3pt}%
  \let\item\@idxitem
}{%
  \clearpage
}
\makeatother

\IfFileExists{\jobname-pw.ind}{\input{\jobname-pw.ind}}{}

\end{document}

      