%% latex-korrekturansicht-vorspann.tex
%% Vorspann für die Korrekturansicht.
%% Lädt die gemeinsame Datei latex-vorspann.tex mit gesetztem Schalter.

\newif\ifkorrekturansicht
\korrekturansichttrue

\input{../tex-inputs/latex-vorspann}


               \section[Hugo von Hofmannsthal an Arthur Schnitzler, 1. 1. 1892]{ Hugo von Hofmannsthal an Arthur Schnitzler, 1. 1. 1892}\nopagebreak\mylabel{v}\rehead{ }\normalsize\beginnumbering\briefempfaengerindex{Schnitzler, Arthur@\textsc{Schnitzler, Arthur}!zzzHofmannsthal, Hugo von@\emph{von Hugo von Hofmannsthal}!1892-01-013@{1. 1. 1892}|(be} \toendnotes[C]{\smallbreak\pagebreak[2]} \Standort{CUL, Schnitzler, B 43.}
\physDesc{Kartenbrief
\newline{}Handschrift: Bleistift, deutsche Kurrent\newline{}Versand: Stempel: »\nobreak{}\oindex{III., Landstrasse@\textbf{III., Landstraße}, \emph{Bezirk (A.BZK)}|pwk}Wien 3/3, 1. 1. 92, 5–6 N\nobreak{}«.  
\newline{}Schnitzler: mit Bleistift datiert: »1/1 92« \newline{}Ordnung: mit Bleistift von unbekannter Hand nummeriert:
                                        »12« und auf der Rückseite der Adressseite
                                    zugefügt: »14.05 / 7.02 / 6.96 / 7.00 /
                                    13.60« }\buchAbdrucke{\weitereDrucke{1) Hugo von Hofmannsthal, Arthur Schnitzler: \emph{Briefwechsel}. Hg. Therese Nickl und Heinrich Schnitzler. Frankfurt am Main: \emph{S. Fischer} 1964, S. 14.} \weitereDrucke{2) Hermann Bahr, Arthur Schnitzler: \emph{Briefwechsel, Aufzeichnungen, Dokumente
                                (1891–1931)}. Hg. Kurt Ifkovits und Martin Anton Müller. Göttingen: \emph{Wallstein} 2018, S. 18–19.} }\toendnotes[C]{\smallbreak}\pstart{}{\pb}Herrn \textsc{D\textsuperscript{r} Arthur Schnitzler}\pend{}\pstart{}\textsc{\textcolor{pink}{Wien}{}\ledrightnote{\textcolor{pink}{Wien}}}\pend{}\pstart{}\textsc{\textcolor{pink}{I. Kärnthnerring 12.}{}\ledrightnote{\textcolor{pink}{Kärntnerring}}}\pend{}{\bigskip}\pstart{}{\pb}Lieber Freund!\pend\pstart
           \textcolor{blue}{Dörmann}{}\ledrightnote{\textcolor{blue}{Felix Dörmann}} will uns ſein neues \label{K_L00059_1v}\edtext{\textcolor{green}{Buch}{}\ledrightnote{→\textcolor{green}{Sensationen}}}{\lemma{\textnormal{\emph{Buch}}}\Cendnote{\textnormal{\textcolor{blue}{Felix Dörmann}: \emph{\textcolor{green}{Sensationen}}. Wien: \emph{Verlag von
                                Leopold Weiss}{ }1892.}}}\label{K_L00059_1h} vorleſen und hat mich gebeten, Sie einzuladen.\pend
           \pstart
           Wenn Sie alſo \label{K_L00059_2v}\edtext{nichts beſſeres}{\lemma{\textnormal{\emph{nichts beſſeres}}}\Cendnote{\textnormal{Schnitzler war bei der Lesung.}}}\label{K_L00059_2h}
                    vorhaben, kommen Sie morgen Samstag, \uline{½ 8} Uhr (pünktlich) \textcolor{brown}{Gewerbeverein}{}\ledrightnote{\textcolor{brown}{Österreichischer Gewerbeverein}}, \textcolor{pink}{Eſchenbachgaſſe}{}\ledrightnote{\textcolor{pink}{Eschenbachgasse}}, 3 Stock, im Secretariat. Es kommen \textcolor{blue}{Salten}{}\ledrightnote{\textcolor{blue}{Felix Salten}}, \textcolor{blue}{Bahr}{}\ledrightnote{\textcolor{blue}{Hermann Bahr}}, Sie und
                    ich. Wenn Sie nicht können, ſagen Sie bitte mir pneumatiſch ab. Ich war heute
                    bei dem Leichenbegängnis von \textcolor{blue}{Richard}{}\ledrightnote{\textcolor{blue}{Richard Beer-Hofmann}}s \textcolor{blue}{Mutter}{}\ledrightnote{→\textcolor{blue}{Bertha Hofmann}}. Soll man ihn
                    beſuchen? \pend
           \pstart
           Herzlichſt{\\[\baselineskip]}\spacefill\mbox{Loris}\pend
           \leftskip=0em{}\endnumbering\briefempfaengerindex{Schnitzler, Arthur@\textsc{Schnitzler, Arthur}!zzzHofmannsthal, Hugo von@\emph{von Hugo von Hofmannsthal}!1892-01-013@{1. 1. 1892}|)be}\mylabel{h}  \normalsize

\doendnotes{C}
\bigskip
\vfill

\clearpage

\footnotesize

\lohead{\textsc{register}}

% Definiere theindex-Environment komplett neu ohne reledmac
\makeatletter
\renewenvironment{theindex}{%
  \section*{\indexname}%
  \setlength{\parindent}{0pt}%
  \setlength{\parskip}{0pt plus 0.3pt}%
  \let\item\@idxitem
}{%
  \clearpage
}
\makeatother

\IfFileExists{\jobname-pw.ind}{\input{\jobname-pw.ind}}{}

\end{document}

      