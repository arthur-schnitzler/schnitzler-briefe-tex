%% latex-korrekturansicht-vorspann.tex
%% Vorspann für die Korrekturansicht.
%% Lädt die gemeinsame Datei latex-vorspann.tex mit gesetztem Schalter.

\newif\ifkorrekturansicht
\korrekturansichttrue

\input{../tex-inputs/latex-vorspann}


               \section[Arthur Schnitzler an Richard Beer-Hofmann, 18. 8. 1924]{ Arthur Schnitzler an Richard Beer-Hofmann, 18. 8. 1924}\nopagebreak\mylabel{v}\rehead{ }\normalsize\beginnumbering\briefempfaengerindex{Beer-Hofmann, Richard@\textsc{Beer-Hofmann, Richard}!zzzSchnitzler, Arthur@\emph{von Arthur Schnitzler}!1924-08-181@{18. 8. 1924}|(be} \toendnotes[C]{\smallbreak\pagebreak[2]} \Standort{YCGL, MSS 31.}
\physDesc{Bildpostkarte
\newline{}Handschrift: Bleistift, lateinische Kurrent\newline{}Versand: Stempel: »\nobreak{}\oindex{Celerina@\textbf{Celerina}, \emph{Besiedelter Ort (A.BSO)}|pwk}Celerina (Graubünden), 18. VIII. 24, 12\nobreak{}«.  }\buchAbdrucke{\weitereDrucke{Arthur Schnitzler, Richard Beer-Hofmann: \emph{Briefwechsel 1891–1931}. Hg. Konstanze Fliedl. Wien, Zürich: \emph{Europaverlag} 1992, S. 229.} }\pstart{}{\pb}\textcolor{pink}{Austria}{}\ledrightnote{\textcolor{pink}{Österreich}}\pend{}\pstart{}Herrn\pend{}\pstart{}Dr. Richard Beer-Hofma{\geminationn}\pend{}\pstart{}\textcolor{pink}{Wien XVIII}{}\ledrightnote{\textcolor{pink}{XVIII., Währing}}\pend{}\pstart{}\textcolor{pink}{Hasenauerstr 59}{}\ledrightnote{\textcolor{pink}{Hasenauerstraße}}\pend{}{\bigskip}\pstart
           \noindent{}\centering{}{\pb}\textcolor{gray}{\textbf{\textcolor{pink}{Celerina}{}\ledrightnote{\textcolor{pink}{Celerina}}}}\pend
           \pstart
           \raggedleft{}{\pb}18. 8. 24\pend
           \pstart
           lieber Richard, ob dieser Gruſs Sie noch in \textcolor{pink}{Wien}{}\ledrightnote{\textcolor{pink}{Wien}} finden mag? – Hier ist’s so herrlich wie nur je; Ende der
               Woche fahre ich weiter, – \textcolor{pink}{Luzern}{}\ledrightnote{\textcolor{pink}{Luzern}}? \textcolor{pink}{Berner Oberland}{}\ledrightnote{\textcolor{pink}{Berner Oberland}}? \textcolor{pink}{Lugano}{}\ledrightnote{\textcolor{pink}{Lugano}}? – Heut
               besuchen mich \textcolor{blue}{Fiſchers}{}\ledrightnote{\textcolor{blue}{Samuel Fischer}{\newline}\textcolor{blue}{Hedwig Fischer}} (aus \textcolor{pink}{Flims}{}\ledrightnote{\textcolor{pink}{Flims}}.) \textcolor{blue}{Lili}{}\ledrightnote{\textcolor{blue}{Lili Schnitzler}} u
                  \textcolor{blue}{Olga}{}\ledrightnote{\textcolor{blue}{Olga Schnitzler}} sind in \textcolor{pink}{Zuoz}{}\ledrightnote{\textcolor{pink}{Zuoz}} (nah von hier) \textcolor{blue}{Heini}{}\ledrightnote{\textcolor{blue}{Heinrich Schnitzler}} schon
               abgereist. Alles herzliche Ihnen und den Ihrigen\pend
           \pstart Ihr \spacefill\mbox{Arthur}\pend{}\endnumbering\briefempfaengerindex{Beer-Hofmann, Richard@\textsc{Beer-Hofmann, Richard}!zzzSchnitzler, Arthur@\emph{von Arthur Schnitzler}!1924-08-181@{18. 8. 1924}|)be}\mylabel{h}  \normalsize

\doendnotes{C}
\bigskip
\vfill

\clearpage

\footnotesize

\lohead{\textsc{register}}

% Definiere theindex-Environment komplett neu ohne reledmac
\makeatletter
\renewenvironment{theindex}{%
  \section*{\indexname}%
  \setlength{\parindent}{0pt}%
  \setlength{\parskip}{0pt plus 0.3pt}%
  \let\item\@idxitem
}{%
  \clearpage
}
\makeatother

\IfFileExists{\jobname-pw.ind}{\input{\jobname-pw.ind}}{}

\end{document}

      