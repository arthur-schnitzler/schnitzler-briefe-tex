%% latex-korrekturansicht-vorspann.tex
%% Vorspann für die Korrekturansicht.
%% Lädt die gemeinsame Datei latex-vorspann.tex mit gesetztem Schalter.

\newif\ifkorrekturansicht
\korrekturansichttrue

\input{../tex-inputs/latex-vorspann}


               \section[Arthur Schnitzler an Robert Adam, 21. 1. 1919]{ Arthur Schnitzler an Robert Adam, 21. 1. 1919}\nopagebreak\mylabel{v}\rehead{ }\normalsize\beginnumbering\briefempfaengerindex{Adam, Robert@\textsc{Adam, Robert}!zzzSchnitzler, Arthur@\emph{von Arthur Schnitzler}!1919-01-211@{21. 1. 1919}|(be} \toendnotes[C]{\smallbreak\pagebreak[2]} \Standort{DLA, 96.34.2/17.}
\physDesc{Postkarte
\newline{}Handschrift: schwarze Tinte, deutsche Kurrent\newline{}Versand: Stempel: »\nobreak{}4\nobreak{}«.  }\pstart{}{\pb}A. S. \textcolor{pink}{Wien XVIII, \textsc{Sternwartestrasse} 71}{}\ledrightnote{\textcolor{pink}{Sternwartestraße}}\pend{}{\bigskip}\pstart{}Herrn Landesgerichtsrath\pend{}\pstart{}\textsc{Dr. Robert Adam Pollak}\pend{}\pstart{}\textcolor{pink}{\textsc{Wien} XII}{}\ledrightnote{\textcolor{pink}{XII., Meidling}}.\pend{}\pstart{}\textcolor{pink}{\textsc{Meidlinger Hptstr.} 58}{}\ledrightnote{\textcolor{pink}{Meidlinger Hauptstraße}}\pend{}{\bigskip}\pstart
           \raggedleft{}{\pb}21. 1. 1919\pend
           \pstart{}verehrteſter Herr Doktor,\pend\pstart
           ſollten Sie eben am Freitag (24.) Abends nach
                        6 Zeit haben, ſo ſind Sie mir willko{\geminationm}en.
                    Gottes Mühlen ſind \textsc{Express}angelegenheiten gegen
                    direktoriale Entſchlüſſe. Davon weiſs ich manches Lied zu ſingen – ohne Lyriker
                    zu ſein.\pend
           \pstart
           Herzlich grüßt Sie Ihr ergebener{\\[\baselineskip]}\spacefill\mbox{Arthur Schnitzler}\pend
           \leftskip=0em{}\endnumbering\briefempfaengerindex{Adam, Robert@\textsc{Adam, Robert}!zzzSchnitzler, Arthur@\emph{von Arthur Schnitzler}!1919-01-211@{21. 1. 1919}|)be}\mylabel{h}  \normalsize

\doendnotes{C}
\bigskip
\vfill

\clearpage

\footnotesize

\lohead{\textsc{register}}

% Definiere theindex-Environment komplett neu ohne reledmac
\makeatletter
\renewenvironment{theindex}{%
  \section*{\indexname}%
  \setlength{\parindent}{0pt}%
  \setlength{\parskip}{0pt plus 0.3pt}%
  \let\item\@idxitem
}{%
  \clearpage
}
\makeatother

\IfFileExists{\jobname-pw.ind}{\input{\jobname-pw.ind}}{}

\end{document}

      