%% latex-korrekturansicht-vorspann.tex
%% Vorspann für die Korrekturansicht.
%% Lädt die gemeinsame Datei latex-vorspann.tex mit gesetztem Schalter.

\newif\ifkorrekturansicht
\korrekturansichttrue

\input{../tex-inputs/latex-vorspann}


               \section[Arthur und Olga Schnitzler an Richard Beer-Hofmann, 25. 5. 1910]{ Arthur und Olga Schnitzler an Richard Beer-Hofmann,
                    25. 5. 1910}\nopagebreak\mylabel{v}\rehead{ }\normalsize\beginnumbering\briefempfaengerindex{Beer-Hofmann, Richard@\textsc{Beer-Hofmann, Richard}!zzzSchnitzler, Olga@\emph{von Olga Schnitzler}!1910-05-251@{25. 5. 1910}|(be}\briefempfaengerindex{Beer-Hofmann, Richard@\textsc{Beer-Hofmann, Richard}!zzzSchnitzler, Arthur@\emph{von Arthur Schnitzler}!1910-05-251@{25. 5. 1910}|(be} \toendnotes[C]{\smallbreak\pagebreak[2]} \Standort{YCGL, MSS 31.}
\physDesc{Bildpostkarte
\newline{}Handschrift Arthur Schnitzler: Bleistift, deutsche Kurrent\newline{}Handschrift Olga Schnitzler: Bleistift, lateinische Kurrent\newline{}Versand: Stempel: »\nobreak{}\oindex{Interlaken@\textbf{Interlaken}, \emph{Besiedelter Ort (A.BSO)}|pwk}Interlaken, 25. V. 10, 9\nobreak{}«.  }\pstart{}{\pb}\textsc{Herrn Dr. Richard Beer-Hofmann}\pend{}\pstart{}\textsc{\textcolor{pink}{Wien XVIII.}{}\ledrightnote{\textcolor{pink}{XVIII., Währing}}}\pend{}\pstart{}\textsc{\textcolor{pink}{Hasenauerstr 59}{}\ledrightnote{\textcolor{pink}{Hasenauerstraße}}}.\pend{}{\bigskip}\pstart
           \noindent{}\centering{}{\pb}\textcolor{gray}{\textbf{\textcolor{pink}{Interlaken}{}\ledrightnote{\textcolor{pink}{Interlaken}} – \textcolor{pink}{Höheweg}{}\ledrightnote{\textcolor{pink}{Höheweg}}. \textcolor{pink}{Jungfrau}{}\ledrightnote{\textcolor{pink}{Jungfrau}}.}}\pend
           \pstart
           \raggedleft{}25. 5. 10.\pend
           \pstart
           {\pb}Hier iſt es über alle Maßen ſchön; alle
                    Erinnerungen übertreffend. Es wi{\geminationm}elt von Hotels,
                        \textcolor{gray}{von} 500–1000 un\textcolor{gray}{d} 1000 \textsc{metern}, – und Ende Juni fänden Sie \introOben{}noch\introOben{} überall was Sie wollen. Wir fahren morgen nach \textsc{\textcolor{pink}{Territet}{}\ledrightnote{\textcolor{pink}{Territet}}}.\pend
           \pstart Herzlichſt Ihr \spacefill\mbox{A.}\pend{}\pstart
           \noindent{}{[}hs. O. Schnitzler:{]} Es wird immer \uline{noch}
                    schöner!\pend
           \pstart Herzlichste Grüsse! \spacefill\mbox{Olga.}\pend{}\endnumbering\briefempfaengerindex{Beer-Hofmann, Richard@\textsc{Beer-Hofmann, Richard}!zzzSchnitzler, Olga@\emph{von Olga Schnitzler}!1910-05-251@{25. 5. 1910}|)be}\briefempfaengerindex{Beer-Hofmann, Richard@\textsc{Beer-Hofmann, Richard}!zzzSchnitzler, Arthur@\emph{von Arthur Schnitzler}!1910-05-251@{25. 5. 1910}|)be}\mylabel{h}  \normalsize

\doendnotes{C}
\bigskip
\vfill

\clearpage

\footnotesize

\lohead{\textsc{register}}

% Definiere theindex-Environment komplett neu ohne reledmac
\makeatletter
\renewenvironment{theindex}{%
  \section*{\indexname}%
  \setlength{\parindent}{0pt}%
  \setlength{\parskip}{0pt plus 0.3pt}%
  \let\item\@idxitem
}{%
  \clearpage
}
\makeatother

\IfFileExists{\jobname-pw.ind}{\input{\jobname-pw.ind}}{}

\end{document}

      