%% latex-korrekturansicht-vorspann.tex
%% Vorspann für die Korrekturansicht.
%% Lädt die gemeinsame Datei latex-vorspann.tex mit gesetztem Schalter.

\newif\ifkorrekturansicht
\korrekturansichttrue

\input{../tex-inputs/latex-vorspann}


               \section[Arthur Schnitzler an Richard Beer-Hofmann, 14. 7. 1916]{ Arthur Schnitzler an Richard Beer-Hofmann, 14. 7. 1916}\nopagebreak\mylabel{v}\rehead{ }\normalsize\beginnumbering\briefempfaengerindex{Beer-Hofmann, Richard@\textsc{Beer-Hofmann, Richard}!zzzSchnitzler, Arthur@\emph{von Arthur Schnitzler}!1916-07-142@{14. 7. 1916}|(be} \toendnotes[C]{\smallbreak\pagebreak[2]} \Standort{YCGL, MSS 31.}
\physDesc{Bildpostkarte
\newline{}Handschrift: Bleistift, deutsche Kurrent\newline{}Versand: Stempel: »\nobreak{}\oindex{Bad Aussee@\textbf{Bad Aussee}, \emph{Besiedelter Ort (A.BSO)}|pwk}Bad Aussee, 15. VII. \textcolor{gray}{16}\nobreak{}«.  
\newline{}Beer-Hofmann: mit Bleistift Datum der Beantwortung vermerkt: »B.
                                       19/VII. 16« }\buchAbdrucke{\weitereDrucke{Arthur Schnitzler, Richard Beer-Hofmann: \emph{Briefwechsel 1891–1931}. Hg. Konstanze Fliedl. Wien, Zürich: \emph{Europaverlag} 1992, S. 221.} }\toendnotes[C]{\smallbreak}\pstart{}{\pb}Hrn \textsc{Dr. Richard}\pend{}\pstart{}\textsc{BeerHofmann}\pend{}\pstart{}\textsc{\textcolor{pink}{Bad Ischl}{}\ledrightnote{\textcolor{pink}{Bad Ischl}}}\pend{}\pstart{}\textsc{\textcolor{pink}{Grazer Straße 54(?)}{}\ledrightnote{\textcolor{pink}{Grazer Straße}}}\pend{}{\bigskip}\pstart
           \noindent{}\centering{}{\pb}\textcolor{gray}{\textbf{\textcolor{pink}{Salzkammergut}{}\ledrightnote{\textcolor{pink}{Salzkammergut}}. \textcolor{pink}{Alt Aussee}{}\ledrightnote{\textcolor{pink}{Altaussee}} mit dem \textcolor{pink}{Loser}{}\ledrightnote{\textcolor{pink}{Loser}}.}}\pend
           \pstart
           \raggedleft{}{\pb}\textcolor{pink}{\textsc{Altaussee}}{}\ledrightnote{\textcolor{pink}{Altaussee}}. 14. 7. 16\pend
           \pstart
           \raggedleft{}\textcolor{pink}{\textsc{Fischerndorf 79}}{}\ledrightnote{\textcolor{pink}{Fischerndorf}}\pend
           \pstart
           \noindent{}\raggedleft{}\textcolor{pink}{\textsc{Anderl Haus}}{}\ledrightnote{\textcolor{pink}{Villa Annerl}}\pend
           \pstart
           lieber Richard, wir ſind in unſerm Haus ſehr gut inſtallirt und
               durchaus gut (daheim und beim \textcolor{pink}{Seewirth}{}\ledrightnote{\textcolor{pink}{Seewirt}}) verſorgt.
               Die Lage unſrer Wohnung iſt wundervoll. Die Landſchaft behagt mir ſehr, ich fliege
               viel aus und arbeite – weniger. Im übrigen bin ich mit der Welt wenig zufrieden. Wird
               man Sie bald hier ſehen? Sie haben (wirklich!) beſſere Verbindung als wir, wegen
               Zurückfahrens. Jedenfalls laſſen Sie möglichſt bald von ſich hören. Herzliche Grüße
               von \textcolor{blue}{uns}{}\ledrightnote{→\textcolor{blue}{Olga Schnitzler}{\newline}→\textcolor{blue}{Lili Schnitzler}{\newline}→\textcolor{blue}{Heinrich Schnitzler}} allen zu
               Ihnen \textcolor{blue}{Allen}{}\ledrightnote{→\textcolor{blue}{Paula Beer-Hofmann}{\newline}→\textcolor{blue}{Gabriel Beer-Hofmann}{\newline}→\textcolor{blue}{Naëmah Beer-Hofmann}{\newline}→\textcolor{blue}{Mirjam Beer-Hofmann}}! \pend
           \pstart Ihr \spacefill\mbox{A.}\pend{}\endnumbering\briefempfaengerindex{Beer-Hofmann, Richard@\textsc{Beer-Hofmann, Richard}!zzzSchnitzler, Arthur@\emph{von Arthur Schnitzler}!1916-07-142@{14. 7. 1916}|)be}\mylabel{h}  \normalsize

\doendnotes{C}
\bigskip
\vfill

\clearpage

\footnotesize

\lohead{\textsc{register}}

% Definiere theindex-Environment komplett neu ohne reledmac
\makeatletter
\renewenvironment{theindex}{%
  \section*{\indexname}%
  \setlength{\parindent}{0pt}%
  \setlength{\parskip}{0pt plus 0.3pt}%
  \let\item\@idxitem
}{%
  \clearpage
}
\makeatother

\IfFileExists{\jobname-pw.ind}{\input{\jobname-pw.ind}}{}

\end{document}

      