%% latex-korrekturansicht-vorspann.tex
%% Vorspann für die Korrekturansicht.
%% Lädt die gemeinsame Datei latex-vorspann.tex mit gesetztem Schalter.

\newif\ifkorrekturansicht
\korrekturansichttrue

\input{../tex-inputs/latex-vorspann}


               \section[Richard Beer-Hofmann an Arthur Schnitzler, 28. 6. 1901]{ Richard Beer-Hofmann an Arthur Schnitzler,
               28. 6. 1901}\nopagebreak\mylabel{v}\rehead{ }\normalsize\beginnumbering\briefempfaengerindex{Schnitzler, Arthur@\textsc{Schnitzler, Arthur}!zzzBeer-Hofmann, Richard@\emph{von Richard Beer-Hofmann}!1901-06-281@{28. 6. 1901}|(be} \toendnotes[C]{\smallbreak\pagebreak[2]} \Standort{CUL, Schnitzler, B 8.}
\physDesc{Brief, 1 Blatt, 2 Seiten
\newline{}Handschrift: blauer Buntstift, lateinische Kurrent\newline{}Ordnung: mit Bleistift von unbekannter Hand nummeriert: »163« }\buchAbdrucke{\weitereDrucke{Arthur Schnitzler, Richard Beer-Hofmann: \emph{Briefwechsel 1891–1931}. Hg. Konstanze Fliedl. Wien, Zürich: \emph{Europaverlag} 1992, S. 152.} }\toendnotes[C]{\smallbreak}\pstart
           \raggedleft{}{\pb}\textcolor{pink}{Pörtschach}{}\ledrightnote{\textcolor{pink}{Pörtschach}}{ }28/VI 1901\pend
           \pstart
           Lieber Arthur! Es war Zeit daß Sie von Sich hören ließen. Ich wußte
               nur durch die \textcolor{green}{\textcolor{brown}{N. Fr Pr}{}\ledrightnote{\textcolor{brown}{Neue Freie Presse}}}{}\ledrightnote{→\textcolor{green}{Kleine Chronik}} daß Sie in \textcolor{pink}{Tirol}{}\ledrightnote{\textcolor{pink}{Tirol}} sind. Ich war – um mir
               Heiterkeit zu holen – 3 Tage in \textcolor{pink}{Venedig}{}\ledrightnote{\textcolor{pink}{Venedig}},
               gleichzeitig mit \textcolor{blue}{Hugo}{}\ledrightnote{\textcolor{blue}{Hugo von Hofmannsthal}}, doch wußten wir von
               einander nichts, und erst als ich zurückkam erfuhr ich daß er auch dort war. Ich habe
               mir aber keine Heiterkeit aus \textcolor{pink}{Venedig}{}\ledrightnote{\textcolor{pink}{Venedig}} geholt.\pend
           \pstart
           {\pb}Ich möchte wissen wann Sie
               herkommen, und ob und wann \textcolor{blue}{Paul}{}\ledrightnote{\textcolor{blue}{Paul Goldmann}} hieherko{\geminationm}t. \textcolor{blue}{Ludassy}{}\ledrightnote{\textcolor{blue}{Julius von Gans-Ludassy}} und \textcolor{blue}{Alexander Engel}{}\ledrightnote{\textcolor{blue}{Alexander Engel}} habe ich hier gesprochen. – \textcolor{blue}{L.}{}\ledrightnote{\textcolor{blue}{Julius von Gans-Ludassy}} erklärte es unsicher daß Sie kämen. \textcolor{blue}{Hirschfeld (Robert)}{}\ledrightnote{\textcolor{blue}{Robert Hirschfeld}} hat uns besucht. Was ist mit
                  \textcolor{blue}{Salten}{}\ledrightnote{\textcolor{blue}{Felix Salten}} und seinem bodenständigen \textcolor{pink}{Brettl}{}\ledrightnote{→\textcolor{pink}{Jung-Wiener Theater zum Lieben Augustin}}; aber wichtiger: Was ist mit
               Ihnen? Ist \textcolor{pink}{Salzburg}{}\ledrightnote{\textcolor{pink}{Salzburg}} noch immer gegen Versti{\geminationm}ung gut? Von Herzen\pend
           \pstart Ihr \spacefill\mbox{Richard}\pend{}\endnumbering\briefempfaengerindex{Schnitzler, Arthur@\textsc{Schnitzler, Arthur}!zzzBeer-Hofmann, Richard@\emph{von Richard Beer-Hofmann}!1901-06-281@{28. 6. 1901}|)be}\mylabel{h}  \normalsize

\doendnotes{C}
\bigskip
\vfill

\clearpage

\footnotesize

\lohead{\textsc{register}}

% Definiere theindex-Environment komplett neu ohne reledmac
\makeatletter
\renewenvironment{theindex}{%
  \section*{\indexname}%
  \setlength{\parindent}{0pt}%
  \setlength{\parskip}{0pt plus 0.3pt}%
  \let\item\@idxitem
}{%
  \clearpage
}
\makeatother

\IfFileExists{\jobname-pw.ind}{\input{\jobname-pw.ind}}{}

\end{document}

      