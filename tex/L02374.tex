%% latex-korrekturansicht-vorspann.tex
%% Vorspann für die Korrekturansicht.
%% Lädt die gemeinsame Datei latex-vorspann.tex mit gesetztem Schalter.

\newif\ifkorrekturansicht
\korrekturansichttrue

\input{../tex-inputs/latex-vorspann}


               \section[Christiane von Hofmannsthal an Arthur Schnitzler, 28. 1. 192{[}2{]}]{ Christiane von Hofmannsthal an Arthur Schnitzler, 28. 1. 192{[}2{]}}\nopagebreak\mylabel{v}\rehead{ }\normalsize\beginnumbering\briefempfaengerindex{Schnitzler, Arthur@\textsc{Schnitzler, Arthur}!zzzHofmannsthal, Christiane von@\emph{von Christiane von Hofmannsthal}!1922-01-281@{28. 1. 192{[}2{]}}|(be} \toendnotes[C]{\smallbreak\pagebreak[2]} \Standort{CUL, Schnitzler, B 43.}
\physDesc{Postkarte
\newline{}Handschrift: schwarze Tinte, lateinische Kurrent\newline{}Versand: Stempel: »\nobreak{}\oindex{Rodaun@\textbf{Rodaun}, \emph{Teil eines besiedelten Ortes (A.BSOX)}|pwk}\textcolor{gray}{R}odau\textcolor{gray}{n}\nobreak{}«.  \newline{}Ordnung: 1) mit Bleistift von \textcolor{blue}{Frieda
                                    Pollak} (?) mit dem Buchstaben »A«
                                 (Abgeschrieben/Abschrift) gekennzeichnet 2) mit Bleistift von unbekannter Hand nummeriert: »\strikeout{375}«3) mit Bleistift von unbekannter Hand nummeriert:
                                    »363«}\buchAbdrucke{\weitereDrucke{Hugo von Hofmannsthal, Arthur Schnitzler: \emph{Briefwechsel}. Hg. Therese Nickl und Heinrich Schnitzler. Frankfurt am Main: \emph{S. Fischer} 1964, S. 392.} }\toendnotes[C]{\smallbreak}\pstart{}{\pb}Herrn Arthur Schnitzler\pend{}\pstart{}\textcolor{pink}{Wien XVIII.}{}\ledrightnote{\textcolor{pink}{XVIII., Währing}}\pend{}\pstart{}\textcolor{pink}{Sternwartestr. 71}{}\ledrightnote{\textcolor{pink}{Sternwartestraße}}.\pend{}{\bigskip}\pstart
           \raggedleft{}{\pb}\label{K_L02374_1v}\edtext{28. I. 21}{\lemma{\textnormal{\emph{28. I. 21}}}\Cendnote{\textnormal{Bei der Jahresangabe handelt es sich um einen Schreibirrtum,
                     wie sich aus der angekündigten Lesung ergibt.}}}\label{K_L02374_1h}\pend
           \pstart{}Lieber Arthur, \pend\pstart
           Im Namen vom \textcolor{blue}{Papa}{}\ledrightnote{→\textcolor{blue}{Hugo von Hofmannsthal}} bitte ich
               Dich, sicher am Freitag{ }¾ 7\textsuperscript{h} abends bei der \textcolor{blue}{Berta Zuckerkandl}{}\ledrightnote{\textcolor{blue}{Berta Zuckerkandl}} zu sein,
               wo \textcolor{blue}{Papa}{}\ledrightnote{→\textcolor{blue}{Hugo von Hofmannsthal}} das \textcolor{green}{Welttheater}{}\ledrightnote{\textcolor{green}{Das Salzburger große Welttheater}} vorliest. Er freut sich besonders auf Dein
               Zuhören.\pend
           \pstart
           Herzliche Grüße von Deiner{\\[\baselineskip]}\spacefill\mbox{Christiane Hofmannsthal}\pend
           \leftskip=0em{}\endnumbering\briefempfaengerindex{Schnitzler, Arthur@\textsc{Schnitzler, Arthur}!zzzHofmannsthal, Christiane von@\emph{von Christiane von Hofmannsthal}!1922-01-281@{28. 1. 192{[}2{]}}|)be}\mylabel{h}  \normalsize

\doendnotes{C}
\bigskip
\vfill

\clearpage

\footnotesize

\lohead{\textsc{register}}

% Definiere theindex-Environment komplett neu ohne reledmac
\makeatletter
\renewenvironment{theindex}{%
  \section*{\indexname}%
  \setlength{\parindent}{0pt}%
  \setlength{\parskip}{0pt plus 0.3pt}%
  \let\item\@idxitem
}{%
  \clearpage
}
\makeatother

\IfFileExists{\jobname-pw.ind}{\input{\jobname-pw.ind}}{}

\end{document}

      