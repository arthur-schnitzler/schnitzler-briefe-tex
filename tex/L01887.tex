%% latex-korrekturansicht-vorspann.tex
%% Vorspann für die Korrekturansicht.
%% Lädt die gemeinsame Datei latex-vorspann.tex mit gesetztem Schalter.

\newif\ifkorrekturansicht
\korrekturansichttrue

\input{../tex-inputs/latex-vorspann}


               \section[Albert Ehrenstein an Arthur Schnitzler, 22. 11. 1909]{ Albert Ehrenstein an Arthur Schnitzler, 22. 11. 1909}\nopagebreak\mylabel{v}\rehead{ }\normalsize\beginnumbering\briefempfaengerindex{Schnitzler, Arthur@\textsc{Schnitzler, Arthur}!zzzEhrenstein, Albert@\emph{von Albert Ehrenstein}!1909-11-221@{22. 11. 1909}|(be} \toendnotes[C]{\smallbreak\pagebreak[2]} \Standort{CUL, Schnitzler, B 30.}
\physDesc{Brief, 1 Blatt, 3 Seiten
\newline{}Handschrift: schwarze Tinte, deutsche Kurrent
\newline{}Schnitzler: mit Bleistift beschriftet: »\textsc{Ehrenste\textcolor{gray}{in}}« }\buchAbdrucke{\weitereDrucke{Albert Ehrenstein: \emph{Briefe}. Hg. Hanni Mittelmann. München: \emph{Boer} 1989, S. 35–36 (Werke, 1).} }\toendnotes[C]{\smallbreak}\pstart
           \noindent{}{\pb}\textcolor{pink}{XVI. \textsc{Ottakringerstr.} 114}{}\ledrightnote{\textcolor{pink}{Ottakringerstraße}}.
                        \hfill 22. XI. 09.
                        \pend
           \pstart{}Sehr geehrter Herr Doktor,\pend\pstart
           Herr \textcolor{blue}{Alfred Polgar}{}\ledrightnote{\textcolor{blue}{Alfred Polgar}}, dem ich, wie Sie wiſſen,
                    Arbeiten unterbreitete, fand großen Gefallen an denſelben und ſchickte mir, der
                    ich ihn übrigens nicht perſönlich kenne, eine in ſchmeichelhafter Weiſe
                    abgefaßte Empfehlung – aber zu meiner Überraſchung an Herrn Profeſſor \textcolor{blue}{Bie}{}\ledrightnote{\textcolor{blue}{Oskar Bie}} für die \textcolor{green}{N. Rundſchau}{}\ledrightnote{\textcolor{green}{Die neue Rundschau}}. Ich konnte nicht umhin, von derſelben Gebrauch zu machen
                    (ſchon um das mir entgegengebrachte Wohlwollen nicht zu kränken), obwohl ich in
                    erſter Linie, die \textcolor{green}{Rundſchau}{}\ledrightnote{\textcolor{green}{Die neue Rundschau}} und Herrn
                    Profeſſor \textcolor{blue}{Bie}{}\ledrightnote{\textcolor{blue}{Oskar Bie}} betreffend, auf die {\pb}von Ihnen mir freundlichſt in Ausſicht geſtellte
                    Fürſprache bei letzterem rechne. Vorgeſtern ſandte ich 6 Skizzen (\textcolor{green}{Saccumum}{}\ledrightnote{\textcolor{green}{Saccumum}}, \textcolor{green}{Mitgefühl}{}\ledrightnote{\textcolor{green}{Mitgefühl}}, \textcolor{green}{Die alte Geſchichte}{}\ledrightnote{\textcolor{green}{Die alte Geschichte}}, \textcolor{green}{Tubutſch}{}\ledrightnote{\textcolor{green}{Tubutsch}}, \textcolor{green}{Baber}{}\ledrightnote{\textcolor{green}{Tod des Zehir eddin Muhammed Baber}} u. \textcolor{green}{Tai-gin}{}\ledrightnote{\textcolor{green}{Tai-Gin}}) an Herrn Profeſſor
                        \textcolor{blue}{Bie}{}\ledrightnote{\textcolor{blue}{Oskar Bie}}.\pend
           \pstart
           Nun weiß ich nicht, ob Sie, ſehr geehrter Herr Doktor, ſchon in \textcolor{pink}{Berlin}{}\ledrightnote{\textcolor{pink}{Berlin}} waren und die Liebenswürdigkeit gehabt haben,
                    meinen Skizzenband »\label{K_L01887_1v}\edtext{Zuſchauer und
                        Tyrannen}{\lemma{\textnormal{\emph{Zuſchauer und
                        Tyrannen}}}\Cendnote{\textnormal{Unter diesem Titel
                        veröffentlichte er keine Novellensammlung, doch ist in seinem Nachlass ein
                        Entwurf der dafür vorgesehenen 19 Novellen überliefert.}}}\label{K_L01887_1h}« – den ich
                    Ihnen vor etwa 14 Tagen mit einem Begleitſchreiben zukommenließ – oder eine
                    ſtrenge Auswahl meiner Novelletten Ihrem \textcolor{blue}{Verleger}{}\ledrightnote{→\textcolor{blue}{Samuel Fischer}} zu geben, oder ob dies noch bevorſteht?\pend
           \pstart
           Jedenfalls möchte ich Sie höflichſt {\pb}bitten, nicht bloß
                    bei dem Herrn \textcolor{blue}{Fiſcher}{}\ledrightnote{\textcolor{blue}{Samuel Fischer}}, ſondern, wenn es
                    angängig iſt, auch bei dem Herrn Profeſſor \textcolor{blue}{Bie}{}\ledrightnote{\textcolor{blue}{Oskar Bie}} für mich zu wirken.\pend
           \pstart
           Für Ihre gewiß erfolgreichen Interventionen im Voraus dankend, bin ich mit dem
                    Ausdrucke vorzüglichſter Hochachtung \pend
           \pstart
           Ihr ergebenſter{\\[\baselineskip]}\spacefill\mbox{Albert Ehrenstein.}\pend
           \leftskip=0em{}\endnumbering\briefempfaengerindex{Schnitzler, Arthur@\textsc{Schnitzler, Arthur}!zzzEhrenstein, Albert@\emph{von Albert Ehrenstein}!1909-11-221@{22. 11. 1909}|)be}\mylabel{h}  \normalsize

\doendnotes{C}
\bigskip
\vfill

\clearpage

\footnotesize

\lohead{\textsc{register}}

% Definiere theindex-Environment komplett neu ohne reledmac
\makeatletter
\renewenvironment{theindex}{%
  \section*{\indexname}%
  \setlength{\parindent}{0pt}%
  \setlength{\parskip}{0pt plus 0.3pt}%
  \let\item\@idxitem
}{%
  \clearpage
}
\makeatother

\IfFileExists{\jobname-pw.ind}{\input{\jobname-pw.ind}}{}

\end{document}

      