%% latex-korrekturansicht-vorspann.tex
%% Vorspann für die Korrekturansicht.
%% Lädt die gemeinsame Datei latex-vorspann.tex mit gesetztem Schalter.

\newif\ifkorrekturansicht
\korrekturansichttrue

\input{../tex-inputs/latex-vorspann}


               \section[Arthur Schnitzler an Gerhart Hauptmann, 30. 11. 1902]{ Arthur Schnitzler an Gerhart Hauptmann, 30. 11. 1902}\nopagebreak\mylabel{v}\rehead{ }\normalsize\beginnumbering\briefempfaengerindex{Hauptmann, Gerhart@\textsc{Hauptmann, Gerhart}!zzzSchnitzler, Arthur@\emph{von Arthur Schnitzler}!1902-11-301@{30. 11. 1903}|(be} \toendnotes[C]{\smallbreak\pagebreak[2]} \Standort{Staatsbibliothek Berlin – Preußischer Kulturbesitz, GHBrBl A:Schnitzler (13).}
\physDesc{Brief, 1 Blatt, 2 Seiten
\newline{}Handschrift: schwarze Tinte, deutsche Kurrent\newline{}Ordnung: Lochung }\toendnotes[C]{\smallbreak}\pstart
           \noindent{}{\pb}von ganzem Herzen, lieber Herr
                        Hauptmann beglückwünſche ich Sie zu Ihrem wahrhaft großen \label{K_L01254_1v}\edtext{Erfolg}{\lemma{\textnormal{\emph{Erfolg}}}\Cendnote{\textnormal{Uraufführung von \emph{\textcolor{green}{Der Arme Heinrich}} im \textcolor{pink}{Burgtheater} am 29. 11. 1902.
                            \textcolor{blue}{Schnitzler} war anwesend.}}}\label{K_L01254_1h}; – i\substVorne{}\textsuperscript{m}\substDazwischen{}ns\substHinten{} innerſte bewegt von der ſchönen \textcolor{green}{Dichtung}{}\ledrightnote{→\textcolor{green}{Der arme Heinrich – Eine deutsche Sage}}, der nicht an Fülle reifſten vielleicht, der
                    aber, die schwellend von verhaltener Kraft, leuchtend in Reinheit, und in
                    reinlichſter {\pb}Einfachheit dahinfließend,
                    Ihren schönſten Werken ſich anschließt und in noch lichtere Höhen deutet.\pend
           \pstart
           In Bewunderung und Freundſchaft drück ich Ihnen die Hand{\\[\baselineskip]}Ihr{\\[\baselineskip]}\spacefill\mbox{Arthur Schnitzler}\pend
           \leftskip=0em{}\pstart
           30. 11. 902\pend
           \endnumbering\briefempfaengerindex{Hauptmann, Gerhart@\textsc{Hauptmann, Gerhart}!zzzSchnitzler, Arthur@\emph{von Arthur Schnitzler}!1902-11-301@{30. 11. 1903}|)be}\mylabel{h}  \normalsize

\doendnotes{C}
\bigskip
\vfill

\clearpage

\footnotesize

\lohead{\textsc{register}}

% Definiere theindex-Environment komplett neu ohne reledmac
\makeatletter
\renewenvironment{theindex}{%
  \section*{\indexname}%
  \setlength{\parindent}{0pt}%
  \setlength{\parskip}{0pt plus 0.3pt}%
  \let\item\@idxitem
}{%
  \clearpage
}
\makeatother

\IfFileExists{\jobname-pw.ind}{\input{\jobname-pw.ind}}{}

\end{document}

      