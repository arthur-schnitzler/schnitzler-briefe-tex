%% latex-korrekturansicht-vorspann.tex
%% Vorspann für die Korrekturansicht.
%% Lädt die gemeinsame Datei latex-vorspann.tex mit gesetztem Schalter.

\newif\ifkorrekturansicht
\korrekturansichttrue

\input{../tex-inputs/latex-vorspann}


               \section[Hugo von Hofmannsthal an Olga Schnitzler, 17. 4. 1920]{ Hugo von Hofmannsthal an Olga Schnitzler, 17. 4. 1920}\nopagebreak\mylabel{v}\rehead{ }\normalsize\beginnumbering\briefempfaengerindex{Schnitzler, Olga@\textsc{Schnitzler, Olga}!zzzHofmannsthal, Hugo von@\emph{von Hugo von Hofmannsthal}!1920-04-172@{17. 4. 1920}|(be} \toendnotes[C]{\smallbreak\pagebreak[2]} \Standort{DLA, A:Schnitzler, HS.NZ85.1.5584.}
\physDesc{Brief, 1 Blatt (Briefpapier mit Trauerrand), 4 Seiten, Umschlag
\newline{}Handschrift: schwarze Tinte, deutsche Kurrent\newline{}Versand: Stempel: »\nobreak{}\oindex{Rodaun@\textbf{Rodaun}, \emph{Teil eines besiedelten Ortes (A.BSOX)}|pwk}Rodaun, 17{[}. 4. 1920{]}\nobreak{}«.  }\toendnotes[C]{\smallbreak}\pstart{}{\pb}\textsc{Hofma{\geminationn}sthal}\pend{}\pstart{}\textsc{\textcolor{pink}{Rodaun}{}\ledrightnote{\textcolor{pink}{Rodaun}}.}\pend{}{\bigskip}\pstart{}\textsc{Frau Olga Schnitzler}\pend{}\pstart{}\textsc{\textcolor{pink}{Wien}{}\ledrightnote{\textcolor{pink}{Wien}}}\pend{}\pstart{}\textsc{\textcolor{pink}{XVIII. Sternwartestrasse \substVorne{}\textsuperscript{5}\substDazwischen{}7\substHinten{}1}{}\ledrightnote{\textcolor{pink}{Sternwartestraße}}.}\pend{}{\bigskip}\pstart
           \raggedleft{}{\pb}\textcolor{pink}{Rodaun}{}\ledrightnote{\textcolor{pink}{Rodaun}}{\\}17. IV.\pend
           \pstart{}liebe Olga\pend\pstart
           mit Schmerz hab ich erfahren, daſs Ihre gute liebe \textcolor{blue}{Schweſter}{}\ledrightnote{→\textcolor{blue}{Elisabeth Steinrück}} von dieſer finſteren Welt und uns allen auf immer
               fortgegangen iſt. Wie freundlich wäre es, ſie noch immer unter den Lebenden zu
               wiſſen. Es ſchien mir eine Güte von ihr, daſs ſie immer noch dableiben wollte. Dieſes
               unvergleichliche, rührende \textcolor{blue}{Weſen}{}\ledrightnote{→\textcolor{blue}{Elisabeth Steinrück}}
               – ich habe ſie ja, {\pb}würde man ſagen, nur wenig
               gekannt: und doch, wie ſehr iſt ſie auch mir geſtorben! – und davon gibt mein innerſtes
               Gefühl mit nachhaltigem Schmerz mir ſelber Zeugnis. Man brauchte ihr nur manchmal
               begegnet zu ſein – mit welcher zarten feinen unauslöſchbaren Schrift ſchrieb ſich
               dieſes \textcolor{blue}{Weſen}{}\ledrightnote{→\textcolor{blue}{Elisabeth Steinrück}} einem ins Herz. Sie
               haben ſo viel {\pb}verloren – mehr als irgend jemand
               ſicherlich, denn Sie waren die frühen Jahre mit ihr verbunden: ſo fällt für Sie ſo
               nichts zugleich dahin.\pend
           \pstart
           Wie viel aber auch \textcolor{blue}{Arthur}{}\ledrightnote{} verloren hat, was für
               eine gute zärtliche Freundin, das kann ich ahnen – ermeſſen kann ja ein Dritter
               ſolche Dinge nie. Sagen Sie es ihm, daſs ich oft u. oft an ihn denke.\pend
           \pstart
           {\pb}Ich bin, liebe Olga, in alter Freundſchaft\pend
           \pstart
           Herzlich Ihr{\\[\baselineskip]}\spacefill\mbox{Hugo H.}\pend
           \leftskip=0em{}\endnumbering\briefempfaengerindex{Schnitzler, Olga@\textsc{Schnitzler, Olga}!zzzHofmannsthal, Hugo von@\emph{von Hugo von Hofmannsthal}!1920-04-172@{17. 4. 1920}|)be}\mylabel{h}  \normalsize

\doendnotes{C}
\bigskip
\vfill

\clearpage

\footnotesize

\lohead{\textsc{register}}

% Definiere theindex-Environment komplett neu ohne reledmac
\makeatletter
\renewenvironment{theindex}{%
  \section*{\indexname}%
  \setlength{\parindent}{0pt}%
  \setlength{\parskip}{0pt plus 0.3pt}%
  \let\item\@idxitem
}{%
  \clearpage
}
\makeatother

\IfFileExists{\jobname-pw.ind}{\input{\jobname-pw.ind}}{}

\end{document}

      