%% latex-korrekturansicht-vorspann.tex
%% Vorspann für die Korrekturansicht.
%% Lädt die gemeinsame Datei latex-vorspann.tex mit gesetztem Schalter.

\newif\ifkorrekturansicht
\korrekturansichttrue

\input{../tex-inputs/latex-vorspann}


               \section[Arthur Schnitzler an Richard Beer-Hofmann, {[}zwischen September 1892 und Ende 1901?{]}]{ Arthur Schnitzler an Richard Beer-Hofmann, {[}zwischen September 1892 und
               Ende 1901?{]}}\nopagebreak\mylabel{v}\rehead{ }\normalsize\beginnumbering\briefempfaengerindex{Beer-Hofmann, Richard@\textsc{Beer-Hofmann, Richard}!zzzSchnitzler, Arthur@\emph{von Arthur Schnitzler}!1892-09-011@{{[}zwischen September 1892
                  und Ende 1901?{]}}|(be} \toendnotes[C]{\smallbreak\pagebreak[2]} \Standort{YCGL, MSS 31.}
\physDesc{Brief, 1 Blatt, 1 Seite, Umschlag
\newline{}Handschrift: Bleistift, deutsche Kurrent\newline{}Versand: ohne postalischen Übermittlungsvermerk }\toendnotes[C]{\smallbreak}\pstart{}{\pb}Herrn Dr \textsc{Richard
                     BeerHofmann}\pend{}\pstart{}\textcolor{pink}{Wien}{}\ledrightnote{\textcolor{pink}{Wien}}. \pend{}\pstart{}\textsc{\textcolor{pink}{I Wollzeile 15}{}\ledrightnote{\textcolor{pink}{Wollzeile}}}.\pend{}{\bigskip}\pstart
           \noindent{}{\pb}\label{K_L00119_1v}\edtext{Herzlichen Gruſs}{\lemma{\textnormal{\emph{Herzlichen Gruſs}}}\Cendnote{\textnormal{Das Korrespondenzstück ist undatiert und
                  kann nur durch die Wohnadresse in der \textcolor{pink}{Wollzeile 15} einem Zeitraum zugeordnet werden.}}}\label{K_L00119_1h}!\pend
           \pstart
           Ihr{\\[\baselineskip]}\spacefill\mbox{Arthur}\pend
           \leftskip=0em{}\endnumbering\briefempfaengerindex{Beer-Hofmann, Richard@\textsc{Beer-Hofmann, Richard}!zzzSchnitzler, Arthur@\emph{von Arthur Schnitzler}!1892-09-011@{{[}zwischen September 1892
                  und Ende 1901?{]}}|)be}\mylabel{h}  \normalsize

\doendnotes{C}
\bigskip
\vfill

\clearpage

\footnotesize

\lohead{\textsc{register}}

% Definiere theindex-Environment komplett neu ohne reledmac
\makeatletter
\renewenvironment{theindex}{%
  \section*{\indexname}%
  \setlength{\parindent}{0pt}%
  \setlength{\parskip}{0pt plus 0.3pt}%
  \let\item\@idxitem
}{%
  \clearpage
}
\makeatother

\IfFileExists{\jobname-pw.ind}{\input{\jobname-pw.ind}}{}

\end{document}

      