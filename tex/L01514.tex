%% latex-korrekturansicht-vorspann.tex
%% Vorspann für die Korrekturansicht.
%% Lädt die gemeinsame Datei latex-vorspann.tex mit gesetztem Schalter.

\newif\ifkorrekturansicht
\korrekturansichttrue

\input{../tex-inputs/latex-vorspann}


               \section[Arthur Schnitzler an Richard Beer-Hofmann, 8. 5. 1905]{ Arthur Schnitzler an Richard Beer-Hofmann, 8. 5. 1905}\nopagebreak\mylabel{v}\rehead{ }\normalsize\beginnumbering\briefempfaengerindex{Beer-Hofmann, Richard@\textsc{Beer-Hofmann, Richard}!zzzSchnitzler, Arthur@\emph{von Arthur Schnitzler}!1905-05-081@{8. 5. 1905}|(be} \toendnotes[C]{\smallbreak\pagebreak[2]} \Standort{YCGL, MSS 31.}
\physDesc{Telegramm
\newline{}Handschrift einer Schreibkraft: Bleistift, lateinische Kurrent\newline{}Versand: »\noindent{}\textcolor{gray}{\textbf{Aufgegeben am}}{ }8./V. \textcolor{gray}{\textbf{190}}5{ }\textcolor{gray}{\textbf{um}}{ }9 \textcolor{gray}{\textbf{Uhr}} 03 \textcolor{gray}{\textbf{Min.}} V\textcolor{gray}{\textbf{Mittag}}{ / }\textcolor{gray}{\textbf{Eingelangt von}}{ }W{ }\textcolor{gray}{×}\-\textcolor{gray}{×}\-\textcolor{gray}{×}{ / }8/V{ }X \textcolor{gray}{\textbf{Uhr}} 40 \textcolor{gray}{\textbf{Min.}} V\textcolor{gray}{\textbf{Mittag}}{ / }\textcolor{gray}{\textbf{Aufgenommen durch}}{ }\textcolor{gray}{×}\-\textcolor{gray}{×}« }\buchAbdrucke{\weitereDrucke{Arthur Schnitzler, Richard Beer-Hofmann: \emph{Briefwechsel 1891–1931}. Hg. Konstanze Fliedl. Wien, Zürich: \emph{Europaverlag} 1992, S. 172.} }\pstart{}{\pb}Richard Beerhofmann\pend{}\pstart{}\textcolor{gray}{\textbf{\textit{\textcolor{pink}{Rodaun}{}\ledrightnote{\textcolor{pink}{Rodaun}}}}}\pend{}{\bigskip}\pstart
           \noindent{}Gratulire von Herzen zum \textcolor{brown}{Schillerpreis}{}\ledrightnote{\textcolor{brown}{Volks-Schillerpreis}} freuten uns
               sehr möchte zur \textcolor{green}{Charolais}{}\ledrightnote{\textcolor{green}{Der Graf von Charolais. Ein Trauerspiel}}premiéren zwei Sitze
               möglichst erste Reihe bitte Nachricht wann wo abholbar viele Grüsse von Haus zu
               Haus\pend
           \pstart = Ihr \spacefill\mbox{Arthur}\pend{}\endnumbering\briefempfaengerindex{Beer-Hofmann, Richard@\textsc{Beer-Hofmann, Richard}!zzzSchnitzler, Arthur@\emph{von Arthur Schnitzler}!1905-05-081@{8. 5. 1905}|)be}\mylabel{h}  \normalsize

\doendnotes{C}
\bigskip
\vfill

\clearpage

\footnotesize

\lohead{\textsc{register}}

% Definiere theindex-Environment komplett neu ohne reledmac
\makeatletter
\renewenvironment{theindex}{%
  \section*{\indexname}%
  \setlength{\parindent}{0pt}%
  \setlength{\parskip}{0pt plus 0.3pt}%
  \let\item\@idxitem
}{%
  \clearpage
}
\makeatother

\IfFileExists{\jobname-pw.ind}{\input{\jobname-pw.ind}}{}

\end{document}

      