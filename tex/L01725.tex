%% latex-korrekturansicht-vorspann.tex
%% Vorspann für die Korrekturansicht.
%% Lädt die gemeinsame Datei latex-vorspann.tex mit gesetztem Schalter.

\newif\ifkorrekturansicht
\korrekturansichttrue

\input{../tex-inputs/latex-vorspann}


               \section[Olga Schnitzler an Richard Beer-Hofmann, {[}19. 10. 1907{]}]{ Olga Schnitzler an Richard Beer-Hofmann,
               {[}19. 10. 1907{]}}\nopagebreak\mylabel{v}\rehead{ }\normalsize\beginnumbering\briefempfaengerindex{Beer-Hofmann, Richard@\textsc{Beer-Hofmann, Richard}!zzzSchnitzler, Olga@\emph{von Olga Schnitzler}!1907-10-192@{{[}19. 10. 1907{]}}|(be} \toendnotes[C]{\smallbreak\pagebreak[2]} \Standort{YCGL, MSS 31.}
\physDesc{Briefkarte, Umschlag
\newline{}Handschrift: Bleistift, lateinische Kurrent\newline{}Versand: ohne postalischen Übermittlungsvermerk }\toendnotes[C]{\smallbreak}\pstart{}{\pb}\textcolor{gray}{\textbf{O. S.}}\pend{}{\bigskip}\pstart{}{\pb}Herrn D\textsuperscript{r} Beer-Hofmann
               \pend{}{\bigskip}\pstart
           \noindent{}{\pb}\textcolor{gray}{\textbf{O. S.}}\pend
           \pstart
           Lieber Herr Doctor, wie schade! Aber wir haben \textcolor{blue}{Speidels}{}\ledrightnote{\textcolor{blue}{Felix Speidel}{\newline}\textcolor{blue}{Else Speidel-Haeberle}} getroffen, die kommen zu uns, nach dem Nachtmal
               liest \textcolor{blue}{er}{}\ledrightnote{→\textcolor{blue}{Felix Speidel}} uns sein {\pb}neues \textcolor{green}{Stück}{}\ledrightnote{→\textcolor{green}{Föhn}} vor.\pend
           \pstart
           Pech, Pech, Pech!\pend
           \pstart
           \textcolor{blue}{Arth.}{}\ledrightnote{} dictiert, lässt Alle herzlichst grüssen.
               Ebenso\pend
           \pstart \spacefill\mbox{O.}\pend{}\endnumbering\briefempfaengerindex{Beer-Hofmann, Richard@\textsc{Beer-Hofmann, Richard}!zzzSchnitzler, Olga@\emph{von Olga Schnitzler}!1907-10-192@{{[}19. 10. 1907{]}}|)be}\mylabel{h}  \normalsize

\doendnotes{C}
\bigskip
\vfill

\clearpage

\footnotesize

\lohead{\textsc{register}}

% Definiere theindex-Environment komplett neu ohne reledmac
\makeatletter
\renewenvironment{theindex}{%
  \section*{\indexname}%
  \setlength{\parindent}{0pt}%
  \setlength{\parskip}{0pt plus 0.3pt}%
  \let\item\@idxitem
}{%
  \clearpage
}
\makeatother

\IfFileExists{\jobname-pw.ind}{\input{\jobname-pw.ind}}{}

\end{document}

      