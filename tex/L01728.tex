%% latex-korrekturansicht-vorspann.tex
%% Vorspann für die Korrekturansicht.
%% Lädt die gemeinsame Datei latex-vorspann.tex mit gesetztem Schalter.

\newif\ifkorrekturansicht
\korrekturansichttrue

\input{../tex-inputs/latex-vorspann}


               \section[Max Mell an Arthur Schnitzler, 4. 11. 1907]{ Max Mell an Arthur Schnitzler, 4. 11. 1907}\nopagebreak\mylabel{v}\rehead{ }\normalsize\beginnumbering\briefempfaengerindex{Schnitzler, Arthur@\textsc{Schnitzler, Arthur}!zzzMell, Max@\emph{von Max Mell}!1907-11-041@{4. 11. 1907}|(be} \toendnotes[C]{\smallbreak\pagebreak[2]} \Standort{DLA, A:Schnitzler, HS.NZ85.1.4055, S. [6].}
\physDesc{maschinelle Abschrift}\toendnotes[C]{\smallbreak}\pstart
           \raggedleft{}{\pb}4. November 1907.\pend
           \pstart\center{}Verehrter Herr Doktor,\pend\pstart
           die Tänzerinnen Schwestern \textcolor{blue}{Wiesenthal}{}\ledrightnote{\textcolor{blue}{Grethe Wiesenthal}{\newline}\textcolor{blue}{Elsa Wiesenthal}{\newline}\textcolor{blue}{Berta Wiesenthal}} veranstalten am \label{K_L01728-1v}\edtext{Mittwoch}{\lemma{\textnormal{\emph{Mittwoch}}}\Cendnote{\textnormal{\textcolor{blue}{Schnitzler} nahm die
                        Einladung nicht an, er war am 6. 11. 1907 auf einer Verbandssitzung.}}}\label{K_L01728-1h} einen
                    Tanzabend – sollte es Sie und Ihre Frau \textcolor{blue}{Gemahlin}{}\ledrightnote{→\textcolor{blue}{Olga Schnitzler}} interessieren, so kommen Sie doch bitte dazu! Es
                    findet im Atelier des Malers \textcolor{blue}{Huber}{}\ledrightnote{\textcolor{blue}{Rudolf Huber-Wiesenthal}}, \textcolor{pink}{IV. Taubstummengasse 2}{}\ledrightnote{\textcolor{pink}{Taubstummengasse}}, statt, um ½ 8
                        abends, und es werden ausser mir nur noch \textcolor{blue}{Kolo Moser}{}\ledrightnote{\textcolor{blue}{Koloman Moser}} und \textcolor{blue}{Josef
                        Hoffmann}{}\ledrightnote{\textcolor{blue}{Josef Hoffmann}} dort sein, allenfalls \textcolor{blue}{Waerndorfer}{}\ledrightnote{\textcolor{blue}{Friedrich Wärndorfer}}. Die \textcolor{blue}{Wiesenthals}{}\ledrightnote{\textcolor{blue}{Grethe Wiesenthal}{\newline}\textcolor{blue}{Elsa Wiesenthal}{\newline}\textcolor{blue}{Berta Wiesenthal}} wären über Ihr Kommen sehr erfreut, ich wurde gebeten, Sie
                    zu benachrichtigen.\pend
           \pstart
           Mit vielen Empfehlungen{\\[\baselineskip]}Ihr stets ergebener{\\[\baselineskip]}\spacefill\mbox{Max
                        Mell}\pend
           \leftskip=0em{}\endnumbering\briefempfaengerindex{Schnitzler, Arthur@\textsc{Schnitzler, Arthur}!zzzMell, Max@\emph{von Max Mell}!1907-11-041@{4. 11. 1907}|)be}\mylabel{h}  \normalsize

\doendnotes{C}
\bigskip
\vfill

\clearpage

\footnotesize

\lohead{\textsc{register}}

% Definiere theindex-Environment komplett neu ohne reledmac
\makeatletter
\renewenvironment{theindex}{%
  \section*{\indexname}%
  \setlength{\parindent}{0pt}%
  \setlength{\parskip}{0pt plus 0.3pt}%
  \let\item\@idxitem
}{%
  \clearpage
}
\makeatother

\IfFileExists{\jobname-pw.ind}{\input{\jobname-pw.ind}}{}

\end{document}

      