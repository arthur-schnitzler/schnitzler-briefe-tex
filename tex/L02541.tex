%% latex-korrekturansicht-vorspann.tex
%% Vorspann für die Korrekturansicht.
%% Lädt die gemeinsame Datei latex-vorspann.tex mit gesetztem Schalter.

\newif\ifkorrekturansicht
\korrekturansichttrue

\input{../tex-inputs/latex-vorspann}


               \section[Arthur Schnitzler an Gabriel Beer-Hofmann, 14. 1. 1931]{ Arthur Schnitzler an Gabriel Beer-Hofmann, 14. 1. 1931}\nopagebreak\mylabel{v}\rehead{ }\normalsize\beginnumbering\briefempfaengerindex{Beer-Hofmann, Gabriel@\textsc{Beer-Hofmann, Gabriel}!zzzSchnitzler, Arthur@\emph{von Arthur Schnitzler}!1931-01-141@{14. 1. 1931}|(be} \toendnotes[C]{\smallbreak\pagebreak[2]} \Standort{DLA, A:Schnitzler, HS.NZ.1.339.}
\physDesc{Brief, 1 Blatt, 1 Seite, Durchschlag?
\newline{}Schreibmaschine
\newline{}Handschrift Frieda Pollak: roter Buntstift, deutsche Kurrent (\noindent{}zwei Unterstreichungen, Beschriftung:
                                                »Beer-Hofmann« und »\textcolor{pink}{U.S.A}«)\newline{}Ordnung: mit schwarzer Tinte von unbekannter Hand die maschinschriftliche Unterschrift »Arthur« um
                                                  »Sc\textcolor{gray}{h}« erweitert }\toendnotes[C]{\smallbreak}\pstart
           \raggedleft{}{\pb}14. 1. 1931. \pend
           \pstart
           Gabriel Beer-Hofmann \textcolor{pink}{Mayflower Hotel}{}\ledrightnote{\textcolor{pink}{Mayflower Hotel}}{ }\textcolor{pink}{Centralpark West New York}{}\ledrightnote{\textcolor{pink}{Central Park West}}\pend
           \pstart
           In froher Zuversicht, mein lieber Gabriel, dass Deine junge liebevoll sichere
                    Führung im Verein mit den vortrefflichen Schauspielern, dem guten alten \textcolor{green}{Anatol}{}\ledrightnote{\textcolor{green}{Anatol}} einen \label{K_L02541_1v}\edtext{neuen Erfolg}{\lemma{\textnormal{\emph{neuen Erfolg}}}\Cendnote{\textnormal{Am 16. 1. 1931 hatte \emph{\textcolor{green}{Anatol}}
                        in der Bearbeitung von \textcolor{blue}{Harley
                            Granville-Parker} und mit \textcolor{blue}{Joseph
                            Schildkraut} am \textcolor{pink}{Lyceum-Theatre} in
                            \textcolor{pink}{New York} Premiere.}}}\label{K_L02541_1h} bringen
                    wird bin ich mit den herzlichsten Wünschen und allen freundschaftlichen
                    Gefühlen\pend
           \pstart
           Dein{\\[\baselineskip]}\spacefill\mbox{Arthur}\pend
           \leftskip=0em{}\endnumbering\briefempfaengerindex{Beer-Hofmann, Gabriel@\textsc{Beer-Hofmann, Gabriel}!zzzSchnitzler, Arthur@\emph{von Arthur Schnitzler}!1931-01-141@{14. 1. 1931}|)be}\mylabel{h}  \normalsize

\doendnotes{C}
\bigskip
\vfill

\clearpage

\footnotesize

\lohead{\textsc{register}}

% Definiere theindex-Environment komplett neu ohne reledmac
\makeatletter
\renewenvironment{theindex}{%
  \section*{\indexname}%
  \setlength{\parindent}{0pt}%
  \setlength{\parskip}{0pt plus 0.3pt}%
  \let\item\@idxitem
}{%
  \clearpage
}
\makeatother

\IfFileExists{\jobname-pw.ind}{\input{\jobname-pw.ind}}{}

\end{document}

      