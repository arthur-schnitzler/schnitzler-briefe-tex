%% latex-korrekturansicht-vorspann.tex
%% Vorspann für die Korrekturansicht.
%% Lädt die gemeinsame Datei latex-vorspann.tex mit gesetztem Schalter.

\newif\ifkorrekturansicht
\korrekturansichttrue

\input{../tex-inputs/latex-vorspann}


               \section[Georg Brandes an Arthur Schnitzler, 7. 1. 1899]{ Georg Brandes an Arthur Schnitzler, 7. 1. 1899}\nopagebreak\mylabel{v}\rehead{ }\normalsize\beginnumbering\briefempfaengerindex{Schnitzler, Arthur@\textsc{Schnitzler, Arthur}!zzzBrandes, Georg@\emph{von Georg Brandes}!1899-01-071@{7. 1. 1899}|(be} \toendnotes[C]{\smallbreak\pagebreak[2]} \Standort{CUL, Schnitzler, B 17.}
\physDesc{Brief, 1 Blatt, 4 Seiten
\newline{}Handschrift: Bleistift, lateinische Kurrent\newline{}Ordnung: mit Bleistift von unbekannter Hand nummeriert: »11« }\buchAbdrucke{\weitereDrucke{Georg Brandes, Arthur Schnitzler: \emph{Ein Briefwechsel}. Hg. Kurt Bergel. Bern: \emph{Francke} 1956, S. 69–70.} }\toendnotes[C]{\smallbreak}\pstart
           \raggedleft{}{\pb}\textcolor{pink}{Kopenhagen}{}\ledrightnote{\textcolor{pink}{Kopenhagen}}{ }7 Jan. 99\pend
           \pstart{}Lieber Dr. Schnitzler, sehr guter Freund\pend\pstart
           Haben Sie Dank für Ihre Zeilen. Was habe ich nicht alles erlebt seit ich Sie sah.
                    Jetzt liege ich wieder zu Bett; die Venenentzündung ist zurückgekehrt.\pend
           \pstart
           Ich blieb ein halbes Jahr in \textcolor{pink}{Italien}{}\ledrightnote{\textcolor{pink}{Italien}}, kam
                    zurück, gab hier zwei Bücher aus, einen Band meiner \textcolor{green}{Gedichte}{}\ledrightnote{→\textcolor{green}{Ungdomsvers [Jugendgedichte]}} (staunen Sie?) und ein \textcolor{green}{Buch über einen verstorbenen Freund}{}\ledrightnote{→\textcolor{green}{Julius Lange}}, das
                    hier einen sehr grossen Erfolg gehabt hat –, in 8 Tagen ausverkauft. Reiste
                    wieder aus, wurde zwei Mal zurückgerufen durch Depeschen, {\pb}weil meine \textcolor{blue}{Mutter}{}\ledrightnote{→\textcolor{blue}{Emilie Brandes}} krank war. Das letzte Mal war ich
                    in \textcolor{pink}{Polen}{}\ledrightnote{\textcolor{pink}{Polen}}, wo ich wegen meines \textcolor{green}{Buches über Polen}{}\ledrightnote{→\textcolor{green}{Polen}} (das deutsch und
                    polnisch übersetzt worden) eingeladen und komisch vergöttert wurde.\pend
           \pstart
           Zurück in einem Zug aus \textcolor{pink}{Lemberg}{}\ledrightnote{\textcolor{pink}{Lviv}}. Sah meine \textcolor{blue}{Mutter}{}\ledrightnote{→\textcolor{blue}{Emilie Brandes}} 14 Tage dann selbst
                    krank, konnte meine \textcolor{blue}{Mutter}{}\ledrightnote{→\textcolor{blue}{Emilie Brandes}}
                    nicht sehen in der letzten Woche ihres Lebens und nicht an ihrer Beerdigung
                    dasein. Ich habe \uline{nie einen einzigen Tag} in \textcolor{pink}{Kopenhagen}{}\ledrightnote{\textcolor{pink}{Kopenhagen}} versäumt meine \textcolor{blue}{Mutter}{}\ledrightnote{→\textcolor{blue}{Emilie Brandes}} zu besuchen.\pend
           \pstart
           Und jetzt liege ich in Streit mit den \textcolor{pink}{Deutschen}{}\ledrightnote{\textcolor{pink}{Deutschland}}
                    wegen der Austreibung der \textcolor{pink}{Dänen}{}\ledrightnote{\textcolor{pink}{Dänemark}} aus \textcolor{pink}{Schleswig}{}\ledrightnote{\textcolor{pink}{Südschleswig}}. Gibt es etwas widerlicheres als {\pb}\textcolor{pink}{Preussen}{}\ledrightnote{\textcolor{pink}{Preußen}}? Nicht \textcolor{pink}{Frankreich}{}\ledrightnote{\textcolor{pink}{Frankreich}} einmal.\pend
           \pstart
           Mit ruhiger geniessender Freude las ich Ihr \textcolor{green}{\uline{Vermächtnis}}{}\ledrightnote{\textcolor{green}{Das Vermächtnis. Schauspiel in drei Akten}}. Es ist ein völlig
                    originales Ding, sehr discret und vornehm, tief pessimistisch und human. (Kennen
                    Sie zufällig eine kleine Erzählung von \textcolor{blue}{Huysmans}{}\ledrightnote{\textcolor{blue}{Joris-Karl Huysmans}} \textcolor{green}{Un dilemme}{}\ledrightnote{\textcolor{green}{Ein Dilemma}} die behandelt
                    ein ähnliches Thema, nur viel gröber oder richtiger ganz anders, aber es ist da
                    ein bischen Verwandtschaft).\pend
           \pstart
           Es ist nur Schade, dass das \textcolor{green}{Stück}{}\ledrightnote{→\textcolor{green}{Das Vermächtnis. Schauspiel in drei Akten}}
               so ganz und gar traurig ist,
                    dann wird es nicht so viel Bühnenerfolg haben können, {\pb}wie ich es wünschte. Der
                    Vater ist wunderbar gezeichnet. Aber überhaupt ich hab Ihr Talent so lieb. Etwas
                    freut mich schon, weil es von Ihnen ist.\pend
           \pstart
           Warum lässt doch unser Freund \textcolor{blue}{Beer Hofmann}{}\ledrightnote{\textcolor{blue}{Richard Beer-Hofmann}}
                    nie von sich hören? Ist er ein bischen faul? Er ist doch ein so feiner
                    Mensch.\pend
           \pstart
           Denken Sie, was es heisst für einen Mann von meinem Temperament still zu liegen,
                    Geduld haben zu sollen und \uline{wieder}, nachdem ich
                    Ein Mal ein halbes Jahr so verlor.\pend
           \pstart
           Behalten Sie mich lieb\pend
           \pstart
           Ihr ergebener{\\[\baselineskip]}\spacefill\mbox{Georg Brandes}\pend
           \leftskip=0em{}\endnumbering\briefempfaengerindex{Schnitzler, Arthur@\textsc{Schnitzler, Arthur}!zzzBrandes, Georg@\emph{von Georg Brandes}!1899-01-071@{7. 1. 1899}|)be}\mylabel{h}  \normalsize

\doendnotes{C}
\bigskip
\vfill

\clearpage

\footnotesize

\lohead{\textsc{register}}

% Definiere theindex-Environment komplett neu ohne reledmac
\makeatletter
\renewenvironment{theindex}{%
  \section*{\indexname}%
  \setlength{\parindent}{0pt}%
  \setlength{\parskip}{0pt plus 0.3pt}%
  \let\item\@idxitem
}{%
  \clearpage
}
\makeatother

\IfFileExists{\jobname-pw.ind}{\input{\jobname-pw.ind}}{}

\end{document}

      