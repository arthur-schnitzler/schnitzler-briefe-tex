%% latex-korrekturansicht-vorspann.tex
%% Vorspann für die Korrekturansicht.
%% Lädt die gemeinsame Datei latex-vorspann.tex mit gesetztem Schalter.

\newif\ifkorrekturansicht
\korrekturansichttrue

\input{../tex-inputs/latex-vorspann}


               \section[Arthur Schnitzler an Hermann Bahr, 24. 8. 1918]{ Arthur Schnitzler an Hermann Bahr, 24. 8. 1918}\nopagebreak\mylabel{v}\rehead{ }\normalsize\beginnumbering\briefempfaengerindex{Bahr, Hermann@\textsc{Bahr, Hermann}!zzzSchnitzler, Arthur@\emph{von Arthur Schnitzler}!1918-08-241@{24. 8. 1918}|(be} \toendnotes[C]{\smallbreak\pagebreak[2]} \Standort{TMW, HS AM 39902 Ba.}
\physDesc{Briefkarte
\newline{}Handschrift: schwarze Tinte, lateinische Kurrent\newline{}Ordnung: Lochung }\buchAbdrucke{\weitereDrucke{1) \emph{24. 8. 1918.} In: Arthur Schnitzler: \emph{The Letters of Arthur Schnitzler to Hermann Bahr}. Edited, annotated, and with an introduction, by Donald G.
                        Daviau. Chapel Hill: \emph{The University of North Carolina Press} 1978, S. 114 (University of North Carolina studies in the Germanic languages
                        and literatures, 89).} \weitereDrucke{2) Hermann Bahr, Arthur Schnitzler: \emph{Briefwechsel, Aufzeichnungen, Dokumente (1891–1931)}. Hg. Kurt Ifkovits und Martin Anton Müller. Göttingen: \emph{Wallstein} 2018, S. 511–512.} }\toendnotes[C]{\smallbreak}\pstart
           \noindent{}{\pb}\textcolor{gray}{\textbf{Dr. Arthur Schnitzler}}\hfill \textcolor{pink}{Wien}{}\ledrightnote{\textcolor{pink}{Wien}}, 24. 8. 18\pend
           \pstart
           \textcolor{gray}{\textbf{\textcolor{pink}{Wien XVIII. Sternwartestrasse 71}{}\ledrightnote{\textcolor{pink}{Sternwartestraße}}}}\pend
           \pstart
           lieber Hermann, ein begabter junger Componist, Musikdirector, (mein
                  \textcolor{blue}{Sohn}{}\ledrightnote{→\textcolor{blue}{Heinrich Schnitzler}}
               studirt Harmonielehre
               u. Clarinette bei ihm) hat deine \textcolor{green}{\textcolor{green}{Pantomime vom braven Mann}{}\ledrightnote{\textcolor{green}{Die Pantomime vom braven Manne}}}{}\ledrightnote{\textcolor{green}{Die Pantomime vom braven Mann op. 32}} in einer mir sehr interessant erscheinenden Weise vertont und möchte nicht nur
               deine nachträgliche Autorisation erbitten sondern hegt den begreiflichen Wunsch, dir
               die Sache einmal vorzuspielen. Vielleicht bist du so gütig und gibst dem jungen
               Künstler (sein Name ist {\pb}\textcolor{blue}{Arthur Johannes Scholz}{}\ledrightnote{\textcolor{blue}{Arthur Johannes Scholz}} – Gelegenheit dazu, wenn du
               dich, was ja (– wenn die \label{K_L02299_1v}\edtext{Zeitungsnachrichten}{\lemma{\textnormal{\emph{Zeitungsnachrichten}}}\Cendnote{\textnormal{Ab
                     18. 8. 1918 wurde mehrfach gemeldet, \textcolor{blue}{Bahr} gehe nicht als Direktor, sondern als künstlerischer
                  Beirat für ein Jahr ans \textcolor{pink}{Burgtheater}. Die
                  offizielle Bestätigung erfolgte erst nach diesem Brief.}}}\label{K_L02299_1h}
               sti{\geminationm}en) nun öfters der Fall sein dürfte, für ein paar Tage
               in \textcolor{pink}{Wien}{}\ledrightnote{\textcolor{pink}{Wien}} aufhältst?\pend
           \pstart
           Wie lang hab ich dich nun schon nicht gesehn und gesprochen. Nun wirds hoffentlich
               nicht mehr so lange dauern wie seit dem letzten Mal!\pend
           \pstart Sei herzlichst gegrüßt von Deinem alten \spacefill\mbox{Art}\pend{}\endnumbering\briefempfaengerindex{Bahr, Hermann@\textsc{Bahr, Hermann}!zzzSchnitzler, Arthur@\emph{von Arthur Schnitzler}!1918-08-241@{24. 8. 1918}|)be}\mylabel{h}  \normalsize

\doendnotes{C}
\bigskip
\vfill

\clearpage

\footnotesize

\lohead{\textsc{register}}

% Definiere theindex-Environment komplett neu ohne reledmac
\makeatletter
\renewenvironment{theindex}{%
  \section*{\indexname}%
  \setlength{\parindent}{0pt}%
  \setlength{\parskip}{0pt plus 0.3pt}%
  \let\item\@idxitem
}{%
  \clearpage
}
\makeatother

\IfFileExists{\jobname-pw.ind}{\input{\jobname-pw.ind}}{}

\end{document}

      