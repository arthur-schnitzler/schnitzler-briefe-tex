%% latex-korrekturansicht-vorspann.tex
%% Vorspann für die Korrekturansicht.
%% Lädt die gemeinsame Datei latex-vorspann.tex mit gesetztem Schalter.

\newif\ifkorrekturansicht
\korrekturansichttrue

\input{../tex-inputs/latex-vorspann}


               \section[Thomas Mann an Arthur Schnitzler, 25. 12. 1927]{ Thomas Mann an Arthur Schnitzler, 25. 12. 1927}\nopagebreak\mylabel{v}\rehead{ }\normalsize\beginnumbering\briefempfaengerindex{Schnitzler, Arthur@\textsc{Schnitzler, Arthur}!zzzMann, Thomas@\emph{von Thomas Mann}!1927-12-252@{25. 12. 1927}|(be} \toendnotes[C]{\smallbreak\pagebreak[2]} \Standort{CUL, Schnitzler, B 67.}
\physDesc{Postkarte
\newline{}Handschrift: schwarze Tinte, deutsche Kurrent\newline{}Versand: Stempel: »\nobreak{}\oindex{Muenchen@\textbf{München}, \emph{https://www.geonames.org/ontologyP.PPLA}|pwk}München, 25. 12. 1927, 9–10\textcolor{gray}{N}\nobreak{}«.  
\newline{}Schnitzler: 1) mit rotem Buntstift beschrieben mit »\textsc{Aph}{[}orismen{]}« 2) mit rotem Buntstift zwei Unterstreichungen}\buchAbdrucke{\weitereDrucke{Hertha Krotkoff: \emph{Arthur Schnitzler – Thomas Mann: Briefe.} In: \emph{Modern Austrian Literature}, Jg. 7 (1974) Nr. 1/2, S. 25.} }\toendnotes[C]{\smallbreak}\pstart{}{\pb}Herrn\pend{}\pstart{}Dr. Arthur \textsc{Schnitzler}\pend{}\pstart{}\textcolor{pink}{Wien XVIII}{}\ledrightnote{\textcolor{pink}{XVIII., Währing}}\pend{}\pstart{}\textcolor{pink}{Sternwarteſtr. 78}{}\ledrightnote{\textcolor{pink}{Sternwartestraße}}.\pend{}{\bigskip}\pstart
           \noindent{}{\pb}\textcolor{gray}{\textbf{\textsc{Dr. Thomas Mann}}}\hfill \textcolor{gray}{\textbf{\textcolor{pink}{MÜNCHEN 27}{}\ledrightnote{\textcolor{pink}{München}}, den}}{ }25. XII. 27.\pend
           \pstart
           \raggedleft{}\textcolor{gray}{\textbf{\textcolor{pink}{POSCHINGERSTR. 1}{}\ledrightnote{\textcolor{pink}{Poschingerstraße}}}}\pend
           \pstart{}Lieber, verehrter Arthur Schnitzler,\pend\pstart
           von Herzen Dank für das Weihnachtsgeſchenk Ihres \textcolor{green}{Spruch-Buches}{}\ledrightnote{→\textcolor{green}{Buch der Sprüche und Bedenken}}, das ſo voll iſt von ſchön und klar
                    geformter Weisheit! Sie ſind ganz darin mit Ihrer Unbeſtechlichkeit, Freiheit
                        \damage{\textcolor{gray}{und}} Güte, und nicht nur im Einzelnen, ſondern als Ganzes iſt es
                    liebenswert.\pend
           \pstart
           Ein glückliches neues Jahr wünſcht Ihnen{\\[\baselineskip]}Ihr treu ergebener{\\[\baselineskip]}\spacefill\mbox{Thomas Mann.}\pend
           \leftskip=0em{}\endnumbering\briefempfaengerindex{Schnitzler, Arthur@\textsc{Schnitzler, Arthur}!zzzMann, Thomas@\emph{von Thomas Mann}!1927-12-252@{25. 12. 1927}|)be}\mylabel{h}  \normalsize

\doendnotes{C}
\bigskip
\vfill

\clearpage

\footnotesize

\lohead{\textsc{register}}

% Definiere theindex-Environment komplett neu ohne reledmac
\makeatletter
\renewenvironment{theindex}{%
  \section*{\indexname}%
  \setlength{\parindent}{0pt}%
  \setlength{\parskip}{0pt plus 0.3pt}%
  \let\item\@idxitem
}{%
  \clearpage
}
\makeatother

\IfFileExists{\jobname-pw.ind}{\input{\jobname-pw.ind}}{}

\end{document}

      