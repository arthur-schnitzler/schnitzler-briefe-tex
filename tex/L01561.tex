%% latex-korrekturansicht-vorspann.tex
%% Vorspann für die Korrekturansicht.
%% Lädt die gemeinsame Datei latex-vorspann.tex mit gesetztem Schalter.

\newif\ifkorrekturansicht
\korrekturansichttrue

\input{../tex-inputs/latex-vorspann}


               \section[Hermann Bahr an Arthur Schnitzler, {[}13. 10. 1905{]}]{ Hermann Bahr an Arthur Schnitzler, {[}13. 10. 1905{]}}\nopagebreak\mylabel{v}\rehead{ }\normalsize\beginnumbering\briefempfaengerindex{Schnitzler, Arthur@\textsc{Schnitzler, Arthur}!zzzBahr, Hermann@\emph{von Hermann Bahr}!1905-10-131@{13. 10. 1905}|(be} \toendnotes[C]{\smallbreak\pagebreak[2]} \Standort{CUL, Schnitzler, B 5b.}
\physDesc{Telegramm
\newline{}Handschrift einer Schreibkraft: blaue Tinte, deutsche Kurrent\newline{}Versand: Uhrzeit teilweise lesbar: »\textcolor{gray}{9} Uhr 10 Min. V. Mittag« 
\newline{}Schnitzler: mit Bleistift datiert: »13/X 905« \newline{}Ordnung: 1) mit Bleistift von unbekannter Hand
                           nummeriert: »135« 2) beschnitten}\buchAbdrucke{\weitereDrucke{Hermann Bahr, Arthur Schnitzler: \emph{Briefwechsel, Aufzeichnungen, Dokumente (1891–1931)}. Hg. Kurt Ifkovits und Martin Anton Müller. Göttingen: \emph{Wallstein} 2018, S. 361.} }\pstart
           \noindent{}{\pb}herzlichſt grüſſt \substVorne{}\textsuperscript{D}\substDazwischen{}d\substHinten{}ich dein getreuer fürſtenfeind{\\}\spacefill\mbox{hermann Bahr}\pend
           \endnumbering\briefempfaengerindex{Schnitzler, Arthur@\textsc{Schnitzler, Arthur}!zzzBahr, Hermann@\emph{von Hermann Bahr}!1905-10-131@{13. 10. 1905}|)be}\mylabel{h}  \normalsize

\doendnotes{C}
\bigskip
\vfill

\clearpage

\footnotesize

\lohead{\textsc{register}}

% Definiere theindex-Environment komplett neu ohne reledmac
\makeatletter
\renewenvironment{theindex}{%
  \section*{\indexname}%
  \setlength{\parindent}{0pt}%
  \setlength{\parskip}{0pt plus 0.3pt}%
  \let\item\@idxitem
}{%
  \clearpage
}
\makeatother

\IfFileExists{\jobname-pw.ind}{\input{\jobname-pw.ind}}{}

\end{document}

      