%% latex-korrekturansicht-vorspann.tex
%% Vorspann für die Korrekturansicht.
%% Lädt die gemeinsame Datei latex-vorspann.tex mit gesetztem Schalter.

\newif\ifkorrekturansicht
\korrekturansichttrue

\input{../tex-inputs/latex-vorspann}


               \section[Arthur Schnitzler an Richard Beer-Hofmann, 22. 8. 1892]{ Arthur Schnitzler an Richard Beer-Hofmann, 22. 8. 1892}\nopagebreak\mylabel{v}\rehead{ }\normalsize\beginnumbering\briefempfaengerindex{Beer-Hofmann, Richard@\textsc{Beer-Hofmann, Richard}!zzzSchnitzler, Arthur@\emph{von Arthur Schnitzler}!1892-08-221@{22. 8. 1892}|(be} \toendnotes[C]{\smallbreak\pagebreak[2]} \Standort{YCGL, MSS 31.}
\physDesc{Brief, 1 Blatt (Briefpapier mit Trauerrand), 3 Seiten, Umschlag
\newline{}Handschrift: schwarze Tinte, deutsche Kurrent\newline{}Versand: 1) Stempel: »\nobreak{}Wien 4/1, 22 8 92, 6–7N\nobreak{}«.  2) Stempel: »\nobreak{}\oindex{Bad Ischl@\textbf{Bad Ischl}, \emph{Besiedelter Ort (A.BSO)}|pwk}{\pb}Ischl, 23 8 9{[}2{]}, 7–8\nobreak{}«. }\buchAbdrucke{\weitereDrucke{Arthur Schnitzler, Richard Beer-Hofmann: \emph{Briefwechsel 1891–1931}. Hg. Konstanze Fliedl. Wien, Zürich: \emph{Europaverlag} 1992, S. 37–38.} }\toendnotes[C]{\smallbreak}\pstart{}{\pb}Herrn Doctor \textsc{Rich. Beer-Hofmann}\pend{}\pstart{}\textsc{\textcolor{pink}{Ischl}{}\ledrightnote{\textcolor{pink}{Bad Ischl}}.}\pend{}\pstart{}\textsc{\textcolor{pink}{Grazerstraße 6}{}\ledrightnote{\textcolor{pink}{Grazer Straße}}}.\pend{}\pstart{}(oder \textcolor{pink}{\textsc{Kreuzplatz}}{}\ledrightnote{\textcolor{pink}{Kreuzplatz}})\pend{}{\bigskip}\pstart
           \noindent{}{\pb}Mein lieber Richard! Warum ſchreiben Sie \textcolor{pink}{Opernring 12}{}\ledrightnote{\textcolor{pink}{Opernring}}; da ich doch \textcolor{pink}{Kärnthnerring
                  12}{}\ledrightnote{\textcolor{pink}{Kärntnerring}} oder \textcolor{pink}{Giſelastr. 11}{}\ledrightnote{\textcolor{pink}{Bösendorferstraße}} wohne?
               Dadurch bekam ich erſt heute Ihren Brief. Nun kann ich Ihnen mittheilen, daſs ich
               ſchon in wenig Tagen, Ende dieſer Woche, in \textcolor{pink}{Iſchl}{}\ledrightnote{\textcolor{pink}{Bad Ischl}}
               einlangen werde. Ich bleibe etwa 8-10 Tage dort und will jedenfalls weiter. Laſſen
               Sie mich Sie übrigens beneiden, {\pb}daſs Sie \uline{verſti{\geminationm}t}{ }ſind; es iſt das
               ſicherſte Zeichen, daſs Sie nicht unglücklich ſind. –\pend
           \pstart
           Könnte unſer lieber \textcolor{blue}{Paul}{}\ledrightnote{\textcolor{blue}{Paul Goldmann}} das nicht geſagt haben?
               – Ein reizendes \textcolor{green}{Feuilleton}{}\ledrightnote{→\textcolor{green}{Spanisches Strandleben}} von ihm erſchien eben
               in der \textcolor{green}{Frkf. Ztg}{}\ledrightnote{\textcolor{green}{Frankfurter Zeitung}}; – aus \textcolor{pink}{San Sebastian}{}\ledrightnote{\textcolor{pink}{San Sebastian}}. –\pend
           \pstart
           Ich freue mich ſehr, Sie bald zu ſehn; und da ich heute ſchon in großen Worten drin
               bin, ſo will ich Ihnen geſtehn, daſs ich mich aufrichtig nach Ihnen ſehne.\pend
           \pstart
           {\pb}\strikeout{Vielleicht} Viele herzliche Grüße{\\[\baselineskip]}der Ihre{\\[\baselineskip]}\spacefill\mbox{Arthur}\pend
           \leftskip=0em{}\pstart
           22. 8. 92.\pend
           \endnumbering\briefempfaengerindex{Beer-Hofmann, Richard@\textsc{Beer-Hofmann, Richard}!zzzSchnitzler, Arthur@\emph{von Arthur Schnitzler}!1892-08-221@{22. 8. 1892}|)be}\mylabel{h}  \normalsize

\doendnotes{C}
\bigskip
\vfill

\clearpage

\footnotesize

\lohead{\textsc{register}}

% Definiere theindex-Environment komplett neu ohne reledmac
\makeatletter
\renewenvironment{theindex}{%
  \section*{\indexname}%
  \setlength{\parindent}{0pt}%
  \setlength{\parskip}{0pt plus 0.3pt}%
  \let\item\@idxitem
}{%
  \clearpage
}
\makeatother

\IfFileExists{\jobname-pw.ind}{\input{\jobname-pw.ind}}{}

\end{document}

      