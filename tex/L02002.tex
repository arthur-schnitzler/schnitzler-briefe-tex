%% latex-korrekturansicht-vorspann.tex
%% Vorspann für die Korrekturansicht.
%% Lädt die gemeinsame Datei latex-vorspann.tex mit gesetztem Schalter.

\newif\ifkorrekturansicht
\korrekturansichttrue

\input{../tex-inputs/latex-vorspann}


               \section[Robert Adam an Arthur Schnitzler, 3. 2. 1911]{ Robert Adam an Arthur Schnitzler, 3. 2. 1911}\nopagebreak\mylabel{v}\rehead{ }\normalsize\beginnumbering\briefempfaengerindex{Schnitzler, Arthur@\textsc{Schnitzler, Arthur}!zzzAdam, Robert@\emph{von Robert Adam}!1911-02-031@{3. 2. 1911}|(be} \toendnotes[C]{\smallbreak\pagebreak[2]} \Standort{DLA, A:Schnitzler, HS.NZ85.1.4230,3.}
\physDesc{Brief, 1 Blatt, 4 Seiten
\newline{}Handschrift: schwarze Tinte, deutsche Kurrent
\newline{}Schnitzler: 1) mit Bleistift beschriftet: »\textsc{Adam}« 2)  mit rotem Buntstift eine Unterstreichung}\Standort{Wien, Österreichische Nationalbibliothek, Cod.ser. 52.266, 79.}
\physDesc{handschriftliche Abschrift
\newline{}Handschrift: schwarze Tinte, Gabelsberger Kurzschrift}\Standort{Wien, Österreichische Nationalbibliothek, Cod.ser. 52.266, 79.}
\physDesc{maschinelle Abschrift
\newline{}Schreibmaschine}\toendnotes[C]{\smallbreak}\pstart
           \raggedleft{}{\pb}\textcolor{pink}{Wien}{}\ledrightnote{\textcolor{pink}{Wien}}, am 3. Febr. 1911\pend
           \pstart{}Hochverehrter Herr Doktor!\pend\pstart
           Ich muß Ihnen leider berichten, daß der Verſuch, an mein Glück zu appellieren,
                    fehlgeſchlagen iſt. Der Verlag \textcolor{brown}{S. Fiſcher}{}\ledrightnote{\textcolor{brown}{S. Fischer Verlag}} hat
                    mir mitgeteilt, daß er den »\textcolor{green}{Neidhard}{}\ledrightnote{\textcolor{green}{Neidhard}}« nicht
                    annehmen konnte. Die dem Schreiben beigefügte ſehr liebenswürdige und eingehende
                    Begründung dieſer Entſcheidung dürfte ſich in einem Punkte mit dem Hauptbedenken
                    berühren, das Sie, hochverehrter Herr Doktor, {\pb}bezüglich des ſtofflichen Aufbaus der Komödie mir gegenüber äußerten. Manches
                    iſt mir in der Begründung der Abweiſung nicht recht verſtändlich. Es will mir
                    ſcheinen, als ob der Verlag bei der Fixierung des Grundthemas meiner Komödie
                    fehlgegriffen hätte; wenigſtens iſt das, was im Schreiben als Thema des Stückes
                    bezeichnet wird, nur ein Teil deſſen, was nach meiner Abſicht Thema ſein ſollte.
                    Iſt dem ſo, ſo muß die Komödie unklarer ſein als ich dachte; und dies wäre
                    jedenfalls ein ſehr arger Fehler. Ich war redlich bemüht, den Grundgedanken
                    hervortreten zu laſſen, {\pb}wenn ich es auch –
                    anders als in der \textcolor{green}{arabiſchen
                        Komödie}{}\ledrightnote{→\textcolor{green}{Die Geschichte des Alî ibn Bekkâr mit Schams an-Nahâr}} – abſichtlich vermied, im Kontexte einfach herauszuſagen, was
                    ich durch die Handlung verſinnbildlichen wollte; die Zwiſchenſpiele, als
                    moderniſierter Chor, ſollten das Amt des Räſonneurs übernehmen.\pend
           \pstart
           Dies ſcheint nicht geglückt zu ſein; und um zu verbeſſern, was noch ſich beſſern
                    läßt, will ich einen Plan, den ich ſchon vordem faßte, nun ausführen; nämlich,
                    wenigſtens in einem kritiſchen Nachwort, das in der Form zweier Briefe von
                    Freunden, eines zerreißenden und eines erhebenden, {\pb}gehalten ſein ſoll, all das klar
                    auseinanderſetzen, was Mangel und gute Abſicht der Komödie (nach Anſicht des
                    Autors) iſt.\pend
           \pstart
           Daß mich das Fehlſchlagen dieſer Hoffnung, obwohl ich’s längſt aufgegeben habe,
                    mir Glück zu vindizieren, arg deprimiert, werden Sie begreifen, hochverehrter
                    Herr Doktor; aber ich will’s übertünchen.\pend
           \pstart
           Dem »\textcolor{brown}{Merker}{}\ledrightnote{\textcolor{brown}{Der Merker}}« habe ich die \textcolor{green}{arabiſche Komödie}{}\ledrightnote{\textcolor{green}{Die Geschichte des Alî ibn Bekkâr mit Schams an-Nahâr}} mit einer Empfehlung des D\textsuperscript{r} \textcolor{blue}{Bittner}{}\ledrightnote{\textcolor{blue}{Julius Bittner}}
                    eingeſendet; vorläufig ohne Reſultat.\pend
           \pstart
           Nehmen Sie mir die Länge dieſes Briefes nicht übel, hochverehrter Herr Doktor,
                    und ſeien Sie herzlich gegrüßt von Ihrem\pend
           \pstart
           dankbar ergebenen{\\[\baselineskip]}\spacefill\mbox{Robert Adam}\pend
           \leftskip=0em{}\endnumbering\briefempfaengerindex{Schnitzler, Arthur@\textsc{Schnitzler, Arthur}!zzzAdam, Robert@\emph{von Robert Adam}!1911-02-031@{3. 2. 1911}|)be}\mylabel{h}  \normalsize

\doendnotes{C}
\bigskip
\vfill

\clearpage

\footnotesize

\lohead{\textsc{register}}

% Definiere theindex-Environment komplett neu ohne reledmac
\makeatletter
\renewenvironment{theindex}{%
  \section*{\indexname}%
  \setlength{\parindent}{0pt}%
  \setlength{\parskip}{0pt plus 0.3pt}%
  \let\item\@idxitem
}{%
  \clearpage
}
\makeatother

\IfFileExists{\jobname-pw.ind}{\input{\jobname-pw.ind}}{}

\end{document}

      