%% latex-korrekturansicht-vorspann.tex
%% Vorspann für die Korrekturansicht.
%% Lädt die gemeinsame Datei latex-vorspann.tex mit gesetztem Schalter.

\newif\ifkorrekturansicht
\korrekturansichttrue

\input{../tex-inputs/latex-vorspann}


               \section[Emil Ludwig an Arthur Schnitzler, 3. 2. 1927]{ Emil Ludwig an Arthur Schnitzler, 3. 2. 1927}\nopagebreak\mylabel{v}\rehead{ }\normalsize\beginnumbering\briefempfaengerindex{Schnitzler, Arthur@\textsc{Schnitzler, Arthur}!zzzLudwig, Emil@\emph{von Emil Ludwig}!1927-02-031@{3. 2. 1927}|(be} \toendnotes[C]{\smallbreak\pagebreak[2]} \Standort{CUL, Schnitzler, B 62.}
\physDesc{Bildpostkarte
\newline{}Handschrift: schwarze Tinte, lateinische Kurrent\newline{}Versand: Stempel: »\nobreak{}\oindex{Ascona@\textbf{Ascona}, \emph{https://www.geonames.org/ontologyP.PPL}|pwk}A\textcolor{gray}{s}cona, 3. II. 27, 18\nobreak{}«.  }\toendnotes[C]{\smallbreak}\pstart{}{\pb}Herrn\pend{}\pstart{}Dr. Arthur Schnitzler\pend{}\pstart{}Adr. \textcolor{brown}{S. Fischer Verlag}{}\ledrightnote{\textcolor{brown}{S. Fischer Verlag}}\pend{}\pstart{}\textcolor{pink}{90, Bülowstr. 90}{}\ledrightnote{\textcolor{pink}{Bülowstraße}}\pend{}\pstart{}\textcolor{pink}{Berlin W. 57}{}\ledrightnote{\textcolor{pink}{Berlin}}\pend{}{\bigskip}\pstart
           \noindent{}\centering{}{\pb}\textcolor{gray}{\textbf{{[}Statue eines Bogenschützens mit Blick auf
                        eine \textcolor{pink}{Brissago-Insel}{}\ledrightnote{\textcolor{pink}{Isole di Brissago}}{]}}}\pend
           \pstart
           Dankbar für hochinteressantes \textcolor{green}{Diagramm}{}\ledrightnote{→\textcolor{green}{Der Geist im Wort und der Geist in der Tat}}, \pend
           \pstart
           grüsst Sie Ihr verehrungsvoll ergebener{\\[\baselineskip]}\spacefill\mbox{Ludwig}\pend
           \leftskip=0em{}\pstart
           \textcolor{pink}{Ascona}{}\ledrightnote{\textcolor{pink}{Ascona}}. 3. 2.\pend
           \endnumbering\briefempfaengerindex{Schnitzler, Arthur@\textsc{Schnitzler, Arthur}!zzzLudwig, Emil@\emph{von Emil Ludwig}!1927-02-031@{3. 2. 1927}|)be}\mylabel{h}  \normalsize

\doendnotes{C}
\bigskip
\vfill

\clearpage

\footnotesize

\lohead{\textsc{register}}

% Definiere theindex-Environment komplett neu ohne reledmac
\makeatletter
\renewenvironment{theindex}{%
  \section*{\indexname}%
  \setlength{\parindent}{0pt}%
  \setlength{\parskip}{0pt plus 0.3pt}%
  \let\item\@idxitem
}{%
  \clearpage
}
\makeatother

\IfFileExists{\jobname-pw.ind}{\input{\jobname-pw.ind}}{}

\end{document}

      