%% latex-korrekturansicht-vorspann.tex
%% Vorspann für die Korrekturansicht.
%% Lädt die gemeinsame Datei latex-vorspann.tex mit gesetztem Schalter.

\newif\ifkorrekturansicht
\korrekturansichttrue

\input{../tex-inputs/latex-vorspann}


               \section[Richard Beer-Hofmann an Arthur Schnitzler, {[}30. 7. 1894{]}]{ Richard Beer-Hofmann an Arthur Schnitzler, {[}30. 7. 1894{]}}\nopagebreak\mylabel{v}\rehead{ }\normalsize\beginnumbering\briefempfaengerindex{Schnitzler, Arthur@\textsc{Schnitzler, Arthur}!zzzBeer-Hofmann, Richard@\emph{von Richard Beer-Hofmann}!1894-07-302@{{[}30. 7. 1894{]}}|(be} \toendnotes[C]{\smallbreak\pagebreak[2]} \Standort{CUL, Schnitzler, B 8.}
\physDesc{Telegramm
\newline{}maschinell\newline{}Versand: Stempel des Telegrafenbeamten: »\textcolor{blue}{Edmund Winter}« 
\newline{}Schnitzler: mit Bleistift datiert: »30{[}/7 94{]}« und nummeriert: »30« \newline{}Ordnung: beschnitten }\buchAbdrucke{\weitereDrucke{Arthur Schnitzler, Richard Beer-Hofmann: \emph{Briefwechsel 1891–1931}. Hg. Konstanze Fliedl. Wien, Zürich: \emph{Europaverlag} 1992, S. 58.} }\toendnotes[C]{\smallbreak}\pstart
           \noindent{}{\pb}\textcolor{pink}{wien}{}\ledrightnote{\textcolor{pink}{Wien}} fr \label{T_L00360_1v}\edtext{\textcolor{pink}{ischl}{}\ledrightnote{\textcolor{pink}{Bad Ischl}}}{\lemma{\textnormal{\emph{ischl}}}\Cendnote{\textnormal{Das »i« von unbekannter
                     Hand mit Bleistift ergänzt.}}}\label{T_L00360_1h} 5806 28 12 20 n\pend
           \pstart
           wir sind am zwejten august{ }vormittag in \textcolor{pink}{salzburg oesterreichischer
                  hof}{}\ledrightnote{\textcolor{pink}{Österreichischer Hof}} bitte es dem \label{K_L00360_1v}\edtext{suendentraum}{\lemma{\textnormal{\emph{suendentraum}}}\Cendnote{\textnormal{unklare Anspielung;
                  eventuell auf \textcolor{blue}{Richard Specht}, dessen
                  dramatische Dichtung \emph{\textcolor{green}{Sündentraum}}{ }1892 erschienen war, oder auf \textcolor{blue}{Adele
                     Sandrock}?}}}\label{K_L00360_1h} der in \textcolor{pink}{wien}{}\ledrightnote{\textcolor{pink}{Wien}} ist nicht zu
               sagen herzlichst \spacefill\mbox{= richard +}\pend
           \endnumbering\briefempfaengerindex{Schnitzler, Arthur@\textsc{Schnitzler, Arthur}!zzzBeer-Hofmann, Richard@\emph{von Richard Beer-Hofmann}!1894-07-302@{{[}30. 7. 1894{]}}|)be}\mylabel{h}  \normalsize

\doendnotes{C}
\bigskip
\vfill

\clearpage

\footnotesize

\lohead{\textsc{register}}

% Definiere theindex-Environment komplett neu ohne reledmac
\makeatletter
\renewenvironment{theindex}{%
  \section*{\indexname}%
  \setlength{\parindent}{0pt}%
  \setlength{\parskip}{0pt plus 0.3pt}%
  \let\item\@idxitem
}{%
  \clearpage
}
\makeatother

\IfFileExists{\jobname-pw.ind}{\input{\jobname-pw.ind}}{}

\end{document}

      