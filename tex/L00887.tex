%% latex-korrekturansicht-vorspann.tex
%% Vorspann für die Korrekturansicht.
%% Lädt die gemeinsame Datei latex-vorspann.tex mit gesetztem Schalter.

\newif\ifkorrekturansicht
\korrekturansichttrue

\input{../tex-inputs/latex-vorspann}


               \section[Hugo von Hofmannsthal an Arthur Schnitzler, 8. 2. 1899]{ Hugo von Hofmannsthal an Arthur Schnitzler, 8. 2. 1899}\nopagebreak\mylabel{v}\rehead{ }\normalsize\beginnumbering\briefempfaengerindex{Schnitzler, Arthur@\textsc{Schnitzler, Arthur}!zzzHofmannsthal, Hugo von@\emph{von Hugo von Hofmannsthal}!1899-02-081@{8. 2. 1899}|(be} \toendnotes[C]{\smallbreak\pagebreak[2]} \Standort{CUL, Schnitzler, B 43.}
\physDesc{Postkarte
\newline{}Handschrift: schwarze Tinte, deutsche Kurrent\newline{}Versand: 1) Rohrpost 2) Stempel: »\nobreak{}\oindex{III., Landstrasse@\textbf{III., Landstraße}, \emph{Bezirk (A.BZK)}|pwk}Wien 3/3, 8 II 99, 3 10N\nobreak{}«. 3) Stempel: »\nobreak{}8 {[}II{]} 99, 3 50N\nobreak{}«. 
\newline{}Schnitzler: mit Bleistift datiert: »8/2 99« \newline{}Ordnung: mit Bleistift von unbekannter Hand nummeriert:
                                        »135« }\buchAbdrucke{\weitereDrucke{Hugo von Hofmannsthal, Arthur Schnitzler: \emph{Briefwechsel}. Hg. Therese Nickl und Heinrich Schnitzler. Frankfurt am Main: \emph{S. Fischer} 1964, S. 118.} }\toendnotes[C]{\smallbreak}\pstart{}{\pb}\textsc{Herrn D\textsuperscript{r} Arthur
                            Schnitzler}\pend{}\pstart{}\textsc{\textcolor{pink}{Wien}{}\ledrightnote{\textcolor{pink}{Wien}}}\pend{}\pstart{}\textsc{\textcolor{pink}{IX Franckgasse 1}{}\ledrightnote{\textcolor{pink}{Frankgasse}}}\pend{}{\bigskip}\pstart
           \noindent{}{\pb}Ich werde ſo frei ſein,
                    heute abend als Mittel gegen Ihre \label{K_L00887_1v}\edtext{Zahnſchmerzen}{\lemma{\textnormal{\emph{Zahnſchmerzen}}}\Cendnote{\textnormal{vgl. A. S.: \emph{Tagebuch}, 3. 2. 1899}}}\label{K_L00887_1h}
                    und gegen den dämoniſchen \textcolor{blue}{Fulda}{}\ledrightnote{\textcolor{blue}{Ludwig Fulda}} den ſehr
                    luſtigen und angenehmen \textcolor{blue}{\textsc{Josi Schönborn}}{}\ledrightnote{\textcolor{blue}{Joseph von Schönborn}} mitzubringen; er wird entweder nach dem Nachtmahl oder (wenn er ſich
                    freimachen kann) ſchon um ½ 9 ko{\geminationm}en.\pend
           \pstart Ihr \spacefill\mbox{Hugo.}\pend{}\endnumbering\briefempfaengerindex{Schnitzler, Arthur@\textsc{Schnitzler, Arthur}!zzzHofmannsthal, Hugo von@\emph{von Hugo von Hofmannsthal}!1899-02-081@{8. 2. 1899}|)be}\mylabel{h}  \normalsize

\doendnotes{C}
\bigskip
\vfill

\clearpage

\footnotesize

\lohead{\textsc{register}}

% Definiere theindex-Environment komplett neu ohne reledmac
\makeatletter
\renewenvironment{theindex}{%
  \section*{\indexname}%
  \setlength{\parindent}{0pt}%
  \setlength{\parskip}{0pt plus 0.3pt}%
  \let\item\@idxitem
}{%
  \clearpage
}
\makeatother

\IfFileExists{\jobname-pw.ind}{\input{\jobname-pw.ind}}{}

\end{document}

      