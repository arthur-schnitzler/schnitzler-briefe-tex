%% latex-korrekturansicht-vorspann.tex
%% Vorspann für die Korrekturansicht.
%% Lädt die gemeinsame Datei latex-vorspann.tex mit gesetztem Schalter.

\newif\ifkorrekturansicht
\korrekturansichttrue

\input{../tex-inputs/latex-vorspann}


               \section[Hermann Bahr an Arthur Schnitzler, 18. 12. 1907]{ Hermann Bahr an Arthur Schnitzler, 18. 12. 1907}\nopagebreak\mylabel{v}\rehead{ }\normalsize\beginnumbering\briefempfaengerindex{Schnitzler, Arthur@\textsc{Schnitzler, Arthur}!zzzBahr, Hermann@\emph{von Hermann Bahr}!1907-12-181@{18. 12. 1907}|(be} \toendnotes[C]{\smallbreak\pagebreak[2]} \Standort{CUL, Schnitzler, B 5b.}
\physDesc{Brief, 1 Blatt, 2 Seiten
\newline{}Handschrift: blaue Tinte, deutsche Kurrent
\newline{}Schnitzler: mit Bleistift beschriftet: »Bahr« \newline{}Ordnung: mit Bleistift von unbekannter Hand
                           nummeriert: »152« }\buchAbdrucke{\weitereDrucke{Hermann Bahr, Arthur Schnitzler: \emph{Briefwechsel, Aufzeichnungen, Dokumente (1891–1931)}. Hg. Kurt Ifkovits und Martin Anton Müller. Göttingen: \emph{Wallstein} 2018, S. 399.} }\toendnotes[C]{\smallbreak}\pstart
           \raggedleft{}{\pb}18. 12. 07\pend
           \pstart\center{}Lieber Arthur!\pend\pstart
           Vertrauen gegen Vertrauen, da ich Dir doch nur helfe, wenn ich ganz rückhaltlos
               aufrichtig bin. Also: \textcolor{blue}{Reinhardt}{}\ledrightnote{\textcolor{blue}{Max Reinhardt}} würde, wenn man
               ihm ſagt, daß Du ſonſt mit \textcolor{blue}{Vallentin}{}\ledrightnote{\textcolor{blue}{Richard Vallentin}} abſchließen
               willſt, ſicher die \textcolor{green}{Beatrice}{}\ledrightnote{\textcolor{green}{Der Schleier der Beatrice. Schauspiel in fünf Akten}} annehmen, damit nur der
               andere ſie nicht habe, dann aber liegen laſſen, ſich mahnen laſſen, Dich verzweifeln
               laſſen, endlich, gedrängt, bedroht, ſie irgendwie, ohne ſich ſelbſt darum zu kümmern,
               von irgendwem ſchnell erledigen laſſen, weil er ſelbſt kein eigentliches Verhältnis
               zu dieſem Stücke hat, und weil es ſchließlich ſeine beſte Eigenschaft iſt, daß alle
               ſeine guten Eigenſchaft{[}en{]} verſagen, wo er {\pb}nicht durch ein ſtarkes inneres Verhältnis gehalten
               wird. Ich würde Dir alſo dringend zu \textcolor{blue}{Vallentin}{}\ledrightnote{\textcolor{blue}{Richard Vallentin}}
               raten und glaube, daß die \textcolor{blue}{Ritſcher}{}\ledrightnote{\textcolor{blue}{Helene Ritscher}}, wenn ſie im
               Sommer bei der \textcolor{blue}{Mildenburg}{}\ledrightnote{\textcolor{blue}{Anna Bahr-Mildenburg}} und gelegentlich auch
               mit mir die Rolle lernt, ſchon was recht Merkwürdiges machen könnte.\pend
           \pstart
           Ich weiß noch nicht, wann ich wieder nach \textcolor{pink}{Berlin}{}\ledrightnote{\textcolor{pink}{Berlin}}
               muß, möchte aber jedenfalls vorher zu Euch, ſo bald Deine \textcolor{blue}{Frau}{}\ledrightnote{→\textcolor{blue}{Olga Schnitzler}}{ }ſo weit iſt, über deren Erkrankung ich,
               ahnungslos, ſehr erschrack, weshalb ich mich ihrer Geneſung gern bald in der Nähe
               erfreuen möchte.\pend
           \pstart
           Herzlichſt{\\[\baselineskip]}Dein alter{\\[\baselineskip]}\spacefill\mbox{Hermann}\pend
           \leftskip=0em{}\pstart
           \noindent{}\label{T_L01742_1v}\edtext{\uline{Frage 1}: \textcolor{blue}{Reinhardt}{}\ledrightnote{\textcolor{blue}{Max Reinhardt}} wird \textcolor{green}{B.}{}\ledrightnote{\textcolor{green}{Der Schleier der Beatrice. Schauspiel in fünf Akten}} nehmen, wenn Du mit
                     \textcolor{blue}{Vallentin}{}\ledrightnote{\textcolor{blue}{Richard Vallentin}} drohſt. \uline{Frage 2}: Ich halte \textcolor{pink}{Hebbeltheater}{}\ledrightnote{\textcolor{pink}{Hebbel-Theater}} für
                  praktiſcher. \uline{Frage 3}: \textcolor{blue}{Reinhardt}{}\ledrightnote{\textcolor{blue}{Max Reinhardt}} müßte man eine Frist von 1\textcolor{gray}{4}
                  Tagen zur Entſcheidung geben.}{\lemma{\textnormal{\emph{Frage … geben.}}}\Cendnote{\textnormal{quer zum
                     Text neben der Grußformel}}}\label{T_L01742_1h}\pend
           \endnumbering\briefempfaengerindex{Schnitzler, Arthur@\textsc{Schnitzler, Arthur}!zzzBahr, Hermann@\emph{von Hermann Bahr}!1907-12-181@{18. 12. 1907}|)be}\mylabel{h}  \normalsize

\doendnotes{C}
\bigskip
\vfill

\clearpage

\footnotesize

\lohead{\textsc{register}}

% Definiere theindex-Environment komplett neu ohne reledmac
\makeatletter
\renewenvironment{theindex}{%
  \section*{\indexname}%
  \setlength{\parindent}{0pt}%
  \setlength{\parskip}{0pt plus 0.3pt}%
  \let\item\@idxitem
}{%
  \clearpage
}
\makeatother

\IfFileExists{\jobname-pw.ind}{\input{\jobname-pw.ind}}{}

\end{document}

      