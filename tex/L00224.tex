%% latex-korrekturansicht-vorspann.tex
%% Vorspann für die Korrekturansicht.
%% Lädt die gemeinsame Datei latex-vorspann.tex mit gesetztem Schalter.

\newif\ifkorrekturansicht
\korrekturansichttrue

\input{../tex-inputs/latex-vorspann}


               \section[Michael Georg Conrad an Arthur Schnitzler, 21. 6. 1893]{ Michael Georg Conrad an Arthur Schnitzler, 21. 6. 1893}\nopagebreak\mylabel{v}\rehead{ }\normalsize\beginnumbering\briefempfaengerindex{Schnitzler, Arthur@\textsc{Schnitzler, Arthur}!zzzConrad, Michael Georg@\emph{von Michael Georg Conrad}!1893-06-211@{21. 6. 1893}|(be} \toendnotes[C]{\smallbreak\pagebreak[2]} \Standort{TMW, HS Schn 1/83/1.}
\physDesc{Postkarte
\newline{}Handschrift: schwarze Tinte, deutsche Kurrent\newline{}Versand: 1) Stempel: »\nobreak{}\oindex{Muenchen@\textbf{München}, \emph{https://www.geonames.org/ontologyP.PPLA}|pwk}Muenchen L., 21. \textcolor{gray}{JU}N{[}I 1893{]}, 4–\textcolor{gray}{5 N}\nobreak{}«.  2) Stempel: »\nobreak{}Wien, 22 6 93, 9–10½V\nobreak{}«. \newline{}Ordnung: 1) mit Bleistift von unbekannter Hand
                                    nummeriert: »2« 2) mit rotem Buntstift von unbekannter Hand nummeriert: »3«}\toendnotes[C]{\smallbreak}\pstart{}{\pb}Herrn \textsc{D\textsuperscript{r}} Arthur Schnitzler\pend{}\pstart{}\textcolor{pink}{Wien I.}{}\ledrightnote{\textcolor{pink}{I., Innere Stadt}}\pend{}\pstart{}\textcolor{pink}{Grillparzerſtr. 7}{}\ledrightnote{\textcolor{pink}{Grillparzerstraße}}.\pend{}{\bigskip}\pstart
           {\pb}\textcolor{pink}{München}{}\ledrightnote{\textcolor{pink}{München}}{ }21. 6. 93.\pend
           \pstart
           Lieber Herr Doktor, eben von einer Wahlreiſe heimgekehrt, finde
                    ich Ihren werten Brief. Hier in Eile die Antwort: Ihre wunderſchönen \textcolor{green}{Gedichte}{}\ledrightnote{→\textcolor{green}{Der gute Irrtum}{\newline}→\textcolor{green}{Ohnmacht}} habe ich mit
                    beſten Empfehlungen an \textcolor{blue}{Hans Merian}{}\ledrightnote{\textcolor{blue}{Hans Merian}} zur
                    Aufnahme in die »\textcolor{green}{Geſellſch.}{}\ledrightnote{\textcolor{green}{Die Gesellschaft. Monatsschrift}}« übergeben. Ich
                    bin überzeugt, daß \uline{nur} redaktionell-techniſche
                    Gründe imſtande ſein können, den Abdruck ſo vortrefflicher Beiträge zu
                    verzögern. Mit Dank und Gruß\pend
           \pstart Ihr ergebener \spacefill\mbox{Conrad.}\pend{}\endnumbering\briefempfaengerindex{Schnitzler, Arthur@\textsc{Schnitzler, Arthur}!zzzConrad, Michael Georg@\emph{von Michael Georg Conrad}!1893-06-211@{21. 6. 1893}|)be}\mylabel{h}  \normalsize

\doendnotes{C}
\bigskip
\vfill

\clearpage

\footnotesize

\lohead{\textsc{register}}

% Definiere theindex-Environment komplett neu ohne reledmac
\makeatletter
\renewenvironment{theindex}{%
  \section*{\indexname}%
  \setlength{\parindent}{0pt}%
  \setlength{\parskip}{0pt plus 0.3pt}%
  \let\item\@idxitem
}{%
  \clearpage
}
\makeatother

\IfFileExists{\jobname-pw.ind}{\input{\jobname-pw.ind}}{}

\end{document}

      