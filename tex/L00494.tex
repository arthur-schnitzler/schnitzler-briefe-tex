%% latex-korrekturansicht-vorspann.tex
%% Vorspann für die Korrekturansicht.
%% Lädt die gemeinsame Datei latex-vorspann.tex mit gesetztem Schalter.

\newif\ifkorrekturansicht
\korrekturansichttrue

\input{../tex-inputs/latex-vorspann}


               \section[Arthur Schnitzler an Richard Beer-Hofmann, 26. 9. 1895]{ Arthur Schnitzler an Richard Beer-Hofmann, 26. 9. 1895}\nopagebreak\mylabel{v}\rehead{ }\normalsize\beginnumbering\briefempfaengerindex{Beer-Hofmann, Richard@\textsc{Beer-Hofmann, Richard}!zzzSchnitzler, Arthur@\emph{von Arthur Schnitzler}!1895-09-261@{26. 9. 1895}|(be} \toendnotes[C]{\smallbreak\pagebreak[2]} \Standort{YCGL, MSS 31.}
\physDesc{Brief, 2 Blätter, 7 Seiten, Umschlag
\newline{}Handschrift: 1) schwarze Tinte, deutsche Kurrent (\noindent{}Umschlag)\hspace{1em}2) Bleistift, deutsche Kurrent\hspace{1em}\newline{}Versand: 1) Stempel: »\nobreak{}Wien, 26. 9. 95, 7–8\nobreak{}«.  2) Stempel: »\nobreak{}\oindex{Gardone Riviera@\textbf{Gardone Riviera}, \emph{Besiedelter Ort (A.BSO)}|pwk}Gardone Riva, 28 9 95\nobreak{}«. 3) Stempel: »\nobreak{}\oindex{I., Innere Stadt@\textbf{I., Innere Stadt}, \emph{Bezirk (A.BZK)}|pwk}Wien 1/1, 1/10 95, 8–9½ V., Bestellt\nobreak{}«. 4) mit blauer Tinte von unbekannter Hand die Nachsendeadresse vermerkt: »\textcolor{pink}{I Wollzeile 15}. \textcolor{pink}{Wien I.}«}\buchAbdrucke{\weitereDrucke{Arthur Schnitzler, Richard Beer-Hofmann: \emph{Briefwechsel 1891–1931}. Hg. Konstanze Fliedl. Wien, Zürich: \emph{Europaverlag} 1992, S. 85–86.} }\toendnotes[C]{\smallbreak}\pstart{}{\pb}Herrn \textsc{Dr. Richard
                     Beer-Hofmann}\pend{}\pstart{}\textcolor{pink}{\textsc{Gardone}}{}\ledrightnote{\textcolor{pink}{Gardone Riviera}}\pend{}\pstart{}\textsc{am \textcolor{pink}{Gardasee}{}\ledrightnote{\textcolor{pink}{Lago di Garda}}}\pend{}\pstart{}\textcolor{pink}{\textsc{Italien}}{}\ledrightnote{\textcolor{pink}{Italien}}\pend{}{\bigskip}\pstart
           \raggedleft{}{\pb}\textcolor{pink}{Wien}{}\ledrightnote{\textcolor{pink}{Wien}}{ }26. 9. 95\pend
           \pstart
           Lieber Richard, heute kam zugleich Ihre Karte vom 23.
               und Ihr Brief vom 24. an. Ich ſende also dieſe Zeilen hier nach \textcolor{pink}{Gardone}{}\ledrightnote{\textcolor{pink}{Gardone Riviera}}; warum ſchreiben Sie nicht, wohin Sie von da
               aus gehen? Eben hat mir die \textcolor{blue}{Tragödin}{}\ledrightnote{→\textcolor{blue}{Adele Sandrock}} telephonirt, es war heut Probe von \textcolor{green}{Liebelei}{}\ledrightnote{\textcolor{green}{Liebelei. Schauspiel in drei Akten}} (ſtatt \textcolor{green}{Don \textsc{Carlos}}{}\ledrightnote{\textcolor{green}{Don Karlos, Infant von Spanien}}) von der ich nichts wußte, und ſie überbot ſich ſelbſt an Liebenswürdigkeiten
               für mich, mein \textcolor{green}{Stück}{}\ledrightnote{→\textcolor{green}{Liebelei. Schauspiel in drei Akten}} und ihre
               Rolle. {\pb}Sie hat heute auf der Probe einen
               »großartigen« Erfolg gehabt, und na, und ſo weiter. Ich denke, die \textsc{\textcolor{green}{Premiere}{}\ledrightnote{→\textcolor{green}{Liebelei. Schauspiel in drei Akten}}} wird am 7. oder 8. oder 9.{ }ſein. Dazu gibt man \textcolor{blue}{\textsc{Giacosa}}{}\ledrightnote{\textcolor{blue}{Giuseppe Giacosa}}, \textcolor{green}{Rechte der Seele}{}\ledrightnote{\textcolor{green}{Rechte der Seele}}. Für einen guten Sitz ſoll
               geſorgt sein. –\pend
           \pstart
           Allmälig hab ich zu arbeiten angefangen. Begonnen hab ich damit, daſs ich ein \textcolor{green}{Stück}{}\ledrightnote{→\textcolor{green}{Das Portrait}} (Einakter) in Verſen, {\pb}den ich vorigen Winter ſchrieb, in mein\introOben{}em\introOben{}{ }\substVorne{}\textsuperscript{\textcolor{gray}{Kästchen}}{\allowbreak}\substDazwischen{}Schreibtiſch\substHinten{} vergrub, – wo e\substVorne{}\textsuperscript{s}\substDazwischen{}r\substHinten{} am tiefſten iſt. Ich hab manchmal die ſtarke Empfindung, daſs mir nie mehr
               etwas gelingen wird – wie \textcolor{blue}{\textsc{Ibsen}}{}\ledrightnote{\textcolor{blue}{Henrik Ibsen}} und – \textcolor{blue}{\textsc{Paul Lindau}}{}\ledrightnote{\textcolor{blue}{Paul Lindau}}. –\pend
           \pstart
           Da die Läufigkeit der Frauen manchmal angenehm war, haben Sie wohl auch was »erlebt«
                  {\dots} wenigſtens {\pb}Anfänge.
               Da drin ſtecken ja die ganzen Erlebniſſe, die Schlüſſe ſind ja dieſelben. (\textcolor{green}{Anatol}{}\ledrightnote{→\textcolor{green}{Anatol}} reibt ſich die Augen. Er
                  ſchlu{\geminationm}ert ſofort wieder ein. Bald ſchläfſt du {\dots}{ }\textsc{etc}. ſiehe \textcolor{green}{\textsc{Hänsel u Grethel}}{}\ledrightnote{\textcolor{green}{Hänsel und Grethel}}) Ich beneide Sie ſo um die Natur. Es iſt ſo ſchön jetzt und ich möchte ganz wo
               anders ſein. Neulich war ich {\pb}in der \textcolor{pink}{Brühl}{}\ledrightnote{\textcolor{pink}{Brühl}}. \textcolor{blue}{Tini}{}\ledrightnote{\textcolor{blue}{Christine Schönberger}} iſt ſehr ſtolz
               geworden. Auch war ein Jägerlieutenant draußen. Dem \textcolor{blue}{Hugo}{}\ledrightnote{\textcolor{blue}{Hugo von Hofmannsthal}} hab ich Ihre Kränkung ausgerichtet, er iſt auch gekränkt. –\pend
           \pstart
           Wie weit iſt der \textcolor{green}{Liebling der Götter}{}\ledrightnote{\textcolor{green}{Der Tod Georgs}} und
               hoffentlich vieler Menschen? – \pend
           \pstart
           {\pb}Leben Sie wohl und ſchreiben Sie mir.
                  Samſtag werde ich wohl das Datum der \textcolor{green}{\textsc{Prém.}}{}\ledrightnote{→\textcolor{green}{Liebelei. Schauspiel in drei Akten}}{ }\textsc{def}\substVorne{}\textsuperscript{.}\substDazwischen{}\textsc{initiv}\substHinten{} kennen.\pend
           \pstart
           Man erkundigt ſich i{\geminationm}erfort und allſeitig nach Ihnen,
               was keine Broſamen, ſondern naive Wahrheiten {\pb}ſind.
               Warum ſoll ichs Ihnen denn verſchweigen? Dazu bin ich nicht 999gradig genug.\pend
           \pstart
           Herzlichen Gruſs, ich freu mich ſchon ſehr auf Sie.{\\[\baselineskip]}Ihr
                  \spacefill\mbox{Arthur.}\pend
           \leftskip=0em{}\endnumbering\briefempfaengerindex{Beer-Hofmann, Richard@\textsc{Beer-Hofmann, Richard}!zzzSchnitzler, Arthur@\emph{von Arthur Schnitzler}!1895-09-261@{26. 9. 1895}|)be}\mylabel{h}  \normalsize

\doendnotes{C}
\bigskip
\vfill

\clearpage

\footnotesize

\lohead{\textsc{register}}

% Definiere theindex-Environment komplett neu ohne reledmac
\makeatletter
\renewenvironment{theindex}{%
  \section*{\indexname}%
  \setlength{\parindent}{0pt}%
  \setlength{\parskip}{0pt plus 0.3pt}%
  \let\item\@idxitem
}{%
  \clearpage
}
\makeatother

\IfFileExists{\jobname-pw.ind}{\input{\jobname-pw.ind}}{}

\end{document}

      