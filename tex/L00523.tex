%% latex-korrekturansicht-vorspann.tex
%% Vorspann für die Korrekturansicht.
%% Lädt die gemeinsame Datei latex-vorspann.tex mit gesetztem Schalter.

\newif\ifkorrekturansicht
\korrekturansichttrue

\input{../tex-inputs/latex-vorspann}


               \section[Richard Beer-Hofmann an Arthur Schnitzler, {[}17. 12. 1895{]}]{ Richard Beer-Hofmann an Arthur Schnitzler,
               {[}17. 12. 1895{]}}\nopagebreak\mylabel{v}\rehead{ }\normalsize\beginnumbering\briefempfaengerindex{Schnitzler, Arthur@\textsc{Schnitzler, Arthur}!zzzBeer-Hofmann, Richard@\emph{von Richard Beer-Hofmann}!1895-12-171@{{[}17. 12. 1895{]}}|(be} \toendnotes[C]{\smallbreak\pagebreak[2]} \Standort{CUL, Schnitzler, B 8.}
\physDesc{Briefkarte
\newline{}Handschrift: schwarze Tinte, lateinische Kurrent
\newline{}Schnitzler: mit Bleistift datiert: »\substVorne{}\textsuperscript{18}\substDazwischen{}17\substHinten{}. 12. 95« \newline{}Ordnung: mit Bleistift von unbekannter Hand nummeriert: »72« }\buchAbdrucke{\weitereDrucke{Arthur Schnitzler, Richard Beer-Hofmann: \emph{Briefwechsel 1891–1931}. Hg. Konstanze Fliedl. Wien, Zürich: \emph{Europaverlag} 1992, S. 89.} }\pstart
           \noindent{}{\pb}Lieber Arthur Sie sind ja sicher morgen um halb eins im \textcolor{pink}{Griensteidel}{}\ledrightnote{\textcolor{pink}{Café Griensteidl}}? Wenn ich nicht \uline{punkt halbeins} dort bin, dann gehen Sie mit \textcolor{blue}{Halbe}{}\ledrightnote{\textcolor{blue}{Max Halbe}} zu \textcolor{blue}{Lou}{}\ledrightnote{\textcolor{blue}{Lou Andreas-Salomé}}. Ich ko{\geminationm}e dann gegen halbzwei ins \textcolor{pink}{Imperial}{}\ledrightnote{\textcolor{pink}{Hotel Imperial}}{ }{\pb}direkt.\pend
           \pstart
           Herzlichst{\\}Ihr{\\}\spacefill\mbox{R.}\pend
           \endnumbering\briefempfaengerindex{Schnitzler, Arthur@\textsc{Schnitzler, Arthur}!zzzBeer-Hofmann, Richard@\emph{von Richard Beer-Hofmann}!1895-12-171@{{[}17. 12. 1895{]}}|)be}\mylabel{h}  \normalsize

\doendnotes{C}
\bigskip
\vfill

\clearpage

\footnotesize

\lohead{\textsc{register}}

% Definiere theindex-Environment komplett neu ohne reledmac
\makeatletter
\renewenvironment{theindex}{%
  \section*{\indexname}%
  \setlength{\parindent}{0pt}%
  \setlength{\parskip}{0pt plus 0.3pt}%
  \let\item\@idxitem
}{%
  \clearpage
}
\makeatother

\IfFileExists{\jobname-pw.ind}{\input{\jobname-pw.ind}}{}

\end{document}

      