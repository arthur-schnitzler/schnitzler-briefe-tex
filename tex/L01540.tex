%% latex-korrekturansicht-vorspann.tex
%% Vorspann für die Korrekturansicht.
%% Lädt die gemeinsame Datei latex-vorspann.tex mit gesetztem Schalter.

\newif\ifkorrekturansicht
\korrekturansichttrue

\input{../tex-inputs/latex-vorspann}


               \section[Arthur Schnitzler an Hugo von Hofmannsthal, 7. 8. 1905]{ Arthur Schnitzler an Hugo von Hofmannsthal, 7. 8. 1905}\nopagebreak\mylabel{v}\rehead{ }\normalsize\beginnumbering\briefempfaengerindex{Hofmannsthal, Hugo von@\textsc{Hofmannsthal, Hugo von}!zzzSchnitzler, Arthur@\emph{von Arthur Schnitzler}!1905-08-072@{7. 8. 1905}|(be} \toendnotes[C]{\smallbreak\pagebreak[2]} \Standort{FDH, Hs-30885,121.}
\physDesc{Brief, 1 Blatt, 4 Seiten
\newline{}Handschrift: Bleistift, deutsche Kurrent}\buchAbdrucke{\weitereDrucke{Hugo von Hofmannsthal, Arthur Schnitzler: \emph{Briefwechsel}. Hg. Therese Nickl und Heinrich Schnitzler. Frankfurt am Main: \emph{S. Fischer} 1964, S. 212.} }\toendnotes[C]{\smallbreak}\pstart
           \raggedleft{}{\pb}7. 8. 90\substVorne{}\textsuperscript{1}\substDazwischen{}5\substHinten{}\pend
           \pstart
           lieber Hugo, erſtens hatte ich begreiflicherweiſe keine Ahnung, daſs
               Sie So{\geminationn}tag{ }ſchon \strikeout{fort} wieder
               fortfahren. Wieſo ich unſer Wiederſehen bis Freitag hinausſchob, werden Sie ſofort
               hören. Heute Montag müſſen wir, wie ſchon ein paar Tage vorherbeſti{\geminationm}t war, weil Hr \textcolor{blue}{Steinrück}{}\ledrightnote{\textcolor{blue}{Albert Steinrück}} gaſtiert, nach \textcolor{pink}{Mödling}{}\ledrightnote{\textcolor{pink}{Mödling}} –
                  Mittwoch wollten {\pb}wir, zu \textcolor{blue}{Heini}{}\ledrightnote{\textcolor{blue}{Heinrich Schnitzler}}’s 3. Geburtstag in den \textcolor{pink}{Prater}{}\ledrightnote{\textcolor{pink}{Prater}}. Um aber nicht allzuſehr aus dem Arbeiten heraus zu ko{\geminationm}en (we{\geminationn} man eben daran iſt
               was abzuſchließen, \textsc{enervirt} einen das ſehr wie Sie ja
               wiſſen) wollte ich zwiſchen den Reiſetagen immer einen Heimtag, und ſo fiel
               naturgemäſs der Freitag erſt auf Sie. {\pb}Nun
               haben Sie indeſs wohl meine Karte erhalten, die Sie für Mittwoch nach
                  \textcolor{pink}{Schönbrunn}{}\ledrightnote{\textcolor{pink}{Schloß Schönbrunn}} bittet (da ſich \textcolor{blue}{Heini}{}\ledrightnote{\textcolor{blue}{Heinrich Schnitzler}} vor die Wahl zwiſchen \label{K_L01540_1v}\edtext{Wurſtl}{\lemma{\textnormal{\emph{Wurſtl}}}\Cendnote{\textnormal{Puppentheater mit dem Hanswurst im \textcolor{pink}{Wurstelprater}}}}\label{K_L01540_1h} u \textsc{Menagerie} geſtellt für
               letztere entschied – u kaum hatte \textcolor{blue}{Heini}{}\ledrightnote{\textcolor{blue}{Heinrich Schnitzler}} das
               ausgeſprochen, ſo war mein erſter Gedanke »Hugo«) – und ich hoffe, auch ohne dieſe
               Karte {\pb}wiſſen Sie, daſs ich mich mindeſtens ebenſo ſehr
               freue \substVorne{}\textsuperscript{\textcolor{gray}{we{\geminationn}}}\substDazwischen{}Sie\substHinten{} wiederzuſehen als umgekehrt. Ich brauche Sie ſogar, abgeſehen von der
               Sehnsucht, Ende der Woche dringend, insbeſondere wegen des einen \textcolor{green}{Stücks}{}\ledrightnote{→\textcolor{green}{Zwischenspiel. Komödie in drei Akten}}. Ich habe Ihnen \textcolor{green}{zwei}{}\ledrightnote{→\textcolor{green}{Der Ruf des Lebens [Filmentwurf]}} vorzuleſen.\pend
           \pstart
           Nun, wir ſprechen hoffentlich ſchon Mittwoch über das Wie, Wo Wann.\pend
           \pstart
           Herzlichst Ihr{\\[\baselineskip]}\spacefill\mbox{A.}\pend
           \leftskip=0em{}\endnumbering\briefempfaengerindex{Hofmannsthal, Hugo von@\textsc{Hofmannsthal, Hugo von}!zzzSchnitzler, Arthur@\emph{von Arthur Schnitzler}!1905-08-072@{7. 8. 1905}|)be}\mylabel{h}  \normalsize

\doendnotes{C}
\bigskip
\vfill

\clearpage

\footnotesize

\lohead{\textsc{register}}

% Definiere theindex-Environment komplett neu ohne reledmac
\makeatletter
\renewenvironment{theindex}{%
  \section*{\indexname}%
  \setlength{\parindent}{0pt}%
  \setlength{\parskip}{0pt plus 0.3pt}%
  \let\item\@idxitem
}{%
  \clearpage
}
\makeatother

\IfFileExists{\jobname-pw.ind}{\input{\jobname-pw.ind}}{}

\end{document}

      