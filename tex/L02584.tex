%% latex-korrekturansicht-vorspann.tex
%% Vorspann für die Korrekturansicht.
%% Lädt die gemeinsame Datei latex-vorspann.tex mit gesetztem Schalter.

\newif\ifkorrekturansicht
\korrekturansichttrue

\input{../tex-inputs/latex-vorspann}


               \section[Arthur Schnitzler an Lotte Bloch-Zavřel, 11. 4. 1927]{ Arthur Schnitzler an Lotte Bloch-Zavřel, 11. 4. 1927}\nopagebreak\mylabel{v}\rehead{ }\normalsize\beginnumbering\briefempfaengerindex{Bloch-Zavřel, Lotte@\textsc{Bloch-Zavřel, Lotte}!zzzSchnitzler, Arthur@\emph{von Arthur Schnitzler}!1927-04-111@{11. 4. 1927}|(be} \toendnotes[C]{\smallbreak\pagebreak[2]} \Standort{DLA, A:Schnitzler, 85.1.405.}
\physDesc{Brief, 1 Blatt, 1 Seite, maschineller Durchschlag
\newline{}Handschrift: roter Buntstift, lateinische Kurrent (\noindent{}Beschriftungen »Bloch«, »Brief « und
                                       »\textcolor{pink}{Berlin}«)
\newline{}Bloch-Zavřel: mit rotem Buntstift drei Unterstreichungen }\pstart
           \raggedleft{}{\pb}11. 4. 1927.\pend
           \pstart{}Verehrte gnädige Frau.\pend\pstart
           Auf Ihre freundliche Anfrage erlaube ich mir zu erwiedern, dass ich gegen eine
               Veröffentlichung des unbeträchtlichen Briefes in Ihrem Sammelbändchen »\textcolor{green}{Briefe an Auguste Hauschner}{}\ledrightnote{\textcolor{green}{Briefe an Auguste Hauschner}}« nichts einzuwenden habe
               und bin mit vorzüglicher Hochachtung\pend
           \pstart Ihr ergebener\pend{}{\bigskip}\pstart
           \noindent{}Frau \textcolor{blue}{Lotte Bloch-Zavřel}{}\ledrightnote{\textcolor{blue}{Lotte Bloch-Zavřel}},\pend
           \pstart
           \textcolor{pink}{Charlottenburg}{}\ledrightnote{\textcolor{pink}{Charlottenburg}}.\pend
           \endnumbering\briefempfaengerindex{Bloch-Zavřel, Lotte@\textsc{Bloch-Zavřel, Lotte}!zzzSchnitzler, Arthur@\emph{von Arthur Schnitzler}!1927-04-111@{11. 4. 1927}|)be}\mylabel{h}  \normalsize

\doendnotes{C}
\bigskip
\vfill

\clearpage

\footnotesize

\lohead{\textsc{register}}

% Definiere theindex-Environment komplett neu ohne reledmac
\makeatletter
\renewenvironment{theindex}{%
  \section*{\indexname}%
  \setlength{\parindent}{0pt}%
  \setlength{\parskip}{0pt plus 0.3pt}%
  \let\item\@idxitem
}{%
  \clearpage
}
\makeatother

\IfFileExists{\jobname-pw.ind}{\input{\jobname-pw.ind}}{}

\end{document}

      