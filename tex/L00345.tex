%% latex-korrekturansicht-vorspann.tex
%% Vorspann für die Korrekturansicht.
%% Lädt die gemeinsame Datei latex-vorspann.tex mit gesetztem Schalter.

\newif\ifkorrekturansicht
\korrekturansichttrue

\input{../tex-inputs/latex-vorspann}


               \section[Karl Kraus an Arthur Schnitzler, 2. 7. 1894]{ Karl Kraus an Arthur Schnitzler, 2. 7. 1894}\nopagebreak\mylabel{v}\rehead{ }\normalsize\beginnumbering\briefempfaengerindex{Schnitzler, Arthur@\textsc{Schnitzler, Arthur}!zzzKraus, Karl@\emph{von Karl Kraus}!1894-07-023@{2. 7. 1894}|(be} \toendnotes[C]{\smallbreak\pagebreak[2]} \Standort{DLA, A:Schnitzler, HS.NZ85.1.3790, S. 18.}
\physDesc{maschinelle Abschrift}\buchAbdrucke{\weitereDrucke{\emph{Karl Kraus und Arthur Schnitzler. Eine Dokumentation.} Hg. Reinhard Urbach. In: \emph{Literatur und Kritik}, Bd. 49, Oktober 1970, S. 521.} }\pstart
           {\pb}\textcolor{pink}{Schliersee}{}\ledrightnote{\textcolor{pink}{Schliersee}},
                            2. 7. 1894.\pend
           \pstart
           Herzliche Grüsse aus \textcolor{pink}{Schliersee}{}\ledrightnote{\textcolor{pink}{Schliersee}} von
                    Ihrem \spacefill\mbox{K. K.}\pend
           \endnumbering\briefempfaengerindex{Schnitzler, Arthur@\textsc{Schnitzler, Arthur}!zzzKraus, Karl@\emph{von Karl Kraus}!1894-07-023@{2. 7. 1894}|)be}\mylabel{h}  \normalsize

\doendnotes{C}
\bigskip
\vfill

\clearpage

\footnotesize

\lohead{\textsc{register}}

% Definiere theindex-Environment komplett neu ohne reledmac
\makeatletter
\renewenvironment{theindex}{%
  \section*{\indexname}%
  \setlength{\parindent}{0pt}%
  \setlength{\parskip}{0pt plus 0.3pt}%
  \let\item\@idxitem
}{%
  \clearpage
}
\makeatother

\IfFileExists{\jobname-pw.ind}{\input{\jobname-pw.ind}}{}

\end{document}

      