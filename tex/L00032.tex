%% latex-korrekturansicht-vorspann.tex
%% Vorspann für die Korrekturansicht.
%% Lädt die gemeinsame Datei latex-vorspann.tex mit gesetztem Schalter.

\newif\ifkorrekturansicht
\korrekturansichttrue

\input{../tex-inputs/latex-vorspann}


               \section[Hugo von Hofmannsthal an Arthur Schnitzler, 12. 8. 1891]{ Hugo von Hofmannsthal an Arthur Schnitzler,
                    12. 8. 1891}\nopagebreak\mylabel{v}\rehead{ }\normalsize\beginnumbering\briefempfaengerindex{Schnitzler, Arthur@\textsc{Schnitzler, Arthur}!zzzHofmannsthal, Hugo von@\emph{von Hugo von Hofmannsthal}!1891-08-122@{12. 8. 1891}|(be} \toendnotes[C]{\smallbreak\pagebreak[2]} \Standort{CUL, Schnitzler, B 43.}
\physDesc{Kartenbrief
\newline{}Handschrift: schwarze Tinte, deutsche Kurrent\newline{}Versand: 1) Stempel: »\nobreak{}\oindex{Strobl@\textbf{Strobl}, \emph{Besiedelter Ort (A.BSO)}|pwk}Strobl, 12. 8. 91\nobreak{}«.  2) Stempel: »\nobreak{}\oindex{VI., Mariahilf@\textbf{VI., Mariahilf}, \emph{Bezirk (A.BZK)}|pwk}Wien VI 1, 13. 8. 91, 8–9½ V.\nobreak{}«. 
\newline{}Schnitzler: auf der Textseite zusätzlich mit Bleistift datiert: »12. 8 91« \newline{}Ordnung: mit Bleistift von unbekannter Hand nummeriert:
                                    »5« }\buchAbdrucke{\weitereDrucke{Hugo von Hofmannsthal, Arthur Schnitzler: \emph{Briefwechsel}. Hg. Therese Nickl und Heinrich Schnitzler. Frankfurt am Main: \emph{S. Fischer} 1964, S. 12.} }\toendnotes[C]{\smallbreak}\pstart{}{\pb}\textsc{D\textsuperscript{r} Arthur Schnitzler}\pend{}\pstart{}\textsc{\textcolor{pink}{Wien}{}\ledrightnote{\textcolor{pink}{Wien}}}\pend{}\pstart{}\textsc{\textcolor{pink}{I Kärthnerring 12}{}\ledrightnote{\textcolor{pink}{Kärntnerring}}}\pend{}{\bigskip}\pstart\center{}{\pb}Lieber Freund!\pend\pstart
           Infolge Feſtvorbereitungen für \label{K_L00032_1v}\edtext{\textcolor{blue}{Kaiſer}{}\ledrightnote{→\textcolor{blue}{Franz Joseph I. von Österreich-Ungarn}}beſuch}{\lemma{\textnormal{\emph{Kaiſerbeſuch}}}\Cendnote{\textnormal{Am 11. 8. 1891
                        besuchte Kaiser \textcolor{blue}{Franz Joseph I.}{ }\textcolor{pink}{Ischl}, um sich dort mit \textcolor{blue}{König Alexander von Serbien} zu treffen.}}}\label{K_L00032_1h} ganz
                    Comité, kurz blöd, mache ich Ihnen folgende Vorſchläge: Da \textcolor{pink}{Strobl}{}\ledrightnote{\textcolor{pink}{Strobl}} Paradies, \textcolor{pink}{Iſchl}{}\ledrightnote{\textcolor{pink}{Bad Ischl}}{ }Schweineſtall ſo erwarte ich sie und \textcolor{blue}{Hoffmann}{}\ledrightnote{\textcolor{blue}{Richard Beer-Hofmann}} an einem der beiden Tage \uline{beſtimmteſtens}.\pend
           \pstart
           Wenn das unmöglich, beſtimmen
                    Sie mir ein \textcolor{pink}{Iſchl}{}\ledrightnote{\textcolor{pink}{Bad Ischl}}er \textsc{rendezvous}. Sehen müſſen wir uns.\pend
           \pstart \spacefill\mbox{Loris.}\pend{}\endnumbering\briefempfaengerindex{Schnitzler, Arthur@\textsc{Schnitzler, Arthur}!zzzHofmannsthal, Hugo von@\emph{von Hugo von Hofmannsthal}!1891-08-122@{12. 8. 1891}|)be}\mylabel{h}  \normalsize

\doendnotes{C}
\bigskip
\vfill

\clearpage

\footnotesize

\lohead{\textsc{register}}

% Definiere theindex-Environment komplett neu ohne reledmac
\makeatletter
\renewenvironment{theindex}{%
  \section*{\indexname}%
  \setlength{\parindent}{0pt}%
  \setlength{\parskip}{0pt plus 0.3pt}%
  \let\item\@idxitem
}{%
  \clearpage
}
\makeatother

\IfFileExists{\jobname-pw.ind}{\input{\jobname-pw.ind}}{}

\end{document}

      