%% latex-korrekturansicht-vorspann.tex
%% Vorspann für die Korrekturansicht.
%% Lädt die gemeinsame Datei latex-vorspann.tex mit gesetztem Schalter.

\newif\ifkorrekturansicht
\korrekturansichttrue

\input{../tex-inputs/latex-vorspann}


               \section[Hermann Bahr an Arthur Schnitzler, 7. 3. 1914]{ Hermann Bahr an Arthur Schnitzler, 7. 3. 1914}\nopagebreak\mylabel{v}\rehead{ }\normalsize\beginnumbering\briefempfaengerindex{Schnitzler, Arthur@\textsc{Schnitzler, Arthur}!zzzBahr, Hermann@\emph{von Hermann Bahr}!1914-03-071@{7. 3. 1914}|(be} \toendnotes[C]{\smallbreak\pagebreak[2]} \Standort{CUL, Schnitzler, B 5b.}
\physDesc{Brief, 1 Blatt, 2 Seiten
\newline{}Handschrift: schwarze Tinte, deutsche Kurrent
\newline{}Schnitzler: 1) mit Bleistift ergänzt »Bahr« 2) mit rotem Buntstift vereinzelte Unterstreichungen\newline{}Ordnung: mit Bleistift von unbekannter Hand nummeriert:
                                    »179« }\buchAbdrucke{\weitereDrucke{Hermann Bahr, Arthur Schnitzler: \emph{Briefwechsel, Aufzeichnungen, Dokumente (1891–1931)}. Hg. Kurt Ifkovits und Martin Anton Müller. Göttingen: \emph{Wallstein} 2018, S. 492–493.} }\toendnotes[C]{\smallbreak}\pstart
           \raggedleft{}{\pb}\textcolor{pink}{Salzburg}{}\ledrightnote{\textcolor{pink}{Salzburg}}{ }7. 3. 14\pend
           \pstart\center{}Lieber Arthur!\pend\pstart
           Ich bin {[}in{]} der letzten Zeit ſo viel herumgegaukelt (\label{K_L02166_1v}\edtext{\textcolor{pink}{Czernowitz}{}\ledrightnote{\textcolor{pink}{Czernowitz}}}{\lemma{\textnormal{\emph{Czernowitz}}}\Cendnote{\textnormal{am 13. 1. 1914}}}\label{K_L02166_1h}, \label{K_L02166_2v}\edtext{\textcolor{pink}{Lemberg}{}\ledrightnote{\textcolor{pink}{Lviv}}}{\lemma{\textnormal{\emph{Lemberg}}}\Cendnote{\textnormal{bereits zuvor, am 12. 1. 1914}}}\label{K_L02166_2h}, \label{K_L02166_3v}\edtext{\textcolor{pink}{Brünn}{}\ledrightnote{\textcolor{pink}{Brünn}}}{\lemma{\textnormal{\emph{Brünn}}}\Cendnote{\textnormal{am 14. 1. 1914}}}\label{K_L02166_3h}, dann \label{K_L02166_4v}\edtext{\textcolor{pink}{Berchtesgaden}{}\ledrightnote{\textcolor{pink}{Berchtesgaden}}}{\lemma{\textnormal{\emph{Berchtesgaden}}}\Cendnote{\textnormal{vom 29. 1. bis zum 4. 2. 1914}}}\label{K_L02166_4h}{ }ſkiend, dann \textcolor{pink}{Münchener}{}\ledrightnote{\textcolor{pink}{München}}{ }\label{K_L02166_5v}\edtext{Suffragetten}{\lemma{\textnormal{\emph{Suffragetten}}}\Cendnote{\textnormal{am 19. 2. 1914 Vortrag über das
                  »Frauenstimmrecht« in \textcolor{pink}{München}}}}\label{K_L02166_5h}, dann
                  \label{K_L02166_6v}\edtext{\textcolor{pink}{Darmſtadt}{}\ledrightnote{\textcolor{pink}{Darmstadt}}}{\lemma{\textnormal{\emph{Darmſtadt}}}\Cendnote{\textnormal{vom 27. 2. bis zum 1. 3. 1914}}}\label{K_L02166_6h} bei Hofe – die Welt iſt ſehr rund), daß ich jetzt erſt dazu komme, Dir zu
               ſagen, wie furchtbar leid mir tat, \textcolor{blue}{Euren}{}\ledrightnote{→\textcolor{blue}{Olga Schnitzler}} lieben Beſuch verſäumt zu haben. So gern möcht ich \textcolor{blue}{Euch Beide}{}\ledrightnote{→\textcolor{blue}{Olga Schnitzler}} wieder einmal ſehen, ſo gern \textcolor{blue}{Euch}{}\ledrightnote{→\textcolor{blue}{Olga Schnitzler}} unſere Behauſung und den
               Park zeigen, ſo viel hätt ich Dich zu fragen, Dir zu ſagen! Hoffentlich {\pb}trifft ſichs das nächſte Mal beſſer. Aber wann wird
               dies nächſte Mal ſein? Wir gehen ja heuer schon zu Pfingſten nach
                  \label{K_L02166_7v}\edtext{\textcolor{pink}{Venedig}{}\ledrightnote{\textcolor{pink}{Venedig}}}{\lemma{\textnormal{\emph{Venedig}}}\Cendnote{\textnormal{vom 6. bis zum
                     25. 6. 1914v}}}\label{K_L02166_7h}, da wir Ende Juni{ }ſchon nach \label{K_L02166_8v}\edtext{\textcolor{pink}{Bayreuth}{}\ledrightnote{\textcolor{pink}{Bayreuth}}}{\lemma{\textnormal{\emph{Bayreuth}}}\Cendnote{\textnormal{vom 27. 7. bis zum
                     14. 8. 1914}}}\label{K_L02166_8h} müſſen, bis Ende Auguſt dort bleiben und uns alſo eigentlich
               jetzt ſchon auf den Herbſt hier freuen, bevor noch der Frühling da iſt.\pend
           \pstart
           Laſſt es Euch immer gut gehen, grüß auch die \textcolor{blue}{Kinder}{}\ledrightnote{→\textcolor{blue}{Heinrich Schnitzler}{\newline}→\textcolor{blue}{Lili Schnitzler}}, wenn ſie gleich nichts von mir wiſſen, herzlich von
               mir und bleibt mir gut, wie ich \textcolor{blue}{Euch}{}\ledrightnote{→\textcolor{blue}{Olga Schnitzler}} immer derſelbe bleiben will, eben dieſer Euer alter\pend
           \pstart \spacefill\mbox{Hermann}\pend{}\endnumbering\briefempfaengerindex{Schnitzler, Arthur@\textsc{Schnitzler, Arthur}!zzzBahr, Hermann@\emph{von Hermann Bahr}!1914-03-071@{7. 3. 1914}|)be}\mylabel{h}  \normalsize

\doendnotes{C}
\bigskip
\vfill

\clearpage

\footnotesize

\lohead{\textsc{register}}

% Definiere theindex-Environment komplett neu ohne reledmac
\makeatletter
\renewenvironment{theindex}{%
  \section*{\indexname}%
  \setlength{\parindent}{0pt}%
  \setlength{\parskip}{0pt plus 0.3pt}%
  \let\item\@idxitem
}{%
  \clearpage
}
\makeatother

\IfFileExists{\jobname-pw.ind}{\input{\jobname-pw.ind}}{}

\end{document}

      