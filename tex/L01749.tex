%% latex-korrekturansicht-vorspann.tex
%% Vorspann für die Korrekturansicht.
%% Lädt die gemeinsame Datei latex-vorspann.tex mit gesetztem Schalter.

\newif\ifkorrekturansicht
\korrekturansichttrue

\input{../tex-inputs/latex-vorspann}


               \section[Hugo von Hofmannsthal an Arthur Schnitzler, 8. 1. 190{[}8?{]}]{ Hugo von Hofmannsthal an Arthur Schnitzler, 8. 1. 190{[}8?{]}}\nopagebreak\mylabel{v}\rehead{ }\normalsize\beginnumbering\briefempfaengerindex{Schnitzler, Arthur@\textsc{Schnitzler, Arthur}!zzzHofmannsthal, Hugo von@\emph{von Hugo von Hofmannsthal}!1908-01-081@{8. 1. 190{[}8?{]}}|(be} \toendnotes[C]{\smallbreak\pagebreak[2]} \Standort{CUL, Schnitzler, B 43.}
\physDesc{Bildpostkarte
\newline{}Handschrift: schwarze Tinte, deutsche Kurrent\newline{}Versand: Stempel: »\nobreak{}\oindex{Sládkovicovo@\textbf{Sládkovičovo}, \emph{Besiedelter Ort (A.BSO)}|pwk}Diószeg, 9\textcolor{gray}{08} {[}JAN{]} 10\nobreak{}«.  
\newline{}Schnitzler: mit Bleistift die Jahreszahl ergänzt: »08« und mit »\textsc{Ho}« beschriftet \newline{}Ordnung: 1) mit Bleistift von unbekannter Hand nummeriert: »\strikeout{280}« 2) mit Bleistift von unbekannter Hand nummeriert: »291«}\buchAbdrucke{\weitereDrucke{Hugo von Hofmannsthal, Arthur Schnitzler: \emph{Briefwechsel}. Hg. Therese Nickl und Heinrich Schnitzler. Frankfurt am Main: \emph{S. Fischer} 1964, S. 235.} }\toendnotes[C]{\smallbreak}\pstart{}{\pb}\textsc{Herrn}\pend{}\pstart{}\textsc{D\textsuperscript{r} Arthur Schnitzler}\pend{}\pstart{}\textcolor{pink}{Wien}{}\ledrightnote{\textcolor{pink}{Wien}}\pend{}\pstart{}\textcolor{pink}{XVIII Spöttelgasse 7}{}\ledrightnote{\textcolor{pink}{Edmund-Weiß-Gasse}}.\pend{}{\bigskip}\pstart
           \noindent{}\centering{}\textcolor{gray}{\textbf{{\pb}\label{K_L01749_1v}\edtext{Ädvözlet
                     \textcolor{pink}{Diószegről}{}\ledrightnote{\textcolor{pink}{Sládkovičovo}}}{\lemma{\textnormal{\emph{Ädvözlet
                     Diószegről}}}\Cendnote{\textnormal{ungarisch:
                        Schöne Grüße aus \textcolor{pink}{Diószeg} (heute \textcolor{pink}{Sládkovičovo}).}}}\label{K_L01749_1h}.}}\pend
           \pstart
           \noindent{}\centering{}\textcolor{gray}{\textbf{\textcolor{pink}{\label{K_L01749_2v}\edtext{Kastély}{\lemma{\textnormal{\emph{Kastély}}}\Cendnote{\textnormal{Im \textcolor{pink}{Kuffner-Schloss} lebten Verwandte \textcolor{blue}{Hofmannsthal}s mütterlicherseits.}}}\label{K_L01749_2h}}{}\ledrightnote{\textcolor{pink}{Kaštiel Kuffnerovcov}}.}}\pend
           \pstart
           {\pb}8 I\pend
           \pstart
           lieber, ich Scheuſal denke ja oft an Sie u. ſchreibe Ihnen nie! Wird
               man nicht jetzt bald miteinander ſpazierengehen können?\pend
           \pstart
           Alles Liebe \textcolor{blue}{Olga}{}\ledrightnote{\textcolor{blue}{Olga Schnitzler}} von uns \textcolor{blue}{beiden}{}\ledrightnote{→\textcolor{blue}{Gertrude von Hofmannsthal}}.\pend
           \pstart Ihr \spacefill\mbox{Hugo.}\pend{}\endnumbering\briefempfaengerindex{Schnitzler, Arthur@\textsc{Schnitzler, Arthur}!zzzHofmannsthal, Hugo von@\emph{von Hugo von Hofmannsthal}!1908-01-081@{8. 1. 190{[}8?{]}}|)be}\mylabel{h}  \normalsize

\doendnotes{C}
\bigskip
\vfill

\clearpage

\footnotesize

\lohead{\textsc{register}}

% Definiere theindex-Environment komplett neu ohne reledmac
\makeatletter
\renewenvironment{theindex}{%
  \section*{\indexname}%
  \setlength{\parindent}{0pt}%
  \setlength{\parskip}{0pt plus 0.3pt}%
  \let\item\@idxitem
}{%
  \clearpage
}
\makeatother

\IfFileExists{\jobname-pw.ind}{\input{\jobname-pw.ind}}{}

\end{document}

      