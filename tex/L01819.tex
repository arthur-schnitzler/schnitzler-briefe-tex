%% latex-korrekturansicht-vorspann.tex
%% Vorspann für die Korrekturansicht.
%% Lädt die gemeinsame Datei latex-vorspann.tex mit gesetztem Schalter.

\newif\ifkorrekturansicht
\korrekturansichttrue

\input{../tex-inputs/latex-vorspann}


               \section[Olga Schnitzler an Richard Beer-Hofmann, {[}27.? 12. 1908{]}]{ Olga Schnitzler an Richard Beer-Hofmann, {[}27.? 12. 1908{]}}\nopagebreak\mylabel{v}\rehead{ }\normalsize\beginnumbering\briefempfaengerindex{Beer-Hofmann, Richard@\textsc{Beer-Hofmann, Richard}!zzzSchnitzler, Olga@\emph{von Olga Schnitzler}!1908-12-272@{{[}27.? 12. 1908{]}}|(be} \toendnotes[C]{\smallbreak\pagebreak[2]} \Standort{YCGL, MSS 31.}
\physDesc{Brief, 1 Blatt, 2 Seiten, Umschlag
\newline{}Handschrift: schwarze Tinte, lateinische Kurrent\newline{}Versand: ohne postalischen Übermittlungsvermerk }\toendnotes[C]{\smallbreak}\pstart{}{\pb}\textcolor{gray}{\textbf{O. S.}}\pend{}{\bigskip}\pstart{}{\pb}Herrn D\textsuperscript{r} Richard
                  Beer-Hofmann\pend{}\pstart{}\textcolor{pink}{Wien XVIII}{}\ledrightnote{\textcolor{pink}{XVIII., Währing}}\pend{}\pstart{}\textcolor{pink}{Hasenauerstr. 59}{}\ledrightnote{\textcolor{pink}{Hasenauerstraße}}.\pend{}{\bigskip}\pstart
           \noindent{}{\pb}\textcolor{gray}{\textbf{O. S.}}\pend
           \pstart
           Lieber Herr Doctor, unser gewohnter \label{K_L01819_1v}\edtext{Sylvester-Familienabend}{\lemma{\textnormal{\emph{Sylvester-Familienabend}}}\Cendnote{\textnormal{siehe A. S.: \emph{Tagebuch}, 31. 12. 1908}}}\label{K_L01819_1h} bei \textcolor{blue}{Mama}{}\ledrightnote{→\textcolor{blue}{Louise Schnitzler}} findet diesmal
               nicht statt, weil \textcolor{blue}{Mama}{}\ledrightnote{→\textcolor{blue}{Louise Schnitzler}} verkühlt
               ist und nicht aufbleiben darf, wir wollen also diesmal den Abend bei uns feiern und
               würden uns sehr sehr freuen, wenn Sie und Frau \textcolor{blue}{Paula}{}\ledrightnote{\textcolor{blue}{Paula Beer-Hofmann}}
               kommen wollten.\hspace*{1.5em}Ich denke noch {\pb}D\textsuperscript{r}{ }\textcolor{blue}{Kaufmann}{}\ledrightnote{\textcolor{blue}{Arthur Kaufmann}}, \textcolor{blue}{Van-Jung}{}\ledrightnote{\textcolor{blue}{Leo Van-Jung}}, \textcolor{blue}{Wassermanns}{}\ledrightnote{\textcolor{blue}{Jakob Wassermann}{\newline}\textcolor{blue}{Julie Wassermann}} und die \textcolor{blue}{Agnes}{}\ledrightnote{\textcolor{blue}{Agnes Ulmann}} zu laden.\pend
           \pstart
           Bitte lassen Sie mir ein bejahendes Wort sagen.\pend
           \pstart
           Mit herzlichen Grüssen an Sie \textcolor{blue}{Beide}{}\ledrightnote{→\textcolor{blue}{Paula Beer-Hofmann}} und die \textcolor{blue}{Kinder}{}\ledrightnote{→\textcolor{blue}{Naëmah Beer-Hofmann}{\newline}→\textcolor{blue}{Mirjam Beer-Hofmann}{\newline}→\textcolor{blue}{Gabriel Beer-Hofmann}}\pend
           \pstart \spacefill\mbox{Olga Schnitzler.}\pend{}\pstart
           \noindent{}Der \label{K_L01819_2v}\edtext{Abend mit \textcolor{blue}{T.}{}\ledrightnote{\textcolor{blue}{Siegfried Trebitsch}}}{\lemma{\textnormal{\emph{Abend mit T.}}}\Cendnote{\textnormal{vgl. A. S.: \emph{Tagebuch}, 26. 12. 1908}}}\label{K_L01819_2h} war gar nicht so arg.\pend
           \endnumbering\briefempfaengerindex{Beer-Hofmann, Richard@\textsc{Beer-Hofmann, Richard}!zzzSchnitzler, Olga@\emph{von Olga Schnitzler}!1908-12-272@{{[}27.? 12. 1908{]}}|)be}\mylabel{h}  \normalsize

\doendnotes{C}
\bigskip
\vfill

\clearpage

\footnotesize

\lohead{\textsc{register}}

% Definiere theindex-Environment komplett neu ohne reledmac
\makeatletter
\renewenvironment{theindex}{%
  \section*{\indexname}%
  \setlength{\parindent}{0pt}%
  \setlength{\parskip}{0pt plus 0.3pt}%
  \let\item\@idxitem
}{%
  \clearpage
}
\makeatother

\IfFileExists{\jobname-pw.ind}{\input{\jobname-pw.ind}}{}

\end{document}

      