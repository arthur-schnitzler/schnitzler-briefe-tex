%% latex-korrekturansicht-vorspann.tex
%% Vorspann für die Korrekturansicht.
%% Lädt die gemeinsame Datei latex-vorspann.tex mit gesetztem Schalter.

\newif\ifkorrekturansicht
\korrekturansichttrue

\input{../tex-inputs/latex-vorspann}


               \section[Arthur Schnitzler an Richard Beer-Hofmann, {[}27. 12. 1892?{]}]{ Arthur Schnitzler an Richard Beer-Hofmann, {[}27. 12. 1892?{]}}\nopagebreak\mylabel{v}\rehead{ }\normalsize\beginnumbering\briefempfaengerindex{Beer-Hofmann, Richard@\textsc{Beer-Hofmann, Richard}!zzzSchnitzler, Arthur@\emph{von Arthur Schnitzler}!1892-12-272@{{[}27. 12. 1892?{]}}|(be} \toendnotes[C]{\smallbreak\pagebreak[2]} \Standort{YCGL, MSS 31.}
\physDesc{Brief, 1 Blatt, 2 Seiten
\newline{}Handschrift: Bleistift, deutsche Kurrent}\toendnotes[C]{\smallbreak}\pstart{}{\pb}Mein lieber Richard, \pend\pstart
           ich muſs Ihnen dieſe \label{K_L00147_1v}\edtext{Karte}{\lemma{\textnormal{\emph{Karte}}}\Cendnote{\textnormal{nicht überlieferte
                  Beilage. Im Folgenden wird klar, dass es sich um eine \textcolor{blue}{Bertha Flegmann} betreffende Sache handelt. Es könnte von                  einer Eintrittskarte für die Liebhaberaufführung der \emph{\textcolor{green}{Aspasia}} in ihrem Salon Ende 1892 die Rede sein, was auch \textcolor{blue}{Schnitzler}s unwilligen Ton in Einklang mit
                  seinen anderen Äußerungen zur Sache brächte.}}}\label{K_L00147_1h} ſchicken. Wenn Sie
                  lie\textcolor{gray}{b}ens\textcolor{gray}{würd} ſind, antworten Sie mir.\pend
           \pstart
           {\pb}herzlich{\\[\baselineskip]}Ihr{\\[\baselineskip]}\spacefill\mbox{Arth}\pend
           \leftskip=0em{}\pstart
           \noindent{}Aber ſo, dſs ich Ihren Brief der Frau \textcolor{blue}{F.}{}\ledrightnote{\textcolor{blue}{Bertha Flegmann}}
                  zeigen kann.\pend
           \endnumbering\briefempfaengerindex{Beer-Hofmann, Richard@\textsc{Beer-Hofmann, Richard}!zzzSchnitzler, Arthur@\emph{von Arthur Schnitzler}!1892-12-272@{{[}27. 12. 1892?{]}}|)be}\mylabel{h}  \normalsize

\doendnotes{C}
\bigskip
\vfill

\clearpage

\footnotesize

\lohead{\textsc{register}}

% Definiere theindex-Environment komplett neu ohne reledmac
\makeatletter
\renewenvironment{theindex}{%
  \section*{\indexname}%
  \setlength{\parindent}{0pt}%
  \setlength{\parskip}{0pt plus 0.3pt}%
  \let\item\@idxitem
}{%
  \clearpage
}
\makeatother

\IfFileExists{\jobname-pw.ind}{\input{\jobname-pw.ind}}{}

\end{document}

      