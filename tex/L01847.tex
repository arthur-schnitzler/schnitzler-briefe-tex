%% latex-korrekturansicht-vorspann.tex
%% Vorspann für die Korrekturansicht.
%% Lädt die gemeinsame Datei latex-vorspann.tex mit gesetztem Schalter.

\newif\ifkorrekturansicht
\korrekturansichttrue

\input{../tex-inputs/latex-vorspann}


               \section[Max Burckhard: Widmungsexemplar Gottfried Wunderlich für Arthur Schnitzler, 20. 6. 1909]{ Max Burckhard: Widmungsexemplar Gottfried Wunderlich für Arthur Schnitzler,
                    20. 6. 1909}\nopagebreak\mylabel{v}\rehead{ }\normalsize\beginnumbering\briefempfaengerindex{Schnitzler, Arthur@\textsc{Schnitzler, Arthur}!zzzBurckhard, Max Eugen@\emph{von Max Eugen Burckhard}!1909-06-201@{20. 6. 1909}|(be} \toendnotes[C]{\smallbreak\pagebreak[2]} \Standort{DLA, G:Schnitzler, Arthur (Sammlung Heinrich Schnitzler).}
\physDesc{Widmung am Vorsatzblatt
\newline{}Handschrift: schwarze Tinte, deutsche Kurrent\newline{}Ordnung: bei der Enteignung des Exemplars 1938 von unbekannter Hand mit
                                    Bleistift ergänzte Information: »Widm.« und zwei
                                    Stempel: \noindent{}\textcolor{gray}{\textbf{\textit{\textcolor{brown}{NATIONAL-BIBLIOTHEK}{ }\textcolor{pink}{WIEN}}}}{ / }\textcolor{gray}{\textbf{\textit{682782-B}}}« }\pstart
           \noindent{}{\pb}Arthur Schnitzler{\\}überreicht in
                    neuem Gewande –\pend
           \pstart
           in herzlicher Verehrung{\\[\baselineskip]}\spacefill\mbox{D\textsuperscript{r} B.}\pend
           \leftskip=0em{}\pstart
           20. 6. 09\pend
           {\bigskip}\pstart
           \noindent{}\centering{}{\pb}\textcolor{gray}{\textbf{\textcolor{green}{Gottfried Wunderlich}{}\ledrightnote{\textcolor{green}{Gottfried Wunderlich. Roman}}}}\pend
           \pstart
           \noindent{}\centering{}\textcolor{gray}{\textbf{Roman}}{\\}\textcolor{gray}{\textbf{von}}{\\}\textcolor{gray}{\textbf{Max Burckhard}}\pend
           \pstart
           \noindent{}\centering{}\textcolor{gray}{\textbf{Dritte Auflage}}\pend
           {\bigskip}\pstart
           \noindent{}\centering{}\textcolor{gray}{\textbf{\textcolor{brown}{S. Fiſcher, Verlag}{}\ledrightnote{\textcolor{brown}{S. Fischer Verlag}}, \textcolor{pink}{Berlin}{}\ledrightnote{\textcolor{pink}{Berlin}}}}\pend
           \pstart
           \noindent{}\centering{}\textcolor{gray}{\textbf{1909}}\pend
           \endnumbering\briefempfaengerindex{Schnitzler, Arthur@\textsc{Schnitzler, Arthur}!zzzBurckhard, Max Eugen@\emph{von Max Eugen Burckhard}!1909-06-201@{20. 6. 1909}|)be}\mylabel{h}  \normalsize

\doendnotes{C}
\bigskip
\vfill

\clearpage

\footnotesize

\lohead{\textsc{register}}

% Definiere theindex-Environment komplett neu ohne reledmac
\makeatletter
\renewenvironment{theindex}{%
  \section*{\indexname}%
  \setlength{\parindent}{0pt}%
  \setlength{\parskip}{0pt plus 0.3pt}%
  \let\item\@idxitem
}{%
  \clearpage
}
\makeatother

\IfFileExists{\jobname-pw.ind}{\input{\jobname-pw.ind}}{}

\end{document}

      