%% latex-korrekturansicht-vorspann.tex
%% Vorspann für die Korrekturansicht.
%% Lädt die gemeinsame Datei latex-vorspann.tex mit gesetztem Schalter.

\newif\ifkorrekturansicht
\korrekturansichttrue

\input{../tex-inputs/latex-vorspann}


               \section[Hugo von Hofmannsthal an Arthur Schnitzler, {[}13. 8. 1899{]}]{ Hugo von Hofmannsthal an Arthur Schnitzler, {[}13. 8. 1899{]}}\nopagebreak\mylabel{v}\rehead{ }\normalsize\beginnumbering\briefempfaengerindex{Schnitzler, Arthur@\textsc{Schnitzler, Arthur}!zzzHofmannsthal, Hugo von@\emph{von Hugo von Hofmannsthal}!1899-08-131@{{[}13. 8. 1899{]}}|(be} \toendnotes[C]{\smallbreak\pagebreak[2]} \Standort{CUL, Schnitzler, B 43.}
\physDesc{Brief, 1 Blatt, 2 Seiten
\newline{}Handschrift: schwarze Tinte, deutsche Kurrent
\newline{}Schnitzler: mit Bleistift datiert: »13/8 99« \newline{}Ordnung: mit Bleistift von unbekannter Hand nummeriert:
                              »155« }\buchAbdrucke{\weitereDrucke{Hugo von Hofmannsthal, Arthur Schnitzler: \emph{Briefwechsel}. Hg. Therese Nickl und Heinrich Schnitzler. Frankfurt am Main: \emph{S. Fischer} 1964, S. 129.} }\pstart
           \centering{}{\pb}Sonntg\pend
           \pstart
           Fand Euer geſtriges Telegra{\geminationm} erſt abends, konnte erſt
               heute \textcolor{pink}{Bozen}{}\ledrightnote{\textcolor{pink}{Bozen}} telegrafieren, erhielt dann Eure
               zweite Depeſche. Möchte Mittwoch erſter Zug \textcolor{pink}{Iſchl}{}\ledrightnote{\textcolor{pink}{Bad Ischl}}
                  anko{\geminationm}en, Tag mit Ihnen verbringen, Rad mitnehmen,
               abends \textcolor{pink}{Auſſee}{}\ledrightnote{\textcolor{pink}{Bad Aussee}} zurück, da ja \textcolor{blue}{Richard}{}\ledrightnote{\textcolor{blue}{Richard Beer-Hofmann}}{ }{\pb}Do{\geminationn}erstag \textcolor{pink}{Auſſee}{}\ledrightnote{\textcolor{pink}{Bad Aussee}} ko{\geminationm}t.\pend
           \pstart
           Vielleicht fahren wir zuſa{\geminationm}en \textcolor{pink}{Hallſtadt}{}\ledrightnote{\textcolor{pink}{Hallstatt}}? oder Sie ko{\geminationm}en ſchon
               Mittwoch \textcolor{pink}{Auſſee}{}\ledrightnote{\textcolor{pink}{Bad Aussee}}? Aber was bei ſchlechtem Wetter?\pend
           \pstart
           Herzlich{\\[\baselineskip]}\spacefill\mbox{Hugo.}\pend
           \leftskip=0em{}\endnumbering\briefempfaengerindex{Schnitzler, Arthur@\textsc{Schnitzler, Arthur}!zzzHofmannsthal, Hugo von@\emph{von Hugo von Hofmannsthal}!1899-08-131@{{[}13. 8. 1899{]}}|)be}\mylabel{h}  \normalsize

\doendnotes{C}
\bigskip
\vfill

\clearpage

\footnotesize

\lohead{\textsc{register}}

% Definiere theindex-Environment komplett neu ohne reledmac
\makeatletter
\renewenvironment{theindex}{%
  \section*{\indexname}%
  \setlength{\parindent}{0pt}%
  \setlength{\parskip}{0pt plus 0.3pt}%
  \let\item\@idxitem
}{%
  \clearpage
}
\makeatother

\IfFileExists{\jobname-pw.ind}{\input{\jobname-pw.ind}}{}

\end{document}

      