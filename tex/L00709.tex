%% latex-korrekturansicht-vorspann.tex
%% Vorspann für die Korrekturansicht.
%% Lädt die gemeinsame Datei latex-vorspann.tex mit gesetztem Schalter.

\newif\ifkorrekturansicht
\korrekturansichttrue

\input{../tex-inputs/latex-vorspann}


               \section[Arthur Schnitzler an Hugo von Hofmannsthal, 21. 7. 1897]{ Arthur Schnitzler an Hugo von Hofmannsthal, 21. 7. 1897}\nopagebreak\mylabel{v}\rehead{ }\normalsize\beginnumbering\briefempfaengerindex{Hofmannsthal, Hugo von@\textsc{Hofmannsthal, Hugo von}!zzzSchnitzler, Arthur@\emph{von Arthur Schnitzler}!1897-07-211@{21. 7. 1897}|(be} \toendnotes[C]{\smallbreak\pagebreak[2]} \Standort{FDH, Hs-30885,62.}
\physDesc{Brief, 1 Blatt, 4 Seiten
\newline{}Handschrift: Bleistift, deutsche Kurrent}\buchAbdrucke{\weitereDrucke{Hugo von Hofmannsthal, Arthur Schnitzler: \emph{Briefwechsel}. Hg. Therese Nickl und Heinrich Schnitzler. Frankfurt am Main: \emph{S. Fischer} 1964, S. 94.} }\toendnotes[C]{\smallbreak}\pstart
           \raggedleft{}{\pb}21/7\pend
           \pstart{} Mein lieber Hugo, \pend\pstart
           daſs wir uns erſt im Herbſt ſehn werden, iſt mir ſehr leid. – Laſſen Sie nur von
                    ſich hören; auch zeigen Sie mir an, wohin ich Ihnen die 2 letzten \textcolor{green}{\textcolor{blue}{\textsc{Mozart}}{}\ledrightnote{\textcolor{blue}{Wolfgang Amadeus Mozart}}bände}{}\ledrightnote{→\textcolor{green}{W. A. Mozart}}{ }ſchicken ſoll.\pend
           \pstart
           \textcolor{blue}{Richard}{}\ledrightnote{\textcolor{blue}{Richard Beer-Hofmann}} iſt nun zu einer wirklichen
                    Radpartie nicht zu bewegen; {\pb}ich aber fahre, we{\geminationn} das Wetter gut iſt, Freitag (mit
                    einem kleinen \textcolor{blue}{Schwager}{}\ledrightnote{→\textcolor{blue}{Carl Reinhard}{\newline}→\textcolor{blue}{Franz Reinhard}}) nach \textcolor{pink}{Salzburg}{}\ledrightnote{\textcolor{pink}{Salzburg}}.
                        Samſtag: \textsc{\textcolor{pink}{Salzb.}{}\ledrightnote{\textcolor{pink}{Salzburg}} – \textcolor{pink}{Berchtesgaden}{}\ledrightnote{\textcolor{pink}{Berchtesgaden}} – \textcolor{pink}{Ramsau}{}\ledrightnote{\textcolor{pink}{Ramsau bei Berchtesgaden}} – \textcolor{pink}{Zell am See}{}\ledrightnote{\textcolor{pink}{Zell am See}}}. So{\geminationn}tag – an der Bahn, ſo weit
                    ich komme, um Mittgs einzuſteigen und am Abend in \textcolor{pink}{Wien}{}\ledrightnote{\textcolor{pink}{Wien}} einzutreffen. –\pend
           \pstart
           {\pb}Neulich war ich in \textcolor{pink}{\textsc{Aussee}}{}\ledrightnote{\textcolor{pink}{Bad Aussee}} bei den \textcolor{blue}{\textsc{Loebs}}{}\ledrightnote{\textcolor{blue}{Louis Loeb}{\newline}\textcolor{blue}{Regina Loeb}}; geſtern waren ſie in \textcolor{pink}{\textsc{Ischl}}{}\ledrightnote{\textcolor{pink}{Bad Ischl}}. \textcolor{blue}{\textsc{Clara}}{}\ledrightnote{\textcolor{blue}{Clara Katharina Pollaczek}} fühlt ſich ſehr verlaſſen von Ihnen. Sie hat es anders ausgedrückt; aber
                    das iſt der Sinn. –\pend
           \pstart
           Sie wiſſen wohl, dſs \textcolor{blue}{\textsc{Burckhard}}{}\ledrightnote{\textcolor{blue}{Max Eugen Burckhard}} die \textcolor{green}{\textsc{Jordan}}{}\ledrightnote{\textcolor{green}{Agnes Jordan. Schauspiel in fünf Akten}} nicht aufführt? – Ich ärgere mich
                    ſehr; umſomehr als ich zu ahnen glau{\pb}be, wo \uline{die} Gründe liegen und \textcolor{blue}{wer}{}\ledrightnote{→\textcolor{blue}{Hermann Bahr}} eigentlich {\dots}{ }ſagen wir »mit«ſchuldig iſt. –\pend
           \pstart
           – Sie ſchreiben mir bald nach \textcolor{pink}{Wien}{}\ledrightnote{\textcolor{pink}{Wien}}, nicht wahr? \pend
           \pstart Ihr \spacefill\mbox{Arthur.}\pend{}\pstart
           \textsc{\textcolor{pink}{Ischl}{}\ledrightnote{\textcolor{pink}{Bad Ischl}}}, 21/7 97.\pend
           \pstart
           Grüßen Sie \textcolor{blue}{P. A.}{}\ledrightnote{\textcolor{blue}{Peter Altenberg}}, we{\geminationn} er ſchon bei Ihnen iſt.\pend
           \endnumbering\briefempfaengerindex{Hofmannsthal, Hugo von@\textsc{Hofmannsthal, Hugo von}!zzzSchnitzler, Arthur@\emph{von Arthur Schnitzler}!1897-07-211@{21. 7. 1897}|)be}\mylabel{h}  \normalsize

\doendnotes{C}
\bigskip
\vfill

\clearpage

\footnotesize

\lohead{\textsc{register}}

% Definiere theindex-Environment komplett neu ohne reledmac
\makeatletter
\renewenvironment{theindex}{%
  \section*{\indexname}%
  \setlength{\parindent}{0pt}%
  \setlength{\parskip}{0pt plus 0.3pt}%
  \let\item\@idxitem
}{%
  \clearpage
}
\makeatother

\IfFileExists{\jobname-pw.ind}{\input{\jobname-pw.ind}}{}

\end{document}

      