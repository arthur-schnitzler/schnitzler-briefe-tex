%% latex-korrekturansicht-vorspann.tex
%% Vorspann für die Korrekturansicht.
%% Lädt die gemeinsame Datei latex-vorspann.tex mit gesetztem Schalter.

\newif\ifkorrekturansicht
\korrekturansichttrue

\input{../tex-inputs/latex-vorspann}


               \section[Arthur Schnitzler an Richard Beer-Hofmann, 16. 9. 1908]{ Arthur Schnitzler an Richard Beer-Hofmann, 16. 9. 1908}\nopagebreak\mylabel{v}\rehead{ }\normalsize\beginnumbering\briefempfaengerindex{Beer-Hofmann, Richard@\textsc{Beer-Hofmann, Richard}!zzzSchnitzler, Arthur@\emph{von Arthur Schnitzler}!1908-09-161@{16. 9. 1908}|(be} \toendnotes[C]{\smallbreak\pagebreak[2]} \Standort{YCGL, MSS 31.}
\physDesc{Brief, 1 Blatt (Briefpapier mit Trauerrand), 3 Seiten, Umschlag
\newline{}Handschrift: schwarze Tinte, deutsche Kurrent\newline{}Versand: Stempel: »\nobreak{}Wien, 16, IX. 08, XII\nobreak{}«.  }\buchAbdrucke{\weitereDrucke{Arthur Schnitzler, Richard Beer-Hofmann: \emph{Briefwechsel 1891–1931}. Hg. Konstanze Fliedl. Wien, Zürich: \emph{Europaverlag} 1992, S. 190.} }\toendnotes[C]{\smallbreak}\pstart{}{\pb}\textcolor{gray}{\textbf{Dr. Arthur Schnitzler}}\pend{}\pstart{}\textcolor{gray}{\textbf{\textcolor{pink}{Wien XVIII. Spoettelgasse 7}{}\ledrightnote{\textcolor{pink}{Edmund-Weiß-Gasse}}.}}\pend{}{\bigskip}\pstart{}{\pb}\textsc{Dr. Richard Beer-Hofmann,}\pend{}\pstart{}\textcolor{pink}{Wien XVIII}{}\ledrightnote{\textcolor{pink}{XVIII., Währing}}\pend{}\pstart{}\textcolor{pink}{\textsc{Hasenauerstr. 59}}{}\ledrightnote{\textcolor{pink}{Hasenauerstraße}}.\pend{}{\bigskip}\pstart
           \noindent{}{\pb}\textcolor{gray}{\textbf{Dr. Arthur Schnitzler}}\hfill 16. 9. 08\pend
           \pstart
           \textcolor{gray}{\textbf{\textcolor{pink}{Wien XVIII. Spoettelgasse 7}{}\ledrightnote{\textcolor{pink}{Edmund-Weiß-Gasse}}.}}\pend
           \pstart
           lieber Richard, geſtern hab ich auf dem Umweg über \textcolor{pink}{Auſſee}{}\ledrightnote{\textcolor{pink}{Bad Aussee}} – wo es Dr \textcolor{blue}{Rudi
                  Kaufmann}{}\ledrightnote{\textcolor{blue}{Rudolf Kaufmann}} der \textcolor{blue}{Agnes Speyer}{}\ledrightnote{\textcolor{blue}{Agnes Ulmann}} erzählt hat,
                  verno{\geminationm}en, daſs man \textcolor{blue}{Paula}{}\ledrightnote{\textcolor{blue}{Paula Beer-Hofmann}} von der überſtandenen Krankheit überhaupt nichts mehr anſieht – ſo
               darf man alſo hoffen, daſs alle Jammergründe verſchwunden ſind. Ihre Karte, aus \textsc{\textcolor{pink}{Seis}{}\ledrightnote{\textcolor{pink}{Seis am Schlern}}} nachgeſchickt, fand ich vorgeſtern Montag früh bei unſrer An{\pb}kunft aus \textcolor{pink}{München}{}\ledrightnote{\textcolor{pink}{München}}
               vor. Haben Sie unſre \label{K_L01790_1v}\edtext{Karte aus \textcolor{pink}{\textsc{Martino}}{}\ledrightnote{\textcolor{pink}{San Martino di Castrozza}}}{\lemma{\textnormal{\emph{Karte aus Martino}}}\Cendnote{\textnormal{nicht
                  überliefert}}}\label{K_L01790_1h} beko{\geminationm}en? –\pend
           \pstart
           Wir ſind mit dem Auto – einem Poſtauto, also keinem \label{K_L01790_2v}\edtext{Nachkaſtl}{\lemma{\textnormal{\emph{Nachkaſtl}}}\Cendnote{\textnormal{vgl. Arthur und Olga Schnitzler an Richard und Paula Beer-Hofmann,
               11. 5. 1908}}}\label{K_L01790_2h} von \textcolor{pink}{Bozen}{}\ledrightnote{\textcolor{pink}{Bozen}} hin u wieder zurückgefahren. In
                  \textcolor{pink}{München}{}\ledrightnote{\textcolor{pink}{München}} war das intereſſanteſte, was wir
               geſehen haben, die \label{K_L01790_3v}\edtext{\textcolor{green}{\textsc{Faust}}{}\ledrightnote{\textcolor{green}{Faust}} Inſcenirung}{\lemma{\textnormal{\emph{Faust Inſcenirung}}}\Cendnote{\textnormal{siehe A. S.: \emph{Tagebuch}, 12. 9. 1908}}}\label{K_L01790_3h} von \textcolor{blue}{\textsc{Erler}}{}\ledrightnote{\textcolor{blue}{Fritz Erler}} im \textcolor{brown}{Künſtleriſchen Theater}{}\ledrightnote{\textcolor{brown}{Münchner Künstlertheater}}. Auch das \label{K_L01790_4v}\edtext{\textcolor{green}{Zwiſchenſpiel}{}\ledrightnote{\textcolor{green}{Zwischenspiel. Komödie in drei Akten}} hab ich erlebt}{\lemma{\textnormal{\emph{Zwiſchenſpiel … erlebt}}}\Cendnote{\textnormal{siehe A. S.: \emph{Tagebuch}, 10. 9. 1908}}}\label{K_L01790_4h}, im \textcolor{pink}{Reſidenztheater}{}\ledrightnote{\textcolor{pink}{Residenztheater München}}, aber es iſt mir ſchon
               beſſer. Von meinem \textcolor{green}{Roman}{}\ledrightnote{→\textcolor{green}{Der Weg ins Freie. Roman}}{ }{\pb}kommt eben die 14.–20. Auflage. Ich werde trotzdem
               nicht \strikeout{a\textcolor{gray}{us}} irre an ihm{ }{\dots}\pend
           \pstart
           Angefangen habe ich manches in \textsc{\textcolor{pink}{Seis}{}\ledrightnote{\textcolor{pink}{Seis am Schlern}}}; darüber mündlich. Wann kommen Sie – ? Ich ſchicke den Brief an Ihre \textcolor{pink}{Wien}{}\ledrightnote{\textcolor{pink}{Wien}}er Adreſſe, da Sie ſchon am 15.{ }\textcolor{pink}{\textsc{Venedig}}{}\ledrightnote{\textcolor{pink}{Venedig}} verlaſſen.\pend
           \pstart
           Ich wünſche von Herzen {\dotstwo} ebenſo wie \textcolor{blue}{Olga}{}\ledrightnote{\textcolor{blue}{Olga Schnitzler}}{ }{\dotstwo}{ }nun Sie wiſſen es \textcolor{blue}{Beide}{}\ledrightnote{→\textcolor{blue}{Paula Beer-Hofmann}}. Grüßen Sie auch die \textcolor{blue}{Kinder}{}\ledrightnote{→\textcolor{blue}{Gabriel Beer-Hofmann}{\newline}→\textcolor{blue}{Naëmah Beer-Hofmann}{\newline}→\textcolor{blue}{Mirjam Beer-Hofmann}}.\pend
           \pstart
           Ihr{\\[\baselineskip]}\spacefill\mbox{Arthur.}\pend
           \leftskip=0em{}\endnumbering\briefempfaengerindex{Beer-Hofmann, Richard@\textsc{Beer-Hofmann, Richard}!zzzSchnitzler, Arthur@\emph{von Arthur Schnitzler}!1908-09-161@{16. 9. 1908}|)be}\mylabel{h}  \normalsize

\doendnotes{C}
\bigskip
\vfill

\clearpage

\footnotesize

\lohead{\textsc{register}}

% Definiere theindex-Environment komplett neu ohne reledmac
\makeatletter
\renewenvironment{theindex}{%
  \section*{\indexname}%
  \setlength{\parindent}{0pt}%
  \setlength{\parskip}{0pt plus 0.3pt}%
  \let\item\@idxitem
}{%
  \clearpage
}
\makeatother

\IfFileExists{\jobname-pw.ind}{\input{\jobname-pw.ind}}{}

\end{document}

      