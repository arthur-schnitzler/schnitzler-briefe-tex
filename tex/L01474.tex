%% latex-korrekturansicht-vorspann.tex
%% Vorspann für die Korrekturansicht.
%% Lädt die gemeinsame Datei latex-vorspann.tex mit gesetztem Schalter.

\newif\ifkorrekturansicht
\korrekturansichttrue

\input{../tex-inputs/latex-vorspann}


               \section[Hermann Bahr an Arthur Schnitzler, 4. {[}12.{]} 1904]{ Hermann Bahr an Arthur Schnitzler, 4. {[}12.{]} 1904}\nopagebreak\mylabel{v}\rehead{ }\normalsize\beginnumbering\briefempfaengerindex{Schnitzler, Arthur@\textsc{Schnitzler, Arthur}!zzzBahr, Hermann@\emph{von Hermann Bahr}!1904-12-041@{4. {[}12.{]} 1904}|(be} \toendnotes[C]{\smallbreak\pagebreak[2]} \Standort{CUL, Schnitzler, B 5b.}
\physDesc{Brief, 1 Blatt, 2 Seiten
\newline{}Handschrift: schwarze Tinte, deutsche Kurrent\newline{}Ordnung: mit Bleistift von unbekannter Hand nummeriert:
                                    »124« }\buchAbdrucke{\weitereDrucke{Hermann Bahr, Arthur Schnitzler: \emph{Briefwechsel, Aufzeichnungen, Dokumente (1891–1931)}. Hg. Kurt Ifkovits und Martin Anton Müller. Göttingen: \emph{Wallstein} 2018, S. 326–327.} }\toendnotes[C]{\smallbreak}\pstart
           \raggedleft{}{\pb}4. \label{T_L01474_1v}\edtext{11.}{\lemma{\textnormal{\emph{11.}}}\Cendnote{\textnormal{Schreibirrtum, durch den Inhalt auf
                        Dezember zu datieren.}}}\label{T_L01474_1h} 04\pend
           \pstart\center{}Lieber Arthur!\pend\pstart
           Bitte, kannſt Du mir den »\textcolor{green}{Puppenſpieler}{}\ledrightnote{\textcolor{green}{Der Puppenspieler}}« gedruckt
               ſchicken? Ich möchte, wenn es mir zuſammengeht, über den Schnitzlerabend
               ausführlicher ſchreiben. Dazu wäre es mir allerdings ſehr lieb, das Buch noch vor
               Donnerſtag zu kriegen. Ja?\pend
           \pstart
           Sehr gern möchte ich Dich auch endlich wieder ſehen. Allerdings bin ich wenig frei,
               da ich mich nun mit einer gewiß törichten \label{K_L01474_1v}\edtext{Leidenſchaft}{\lemma{\textnormal{\emph{Leidenſchaft}}}\Cendnote{\textnormal{die
                  Bekanntschaft mit seiner späteren zweiten Frau, der Opernsängerin \textcolor{blue}{Anna von Mildenburg}}}}\label{K_L01474_1h}, der ich aber momentan so viel
               unſagbares Glück verdanke, wie ich nie im Leben kannte (\label{LL286-1v}vielleicht wird man ſo ganz transparenter Seligkeiten erſt im
                  Angeſicht des Todes fähig\label{LL286-1h}), aufs Hören von Musik geworfen habe, wovon ich
               dann manchmal in einer Ermattung mit {\pb}vollſtändigem
               Verſagen und Verſiegen jeder Kraft zurückbleibe. \textsc{Vita
                  minima}, die auch ihre ſchönen Schauder hat. Wie eben jetzt, ſonſt würde ich
               Dir dieſen Unsinn nicht ſchreiben, \textsc{enfin} ich wollte ſagen:
               ich möchte Dich gern wiederſehen und hoffe bald zu Dir zu kommen. Und was würdeſt Du
               zu der Idee ſagen: zu Weihnachten uns in \textcolor{pink}{Lueg}{}\ledrightnote{\textcolor{pink}{Lueg am Wolfgangsee}}{ }\introOben{}am \textcolor{pink}{Wolfgangſee}{}\ledrightnote{\textcolor{pink}{Wolfgangsee}}\introOben{} zu treffen, wo ich ein paar \label{K_L01474_2v}\edtext{Tage beim \textcolor{blue}{Burckhard}{}\ledrightnote{\textcolor{blue}{Max Eugen Burckhard}}}{\lemma{\textnormal{\emph{Tage beim Burckhard}}}\Cendnote{\textnormal{\textcolor{blue}{Bahr} fährt am 24. und bleibt bis
                     27. 12. 1904 und verpasst \textcolor{blue}{Schnitzler} knapp.}}}\label{K_L01474_2h} hauſen will? Ich wollte eigentlich nach \textcolor{pink}{Athen}{}\ledrightnote{\textcolor{pink}{Athen}}, aber da müßte ich am 20. von \textcolor{pink}{Trieſt}{}\ledrightnote{\textcolor{pink}{Triest}} weg und \label{K_L01474_3v}\edtext{am
                  22. ist der \textcolor{green}{Triſtan}{}\ledrightnote{\textcolor{green}{Tristan und Isolde}}}{\lemma{\textnormal{\emph{am
                  22. ist der Triſtan}}}\Cendnote{\textnormal{Die Aufführung von \emph{\textcolor{green}{Tristan und Isolde}} war noch am 8. 12. 1904 für
                  den 22. angesetzt (vgl. Brief Bahrs an \textcolor{blue}{Anna
                        Mildenburg}, 8. 12. 1904, \emph{Theatermuseum
                        Wien}, AM 43853 BaM), wurde aber auf den 23. 12. 1904
                  verschoben.}}}\label{K_L01474_3h}, der für mich jetzt – ganz real und ganz phyſiſch geſprochen –
               das höchſte Wolſein iſt, mehr als Sonne und Meer.\pend
           \pstart
           Entſchuldige den verworrenen Ton dieſes Briefes, grüße Frau \textcolor{blue}{Olga}{}\ledrightnote{\textcolor{blue}{Olga Schnitzler}} und den \textcolor{blue}{Heinrich}{}\ledrightnote{\textcolor{blue}{Heinrich Schnitzler}}
               herzlichſt und ſei es ſelbſt von{\\[\baselineskip]}Deinem{\\[\baselineskip]}\spacefill\mbox{Hermann}\pend
           \leftskip=0em{}\endnumbering\briefempfaengerindex{Schnitzler, Arthur@\textsc{Schnitzler, Arthur}!zzzBahr, Hermann@\emph{von Hermann Bahr}!1904-12-041@{4. {[}12.{]} 1904}|)be}\mylabel{h}  \normalsize

\doendnotes{C}
\bigskip
\vfill

\clearpage

\footnotesize

\lohead{\textsc{register}}

% Definiere theindex-Environment komplett neu ohne reledmac
\makeatletter
\renewenvironment{theindex}{%
  \section*{\indexname}%
  \setlength{\parindent}{0pt}%
  \setlength{\parskip}{0pt plus 0.3pt}%
  \let\item\@idxitem
}{%
  \clearpage
}
\makeatother

\IfFileExists{\jobname-pw.ind}{\input{\jobname-pw.ind}}{}

\end{document}

      