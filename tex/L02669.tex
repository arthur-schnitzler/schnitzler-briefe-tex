\documentclass[twoside=false,titlepage=false,open=any, parskip=never, fontsize=12pt, headings=small, chapterprefix=false, appendixprefix=false]{scrbook}
\addtolength{\oddsidemargin}{\evensidemargin}
\setlength{\oddsidemargin}{.5\oddsidemargin}
\setlength{\evensidemargin}{\oddsidemargin}

\usepackage[{textwidth=13cm,textheight=23cm,marginpar=3cm, left=2cm}]{geometry}
%\usepackage[textwidth=80mm, layoutwidth=170mm, paperheight =297mm, paperwidth  =210mm, layoutvoffset= 20mm,layouthoffset= 20mm]{geometry}
%\usepackage[paperheight =297mm, paperwidth  =210mm, layoutheight=230mm, layoutwidth=158mm, layoutvoffset= 20mm, layouthoffset= 20mm, textwidth=150mm, textheight=185mm, showcrop=false]{geometry}
%sepackage[paperheight=230mm, paperwidth=138mm, textwidth=100mm, textheight=185mm]{geometry}
 \usepackage[usenames, dvipsnames]{xcolor}
\usepackage{scrlayer-scrpage}
\usepackage{hyphenat}
\usepackage{fontspec}
\usepackage{moresize}
\usepackage[english, french, greek, ngerman]{babel}
%\usepackage{ipa}  für das Seitenwechselzeichens
\usepackage[babel]{microtype}
\usepackage[dash, dot]{dashundergaps}
\usepackage{soul}
\usepackage{ragged2e}
\usepackage[makeindex, protected]{splitidx}
\usepackage[itemlayout=abshang,hangindent=0.85em, subindent=0em, subsubindent=1em, justific=RaggedRight, columns=1, columnsep=0pt, indentunit=1em, totoc=false]{idxlayout}
\usepackage{scrhack}
\usepackage{xpatch}
\usepackage{reledmac}
\usepackage{refcount} % Für die Seitenverweise 1–3 etc. 
\usepackage{etoolbox}
\usepackage{framed}
\usepackage[export]{adjustbox} % loads also graphicx, für Bildgröße autom. maximal
\usepackage{float} %ermöglicht exakte Bildpositionierung
\usepackage{mdframed}
\usepackage{enumitem}
\usepackage{relsize}
\usepackage{longtable}
\usepackage{chngcntr} % Sectionnummern durchgehend
\usepackage{hanging} % Für hängende Absätze
\usepackage[rightmargin=0em, leftmargin=1em, indentfirst=false]{quoting} % Für die geänderte quote-Umgebung in den Hrsg-Texten
%\usepackage{fontawesome}
\usepackage{ellipsis}
\RequirePackage{hyphsubst}%
\HyphSubstIfExists{ngerman-x-latest}{\HyphSubstLet{ngerman}{ngerman-x-latest}}{} 
\listfiles
\usepackage[noadjust]{marginnote}

\KOMAoptions{toc=chapterentrydotfill, toc=flat}
\addtokomafont{chapterentrypagenumber}{\mdseries}
\setkomafont{chapterentry}{\normalfont\mdseries}
\setkomafont{partentry}{\normalfont\mdseries}
\RedeclareSectionCommand[tocbeforeskip=0pt]{chapter}

\setlength{\skip\footins}{4mm plus 2mm} % Abstand Fussnote Text
\interfootnotelinepenalty=10000 % Kein Seitenwechsel in Fuss

%\DeclareTextFontCommand{\emph}{\textit}

% Der Befehl erlaubt rechtsbündig bei Unterschriften, die nicht mehr in die Zeile passen
\def\spacefill{\hspace{\fill}\mbox{}\linebreak[0]\hspace*{\fill}}
\usepackage{atbegshi}
\usepackage{zref-abspage}
\usepackage{perpage}
\usepackage{zref-user}
\usepackage{tikz}
\usepackage{ulem}
\usetikzlibrary{calc,decorations.pathmorphing}

\PassOptionsToPackage{gray}{xcolor}
\definecolor{gray}{gray}{0.6}

\doublehyphendemerits=1000000 % das hier verhindert zu viele aufeinanderfolgende Trennstriche am Zeilenende


\usepackage{zref-abspage}
\usepackage{zref-user}
\usepackage{tikz}
\usepackage{atbegshi}
\usepackage{ulem}
\usetikzlibrary{calc,decorations.pathmorphing}

\PassOptionsToPackage{gray}{xcolor}
\definecolor{gray}{gray}{0.6}

\doublehyphendemerits=1000000 % das hier verhindert zu viele aufeinanderfolgende Trennstriche am Zeilenende

\makeatletter
\newcommand{\currentsidemargin}{%
  \ifodd\zref@extract{textarea-\thetextarea}{abspage}%
    \oddsidemargin%
  \else%
    \evensidemargin%
  \fi%
}

\newcounter{textarea}
\newcommand{\settextarea}{%
   \stepcounter{textarea}%
   \zlabel{textarea-\thetextarea}%
   \begin{tikzpicture}[overlay,remember picture]
    % Helper nodes
    \path (current page.north west) ++(\hoffset, -\voffset)
        node[anchor=north west, shape=rectangle, inner sep=0, minimum width=\paperwidth, minimum height=\paperheight]
        (pagearea) {};
    \path (pagearea.north west) ++(1in+\currentsidemargin,-1in-\topmargin-\headheight-\headsep)
        node[anchor=north west, shape=rectangle, inner sep=0, minimum width=\textwidth, minimum height=7pt]
        (textarea) {};
  \end{tikzpicture}%
}

\tikzset{tikzul/.style={yshift=-.75\dp\strutbox}}

\newcounter{tikzul}%
\newcommand\tikzul[1][]{%
    \begingroup
    \global\tikzullinewidth\linewidth
    \def\tikzulsetting{[#1]}%
    \stepcounter{tikzul}%
    \settextarea
    \zlabel{tikzul-begin-\thetikzul}%
    \tikz[overlay,remember picture,tikzul] \coordinate (tikzul-\thetikzul) at (0,0);% Modified \tikzmark macro
    \ifnum\zref@extract{tikzul-begin-\thetikzul}{abspage}=\zref@extract{tikzul-end-\thetikzul}{abspage}
    \else
        \AtBeginShipoutNext{\tikzul@endpage{#1}}%
    \fi
    \bgroup
    \def\par{\ifhmode\unskip\fi\egroup\par\@ifnextchar\noindent{\noindent\tikzul[#1]}{\tikzul[#1]\bgroup}}%
    \aftergroup\endtikzul
    \let\@let@token=%
}
\newlength\tikzullinewidth


\def\tikzul@endpage#1{%
\setbox\AtBeginShipoutBox\hbox{%
\box\AtBeginShipoutBox
\hbox{%
\begin{tikzpicture}[overlay,remember picture,tikzul]
\draw[#1]
    let \p1 = (tikzul-\thetikzul), \p2 = ([xshift=\tikzullinewidth+\@totalleftmargin]textarea.south west) in
    \ifdim\dimexpr\y1-\y2<.5\baselineskip
        (\x1,\y1) -- (\x2,\y1)
    \else
        let \p3 = ([xshift=\@totalleftmargin]textarea.west) in
        (\x1,\y1) -- +(\tikzullinewidth-\x1+\x3,0)
        % (\x3,\y2) -- (\x2,\y2)
        (\x3,\y1)
       \myloop{\y1-\y2+.5\baselineskip}{%
           ++(0,-\baselineskip) -- +(\tikzullinewidth,0)
       }%
    \fi
;
\end{tikzpicture}%
}}%
}%


\def\endtikzul{%
    \zlabel{tikzul-end-\thetikzul}%
    \ifnum\zref@extract{tikzul-begin-\thetikzul}{abspage}=\zref@extract{tikzul-end-\thetikzul}{abspage}
    \begin{tikzpicture}[overlay,remember picture,tikzul]
        \expandafter\draw\tikzulsetting
            let \p1 = (tikzul-\thetikzul), \p2 = (0,0) in
            \ifdim\y1=\y2
                (\x1,\y1) -- (\x2,\y2)
            \else
                let \p3 = ([xshift=\@totalleftmargin]textarea.west), \p4 = ([xshift=-\rightmargin]textarea.east) in
                (\x1,\y1) -- +(\tikzullinewidth-\x1+\x3,0)
                (\x3,\y2) -- (\x2,\y2)
                (\x3,\y1)
                \myloop{\y1-\y2}{%
                    ++(0,-\baselineskip) -- +(\tikzullinewidth,0)
                }%
            \fi
        ;
    \end{tikzpicture}%
    \else
    \settextarea
    \begin{tikzpicture}[overlay,remember picture,tikzul]
        \expandafter\draw\tikzulsetting
            let \p1 = ([xshift=\@totalleftmargin,yshift=-.5\baselineskip]textarea.north west), \p2 = (0,0) in
            \ifdim\dimexpr\y1-\y2<.5\baselineskip
                (\x1,\y2) -- (\x2,\y2)
            \else
                let \p3 = ([xshift=\@totalleftmargin]textarea.west), \p4 = ([xshift=-\rightmargin]textarea.east) in
                (\x3,\y2) -- (\x2,\y2)
                (\x3,\y2)
                \myloop{\y1-\y2}{%
                    ++(0,+\baselineskip) -- +(\tikzullinewidth,0)
                }
            \fi
        ;
    \end{tikzpicture}%
    \fi
    \endgroup
}

% -------------------------------------------------------------- Additions by Peter Grill

\tikzset{tikzst/.style={yshift=0.5\dp\strutbox}}

\newcounter{tikzst}%
\newcommand\tikzst[1][]{%
    \begingroup
    \global\tikzstlinewidth\linewidth
    \def\tikzstsetting{[#1]}%
    \stepcounter{tikzst}%
    \settextarea
    \zlabel{tikzst-begin-\thetikzst}%
    \tikz[overlay,remember picture,tikzst] \coordinate (tikzst-\thetikzst) at (0,0);% Modified \tikzmark macro
    \ifnum\zref@extract{tikzst-begin-\thetikzst}{abspage}=\zref@extract{tikzst-end-\thetikzst}{abspage}
    \else
        \AtBeginShipoutNext{\tikzst@endpage{#1}}%
    \fi
    \bgroup
    \def\par{\ifhmode\unskip\fi\egroup\par\@ifnextchar\noindent{\noindent\tikzst[#1]}{\tikzst[#1]\bgroup}}%
    \aftergroup\endtikzst
    \let\@let@token=%
}
\newlength\tikzstlinewidth


\def\tikzst@endpage#1{%
\setbox\AtBeginShipoutBox\hbox{%
\box\AtBeginShipoutBox
\hbox{%
\begin{tikzpicture}[overlay,remember picture,tikzst]
\draw[#1]
    let \p1 = (tikzst-\thetikzst), \p2 = ([xshift=\tikzstlinewidth+\@totalleftmargin]textarea.south west) in
    \ifdim\dimexpr\y1-\y2<.5\baselineskip
        (\x1,\y1) -- (\x2,\y1)
    \else
        let \p3 = ([xshift=\@totalleftmargin]textarea.west) in
        (\x1,\y1) -- +(\tikzstlinewidth-\x1+\x3,0)
        % (\x3,\y2) -- (\x2,\y2)
        (\x3,\y1)
       \myloop{\y1-\y2+.5\baselineskip}{%
           ++(0,-\baselineskip) -- +(\tikzstlinewidth,0)
       }%
    \fi
;
\end{tikzpicture}%
}}%
}%


\def\endtikzst{%
    \zlabel{tikzst-end-\thetikzst}%
    \ifnum\zref@extract{tikzst-begin-\thetikzst}{abspage}=\zref@extract{tikzst-end-\thetikzst}{abspage}
    \begin{tikzpicture}[overlay,remember picture,tikzst]
        \expandafter\draw\tikzstsetting
            let \p1 = (tikzst-\thetikzst), \p2 = (0,0) in
            \ifdim\y1=\y2
                (\x1,\y1) -- (\x2,\y2)
            \else
                let \p3 = ([xshift=\@totalleftmargin]textarea.west), \p4 = ([xshift=-\rightmargin]textarea.east) in
                (\x1,\y1) -- +(\tikzstlinewidth-\x1+\x3,0)
                (\x3,\y2) -- (\x2,\y2)
                (\x3,\y1)
                \myloop{\y1-\y2}{%
                    ++(0,-\baselineskip) -- +(\tikzstlinewidth,0)
                }%
            \fi
        ;
    \end{tikzpicture}%
    \else
    \settextarea
    \begin{tikzpicture}[overlay,remember picture,tikzst]
        \expandafter\draw\tikzstsetting
            let \p1 = ([xshift=\@totalleftmargin,yshift=-.5\baselineskip]textarea.north west), \p2 = (0,0) in
            \ifdim\dimexpr\y1-\y2<.5\baselineskip
                (\x1,\y2) -- (\x2,\y2)
            \else
                let \p3 = ([xshift=\@totalleftmargin]textarea.west), \p4 = ([xshift=-\rightmargin]textarea.east) in
                (\x3,\y2) -- (\x2,\y2)
                (\x3,\y2)
                \myloop{\y1-\y2}{%
                    ++(0,+\baselineskip) -- +(\tikzstlinewidth,0)
                }
            \fi
        ;
    \end{tikzpicture}%
    \fi
    \endgroup
}
% --------------------------------------------------------------

\def\myloop#1#2#3{%
    #3%
    \ifdim\dimexpr#1>1.1\baselineskip
        #2%
        \expandafter\myloop\expandafter{\the\dimexpr#1-\baselineskip\relax}{#2}%
    \fi
}

\makeatother






\def\myloop#1#2#3{%
    #3%
    \ifdim\dimexpr#1>1.1\baselineskip
        #2%
        \expandafter\myloop\expandafter{\the\dimexpr#1-\baselineskip\relax}{#2}%
    \fi
}

\makeatother
%\newcommand{\damage}[1]{\tikzul[gray,line width=0.15\ht\strutbox,semitransparent]{#1}}
%\newcommand{\strikeout}[1]{\tikzst[black]{#1}}

\newcommand{\damage}[1]{\textcolor{orange}{#1}}
\newcommand{\strikeout}[1]{\sout{#1}}


\setlength{\parindent}{1em}

% Mehr als drei Auslassungspunkte 

\newcommand{\dotsseven}{%
.\kern\ellipsisgap 
.\kern\ellipsisgap
.\kern\ellipsisgap 
.\kern\ellipsisgap
.\kern\ellipsisgap
.\kern\ellipsisgap 
.\kern\ellipsisgap 	
\relax}

\newcommand{\dotssix}{%
.\kern\ellipsisgap 
.\kern\ellipsisgap
.\kern\ellipsisgap
.\kern\ellipsisgap
.\kern\ellipsisgap 
.\kern\ellipsisgap 
\relax}

\newcommand{\dotsfive}{%
.\kern\ellipsisgap 
.\kern\ellipsisgap
.\kern\ellipsisgap
.\kern\ellipsisgap 
.\kern\ellipsisgap 
\relax}

\newcommand{\dotsfour}{%
.\kern\ellipsisgap 
.\kern\ellipsisgap
.\kern\ellipsisgap
.\kern\ellipsisgap 
\relax}

\newcommand{\dotstwo}{%
.\kern\ellipsisgap 
.\kern\ellipsisgap
\relax}


% Silbentrennung
\selectlanguage{ngerman}
\hyphenation{Re-kours EP-STEIN Her-vay-vor-les-ung Steu-er-sa-chen Öst-reich Burck-hard Keuch-hus-ten Oedi-pus-auf-führ-un-gen Hi-obs-post Kärnt-ner-ring Vei-tlis-sen-gas-se Franck-gas-se Rath-hau-se Sechs-schg Stu-bai-thal Tha-deusz Volks-th Halb-mo-nats-schrift JAHR-ES-ZEI-TEN Te-le-phon mit-ge-theilt Ge-schäfts-ver-bin-dung hoch-müth-ig Ueber-zeu-gung bis-chen Au-tor-rech-te Hof-manns-thal Nor-deijk Irre-seins Tschap-perl mit-zu-thei-len Aeu-ße-rung be-thö-ren Kü-ni-gel Be-ur-thei-lung Kuenst-lern ko-moe-di-sche hae-mor-rha-gi-scher Doer-mann Wash-burn flei-ssig haute Buddh-ist Preu-ssen Lin-den-café Mit-theil-un-gen An-theil Lieu-te-nant oes-terr Rieg-ner Oes-ter-reich gro-ssem Fran-zo-sen-thum Roche Lili Ent-schlie-ssun-gen äu-ssert wuen-sche Trans-ac-tio-nen Ue-ber-win-dung Eu-gene Stra-ssen-dir-ne qua-tre Deutsch-öst-er-reich Deutsch-öst-er-reichs Bjørn-stjer-ne noth-ing Edit-ed Olga Ar-naud Mer-gent-heim Léon-tine Polla-czek Brion Barre Hoch-sin-ger Ka-tha-rina Arouet Va-len-ci-ennes Ueber-win-dung Type-writer-in Tolstoi-buch Schnitzler Copier-buche Schiller Intel-lek-tuell-en-as-so-zi-a-tion Salten Devrient Grien-steidl Ge-sell-ſchaft ein-ge-ſchloſ-ſen Fort-ſetz-un-gen Bor-dell-ſtück fort-ſchrei-ten wirk-ſam-es ſchrift-ſtel-ler-i-ſchen hin-weg-ſe-hen Gerichts-saal-be-richt-er-ſtat-ter}



% Sonderbefehl für .–
\def\dotdash{\nobreak\hspace{0pt}.–}  %ACHTUNG BEIM ERSETZEN: LEERZEICHEN DANACH 
\def\commadash{\nobreak\hspace{0pt},–}
\def\excdash{\nobreak\hspace{0pt}!–}
\def\semicolondash{\nobreak\hspace{0pt};–}
\def\parentdotdash{\nobreak\hspace{0pt}).–}
\def\slashislash{\,\slash\,\allowbreak\hspace{0pt}}

\newcommand{\strich}{\makebox[1em][l]{– }}


% Seite einrichten

% Farbe definieren
%\setmainfont[RawFeature={-liga}, 
%SmallCapsFont=WSVgara-Caps, 
%ItalicFont=WSVgara-Italic, 
%BoldFont=WSVgara-Bold,
%BoldItalicFont=WSVgara-BoldItalic
%]{WSVgara}
%\setsansfont[RawFeature={-liga}, 
%SmallCapsFont=WSVgara-Caps, 
%ItalicFont=WSVgara-Italic, 
%BoldFont=WSVgara-Bold,
%BoldItalicFont=WSVgara-BoldItalic
%]{WSVgara}

%\setmainfont{Brill}
%\setsansfont{Brill}

%\setmainfont[ItalicFont=SinaNova-Italic, 
%BoldFont=SinaNova-Bold,
%BoldItalicFont=SinaNova-BoldItalic
%]{SinaNova-Regular}
%\setsansfont[ItalicFont=SinaNova-Italic, 
%BoldFont=SinaNova-Bold,
%BoldItalicFont=SinaNova-BoldItalic
%]{SinaNova-Regular}



\def\labelitemi{--}

% Geminationsstrich, U-Strich

 \newcommand{\overbar}[1]{$\overline{\hbox{#1}}$}


% Ausrufezeichen in den Index kriegen
\newcommand{\rufezeichen}{"!}

% Griechisch
	
%\newfontfamily\greekfont{GaramondPremrPro}
%\newcommand\griechisch[1]{\greekfont{}#1{}\normalfont}

%\newfontfamily\sansseriffont[HyphenChar=None, RawFeature={-liga}, Scale=1.03]{TheSans-Regular}
%\newfontfamily\sansseriffont{uarial}


%\newfontfamily\sansseriffont[HyphenChar=None, LetterSpace=1.0, RawFeature={-liga}]{TheSans-SemiBold}
%\newcommand\sansseriff[1]{\sffamily{}#1{}\normalfont}

\newcommand{\mini}{\,}


\newcommand{\key}{\textsuperscript{\textcolor{red}{KEY}}}


%% Sperrung (Package Soul)
%% Hier ist die Sperrung definiert. Sperrung erreicht man mit \so{gesperrtes Wort}
\sodef\so{}{.14em}{.4em plus.1em minus .1em}{.4em plus.1em minus .1em}

% SCHRIFTEN
\setkomafont{disposition}{}
\addtokomafont{caption}{\small}
\addtokomafont{captionlabel}{\small}

%% Schrift der Kopf und Fußzeile
\renewcommand*{\headfont}{\normalfont}
\setkomafont{pagehead}{\footnotesize\addfontfeature{LetterSpace=10.0}}
\setkomafont{pagenumber}{\normalfont\normalsize}
\ohead[]{\pagemark}% Seitenzahl (c = centered) 
\ofoot[]{}


 
% Flatterndes Seitenende
\raggedbottom

% Fussnoten neu Anfangen

\makeatletter
\pretocmd{\@schapter}{\setcounter{footnote}{0}}{}{}
\pretocmd{\@chapter}{\setcounter{footnote}{0}}{}{}
\pretocmd{\@section}{\setcounter{footnote}{0}}{}{}
\makeatother


% Section Nummern durchgehend

\RedeclareSectionCommand[
  counterwithout=chapter
]{section}

% Section Punkt

\renewcommand*{\sectionformat}{}
\renewcommand*{\partformat}{}


% Marginpar Schrift

\newkomafont{margin}{\footnotesize} 
\makeatletter 
\let\MarginParOriginal\marginpar 
\renewcommand*{\marginpar}{\@dblarg\@marginpar} 
\newcommand{\@marginpar}[2][]{% 
  \MarginParOriginal[\usekomafont{margin}{#1\par}]{\usekomafont{margin}{#2\par}} 
} 
\makeatother 



\let\oldbeginnumbering\beginnumbering

\def\beginnumbering{\oldbeginnumbering\par\nopagebreak}


% Fußnoten linksbündig
\deffootnote{1.5em}{1em}{% 
\makebox[1.5em][l]{\thefootnotemark}%
}


% Fussnotenlineal (wobei für reledmac wohl was anderes gilt)
\let\normalfootnoterule\footnoterule
\setfootnoterule{0pt}
\let\normalfootnoterule\footnoterule


\setlength{\skip\footins}{8mm plus 2mm} % Abstand Fussnote Text
\interfootnotelinepenalty=10000 % Kein Seitenwechsel in Fuss

%% Kapitelüberschriften
\renewcommand*{\raggedchapter}{\centering} 
\renewcommand*{\raggedsection}{%
 \CenteringLeftskip=1cm plus 1em\relax 
 \CenteringRightskip=1cm plus 1em\relax 
 \Centering\footnotesize\thesection{}.\ }
\setkomafont{section}{\footnotesize}
\setkomafont{chapter}{\normalfont\Large}
\renewcommand{\chapterpagestyle}{empty}%The first page in each chapter won't have any heading or footer, especially no page number

% section ohne führende Kapitelnummer
\renewcommand*\thesection{\arabic{section}}

% Bildunterschrift ohne Nummer
\renewcommand*{\figureformat}{}
\renewcommand*{\captionformat}{}

% Abstand Bild
\setlength{\textfloatsep}{\baselineskip}

%% Zeilennummern
\firstlinenum{0} \linenumincrement{5}
\lineation{section} %Jeder Abschnitt wird durchnummeriert
\renewcommand{\numlabfont}{\ssmall} %Schriftgröße Zeilennummern

%\AtBeginEnvironment{multicols}{\RaggedRight} % Linksbündig in Spalten


% SEITENUMBRÜCHE IM TEXT MARKIEREN

%% Seitenumbrüche


\newcommand{\Theight}{\dimexpr\fontcharht\font`W}
\newcommand{\pbposition}{\depth}
\newcommand{\pb}{\nobreak\hspace{0pt}\raisebox{-0.1em}{\raisebox{\pbposition}{\textnormal{|}}}\nobreak\hspace{0pt}}

% EINFÜGUNGEN IM TEXT MARKIEREN

\renewcaptionname{ngerman}{\contentsname}{Inhalt}           %Table of contents


\newcommand{\introOben}{\textnormal{\raisebox{\Theight}{\raisebox{-\height}{\small{v}\normalsize}}}}
\newcommand{\introUnten}{\textnormal{\raisebox{\Theight}{\raisebox{-\height}{\small{v}\normalsize}}}}
\newcommand{\introMitteVorne}{\textnormal{\raisebox{\Theight}{\raisebox{-\height}{\small{v}\normalsize}}}}
\newcommand{\introMitteHinten}{\textnormal{\raisebox{\Theight}{\raisebox{-\height}{\small{v}\normalsize}}}}
\newcommand{\substVorne}{\textnormal{\raisebox{\Theight}{\raisebox{-\height}{\rotatebox[origin=c]{180}{v}\normalsize}}}}
\newcommand{\substDazwischen}{}
\newcommand{\substHinten}{\textnormal{\raisebox{\Theight}{\raisebox{-\height}{\small{v}\normalsize}}}}


% MARGINALSPALTE
\setlength\ledrsnotewidth{1.5cm}


% FUSSNOTE
%% Im Apparat f. und ff.
\Xtwolines{f.}
\Xtwolinesbutnotmore

%% Sperrungen bei Lemmas im Apparat
%\pretocmd{\so}{\null}{}{}
% Hab ich auskommentiert: Hat einen Fehler ergeben, denn plötzlich war ein Abstand vor Absätzen, die mit einer Sperrung beginnen

%% Zeilennummerierung Abstand zum Lemma
\Xboxlinenum{5mm}

%% Bei zwei Apparateinträgen in einer Zeile wird nur beim ersten Mal die Zeile gezählt
\Xnumberonlyfirstinline
\Xnumberonlyfirstintwolines
\Xinplaceofnumber{1em}
\Xhangindent{1em}

% ENDNOTEN
\Xendlemmadisablefontselection[A]
\renewcommand*{\printnpnum}[1]{{\noindent}\tiny}
\Xendparagraph[A] % Endnoten in einem Absatz
%\Xendtwolines{\tiny{f.}}
\Xendbeforepagenumber{} 
\Xendnotenumfont[A]{\tiny}
\Xendboxlinenum[A]{0em}
\Xendlemmaseparator{$\rbracket$}
\Xendnotefontsize[A]{\footnotesize}
\Xendhangindent[A]{1em}
\Xendlemmafont[A]{\itshape}
\Xendlemmafont[B]{\bfseries}
\Xendnotefontsize[B]{\footnotesize}
\Xendnotenumfont{\footnotesize}
\Xendlineprefixsingle[C]{\tiny}
\Xendlineprefixmore[C]{\tiny}
\Xendlemmadisablefontselection
\Xendlemmafont{\itshape}
\Xendlinerangeseparator{\tiny{--}}
\Xendhangindent{4em}
\Xendboxlinenum{3.6em}
\Xendafternumber{0.4em}
\Xendboxlinenumalign{R}

%\Xendboxstartlinenum{3.5em}
%\Xendboxendlinenum{1em}


%% Kaufmanns-Und (=)
            
            

\newcommand{\kaufmannsund}{\&} 

%% Tabelle Zellensprung
% Ein weiterer Anlass, das Kaufmannsund in der Übergabe zu vermeiden:

\newcommand{\zellensprung}{ \& }

%% INDEX
    
    \makeindex 
    \newcommand*\lettergroup[1]{}
    
        \newcommand{\pw}[1]{#1}
        \newcommand{\pwt}[1]{\textbf{#1}}
        \newcommand{\pws}[1]{\upshape{\textbf{#1}}}
            
        \newcommand{\pwe}[1]{\textbf{\emph{#1}}}
             
    \newcommand{\pwk}[1]{#1\textsuperscript{\tiny{K}}}
    \newcommand{\pwv}[1]{\emph{#1}}
     \newcommand{\pwkv}[1]{\emph{#1}\textsuperscript{\tiny{K}}}
               \newcommand{\pwuv}[1]{\emph{#1}?}
               \newcommand{\pwu}[1]{#1?}
 \newcommand{\range}[2]{{\def\pw##1{##1}#1}--#2}

\newcommand{\buch}[1]{#1}


%% MEHRERE INDIZES

\newindex[Register]{pw}
%\newindex[Institutionen Organisationen Periodika und Unternehmen]{org}
%\newindex[Institutionen und Orte]{o}
\newindex[Korrespondenzpartner]{briefe-out}
\newindex[Gedruckte Quellen]{buch-abdruck}

\newcommand\briefsenderindex[1]{\sindex[briefe-out]{#1}}
\newcommand\briefempfaengerindex[1]{\sindex[briefe-out]{#1}}

\newcommand\buchabdruck[1]{\sindex[buch-abdruck]{#1}}
\renewcommand\buchabdruck[1]{}



%% Symbole

%\newcommand{\symaddr}{\includegraphics[height=6pt]{symbol/noun_637366.png}}
%\newcommand{\symweiteredrucke}{\includegraphics[height=6pt]{symbol/noun_634729.png}}
%\newcommand{\symdruckvorlage}{\includegraphics[height=6pt]{symbol/noun_637409.png}}
%\newcommand{\symstandort}{\includegraphics[height=6pt]{symbol/noun_634216.png}}
%\newcommand{\symhead}{\includegraphics[height=6pt]{symbol/noun_1162030_cc.png}}


\newcommand{\symaddr}{A}
\newcommand{\symweiteredrucke}{D}
\newcommand{\symdruckvorlage}{V}
\newcommand{\symstandort}{O}
\newcommand{\symhead}{H}



\newcommand\anhangTitel[2]{\toendnotes[C]{\hangpara{4em}{1}{\makebox[4em][l]{\textbf{#1}}\textbf{#2}}\endgraf}}
\newcommand\Adresse[1]{\toendnotes[C]{\hangpara{4em}{1}{\makebox[4em][l]{\makebox[3.6em][r]{\symaddr}}}#1\endgraf}}

\newcommand\buchAlsQuelle[1]{\toendnotes[C]{\footnotesize\par\hangpara{4em}{1}{\makebox[4em][l]{\makebox[3.6em][r]{\symdruckvorlage}}}#1\endgraf}}
\newcommand\buchAbdrucke[1]{\toendnotes[C]{\footnotesize\par\hangpara{4em}{1}{\makebox[4em][l]{\makebox[3.6em][r]{\symweiteredrucke}}}#1\endgraf}}
\newcommand\Standort[1]{\toendnotes[C]{\footnotesize\hangpara{4em}{1}{\makebox[4em][l]{\makebox[3.6em][r]{\symstandort}}}#1\endgraf}}
\newcommand\biographical[1]{\toendnotes[C]{\footnotesize\hangpara{4em}{1}{\makebox[4em][l]{\makebox[3.6em][r]{\symhead}}}#1\endgraf}}
\newcommand\biographicalOhne[1]{\toendnotes[C]{\footnotesize\hangpara{4em}{1}{\makebox[4em][l]{\makebox[3.6em][r]{}}}#1\endgraf}}



\newcommand\datumImAnhang[1]{\toendnotes[C]{#1}}

\let\newcell&

\newcommand\physDesc[1]{\toendnotes[C]{\hangpara{4em}{0}#1\endgraf}}
\newcommand\weitereDrucke[1]{#1}


% Schnitzler Tagebuch Auszüge
\newcommand{\prgrph}[1]{\endgraf\medskip\noindent\textbf{#1}\newline}


%% VERWEISE
% Dieser Befehl vom Typ
% \verweis{FW_V_schwn_A}{FW_V_schwn_E} 
% dient den Verweisen auf den Text von Kommentar und Herausgebereingriffen. Ihm werden die Namen der beiden Labels – Anfang und Ende – übergeben und er setzt den Anfang und entscheidet ob f. oder ff. folgt 


\newcounter{mystart}
\newcounter{mystop}
\newcounter{phantom}

\newcommand*\myrangeref[2]{%
  \setcounterpageref{mystart}{#1}%
  \setcounterpageref{mystop}{#2}%
  \ifnum\value{mystop}<\value{mystart}%
    \typeout{[myrangeref] Strange...stop (#2) before start (#1).}%
    \pageref{#2}--\pageref{#1}%
  \else
    \pageref{#1}%
    \ifnum\value{mystart}<\value{mystop}%
      \addtocounter{mystop}{-1}%
      \ifnum\value{mystart}<\value{mystop}%
        \,ff.
        %--\pageref{#2}%%
      \else
        \,f.
         %%--\pageref{#2}%
              \fi
    \fi
  \fi
}
            
\newcommand*\myrangerefkasten[2]{%
  \setcounterpageref{mystart}{#1}%
  \setcounterpageref{mystop}{#2}%
  \ifnum\value{mystop}<\value{mystart}%
    \typeout{[myrangeref] Strange...stop (#2) before start (#1).}%
    \pageref{#2}--\pageref{#1}%
  \else
    \makebox[12pt][r]{\pageref{#1}}%
    \ifnum\value{mystart}<\value{mystop}%
      \addtocounter{mystop}{-1}%
      \ifnum\value{mystart}<\value{mystop}%
        --\pageref{#2}%%
      \else
         --\pageref{#2}%
         % alternativ hierher: f.
      \fi
    \fi
  \fi
}


\newcommand*\mylabel[1]{%
  \refstepcounter{phantom}%
  \label{#1}%
}

\newenvironment{anhang}{\vspace{1cm}
}{}

\emfontdeclare{\itshape}

%% RAHMEN SEITLICH

\newlength{\leftbarwidth}
\setlength{\leftbarwidth}{3pt}
\newlength{\leftbarsep}
\setlength{\leftbarsep}{10pt}

\renewenvironment{leftbar}[1][\hsize]
{% 
\def\FrameCommand 
{%
{\hspace{-7pt} \color{black} \vrule width 0.5pt}%
\hspace{0pt}%must no space.
\fboxsep=\FrameSep\colorbox{white}%
}%
\MakeFramed{\hsize#1\advance\hsize-\width\FrameRestore}%
}
{\endMakeFramed}
\setlength{\FrameSep}{5pt}

\newmdenv[topline=false, leftline=true, rightline=true, bottomline=false,%
  linewidth=0.5pt, leftmargin=30pt, rightmargin=30pt, %
  skipabove=8pt, skipbelow=8pt]{mdbar}

% Überstreichung (OVERLINE)

\makeatletter
\newcommand*{\textoverline}[1]{$\overline{\hbox{#1}}\m@th$}
\makeatother

% Rahmen für Hintergrundfarbe
\fboxsep0mm

% Befehl für gekürzte Texte

\newcommand{\kuerzung}{, Auszug}

% Verse 

\setlength{\stanzaindentbase}{20pt} %Play with it later.
\setstanzaindents{5,1,1}
\setcounter{stanzaindentsrepetition}{2}
\newcommand{\stanzaend}{\&}
\sethangingsymbol{\protect\hfill}
\AtEveryStopStanza{\vspace{0.25\baselineskip}} %Abstand zwischen Strophen


% Versuch eines Grid

\RedeclareSectionCommand[
  beforeskip=3\baselineskip,
  afterskip=\baselineskip
]{chapter}
\RedeclareSectionCommand[
  beforeskip=2\baselineskip,
  afterskip=\baselineskip
]{section}

\newcommand\adjacent[2][]{%
  \bgroup
  \RedeclareSectionCommand[
    beforeskip=2\baselineskip,
    afterskip=\baselineskip,
  ]{chapter}%
  \if\relax\detokenize{#1}\relax
    \addchap{#2}%
  \else
    \addchap[#1]{#2}%
  \fi
  \egroup
  \section
}


%change the part format in table of contents
\renewcaptionname{ngerman}{\contentsname}{Inhalt} 


% Inhaltsverzeichnis

\AtBeginDocument{%
  \addtocontents{toc}{\protect\label{toc}}%
}

\renewcaptionname{ngerman}{\contentsname}{Verzeichnis der Dokumente} 
 
 
   \DeclareTOCStyleEntry[
  beforeskip=15pt,
  entryformat=\normalsize\normalfont\centering,
  pagenumberformat=\nullfont,
  linefill={},
  raggedentrytext=true
]{part}{part}

  \DeclareTOCStyleEntry[
  beforeskip=5pt,
  entryformat=\normalsize\normalfont\centering,
  pagenumberformat=\nullfont,
  linefill={},
  raggedentrytext=true
]{chapter}{chapter}

\DeclareTOCStyleEntry[
  onstarthigherlevel=\vspace*{0.5\baselineskip}\nobreak,
  indent=0pt,
  entryformat=\normalsize\def\autodot{.},
  pagenumberformat=\normalsize,
  raggedentrytext=true
]{section}{section}



 
% Das folgende auskommentiert, funktionierte nicht mehr, ging aber in Bahr/Schnitzler. Sollte eigentlich dazu dienen, beim Inhaltsverzeichnis die Nummern rechtsbündig zu setzen

 \iffalse
 
  \DeclareTOCStyleEntry[
  beforeskip=5pt,
  entryformat=\normalsize\normalfont\centering,
  pagenumberformat=\nullfont,
  linefill={},
  raggedentrytext=true
]{chapter}{chapter}

\DeclareTOCStyleEntry[
  onstarthigherlevel=\vspace*{0.5\baselineskip}\nobreak,
  indent=0pt,
  entryformat=\normalsize\def\autodot{.},
  pagenumberformat=\normalsize,
  raggedentrytext=true
]{section}{section}
 
 
  \newcommand*\sectionnumberbox[1]{\hfill #1\hspace{.6em}}

\newlength{\zweiziffern}
\newlength{\dreiziffern}
\newlength{\vierziffern}
\settowidth{\zweiziffern}{9999}
\settowidth{\dreiziffern}{99999}
\settowidth{\vierziffern}{99999999}
 
\BeforeStartingTOC[toc]{\value{tocdepth}=\sectiontocdepth}


\DeclareTOCStyleEntry[
  onstarthigherlevel=\vspace*{0.5\baselineskip}\nobreak,
  indent=0pt,
  entryformat=\normalsize\def\autodot{.},
  entrynumberformat=\sectionnumberbox,
  pagenumberformat=\normalsize,
  numwidth=\zweiziffern,
  raggedentrytext=true
]{section}{section}

\newcommand{\toccheck}{\ifnum \value{section}=76 \addtocontents{toc}{\protect\DeclareTOCStyleEntry[numwidth=\dreiziffern]{section}{section}} \else \ifnum \value{section}=990 \addtocontents{toc}{\protect\DeclareTOCStyleEntry[numwidth=\vierziffern]{section}{section}} \fi \fi}
\fi



% Längen für Tabellen
\newlength{\longeste}
\newlength{\longestz}
\newlength{\longestd}
\newlength{\longestv}
\newlength{\longestf}

\newcommand\halbtextwidth{0.9\textwidth}

\newcommand\pwindex[1]{{\sindex[pw]{#1}}}
\newcommand\oindex[1]{{\sindex[pw]{#1}}}
\newcommand\orgindex[1]{{\sindex[pw]{#1}}}

\renewcommand\oindex[1]{{{\sindex[pw]{#1}}}}
\renewcommand\orgindex[1]{{{\sindex[pw]{#1}}}}



% INDEX

%\renewcommand\pwindex[1]{}
%\renewcommand\oindex[1]{}
%\renewcommand\orgindex[1]{}
%\renewcommand\buchabdruck[1]{}


\newcommand\url[1]{\mbox{#1}}
\renewcommand\ngermanhyphenmins{33}

\makeatletter
\newcommand*{\geminationm}{$\overline{\hbox{m}}\m@th$}
\newcommand*{\geminationn}{$\overline{\hbox{n}}\m@th$}
\makeatother

%part
\renewcommand{\partmarkformat}{}
\renewcommand{\partheadmidvskip}{\enskip}
\renewcommand{\partformat}{}
\setkomafont{partnumber}{\usekomafont{part}}


%\geometry{headsep=8pt} % Abstand Kopfzeile - Text
%% DOKUMENT

\begin{document}

% Section ohne Nummer
\renewcommand*{\raggedsection}{%
 \CenteringLeftskip=1cm plus 1em\relax 
 \CenteringRightskip=1cm plus 1em\relax 
 \Centering\normalsize}



\widowpenalty=10000         % avoid widows
\clubpenalty=10000          % avoid orphans

\sloppy
\setlength{\parindent}{0em}

\setlength{\ledlsnotewidth}{4cm}
\setlength{\ledrsnotewidth}{4cm}
\renewcommand*{\ledlsnotefontsetup}{\scriptsize\sffamily}% left
\renewcommand*{\ledrsnotefontsetup}{\scriptsize\sffamily}% left
\thispagestyle{empty} 

               \section[Paul Goldmann an Arthur Schnitzler, 27. 10. 1891]{ Paul Goldmann an Arthur Schnitzler, 27. 10. 1891}\nopagebreak\mylabel{v}\rehead{ }\normalsize\beginnumbering\briefempfaengerindex{Schnitzler, Arthur@\textsc{Schnitzler, Arthur}!zzzGoldmann, Paul@\emph{von Paul Goldmann}!1891-10-271@{27. 10. 1891}|(be} \toendnotes[C]{\smallbreak\pagebreak[2]} \Standort{DLA, A:Schnitzler, HS.NZ85.1.3162.}
\physDesc{Brief, 3 Blätter, 10 Seiten
\newline{}Handschrift: blaue Tinte, deutsche Kurrent
\newline{}Schnitzler: mit rotem Buntstift zwei Unterstreichungen }\toendnotes[C]{\smallbreak}\pstart
           \noindent{}\centering{}{\pb}\textcolor{gray}{\textbf{Dr. jur. Paul Goldmann}}\pend
           \pstart
           \noindent{}\centering{}\textcolor{gray}{\textbf{\begin{otherlanguage}{french}Correspondant de la »\textcolor{brown}{Gazette de Francfort}{}\ledrightnote{\textcolor{brown}{Frankfurter Zeitung}}«\end{otherlanguage}}}\pend
           \pstart
           \noindent{}\centering{}\textcolor{gray}{\textbf{\textcolor{pink}{\begin{otherlanguage}{french}Bruxelles, 21, rue des Plantes\end{otherlanguage}}{}\ledrightnote{\textcolor{pink}{rue des Plantes}}.}}\pend
           \pstart
           \raggedleft{}\textcolor{pink}{Brüſſel}{}\ledrightnote{\textcolor{pink}{Brüssel}}, 27.
                  October 91.\pend
           \pstart\center{}Mein lieber Arthur!\pend\pstart
           Ich entſchließe mich nicht leicht zum Schreiben an Dich, offen geſtanden. Denn ich
               komme mir vor, wie \strikeout{einer} ein läſtiger Mahner, der
               eine Gefühlsſchuld eintreiben will, zu deren Honorirung nicht mehr der nöthige
               Beſtand vorhanden iſt. Alle Symptome ſprechen mir dafür, daß das gekommen iſt, was
               kommen mußte: Daß ich für Euch ein Stück Vergangenheit geworden bin; und als ſolches
               habe ich natürlich weit hinter den Sachen Eurer Gegenwart zurückzuſtehen. Ich bin
               eine Erinnerung für einſame Sonntag Nachmittage geworden{\dotsfive}\pend
           \pstart
           Alſo einiges von mir. In \textcolor{pink}{Brüſſel}{}\ledrightnote{\textcolor{pink}{Brüssel}} geht es mir
               jetzt etwas beſſer – moraliſch wenigſtens. Ich bin den Leuten hier ein klein wenig
               näher getreten, habe {\pb}manchen lieben Menſchen,
               manche ſchöne Künſtlernatur gefunden und bin mit dem Einen oder dem Andern wenn auch
               nicht Freund, ſo doch gut bekannt geworden. \strikeout{\textcolor{gray}{×}} Sogar ein kleines Milieu junger Künſtler und Lebemänner in meinem Alter, ein
                  \label{K_L02669-1v}\edtext{\textsc{Milieu} der \textsc{Hectors} und \textsc{Gastons}}{\lemma{\textnormal{\emph{Milieu … Gastons}}}\Cendnote{\textnormal{Er dürfte sich auf die zwei verarmten
                  adeligen Lebemänner Hector de Montmeyran und Gaston de Presle aus der Komödie \emph{\textcolor{green}{Le Gendre de M. Poirier}} (1854) von \textcolor{blue}{Émile Augier} und \textcolor{blue}{Jules Sandeau} beziehen.}}}\label{K_L02669-1h}, habe ich gefunden. Am
               meiſten verkehre ich mit \textsc{\textcolor{blue}{Chainaye}{}\ledrightnote{\textcolor{blue}{Hector Chainaye}}}, dem jüngſten \textcolor{blue}{Redacteur}{}\ledrightnote{→\textcolor{blue}{Hector Chainaye}} der \textcolor{brown}{\textsc{Indépendance Belge}}{}\ledrightnote{\textcolor{brown}{L’Indépendance Belge}}: enragirter \textcolor{blue}{Wallone}{}\ledrightnote{→\textcolor{blue}{Hector Chainaye}}
               und \label{K_L02669-44v}\edtext{\textcolor{blue}{Romane}{}\ledrightnote{→\textcolor{blue}{Hector Chainaye}}}{\lemma{\textnormal{\emph{Romane}}}\Cendnote{\textnormal{»Belgique romane« ist ein Überbegriff
                  für mehrere Dialekte. Der bedeutendste ist der wallonische.}}}\label{K_L02669-44h}, reiches
               künſtleriſches Sentiment, \textcolor{blue}{Stimmungsmenſch}{}\ledrightnote{→\textcolor{blue}{Hector Chainaye}}, melancholiſches Talent, \textcolor{blue}{Verfaſſer}{}\ledrightnote{→\textcolor{blue}{Hector Chainaye}} myſtiſch-empfindſamer \label{K_L02669-3v}\edtext{Gedichte in Proſa}{\lemma{\textnormal{\emph{Gedichte in Proſa}}}\Cendnote{\textnormal{Prosagedichte \textcolor{blue}{Hector Chainaye}s finden sich
                  zum Beispiel in seinem Band \emph{\textcolor{green}{L’Âme des choses}}
                     (1935). Viele der darin enthaltenen Gedichte wurden bereits
                  zwischen 1886 und 1888 in Zeitschriften wie \emph{\textcolor{green}{La Wallonie}}, \emph{\textcolor{green}{La Basoche}} und \emph{\textcolor{green}{La Jeune Belgique}}
                  veröffentlicht.}}}\label{K_L02669-3h}, blond, krank, \strikeout{ſ}
               geiſtſprühend und luſtig in der Converſation bei dem Allen und – was das beſte iſt –
               mit einigen \strikeout{k\textcolor{gray}{l}} Zügen, die entfernt an Dich erinnern. Nach Beſiegung des Deutſchenhaſſes, der
               Verſtändigungsſchwierigkeiten, des Mißtrauens gegen den Fremden \textsc{etc. etc.} bin ich ihm näher getreten. Und in dieſe\substVorne{}\textsuperscript{m}\substDazwischen{}n\substHinten{}{ }{\pb}Tagen ſtehe ich ihm rathend zur Seite bei einem
               großen Bruch mit ſeiner \label{K_L02669-2v}\edtext{\textcolor{blue}{Maitreſſe}{}\ledrightnote{→\textcolor{blue}{?? [Partnerin von Hector Chainaye, 1891]}}}{\lemma{\textnormal{\emph{Maitreſſe}}}\Cendnote{\textnormal{nicht identifiziert}}}\label{K_L02669-2h}, die ſich zu
               tödten droht \textsc{etc. etc.} (ſiehe \label{K_L02669-4v}\edtext{\textsc{\textcolor{blue}{Jeannette}{}\ledrightnote{\textcolor{blue}{Jeanette Heeger}}}}{\lemma{\textnormal{\emph{Jeannette}}}\Cendnote{\textnormal{\textcolor{blue}{Jeannette Heeger}, Geliebte \textcolor{blue}{Schnitzler}s, unternahm am 18. 12. 1889 einen
                  Suizidversuch mit einer Pistole.}}}\label{K_L02669-4h}.) Ein närriſches Ding, das Leben, – nicht
               wahr? Außerdem haben ſich meine Beziehungen zu den \textcolor{pink}{Brüſſel}{}\ledrightnote{\textcolor{pink}{Brüssel}}er Journaliſten ſichtlich verbeſſert. Es iſt ein geradezu enormer
               Unterſchied zwiſchen den \textcolor{pink}{Brüſſel}{}\ledrightnote{\textcolor{pink}{Brüssel}}er und den \textcolor{pink}{Wien}{}\ledrightnote{\textcolor{pink}{Wien}}er Collegen. Hier ſind es – von wenigen
               Ausnahmen abgeſehen – liebe, gute Burſchen mit prächtigem Benehmen, voll Gefälligkeit
               und Liebenswürdigkeit, und manch’ eine ſchöne Künftlernatur iſt auch hier darunter –
               Leute, die den Journalismus machen, um Brod zu verdienen, aber im Übrigen \label{K_L02669-5v}\edtext{\textsc{\begin{otherlanguage}{french}s’en fichent\end{otherlanguage}}}{\lemma{\textnormal{\emph{s’en fichent}}}\Cendnote{\textnormal{französisch: sich nicht kümmern}}}\label{K_L02669-5h}
               und warmen Herzens der Kunſt anhängen. Ich mache hier eifrige Propaganda für die
                  \label{K_L02669-11v}\edtext{\textcolor{pink}{Norweger}{}\ledrightnote{\textcolor{pink}{Norwegen}}}{\lemma{\textnormal{\emph{Norweger}}}\Cendnote{\textnormal{Gemeint sein dürfte vor allem \textcolor{blue}{Henrik Ibsen}, eventuell auch \textcolor{blue}{Knut Hamsun}. In der im Folgenden erwähnten
                     \textcolor{green}{Zeitungsmeldung} von \textcolor{blue}{Charles Tardieu} wird allgemein von der \textcolor{blue}{Ibsen}-Schule gesprochen und vor allem der
                     \textcolor{blue}{Schwede}{ }\textcolor{blue}{August Strindberg} behandelt.}}}\label{K_L02669-11h}, und
                  \textsc{\textcolor{blue}{Tardieu}{}\ledrightnote{\textcolor{blue}{Charles Tardieu}}}, der \textcolor{blue}{Chefredacteur}{}\ledrightnote{→\textcolor{blue}{Charles Tardieu}} der
                  \textsc{\textcolor{brown}{Indépendance}{}\ledrightnote{\textcolor{brown}{L’Indépendance Belge}}}, der unter den intereſſanten {\pb}hieſigen \strikeout{\textcolor{gray}{S}} Collegen vielleicht der intereſſanteſte iſt, hat dieſe meine Bemühungen ſammt
               Citat meines Namens in der \textsc{\textcolor{green}{Indép.}{}\ledrightnote{\textcolor{green}{L’Indépendance Belge}}}{ }\label{K_L02669-6v}\edtext{verewigt}{\lemma{\textnormal{\emph{verewigt}}}\Cendnote{\textnormal{\textcolor{blue}{Charles Tardieu}: \emph{\textcolor{green}{Théâtres et beaux-arts}}. In: \emph{\textcolor{green}{L’Indépendance Belge}}, Jg. 62, H. 281,
                        8. 10. 1891, Abendausgabe, S. 3: »\begin{otherlanguage}{french}Voilà qui nous mène en \textcolor{pink}{Scandinavie} et de là à \textcolor{pink}{Berlin}
                        et \textcolor{pink}{Munich}, où l’école \textcolor{blue}{ibsén}ienne a un public enthousiaste.
                        Mais que parlons-nous encore d’\textcolor{blue}{Ibsen}?
                           L’\textcolor{blue}{auteur} du \textcolor{green}{\emph{Canard sauvage}} est absolument distancé dans son \textcolor{pink}{pays}. Novateur et réformateur en Allemagne et en
                        France, il est déjà ›vieux jeu‹ dans sa \textcolor{pink}{Norvège}. Notre \textcolor{blue}{confrère} de la \textcolor{brown}{\emph{Gazette de Francfort}}, le docteur \textcolor{blue}{Goldmann}, très au
                        courant des curiosités et nouveautés littéraires, nous expliquait cela
                        dernièrement, et il nous prédisait le prochain avènement d’\textcolor{blue}{Auguste Strindberg}, un \textcolor{blue}{dramaturge}{ }\textcolor{pink}{suédois} et \textcolor{blue}{niet{[}z{]}schien}. \textcolor{pink}{Suédois}? vous comprenez. Mais pour ›\textcolor{blue}{niet{[}z{]}schien}‹
                        sachez que \textcolor{blue}{Frédéric
                              Niet{[}z{]}sche} est, comme eût dit \textcolor{blue}{Stendhal}, ›l’expression la plus
                        récente‹ de la philosophie allemande. Or, voici que la prédiction se
                        vérifie. Le \textcolor{brown}{Théâtre Libre} de
                           \textcolor{pink}{Berlin} et celui de \textcolor{pink}{Munich} monteront cet hiver \textcolor{green}{\emph{Mademoiselle Julie}}, de M. \textcolor{blue}{Auguste Strindberg}, une
                           \textcolor{green}{tragédie}
                        naturaliste à trois personnages, en un acte et une nuit. En deux mots Mlle
                        Julie, hystérique par atavisme, est amoureuse du domestique de son père.
                        Elle fait littéralement le siège du valet qui lutte et-succombe. Tous deux
                        se préparent à s’enfuir. Mais la cuisinière raisonne les deux amants, les
                        rappelle au sentiment des convenances sociales, et, ma foi, réussit à les
                        calmer. La toile tombe sur une rupture, definitive, espérons-le. Il est
                        probable que l’analyse des caractères ajoute à l'intérêt de cette donnée,
                        déjà séduisante par elle même. De quoi s’agit-il après tout? D'un accident.
                        A quoi bon se troubler et déranger sa vie pour si peu de chose? Christine
                        est dans le vrai. On voit bien qu'elle sait l'art d'accommoder les
                        restes.\end{otherlanguage}«}}}\label{K_L02669-6h}, worauf dann die \textcolor{green}{Notiz}{}\ledrightnote{→\textcolor{green}{Théâtres et beaux-arts}} mit »\label{K_L02669-7v}\edtext{\textcolor{green}{\textsc{\begin{otherlanguage}{french}notre \textcolor{blue}{confrère}{}\ledrightnote{→\textcolor{blue}{Paul Goldmann}} le docteur \textcolor{blue}{Goldmann}{}\ledrightnote{\textcolor{blue}{Paul Goldmann}} de le \textcolor{brown}{Gazette de Francfort}{}\ledrightnote{\textcolor{brown}{Frankfurter Zeitung}}\end{otherlanguage}}}{}\ledrightnote{→\textcolor{green}{Théâtres et beaux-arts}}}{\lemma{\textnormal{\emph{notre … Francfort}}}\Cendnote{\textnormal{französisch: unser Kollege Dr. Goldmann
                  von der \emph{\textcolor{brown}{Frankfurter Zeitung}}}}}\label{K_L02669-7h}« die Runde durch die \textcolor{pink}{Pariſ}{}\ledrightnote{\textcolor{pink}{Paris}}er Preſſe, vom
                  \label{K_L02669-8v}\edtext{\textsc{\textcolor{green}{Figaro}{}\ledrightnote{\textcolor{green}{Le Figaro}}}}{\lemma{\textnormal{\emph{Figaro}}}\Cendnote{\textnormal{\textcolor{blue}{Georges Boyer}: \emph{\textcolor{green}{Courrier des Théâtres}}. In: \emph{\textcolor{green}{Le Figaro}}, Jg. 37, H. 286, 13. 10. 1891,
                     S. 3.}}}\label{K_L02669-8h} bis zum \label{K_L02669-666v}\edtext{\textsc{\textcolor{brown}{Rappel}{}\ledrightnote{\textcolor{brown}{Le Rappel}}}}{\lemma{\textnormal{\emph{Rappel}}}\Cendnote{\textnormal{nicht nachgewiesen}}}\label{K_L02669-666h}, gemacht
               hat. Auch d\substVorne{}\textsuperscript{\textcolor{gray}{ie}}\substDazwischen{}er\substHinten{} Verkehr \substVorne{}\textsuperscript{zu\textcolor{gray}{r}}\substDazwischen{}mit der\substHinten{} officiellen Welt iſt angenehm. Ich werde von mehreren Miniſtern mit allen
               meinem Range gebührenden Ehren empfangen \textsc{etc.} Außerdem iſt
               die \textcolor{pink}{Stadt}{}\ledrightnote{→\textcolor{pink}{Brüssel}} mit ihrem \substVorne{}\textsuperscript{Schein}{\allowbreak}\substDazwischen{}Abglanz\substHinten{} franzöſiſchen Kunſtlebens recht intereſſant, und es gibt ſchöne Abende im
               Theater und im Concert. Endlich das herrliche Hiſtoriſche. Die alte \textcolor{pink}{niederländiſche}{}\ledrightnote{\textcolor{pink}{Niederlande}} Malerei. Ich beginne hier langſam zu
               begreifen, was das für Dinger ſind, die \textsc{\textcolor{blue}{Rubens}{}\ledrightnote{\textcolor{blue}{Peter Paul Rubens}}}, \textsc{\textcolor{blue}{van Dyck}{}\ledrightnote{\textcolor{blue}{Anthonis van Dyck}}} und \textsc{\textcolor{blue}{Rembrandt}{}\ledrightnote{\textcolor{blue}{Rembrandt van Rijn}}}. Und das iſt ein Quell neuer und {\pb}ungeahnter
               Genüſſe.\pend
           \pstart
           Das ſind die guten Seiten. Aber die böſen ſind geblieben, ſind vielleicht noch
               troſtloſer als zuvor, und haben nur die Geſichter zum Theil gewechſelt. Keine
               Zukunft, keine Zukunft. Die Möglichkeit, ſich ein Vermögen zu machen, exiſtirt nicht.
               Mein Gehalt iſt jämmerlich und wird nicht geſteigert. Die großen \label{K_L02669-999v}\edtext{Pflichten, die ich gegen die \textcolor{blue}{Meinen}{}\ledrightnote{→\textcolor{blue}{Vally Rosengart}{\newline}→\textcolor{blue}{Clementine Goldmann}}}{\lemma{\textnormal{\emph{Pflichten, … Meinen}}}\Cendnote{\textnormal{siehe Paul Goldmann an Arthur Schnitzler, 27. 4. 1891}}}\label{K_L02669-999h} habe, treten immer drohender an mich heran. Und außerdem werde ich von Seiten
               des \textcolor{brown}{Blatt}{}\ledrightnote{→\textcolor{brown}{Frankfurter Zeitung}}es genau ſo gemein und
               ungerecht behandelt, wie es mir in \textcolor{pink}{Wien}{}\ledrightnote{\textcolor{pink}{Wien}} geſchehen
               – H. \textsc{\textcolor{blue}{Sonnemann}{}\ledrightnote{\textcolor{blue}{Leopold Sonnemann}}}, der \textcolor{blue}{Chef}{}\ledrightnote{→\textcolor{blue}{Leopold Sonnemann}} und \textcolor{blue}{Gebieter}{}\ledrightnote{→\textcolor{blue}{Leopold Sonnemann}}, iſt ein \strikeout{\textcolor{gray}{erbarmu}} erbarmungsloſer \textcolor{blue}{Blutſauger}{}\ledrightnote{→\textcolor{blue}{Leopold Sonnemann}}, der verlangt, daß ſich ſeine Leute zu Tode ſchinden und der
               ihnen auch {\pb}dann noch beim kleinſten Verſehen
               heftige Vorwürfe macht. Außerdem ſitzt eine \label{K_L02669-10v}\edtext{\textcolor{blue}{Canaille}{}\ledrightnote{→\textcolor{blue}{?? [Vorgesetzter Paul Goldmanns 1891]}}}{\lemma{\textnormal{\emph{Canaille}}}\Cendnote{\textnormal{Schurke, Bösewicht}}}\label{K_L02669-10h} in der \textcolor{brown}{Redaction}{}\ledrightnote{→\textcolor{brown}{Frankfurter Zeitung}}, ein \textcolor{blue}{Menſch}{}\ledrightnote{→\textcolor{blue}{?? [Vorgesetzter Paul Goldmanns 1891]}}, der mich kaum kennt, dem ich nie
               etwas gethan habe und der mich trotzdem haßt, Gott weiß warum. Er iſt zum Unglück
               mein unmittelbarer \textcolor{blue}{Vorgeſetzter}{}\ledrightnote{→\textcolor{blue}{?? [Vorgesetzter Paul Goldmanns 1891]}}, und ihm habe ich es zu danken, daß \strikeout{\textcolor{gray}{man}} meine Ernennung für den \textcolor{pink}{Pariſ}{}\ledrightnote{\textcolor{pink}{Paris}}er Poſten,
               welche im Zuge war, unterblieb, weil ich mit der \label{K_L02669-777v}\edtext{Nachricht vom Tode \textsc{\textcolor{blue}{Boulanger}{}\ledrightnote{\textcolor{blue}{Georges Boulanger}}s}}{\lemma{\textnormal{\emph{Nachricht … Boulangers}}}\Cendnote{\textnormal{\textcolor{blue}{Georges Boulanger} hatte am
                     30. 9. 1891 in \textcolor{pink}{Ixelles}
                  Suizid begangen.}}}\label{K_L02669-777h} eine Stunde ſpäter gekommen, als die officielle
               Telegraphenagentur – die \textsc{\textcolor{brown}{Agence Havas}{}\ledrightnote{\textcolor{brown}{Agence Havas}}}! Und ähnliche Schurkereien. Ich leide entſetzlich darunter und ſehne mich
               blutenden Herzens mehr als je nach Erlöſung. Ein kleines Capital und Rückkehr nach
                  \textcolor{pink}{Wien}{}\ledrightnote{\textcolor{pink}{Wien}}. Denn das iſt nach wie vor das oberſte
               Ziel meiner Wünſche. Es vergeht nach wie vor kein Tag, {\pb}wo ich nicht zehn-, zwanzigmal an Dich und die
               theure \textcolor{pink}{Stadt}{}\ledrightnote{→\textcolor{pink}{Wien}} denke. Und als das
                  \textcolor{brown}{Orcheſter der \textsc{Pompiers}}{}\ledrightnote{\textcolor{brown}{Orchestre municipal des pompiers de Bruxelles}}{ }Sonntag die Straßen mit dem \textcolor{green}{Schrammel-Marſch}{}\ledrightnote{→\textcolor{green}{Wien bleibt Wien}} durchzog, lief ich
               hinterher und wiſchte mir, wie der bekannte Vater im \textcolor{green}{Singſpiel}{}\ledrightnote{→\textcolor{green}{?? [Singspiel, in dem sich ein Vater Tränen der Rührung aus den Augen wischt]}}, die Thränen mit dem Rockärmel
               ab. Aber ich habe keine Hoffnung. Mein Leben wird in harter Sklaverei verfließen,
               fern von Allem, was ich lieb habe; und zu großen befreienden Werken habe ich weder
               das genügende Talent, noch die genügende Energie{\dotsfive}\pend
           \pstart
           Wollte ich nun alle die Fragen aufſchreiben, die ich an Dich zu richten habe, es
               ginge noch ein Briefbogen darauf. Aber ich thue es nicht; denn ich weiß, daß du mir
               ſie eh’ nicht beantworten wirſt. Der lange Brief\strikeout{,} von
               Dir, der nicht kommt, ſagt mir viel mehr, als \strikeout{ein}
               einer, der gekom{\pb}men wäre. Du haſt Recht, mein
               lieber Alter; es gibt auch in der Freundſchaft »\label{K_L02669-45v}\edtext{\textcolor{green}{Epiſoden}{}\ledrightnote{→\textcolor{green}{Episode}}}{\lemma{\textnormal{\emph{Epiſoden}}}\Cendnote{\textnormal{Anspielung auf \textcolor{blue}{Schnitzler}s Einakter \textcolor{green}{Epiſode}}}}\label{K_L02669-45h}«. Jeder verbraucht halt in ſeinem Leben eine gewiſſe Anzahl Menſchen, und von
               mir iſt nur mehr der letzte Bodenſatz vorhanden. Dir iſt kein Vorwurf zu machen. Es
               iſt die Natur, die es ſo eingerichtet, daß das Vergeſſen in der ſeeliſchen Welt genau
               ſo \strikeout{meh} mechaniſch und nothwendig und mit denſelben
               Endzwecken vor ſich geht, wie das Verdauen in der körperlichen{\dotsfour}\pend
           \pstart
           Mir brennt das Gewiſſen oft, wenn ich daran denke, daß ich \textsc{\textcolor{blue}{Loris}{}\ledrightnote{\textcolor{blue}{Hugo von Hofmannsthal}}} und \textsc{\textcolor{blue}{Richard}{}\ledrightnote{\textcolor{blue}{Richard Beer-Hofmann}}} noch nicht auf ihre Brieſe geantwortet habe. Aber mir lähmt der Gedanke die zum
               Schreiben angeſetzte Hand, daß ſie, wenn ſie meinen Brief erhalten, die Empfindung
               haben könnten: was will der Menſch eigentlich von mir? Grüße die \textcolor{blue}{Zwei}{}\ledrightnote{→\textcolor{blue}{Hugo von Hofmannsthal}{\newline}→\textcolor{blue}{Richard Beer-Hofmann}} bitte viel {\pb}tauſend Mal von mir und ſage ihnen in meinem Namen
               alles Liebe und Gute, was ſich finden läßt{\dots}\pend
           \pstart
           Deinem \textcolor{blue}{Bruder}{}\ledrightnote{→\textcolor{blue}{Julius Schnitzler}} und \textsc{\textcolor{blue}{Kapper}{}\ledrightnote{\textcolor{blue}{Friedrich Kapper}}} herzlichſte Grüße. Den Deinen ergebene Empfehlungen. Dir ſelbſt – ſchweres
               Problem. Ich möchte Dir am Liebſten meinen Segen geben, ſo abgeſchieden komme ich mir
               Dir gegenüber vor.\pend
           \pstart
           Dein {\\[\baselineskip]}treuer {\\[\baselineskip]}\spacefill\mbox{Paul Goldmann.}\pend
           \leftskip=0em{}\pstart
           \noindent{}Drei Bitten 1.) ſag’ doch dem \textcolor{blue}{Schuft}{}\ledrightnote{→\textcolor{blue}{Jaques Joachim}}, dem Dr. \textsc{\textcolor{blue}{Joachim}{}\ledrightnote{\textcolor{blue}{Jaques Joachim}}}, wenn er die ihm geſchickte kleine \textcolor{green}{Arbeit}{}\ledrightnote{→\textcolor{green}{Die drei Elixire}} nicht brauchen kann, ſo ſoll er mir
                  ſie augenblicklich zurückſenden, weil ich Verwendung {\pb}dafür habe; auch ſoll er mir dasjenige \label{K_L02669-12v}\edtext{Heft der »\textcolor{green}{Modernen \uline{Dichtung}}{}\ledrightnote{\textcolor{green}{Moderne Dichtung. Monatsschrift für Literatur und Kritik}}}{\lemma{\textnormal{\emph{Heft … Dichtung}}}\Cendnote{\textnormal{\textcolor{blue}{Paul Goldmann}: \emph{\textcolor{green}{Was einem so einfällt}}. In: \emph{\textcolor{green}{Moderne Dichtung}}, Jg. 1, Bd. 2, H. 1,
                        S. 521–522.}}}\label{K_L02669-12h}« (\label{K_L02669-43v}\edtext{nicht \textcolor{green}{Rundſchau}{}\ledrightnote{\textcolor{green}{Moderne Rundschau}}}{\lemma{\textnormal{\emph{nicht Rundſchau}}}\Cendnote{\textnormal{\textcolor{blue}{Paul Goldmann}: \emph{\textcolor{green}{Nämlich}}. In: \emph{\textcolor{green}{Moderne
                           Rundschau}}, Jg. 1, Bd. 3, H. 1, 1. 4. 1891,
                     S. 34.}}}\label{K_L02669-43h}) ſchicken, in dem \textcolor{green}{Aphorismen}{}\ledrightnote{→\textcolor{green}{Nämlich}} von mir erſchienen ſind; ich brauche ſie
                  dringend und zahle \strikeout{e\textcolor{gray}{n}} eventuell dem Buchhändler dafür 2.) haſt Du eine Ahnung, was zwiſchen \textsc{\strikeout{Herz}}{ }\textsc{\textcolor{blue}{Herzl}{}\ledrightnote{\textcolor{blue}{Theodor Herzl}}} und ſeiner \textcolor{blue}{Frau}{}\ledrightnote{→\textcolor{blue}{Julie Herzl}}{ }\label{K_L02669-13v}\edtext{vorgegangen}{\lemma{\textnormal{\emph{vorgegangen}}}\Cendnote{\textnormal{Möglicherweise hörte \textcolor{blue}{Goldmann} von der Ehekrise der \textcolor{blue}{Herzl}s. \textcolor{blue}{Theodor Herzl} teilte seinem \textcolor{blue}{Schwiegervater} im Mai 1891
                     mit, dass er die Scheidung wolle. \textcolor{blue}{Julie
                        Herzl}, mit der \textcolor{blue}{Theodor Herzl} bis
                     zu seinem Tod verheiratet blieb, war zu dieser Zeit schwanger. XXXX
                        Literaturangabe: Briefe, Bd. 1?}}}\label{K_L02669-13h}? 3.) Weißt Du vielleicht – nicht lachen, bitte! – den Namen einer \strikeout{T}{ }\uline{guten}{ }\strikeout{Tr\textcolor{gray}{u}} Truppe \textcolor{pink}{Tirol}{}\ledrightnote{\textcolor{pink}{Tirol}}er Sänger, \introOben{}an\introOben{} welche man ſich wenden könnte, um ſie zu einer Reiſe
                  nach \textcolor{pink}{Brüſſel}{}\ledrightnote{\textcolor{pink}{Brüssel}} zu veranlaſſen?\pend
           \endnumbering\briefempfaengerindex{Schnitzler, Arthur@\textsc{Schnitzler, Arthur}!zzzGoldmann, Paul@\emph{von Paul Goldmann}!1891-10-271@{27. 10. 1891}|)be}\mylabel{h}\begin{anhang}\end{anhang}
         \normalsize

\newenvironment{esempio}[3]%
{
    \vspace{1.5ex}
    \rlap{\underline{#1}}
    \par
    \setlength{\parindent}{0cm}
    \nopagebreak
    \leftskip=#2cm
    \rightskip=#3cm
}
{
    \par
}

\doendnotes{C}
\bigskip

\printindex[pw]


\end{document}
      