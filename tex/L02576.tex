%% latex-korrekturansicht-vorspann.tex
%% Vorspann für die Korrekturansicht.
%% Lädt die gemeinsame Datei latex-vorspann.tex mit gesetztem Schalter.

\newif\ifkorrekturansicht
\korrekturansichttrue

\input{../tex-inputs/latex-vorspann}


               \section[Therese Rie-Andro an Arthur Schnitzler, 30. 9. 1923]{ Therese Rie-Andro an Arthur Schnitzler, 30. 9. 1923}\nopagebreak\mylabel{v}\rehead{ }\normalsize\beginnumbering\briefempfaengerindex{Schnitzler, Arthur@\textsc{Schnitzler, Arthur}!zzzRie, Therese@\emph{von Therese Rie}!1923-09-301@{30. 9. 1923}|(be} \toendnotes[C]{\smallbreak\pagebreak[2]} \Standort{CUL, Schnitzler, B 658.}
\physDesc{Brief, 1 Blatt, 1 Seite
\newline{}Schreibmaschine
\newline{}Handschrift: blaue Tinte, lateinische Kurrent (\noindent{}eine Unterstreichung, Grußformel und Unterschrift)
\newline{}Schnitzler: 1) mit Bleistift zwischen erstem
                                 und zweitem Absatz: »\textsc{\textcolor{green}{Fliederbusch}}« 2) mit rotem Buntstift beschriftet: »\textsc{Rie-Andro (\textcolor{green}{Fliederbusch})}« und fünf Unterstreichungen}\toendnotes[C]{\smallbreak}\pstart
           \raggedleft{}{\pb}\textcolor{pink}{Wien}{}\ledrightnote{\textcolor{pink}{Wien}}, 30. 9. 23\pend
           \pstart
           \raggedleft{}\textcolor{pink}{IV. Schönburgstr. 48}{}\ledrightnote{\textcolor{pink}{Schönburgstraße}}\pend
           \pstart{}Verehrter Herr Doktor,\pend\pstart
           Haben Sie sehr sehr herzlichen Dank! Ich habe mich einen ganzen Nachmittag meiner
               Lieblingsbeschäftigung hingeben können: zu lachen. Wenn Sie freilich auch das »ernste
               Lachen« mit dazu rechnen wollen, das einen überkommt, wenn man das Allzumenschliche
               blosgelegt sieht. Es ist ein sehr \uline{weises}{ }\textcolor{green}{Stück}{}\ledrightnote{→\textcolor{green}{Fink und Fliederbusch. Komödie in drei Akten}} und ich weiss jetzt genau, warum ich es damals so besonders
               liebte!\pend
           \pstart
           Ich schicke Ihnen zugleich den versprochenen \textcolor{green}{\textcolor{blue}{Rolland}{}\ledrightnote{\textcolor{blue}{Romain Rolland}}}{}\ledrightnote{→\textcolor{green}{Musikalische Reise ins Land der Vergangenheit}}; Sie hätten ihn längst bekommen, aber ich wusste, dass Sie verreist waren.
               Hoffentlich interessiert er Sie – umsomehr, als Sie, wie \textcolor{blue}{Stefan Zweig}{}\ledrightnote{\textcolor{blue}{Stefan Zweig}} mir in \textcolor{pink}{Salzburg}{}\ledrightnote{\textcolor{pink}{Salzburg}}
               erzählte, mit \textcolor{blue}{Rolland}{}\ledrightnote{\textcolor{blue}{Romain Rolland}} dort \label{K_L02576-1v}\edtext{zusammen waren}{\lemma{\textnormal{\emph{zusammen waren}}}\Cendnote{\textnormal{siehe A. S.: \emph{Tagebuch}, 3. 8. 1923}}}\label{K_L02576-1h}. Ein paar Aufsätze finde ich ja ein bischen langweilig, aber der \textcolor{green}{\textcolor{blue}{Händel}{}\ledrightnote{\textcolor{blue}{Georg Friedrich Händel}}}{}\ledrightnote{→\textcolor{green}{Musikalische Reise ins Land der Vergangenheit}} ist ergreifend schön für meinen Geschmack. Auch \textcolor{green}{\textcolor{blue}{Metastasio}{}\ledrightnote{\textcolor{blue}{Pietro Metastasio}}}{}\ledrightnote{→\textcolor{green}{Musikalische Reise ins Land der Vergangenheit}} mit seinem ganz modernen Musikdramatiker-Empfinden hat mir sehr gefallen und
               der musikwütige \textcolor{pink}{Engländer}{}\ledrightnote{\textcolor{pink}{England}}, der Musik so sehr
               vergöttert und so ungern bezahlt, ist auch nicht schlecht.\pend
           \pstart
           – – Sonderbar ist mirs immer, dass \textcolor{blue}{Rolland}{}\ledrightnote{\textcolor{blue}{Romain Rolland}} sich
               um \textcolor{blue}{J. S. Bach}{}\ledrightnote{\textcolor{blue}{Johann Sebastian Bach}} jedes Mal mit ein paar bewundernden
               Worten herumdrückt; aber ihm nie recht in die Nähe will. Vielleicht gibts da trotz
               allem doch nationale Schranken – oder er hat die \textcolor{green}{Hmoll-Messe}{}\ledrightnote{\textcolor{green}{h-Moll-Messe}} nie ordentlich gehö\textcolor{gray}{rt}\pend
           \pstart
           Nochmals herzlichsten Dark und viele Grüsse1\pend
           \pstart
           {[}hs.:{]} \textcolor{gray}{Ihre}{\\[\baselineskip]}\spacefill\mbox{Therese Rie.}\pend
           \leftskip=0em{}\endnumbering\briefempfaengerindex{Schnitzler, Arthur@\textsc{Schnitzler, Arthur}!zzzRie, Therese@\emph{von Therese Rie}!1923-09-301@{30. 9. 1923}|)be}\mylabel{h}  \normalsize

\doendnotes{C}
\bigskip
\vfill

\clearpage

\footnotesize

\lohead{\textsc{register}}

% Definiere theindex-Environment komplett neu ohne reledmac
\makeatletter
\renewenvironment{theindex}{%
  \section*{\indexname}%
  \setlength{\parindent}{0pt}%
  \setlength{\parskip}{0pt plus 0.3pt}%
  \let\item\@idxitem
}{%
  \clearpage
}
\makeatother

\IfFileExists{\jobname-pw.ind}{\input{\jobname-pw.ind}}{}

\end{document}

      