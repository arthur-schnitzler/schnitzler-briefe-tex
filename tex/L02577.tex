%% latex-korrekturansicht-vorspann.tex
%% Vorspann für die Korrekturansicht.
%% Lädt die gemeinsame Datei latex-vorspann.tex mit gesetztem Schalter.

\newif\ifkorrekturansicht
\korrekturansichttrue

\input{../tex-inputs/latex-vorspann}


               \section[Therese Rie-Andro an Arthur Schnitzler, 6. 1. 1928]{ Therese Rie-Andro an Arthur Schnitzler, 6. 1. 1928}\nopagebreak\mylabel{v}\rehead{ }\normalsize\beginnumbering\briefempfaengerindex{Schnitzler, Arthur@\textsc{Schnitzler, Arthur}!zzzRie, Therese@\emph{von Therese Rie}!1928-01-061@{6. 1. 1928}|(be} \toendnotes[C]{\smallbreak\pagebreak[2]} \Standort{CUL, Schnitzler, B 658.}
\physDesc{Brief, 1 Blatt, 1 Seite
\newline{}Handschrift: blaue Tinte, lateinische Kurrent
\newline{}Schnitzler: 1) mit Bleistift beschriftet: »\textsc{Andro}« 2) mit rotem Buntstift eine Unterstreichung}\toendnotes[C]{\smallbreak}\pstart
           \raggedleft{}{\pb}\textcolor{pink}{Wien}{}\ledrightnote{\textcolor{pink}{Wien}}, Dreikönig
                     1928.\pend
           \pstart
           \raggedleft{}\textcolor{pink}{IV, Schönburgstr. 48}{}\ledrightnote{\textcolor{pink}{Schönburgstraße}}.\pend
           \pstart{}Verehrter Herr Doktor,\pend\pstart
           Ich habe mich so in Ihr \label{K_L02577-1v}\edtext{\textcolor{green}{Buch}{}\ledrightnote{→\textcolor{green}{Der Geist im Wort und der Geist in der Tat}}}{\lemma{\textnormal{\emph{Buch}}}\Cendnote{\textnormal{\textcolor{blue}{Schnitzler} übersandte ihr nach dem letzten Brief \emph{\textcolor{green}{Der Geist im Wort und der Geist in der Tat}}.}}}\label{K_L02577-1h} verlesen,
               daſs ich vergessen habe, Ihnen zu danken – und es war doch so lieb von Ihnen! So darf
               ich Ihnen heute zweimal Dank sagen: einmal für Ihre Freundlichkeit und dann dafür,
               dass Sie den Unterschied zwischen Kontinualiſchem und Aktualiſchem (in allen Formen)
               so aufgezeigt haben, wie noch niemand vorher.\pend
           \pstart
           Ihre{\\[\baselineskip]}\spacefill\mbox{Therese Rie – Andro.}\pend
           \leftskip=0em{}\endnumbering\briefempfaengerindex{Schnitzler, Arthur@\textsc{Schnitzler, Arthur}!zzzRie, Therese@\emph{von Therese Rie}!1928-01-061@{6. 1. 1928}|)be}\mylabel{h}  \normalsize

\doendnotes{C}
\bigskip
\vfill

\clearpage

\footnotesize

\lohead{\textsc{register}}

% Definiere theindex-Environment komplett neu ohne reledmac
\makeatletter
\renewenvironment{theindex}{%
  \section*{\indexname}%
  \setlength{\parindent}{0pt}%
  \setlength{\parskip}{0pt plus 0.3pt}%
  \let\item\@idxitem
}{%
  \clearpage
}
\makeatother

\IfFileExists{\jobname-pw.ind}{\input{\jobname-pw.ind}}{}

\end{document}

      