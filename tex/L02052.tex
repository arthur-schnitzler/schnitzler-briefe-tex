%% latex-korrekturansicht-vorspann.tex
%% Vorspann für die Korrekturansicht.
%% Lädt die gemeinsame Datei latex-vorspann.tex mit gesetztem Schalter.

\newif\ifkorrekturansicht
\korrekturansichttrue

\input{../tex-inputs/latex-vorspann}


               \section[Stefan Großmann an Arthur Schnitzler, 5. 2. 1912]{ Stefan Großmann an Arthur Schnitzler, 5. 2. 1912}\nopagebreak\mylabel{v}\rehead{ }\normalsize\beginnumbering\briefempfaengerindex{Schnitzler, Arthur@\textsc{Schnitzler, Arthur}!zzzGrossmann, Stefan@\emph{von Stefan Großmann}!1912-02-051@{5. 2. 1912}|(be} \toendnotes[C]{\smallbreak\pagebreak[2]} \Standort{CUL, Schnitzler, B 34.}
\physDesc{Brief, 1 Blatt, 4 Seiten
\newline{}Schreibmaschine
\newline{}Handschrift: schwarze Tinte, deutsche Kurrent
\newline{}Schnitzler: 1) mit Bleistift beschriftet: »\textsc{Großma{\geminationn}}« 2) mit rotem Buntstift eine Unterstreichung\newline{}Ordnung: mit Bleistift von unbekannter Hand nummeriert:
                              »11« }\toendnotes[C]{\smallbreak}\pstart
           \noindent{}\centering{}{\pb}\textcolor{gray}{\textbf{\textsc{Hotel}}}\pend
           \pstart
           \noindent{}\centering{}\textcolor{gray}{\textbf{\textcolor{pink}{\textsc{Vier Jahreszeiten}}{}\ledrightnote{\textcolor{pink}{Hotel Vier Jahreszeiten}}}}\pend
           \pstart
           \noindent{}\centering{}\textcolor{gray}{\textbf{TELEGRAMM-ADRESSE: JAHRESZEITENTYP, \textcolor{pink}{MÜNCHEN}{}\ledrightnote{\textcolor{pink}{München}}.}}\pend
           \pstart
           \noindent{}\centering{}\textcolor{gray}{\textbf{Lieber’s Code – International Hôtel-Code.}}\pend
           \pstart
           \noindent{}\centering{}\textcolor{gray}{\textbf{TELEFON 23073–23076}}\pend
           \pstart
           \raggedleft{}\textcolor{gray}{\textbf{\textcolor{pink}{MÜNCHEN}{}\ledrightnote{\textcolor{pink}{München}},}}{ }5 Februar 1912\pend
           \pstart
           \uline{Auf der Durchreiſe.}\pend
           \pstart
           Nachdem ich nun in \textcolor{pink}{München}{}\ledrightnote{\textcolor{pink}{München}}{ }\strikeout{\textcolor{gray}{ſa}} »\textcolor{green}{Das weite Land}{}\ledrightnote{\textcolor{green}{Das weite Land. Tragikomödie in fünf Akten}}«
               mit Hrn \textcolor{blue}{\textsc{Steinrück}}{}\ledrightnote{\textcolor{blue}{Albert Steinrück}} ſah, möchte ich Ihnen, verehrter Herr Schnitzler, – wiewohl Sie gewiſs auf
               dieſe Correctur gar kein Gewicht legen – ſagen, daſs ich nun erſt das Werk wirklich
               gefühlt habe und das verfluchte Zeitungshandwerk anklage, welches Einen zwingt, im
               Handumdrehen {\pb}ein paar \label{K_L02052_1v}\edtext{\textcolor{green}{leicht-fertige Dinge}{}\ledrightnote{→\textcolor{green}{Schnitzlers »Weites Land«. Erste Aufführung im Burgtheater}}}{\lemma{\textnormal{\emph{leicht-fertige Dinge}}}\Cendnote{\textnormal{Seine Kritik
                     fasste er am Ende der Rezension der Uraufführung (\textcolor{blue}{st. gr.}: \emph{\textcolor{green}{Schnitzlers »Weites Land«. Erste Aufführung im
                           Burgtheater}}. In: \emph{\textcolor{green}{Arbeiter-Zeitung}}, Jg. 23, Nr. 284, 15. 11. 1910,
                        S. 3–4.) zusammen: »Das Publikum nahm das übergrübelte
                           Schauspiel mit großem Interesse auf und gab sich auch den zarten,
                           eigentlich novellistischen Reizen der Dichtung mit außerordentlicher
                           Bereitwilligkeit hin. Nach jedem Akt wurde \textcolor{blue}{Schnitzler} hervorgerufen und dankte in etwas müder
                           Haltung.«}}}\label{K_L02052_1h} innerhalb einiger Stunden
               über eine Dichtung zu ſagen.\pend
           \pstart
           Durch Hrn \textcolor{blue}{\uline{Steinrück}}{}\ledrightnote{\textcolor{blue}{Albert Steinrück}} ſah ich erſt, wie viel menſchliche Stärke im \textcolor{green}{Hofreiter}{}\ledrightnote{\textcolor{green}{Das weite Land. Tragikomödie in fünf Akten}} ſteckt, wie viel ſittliche Energie bei aller Freiheit, wie viel
               Willens-training bei aller Ungebundenheit.\pend
           \pstart
           Das verdammte Geſetz der Nähe verwirrt Einen oft, ich ſah nur {\pb}das Äußerliche, die \textcolor{pink}{Wien}{}\ledrightnote{\textcolor{pink}{Wien}}er Nichtsthuer-atmosphäre, das war oberflächlich und anmaßend.\pend
           \pstart
           Es liegt mir daran, Ihnen zu ſagen, daſs ich das Werk geſtern mit einer Art Bangen
               mitgefühlt habe und einen tiefen, nicht ſchnell zu verwiſchenden Eindruck nach Hauſe
               trage.\pend
           \pstart
           Ich ſchreibe Ihnen dies mitten auf einer Forſchungsreiſe nach Talenten durch ganz \textcolor{pink}{Deutſchland}{}\ledrightnote{\textcolor{pink}{Deutschland}} und nur deshalb, {\pb}weil ich \uline{mir}
               durch dieſes Geſtändnis eine erleichterte Viertelſtunde machen will.\pend
           \pstart
           Sehr ergeben:{\\[\baselineskip]}\spacefill\mbox{Stefan Großmann}\pend
           \leftskip=0em{}\endnumbering\briefempfaengerindex{Schnitzler, Arthur@\textsc{Schnitzler, Arthur}!zzzGrossmann, Stefan@\emph{von Stefan Großmann}!1912-02-051@{5. 2. 1912}|)be}\mylabel{h}  \normalsize

\doendnotes{C}
\bigskip
\vfill

\clearpage

\footnotesize

\lohead{\textsc{register}}

% Definiere theindex-Environment komplett neu ohne reledmac
\makeatletter
\renewenvironment{theindex}{%
  \section*{\indexname}%
  \setlength{\parindent}{0pt}%
  \setlength{\parskip}{0pt plus 0.3pt}%
  \let\item\@idxitem
}{%
  \clearpage
}
\makeatother

\IfFileExists{\jobname-pw.ind}{\input{\jobname-pw.ind}}{}

\end{document}

      