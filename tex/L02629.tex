%% latex-korrekturansicht-vorspann.tex
%% Vorspann für die Korrekturansicht.
%% Lädt die gemeinsame Datei latex-vorspann.tex mit gesetztem Schalter.

\newif\ifkorrekturansicht
\korrekturansichttrue

\input{../tex-inputs/latex-vorspann}


               \section[Paul Goldmann an Arthur Schnitzler, 3. 7. 1894]{ Paul Goldmann an Arthur Schnitzler, 3. 7. 1894}\nopagebreak\mylabel{v}\rehead{ }\normalsize\beginnumbering\briefempfaengerindex{Schnitzler, Arthur@\textsc{Schnitzler, Arthur}!zzzGoldmann, Paul@\emph{von Paul Goldmann}!1894-07-031@{3. 7. {[}1894{]}}|(be} \toendnotes[C]{\smallbreak\pagebreak[2]} \Standort{DLA, A:Schnitzler, HS.NZ85.1.3164.}
\physDesc{Postkarte
\newline{}Handschrift: 1) schwarze Tinte, deutsche Kurrent\hspace{1em}2) schwarze Tinte, lateinische Kurrent (\noindent{}Adresse)\hspace{1em}\newline{}Versand: 1) Stempel: »\nobreak{}\oindex{Place de la Bourse@\textbf{Place de la Bourse}, \emph{Platz (K.PLT)}|pwk}Paris Pl. de la Bourse, \textcolor{gray}{3} Juil 94\nobreak{}«.  2) Stempel: »\nobreak{}\oindex{IX., Alsergrund@\textbf{IX., Alsergrund}, \emph{Bezirk (A.BZK)}|pwk}Wien 9/3 72, 5. 7. 94, 8.V, Bestellt\nobreak{}«. 
\newline{}Schnitzler: mit Bleistift das Datum »3/7 94« vermerkt }\toendnotes[C]{\smallbreak}\pstart{}
u                  {\pb}\textcolor{pink}{\begin{otherlanguage}{french}Autriche\end{otherlanguage}}{}\ledrightnote{\textcolor{pink}{Österreich}}.\pend{}\pstart{}Herrn Dr. Arthur Schnitzler \pend{}\pstart{}\textcolor{pink}{IX. Frankgaße 1}{}\ledrightnote{\textcolor{pink}{Frankgasse}}\pend{}\pstart{}\textcolor{pink}{Wien}{}\ledrightnote{\textcolor{pink}{Wien}}\pend{}{\bigskip}\pstart
           {\pb}\textsc{\textcolor{pink}{Paris}{}\ledrightnote{\textcolor{pink}{Paris}}}, 3. Juli.\pend
           \pstart{}Liebſter Freund,\pend\pstart
           Bitte ſchicke mir die \textcolor{pink}{Adreſſe}{}\ledrightnote{→\textcolor{pink}{Frankgasse}}
               Deines \textcolor{blue}{Bruder}{}\ledrightnote{→\textcolor{blue}{Julius Schnitzler}}s oder \strikeout{ſei} des \label{K_L02629-2v}\edtext{Locales}{\lemma{\textnormal{\emph{Locales}}}\Cendnote{\textnormal{nicht identifizert}}}\label{K_L02629-2h},
               in dem er die \label{K_L02629-1v}\edtext{Hochzeit}{\lemma{\textnormal{\emph{Hochzeit}}}\Cendnote{\textnormal{\textcolor{blue}{Julius Schnitzler} und \textcolor{blue}{Helene Altmann} heirateten am 8. 7. 1894.}}}\label{K_L02629-1h} feiert.\pend
           \pstart
           Und warum ſchreibſt Du mir nicht? Herzlichſt {\\}Dein {\\}\spacefill\mbox{P. G.}\pend
           \endnumbering\briefempfaengerindex{Schnitzler, Arthur@\textsc{Schnitzler, Arthur}!zzzGoldmann, Paul@\emph{von Paul Goldmann}!1894-07-031@{3. 7. {[}1894{]}}|)be}\mylabel{h}  \normalsize

\doendnotes{C}
\bigskip
\vfill

\clearpage

\footnotesize

\lohead{\textsc{register}}

% Definiere theindex-Environment komplett neu ohne reledmac
\makeatletter
\renewenvironment{theindex}{%
  \section*{\indexname}%
  \setlength{\parindent}{0pt}%
  \setlength{\parskip}{0pt plus 0.3pt}%
  \let\item\@idxitem
}{%
  \clearpage
}
\makeatother

\IfFileExists{\jobname-pw.ind}{\input{\jobname-pw.ind}}{}

\end{document}

      