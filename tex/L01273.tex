%% latex-korrekturansicht-vorspann.tex
%% Vorspann für die Korrekturansicht.
%% Lädt die gemeinsame Datei latex-vorspann.tex mit gesetztem Schalter.

\newif\ifkorrekturansicht
\korrekturansichttrue

\input{../tex-inputs/latex-vorspann}


               \section[Arthur Schnitzler an Richard Beer-Hofmann, 27. 2. 1903]{ Arthur Schnitzler an Richard Beer-Hofmann,
               27. 2. 1903}\nopagebreak\mylabel{v}\rehead{ }\normalsize\beginnumbering\briefempfaengerindex{Beer-Hofmann, Richard@\textsc{Beer-Hofmann, Richard}!zzzSchnitzler, Arthur@\emph{von Arthur Schnitzler}!1903-02-271@{27. 2. 1903}|(be} \toendnotes[C]{\smallbreak\pagebreak[2]} \Standort{YCGL, MSS 31.}
\physDesc{Bildpostkarte
\newline{}Handschrift: Bleistift, deutsche Kurrent\newline{}Versand: 1) Stempel: »\nobreak{}\oindex{Berlin@\textbf{Berlin}, \emph{https://www.geonames.org/ontologyP.PPLC}|pwk}Berlin W, 1. 3. 03, 8–9N\nobreak{}«.  2) Stempel: »\nobreak{}\oindex{Rodaun@\textbf{Rodaun}, \emph{Teil eines besiedelten Ortes (A.BSOX)}|pwk}R{[}odaun{]}, 3. 3. 03, \textcolor{gray}{7}{[}–8{]}V\nobreak{}«. \newline{}Ordnung: mit Bleistift von unbekannter Hand datiert: »27. 2.« }\pstart{}{\pb}\textsc{Herrn Dr Rich.
                     Beer-Hofmann}\pend{}\pstart{}\textcolor{pink}{\textsc{Rodaun}}{}\ledrightnote{\textcolor{pink}{Rodaun}}{ }\textsuperscript{b}/\textcolor{pink}{Wien}{}\ledrightnote{\textcolor{pink}{Wien}}\pend{}\pstart{}\textsc{\textcolor{pink}{Liesinger Straße 1}{}\ledrightnote{\textcolor{pink}{Liesingerstraße}}.}\pend{}{\bigskip}\pstart
           \noindent{}\centering{}{\pb}\textcolor{gray}{\textbf{Gruss aus
                     Berlin}}\pend
           \pstart
           \noindent{}\centering{}\textcolor{gray}{\textbf{Elektrische Hochbahn}}\pend
           \pstart
           \noindent{}\centering{}\textcolor{gray}{\textbf{Ueberbrückung in der \textcolor{pink}{Bülow-Strasse}{}\ledrightnote{\textcolor{pink}{Bülowstraße}}}}\pend
           \pstart
           27/2 /903\hspace*{1.5em}\textsc{\textcolor{pink}{Palast-Hotel}{}\ledrightnote{\textcolor{pink}{Palasthotel Berlin}}}\pend
           \pstart Herzlichſt Ihr \spacefill\mbox{A.}\pend{}\endnumbering\briefempfaengerindex{Beer-Hofmann, Richard@\textsc{Beer-Hofmann, Richard}!zzzSchnitzler, Arthur@\emph{von Arthur Schnitzler}!1903-02-271@{27. 2. 1903}|)be}\mylabel{h}  \normalsize

\doendnotes{C}
\bigskip
\vfill

\clearpage

\footnotesize

\lohead{\textsc{register}}

% Definiere theindex-Environment komplett neu ohne reledmac
\makeatletter
\renewenvironment{theindex}{%
  \section*{\indexname}%
  \setlength{\parindent}{0pt}%
  \setlength{\parskip}{0pt plus 0.3pt}%
  \let\item\@idxitem
}{%
  \clearpage
}
\makeatother

\IfFileExists{\jobname-pw.ind}{\input{\jobname-pw.ind}}{}

\end{document}

      