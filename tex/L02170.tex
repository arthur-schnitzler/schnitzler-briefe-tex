%% latex-korrekturansicht-vorspann.tex
%% Vorspann für die Korrekturansicht.
%% Lädt die gemeinsame Datei latex-vorspann.tex mit gesetztem Schalter.

\newif\ifkorrekturansicht
\korrekturansichttrue

\input{../tex-inputs/latex-vorspann}


               \section[Bertha von Suttner an Arthur und Olga Schnitzler, 30. 3. 1914]{ Bertha von Suttner an Arthur und Olga Schnitzler,
                    30. 3. 1914}\nopagebreak\mylabel{v}\rehead{ }\normalsize\beginnumbering\briefempfaengerindex{Schnitzler, Olga@\textsc{Schnitzler, Olga}!zzzSuttner, Bertha von@\emph{von Bertha von Suttner}!1914-03-302@{30. 3. 1914}|(be}\briefempfaengerindex{Schnitzler, Arthur@\textsc{Schnitzler, Arthur}!zzzSuttner, Bertha von@\emph{von Bertha von Suttner}!1914-03-302@{30. 3. 1914}|(be} \toendnotes[C]{\smallbreak\pagebreak[2]} \Standort{DLA, A:Schnitzler, HS.NZ66.198.}
\physDesc{Brief, 1 Blatt (mit Krone in Golddruck), 1 Seite
\newline{}Handschrift: schwarze Tinte, deutsche Kurrent
\newline{}Schnitzler: 1) mit Bleistift beschriftet: »\textsc{Suttner}« 2) mit rotem Buntstift eine Unterstreichung}\Standort{DLA, A:Schnitzler, HS.NZ85.1.4773.}
\physDesc{1 Blatt, 1 Seite, maschinelle Abschrift}\toendnotes[C]{\smallbreak}\pstart
           \raggedleft{}{\pb}30/III 1914\pend
           \pstart\center{}Geehrter Dichter und liebe Dichtersgattin\pend\pstart
           Das war mir u. noch jemand anders eine herbe Enttäuſchung geſtern: zuerſt zu- und
                    dann abgeſagt! Das müſſen Sie wieder gutmachen. Eine Dame kam \uline{nur}, weil ſie ſich ſo ſehr auf Ihr in Ausſicht
                    geſtelltes Erſcheinen \strikeout{ſo} freute. Und ſie nahm
                    mir das Verſprechen ab ſie bei der nächſten Gelegenheit wieder zu rufen. Es iſt
                    die \label{K_L02170_1v}\edtext{Pr.}{\lemma{\textnormal{\emph{Pr.}}}\Cendnote{\textnormal{Prinzessin}}}\label{K_L02170_1h}{ }\textcolor{blue}{\textsc{Lothar Metternich}}{}\ledrightnote{\textcolor{blue}{Karoline Franziska von Metternich-Winneburg}}
                    (Schwägerin der Fürſtin \textcolor{blue}{\textsc{Pauline}}{}\ledrightnote{\textcolor{blue}{Pauline von Metternich-Sándor}}). Die wäre glücklich, mit Ihnen zuſammenzukommen.
                    Alſo bitte: beſtimmen Sie einen der 3 Tage dieſer Woche: Donnerſtag, Freitag
                    oder Samſtag – und {\pb}ich arrangiere einen ganz
                    intimen kleinen Nachmittags-Gedankenaustauſch nur Sie beide, meine Freundin \textcolor{blue}{\textsc{Metternich}}{}\ledrightnote{\textcolor{blue}{Karoline Franziska von Metternich-Winneburg}} und
                    höchſtens noch zwei drei Perſonen (5 Uhr)\pend
           \pstart
           Einer lieben Antwort gewertig{\\[\baselineskip]}\spacefill\mbox{Bertha Suttner}\pend
           \leftskip=0em{}\endnumbering\briefempfaengerindex{Schnitzler, Olga@\textsc{Schnitzler, Olga}!zzzSuttner, Bertha von@\emph{von Bertha von Suttner}!1914-03-302@{30. 3. 1914}|)be}\briefempfaengerindex{Schnitzler, Arthur@\textsc{Schnitzler, Arthur}!zzzSuttner, Bertha von@\emph{von Bertha von Suttner}!1914-03-302@{30. 3. 1914}|)be}\mylabel{h}  \normalsize

\doendnotes{C}
\bigskip
\vfill

\clearpage

\footnotesize

\lohead{\textsc{register}}

% Definiere theindex-Environment komplett neu ohne reledmac
\makeatletter
\renewenvironment{theindex}{%
  \section*{\indexname}%
  \setlength{\parindent}{0pt}%
  \setlength{\parskip}{0pt plus 0.3pt}%
  \let\item\@idxitem
}{%
  \clearpage
}
\makeatother

\IfFileExists{\jobname-pw.ind}{\input{\jobname-pw.ind}}{}

\end{document}

      