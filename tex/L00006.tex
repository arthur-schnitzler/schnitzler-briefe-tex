%% latex-korrekturansicht-vorspann.tex
%% Vorspann für die Korrekturansicht.
%% Lädt die gemeinsame Datei latex-vorspann.tex mit gesetztem Schalter.

\newif\ifkorrekturansicht
\korrekturansichttrue

\input{../tex-inputs/latex-vorspann}


               \section[Arthur Schnitzler an Wilhelm Bölsche, 14. 10. 1890]{ Arthur Schnitzler an Wilhelm Bölsche, 14. 10. 1890}\nopagebreak\mylabel{v}\rehead{ }\normalsize\beginnumbering\briefempfaengerindex{Boelsche, Wilhelm@\textsc{Bölsche, Wilhelm}!zzzSchnitzler, Arthur@\emph{von Arthur Schnitzler}!1890-10-141@{14. 10. 1890}|(be} \toendnotes[C]{\smallbreak\pagebreak[2]} \Standort{Wrocław, Biblioteka Uniwersytecka, Böl.Pis 1759.}
\physDesc{Brief, 1 Blatt, 2 Seiten
\newline{}Handschrift: schwarze Tinte, deutsche Kurrent}\buchAbdrucke{\weitereDrucke{1) Alois Woldan: \emph{Arthur Schnitzler – Briefe an Wilhelm Bölsche.} In: \emph{Germanica Wratislaviensia} (1987) Nr. 77, S. 458.} \weitereDrucke{2) Wilhelm Bölsche: \emph{Briefwechsel. Mit Autoren der Freien Bühne}. Hg. Gerd-Hermann Susen. Berlin: \emph{Weidler} 2010, S. 668 (Werke und Briefe. Wissenschaftliche Ausgabe, Briefe I).} }\toendnotes[C]{\smallbreak}\pstart{}{\pb}Sehr geehrter Herr Redakteur!\pend\pstart
           Ihrer freundlichen Aufforderung gemäß, die ich mir erlaubt habe, nicht als einfache
               Höflichkeitsform zu betrachten, ſende ich Ihnen hier etwas andres – nur ein \textcolor{green}{Gedicht}{}\ledrightnote{→\textcolor{green}{Morgenandacht}}, wie Sie ſehen, von dem
               ich aber vielleicht annehmen kann, daſs es nicht ganz aus dem Stil Ihres \textcolor{green}{Blattes}{}\ledrightnote{→\textcolor{green}{Freie Bühne für modernes Leben}} fällt. Wollen Sie die
               große Liebenswürdigkeit haben (bei Gedichten iſt das wirklich eine große
               Liebenswürdigkeit) mir {\pb}die »\label{K_L00006_1v}\edtext{\textcolor{green}{Morgenandacht}{}\ledrightnote{\textcolor{green}{Morgenandacht}}}{\lemma{\textnormal{\emph{Morgenandacht}}}\Cendnote{\textnormal{Nach der Ablehnung durch Bölsche am
                     25. 10. 1890{ }sandte Schnitzler das \textcolor{green}{Gedicht} umgehend an \textcolor{blue}{Michael Georg Conrad}; dieser druckte es in der \emph{\textcolor{green}{Gesellschaft}} im Februar 1891; vgl. Michael Georg Conrad an Arthur Schnitzler,
                    14. 11. 1890}}}\label{K_L00006_1h}« zurückzuſchicken, wenn Sie ſie nicht
               brauchen können? –\pend
           \pstart
           Hochachtungsvoll{\\[\baselineskip]}\spacefill\mbox{Dr. med. Arthur Schnitzler}\pend
           \leftskip=0em{}\pstart
           \noindent{}\textsc{\textcolor{pink}{Wien I. Giselastraße 11}{}\ledrightnote{\textcolor{pink}{Bösendorferstraße}}.}{\\}\textsc{14. Oktober 1890}.\pend
           \endnumbering\briefempfaengerindex{Boelsche, Wilhelm@\textsc{Bölsche, Wilhelm}!zzzSchnitzler, Arthur@\emph{von Arthur Schnitzler}!1890-10-141@{14. 10. 1890}|)be}\mylabel{h}  \normalsize

\doendnotes{C}
\bigskip
\vfill

\clearpage

\footnotesize

\lohead{\textsc{register}}

% Definiere theindex-Environment komplett neu ohne reledmac
\makeatletter
\renewenvironment{theindex}{%
  \section*{\indexname}%
  \setlength{\parindent}{0pt}%
  \setlength{\parskip}{0pt plus 0.3pt}%
  \let\item\@idxitem
}{%
  \clearpage
}
\makeatother

\IfFileExists{\jobname-pw.ind}{\input{\jobname-pw.ind}}{}

\end{document}

      