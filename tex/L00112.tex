%% latex-korrekturansicht-vorspann.tex
%% Vorspann für die Korrekturansicht.
%% Lädt die gemeinsame Datei latex-vorspann.tex mit gesetztem Schalter.

\newif\ifkorrekturansicht
\korrekturansichttrue

\input{../tex-inputs/latex-vorspann}


               \section[Arthur Schnitzler an Hugo von Hofmannsthal, {[}6. 8. 1892{]}]{ Arthur Schnitzler an Hugo von Hofmannsthal, {[}6. 8. 1892{]}}\nopagebreak\mylabel{v}\rehead{ }\normalsize\beginnumbering\briefempfaengerindex{Hofmannsthal, Hugo von@\textsc{Hofmannsthal, Hugo von}!zzzSchnitzler, Arthur@\emph{von Arthur Schnitzler}!1892-08-061@{6. 8. 1892}|(be} \toendnotes[C]{\smallbreak\pagebreak[2]} \Standort{FDH, Hs-30885,24.}
\physDesc{Brief, 2 Blätter, 6 Seiten
\newline{}Handschrift: schwarze Tinte, deutsche Kurrent\newline{}Ordnung: mit Bleistift von Schnitzler mutmaßlich bei der Durchsicht der Briefe 1929 das erste Blatt beschriftet:
                                    »\textcolor{pink}{Wien}« und datiert: »6. 8. 92«. Das zweite Blatt datiert:
                                  »(6. 8. 92{[}){]}« }\buchAbdrucke{\weitereDrucke{1) Hugo von Hofmannsthal, Arthur Schnitzler: \emph{Briefwechsel}. Hg. Therese Nickl und Heinrich Schnitzler. Frankfurt am Main: \emph{S. Fischer} 1964, S. 27–28.} \weitereDrucke{2) Hermann Bahr, Arthur Schnitzler: \emph{Briefwechsel, Aufzeichnungen, Dokumente
                                (1891–1931)}. Hg. Kurt Ifkovits und Martin Anton Müller. Göttingen: \emph{Wallstein} 2018.} }\toendnotes[C]{\smallbreak}\pstart{}{\pb}Mein lieber Loris,\pend\pstart
           vielen Dank für den überſandten Brief. Es ſtehen geſcheidte Sachen drin. Es iſt
                    ſogar möglich, daſs die \textcolor{blue}{H.}{}\ledrightnote{\textcolor{blue}{Marie Herzfeld}} mit all ihrem
                    Tadel Recht hat: gewiſs aber hat ſie manches zu loben vergeſſen. Daſs ſie den
                        »\textcolor{green}{Sohn}{}\ledrightnote{\textcolor{green}{Der Sohn. Aus den Papieren eines Arztes}}« ſo beſonders gut findet zeigt
                    mir, daſs ſie ein wenig vom \textcolor{pink}{Berliner}{}\ledrightnote{\textcolor{pink}{Berlin}}-\textcolor{blue}{Bölſche}{}\ledrightnote{\textcolor{blue}{Wilhelm Bölsche}}thum
                    beeinflußt iſt. Ich habe den Eindruck, daſs ſie alles einzelne an mir verſteht,
                    wie das bei ihrer kritiſchen {\pb}Begabung
                    ſelbſtverſtändlich – nur meine \uline{Atmosphäre}
                    nicht. –\pend
           \pstart
           Das \textcolor{green}{Anatol}{}\ledrightnote{\textcolor{green}{Anatol}}-Buch erſcheint im \textsc{\textcolor{brown}{Bibliogr. Bureau}{}\ledrightnote{\textcolor{brown}{Bibliographisches Bureau}}, \textcolor{pink}{Berlin}{}\ledrightnote{\textcolor{pink}{Berlin}}}. –\pend
           \pstart
           Von \textcolor{blue}{Blumenthal}{}\ledrightnote{\textcolor{blue}{Oskar Blumenthal}} hab ich Nachricht: 2. Quartal,
                    d. h. Jaenner–März 93 Etwas ſpät! Umſomehr als ich
                    heute aus \textcolor{pink}{Prag}{}\ledrightnote{\textcolor{pink}{Prag}} die Mittheilung erhalte, daſs
                    das \textcolor{green}{Stück}{}\ledrightnote{→\textcolor{green}{Anatol}} im Oktober dranko{\geminationm}en dürfte! Zugleich hat man mir meine Luſtſpiele
                    von dort retournirt, da ſie für eine Provinzbühne zu gewagt ſeien.\pend
           \pstart
           {\pb}– \textcolor{blue}{\textsc{Schupp}}{}\ledrightnote{\textcolor{blue}{Falk Schupp}} iſt Secretär des Preſsausſchuſses für d. \textsc{\textcolor{pink}{Chicago}{}\ledrightnote{\textcolor{pink}{Chicago}}. \textcolor{brown}{W. A.}{}\ledrightnote{\textcolor{brown}{Weltausstellung 1893}}} –\pend
           \pstart
           – \textsc{Von \textcolor{blue}{Theodor Herzl}{}\ledrightnote{\textcolor{blue}{Theodor Herzl}}} hab ich einen reizenden Brief beko{\geminationm}en. –\pend
           \pstart
           Vielleicht ſehen wir uns doch im Laufe dieſes So{\geminationm}ers. Ich habe nämlich keine Einberufung zur Waffenübung beko{\geminationm}en, und fahre vielleicht Ende Auguſt
                    nach \textcolor{pink}{Iſchl}{}\ledrightnote{\textcolor{pink}{Bad Ischl}}. – Wohin gehn Sie im
                        September? –\pend
           \pstart
           – Ich kam die letzten Tage nicht zum Schreiben; die äußerliche Thätigkeit ſtört
                    doch. Hoffentlich bald! – Sie {\pb}ko{\geminationm}en ja ſicher mit den ganzen 5 Akten zurück!
                    ––\pend
           \pstart
           Haben Sie Recht, von einem »\uline{herrſchenden}
                    Novellendrama« zu ſprechen? – Berechtigung hat die Form gewiſs – ſobald nur \uline{ein} bedeutender Menſch da iſt, der daran Freude
                    findet. Ueber den gewiſſen Fundamentalſatz: »Das iſt eben kein rechtes Drama,
                    das nicht von der Bühne herab wirkt (oder gar ›auf die Menge‹ wirkt\strikeout{«})« hab ich {\pb}mich
                        i{\geminationm}er geärgert. Eventuell will ich mir, mir ganz
                    allein was vorſpielen laſſen! – Na, Sie wiſſen ja, \textcolor{blue}{Kulka}{}\ledrightnote{\textcolor{blue}{Julius Kulka}} hat ja das wichtigſte über dieſes Thema ſchon gesagt.
                    –\pend
           \pstart
           – Wa{\geminationn} wird man ſich Briefe phonographiren können? –
                    Die Zeit ſeh ich ko{\geminationm}en, wo die Leute über unſre
                    mühſelige Correſpondenzerei lächeln und ſtaunen werden.\pend
           \pstart
           {\pb}Auf dieſer Seite ſteht nur mehr, daſs ich Sie,
                    liebſter Freund, aufs Herzlichſte grüße!\pend
           \pstart
           Ganz der Ihre{\\[\baselineskip]}\spacefill\mbox{Arthur.}\pend
           \leftskip=0em{}\pstart
           \noindent{}Was macht \textcolor{blue}{\textsc{Richard}}{}\ledrightnote{\textcolor{blue}{Richard Beer-Hofmann}}? –\pend
           \pstart
           – Mit \textcolor{blue}{\textsc{Schwarzkopf}}{}\ledrightnote{\textcolor{blue}{Gustav Schwarzkopf}} war ich einige Male auf dem Land. –\pend
           \pstart
           \textcolor{blue}{\textsc{Bahr}}{}\ledrightnote{\textcolor{blue}{Hermann Bahr}} iſt verzweifelt; – er wurde einberufen und fahndet nun nach
                        einer Befreiung. –\pend
           \endnumbering\briefempfaengerindex{Hofmannsthal, Hugo von@\textsc{Hofmannsthal, Hugo von}!zzzSchnitzler, Arthur@\emph{von Arthur Schnitzler}!1892-08-061@{6. 8. 1892}|)be}\mylabel{h}  \normalsize

\doendnotes{C}
\bigskip
\vfill

\clearpage

\footnotesize

\lohead{\textsc{register}}

% Definiere theindex-Environment komplett neu ohne reledmac
\makeatletter
\renewenvironment{theindex}{%
  \section*{\indexname}%
  \setlength{\parindent}{0pt}%
  \setlength{\parskip}{0pt plus 0.3pt}%
  \let\item\@idxitem
}{%
  \clearpage
}
\makeatother

\IfFileExists{\jobname-pw.ind}{\input{\jobname-pw.ind}}{}

\end{document}

      