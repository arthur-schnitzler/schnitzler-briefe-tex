%% latex-korrekturansicht-vorspann.tex
%% Vorspann für die Korrekturansicht.
%% Lädt die gemeinsame Datei latex-vorspann.tex mit gesetztem Schalter.

\newif\ifkorrekturansicht
\korrekturansichttrue

\input{../tex-inputs/latex-vorspann}


               \section[Hugo von Hofmannsthal an Arthur Schnitzler, 15. 6. 1907]{ Hugo von Hofmannsthal an Arthur Schnitzler, 15. 6. 1907}\nopagebreak\mylabel{v}\rehead{ }\normalsize\beginnumbering\briefempfaengerindex{Schnitzler, Arthur@\textsc{Schnitzler, Arthur}!zzzHofmannsthal, Hugo von@\emph{von Hugo von Hofmannsthal}!1907-06-151@{15. 6. 1907}|(be} \toendnotes[C]{\smallbreak\pagebreak[2]} \Standort{CUL, Schnitzler, B 43.}
\physDesc{Bildpostkarte
\newline{}Handschrift: schwarze Tinte, deutsche Kurrent\newline{}Versand: Stempel: »\nobreak{}\oindex{Venedig@\textbf{Venedig}, \emph{Besiedelter Ort (A.BSO)}|pwk}Sezioni Riunite Venezia, 15 6–07, 6S\nobreak{}«.  
\newline{}Schnitzler: mit Bleistift datiert: »Jun 907« \newline{}Ordnung: 1) mit Bleistift von unbekannter Hand nummeriert:
                              »279« 2) mit Bleistift von unbekannter Hand nummeriert: »280«}\buchAbdrucke{\weitereDrucke{Hugo von Hofmannsthal, Arthur Schnitzler: \emph{Briefwechsel}. Hg. Therese Nickl und Heinrich Schnitzler. Frankfurt am Main: \emph{S. Fischer} 1964, S. 229.} }\toendnotes[C]{\smallbreak}\pstart{}{\pb}\textsc{Herrn D\textsuperscript{r} Arthur Schnitzler}\pend{}\pstart{}\textcolor{pink}{\textsc{Wien}}{}\ledrightnote{\textcolor{pink}{Wien}}\pend{}\pstart{}\textcolor{pink}{\textsc{XVII. Spöttelgasse 7}.}{}\ledrightnote{\textcolor{pink}{Edmund-Weiß-Gasse}}\pend{}{\bigskip}\pstart
           \noindent{}\centering{}\textcolor{gray}{\textbf{{\pb}\textcolor{pink}{Duino}{}\ledrightnote{\textcolor{pink}{Duino}} bei \textcolor{pink}{Sistiana}{}\ledrightnote{\textcolor{pink}{Sistiana}}}}.
               \pend
           \pstart
           {\pb}Hier haben wir ein paar ſehr
               angenehme Tage verbracht. Am \textcolor{pink}{Lido}{}\ledrightnote{\textcolor{pink}{Lido}} hoffe ich manches
               zu arbeiten.\pend
           \pstart
           Herzlich{\\[\baselineskip]}\spacefill\mbox{Hugo.}\pend
           \leftskip=0em{}\pstart
           \noindent{}\label{T_L01684_1v}\edtext{+ unſere Zi{\geminationm}er}{\lemma{\textnormal{\emph{+ unſere Zier}}}\Cendnote{\textnormal{Verweis auf eine ebenfalls mit
                        »+« markierte Stelle auf dem Kartenmotiv.}}}\label{T_L01684_1h}\pend
           \endnumbering\briefempfaengerindex{Schnitzler, Arthur@\textsc{Schnitzler, Arthur}!zzzHofmannsthal, Hugo von@\emph{von Hugo von Hofmannsthal}!1907-06-151@{15. 6. 1907}|)be}\mylabel{h}  \normalsize

\doendnotes{C}
\bigskip
\vfill

\clearpage

\footnotesize

\lohead{\textsc{register}}

% Definiere theindex-Environment komplett neu ohne reledmac
\makeatletter
\renewenvironment{theindex}{%
  \section*{\indexname}%
  \setlength{\parindent}{0pt}%
  \setlength{\parskip}{0pt plus 0.3pt}%
  \let\item\@idxitem
}{%
  \clearpage
}
\makeatother

\IfFileExists{\jobname-pw.ind}{\input{\jobname-pw.ind}}{}

\end{document}

      