%% latex-korrekturansicht-vorspann.tex
%% Vorspann für die Korrekturansicht.
%% Lädt die gemeinsame Datei latex-vorspann.tex mit gesetztem Schalter.

\newif\ifkorrekturansicht
\korrekturansichttrue

\input{../tex-inputs/latex-vorspann}


               \section[Arthur Schnitzler an Richard Beer-Hofmann, 8. 5. 1921]{ Arthur Schnitzler an Richard Beer-Hofmann, 8. 5. 1921}\nopagebreak\mylabel{v}\rehead{ }\normalsize\beginnumbering\briefempfaengerindex{Beer-Hofmann, Richard@\textsc{Beer-Hofmann, Richard}!zzzSchnitzler, Arthur@\emph{von Arthur Schnitzler}!1921-05-081@{8. 5. 1921}|(be} \toendnotes[C]{\smallbreak\pagebreak[2]} \Standort{YCGL, MSS 31.}
\physDesc{Bildpostkarte
\newline{}Handschrift: Bleistift, deutsche Kurrent\newline{}Versand: 1) Stempel: »\nobreak{}\oindex{Muenchen@\textbf{München}, \emph{https://www.geonames.org/ontologyP.PPLA}|pwk}München, 8. V. 21, 1–2N\nobreak{}«.  2) mit blauem Buntstift von unbekannter Hand bei der Ortsangabe der
                                 Adresse die Bezirksnummer ergänzt: »XVIII«}\toendnotes[C]{\smallbreak}\pstart{}{\pb}Hrn \textsc{Richard Beer Hofmann}\pend{}\pstart{}\textcolor{pink}{Wien}{}\ledrightnote{\textcolor{pink}{Wien}}\pend{}\pstart{}\textsc{\textcolor{pink}{Hasenauerstr 59}{}\ledrightnote{\textcolor{pink}{Hasenauerstraße}}}.\pend{}{\bigskip}\pstart
           \noindent{}\centering{}{\pb}\textcolor{gray}{\textbf{\textcolor{pink}{\textbf{München}}{}\ledrightnote{\textcolor{pink}{München}}\hspace*{1.5em}Partie an der \textcolor{pink}{Isar}{}\ledrightnote{\textcolor{pink}{Isar}} – Blick gegen Süden}}\pend
           \pstart
           {\pb}Herzliche Grüße\pend
           \pstart
           Ihr{\\[\baselineskip]}\spacefill\mbox{Arthur}\pend
           \leftskip=0em{}\pstart
           \raggedleft{}\label{T_L02366-1v}\edtext{8. 5. 1921}{\lemma{\textnormal{\emph{8. 5. 1921}}}\Cendnote{\textnormal{ab hier seitlich zum Text}}}\label{T_L02366-1h}\pend
           \pstart
           \label{KLL02366_Beer-Hofmann-1v}\edtext{Geſtern}{\lemma{\textnormal{\emph{Geſtern}}}\Cendnote{\textnormal{siehe A. S.: \emph{Tagebuch}, 7. 5. 1921}}}\label{KLL02366_Beer-Hofmann-1h}, in \textcolor{pink}{Harlaching}{}\ledrightnote{\textcolor{pink}{Harlaching}}, hab ich viel Ihrer, und
                  schöner Tage gedacht.\pend
           \endnumbering\briefempfaengerindex{Beer-Hofmann, Richard@\textsc{Beer-Hofmann, Richard}!zzzSchnitzler, Arthur@\emph{von Arthur Schnitzler}!1921-05-081@{8. 5. 1921}|)be}\mylabel{h}  \normalsize

\doendnotes{C}
\bigskip
\vfill

\clearpage

\footnotesize

\lohead{\textsc{register}}

% Definiere theindex-Environment komplett neu ohne reledmac
\makeatletter
\renewenvironment{theindex}{%
  \section*{\indexname}%
  \setlength{\parindent}{0pt}%
  \setlength{\parskip}{0pt plus 0.3pt}%
  \let\item\@idxitem
}{%
  \clearpage
}
\makeatother

\IfFileExists{\jobname-pw.ind}{\input{\jobname-pw.ind}}{}

\end{document}

      