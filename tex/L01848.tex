%% latex-korrekturansicht-vorspann.tex
%% Vorspann für die Korrekturansicht.
%% Lädt die gemeinsame Datei latex-vorspann.tex mit gesetztem Schalter.

\newif\ifkorrekturansicht
\korrekturansichttrue

\input{../tex-inputs/latex-vorspann}


               \section[Arthur Schnitzler an Hermann Bahr, 22. 6. 1909]{ Arthur Schnitzler an Hermann Bahr, 22. 6. 1909}\nopagebreak\mylabel{v}\rehead{ }\normalsize\beginnumbering\briefempfaengerindex{Bahr, Hermann@\textsc{Bahr, Hermann}!zzzSchnitzler, Arthur@\emph{von Arthur Schnitzler}!1909-06-221@{22. 6. 1909}|(be} \toendnotes[C]{\smallbreak\pagebreak[2]} \Standort{TMW, HS AM 60167 Ba.}
\physDesc{Briefkarten, 2 Karten, 3 Seiten
\newline{}Handschrift: schwarze Tinte, deutsche Kurrent\newline{}Ordnung: Lochung }\buchAbdrucke{\weitereDrucke{1) \emph{22. 6. 1909, Abschrift.} In: Arthur Schnitzler: \emph{The Letters of Arthur Schnitzler to Hermann Bahr}. Edited, annotated, and with an introduction, by Donald G.
                        Daviau. Chapel Hill: \emph{The University of North Carolina Press} 1978, S. 103 (University of North Carolina studies in the Germanic languages
                        and literatures, 89).} \weitereDrucke{2) Hermann Bahr, Arthur Schnitzler: \emph{Briefwechsel, Aufzeichnungen, Dokumente (1891–1931)}. Hg. Kurt Ifkovits und Martin Anton Müller. Göttingen: \emph{Wallstein} 2018, S. 418.} }\toendnotes[C]{\smallbreak}\pstart
           \noindent{}{\pb}\textcolor{gray}{\textbf{Dr. Arthur Schnitzler}}\hfill 22. 6. 09\pend
           \pstart
           \textcolor{gray}{\textbf{\textcolor{pink}{Wien XVIII. Spoettelgasse 7}{}\ledrightnote{\textcolor{pink}{Edmund-Weiß-Gasse}}.}}\pend
           \pstart
           mein lieber Herma{\geminationn}, geſtern iſt das \textcolor{green}{Tagebuch}{}\ledrightnote{\textcolor{green}{Tagebuch [Berlin: Paul Cassirer]}} geko{\geminationm}en und
               neulich die \textcolor{green}{Drut}{}\ledrightnote{\textcolor{green}{Drut}}, die meine \textcolor{blue}{Frau}{}\ledrightnote{→\textcolor{blue}{Olga Schnitzler}}{ }ſofort für ſich beanſprucht und mit großem
               Entzücken geleſen hat. Auch \textcolor{blue}{Burkhard}{}\ledrightnote{\textcolor{blue}{Max Eugen Burckhard}} hat mir in
                  \textsc{\textcolor{pink}{St Gilgen}{}\ledrightnote{\textcolor{pink}{St. Gilgen}}} viel ſchönes darüber geſagt. Ja ſo ſpricht man übereinander und ſieht und
               ſpricht ſich nie. \label{K_L01848_1v}\edtext{Einer w\damage{ird}{ }{\pb}übrig bleiben und
               ſagen: »{\dots}{ }Schade{\dotsfour}«}{\lemma{\textnormal{\emph{Einer … Schade«}}}\Cendnote{\textnormal{vgl. Hermann Bahr an Arthur Schnitzler, 28. 6. 1909, Arthur Schnitzler an Hermann Bahr, 16. 2. 1930}}}\label{K_L01848_1h}\pend
           \pstart
           \damage{Wi}r ſind von \textcolor{pink}{Gilgen}{}\ledrightnote{\textcolor{pink}{St. Gilgen}} zurückgehetzt, weil unſer
                  \textcolor{blue}{Bub}{}\ledrightnote{→\textcolor{blue}{Heinrich Schnitzler}} eine Art Keuchhuſten
               hat, recht leicht bis jetzt. Nächſte Woche fahren wir nach \textcolor{pink}{Edlach}{}\ledrightnote{\textcolor{pink}{Edlach}}, ich mit der \textcolor{green}{Drut}{}\ledrightnote{\textcolor{green}{Drut}} und
               dem \textcolor{green}{Tagebuch}{}\ledrightnote{\textcolor{green}{Tagebuch [Berlin: Paul Cassirer]}} und freu mich ſchon ſehr. Mit dem
               Danken ko{\geminationm}t man ja nicht nach bei dir. Ich war auch
               nicht ſehr faul – aber wie ko{\geminationm}t man ſich gegen dich vor!
               Mit \textcolor{blue}{Burckhard}{}\ledrightnote{\textcolor{blue}{Max Eugen Burckhard}} war ich auf ſeiner {\pb}Alm oben; ich finde es
               geht ihm recht gut, er war lebendig, fidel geradezu und jung.\pend
           \pstart
           \textcolor{blue}{Wir}{}\ledrightnote{→\textcolor{blue}{Olga Schnitzler}} grüßen dich
               herzlichſt.{\\[\baselineskip]}Dein getreuer{\\[\baselineskip]}\spacefill\mbox{Arthur}\pend
           \leftskip=0em{}\endnumbering\briefempfaengerindex{Bahr, Hermann@\textsc{Bahr, Hermann}!zzzSchnitzler, Arthur@\emph{von Arthur Schnitzler}!1909-06-221@{22. 6. 1909}|)be}\mylabel{h}  \normalsize

\doendnotes{C}
\bigskip
\vfill

\clearpage

\footnotesize

\lohead{\textsc{register}}

% Definiere theindex-Environment komplett neu ohne reledmac
\makeatletter
\renewenvironment{theindex}{%
  \section*{\indexname}%
  \setlength{\parindent}{0pt}%
  \setlength{\parskip}{0pt plus 0.3pt}%
  \let\item\@idxitem
}{%
  \clearpage
}
\makeatother

\IfFileExists{\jobname-pw.ind}{\input{\jobname-pw.ind}}{}

\end{document}

      