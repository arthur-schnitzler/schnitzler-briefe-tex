%% latex-korrekturansicht-vorspann.tex
%% Vorspann für die Korrekturansicht.
%% Lädt die gemeinsame Datei latex-vorspann.tex mit gesetztem Schalter.

\newif\ifkorrekturansicht
\korrekturansichttrue

\input{../tex-inputs/latex-vorspann}


               \section[Arthur Schnitzler an Richard Beer-Hofmann, 1. 4. 1898]{ Arthur Schnitzler an Richard Beer-Hofmann, 1. 4. 1898}\nopagebreak\mylabel{v}\rehead{ }\normalsize\beginnumbering\briefempfaengerindex{Beer-Hofmann, Richard@\textsc{Beer-Hofmann, Richard}!zzzSchnitzler, Arthur@\emph{von Arthur Schnitzler}!1898-04-011@{1. 4. 1898}|(be} \toendnotes[C]{\smallbreak\pagebreak[2]} \Standort{YCGL, MSS 31.}
\physDesc{Briefkarte, Umschlag
\newline{}Handschrift: schwarze Tinte, deutsche Kurrent\newline{}Versand: 1) Stempel: »\nobreak{}\oindex{IX., Alsergrund@\textbf{IX., Alsergrund}, \emph{Bezirk (A.BZK)}|pwk}Wien 9/3, 1. 4. 98, 5–6 N\nobreak{}«.  2) Stempel: »\nobreak{}\oindex{I., Innere Stadt@\textbf{I., Innere Stadt}, \emph{Bezirk (A.BZK)}|pwk}{[}Wi{]}en 1/1, {[}2.{]} 4. 98, {[}7–8{]}½ N, {[}Best{]}ellt\nobreak{}«. }\toendnotes[C]{\smallbreak}\pstart{}{\pb}Herrn \textsc{Dr. Rich.
                     Beer-Hofmann}\pend{}\pstart{}\textcolor{pink}{Wien}{}\ledrightnote{\textcolor{pink}{Wien}}\pend{}\pstart{}\textcolor{pink}{\textsc{I.
                     Wollzeile 15}}{}\ledrightnote{\textcolor{pink}{Wollzeile}}\pend{}{\bigskip}\pstart
           \noindent{}{\pb}Lieber Richard, verſäumen Sie gewiſs nicht, an \textcolor{blue}{Paul}{}\ledrightnote{\textcolor{blue}{Paul Goldmann}} (\textcolor{pink}{\textsc{Genua}}{}\ledrightnote{\textcolor{pink}{Genua}}, \textsc{ferma in posta}), natürlich
               gleich, ein paar Worte des \label{K_L00789_1v}\edtext{Abſchieds}{\lemma{\textnormal{\emph{Abſchieds}}}\Cendnote{\textnormal{\textcolor{blue}{Goldmann} bestieg am
                     5. 4. 1898 in \textcolor{pink}{Genua} ein Schiff nach \textcolor{pink}{China}.}}}\label{K_L00789_1h} zu ſchreiben. – Laſſen Sie mich wegen So{\geminationn}tag was {\pb}wiſſen,
                  we{\geminationn}{ }Sie frei ſind. –\pend
           \pstart
           Im Fall ſchlechten Wetters bin ich übrigens Samſtg Abds im \textcolor{pink}{Pucher}{}\ledrightnote{\textcolor{pink}{Café Pucher}}.\pend
           \pstart
           Herzlichſt Ihr {\\[\baselineskip]}\spacefill\mbox{Arthur}\pend
           \leftskip=0em{}\endnumbering\briefempfaengerindex{Beer-Hofmann, Richard@\textsc{Beer-Hofmann, Richard}!zzzSchnitzler, Arthur@\emph{von Arthur Schnitzler}!1898-04-011@{1. 4. 1898}|)be}\mylabel{h}  \normalsize

\doendnotes{C}
\bigskip
\vfill

\clearpage

\footnotesize

\lohead{\textsc{register}}

% Definiere theindex-Environment komplett neu ohne reledmac
\makeatletter
\renewenvironment{theindex}{%
  \section*{\indexname}%
  \setlength{\parindent}{0pt}%
  \setlength{\parskip}{0pt plus 0.3pt}%
  \let\item\@idxitem
}{%
  \clearpage
}
\makeatother

\IfFileExists{\jobname-pw.ind}{\input{\jobname-pw.ind}}{}

\end{document}

      