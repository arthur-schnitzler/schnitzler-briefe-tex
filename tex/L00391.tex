%% latex-korrekturansicht-vorspann.tex
%% Vorspann für die Korrekturansicht.
%% Lädt die gemeinsame Datei latex-vorspann.tex mit gesetztem Schalter.

\newif\ifkorrekturansicht
\korrekturansichttrue

\input{../tex-inputs/latex-vorspann}


               \section[Richard Beer-Hofmann an Arthur Schnitzler, 23. 10. 1894]{ Richard Beer-Hofmann an Arthur Schnitzler, 23. 10. 1894}\nopagebreak\mylabel{v}\rehead{ }\normalsize\beginnumbering\briefempfaengerindex{Schnitzler, Arthur@\textsc{Schnitzler, Arthur}!zzzBeer-Hofmann, Richard@\emph{von Richard Beer-Hofmann}!1894-10-231@{23. 10. 1894}|(be} \toendnotes[C]{\smallbreak\pagebreak[2]} \Standort{CUL, Schnitzler, B 8.}
\physDesc{Brief, 1 Blatt, 4 Seiten
\newline{}Handschrift: Bleistift, lateinische Kurrent
\newline{}Schnitzler: mit Bleistift datiert: »23/10 94« und nummeriert »51« }\buchAbdrucke{\weitereDrucke{Arthur Schnitzler, Richard Beer-Hofmann: \emph{Briefwechsel 1891–1931}. Hg. Konstanze Fliedl. Wien, Zürich: \emph{Europaverlag} 1992, S. 68.} }\toendnotes[C]{\smallbreak}\pstart
           \noindent{}{\pb}Lieber Arthur! Soeben
               erhalte ich Ihren »\textcolor{blue}{Sudermann}{}\ledrightnote{\textcolor{blue}{Hermann Sudermann}}«brief, er hat sich
               mit meinem gestrigen gekreuzt, wo ich von »\textcolor{green}{Schmetterlingsschlacht}{}\ledrightnote{\textcolor{green}{Die Schmetterlingsschlacht}}« sprach. Also ich habe richtig empfunden. Schön
               wär es wenn »\textcolor{green}{Liebelei}{}\ledrightnote{\textcolor{green}{Liebelei. Schauspiel in drei Akten}}« am \textcolor{pink}{Burgtheater}{}\ledrightnote{\textcolor{pink}{Burgtheater}} drankäme – sehr {\pb}schön, der Erfolg der Aufführung
               wäre beinahe nebensächlich \strikeout{neb} gegenüber dem Erfolg
               der Annahme. Freilich, \textcolor{blue}{Schönthan}{}\ledrightnote{\textcolor{blue}{Franz von Schönthan-Pernwald}} und \textcolor{blue}{Rudolf Lothar}{}\ledrightnote{\textcolor{blue}{Rudolf Lothar}} und das \textcolor{green}{Buch
                  Hiob}{}\ledrightnote{\textcolor{green}{Das Buch Hiob. Schauspiel in einem Akt}}, spielt man auch am \textcolor{pink}{Burgtea{\pb}ter}{}\ledrightnote{\textcolor{pink}{Burgtheater}}. Nur \uline{wir} würden eigentlich erstaunt sein daß »\textcolor{green}{Liebelei}{}\ledrightnote{\textcolor{green}{Liebelei. Schauspiel in drei Akten}}« angeno{\geminationm}en wird,
               und finden die Annahme all’ des Andern begreiflich. Nein arrogant sind wir nicht. In
                  \textcolor{pink}{Pompei\strikeout{j}}{}\ledrightnote{\textcolor{pink}{Pompei}} war ich heute; ich bin ganz krank \strikeout{nach} vor Sehnsucht nach {\pb}wirklichen \textcolor{pink}{römischen}{}\ledrightnote{\textcolor{pink}{Rom}} Bädern. Im Culturraffinement sind wir noch alle Barbaren.
               Ja – Theater wollten Sie wissen?\pend
           \settowidth{\longeste}{La martire (Samarra)}\settowidth{\longestz}{Mailand}\settowidth{\longestd}{}\settowidth{\longestv}{}\settowidth{\longestf}{}\addtolength\longeste{1em}
        \addtolength\longestz{1em}
      \pstart\noindent\makebox[\the\longeste][l]{\textcolor{green}{La martire}{}\ledrightnote{\textcolor{green}{La Martire}} (\textcolor{blue}{Samarra}{}\ledrightnote{\textcolor{blue}{Spyros Samaras}})}\makebox[\the\longestz][l]{\textcolor{pink}{Mailand}{}\ledrightnote{\textcolor{pink}{Mailand}}}
                  \pend\pstart\noindent\makebox[\the\longeste][l]{\textcolor{green}{Medici}{}\ledrightnote{\textcolor{green}{I Medici}}}\makebox[\the\longestz][l]{}
                  \pend\settowidth{\longeste}{Premiere von}\settowidth{\longestz}{Puppenfeela fata del bambolitalienisch richtig: La fata delle bambole}\settowidth{\longestd}{Rom}\settowidth{\longestv}{}\settowidth{\longestf}{}\addtolength\longeste{1em}
        \addtolength\longestz{1em}
        \addtolength\longestd{1em}
      \pstart\noindent\makebox[\the\longeste][l]{Premiere von}\makebox[\the\longestz][l]{\textcolor{green}{Ennemico del popolo}{}\ledrightnote{\textcolor{green}{Ein Volksfeind}}}
                  \makebox[\the\longestd][l]{\textcolor{pink}{Rom}{}\ledrightnote{\textcolor{pink}{Rom}}}\pend\pstart\noindent\makebox[\the\longeste][l]{\hspace*{2.5em}“\hspace*{1.5em}“\hspace*{1em}}\makebox[\the\longestz][l]{\textcolor{green}{Puppenfee}{}\ledrightnote{\textcolor{green}{Die Puppenfee}}{ }\label{K_L00391_1v}\edtext{\textcolor{green}{la fata del bambol}{}\ledrightnote{\textcolor{green}{Die Puppenfee}}}{\lemma{\textnormal{\emph{la fata del bambol}}}\Cendnote{\textnormal{italienisch richtig: \emph{\textcolor{green}{La fata delle bambole}}}}}\label{K_L00391_1h}}
                  \makebox[\the\longestd][l]{}\pend\pstart
           Varietés, Operetten etc. überall.\pend
           \pstart Herzlichst Ihr \spacefill\mbox{Richard.}\pend{}\pstart
           \noindent{}der sich auf Sie freut\pend
           \pstart
           \raggedleft{}\textcolor{pink}{Neapel}{}\ledrightnote{\textcolor{pink}{Neapel}}{ }23/X 94.\pend
           \endnumbering\briefempfaengerindex{Schnitzler, Arthur@\textsc{Schnitzler, Arthur}!zzzBeer-Hofmann, Richard@\emph{von Richard Beer-Hofmann}!1894-10-231@{23. 10. 1894}|)be}\mylabel{h}  \normalsize

\doendnotes{C}
\bigskip
\vfill

\clearpage

\footnotesize

\lohead{\textsc{register}}

% Definiere theindex-Environment komplett neu ohne reledmac
\makeatletter
\renewenvironment{theindex}{%
  \section*{\indexname}%
  \setlength{\parindent}{0pt}%
  \setlength{\parskip}{0pt plus 0.3pt}%
  \let\item\@idxitem
}{%
  \clearpage
}
\makeatother

\IfFileExists{\jobname-pw.ind}{\input{\jobname-pw.ind}}{}

\end{document}

      