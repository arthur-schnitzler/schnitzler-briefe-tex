%% latex-korrekturansicht-vorspann.tex
%% Vorspann für die Korrekturansicht.
%% Lädt die gemeinsame Datei latex-vorspann.tex mit gesetztem Schalter.

\newif\ifkorrekturansicht
\korrekturansichttrue

\input{../tex-inputs/latex-vorspann}


               \section[Arthur Schnitzler an Richard Beer-Hofmann, {[}zwischen 7. 10. 1891 und Ende April 1892{]}]{ Arthur Schnitzler an Richard Beer-Hofmann, {[}zwischen 7. 10. 1891 und
               Ende April 1892{]}}\nopagebreak\mylabel{v}\rehead{ }\normalsize\beginnumbering\briefempfaengerindex{Beer-Hofmann, Richard@\textsc{Beer-Hofmann, Richard}!zzzSchnitzler, Arthur@\emph{von Arthur Schnitzler}!1891-10-071@{{[}zwischen 7. 10. 1891 und
                  30. 4. 1892{]}}|(be} \toendnotes[C]{\smallbreak\pagebreak[2]} \Standort{YCGL, MSS 31.}
\physDesc{Briefkarte, Umschlag
\newline{}Handschrift: Bleistift, deutsche Kurrent\newline{}Versand: ohne postalischen Übermittlungsvermerk }\toendnotes[C]{\smallbreak}\pstart{}{\pb}\textsc{Herrn Dr. Rich Beer-Hofma{\geminationn}}\pend{}\pstart{}\textsc{\textcolor{pink}{Wien}{}\ledrightnote{\textcolor{pink}{Wien}}}\pend{}\pstart{}\textsc{\textcolor{pink}{III Seidlgasse 30}{}\ledrightnote{\textcolor{pink}{Seidlgasse}}}.\pend{}{\bigskip}\pstart{}{\pb}Lieber Richard,\pend\pstart
           Ich bin \label{K_L00043_1v}\edtext{heute}{\lemma{\textnormal{\emph{heute}}}\Cendnote{\textnormal{Das undatierte Korrespondenstück ist
                  womöglich auf den ersten Nachmittagsaufenthalt \textcolor{blue}{Beer-Hofmann}s bei \textcolor{blue}{Schnitzler},
                  jedenfalls aber frühestens auf diesen einzuordnen. Da \textcolor{blue}{Beer-Hofmann} nur bis Ende April 1892 in der \textcolor{pink}{Seidlgasse} wohnte, gibt das die hintere zeitliche
                  Grenze an.}}}\label{K_L00043_1h} Nachmittag zu Hauſe u habe auch die andern {\pb}verſtändigt. We{\geminationn} Sie
               nichts beſſeres vorhaben, ko{\geminationm}en Sie?\pend
           \pstart
           Herzlich{\\[\baselineskip]}\spacefill\mbox{Arthur}\pend
           \leftskip=0em{}\endnumbering\briefempfaengerindex{Beer-Hofmann, Richard@\textsc{Beer-Hofmann, Richard}!zzzSchnitzler, Arthur@\emph{von Arthur Schnitzler}!1891-10-071@{{[}zwischen 7. 10. 1891 und
                  30. 4. 1892{]}}|)be}\mylabel{h}  \normalsize

\doendnotes{C}
\bigskip
\vfill

\clearpage

\footnotesize

\lohead{\textsc{register}}

% Definiere theindex-Environment komplett neu ohne reledmac
\makeatletter
\renewenvironment{theindex}{%
  \section*{\indexname}%
  \setlength{\parindent}{0pt}%
  \setlength{\parskip}{0pt plus 0.3pt}%
  \let\item\@idxitem
}{%
  \clearpage
}
\makeatother

\IfFileExists{\jobname-pw.ind}{\input{\jobname-pw.ind}}{}

\end{document}

      