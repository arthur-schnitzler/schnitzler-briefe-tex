%% latex-korrekturansicht-vorspann.tex
%% Vorspann für die Korrekturansicht.
%% Lädt die gemeinsame Datei latex-vorspann.tex mit gesetztem Schalter.

\newif\ifkorrekturansicht
\korrekturansichttrue

\input{../tex-inputs/latex-vorspann}


               \section[Arthur Schnitzler an Richard Beer-Hofmann, 23. 8. 1903]{ Arthur Schnitzler an Richard Beer-Hofmann, 23. 8. 1903}\nopagebreak\mylabel{v}\rehead{ }\normalsize\beginnumbering\briefempfaengerindex{Beer-Hofmann, Richard@\textsc{Beer-Hofmann, Richard}!zzzSchnitzler, Arthur@\emph{von Arthur Schnitzler}!1903-08-231@{23. 8. 1903}|(be} \toendnotes[C]{\smallbreak\pagebreak[2]} \Standort{YCGL, MSS 31.}
\physDesc{Telegramm
\newline{}Handschrift einer Schreibkraft: Bleistift, deutsche Kurrent\newline{}Versand: »\noindent{}\textcolor{gray}{\textbf{Dienstliche Angaben.}} RP{ / }\textcolor{gray}{\textbf{Aufgegeben am}}{ }2\textcolor{gray}{0}/8 \textcolor{gray}{\textbf{190}}2{ }\textcolor{gray}{\textbf{um {\dots} Uhr {\dots} Min {\dots} Mittag}}{ / }\textcolor{gray}{\textbf{Eingelangt von}} 1584 \textcolor{pink}{Wien}{ }\textcolor{gray}{\textbf{auf Leitung Nr. {\dots} am}}{ }22/8\textcolor{gray}{\textbf{190.}}{ }\textcolor{gray}{\textbf{um}}{ }9 \textcolor{gray}{\textbf{Uhr}} 9 \textcolor{gray}{\textbf{Min. {\dots} Mittag}}{ / }\textcolor{gray}{\textbf{Aufgenommen durch}}{ }Ba{ / }\textcolor{gray}{\textbf{Von}}{ }\textcolor{pink}{Wien}{ }\textcolor{gray}{\textbf{Aufgabe-Nr.}} 9999{ }\textcolor{gray}{\textbf{mit}} 31 \textcolor{gray}{\textbf{Taxworten ({\dots} Worten {\dots} Chiffern)}}« }\buchAbdrucke{\weitereDrucke{Arthur Schnitzler, Richard Beer-Hofmann: \emph{Briefwechsel 1891–1931}. Hg. Konstanze Fliedl. Wien, Zürich: \emph{Europaverlag} 1992, S. 163.} }\toendnotes[C]{\smallbreak}\pstart{}{\pb}Richard Beer Hofma{\geminationn}\pend{}\pstart{}\textcolor{pink}{Lieſinger Straſſe 2}{}\ledrightnote{\textcolor{pink}{Liesingerstraße}}\pend{}\pstart{}\textcolor{gray}{\textbf{\textit{\textcolor{pink}{Rodaun}{}\ledrightnote{\textcolor{pink}{Rodaun}}}}}\pend{}{\bigskip}\pstart
           \noindent{}{\pb}Ich bitte Sie allſo Mittwoch{ }halb eins{ }\label{K_L01312_1v}\edtext{\textcolor{pink}{Schopenhauergaſſe 37}{}\ledrightnote{\textcolor{pink}{Schopenhauerstraße}}}{\lemma{\textnormal{\emph{Schopenhauergaſſe 37}}}\Cendnote{\textnormal{\textcolor{pink}{Schopenhauerstraße 39} war die Adresse der \textcolor{brown}{Synagoge}, in der am 26. 8. 1903 die Trauung
                  von \textcolor{blue}{Schnitzler} und \textcolor{blue}{Olga Gussmann} stattfand.}}}\label{K_L01312_1h} als Beistand zu fungiren und
               die Sache durchaus als vertraulich zu behandeln\pend
           \pstart Herzlichſt Ihr \spacefill\mbox{Arthur}\pend{}\endnumbering\briefempfaengerindex{Beer-Hofmann, Richard@\textsc{Beer-Hofmann, Richard}!zzzSchnitzler, Arthur@\emph{von Arthur Schnitzler}!1903-08-231@{23. 8. 1903}|)be}\mylabel{h}  \normalsize

\doendnotes{C}
\bigskip
\vfill

\clearpage

\footnotesize

\lohead{\textsc{register}}

% Definiere theindex-Environment komplett neu ohne reledmac
\makeatletter
\renewenvironment{theindex}{%
  \section*{\indexname}%
  \setlength{\parindent}{0pt}%
  \setlength{\parskip}{0pt plus 0.3pt}%
  \let\item\@idxitem
}{%
  \clearpage
}
\makeatother

\IfFileExists{\jobname-pw.ind}{\input{\jobname-pw.ind}}{}

\end{document}

      