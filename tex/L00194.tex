%% latex-korrekturansicht-vorspann.tex
%% Vorspann für die Korrekturansicht.
%% Lädt die gemeinsame Datei latex-vorspann.tex mit gesetztem Schalter.

\newif\ifkorrekturansicht
\korrekturansichttrue

\input{../tex-inputs/latex-vorspann}


               \section[Michael Georg Conrad an Arthur Schnitzler, 28. 3. 1893]{ Michael Georg Conrad an Arthur Schnitzler, 28. 3. 1893}\nopagebreak\mylabel{v}\rehead{ }\normalsize\beginnumbering\briefempfaengerindex{Schnitzler, Arthur@\textsc{Schnitzler, Arthur}!zzzConrad, Michael Georg@\emph{von Michael Georg Conrad}!1893-03-281@{28. 3. 1893}|(be} \toendnotes[C]{\smallbreak\pagebreak[2]} \Standort{CUL, Schnitzler, B 22.}
\physDesc{Postkarte
\newline{}Handschrift: schwarze Tinte, deutsche Kurrent\newline{}Versand: 1) Stempel: »\nobreak{}\oindex{Muenchen@\textbf{München}, \emph{https://www.geonames.org/ontologyP.PPLA}|pwk}München I, 28 Mär {[}93{]}, 7–8 N\nobreak{}«.  2) Stempel: »\nobreak{}Wien 1/\nobreak{}«. \newline{}Ordnung: mit rotem Buntstift von unbekannter Hand
                                    nummeriert: »2« }\toendnotes[C]{\smallbreak}\pstart{}{\pb}Herrn \textsc{D\textsuperscript{r}} Arthur Schnitzler\pend{}\pstart{}\textcolor{pink}{Wien I}{}\ledrightnote{\textcolor{pink}{I., Innere Stadt}}.\pend{}\pstart{}\textcolor{pink}{Grillparzerſtr. 7}{}\ledrightnote{\textcolor{pink}{Grillparzerstraße}}.\pend{}{\bigskip}\pstart
           \noindent{}{\pb}\textcolor{pink}{München}{}\ledrightnote{\textcolor{pink}{München}}, \textcolor{pink}{Steinsdorfſtr. 7}{}\ledrightnote{\textcolor{pink}{Steinsdorfstraße}}.\pend
           \pstart
           \raggedleft{}28. 3. 93.\pend
           \pstart
           Beſten Dank für Ihre Zuſchrift, ſehr geehrter Herr Doktor! Haben Sie ſeit
                        \textcolor{green}{A. L.}{}\ledrightnote{\textcolor{green}{Alkandi’s Lied}} nichts mehr veröffentlicht? Stünde
                    ich mit der Leitung d. \textcolor{brown}{Hoftheaters}{}\ledrightnote{\textcolor{brown}{Königliche Hof- und Nationaltheater München}} beſſer,
                    würde ich gern perſönlich für Ihr \textcolor{green}{Werk}{}\ledrightnote{→\textcolor{green}{Alkandi’s Lied}}
                    eintreten. Aber ich habe von dieſer Seite auch noch nichts als Kränkungen
                    erfahren. Mit hochachtungsvollem Gruße\hspace*{1.5em}Ihr ergebener\pend
           \pstart \spacefill\mbox{Dr. Conrad.}\pend{}\endnumbering\briefempfaengerindex{Schnitzler, Arthur@\textsc{Schnitzler, Arthur}!zzzConrad, Michael Georg@\emph{von Michael Georg Conrad}!1893-03-281@{28. 3. 1893}|)be}\mylabel{h}  \normalsize

\doendnotes{C}
\bigskip
\vfill

\clearpage

\footnotesize

\lohead{\textsc{register}}

% Definiere theindex-Environment komplett neu ohne reledmac
\makeatletter
\renewenvironment{theindex}{%
  \section*{\indexname}%
  \setlength{\parindent}{0pt}%
  \setlength{\parskip}{0pt plus 0.3pt}%
  \let\item\@idxitem
}{%
  \clearpage
}
\makeatother

\IfFileExists{\jobname-pw.ind}{\input{\jobname-pw.ind}}{}

\end{document}

      