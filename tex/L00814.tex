%% latex-korrekturansicht-vorspann.tex
%% Vorspann für die Korrekturansicht.
%% Lädt die gemeinsame Datei latex-vorspann.tex mit gesetztem Schalter.

\newif\ifkorrekturansicht
\korrekturansichttrue

\input{../tex-inputs/latex-vorspann}


               \section[Arthur Schnitzler an Richard Beer-Hofmann, 6. 7. 1898]{ Arthur Schnitzler an Richard Beer-Hofmann, 6. 7. 1898}\nopagebreak\mylabel{v}\rehead{ }\normalsize\beginnumbering\briefempfaengerindex{Beer-Hofmann, Richard@\textsc{Beer-Hofmann, Richard}!zzzSchnitzler, Arthur@\emph{von Arthur Schnitzler}!1898-07-061@{6. 7. 1898}|(be} \toendnotes[C]{\smallbreak\pagebreak[2]} \Standort{YCGL, MSS 31.}
\physDesc{Brief, 1 Blatt, 4 Seiten, Umschlag
\newline{}Handschrift: Bleistift, deutsche Kurrent\newline{}Versand: 1) Stempel: »\nobreak{}\oindex{IX., Alsergrund@\textbf{IX., Alsergrund}, \emph{Bezirk (A.BZK)}|pwk}Wien 9/3 72, 6. 7. 98, 4–5N\nobreak{}«.  2) Stempel: »\nobreak{}\oindex{Steindorf am Ossiacher See@\textbf{Steindorf am Ossiacher See}, \emph{http://www.geonames.org/ontologyA.ADM3}|pwk}{\pb}Steindorf am
                              Ossiacher See, 7 7 98\nobreak{}«. }\buchAbdrucke{\weitereDrucke{Arthur Schnitzler, Richard Beer-Hofmann: \emph{Briefwechsel 1891–1931}. Hg. Konstanze Fliedl. Wien, Zürich: \emph{Europaverlag} 1992, S. 122.} }\toendnotes[C]{\smallbreak}\pstart{}{\pb}Herrn \textsc{Dr. Rich.
                     Beer-Hofmann}\pend{}\pstart{}\textsc{\textcolor{pink}{Steindorf}{}\ledrightnote{\textcolor{pink}{Steindorf am Ossiacher See}}}\pend{}\pstart{}\textsc{am }\textcolor{pink}{\textsc{Ossiacher}ſee}{}\ledrightnote{\textcolor{pink}{Ossiacher See}}\pend{}\pstart{}\textcolor{pink}{\textsc{Kärnthen}}{}\ledrightnote{\textcolor{pink}{Kärnten}}\pend{}{\bigskip}\pstart
           \raggedleft{}{\pb}6/7. 98\pend
           \pstart
           Mein lieber Richard, das iſt aber wirklich Verfolgungswahn. Man ka{\geminationn} unmöglich ernſthaft darüber reden. Ich habe nach Ihrem
               Telegr das lautete Nr. 16, 1. Juni, ſowohl \strikeout{mir} Nr 16, als 1. Juni{ }ſchicken laſſen – was mir umſo leichter war als \textcolor{brown}{\textsc{Eisenstein}}{}\ledrightnote{\textcolor{brown}{J. Eisenstein {\kaufmannsund} Co.}} beide Nrn gleich auf Ihre Rechnung ſchrieb. – \pend
           \pstart
           {\pb}– Sie ſcheinen im ganzen nervöſer zu ſein, als ich
               gern hören möchte; vielleicht haben Sie doch Luſt, mich ſo zwiſchen 20.
               u 26. Juli irgendwo im \textcolor{pink}{Salzburg}{}\ledrightnote{\textcolor{pink}{Salzburg (Land)}}iſchen
               zu treffen? Der Auguſt iſt mir noch verſchwo{\geminationm}en. \textcolor{blue}{Hugo}{}\ledrightnote{\textcolor{blue}{Hugo von Hofmannsthal}} hat erſt vom 9. Auguſt an
               Zeit – wir möchten gern in die \textcolor{pink}{Schweiz}{}\ledrightnote{\textcolor{pink}{Schweiz}}; überlegen
               Sie ſich das. –\pend
           \pstart
           {\pb}– Die \textcolor{green}{3 Einakter}{}\ledrightnote{→\textcolor{green}{Der grüne Kakadu – Paracelsus – Die Gefährtin. Drei Einakter}} heißen: \textcolor{green}{Paracelſus}{}\ledrightnote{\textcolor{green}{Paracelsus. Versspiel in einem Akt}}, \textcolor{green}{Die Gefährtin}{}\ledrightnote{\textcolor{green}{Die Gefährtin. Schauspiel in einem Akt}}, \textcolor{green}{Der
                  grüne Kakadu}{}\ledrightnote{\textcolor{green}{Der grüne Kakadu. Groteske in einem Akt}}. Die beiden erſten (\textcolor{green}{P.}{}\ledrightnote{\textcolor{green}{Paracelsus. Versspiel in einem Akt}} in
               Verſen) hab ich \textcolor{blue}{Hugo}{}\ledrightnote{\textcolor{blue}{Hugo von Hofmannsthal}} Nachts vor ſeiner Abreiſe
               nach \textcolor{pink}{Czortkow}{}\ledrightnote{\textcolor{pink}{Tschortkiw}}{ }\label{K_L00814_1v}\edtext{vorgeleſen}{\lemma{\textnormal{\emph{vorgeleſen}}}\Cendnote{\textnormal{vgl. A. S.: \emph{Tagebuch}, 28. 6. 1898}}}\label{K_L00814_1h}; ſie
               ſcheinen – nein, nein, ſie haben ihm ſehr gut gefallen – insbeſondre im \textcolor{green}{P.}{}\ledrightnote{\textcolor{green}{Paracelsus. Versspiel in einem Akt}} findet er auch nicht eine Zeile zu ändern.\pend
           \pstart
           – Mein neues \textcolor{green}{Stück}{}\ledrightnote{→\textcolor{green}{Der Schleier der Beatrice. Schauspiel in fünf Akten}} hat
               unterdeſſen ſonderbare Wandlungen durchgemacht – {\pb}es
               ſpielt wo anders u zu einer andren Zeit, als ich anfangs vermuthete; – jetzt iſt es
               aber dort, wo es ſein ſoll. (5 Akte.) Ich möchte es im Sommer ſchreiben, auf der
               Reiſe, freue mich ſehr darauf.\pend
           \pstart
           – Die Arbeit bedeutet alles mögliche für mich – nicht \uuline{die}, sondern die \uuline{\edtext{Arbeit}{\Cendnote{dreifach unterstrichen}}}.\pend
           \pstart
           – Einen Traum von \label{K_L00814_2v}\edtext{Flirt}{\lemma{\textnormal{\emph{Flirt}}}\Cendnote{\textnormal{\textcolor{blue}{Beer-Hofmann}s Hund}}}\label{K_L00814_2h} will ich Ihnen nicht erzählen; ſchreiben Sie
               mir bald, dſs es Ihnen und dem \textcolor{green}{Götterliebling}{}\ledrightnote{\textcolor{green}{Der Tod Georgs}} und
               den Ihren gut geht. Von Herzen Ihr \spacefill\mbox{Arthur.}\pend
           \endnumbering\briefempfaengerindex{Beer-Hofmann, Richard@\textsc{Beer-Hofmann, Richard}!zzzSchnitzler, Arthur@\emph{von Arthur Schnitzler}!1898-07-061@{6. 7. 1898}|)be}\mylabel{h}  \normalsize

\doendnotes{C}
\bigskip
\vfill

\clearpage

\footnotesize

\lohead{\textsc{register}}

% Definiere theindex-Environment komplett neu ohne reledmac
\makeatletter
\renewenvironment{theindex}{%
  \section*{\indexname}%
  \setlength{\parindent}{0pt}%
  \setlength{\parskip}{0pt plus 0.3pt}%
  \let\item\@idxitem
}{%
  \clearpage
}
\makeatother

\IfFileExists{\jobname-pw.ind}{\input{\jobname-pw.ind}}{}

\end{document}

      