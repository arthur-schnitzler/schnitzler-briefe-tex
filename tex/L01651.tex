%% latex-korrekturansicht-vorspann.tex
%% Vorspann für die Korrekturansicht.
%% Lädt die gemeinsame Datei latex-vorspann.tex mit gesetztem Schalter.

\newif\ifkorrekturansicht
\korrekturansichttrue

\input{../tex-inputs/latex-vorspann}


               \section[Hugo von Hofmannsthal an Arthur Schnitzler, {[}14. 1. 1907{]}]{ Hugo von Hofmannsthal an Arthur Schnitzler, {[}14. 1. 1907{]}}\nopagebreak\mylabel{v}\rehead{ }\normalsize\beginnumbering\briefempfaengerindex{Schnitzler, Arthur@\textsc{Schnitzler, Arthur}!zzzHofmannsthal, Hugo von@\emph{von Hugo von Hofmannsthal}!1907-01-141@{{[}14. 1. 1907{]}}|(be} \toendnotes[C]{\smallbreak\pagebreak[2]} \Standort{CUL, Schnitzler, B 43.}
\physDesc{Brief, 1 Blatt, 2 Seiten
\newline{}Handschrift: schwarze Tinte, deutsche Kurrent
\newline{}Schnitzler: mit Bleistift datiert: »14/1 907« \newline{}Ordnung: 1) mit Bleistift von unbekannter Hand nummeriert: »\strikeout{264}« 2) mit Bleistift von unbekannter Hand nummeriert:
                                    »270«}\buchAbdrucke{\weitereDrucke{Hugo von Hofmannsthal, Arthur Schnitzler: \emph{Briefwechsel}. Hg. Therese Nickl und Heinrich Schnitzler. Frankfurt am Main: \emph{S. Fischer} 1964, S. 226.} }\toendnotes[C]{\smallbreak}\pstart{}{\pb}mein lieber Arthur\pend\pstart
           es iſt mir natürlich äußerſt zuwider, gerade Ihnen auf einen directen Wunſch \strikeout{ſie} »nein« zu ſagen, aber das geht abſolut nicht\pend
           \pstart
           1.) (und das dürfte ſchon hinreichen) bin ich 2\textsuperscript{te}
                  Hälfte Februar fort\pend
           \pstart
           2.) habe ich mir präcis vorgeno{\geminationm}en, wohl noch Vorträge
               zu halten nie mehr aber verſa{\geminationm}elten Schweinen meine
               schönen Werke vorzuleſen\pend
           \pstart
           {\pb}3 würde ein öffentliches Leſen
               (wenn auch \label{K_L01651_1v}\edtext{zu wohlthätigem Zweck}{\lemma{\textnormal{\emph{zu wohlthätigem Zweck}}}\Cendnote{\textnormal{Am 10. 2. 1907 lasen \textcolor{blue}{Jakob Wassermann} seinen Aufsatz \emph{\textcolor{green}{Das Los der Juden}}, \textcolor{blue}{Richard
                     Beer-Hofmann} Gedichte (darunter \emph{\textcolor{green}{Schlaflied
                     für Mirjam}}), \textcolor{blue}{Felix Salten} seine Novelle
                     \emph{\textcolor{green}{Der Ernst des Lebens}} sowie \textcolor{blue}{Schnitzler}{ }\emph{\textcolor{green}{Lieutenant Gustl}} vor.}}}\label{K_L01651_1h}) die Demonſtration
               die in meiner jetzigen \label{K_L01651_2v}\edtext{kl. \textcolor{green}{Veranſtaltung}{}\ledrightnote{→\textcolor{green}{Der Dichter und diese Zeit}}}{\lemma{\textnormal{\emph{kl. Veranſtaltung}}}\Cendnote{\textnormal{Am 17. 1. 1907 hielt \textcolor{blue}{Hofmannsthal} den Vortrag \emph{\textcolor{green}{Der Dichter und diese Zeit}} im \emph{\textcolor{brown}{Kunstsalon Miethke}} vor geladenen, zehn Kronen zahlenden Gästen.}}}\label{K_L01651_2h}
               liegt (Hinauswurf von Preſſe und Premièrenpack) geradezu auf den Kopf ſtellen.\pend
           \pstart
           Ihr{\\[\baselineskip]}\spacefill\mbox{Hugo.}\pend
           \leftskip=0em{}\endnumbering\briefempfaengerindex{Schnitzler, Arthur@\textsc{Schnitzler, Arthur}!zzzHofmannsthal, Hugo von@\emph{von Hugo von Hofmannsthal}!1907-01-141@{{[}14. 1. 1907{]}}|)be}\mylabel{h}  \normalsize

\doendnotes{C}
\bigskip
\vfill

\clearpage

\footnotesize

\lohead{\textsc{register}}

% Definiere theindex-Environment komplett neu ohne reledmac
\makeatletter
\renewenvironment{theindex}{%
  \section*{\indexname}%
  \setlength{\parindent}{0pt}%
  \setlength{\parskip}{0pt plus 0.3pt}%
  \let\item\@idxitem
}{%
  \clearpage
}
\makeatother

\IfFileExists{\jobname-pw.ind}{\input{\jobname-pw.ind}}{}

\end{document}

      