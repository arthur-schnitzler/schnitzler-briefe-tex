%% latex-korrekturansicht-vorspann.tex
%% Vorspann für die Korrekturansicht.
%% Lädt die gemeinsame Datei latex-vorspann.tex mit gesetztem Schalter.

\newif\ifkorrekturansicht
\korrekturansichttrue

\input{../tex-inputs/latex-vorspann}


               \section[Arthur Schnitzler an Richard Beer-Hofmann, 7. 9. 1896]{ Arthur Schnitzler an Richard Beer-Hofmann, 7. 9. 1896}\nopagebreak\mylabel{v}\rehead{ }\normalsize\beginnumbering\briefempfaengerindex{Beer-Hofmann, Richard@\textsc{Beer-Hofmann, Richard}!zzzSchnitzler, Arthur@\emph{von Arthur Schnitzler}!1896-09-071@{7. 9. 1896}|(be} \toendnotes[C]{\smallbreak\pagebreak[2]} \Standort{YCGL, MSS 31.}
\physDesc{Kartenbrief
\newline{}Handschrift: Bleistift, deutsche Kurrent\newline{}Versand: 1) Stempel: »\nobreak{}\oindex{I., Innere Stadt@\textbf{I., Innere Stadt}, \emph{Bezirk (A.BZK)}|pwk}Wien 1/{[}1{]}, 8. 9. {[}96{]}, 8–9 {[}V{]}\nobreak{}«.  2) Stempel: »\nobreak{}\oindex{Baden bei Wien@\textbf{Baden bei Wien}, \emph{Besiedelter Ort (A.BSO)}|pwk}Baden 1, 8. 9. 96, 11–2N, Bestellt\nobreak{}«. }\buchAbdrucke{\weitereDrucke{Arthur Schnitzler, Richard Beer-Hofmann: \emph{Briefwechsel 1891–1931}. Hg. Konstanze Fliedl. Wien, Zürich: \emph{Europaverlag} 1992, S. 95–96.} }\toendnotes[C]{\smallbreak}\pstart{}{\pb}Herrn \textsc{Dr. Rich.
                     Beer-Hofmann}\pend{}\pstart{}\textsc{\textcolor{pink}{Baden bei Wien}{}\ledrightnote{\textcolor{pink}{Baden bei Wien}}}\pend{}\pstart{}\textsc{\textcolor{pink}{Franzensgassse 54}{}\ledrightnote{\textcolor{pink}{Kaiser-Franz-Ring}}}, Thür 8\pend{}{\bigskip}\pstart
           \raggedleft{}{\pb}Montag\pend
           \pstart
           Lieber Richard, Ihre Karte hab ich bekommen. Morgen wollte ich zu
               Ihnen; aber plötzlich iſt \textcolor{blue}{\textsc{Sorma}}{}\ledrightnote{\textcolor{blue}{Agnes Sorma}} u \textcolor{blue}{Gemahl}{}\ledrightnote{→\textcolor{blue}{Demetrius Mito von Minotto}} in \textcolor{pink}{Wien}{}\ledrightnote{\textcolor{pink}{Wien}} und ich ſpeiſe morgen mit ihnen. Ich ka{\geminationn} Ihnen alſo noch nicht genau ſagen, wann ich nach \textcolor{pink}{Baden}{}\ledrightnote{\textcolor{pink}{Baden bei Wien}} fahre. Wie lange bleiben Sie noch draußen?
               Arbeiten Sie? Haben Sie mit \textcolor{blue}{Fiſcher}{}\ledrightnote{\textcolor{blue}{Samuel Fischer}}, mit \textcolor{blue}{Brahm}{}\ledrightnote{\textcolor{blue}{Otto Brahm}} geſprochen? – Von \textcolor{blue}{Hugo}{}\ledrightnote{\textcolor{blue}{Hugo von Hofmannsthal}} weiſs ich auch nichts, vor 8 Tagen hab ich ihm nach \textcolor{pink}{Alt-Auſſee}{}\ledrightnote{\textcolor{pink}{Altaussee}} geſchrieben. – \textcolor{blue}{Burckhard}{}\ledrightnote{\textcolor{blue}{Max Eugen Burckhard}} hat \textcolor{green}{Freiwild}{}\ledrightnote{\textcolor{green}{Freiwild. Schauspiel in 3 Akten}}
               geleſen u gratulirt \textcolor{blue}{Brahm}{}\ledrightnote{\textcolor{blue}{Otto Brahm}}, ders aufführen darf;
               hälts für den »pupillarſichern Senſationserfolg{[}«{]}, fährt nach \textcolor{pink}{Berlin}{}\ledrightnote{\textcolor{pink}{Berlin}} zur \textsc{Première}. –\pend
           \pstart
           Herzlich Ihr{\\[\baselineskip]}\spacefill\mbox{Arthur}\pend
           \leftskip=0em{}\endnumbering\briefempfaengerindex{Beer-Hofmann, Richard@\textsc{Beer-Hofmann, Richard}!zzzSchnitzler, Arthur@\emph{von Arthur Schnitzler}!1896-09-071@{7. 9. 1896}|)be}\mylabel{h}  \normalsize

\doendnotes{C}
\bigskip
\vfill

\clearpage

\footnotesize

\lohead{\textsc{register}}

% Definiere theindex-Environment komplett neu ohne reledmac
\makeatletter
\renewenvironment{theindex}{%
  \section*{\indexname}%
  \setlength{\parindent}{0pt}%
  \setlength{\parskip}{0pt plus 0.3pt}%
  \let\item\@idxitem
}{%
  \clearpage
}
\makeatother

\IfFileExists{\jobname-pw.ind}{\input{\jobname-pw.ind}}{}

\end{document}

      