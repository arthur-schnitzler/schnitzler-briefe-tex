%% latex-korrekturansicht-vorspann.tex
%% Vorspann für die Korrekturansicht.
%% Lädt die gemeinsame Datei latex-vorspann.tex mit gesetztem Schalter.

\newif\ifkorrekturansicht
\korrekturansichttrue

\input{../tex-inputs/latex-vorspann}


               \section[Arthur Schnitzler an Richard Beer-Hofmann, 15. 5. 1922]{ Arthur Schnitzler an Richard Beer-Hofmann, 15. 5. 1922}\nopagebreak\mylabel{v}\rehead{ }\normalsize\beginnumbering\briefempfaengerindex{Beer-Hofmann, Richard@\textsc{Beer-Hofmann, Richard}!zzzSchnitzler, Arthur@\emph{von Arthur Schnitzler}!1922-05-151@{15. 5. 1921}|(be} \toendnotes[C]{\smallbreak\pagebreak[2]} \Standort{YCGL, MSS 31.}
\physDesc{Bildpostkarte
\newline{}Handschrift: Bleistift, lateinische Kurrent\newline{}Versand: Stempel: »\nobreak{}\oindex{Nuernberg@\textbf{Nürnberg}, \emph{https://www.geonames.org/ontologyP.PPL}|pwk}Nürnberg, 15. 5. 22, N3–4\nobreak{}«.  }\pstart{}{\pb}Hrn Dr. Richard Beer-Hofmann\pend{}\pstart{}\textcolor{pink}{Wien XVIII}{}\ledrightnote{\textcolor{pink}{XVIII., Währing}}\pend{}\pstart{}\textcolor{pink}{Hasenauerstr 59}{}\ledrightnote{\textcolor{pink}{Hasenauerstraße}}.\pend{}{\bigskip}\pstart
           \noindent{}\centering{}{\pb}\textcolor{gray}{\textbf{\textcolor{pink}{Nürnberg}{}\ledrightnote{\textcolor{pink}{Nürnberg}}}}\pend
           \pstart
           \noindent{}\centering{}\textcolor{gray}{\textbf{Blick von der \textcolor{pink}{Museumsbrücke}{}\ledrightnote{\textcolor{pink}{Museumsbrücke Nürnberg}} zum \textcolor{pink}{Heiliggeist-Spital}{}\ledrightnote{\textcolor{pink}{Heilig-Geist-Spital}}.}}\pend
           \pstart
           \raggedleft{}{\pb}\textcolor{pink}{Nürnberg}{}\ledrightnote{\textcolor{pink}{Nürnberg}}{ }15. 5. 22\pend
           \pstart
           Herzliche Grüße!\pend
           \pstart
           Ihr{\\[\baselineskip]}\spacefill\mbox{Arthur}\pend
           \leftskip=0em{}\endnumbering\briefempfaengerindex{Beer-Hofmann, Richard@\textsc{Beer-Hofmann, Richard}!zzzSchnitzler, Arthur@\emph{von Arthur Schnitzler}!1922-05-151@{15. 5. 1921}|)be}\mylabel{h}  \normalsize

\doendnotes{C}
\bigskip
\vfill

\clearpage

\footnotesize

\lohead{\textsc{register}}

% Definiere theindex-Environment komplett neu ohne reledmac
\makeatletter
\renewenvironment{theindex}{%
  \section*{\indexname}%
  \setlength{\parindent}{0pt}%
  \setlength{\parskip}{0pt plus 0.3pt}%
  \let\item\@idxitem
}{%
  \clearpage
}
\makeatother

\IfFileExists{\jobname-pw.ind}{\input{\jobname-pw.ind}}{}

\end{document}

      