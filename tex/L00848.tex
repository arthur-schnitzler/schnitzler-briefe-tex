%% latex-korrekturansicht-vorspann.tex
%% Vorspann für die Korrekturansicht.
%% Lädt die gemeinsame Datei latex-vorspann.tex mit gesetztem Schalter.

\newif\ifkorrekturansicht
\korrekturansichttrue

\input{../tex-inputs/latex-vorspann}


               \section[Arthur Schnitzler an Georg Brandes, nicht abgesandt, 22. 9. 1898]{ Arthur Schnitzler an Georg Brandes, nicht abgesandt,
                    22. 9. 1898}\nopagebreak\mylabel{v}\rehead{ }\normalsize\beginnumbering\briefempfaengerindex{Brandes, Georg@\textsc{Brandes, Georg}!zzzSchnitzler, Arthur@\emph{von Arthur Schnitzler}!1898-09-221@{22. 9. 1898}|(be} \toendnotes[C]{\smallbreak\pagebreak[2]} \Standort{DLA, A:Schnitzler, HS.NZ85.1.440.}
\physDesc{Brief, 1 Blatt, 4 Seiten
\newline{}Handschrift: 1) schwarze Tinte, deutsche Kurrent\hspace{1em}2) Bleistift, deutsche Kurrent (\noindent{}Ergänzung: »(nicht abgesand{[}t{]}«)\hspace{1em}3) roter Buntstift (\noindent{}eine Unterstreichung)\hspace{1em}}\toendnotes[C]{\smallbreak}\pstart
           \noindent{} (nicht abgeſand{[}t{]}\pend
           \pstart{}{\pb}Verehrteſter Herr Brandes,\pend\pstart
           ich ſchicke Ihnen heute das \textcolor{green}{Stück}{}\ledrightnote{→\textcolor{green}{Das Vermächtnis. Schauspiel in drei Akten}}, welches nächſtens aufgeführt wird; es iſt das Bühnenmanuscript;
                    als Buch hab ich es noch nicht drucken laſſen, weil ich hoffe, daſs mir bei den
                    Proben noch manches einfallen wird, um den zweiten und den Beginn des 3. Aktes
                    höher zu bringen; und das erſcheint mir recht nothwendig. –\pend
           \pstart
           – Heut hab ich eine Zeitſchrift »\textcolor{green}{Das neue
                        Jahrhundert}{}\ledrightnote{\textcolor{green}{Das neue Jahrhundert}}« zugeſchickt erhalten, mit Ihrem \label{K_L00848_1v}\edtext{\textcolor{green}{Artikel}{}\ledrightnote{→\textcolor{green}{Jeanne Marni}}}{\lemma{\textnormal{\emph{Artikel}}}\Cendnote{\textnormal{\textcolor{blue}{Georg Brandes}: \emph{\textcolor{green}{Jeanne Marni}}. In: \emph{\textcolor{green}{Das
                                neue Jahrhundert}}, Jg. 1, H. 1, 1. 10. 1898,
                            S. 14–19.}}}\label{K_L00848_1h} über die \textcolor{blue}{\textsc{Marni}}{}\ledrightnote{\textcolor{blue}{Jeanne Marni}}. {\pb}Zu dieſem Artikel ſteht auch eine
                    unendlich liebenswürdige Bemerkung über mein erſtes \textcolor{green}{Buch}{}\ledrightnote{→\textcolor{green}{Anatol}}. Und doch wärs mir lieber geweſen, Sie hätten
                    geſchrieben, jenes Buch iſt nicht viel werth, aber ſein Autor hat ſpäter
                    beſſeres gemacht. Sie werden gleich wiſſen, warum ich das ſagen darf. Nach dem
                        \textcolor{green}{Anatol}{}\ledrightnote{\textcolor{green}{Anatol}} hab’ ich Ihnen das \textcolor{green}{Märchen}{}\ledrightnote{\textcolor{green}{Das Märchen. Schauspiel in drei Aufzügen}} geſchickt und da haben Sie mir geſchrieben: »Sie
                    haben hier eine viel höhere Stufe erreicht als in Ihrem früheren \textcolor{green}{Buch}{}\ledrightnote{→\textcolor{green}{Anatol}}« – und ebenſo ſchienen Sie – in
                    einem Brief an mich, wie in einer \textcolor{green}{Bemerkung}{}\ledrightnote{→\textcolor{green}{To Forestillinger af Henrik IV}}{ }{\pb}in »\textcolor{green}{\textsc{Politiken}}{}\ledrightnote{\textcolor{green}{Politiken}}« die »\textcolor{green}{Liebelei}{}\ledrightnote{\textcolor{green}{Liebelei. Schauspiel in drei Akten}}« höher zu ſchätzen als
                    die frühern Sachen. – Und heute ſteht in Ihrem \textcolor{green}{Artikel}{}\ledrightnote{→\textcolor{green}{Jeanne Marni}} – »Sch. hat die Fähigkeit, die er hier \introOben{}\textcolor{green}{(Anatol)}{}\ledrightnote{\textcolor{green}{Anatol}}\introOben{} bewieſen, nicht weiterentwickelt.« – Ich glaube nicht, daſs es dumme
                    Empfindlichkeit iſt wenn mich dieſe Bemerkung ein bischen verſti{\geminationm}t hat – denn von Menſchen, deren Urtheil uns hoch
                    ſteht, möchten wir alles hören – nur nicht; daſs ſie uns ſtehen bleiben oder gar
                    herunter ſteigen ſehen. Es ist ja wirklich \substVorne{}\textsuperscript{das}\substDazwischen{}nicht\substHinten{} weſentlicher, daſs wir gelegentlich was anſtändges ſchreiben, ſondern
                        {\pb}daſs wir uns in ſteter Entwicklung befinden –
                    und, wie Sie ſehen, hatte ich nicht Urſache zu glauben, daſs Sie \uline{gerade} das bei mir zu bemerken meinen – und ich
                    bin vielleicht ein wenig ſtolz darauf geweſen.\pend
           \pstart
           Darum, mein verehrter Herr Brandes, müſſen Sie mir verzeihen, daſs ich Ihnen
                    heute dieſen möglicherweiſe kindiſchen Brief ſchreibe; ich werde mich
                    wahrſcheinlich morgen ſchon ſeiner ſchämen.\pend
           \pstart Seien Sie in herzlicher Ergebenheit gegrüßt von Ihrem
                        \spacefill\mbox{ArthurSchnitzler}\pend{}\pstart
           \textcolor{pink}{Wien}{}\ledrightnote{\textcolor{pink}{Wien}}{ }22. 9. 98.\pend
           \endnumbering\briefempfaengerindex{Brandes, Georg@\textsc{Brandes, Georg}!zzzSchnitzler, Arthur@\emph{von Arthur Schnitzler}!1898-09-221@{22. 9. 1898}|)be}\mylabel{h}  \normalsize

\doendnotes{C}
\bigskip
\vfill

\clearpage

\footnotesize

\lohead{\textsc{register}}

% Definiere theindex-Environment komplett neu ohne reledmac
\makeatletter
\renewenvironment{theindex}{%
  \section*{\indexname}%
  \setlength{\parindent}{0pt}%
  \setlength{\parskip}{0pt plus 0.3pt}%
  \let\item\@idxitem
}{%
  \clearpage
}
\makeatother

\IfFileExists{\jobname-pw.ind}{\input{\jobname-pw.ind}}{}

\end{document}

      