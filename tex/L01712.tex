%% latex-korrekturansicht-vorspann.tex
%% Vorspann für die Korrekturansicht.
%% Lädt die gemeinsame Datei latex-vorspann.tex mit gesetztem Schalter.

\newif\ifkorrekturansicht
\korrekturansichttrue

\input{../tex-inputs/latex-vorspann}


               \section[Arthur Schnitzler an Stefan Großmann, 28. 9. 1907]{ Arthur Schnitzler an Stefan Großmann, 28. 9. 1907}\nopagebreak\mylabel{v}\rehead{ }\normalsize\beginnumbering\briefempfaengerindex{Grossmann, Stefan@\textsc{Großmann, Stefan}!zzzSchnitzler, Arthur@\emph{von Arthur Schnitzler}!1907-09-281@{28. 9. 1907}|(be} \toendnotes[C]{\smallbreak\pagebreak[2]} \Standort{DLA, A:Schnitzler, HS.NZ85.1.896.}
\physDesc{Brief, 1 Blatt, 1 Seite, maschineller Durchschlag
\newline{}Schreibmaschine
\newline{}Handschrift: roter Buntstift, deutsche Kurrent (\noindent{}Korrektur eines Satzzeichens, eine Unterstreichung)}\toendnotes[C]{\smallbreak}\pstart
           \raggedleft{}{\pb}28. Sept. 07. \pend
           \pstart{}Sehr geehrter Herr Grossmann,\pend\pstart
           Ihre freundliche Einladun\textcolor{gray}{g} an einem Abend vor Mitgliedern der \textcolor{brown}{freien Volksbühne}{}\ledrightnote{\textcolor{brown}{Wiener Freie Volksbühne}} zu lesen nehme ich gern an. Nur
               bitte ich Sie einen kleinen Saal zu wählen, von einem Fassungsraum für höchstens
               fünf- bis sechshundert Personen, da meine Stimme in ei\textcolor{gray}{nem} grössern
               Saale nicht weit genug trägt. Auch glaub ich nicht, dass ich mit meinen Stimmmitteln
               einen Abend allein bestreiten kann, wenigstens einen, der länger \label{T_L01712_1v}\edtext{währte}{\lemma{\textnormal{\emph{währte}}}\Cendnote{\textnormal{geschrieben: »wehrte«}}}\label{T_L01712_1h}, als eine Stunde. Vielleicht
               arrangieren Sie es so, dass noch ein zweiter Autor am gleichen Abend liest\substVorne{}\textsuperscript{.}\substDazwischen{}?\substHinten{} Wollen Sie mir nicht auch einen Vorschlag hinsichtlich des Programms
               machen?\pend
           \pstart
           Mit vorzüglicher Hochachtung{\\[\baselineskip]}Ihr ergebener\pend
           \leftskip=0em{}{\bigskip}\pstart
           \noindent{}Herrn Stefan Grossmann, \textcolor{pink}{Wien}{}\ledrightnote{\textcolor{pink}{Wien}}\pend
           \endnumbering\briefempfaengerindex{Grossmann, Stefan@\textsc{Großmann, Stefan}!zzzSchnitzler, Arthur@\emph{von Arthur Schnitzler}!1907-09-281@{28. 9. 1907}|)be}\mylabel{h}  \normalsize

\doendnotes{C}
\bigskip
\vfill

\clearpage

\footnotesize

\lohead{\textsc{register}}

% Definiere theindex-Environment komplett neu ohne reledmac
\makeatletter
\renewenvironment{theindex}{%
  \section*{\indexname}%
  \setlength{\parindent}{0pt}%
  \setlength{\parskip}{0pt plus 0.3pt}%
  \let\item\@idxitem
}{%
  \clearpage
}
\makeatother

\IfFileExists{\jobname-pw.ind}{\input{\jobname-pw.ind}}{}

\end{document}

      