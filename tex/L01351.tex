%% latex-korrekturansicht-vorspann.tex
%% Vorspann für die Korrekturansicht.
%% Lädt die gemeinsame Datei latex-vorspann.tex mit gesetztem Schalter.

\newif\ifkorrekturansicht
\korrekturansichttrue

\input{../tex-inputs/latex-vorspann}


               \section[Arthur Schnitzler u. a. an Hermann Bahr, 14. 12. 1903]{ Arthur Schnitzler u. a. an Hermann Bahr, 14. 12. 1903}\nopagebreak\mylabel{v}\rehead{ }\normalsize\beginnumbering\briefempfaengerindex{Bahr, Hermann@\textsc{Bahr, Hermann}!zzzBeer-Hofmann, Richard@\emph{von Richard Beer-Hofmann}!1903-12-142@{14. 12. 1903}|(be}\briefempfaengerindex{Bahr, Hermann@\textsc{Bahr, Hermann}!zzzHofmannsthal, Gertrude von@\emph{von Gertrude von Hofmannsthal}!1903-12-142@{14. 12. 1903}|(be}\briefempfaengerindex{Bahr, Hermann@\textsc{Bahr, Hermann}!zzzHofmannsthal, Hugo von@\emph{von Hugo von Hofmannsthal}!1903-12-142@{14. 12. 1903}|(be}\briefempfaengerindex{Bahr, Hermann@\textsc{Bahr, Hermann}!zzzSchnitzler, Olga@\emph{von Olga Schnitzler}!1903-12-142@{14. 12. 1903}|(be}\briefempfaengerindex{Bahr, Hermann@\textsc{Bahr, Hermann}!zzzSchnitzler, Arthur@\emph{von Arthur Schnitzler}!1903-12-142@{14. 12. 1903}|(be} \toendnotes[C]{\smallbreak\pagebreak[2]} \Standort{TMW, HS AM 49103 Ba.}
\physDesc{Bildpostkarte
\newline{}Handschrift Arthur Schnitzler: Bleistift, deutsche Kurrent\newline{}Handschrift Olga Schnitzler: Bleistift\newline{}Handschrift Hugo von Hofmannsthal: Bleistift, lateinische Kurrent\newline{}Handschrift Gertrude von Hofmannsthal: Bleistift\newline{}Handschrift Richard Beer-Hofmann: Bleistift\newline{}Versand: 1) Stempel: »\nobreak{}\oindex{XVIII., Waehring@\textbf{XVIII., Währing}, \emph{Bezirk (A.BZK)}|pwk}18/1 Wien, 15. 12. 03, 6–7 V\nobreak{}«.  2) Stempel: »\nobreak{}Bestellt, \oindex{XIII., Hietzing@\textbf{XIII., Hietzing}, \emph{Bezirk (A.BZK)}|pwk}Wien 13/7, 15{[}.{]} 12. 03, 11.\nobreak{}«. \newline{}Ordnung: Lochung }\buchAbdrucke{\weitereDrucke{1) Hugo von Hofmannsthal, Gerty von Hofmannsthal, Hermann Bahr: \emph{Briefwechsel 1891–1934}. Hg. und kommentiert von Elsbeth Dangel-Pelloquin. Göttingen: \emph{Wallstein} 2013, S. 229.} \weitereDrucke{2) Hermann Bahr, Arthur Schnitzler: \emph{Briefwechsel, Aufzeichnungen, Dokumente (1891–1931)}. Hg. Kurt Ifkovits und Martin Anton Müller. Göttingen: \emph{Wallstein} 2018, S. 286.} }\toendnotes[C]{\smallbreak}\pstart{}{\pb}Herrn Hermann
                  Bahr\pend{}\pstart{}\textcolor{pink}{Wien Ob St Veit}{}\ledrightnote{\textcolor{pink}{Ober Sankt Veit}}\pend{}\pstart{}\textcolor{pink}{Veitliſſengaſſe}{}\ledrightnote{\textcolor{pink}{Veitlissengasse}}\pend{}{\bigskip}\pstart
           \noindent{}\centering{}{\pb}\textcolor{gray}{\textbf{\textcolor{pink}{Hietzing}{}\ledrightnote{\textcolor{pink}{XIV., Penzing}}, den {\dots} 19{\dots}}}\pend
           \pstart
           \noindent{}\centering{}\textcolor{gray}{\textbf{\textcolor{pink}{Auhofstrasse 1}{}\ledrightnote{\textcolor{pink}{Auhofstraße}}.}}\pend
           \pstart
           \noindent{}\textcolor{gray}{\textbf{Schöner Gruss vom reizenden Etablissement ›\textcolor{pink}{Ottakringerbräu}{}\ledrightnote{\textcolor{pink}{Ottakringer Bräu}}‹ ›zum schwarzen Hahn‹, wo es so gemüthlich
                     ist.}}\pend
           \pstart
           {\pb}\textcolor{green}{Dem Meister}{}\ledrightnote{\textcolor{green}{Der Meister}}. – 10{\%}!\pend
           \pstart
           14. 12. 03\pend
           \pstart
           \label{K_L01351_1v}\edtext{\textcolor{green}{Unter sich}{}\ledrightnote{\textcolor{green}{Unter sich. Ein Arme-Leut’-Stück}}}{\lemma{\textnormal{\emph{Unter sich}}}\Cendnote{\textnormal{Die \emph{\textcolor{brown}{11 Scharfrichter}} hatten \textcolor{blue}{Bahrs}{ }Schauspiel \emph{\textcolor{green}{Unter
                     sich}} am 9. 12. 1903 in \textcolor{pink}{Wien}
                  gegeben.}}}\label{K_L01351_1h}:\pend
           \pstart \spacefill\mbox{Arthur}\pend{}\pstart
           \noindent{}\spacefill\mbox{{[}hs. O. Schnitzler:{]} Olga S.}{\\}\spacefill\mbox{{[}hs. Hofmannsthal:{]} Hugo (ich bin nicht der \textcolor{green}{\label{K_L01351_2v}\edtext{Onkel}{\lemma{\textnormal{\emph{Onkel}}}\Cendnote{\textnormal{Figur aus \emph{\textcolor{green}{Unter
                           sich}}.}}}\label{K_L01351_2h}}{}\ledrightnote{→\textcolor{green}{Unter sich. Ein Arme-Leut’-Stück}})}{\\}\spacefill\mbox{{[}hs. G. Hofmannsthal:{]} Gerty}{\\}\spacefill\mbox{{[}hs. Beer-Hofmann:{]} Richard}\pend
           \endnumbering\briefempfaengerindex{Bahr, Hermann@\textsc{Bahr, Hermann}!zzzBeer-Hofmann, Richard@\emph{von Richard Beer-Hofmann}!1903-12-142@{14. 12. 1903}|)be}\briefempfaengerindex{Bahr, Hermann@\textsc{Bahr, Hermann}!zzzHofmannsthal, Gertrude von@\emph{von Gertrude von Hofmannsthal}!1903-12-142@{14. 12. 1903}|)be}\briefempfaengerindex{Bahr, Hermann@\textsc{Bahr, Hermann}!zzzHofmannsthal, Hugo von@\emph{von Hugo von Hofmannsthal}!1903-12-142@{14. 12. 1903}|)be}\briefempfaengerindex{Bahr, Hermann@\textsc{Bahr, Hermann}!zzzSchnitzler, Olga@\emph{von Olga Schnitzler}!1903-12-142@{14. 12. 1903}|)be}\briefempfaengerindex{Bahr, Hermann@\textsc{Bahr, Hermann}!zzzSchnitzler, Arthur@\emph{von Arthur Schnitzler}!1903-12-142@{14. 12. 1903}|)be}\mylabel{h}  \normalsize

\doendnotes{C}
\bigskip
\vfill

\clearpage

\footnotesize

\lohead{\textsc{register}}

% Definiere theindex-Environment komplett neu ohne reledmac
\makeatletter
\renewenvironment{theindex}{%
  \section*{\indexname}%
  \setlength{\parindent}{0pt}%
  \setlength{\parskip}{0pt plus 0.3pt}%
  \let\item\@idxitem
}{%
  \clearpage
}
\makeatother

\IfFileExists{\jobname-pw.ind}{\input{\jobname-pw.ind}}{}

\end{document}

      