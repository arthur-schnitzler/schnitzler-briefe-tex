%% latex-korrekturansicht-vorspann.tex
%% Vorspann für die Korrekturansicht.
%% Lädt die gemeinsame Datei latex-vorspann.tex mit gesetztem Schalter.

\newif\ifkorrekturansicht
\korrekturansichttrue

\input{../tex-inputs/latex-vorspann}


               \section[Arthur Schnitzler: Widmungsexemplar Anatol für Hugo von Hofmannsthal, 4. 12. 1900]{ Arthur Schnitzler: Widmungsexemplar Anatol für Hugo von
                    Hofmannsthal, 4. 12. 1900}\nopagebreak\mylabel{v}\rehead{ }\normalsize\beginnumbering\briefempfaengerindex{Hofmannsthal, Hugo von@\textsc{Hofmannsthal, Hugo von}!zzzSchnitzler, Arthur@\emph{von Arthur Schnitzler}!1900-12-041@{4. 12. 1900}|(be} \toendnotes[C]{\smallbreak\pagebreak[2]} \Standort{FDH, FDH 258.}
\physDesc{Widmung am Vortitel
\newline{}Handschrift: schwarze Tinte, deutsche Kurrent\newline{}Ordnung: mit Bleistift von unbekannter Hand beschriftet: »HvH-S. CI, 48« }\buchAbdrucke{\weitereDrucke{Hugo von Hofmannsthal: \emph{Bibliothek}. Hg. Ellen Ritter † in Zusammenarbeit mit Dalia Bukauskaité und
                                Konrad Heumann. Frankfurt am Main: \emph{S. Fischer} 2011, S. 603 (Sämtliche Werke. Kritische Ausgabe, XL).} }\pstart
           \noindent{}{\pb}Meinem lieben Hugo\pend
           \pstart \spacefill\mbox{Arthur}\pend{}\pstart
           \textcolor{pink}{Wien}{}\ledrightnote{\textcolor{pink}{Wien}}{ }4. Dezember 900.\pend
           {\bigskip}\pstart
           \noindent{}\centering{}\textcolor{gray}{\textbf{\textcolor{green}{ANATOL}{}\ledrightnote{\textcolor{green}{Anatol}}}}\pend
           {\bigskip}\pstart
           \noindent{}\centering{}{\pb}\textcolor{gray}{\textbf{ARTHUR SCHNITZLER}}\pend
           \pstart
           \noindent{}\centering{}\textcolor{gray}{\textbf{\textcolor{green}{ANATOL}{}\ledrightnote{\textcolor{green}{Anatol}}}}\pend
           \pstart
           \noindent{}\centering{}\textcolor{gray}{\textbf{illustriert von}}\pend
           \pstart
           \noindent{}\centering{}\textcolor{gray}{\textbf{\textcolor{blue}{M. COSCHELL.}{}\ledrightnote{\textcolor{blue}{Moritz Coschell}}}}\pend
           {\bigskip}\pstart
           \noindent{}\centering{}\textcolor{gray}{\textbf{Erste bis dritte Auflage}}\pend
           \pstart
           \noindent{}\centering{}\textcolor{gray}{\textbf{\textcolor{pink}{Berlin}{}\ledrightnote{\textcolor{pink}{Berlin}}}}\pend
           \pstart
           \noindent{}\centering{}\textcolor{gray}{\textbf{\textcolor{brown}{\so{S. Fiſcher, Verlag}}{}\ledrightnote{\textcolor{brown}{S. Fischer Verlag}}}}\pend
           \pstart
           \noindent{}\centering{}\textcolor{gray}{\textbf{1901.}}\pend
           \endnumbering\briefempfaengerindex{Hofmannsthal, Hugo von@\textsc{Hofmannsthal, Hugo von}!zzzSchnitzler, Arthur@\emph{von Arthur Schnitzler}!1900-12-041@{4. 12. 1900}|)be}\mylabel{h}  \normalsize

\doendnotes{C}
\bigskip
\vfill

\clearpage

\footnotesize

\lohead{\textsc{register}}

% Definiere theindex-Environment komplett neu ohne reledmac
\makeatletter
\renewenvironment{theindex}{%
  \section*{\indexname}%
  \setlength{\parindent}{0pt}%
  \setlength{\parskip}{0pt plus 0.3pt}%
  \let\item\@idxitem
}{%
  \clearpage
}
\makeatother

\IfFileExists{\jobname-pw.ind}{\input{\jobname-pw.ind}}{}

\end{document}

      