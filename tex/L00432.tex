%% latex-korrekturansicht-vorspann.tex
%% Vorspann für die Korrekturansicht.
%% Lädt die gemeinsame Datei latex-vorspann.tex mit gesetztem Schalter.

\newif\ifkorrekturansicht
\korrekturansichttrue

\input{../tex-inputs/latex-vorspann}


               \section[Karl Kraus an Arthur Schnitzler, 25. 4. 1895]{ Karl Kraus an Arthur Schnitzler, 25. 4. 1895}\nopagebreak\mylabel{v}\rehead{ }\normalsize\beginnumbering\briefempfaengerindex{Schnitzler, Arthur@\textsc{Schnitzler, Arthur}!zzzKraus, Karl@\emph{von Karl Kraus}!1895-04-251@{25. 4. 1895}|(be} \toendnotes[C]{\smallbreak\pagebreak[2]} \Standort{CUL, Schnitzler, B 55.}
\physDesc{Brief, 1 Blatt, 1 Seite
\newline{}Handschrift: schwarze Tinte, deutsche Kurrent}\buchAbdrucke{\weitereDrucke{\emph{Karl Kraus und Arthur Schnitzler. Eine Dokumentation.} Hg. Reinhard Urbach. In: \emph{Literatur und Kritik}, Bd. 49, Oktober 1970, S. 522.} }\toendnotes[C]{\smallbreak}\pstart
           \noindent{}{\pb}\textcolor{gray}{\textbf{KARL KRAUS}}\hfill \textcolor{gray}{\textbf{\textcolor{pink}{WIEN}{}\ledrightnote{\textcolor{pink}{Wien}}}}, 25. 4. \textcolor{gray}{\textbf{189}}5.\pend
           \pstart
           \raggedleft{}\textcolor{gray}{\textbf{\textsc{\textcolor{pink}{I. Maximilianstrasse 13}{}\ledrightnote{\textcolor{pink}{Mahlerstraße}}}}}.\pend
           \pstart{}Lieber Doktor,\pend\pstart
           zu unſerer Wette:\pend
           \pstart
           Ich erkundigte mich im Regiezimmer des \textcolor{pink}{Burgtheater}{}\ledrightnote{\textcolor{pink}{Burgtheater}}s und Herr \textcolor{blue}{\textsc{Lorai}}{}\ledrightnote{\textcolor{blue}{Christian Lorey}} hat mir folgende Auskunft
                    ertheilt:\pend
           \pstart
           »Herr \textcolor{blue}{Schreiner}{}\ledrightnote{\textcolor{blue}{Jakob Schreiner}} hat den \textcolor{green}{Lerſe}{}\ledrightnote{→\textcolor{green}{Götz von Berlichingen}} in ›\textcolor{green}{Götz v.
                        Berlichingen}{}\ledrightnote{\textcolor{green}{Götz von Berlichingen}}‹ \uuline{ſehr häufig}
                    geſpielt.«\pend
           \pstart
           – »Das ſind die kurzen Sätze. Ich kann nichts dafür. – – – – –«\pend
           \pstart
           Beſtens grüßend{\\[\baselineskip]}Ihr ganz ergebener{\\[\baselineskip]}\spacefill\mbox{KarlKraus}\pend
           \leftskip=0em{}\pstart
           \noindent{}\textsc{NB}. Herr \textcolor{blue}{\textsc{Lorai}}{}\ledrightnote{\textcolor{blue}{Christian Lorey}} wird Ihnen die mir gegebenen
                        Auskünfte gerne wiederholen.\pend
           \endnumbering\briefempfaengerindex{Schnitzler, Arthur@\textsc{Schnitzler, Arthur}!zzzKraus, Karl@\emph{von Karl Kraus}!1895-04-251@{25. 4. 1895}|)be}\mylabel{h}  \normalsize

\doendnotes{C}
\bigskip
\vfill

\clearpage

\footnotesize

\lohead{\textsc{register}}

% Definiere theindex-Environment komplett neu ohne reledmac
\makeatletter
\renewenvironment{theindex}{%
  \section*{\indexname}%
  \setlength{\parindent}{0pt}%
  \setlength{\parskip}{0pt plus 0.3pt}%
  \let\item\@idxitem
}{%
  \clearpage
}
\makeatother

\IfFileExists{\jobname-pw.ind}{\input{\jobname-pw.ind}}{}

\end{document}

      