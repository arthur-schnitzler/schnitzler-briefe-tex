%% latex-korrekturansicht-vorspann.tex
%% Vorspann für die Korrekturansicht.
%% Lädt die gemeinsame Datei latex-vorspann.tex mit gesetztem Schalter.

\newif\ifkorrekturansicht
\korrekturansichttrue

\input{../tex-inputs/latex-vorspann}


               \section[Richard Beer-Hofmann an Arthur Schnitzler, {[}28. 12. 1892?{]}]{ Richard Beer-Hofmann an Arthur Schnitzler, {[}28. 12. 1892?{]}}\nopagebreak\mylabel{v}\rehead{ }\normalsize\beginnumbering\briefempfaengerindex{Schnitzler, Arthur@\textsc{Schnitzler, Arthur}!zzzBeer-Hofmann, Richard@\emph{von Richard Beer-Hofmann}!1892-12-281@{{[}28. 12. 1892?{]}}|(be} \toendnotes[C]{\smallbreak\pagebreak[2]} \Standort{CUL, Schnitzler, B 8.}
\physDesc{Brief, 1 Blatt, 2 Seiten
\newline{}Handschrift: blauer Buntstift, lateinische Kurrent
\newline{}Schnitzler: mit Bleistift nummeriert: »15« }\toendnotes[C]{\smallbreak}\pstart{}{\pb}Lieber Arthur!\pend\pstart
           Frau \textcolor{blue}{Flegmann}{}\ledrightnote{\textcolor{blue}{Bertha Flegmann}} hat \uline{uns} für nächsten \label{K_L00148_1v}\edtext{Freitag}{\lemma{\textnormal{\emph{Freitag}}}\Cendnote{\textnormal{Der Brief ist undatiert. Am
                     23. 12. 1892 wird im Brief \textcolor{blue}{Hofmannsthal}s an \textcolor{blue}{Schnitzler}{ }\emph{\textcolor{green}{Aspasia}} erwähnt. Es dürfte sich um
                  Vorbereitungen zu einer Privataufführung bei \textcolor{blue}{Bertha
                     Flegmann} gehandelt haben. Da \textcolor{blue}{Schnitzler} am Donnerstag, dem 29. 12. 1892 für den
                  morgigen Tag ein Treffen bei \textcolor{blue}{Flegmann} absagt,
                  scheint dieses Korrespondenzstück der wahrscheinliche Vorgänger desselben zu sein.}}}\label{K_L00148_1h}
               eingeladen (\textcolor{green}{Aspasia}{}\ledrightnote{\textcolor{green}{Aspasia}}) ich refusire daher \textcolor{blue}{Singer}{}\ledrightnote{\textcolor{blue}{Alexander Singer}}. {\pb}Sie hoffentlich auch.\pend
           \pstart
           Herzlichst{\\[\baselineskip]}\spacefill\mbox{Richard}\pend
           \leftskip=0em{}\endnumbering\briefempfaengerindex{Schnitzler, Arthur@\textsc{Schnitzler, Arthur}!zzzBeer-Hofmann, Richard@\emph{von Richard Beer-Hofmann}!1892-12-281@{{[}28. 12. 1892?{]}}|)be}\mylabel{h}  \normalsize

\doendnotes{C}
\bigskip
\vfill

\clearpage

\footnotesize

\lohead{\textsc{register}}

% Definiere theindex-Environment komplett neu ohne reledmac
\makeatletter
\renewenvironment{theindex}{%
  \section*{\indexname}%
  \setlength{\parindent}{0pt}%
  \setlength{\parskip}{0pt plus 0.3pt}%
  \let\item\@idxitem
}{%
  \clearpage
}
\makeatother

\IfFileExists{\jobname-pw.ind}{\input{\jobname-pw.ind}}{}

\end{document}

      