%% latex-korrekturansicht-vorspann.tex
%% Vorspann für die Korrekturansicht.
%% Lädt die gemeinsame Datei latex-vorspann.tex mit gesetztem Schalter.

\newif\ifkorrekturansicht
\korrekturansichttrue

\input{../tex-inputs/latex-vorspann}


               \section[Arthur Schnitzler an Hugo von Hofmannsthal, 2. 9. 1896]{ Arthur Schnitzler an Hugo von Hofmannsthal,
                    2. 9. 1896}\nopagebreak\mylabel{v}\rehead{ }\normalsize\beginnumbering\briefempfaengerindex{Hofmannsthal, Hugo von@\textsc{Hofmannsthal, Hugo von}!zzzSchnitzler, Arthur@\emph{von Arthur Schnitzler}!1896-09-021@{2. 9. 1896}|(be} \toendnotes[C]{\smallbreak\pagebreak[2]} \Standort{FDH, Hs-30885,52.}
\physDesc{Brief, 1 Blatt, 4 Seiten
\newline{}Handschrift: schwarze Tinte, deutsche Kurrent}\buchAbdrucke{\weitereDrucke{Hugo von Hofmannsthal, Arthur Schnitzler: \emph{Briefwechsel}. Hg. Therese Nickl und Heinrich Schnitzler. Frankfurt am Main: \emph{S. Fischer} 1964, S. 74–75.} }\toendnotes[C]{\smallbreak}\pstart
           \raggedleft{}{\pb}\textcolor{pink}{Wien}{}\ledrightnote{\textcolor{pink}{Wien}}{ }2. 9. 96.\pend
           \pstart{}Lieber Hugo,\pend\pstart
           Ihren ſo \textcolor{blue}{gemeinſchaftlichen}{}\ledrightnote{→\textcolor{blue}{Hermine von Schaffgotsch}}
                    Brief hab ich in \textcolor{pink}{Berlin}{}\ledrightnote{\textcolor{pink}{Berlin}} beko{\geminationm}en und hab mich ſehr darüber gefreut. Sind Sie
                    noch in \textcolor{pink}{Altausſee}{}\ledrightnote{\textcolor{pink}{Altaussee}}? Jedenfalls ſende ich Ihnen
                    dahin meine herzlichſten Grüße und hoffe Sie bald in \textcolor{pink}{Wien}{}\ledrightnote{\textcolor{pink}{Wien}} zu ſehn. Ich war in \textcolor{pink}{Berlin}{}\ledrightnote{\textcolor{pink}{Berlin}}{ }{\pb}4 Tage; das bis zur Unkenntlichkeit umgearbeitete
                        \textcolor{green}{Stück}{}\ledrightnote{→\textcolor{green}{Freiwild. Schauspiel in 3 Akten}} hab ich dem \textcolor{blue}{Brahm}{}\ledrightnote{\textcolor{blue}{Otto Brahm}} vorgeleſen, der es, nicht ohne
                    ausgeſprochenes Vergnügen, gleich angeno{\geminationm}en hat. Er
                    wollte es ſchon im September aufführen, wogegen ich mich wehre;
                    wohl mit Erfolg. –\pend
           \pstart
           Auch in \textcolor{pink}{München}{}\ledrightnote{\textcolor{pink}{München}} war ich 2 Tage, und ſeit
                        \label{K_L00582_1v}\edtext{Samstag Früh}{\lemma{\textnormal{\emph{Samstag Früh}}}\Cendnote{\textnormal{29. 8. 1896}}}\label{K_L00582_1h} bin ich wieder zu Hauſe, wo ich eben einen
                        {\pb}der wildeſten Schnupfen durchlebe. So kann ich
                    nicht mit der nötigen Geiſtesfriſche auf die Vierzeiler antworten, obwohl ich
                    mehr als dreifachen Sinn darin erkannt zu haben glaube.\pend
           \pstart
           Daſs ich Ihre \textcolor{green}{Novelle}{}\ledrightnote{→\textcolor{green}{Geschichte der beiden Liebespaare}} nicht
                    hören ſoll, beleidigt mich – nur \textcolor{blue}{Richard}{}\ledrightnote{\textcolor{blue}{Richard Beer-Hofmann}}{ }ſoll das \label{K_L00582_2v}\edtext{Vorrecht}{\lemma{\textnormal{\emph{Vorrecht}}}\Cendnote{\textnormal{\textcolor{blue}{Hofmannsthal} hatte \emph{\textcolor{green}{Geschichte der beiden Liebespaare}} nach harter Kritik
                        von \textcolor{blue}{Beer-Hofmann} zurückgelegt.}}}\label{K_L00582_2h}
                    haben, Sachen zu leſen, die Sie nicht für gelungen halten?\pend
           \pstart
           Ich wollte, es käme mir einmal {\pb}was von Ihnen vor
                    Augen mit ſchönen jungen Fehlern!\pend
           \pstart
           Wie ko{\geminationm}en Sie plötzlich aufs Theaterſpielen? Ich war
                    ganz erſchüttert!\pend
           \pstart
           – Aber Zuſa{\geminationm}enſein werden wir hoffentlich oft – und
                    ohne das, was Sie »Halbwahres« ne{\geminationn}en, was aber was
                    ganz andres iſt.\pend
           \pstart
           Wüßt ich nur ganz genau was! In \textcolor{pink}{\textsc{Upsala}}{}\ledrightnote{\textcolor{pink}{Uppsala}} hab ich drüber nachgedacht – \uline{wirklich} in
                        \textcolor{pink}{\textsc{Upsala}}{}\ledrightnote{\textcolor{pink}{Uppsala}}! –\pend
           \pstart Herzliche Grüße! Ihr \spacefill\mbox{Arthur}\pend{}\endnumbering\briefempfaengerindex{Hofmannsthal, Hugo von@\textsc{Hofmannsthal, Hugo von}!zzzSchnitzler, Arthur@\emph{von Arthur Schnitzler}!1896-09-021@{2. 9. 1896}|)be}\mylabel{h}  \normalsize

\doendnotes{C}
\bigskip
\vfill

\clearpage

\footnotesize

\lohead{\textsc{register}}

% Definiere theindex-Environment komplett neu ohne reledmac
\makeatletter
\renewenvironment{theindex}{%
  \section*{\indexname}%
  \setlength{\parindent}{0pt}%
  \setlength{\parskip}{0pt plus 0.3pt}%
  \let\item\@idxitem
}{%
  \clearpage
}
\makeatother

\IfFileExists{\jobname-pw.ind}{\input{\jobname-pw.ind}}{}

\end{document}

      