%% latex-korrekturansicht-vorspann.tex
%% Vorspann für die Korrekturansicht.
%% Lädt die gemeinsame Datei latex-vorspann.tex mit gesetztem Schalter.

\newif\ifkorrekturansicht
\korrekturansichttrue

\input{../tex-inputs/latex-vorspann}


               \section[Arthur Schnitzler an Richard Beer-Hofmann, {[}5.?{]} 5. 1902]{ Arthur Schnitzler an Richard Beer-Hofmann, {[}5.?{]} 5. 1902}\nopagebreak\mylabel{v}\rehead{ }\normalsize\beginnumbering\briefempfaengerindex{Beer-Hofmann, Richard@\textsc{Beer-Hofmann, Richard}!zzzSchnitzler, Arthur@\emph{von Arthur Schnitzler}!1903-05-051@{{[}5.?{]} 5. 1902}|(be} \toendnotes[C]{\smallbreak\pagebreak[2]} \Standort{YCGL, MSS 31.}
\physDesc{Brief, 1 Blatt, 1 Seite, Umschlag
\newline{}Handschrift: 1) Bleistift, deutsche Kurrent\hspace{1em}2) schwarze Tinte, deutsche Kurrent (\noindent{}Umschlag)\hspace{1em}\newline{}Versand: 1) Einschreiben 2) Stempel: »\nobreak{}\oindex{IX., Alsergrund@\textbf{IX., Alsergrund}, \emph{Bezirk (A.BZK)}|pwk}\textcolor{gray}{Wien 9/3}, \textcolor{gray}{5}{[}. 5. 1902{]}, 5\nobreak{}«. 3) Stempel: »\nobreak{}\oindex{Rodaun@\textbf{Rodaun}, \emph{Teil eines besiedelten Ortes (A.BSOX)}|pwk}{\pb}Rodaun, 5/5 02\nobreak{}«. }\toendnotes[C]{\smallbreak}\pstart{}{\pb}\textsc{rec.}\pend{}\pstart{}\textsc{Dr. Richard Beer-Hofmann}\pend{}\pstart{}\textcolor{pink}{\textsc{Rodaun}}{}\ledrightnote{\textcolor{pink}{Rodaun}}\pend{}\pstart{}\textcolor{pink}{\textsc{Liesinger}ſtraße 2}{}\ledrightnote{\textcolor{pink}{Liesingerstraße}}. \pend{}\pstart{}\textsc{Reco{\geminationm}andirt}.
                    \pend{}{\bigskip}\pstart
           \noindent{}{\pb}Auf Wiederſehen\pend
           \pstart
           herzlichſt{\\[\baselineskip]}\spacefill\mbox{A.}\pend
           \leftskip=0em{}\pstart
           \noindent{}An \textcolor{blue}{Eger}{}\ledrightnote{\textcolor{blue}{Paul Eger}} (\label{K_L01218_1v}\edtext{\textcolor{green}{\textsc{Peer Gynt}}{}\ledrightnote{\textcolor{green}{Peer Gynt}}}{\lemma{\textnormal{\emph{Peer Gynt}}}\Cendnote{\textnormal{Am 7. 5. 1902 veranstaltete der \emph{\textcolor{brown}{Akademische Verein für Kunst und
                                Literatur}} im \textcolor{pink}{Deutschen
                                Volkstheater} die Generalprobe für eine zweimalige
                            Wohltätigkeitsaufführung.}}}\label{K_L01218_1h} ſchrieb ich eben)\pend
           \endnumbering\briefempfaengerindex{Beer-Hofmann, Richard@\textsc{Beer-Hofmann, Richard}!zzzSchnitzler, Arthur@\emph{von Arthur Schnitzler}!1903-05-051@{{[}5.?{]} 5. 1902}|)be}\mylabel{h}  \normalsize

\doendnotes{C}
\bigskip
\vfill

\clearpage

\footnotesize

\lohead{\textsc{register}}

% Definiere theindex-Environment komplett neu ohne reledmac
\makeatletter
\renewenvironment{theindex}{%
  \section*{\indexname}%
  \setlength{\parindent}{0pt}%
  \setlength{\parskip}{0pt plus 0.3pt}%
  \let\item\@idxitem
}{%
  \clearpage
}
\makeatother

\IfFileExists{\jobname-pw.ind}{\input{\jobname-pw.ind}}{}

\end{document}

      