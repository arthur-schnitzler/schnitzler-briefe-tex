%% latex-korrekturansicht-vorspann.tex
%% Vorspann für die Korrekturansicht.
%% Lädt die gemeinsame Datei latex-vorspann.tex mit gesetztem Schalter.

\newif\ifkorrekturansicht
\korrekturansichttrue

\input{../tex-inputs/latex-vorspann}


               \section[Arthur Schnitzler an Richard Beer-Hofmann, 22. 8. 1914]{ Arthur Schnitzler an Richard Beer-Hofmann, 22. 8. 1914}\nopagebreak\mylabel{v}\rehead{ }\normalsize\beginnumbering\briefempfaengerindex{Beer-Hofmann, Richard@\textsc{Beer-Hofmann, Richard}!zzzSchnitzler, Arthur@\emph{von Arthur Schnitzler}!1914-08-221@{22. 8. 1914}|(be} \toendnotes[C]{\smallbreak\pagebreak[2]} \Standort{YCGL, MSS 31.}
\physDesc{Kartenbrief
\newline{}Handschrift: Bleistift, deutsche Kurrent\newline{}Versand: Stempel: »\nobreak{}\oindex{Bad Ischl@\textbf{Bad Ischl}, \emph{Besiedelter Ort (A.BSO)}|pwk}{[}Bad{]} Ischl, 2\textcolor{gray}{2}. VIII. {[}1914{]}\nobreak{}«.  
\newline{}Beer-Hofmann: mit blauem Buntstift den Erhalt und die Beantwortung markiert: »\noindent{}E.B{ / }24/VIII 14{ }\textsc{Telegr.}« }\buchAbdrucke{\weitereDrucke{Arthur Schnitzler, Richard Beer-Hofmann: \emph{Briefwechsel 1891–1931}. Hg. Konstanze Fliedl. Wien, Zürich: \emph{Europaverlag} 1992, S. 220.} }\toendnotes[C]{\smallbreak}\pstart{}{\pb}Abſ. \textsc{Schnitzler, \textcolor{pink}{Ischl}{}\ledrightnote{\textcolor{pink}{Bad Ischl}}, \textcolor{pink}{Kaiserkrone}{}\ledrightnote{\textcolor{pink}{Hotel Kaiserkrone}}}\pend{}{\bigskip}\pstart{}Herrn \textsc{Dr. Richard Beer-Hofmann}\pend{}\pstart{}\strikeout{\textcolor{pink}{\textsc{Untera}}{}\ledrightnote{\textcolor{pink}{Unterach am Attersee}}}\pend{}\pstart{}\textsc{\textcolor{pink}{Weißenbach}{}\ledrightnote{\textcolor{pink}{Weißenbach am Attersee}}.}\pend{}\pstart{}\textsc{Am}{ }\textcolor{pink}{\textsc{Atter}ſee}{}\ledrightnote{\textcolor{pink}{Attersee}}\pend{}{\bigskip}\pstart
           \raggedleft{}{\pb}\textcolor{pink}{\textsc{Ischl}}{}\ledrightnote{\textcolor{pink}{Bad Ischl}}, 22/8 914. \pend
           \pstart{}lieber Richard,\pend\pstart
           wir ſind recht reiſemüde nach dieſer höchſt unbequemen überlangen Fahrt – wollen hier
               eigentlich nur ein paar Tage ausruhn und nicht mehr hin u her radeln. Vielleicht
               entſchließen Sie ſich mit \textcolor{blue}{Paula}{}\ledrightnote{\textcolor{blue}{Paula Beer-Hofmann}},
                  Montag oder \label{KLL02194_Beer-Hofmann-1v}\edtext{Dinſtag}{\lemma{\textnormal{\emph{Dinſtag}}}\Cendnote{\textnormal{vgl. A. S.: \emph{Tagebuch}, 25. 8. 1914}}}\label{KLL02194_Beer-Hofmann-1h} herüberzufahren? Es wäre ſehr ſchön! Wir dürften Mittwoch oder
                  Donnerſtg\label{KLL02194_Beer-Hofmann-2v}\edtext{{ }heimfahren}{\lemma{\textnormal{\emph{ heimfahren}}}\Cendnote{\textnormal{Das verzögerte sich bis 30. 9. 1914.}}}\label{KLL02194_Beer-Hofmann-2h}.\pend
           \pstart
           Wie lange bleiben Sie überhaupt noch?\pend
           \pstart
           Wir grüßen Sie alle herzlichſt!\pend
           \pstart
           Ihr{\\[\baselineskip]}\spacefill\mbox{Arthur}\pend
           \leftskip=0em{}\pstart
           Vielleicht machen Sie \introOben{}etwas\introOben{} mit \textcolor{blue}{Saltens}{}\ledrightnote{\textcolor{blue}{Felix Salten}{\newline}\textcolor{blue}{Ottilie Salten}} ab, dem ich in ähnlichem Sinn ſchreibe\pend
           \endnumbering\briefempfaengerindex{Beer-Hofmann, Richard@\textsc{Beer-Hofmann, Richard}!zzzSchnitzler, Arthur@\emph{von Arthur Schnitzler}!1914-08-221@{22. 8. 1914}|)be}\mylabel{h}  \normalsize

\doendnotes{C}
\bigskip
\vfill

\clearpage

\footnotesize

\lohead{\textsc{register}}

% Definiere theindex-Environment komplett neu ohne reledmac
\makeatletter
\renewenvironment{theindex}{%
  \section*{\indexname}%
  \setlength{\parindent}{0pt}%
  \setlength{\parskip}{0pt plus 0.3pt}%
  \let\item\@idxitem
}{%
  \clearpage
}
\makeatother

\IfFileExists{\jobname-pw.ind}{\input{\jobname-pw.ind}}{}

\end{document}

      