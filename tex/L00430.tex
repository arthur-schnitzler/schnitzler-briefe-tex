%% latex-korrekturansicht-vorspann.tex
%% Vorspann für die Korrekturansicht.
%% Lädt die gemeinsame Datei latex-vorspann.tex mit gesetztem Schalter.

\newif\ifkorrekturansicht
\korrekturansichttrue

\input{../tex-inputs/latex-vorspann}


               \section[Friedrich M. Fels an Arthur Schnitzler, 23. 4. 1895]{ Friedrich M. Fels an Arthur Schnitzler, 23. 4. 1895}\nopagebreak\mylabel{v}\rehead{ }\normalsize\beginnumbering\briefempfaengerindex{Schnitzler, Arthur@\textsc{Schnitzler, Arthur}!zzzFels, Friedrich Michael@\emph{von Friedrich Michael Fels}!1895-04-231@{23. 4. 1895}|(be} \toendnotes[C]{\smallbreak\pagebreak[2]} \Standort{DLA, A:Schnitzler, HS.NZ85.1.2956.}
\physDesc{Kartenbrief
\newline{}Handschrift: schwarze Tinte, lateinische Kurrent\newline{}Versand: 1) Stempel: »\nobreak{}\oindex{I., Innere Stadt@\textbf{I., Innere Stadt}, \emph{Bezirk (A.BZK)}|pwk}Wien 1/1, 23. 5. 1895, 1–N\nobreak{}«.  2) Stempel: »\nobreak{}\oindex{IX., Alsergrund@\textbf{IX., Alsergrund}, \emph{Bezirk (A.BZK)}|pwk}Wien 9/3, 2\textcolor{gray}{3}. 5. 1895, 3, Bestellt\nobreak{}«. 
\newline{}Schnitzler: mit Bleistift datiert: »23/4 95« und nummeriert: »21« }\toendnotes[C]{\smallbreak}\pstart{}{\pb}Herrn Dr. Arthur Schnitzler\pend{}\pstart{}\textcolor{pink}{Wien}{}\ledrightnote{\textcolor{pink}{Wien}}\pend{}\pstart{}\textcolor{pink}{IX, Frankgaſse 1}{}\ledrightnote{\textcolor{pink}{Frankgasse}}\pend{}{\bigskip}\pstart{}{\pb}Lieber Dr. Schnitzler,\pend\pstart
           In der \textcolor{green}{Gegenwart}{}\ledrightnote{\textcolor{green}{Die Gegenwart. Zeitschrift für Literatur, Wirtschaftsleben und Kunst}} vom 20. d.{ }ſteht eine \label{K_L00430_1v}\edtext{\textcolor{green}{Besprechung}{}\ledrightnote{→\textcolor{green}{Sterben. Novelle von Arthur Schnitzler}}}{\lemma{\textnormal{\emph{Besprechung}}}\Cendnote{\textnormal{»\textcolor{green}{\so{Sterben}}. Novelle von \so{Arthur Schnitzler}. (\textcolor{pink}{Berlin}, \textcolor{brown}{S. Fischer}.) Ein trauriges und peinliches, aber ein feines und
                            bedeutendes Werk eines echten Künstlers. Das Sterben eines
                            Schwindsüchtigen wird geschildert, sein Ringen mit Leben und Tod, das
                            langsame Scheiden von der Geliebten, die Empörung, das Aufbäumen, der
                            Todeskampf. Jeder Zug ist beobachtet und wahr, nichts übertrieben, dem
                            Dramatischen wird discret aus dem Wege gegangen, wie jeglichem
                            schildernden Naturalismus. Das Zuständliche ist knapp, die Menschen ohne
                            Individualisirung gezeichnet – sie haben auch keinen Familiennamen –,
                            aber die Seelenanalyse ist voll feiner Züge, so daß uns weder Ekel noch
                            Schauer erfaßt und die rein menschliche Theilnahme bis zuletzt rege
                            bleibt.« ([O. V.], in: \emph{\textcolor{green}{Die
                                Gegenwart}}, Bd. 47, Nr. 16, 20. 4. 1895,
                            S. 255.)}}}\label{K_L00430_1h} Ihrer \textcolor{green}{Novelle}{}\ledrightnote{→\textcolor{green}{Sterben. Novelle}}, sehr knapp und \uline{sehr} anerke{\geminationn}enend, dabei sehr vernünftig
                    – ungefähr so, wie wir selbst darüber schreiben würden.\pend
           \pstart
           Herzlichst{\\[\baselineskip]}\spacefill\mbox{Fels}\pend
           \leftskip=0em{}\endnumbering\briefempfaengerindex{Schnitzler, Arthur@\textsc{Schnitzler, Arthur}!zzzFels, Friedrich Michael@\emph{von Friedrich Michael Fels}!1895-04-231@{23. 4. 1895}|)be}\mylabel{h}  \normalsize

\doendnotes{C}
\bigskip
\vfill

\clearpage

\footnotesize

\lohead{\textsc{register}}

% Definiere theindex-Environment komplett neu ohne reledmac
\makeatletter
\renewenvironment{theindex}{%
  \section*{\indexname}%
  \setlength{\parindent}{0pt}%
  \setlength{\parskip}{0pt plus 0.3pt}%
  \let\item\@idxitem
}{%
  \clearpage
}
\makeatother

\IfFileExists{\jobname-pw.ind}{\input{\jobname-pw.ind}}{}

\end{document}

      