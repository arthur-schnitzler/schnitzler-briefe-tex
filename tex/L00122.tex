%% latex-korrekturansicht-vorspann.tex
%% Vorspann für die Korrekturansicht.
%% Lädt die gemeinsame Datei latex-vorspann.tex mit gesetztem Schalter.

\newif\ifkorrekturansicht
\korrekturansichttrue

\input{../tex-inputs/latex-vorspann}


               \section[Arthur Schnitzler an Hugo von Hofmannsthal, 11. 9. 1892]{ Arthur Schnitzler an Hugo von Hofmannsthal, 11. 9. 1892}\nopagebreak\mylabel{v}\rehead{ }\normalsize\beginnumbering\briefempfaengerindex{Hofmannsthal, Hugo von@\textsc{Hofmannsthal, Hugo von}!zzzSchnitzler, Arthur@\emph{von Arthur Schnitzler}!1892-09-111@{11. 9. 1892}|(be} \toendnotes[C]{\smallbreak\pagebreak[2]} \Standort{FDH, Hs-30885,25.}
\physDesc{Brief, 1 Blatt, 3 Seiten
\newline{}Handschrift: schwarze Tinte, deutsche Kurrent}\buchAbdrucke{\weitereDrucke{Hugo von Hofmannsthal, Arthur Schnitzler: \emph{Briefwechsel}. Hg. Therese Nickl und Heinrich Schnitzler. Frankfurt am Main: \emph{S. Fischer} 1964, S. 29.} }\pstart
           \raggedleft{}{\pb}11. 9. 92.\pend
           \pstart{}Lieber Loris. –\pend\pstart
           Heute verlaſſe ich \textcolor{pink}{Iſchl}{}\ledrightnote{\textcolor{pink}{Bad Ischl}}. Ueber den \textcolor{pink}{Brenner}{}\ledrightnote{\textcolor{pink}{Brenner}} nach \textcolor{pink}{Riva}{}\ledrightnote{\textcolor{pink}{Riva del Garda}} am \textcolor{pink}{Gardaſee}{}\ledrightnote{\textcolor{pink}{Lago di Garda}}, wo ich wohl
                    einige Zeit, dh. 5–8 Tage verbleibe. Dann \textcolor{pink}{Semmering}{}\ledrightnote{\textcolor{pink}{Semmering}}, denk’ ich, dann \textcolor{pink}{Wien}{}\ledrightnote{\textcolor{pink}{Wien}}.
                    Neulich auf dem \textcolor{pink}{Schafberg}{}\ledrightnote{\textcolor{pink}{Schafberg (Wien)}} geweſen – tiefer
                    Schnee, Geſtöber. –\pend
           \pstart
           Hier auch weiterhin nichts gethan. Der Tag vergeht doch. Das \textcolor{green}{Journal}{}\ledrightnote{\textcolor{green}{Journal des Goncourt. Mémoires de la vie littéraire}} v d \textcolor{blue}{Goncourts}{}\ledrightnote{\textcolor{blue}{Edmond Huot de Goncourt}{\newline}\textcolor{blue}{Jules Huot de Goncourt}} geleſen, Karten geſpielt, in den Straßen herum,
                    faſt i{\geminationm}er Regen. {\pb}Jetzt
                    will ich packen, was ich nicht kann.\pend
           \pstart
           Wenn Sie mir nach \textcolor{pink}{Riva}{}\ledrightnote{\textcolor{pink}{Riva del Garda}}{ }ſchreiben wollen, ein paar Zeilen, was ſehr
                    hübſch wäre, \textsc{post rest}, bitte. –\pend
           \pstart
           Mich frieren die Fingerſpitzen. Im Zi{\geminationm}er iſt es
                    kalt. Im Hotel wird i{\geminationm}erfort geklingelt, kein
                    Menſch weiſs warum. Schritte im Corridor: i{\geminationm}er, als
                        we{\geminationn}{ }ſie gerad zu meiner Thür kämen. Alles in
                    Wolken. {\pb}Freue mich, noch nicht nach \textcolor{pink}{Wien}{}\ledrightnote{\textcolor{pink}{Wien}} zu reiſen.\pend
           \pstart
           Herzlichſt der Ihre{\\[\baselineskip]}\spacefill\mbox{Arthur.}\pend
           \leftskip=0em{}\endnumbering\briefempfaengerindex{Hofmannsthal, Hugo von@\textsc{Hofmannsthal, Hugo von}!zzzSchnitzler, Arthur@\emph{von Arthur Schnitzler}!1892-09-111@{11. 9. 1892}|)be}\mylabel{h}  \normalsize

\doendnotes{C}
\bigskip
\vfill

\clearpage

\footnotesize

\lohead{\textsc{register}}

% Definiere theindex-Environment komplett neu ohne reledmac
\makeatletter
\renewenvironment{theindex}{%
  \section*{\indexname}%
  \setlength{\parindent}{0pt}%
  \setlength{\parskip}{0pt plus 0.3pt}%
  \let\item\@idxitem
}{%
  \clearpage
}
\makeatother

\IfFileExists{\jobname-pw.ind}{\input{\jobname-pw.ind}}{}

\end{document}

      