%% latex-korrekturansicht-vorspann.tex
%% Vorspann für die Korrekturansicht.
%% Lädt die gemeinsame Datei latex-vorspann.tex mit gesetztem Schalter.

\newif\ifkorrekturansicht
\korrekturansichttrue

\input{../tex-inputs/latex-vorspann}


               \section[Arthur Schnitzler und Hugo von Hofmannsthal an Richard Beer-Hofmann, 31. 8. 1899]{ Arthur Schnitzler und Hugo von Hofmannsthal an Richard Beer-Hofmann,
               31. 8. 1899}\nopagebreak\mylabel{v}\rehead{ }\normalsize\beginnumbering\briefempfaengerindex{Beer-Hofmann, Richard@\textsc{Beer-Hofmann, Richard}!zzzHofmannsthal, Hugo von@\emph{von Hugo von Hofmannsthal}!1899-08-311@{31. 8. 1899}|(be}\briefempfaengerindex{Beer-Hofmann, Richard@\textsc{Beer-Hofmann, Richard}!zzzSchnitzler, Arthur@\emph{von Arthur Schnitzler}!1899-08-311@{31. 8. 1899}|(be} \toendnotes[C]{\smallbreak\pagebreak[2]} \Standort{YCGL, MSS 31.}
\physDesc{Postkarte
\newline{}Handschrift Arthur Schnitzler: Bleistift, deutsche Kurrent\newline{}Handschrift Hugo von Hofmannsthal: Bleistift\newline{}Versand: 1) Stempel: »\nobreak{}\oindex{Bad Ischl@\textbf{Bad Ischl}, \emph{Besiedelter Ort (A.BSO)}|pwk}Ischl, 31. 8. 99, 12–1N\nobreak{}«.  2) Stempel: »\nobreak{}\oindex{Seeboden@\textbf{Seeboden}, \emph{http://www.geonames.org/ontologyA.ADM3}|pwk}Seeboden, 1 9 99\nobreak{}«. }\buchAbdrucke{\weitereDrucke{Arthur Schnitzler, Richard Beer-Hofmann: \emph{Briefwechsel 1891–1931}. Hg. Konstanze Fliedl. Wien, Zürich: \emph{Europaverlag} 1992, S. 134.} }\toendnotes[C]{\smallbreak}\pstart{}{\pb}\textsc{Herrn Dr Richard Beer-Hofmann}\pend{}\pstart{}\textcolor{pink}{\textsc{Seeboden am Millstätter}ſee}{}\ledrightnote{\textcolor{pink}{Seeboden}}\pend{}\pstart{}\textcolor{pink}{\textsc{Villa Platzer}}{}\ledrightnote{\textcolor{pink}{Villa Platzer}}\pend{}\pstart{}\textcolor{pink}{\textsc{Kärnthen}}{}\ledrightnote{\textcolor{pink}{Kärnten}}\pend{}{\bigskip}\pstart
           \noindent{}\centering{}{\pb}\label{K_L00966_1v}\edtext{Gigant!}{\lemma{\textnormal{\emph{Gigant!}}}\Cendnote{\textnormal{Reaktion darauf, dass der notorisch langsam arbeitende \textcolor{blue}{Beer-Hofmann} sein neues Stück, das unter dem Titel \emph{\textcolor{green}{Der Graf von Charolais}} veröffentlicht werden
                  sollte, begonnen hatte.}}}\label{K_L00966_1h}\pend
           \pstart
           \spacefill\mbox{Arthur}{\\[\baselineskip]}\spacefill\mbox{{[}hs. Hofmannsthal:{]} Hugo}\pend
           \leftskip=0em{}\endnumbering\briefempfaengerindex{Beer-Hofmann, Richard@\textsc{Beer-Hofmann, Richard}!zzzHofmannsthal, Hugo von@\emph{von Hugo von Hofmannsthal}!1899-08-311@{31. 8. 1899}|)be}\briefempfaengerindex{Beer-Hofmann, Richard@\textsc{Beer-Hofmann, Richard}!zzzSchnitzler, Arthur@\emph{von Arthur Schnitzler}!1899-08-311@{31. 8. 1899}|)be}\mylabel{h}  \normalsize

\doendnotes{C}
\bigskip
\vfill

\clearpage

\footnotesize

\lohead{\textsc{register}}

% Definiere theindex-Environment komplett neu ohne reledmac
\makeatletter
\renewenvironment{theindex}{%
  \section*{\indexname}%
  \setlength{\parindent}{0pt}%
  \setlength{\parskip}{0pt plus 0.3pt}%
  \let\item\@idxitem
}{%
  \clearpage
}
\makeatother

\IfFileExists{\jobname-pw.ind}{\input{\jobname-pw.ind}}{}

\end{document}

      