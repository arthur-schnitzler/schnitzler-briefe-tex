%% latex-korrekturansicht-vorspann.tex
%% Vorspann für die Korrekturansicht.
%% Lädt die gemeinsame Datei latex-vorspann.tex mit gesetztem Schalter.

\newif\ifkorrekturansicht
\korrekturansichttrue

\input{../tex-inputs/latex-vorspann}


               \section[Richard Beer-Hofmann an Arthur Schnitzler, 10. 8. 1914]{ Richard Beer-Hofmann an Arthur Schnitzler, 10. 8. 1914}\nopagebreak\mylabel{v}\rehead{ }\normalsize\beginnumbering\briefempfaengerindex{Schnitzler, Arthur@\textsc{Schnitzler, Arthur}!zzzBeer-Hofmann, Richard@\emph{von Richard Beer-Hofmann}!1914-08-101@{10. 8. 1914}|(be} \toendnotes[C]{\smallbreak\pagebreak[2]} \Standort{CUL, Schnitzler, B 8.}
\physDesc{Bildpostkarte
\newline{}Handschrift: Bleistift, lateinische Kurrent\newline{}Versand: 1) Stempel: »\nobreak{}\oindex{Weissenbach am Attersee@\textbf{Weißenbach am Attersee}, \emph{http://www.geonames.org/ontologyA.ADM3}|pwk}Weissenbach am Attersee, 1\textcolor{gray}{1}. VII. 14\nobreak{}«.  2) Stempel: »\nobreak{}\oindex{Celerina@\textbf{Celerina}, \emph{Besiedelter Ort (A.BSO)}|pwk}Celerina (Graubünden), 16. VII. 14, 1\nobreak{}«. 3) postalischer Nachsendevermerk: »\textcolor{pink}{Hotel Lattmann}, \textcolor{pink}{Ragaz}« 4) Stempel: »\nobreak{}\oindex{Bad Ragaz@\textbf{Bad Ragaz}, \emph{https://www.geonames.org/ontologyP.PPL}|pwk}Ragaz, 17. VII. 14, 3\nobreak{}«. 5) postalischer Nachsendevermerk: »\textcolor{pink}{Wien XVIII}, \textcolor{pink}{Sternwartestr. 71}« \newline{}Ordnung: mit Bleistift von unbekannter Hand nummeriert:
                                    »259« }\buchAbdrucke{\weitereDrucke{Arthur Schnitzler, Richard Beer-Hofmann: \emph{Briefwechsel 1891–1931}. Hg. Konstanze Fliedl. Wien, Zürich: \emph{Europaverlag} 1992, S. 220.} }\toendnotes[C]{\smallbreak}\pstart{}{\pb}Herrn D\textsuperscript{r} Arthur Schnitzler\pend{}\pstart{}\textcolor{pink}{Schweiz}{}\ledrightnote{\textcolor{pink}{Schweiz}}\pend{}\pstart{}\textcolor{pink}{Celerina}{}\ledrightnote{\textcolor{pink}{Celerina}}\pend{}\pstart{}\textcolor{pink}{Cresta Palace}{}\ledrightnote{\textcolor{pink}{Cresta Palace}}\pend{}{\bigskip}\pstart
           \noindent{}\centering{}{\pb}\textcolor{gray}{\textbf{\textcolor{pink}{Salzkammergut}{}\ledrightnote{\textcolor{pink}{Salzkammergut}}. \textcolor{pink}{Weissenbach am Attersee}{}\ledrightnote{\textcolor{pink}{Weißenbach am Attersee}}.}}\pend
           \pstart
           \raggedleft{}10/VIII. 14\pend
           \pstart
           Lieber Arthur! Ich war für zwei Tage – getrieben von Unruhe – in \textcolor{pink}{Wien}{}\ledrightnote{\textcolor{pink}{Wien}} und sah dass es zwecklos wäre \uline{jetzt} dorthin mit den \textcolor{blue}{Kindern}{}\ledrightnote{→\textcolor{blue}{Naëmah Beer-Hofmann}{\newline}→\textcolor{blue}{Mirjam Beer-Hofmann}{\newline}→\textcolor{blue}{Gabriel Beer-Hofmann}} zurückzugehen. So bleibe
               ich noch – wie lange? – hier. \uline{Zu} weit vom Schuss sein
               ist auch unerträglich. Was ists mit \textcolor{blue}{Kaufmann}{}\ledrightnote{\textcolor{blue}{Arthur Kaufmann}}, \textcolor{blue}{Leo}{}\ledrightnote{\textcolor{blue}{Leo Van-Jung}}, \textcolor{blue}{Bella}{}\ledrightnote{\textcolor{blue}{Isabella Vengerova}}?\pend
           \pstart
           Alles Herzliche von uns!{\\[\baselineskip]}\spacefill\mbox{Richard}\pend
           \leftskip=0em{}\endnumbering\briefempfaengerindex{Schnitzler, Arthur@\textsc{Schnitzler, Arthur}!zzzBeer-Hofmann, Richard@\emph{von Richard Beer-Hofmann}!1914-08-101@{10. 8. 1914}|)be}\mylabel{h}  \normalsize

\doendnotes{C}
\bigskip
\vfill

\clearpage

\footnotesize

\lohead{\textsc{register}}

% Definiere theindex-Environment komplett neu ohne reledmac
\makeatletter
\renewenvironment{theindex}{%
  \section*{\indexname}%
  \setlength{\parindent}{0pt}%
  \setlength{\parskip}{0pt plus 0.3pt}%
  \let\item\@idxitem
}{%
  \clearpage
}
\makeatother

\IfFileExists{\jobname-pw.ind}{\input{\jobname-pw.ind}}{}

\end{document}

      