%% latex-korrekturansicht-vorspann.tex
%% Vorspann für die Korrekturansicht.
%% Lädt die gemeinsame Datei latex-vorspann.tex mit gesetztem Schalter.

\newif\ifkorrekturansicht
\korrekturansichttrue

\input{../tex-inputs/latex-vorspann}


               \section[Hugo von Hofmannsthal an Arthur Schnitzler, 14. 6. 1901]{ Hugo von Hofmannsthal an Arthur Schnitzler, 14. 6. 1901}\nopagebreak\mylabel{v}\rehead{ }\normalsize\beginnumbering\briefempfaengerindex{Schnitzler, Arthur@\textsc{Schnitzler, Arthur}!zzzHofmannsthal, Hugo von@\emph{von Hugo von Hofmannsthal}!1901-06-141@{14. 6. 1901}|(be} \toendnotes[C]{\smallbreak\pagebreak[2]} \Standort{CUL, Schnitzler, B 43.}
\physDesc{Bildpostkarte
\newline{}Handschrift: schwarze Tinte, deutsche Kurrent\newline{}Versand: 1) Stempel: »\nobreak{}\oindex{Lido@\textbf{Lido}, \emph{Teil eines besiedelten Ortes (A.BSOX)}|pwk}Grand Hôtel des Bains Dépendance et Châlets Lido –
                              Venise F. Schlössing directeur\nobreak{}«.  2) Stempel: »\nobreak{}\oindex{Lido@\textbf{Lido}, \emph{Teil eines besiedelten Ortes (A.BSOX)}|pwk}Elisabetta di Lido (Venezia), 14 6 01\nobreak{}«. 3) Stempel: »\nobreak{}16. 6. 01, 9.V\nobreak{}«. 
\newline{}Schnitzler: mit Bleistift datiert: »14/6 901« \newline{}Ordnung: mit Bleistift von unbekannter Hand nummeriert:
                              »174« }\buchAbdrucke{\weitereDrucke{Hugo von Hofmannsthal, Arthur Schnitzler: \emph{Briefwechsel}. Hg. Therese Nickl und Heinrich Schnitzler. Frankfurt am Main: \emph{S. Fischer} 1964, S. 147.} }\toendnotes[C]{\smallbreak}\pstart{}{\pb}\textsc{Herrn D\textsuperscript{r} Arthur Schnitzler}\pend{}\pstart{}\textcolor{pink}{\textsc{Wien}}{}\ledrightnote{\textcolor{pink}{Wien}}\pend{}\pstart{}\textcolor{pink}{\textsc{IX. Frankgasse 1}}{}\ledrightnote{\textcolor{pink}{Frankgasse}}. \pend{}\pstart{}\textsc{\textcolor{pink}{Austria}{}\ledrightnote{\textcolor{pink}{Österreich}}}\pend{}{\bigskip}\pstart
           \noindent{}\centering{}\textcolor{gray}{\textbf{{\pb}\textcolor{pink}{Bagni di Lido}{}\ledrightnote{\textcolor{pink}{Lido}}}}\pend
           \pstart
           \noindent{}\centering{}\textcolor{gray}{\textbf{\textcolor{pink}{Venezia}{}\ledrightnote{\textcolor{pink}{Venedig}}}}\pend
           \pstart
           Wir thun \textcolor{pink}{hier}{}\ledrightnote{→\textcolor{pink}{Lido}}{ }See-baden und ich leſe dazu \textcolor{green}{die natürliche Tochter}{}\ledrightnote{\textcolor{green}{Die natürliche Tochter}}. Hoffentlich liegt in \textcolor{pink}{Rodaun}{}\ledrightnote{\textcolor{pink}{Rodaun}} in 8 Tagen eine Zeile von Ihnen. Viele Grüße von \textcolor{blue}{Gerty}{}\ledrightnote{\textcolor{blue}{Gertrude von Hofmannsthal}}. Von Herzen Ihr\pend
           \pstart \spacefill\mbox{Hugo}\pend{}\endnumbering\briefempfaengerindex{Schnitzler, Arthur@\textsc{Schnitzler, Arthur}!zzzHofmannsthal, Hugo von@\emph{von Hugo von Hofmannsthal}!1901-06-141@{14. 6. 1901}|)be}\mylabel{h}  \normalsize

\doendnotes{C}
\bigskip
\vfill

\clearpage

\footnotesize

\lohead{\textsc{register}}

% Definiere theindex-Environment komplett neu ohne reledmac
\makeatletter
\renewenvironment{theindex}{%
  \section*{\indexname}%
  \setlength{\parindent}{0pt}%
  \setlength{\parskip}{0pt plus 0.3pt}%
  \let\item\@idxitem
}{%
  \clearpage
}
\makeatother

\IfFileExists{\jobname-pw.ind}{\input{\jobname-pw.ind}}{}

\end{document}

      