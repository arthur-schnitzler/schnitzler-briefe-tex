%% latex-korrekturansicht-vorspann.tex
%% Vorspann für die Korrekturansicht.
%% Lädt die gemeinsame Datei latex-vorspann.tex mit gesetztem Schalter.

\newif\ifkorrekturansicht
\korrekturansichttrue

\input{../tex-inputs/latex-vorspann}


               \section[Hugo von Hofmannsthal an Arthur Schnitzler, 1. 2. 1904]{ Hugo von Hofmannsthal an Arthur Schnitzler, 1. 2. 1904}\nopagebreak\mylabel{v}\rehead{ }\normalsize\beginnumbering\briefempfaengerindex{Schnitzler, Arthur@\textsc{Schnitzler, Arthur}!zzzHofmannsthal, Hugo von@\emph{von Hugo von Hofmannsthal}!1904-02-012@{1. 2. 1904}|(be} \toendnotes[C]{\smallbreak\pagebreak[2]} \Standort{CUL, Schnitzler, B 43.}
\physDesc{Postkarte
\newline{}Handschrift: schwarze Tinte, deutsche Kurrent\newline{}Versand: 1) Stempel: »\nobreak{}\oindex{Rodaun@\textbf{Rodaun}, \emph{Teil eines besiedelten Ortes (A.BSOX)}|pwk}Rodaun, 1 2 {[}1904{]}, 9–12N\nobreak{}«.  2) Stempel: »\nobreak{}\oindex{XVIII., Waehring@\textbf{XVIII., Währing}, \emph{Bezirk (A.BZK)}|pwk}18/1 Wien 110, 2 2 04, 8.V\nobreak{}«. \newline{}Ordnung: 1) mit Bleistift von unbekannter Hand
                           nummeriert: »214« 2) mit Bleistift von unbekannter Hand nummeriert: »214«}\buchAbdrucke{\weitereDrucke{Hugo von Hofmannsthal, Arthur Schnitzler: \emph{Briefwechsel}. Hg. Therese Nickl und Heinrich Schnitzler. Frankfurt am Main: \emph{S. Fischer} 1964, S. 182.} }\toendnotes[C]{\smallbreak}\pstart{}{\pb}\textsc{Herrn D\textsuperscript{r} Arthur
                  Schnitzler}\pend{}\pstart{}\textcolor{pink}{\textsc{Wien}}{}\ledrightnote{\textcolor{pink}{Wien}}\pend{}\pstart{}\textcolor{pink}{\textsc{\damage{\textcolor{gray}{XV}}III Spöttelgasse 7}}{}\ledrightnote{\textcolor{pink}{Edmund-Weiß-Gasse}}\pend{}\pstart{}neben \textsc{\textcolor{pink}{Türkenschanzstrasse}{}\ledrightnote{\textcolor{pink}{Türkenschanzstrasse}}}\pend{}{\bigskip}\pstart{}{\pb}lieber, \pend\pstart
           \textcolor{blue}{Edgar Karg}{}\ledrightnote{\textcolor{blue}{Edgar von Karg-Bebenburg}}, der Marineur, hat Sie ſehr gern und
               möchte Sie ſehr gern wieder ſehen und auch Ihre \textcolor{blue}{Frau}{}\ledrightnote{→\textcolor{blue}{Olga Schnitzler}} kennen. Da Ihr nun nie zu uns kommt und diesmal wieder
               abgeſagt habt, ſo habe ich ihn für morgen zum Nachtmahl in die \textcolor{pink}{Spöttelgaſſe N\textsuperscript{o} 7}{}\ledrightnote{\textcolor{pink}{Edmund-Weiß-Gasse}} eingeladen.\pend
           \pstart
           Von Herzen Ihr\pend
           \pstart \spacefill\mbox{Hugo}\pend{}\pstart
           \noindent{}\textcolor{pink}{Rodaun}{}\ledrightnote{\textcolor{pink}{Rodaun}}{ }Montag.\pend
           \endnumbering\briefempfaengerindex{Schnitzler, Arthur@\textsc{Schnitzler, Arthur}!zzzHofmannsthal, Hugo von@\emph{von Hugo von Hofmannsthal}!1904-02-012@{1. 2. 1904}|)be}\mylabel{h}  \normalsize

\doendnotes{C}
\bigskip
\vfill

\clearpage

\footnotesize

\lohead{\textsc{register}}

% Definiere theindex-Environment komplett neu ohne reledmac
\makeatletter
\renewenvironment{theindex}{%
  \section*{\indexname}%
  \setlength{\parindent}{0pt}%
  \setlength{\parskip}{0pt plus 0.3pt}%
  \let\item\@idxitem
}{%
  \clearpage
}
\makeatother

\IfFileExists{\jobname-pw.ind}{\input{\jobname-pw.ind}}{}

\end{document}

      