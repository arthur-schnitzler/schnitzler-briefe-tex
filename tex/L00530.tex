%% latex-korrekturansicht-vorspann.tex
%% Vorspann für die Korrekturansicht.
%% Lädt die gemeinsame Datei latex-vorspann.tex mit gesetztem Schalter.

\newif\ifkorrekturansicht
\korrekturansichttrue

\input{../tex-inputs/latex-vorspann}


               \section[Lou Andreas-Salomé an Arthur Schnitzler, 28. 1. 1896]{ Lou Andreas-Salomé an Arthur Schnitzler, 28. 1. 1896}\nopagebreak\mylabel{v}\rehead{ }\normalsize\beginnumbering\briefempfaengerindex{Schnitzler, Arthur@\textsc{Schnitzler, Arthur}!zzzAndreas-Salome, Lou@\emph{von Lou Andreas-Salomé}!1896-01-281@{28. 1. 1896}|(be} \toendnotes[C]{\smallbreak\pagebreak[2]} \Standort{CUL, Schnitzler, B 3.}
\physDesc{Kartenbrief
\newline{}Handschrift: schwarze Tinte, deutsche Kurrent\newline{}Versand: 1) Stempel: »\nobreak{}\oindex{I., Innere Stadt@\textbf{I., Innere Stadt}, \emph{Bezirk (A.BZK)}|pwk}Wien 1/1, 28. 1. 96, 9 10 N\nobreak{}«.  2) Stempel: »\nobreak{}\oindex{IX., Alsergrund@\textbf{IX., Alsergrund}, \emph{Bezirk (A.BZK)}|pwk}Wien {[}9/3{]}, 29.1{[}.96{]}, 8 {[}V{]}\nobreak{}«. \newline{}Ordnung: mit Bleistift von unbekannter Hand nummeriert: »17« }\toendnotes[C]{\smallbreak}\pstart{}{\pb}Herrn \textsc{D\textsuperscript{r}}\pend{}\pstart{}\textsc{Arthur Schnitzler}\pend{}\pstart{}\textsc{\textcolor{pink}{Wien IX}{}\ledrightnote{\textcolor{pink}{IX., Alsergrund}}}\pend{}\pstart{}\textcolor{pink}{Frankgasse 1}{}\ledrightnote{\textcolor{pink}{Frankgasse}}.
                    \pend{}{\bigskip}\pstart
           \noindent{}{\pb}Lieber Herr \textsc{D\textsuperscript{r}}, danke für Ihren Beſuch. ich ſchlief ſo feſt, daß ich Sie
                    nicht einmal klopfen gehört habe. Sie werden vor mir in \textcolor{pink}{Berlin}{}\ledrightnote{\textcolor{pink}{Berlin}}{ }ſein: wollen Sie ſo gut ſein, mir \uline{hierher nach \textcolor{pink}{Wien}{}\ledrightnote{\textcolor{pink}{Wien}}} eine Karte mit Angabe Ihrer Hôteladreſſe zu ſchicken? ich ſuche Sie gleich
                    auf, ſobald ich ankomme, – \uline{wenn} ich ankomme.
                    Aber ich weiß es, von Stunde zu Stunde, nicht, wann das ſein wird.\pend
           \pstart
           Sie werden gewiß viel Freude in \textcolor{pink}{Berlin}{}\ledrightnote{\textcolor{pink}{Berlin}}
                    erleben; ich wünſche Ihnen eine gute \textcolor{green}{Beſetzung}{}\ledrightnote{→\textcolor{green}{Liebelei. Schauspiel in drei Akten}} und viel, viel Glück.\pend
           \pstart
           Herzlich Ihre{\\[\baselineskip]}\spacefill\mbox{LouAS.}\pend
           \leftskip=0em{}\endnumbering\briefempfaengerindex{Schnitzler, Arthur@\textsc{Schnitzler, Arthur}!zzzAndreas-Salome, Lou@\emph{von Lou Andreas-Salomé}!1896-01-281@{28. 1. 1896}|)be}\mylabel{h}  \normalsize

\doendnotes{C}
\bigskip
\vfill

\clearpage

\footnotesize

\lohead{\textsc{register}}

% Definiere theindex-Environment komplett neu ohne reledmac
\makeatletter
\renewenvironment{theindex}{%
  \section*{\indexname}%
  \setlength{\parindent}{0pt}%
  \setlength{\parskip}{0pt plus 0.3pt}%
  \let\item\@idxitem
}{%
  \clearpage
}
\makeatother

\IfFileExists{\jobname-pw.ind}{\input{\jobname-pw.ind}}{}

\end{document}

      