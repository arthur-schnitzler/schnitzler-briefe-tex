%% latex-korrekturansicht-vorspann.tex
%% Vorspann für die Korrekturansicht.
%% Lädt die gemeinsame Datei latex-vorspann.tex mit gesetztem Schalter.

\newif\ifkorrekturansicht
\korrekturansichttrue

\input{../tex-inputs/latex-vorspann}


               \section[Arthur Schnitzler an Hermann Bahr, 6. 6. 1922]{ Arthur Schnitzler an Hermann Bahr, 6. 6. 1922}\nopagebreak\mylabel{v}\rehead{ }\normalsize\beginnumbering\briefempfaengerindex{Bahr, Hermann@\textsc{Bahr, Hermann}!zzzSchnitzler, Arthur@\emph{von Arthur Schnitzler}!1922-06-061@{6. 6. 1922}|(be} \toendnotes[C]{\smallbreak\pagebreak[2]} \Standort{TMW, HS AM 60137 Ba.}
\physDesc{Postkarte
\newline{}Handschrift: schwarze Tinte, deutsche Kurrent\newline{}Versand: 1) Stempel: »\nobreak{}Wien, 7. VI. 22, 8\nobreak{}«.  2) mit Bleistift von unbekannter Hand Ergänzung der Adresse:
                                    »NW 18«, die erste Ziffer überschrieben mit:
                                    »3«}\buchAbdrucke{\weitereDrucke{1) \emph{6. 6. 1922, Abschrift.} In: Arthur Schnitzler: \emph{The Letters of Arthur Schnitzler to Hermann Bahr}. Edited, annotated, and with an introduction, by Donald G.
                        Daviau. Chapel Hill: \emph{The University of North Carolina Press} 1978, S. 116 (University of North Carolina studies in the Germanic languages
                        and literatures, 89).} \weitereDrucke{2) Hermann Bahr, Arthur Schnitzler: \emph{Briefwechsel, Aufzeichnungen, Dokumente (1891–1931)}. Hg. Kurt Ifkovits und Martin Anton Müller. Göttingen: \emph{Wallstein} 2018, S. 561.} }\toendnotes[C]{\smallbreak}\pstart{}{\pb}\textsc{A. S.}\pend{}\pstart{}\textcolor{pink}{Wien XVIII}{}\ledrightnote{\textcolor{pink}{XVIII., Währing}}\pend{}\pstart{}\textsc{\textcolor{pink}{Sternwstr 71}{}\ledrightnote{\textcolor{pink}{Sternwartestraße}}}\pend{}{\bigskip}\pstart{}Herr Hermann Bahr\pend{}\pstart{}\textcolor{pink}{München}{}\ledrightnote{\textcolor{pink}{München}}\pend{}\pstart{}\textcolor{pink}{Barerſtraße.}{}\ledrightnote{\textcolor{pink}{Barerstraße}}\pend{}{\bigskip}\pstart
           \raggedleft{}{\pb}\textcolor{pink}{Wien}{}\ledrightnote{\textcolor{pink}{Wien}}, 6. 6. 22\pend
           \pstart
           Mein lieber Hermann, laß dir vorläufig auf dieſem Weg für die
               ausführlichen, freundſchaftlichen warmherzigen \textcolor{green}{Grüße}{}\ledrightnote{→\textcolor{green}{Brief an Arthur Schnitzler}{\newline}→\textcolor{green}{Arthur Schnitzler zu seinem sechzigsten Geburtstag}}{ }\substVorne{}\textsuperscript{ſ}\substDazwischen{}d\substHinten{}anken, die du mir durch die Zeitungen zu meinem Geburtstag geſandt hast. In
               dieſem So{\geminationm}er hoffe ich zuverſichtlich dir
               endlich wieder die Hand drücken zu kö{\geminationn}en. Ich
                  \textcolor{gray}{nehme} an, du bleibſt vorläufig in \textcolor{pink}{München}{}\ledrightnote{\textcolor{pink}{München}}, {\pb}ich
               komme wohl durch und darf dich aufſuchen!\pend
           \pstart
           Mit tauſen{[}d{]} Grüßen,{\\[\baselineskip]}Dein getreuer{\\[\baselineskip]}\spacefill\mbox{Arthur }\pend
           \leftskip=0em{}\endnumbering\briefempfaengerindex{Bahr, Hermann@\textsc{Bahr, Hermann}!zzzSchnitzler, Arthur@\emph{von Arthur Schnitzler}!1922-06-061@{6. 6. 1922}|)be}\mylabel{h}  \normalsize

\doendnotes{C}
\bigskip
\vfill

\clearpage

\footnotesize

\lohead{\textsc{register}}

% Definiere theindex-Environment komplett neu ohne reledmac
\makeatletter
\renewenvironment{theindex}{%
  \section*{\indexname}%
  \setlength{\parindent}{0pt}%
  \setlength{\parskip}{0pt plus 0.3pt}%
  \let\item\@idxitem
}{%
  \clearpage
}
\makeatother

\IfFileExists{\jobname-pw.ind}{\input{\jobname-pw.ind}}{}

\end{document}

      