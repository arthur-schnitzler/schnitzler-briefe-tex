%% latex-korrekturansicht-vorspann.tex
%% Vorspann für die Korrekturansicht.
%% Lädt die gemeinsame Datei latex-vorspann.tex mit gesetztem Schalter.

\newif\ifkorrekturansicht
\korrekturansichttrue

\input{../tex-inputs/latex-vorspann}


               \section[Hermann Bahr an Arthur Schnitzler, 10. 7. {[}1902{]}]{ Hermann Bahr an Arthur Schnitzler, 10. 7. {[}1902{]}}\nopagebreak\mylabel{v}\rehead{ }\normalsize\beginnumbering\briefempfaengerindex{Schnitzler, Arthur@\textsc{Schnitzler, Arthur}!zzzBahr, Hermann@\emph{von Hermann Bahr}!1902-07-101@{10. 7. 1902}|(be} \toendnotes[C]{\smallbreak\pagebreak[2]} \Standort{CUL, Schnitzler, B 5b.}
\physDesc{Brief, 1 Blatt (Briefpapier mit Trauerrand), 4 Seiten
\newline{}Handschrift: schwarze Tinte, deutsche Kurrent\newline{}Ordnung: mit Bleistift von unbekannter Hand nummeriert: »90« }\buchAbdrucke{\weitereDrucke{Hermann Bahr, Arthur Schnitzler: \emph{Briefwechsel, Aufzeichnungen, Dokumente (1891–1931)}. Hg. Kurt Ifkovits und Martin Anton Müller. Göttingen: \emph{Wallstein} 2018, S. 241.} }\pstart
           \raggedleft{}{\pb}10. Juli\pend
           \pstart\center{}Lieber Arthur!\pend\pstart
           Denſelben Wiſch hat \textcolor{blue}{\textsc{Burckhard}}{}\ledrightnote{\textcolor{blue}{Max Eugen Burckhard}} bekommen, voriges Jahr \textcolor{blue}{\textsc{Karlweis}}{}\ledrightnote{\textcolor{blue}{Carl Karlweis}} und \textcolor{blue}{\textsc{Chiavacci}}{}\ledrightnote{\textcolor{blue}{Vincenz Chiavacci}}, und mit derſelben Wirkung: einer
               Anfrage bei mir. Gesetzlich biſt Du verpflichtet, eine Antwort zu geben. \uline{Ich} werde aber, wenn ich jemals befragt werde,
               antworten, daß ich das Einkommen {\pb}auch meiner
               nächſten Freunde weder kenne noch mir darüber Gedanken mache, weil es mich gar nicht
               intereſſiert.\pend
           \pstart
           Übrigens theile ich Dir der Wahrheit gemäß mit: 1) Daß in der Zeit vom 1. Januar bis
               zum 31. December 1901 überhaupt kein Stück von mir in \textcolor{pink}{Berlin}{}\ledrightnote{\textcolor{pink}{Berlin}} aufgeführt wurde; {\pb}2) Daß in \textcolor{pink}{Wien}{}\ledrightnote{\textcolor{pink}{Wien}} am \textcolor{pink}{Deutſchen
                  Volkstheater}{}\ledrightnote{\textcolor{pink}{Volkstheater}} noch »\textcolor{green}{Wienerinnen}{}\ledrightnote{\textcolor{green}{Wienerinnen}}« weiter
               gegeben wurde, daß aber der eigentliche Zug dieſes im Oktober 1900 zum erſten Mal
               aufgeführten Stückes im Jänner 1901 bereits vorüber war.\pend
           \pstart
           3) Daß in \textcolor{pink}{Wien}{}\ledrightnote{\textcolor{pink}{Wien}} am \textcolor{pink}{Burgtheater}{}\ledrightnote{\textcolor{pink}{Burgtheater}} der »\textcolor{green}{Apoſtel}{}\ledrightnote{\textcolor{green}{Der Apostel}}« im November
               und December 1901 zehn Mal gegeben, die {\pb}Tantièmen
               hiefür erſt am 4. Januar verrechnet, erſt im Februar von mir behoben wurden und alſo
               nicht \textsc{pro} 1901 fatiert werden konnten. Und nun rechne Dir
               meine Reichthümer aus! Roman oder Novelle habe ich 1901 keine geſchrieben.\pend
           \pstart
           Herzlichſt{\\[\baselineskip]}Dein alter{\\[\baselineskip]}\spacefill\mbox{Hermann}\pend
           \leftskip=0em{}\endnumbering\briefempfaengerindex{Schnitzler, Arthur@\textsc{Schnitzler, Arthur}!zzzBahr, Hermann@\emph{von Hermann Bahr}!1902-07-101@{10. 7. 1902}|)be}\mylabel{h}  \normalsize

\doendnotes{C}
\bigskip
\vfill

\clearpage

\footnotesize

\lohead{\textsc{register}}

% Definiere theindex-Environment komplett neu ohne reledmac
\makeatletter
\renewenvironment{theindex}{%
  \section*{\indexname}%
  \setlength{\parindent}{0pt}%
  \setlength{\parskip}{0pt plus 0.3pt}%
  \let\item\@idxitem
}{%
  \clearpage
}
\makeatother

\IfFileExists{\jobname-pw.ind}{\input{\jobname-pw.ind}}{}

\end{document}

      