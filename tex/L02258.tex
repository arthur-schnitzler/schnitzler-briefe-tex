%% latex-korrekturansicht-vorspann.tex
%% Vorspann für die Korrekturansicht.
%% Lädt die gemeinsame Datei latex-vorspann.tex mit gesetztem Schalter.

\newif\ifkorrekturansicht
\korrekturansichttrue

\input{../tex-inputs/latex-vorspann}


               \section[Arthur Schnitzler an Arno Holz, 28. 2. 1917]{ Arthur Schnitzler an Arno Holz, 28. 2. 1917}\nopagebreak\mylabel{v}\rehead{ }\normalsize\beginnumbering\briefempfaengerindex{Holz, Arno@\textsc{Holz, Arno}!zzzSchnitzler, Arthur@\emph{von Arthur Schnitzler}!1917-02-281@{28. 2. 1917}|(be} \toendnotes[C]{\smallbreak\pagebreak[2]} \Standort{DLA, A:Schnitzler, HS.NZ85.1.1040.}
\physDesc{Brief, 1 Blatt, 1 Seite, maschineller Durchschlag
\newline{}Schreibmaschine
\newline{}Handschrift: Bleistift, lateinische Kurrent (\noindent{}Beschriftung: »Arno Holz« und drei Unterstreichungen)}\toendnotes[C]{\smallbreak}\pstart
           \raggedleft{}{\pb}28. 2. 1917\pend
           \pstart\center{}Sehr verehrter Herr Holz.\pend\pstart
           Möchten Sie mir vielleicht gütigst einige Prospekte oder was dafür gelten könnte
                    über die von Ihnen projektierte Ausgabe der »\textcolor{green}{Blechschmiede}{}\ledrightnote{\textcolor{green}{Die Blechschmiede}}« zukommen lassen? Es wäre wohl denkbar, dass man Ihnen
                    eventuell auch durch meinen \textcolor{blue}{Buchhändler}{}\ledrightnote{→\textcolor{blue}{Hugo Heller}}, der hauptsächlich in Bibliophilenkreisen bekannt ist,
                    eine Anzahl Subscribenten verschaffen könnte. Auf mich muss ich Sie leider
                    bitten diesmal zu verzichten. Meine Einnahmen sind so erheblich gesunken, meine
                    Ausgaben so ungeheuerlich gestiegen, dass \introOben{}ich\introOben{} es mir
                    leider versagen muss, für ein Buch und wäre es das allerschönste hundert Mark zu
                    verausgaben.\pend
           \pstart
           In besonderer Hochachtung{\\[\baselineskip]} Ihr sehr ergebener\pend
           \leftskip=0em{}\endnumbering\briefempfaengerindex{Holz, Arno@\textsc{Holz, Arno}!zzzSchnitzler, Arthur@\emph{von Arthur Schnitzler}!1917-02-281@{28. 2. 1917}|)be}\mylabel{h}  \normalsize

\doendnotes{C}
\bigskip
\vfill

\clearpage

\footnotesize

\lohead{\textsc{register}}

% Definiere theindex-Environment komplett neu ohne reledmac
\makeatletter
\renewenvironment{theindex}{%
  \section*{\indexname}%
  \setlength{\parindent}{0pt}%
  \setlength{\parskip}{0pt plus 0.3pt}%
  \let\item\@idxitem
}{%
  \clearpage
}
\makeatother

\IfFileExists{\jobname-pw.ind}{\input{\jobname-pw.ind}}{}

\end{document}

      