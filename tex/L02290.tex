%% latex-korrekturansicht-vorspann.tex
%% Vorspann für die Korrekturansicht.
%% Lädt die gemeinsame Datei latex-vorspann.tex mit gesetztem Schalter.

\newif\ifkorrekturansicht
\korrekturansichttrue

\input{../tex-inputs/latex-vorspann}


               \section[Arthur Schnitzler an Robert Adam, 24. 7. 1918]{ Arthur Schnitzler an Robert Adam, 24. 7. 1918}\nopagebreak\mylabel{v}\rehead{ }\normalsize\beginnumbering\briefempfaengerindex{Adam, Robert@\textsc{Adam, Robert}!zzzSchnitzler, Arthur@\emph{von Arthur Schnitzler}!1918-07-241@{24. 7. 1918}|(be} \toendnotes[C]{\smallbreak\pagebreak[2]} \Standort{DLA, 96.34.2/10.}
\physDesc{Postkarte
\newline{}Handschrift: Bleistift, lateinische Kurrent\newline{}Versand: Stempel: »\nobreak{}24. VII. 18, 5\nobreak{}«.  }\pstart{}{\pb}\textcolor{gray}{\textbf{D\textsuperscript{R} ARTHUR
                                SCHNITZLER}}\pend{}\pstart{}\textcolor{gray}{\textbf{\textcolor{pink}{WIEN, XVIII. STERNWARTESTRASSE 71}{}\ledrightnote{\textcolor{pink}{Sternwartestraße}}.}}\pend{}{\bigskip}\pstart{}Herrn Dr.\pend{}\pstart{}Robert Adam Pollak\pend{}\pstart{}\textcolor{pink}{Andorf nahe Schärding}{}\ledrightnote{\textcolor{pink}{Andorf}}.\pend{}\pstart{}\textcolor{pink}{Innviertel}{}\ledrightnote{\textcolor{pink}{Innviertel}}.\pend{}{\bigskip}\pstart
           \raggedleft{}{\pb}24. 7. 18\pend
           \pstart
           lieber Herr Doctor,  vielen Dank für Ihre freundliche
                    Nachricht; ich freue mich, daß Sie’s so gut getroffen haben und hoffe Sie
                    bringen we{\geminationn} schon nichts geschriebenes doch eine
                    fruchtbare Sti{\geminationm}ung mit nach Hause. Ich bleibe wohl
                    bis Mitte August hier, um da{\geminationn} nach \textcolor{pink}{Bayern}{}\ledrightnote{\textcolor{pink}{Bayern}} abzureisen, u. zw. hätte ich nicht übel
                    Lust, donauaufwärts bis \textcolor{pink}{Passau}{}\ledrightnote{\textcolor{pink}{Passau}} zu fahren, u von dort erst nach \textcolor{pink}{München}{}\ledrightnote{\textcolor{pink}{München}}{ }{\pb}u. in weiterem Verlauf \textcolor{pink}{Partenkirchen}{}\ledrightnote{\textcolor{pink}{Partenkirchen}} abzubiegen. Wir sehen uns vielleicht noch vorher?
                    Herzlich grüßt Sie Ihr ergebner\pend
           \pstart \spacefill\mbox{A. S.}\pend{}\endnumbering\briefempfaengerindex{Adam, Robert@\textsc{Adam, Robert}!zzzSchnitzler, Arthur@\emph{von Arthur Schnitzler}!1918-07-241@{24. 7. 1918}|)be}\mylabel{h}  \normalsize

\doendnotes{C}
\bigskip
\vfill

\clearpage

\footnotesize

\lohead{\textsc{register}}

% Definiere theindex-Environment komplett neu ohne reledmac
\makeatletter
\renewenvironment{theindex}{%
  \section*{\indexname}%
  \setlength{\parindent}{0pt}%
  \setlength{\parskip}{0pt plus 0.3pt}%
  \let\item\@idxitem
}{%
  \clearpage
}
\makeatother

\IfFileExists{\jobname-pw.ind}{\input{\jobname-pw.ind}}{}

\end{document}

      