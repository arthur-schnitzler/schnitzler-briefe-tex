%% latex-korrekturansicht-vorspann.tex
%% Vorspann für die Korrekturansicht.
%% Lädt die gemeinsame Datei latex-vorspann.tex mit gesetztem Schalter.

\newif\ifkorrekturansicht
\korrekturansichttrue

\input{../tex-inputs/latex-vorspann}


               \section[Arthur Schnitzler an Hugo von Hofmannsthal, 4. 8. 1900]{ Arthur Schnitzler an Hugo von Hofmannsthal, 4. 8. 1900}\nopagebreak\mylabel{v}\rehead{ }\normalsize\beginnumbering\briefempfaengerindex{Hofmannsthal, Hugo von@\textsc{Hofmannsthal, Hugo von}!zzzSchnitzler, Arthur@\emph{von Arthur Schnitzler}!1900-08-041@{4. 8. 1900}|(be} \toendnotes[C]{\smallbreak\pagebreak[2]} \Standort{FDH, Hs-30885,1.}
\physDesc{Brief, 1 Blatt, 3 Seiten
\newline{}Handschrift: schwarze Tinte, deutsche Kurrent}\buchAbdrucke{\weitereDrucke{Hugo von Hofmannsthal, Arthur Schnitzler: \emph{Briefwechsel}. Hg. Therese Nickl und Heinrich Schnitzler. Frankfurt am Main: \emph{S. Fischer} 1964, S. 144.} }\pstart
           \raggedleft{}{\pb}\textcolor{pink}{Iſchl}{}\ledrightnote{\textcolor{pink}{Bad Ischl}}, 4. 8. 900.\pend
           \pstart
           Mein lieber Hugo, ich bin ein paar Tage in \textcolor{pink}{Auſſee}{}\ledrightnote{\textcolor{pink}{Bad Aussee}} geweſen, jetzt in \textcolor{pink}{Iſchl}{}\ledrightnote{\textcolor{pink}{Bad Ischl}}, \textcolor{pink}{\textsc{Pension Petter}}{}\ledrightnote{\textcolor{pink}{Hotel und Pension Rudolfshöhe (Leopold Petter)}}, habe vor meinem Fenſter, auch jetzt, während ich ſchreibe, den ſchmalen
                    Weg, auf dem wir im vorigen Jahr nach dem Eſſen immer ſpazieren gegangen ſind
                    und über \textcolor{green}{Schleier}{}\ledrightnote{\textcolor{green}{Der Schleier der Beatrice. Schauspiel in fünf Akten}} und \textcolor{green}{Bergwerk}{}\ledrightnote{\textcolor{green}{Das Bergwerk zu Falun}} geſpro{\pb}chen haben. Heuer
                    geht es mir hier nicht ſo gut. Am 10. wahrſcheinlich fahr ich weg,
                    am 12. dürft ich in \textcolor{pink}{Salzburg}{}\ledrightnote{\textcolor{pink}{Salzburg}}{ }ſein und freue mich ſehr Sie dort noch
                    anzutreffen u. Ihnen mündlich ſagen zu können, wie ſehr von Herzen ich Ihnen
                    Glück wünſche. Aber bevor ich \textcolor{pink}{Iſchl}{}\ledrightnote{\textcolor{pink}{Bad Ischl}} verlaſſe,
                    ſchreib ich Ihnen noch ein Wort und höre vielleicht auch noch von Ihnen. Sie
                    wiſſen ja, {\pb}dſs \textcolor{blue}{Richard}{}\ledrightnote{\textcolor{blue}{Richard Beer-Hofmann}} auch nach \textcolor{pink}{S.}{}\ledrightnote{\textcolor{pink}{Salzburg}} ko{\geminationm}t, vielleicht auch \textcolor{blue}{Goldmann}{}\ledrightnote{\textcolor{blue}{Paul Goldmann}}.\pend
           \pstart
           Am 13.{ }Nachmittag dürften wir aufbrechen; ſpäteſtens am
                        14.\hspace*{2.5em}Auf Wiederſehen! Ihr
                        \spacefill\mbox{Arthur.}\pend
           \endnumbering\briefempfaengerindex{Hofmannsthal, Hugo von@\textsc{Hofmannsthal, Hugo von}!zzzSchnitzler, Arthur@\emph{von Arthur Schnitzler}!1900-08-041@{4. 8. 1900}|)be}\mylabel{h}  \normalsize

\doendnotes{C}
\bigskip
\vfill

\clearpage

\footnotesize

\lohead{\textsc{register}}

% Definiere theindex-Environment komplett neu ohne reledmac
\makeatletter
\renewenvironment{theindex}{%
  \section*{\indexname}%
  \setlength{\parindent}{0pt}%
  \setlength{\parskip}{0pt plus 0.3pt}%
  \let\item\@idxitem
}{%
  \clearpage
}
\makeatother

\IfFileExists{\jobname-pw.ind}{\input{\jobname-pw.ind}}{}

\end{document}

      