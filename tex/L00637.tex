%% latex-korrekturansicht-vorspann.tex
%% Vorspann für die Korrekturansicht.
%% Lädt die gemeinsame Datei latex-vorspann.tex mit gesetztem Schalter.

\newif\ifkorrekturansicht
\korrekturansichttrue

\input{../tex-inputs/latex-vorspann}


               \section[Richard Beer-Hofmann an Arthur Schnitzler, 14. 1. 1897]{ Richard Beer-Hofmann an Arthur Schnitzler,
               14. 1. 1897}\nopagebreak\mylabel{v}\rehead{ }\normalsize\beginnumbering\briefempfaengerindex{Schnitzler, Arthur@\textsc{Schnitzler, Arthur}!zzzBeer-Hofmann, Richard@\emph{von Richard Beer-Hofmann}!1897-01-141@{14. 1. 1897}|(be} \toendnotes[C]{\smallbreak\pagebreak[2]} \Standort{CUL, Schnitzler, B 8.}
\physDesc{Brief, 1 Blatt, 1 Seite
\newline{}Handschrift: blauer Buntstift, lateinische Kurrent\newline{}Ordnung: mit Bleistift von unbekannter Hand nummeriert: »92« }\buchAbdrucke{\weitereDrucke{Arthur Schnitzler, Richard Beer-Hofmann: \emph{Briefwechsel 1891–1931}. Hg. Konstanze Fliedl. Wien, Zürich: \emph{Europaverlag} 1992, S. 100.} }\pstart
           \raggedleft{}{\pb}14/I 97\pend
           \pstart
           Lieber Arthur! Wenn Sie morgen – Freitag – irgendwann
               zwischen 4 und 8{ }\uline{Zeit haben} – nur dann – bitte\strikeout{n}
                schauen Sie ein wenig zu mir. \textcolor{blue}{Paula}{}\ledrightnote{\textcolor{blue}{Paula Beer-Hofmann}}
               klagt über Stechen und Schmerzen{[}.{]} Herzlichst\pend
           \pstart Ihr \spacefill\mbox{Richard}\pend{}\endnumbering\briefempfaengerindex{Schnitzler, Arthur@\textsc{Schnitzler, Arthur}!zzzBeer-Hofmann, Richard@\emph{von Richard Beer-Hofmann}!1897-01-141@{14. 1. 1897}|)be}\mylabel{h}  \normalsize

\doendnotes{C}
\bigskip
\vfill

\clearpage

\footnotesize

\lohead{\textsc{register}}

% Definiere theindex-Environment komplett neu ohne reledmac
\makeatletter
\renewenvironment{theindex}{%
  \section*{\indexname}%
  \setlength{\parindent}{0pt}%
  \setlength{\parskip}{0pt plus 0.3pt}%
  \let\item\@idxitem
}{%
  \clearpage
}
\makeatother

\IfFileExists{\jobname-pw.ind}{\input{\jobname-pw.ind}}{}

\end{document}

      