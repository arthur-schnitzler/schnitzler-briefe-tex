%% latex-korrekturansicht-vorspann.tex
%% Vorspann für die Korrekturansicht.
%% Lädt die gemeinsame Datei latex-vorspann.tex mit gesetztem Schalter.

\newif\ifkorrekturansicht
\korrekturansichttrue

\input{../tex-inputs/latex-vorspann}


               \section[Hermann Bahr an Arthur Schnitzler, 17. 9. 1905]{ Hermann Bahr an Arthur Schnitzler, 17. 9. 1905}\nopagebreak\mylabel{v}\rehead{ }\normalsize\beginnumbering\briefempfaengerindex{Schnitzler, Arthur@\textsc{Schnitzler, Arthur}!zzzBahr, Hermann@\emph{von Hermann Bahr}!1905-09-172@{17. 9. 1905}|(be} \toendnotes[C]{\smallbreak\pagebreak[2]} \Standort{TMW, HS AM 39978 Ba und AM 39979 Ba.}
\physDesc{maschinelle Abschrift\newline{}Ordnung: Original nicht nachweisbar; auf der Mappe in der Cambridge University Library
                                 hat \textcolor{blue}{Heinrich Schnitzler} vermerkt,
                                 dass \textcolor{blue}{Olga Schnitzler} diesen Brief am
                                 15. 8. 1936 entnommen habe. }\buchAbdrucke{\weitereDrucke{Hermann Bahr, Arthur Schnitzler: \emph{Briefwechsel, Aufzeichnungen, Dokumente (1891–1931)}. Hg. Kurt Ifkovits und Martin Anton Müller. Göttingen: \emph{Wallstein} 2018, S. 351–352.} }\toendnotes[C]{\smallbreak}\pstart
           \raggedleft{}{\pb}17. 9. 1905\pend
           \pstart{}Lieber Arthur!\pend\pstart
           Ich war sehr verstimmt Dich heute verfehlt zu haben – ich bin sonst Vormittag fast
               immer zu Haus, nur heute musste ich ins \textcolor{pink}{Jubiläumstheater}{}\ledrightnote{\textcolor{pink}{Volksoper}}, da dieses, seines Patriotismus wegen, ausersehen ist, den
                  »\label{K_L01548_1v}\edtext{\textcolor{green}{Klub der Erlöser}{}\ledrightnote{\textcolor{green}{Der Klub der Erlöser}}}{\lemma{\textnormal{\emph{Klub der Erlöser}}}\Cendnote{\textnormal{Das Schauspiel wurde Ende
                     November von der Zensur nicht zur Aufführung zugelassen.}}}\label{K_L01548_1h}« zu
               bringen, den ich Dir nächstens schicke, er ist eine Parallele zu »\textcolor{green}{Unter sich}{}\ledrightnote{\textcolor{green}{Unter sich. Ein Arme-Leut’-Stück}}«. Nun habe ich sogleich den »\textcolor{green}{Ruf des Lebens}{}\ledrightnote{\textcolor{green}{Der Ruf des Lebens. Schauspiel in drei Akten}}« gelesen. Ich danke Dir herzlichst für die
               Absicht, ihn mir zu widmen, und Du machst mir eine sehr grosse Freude, wenn Du es
               wirklich tust. Seine »Gesinnung« (ich find im Augenblick nur dieses dumme Wort) hat
               mich sehr ergriffen und in \label{LL294-1v}dieser ungeheueren
                  Angst, die er ausdrückt und mitteilt, der Angst das Leben zu versäumen, das
                  einzige, das Höchste, geht er mir sehr nahe\label{LL294-1h}, ja ich glaube, dass Du noch
               nie so tief in das Gemüt unserer Generation und ihre letzte Sehnsucht eingedrungen
               bist. (An meinen »\label{K_L01548_2v}\edtext{\textcolor{green}{armen Narren}{}\ledrightnote{\textcolor{green}{Der arme Narr}}}{\lemma{\textnormal{\emph{armen Narren}}}\Cendnote{\textnormal{\textcolor{blue}{Hermann Bahr}: \emph{\textcolor{green}{Der arme Narr. Schauspiel in einem Akt}}. In: \emph{\textcolor{green}{Österreichische Rundschau}}, Jg. 4, H. 48,
                        28. 9. 1905, S. 396–407; H. 49, 5. 10. 1905,
                     S. 444–451; H. 50, 12. 10. 1905, S. 490–497.}}}\label{K_L01548_2h}«, von
               dem ich nur noch kein Exemplar für Dich frei habe, und einem kleinen \label{K_L01548_3v}\edtext{\textcolor{green}{Kainzbüchel}{}\ledrightnote{→\textcolor{green}{Josef Kainz}}}{\lemma{\textnormal{\emph{Kainzbüchel}}}\Cendnote{\textnormal{Hermann Bahr: \emph{\textcolor{green}{Josef Kainz}}. Wien, Leipzig:
                        \emph{\textcolor{brown}{Wiener Verlag}}{ }1906.}}}\label{K_L01548_3h}, das bei \textcolor{blue}{Freund}{}\ledrightnote{\textcolor{blue}{Fritz Freund}} kommt, wirst
                  \label{LL294-2v}Du sehen, dass mir dies, gerade dies und
                  eigentlich nur dies allein unser eigentliches Problem scheint, von dem mir alle
                  anderen unserer Forderungen oder Fragen nur Abwandlungen oder Variationen
                  scheinen\label{LL294-2h}). Was nun die Ausführung betrifft, einstweilen unter {\pb}dem ersten Eindruck nur folgendes: prachtvoll finde ich den Vater, von einer
               Plastik, die vielleicht noch nie eine Figur von Dir gehabt hat, ebenso stehen mir
               Marie und die gleich von mir geliebte Katharina wunderbar lebendig da, auch Dr.
               Schindler und Rainer sehe und höre ich, wenn schon ferner und stiller als jene.
               Dagegen (die Schuld mag an mir liegen, ich will Dir auch nur meinen ersten Eindruck
               sagen, wie sich ja schliesslich auch das Publikum immer nur an den unmittelbaren
               Eindruck hält) dagegen sehe ich den \textcolor{green}{Obersten}{}\ledrightnote{→\textcolor{green}{Der Ruf des Lebens. Schauspiel in drei Akten}}, seine Frau und \textcolor{green}{Max}{}\ledrightnote{→\textcolor{green}{Der Ruf des Lebens. Schauspiel in drei Akten}} gar nicht. Den Obersten \uline{kann} ich mir
               denken, und es reizt mich sehr, mir ihn zu denken, er geht mir nach, ich ihm, und ich
               dichte mir sein ganzes Leben hinzu, bald dieses, bald jenes, aber dies bleibt meiner
               Willkür frei, ich \uline{muss} nicht, denn es ist doch zu
               wenig von seiner Vergangenheit da, und nichts, das mich zwingen würde, daraus sein
               ganzes Wesen zu erkennen. Was noch mehr für seine Frau und vom \textcolor{green}{L{[}i{]}eutenant}{}\ledrightnote{→\textcolor{green}{Der Ruf des Lebens. Schauspiel in drei Akten}} gilt. Die
               Kritik wird deshalb den zweiten Akt zu stark an Handlung und melodramatisch oder
               boulevarddramatisch oder dergleichen finden. Er ist es nicht, gewiss nicht, nur
               scheint mir der Ausgleich zwischen der auf die Handlung verteilten Kraft und der in
               die Figuren gelegten nicht völlig getroffen. Woher auch wohl das Gefühl stammt, das
               ich sehr lebhaft hatte, der Akt sei viel zu kurz, als ob alles nur angedeutet wäre,
               besonders an der sehr ruhig breiten Ausführung im ersten und dann wieder im dritten
               Akt gemessen. Doch über all das mündlich, sehr bald, wir müssen uns endlich einmal
               gründlich sehen. Grüss Deine liebe \textcolor{blue}{Frau}{}\ledrightnote{→\textcolor{blue}{Olga Schnitzler}} bestens und sei herzlichst gegrüsst von Deinem\pend
           \pstart \spacefill\mbox{H.}\pend{}\pstart
           \noindent{}Brauchst Du das Manuscript zurück? »\textcolor{green}{Zwischenspiel}{}\ledrightnote{\textcolor{green}{Zwischenspiel. Komödie in drei Akten}}« les ich morgen.\pend
           \endnumbering\briefempfaengerindex{Schnitzler, Arthur@\textsc{Schnitzler, Arthur}!zzzBahr, Hermann@\emph{von Hermann Bahr}!1905-09-172@{17. 9. 1905}|)be}\mylabel{h}  \normalsize

\doendnotes{C}
\bigskip
\vfill

\clearpage

\footnotesize

\lohead{\textsc{register}}

% Definiere theindex-Environment komplett neu ohne reledmac
\makeatletter
\renewenvironment{theindex}{%
  \section*{\indexname}%
  \setlength{\parindent}{0pt}%
  \setlength{\parskip}{0pt plus 0.3pt}%
  \let\item\@idxitem
}{%
  \clearpage
}
\makeatother

\IfFileExists{\jobname-pw.ind}{\input{\jobname-pw.ind}}{}

\end{document}

      