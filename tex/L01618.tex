%% latex-korrekturansicht-vorspann.tex
%% Vorspann für die Korrekturansicht.
%% Lädt die gemeinsame Datei latex-vorspann.tex mit gesetztem Schalter.

\newif\ifkorrekturansicht
\korrekturansichttrue

\input{../tex-inputs/latex-vorspann}


               \section[Hugo und Gerty von Hofmannsthal, Hans Schlesinger an Arthur Schnitzler, 29. 7. 1906]{ Hugo und Gerty von Hofmannsthal, Hans Schlesinger an Arthur Schnitzler,
               29. 7. 1906}\nopagebreak\mylabel{v}\rehead{ }\normalsize\beginnumbering\briefempfaengerindex{Schnitzler, Arthur@\textsc{Schnitzler, Arthur}!zzzSchlesinger, Hans Bernhard@\emph{von Hans Bernhard Schlesinger}!1906-07-291@{29. 7. 1906}|(be}\briefempfaengerindex{Schnitzler, Arthur@\textsc{Schnitzler, Arthur}!zzzHofmannsthal, Gertrude von@\emph{von Gertrude von Hofmannsthal}!1906-07-291@{29. 7. 1906}|(be}\briefempfaengerindex{Schnitzler, Arthur@\textsc{Schnitzler, Arthur}!zzzHofmannsthal, Hugo von@\emph{von Hugo von Hofmannsthal}!1906-07-291@{29. 7. 1906}|(be} \toendnotes[C]{\smallbreak\pagebreak[2]} \Standort{CUL, Schnitzler, B 43.}
\physDesc{Bildpostkarte
\newline{}Handschrift Hugo von Hofmannsthal: Bleistift, lateinische Kurrent\newline{}Handschrift Gertrude von Hofmannsthal: Bleistift, lateinische Kurrent\newline{}Handschrift Hans Bernhard Schlesinger: Bleistift, lateinische Kurrent\newline{}Versand: 1) Stempel: »\nobreak{}\oindex{Salzburg@\textbf{Salzburg}, \emph{Besiedelter Ort (A.BSO)}|pwk}Salzburg, 29. V. {[}06{]}, {[}3-{]}4\nobreak{}«.  2) Stempel: »\nobreak{}\oindex{Helsingør@\textbf{Helsingør}, \emph{Besiedelter Ort (A.BSO)}|pwk}Helsingør, 31. 7. 06, 9\nobreak{}«. \newline{}Ordnung: 1) mit Bleistift von unbekannter Hand nummeriert: »\strikeout{263}« 2) mit Bleistift von unbekannter Hand nummeriert: »264«}\buchAbdrucke{\weitereDrucke{Hugo von Hofmannsthal, Arthur Schnitzler: \emph{Briefwechsel}. Hg. Therese Nickl und Heinrich Schnitzler. Frankfurt am Main: \emph{S. Fischer} 1964, S. 220.} }\pstart{}{\pb}Herrn D\textsuperscript{r}\pend{}\pstart{}Arthur Schnitzler\pend{}\pstart{}\textcolor{pink}{Marienlyst}{}\ledrightnote{\textcolor{pink}{Marienlyst}}\pend{}\pstart{}\textcolor{pink}{Kurhaus}{}\ledrightnote{\textcolor{pink}{Kurhotellet}}\pend{}\pstart{}per \textcolor{pink}{Kopenhagen}{}\ledrightnote{\textcolor{pink}{Kopenhagen}}\pend{}\pstart{}\textcolor{pink}{Dänemark}{}\ledrightnote{\textcolor{pink}{Dänemark}}\pend{}{\bigskip}\pstart
           \noindent{}\centering{}\textcolor{gray}{\textbf{{\pb}\textcolor{pink}{Salzburg}{}\ledrightnote{\textcolor{pink}{Salzburg}}. – \textcolor{pink}{Kiosk
                        Tomaselli}{}\ledrightnote{\textcolor{pink}{Café Tomaselli}}.}}\pend
           \pstart
           \raggedleft{}29. VII\pend
           \pstart
           Früherer Jahre herzlich gedenkend\pend
           \pstart \spacefill\mbox{Hugo}\pend{}\pstart
           {[}hs. G. Hofmannsthal:{]} Auf dem Weg nach \textcolor{pink}{Baireuth}{}\ledrightnote{\textcolor{pink}{Bayreuth}}. Herzlichst \spacefill\mbox{Gerty}\pend
           \pstart
           {[}hs. Schlesinger:{]} Grüße \spacefill\mbox{Hans.}\pend
           \endnumbering\briefempfaengerindex{Schnitzler, Arthur@\textsc{Schnitzler, Arthur}!zzzSchlesinger, Hans Bernhard@\emph{von Hans Bernhard Schlesinger}!1906-07-291@{29. 7. 1906}|)be}\briefempfaengerindex{Schnitzler, Arthur@\textsc{Schnitzler, Arthur}!zzzHofmannsthal, Gertrude von@\emph{von Gertrude von Hofmannsthal}!1906-07-291@{29. 7. 1906}|)be}\briefempfaengerindex{Schnitzler, Arthur@\textsc{Schnitzler, Arthur}!zzzHofmannsthal, Hugo von@\emph{von Hugo von Hofmannsthal}!1906-07-291@{29. 7. 1906}|)be}\mylabel{h}  \normalsize

\doendnotes{C}
\bigskip
\vfill

\clearpage

\footnotesize

\lohead{\textsc{register}}

% Definiere theindex-Environment komplett neu ohne reledmac
\makeatletter
\renewenvironment{theindex}{%
  \section*{\indexname}%
  \setlength{\parindent}{0pt}%
  \setlength{\parskip}{0pt plus 0.3pt}%
  \let\item\@idxitem
}{%
  \clearpage
}
\makeatother

\IfFileExists{\jobname-pw.ind}{\input{\jobname-pw.ind}}{}

\end{document}

      