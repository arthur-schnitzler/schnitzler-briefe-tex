%% latex-korrekturansicht-vorspann.tex
%% Vorspann für die Korrekturansicht.
%% Lädt die gemeinsame Datei latex-vorspann.tex mit gesetztem Schalter.

\newif\ifkorrekturansicht
\korrekturansichttrue

\input{../tex-inputs/latex-vorspann}


               \section[Richard Beer-Hofmann an Arthur Schnitzler, 9. 12. 1909]{ Richard Beer-Hofmann an Arthur Schnitzler, 9. 12. 1909}\nopagebreak\mylabel{v}\rehead{ }\normalsize\beginnumbering\briefempfaengerindex{Schnitzler, Arthur@\textsc{Schnitzler, Arthur}!zzzBeer-Hofmann, Richard@\emph{von Richard Beer-Hofmann}!1909-12-091@{9. 12. 1909}|(be} \toendnotes[C]{\smallbreak\pagebreak[2]} \Standort{CUL, Schnitzler, B 8.}
\physDesc{Kartenbrief, 1 Blatt, 3 Seiten
\newline{}Handschrift: Bleistift, lateinische Kurrent\newline{}Versand: ohne postalischen Übermittlungsvermerk 
\newline{}Schnitzler: mit Bleistift beschriftet: »\textsc{Beerh.}« \newline{}Ordnung: mit Bleistift von unbekannter Hand nummeriert:
                              »226« }\buchAbdrucke{\weitereDrucke{Arthur Schnitzler, Richard Beer-Hofmann: \emph{Briefwechsel 1891–1931}. Hg. Konstanze Fliedl. Wien, Zürich: \emph{Europaverlag} 1992, S. 196.} }\toendnotes[C]{\smallbreak}\pstart{}{\pb}nicht dringend\pend{}\pstart{}Herrn\pend{}\pstart{}Arthur Schnitzler\pend{}\pstart{}\textcolor{pink}{Spöttelgasse 7}{}\ledrightnote{\textcolor{pink}{Edmund-Weiß-Gasse}}\pend{}{\bigskip}\pstart
           \raggedleft{}{\pb}9./XII. 09\pend
           \pstart
           Lieber Arthur! Soeben überfällt mich folgendes Telegra{\geminationm}: »Bin morgen, Freitag 2{ }\textcolor{pink}{Wien}{}\ledrightnote{\textcolor{pink}{Wien}} wäre sehr dankbar wenn mich 3 Uhr{ }\textcolor{pink}{Hasenauerstr}{}\ledrightnote{\textcolor{pink}{Hasenauerstraße}} erwarten und mir baldmöglichst
               consultation Arthur Schnitzler ermoeglichen wollten herzlichst \textcolor{blue}{poldi andrian}{}\ledrightnote{\textcolor{blue}{Leopold von Andrian-Werburg}}«. Ich sehe Sie ja morgen Vorm (\strikeout{voraussichtlich} – hoffentlich) schreibe {\pb}Ihnen aber jetzt, – damit Sie es sich
               einteilen können. Entweder – dass ich ihn zu Ihnen hinüberschicke, oder dass Sie zu mir herüberko{\geminationm}en. \label{K_L01895-1v}\edtext{Grossvater \textcolor{blue}{Giacomo}{}\ledrightnote{\textcolor{blue}{Giacomo Meyerbeer}}}{\lemma{\textnormal{\emph{Grossvater Giacomo}}}\Cendnote{\textnormal{\textcolor{blue}{Leopold Andrian} war mütterlicherseits ein Enkel des Komponisten \textcolor{blue}{Giacomo Meyerbeer}.}}}\label{K_L01895-1h}’s Nerven?\pend
           \pstart
           Herzlichst{\\[\baselineskip]}\spacefill\mbox{Richard}\pend
           \leftskip=0em{}\endnumbering\briefempfaengerindex{Schnitzler, Arthur@\textsc{Schnitzler, Arthur}!zzzBeer-Hofmann, Richard@\emph{von Richard Beer-Hofmann}!1909-12-091@{9. 12. 1909}|)be}\mylabel{h}  \normalsize

\doendnotes{C}
\bigskip
\vfill

\clearpage

\footnotesize

\lohead{\textsc{register}}

% Definiere theindex-Environment komplett neu ohne reledmac
\makeatletter
\renewenvironment{theindex}{%
  \section*{\indexname}%
  \setlength{\parindent}{0pt}%
  \setlength{\parskip}{0pt plus 0.3pt}%
  \let\item\@idxitem
}{%
  \clearpage
}
\makeatother

\IfFileExists{\jobname-pw.ind}{\input{\jobname-pw.ind}}{}

\end{document}

      