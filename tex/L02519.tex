%% latex-korrekturansicht-vorspann.tex
%% Vorspann für die Korrekturansicht.
%% Lädt die gemeinsame Datei latex-vorspann.tex mit gesetztem Schalter.

\newif\ifkorrekturansicht
\korrekturansichttrue

\input{../tex-inputs/latex-vorspann}


               \section[Christiane von Hofmannsthal an Arthur Schnitzler, 13. 8. 1929]{ Christiane von Hofmannsthal an Arthur Schnitzler,
                    13. 8. 1929}\nopagebreak\mylabel{v}\rehead{ }\normalsize\beginnumbering\briefempfaengerindex{Schnitzler, Arthur@\textsc{Schnitzler, Arthur}!zzzHofmannsthal, Christiane von@\emph{von Christiane von Hofmannsthal}!1929-08-131@{13. 8. 1929}|(be} \toendnotes[C]{\smallbreak\pagebreak[2]} \Standort{CUL, Schnitzler, B 43.}
\physDesc{Brief, 1 Blatt (Briefpapier mit Trauerrand), 1 Seite
\newline{}Schreibmaschine
\newline{}Handschrift: 1) schwarze Tinte (\noindent{}Unterschrift)\hspace{1em}2) Bleistift, lateinische Kurrent (\noindent{}Fußnote,
                                        Fußnotenzeichen)\hspace{1em}
\newline{}Schnitzler: mit rotem Buntstift fünf Unterstreichungen }\toendnotes[C]{\smallbreak}\pstart
           \raggedleft{}{\pb}\textcolor{pink}{Bad Aussee}{}\ledrightnote{\textcolor{pink}{Bad Aussee}}, am 13. August 1929\pend
           \pstart{}Lieber Arthur,\pend\pstart
           Danke für Deinen lieben Brief, ich erwarte also die Briefe von Frl. \textcolor{blue}{Pollack}{}\ledrightnote{\textcolor{blue}{Frieda Pollak}} zu bekommen.\pend
           \pstart
           Wenn wir einen oder den anderen für das neue \label{K_L02519_1v}\edtext{\textcolor{green}{Rundschau}{}\ledrightnote{\textcolor{green}{Die neue Rundschau}}heft}{\lemma{\textnormal{\emph{Rundschauheft}}}\Cendnote{\textnormal{Im November erschienen erstmals Texte aus dem
                        Nachlass in der \emph{\textcolor{green}{Neuen Deutschen Rundschau}} (\emph{\textcolor{green}{Aus dem Nachlass}}, Jg. 40, H. 11,
                            S. 613–625), aber keine Briefe. Diese folgten erst im April
                            1930 (\emph{\textcolor{green}{Aus dem Nachlass}}, Jg. 41, H. 4,
                            S. 497–519).}}}\label{K_L02519_1h} für geeignet halten, werden wir ihn Dir
                    vorher zur Einsicht übersenden.\pend
           \pstart
           Bezüglich des \textcolor{blue}{Franzosen}{}\ledrightnote{→\textcolor{blue}{?? [Franzose, der sich für Hofmannsthal interessiert 1929]}}
                    weiss ich nicht recht, was da zu empfehlen wäre, als \textcolor{blue}{Papas}{}\ledrightnote{→\textcolor{blue}{Hugo von Hofmannsthal}} Werke selber? Es gibt eine ganz
                    brave \textcolor{pink}{französische}{}\ledrightnote{\textcolor{pink}{Frankreich}}{ }\label{K_L02519_2v}\edtext{\textcolor{green}{Thèse de Doctorat}{}\ledrightnote{→\textcolor{green}{La poésie autrichienne de Hofmannsthal à Rilke}}}{\lemma{\textnormal{\emph{Thèse de Doctorat}}}\Cendnote{\textnormal{\textcolor{blue}{Geneviève Bianquis}: \emph{\textcolor{green}{La poésie autrichienne de Hofmannsthal à Rilke}}.
                            Paris: \emph{\textcolor{brown}{Presses universitaires de France}}{ }1926.}}}\label{K_L02519_2h} von einer Mlle. \textcolor{blue}{Genevieve
                        Bianquis}{}\ledrightnote{\textcolor{blue}{Geneviève Bianquis}},\footnote{\noindent{}{[}hs.:{]} auch in Buchform erschienen.} wo alles sehr gewissenhaft, aber weiter nicht hervorragend\introOben{}es\introOben{} drinsteht, und dann gibts wohl nur einzelne Aufsätze
                    von Leuten über spezielle Sachen, aber da weiss ich auch nicht, was ich da
                    empfehlen soll. Vielleicht fällt Dir noch was Gescheites ein.\pend
           \pstart
           Hier ist es hässlich und regnerisch wie immer und eher traurig und zuviel
                    bekannte Menschen.\pend
           \pstart
           Sonst geht es soweit ganz gut.\pend
           \pstart
           Ich freue mich sehr, Dich im Herbst wiederzusehen und Deine
                    Ratschläge bekommen zu können.\pend
           \pstart
           Alles Liebe Deine{\\[\baselineskip]}\spacefill\mbox{{[}hs.:{]} Christiane}\pend
           \leftskip=0em{}\endnumbering\briefempfaengerindex{Schnitzler, Arthur@\textsc{Schnitzler, Arthur}!zzzHofmannsthal, Christiane von@\emph{von Christiane von Hofmannsthal}!1929-08-131@{13. 8. 1929}|)be}\mylabel{h}  \normalsize

\doendnotes{C}
\bigskip
\vfill

\clearpage

\footnotesize

\lohead{\textsc{register}}

% Definiere theindex-Environment komplett neu ohne reledmac
\makeatletter
\renewenvironment{theindex}{%
  \section*{\indexname}%
  \setlength{\parindent}{0pt}%
  \setlength{\parskip}{0pt plus 0.3pt}%
  \let\item\@idxitem
}{%
  \clearpage
}
\makeatother

\IfFileExists{\jobname-pw.ind}{\input{\jobname-pw.ind}}{}

\end{document}

      