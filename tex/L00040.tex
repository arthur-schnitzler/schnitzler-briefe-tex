%% latex-korrekturansicht-vorspann.tex
%% Vorspann für die Korrekturansicht.
%% Lädt die gemeinsame Datei latex-vorspann.tex mit gesetztem Schalter.

\newif\ifkorrekturansicht
\korrekturansichttrue

\input{../tex-inputs/latex-vorspann}


               \section[Arthur Schnitzler an Richard Beer-Hofmann, 16. 9. 1891]{ Arthur Schnitzler an Richard Beer-Hofmann, 16. 9. 1891}\nopagebreak\mylabel{v}\rehead{ }\normalsize\beginnumbering\briefempfaengerindex{Beer-Hofmann, Richard@\textsc{Beer-Hofmann, Richard}!zzzSchnitzler, Arthur@\emph{von Arthur Schnitzler}!1891-09-161@{16. 9. 1891}|(be} \toendnotes[C]{\smallbreak\pagebreak[2]} \Standort{YCGL, MSS 31.}
\physDesc{Briefkarte, Umschlag
\newline{}Handschrift: schwarze Tinte, deutsche Kurrent\newline{}Versand: 1) Stempel: »\nobreak{}\oindex{VIII., Josefstadt@\textbf{VIII., Josefstadt}, \emph{Bezirk (A.BZK)}|pwk}Wien 8, 16 9 91, \textcolor{gray}{5} N\nobreak{}«.  2) Stempel: »\nobreak{}\oindex{III., Landstrasse@\textbf{III., Landstraße}, \emph{Bezirk (A.BZK)}|pwk}Wien 3/2, 16-9 91, 5 7. N, Bestellt\nobreak{}«. }\toendnotes[C]{\smallbreak}\pstart{}{\pb}\textsc{Herrn Dr. Rich. Beer Hofmann}\pend{}\pstart{}\textsc{\textcolor{pink}{Wien}{}\ledrightnote{\textcolor{pink}{Wien}}}\pend{}\pstart{}\textsc{\textcolor{pink}{III. Seidlgasse 30}{}\ledrightnote{\textcolor{pink}{Seidlgasse}}.}\pend{}{\bigskip}\pstart
           \noindent{}{\pb}Lieber Freund, man will Sie bereits vor 14 Tagen in \textcolor{pink}{Baden}{}\ledrightnote{\textcolor{pink}{Baden bei Wien}} geſehen haben. Sind Sie da? Ich verreiſe am
                  Samſtag auf etwa 8 Tage nach \textcolor{pink}{\textsc{Halle an der Saale}}{}\ledrightnote{\textcolor{pink}{Halle an der Saale}} zur \label{K_L00040_1v}\edtext{Natur{\pb}forſcherverſa{\geminationm}lung}{\lemma{\textnormal{\emph{Naturforſcherverſalung}}}\Cendnote{\textnormal{Die 64. Versammlung der Gesellschaft Deutscher Naturforscher und
                  Ärzte fand vom 21. bis 25. 9. 1891 in \textcolor{pink}{Halle an der
                     Saale}{ }statt.}}}\label{K_L00040_1h}. – Wie ſteht’s mit \textcolor{pink}{Italien}{}\ledrightnote{\textcolor{pink}{Italien}}? Ka{\geminationn} ich für
               den Anfang Oktober auf Sie rechnen?\pend
           \pstart
           Herzlich Ihr{\\[\baselineskip]}\spacefill\mbox{Arthur}\pend
           \leftskip=0em{}\endnumbering\briefempfaengerindex{Beer-Hofmann, Richard@\textsc{Beer-Hofmann, Richard}!zzzSchnitzler, Arthur@\emph{von Arthur Schnitzler}!1891-09-161@{16. 9. 1891}|)be}\mylabel{h}  \normalsize

\doendnotes{C}
\bigskip
\vfill

\clearpage

\footnotesize

\lohead{\textsc{register}}

% Definiere theindex-Environment komplett neu ohne reledmac
\makeatletter
\renewenvironment{theindex}{%
  \section*{\indexname}%
  \setlength{\parindent}{0pt}%
  \setlength{\parskip}{0pt plus 0.3pt}%
  \let\item\@idxitem
}{%
  \clearpage
}
\makeatother

\IfFileExists{\jobname-pw.ind}{\input{\jobname-pw.ind}}{}

\end{document}

      