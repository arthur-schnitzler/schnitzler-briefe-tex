%% latex-korrekturansicht-vorspann.tex
%% Vorspann für die Korrekturansicht.
%% Lädt die gemeinsame Datei latex-vorspann.tex mit gesetztem Schalter.

\newif\ifkorrekturansicht
\korrekturansichttrue

\input{../tex-inputs/latex-vorspann}


               \section[Hugo von Hofmannsthal an Arthur Schnitzler, {[}23. 11. 1892?{]}]{ Hugo von Hofmannsthal an Arthur Schnitzler, {[}23. 11. 1892?{]}}\nopagebreak\mylabel{v}\rehead{ }\normalsize\beginnumbering\briefempfaengerindex{Schnitzler, Arthur@\textsc{Schnitzler, Arthur}!zzzHofmannsthal, Hugo von@\emph{von Hugo von Hofmannsthal}!1892-11-231@{{[}23. 11. 1892?{]}}|(be} \toendnotes[C]{\smallbreak\pagebreak[2]} \Standort{CUL, Schnitzler, B 43.}
\physDesc{Zwei Briefkarten, die zweite Karte nur in Abschrift überliefert
\newline{}Handschrift: 1) blaue Tinte, deutsche Kurrent (\noindent{}bis »macht aber nichts.«)\hspace{1em}2) schwarze Tinte, deutsche Kurrent (\noindent{}bis »Robert E«)\hspace{1em}3) Bleistift, deutsche Kurrent (\noindent{}ab »hrhardt und Paul Horn«)\hspace{1em}\newline{}Ordnung: mit Bleistift von unbekannter Hand nummeriert:
                                 »8« }\buchAbdrucke{\weitereDrucke{Hugo von Hofmannsthal, Arthur Schnitzler: \emph{Briefwechsel}. Hg. Therese Nickl und Heinrich Schnitzler. Frankfurt am Main: \emph{S. Fischer} 1964, S. 31.} }\toendnotes[C]{\smallbreak}\pstart
           \raggedleft{}{\pb}\label{K_L00138_1v}\edtext{Mittwoch}{\lemma{\textnormal{\emph{Mittwoch}}}\Cendnote{\textnormal{Die Datierung beruht auf dem Brief
                     vom 24. 11. 1892 (Arthur Schnitzler an Hugo von Hofmannsthal, 24. 11. 1892), bei dem
                     es sich um die Antwort auf diese Karte handeln dürfte.}}}\label{K_L00138_1h}\pend
           \pstart{}Lieber Arthur\pend\pstart
           Ich ſchreibe zufällig \label{T_L00138-1v}\edtext{an \textcolor{blue}{Richard}{}\ledrightnote{\textcolor{blue}{Richard Beer-Hofmann}}s Schreibtiſch}{\lemma{\textnormal{\emph{an Richards Schreibtiſch}}}\Cendnote{\textnormal{Papier und der verwendete blaue Stift entsprechen den Briefen
                     \textcolor{blue}{Richard Beer-Hofmann}s.}}}\label{T_L00138-1h}, das macht
               aber nichts. Ich möchte Ihnen nämlich etwas ſagen: \strikeout{wir} wir ſollten doch einmal wieder ein bischen unter uns zuſammenkommen.
                  \textcolor{blue}{Robert}{}\ledrightnote{\textcolor{blue}{Robert Ehrhart von Ehrhartstein}}{ }\textcolor{blue}{Ehrhardt}{}\ledrightnote{\textcolor{blue}{Robert Ehrhart von Ehrhartstein}} und \textcolor{blue}{\textsc{Paul Horn}}{}\ledrightnote{\textcolor{blue}{Paul Horn}} und alle ſind ja jeder in ſeiner Art ſehr nett, aber immer, das vergröbert und
               encanailliert naturgemäß Thema und Ton. Ich gehe deshalb nicht zu {\pb}\textcolor{pink}{Pfob}{}\ledrightnote{\textcolor{pink}{Café Pfob}}. Meinen Sie nicht auch? Wir haben ja sehr gut
               ohne das alles existiert. Uebrigens auf Wiedersehen Sonntag.\pend
           \pstart Ihr \spacefill\mbox{Loris}\pend{}\endnumbering\briefempfaengerindex{Schnitzler, Arthur@\textsc{Schnitzler, Arthur}!zzzHofmannsthal, Hugo von@\emph{von Hugo von Hofmannsthal}!1892-11-231@{{[}23. 11. 1892?{]}}|)be}\mylabel{h}  \normalsize

\doendnotes{C}
\bigskip
\vfill

\clearpage

\footnotesize

\lohead{\textsc{register}}

% Definiere theindex-Environment komplett neu ohne reledmac
\makeatletter
\renewenvironment{theindex}{%
  \section*{\indexname}%
  \setlength{\parindent}{0pt}%
  \setlength{\parskip}{0pt plus 0.3pt}%
  \let\item\@idxitem
}{%
  \clearpage
}
\makeatother

\IfFileExists{\jobname-pw.ind}{\input{\jobname-pw.ind}}{}

\end{document}

      