%% latex-korrekturansicht-vorspann.tex
%% Vorspann für die Korrekturansicht.
%% Lädt die gemeinsame Datei latex-vorspann.tex mit gesetztem Schalter.

\newif\ifkorrekturansicht
\korrekturansichttrue

\input{../tex-inputs/latex-vorspann}


               \section[Friedrich M. Fels an Arthur Schnitzler, 1{[}6{]}. 2. 1893]{ Friedrich M. Fels an Arthur Schnitzler, 1{[}6{]}. 2. 1893}\nopagebreak\mylabel{v}\rehead{ }\normalsize\beginnumbering\briefempfaengerindex{Schnitzler, Arthur@\textsc{Schnitzler, Arthur}!zzzFels, Friedrich Michael@\emph{von Friedrich Michael Fels}!1893-02-161@{1{[}6{]}. 2. 1893}|(be} \toendnotes[C]{\smallbreak\pagebreak[2]} \Standort{DLA, A:Schnitzler, HS.NZ85.1.2956.}
\physDesc{Brief, 1 Blatt, 3 Seiten
\newline{}Handschrift: schwarze Tinte, lateinische Kurrent
\newline{}Schnitzler: mit Bleistift nummeriert: »8« und unterhalb der
            Datumsangabe klein »16« vermerkt }\toendnotes[C]{\smallbreak}\pstart
           \raggedleft{}{\pb}\textcolor{pink}{Meran-Obermais, Hotel Erzherzog Rainer}{}\ledrightnote{\textcolor{pink}{Erzherzog Rainer}}{\\}\label{K_L00176_1v}\edtext{18. Februar 1893}{\lemma{\textnormal{\emph{18. Februar 1893}}}\Cendnote{\textnormal{Obzwar eindeutig auf den
                                18. datiert, geht aus dem Korrespondenzstück \textcolor{blue}{Schnitzler}s an \textcolor{blue}{Hofmannsthal} hervor, dass er an diesem Tag
                            bereits in \textcolor{pink}{Wien} war.}}}\label{K_L00176_1h}. \pend
           \pstart{}Lieber Dr. Schnitzler!\pend\pstart
           Verzeihen Sie, daſs ich Ihnen heute erst schreibe; aber erst gestern hat sich
                    entschieden, wo ich wohne, – und ich bin i{\geminationm}er so
                    müde! Aber ich will der Reihe nach erzählen.\pend
           \pstart
           Die Fahrt war furchtbar ermüdend: zum Mittageſsen in \textcolor{pink}{Franzensfeste}{}\ledrightnote{\textcolor{pink}{Franzensfeste}} 20 Minuten Aufenthalt, in \textcolor{pink}{Villach}{}\ledrightnote{\textcolor{pink}{Villach}} 15 – das war alles. Zum Glück hatte ich
                    verhältnismäſsig angenehme Gesellschaft, darunter Dr. \textcolor{blue}{Rullma{\geminationn}}{}\ledrightnote{\textcolor{blue}{Wilhelm Rullmann}}, den Redakteur des \label{K_L00176_2v}\edtext{\textcolor{brown}{\textcolor{brown}{Grazer Tagblatt}{}\ledrightnote{\textcolor{brown}{Grazer Tagblatt}}}{}\ledrightnote{→\textcolor{brown}{Tagespost}}s}{\lemma{\textnormal{\emph{Grazer Tagblatts}}}\Cendnote{\textnormal{Dies ist falsch, er arbeitete
                        für die \emph{\textcolor{brown}{Grazer Tagespost}}.}}}\label{K_L00176_2h}. Er lebt
                    jetzt auch hier, wohnt aber unten in der \textcolor{pink}{Stadt}{}\ledrightnote{→\textcolor{pink}{Meran}}.\pend
           \pstart
           Dr. \textcolor{blue}{Schreiber}{}\ledrightnote{\textcolor{blue}{Joseph Schreiber}}{ }ſa{\geminationm}t \textcolor{blue}{Gemahlin}{}\ledrightnote{→\textcolor{blue}{Clara Schreiber}} haben mich
                    äuſserst freundlich und liebenswürdig empfangen; letztere läſst bestens danken.
                    Sehr unangenehm aber waren die Eröffnungen, die mir ihr Herr \textcolor{blue}{Gemahl}{}\ledrightnote{→\textcolor{blue}{Joseph Schreiber}} machte. Nachdem er konstatiert
                    hatte, daſs ich im höchsten Grad anämisch sei, erklärte er mir rund heraus, von
                    einer Heilung bi{\geminationn}en 4 Wochen – ich getraute mich
                    gar nicht mehr, von 16 Tagen zu sprechen – kö{\geminationn}e
                    überhaupt nicht die Rede sein; \uline{vor
                            15. Mai}{ }{\pb}\uline{d. h. vor 3 Monaten} kö{\geminationn}e er mich nicht entlaſsen. Dabei sagte er nicht
                    etwa: We{\geminationn} Sie früher fortgehen, werden Sie später
                    die Folgen zu spüren haben – o nein! sondern ganz einfach: »Sie werden vor
                    3 Monaten nicht arbeitsfähig sein!« Das ist doch ein Argument, das zieht.\pend
           \pstart
           Sehen Sie, lieber Dr., ich hatte Recht, als ich meinte, es sei fertig mit mir.
                    Die Aussichten auf die \textcolor{brown}{deutsche Zeitung}{}\ledrightnote{\textcolor{brown}{Deutsche Zeitung}}{ }ſind doch entschieden vorbei, und auch die \textcolor{brown}{Kunstchronik}{}\ledrightnote{\textcolor{brown}{Allgemeine Kunst-Chronik}} wird bei einer so langen
                    Abwesenheit verloren sein. Also stehe ich, we{\geminationn} ich
                    nach \textcolor{pink}{Wien}{}\ledrightnote{\textcolor{pink}{Wien}} ko{\geminationm}e,
                    wieder ohne jede Einnahme da, der Mildthätigkeit überlaſsen. – Auf der andern
                    Seite sehe ich absolut nicht ein, wie so lange den Aufenthalt in \textcolor{pink}{Meran}{}\ledrightnote{\textcolor{pink}{Meran}} bestreiten. Die Pension im Hotel ohne
                    Wein, Licht und Heizung beträgt 3 fl (ich habe, als Journalist, von den üblichen
                    4 fl einen abgehandelt. Alle Leute, auch Dr. \textcolor{blue}{Schreiber}{}\ledrightnote{\textcolor{blue}{Joseph Schreiber}}, haben mir zum Hotel geraten, weil ich hier Gesellschaft und
                    mehr Anregung finde als im Privatquartier; auch sei’s nicht teuerer); da ich
                    absolut nicht gehen ka{\geminationn} und \uline{darf}, muſs ich mir jeden Tag einen Rollwagen nehmen, der
                    fl 1.–1.20 kostet; nehmen Sie dazu Wein, Licht, Heizung, Cigarren etc – so kö{\geminationn}en Sie sich ungefähr einen Begriff von den
                    Ausgaben machen. Dagegen werde ich noch einnehmen: \pend
           \pstart
           {\pb}1) die Su{\geminationm}e,
                    die Sie so gütig waren, mir zu versprechen\pend
           \pstart
           2) das Ergebnis zweier Sa{\geminationm}lungen, die \textcolor{blue}{Steinbach}{}\ledrightnote{\textcolor{blue}{Josef Steinbach}} bei der \textcolor{brown}{Neuen Freien Preſse}{}\ledrightnote{\textcolor{brown}{Neue Freie Presse}} und \textcolor{blue}{Gelber}{}\ledrightnote{\textcolor{blue}{Ludwig Gelber}} beim \textcolor{brown}{Neuen Tagblatt}{}\ledrightnote{\textcolor{brown}{Neues Wiener Tagblatt}}
                    veranstalten werden (we{\geminationn}{ }ſie es thun!)\pend
           \pstart
           3) eine Unterstützung von je 50 fl, die ich vielleicht! von der \textcolor{brown}{Concordia}{}\ledrightnote{\textcolor{brown}{Concordia}} und von der \textcolor{brown}{Schillerstiftung}{}\ledrightnote{\textcolor{brown}{Deutsche Schillerstiftung}} erhalte. – Das ist zwar viel, aber es reicht doch
                    nicht. – –\pend
           \pstart
           Jetzt leben Sie wol – meine Hand ist müde, und Sie wiſsen alles Wichtige – und
                    seien Sie nebst \textcolor{blue}{Beer-Hofma{\geminationn}}{}\ledrightnote{\textcolor{blue}{Richard Beer-Hofmann}}, \textcolor{blue}{Loris}{}\ledrightnote{\textcolor{blue}{Hugo von Hofmannsthal}} und den andern herzlich
                    gegrüſst von\pend
           \pstart
           Ihrem{\\[\baselineskip]}\spacefill\mbox{Fels}\pend
           \leftskip=0em{}\pstart
           \noindent{}Für wie schwach mich \textcolor{blue}{Schreiber}{}\ledrightnote{\textcolor{blue}{Joseph Schreiber}} erklärt,
                            kö{\geminationn}en Sie aus meiner Kurvorschrift
                        ersehen:\pend
           \pstart
           1) ¼ Ltr Milch mit 1 Kaffeelöffel Cognac 4mal tägl.\pend
           \pstart
           2) Waschung 27°, Halbbad 26° mit \uline{sanften}
                        Frottierungen und Übergieſsungen. »Man ka{\geminationn} ja
                        mit Ihnen nichts anfangen.«\pend
           \endnumbering\briefempfaengerindex{Schnitzler, Arthur@\textsc{Schnitzler, Arthur}!zzzFels, Friedrich Michael@\emph{von Friedrich Michael Fels}!1893-02-161@{1{[}6{]}. 2. 1893}|)be}\mylabel{h}  \normalsize

\doendnotes{C}
\bigskip
\vfill

\clearpage

\footnotesize

\lohead{\textsc{register}}

% Definiere theindex-Environment komplett neu ohne reledmac
\makeatletter
\renewenvironment{theindex}{%
  \section*{\indexname}%
  \setlength{\parindent}{0pt}%
  \setlength{\parskip}{0pt plus 0.3pt}%
  \let\item\@idxitem
}{%
  \clearpage
}
\makeatother

\IfFileExists{\jobname-pw.ind}{\input{\jobname-pw.ind}}{}

\end{document}

      