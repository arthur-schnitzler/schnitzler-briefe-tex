%% latex-korrekturansicht-vorspann.tex
%% Vorspann für die Korrekturansicht.
%% Lädt die gemeinsame Datei latex-vorspann.tex mit gesetztem Schalter.

\newif\ifkorrekturansicht
\korrekturansichttrue

\input{../tex-inputs/latex-vorspann}


               \section[Olga Schnitzler an Richard Beer-Hofmann, {[}18. 10. 1907{]}]{ Olga Schnitzler an Richard Beer-Hofmann, {[}18. 10. 1907{]}}\nopagebreak\mylabel{v}\rehead{ }\normalsize\beginnumbering\briefempfaengerindex{Beer-Hofmann, Richard@\textsc{Beer-Hofmann, Richard}!zzzSchnitzler, Olga@\emph{von Olga Schnitzler}!1907-10-181@{{[}18. 10. 1907{]}}|(be} \toendnotes[C]{\smallbreak\pagebreak[2]} \Standort{YCGL, MSS 31.}
\physDesc{Briefkarte, Umschlag
\newline{}Handschrift: schwarze Tinte, lateinische Kurrent\newline{}Versand: ohne postalischen Übermittlungsvermerk }\toendnotes[C]{\smallbreak}\pstart{}{\pb}\textcolor{gray}{\textbf{O. S.}}\pend{}{\bigskip}\pstart{}{\pb}Herrn D\textsuperscript{r} Richard
                  Beer-Hofmann \pend{}{\bigskip}\pstart
           \noindent{}{\pb}\textcolor{gray}{\textbf{O. S.}}\pend
           \pstart
           Lieber Herr Doctor, ich habe gestern im \label{K_L01723_1v}\edtext{Antiquitäten-Geschäft}{\lemma{\textnormal{\emph{Antiquitäten-Geschäft}}}\Cendnote{\textnormal{Es dürfte sich um ein temporäres Geschäft aus dem Nachlass des
                     1904 verstorbenen Sammlers und Schätzmeisters \textcolor{blue}{Heinrich Cubasch} gehandelt haben.}}}\label{K_L01723_1h} im Gebäude des \textcolor{brown}{Central-Bades}{}\ledrightnote{\textcolor{brown}{Zentralbad}}, \textcolor{pink}{Weihburggasse}{}\ledrightnote{\textcolor{pink}{Weihburggasse}}, eine herrliche Spitze gesehen; sie hängt in der Auslage, hat
               ungefähr diese Form: {[}Umriss einer Zigarrenspitze{]}\pend
           \pstart
           {\pb}Es ist noch ein zweites, ebensolches Stück da, die
               beiden kosten 60 fl. Vielleicht interessieren Sie sich dafür. – Auf Wiedersehen
                  \label{K_L01723_2v}\edtext{morgen}{\lemma{\textnormal{\emph{morgen}}}\Cendnote{\textnormal{Das ermöglicht die Datierung. Vgl. A. S.: \emph{Tagebuch}, 19. 10. 1907}}}\label{K_L01723_2h} in der General-Probe der »\textcolor{brown}{Fledermaus}{}\ledrightnote{\textcolor{brown}{Cabaret Fledermaus}}«.\pend
           \pstart
           Vo\substVorne{}\textsuperscript{m}\substDazwischen{}n\substHinten{} uns zu Ihnen \textcolor{blue}{Beiden}{}\ledrightnote{→\textcolor{blue}{Paula Beer-Hofmann}}
               die herzlichsten Grüsse!{\\[\baselineskip]}\spacefill\mbox{OlgaS.}\pend
           \leftskip=0em{}\pstart
           Freitag.\pend
           \endnumbering\briefempfaengerindex{Beer-Hofmann, Richard@\textsc{Beer-Hofmann, Richard}!zzzSchnitzler, Olga@\emph{von Olga Schnitzler}!1907-10-181@{{[}18. 10. 1907{]}}|)be}\mylabel{h}  \normalsize

\doendnotes{C}
\bigskip
\vfill

\clearpage

\footnotesize

\lohead{\textsc{register}}

% Definiere theindex-Environment komplett neu ohne reledmac
\makeatletter
\renewenvironment{theindex}{%
  \section*{\indexname}%
  \setlength{\parindent}{0pt}%
  \setlength{\parskip}{0pt plus 0.3pt}%
  \let\item\@idxitem
}{%
  \clearpage
}
\makeatother

\IfFileExists{\jobname-pw.ind}{\input{\jobname-pw.ind}}{}

\end{document}

      