%% latex-korrekturansicht-vorspann.tex
%% Vorspann für die Korrekturansicht.
%% Lädt die gemeinsame Datei latex-vorspann.tex mit gesetztem Schalter.

\newif\ifkorrekturansicht
\korrekturansichttrue

\input{../tex-inputs/latex-vorspann}


               \section[Hugo von Hofmannsthal an Arthur Schnitzler, {[}6. 8. 1915{]}]{ Hugo von Hofmannsthal an Arthur Schnitzler, {[}6. 8. 1915{]}}\nopagebreak\mylabel{v}\rehead{ }\normalsize\beginnumbering\briefempfaengerindex{Schnitzler, Arthur@\textsc{Schnitzler, Arthur}!zzzHofmannsthal, Hugo von@\emph{von Hugo von Hofmannsthal}!1915-08-061@{{[}6. 8. 1915{]}}|(be} \toendnotes[C]{\smallbreak\pagebreak[2]} \Standort{CUL, Schnitzler, B 43.}
\physDesc{Brief, 1 Blatt, 2 Seiten, Umschlag
\newline{}Handschrift: Bleistift, lateinische Kurrent
\newline{}Schnitzler: mit Bleistift beschriftet: »\textsc{Hofmannsthal}« und datiert: »6/8 19{[}1{]}5« \newline{}Ordnung: 1) mit Bleistift von unbekannter Hand nummeriert: »\strikeout{342}« 2) mit Bleistift von unbekannter Hand
                                            nummeriert: »74a?«}\buchAbdrucke{\weitereDrucke{Hugo von Hofmannsthal, Arthur Schnitzler: \emph{Briefwechsel}. Hg. Therese Nickl und Heinrich Schnitzler. Frankfurt am Main: \emph{S. Fischer} 1964, S. 278.} }\toendnotes[C]{\smallbreak}\pstart{}{\pb}Herrn D\textsuperscript{r} Arthur Schnitzler\pend{}{\bigskip}\pstart
           \noindent{}{\pb}\textcolor{gray}{\textbf{\textcolor{pink}{HOTEL KAISERKRONE}{}\ledrightnote{\textcolor{pink}{Hotel Kaiserkrone}}}}\hfill \textcolor{gray}{\textbf{\textcolor{pink}{Bad Ischl}{}\ledrightnote{\textcolor{pink}{Bad Ischl}}, am ..........}}\pend
           \pstart
           \textcolor{gray}{\textbf{\textcolor{pink}{BAD ISCHL}{}\ledrightnote{\textcolor{pink}{Bad Ischl}}}}\pend
           \pstart
           \textcolor{gray}{\textbf{ZENTRALE LAGE}}\pend
           \pstart
           \textcolor{gray}{\textbf{RESTAURATIONSGARTEN, BÄDER}}\pend
           \pstart
           \textcolor{gray}{\textbf{LIFT, MODERNER KOMFORT}}\pend
           \pstart{}Viele Grüsse, lieber Arthur,\pend\pstart
           Ihnen und \textcolor{blue}{Olga}{}\ledrightnote{\textcolor{blue}{Olga Schnitzler}}. Ich ko{\geminationm}e alle 3–4 Tage für 24 Stunden herüber.
                    Vielleicht besuchen Sie \textcolor{blue}{Papa}{}\ledrightnote{→\textcolor{blue}{Hugo August von Hofmannsthal}} einmal, es ist ganz nahe, \textcolor{pink}{Tänzlgasse 10}{}\ledrightnote{\textcolor{pink}{Tänzelgasse}}, vormittag sitzt er i{\geminationm}er in seinem Garten.\pend
           \pstart
           Von Herzen Ihr{\\[\baselineskip]}\spacefill\mbox{Hugo.}\pend
           \leftskip=0em{}\endnumbering\briefempfaengerindex{Schnitzler, Arthur@\textsc{Schnitzler, Arthur}!zzzHofmannsthal, Hugo von@\emph{von Hugo von Hofmannsthal}!1915-08-061@{{[}6. 8. 1915{]}}|)be}\mylabel{h}  \normalsize

\doendnotes{C}
\bigskip
\vfill

\clearpage

\footnotesize

\lohead{\textsc{register}}

% Definiere theindex-Environment komplett neu ohne reledmac
\makeatletter
\renewenvironment{theindex}{%
  \section*{\indexname}%
  \setlength{\parindent}{0pt}%
  \setlength{\parskip}{0pt plus 0.3pt}%
  \let\item\@idxitem
}{%
  \clearpage
}
\makeatother

\IfFileExists{\jobname-pw.ind}{\input{\jobname-pw.ind}}{}

\end{document}

      