%% latex-korrekturansicht-vorspann.tex
%% Vorspann für die Korrekturansicht.
%% Lädt die gemeinsame Datei latex-vorspann.tex mit gesetztem Schalter.

\newif\ifkorrekturansicht
\korrekturansichttrue

\input{../tex-inputs/latex-vorspann}


               \section[Paul Goldmann an Arthur Schnitzler, 9. 8. {[}1894{]}]{ Paul Goldmann an Arthur Schnitzler, 9. 8. {[}1894{]}}\nopagebreak\mylabel{v}\rehead{ }\normalsize\beginnumbering\briefempfaengerindex{Schnitzler, Arthur@\textsc{Schnitzler, Arthur}!zzzGoldmann, Paul@\emph{von Paul Goldmann}!1894-08-091@{9. 8. {[}1894{]}}|(be} \toendnotes[C]{\smallbreak\pagebreak[2]} \Standort{DLA, A:Schnitzler, HS.NZ85.1.3164.}
\physDesc{Brief, 1 Blatt, 4 Seiten
\newline{}Handschrift: schwarze Tinte, deutsche Kurrent
\newline{}Schnitzler: mit Bleistift auf dem ersten Blatt die Jahreszahl
                                       »94« vermerkt }\toendnotes[C]{\smallbreak}\pstart
           \noindent{}{\pb}\textcolor{gray}{\textbf{\textcolor{brown}{Frankfurter Zeitung}{}\ledrightnote{\textcolor{brown}{Frankfurter Zeitung}}.}}\hfill \textsc{\textcolor{pink}{Paris}{}\ledrightnote{\textcolor{pink}{Paris}}}, 9. Auguſt.\pend
           \pstart
           \textcolor{gray}{\textbf{(\textcolor{brown}{Gazette de
                  Francfort}{}\ledrightnote{\textcolor{brown}{Frankfurter Zeitung}}.)}}\pend
           \pstart
           \textcolor{gray}{\textbf{\begin{otherlanguage}{french}Fondateur\end{otherlanguage}{ }\textbf{M. \textcolor{blue}{L.
                  Sonnemann}{}\ledrightnote{\textcolor{blue}{Leopold Sonnemann}}}.}}\pend
           \pstart
           \textcolor{gray}{\textbf{\begin{otherlanguage}{french}Journal politique,
                        financier,\end{otherlanguage}}}\pend
           \pstart
           \textcolor{gray}{\textbf{\begin{otherlanguage}{french}commercial et
                     littéraire.\end{otherlanguage}}}\pend
           \pstart
           \textcolor{gray}{\textbf{\begin{otherlanguage}{french}\textbf{Paraissant trois fois
                           par jour}\end{otherlanguage}}}.\pend
           \pstart
           \textcolor{gray}{\textbf{–}}\pend
           \pstart
           \textcolor{gray}{\textbf{\begin{otherlanguage}{french}\textbf{Bureaux à \textcolor{pink}{Paris}{}\ledrightnote{\textcolor{pink}{Paris}}:}\end{otherlanguage}}}\pend
           \pstart
           \textcolor{gray}{\textbf{\begin{otherlanguage}{french}\textcolor{pink}{\textbf{24. Rue Feydeau}}{}\ledrightnote{\textcolor{pink}{rue Feydeau}}.\end{otherlanguage}}}\pend
           \pstart\center{}Mein lieber Freund,\pend\pstart
           Alles kracht plötzlich zuſammen. Geſtern erhielt ich \textsc{\begin{otherlanguage}{french}Ordre\end{otherlanguage}} von meinem \textcolor{brown}{Journal}{}\ledrightnote{→\textcolor{brown}{Frankfurter Zeitung}}, ſofort meinen Urlaub anzutreten und
               nach \textsc{\textcolor{pink}{Orange}{}\ledrightnote{\textcolor{pink}{Orange}}} zu fahren, um
               über die Aufführungen im \textcolor{brown}{antiken Theater}{}\ledrightnote{\textcolor{brown}{Theater Orange}} zu
               berichten. Es iſt ekelhaft und gemein, aber da gibt es keine Weigerung. Demgemäß
               ändern ſich ſämmtliche Dispoſitionen. Mein Urlaub geht auf dieſe Weiſe bereits am
                  12. September zu Ende. {\pb}Und da ich
               die letzten acht Tage in \uline{\textcolor{pink}{Frankfurt}{}\ledrightnote{\textcolor{pink}{Frankfurt am Main}} verbringen muß}, ſo könnte ich nur zwiſchen dem
                  20. Auguſt und 3. September mit \textcolor{blue}{Euch}{}\ledrightnote{→\textcolor{blue}{Hugo von Hofmannsthal}{\newline}→\textcolor{blue}{Richard Beer-Hofmann}}\label{K_L02610-2v}\edtext{zuſammen}{\lemma{\textnormal{\emph{zuſammen}}}\Cendnote{\textnormal{er schreibt »zuſammen zu«}}}\label{K_L02610-2h} ſein. Ich
               würde Alles thun, um dieſes Zuſammenſein zu ermöglichen, keine Reiſe ſcheuen \textsc{etc}. Ich habe ein ſolches Bedürfniß danach, mir Eure lieben
               Geſichter aufzufriſchen, mit \textcolor{blue}{Euch}{}\ledrightnote{→\textcolor{blue}{Hugo von Hofmannsthal}{\newline}→\textcolor{blue}{Richard Beer-Hofmann}} zu plaudern und mich bei \textcolor{blue}{Euch}{}\ledrightnote{→\textcolor{blue}{Hugo von Hofmannsthal}{\newline}→\textcolor{blue}{Richard Beer-Hofmann}} moraliſch und
               geiſtig zu kräftigen. Ich wäre tief traurig, wenn dieſes Zuſammenſein unmöglich wäre.
               Kann {\pb} ich nicht \textcolor{blue}{Alle}{}\ledrightnote{→\textcolor{blue}{Hugo von Hofmannsthal}{\newline}→\textcolor{blue}{Richard Beer-Hofmann}} ſehen, ſo möchte ich wenigſtens mit
               Einem Zuſammenſein, am Liebſten natürlich mit Dir.\pend
           \pstart
           Kurzum: Könntet Ihr die Reiſe in \textcolor{pink}{Tirol}{}\ledrightnote{\textcolor{pink}{Tirol}{\newline}\textcolor{pink}{Südtirol}} um acht
               Tage \label{K_L02610-1v}\edtext{früher beginnen}{\lemma{\textnormal{\emph{früher beginnen}}}\Cendnote{\textnormal{Am 23. 8. 1894 kam Goldmann in \textcolor{pink}{Bad Ischl} an, er reiste also nicht nach \textcolor{pink}{Tirol}.}}}\label{K_L02610-1h}, ſo käme ich direct aus \textcolor{pink}{Südfrankreich}{}\ledrightnote{\textcolor{pink}{Frankreich}} nach \textcolor{pink}{Tirol}{}\ledrightnote{\textcolor{pink}{Tirol}{\newline}\textcolor{pink}{Südtirol}}.
               Am Liebſten wäre es mir freilich, wenn wir uns in \textcolor{pink}{Italien}{}\ledrightnote{\textcolor{pink}{Italien}} treffen könnten. \textsc{\textcolor{pink}{Pisa}{}\ledrightnote{\textcolor{pink}{Pisa}}}{ }\textsc{\textcolor{pink}{Genua}{}\ledrightnote{\textcolor{pink}{Genua}}}, \textsc{\textcolor{pink}{Florenz}{}\ledrightnote{\textcolor{pink}{Florenz}}}, \textsc{\textcolor{pink}{Venedig}{}\ledrightnote{\textcolor{pink}{Venedig}}}. Wie herrlich wäre
               es z. B., wenn wir acht Tage in \textcolor{pink}{Venedig}{}\ledrightnote{\textcolor{pink}{Venedig}}{ }\strikeout{be} bummeln könnten! Sollteſt Du das nicht zu {\pb}machen vermögen? Aber ich mache dir keine weitern
               Vorſchläge und überlaſſe Alles deiner Güte und Freundſchaft.\pend
           \pstart
           Schreibe mir ſofort nach dem Empfang dieſes Briefes an meine \textcolor{pink}{Pariſ}{}\ledrightnote{\textcolor{pink}{Paris}}er Adresse, oder telegraphiere mir dorthin (\textsc{Goldmann}, \textsc{\textcolor{pink}{Paris, 24. Feydeau}{}\ledrightnote{\textcolor{pink}{rue Feydeau}}}). Ich habe \begin{otherlanguage}{french}Ordre\end{otherlanguage} gegeben, daß mir Briefe nachgeſchickt und Telegramme
               nachtelegraphirt werden. Gib mir auch an, wohin ich dir brieflich oder telegraphiſch
               antworten kann?\pend
           \pstart
           Von Herzen{\\[\baselineskip]} Dein{\\[\baselineskip]}\spacefill\mbox{Paul Goldmann}\pend
           \leftskip=0em{}\pstart
           \noindent{}{\pb}\label{T_L02610-1v}\edtext{Tauſend Dank für den lieben Brief aus \textsc{\textcolor{pink}{Salzburg}{}\ledrightnote{\textcolor{pink}{Salzburg}}}}{\lemma{\textnormal{\emph{Tauſend … Salzburg}}}\Cendnote{\textnormal{auf der ersten Seite, unterhalb des Textes}}}\label{T_L02610-1h}\pend
           \endnumbering\briefempfaengerindex{Schnitzler, Arthur@\textsc{Schnitzler, Arthur}!zzzGoldmann, Paul@\emph{von Paul Goldmann}!1894-08-091@{9. 8. {[}1894{]}}|)be}\mylabel{h}  \normalsize

\doendnotes{C}
\bigskip
\vfill

\clearpage

\footnotesize

\lohead{\textsc{register}}

% Definiere theindex-Environment komplett neu ohne reledmac
\makeatletter
\renewenvironment{theindex}{%
  \section*{\indexname}%
  \setlength{\parindent}{0pt}%
  \setlength{\parskip}{0pt plus 0.3pt}%
  \let\item\@idxitem
}{%
  \clearpage
}
\makeatother

\IfFileExists{\jobname-pw.ind}{\input{\jobname-pw.ind}}{}

\end{document}

      