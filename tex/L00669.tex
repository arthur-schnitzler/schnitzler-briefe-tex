%% latex-korrekturansicht-vorspann.tex
%% Vorspann für die Korrekturansicht.
%% Lädt die gemeinsame Datei latex-vorspann.tex mit gesetztem Schalter.

\newif\ifkorrekturansicht
\korrekturansichttrue

\input{../tex-inputs/latex-vorspann}


               \section[Hugo von Hofmannsthal an Arthur Schnitzler, 24. 4. {[}1897{]}]{ Hugo von Hofmannsthal an Arthur Schnitzler, 24. 4. {[}1897{]}}\nopagebreak\mylabel{v}\rehead{ }\normalsize\beginnumbering\briefempfaengerindex{Schnitzler, Arthur@\textsc{Schnitzler, Arthur}!zzzHofmannsthal, Hugo von@\emph{von Hugo von Hofmannsthal}!1897-04-241@{24. 4. {[}1897{]}}|(be} \toendnotes[C]{\smallbreak\pagebreak[2]} \Standort{CUL, Schnitzler, B 43.}
\physDesc{Brief, 2 Blätter, 5 Seiten
\newline{}Handschrift: 1) schwarze Tinte, deutsche Kurrent\hspace{1em}2) Bleistift, deutsche Kurrent (\noindent{}ab »Eben kommt«)\hspace{1em}
\newline{}Schnitzler: mit Bleistift die Jahreszahl ergänzt: »97« \newline{}Ordnung: 1) mit Bleistift von unbekannter Hand nummeriert: »88« und paginiert 1–2 2) mit Bleistift von unbekannter Hand nummeriert: »87«}\buchAbdrucke{\weitereDrucke{Hugo von Hofmannsthal, Arthur Schnitzler: \emph{Briefwechsel}. Hg. Therese Nickl und Heinrich Schnitzler. Frankfurt am Main: \emph{S. Fischer} 1964, S. 80.} }\toendnotes[C]{\smallbreak}\pstart
           \raggedleft{}{\pb}\textcolor{pink}{Wien}{}\ledrightnote{\textcolor{pink}{Wien}}{ }24\textsuperscript{ten} April\pend
           \pstart{}mein lieber Arthur\pend\pstart
           zuerſt kommt eine dumme Geſchichte, dann anderes. Die »\textcolor{green}{Mimi}{}\ledrightnote{\textcolor{green}{Mimi}}« von der \textcolor{blue}{Clara
                        Loeb}{}\ledrightnote{\textcolor{blue}{Clara Katharina Pollaczek}}{ }ſteht ſeit 10 Tagen in der »\textcolor{green}{Freien Bühne}{}\ledrightnote{\textcolor{green}{Freie Bühne für den Entwickelungskampf der Zeit}}«, natürlich iſt es herausgekommen von wem es
                    iſt.\pend
           \pstart
           Zum Theil hat die \textcolor{blue}{Minnie B.}{}\ledrightnote{\textcolor{blue}{Marianne Benedict}} einen recht
                    überflüſſigen Tratſch angefangen (komiſch muſs ſich das alles in \textcolor{pink}{Paris}{}\ledrightnote{\textcolor{pink}{Paris}}{ }{\pb}anhören) andersſeits hat
                    jemand recht gemeiner den Eltern \textcolor{blue}{Loeb}{}\ledrightnote{\textcolor{blue}{Louis Loeb}{\newline}\textcolor{blue}{Regina Loeb}}
                    einen anonymen Brief geſchrieben, kurz heute Früh läſst mich die \textcolor{blue}{Mutter}{}\ledrightnote{→\textcolor{blue}{Regina Loeb}} bitten hinzukommen.
                    Die \textcolor{blue}{Clara}{}\ledrightnote{\textcolor{blue}{Clara Katharina Pollaczek}} war nicht zu ſehen, die \textcolor{blue}{Anna}{}\ledrightnote{\textcolor{blue}{Anna Epstein}} und die \textcolor{blue}{Mutter}{}\ledrightnote{→\textcolor{blue}{Regina Loeb}} verweint wie bei einem Leichenbegängnis, der
                        \textcolor{blue}{Vater}{}\ledrightnote{→\textcolor{blue}{Louis Loeb}} ganz blaſs und
                    mit zitternder Stimme. Das weitere iſt unintereſſant; ich glaube daſs ich ſie
                    doch ein biſſel herumgekriegt {\pb}habe; \uuline{Ihre} active Theilnahme hab ich
                    verſchwiegen, weil die \textcolor{blue}{Mutter}{}\ledrightnote{→\textcolor{blue}{Regina Loeb}} ohnehin eine ſchlechte moraliſche Meinung von Ihnen hat,
                    während ich doch ſo brav und anſtändig bin. (Hoch!)\pend
           \pstart
           Zum Schluſs waren ſie faſt gerührt über mich und vielleicht laſſen \strikeout{S}ſie mich noch die Männer für die Mädeln ausſuchen.
                    Von Ihnen aber will ich nur zweierlei: 1.) wenn irgend jemand bei Ihnen anfragt
                    (bei der rätſelhaften Stellung, die {\pb}die \textcolor{blue}{Minnie}{}\ledrightnote{\textcolor{blue}{Marianne Benedict}} zu der Geſchichte hat, iſt alles möglich) ſo
                    wiſſen Sie einfach nicht, wer die \textcolor{blue}{Verfaſſerin}{}\ledrightnote{→\textcolor{blue}{Clara Katharina Pollaczek}} iſt.\pend
           \pstart
           2.) Sie müſſen ſo gut ſein, ſofort an \textcolor{blue}{Fiſcher}{}\ledrightnote{\textcolor{blue}{Samuel Fischer}}{ }ſchreiben, daſs der Druck des Buches
                    unterbleibt und er das Manuſcript umgehend an mich zurück ſchicken ſoll. Sie
                    müſſen das von Ihrem \textcolor{blue}{Verleger}{}\ledrightnote{→\textcolor{blue}{Samuel Fischer}} als perſönliche Gefälligkeit verlangen. Ich habe es den \textcolor{blue}{Eltern}{}\ledrightnote{→\textcolor{blue}{Louis Loeb}{\newline}→\textcolor{blue}{Regina Loeb}} beſtimmt
                    verſprochen, mir zu liebe tut er es aber vielleicht nicht, weil {\pb}es ihm etwa unbequem iſt.
                    Alſo bitte, \uline{ſofort}.\pend
           \pstart
           \centering{}Das Andere.\pend
           \pstart
           \noindent{}was eſſen Sie in \textcolor{pink}{Paris}{}\ledrightnote{\textcolor{pink}{Paris}}{ }ſtatt des gemiſchten Hausbrotes?\pend
           \pstart
           \noindent{}\label{T_L00669_1v}\edtext{Eben kommt \textcolor{blue}{Hirschfeld}{}\ledrightnote{\textcolor{blue}{Georg Hirschfeld}}.}{\lemma{\textnormal{\emph{Eben kommt Hirschfeld.}}}\Cendnote{\textnormal{ab
                        hier Bleistift.}}}\label{T_L00669_1h}\pend
           \pstart
           Muſs für heute ſchließen.\pend
           \pstart
           Grüße \textcolor{blue}{Goldmann}{}\ledrightnote{\textcolor{blue}{Paul Goldmann}}.\pend
           \pstart
           Ihr{\\[\baselineskip]}\spacefill\mbox{Hugo}\pend
           \leftskip=0em{}\endnumbering\briefempfaengerindex{Schnitzler, Arthur@\textsc{Schnitzler, Arthur}!zzzHofmannsthal, Hugo von@\emph{von Hugo von Hofmannsthal}!1897-04-241@{24. 4. {[}1897{]}}|)be}\mylabel{h}  \normalsize

\doendnotes{C}
\bigskip
\vfill

\clearpage

\footnotesize

\lohead{\textsc{register}}

% Definiere theindex-Environment komplett neu ohne reledmac
\makeatletter
\renewenvironment{theindex}{%
  \section*{\indexname}%
  \setlength{\parindent}{0pt}%
  \setlength{\parskip}{0pt plus 0.3pt}%
  \let\item\@idxitem
}{%
  \clearpage
}
\makeatother

\IfFileExists{\jobname-pw.ind}{\input{\jobname-pw.ind}}{}

\end{document}

      