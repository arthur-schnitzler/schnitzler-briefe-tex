%% latex-korrekturansicht-vorspann.tex
%% Vorspann für die Korrekturansicht.
%% Lädt die gemeinsame Datei latex-vorspann.tex mit gesetztem Schalter.

\newif\ifkorrekturansicht
\korrekturansichttrue

\input{../tex-inputs/latex-vorspann}


               \section[Hugo von Hofmannsthal an Arthur Schnitzler, 16. 6. 1894]{ Hugo von Hofmannsthal an Arthur Schnitzler, 16. 6. 1894}\nopagebreak\mylabel{v}\rehead{ }\normalsize\beginnumbering\briefempfaengerindex{Schnitzler, Arthur@\textsc{Schnitzler, Arthur}!zzzHofmannsthal, Hugo von@\emph{von Hugo von Hofmannsthal}!1894-06-161@{16. 6. 1894}|(be} \toendnotes[C]{\smallbreak\pagebreak[2]} \Standort{CUL, Schnitzler, B 43b/1.}
\physDesc{Kartenbrief
\newline{}Handschrift: Bleistift, deutsche Kurrent\newline{}Versand: 1) Stempel: »\nobreak{}\oindex{III., Landstrasse@\textbf{III., Landstraße}, \emph{Bezirk (A.BZK)}|pwk}Wien 3/3, 16. 6. 94, 5–6 N\nobreak{}«.  2) Stempel: »\nobreak{}Bestellt, \oindex{IX., Alsergrund@\textbf{IX., Alsergrund}, \emph{Bezirk (A.BZK)}|pwk}Wien 9/3, 17. 6. 94, 8. V\nobreak{}«. 
\newline{}Schnitzler: mit Bleistift das Datum ergänzt: »16/6 94« \newline{}Ordnung: mit Bleistift von unbekannter Hand nummeriert:
                                    »66« }\buchAbdrucke{\weitereDrucke{1) Hugo von Hofmannsthal, Arthur Schnitzler: \emph{Briefwechsel}. Hg. Therese Nickl und Heinrich Schnitzler. Frankfurt am Main: \emph{S. Fischer} 1964, S. 52.} \weitereDrucke{2) Hermann Bahr, Arthur Schnitzler: \emph{Briefwechsel, Aufzeichnungen, Dokumente (1891–1931)}. Hg. Kurt Ifkovits und Martin Anton Müller. Göttingen: \emph{Wallstein} 2018, S. 73.} }\toendnotes[C]{\smallbreak}\pstart{}\textsc{{\pb}Herrn D\textsuperscript{r} Arthur Schnitzler}\pend{}\pstart{}\textsc{\textcolor{pink}{IX}{}\ledrightnote{\textcolor{pink}{IX., Alsergrund}}}\pend{}\pstart{}\textsc{\textcolor{pink}{Frankgasse 1}{}\ledrightnote{\textcolor{pink}{Frankgasse}}}\pend{}{\bigskip}\pstart
           \noindent{}{\pb}lieber, ich werde dem
                  \textcolor{blue}{Bahr}{}\ledrightnote{\textcolor{blue}{Hermann Bahr}}{ }\label{K_L00339_1v}\edtext{das Mitgehen}{\lemma{\textnormal{\emph{das Mitgehen}}}\Cendnote{\textnormal{nach \textcolor{pink}{Mödling} zu \textcolor{blue}{Christine Schönberger}, der Wirtstochter des \textcolor{pink}{Goldenen Stern}. Diese dürfte in der \emph{\textcolor{green}{Liebelei}} porträtiert sein, vgl. \textcolor{blue}{Bahr} an \textcolor{blue}{Gerty
                        Schlesinger}, 30. 6. 1898 und Valerie Reichert-Heidt: \emph{Das Urbild der
                        Christine}. In: \emph{Neues Österreich}, Jg. 11,
                     Nr. 3208, 13. 11. 1955, S. 17–18.}}}\label{K_L00339_1h}
               ausreden.\pend
           \pstart
           Wenn es \uline{unzweifelhaft} hübſch iſt, weder drohend noch
               regneriſch, erwart ich Sie um Punkt \label{K_L00339_2v}\edtext{¼
                  4}{\lemma{\textnormal{\emph{¼
                  4}}}\Cendnote{\textnormal{15 Uhr 45}}}\label{K_L00339_2h} unter
               den Arkaden der \textcolor{pink}{Oper}{}\ledrightnote{\textcolor{pink}{Oper}}, wo die \textcolor{pink}{Guttmann’ſche \label{K_L00339_3v}\edtext{Kalienhandlung}{\lemma{\textnormal{\emph{Kalienhandlung}}}\Cendnote{\textnormal{gemeint:
                     Musikalienhandlung}}}\label{K_L00339_3h}}{}\ledrightnote{\textcolor{pink}{Musikalienhandlung Albert J. Gutmann}} iſt. Recht? Dadurch erſparen wir ½ Stunde.\pend
           \pstart
           Ihr{\\[\baselineskip]}\spacefill\mbox{Hugo.}\pend
           \leftskip=0em{}\endnumbering\briefempfaengerindex{Schnitzler, Arthur@\textsc{Schnitzler, Arthur}!zzzHofmannsthal, Hugo von@\emph{von Hugo von Hofmannsthal}!1894-06-161@{16. 6. 1894}|)be}\mylabel{h}  \normalsize

\doendnotes{C}
\bigskip
\vfill

\clearpage

\footnotesize

\lohead{\textsc{register}}

% Definiere theindex-Environment komplett neu ohne reledmac
\makeatletter
\renewenvironment{theindex}{%
  \section*{\indexname}%
  \setlength{\parindent}{0pt}%
  \setlength{\parskip}{0pt plus 0.3pt}%
  \let\item\@idxitem
}{%
  \clearpage
}
\makeatother

\IfFileExists{\jobname-pw.ind}{\input{\jobname-pw.ind}}{}

\end{document}

      