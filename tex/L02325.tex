%% latex-korrekturansicht-vorspann.tex
%% Vorspann für die Korrekturansicht.
%% Lädt die gemeinsame Datei latex-vorspann.tex mit gesetztem Schalter.

\newif\ifkorrekturansicht
\korrekturansichttrue

\input{../tex-inputs/latex-vorspann}


               \section[Robert Adam an Arthur Schnitzler, 19. 8. 1919]{ Robert Adam an Arthur Schnitzler, 19. 8. 1919}\nopagebreak\mylabel{v}\rehead{ }\normalsize\beginnumbering\briefempfaengerindex{Schnitzler, Arthur@\textsc{Schnitzler, Arthur}!zzzAdam, Robert@\emph{von Robert Adam}!1919-08-191@{19. 8. 1919}|(be} \toendnotes[C]{\smallbreak\pagebreak[2]} \Standort{CUL, Schnitzler, B 1.}
\physDesc{Brief, 1 Blatt, 3 Seiten
\newline{}Handschrift: blaue Tinte, deutsche Kurrent
\newline{}Schnitzler: 1) mit Bleistift beschriftet: »\textsc{Adam}« 2) mit rotem Buntstift zwei Unterstreichungen\newline{}Ordnung: mit Bleistift von unbekannter Hand nummeriert: »13« }\Standort{Wien, Österreichische Nationalbibliothek, Cod.ser. 52.268, 21 recto und 23.}
\physDesc{handschriftliche Abschrift
\newline{}Handschrift: schwarze Tinte, Gabelsberger Kurzschrift}\Standort{Wien, Österreichische Nationalbibliothek, Cod.ser. 52.268, 21 recto und 23.}
\physDesc{maschinelle Abschrift
\newline{}Schreibmaschine}\toendnotes[C]{\smallbreak}\pstart
           \raggedleft{}{\pb}\textcolor{pink}{Wien}{}\ledrightnote{\textcolor{pink}{Wien}}, am 19. August 1919\pend
           \pstart\center{}Hochverehrter Herr Doktor!\pend\pstart
           Von \textcolor{pink}{Wegſcheid bei Maria Zell}{}\ledrightnote{\textcolor{pink}{Wegscheid}} zurückgekehrt, wo
                    ich nach vollbrachter \textcolor{pink}{Karlsbad}{}\ledrightnote{\textcolor{pink}{Karlsbad}}er Kur \textcolor{blue}{Frau}{}\ledrightnote{→\textcolor{blue}{Maria Pollak}} und \textcolor{blue}{Kind}{}\ledrightnote{→\textcolor{blue}{Viktor Franz Patzner}} aufſuchte, um ſie glücklich
                    heimzubringen, finde ich Ihre Karte vor, die mir nach \textcolor{pink}{Karlsbad}{}\ledrightnote{\textcolor{pink}{Karlsbad}} nachgeſchickt und von dort zurückgeſendet worden
                    war. Ich freue mich darauf, Ihnen über meine Schickſale bei Ihrer Rückkehr
                    mündlich berichten zu können; erfreulich sind ſie ſchließlich nicht. Wenn Ärger,
                    wie die Ärzte behaupten, auf die Folgeerſcheinungen von Magengeſchwüren
                    ungünſtig einwirkt, so trägt das \textcolor{pink}{Deutſche
                        Volkstheater}{}\ledrightnote{\textcolor{pink}{Volkstheater}} zum guten Teile Schuld daran, daß ich mich durch vier
                    Wochen in \textcolor{pink}{Karlsbad}{}\ledrightnote{\textcolor{pink}{Karlsbad}} mit Felſenquelle und
                    Moorumſchlägen abgeben mußte. Der »\textcolor{green}{Fremde}{}\ledrightnote{\textcolor{green}{Der Fremde}}« hat
                    alle intereſſiert: den D\textsuperscript{r}{ }\textcolor{blue}{\textsc{Glücksmann}}{}\ledrightnote{\textcolor{blue}{Heinrich Glücksmann}}, {\pb}den D\textsuperscript{r}{ }\textcolor{blue}{\textsc{Waniek}}{}\ledrightnote{\textcolor{blue}{Wolfgang Waniek}}, den D\textsuperscript{r}{ }\textcolor{blue}{\textsc{Rosenthal}}{}\ledrightnote{\textcolor{blue}{Friedrich Rosenthal}} und den \textcolor{blue}{Direktor}{}\ledrightnote{→\textcolor{blue}{Alfred Bernau}}, und
                    ich war ſchon faſt meiner Sache ſicher: bis der \textcolor{blue}{Direktor}{}\ledrightnote{→\textcolor{blue}{Alfred Bernau}} mir ſeinen Entſchluß bekanntgab, das Stück
                    doch nicht zu geben, da es keine ſich ſteigernde Handlung und daher keine
                    Ausſicht auf Erfolg habe. Seither war der »\textcolor{green}{Fremde}{}\ledrightnote{\textcolor{green}{Der Fremde}}« auch ſchon im \textcolor{pink}{Burgtheater}{}\ledrightnote{\textcolor{pink}{Burgtheater}} und
                    wurde mit anerkennenswerter Eile und einem Formular retourniert. Von dem \textcolor{green}{\textcolor{pink}{Welſ}{}\ledrightnote{\textcolor{pink}{Wels}}er Stück}{}\ledrightnote{→\textcolor{green}{Yppl. Idylle in fünf Akten}} wollte D\textsuperscript{r}{ }\textcolor{blue}{\textsc{Waniek}}{}\ledrightnote{\textcolor{blue}{Wolfgang Waniek}} ohne Umarbeitung, die er am liebſten von einem Kompagnon – \textcolor{blue}{\textsc{Engel}}{}\ledrightnote{\textcolor{blue}{Alexander Engel}}{ }\strikeout{oder \textcolor{blue}{Landerberg}{}\ledrightnote{\textcolor{blue}{Landerberg}}} oder ſonſt wem – vorgenommen wüßte, überhaupt nichts wiſſen; und zu einer
                    ſolchen Arbeit fehlte es mir bisher an Luſt und an Stimmung. –\pend
           \pstart
           Es iſt ſehr traurig, daß auch die \textcolor{green}{Märchenkomödie}{}\ledrightnote{\textcolor{green}{Märchenkomödie}}, die ich in \textcolor{pink}{Karlsbad}{}\ledrightnote{\textcolor{pink}{Karlsbad}}
                    fleißig ſkizziert habe, keine Bühne finden wird, da der Stoff derart iſt, daß
                    überhaupt nur wenige begreifen werden, wie man zu ihm habe gelangen können: was
                    mich aber nicht abhalten ſoll, die Arbeit, die mich perſönlich intereſſiert, {\pb}zu Ende zu bringen, obwohl ſie mich,
                    der Anlage nach, viel Zeit und Mühe koſten wird. Ich hoffe, daß Sie,
                    hochverehrter Herr Doktor, dereinſt meine Stoffwahl nicht allzuſehr ſchelten
                    werden.\pend
           \pstart
           Indem ich Ihnen angenehmen Abſchluß des Sommeraufenthalts wünſche, bin ich mit
                    den herzlichſten Grüßen Ihr ſehr ergebener\pend
           \pstart \spacefill\mbox{D\textsuperscript{r}RAdam}\pend{}\endnumbering\briefempfaengerindex{Schnitzler, Arthur@\textsc{Schnitzler, Arthur}!zzzAdam, Robert@\emph{von Robert Adam}!1919-08-191@{19. 8. 1919}|)be}\mylabel{h}  \normalsize

\doendnotes{C}
\bigskip
\vfill

\clearpage

\footnotesize

\lohead{\textsc{register}}

% Definiere theindex-Environment komplett neu ohne reledmac
\makeatletter
\renewenvironment{theindex}{%
  \section*{\indexname}%
  \setlength{\parindent}{0pt}%
  \setlength{\parskip}{0pt plus 0.3pt}%
  \let\item\@idxitem
}{%
  \clearpage
}
\makeatother

\IfFileExists{\jobname-pw.ind}{\input{\jobname-pw.ind}}{}

\end{document}

      