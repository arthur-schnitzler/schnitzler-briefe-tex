%% latex-korrekturansicht-vorspann.tex
%% Vorspann für die Korrekturansicht.
%% Lädt die gemeinsame Datei latex-vorspann.tex mit gesetztem Schalter.

\newif\ifkorrekturansicht
\korrekturansichttrue

\input{../tex-inputs/latex-vorspann}


               \section[Arthur Schnitzler an Richard Beer-Hofmann, 2{[}6?{]}. 7. 1896]{ Arthur Schnitzler an Richard Beer-Hofmann,
               2{[}6?{]}. 7. 1896}\nopagebreak\mylabel{v}\rehead{ }\normalsize\beginnumbering\briefempfaengerindex{Beer-Hofmann, Richard@\textsc{Beer-Hofmann, Richard}!zzzSchnitzler, Arthur@\emph{von Arthur Schnitzler}!1896-07-262@{2{[}6?{]}. 7. 1896}|(be} \toendnotes[C]{\smallbreak\pagebreak[2]} \Standort{YCGL, MSS 31.}
\physDesc{Postkarte
\newline{}Handschrift: Bleistift, deutsche Kurrent\newline{}Versand: 1) Stempel: »\nobreak{}\oindex{Oslo@\textbf{Oslo}, \emph{http://www.geonames.org/ontologyP.PPLC}|pwk}Kristiania, {[}26{]} VII 96\nobreak{}«.  2) Stempel: »\nobreak{}\oindex{Kopenhagen@\textbf{Kopenhagen}, \emph{Besiedelter Ort (A.BSO)}|pwk}Kjøbenhavn, 27. 7. 96, 10MB\nobreak{}«. }\pstart{}{\pb}\textsc{Dr. \strikeout{Paul} Rich.
                     Beer-Hofmann}\pend{}\pstart{}\textcolor{pink}{\textsc{Kopenhagen}}{}\ledrightnote{\textcolor{pink}{Kopenhagen}}\pend{}\pstart{}\textsc{post rest.}\pend{}{\bigskip}\pstart
           \noindent{}{\pb}Lieber Richard, { }ſobald Sie Ihre Adreſſe \textcolor{pink}{\textsc{Kopenhagen}}{}\ledrightnote{\textcolor{pink}{Kopenhagen}} u näheres \textcolor{pink}{\textsc{Skodsborg}}{}\ledrightnote{\textcolor{pink}{Skodsborg}} wiſſen, telegrafir\textcolor{gray}{en}
               Sie \textsc{\textcolor{blue}{Goldmann}{}\ledrightnote{\textcolor{blue}{Paul Goldmann}}}{ }\textsc{\textcolor{pink}{Paris}{}\ledrightnote{\textcolor{pink}{Paris}}}{ }\textsc{\textcolor{pink}{24 rue Feydeau}{}\ledrightnote{\textcolor{pink}{rue Feydeau}}}\pend
           \pstart
           Herzlich Ihr{\\[\baselineskip]}Arthur\pend
           \leftskip=0em{}\endnumbering\briefempfaengerindex{Beer-Hofmann, Richard@\textsc{Beer-Hofmann, Richard}!zzzSchnitzler, Arthur@\emph{von Arthur Schnitzler}!1896-07-262@{2{[}6?{]}. 7. 1896}|)be}\mylabel{h}  \normalsize

\doendnotes{C}
\bigskip
\vfill

\clearpage

\footnotesize

\lohead{\textsc{register}}

% Definiere theindex-Environment komplett neu ohne reledmac
\makeatletter
\renewenvironment{theindex}{%
  \section*{\indexname}%
  \setlength{\parindent}{0pt}%
  \setlength{\parskip}{0pt plus 0.3pt}%
  \let\item\@idxitem
}{%
  \clearpage
}
\makeatother

\IfFileExists{\jobname-pw.ind}{\input{\jobname-pw.ind}}{}

\end{document}

      