%% latex-korrekturansicht-vorspann.tex
%% Vorspann für die Korrekturansicht.
%% Lädt die gemeinsame Datei latex-vorspann.tex mit gesetztem Schalter.

\newif\ifkorrekturansicht
\korrekturansichttrue

\input{../tex-inputs/latex-vorspann}


               \section[Arthur Schnitzler an Peter Altenberg, 29. 10. 1896]{ Arthur Schnitzler an Peter Altenberg, 29. 10. 1896}\nopagebreak\mylabel{v}\rehead{ }\normalsize\beginnumbering\briefempfaengerindex{Altenberg, Peter@\textsc{Altenberg, Peter}!zzzSchnitzler, Arthur@\emph{von Arthur Schnitzler}!1896-10-291@{29. 10. 1896}|(be} \toendnotes[C]{\smallbreak\pagebreak[2]} \Standort{Wienbibliothek im Rathaus, H.I.N.-137077.}
\physDesc{Brief, 1 Blatt, 2 Seiten, Fotokopie
\newline{}Altenberg: Ergänzung, nur zwei der vier Zeilen der Notiz sind ansatzweise zu
                                 entziffern: »\noindent{}\textsc{Lendway}{ / }\textsc{II. \textcolor{gray}{A}\textcolor{gray}{×}\-\textcolor{gray}{×}\-\textcolor{gray}{×}\-\textcolor{gray}{×}\-\textcolor{gray}{×}gaſſe 5}«. \textcolor{blue}{Karl Kraus} beschrieb
                                 diesen Text: »Der Wert des Autogramms ist allerdings
                                    beträchtlich erhöht durch eine Randnotiz Peter Altenbergs, der
                                    die ihm widerfahrene literarische Weihe mit den Adressen eines
                                    Nachtcafés und offenbar einer von dessen Besucherinnen quittiert
                                    hat«. \textcolor{green}{Die Fackel},
                                 Jg. 24, Nr. 608–612, Ende Dezember
                                 1922, S. 52. \newline{}Ordnung: Im Nachlass von \textcolor{blue}{Karl Kraus}
                                 überliefert. \textcolor{blue}{Kraus} ergänzte
                                 (vor der Kopie) am Objekt: »handſchriftliche Notiz von Peter
                                    Altenberg. Das Dokument 1896 von ihm empfangen. \textcolor{pink}{Wien}, im November
                                          1922\hspace*{1.5em}\textcolor{blue}{Karl Kraus}« \newline{}Zusatz: \textcolor{blue}{Kraus} ließ das Original
                                 versteigern. Schnitzler bot selber mit, wurde aber überboten.
                                    Vgl. \emph{Briefe} II,293–296 und
                                    \textcolor{green}{Die Fackel} von Ende
                                    1922 bis Anfang 1923 }\buchAbdrucke{\weitereDrucke{1) \emph{Vorlesung Karl Kraus [Programm]}. (26. 11. 1922).} \weitereDrucke{2) \pwindex{Fackel@\emph{Die Fackel}|pwk}\emph{Die Fackel}, Jg. 24, Nr. 608–612, Ende Dezember 1922, S. 51.} \weitereDrucke{3) Reinhard Urbach: \emph{»Schwätzer sind Verbrecher«. Bemerkungen zu Schnitzlers Dramenfragment »Das Wort«.} In: \emph{Literatur und Kritik}, Jg. 3 (1968), S. 292–304, hier S. 293.} }\toendnotes[C]{\smallbreak}\pstart{}{\pb}Lieber Herr Peter Altenberg,\pend\pstart
           geſtern ſprach ich mit \textcolor{blue}{\textsc{Gerhard Hauptmann}}{}\ledrightnote{\textcolor{blue}{Gerhart Hauptmann}}, der ſich über Ihr \textcolor{green}{Buch}{}\ledrightnote{→\textcolor{green}{Wie ich es sehe}} in
               unendlich ſympathiſcher Weiſe äußerte u. unter anderm sagte, ſeit \uline{Jahren} habe kein \textcolor{green}{Buch}{}\ledrightnote{→\textcolor{green}{Wie ich es sehe}} einen ſo ſtarken Eindruck auf ihn gemacht als das
               Ihre.\pend
           \pstart
           Da dieſe Bemerkung für Sie \label{K_L00611_1v}\edtext{intereſſant
                  ſein}{\lemma{\textnormal{\emph{intereſſant
                  ſein}}}\Cendnote{\textnormal{Für \textcolor{blue}{Altenberg} bot sie den Anlass, \textcolor{blue}{Hauptmann} direkt einen Brief zu schreiben. (\emph{Selbsterfindung eines Dichters}, S. 80.)}}}\label{K_L00611_1h}
               dürfte und ſie ſonſt kaum an Sie {\pb}gelangen
               könnte, fühle ich mich in gewiſſem Sinne angenehm verpflichtet, ſie Ihnen
               mitzutheilen.\pend
           \pstart
           Mit beſtem Gruſs Ihr ergebener{\\[\baselineskip]}\spacefill\mbox{ArthurSchnitzler}\pend
           \leftskip=0em{}\pstart
           \textcolor{pink}{Berlin}{}\ledrightnote{\textcolor{pink}{Berlin}}, 29. X. 96.\pend
           \endnumbering\briefempfaengerindex{Altenberg, Peter@\textsc{Altenberg, Peter}!zzzSchnitzler, Arthur@\emph{von Arthur Schnitzler}!1896-10-291@{29. 10. 1896}|)be}\mylabel{h}  \normalsize

\doendnotes{C}
\bigskip
\vfill

\clearpage

\footnotesize

\lohead{\textsc{register}}

% Definiere theindex-Environment komplett neu ohne reledmac
\makeatletter
\renewenvironment{theindex}{%
  \section*{\indexname}%
  \setlength{\parindent}{0pt}%
  \setlength{\parskip}{0pt plus 0.3pt}%
  \let\item\@idxitem
}{%
  \clearpage
}
\makeatother

\IfFileExists{\jobname-pw.ind}{\input{\jobname-pw.ind}}{}

\end{document}

      