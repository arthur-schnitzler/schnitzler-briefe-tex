%% latex-korrekturansicht-vorspann.tex
%% Vorspann für die Korrekturansicht.
%% Lädt die gemeinsame Datei latex-vorspann.tex mit gesetztem Schalter.

\newif\ifkorrekturansicht
\korrekturansichttrue

\input{../tex-inputs/latex-vorspann}


               \section[Arthur Schnitzler an Richard Beer-Hofmann, 14. 9. 1905]{ Arthur Schnitzler an Richard Beer-Hofmann, 14. 9. 1905}\nopagebreak\mylabel{v}\rehead{ }\normalsize\beginnumbering\briefempfaengerindex{Beer-Hofmann, Richard@\textsc{Beer-Hofmann, Richard}!zzzSchnitzler, Arthur@\emph{von Arthur Schnitzler}!1905-09-141@{14. 9. 1905}|(be} \toendnotes[C]{\smallbreak\pagebreak[2]} \Standort{YCGL, MSS 31.}
\physDesc{Brief, 1 Blatt (Briefpapier mit Trauerrand), 4 Seiten, Umschlag
\newline{}Handschrift: schwarze Tinte, deutsche Kurrent\newline{}Versand: 1) Stempel: »\nobreak{}\oindex{XVIII., Waehring@\textbf{XVIII., Währing}, \emph{Bezirk (A.BZK)}|pwk}18/1 Wien, 14. IX. 0\textcolor{gray}{5}, 6\nobreak{}«.  2) Stempel: »\nobreak{}\oindex{Rodaun@\textbf{Rodaun}, \emph{Teil eines besiedelten Ortes (A.BSOX)}|pwk}Rodaun\nobreak{}«. 3) Stempel: »\nobreak{}\oindex{Venedig@\textbf{Venedig}, \emph{Besiedelter Ort (A.BSO)}|pwk}\textcolor{gray}{Venezia}\nobreak{}«. 4) mit blauer Tinte von unbekannter Hand in Lateinschrift
                           zur Adresse hinzugefügt: »derzeit \uline{\textcolor{pink}{Venedig}}{ }\textcolor{pink}{Lido Grand Hotel}{ }\uline{\textcolor{pink}{Italien}}«}\buchAbdrucke{\weitereDrucke{Arthur Schnitzler, Richard Beer-Hofmann: \emph{Briefwechsel 1891–1931}. Hg. Konstanze Fliedl. Wien, Zürich: \emph{Europaverlag} 1992, S. 175.} }\toendnotes[C]{\smallbreak}\pstart{}{\pb}\textcolor{gray}{\textbf{Dr. Arthur Schnitzler}}\pend{}\pstart{}\textcolor{gray}{\textbf{\textcolor{pink}{Wien XVIII. Spoettelgasse 7}{}\ledrightnote{\textcolor{pink}{Edmund-Weiß-Gasse}}.}}\pend{}{\bigskip}\pstart{}{\pb}\textsc{Herrn Dr. Richard Beer-Hofmann}\pend{}\pstart{}\textcolor{pink}{\textsc{Rodaun}}{}\ledrightnote{\textcolor{pink}{Rodaun}}\pend{}\pstart{}\textsc{bei \textcolor{pink}{Liesing}{}\ledrightnote{\textcolor{pink}{XXIII., Liesing}}}\pend{}\pstart{}\textcolor{pink}{\textsc{Liesingerstr.} 2}{}\ledrightnote{\textcolor{pink}{Liesingerstraße}}.\pend{}\pstart{}ev. nachzuſenden.\pend{}{\bigskip}\pstart
           \raggedleft{}{\pb}\textcolor{pink}{Wien}{}\ledrightnote{\textcolor{pink}{Wien}}{ }14. 9. 905\pend
           \pstart
           lieber Richard, ich habe erwartet, eine Nachricht von Ihnen zu
                  beko{\geminationm}en, we{\geminationn} Sie
               irgendwo gelandet ſind, und da ich nicht weiſs, ob Sie ſchon, noch, überhaupt am \textcolor{pink}{Lido}{}\ledrightnote{\textcolor{pink}{Lido}} ſind und in welchem Hotel, richte ich dieſe
               Zeilen an Ihre \textcolor{pink}{Rodaun}{}\ledrightnote{\textcolor{pink}{Rodaun}}er Adreſſe. Der Brief an \textcolor{blue}{\textsc{Mir. Horwitz}}{}\ledrightnote{\textcolor{blue}{Mirjam Horwitz}} iſt längſt beſorgt, übrigens ko{\geminationm}t Adreſſatin
               morgen hier an (mit dem \textcolor{brown}{\textsc{Roland}}{}\ledrightnote{\textcolor{brown}{Roland von Berlin}} von \textcolor{pink}{Berlin}{}\ledrightnote{\textcolor{pink}{Berlin}}, was kein Liebhaber, ſondern ein \textsc{Caba{\pb}ret} iſt). Ob und wann ich in dieſem
               Herbſt noch wegkomme, iſt ungewiſs, da ich wahrſcheinlich ſehr bald \textcolor{pink}{Burgtheater}{}\ledrightnote{\textcolor{pink}{Burgtheater}}proben haben dürfte. (Sie haben wohl geleſen; näheres
               mündlich, die Sache iſt mir höchſt angenehm; \textcolor{blue}{Schl.}{}\ledrightnote{\textcolor{blue}{Paul Schlenther}} hatte ſich über \textcolor{blue}{Brahm}{}\ledrightnote{\textcolor{blue}{Otto Brahm}} an mich
               gewandt.) Auch mit dem zweiten \textcolor{green}{Stück}{}\ledrightnote{→\textcolor{green}{Der Ruf des Lebens. Schauspiel in drei Akten}}, das zur Zeit der Vorleſung im 3. Akt noch höchſt unſicher war, bin
               ich jetzt glaub ich leidlich fertig – oder ka{\geminationn} nur ni{\geminationm}er weiter, was aufs gleiche {\pb}herausko{\geminationm}t. –
               Wahrſcheinlich kriegt auch das zweite der \textcolor{blue}{Brahm}{}\ledrightnote{\textcolor{blue}{Otto Brahm}};
               mit \textcolor{blue}{\textsc{Reinhardt}}{}\ledrightnote{\textcolor{blue}{Max Reinhardt}} und den Seinen iſt einfach nicht zu
               verhandeln. Sie depeſchiren einem von Briefen, die auf dem Wege ſind – und die nie
               geſchrieben wurden – und das iſt noch nicht das ärgſte. Auch darüber mündlich. –\pend
           \pstart
           Sagen Sie mir doch ein Wort, wo Sie ſind, wie lang Sie bleiben, wann Sie kommen, wie
               es \textcolor{blue}{Paula}{}\ledrightnote{\textcolor{blue}{Paula Beer-Hofmann}} geht und den \textcolor{blue}{Kindern}{}\ledrightnote{→\textcolor{blue}{Naëmah Beer-Hofmann}{\newline}→\textcolor{blue}{Mirjam Beer-Hofmann}{\newline}→\textcolor{blue}{Gabriel Beer-Hofmann}} –\pend
           \pstart
           {\pb}Wir ſpielen täglich Tennis, und bald hoff ich wieder
               in ein geordnetes Arbeiten zu gerathen. \textcolor{blue}{Olga}{}\ledrightnote{\textcolor{blue}{Olga Schnitzler}}, die
               Sie alle herzlich grüßt, iſt ſehr wohl, \textcolor{blue}{Heinrich}{}\ledrightnote{\textcolor{blue}{Heinrich Schnitzler}}
               desgleichen – ſchreiben Sie bitte!\pend
           \pstart
           Von Herzen Ihr{\\[\baselineskip]}\spacefill\mbox{A.}\pend
           \leftskip=0em{}\endnumbering\briefempfaengerindex{Beer-Hofmann, Richard@\textsc{Beer-Hofmann, Richard}!zzzSchnitzler, Arthur@\emph{von Arthur Schnitzler}!1905-09-141@{14. 9. 1905}|)be}\mylabel{h}  \normalsize

\doendnotes{C}
\bigskip
\vfill

\clearpage

\footnotesize

\lohead{\textsc{register}}

% Definiere theindex-Environment komplett neu ohne reledmac
\makeatletter
\renewenvironment{theindex}{%
  \section*{\indexname}%
  \setlength{\parindent}{0pt}%
  \setlength{\parskip}{0pt plus 0.3pt}%
  \let\item\@idxitem
}{%
  \clearpage
}
\makeatother

\IfFileExists{\jobname-pw.ind}{\input{\jobname-pw.ind}}{}

\end{document}

      