%% latex-korrekturansicht-vorspann.tex
%% Vorspann für die Korrekturansicht.
%% Lädt die gemeinsame Datei latex-vorspann.tex mit gesetztem Schalter.

\newif\ifkorrekturansicht
\korrekturansichttrue

\input{../tex-inputs/latex-vorspann}


               \section[Olga und Arthur Schnitzler an Richard und Paula Beer-Hofmann, {[}19. 5. 1914{]}]{ Olga und Arthur Schnitzler an Richard und Paula Beer-Hofmann,
               {[}19. 5. 1914{]}}\nopagebreak\mylabel{v}\rehead{ }\normalsize\beginnumbering\briefempfaengerindex{Beer-Hofmann, Paula@\textsc{Beer-Hofmann, Paula}!zzzSchnitzler, Arthur@\emph{von Arthur Schnitzler}!1914-05-191@{{[}19. 5. 1914{]}}|(be}\briefempfaengerindex{Beer-Hofmann, Paula@\textsc{Beer-Hofmann, Paula}!zzzSchnitzler, Olga@\emph{von Olga Schnitzler}!1914-05-191@{{[}19. 5. 1914{]}}|(be}\briefempfaengerindex{Beer-Hofmann, Richard@\textsc{Beer-Hofmann, Richard}!zzzSchnitzler, Arthur@\emph{von Arthur Schnitzler}!1914-05-191@{{[}19. 5. 1914{]}}|(be}\briefempfaengerindex{Beer-Hofmann, Richard@\textsc{Beer-Hofmann, Richard}!zzzSchnitzler, Olga@\emph{von Olga Schnitzler}!1914-05-191@{{[}19. 5. 1914{]}}|(be} \toendnotes[C]{\smallbreak\pagebreak[2]} \Standort{YCGL, MSS 31.}
\physDesc{Bildpostkarte
\newline{}Handschrift Arthur Schnitzler: Bleistift, deutsche Kurrent\newline{}Handschrift Olga Schnitzler: Bleistift, lateinische Kurrent\newline{}Versand: 1) Stempel: »\nobreak{}\textcolor{gray}{20. 5.}\nobreak{}«.  2) Der Versandweg ist unklar, da eine Briefmarke des deutschen Reiches zum Einsatz kommt, die 
                                 Kreuzfahrt aber am 20. 5. 1910 von Süden 
                                 kommend erst \textcolor{pink}{Southampton} erreicht\newline{}Ordnung: mit Bleistift von unbekannter Hand datiert:
                              »19. 5. 14.« }\pstart{}{\pb}\textcolor{pink}{Austria}{}\ledrightnote{\textcolor{pink}{Österreich}}\pend{}\pstart{}Herrn u. Frau\pend{}\pstart{}D\textsuperscript{r} Richard Beer-Hofmann\pend{}\pstart{}\textcolor{pink}{Vienna}{}\ledrightnote{\textcolor{pink}{Wien}}\pend{}\pstart{}\textcolor{pink}{XVIII Hasenauerstr. 59}{}\ledrightnote{\textcolor{pink}{Hasenauerstraße}}.\pend{}{\bigskip}\pstart
           \noindent{}\centering{}{\pb}\textcolor{gray}{\textbf{\textcolor{brown}{NORDDEUTSCHER LLOYD BREMEN}{}\ledrightnote{\textcolor{brown}{Norddeutscher Lloyd}}}}{\\}\textcolor{gray}{\textbf{Reichspostdampfer »York«}}\pend
           \pstart
           \noindent{}{\pb}Herzliche Grüsse!\pend
           \pstart
           morgen hoffen wir, in \textcolor{pink}{Southampten}{}\ledrightnote{\textcolor{pink}{Southampton}} zu sein,
               übermorgen \textcolor{pink}{Antwerpen}{}\ledrightnote{\textcolor{pink}{Antwerpen}}, Freitag{ }\textcolor{pink}{Amsterdam}{}\ledrightnote{\textcolor{pink}{Amsterdam}}.\pend
           \pstart Ihre\spacefill\mbox{OlgaS.}\pend{}\pstart
           {\pb}{[}hs. Schnitzler:{]} Herzlichſt Ihr\spacefill\mbox{Arthur}\pend
           \pstart
           Eben passiren wir ziemlich ſeitlich den \textcolor{pink}{Golf von Biscaya}{}\ledrightnote{\textcolor{pink}{Biskaya}} bei
               ruhiger See.\pend
           \pstart
           Die Fahrt heut den 7. Tag, herrlich\pend
           \endnumbering\briefempfaengerindex{Beer-Hofmann, Paula@\textsc{Beer-Hofmann, Paula}!zzzSchnitzler, Arthur@\emph{von Arthur Schnitzler}!1914-05-191@{{[}19. 5. 1914{]}}|)be}\briefempfaengerindex{Beer-Hofmann, Paula@\textsc{Beer-Hofmann, Paula}!zzzSchnitzler, Olga@\emph{von Olga Schnitzler}!1914-05-191@{{[}19. 5. 1914{]}}|)be}\briefempfaengerindex{Beer-Hofmann, Richard@\textsc{Beer-Hofmann, Richard}!zzzSchnitzler, Arthur@\emph{von Arthur Schnitzler}!1914-05-191@{{[}19. 5. 1914{]}}|)be}\briefempfaengerindex{Beer-Hofmann, Richard@\textsc{Beer-Hofmann, Richard}!zzzSchnitzler, Olga@\emph{von Olga Schnitzler}!1914-05-191@{{[}19. 5. 1914{]}}|)be}\mylabel{h}  \normalsize

\doendnotes{C}
\bigskip
\vfill

\clearpage

\footnotesize

\lohead{\textsc{register}}

% Definiere theindex-Environment komplett neu ohne reledmac
\makeatletter
\renewenvironment{theindex}{%
  \section*{\indexname}%
  \setlength{\parindent}{0pt}%
  \setlength{\parskip}{0pt plus 0.3pt}%
  \let\item\@idxitem
}{%
  \clearpage
}
\makeatother

\IfFileExists{\jobname-pw.ind}{\input{\jobname-pw.ind}}{}

\end{document}

      