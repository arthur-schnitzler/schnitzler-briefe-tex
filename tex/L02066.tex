%% latex-korrekturansicht-vorspann.tex
%% Vorspann für die Korrekturansicht.
%% Lädt die gemeinsame Datei latex-vorspann.tex mit gesetztem Schalter.

\newif\ifkorrekturansicht
\korrekturansichttrue

\input{../tex-inputs/latex-vorspann}


               \section[Arthur Schnitzler an Richard Beer-Hofmann, 15. 5. 1912]{ Arthur Schnitzler an Richard Beer-Hofmann, 15. 5. 1912}\nopagebreak\mylabel{v}\rehead{ }\normalsize\beginnumbering\briefempfaengerindex{Beer-Hofmann, Richard@\textsc{Beer-Hofmann, Richard}!zzzSchnitzler, Arthur@\emph{von Arthur Schnitzler}!1912-05-151@{15. 5. 1912}|(be} \toendnotes[C]{\smallbreak\pagebreak[2]} \Standort{YCGL, MSS 31.}
\physDesc{Visitenkarte mit Trauerrand
\newline{}Handschrift: Bleistift, deutsche Kurrent
\newline{}Beer-Hofmann: mit blauem Buntstift datiert: »15/5 12{ }\textcolor{pink}{\textsc{Venedig}}« }\pstart
           \noindent{}{\pb}Bin ſchon da. Wir wohnen \uline{\textcolor{pink}{\textsc{Hotel Europe}}{}\ledrightnote{\textcolor{pink}{Hotel de l’Europe}}}. Vorläufig \textsc{provisor.} Zimmer. Bitte holen Sie uns etwa
               um 10 ab oder ſchauen Sie vorbei oder telepho{\pb}niren Sie.\pend
           \pstart
           Herzlichſt{\\[\baselineskip]}Ihr\pend
           \leftskip=0em{}\pstart
           \centering{}\textcolor{gray}{\textbf{D\textsuperscript{r} Arthur Schnitzler}}\pend
           \endnumbering\briefempfaengerindex{Beer-Hofmann, Richard@\textsc{Beer-Hofmann, Richard}!zzzSchnitzler, Arthur@\emph{von Arthur Schnitzler}!1912-05-151@{15. 5. 1912}|)be}\mylabel{h}  \normalsize

\doendnotes{C}
\bigskip
\vfill

\clearpage

\footnotesize

\lohead{\textsc{register}}

% Definiere theindex-Environment komplett neu ohne reledmac
\makeatletter
\renewenvironment{theindex}{%
  \section*{\indexname}%
  \setlength{\parindent}{0pt}%
  \setlength{\parskip}{0pt plus 0.3pt}%
  \let\item\@idxitem
}{%
  \clearpage
}
\makeatother

\IfFileExists{\jobname-pw.ind}{\input{\jobname-pw.ind}}{}

\end{document}

      