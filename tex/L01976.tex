%% latex-korrekturansicht-vorspann.tex
%% Vorspann für die Korrekturansicht.
%% Lädt die gemeinsame Datei latex-vorspann.tex mit gesetztem Schalter.

\newif\ifkorrekturansicht
\korrekturansichttrue

\input{../tex-inputs/latex-vorspann}


               \section[Richard Beer-Hofmann an Arthur Schnitzler, 3. 11. 1910]{ Richard Beer-Hofmann an Arthur Schnitzler, 3. 11. 1910}\nopagebreak\mylabel{v}\rehead{ }\normalsize\beginnumbering\briefempfaengerindex{Schnitzler, Arthur@\textsc{Schnitzler, Arthur}!zzzBeer-Hofmann, Richard@\emph{von Richard Beer-Hofmann}!1910-11-031@{3. 11. 1910}|(be} \toendnotes[C]{\smallbreak\pagebreak[2]} \Standort{CUL, Schnitzler, B 8.}
\physDesc{Brief, 1 Blatt, 2 Seiten
\newline{}Handschrift: Bleistift, lateinische Kurrent
\newline{}Schnitzler: mit Bleistift beschriftet: »\textsc{B. H}« \newline{}Ordnung: mit Bleistift von unbekannter Hand nummeriert:
                              »238« }\buchAbdrucke{\weitereDrucke{Arthur Schnitzler, Richard Beer-Hofmann: \emph{Briefwechsel 1891–1931}. Hg. Konstanze Fliedl. Wien, Zürich: \emph{Europaverlag} 1992, S. 213.} }\pstart
           \raggedleft{}{\pb}3. November 1910\pend
           \pstart
           Lieber Arthur! \textcolor{blue}{Leo}{}\ledrightnote{\textcolor{blue}{Leo Van-Jung}} – den ich gestern sah, – bittet um Folgendes:
               Eine Frau \textcolor{blue}{Moller}{}\ledrightnote{\textcolor{blue}{Alice Moller}} (etwas Snob), Schülerin von ihm,
               will einen Autorenabend zu Gunsten des Vereines »\textcolor{brown}{Mutterschutz}{}\ledrightnote{\textcolor{brown}{Bund für Mutterschutz}}« machen. Möchte dass Sie – gegen von Ihnen zu besti{\geminationm}endes Honorar – lesen. Ausser Ihnen nur »würdige
               Entourage«. \textcolor{blue}{Salten}{}\ledrightnote{\textcolor{blue}{Felix Salten}} soll principiell nichts dagegen
               haben. Im \textcolor{pink}{Kl. Musikvereinssaal}{}\ledrightnote{\textcolor{pink}{Musikverein}}. \textcolor{blue}{Leo}{}\ledrightnote{\textcolor{blue}{Leo Van-Jung}} frägt bei Ihnen an, um Ihnen – u Frau \textcolor{blue}{M.}{}\ledrightnote{\textcolor{blue}{Alice Moller}} den Besuch \introOben{}eventuell\introOben{} zu ersparen.
               Er bittet mich Ihnen zu sagen, dass er gar nichts bei der Sache zu tun hat, Sie sich
               um seinetwillen nicht mehr {\pb}Freundlichkeit i. d. Absage (oder Annahme) auferlegen sollen, als \strikeout{es} Ihnen passt. Er hat nur Frau \textcolor{blue}{M.}{}\ledrightnote{\textcolor{blue}{Alice Moller}} zugesagt Sie vorerst zu fragen, da im Falle Ihrer princip.
               Abgeneigtheit jede weitere Belästigung für Sie entfällt\pend
           \pstart
           Er erwartet – durch mich – von Ihnen nur ein »Ja« oder »Nein«; \introOben{}Mit\introOben{} Motivirungen sollen Sie Sich nicht mühen –\pend
           \pstart
           Bitte noch heute um Antwort.
               Herzlichst\pend
           \pstart
           Ihr{\\[\baselineskip]}\spacefill\mbox{Richard}\pend
           \leftskip=0em{}\endnumbering\briefempfaengerindex{Schnitzler, Arthur@\textsc{Schnitzler, Arthur}!zzzBeer-Hofmann, Richard@\emph{von Richard Beer-Hofmann}!1910-11-031@{3. 11. 1910}|)be}\mylabel{h}  \normalsize

\doendnotes{C}
\bigskip
\vfill

\clearpage

\footnotesize

\lohead{\textsc{register}}

% Definiere theindex-Environment komplett neu ohne reledmac
\makeatletter
\renewenvironment{theindex}{%
  \section*{\indexname}%
  \setlength{\parindent}{0pt}%
  \setlength{\parskip}{0pt plus 0.3pt}%
  \let\item\@idxitem
}{%
  \clearpage
}
\makeatother

\IfFileExists{\jobname-pw.ind}{\input{\jobname-pw.ind}}{}

\end{document}

      