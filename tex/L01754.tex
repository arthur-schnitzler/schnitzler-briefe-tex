%% latex-korrekturansicht-vorspann.tex
%% Vorspann für die Korrekturansicht.
%% Lädt die gemeinsame Datei latex-vorspann.tex mit gesetztem Schalter.

\newif\ifkorrekturansicht
\korrekturansichttrue

\input{../tex-inputs/latex-vorspann}


               \section[Karl Kraus an Arthur Schnitzler, 20. 1. 1908]{ Karl Kraus an Arthur Schnitzler, 20. 1. 1908}\nopagebreak\mylabel{v}\rehead{ }\normalsize\beginnumbering\briefempfaengerindex{Schnitzler, Arthur@\textsc{Schnitzler, Arthur}!zzzKraus, Karl@\emph{von Karl Kraus}!1908-01-201@{20. 1. 1908}|(be} \toendnotes[C]{\smallbreak\pagebreak[2]} \Standort{DLA, A:Schnitzler, HS.NZ85.1.5731.}
\physDesc{Brief, 1 Blatt, 1 Seite
\newline{}Handschrift: schwarze Tinte, deutsche Kurrent
\newline{}Schnitzler: mit Bleistift beschriftet: »\textsc{Carl Kraus}« und abgehakt, womöglich als Zeichen, dass es
                                 abgeschrieben wurde }\buchAbdrucke{\weitereDrucke{\emph{Karl Kraus und Arthur Schnitzler. Eine Dokumentation.} Hg. Reinhard Urbach. In: \emph{Literatur und Kritik}, Bd. 49, Oktober 1970, S. 522.} }\pstart
           \raggedleft{}{\pb}\textcolor{pink}{Wien}{}\ledrightnote{\textcolor{pink}{Wien}}{ }20. 1. 08\pend
           \pstart
           Eine in \textcolor{pink}{New York}{}\ledrightnote{\textcolor{pink}{New York City}} lebende Freundin, Mrs. \textcolor{blue}{Fox}{}\ledrightnote{\textcolor{blue}{Kete Fox}} – die als \textcolor{blue}{Kete
                  Parsenow}{}\ledrightnote{\textcolor{blue}{Kete Fox}} vor einigen Jahren im \textcolor{pink}{Berliner Kleinen
                  Theater}{}\ledrightnote{\textcolor{pink}{Kleines Theater}}{ }\textcolor{green}{Salome}{}\ledrightnote{\textcolor{green}{Salomé. Drame en une acte}}, in »\textcolor{green}{Rausch}{}\ledrightnote{\textcolor{green}{Rausch}}«, »\textcolor{green}{Nachtasyl}{}\ledrightnote{\textcolor{green}{Nachtasyl. Szenen aus der Tiefe in vier Aufzügen}}« etc. geſpielt hat –,
               erſucht mich Sie zu fragen, ob Sie geneigt wären, ihr das Recht der \textcolor{pink}{engliſchen}{}\ledrightnote{\textcolor{pink}{England}} Überſetzung und Aufführung Ihres »\textcolor{green}{Schleiers der Beatrice}{}\ledrightnote{\textcolor{green}{Der Schleier der Beatrice. Schauspiel in fünf Akten}}« zu erteilen. Für einen freundlichen
               Beſcheid an meine oder die Adreſſe: \textcolor{blue}{Mrs. \textcolor{blue}{A. C. Fox}{}\ledrightnote{\textcolor{blue}{Albert Claughten Fox}}}{}\ledrightnote{\textcolor{blue}{Kete Fox}}, \textcolor{pink}{New-Yersey}{}\ledrightnote{\textcolor{pink}{New Jersey}}{ }\textcolor{pink}{U.S.A.}{}\ledrightnote{\textcolor{pink}{Vereinigte Staaten von Amerika (USA)}}{ }\textcolor{pink}{Addison Street}{}\ledrightnote{\textcolor{pink}{Addison Street}}, wäre ich Ihnen sehr verbunden.\pend
           \pstart
           Ich geſtatte mir bei dieſer Gelegenheit, Sie zum \textcolor{brown}{Grillparzer-Preis}{}\ledrightnote{\textcolor{brown}{Franz-Grillparzer-Preis}} zu beglückwünſchen, und bin mit hochachtungsvollem
               Gruß\pend
           \pstart
           Ihr ganz ergebener{\\[\baselineskip]}\spacefill\mbox{Karl Kraus}\pend
           \leftskip=0em{}\pstart
           \noindent{}\raggedleft{}\textcolor{pink}{Wien IV. Schwindg. 3, Th. 3}{}\ledrightnote{\textcolor{pink}{Schwindgasse}}\pend
           \endnumbering\briefempfaengerindex{Schnitzler, Arthur@\textsc{Schnitzler, Arthur}!zzzKraus, Karl@\emph{von Karl Kraus}!1908-01-201@{20. 1. 1908}|)be}\mylabel{h}  \normalsize

\doendnotes{C}
\bigskip
\vfill

\clearpage

\footnotesize

\lohead{\textsc{register}}

% Definiere theindex-Environment komplett neu ohne reledmac
\makeatletter
\renewenvironment{theindex}{%
  \section*{\indexname}%
  \setlength{\parindent}{0pt}%
  \setlength{\parskip}{0pt plus 0.3pt}%
  \let\item\@idxitem
}{%
  \clearpage
}
\makeatother

\IfFileExists{\jobname-pw.ind}{\input{\jobname-pw.ind}}{}

\end{document}

      