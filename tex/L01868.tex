%% latex-korrekturansicht-vorspann.tex
%% Vorspann für die Korrekturansicht.
%% Lädt die gemeinsame Datei latex-vorspann.tex mit gesetztem Schalter.

\newif\ifkorrekturansicht
\korrekturansichttrue

\input{../tex-inputs/latex-vorspann}


               \section[Hermann Bahr und Anna Bahr-Mildenburg an Arthur Schnitzler, 25. 8. 1909]{ Hermann Bahr und Anna Bahr-Mildenburg an Arthur Schnitzler,
               25. 8. 1909}\nopagebreak\mylabel{v}\rehead{ }\normalsize\beginnumbering\briefempfaengerindex{Schnitzler, Arthur@\textsc{Schnitzler, Arthur}!zzzBahr-Mildenburg, Anna@\emph{von Anna Bahr-Mildenburg}!1909-08-251@{25. 8. 1909}|(be}\briefempfaengerindex{Schnitzler, Arthur@\textsc{Schnitzler, Arthur}!zzzBahr, Hermann@\emph{von Hermann Bahr}!1909-08-251@{25. 8. 1909}|(be} \toendnotes[C]{\smallbreak\pagebreak[2]} \Standort{CUL, Schnitzler, B 5b.}
\physDesc{Bildpostkarte
\newline{}Handschrift Hermann Bahr: Bleistift, deutsche Kurrent\newline{}Handschrift Anna Bahr-Mildenburg: schwarze Tinte, lateinische Kurrent\newline{}Versand: 1) Stempel: »\nobreak{}\oindex{Zell am Ziller@\textbf{Zell am Ziller}, \emph{Besiedelter Ort (A.BSO)}|pwk}Zell am Ziller\nobreak{}«.  2) als Beilage zu einem Brief von \textcolor{blue}{Olga} an Schnitzler
            übermittelt. Mit Tinte von ihrer Hand ergänzt: »\textsc{Bevor ich den Brief schliesse, kommt diese Karte!}«
\newline{}Schnitzler: mit Bleistift ergänzt »Bahr« \newline{}Ordnung: mit Bleistift von unbekannter Hand nummeriert:
                                    »159« }\buchAbdrucke{\weitereDrucke{Hermann Bahr, Arthur Schnitzler: \emph{Briefwechsel, Aufzeichnungen, Dokumente (1891–1931)}. Hg. Kurt Ifkovits und Martin Anton Müller. Göttingen: \emph{Wallstein} 2018, S. 423.} }\toendnotes[C]{\smallbreak}\pstart{}{\pb}\textsc{D\textsuperscript{r} Arthur Schnitzler}\pend{}\pstart{}\textsc{\textcolor{pink}{Wien XVIII}{}\ledrightnote{\textcolor{pink}{XVIII., Währing}}}\pend{}\pstart{}\textsc{\textcolor{pink}{Spöttelgasse 7}{}\ledrightnote{\textcolor{pink}{Edmund-Weiß-Gasse}}}\pend{}{\bigskip}\pstart
           \noindent{}\centering{}\textcolor{gray}{\textbf{{\pb}Gruss von \textcolor{pink}{Zell
                        im Zillerthal!}{}\ledrightnote{\textcolor{pink}{Zell am Ziller}}}}\pend
           \pstart
           \centering{}{\pb}25. 8. 09\pend
           \pstart
           Dich und Deine liebe \textcolor{blue}{Frau}{}\ledrightnote{→\textcolor{blue}{Olga Schnitzler}} grüßen
               herzlichſt\pend
           \pstart \spacefill\mbox{Hermann und {[}hs. Bahr-Mildenburg:{]} Anna \label{K_L01868_1v}\edtext{Bahr-Mildenburg}{\lemma{\textnormal{\emph{Bahr-Mildenburg}}}\Cendnote{\textnormal{Die Hochzeit, nach der sie den Doppelnamen führte, hatte am 22. 8. 1909 in \textcolor{pink}{Aigen} (heute: Stadtteil von \textcolor{pink}{Salzburg})
                     stattgefunden.}}}\label{K_L01868_1h}}\pend{}\endnumbering\briefempfaengerindex{Schnitzler, Arthur@\textsc{Schnitzler, Arthur}!zzzBahr-Mildenburg, Anna@\emph{von Anna Bahr-Mildenburg}!1909-08-251@{25. 8. 1909}|)be}\briefempfaengerindex{Schnitzler, Arthur@\textsc{Schnitzler, Arthur}!zzzBahr, Hermann@\emph{von Hermann Bahr}!1909-08-251@{25. 8. 1909}|)be}\mylabel{h}  \normalsize

\doendnotes{C}
\bigskip
\vfill

\clearpage

\footnotesize

\lohead{\textsc{register}}

% Definiere theindex-Environment komplett neu ohne reledmac
\makeatletter
\renewenvironment{theindex}{%
  \section*{\indexname}%
  \setlength{\parindent}{0pt}%
  \setlength{\parskip}{0pt plus 0.3pt}%
  \let\item\@idxitem
}{%
  \clearpage
}
\makeatother

\IfFileExists{\jobname-pw.ind}{\input{\jobname-pw.ind}}{}

\end{document}

      