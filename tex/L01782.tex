%% latex-korrekturansicht-vorspann.tex
%% Vorspann für die Korrekturansicht.
%% Lädt die gemeinsame Datei latex-vorspann.tex mit gesetztem Schalter.

\newif\ifkorrekturansicht
\korrekturansichttrue

\input{../tex-inputs/latex-vorspann}


               \section[Hugo von Hofmannsthal an Arthur Schnitzler, 11. 7. 1908]{ Hugo von Hofmannsthal an Arthur Schnitzler, 11. 7. 1908}\nopagebreak\mylabel{v}\rehead{ }\normalsize\beginnumbering\briefempfaengerindex{Schnitzler, Arthur@\textsc{Schnitzler, Arthur}!zzzHofmannsthal, Hugo von@\emph{von Hugo von Hofmannsthal}!1908-07-111@{11. 7. 1908}|(be} \toendnotes[C]{\smallbreak\pagebreak[2]} \Standort{CUL, Schnitzler, B 43.}
\physDesc{Bildpostkarte
\newline{}Handschrift: Bleistift, deutsche Kurrent\newline{}Versand: Stempel: »\nobreak{}\oindex{Salzburg@\textbf{Salzburg}, \emph{Besiedelter Ort (A.BSO)}|pwk}Salzburg 2, 11. VII. 08, 6\nobreak{}«.  
\newline{}Schnitzler: mit Bleistift beschriftet: »\textsc{Hofma}« und »Hugo« \newline{}Ordnung: 1) mit Bleistift von unbekannter Hand nummeriert:
                              »328« 2) mit Bleistift von unbekannter Hand nummeriert: »298«}\buchAbdrucke{\weitereDrucke{Hugo von Hofmannsthal, Arthur Schnitzler: \emph{Briefwechsel}. Hg. Therese Nickl und Heinrich Schnitzler. Frankfurt am Main: \emph{S. Fischer} 1964, S. 237.} }\pstart{}{\pb}\textsc{Herrn}\pend{}\pstart{}\textsc{D\textsuperscript{r} Arthur Schnitzler}\pend{}\pstart{}\textcolor{pink}{\textsc{Seis am Schlern}}{}\ledrightnote{\textcolor{pink}{Seis am Schlern}}\pend{}\pstart{}\textcolor{pink}{\textsc{Villa Heufler}}{}\ledrightnote{\textcolor{pink}{Villa Heufler}}\pend{}\pstart{}\textcolor{pink}{\textsc{Süd Tirol}}{}\ledrightnote{\textcolor{pink}{Südtirol}}\pend{}{\bigskip}\pstart
           \noindent{}\centering{}\textcolor{gray}{\textbf{{\pb}\textcolor{pink}{Salzburg}{}\ledrightnote{\textcolor{pink}{Salzburg}} – \textcolor{pink}{Kiosk
                        Tomaselli}{}\ledrightnote{\textcolor{pink}{Café Tomaselli}}.}}\pend
           \pstart
           \raggedleft{}11. VII.\pend
           \pstart
           Wie lange waren wir ſchon nicht hier zuſa{\geminationm}en!\pend
           \pstart
           Ich gehe (weiterarbeitend) heute in die \textcolor{pink}{Fuſch}{}\ledrightnote{\textcolor{pink}{Bad Fusch}}, von
               wo ich Ihnen ſchreibe.\pend
           \pstart  Herzlichſt \spacefill\mbox{Hugo.}\pend{}\endnumbering\briefempfaengerindex{Schnitzler, Arthur@\textsc{Schnitzler, Arthur}!zzzHofmannsthal, Hugo von@\emph{von Hugo von Hofmannsthal}!1908-07-111@{11. 7. 1908}|)be}\mylabel{h}  \normalsize

\doendnotes{C}
\bigskip
\vfill

\clearpage

\footnotesize

\lohead{\textsc{register}}

% Definiere theindex-Environment komplett neu ohne reledmac
\makeatletter
\renewenvironment{theindex}{%
  \section*{\indexname}%
  \setlength{\parindent}{0pt}%
  \setlength{\parskip}{0pt plus 0.3pt}%
  \let\item\@idxitem
}{%
  \clearpage
}
\makeatother

\IfFileExists{\jobname-pw.ind}{\input{\jobname-pw.ind}}{}

\end{document}

      