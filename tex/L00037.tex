%% latex-korrekturansicht-vorspann.tex
%% Vorspann für die Korrekturansicht.
%% Lädt die gemeinsame Datei latex-vorspann.tex mit gesetztem Schalter.

\newif\ifkorrekturansicht
\korrekturansichttrue

\input{../tex-inputs/latex-vorspann}


               \section[Eduard Michael Kafka an Arthur Schnitzler, 30. 8. 1891]{ Eduard Michael Kafka an Arthur Schnitzler, 30. 8. 1891}\nopagebreak\mylabel{v}\rehead{ }\normalsize\beginnumbering\briefempfaengerindex{Schnitzler, Arthur@\textsc{Schnitzler, Arthur}!zzzKafka, Eduard Michael@\emph{von Eduard Michael Kafka}!1891-08-301@{30. 8. 1891}|(be} \toendnotes[C]{\smallbreak\pagebreak[2]} \Standort{DLA, A:Schnitzler, HS.NZ85.1.3604.}
\physDesc{Brief, 1 Blatt, 2 Seiten
\newline{}Handschrift: schwarze Tinte, deutsche Kurrent
\newline{}Schnitzler: mit rotem Buntstift beschriftet: »\textsc{Kafka}« und nummeriert: »(2)«. mit rotem Buntstift eine Unterstreichung  }\toendnotes[C]{\smallbreak}\pstart
           \noindent{}\centering{}{\pb}\textcolor{brown}{\textcolor{gray}{\textbf{\textbf{Moderne Dichtung.}}}}{}\ledrightnote{\textcolor{brown}{Moderne Dichtung/Moderne Rundschau}}\pend
           \pstart
           \noindent{}\centering{}\textcolor{gray}{\textbf{\so{Monatsſchrift für Literatur und Kritik}.}}\pend
           \pstart
           \noindent{}\centering{}\textcolor{gray}{\textbf{Herausgeber: \textbf{E. M. Kafka, \textcolor{pink}{Wien}{}\ledrightnote{\textcolor{pink}{Wien}}.} – Verlag: \textbf{\textcolor{brown}{Holzwarth {\kaufmannsund}
                           Ortony}{}\ledrightnote{\textcolor{brown}{Holzwarth {\kaufmannsund}  Ortony}}, \textcolor{pink}{Wien}{}\ledrightnote{\textcolor{pink}{Wien}}.}}}\pend
           \pstart
           \raggedleft{}\textcolor{pink}{Brünn}{}\ledrightnote{\textcolor{pink}{Brünn}}, \textcolor{gray}{\textbf{\textbf{\textcolor{pink}{Wien}{}\ledrightnote{\textcolor{pink}{Wien}},}}}{ }30. August \textcolor{gray}{\textbf{189}}1\pend
           \pstart{}Sehr verehrter Herr Doctor,\pend\pstart
           ich lade Sie freundlichſt ein, an einem »\label{K_L00037_1v}\edtext{\textsc{Oesterreichischen Jahrbuch für moderne Literatur}}{\lemma{\textnormal{\emph{Oesterreichischen … Literatur}}}\Cendnote{\textnormal{Das Jahrbuch sollte Beiträge von 42
                  Schriftstellern enthalten, wurde aber nicht verwirklicht. Vgl. den Brief \textcolor{blue}{Kafka}s an \textcolor{blue}{Ferdinand von Saar} vom 25. 8. 1891, in: \emph{Jugend in Wien. Literatur um 1900}. Ausstellung und
                     Katalog von Ludwig Greve und Werner Volke. München:
                        \emph{Kösel}{ }1974, S. 98.}}}\label{K_L00037_1h}« mitzuarbeiten, das ich anfangs
                  November herauszugeben beabſichtige. Und zwar erbitte ich mir für
               dasſelbe vor allem »\textcolor{green}{die \textsc{Elixire}}{}\ledrightnote{\textcolor{green}{Die drei Elixire}}«, u. wäre Ihnen ganz außerordentlich verbunden, könnte ich hiezu noch eine
               bisher ungedruckte \uline{Bluette} erhalten. Aus dem \textcolor{green}{\textsc{Anatol-Cyclus}}{}\ledrightnote{\textcolor{green}{Anatol}} haben Sie ja noch Etwas, – wenn ich nicht irre. Wenn möglich, bäte ich um recht
               baldige Zuſendung, da das Buch bereits anfangs September in Angriff genommen, alſo
               mit der Drucklegung begonnen werden wird.\pend
           \pstart
           Ich bäte ferner um Zuſendung Ihres »\textcolor{green}{Märchen}{}\ledrightnote{\textcolor{green}{Das Märchen. Schauspiel in drei Aufzügen}}«, um
               dasſelbe dem \textcolor{blue}{Direktor}{}\ledrightnote{→\textcolor{blue}{Adolf Baumann}} des
               Brünner \textcolor{pink}{Stadttheater}{}\ledrightnote{\textcolor{pink}{Stadttheater}} zu übermit{\pb}teln. Derſelbe verſprach mir, das Stück binnen 3 Tagen
               gelesen u. ſich bezüglich einer ev. Aufführung entſchieden zu haben. Wenn möglich, ſo
               wär es am beſten, wenn die Einreichung \uline{jetzt}
               geſchähe, da mir \textcolor{blue}{Baumann}{}\ledrightnote{\textcolor{blue}{Adolf Baumann}} mittheilt, daſs er auf
               Suche \introOben{}nach Novitäten\introOben{} iſt.\pend
           \pstart
           Was meine Geſundheit betrifft, ſo vermag ich leider nichts beſonders Günſtiges zu
               vermelden. Doch hoffe ich immerhin, in 4–6 Wochen wieder nach \textcolor{pink}{Wien}{}\ledrightnote{\textcolor{pink}{Wien}} zurückkehren zu können.\pend
           \pstart
           Sie würden mich durch ein paar Zeilen ſehr erfreuen. Auch bitte ich Sie recht ſehr,
               mich Ihrem Herrn \textcolor{blue}{Bruder}{}\ledrightnote{→\textcolor{blue}{Julius Schnitzler}}, der
               wohl ſehr böſe auf mich ſein wird, weil ich mich wirklich recht unartig ihm gegenüber
               benommen habe, frdlchst zu empfehlen. Es rächt ſich jetzt an mir, in unangenehmſter
               Weiſe, daſs ich ihm ſo vorzeitig Reißaus genommen!\pend
           \pstart
           Mit herzlichen Grüßen{\\[\baselineskip]}Ihr{\\[\baselineskip]}Sie aufrichtig hochſchätzender{\\[\baselineskip]}\spacefill\mbox{EMKafka}\pend
           \leftskip=0em{}\pstart
           \noindent{}\textcolor{pink}{Brünn}{}\ledrightnote{\textcolor{pink}{Brünn}}, \textcolor{pink}{Straßengaſſe 36}{}\ledrightnote{\textcolor{pink}{Hybešova}}\pend
           \pstart
           \label{T_L00037_1v}\edtext{\textcolor{gray}{\textbf{Alle den Inhalt der »\textcolor{brown}{Modernen
                        Dichtung}{}\ledrightnote{\textcolor{brown}{Moderne Dichtung/Moderne Rundschau}}« betreffenden Zuſchriften und Sendungen wolle man an die
                     Redaktion: \textcolor{pink}{Wien}{}\ledrightnote{\textcolor{pink}{Wien}}, \textcolor{pink}{VIII., Buchfeldgasse 8}{}\ledrightnote{\textcolor{pink}{Buchfeldgasse}} (Sprechſtunden 2–4), alle auf die
                     Adminiſtration und Expedition bezüglichen Zuſchriften, Geldſendungen etc.
                     jedoch an den \textcolor{brown}{Verlag}{}\ledrightnote{\textcolor{brown}{Holzwarth {\kaufmannsund}  Ortony}}: \textcolor{pink}{Wien}{}\ledrightnote{\textcolor{pink}{Wien}}, \textcolor{pink}{IX.,
                        Liechtenſteinſtraße 3}{}\ledrightnote{\textcolor{pink}{Liechtensteinstraße}}, richten.}}}{\lemma{\textnormal{\emph{Alle … richten.}}}\Cendnote{\textnormal{quer am Rand der ersten Seite}}}\label{T_L00037_1h}\pend
           \endnumbering\briefempfaengerindex{Schnitzler, Arthur@\textsc{Schnitzler, Arthur}!zzzKafka, Eduard Michael@\emph{von Eduard Michael Kafka}!1891-08-301@{30. 8. 1891}|)be}\mylabel{h}  \normalsize

\doendnotes{C}
\bigskip
\vfill

\clearpage

\footnotesize

\lohead{\textsc{register}}

% Definiere theindex-Environment komplett neu ohne reledmac
\makeatletter
\renewenvironment{theindex}{%
  \section*{\indexname}%
  \setlength{\parindent}{0pt}%
  \setlength{\parskip}{0pt plus 0.3pt}%
  \let\item\@idxitem
}{%
  \clearpage
}
\makeatother

\IfFileExists{\jobname-pw.ind}{\input{\jobname-pw.ind}}{}

\end{document}

      