%% latex-korrekturansicht-vorspann.tex
%% Vorspann für die Korrekturansicht.
%% Lädt die gemeinsame Datei latex-vorspann.tex mit gesetztem Schalter.

\newif\ifkorrekturansicht
\korrekturansichttrue

\input{../tex-inputs/latex-vorspann}


               \section[Lou Andreas-Salomé an Arthur Schnitzler, {[}22. 4. 1901{]}]{ Lou Andreas-Salomé an Arthur Schnitzler,
                    {[}22. 4. 1901{]}}\nopagebreak\mylabel{v}\rehead{ }\normalsize\beginnumbering\briefempfaengerindex{Schnitzler, Arthur@\textsc{Schnitzler, Arthur}!zzzAndreas-Salome, Lou@\emph{von Lou Andreas-Salomé}!1901-04-221@{{[}22. 4. 1901{]}}|(be} \toendnotes[C]{\smallbreak\pagebreak[2]} \Standort{CUL, Schnitzler, B 3.}
\physDesc{Briefkarte
\newline{}Handschrift: schwarze Tinte, deutsche Kurrent
\newline{}Schnitzler: 1) mit Bleistift datiert: »22/4 901« 2) mit rotem Buntstift eine Unterstreichung\newline{}Ordnung: mit Bleistift von unbekannter Hand nummeriert:
                                        »18« }\toendnotes[C]{\smallbreak}\pstart{}{\pb}Lieber Herr Doktor,\pend\pstart
           ſehr freu ich mich darüber, Ihr neues \textcolor{green}{Buch}{}\ledrightnote{→\textcolor{green}{Frau Bertha Garlan. Roman}} von Ihnen zu empfangen, nachdem ich die
                    Bekanntſchaft mit Frau \textcolor{green}{\textsc{Bertha Garlan}}{}\ledrightnote{→\textcolor{green}{Frau Bertha Garlan. Roman}} und Frau \textcolor{green}{\textsc{Rupius}}{}\ledrightnote{→\textcolor{green}{Frau Bertha Garlan. Roman}} in der \label{K_L01111_1v}\edtext{\textcolor{green}{\textsc{N. D. Rundschau}}{}\ledrightnote{\textcolor{green}{Neue Deutsche Rundschau}}}{\lemma{\textnormal{\emph{N. D. Rundschau}}}\Cendnote{\textnormal{Nachdem \emph{\textcolor{green}{Frau Bertha Garlan}} in drei Teilen zwischen
                            Januar und März 1901 in der \emph{\textcolor{green}{Neuen Deutschen Rundschau}} erschienen
                        war, wurde die Buchausgabe Mitte April ausgeliefert (\emph{\textcolor{green}{Frau Bertha Garlan}}. Roman. Berlin:
                                \emph{\textcolor{brown}{S. Fischer}}{ }1901.)}}}\label{K_L01111_1h} gemacht habe. Um Frau \textcolor{green}{\textsc{Rupius}}{}\ledrightnote{→\textcolor{green}{Frau Bertha Garlan. Roman}} focht ich ſogar mit \textcolor{blue}{Frieda Bülow}{}\ledrightnote{\textcolor{blue}{Frieda von Bülow}}
                    einen großen Streit aus; ich hielt es mit Herrn \textcolor{green}{\textsc{Rupius}}{}\ledrightnote{→\textcolor{green}{Frau Bertha Garlan. Roman}}.\pend
           \pstart
           Hoffentlich geht es Ihnen drüben in \textcolor{pink}{Wien}{}\ledrightnote{\textcolor{pink}{Wien}}{ }ſo
                    gut, wie mir hier, wo ich zwar nur zur Hälfte bin, denn {\pb}am liebſten ſind mein \textcolor{blue}{Mann}{}\ledrightnote{→\textcolor{blue}{Friedrich Carl Andreas}} und ich in \textcolor{pink}{Rußland}{}\ledrightnote{\textcolor{pink}{Russland}} und reiſen auch demnächſt wieder
                    auf lange dorthin. Erſt ſeit ein paar Jahren kenne ich meine \textcolor{pink}{ruſſiſche}{}\ledrightnote{\textcolor{pink}{Russland}} Heimath in ihrem weitern Umkreis, mit ihren
                    Landſchaften und Menſchen; ſeitdem weiß ich erſt, daß ſie meine Heimath iſt, und
                    daß ich eigentlich dort lebe.\pend
           \pstart
           Herzlichen Gruß Ihnen allen!{\\[\baselineskip]}\spacefill\mbox{Frau Lou.}\pend
           \leftskip=0em{}\endnumbering\briefempfaengerindex{Schnitzler, Arthur@\textsc{Schnitzler, Arthur}!zzzAndreas-Salome, Lou@\emph{von Lou Andreas-Salomé}!1901-04-221@{{[}22. 4. 1901{]}}|)be}\mylabel{h}  \normalsize

\doendnotes{C}
\bigskip
\vfill

\clearpage

\footnotesize

\lohead{\textsc{register}}

% Definiere theindex-Environment komplett neu ohne reledmac
\makeatletter
\renewenvironment{theindex}{%
  \section*{\indexname}%
  \setlength{\parindent}{0pt}%
  \setlength{\parskip}{0pt plus 0.3pt}%
  \let\item\@idxitem
}{%
  \clearpage
}
\makeatother

\IfFileExists{\jobname-pw.ind}{\input{\jobname-pw.ind}}{}

\end{document}

      