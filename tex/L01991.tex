%% latex-korrekturansicht-vorspann.tex
%% Vorspann für die Korrekturansicht.
%% Lädt die gemeinsame Datei latex-vorspann.tex mit gesetztem Schalter.

\newif\ifkorrekturansicht
\korrekturansichttrue

\input{../tex-inputs/latex-vorspann}


               \section[Georg Brandes an Arthur Schnitzler, 18. 12. 1910]{ Georg Brandes an Arthur Schnitzler, 18. 12. 1910}\nopagebreak\mylabel{v}\rehead{ }\normalsize\beginnumbering\briefempfaengerindex{Schnitzler, Arthur@\textsc{Schnitzler, Arthur}!zzzBrandes, Georg@\emph{von Georg Brandes}!1910-12-181@{18. 12. 1910}|(be} \toendnotes[C]{\smallbreak\pagebreak[2]} \Standort{CUL, Schnitzler, B 17.}
\physDesc{Brief, 1 Blatt, 3 Seiten
\newline{}Handschrift: schwarze Tinte, lateinische Kurrent
\newline{}Schnitzler: mit Bleistift beschriftet: »\textsc{Brandes}« \newline{}Ordnung: mit Bleistift von unbekannter Hand nummeriert:
                                        »34« }\buchAbdrucke{\weitereDrucke{Georg Brandes, Arthur Schnitzler: \emph{Ein Briefwechsel}. Hg. Kurt Bergel. Bern: \emph{Francke} 1956, S. 98.} }\toendnotes[C]{\smallbreak}\pstart
           \raggedleft{}{\pb}\textcolor{pink}{\uline{Kopenhagen}}{}\ledrightnote{\textcolor{pink}{Kopenhagen}}{\\}18. 12. 10\pend
           \pstart{}Verehrter Freund\pend\pstart
           Wenn ich Sie lese, thut es mir leid, dass ich so weit von Ihnen wohne und so
                    selten Gelegenheit habe, mit Ihnen einige Worte zu wechseln.\pend
           \pstart
           \textcolor{green}{\uline{Medardus}}{}\ledrightnote{\textcolor{green}{Der junge Medardus. Dramatische Historie in einem Vorspiel und fünf Aufzügen}} habe ich sehr genau gelesen, laut vorgelesen, um es recht zu würdigen. Sie
                    haben dort ein reiches Bild aufgerollt. Mit Ueberraschung und Freude erfuhr ich
                    aus einer Zeitungs\label{K_L01991_1v}\edtext{notits}{\lemma{\textnormal{\emph{notits}}}\Cendnote{\textnormal{dänisch: Notiz}}}\label{K_L01991_1h}, dass das Stück
                    trotz seiner epischen Anlage erfolgreich aufgeführt worden ist. Die – im \textcolor{blue}{Goethe}{}\ledrightnote{\textcolor{blue}{Johann Wolfgang von Goethe}}schen Sinn über \textcolor{blue}{Kleist}{}\ledrightnote{\textcolor{blue}{Heinrich von Kleist}} – \strikeout{\textcolor{gray}{V}} fesselnde »\label{K_L01991_2v}\edtext{Verwirrung des
                        Gefühls}{\lemma{\textnormal{\emph{Verwirrung des
                        Gefühls}}}\Cendnote{\textnormal{Äußerung \textcolor{blue}{Goethe}s in seinem \textcolor{green}{Tagebuch}, 13. 7. 1807}}}\label{K_L01991_2h}« in
                        \textcolor{green}{Medardus}{}\ledrightnote{→\textcolor{green}{Der junge Medardus. Dramatische Historie in einem Vorspiel und fünf Aufzügen}} ist so recht
                    Ihre Domäne. {\pb}Sehr fein ist
                    die schwache Andeutung \strikeout{der} einer geistigen
                    Verwandtschaft zwischen \textcolor{green}{Helene}{}\ledrightnote{→\textcolor{green}{Der junge Medardus. Dramatische Historie in einem Vorspiel und fünf Aufzügen}} und \textcolor{green}{Napoleon}{}\ledrightnote{→\textcolor{green}{Der junge Medardus. Dramatische Historie in einem Vorspiel und fünf Aufzügen}}.\pend
           \pstart
           Die ganze \textcolor{pink}{Wien}{}\ledrightnote{\textcolor{pink}{Wien}}eratmosphäre vor 100 Jahren haben
                    Sie geben wollen. Und wenn ich nicht irre, lag es Ihnen besonders am Herzen, zu
                    zeigen, auf welchem Hintergrund von Spiessbürgerlichkeit und lässiger
                    Frivolität, die in \textcolor{pink}{Wien}{}\ledrightnote{\textcolor{pink}{Wien}} zu Hause \substVorne{}\textsuperscript{sind}\substDazwischen{}waren\substHinten{}, und auf welchem Hintergrund von unnationalem Wesen und Gehorsam dem
                    Eroberer gegenüber, die in \textcolor{pink}{Deutschland}{}\ledrightnote{\textcolor{pink}{Deutschland}}
                    hervortraten, der Heroismus einiger Weniger sich geltend machte. Eine
                    nachsichtige Menschenverachtung durchdringt das Schaupiel und findet u. a. in
                    mir ein Echo.\pend
           \pstart
           Ich möchte immer gerne wissen, wie es {\pb}Ihnen geht und wie es \textcolor{blue}{Beer-Hofmann}{}\ledrightnote{\textcolor{blue}{Richard Beer-Hofmann}} geht, den ich (vor 16 Jahren,
                    glaube ich) mit Ihnen kennen lernte.\pend
           \pstart
           Ueber mich selbst ist nichts Interessantes, wenigstens nichts besonders Gutes zu
                    melden. Ich bin nicht krank.\pend
           \pstart
           Haben Sie für die Treue Dank, womit Sie bei jeder neuen Arbeit auch an mich
                    denken.\pend
           \pstart
           Ich bin Ihr unveränderlicher Freund{\\[\baselineskip]}\spacefill\mbox{Georg Brandes}\pend
           \leftskip=0em{}\endnumbering\briefempfaengerindex{Schnitzler, Arthur@\textsc{Schnitzler, Arthur}!zzzBrandes, Georg@\emph{von Georg Brandes}!1910-12-181@{18. 12. 1910}|)be}\mylabel{h}  \normalsize

\doendnotes{C}
\bigskip
\vfill

\clearpage

\footnotesize

\lohead{\textsc{register}}

% Definiere theindex-Environment komplett neu ohne reledmac
\makeatletter
\renewenvironment{theindex}{%
  \section*{\indexname}%
  \setlength{\parindent}{0pt}%
  \setlength{\parskip}{0pt plus 0.3pt}%
  \let\item\@idxitem
}{%
  \clearpage
}
\makeatother

\IfFileExists{\jobname-pw.ind}{\input{\jobname-pw.ind}}{}

\end{document}

      