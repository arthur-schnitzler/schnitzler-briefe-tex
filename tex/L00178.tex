%% latex-korrekturansicht-vorspann.tex
%% Vorspann für die Korrekturansicht.
%% Lädt die gemeinsame Datei latex-vorspann.tex mit gesetztem Schalter.

\newif\ifkorrekturansicht
\korrekturansichttrue

\input{../tex-inputs/latex-vorspann}


               \section[Arthur Schnitzler an Hugo von Hofmannsthal, 18. 2. 1893]{ Arthur Schnitzler an Hugo von Hofmannsthal, 18. 2. 1893}\nopagebreak\mylabel{v}\rehead{ }\normalsize\beginnumbering\briefempfaengerindex{Hofmannsthal, Hugo von@\textsc{Hofmannsthal, Hugo von}!zzzSchnitzler, Arthur@\emph{von Arthur Schnitzler}!1893-02-181@{18. 2. 1893}|(be} \toendnotes[C]{\smallbreak\pagebreak[2]} \Standort{FDH, Hs-30885,34.}
\physDesc{Brief, 1 Blatt, 4 Seiten
\newline{}Handschrift: Bleistift, deutsche Kurrent\newline{}Ordnung: von Schnitzler mutmaßlich während der Durchsicht der Briefe 1929 mit Bleistift am oberen
                                    Blattrand zusätzlich datiert: »18/2 93« }\buchAbdrucke{\weitereDrucke{1) Hugo von Hofmannsthal, Arthur Schnitzler: \emph{Briefwechsel}. Hg. Therese Nickl und Heinrich Schnitzler. Frankfurt am Main: \emph{S. Fischer} 1964, S. 36.} \weitereDrucke{2) Hermann Bahr, Arthur Schnitzler: \emph{Briefwechsel, Aufzeichnungen, Dokumente
                                (1891–1931)}. Hg. Kurt Ifkovits und Martin Anton Müller. Göttingen: \emph{Wallstein} 2018.} }\toendnotes[C]{\smallbreak}\pstart{}{\pb}Lieber Hugo,\pend\pstart
           bitte leſen Sie \label{K_L00178_1v}\edtext{beiliegenden
                        Brief}{\lemma{\textnormal{\emph{beiliegenden
                        Brief}}}\Cendnote{\textnormal{Zwei Briefe \textcolor{blue}{Fels}’ aus dem \textcolor{pink}{Hotel
                            Erzherzog Rainer in Meran-Obermais} (\emph{Deutsches Literaturarchiv}, A:Schnitzler, 85.1.2956) sind mit
                            18. 2. 1893 datiert, wobei sich erschließen lässt, dass
                        einer am Tag vor dem anderen verfasst ist. Mit Bleistift wurde zum ersten
                        Datum »16«, zum zweiten »17« geschrieben. \textcolor{blue}{Schnitzler} dürfte \textcolor{blue}{Hofmannsthal}
                        den ersten mitteilen, der die Ankunft in \textcolor{pink}{Meran} schildert. Für die Rekonvaleszenz sind drei Monate
                        angesetzt, weswegen \textcolor{blue}{Fels} fürchtet, keine
                        Stelle bei der \emph{\textcolor{brown}{Deutschen Zeitung}} zu
                        bekommen.}}}\label{K_L00178_1h}. Und dann fragen Sie gütigſt \textcolor{blue}{Bahr}{}\ledrightnote{\textcolor{blue}{Hermann Bahr}}, wie die Ausſichten des Dr. \textcolor{blue}{\textsc{Fels}}{}\ledrightnote{\textcolor{blue}{Friedrich Michael Fels}} bei der \textcolor{brown}{Dtſch Ztg}{}\ledrightnote{\textcolor{brown}{Deutsche Zeitung}}{ }ſtehn, und wann er eintreffen müſſte. Es wäre
                    mir höchſt erwünſcht, darüber vollko{\geminationm}ene Klarheit
                    zu haben. Sie erſehen auch {\pb}weiters aus dem Brief,
                    daſs auf Ihre liebenswürdige Zuſage, eine neuerliche Sa{\geminationm}lg zu veranſtalten, reflectirt wird. Je früher
                    mir Ihre Reſultate in jeder Richtung bekannt werden, umſo dankbarer bin ich
                    Ihnen im Namen unſres Kranken.\pend
           \pstart
           – Wa{\geminationn} werden wir wieder einmal geſcheidte Dinge {\pb}miteinander ſprechen? Was machen Sie? Ich wäre ſehr
                    erfreut, wieder einmal mit Ihnen zusa{\geminationm}en zu ſein.
                    Ich bin jeden Abend nach 10 im \textcolor{pink}{Central}{}\ledrightnote{\textcolor{pink}{Café Central}},
                    Dienſtag, Donnerſtag, Samſtag ſicher. Den beigelegten Brief bitte mir mit Ihrer
                    frdl Antwort gef rückzuſenden.\pend
           \pstart
           {\pb}Herzlich der Ihre{\\[\baselineskip]}\spacefill\mbox{Arthur.}\pend
           \leftskip=0em{}\pstart
           \raggedleft{}18. 2. 93\pend
           \endnumbering\briefempfaengerindex{Hofmannsthal, Hugo von@\textsc{Hofmannsthal, Hugo von}!zzzSchnitzler, Arthur@\emph{von Arthur Schnitzler}!1893-02-181@{18. 2. 1893}|)be}\mylabel{h}  \normalsize

\doendnotes{C}
\bigskip
\vfill

\clearpage

\footnotesize

\lohead{\textsc{register}}

% Definiere theindex-Environment komplett neu ohne reledmac
\makeatletter
\renewenvironment{theindex}{%
  \section*{\indexname}%
  \setlength{\parindent}{0pt}%
  \setlength{\parskip}{0pt plus 0.3pt}%
  \let\item\@idxitem
}{%
  \clearpage
}
\makeatother

\IfFileExists{\jobname-pw.ind}{\input{\jobname-pw.ind}}{}

\end{document}

      