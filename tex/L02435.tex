%% latex-korrekturansicht-vorspann.tex
%% Vorspann für die Korrekturansicht.
%% Lädt die gemeinsame Datei latex-vorspann.tex mit gesetztem Schalter.

\newif\ifkorrekturansicht
\korrekturansichttrue

\input{../tex-inputs/latex-vorspann}


               \section[Gertrud Rung an Arthur Schnitzler, 1. 3. 1925]{ Gertrud Rung an Arthur Schnitzler, 1. 3. 1925}\nopagebreak\mylabel{v}\rehead{ }\normalsize\beginnumbering\briefempfaengerindex{Schnitzler, Arthur@\textsc{Schnitzler, Arthur}!zzzRung, Gertrud@\emph{von Gertrud Rung}!1925-03-011@{1. 3. 1925}|(be} \toendnotes[C]{\smallbreak\pagebreak[2]} \Standort{CUL, Schnitzler, B 17.}
\physDesc{Brief, 1 Blatt, 1 Seite
\newline{}Handschrift: schwarze Tinte, lateinische Kurrent
\newline{}Schnitzler: mit Bleistift beschriftet: »\textsc{Brandes}« \newline{}Ordnung: mit Bleistift von unbekannter Hand nummeriert:
                                        »57« }\buchAbdrucke{\weitereDrucke{Georg Brandes, Arthur Schnitzler: \emph{Ein Briefwechsel}. Hg. Kurt Bergel. Bern: \emph{Francke} 1956, S. 145.} }\pstart
           \centering{}{\pb}\textcolor{pink}{Kopenhagen}{}\ledrightnote{\textcolor{pink}{Kopenhagen}}{ }1 März 1925\pend
           \pstart{}Hochverehrter Herr Schnitzler.\pend\pstart
           Professor \textcolor{blue}{Brandes}{}\ledrightnote{\textcolor{blue}{Georg Brandes}} wird am 24
                        März, Abends, in \textcolor{pink}{Berlin}{}\ledrightnote{\textcolor{pink}{Berlin}}
                    eintreffen und hofft gleich einer der folgenden Tage die Freude zu haben sich
                    mit Ihnen zusammen zu treffen.\pend
           \pstart
           Mit den herzlichsten Grüßen des Professor \textcolor{blue}{Georg Brandes}{}\ledrightnote{\textcolor{blue}{Georg Brandes}}{\\[\baselineskip]} in Verehrung{\\[\baselineskip]}\spacefill\mbox{G. Rung Sekretär.}\pend
           \leftskip=0em{}\endnumbering\briefempfaengerindex{Schnitzler, Arthur@\textsc{Schnitzler, Arthur}!zzzRung, Gertrud@\emph{von Gertrud Rung}!1925-03-011@{1. 3. 1925}|)be}\mylabel{h}  \normalsize

\doendnotes{C}
\bigskip
\vfill

\clearpage

\footnotesize

\lohead{\textsc{register}}

% Definiere theindex-Environment komplett neu ohne reledmac
\makeatletter
\renewenvironment{theindex}{%
  \section*{\indexname}%
  \setlength{\parindent}{0pt}%
  \setlength{\parskip}{0pt plus 0.3pt}%
  \let\item\@idxitem
}{%
  \clearpage
}
\makeatother

\IfFileExists{\jobname-pw.ind}{\input{\jobname-pw.ind}}{}

\end{document}

      