%% latex-korrekturansicht-vorspann.tex
%% Vorspann für die Korrekturansicht.
%% Lädt die gemeinsame Datei latex-vorspann.tex mit gesetztem Schalter.

\newif\ifkorrekturansicht
\korrekturansichttrue

\input{../tex-inputs/latex-vorspann}


               \section[Richard Beer-Hofmann an Arthur Schnitzler, 8. 3. 1908]{ Richard Beer-Hofmann an Arthur Schnitzler, 8. 3. 1908}\nopagebreak\mylabel{v}\rehead{ }\normalsize\beginnumbering\briefempfaengerindex{Schnitzler, Arthur@\textsc{Schnitzler, Arthur}!zzzBeer-Hofmann, Richard@\emph{von Richard Beer-Hofmann}!1908-03-081@{8. 3. 1908}|(be} \toendnotes[C]{\smallbreak\pagebreak[2]} \Standort{CUL, Schnitzler, B 8.}
\physDesc{Bildpostkarte
\newline{}Handschrift: schwarze Tinte, lateinische Kurrent\newline{}Versand: Stempel: »\nobreak{}\oindex{Berlin@\textbf{Berlin}, \emph{https://www.geonames.org/ontologyP.PPLC}|pwk}Berlin 64, 8. 3. 08, 7–8\nobreak{}«.  
\newline{}Schnitzler: mit Bleistift datiert: »8. 3. 08« \newline{}Ordnung: 1) mit Bleistift von unbekannter Hand nummeriert: »\strikeout{217}« 2) mit Bleistift von unbekannter Hand nummeriert: »215«}\pstart{}{\pb}Herrn\pend{}\pstart{}Arthur Schnitzler\pend{}\pstart{}\textcolor{pink}{\textsc{Wien}}{}\ledrightnote{\textcolor{pink}{Wien}}\pend{}\pstart{}\textcolor{pink}{XVIII Spöttelgasse 7}{}\ledrightnote{\textcolor{pink}{Edmund-Weiß-Gasse}}.
               \pend{}{\bigskip}\pstart
           \noindent{}\centering{}{\pb}\textcolor{gray}{\textbf{\textbf{Gruss aus \textcolor{pink}{Berlin}{}\ledrightnote{\textcolor{pink}{Berlin}}}}}\pend
           \pstart
           \noindent{}\centering{}\textcolor{gray}{\textbf{Der neue \textcolor{pink}{Dom}{}\ledrightnote{\textcolor{pink}{Berliner Dom}} mit der \textcolor{pink}{Friedrichsbrücke}{}\ledrightnote{\textcolor{pink}{Friedrichsbrücke}}}}\pend
           \pstart
           \noindent{}\centering{}\textcolor{gray}{\textbf{\textcolor{pink}{\so{Börse}}{}\ledrightnote{\textcolor{pink}{Börse}}}}\pend
           \pstart
           \noindent{}{\pb}Herzliche Grüsse!\pend
           \pstart \spacefill\mbox{Richard.}\pend{}\endnumbering\briefempfaengerindex{Schnitzler, Arthur@\textsc{Schnitzler, Arthur}!zzzBeer-Hofmann, Richard@\emph{von Richard Beer-Hofmann}!1908-03-081@{8. 3. 1908}|)be}\mylabel{h}  \normalsize

\doendnotes{C}
\bigskip
\vfill

\clearpage

\footnotesize

\lohead{\textsc{register}}

% Definiere theindex-Environment komplett neu ohne reledmac
\makeatletter
\renewenvironment{theindex}{%
  \section*{\indexname}%
  \setlength{\parindent}{0pt}%
  \setlength{\parskip}{0pt plus 0.3pt}%
  \let\item\@idxitem
}{%
  \clearpage
}
\makeatother

\IfFileExists{\jobname-pw.ind}{\input{\jobname-pw.ind}}{}

\end{document}

      