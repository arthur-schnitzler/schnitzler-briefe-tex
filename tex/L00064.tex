%% latex-korrekturansicht-vorspann.tex
%% Vorspann für die Korrekturansicht.
%% Lädt die gemeinsame Datei latex-vorspann.tex mit gesetztem Schalter.

\newif\ifkorrekturansicht
\korrekturansichttrue

\input{../tex-inputs/latex-vorspann}


               \section[Hugo von Hofmannsthal an Arthur Schnitzler, 18. 1. 1892]{ Hugo von Hofmannsthal an Arthur Schnitzler, 18. 1. 1892}\nopagebreak\mylabel{v}\rehead{ }\normalsize\beginnumbering\briefempfaengerindex{Schnitzler, Arthur@\textsc{Schnitzler, Arthur}!zzzHofmannsthal, Hugo von@\emph{von Hugo von Hofmannsthal}!1892-01-182@{18. 1. 1892}|(be} \toendnotes[C]{\smallbreak\pagebreak[2]} \Standort{CUL, Schnitzler, B 43.}
\physDesc{Postkarte
\newline{}Handschrift: Bleistift, deutsche Kurrent\newline{}Versand: 1) Stempel: »\nobreak{}Wien 3/1, 18. 1. 92, 1–2V\nobreak{}«.  2) Stempel: »\nobreak{}Wien Kärntnerring, 18. 1. 92, 1–2N\nobreak{}«. 
\newline{}Schnitzler: mit Bleistift auf der Text- und der Anschriftenseite datiert: »18/1 92« \newline{}Ordnung: mit Bleistift von unbekannter Hand nummeriert:
                                        »16« }\buchAbdrucke{\weitereDrucke{1) Hugo von Hofmannsthal: \emph{Briefe. 1890–1901}. Berlin: \emph{S. Fischer} 1935, S. 17.} \weitereDrucke{2) Hugo von Hofmannsthal, Arthur Schnitzler: \emph{Briefwechsel}. Hg. Therese Nickl und Heinrich Schnitzler. Frankfurt am Main: \emph{S. Fischer} 1964, S. 15.} }\toendnotes[C]{\smallbreak}\pstart{}{\pb}Herrn \textsc{D\textsuperscript{r} Arthur Schnitzler}\pend{}\pstart{}\textsc{\textcolor{pink}{I Kärnthnerring 12}{}\ledrightnote{\textcolor{pink}{Kärntnerring}}}\pend{}\pstart{}\textsc{\textcolor{pink}{Wien}{}\ledrightnote{\textcolor{pink}{Wien}}}\pend{}\pstart{}\textsc{2 Stiege 3 Stock}\pend{}{\bigskip}\pstart{}{\pb}Geſchätzter
                        Herr.\pend\pstart
           \label{K_L00064_1v}\edtext{Dienſtag}{\lemma{\textnormal{\emph{Dienſtag}}}\Cendnote{\textnormal{der 19. 1. 1892}}}\label{K_L00064_1h} um 12 Uhr bin
                    ich ſehr natürlich in der \textcolor{pink}{Schule}{}\ledrightnote{→\textcolor{pink}{Akademisches Gymnasium}}, dann mache
                    ich Aufgaben und von 3–4 habe ich Deutſchſtunde. Aber
                        Mittwoch um ½ 1 möchte ich ins \textcolor{pink}{\textsc{Hotel Kummer}}{}\ledrightnote{\textcolor{pink}{Hotel Kummer}} kommen können. Wenn Sie mir nicht mehr antworten, betrachte ich
                    dieſen Antrag als abgelehnt und komme erſt \textsc{Freitag}{ }2 Uhr zu \textcolor{blue}{\textsc{Bératon}}{}\ledrightnote{\textcolor{blue}{Ferry Bératon}}{ }ſitzen.\pend
           \pstart \spacefill\mbox{Loris}\pend{}\endnumbering\briefempfaengerindex{Schnitzler, Arthur@\textsc{Schnitzler, Arthur}!zzzHofmannsthal, Hugo von@\emph{von Hugo von Hofmannsthal}!1892-01-182@{18. 1. 1892}|)be}\mylabel{h}  \normalsize

\doendnotes{C}
\bigskip
\vfill

\clearpage

\footnotesize

\lohead{\textsc{register}}

% Definiere theindex-Environment komplett neu ohne reledmac
\makeatletter
\renewenvironment{theindex}{%
  \section*{\indexname}%
  \setlength{\parindent}{0pt}%
  \setlength{\parskip}{0pt plus 0.3pt}%
  \let\item\@idxitem
}{%
  \clearpage
}
\makeatother

\IfFileExists{\jobname-pw.ind}{\input{\jobname-pw.ind}}{}

\end{document}

      