%% latex-korrekturansicht-vorspann.tex
%% Vorspann für die Korrekturansicht.
%% Lädt die gemeinsame Datei latex-vorspann.tex mit gesetztem Schalter.

\newif\ifkorrekturansicht
\korrekturansichttrue

\input{../tex-inputs/latex-vorspann}


               \section[Hermann Bahr an Arthur Schnitzler, 5. 2. {[}1896{]}]{ Hermann Bahr an Arthur Schnitzler, 5. 2. {[}1896{]}}\nopagebreak\mylabel{v}\rehead{ }\normalsize\beginnumbering\briefempfaengerindex{Schnitzler, Arthur@\textsc{Schnitzler, Arthur}!zzzBahr, Hermann@\emph{von Hermann Bahr}!1896-02-051@{5. 2. {[}1896{]}}|(be} \toendnotes[C]{\smallbreak\pagebreak[2]} \Standort{CUL, Schnitzler, B 5b.}
\physDesc{Brief, 1 Blatt, 2 Seiten
\newline{}Handschrift: schwarze Tinte, deutsche Kurrent\newline{}Ordnung: mit Bleistift von unbekannter Hand nummeriert: »35« }\buchAbdrucke{\weitereDrucke{Hermann Bahr, Arthur Schnitzler: \emph{Briefwechsel, Aufzeichnungen, Dokumente (1891–1931)}. Hg. Kurt Ifkovits und Martin Anton Müller. Göttingen: \emph{Wallstein} 2018, S. 116.} }\toendnotes[C]{\smallbreak}\pstart
           \noindent{}{\pb}\textcolor{gray}{\textbf{»\textcolor{brown}{Die Zeit}{}\ledrightnote{\textcolor{brown}{Die Zeit. Wiener Wochenschrift}}«}}\hfill \textcolor{gray}{\textbf{\textbf{\textcolor{pink}{Wien}{}\ledrightnote{\textcolor{pink}{Wien}}}, den }}5. Febr. \textcolor{gray}{\textbf{189}}\pend
           \pstart
           \textcolor{gray}{\textbf{Wiener Wochenſchrift}}\hfill \textcolor{gray}{\textbf{\textcolor{pink}{IX/3, Günthergaſſe 1}{}\ledrightnote{\textcolor{pink}{Günthergasse}}.}}\pend
           \pstart
           \textcolor{gray}{\textbf{\textbf{Herausgeber}:}}{\\}\textcolor{gray}{\textbf{Profeſſor Dr. \textcolor{blue}{I. Singer}{}\ledrightnote{\textcolor{blue}{Isidor Singer}},
                        \textcolor{blue}{Hermann Bahr}{}\ledrightnote{\textcolor{blue}{Hermann Bahr}}, Dr. \textcolor{blue}{Heinrich Kanner}{}\ledrightnote{\textcolor{blue}{Heinrich Kanner}}.}}\pend
           \pstart
           \textcolor{gray}{\textbf{Telephon Nr. 6415.}}\pend
           \pstart{}Lieber Arthur!\pend\pstart
           Vor allem meine herzlichſten und wärmſten Glückwünſche dazu, daß Du nun auch in \textcolor{pink}{Berlin}{}\ledrightnote{\textcolor{pink}{Berlin}} denſelben großen \label{K_L00532_1v}\edtext{Erfolg}{\lemma{\textnormal{\emph{Erfolg}}}\Cendnote{\textnormal{\emph{\textcolor{green}{Liebelei}} wurde am 4. 2. 1896 zum ersten Mal in der Inszenierung von \textcolor{blue}{Brahm} am \textcolor{pink}{Deutschen Theater}
                  gegeben.}}}\label{K_L00532_1h} gehabt haſt, wie \substVorne{}\textsuperscript{ſchon }{\allowbreak}\substDazwischen{}überall\substHinten{}.\pend
           \pstart
           Ferner theile ich Dir mit, daß \textcolor{blue}{Langkammer}{}\ledrightnote{\textcolor{blue}{Karl Langkammer}} für das
                  »\textcolor{green}{Märchen}{}\ledrightnote{\textcolor{green}{Das Märchen. Schauspiel in drei Aufzügen}}« begeiſtert iſt, bei der \label{K_L00532_2v}\edtext{neuen Faſſung}{\lemma{\textnormal{\emph{neuen Faſſung}}}\Cendnote{\textnormal{Die Buchausgabe von
                     1894 weicht von der Textvorlage der Uraufführung ab.}}}\label{K_L00532_2h} (und
               einigen geringfügigen Aenderungen) einen Erfolg für ſicher hält und die Aufführung
               des Stückes beim {\pb}Direktionsrath gleich nach der
                  Generalverſa{\geminationm}lung \label{K_L00532_3v}\edtext{beantragen wird}{\lemma{\textnormal{\emph{beantragen wird}}}\Cendnote{\textnormal{Am 7. 9. 1896
                  retourniert \textcolor{blue}{Langkammer} das Drama, die
                  Inszenierung findet nicht statt.}}}\label{K_L00532_3h}. Vorher will er es nicht, weil einer der
               Hauptpunkte gegen \textcolor{blue}{Müller}{}\ledrightnote{\textcolor{blue}{Leopold Müller}} die Überladung des \textcolor{pink}{Theaters}{}\ledrightnote{→\textcolor{pink}{Raimund-Theater}} mit ſchon aufgeführten
               Stücken iſt.\pend
           \pstart
           Herzlich{\\[\baselineskip]}Dein treuer{\\[\baselineskip]}\spacefill\mbox{HermannB}\pend
           \leftskip=0em{}\pstart
           \textcolor{gray}{\textbf{\label{T_L00532_1v}\edtext{Alle für »\textcolor{brown}{Die Zeit}{}\ledrightnote{\textcolor{brown}{Die Zeit. Wiener Wochenschrift}}« beſtimmten Zuſchriften und Sendungen ſind an die Redaction der »\textcolor{brown}{Zeit}{}\ledrightnote{\textcolor{brown}{Die Zeit. Wiener Wochenschrift}}« und \textbf{nicht} an die Perſon eines der Herausgeber zu richten.}{\lemma{\textnormal{\emph{Alle … richten.}}}\Cendnote{\textnormal{am unteren Rand der
                     ersten Seite}}}\label{T_L00532_1h}}}\pend
           \endnumbering\briefempfaengerindex{Schnitzler, Arthur@\textsc{Schnitzler, Arthur}!zzzBahr, Hermann@\emph{von Hermann Bahr}!1896-02-051@{5. 2. {[}1896{]}}|)be}\mylabel{h}  \normalsize

\doendnotes{C}
\bigskip
\vfill

\clearpage

\footnotesize

\lohead{\textsc{register}}

% Definiere theindex-Environment komplett neu ohne reledmac
\makeatletter
\renewenvironment{theindex}{%
  \section*{\indexname}%
  \setlength{\parindent}{0pt}%
  \setlength{\parskip}{0pt plus 0.3pt}%
  \let\item\@idxitem
}{%
  \clearpage
}
\makeatother

\IfFileExists{\jobname-pw.ind}{\input{\jobname-pw.ind}}{}

\end{document}

      