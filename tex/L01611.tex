%% latex-korrekturansicht-vorspann.tex
%% Vorspann für die Korrekturansicht.
%% Lädt die gemeinsame Datei latex-vorspann.tex mit gesetztem Schalter.

\newif\ifkorrekturansicht
\korrekturansichttrue

\input{../tex-inputs/latex-vorspann}


               \section[Richard Beer-Hofmann an Arthur Schnitzler, 13. 7. 1906]{ Richard Beer-Hofmann an Arthur Schnitzler, 13. 7. 1906}\nopagebreak\mylabel{v}\rehead{ }\normalsize\beginnumbering\briefempfaengerindex{Schnitzler, Arthur@\textsc{Schnitzler, Arthur}!zzzBeer-Hofmann, Richard@\emph{von Richard Beer-Hofmann}!1906-07-132@{13. 7. 1906}|(be} \toendnotes[C]{\smallbreak\pagebreak[2]} \Standort{CUL, Schnitzler, B 8.}
\physDesc{Bildpostkarte
\newline{}Handschrift: Bleistift, lateinische Kurrent\newline{}Versand: 1) Stempel: »\nobreak{}\oindex{Rodaun@\textbf{Rodaun}, \emph{Teil eines besiedelten Ortes (A.BSOX)}|pwk}{[}Roda{]}un\nobreak{}«.  2) Stempel: »\nobreak{}\oindex{Helsingør@\textbf{Helsingør}, \emph{Besiedelter Ort (A.BSO)}|pwk}Helsingør, 16. 7. 06, 10-11E\nobreak{}«. \newline{}Ordnung: mit Bleistift von unbekannter Hand nummeriert: »206« }\buchAbdrucke{\weitereDrucke{Arthur Schnitzler, Richard Beer-Hofmann: \emph{Briefwechsel 1891–1931}. Hg. Konstanze Fliedl. Wien, Zürich: \emph{Europaverlag} 1992, S. 179.} }\toendnotes[C]{\smallbreak}\pstart{}{\pb}Herrn\pend{}\pstart{}Dr Arthur Schnitzler\pend{}\pstart{}\textcolor{pink}{Marienlyst}{}\ledrightnote{\textcolor{pink}{Marienlyst}}\pend{}\pstart{}\textcolor{pink}{Kurhaus}{}\ledrightnote{\textcolor{pink}{Kurhotellet}}\pend{}\pstart{}\textcolor{pink}{Dänemark}{}\ledrightnote{\textcolor{pink}{Dänemark}}\pend{}{\bigskip}\pstart
           \noindent{}\centering{}{\pb}\textcolor{gray}{\textbf{\textcolor{pink}{Kirche und Dreifaltigkeitssäule}{}\ledrightnote{\textcolor{pink}{Stift Heiligenkreuz}}.}}\hspace*{1.5em}\textcolor{gray}{\textbf{Gruss aus \textcolor{pink}{Heiligenkreuz}{}\ledrightnote{\textcolor{pink}{Heiligenkreuz}} im \textcolor{pink}{Wienerwald}{}\ledrightnote{\textcolor{pink}{Wienerwald}}.}}\pend
           \pstart
           \raggedleft{}13/VII 06\pend
           \stanza{}»Lang ist es her«\newverse{}spielt ein Kind\newverse{}auf dem versti{\geminationm}ten Klavier\newverse{}\textcolor{green}{»\label{K_L01611_1v}\edtext{Es ist ein
                     alter Klimperkasten}{\lemma{\textnormal{\emph{Es … Klimperkasten}}}\Cendnote{\textnormal{Es handelt sich um die letzten beiden Verse von \textcolor{blue}{Schnitzler}s Gedicht \emph{\textcolor{green}{Am Flügel}} mit der Textabweichung »alter« statt dem in
                        der ersten gedruckten Fassung stehenden »dummer«.}}}\label{K_L01611_1h}}{}\ledrightnote{→\textcolor{green}{Am Flügel}}\newverse{}\textcolor{green}{wahrscheinlich war er’s damals
                     schon!«}{}\ledrightnote{→\textcolor{green}{Am Flügel}}\stanzaend{}\pstart \spacefill\mbox{Richard}\pend{}\endnumbering\briefempfaengerindex{Schnitzler, Arthur@\textsc{Schnitzler, Arthur}!zzzBeer-Hofmann, Richard@\emph{von Richard Beer-Hofmann}!1906-07-132@{13. 7. 1906}|)be}\mylabel{h}  \normalsize

\doendnotes{C}
\bigskip
\vfill

\clearpage

\footnotesize

\lohead{\textsc{register}}

% Definiere theindex-Environment komplett neu ohne reledmac
\makeatletter
\renewenvironment{theindex}{%
  \section*{\indexname}%
  \setlength{\parindent}{0pt}%
  \setlength{\parskip}{0pt plus 0.3pt}%
  \let\item\@idxitem
}{%
  \clearpage
}
\makeatother

\IfFileExists{\jobname-pw.ind}{\input{\jobname-pw.ind}}{}

\end{document}

      