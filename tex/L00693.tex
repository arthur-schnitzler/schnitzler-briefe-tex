%% latex-korrekturansicht-vorspann.tex
%% Vorspann für die Korrekturansicht.
%% Lädt die gemeinsame Datei latex-vorspann.tex mit gesetztem Schalter.

\newif\ifkorrekturansicht
\korrekturansichttrue

\input{../tex-inputs/latex-vorspann}


               \section[Hugo von Hofmannsthal an Arthur Schnitzler, 6. 7. {[}1897{]}]{ Hugo von Hofmannsthal an Arthur Schnitzler, 6. 7. {[}1897{]}}\nopagebreak\mylabel{v}\rehead{ }\normalsize\beginnumbering\briefempfaengerindex{Schnitzler, Arthur@\textsc{Schnitzler, Arthur}!zzzHofmannsthal, Hugo von@\emph{von Hugo von Hofmannsthal}!1897-07-061@{6. 7. {[}1897{]}}|(be} \toendnotes[C]{\smallbreak\pagebreak[2]} \Standort{CUL, Schnitzler, B 43.}
\physDesc{Brief, 1 Blatt (gedrucktes Wappen in blauer Farbe), 3 Seiten
\newline{}Handschrift: schwarze Tinte, deutsche Kurrent
\newline{}Schnitzler: mit Bleistift die Jahreszahl ergänzt: »97« \newline{}Ordnung: mit Bleistift von unbekannter Hand nummeriert:
                                        »92« }\buchAbdrucke{\weitereDrucke{Hugo von Hofmannsthal, Arthur Schnitzler: \emph{Briefwechsel}. Hg. Therese Nickl und Heinrich Schnitzler. Frankfurt am Main: \emph{S. Fischer} 1964, S. 88.} }\toendnotes[C]{\smallbreak}\pstart
           \raggedleft{}{\pb}\textcolor{pink}{Bad Fuſch}{}\ledrightnote{\textcolor{pink}{Bad Fusch}}, 6. July.\pend
           \pstart{}mein lieber Arthur,\pend\pstart
           ich lebe ſehr ſtill und recht zufrieden, verſuche hie und da Verſe zu machen und
                    komme mir merkwürdig unſicher und entwöhnt vor, ſchmiere an meiner \textcolor{green}{Doctorsarbeit}{}\ledrightnote{→\textcolor{green}{Über den Sprachgebrauch bei den Dichtern der Pléjade}} und finde daſs
                        »\textcolor{green}{Fauſt}{}\ledrightnote{\textcolor{green}{Faust}}« von \textcolor{blue}{Goethe}{}\ledrightnote{\textcolor{blue}{Johann Wolfgang von Goethe}} ein ſehr angenehmes Buch iſt, in welchem das Schöne und das
                    Kluge wundervoll ineinander aufgehen, was man denn wohl heitere {\pb}Weisheit nennen kann. Anders
                    wieder die \textcolor{green}{italieniſche Reiſe}{}\ledrightnote{\textcolor{green}{Italienische Reise}}, die einem einen
                    guten Begriff von der Friſche und kraftvollen Naivetät eines drei- oder
                    vierundvierzigjährigen Menſchen geben kann.\pend
           \pstart
           Die \textcolor{green}{\textcolor{blue}{Mozart}{}\ledrightnote{\textcolor{blue}{Wolfgang Amadeus Mozart}}biographie}{}\ledrightnote{→\textcolor{green}{W. A. Mozart}} enthält viel
                    weniger menſchliches, als ich erwartet hätte, zumindeſt in dieſem Theil; nur
                    hübſche kindiſche Briefe aus \textcolor{pink}{Italien}{}\ledrightnote{\textcolor{pink}{Italien}}.
                    Vielleicht ſchicken Sie mir gelegentlich hieher den 2\textsuperscript{ten} Band, ich Ihnen {\pb}den erſten. Denn nach \textcolor{pink}{Salzburg}{}\ledrightnote{\textcolor{pink}{Salzburg}} ko{\geminationm} ich nur mit einem ſehr kleinen Koffer. Daſs mir
                        \textcolor{blue}{Richard}{}\ledrightnote{\textcolor{blue}{Richard Beer-Hofmann}} abſolut nicht ſchreibt, bedeutet
                    doch wohl nichts beſonderes, am wenigſten daſs er viel arbeitet?\pend
           \pstart
           Ich wäre ſehr froh über einige Nachricht von Euch beiden.\pend
           \pstart
           Herzlich der Ihre{\\[\baselineskip]}\spacefill\mbox{Hugo.}\pend
           \leftskip=0em{}\endnumbering\briefempfaengerindex{Schnitzler, Arthur@\textsc{Schnitzler, Arthur}!zzzHofmannsthal, Hugo von@\emph{von Hugo von Hofmannsthal}!1897-07-061@{6. 7. {[}1897{]}}|)be}\mylabel{h}  \normalsize

\doendnotes{C}
\bigskip
\vfill

\clearpage

\footnotesize

\lohead{\textsc{register}}

% Definiere theindex-Environment komplett neu ohne reledmac
\makeatletter
\renewenvironment{theindex}{%
  \section*{\indexname}%
  \setlength{\parindent}{0pt}%
  \setlength{\parskip}{0pt plus 0.3pt}%
  \let\item\@idxitem
}{%
  \clearpage
}
\makeatother

\IfFileExists{\jobname-pw.ind}{\input{\jobname-pw.ind}}{}

\end{document}

      