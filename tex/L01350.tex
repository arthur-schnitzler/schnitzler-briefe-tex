%% latex-korrekturansicht-vorspann.tex
%% Vorspann für die Korrekturansicht.
%% Lädt die gemeinsame Datei latex-vorspann.tex mit gesetztem Schalter.

\newif\ifkorrekturansicht
\korrekturansichttrue

\input{../tex-inputs/latex-vorspann}


               \section[Arthur Schnitzler an Hermann Bahr, 13. 12. 1903]{ Arthur Schnitzler an Hermann Bahr, 13. 12. 1903}\nopagebreak\mylabel{v}\rehead{ }\normalsize\beginnumbering\briefempfaengerindex{Bahr, Hermann@\textsc{Bahr, Hermann}!zzzSchnitzler, Arthur@\emph{von Arthur Schnitzler}!1903-12-131@{13. 12. 1903}|(be} \toendnotes[C]{\smallbreak\pagebreak[2]} \Standort{TMW, HS AM 23362 Ba.}
\physDesc{Telegramm
\newline{}maschinell\newline{}Versand: 1) mit schwarzer Tinte von »\textcolor{blue}{Schott}« signiert und mit weiterer Empfängeradresse versehen: »\textcolor{pink}{N.W.7 Hotel de Rom} zu
                                    bestellen« 2) Stempel: »\nobreak{}\oindex{Berlin@\textbf{Berlin}, \emph{https://www.geonames.org/ontologyP.PPLC}|pwk}Berlin N. W. 6, 13. 12. 03., 12\textsuperscript{20}\nobreak{}«. 3) Stempel: »\nobreak{}Ausgefertigt, 13 Dec. {[}1903{]}\nobreak{}«. 4) »\textcolor{gray}{\textbf{\textbf{Aufgenommen} von}} W \textcolor{gray}{\textbf{den}}{ }13\textcolor{gray}{\textbf{/}}12 um 11 \textcolor{gray}{\textbf{Uhr}} 57 \textcolor{gray}{\textbf{M.}}
                                       m{ }\textcolor{gray}{\textbf{durch}}{ }\textcolor{gray}{MW}«}\buchAbdrucke{\weitereDrucke{1) \emph{13. 12. 1903.} In: Arthur Schnitzler: \emph{The Letters of Arthur Schnitzler to Hermann Bahr}. Edited, annotated, and with an introduction, by Donald G.
                        Daviau. Chapel Hill: \emph{The University of North Carolina Press} 1978, S. 82 (University of North Carolina studies in the Germanic languages
                        and literatures, 89).} \weitereDrucke{2) Hermann Bahr, Arthur Schnitzler: \emph{Briefwechsel, Aufzeichnungen, Dokumente (1891–1931)}. Hg. Kurt Ifkovits und Martin Anton Müller. Göttingen: \emph{Wallstein} 2018, S. 284.} }\toendnotes[C]{\smallbreak}\pstart{}{\pb}hermann bahr \textcolor{pink}{berlin deutsches theater}{}\ledrightnote{\textcolor{pink}{Deutsches Theater Berlin}}\pend{}{\bigskip}\pstart
           \noindent{}{\pb}\textcolor{gray}{\textbf{Telegramm}} fr \textcolor{pink}{wien}{}\ledrightnote{\textcolor{pink}{Wien}} 110+ 466
                  12 13{ }11 m{ }\textcolor{gray}{\textbf{W.}}{ }\textcolor{gray}{\textbf{190}}3\pend
           \pstart
           herzlichen \label{K_L01350_1v}\edtext{glueckwunsch}{\lemma{\textnormal{\emph{glueckwunsch}}}\Cendnote{\textnormal{am 12. 12. 1903 Uraufführung
                  von \emph{\textcolor{green}{Der Meister}} im \textcolor{pink}{Deutschen Theater} in \textcolor{pink}{Berlin}.}}}\label{K_L01350_1h} und
               gruss dein arthur schnitzler\pend
           \endnumbering\briefempfaengerindex{Bahr, Hermann@\textsc{Bahr, Hermann}!zzzSchnitzler, Arthur@\emph{von Arthur Schnitzler}!1903-12-131@{13. 12. 1903}|)be}\mylabel{h}  \normalsize

\doendnotes{C}
\bigskip
\vfill

\clearpage

\footnotesize

\lohead{\textsc{register}}

% Definiere theindex-Environment komplett neu ohne reledmac
\makeatletter
\renewenvironment{theindex}{%
  \section*{\indexname}%
  \setlength{\parindent}{0pt}%
  \setlength{\parskip}{0pt plus 0.3pt}%
  \let\item\@idxitem
}{%
  \clearpage
}
\makeatother

\IfFileExists{\jobname-pw.ind}{\input{\jobname-pw.ind}}{}

\end{document}

      