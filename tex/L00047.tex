%% latex-korrekturansicht-vorspann.tex
%% Vorspann für die Korrekturansicht.
%% Lädt die gemeinsame Datei latex-vorspann.tex mit gesetztem Schalter.

\newif\ifkorrekturansicht
\korrekturansichttrue

\input{../tex-inputs/latex-vorspann}


               \section[Arthur Schnitzler an Richard Beer-Hofmann, 5. 12. 1891]{ Arthur Schnitzler an Richard Beer-Hofmann, 5. 12. 1891}\nopagebreak\mylabel{v}\rehead{ }\normalsize\beginnumbering\briefempfaengerindex{Beer-Hofmann, Richard@\textsc{Beer-Hofmann, Richard}!zzzSchnitzler, Arthur@\emph{von Arthur Schnitzler}!1891-12-051@{5. 12. 1891}|(be} \toendnotes[C]{\smallbreak\pagebreak[2]} \Standort{YCGL, MSS 31.}
\physDesc{Postkarte
\newline{}Handschrift: Bleistift, deutsche Kurrent\newline{}Versand: 1) Stempel: »\nobreak{}Margarethen Wien, 5/12 91, 5 A\nobreak{}«.  2) Stempel: »\nobreak{}Wien 3/2, 6-12 91, 8 12 V, Bestellt\nobreak{}«. }\toendnotes[C]{\smallbreak}\pstart{}{\pb}\textsc{Hrn Dr Rich. Beer Hofma{\geminationn}}\pend{}\pstart{}\textsc{\textcolor{pink}{Wien}{}\ledrightnote{\textcolor{pink}{Wien}}}\pend{}\pstart{}\textsc{\textcolor{pink}{III. Seidlgasse 30}{}\ledrightnote{\textcolor{pink}{Seidlgasse}}.}\pend{}{\bigskip}\pstart
           \noindent{}{\pb}Lieber Richard, morgen So{\geminationn}tag{ }3 ½ Uhr bei mir.
               Ja? Die \textcolor{blue}{andern}{}\ledrightnote{→\textcolor{blue}{Hugo von Hofmannsthal}{\newline}→\textcolor{blue}{Felix Salten}} kommen ſicher \pend
           \pstart
           Herzlichſt Ihr{\\[\baselineskip]}\spacefill\mbox{Arth Schn}\pend
           \leftskip=0em{}\endnumbering\briefempfaengerindex{Beer-Hofmann, Richard@\textsc{Beer-Hofmann, Richard}!zzzSchnitzler, Arthur@\emph{von Arthur Schnitzler}!1891-12-051@{5. 12. 1891}|)be}\mylabel{h}  \normalsize

\doendnotes{C}
\bigskip
\vfill

\clearpage

\footnotesize

\lohead{\textsc{register}}

% Definiere theindex-Environment komplett neu ohne reledmac
\makeatletter
\renewenvironment{theindex}{%
  \section*{\indexname}%
  \setlength{\parindent}{0pt}%
  \setlength{\parskip}{0pt plus 0.3pt}%
  \let\item\@idxitem
}{%
  \clearpage
}
\makeatother

\IfFileExists{\jobname-pw.ind}{\input{\jobname-pw.ind}}{}

\end{document}

      