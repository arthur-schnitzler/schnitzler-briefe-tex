%% latex-korrekturansicht-vorspann.tex
%% Vorspann für die Korrekturansicht.
%% Lädt die gemeinsame Datei latex-vorspann.tex mit gesetztem Schalter.

\newif\ifkorrekturansicht
\korrekturansichttrue

\input{../tex-inputs/latex-vorspann}


               \section[Hermann Bahr an Arthur Schnitzler, 21. 5. 1903]{ Hermann Bahr an Arthur Schnitzler, 21. 5. 1903}\nopagebreak\mylabel{v}\rehead{ }\normalsize\beginnumbering\briefempfaengerindex{Schnitzler, Arthur@\textsc{Schnitzler, Arthur}!zzzBahr, Hermann@\emph{von Hermann Bahr}!1903-05-211@{21. 5. 1903}|(be} \toendnotes[C]{\smallbreak\pagebreak[2]} \Standort{CUL, Schnitzler, B 5b.}
\physDesc{Postkarte
\newline{}Handschrift: schwarze Tinte, deutsche Kurrent\newline{}Versand: 1) Stempel: »\nobreak{}\oindex{Edlach@\textbf{Edlach}, \emph{Besiedelter Ort (A.BSO)}|pwk}Edlach b. Reichenau in N.OE., 22 5 03, 8–12V\nobreak{}«.  2) Stempel: »\nobreak{}\oindex{IX., Alsergrund@\textbf{IX., Alsergrund}, \emph{Bezirk (A.BZK)}|pwk}Wien 9/3, 22 5. 03, 1.N, Bestellt\nobreak{}«. 
\newline{}Schnitzler: mit Bleistift die Jahreszahl »903.« ergänzt \newline{}Ordnung: mit Bleistift von unbekannter Hand nummeriert: »99« }\buchAbdrucke{\weitereDrucke{Hermann Bahr, Arthur Schnitzler: \emph{Briefwechsel, Aufzeichnungen, Dokumente (1891–1931)}. Hg. Kurt Ifkovits und Martin Anton Müller. Göttingen: \emph{Wallstein} 2018, S. 265.} }\toendnotes[C]{\smallbreak}\pstart{}{\pb}Herrn \textsc{D\textsuperscript{r} Arthur
                  Schnitzler}\pend{}\pstart{}\textcolor{pink}{Wien IX}{}\ledrightnote{\textcolor{pink}{IX., Alsergrund}}\pend{}\pstart{}\textcolor{pink}{Frankgaſſe 1}{}\ledrightnote{\textcolor{pink}{Frankgasse}}.\pend{}{\bigskip}\pstart
           {\pb}\textcolor{pink}{Edlach Anſtalt D\textsuperscript{r}
                     Konried}{}\ledrightnote{\textcolor{pink}{Kuranstalt Dr. Konried}}\hfill 21. 5.\pend
           \pstart
           Lieber Arthur! Ich habe keine Ahnung, was Du eigentlich meinſt. Ich
               bin ſeit drei Jahren Mitglied des \label{K_L01291_1v}\edtext{\textcolor{pink}{Münchner}{}\ledrightnote{\textcolor{pink}{München}}{ }\textcolor{brown}{Penſionsfonds}{}\ledrightnote{→\textcolor{brown}{Pensionsanstalt deutscher Journalisten und Schriftsteller}}}{\lemma{\textnormal{\emph{Münchner Penſionsfonds}}}\Cendnote{\textnormal{\textcolor{blue}{Bahr} meint denselben Pensionsfonds wie \textcolor{blue}{Schnitzler}, dieser hatte seinen Sitz in \textcolor{pink}{München}.}}}\label{K_L01291_1h} und zahle dafür ſehr wenig;
               ich glaube 6 oder 8 Mark pro Quartal. Von einer anderen »Zeichnung« iſt mir nichts
               bekannt. Ich komme übrigens Montag zurück u. werde mich dann erkundigen.\pend
           \pstart
           Herzlichſt{\\[\baselineskip]}Dein{\\[\baselineskip]}\spacefill\mbox{Hermann}\pend
           \leftskip=0em{}\endnumbering\briefempfaengerindex{Schnitzler, Arthur@\textsc{Schnitzler, Arthur}!zzzBahr, Hermann@\emph{von Hermann Bahr}!1903-05-211@{21. 5. 1903}|)be}\mylabel{h}  \normalsize

\doendnotes{C}
\bigskip
\vfill

\clearpage

\footnotesize

\lohead{\textsc{register}}

% Definiere theindex-Environment komplett neu ohne reledmac
\makeatletter
\renewenvironment{theindex}{%
  \section*{\indexname}%
  \setlength{\parindent}{0pt}%
  \setlength{\parskip}{0pt plus 0.3pt}%
  \let\item\@idxitem
}{%
  \clearpage
}
\makeatother

\IfFileExists{\jobname-pw.ind}{\input{\jobname-pw.ind}}{}

\end{document}

      