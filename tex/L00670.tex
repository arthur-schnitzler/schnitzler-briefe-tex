%% latex-korrekturansicht-vorspann.tex
%% Vorspann für die Korrekturansicht.
%% Lädt die gemeinsame Datei latex-vorspann.tex mit gesetztem Schalter.

\newif\ifkorrekturansicht
\korrekturansichttrue

\input{../tex-inputs/latex-vorspann}


               \section[Arthur Schnitzler an Richard Beer-Hofmann, 26. 4. 1897]{ Arthur Schnitzler an Richard Beer-Hofmann,
               26. 4. 1897}\nopagebreak\mylabel{v}\rehead{ }\normalsize\beginnumbering\briefempfaengerindex{Beer-Hofmann, Richard@\textsc{Beer-Hofmann, Richard}!zzzSchnitzler, Arthur@\emph{von Arthur Schnitzler}!1897-04-261@{26. 4. 1897}|(be} \toendnotes[C]{\smallbreak\pagebreak[2]} \Standort{YCGL, MSS 31.}
\physDesc{Brief, 2 Blätter (Briefpapier mit Trauerrand), 8 Seiten, Umschlag
\newline{}Handschrift: schwarze Tinte, deutsche Kurrent\newline{}Versand: 1) Stempel: »\nobreak{}\oindex{rue La Fayette@\textbf{rue La Fayette}, \emph{Straße (K.STR)}|pwk}Paris 51 R. Lafayette, 26 Avril 97, 8\textsuperscript{E}\nobreak{}«.  2) Stempel: »\nobreak{}\oindex{I., Innere Stadt@\textbf{I., Innere Stadt}, \emph{Bezirk (A.BZK)}|pwk}Wien 1/1, 28. 4. 97, 9–10½V., Bestellt\nobreak{}«. }\buchAbdrucke{\weitereDrucke{1) Arthur Schnitzler: \emph{Briefe 1875–1912}. Hg. Therese Nickl und Heinrich Schnitzler. Frankfurt am Main: \emph{S. Fischer} 1981, S. 317–318.} \weitereDrucke{2) Arthur Schnitzler, Richard Beer-Hofmann: \emph{Briefwechsel 1891–1931}. Hg. Konstanze Fliedl. Wien, Zürich: \emph{Europaverlag} 1992, S. 102–103.} }\toendnotes[C]{\smallbreak}\pstart{}{\pb}Herrn \textsc{Dr. Richard
                     Beer-Hofmann}\pend{}\pstart{}\textsc{\textcolor{pink}{Wien}{}\ledrightnote{\textcolor{pink}{Wien}}}\pend{}\pstart{}\textsc{\textcolor{pink}{I. Bezirk}{}\ledrightnote{\textcolor{pink}{I., Innere Stadt}}}\pend{}\pstart{}\textcolor{pink}{\textsc{Wollzeile 15}}{}\ledrightnote{\textcolor{pink}{Wollzeile}}.\pend{}\pstart{}\textcolor{pink}{\textsc{Autriche}}{}\ledrightnote{\textcolor{pink}{Österreich}}\pend{}{\bigskip}\pstart
           \raggedleft{}{\pb}26. 4. 97.\pend
           \pstart{}Lieber Richard,\pend\pstart
           allerdings würden Sie für \textcolor{pink}{Paris}{}\ledrightnote{\textcolor{pink}{Paris}} einige hundert Jahre
               brauchen!\pend
           \pstart
           Nur die \textsc{Bouquinerien}! – Und die \label{K_L00670_1v}\edtext{\textsc{Emaux}}{\lemma{\textnormal{\emph{Emaux}}}\Cendnote{\textnormal{französisch: Emailarbeiten}}}\label{K_L00670_1h} aus dem 16 u
               17. Jahrhundert im \textcolor{pink}{\textsc{Louvre}}{}\ledrightnote{\textcolor{pink}{Louvre}} – \pend
           \pstart
           Ich ſchreibe ſo beiläufig her, wo\substVorne{}\textsuperscript{rin}\substDazwischen{}bei\substHinten{} ich am
               heftigſten an Sie gedacht – {\pb}– und die \textsc{Chinoiserien} im \textcolor{pink}{\textsc{Guimet}}{}\ledrightnote{\textcolor{pink}{Museum Guimet}} –\pend
           \pstart
           Wäre ich \textcolor{blue}{Altenberg}{}\ledrightnote{\textcolor{blue}{Peter Altenberg}}{ }ſo würde ich sagen:\pend
           \pstart
           \textcolor{pink}{Paris}{}\ledrightnote{\textcolor{pink}{Paris}} iſt »die« Stadt {\dotsfive}{ }\textsc{La ville}{ }{\dotsseven}\pend
           \pstart
           \textcolor{pink}{Paris}{}\ledrightnote{\textcolor{pink}{Paris}} iſt \textsc{la grande ville}{ }{\dotsfour}\pend
           \pstart
           \numberlinefalse{}–\numberlinetrue{}\pend
           \pstart
           Im Ernſt geſprochen (im Gegenſatz zu \textcolor{blue}{Altenberg}{}\ledrightnote{\textcolor{blue}{Peter Altenberg}}.):
               Die \uline{Form} für alles iſt da, \introOben{}das
                  iſt\introOben{} das weſentliche: die ganz {\pb}großen
                  \introOben{}ſchöpferiſchen\introOben{} Talente ſcheinen heute noch zu fehlen.
               Dagegen ſind die \textsc{reproducirenden} da; die ununterbrochen für
               die Form ſorgen. Auch die Decoration iſt für alles da; jederzeit können die großen
               Künſtler auftreten, ohne sich um etwas andres kü{\geminationm}e\textcolor{gray}{rn} zu
               müſſen als um ihr Genie. – Auch große Menſchen {\pb}jeder Art finden alles bereit; der \textcolor{pink}{\textsc{Concorde}-Platz}{}\ledrightnote{\textcolor{pink}{Place de la Concorde}}{ }ſcheint
               eigentlich nur auf einen neuen \textcolor{blue}{Napoleon}{}\ledrightnote{→\textcolor{blue}{Napoleon Bonaparte}} zu warten.\pend
           \pstart
           – Aber dieſen Brief hab ich nur angefangen um mich bei Ihnen nach Ihnen zu
               erkundigen. Wie geht es \textcolor{blue}{Paula}{}\ledrightnote{\textcolor{blue}{Paula Beer-Hofmann}}? Bei »\textcolor{blue}{uns}{}\ledrightnote{→\textcolor{blue}{Marie Reinhard}}« – mit »Rieſen{\pb}ſchritten«.\pend
           \pstart
           Bleiben Sie in \textcolor{pink}{Wien}{}\ledrightnote{\textcolor{pink}{Wien}}? – \pend
           \pstart
           – Darüber ſein Sie ruhig: zu einem »wirklichen« Brief ko{\geminationm} ich hier nicht.\pend
           \pstart
           \textcolor{blue}{Graf}{}\ledrightnote{\textcolor{blue}{Max Graf}} iſt hier, Sie wiſſen ja, dem Sie eine
               zärtliche Empfehlung an \textcolor{blue}{Paul}{}\ledrightnote{\textcolor{blue}{Paul Goldmann}} gegeben. Den treff
               ich natürlich immer. {\pb}Alſo könnte der kleine \textcolor{blue}{Kraus}{}\ledrightnote{\textcolor{blue}{Karl Kraus}} bald einen Artikel über die Flucht aus \textcolor{pink}{Wien}{}\ledrightnote{\textcolor{pink}{Wien}}{ }ſchreiben. –\pend
           \pstart
           Wie leben Sie? – \pend
           \pstart
           Ich: Vormittg \textcolor{pink}{\textsc{Louvre}}{}\ledrightnote{\textcolor{pink}{Louvre}} oder \textcolor{pink}{\textsc{Luxemburg}}{}\ledrightnote{\textcolor{pink}{Jardin du Luxembourg}} oder so was; Abends immer im Theater. Entzückend die ganz kleinen. Es wi{\geminationm}elt von »Flohtheatern des Arthur Schnitzler«.\pend
           \pstart
           {\pb}Geſtern oder vorgeſtern Nachm in einem dieſer
               kleinen »\textsc{la Bodinière}« Aufführung von \introOben{}\textcolor{pink}{franzöſ.}{}\ledrightnote{\textcolor{pink}{Frankreich}}\introOben{} Muſik des 16. u 17. Jahrhunderts.\pend
           \pstart
           – In andern werden dieſe hübſchen Kleinigkeiten von \textcolor{blue}{\textsc{Lavedan}}{}\ledrightnote{\textcolor{blue}{Henri Léon Lavedan}}, von \textcolor{blue}{\textsc{Courteline}}{}\ledrightnote{\textcolor{blue}{Georges Courteline}} aufgeführt. Oder, wie ich \label{K_L00670_2v}\edtext{neulich}{\lemma{\textnormal{\emph{neulich}}}\Cendnote{\textnormal{am
                     20. 4. 1897}}}\label{K_L00670_2h} in der »\textcolor{pink}{\textsc{Roulotte}}{}\ledrightnote{\textcolor{pink}{La Roulotte}}« ſah, ein Volkslied von zwölf Zeilen wird einfach »aufgeführt«. Er und {\pb}Sie – kein lebendes Bild, was beka{\geminationn}tlich ſehr
               todt ist, ſondern ſie \uline{ſpielen} das Volkslied. –\pend
           \pstart
           Überhaupt »hier ka{\geminationn} man ſchon einmal alles haben«.\pend
           \pstart
           Schreiben Sie mir bald.\pend
           \pstart
           Adreſſe \textcolor{pink}{5 \textsc{rue de Maubeuge}}{}\ledrightnote{\textcolor{pink}{rue de Maubeuge}}\pend
           \pstart
           Herzlichst Ihr{\\[\baselineskip]}\spacefill\mbox{Arthur.}\pend
           \leftskip=0em{}\pstart
           \textcolor{blue}{Paul}{}\ledrightnote{\textcolor{blue}{Paul Goldmann}}{ }ſchon 9 Tage in \textcolor{pink}{Frankfurt}{}\ledrightnote{\textcolor{pink}{Frankfurt am Main}}; ko{\geminationm}t bald. –\pend
           \endnumbering\briefempfaengerindex{Beer-Hofmann, Richard@\textsc{Beer-Hofmann, Richard}!zzzSchnitzler, Arthur@\emph{von Arthur Schnitzler}!1897-04-261@{26. 4. 1897}|)be}\mylabel{h}  \normalsize

\doendnotes{C}
\bigskip
\vfill

\clearpage

\footnotesize

\lohead{\textsc{register}}

% Definiere theindex-Environment komplett neu ohne reledmac
\makeatletter
\renewenvironment{theindex}{%
  \section*{\indexname}%
  \setlength{\parindent}{0pt}%
  \setlength{\parskip}{0pt plus 0.3pt}%
  \let\item\@idxitem
}{%
  \clearpage
}
\makeatother

\IfFileExists{\jobname-pw.ind}{\input{\jobname-pw.ind}}{}

\end{document}

      