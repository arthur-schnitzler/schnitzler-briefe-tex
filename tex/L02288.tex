%% latex-korrekturansicht-vorspann.tex
%% Vorspann für die Korrekturansicht.
%% Lädt die gemeinsame Datei latex-vorspann.tex mit gesetztem Schalter.

\newif\ifkorrekturansicht
\korrekturansichttrue

\input{../tex-inputs/latex-vorspann}


               \section[Arthur Schnitzler an Robert Adam, 15. 6. 1918]{ Arthur Schnitzler an Robert Adam, 15. 6. 1918}\nopagebreak\mylabel{v}\rehead{ }\normalsize\beginnumbering\briefempfaengerindex{Adam, Robert@\textsc{Adam, Robert}!zzzSchnitzler, Arthur@\emph{von Arthur Schnitzler}!1918-06-151@{15. 6. 1918}|(be} \toendnotes[C]{\smallbreak\pagebreak[2]} \Standort{DLA, 96.34.2/9.}
\physDesc{Postkarte
\newline{}Handschrift: schwarze Tinte, deutsche Kurrent\newline{}Versand: Stempel: »\nobreak{}15. {[}6. 1918{]}\nobreak{}«.  }\pstart{}{\pb}\textcolor{pink}{Wien XVIII. \textsc{Sternwartestr}. 71}{}\ledrightnote{\textcolor{pink}{VIII., Josefstadt}}.\pend{}{\bigskip}\pstart{}Herrn \textsc{Dr. Robert Adam}\pend{}\pstart{}\textsc{Pollak}.\pend{}\pstart{}\textcolor{pink}{\textsc{Wien} XII}{}\ledrightnote{\textcolor{pink}{XII., Meidling}}\pend{}\pstart{}\textcolor{pink}{\textsc{Meidlinger Hauptstr} 58}{}\ledrightnote{\textcolor{pink}{Meidlinger Hauptstraße}}.\pend{}{\bigskip}\pstart
           \noindent{}{\pb}\textcolor{gray}{\textbf{A. S.}}\hfill 15. 6. 18\pend
           \pstart{}Verehrter Herr Doktor, \pend\pstart
           vielen Dank für Ihren lieben Brief. Paſſt es Ihnen, ſo erwarte ich Sie gern am
                        Mitwwoch (19.) gegen 7 Uhr Abends Oder geben Sie
                    ſelbſt einen anderen Tag an. Jedenfalls ſreue ich mich Sie bald
                    wiederzuſehen.\pend
           \pstart
           Herzlich grüßt Sie ſehr ergeb{\\[\baselineskip]}\spacefill\mbox{Arthur Schnitzler.}\pend
           \leftskip=0em{}\endnumbering\briefempfaengerindex{Adam, Robert@\textsc{Adam, Robert}!zzzSchnitzler, Arthur@\emph{von Arthur Schnitzler}!1918-06-151@{15. 6. 1918}|)be}\mylabel{h}  \normalsize

\doendnotes{C}
\bigskip
\vfill

\clearpage

\footnotesize

\lohead{\textsc{register}}

% Definiere theindex-Environment komplett neu ohne reledmac
\makeatletter
\renewenvironment{theindex}{%
  \section*{\indexname}%
  \setlength{\parindent}{0pt}%
  \setlength{\parskip}{0pt plus 0.3pt}%
  \let\item\@idxitem
}{%
  \clearpage
}
\makeatother

\IfFileExists{\jobname-pw.ind}{\input{\jobname-pw.ind}}{}

\end{document}

      