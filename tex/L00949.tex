%% latex-korrekturansicht-vorspann.tex
%% Vorspann für die Korrekturansicht.
%% Lädt die gemeinsame Datei latex-vorspann.tex mit gesetztem Schalter.

\newif\ifkorrekturansicht
\korrekturansichttrue

\input{../tex-inputs/latex-vorspann}


               \section[Hugo von Hofmannsthal an Arthur Schnitzler, 20. 7. {[}1899{]}]{ Hugo von Hofmannsthal an Arthur Schnitzler, 20. 7. {[}1899{]}}\nopagebreak\mylabel{v}\rehead{ }\normalsize\beginnumbering\briefempfaengerindex{Schnitzler, Arthur@\textsc{Schnitzler, Arthur}!zzzHofmannsthal, Hugo von@\emph{von Hugo von Hofmannsthal}!1899-07-202@{20. 7. {[}1899{]}}|(be} \toendnotes[C]{\smallbreak\pagebreak[2]} \Standort{CUL, Schnitzler, B 43.}
\physDesc{Brief, 2 Blätter, 7 Seiten
\newline{}Handschrift: schwarze Tinte, deutsche Kurrent\newline{}Ordnung: 1) mit Bleistift von unbekannter Hand nummeriert: »\strikeout{154}« 2) mit Bleistift von unbekannter Hand nummeriert:
                                                »152.«. Diese Hand dürfte auch für
                                            die Paginierung der beiden Blätter mit
                                            »1« respektive »2«
                                            verantwortlich sein}\buchAbdrucke{\weitereDrucke{1) Hugo von Hofmannsthal: \emph{Briefe. 1890–1901}. Berlin: \emph{S. Fischer} 1935, S. 288–289.} \weitereDrucke{2) Hugo von Hofmannsthal, Arthur Schnitzler: \emph{Briefwechsel}. Hg. Therese Nickl und Heinrich Schnitzler. Frankfurt am Main: \emph{S. Fischer} 1964, S. 126–127.} }\toendnotes[C]{\smallbreak}\pstart
           \noindent{}{\pb}\textcolor{gray}{\textbf{\label{T_L00949-1v}\edtext{hvH}{\lemma{\textnormal{\emph{hvH}}}\Cendnote{\textnormal{gedrucktes Monogramm mit Krone
                                in blauer Farbe}}}\label{T_L00949-1h}}}\pend
           \pstart
           \raggedleft{}\textcolor{pink}{Marienbad}{}\ledrightnote{\textcolor{pink}{Marienbad}}\pend
           \pstart
           \raggedleft{}20 VII\pend
           \pstart{}mein lieber Arthur\pend\pstart
           ich möchte Ihnen gern einen viel ausführlicheren Brief ſchreiben, möchte auch
                    gern über \textcolor{blue}{Richard}{}\ledrightnote{\textcolor{blue}{Richard Beer-Hofmann}} vieles ſagen, aber ich bin
                    ſo unglaublich abgeſpannt, ſobald meine tägliche wie im Fieber eintretende
                    Arbeitszeit vorüber iſt, daſs {\pb}ich kaum im Stand bin die Feder zu halten.\pend
           \pstart
           Ich war mit meinen Nerven noch nie ſo herunter: ein geräuschvoller Speiſeſaal
                    macht mir heftige phyſiſche Schmerzen im Genick und lauter ſolche Dummheiten.
                    Ich werde nach dem 28\textsuperscript{ten} mindeſtens 14 Tage zu arbeiten aufhören {\pb}und das Landleben führen,
                        da\strikeout{ſ}s mir allein ganz wohl thut: \textsc{tennys} Bad und vielerlei harmloſe Geſellſchaft. Ich
                    gehe daher nach \textcolor{pink}{Alt-Auſſee}{}\ledrightnote{\textcolor{pink}{Altaussee}} entweder zu den
                        \textcolor{blue}{\textsc{Franckensteins}}{}\ledrightnote{\textcolor{blue}{Clemens von Franckenstein}{\newline}\textcolor{blue}{Georg von Franckenstein}} oder zum \textcolor{pink}{\textsc{Seewirth}}{}\ledrightnote{\textcolor{pink}{Seewirt}}. Vor einer Radreiſe, \uline{jetzt}, hätte ich bei
                    meinem übermäßig montirten und ruheloſen Zuſtand direct Angſt. {\pb}Ich werd mich ſchon wieder
                    in Ordnung bringen.\pend
           \pstart
           Mein \textcolor{green}{Stück}{}\ledrightnote{→\textcolor{green}{Das Bergwerk zu Falun}} iſt ein
                    fünfactiges märchenartiges Trauerſpiel, in Verſen. 2 Acte ſind nahezu fertig.
                    Ich habe noch nie ſo gern an etwas gearbeitet. Fangen Sie nur auch zu arbeiten
                    an.\pend
           \pstart
           Oder machen Sie jetzt mit \textcolor{blue}{Salten}{}\ledrightnote{\textcolor{blue}{Felix Salten}} eine Radtour
                        {\pb}und laſſen für mich und
                    für September nur den Weg \textcolor{pink}{\textsc{Passau}}{}\ledrightnote{\textcolor{pink}{Passau}} – \textcolor{pink}{\textsc{Nürnberg}}{}\ledrightnote{\textcolor{pink}{Nürnberg}} – \textcolor{pink}{Rothenburg}{}\ledrightnote{\textcolor{pink}{Rothenburg ob der Tauber}} – \textcolor{pink}{München}{}\ledrightnote{\textcolor{pink}{München}} – \textcolor{pink}{Salzburg}{}\ledrightnote{\textcolor{pink}{Salzburg}} in
                    Reſerve. Das wäre ſchön!\pend
           \pstart
           Und um den 15. Auguſt träfen wir uns bei \textcolor{blue}{Richard}{}\ledrightnote{\textcolor{blue}{Richard Beer-Hofmann}}, verbrächten immer den halben Tag arbeitend,
                    gingen dann {\pb}nach \textcolor{pink}{Salzburg}{}\ledrightnote{\textcolor{pink}{Salzburg}}, noch mehr arbeitend und träten
                        Anfang September die Reiſe an. Mir folgen, ich bin der
                    Geſcheidtere!\pend
           \pstart
           Herzlich Ihr{\\[\baselineskip]}\spacefill\mbox{Hugo}\pend
           \leftskip=0em{}\pstart
           \noindent{}\textsc{P. S.}\pend
           \pstart
           Es iſt nicht ernſt, daſs ich der Geſcheidtere bin. Sonſt sind Sie vielleicht
                        beleidigt.\pend
           \pstart
           \centering{}{\pb}Immer ſchreiben!\pend
           \endnumbering\briefempfaengerindex{Schnitzler, Arthur@\textsc{Schnitzler, Arthur}!zzzHofmannsthal, Hugo von@\emph{von Hugo von Hofmannsthal}!1899-07-202@{20. 7. {[}1899{]}}|)be}\mylabel{h}  \normalsize

\doendnotes{C}
\bigskip
\vfill

\clearpage

\footnotesize

\lohead{\textsc{register}}

% Definiere theindex-Environment komplett neu ohne reledmac
\makeatletter
\renewenvironment{theindex}{%
  \section*{\indexname}%
  \setlength{\parindent}{0pt}%
  \setlength{\parskip}{0pt plus 0.3pt}%
  \let\item\@idxitem
}{%
  \clearpage
}
\makeatother

\IfFileExists{\jobname-pw.ind}{\input{\jobname-pw.ind}}{}

\end{document}

      