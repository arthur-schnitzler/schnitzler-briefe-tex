%% latex-korrekturansicht-vorspann.tex
%% Vorspann für die Korrekturansicht.
%% Lädt die gemeinsame Datei latex-vorspann.tex mit gesetztem Schalter.

\newif\ifkorrekturansicht
\korrekturansichttrue

\input{../tex-inputs/latex-vorspann}


               \section[Arthur Schnitzler an Richard Beer-Hofmann, 27. 3. 1903]{ Arthur Schnitzler an Richard Beer-Hofmann, 27. 3. 1903}\nopagebreak\mylabel{v}\rehead{ }\normalsize\beginnumbering\briefempfaengerindex{Beer-Hofmann, Richard@\textsc{Beer-Hofmann, Richard}!zzzSchnitzler, Arthur@\emph{von Arthur Schnitzler}!1903-03-271@{27. 3. 1903}|(be} \toendnotes[C]{\smallbreak\pagebreak[2]} \Standort{YCGL, MSS 31.}
\physDesc{Brief, 1 Blatt, 3 Seiten, Umschlag
\newline{}Handschrift: Bleistift, deutsche Kurrent\newline{}Versand: 1) Stempel: »\nobreak{}\oindex{IX., Alsergrund@\textbf{IX., Alsergrund}, \emph{Bezirk (A.BZK)}|pwk}9/3 Wien, 27. 3. \textcolor{gray}{03}, 11–12V\nobreak{}«.  2) Stempel: »\nobreak{}\oindex{Rodaun@\textbf{Rodaun}, \emph{Teil eines besiedelten Ortes (A.BSOX)}|pwk}{\pb}Rodaun, 27. 3. 03, 11–12V\nobreak{}«. }\buchAbdrucke{\weitereDrucke{Arthur Schnitzler, Richard Beer-Hofmann: \emph{Briefwechsel 1891–1931}. Hg. Konstanze Fliedl. Wien, Zürich: \emph{Europaverlag} 1992, S. 162.} }\toendnotes[C]{\smallbreak}\pstart{}{\pb}Herrn\pend{}\pstart{}\textsc{Dr. Richard Beer-Hofmann}\pend{}\pstart{}\textcolor{pink}{Rodaun}{}\ledrightnote{\textcolor{pink}{Rodaun}}\pend{}\pstart{}bei \textcolor{pink}{Lieſing}{}\ledrightnote{\textcolor{pink}{XXIII., Liesing}}\pend{}\pstart{}\textcolor{pink}{Lieſinger Straße 2}{}\ledrightnote{\textcolor{pink}{Liesingerstraße}}. \pend{}{\bigskip}\pstart
           \raggedleft{}{\pb}27./3 903.\pend
           \pstart{}mein lieber Richard,\pend\pstart
           \textcolor{green}{Lear}{}\ledrightnote{\textcolor{green}{König Lear}} hab ich \label{K_L01280_1v}\edtext{heuer}{\lemma{\textnormal{\emph{heuer}}}\Cendnote{\textnormal{Gemeint ist
                  die Theatersaison. Vgl. A. S.: \emph{Tagebuch}, 28. 9. 1902}}}\label{K_L01280_1h} ſchon einmal geſehen; übrigens ſind fünf in einer Loge zu viel,
                  un\textcolor{gray}{d} man hätte weder was von \textcolor{blue}{\textsc{Shakespeare}}{}\ledrightnote{\textcolor{blue}{William Shakespeare}} noch von einander\pend
           \pstart
           Man könnte ſich ſchon viel öfter ſehen, we{\geminationn} man nicht ſo
               ſchwerfällig wäre, was nicht {\pb}nur auf Sie, ſondern
               eigentlich viel mehr auf mich geht. Übrigens hab ich von Tag zu Tag irgend was
               telephonisches von Ihnen erwartet. Auch denk ich im Laufe der nächſten Woche einmal,
                  Vormittags, vielleicht mit \textcolor{blue}{Olga}{}\ledrightnote{\textcolor{blue}{Olga Schnitzler}},
               in \textcolor{pink}{Rodaun}{}\ledrightnote{\textcolor{pink}{Rodaun}} aufzutauchen.\pend
           \pstart
           Grüß Sie Gott und verſichern {\pb}Sie \textcolor{blue}{Hugo}{}\ledrightnote{\textcolor{blue}{Hugo von Hofmannsthal}}, dem begabten \label{K_L01280_2v}\edtext{Adreſſenschreiber}{\lemma{\textnormal{\emph{Adreſſenschreiber}}}\Cendnote{\textnormal{Die Adressierung des Briefes vom 26. 3. 1903 stammte von \textcolor{blue}{Hofmannsthal}.}}}\label{K_L01280_2h}, das gleiche.\pend
           \pstart
           Der Ihrige,{\\[\baselineskip]}\spacefill\mbox{A.}\pend
           \leftskip=0em{}\endnumbering\briefempfaengerindex{Beer-Hofmann, Richard@\textsc{Beer-Hofmann, Richard}!zzzSchnitzler, Arthur@\emph{von Arthur Schnitzler}!1903-03-271@{27. 3. 1903}|)be}\mylabel{h}  \normalsize

\doendnotes{C}
\bigskip
\vfill

\clearpage

\footnotesize

\lohead{\textsc{register}}

% Definiere theindex-Environment komplett neu ohne reledmac
\makeatletter
\renewenvironment{theindex}{%
  \section*{\indexname}%
  \setlength{\parindent}{0pt}%
  \setlength{\parskip}{0pt plus 0.3pt}%
  \let\item\@idxitem
}{%
  \clearpage
}
\makeatother

\IfFileExists{\jobname-pw.ind}{\input{\jobname-pw.ind}}{}

\end{document}

      