%% latex-korrekturansicht-vorspann.tex
%% Vorspann für die Korrekturansicht.
%% Lädt die gemeinsame Datei latex-vorspann.tex mit gesetztem Schalter.

\newif\ifkorrekturansicht
\korrekturansichttrue

\input{../tex-inputs/latex-vorspann}


               \section[Arthur Schnitzler an Hermann Bahr, 6. 9. 1898]{ Arthur Schnitzler an Hermann Bahr, 6. 9. 1898}\nopagebreak\mylabel{v}\rehead{ }\normalsize\beginnumbering\briefempfaengerindex{Bahr, Hermann@\textsc{Bahr, Hermann}!zzzSchnitzler, Arthur@\emph{von Arthur Schnitzler}!1898-09-061@{6. 9. 1898}|(be} \toendnotes[C]{\smallbreak\pagebreak[2]} \Standort{TMW, HS AM 60158 Ba.}
\physDesc{Briefkarte
\newline{}Handschrift: schwarze Tinte, deutsche Kurrent\newline{}Ordnung: Lochung }\buchAbdrucke{\weitereDrucke{1) \emph{6. 9. 1898, Abschrift.} In: Arthur Schnitzler: \emph{The Letters of Arthur Schnitzler to Hermann Bahr}. Edited, annotated, and with an introduction, by Donald G.
                        Daviau. Chapel Hill: \emph{The University of North Carolina Press} 1978, S. 64 (University of North Carolina studies in the Germanic languages
                        and literatures, 89).} \weitereDrucke{2) Hermann Bahr, Arthur Schnitzler: \emph{Briefwechsel, Aufzeichnungen, Dokumente (1891–1931)}. Hg. Kurt Ifkovits und Martin Anton Müller. Göttingen: \emph{Wallstein} 2018, S. 163.} }\toendnotes[C]{\smallbreak}\pstart
           \noindent{}{\pb}Lieber Hermann,
               ich war neulich in der \textcolor{pink}{Redaction}{}\ledrightnote{\textcolor{pink}{Redaktion der »Zeit«}} u habe
               dich nicht getroffen. Auf dieſem Weg alſo meine herzlichſte Theilnahme zu dem \label{K_L00845_1v}\edtext{Hinſcheiden deines \textcolor{blue}{Vaters}{}\ledrightnote{→\textcolor{blue}{Alois Bahr}}}{\lemma{\textnormal{\emph{Hinſcheiden … Vaters}}}\Cendnote{\textnormal{\textcolor{blue}{Alois Bahr}
                   starb am 5. 9. 1898 in
                     \textcolor{pink}{Salzburg}.}}}\label{K_L00845_1h}.\pend
           \pstart
           Wenn du wieder in \textcolor{pink}{Wien}{}\ledrightnote{\textcolor{pink}{Wien}} biſt, ſehen wir uns
               hoffentlich bald. Mit den herzlichſten Grüßen dein\pend
           \pstart \spacefill\mbox{Arthur Schnitzler}\pend{}\pstart
           6. 9. 98.\pend
           \endnumbering\briefempfaengerindex{Bahr, Hermann@\textsc{Bahr, Hermann}!zzzSchnitzler, Arthur@\emph{von Arthur Schnitzler}!1898-09-061@{6. 9. 1898}|)be}\mylabel{h}  \normalsize

\doendnotes{C}
\bigskip
\vfill

\clearpage

\footnotesize

\lohead{\textsc{register}}

% Definiere theindex-Environment komplett neu ohne reledmac
\makeatletter
\renewenvironment{theindex}{%
  \section*{\indexname}%
  \setlength{\parindent}{0pt}%
  \setlength{\parskip}{0pt plus 0.3pt}%
  \let\item\@idxitem
}{%
  \clearpage
}
\makeatother

\IfFileExists{\jobname-pw.ind}{\input{\jobname-pw.ind}}{}

\end{document}

      