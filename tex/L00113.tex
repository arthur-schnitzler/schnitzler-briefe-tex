%% latex-korrekturansicht-vorspann.tex
%% Vorspann für die Korrekturansicht.
%% Lädt die gemeinsame Datei latex-vorspann.tex mit gesetztem Schalter.

\newif\ifkorrekturansicht
\korrekturansichttrue

\input{../tex-inputs/latex-vorspann}


               \section[Arthur Schnitzler an Richard Beer-Hofmann, 11. 8. 1892]{ Arthur Schnitzler an Richard Beer-Hofmann, 11. 8. 1892}\nopagebreak\mylabel{v}\rehead{ }\normalsize\beginnumbering\briefempfaengerindex{Beer-Hofmann, Richard@\textsc{Beer-Hofmann, Richard}!zzzSchnitzler, Arthur@\emph{von Arthur Schnitzler}!1892-08-111@{11. 8. 1892}|(be} \toendnotes[C]{\smallbreak\pagebreak[2]} \Standort{YCGL, MSS 31.}
\physDesc{Brief, 1 Blatt (Briefpapier mit Trauerrand), 1 Seite, Umschlag
\newline{}Handschrift: schwarze Tinte, deutsche Kurrent\newline{}Versand: 1) Stempel: »\nobreak{}{\pb}Wien 1/1, 12 8 92, 10\nobreak{}«.  2) Stempel: »\nobreak{}\oindex{Bad Ischl@\textbf{Bad Ischl}, \emph{Besiedelter Ort (A.BSO)}|pwk}Ischl, 1\textcolor{gray}{4} 8 92, 7–8N\nobreak{}«. }\buchAbdrucke{\weitereDrucke{Arthur Schnitzler, Richard Beer-Hofmann: \emph{Briefwechsel 1891–1931}. Hg. Konstanze Fliedl. Wien, Zürich: \emph{Europaverlag} 1992, S. 36.} }\pstart{}{\pb}Herrn \textsc{Dr. Richard
                     Beer-Hofmann}\pend{}\pstart{}\textsc{\textcolor{pink}{Ischl}{}\ledrightnote{\textcolor{pink}{Bad Ischl}}}\pend{}\pstart{}\textsc{\textcolor{pink}{Grazerstraße 6}{}\ledrightnote{\textcolor{pink}{Grazer Straße}}}.\pend{}{\bigskip}\pstart
           \raggedleft{}{\pb}11. 8. 92\pend
           \pstart
           Mein lieber Richard! Mir aber auch keine Zeile zu ſchreiben!! Ich
               will Sie heut auch nur fragen, 1) wie lang Sie in \textcolor{pink}{Iſchl}{}\ledrightnote{\textcolor{pink}{Bad Ischl}} zu bleiben gedenken und 2) ob Sie ſich entſchließen kö{\geminationn}ten, von \textcolor{pink}{Iſchl}{}\ledrightnote{\textcolor{pink}{Bad Ischl}} aus in
               den erſten Septembertagen weiterzureiſen. Ich brauch Ihnen wohl nicht zu verſichern,
               daſs ich Sie nicht verbannen, ſondern nur eventuell \introOben{}an Ihnen\introOben{}
               einen liebenswürdigen Reiſebegleiter haben will. –\pend
           \pstart
           Mehr ſchreibe ich Ihnen heute nicht: Ihre Verpflichtg während des So{\geminationm}ers war es zu dichten und zu trachten, und ich bin
               begierig was von Ihnen zu erfahren. Und ich – ! ach Gott! – Und \uline{doch} hab ich was geſchrieben! –\pend
           \pstart
           Herzlichſt Ihr{\\[\baselineskip]}\spacefill\mbox{Arthur.}\pend
           \leftskip=0em{}\endnumbering\briefempfaengerindex{Beer-Hofmann, Richard@\textsc{Beer-Hofmann, Richard}!zzzSchnitzler, Arthur@\emph{von Arthur Schnitzler}!1892-08-111@{11. 8. 1892}|)be}\mylabel{h}  \normalsize

\doendnotes{C}
\bigskip
\vfill

\clearpage

\footnotesize

\lohead{\textsc{register}}

% Definiere theindex-Environment komplett neu ohne reledmac
\makeatletter
\renewenvironment{theindex}{%
  \section*{\indexname}%
  \setlength{\parindent}{0pt}%
  \setlength{\parskip}{0pt plus 0.3pt}%
  \let\item\@idxitem
}{%
  \clearpage
}
\makeatother

\IfFileExists{\jobname-pw.ind}{\input{\jobname-pw.ind}}{}

\end{document}

      