%% latex-korrekturansicht-vorspann.tex
%% Vorspann für die Korrekturansicht.
%% Lädt die gemeinsame Datei latex-vorspann.tex mit gesetztem Schalter.

\newif\ifkorrekturansicht
\korrekturansichttrue

\input{../tex-inputs/latex-vorspann}


               \section[Thomas Mann an Arthur Schnitzler, 15. 2. 1904]{ Thomas Mann an Arthur Schnitzler, 15. 2. 1904}\nopagebreak\mylabel{v}\rehead{ }\normalsize\beginnumbering\briefempfaengerindex{Schnitzler, Arthur@\textsc{Schnitzler, Arthur}!zzzMann, Thomas@\emph{von Thomas Mann}!1904-02-152@{15. 2. 1904}|(be} \toendnotes[C]{\smallbreak\pagebreak[2]} \Standort{CUL, Schnitzler, B 67.}
\physDesc{Briefkarte
\newline{}Handschrift: schwarze Tinte, deutsche Kurrent}\Standort{DLA, A:Schnitzler, HS.NZ85.1.3986, S. 2.}
\physDesc{maschinelle Abschrift
\newline{}Schreibmaschine\newline{}Zusatz: die Abschrift noch zu Lebzeiten
                                    Schnitzlers hergestellt }\buchAbdrucke{\weitereDrucke{Hertha Krotkoff: \emph{Arthur Schnitzler – Thomas Mann: Briefe.} In: \emph{Modern Austrian Literature}, Jg. 7 (1974) Nr. 1/2, S. 13.} }\toendnotes[C]{\smallbreak}\pstart
           {\pb}\textcolor{pink}{München}{}\ledrightnote{\textcolor{pink}{München}} d. 15. II. 1904{\\}\textcolor{pink}{Konradſtraße 11 pt}{}\ledrightnote{\textcolor{pink}{Konradstraße}}\pend
           \pstart{}Verehrter Herr:\pend\pstart
           Durch unſeren gemeinſamen \textcolor{blue}{Verleger}{}\ledrightnote{→\textcolor{blue}{Samuel Fischer}} erhielt ich das Exemplar Ihres Schauſpiels »\textcolor{green}{Der einſame Weg}{}\ledrightnote{\textcolor{green}{Der einsame Weg. Schauspiel in fünf Akten}}« mit Ihrer liebenswürdigen Widmung und
                    beeile mich, Ihnen meine Freude und Dankbarkeit auszudrücken. Seien Sie
                    überzeugt, {\pb}daß ich den freundlichen
                    Beifall des Dichters der »\textcolor{green}{Lebendigen Stunden}{}\ledrightnote{\textcolor{green}{Lebendige Stunden. Vier Einakter}}«
                    von ganzem Herzen zu ſchätzen weiß!\pend
           \pstart
           Ihr ergebener{\\[\baselineskip]}\spacefill\mbox{Thomas Mann.}\pend
           \leftskip=0em{}\endnumbering\briefempfaengerindex{Schnitzler, Arthur@\textsc{Schnitzler, Arthur}!zzzMann, Thomas@\emph{von Thomas Mann}!1904-02-152@{15. 2. 1904}|)be}\mylabel{h}  \normalsize

\doendnotes{C}
\bigskip
\vfill

\clearpage

\footnotesize

\lohead{\textsc{register}}

% Definiere theindex-Environment komplett neu ohne reledmac
\makeatletter
\renewenvironment{theindex}{%
  \section*{\indexname}%
  \setlength{\parindent}{0pt}%
  \setlength{\parskip}{0pt plus 0.3pt}%
  \let\item\@idxitem
}{%
  \clearpage
}
\makeatother

\IfFileExists{\jobname-pw.ind}{\input{\jobname-pw.ind}}{}

\end{document}

      