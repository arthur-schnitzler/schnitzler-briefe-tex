%% latex-korrekturansicht-vorspann.tex
%% Vorspann für die Korrekturansicht.
%% Lädt die gemeinsame Datei latex-vorspann.tex mit gesetztem Schalter.

\newif\ifkorrekturansicht
\korrekturansichttrue

\input{../tex-inputs/latex-vorspann}


               \section[Arthur Schnitzler an Richard Beer-Hofmann, 12. 7. 1910]{ Arthur Schnitzler an Richard Beer-Hofmann, 12. 7. 1910}\nopagebreak\mylabel{v}\rehead{ }\normalsize\beginnumbering\briefempfaengerindex{Beer-Hofmann, Richard@\textsc{Beer-Hofmann, Richard}!zzzSchnitzler, Arthur@\emph{von Arthur Schnitzler}!1910-07-121@{12. 7. 1910}|(be} \toendnotes[C]{\smallbreak\pagebreak[2]} \Standort{CUL, Schnitzler, B 8.1, S. 137.}
\physDesc{maschinelle Abschrift
\newline{}Schreibmaschine\newline{}Ordnung: von unbekannter Hand als Briefnummer 297 gekennzeichnet }\buchAbdrucke{\weitereDrucke{Arthur Schnitzler, Richard Beer-Hofmann: \emph{Briefwechsel 1891–1931}. Hg. Konstanze Fliedl. Wien, Zürich: \emph{Europaverlag} 1992, S. 210–211.} }\toendnotes[C]{\smallbreak}\pstart
           \raggedleft{}{\pb}\textcolor{pink}{Wien}{}\ledrightnote{\textcolor{pink}{Wien}}, 12. 7. 1910.\pend
           \pstart
           Mein lieber Richard, wir waren ein paar Tage auf dem \textcolor{pink}{Semmering}{}\ledrightnote{\textcolor{pink}{Semmering}} – \textcolor{blue}{Mama}{}\ledrightnote{→\textcolor{blue}{Louise Schnitzler}}’s Geburtstag, \textcolor{blue}{englische Verwandte}{}\ledrightnote{→\textcolor{blue}{Felix Markbreiter}{\newline}→\textcolor{blue}{Amelia Margaret Markbreiter}{\newline}→\textcolor{blue}{Andrée Marie Markbreiter}{\newline}→\textcolor{blue}{Julie Markbreiter}}, \textcolor{blue}{Brahm}{}\ledrightnote{\textcolor{blue}{Otto Brahm}}, \textcolor{blue}{Kainz}{}\ledrightnote{\textcolor{blue}{Josef Kainz}} – und Ihr Brief erwartete
               mich, als ich unsere schon in Zerstörung begriffene \textcolor{pink}{Wohnung}{}\ledrightnote{→\textcolor{pink}{Edmund-Weiß-Gasse}} wieder betrat. Ich freu mich sehr, dass
               Sie das \textcolor{green}{Stück}{}\ledrightnote{→\textcolor{green}{Das weite Land. Tragikomödie in fünf Akten}} gut finden und
               glaube auch gern Ihrer Voraussage, dass ich noch Freude an meiner \textcolor{green}{Tragikomödie}{}\ledrightnote{→\textcolor{green}{Das weite Land. Tragikomödie in fünf Akten}} haben werde – nur bin ich nicht
               sicher, ob das schon bei Gelegenheit der ersten Aufführung sein wird {\dotstwo} was ebensowohl mit Publikumspsychologie als mit
               Schauspielerconstellation zusammenhängt. Ueber all dies, – auch über die Liebe der
                  \textcolor{green}{Genia}{}\ledrightnote{→\textcolor{green}{Das weite Land. Tragikomödie in fünf Akten}}’s zu den \textcolor{green}{Hofreiter}{}\ledrightnote{→\textcolor{green}{Das weite Land. Tragikomödie in fünf Akten}}s (die vorkommt! öfters als die zu
               edlern Exemplaren!) näheres, hoffentlich, noch in diesem Sommer. Vorläufig bin ich
               etwas gerührt und fast etwas beschämt, dass Sie mir einen so langen und schönen Brief
               geschrieben haben. (Wenn es aber als Ausrede benützt werden soll, dass Sie im »Traum«
               nicht weiter gekommen sind, so wasch ich meine Hände in Unschuld.) Morgen kommen
               meine Bücher in die \textcolor{pink}{Sternwartestrasse}{}\ledrightnote{\textcolor{pink}{Sternwartestraße}}; und wir
               hoffen Samstag oder Sonntag zum ersten Mal drüben zu schlafen. Ihr \textcolor{green}{\textcolor{blue}{Mirjam}{}\ledrightnote{\textcolor{blue}{Mirjam Beer-Hofmann}}-Gedicht}{}\ledrightnote{→\textcolor{green}{Schlaflied für Mirjam}} (für dessen Sendung ich
               herzlich danke) kann ich jetzt von der braven \textcolor{blue}{Frieda}{}\ledrightnote{\textcolor{blue}{Frieda Pollak}} nicht abschreiben lassen, weil sie in \textcolor{pink}{Alt-Aussee}{}\ledrightnote{\textcolor{pink}{Altaussee}}{ }\textcolor{pink}{Salzberggasse 46}{}\ledrightnote{\textcolor{pink}{Salzbergstraße}} lebt, ohne Schreibmaschine. Aber
               ich will nächste Woche, wenn wir so weit sind, ihre \textcolor{blue}{Vertreterin}{}\ledrightnote{→\textcolor{blue}{Grethe Hoffmann}} kommen lassen.\pend
           \pstart
           Und wie geht es Ihnen? Sind Sie mit Wohnung und allem übrigen zufrieden? Und \textcolor{blue}{Paula}{}\ledrightnote{\textcolor{blue}{Paula Beer-Hofmann}}? Und die \textcolor{blue}{Kinder}{}\ledrightnote{→\textcolor{blue}{Gabriel Beer-Hofmann}{\newline}→\textcolor{blue}{Mirjam Beer-Hofmann}{\newline}→\textcolor{blue}{Naëmah Beer-Hofmann}}?\pend
           \pstart
           Wir grüssen Euch alle vielmals.\pend
           \pstart Herzlichst Ihr \spacefill\mbox{Arthur.}\pend{}\pstart
           \noindent{}(nach \textcolor{pink}{Ischl}{}\ledrightnote{\textcolor{pink}{Bad Ischl}})\pend
           \endnumbering\briefempfaengerindex{Beer-Hofmann, Richard@\textsc{Beer-Hofmann, Richard}!zzzSchnitzler, Arthur@\emph{von Arthur Schnitzler}!1910-07-121@{12. 7. 1910}|)be}\mylabel{h}  \normalsize

\doendnotes{C}
\bigskip
\vfill

\clearpage

\footnotesize

\lohead{\textsc{register}}

% Definiere theindex-Environment komplett neu ohne reledmac
\makeatletter
\renewenvironment{theindex}{%
  \section*{\indexname}%
  \setlength{\parindent}{0pt}%
  \setlength{\parskip}{0pt plus 0.3pt}%
  \let\item\@idxitem
}{%
  \clearpage
}
\makeatother

\IfFileExists{\jobname-pw.ind}{\input{\jobname-pw.ind}}{}

\end{document}

      