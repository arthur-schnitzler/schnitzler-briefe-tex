%% latex-korrekturansicht-vorspann.tex
%% Vorspann für die Korrekturansicht.
%% Lädt die gemeinsame Datei latex-vorspann.tex mit gesetztem Schalter.

\newif\ifkorrekturansicht
\korrekturansichttrue

\input{../tex-inputs/latex-vorspann}


               \section[Hermann Bahr an Arthur Schnitzler, 20. 5. 1902]{ Hermann Bahr an Arthur Schnitzler, 20. 5. 1902}\nopagebreak\mylabel{v}\rehead{ }\normalsize\beginnumbering\briefempfaengerindex{Schnitzler, Arthur@\textsc{Schnitzler, Arthur}!zzzBahr, Hermann@\emph{von Hermann Bahr}!1902-05-201@{20. 5. 1902}|(be} \toendnotes[C]{\smallbreak\pagebreak[2]} \Standort{CUL, Schnitzler, B 5b.}
\physDesc{Karte mit Trauerrand
\newline{}Druck
\newline{}Handschrift: schwarze Tinte, deutsche Kurrent\newline{}Ordnung: mit Bleistift von unbekannter Hand nummeriert:
                                    »89« }\buchAbdrucke{\weitereDrucke{Hermann Bahr, Arthur Schnitzler: \emph{Briefwechsel, Aufzeichnungen, Dokumente (1891–1931)}. Hg. Kurt Ifkovits und Martin Anton Müller. Göttingen: \emph{Wallstein} 2018, S. 238.} }\toendnotes[C]{\smallbreak}\pstart
           \noindent{}{\pb}\textcolor{gray}{\textbf{Für die vielen Beweise herzlicher Teilnahme bei dem
                     Hinscheiden und der Beerdigung unserer lieben, unvergesslichen Mutter,
                     Schwiegermutter, Schwägerin u. Tante}}\pend
           \pstart
           \textcolor{gray}{\textbf{Frau \textcolor{blue}{Mina Bahr geb. von
                        Weidlich}{}\ledrightnote{\textcolor{blue}{Wilhelmine Bahr}}}}\pend
           \pstart
           \textcolor{gray}{\textbf{sprechen ihren innigsten Dank aus}}\pend
           \leftskip=3em{}\pstart
           \noindent{}\textcolor{gray}{\textbf{\textcolor{pink}{Salzburg}{}\ledrightnote{\textcolor{pink}{Salzburg}}, 19. Mai
                        1902}}\pend
           \leftskip=0em{}\pstart
           \noindent{}\raggedleft{}\textcolor{gray}{\textbf{Die tieftrauernd Hinterbliebenen.}}\pend
           \pstart
           {\pb}Wie eine fixe Idee verfolgt mich dieſe ganzen Tage
               der Satz: \label{K_L01221_1v}\edtext{\textcolor{green}{es gibt alſo Fälle, wo \textcolor{pink}{Salzburg}{}\ledrightnote{\textcolor{pink}{Salzburg}} nicht wirkt}{}\ledrightnote{→\textcolor{green}{Lebendige Stunden. Vier Einakter}}}{\lemma{\textnormal{\emph{es … wirkt}}}\Cendnote{\textnormal{Vgl. \textcolor{blue}{Bahr}s Feuilleton \emph{\textcolor{green}{Lebendige Stunden (Vier Einacter von Arthur Schnitzler:
                        »Lebendige Stunden«, »Die Frau mit dem Dolche«, »Die letzten Masken« und
                        »Literatur«. Zum ersten Male aufgeführt im Carl-Theater am
                           6. Mai 1902. Erste Vorstellung des Berliner Deutschen
                        Theaters)}} und Vgl. A. S.: \emph{Tagebuch}, 11. 9. 1911.}}}\label{K_L01221_1h}.\pend
           \pstart
           Es dankt Dir ſehr{\\[\baselineskip]}Dein{\\[\baselineskip]}\spacefill\mbox{Hermann}\pend
           \leftskip=0em{}\pstart
           \textcolor{pink}{Salzburg}{}\ledrightnote{\textcolor{pink}{Salzburg}}{ }20. 5.\pend
           \endnumbering\briefempfaengerindex{Schnitzler, Arthur@\textsc{Schnitzler, Arthur}!zzzBahr, Hermann@\emph{von Hermann Bahr}!1902-05-201@{20. 5. 1902}|)be}\mylabel{h}  \normalsize

\doendnotes{C}
\bigskip
\vfill

\clearpage

\footnotesize

\lohead{\textsc{register}}

% Definiere theindex-Environment komplett neu ohne reledmac
\makeatletter
\renewenvironment{theindex}{%
  \section*{\indexname}%
  \setlength{\parindent}{0pt}%
  \setlength{\parskip}{0pt plus 0.3pt}%
  \let\item\@idxitem
}{%
  \clearpage
}
\makeatother

\IfFileExists{\jobname-pw.ind}{\input{\jobname-pw.ind}}{}

\end{document}

      