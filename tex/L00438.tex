%% latex-korrekturansicht-vorspann.tex
%% Vorspann für die Korrekturansicht.
%% Lädt die gemeinsame Datei latex-vorspann.tex mit gesetztem Schalter.

\newif\ifkorrekturansicht
\korrekturansichttrue

\input{../tex-inputs/latex-vorspann}


               \section[Hermann Bahr an Arthur Schnitzler, {[}8.? 5. 1895{]}]{ Hermann Bahr an Arthur Schnitzler, {[}8.? 5. 1895{]}}\nopagebreak\mylabel{v}\rehead{ }\normalsize\beginnumbering\briefempfaengerindex{Schnitzler, Arthur@\textsc{Schnitzler, Arthur}!zzzBahr, Hermann@\emph{von Hermann Bahr}!1895-05-081@{{[}8.? 5. 1895{]}}|(be} \toendnotes[C]{\smallbreak\pagebreak[2]} \Standort{CUL, Schnitzler, B 5b.}
\physDesc{Brief, 1 Blatt, 1 Seite
\newline{}Handschrift: schwarze Tinte, deutsche Kurrent
\newline{}Schnitzler: mit Bleistift datiert: »\textcolor{gray}{8}/5 95« \newline{}Ordnung: 1) mit rotem Buntstift von unbekannter Hand nummeriert: »27« 2) mit Bleistift von unbekannter Hand nummeriert: »27«}\buchAbdrucke{\weitereDrucke{Hermann Bahr, Arthur Schnitzler: \emph{Briefwechsel, Aufzeichnungen, Dokumente (1891–1931)}. Hg. Kurt Ifkovits und Martin Anton Müller. Göttingen: \emph{Wallstein} 2018, S. 101.} }\toendnotes[C]{\smallbreak}\pstart
           \noindent{}{\pb}\textcolor{gray}{\textbf{»\textcolor{brown}{Die
                        Zeit}{}\ledrightnote{\textcolor{brown}{Die Zeit. Wiener Wochenschrift}}«}}\hfill \textcolor{gray}{\textbf{\textbf{\textcolor{pink}{Wien}{}\ledrightnote{\textcolor{pink}{Wien}}}, den }}8/10 \textcolor{gray}{\textbf{189}}\pend
           \pstart
           \textcolor{gray}{\textbf{Wiener Wochenſchrift}}\hfill \textcolor{gray}{\textbf{\textcolor{pink}{IX/3, Günthergaſſe 1}{}\ledrightnote{\textcolor{pink}{Günthergasse}}.}}\pend
           \pstart
           \textcolor{gray}{\textbf{\textbf{Herausgeber}:}}{\\}\textcolor{gray}{\textbf{Profeſſor Dr. \textcolor{blue}{I. Singer}{}\ledrightnote{\textcolor{blue}{Isidor Singer}}, \textcolor{blue}{Hermann Bahr}{}\ledrightnote{\textcolor{blue}{Hermann Bahr}},
                        Dr. \textcolor{blue}{Heinrich Kanner}{}\ledrightnote{\textcolor{blue}{Heinrich Kanner}}.}}\pend
           \pstart
           \textcolor{gray}{\textbf{Telephon Nr. 6415.}}\pend
           \pstart\center{}Lieber Thuri!\pend\pstart
           Herzlichen Dank für Deine lieben \label{K_L00438_1v}\edtext{Wünſche}{\lemma{\textnormal{\emph{Wünſche}}}\Cendnote{\textnormal{nicht überliefert. \textcolor{blue}{Schnitzler} dürfte auf die Meldung des
                  Abendblatts der \emph{\textcolor{brown}{Neuen Freien Presse}} vom
                        6. 5. 1895, S. 1 (oder eine vergleichbare
                  Zeitungsnotiz) reagiert haben: »Gestern hat im \textcolor{pink}{Rathhause} die Civiltrauung des Schriftstellers \textcolor{blue}{Hermann \so{Bahr}} mit Fräulein \textcolor{blue}{Rosa \so{Joël}}{ }stattgefunden. Beistände des Bräutigams waren Herr \textcolor{blue}{Adalbert v. \so{Goldschmidt}} und Herr Dr. \textcolor{blue}{Heinrich \so{Müller}}.« \textcolor{blue}{Bahr} lebte mit ihr bis
                  zur Jahrhundertwende in gemeinsamem Haushalt. 1909 wurde die
                  Scheidung erwirkt.}}}\label{K_L00438_1h} von\pend
           \pstart
           Deinem alten{\\[\baselineskip]}\spacefill\mbox{Hermann}\pend
           \leftskip=0em{}\endnumbering\briefempfaengerindex{Schnitzler, Arthur@\textsc{Schnitzler, Arthur}!zzzBahr, Hermann@\emph{von Hermann Bahr}!1895-05-081@{{[}8.? 5. 1895{]}}|)be}\mylabel{h}  \normalsize

\doendnotes{C}
\bigskip
\vfill

\clearpage

\footnotesize

\lohead{\textsc{register}}

% Definiere theindex-Environment komplett neu ohne reledmac
\makeatletter
\renewenvironment{theindex}{%
  \section*{\indexname}%
  \setlength{\parindent}{0pt}%
  \setlength{\parskip}{0pt plus 0.3pt}%
  \let\item\@idxitem
}{%
  \clearpage
}
\makeatother

\IfFileExists{\jobname-pw.ind}{\input{\jobname-pw.ind}}{}

\end{document}

      