%% latex-korrekturansicht-vorspann.tex
%% Vorspann für die Korrekturansicht.
%% Lädt die gemeinsame Datei latex-vorspann.tex mit gesetztem Schalter.

\newif\ifkorrekturansicht
\korrekturansichttrue

\input{../tex-inputs/latex-vorspann}


               \section[Thomas und Katia Mann an Arthur Schnitzler, 23. 5. 1926]{ Thomas und Katia Mann an Arthur Schnitzler,
                    23. 5. 1926}\nopagebreak\mylabel{v}\rehead{ }\normalsize\beginnumbering\briefempfaengerindex{Schnitzler, Arthur@\textsc{Schnitzler, Arthur}!zzzMann, Katia@\emph{von Katia Mann}!1926-05-231@{23. 5. 1926}|(be}\briefempfaengerindex{Schnitzler, Arthur@\textsc{Schnitzler, Arthur}!zzzMann, Thomas@\emph{von Thomas Mann}!1926-05-231@{23. 5. 1926}|(be} \toendnotes[C]{\smallbreak\pagebreak[2]} \Standort{CUL, Schnitzler, B 67.}
\physDesc{Bildpostkarte
\newline{}Handschrift: schwarze Tinte, deutsche Kurrent\newline{}Versand: Stempel: »\nobreak{}\oindex{Arosa@\textbf{Arosa}, \emph{https://www.geonames.org/ontologyP.PPL}|pwk}Arosa, 24. V. 26, 12\nobreak{}«.  
\newline{}Schnitzler: mit rotem Buntstift eine Unterstreichung }\buchAbdrucke{\weitereDrucke{Hertha Krotkoff: \emph{Arthur Schnitzler – Thomas Mann: Briefe.} In: \emph{Modern Austrian Literature}, Jg. 7 (1974) Nr. 1/2, S. 24.} }\pstart{}{\pb}\textsc{Herrn}\pend{}\pstart{}\textsc{Dr. Arthur Schnitzler}\pend{}\pstart{}\textsc{\textcolor{pink}{Wien}{}\ledrightnote{\textcolor{pink}{Wien}}}\pend{}\pstart{}\textsc{\textcolor{pink}{Sternwartstr.}{}\ledrightnote{\textcolor{pink}{Sternwartestraße}}}\pend{}{\bigskip}\pstart
           \noindent{}\centering{}\textcolor{gray}{\textbf{{\pb}\textcolor{pink}{Arosa}{}\ledrightnote{\textcolor{pink}{Arosa}}. \textcolor{pink}{Bergkirchli}{}\ledrightnote{\textcolor{pink}{Bergkirchli}} m. Weisshorn.}}\pend
           \pstart
           {\pb}\textcolor{pink}{Aroſa}{}\ledrightnote{\textcolor{pink}{Arosa}} den
                            23. V. 26.\pend
           \pstart{}Lieber und verehrter Dr. Schnitzler,\pend\pstart
           hingeriſſen haben wir die \textcolor{green}{Traumnovelle}{}\ledrightnote{\textcolor{green}{Traumnovelle}} hier
                    geleſen, beide in einem Zuge, atemlos, und begrüßen Sie voller Bewunderung.\pend
           \pstart \spacefill\mbox{Thomas und Katja Mann.}\pend{}\endnumbering\briefempfaengerindex{Schnitzler, Arthur@\textsc{Schnitzler, Arthur}!zzzMann, Katia@\emph{von Katia Mann}!1926-05-231@{23. 5. 1926}|)be}\briefempfaengerindex{Schnitzler, Arthur@\textsc{Schnitzler, Arthur}!zzzMann, Thomas@\emph{von Thomas Mann}!1926-05-231@{23. 5. 1926}|)be}\mylabel{h}  \normalsize

\doendnotes{C}
\bigskip
\vfill

\clearpage

\footnotesize

\lohead{\textsc{register}}

% Definiere theindex-Environment komplett neu ohne reledmac
\makeatletter
\renewenvironment{theindex}{%
  \section*{\indexname}%
  \setlength{\parindent}{0pt}%
  \setlength{\parskip}{0pt plus 0.3pt}%
  \let\item\@idxitem
}{%
  \clearpage
}
\makeatother

\IfFileExists{\jobname-pw.ind}{\input{\jobname-pw.ind}}{}

\end{document}

      