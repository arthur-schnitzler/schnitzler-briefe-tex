%% latex-korrekturansicht-vorspann.tex
%% Vorspann für die Korrekturansicht.
%% Lädt die gemeinsame Datei latex-vorspann.tex mit gesetztem Schalter.

\newif\ifkorrekturansicht
\korrekturansichttrue

\input{../tex-inputs/latex-vorspann}


               \section[Georg Brandes an Arthur Schnitzler, 6. 10. 1896]{ Georg Brandes an Arthur Schnitzler, 6. 10. 1896}\nopagebreak\mylabel{v}\rehead{ }\normalsize\beginnumbering\briefempfaengerindex{Schnitzler, Arthur@\textsc{Schnitzler, Arthur}!zzzBrandes, Georg@\emph{von Georg Brandes}!1896-10-061@{6. 10. 1896}|(be} \toendnotes[C]{\smallbreak\pagebreak[2]} \Standort{CUL, Schnitzler, B 17.}
\physDesc{Postkarte
\newline{}Handschrift: blaue Tinte, lateinische Kurrent\newline{}Versand: 1) Stempel: »\nobreak{}\oindex{Kopenhagen@\textbf{Kopenhagen}, \emph{Besiedelter Ort (A.BSO)}|pwk}Kjobenhavn, 6. 10.96, 5–5E\nobreak{}«.  2) Stempel: »\nobreak{}\oindex{III., Landstrasse@\textbf{III., Landstraße}, \emph{Bezirk (A.BZK)}|pwk}Wien 3/3, 8. 10.96, 8.V\nobreak{}«. \newline{}Ordnung: von unbekannter Hand nummeriert: »3« }\buchAbdrucke{\weitereDrucke{Georg Brandes, Arthur Schnitzler: \emph{Ein Briefwechsel}. Hg. Kurt Bergel. Bern: \emph{Francke} 1956, S. 58.} }\toendnotes[C]{\smallbreak}\pstart{}{\pb}Herrn Dr. Arthur
                        Schnitzler\pend{}\pstart{}\textcolor{pink}{Frankgasse 1}{}\ledrightnote{\textcolor{pink}{Frankgasse}}\pend{}\pstart{}\textcolor{pink}{Wien IX}{}\ledrightnote{\textcolor{pink}{IX., Alsergrund}}\pend{}{\bigskip}\pstart
           \raggedleft{}{\pb}\textcolor{pink}{Kopenhagen}{}\ledrightnote{\textcolor{pink}{Kopenhagen}}{ }6 Oct\pend
           \pstart
           Lieber Herr Schnitzler! Könnten Sie mir nicht ein Bischen zu
                    Hülfe kommen. Mir wird ein Numero der \textcolor{brown}{\uline{Zeit}}{}\ledrightnote{\textcolor{brown}{Die Zeit. Wiener Wochenschrift}} geschickt, worin als von mir
                    eingesandt ein \textcolor{green}{Bruchstück}{}\ledrightnote{→\textcolor{green}{Censur in Polen}}
                    meines alten \textcolor{green}{Buches}{}\ledrightnote{→\textcolor{green}{Polen}} über \textcolor{pink}{Polen}{}\ledrightnote{\textcolor{pink}{Polen}}
               sich
                    findet. Es ist vor \uline{10} Jahren herausgegeben, und
                    die Zeitangaben passen darauf; nun steht es da als von \uline{heute}
               stammend. Wenn
                    ich doch wenigstens eine Correctur dieser Sachen sähe! Es wimmelt von
                    Missverständnissen. Die Fehler sind derart dass das \textcolor{pink}{dänische}{}\ledrightnote{\textcolor{pink}{Dänemark}} Wort \uline{Rædsel} (horror,
                    horreur, Schrecken) übersetzt ist \uline{Räthsel}.\hspace*{5em} Ich erfahre, dass kürzlich in \textcolor{pink}{Berlin}{}\ledrightnote{\textcolor{pink}{Berlin}} ein Buch mit meinem Namen versehen erschienen ist
                        \textcolor{green}{\uline{Aus dem Reiche des Absolutismus}}{}\ledrightnote{\textcolor{green}{Eindrücke aus Russland}} (!)
                    Welcher Titel. Es sind wohl meine »\textcolor{green}{Eindrücke aus
                        Rusland}{}\ledrightnote{\textcolor{green}{Eindrücke aus Russland}}«. Es ist mir nicht geschickt worden. \introOben{}Es ist
                        der 9\textsuperscript{te} nicht autorisirte Band von mir in Einem
                        Jahre.\introOben{}\pend
           \pstart Ihr ergebener \spacefill\mbox{Georg Brandes}\pend{}\endnumbering\briefempfaengerindex{Schnitzler, Arthur@\textsc{Schnitzler, Arthur}!zzzBrandes, Georg@\emph{von Georg Brandes}!1896-10-061@{6. 10. 1896}|)be}\mylabel{h}  \normalsize

\doendnotes{C}
\bigskip
\vfill

\clearpage

\footnotesize

\lohead{\textsc{register}}

% Definiere theindex-Environment komplett neu ohne reledmac
\makeatletter
\renewenvironment{theindex}{%
  \section*{\indexname}%
  \setlength{\parindent}{0pt}%
  \setlength{\parskip}{0pt plus 0.3pt}%
  \let\item\@idxitem
}{%
  \clearpage
}
\makeatother

\IfFileExists{\jobname-pw.ind}{\input{\jobname-pw.ind}}{}

\end{document}

      