%% latex-korrekturansicht-vorspann.tex
%% Vorspann für die Korrekturansicht.
%% Lädt die gemeinsame Datei latex-vorspann.tex mit gesetztem Schalter.

\newif\ifkorrekturansicht
\korrekturansichttrue

\input{../tex-inputs/latex-vorspann}


               \section[Arthur und Olga Schnitzler an Richard Beer-Hofmann, 22. 5. 1910]{ Arthur und Olga Schnitzler an Richard Beer-Hofmann, 22. 5. 1910}\nopagebreak\mylabel{v}\rehead{ }\normalsize\beginnumbering\briefempfaengerindex{Beer-Hofmann, Richard@\textsc{Beer-Hofmann, Richard}!zzzSchnitzler, Olga@\emph{von Olga Schnitzler}!1910-05-221@{22. 5. 1910}|(be}\briefempfaengerindex{Beer-Hofmann, Richard@\textsc{Beer-Hofmann, Richard}!zzzSchnitzler, Arthur@\emph{von Arthur Schnitzler}!1910-05-221@{22. 5. 1910}|(be} \toendnotes[C]{\smallbreak\pagebreak[2]} \Standort{YCGL, MSS 31.}
\physDesc{Bildpostkarte
\newline{}Handschrift Arthur Schnitzler: Bleistift, deutsche Kurrent\newline{}Handschrift Olga Schnitzler: Bleistift, lateinische Kurrent\newline{}Versand: 1) Stempel: »\nobreak{}\oindex{Buergenstock@\textbf{Bürgenstock}, \emph{Berg (N.BRG)}|pwk}Bürgenstock b. Luzern
                                       Hammetschwand-Lift, 1132 m. {[}ü.{]} M. Höhe des Lifts
                                       170 m.\nobreak{}«.  2) Stempel: »\nobreak{}\oindex{Luzern@\textbf{Luzern}, \emph{Besiedelter Ort (A.BSO)}|pwk}Luzern, 22 V 10, 9\nobreak{}«. \newline{}Ordnung: mit Bleistift von unbekannter Hand datiert: »22. 5.« }\toendnotes[C]{\smallbreak}\pstart{}{\pb}\textsc{Dr. Richard Beer-Hofma{\geminationn}}\pend{}\pstart{}\textcolor{pink}{Wien XVIII}{}\ledrightnote{\textcolor{pink}{XVIII., Währing}}\pend{}\pstart{}\textsc{\textcolor{pink}{Hasenauerstr 59}{}\ledrightnote{\textcolor{pink}{Hasenauerstraße}}}\pend{}{\bigskip}\pstart
           \noindent{}\centering{}{\pb}\textcolor{gray}{\textbf{\textbf{\label{T_L01933-1v}\edtext{\textcolor{pink}{Bürgenstock}{}\ledrightnote{\textcolor{pink}{Bürgenstock}}}{\lemma{\textnormal{\emph{Bürgenstock}}}\Cendnote{\textnormal{die Bilderläuterung zweimal auf
                           der Karte; bei jener auf der Adressseite von \textcolor{blue}{Schnitzler} die Ortsangabe unterstrichen}}}\label{T_L01933-1h}.}
                     Ausblick vom \textcolor{pink}{Känzeli}{}\ledrightnote{\textcolor{pink}{Känzeli}} am \textcolor{pink}{Felsenweg}{}\ledrightnote{\textcolor{pink}{Felsenweg}} auf \textcolor{pink}{Rigi}{}\ledrightnote{\textcolor{pink}{Rigi}} u. \textcolor{pink}{Vierwaldstättersee}{}\ledrightnote{\textcolor{pink}{Vierwaldstättersee}}.}}\pend
           \pstart
           {\pb}Wär’ wieder was für Sie! Wundervoll! Reiſen Sie mit
                  \textcolor{blue}{Paula}{}\ledrightnote{\textcolor{blue}{Paula Beer-Hofmann}}{ }\substVorne{}\textsuperscript{\textcolor{gray}{auch}}\substDazwischen{}nach\substHinten{}{ }\textcolor{pink}{Luzern}{}\ledrightnote{\textcolor{pink}{Luzern}} und beſichtigen die Ufer. (\textcolor{pink}{Wien}{}\ledrightnote{\textcolor{pink}{Wien}}–\textcolor{pink}{Zürich}{}\ledrightnote{\textcolor{pink}{Zürich}}, Schlafwagen 2. \textsc{Classe}, (wie wir) ab \textcolor{pink}{Wien}{}\ledrightnote{\textcolor{pink}{Wien}} 8
               Abend, \textcolor{pink}{Zürich}{}\ledrightnote{\textcolor{pink}{Zürich}} 2 Uhr \introOben{}Mg\introOben{}, \textcolor{pink}{Luzern}{}\ledrightnote{\textcolor{pink}{Luzern}} 4.30.)\pend
           \pstart
           Das wünſcht Ihnen{\\[\baselineskip]}herzlichſt Ihr \spacefill\mbox{A.}{\\[\baselineskip]}{[}hs. O. Schnitzler:{]} und \spacefill\mbox{Olga.}\pend
           \leftskip=0em{}\endnumbering\briefempfaengerindex{Beer-Hofmann, Richard@\textsc{Beer-Hofmann, Richard}!zzzSchnitzler, Olga@\emph{von Olga Schnitzler}!1910-05-221@{22. 5. 1910}|)be}\briefempfaengerindex{Beer-Hofmann, Richard@\textsc{Beer-Hofmann, Richard}!zzzSchnitzler, Arthur@\emph{von Arthur Schnitzler}!1910-05-221@{22. 5. 1910}|)be}\mylabel{h}  \normalsize

\doendnotes{C}
\bigskip
\vfill

\clearpage

\footnotesize

\lohead{\textsc{register}}

% Definiere theindex-Environment komplett neu ohne reledmac
\makeatletter
\renewenvironment{theindex}{%
  \section*{\indexname}%
  \setlength{\parindent}{0pt}%
  \setlength{\parskip}{0pt plus 0.3pt}%
  \let\item\@idxitem
}{%
  \clearpage
}
\makeatother

\IfFileExists{\jobname-pw.ind}{\input{\jobname-pw.ind}}{}

\end{document}

      