%% latex-korrekturansicht-vorspann.tex
%% Vorspann für die Korrekturansicht.
%% Lädt die gemeinsame Datei latex-vorspann.tex mit gesetztem Schalter.

\newif\ifkorrekturansicht
\korrekturansichttrue

\input{../tex-inputs/latex-vorspann}


               \section[Gerty von Hofmannsthal an Arthur Schnitzler, 9. 4. 1930]{ Gerty von Hofmannsthal an Arthur Schnitzler, 9. 4. 1930}\nopagebreak\mylabel{v}\rehead{ }\normalsize\beginnumbering\briefempfaengerindex{Schnitzler, Arthur@\textsc{Schnitzler, Arthur}!zzzHofmannsthal, Gertrude von@\emph{von Gertrude von Hofmannsthal}!1930-04-091@{9. 4. 1930}|(be} \toendnotes[C]{\smallbreak\pagebreak[2]} \Standort{CUL, Schnitzler, B 43.}
\physDesc{Brief, 1 Blatt (Briefpapier mit Trauerrand), 2 Seiten
\newline{}Schreibmaschine
\newline{}Handschrift: schwarze Tinte, deutsche Kurrent (\noindent{}Unterschrift)
\newline{}Schnitzler: mit rotem Buntstift mehrere Unterstreichungen }\toendnotes[C]{\smallbreak}\pstart
           \noindent{}{\pb}IV Mozartgasse 4\hfill \textcolor{pink}{Wien}{}\ledrightnote{\textcolor{pink}{Wien}} d. 9/IV 30\pend
           \pstart
           Telephon U 43384\pend
           \pstart
           Lieber Arthur, ich habe heute versucht Sie anzurufen hörte
                    aber, dass Sie eine andere \label{K_L02535_1v}\edtext{Geheimnummer}{\lemma{\textnormal{\emph{Geheimnummer}}}\Cendnote{\textnormal{vgl. Arthur Schnitzler an Gerty von Hofmannsthal, 17. 2. 1931}}}\label{K_L02535_1h} haben, wahrscheinlich sind Sie zu viel angerufen worden, darum sage ich
                    Ihnen heute meine Bitte schriftlich\pend
           \pstart
           Mein Advokat Dr \textcolor{blue}{Weinmann}{}\ledrightnote{\textcolor{blue}{Leonhard Weinmann}} würde so sehr eine
                    Unterredung mit Ihnen wünschen, es handelt sich wegen der Erbsteuer um
                    Bestimmung der Autoreneinkünfte, die man möglichst gering angeben muss, weil es
                    als Kapital angesehen wird (was wirklich recht ungerecht ist, finde ich dass es
                    doch sicher sehr schwankend sein wird) Ich konnte Dr \textcolor{blue}{W.}{}\ledrightnote{\textcolor{blue}{Leonhard Weinmann}} niemanden andern nennen als Sie, als bester Freund und
                    auch als Autor, der competent ist seine Meinung zu sagen. Was die Opern betrifft
                    hat \textcolor{blue}{Schalk}{}\ledrightnote{\textcolor{blue}{Franz Schalk}} eine Art Gutachten gegeben.
                        Dr \textcolor{blue}{W.}{}\ledrightnote{\textcolor{blue}{Leonhard Weinmann}} wird Ihnen das alles besser
                    erklären können als ich. Wollen Sie also die grosse Güte haben den Mann einmal
                    in nächster Zeit zu \label{T_L02535_1v}\edtext{einer}{\lemma{\textnormal{\emph{einer}}}\Cendnote{\textnormal{Sie schreibt:
                        »einen«}}}\label{T_L02535_1h} Ihnen passenden Stunde zu empfangen?
                    Natürlich müsste ich es einige Tage früher wissen, da der \textcolor{blue}{Mann}{}\ledrightnote{\textcolor{blue}{Leonhard Weinmann}} sehr beschäftigt ist und auch oft Verhandlungen hat.
                    Bitte rufen Sie mich einmal zwischen 10–11 vorm an, wo ich fast
                    immer zuhaus bin und lassen Sie mich ein Wort wissen.\pend
           \pstart
           Ich war drei Wochen in \textcolor{pink}{Berlin}{}\ledrightnote{\textcolor{pink}{Berlin}}, habe \textcolor{blue}{Olga}{}\ledrightnote{\textcolor{blue}{Olga Schnitzler}} gesehen, die ich sehr wohl fand und war
                    entzückt über die Wohnung, die ich so besonders geschmackvoll fand. \textcolor{blue}{Heini}{}\ledrightnote{\textcolor{blue}{Heinrich Schnitzler}} konnte ich leider nicht sehen. \textcolor{blue}{Raimund}{}\ledrightnote{\textcolor{blue}{Raimund von Hofmannsthal}} ist jetzt bis auf \label{T_L02535_2v}\edtext{weiteres}{\lemma{\textnormal{\emph{weiteres}}}\Cendnote{\textnormal{Sie schreibt: »wieteres«}}}\label{T_L02535_2h} in \textcolor{pink}{Berlin}{}\ledrightnote{\textcolor{pink}{Berlin}} bei einer Filmsache und ich glaube dass
                    es aussichtsreich ist. Ich selbst bin seit gestern in der neuen Wohnung und
                    gewöhne mich langsam. Es hat gegenüber der \textcolor{pink}{Stallburggasse}{}\ledrightnote{\textcolor{pink}{Stallburggasse}} viele Vorteile.\pend
           \pstart
           Ich hoffe Sie schauen sichs einmal an. Sie werden viele bekannte Dinge hier
                    vorfinden, die Sie an die \textcolor{blue}{Elternwohnung}{}\ledrightnote{\textcolor{blue}{Hugo August von Hofmannsthal}{\newline}\textcolor{blue}{Anna von Hofmannsthal}} und an \textcolor{blue}{Hugo}{}\ledrightnote{\textcolor{blue}{Hugo von Hofmannsthal}}{ }{\pb}erinnern werden! – – – – alles das
                    ist ja so traurig!\pend
           \pstart
           Viel Herzliches{\\[\baselineskip]}Ihre{\\[\baselineskip]}\spacefill\mbox{{[}hs.:{]} Gerty}\pend
           \leftskip=0em{}\endnumbering\briefempfaengerindex{Schnitzler, Arthur@\textsc{Schnitzler, Arthur}!zzzHofmannsthal, Gertrude von@\emph{von Gertrude von Hofmannsthal}!1930-04-091@{9. 4. 1930}|)be}\mylabel{h}  \normalsize

\doendnotes{C}
\bigskip
\vfill

\clearpage

\footnotesize

\lohead{\textsc{register}}

% Definiere theindex-Environment komplett neu ohne reledmac
\makeatletter
\renewenvironment{theindex}{%
  \section*{\indexname}%
  \setlength{\parindent}{0pt}%
  \setlength{\parskip}{0pt plus 0.3pt}%
  \let\item\@idxitem
}{%
  \clearpage
}
\makeatother

\IfFileExists{\jobname-pw.ind}{\input{\jobname-pw.ind}}{}

\end{document}

      