%% latex-korrekturansicht-vorspann.tex
%% Vorspann für die Korrekturansicht.
%% Lädt die gemeinsame Datei latex-vorspann.tex mit gesetztem Schalter.

\newif\ifkorrekturansicht
\korrekturansichttrue

\input{../tex-inputs/latex-vorspann}


               \section[Hugo von Hofmannsthal an Arthur Schnitzler, 5. 8. {[}1912{]}]{ Hugo von Hofmannsthal an Arthur Schnitzler, 5. 8. {[}1912{]}}\nopagebreak\mylabel{v}\rehead{ }\normalsize\beginnumbering\briefempfaengerindex{Schnitzler, Arthur@\textsc{Schnitzler, Arthur}!zzzHofmannsthal, Hugo von@\emph{von Hugo von Hofmannsthal}!1912-08-052@{5. 8. {[}1912{]}}|(be} \toendnotes[C]{\smallbreak\pagebreak[2]} \Standort{CUL, Schnitzler, B 43.}
\physDesc{Briefkarte
\newline{}Handschrift: schwarze Tinte, deutsche Kurrent
\newline{}Schnitzler: mit Bleistift die Jahreszahl ergänzt: »912« \newline{}Ordnung: 1) mit Bleistift von unbekannter Hand nummeriert: »\strikeout{329}« 2) mit Bleistift von unbekannter Hand nummeriert: »339«}\buchAbdrucke{\weitereDrucke{Hugo von Hofmannsthal, Arthur Schnitzler: \emph{Briefwechsel}. Hg. Therese Nickl und Heinrich Schnitzler. Frankfurt am Main: \emph{S. Fischer} 1964, S. 268.} }\toendnotes[C]{\smallbreak}\pstart
           \raggedleft{}{\pb}5 VIII.{ }\textcolor{pink}{\textsc{Aussee}}{}\ledrightnote{\textcolor{pink}{Bad Aussee}}.\pend
           \pstart{}mein lieber Arthur\pend\pstart
           ich bin froh, aus Ihren Karten zu ſehen daſs es Euch gut geht. Uns gehts auch gut.
               Mir iſt dieſe Landſchaft die ſchönſte und liebſte, und daſs hie und da Leute ſind,
               die man kennt, tut mir auch nichts, man iſt dennoch ſo viel allein und ſo meilenweit
               von ihnen als man will. Mir iſt ſchon {\pb}Jahre lang nicht ſo viel und
               vielerlei eingefallen, macht man auch nicht alles ſo iſt das Einfallen doch ein
               großes Vergnügen.\pend
           \pstart
           Unter andern Büchern les ich den \textcolor{green}{\textcolor{blue}{Varnhagen}{}\ledrightnote{\textcolor{blue}{Karl August von Varnhagen-Ense}}}{}\ledrightnote{→\textcolor{green}{Tagebücher}}, finde ihn äuſſerſt intereſſant.\hspace*{1.5em}Kommt doch im
                  September hier vorbei, ich ſag wieder mein Sprücherl: man wird auf
               einmal todt ſein und dann wird einem \uline{sehr} leid ſein
               daſs man ſich nicht öfter geſehen hat. \label{T_L02083_1v}\edtext{Schreiben Sie wieder einmal ein
               kleines Karterl}{\lemma{\textnormal{\emph{Schreiben … Karterl}}}\Cendnote{\textnormal{quer am linken Rand}}}\label{T_L02083_1h}.\pend
           \pstart Ihr \spacefill\mbox{Hugo}\pend{}\pstart
           \noindent{}\label{T_L02083_2v}\edtext{Viele Grüße
                     \textcolor{blue}{Olga}{}\ledrightnote{\textcolor{blue}{Olga Schnitzler}} von uns beiden.}{\lemma{\textnormal{\emph{Viele … beiden.}}}\Cendnote{\textnormal{quer am rechten Rand der ersten
                     Seite}}}\label{T_L02083_2h}\pend
           \endnumbering\briefempfaengerindex{Schnitzler, Arthur@\textsc{Schnitzler, Arthur}!zzzHofmannsthal, Hugo von@\emph{von Hugo von Hofmannsthal}!1912-08-052@{5. 8. {[}1912{]}}|)be}\mylabel{h}  \normalsize

\doendnotes{C}
\bigskip
\vfill

\clearpage

\footnotesize

\lohead{\textsc{register}}

% Definiere theindex-Environment komplett neu ohne reledmac
\makeatletter
\renewenvironment{theindex}{%
  \section*{\indexname}%
  \setlength{\parindent}{0pt}%
  \setlength{\parskip}{0pt plus 0.3pt}%
  \let\item\@idxitem
}{%
  \clearpage
}
\makeatother

\IfFileExists{\jobname-pw.ind}{\input{\jobname-pw.ind}}{}

\end{document}

      