%% latex-korrekturansicht-vorspann.tex
%% Vorspann für die Korrekturansicht.
%% Lädt die gemeinsame Datei latex-vorspann.tex mit gesetztem Schalter.

\newif\ifkorrekturansicht
\korrekturansichttrue

\input{../tex-inputs/latex-vorspann}


               \section[Hugo von Hofmannsthal an Arthur Schnitzler, 16. 7. {[}1896{]}]{ Hugo von Hofmannsthal an Arthur Schnitzler, 16. 7. {[}1896{]}}\nopagebreak\mylabel{v}\rehead{ }\normalsize\beginnumbering\briefempfaengerindex{Schnitzler, Arthur@\textsc{Schnitzler, Arthur}!zzzHofmannsthal, Hugo von@\emph{von Hugo von Hofmannsthal}!1896-07-161@{16. 7. {[}1896{]}}|(be} \toendnotes[C]{\smallbreak\pagebreak[2]} \Standort{CUL, Schnitzler, B 43.}
\physDesc{Brief, 2 Blätter (erstes Blatt mit aufgeprägtem Wappen), 8 Seiten
\newline{}Handschrift: schwarze Tinte, deutsche Kurrent
\newline{}Schnitzler: mit Bleistift die Jahreszahl ergänzt: »96« \newline{}Ordnung: mit Bleistift von unbekannter Hand nummeriert:
                                        »78.1« und, am zweiten Blatt,
                                        »78.2« }\buchAbdrucke{\weitereDrucke{Hugo von Hofmannsthal, Arthur Schnitzler: \emph{Briefwechsel}. Hg. Therese Nickl und Heinrich Schnitzler. Frankfurt am Main: \emph{S. Fischer} 1964, S. 69–70.} }\toendnotes[C]{\smallbreak}\pstart
           \raggedleft{}{\pb}\textcolor{pink}{Fusch}{}\ledrightnote{\textcolor{pink}{Bad Fusch}}. 16. Juli.\pend
           \pstart{}mein lieber Arthur!\pend\pstart
           Über das \textcolor{green}{Stück}{}\ledrightnote{→\textcolor{green}{Freiwild. Schauspiel in 3 Akten}} hab ich öfter nachgedacht, bin
                    aber nicht über gewiſſe allgemeine Gedanken hinausgekommen. Ich fahre morgen
                    nach \textcolor{pink}{Salzburg}{}\ledrightnote{\textcolor{pink}{Salzburg}} und bin dort 2, 3 Tage mit
                        \textcolor{blue}{Richard}{}\ledrightnote{\textcolor{blue}{Richard Beer-Hofmann}} zuſa{\geminationm}en. Dann geht er nach \textcolor{pink}{Dänemark}{}\ledrightnote{\textcolor{pink}{Dänemark}}, ich nach
                        \textcolor{pink}{Auſſee}{}\ledrightnote{\textcolor{pink}{Altaussee}}. Vielleicht finden wir zuſammen
                    etwas Geſcheites.\pend
           \pstart
           {\pb}Der 2\textsuperscript{te}{ }\textcolor{green}{Act}{}\ledrightnote{→\textcolor{green}{Freiwild. Schauspiel in 3 Akten}} muſs alles wirklich Dramatiſche
                    enthalten, alle Wucht, alles Pathos, alle Grauſamkeit und alle\strikeout{s} innere Verſöhnung, dann ſind der erſte und
                    dritte Act, die den Vorgang nur von außen zeigen und an denen ſich ohne
                    Verderbnis nicht viel verinnerlichen läſst, gerechtfertigt und gerettet, wie
                        \textcolor{pink}{japaniſche}{}\ledrightnote{\textcolor{pink}{Japan}} Laternen wenn man hinter ihren Bildern ein Licht anzündet. Es liegt
                        {\pb}nun im Weſen ihrer
                    Compoſition, daſs Ihnen gerade Wucht und das Schickſalmäßige, Unabwendbare
                    ſchwer wird: (beſonders wenn nicht eine weibliche Figur das ganze trägt.)
                    Deswegen muſs aber gerade hier die Frauenrolle ausgenützt werden (jetzt läuft
                    ſie nutzlos, ja ſtörend dazwiſchen): der gehaltene Ton, den der Held allen
                    Männern gegenüber hat, kann dem Mädchen gegenüber ſo völlig wegfallen wie etwa
                    in einem Monolog: es liegt ſogar eine natürliche tiefe Coquetterie darin, {\pb}vor der geliebten Frau die
                    Schwere und grauſame Sonderbarkeit einer Situation einzuſehen und einzugeſtehen.
                    (Das entgegengeſetzte, viel dürftigere Motiv war das Verheimlichen in der
                        »\textcolor{green}{Liebelei}{}\ledrightnote{\textcolor{green}{Liebelei. Schauspiel in drei Akten}}«)\pend
           \pstart
           An ſich, von außen geſehen (ſo wie der \textcolor{green}{erſte}{}\ledrightnote{→\textcolor{green}{Freiwild. Schauspiel in 3 Akten}}
                    und \textcolor{green}{dritte Act}{}\ledrightnote{→\textcolor{green}{Freiwild. Schauspiel in 3 Akten}} es bringt) ſind ja alle
                    Tragödien des Lebens nur unangenehme Begebenheiten, die mit einem Unglücksfall
                    enden. Die Tragik muß man (und \uuline{darf} man!!) in
                    die Auffaſſung legen, welche die Hauptperſon von der durch ſie angeſtifteten {\pb}innerlich unrettbaren \substVorne{}\textsuperscript{Perſon}{\allowbreak}\substDazwischen{}Situation\substHinten{} plötzlich zu haben anfängt, dagegen ankämpft, und ſchließlich darein
                    verſinkt wie ein Ertrinkender. Nun haben Sie einmal (beim Erfinden des \textcolor{green}{Stoffes}{}\ledrightnote{→\textcolor{green}{Freiwild. Schauspiel in 3 Akten}}) die durch das verweigerte Duell für
                    eine beſtimmte Art Menſchen ſich ergebende Situation als tragiſch, d. h. als
                    einen tiefen unlösbaren inneren Widerſpruch in ſich tragend erkannt: {\pb}ſuchen Sie dieſe Stimmung
                    wiederherzuſtellen. Sie war wahrſcheinlich rhetoriſch. Individualiſieren Sie
                    dieſe Rhetorik und legen Sie ſie der Hauptperſon in den Mund, verſtärken und
                    verdichten Sie ſie (reine Rhetorik iſt immer dünn) durch retardierende,
                    menſchliche, zuſtändliche Motive (wie Sie in der \textcolor{green}{Liebelei}{}\ledrightnote{\textcolor{green}{Liebelei. Schauspiel in drei Akten}} ein faſt-nichts von Vorgang aufgeſchwemmt haben und ihm
                    Dichte gegeben) {\pb}und fürchten
                    Sie ſich nicht vor Ihrem eigenen Feuer. Es wird nie nackt brennen, da immer die
                    bunten Schirme des erſten und letzten Actes davorſtehen werden. Die Schwäche und
                    Zaghaftigkeit im Ton des \textcolor{green}{2\textsuperscript{ten} Actes}{}\ledrightnote{→\textcolor{green}{Freiwild. Schauspiel in 3 Akten}} (vergleichen Sie mit \textcolor{blue}{\textsc{Shakeſpeare}}{}\ledrightnote{\textcolor{blue}{William Shakespeare}}!) iſt nur entſtanden, weil der Held und das Mädel ſo furchtbar wenig
                    individualiſiert ſind: in einem papierdünnen Herd kann man dann freilich kein
                    großes Feuer anmachen. {\pb}Verſtärken \introOben{}(d. h. determinieren)\introOben{}
               Sie das Verhältnis
                    zu dem Mädel, ſo wird es nicht nur ſich ſelbſt tragen ſondern die ganzen
                    tragiſchen Eingeſtändniſſe und Irrläufe des Helden werden darauf ruhen können,
                    und ganz ohne Künſtelei. Nur müſſen Sie ſich hüten, das Verhältnis \uline{übermäßig} zu individualiſieren; ſo fern von der
                        \textcolor{green}{Anatol}{}\ledrightnote{\textcolor{green}{Anatol}}-manier als möglich.\pend
           \pstart
           Womöglich ſo behandelt und geſehen, wie Sie gewöhnlich Nebenfiguren ſehen: mit
                    einer ſcheinbar geringeren Liebe, die aber zuträglicher iſt und mehr Leben
                        giebt.\hspace*{2.5em}Nächſtens etwas anderes.\pend
           \pstart Ihr\spacefill\mbox{Hugo.}\pend{}\endnumbering\briefempfaengerindex{Schnitzler, Arthur@\textsc{Schnitzler, Arthur}!zzzHofmannsthal, Hugo von@\emph{von Hugo von Hofmannsthal}!1896-07-161@{16. 7. {[}1896{]}}|)be}\mylabel{h}  \normalsize

\doendnotes{C}
\bigskip
\vfill

\clearpage

\footnotesize

\lohead{\textsc{register}}

% Definiere theindex-Environment komplett neu ohne reledmac
\makeatletter
\renewenvironment{theindex}{%
  \section*{\indexname}%
  \setlength{\parindent}{0pt}%
  \setlength{\parskip}{0pt plus 0.3pt}%
  \let\item\@idxitem
}{%
  \clearpage
}
\makeatother

\IfFileExists{\jobname-pw.ind}{\input{\jobname-pw.ind}}{}

\end{document}

      