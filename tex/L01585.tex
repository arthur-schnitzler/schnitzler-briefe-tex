%% latex-korrekturansicht-vorspann.tex
%% Vorspann für die Korrekturansicht.
%% Lädt die gemeinsame Datei latex-vorspann.tex mit gesetztem Schalter.

\newif\ifkorrekturansicht
\korrekturansichttrue

\input{../tex-inputs/latex-vorspann}


               \section[Lou Andreas-Salomé an Arthur Schnitzler, 19. 2. 1906]{ Lou Andreas-Salomé an Arthur Schnitzler,
                    19. 2. 1906}\nopagebreak\mylabel{v}\rehead{ }\normalsize\beginnumbering\briefempfaengerindex{Schnitzler, Arthur@\textsc{Schnitzler, Arthur}!zzzAndreas-Salome, Lou@\emph{von Lou Andreas-Salomé}!1906-02-191@{19. 2. 1906}|(be} \toendnotes[C]{\smallbreak\pagebreak[2]} \Standort{CUL, Schnitzler, B 3.}
\physDesc{Postkarte
\newline{}Handschrift: schwarze Tinte, deutsche Kurrent\newline{}Versand: 1) Stempel: »\nobreak{}\oindex{Berlin@\textbf{Berlin}, \emph{https://www.geonames.org/ontologyP.PPLC}|pwk}Berlin W 50, 19. 2. 06, 9 10 N\nobreak{}«.  2) Stempel: »\nobreak{}20. 2. 06\nobreak{}«. 
\newline{}Schnitzler: 1) mit Bleistift datiert: »19/2 06« 2) mit rotem Buntstift eine Unterstreichung\newline{}Ordnung: mit Bleistift von unbekannter Hand nummeriert:
                                        »20« }\toendnotes[C]{\smallbreak}\pstart{}{\pb}Herrn \textsc{D\textsuperscript{r} Arthur Schnitzler}\pend{}\pstart{}\textsc{Berlin C.}\pend{}\pstart{}\textsc{\textcolor{pink}{Hôtel Continental}{}\ledrightnote{\textcolor{pink}{Hotel Continental}}.}\pend{}{\bigskip}\pstart
           \noindent{}{\pb}Lieber Doktor \textsc{Schnitzler}, darf ich
                    Sie um die Erlaubniß bitten, am \label{K_L01585_1v}\edtext{Freitag}{\lemma{\textnormal{\emph{Freitag}}}\Cendnote{\textnormal{Die Generalprobe von \emph{\textcolor{green}{Der Ruf des Lebens}} am \textcolor{pink}{Deutschen Theater in Berlin} fand am Freitag, den
                            23. 2. 1906, statt, die Uraufführung am Folgetag,
                        beide in Anwesenheit \textcolor{blue}{Schnitzlers}.}}}\label{K_L01585_1h} der \textcolor{green}{Generalprobe}{}\ledrightnote{→\textcolor{green}{Der Ruf des Lebens. Schauspiel in drei Akten}} beiwohnen zu dürfen? Wenn Sie »Ja« dazu
                    ſagen, machen Sie mir eine große Freude! Ich glaube, \textcolor{blue}{\textsc{Brahm}}{}\ledrightnote{\textcolor{blue}{Otto Brahm}} würde nichts dagegen haben weil ich ja auch bei \textcolor{blue}{Hauptmann}{}\ledrightnote{\textcolor{blue}{Gerhart Hauptmann}}’ſchen Generalproben öfters (auch letztes
                        \label{K_L01585_2v}\edtext{\textcolor{green}{Mal}{}\ledrightnote{→\textcolor{green}{Und Pippa tanzt!}}}{\lemma{\textnormal{\emph{Mal}}}\Cendnote{\textnormal{\emph{\textcolor{green}{Und Pippa tanzt. Ein Glashüttenmärchen}},
                        hatte am 19. 1. 1906 Uraufführung am \textcolor{pink}{Deutschen Theater}.}}}\label{K_L01585_2h}) zugegen war. Wollen Sie
                    mir's ſchreiben in die \textcolor{pink}{\textsc{Marburger}ſtr. 4}{}\ledrightnote{\textcolor{pink}{Marburger Straße}}, \textcolor{pink}{Hospiz
                        des Weſtens}{}\ledrightnote{\textcolor{pink}{Hospiz des Westens}}? \pend
           \pstart
           In alter Verehrung Ihre{\\[\baselineskip]}\spacefill\mbox{Frau Lou.}\pend
           \leftskip=0em{}\endnumbering\briefempfaengerindex{Schnitzler, Arthur@\textsc{Schnitzler, Arthur}!zzzAndreas-Salome, Lou@\emph{von Lou Andreas-Salomé}!1906-02-191@{19. 2. 1906}|)be}\mylabel{h}  \normalsize

\doendnotes{C}
\bigskip
\vfill

\clearpage

\footnotesize

\lohead{\textsc{register}}

% Definiere theindex-Environment komplett neu ohne reledmac
\makeatletter
\renewenvironment{theindex}{%
  \section*{\indexname}%
  \setlength{\parindent}{0pt}%
  \setlength{\parskip}{0pt plus 0.3pt}%
  \let\item\@idxitem
}{%
  \clearpage
}
\makeatother

\IfFileExists{\jobname-pw.ind}{\input{\jobname-pw.ind}}{}

\end{document}

      