%% latex-korrekturansicht-vorspann.tex
%% Vorspann für die Korrekturansicht.
%% Lädt die gemeinsame Datei latex-vorspann.tex mit gesetztem Schalter.

\newif\ifkorrekturansicht
\korrekturansichttrue

\input{../tex-inputs/latex-vorspann}


               \section[Michael Georg Conrad an Arthur Schnitzler, 22. 1. 1904]{ Michael Georg Conrad an Arthur Schnitzler, 22. 1. 1904}\nopagebreak\mylabel{v}\rehead{ }\normalsize\beginnumbering\briefempfaengerindex{Schnitzler, Arthur@\textsc{Schnitzler, Arthur}!zzzConrad, Michael Georg@\emph{von Michael Georg Conrad}!1904-01-221@{22. 1. 1904}|(be} \toendnotes[C]{\smallbreak\pagebreak[2]} \Standort{CUL, Schnitzler, B 22.}
\physDesc{Postkarte
\newline{}Handschrift: schwarze Tinte, deutsche Kurrent\newline{}Versand: 1) Stempel: »\nobreak{}\oindex{Muenchen@\textbf{München}, \emph{https://www.geonames.org/ontologyP.PPLA}|pwk}München 26, 22 Jan 04, 6–7 N\nobreak{}«.  2) Stempel: »\nobreak{}\oindex{IX., Alsergrund@\textbf{IX., Alsergrund}, \emph{Bezirk (A.BZK)}|pwk}Wien 9/3 73, 23. 1. 04, 11. V\nobreak{}«. 3) Stempel: »\nobreak{}\oindex{I., Innere Stadt@\textbf{I., Innere Stadt}, \emph{Bezirk (A.BZK)}|pwk}Wien 110, 23. 1. 04, 3. N\nobreak{}«. 4) nachgesandt nach: \textcolor{pink}{Spöttelg 7}\hspace*{1.5em}XVIII/I}\toendnotes[C]{\smallbreak}\pstart{}{\pb}Hochwohlgeboren\pend{}\pstart{}Herrn \textsc{D\textsuperscript{r}} Arthur Schnitzler\pend{}\pstart{}Dichter\pend{}\pstart{}\textsc{\textcolor{pink}{Wien XII}{}\ledrightnote{\textcolor{pink}{XII., Meidling}}}.\pend{}\pstart{}\textsc{\textcolor{pink}{Frankgasse 1}{}\ledrightnote{\textcolor{pink}{Frankgasse}}.}\pend{}{\bigskip}\pstart
           \noindent{}{\pb}\textcolor{pink}{München}{}\ledrightnote{\textcolor{pink}{München}}, \textcolor{pink}{Steinsdorfſtr. 7}{}\ledrightnote{\textcolor{pink}{Steinsdorfstraße}}\pend
           \pstart
           \raggedleft{}22. 1. 04.\pend
           \pstart
           Lieber Herr Doktor, ein mediumiſtiſches Schreibweibchen, \textcolor{blue}{Frau Marie Knorr-Schmidt}{}\ledrightnote{\textcolor{blue}{Marie Knorr-Schmidt}} aus \textcolor{pink}{Meerane}{}\ledrightnote{\textcolor{pink}{Meerane}} in \textcolor{pink}{Sachſen}{}\ledrightnote{\textcolor{pink}{Sachsen}}, \textcolor{pink}{Bismarckſtr. 3}{}\ledrightnote{\textcolor{pink}{Innere Crimmitschauer Straße}}, will Sie ein wenig anöden mit
                    Dichteleien aus der vierten Dimenſion. Das \textcolor{green}{Buch}{}\ledrightnote{→\textcolor{green}{Evoë! Ein Schritt zur Lichtung des Seelenlebens}} geht Ihnen heute zu. Bitte, werfen Sie einen
                    Blick hinein. Ich habe nämlich der Dame – um endlich Ruhe zu kriegen –
                    verſprochen, Sie durch inſtändiges Bitten dahin zu bringen, daß Sie einen Blick
                    hineinwerfen. Dann nehmen Sie eine Postkarte und beſtätigen mir: Ich habe einen
                    Blick hineingeworfen. Das genügt. \textsc{Voilà tout}. Der
                    Geiſter-Dichter aus der vierten Dimenſion wird beſchwichtigt und wir können uns
                    wieder wichtigen Dingen widmen. Gruß! \spacefill\mbox{C.}\pend
           \endnumbering\briefempfaengerindex{Schnitzler, Arthur@\textsc{Schnitzler, Arthur}!zzzConrad, Michael Georg@\emph{von Michael Georg Conrad}!1904-01-221@{22. 1. 1904}|)be}\mylabel{h}  \normalsize

\doendnotes{C}
\bigskip
\vfill

\clearpage

\footnotesize

\lohead{\textsc{register}}

% Definiere theindex-Environment komplett neu ohne reledmac
\makeatletter
\renewenvironment{theindex}{%
  \section*{\indexname}%
  \setlength{\parindent}{0pt}%
  \setlength{\parskip}{0pt plus 0.3pt}%
  \let\item\@idxitem
}{%
  \clearpage
}
\makeatother

\IfFileExists{\jobname-pw.ind}{\input{\jobname-pw.ind}}{}

\end{document}

      