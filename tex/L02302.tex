%% latex-korrekturansicht-vorspann.tex
%% Vorspann für die Korrekturansicht.
%% Lädt die gemeinsame Datei latex-vorspann.tex mit gesetztem Schalter.

\newif\ifkorrekturansicht
\korrekturansichttrue

\input{../tex-inputs/latex-vorspann}


               \section[Richard Beer-Hofmann an Arthur Schnitzler, 28. 8. 1918]{ Richard Beer-Hofmann an Arthur Schnitzler,
               28. 8. 1918}\nopagebreak\mylabel{v}\rehead{ }\normalsize\beginnumbering\briefempfaengerindex{Schnitzler, Arthur@\textsc{Schnitzler, Arthur}!zzzBeer-Hofmann, Richard@\emph{von Richard Beer-Hofmann}!1918-08-281@{28. 8. 1918}|(be} \toendnotes[C]{\smallbreak\pagebreak[2]} \Standort{CUL, Schnitzler, B 8.}
\physDesc{Postkarte
\newline{}Handschrift: Bleistift, lateinische Kurrent\newline{}Versand: Stempel: »\nobreak{}\oindex{Bad Ischl@\textbf{Bad Ischl}, \emph{Besiedelter Ort (A.BSO)}|pwk}Bad Ischl, 29. VIII. 18, 5\nobreak{}«.  \newline{}Ordnung: mit Bleistift von unbekannter Hand nummeriert: »267« }\buchAbdrucke{\weitereDrucke{Arthur Schnitzler, Richard Beer-Hofmann: \emph{Briefwechsel 1891–1931}. Hg. Konstanze Fliedl. Wien, Zürich: \emph{Europaverlag} 1992, S. 226.} }\toendnotes[C]{\smallbreak}\pstart{}{\pb}Herrn\pend{}\pstart{}D\textsc{r} Arthur Schnitzler\pend{}\pstart{}\textcolor{pink}{Partenkirchen}{}\ledrightnote{\textcolor{pink}{Partenkirchen}}\pend{}\pstart{}\textcolor{pink}{Haus Tannenberg}{}\ledrightnote{\textcolor{pink}{Haus Tannenberg}}\pend{}{\bigskip}\pstart
           \raggedleft{}{\pb}\textcolor{pink}{Bad-Ischl}{}\ledrightnote{\textcolor{pink}{Bad Ischl}}{ }28. VIII. 18.\pend
           \pstart
           Lieber Arthur! Schade, dass Sie nicht nach \textcolor{pink}{Salzburg}{}\ledrightnote{\textcolor{pink}{Salzburg}} kamen. Über meinen Aufführungstermin wurde erst –
               nachdem wir 10 Tage beisa{\geminationm}en waren, gesprochen, da ich
               nicht fragte. Fest steht \strikeout{erst} nur \introOben{}(wenn es fest steht!\introOben{}): Als erstes: »\textcolor{green}{Wie es
                  Euch gefällt}{}\ledrightnote{\textcolor{green}{Wie es euch gefällt}}«. Als zweites »\textcolor{green}{Jaakobs
               Traum}{}\ledrightnote{\textcolor{green}{Jaákobs Traum. Ein Vorspiel}}«. Alles andere noch unbesti{\geminationm}t.
               Wann wollen Sie wieder in \textcolor{pink}{Wien}{}\ledrightnote{\textcolor{pink}{Wien}} sein? Ich dürfte {\pb}16. od. 17 Sept. kommen. Herzliche Grüsse Ihnen und Ihrer
                  \textcolor{blue}{Frau}{}\ledrightnote{→\textcolor{blue}{Olga Schnitzler}}, und auch Ihrer \textcolor{blue}{Schwägerin}{}\ledrightnote{→\textcolor{blue}{Elisabeth Steinrück}} und \textcolor{blue}{Steinrück}{}\ledrightnote{\textcolor{blue}{Albert Steinrück}}.\pend
           \pstart
           Ihr{\\[\baselineskip]}\spacefill\mbox{Richard}\pend
           \leftskip=0em{}\endnumbering\briefempfaengerindex{Schnitzler, Arthur@\textsc{Schnitzler, Arthur}!zzzBeer-Hofmann, Richard@\emph{von Richard Beer-Hofmann}!1918-08-281@{28. 8. 1918}|)be}\mylabel{h}  \normalsize

\doendnotes{C}
\bigskip
\vfill

\clearpage

\footnotesize

\lohead{\textsc{register}}

% Definiere theindex-Environment komplett neu ohne reledmac
\makeatletter
\renewenvironment{theindex}{%
  \section*{\indexname}%
  \setlength{\parindent}{0pt}%
  \setlength{\parskip}{0pt plus 0.3pt}%
  \let\item\@idxitem
}{%
  \clearpage
}
\makeatother

\IfFileExists{\jobname-pw.ind}{\input{\jobname-pw.ind}}{}

\end{document}

      