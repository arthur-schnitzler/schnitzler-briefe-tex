%% latex-korrekturansicht-vorspann.tex
%% Vorspann für die Korrekturansicht.
%% Lädt die gemeinsame Datei latex-vorspann.tex mit gesetztem Schalter.

\newif\ifkorrekturansicht
\korrekturansichttrue

\input{../tex-inputs/latex-vorspann}


               \section[Lou Andreas-Salomé an Arthur Schnitzler, Juni 1908]{ Lou Andreas-Salomé an Arthur Schnitzler, Juni 1908}\nopagebreak\mylabel{v}\rehead{ }\normalsize\beginnumbering\briefempfaengerindex{Schnitzler, Arthur@\textsc{Schnitzler, Arthur}!zzzAndreas-Salome, Lou@\emph{von Lou Andreas-Salomé}!1908-06-011@{Juni 1908}|(be} \toendnotes[C]{\smallbreak\pagebreak[2]} \Standort{CUL, Schnitzler, B 3.}
\physDesc{Visitenkarte
\newline{}Handschrift: schwarze Tinte, deutsche Kurrent\newline{}Ordnung: mit Bleistift von unbekannter Hand nummeriert: »21« }\pstart
           \noindent{}\centering{}{\pb}\textcolor{gray}{\textbf{Lou Andreas-Salomé}}\pend
           \pstart{}Lieber \textsc{Arthur Schnitzler}!\pend\pstart
           Frl. \textcolor{blue}{\textsc{Lulu v. Strauss u Torney}}{}\ledrightnote{\textcolor{blue}{Lulu von Strauss und Torney}} kommt fremd nach \textcolor{pink}{Wien}{}\ledrightnote{\textcolor{pink}{Wien}}; ich habe ſie
                    gebeten zu Ihnen zu gehen, mit den ſchönſten Grü{\pb}ßen an Sie und Ihre \textcolor{blue}{Frau}{}\ledrightnote{\textcolor{blue}{Olga Schnitzler}}\pend
           \pstart
           von{\\[\baselineskip]}\spacefill\mbox{Frau Lou.}\pend
           \leftskip=0em{}\pstart
           \noindent{}\textsc{\textcolor{pink}{Göttingen}{}\ledrightnote{\textcolor{pink}{Göttingen}}, Juni 1908}\pend
           \endnumbering\briefempfaengerindex{Schnitzler, Arthur@\textsc{Schnitzler, Arthur}!zzzAndreas-Salome, Lou@\emph{von Lou Andreas-Salomé}!1908-06-011@{Juni 1908}|)be}\mylabel{h}  \normalsize

\doendnotes{C}
\bigskip
\vfill

\clearpage

\footnotesize

\lohead{\textsc{register}}

% Definiere theindex-Environment komplett neu ohne reledmac
\makeatletter
\renewenvironment{theindex}{%
  \section*{\indexname}%
  \setlength{\parindent}{0pt}%
  \setlength{\parskip}{0pt plus 0.3pt}%
  \let\item\@idxitem
}{%
  \clearpage
}
\makeatother

\IfFileExists{\jobname-pw.ind}{\input{\jobname-pw.ind}}{}

\end{document}

      