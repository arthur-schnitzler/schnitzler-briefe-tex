%% latex-korrekturansicht-vorspann.tex
%% Vorspann für die Korrekturansicht.
%% Lädt die gemeinsame Datei latex-vorspann.tex mit gesetztem Schalter.

\newif\ifkorrekturansicht
\korrekturansichttrue

\input{../tex-inputs/latex-vorspann}


               \section[Paul Goldmann an Arthur Schnitzler, 15. 6. {[}1894{]}]{ Paul Goldmann an Arthur Schnitzler, 15. 6. {[}1894{]}}\nopagebreak\mylabel{v}\rehead{ }\normalsize\beginnumbering\briefempfaengerindex{Schnitzler, Arthur@\textsc{Schnitzler, Arthur}!zzzGoldmann, Paul@\emph{von Paul Goldmann}!1894-06-152@{15. 6. {[}1894{]}}|(be} \toendnotes[C]{\smallbreak\pagebreak[2]} \Standort{DLA, A:Schnitzler, HS.NZ85.1.3164.}
\physDesc{Brief, 1 Blatt, 3 Seiten
\newline{}Handschrift: schwarze Tinte, deutsche Kurrent
\newline{}Schnitzler: 1) mit Bleistift auf dem ersten Blatt die Jahreszahl »94« vermerkt 2) mit rotem Buntstift eine Unterstreichung}\toendnotes[C]{\smallbreak}\pstart
           \noindent{}{\pb}\textcolor{gray}{\textbf{\textcolor{brown}{Frankfurter Zeitung}{}\ledrightnote{\textcolor{brown}{Frankfurter Zeitung}}.}}\hfill \textsc{\textcolor{pink}{Paris}{}\ledrightnote{\textcolor{pink}{Paris}}}, 15. Juni.\pend
           \pstart
           \textcolor{gray}{\textbf{(\textcolor{brown}{Gazette de
                     Francfort}{}\ledrightnote{\textcolor{brown}{Frankfurter Zeitung}}.)}}\pend
           \pstart
           \textcolor{gray}{\textbf{Fondateur \textbf{M. \textcolor{blue}{L. Sonnemann}{}\ledrightnote{\textcolor{blue}{Leopold Sonnemann}}}.}}\pend
           \pstart
           \textcolor{gray}{\textbf{\begin{otherlanguage}{french}Journal politique, financier,\end{otherlanguage}}}\pend
           \pstart
           \textcolor{gray}{\textbf{\begin{otherlanguage}{french}commercial et littéraire.\end{otherlanguage}}}\pend
           \pstart
           \textcolor{gray}{\textbf{\begin{otherlanguage}{french}\textbf{Paraissant trois fois par jour.}\end{otherlanguage}}}\pend
           \pstart
           \textcolor{gray}{\textbf{–}}\pend
           \pstart
           \textcolor{gray}{\textbf{\begin{otherlanguage}{french}\textbf{Bureau à \textcolor{pink}{Paris}{}\ledrightnote{\textcolor{pink}{Paris}}:}\end{otherlanguage}}}\pend
           \pstart
           \textcolor{gray}{\textbf{\begin{otherlanguage}{french}\textcolor{pink}{24. Rue Feydeau}{}\ledrightnote{\textcolor{pink}{rue Feydeau}}.\end{otherlanguage}}}\pend
           \pstart{}Mein lieber Freund,\pend\pstart
           Ich bin ſehr beſchäftigt. Darum nur wenige Zeilen.\pend
           \pstart
           1.) Wärmſten Dank für Deinen lieben Brief aus \label{K_L02625-1v}\edtext{\textsc{\textcolor{pink}{Muenchen}{}\ledrightnote{\textcolor{pink}{München}}}}{\lemma{\textnormal{\emph{Muenchen}}}\Cendnote{\textnormal{Zwichen 2. 6. 1894 und 8. 6. 1894 hielt sich \textcolor{blue}{Schnitzler} in \textcolor{pink}{München}
                  auf.}}}\label{K_L02625-1h}. Er erklärt Manches und läßt Manches im Unklaren. All’ das iſt ſehr
               ſchwer brieflich abzumachen. Auch das, was mich erregt, läßt ſich kaum ſo
               niederſchreiben. Ich möchte mit Dir ſprechen, aber vielleicht iſt es am Beſten gar
               nicht mehr darüber zu {\pb}reden. Die Dinge müſſen ihren
               Lauf gehen.\pend
           \pstart
           2.) Haſt Du die \label{K_L02625-5v}\edtext{»\textsc{\textcolor{green}{Revue Blanche}{}\ledrightnote{\textcolor{green}{La Revue blanche}}}«}{\lemma{\textnormal{\emph{»Revue Blanche«}}}\Cendnote{\textnormal{Die wohl für den \emph{\textcolor{green}{Mercure de France}} gedachte (siehe Paul Goldmann an Arthur Schnitzler, 29. 5. [1894]) \textcolor{green}{Besprechung} von \textcolor{blue}{Schnitzler}s Schauspiel \emph{\textcolor{green}{Das Märchen}}
                  erschien in der \emph{\textcolor{green}{Revue blanche}}, \textcolor{blue}{Henri Albert}: \emph{\textcolor{green}{Les Lettres allemandes. Drames Nouveaux}}. In: \emph{\textcolor{green}{La Revue Blanche}}, Jg. 6, Nr. 32,
                        Juni 1984, S. 556–560, hier S. 560. Dem \emph{\textcolor{green}{Tagebuch}} ist zu entnehmen, dass \textcolor{blue}{Schnitzler} die \textcolor{green}{Besprechung} las, vgl. A. S.: \emph{Tagebuch}, 11. 6. 1894.}}}\label{K_L02625-5h} erhalten.\pend
           \pstart
           3.) Können wir \label{K_L02625-7v}\edtext{im Auguſt zuſammenreiſen}{\lemma{\textnormal{\emph{im Auguſt zuſammenreiſen}}}\Cendnote{\textnormal{Vom 23. 8. 1894 bis zum 3. 9. 1894 verbrachten
                     \textcolor{blue}{Schnitzler} und \textcolor{blue}{Goldmann} einige Zeit gemeinsam in \textcolor{pink}{Bad Ischl} und \textcolor{pink}{Bad
                  Aussee}.}}}\label{K_L02625-7h}? Bitte, antworte mir umgehend, denn ich muß jetzt bereits
               anfangen, eventuelle Vorkehrungen zu treffen.\pend
           \pstart
           4.) Was weißt Du von \textsc{\textcolor{pink}{Muenchen}{}\ledrightnote{\textcolor{pink}{München}}} zu erzählen? Haſt Du den \textsc{\textcolor{green}{\textcolor{blue}{Altdorfer}{}\ledrightnote{\textcolor{blue}{Albrecht Altdorfer}}}{}\ledrightnote{→\textcolor{green}{Laubwald mit dem heiligen Georg}}} geſehen, von dem ich Dir \label{K_L02625-2v}\edtext{ſchrieb}{\lemma{\textnormal{\emph{ſchrieb}}}\Cendnote{\textnormal{siehe Paul Goldmann an Arthur Schnitzler, 1. 6. [1894]}}}\label{K_L02625-2h}? Wie gehts Dir \strikeout{J} geſundheitlich?\pend
           \pstart
           {\pb}5.) \textsc{\textcolor{blue}{Herzl}{}\ledrightnote{\textcolor{blue}{Theodor Herzl}}}, den ich verſchiedentlich von Dir gegrüßt, läßt Dich verſchiedentlich wieder
               grüßen. Desgleichen \textsc{\textcolor{blue}{Henri Albert}{}\ledrightnote{\textcolor{blue}{Henri Albert}}}. Ich habe dieſer Tage den \label{K_L02625-3v}\edtext{Bürſten-Abzug}{\lemma{\textnormal{\emph{Bürſten-Abzug}}}\Cendnote{\textnormal{Probeabzug}}}\label{K_L02625-3h} der
                  \label{K_L02625-4v}\edtext{»\textsc{\textcolor{green}{Emplettes de Noël}{}\ledrightnote{\textcolor{green}{Les Emplettes de Noël}}}«}{\lemma{\textnormal{\emph{»Emplettes de Noël«}}}\Cendnote{\textnormal{\textcolor{blue}{Henri Albert}s \textcolor{green}{Übersetzung} von Schnitzlers \emph{\textcolor{green}{Anatol}}-Einakter \emph{\textcolor{green}{Weihnachts-Einkäufe}}}}}\label{K_L02625-4h} geſehen, die \label{K_L02625-6v}\edtext{in der »\textsc{\textcolor{green}{Idée Libre}{}\ledrightnote{\textcolor{green}{L'Idée libre. Revue mensuelle de Littérature et d'Art}}}« erſcheinen}{\lemma{\textnormal{\emph{in … erſcheinen}}}\Cendnote{\textnormal{\textcolor{blue}{Arthur Schnitzler}: \emph{\textcolor{green}{Les Emplettes de Noël}}. Übersetzung \textcolor{blue}{Henri Albert}. In: \emph{\textcolor{green}{L'Idée
                        libre. Revue mensuelle de Littérature et d'Art}}, Jg. 3,
                     Nr. 5–6, Mai–Juni 1984, S. 215–225. Am 21. 7. 1894 notiert
                  Schnitzler in seinem \emph{\textcolor{green}{Tagebuch}}:
                     »Schlecht übersetzt.«. \textcolor{blue}{Albert} gegenüber dürfte er aber ein anderes Urteil geäußert haben, denn
                  dieser antwortet ihm in einem Brief am 6. 8. 1894: »Dass Ihnen
                     meine Uebersetzung so gut gefallen hat, hat mich hoch erfreut.« (\emph{DLA}, HS.1985.1.2331,3)}}}\label{K_L02625-6h} werden, da die
               andern auf Monat und Jahr hinaus keinen Platz haben.\pend
           \pstart
           6.) \label{K_L02625-8v}\edtext{Lies »\textsc{\textcolor{green}{Caligula}{}\ledrightnote{\textcolor{green}{Caligula – Eine Studie über römischen Cäsarenwahnsinn}}}« von \textsc{\textcolor{blue}{Quidde}{}\ledrightnote{\textcolor{blue}{Ludwig Quidde}}}!}{\lemma{\textnormal{\emph{Lies … Quidde!}}}\Cendnote{\textnormal{Eine Lektüre der kleinen Studie
                  über den Cäsarenwahn durch \textcolor{blue}{Schnitzler}, die
                  von den Zeitgenossinnen und Zeitgenossen als Schmähschrift gegen \textcolor{blue}{Wilhelm II.} gelesen wurde, ist bislang nicht belegt.}}}\label{K_L02625-8h}\pend
           \pstart
           7.) Viele treue Grüße! {\\[\baselineskip]}Dein {\\[\baselineskip]}\spacefill\mbox{Paul Goldmann}\pend
           \leftskip=0em{}\endnumbering\briefempfaengerindex{Schnitzler, Arthur@\textsc{Schnitzler, Arthur}!zzzGoldmann, Paul@\emph{von Paul Goldmann}!1894-06-152@{15. 6. {[}1894{]}}|)be}\mylabel{h}  \normalsize

\doendnotes{C}
\bigskip
\vfill

\clearpage

\footnotesize

\lohead{\textsc{register}}

% Definiere theindex-Environment komplett neu ohne reledmac
\makeatletter
\renewenvironment{theindex}{%
  \section*{\indexname}%
  \setlength{\parindent}{0pt}%
  \setlength{\parskip}{0pt plus 0.3pt}%
  \let\item\@idxitem
}{%
  \clearpage
}
\makeatother

\IfFileExists{\jobname-pw.ind}{\input{\jobname-pw.ind}}{}

\end{document}

      