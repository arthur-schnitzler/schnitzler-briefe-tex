%% latex-korrekturansicht-vorspann.tex
%% Vorspann für die Korrekturansicht.
%% Lädt die gemeinsame Datei latex-vorspann.tex mit gesetztem Schalter.

\newif\ifkorrekturansicht
\korrekturansichttrue

\input{../tex-inputs/latex-vorspann}


               \section[Arthur von Suttner an Arthur Schnitzler, 3. 5. 1893]{ Arthur von Suttner an Arthur Schnitzler,
                    3. 5. 1893}\nopagebreak\mylabel{v}\rehead{ }\normalsize\beginnumbering\briefempfaengerindex{Schnitzler, Arthur@\textsc{Schnitzler, Arthur}!zzzSuttner, Arthur Gundaccar von@\emph{von Arthur Gundaccar von Suttner}!1893-05-032@{3. 5. 1893}|(be} \toendnotes[C]{\smallbreak\pagebreak[2]} \Standort{CUL, Schnitzler, B 104.}
\physDesc{Brief, 1 Blatt, 2 Seiten
\newline{}Handschrift: schwarze Tinte, deutsche Kurrent
\newline{}Schnitzler: mit Bleistift beschriftet: »\textsc{Suttner}« }\Standort{DLA, A:Schnitzler, HS.NZ85.1.4773.}
\physDesc{1 Blatt, 1 Seite, maschinelle Abschrift}\toendnotes[C]{\smallbreak}\pstart
           \noindent{}{\pb}\textcolor{pink}{\textsc{Schloss Harmannsdorf}}{}\ledrightnote{\textcolor{pink}{Schloss Harmannsdorf}}\hfill \textcolor{gray}{\textbf{am}}{ }3/V \textcolor{gray}{\textbf{189}}3\pend
           \pstart
           \textsc{b/\textcolor{pink}{Eggenburg}{}\ledrightnote{\textcolor{pink}{Eggenburg}}.}\pend
           \pstart{}Sehr geehrter Herr,\pend\pstart
           Geſtatten Sie einem Ihnen perſönlich Unbekannten, Ihnen ſein warmes Beileid zu
                    dem ſchweren Verluſte auszudrücken. Nicht allein Sie, – die Wiſſenſchaft, – die
                    Menſchheit hat viel verloren. Ich habe den trefflichen \textcolor{blue}{Mann}{}\ledrightnote{→\textcolor{blue}{Johann Schnitzler}} gekannt, der in ſeiner ganzen
                    Vollkraft den \uline{wahren} Heldentod geſtorben iſt,
                    auf dem \uline{wahren} Felde der Ehre – zur Rettung
                    eines Mitmenſchen.\pend
           \pstart
           Meine \textcolor{blue}{Frau}{}\ledrightnote{→\textcolor{blue}{Bertha von Suttner}} ſchließt ſich
                    mir an, und ich bitte, die Verſicherung unſerer wärmſten, unſerer herzlichſten
                        {\pb}Teilnahme für ſich und Ihre
                    Familie in Empfang zu nehmen.\pend
           \pstart
           In vorzüglicher Hochachtung{\\[\baselineskip]}\spacefill\mbox{A. G. v. Suttner}\pend
           \leftskip=0em{}\endnumbering\briefempfaengerindex{Schnitzler, Arthur@\textsc{Schnitzler, Arthur}!zzzSuttner, Arthur Gundaccar von@\emph{von Arthur Gundaccar von Suttner}!1893-05-032@{3. 5. 1893}|)be}\mylabel{h}  \normalsize

\doendnotes{C}
\bigskip
\vfill

\clearpage

\footnotesize

\lohead{\textsc{register}}

% Definiere theindex-Environment komplett neu ohne reledmac
\makeatletter
\renewenvironment{theindex}{%
  \section*{\indexname}%
  \setlength{\parindent}{0pt}%
  \setlength{\parskip}{0pt plus 0.3pt}%
  \let\item\@idxitem
}{%
  \clearpage
}
\makeatother

\IfFileExists{\jobname-pw.ind}{\input{\jobname-pw.ind}}{}

\end{document}

      