%% latex-korrekturansicht-vorspann.tex
%% Vorspann für die Korrekturansicht.
%% Lädt die gemeinsame Datei latex-vorspann.tex mit gesetztem Schalter.

\newif\ifkorrekturansicht
\korrekturansichttrue

\input{../tex-inputs/latex-vorspann}


               \section[Hugo von Hofmannsthal an Arthur Schnitzler, 4. 10. 1900]{ Hugo von Hofmannsthal an Arthur Schnitzler, 4. 10. 1900}\nopagebreak\mylabel{v}\rehead{ }\normalsize\beginnumbering\briefempfaengerindex{Schnitzler, Arthur@\textsc{Schnitzler, Arthur}!zzzHofmannsthal, Hugo von@\emph{von Hugo von Hofmannsthal}!1900-10-041@{4. 10. 1900}|(be} \toendnotes[C]{\smallbreak\pagebreak[2]} \Standort{CUL, Schnitzler, B 43.}
\physDesc{Bildpostkarte
\newline{}Handschrift: schwarze Tinte, deutsche Kurrent\newline{}Versand: 1) Stempel: »\nobreak{}\oindex{Ouchy@\textbf{Ouchy}, \emph{http://www.geonames.org/ontologyP.PPL}|pwk}Ouchy, 4. X. 00, 4\nobreak{}«.  2) Stempel: »\nobreak{}\oindex{IX., Alsergrund@\textbf{IX., Alsergrund}, \emph{Bezirk (A.BZK)}|pwk}Wien 9/3 72, 6. 10. 00, 8.V, Bestellt\nobreak{}«. \newline{}Ordnung: 1) mit Bleistift von unbekannter Hand
                           nummeriert: »167« 2) mit Bleistift von unbekannter Hand nummeriert: »174«}\buchAbdrucke{\weitereDrucke{Hugo von Hofmannsthal, Arthur Schnitzler: \emph{Briefwechsel}. Hg. Therese Nickl und Heinrich Schnitzler. Frankfurt am Main: \emph{S. Fischer} 1964, S. 144.} }\toendnotes[C]{\smallbreak}\pstart{}{\pb}\textsc{Herrn D\textsuperscript{r} Arthur Schnitzler}\pend{}\pstart{}\textsc{\textcolor{pink}{Wien}{}\ledrightnote{\textcolor{pink}{Wien}}}\pend{}\pstart{}\textsc{\textcolor{pink}{IX. Franckgasse 1.}{}\ledrightnote{\textcolor{pink}{Frankgasse}}}\pend{}{\bigskip}\pstart
           \noindent{}\centering{}\textcolor{gray}{\textbf{{\pb}\textcolor{pink}{Ouchy}{}\ledrightnote{\textcolor{pink}{Ouchy}} – \textcolor{pink}{Hôtel Beau
                     Rivage}{}\ledrightnote{\textcolor{pink}{Beau-Rivage Palace}}.}}\pend
           \pstart
           4. X. 1900.\pend
           \pstart
           In \textcolor{pink}{\textsc{Ouchy}}{}\ledrightnote{\textcolor{pink}{Ouchy}} haben wir einmal zuſa{\geminationm}en aufs Dampfſchiff \label{K_L01075_1v}\edtext{gewartet}{\lemma{\textnormal{\emph{gewartet}}}\Cendnote{\textnormal{siehe A. S.: \emph{Tagebuch}, 14. 8. 1898}}}\label{K_L01075_1h} und über
               färbige Strümpfe geſprochen. Dann war der ſchöne Abend in \textcolor{pink}{\textsc{Chillon}}{}\ledrightnote{\textcolor{pink}{Schloss Chillon}} und in \textcolor{pink}{\textsc{Glion}}{}\ledrightnote{\textcolor{pink}{Glion}}.\pend
           \pstart
           Auf baldiges Wiederſehen{\\[\baselineskip]}Ihr\spacefill\mbox{Hugo}\pend
           \leftskip=0em{}\endnumbering\briefempfaengerindex{Schnitzler, Arthur@\textsc{Schnitzler, Arthur}!zzzHofmannsthal, Hugo von@\emph{von Hugo von Hofmannsthal}!1900-10-041@{4. 10. 1900}|)be}\mylabel{h}  \normalsize

\doendnotes{C}
\bigskip
\vfill

\clearpage

\footnotesize

\lohead{\textsc{register}}

% Definiere theindex-Environment komplett neu ohne reledmac
\makeatletter
\renewenvironment{theindex}{%
  \section*{\indexname}%
  \setlength{\parindent}{0pt}%
  \setlength{\parskip}{0pt plus 0.3pt}%
  \let\item\@idxitem
}{%
  \clearpage
}
\makeatother

\IfFileExists{\jobname-pw.ind}{\input{\jobname-pw.ind}}{}

\end{document}

      