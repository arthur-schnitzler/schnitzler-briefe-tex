%% latex-korrekturansicht-vorspann.tex
%% Vorspann für die Korrekturansicht.
%% Lädt die gemeinsame Datei latex-vorspann.tex mit gesetztem Schalter.

\newif\ifkorrekturansicht
\korrekturansichttrue

\input{../tex-inputs/latex-vorspann}


               \section[Arthur Schnitzler an Richard Beer-Hofmann, 1. 8. 1906]{ Arthur Schnitzler an Richard Beer-Hofmann, 1. 8. 1906}\nopagebreak\mylabel{v}\rehead{ }\normalsize\beginnumbering\briefempfaengerindex{Beer-Hofmann, Richard@\textsc{Beer-Hofmann, Richard}!zzzSchnitzler, Arthur@\emph{von Arthur Schnitzler}!1906-08-011@{1. 8. 1906}|(be} \toendnotes[C]{\smallbreak\pagebreak[2]} \Standort{YCGL, MSS 31.}
\physDesc{Bildpostkarte
\newline{}Handschrift: Bleistift, deutsche Kurrent\newline{}Versand: 1) Stempel: »\nobreak{}\oindex{Skodsborg@\textbf{Skodsborg}, \emph{Besiedelter Ort (A.BSO)}|pwk}Skodsborg, 1. 8. 06, \textcolor{gray}{D}17F\nobreak{}«.  2) Stempel: »\nobreak{}\oindex{Rodaun@\textbf{Rodaun}, \emph{Teil eines besiedelten Ortes (A.BSOX)}|pwk}Rodaun\nobreak{}«. }\toendnotes[C]{\smallbreak}\pstart{}{\pb}\textsc{Dr. Richard Beer-Hofmann}\pend{}\pstart{}\textcolor{pink}{\textsc{Rodaun bei Wien}}{}\ledrightnote{\textcolor{pink}{Rodaun}}\pend{}\pstart{}\textcolor{pink}{\textsc{Liesingerstr. –}}{}\ledrightnote{\textcolor{pink}{Liesingerstraße}}\pend{}\pstart{}\textsc{\textcolor{pink}{Austria}{}\ledrightnote{\textcolor{pink}{Österreich}}}\pend{}{\bigskip}\pstart
           \noindent{}{\pb}\textcolor{gray}{\textbf{\textcolor{pink}{Skodsborg Badehotel}{}\ledrightnote{\textcolor{pink}{Badehotel}}}}\pend
           \pstart
           \centering{}1/8 906.\pend
           \pstart
           \label{K_L01619_1v}\edtext{Domino.}{\lemma{\textnormal{\emph{Domino.}}}\Cendnote{\textnormal{Einzelne Begriffe sind mit bestimmten Stellen des
                  Ansichtkartenmotivs verbunden, bestimmte Themen des gemeinsamen Aufenthalts
                     1896 evozierend (vgl. A. S.: \emph{Tagebuch}, 14. 8. 1896). Nachdem \textcolor{blue}{Schnitzler} damals in der Dependance des Hotels wohnte, dürfte dieser
                  Begriff die Terrasse vor seiner Unterkunft markieren, während das folgende
                     »Jugend?« die Zimmer von \textcolor{blue}{Beer-Hofmann} und \textcolor{blue}{Paula Lissy}
                  einzeichnet.}}}\label{K_L01619_1h}\pend
           \pstart
           Jugend?\pend
           \pstart
           \label{T_L01619_1v}\edtext{Bad}{\lemma{\textnormal{\emph{Bad}}}\Cendnote{\textnormal{Verbindungslinie zum Wasser}}}\label{T_L01619_1h}\pend
           \pstart
           {\pb}Herzliche Grüße,\pend
           \pstart
           Ihr{\\[\baselineskip]}\spacefill\mbox{A.}\pend
           \leftskip=0em{}\endnumbering\briefempfaengerindex{Beer-Hofmann, Richard@\textsc{Beer-Hofmann, Richard}!zzzSchnitzler, Arthur@\emph{von Arthur Schnitzler}!1906-08-011@{1. 8. 1906}|)be}\mylabel{h}  \normalsize

\doendnotes{C}
\bigskip
\vfill

\clearpage

\footnotesize

\lohead{\textsc{register}}

% Definiere theindex-Environment komplett neu ohne reledmac
\makeatletter
\renewenvironment{theindex}{%
  \section*{\indexname}%
  \setlength{\parindent}{0pt}%
  \setlength{\parskip}{0pt plus 0.3pt}%
  \let\item\@idxitem
}{%
  \clearpage
}
\makeatother

\IfFileExists{\jobname-pw.ind}{\input{\jobname-pw.ind}}{}

\end{document}

      