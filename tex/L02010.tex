%% latex-korrekturansicht-vorspann.tex
%% Vorspann für die Korrekturansicht.
%% Lädt die gemeinsame Datei latex-vorspann.tex mit gesetztem Schalter.

\newif\ifkorrekturansicht
\korrekturansichttrue

\input{../tex-inputs/latex-vorspann}


               \section[Arthur Schnitzler an Robert Adam, 11. 2. 1911]{ Arthur Schnitzler an Robert Adam, 11. 2. 1911}\nopagebreak\mylabel{v}\rehead{ }\normalsize\beginnumbering\briefempfaengerindex{Adam, Robert@\textsc{Adam, Robert}!zzzSchnitzler, Arthur@\emph{von Arthur Schnitzler}!1911-02-111@{11. 2. 1911}|(be} \toendnotes[C]{\smallbreak\pagebreak[2]} \Standort{DLA, 96.34.1/5.}
\physDesc{Brief, 1 Blatt, 1 Seite, Umschlag
\newline{}Schreibmaschine
\newline{}Handschrift: schwarze Tinte (\noindent{}Unterschrift)\newline{}Versand: Stempel: »\nobreak{}Wien\nobreak{}«.  }\Standort{DLA, A:Schnitzler, 85.1.1621.}
\physDesc{Brief, 1 Blatt, 1 Seite, Umschlag, maschineller Durchschlag
\newline{}Schreibmaschine
\newline{}Handschrift: roter Buntstift, lateinische Kurrent (\noindent{}Beschriftung »Adam«)}\toendnotes[C]{\smallbreak}\pstart{}{\pb}\textcolor{gray}{\textbf{Dr. Arthur Schnitzler}}\pend{}\pstart{}\textcolor{pink}{\textcolor{gray}{\textbf{Wien, XVIII. Sternwartestrasse 71}}}{}\ledrightnote{\textcolor{pink}{Sternwartestraße}}\pend{}{\bigskip}\pstart{}{\pb}Herrn Robert Adam\pend{}\pstart{}\textcolor{pink}{\so{Wien XII}}{}\ledrightnote{\textcolor{pink}{XII., Meidling}}.\pend{}\pstart{}\textcolor{pink}{Meidlinger Hauptstraße 56}{}\ledrightnote{\textcolor{pink}{Meidlinger Hauptstraße}}.\pend{}{\bigskip}\pstart
           {\pb}\textcolor{gray}{\textbf{Dr. Arthur Schnitzler}}\hfill 11. 2. 1911.\pend
           \pstart
           \textcolor{gray}{\textbf{\textcolor{pink}{Wien XVIII. Sternwartestrasse 71}{}\ledrightnote{\textcolor{pink}{Sternwartestraße}}}}\pend
           \pstart\center{}Sehr geehrter Herr Adam.\pend\pstart
           Es tut mir leid, dass Ihnen bei \textcolor{brown}{S. Fischer}{}\ledrightnote{\textcolor{brown}{S. Fischer Verlag}} kein
               Erfolg beschieden war. Ob ein weiteres Herumschicken des \textcolor{green}{Manuscriptes}{}\ledrightnote{→\textcolor{green}{Neidhard}} an Verleger Ihre Sache fördern könnte, ist
               schwer zu entscheiden. Von der Wertlosigkeit meiner Empfehlung haben Sie sich wohl
               überzeugt. Versuche einzelne Szenen bei Zeitschriften unterzubringen, sollten Sie
               keineswegs unterlassen. Hier kämen meines Erachtens »\textcolor{brown}{Merker}{}\ledrightnote{\textcolor{brown}{Der Merker}}« und »\textcolor{brown}{Schaubühne}{}\ledrightnote{\textcolor{brown}{Die Schaubühne / Die Weltbühne}}« vor allem in
               Betracht.\pend
           \pstart
           Mit verbindlichen Grüssen{\\[\baselineskip]}Ihr ergebener{\\[\baselineskip]}\spacefill\mbox{{[}hs.:{]} ArthSchnitzler}\pend
           \leftskip=0em{}\pstart
           \noindent{}{[}ms.:{]} Herrn Robert Adam, \textcolor{pink}{Wien}{}\ledrightnote{\textcolor{pink}{Wien}}.\pend
           \endnumbering\briefempfaengerindex{Adam, Robert@\textsc{Adam, Robert}!zzzSchnitzler, Arthur@\emph{von Arthur Schnitzler}!1911-02-111@{11. 2. 1911}|)be}\mylabel{h}  \normalsize

\doendnotes{C}
\bigskip
\vfill

\clearpage

\footnotesize

\lohead{\textsc{register}}

% Definiere theindex-Environment komplett neu ohne reledmac
\makeatletter
\renewenvironment{theindex}{%
  \section*{\indexname}%
  \setlength{\parindent}{0pt}%
  \setlength{\parskip}{0pt plus 0.3pt}%
  \let\item\@idxitem
}{%
  \clearpage
}
\makeatother

\IfFileExists{\jobname-pw.ind}{\input{\jobname-pw.ind}}{}

\end{document}

      