%% latex-korrekturansicht-vorspann.tex
%% Vorspann für die Korrekturansicht.
%% Lädt die gemeinsame Datei latex-vorspann.tex mit gesetztem Schalter.

\newif\ifkorrekturansicht
\korrekturansichttrue

\input{../tex-inputs/latex-vorspann}


               \section[Friedrich M. Fels an Arthur Schnitzler, 22. 10. 1894]{ Friedrich M. Fels an Arthur Schnitzler, 22. 10. 1894}\nopagebreak\mylabel{v}\rehead{ }\normalsize\beginnumbering\briefempfaengerindex{Schnitzler, Arthur@\textsc{Schnitzler, Arthur}!zzzFels, Friedrich Michael@\emph{von Friedrich Michael Fels}!1894-10-222@{22. 10. 1894}|(be} \toendnotes[C]{\smallbreak\pagebreak[2]} \Standort{DLA, A:Schnitzler, HS.NZ85.1.2956.}
\physDesc{Postkarte
\newline{}Handschrift: schwarze Tinte, lateinische Kurrent\newline{}Versand: 1) Stempel: »\nobreak{}\oindex{I., Innere Stadt@\textbf{I., Innere Stadt}, \emph{Bezirk (A.BZK)}|pwk}Wien 1/1, 22. 10. 94, 3–4N\nobreak{}«.  2) Stempel: »\nobreak{}\oindex{IX., Alsergrund@\textbf{IX., Alsergrund}, \emph{Bezirk (A.BZK)}|pwk}Wien 9/3, 22. 10. 94, 5.N, Bestellt\nobreak{}«. 
\newline{}Schnitzler: mit Bleistift nummeriert: »16« und datiert: »22/10 94« }\pstart{}{\pb}Herrn Dr. Arthur
                        Schnitzler\pend{}\pstart{}\textcolor{pink}{Wien}{}\ledrightnote{\textcolor{pink}{Wien}}\pend{}\pstart{}\textcolor{pink}{IX, Frankgaſse 1}{}\ledrightnote{\textcolor{pink}{Frankgasse}}\pend{}{\bigskip}\pstart
           \noindent{}{\pb}Lieber Doktor Schnitzler! Bei \textcolor{blue}{L.}{}\ledrightnote{\textcolor{blue}{Julius von Gans-Ludassy}} leider noch nichts entschieden, da er noch nicht gelesen hat; ich
                    soll in ein paar Tagen wieder ko{\geminationm}en; doch hat er
                    keinen besti{\geminationm}ten Termin angegeben, wohl um sich das
                    Recht zu erhalten, \introOben{}da{\geminationn}\introOben{} i{\geminationm}er noch nicht gelesen zu haben. Mit \textcolor{blue}{J. J. D.}{}\ledrightnote{\textcolor{blue}{Jakob Julius David}} habe ich ausführlich gesprochen, und
                    er hat mir gesagt, er kö{\geminationn}e, möge es mit \textcolor{blue}{L.}{}\ledrightnote{\textcolor{blue}{Julius von Gans-Ludassy}} ausgehen, wie i{\geminationm}er es wolle, monatlich 2 Feuill. von mir bringen
                    (à 10 fl). I{\geminationm}erhin etwas. Zu \textcolor{blue}{H. B.}{}\ledrightnote{\textcolor{blue}{Hermann Bahr}} gehe ich, sowie von \textcolor{blue}{L.}{}\ledrightnote{\textcolor{blue}{Julius von Gans-Ludassy}} die Arbeit zurückko{\geminationm}t.\pend
           \pstart
           Herzlichen Gruſs{\\[\baselineskip]}\spacefill\mbox{F.}\pend
           \leftskip=0em{}\endnumbering\briefempfaengerindex{Schnitzler, Arthur@\textsc{Schnitzler, Arthur}!zzzFels, Friedrich Michael@\emph{von Friedrich Michael Fels}!1894-10-222@{22. 10. 1894}|)be}\mylabel{h}  \normalsize

\doendnotes{C}
\bigskip
\vfill

\clearpage

\footnotesize

\lohead{\textsc{register}}

% Definiere theindex-Environment komplett neu ohne reledmac
\makeatletter
\renewenvironment{theindex}{%
  \section*{\indexname}%
  \setlength{\parindent}{0pt}%
  \setlength{\parskip}{0pt plus 0.3pt}%
  \let\item\@idxitem
}{%
  \clearpage
}
\makeatother

\IfFileExists{\jobname-pw.ind}{\input{\jobname-pw.ind}}{}

\end{document}

      