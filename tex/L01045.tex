%% latex-korrekturansicht-vorspann.tex
%% Vorspann für die Korrekturansicht.
%% Lädt die gemeinsame Datei latex-vorspann.tex mit gesetztem Schalter.

\newif\ifkorrekturansicht
\korrekturansichttrue

\input{../tex-inputs/latex-vorspann}


               \section[Arthur Schnitzler an Richard Beer-Hofmann, 19. 6. 1900]{ Arthur Schnitzler an Richard Beer-Hofmann, 19. 6. 1900}\nopagebreak\mylabel{v}\rehead{ }\normalsize\beginnumbering\briefempfaengerindex{Beer-Hofmann, Richard@\textsc{Beer-Hofmann, Richard}!zzzSchnitzler, Arthur@\emph{von Arthur Schnitzler}!1900-06-191@{19. 6. 1900}|(be} \toendnotes[C]{\smallbreak\pagebreak[2]} \Standort{YCGL, MSS 31.}
\physDesc{Brief, 1 Blatt, 4 Seiten, Umschlag
\newline{}Handschrift: Bleistift, deutsche Kurrent\newline{}Versand: 1) Stempel: »\nobreak{}\oindex{I., Innere Stadt@\textbf{I., Innere Stadt}, \emph{Bezirk (A.BZK)}|pwk}Wien 1/1 1, 19. 6. 00, 11–12N\nobreak{}«.  2) Stempel: »\nobreak{}\oindex{Altaussee@\textbf{Altaussee}, \emph{http://www.geonames.org/ontologyA.ADM3}|pwk}{\pb}Alt-Aussee, 20/6 00\nobreak{}«. }\buchAbdrucke{\weitereDrucke{Arthur Schnitzler, Richard Beer-Hofmann: \emph{Briefwechsel 1891–1931}. Hg. Konstanze Fliedl. Wien, Zürich: \emph{Europaverlag} 1992, S. 145.} }\toendnotes[C]{\smallbreak}\pstart{}{\pb}Herrn \textsc{Dr. Richard
                        Beer-Hofma{\geminationn}}\pend{}\pstart{}\textsc{\textcolor{pink}{Altaussee}{}\ledrightnote{\textcolor{pink}{Altaussee}}.}\pend{}{\bigskip}\pstart
           \raggedleft{}{\pb}19/6. 900.\pend
           \pstart
           lieber Richard, es iſt ziemlich unglaublich, dſs Sie gar nichts
               abſolut nichts von ſich hören laſſen. Ich möchte gern gegen Ende dieſe\textcolor{gray}{s} auf 2–3 Tage
               nach \textcolor{pink}{Altauſſee}{}\ledrightnote{\textcolor{pink}{Altaussee}} ko{\geminationm}en,
                  {\pb}iſt es Ihnen recht?\pend
           \pstart
           \textcolor{blue}{Goldmann}{}\ledrightnote{\textcolor{blue}{Paul Goldmann}}{ }ſchreibt mir wegen einer event. Fußtour
                  Anfg Auguſt, auch \textcolor{blue}{Kerr}{}\ledrightnote{\textcolor{blue}{Alfred Kerr}} möchte
               ſich anſchließen, mir wäre die Zeit nach 20. Juli eigentlich lieber;
               auch darüber ſpre{\pb}chen wir wohl. Mir geht es innerlich
               nicht gut. Denken Sie übrigens, dſs \textcolor{blue}{\textsc{Schlenther}}{}\ledrightnote{\textcolor{blue}{Paul Schlenther}}
               die \textcolor{green}{\textsc{Bea}.}{}\ledrightnote{\textcolor{green}{Der Schleier der Beatrice. Schauspiel in fünf Akten}}{ }\uline{nicht} aufführen will. (Natürlich verblümt.) Näheres
               auch darüber mündlich. Ich war u. bin noch wüthend {\pb}drüber. – Meine \textcolor{green}{Novelle}{}\ledrightnote{→\textcolor{green}{Frau Bertha Garlan. Roman}} iſt
               fertig. Nicht ſchlecht. Einiges kleinere halbfertig. Zu größerm keine rechte
               Luſt. –\pend
           \pstart
           \textcolor{blue}{Hugo}{}\ledrightnote{\textcolor{blue}{Hugo von Hofmannsthal}} iſt in der \textcolor{pink}{Brühl}{}\ledrightnote{\textcolor{pink}{Brühl}}. \textcolor{blue}{Guſtav}{}\ledrightnote{\textcolor{blue}{Gustav Schwarzkopf}} auch.\pend
           \pstart
           Herzlichſt Ihr{\\[\baselineskip]}\spacefill\mbox{Arthur}\pend
           \leftskip=0em{}\endnumbering\briefempfaengerindex{Beer-Hofmann, Richard@\textsc{Beer-Hofmann, Richard}!zzzSchnitzler, Arthur@\emph{von Arthur Schnitzler}!1900-06-191@{19. 6. 1900}|)be}\mylabel{h}  \normalsize

\doendnotes{C}
\bigskip
\vfill

\clearpage

\footnotesize

\lohead{\textsc{register}}

% Definiere theindex-Environment komplett neu ohne reledmac
\makeatletter
\renewenvironment{theindex}{%
  \section*{\indexname}%
  \setlength{\parindent}{0pt}%
  \setlength{\parskip}{0pt plus 0.3pt}%
  \let\item\@idxitem
}{%
  \clearpage
}
\makeatother

\IfFileExists{\jobname-pw.ind}{\input{\jobname-pw.ind}}{}

\end{document}

      