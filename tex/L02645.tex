%% latex-korrekturansicht-vorspann.tex
%% Vorspann für die Korrekturansicht.
%% Lädt die gemeinsame Datei latex-vorspann.tex mit gesetztem Schalter.

\newif\ifkorrekturansicht
\korrekturansichttrue

\input{../tex-inputs/latex-vorspann}


               \section[Paul Goldmann an Arthur Schnitzler, 1. 12. 1889]{ Paul Goldmann an Arthur Schnitzler, 1. 12. 1889}\nopagebreak\mylabel{v}\rehead{ }\normalsize\beginnumbering\briefempfaengerindex{Schnitzler, Arthur@\textsc{Schnitzler, Arthur}!zzzGoldmann, Paul@\emph{von Paul Goldmann}!1889-12-011@{1. 12. 1889}|(be} \toendnotes[C]{\smallbreak\pagebreak[2]} \Standort{DLA, A:Schnitzler, HS.NZ85.1.3162.}
\physDesc{Brief, 1 Blatt, 4 Seiten
\newline{}Handschrift: blaue Tinte, deutsche Kurrent}\toendnotes[C]{\smallbreak}\pstart
           \noindent{}\centering{}{\pb}\textcolor{gray}{\textbf{\textbf{Adminiſtration: \textcolor{pink}{VII.
                           Seidengaſſe 7}{}\ledrightnote{\textcolor{pink}{Seidengasse}}} (\textcolor{brown}{Jos. Eberle {\kaufmannsund} Co.}{}\ledrightnote{\textcolor{brown}{Josef Eberle  Stein-, Buch und Musikaliendruckerei}})}}\pend
           \pstart
           \noindent{}\centering{}\textcolor{gray}{\textbf{\textcolor{brown}{An der Schönen Blauen Donau}{}\ledrightnote{\textcolor{brown}{An der schönen blauen Donau}}}}\pend
           \pstart
           \noindent{}\centering{}\textcolor{gray}{\textbf{Chef-Redacteur: Dr. \textcolor{blue}{F.
                        Mamroth}{}\ledrightnote{\textcolor{blue}{Fedor Mamroth}}. – Redaction: \textcolor{pink}{IX.,
                        Berggaſſe 31}{}\ledrightnote{\textcolor{pink}{Berggasse}}.}}\pend
           \pstart
           \raggedleft{}\textcolor{gray}{\textbf{\textcolor{pink}{Wien}{}\ledrightnote{\textcolor{pink}{Wien}}, den}}{ }1. December \textcolor{gray}{\textbf{18}}89.\pend
           \pstart\center{}Lieber Freund!\pend\pstart
           Weniger die ſchiefe und ungemein verzwickte Logik Ihres lieben Briefes, als vielmehr
               deſſen Liebenswürdigkeit bringen mich zu der Überzeugung, daß ich in einem Punkte
               jedenfalls Unrecht hatte: in meiner gereizten Auffaſſung der ganzen \label{K_L02645-1v}\edtext{Streitfrage}{\lemma{\textnormal{\emph{Streitfrage}}}\Cendnote{\textnormal{Der Vorgang, der einen Streit ausgelöst hat, ist nur durch
                  diesen und den folgenden Brief (Paul Goldmann an Arthur Schnitzler, 6. 12. 1889) belegt.
                  Wie sich daraus ergibt, hatte sich \textcolor{blue}{Goldmann}
                  mit einer nicht näher identifizierten \textcolor{blue}{Frau} in der Straßenbahn über \textcolor{blue}{Schnitzler} unterhalten. Das Gespräch wurde belauscht und
                     \textcolor{blue}{Schnitzler}{ }\textcolor{blue}{rapportiert}. Worüber genau
                  gesprochen wurde, bleibt unklar; es könnte sich um eine der mehreren zu diesem
                  Zeitpunkt parallel laufenden aktiven Beziehungen \textcolor{blue}{Schnitzler}s gehandelt haben: um jene mit der verheirateten \textcolor{blue}{Olga Waissnix} oder um die jeweiligen
                  Beziehungen mit \textcolor{blue}{Jeanette Heeger}, \textcolor{blue}{Marie Glümer} oder \textcolor{blue}{Helene Herz}.}}}\label{K_L02645-1h}. Aber es war gerade geſtern ein Tag höchſter Nervoſität für mich. Das war der
               phyſiſche Grund; und dann habe ich mich wüthend geärgert, daß in mein Verhältniß zu
               Ihnen, das mir biſher ſo viel Freude gemacht, ein Mißton gekommen war – das war der
               pſychiſche Grund. Ich will auf die Sache ſelbſt gar nicht mehr eingehen, obwohl ich
               überzeugt bin, daß mich nicht einmal der Vorwurf der Unvorſich{\pb}tigkeit trifft. Woher wiſſen Sie
               denn überhaupt, ob das \textcolor{blue}{Mädel}{}\ledrightnote{→\textcolor{blue}{?? [Frau, die mit Goldmann in der Straßenbahn spricht, Ende November 1889]}}
               Ihren Namen genannt hat, oder ob ich das war? Dieſes Thatbeſtandes-Moment hätten Sie
               doch erſt aufnehmen müſſen, ehe Sie Ihr Verdict fällten. Ich meine nach wie vor, daß
               ich nur \uline{eine} ſchuldhafte Handlung begangen habe,
               nämlich die, daß ich auf der Tramway überhaupt gefahren bin. Und ich ſehe, ich werde
               mir in Zukunft, um Ihnen Unannehmlichkeiten zu erſparen, das Tramway-Fahren
               abgewöhnen müſſen.\pend
           \pstart
           Aber laſſen wir das wirklich begraben ſein. \uline{Sie} haben
               ſich gekränkt, \uline{ich} habe mich gekränkt; ein Dienſtmann
               hat 30 und die Poſt 6 Kr. verdient; damit hat die ganze Affaire, meine ich, Wirkungen
               genug gehabt, und ſie kann jetzt geruhig vom Erdboden verſchwinden. Reden wir nicht
               mehr davon – ich bin ganz Ihrer Anſicht.\pend
           \pstart
           Nun noch ein Wort für die Zukunft. Es wird ſelbſtverſtändlich wieder vorkommen, daß
               Sie Gelegenheit haben werden, ſich über mich zu ärgern, obwohl – wie Sie überzeugt
               ſein können – von meiner Seite Alles geſchehen wird, um das zu {\pb}vermeiden. Aber das iſt nun einmal
               ſo: wozu hätte man einen guten Freund anders, als um ſich hier und da über ihn zu
               ärgern! Ich bin auch ganz Ihrer Anſicht, daß man jeden ſolchen Zwiſchenfall zur
               Sprache bringen ſoll; dazu iſt man befreundet, daß man ſich gegenſeitig ausſpricht.
               Nur bitte ich Sie um Eines: keine Briefe mehr in Zukunft. Ich kann mir nicht \strikeout{ſch} helfen: ſür mich hat ſo ein Wiſch Papier, der mir
                  \strikeout{S\textcolor{gray}{a}} allerlei unangenehme Sachen ſagt, ohne daß ich in der Lage bin, mich ihm
               gegenüber zu vertheidigen, immer etwas verteufelt Odioſes. Alſo reden Sie zu mir von
               Mund zu Ohr, wenn Sie etwas gegen mich haben. Und ich werde das Gleiche mit Ihnen
               thun. So ein Brief iſt wie ein Dritter, der ſich in etwas hineinmiſcht, das nur zwei
               allein angeht. Alſo, nicht wahr, den Gefallen thun Sie mir \strikeout{ih} in Zukunft?{\dotsfour}\pend
           \pstart
           {\pb}Und nun nehme ich eine neue Seite,
                  \strikeout{und} wie man das immer thun ſoll, wenn man mit ſich
               in’s Reine gekommen iſt und wenn Alles wieder gut geworden. Und frage Sie, ob man
                  heut{ }Abend auf das Vergnügen Ihrer Geſellſchaft beim \label{K_L02645-2v}\edtext{Souper}{\lemma{\textnormal{\emph{Souper}}}\Cendnote{\textnormal{Erst am Folgetag, dem 2. 12. 1889, kam es zum neuerlichen
                  gemeinsamen Abendessen; gemeinsam mit \textcolor{blue}{Friedrich
                     Kapper} und \textcolor{blue}{Jeanette Heeger}.}}}\label{K_L02645-2h}
                  rechnen\strikeout{\textcolor{gray}{×}} kann. Oder wann ſonſt, wenn nicht heut{ }Abend? Und wenn heut{ }Abend – wo und zu welcher Stunde?\pend
           \pstart
           Mein Bote wartet auf Antwort.\pend
           \pstart
           Herzlichſten Gruß! {\\[\baselineskip]}Ihr {\\[\baselineskip]}\spacefill\mbox{Paul Goldmann.}\pend
           \leftskip=0em{}\endnumbering\briefempfaengerindex{Schnitzler, Arthur@\textsc{Schnitzler, Arthur}!zzzGoldmann, Paul@\emph{von Paul Goldmann}!1889-12-011@{1. 12. 1889}|)be}\mylabel{h}  \normalsize

\doendnotes{C}
\bigskip
\vfill

\clearpage

\footnotesize

\lohead{\textsc{register}}

% Definiere theindex-Environment komplett neu ohne reledmac
\makeatletter
\renewenvironment{theindex}{%
  \section*{\indexname}%
  \setlength{\parindent}{0pt}%
  \setlength{\parskip}{0pt plus 0.3pt}%
  \let\item\@idxitem
}{%
  \clearpage
}
\makeatother

\IfFileExists{\jobname-pw.ind}{\input{\jobname-pw.ind}}{}

\end{document}

      