%% latex-korrekturansicht-vorspann.tex
%% Vorspann für die Korrekturansicht.
%% Lädt die gemeinsame Datei latex-vorspann.tex mit gesetztem Schalter.

\newif\ifkorrekturansicht
\korrekturansichttrue

\input{../tex-inputs/latex-vorspann}


               \section[Arthur Schnitzler an Hugo von Hofmannsthal, 26. 11. 1908]{ Arthur Schnitzler an Hugo von Hofmannsthal, 26. 11. 1908}\nopagebreak\mylabel{v}\rehead{ }\normalsize\beginnumbering\briefempfaengerindex{Hofmannsthal, Hugo von@\textsc{Hofmannsthal, Hugo von}!zzzSchnitzler, Arthur@\emph{von Arthur Schnitzler}!1908-11-261@{26. 11. 1908}|(be} \toendnotes[C]{\smallbreak\pagebreak[2]} \Standort{FDH, Hs-30885,133.}
\physDesc{Brief, 1 Blatt, 3 Seiten
\newline{}Handschrift: schwarze Tinte, deutsche Kurrent}\buchAbdrucke{\weitereDrucke{1) Hugo von Hofmannsthal, Arthur Schnitzler: \emph{Briefwechsel}. Hg. Therese Nickl und Heinrich Schnitzler. Frankfurt am Main: \emph{S. Fischer} 1964, S. 242.} \weitereDrucke{2) Hermann Bahr, Arthur Schnitzler: \emph{Briefwechsel, Aufzeichnungen, Dokumente
                                (1891–1931)}. Hg. Kurt Ifkovits und Martin Anton Müller. Göttingen: \emph{Wallstein} 2018, S. 411.} }\toendnotes[C]{\smallbreak}\pstart
           \raggedleft{}{\pb}2\substVorne{}\textsuperscript{5}\substDazwischen{}6\substHinten{}. 11. 08\pend
           \pstart
           \textcolor{gray}{\textbf{Dr. Arthur Schnitzler}}{\\}\textcolor{gray}{\textbf{\textcolor{pink}{Wien XVIII. Spoettelgasse 7}{}\ledrightnote{\textcolor{pink}{Edmund-Weiß-Gasse}}.}}\pend
           \pstart
           mein lieber Hugo,  geſtern waren wir in \label{K_L01809_1v}\edtext{\textcolor{green}{2 × 2 = 5}{}\ledrightnote{\textcolor{green}{2 × 2 = 5}}}{\lemma{\textnormal{\emph{2 × 2 = 5}}}\Cendnote{\textnormal{von \textcolor{blue}{Gustav Wied}}}}\label{K_L01809_1h} (\uline{unbedingt}
                    anzuſehen, ſchon, u. beſonders wegen \textcolor{blue}{Ethofer}{}\ledrightnote{\textcolor{blue}{Anton Edthofer}}) vorgeſtern beim \textcolor{green}{Krampus}{}\ledrightnote{\textcolor{green}{Caph}}, heut
                    gehn wir ins Tonkünſtlerconcert, Samſtag zum \textcolor{blue}{\textsc{Dohnanyi}}{}\ledrightnote{\textcolor{blue}{Ernst von Dohnányi}}, So{\geminationn}tag
                    zum \textsc{\textcolor{blue}{Heine}{}\ledrightnote{\textcolor{blue}{Heinrich Heine}}
                        Abend} – es gibt ſo verhexte Wochen; hingegen wollen wir am Montag oder
                    Dinſtag für 2 Tage auf den \textcolor{pink}{Se{\geminationm}ering}{}\ledrightnote{\textcolor{pink}{Semmering}}, es wäre ſehr ſchön, we{\geminationn} Sie u \textcolor{blue}{Gerty}{}\ledrightnote{\textcolor{blue}{Gertrude von Hofmannsthal}}
                    auch hinauf kämen; schrei{\pb}ben Sie mir ein Wort.
                    (Nicht unmöglich, dſs auch \textcolor{blue}{Waſſermann}{}\ledrightnote{\textcolor{blue}{Jakob Wassermann}} u \textcolor{blue}{Thomas Ma{\geminationn}}{}\ledrightnote{\textcolor{blue}{Thomas Mann}} (mit
                    dem wir geſtern Mittag bei \textcolor{blue}{W.}{}\ledrightnote{\textcolor{blue}{Jakob Wassermann}} zuſa{\geminationm}en waren) hinaufkommen.)\pend
           \pstart
           – Es freut mich, dſs Sie meine Anſicht von den \textcolor{blue}{Winterſtein}{}\ledrightnote{\textcolor{blue}{Alfred von Winterstein}}’ſchen \textcolor{green}{Gedichten}{}\ledrightnote{→\textcolor{green}{[Gedichte]}} theilen. \label{K_L01809_2v}\edtext{Einmal}{\lemma{\textnormal{\emph{Einmal}}}\Cendnote{\textnormal{vgl. A. S.: \emph{Tagebuch}, 13. 12. 1906}}}\label{K_L01809_2h} hab
                    ich ſchon an \textcolor{blue}{Bie}{}\ledrightnote{\textcolor{blue}{Oskar Bie}} geſchrieben u ihm \textcolor{green}{Gedichte}{}\ledrightnote{→\textcolor{green}{[Gedichte]}} von \textcolor{blue}{W.}{}\ledrightnote{\textcolor{blue}{Alfred von Winterstein}} geſchickt, es waren aber viel ſchwächere
                    als diesmal; we{\geminationn} Sie glauben, ſo könnte man doch
                    die \textcolor{brown}{\textsc{N. Rdsch}}{}\ledrightnote{\textcolor{brown}{Neue Rundschau, Neue Deutsche Rundschau, Freie Bühne}} noch einmal {\pb}verſuchen; ein paar Zeilen von
                    Ihnen denk ich wären von allergrößtem Werth. Übrigens ſchreib ich auch an den
                    Baron \textcolor{blue}{W.}{}\ledrightnote{\textcolor{blue}{Alfred von Winterstein}}, vielleicht hat er eine andre Bitte
                    an Sie. –\pend
           \pstart
           Also auf ſehr baldiges Wiederſehen; u herzliche Grüße.\pend
           \pstart
           Ihr{\\[\baselineskip]}\spacefill\mbox{Arthur}\pend
           \leftskip=0em{}\endnumbering\briefempfaengerindex{Hofmannsthal, Hugo von@\textsc{Hofmannsthal, Hugo von}!zzzSchnitzler, Arthur@\emph{von Arthur Schnitzler}!1908-11-261@{26. 11. 1908}|)be}\mylabel{h}  \normalsize

\doendnotes{C}
\bigskip
\vfill

\clearpage

\footnotesize

\lohead{\textsc{register}}

% Definiere theindex-Environment komplett neu ohne reledmac
\makeatletter
\renewenvironment{theindex}{%
  \section*{\indexname}%
  \setlength{\parindent}{0pt}%
  \setlength{\parskip}{0pt plus 0.3pt}%
  \let\item\@idxitem
}{%
  \clearpage
}
\makeatother

\IfFileExists{\jobname-pw.ind}{\input{\jobname-pw.ind}}{}

\end{document}

      