%% latex-korrekturansicht-vorspann.tex
%% Vorspann für die Korrekturansicht.
%% Lädt die gemeinsame Datei latex-vorspann.tex mit gesetztem Schalter.

\newif\ifkorrekturansicht
\korrekturansichttrue

\input{../tex-inputs/latex-vorspann}


               \section[Arthur Schnitzler an Richard Beer-Hofmann, 19. 8. 1907]{ Arthur Schnitzler an Richard Beer-Hofmann, 19. 8. 1907}\nopagebreak\mylabel{v}\rehead{ }\normalsize\beginnumbering\briefempfaengerindex{Beer-Hofmann, Richard@\textsc{Beer-Hofmann, Richard}!zzzSchnitzler, Arthur@\emph{von Arthur Schnitzler}!1907-08-191@{19. 8. 1907}|(be} \toendnotes[C]{\smallbreak\pagebreak[2]} \Standort{YCGL, MSS 31.}
\physDesc{Bildpostkarte
\newline{}Handschrift: Bleistift, deutsche Kurrent\newline{}Versand: 1) Stempel: »\nobreak{}\oindex{Wildbad Waldbrunn@\textbf{Wildbad Waldbrunn}, \emph{Hotel (K.HTL)}|pwk}Wildbad Waldbrunn, 19. Aug. 1907\nobreak{}«.  2) Stempel: »\nobreak{}\oindex{Welsberg-Taisten@\textbf{Welsberg-Taisten}, \emph{Besiedelter Ort (A.BSO)}|pwk}Wel{[}sberg{]}, 19. \textcolor{gray}{8}. 07\nobreak{}«. 3) Stempel: »\nobreak{}\oindex{XVIII., Waehring@\textbf{XVIII., Währing}, \emph{Bezirk (A.BZK)}|pwk}18/1 Wien 110, 22. VIII. 07, IX\nobreak{}«. 4) mit schwarzer Tinte von unbekannter Hand nachgesandt nach: »\textcolor{pink}{Kärnten}{ }\textcolor{pink}{Villach}{ }\textcolor{pink}{Hotel Moser}«
\newline{}Beer-Hofmann: mit Bleistift das Datum der Beantwortung vermerkt: »B
                                       23/VIII 07« }\buchAbdrucke{\weitereDrucke{Arthur Schnitzler, Richard Beer-Hofmann: \emph{Briefwechsel 1891–1931}. Hg. Konstanze Fliedl. Wien, Zürich: \emph{Europaverlag} 1992, S. 182–183.} }\toendnotes[C]{\smallbreak}\pstart{}{\pb}\textsc{Dr. Richard}\pend{}\pstart{}\textsc{Beer-Hofmann}\pend{}\pstart{}\textcolor{pink}{Wien XVIII}{}\ledrightnote{\textcolor{pink}{XVIII., Währing}}\pend{}\pstart{}\textcolor{pink}{\textsc{Hasenauerstr} 59}{}\ledrightnote{\textcolor{pink}{Hasenauerstraße}}. \pend{}{\bigskip}\pstart
           \noindent{}\centering{}{\pb}\textcolor{gray}{\textbf{\textcolor{pink}{Wildbad Waldbrunn}{}\ledrightnote{\textcolor{pink}{Wildbad Waldbrunn}} bei \textcolor{pink}{Welsberg}{}\ledrightnote{\textcolor{pink}{Welsberg-Taisten}} im \textcolor{pink}{Pustertale}{}\ledrightnote{\textcolor{pink}{Pustertal}}.}}\pend
           \pstart
           \raggedleft{}{\pb}19. 8. 907\pend
           \pstart
           lieber Richard; wir bleiben hier bis 26.
                  (27.. 28)?; (Donnerſtag ko{\geminationm}t vielleicht \label{K_L01700_1v}\edtext{\textcolor{blue}{Goldmann}{}\ledrightnote{\textcolor{blue}{Paul Goldmann}}}{\lemma{\textnormal{\emph{Goldmann}}}\Cendnote{\textnormal{Er kam nicht.}}}\label{K_L01700_1h}) da{\geminationn}{ }\textcolor{pink}{Dolomitenſtraße}{}\ledrightnote{\textcolor{pink}{Große Dolomitenstraße}} (wahrſcheinlich \textcolor{pink}{Grödner Thal}{}\ledrightnote{\textcolor{pink}{Val Badia}} – \textcolor{pink}{Grödner Joch}{}\ledrightnote{\textcolor{pink}{Grödner Joch}}{ }{[}–{]}{ }\textcolor{pink}{Pordoj}{}\ledrightnote{\textcolor{pink}{Pordoijoch}} – \textcolor{pink}{Vigo}{}\ledrightnote{\textcolor{pink}{Vigo di Fassa}} –
                  \textcolor{pink}{Karerſee}{}\ledrightnote{\textcolor{pink}{Karersee}}, – \textcolor{pink}{Bozen}{}\ledrightnote{\textcolor{pink}{Bozen}}.) da{\geminationn}{ }\textcolor{pink}{Meran}{}\ledrightnote{\textcolor{pink}{Meran}}, eventuell \textcolor{pink}{Gardaſee}{}\ledrightnote{\textcolor{pink}{Lago di Garda}}, zurück zwiſchen 5 u. 10, mit event.
               Aufenthalt (\textcolor{pink}{I{\geminationn}sbruck}{}\ledrightnote{\textcolor{pink}{Innsbruck}},
                  \textcolor{pink}{Salzburg}{}\ledrightnote{\textcolor{pink}{Salzburg}}.) Auf Wiederſehen hoffentlich.\pend
           \pstart
           Herzlichſt Ihr{\\[\baselineskip]}\spacefill\mbox{A.}\pend
           \leftskip=0em{}\pstart
           \noindent{}Schreiben Sie ein Wort! Viele Grüße über die Lande.\pend
           \endnumbering\briefempfaengerindex{Beer-Hofmann, Richard@\textsc{Beer-Hofmann, Richard}!zzzSchnitzler, Arthur@\emph{von Arthur Schnitzler}!1907-08-191@{19. 8. 1907}|)be}\mylabel{h}  \normalsize

\doendnotes{C}
\bigskip
\vfill

\clearpage

\footnotesize

\lohead{\textsc{register}}

% Definiere theindex-Environment komplett neu ohne reledmac
\makeatletter
\renewenvironment{theindex}{%
  \section*{\indexname}%
  \setlength{\parindent}{0pt}%
  \setlength{\parskip}{0pt plus 0.3pt}%
  \let\item\@idxitem
}{%
  \clearpage
}
\makeatother

\IfFileExists{\jobname-pw.ind}{\input{\jobname-pw.ind}}{}

\end{document}

      