%% latex-korrekturansicht-vorspann.tex
%% Vorspann für die Korrekturansicht.
%% Lädt die gemeinsame Datei latex-vorspann.tex mit gesetztem Schalter.

\newif\ifkorrekturansicht
\korrekturansichttrue

\input{../tex-inputs/latex-vorspann}


               \section[Hugo von Hofmannsthal an Arthur Schnitzler, 12. 11. 1907]{ Hugo von Hofmannsthal an Arthur Schnitzler, 12. 11. 1907}\nopagebreak\mylabel{v}\rehead{ }\normalsize\beginnumbering\briefempfaengerindex{Schnitzler, Arthur@\textsc{Schnitzler, Arthur}!zzzHofmannsthal, Hugo von@\emph{von Hugo von Hofmannsthal}!1907-11-121@{12. 11. 1907}|(be} \toendnotes[C]{\smallbreak\pagebreak[2]} \Standort{CUL, Schnitzler, B 43.}
\physDesc{Postkarte
\newline{}Handschrift: Bleistift, lateinische Kurrent\newline{}Versand: 1) Rohrpost 2) Stempel: »\nobreak{}\oindex{XVII., Hernals@\textbf{XVII., Hernals}, \emph{Bezirk (A.BZK)}|pwk}1\textcolor{gray}{7}/1 Wien, 12 XI 07, 8\textsuperscript{30}N\nobreak{}«. 3) Stempel: »\nobreak{}\oindex{XVIII., Waehring@\textbf{XVIII., Währing}, \emph{Bezirk (A.BZK)}|pwk}18/1 Wien  111, 12 XI 07, 9\textsuperscript{10}\nobreak{}«. 
\newline{}Schnitzler: mit Bleistift datiert: »Nov 907« \newline{}Ordnung: mit Bleistift von unbekannter Hand nummeriert: »284 287« }\buchAbdrucke{\weitereDrucke{Hugo von Hofmannsthal, Arthur Schnitzler: \emph{Briefwechsel}. Hg. Therese Nickl und Heinrich Schnitzler. Frankfurt am Main: \emph{S. Fischer} 1964, S. 233.} }\pstart{}{\pb}D\textsuperscript{r} A Schnitzler\pend{}\pstart{}\textcolor{pink}{Wien}{}\ledrightnote{\textcolor{pink}{Wien}}\pend{}\pstart{}\textcolor{pink}{XVII Spöttelgasse 7}{}\ledrightnote{\textcolor{pink}{Edmund-Weiß-Gasse}}\pend{}{\bigskip}\pstart
           {\pb}Dienstg{ }abend.\pend
           \pstart
           Könnten wir zur nötig. Aufheiterung morgen \introOben{}= \uline{Mittwoch}\introOben{}{ }7\textsuperscript{h} zu Euch? (event. zusa{\geminationm}en dann in ein
               Variété)\pend
           \pstart
           Antwort trifft uns (teleph oder pneumat) bis 3\textsuperscript{h} 30
            bei \textcolor{blue}{Schlesinger}{}\ledrightnote{\textcolor{blue}{Franziska Schlesinger}}\pend
           \pstart
           \textcolor{pink}{Elisabethstrasse 6}{}\ledrightnote{\textcolor{pink}{Elisabethstraße}}\pend
           \pstart
           Telephon 229\pend
           \endnumbering\briefempfaengerindex{Schnitzler, Arthur@\textsc{Schnitzler, Arthur}!zzzHofmannsthal, Hugo von@\emph{von Hugo von Hofmannsthal}!1907-11-121@{12. 11. 1907}|)be}\mylabel{h}  \normalsize

\doendnotes{C}
\bigskip
\vfill

\clearpage

\footnotesize

\lohead{\textsc{register}}

% Definiere theindex-Environment komplett neu ohne reledmac
\makeatletter
\renewenvironment{theindex}{%
  \section*{\indexname}%
  \setlength{\parindent}{0pt}%
  \setlength{\parskip}{0pt plus 0.3pt}%
  \let\item\@idxitem
}{%
  \clearpage
}
\makeatother

\IfFileExists{\jobname-pw.ind}{\input{\jobname-pw.ind}}{}

\end{document}

      