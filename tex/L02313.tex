%% latex-korrekturansicht-vorspann.tex
%% Vorspann für die Korrekturansicht.
%% Lädt die gemeinsame Datei latex-vorspann.tex mit gesetztem Schalter.

\newif\ifkorrekturansicht
\korrekturansichttrue

\input{../tex-inputs/latex-vorspann}


               \section[Hugo von Hofmannsthal an Arthur Schnitzler, {[}Anfang Dezember 1918{]}]{ Hugo von Hofmannsthal an Arthur Schnitzler, {[}Anfang Dezember
               1918{]}}\nopagebreak\mylabel{v}\rehead{ }\normalsize\beginnumbering\briefempfaengerindex{Schnitzler, Arthur@\textsc{Schnitzler, Arthur}!zzzHofmannsthal, Hugo von@\emph{von Hugo von Hofmannsthal}!1918-12-011@{{[}Anfang Dezember
                  1918{]}}|(be} \toendnotes[C]{\smallbreak\pagebreak[2]} \Standort{CUL, Schnitzler, B 43.}
\physDesc{Brief, 1 Blatt, 1 Seite
\newline{}Handschrift: schwarze Tinte, deutsche Kurrent
\newline{}Schnitzler: 1) mit Bleistift datiert: »Anf Dez. 918« und beschriftet: »\textsc{Hugo}« 2) mit rotem Buntstift eine Unterstreichung\newline{}Ordnung: 1) mit Bleistift von \textcolor{blue}{Frieda Pollak} (?) mit dem Buchstaben »A« (Abgeschrieben/Abschrift) gekennzeichnet 2) mit Bleistift von unbekannter Hand nummeriert: »\strikeout{351}«3) mit Bleistift von unbekannter Hand nummeriert: »360«}\buchAbdrucke{\weitereDrucke{Hugo von Hofmannsthal, Arthur Schnitzler: \emph{Briefwechsel}. Hg. Therese Nickl und Heinrich Schnitzler. Frankfurt am Main: \emph{S. Fischer} 1964, S. 288.} }\toendnotes[C]{\smallbreak}\pstart
           \raggedleft{}{\pb}\textcolor{pink}{Wien}{}\ledrightnote{\textcolor{pink}{Wien}}{\\}\textcolor{pink}{Stallburggaſſe 2}{}\ledrightnote{\textcolor{pink}{Stallburggasse}}\pend
           \pstart{}mein lieber Arthur \pend\pstart
           ſeit mehr als 10 Tagen ſind wir ganz herinnen, \textcolor{blue}{Gerty}{}\ledrightnote{\textcolor{blue}{Gertrude von Hofmannsthal}} ist hier krank geworden, befindet ſich aber ſchon wieder wohl und
                  Sonntag werden wir für einige Zeit wieder hinausziehen, doch läſst
               ſich draußen in einem finſteren und kaum über {\pb}11° heizbaren \textcolor{pink}{Haus}{}\ledrightnote{→\textcolor{pink}{Hofmannsthal-Schlössl}} mehr vegetieren als leben.\hspace*{1.5em}– Aber nicht davon wollte ich ſprechen ſondern ſagen
               daſs ich Sie und \textcolor{blue}{Olga}{}\ledrightnote{\textcolor{blue}{Olga Schnitzler}} unendlich gern ſehen möchte
               und in dieſen Tagen durch wiederholtes Anrufen vergeblich dies zu betätigen verſucht
               habe. Ich wollte anfragen ob ich eines Vormittags zu Ihnen hinausko{\geminationm}en könnte! Indeſſen hab ich aber gehört daſs Sie {\pb}Proben zum \textcolor{green}{Profeſſor Bernhardi}{}\ledrightnote{\textcolor{green}{Professor Bernhardi. Komödie in fünf Akten}} haben – ſo nehme ich an daſs Ihre Vormittage
               beſetzt ſind und zwar wie ich hoffe in einer Weiſe die Sie über das halb Gräſsliche
               halb Läppiſche das uns umgibt einigermaßen hinaushebt wofür ich Sie gewiſſermaßen
               beneide.\pend
           \pstart
           Bitte wenn das vorbei iſt, {\pb}ſo
               ſchreiben Sie mir eine Zeile und vielleicht ko{\geminationm}t Ihr
               dann endlich einmal in die \textcolor{pink}{Stallburggaſſe}{}\ledrightnote{\textcolor{pink}{Stallburggasse}}, etwa
               mit einem Concert es verbindend – oder wenn Ihnen das lieber ist, ſo ko{\geminationm}e ich hinaus.\hspace*{1.5em}Ihnen
               und \textcolor{blue}{Olga}{}\ledrightnote{\textcolor{blue}{Olga Schnitzler}} alles Liebe \pend
           \pstart
           von Ihrem{\\[\baselineskip]}\spacefill\mbox{Hugo.}\pend
           \leftskip=0em{}\endnumbering\briefempfaengerindex{Schnitzler, Arthur@\textsc{Schnitzler, Arthur}!zzzHofmannsthal, Hugo von@\emph{von Hugo von Hofmannsthal}!1918-12-011@{{[}Anfang Dezember
                  1918{]}}|)be}\mylabel{h}  \normalsize

\doendnotes{C}
\bigskip
\vfill

\clearpage

\footnotesize

\lohead{\textsc{register}}

% Definiere theindex-Environment komplett neu ohne reledmac
\makeatletter
\renewenvironment{theindex}{%
  \section*{\indexname}%
  \setlength{\parindent}{0pt}%
  \setlength{\parskip}{0pt plus 0.3pt}%
  \let\item\@idxitem
}{%
  \clearpage
}
\makeatother

\IfFileExists{\jobname-pw.ind}{\input{\jobname-pw.ind}}{}

\end{document}

      