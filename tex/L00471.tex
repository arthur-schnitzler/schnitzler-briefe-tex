%% latex-korrekturansicht-vorspann.tex
%% Vorspann für die Korrekturansicht.
%% Lädt die gemeinsame Datei latex-vorspann.tex mit gesetztem Schalter.

\newif\ifkorrekturansicht
\korrekturansichttrue

\input{../tex-inputs/latex-vorspann}


               \section[Hugo von Hofmannsthal an Arthur Schnitzler, 9. 8. {[}1895{]}]{ Hugo von Hofmannsthal an Arthur Schnitzler, 9. 8. {[}1895{]}}\nopagebreak\mylabel{v}\rehead{ }\normalsize\beginnumbering\briefempfaengerindex{Schnitzler, Arthur@\textsc{Schnitzler, Arthur}!zzzHofmannsthal, Hugo von@\emph{von Hugo von Hofmannsthal}!1895-08-092@{9. 8. {[}1895{]}}|(be} \toendnotes[C]{\smallbreak\pagebreak[2]} \Standort{CUL, Schnitzler, B 43.}
\physDesc{Brief, 1 Blatt, 4 Seiten
\newline{}Handschrift: schwarze Tinte, deutsche Kurrent
\newline{}Schnitzler: mit Bleistift die Jahreszahl ergänzt: »95«
            und nummeriert: »74« }\buchAbdrucke{\weitereDrucke{1) Hugo von Hofmannsthal: \emph{Briefe. 1890–1901}. Berlin: \emph{S. Fischer} 1935, S. 164–165.} \weitereDrucke{2) Hugo von Hofmannsthal, Arthur Schnitzler: \emph{Briefwechsel}. Hg. Therese Nickl und Heinrich Schnitzler. Frankfurt am Main: \emph{S. Fischer} 1964, S. 58–59.} }\pstart
           \raggedleft{}{\pb}\textcolor{pink}{Göding}{}\ledrightnote{\textcolor{pink}{Hodonín}}{ }9. Auguſt\pend
           \pstart{}lieber Arthur\pend\pstart
           es iſt doch ſehr merkwürdig, ſo wider ſeine Natur zu leben, wie ich es jetzt
                    thue, unter Menſchen, denen jeder Antheil ſchon faſt wie Affectation erſcheint.
                    Ich bin begierig, wie ich das ſehen werde, wenn ich von dem unmittelbaren Zwang
                    befreit bin. Euch vermuthe ich mit den \textcolor{pink}{däniſchen}{}\ledrightnote{\textcolor{pink}{Dänemark}} Buchten und der \textcolor{pink}{München}{}\ledrightnote{\textcolor{pink}{München}}er
                    Bilderausſtellung in {\pb}Gedanken ſo ſpielend, wie mit Spielereien die noch in der Schachtel ſind. Es
                    kränkt mich, daſs mir der \textcolor{blue}{Richard}{}\ledrightnote{\textcolor{blue}{Richard Beer-Hofmann}} nicht
                    ſchreibt. Seit 6 Wochen hat er mir \uline{einen} Brief
                    geſchrieben, obwohl er weiß, daſs ich eine kindiſche Freude über jeden Brief
                    hab, und hier wirklich wenig habe was mir Freud macht. Sonntag iſt das Rennen.
                    Wenn ich an die Bretterwand hinflieg und mir das Genick brech (unwahrſcheinlich,
                        {\pb}aber möglich) ſollt Ihr
                    meine vielen Notizen auf Zetteln herausgeben, in Gedankengruppen geordnet, mit
                    einem ſehr einfachen, die Aſſociationen aufdeckenden Commentar. Denn meine
                    Gedanken gehören alle zuſammen, weil ich von der Einheit der Welt ſehr ſtark
                    durchdrungen bin. Ich glaub ſogar ein Dichter iſt eben ein Menſch, dem in guten
                    Stunden die Gedanken »ausgehen« wie man beim Patiencelegen ſagt. – Am
                            15\textsuperscript{ten} iſt Abmarſch {\pb}nach
                        \textcolor{pink}{Znaim}{}\ledrightnote{\textcolor{pink}{Znaim}}, dann \textcolor{pink}{Stockerau}{}\ledrightnote{\textcolor{pink}{Stockerau}} etc. etc. Bitte alſo Briefe vom 14\textsuperscript{ten} an nach \textcolor{pink}{Wien}{}\ledrightnote{\textcolor{pink}{Wien}} richten, von wo ſie
                    nachgeſchickt werden.\pend
           \pstart
           Auf Wiederſehen!{\\[\baselineskip]}\spacefill\mbox{Hugo.}\pend
           \leftskip=0em{}\pstart
           \noindent{}Bitte können Sie in Erfahrung bringen ob D\textsuperscript{r}{ }\textcolor{blue}{Mamroth}{}\ledrightnote{\textcolor{blue}{Fedor Mamroth}} nicht mehr bei der \textcolor{brown}{Frankf.}{}\ledrightnote{\textcolor{brown}{Frankfurter Zeitung}} iſt, oder beurlaubt? und mir das
                        ſchreiben? \pend
           \endnumbering\briefempfaengerindex{Schnitzler, Arthur@\textsc{Schnitzler, Arthur}!zzzHofmannsthal, Hugo von@\emph{von Hugo von Hofmannsthal}!1895-08-092@{9. 8. {[}1895{]}}|)be}\mylabel{h}  \normalsize

\doendnotes{C}
\bigskip
\vfill

\clearpage

\footnotesize

\lohead{\textsc{register}}

% Definiere theindex-Environment komplett neu ohne reledmac
\makeatletter
\renewenvironment{theindex}{%
  \section*{\indexname}%
  \setlength{\parindent}{0pt}%
  \setlength{\parskip}{0pt plus 0.3pt}%
  \let\item\@idxitem
}{%
  \clearpage
}
\makeatother

\IfFileExists{\jobname-pw.ind}{\input{\jobname-pw.ind}}{}

\end{document}

      