%% latex-korrekturansicht-vorspann.tex
%% Vorspann für die Korrekturansicht.
%% Lädt die gemeinsame Datei latex-vorspann.tex mit gesetztem Schalter.

\newif\ifkorrekturansicht
\korrekturansichttrue

\input{../tex-inputs/latex-vorspann}


               \section[Friedrich M. Fels an Arthur Schnitzler, 4. 10. 1895]{ Friedrich M. Fels an Arthur Schnitzler, 4. 10. 1895}\nopagebreak\mylabel{v}\rehead{ }\normalsize\beginnumbering\briefempfaengerindex{Schnitzler, Arthur@\textsc{Schnitzler, Arthur}!zzzFels, Friedrich Michael@\emph{von Friedrich Michael Fels}!1895-10-043@{4. 10. 1895}|(be} \toendnotes[C]{\smallbreak\pagebreak[2]} \Standort{DLA, A:Schnitzler, HS.NZ85.1.2956.}
\physDesc{Brief, 1 Blatt, 3 Seiten
\newline{}Handschrift: schwarze Tinte, lateinische Kurrent
\newline{}Schnitzler: mit Bleistift nummeriert: »26« }\toendnotes[C]{\smallbreak}\pstart
           \raggedleft{}{\pb}\textcolor{pink}{Zürich I, Schifflände 30}{}\ledrightnote{\textcolor{pink}{Schifflände}},
                        III. Stock{\\}am 4. Oktober 1895\pend
           \pstart\center{}Lieber Doktor Schnitzler!\pend\pstart
           Wie Sie aus der Datierung ersehen, bin ich, dank Ihrer und \textcolor{blue}{Beer-Hofma{\geminationn}}{}\ledrightnote{\textcolor{blue}{Richard Beer-Hofmann}}s Hilfe, wieder im Besitze einer eigenen Wohnung. Ich danke Ihnen herzlich.
                    Ich wohne jetzt bei einer beka{\geminationn}ten \textcolor{blue}{Familie}{}\ledrightnote{→\textcolor{blue}{Julius Ott}{\newline}→\textcolor{blue}{Anna Elisabetha Ott}}, zusa{\geminationm}en mit einem \textcolor{blue}{Freunde}{}\ledrightnote{→\textcolor{blue}{Meichl}}, einem alten Herrn, \textcolor{pink}{Wien}{}\ledrightnote{\textcolor{pink}{Wien}}er, Schwager von \textcolor{blue}{Dreher}{}\ledrightnote{\textcolor{blue}{Carl Anton Dreher}} in \textcolor{pink}{Schwechat}{}\ledrightnote{\textcolor{pink}{Schwechat}}, der früher
                    lange Jahre in \textcolor{pink}{Amerika}{}\ledrightnote{\textcolor{pink}{Amerika}} und \textcolor{pink}{Deutschland}{}\ledrightnote{\textcolor{pink}{Deutschland}} ein groſser Fabrikant war, da{\geminationn} fallierte und nun in seinen alten Tagen als
                    Reisender eines Papiergeschäfts mühsam sein Leben fristet. Wir haben zusa{\geminationm}en ein groſses Wohnzi{\geminationm}er, ein Kabinet und einen Alkoven, wofür wir 50 francs zahlen – gewiſs billig.
                    Na, der Teufel wird schon weiterhelfen.\pend
           \pstart
           Ich hätte noch eine Bitte. Wären Sie so freundlich, bei \textcolor{blue}{Beer-Hofma{\geminationn}}{}\ledrightnote{\textcolor{blue}{Richard Beer-Hofmann}} nachzufragen, ob er vielleicht wieder einen {\pb}alten Anzug hat; das Porto ka{\geminationn} ja nicht viel kosten. Und ich bin absolut
                    auſserstande, mir selbst einen beizubringen. Seien Sie nicht böse, und besten
                    Dank im vorhinein.\pend
           \pstart
           Ich schreibe wirklich einen \textcolor{green}{Aufsatz}{}\ledrightnote{→\textcolor{green}{Die Volkslieder der Bulgaren}} für \textcolor{blue}{Wengraf}{}\ledrightnote{\textcolor{blue}{Edmund Wengraf}} und \textcolor{blue}{Osten}{}\ledrightnote{\textcolor{blue}{Heinrich Osten}} und werde da{\geminationn}{ }\label{K_L00499_1v}\edtext{einen}{\lemma{\textnormal{\emph{einen}}}\Cendnote{\textnormal{nicht nachgewiesen}}}\label{K_L00499_1h} für die \textcolor{green}{Preſse}{}\ledrightnote{\textcolor{green}{Die Presse}}{ }ſchreiben. Apropos \textcolor{brown}{Preſse}{}\ledrightnote{\textcolor{brown}{Die Presse}}: Dr. \textcolor{blue}{Hirschfeld}{}\ledrightnote{\textcolor{blue}{Robert Hirschfeld}} muſs ja
                    jetzt wieder in \textcolor{pink}{Wien}{}\ledrightnote{\textcolor{pink}{Wien}}{ }ſein, und Sie kö{\geminationn}ten vielleicht bei Gelegenheit mit ihm sprechen, ob es sich nicht machen
                    lieſse, daſs ich für das Blatt die \textcolor{pink}{Schweiz}{}\ledrightnote{\textcolor{pink}{Schweiz}}er
                    Korrespondenz, auch über Politik und Volkswirtschaft, übernähme. Ich haben
                        bego{\geminationn}en, mich in die Verhältniſse einzuleben,
                    und glaube, daſs ich genügen würde.\pend
           \pstart
           Daſs \textcolor{blue}{Mackay}{}\ledrightnote{\textcolor{blue}{John Henry Mackay}} Ihnen gefallen hat, freut mich.
                    Auch ich habe ihn gern. Er hat, bei viel Schlauheit und einiger Reserviertheit,
                    viele liebenswürdige Seiten, vor allem eine sehr angenehme Naivetät. Naiv ist
                    zwar auch \textcolor{blue}{Henckell}{}\ledrightnote{\textcolor{blue}{Karl Friedrich Henckell}}, dabei aber entsetzlich
                    langweilig und geistlos. Sie haben mich einen Antisemiten gena{\geminationn}t, aber – mit Ariern verkehrt es sich wirklich zu
                    schwer.\pend
           \pstart
           {\pb}Nehmen Sie mir meine neue Bitte nicht
                    übel, grüßen Sie \textcolor{blue}{Beer-Hofma{\geminationn}}{}\ledrightnote{\textcolor{blue}{Richard Beer-Hofmann}}, \textcolor{blue}{Loris}{}\ledrightnote{\textcolor{blue}{Hugo von Hofmannsthal}}, \textcolor{blue}{Hirschfeld}{}\ledrightnote{\textcolor{blue}{Robert Hirschfeld}} etc von mir und seien Sie selbst herzlichst gegrüßt\pend
           \pstart
           von{\\[\baselineskip]}Ihrem{\\[\baselineskip]}\spacefill\mbox{Fels}\pend
           \leftskip=0em{}\pstart
           \noindent{}Was sagen Sie zu \textcolor{blue}{Mackay}{}\ledrightnote{\textcolor{blue}{John Henry Mackay}}s neuestem \textcolor{green}{Buch}{}\ledrightnote{→\textcolor{green}{Albert Schnell’s Untergang. Eine Geschichte ohne Handlung}}? Erscheint bald
                        wieder etwas von Ihnen? Wie stehts mit der \textcolor{green}{Aufführung}{}\ledrightnote{→\textcolor{green}{Liebelei. Schauspiel in drei Akten}}? \textcolor{green}{\textcolor{blue}{David}{}\ledrightnote{\textcolor{blue}{Jakob Julius David}}}{}\ledrightnote{→\textcolor{green}{Ein Regentag}} ko{\geminationm}t also am 12. daran; ich
                        bin begierig.\pend
           \endnumbering\briefempfaengerindex{Schnitzler, Arthur@\textsc{Schnitzler, Arthur}!zzzFels, Friedrich Michael@\emph{von Friedrich Michael Fels}!1895-10-043@{4. 10. 1895}|)be}\mylabel{h}  \normalsize

\doendnotes{C}
\bigskip
\vfill

\clearpage

\footnotesize

\lohead{\textsc{register}}

% Definiere theindex-Environment komplett neu ohne reledmac
\makeatletter
\renewenvironment{theindex}{%
  \section*{\indexname}%
  \setlength{\parindent}{0pt}%
  \setlength{\parskip}{0pt plus 0.3pt}%
  \let\item\@idxitem
}{%
  \clearpage
}
\makeatother

\IfFileExists{\jobname-pw.ind}{\input{\jobname-pw.ind}}{}

\end{document}

      