%% latex-korrekturansicht-vorspann.tex
%% Vorspann für die Korrekturansicht.
%% Lädt die gemeinsame Datei latex-vorspann.tex mit gesetztem Schalter.

\newif\ifkorrekturansicht
\korrekturansichttrue

\input{../tex-inputs/latex-vorspann}


               \section[Hugo von Hofmannsthal an Arthur Schnitzler, {[}13. 1. 1893{]}]{ Hugo von Hofmannsthal an Arthur Schnitzler, {[}13. 1. 1893{]}}\nopagebreak\mylabel{v}\rehead{ }\normalsize\beginnumbering\briefempfaengerindex{Schnitzler, Arthur@\textsc{Schnitzler, Arthur}!zzzHofmannsthal, Hugo von@\emph{von Hugo von Hofmannsthal}!1893-01-131@{{[}13. 1. 1893{]}}|(be} \toendnotes[C]{\smallbreak\pagebreak[2]} \Standort{CUL, Schnitzler, B 43.}
\physDesc{Briefkarte mit aufgeprägtem Wappen
\newline{}Handschrift: schwarze Tinte, deutsche Kurrent
\newline{}Schnitzler: mit Bleistift datiert: »13/1 93« \newline{}Ordnung: mit Bleistift von unbekannter Hand nummeriert:
                                        »39« }\buchAbdrucke{\weitereDrucke{Hugo von Hofmannsthal, Arthur Schnitzler: \emph{Briefwechsel}. Hg. Therese Nickl und Heinrich Schnitzler. Frankfurt am Main: \emph{S. Fischer} 1964, S. 35.} }\toendnotes[C]{\smallbreak}\pstart
           \raggedleft{}{\pb}Freitag.\pend
           \pstart{}mein lieber Arthur.\pend\pstart
           Ich habe den \textcolor{green}{Sitz}{}\ledrightnote{→\textcolor{green}{Die Räuber}} für \textsc{Samstag} natürlich
                    genommen, kann aber leider nicht gehen, weil am ſelben Abend eine \label{K_L00159_1v}\edtext{Vorleſung \textcolor{blue}{F. v. \textsc{Saars}}{}\ledrightnote{\textcolor{blue}{Ferdinand von Saar}}}{\lemma{\textnormal{\emph{Vorleſung F. v. Saars}}}\Cendnote{\textnormal{Die Lesung fand am
                            14. 1. 1893 im \textcolor{pink}{Kleinen
                            Musikvereinssaal} statt.}}}\label{K_L00159_1h}{ }ſtattfindet, zu der zu kommen ich ſeit langer
                    Zeit verſprochen habe. Ich hoffe aber beſtimmt, wenn mir nicht abgeſchrieben
                    wird, \textcolor{blue}{Richard}{}\ledrightnote{\textcolor{blue}{Richard Beer-Hofmann}} u. \textcolor{blue}{Salten}{}\ledrightnote{\textcolor{blue}{Felix Salten}} am Sonntag bei Ihnen zu treffen und
                    wünſche Euch für \textsc{Samstag} beſte Unterhaltung.\pend
           \pstart
           Herzlichſt Ihr{\\[\baselineskip]}\spacefill\mbox{Hugo}\pend
           \leftskip=0em{}\pstart
           \noindent{}ehemals Schriftſteller.\pend
           \endnumbering\briefempfaengerindex{Schnitzler, Arthur@\textsc{Schnitzler, Arthur}!zzzHofmannsthal, Hugo von@\emph{von Hugo von Hofmannsthal}!1893-01-131@{{[}13. 1. 1893{]}}|)be}\mylabel{h}  \normalsize

\doendnotes{C}
\bigskip
\vfill

\clearpage

\footnotesize

\lohead{\textsc{register}}

% Definiere theindex-Environment komplett neu ohne reledmac
\makeatletter
\renewenvironment{theindex}{%
  \section*{\indexname}%
  \setlength{\parindent}{0pt}%
  \setlength{\parskip}{0pt plus 0.3pt}%
  \let\item\@idxitem
}{%
  \clearpage
}
\makeatother

\IfFileExists{\jobname-pw.ind}{\input{\jobname-pw.ind}}{}

\end{document}

      