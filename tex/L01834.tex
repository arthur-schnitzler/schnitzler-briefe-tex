%% latex-korrekturansicht-vorspann.tex
%% Vorspann für die Korrekturansicht.
%% Lädt die gemeinsame Datei latex-vorspann.tex mit gesetztem Schalter.

\newif\ifkorrekturansicht
\korrekturansichttrue

\input{../tex-inputs/latex-vorspann}


               \section[Arthur Schnitzler an Hugo von Hofmannsthal, 25. 3. 1909]{ Arthur Schnitzler an Hugo von Hofmannsthal, 25. 3. 1909}\nopagebreak\mylabel{v}\rehead{ }\normalsize\beginnumbering\briefempfaengerindex{Hofmannsthal, Hugo von@\textsc{Hofmannsthal, Hugo von}!zzzSchnitzler, Arthur@\emph{von Arthur Schnitzler}!1909-03-251@{25. 3. 1909}|(be} \toendnotes[C]{\smallbreak\pagebreak[2]} \Standort{FDH, Hs-30885,134.}
\physDesc{Brief, 1 Blatt, 2 Seiten
\newline{}Handschrift: schwarze Tinte, deutsche Kurrent}\buchAbdrucke{\weitereDrucke{Hugo von Hofmannsthal, Arthur Schnitzler: \emph{Briefwechsel}. Hg. Therese Nickl und Heinrich Schnitzler. Frankfurt am Main: \emph{S. Fischer} 1964, S. 244.} }\pstart
           \noindent{}{\pb}\textcolor{gray}{\textbf{Dr. Arthur
                        Schnitzler}}\hfill 25. 3. 09\pend
           \pstart
           \textcolor{gray}{\textbf{\textcolor{pink}{Wien XVIII. Spoettelgasse 7}{}\ledrightnote{\textcolor{pink}{Edmund-Weiß-Gasse}}.}}\pend
           \pstart
           lieber Hugo, die \textcolor{green}{Elektra}{}\ledrightnote{\textcolor{green}{Elektra (op. 58)}} hat mir
               bei der Generalprobe ſchon einen ſtarken Eindruck gemacht, und geſtern
                  Abend einen noch viel ſtärkeren. Einen reineren hatt’ ich zwiſchen
               Generalprobe und Aufführung, da ich geſtern früh Ihre unver\textcolor{blue}{ſtrauß}{}\ledrightnote{\textcolor{blue}{Richard Strauss}}te \textcolor{green}{Elektra}{}\ledrightnote{\textcolor{green}{Elektra (op. 58)}} wieder las,
               die etwas einfach bewunderungs{\pb}würdiges vorſtellt und der
               ich für meinen Theil geſtern \introOben{}Abend\introOben{} noch
               heftiger applaudirt habe als der wahrhaft\strikeout{igen}
               mächtigen Musik-Begleitung \substVorne{}\textsuperscript{(}\substDazwischen{}(ein Wort\substHinten{} das hier in höchſtem Sinn zu nehmen wäre).\pend
           \pstart
           \textcolor{blue}{Olga}{}\ledrightnote{\textcolor{blue}{Olga Schnitzler}}{ }ſchließt ſich meiner Anſicht, ebenſowie meinen
               Grüßen und Glückwünſchen aufs wärmſte an.\pend
           \pstart
           Ihr{\\[\baselineskip]}\spacefill\mbox{Arthur.}\pend
           \leftskip=0em{}\endnumbering\briefempfaengerindex{Hofmannsthal, Hugo von@\textsc{Hofmannsthal, Hugo von}!zzzSchnitzler, Arthur@\emph{von Arthur Schnitzler}!1909-03-251@{25. 3. 1909}|)be}\mylabel{h}  \normalsize

\doendnotes{C}
\bigskip
\vfill

\clearpage

\footnotesize

\lohead{\textsc{register}}

% Definiere theindex-Environment komplett neu ohne reledmac
\makeatletter
\renewenvironment{theindex}{%
  \section*{\indexname}%
  \setlength{\parindent}{0pt}%
  \setlength{\parskip}{0pt plus 0.3pt}%
  \let\item\@idxitem
}{%
  \clearpage
}
\makeatother

\IfFileExists{\jobname-pw.ind}{\input{\jobname-pw.ind}}{}

\end{document}

      