%% latex-korrekturansicht-vorspann.tex
%% Vorspann für die Korrekturansicht.
%% Lädt die gemeinsame Datei latex-vorspann.tex mit gesetztem Schalter.

\newif\ifkorrekturansicht
\korrekturansichttrue

\input{../tex-inputs/latex-vorspann}


               \section[Max Burckhard an Arthur Schnitzler, 20. 11. 1908]{ Max Burckhard an Arthur Schnitzler, 20. 11. 1908}\nopagebreak\mylabel{v}\rehead{ }\normalsize\beginnumbering\briefempfaengerindex{Schnitzler, Arthur@\textsc{Schnitzler, Arthur}!zzzBurckhard, Max Eugen@\emph{von Max Eugen Burckhard}!1908-11-201@{20. 11. 1908}|(be} \toendnotes[C]{\smallbreak\pagebreak[2]} \Standort{CUL, Schnitzler, B 20.}
\physDesc{Brief, 1 Blatt, 1 Seite
\newline{}Handschrift: schwarze Tinte, deutsche Kurrent\newline{}Ordnung: mit Bleistift von unbekannter Hand nummeriert: »24« }\toendnotes[C]{\smallbreak}\pstart
           \noindent{}{\pb}\textcolor{gray}{\textbf{D\textsuperscript{r.} Max Burckhard}}\hfill \textcolor{gray}{\textbf{\textcolor{pink}{Wien, IX. Porzellangasse 48}{}\ledrightnote{\textcolor{pink}{Porzellangasse}}}}{ }20. XI. 08\pend
           \pstart
           \raggedleft{}\textcolor{gray}{\textbf{\strikeout{St. Gilgen}}}\hspace*{4em}\pend
           \pstart{}Sehr verehrter lieber Herr Doctor!\pend\pstart
           Anbei die \textcolor{green}{3 \textcolor{brown}{Lloyd}{}\ledrightnote{\textcolor{brown}{Pester Lloyd}}-Geſchichten}{}\ledrightnote{→\textcolor{green}{Scala Santa}{\newline}→\textcolor{green}{Der Hund}{\newline}→\textcolor{green}{Ich und mein Bruder}} – ich glaube, wir
                    haben nur von diesen 3 Sachen geſprochen, wenigſtens weiß ich momentan ſonſt
                    nichts und nur ſo ein dunkles Dä{\geminationm}ern ist mir, als
                    wäre noch von was anderm die Rede geweſen außer der \label{K_L01807_1v}\edtext{\textcolor{green}{Generalprobe}{}\ledrightnote{→\textcolor{green}{Die verflixten Frauenzimmer}}}{\lemma{\textnormal{\emph{Generalprobe}}}\Cendnote{\textnormal{Die Generalprobe der vier Einakter \textcolor{blue}{Burckhards}, \emph{\textcolor{green}{Die verflixten Frauenzimmer}}, fand am
                            27. 11. 1908, die Uraufführung am Folgetag am \textcolor{pink}{Deutschen Volkstheater} statt.}}}\label{K_L01807_1h}
                    natürlich, hinſichtlich derer man mir geſagt hat, es genüge zum Einlaſs meine
                    Viſitkarte für Sie, die ich mir also hiermit, herzlichſt um Ihre freundliche
                    Aſſiſtenz bittend, anzuschließen erlaube.\pend
           \pstart
           Mit Handkuſs an die verehrte gnädige Frau und herzlichſten Grüßen Ihr{\\[\baselineskip]}\spacefill\mbox{D\textsuperscript{r}Burckhard}\pend
           \leftskip=0em{}\endnumbering\briefempfaengerindex{Schnitzler, Arthur@\textsc{Schnitzler, Arthur}!zzzBurckhard, Max Eugen@\emph{von Max Eugen Burckhard}!1908-11-201@{20. 11. 1908}|)be}\mylabel{h}  \normalsize

\doendnotes{C}
\bigskip
\vfill

\clearpage

\footnotesize

\lohead{\textsc{register}}

% Definiere theindex-Environment komplett neu ohne reledmac
\makeatletter
\renewenvironment{theindex}{%
  \section*{\indexname}%
  \setlength{\parindent}{0pt}%
  \setlength{\parskip}{0pt plus 0.3pt}%
  \let\item\@idxitem
}{%
  \clearpage
}
\makeatother

\IfFileExists{\jobname-pw.ind}{\input{\jobname-pw.ind}}{}

\end{document}

      