%% latex-korrekturansicht-vorspann.tex
%% Vorspann für die Korrekturansicht.
%% Lädt die gemeinsame Datei latex-vorspann.tex mit gesetztem Schalter.

\newif\ifkorrekturansicht
\korrekturansichttrue

\input{../tex-inputs/latex-vorspann}


               \section[Arthur Schnitzler an Hugo Hofmannsthal, 26. 2. 1927]{ Arthur Schnitzler an Hugo Hofmannsthal, 26. 2. 1927}\nopagebreak\mylabel{v}\rehead{ }\normalsize\beginnumbering\briefempfaengerindex{Hofmannsthal, Hugo von@\textsc{Hofmannsthal, Hugo von}!zzzSchnitzler, Arthur@\emph{von Arthur Schnitzler}!1927-02-261@{26. 2. 1927}|(be} \toendnotes[C]{\smallbreak\pagebreak[2]} \Standort{FDH, Hs-30885,157.}
\physDesc{Postkarte
\newline{}Handschrift: Bleistift, lateinische Kurrent}\buchAbdrucke{\weitereDrucke{Hugo von Hofmannsthal, Arthur Schnitzler: \emph{Briefwechsel}. Hg. Therese Nickl und Heinrich Schnitzler. Frankfurt am Main: \emph{S. Fischer} 1964, S. 307.} }\toendnotes[C]{\smallbreak}\pstart{}{\pb}\label{T_L02482-1v}\edtext{\textcolor{gray}{\textbf{A. S.}}}{\lemma{\textnormal{\emph{A. S.}}}\Cendnote{\textnormal{ovaler Absenderkleber}}}\label{T_L02482-1h}\pend{}\pstart{}\textcolor{pink}{\textcolor{gray}{\textbf{WIEN, XVIII.}}}{}\ledrightnote{\textcolor{pink}{XVIII., Währing}}\pend{}\pstart{}\textcolor{pink}{\textcolor{gray}{\textbf{STERNWARTESTR. 71}}}{}\ledrightnote{\textcolor{pink}{Sternwartestraße}}\pend{}{\bigskip}\pstart{}Herrn Hugo v Hofmannsthal,\pend{}\pstart{}\textcolor{pink}{Rodaun}{}\ledrightnote{\textcolor{pink}{Badgasse}}\pend{}\pstart{}bei \textcolor{pink}{Wien-Liesing}{}\ledrightnote{\textcolor{pink}{Rodaun}}\pend{}{\bigskip}\pstart
           \raggedleft{}{\pb}\textcolor{pink}{Wien}{}\ledrightnote{\textcolor{pink}{Wien}}, 26. 2. 927\pend
           \pstart
           mein lieber Hugo, ich danke Ihnen für Ihren Gruſs aus \textcolor{pink}{Girgenti}{}\ledrightnote{\textcolor{pink}{Agrigento}}.\pend
           \pstart
           Der treffliche Regisseur \textcolor{blue}{\uline{Schulbaur}}{}\ledrightnote{\textcolor{blue}{Heinz Schulbaur}}, früher \textcolor{pink}{Volkstheater}{}\ledrightnote{\textcolor{pink}{Volkstheater}} wendet sich an mich:
                    ich möchte seine Bitte bei Ihnen unterstützen. Er will in der \textcolor{pink}{Akademie}{}\ledrightnote{\textcolor{pink}{Hochschule und Akademie für Musik und Darstellende Kunst}} mit seinen Schülern den \textcolor{green}{weißen Fächer}{}\ledrightnote{\textcolor{green}{Der weiße Fächer. Ein Zwischenspiel}} aufführen. Sie werden wohl nichts dagegen
                    haben, so wenig ich mich gegen dergleichen zu wehren pflege.\pend
           \pstart
           Auf Wiedersehen nach Ihrer Rückkehr\hspace*{1em}Ich wünsche
                    Ihnen weiterhin schöne \textcolor{pink}{Sicilianer}{}\ledrightnote{\textcolor{pink}{Sizilien}} Tage. Ich war
                        1904 in \label{K_L02482_1v}\edtext{\textcolor{pink}{Taormina}{}\ledrightnote{\textcolor{pink}{Taormina}}}{\lemma{\textnormal{\emph{Taormina}}}\Cendnote{\textnormal{vgl. A. S.: \emph{Tagebuch}, 19. 5. 1904}}}\label{K_L02482_1h} u \label{K_L02482_2v}\edtext{\textcolor{pink}{Syrakus}{}\ledrightnote{\textcolor{pink}{Syrakus}}}{\lemma{\textnormal{\emph{Syrakus}}}\Cendnote{\textnormal{vgl. A. S.: \emph{Tagebuch}, 17. 5. 1904}}}\label{K_L02482_2h}.\pend
           \pstart Herzlichst Ihr \spacefill\mbox{Arthur}\pend{}\endnumbering\briefempfaengerindex{Hofmannsthal, Hugo von@\textsc{Hofmannsthal, Hugo von}!zzzSchnitzler, Arthur@\emph{von Arthur Schnitzler}!1927-02-261@{26. 2. 1927}|)be}\mylabel{h}  \normalsize

\doendnotes{C}
\bigskip
\vfill

\clearpage

\footnotesize

\lohead{\textsc{register}}

% Definiere theindex-Environment komplett neu ohne reledmac
\makeatletter
\renewenvironment{theindex}{%
  \section*{\indexname}%
  \setlength{\parindent}{0pt}%
  \setlength{\parskip}{0pt plus 0.3pt}%
  \let\item\@idxitem
}{%
  \clearpage
}
\makeatother

\IfFileExists{\jobname-pw.ind}{\input{\jobname-pw.ind}}{}

\end{document}

      