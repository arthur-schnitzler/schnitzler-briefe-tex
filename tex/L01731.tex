%% latex-korrekturansicht-vorspann.tex
%% Vorspann für die Korrekturansicht.
%% Lädt die gemeinsame Datei latex-vorspann.tex mit gesetztem Schalter.

\newif\ifkorrekturansicht
\korrekturansichttrue

\input{../tex-inputs/latex-vorspann}


               \section[Arthur Schnitzler an Hugo von Hofmannsthal, 15. 11. {[}1907{]}]{ Arthur Schnitzler an Hugo von Hofmannsthal, 15. 11. {[}1907{]}}\nopagebreak\mylabel{v}\rehead{ }\normalsize\beginnumbering\briefempfaengerindex{Hofmannsthal, Hugo von@\textsc{Hofmannsthal, Hugo von}!zzzSchnitzler, Arthur@\emph{von Arthur Schnitzler}!1907-11-151@{15. 11. {[}1907{]}}|(be} \toendnotes[C]{\smallbreak\pagebreak[2]} \Standort{FDH, Hs-30885,130.}
\physDesc{Brief, 1 Blatt, 2 Seiten
\newline{}Handschrift: schwarze Tinte, lateinische Kurrent\newline{}Ordnung: 1) von Schnitzler mutmaßlich bei der Durchsicht der Briefe
                                 1929 mit
                           Bleistift datiert: »912?« 2) mit rotem Buntstift von unbekannter
                           Hand die letzte Ziffer der ergänzten Jahresangabe zu »0« korrigiert}\buchAbdrucke{\weitereDrucke{Hugo von Hofmannsthal, Arthur Schnitzler: \emph{Briefwechsel}. Hg. Therese Nickl und Heinrich Schnizler. Frankfurt am Main: \emph{S. Fischer} 1964, S. 234.} }\toendnotes[C]{\smallbreak}\pstart
           {\pb}\textcolor{gray}{\textbf{Dr. Arthur Schnitzler}}\hfill 15.
                     11.\pend
           \pstart
           \textcolor{gray}{\textbf{\textcolor{pink}{Wien XVIII.
                        Spoettelgasse 7}{}\ledrightnote{\textcolor{pink}{Edmund-Weiß-Gasse}}.}}\pend
           \pstart
           liebster Hugo, wir dürfen also annehmen, dass \substVorne{}\textsuperscript{Sie}\substDazwischen{}Ihr\substHinten{} am
                  Montag ko{\geminationm}\substVorne{}\textsuperscript{en}\substDazwischen{}t\substHinten{}. Wollen Sie Ihren \textcolor{blue}{Papa}{}\ledrightnote{→\textcolor{blue}{Hugo August von Hofmannsthal}} mitbringen? Sie wissen wie
               wir uns freuen, ihn bei uns zu sehen. Aber auch wie gern wir mit Euch allein sind
               wissen Sie. Also möcht ichs ganz Ihnen überlassen, ob wir {\pb}Ihren \textcolor{blue}{Papa}{}\ledrightnote{→\textcolor{blue}{Hugo August von Hofmannsthal}} auch zu uns
               bitten. We{\geminationn}{ }\uline{ja}, theilen Sie mirs (mit seiner Adresse) rasch auf einer
               Karte mit. Auch vielleicht, ob Ihnen \textcolor{blue}{Skopf}{}\ledrightnote{\textcolor{blue}{Gustav Schwarzkopf}}
               angenehm wäre.\pend
           \pstart
           Herzlichst{\\[\baselineskip]}Ihr\spacefill\mbox{A.}\pend
           \leftskip=0em{}\endnumbering\briefempfaengerindex{Hofmannsthal, Hugo von@\textsc{Hofmannsthal, Hugo von}!zzzSchnitzler, Arthur@\emph{von Arthur Schnitzler}!1907-11-151@{15. 11. {[}1907{]}}|)be}\mylabel{h}  \normalsize

\doendnotes{C}
\bigskip
\vfill

\clearpage

\footnotesize

\lohead{\textsc{register}}

% Definiere theindex-Environment komplett neu ohne reledmac
\makeatletter
\renewenvironment{theindex}{%
  \section*{\indexname}%
  \setlength{\parindent}{0pt}%
  \setlength{\parskip}{0pt plus 0.3pt}%
  \let\item\@idxitem
}{%
  \clearpage
}
\makeatother

\IfFileExists{\jobname-pw.ind}{\input{\jobname-pw.ind}}{}

\end{document}

      