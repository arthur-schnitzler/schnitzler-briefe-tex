%% latex-korrekturansicht-vorspann.tex
%% Vorspann für die Korrekturansicht.
%% Lädt die gemeinsame Datei latex-vorspann.tex mit gesetztem Schalter.

\newif\ifkorrekturansicht
\korrekturansichttrue

\input{../tex-inputs/latex-vorspann}


               \section[Hugo von Hofmannsthal an Arthur Schnitzler, {[}30. 11. 1898{]}]{ Hugo von Hofmannsthal an Arthur Schnitzler, {[}30. 11. 1898{]}}\nopagebreak\mylabel{v}\rehead{ }\normalsize\beginnumbering\briefempfaengerindex{Schnitzler, Arthur@\textsc{Schnitzler, Arthur}!zzzHofmannsthal, Hugo von@\emph{von Hugo von Hofmannsthal}!1898-11-302@{{[}30. 11. 1898{]}}|(be} \toendnotes[C]{\smallbreak\pagebreak[2]} \Standort{CUL, Schnitzler, B 43.}
\physDesc{Brief, 1 Blatt, 2 Seiten
\newline{}Handschrift: schwarze Tinte, deutsche Kurrent
\newline{}Schnitzler: mit Bleistift datiert: »30/11 98« \newline{}Ordnung: 1) mit Bleistift von unbekannter Hand nummeriert: »127« 2) mit Bleistift von unbekannter Hand nummeriert:
                                        »130«}\buchAbdrucke{\weitereDrucke{Hugo von Hofmannsthal, Arthur Schnitzler: \emph{Briefwechsel}. Hg. Therese Nickl und Heinrich Schnitzler. Frankfurt am Main: \emph{S. Fischer} 1964, S. 114.} }\toendnotes[C]{\smallbreak}\pstart{}{\pb}\textsc{Herrn D\textsuperscript{r} Arthur
                            Schnitzler}\pend{}\pstart{}\textcolor{pink}{\textsc{Frankgasse} 1}{}\ledrightnote{\textcolor{pink}{Frankgasse}}\pend{}{\bigskip}\pstart{}{\pb}lieber Arthur\pend\pstart
           an der \label{K_L00862_1v}\edtext{Caſſa}{\lemma{\textnormal{\emph{Caſſa}}}\Cendnote{\textnormal{Am Abend fand die Uraufführung von \emph{\textcolor{green}{Das Vermächtnis}}
                   statt. Vgl. A. S.: \emph{Tagebuch}, 30. 11. 1898.}}}\label{K_L00862_1h} beko{\geminationm}en die Leute die Auskunft, daſs die Sitze und
                    Logen durch Sie direct zu beziehen ſind, alſo was ſoll machen!\pend
           \pstart \spacefill\mbox{Hugo}\pend{}\pstart
           \noindent{}Es handelt ſich um die Loge »\textcolor{blue}{Frankenſtein}{}\ledrightnote{\textcolor{blue}{Frankenstein}}.«\pend
           \endnumbering\briefempfaengerindex{Schnitzler, Arthur@\textsc{Schnitzler, Arthur}!zzzHofmannsthal, Hugo von@\emph{von Hugo von Hofmannsthal}!1898-11-302@{{[}30. 11. 1898{]}}|)be}\mylabel{h}  \normalsize

\doendnotes{C}
\bigskip
\vfill

\clearpage

\footnotesize

\lohead{\textsc{register}}

% Definiere theindex-Environment komplett neu ohne reledmac
\makeatletter
\renewenvironment{theindex}{%
  \section*{\indexname}%
  \setlength{\parindent}{0pt}%
  \setlength{\parskip}{0pt plus 0.3pt}%
  \let\item\@idxitem
}{%
  \clearpage
}
\makeatother

\IfFileExists{\jobname-pw.ind}{\input{\jobname-pw.ind}}{}

\end{document}

      