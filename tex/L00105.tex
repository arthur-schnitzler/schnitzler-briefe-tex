%% latex-korrekturansicht-vorspann.tex
%% Vorspann für die Korrekturansicht.
%% Lädt die gemeinsame Datei latex-vorspann.tex mit gesetztem Schalter.

\newif\ifkorrekturansicht
\korrekturansichttrue

\input{../tex-inputs/latex-vorspann}


               \section[Hugo von Hofmannsthal an Arthur Schnitzler, 19. 7. {[}1892{]}]{ Hugo von Hofmannsthal an Arthur Schnitzler, 19. 7. {[}1892{]}}\nopagebreak\mylabel{v}\rehead{ }\normalsize\beginnumbering\briefempfaengerindex{Schnitzler, Arthur@\textsc{Schnitzler, Arthur}!zzzHofmannsthal, Hugo von@\emph{von Hugo von Hofmannsthal}!1892-07-191@{19. 7. {[}1892{]}}|(be} \toendnotes[C]{\smallbreak\pagebreak[2]} \Standort{CUL, Schnitzler, B 43.}
\physDesc{Brief, 1 Blatt (Briefpapier mit aufgeprägtem Wappen), 4 Seiten
\newline{}Handschrift: schwarze Tinte, deutsche Kurrent
\newline{}Schnitzler: mit Bleistift die Jahreszahl ergänzt: »92« \newline{}Ordnung: mit Bleistift von unbekannter Hand nummeriert:
                                    »26« }\buchAbdrucke{\weitereDrucke{1) Hugo von Hofmannsthal: \emph{Briefe an Freunde.} In: \emph{Die neue Rundschau}, Jg. 41, Nr. 4, April 1930, S. 512–513.} \weitereDrucke{2) Hugo von Hofmannsthal: \emph{Briefe. 1890–1901}. Berlin: \emph{S. Fischer} 1935, S. 48–50.} \weitereDrucke{3) Hugo von Hofmannsthal, Arthur Schnitzler: \emph{Briefwechsel}. Hg. Therese Nickl und Heinrich Schnitzler. Frankfurt am Main: \emph{S. Fischer} 1964, S. 23–24.} \weitereDrucke{4) Hermann Bahr, Arthur Schnitzler: \emph{Briefwechsel, Aufzeichnungen, Dokumente (1891–1931)}. Hg. Kurt Ifkovits und Martin Anton Müller. Göttingen: \emph{Wallstein} 2018, S. 25.} }\toendnotes[C]{\smallbreak}\pstart
           \raggedleft{}{\pb}\textcolor{pink}{Fuſch}{}\ledrightnote{\textcolor{pink}{Bad Fusch}}. 19. Juli.\pend
           \pstart{}lieber Arthur,\pend\pstart
           an Ihrem guten und lieben Brief ſtört mich nur die Nachricht, wie viel Arbeit Sie
               sich jetzt zumuthen wollen. Deshalb wünſche ich für Sie ſoſehr den äußeren Erfolg,
               den Sie als Künſtler vor ſich ſelbſt und vor uns gewiſs nicht nothwendig haben, damit
               ſich die Perſpectiven, in denen Sie ſelbſt und Ihr \textcolor{blue}{Vater}{}\ledrightnote{→\textcolor{blue}{Johann Schnitzler}} Ihr äußeres Leben, Ziele, Pflichten\strikeout{,} und Stil der Lebensführung, anſchauen, endlich
               ändern. Vorläufig iſt es ja ſehr gut, daſs Sie nachts ſchaffen und ſo reich und
               lebhaft aufnehmen können, wie Ihre \textcolor{blue}{Hebbel}{}\ledrightnote{\textcolor{blue}{Friedrich Hebbel}}eindrücke
               dies zeigen. Gewiſs iſt \textcolor{blue}{Hebbel}{}\ledrightnote{\textcolor{blue}{Friedrich Hebbel}} ein ſehr großer,
               tiefer und reicher Geiſt, mit den innerlichſten und eindringendſten {\pb}Anſchauungen vom Weſen der
               Naturdinge und des Menſchen, aufwühlend und anregend wie keiner ſonſt, sodaſs ſich
               einem die geheimſten, ſonſt erſtarrten inneren Tiefen regen und das eigentlich
               Dämoniſche in uns, das naturverwandte, dumpf und berauſchend mittönt. Eine
               Überſchrift bei \textcolor{blue}{Goethe}{}\ledrightnote{\textcolor{blue}{Johann Wolfgang von Goethe}} irgendwo: »\textcolor{green}{Urworte; orphiſch}{}\ledrightnote{\textcolor{green}{Urworte. Orphisch}}« ſuggeriert mir immer den Duft der Poeſie \textcolor{blue}{Hebbel}{}\ledrightnote{\textcolor{blue}{Friedrich Hebbel}}s.\pend
           \pstart
           \textcolor{blue}{Papa}{}\ledrightnote{→\textcolor{blue}{Hugo August von Hofmannsthal}} iſt befriedigend wohl und
               grüßt Sie, \textcolor{blue}{Bahr}{}\ledrightnote{\textcolor{blue}{Hermann Bahr}} und \textcolor{blue}{Salten}{}\ledrightnote{\textcolor{blue}{Felix Salten}}.\pend
           \pstart
           Ich habe mich vor einer gewiſſen inneren Öde und Abſpannung in die \textcolor{green}{Tragödie}{}\ledrightnote{→\textcolor{green}{Ascanio und Gioconda}} gerettet; eine 5 actige \textcolor{green}{Renaiſſancetragödie}{}\ledrightnote{→\textcolor{green}{Ascanio und Gioconda}}, dramatiſierte
               Novelle, äußerlich im Stil von \textcolor{green}{Romeo u. Julie}{}\ledrightnote{\textcolor{green}{Romeo und Julia}}, für
               die wirkliche brutale Bühne gearbeitet, mit {\pb}großem, ſchlankem Aufbau und
               grellen Farbenflecken, Freskotechnik; ich hoffe vorläufig noch genug lebendige
               Pſychologie in mir zu haben, um das große Gerippe mit lebendigem Fleiſch zu
               umkleiden; ich arbeite ohne Scenarium, mit einzelnen, ſuggeſtiven Notizen;
               geſchrieben habe ich bis jetzt ein paar Scenen aus dem 2\textsuperscript{ten} und eine aus dem 5\textsuperscript{ten} Act; das iſt zwar
               nicht viel aber ich ſehe alles andere recht deutlich und arbeite leicht. Was mich
               lockt und worauf ich eigentlich innerlich hinarbeite, iſt die eigenthümlich
               dunkelglühende, dionyſiſche Luſt im Erfinden und Ausführen tragiſcher Menſchen in
               tragiſchen Situationen; dieſe Luſt, deren ſymboliſches Aequivalent etwa das Anhören
                  {\pb}feierlicher,
               prunkvoll-trauriger Muſik iſt oder das Anſchauen mancher Bilder der \textsc{Renaissance}, mit dunkelgoldnen Panzern und blaſſen ſchönen
               Profilen auf ſehr finſterem Grund. Es wäre ſehr schön, wenn Octobernachmittage
               würden, mit dieſen zwei Leſepremièren. Wie weit iſt die \textcolor{green}{Familie}{}\ledrightnote{\textcolor{green}{Familie}}? \hspace*{2em}\textsc{\textcolor{blue}{Richard}{}\ledrightnote{\textcolor{blue}{Richard Beer-Hofmann}}}{ }ſchreibt mir, ungern und nur weil er von \textcolor{blue}{Papas}{}\ledrightnote{→\textcolor{blue}{Hugo August von Hofmannsthal}} Krankheit gehört hat; er
               ist verſtimmt, arbeitet aber doch an einer ſeiner \textcolor{green}{Novellen}{}\ledrightnote{→\textcolor{green}{Das Kind}}. Wann iſt Ihre Waffenübung? was ist es mit der
               Verlagsanſtalt für \textcolor{green}{Anatol}{}\ledrightnote{\textcolor{green}{Anatol}}? laſſen Sie ſich doch ja
               nicht durch ganz gleichgiltige Miſserfolge vom Weiterſuchen abſchrecken. Bitte,
               ſchreiben Sie mir bald, Briefe bekommen iſt hier das luſtigſte.\pend
           \pstart \spacefill\mbox{Loris.}\pend{}\endnumbering\briefempfaengerindex{Schnitzler, Arthur@\textsc{Schnitzler, Arthur}!zzzHofmannsthal, Hugo von@\emph{von Hugo von Hofmannsthal}!1892-07-191@{19. 7. {[}1892{]}}|)be}\mylabel{h}  \normalsize

\doendnotes{C}
\bigskip
\vfill

\clearpage

\footnotesize

\lohead{\textsc{register}}

% Definiere theindex-Environment komplett neu ohne reledmac
\makeatletter
\renewenvironment{theindex}{%
  \section*{\indexname}%
  \setlength{\parindent}{0pt}%
  \setlength{\parskip}{0pt plus 0.3pt}%
  \let\item\@idxitem
}{%
  \clearpage
}
\makeatother

\IfFileExists{\jobname-pw.ind}{\input{\jobname-pw.ind}}{}

\end{document}

      