%% latex-korrekturansicht-vorspann.tex
%% Vorspann für die Korrekturansicht.
%% Lädt die gemeinsame Datei latex-vorspann.tex mit gesetztem Schalter.

\newif\ifkorrekturansicht
\korrekturansichttrue

\input{../tex-inputs/latex-vorspann}


               \section[Arthur Schnitzler an Hugo von Hofmannsthal, {[}15. 5. 1894?{]}]{ Arthur Schnitzler an Hugo von Hofmannsthal, {[}15. 5. 1894?{]}}\nopagebreak\mylabel{v}\rehead{ }\normalsize\beginnumbering\briefempfaengerindex{Hofmannsthal, Hugo von@\textsc{Hofmannsthal, Hugo von}!zzzSchnitzler, Arthur@\emph{von Arthur Schnitzler}!1894-05-152@{{[}15. 5. 1894?{]}}|(be} \toendnotes[C]{\smallbreak\pagebreak[2]} \Standort{FDH, Hs-30885,31.}
\physDesc{Briefkarte
\newline{}Handschrift: schwarze Tinte, deutsche Kurrent}\buchAbdrucke{\weitereDrucke{Hugo von Hofmannsthal, Arthur Schnitzler: \emph{Briefwechsel}. Hg. Therese Nickl und Heinrich Schnitzler. Frankfurt am Main: \emph{S. Fischer} 1964, S. 32.} }\toendnotes[C]{\smallbreak}\pstart
           \noindent{}{\pb}Lieber Hugo! \textcolor{blue}{Fels}{}\ledrightnote{\textcolor{blue}{Friedrich Michael Fels}} hat ſich wieder gemeldet. Können Sie im Lauf
                  \label{K_L00324_1v}\edtext{dieſes Monats}{\lemma{\textnormal{\emph{dieſes Monats}}}\Cendnote{\textnormal{Die Einordnung des undatierten Stückes ist
                  schwierig. Der Februar 1893, in dem die Hilfe für \textcolor{blue}{Fels} zentral in der Korrespondenz ist, scheint sich durch
                  die Mitteilung der Wohnadresse in der \textcolor{pink}{Exnerstraße} auszuschließen, da \textcolor{blue}{Hofmannsthal} am 9. 2. 1893 explizit nach der Adresse fragt,
                  dieses Korrespondenzstück aber nicht die Antwort darauf ist. Hingegen können der
                  Brief \textcolor{blue}{Schnitzler}s an \textcolor{blue}{Beer-Hofmann} vom 15. 5. 1894 – in dem er um
                  Hilfe für \textcolor{blue}{Fels} bittet und dessen Adresse
                  mitteilt, als Hinweis genommen werden, dass auch dieses Korrespondenzstück an
                  diesem Tag verfasst ist.}}}\label{K_L00324_1h} noch was thun, ſo wäre es ihm, ja auch mir recht
               angenehm. Er wohnt, für alle Fälle ſei es Ihnen mitgetheilt, \textsc{\textcolor{pink}{XVIII. Exnerstraße 3}{}\ledrightnote{\textcolor{pink}{Krütznergasse}}}. Es ſcheint wirklich, dß er vom nächſten Monat {\pb}an
               nicht auf uns mehr angewieſen ſein wird.\pend
           \pstart
           Herzliche Grüße.{\\[\baselineskip]}Ihr \spacefill\mbox{Arthur}\pend
           \leftskip=0em{}\endnumbering\briefempfaengerindex{Hofmannsthal, Hugo von@\textsc{Hofmannsthal, Hugo von}!zzzSchnitzler, Arthur@\emph{von Arthur Schnitzler}!1894-05-152@{{[}15. 5. 1894?{]}}|)be}\mylabel{h}  \normalsize

\doendnotes{C}
\bigskip
\vfill

\clearpage

\footnotesize

\lohead{\textsc{register}}

% Definiere theindex-Environment komplett neu ohne reledmac
\makeatletter
\renewenvironment{theindex}{%
  \section*{\indexname}%
  \setlength{\parindent}{0pt}%
  \setlength{\parskip}{0pt plus 0.3pt}%
  \let\item\@idxitem
}{%
  \clearpage
}
\makeatother

\IfFileExists{\jobname-pw.ind}{\input{\jobname-pw.ind}}{}

\end{document}

      