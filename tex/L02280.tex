%% latex-korrekturansicht-vorspann.tex
%% Vorspann für die Korrekturansicht.
%% Lädt die gemeinsame Datei latex-vorspann.tex mit gesetztem Schalter.

\newif\ifkorrekturansicht
\korrekturansichttrue

\input{../tex-inputs/latex-vorspann}


               \section[Hugo von Hofmannsthal an Arthur Schnitzler, 13. 11. {[}1917{]}]{ Hugo von Hofmannsthal an Arthur Schnitzler, 13. 11. {[}1917{]}}\nopagebreak\mylabel{v}\rehead{ }\normalsize\beginnumbering\briefempfaengerindex{Schnitzler, Arthur@\textsc{Schnitzler, Arthur}!zzzHofmannsthal, Hugo von@\emph{von Hugo von Hofmannsthal}!1917-11-131@{13. 11. {[}1917{]}}|(be} \toendnotes[C]{\smallbreak\pagebreak[2]} \Standort{CUL, Schnitzler, B 43.}
\physDesc{Kartenbrief
\newline{}Handschrift: schwarze Tinte, deutsche Kurrent\newline{}Versand: Stempel: »\nobreak{}\oindex{Rodaun@\textbf{Rodaun}, \emph{Teil eines besiedelten Ortes (A.BSOX)}|pwk}Rodaun, 13. 1{[}1. 1917{]}, 2 N\nobreak{}«.  
\newline{}Schnitzler: 1) mit Bleistift beschriftet: »\textsc{Hugo}«, datiert »18?« 2) mit rotem Buntstift eine Unterstreichung\newline{}Ordnung: 1) mit Bleistift von \textcolor{blue}{Frieda Pollak} (?) mit dem Buchstaben »A« (Abgeschrieben/Abschrift) gekennzeichnet 2) mit Bleistift von unbekannter Hand
                                    nummeriert: »390«}\buchAbdrucke{\weitereDrucke{Hugo von Hofmannsthal, Arthur Schnitzler: \emph{Briefwechsel}. Hg. Therese Nickl und Heinrich Schnitzler. Frankfurt am Main: \emph{S. Fischer} 1964, S. 282.} }\toendnotes[C]{\smallbreak}\pstart{}{\pb}\textsc{Herrn D\textsuperscript{r} Arthur
                            Schnitzler}\pend{}\pstart{}\textsc{\textcolor{pink}{Wien}{}\ledrightnote{\textcolor{pink}{Wien}}}\pend{}\pstart{}\textsc{\textcolor{pink}{XVIII}{}\ledrightnote{\textcolor{pink}{XVIII., Währing}}}\pend{}\pstart{}\textsc{\textcolor{pink}{Sternwartestrasse 71}{}\ledrightnote{\textcolor{pink}{Sternwartestraße}}}\pend{}{\bigskip}\pstart
           \raggedleft{}{\pb}\textcolor{pink}{\textsc{R}}{}\ledrightnote{\textcolor{pink}{Rodaun}}{ }\label{K_L02280_1v}\edtext{12 XI}{\lemma{\textnormal{\emph{12 XI}}}\Cendnote{\textnormal{Hier ist ein Irrtum des
                            Verfassers anzunehmen. Sowohl der Poststempel als auch der Verweis auf
                            die »morgige« Uraufführung verweisen auf den
                                13. 11. 1917 als Tag der Abfassung.}}}\label{K_L02280_1h}\pend
           \pstart
           mein lieber Arthur\hspace*{1.5em}der \label{K_L02280_2v}\edtext{dritte}{\lemma{\textnormal{\emph{dritte}}}\Cendnote{\textnormal{Der erste erschien 1907, der zweite
                            1914, der dritte Ende November 1917.}}}\label{K_L02280_2h}
                    Band meiner \textcolor{green}{Proſaarbeiten}{}\ledrightnote{\textcolor{green}{Die prosaischen Schriften}} wird in dieſen Tagen
                    durch \textcolor{blue}{Fiſcher}{}\ledrightnote{\textcolor{blue}{Samuel Fischer}} an Sie geſchickt werden, bitte
                    nehmen Sie ihn wie aus meiner Hand, ich habe den Auftrag gegeben, diesmal direct
                    zu ſchicken, weil man ja weder Papier noch Spagat mehr hat, um von Haus Bücher
                    zu verſenden. Und ſo iſt man ſchließlich auch voneinander abgeſchnitten, durch
                    die Einſchränkung der Verkehrsmittel u. die Unmöglichkeit, eine Abendmahlzeit
                    herzuſtellen.\pend
           \pstart
           Wenn ich \label{K_L02280_3v}\edtext{aus \textcolor{pink}{Deutſchland}{}\ledrightnote{\textcolor{pink}{Deutschland}} zurückkomme}{\lemma{\textnormal{\emph{aus … zurückkomme}}}\Cendnote{\textnormal{Die Reise
                        dauerte vom 20. 11. 1917 bis zum
                            8. 12. 1917.}}}\label{K_L02280_3h}, Mitte December,
                    ſo hoffe ich daſs Sie u. \textcolor{blue}{Olga}{}\ledrightnote{\textcolor{blue}{Olga Schnitzler}} einmal gegen
                    Abend in unſere kleine \textcolor{pink}{Stadtwohnung}{}\ledrightnote{→\textcolor{pink}{Stallburggasse}} ko{\geminationm}en werden. Indeſſen freue
                    ich mich auf \label{K_L02280_4v}\edtext{morgen Abend}{\lemma{\textnormal{\emph{morgen Abend}}}\Cendnote{\textnormal{Uraufführung von \emph{\textcolor{green}{Fink und Fliederbusch}} am 14. 11. 1917 im \textcolor{pink}{Deutschen Volkstheater}.}}}\label{K_L02280_4h}, und werde für das
                    Ernſte u. für den Spaß in Ihrer \textcolor{green}{Comödie}{}\ledrightnote{→\textcolor{green}{Fink und Fliederbusch. Komödie in drei Akten}} ein guter Zuhörer ſein.\pend
           \pstart Herzlich Ihr\spacefill\mbox{Hugo}\pend{}\endnumbering\briefempfaengerindex{Schnitzler, Arthur@\textsc{Schnitzler, Arthur}!zzzHofmannsthal, Hugo von@\emph{von Hugo von Hofmannsthal}!1917-11-131@{13. 11. {[}1917{]}}|)be}\mylabel{h}  \normalsize

\doendnotes{C}
\bigskip
\vfill

\clearpage

\footnotesize

\lohead{\textsc{register}}

% Definiere theindex-Environment komplett neu ohne reledmac
\makeatletter
\renewenvironment{theindex}{%
  \section*{\indexname}%
  \setlength{\parindent}{0pt}%
  \setlength{\parskip}{0pt plus 0.3pt}%
  \let\item\@idxitem
}{%
  \clearpage
}
\makeatother

\IfFileExists{\jobname-pw.ind}{\input{\jobname-pw.ind}}{}

\end{document}

      