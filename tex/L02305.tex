%% latex-korrekturansicht-vorspann.tex
%% Vorspann für die Korrekturansicht.
%% Lädt die gemeinsame Datei latex-vorspann.tex mit gesetztem Schalter.

\newif\ifkorrekturansicht
\korrekturansichttrue

\input{../tex-inputs/latex-vorspann}


               \section[Erhard Buschbeck an Arthur Schnitzler, 24. 9. 1918]{ Erhard Buschbeck an Arthur Schnitzler, 24. 9. 1918}\nopagebreak\mylabel{v}\rehead{ }\normalsize\beginnumbering\briefempfaengerindex{Schnitzler, Arthur@\textsc{Schnitzler, Arthur}!zzzBuschbeck, Erhard@\emph{von Erhard Buschbeck}!1918-09-241@{24. 9. 1918}|(be} \toendnotes[C]{\smallbreak\pagebreak[2]} \Standort{CUL, Schnitzler, B 5b.}
\physDesc{Brief, 1 Blatt, 1 Seite
\newline{}Handschrift: schwarze Tinte, lateinische Kurrent
\newline{}Schnitzler: 1) mit Bleistift ergänzt: »Bahr.« und Vermerk »\textsc{A}«, vermutlich für »Abzuschreiben«/»Abschrift« 2) mit rotem Buntstift eine Unterstreichung\newline{}Ordnung: mit Bleistift von unbekannter Hand nummeriert:
                                    »183« }\buchAbdrucke{\weitereDrucke{Hermann Bahr, Arthur Schnitzler: \emph{Briefwechsel, Aufzeichnungen, Dokumente (1891–1931)}. Hg. Kurt Ifkovits und Martin Anton Müller. Göttingen: \emph{Wallstein} 2018, S. 520.} }\toendnotes[C]{\smallbreak}\pstart
           \noindent{}{\pb}\textcolor{gray}{\textbf{\textit{\label{T_L02305-1v}\edtext{\textcolor{brown}{k. k. Hofburgtheater}{}\ledrightnote{\textcolor{brown}{Burgtheater}}}{\lemma{\textnormal{\emph{k. k. Hofburgtheater}}}\Cendnote{\textnormal{Prägestempel}}}\label{T_L02305-1h}}}}\hfill \textcolor{pink}{Wien}{}\ledrightnote{\textcolor{pink}{Wien}}, 24. Sept. 1918.\pend
           \pstart
           \textcolor{gray}{\textbf{Direction}}\pend
           \pstart{}Sehr geehrter Herr Doktor,\pend\pstart
           \textcolor{blue}{Hermann Bahr}{}\ledrightnote{\textcolor{blue}{Hermann Bahr}} hat mich gebeten, Ihnen zu sagen,
               daß ein Beschluss vorliegt, die Generalproben vorläufig nicht mehr öffentlich
               abzuhalten und nur die Vertreter der \textcolor{pink}{Wiener}{}\ledrightnote{\textcolor{pink}{Wien}}
               Tagespresse und Mitglieder des Hauses einzulassen. Es ist ihm sehr schmerzlich, daß
               er infolge der Verreisung des General-Intendanten und Major \textcolor{blue}{Michels}{}\ledrightnote{\textcolor{blue}{Robert Michel}} bis zu diesem Freitag eine Ausnahme für Sie,
               hochgeehrter Herr Doktor, wird nicht mehr erreichen können. \textcolor{blue}{Bahr}{}\ledrightnote{\textcolor{blue}{Hermann Bahr}} glaubt aber sicher, daß das für die kommenden Male nach
               einer Intervention bei Exc. \textcolor{blue}{Andrian}{}\ledrightnote{\textcolor{blue}{Leopold von Andrian-Werburg}} ohne weiteres
               wird geschehen können. Daß es ganz seinen Wünschen entspricht und es ihm natürlich
               sehr lieb \introOben{}und wertvoll\introOben{} wäre, Arthur Schnitzler dabei zu
               wissen, soll ich Ihnen, sehr geehrter Herr Dr., noch ganz besonders sagen.\pend
           \pstart
           In größter Hochachtung{\\[\baselineskip]}ergebenst{\\[\baselineskip]}\spacefill\mbox{ErhardBuschbeck}\pend
           \leftskip=0em{}\endnumbering\briefempfaengerindex{Schnitzler, Arthur@\textsc{Schnitzler, Arthur}!zzzBuschbeck, Erhard@\emph{von Erhard Buschbeck}!1918-09-241@{24. 9. 1918}|)be}\mylabel{h}  \normalsize

\doendnotes{C}
\bigskip
\vfill

\clearpage

\footnotesize

\lohead{\textsc{register}}

% Definiere theindex-Environment komplett neu ohne reledmac
\makeatletter
\renewenvironment{theindex}{%
  \section*{\indexname}%
  \setlength{\parindent}{0pt}%
  \setlength{\parskip}{0pt plus 0.3pt}%
  \let\item\@idxitem
}{%
  \clearpage
}
\makeatother

\IfFileExists{\jobname-pw.ind}{\input{\jobname-pw.ind}}{}

\end{document}

      