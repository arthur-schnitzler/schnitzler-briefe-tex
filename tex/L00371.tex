%% latex-korrekturansicht-vorspann.tex
%% Vorspann für die Korrekturansicht.
%% Lädt die gemeinsame Datei latex-vorspann.tex mit gesetztem Schalter.

\newif\ifkorrekturansicht
\korrekturansichttrue

\input{../tex-inputs/latex-vorspann}


               \section[Hugo von Hofmannsthal an Arthur Schnitzler, {[}20. 9. 1894{]}]{ Hugo von Hofmannsthal an Arthur Schnitzler, {[}20. 9. 1894{]}}\nopagebreak\mylabel{v}\rehead{ }\normalsize\beginnumbering\briefempfaengerindex{Schnitzler, Arthur@\textsc{Schnitzler, Arthur}!zzzHofmannsthal, Hugo von@\emph{von Hugo von Hofmannsthal}!1894-09-201@{{[}20. 9. 1894{]}}|(be} \toendnotes[C]{\smallbreak\pagebreak[2]} \Standort{CUL, Schnitzler, B 43.}
\physDesc{Briefkarte (aufgeprägtes Wappen, floraler Jugendstil-Karton)
\newline{}Handschrift: schwarze Tinte, deutsche Kurrent
\newline{}Schnitzler: mit Bleistift datiert: »20/9 94« und nummeriert: »67« }\buchAbdrucke{\weitereDrucke{Hugo von Hofmannsthal, Arthur Schnitzler: \emph{Briefwechsel}. Hg. Therese Nickl und Heinrich Schnitzler. Frankfurt am Main: \emph{S. Fischer} 1964, S. 52.} }\toendnotes[C]{\smallbreak}\pstart{}{\pb}lieber,\pend\pstart
           \textcolor{green}{Sterben}{}\ledrightnote{\textcolor{green}{Sterben. Novelle}}. \uline{Abſolut} keine Punkte. Beſſer Novelle als Erzählung, am beſten einfach
                    »von A. S.«\pend
           \pstart
           Bitte hat Ihnen \textcolor{blue}{Stern}{}\ledrightnote{\textcolor{blue}{Julius Stern}} wegen \label{K_L00371_1v}\edtext{\textcolor{green}{Generalprobe}{}\ledrightnote{→\textcolor{green}{Fürst Malachoff}}}{\lemma{\textnormal{\emph{Generalprobe}}}\Cendnote{\textnormal{Zumindest \textcolor{blue}{Schnitzler} besuchte die Uraufführung am
                            22. 9. 1894 im \textcolor{pink}{Carl-Theater}.}}}\label{K_L00371_1h} was ſagen laſſen? \pend
           \pstart \spacefill\mbox{Hugo.}\pend{}\endnumbering\briefempfaengerindex{Schnitzler, Arthur@\textsc{Schnitzler, Arthur}!zzzHofmannsthal, Hugo von@\emph{von Hugo von Hofmannsthal}!1894-09-201@{{[}20. 9. 1894{]}}|)be}\mylabel{h}  \normalsize

\doendnotes{C}
\bigskip
\vfill

\clearpage

\footnotesize

\lohead{\textsc{register}}

% Definiere theindex-Environment komplett neu ohne reledmac
\makeatletter
\renewenvironment{theindex}{%
  \section*{\indexname}%
  \setlength{\parindent}{0pt}%
  \setlength{\parskip}{0pt plus 0.3pt}%
  \let\item\@idxitem
}{%
  \clearpage
}
\makeatother

\IfFileExists{\jobname-pw.ind}{\input{\jobname-pw.ind}}{}

\end{document}

      