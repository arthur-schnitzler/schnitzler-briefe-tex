%% latex-korrekturansicht-vorspann.tex
%% Vorspann für die Korrekturansicht.
%% Lädt die gemeinsame Datei latex-vorspann.tex mit gesetztem Schalter.

\newif\ifkorrekturansicht
\korrekturansichttrue

\input{../tex-inputs/latex-vorspann}


               \section[Max Burckhard an Arthur Schnitzler, 5. 2. 1898]{ Max Burckhard an Arthur Schnitzler, 5. 2. 1898}\nopagebreak\mylabel{v}\rehead{ }\normalsize\beginnumbering\briefempfaengerindex{Schnitzler, Arthur@\textsc{Schnitzler, Arthur}!zzzBurckhard, Max Eugen@\emph{von Max Eugen Burckhard}!1898-02-051@{5. 2. 1898}|(be} \toendnotes[C]{\smallbreak\pagebreak[2]} \Standort{CUL, Schnitzler, B 20.}
\physDesc{Brief, 1 Blatt, 1 Seite
\newline{}Handschrift: schwarze Tinte, deutsche Kurrent\newline{}Ordnung: mit Bleistift von unbekannter Hand nummeriert:
                                    »12« }\toendnotes[C]{\smallbreak}\pstart
           \raggedleft{}{\pb}\textcolor{pink}{Wien}{}\ledrightnote{\textcolor{pink}{Wien}}{ }5. 2. 98.\pend
           \pstart{}Sehr verehrter lieber Herr Doctor!\pend\pstart
           Ich gratuliere Ihnen \uline{von Herzen} zu Ihrem geſtrigen
               ſchönen \label{K_L00773_1v}\edtext{Erfolg}{\lemma{\textnormal{\emph{Erfolg}}}\Cendnote{\textnormal{die \textcolor{pink}{Wien}er
                  Premiere von \emph{\textcolor{green}{Freiwild}} am \textcolor{pink}{Carl-Theater} am 4. 2. 1898.}}}\label{K_L00773_1h}, den mir die Morgenblätter melden. Adieu ſage
               ich \uline{Ihnen} nicht, denn wir bleiben ja doch gute
               Nachbarn und ich darf ja auch ſagen \uline{gute Freunde}.
               Habe ich einmal ein biſſel Luft, ſo bin ich ſo frei zu Ihnen hinabzukommen und Ihnen
               auch noch mündlich zu sagen, wie herzlich mich Ihre Anweſenheit am \label{K_L00773_2v}\edtext{Mittwoch}{\lemma{\textnormal{\emph{Mittwoch}}}\Cendnote{\textnormal{beim Bankett zu Ehren \textcolor{blue}{Burckhards}, das als Reaktion auf dessen Ablösung als Direktor des \textcolor{pink}{Burgtheater}s, am 2. 2. 1898 stattfand.}}}\label{K_L00773_2h} gefreut hat. Ihr
               Sie aufrichtig verehrender\pend
           \pstart
           \spacefill\mbox{D\textsuperscript{r}Burckhard}{\\[\baselineskip]}Herzlichſte Grüße!\pend
           \leftskip=0em{}\endnumbering\briefempfaengerindex{Schnitzler, Arthur@\textsc{Schnitzler, Arthur}!zzzBurckhard, Max Eugen@\emph{von Max Eugen Burckhard}!1898-02-051@{5. 2. 1898}|)be}\mylabel{h}  \normalsize

\doendnotes{C}
\bigskip
\vfill

\clearpage

\footnotesize

\lohead{\textsc{register}}

% Definiere theindex-Environment komplett neu ohne reledmac
\makeatletter
\renewenvironment{theindex}{%
  \section*{\indexname}%
  \setlength{\parindent}{0pt}%
  \setlength{\parskip}{0pt plus 0.3pt}%
  \let\item\@idxitem
}{%
  \clearpage
}
\makeatother

\IfFileExists{\jobname-pw.ind}{\input{\jobname-pw.ind}}{}

\end{document}

      