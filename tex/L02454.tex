%% latex-korrekturansicht-vorspann.tex
%% Vorspann für die Korrekturansicht.
%% Lädt die gemeinsame Datei latex-vorspann.tex mit gesetztem Schalter.

\newif\ifkorrekturansicht
\korrekturansichttrue

\input{../tex-inputs/latex-vorspann}


               \section[Hugo Hofmannsthal an Arthur Schnitzler, 14. 11. 1925]{ Hugo Hofmannsthal an Arthur Schnitzler, 14. 11. 1925}\nopagebreak\mylabel{v}\rehead{ }\normalsize\beginnumbering\briefempfaengerindex{Schnitzler, Arthur@\textsc{Schnitzler, Arthur}!zzzHofmannsthal, Hugo von@\emph{von Hugo von Hofmannsthal}!1925-11-141@{14. 11. 1925}|(be} \toendnotes[C]{\smallbreak\pagebreak[2]} \Standort{CUL, Schnitzler, B 43.}
\physDesc{Brief, 1 Blatt, 1 Seite
\newline{}Handschrift: schwarze Tinte, lateinische Kurrent
\newline{}Schnitzler: 1) mit Bleistift beschriftet: »\textsc{Hugo}« 2) mit rotem Buntstift mehrere Unterstreichungen\newline{}Ordnung: 1) mit Bleistift von unbekannter Hand nummeriert: »\strikeout{369}« 2) mit Bleistift von unbekannter Hand nummeriert: »378«}\buchAbdrucke{\weitereDrucke{Hugo von Hofmannsthal, Arthur Schnitzler: \emph{Briefwechsel}. Hg. Therese Nickl und Heinrich Schnitzler. Frankfurt am Main: \emph{S. Fischer} 1964, S. 302.} }\toendnotes[C]{\smallbreak}\pstart
           {\pb}\textcolor{pink}{Bad Aussee}{}\ledrightnote{\textcolor{pink}{Bad Aussee}}{ }14 XI 25. \pend
           \pstart{}lieber Arthur\pend\pstart
           eben ko{\geminationm}t ein kleines \textcolor{green}{Buch}{}\ledrightnote{→\textcolor{green}{Die Frau des Richters. Novelle}}: eine Erzählung von Ihrer Hand, und ich freue mich
               äußerst darauf, sie abends zu lesen: ein Vorgefühl (genährt durch Hineinschauen) sagt
               mir, dass sie an meine besonderen Lieblinge: »\textcolor{green}{Leisenbohg}{}\ledrightnote{\textcolor{green}{Das Schicksal des Freiherrn von Leisenbohg. Novellette}}« und »\textcolor{green}{Cassian}{}\ledrightnote{\textcolor{green}{Der tapfere Cassian. Puppenspiel in einem Akt}}«, \label{T_L02454_1v}\edtext{angrenzt}{\lemma{\textnormal{\emph{angrenzt}}}\Cendnote{\textnormal{Er schreibt »angränzt«.}}}\label{T_L02454_1h}.\pend
           \pstart
           Arthur, aber haben Sie in \textcolor{pink}{Berlin}{}\ledrightnote{\textcolor{pink}{Berlin}} den »\textcolor{green}{Turm}{}\ledrightnote{\textcolor{green}{Der Turm. Ein Trauerspiel}}« beko{\geminationm}en?\hspace*{1.5em}Fast ko{\geminationm}t mir der Gedanke,
               dass \uline{nicht}. Und diese Exemplare einer (vorläufigen)
               mehr nur Luxusausgabe sind wenige, es täte mir leid, we{\geminationn} eines verloren wäre.\hspace*{1.5em}Würden Sie eventuell ans \textcolor{pink}{Esplanade}{}\ledrightnote{\textcolor{pink}{Hotel Esplanade}} ein reclamierendes Wort schreiben? Mir liegt viel daran, diese
               Arbeit endlich in Ihren Händen zu wissen!\hspace*{1.5em}– Ich bin,
               in großer Stille, sehr anhaltend fleissig.\pend
           \pstart Ihr\spacefill\mbox{Hugo.}\pend{}\endnumbering\briefempfaengerindex{Schnitzler, Arthur@\textsc{Schnitzler, Arthur}!zzzHofmannsthal, Hugo von@\emph{von Hugo von Hofmannsthal}!1925-11-141@{14. 11. 1925}|)be}\mylabel{h}  \normalsize

\doendnotes{C}
\bigskip
\vfill

\clearpage

\footnotesize

\lohead{\textsc{register}}

% Definiere theindex-Environment komplett neu ohne reledmac
\makeatletter
\renewenvironment{theindex}{%
  \section*{\indexname}%
  \setlength{\parindent}{0pt}%
  \setlength{\parskip}{0pt plus 0.3pt}%
  \let\item\@idxitem
}{%
  \clearpage
}
\makeatother

\IfFileExists{\jobname-pw.ind}{\input{\jobname-pw.ind}}{}

\end{document}

      