%% latex-korrekturansicht-vorspann.tex
%% Vorspann für die Korrekturansicht.
%% Lädt die gemeinsame Datei latex-vorspann.tex mit gesetztem Schalter.

\newif\ifkorrekturansicht
\korrekturansichttrue

\input{../tex-inputs/latex-vorspann}


               \section[Laura Marholm an Arthur Schnitzler, 24. 4. 1895]{ Laura Marholm an Arthur Schnitzler, 24. 4. 1895}\nopagebreak\mylabel{v}\rehead{ }\normalsize\beginnumbering\briefempfaengerindex{Schnitzler, Arthur@\textsc{Schnitzler, Arthur}!zzzMarholm, Laura@\emph{von Laura Marholm}!1895-04-241@{24. 4. 1895}|(be} \toendnotes[C]{\smallbreak\pagebreak[2]} \Standort{TMW, HS Schn 3/65/1.}
\physDesc{Brief, 1 Blatt, 3 Seiten
\newline{}Handschrift: schwarze Tinte, lateinische Kurrent}\toendnotes[C]{\smallbreak}\pstart
           \raggedleft{}{\pb}\textcolor{pink}{Schliersee}{}\ledrightnote{\textcolor{pink}{Schliersee}}, \textcolor{pink}{Oberbaiern}{}\ledrightnote{\textcolor{pink}{Oberbayern}},{\\}24 April 95.\pend
           \pstart{}Sehr geehrter Herr Doctor.\pend\pstart
           Wie ich Ihren Brief aufmachte, las ich erst: »mein \textcolor{blue}{Vater}{}\ledrightnote{→\textcolor{blue}{Johann Schnitzler}} ist schon zwei Tage lang todt« und erschrak, –
                    Sie hätten um ein Haar einen Condolenzbrief bekommen; da las ich ihn noch
                    einmal, weil mir soviel Gutes drin gesagt wurde, was ich im Einzelnen auf seine
                    Richtigkeit durchgehen wollte, – das, was Sie über die Hauptlinie sagen, machte
                    mir eine besondere Freude, denn das meine ich selbst ist im Guten und Üblem der
                    Punkt auf dem meine Anlage fußt. Nur beim zweiten Lesen sehe ich, daß es 2 Jahre
                    sind und mir wurde ganz flau{\dots} sie haben mir so
                    grundernsthaft geschrieben, Sie hätten auch ein bischen lachen können. Jetzt
                    glaube ich, Sie thun es heimlich.\pend
           \pstart
           Natürlich bitte ich Sie, das häßliche \textcolor{green}{Buch}{}\ledrightnote{→\textcolor{green}{Das Buch der Frauen}} zu behalten, im Austausch von »\textcolor{green}{Sterben}{}\ledrightnote{\textcolor{green}{Sterben. Novelle}}« {\pb}das ich von Ihnen
                    erhielt. Ich schrieb Ihnen damals über das Buch nichts – – wenn ich Ihnen den
                    Grund sage, werden Sie es verstehen. \textcolor{blue}{Ola}{}\ledrightnote{\textcolor{blue}{Ola Hansson}} las
                    es und fand es sehr gut und fein.\pend
           \pstart
           Aber ich konnte es nicht leiden – aus einem ganz subjectiven Grund {\dots} ich konnte mich damals keine Nacht zu Bett legen,
                    ohne daß das kam, wovon das ganze Buch handelt. Sobald ich das Licht auslöschte
                    und es ganz schwarz war, kam regelmäßig dies furchtbare Grauen vor dem Aufhören,
                    nicht dem Sterben, aber dem Nichtmehrsein und nicht blos dem persönlichen
                    Nichtmehrsein, sondern dem von meinen Liebsten, von dieser Weltkugel{\dotsfour} Ich betrachtete es gar nicht als etwas
                    Krankhaftes, nur als einen Ausschlag von Vitalitätsgefühl, aber in der tiefen
                        \textcolor{pink}{Schliersee}{}\ledrightnote{\textcolor{pink}{Schliersee}}r Einsamkeit, die mein \textcolor{blue}{Mann}{}\ledrightnote{→\textcolor{blue}{Ola Hansson}} liebt, war es bei mir,
                    Tag und Nacht, immer, und steigerte sich jedesmal beim Einschlafen zu einem
                    unsagbaren Angstgefühl. Darum mochte ich Ihr \textcolor{green}{Buch}{}\ledrightnote{→\textcolor{green}{Sterben. Novelle}} nicht, das ganz auf dieser einen Note gespielt
                    wird, es potenzierte mein Eigenes zu stark{\dotsfour}\pend
           \pstart
           Jetzt ist es vorbei. Und an einem sehr schönen, duftenden, schwirrenden Tage will
                    ich »\textcolor{green}{Sterben}{}\ledrightnote{\textcolor{green}{Sterben. Novelle}}« wieder lesen. Wenn ich fühle,
                    daß {\pb}ich es
                    kann.\pend
           \pstart
           Sie sind der einzige von allen Jungen, von dem ich etwas ganz Besonderes erwarten
                    könnte, – dagegen bin ich nicht sicher, daß es Sie dauernd interessiren wird zu
                    schreiben. Produciren ist doch auch nur eine Art von Stimulanz-Genuß {\dots} aber wieviele Stoffe können Naturen wie Sie
                    stimuliren? Da Sie doch viel zu durchgebildet und von zu guter Herkunft sind als
                    daß die äusserlichen Eitelkeits- und Erfolgsrücksichten viel für Sie bedeuten
                    könnten.\pend
           \pstart
           Aber Ihr nächstes Buch schicken Sie mir wieder? nicht wahr?\pend
           \pstart
           Mit verbindlichem Gruß{\\[\baselineskip]} Ihre ergebene{\\[\baselineskip]}\spacefill\mbox{Laura Hansson-Marholm}\pend
           \leftskip=0em{}\endnumbering\briefempfaengerindex{Schnitzler, Arthur@\textsc{Schnitzler, Arthur}!zzzMarholm, Laura@\emph{von Laura Marholm}!1895-04-241@{24. 4. 1895}|)be}\mylabel{h}  \normalsize

\doendnotes{C}
\bigskip
\vfill

\clearpage

\footnotesize

\lohead{\textsc{register}}

% Definiere theindex-Environment komplett neu ohne reledmac
\makeatletter
\renewenvironment{theindex}{%
  \section*{\indexname}%
  \setlength{\parindent}{0pt}%
  \setlength{\parskip}{0pt plus 0.3pt}%
  \let\item\@idxitem
}{%
  \clearpage
}
\makeatother

\IfFileExists{\jobname-pw.ind}{\input{\jobname-pw.ind}}{}

\end{document}

      