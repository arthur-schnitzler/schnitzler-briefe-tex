%% latex-korrekturansicht-vorspann.tex
%% Vorspann für die Korrekturansicht.
%% Lädt die gemeinsame Datei latex-vorspann.tex mit gesetztem Schalter.

\newif\ifkorrekturansicht
\korrekturansichttrue

\input{../tex-inputs/latex-vorspann}


               \section[Hugo von Hofmannsthal an Arthur Schnitzler, 19. 9. {[}1909{]}]{ Hugo von Hofmannsthal an Arthur Schnitzler, 19. 9. {[}1909{]}}\nopagebreak\mylabel{v}\rehead{ }\normalsize\beginnumbering\briefempfaengerindex{Schnitzler, Arthur@\textsc{Schnitzler, Arthur}!zzzHofmannsthal, Hugo von@\emph{von Hugo von Hofmannsthal}!1909-09-191@{19. 9. {[}1909{]}}|(be} \toendnotes[C]{\smallbreak\pagebreak[2]} \Standort{CUL, Schnitzler, B 43.}
\physDesc{Brief, 1 Blatt, 4 Seiten
\newline{}Handschrift: schwarze Tinte, deutsche Kurrent
\newline{}Schnitzler: mit Bleistift datiert: »19/X 909.« und beschriftet: »\textsc{Hofma{\geminationn}sthal}« \newline{}Ordnung: 1) mit Bleistift von unbekannter Hand nummeriert: »309« 2) mit Bleistift von unbekannter Hand nummeriert: »307«}\buchAbdrucke{\weitereDrucke{Hugo von Hofmannsthal, Arthur Schnitzler: \emph{Briefwechsel}. Hg. Therese Nickl und Heinrich Schnitzler. Frankfurt am Main: \emph{S. Fischer} 1964, S. 246.} }\toendnotes[C]{\smallbreak}\pstart
           {\pb}19 IX.\hfill \textcolor{pink}{\textsc{Aussee Obertressen} 14}{}\ledrightnote{\textcolor{pink}{Obertressen}}.\pend
           \pstart{}mein guter lieber Arthur \pend\pstart
           ich freue mich von ganzem Herzen daſs Ihr ein zweites \textcolor{blue}{Kind}{}\ledrightnote{→\textcolor{blue}{Lili Schnitzler}} habt. Ich kann mir denken daſs Sie es ſich im Stillen
               gewünſcht haben, und es iſt zu nett von \textcolor{blue}{Olga}{}\ledrightnote{\textcolor{blue}{Olga Schnitzler}}, daſs
               Sie es Ihnen ſofort geſchenkt hat. Ja, ja, die {\pb}eigenen Frauen ſind doch etwas
               ſehr nettes und vielleicht noch netter als die Frauen der Andern, was meinen Sie, Sie
               geübter \textsc{roué, emeritierter \textcolor{green}{Anatol}{}\ledrightnote{→\textcolor{green}{Anatol}} etc}., Sie \textcolor{green}{Julian Fichtner}{}\ledrightnote{→\textcolor{green}{Der einsame Weg. Schauspiel in fünf Akten}}, \textcolor{green}{Waldemar von Sala}{}\ledrightnote{→\textcolor{green}{Der einsame Weg. Schauspiel in fünf Akten}} – nein der \textcolor{green}{Sala}{}\ledrightnote{→\textcolor{green}{Der einsame Weg. Schauspiel in fünf Akten}} bin ja ich!\pend
           \pstart
           Kurz, ich freue mich ſehr, daſs für \textcolor{blue}{\textsc{Heini}}{}\ledrightnote{\textcolor{blue}{Heinrich Schnitzler}} der \textcolor{green}{einſame Weg}{}\ledrightnote{→\textcolor{green}{Der einsame Weg. Schauspiel in fünf Akten}} nun zu Ende
               iſt und eine kleine {\pb}\textcolor{green}{Dämmerſeele}{}\ledrightnote{→\textcolor{green}{Die Fremde}} ihm Geſellſchaft
               leiſten wird, die ſich hoffentlich bald zu einer \textcolor{green}{griechiſchen Tänzerin}{}\ledrightnote{→\textcolor{green}{Die griechische Tänzerin. Novellette}} entwickelt.\pend
           \pstart
           Ich hab Sie ſehr lieb, mein lieber Arthur, und auch Ihre Arbeiten habe ich ſehr lieb,
               das gehört ja dazu. – Von dieſen allen hat mir aber die letzte: »\textcolor{green}{Brüderlein \textsc{Medardus} Hiergeiſt}{}\ledrightnote{\textcolor{green}{Der junge Medardus. Dramatische Historie in einem Vorspiel und fünf Aufzügen}}« den
               allerſchwächſten Eindruck gemacht, ſowohl die Geſtalten als die Fabel. {\pb}Kommt das vielleicht daher, weil
               ich beides nicht kenne?\pend
           \pstart
           Ich habe eine \textcolor{green}{Spieloper}{}\ledrightnote{→\textcolor{green}{Der Rosenkavalier}} gemacht,
               die glaub ich hübſch iſt. (Nicht ſo hübſch wie der \textcolor{green}{tapfere Caſſian}{}\ledrightnote{\textcolor{green}{Der tapfere Cassian. Puppenspiel in einem Akt}}) Und ferner bilde ich mir in den letzten Tagen ſtark ein
               daſs ich meine (äußerſt ſehr veränderte) \textcolor{green}{Florindocomödie}{}\ledrightnote{→\textcolor{green}{Cristinas Heimreise. Komödie}} in den nächſten Wochen fertig kriegen
               werde. Ich werde mich zu dieſem Zweck etwas iſolieren, vielleicht in \textcolor{pink}{München}{}\ledrightnote{\textcolor{pink}{München}} oder ſo.\hspace*{1.5em}Auf ein gutes
               Wiederſehen und vieles \uline{ſehr herzliche} an \textcolor{blue}{Olga}{}\ledrightnote{\textcolor{blue}{Olga Schnitzler}}.\pend
           \pstart Ihr \spacefill\mbox{Arthur}\pend{}\endnumbering\briefempfaengerindex{Schnitzler, Arthur@\textsc{Schnitzler, Arthur}!zzzHofmannsthal, Hugo von@\emph{von Hugo von Hofmannsthal}!1909-09-191@{19. 9. {[}1909{]}}|)be}\mylabel{h}  \normalsize

\doendnotes{C}
\bigskip
\vfill

\clearpage

\footnotesize

\lohead{\textsc{register}}

% Definiere theindex-Environment komplett neu ohne reledmac
\makeatletter
\renewenvironment{theindex}{%
  \section*{\indexname}%
  \setlength{\parindent}{0pt}%
  \setlength{\parskip}{0pt plus 0.3pt}%
  \let\item\@idxitem
}{%
  \clearpage
}
\makeatother

\IfFileExists{\jobname-pw.ind}{\input{\jobname-pw.ind}}{}

\end{document}

      