%% latex-korrekturansicht-vorspann.tex
%% Vorspann für die Korrekturansicht.
%% Lädt die gemeinsame Datei latex-vorspann.tex mit gesetztem Schalter.

\newif\ifkorrekturansicht
\korrekturansichttrue

\input{../tex-inputs/latex-vorspann}


               \section[Arthur Schnitzler an Richard Dehmel, {[}14. 11. 1908?{]}]{ Arthur Schnitzler an Richard Dehmel,
                    {[}14. 11. 1908?{]}}\nopagebreak\mylabel{v}\rehead{ }\normalsize\beginnumbering\briefempfaengerindex{Dehmel, Richard@\textsc{Dehmel, Richard}!zzzSchnitzler, Arthur@\emph{von Arthur Schnitzler}!1908-11-142@{{[}14. 11. 1908?{]}}|(be} \toendnotes[C]{\smallbreak\pagebreak[2]} \Standort{Hamburg, Staats- und Universitätsbibliothek, DA:Br:S:619.}
\physDesc{Brief, 1 Blatt, 2 Seiten
\newline{}Handschrift: schwarze Tinte, lateinische Kurrent\newline{}Ordnung: mit grüner Tinte von unbekannter Hand datiert: »Nov. 1908« }\pstart
           \noindent{}{\pb}\textcolor{gray}{\textbf{Dr. Arthur Schnitzler}}\hfill \uline{Samstag}\pend
           \pstart
           \textcolor{gray}{\textbf{\textcolor{pink}{Wien XVIII. Spoettelgasse 7}{}\ledrightnote{\textcolor{pink}{Edmund-Weiß-Gasse}}.}}\pend
           \pstart{}Verehrter Herr Dehmel,\pend\pstart
           schon bevor \strikeout{ich} Ihre Karte kam, hatte ich \textcolor{blue}{Wassermann}{}\ledrightnote{\textcolor{blue}{Jakob Wassermann}} geschrieben, er möge Sie bitten,
                    morgen Abend mit ihm zu uns zu kommen. Es wäre uns sehr unlieb, auf dieses
                    Vergnügen verzichten \introOben{}zu\introOben{} müssen – ich schlage Ihnen
                    daher vor, ko{\geminationm}en Sie recht früh, sagen wir
                        6, ½ 7 zu uns – und sagen {\pb}Sie uns auf einer Rohrpostkarte, um wie viel Uhr Sie das Nachtmahl bereit
                    wünschen.\pend
           \pstart
           Mit herzlichen Grüßen{\\[\baselineskip]}Ihr{\\[\baselineskip]}\spacefill\mbox{Arthur Schnitzler}\pend
           \leftskip=0em{}\endnumbering\briefempfaengerindex{Dehmel, Richard@\textsc{Dehmel, Richard}!zzzSchnitzler, Arthur@\emph{von Arthur Schnitzler}!1908-11-142@{{[}14. 11. 1908?{]}}|)be}\mylabel{h}  \normalsize

\doendnotes{C}
\bigskip
\vfill

\clearpage

\footnotesize

\lohead{\textsc{register}}

% Definiere theindex-Environment komplett neu ohne reledmac
\makeatletter
\renewenvironment{theindex}{%
  \section*{\indexname}%
  \setlength{\parindent}{0pt}%
  \setlength{\parskip}{0pt plus 0.3pt}%
  \let\item\@idxitem
}{%
  \clearpage
}
\makeatother

\IfFileExists{\jobname-pw.ind}{\input{\jobname-pw.ind}}{}

\end{document}

      