%% latex-korrekturansicht-vorspann.tex
%% Vorspann für die Korrekturansicht.
%% Lädt die gemeinsame Datei latex-vorspann.tex mit gesetztem Schalter.

\newif\ifkorrekturansicht
\korrekturansichttrue

\input{../tex-inputs/latex-vorspann}


               \section[Hugo von Hofmannsthal und Richard Beer-Hofmann an Arthur Schnitzler, {[}15.? 2. 1903{]}]{ Hugo von Hofmannsthal und Richard Beer-Hofmann an Arthur Schnitzler,
               {[}15.? 2. 1903{]}}\nopagebreak\mylabel{v}\rehead{ }\normalsize\beginnumbering\briefempfaengerindex{Schnitzler, Arthur@\textsc{Schnitzler, Arthur}!zzzBeer-Hofmann, Richard@\emph{von Richard Beer-Hofmann}!1903-02-151@{{[}15. 2. 1903{]}}|(be}\briefempfaengerindex{Schnitzler, Arthur@\textsc{Schnitzler, Arthur}!zzzHofmannsthal, Hugo von@\emph{von Hugo von Hofmannsthal}!1903-02-151@{{[}15. 2. 1903{]}}|(be} \toendnotes[C]{\smallbreak\pagebreak[2]} \Standort{CUL, Schnitzler, B 43.}
\physDesc{Brief, 1 Blatt, 4 Seiten
\newline{}Handschrift Richard Beer-Hofmann: schwarze Tinte, lateinische Kurrent\newline{}Handschrift Hugo von Hofmannsthal: schwarze Tinte, deutsche Kurrent\newline{}Ordnung: 1) mit Bleistift von unbekannter Hand datiert: »15/2 903.« 2) mit Bleistift von unbekannter Hand nummeriert: »\strikeout{213}«3) mit Bleistift von unbekannter Hand nummeriert: »194«}\buchAbdrucke{\weitereDrucke{1) Hugo von Hofmannsthal, Arthur Schnitzler: \emph{Briefwechsel}. Hg. Therese Nickl und Heinrich Schnitzler. Frankfurt am Main: \emph{S. Fischer} 1964, S. 167–168.} \weitereDrucke{2) Arthur Schnitzler, Richard Beer-Hofmann: \emph{Briefwechsel 1891–1931}. Hg. Konstanze Fliedl. Wien, Zürich: \emph{Europaverlag} 1992, S. 160–161.} }\toendnotes[C]{\smallbreak}\pstart{}{\pb}lieber Pornograph\pend\pstart
           wir denken es käme darauf an was für ein Verlag Ihr \textcolor{green}{Schmutzwerk}{}\ledrightnote{→\textcolor{green}{Reigen. Zehn Dialoge}} herausgibt. Iſt es etwa \textcolor{brown}{\textsc{Grimm}}{}\ledrightnote{\textcolor{brown}{Gustav Grimm Verlag}} in \textcolor{pink}{\textsc{Búda-Pest}}{}\ledrightnote{\textcolor{pink}{Budapest}}? Dazu würden wir nicht rathen. Iſt es aber ein ernſter Verlag, die Ausſtattung
               ſehr ernſthaft und anſtändig (Illuſtrationen \textsc{à la}{ }\textcolor{blue}{\textsc{Coschelle}}{}\ledrightnote{\textcolor{blue}{Moritz Coschell}} würden dieſe \label{K_L01269_1v}\edtext{\textsc{Cochonnerie}}{\lemma{\textnormal{\emph{Cochonnerie}}}\Cendnote{\textnormal{französisch: Ferkelei}}}\label{K_L01269_1h} zum Gelächter \textcolor{pink}{Europas}{}\ledrightnote{\textcolor{pink}{Europa}} machen) dann geht es immerhin. Denn
               ſchließlich {\pb}iſt es ja Ihr beſtes
                  \textcolor{green}{Buch}{}\ledrightnote{→\textcolor{green}{Reigen. Zehn Dialoge}}, Sie Schmutzfink. Weder
               iſt es ſo confus wie das \textcolor{green}{Vermächtnis}{}\ledrightnote{\textcolor{green}{Das Vermächtnis. Schauspiel in drei Akten}}, noch ſo glatt
               wie die \textcolor{green}{Liebelei}{}\ledrightnote{\textcolor{green}{Liebelei. Schauspiel in drei Akten}}, noch ſo \textsc{snobish} wie die \textcolor{green}{\textsc{Beatrice}}{}\ledrightnote{\textcolor{green}{Der Schleier der Beatrice. Schauspiel in fünf Akten}}, noch ſo unsäglich langweilig wie Ihre läppiſchen Novellen, kurz, natürlich
               ſollen Sie es herausgeben, unter dem \textsc{Pseudonym}{ }\textcolor{blue}{\textsc{Ludassy}}{}\ledrightnote{\textcolor{blue}{Julius von Gans-Ludassy}} oder auch unter Ihrem eigenen Namen. Aber in einer {\pb}anständigen Form. Das iſt unſere
               Anſicht.\pend
           \pstart
           {[}hs. Beer-Hofmann:{]} Sie müssen soviel Geld dafür beko{\geminationm}en (im \uuline{Vorhinein}, de{\geminationn} im Nachhinein wird es confiscirt) daß Sie Sich
               jedenfalls darüber mehr freuen, als Sie Sich später über das Schwätzen der Leute
               ärgern. Viele Leute werden es als Ihr erectiefstes Werk bezeichnen. Ob \uline{ich} es an Ihrer Stelle herausgeben würde weiß ich nicht;
               jedenfalls würde ich \uline{Sie} um Rath gefragt haben; geben
               Sie ihn mir also!\pend
           \pstart
           {[}hs. Hofmannsthal:{]} Ob ich es an Ihrer Stelle herausgegeben hätte? Unbedingt,
               gegen einen beträchtlichen Vorſchuſs und unter Ihrem Namen. (Der Vorſchuſs natürlich
               unter meinem Namen zahlbar.)\pend
           \pstart
           Verſtehen Sie alſo, was wir Ihnen gerathen haben?\pend
           \pstart
           {[}hs. Beer-Hofmann:{]} Ernstlich:\pend
           \settowidth{\longeste}{3) Ausstattung}\settowidth{\longestz}{entscheiden}\settowidth{\longestd}{}\settowidth{\longestv}{}\settowidth{\longestf}{}\addtolength\longeste{1em}
        \addtolength\longestz{1em}
      \pstart\noindent\makebox[\the\longeste][l]{1) Summe}\makebox[\the\longestz][l]{}
                  \pend\pstart\noindent\makebox[\the\longeste][l]{2.) Verlag}\makebox[\the\longestz][l]{entscheiden}
                  \pend\pstart\noindent\makebox[\the\longeste][l]{3) Ausstattung}\makebox[\the\longestz][l]{}
                  \pend\pstart
           1.) Sehr groß, 2.) Sehr ernst (die war’s nicht, der’s geschah) 3.) Würdig, d. h.
               Papier stark – wie Ihr Talent Format einfach, und eher groß, ja nicht Taschenformat
               oder zierlich.\pend
           \pstart
           {[}hs. Hofmannsthal:{]} Genug. \spacefill\mbox{Hugo}\pend
           \pstart
           {[}hs. Beer-Hofmann:{]} Ja! \spacefill\mbox{Richard}\pend
           \pstart
           \noindent{}\label{T_L01269_1v}\edtext{Dieser Brief kann als Vorrede
                  abgedruckt werden!}{\lemma{\textnormal{\emph{Dieser … werden!}}}\Cendnote{\textnormal{quer am linken Rand
                     der letzten Seite}}}\label{T_L01269_1h}\pend
           \endnumbering\briefempfaengerindex{Schnitzler, Arthur@\textsc{Schnitzler, Arthur}!zzzBeer-Hofmann, Richard@\emph{von Richard Beer-Hofmann}!1903-02-151@{{[}15. 2. 1903{]}}|)be}\briefempfaengerindex{Schnitzler, Arthur@\textsc{Schnitzler, Arthur}!zzzHofmannsthal, Hugo von@\emph{von Hugo von Hofmannsthal}!1903-02-151@{{[}15. 2. 1903{]}}|)be}\mylabel{h}  \normalsize

\doendnotes{C}
\bigskip
\vfill

\clearpage

\footnotesize

\lohead{\textsc{register}}

% Definiere theindex-Environment komplett neu ohne reledmac
\makeatletter
\renewenvironment{theindex}{%
  \section*{\indexname}%
  \setlength{\parindent}{0pt}%
  \setlength{\parskip}{0pt plus 0.3pt}%
  \let\item\@idxitem
}{%
  \clearpage
}
\makeatother

\IfFileExists{\jobname-pw.ind}{\input{\jobname-pw.ind}}{}

\end{document}

      