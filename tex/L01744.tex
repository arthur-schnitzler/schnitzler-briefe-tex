%% latex-korrekturansicht-vorspann.tex
%% Vorspann für die Korrekturansicht.
%% Lädt die gemeinsame Datei latex-vorspann.tex mit gesetztem Schalter.

\newif\ifkorrekturansicht
\korrekturansichttrue

\input{../tex-inputs/latex-vorspann}


               \section[Hermann Bahr an Arthur Schnitzler, 23. 12. 1907]{ Hermann Bahr an Arthur Schnitzler, 23. 12. 1907}\nopagebreak\mylabel{v}\rehead{ }\normalsize\beginnumbering\briefempfaengerindex{Schnitzler, Arthur@\textsc{Schnitzler, Arthur}!zzzBahr, Hermann@\emph{von Hermann Bahr}!1907-12-231@{23. 12. 1907}|(be} \toendnotes[C]{\smallbreak\pagebreak[2]} \Standort{CUL, Schnitzler, B 5b.}
\physDesc{Brief, 1 Blatt, 4 Seiten
\newline{}Handschrift: blaue Tinte, deutsche Kurrent
\newline{}Schnitzler: 1) mit Bleistift beschriftet: »Bahr« 2) mit rotem Buntstift vereinzelte Unterstreichungen\newline{}Ordnung: mit Bleistift von unbekannter Hand nummeriert:
                              »153« }\buchAbdrucke{\weitereDrucke{Hermann Bahr, Arthur Schnitzler: \emph{Briefwechsel, Aufzeichnungen, Dokumente (1891–1931)}. Hg. Kurt Ifkovits und Martin Anton Müller. Göttingen: \emph{Wallstein} 2018, S. 400.} }\toendnotes[C]{\smallbreak}\pstart
           \raggedleft{}{\pb}23. 12. 07\pend
           \pstart\center{}Lieber Arthur!\pend\pstart
           Danke ſchön für Deinen Brief. Ich möchte nicht, daß Du falſch deuteſt, was ich über
                  \textcolor{blue}{Reinhardts}{}\ledrightnote{\textcolor{blue}{Max Reinhardt}} Verhältnis zu Deinen Werken
               ſchrieb. Er bemüht ſich ſehr, ihnen gerecht zu ſein, aber ich habe immer das Gefühl,
               daß ihm das innere Verſtehen dafür fehlt; und es ist ſchon ſehr bös, wenn einer ſich
               erſt bemühen muß. Aber am guten Willen fehlts ihm ſicher nicht. Nur daß dieſer dabei
               leider ſchließlich gar nichts nützt. – Der \textcolor{blue}{Ritſcher}{}\ledrightnote{\textcolor{blue}{Helene Ritscher}} müßte geſagt werden, daß ſie Anfang Mai oder im September hier ſein
               ſoll. Die \textcolor{blue}{Mildenburg}{}\ledrightnote{\textcolor{blue}{Anna Bahr-Mildenburg}}{ }{\pb}hat eine merkwürdige Macht über ſie, ſodaß ſie
               nicht blos aus ihr heraus holen, ſondern ſogar bis zu einem gewiſſen Grad in ſie
               hinein pumpen kann. Ihr würde ich das Darſtelleriſche ganz überlaſſen, ohne ſelbſt
               dreinzureden; bei zweien kommt nichts heraus. Ich aber würde mit großer Paſſion den
                  \textcolor{blue}{Strakoſch}{}\ledrightnote{\textcolor{blue}{Alexander Strakosch}} machen und dem Mädel den Rhythmus
               der Verſe ein\substVorne{}\textsuperscript{t}\substDazwischen{}b\substHinten{}läuen, wovon ich aus Erfahrung weiß, daß ichs kann. Wenn es ſchließlich
               trotzdem ſcheußlich wird, können wir nichts {\pb}dafür.
                  \uline{Garantieren} könnte ich für die \textcolor{blue}{Höflich}{}\ledrightnote{\textcolor{blue}{Lucie Höflich}} ja auch nicht, die freilich einen vagen Schimmer von
               Seele oder Poeſie oder wie man das nennt für die Rolle hätte, den das Chaotiſche, das
               die \textcolor{blue}{Ritſcher}{}\ledrightnote{\textcolor{blue}{Helene Ritscher}}{ }ſehr ſtark hat, vielleicht nicht
               völlig erſetzen kann.\pend
           \pstart
           Ich ſelbſt habe vor Anſteckungen gar keine Furcht, muß aber auf meine \label{K_L01744_1v}\edtext{\textcolor{blue}{\textcolor{blue}{Frauen}{}\ledrightnote{→\textcolor{blue}{Rosa Bahr}{\newline}→\textcolor{blue}{Eugenie von Roth}}}{}\ledrightnote{→\textcolor{blue}{Anna Bahr-Mildenburg}}}{\lemma{\textnormal{\emph{Frauen}}}\Cendnote{\textnormal{Gemeint ist in jedem Fall seine Partnerin
                     \textcolor{blue}{Anna von Mildenburg}, eventuell mit ihrer
                  Gesellschafterin \textcolor{blue}{Eugenie Roth}. Vielleicht
                  inkludiert er auch seine erste Frau, \textcolor{blue}{Rosa}, mit
                  der er noch verheiratet war.}}}\label{K_L01744_1h} Rückſicht nehmen, hoffe jedoch, da ich
               früheſtens erſt am 15. Januar zu \textcolor{blue}{Reinhardt}{}\ledrightnote{\textcolor{blue}{Max Reinhardt}} zurückkehre, daß Deine {\pb}liebe
                  \textcolor{blue}{Frau}{}\ledrightnote{→\textcolor{blue}{Olga Schnitzler}}, der ich das Allerbeſte
               wünſche, \substVorne{}\textsuperscript{\textcolor{gray}{ſ}}\substDazwischen{}n\substHinten{}och vorher ſo weit \strikeout{ſ \textcolor{gray}{e\textcolor{gray}{×}}h\textcolor{gray}{×}}{ }ſein wird, daß ich zu Euch kann, was ich Dich bitte, mich gleich wiſſen zu
               laſſen.\pend
           \pstart
           Herzlichſt{\\[\baselineskip]}mit den wärmſten Weihnachtswünſchen{\\[\baselineskip]}Dein{\\[\baselineskip]}\spacefill\mbox{H}\pend
           \leftskip=0em{}\endnumbering\briefempfaengerindex{Schnitzler, Arthur@\textsc{Schnitzler, Arthur}!zzzBahr, Hermann@\emph{von Hermann Bahr}!1907-12-231@{23. 12. 1907}|)be}\mylabel{h}  \normalsize

\doendnotes{C}
\bigskip
\vfill

\clearpage

\footnotesize

\lohead{\textsc{register}}

% Definiere theindex-Environment komplett neu ohne reledmac
\makeatletter
\renewenvironment{theindex}{%
  \section*{\indexname}%
  \setlength{\parindent}{0pt}%
  \setlength{\parskip}{0pt plus 0.3pt}%
  \let\item\@idxitem
}{%
  \clearpage
}
\makeatother

\IfFileExists{\jobname-pw.ind}{\input{\jobname-pw.ind}}{}

\end{document}

      