%% latex-korrekturansicht-vorspann.tex
%% Vorspann für die Korrekturansicht.
%% Lädt die gemeinsame Datei latex-vorspann.tex mit gesetztem Schalter.

\newif\ifkorrekturansicht
\korrekturansichttrue

\input{../tex-inputs/latex-vorspann}


               \section[Arthur Schnitzler an Richard Beer-Hofmann, 4. 5. 1922]{ Arthur Schnitzler an Richard Beer-Hofmann, 4. 5. 1922}\nopagebreak\mylabel{v}\rehead{ }\normalsize\beginnumbering\briefempfaengerindex{Beer-Hofmann, Richard@\textsc{Beer-Hofmann, Richard}!zzzSchnitzler, Arthur@\emph{von Arthur Schnitzler}!1922-05-041@{4. 5. 1922}|(be} \toendnotes[C]{\smallbreak\pagebreak[2]} \Standort{CUL, Schnitzler, B 8.1, S. 156.}
\physDesc{maschinelle Abschrift
\newline{}Schreibmaschine\newline{}Ordnung: von unbekannter Hand als Briefnummer »352« gekennzeichnet }\buchAbdrucke{\weitereDrucke{Arthur Schnitzler, Richard Beer-Hofmann: \emph{Briefwechsel 1891–1931}. Hg. Konstanze Fliedl. Wien, Zürich: \emph{Europaverlag} 1992, S. 228.} }\toendnotes[C]{\smallbreak}\pstart
           \raggedleft{}{\pb}\textcolor{pink}{Haag}{}\ledrightnote{\textcolor{pink}{Den Haag}}, 4. 5. 1922.\pend
           \pstart
           Herzliche Grüsse! Ich habe hier angenehme bewegte Tage verbracht, in \textcolor{pink}{Rotterdam}{}\ledrightnote{\textcolor{pink}{Rotterdam}}, \textcolor{pink}{Haag}{}\ledrightnote{\textcolor{pink}{Den Haag}}, \textcolor{pink}{Amsterdam}{}\ledrightnote{\textcolor{pink}{Amsterdam}} mit Erfolg (in jeder Hinsicht) gelesen,
               fahre Samstag 6. aufs Land nach \textcolor{pink}{Osterbeck}{}\ledrightnote{\textcolor{pink}{Oosterbeek}} zu \textcolor{blue}{Brevées}{}\ledrightnote{\textcolor{blue}{Egbertje Alida Brevée}{\newline}\textcolor{blue}{Isaac Brevée}}, circa am
                  8. nach \textcolor{pink}{Berlin}{}\ledrightnote{\textcolor{pink}{Berlin}} weiter; dürfte
               zwischen 18. u. 20. in \textcolor{pink}{Wien}{}\ledrightnote{\textcolor{pink}{Wien}}
               sein. Grüssen Sie \textcolor{blue}{Paula}{}\ledrightnote{\textcolor{blue}{Paula Beer-Hofmann}} und die \textcolor{blue}{Kinder}{}\ledrightnote{→\textcolor{blue}{Naëmah Beer-Hofmann}{\newline}→\textcolor{blue}{Mirjam Beer-Hofmann}{\newline}→\textcolor{blue}{Gabriel Beer-Hofmann}}! Ihr \spacefill\mbox{A.}\pend
           \pstart
           \noindent{}(nach \textcolor{pink}{Wien}{}\ledrightnote{\textcolor{pink}{Wien}})\pend
           \endnumbering\briefempfaengerindex{Beer-Hofmann, Richard@\textsc{Beer-Hofmann, Richard}!zzzSchnitzler, Arthur@\emph{von Arthur Schnitzler}!1922-05-041@{4. 5. 1922}|)be}\mylabel{h}  \normalsize

\doendnotes{C}
\bigskip
\vfill

\clearpage

\footnotesize

\lohead{\textsc{register}}

% Definiere theindex-Environment komplett neu ohne reledmac
\makeatletter
\renewenvironment{theindex}{%
  \section*{\indexname}%
  \setlength{\parindent}{0pt}%
  \setlength{\parskip}{0pt plus 0.3pt}%
  \let\item\@idxitem
}{%
  \clearpage
}
\makeatother

\IfFileExists{\jobname-pw.ind}{\input{\jobname-pw.ind}}{}

\end{document}

      