%% latex-korrekturansicht-vorspann.tex
%% Vorspann für die Korrekturansicht.
%% Lädt die gemeinsame Datei latex-vorspann.tex mit gesetztem Schalter.

\newif\ifkorrekturansicht
\korrekturansichttrue

\input{../tex-inputs/latex-vorspann}


               \section[Richard Beer-Hofmann an Arthur Schnitzler, 19. 9. 1896]{ Richard Beer-Hofmann an Arthur Schnitzler, 19. 9. 1896}\nopagebreak\mylabel{v}\rehead{ }\normalsize\beginnumbering\briefempfaengerindex{Schnitzler, Arthur@\textsc{Schnitzler, Arthur}!zzzBeer-Hofmann, Richard@\emph{von Richard Beer-Hofmann}!1896-09-192@{19. 9. 1896}|(be} \toendnotes[C]{\smallbreak\pagebreak[2]} \Standort{CUL, Schnitzler, B 8.}
\physDesc{Brief, 3 Blätter, 9 Seiten
\newline{}Handschrift: 1) Bleistift, lateinische Kurrent (\noindent{}3. Blatt)\hspace{1em}2) blauer Buntstift, lateinische Kurrent (\noindent{}1.–2. Blatt)\hspace{1em}\newline{}Ordnung: mit Bleistift von unbekannter Hand nummeriert:
                                    »86« beziehungsweise
                                 »86a?« }\buchAbdrucke{\weitereDrucke{Arthur Schnitzler, Richard Beer-Hofmann: \emph{Briefwechsel 1891–1931}. Hg. Konstanze Fliedl. Wien, Zürich: \emph{Europaverlag} 1992, S. 97–98.} }\toendnotes[C]{\smallbreak}\pstart
           \centering{}{\pb}\textcolor{pink}{Baden}{}\ledrightnote{\textcolor{pink}{Baden bei Wien}}{ }19/IX 96\pend
           \pstart
           Lieber Arthur! Ich bin schon Mittwoch{ }Abends in \textcolor{pink}{Wien}{}\ledrightnote{\textcolor{pink}{Wien}} und möchte gerne den
                  Abend mit Ihnen beisa{\geminationm}en sein. Schreiben
               Sie mir ob Sie frei sind und wann Sie mich abholen möchten. Außerdem, bitte, nehmen
               Sie mir für Donnerstag (\textcolor{blue}{Dörmann}{}\ledrightnote{\textcolor{blue}{Felix Dörmann}}?)
               einen \textcolor{green}{Sitz}{}\ledrightnote{→\textcolor{green}{Sein Sohn}} (neben sich – oder {\pb}Ecke) ins \textcolor{pink}{Raimundtheater}{}\ledrightnote{\textcolor{pink}{Raimund-Theater}} – ja?\pend
           \pstart
           Schließlich dachte ich heute Nachmittag an »\textcolor{green}{Liebelei}{}\ledrightnote{\textcolor{green}{Liebelei. Schauspiel in drei Akten}}« und »\textcolor{green}{Freiwild}{}\ledrightnote{\textcolor{green}{Freiwild. Schauspiel in 3 Akten}}«. Sie machen das
               Leben – wissen Sie das \uuline{Leben} (nicht das Leben das »so
               ist {\pb}wie – –{[}«{]})
               sehr schwer. Duellirt man sich – wird man unfehlbar erschossen; Duellirt man sich
               nicht, – no da wird man doch erst recht erschossen – das ist schrecklich. Im übrigen
               könnten Sie nicht 6 Akte aus den zwei Stücken {\pb}machen? Nur i{\geminationm}er abwechselnd einen Akt von \textcolor{green}{Liebelei}{}\ledrightnote{\textcolor{green}{Liebelei. Schauspiel in drei Akten}} und \textcolor{green}{Freiwild}{}\ledrightnote{\textcolor{green}{Freiwild. Schauspiel in 3 Akten}} spielen
               lassen?\pend
           \pstart
           Der \textcolor{green}{Lobheimer}{}\ledrightnote{→\textcolor{green}{Liebelei. Schauspiel in drei Akten}} wird im I Akt nicht
               gefordert, sondern statt des \label{K_L00594_1v}\edtext{\textcolor{blue}{Mitterwurzer}{}\ledrightnote{\textcolor{blue}{Friedrich Mitterwurzer}}}{\lemma{\textnormal{\emph{Mitterwurzer}}}\Cendnote{\textnormal{Dieser hatte in der Uraufführung den
                  »Herrn«, den betrogenen Ehemann, gespielt.}}}\label{K_L00594_1h} ko{\geminationm}t ein Briefträger – der auch zweimal läutet, {\pb}mit einem Expressbrief – der \strikeout{\textcolor{blue}{Pau}{}\ledrightnote{\textcolor{blue}{Paul Goldmann}}}{ }\textcolor{green}{Fritz}{}\ledrightnote{→\textcolor{green}{Liebelei. Schauspiel in drei Akten}} soll aufs Land zu seinen
               Eltern. Im II Akt (I. Akt \substVorne{}\textsuperscript{\textcolor{green}{Liebelei}{}\ledrightnote{\textcolor{green}{Liebelei. Schauspiel in drei Akten}}}{\allowbreak}\substDazwischen{}\textcolor{green}{Freiwild}{}\ledrightnote{\textcolor{green}{Freiwild. Schauspiel in 3 Akten}}\substHinten{}) \substVorne{}\textsuperscript{wird er gefordert}{\allowbreak}\substDazwischen{}beleidigt er –\substHinten{}.\pend
           \pstart
           Im III Akt fährt er nach \textcolor{pink}{Wien}{}\ledrightnote{\textcolor{pink}{Wien}} Abschied nehmen (II Akt
                  \textcolor{green}{Liebelei}{}\ledrightnote{\textcolor{green}{Liebelei. Schauspiel in drei Akten}}).\pend
           \pstart
           Im IV Akt (II Akt {\pb}\textcolor{green}{Freiwild}{}\ledrightnote{\textcolor{green}{Freiwild. Schauspiel in 3 Akten}}) überlegt er sich die Sache. Im V Akt
               (III Akt \textcolor{green}{Freiwild}{}\ledrightnote{\textcolor{green}{Freiwild. Schauspiel in 3 Akten}}) wird er todtgeschossen –
               »Gruppe« sagt die \textcolor{blue}{Sandrock}{}\ledrightnote{\textcolor{blue}{Adele Sandrock}}. Im VI Akt (III Akt
                  \textcolor{green}{Liebelei}{}\ledrightnote{\textcolor{green}{Liebelei. Schauspiel in drei Akten}}) teilt mans {\pb}dem »süßen Mädel« mit. Sehr feine
               Verkettung: \textcolor{blue}{Sonnenthal}{}\ledrightnote{\textcolor{blue}{Adolf von Sonnenthal}} ist Geigenspieler am \textcolor{pink}{Josefstädtertheater}{}\ledrightnote{\textcolor{pink}{Theater in der Josefstadt}}! Die Schauspielerin ist an der \textcolor{pink}{Josefstadt}{}\ledrightnote{\textcolor{pink}{VIII., Josefstadt}}, im So{\geminationm}er im
                  Bade{\pb}ort – \textcolor{pink}{Ischl}{}\ledrightnote{\textcolor{pink}{Bad Ischl}} – Ha! Bitte schlagen Sie mich nicht todt.\pend
           \pstart
           Herzlichst{\\[\baselineskip]}\spacefill\mbox{Richard}\pend
           \leftskip=0em{}\pstart
           \noindent{}{\pb}Da ich sehe daß das Couvert
                  durchsichtig ist und das »Todtschlagen« die Polizei beunruhigen könnte so nehme
                  ich noch ein Couvert drüber.\pend
           \pstart
           \raggedleft{}R\pend
           \endnumbering\briefempfaengerindex{Schnitzler, Arthur@\textsc{Schnitzler, Arthur}!zzzBeer-Hofmann, Richard@\emph{von Richard Beer-Hofmann}!1896-09-192@{19. 9. 1896}|)be}\mylabel{h}  \normalsize

\doendnotes{C}
\bigskip
\vfill

\clearpage

\footnotesize

\lohead{\textsc{register}}

% Definiere theindex-Environment komplett neu ohne reledmac
\makeatletter
\renewenvironment{theindex}{%
  \section*{\indexname}%
  \setlength{\parindent}{0pt}%
  \setlength{\parskip}{0pt plus 0.3pt}%
  \let\item\@idxitem
}{%
  \clearpage
}
\makeatother

\IfFileExists{\jobname-pw.ind}{\input{\jobname-pw.ind}}{}

\end{document}

      