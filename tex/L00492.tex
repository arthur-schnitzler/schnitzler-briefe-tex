%% latex-korrekturansicht-vorspann.tex
%% Vorspann für die Korrekturansicht.
%% Lädt die gemeinsame Datei latex-vorspann.tex mit gesetztem Schalter.

\newif\ifkorrekturansicht
\korrekturansichttrue

\input{../tex-inputs/latex-vorspann}


               \section[Richard Beer-Hofmann an Arthur Schnitzler, 24. 9. 1895]{ Richard Beer-Hofmann an Arthur Schnitzler, 24. 9. 1895}\nopagebreak\mylabel{v}\rehead{ }\normalsize\beginnumbering\briefempfaengerindex{Schnitzler, Arthur@\textsc{Schnitzler, Arthur}!zzzBeer-Hofmann, Richard@\emph{von Richard Beer-Hofmann}!1895-09-241@{24. 9. 1895}|(be} \toendnotes[C]{\smallbreak\pagebreak[2]} \Standort{CUL, Schnitzler, B 8.}
\physDesc{Brief, 1 Blatt, 4 Seiten
\newline{}Handschrift: Bleistift, lateinische Kurrent
\newline{}Schnitzler: mit Bleistift nummeriert: »64« }\buchAbdrucke{\weitereDrucke{1) Arthur Schnitzler, Richard Beer-Hofmann: \emph{Briefwechsel 1891–1931}. Hg. Konstanze Fliedl. Wien, Zürich: \emph{Europaverlag} 1992, S. 84–85.} \weitereDrucke{2) Hermann Bahr, Arthur Schnitzler: \emph{Briefwechsel, Aufzeichnungen, Dokumente (1891–1931)}. Hg. Kurt Ifkovits und Martin Anton Müller. Göttingen: \emph{Wallstein} 2018.} }\toendnotes[C]{\smallbreak}\pstart
           \raggedleft{}{\pb}\textcolor{pink}{Gardone}{}\ledrightnote{\textcolor{pink}{Gardone Riviera}}, Dienstag 24/IX 95\pend
           \pstart
           Lieber Arthur! Soeben erhalte ich von \textcolor{pink}{Riva}{}\ledrightnote{\textcolor{pink}{Riva del Garda}} nachgesandt Ihren Brief vom 21/IX. \uline{\textcolor{blue}{Fels}{}\ledrightnote{\textcolor{blue}{Friedrich Michael Fels}} – \label{K_L00492_1v}\edtext{Hekuba}{\lemma{\textnormal{\emph{Hekuba}}}\Cendnote{\textnormal{sprichwörtlicher Ausruf, der »Ist mir gleichgültig« bedeutet}}}\label{K_L00492_1h}} senden Sie bitte für mich ebensoviel als Sie bereits gesandt haben. Wie zuwider
               müssen wir ihm sein! Später oder früher werden wir es auch merken.\pend
           \pstart
           Hier ist{[}’s{]} wunderschön; der See 20 Grad Wärme – und etwas zu
               heiß, wodurch mein Arbeiten wieder stockt.\pend
           \pstart
           {\pb}Das mit dem »Blaßwerden guter
               Stücke« hat auch mich immer sehr traurig gemacht.\pend
           \stanza{}»\textcolor{green}{Alles entführet die Zeit; die
                     flüchtigen Jahre verändern}{}\ledrightnote{→\textcolor{green}{Epigramme}}\newverse{}\textcolor{green}{Ganz allmählich Gestalt, Namen und
                     Glück und Natur.}{}\ledrightnote{→\textcolor{green}{Epigramme}}{[}«{]}\stanzaend{}\pstart
           Das ist aber nicht von mir sondern von \textcolor{blue}{Plato}{}\ledrightnote{\textcolor{blue}{Platon}}!
               Wirklich! \pend
           \pstart
           Schreiben Sie mir doch recht viel oder zumindest oft, Sie sehen wie pünktlich ich
               antworte. Sagen Sie, sind in \textcolor{pink}{Wien}{}\ledrightnote{\textcolor{pink}{Wien}} auch alle Frauen
               jetzt läufig (l-ä-u-f-i-g)? {\pb}Hier
                  \strikeout{au} oder viel mehr auf der Reise schien es so.
               Manchmal angenehm, manchmal komisch und manchmal widerlich.\pend
           \pstart
           Daß \textcolor{blue}{Burkhardt}{}\ledrightnote{\textcolor{blue}{Max Eugen Burckhard}} die »Enthüllung von Frl. \textcolor{blue}{Dandler}{}\ledrightnote{\textcolor{blue}{Anna Dandler}}« (\textcolor{pink}{München}{}\ledrightnote{\textcolor{pink}{München}}?) lieber wäre als \label{K_L00492_2v}\edtext{die
                  \textcolor{blue}{Laubes}{}\ledrightnote{\textcolor{blue}{Heinrich Laube}}}{\lemma{\textnormal{\emph{die
                  Laubes}}}\Cendnote{\textnormal{Am 18. 9. 1895 wurde im
                  Geburtsort \textcolor{blue}{Heinrich Laubes}, in \textcolor{pink}{Sprottau}, ein Denkmal für diesen eingeweiht.}}}\label{K_L00492_2h} begreife
               ich. Die \textcolor{blue}{Dandler}{}\ledrightnote{\textcolor{blue}{Anna Dandler}} ist übrigens {\pb}auch \textcolor{blue}{Bahr}{}\ledrightnote{\textcolor{blue}{Hermann Bahr}}s Geschmack, voraussichtlich auch der Doctor \textcolor{blue}{Lueger}{}\ledrightnote{\textcolor{blue}{Karl Lueger}}s. Das{[}s{]} die \textcolor{blue}{Kallina}{}\ledrightnote{\textcolor{blue}{Anna Kallina}} überraschen wird, freut mich, vielleicht überrascht sie
               auch mich; jedenfalls grüßen Sie sie von mir – sie hat wirklich schöne Augen.
               Übrigens ist sie Ihnen so sympathisch weil \textcolor{blue}{Bahr}{}\ledrightnote{\textcolor{blue}{Hermann Bahr}}
               sie gar nicht mag – was? Wann ist \textcolor{green}{Liebelei}{}\ledrightnote{\textcolor{green}{Liebelei. Schauspiel in drei Akten}}? Das muß
               ich nämlich genau wissen, wegen meiner Ankunft!\pend
           \pstart Herzlichst Ihr \spacefill\mbox{Richard}\pend{}\endnumbering\briefempfaengerindex{Schnitzler, Arthur@\textsc{Schnitzler, Arthur}!zzzBeer-Hofmann, Richard@\emph{von Richard Beer-Hofmann}!1895-09-241@{24. 9. 1895}|)be}\mylabel{h}  \normalsize

\doendnotes{C}
\bigskip
\vfill

\clearpage

\footnotesize

\lohead{\textsc{register}}

% Definiere theindex-Environment komplett neu ohne reledmac
\makeatletter
\renewenvironment{theindex}{%
  \section*{\indexname}%
  \setlength{\parindent}{0pt}%
  \setlength{\parskip}{0pt plus 0.3pt}%
  \let\item\@idxitem
}{%
  \clearpage
}
\makeatother

\IfFileExists{\jobname-pw.ind}{\input{\jobname-pw.ind}}{}

\end{document}

      