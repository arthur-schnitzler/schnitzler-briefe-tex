%% latex-korrekturansicht-vorspann.tex
%% Vorspann für die Korrekturansicht.
%% Lädt die gemeinsame Datei latex-vorspann.tex mit gesetztem Schalter.

\newif\ifkorrekturansicht
\korrekturansichttrue

\input{../tex-inputs/latex-vorspann}


               \section[Arthur Schnitzler an Georg Brandes, 27. 2. 1913]{ Arthur Schnitzler an Georg Brandes, 27. 2. 1913}\nopagebreak\mylabel{v}\rehead{ }\normalsize\beginnumbering\briefempfaengerindex{Brandes, Georg@\textsc{Brandes, Georg}!zzzSchnitzler, Arthur@\emph{von Arthur Schnitzler}!1913-02-271@{27. 2. 1913}|(be} \toendnotes[C]{\smallbreak\pagebreak[2]} \Standort{Kopenhagen, Det Kongelige Bibliotek, Georg Brandes Arkiv, box 125.}
\physDesc{Brief, 2 Blätter, 3 Seiten (Seite 2 und 3 mit Schreibmaschine paginiert)
\newline{}Schreibmaschine
\newline{}Handschrift: schwarze Tinte (\noindent{}zwei Streichungen, Unterschrift)\newline{}Ordnung: mit Bleistift von unbekannter Hand auf dem ersten Blatt
                                 nummeriert: »34.«; das zweite Blatt datiert mit: »27/2 13« }\buchAbdrucke{\weitereDrucke{1) Georg Brandes, Arthur Schnitzler: \emph{Ein Briefwechsel}. Hg. Kurt Bergel. Bern: \emph{Francke} 1956, S. 106–107.} \weitereDrucke{2) Arthur Schnitzler: \emph{Briefe 1913–1931}. Hg. Peter Michael Braunwarth, Richard Miklin, Susanne Pertlik und Heinrich Schnitzler. Frankfurt am Main: \emph{S. Fischer} 1984, S. 12–13.} }\toendnotes[C]{\smallbreak}\pstart
           \noindent{}{\pb}\textcolor{gray}{\textbf{Dr. Arthur Schnitzler}}{\\}\textcolor{gray}{\textbf{\textcolor{pink}{Wien XVIII. Sternwartestrasse 71}{}\ledrightnote{\textcolor{pink}{Sternwartestraße}}}}\pend
           \pstart
           \raggedleft{}27. 2. 1913\pend
           \pstart\center{}Lieber und verehrter Freund.\pend\pstart
           Ihr \label{K_L02114_1v}\edtext{\textcolor{green}{Bild}{}\ledrightnote{→\textcolor{green}{[Georg Brandes]}}}{\lemma{\textnormal{\emph{Bild}}}\Cendnote{\textnormal{Es dürfte sich um die Fotografie
                  handeln, die abgebildet ist in: \textcolor{blue}{Arthur Schnitzler}: \emph{Sein
                        Leben · Sein Werk · Seine Zeit}. Hg. Heinrich Schnitzler, Christian
                     Brandstätter und Reinhard Urbach. Frankfurt am Main: \emph{\textcolor{brown}{S. Fischer}}{ }1981, S. 73.}}}\label{K_L02114_1h} ist aus \textcolor{pink}{Paris}{}\ledrightnote{\textcolor{pink}{Paris}} eingetroffen, es ist ausserordentlich gelungen, hat uns grosse Freude
               gemacht und wir sagen Ihnen herzlichen Dank dafür.\pend
           \pstart
           Im »\textcolor{brown}{Merker}{}\ledrightnote{\textcolor{brown}{Der Merker}}« habe ich eben Ihren höchst anregenden
                  \label{K_L02114_2v}\edtext{\textcolor{green}{Artikel}{}\ledrightnote{→\textcolor{green}{Theater und Schauspiele in Deutschland}}}{\lemma{\textnormal{\emph{Artikel}}}\Cendnote{\textnormal{\textcolor{blue}{Georg Brandes}: \emph{\textcolor{green}{Theater und Schauspiele in Deutschland}}. In: \emph{\textcolor{green}{Der Merker}}, Jg. 4, H. 3, 1. Februar-Heft 1913,
                     S. 95–99.}}}\label{K_L02114_2h} über Theater in \textcolor{pink}{Deutschland}{}\ledrightnote{\textcolor{pink}{Deutschland}} gelesen. Dass meine neue Komödie »\textcolor{green}{Professor Bernhardi}{}\ledrightnote{\textcolor{green}{Professor Bernhardi. Komödie in fünf Akten}}« Sie so lebhaft interessiert hat, ist mir sehr lieb. Es
               ist über dieses Stück gar viel herumgeredet und – nicht immer bonafide – herumgeschwätzt\substVorne{}\textsuperscript{ worden}{\allowbreak}\substDazwischen{},\substHinten{} und auch Sie, verehrter Freund, sind wie speziell aus einer Ihrer
               Bemerkungen hervorgeht, über die Entstehungsgeschichte meines Stückes nicht ganz
               richtig informiert worden. Die \textcolor{green}{Komödie}{}\ledrightnote{→\textcolor{green}{Professor Bernhardi. Komödie in fünf Akten}} behandelt nicht eigentlich »ein Lebensschicksal, wie es mein Vater
               erfahren hat, der Inhalt ist vielmehr frei erfunden. Mein {\pb}\textcolor{blue}{Vater}{}\ledrightnote{→\textcolor{blue}{Johann Schnitzler}} hat wohl seinerzeit, mit
               Freunden zusammen, ein \textcolor{pink}{Krankeninstitut}{}\ledrightnote{→\textcolor{pink}{Allgemeine Poliklinik}} in der Art des Elisabethinums gegründet, hat es gegen
               mancherlei Anfeindungen mit Aufgebot seiner ganzen Begabung und Tatkraft, natürlich
               nicht ohne die Mithilfe ausgezeichneter Arbeits- und Kampfgefährten, zu hoher Blüte
               gebracht und musste insbesondere gegen Schlus seines Lebens von mancher Seite Undank
               und Kränkung erfahren; – aber wenn sein Ausscheiden aus dem von ihm gegründeten
               Institut vielleicht auch Einem oder dem Andern nicht unangenehm gewesen wäre, er ist
               keineswegs »hinausintrigiert« worden, ja, \strikeout{es} ist
               sogar als Direktor des \textcolor{pink}{Instituts}{}\ledrightnote{→\textcolor{pink}{Allgemeine Poliklinik}} am
                  2. Mai 1893 gestorben. Uebrigens hat mein Titelheld, der »\textcolor{green}{Professor Bernhardi}{}\ledrightnote{→\textcolor{green}{Professor Bernhardi. Komödie in fünf Akten}}«, von meinem
                  \textcolor{blue}{Vater}{}\ledrightnote{→\textcolor{blue}{Johann Schnitzler}} nur wenige Züge
               entliehen,und auch die anderen Figuren meines Stückes sind, mit der freilich
               unerlässlichen Benützung von Wirklichkeitszügen so frei gestaltet, dass nur
               Kunstfremde, an denen es natürlich {\pb}niemals
               mangelt, hier von einem Schlüsselstück reden k\substVorne{}\textsuperscript{o}\substDazwischen{}ö\substHinten{}nnten. Meine \textcolor{green}{Komödie}{}\ledrightnote{→\textcolor{green}{Professor Bernhardi. Komödie in fünf Akten}} hat
               keine andere Wahrheit als die, dass sich die Handlung genau so, wie ich sie
                  erfunden\strikeout{,} habe, zugetragen haben könnte, – zum
               mindesten in \textcolor{pink}{Wien}{}\ledrightnote{\textcolor{pink}{Wien}} zu Ende des vorigen
               Jahrhunderts.\pend
           \pstart
           Ich sende Ihnen diese Zeilen nach \textcolor{pink}{Kopenhagen}{}\ledrightnote{\textcolor{pink}{Kopenhagen}},
               freilich ohne zu wissen, ob Sie jetzt schon zurück sind. Man schickt Ihnen den Brief
               wohl nach, sei es nach \textcolor{pink}{Paris}{}\ledrightnote{\textcolor{pink}{Paris}} oder anderswohin.
               Kommen Sie vielleicht über \textcolor{pink}{Wien}{}\ledrightnote{\textcolor{pink}{Wien}}, wenn Sie heimreisen?
               Oder wo sonst werden Sie im Frühjahr sein? Es wäre schön einander einmal im Süden zu
               begegnen.\pend
           \pstart
           Mit herzlichen Grüssen{\\[\baselineskip]}Ihr{\\[\baselineskip]}\spacefill\mbox{{[}hs.:{]} ArthurSchnitzler}\pend
           \leftskip=0em{}\pstart
           \noindent{}{[}ms.:{]} Herrn Georg Brandes, \textcolor{pink}{Kopenhagen}{}\ledrightnote{\textcolor{pink}{Kopenhagen}}.\pend
           \endnumbering\briefempfaengerindex{Brandes, Georg@\textsc{Brandes, Georg}!zzzSchnitzler, Arthur@\emph{von Arthur Schnitzler}!1913-02-271@{27. 2. 1913}|)be}\mylabel{h}  \normalsize

\doendnotes{C}
\bigskip
\vfill

\clearpage

\footnotesize

\lohead{\textsc{register}}

% Definiere theindex-Environment komplett neu ohne reledmac
\makeatletter
\renewenvironment{theindex}{%
  \section*{\indexname}%
  \setlength{\parindent}{0pt}%
  \setlength{\parskip}{0pt plus 0.3pt}%
  \let\item\@idxitem
}{%
  \clearpage
}
\makeatother

\IfFileExists{\jobname-pw.ind}{\input{\jobname-pw.ind}}{}

\end{document}

      