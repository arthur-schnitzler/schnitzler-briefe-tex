%% latex-korrekturansicht-vorspann.tex
%% Vorspann für die Korrekturansicht.
%% Lädt die gemeinsame Datei latex-vorspann.tex mit gesetztem Schalter.

\newif\ifkorrekturansicht
\korrekturansichttrue

\input{../tex-inputs/latex-vorspann}


               \section[Franz Blei an Arthur Schnitzler, 10. 7. 1910]{ Franz Blei an Arthur Schnitzler, 10. 7. 1910}\nopagebreak\mylabel{v}\rehead{ }\normalsize\beginnumbering\briefempfaengerindex{Schnitzler, Arthur@\textsc{Schnitzler, Arthur}!zzzBlei, Franz@\emph{von Franz Blei}!1910-07-101@{10. 7. 1910}|(be} \toendnotes[C]{\smallbreak\pagebreak[2]} \Standort{CUL, Schnitzler, B 14.}
\physDesc{Brief, 1 Blatt, 1 Seite
\newline{}Handschrift: schwarze Tinte, lateinische Kurrent
\newline{}Schnitzler: 1) mit Bleistift beschriftet: »\textsc{Blei}« 2) mit rotem Buntstift zwei Unterstreichungen\newline{}Ordnung: 1) mit Bleistift von unbekannter Hand nummeriert: »\strikeout{5}« 2) mit Bleistift von unbekannter Hand nummeriert:
                                 »6«}\toendnotes[C]{\smallbreak}\leftskip=3em{}\pstart
           \noindent{}{\pb}\textcolor{pink}{Forte dei Marmi, Versilia, Ital.}{}\ledrightnote{\textcolor{pink}{Forte dei Marmi}}{\\}\textcolor{pink}{Casa Vignolo}{}\ledrightnote{\textcolor{pink}{Casa Vignolo}}.\pend
           \leftskip=0em{}\pstart{}Wertester Herr Schnitzler,\pend\pstart
           der Verleger \textcolor{blue}{Georg Müller}{}\ledrightnote{\textcolor{blue}{Georg Müller}}, \textcolor{pink}{München, Josefsplatz 7}{}\ledrightnote{\textcolor{pink}{Josephsplatz}} möchte gerne in einem schönen Druck von
               600 Exemplaren den »\label{K_L01943_1v}\edtext{\textcolor{green}{Reigen}{}\ledrightnote{\textcolor{green}{Reigen. Zehn Dialoge}}« herausgeben}{\lemma{\textnormal{\emph{Reigen« herausgeben}}}\Cendnote{\textnormal{Nicht verwirklicht, Briefe \textcolor{blue}{Georg
                     Müller}s finden sich nicht in \textcolor{blue}{Schnitzler}s Nachlass. Das Vorhaben war Teil einer größeren Buchreihe, die
                  auch, neben anderen, \textcolor{blue}{Bahr}s \emph{\textcolor{green}{Die Mutter}} und \emph{\textcolor{green}{Der Garten der
                     Erkenntnis}} von \textcolor{blue}{Leopold von
                     Andrian-Werburg} hätte enthalten sollen. (Vgl. Hartmut Walravens,
                     Angela Reinthal: \emph{Franz Blei als Berater des Verlages Georg
                        Müller. Franz Bleis Briefe an Georg Müller}. Wien: \emph{Verlag
                        der Österreichischen Akademie der Wissenschaften}{ }2015, S. 77–79, S. 118–120, S. 129.)}}}\label{K_L01943_1h}, und ich möchte
               das empfehlend unterstützen. Wenn Sie prinzipiell damit einverstanden sind, bitte ich
               Sie, sich mit \textcolor{blue}{G. Müller}{}\ledrightnote{\textcolor{blue}{Georg Müller}} zu verständigen.\pend
           \pstart
           Mit bestem Grusse{\\[\baselineskip]}\textcolor{gray}{Ihr}{\\[\baselineskip]}\spacefill\mbox{Frz Blei}\pend
           \leftskip=0em{}\pstart
           10. Juli 1910\pend
           \endnumbering\briefempfaengerindex{Schnitzler, Arthur@\textsc{Schnitzler, Arthur}!zzzBlei, Franz@\emph{von Franz Blei}!1910-07-101@{10. 7. 1910}|)be}\mylabel{h}  \normalsize

\doendnotes{C}
\bigskip
\vfill

\clearpage

\footnotesize

\lohead{\textsc{register}}

% Definiere theindex-Environment komplett neu ohne reledmac
\makeatletter
\renewenvironment{theindex}{%
  \section*{\indexname}%
  \setlength{\parindent}{0pt}%
  \setlength{\parskip}{0pt plus 0.3pt}%
  \let\item\@idxitem
}{%
  \clearpage
}
\makeatother

\IfFileExists{\jobname-pw.ind}{\input{\jobname-pw.ind}}{}

\end{document}

      