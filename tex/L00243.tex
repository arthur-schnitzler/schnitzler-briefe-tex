%% latex-korrekturansicht-vorspann.tex
%% Vorspann für die Korrekturansicht.
%% Lädt die gemeinsame Datei latex-vorspann.tex mit gesetztem Schalter.

\newif\ifkorrekturansicht
\korrekturansichttrue

\input{../tex-inputs/latex-vorspann}


               \section[Richard Beer-Hofmann an Arthur Schnitzler, {[}25. 7. 1893{]}]{ Richard Beer-Hofmann an Arthur Schnitzler, {[}25. 7. 1893{]}}\nopagebreak\mylabel{v}\rehead{ }\normalsize\beginnumbering\briefempfaengerindex{Schnitzler, Arthur@\textsc{Schnitzler, Arthur}!zzzBeer-Hofmann, Richard@\emph{von Richard Beer-Hofmann}!1893-07-252@{{[}25. 7. 1893{]}}|(be} \toendnotes[C]{\smallbreak\pagebreak[2]} \Standort{CUL, Schnitzler, B 8.}
\physDesc{Brief, 2 Blätter, 5 Seiten
\newline{}Handschrift: Bleistift, deutsche Kurrent
\newline{}Schnitzler: 1) mit Bleistift datiert: »29/7 93« 2) mit Bleistift nummeriert: »21.«}\buchAbdrucke{\weitereDrucke{Arthur Schnitzler, Richard Beer-Hofmann: \emph{Briefwechsel 1891–1931}. Hg. Konstanze Fliedl. Wien, Zürich: \emph{Europaverlag} 1992, S. 47–48.} }\toendnotes[C]{\smallbreak}\pstart
           \raggedleft{}{\pb}\textcolor{pink}{Salzburg}{}\ledrightnote{\textcolor{pink}{Salzburg}}{ }Dienst.{ }Nachmittag{\\}\uline{bei \textcolor{pink}{Tomaselli}{}\ledrightnote{\textcolor{pink}{Café Tomaselli}}}\pend
           \pstart
           Lieber Arthur! Soeben erhalte ich Ihren Brief nachgeschickt – ich
               bin in \textcolor{pink}{Salzburg}{}\ledrightnote{\textcolor{pink}{Salzburg}}; vielen Dank für Ihre Mühe – Ich bin
               seit Samst.{ }Nachm. hier – von Samstag{ }Abends bis gestern{ }Mittag in Gesellschaft. Lesen Sie die alte \textcolor{green}{Presse}{}\ledrightnote{\textcolor{green}{Die Presse}}, von Freitag »\textcolor{green}{Ischler
                  Brief}{}\ledrightnote{\textcolor{green}{Aus Ischl}}«\substVorne{}\textsuperscript{,}\substDazwischen{}:\substHinten{} ganz vernünftig {\pb}anerkennungsvoll, hält es nur für die Bühne zu stark. Aber \uuline{lesen Sie selbst}. Mich beschimpft man noch manchmal, vom moralischen
               Standp. aus.\pend
           \pstart
           Jemand – ich glaube \uline{Frau}{ }\textcolor{blue}{Wald\textcolor{gray}{ner}}{}\ledrightnote{\textcolor{blue}{Waldner}}, \textcolor{blue}{er}{}\ledrightnote{→\textcolor{blue}{Waldner}} ist doch nicht so
                  du{\geminationm} – behauptete es wäre irgendetwas zwischen Ihnen
               und \textcolor{blue}{M. B{\dotssix}t}{}\ledrightnote{\textcolor{blue}{Hermine von Schaffgotsch}} im Zuge
               gewesen; aber {\pb}nachdem Sie
               derartige Sachen, \uuline{\edtext{aus Ihrem Leben!}{\Cendnote{achtfach unterstrichen}}} auf die Bühne
                  bringe{[}n{]}, scheine man eingesehen zu haben daß es denn doch
               nicht gienge; \textcolor{blue}{Jarno}{}\ledrightnote{\textcolor{blue}{Josef Jarno}} habe ich ein einziges mal
               gesprochen. {\pb}Er kam zur \textcolor{blue}{Wreden}{}\ledrightnote{\textcolor{blue}{Grethe Wreden}}, während ich u. \textcolor{blue}{Paul Horn}{}\ledrightnote{\textcolor{blue}{Paul Horn}} dort waren. Sind Sie mit \textcolor{blue}{Julius Bauer}{}\ledrightnote{\textcolor{blue}{Julius Bauer}} zufrieden? Hier ist’s herrlich! ich schreibe ein
               wenig und feiere Orgien im Entbinden von Plänen; ich ergreife Pauschalbesitz von \textcolor{pink}{Salzburg}{}\ledrightnote{\textcolor{pink}{Salzburg}} – sagen Sie es \textcolor{blue}{Salten}{}\ledrightnote{\textcolor{blue}{Felix Salten}}, den ich herzlich grüße. Sie auch \spacefill\mbox{Richard}\pend
           \pstart
           \noindent{}{\pb}Soeben fällt mir ein daß ich
                  bez. \textcolor{brown}{Verlag v. Freund}{}\ledrightnote{\textcolor{brown}{Freund {\kaufmannsund} Jeckel}} nicht geantwortet habe. \textcolor{blue}{Flegmann}{}\ledrightnote{\textcolor{blue}{Bertha Flegmann}} bat mich Ihnen mitzuteilen daß \textcolor{blue}{Freund}{}\ledrightnote{\textcolor{blue}{Carl Freund}} nicht in \textcolor{pink}{Berlin}{}\ledrightnote{\textcolor{pink}{Berlin}}, \uline{nicht} in d. Bädern sei, sondern
                  in der – \textcolor{pink}{Dauphinée}{}\ledrightnote{\textcolor{pink}{Dauphiné}} – bitte nachzusehen ob die
                  Orthographie richtig – Bis zu seiner Rückkehr kann man nichts tun\pend
           \pstart
           \raggedleft{}\spacefill\mbox{R.}\pend
           \pstart
           \noindent{}\label{T_L00243_1v}\edtext{Ich reise morgen nach \textcolor{pink}{Ischl}{}\ledrightnote{\textcolor{pink}{Bad Ischl}} zurück.}{\lemma{\textnormal{\emph{Ich … zurück.}}}\Cendnote{\textnormal{quer am
                     rechten Rand der vierten Seite}}}\label{T_L00243_1h}\pend
           \endnumbering\briefempfaengerindex{Schnitzler, Arthur@\textsc{Schnitzler, Arthur}!zzzBeer-Hofmann, Richard@\emph{von Richard Beer-Hofmann}!1893-07-252@{{[}25. 7. 1893{]}}|)be}\mylabel{h}  \normalsize

\doendnotes{C}
\bigskip
\vfill

\clearpage

\footnotesize

\lohead{\textsc{register}}

% Definiere theindex-Environment komplett neu ohne reledmac
\makeatletter
\renewenvironment{theindex}{%
  \section*{\indexname}%
  \setlength{\parindent}{0pt}%
  \setlength{\parskip}{0pt plus 0.3pt}%
  \let\item\@idxitem
}{%
  \clearpage
}
\makeatother

\IfFileExists{\jobname-pw.ind}{\input{\jobname-pw.ind}}{}

\end{document}

      