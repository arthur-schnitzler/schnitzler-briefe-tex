%% latex-korrekturansicht-vorspann.tex
%% Vorspann für die Korrekturansicht.
%% Lädt die gemeinsame Datei latex-vorspann.tex mit gesetztem Schalter.

\newif\ifkorrekturansicht
\korrekturansichttrue

\input{../tex-inputs/latex-vorspann}


               \section[Hermann Bahr an Arthur Schnitzler, 27. 7. {[}1904{]}]{ Hermann Bahr an Arthur Schnitzler, 27. 7. {[}1904{]}}\nopagebreak\mylabel{v}\rehead{ }\normalsize\beginnumbering\briefempfaengerindex{Schnitzler, Arthur@\textsc{Schnitzler, Arthur}!zzzBahr, Hermann@\emph{von Hermann Bahr}!1904-07-271@{27. 7. {[}1904{]}}|(be} \toendnotes[C]{\smallbreak\pagebreak[2]} \Standort{CUL, Schnitzler, B 5b.}
\physDesc{Brief, 1 Blatt, 1 Seite
\newline{}Handschrift: schwarze Tinte, deutsche Kurrent\newline{}Ordnung: mit Bleistift von unbekannter Hand nummeriert: »118« }\buchAbdrucke{\weitereDrucke{Hermann Bahr, Arthur Schnitzler: \emph{Briefwechsel, Aufzeichnungen, Dokumente (1891–1931)}. Hg. Kurt Ifkovits und Martin Anton Müller. Göttingen: \emph{Wallstein} 2018, S. 309.} }\toendnotes[C]{\smallbreak}\pstart
           \raggedleft{}{\pb}27. 7.\pend
           \pstart\center{}Lieber Arthur!\pend\pstart
           Ich bin einige Zeit ganz in mein neues \textcolor{green}{Stück}{}\ledrightnote{→\textcolor{green}{Sanna. Schauspiel in fünf Aufzügen}} verloren geweſen, das jetzt fertig iſt. Dann hieß es, daß Du nach \textcolor{pink}{Reichenau}{}\ledrightnote{\textcolor{pink}{Reichenau an der Rax}} biſt. Nun geh ich morgen auf acht oder zehn
               Tage nach \textcolor{pink}{Salzburg}{}\ledrightnote{\textcolor{pink}{Salzburg}}, \textcolor{pink}{Bayreuth}{}\ledrightnote{\textcolor{pink}{Bayreuth}}, \textcolor{pink}{München}{}\ledrightnote{\textcolor{pink}{München}}. \label{K_L01419_1v}\edtext{Zurück}{\lemma{\textnormal{\emph{Zurück}}}\Cendnote{\textnormal{\textcolor{blue}{Bahr} kehrte am 3. 8. nach \textcolor{pink}{Wien}
                  zurück.}}}\label{K_L01419_1h}, will ich mich gleich bei Dir melden, um endlich wieder einmal mit
               Dir zu ſein, wonach ſchon ſehr verlangt Deinem\pend
           \pstart
           Dich und Deine liebe \textcolor{blue}{Frau}{}\ledrightnote{→\textcolor{blue}{Olga Schnitzler}}
               herzlichſt grüßenden{\\[\baselineskip]}\spacefill\mbox{Hermann}\pend
           \leftskip=0em{}\endnumbering\briefempfaengerindex{Schnitzler, Arthur@\textsc{Schnitzler, Arthur}!zzzBahr, Hermann@\emph{von Hermann Bahr}!1904-07-271@{27. 7. {[}1904{]}}|)be}\mylabel{h}  \normalsize

\doendnotes{C}
\bigskip
\vfill

\clearpage

\footnotesize

\lohead{\textsc{register}}

% Definiere theindex-Environment komplett neu ohne reledmac
\makeatletter
\renewenvironment{theindex}{%
  \section*{\indexname}%
  \setlength{\parindent}{0pt}%
  \setlength{\parskip}{0pt plus 0.3pt}%
  \let\item\@idxitem
}{%
  \clearpage
}
\makeatother

\IfFileExists{\jobname-pw.ind}{\input{\jobname-pw.ind}}{}

\end{document}

      