%% latex-korrekturansicht-vorspann.tex
%% Vorspann für die Korrekturansicht.
%% Lädt die gemeinsame Datei latex-vorspann.tex mit gesetztem Schalter.

\newif\ifkorrekturansicht
\korrekturansichttrue

\input{../tex-inputs/latex-vorspann}


               \section[Bertha von Suttner an Arthur Schnitzler, 22. 10. 1913]{ Bertha von Suttner an Arthur Schnitzler,
                    22. 10. 1913}\nopagebreak\mylabel{v}\rehead{ }\normalsize\beginnumbering\briefempfaengerindex{Schnitzler, Arthur@\textsc{Schnitzler, Arthur}!zzzSuttner, Bertha von@\emph{von Bertha von Suttner}!1913-10-221@{22. 10. 1913}|(be} \toendnotes[C]{\smallbreak\pagebreak[2]} \Standort{CUL, Schnitzler, B 104.}
\physDesc{Brief, 1 Blatt (mit Krone in Golddruck), 2 Seiten
\newline{}Handschrift: schwarze Tinte, deutsche Kurrent
\newline{}Schnitzler: 1) mit Bleistift beschriftet: »\textsc{Suttner}« 2)  mit rotem Buntstift eine Anstreichung}\Standort{DLA, A:Schnitzler, HS.NZ85.1.4773.}
\physDesc{1 Blatt, 1 Seite, maschinelle Abschrift}\toendnotes[C]{\smallbreak}\pstart
           \noindent{}\centering{}{\pb}\textcolor{pink}{\textsc{Zedlitzgasse 7 Wien}}{}\ledrightnote{\textcolor{pink}{Zedlitzgasse}}\pend
           \pstart
           \raggedleft{}22/10 1913\pend
           \pstart{}Verehrter Dichter\pend\pstart
           In einer \label{K_L02153_1v}\edtext{Angelegenheit}{\lemma{\textnormal{\emph{Angelegenheit}}}\Cendnote{\textnormal{vgl. A. S.: \emph{Tagebuch}, 29. 10. 1913}}}\label{K_L02153_1h}, die Sie und mich angeht, wäre mir eine Rückſprache ſehr erwünſcht.\pend
           \pstart
           Wie ſollen wir das machen? Ich wäre auch gern bereit, zu einer Stunde, wo Sie u.
                    Frau D\textsuperscript{r}{ }\textcolor{blue}{Schnitzler}{}\ledrightnote{\textcolor{blue}{Olga Schnitzler}} ein paar Freunde um ſich haben,
                    nach der \label{K_L02153_2v}\edtext{\textcolor{pink}{Sternwartegaſſe}{}\ledrightnote{\textcolor{pink}{Sternwartestraße}}}{\lemma{\textnormal{\emph{Sternwartegaſſe}}}\Cendnote{\textnormal{richtig: \textcolor{pink}{Sternwartestraße}}}}\label{K_L02153_2h} zu kommen. Da {\pb}würde ich Sie
                    um nichts von Ihrer Arbeitszeit berauben, und zugleich das Vergnügen einer
                    gemüthlichen Unterhaltung mit Ihnen beiden \strikeout{ge}
                    haben.\pend
           \pstart
           Mit ausgezeichneter Hochachtung{\\[\baselineskip]}Ihre erg.{\\[\baselineskip]}\spacefill\mbox{Bertha v. Suttner}\pend
           \leftskip=0em{}\endnumbering\briefempfaengerindex{Schnitzler, Arthur@\textsc{Schnitzler, Arthur}!zzzSuttner, Bertha von@\emph{von Bertha von Suttner}!1913-10-221@{22. 10. 1913}|)be}\mylabel{h}  \normalsize

\doendnotes{C}
\bigskip
\vfill

\clearpage

\footnotesize

\lohead{\textsc{register}}

% Definiere theindex-Environment komplett neu ohne reledmac
\makeatletter
\renewenvironment{theindex}{%
  \section*{\indexname}%
  \setlength{\parindent}{0pt}%
  \setlength{\parskip}{0pt plus 0.3pt}%
  \let\item\@idxitem
}{%
  \clearpage
}
\makeatother

\IfFileExists{\jobname-pw.ind}{\input{\jobname-pw.ind}}{}

\end{document}

      