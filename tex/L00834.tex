%% latex-korrekturansicht-vorspann.tex
%% Vorspann für die Korrekturansicht.
%% Lädt die gemeinsame Datei latex-vorspann.tex mit gesetztem Schalter.

\newif\ifkorrekturansicht
\korrekturansichttrue

\input{../tex-inputs/latex-vorspann}


               \section[Arthur Schnitzler an Richard Beer-Hofmann, 13. 8. 1898]{ Arthur Schnitzler an Richard Beer-Hofmann, 13. 8. 1898}\nopagebreak\mylabel{v}\rehead{ }\normalsize\beginnumbering\briefempfaengerindex{Beer-Hofmann, Richard@\textsc{Beer-Hofmann, Richard}!zzzSchnitzler, Arthur@\emph{von Arthur Schnitzler}!1898-08-131@{13. 8. 1898}|(be} \toendnotes[C]{\smallbreak\pagebreak[2]} \Standort{YCGL, MSS 31.}
\physDesc{Postkarte
\newline{}Handschrift: Bleistift, deutsche Kurrent\newline{}Versand: 1) Stempel: »\nobreak{}\oindex{Biel@\textbf{Biel}, \emph{http://www.geonames.org/ontologyP.PPLA2}|pwk}Bienne, 13. VIII. 98, 5\nobreak{}«.  2) Stempel: »\nobreak{}\oindex{Steindorf am Ossiacher See@\textbf{Steindorf am Ossiacher See}, \emph{http://www.geonames.org/ontologyA.ADM3}|pwk}Steindorf am Ossiacher See, 15 8 98\nobreak{}«. \newline{}Ordnung: mit Bleistift von unbekannter Hand datiert:
                                 »13. 8.« }\buchAbdrucke{\weitereDrucke{Arthur Schnitzler, Richard Beer-Hofmann: \emph{Briefwechsel 1891–1931}. Hg. Konstanze Fliedl. Wien, Zürich: \emph{Europaverlag} 1992, S. 124.} }\toendnotes[C]{\smallbreak}\pstart{}{\pb}\textsc{Dr. Richard Beer-Hofmann}\pend{}\pstart{}\textsc{\textcolor{pink}{Steindorf}{}\ledrightnote{\textcolor{pink}{Steindorf am Ossiacher See}}}\pend{}\pstart{}\textsc{am }\textcolor{pink}{\textsc{Ossiacher}ſee}{}\ledrightnote{\textcolor{pink}{Ossiacher See}}.\pend{}\pstart{}\textcolor{pink}{\textsc{Kärnthen}}{}\ledrightnote{\textcolor{pink}{Kärnten}}.\pend{}{\bigskip}\pstart
           \noindent{}{\pb}Unſer lieber Richard, wir denken (ſagt \textcolor{blue}{Hugo}{}\ledrightnote{\textcolor{blue}{Hugo von Hofmannsthal}}) oft an Sie (ſage ich) – ſchreiben Sie uns gleich (ſage
               ich) \textcolor{pink}{\textsc{Genf}}{}\ledrightnote{\textcolor{pink}{Genf}}{ }\textsc{post rest} (ſagt \textcolor{blue}{Hugo}{}\ledrightnote{\textcolor{blue}{Hugo von Hofmannsthal}}),
               wo wir \label{K_L00834_1v}\edtext{Mittwoch}{\lemma{\textnormal{\emph{Mittwoch}}}\Cendnote{\textnormal{vgl. A. S.: \emph{Tagebuch}, 17. 8. 1898}}}\label{K_L00834_1h} ſind. Ich möchte irgendwo am \textcolor{pink}{Genferſee}{}\ledrightnote{\textcolor{pink}{Genfer See}} bleiben, \textcolor{blue}{Hugo}{}\ledrightnote{\textcolor{blue}{Hugo von Hofmannsthal}} geht
               wahrſcheinlich nach \textcolor{pink}{Lugano}{}\ledrightnote{\textcolor{pink}{Lugano}}, doch ist es möglich,
                  {[}d{]}ſs wir beide \introOben{}eine Zeit lang\introOben{} zuſa{\geminationm}en bleiben, hier oder dort. Von Herzen Ihr
                  \spacefill\mbox{Arthur}\pend
           \endnumbering\briefempfaengerindex{Beer-Hofmann, Richard@\textsc{Beer-Hofmann, Richard}!zzzSchnitzler, Arthur@\emph{von Arthur Schnitzler}!1898-08-131@{13. 8. 1898}|)be}\mylabel{h}  \normalsize

\doendnotes{C}
\bigskip
\vfill

\clearpage

\footnotesize

\lohead{\textsc{register}}

% Definiere theindex-Environment komplett neu ohne reledmac
\makeatletter
\renewenvironment{theindex}{%
  \section*{\indexname}%
  \setlength{\parindent}{0pt}%
  \setlength{\parskip}{0pt plus 0.3pt}%
  \let\item\@idxitem
}{%
  \clearpage
}
\makeatother

\IfFileExists{\jobname-pw.ind}{\input{\jobname-pw.ind}}{}

\end{document}

      