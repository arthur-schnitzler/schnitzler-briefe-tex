%% latex-korrekturansicht-vorspann.tex
%% Vorspann für die Korrekturansicht.
%% Lädt die gemeinsame Datei latex-vorspann.tex mit gesetztem Schalter.

\newif\ifkorrekturansicht
\korrekturansichttrue

\input{../tex-inputs/latex-vorspann}


               \section[Jakob Julius David an Arthur Schnitzler, 28. 2. 1899]{ Jakob Julius David an Arthur Schnitzler, 28. 2. 1899}\nopagebreak\mylabel{v}\rehead{ }\normalsize\beginnumbering\briefempfaengerindex{Schnitzler, Arthur@\textsc{Schnitzler, Arthur}!zzzDavid, Jakob Julius@\emph{von Jakob Julius David}!1899-02-281@{28. 2. 1899}|(be} \toendnotes[C]{\smallbreak\pagebreak[2]} \Standort{CUL, Schnitzler, B 25.}
\physDesc{Postkarte
\newline{}Handschrift: schwarze Tinte, lateinische Kurrent\newline{}Versand: 1) Stempel: »\nobreak{}\oindex{II., Leopoldstadt@\textbf{II., Leopoldstadt}, \emph{Bezirk (A.BZK)}|pwk}Wien 2/3, 28. 2. 99, 3–4N\nobreak{}«.  2) Stempel: »\nobreak{}\oindex{IX., Alsergrund@\textbf{IX., Alsergrund}, \emph{Bezirk (A.BZK)}|pwk}Wien 9/3, 28. 2. 99, 6.N, Bestellt\nobreak{}«. \newline{}Ordnung: mit Bleistift von unbekannter Hand nummeriert: »7« }\toendnotes[C]{\smallbreak}\pstart{}{\pb}Herrn D\textsuperscript{r} Arthur Schnitzler\pend{}\pstart{}\textcolor{pink}{IX.}{}\ledrightnote{\textcolor{pink}{IX., Alsergrund}}{ }\textcolor{pink}{Franckgaße N\textsuperscript{o}. 1.}{}\ledrightnote{\textcolor{pink}{Frankgasse}}\pend{}{\bigskip}\pstart\center{}{\pb}Lieber Freund!\pend\pstart
           Noch ganz im Eindruck – meine aufrichtige Freude! Zwei von den \textcolor{green}{Sachen}{}\ledrightnote{→\textcolor{green}{Der grüne Kakadu – Paracelsus – Die Gefährtin. Drei Einakter}} haben mir
                    imponirt und ich will nicht hinterm Berg halten mit meiner Meinung. Wünschen Sie
                    die Büc\damage{he}r wieder, so stehen sie Ihnen zur Verfügung.\pend
           \pstart
           Wider meine Gewohnheit bitte ich Sie zur dritten oder vierten \textcolor{green}{Vorstellung}{}\ledrightnote{→\textcolor{green}{Der grüne Kakadu – Paracelsus – Die Gefährtin. Drei Einakter}} um
                    zwei Karten. Ich möchte mir die Sachen noch einmal und nicht im \textcolor{green}{Premièren-Ru{\geminationm}el}{}\ledrightnote{→\textcolor{green}{Der grüne Kakadu – Paracelsus – Die Gefährtin. Drei Einakter}} ansehn.\pend
           \pstart
           Bestens Ihr{\\[\baselineskip]}\spacefill\mbox{David}\pend
           \leftskip=0em{}\endnumbering\briefempfaengerindex{Schnitzler, Arthur@\textsc{Schnitzler, Arthur}!zzzDavid, Jakob Julius@\emph{von Jakob Julius David}!1899-02-281@{28. 2. 1899}|)be}\mylabel{h}  \normalsize

\doendnotes{C}
\bigskip
\vfill

\clearpage

\footnotesize

\lohead{\textsc{register}}

% Definiere theindex-Environment komplett neu ohne reledmac
\makeatletter
\renewenvironment{theindex}{%
  \section*{\indexname}%
  \setlength{\parindent}{0pt}%
  \setlength{\parskip}{0pt plus 0.3pt}%
  \let\item\@idxitem
}{%
  \clearpage
}
\makeatother

\IfFileExists{\jobname-pw.ind}{\input{\jobname-pw.ind}}{}

\end{document}

      