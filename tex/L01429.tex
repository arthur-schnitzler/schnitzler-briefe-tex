%% latex-korrekturansicht-vorspann.tex
%% Vorspann für die Korrekturansicht.
%% Lädt die gemeinsame Datei latex-vorspann.tex mit gesetztem Schalter.

\newif\ifkorrekturansicht
\korrekturansichttrue

\input{../tex-inputs/latex-vorspann}


               \section[Hugo von Hofmannsthal an Arthur Schnitzler, 18. 8. 1904]{ Hugo von Hofmannsthal an Arthur Schnitzler, 18. 8. 1904}\nopagebreak\mylabel{v}\rehead{ }\normalsize\beginnumbering\briefempfaengerindex{Schnitzler, Arthur@\textsc{Schnitzler, Arthur}!zzzHofmannsthal, Hugo von@\emph{von Hugo von Hofmannsthal}!1904-08-181@{18. 8. 1904}|(be} \toendnotes[C]{\smallbreak\pagebreak[2]} \Standort{CUL, Schnitzler, B 43.}
\physDesc{Postkarte
\newline{}Handschrift: schwarze Tinte, deutsche Kurrent\newline{}Versand: 1) Stempel: »\nobreak{}\oindex{Bad Aussee@\textbf{Bad Aussee}, \emph{Besiedelter Ort (A.BSO)}|pwk}Aussee in Steiermark, 18 8 04\nobreak{}«.  2) Stempel: »\nobreak{}\oindex{XVIII., Waehring@\textbf{XVIII., Währing}, \emph{Bezirk (A.BZK)}|pwk}18/1 Wien, 19. 8. 04, 3.N, Bestellt\nobreak{}«. \newline{}Ordnung: 1) mit Bleistift von unbekannter Hand
                           nummeriert: »\strikeout{223}« 2) mit Bleistift von unbekannter Hand nummeriert: »233«}\buchAbdrucke{\weitereDrucke{Hugo von Hofmannsthal, Arthur Schnitzler: \emph{Briefwechsel}. Hg. Therese Nickl und Heinrich Schnitzler. Frankfurt am Main: \emph{S. Fischer} 1964, S. 197.} }\toendnotes[C]{\smallbreak}\pstart{}{\pb}\textsc{Herrn D\textsuperscript{r} Arthur
                  Schnitzler}\pend{}\pstart{}\textcolor{pink}{\textsc{Wien}}{}\ledrightnote{\textcolor{pink}{Wien}}\pend{}\pstart{}\textcolor{pink}{\textsc{XVIII Spöttelgasse 7}.}{}\ledrightnote{\textcolor{pink}{Edmund-Weiß-Gasse}}\pend{}{\bigskip}\pstart
           \raggedleft{}{\pb}18. VIII.\pend
           \pstart
           lieber, den Inhalt Ihres großen Briefes werde ich mit \textcolor{blue}{\textsc{V. S.}}{}\ledrightnote{\textcolor{blue}{Robert Gilbert Vansittart}} genau
               durchſprechen und ſeiner Energie wird es gewiſs gelingen, Ordnung in die Sache zu
               bringen. Jetzt etwas anderes; bitte ſchreiben Sie mir gleich, ob Ihr Plan,
                  erſte Tage September{ }\textcolor{pink}{Salzka{\geminationm}ergut}{}\ledrightnote{\textcolor{pink}{Salzkammergut}} feſtſteht.
                  \textcolor{blue}{Gerty}{}\ledrightnote{\textcolor{blue}{Gertrude von Hofmannsthal}} bringt 25\textsuperscript{ten} VIII die \textcolor{blue}{Kinder}{}\ledrightnote{→\textcolor{blue}{Christiane von Hofmannsthal}{\newline}→\textcolor{blue}{Raimund von Hofmannsthal}{\newline}→\textcolor{blue}{Franz von Hofmannsthal}} nach \textcolor{pink}{Rodaun}{}\ledrightnote{\textcolor{pink}{Rodaun}} zurück
               und wäre \uline{ſehr} erfreut, mit Ihnen und \textcolor{blue}{Olga}{}\ledrightnote{\textcolor{blue}{Olga Schnitzler}} etwa den 2\textsuperscript{ten},
                     3\textsuperscript{ten} September nach \textcolor{pink}{Iſchl}{}\ledrightnote{\textcolor{pink}{Bad Ischl}} zu fahren.\pend
           \pstart Ihr \spacefill\mbox{Hugo}.\pend{}\endnumbering\briefempfaengerindex{Schnitzler, Arthur@\textsc{Schnitzler, Arthur}!zzzHofmannsthal, Hugo von@\emph{von Hugo von Hofmannsthal}!1904-08-181@{18. 8. 1904}|)be}\mylabel{h}  \normalsize

\doendnotes{C}
\bigskip
\vfill

\clearpage

\footnotesize

\lohead{\textsc{register}}

% Definiere theindex-Environment komplett neu ohne reledmac
\makeatletter
\renewenvironment{theindex}{%
  \section*{\indexname}%
  \setlength{\parindent}{0pt}%
  \setlength{\parskip}{0pt plus 0.3pt}%
  \let\item\@idxitem
}{%
  \clearpage
}
\makeatother

\IfFileExists{\jobname-pw.ind}{\input{\jobname-pw.ind}}{}

\end{document}

      