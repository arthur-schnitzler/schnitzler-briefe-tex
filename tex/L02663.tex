%% latex-korrekturansicht-vorspann.tex
%% Vorspann für die Korrekturansicht.
%% Lädt die gemeinsame Datei latex-vorspann.tex mit gesetztem Schalter.

\newif\ifkorrekturansicht
\korrekturansichttrue

\input{../tex-inputs/latex-vorspann}


               \section[Paul Goldmann an Arthur Schnitzler, 16. 5. 1891]{ Paul Goldmann an Arthur Schnitzler, 16. 5. 1891}\nopagebreak\mylabel{v}\rehead{ }\normalsize\beginnumbering\briefempfaengerindex{Schnitzler, Arthur@\textsc{Schnitzler, Arthur}!zzzGoldmann, Paul@\emph{von Paul Goldmann}!1891-05-161@{16. 5. 1891}|(be} \toendnotes[C]{\smallbreak\pagebreak[2]} \Standort{DLA, A:Schnitzler, HS.NZ85.1.3162.}
\physDesc{Brief, 2 Blätter, 5 Seiten
\newline{}Handschrift: schwarze Tinte, deutsche Kurrent}\toendnotes[C]{\smallbreak}\pstart
           \centering{}{\pb}\textsc{\textcolor{pink}{Brüssel}{}\ledrightnote{\textcolor{pink}{Brüssel}}}{ }16. \textsc{mai} 1891.\pend
           \pstart\center{}Mein lieber Arthur!\pend\pstart
           Dein Brief als erſten Freundesgruß i\substVorne{}\textsuperscript{m}\substDazwischen{}n\substHinten{} fremder \textcolor{pink}{Stadt}{}\ledrightnote{→\textcolor{pink}{Brüssel}} – das
               hat mir aufrichtig wohlgethan. Sei von Herzen bedankt für Deine Treue! {\dots} Wenn ich Dir von unterwegs eine Karte ſchickte, ſo
               geschah das nicht, um Dir zu ſchreiben, ſondern um Dir einen Beweis zu geben, daß ich
               mitten im \textcolor{gray}{Wirrwar} der neuen Eindrücke und im Fieber der Arbeit
               Deiner denke. Das war eine harte Zeit – dieſe ſechs Tage\textcolor{gray}{.} Morgens
               in der Regel um fünf Uhr aufſtehen, um die Bergleute noch vor der Einfahrt in den
               Schacht zu ſehen, ſtundenlang im glühenden Sonnenbrand über ſtaubige \label{K_L02663-5v}\edtext{\begin{otherlanguage}{french}Chauſſéen\end{otherlanguage}}{\lemma{\textnormal{\emph{Chauſſéen}}}\Cendnote{\textnormal{französisch: Landstraßen}}}\label{K_L02663-5h} wandern,
               ſich täglich von \textcolor{gray}{vertrackten} Localbahnen das Herz aus dem Leibe
               ſchütteln laſſen, und Abends, todtmüde, den Bericht ſchreiben (um ihn dann\strikeout{)}, einige Tage ſpäter, elend zuſammengeſtrichen oder
               gar nicht im \textcolor{green}{Blatte}{}\ledrightnote{→\textcolor{green}{Frankfurter Zeitung}} zu
               finden). Endlich \strikeout{i\textcolor{gray}{n}} bin ich heut nach \textcolor{pink}{Brüſſel}{}\ledrightnote{\textcolor{pink}{Brüssel}} gekommen; aber ſei es nun die Nervenreaction gegen die
               Überanſtrengung der letzten Tage, ſei es das Erwachen des Bewußtſeins aus dem Rauſche
               der Arbeit – ich fühle mich todtenbang und pſychiſch elend. Und als ich Deinen Brief
               las, war es ein veritables tiefes, tiefes Heimweh nach \textcolor{pink}{Wien}{}\ledrightnote{\textcolor{pink}{Wien}}, das mir durch das Herz ſchnitt, {\pb}wie
               nur ein Heimweh ſchneiden kann. Und es war nicht blos ein Heimweh nach \textcolor{pink}{Wien}{}\ledrightnote{\textcolor{pink}{Wien}}, ſondern eine Sehnſucht nach der beſſeren Welt
               dort, die ich auf immer verloren. Du kennſt ja meinen Neid mit der umgekehrten
               Spitze, der ſich nicht gegen den Andern ſondern gegen mich ſelbſt kehrt. Und ſo war
               es mir ein gar bitteres Gefühl, als ich von Deinen Erfolgen las, daß ich ſo ganz aus
               der Reihe Jener geriſſen bin, die nach dem hohen Ziele ſtreben, das nicht mehr das
                  \strikeout{d\textcolor{gray}{e}} meine ſein darf. Wir ſind eine Zeitlang Seite an Seite gewandert; jetzt bin
               ich an einem Stein am Wege unterwegs ſtehen geblieben und ſehe Dir wehmüthig nach,
               wie Du emporſteigſt. Das iſt die Schlacke, die meine Empfindung der Freude an deinen
               Erfolgen aufſetzt; wir ſind eben Alle keine Menſchen der reinen Empfindungen; vom
               Herzen, dem d\substVorne{}\textsuperscript{a}\substDazwischen{}ie\substHinten{}s Gefühl entſtrömt, tropft immer ein wenig Ich mit hinein. Ich ſage Dir das
               eigentlich nur, um auf der andern Seite das Recht zu haben, von der warmen
               Aufrichtigkeit meiner Mitfreude zu ſprechen. Nur ſo weiter! Stark und tapfer! Und ich
               habe nur einen Wunſch für Dich: daß \strikeout{\textcolor{gray}{al}} Dir \strikeout{gelingen} die Kraft werde, all’ das Schöne
               aus Dir herauszuarbeiten, was – meiner feſten Überzeugung nach – in Dir ſteckt. Die
               Kritiken ſchickſt Du mir wohl alle; Du bekommſt ſie pünktlich zurück; ebenſo werde
               ich Dich, wenn ich mich erſt ein wenig eingearbeitet und mir Zeit genommen habe, um
               alle \uline{drei} Acte des \textcolor{green}{Stück}{}\ledrightnote{→\textcolor{green}{Das Märchen. Schauspiel in drei Aufzügen}}es{\pb}
               bitten. Desgleichen ſollſt Du mir \uline{bald}{ }Folgendes schreiben: 1.) wie Du Deinen Tag
               verbringſt, mit genauer trockener Aufzählung der regelmäßigen Beſchäftigung von Früh
               bis Abend 2.) ob \label{K_L02663-11v}\edtext{\textsc{\textcolor{blue}{Schwarzkopf}{}\ledrightnote{\textcolor{blue}{Gustav Schwarzkopf}}} dein \textcolor{green}{Stück}{}\ledrightnote{→\textcolor{green}{Das Märchen. Schauspiel in drei Aufzügen}}
               bereits geleſen}{\lemma{\textnormal{\emph{Schwarzkopf … geleſen}}}\Cendnote{\textnormal{\textcolor{blue}{Gustav Schwarzkopf} dürfte \emph{\textcolor{green}{Das Märchen}} erst am 25. 6. 1891 kennengelernt haben, als \textcolor{blue}{Schnitzler} es ihm und anderen \textcolor{blue}{Freunden} vorlas.}}}\label{K_L02663-11h} hat? 3.) ob Du noch mit \label{K_L02663-12v}\edtext{\textcolor{brown}{Jung-Wien}{}\ledrightnote{\textcolor{brown}{Jung Wien}}}{\lemma{\textnormal{\emph{Jung-Wien}}}\Cendnote{\textnormal{Gemeint ist ein loser \textcolor{brown}{Verein}, bei dem immer Dienstags neue
                  Texte vorgelesen wurden. Das erste Treffen (in Beteiligung \textcolor{blue}{Goldmann}s) fand am 17. 3. 1891 in der Weinhandlung \textcolor{pink}{Wieninger} statt, das letzte, das \textcolor{blue}{Schnitzler} erwähnte, am 5. 5. 1891, eventuell
                  auch am darauffolgenden Dienstag.}}}\label{K_L02663-12h} verkehrſt? 4.) ob Du noch zu \label{K_L02663-14v}\edtext{\textsc{\textcolor{blue}{Fanjung}{}\ledrightnote{→\textcolor{blue}{Boris Van-Jung}{\newline}→\textcolor{blue}{Leo Van-Jung}}}’s }{\lemma{\textnormal{\emph{Fanjung’s }}}\Cendnote{\textnormal{Das Brüderpaar \textcolor{blue}{Leo} und \textcolor{blue}{Boris Van-Jung} erwähnte \textcolor{blue}{Schnitzler}
                  im \emph{\textcolor{green}{Tagebuch}} im Jahr 1891 nur am
                     5. 2. 1891, in den
                  Folgejahren jedoch öfter.}}}\label{K_L02663-14h} kommſt? 5.) wer jetzt Deinen hauptſächlichen
               Verkehr bildet? 6.) was \textsc{\textcolor{blue}{Olga}{}\ledrightnote{\textcolor{blue}{Olga Waissnix}}} macht? 7.) was Du lieſt? und 8.) was Du zu ſchreiben gedenkſt? –
                  j\textcolor{gray}{a} richtig und 9.) noch was Du für den \label{K_L02663-55v}\edtext{Sommer vorhaſt}{\lemma{\textnormal{\emph{Sommer vorhaſt}}}\Cendnote{\textnormal{\textcolor{blue}{Schnitzler} verbrachte den Sommer
                     1891 unter anderem in \textcolor{pink}{Baden}, \textcolor{pink}{Ischl} und \textcolor{pink}{Halle
                     an der Saale}.}}}\label{K_L02663-55h}? Du wirſt zwar nach Beantwortung all’ dieſer Fragen
               ſo erſchöpft von der Anſtrengung ſein, daß Du wirſt eine einwöchentliche
               Kaltwaſſerkur gebrauchen müſſen (\label{K_L02663-4v}\edtext{Briefkaſtenwitz}{\lemma{\textnormal{\emph{Briefkaſtenwitz}}}\Cendnote{\textnormal{Unklare Anspielung;
                  eventuell bezieht sich der Ausdruck »Briefkasten« auf einen in vielen
                  Zeitschriften enthaltenen Abschnitt, in der unter dem Titel »Briefkasten«
                  Antworten der Herausgeberinnen und Herausgeber auf Zuschriften des Publikums in
                  knapper, oft auch satirischer Form gegeben wurden.}}}\label{K_L02663-4h}) – aber Du thuſt mir’s
               wohl aus alter Freundſchaft.\pend
           \pstart
           Meinen gegenwärtigen Lebensinhalt wirſt du wohl aus dem, was am Eingang dieses
               Briefes ſteht, zur Genüge erkennen. \textcolor{pink}{Brüſſel}{}\ledrightnote{\textcolor{pink}{Brüssel}} ſagt
               mir vorläufig gar nichts – es ſei denn, daß es eine unſäglich theure \textcolor{pink}{Stadt}{}\ledrightnote{→\textcolor{pink}{Brüssel}} iſt und daß ich keine Ahnung habe, wie
               ich hier mit meinem kleinen Gehalt und meinen großen Schulden leben ſoll. Große
               Sorgen machen mir ferner die äußerſt verzwickten politiſchen Verhältniſſe, in die
               mich einzuarbeiten ich Monate Zeit haben müßte, während man {\pb}mein ſofortiges Treten in Action verlangt ſowie
               meine Unkenntniß im Franzöſiſchen. Meine Fähigkeit zu verſtehen iſt gleich Null; und
               wenn es noch vier Grad weniger gibt als Null, ſo bezeichnet dieſes meine Fähigkeit
               mich verſtändlich zu machen. Von ſelbſt wird das nicht kommen; Alle lügen, die ſagen,
               man lerne die Sprache durch einen Aufenthalt im fremden Lande von ſelbſt; und Zeit
               zum Studiren habe ich abſolut nicht. Zwei Eigenthümlichkeiten von \textcolor{pink}{Belgien}{}\ledrightnote{\textcolor{pink}{Belgien}} ſind mir beſonders ins Auge gefallen: es iſt ein \textcolor{pink}{Land}{}\ledrightnote{→\textcolor{pink}{Belgien}}, in dem es keine
               Zahnſtocher gibt, und in dem man die Thürklinken durch einen Druck von unten nach
               oben öffnet. Außerdem ſind die Kellner hier von einer unerhörten Unhöflichkeit und
               Schlamperei, und ich muß oft an Dich denken, der Du – nachdem Du mit Kellnern keinen
               Spaß verſtehſt – längſt einem dieſer Kerle ein Meſſer in den Leib geſtoßen haben
               würdeſt, hofſentlich gewinnen die Dinge ein freundlicheres Ausſehen für mich. Heut komme ich mir – wie nie vorher – vor wie in der
               Verbannung, und alle meine Wünſche regen ſich, um dieſen Brief zu begleiten in das
                  \label{K_L02663-999v}\edtext{trauliche, von Cigarettendampf
               erfüllte Zimmer}{\lemma{\textnormal{\emph{trauliche, … Zimmer}}}\Cendnote{\textnormal{hierbei dürfte es sich
                  um eine Beschreibung von \textcolor{blue}{Schnitzler}s Zimmer
                  handeln}}}\label{K_L02663-999h} mit dem Divan, in deſſen reichen und coquett geordneten Kiſſen es
               ſich ſo weich ruht und von dem man einen Ausblick hat auf das »\label{K_L02663-66v}\edtext{Pfühl}{\lemma{\textnormal{\emph{Pfühl}}}\Cendnote{\textnormal{österreichisch: Polster}}}\label{K_L02663-66h}« im \label{K_L02663-444v}\edtext{Alkoven}{\lemma{\textnormal{\emph{Alkoven}}}\Cendnote{\textnormal{Bettnische}}}\label{K_L02663-444h} und die \label{K_L02663-22v}\edtext{Landſchaft mit dem unglaublichen Mond}{\lemma{\textnormal{\emph{Landſchaft … Mond}}}\Cendnote{\textnormal{Sofern hier ein Bildobjekt (Gemälde, Stich, …) alludiert wird, so ist nicht klar,
                  welches gemeint ist.}}}\label{K_L02663-22h} darüber{\dots} Gott grüße Dich,
               mein lieber kleiner Arthur! Ich umarme Dich in alter Freundſchaft und drücke Dir
               beide Hände dazu.\pend
           \pstart Dein treuer \spacefill\mbox{Paul Goldmann.}\pend{}\pstart
           \noindent{}Sobald ich eine Adreſſe habe, theile ich ſie Dir mit{\dots}\pend
           \pstart
           Empfiehl’ mich den Deinen! Die Meinigen haben Dich {\pb}mehreremale grüßen laſſen, aber ich habe immer
                  vergeſſen, Dir’s zu ſchreiben{\dots}{ }\textsc{\begin{otherlanguage}{french}À propos\end{otherlanguage}}: wenn Du Herauskriegen könnteſt, warum mir der Schurke, der \label{K_L02663-99v}\edtext{\textsc{\textcolor{blue}{Beer-Hoffmann}{}\ledrightnote{\textcolor{blue}{Richard Beer-Hofmann}}}, nicht ſchreibt}{\lemma{\textnormal{\emph{Beer-Hoffmann, nicht ſchreibt}}}\Cendnote{\textnormal{Der erste
                     überlieferte Brief \textcolor{blue}{Goldmann}s an \textcolor{blue}{Beer-Hofmann} ist vom
                        10. 4. 1891, danach ist eine Lücke bis zum November des Jahres. (\emph{Houghton Library}, MS Ger 183, Box 4.)}}}\label{K_L02663-99h}
                  wäre ich Dir ſehr dankbar.\pend
           \endnumbering\briefempfaengerindex{Schnitzler, Arthur@\textsc{Schnitzler, Arthur}!zzzGoldmann, Paul@\emph{von Paul Goldmann}!1891-05-161@{16. 5. 1891}|)be}\mylabel{h}  \normalsize

\doendnotes{C}
\bigskip
\vfill

\clearpage

\footnotesize

\lohead{\textsc{register}}

% Definiere theindex-Environment komplett neu ohne reledmac
\makeatletter
\renewenvironment{theindex}{%
  \section*{\indexname}%
  \setlength{\parindent}{0pt}%
  \setlength{\parskip}{0pt plus 0.3pt}%
  \let\item\@idxitem
}{%
  \clearpage
}
\makeatother

\IfFileExists{\jobname-pw.ind}{\input{\jobname-pw.ind}}{}

\end{document}

      