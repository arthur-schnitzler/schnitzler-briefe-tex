%% latex-korrekturansicht-vorspann.tex
%% Vorspann für die Korrekturansicht.
%% Lädt die gemeinsame Datei latex-vorspann.tex mit gesetztem Schalter.

\newif\ifkorrekturansicht
\korrekturansichttrue

\input{../tex-inputs/latex-vorspann}


               \section[Arthur Schnitzler an Hugo von Hofmannsthal, 29. 9. 1899]{ Arthur Schnitzler an Hugo von Hofmannsthal, 29. 9. 1899}\nopagebreak\mylabel{v}\rehead{ }\normalsize\beginnumbering\briefempfaengerindex{Hofmannsthal, Hugo von@\textsc{Hofmannsthal, Hugo von}!zzzSchnitzler, Arthur@\emph{von Arthur Schnitzler}!1899-09-292@{29. 9. 1899}|(be} \toendnotes[C]{\smallbreak\pagebreak[2]} \Standort{FDH, Hs-30885,87.}
\physDesc{Brief, 1 Blatt, 4 Seiten
\newline{}Handschrift: Bleistift, deutsche Kurrent\newline{}Ordnung: mit Bleistift von Schnitzler mutmaßlich bei der
                                 Durchsicht der Korrespondenz 1929 Ergänzung der Jahreszahl
                                 »99« sowie des Ortes »\textcolor{pink}{\textsc{Wiesbaden}}« }\buchAbdrucke{\weitereDrucke{Hugo von Hofmannsthal, Arthur Schnitzler: \emph{Briefwechsel}. Hg. Therese Nickl und Heinrich Schnitzler. Frankfurt am Main: \emph{S. Fischer} 1964, S. 131.} }\toendnotes[C]{\smallbreak}\pstart
           \raggedleft{}{\pb}Freitag 29. 9.\pend
           \pstart
           mein lieber Hugo, das geht ſchon ſo mit den Stücken. Am leichteſten
               ſind ſie we{\geminationn}{ }ſie einem grad einfallen, – da ſind ſie beinah
               fertig. Über meines will ich nichts ſagen – mein Vertrauen wechſelt; das höchſte und
               wohl auch das höhere iſt mir nun einmal {\pb}verſagt; ich will
               für die Momente dankbar ſein, in denen ich eine gewiſſe innere Fülle empfinde.  –\pend
           \pstart
           Ich bleibe hier noch bis zum Dinſtag, fahre da{\geminationn} nach \textcolor{pink}{Berlin}{}\ledrightnote{\textcolor{pink}{Berlin}} (\textcolor{pink}{\textsc{Hotel Savoy}}{}\ledrightnote{\textcolor{pink}{Hotel Savoy}}, bitte ſchreiben Sie mir hin)\pend
           \pstart
           – Die paar Tage mit \textcolor{green}{\textsc{Beatrice}}{}\ledrightnote{\textcolor{green}{Der Schleier der Beatrice. Schauspiel in fünf Akten}}{ }{\pb}(\textcolor{pink}{München}{}\ledrightnote{\textcolor{pink}{München}}, \textcolor{pink}{Nürnberg}{}\ledrightnote{\textcolor{pink}{Nürnberg}}) waren ziemlich, ja ganz ungeſtört;
               eigentlich wirklich hübſch. Seit zehn Tagen hab ich erſt einmal, ganz flüchtig von
                  \textcolor{blue}{ihr}{}\ledrightnote{→\textcolor{blue}{Marie Reinhard}} gehört. – In \textcolor{pink}{Frankfurt}{}\ledrightnote{\textcolor{pink}{Frankfurt am Main}} freute ich mich \textcolor{blue}{Paul Goldm}{}\ledrightnote{\textcolor{blue}{Paul Goldmann}} in ſozuſagen glücklichrer Sti{\geminationm}ung zu ſehn als je. – Hier leb ich ganz allein, in
               einem ſchönen, angenehmen \textcolor{pink}{Hotel}{}\ledrightnote{→\textcolor{pink}{Hôtel du Parc & Bristol}},
               bin heut (i{\geminationm}er ſchlechtes Wetter) zum erſten Mal
               geradelt; arbeite nicht wenig; habe natürlich zuweilen Stunden von einer
               unbeſchreiblichen Traurigkeit. Ich glaube, ich werde immer mehr arbeiten, ſolang’s
               eben geht.\pend
           \pstart Von Herzen Ihr \spacefill\mbox{Arthur.}\pend{}\endnumbering\briefempfaengerindex{Hofmannsthal, Hugo von@\textsc{Hofmannsthal, Hugo von}!zzzSchnitzler, Arthur@\emph{von Arthur Schnitzler}!1899-09-292@{29. 9. 1899}|)be}\mylabel{h}  \normalsize

\doendnotes{C}
\bigskip
\vfill

\clearpage

\footnotesize

\lohead{\textsc{register}}

% Definiere theindex-Environment komplett neu ohne reledmac
\makeatletter
\renewenvironment{theindex}{%
  \section*{\indexname}%
  \setlength{\parindent}{0pt}%
  \setlength{\parskip}{0pt plus 0.3pt}%
  \let\item\@idxitem
}{%
  \clearpage
}
\makeatother

\IfFileExists{\jobname-pw.ind}{\input{\jobname-pw.ind}}{}

\end{document}

      