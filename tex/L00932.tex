%% latex-korrekturansicht-vorspann.tex
%% Vorspann für die Korrekturansicht.
%% Lädt die gemeinsame Datei latex-vorspann.tex mit gesetztem Schalter.

\newif\ifkorrekturansicht
\korrekturansichttrue

\input{../tex-inputs/latex-vorspann}


               \section[Richard Beer-Hofmann an Arthur Schnitzler, 3. 7. 1899]{ Richard Beer-Hofmann an Arthur Schnitzler, 3. 7. 1899}\nopagebreak\mylabel{v}\rehead{ }\normalsize\beginnumbering\briefempfaengerindex{Schnitzler, Arthur@\textsc{Schnitzler, Arthur}!zzzBeer-Hofmann, Richard@\emph{von Richard Beer-Hofmann}!1899-07-031@{3. 7. 1899}|(be} \toendnotes[C]{\smallbreak\pagebreak[2]} \Standort{CUL, Schnitzler, B 8.}
\physDesc{Bildpostkarte
\newline{}Handschrift: Bleistift, lateinische Kurrent\newline{}Versand: 1) Stempel: »\nobreak{}\oindex{Seeboden@\textbf{Seeboden}, \emph{http://www.geonames.org/ontologyA.ADM3}|pwk}\textcolor{gray}{S}eebod\textcolor{gray}{en}, 3 \textcolor{gray}{7 99}\nobreak{}«.  2) Stempel: »\nobreak{}\oindex{IX., Alsergrund@\textbf{IX., Alsergrund}, \emph{Bezirk (A.BZK)}|pwk}Wien 9/3 72, 4. 7. 99, 10.V, Bestellt\nobreak{}«. 
\newline{}Schnitzler: mit Bleistift mit Empfangsdatum versehen: »4/7 99« \newline{}Ordnung: mit Bleistift von unbekannter Hand nummeriert:
                                    »130« }\buchAbdrucke{\weitereDrucke{Arthur Schnitzler, Richard Beer-Hofmann: \emph{Briefwechsel 1891–1931}. Hg. Konstanze Fliedl. Wien, Zürich: \emph{Europaverlag} 1992, S. 131.} }\toendnotes[C]{\smallbreak}\pstart{}{\pb}Herrn\pend{}\pstart{}D\textsuperscript{r} Arthur Schnitzler\pend{}\pstart{}\textcolor{pink}{Wien}{}\ledrightnote{\textcolor{pink}{Wien}}\pend{}\pstart{}\textcolor{pink}{IX Frankgasse 1}{}\ledrightnote{\textcolor{pink}{Frankgasse}}\pend{}{\bigskip}\pstart
           \noindent{}\centering{}\textcolor{gray}{\textbf{{\pb}Gruss aus \textcolor{pink}{Seeboden}{}\ledrightnote{\textcolor{pink}{Seeboden}}.}}\pend
           \pstart
           Lieber Arthur! Im Vorhinein mit Ihren Plänen einverstanden. Bitte
               verständigen Sie auch gelegentlich \textcolor{blue}{Mayer}{}\ledrightnote{\textcolor{blue}{Oskar Mayer}}, dem ich
               übrigens auch schreiben werde. Das bewußte, heißt »\label{K_L00932_1v}\edtext{Sanitas}{\lemma{\textnormal{\emph{Sanitas}}}\Cendnote{\textnormal{In der
                  Werbung wurde Sanitas als »Das neue antiseptische desinficirende und
                     hygienische Mittel«, »Unentbehrlich für jeden Haushalt«
                  angepriesen.}}}\label{K_L00932_1h}« erhältlich in d. großen \textcolor{brown}{Papiergeschäft}{}\ledrightnote{→\textcolor{brown}{Joseph Lustig {\kaufmannsund} Co.}} am \textcolor{pink}{Hohen
                  Markt}{}\ledrightnote{\textcolor{pink}{Hoher Markt}}. Herzlichst \spacefill\mbox{R.}\pend
           \endnumbering\briefempfaengerindex{Schnitzler, Arthur@\textsc{Schnitzler, Arthur}!zzzBeer-Hofmann, Richard@\emph{von Richard Beer-Hofmann}!1899-07-031@{3. 7. 1899}|)be}\mylabel{h}  \normalsize

\doendnotes{C}
\bigskip
\vfill

\clearpage

\footnotesize

\lohead{\textsc{register}}

% Definiere theindex-Environment komplett neu ohne reledmac
\makeatletter
\renewenvironment{theindex}{%
  \section*{\indexname}%
  \setlength{\parindent}{0pt}%
  \setlength{\parskip}{0pt plus 0.3pt}%
  \let\item\@idxitem
}{%
  \clearpage
}
\makeatother

\IfFileExists{\jobname-pw.ind}{\input{\jobname-pw.ind}}{}

\end{document}

      