%% latex-korrekturansicht-vorspann.tex
%% Vorspann für die Korrekturansicht.
%% Lädt die gemeinsame Datei latex-vorspann.tex mit gesetztem Schalter.

\newif\ifkorrekturansicht
\korrekturansichttrue

\input{../tex-inputs/latex-vorspann}


               \section[Karl Kraus an Arthur Schnitzler, 31. 10. 1892]{ Karl Kraus an Arthur Schnitzler, 31. 10. 1892}\nopagebreak\mylabel{v}\rehead{ }\normalsize\beginnumbering\briefempfaengerindex{Schnitzler, Arthur@\textsc{Schnitzler, Arthur}!zzzKraus, Karl@\emph{von Karl Kraus}!1892-10-311@{31. 10. 1892}|(be} \toendnotes[C]{\smallbreak\pagebreak[2]} \Standort{CUL, Schnitzler, B 55.}
\physDesc{Brief, 1 Blatt, 3 Seiten
\newline{}Handschrift: schwarze Tinte, deutsche Kurrent
\newline{}Schnitzler: mit Bleistift beschriftet: »\textsc{Karl Kraus}« }\buchAbdrucke{\weitereDrucke{\emph{Karl Kraus und Arthur Schnitzler. Eine Dokumentation.} Hg. Reinhard Urbach. In: \emph{Literatur und Kritik}, Bd. 49, Oktober 1970, S. 513.} }\toendnotes[C]{\smallbreak}\pstart
           {\pb}am 31. Oktober
                  1892.\pend
           \pstart\center{}Sehr verehrter Herr Doctor!\pend\pstart
           Herzlichſten und aufrichtigſten Dank für die Überſendung Ihres \textcolor{green}{Buches}{}\ledrightnote{→\textcolor{green}{Anatol}} und für die liebenswürdige Widmung!\pend
           \pstart
           Sie können ſich vorſtellen, \uline{wie} ich mich damit
               gefreut habe. Das iſt ja ein prächtiges \textcolor{green}{Buch}{}\ledrightnote{→\textcolor{green}{Anatol}}! und der \textcolor{green}{Prolog}{}\ledrightnote{→\textcolor{green}{Prolog [zum Anatol]}} von \textcolor{blue}{Loris}{}\ledrightnote{\textcolor{blue}{Hugo von Hofmannsthal}} iſt
               ſehr herzig. Aber ich bezahle Sie mit Undank. Denn – denken Sie ſich nur nur: ich –
               will – {\pb}eine – \textcolor{green}{Kritik}{}\ledrightnote{→\textcolor{green}{Arthur Schnitzler, Anatol}} – drüber ſchreiben!! Nun ja, wenn ein
               Buch einmal \uline{in meine Klauen} kommt!\pend
           \pstart
           U. zw. entweder »\textcolor{green}{\uline{Geſellſchaft}}{}\ledrightnote{\textcolor{green}{Die Gesellschaft. Monatsschrift}}« (\label{K_L00130_1v}\edtext{Dezemberheft}{\lemma{\textnormal{\emph{Dezemberheft}}}\Cendnote{\textnormal{Die Rezension erschien erst im ersten Heft
                  des neuen Jahres (\textcolor{blue}{Karl Kraus}: \emph{\textcolor{green}{Arthur Schnitzler, Anatol}}. In: \emph{\textcolor{green}{Die
                        Gesellschaft}}, Jg. 9, H. 1, 1. 1. 1893, S. 109–110).
                  Die Verschiebung auf das Januarheft könnte dadurch verursacht sein, dass im
                  Dezember bereits zwei Rezensionen von \textcolor{blue}{Kraus}
                  erschienen.}}}\label{K_L00130_1h}) oder »\textcolor{brown}{\uline{W\textsuperscript{r.} Allgemeine}}{}\ledrightnote{\textcolor{brown}{Wiener Allgemeine Zeitung}}« – oder Feuilleton mit anderen Sachen.\pend
           \pstart
           Auguſtheft der »\textcolor{green}{\uline{Geſellſchaft}}{}\ledrightnote{\textcolor{green}{Die Gesellschaft. Monatsschrift}} (\textcolor{green}{\textcolor{pink}{Burgtheater}{}\ledrightnote{\textcolor{pink}{Burgtheater}}aufsatz}{}\ledrightnote{→\textcolor{green}{Das Burgtheater und die letzte Saison}}) bekam ich unlängſt zurück
               und ſende Ihnen noch heute. Er iſt leider in nicht ſehr salonfähigem Zuſtand, und
               leider – mein \uline{einziges Exemplar!}\pend
           \pstart
           {\pb}Ich hab’ Sie (von weitem allerdings) bei
               der \label{K_L00130_2v}\edtext{Premiere}{\lemma{\textnormal{\emph{Premiere}}}\Cendnote{\textnormal{am 29. 10. 1892 im \textcolor{pink}{Deutschen Volkstheater}; ein Besuch \textcolor{blue}{Schnitzler}s ist nicht in seinem \emph{\textcolor{green}{Tagebuch}} verzeichnet.}}}\label{K_L00130_2h} der »\textcolor{green}{Orientreise}{}\ledrightnote{\textcolor{green}{Die Orientreise}}« gesehn. Nun, \uline{das} iſt doch ein Schund? \uline{Wie} hat es \uline{Ihnen} ge- resp. missfallen?\pend
           \pstart
           Ach, nochmals ergebenſt Dank für Ihre Liebenswürdigkeit und schönſten Gruß\pend
           \pstart
           von Ihrem{\\[\baselineskip]}hochachtungsvollen{\\[\baselineskip]}\spacefill\mbox{Karl Kraus}\pend
           \leftskip=0em{}\pstart
           \noindent{}\textcolor{pink}{I. Maximilianstr. 13\textsuperscript{I.}}{}\ledrightnote{\textcolor{pink}{Mahlerstraße}}\pend
           \endnumbering\briefempfaengerindex{Schnitzler, Arthur@\textsc{Schnitzler, Arthur}!zzzKraus, Karl@\emph{von Karl Kraus}!1892-10-311@{31. 10. 1892}|)be}\mylabel{h}  \normalsize

\doendnotes{C}
\bigskip
\vfill

\clearpage

\footnotesize

\lohead{\textsc{register}}

% Definiere theindex-Environment komplett neu ohne reledmac
\makeatletter
\renewenvironment{theindex}{%
  \section*{\indexname}%
  \setlength{\parindent}{0pt}%
  \setlength{\parskip}{0pt plus 0.3pt}%
  \let\item\@idxitem
}{%
  \clearpage
}
\makeatother

\IfFileExists{\jobname-pw.ind}{\input{\jobname-pw.ind}}{}

\end{document}

      