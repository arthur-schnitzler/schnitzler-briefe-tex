%% latex-korrekturansicht-vorspann.tex
%% Vorspann für die Korrekturansicht.
%% Lädt die gemeinsame Datei latex-vorspann.tex mit gesetztem Schalter.

\newif\ifkorrekturansicht
\korrekturansichttrue

\input{../tex-inputs/latex-vorspann}


               \section[Hugo von Hofmannsthal an Arthur Schnitzler, 16. 12. 1904]{ Hugo von Hofmannsthal an Arthur Schnitzler, 16. 12. 1904}\nopagebreak\mylabel{v}\rehead{ }\normalsize\beginnumbering\briefempfaengerindex{Schnitzler, Arthur@\textsc{Schnitzler, Arthur}!zzzHofmannsthal, Hugo von@\emph{von Hugo von Hofmannsthal}!1904-12-161@{16. 12. 1904}|(be} \toendnotes[C]{\smallbreak\pagebreak[2]} \Standort{CUL, Schnitzler, B 43.}
\physDesc{Postkarte
\newline{}Handschrift: schwarze Tinte, lateinische Kurrent\newline{}Versand: 1) Stempel: »\nobreak{}\oindex{Rodaun@\textbf{Rodaun}, \emph{Teil eines besiedelten Ortes (A.BSOX)}|pwk}Rodaun, 16 12 04, \textcolor{gray}{6}N\nobreak{}«.  2) Stempel: »\nobreak{}\oindex{XVIII., Waehring@\textbf{XVIII., Währing}, \emph{Bezirk (A.BZK)}|pwk}18/2 Wien 113, 17. 12. \textcolor{gray}{0}4, Bestellt\nobreak{}«. 3) mit Tinte von unbekannter Hand die Bezirksnummer um den
                                 Postrayon erweitert: »/1«, was im Zusammenhang mit
                                 dem Empfangsstempel vom Postrayon 18/2 stehen dürfte
\newline{}Schnitzler: mit Bleistift datiert: »17/12 904« \newline{}Ordnung: 1) mit Bleistift von unbekannter Hand nummeriert:
                                    »219« 2) mit Bleistift von unbekannter Hand nummeriert:
                                    »244«}\buchAbdrucke{\weitereDrucke{Hugo von Hofmannsthal, Arthur Schnitzler: \emph{Briefwechsel}. Hg. Therese Nickl und Heinrich Schnitzler. Frankfurt am Main: \emph{S. Fischer} 1964, S. 208.} }\toendnotes[C]{\smallbreak}\pstart{}{\pb}Herrn D\textsuperscript{r} Arthur Schnitzler\pend{}\pstart{}\textcolor{pink}{Wien}{}\ledrightnote{\textcolor{pink}{Wien}}\pend{}\pstart{}\textcolor{pink}{XVIII Spöttelgasse 7}{}\ledrightnote{\textcolor{pink}{Edmund-Weiß-Gasse}}\pend{}{\bigskip}\pstart
           {\pb}Freitag.\pend
           \pstart
           Freuen uns auf \label{K_L01480_1v}\edtext{Mittwoch}{\lemma{\textnormal{\emph{Mittwoch}}}\Cendnote{\textnormal{vgl. A. S.: \emph{Tagebuch}, 21. 12. 1890}}}\label{K_L01480_1h}.\pend
           \pstart
           Wir beide möchten schon gegen ½ 7 ko{\geminationm}en,
                  \textcolor{blue}{Papa}{}\ledrightnote{→\textcolor{blue}{Hugo August von Hofmannsthal}} etwas später.\pend
           \pstart
           Herzlich{\\[\baselineskip]}\spacefill\mbox{Hugo}\pend
           \leftskip=0em{}\pstart
           \noindent{}\textcolor{blue}{Richard}{}\ledrightnote{\textcolor{blue}{Richard Beer-Hofmann}} ist \textcolor{pink}{dort}{}\ledrightnote{\textcolor{pink}{Berlin}}. Herzzerreißende \textcolor{green}{Première}{}\ledrightnote{→\textcolor{green}{Der Graf von Charolais. Ein Trauerspiel}} soll 23\textsuperscript{ten} sein. \textcolor{blue}{Höflich}{}\ledrightnote{\textcolor{blue}{Lucie Höflich}} und \textcolor{blue}{Sorma}{}\ledrightnote{\textcolor{blue}{Agnes Sorma}} hat er schon nahezu umgebracht.\pend
           \endnumbering\briefempfaengerindex{Schnitzler, Arthur@\textsc{Schnitzler, Arthur}!zzzHofmannsthal, Hugo von@\emph{von Hugo von Hofmannsthal}!1904-12-161@{16. 12. 1904}|)be}\mylabel{h}  \normalsize

\doendnotes{C}
\bigskip
\vfill

\clearpage

\footnotesize

\lohead{\textsc{register}}

% Definiere theindex-Environment komplett neu ohne reledmac
\makeatletter
\renewenvironment{theindex}{%
  \section*{\indexname}%
  \setlength{\parindent}{0pt}%
  \setlength{\parskip}{0pt plus 0.3pt}%
  \let\item\@idxitem
}{%
  \clearpage
}
\makeatother

\IfFileExists{\jobname-pw.ind}{\input{\jobname-pw.ind}}{}

\end{document}

      