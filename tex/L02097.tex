%% latex-korrekturansicht-vorspann.tex
%% Vorspann für die Korrekturansicht.
%% Lädt die gemeinsame Datei latex-vorspann.tex mit gesetztem Schalter.

\newif\ifkorrekturansicht
\korrekturansichttrue

\input{../tex-inputs/latex-vorspann}


               \section[Lou Andreas-Salomé an Arthur Schnitzler, {[}vor dem 15. 11. 1912?{]}]{ Lou Andreas-Salomé an Arthur Schnitzler, {[}vor dem
                    15. 11. 1912?{]}}\nopagebreak\mylabel{v}\rehead{ }\normalsize\beginnumbering\briefempfaengerindex{Schnitzler, Arthur@\textsc{Schnitzler, Arthur}!zzzAndreas-Salome, Lou@\emph{von Lou Andreas-Salomé}!1912-11-141@{{[}vor dem 15. 11. 1912?{]}}|(be} \toendnotes[C]{\smallbreak\pagebreak[2]} \Standort{CUL, Schnitzler, B 3.}
\physDesc{Briefkarte
\newline{}Handschrift: schwarze Tinte, deutsche Kurrent
\newline{}Schnitzler: 1) mit Bleistift beschriftet: »\substVorne{}\textsuperscript{A}\substDazwischen{}L\substHinten{}ou Salom\textcolor{gray}{e} Andr.« 2) mit rotem Buntstift eine Unterstreichung}\toendnotes[C]{\smallbreak}\pstart{}{\pb}Lieber Herr Doktor,\pend\pstart
           ich bin mit einem jungen \label{K_L02097_1v}\edtext{\textcolor{blue}{Mädchen}{}\ledrightnote{→\textcolor{blue}{Ellen Delp}}}{\lemma{\textnormal{\emph{Mädchen}}}\Cendnote{\textnormal{Die Datierung basiert auf der Annahme,
                        dass es sich bei ihr um \textcolor{blue}{Ellen Delp}
                        handelt. Vgl. A. S.: \emph{Tagebuch}, 15. 11. 1912.}}}\label{K_L02097_1h} hier in \textcolor{pink}{Wien}{}\ledrightnote{\textcolor{pink}{Wien}}, und würde mich
                    freuen, wenn wir Ihnen einen Beſuch abſtatten dürften, falls dies keine Störung
                    für Sie bedeutet? Dienstag oder Donnerstag oder
                        Freitag brauchten Sie nur die Stunde zu beſtimmen, wir halten
                    uns frei.\pend
           \pstart
           Mit herzlichem Gruß Ihre{\\[\baselineskip]}\spacefill\mbox{Lou Andreas-Salomé}\pend
           \leftskip=0em{}\endnumbering\briefempfaengerindex{Schnitzler, Arthur@\textsc{Schnitzler, Arthur}!zzzAndreas-Salome, Lou@\emph{von Lou Andreas-Salomé}!1912-11-141@{{[}vor dem 15. 11. 1912?{]}}|)be}\mylabel{h}  \normalsize

\doendnotes{C}
\bigskip
\vfill

\clearpage

\footnotesize

\lohead{\textsc{register}}

% Definiere theindex-Environment komplett neu ohne reledmac
\makeatletter
\renewenvironment{theindex}{%
  \section*{\indexname}%
  \setlength{\parindent}{0pt}%
  \setlength{\parskip}{0pt plus 0.3pt}%
  \let\item\@idxitem
}{%
  \clearpage
}
\makeatother

\IfFileExists{\jobname-pw.ind}{\input{\jobname-pw.ind}}{}

\end{document}

      