%% latex-korrekturansicht-vorspann.tex
%% Vorspann für die Korrekturansicht.
%% Lädt die gemeinsame Datei latex-vorspann.tex mit gesetztem Schalter.

\newif\ifkorrekturansicht
\korrekturansichttrue

\input{../tex-inputs/latex-vorspann}


               \section[Arthur Schnitzler an Richard Beer-Hofmann, 18. 7. 1909]{ Arthur Schnitzler an Richard Beer-Hofmann, 18. 7. 1909}\nopagebreak\mylabel{v}\rehead{ }\normalsize\beginnumbering\briefempfaengerindex{Beer-Hofmann, Richard@\textsc{Beer-Hofmann, Richard}!zzzSchnitzler, Arthur@\emph{von Arthur Schnitzler}!1909-07-181@{18. 7. 1909}|(be} \toendnotes[C]{\smallbreak\pagebreak[2]} \Standort{YCGL, MSS 31.}
\physDesc{Bildpostkarte
\newline{}Handschrift: Bleistift, deutsche Kurrent\newline{}Versand: Stempel: »\nobreak{}\oindex{Edlach@\textbf{Edlach}, \emph{Besiedelter Ort (A.BSO)}|pwk}Edlach {[}bei
                                                  Reic{]}\textcolor{gray}{henau N.Ö.}, \textcolor{gray}{1}9 7 09, 8\nobreak{}«.  \newline{}Ordnung: mit Bleistift von unbekannter Hand datiert: »19. 7.« }\pstart{}{\pb}Hrn \textsc{Dr. Richard
                            Beer-Hofmann}\pend{}\pstart{}\textsc{\textcolor{pink}{Wien XVIII}{}\ledrightnote{\textcolor{pink}{XVIII., Währing}}}\pend{}\pstart{}\textsc{\textcolor{pink}{Hasenauerstr 59}{}\ledrightnote{\textcolor{pink}{Hasenauerstraße}}}\pend{}\pstart{}\textsc{(nachzusenden)}\pend{}{\bigskip}\pstart
           \noindent{}\centering{}{\pb}\textcolor{gray}{\textbf{\textcolor{pink}{Hôtel Edlacherhof.}{}\ledrightnote{\textcolor{pink}{Hotel Edlacherhof}}}}\pend
           \pstart
           \noindent{}\centering{}\textcolor{gray}{\textbf{\textcolor{pink}{Edlach bei Reichenau, N.-Oe.}{}\ledrightnote{\textcolor{pink}{Edlach}},
                            520 m.}}\pend
           \pstart
           \raggedleft{}{\pb}18. \textcolor{gray}{7. 09}\pend
           \pstart
           lieber Richard, ſtatt in \textcolor{pink}{Wien}{}\ledrightnote{\textcolor{pink}{Wien}}
                    ſein Sie vom 1.–15. 8. lieber in \textcolor{pink}{Edlach}{}\ledrightnote{\textcolor{pink}{Edlach}}. \textcolor{blue}{Heini}{}\ledrightnote{\textcolor{blue}{Heinrich Schnitzler}} ko{\geminationm}t erſt übermorgen zu uns heraus. Wie gehts Ihnen
                    Allen?\pend
           \pstart
           Herzlichſt{\\[\baselineskip]}Ihr \spacefill\mbox{Arthur}\pend
           \leftskip=0em{}\endnumbering\briefempfaengerindex{Beer-Hofmann, Richard@\textsc{Beer-Hofmann, Richard}!zzzSchnitzler, Arthur@\emph{von Arthur Schnitzler}!1909-07-181@{18. 7. 1909}|)be}\mylabel{h}  \normalsize

\doendnotes{C}
\bigskip
\vfill

\clearpage

\footnotesize

\lohead{\textsc{register}}

% Definiere theindex-Environment komplett neu ohne reledmac
\makeatletter
\renewenvironment{theindex}{%
  \section*{\indexname}%
  \setlength{\parindent}{0pt}%
  \setlength{\parskip}{0pt plus 0.3pt}%
  \let\item\@idxitem
}{%
  \clearpage
}
\makeatother

\IfFileExists{\jobname-pw.ind}{\input{\jobname-pw.ind}}{}

\end{document}

      