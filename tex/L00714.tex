%% latex-korrekturansicht-vorspann.tex
%% Vorspann für die Korrekturansicht.
%% Lädt die gemeinsame Datei latex-vorspann.tex mit gesetztem Schalter.

\newif\ifkorrekturansicht
\korrekturansichttrue

\input{../tex-inputs/latex-vorspann}


               \section[Richard Beer-Hofmann an Arthur Schnitzler, 7. 8. 1897]{ Richard Beer-Hofmann an Arthur Schnitzler, 7. 8. 1897}\nopagebreak\mylabel{v}\rehead{ }\normalsize\beginnumbering\briefempfaengerindex{Schnitzler, Arthur@\textsc{Schnitzler, Arthur}!zzzBeer-Hofmann, Richard@\emph{von Richard Beer-Hofmann}!1897-08-071@{7. 8. 1897}|(be} \toendnotes[C]{\smallbreak\pagebreak[2]} \Standort{CUL, Schnitzler, B 8.}
\physDesc{Telegramm
\newline{}maschinell\newline{}Versand: 1) Stempel: »\nobreak{}\oindex{Bad Ischl@\textbf{Bad Ischl}, \emph{Besiedelter Ort (A.BSO)}|pwk}Ischl, 7 Aug. 1897, Suchý.\nobreak{}«.  2) Stempel: »\nobreak{}Ausgefertigt, 7 Aug., 12\textsuperscript{37}\nobreak{}«. 3) Stempel: »\nobreak{}\oindex{IX., Alsergrund@\textbf{IX., Alsergrund}, \emph{Bezirk (A.BZK)}|pwk}Wien 9/3, 7 VIII 97, 1 30N\nobreak{}«. \newline{}Ordnung: 1) beschnitten 2) mit Bleistift von unbekannter Hand nummeriert:
                                    »103«}\buchAbdrucke{\weitereDrucke{Arthur Schnitzler, Richard Beer-Hofmann: \emph{Briefwechsel 1891–1931}. Hg. Konstanze Fliedl. Wien, Zürich: \emph{Europaverlag} 1992, S. 112.} }\pstart{}{\pb}arthur schnitzler \textcolor{pink}{win 9}{}\ledrightnote{\textcolor{pink}{IX., Alsergrund}}{ }\textcolor{pink}{frankg 1}{}\ledrightnote{\textcolor{pink}{Frankgasse}}\pend{}{\bigskip}\pstart
           \noindent{}wfr \textcolor{pink}{ischl}{}\ledrightnote{\textcolor{pink}{Bad Ischl}} 3549 27 7{ }10/38+\pend
           \pstart
           konte gestern fluechtyg sprechn kain bsondrer grund langewejle wahrschainlych ich
               rejse morgn \textcolor{pink}{salzburg}{}\ledrightnote{\textcolor{pink}{Salzburg}} bin mittwoch oder
                  donerstag in \textcolor{pink}{win}{}\ledrightnote{\textcolor{pink}{Wien}} herzlychst
               rychard=\pend
           \endnumbering\briefempfaengerindex{Schnitzler, Arthur@\textsc{Schnitzler, Arthur}!zzzBeer-Hofmann, Richard@\emph{von Richard Beer-Hofmann}!1897-08-071@{7. 8. 1897}|)be}\mylabel{h}  \normalsize

\doendnotes{C}
\bigskip
\vfill

\clearpage

\footnotesize

\lohead{\textsc{register}}

% Definiere theindex-Environment komplett neu ohne reledmac
\makeatletter
\renewenvironment{theindex}{%
  \section*{\indexname}%
  \setlength{\parindent}{0pt}%
  \setlength{\parskip}{0pt plus 0.3pt}%
  \let\item\@idxitem
}{%
  \clearpage
}
\makeatother

\IfFileExists{\jobname-pw.ind}{\input{\jobname-pw.ind}}{}

\end{document}

      