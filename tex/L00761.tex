%% latex-korrekturansicht-vorspann.tex
%% Vorspann für die Korrekturansicht.
%% Lädt die gemeinsame Datei latex-vorspann.tex mit gesetztem Schalter.

\newif\ifkorrekturansicht
\korrekturansichttrue

\input{../tex-inputs/latex-vorspann}


               \section[Hugo von Hofmannsthal an Arthur Schnitzler, {[}10.? 1. 1898{]}]{ Hugo von Hofmannsthal an Arthur Schnitzler, {[}10.? 1. 1898{]}}\nopagebreak\mylabel{v}\rehead{ }\normalsize\beginnumbering\briefempfaengerindex{Schnitzler, Arthur@\textsc{Schnitzler, Arthur}!zzzHofmannsthal, Hugo von@\emph{von Hugo von Hofmannsthal}!1898-01-101@{{[}10.? 1. 1898{]}}|(be} \toendnotes[C]{\smallbreak\pagebreak[2]} \Standort{CUL, Schnitzler, B 43.}
\physDesc{Briefkarte
\newline{}Handschrift: schwarze Tinte, deutsche Kurrent
\newline{}Schnitzler: mit Bleistift datiert: »? Jann 98« \newline{}Ordnung: mit Bleistift von unbekannter Hand nummeriert:
                                    »104« }\buchAbdrucke{\weitereDrucke{Hugo von Hofmannsthal, Arthur Schnitzler: \emph{Briefwechsel}. Hg. Therese Nickl und Heinrich Schnitzler. Frankfurt am Main: \emph{S. Fischer} 1964, S. 98.} }\toendnotes[C]{\smallbreak}\pstart
           \raggedleft{}{\pb}\label{K_L00761_1v}\edtext{Montag}{\lemma{\textnormal{\emph{Montag}}}\Cendnote{\textnormal{Am 5. 1. 1898 wiederholt \textcolor{blue}{Brahm} in einem Brief an \textcolor{blue}{Schnitzler}, dass er \emph{\textcolor{green}{Der Kaiser und
                           Hexe}} für misslungen halte. Er hatte sich also seine Meinung
                        gebildet, wenngleich sich das so lesen lässt, dass diese noch nicht
                        kommuniziert war. Entsprechend könnte der Brief am darauffolgenden Montag
                        geschrieben sein.}}}\label{K_L00761_1h}\pend
           \pstart{}mein lieber Arthur,\pend\pstart
           »\textcolor{green}{Kaiſer und Hexe}{}\ledrightnote{\textcolor{green}{Der Kaiser und die Hexe}}« gefällt \textcolor{blue}{Brahm}{}\ledrightnote{\textcolor{blue}{Otto Brahm}} nicht ſehr (offenbar) und er wird es \uline{nicht}{ }ſpielen.\pend
           \pstart
           Die künftigen Beziehungen der \textcolor{blue}{\textsc{Sorma}}{}\ledrightnote{\textcolor{blue}{Agnes Sorma}} zum »\textcolor{pink}{Deutſchen Theater}{}\ledrightnote{\textcolor{pink}{Deutsches Theater Berlin}}« ſind ſehr unſicher; er
               denkt {\pb}alſo daran, die beiden
               anderen \textcolor{green}{Stücke}{}\ledrightnote{→\textcolor{green}{Die Frau im Fenster}{\newline}→\textcolor{green}{Die Hochzeit der Sobeide}} oder nur
               die »\textcolor{green}{junge Frau}{}\ledrightnote{\textcolor{green}{Die Frau im Fenster}}« mit einem (fremden) Einacter
               heuer, ohne die \textcolor{blue}{\textsc{Sorma}}{}\ledrightnote{\textcolor{blue}{Agnes Sorma}}, zu ſpielen etc{\dots} lauter unangenehme Sachen, worüber
               weiter nichts zu reden. Morgen{ }abend bin \uline{leider} nicht frei.\pend
           \pstart Ihr\spacefill\mbox{Hugo.}\pend{}\endnumbering\briefempfaengerindex{Schnitzler, Arthur@\textsc{Schnitzler, Arthur}!zzzHofmannsthal, Hugo von@\emph{von Hugo von Hofmannsthal}!1898-01-101@{{[}10.? 1. 1898{]}}|)be}\mylabel{h}  \normalsize

\doendnotes{C}
\bigskip
\vfill

\clearpage

\footnotesize

\lohead{\textsc{register}}

% Definiere theindex-Environment komplett neu ohne reledmac
\makeatletter
\renewenvironment{theindex}{%
  \section*{\indexname}%
  \setlength{\parindent}{0pt}%
  \setlength{\parskip}{0pt plus 0.3pt}%
  \let\item\@idxitem
}{%
  \clearpage
}
\makeatother

\IfFileExists{\jobname-pw.ind}{\input{\jobname-pw.ind}}{}

\end{document}

      