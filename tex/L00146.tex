%% latex-korrekturansicht-vorspann.tex
%% Vorspann für die Korrekturansicht.
%% Lädt die gemeinsame Datei latex-vorspann.tex mit gesetztem Schalter.

\newif\ifkorrekturansicht
\korrekturansichttrue

\input{../tex-inputs/latex-vorspann}


               \section[Arthur Schnitzler an Richard Beer-Hofmann, 27. 12. 1892]{ Arthur Schnitzler an Richard Beer-Hofmann, 27. 12. 1892}\nopagebreak\mylabel{v}\rehead{ }\normalsize\beginnumbering\briefempfaengerindex{Beer-Hofmann, Richard@\textsc{Beer-Hofmann, Richard}!zzzSchnitzler, Arthur@\emph{von Arthur Schnitzler}!1892-12-271@{27. 12. 1892}|(be} \toendnotes[C]{\smallbreak\pagebreak[2]} \Standort{YCGL, MSS 31.}
\physDesc{Brief, 1 Blatt (Die Innenseite des Doppelblatts ist über die ganze Breite beschrieben), 2 Seiten, Umschlag
\newline{}Handschrift: Bleistift, deutsche Kurrent\newline{}Versand: 1) Stempel: »\nobreak{}Wien 9/{[}3{]}, 27. 12 92, 4–5\nobreak{}«.  2) Stempel: »\nobreak{}Wien 1{[}/1{]}, 28{[}.{]} 12. 92, 8–9½ V\nobreak{}«. }\buchAbdrucke{\weitereDrucke{Arthur Schnitzler, Richard Beer-Hofmann: \emph{Briefwechsel 1891–1931}. Hg. Konstanze Fliedl. Wien, Zürich: \emph{Europaverlag} 1992, S. 41.} }\pstart{}{\pb}\textsc{Herrn Doctor Richard Beer-Hofmann}\pend{}\pstart{}\textsc{\textcolor{pink}{Wien}{}\ledrightnote{\textcolor{pink}{Wien}}}\pend{}\pstart{}\textcolor{pink}{I Wollzeile 15}{}\ledrightnote{\textcolor{pink}{Wollzeile}}.\pend{}{\bigskip}\pstart
           \raggedleft{}{\pb}27/12 92\pend
           \pstart{}Lieber Richard,\pend\pstart
           hier der Sitz, leider nur mehr à \substVorne{}\textsuperscript{\textcolor{gray}{3}}\substDazwischen{}2\substHinten{} fl erhältlich\pend
           \pstart
           Herzlich Ihr{\\[\baselineskip]}\spacefill\mbox{Arthur}\pend
           \leftskip=0em{}\pstart
           \noindent{}\textcolor{blue}{\textsc{Horn}}{}\ledrightnote{\textcolor{blue}{Paul Horn}}{ }ſchreibt mir,
                     {\pb}daſs er Donnerſtag erſt um
                     8 zu Frau \textcolor{blue}{\textsc{Flegma{\geminationn}}}{}\ledrightnote{\textcolor{blue}{Bertha Flegmann}} ko{\geminationm}t, Gott wie mies iſt mir! Bitte ko{\geminationm}en Sie ſicher, vielleicht gelingt uns das Mislingen
                  (Wortſpiel).\pend
           \endnumbering\briefempfaengerindex{Beer-Hofmann, Richard@\textsc{Beer-Hofmann, Richard}!zzzSchnitzler, Arthur@\emph{von Arthur Schnitzler}!1892-12-271@{27. 12. 1892}|)be}\mylabel{h}  \normalsize

\doendnotes{C}
\bigskip
\vfill

\clearpage

\footnotesize

\lohead{\textsc{register}}

% Definiere theindex-Environment komplett neu ohne reledmac
\makeatletter
\renewenvironment{theindex}{%
  \section*{\indexname}%
  \setlength{\parindent}{0pt}%
  \setlength{\parskip}{0pt plus 0.3pt}%
  \let\item\@idxitem
}{%
  \clearpage
}
\makeatother

\IfFileExists{\jobname-pw.ind}{\input{\jobname-pw.ind}}{}

\end{document}

      