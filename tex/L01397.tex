%% latex-korrekturansicht-vorspann.tex
%% Vorspann für die Korrekturansicht.
%% Lädt die gemeinsame Datei latex-vorspann.tex mit gesetztem Schalter.

\newif\ifkorrekturansicht
\korrekturansichttrue

\input{../tex-inputs/latex-vorspann}


               \section[Hermann Bahr an Arthur Schnitzler, 28. 4. 1904]{ Hermann Bahr an Arthur Schnitzler, 28. 4. 1904}\nopagebreak\mylabel{v}\rehead{ }\normalsize\beginnumbering\briefempfaengerindex{Schnitzler, Arthur@\textsc{Schnitzler, Arthur}!zzzBahr, Hermann@\emph{von Hermann Bahr}!1904-04-281@{28. 4. 1904}|(be} \toendnotes[C]{\smallbreak\pagebreak[2]} \Standort{CUL, Schnitzler, B 5b.}
\physDesc{Postkarte
\newline{}Handschrift: schwarze Tinte, deutsche Kurrent\newline{}Versand: 1) Rohrpost 2) Stempel: »\nobreak{}\oindex{XIII., Hietzing@\textbf{XIII., Hietzing}, \emph{Bezirk (A.BZK)}|pwk}Wien 13/5, 28. IV. 04\nobreak{}«. 3) Stempel: »\nobreak{}28. IV. 04\nobreak{}«. 4) Stempel: »\nobreak{}\oindex{XVIII., Waehring@\textbf{XVIII., Währing}, \emph{Bezirk (A.BZK)}|pwk}Wien 18, 4.10N\nobreak{}«. \newline{}Ordnung: mit Bleistift von unbekannter Hand nummeriert: »116« }\buchAbdrucke{\weitereDrucke{Hermann Bahr, Arthur Schnitzler: \emph{Briefwechsel, Aufzeichnungen, Dokumente (1891–1931)}. Hg. Kurt Ifkovits und Martin Anton Müller. Göttingen: \emph{Wallstein} 2018, S. 306–307.} }\toendnotes[C]{\smallbreak}\pstart{}{\pb}Pneumatisch\pend{}\pstart{}Herrn \textsc{D\textsuperscript{r} Arthur Schnitzler}\pend{}\pstart{}\textcolor{pink}{\textsc{Wien XVIII}}{}\ledrightnote{\textcolor{pink}{XVIII., Währing}}\pend{}\pstart{}\textcolor{pink}{\textsc{Spöttelgasse 7}}{}\ledrightnote{\textcolor{pink}{Edmund-Weiß-Gasse}}\pend{}{\bigskip}\pstart
           \raggedleft{}{\pb}28. \textcolor{gray}{4}\pend
           \pstart{}Lieber Arthur!\pend\pstart
           Dein Brief u Deine Karten kamen um Viertel nach zehn abends an, ich hätte nicht vor
               elf in \textcolor{pink}{Hietzing}{}\ledrightnote{\textcolor{pink}{XIII., Hietzing}}{ }ſein können u \label{K_L01397_1v}\edtext{\textcolor{blue}{Euch}{}\ledrightnote{→\textcolor{blue}{Richard Beer-Hofmann}{\newline}→\textcolor{blue}{Paula Beer-Hofmann}{\newline}→\textcolor{blue}{Gertrude von Hofmannsthal}{\newline}→\textcolor{blue}{Felix Salten}{\newline}→\textcolor{blue}{Olga Schnitzler}}}{\lemma{\textnormal{\emph{Euch}}}\Cendnote{\textnormal{Anwesend waren \textcolor{blue}{Richard} und \textcolor{blue}{Paula
                     Beer-Hofmann}, \textcolor{blue}{Gerty Hofmannsthal}, \textcolor{blue}{Felix Salten} und \textcolor{blue}{Arthur} und \textcolor{blue}{Olga Schnitzler}.}}}\label{K_L01397_1h}
               dann wol nicht mehr getroffen. Mir war ſehr leid. Könnteſt Du mir \label{K_L01397_2v}\edtext{Samſtag}{\lemma{\textnormal{\emph{Samſtag}}}\Cendnote{\textnormal{Am 30. 4. Zum gewünschten Treffen dürfte es nicht
                  gekommen sein, da \textcolor{blue}{Schnitzler} an diesem Tag
                  seine \textcolor{pink}{Italien}reise begann.}}}\label{K_L01397_2h} zwiſchen \substVorne{}\textsuperscript{\textcolor{gray}{×}\-\textcolor{gray}{×}}\substDazwischen{}fünf\substHinten{} und ſechs ein Rendezvous in der Stadt geben?\pend
           \pstart
           Herzlichſt{\\[\baselineskip]}mit vielen Grüßen an Deine \textcolor{blue}{Fr.}{}\ledrightnote{→\textcolor{blue}{Olga Schnitzler}}\hspace*{1.5em}\spacefill\mbox{Herm}\pend
           \leftskip=0em{}\endnumbering\briefempfaengerindex{Schnitzler, Arthur@\textsc{Schnitzler, Arthur}!zzzBahr, Hermann@\emph{von Hermann Bahr}!1904-04-281@{28. 4. 1904}|)be}\mylabel{h}  \normalsize

\doendnotes{C}
\bigskip
\vfill

\clearpage

\footnotesize

\lohead{\textsc{register}}

% Definiere theindex-Environment komplett neu ohne reledmac
\makeatletter
\renewenvironment{theindex}{%
  \section*{\indexname}%
  \setlength{\parindent}{0pt}%
  \setlength{\parskip}{0pt plus 0.3pt}%
  \let\item\@idxitem
}{%
  \clearpage
}
\makeatother

\IfFileExists{\jobname-pw.ind}{\input{\jobname-pw.ind}}{}

\end{document}

      