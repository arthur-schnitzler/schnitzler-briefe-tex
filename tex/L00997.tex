%% latex-korrekturansicht-vorspann.tex
%% Vorspann für die Korrekturansicht.
%% Lädt die gemeinsame Datei latex-vorspann.tex mit gesetztem Schalter.

\newif\ifkorrekturansicht
\korrekturansichttrue

\input{../tex-inputs/latex-vorspann}


               \section[Arthur Schnitzler an Hugo von Hofmannsthal, {[}18. 11. 1899?{]}]{ Arthur Schnitzler an Hugo von Hofmannsthal,
                    {[}18. 11. 1899?{]}}\nopagebreak\mylabel{v}\rehead{ }\normalsize\beginnumbering\briefempfaengerindex{Hofmannsthal, Hugo von@\textsc{Hofmannsthal, Hugo von}!zzzSchnitzler, Arthur@\emph{von Arthur Schnitzler}!1899-11-181@{{[}18. 11. 1899?{]}}|(be} \toendnotes[C]{\smallbreak\pagebreak[2]} \Standort{FDH, Hs-30885,89.}
\physDesc{Brief, 1 Blatt, 2 Seiten
\newline{}Handschrift: Bleistift, deutsche Kurrent\newline{}Ordnung: von Schnitzler mutmaßlich bei der Durchsicht der Briefe
                                   1929 mit
                                    Bleistift datiert: »99?« }\buchAbdrucke{\weitereDrucke{Hugo von Hofmannsthal, Arthur Schnitzler: \emph{Briefwechsel}. Hg. Therese Nickl und Heinrich Schnitzler. Frankfurt am Main: \emph{S. Fischer} 1964, S. 117.} }\toendnotes[C]{\smallbreak}\pstart{}{\pb}Mein lieber Hugo,\pend\pstart
           Sie sehen, ich \label{K_L00997_1v}\edtext{ka{\geminationn}
                    nicht ko{\geminationm}en}{\lemma{\textnormal{\emph{ka
                    nicht koen}}}\Cendnote{\textnormal{Die Datierung dieses Briefes ist
                        mit vielen Zweifeln behaftet. Sofern die handschriftlich von \textcolor{blue}{Schnitzler} angebrachte Jahresangabe
                        zutrifft – sie ist mit Fragezeichen versehen – ist dies die beste
                        Platzierung innerhalb der überlieferten Dokumente dieses Jahres. \textcolor{blue}{Hofmannsthal} bat am 17. 11. 1899 um ein Treffen für den Folgetag, das bei \textcolor{blue}{Beer-Hofmann} begonnen und dann ins
                        Kaffeehaus geführt hätte. Das Treffen kam nicht zu Stande und dieses
                        Schreiben könnte die Absage darstellen. Unbeantwortet bleibt damit aber,
                        warum er \textcolor{blue}{Beer-Hofmann} anzurufen
                        gedenkt.}}}\label{K_L00997_1h}, auch nicht ins Café{\dots}\pend
           \pstart
           Alles Gute Ihnen!\pend
           \pstart
           – Ich werde möglicherweiſe \textcolor{blue}{Richard}{}\ledrightnote{\textcolor{blue}{Richard Beer-Hofmann}} ſpät
                    Nachts im Café te{\pb}lephoniſch anrufen.\pend
           \pstart
           Ihr treuer{\\}\spacefill\mbox{Arthur}\pend
           \endnumbering\briefempfaengerindex{Hofmannsthal, Hugo von@\textsc{Hofmannsthal, Hugo von}!zzzSchnitzler, Arthur@\emph{von Arthur Schnitzler}!1899-11-181@{{[}18. 11. 1899?{]}}|)be}\mylabel{h}  \normalsize

\doendnotes{C}
\bigskip
\vfill

\clearpage

\footnotesize

\lohead{\textsc{register}}

% Definiere theindex-Environment komplett neu ohne reledmac
\makeatletter
\renewenvironment{theindex}{%
  \section*{\indexname}%
  \setlength{\parindent}{0pt}%
  \setlength{\parskip}{0pt plus 0.3pt}%
  \let\item\@idxitem
}{%
  \clearpage
}
\makeatother

\IfFileExists{\jobname-pw.ind}{\input{\jobname-pw.ind}}{}

\end{document}

      