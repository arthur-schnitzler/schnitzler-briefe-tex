%% latex-korrekturansicht-vorspann.tex
%% Vorspann für die Korrekturansicht.
%% Lädt die gemeinsame Datei latex-vorspann.tex mit gesetztem Schalter.

\newif\ifkorrekturansicht
\korrekturansichttrue

\input{../tex-inputs/latex-vorspann}


               \section[Arthur Schnitzler an Richard Beer-Hofmann, 31. 7. 1896]{ Arthur Schnitzler an Richard Beer-Hofmann,
               31. 7. 1896}\nopagebreak\mylabel{v}\rehead{ }\normalsize\beginnumbering\briefempfaengerindex{Beer-Hofmann, Richard@\textsc{Beer-Hofmann, Richard}!zzzSchnitzler, Arthur@\emph{von Arthur Schnitzler}!1896-07-311@{31. 7. 1896}|(be} \toendnotes[C]{\smallbreak\pagebreak[2]} \Standort{YCGL, MSS 31.}
\physDesc{Postkarte
\newline{}Handschrift: Bleistift, deutsche Kurrent\newline{}Versand: Stempel: »\nobreak{}\oindex{Schweden@\textbf{Schweden}, \emph{https://www.geonames.org/ontologyA.PCLI}|pwk}PKXPN \textcolor{gray}{354A}, 31 7 1896\nobreak{}«.  }\pstart{}{\pb}Herrn \textsc{Dr. Richard
                     Beer-Hofmann}\pend{}\pstart{}\textcolor{pink}{\textsc{Kopenhagen}}{}\ledrightnote{\textcolor{pink}{Kopenhagen}}\pend{}\pstart{}\textcolor{pink}{\textsc{Hotel König v. Dänemark}}{}\ledrightnote{\textcolor{pink}{Hotel König von Dänemark}}\pend{}{\bigskip}\pstart
           \noindent{}{\pb}Lieber Richard, i{\geminationm}erhin iſt es möglich, daſs ich ſchon So{\geminationn}tag{ }\uline{früh} in \textcolor{pink}{K.}{}\ledrightnote{\textcolor{pink}{Kopenhagen}} bin –
               \textcolor{gray}{ſoweit} ich nach den \textcolor{pink}{\textsc{Trollhetta} Fällen}{}\ledrightnote{\textcolor{pink}{Trollhättan-Fälle}} nicht zu müd bin, gleich eine zweite Nacht
               weiterzufahren.\pend
           \pstart Herzlich Ihr \spacefill\mbox{Arthur}\pend{}\pstart
           \noindent{}Bitte beſtellen Sie mir ein Zimmer.\pend
           \endnumbering\briefempfaengerindex{Beer-Hofmann, Richard@\textsc{Beer-Hofmann, Richard}!zzzSchnitzler, Arthur@\emph{von Arthur Schnitzler}!1896-07-311@{31. 7. 1896}|)be}\mylabel{h}  \normalsize

\doendnotes{C}
\bigskip
\vfill

\clearpage

\footnotesize

\lohead{\textsc{register}}

% Definiere theindex-Environment komplett neu ohne reledmac
\makeatletter
\renewenvironment{theindex}{%
  \section*{\indexname}%
  \setlength{\parindent}{0pt}%
  \setlength{\parskip}{0pt plus 0.3pt}%
  \let\item\@idxitem
}{%
  \clearpage
}
\makeatother

\IfFileExists{\jobname-pw.ind}{\input{\jobname-pw.ind}}{}

\end{document}

      