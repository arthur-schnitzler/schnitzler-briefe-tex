%% latex-korrekturansicht-vorspann.tex
%% Vorspann für die Korrekturansicht.
%% Lädt die gemeinsame Datei latex-vorspann.tex mit gesetztem Schalter.

\newif\ifkorrekturansicht
\korrekturansichttrue

\input{../tex-inputs/latex-vorspann}


               \section[Max Burckhard an Arthur Schnitzler, 8. 4. 1910]{ Max Burckhard an Arthur Schnitzler, 8. 4. 1910}\nopagebreak\mylabel{v}\rehead{ }\normalsize\beginnumbering\briefempfaengerindex{Schnitzler, Arthur@\textsc{Schnitzler, Arthur}!zzzBurckhard, Max Eugen@\emph{von Max Eugen Burckhard}!1910-04-081@{8. 4. 1910}|(be} \toendnotes[C]{\smallbreak\pagebreak[2]} \Standort{TMW, HS Schn 1/73/1.}
\physDesc{Brief, 1 Blatt, 1 Seite
\newline{}Handschrift: schwarze Tinte, deutsche Kurrent}\pstart
           \noindent{}{\pb}\textcolor{gray}{\textbf{Dr. Max Burckhard}}\hfill \textcolor{gray}{\textbf{\textcolor{pink}{Wien, IX. Porzellangasse 48}{}\ledrightnote{\textcolor{pink}{Porzellangasse}}}}{ }..........\pend
           \pstart
           \raggedleft{}\textcolor{gray}{\textbf{\textcolor{pink}{St. Gilgen}{}\ledrightnote{\textcolor{pink}{St. Gilgen}}}}{ }8. 4. 10\pend
           \pstart{}Lieber verehrter Herr Doctor!\pend\pstart
           Ich habe Zweifel, ob ein Brief, den ich geſtern an Sie ſchrieb, aufgegeben wurde,
                    und ſage daher vorſichtsweiſe heute nochmal Dank für Ihren lieben Brief, den ich
                    bei der Rückkehr aus \textcolor{pink}{Portofino}{}\ledrightnote{\textcolor{pink}{Portofino}} vorfand. Mich
                    hat es außerordentlich gefreut, daſs \textcolor{green}{Trinacria}{}\ledrightnote{\textcolor{green}{Trinacria}}
                    Sie intereſſiert hat, da ich bei perſönlichen Reminiscenzen i{\geminationm}er ganz beſonders unſicher bin über die Wirkung
                    auf andere. Ich habe \textcolor{pink}{Sicilien}{}\ledrightnote{\textcolor{pink}{Sizilien}}{ }ſo gerne gewonnen, daſs ich fünfmal unten war
                    und bei ſolchen Gelegenheiten nicht nur ſehr viel herumgeradelt u -gekraxelt
                    bin, ſondern auch bis in die Tiefe archäologiſcher Localſtudien geſunken
                    bin.\pend
           \pstart
           Auf ſehr baldiges Wiederſehen in \textcolor{pink}{Wien}{}\ledrightnote{\textcolor{pink}{Wien}}, und
                    hoffentlich wieder in \textcolor{pink}{St. Gilgen}{}\ledrightnote{\textcolor{pink}{St. Gilgen}}. Mit Handkuß u
                    herzl Grüßen{\\[\baselineskip]}Ihr{\\[\baselineskip]}\spacefill\mbox{D\textsuperscript{r}Burckhard}\pend
           \leftskip=0em{}\endnumbering\briefempfaengerindex{Schnitzler, Arthur@\textsc{Schnitzler, Arthur}!zzzBurckhard, Max Eugen@\emph{von Max Eugen Burckhard}!1910-04-081@{8. 4. 1910}|)be}\mylabel{h}  \normalsize

\doendnotes{C}
\bigskip
\vfill

\clearpage

\footnotesize

\lohead{\textsc{register}}

% Definiere theindex-Environment komplett neu ohne reledmac
\makeatletter
\renewenvironment{theindex}{%
  \section*{\indexname}%
  \setlength{\parindent}{0pt}%
  \setlength{\parskip}{0pt plus 0.3pt}%
  \let\item\@idxitem
}{%
  \clearpage
}
\makeatother

\IfFileExists{\jobname-pw.ind}{\input{\jobname-pw.ind}}{}

\end{document}

      