%% latex-korrekturansicht-vorspann.tex
%% Vorspann für die Korrekturansicht.
%% Lädt die gemeinsame Datei latex-vorspann.tex mit gesetztem Schalter.

\newif\ifkorrekturansicht
\korrekturansichttrue

\input{../tex-inputs/latex-vorspann}


               \section[Arthur Schnitzler an Albert Ehrenstein, 12. 2. 1910]{ Arthur Schnitzler an Albert Ehrenstein, 12. 2. 1910}\nopagebreak\mylabel{v}\rehead{ }\normalsize\beginnumbering\briefempfaengerindex{Ehrenstein, Albert@\textsc{Ehrenstein, Albert}!zzzSchnitzler, Arthur@\emph{von Arthur Schnitzler}!1910-02-121@{12. 2. 1910}|(be} \toendnotes[C]{\smallbreak\pagebreak[2]} \Standort{Jerusalem, The National Library of Israel, ARC. Ms. Var. 306 1 118.}
\physDesc{Brief, 1 Blatt, 1 Seite
\newline{}Schreibmaschine
\newline{}Handschrift: schwarze Tinte (\noindent{}Unterschrift)}\Standort{DLA, A:Schnitzler, 85.1.642,3.}
\physDesc{Brief, maschineller Durchschlag
\newline{}Schreibmaschine
\newline{}Handschrift: roter Buntstift, lateinische Kurrent (\noindent{}Beschriftung:
                                                »Ehrenstein«)}\toendnotes[C]{\smallbreak}\pstart
           \noindent{}{\pb}\textcolor{gray}{\textbf{Dr. Arthur Schnitzler}}\hfill 12. 2. 1910.\pend
           \pstart
           \textcolor{gray}{\textbf{\textcolor{pink}{Wien XVIII. Spoettelgasse 7}{}\ledrightnote{\textcolor{pink}{Edmund-Weiß-Gasse}}.}}\pend
           \pstart{}Lieber Herr Ehrenstein!\pend\pstart
           Aus dem Brief von \textcolor{blue}{Bie}{}\ledrightnote{\textcolor{blue}{Oskar Bie}} an Sie ist zu entnehmen,
                    dass er »\textcolor{green}{Tubutsch}{}\ledrightnote{\textcolor{green}{Tubutsch}}« nicht veröffentlichen will,
                    dass aber für Ihre nächsten Einsendungen aufrichtiges Interesse und daher auch
                    Druckchancen vorhanden sind. Das mit dem \textcolor{pink}{Wien}{}\ledrightnote{\textcolor{pink}{Wien}}er
                    Leben müssen Sie nicht so wörtlich nehmen. Was die \textcolor{blue}{Schröder}{}\ledrightnote{\textcolor{blue}{Rudolf Alexander Schröder}}’sche \textcolor{green}{\textcolor{blue}{Homer}{}\ledrightnote{\textcolor{blue}{Homer}}überſetzung}{}\ledrightnote{→\textcolor{green}{Odyssee}} anbelangt, so
                    bringen Sie diesen Wunsch vielleicht Dr. \textcolor{blue}{Bie}{}\ledrightnote{\textcolor{blue}{Oskar Bie}}
                    direkt schriftlich zur Kenntnis.\pend
           \pstart
           \textcolor{green}{Medardus}{}\ledrightnote{\textcolor{green}{Der junge Medardus. Dramatische Historie in einem Vorspiel und fünf Aufzügen}} hätte am Tage der Erstaufführung im
                    Buchhandel erscheinen sollen, zurückgezogen wurde er nie, vielmehr ist er gerade
                    in den letzten Tagen angenommen worden und soll im Herbst gespielt werden, bei
                    welcher Gelegenheit auch das Buch herauskommen wird.\pend
           \pstart
           Auf Wiedersehen und besten Gruss!{\\[\baselineskip]}Ihr{\\[\baselineskip]}\spacefill\mbox{{[}hs. Schnitzler:{]} Arthur Schnitzler}\pend
           \leftskip=0em{}\endnumbering\briefempfaengerindex{Ehrenstein, Albert@\textsc{Ehrenstein, Albert}!zzzSchnitzler, Arthur@\emph{von Arthur Schnitzler}!1910-02-121@{12. 2. 1910}|)be}\mylabel{h}  \normalsize

\doendnotes{C}
\bigskip
\vfill

\clearpage

\footnotesize

\lohead{\textsc{register}}

% Definiere theindex-Environment komplett neu ohne reledmac
\makeatletter
\renewenvironment{theindex}{%
  \section*{\indexname}%
  \setlength{\parindent}{0pt}%
  \setlength{\parskip}{0pt plus 0.3pt}%
  \let\item\@idxitem
}{%
  \clearpage
}
\makeatother

\IfFileExists{\jobname-pw.ind}{\input{\jobname-pw.ind}}{}

\end{document}

      