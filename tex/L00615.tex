%% latex-korrekturansicht-vorspann.tex
%% Vorspann für die Korrekturansicht.
%% Lädt die gemeinsame Datei latex-vorspann.tex mit gesetztem Schalter.

\newif\ifkorrekturansicht
\korrekturansichttrue

\input{../tex-inputs/latex-vorspann}


               \section[Richard Beer-Hofmann an Arthur Schnitzler, {[}4. 11. 1896{]}]{ Richard Beer-Hofmann an Arthur Schnitzler, {[}4. 11. 1896{]}}\nopagebreak\mylabel{v}\rehead{ }\normalsize\beginnumbering\briefempfaengerindex{Schnitzler, Arthur@\textsc{Schnitzler, Arthur}!zzzBeer-Hofmann, Richard@\emph{von Richard Beer-Hofmann}!1896-11-041@{{[}4. 11. 1896{]}}|(be} \toendnotes[C]{\smallbreak\pagebreak[2]} \Standort{CUL, Schnitzler, B 8.}
\physDesc{Telegramm
\newline{}maschinell\newline{}Ordnung: mit Bleistift von unbekannter Hand
                                    nummeriert: »87« }\buchAbdrucke{\weitereDrucke{Arthur Schnitzler, Richard Beer-Hofmann: \emph{Briefwechsel 1891–1931}. Hg. Konstanze Fliedl. Wien, Zürich: \emph{Europaverlag} 1992, S. 99.} }\toendnotes[C]{\smallbreak}\pstart
           {\pb}\textcolor{pink}{\textcolor{gray}{b}}{}\ledrightnote{\textcolor{pink}{Berlin}}{ }fr{ }\textcolor{pink}{\textcolor{gray}{wien}}{}\ledrightnote{\textcolor{pink}{Wien}}{ }502{ }21{ }12/50=\pend
           \pstart
           ich freue mich sehr. es gratuliren \textcolor{blue}{schwarze koepfe}{}\ledrightnote{\textcolor{blue}{Gustav Schwarzkopf}{\newline}\textcolor{blue}{Max Schwarzkopf}{\newline}\textcolor{blue}{Emil Schwarzkopf}}, \textcolor{blue}{leon}{}\ledrightnote{\textcolor{blue}{Victor Léon}} und \textcolor{blue}{bruder}{}\ledrightnote{→\textcolor{blue}{Leo Feld}} doktor \textcolor{blue}{engl}{}\ledrightnote{\textcolor{blue}{Alexander Engel}} = \spacefill\mbox{richard +}\pend
           \endnumbering\briefempfaengerindex{Schnitzler, Arthur@\textsc{Schnitzler, Arthur}!zzzBeer-Hofmann, Richard@\emph{von Richard Beer-Hofmann}!1896-11-041@{{[}4. 11. 1896{]}}|)be}\mylabel{h}  \normalsize

\doendnotes{C}
\bigskip
\vfill

\clearpage

\footnotesize

\lohead{\textsc{register}}

% Definiere theindex-Environment komplett neu ohne reledmac
\makeatletter
\renewenvironment{theindex}{%
  \section*{\indexname}%
  \setlength{\parindent}{0pt}%
  \setlength{\parskip}{0pt plus 0.3pt}%
  \let\item\@idxitem
}{%
  \clearpage
}
\makeatother

\IfFileExists{\jobname-pw.ind}{\input{\jobname-pw.ind}}{}

\end{document}

      