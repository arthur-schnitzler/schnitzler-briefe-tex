%% latex-korrekturansicht-vorspann.tex
%% Vorspann für die Korrekturansicht.
%% Lädt die gemeinsame Datei latex-vorspann.tex mit gesetztem Schalter.

\newif\ifkorrekturansicht
\korrekturansichttrue

\input{../tex-inputs/latex-vorspann}


               \section[Hugo Hofmannsthal an Arthur Schnitzler, 3. 3. 1925]{ Hugo Hofmannsthal an Arthur Schnitzler, 3. 3. 1925}\nopagebreak\mylabel{v}\rehead{ }\normalsize\beginnumbering\briefempfaengerindex{Schnitzler, Arthur@\textsc{Schnitzler, Arthur}!zzzHofmannsthal, Hugo von@\emph{von Hugo von Hofmannsthal}!1925-03-031@{3. 3. 1925}|(be} \toendnotes[C]{\smallbreak\pagebreak[2]} \Standort{CUL, Schnitzler, B 43.}
\physDesc{Bildpostkarte
\newline{}Handschrift: Bleistift, lateinische Kurrent\newline{}Versand: Stempel: »\nobreak{}\oindex{Bahnhof Marseille@\textbf{Bahnhof Marseille}, \emph{Bahnhofsgebäude (K.BHF)}|pwk}Marseille – Gare B\textsuperscript{ches} du Rhône, 4-III 1925, 17\textsuperscript{30}\nobreak{}«.  \newline{}Ordnung: 1) mit Bleistift von unbekannter Hand nummeriert: »\strikeout{388}« 2) mit Bleistift von unbekannter Hand nummeriert: »388«}\buchAbdrucke{\weitereDrucke{Hugo von Hofmannsthal, Arthur Schnitzler: \emph{Briefwechsel}. Hg. Therese Nickl und Heinrich Schnitzler. Frankfurt am Main: \emph{S. Fischer} 1964, S. 301.} }\pstart{}{\pb}Arthur Schnitzler\pend{}\pstart{}\textcolor{pink}{XVIII Sternwartestrasse 71}{}\ledrightnote{\textcolor{pink}{Sternwartestraße}}\pend{}\pstart{}\textcolor{pink}{Wien}{}\ledrightnote{\textcolor{pink}{Wien}}\pend{}\pstart{}\textcolor{pink}{Autriche}{}\ledrightnote{\textcolor{pink}{Österreich}}\pend{}{\bigskip}\pstart
           \noindent{}{\pb}\textcolor{gray}{\textbf{\begin{otherlanguage}{french}\textcolor{pink}{AVIGNON}{}\ledrightnote{\textcolor{pink}{Avignon}} – N.-D des Doms (\textcolor{pink}{Cathédrale}{}\ledrightnote{\textcolor{pink}{Kathedrale von Avignon}}) Tour Campane et partie du Palais construite à
                     l’avènement du Pape \textcolor{blue}{Jean XXII}{}\ledrightnote{\textcolor{blue}{Johannes XXII.}}
                        (1316), sur l’emplacement de l’Eglise St-Etienne. \textcolor{blue}{Benoit XII}{}\ledrightnote{\textcolor{blue}{Benedikt XII.}},
                        (1335–1342) sur les plans de Pierre Obrerie
                     architecte français compléta l’œuvre de son prédécesseur. Cette partie du
                     monument renferme aujourd’hui les Archives départementales.\end{otherlanguage}}}\pend
           \pstart
           {\pb}Da ist schon die erste Karte mit
               vielen Grüßen u. Gedanken!\pend
           \pstart \spacefill\mbox{Hugo.}\pend{}\pstart
           \textcolor{pink}{Avignon}{}\ledrightnote{\textcolor{pink}{Avignon}}{ }3 III 1925.\pend
           \endnumbering\briefempfaengerindex{Schnitzler, Arthur@\textsc{Schnitzler, Arthur}!zzzHofmannsthal, Hugo von@\emph{von Hugo von Hofmannsthal}!1925-03-031@{3. 3. 1925}|)be}\mylabel{h}  \normalsize

\doendnotes{C}
\bigskip
\vfill

\clearpage

\footnotesize

\lohead{\textsc{register}}

% Definiere theindex-Environment komplett neu ohne reledmac
\makeatletter
\renewenvironment{theindex}{%
  \section*{\indexname}%
  \setlength{\parindent}{0pt}%
  \setlength{\parskip}{0pt plus 0.3pt}%
  \let\item\@idxitem
}{%
  \clearpage
}
\makeatother

\IfFileExists{\jobname-pw.ind}{\input{\jobname-pw.ind}}{}

\end{document}

      