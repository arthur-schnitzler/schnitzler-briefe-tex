%% latex-korrekturansicht-vorspann.tex
%% Vorspann für die Korrekturansicht.
%% Lädt die gemeinsame Datei latex-vorspann.tex mit gesetztem Schalter.

\newif\ifkorrekturansicht
\korrekturansichttrue

\input{../tex-inputs/latex-vorspann}


               \section[Paul Goldmann an Arthur Schnitzler, 25. 7. 1891]{ Paul Goldmann an Arthur Schnitzler, 25. 7. 1891}\nopagebreak\mylabel{v}\rehead{ }\normalsize\beginnumbering\briefempfaengerindex{Schnitzler, Arthur@\textsc{Schnitzler, Arthur}!zzzGoldmann, Paul@\emph{von Paul Goldmann}!1891-07-251@{25. 7. 1891}|(be} \toendnotes[C]{\smallbreak\pagebreak[2]} \Standort{DLA, A:Schnitzler, HS.NZ85.1.3162.}
\physDesc{Postkarte
\newline{}Handschrift: 1) Bleistift, deutsche Kurrent\hspace{1em}2) Bleistift, lateinische Kurrent (\noindent{}Adresse)\hspace{1em}\newline{}Versand: 1) Stempel: »\nobreak{}\oindex{Koeln@\textbf{Köln}, \emph{Besiedelter Ort (A.BSO)}|pwk}Cöln (Rhein{[}land){]}, 25 7 91, Zug 13\nobreak{}«.  2) Stempel: »\nobreak{}\oindex{I., Innere Stadt@\textbf{I., Innere Stadt}, \emph{Bezirk (A.BZK)}|pwk}Wien 1/1, 27/7 91, 9½–11V, Bestellt\nobreak{}«. 
\newline{}Schnitzler: mit Bleistift das Datum »15/ 7 91« vermerkt }\toendnotes[C]{\smallbreak}\pstart{}{\pb}\textcolor{pink}{Österreich}{}\ledrightnote{\textcolor{pink}{Österreich}}!\pend{}\pstart{}Herrn\pend{}\pstart{}Dr. Arthur Schnitzler\pend{}\pstart{}\textcolor{pink}{Wien}{}\ledrightnote{\textcolor{pink}{Wien}}\pend{}\pstart{}\textcolor{pink}{I, Giselastraße 11}{}\ledrightnote{\textcolor{pink}{Bösendorferstraße}}.\pend{}{\bigskip}\pstart
           \noindent{}{\pb}\textsc{\textcolor{pink}{Köln}{}\ledrightnote{\textcolor{pink}{Köln}}}, 25. 7. – 1 Uhr Nachts. Mein lieber Arthur! Ich kehre nach \textcolor{pink}{Brüſſel}{}\ledrightnote{\textcolor{pink}{Brüssel}} zurück von einem 7 tägigen Aufenthalt, den ich
                  \textcolor{gray}{in}{ }\textsc{\textcolor{pink}{Frankfurt}{}\ledrightnote{\textcolor{pink}{Frankfurt am Main}}} in Familien u. Redactionsangelegenheiten geno{\geminationm}en.
               Ärgerniß u. Kümmerniß ringsum. Ich denke Dein in Treue und Schmerzen. Oh, mein lieber
               Arthur und immer liebes \textcolor{pink}{Wien}{}\ledrightnote{\textcolor{pink}{Wien}}! So fahre ich in die
               Nacht hinein wie ein Verdammter und Verfluchter! {\dots}\pend
           \pstart
           Gott behüte Dich!{\\[\baselineskip]}Dein {\\[\baselineskip]}\spacefill\mbox{Paul}\pend
           \leftskip=0em{}\pstart
           \noindent{}\label{T_L02667-1v}\edtext{Auf den Knien geſchrieben.}{\lemma{\textnormal{\emph{Auf … geſchrieben.}}}\Cendnote{\textnormal{am oberen Rand}}}\label{T_L02667-1h}\pend
           \endnumbering\briefempfaengerindex{Schnitzler, Arthur@\textsc{Schnitzler, Arthur}!zzzGoldmann, Paul@\emph{von Paul Goldmann}!1891-07-251@{25. 7. 1891}|)be}\mylabel{h}  \normalsize

\doendnotes{C}
\bigskip
\vfill

\clearpage

\footnotesize

\lohead{\textsc{register}}

% Definiere theindex-Environment komplett neu ohne reledmac
\makeatletter
\renewenvironment{theindex}{%
  \section*{\indexname}%
  \setlength{\parindent}{0pt}%
  \setlength{\parskip}{0pt plus 0.3pt}%
  \let\item\@idxitem
}{%
  \clearpage
}
\makeatother

\IfFileExists{\jobname-pw.ind}{\input{\jobname-pw.ind}}{}

\end{document}

      