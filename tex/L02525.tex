%% latex-korrekturansicht-vorspann.tex
%% Vorspann für die Korrekturansicht.
%% Lädt die gemeinsame Datei latex-vorspann.tex mit gesetztem Schalter.

\newif\ifkorrekturansicht
\korrekturansichttrue

\input{../tex-inputs/latex-vorspann}


               \section[Gerty von Hofmannsthal an Arthur Schnitzler, 23. 11. 1929]{ Gerty von Hofmannsthal an Arthur Schnitzler, 23. 11. 1929}\nopagebreak\mylabel{v}\rehead{ }\normalsize\beginnumbering\briefempfaengerindex{Schnitzler, Arthur@\textsc{Schnitzler, Arthur}!zzzHofmannsthal, Gertrude von@\emph{von Gertrude von Hofmannsthal}!1929-11-231@{23. 11. 1929}|(be} \toendnotes[C]{\smallbreak\pagebreak[2]} \Standort{CUL, Schnitzler, B 43.}
\physDesc{Brief, 1 Blatt, 1 Seite
\newline{}Schreibmaschine
\newline{}Handschrift: schwarze Tinte, lateinische Kurrent (\noindent{}Angabe der Straße, Unterschrift)
\newline{}Schnitzler: mit rotem Buntstift beschriftet: »\textsc{Hofm}« und zwei Unterstreichungen vorgenommen }\toendnotes[C]{\smallbreak}\pstart
           \raggedleft{}{\pb}\textcolor{pink}{Wien}{}\ledrightnote{\textcolor{pink}{Wien}} d 23/IX 29\pend
           \pstart
           \raggedleft{}{[}hs.:{]} \textcolor{pink}{I Stallburggasse 2}{}\ledrightnote{\textcolor{pink}{Stallburggasse}}\pend
           \pstart
           {[}ms.:{]} Lieber Arthur, darf ich Sie heute um einen Rat fragen in einer
                    geschäftlichen Angelegenheit: Die \textcolor{brown}{Zentralstelle der
                        Bühnenautoren und Verleger}{}\ledrightnote{\textcolor{brown}{Zentralstelle der Bühnen-Autoren und -Verleger}} reclamiert eine 3{\%}tige Tantiemenabgabe aus Eingängen aus \textcolor{pink}{Oest{[}e{]}rreich}{}\ledrightnote{\textcolor{pink}{Österreich}} und \textcolor{pink}{C.S.R.}{}\ledrightnote{\textcolor{pink}{Tschechoslowakei}}\pend
           \pstart
           Ich weiss dass auch \textcolor{blue}{Hugo}{}\ledrightnote{\textcolor{blue}{Hugo von Hofmannsthal}} dies tat wenn es
                    sich um ein Werk wie \textcolor{green}{Jedermann}{}\ledrightnote{\textcolor{green}{Jedermann. Das Spiel vom Sterben des reichen Mannes}} gehandelt hat
                    welches er für \textcolor{pink}{Oesterreich}{}\ledrightnote{\textcolor{pink}{Österreich}} selbst zum
                    Vertrieb hatte und ich weiss auch dass er voriges Jahr im Mai für
                    die Aufführungen in \textcolor{pink}{Salzburg}{}\ledrightnote{\textcolor{pink}{Salzburg}} dem Verein
                        120 \label{K_L02525_1v}\edtext{S.}{\lemma{\textnormal{\emph{S.}}}\Cendnote{\textnormal{Schilling}}}\label{K_L02525_1h} anwies (was unter uns gesagt keine 3{\%} der Einnahmen war) Da die heurigen Einnahmen doch
                    eine ziemliche Höhe hatten und auch die \textcolor{brown}{Josefstadt}{}\ledrightnote{\textcolor{brown}{Theater in der Josefstadt}} den \textcolor{green}{Schwierigen}{}\ledrightnote{\textcolor{green}{Der Schwierige. Lustspiel in drei Akten}} direct
                    mit mir abrechnete so wären 3{\%}{ }\so{ehrlich} abgerechnet doch ganz
                    viel.\pend
           \pstart
           Nun habe ich bei \textcolor{brown}{Fischer}{}\ledrightnote{\textcolor{brown}{S. Fischer Verlag}} nachgesehen und gesehen
                    dass er in \textcolor{pink}{Deutschland}{}\ledrightnote{\textcolor{pink}{Deutschland}} immer 2{\%} bei den \label{T_L02525_1v}\edtext{Abrechnungen}{\lemma{\textnormal{\emph{Abrechnungen}}}\Cendnote{\textnormal{Sie schreibt:
                            »Abrechrenungen«}}}\label{T_L02525_1h} abzieht. Warum also 3{\%} hier? Ferner ob Sie glauben dass ich nach unten
                    abrunden kann in der Berechnung, oder ob der \textcolor{brown}{Verein}{}\ledrightnote{→\textcolor{brown}{Zentralstelle der Bühnen-Autoren und -Verleger}} das Recht hat nachzuforschen wie viel tatsächlich
                    die Einnahmen waren. Ich verstehe ja gar nicht die Rechte, die dieser \textcolor{brown}{Verein}{}\ledrightnote{→\textcolor{brown}{Zentralstelle der Bühnen-Autoren und -Verleger}} hat, und welche
                    Vorteile man wiederum hat wenn man ihm angehört – aber vielleicht muss das eben
                    sein, sonst würde Fischer ja auch nicht die Percente gleich automatisch
                    zahlen.\pend
           \pstart
           Also meine Frage: muss ich \so{ehrlich} sein?\pend
           \pstart
           2/ ist 3{\%} berechtigt?\pend
           \pstart
           Ich schreibe dies, weil mein Telephon so schnell abschnappt. Aber wenn Sie so
                    lieb sind mich anzurufen und mir die Antwort sagen R 23757, (am besten zwischen
                        ½10–11), so wäre ich sehr dankbar\pend
           \pstart
           \label{T_L02525_2v}\edtext{Herzlichst}{\lemma{\textnormal{\emph{Herzlichst}}}\Cendnote{\textnormal{Sie schreibt:
                            »Herzlchst«}}}\label{T_L02525_2h}{\\[\baselineskip]}Ihre{\\[\baselineskip]}\spacefill\mbox{{[}hs.:{]} Gerty}\pend
           \leftskip=0em{}\endnumbering\briefempfaengerindex{Schnitzler, Arthur@\textsc{Schnitzler, Arthur}!zzzHofmannsthal, Gertrude von@\emph{von Gertrude von Hofmannsthal}!1929-11-231@{23. 11. 1929}|)be}\mylabel{h}  \normalsize

\doendnotes{C}
\bigskip
\vfill

\clearpage

\footnotesize

\lohead{\textsc{register}}

% Definiere theindex-Environment komplett neu ohne reledmac
\makeatletter
\renewenvironment{theindex}{%
  \section*{\indexname}%
  \setlength{\parindent}{0pt}%
  \setlength{\parskip}{0pt plus 0.3pt}%
  \let\item\@idxitem
}{%
  \clearpage
}
\makeatother

\IfFileExists{\jobname-pw.ind}{\input{\jobname-pw.ind}}{}

\end{document}

      