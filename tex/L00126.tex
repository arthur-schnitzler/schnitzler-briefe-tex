%% latex-korrekturansicht-vorspann.tex
%% Vorspann für die Korrekturansicht.
%% Lädt die gemeinsame Datei latex-vorspann.tex mit gesetztem Schalter.

\newif\ifkorrekturansicht
\korrekturansichttrue

\input{../tex-inputs/latex-vorspann}


               \section[Arthur Schnitzler an Richard Beer-Hofmann, 2. 10. 1892]{ Arthur Schnitzler an Richard Beer-Hofmann, 2. 10. 1892}\nopagebreak\mylabel{v}\rehead{ }\normalsize\beginnumbering\briefempfaengerindex{Beer-Hofmann, Richard@\textsc{Beer-Hofmann, Richard}!zzzSchnitzler, Arthur@\emph{von Arthur Schnitzler}!1892-10-021@{2. 10. 1892}|(be} \toendnotes[C]{\smallbreak\pagebreak[2]} \Standort{YCGL, MSS 31.}
\physDesc{Postkarte
\newline{}Handschrift: Bleistift, deutsche Kurrent\newline{}Versand: 1) Rohrpost 2) Stempel: »\nobreak{}\oindex{I., Innere Stadt@\textbf{I., Innere Stadt}, \emph{Bezirk (A.BZK)}|pwk}Wien 1/1, 2-X 93, 7 10N\nobreak{}«. 3) Stempel: »\nobreak{}\oindex{I., Innere Stadt@\textbf{I., Innere Stadt}, \emph{Bezirk (A.BZK)}|pwk}Wien 1/1, 2 X {[}92{]}, 7 40N\nobreak{}«. }\buchAbdrucke{\weitereDrucke{Arthur Schnitzler, Richard Beer-Hofmann: \emph{Briefwechsel 1891–1931}. Hg. Konstanze Fliedl. Wien, Zürich: \emph{Europaverlag} 1992, S. 39.} }\toendnotes[C]{\smallbreak}\pstart{}{\pb}Hrn \textsc{Dr. Richard Beer
                     Hofmann}\pend{}\pstart{}\textsc{\textcolor{pink}{Wien}{}\ledrightnote{\textcolor{pink}{Wien}}}\pend{}\pstart{}\textcolor{pink}{I. \textsc{Wollzeile 15}}{}\ledrightnote{\textcolor{pink}{Wollzeile}}\pend{}{\bigskip}\pstart
           \noindent{}{\pb}Lieber Richard! \textsc{\textcolor{blue}{Torres.}{}\ledrightnote{\textcolor{blue}{Carl von Torresani-Lanzenfeld}}} holt mich \label{K_L00126_1v}\edtext{\uline{Montag}}{\lemma{\textnormal{\emph{Montag}}}\Cendnote{\textnormal{Obzwar am Poststempel – sofern er
                  sich auf das Jahr bezieht – eindeutig 93
                   steht, scheint dies doch
                  durch den Inhalt ausgeschlossen. \textcolor{blue}{Schnitzler} war
                  am Sonntag, 2. 10. 1892 in \emph{\textcolor{green}{Gefallene
                     Engel}}, am Folgetag wurde er von \textcolor{blue}{Torresani} für das \textcolor{pink}{Ausstellungstheater}
                  abgeholt.}}}\label{K_L00126_1h}{ }Nachmittag vor 5 Uhr für die \textcolor{pink}{Ausſtellung}{}\ledrightnote{→\textcolor{pink}{Internationales Ausstellungstheater im k.k. Prater}} ab; bitte ko{\geminationm}en
               Sie auch zu mir. \uline{So{\geminationn}tag} denke ich zu den »\textcolor{green}{gefallenen Engeln}{}\ledrightnote{\textcolor{green}{Gefallene Engel. Volkstück in drei Aufzügen}}« zu gehn,
               wenn ich ordentliche Sitze beko{\geminationm}e. Jedenfalls bin ich um
                  5, ½ 6 zu Hauſe.\pend
           \pstart
           Herzlich grüßend Ihr{\\[\baselineskip]}\spacefill\mbox{Arthur}\pend
           \leftskip=0em{}\endnumbering\briefempfaengerindex{Beer-Hofmann, Richard@\textsc{Beer-Hofmann, Richard}!zzzSchnitzler, Arthur@\emph{von Arthur Schnitzler}!1892-10-021@{2. 10. 1892}|)be}\mylabel{h}  \normalsize

\doendnotes{C}
\bigskip
\vfill

\clearpage

\footnotesize

\lohead{\textsc{register}}

% Definiere theindex-Environment komplett neu ohne reledmac
\makeatletter
\renewenvironment{theindex}{%
  \section*{\indexname}%
  \setlength{\parindent}{0pt}%
  \setlength{\parskip}{0pt plus 0.3pt}%
  \let\item\@idxitem
}{%
  \clearpage
}
\makeatother

\IfFileExists{\jobname-pw.ind}{\input{\jobname-pw.ind}}{}

\end{document}

      