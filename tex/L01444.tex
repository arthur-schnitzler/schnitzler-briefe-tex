%% latex-korrekturansicht-vorspann.tex
%% Vorspann für die Korrekturansicht.
%% Lädt die gemeinsame Datei latex-vorspann.tex mit gesetztem Schalter.

\newif\ifkorrekturansicht
\korrekturansichttrue

\input{../tex-inputs/latex-vorspann}


               \section[Arthur Schnitzler an Richard Beer-Hofmann, 14. 9. 1904]{ Arthur Schnitzler an Richard Beer-Hofmann, 14. 9. 1904}\nopagebreak\mylabel{v}\rehead{ }\normalsize\beginnumbering\briefempfaengerindex{Beer-Hofmann, Richard@\textsc{Beer-Hofmann, Richard}!zzzSchnitzler, Arthur@\emph{von Arthur Schnitzler}!1904-09-141@{14. 9. 1904}|(be} \toendnotes[C]{\smallbreak\pagebreak[2]} \Standort{YCGL, MSS 31.}
\physDesc{Brief, 1 Blatt, 3 Seiten, Umschlag
\newline{}Handschrift: Bleistift, deutsche Kurrent\newline{}Versand: 1) Stempel: »\nobreak{}\oindex{St. Gilgen@\textbf{St. Gilgen}, \emph{Besiedelter Ort (A.BSO)}|pwk}St. Gilgen, 14. 9. 04, 3–4N\nobreak{}«.  2) Stempel: »\nobreak{}\oindex{Bad Aussee@\textbf{Bad Aussee}, \emph{Besiedelter Ort (A.BSO)}|pwk}{\pb}\textcolor{gray}{A}ussee in Steiermark, 15 9 04\nobreak{}«. }\buchAbdrucke{\weitereDrucke{Arthur Schnitzler, Richard Beer-Hofmann: \emph{Briefwechsel 1891–1931}. Hg. Konstanze Fliedl. Wien, Zürich: \emph{Europaverlag} 1992, S. 166–167.} }\toendnotes[C]{\smallbreak}\pstart{}{\pb}\textsc{Herrn Dr Rich. Beer-Hofmann}\pend{}\pstart{}\textsc{\textcolor{pink}{Markt Aussee}{}\ledrightnote{\textcolor{pink}{Bad Aussee}}}\pend{}\pstart{}\textcolor{pink}{\textsc{Villa Frühling}}{}\ledrightnote{\textcolor{pink}{Villa Frühling}}.
               \pend{}{\bigskip}\pstart
           \raggedleft{}{\pb}\textcolor{pink}{\textsc{Lueg}}{}\ledrightnote{\textcolor{pink}{Lueg am Wolfgangsee}}, 14. 9. 904\pend
           \pstart
           lieber Richard, eben ko{\geminationm}t, wie ich im
               Begriff bin Ihnen zu telegrafiren, \substVorne{}\textsuperscript{ein}\substDazwischen{}Ihr\substHinten{} Brief. Wir möchten Samſtag den 17. von hier nach \textcolor{pink}{Salzburg}{}\ledrightnote{\textcolor{pink}{Salzburg}} reiſen und dort einige Tage bleiben. (Möchten diesmal
               verſuchsweiſe \textcolor{pink}{Nelböck}{}\ledrightnote{\textcolor{pink}{Hotel und Pension Nelböck}} wohnen.) Ich ſchlage Ihnen
               nun vor, Freitag nach \textcolor{pink}{\textsc{Lueg}}{}\ledrightnote{\textcolor{pink}{Lueg am Wolfgangsee}} zu ko{\geminationm}en und Samſtag mit uns zu
               fahren, oder uns \strikeout{vielleicht}{ }{\pb}zu ſchreiben, wann Sie in \textcolor{pink}{\textsc{Lueg}}{}\ledrightnote{\textcolor{pink}{Lueg am Wolfgangsee}} durchkommen, ſo daſs wir hier zu Ihnen einſteigen. (Der Zug, der \textcolor{pink}{Iſchl}{}\ledrightnote{\textcolor{pink}{Bad Ischl}}{ }8.55 früh verläßt u 9.59{ }\textcolor{pink}{\textsc{Lueg}}{}\ledrightnote{\textcolor{pink}{Lueg am Wolfgangsee}} paſſirt, wäre mir der weitaus ſympathiſcheſte.) In \textcolor{pink}{Salzburg}{}\ledrightnote{\textcolor{pink}{Salzburg}} möcht ich bis mindeſtens 21., 22.
               bleiben; von dort fahren wir aller Wahrſcheinlichkeit direct nach \textcolor{pink}{Wien}{}\ledrightnote{\textcolor{pink}{Wien}}.\pend
           \pstart
           Telegrafiren Sie bitte Ihre Entſcheidg, ev. auch wo Sie in \textcolor{pink}{Salzb.}{}\ledrightnote{\textcolor{pink}{Salzburg}} zu {\pb}wohnen gedenken, und ob Sie nicht
               vielleicht von Freitag bis So{\geminationn}tag in \textcolor{pink}{\textsc{Lueg}}{}\ledrightnote{\textcolor{pink}{Lueg am Wolfgangsee}} bleiben und mir hier den \textcolor{green}{Grafen \textsc{Ch}.}{}\ledrightnote{\textcolor{green}{Der Graf von Charolais. Ein Trauerspiel}} vorleſen möchten.\pend
           \pstart
           Für alle Fälle hoff ich ſind wir noch ein paar Tage beiſammen.\pend
           \pstart
           Herzlichſt Ihr{\\[\baselineskip]}\spacefill\mbox{A.}\pend
           \leftskip=0em{}\pstart
           \noindent{}Grüße von \textcolor{pink}{Gaſthof}{}\ledrightnote{→\textcolor{pink}{Hotel und Pension Lueg}} zu \textcolor{pink}{Villa}{}\ledrightnote{→\textcolor{pink}{Villa Frühling}}.\pend
           \endnumbering\briefempfaengerindex{Beer-Hofmann, Richard@\textsc{Beer-Hofmann, Richard}!zzzSchnitzler, Arthur@\emph{von Arthur Schnitzler}!1904-09-141@{14. 9. 1904}|)be}\mylabel{h}  \normalsize

\doendnotes{C}
\bigskip
\vfill

\clearpage

\footnotesize

\lohead{\textsc{register}}

% Definiere theindex-Environment komplett neu ohne reledmac
\makeatletter
\renewenvironment{theindex}{%
  \section*{\indexname}%
  \setlength{\parindent}{0pt}%
  \setlength{\parskip}{0pt plus 0.3pt}%
  \let\item\@idxitem
}{%
  \clearpage
}
\makeatother

\IfFileExists{\jobname-pw.ind}{\input{\jobname-pw.ind}}{}

\end{document}

      