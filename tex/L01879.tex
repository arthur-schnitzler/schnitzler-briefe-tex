%% latex-korrekturansicht-vorspann.tex
%% Vorspann für die Korrekturansicht.
%% Lädt die gemeinsame Datei latex-vorspann.tex mit gesetztem Schalter.

\newif\ifkorrekturansicht
\korrekturansichttrue

\input{../tex-inputs/latex-vorspann}


               \section[Max Burckhard an Arthur Schnitzler, 14. 10. 1909]{ Max Burckhard an Arthur Schnitzler, 14. 10. 1909}\nopagebreak\mylabel{v}\rehead{ }\normalsize\beginnumbering\briefempfaengerindex{Schnitzler, Arthur@\textsc{Schnitzler, Arthur}!zzzBurckhard, Max Eugen@\emph{von Max Eugen Burckhard}!1909-10-141@{14. 10. 1909}|(be} \toendnotes[C]{\smallbreak\pagebreak[2]} \Standort{CUL, Schnitzler, B 20.}
\physDesc{Telegramm
\newline{}Handschrift einer Schreibkraft: Bleistift, deutsche Kurrent\newline{}Versand: »\noindent{}\textcolor{pink}{Wien} tel \textcolor{gray}{\textbf{Nr.}} 729 \textcolor{gray}{\textbf{Taxw.}} 40 \textcolor{gray}{\textbf{Ch.{\dots})
                                          aufgegeben am}}{ }14\textcolor{gray}{\textbf{/}}10 \textcolor{gray}{\textbf{19}}{[}0{]}9{ }\textcolor{gray}{\textbf{um}}{ }11 \textcolor{gray}{\textbf{Uhr}} 40 \textcolor{gray}{\textbf{M.}}\textcolor{gray}{V}\textcolor{gray}{\textbf{Mittag}}\textcolor{gray}{\textbf{.}}« 
\newline{}Schnitzler: mit Bleistift datiert: »14/X 09« }\toendnotes[C]{\smallbreak}\pstart
           \noindent{}{\pb}Ich hatte ſie ſchon ſelbſt bitten wollen
               bin nur in der Hetzjagd noch nicht in die \textcolor{pink}{Spöttelgaſſe}{}\ledrightnote{\textcolor{pink}{Edmund-Weiß-Gasse}} gekommen. ich teile Ihnen die \label{K_L01879_1v}\edtext{directoriale Zuſtimmung}{\lemma{\textnormal{\emph{directoriale Zuſtimmung}}}\Cendnote{\textnormal{zur Teilnahme an der Generalprobe von \emph{\textcolor{green}{Jene Asra}} am 15. 10. 1909}}}\label{K_L01879_1h} mit und meine herzlichſte Freude mit Gruß
                  \spacefill\mbox{Doktor Burckhard +}\pend
           \endnumbering\briefempfaengerindex{Schnitzler, Arthur@\textsc{Schnitzler, Arthur}!zzzBurckhard, Max Eugen@\emph{von Max Eugen Burckhard}!1909-10-141@{14. 10. 1909}|)be}\mylabel{h}  \normalsize

\doendnotes{C}
\bigskip
\vfill

\clearpage

\footnotesize

\lohead{\textsc{register}}

% Definiere theindex-Environment komplett neu ohne reledmac
\makeatletter
\renewenvironment{theindex}{%
  \section*{\indexname}%
  \setlength{\parindent}{0pt}%
  \setlength{\parskip}{0pt plus 0.3pt}%
  \let\item\@idxitem
}{%
  \clearpage
}
\makeatother

\IfFileExists{\jobname-pw.ind}{\input{\jobname-pw.ind}}{}

\end{document}

      