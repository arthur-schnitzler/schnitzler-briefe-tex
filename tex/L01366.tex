%% latex-korrekturansicht-vorspann.tex
%% Vorspann für die Korrekturansicht.
%% Lädt die gemeinsame Datei latex-vorspann.tex mit gesetztem Schalter.

\newif\ifkorrekturansicht
\korrekturansichttrue

\input{../tex-inputs/latex-vorspann}


               \section[Hugo von Hofmannsthal an Arthur Schnitzler, 28. 1. 1904]{ Hugo von Hofmannsthal an Arthur Schnitzler, 28. 1. 1904}\nopagebreak\mylabel{v}\rehead{ }\normalsize\beginnumbering\briefempfaengerindex{Schnitzler, Arthur@\textsc{Schnitzler, Arthur}!zzzHofmannsthal, Hugo von@\emph{von Hugo von Hofmannsthal}!1904-01-281@{28. 1. 1904}|(be} \toendnotes[C]{\smallbreak\pagebreak[2]} \Standort{CUL, Schnitzler, B 43.}
\physDesc{Postkarte
\newline{}Handschrift: schwarze Tinte, deutsche Kurrent\newline{}Versand: 1) Stempel: »\nobreak{}\oindex{Bahnhof@\textbf{Bahnhof}, \emph{Bahnhofsgebäude (K.BHF)}|pwk}Venezia Ferrovia, 28{[}-1{]}-04\nobreak{}«.  2) Stempel: »\nobreak{}\oindex{XVIII., Waehring@\textbf{XVIII., Währing}, \emph{Bezirk (A.BZK)}|pwk}18/1 Wien, 30. 1. 04, 8.V, Bestellt\nobreak{}«. 
\newline{}Schnitzler: mit Bleistift datiert: »30. 1. 903« \newline{}Ordnung: 1) mit Bleistift von unbekannter Hand nummeriert: »\strikeout{\textcolor{gray}{248}}« 2) mit Bleistift von unbekannter Hand nummeriert:
                                    »213«}\buchAbdrucke{\weitereDrucke{Hugo von Hofmannsthal, Arthur Schnitzler: \emph{Briefwechsel}. Hg. Therese Nickl und Heinrich Schnitzler. Frankfurt am Main: \emph{S. Fischer} 1964, S. 182.} }\pstart{}{\pb}\textsc{Herrn D\textsuperscript{r} Arthur Schnitzler}\pend{}\pstart{}\textcolor{pink}{\textsc{Wien}}{}\ledrightnote{\textcolor{pink}{Wien}}\pend{}\pstart{}\textcolor{pink}{\textsc{XVIII Spöttelgasse 7}}{}\ledrightnote{\textcolor{pink}{Edmund-Weiß-Gasse}}\pend{}\pstart{}neben \textcolor{pink}{\textsc{Türkenschanz}park}{}\ledrightnote{\textcolor{pink}{Türkenschanzpark}}\pend{}\pstart{}\textsc{\textcolor{pink}{Austria}{}\ledrightnote{\textcolor{pink}{Österreich}}}\pend{}{\bigskip}\pstart
           \noindent{}{\pb}Ich bin Samstag{ }früh in \textcolor{pink}{Rodaun}{}\ledrightnote{\textcolor{pink}{Rodaun}}. Es wäre doch ſehr ſchön
               wenn man ſich noch vor Eurer Abreiſe, die ich nahe vermuthe, ſehen würde, vielleicht
               kommt Ihr Sonntag zu uns nachtmahlen, ſo gegen 6 Uhr ſchon ko{\geminationm}end!!\pend
           \pstart
           Ihr{\\[\baselineskip]}\spacefill\mbox{Hugo}\pend
           \leftskip=0em{}\endnumbering\briefempfaengerindex{Schnitzler, Arthur@\textsc{Schnitzler, Arthur}!zzzHofmannsthal, Hugo von@\emph{von Hugo von Hofmannsthal}!1904-01-281@{28. 1. 1904}|)be}\mylabel{h}  \normalsize

\doendnotes{C}
\bigskip
\vfill

\clearpage

\footnotesize

\lohead{\textsc{register}}

% Definiere theindex-Environment komplett neu ohne reledmac
\makeatletter
\renewenvironment{theindex}{%
  \section*{\indexname}%
  \setlength{\parindent}{0pt}%
  \setlength{\parskip}{0pt plus 0.3pt}%
  \let\item\@idxitem
}{%
  \clearpage
}
\makeatother

\IfFileExists{\jobname-pw.ind}{\input{\jobname-pw.ind}}{}

\end{document}

      