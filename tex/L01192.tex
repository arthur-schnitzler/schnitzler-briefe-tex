%% latex-korrekturansicht-vorspann.tex
%% Vorspann für die Korrekturansicht.
%% Lädt die gemeinsame Datei latex-vorspann.tex mit gesetztem Schalter.

\newif\ifkorrekturansicht
\korrekturansichttrue

\input{../tex-inputs/latex-vorspann}


               \section[Hermann Bahr an Arthur Schnitzler, 30. 12. {[}1901{]}]{ Hermann Bahr an Arthur Schnitzler, 30. 12. {[}1901{]}}\nopagebreak\mylabel{v}\rehead{ }\normalsize\beginnumbering\briefempfaengerindex{Schnitzler, Arthur@\textsc{Schnitzler, Arthur}!zzzBahr, Hermann@\emph{von Hermann Bahr}!1901-12-301@{30. 12. 1901}|(be} \toendnotes[C]{\smallbreak\pagebreak[2]} \Standort{CUL, Schnitzler, B 5b.}
\physDesc{Brief, 1 Blatt, 4 Seiten
\newline{}Handschrift: schwarze Tinte, deutsche Kurrent
\newline{}Schnitzler: mit Bleistift die Jahreszahl »901« ergänzt \newline{}Ordnung: mit Bleistift von unbekannter Hand nummeriert:
                                    »84« }\buchAbdrucke{\weitereDrucke{Hermann Bahr, Arthur Schnitzler: \emph{Briefwechsel, Aufzeichnungen, Dokumente (1891–1931)}. Hg. Kurt Ifkovits und Martin Anton Müller. Göttingen: \emph{Wallstein} 2018, S. 220–221.} }\toendnotes[C]{\smallbreak}\pstart
           \noindent{}\centering{}{\pb}\textcolor{gray}{\textbf{\textcolor{brown}{Redaktion des Neuen Wiener Tagblatt}{}\ledrightnote{\textcolor{brown}{Neues Wiener Tagblatt}}}}\pend
           \pstart
           \noindent{}\centering{}\textcolor{gray}{\textbf{\textsc{\textcolor{pink}{Wien, I., Rothenturmstrasse,
                        Steyrerhof}{}\ledrightnote{\textcolor{pink}{Steyrerhof}}.}}}\pend
           \pstart
           \noindent{}\centering{}\textcolor{gray}{\textbf{Telegramm-Adresse: \textcolor{brown}{Tagblatt}{}\ledrightnote{\textcolor{brown}{Neues Wiener Tagblatt}},
                        \textcolor{pink}{Steyrerhof, Wien}{}\ledrightnote{\textcolor{pink}{Steyrerhof}}. – Telephon Nr. 384.
                     Staats-Telephon Nr. 36.}}\pend
           \pstart
           30. 12.\pend
           \pstart\center{}Lieber Arthur!\pend\pstart
           Danke ſehr für Deine liebe Karte. Du könnteſt mir allerdings in \textcolor{pink}{Berlin}{}\ledrightnote{\textcolor{pink}{Berlin}} einen ſehr, ſehr großen Dienst erweiſen, wenn Du
               gelegentlich mit \textcolor{blue}{Brahm}{}\ledrightnote{\textcolor{blue}{Otto Brahm}} über mich ſprechen und ihm
               klar machen würdeſt, daß ich, bei allem, was man gegen mich ſagen kann, doch
               ſchließlich auch Jemand bin und daß ich gern in ein, wenn auch kühles, doch
               anſtändiges Verhältnis gegenſeitiger Duldung und beding{\pb}ter Anerkennung \introOben{}zu ihm\introOben{}
               kommen möchte. Ich leide ſehr unter meiner Erfolgloſigkeit in \textcolor{pink}{Deutſchland}{}\ledrightnote{\textcolor{pink}{Deutschland}} und bin ſchon ſo beſcheiden geworden, daß ich es als
               einen großen Erfolg empfinden würde, wenn er ſich nur entschließen könnte, ein Stück
               von mir anzunehmen und aufzuführen, meinetwegen in der ſchlechteſten Zeit, weil es
               mir dabei gar nicht auf die Tantièmen ankommt, ſondern auf den »literarischen
               Stempel«, {\pb}den nun das \textcolor{pink}{Deutſche Theater}{}\ledrightnote{\textcolor{pink}{Deutsches Theater Berlin}} einmal ſeinen Autoren gibt und der mir noch immer fehlt,
               und darauf, von ſeiner »Clique« ernſt genommen zu werden. Er hat mir über den
                  »\label{K_L01192_1v}\edtext{\textcolor{green}{Krampus}{}\ledrightnote{\textcolor{green}{Der Krampus}}}{\lemma{\textnormal{\emph{Krampus}}}\Cendnote{\textnormal{\textcolor{blue}{Hermann Bahr}: \emph{\textcolor{green}{Der Krampus. Lustspiel in drei Aufzügen}}. München: \emph{\textcolor{brown}{Albert Langen}}{ }1902 (vordatiert von Dezember 1901).}}}\label{K_L01192_1h}«
               ſehr anerkennend geſprochen, ihn aber ſchließlich leider doch abgelehnt; ich werde
               ihn nun einladen, der \textcolor{pink}{Hamburger}{}\ledrightnote{\textcolor{pink}{Hamburg}}{ }\label{K_L01192_2v}\edtext{Première}{\lemma{\textnormal{\emph{Première}}}\Cendnote{\textnormal{Letztlich erfolgte die Aufführung in \textcolor{pink}{Hamburg} am 14. 1. 1902 unter dem Titel \emph{\textcolor{green}{Der Herr Hofrat}}.}}}\label{K_L01192_2h} (am 12 oder 13 Januar) beizuwohnen;
               freilich ohne viel Hoffnung, \strikeout{\textcolor{gray}{ohne}} ihn noch {\pb}umzuſtimmen. Aber vielleicht
               bringſt Du ihn doch ſo weit, daß er ſich, wenn ich ihm wieder ein Stück ſchicke, es
               wenigſtens mit nicht im Vorhinein feindlichen Augen anſieht.\pend
           \pstart
           Aber bitte, thu das nur, wenn es ſich leicht machen läßt, ohne Dir unbequem zu
               ſein.\pend
           \pstart
           Ich bin rieſig neugierig auf \label{K_L01192_3v}\edtext{Samſtag}{\lemma{\textnormal{\emph{Samſtag}}}\Cendnote{\textnormal{Uraufführung von \emph{\textcolor{green}{Lebendige Stunden}} am 4. 1. 1902 im \emph{\textcolor{brown}{Lessingtheater}} in \textcolor{pink}{Berlin}}}}\label{K_L01192_3h}; mehr
               auszuſprechen verbietet mir mein Aberglaube.\pend
           \pstart
           Herzlichſt{\\[\baselineskip]}Dein alter{\\[\baselineskip]}\spacefill\mbox{HermannB}\pend
           \leftskip=0em{}\pstart
           \noindent{}\textsc{Prost Neujahr!}\pend
           \pstart
           \label{T_L01192_1v}\edtext{\label{K_L01192_4v}\edtext{Den \textcolor{blue}{Novelli}{}\ledrightnote{\textcolor{blue}{Ermete Novelli}}, der über den »\textcolor{green}{Kakadu}{}\ledrightnote{\textcolor{green}{Der grüne Kakadu. Groteske in einem Akt}}« noch
                  immer nichts hören ließ, habe ich gestern \substVorne{}\textsuperscript{d}\substDazwischen{}D\substHinten{}ringend gemahnt.}{\lemma{\textnormal{\emph{Den … gemahnt.}}}\Cendnote{\textnormal{In den
                     Korrespondenzstücken, die von \textcolor{blue}{Novelli} im
                     Nachlass \textcolor{blue}{Bahrs} überliefert sind, findet
                     sich darüber kein näherer Aufschluss.}}}\label{K_L01192_4h}}{\lemma{\textnormal{\emph{Den … gemahnt.}}}\Cendnote{\textnormal{quer am rechten Rand}}}\label{T_L01192_1h}\pend
           \endnumbering\briefempfaengerindex{Schnitzler, Arthur@\textsc{Schnitzler, Arthur}!zzzBahr, Hermann@\emph{von Hermann Bahr}!1901-12-301@{30. 12. 1901}|)be}\mylabel{h}  \normalsize

\doendnotes{C}
\bigskip
\vfill

\clearpage

\footnotesize

\lohead{\textsc{register}}

% Definiere theindex-Environment komplett neu ohne reledmac
\makeatletter
\renewenvironment{theindex}{%
  \section*{\indexname}%
  \setlength{\parindent}{0pt}%
  \setlength{\parskip}{0pt plus 0.3pt}%
  \let\item\@idxitem
}{%
  \clearpage
}
\makeatother

\IfFileExists{\jobname-pw.ind}{\input{\jobname-pw.ind}}{}

\end{document}

      