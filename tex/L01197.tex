%% latex-korrekturansicht-vorspann.tex
%% Vorspann für die Korrekturansicht.
%% Lädt die gemeinsame Datei latex-vorspann.tex mit gesetztem Schalter.

\newif\ifkorrekturansicht
\korrekturansichttrue

\input{../tex-inputs/latex-vorspann}


               \section[Hermann Bahr an Arthur Schnitzler, 9. 1. 1902]{ Hermann Bahr an Arthur Schnitzler, 9. 1. 1902}\nopagebreak\mylabel{v}\rehead{ }\normalsize\beginnumbering\briefempfaengerindex{Schnitzler, Arthur@\textsc{Schnitzler, Arthur}!zzzBahr, Hermann@\emph{von Hermann Bahr}!1902-01-091@{9. 1. 1902}|(be} \toendnotes[C]{\smallbreak\pagebreak[2]} \Standort{CUL, Schnitzler, B 5b.}
\physDesc{Brief, 1 Blatt, 2 Seiten
\newline{}Handschrift: schwarze Tinte, deutsche Kurrent
\newline{}Schnitzler: mit Bleistift die Jahreszahl »902« ergänzt \newline{}Ordnung: mit Bleistift von unbekannter Hand nummeriert:
                                    »85« }\buchAbdrucke{\weitereDrucke{Hermann Bahr, Arthur Schnitzler: \emph{Briefwechsel, Aufzeichnungen, Dokumente (1891–1931)}. Hg. Kurt Ifkovits und Martin Anton Müller. Göttingen: \emph{Wallstein} 2018, S. 223.} }\toendnotes[C]{\smallbreak}\pstart
           \noindent{}\centering{}{\pb}\textcolor{gray}{\textbf{\textcolor{brown}{Redaktion des Neuen Wiener Tagblatt}{}\ledrightnote{\textcolor{brown}{Neues Wiener Tagblatt}}}}\pend
           \pstart
           \noindent{}\centering{}\textcolor{gray}{\textbf{\textsc{\textcolor{pink}{Wien, I., Rothenturmstrasse,
                        Steyrerhof}{}\ledrightnote{\textcolor{pink}{Steyrerhof}}.}}}\pend
           \pstart
           \noindent{}\centering{}\textcolor{gray}{\textbf{Telegramm-Adresse: \textcolor{brown}{Tagblatt}{}\ledrightnote{\textcolor{brown}{Neues Wiener Tagblatt}},
                        \textcolor{pink}{Steyrerhof, Wien}{}\ledrightnote{\textcolor{pink}{Steyrerhof}}. – Telephon Nr. 384.
                     Staats-Telephon Nr. 36.}}\pend
           \pstart
           \raggedleft{}9/I\pend
           \pstart\center{}Lieber Arthur!\pend\pstart
           Eben erfahre ich von meinem Sendboten, der bei \textcolor{blue}{Schlenther}{}\ledrightnote{\textcolor{blue}{Paul Schlenther}} war\pend
           \pstart
           1) Schnitzler bekommt den \textcolor{brown}{Grillparzerpreis}{}\ledrightnote{\textcolor{brown}{Franz-Grillparzer-Preis}}{ }\uline{nicht};\pend
           \pstart
           2) \textcolor{blue}{Schlenther}{}\ledrightnote{\textcolor{blue}{Paul Schlenther}} bezeichnet es als abſolut falſch,
               wenn man meine, Schnitzler ſei durch die \textcolor{green}{Guſtl}{}\ledrightnote{\textcolor{green}{Lieutenant Gustl. Novelle}}-Affaire \textcolor{pink}{burgtheaterunfähig}{}\ledrightnote{\textcolor{pink}{Burgtheater}} geworden;
               diese Auffaſſung bestehe weder in der Intendanz noch bei ihm ſelbſt; die »\textcolor{green}{Lebendigen Stunden}{}\ledrightnote{\textcolor{green}{Lebendige Stunden. Vier Einakter}}« kenne er leider nicht.\pend
           \pstart
           Ich \label{K_L01197_1v}\edtext{fahre in einer Stunde ab}{\lemma{\textnormal{\emph{fahre in einer Stunde ab}}}\Cendnote{\textnormal{zur Premiere von \emph{\textcolor{green}{Der Krampus}} in \textcolor{pink}{Hamburg}}}}\label{K_L01197_1h}. Überleg Dir, bis {\pb}ich
               wiederkomm’, ob ich nicht doch mit den \textcolor{green}{Stücken}{}\ledrightnote{→\textcolor{green}{Die Frau mit dem Dolche}{\newline}→\textcolor{green}{Literatur}{\newline}→\textcolor{green}{Lebendige Stunden}} reſolut hingehen darf.\pend
           \pstart
           Herzlichſt{\\[\baselineskip]}\spacefill\mbox{Hermann}\pend
           \leftskip=0em{}\endnumbering\briefempfaengerindex{Schnitzler, Arthur@\textsc{Schnitzler, Arthur}!zzzBahr, Hermann@\emph{von Hermann Bahr}!1902-01-091@{9. 1. 1902}|)be}\mylabel{h}  \normalsize

\doendnotes{C}
\bigskip
\vfill

\clearpage

\footnotesize

\lohead{\textsc{register}}

% Definiere theindex-Environment komplett neu ohne reledmac
\makeatletter
\renewenvironment{theindex}{%
  \section*{\indexname}%
  \setlength{\parindent}{0pt}%
  \setlength{\parskip}{0pt plus 0.3pt}%
  \let\item\@idxitem
}{%
  \clearpage
}
\makeatother

\IfFileExists{\jobname-pw.ind}{\input{\jobname-pw.ind}}{}

\end{document}

      