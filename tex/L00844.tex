%% latex-korrekturansicht-vorspann.tex
%% Vorspann für die Korrekturansicht.
%% Lädt die gemeinsame Datei latex-vorspann.tex mit gesetztem Schalter.

\newif\ifkorrekturansicht
\korrekturansichttrue

\input{../tex-inputs/latex-vorspann}


               \section[Richard Beer-Hofmann und Hugo von Hofmannsthal an Arthur Schnitzler, 5. 9. 1898]{ Richard Beer-Hofmann und Hugo von Hofmannsthal an Arthur Schnitzler,
               5. 9. 1898}\nopagebreak\mylabel{v}\rehead{ }\normalsize\beginnumbering\briefempfaengerindex{Schnitzler, Arthur@\textsc{Schnitzler, Arthur}!zzzHofmannsthal, Hugo von@\emph{von Hugo von Hofmannsthal}!1898-09-051@{5. 9. 1898}|(be}\briefempfaengerindex{Schnitzler, Arthur@\textsc{Schnitzler, Arthur}!zzzBeer-Hofmann, Richard@\emph{von Richard Beer-Hofmann}!1898-09-051@{5. 9. 1898}|(be} \toendnotes[C]{\smallbreak\pagebreak[2]} \Standort{CUL, Schnitzler, B 8.}
\physDesc{Bildpostkarte
\newline{}Handschrift Richard Beer-Hofmann: Bleistift, lateinische Kurrent\newline{}Handschrift Hugo von Hofmannsthal: Bleistift, lateinische Kurrent\newline{}Versand: 1) Stempel: »\nobreak{}\oindex{Lugano@\textbf{Lugano}, \emph{Besiedelter Ort (A.BSO)}|pwk}Lugano, 5. IX. 98, IX\nobreak{}«.  2) Stempel: »\nobreak{}\oindex{IX., Alsergrund@\textbf{IX., Alsergrund}, \emph{Bezirk (A.BZK)}|pwk}Wien 9/3 72, 7. 9. 98, 8.N, Bestellt\nobreak{}«. \newline{}Ordnung: mit Bleistift von unbekannter Hand nummeriert:
                                    »122« }\buchAbdrucke{\weitereDrucke{Arthur Schnitzler, Richard Beer-Hofmann: \emph{Briefwechsel 1891–1931}. Hg. Konstanze Fliedl. Wien, Zürich: \emph{Europaverlag} 1992, S. 124–125.} }\toendnotes[C]{\smallbreak}\pstart{}{\pb}Herrn Arthur D\textsuperscript{r} Schnitzler\pend{}\pstart{}\textcolor{pink}{Wien}{}\ledrightnote{\textcolor{pink}{Wien}}\pend{}\pstart{}\strikeout{Wien} im IX.\pend{}\pstart{}\textcolor{pink}{Frankgasse 1}{}\ledrightnote{\textcolor{pink}{Frankgasse}}\pend{}\pstart{}\textcolor{pink}{Autriche}{}\ledrightnote{\textcolor{pink}{Österreich}}\pend{}\pstart{}\textcolor{pink}{Austria}{}\ledrightnote{\textcolor{pink}{Österreich}}\pend{}{\bigskip}\pstart
           \noindent{}\centering{}\textcolor{gray}{\textbf{{\pb}\textcolor{pink}{Villa Ceresio}{}\ledrightnote{\textcolor{pink}{Villa Ceresio}}}}\pend
           \pstart
           \noindent{}\centering{}\textcolor{gray}{\textbf{\textcolor{pink}{Hôtel du Park}{}\ledrightnote{\textcolor{pink}{Hôtel du Parc}}}}\pend
           \pstart
           \noindent{}\centering{}\textcolor{gray}{\textbf{\textcolor{pink}{Lugano}{}\ledrightnote{\textcolor{pink}{Lugano}}}}\pend
           \pstart
           \noindent{}\centering{}\textcolor{gray}{\textbf{\textcolor{pink}{Villa Beauséjour}{}\ledrightnote{\textcolor{pink}{Villa Beauséjour}}}}\pend
           \pstart
           \noindent{}\centering{}\textcolor{gray}{\textbf{\textcolor{pink}{Belvédère}{}\ledrightnote{\textcolor{pink}{Belvédère}}}}\pend
           \pstart
           {\pb}Lieber Arthur, ich hab mir den größeren Thurm geno{\geminationm}en. Wir fahren Mittwoch von \textcolor{pink}{Mailand}{}\ledrightnote{\textcolor{pink}{Mailand}} hin um die beiden ab\introOben{}zu\introOben{}holen –
                  \textcolor{blue}{Hugo}{}\ledrightnote{\textcolor{blue}{Hugo von Hofmannsthal}} hat heute in 2 Operationen (Vor ×
               Nachm.) den »\textcolor{green}{Götterlibling}{}\ledrightnote{\textcolor{green}{Der Tod Georgs}}« (jetzt heißt er »\textcolor{green}{Der Tod Georgs}{}\ledrightnote{\textcolor{green}{Der Tod Georgs}}«) erlitten. Vorher hat er sich die
                  Hühneraugen\footnote{\noindent{}Der \textcolor{blue}{Hugo}{}\ledrightnote{\textcolor{blue}{Hugo von Hofmannsthal}} behauptet »Hühneraugen« kann man
                     gar nicht lesen. Dazu ist doch der »\textcolor{green}{Secolo}{}\ledrightnote{\textcolor{green}{Il Secolo}}«
                     da. \spacefill\mbox{R.}\par\noindent \label{T_L00844_1v}\toendnotes[C]{\begin{minipage}[t]{4em}{\makebox[3.6em][r]{\tiny{Fußnote}}}\end{minipage}\begin{minipage}[t]{\dimexpr\linewidth-4em}\textit{Der Hugo sagt das versteht kein Mensch. Ich mein zum lesen
                     ist der Secolo da.}\,{]} über die Abbildung geschrieben\end{minipage}\par}Der \textcolor{blue}{Hugo}{}\ledrightnote{\textcolor{blue}{Hugo von Hofmannsthal}} sagt das versteht kein Mensch. Ich mein zum lesen
                     ist der \textcolor{green}{Secolo}{}\ledrightnote{\textcolor{green}{Il Secolo}} da.\label{T_L00844_1h}} schneiden lassen. Diese Operation gelang auch. Der \textcolor{green}{Götterl.}{}\ledrightnote{\textcolor{green}{Der Tod Georgs}} ist ein »\label{K_L00844_1v}\edtext{meschugener Fisch}{\lemma{\textnormal{\emph{meschugener Fisch}}}\Cendnote{\textnormal{stehender Ausdruck in
                  der jüdischen Kultur, sinngemäß: verrückter Kerl}}}\label{K_L00844_1h}« darin scheint sich
                  \textcolor{blue}{Hugo}{}\ledrightnote{\textcolor{blue}{Hugo von Hofmannsthal}}s Urtheil zu resumiren.
                  \spacefill\mbox{R.}\pend
           \pstart
           \noindent{}{[}hs. Hofmannsthal:{]} \label{T_L00844_2v}\edtext{Das Schwein lasst mir keinen Platz und
               sagt mir auch keinen Stoff.}{\lemma{\textnormal{\emph{Das … Stoff.}}}\Cendnote{\textnormal{am oberen Rand
                  auf dem Kopf}}}\label{T_L00844_2h}\pend
           \pstart \label{T_L00844_3v}\edtext{Herzlich Hugo kleinerer
                  Thurmbesitzer}{\lemma{\textnormal{\emph{Herzlich … Thurmbesitzer}}}\Cendnote{\textnormal{quer am linken
                  Rand}}}\label{T_L00844_3h}\pend{}\pstart
           \noindent{}{[}hs. Beer-Hofmann:{]} Er will \label{T_L00844_4v}\edtext{i{\geminationm}er einen Stoff von mir haben weil ich ein alter Jud
                  bin.}{\lemma{\textnormal{\emph{ier … bin.}}}\Cendnote{\textnormal{diagonal über den Text
                  geschrieben}}}\label{T_L00844_4h}\pend
           \endnumbering\briefempfaengerindex{Schnitzler, Arthur@\textsc{Schnitzler, Arthur}!zzzHofmannsthal, Hugo von@\emph{von Hugo von Hofmannsthal}!1898-09-051@{5. 9. 1898}|)be}\briefempfaengerindex{Schnitzler, Arthur@\textsc{Schnitzler, Arthur}!zzzBeer-Hofmann, Richard@\emph{von Richard Beer-Hofmann}!1898-09-051@{5. 9. 1898}|)be}\mylabel{h}  \normalsize

\doendnotes{C}
\bigskip
\vfill

\clearpage

\footnotesize

\lohead{\textsc{register}}

% Definiere theindex-Environment komplett neu ohne reledmac
\makeatletter
\renewenvironment{theindex}{%
  \section*{\indexname}%
  \setlength{\parindent}{0pt}%
  \setlength{\parskip}{0pt plus 0.3pt}%
  \let\item\@idxitem
}{%
  \clearpage
}
\makeatother

\IfFileExists{\jobname-pw.ind}{\input{\jobname-pw.ind}}{}

\end{document}

      