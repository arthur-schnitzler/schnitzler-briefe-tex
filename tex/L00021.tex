%% latex-korrekturansicht-vorspann.tex
%% Vorspann für die Korrekturansicht.
%% Lädt die gemeinsame Datei latex-vorspann.tex mit gesetztem Schalter.

\newif\ifkorrekturansicht
\korrekturansichttrue

\input{../tex-inputs/latex-vorspann}


               \section[Arthur Schnitzler an Richard Beer-Hofmann, 24. 6. 1891]{ Arthur Schnitzler an Richard Beer-Hofmann,
               24. 6. 1891}\nopagebreak\mylabel{v}\rehead{ }\normalsize\beginnumbering\briefempfaengerindex{Beer-Hofmann, Richard@\textsc{Beer-Hofmann, Richard}!zzzSchnitzler, Arthur@\emph{von Arthur Schnitzler}!1891-06-241@{24. 6. 1891}|(be} \toendnotes[C]{\smallbreak\pagebreak[2]} \Standort{YCGL, MSS 31.}
\physDesc{Postkarte
\newline{}Handschrift: Bleistift, deutsche Kurrent\newline{}Versand: 1) Rohrpost 2) Stempel: »\nobreak{}Wien Kärntnerring, 24 6 91, 11\nobreak{}«. 3) Stempel: »\nobreak{}Wien Landstr\textcolor{gray}{.}
                              Hauptstr., 24/6 91, 1–2 N\nobreak{}«. }\toendnotes[C]{\smallbreak}\pstart{}{\pb}\textsc{Herrn Dr R. Beer Hofmann}\pend{}\pstart{}\textsc{\textcolor{pink}{Wien}{}\ledrightnote{\textcolor{pink}{Wien}}}\pend{}\pstart{}\textsc{\textcolor{pink}{III. Seidlgasse 30}{}\ledrightnote{\textcolor{pink}{Seidlgasse}}}. \pend{}{\bigskip}\pstart
           \noindent{}{\pb}Lieber Richard, ich habe einen völlig freien Abend vor mir, we{\geminationn} es Ihnen alſo recht iſt, treffen wir uns. Haben Sie
               die Abſicht, \label{K_L00021_1v}\edtext{eventuell aufs Land}{\lemma{\textnormal{\emph{eventuell aufs Land}}}\Cendnote{\textnormal{Schnitzler fuhr mit \textcolor{blue}{Beer-Hofmann} und \textcolor{blue}{Salten}
                  zur \textcolor{pink}{Türkenschanze}.}}}\label{K_L00021_1h}, ſo holen Sie mich
               vielleicht zwiſchen 5 u ½ 6 ab – Erſcheinen Sie nicht, ſo
               werd ich \uline{\textsc{ca}}{ }6, 7 im \textcolor{pink}{\textsc{Griensteidl}}{}\ledrightnote{\textcolor{pink}{Café Griensteidl}}{ }ſein.\pend
           \pstart Herzlich grüß\textcolor{gray}{end} Ihr\spacefill\mbox{Arth Schnitzler}\pend{}\endnumbering\briefempfaengerindex{Beer-Hofmann, Richard@\textsc{Beer-Hofmann, Richard}!zzzSchnitzler, Arthur@\emph{von Arthur Schnitzler}!1891-06-241@{24. 6. 1891}|)be}\mylabel{h}  \normalsize

\doendnotes{C}
\bigskip
\vfill

\clearpage

\footnotesize

\lohead{\textsc{register}}

% Definiere theindex-Environment komplett neu ohne reledmac
\makeatletter
\renewenvironment{theindex}{%
  \section*{\indexname}%
  \setlength{\parindent}{0pt}%
  \setlength{\parskip}{0pt plus 0.3pt}%
  \let\item\@idxitem
}{%
  \clearpage
}
\makeatother

\IfFileExists{\jobname-pw.ind}{\input{\jobname-pw.ind}}{}

\end{document}

      