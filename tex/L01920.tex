%% latex-korrekturansicht-vorspann.tex
%% Vorspann für die Korrekturansicht.
%% Lädt die gemeinsame Datei latex-vorspann.tex mit gesetztem Schalter.

\newif\ifkorrekturansicht
\korrekturansichttrue

\input{../tex-inputs/latex-vorspann}


               \section[Stefan Großmann an Arthur Schnitzler, 31. 3. 1910]{ Stefan Großmann an Arthur Schnitzler, 31. 3. 1910}\nopagebreak\mylabel{v}\rehead{ }\normalsize\beginnumbering\briefempfaengerindex{Schnitzler, Arthur@\textsc{Schnitzler, Arthur}!zzzGrossmann, Stefan@\emph{von Stefan Großmann}!1910-03-311@{31. 3. 1910}|(be} \toendnotes[C]{\smallbreak\pagebreak[2]} \Standort{CUL, Schnitzler, B 34.}
\physDesc{Brief, 1 Blatt, 2 Seiten
\newline{}Handschrift: 1) schwarze Tinte, deutsche Kurrent\hspace{1em}2) schwarze Tinte, lateinische Kurrent (\noindent{}bis einschließlich der
                           Aufzählung der Schauspieler)\hspace{1em}
\newline{}Schnitzler: mit rotem Buntstift eine Unterstreichung \newline{}Ordnung: mit Bleistift von unbekannter Hand nummeriert: »7« }\toendnotes[C]{\smallbreak}\pstart
           \noindent{}{\pb}\textcolor{gray}{\textbf{STEFAN GROSSMANN}}\hfill \textcolor{pink}{WIEN, I.}{}\ledrightnote{\textcolor{pink}{I., Innere Stadt}}, \textcolor{pink}{GRABEN
                        29a}{}\ledrightnote{\textcolor{pink}{Graben}}\pend
           \pstart
           \centering{}31. III. 10\pend
           \pstart{}Sehr verehrter Herr!\pend\pstart
           Aufrichtigen Dank für Ihre gütige Erlaubnis. Der \textcolor{brown}{Verein}{}\ledrightnote{→\textcolor{brown}{Wiener Freie Volksbühne}} (der langsam in eine bürgerliche Breite kommt, es
               gehören ihm heute schon 12000 Mitglieder an) bittet Sie, zu gestatten, dass wir dem
                  \textcolor{blue}{Verleger}{}\ledrightnote{→\textcolor{blue}{Samuel Fischer}} 5{\%} Tantieme zahlen. Reicher sind wir noch nicht.\pend
           \pstart
           Ich verstehe vollkommen, dass Ihnen die Anfügung der »\textcolor{green}{Frage an das Schicksal}{}\ledrightnote{\textcolor{green}{Die Frage an das Schicksal}}« nicht gefällt. Aber die \textcolor{brown}{Neue W\textsuperscript{r} Bühne}{}\ledrightnote{\textcolor{brown}{Neue Wiener Bühne}} behauptet, für den
                  {[}»{]}\textcolor{green}{Puppenspieler}{}\ledrightnote{\textcolor{green}{Der Puppenspieler}}« absolut nicht die Zeit für nöthige
               Proben zu haben. So musste ich, wider besseres Wissen, {\pb}im Interesse der guten Ausarbeitung der »\textcolor{green}{Literatur}{}\ledrightnote{\textcolor{green}{Literatur}}« und »\textcolor{green}{Masken}{}\ledrightnote{\textcolor{green}{Die letzten Masken}}« einwilligen.\pend
           \pstart
           In \textcolor{green}{Literatur}{}\ledrightnote{\textcolor{green}{Literatur}} sind \uline{\textcolor{blue}{Charlé}{}\ledrightnote{\textcolor{blue}{Gustav Charlé}}}, Fr \textcolor{blue}{v. \uline{Linden}}{}\ledrightnote{\textcolor{blue}{Constance von Linden}} (die ausgezeichnet wird), Hr \textcolor{blue}{\uline{Ziegler}}{}\ledrightnote{\textcolor{blue}{Hans Ziegler}}, – in \textcolor{green}{Masken}{}\ledrightnote{\textcolor{green}{Die letzten Masken}} Herr \textcolor{blue}{\uline{Charlé}}{}\ledrightnote{\textcolor{blue}{Gustav Charlé}}, Herr \textcolor{blue}{\uline{Heyse}}{}\ledrightnote{\textcolor{blue}{Emil Heyse}} (\textcolor{green}{Weihgast}{}\ledrightnote{→\textcolor{green}{Die letzten Masken}}) beſchäftigt.\pend
           \pstart
           Gern würde ich Sie einmal als Gaſt bei einer Aufführung des \uline{\textcolor{green}{\textsc{Halben Held}}{}\ledrightnote{\textcolor{green}{Ein halber Held. Tragödie}}} v \textcolor{blue}{\textsc{H Eulenberg}}{}\ledrightnote{\textcolor{blue}{Herbert Eulenberg}} begrüßen, auch deshalb, weil es eine paſſable
               Regiſſeurarbeit von mir iſt. Wollen Sie unſer Gaſt ſein?\pend
           \pstart
           Ich habe die Hoffnung, daſs Sie mich als Regiſſeur noch einmal werden brauchen
               können. – – –\pend
           \pstart
           Mit den beſten Gefühlen{\\[\baselineskip]}\uline{aufrichtig ergeben}:
                  \spacefill\mbox{Stefan Großmann}\pend
           \leftskip=0em{}\endnumbering\briefempfaengerindex{Schnitzler, Arthur@\textsc{Schnitzler, Arthur}!zzzGrossmann, Stefan@\emph{von Stefan Großmann}!1910-03-311@{31. 3. 1910}|)be}\mylabel{h}  \normalsize

\doendnotes{C}
\bigskip
\vfill

\clearpage

\footnotesize

\lohead{\textsc{register}}

% Definiere theindex-Environment komplett neu ohne reledmac
\makeatletter
\renewenvironment{theindex}{%
  \section*{\indexname}%
  \setlength{\parindent}{0pt}%
  \setlength{\parskip}{0pt plus 0.3pt}%
  \let\item\@idxitem
}{%
  \clearpage
}
\makeatother

\IfFileExists{\jobname-pw.ind}{\input{\jobname-pw.ind}}{}

\end{document}

      