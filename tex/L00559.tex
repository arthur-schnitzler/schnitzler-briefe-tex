%% latex-korrekturansicht-vorspann.tex
%% Vorspann für die Korrekturansicht.
%% Lädt die gemeinsame Datei latex-vorspann.tex mit gesetztem Schalter.

\newif\ifkorrekturansicht
\korrekturansichttrue

\input{../tex-inputs/latex-vorspann}


               \section[Arthur Schnitzler an Richard Beer-Hofmann, 2. 7. 1896]{ Arthur Schnitzler an Richard Beer-Hofmann, 2. 7. 1896}\nopagebreak\mylabel{v}\rehead{ }\normalsize\beginnumbering\briefempfaengerindex{Beer-Hofmann, Richard@\textsc{Beer-Hofmann, Richard}!zzzSchnitzler, Arthur@\emph{von Arthur Schnitzler}!1896-07-021@{2. 7. 1896}|(be} \toendnotes[C]{\smallbreak\pagebreak[2]} \Standort{YCGL, MSS 31.}
\physDesc{Postkarte
\newline{}Handschrift: schwarze Tinte, deutsche Kurrent\newline{}Versand: 1) Stempel: »\nobreak{}\oindex{VI., Mariahilf@\textbf{VI., Mariahilf}, \emph{Bezirk (A.BZK)}|pwk}Wien 6/1, 2. 7. 96, 1–2N\nobreak{}«.  2) Stempel: »\nobreak{}\oindex{St. Gilgen@\textbf{St. Gilgen}, \emph{Besiedelter Ort (A.BSO)}|pwk}St. Gilgen, 3 7 96\nobreak{}«. }\buchAbdrucke{\weitereDrucke{Arthur Schnitzler, Richard Beer-Hofmann: \emph{Briefwechsel 1891–1931}. Hg. Konstanze Fliedl. Wien, Zürich: \emph{Europaverlag} 1992, S. 92.} }\pstart{}{\pb}Herrn \textsc{Dr. Richard
                     Beer-Hofmann}\pend{}\pstart{}\textcolor{pink}{\textsc{Fürberg am Wolfgangsee}}{}\ledrightnote{\textcolor{pink}{Fürberg}}\pend{}{\bigskip}\pstart
           \noindent{}{\pb}Lieber Richard, wenn Sie nicht in der Correſpondzkartensti{\geminationm}g ſind, raffen Sie ſich zu einem Brief auf. \textcolor{blue}{Paul}{}\ledrightnote{\textcolor{blue}{Paul Goldmann}} ko{\geminationm}t nach \textcolor{pink}{Dänemark}{}\ledrightnote{\textcolor{pink}{Dänemark}}. Schreiben Sie ihm. Ich reiſe
                  Freitag Abend \textcolor{pink}{Hamburg}{}\ledrightnote{\textcolor{pink}{Hamburg}}. Dort \textsc{post rest} hoff ich Nachricht von Ihnen zu finden. Am
                  7. geht mein Schiff ab. Nach \textcolor{pink}{\textsc{Trondjhem}}{}\ledrightnote{\textcolor{pink}{Trondheim}}{ }ſenden Sie eine \introOben{}(briefl.)\introOben{}
               Nachricht am 9. Juli; eine zweite am 18. Juli. – Telegra{\geminationm}e wiſſen Sie ja. Gehen Sie nicht nach \textcolor{pink}{München}{}\ledrightnote{\textcolor{pink}{München}}? Vielleicht doch mit mir auf der Rückreiſe. –\pend
           \pstart
           Seien Sie vielmals herzlich gegrüßt u grüßen Sie \textcolor{blue}{Paula}{}\ledrightnote{\textcolor{blue}{Paula Beer-Hofmann}}.{\\[\baselineskip]}Ihr \spacefill\mbox{ArthurSch}\pend
           \leftskip=0em{}\endnumbering\briefempfaengerindex{Beer-Hofmann, Richard@\textsc{Beer-Hofmann, Richard}!zzzSchnitzler, Arthur@\emph{von Arthur Schnitzler}!1896-07-021@{2. 7. 1896}|)be}\mylabel{h}  \normalsize

\doendnotes{C}
\bigskip
\vfill

\clearpage

\footnotesize

\lohead{\textsc{register}}

% Definiere theindex-Environment komplett neu ohne reledmac
\makeatletter
\renewenvironment{theindex}{%
  \section*{\indexname}%
  \setlength{\parindent}{0pt}%
  \setlength{\parskip}{0pt plus 0.3pt}%
  \let\item\@idxitem
}{%
  \clearpage
}
\makeatother

\IfFileExists{\jobname-pw.ind}{\input{\jobname-pw.ind}}{}

\end{document}

      