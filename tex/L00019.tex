%% latex-korrekturansicht-vorspann.tex
%% Vorspann für die Korrekturansicht.
%% Lädt die gemeinsame Datei latex-vorspann.tex mit gesetztem Schalter.

\newif\ifkorrekturansicht
\korrekturansichttrue

\input{../tex-inputs/latex-vorspann}


               \section[Friedrich M. Fels an Arthur Schnitzler, 18. 6. 1891]{ Friedrich M. Fels an Arthur Schnitzler,
                    18. 6. 1891}\nopagebreak\mylabel{v}\rehead{ }\normalsize\beginnumbering\briefempfaengerindex{Schnitzler, Arthur@\textsc{Schnitzler, Arthur}!zzzFels, Friedrich Michael@\emph{von Friedrich Michael Fels}!1891-06-181@{18. 6. 1891}|(be} \toendnotes[C]{\smallbreak\pagebreak[2]} \Standort{DLA, A:Schnitzler, HS.NZ85.1.2956.}
\physDesc{Brief, 1 Blatt, 1 Seite
\newline{}Handschrift: schwarze Tinte, lateinische Kurrent
\newline{}Schnitzler: 1) mit Bleistift beschriftet mit »\textsc{Fels}« 2) mit rotem Buntstift beschriftet mit »\textsc{Fels}«}\toendnotes[C]{\smallbreak}\pstart
           \noindent{}{\pb}\textcolor{gray}{\textbf{»\textcolor{brown}{Moderne
                                Rundſchau}{}\ledrightnote{\textcolor{brown}{Moderne Dichtung/Moderne Rundschau}}«}}\hfill \textcolor{gray}{\textbf{Redaction:}}\pend
           \pstart
           \textcolor{gray}{\textbf{Halbmonatsſchrift}}\hfill \textcolor{gray}{\textbf{\textcolor{pink}{VIII., Buchfeldgaſſe 8}{}\ledrightnote{\textcolor{pink}{Buchfeldgasse}}}}\pend
           \pstart
           \textcolor{gray}{\textbf{Herausgegeben von \textbf{Dr. \textcolor{blue}{J. Joachim}{}\ledrightnote{\textcolor{blue}{Jaques Joachim}}} und \textbf{\textcolor{blue}{E. M. Kafka}{}\ledrightnote{\textcolor{blue}{Eduard Michael Kafka}}}}}\pend
           \pstart
           \textcolor{gray}{\textbf{Verlag von \textbf{\textcolor{brown}{Leopold Weiß}{}\ledrightnote{\textcolor{brown}{Leopold Weiss}}}}}\hfill \textcolor{gray}{\textbf{Adminiſtration:}}\pend
           \pstart
           \raggedleft{}\textcolor{gray}{\textbf{\textcolor{pink}{I., Tuchlauben 7}{}\ledrightnote{\textcolor{pink}{Tuchlauben}}}}\pend
           \pstart
           \raggedleft{}\textcolor{gray}{\textbf{\textcolor{pink}{Wien}{}\ledrightnote{\textcolor{pink}{Wien}}}}\pend
           \pstart
           \raggedleft{}\textcolor{gray}{\textbf{am}}{ }18. Juni \textcolor{gray}{\textbf{189}}1\pend
           \pstart\center{}Lieber Herr Doktor!\pend\pstart
           Haben Sie keine Skizze von 2–3 Druckseiten fertig? Wir brauchen für das nächste
                    Heft unumgänglich eine so kurze, da \label{K_L00019_1v}\edtext{\textcolor{green}{\textcolor{blue}{Held}{}\ledrightnote{\textcolor{blue}{Franz Held}}}{}\ledrightnote{→\textcolor{green}{Ein Pyrrhus-Sieg. Geschichte eines glücklichen Pechvogels}}}{\lemma{\textnormal{\emph{Held}}}\Cendnote{\textnormal{\emph{\textcolor{green}{Ein Pyrrhus-Sieg. Geschichte eines glücklichen
                            Pechvogels}} erschien in sechs Teilen zwischen 15. 5.
                        und 1. 8. 1891.}}}\label{K_L00019_1h} und besonders \label{K_L00019_2v}\edtext{\textcolor{blue}{\textcolor{green}{David}{}\ledrightnote{→\textcolor{green}{Hagars Sohn. Schauspiel in vier Acten}}}{}\ledrightnote{\textcolor{blue}{Jakob Julius David}}}{\lemma{\textnormal{\emph{David}}}\Cendnote{\textnormal{\emph{\textcolor{green}{Hagars Sohn. Schauspiel in vier Acten}}
                        erschien in vier Teilen zwischen 1. 6. und
                            15. 7. 1891.}}}\label{K_L00019_2h} zu viel Raum in Anspruch nehmen;
                    vorrätig haben wir aber nur längere Novelletten. Sie würden uns ausserordentlich
                    verpflichten, we{\geminationn} Sie uns etwas gäben; \textcolor{blue}{Kafka}{}\ledrightnote{\textcolor{blue}{Eduard Michael Kafka}}{ }ſprach von einem \label{K_L00019_3v}\edtext{\textcolor{green}{Märchen}{}\ledrightnote{→\textcolor{green}{Die drei Elixire}}}{\lemma{\textnormal{\emph{Märchen}}}\Cendnote{\textnormal{Unter dem Namen \emph{\textcolor{brown}{Jung Wien}} agierte ein Verein, der sich
                  zumindest zwischen 17. 3. 1891 und
                  5. 5. 1891 wöchentlich in der Weinhandlung
                  \textcolor{pink}{Wieninger} traf. Am 14. 4. 1891
                  las \textcolor{blue}{Schnitzler} dort \emph{\textcolor{green}{Die drei Elixire}} vor. Eine Lesung
                  aus dem Theaterstück \emph{\textcolor{green}{Das Märchen}}, das er gerade schrieb, ist zu diesem Zeitpunkt eher unwahrscheinlich.
               }}}\label{K_L00019_3h}, das Sie bei \textcolor{pink}{Wieninger}{}\ledrightnote{\textcolor{pink}{Joseph G. Wieninger, Weinhandlung}} vorgelesen
                    haben sollen – wohl \label{K_L00019_4v}\edtext{ehe ich dem Kreise
                        angehörte}{\lemma{\textnormal{\emph{ehe … angehörte}}}\Cendnote{\textnormal{\textcolor{blue}{Friedrich M. Fels} wird in \textcolor{blue}{Schnitzler}s
                          \emph{\textcolor{green}{Tagebuch}} erstmals am 21. 4. 1891 erwähnt.}}}\label{K_L00019_4h}.\pend
           \pstart Mit bestem Gruss\pend{}\pstart
           \raggedleft{}\textcolor{gray}{\textbf{\textit{Redaction der »\textcolor{brown}{Modernen
                            Rundſchau}{}\ledrightnote{\textcolor{brown}{Moderne Dichtung/Moderne Rundschau}}.«}}}\pend
           \pstart
           \noindent{}\spacefill\mbox{I. V.\hspace*{1.5em}Friedr. M. Fels}\pend
           \endnumbering\briefempfaengerindex{Schnitzler, Arthur@\textsc{Schnitzler, Arthur}!zzzFels, Friedrich Michael@\emph{von Friedrich Michael Fels}!1891-06-181@{18. 6. 1891}|)be}\mylabel{h}  \normalsize

\doendnotes{C}
\bigskip
\vfill

\clearpage

\footnotesize

\lohead{\textsc{register}}

% Definiere theindex-Environment komplett neu ohne reledmac
\makeatletter
\renewenvironment{theindex}{%
  \section*{\indexname}%
  \setlength{\parindent}{0pt}%
  \setlength{\parskip}{0pt plus 0.3pt}%
  \let\item\@idxitem
}{%
  \clearpage
}
\makeatother

\IfFileExists{\jobname-pw.ind}{\input{\jobname-pw.ind}}{}

\end{document}

      