%% latex-korrekturansicht-vorspann.tex
%% Vorspann für die Korrekturansicht.
%% Lädt die gemeinsame Datei latex-vorspann.tex mit gesetztem Schalter.

\newif\ifkorrekturansicht
\korrekturansichttrue

\input{../tex-inputs/latex-vorspann}


               \section[Arthur Schnitzler an Richard Beer-Hofmann, 22. 6. 1895]{ Arthur Schnitzler an Richard Beer-Hofmann, 22. 6. 1895}\nopagebreak\mylabel{v}\rehead{ }\normalsize\beginnumbering\briefempfaengerindex{Beer-Hofmann, Richard@\textsc{Beer-Hofmann, Richard}!zzzSchnitzler, Arthur@\emph{von Arthur Schnitzler}!1895-06-221@{22. 6. 1895}|(be} \toendnotes[C]{\smallbreak\pagebreak[2]} \Standort{YCGL, MSS 31.}
\physDesc{Brief, 2 Blätter, 7 Seiten, Umschlag
\newline{}Handschrift: 1) Bleistift, deutsche Kurrent\hspace{1em}2) schwarze Tinte, deutsche Kurrent (\noindent{}Umschlag)\hspace{1em}\newline{}Versand: 1) Stempel: »\nobreak{}\oindex{I., Innere Stadt@\textbf{I., Innere Stadt}, \emph{Bezirk (A.BZK)}|pwk}Wien
                                                  {[}1{]}/1, 22. {[}6{]}. 95, 8–9\nobreak{}«.  2) Stempel: »\nobreak{}\oindex{Caslau@\textbf{Caslau}, \emph{Besiedelter Ort (A.BSO)}|pwk}{\pb}Časlau Časlav, 23 / 6 / 95, 8–9\nobreak{}«. }\buchAbdrucke{\weitereDrucke{Arthur Schnitzler, Richard Beer-Hofmann: \emph{Briefwechsel 1891–1931}. Hg. Konstanze Fliedl. Wien, Zürich: \emph{Europaverlag} 1992, S. 75.} }\toendnotes[C]{\smallbreak}\pstart{}{\pb}Herrn kuk. u. a.
                    Lieutenant\pend{}\pstart{}\textsc{Dr. Richard Beer-Hofmann}\pend{}\pstart{}im k. k. Landw. Inf.-Regmt\pend{}\pstart{}\textsc{\textcolor{pink}{Caslau Nr 12}{}\ledrightnote{\textcolor{pink}{Caslau}}.}\pend{}{\bigskip}\pstart{}{\pb}Lieber Richard\pend\pstart
           wann ko{\geminationm}en Sie? Werden Sie mich noch hier antreffen?
                    Ich verreiſe wahrscheinlich am 2. Juli.\pend
           \pstart
           {\pb}\textcolor{blue}{\textsc{Hugo}}{}\ledrightnote{\textcolor{blue}{Hugo von Hofmannsthal}}{ }ſoll heute in \textcolor{pink}{Wien}{}\ledrightnote{\textcolor{pink}{Wien}}{ }ſein, telephonirte mir ſein \textcolor{blue}{Vater}{}\ledrightnote{→\textcolor{blue}{Hugo August von Hofmannsthal}}; vielleicht treff ich ihn heute
                    Abend. – \textcolor{blue}{\textsc{Salten}}{}\ledrightnote{\textcolor{blue}{Felix Salten}}{ }ſeh ich ſelten, \textcolor{blue}{\textsc{Schwarzkopf}}{}\ledrightnote{\textcolor{blue}{Gustav Schwarzkopf}} faſt gar nicht. {\pb}Daſs ich ein \textcolor{green}{Stück}{}\ledrightnote{→\textcolor{green}{Freiwild. Schauspiel in 3 Akten}}{ }ſchreibe, wiſſen Sie? Vielleicht beend’ ich
                    den 1. Akt noch in \textcolor{pink}{Wien}{}\ledrightnote{\textcolor{pink}{Wien}}. – \textcolor{blue}{Burckhard}{}\ledrightnote{\textcolor{blue}{Max Eugen Burckhard}}{ }ſprach ich neulich; Nachts – im Dunkel unsrer
                        {\pb}gemeinſchaftlichen Stiege. Er iſt ein
                    Wurſtl. – Ich war bei \textcolor{blue}{\textsc{Sonnenthal}}{}\ledrightnote{\textcolor{blue}{Adolf von Sonnenthal}} – der wird nemlich den \textcolor{green}{Vater}{}\ledrightnote{→\textcolor{green}{Liebelei. Schauspiel in drei Akten}} geben. Und, wie \textcolor{blue}{B.}{}\ledrightnote{\textcolor{blue}{Max Eugen Burckhard}}
                    verſichert, \textcolor{blue}{Mitter{\pb}wurzer}{}\ledrightnote{\textcolor{blue}{Friedrich Mitterwurzer}} den »\textcolor{green}{Herrn}{}\ledrightnote{→\textcolor{green}{Liebelei. Schauspiel in drei Akten}}«. –\pend
           \pstart
           Ich habe geradezu eine Sehnſucht, wieder mit Ihnen zu plaudern. »Geradezu« – das
                    ſoll der Sentimentalität den Kragen umdrehen.\pend
           \pstart
           {\pb}Wie geht’s Ihnen? Schreiben Sie bitte. –\pend
           \pstart
           Den »\textcolor{green}{alten Dichter}{}\ledrightnote{\textcolor{green}{Später Ruhm}}« werd ich dem \textcolor{blue}{\textsc{Bahr}}{}\ledrightnote{\textcolor{blue}{Hermann Bahr}} für die \textcolor{brown}{Zeit}{}\ledrightnote{\textcolor{brown}{Die Zeit. Wiener Wochenschrift}} geben, we{\geminationn} er ihn bringen will. Im Prinzip iſt er ein{\pb}verſtanden.\pend
           \pstart
           Seien Sie herzlich gegrüßt.{\\[\baselineskip]}Ihr \spacefill\mbox{Arthur}\pend
           \leftskip=0em{}\endnumbering\briefempfaengerindex{Beer-Hofmann, Richard@\textsc{Beer-Hofmann, Richard}!zzzSchnitzler, Arthur@\emph{von Arthur Schnitzler}!1895-06-221@{22. 6. 1895}|)be}\mylabel{h}  \normalsize

\doendnotes{C}
\bigskip
\vfill

\clearpage

\footnotesize

\lohead{\textsc{register}}

% Definiere theindex-Environment komplett neu ohne reledmac
\makeatletter
\renewenvironment{theindex}{%
  \section*{\indexname}%
  \setlength{\parindent}{0pt}%
  \setlength{\parskip}{0pt plus 0.3pt}%
  \let\item\@idxitem
}{%
  \clearpage
}
\makeatother

\IfFileExists{\jobname-pw.ind}{\input{\jobname-pw.ind}}{}

\end{document}

      