%% latex-korrekturansicht-vorspann.tex
%% Vorspann für die Korrekturansicht.
%% Lädt die gemeinsame Datei latex-vorspann.tex mit gesetztem Schalter.

\newif\ifkorrekturansicht
\korrekturansichttrue

\input{../tex-inputs/latex-vorspann}


               \section[Arthur Schnitzler an Georg Brandes, 29. 4. 1911]{ Arthur Schnitzler an Georg Brandes, 29. 4. 1911}\nopagebreak\mylabel{v}\rehead{ }\normalsize\beginnumbering\briefempfaengerindex{Brandes, Georg@\textsc{Brandes, Georg}!zzzSchnitzler, Arthur@\emph{von Arthur Schnitzler}!1911-04-291@{29. 4. 1911}|(be} \toendnotes[C]{\smallbreak\pagebreak[2]} \Standort{Kopenhagen, Det Kongelige Bibliotek, Georg Brandes Arkiv, box 125.}
\physDesc{Bildpostkarte
\newline{}Handschrift: Bleistift, deutsche Kurrent\newline{}Versand: 1) Stempel: »\nobreak{}\oindex{Garmisch-Partenkirchen@\textbf{Garmisch-Partenkirchen}, \emph{Besiedelter Ort (A.BSO)}|pwk}P{[}arten{]}kirchen, {[}29{]}. Apr. {[}11{]}, 6–7Nm\nobreak{}«.  2) mit blauem Buntstift
                                    »138/31« über dem Adressfeld notiert\newline{}Ordnung: 1) die linke Ecke abgerissen, eine
                                    Briefmarke entfernt 2) mit Bleistift von unbekannter Hand nummeriert:
                                    »31«}\buchAbdrucke{\weitereDrucke{Georg Brandes, Arthur Schnitzler: \emph{Ein Briefwechsel}. Hg. Kurt Bergel. Bern: \emph{Francke} 1956, S. 101.} }\pstart{}{\pb}Herrn Prof. \textsc{Georg
                            Brandes}\pend{}\pstart{}\textcolor{pink}{\textsc{Paris}}{}\ledrightnote{\textcolor{pink}{Paris}}\pend{}\pstart{}\textcolor{pink}{\textsc{Hotel Lutetia}}{}\ledrightnote{\textcolor{pink}{Hôtel Lutetia}}\pend{}\pstart{}\textcolor{pink}{\textsc{Boulevrd Pasqual}}{}\ledrightnote{\textcolor{pink}{Boulevard Raspail}}\pend{}{\bigskip}\pstart
           \noindent{}\centering{}\textcolor{gray}{\textbf{{\pb}\textcolor{pink}{Garmisch}{}\ledrightnote{\textcolor{pink}{Garmisch-Partenkirchen}} geg. Wetterstein}}\pend
           \pstart
           {\pb}\textcolor{pink}{\textsc{Partenkirchen}}{}\ledrightnote{\textcolor{pink}{Garmisch-Partenkirchen}}, 29. 4. 11\pend
           \pstart
           Ihre Karte, verehrter Herr Br\damage{andes,} und die Druckſchriften ſind uns nach \textcolor{pink}{Mentone}{}\ledrightnote{\textcolor{pink}{Menton}} nachgewandert, u. am Ende meiner Reiſe, dank ich und grüß ich
                    herzlichſt in alter Treue.\pend
           \pstart
           Ihr{\\[\baselineskip]}\spacefill\mbox{Arth Schni}\pend
           \leftskip=0em{}\endnumbering\briefempfaengerindex{Brandes, Georg@\textsc{Brandes, Georg}!zzzSchnitzler, Arthur@\emph{von Arthur Schnitzler}!1911-04-291@{29. 4. 1911}|)be}\mylabel{h}  \normalsize

\doendnotes{C}
\bigskip
\vfill

\clearpage

\footnotesize

\lohead{\textsc{register}}

% Definiere theindex-Environment komplett neu ohne reledmac
\makeatletter
\renewenvironment{theindex}{%
  \section*{\indexname}%
  \setlength{\parindent}{0pt}%
  \setlength{\parskip}{0pt plus 0.3pt}%
  \let\item\@idxitem
}{%
  \clearpage
}
\makeatother

\IfFileExists{\jobname-pw.ind}{\input{\jobname-pw.ind}}{}

\end{document}

      