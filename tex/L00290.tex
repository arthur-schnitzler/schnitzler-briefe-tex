%% latex-korrekturansicht-vorspann.tex
%% Vorspann für die Korrekturansicht.
%% Lädt die gemeinsame Datei latex-vorspann.tex mit gesetztem Schalter.

\newif\ifkorrekturansicht
\korrekturansichttrue

\input{../tex-inputs/latex-vorspann}


               \section[Arthur Schnitzler an Max Burckhard, 14. 1. 1894]{ Arthur Schnitzler an Max Burckhard, 14. 1. 1894}\nopagebreak\mylabel{v}\rehead{ }\normalsize\beginnumbering\briefempfaengerindex{Burckhard, Max Eugen@\textsc{Burckhard, Max Eugen}!zzzSchnitzler, Arthur@\emph{von Arthur Schnitzler}!1894-01-141@{14. 1. 1894}|(be} \toendnotes[C]{\smallbreak\pagebreak[2]} \buchAlsQuelle{\pwindex{Schnitzlers Einzug ins Burgtheater@\emph{Schnitzlers Einzug ins Burgtheater}|pwk}\pwindex{Neue Freie Presse@\emph{Neue Freie Presse}|pwk}Karl Glossy: \emph{Schnitzlers Einzug ins Burgtheater. Unbekannte Briefe des Dichters.} In: \emph{Neue Freie Presse}, Nr. 24162, 19. 12. 1931, S. 14.}\buchAbdrucke{\weitereDrucke{1) \pwindex{Schnitzlers Einzug ins Burgtheater@\emph{Schnitzlers Einzug ins Burgtheater}|pwk}Karl Glossy: \emph{Schnitzlers Einzug ins Burgtheater. Unbekannte Briefe des Dichters.} In: \emph{Wiener Studien und Dokumente}. Zum 85. Geburtstag des Verfassers hg. von seinen
                                Freunden. Wien: \emph{Steyrermühl} 1933, S. 166–168.} \weitereDrucke{2) Hans-Ulrich Lindken: \emph{Arthur Schnitzler. Aspekte und Akzente. Materialien zu
                                Leben und Werk}. Frankfurt am Main, Bern, Göttingen: \emph{Peter Lang} 1984, S. 243–246 (Europäische Hochschulschriften, Reihe 1, Deutsche
                                Sprache und Literatur, 754).} }\toendnotes[C]{\smallbreak}\pstart
           \noindent{}{\pb}\so{Schnitzler an Burckhard}, 14. Januar
                    1894: »Sehr verehrter Herr Direktor! Vor etwa drei Vierteljahren habe ich
                    Ihnen durch den Verlag \label{K_L00290_1v}\edtext{\textcolor{brown}{Entſch}{}\ledrightnote{\textcolor{brown}{A. Entsch}}}{\lemma{\textnormal{\emph{Entſch}}}\Cendnote{\textnormal{Der Verlag \emph{\textcolor{brown}{A. Entsch}} dürfte den Bühnenvertrieb von \emph{\textcolor{green}{Anatol}} verwaltet haben. Dieser erschien bereits Ende
                            1892, vordatiert auf 1893, im \emph{\textcolor{brown}{Bibliographischen Bureau}}.}}}\label{K_L00290_1h} in \textcolor{pink}{Berlin}{}\ledrightnote{\textcolor{pink}{Berlin}} ein \textcolor{green}{Buch}{}\ledrightnote{→\textcolor{green}{Anatol}} einſenden laſſen, welches unter anderm drei Luſtspiele enthält,
                    die ſich vielleicht zur Aufführung eignen. Erlauben Sie mir, ſehr geehrter Herr
                    Direktor, Sie jetzt auf dieſelben aufmerkſam zu machen, zu einer Zeit, wo ſowohl
                    die Stimmung des Publikums als auch die Geſtaltung des Repertoires Einaktern
                    günſtiger geworden ſcheint. Die drei ſehr kurzen Stücke ſind: ›\textcolor{green}{Frage an das Schickſal}{}\ledrightnote{\textcolor{green}{Die Frage an das Schicksal}}‹, ›\textcolor{green}{Epiſode}{}\ledrightnote{\textcolor{green}{Episode}}‹ und ›\textcolor{green}{Abſchiedsſouper}{}\ledrightnote{\textcolor{green}{Abschiedssouper}}‹, von
                    welchen vielleicht das dritte in Anbetracht des etwas frivolen Tones auf der \textcolor{pink}{Hofbühne}{}\ledrightnote{→\textcolor{pink}{Burgtheater}} nicht möglich
                    erſcheinen ſollte, ſo dürften ſich die zwei erſten um ſo eher für eine ſolche
                    eignen. Ich will über die kleinen Stückchen weiter nichts ſagen, möchte Sie,
                    verehrter Herr Direktor, nur bitten, ſie gütigſt einmal Ihrer Aufmerkſamkeit zu
                    würdigen. Ich bin mit vorzüglicher Hochachtung Ihr ſehr ergebener Dr. Arthur
                    Schnitzler.«\pend
           \endnumbering\briefempfaengerindex{Burckhard, Max Eugen@\textsc{Burckhard, Max Eugen}!zzzSchnitzler, Arthur@\emph{von Arthur Schnitzler}!1894-01-141@{14. 1. 1894}|)be}\mylabel{h}  \normalsize

\doendnotes{C}
\bigskip
\vfill

\clearpage

\footnotesize

\lohead{\textsc{register}}

% Definiere theindex-Environment komplett neu ohne reledmac
\makeatletter
\renewenvironment{theindex}{%
  \section*{\indexname}%
  \setlength{\parindent}{0pt}%
  \setlength{\parskip}{0pt plus 0.3pt}%
  \let\item\@idxitem
}{%
  \clearpage
}
\makeatother

\IfFileExists{\jobname-pw.ind}{\input{\jobname-pw.ind}}{}

\end{document}

      