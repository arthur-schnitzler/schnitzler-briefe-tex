%% latex-korrekturansicht-vorspann.tex
%% Vorspann für die Korrekturansicht.
%% Lädt die gemeinsame Datei latex-vorspann.tex mit gesetztem Schalter.

\newif\ifkorrekturansicht
\korrekturansichttrue

\input{../tex-inputs/latex-vorspann}


               \section[Arthur Schnitzler an Hugo von Hofmannsthal, 29. 6. 1896]{ Arthur Schnitzler an Hugo von Hofmannsthal, 29. 6. 1896}\nopagebreak\mylabel{v}\rehead{ }\normalsize\beginnumbering\briefempfaengerindex{Hofmannsthal, Hugo von@\textsc{Hofmannsthal, Hugo von}!zzzSchnitzler, Arthur@\emph{von Arthur Schnitzler}!1896-06-291@{29. 6. 1896}|(be} \toendnotes[C]{\smallbreak\pagebreak[2]} \Standort{FDH, Hs-30885,50.}
\physDesc{Brief, 1 Blatt, 4 Seiten
\newline{}Handschrift: schwarze Tinte, deutsche Kurrent}\buchAbdrucke{\weitereDrucke{Hugo von Hofmannsthal, Arthur Schnitzler: \emph{Briefwechsel}. Hg. Therese Nickl und Heinrich Schnitzler. Frankfurt am Main: \emph{S. Fischer} 1964, S. 68–69.} }\toendnotes[C]{\smallbreak}\pstart
           \raggedleft{}{\pb}\textcolor{pink}{Wien}{}\ledrightnote{\textcolor{pink}{Wien}}{ }29. Juni 96\pend
           \pstart
           Mein lieber Hugo, ich lege Ihnen einen Zettel bei, da ſteht
                    drauf, wo ich für Briefe zu erreichen bin, u. bis wann. In \textcolor{pink}{Wien}{}\ledrightnote{\textcolor{pink}{Wien}} bin ich noch bis zum Freitag
                    (ſpäteſtens) (3. Juli). –\pend
           \pstart
           Ich wollte eben niederſchreiben, daſs ich mich »freue« u. habe gezögert, weil die
                    Freude nicht ganz rein iſt. Es iſt, durch heftigeres Erklin{\pb}gen \label{K_L00557_1v}\edtext{früherer Lebensbeziehungen}{\lemma{\textnormal{\emph{früherer Lebensbeziehungen}}}\Cendnote{\textnormal{In den
                        vorangehenden Tagen stand er in Kontakt mit \textcolor{blue}{Olga Waissnix} und \textcolor{blue}{Marie
                        Glümer}.}}}\label{K_L00557_1h}, in der letzten Zeit wieder manche Unruhe in mich
                    gekommen, die in manchen Stunden, beſonders Abendſtunden allein auf dem Land,
                    ſchmerzlich bewegt. Nun weiſs ich nicht, ob ſich das da oben gänzlich beruhigen
                    wird oder ob nicht vielleicht noch dunklere Traurigkeit ko{\geminationm}en mag. Ich leide gewiſs an {\pb}einer gewiſſen \introOben{}(\introOben{}ſentimentalen\introOben{}!)\introOben{} Ueberempfindlichkeit für
                    gewiſſe Begriffe, wie Ferne, Einſamkeit, und Vergangen. Das hängt wohl mit
                        meine\textcolor{gray}{n} mangelnden Fähigkeit\textcolor{gray}{en}{ }\introOben{}abzuſchließen\introOben{} zusa{\geminationm}en. Abzuſchließen, in jedem Sinn. Fehler meines
                    Lebens und meiner Kunſt ſind daraus zu erklären.\pend
           \pstart
           – Das \textcolor{green}{Stück}{}\ledrightnote{→\textcolor{green}{Freiwild. Schauspiel in 3 Akten}} reiſt natürlich
                    mit; {\pb}iſt Ihnen noch was dazu eingefallen?\pend
           \pstart
           – Iſt das \textcolor{green}{eine}{}\ledrightnote{→\textcolor{green}{Geschichte der beiden Liebespaare}} Ihrer \label{K_L00557_2v}\edtext{Soldatengeſchichten}{\lemma{\textnormal{\emph{Soldatengeſchichten}}}\Cendnote{\textnormal{Mehrere Texte aus der Zeit
                        spielen im Milieu des Militärs.}}}\label{K_L00557_2h}, die Sie ſchreiben? –\pend
           \pstart
           Sie hören ſehr bald von mir u. laſſen mich wohl auch nicht lang ohne Nachricht.
                    Empfehlen Sie mich Ihren \textcolor{blue}{Eltern}{}\ledrightnote{→\textcolor{blue}{Hugo August von Hofmannsthal}{\newline}→\textcolor{blue}{Anna von Hofmannsthal}}. Seien Sie herzlich gegrüßt.\pend
           \pstart Ihr \spacefill\mbox{Arthur}\pend{}\endnumbering\briefempfaengerindex{Hofmannsthal, Hugo von@\textsc{Hofmannsthal, Hugo von}!zzzSchnitzler, Arthur@\emph{von Arthur Schnitzler}!1896-06-291@{29. 6. 1896}|)be}\mylabel{h}  \normalsize

\doendnotes{C}
\bigskip
\vfill

\clearpage

\footnotesize

\lohead{\textsc{register}}

% Definiere theindex-Environment komplett neu ohne reledmac
\makeatletter
\renewenvironment{theindex}{%
  \section*{\indexname}%
  \setlength{\parindent}{0pt}%
  \setlength{\parskip}{0pt plus 0.3pt}%
  \let\item\@idxitem
}{%
  \clearpage
}
\makeatother

\IfFileExists{\jobname-pw.ind}{\input{\jobname-pw.ind}}{}

\end{document}

      