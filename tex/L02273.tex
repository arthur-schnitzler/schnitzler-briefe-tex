%% latex-korrekturansicht-vorspann.tex
%% Vorspann für die Korrekturansicht.
%% Lädt die gemeinsame Datei latex-vorspann.tex mit gesetztem Schalter.

\newif\ifkorrekturansicht
\korrekturansichttrue

\input{../tex-inputs/latex-vorspann}


               \section[Richard Beer-Hofmann an Arthur Schnitzler, 5. 9. 1917]{ Richard Beer-Hofmann an Arthur Schnitzler, 5. 9. 1917}\nopagebreak\mylabel{v}\rehead{ }\normalsize\beginnumbering\briefempfaengerindex{Schnitzler, Arthur@\textsc{Schnitzler, Arthur}!zzzBeer-Hofmann, Richard@\emph{von Richard Beer-Hofmann}!1917-09-051@{5. 9. 1917}|(be} \toendnotes[C]{\smallbreak\pagebreak[2]} \Standort{CUL, Schnitzler, B 8.}
\physDesc{Postkarte
\newline{}Handschrift: Bleistift, lateinische Kurrent\newline{}Versand: Stempel: »\nobreak{}\oindex{Bad Ischl@\textbf{Bad Ischl}, \emph{Besiedelter Ort (A.BSO)}|pwk}Bad Is{[}chl{]}, 5. IX. 17, VII\nobreak{}«.  \newline{}Ordnung: mit Bleistift von unbekannter Hand nummeriert:
                                    »265« }\toendnotes[C]{\smallbreak}\pstart{}{\pb}Beer-Hofmann, \textcolor{pink}{Bad Ischl}{}\ledrightnote{\textcolor{pink}{Bad Ischl}}\pend{}{\bigskip}\pstart{}Herrn\pend{}\pstart{}D\textsuperscript{r} Arthur Schnitzler\pend{}\pstart{}\textcolor{pink}{Partenkirchen}{}\ledrightnote{\textcolor{pink}{Partenkirchen}}\pend{}\pstart{}\textcolor{pink}{Haus Tannenberg}{}\ledrightnote{\textcolor{pink}{Haus Tannenberg}}\pend{}{\bigskip}\pstart
           \noindent{}{\pb}Lieber Artur! Von \textcolor{pink}{Salzburg}{}\ledrightnote{\textcolor{pink}{Salzburg}} zurück,
               finde ich Ihre Karte vor. Ich glaube dass ich kaum vor 28–30ten
                  Sept in \textcolor{pink}{Berlin}{}\ledrightnote{\textcolor{pink}{Berlin}} sein dürfte. Jedenfalls
               verständig ich Sie (– falls Sie nicht in \textcolor{pink}{Wien}{}\ledrightnote{\textcolor{pink}{Wien}} sein
               sollten Ihre jeweilige Adresse!) wann ich reise.\pend
           \pstart
           Der \label{K_L02273-1v}\edtext{zweite Brief}{\lemma{\textnormal{\emph{zweite Brief}}}\Cendnote{\textnormal{siehe Arthur Schnitzler an Richard Beer-Hofmann, 23. 7. 1917}}}\label{K_L02273-1h} ist angelangt, – ich habe ein böses Gewissen so säumig
                  {\pb}gewesen zu sein – aber Versti{\geminationm}tsein behält man für sich – oder nicht »man« – aber
               ich.\pend
           \pstart
           Herzliche Grüsse Ihnen, Ihrer \textcolor{blue}{Frau}{}\ledrightnote{→\textcolor{blue}{Olga Schnitzler}} und Ihrer \textcolor{blue}{Schwägerin}{}\ledrightnote{→\textcolor{blue}{Elisabeth Steinrück}}! Ihr{\\[\baselineskip]}\spacefill\mbox{Richard}\pend
           \leftskip=0em{}\endnumbering\briefempfaengerindex{Schnitzler, Arthur@\textsc{Schnitzler, Arthur}!zzzBeer-Hofmann, Richard@\emph{von Richard Beer-Hofmann}!1917-09-051@{5. 9. 1917}|)be}\mylabel{h}  \normalsize

\doendnotes{C}
\bigskip
\vfill

\clearpage

\footnotesize

\lohead{\textsc{register}}

% Definiere theindex-Environment komplett neu ohne reledmac
\makeatletter
\renewenvironment{theindex}{%
  \section*{\indexname}%
  \setlength{\parindent}{0pt}%
  \setlength{\parskip}{0pt plus 0.3pt}%
  \let\item\@idxitem
}{%
  \clearpage
}
\makeatother

\IfFileExists{\jobname-pw.ind}{\input{\jobname-pw.ind}}{}

\end{document}

      