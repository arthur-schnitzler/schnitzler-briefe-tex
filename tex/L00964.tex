%% latex-korrekturansicht-vorspann.tex
%% Vorspann für die Korrekturansicht.
%% Lädt die gemeinsame Datei latex-vorspann.tex mit gesetztem Schalter.

\newif\ifkorrekturansicht
\korrekturansichttrue

\input{../tex-inputs/latex-vorspann}


               \section[Arthur Schnitzler an Gerhart Hauptmann, 25. 8. 1899]{ Arthur Schnitzler an Gerhart Hauptmann, 25. 8. 1899}\nopagebreak\mylabel{v}\rehead{ }\normalsize\beginnumbering\briefempfaengerindex{Hauptmann, Gerhart@\textsc{Hauptmann, Gerhart}!zzzSchnitzler, Arthur@\emph{von Arthur Schnitzler}!1899-08-251@{25. 8. 1899}|(be} \toendnotes[C]{\smallbreak\pagebreak[2]} \Standort{Staatsbibliothek Berlin – Preußischer Kulturbesitz, GHBrBl A:Schnitzler (2,3).}
\physDesc{Brief, 1 Blatt, 4 Seiten
\newline{}Handschrift: schwarze Tinte, deutsche Kurrent\newline{}Ordnung: mit Bleistift von unbekannter Hand nummeriert: »2« }\buchAbdrucke{\weitereDrucke{Arthur Schnitzler: \emph{Briefe 1875–1912}. Hg. Therese Nickl und Heinrich Schnitzler. Frankfurt am Main: \emph{S. Fischer} 1981, S. 373.} }\toendnotes[C]{\smallbreak}\pstart
           \raggedleft{}{\pb}\textcolor{pink}{Ischl, Rudolfshöhe}{}\ledrightnote{\textcolor{pink}{Hotel und Pension Rudolfshöhe (Leopold Petter)}}{\\}25. 8. 9\textcolor{gray}{9}.
                    \pend
           \pstart{}Lieber Herr Hauptmann,\pend\pstart
           etwas verſpätet danke ich Ihnen für Ihre freundliche Antwort. Ich darf Ihnen wohl
                    ſagen, dſs ich ſie ungefähr ſo erwartet und an Ihrer Stelle dieſelbe gegeben
                    hätte. Nun iſt der \textcolor{blue}{Heraus{\pb}geber}{}\ledrightnote{→\textcolor{blue}{Isidor Singer}} von der ganzen Idee mit den vielen
                    Namen und den großen Namen abgeko{\geminationm}en, was ich ſehr
                    vernünftig finde.\pend
           \pstart
           Ich bin jetzt in \textcolor{pink}{Iſchl}{}\ledrightnote{\textcolor{pink}{Bad Ischl}}, \textcolor{blue}{Hofmannsthal}{}\ledrightnote{\textcolor{blue}{Hugo von Hofmannsthal}} desgleichen, in derſelben \textcolor{pink}{Pension}{}\ledrightnote{→\textcolor{pink}{Hotel und Pension Rudolfshöhe (Leopold Petter)}}, und jeder von uns hat einen
                    eigenen {\pb}Balkon zum Dichten.\pend
           \pstart
           Es freut mich dſs Sie ſich ſo freundlich meiner erinnern und mich bald einmal
                    wieder zu sehen wünschen – aber ob \uline{inner}halb
                    oder \uline{außer}halb der Stadtmauern kann ich Ihrem
                    Brief nicht entnehmen: in Ihrer Schrift ſieht {\pb}»innen«
                    genau ſo aus wie »außen« – ſo arg iſts bei mir hoffentlich nicht.\pend
           \pstart
           Wie immer und wo i{\geminationm}er; Sie können mir glauben daſs
                    es wenige Menſchen gibt, die ich ſo gerne bald wiederſehen möchte als Sie.\pend
           \pstart
           Ganz der Ihre{\\[\baselineskip]}Arthur Schnitzler\pend
           \leftskip=0em{}\endnumbering\briefempfaengerindex{Hauptmann, Gerhart@\textsc{Hauptmann, Gerhart}!zzzSchnitzler, Arthur@\emph{von Arthur Schnitzler}!1899-08-251@{25. 8. 1899}|)be}\mylabel{h}  \normalsize

\doendnotes{C}
\bigskip
\vfill

\clearpage

\footnotesize

\lohead{\textsc{register}}

% Definiere theindex-Environment komplett neu ohne reledmac
\makeatletter
\renewenvironment{theindex}{%
  \section*{\indexname}%
  \setlength{\parindent}{0pt}%
  \setlength{\parskip}{0pt plus 0.3pt}%
  \let\item\@idxitem
}{%
  \clearpage
}
\makeatother

\IfFileExists{\jobname-pw.ind}{\input{\jobname-pw.ind}}{}

\end{document}

      