%% latex-korrekturansicht-vorspann.tex
%% Vorspann für die Korrekturansicht.
%% Lädt die gemeinsame Datei latex-vorspann.tex mit gesetztem Schalter.

\newif\ifkorrekturansicht
\korrekturansichttrue

\input{../tex-inputs/latex-vorspann}


               \section[Arthur Schnitzler an Richard Beer-Hofmann, 16. 11. 1902]{ Arthur Schnitzler an Richard Beer-Hofmann, 16. 11. 1902}\nopagebreak\mylabel{v}\rehead{ }\normalsize\beginnumbering\briefempfaengerindex{Beer-Hofmann, Richard@\textsc{Beer-Hofmann, Richard}!zzzSchnitzler, Arthur@\emph{von Arthur Schnitzler}!1902-11-161@{16. 11. 1902}|(be} \toendnotes[C]{\smallbreak\pagebreak[2]} \Standort{YCGL, MSS 31.}
\physDesc{Brief, 1 Blatt, 1 Seite, Umschlag
\newline{}Handschrift: Bleistift, deutsche Kurrent\newline{}Versand: 1) Stempel: »\nobreak{}\oindex{IX., Alsergrund@\textbf{IX., Alsergrund}, \emph{Bezirk (A.BZK)}|pwk}Wien 9/1, 17. 11. 02, 11–12V\nobreak{}«.  2) Stempel: »\nobreak{}\oindex{Rodaun@\textbf{Rodaun}, \emph{Teil eines besiedelten Ortes (A.BSOX)}|pwk}{\pb}Rodaun, 17. 11. 02, 2–4N\nobreak{}«. \newline{}Ordnung: mit Bleistift von unbekannter Hand datiert: »16. 11.« }\buchAbdrucke{\weitereDrucke{Arthur Schnitzler, Richard Beer-Hofmann: \emph{Briefwechsel 1891–1931}. Hg. Konstanze Fliedl. Wien, Zürich: \emph{Europaverlag} 1992, S. 159.} }\pstart{}{\pb}\textsc{Herrn Dr Rich
                     Beer-Hofmann}\pend{}\pstart{}\textcolor{pink}{\textsc{Rodaun}}{}\ledrightnote{\textcolor{pink}{Rodaun}}\pend{}\pstart{}\textsc{\textcolor{pink}{Liesinger Straße 2}{}\ledrightnote{\textcolor{pink}{Liesingerstraße}}}\pend{}{\bigskip}\pstart
           \raggedleft{}{\pb}16. 11. 902\pend
           \pstart
           lieber Richard, die nächſte Zeit ko{\geminationm}
               ich kaum nach \textcolor{pink}{Rodaun}{}\ledrightnote{\textcolor{pink}{Rodaun}}; die Vormittage ſind zu kurz,
               Nachmittg arbeite ich. Könnte man ſich de{\geminationn} nicht in \textcolor{pink}{Wien}{}\ledrightnote{\textcolor{pink}{Wien}} ſehn? Sie ko{\geminationm}en ja ſo oft herein. Das wär
               doch fürs erſte viel einfacher. Herzlichſt\pend
           \pstart Ihr \spacefill\mbox{A.}\pend{}\endnumbering\briefempfaengerindex{Beer-Hofmann, Richard@\textsc{Beer-Hofmann, Richard}!zzzSchnitzler, Arthur@\emph{von Arthur Schnitzler}!1902-11-161@{16. 11. 1902}|)be}\mylabel{h}  \normalsize

\doendnotes{C}
\bigskip
\vfill

\clearpage

\footnotesize

\lohead{\textsc{register}}

% Definiere theindex-Environment komplett neu ohne reledmac
\makeatletter
\renewenvironment{theindex}{%
  \section*{\indexname}%
  \setlength{\parindent}{0pt}%
  \setlength{\parskip}{0pt plus 0.3pt}%
  \let\item\@idxitem
}{%
  \clearpage
}
\makeatother

\IfFileExists{\jobname-pw.ind}{\input{\jobname-pw.ind}}{}

\end{document}

      