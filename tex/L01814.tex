%% latex-korrekturansicht-vorspann.tex
%% Vorspann für die Korrekturansicht.
%% Lädt die gemeinsame Datei latex-vorspann.tex mit gesetztem Schalter.

\newif\ifkorrekturansicht
\korrekturansichttrue

\input{../tex-inputs/latex-vorspann}


               \section[Arthur Schnitzler an Richard Beer-Hofmann, 2{[}8{]}. 11. 1908]{ Arthur Schnitzler an Richard Beer-Hofmann, 2{[}8{]}. 11. 1908}\nopagebreak\mylabel{v}\rehead{ }\normalsize\beginnumbering\briefempfaengerindex{Beer-Hofmann, Richard@\textsc{Beer-Hofmann, Richard}!zzzSchnitzler, Arthur@\emph{von Arthur Schnitzler}!1908-11-284@{2{[}8{]}. 11. 1908}|(be} \toendnotes[C]{\smallbreak\pagebreak[2]} \Standort{YCGL, MSS 31.}
\physDesc{Brief, 1 Blatt, 3 Seiten, Umschlag
\newline{}Handschrift: Bleistift, deutsche Kurrent}\buchAbdrucke{\weitereDrucke{Arthur Schnitzler, Richard Beer-Hofmann: \emph{Briefwechsel 1891–1931}. Hg. Konstanze Fliedl. Wien, Zürich: \emph{Europaverlag} 1992, S. 192.} }\toendnotes[C]{\smallbreak}\pstart{}{\pb}\textcolor{gray}{\textbf{Dr. Arthur Schnitzler}}\pend{}\pstart{}\textcolor{gray}{\textbf{\textcolor{pink}{Wien XVIII. Spoettelgasse 7}{}\ledrightnote{\textcolor{pink}{Edmund-Weiß-Gasse}}.}}\pend{}{\bigskip}\pstart{}{\pb}\textsc{Dr. Richard Beer-Hofma{\geminationn}}\pend{}\pstart{}\textcolor{pink}{Wien}{}\ledrightnote{\textcolor{pink}{XVIII., Währing}}. \pend{}{\bigskip}\pstart
           \noindent{}{\pb}II.\pend
           \pstart
           \textcolor{gray}{\textbf{Dr. Arthur Schnitzler}}\hfill \label{K_L01814_1v}\edtext{29. 11.}{\lemma{\textnormal{\emph{29. 11.}}}\Cendnote{\textnormal{Bei der Datierung ist \textcolor{blue}{Schnitzler} ein Fehler
                           unterlaufen.}}}\label{K_L01814_1h}\pend
           \pstart
           \textcolor{gray}{\textbf{\textcolor{pink}{Wien XVIII. Spoettelgasse 7}{}\ledrightnote{\textcolor{pink}{Edmund-Weiß-Gasse}}.}}\pend
           \pstart
           Eben ſchrieb ich Ihnen \label{K_L01814_2v}\edtext{den beiliegd Brief}{\lemma{\textnormal{\emph{den beiliegd Brief}}}\Cendnote{\textnormal{Es dürfte sich um
                  den zweiten Brief vom [28. 11. 1908?] handeln. Da der Briefumschlag ohne
                  Briefmarke geblieben ist, dürfte er in den anderen eingelegt gewesen sein.}}}\label{K_L01814_2h}.
               Bleibt alſo nichts andres übrig als den morgigen Abend abzuwarten.\pend
           \pstart
           {\pb}Falls \textcolor{blue}{\textsc{Kerr}}{}\ledrightnote{\textcolor{blue}{Alfred Kerr}} bei Ihnen ſchriftlich anfrägt, ſo ſchlagen Sie vielleicht auch für morgen Abend
                  \textcolor{pink}{\textsc{Meissl}}{}\ledrightnote{\textcolor{pink}{Meissl & Schadn}} vor. Den ganzen Tag über hab ich \label{K_L01814_3v}\edtext{morgen »geſchäftliche«
                  Beſprechungen}{\lemma{\textnormal{\emph{morgen … Beſprechungen}}}\Cendnote{\textnormal{Das erlaubt die sichere Datierung dieses Korrespondenzstücks. Vgl. A. S.: \emph{Tagebuch}, 29. 11. 1908}}}\label{K_L01814_3h}{ }{\pb}(\textsc{\textcolor{blue}{Dohnanyi}{}\ledrightnote{\textcolor{blue}{Ernst von Dohnányi}}, \textcolor{blue}{Straus}{}\ledrightnote{\textcolor{blue}{Oscar Straus}}, \textcolor{blue}{Herzmansky}{}\ledrightnote{\textcolor{blue}{Bernhard Herzmansky}}}.)\pend
           \pstart
           Ihr{\\[\baselineskip]}\spacefill\mbox{A.}\pend
           \leftskip=0em{}\endnumbering\briefempfaengerindex{Beer-Hofmann, Richard@\textsc{Beer-Hofmann, Richard}!zzzSchnitzler, Arthur@\emph{von Arthur Schnitzler}!1908-11-284@{2{[}8{]}. 11. 1908}|)be}\mylabel{h}  \normalsize

\doendnotes{C}
\bigskip
\vfill

\clearpage

\footnotesize

\lohead{\textsc{register}}

% Definiere theindex-Environment komplett neu ohne reledmac
\makeatletter
\renewenvironment{theindex}{%
  \section*{\indexname}%
  \setlength{\parindent}{0pt}%
  \setlength{\parskip}{0pt plus 0.3pt}%
  \let\item\@idxitem
}{%
  \clearpage
}
\makeatother

\IfFileExists{\jobname-pw.ind}{\input{\jobname-pw.ind}}{}

\end{document}

      