%% latex-korrekturansicht-vorspann.tex
%% Vorspann für die Korrekturansicht.
%% Lädt die gemeinsame Datei latex-vorspann.tex mit gesetztem Schalter.

\newif\ifkorrekturansicht
\korrekturansichttrue

\input{../tex-inputs/latex-vorspann}


               \section[Paul Goldmann an Arthur Schnitzler, 29. 11. 1891]{ Paul Goldmann an Arthur Schnitzler, 29. 11. 1891}\nopagebreak\mylabel{v}\rehead{ }\normalsize\beginnumbering\briefempfaengerindex{Schnitzler, Arthur@\textsc{Schnitzler, Arthur}!zzzGoldmann, Paul@\emph{von Paul Goldmann}!1891-11-291@{29. 11. 1891}|(be} \toendnotes[C]{\smallbreak\pagebreak[2]} \Standort{DLA, A:Schnitzler, HS.NZ85.1.3162.}
\physDesc{Postkarte
\newline{}Handschrift: 1) schwarze Tinte, deutsche Kurrent\hspace{1em}2) schwarze Tinte, lateinische Kurrent (\noindent{}Adresse)\hspace{1em}\newline{}Versand: 1) Stempel: »\nobreak{}\oindex{Amsterdam@\textbf{Amsterdam}, \emph{http://www.geonames.org/ontologyP.PPLC}|pwk}Amste\textcolor{gray}{rdam}, 30 Nov 91, 10–11V\nobreak{}«.  2) Stempel: »\nobreak{}Wien 1/1, 2/12. 91, 9½–11V., Bestellt\nobreak{}«. 
\newline{}Schnitzler: mit Bleistift das Datum »30/11 91« vermerkt }\pstart{}{\pb}\textcolor{pink}{\begin{otherlanguage}{french}Autriche\end{otherlanguage}}{}\ledrightnote{\textcolor{pink}{Österreich}}!
               \pend{}\pstart{}Herrn
               \pend{}\pstart{}Dr. Arthur Schnitzler
               \pend{}\pstart{}\textcolor{pink}{Wien}{}\ledrightnote{\textcolor{pink}{Wien}}\pend{}\pstart{}\textcolor{pink}{I. Giselastraße 11}{}\ledrightnote{\textcolor{pink}{Bösendorferstraße}}.\pend{}{\bigskip}\pstart
           \centering{}{\pb}\textcolor{pink}{Amſterdam}{}\ledrightnote{\textcolor{pink}{Amsterdam}}, 29. November\pend
           \pstart
           Mein lieber Arthur! So ein Bildernarr bin ich
               geworden, daß ich noch im Fluge zwei Tage zufammengerafft habe, um in \textsc{\textcolor{pink}{Haarlem}{}\ledrightnote{\textcolor{pink}{Haarlem}}} die \textsc{\textcolor{blue}{Frans Hals}{}\ledrightnote{\textcolor{blue}{Frans Hals}}} und in \textsc{\textcolor{pink}{Amsterdam}{}\ledrightnote{\textcolor{pink}{Amsterdam}}} die \textsc{\textcolor{blue}{Rembrandt}{}\ledrightnote{\textcolor{blue}{Rembrandt van Rijn}}} zu ſehen. Zwei herrliche Tage voll Schönheiten und Seltſamkeiten. Und daß ich
               über all’ dem Dein gedacht, ſollen Dir dieſe Zeilen ein Zeichen ſein. Schreib’ mir,
               bitte, ein Wort nach \textcolor{brown}{\textsc{\textcolor{pink}{Paris, Rue Vivienne 51}{}\ledrightnote{\textcolor{pink}{rue Vivienne}}}, »\textsc{\textcolor{brown}{\begin{otherlanguage}{french}Gazette de Francfort\end{otherlanguage}}{}\ledrightnote{\textcolor{brown}{Frankfurter Zeitung}}}}{}\ledrightnote{\textcolor{brown}{Pariser Büro der Frankfurter Zeitung}}«. Grüß’ Dich Gott! Dein \spacefill\mbox{Paul Goldmann}\pend
           \endnumbering\briefempfaengerindex{Schnitzler, Arthur@\textsc{Schnitzler, Arthur}!zzzGoldmann, Paul@\emph{von Paul Goldmann}!1891-11-291@{29. 11. 1891}|)be}\mylabel{h}  \normalsize

\doendnotes{C}
\bigskip
\vfill

\clearpage

\footnotesize

\lohead{\textsc{register}}

% Definiere theindex-Environment komplett neu ohne reledmac
\makeatletter
\renewenvironment{theindex}{%
  \section*{\indexname}%
  \setlength{\parindent}{0pt}%
  \setlength{\parskip}{0pt plus 0.3pt}%
  \let\item\@idxitem
}{%
  \clearpage
}
\makeatother

\IfFileExists{\jobname-pw.ind}{\input{\jobname-pw.ind}}{}

\end{document}

      