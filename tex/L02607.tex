%% latex-korrekturansicht-vorspann.tex
%% Vorspann für die Korrekturansicht.
%% Lädt die gemeinsame Datei latex-vorspann.tex mit gesetztem Schalter.

\newif\ifkorrekturansicht
\korrekturansichttrue

\input{../tex-inputs/latex-vorspann}


               \section[Paul Goldmann an Arthur Schnitzler, 23. 1. 1894]{ Paul Goldmann an Arthur Schnitzler, 23. 1. 1894}\nopagebreak\mylabel{v}\rehead{ }\normalsize\beginnumbering\briefempfaengerindex{Schnitzler, Arthur@\textsc{Schnitzler, Arthur}!zzzGoldmann, Paul@\emph{von Paul Goldmann}!1894-01-231@{23. 1. 1894}|(be} \toendnotes[C]{\smallbreak\pagebreak[2]} \Standort{DLA, A:Schnitzler, HS.NZ85.1.3164.}
\physDesc{Postkarte
\newline{}Handschrift Arthur Schnitzler: 1) schwarze Tinte, deutsche Kurrent\hspace{1em}2) schwarze Tinte, lateinische Kurrent (\noindent{}Adresse)\hspace{1em}\newline{}Versand: 1) Stempel: »\nobreak{}\oindex{Place de la Bourse@\textbf{Place de la Bourse}, \emph{Platz (K.PLT)}|pwk}{[}Paris{]} Pl. de la Bourse, {[}23{]} Janv. 94\nobreak{}«.  2) Stempel: »\nobreak{}\oindex{IX., Alsergrund@\textbf{IX., Alsergrund}, \emph{Bezirk (A.BZK)}|pwk}Wien 9/3 \textcolor{gray}{7}2, 25. 1. 94, 9.V, B{[}est{]}e{[}llt{]}\nobreak{}«. }\toendnotes[C]{\smallbreak}\pstart{}{\pb}\textcolor{pink}{\begin{otherlanguage}{french}Autriche\end{otherlanguage}}{}\ledrightnote{\textcolor{pink}{Österreich}} .\pend{}\pstart{}Herrn\pend{}\pstart{}Dr. Arthur Schnitzler\pend{}\pstart{}\textcolor{pink}{IX. Frankgaße 1}{}\ledrightnote{\textcolor{pink}{Frankgasse}}\pend{}\pstart{}\textcolor{pink}{Wien}{}\ledrightnote{\textcolor{pink}{Wien}}. \pend{}{\bigskip}\pstart
           \centering{}{\pb}\textsc{\textcolor{pink}{Paris}{}\ledrightnote{\textcolor{pink}{Paris}}}{ }23. 1. 94\pend
           \pstart
           Sofort – ehe es \label{K_L02607-1v}\edtext{verboten}{\lemma{\textnormal{\emph{verboten}}}\Cendnote{\textnormal{Zu Kritik an \textcolor{blue}{Niemann}s \emph{\textcolor{green}{Der
                     Junggeſell}} siehe etwa \textcolor{blue}{Adolf Silberstein}: \emph{\textcolor{green}{Heirathen oder nicht?}} In: \emph{\textcolor{green}{Pester Lloyd}}, Nr. 151, 23. 6. 1894,
                     S. [3–4].}}}\label{K_L02607-1h} wird – kommen laſſen: \textsc{\textcolor{blue}{August Niemann}{}\ledrightnote{\textcolor{blue}{August Niemann}}}: \textcolor{green}{Der Junggeſell}{}\ledrightnote{\textcolor{green}{Der Junggesell. Humoreske}}. \textcolor{pink}{Berlin}{}\ledrightnote{\textcolor{pink}{Berlin}}, \textcolor{brown}{Philoſophiſch-Hiſtoriſcher Verlag}{}\ledrightnote{\textcolor{brown}{Philosophisch-historischer Verlag Dr. R. Salinger}}, \textsc{\textcolor{blue}{Dr. R. Salinger}{}\ledrightnote{\textcolor{blue}{Richard Salinger}}}, 1894.\pend
           \pstart
           Grüße,{\\[\baselineskip]}\spacefill\mbox{Paul Goldmann.}\pend
           \leftskip=0em{}\endnumbering\briefempfaengerindex{Schnitzler, Arthur@\textsc{Schnitzler, Arthur}!zzzGoldmann, Paul@\emph{von Paul Goldmann}!1894-01-231@{23. 1. 1894}|)be}\mylabel{h}  \normalsize

\doendnotes{C}
\bigskip
\vfill

\clearpage

\footnotesize

\lohead{\textsc{register}}

% Definiere theindex-Environment komplett neu ohne reledmac
\makeatletter
\renewenvironment{theindex}{%
  \section*{\indexname}%
  \setlength{\parindent}{0pt}%
  \setlength{\parskip}{0pt plus 0.3pt}%
  \let\item\@idxitem
}{%
  \clearpage
}
\makeatother

\IfFileExists{\jobname-pw.ind}{\input{\jobname-pw.ind}}{}

\end{document}

      