%% latex-korrekturansicht-vorspann.tex
%% Vorspann für die Korrekturansicht.
%% Lädt die gemeinsame Datei latex-vorspann.tex mit gesetztem Schalter.

\newif\ifkorrekturansicht
\korrekturansichttrue

\input{../tex-inputs/latex-vorspann}


               \section[Richard Beer-Hofmann an Arthur Schnitzler, 5. 6. 1897]{ Richard Beer-Hofmann an Arthur Schnitzler, 5. 6. 1897}\nopagebreak\mylabel{v}\rehead{ }\normalsize\beginnumbering\briefempfaengerindex{Schnitzler, Arthur@\textsc{Schnitzler, Arthur}!zzzBeer-Hofmann, Richard@\emph{von Richard Beer-Hofmann}!1897-06-052@{5. 6. 1897}|(be} \toendnotes[C]{\smallbreak\pagebreak[2]} \Standort{CUL, Schnitzler, B 8.}
\physDesc{Brief, 1 Blatt, 4 Seiten
\newline{}Handschrift: blauer Buntstift, lateinische Kurrent\newline{}Ordnung: mit Bleistift von unbekannter Hand nummeriert: »97« }\buchAbdrucke{\weitereDrucke{Arthur Schnitzler, Richard Beer-Hofmann: \emph{Briefwechsel 1891–1931}. Hg. Konstanze Fliedl. Wien, Zürich: \emph{Europaverlag} 1992, S. 107.} }\toendnotes[C]{\smallbreak}\pstart
           \centering{}{\pb}\textcolor{pink}{Ischl}{}\ledrightnote{\textcolor{pink}{Bad Ischl}}{ }5/VI 97\pend
           \pstart
           Lieber Arthur, ich bin seit Donnerstag hier und wünsche
               Sie recht bald auch hier zu haben. Wann ko{\geminationm}en Sie voraussichtlich, und wie lange {\pb}bleiben Sie hier? Anderes hoffe ich
               mündlich von Ihnen zu erfahren.\pend
           \pstart
           Ko{\geminationm}t \textcolor{blue}{Goldmann}{}\ledrightnote{\textcolor{blue}{Paul Goldmann}}, und
               wann?\pend
           \pstart
           Hier ist es wunderschön und da ich nur 4 Lectionen {\pb}im Radfahren erst in \textcolor{pink}{Wien}{}\ledrightnote{\textcolor{pink}{Wien}} geno{\geminationm}en habe, muß ich hier
               weiterlernen. Ich hoffe aber daß Rad u. Radfahren mich nicht hindern werden am \textcolor{green}{{\dots}ten Capitel}{}\ledrightnote{→\textcolor{green}{Der Tod Georgs}}{ }{\pb}zu arbeiten.\pend
           \pstart
           Herzlichst{\\[\baselineskip]}Ihr \spacefill\mbox{Richard}\pend
           \leftskip=0em{}\endnumbering\briefempfaengerindex{Schnitzler, Arthur@\textsc{Schnitzler, Arthur}!zzzBeer-Hofmann, Richard@\emph{von Richard Beer-Hofmann}!1897-06-052@{5. 6. 1897}|)be}\mylabel{h}  \normalsize

\doendnotes{C}
\bigskip
\vfill

\clearpage

\footnotesize

\lohead{\textsc{register}}

% Definiere theindex-Environment komplett neu ohne reledmac
\makeatletter
\renewenvironment{theindex}{%
  \section*{\indexname}%
  \setlength{\parindent}{0pt}%
  \setlength{\parskip}{0pt plus 0.3pt}%
  \let\item\@idxitem
}{%
  \clearpage
}
\makeatother

\IfFileExists{\jobname-pw.ind}{\input{\jobname-pw.ind}}{}

\end{document}

      