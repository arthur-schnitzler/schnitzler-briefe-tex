%% latex-korrekturansicht-vorspann.tex
%% Vorspann für die Korrekturansicht.
%% Lädt die gemeinsame Datei latex-vorspann.tex mit gesetztem Schalter.

\newif\ifkorrekturansicht
\korrekturansichttrue

\input{../tex-inputs/latex-vorspann}


               \section[Arthur Schnitzler: Widmungsexemplar Reichtum für Hugo von Hofmannsthal, {[}nach Mitte Oktober 1891?{]}]{ Arthur Schnitzler: Widmungsexemplar Reichtum für Hugo von
                    Hofmannsthal, {[}nach Mitte Oktober 1891?{]}}\nopagebreak\mylabel{v}\rehead{ }\normalsize\beginnumbering\briefempfaengerindex{Hofmannsthal, Hugo von@\textsc{Hofmannsthal, Hugo von}!zzzSchnitzler, Arthur@\emph{von Arthur Schnitzler}!1891-10-151@{{[}nach Mitte Oktober 1891?{]}}|(be} \toendnotes[C]{\smallbreak\pagebreak[2]} \Standort{FDH, FDH 3239.}
\physDesc{Widmung am Umschlag
\newline{}Handschrift: schwarze Tinte, deutsche Kurrent}\buchAbdrucke{\weitereDrucke{Hugo von Hofmannsthal: \emph{Bibliothek}. Hg. Ellen Ritter † in Zusammenarbeit mit Dalia Bukauskaité und
                                Konrad Heumann. Frankfurt am Main: \emph{S. Fischer} 2011, S. 605 (Sämtliche Werke. Kritische Ausgabe, XL).} }\toendnotes[C]{\smallbreak}\pstart
           \noindent{}{\pb}Meinem \damage{li}eben Freunde \textcolor{gray}{L}\damage{oris}\pend
           \pstart \spacefill\mbox{Arth}\pend{}{\bigskip}\pstart
           \noindent{}\centering{}\textcolor{gray}{\textbf{\textcolor{green}{\so{Reichtum}}{}\ledrightnote{\textcolor{green}{Reichtum. Erzählung}}}}\pend
           \pstart
           \noindent{}\centering{}\textcolor{gray}{\textbf{Erzählung}}{\\}\textcolor{gray}{\textbf{von}}{\\}\textcolor{gray}{\textbf{Arthur Schnitzler.}}\pend
           {\bigskip}\pstart
           \noindent{}\centering{}\textcolor{gray}{\textbf{\label{K_L00044_1v}\edtext{Separat-Abdruck}{\lemma{\textnormal{\emph{Separat-Abdruck}}}\Cendnote{\textnormal{In seinem Brief vom 11. 9. 1891 schreibt \textcolor{blue}{Schnitzler}, noch mehrere Änderungen an der
                            Zeitschriftenfassung für den Separatabdruck vornehmen zu wollen. Es ist
                            anzunehmen, dass dieser Druck zeitnah zum Abdruck des 4. Teils am
                                15. 10. 1891 fertiggestellt wurde.}}}\label{K_L00044_1h} aus der »\textcolor{green}{Modernen Rundſchau}{}\ledrightnote{\textcolor{green}{Moderne Rundschau}}«.}}\pend
           \pstart
           \noindent{}\centering{}\textcolor{gray}{\textbf{\so{Halbmonatſchrift.}}}\pend
           \pstart
           \noindent{}\centering{}\textcolor{gray}{\textbf{Herausgegeben von \textcolor{blue}{\textbf{J. Joachim}}{}\ledrightnote{\textcolor{blue}{Jaques Joachim}} und \textcolor{blue}{\textbf{E. M. Kafka}}{}\ledrightnote{\textcolor{blue}{Eduard Michael Kafka}}.}}\pend
           \pstart
           \noindent{}\centering{}\textcolor{gray}{\textbf{Druck von \textcolor{brown}{Carl Steinhardt {\kaufmannsund} Cie.}{}\ledrightnote{\textcolor{brown}{Carl Steinhardt {\kaufmannsund} Co.}} (verantw. Leiter \textcolor{blue}{Guſtav Röttig}{}\ledrightnote{\textcolor{blue}{Gustav Röttig}}), \textcolor{pink}{Wien}{}\ledrightnote{\textcolor{pink}{Wien}}, \textcolor{pink}{IX.,
                            Hahngaſſe 12}{}\ledrightnote{\textcolor{pink}{Hahngasse}}.}}\pend
           \endnumbering\briefempfaengerindex{Hofmannsthal, Hugo von@\textsc{Hofmannsthal, Hugo von}!zzzSchnitzler, Arthur@\emph{von Arthur Schnitzler}!1891-10-151@{{[}nach Mitte Oktober 1891?{]}}|)be}\mylabel{h}  \normalsize

\doendnotes{C}
\bigskip
\vfill

\clearpage

\footnotesize

\lohead{\textsc{register}}

% Definiere theindex-Environment komplett neu ohne reledmac
\makeatletter
\renewenvironment{theindex}{%
  \section*{\indexname}%
  \setlength{\parindent}{0pt}%
  \setlength{\parskip}{0pt plus 0.3pt}%
  \let\item\@idxitem
}{%
  \clearpage
}
\makeatother

\IfFileExists{\jobname-pw.ind}{\input{\jobname-pw.ind}}{}

\end{document}

      