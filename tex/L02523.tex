%% latex-korrekturansicht-vorspann.tex
%% Vorspann für die Korrekturansicht.
%% Lädt die gemeinsame Datei latex-vorspann.tex mit gesetztem Schalter.

\newif\ifkorrekturansicht
\korrekturansichttrue

\input{../tex-inputs/latex-vorspann}


               \section[Arthur Schnitzler an Thomas Mann, 18. 11. 1929]{ Arthur Schnitzler an Thomas Mann, 18. 11. 1929}\nopagebreak\mylabel{v}\rehead{ }\normalsize\beginnumbering\briefempfaengerindex{Mann, Thomas@\textsc{Mann, Thomas}!zzzSchnitzler, Arthur@\emph{von Arthur Schnitzler}!1929-11-181@{18. 11. 1929}|(be} \toendnotes[C]{\smallbreak\pagebreak[2]} \Standort{Zürich, Thomas-Mann-Archiv, B-II-SCHNM-4.}
\physDesc{Brief, 1 Blatt (Briefpapier mit Trauerrand), 2 Seiten, Umschlag mit Trauerrand
\newline{}Handschrift: schwarze Tinte, lateinische Kurrent\newline{}Versand: Stempel: »\nobreak{}\oindex{XVIII., Waehring@\textbf{XVIII., Währing}, \emph{Bezirk (A.BZK)}|pwk}18/1 Wien 110, 18. XI. 29, 17\nobreak{}«.  }\pstart{}{\pb}\textcolor{gray}{\textbf{A. S.}}\pend{}\pstart{}\textcolor{pink}{\textcolor{gray}{\textbf{WIEN, XVIII.}}}{}\ledrightnote{\textcolor{pink}{XVIII., Währing}}\pend{}\pstart{}\textcolor{pink}{\textcolor{gray}{\textbf{STERNWARTESTR. 71}}}{}\ledrightnote{\textcolor{pink}{Sternwartestraße}}\pend{}{\bigskip}\pstart{}{\pb}Herrn Thomas Mann\pend{}\pstart{}\textcolor{pink}{München}{}\ledrightnote{\textcolor{pink}{München}}\pend{}\pstart{}\textcolor{pink}{Puschingerstr. 1}{}\ledrightnote{\textcolor{pink}{Poschingerstraße}}.\pend{}{\bigskip}\pstart
           \raggedleft{}{\pb}\textcolor{pink}{Wien}{}\ledrightnote{\textcolor{pink}{Wien}}, 18. 11. 924\pend
           \pstart{}Mein lieber und verehrter Thomas Mann,\pend\pstart
           Sie und der \textcolor{brown}{Nobelpreis}{}\ledrightnote{\textcolor{brown}{Nobelpreis}} Sie gehören schon lang
                    zusammen – womit ich keineswegs die Bedeutung von Preisen überhaupt überschätzen
                    möchte. Trotzdem freut es Einen – und ich hoffe, auch Sie haben sich
                    gefreut.\pend
           \pstart
           Im übrigen glaub ich, dſs ich Ihnen weiter nicht viel sagen muſs. Sie wissen was
                    Sie der Welt, – Sie wissen auch was {\pb}mir sind. Ich liebe Ihre Haltung, Ihr Werk, ich liebe Sie. Von meiner
                    Bewunderung spreche ich nicht, – ich finde, hier ist beides, Bewunderung und
                    Liebe eins.\pend
           \pstart
           Bleiben Sie der Sie sind, und lange; damit ist auch etwas ausgedrückt, daſs Sie
                    immer mehr werden.\pend
           \pstart
           Glückwunsch und Gruß, und auf Wiedersehen, hoffentlich.\pend
           \pstart
           Ihr{\\[\baselineskip]}\spacefill\mbox{ArthSchnitzler}\pend
           \leftskip=0em{}\endnumbering\briefempfaengerindex{Mann, Thomas@\textsc{Mann, Thomas}!zzzSchnitzler, Arthur@\emph{von Arthur Schnitzler}!1929-11-181@{18. 11. 1929}|)be}\mylabel{h}  \normalsize

\doendnotes{C}
\bigskip
\vfill

\clearpage

\footnotesize

\lohead{\textsc{register}}

% Definiere theindex-Environment komplett neu ohne reledmac
\makeatletter
\renewenvironment{theindex}{%
  \section*{\indexname}%
  \setlength{\parindent}{0pt}%
  \setlength{\parskip}{0pt plus 0.3pt}%
  \let\item\@idxitem
}{%
  \clearpage
}
\makeatother

\IfFileExists{\jobname-pw.ind}{\input{\jobname-pw.ind}}{}

\end{document}

      