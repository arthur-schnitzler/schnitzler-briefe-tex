%% latex-korrekturansicht-vorspann.tex
%% Vorspann für die Korrekturansicht.
%% Lädt die gemeinsame Datei latex-vorspann.tex mit gesetztem Schalter.

\newif\ifkorrekturansicht
\korrekturansichttrue

\input{../tex-inputs/latex-vorspann}


               \section[Jakob Julius David an Arthur Schnitzler, {[}27. 3. 1895{]}]{ Jakob Julius David an Arthur Schnitzler, {[}27. 3. 1895{]}}\nopagebreak\mylabel{v}\rehead{ }\normalsize\beginnumbering\briefempfaengerindex{Schnitzler, Arthur@\textsc{Schnitzler, Arthur}!zzzDavid, Jakob Julius@\emph{von Jakob Julius David}!1895-03-272@{{[}27. 3. 1895{]}}|(be} \toendnotes[C]{\smallbreak\pagebreak[2]} \Standort{CUL, Schnitzler, B 25.}
\physDesc{Brief, 1 Blatt, 1 Seite
\newline{}Handschrift: schwarze Tinte, lateinische Kurrent
\newline{}Schnitzler: mit Bleistift zuerst unleserlich datiert, das dann gestrichen und
            neuerlich: »27/3 95.« \newline{}Ordnung: mit Bleistift von unbekannter Hand nummeriert: »2« }\toendnotes[C]{\smallbreak}\pstart\center{}{\pb}Werther Herr
                        Doctor!\pend\pstart
           \label{K_L00427_1v}\edtext{\textcolor{green}{Es}{}\ledrightnote{→\textcolor{green}{Buchmacher und Künstler}}}{\lemma{\textnormal{\emph{Es}}}\Cendnote{\textnormal{Wohl diese Rezension von \emph{\textcolor{green}{Sterben}}: \textcolor{blue}{Felix
                                Poppenberg}: \emph{\textcolor{green}{Buchmacher und
                            Künstler}}. In: \emph{\textcolor{green}{Das Magazin für
                                Litteratur}}, Jg. 64, Nr. 9,
                                2. 3. 1895, Sp. 265–270, hier
                                Sp. 269–270.}}}\label{K_L00427_1h} war im Februar oder
                        März. \textcolor{blue}{Neuma{\geminationn}-Hofer}{}\ledrightnote{\textcolor{blue}{Gilbert Otto Neumann-Hofer}} stellt Ihnen die \textcolor{green}{Nu{\geminationm}er}{}\ledrightnote{→\textcolor{green}{Magazin für die Literatur des Auslandes}}
                    sicher zur Verfügung. Ich weiß nicht, wohin ich das Ding kramte. Mir liegt an
                    den Sachen so gar nichts.\pend
           \pstart
           Herzlichst{\\[\baselineskip]}\spacefill\mbox{David}\pend
           \leftskip=0em{}\endnumbering\briefempfaengerindex{Schnitzler, Arthur@\textsc{Schnitzler, Arthur}!zzzDavid, Jakob Julius@\emph{von Jakob Julius David}!1895-03-272@{{[}27. 3. 1895{]}}|)be}\mylabel{h}  \normalsize

\doendnotes{C}
\bigskip
\vfill

\clearpage

\footnotesize

\lohead{\textsc{register}}

% Definiere theindex-Environment komplett neu ohne reledmac
\makeatletter
\renewenvironment{theindex}{%
  \section*{\indexname}%
  \setlength{\parindent}{0pt}%
  \setlength{\parskip}{0pt plus 0.3pt}%
  \let\item\@idxitem
}{%
  \clearpage
}
\makeatother

\IfFileExists{\jobname-pw.ind}{\input{\jobname-pw.ind}}{}

\end{document}

      