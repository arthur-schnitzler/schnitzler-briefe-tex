%% latex-korrekturansicht-vorspann.tex
%% Vorspann für die Korrekturansicht.
%% Lädt die gemeinsame Datei latex-vorspann.tex mit gesetztem Schalter.

\newif\ifkorrekturansicht
\korrekturansichttrue

\input{../tex-inputs/latex-vorspann}


               \section[Stefan Großmann an Arthur Schnitzler, {[}7.{]} 2. 1911]{ Stefan Großmann an Arthur Schnitzler,
                    {[}7.{]} 2. 1911}\nopagebreak\mylabel{v}\rehead{ }\normalsize\beginnumbering\briefempfaengerindex{Schnitzler, Arthur@\textsc{Schnitzler, Arthur}!zzzGrossmann, Stefan@\emph{von Stefan Großmann}!1911-02-072@{{[}7.{]} 2. 1911}|(be} \toendnotes[C]{\smallbreak\pagebreak[2]} \Standort{CUL, Schnitzler, B 34.}
\physDesc{Brief, 1 Blatt, 1 Seite
\newline{}Handschrift: schwarze Tinte, deutsche Kurrent
\newline{}Schnitzler: 1) Datum  mit Bleistift geändert zu »7.« 2) mit rotem Buntstift zwei Unterstreichungen\newline{}Ordnung: mit Bleistift von unbekannter Hand nummeriert: »9« }\pstart
           {\pb}\textcolor{gray}{\textbf{STEFAN GROHSMANN}}\hfill \textcolor{gray}{\textbf{\textcolor{pink}{WIEN}{}\ledrightnote{\textcolor{pink}{Wien}},}}{ }11. Februar 1911\pend
           \pstart
           \textcolor{gray}{\textbf{LEITER DER \textcolor{brown}{FREIEN
                                    VOLKSBÜHNE}{}\ledrightnote{\textcolor{brown}{Wiener Freie Volksbühne}}}}\hfill \textcolor{gray}{\textbf{\textcolor{pink}{VI. UFERGASSE 18}{}\ledrightnote{\textcolor{pink}{Linke Wienzeile}}.}}\pend
           \pstart\center{}Sehr verehrter Herr.\pend\pstart
           Verzeihen Sie, daſs ich Ihre werthvolle Zeit für zwei Minuten mit einer
                    Klatſchgeſchichte \strikeout{b} in Anſpruch nehmen
                    muſs.\pend
           \pstart
           Ein junger Literat (von Talent) Herr \uline{\textcolor{blue}{\textsc{Ehrenstein}}{}\ledrightnote{\textcolor{blue}{Albert Ehrenstein}}} erzählt verſchiedenen Leuten, u. A. auch dem \textcolor{blue}{\textcolor{brown}{Fackel}{}\ledrightnote{\textcolor{brown}{Die Fackel}}kraus}{}\ledrightnote{\textcolor{blue}{Karl Kraus}}, Sie hätten ihm
                    »beſtätigt«, daſs ich meine Macht als Kritiker zu erotiſchen Erpreſſungen an
                    Schauſpielerinnen ausgenutzt hätte.\pend
           \pstart
           Ich weiß wohl, daſs derlei Klatſchgeſchichten zu dem Koth gehören, der jeden
                    Schnell-Schreibenden befleckt, aber ich bitte Sie doch um eine Silbe darüber,
                    daſs Sie eine ſolche »Beſtätigung« nicht gaben, wie Sie ſie ja auch nicht geben
                    konnten.\pend
           \pstart
           Verzeihen Sie die lästige Behelligung!! Wäre Ihr Name in der dummen Geſchichte
                    nicht eitel genannt worden, hätte ich sie nicht beachtet.\pend
           \pstart
           Mit aufrichtigſter Hochſchätzung:{\\[\baselineskip]}\spacefill\mbox{Stefan Großmann}\pend
           \leftskip=0em{}\endnumbering\briefempfaengerindex{Schnitzler, Arthur@\textsc{Schnitzler, Arthur}!zzzGrossmann, Stefan@\emph{von Stefan Großmann}!1911-02-072@{{[}7.{]} 2. 1911}|)be}\mylabel{h}  \normalsize

\doendnotes{C}
\bigskip
\vfill

\clearpage

\footnotesize

\lohead{\textsc{register}}

% Definiere theindex-Environment komplett neu ohne reledmac
\makeatletter
\renewenvironment{theindex}{%
  \section*{\indexname}%
  \setlength{\parindent}{0pt}%
  \setlength{\parskip}{0pt plus 0.3pt}%
  \let\item\@idxitem
}{%
  \clearpage
}
\makeatother

\IfFileExists{\jobname-pw.ind}{\input{\jobname-pw.ind}}{}

\end{document}

      