%% latex-korrekturansicht-vorspann.tex
%% Vorspann für die Korrekturansicht.
%% Lädt die gemeinsame Datei latex-vorspann.tex mit gesetztem Schalter.

\newif\ifkorrekturansicht
\korrekturansichttrue

\input{../tex-inputs/latex-vorspann}


               \section[Arthur Schnitzler an Richard Beer-Hofmann, {[}zwischen 24. 9. 1892 und 1. 5. 1901{]}]{ Arthur Schnitzler an Richard Beer-Hofmann, {[}zwischen 24. 9. 1892 und
               1. 5. 1901{]}}\nopagebreak\mylabel{v}\rehead{ }\normalsize\beginnumbering\briefempfaengerindex{Beer-Hofmann, Richard@\textsc{Beer-Hofmann, Richard}!zzzSchnitzler, Arthur@\emph{von Arthur Schnitzler}!1892-09-241@{{[}zwischen 24. 9. 1892 und
                  1. 5. 1901{]}}|(be} \toendnotes[C]{\smallbreak\pagebreak[2]} \Standort{YCGL, MSS 31.}
\physDesc{Briefkarte, Umschlag
\newline{}Handschrift: Bleistift, deutsche Kurrent\newline{}Versand: ohne postalischen Übermittlungsvermerk }\toendnotes[C]{\smallbreak}\pstart{}{\pb}Hrn \textsc{Dr. Rich.
                     Beer-Hofmann}\pend{}\pstart{}\textcolor{pink}{Wien}{}\ledrightnote{\textcolor{pink}{Wien}}\pend{}\pstart{}\textsc{\textcolor{pink}{I. \label{K_L00120_1v}\edtext{Wollzeile 15}{\lemma{\textnormal{\emph{Wollzeile 15}}}\Cendnote{\textnormal{Das
                           Korrespondenzstück ist undatiert. Durch die Übersiedlung \textcolor{blue}{Beer-Hofmann}s im September
                                 1892 in die \textcolor{pink}{Wollzeile}, wo er bis Ende April 1901 wohnte, ist eine grobe Zuordnung
                           möglich.}}}\label{K_L00120_1h}}{}\ledrightnote{\textcolor{pink}{Wollzeile}}.}\pend{}{\bigskip}\pstart
           \noindent{}{\pb}lieber Richard,  hier iſt der \textcolor{blue}{Herr}{}\ledrightnote{→\textcolor{blue}{?? [Mann ohne Winterrock]}} mit dem Winterrock, \textsc{resp.}
               ohne den Winterrock.\pend
           \pstart Ihr \spacefill\mbox{Arthur.}\pend{}\pstart
           \noindent{}{\pb}Vielleicht geben Sie ihm auch ein paar Kreuzer. Er
                  fährt nach \textcolor{pink}{Linz}{}\ledrightnote{\textcolor{pink}{Linz}}.\pend
           \endnumbering\briefempfaengerindex{Beer-Hofmann, Richard@\textsc{Beer-Hofmann, Richard}!zzzSchnitzler, Arthur@\emph{von Arthur Schnitzler}!1892-09-241@{{[}zwischen 24. 9. 1892 und
                  1. 5. 1901{]}}|)be}\mylabel{h}  \normalsize

\doendnotes{C}
\bigskip
\vfill

\clearpage

\footnotesize

\lohead{\textsc{register}}

% Definiere theindex-Environment komplett neu ohne reledmac
\makeatletter
\renewenvironment{theindex}{%
  \section*{\indexname}%
  \setlength{\parindent}{0pt}%
  \setlength{\parskip}{0pt plus 0.3pt}%
  \let\item\@idxitem
}{%
  \clearpage
}
\makeatother

\IfFileExists{\jobname-pw.ind}{\input{\jobname-pw.ind}}{}

\end{document}

      