%% latex-korrekturansicht-vorspann.tex
%% Vorspann für die Korrekturansicht.
%% Lädt die gemeinsame Datei latex-vorspann.tex mit gesetztem Schalter.

\newif\ifkorrekturansicht
\korrekturansichttrue

\input{../tex-inputs/latex-vorspann}


               \section[Gertrud Rung an Arthur Schnitzler, 17. 2. 1925]{ Gertrud Rung an Arthur Schnitzler, 17. 2. 1925}\nopagebreak\mylabel{v}\rehead{ }\normalsize\beginnumbering\briefempfaengerindex{Schnitzler, Arthur@\textsc{Schnitzler, Arthur}!zzzRung, Gertrud@\emph{von Gertrud Rung}!1925-02-171@{17. 2. 1925}|(be} \toendnotes[C]{\smallbreak\pagebreak[2]} \Standort{CUL, Schnitzler, B 17.}
\physDesc{Brief, 1 Blatt, 2 Seiten
\newline{}Handschrift: schwarze Tinte, lateinische Kurrent
\newline{}Schnitzler: 1) mit Bleistift beschriftet: »\noindent{}(\textsc{Brandes}{ / }\textsc{Rung})« 2) mit rotem Buntstift vereinzelte Unterstreichungen\newline{}Ordnung: mit Bleistift von unbekannter Hand nummeriert:
                                    »56« }\buchAbdrucke{\weitereDrucke{Georg Brandes, Arthur Schnitzler: \emph{Ein Briefwechsel}. Hg. Kurt Bergel. Bern: \emph{Francke} 1956, S. 144.} }\toendnotes[C]{\smallbreak}\pstart
           \raggedleft{}{\pb}\textcolor{pink}{Kopenhagen}{}\ledrightnote{\textcolor{pink}{Kopenhagen}}{ }17-2-25\pend
           \pstart{}Hochverehrter Herr.\pend\pstart
           Dr \textcolor{blue}{Georg Brandes}{}\ledrightnote{\textcolor{blue}{Georg Brandes}} bittet Sie dringend ihm nicht zu
               verübeln, daß er Ihnen diesmal nicht persönlich schreibt. Die bevorstehende
               Vortragsreise nimmt die Zeit des Doktors derartig in Anspruch, daß er zu müde ist
               sein Correspondenz selber zu führen.\pend
           \pstart
           Dr \textcolor{blue}{Brandes}{}\ledrightnote{\textcolor{blue}{Georg Brandes}} beauftragt mich deshalb, Ihnen,
               hochverehrter Herr, zu sagen, daß es ihm eine ganz besondere Freude sein wird sich
               mit Ihnen irgendwo zusammen zu treffen.\hspace*{1.5em}Der erste
                  {\pb}\label{K_L02433_1v}\edtext{Vortrag}{\lemma{\textnormal{\emph{Vortrag}}}\Cendnote{\textnormal{\textcolor{blue}{Brandes} hielt in \textcolor{pink}{Berlin} nur einen Vortrag: am 31. 3. 1925 im \textcolor{pink}{Blüthner-Saal} zum Thema: \emph{\textcolor{green}{Das
                     heutige Europa}}. \textcolor{blue}{Brandes}’ Aufenthalt
                  und sein Vortrag fanden in der \textcolor{pink}{Berlin}er Presse
                  große Resonanz (vgl. \textcolor{blue}{A. F. Cohn}:
                        \emph{\textcolor{green}{Georg Brandes in Berlin}}. In: \emph{\textcolor{green}{Berliner Tageblatt}}, Jg. 54, Nr. 153,
                        31. 3. 1925, Abend-Ausgabe, S. 4. Anschließend fuhr \textcolor{blue}{Brandes} nach \textcolor{pink}{Wien}, um dort am 8. 4. 1931 den Vortrag zu
                  wiederholen.}}}\label{K_L02433_1h} soll in \textcolor{pink}{Berlin}{}\ledrightnote{\textcolor{pink}{Berlin}} am 25
                  März stattfinden, der zweite folgt innerhalb einer Woche. Dr \textcolor{blue}{Brandes}{}\ledrightnote{\textcolor{blue}{Georg Brandes}} weißt noch nicht in welchem Hotel er
               wohnen wird, weil sein \textcolor{blue}{Impresario}{}\ledrightnote{→\textcolor{blue}{J.  Span}} dies für ihn arrangieren wird.\hspace*{1.5em}Dr \textcolor{blue}{Brandes bittet}{}\ledrightnote{\textcolor{blue}{Georg Brandes}} Sie deshalb die Güte haben zu
               wollen bei diesem Herrn, \textcolor{blue}{J. Span}{}\ledrightnote{\textcolor{blue}{J.  Span}}, \textcolor{pink}{Berlinerstraße 149 Charlottenburg}{}\ledrightnote{\textcolor{pink}{Straße des 17. Juni}}, Ihre Adresse abzugeben, so
               daß er sich gleich nach seiner Ankunft in Verbindung mit Ihnen setzen kann.\pend
           \pstart
           Dr \textcolor{blue}{Brandes}{}\ledrightnote{\textcolor{blue}{Georg Brandes}} bittet Sie um seine Ergebenheit und
               warme Freundschaft versichert zu sein und grüßt Sie auf das herzlichste.\pend
           \pstart
           Mit vorzüglicher Hochachtung{\\[\baselineskip]}für Dr. \textcolor{blue}{Georg
                  Brandes}{}\ledrightnote{\textcolor{blue}{Georg Brandes}}{\\[\baselineskip]}\spacefill\mbox{G. Rung / Sekretär.}\pend
           \leftskip=0em{}\endnumbering\briefempfaengerindex{Schnitzler, Arthur@\textsc{Schnitzler, Arthur}!zzzRung, Gertrud@\emph{von Gertrud Rung}!1925-02-171@{17. 2. 1925}|)be}\mylabel{h}  \normalsize

\doendnotes{C}
\bigskip
\vfill

\clearpage

\footnotesize

\lohead{\textsc{register}}

% Definiere theindex-Environment komplett neu ohne reledmac
\makeatletter
\renewenvironment{theindex}{%
  \section*{\indexname}%
  \setlength{\parindent}{0pt}%
  \setlength{\parskip}{0pt plus 0.3pt}%
  \let\item\@idxitem
}{%
  \clearpage
}
\makeatother

\IfFileExists{\jobname-pw.ind}{\input{\jobname-pw.ind}}{}

\end{document}

      