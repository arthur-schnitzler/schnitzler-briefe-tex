%% latex-korrekturansicht-vorspann.tex
%% Vorspann für die Korrekturansicht.
%% Lädt die gemeinsame Datei latex-vorspann.tex mit gesetztem Schalter.

\newif\ifkorrekturansicht
\korrekturansichttrue

\input{../tex-inputs/latex-vorspann}


               \section[Arthur Schnitzler an Richard Beer-Hofmann, {[}zwischen 5. 5. 1893 und 2. 5. 1894{]}]{ Arthur Schnitzler an Richard Beer-Hofmann, {[}zwischen 5. 5. 1893 und
               2. 5. 1894{]}}\nopagebreak\mylabel{v}\rehead{ }\normalsize\beginnumbering\briefempfaengerindex{Beer-Hofmann, Richard@\textsc{Beer-Hofmann, Richard}!zzzSchnitzler, Arthur@\emph{von Arthur Schnitzler}!1893-05-053@{{[}zwischen 5. 5. 1893 und
                  2. 5. 1894{]}}|(be} \toendnotes[C]{\smallbreak\pagebreak[2]} \Standort{YCGL, MSS 31.}
\physDesc{Visitenkarte mit Trauerrand
\newline{}Handschrift: Bleistift, deutsche Kurrent}\toendnotes[C]{\smallbreak}\pstart
           \noindent{}\centering{}{\pb}\textcolor{gray}{\textbf{\label{K_L00213-1v}\edtext{Dr. Arthur
                     Schnitzler}{\lemma{\textnormal{\emph{Dr. Arthur
                     Schnitzler}}}\Cendnote{\textnormal{Der Trauerrand verortet
                        diese Visitenkarte in das Jahr nach dem Ableben von \textcolor{blue}{Johann Schnitzler}.}}}\label{K_L00213-1h}}}\pend
           \pstart
           \noindent{}herzlich grüßend\pend
           \pstart
           {\pb}\textsc{Herrn Dr. Rich. Beer Hofmann}\pend
           \pstart
           \textsc{\textcolor{pink}{I. Wollzeile}{}\ledrightnote{\textcolor{pink}{Wollzeile}} 15.}\pend
           \endnumbering\briefempfaengerindex{Beer-Hofmann, Richard@\textsc{Beer-Hofmann, Richard}!zzzSchnitzler, Arthur@\emph{von Arthur Schnitzler}!1893-05-053@{{[}zwischen 5. 5. 1893 und
                  2. 5. 1894{]}}|)be}\mylabel{h}  \normalsize

\doendnotes{C}
\bigskip
\vfill

\clearpage

\footnotesize

\lohead{\textsc{register}}

% Definiere theindex-Environment komplett neu ohne reledmac
\makeatletter
\renewenvironment{theindex}{%
  \section*{\indexname}%
  \setlength{\parindent}{0pt}%
  \setlength{\parskip}{0pt plus 0.3pt}%
  \let\item\@idxitem
}{%
  \clearpage
}
\makeatother

\IfFileExists{\jobname-pw.ind}{\input{\jobname-pw.ind}}{}

\end{document}

      