%% latex-korrekturansicht-vorspann.tex
%% Vorspann für die Korrekturansicht.
%% Lädt die gemeinsame Datei latex-vorspann.tex mit gesetztem Schalter.

\newif\ifkorrekturansicht
\korrekturansichttrue

\input{../tex-inputs/latex-vorspann}


               \section[Georg Brandes an Arthur Schnitzler, 11. 6. 1923]{ Georg Brandes an Arthur Schnitzler, 11. 6. 1923}\nopagebreak\mylabel{v}\rehead{ }\normalsize\beginnumbering\briefempfaengerindex{Schnitzler, Arthur@\textsc{Schnitzler, Arthur}!zzzBrandes, Georg@\emph{von Georg Brandes}!1923-06-111@{11. 6. 1923}|(be} \toendnotes[C]{\smallbreak\pagebreak[2]} \Standort{CUL, Schnitzler, B 17.}
\physDesc{Postkarte
\newline{}Handschrift: schwarze Tinte, lateinische Kurrent\newline{}Versand: Stempel: »\nobreak{}\oindex{Hornbæk@\textbf{Hornbæk}, \emph{https://www.geonames.org/ontologyP.PPL}|pwk}Hornbæk, 11. 6. 23, 6–8 E\nobreak{}«.  
\newline{}Schnitzler: 1) Markierung  (?) mit Bleistift: »\textsc{\uline{A}}« (für: Abgeschrieben?) 2) mit rotem Buntstift zwei Unterstreichungen\newline{}Ordnung: mit Bleistift von unbekannter Hand nummeriert: »54« }\buchAbdrucke{\weitereDrucke{Georg Brandes, Arthur Schnitzler: \emph{Ein Briefwechsel}. Hg. Kurt Bergel. Bern: \emph{Francke} 1956, S. 138–139.} }\toendnotes[C]{\smallbreak}\pstart{}{\pb}Herrn Dr. Arthur
                        Schnitzler\pend{}\pstart{}\textcolor{pink}{Sternwartestrasse 71}{}\ledrightnote{\textcolor{pink}{Sternwartestraße}}\pend{}\pstart{}\textcolor{pink}{Wien XVIII}{}\ledrightnote{\textcolor{pink}{XVIII., Währing}}\pend{}{\bigskip}\pstart
           \raggedleft{}{\pb}\textcolor{pink}{Kopenhagen}{}\ledrightnote{\textcolor{pink}{Kopenhagen}}{ }11 Juni 23\pend
           \pstart
           Liebster Schnitzler\hspace*{3.5em}Seien Sie bedankt für die Güte, die Sie nicht
                    weniger als drei mal einen Patienten aufsuchen lies. Ich war und bin Ihnen von
                    ganzem Herzen dankbar. Ich hoffe dass Sie in \textcolor{pink}{Stockholm}{}\ledrightnote{\textcolor{pink}{Stockholm}} gute Erfahrungen machte{[}n{]}. Ich habe leider keine \textcolor{pink}{schwedische}{}\ledrightnote{\textcolor{pink}{Schweden}} Zeitung gesehen. Ich habe den
                    Wunsch, dass es Ihnen in der hübschen \textcolor{pink}{Stadt}{}\ledrightnote{→\textcolor{pink}{Stockholm}} gut ging und dass Sie was verdienten. Die \textcolor{pink}{schwedische}{}\ledrightnote{\textcolor{pink}{Schweden}} Krone ist viel mehr werth als die
                        \textcolor{pink}{dänische}{}\ledrightnote{\textcolor{pink}{Dänemark}}.\pend
           \pstart
           Ich bin augenblicklich auf dem Lande (\textcolor{pink}{Hornbæk}{}\ledrightnote{\textcolor{pink}{Hornbæk}},
                        \textcolor{pink}{Villa Iris}{}\ledrightnote{\textcolor{pink}{Villa Iris}}) um mich zu erholen, und es
                    geht mir sehr gut, wäre nur nicht der Sommer so schlecht, das Wetter so kalt und
                    regnerisch. Ich habe recht viel gearbeitet, gebe die 6\textsuperscript{te} Ausgabe meiner alten vor halbhundert Jahren geschriebenen \textcolor{green}{Hauptströmungen}{}\ledrightnote{\textcolor{green}{Hauptströmungen der Literatur des neunzehnten Jahrhunderts}} heraus, in vermehrter und
                    verbesserter Gestalt, merze {\pb}Irrthümer aus und füge Binsenwahrheiten hinzu.\pend
           \pstart
           Es war eine wahre Freude für mich, Sie wiederzusehen, anscheinend unangefochten
                    von all dem Ungemach, das sich über Ihr \textcolor{pink}{Land}{}\ledrightnote{→\textcolor{pink}{Österreich}} wie über ganz \textcolor{pink}{Europa}{}\ledrightnote{\textcolor{pink}{Europa}} gestürzt hat. Sie haben augenscheinlich nicht weniger
                    Widerstandskraft als Ihr jugendlicher Verehrer\pend
           \pstart \spacefill\mbox{G. B.}\pend{}\pstart
           \noindent{}\label{T_L02401_1v}\edtext{Grüssen Sie den \textcolor{blue}{Sohn}{}\ledrightnote{→\textcolor{blue}{Heinrich Schnitzler}}, von dem Sie mir sprachen}{\lemma{\textnormal{\emph{Grüssen … sprachen}}}\Cendnote{\textnormal{am linken Rand}}}\label{T_L02401_1h}\pend
           \endnumbering\briefempfaengerindex{Schnitzler, Arthur@\textsc{Schnitzler, Arthur}!zzzBrandes, Georg@\emph{von Georg Brandes}!1923-06-111@{11. 6. 1923}|)be}\mylabel{h}  \normalsize

\doendnotes{C}
\bigskip
\vfill

\clearpage

\footnotesize

\lohead{\textsc{register}}

% Definiere theindex-Environment komplett neu ohne reledmac
\makeatletter
\renewenvironment{theindex}{%
  \section*{\indexname}%
  \setlength{\parindent}{0pt}%
  \setlength{\parskip}{0pt plus 0.3pt}%
  \let\item\@idxitem
}{%
  \clearpage
}
\makeatother

\IfFileExists{\jobname-pw.ind}{\input{\jobname-pw.ind}}{}

\end{document}

      