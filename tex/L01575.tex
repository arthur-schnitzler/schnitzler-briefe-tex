%% latex-korrekturansicht-vorspann.tex
%% Vorspann für die Korrekturansicht.
%% Lädt die gemeinsame Datei latex-vorspann.tex mit gesetztem Schalter.

\newif\ifkorrekturansicht
\korrekturansichttrue

\input{../tex-inputs/latex-vorspann}


               \section[Adolf Treibl an Arthur Schnitzler, {[}22.? 1. 1906{]}]{ Adolf Treibl an Arthur Schnitzler, {[}22.? 1. 1906{]}}\nopagebreak\mylabel{v}\rehead{ }\normalsize\beginnumbering\briefempfaengerindex{Schnitzler, Arthur@\textsc{Schnitzler, Arthur}!zzzTreibl, Adolf@\emph{von Adolf Treibl}!1906-01-221@{{[}22.? 1. 1906{]}}|(be} \toendnotes[C]{\smallbreak\pagebreak[2]} \Standort{DLA, A:Schnitzler, HS.NZ85.1.4815,1.}
\physDesc{Brief, 2 Blätter, 5 Seiten
\newline{}Handschrift: schwarze Tinte, deutsche Kurrent
\newline{}Schnitzler: mit Bleistift beschriftet: »\textsc{Ehrenstein (Treibl}« }\toendnotes[C]{\smallbreak}\pstart
           \noindent{}{\pb}\textsc{Euer Hochwohlgeboren}\pend
           \pstart{}Hochverehrter Herr \textsc{Doctor}\pend\pstart
           Die Woche fängt für mich gut an. Schon am \textsc{Montag}{ }morgen muß ich ein Vergehen beichten. Dieſer Brief hätte Euer
                    Hochwohlgeboren ſchon \textsc{Samstag} zugehen ſollen. Aber ſo ſind wir Menſchen. Im Unglück zerknirſcht und
                    demütig, wird doch {\pb}kaum daß es beſſer geht, der
                    alte Schlendrian eingeſchlagen und die kleine, kleinliche Tagesarbeit erſcheint
                    wichtiger, als Treue und Dankbarkeit zu bezeugen. Das iſt nur eine
                    Selbſtanklage. Die Familie \textcolor{blue}{Ehrenstein}{}\ledrightnote{\textcolor{blue}{Alexander Ehrenstein}{\newline}\textcolor{blue}{Charlotte Ehrenstein}}
                    trifft kein Verſchulden.\pend
           \pstart
           \textcolor{blue}{\textsc{Albert}}{}\ledrightnote{\textcolor{blue}{Albert Ehrenstein}} befindet ſich am Wege der Beſſerung und iſt mit Zuſtimmung des \textsc{Prima{\pb}rius D\textsuperscript{r}}{ }\textcolor{blue}{\textsc{Kornfeld}}{}\ledrightnote{\textcolor{blue}{Sigmund Kornfeld}}, der vorgeſtern dort war und heute wieder kommt in häuslicher Pflege
                    belaſſen worden. Der krankhafte Erregungszuſtand iſt im Abflauen. Seine
                    Handlungsweiſe vom \label{K_L01575_1v}\edtext{vorigen
                        Sonntag}{\lemma{\textnormal{\emph{vorigen
                        Sonntag}}}\Cendnote{\textnormal{vgl. A. S.: \emph{Tagebuch}, 14. 1. 1906}}}\label{K_L01575_1h} erkennt \textcolor{blue}{\textsc{Albert}}{}\ledrightnote{\textcolor{blue}{Albert Ehrenstein}}{ }ſchon als abnormal. Sein Gang iſt ſchon
                    natürlicher, drückt bei weitem nicht mehr die gehobene Stimmung eines Siegers
                    aus. Unnützes {\pb}Lachen kommt nicht vor, doch hat
                    er noch namentlich abends Angſtgefühle und findet auch noch – wenn auch ſeltener
                    – Beziehungen litterariſcher Größen zu ſich und seinem Verhalten.\pend
           \pstart
           \textsc{D\textsuperscript{r}}{ }\textcolor{blue}{\textsc{Kornfeld}}{}\ledrightnote{\textcolor{blue}{Sigmund Kornfeld}} ordnete unter anderem auch gelinde geiſtige Beſchäftigung an und \textcolor{blue}{\textsc{Albert}}{}\ledrightnote{\textcolor{blue}{Albert Ehrenstein}} hat geſtern im \textcolor{blue}{\textsc{Herder}}{}\ledrightnote{\textcolor{blue}{Johann Gottfried von Herder}} geleſen u darüber eine \textsc{Kritik} zu liefern gehabt.
                    Daß Gott erbarme wie \textcolor{blue}{Herder}{}\ledrightnote{\textcolor{blue}{Johann Gottfried von Herder}} wegkam. Er ſelbſt
                        be{\pb}zeichnete die Arbeit ironiſierend als
                    »Schularbeit« und klaſſifizierte ſie mit »nicht genügend«.\pend
           \pstart
           Mit vielem und herzlichen Dank für Ihre Teilnahme an das Geſchick des \textcolor{blue}{Kranken}{}\ledrightnote{→\textcolor{blue}{Albert Ehrenstein}} bitte ich um
                    Entſchuldigung, wenn ich ſo frei ſein werde dieſer Tage weiter zu berichten \pend
           \pstart
           In vollkommener Hochachtung{\\[\baselineskip]}ergebſt{\\[\baselineskip]}\spacefill\mbox{Ad. Treibl}\pend
           \leftskip=0em{}\endnumbering\briefempfaengerindex{Schnitzler, Arthur@\textsc{Schnitzler, Arthur}!zzzTreibl, Adolf@\emph{von Adolf Treibl}!1906-01-221@{{[}22.? 1. 1906{]}}|)be}\mylabel{h}  \normalsize

\doendnotes{C}
\bigskip
\vfill

\clearpage

\footnotesize

\lohead{\textsc{register}}

% Definiere theindex-Environment komplett neu ohne reledmac
\makeatletter
\renewenvironment{theindex}{%
  \section*{\indexname}%
  \setlength{\parindent}{0pt}%
  \setlength{\parskip}{0pt plus 0.3pt}%
  \let\item\@idxitem
}{%
  \clearpage
}
\makeatother

\IfFileExists{\jobname-pw.ind}{\input{\jobname-pw.ind}}{}

\end{document}

      