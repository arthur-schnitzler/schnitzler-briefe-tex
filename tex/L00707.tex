%% latex-korrekturansicht-vorspann.tex
%% Vorspann für die Korrekturansicht.
%% Lädt die gemeinsame Datei latex-vorspann.tex mit gesetztem Schalter.

\newif\ifkorrekturansicht
\korrekturansichttrue

\input{../tex-inputs/latex-vorspann}


               \section[Arthur Schnitzler an Richard Beer-Hofmann, 20. 7. 1897]{ Arthur Schnitzler an Richard Beer-Hofmann, 20. 7. 1897}\nopagebreak\mylabel{v}\rehead{ }\normalsize\beginnumbering\briefempfaengerindex{Beer-Hofmann, Richard@\textsc{Beer-Hofmann, Richard}!zzzSchnitzler, Arthur@\emph{von Arthur Schnitzler}!1897-07-201@{20. 7. 1897}|(be} \toendnotes[C]{\smallbreak\pagebreak[2]} \Standort{YCGL, MSS 31.}
\physDesc{Visitenkarte
\newline{}Handschrift: Bleistift, deutsche Kurrent}\buchAbdrucke{\weitereDrucke{Arthur Schnitzler, Richard Beer-Hofmann: \emph{Briefwechsel 1891–1931}. Hg. Konstanze Fliedl. Wien, Zürich: \emph{Europaverlag} 1992, S. 111.} }\toendnotes[C]{\smallbreak}\pstart{}{\pb}\label{T_L00707-1v}\edtext{Lieber Richard.}{\lemma{\textnormal{\emph{Lieber Richard.}}}\Cendnote{\textnormal{der gesamte Text ignoriert den Vordruck und ist quer zu dessen Ausrichtung verfasst}}}\label{T_L00707-1h}\pend\pstart
           1.) Ich fahr heut 4 Uhr{ }\textcolor{pink}{Hallſtadt}{}\ledrightnote{\textcolor{pink}{Hallstatt}}{ }\textcolor{blue}{\textsc{Loebs}}{}\ledrightnote{\textcolor{blue}{Louis Loeb}{\newline}\textcolor{blue}{Regina Loeb}} (die mit der Bahn).\pend
           \pstart
           2.) \textcolor{blue}{Hugo}{}\ledrightnote{\textcolor{blue}{Hugo von Hofmannsthal}} a) aergert ſich, dſs Sie ihm nicht
               ſchreiben\pend
           \pstart
           b) ka{\geminationn} nicht aus der \textcolor{pink}{\textsc{Fusch}}{}\ledrightnote{\textcolor{pink}{Fusch an der Großglocknerstraße}} fort.\pend
           \pstart
           (Was unſere Partie hoffent. nicht hindert)\pend
           \pstart
           3.) In \textcolor{pink}{Gmunden}{}\ledrightnote{\textcolor{pink}{Gmunden}}{ }ſoll 22. (übermorgen) \textcolor{green}{\uline{Freiwild}}{}\ledrightnote{\textcolor{green}{Freiwild. Schauspiel in 3 Akten}}{ }ſein (\label{K_L00707_1v}\edtext{\textcolor{green}{\textcolor{green}{Fremdenblatt}{}\ledrightnote{\textcolor{green}{Fremden-Blatt}}}{}\ledrightnote{→\textcolor{green}{Man schreibt uns aus Gmunden}}}{\lemma{\textnormal{\emph{Fremdenblatt}}}\Cendnote{\textnormal{»– Man schreibt uns aus \textcolor{pink}{\so{Gmunden}}: Das hiesige Saisontheater sieht einer interessanten Première entgegen.
                        \textcolor{blue}{Arthur \so{Schnitzler}}’s ›\textcolor{green}{\so{Freiwild}}« gelangt hier Donnerstag den 22. d., von Direktor \textcolor{blue}{\so{Cavar}} inszenirt, zum erstenmale (in \textcolor{pink}{Oesterreich}) zur Aufführung, mit jenen Einschränkungen natürlich,
                     welche die Zensur für nothwendig erachtet hat. In der Novität sind die besten
                     Kräfte beschäftigt, über welche das hiesige Theater verfügt, u. A. die Naive
                     Fräulein \textcolor{blue}{\so{Großmüller}}, welche für die nächste Saison an das \textcolor{brown}{Deutsche Volkstheater} engagirt ist, und Herr \textcolor{blue}{Alexander \so{Rottmann}}, der in einer Aufführung von \textcolor{blue}{Ohnet}’s
                        ›\textcolor{green}{Hüttenbesitzer}‹ durch die diskrete
                     Anwendung seiner schönen Mittel und die Natürlichkeit seiner Darstellung des
                     Philippe Derblay einen vollen Erfolg erzielt hat.« (\emph{\textcolor{green}{Fremden-Blatt}}, Jg. 51, Nr. 198,
                        19. 7. 1897, Abend-Blatt, S. 6)}}}\label{K_L00707_1h}) mit cenſurellen
               Aenderungen. Ich hab an \textcolor{blue}{\textsc{Cavar}}{}\ledrightnote{\textcolor{blue}{Alfred Cavar}} telegrafirt, mir {\pb}ſofort die Aenderg
               mitzutheilen. Geſindel, mi\textcolor{gray}{ch} nicht vorher zu verſtändg. (Kämen Sie
                     Do{\geminationn}erſtg mit mir hinüber?)\pend
           \pstart
           4.) Schaun Sie nach dem Nachtmahl zu mir herauf oder laſſen mir ſagen, wo Sie
               ſind.\pend
           \pstart
           Herzl Gruß{\\[\baselineskip]}Ihr \spacefill\mbox{A.}\pend
           \leftskip=0em{}\pstart
           \noindent{}\centering{}\textcolor{gray}{\textbf{D\textsuperscript{r} Arthur Schnitzler}}\pend
           \pstart
           \noindent{}\raggedleft{}\textcolor{pink}{Wien}{}\ledrightnote{\textcolor{pink}{Wien}}\pend
           \endnumbering\briefempfaengerindex{Beer-Hofmann, Richard@\textsc{Beer-Hofmann, Richard}!zzzSchnitzler, Arthur@\emph{von Arthur Schnitzler}!1897-07-201@{20. 7. 1897}|)be}\mylabel{h}  \normalsize

\doendnotes{C}
\bigskip
\vfill

\clearpage

\footnotesize

\lohead{\textsc{register}}

% Definiere theindex-Environment komplett neu ohne reledmac
\makeatletter
\renewenvironment{theindex}{%
  \section*{\indexname}%
  \setlength{\parindent}{0pt}%
  \setlength{\parskip}{0pt plus 0.3pt}%
  \let\item\@idxitem
}{%
  \clearpage
}
\makeatother

\IfFileExists{\jobname-pw.ind}{\input{\jobname-pw.ind}}{}

\end{document}

      