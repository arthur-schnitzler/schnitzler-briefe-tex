%% latex-korrekturansicht-vorspann.tex
%% Vorspann für die Korrekturansicht.
%% Lädt die gemeinsame Datei latex-vorspann.tex mit gesetztem Schalter.

\newif\ifkorrekturansicht
\korrekturansichttrue

\input{../tex-inputs/latex-vorspann}


               \section[Fedor Mamroth an Arthur Schnitzler, 4. 4. 1894]{ Fedor Mamroth an Arthur Schnitzler, 4. 4. 1894}\nopagebreak\mylabel{v}\rehead{ }\normalsize\beginnumbering\briefempfaengerindex{Schnitzler, Arthur@\textsc{Schnitzler, Arthur}!zzzMamroth, Fedor@\emph{von Fedor Mamroth}!1894-04-041@{4. 4. 1894}|(be} \toendnotes[C]{\smallbreak\pagebreak[2]} \Standort{CUL, Schnitzler, B 68.}
\physDesc{Brief, 1 Blatt, 1 Seite
\newline{}Handschrift einer Schreibkraft: schwarze Tinte, deutsche Kurrent
\newline{}Schnitzler: 1) mit Bleistift nummeriert: »6« und  2) mit rotem Buntstift
            beschriftet: »\textsc{Mam}« und zwei Unterstreichungen}\toendnotes[C]{\smallbreak}\pstart
           \noindent{}{\pb}\textcolor{brown}{\textcolor{gray}{\textbf{\textcolor{brown}{Frankfurter Zeitung}{}\ledrightnote{\textcolor{brown}{Frankfurter Zeitung}}}}}{}\ledrightnote{\textcolor{brown}{Frankfurter Zeitung}}\hfill \textcolor{gray}{\textbf{\textcolor{pink}{Frankfurt a. M.}{}\ledrightnote{\textcolor{pink}{Frankfurt am Main}},}}{ }4/4 \textcolor{gray}{\textbf{189}}4.\pend
           \pstart
           \textcolor{gray}{\textbf{und}}{\\}\textcolor{gray}{\textbf{Handelsblatt.}}\pend
           \pstart
           \textcolor{gray}{\textbf{Redaction.\footnote{\noindent{}\textcolor{gray}{\textbf{Für die Redaktion bestimmte Briefe und
                                        Sendungen wolle man \so{nicht} an die
                                        Perſon eines Redakteurs, ſondern ſtets \textbf{an
                                            die Redaktion der Frankfurter Zeitung}
                                        adreſſiren}}.}}}\pend
           \pstart
           \textcolor{gray}{\textbf{Telegramm-Adreſſe:}}\pend
           \pstart
           \textcolor{gray}{\textbf{\textcolor{brown}{Zeitung Frankfurt Main}{}\ledrightnote{\textcolor{brown}{Frankfurter Zeitung}}.}}\pend
           \pstart{}Hochgeehrter Herr Doktor.\pend\pstart
           Ich veröffentliche gegenwärtig einen großen \textcolor{green}{Roman}{}\ledrightnote{→\textcolor{green}{Kraft}}, dem ſich unmittelbar ein \label{K_L00311_1v}\edtext{anderer}{\lemma{\textnormal{\emph{anderer}}}\Cendnote{\textnormal{Das war dann nicht der Fall,
                    in Folge erschienen Novellen und Erzählungen verschiedener Autoren.}}}\label{K_L00311_1h} von \textcolor{blue}{\textsc{M. Prevost}}{}\ledrightnote{\textcolor{blue}{Marcel Prévost}} anreihen wird. Ich bin deshalb auf lange Zeit hinaus außer ſtande,
                    für kleine novelliſtiſche Arbeiten Raum zu finden u. muß Ihnen deßhalb Ihr ſehr
                    ſchönes \textcolor{green}{\textsc{Pastell}}{}\ledrightnote{→\textcolor{green}{Blumen}} zu meinem lebhaften Bedauern retournieren. Ich empfehle mich mit
                    herzlichem Gruß.\pend
           \pstart
           Hochachtungsvoll{\\[\baselineskip]}Ihr ergebener{\\[\baselineskip]}per{\\[\baselineskip]}\spacefill\mbox{D\textsuperscript{r.} F. Mamroth}\pend
           \leftskip=0em{}\endnumbering\briefempfaengerindex{Schnitzler, Arthur@\textsc{Schnitzler, Arthur}!zzzMamroth, Fedor@\emph{von Fedor Mamroth}!1894-04-041@{4. 4. 1894}|)be}\mylabel{h}  \normalsize

\doendnotes{C}
\bigskip
\vfill

\clearpage

\footnotesize

\lohead{\textsc{register}}

% Definiere theindex-Environment komplett neu ohne reledmac
\makeatletter
\renewenvironment{theindex}{%
  \section*{\indexname}%
  \setlength{\parindent}{0pt}%
  \setlength{\parskip}{0pt plus 0.3pt}%
  \let\item\@idxitem
}{%
  \clearpage
}
\makeatother

\IfFileExists{\jobname-pw.ind}{\input{\jobname-pw.ind}}{}

\end{document}

      