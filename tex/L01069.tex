%% latex-korrekturansicht-vorspann.tex
%% Vorspann für die Korrekturansicht.
%% Lädt die gemeinsame Datei latex-vorspann.tex mit gesetztem Schalter.

\newif\ifkorrekturansicht
\korrekturansichttrue

\input{../tex-inputs/latex-vorspann}


               \section[Stefan Großmann an Arthur Schnitzler, 30. 8. 1900]{ Stefan Großmann an Arthur Schnitzler, 30. 8. 1900}\nopagebreak\mylabel{v}\rehead{ }\normalsize\beginnumbering\briefempfaengerindex{Schnitzler, Arthur@\textsc{Schnitzler, Arthur}!zzzGrossmann, Stefan@\emph{von Stefan Großmann}!1900-08-301@{30. 8. 1900}|(be} \toendnotes[C]{\smallbreak\pagebreak[2]} \Standort{CUL, Schnitzler, B 34.}
\physDesc{Brief, 1 Blatt, 2 Seiten
\newline{}Handschrift: schwarze Tinte, deutsche Kurrent
\newline{}Schnitzler: 1) mit Bleistift beschriftet: »leihen« 2) mit rotem Buntstift eine Unterstreichung\newline{}Ordnung: mit Bleistift von unbekannter Hand
                           nummeriert: »2« }\pstart
           \raggedleft{}{\pb}\textcolor{pink}{Wien}{}\ledrightnote{\textcolor{pink}{Wien}}, den 30. Auguſt 1900\pend
           \pstart{}ſehr geehrter Herr Doctor,\pend\pstart
           Schon seit einiger Zeit möchte ich Sie, verehrter Herr, bitten, mir – wenn es Ihnen
               möglich iſt – ein \introOben{}etwa\introOben{} überflüſſiges Exemplar des »\textcolor{green}{\textsc{Reigen}}{}\ledrightnote{\textcolor{green}{Reigen. Zehn Dialoge}}« gütigſt leihen oder ſchenken zu wollen.\pend
           \pstart
           Ich fürchte, daſs es mir im Moment nicht möglich ſein wird Ihren Glauben an meinen
               einſeitigen aeſthetischen Doctrinarismus zu erſchüttern und beſchränke mich daher
               Ihnen zu ſagen, daſs ich Ihnen für die Zuſendung des Buches, {\pb}auf deſſen Lecture ich ſchon ſehr geſpannt
               bin, \uline{aufrichtig} und \uline{herzlichſt} danke.\pend
           \pstart
           Sehr ergeben:{\\[\baselineskip]}\spacefill\mbox{StefanGroßmann}\pend
           \leftskip=0em{}\pstart
           \noindent{}\raggedleft{}\textcolor{pink}{VIII. \textsc{Langega}ſſ\textsc{e} 52{\\}Th. 12}{}\ledrightnote{\textcolor{pink}{Lange Gasse}}\pend
           \endnumbering\briefempfaengerindex{Schnitzler, Arthur@\textsc{Schnitzler, Arthur}!zzzGrossmann, Stefan@\emph{von Stefan Großmann}!1900-08-301@{30. 8. 1900}|)be}\mylabel{h}  \normalsize

\doendnotes{C}
\bigskip
\vfill

\clearpage

\footnotesize

\lohead{\textsc{register}}

% Definiere theindex-Environment komplett neu ohne reledmac
\makeatletter
\renewenvironment{theindex}{%
  \section*{\indexname}%
  \setlength{\parindent}{0pt}%
  \setlength{\parskip}{0pt plus 0.3pt}%
  \let\item\@idxitem
}{%
  \clearpage
}
\makeatother

\IfFileExists{\jobname-pw.ind}{\input{\jobname-pw.ind}}{}

\end{document}

      