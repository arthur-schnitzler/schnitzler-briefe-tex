%% latex-korrekturansicht-vorspann.tex
%% Vorspann für die Korrekturansicht.
%% Lädt die gemeinsame Datei latex-vorspann.tex mit gesetztem Schalter.

\newif\ifkorrekturansicht
\korrekturansichttrue

\input{../tex-inputs/latex-vorspann}


               \section[Hugo von Hofmannsthal an Arthur Schnitzler, {[}19. 11. 1899?{]}]{ Hugo von Hofmannsthal an Arthur Schnitzler, {[}19. 11. 1899?{]}}\nopagebreak\mylabel{v}\rehead{ }\normalsize\beginnumbering\briefempfaengerindex{Schnitzler, Arthur@\textsc{Schnitzler, Arthur}!zzzHofmannsthal, Hugo von@\emph{von Hugo von Hofmannsthal}!1899-11-191@{{[}19. 11. 1899?{]}}|(be} \toendnotes[C]{\smallbreak\pagebreak[2]} \Standort{CUL, Schnitzler, B 43.}
\physDesc{Brief, 1 Blatt, 1 Seite
\newline{}Handschrift: Bleistift, deutsche Kurrent
\newline{}Schnitzler: mit Bleistift datiert: »\substVorne{}\textsuperscript{De}\substDazwischen{}Nov\substHinten{} 99.« \newline{}Ordnung: 1) mit Bleistift von unbekannter Hand nummeriert:
                                    »158« 2) mit Bleistift von unbekannter Hand nummeriert:
                                    »161«}\buchAbdrucke{\weitereDrucke{Hugo von Hofmannsthal, Arthur Schnitzler: \emph{Briefwechsel}. Hg. Therese Nickl und Heinrich Schnitzler. Frankfurt am Main: \emph{S. Fischer} 1964, S. 134.} }\toendnotes[C]{\smallbreak}\pstart{}{\pb}lieber\pend\pstart
           leider \label{K_L00998_1v}\edtext{treffe ich Sie \uline{nie}}{\lemma{\textnormal{\emph{treffe ich Sie nie}}}\Cendnote{\textnormal{Dieser Brief ist nur tentativ zu
                  datieren. Im von \textcolor{blue}{Schnitzler} angegebenen Monat
                  findet das erste Treffen zwischen den beiden am 26. 11. 1899 statt. Am 5. 11. 1899 ist \textcolor{blue}{Schnitzler} bei \textcolor{blue}{Beer-Hofmann} und
                  ärgert sich über \textcolor{blue}{Hofmannsthal}, was
                  möglicherweise auf ein Zusammentreffen verweist. Offenbar war zu dieser Zeit ein
                  regelmäßiges Treffen am Sonntag geplant, das \textcolor{blue}{Schnitzler} aber erst am Monatsende einhalten konnte. Verzichtet man
                  darauf, das »nie« als Übertreibung zu betrachten und einen gewissen Abstand
                  zwischen den Treffen anzunehmen, bliebe der Sonntag, 19. 11. 1899 als
                  mögliches Datum.}}}\label{K_L00998_1h}.\pend
           \pstart
           Heute abend ko{\geminationm}e ich \uline{eventuell} zu \textcolor{blue}{Richard}{}\ledrightnote{\textcolor{blue}{Richard Beer-Hofmann}} doch nicht vor
                  ½ 9.\pend
           \pstart
           Herzlich{\\[\baselineskip]}\spacefill\mbox{Hugo}\pend
           \leftskip=0em{}\endnumbering\briefempfaengerindex{Schnitzler, Arthur@\textsc{Schnitzler, Arthur}!zzzHofmannsthal, Hugo von@\emph{von Hugo von Hofmannsthal}!1899-11-191@{{[}19. 11. 1899?{]}}|)be}\mylabel{h}  \normalsize

\doendnotes{C}
\bigskip
\vfill

\clearpage

\footnotesize

\lohead{\textsc{register}}

% Definiere theindex-Environment komplett neu ohne reledmac
\makeatletter
\renewenvironment{theindex}{%
  \section*{\indexname}%
  \setlength{\parindent}{0pt}%
  \setlength{\parskip}{0pt plus 0.3pt}%
  \let\item\@idxitem
}{%
  \clearpage
}
\makeatother

\IfFileExists{\jobname-pw.ind}{\input{\jobname-pw.ind}}{}

\end{document}

      