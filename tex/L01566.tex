%% latex-korrekturansicht-vorspann.tex
%% Vorspann für die Korrekturansicht.
%% Lädt die gemeinsame Datei latex-vorspann.tex mit gesetztem Schalter.

\newif\ifkorrekturansicht
\korrekturansichttrue

\input{../tex-inputs/latex-vorspann}


               \section[Max Burckhard: Widmungsexemplar Franz Stelzhamer und die oberösterreichische Dialektdichtung für Arthur Schnitzler, {[}27. 10. 1905?{]}]{ Max Burckhard: Widmungsexemplar Franz Stelzhamer und die
                    oberösterreichische Dialektdichtung für Arthur Schnitzler, {[}27. 10. 1905?{]}}\nopagebreak\mylabel{v}\rehead{ }\normalsize\beginnumbering\briefempfaengerindex{Schnitzler, Arthur@\textsc{Schnitzler, Arthur}!zzzBurckhard, Max Eugen@\emph{von Max Eugen Burckhard}!1905-10-272@{{[}27. 10. 1905?{]}}|(be} \toendnotes[C]{\smallbreak\pagebreak[2]} \Standort{DLA, G:Schnitzler, Arthur (Sammlung Heinrich Schnitzler).}
\physDesc{Widmung am Vorsatzblatt
\newline{}Handschrift: schwarze Tinte, deutsche Kurrent}\toendnotes[C]{\smallbreak}\pstart
           \noindent{}{\pb}S. l. Arthur Schnitzler\pend
           \pstart herzlichſt\spacefill\mbox{MaxBurckhard}\pend{}{\bigskip}\pstart
           \noindent{}\centering{}{\pb}\textcolor{gray}{\textbf{\textcolor{green}{Franz Stelzhamer und die oberöſterreichiſche
                            Dialektdichtung}{}\ledrightnote{\textcolor{green}{Franz Stelzhamer und die oberösterreichische Dialektdichtung}}.}}\pend
           \pstart
           \noindent{}\centering{}\textcolor{gray}{\textbf{\textsuperscript{von} Max Burckhart.}}\pend
           \pstart
           \noindent{}\centering{}\textcolor{gray}{\textbf{Zeichnungen von \textcolor{blue}{Leop.
                            Forſtner}{}\ledrightnote{\textcolor{blue}{Leopold Forstner}}.}}\pend
           {\bigskip}\pstart
           \noindent{}\centering{}\textcolor{gray}{\textbf{\label{K_L01566_1v}\edtext{\textcolor{brown}{Wiener Verlag}{}\ledrightnote{\textcolor{brown}{Wiener Verlag}}}{\lemma{\textnormal{\emph{Wiener Verlag}}}\Cendnote{\textnormal{am 13. 11. 1905 vom \emph{\textcolor{green}{Börsenblatt für den deutschen
                     Buchhandel}} als Neuerscheinung gemeldet. Da
                            zugleich auch das Buch \emph{\textcolor{green}{Charakterbilder aus
                                Oberoesterreich}} erschien, ist es naheliegend, dass \textcolor{blue}{Burckhard}{ }\textcolor{blue}{Schnitzler} beide Bücher zugleich
                            zukommen ließ.}}}\label{K_L01566_1h}{ }\textcolor{pink}{Wien}{}\ledrightnote{\textcolor{pink}{Wien}}{ }u.{ }\textcolor{pink}{Leipzig}{}\ledrightnote{\textcolor{pink}{Leipzig}}.}}\pend
           \endnumbering\briefempfaengerindex{Schnitzler, Arthur@\textsc{Schnitzler, Arthur}!zzzBurckhard, Max Eugen@\emph{von Max Eugen Burckhard}!1905-10-272@{{[}27. 10. 1905?{]}}|)be}\mylabel{h}  \normalsize

\doendnotes{C}
\bigskip
\vfill

\clearpage

\footnotesize

\lohead{\textsc{register}}

% Definiere theindex-Environment komplett neu ohne reledmac
\makeatletter
\renewenvironment{theindex}{%
  \section*{\indexname}%
  \setlength{\parindent}{0pt}%
  \setlength{\parskip}{0pt plus 0.3pt}%
  \let\item\@idxitem
}{%
  \clearpage
}
\makeatother

\IfFileExists{\jobname-pw.ind}{\input{\jobname-pw.ind}}{}

\end{document}

      