%% latex-korrekturansicht-vorspann.tex
%% Vorspann für die Korrekturansicht.
%% Lädt die gemeinsame Datei latex-vorspann.tex mit gesetztem Schalter.

\newif\ifkorrekturansicht
\korrekturansichttrue

\input{../tex-inputs/latex-vorspann}


               \section[Jakob Julius David an Arthur Schnitzler, {[}8. 3. 1899?{]}]{ Jakob Julius David an Arthur Schnitzler, {[}8. 3. 1899?{]}}\nopagebreak\mylabel{v}\rehead{ }\normalsize\beginnumbering\briefempfaengerindex{Schnitzler, Arthur@\textsc{Schnitzler, Arthur}!zzzDavid, Jakob Julius@\emph{von Jakob Julius David}!1899-03-082@{{[}8. 3. 1899?{]}}|(be} \toendnotes[C]{\smallbreak\pagebreak[2]} \Standort{TMW, HS Schn 1/93/1.}
\physDesc{Visitenkarte
\newline{}Handschrift: schwarze Tinte, lateinische Kurrent\newline{}Ordnung: mit Bleistift von unbekannter Hand nummeriert:
                                    »6a« }\toendnotes[C]{\smallbreak}\pstart
           \noindent{}\centering{}{\pb}\textcolor{gray}{\textbf{D\textsuperscript{r} J. J. David}}\pend
           \pstart\center{}{\pb}Verehrter Herr!\pend\pstart
           schön Dank. Die Logen \label{K_L00902_1v}\edtext{opponirten auch
                  gestern}{\lemma{\textnormal{\emph{opponirten auch
                  gestern}}}\Cendnote{\textnormal{Die Karte ist undatiert.
                  Sofern sie einen Anschluss an eine erhaltene Kommunikation darstellt, bietet sich
                  der 8. 3. 1899 an. Am Vortag dürfte \textcolor{blue}{David} die ihm von \textcolor{blue}{Schnitzler}
                  verschafften Freikarten für einen neuerlichen Besuch der drei Einakter \emph{\textcolor{green}{Der grüne Kakadu – Paracelsus – Die Gefährtin}}
                  benutzt haben. Bereits in seiner Rezension der Uraufführung – \emph{\textcolor{green}{Aus ungleichen Tagen}} – hatte er von der geteilten Aufnahme
                  durch das Publikum berichtet.}}}\label{K_L00902_1h}. So beßer, wenn sie sich daran ärgern.\pend
           \pstart
           Es wird mich immer freuen, wenn sich Gelegenheit zu einer Aussprache gäbe.\pend
           \pstart
           Bestens Ihr{\\[\baselineskip]}\spacefill\mbox{David}\pend
           \leftskip=0em{}\endnumbering\briefempfaengerindex{Schnitzler, Arthur@\textsc{Schnitzler, Arthur}!zzzDavid, Jakob Julius@\emph{von Jakob Julius David}!1899-03-082@{{[}8. 3. 1899?{]}}|)be}\mylabel{h}  \normalsize

\doendnotes{C}
\bigskip
\vfill

\clearpage

\footnotesize

\lohead{\textsc{register}}

% Definiere theindex-Environment komplett neu ohne reledmac
\makeatletter
\renewenvironment{theindex}{%
  \section*{\indexname}%
  \setlength{\parindent}{0pt}%
  \setlength{\parskip}{0pt plus 0.3pt}%
  \let\item\@idxitem
}{%
  \clearpage
}
\makeatother

\IfFileExists{\jobname-pw.ind}{\input{\jobname-pw.ind}}{}

\end{document}

      