%% latex-korrekturansicht-vorspann.tex
%% Vorspann für die Korrekturansicht.
%% Lädt die gemeinsame Datei latex-vorspann.tex mit gesetztem Schalter.

\newif\ifkorrekturansicht
\korrekturansichttrue

\input{../tex-inputs/latex-vorspann}


               \section[Hugo von Hofmannsthal an Arthur Schnitzler, 1. 7. {[}1903{]}]{ Hugo von Hofmannsthal an Arthur Schnitzler, 1. 7. {[}1903{]}}\nopagebreak\mylabel{v}\rehead{ }\normalsize\beginnumbering\briefempfaengerindex{Schnitzler, Arthur@\textsc{Schnitzler, Arthur}!zzzHofmannsthal, Hugo von@\emph{von Hugo von Hofmannsthal}!1903-07-011@{1. 7. {[}1903{]}}|(be} \toendnotes[C]{\smallbreak\pagebreak[2]} \Standort{CUL, Schnitzler, B 43.}
\physDesc{Brief, 1 Blatt, 4 Seiten
\newline{}Handschrift: schwarze Tinte, deutsche Kurrent\newline{}Ordnung: 1) mit Bleistift von unbekannter Hand nummeriert: »\strikeout{263}« 2) mit Bleistift von unbekannter Hand nummeriert: »262«}\buchAbdrucke{\weitereDrucke{Hugo von Hofmannsthal, Arthur Schnitzler: \emph{Briefwechsel}. Hg. Therese Nickl und Heinrich Schnitzler. Frankfurt am Main: \emph{S. Fischer} 1964, S. 172–173.} }\toendnotes[C]{\smallbreak}\pstart
           \raggedleft{}{\pb}1\textsuperscript{ten} July{\\}\textcolor{pink}{Gaſthof Poſt, am Brenner}{}\ledrightnote{\textcolor{pink}{Gasthof Post}}.\pend
           \pstart
           lieber, hier, wo wir vor einem Jahr \label{K_L01301_1v}\edtext{zuſammen geſeſſen}{\lemma{\textnormal{\emph{zuſammen geſeſſen}}}\Cendnote{\textnormal{vgl. A. S.: \emph{Tagebuch}, 3. 7. 1902}}}\label{K_L01301_1h} ſind – es iſt
               ein Jahr faſt auf den Tag genau – finde ich Ihren lieben Brief. Erinnern Sie ſich? es
               war an dem ſchönen Tag, wo wir im \textcolor{pink}{\textsc{Stubai}thal}{}\ledrightnote{\textcolor{pink}{Stubaital}} waren und ich Ihnen Complimente gemacht
               habe, wir dann in \textcolor{pink}{\textsc{Windischmatrei}}{}\ledrightnote{\textcolor{pink}{Matrei in Osttirol}} Forellen gegeſſen haben und die \textcolor{blue}{\textsc{Lisl}}{}\ledrightnote{\textcolor{blue}{Elisabeth Steinrück}} aus \textcolor{pink}{Berlin}{}\ledrightnote{\textcolor{pink}{Berlin}}{ }{\pb}geſchrieben hat, daß der \textcolor{blue}{Goldmann}{}\ledrightnote{\textcolor{blue}{Paul Goldmann}} ihr kein Geld leiht.\pend
           \pstart
           Wir haben ein paar ſehr ſchöne Tage in \textcolor{pink}{Italien}{}\ledrightnote{\textcolor{pink}{Italien}}
               verbracht, das \textcolor{pink}{Ampezzo-thal}{}\ledrightnote{\textcolor{pink}{Valle d’Ampezzo}} hinunter bis \textcolor{pink}{\textsc{Vicenza}}{}\ledrightnote{\textcolor{pink}{Vicenza}} und durchs \textcolor{pink}{\textsc{Val sugana}}{}\ledrightnote{\textcolor{pink}{Val Sugana}} zurück.\hspace*{1.5em}So ſchön iſt dieſes Land!\pend
           \pstart
           Trotzdem werde ich nicht mit Ihnen um den 10\textsuperscript{ten}
                  Auguſt in dieſe Gegenden fahren. Ich werde um den 10\textsuperscript{ten} Auguſt in \textcolor{pink}{Weimar}{}\ledrightnote{\textcolor{pink}{Weimar}}{ }ſein. Die Einladung dazu geht direct von der \textcolor{blue}{Erbgroßherzogin}{}\ledrightnote{→\textcolor{blue}{Pauline Sachsen-Weimar}} aus, indirect
                  \strikeout{zu} von \textcolor{blue}{Keſſler}{}\ledrightnote{\textcolor{blue}{Harry von Kessler}}, der an dieſem {\pb}kleinen Hof ſeit einiger Zeit eine nicht recht definierbare Art von
               Intendantenſtellung einnimmt. Sie wollen meinem Hinkommen zu Ehren dort auf dem
               kleinen Naturtheater in \textcolor{pink}{Belvedere}{}\ledrightnote{\textcolor{pink}{Belvedere}} – auf welchem \textcolor{blue}{Goethe}{}\ledrightnote{\textcolor{blue}{Johann Wolfgang von Goethe}} den \textcolor{green}{Oreſt}{}\ledrightnote{→\textcolor{green}{Iphigenie auf Tauris}}{ }ſpielte – den \textcolor{green}{Tod des
                  Tizian}{}\ledrightnote{\textcolor{green}{Der Tod des Tizian}} von den hübſcheſten Hofdamen und Pagen – wirklichen Pagen – ſpielen
               laſſen. Es macht mir natürlich Spaß, auch kenne ich \textcolor{pink}{Weimar}{}\ledrightnote{\textcolor{pink}{Weimar}} gar nicht. –\pend
           \pstart
           Das nähere darüber und über ſonſtige Pläne mündlich.\pend
           \pstart
           Wir gehen {\pb}noch für 10–12 Tage an
               den \textcolor{pink}{Grundlſee}{}\ledrightnote{\textcolor{pink}{Grundlsee}}.\pend
           \settowidth{\longeste}{Adreſſe H. H. bei}\settowidth{\longestz}{Frau Lili Geyger}\settowidth{\longestd}{}\settowidth{\longestv}{}\settowidth{\longestf}{}\addtolength\longeste{1em}
        \addtolength\longestz{1em}
      \pstart\noindent\makebox[\the\longeste][l]{Adreſſe \textsc{H. H.} bei}\makebox[\the\longestz][l]{\textsc{\uline{Frau \textcolor{blue}{Lili Geyger}{}\ledrightnote{\textcolor{blue}{Lili Schalk}}}}}
                  \pend\pstart\noindent\makebox[\the\longeste][l]{}\makebox[\the\longestz][l]{\textcolor{pink}{\textsc{Grundlsee}}{}\ledrightnote{\textcolor{pink}{Grundlsee}}}
                  \pend\pstart\noindent\makebox[\the\longeste][l]{}\makebox[\the\longestz][l]{\textcolor{pink}{\textsc{Archkogel 13}}{}\ledrightnote{\textcolor{pink}{Archkogel}}}
                  \pend\pstart
           Von Herzen{\\[\baselineskip]}\spacefill\mbox{Hugo.}\pend
           \leftskip=0em{}\pstart
           \noindent{}Grüße für \textcolor{blue}{Olga}{}\ledrightnote{\textcolor{blue}{Olga Schnitzler}} und \textcolor{blue}{Heinrich}{}\ledrightnote{\textcolor{blue}{Heinrich Schnitzler}} das Kind. Es war abſolut unleſerlich, welches \textcolor{green}{(franzöſiſche??) Buch}{}\ledrightnote{→\textcolor{green}{Die drei Musketiere}} Sie auf
                  der Reiſe ſehr genoſſen haben.\pend
           \endnumbering\briefempfaengerindex{Schnitzler, Arthur@\textsc{Schnitzler, Arthur}!zzzHofmannsthal, Hugo von@\emph{von Hugo von Hofmannsthal}!1903-07-011@{1. 7. {[}1903{]}}|)be}\mylabel{h}  \normalsize

\doendnotes{C}
\bigskip
\vfill

\clearpage

\footnotesize

\lohead{\textsc{register}}

% Definiere theindex-Environment komplett neu ohne reledmac
\makeatletter
\renewenvironment{theindex}{%
  \section*{\indexname}%
  \setlength{\parindent}{0pt}%
  \setlength{\parskip}{0pt plus 0.3pt}%
  \let\item\@idxitem
}{%
  \clearpage
}
\makeatother

\IfFileExists{\jobname-pw.ind}{\input{\jobname-pw.ind}}{}

\end{document}

      