\documentclass[twoside=false,titlepage=false,open=any, parskip=never, fontsize=12pt, headings=small, chapterprefix=false, appendixprefix=false]{scrbook}
\addtolength{\oddsidemargin}{\evensidemargin}
\setlength{\oddsidemargin}{.5\oddsidemargin}
\setlength{\evensidemargin}{\oddsidemargin}

\usepackage[{textwidth=13cm,textheight=23cm,marginpar=3cm, left=2cm}]{geometry}
%\usepackage[textwidth=80mm, layoutwidth=170mm, paperheight =297mm, paperwidth  =210mm, layoutvoffset= 20mm,layouthoffset= 20mm]{geometry}
%\usepackage[paperheight =297mm, paperwidth  =210mm, layoutheight=230mm, layoutwidth=158mm, layoutvoffset= 20mm, layouthoffset= 20mm, textwidth=150mm, textheight=185mm, showcrop=false]{geometry}
%sepackage[paperheight=230mm, paperwidth=138mm, textwidth=100mm, textheight=185mm]{geometry}
 \usepackage[usenames, dvipsnames]{xcolor}
\usepackage{scrlayer-scrpage}
\usepackage{hyphenat}
\usepackage{fontspec}
\usepackage{moresize}
\usepackage[english, french, greek, ngerman]{babel}
%\usepackage{ipa}  für das Seitenwechselzeichens
\usepackage[babel]{microtype}
\usepackage[dash, dot]{dashundergaps}
\usepackage{soul}
\usepackage{ragged2e}
\usepackage[makeindex, protected]{splitidx}
\usepackage[itemlayout=abshang,hangindent=0.85em, subindent=0em, subsubindent=1em, justific=RaggedRight, columns=1, columnsep=0pt, indentunit=1em, totoc=false]{idxlayout}
\usepackage{scrhack}
\usepackage{xpatch}
\usepackage{reledmac}
\usepackage{refcount} % Für die Seitenverweise 1–3 etc. 
\usepackage{etoolbox}
\usepackage{framed}
\usepackage[export]{adjustbox} % loads also graphicx, für Bildgröße autom. maximal
\usepackage{float} %ermöglicht exakte Bildpositionierung
\usepackage{mdframed}
\usepackage{enumitem}
\usepackage{relsize}
\usepackage{longtable}
\usepackage{chngcntr} % Sectionnummern durchgehend
\usepackage{hanging} % Für hängende Absätze
\usepackage[rightmargin=0em, leftmargin=1em, indentfirst=false]{quoting} % Für die geänderte quote-Umgebung in den Hrsg-Texten
%\usepackage{fontawesome}
\usepackage{ellipsis}
\RequirePackage{hyphsubst}%
\HyphSubstIfExists{ngerman-x-latest}{\HyphSubstLet{ngerman}{ngerman-x-latest}}{} 
\listfiles
\usepackage[noadjust]{marginnote}

\KOMAoptions{toc=chapterentrydotfill, toc=flat}
\addtokomafont{chapterentrypagenumber}{\mdseries}
\setkomafont{chapterentry}{\normalfont\mdseries}
\setkomafont{partentry}{\normalfont\mdseries}
\RedeclareSectionCommand[tocbeforeskip=0pt]{chapter}

\setlength{\skip\footins}{4mm plus 2mm} % Abstand Fussnote Text
\interfootnotelinepenalty=10000 % Kein Seitenwechsel in Fuss

%\DeclareTextFontCommand{\emph}{\textit}

% Der Befehl erlaubt rechtsbündig bei Unterschriften, die nicht mehr in die Zeile passen
\def\spacefill{\hspace{\fill}\mbox{}\linebreak[0]\hspace*{\fill}}
\usepackage{atbegshi}
\usepackage{zref-abspage}
\usepackage{perpage}
\usepackage{zref-user}
\usepackage{tikz}
\usepackage{ulem}
\usetikzlibrary{calc,decorations.pathmorphing}
\setmainfont[Path=../fonts/,
  Extension=.otf,
  UprightFont=*-Regular,
  ItalicFont=*-Italic]{EBGaramond12}


\PassOptionsToPackage{gray}{xcolor}
\definecolor{gray}{gray}{0.6}

\doublehyphendemerits=1000000 % das hier verhindert zu viele aufeinanderfolgende Trennstriche am Zeilenende


\usepackage{zref-abspage}
\usepackage{zref-user}
\usepackage{tikz}
\usepackage{atbegshi}
\usepackage{ulem}
\usetikzlibrary{calc,decorations.pathmorphing}

\PassOptionsToPackage{gray}{xcolor}
\definecolor{gray}{gray}{0.6}

\doublehyphendemerits=1000000 % das hier verhindert zu viele aufeinanderfolgende Trennstriche am Zeilenende

\makeatletter
\newcommand{\currentsidemargin}{%
  \ifodd\zref@extract{textarea-\thetextarea}{abspage}%
    \oddsidemargin%
  \else%
    \evensidemargin%
  \fi%
}

\newcounter{textarea}
\newcommand{\settextarea}{%
   \stepcounter{textarea}%
   \zlabel{textarea-\thetextarea}%
   \begin{tikzpicture}[overlay,remember picture]
    % Helper nodes
    \path (current page.north west) ++(\hoffset, -\voffset)
        node[anchor=north west, shape=rectangle, inner sep=0, minimum width=\paperwidth, minimum height=\paperheight]
        (pagearea) {};
    \path (pagearea.north west) ++(1in+\currentsidemargin,-1in-\topmargin-\headheight-\headsep)
        node[anchor=north west, shape=rectangle, inner sep=0, minimum width=\textwidth, minimum height=7pt]
        (textarea) {};
  \end{tikzpicture}%
}

\tikzset{tikzul/.style={yshift=-.75\dp\strutbox}}

\newcounter{tikzul}%
\newcommand\tikzul[1][]{%
    \begingroup
    \global\tikzullinewidth\linewidth
    \def\tikzulsetting{[#1]}%
    \stepcounter{tikzul}%
    \settextarea
    \zlabel{tikzul-begin-\thetikzul}%
    \tikz[overlay,remember picture,tikzul] \coordinate (tikzul-\thetikzul) at (0,0);% Modified \tikzmark macro
    \ifnum\zref@extract{tikzul-begin-\thetikzul}{abspage}=\zref@extract{tikzul-end-\thetikzul}{abspage}
    \else
        \AtBeginShipoutNext{\tikzul@endpage{#1}}%
    \fi
    \bgroup
    \def\par{\ifhmode\unskip\fi\egroup\par\@ifnextchar\noindent{\noindent\tikzul[#1]}{\tikzul[#1]\bgroup}}%
    \aftergroup\endtikzul
    \let\@let@token=%
}
\newlength\tikzullinewidth


\def\tikzul@endpage#1{%
\setbox\AtBeginShipoutBox\hbox{%
\box\AtBeginShipoutBox
\hbox{%
\begin{tikzpicture}[overlay,remember picture,tikzul]
\draw[#1]
    let \p1 = (tikzul-\thetikzul), \p2 = ([xshift=\tikzullinewidth+\@totalleftmargin]textarea.south west) in
    \ifdim\dimexpr\y1-\y2<.5\baselineskip
        (\x1,\y1) -- (\x2,\y1)
    \else
        let \p3 = ([xshift=\@totalleftmargin]textarea.west) in
        (\x1,\y1) -- +(\tikzullinewidth-\x1+\x3,0)
        % (\x3,\y2) -- (\x2,\y2)
        (\x3,\y1)
       \myloop{\y1-\y2+.5\baselineskip}{%
           ++(0,-\baselineskip) -- +(\tikzullinewidth,0)
       }%
    \fi
;
\end{tikzpicture}%
}}%
}%


\def\endtikzul{%
    \zlabel{tikzul-end-\thetikzul}%
    \ifnum\zref@extract{tikzul-begin-\thetikzul}{abspage}=\zref@extract{tikzul-end-\thetikzul}{abspage}
    \begin{tikzpicture}[overlay,remember picture,tikzul]
        \expandafter\draw\tikzulsetting
            let \p1 = (tikzul-\thetikzul), \p2 = (0,0) in
            \ifdim\y1=\y2
                (\x1,\y1) -- (\x2,\y2)
            \else
                let \p3 = ([xshift=\@totalleftmargin]textarea.west), \p4 = ([xshift=-\rightmargin]textarea.east) in
                (\x1,\y1) -- +(\tikzullinewidth-\x1+\x3,0)
                (\x3,\y2) -- (\x2,\y2)
                (\x3,\y1)
                \myloop{\y1-\y2}{%
                    ++(0,-\baselineskip) -- +(\tikzullinewidth,0)
                }%
            \fi
        ;
    \end{tikzpicture}%
    \else
    \settextarea
    \begin{tikzpicture}[overlay,remember picture,tikzul]
        \expandafter\draw\tikzulsetting
            let \p1 = ([xshift=\@totalleftmargin,yshift=-.5\baselineskip]textarea.north west), \p2 = (0,0) in
            \ifdim\dimexpr\y1-\y2<.5\baselineskip
                (\x1,\y2) -- (\x2,\y2)
            \else
                let \p3 = ([xshift=\@totalleftmargin]textarea.west), \p4 = ([xshift=-\rightmargin]textarea.east) in
                (\x3,\y2) -- (\x2,\y2)
                (\x3,\y2)
                \myloop{\y1-\y2}{%
                    ++(0,+\baselineskip) -- +(\tikzullinewidth,0)
                }
            \fi
        ;
    \end{tikzpicture}%
    \fi
    \endgroup
}

% -------------------------------------------------------------- Additions by Peter Grill

\tikzset{tikzst/.style={yshift=0.5\dp\strutbox}}

\newcounter{tikzst}%
\newcommand\tikzst[1][]{%
    \begingroup
    \global\tikzstlinewidth\linewidth
    \def\tikzstsetting{[#1]}%
    \stepcounter{tikzst}%
    \settextarea
    \zlabel{tikzst-begin-\thetikzst}%
    \tikz[overlay,remember picture,tikzst] \coordinate (tikzst-\thetikzst) at (0,0);% Modified \tikzmark macro
    \ifnum\zref@extract{tikzst-begin-\thetikzst}{abspage}=\zref@extract{tikzst-end-\thetikzst}{abspage}
    \else
        \AtBeginShipoutNext{\tikzst@endpage{#1}}%
    \fi
    \bgroup
    \def\par{\ifhmode\unskip\fi\egroup\par\@ifnextchar\noindent{\noindent\tikzst[#1]}{\tikzst[#1]\bgroup}}%
    \aftergroup\endtikzst
    \let\@let@token=%
}
\newlength\tikzstlinewidth


\def\tikzst@endpage#1{%
\setbox\AtBeginShipoutBox\hbox{%
\box\AtBeginShipoutBox
\hbox{%
\begin{tikzpicture}[overlay,remember picture,tikzst]
\draw[#1]
    let \p1 = (tikzst-\thetikzst), \p2 = ([xshift=\tikzstlinewidth+\@totalleftmargin]textarea.south west) in
    \ifdim\dimexpr\y1-\y2<.5\baselineskip
        (\x1,\y1) -- (\x2,\y1)
    \else
        let \p3 = ([xshift=\@totalleftmargin]textarea.west) in
        (\x1,\y1) -- +(\tikzstlinewidth-\x1+\x3,0)
        % (\x3,\y2) -- (\x2,\y2)
        (\x3,\y1)
       \myloop{\y1-\y2+.5\baselineskip}{%
           ++(0,-\baselineskip) -- +(\tikzstlinewidth,0)
       }%
    \fi
;
\end{tikzpicture}%
}}%
}%


\def\endtikzst{%
    \zlabel{tikzst-end-\thetikzst}%
    \ifnum\zref@extract{tikzst-begin-\thetikzst}{abspage}=\zref@extract{tikzst-end-\thetikzst}{abspage}
    \begin{tikzpicture}[overlay,remember picture,tikzst]
        \expandafter\draw\tikzstsetting
            let \p1 = (tikzst-\thetikzst), \p2 = (0,0) in
            \ifdim\y1=\y2
                (\x1,\y1) -- (\x2,\y2)
            \else
                let \p3 = ([xshift=\@totalleftmargin]textarea.west), \p4 = ([xshift=-\rightmargin]textarea.east) in
                (\x1,\y1) -- +(\tikzstlinewidth-\x1+\x3,0)
                (\x3,\y2) -- (\x2,\y2)
                (\x3,\y1)
                \myloop{\y1-\y2}{%
                    ++(0,-\baselineskip) -- +(\tikzstlinewidth,0)
                }%
            \fi
        ;
    \end{tikzpicture}%
    \else
    \settextarea
    \begin{tikzpicture}[overlay,remember picture,tikzst]
        \expandafter\draw\tikzstsetting
            let \p1 = ([xshift=\@totalleftmargin,yshift=-.5\baselineskip]textarea.north west), \p2 = (0,0) in
            \ifdim\dimexpr\y1-\y2<.5\baselineskip
                (\x1,\y2) -- (\x2,\y2)
            \else
                let \p3 = ([xshift=\@totalleftmargin]textarea.west), \p4 = ([xshift=-\rightmargin]textarea.east) in
                (\x3,\y2) -- (\x2,\y2)
                (\x3,\y2)
                \myloop{\y1-\y2}{%
                    ++(0,+\baselineskip) -- +(\tikzstlinewidth,0)
                }
            \fi
        ;
    \end{tikzpicture}%
    \fi
    \endgroup
}
% --------------------------------------------------------------

\def\myloop#1#2#3{%
    #3%
    \ifdim\dimexpr#1>1.1\baselineskip
        #2%
        \expandafter\myloop\expandafter{\the\dimexpr#1-\baselineskip\relax}{#2}%
    \fi
}

\makeatother






\def\myloop#1#2#3{%
    #3%
    \ifdim\dimexpr#1>1.1\baselineskip
        #2%
        \expandafter\myloop\expandafter{\the\dimexpr#1-\baselineskip\relax}{#2}%
    \fi
}

\makeatother
%\newcommand{\damage}[1]{\tikzul[gray,line width=0.15\ht\strutbox,semitransparent]{#1}}
%\newcommand{\strikeout}[1]{\tikzst[black]{#1}}

\newcommand{\damage}[1]{\textcolor{orange}{#1}}
\newcommand{\strikeout}[1]{\sout{#1}}


\setlength{\parindent}{1em}

% Mehr als drei Auslassungspunkte 

\newcommand{\dotsseven}{%
.\kern\ellipsisgap 
.\kern\ellipsisgap
.\kern\ellipsisgap 
.\kern\ellipsisgap
.\kern\ellipsisgap
.\kern\ellipsisgap 
.\kern\ellipsisgap 	
\relax}

\newcommand{\dotssix}{%
.\kern\ellipsisgap 
.\kern\ellipsisgap
.\kern\ellipsisgap
.\kern\ellipsisgap
.\kern\ellipsisgap 
.\kern\ellipsisgap 
\relax}

\newcommand{\dotsfive}{%
.\kern\ellipsisgap 
.\kern\ellipsisgap
.\kern\ellipsisgap
.\kern\ellipsisgap 
.\kern\ellipsisgap 
\relax}

\newcommand{\dotsfour}{%
.\kern\ellipsisgap 
.\kern\ellipsisgap
.\kern\ellipsisgap
.\kern\ellipsisgap 
\relax}

\newcommand{\dotstwo}{%
.\kern\ellipsisgap 
.\kern\ellipsisgap
\relax}


% Silbentrennung
\selectlanguage{ngerman}
\hyphenation{Re-kours EP-STEIN Her-vay-vor-les-ung Steu-er-sa-chen Öst-reich Burck-hard Keuch-hus-ten Oedi-pus-auf-führ-un-gen Hi-obs-post Kärnt-ner-ring Vei-tlis-sen-gas-se Franck-gas-se Rath-hau-se Sechs-schg Stu-bai-thal Tha-deusz Volks-th Halb-mo-nats-schrift JAHR-ES-ZEI-TEN Te-le-phon mit-ge-theilt Ge-schäfts-ver-bin-dung hoch-müth-ig Ueber-zeu-gung bis-chen Au-tor-rech-te Hof-manns-thal Nor-deijk Irre-seins Tschap-perl mit-zu-thei-len Aeu-ße-rung be-thö-ren Kü-ni-gel Be-ur-thei-lung Kuenst-lern ko-moe-di-sche hae-mor-rha-gi-scher Doer-mann Wash-burn flei-ssig haute Buddh-ist Preu-ssen Lin-den-café Mit-theil-un-gen An-theil Lieu-te-nant oes-terr Rieg-ner Oes-ter-reich gro-ssem Fran-zo-sen-thum Roche Lili Ent-schlie-ssun-gen äu-ssert wuen-sche Trans-ac-tio-nen Ue-ber-win-dung Eu-gene Stra-ssen-dir-ne qua-tre Deutsch-öst-er-reich Deutsch-öst-er-reichs Bjørn-stjer-ne noth-ing Edit-ed Olga Ar-naud Mer-gent-heim Léon-tine Polla-czek Brion Barre Hoch-sin-ger Ka-tha-rina Arouet Va-len-ci-ennes Ueber-win-dung Type-writer-in Tolstoi-buch Schnitzler Copier-buche Schiller Intel-lek-tuell-en-as-so-zi-a-tion Salten Devrient Grien-steidl Ge-sell-ſchaft ein-ge-ſchloſ-ſen Fort-ſetz-un-gen Bor-dell-ſtück fort-ſchrei-ten wirk-ſam-es ſchrift-ſtel-ler-i-ſchen hin-weg-ſe-hen Gerichts-saal-be-richt-er-ſtat-ter}



% Sonderbefehl für .–
\def\dotdash{\nobreak\hspace{0pt}.–}  %ACHTUNG BEIM ERSETZEN: LEERZEICHEN DANACH 
\def\commadash{\nobreak\hspace{0pt},–}
\def\excdash{\nobreak\hspace{0pt}!–}
\def\semicolondash{\nobreak\hspace{0pt};–}
\def\parentdotdash{\nobreak\hspace{0pt}).–}
\def\slashislash{\,\slash\,\allowbreak\hspace{0pt}}

\newcommand{\strich}{\makebox[1em][l]{– }}


% Seite einrichten

% Farbe definieren
%\setmainfont[RawFeature={-liga}, 
%SmallCapsFont=WSVgara-Caps, 
%ItalicFont=WSVgara-Italic, 
%BoldFont=WSVgara-Bold,
%BoldItalicFont=WSVgara-BoldItalic
%]{WSVgara}
%\setsansfont[RawFeature={-liga}, 
%SmallCapsFont=WSVgara-Caps, 
%ItalicFont=WSVgara-Italic, 
%BoldFont=WSVgara-Bold,
%BoldItalicFont=WSVgara-BoldItalic
%]{WSVgara}

%\setmainfont{Brill}
%\setsansfont{Brill}

%\setmainfont[ItalicFont=SinaNova-Italic, 
%BoldFont=SinaNova-Bold,
%BoldItalicFont=SinaNova-BoldItalic
%]{SinaNova-Regular}
%\setsansfont[ItalicFont=SinaNova-Italic, 
%BoldFont=SinaNova-Bold,
%BoldItalicFont=SinaNova-BoldItalic
%]{SinaNova-Regular}



\def\labelitemi{--}

% Geminationsstrich, U-Strich

 \newcommand{\overbar}[1]{$\overline{\hbox{#1}}$}


% Ausrufezeichen in den Index kriegen
\newcommand{\rufezeichen}{"!}

% Griechisch
	
%\newfontfamily\greekfont{GaramondPremrPro}
%\newcommand\griechisch[1]{\greekfont{}#1{}\normalfont}
\newcommand\griechisch[1]{#1}


%\newfontfamily\sansseriffont[HyphenChar=None, RawFeature={-liga}, Scale=1.03]{TheSans-Regular}
%\newfontfamily\sansseriffont{uarial}


%\newfontfamily\sansseriffont[HyphenChar=None, LetterSpace=1.0, RawFeature={-liga}]{TheSans-SemiBold}
%\newcommand\sansseriff[1]{\sffamily{}#1{}\normalfont}

\newcommand{\mini}{\,}


\newcommand{\key}{\textsuperscript{\textcolor{red}{KEY}}}


%% Sperrung (Package Soul)
%% Hier ist die Sperrung definiert. Sperrung erreicht man mit \so{gesperrtes Wort}
\sodef\so{}{.14em}{.4em plus.1em minus .1em}{.4em plus.1em minus .1em}

% SCHRIFTEN
\setkomafont{disposition}{}
\addtokomafont{caption}{\small}
\addtokomafont{captionlabel}{\small}

%% Schrift der Kopf und Fußzeile
\renewcommand*{\headfont}{\normalfont}
\setkomafont{pagehead}{\footnotesize\addfontfeature{LetterSpace=10.0}}
\setkomafont{pagenumber}{\normalfont\normalsize}
\ohead[]{\pagemark}% Seitenzahl (c = centered) 
\ofoot[]{}


 
% Flatterndes Seitenende
\raggedbottom

% Fussnoten neu Anfangen

\makeatletter
\pretocmd{\@schapter}{\setcounter{footnote}{0}}{}{}
\pretocmd{\@chapter}{\setcounter{footnote}{0}}{}{}
\pretocmd{\@section}{\setcounter{footnote}{0}}{}{}
\makeatother


% Section Nummern durchgehend

\RedeclareSectionCommand[
  counterwithout=chapter
]{section}

% Section Punkt

\renewcommand*{\sectionformat}{}
\renewcommand*{\partformat}{}


% Marginpar Schrift

\newkomafont{margin}{\footnotesize} 
\makeatletter 
\let\MarginParOriginal\marginpar 
\renewcommand*{\marginpar}{\@dblarg\@marginpar} 
\newcommand{\@marginpar}[2][]{% 
  \MarginParOriginal[\usekomafont{margin}{#1\par}]{\usekomafont{margin}{#2\par}} 
} 
\makeatother 



\let\oldbeginnumbering\beginnumbering

\def\beginnumbering{\oldbeginnumbering\par\nopagebreak}


% Fußnoten linksbündig
\deffootnote{1.5em}{1em}{% 
\makebox[1.5em][l]{\thefootnotemark}%
}


% Fussnotenlineal (wobei für reledmac wohl was anderes gilt)
\let\normalfootnoterule\footnoterule
\setfootnoterule{0pt}
\let\normalfootnoterule\footnoterule


\setlength{\skip\footins}{8mm plus 2mm} % Abstand Fussnote Text
\interfootnotelinepenalty=10000 % Kein Seitenwechsel in Fuss

%% Kapitelüberschriften
\renewcommand*{\raggedchapter}{\centering} 
\renewcommand*{\raggedsection}{%
 \CenteringLeftskip=1cm plus 1em\relax 
 \CenteringRightskip=1cm plus 1em\relax 
 \Centering\footnotesize\thesection{}.\ }
\setkomafont{section}{\footnotesize}
\setkomafont{chapter}{\normalfont\Large}
\renewcommand{\chapterpagestyle}{empty}%The first page in each chapter won't have any heading or footer, especially no page number

% section ohne führende Kapitelnummer
\renewcommand*\thesection{\arabic{section}}

% Bildunterschrift ohne Nummer
\renewcommand*{\figureformat}{}
\renewcommand*{\captionformat}{}

% Abstand Bild
\setlength{\textfloatsep}{\baselineskip}

%% Zeilennummern
\firstlinenum{0} \linenumincrement{5}
\lineation{section} %Jeder Abschnitt wird durchnummeriert
\renewcommand{\numlabfont}{\ssmall} %Schriftgröße Zeilennummern

%\AtBeginEnvironment{multicols}{\RaggedRight} % Linksbündig in Spalten


% SEITENUMBRÜCHE IM TEXT MARKIEREN

%% Seitenumbrüche


\newcommand{\Theight}{\dimexpr\fontcharht\font`W}
\newcommand{\pbposition}{\depth}
\newcommand{\pb}{\nobreak\hspace{0pt}\raisebox{-0.1em}{\raisebox{\pbposition}{\textnormal{|}}}\nobreak\hspace{0pt}}

% EINFÜGUNGEN IM TEXT MARKIEREN

\renewcaptionname{ngerman}{\contentsname}{Inhalt}           %Table of contents


\newcommand{\introOben}{\textnormal{\raisebox{\Theight}{\raisebox{-\height}{\small{v}\normalsize}}}}
\newcommand{\introUnten}{\textnormal{\raisebox{\Theight}{\raisebox{-\height}{\small{v}\normalsize}}}}
\newcommand{\introMitteVorne}{\textnormal{\raisebox{\Theight}{\raisebox{-\height}{\small{v}\normalsize}}}}
\newcommand{\introMitteHinten}{\textnormal{\raisebox{\Theight}{\raisebox{-\height}{\small{v}\normalsize}}}}
\newcommand{\substVorne}{\textnormal{\raisebox{\Theight}{\raisebox{-\height}{\rotatebox[origin=c]{180}{v}\normalsize}}}}
\newcommand{\substDazwischen}{}
\newcommand{\substHinten}{\textnormal{\raisebox{\Theight}{\raisebox{-\height}{\small{v}\normalsize}}}}


% MARGINALSPALTE
\setlength\ledrsnotewidth{1.5cm}


% FUSSNOTE
%% Im Apparat f. und ff.
\Xtwolines{f.}
\Xtwolinesbutnotmore

%% Sperrungen bei Lemmas im Apparat
%\pretocmd{\so}{\null}{}{}
% Hab ich auskommentiert: Hat einen Fehler ergeben, denn plötzlich war ein Abstand vor Absätzen, die mit einer Sperrung beginnen

%% Zeilennummerierung Abstand zum Lemma
\Xboxlinenum{5mm}

%% Bei zwei Apparateinträgen in einer Zeile wird nur beim ersten Mal die Zeile gezählt
\Xnumberonlyfirstinline
\Xnumberonlyfirstintwolines
\Xinplaceofnumber{1em}
\Xhangindent{1em}

% ENDNOTEN
\Xendlemmadisablefontselection[A]
\renewcommand*{\printnpnum}[1]{{\noindent}\tiny}
\Xendparagraph[A] % Endnoten in einem Absatz
%\Xendtwolines{\tiny{f.}}
\Xendbeforepagenumber{} 
\Xendnotenumfont[A]{\tiny}
\Xendboxlinenum[A]{0em}
\Xendlemmaseparator{$\rbracket$}
\Xendnotefontsize[A]{\footnotesize}
\Xendhangindent[A]{1em}
\Xendlemmafont[A]{\itshape}
\Xendlemmafont[B]{\bfseries}
\Xendnotefontsize[B]{\footnotesize}
\Xendnotenumfont{\footnotesize}
\Xendlineprefixsingle[C]{\tiny}
\Xendlineprefixmore[C]{\tiny}
\Xendlemmadisablefontselection
\Xendlemmafont{\itshape}
\Xendlinerangeseparator{\tiny{--}}
\Xendhangindent{4em}
\Xendboxlinenum{3.6em}
\Xendafternumber{0.4em}
\Xendboxlinenumalign{R}

%\Xendboxstartlinenum{3.5em}
%\Xendboxendlinenum{1em}


%% Kaufmanns-Und (=)
            
            

\newcommand{\kaufmannsund}{\&} 

%% Tabelle Zellensprung
% Ein weiterer Anlass, das Kaufmannsund in der Übergabe zu vermeiden:

\newcommand{\zellensprung}{ \& }

%% INDEX
    
    \makeindex 
    \newcommand*\lettergroup[1]{}
    
        \newcommand{\pw}[1]{#1}
        \newcommand{\pwt}[1]{\textbf{#1}}
        \newcommand{\pws}[1]{\upshape{\textbf{#1}}}
            
        \newcommand{\pwe}[1]{\textbf{\emph{#1}}}
             
    \newcommand{\pwk}[1]{#1\textsuperscript{\tiny{K}}}
    \newcommand{\pwv}[1]{\emph{#1}}
     \newcommand{\pwkv}[1]{\emph{#1}\textsuperscript{\tiny{K}}}
               \newcommand{\pwuv}[1]{\emph{#1}?}
               \newcommand{\pwu}[1]{#1?}
 \newcommand{\range}[2]{{\def\pw##1{##1}#1}--#2}

\newcommand{\buch}[1]{#1}


%% MEHRERE INDIZES

\newindex[Register]{pw}
%\newindex[Institutionen Organisationen Periodika und Unternehmen]{org}
%\newindex[Institutionen und Orte]{o}
\newindex[Korrespondenzpartner]{briefe-out}
\newindex[Gedruckte Quellen]{buch-abdruck}

\newcommand\briefsenderindex[1]{\sindex[briefe-out]{#1}}
\newcommand\briefempfaengerindex[1]{\sindex[briefe-out]{#1}}

\newcommand\buchabdruck[1]{\sindex[buch-abdruck]{#1}}
\renewcommand\buchabdruck[1]{}



%% Symbole

%\newcommand{\symaddr}{\includegraphics[height=6pt]{symbol/noun_637366.png}}
%\newcommand{\symweiteredrucke}{\includegraphics[height=6pt]{symbol/noun_634729.png}}
%\newcommand{\symdruckvorlage}{\includegraphics[height=6pt]{symbol/noun_637409.png}}
%\newcommand{\symstandort}{\includegraphics[height=6pt]{symbol/noun_634216.png}}
%\newcommand{\symhead}{\includegraphics[height=6pt]{symbol/noun_1162030_cc.png}}


\newcommand{\symaddr}{A}
\newcommand{\symweiteredrucke}{D}
\newcommand{\symdruckvorlage}{V}
\newcommand{\symstandort}{O}
\newcommand{\symhead}{H}



\newcommand\anhangTitel[2]{\toendnotes[C]{\hangpara{4em}{1}{\makebox[4em][l]{\textbf{#1}}\textbf{#2}}\endgraf}}
\newcommand\Adresse[1]{\toendnotes[C]{\hangpara{4em}{1}{\makebox[4em][l]{\makebox[3.6em][r]{\symaddr}}}#1\endgraf}}

\newcommand\buchAlsQuelle[1]{\toendnotes[C]{\footnotesize\par\hangpara{4em}{1}{\makebox[4em][l]{\makebox[3.6em][r]{\symdruckvorlage}}}#1\endgraf}}
\newcommand\buchAbdrucke[1]{\toendnotes[C]{\footnotesize\par\hangpara{4em}{1}{\makebox[4em][l]{\makebox[3.6em][r]{\symweiteredrucke}}}#1\endgraf}}
\newcommand\Standort[1]{\toendnotes[C]{\footnotesize\hangpara{4em}{1}{\makebox[4em][l]{\makebox[3.6em][r]{\symstandort}}}#1\endgraf}}
\newcommand\biographical[1]{\toendnotes[C]{\footnotesize\hangpara{4em}{1}{\makebox[4em][l]{\makebox[3.6em][r]{\symhead}}}#1\endgraf}}
\newcommand\biographicalOhne[1]{\toendnotes[C]{\footnotesize\hangpara{4em}{1}{\makebox[4em][l]{\makebox[3.6em][r]{}}}#1\endgraf}}



\newcommand\datumImAnhang[1]{\toendnotes[C]{#1}}

\let\newcell&

\newcommand\physDesc[1]{\toendnotes[C]{\hangpara{4em}{0}#1\endgraf}}
\newcommand\weitereDrucke[1]{#1}


% Schnitzler Tagebuch Auszüge
\newcommand{\prgrph}[1]{\endgraf\medskip\noindent\textbf{#1}\newline}


%% VERWEISE
% Dieser Befehl vom Typ
% \verweis{FW_V_schwn_A}{FW_V_schwn_E} 
% dient den Verweisen auf den Text von Kommentar und Herausgebereingriffen. Ihm werden die Namen der beiden Labels – Anfang und Ende – übergeben und er setzt den Anfang und entscheidet ob f. oder ff. folgt 


\newcounter{mystart}
\newcounter{mystop}
\newcounter{phantom}

\newcommand*\myrangeref[2]{%
  \setcounterpageref{mystart}{#1}%
  \setcounterpageref{mystop}{#2}%
  \ifnum\value{mystop}<\value{mystart}%
    \typeout{[myrangeref] Strange...stop (#2) before start (#1).}%
    \pageref{#2}--\pageref{#1}%
  \else
    \pageref{#1}%
    \ifnum\value{mystart}<\value{mystop}%
      \addtocounter{mystop}{-1}%
      \ifnum\value{mystart}<\value{mystop}%
        \,ff.
        %--\pageref{#2}%%
      \else
        \,f.
         %%--\pageref{#2}%
              \fi
    \fi
  \fi
}
            
\newcommand*\myrangerefkasten[2]{%
  \setcounterpageref{mystart}{#1}%
  \setcounterpageref{mystop}{#2}%
  \ifnum\value{mystop}<\value{mystart}%
    \typeout{[myrangeref] Strange...stop (#2) before start (#1).}%
    \pageref{#2}--\pageref{#1}%
  \else
    \makebox[12pt][r]{\pageref{#1}}%
    \ifnum\value{mystart}<\value{mystop}%
      \addtocounter{mystop}{-1}%
      \ifnum\value{mystart}<\value{mystop}%
        --\pageref{#2}%%
      \else
         --\pageref{#2}%
         % alternativ hierher: f.
      \fi
    \fi
  \fi
}


\newcommand*\mylabel[1]{%
  \refstepcounter{phantom}%
  \label{#1}%
}

\newenvironment{anhang}{\vspace{1cm}
}{}

\emfontdeclare{\itshape}

%% RAHMEN SEITLICH

\newlength{\leftbarwidth}
\setlength{\leftbarwidth}{3pt}
\newlength{\leftbarsep}
\setlength{\leftbarsep}{10pt}

\renewenvironment{leftbar}[1][\hsize]
{% 
\def\FrameCommand 
{%
{\hspace{-7pt} \color{black} \vrule width 0.5pt}%
\hspace{0pt}%must no space.
\fboxsep=\FrameSep\colorbox{white}%
}%
\MakeFramed{\hsize#1\advance\hsize-\width\FrameRestore}%
}
{\endMakeFramed}
\setlength{\FrameSep}{5pt}

\newmdenv[topline=false, leftline=true, rightline=true, bottomline=false,%
  linewidth=0.5pt, leftmargin=30pt, rightmargin=30pt, %
  skipabove=8pt, skipbelow=8pt]{mdbar}

% Überstreichung (OVERLINE)

\makeatletter
\newcommand*{\textoverline}[1]{$\overline{\hbox{#1}}\m@th$}
\makeatother

% Rahmen für Hintergrundfarbe
\fboxsep0mm

% Befehl für gekürzte Texte

\newcommand{\kuerzung}{, Auszug}

% Verse 

\setlength{\stanzaindentbase}{20pt} %Play with it later.
\setstanzaindents{5,1,1}
\setcounter{stanzaindentsrepetition}{2}
\newcommand{\stanzaend}{\&}
\sethangingsymbol{\protect\hfill}
\AtEveryStopStanza{\vspace{0.25\baselineskip}} %Abstand zwischen Strophen


% Versuch eines Grid

\RedeclareSectionCommand[
  beforeskip=3\baselineskip,
  afterskip=\baselineskip
]{chapter}
\RedeclareSectionCommand[
  beforeskip=2\baselineskip,
  afterskip=\baselineskip
]{section}

\newcommand\adjacent[2][]{%
  \bgroup
  \RedeclareSectionCommand[
    beforeskip=2\baselineskip,
    afterskip=\baselineskip,
  ]{chapter}%
  \if\relax\detokenize{#1}\relax
    \addchap{#2}%
  \else
    \addchap[#1]{#2}%
  \fi
  \egroup
  \section
}


%change the part format in table of contents
\renewcaptionname{ngerman}{\contentsname}{Inhalt} 


% Inhaltsverzeichnis

\AtBeginDocument{%
  \addtocontents{toc}{\protect\label{toc}}%
}

\renewcaptionname{ngerman}{\contentsname}{Verzeichnis der Dokumente} 
 
 
   \DeclareTOCStyleEntry[
  beforeskip=15pt,
  entryformat=\normalsize\normalfont\centering,
  pagenumberformat=\nullfont,
  linefill={},
  raggedentrytext=true
]{part}{part}

  \DeclareTOCStyleEntry[
  beforeskip=5pt,
  entryformat=\normalsize\normalfont\centering,
  pagenumberformat=\nullfont,
  linefill={},
  raggedentrytext=true
]{chapter}{chapter}

\DeclareTOCStyleEntry[
  onstarthigherlevel=\vspace*{0.5\baselineskip}\nobreak,
  indent=0pt,
  entryformat=\normalsize\def\autodot{.},
  pagenumberformat=\normalsize,
  raggedentrytext=true
]{section}{section}



 
% Das folgende auskommentiert, funktionierte nicht mehr, ging aber in Bahr/Schnitzler. Sollte eigentlich dazu dienen, beim Inhaltsverzeichnis die Nummern rechtsbündig zu setzen

 \iffalse
 
  \DeclareTOCStyleEntry[
  beforeskip=5pt,
  entryformat=\normalsize\normalfont\centering,
  pagenumberformat=\nullfont,
  linefill={},
  raggedentrytext=true
]{chapter}{chapter}

\DeclareTOCStyleEntry[
  onstarthigherlevel=\vspace*{0.5\baselineskip}\nobreak,
  indent=0pt,
  entryformat=\normalsize\def\autodot{.},
  pagenumberformat=\normalsize,
  raggedentrytext=true
]{section}{section}
 
 
  \newcommand*\sectionnumberbox[1]{\hfill #1\hspace{.6em}}

\newlength{\zweiziffern}
\newlength{\dreiziffern}
\newlength{\vierziffern}
\settowidth{\zweiziffern}{9999}
\settowidth{\dreiziffern}{99999}
\settowidth{\vierziffern}{99999999}
 
\BeforeStartingTOC[toc]{\value{tocdepth}=\sectiontocdepth}


\DeclareTOCStyleEntry[
  onstarthigherlevel=\vspace*{0.5\baselineskip}\nobreak,
  indent=0pt,
  entryformat=\normalsize\def\autodot{.},
  entrynumberformat=\sectionnumberbox,
  pagenumberformat=\normalsize,
  numwidth=\zweiziffern,
  raggedentrytext=true
]{section}{section}

\newcommand{\toccheck}{\ifnum \value{section}=76 \addtocontents{toc}{\protect\DeclareTOCStyleEntry[numwidth=\dreiziffern]{section}{section}} \else \ifnum \value{section}=990 \addtocontents{toc}{\protect\DeclareTOCStyleEntry[numwidth=\vierziffern]{section}{section}} \fi \fi}
\fi



% Längen für Tabellen
\newlength{\longeste}
\newlength{\longestz}
\newlength{\longestd}
\newlength{\longestv}
\newlength{\longestf}

\newcommand\halbtextwidth{0.9\textwidth}

\newcommand\pwindex[1]{{\sindex[pw]{#1}}}
\newcommand\oindex[1]{{\sindex[pw]{#1}}}
\newcommand\orgindex[1]{{\sindex[pw]{#1}}}

\renewcommand\oindex[1]{{{\sindex[pw]{#1}}}}
\renewcommand\orgindex[1]{{{\sindex[pw]{#1}}}}



% INDEX

%\renewcommand\pwindex[1]{}
%\renewcommand\oindex[1]{}
%\renewcommand\orgindex[1]{}
%\renewcommand\buchabdruck[1]{}


\newcommand\url[1]{\mbox{#1}}
\renewcommand\ngermanhyphenmins{33}

\makeatletter
\newcommand*{\geminationm}{$\overline{\hbox{m}}\m@th$}
\newcommand*{\geminationn}{$\overline{\hbox{n}}\m@th$}
\makeatother

%part
\renewcommand{\partmarkformat}{}
\renewcommand{\partheadmidvskip}{\enskip}
\renewcommand{\partformat}{}
\setkomafont{partnumber}{\usekomafont{part}}


%\geometry{headsep=8pt} % Abstand Kopfzeile - Text
%% DOKUMENT

\begin{document}

% Section ohne Nummer
\renewcommand*{\raggedsection}{%
 \CenteringLeftskip=1cm plus 1em\relax 
 \CenteringRightskip=1cm plus 1em\relax 
 \Centering\normalsize}



\widowpenalty=10000         % avoid widows
\clubpenalty=10000          % avoid orphans

\sloppy
\setlength{\parindent}{0em}

\setlength{\ledlsnotewidth}{4cm}
\setlength{\ledrsnotewidth}{4cm}
\renewcommand*{\ledlsnotefontsetup}{\scriptsize\sffamily}% left
\renewcommand*{\ledrsnotefontsetup}{\scriptsize\sffamily}% left
\thispagestyle{empty} 

               \section[Arthur Schnitzler an Richard Beer-Hofmann, 26. 1. 1905]{ Arthur Schnitzler an Richard Beer-Hofmann, 26. 1. 1905}\nopagebreak\mylabel{v}\rehead{ }\normalsize\beginnumbering\briefempfaengerindex{Beer-Hofmann, Richard@\textsc{Beer-Hofmann, Richard}!zzzSchnitzler, Arthur@\emph{von Arthur Schnitzler}!1905-01-261@{26. 1. 1905}|(be} \toendnotes[C]{\smallbreak\pagebreak[2]} \Standort{YCGL, MSS 31.}
\physDesc{Brief, 1 Blatt, 1 Seite, Umschlag
\newline{}Handschrift: Bleistift, deutsche Kurrent\newline{}Beilage: gedruckte Beilage, 1 Blatt, 2 Seiten \newline{}Versand: 1) Stempel: »\nobreak{}\oindex{XVIII., Waehring@\textbf{XVIII., Währing}, \emph{Bezirk (A.BZK)}|pwk}Wien 110, 26. 1. 05, 6\nobreak{}«.  2) Stempel: »\nobreak{}\oindex{Rodaun@\textbf{Rodaun}, \emph{Teil eines besiedelten Ortes (A.BSOX)}|pwk}{\pb}Rodaun, 27 {[}1 05{]}\nobreak{}«. \newline{}Ordnung: mit Bleistift von unbekannter Hand die Beilage beschriftet:
                                    »Zum 26. 1. 1905« \newline{}Editorischer Hinweis: Die Vorderseite der Beilage (Seite 17) wird, da sie nur das
                                 Schlussfragment einer anderen Schiffsreise enthält, in der
                                 Wiedergabe nicht berücksichtigt. }\toendnotes[C]{\smallbreak}\pstart{}{\pb}\textsc{Herrn Dr. Richard Beer-Hofma{\geminationn}}\pend{}\pstart{}\textsc{\textcolor{pink}{Rodaun \introOben{}bei Liesing\introOben{}}{}\ledrightnote{\textcolor{pink}{Rodaun}}}\pend{}\pstart{}\textcolor{pink}{\textsc{Liesingerstr. 2}}{}\ledrightnote{\textcolor{pink}{Liesingerstraße}}. \pend{}{\bigskip}\pstart
           \raggedleft{}{\pb}26. 1. 905.\pend
           \pstart{}lieber Richard, \pend\pstart
           lockt Sie das nicht? – Würde Ihnen u \textcolor{blue}{Paula}{}\ledrightnote{\textcolor{blue}{Paula Beer-Hofmann}}
               wohlthun. Ich würde mit \textcolor{blue}{Olga}{}\ledrightnote{\textcolor{blue}{Olga Schnitzler}} gleichfalls
               fahren.\pend
           \pstart
           Schicken Sie mir dieſes Blatt zurück. Seien Sie gegrüßt.\pend
           \pstart
           Herzlichſt{\\[\baselineskip]}Ihr{\\[\baselineskip]}\spacefill\mbox{A.}\pend
           \leftskip=0em{}{\bigskip}\pstart
           \noindent{}\centering{}{\pb}\textcolor{gray}{\textbf{\textcolor{brown}{SCHENKER}{}\ledrightnote{\textcolor{brown}{Internationales Reise-Bureau Schenker}}’S REISE-BULLETIN. — JANUAR–FEBRUAR 1905.}}\pend
           \pstart
           \noindent{}\centering{}\textcolor{gray}{\textbf{\textcolor{pink}{Mittelmeer}{}\ledrightnote{\textcolor{pink}{Mittelmeer}}-, Orient- und Sonderfahrten mit
                  Dampfern der \textcolor{brown}{Hamburg-Amerika-Linie}{}\ledrightnote{\textcolor{brown}{Hamburg-Amerika-Linie}}.}}\pend
           {\bigskip}\pstart
           \noindent{}\centering{}\textcolor{gray}{\textbf{\textcolor{pink}{Mittelmeer}{}\ledrightnote{\textcolor{pink}{Mittelmeer}}fahrten mit dem
                  Doppelschrauben-Dampfer »Meteor«.}}\pend
           \pstart
           \noindent{}\textcolor{gray}{\textbf{\textbf{Erste \textcolor{pink}{Mittelmeer}{}\ledrightnote{\textcolor{pink}{Mittelmeer}}fahrt.}
                  (Von \textcolor{pink}{Hamburg}{}\ledrightnote{\textcolor{pink}{Hamburg}} ins \textcolor{pink}{Mittelmeer}{}\ledrightnote{\textcolor{pink}{Mittelmeer}}.) Abfahrt von \textcolor{pink}{Hamburg}{}\ledrightnote{\textcolor{pink}{Hamburg}}{ }\textbf{26. Oktober 1904}. Besucht werden die Häfen: \textcolor{pink}{Dover}{}\ledrightnote{\textcolor{pink}{Dover}}, \textcolor{pink}{Lissabon}{}\ledrightnote{\textcolor{pink}{Lissabon}}, \textcolor{pink}{Funchal}{}\ledrightnote{\textcolor{pink}{Funchal}}, \textcolor{pink}{Teneriffa}{}\ledrightnote{\textcolor{pink}{Teneriffa}}, \textcolor{pink}{Tanger}{}\ledrightnote{\textcolor{pink}{Tanger}}, \textcolor{pink}{Gibraltar}{}\ledrightnote{\textcolor{pink}{Gibraltar}}, \textcolor{pink}{Oran}{}\ledrightnote{\textcolor{pink}{Oran}}, \textcolor{pink}{Algier}{}\ledrightnote{\textcolor{pink}{Algiers}},
                     \textcolor{pink}{Tunis}{}\ledrightnote{\textcolor{pink}{Tunis}}, \textcolor{pink}{Palermo}{}\ledrightnote{\textcolor{pink}{Palermo}} (\textcolor{pink}{Monreale}{}\ledrightnote{\textcolor{pink}{Monreale}}), \textcolor{pink}{Gibraltar}{}\ledrightnote{\textcolor{pink}{Gibraltar}}, \textcolor{pink}{Neapel}{}\ledrightnote{\textcolor{pink}{Neapel}} (\textcolor{pink}{Vesuv}{}\ledrightnote{\textcolor{pink}{Vesuv}}, \textcolor{pink}{Pompeji}{}\ledrightnote{\textcolor{pink}{Pompei}}
                  etc.), \textcolor{pink}{Genua}{}\ledrightnote{\textcolor{pink}{Genua}}. Reisedauer 24 Tage. Fahrpreise von
                     \textbf{K 570.—} an aufwärts.}}\pend
           \pstart
           \textcolor{gray}{\textbf{\textbf{Zweite \textcolor{pink}{Mittelmeer}{}\ledrightnote{\textcolor{pink}{Mittelmeer}}fahrt.}
                  Abfahrt von \textcolor{pink}{Genua}{}\ledrightnote{\textcolor{pink}{Genua}}{ }\textbf{22. November 1904}. Besucht werden die Häfen: \textcolor{pink}{Villafranca}{}\ledrightnote{\textcolor{pink}{Villefranche-sur-Mer}}
                     (\textcolor{pink}{Nizza}{}\ledrightnote{\textcolor{pink}{Nizza}}, \textcolor{pink}{Monte
                     Carlo}{}\ledrightnote{\textcolor{pink}{Monte Carlo}}), \label{T_L01495_1v}\edtext{\textcolor{pink}{Ajaccio}{}\ledrightnote{\textcolor{pink}{Ajaccio}}}{\lemma{\textnormal{\emph{Ajaccio}}}\Cendnote{\textnormal{korrigiert
                        aus: »Ajjaccio«}}}\label{T_L01495_1h}, \textcolor{pink}{Barcelona}{}\ledrightnote{\textcolor{pink}{Barcelona}}, \textcolor{pink}{Algier}{}\ledrightnote{\textcolor{pink}{Algiers}}, \textcolor{pink}{Tunis}{}\ledrightnote{\textcolor{pink}{Tunis}}, \textcolor{pink}{Palermo}{}\ledrightnote{\textcolor{pink}{Palermo}} (\textcolor{pink}{Monreale}{}\ledrightnote{\textcolor{pink}{Monreale}}), \textcolor{pink}{Messina}{}\ledrightnote{\textcolor{pink}{Messina}}, \textcolor{pink}{Neapel}{}\ledrightnote{\textcolor{pink}{Neapel}} (\textcolor{pink}{Vesuv}{}\ledrightnote{\textcolor{pink}{Vesuv}}, \textcolor{pink}{Pompeji}{}\ledrightnote{\textcolor{pink}{Pompei}} etc.), \textcolor{pink}{Genua}{}\ledrightnote{\textcolor{pink}{Genua}}. Reisedauer 14 Tage. Fahrpreise von \textbf{K 330.—} an aufwärts.}}\pend
           \pstart
           \textcolor{gray}{\textbf{\textbf{Dritte \textcolor{pink}{Mittelmeer}{}\ledrightnote{\textcolor{pink}{Mittelmeer}}fahrt.}
                  Abfahrt von \textcolor{pink}{Genua}{}\ledrightnote{\textcolor{pink}{Genua}}{ }\textbf{8. Dezember 1904}. Fahrplan ebenso wie bei der zweiten \textcolor{pink}{Mittelmeer}{}\ledrightnote{\textcolor{pink}{Mittelmeer}}fahrt. Reisedauer 14 Tage. Fahrpreise von \textbf{K 330.—} an aufwärts.}}\pend
           \pstart
           \textcolor{gray}{\textbf{\textbf{Vierte \textcolor{pink}{Mittelmeer}{}\ledrightnote{\textcolor{pink}{Mittelmeer}}fahrt.}
                  (Bis nach Konstantinopel.) Abfahrt von \textcolor{pink}{Genua}{}\ledrightnote{\textcolor{pink}{Genua}}{ }\textbf{5. Jänner 1905}. Besucht werden die Häfen: \textcolor{pink}{Villafranca}{}\ledrightnote{\textcolor{pink}{Villefranche-sur-Mer}}
                     (\textcolor{pink}{Nizza}{}\ledrightnote{\textcolor{pink}{Nizza}}, \textcolor{pink}{Monte
                     Carlo}{}\ledrightnote{\textcolor{pink}{Monte Carlo}}), \textcolor{pink}{Ajaccio}{}\ledrightnote{\textcolor{pink}{Ajaccio}}, \textcolor{pink}{Algier}{}\ledrightnote{\textcolor{pink}{Algiers}}, \textcolor{pink}{Konstantinopel}{}\ledrightnote{\textcolor{pink}{Istanbul}}, \textcolor{pink}{Smyrna}{}\ledrightnote{\textcolor{pink}{Izmir}}, \textcolor{pink}{Piräus}{}\ledrightnote{\textcolor{pink}{Piraeus}} (\textcolor{pink}{Athen}{}\ledrightnote{\textcolor{pink}{Athen}}), \textcolor{pink}{Syrakus}{}\ledrightnote{\textcolor{pink}{Syrakus}}, \textcolor{pink}{Messina}{}\ledrightnote{\textcolor{pink}{Messina}}, \textcolor{pink}{Palermo}{}\ledrightnote{\textcolor{pink}{Palermo}} (\textcolor{pink}{Monreale}{}\ledrightnote{\textcolor{pink}{Monreale}}), \textcolor{pink}{Neapel}{}\ledrightnote{\textcolor{pink}{Neapel}} (\textcolor{pink}{Vesuv}{}\ledrightnote{\textcolor{pink}{Vesuv}}, \textcolor{pink}{Pompeji}{}\ledrightnote{\textcolor{pink}{Pompei}} etc.), \textcolor{pink}{Genua}{}\ledrightnote{\textcolor{pink}{Genua}}. Reisedauer 25 Tage. Fahrpreise von \textbf{K 600.—} an aufwärts.}}\pend
           \pstart
           \textcolor{gray}{\textbf{\textbf{Fünfte \textcolor{pink}{Mittelmeer}{}\ledrightnote{\textcolor{pink}{Mittelmeer}}fahrt.}
                  (Bis nach Konstantinopel.) Abfahrt von \textcolor{pink}{Genua}{}\ledrightnote{\textcolor{pink}{Genua}}{ }\textbf{5. Februar 1905}. Fahrplan ebenso wie bei der vierten \textcolor{pink}{Mittelmeer}{}\ledrightnote{\textcolor{pink}{Mittelmeer}}fahrt. Reisedauer 25 Tage. Fahrpreise von \textbf{K 600.—} an aufwärts.}}\pend
           \pstart
           \textcolor{gray}{\textbf{\textbf{Sechste \textcolor{pink}{Mittelmeer}{}\ledrightnote{\textcolor{pink}{Mittelmeer}}fahrt.}
                  (Im \textcolor{pink}{Mittelmeer}{}\ledrightnote{\textcolor{pink}{Mittelmeer}} und Adriatischen Meer.) Abfahrt
                  von \textcolor{pink}{Genua}{}\ledrightnote{\textcolor{pink}{Genua}}{ }\textbf{5. März 1905}. Besucht werden die Häfen: \textcolor{pink}{Villafranca}{}\ledrightnote{\textcolor{pink}{Villefranche-sur-Mer}}
                     (\textcolor{pink}{Nizza}{}\ledrightnote{\textcolor{pink}{Nizza}}, \textcolor{pink}{Monte
                     Carlo}{}\ledrightnote{\textcolor{pink}{Monte Carlo}}), \textcolor{pink}{Ajaccio}{}\ledrightnote{\textcolor{pink}{Ajaccio}}, \textcolor{pink}{Neapel}{}\ledrightnote{\textcolor{pink}{Neapel}} (\textcolor{pink}{Vesuv}{}\ledrightnote{\textcolor{pink}{Vesuv}}, \textcolor{pink}{Pompeji}{}\ledrightnote{\textcolor{pink}{Pompei}} etc.), \textcolor{pink}{Palermo}{}\ledrightnote{\textcolor{pink}{Palermo}} (\textcolor{pink}{Monreale}{}\ledrightnote{\textcolor{pink}{Monreale}}), \textcolor{pink}{Messina}{}\ledrightnote{\textcolor{pink}{Messina}}, \textcolor{pink}{Korfu}{}\ledrightnote{\textcolor{pink}{Korfu}}, \textcolor{pink}{Cattaro}{}\ledrightnote{\textcolor{pink}{Kotor}}, \textcolor{pink}{Gravosa}{}\ledrightnote{\textcolor{pink}{Dubrovnik}} (\textcolor{pink}{Ragusa}{}\ledrightnote{\textcolor{pink}{Dubrovnik}}), \textcolor{pink}{Spalato}{}\ledrightnote{\textcolor{pink}{Split}}, \textcolor{pink}{Abbazia}{}\ledrightnote{\textcolor{pink}{Opatija}} (\textcolor{pink}{Fiume}{}\ledrightnote{\textcolor{pink}{Rijeka}}), \textcolor{pink}{Triest}{}\ledrightnote{\textcolor{pink}{Triest}}
                     (\textcolor{pink}{Miramare}{}\ledrightnote{\textcolor{pink}{Schloss Miramare}}), \textcolor{pink}{Venedig}{}\ledrightnote{\textcolor{pink}{Venedig}}. Reisedauer 14 Tage. Fahrpreise von \textbf{K 330.—}
                  an aufwärts.}}\pend
           \pstart
           \textcolor{gray}{\textbf{\textbf{Siebente \textcolor{pink}{Mittelmeer}{}\ledrightnote{\textcolor{pink}{Mittelmeer}}fahrt.}
                  (Im \textcolor{pink}{Mittelmeer}{}\ledrightnote{\textcolor{pink}{Mittelmeer}} und Adriatischen Meer.) Abfahrt
                  von \textcolor{pink}{Venedig}{}\ledrightnote{\textcolor{pink}{Venedig}}{ }\textbf{21. März 1905}. Besucht werden die Häfen: \textcolor{pink}{Triest}{}\ledrightnote{\textcolor{pink}{Triest}} (\textcolor{pink}{Miramare}{}\ledrightnote{\textcolor{pink}{Schloss Miramare}}), \textcolor{pink}{Abbazia}{}\ledrightnote{\textcolor{pink}{Opatija}} (\textcolor{pink}{Fiume}{}\ledrightnote{\textcolor{pink}{Rijeka}}), \textcolor{pink}{Spalato}{}\ledrightnote{\textcolor{pink}{Split}}, \textcolor{pink}{Gravosa}{}\ledrightnote{\textcolor{pink}{Dubrovnik}} (\textcolor{pink}{Ragusa}{}\ledrightnote{\textcolor{pink}{Dubrovnik}}), \textcolor{pink}{Cattaro}{}\ledrightnote{\textcolor{pink}{Kotor}}, \textcolor{pink}{Korfu}{}\ledrightnote{\textcolor{pink}{Korfu}}, \textcolor{pink}{Syrakus}{}\ledrightnote{\textcolor{pink}{Syrakus}}. \textcolor{pink}{Messina}{}\ledrightnote{\textcolor{pink}{Messina}}, \textcolor{pink}{Palermo}{}\ledrightnote{\textcolor{pink}{Palermo}} (\textcolor{pink}{Monreale}{}\ledrightnote{\textcolor{pink}{Monreale}}), \textcolor{pink}{Neapel}{}\ledrightnote{\textcolor{pink}{Neapel}} (\textcolor{pink}{Vesuv}{}\ledrightnote{\textcolor{pink}{Vesuv}}, \textcolor{pink}{Pompeji}{}\ledrightnote{\textcolor{pink}{Pompei}} etc.), \textcolor{pink}{Genua}{}\ledrightnote{\textcolor{pink}{Genua}}. Reisedauer 14 Tage. Fahrpreise von \textbf{K 330.—} an aufwärts.}}\pend
           \pstart
           \textcolor{gray}{\textbf{\textbf{Achte \textcolor{pink}{Mittelmeer}{}\ledrightnote{\textcolor{pink}{Mittelmeer}}fahrt.}
                  (Vom \textcolor{pink}{Mittelmeer}{}\ledrightnote{\textcolor{pink}{Mittelmeer}} nach \textcolor{pink}{Hamburg}{}\ledrightnote{\textcolor{pink}{Hamburg}}.) Abfahrt von \textcolor{pink}{Genua}{}\ledrightnote{\textcolor{pink}{Genua}}{ }\textbf{5. April 1905}. Besucht werden die Häfen: \textcolor{pink}{Villafranca}{}\ledrightnote{\textcolor{pink}{Villefranche-sur-Mer}}
                     (\textcolor{pink}{Nizza}{}\ledrightnote{\textcolor{pink}{Nizza}}, \textcolor{pink}{Monte
                     Carlo}{}\ledrightnote{\textcolor{pink}{Monte Carlo}}), \textcolor{pink}{Ajaccio}{}\ledrightnote{\textcolor{pink}{Ajaccio}}, \textcolor{pink}{Barcelona}{}\ledrightnote{\textcolor{pink}{Barcelona}}, \textcolor{pink}{Algier}{}\ledrightnote{\textcolor{pink}{Algiers}}, \textcolor{pink}{Gibraltar}{}\ledrightnote{\textcolor{pink}{Gibraltar}}, \textcolor{pink}{Lissabon}{}\ledrightnote{\textcolor{pink}{Lissabon}}, \textcolor{pink}{Dover}{}\ledrightnote{\textcolor{pink}{Dover}}, \textcolor{pink}{Hamburg}{}\ledrightnote{\textcolor{pink}{Hamburg}}. Reisedauer 16 Tage, Fahrpreise von \textbf{K 390.—} an aufwärts.}}\pend
           {\bigskip}\pstart
           \noindent{}\centering{}\textcolor{gray}{\textbf{Mit dem Doppelschrauben-Schnelldampfer »Prinzessin Victoria
                  Luise«.}}\pend
           \pstart
           \noindent{}\textcolor{gray}{\textbf{\textbf{Neunte \textcolor{pink}{Mittelmeer}{}\ledrightnote{\textcolor{pink}{Mittelmeer}}fahrt.}
                  Abfahrt von \textcolor{pink}{New-York}{}\ledrightnote{\textcolor{pink}{New York City}}{ }\textbf{4. April 1905}. Besucht werden die Häfen: \textcolor{pink}{Ponta Delgada}{}\ledrightnote{\textcolor{pink}{Ponta Delgada}},
                     \textcolor{pink}{Funchal}{}\ledrightnote{\textcolor{pink}{Funchal}}, \textcolor{pink}{Santa
                     Cruz}{}\ledrightnote{\textcolor{pink}{Santa Cruz}}, \textcolor{pink}{Gibraltar}{}\ledrightnote{\textcolor{pink}{Gibraltar}}, \textcolor{pink}{Algier}{}\ledrightnote{\textcolor{pink}{Algiers}}, \textcolor{pink}{Palermo}{}\ledrightnote{\textcolor{pink}{Palermo}} (\textcolor{pink}{Monreale}{}\ledrightnote{\textcolor{pink}{Monreale}}), \textcolor{pink}{Neapel}{}\ledrightnote{\textcolor{pink}{Neapel}}, \textcolor{pink}{Genua}{}\ledrightnote{\textcolor{pink}{Genua}}. Reisedauer 25 Tage.
                  Fahrpreise von \textbf{K 360.—} an aufwärts.}}\pend
           \pstart
           \textcolor{gray}{\textbf{\textbf{Zehnte \textcolor{pink}{Mittelmeer}{}\ledrightnote{\textcolor{pink}{Mittelmeer}}fahrt.}
                  Abfahrt von \textcolor{pink}{Genua}{}\ledrightnote{\textcolor{pink}{Genua}}{ }\textbf{30. April 1905}. Besucht werden die Häfen: \textcolor{pink}{Villafranca}{}\ledrightnote{\textcolor{pink}{Villefranche-sur-Mer}}
                     (\textcolor{pink}{Nizza}{}\ledrightnote{\textcolor{pink}{Nizza}}, \textcolor{pink}{Monte
                     Carlo}{}\ledrightnote{\textcolor{pink}{Monte Carlo}}), \textcolor{pink}{Ajaccio}{}\ledrightnote{\textcolor{pink}{Ajaccio}}, \textcolor{pink}{Cagliari}{}\ledrightnote{\textcolor{pink}{Cagliari}}, \textcolor{pink}{Tunis}{}\ledrightnote{\textcolor{pink}{Tunis}}, \textcolor{pink}{Algier}{}\ledrightnote{\textcolor{pink}{Algiers}}, \textcolor{pink}{Oran}{}\ledrightnote{\textcolor{pink}{Oran}},
                     \textcolor{pink}{Gibraltar}{}\ledrightnote{\textcolor{pink}{Gibraltar}}, \textcolor{pink}{Lissabon}{}\ledrightnote{\textcolor{pink}{Lissabon}}, \textcolor{pink}{Oporto}{}\ledrightnote{\textcolor{pink}{Porto}}, \textcolor{pink}{Dover}{}\ledrightnote{\textcolor{pink}{Dover}}, \textcolor{pink}{Hamburg}{}\ledrightnote{\textcolor{pink}{Hamburg}}. Reisedauer
                  14 Tage. Fahrpreise von \textbf{K 312.—} an aufwärts.}}\pend
           {\bigskip}\pstart
           \noindent{}\centering{}\textcolor{gray}{\textbf{Westindienfahrten mit dem Doppelschrauben-Schnelldampfer
                  »Prinzessin Victoria Luise«.}}\pend
           \pstart
           \noindent{}\textcolor{gray}{\textbf{\textbf{Erste Westindienfahrt.} Abfahrt von \textcolor{pink}{New-York}{}\ledrightnote{\textcolor{pink}{New York City}}{ }\textbf{12. Jänner 1905}. Besucht werden die Häfen: \textcolor{pink}{St. Thomas}{}\ledrightnote{\textcolor{pink}{Saint Thomas}},
                     \textcolor{pink}{St. Pierre (Martinique)}{}\ledrightnote{\textcolor{pink}{Saint-Pierre}}, \textcolor{pink}{Fort de France}{}\ledrightnote{\textcolor{pink}{Fort-de-France}}, \textcolor{pink}{San Juan
                     (Puerto Rico)}{}\ledrightnote{\textcolor{pink}{San Juan}}, \textcolor{pink}{Santiago de Cuba}{}\ledrightnote{\textcolor{pink}{Santiago de Cuba}}, \textcolor{pink}{Havana}{}\ledrightnote{\textcolor{pink}{Havana}}, \textcolor{pink}{Nassau}{}\ledrightnote{\textcolor{pink}{Nassau}}. Reisedauer 18 Tage. Fahrpreise von \textbf{K 600.—}
                  an aufwärts.}}\pend
           \pstart
           \textcolor{gray}{\textbf{\textbf{Zweite Westindienfahrt.} Abfahrt von New-York { }\textbf{2. Februar 1905}. Besucht werden die Häfen : \textcolor{pink}{St. Thomas}{}\ledrightnote{\textcolor{pink}{Saint Thomas}},
                     \textcolor{pink}{San Juan (Puerto Rico)}{}\ledrightnote{\textcolor{pink}{San Juan}}, \textcolor{pink}{Fort de France (Martinique)}{}\ledrightnote{\textcolor{pink}{Fort-de-France}}, \textcolor{pink}{St. Pierre}{}\ledrightnote{\textcolor{pink}{Saint-Pierre}}, \textcolor{pink}{Bridgetown (Barbados)}{}\ledrightnote{\textcolor{pink}{Bridgetown}}, \textcolor{pink}{Port of Spain (Trinidad)}{}\ledrightnote{\textcolor{pink}{Port of Spain}}, \textcolor{pink}{La Brea Point}{}\ledrightnote{\textcolor{pink}{Punta Brea}}, \textcolor{pink}{Port of
                     Spain}{}\ledrightnote{\textcolor{pink}{Port of Spain}}, \textcolor{pink}{La Guayra (Caracas)}{}\ledrightnote{\textcolor{pink}{La Guaira}}, \textcolor{pink}{Puerto Cabello}{}\ledrightnote{\textcolor{pink}{Puerto Cabello}}, \textcolor{pink}{Curaçao}{}\ledrightnote{\textcolor{pink}{Curaçao}}, \textcolor{pink}{Kingston (Jamaika)}{}\ledrightnote{\textcolor{pink}{Kingston}}\textcolor{pink}{Santiago de Cuba}{}\ledrightnote{\textcolor{pink}{Santiago de Cuba}}, \textcolor{pink}{Havana}{}\ledrightnote{\textcolor{pink}{Havana}}, \textcolor{pink}{Nassau}{}\ledrightnote{\textcolor{pink}{Nassau}}. Reisedauer 28 Tage.
                  Fahrpreise von \textbf{K 840.—} an aufwärts.}}\pend
           \pstart
           \textcolor{gray}{\textbf{\textbf{Dritte Westindienfahrt.} Abfahrt von New-York { }\textbf{7. März 1905}. Besucht werden die Häfen: \textcolor{pink}{Nassau}{}\ledrightnote{\textcolor{pink}{Nassau}}, \textcolor{pink}{Havana}{}\ledrightnote{\textcolor{pink}{Havana}}, \textcolor{pink}{Santiago
                     de Cuba}{}\ledrightnote{\textcolor{pink}{Santiago de Cuba}}, \textcolor{pink}{Kingston (Jamaika)}{}\ledrightnote{\textcolor{pink}{Kingston}}, \textcolor{pink}{San Juan (Puerto Rico)}{}\ledrightnote{\textcolor{pink}{San Juan}}, \textcolor{pink}{St. Thomas}{}\ledrightnote{\textcolor{pink}{Saint Thomas}}, \textcolor{pink}{Bridgetown
                     (Barbados)}{}\ledrightnote{\textcolor{pink}{Bridgetown}}, \textcolor{pink}{Fort de France
                  (Martinique)}{}\ledrightnote{\textcolor{pink}{Fort-de-France}}, \textcolor{pink}{St. Pierre}{}\ledrightnote{\textcolor{pink}{Saint-Pierre}}, \textcolor{pink}{San Juan (Puerto Rico)}{}\ledrightnote{\textcolor{pink}{San Juan}}, \textcolor{pink}{Bermudas}{}\ledrightnote{\textcolor{pink}{Bermuda}}. Reisedauer 24 Tage. Fahrpreise von \textbf{K 600.—}
                  an aufwärts.}}\pend
           \pstart
           \textcolor{gray}{\textbf{\textbf{Große Orientreise}, per Dampfer »Moltke« ab \textcolor{pink}{Genua}{}\ledrightnote{\textcolor{pink}{Genua}}{ }\textbf{20. Februar 1905}. \textcolor{pink}{Genua}{}\ledrightnote{\textcolor{pink}{Genua}}, \textcolor{pink}{Villafranka}{}\ledrightnote{\textcolor{pink}{Villefranche-sur-Mer}} (\textcolor{pink}{Nizza}{}\ledrightnote{\textcolor{pink}{Nizza}}), \textcolor{pink}{Syrakus}{}\ledrightnote{\textcolor{pink}{Syrakus}}, \textcolor{pink}{Malta}{}\ledrightnote{\textcolor{pink}{Malta}}, \textcolor{pink}{Alexandria}{}\ledrightnote{\textcolor{pink}{Alexandria}} (\textcolor{pink}{Cairo}{}\ledrightnote{\textcolor{pink}{Kairo}} etc.), \textcolor{pink}{Beirut}{}\ledrightnote{\textcolor{pink}{Beirut}}, (\textcolor{pink}{Damascus}{}\ledrightnote{\textcolor{pink}{Damascus}}, \textcolor{pink}{Baalbec}{}\ledrightnote{\textcolor{pink}{Baalbek}}), \textcolor{pink}{Jaffa}{}\ledrightnote{\textcolor{pink}{Jaffa}}, (\textcolor{pink}{Jerusalem}{}\ledrightnote{\textcolor{pink}{Jerusalem}} etc.), \textcolor{pink}{Beirut}{}\ledrightnote{\textcolor{pink}{Beirut}}, \textcolor{pink}{Alexandria}{}\ledrightnote{\textcolor{pink}{Alexandria}}, \textcolor{pink}{Jaffa}{}\ledrightnote{\textcolor{pink}{Jaffa}}, \textcolor{pink}{Konstantinopel}{}\ledrightnote{\textcolor{pink}{Istanbul}}, \textcolor{pink}{Athen}{}\ledrightnote{\textcolor{pink}{Athen}}, \textcolor{pink}{Nauplia}{}\ledrightnote{\textcolor{pink}{Nafplion}} (\textcolor{pink}{Eleusis}{}\ledrightnote{\textcolor{pink}{Elefsina}}, \textcolor{pink}{Mykenae}{}\ledrightnote{\textcolor{pink}{Mycenae}}), \textcolor{pink}{Messina}{}\ledrightnote{\textcolor{pink}{Messina}}, \textcolor{pink}{Palermo}{}\ledrightnote{\textcolor{pink}{Palermo}}, \textcolor{pink}{Neapel}{}\ledrightnote{\textcolor{pink}{Neapel}}, (\textcolor{pink}{Vesuv}{}\ledrightnote{\textcolor{pink}{Vesuv}}, \textcolor{pink}{Pompeji}{}\ledrightnote{\textcolor{pink}{Pompei}}, \textcolor{pink}{Capri}{}\ledrightnote{\textcolor{pink}{Capri}}, \textcolor{pink}{Rom}{}\ledrightnote{\textcolor{pink}{Rom}}), \textcolor{pink}{Genua}{}\ledrightnote{\textcolor{pink}{Genua}}. Reisedauer 43 Tage.
                  Fahrpreise von \textbf{K 2520.—} an aufwärts.}}\pend
           {\bigskip}\pstart
           \noindent{}\centering{}\textcolor{gray}{\textbf{Nähere Auskünfte werden erteilt und Plätze reserviert im}}\pend
           \pstart
           \noindent{}\textcolor{gray}{\textbf{\textcolor{brown}{Internationalen Reise-Bureau Schenker {\kaufmannsund} Co.}{}\ledrightnote{\textcolor{brown}{Internationales Reise-Bureau Schenker}}{ }\textcolor{pink}{Wien, I. Schottenring Nr. 3}{}\ledrightnote{\textcolor{pink}{Schottenring}}.}}\pend
           \pstart
           \centering{}\textcolor{gray}{\textbf{18}}\pend
           \endnumbering\briefempfaengerindex{Beer-Hofmann, Richard@\textsc{Beer-Hofmann, Richard}!zzzSchnitzler, Arthur@\emph{von Arthur Schnitzler}!1905-01-261@{26. 1. 1905}|)be}\mylabel{h}  
         \normalsize

\newenvironment{esempio}[3]%
{
    \vspace{1.5ex}
    \rlap{\underline{#1}}
    \par
    \setlength{\parindent}{0cm}
    \nopagebreak
    \leftskip=#2cm
    \rightskip=#3cm
}
{
    \par
}

\doendnotes{C}
\bigskip

\printindex[pw]


\end{document}
      