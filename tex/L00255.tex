%% latex-korrekturansicht-vorspann.tex
%% Vorspann für die Korrekturansicht.
%% Lädt die gemeinsame Datei latex-vorspann.tex mit gesetztem Schalter.

\newif\ifkorrekturansicht
\korrekturansichttrue

\input{../tex-inputs/latex-vorspann}


               \section[Karl Kraus an Arthur Schnitzler, 12. 8. 1893]{ Karl Kraus an Arthur Schnitzler, 12. 8. 1893}\nopagebreak\mylabel{v}\rehead{ }\normalsize\beginnumbering\briefempfaengerindex{Schnitzler, Arthur@\textsc{Schnitzler, Arthur}!zzzKraus, Karl@\emph{von Karl Kraus}!1893-08-124@{12. 8. 1893}|(be} \toendnotes[C]{\smallbreak\pagebreak[2]} \Standort{CUL, Schnitzler, B 55.}
\physDesc{Brief, 1 Blatt, 4 Seiten
\newline{}Handschrift: schwarze Tinte, deutsche Kurrent}\buchAbdrucke{\weitereDrucke{\emph{Karl Kraus und Arthur Schnitzler. Eine Dokumentation.} Hg. Reinhard Urbach. In: \emph{Literatur und Kritik}, Bd. 49, Oktober 1970, S. 520.} }\toendnotes[C]{\smallbreak}\pstart
           {\pb}\textcolor{pink}{Ischl, Ramsauer}{}\ledrightnote{\textcolor{pink}{Ramsauer Garni Café}},
                        12. 8. 93.\pend
           \pstart
           Liebſter Doktor! Eben holte ich mir von der Post den Brief u.
                    beeile mich, Ihnen auf Ihr Schreiben zu antworten: ich bin über die Auskunft des
                    Herrn \textcolor{blue}{Entſch}{}\ledrightnote{\textcolor{blue}{Theodor Entsch}} ganz paff – es iſt mir \uline{nie im Traume eingefallen}, dem \textcolor{brown}{Magazin}{}\ledrightnote{\textcolor{brown}{Magazin für die Literatur des Auslandes}} eine derartige aus der Luft gegriffene \label{K_L00255_1v}\edtext{\textcolor{green}{Mittheilung}{}\ledrightnote{→\textcolor{green}{[Meldung: Märchen im Lessingtheater]}}}{\lemma{\textnormal{\emph{Mittheilung}}}\Cendnote{\textnormal{Auf S. 469 der Nr. 29 vom
                            22. 7. 1893{ }stand: »Am \textcolor{brown}{Lessingtheater} kommen ferner noch im Laufe des
                            Sommers ein Drama von \textcolor{blue}{\so{Fedor von Zobeltitz}}: ›\textcolor{green}{Ohne Geläut}‹ und ein
                            dreiaktiges Schauspiel von \textsc{Dr}. \textcolor{blue}{Arthur \so{Schnitzler}} in \textcolor{pink}{Wien}: ›\textcolor{green}{Das Märchen}‹, zur Aufführung.«}}}\label{K_L00255_1h} zu
                    machen – das wäre dann eine höchſt abgeſchmackte Fopperei \introOben{}von\introOben{} meiner Seite geweſen, wenn ich Ihnen dann »freudig \uline{überraſcht}« das Blatt ſchicken konnte: »Sehen
                    Sie, da ſteht was über das ›\textcolor{green}{Märchen}{}\ledrightnote{\textcolor{green}{Das Märchen. Schauspiel in drei Aufzügen}}‹ drin!«
                    Wie geſagt, liebſter Herr Doktor, \uuline{nie und nimmer}
                    würde mir ſoetwas einfallen, ich habe \uuline{nie} (Sie
                    wiſſen ja, bei {\pb}\textcolor{green}{Abſchiedssouper}{}\ledrightnote{\textcolor{green}{Abschiedssouper}} habe ich Sie zu erst brieflich
                    befragt) Herrn \textcolor{blue}{Neumann-Hofer}{}\ledrightnote{\textcolor{blue}{Gilbert Otto Neumann-Hofer}} den
                    Aufführungstermin Ihres \textcolor{green}{Märchen}{}\ledrightnote{\textcolor{green}{Das Märchen. Schauspiel in drei Aufzügen}} geſchrieben:
                    das wäre doch meinerſeits eine recht ungeſchickte Reklame für Sie geweſen. Das
                    Ganze muſs unbedingt auf einem \uline{Irrthum} beruhen,
                    vielleicht erklärt es ſich daraus, daſs ich einmal – Sie haben’s ja geleſen – im
                        \textcolor{green}{Magazin}{}\ledrightnote{\textcolor{green}{Magazin für die Literatur des Auslandes}} gelegentlich der \textcolor{green}{\textcolor{green}{Anatol}{}\ledrightnote{\textcolor{green}{Anatol}}-recenſion}{}\ledrightnote{→\textcolor{green}{Wiener Dichter}} auch Ihr \textcolor{green}{\uline{Märchen}}{}\ledrightnote{\textcolor{green}{Das Märchen. Schauspiel in drei Aufzügen}} als beachtenswertes Schauspiel erwähnte.\pend
           \pstart
           Mir iſt die ganze Sache \uuline{ſehr peinlich}, glauben
                    Sie mir! {\pb}Jawohl, wenn Sie mir ſelbſt
                    den \strikeout{I} Inhalt dieſer vielbeſprochenen \textcolor{green}{\textcolor{green}{Märchen}{}\ledrightnote{\textcolor{green}{Das Märchen. Schauspiel in drei Aufzügen}}notiz}{}\ledrightnote{→\textcolor{green}{[Meldung: Märchen im Lessingtheater]}} geſagt hätten, mit
                    Vergnügen \uuline{hätte} ich, um Ihnen zu dienen, dem \textcolor{brown}{Magazin}{}\ledrightnote{\textcolor{brown}{Magazin für die Literatur des Auslandes}} die Notiz mitgetheilt – aber ſo – wie
                    werde ich ſo plump ſein, ſo etwas aus der Luft zu greifen oder aus dem Finger zu
                    zutzeln und dann Ihnen das Heft mit »freudig–überraschter« Miene noch zu\introOben{}zu\introOben{}senden? Ich bitte Sie, mir nicht böſe zu ſein, daſs
                    ich Ihnen (\uuline{unverſchuldet}!) ſolche
                    Unannehmlichkeiten bereite – aber mich ſelbſt {\pb}berührt die Angelegenheit noch \uuline{viel} unangenehmer. \uuline{Selbſtverſtändlich}{ }ſchreibe ich ſofort dem \textcolor{brown}{Magazin}{}\ledrightnote{\textcolor{brown}{Magazin für die Literatur des Auslandes}} u. erſuche um Aufklärung. Der \introOben{}\textcolor{blue}{Entſch}{}\ledrightnote{\textcolor{blue}{Theodor Entsch}}\introOben{}brief liegt bei. Ich bin mit den herzlichſten Grüßen Ihr\pend
           \pstart \spacefill\mbox{KarlKraus.}\pend{}\pstart
           \noindent{}\uuline{NB.} um von freundlicheren Sachen zu
                        ſprechen: \textcolor{blue}{Beer Hofmann}{}\ledrightnote{\textcolor{blue}{Richard Beer-Hofmann}}s »\textcolor{green}{Kind}{}\ledrightnote{→\textcolor{green}{Das Kind}}« iſt ein prächtiger, geſunder
                        Bengel. Der grauſame Vater will es – \label{K_L00255_2v}\edtext{verlegen}{\lemma{\textnormal{\emph{verlegen}}}\Cendnote{\textnormal{\textcolor{blue}{Richard Beer-Hofmann}: \emph{\textcolor{green}{Novellen}}. Berlin:
                                    \emph{Freund {\kaufmannsund}
                                    Jeckel}{ }1893. Enthält: \emph{\textcolor{green}{Das Kind}} und \emph{\textcolor{green}{Camelias}}. Erschienen Anfang
                                Dezember 1893.}}}\label{K_L00255_2h} laſſen.\pend
           \endnumbering\briefempfaengerindex{Schnitzler, Arthur@\textsc{Schnitzler, Arthur}!zzzKraus, Karl@\emph{von Karl Kraus}!1893-08-124@{12. 8. 1893}|)be}\mylabel{h}  \normalsize

\doendnotes{C}
\bigskip
\vfill

\clearpage

\footnotesize

\lohead{\textsc{register}}

% Definiere theindex-Environment komplett neu ohne reledmac
\makeatletter
\renewenvironment{theindex}{%
  \section*{\indexname}%
  \setlength{\parindent}{0pt}%
  \setlength{\parskip}{0pt plus 0.3pt}%
  \let\item\@idxitem
}{%
  \clearpage
}
\makeatother

\IfFileExists{\jobname-pw.ind}{\input{\jobname-pw.ind}}{}

\end{document}

      