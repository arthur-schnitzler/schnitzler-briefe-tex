%% latex-korrekturansicht-vorspann.tex
%% Vorspann für die Korrekturansicht.
%% Lädt die gemeinsame Datei latex-vorspann.tex mit gesetztem Schalter.

\newif\ifkorrekturansicht
\korrekturansichttrue

\input{../tex-inputs/latex-vorspann}


               \section[Arthur Schnitzler an Richard Beer-Hofmann, 20. 10. 1894]{ Arthur Schnitzler an Richard Beer-Hofmann, 20. 10. 1894}\nopagebreak\mylabel{v}\rehead{ }\normalsize\beginnumbering\briefempfaengerindex{Beer-Hofmann, Richard@\textsc{Beer-Hofmann, Richard}!zzzSchnitzler, Arthur@\emph{von Arthur Schnitzler}!1894-10-201@{20. 10. 1894}|(be} \toendnotes[C]{\smallbreak\pagebreak[2]} \Standort{YCGL, MSS 31.}
\physDesc{Brief, 3 Blätter, 12 Seiten, Umschlag
\newline{}Handschrift: Bleistift, deutsche Kurrent\newline{}Versand: 1) Stempel: »\nobreak{}\oindex{I., Innere Stadt@\textbf{I., Innere Stadt}, \emph{Bezirk (A.BZK)}|pwk}Wien 1/1, 20. 10. 94, \textcolor{gray}{7-8}N\nobreak{}«.  2) Stempel: »\nobreak{}\oindex{Neapel@\textbf{Neapel}, \emph{Besiedelter Ort (A.BSO)}|pwk}\textcolor{gray}{Nap}o\textcolor{gray}{l}i, \textcolor{gray}{23} 10-94, 3 S\nobreak{}«. }\buchAbdrucke{\weitereDrucke{1) Arthur Schnitzler: \emph{Briefe 1875–1912}. Hg. Therese Nickl und Heinrich Schnitzler. Frankfurt am Main: \emph{S. Fischer} 1981, S. 232–233.} \weitereDrucke{2) Arthur Schnitzler, Richard Beer-Hofmann: \emph{Briefwechsel 1891–1931}. Hg. Konstanze Fliedl. Wien, Zürich: \emph{Europaverlag} 1992, S. 66–67.} \weitereDrucke{3) Arthur Schnitzler: \emph{Briefe.} In: \emph{Die Neue Rundschau}, Bd. 68 (1957) Nr. 1, S. 88–89.} \weitereDrucke{4) Hermann Bahr, Arthur Schnitzler: \emph{Briefwechsel, Aufzeichnungen, Dokumente (1891–1931)}. Hg. Kurt Ifkovits und Martin Anton Müller. Göttingen: \emph{Wallstein} 2018.} }\toendnotes[C]{\smallbreak}\pstart{}{\pb}\textsc{Dr. Arthur Schnitzler}, \textcolor{pink}{Wien,
                     IX. Frankgaſſe 1.}{}\ledrightnote{\textcolor{pink}{Frankgasse}}\pend{}{\bigskip}\pstart{}{\pb}\textsc{\textcolor{pink}{Italien}{}\ledrightnote{\textcolor{pink}{Italien}}}\pend{}\pstart{}\textsc{Dr. Richard Beer Hofmann}\pend{}\pstart{}\textsc{\textcolor{pink}{Neapel}{}\ledrightnote{\textcolor{pink}{Neapel}}}\pend{}\pstart{}\textsc{\textcolor{pink}{Hotel Hassler}{}\ledrightnote{\textcolor{pink}{Hôtel Hassler}}}\pend{}{\bigskip}\pstart
           \raggedleft{}{\pb}20. 10. 94\pend
           \pstart{}Lieber Richard. –\pend\pstart
           \textcolor{green}{Schmetterlingsſchlacht}{}\ledrightnote{\textcolor{green}{Die Schmetterlingsschlacht}}: Erſter Akt ſehr gut, voll
               glänzenden, nur zuweilen etwas abſichtlichen Details;– machte erwartungsvolle
               treffliche Sti{\geminationm}ung. Zweiter Akt läßt ſich nicht übel an;
               befremdet bereits durch einige Trivialitäten – enttäuſcht aber noch nicht recht. Der
               dritte Akt {\pb}ſchwach, ungeſchickt, ohne ſelbſt den
               ſtofflichen Inhalt, der in ihm ſteckt, auszuſchöpfen; verſti{\geminationm}end, mit einem affectirten, pſychologiſch falſchen,
               enervirenden Schluſs. Der letzte Akt kurzweg kläglich, geradezu erbitternd. – \textcolor{blue}{Suderma{\geminationn}}{}\ledrightnote{\textcolor{blue}{Hermann Sudermann}}{ }ſcheint doch nur der große Meiſter der erſten Akte
               zu ſein. – (\textcolor{green}{Ehre}{}\ledrightnote{\textcolor{green}{Die Ehre}}, \textcolor{green}{Sodom}{}\ledrightnote{\textcolor{green}{Sodom’s Ende}}, \textcolor{green}{Heimath}{}\ledrightnote{\textcolor{green}{Heimat}} – {\pb}überall der erſte Akt am beſten.) – Einige Figuren der
                  \textcolor{green}{Schmett.}{}\ledrightnote{\textcolor{green}{Die Schmetterlingsschlacht}} famos, andre unerlaubt läppiſch. Das
               ganze Stück nicht einer glücklichen Eingebung entſta{\geminationm}end, ſondern recht mühſelig und ohne Glück conſtruirt. Das ärgſte war zu vermeiden,
                  we{\geminationn} 3. u 4. Akt zu einem zusa{\geminationm}enge{\pb}zogen werden und
               die Rolle der naiven \textcolor{green}{Roſi}{}\ledrightnote{→\textcolor{green}{Die Schmetterlingsschlacht}} aus der
               gemeinen Theaterſchablone ins menſchliche hinaufgehoben wird. Die Darſtellung ist
               großartig; ſie lügt geradezu Seelen in die Puppen. – Um die \textcolor{green}{\textsc{Schm}.}{}\ledrightnote{\textcolor{green}{Die Schmetterlingsschlacht}} für \textcolor{blue}{Sud.’s}{}\ledrightnote{\textcolor{blue}{Hermann Sudermann}} beſtes Stück zu halten, muß man entweder nichts verſtehn – oder \textcolor{blue}{\textsc{Herma{\geminationn}{ }{\pb}Bahr}}{}\ledrightnote{\textcolor{blue}{Hermann Bahr}}{ }ſein. Ueber ſeine \label{K_L00387_1v}\edtext{\textcolor{green}{Kritik}{}\ledrightnote{→\textcolor{green}{Burgtheater (»Die Schmetterlingsschlacht«. Komödie in vier Akten von Hermann Sudermann. Zum ersten Mal aufgeführt am 6. October 1894)}}}{\lemma{\textnormal{\emph{Kritik}}}\Cendnote{\textnormal{\textcolor{blue}{Hermann Bahr}: \emph{\textcolor{green}{Burgtheater (»Die Schmetterlingsschlacht«. Komödie in vier Akten von
                        Hermann Sudermann. Zum ersten Mal aufgeführt am 6. October
                        1894)}}. In: \emph{\textcolor{green}{Die Zeit}}, Bd. 1,
                     H. 2, 13. 10. 1894, S. 26.}}}\label{K_L00387_1h} und noch vieles andre hab
               ich geſtern erſt zwei Stunden mit ihm geplauſcht. Ich zweifle gar nicht: er will
               immer intereſſant, i{\geminationm}er geiſtvoll, i{\geminationm}er bizarr ſein, und es gelingt ihm faſt i{\geminationm}er – aber we{\geminationn}\substVorne{}\textsuperscript{seine}\substDazwischen{}die\substHinten{} Originalität {\pb}und die Bizarrerie – ja ſagen
               wir zuweilen ſelbſt die Tiefe ſeiner künſtleriſchen Anſchauungen mit der Wahrheit
                  zuſa{\geminationm}enfällt, ſo iſt das gewiſs mehr Zufall als der
               ſchöne Drang nach kritiſcher Ehrlichkeit. Und was könnte dieſer Menſch nicht {\pb}leiſten, wenn er zu ſeinen außerordentlichen
               Eigenſchaften auch noch die der Verläßlichkeit hätte. Er iſt einer von den glänzenden
               – aber nicht einer von den Echten. –\pend
           \pstart
           Heut geh ich zur \textsc{Première} von den \textcolor{green}{Komödianten}{}\ledrightnote{\textcolor{green}{Comödianten}}. Haben Sie auch in \textsc{theatralibus} was {\pb}geſehen? Gehn Sie nach \textcolor{pink}{\textsc{Sicilien}}{}\ledrightnote{\textcolor{pink}{Sizilien}}? –\pend
           \pstart
           Heute holt der \textcolor{blue}{Abſchreiber}{}\ledrightnote{→\textcolor{blue}{?? [Schreibkraft für Arthur Schnitzler]}}
               meinen letzten \textcolor{green}{Akt}{}\ledrightnote{→\textcolor{green}{Liebelei. Schauspiel in drei Akten}}. In acht Tagen
               hoff’ ichs einreichen zu können. – \strikeout{Auch}{ }\textcolor{blue}{\textsc{Hugo}}{}\ledrightnote{\textcolor{blue}{Hugo von Hofmannsthal}} und \textcolor{blue}{Salten}{}\ledrightnote{\textcolor{blue}{Felix Salten}} finden: \textcolor{pink}{Burgtheater}{}\ledrightnote{\textcolor{pink}{Burgtheater}}. \textcolor{blue}{\textsc{Bahr}}{}\ledrightnote{\textcolor{blue}{Hermann Bahr}} hat auch ſchon mit \textcolor{blue}{\textsc{Burckh}}{}\ledrightnote{\textcolor{blue}{Max Eugen Burckhard}}. geſprochen und \textcolor{blue}{Burckh}{}\ledrightnote{\textcolor{blue}{Max Eugen Burckhard}}. {\pb}»erwartet« das \textcolor{green}{Stück}{}\ledrightnote{→\textcolor{green}{Liebelei. Schauspiel in drei Akten}}. Charakteriſtiſch übrigens, daſs \textcolor{blue}{Bahr}{}\ledrightnote{\textcolor{blue}{Hermann Bahr}}, nachdem er mit \textcolor{blue}{\textsc{Burckh}}{}\ledrightnote{\textcolor{blue}{Max Eugen Burckhard}} geſprochen und nachdem er von dem \textcolor{green}{Stück}{}\ledrightnote{→\textcolor{green}{Liebelei. Schauspiel in drei Akten}} nichts wußte als, was ihm \textcolor{blue}{Hugo}{}\ledrightnote{\textcolor{blue}{Hugo von Hofmannsthal}} geſagt, daſs es ſehr gut und »\textcolor{pink}{Burgtheater}{}\ledrightnote{\textcolor{pink}{Burgtheater}}« ſei, mir gegenüber äußerte: {\pb}»Ich hab’ die Empfindung, daſs es ins \textcolor{pink}{Raimundtheater}{}\ledrightnote{\textcolor{pink}{Raimund-Theater}} gehört.« – Man ka{\geminationn} übrigens
               weniger als je ans \textcolor{pink}{Raimundth}{}\ledrightnote{\textcolor{pink}{Raimund-Theater}}. denken – es wird dort
               geſpielt wie an einem Provinztheater, wo die Leut eben zehn Proben haben, {\pb}ſtatt einer oder zwei. Aber dadurch kriegen die Herren
                  \textcolor{blue}{Heding}{}\ledrightnote{\textcolor{blue}{Edmund Heding}} und \textcolor{blue}{Nerz}{}\ledrightnote{\textcolor{blue}{Ludwig Nerz}} u. ſ. w. nicht mehr Talent als ſie haben. – \textcolor{pink}{Burgtheater}{}\ledrightnote{\textcolor{pink}{Burgtheater}}verſuch muſs natürlich ſtrenges Geheimnis bleiben, da ich ja
               dann, we{\geminationn}{ }\textcolor{blue}{B.}{}\ledrightnote{\textcolor{blue}{Max Eugen Burckhard}} es reſusirt {\pb}beim \textcolor{pink}{Volkstheater}{}\ledrightnote{\textcolor{pink}{Volkstheater}} einreichen will. – \pend
           \pstart
           Ich freue mich auf Ihre Rückkehr. – \pend
           \pstart
           Herzlichen Gruſs{\\[\baselineskip]}Ihr \spacefill\mbox{Arthur}\pend
           \leftskip=0em{}\endnumbering\briefempfaengerindex{Beer-Hofmann, Richard@\textsc{Beer-Hofmann, Richard}!zzzSchnitzler, Arthur@\emph{von Arthur Schnitzler}!1894-10-201@{20. 10. 1894}|)be}\mylabel{h}  \normalsize

\doendnotes{C}
\bigskip
\vfill

\clearpage

\footnotesize

\lohead{\textsc{register}}

% Definiere theindex-Environment komplett neu ohne reledmac
\makeatletter
\renewenvironment{theindex}{%
  \section*{\indexname}%
  \setlength{\parindent}{0pt}%
  \setlength{\parskip}{0pt plus 0.3pt}%
  \let\item\@idxitem
}{%
  \clearpage
}
\makeatother

\IfFileExists{\jobname-pw.ind}{\input{\jobname-pw.ind}}{}

\end{document}

      