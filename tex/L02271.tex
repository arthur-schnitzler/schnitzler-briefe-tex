%% latex-korrekturansicht-vorspann.tex
%% Vorspann für die Korrekturansicht.
%% Lädt die gemeinsame Datei latex-vorspann.tex mit gesetztem Schalter.

\newif\ifkorrekturansicht
\korrekturansichttrue

\input{../tex-inputs/latex-vorspann}


               \section[Arthur Schnitzler an Richard Beer-Hofmann, 1. 9. 1917]{ Arthur Schnitzler an Richard Beer-Hofmann, 1. 9. 1917}\nopagebreak\mylabel{v}\rehead{ }\normalsize\beginnumbering\briefempfaengerindex{Beer-Hofmann, Richard@\textsc{Beer-Hofmann, Richard}!zzzSchnitzler, Arthur@\emph{von Arthur Schnitzler}!1917-09-011@{1. 9. 1917}|(be} \toendnotes[C]{\smallbreak\pagebreak[2]} \Standort{YCGL, MSS 31.}
\physDesc{Postkarte
\newline{}Handschrift: Bleistift, deutsche Kurrent\newline{}Versand: Stempel: »\nobreak{}Wien, 2. IX. 17, X\nobreak{}«.  
\newline{}Beer-Hofmann: mit blauem Buntstift das Datum der Beantwortung
                                    vermerkt: »B. 4./IX 17« }\toendnotes[C]{\smallbreak}\pstart{}{\pb}\textsc{Dr Schnitzler \textcolor{pink}{Wien XVIII}{}\ledrightnote{\textcolor{pink}{XVIII., Währing}}}\pend{}\pstart{}\textsc{\textcolor{pink}{Sternwartestr 71}{}\ledrightnote{\textcolor{pink}{Sternwartestraße}}.}\pend{}{\bigskip}\pstart{}\textsc{Herrn Dr. Richard Beer-Hofmann}\pend{}\pstart{}\textsc{\textcolor{pink}{Bad Ischl}{}\ledrightnote{\textcolor{pink}{Bad Ischl}}}\pend{}\pstart{}\textsc{\textcolor{pink}{Grazerstr 56}{}\ledrightnote{\textcolor{pink}{Grazer Straße}}}\pend{}{\bigskip}\pstart
           \raggedleft{}{\pb}1. 9. 1917\pend
           \pstart
           lieber Richard, \textcolor{blue}{Gustav}{}\ledrightnote{\textcolor{blue}{Gustav Schwarzkopf}} erzählt mir dſs Sie vielleicht nach \textcolor{pink}{Berlin}{}\ledrightnote{\textcolor{pink}{Berlin}} fahren. Bitte ſchreiben Sie mir nach \textcolor{pink}{Partenkirchen}{}\ledrightnote{\textcolor{pink}{Partenkirchen}} (\textcolor{pink}{Haus
                  Tannenberg}{}\ledrightnote{\textcolor{pink}{Haus Tannenberg}}) wohin ich Montag meiner \textcolor{blue}{Frau}{}\ledrightnote{→\textcolor{blue}{Olga Schnitzler}} nachreiſe, wann Sie etwa in \textcolor{pink}{Berlin}{}\ledrightnote{\textcolor{pink}{Berlin}} sein dürften.\pend
           \pstart
           Haben Sie meinen letzten (zweiten) {\pb}\label{K_L02271-1v}\edtext{Brief}{\lemma{\textnormal{\emph{Brief}}}\Cendnote{\textnormal{siehe Arthur Schnitzler an Richard Beer-Hofmann, 23. 7. 1917}}}\label{K_L02271-1h} erhalten?\pend
           \pstart
           Herzliche Grüße Ihnen und den Ihrigen{\\[\baselineskip]}Ihr{\\[\baselineskip]}\spacefill\mbox{Arth}\pend
           \leftskip=0em{}\endnumbering\briefempfaengerindex{Beer-Hofmann, Richard@\textsc{Beer-Hofmann, Richard}!zzzSchnitzler, Arthur@\emph{von Arthur Schnitzler}!1917-09-011@{1. 9. 1917}|)be}\mylabel{h}  \normalsize

\doendnotes{C}
\bigskip
\vfill

\clearpage

\footnotesize

\lohead{\textsc{register}}

% Definiere theindex-Environment komplett neu ohne reledmac
\makeatletter
\renewenvironment{theindex}{%
  \section*{\indexname}%
  \setlength{\parindent}{0pt}%
  \setlength{\parskip}{0pt plus 0.3pt}%
  \let\item\@idxitem
}{%
  \clearpage
}
\makeatother

\IfFileExists{\jobname-pw.ind}{\input{\jobname-pw.ind}}{}

\end{document}

      