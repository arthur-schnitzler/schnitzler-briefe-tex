%% latex-korrekturansicht-vorspann.tex
%% Vorspann für die Korrekturansicht.
%% Lädt die gemeinsame Datei latex-vorspann.tex mit gesetztem Schalter.

\newif\ifkorrekturansicht
\korrekturansichttrue

\input{../tex-inputs/latex-vorspann}


               \section[Peter Altenberg an Arthur Schnitzler, {[}10. 10. 1895{]}]{ Peter Altenberg an Arthur Schnitzler, {[}10. 10. 1895{]}}\nopagebreak\mylabel{v}\rehead{ }\normalsize\beginnumbering\briefempfaengerindex{Schnitzler, Arthur@\textsc{Schnitzler, Arthur}!zzzAltenberg, Peter@\emph{von Peter Altenberg}!1895-10-102@{{[}10. 10. 1895{]}}|(be} \toendnotes[C]{\smallbreak\pagebreak[2]} \Standort{CUL, Schnitzler, B 2.}
\physDesc{Brief, 1 Blatt, 1 Seite
\newline{}Handschrift: schwarze Tinte, deutsche Kurrent
\newline{}Schnitzler: mit Bleistift datiert: »Oct. 95« und nummeriert: »5« \newline{}Ordnung: mit Bleistift von unbekannter Hand nummeriert:
                                 »4« }\buchAbdrucke{\weitereDrucke{Peter Altenberg: \emph{Die Selbsterfindung eines Dichters. Briefe und Dokumente
                        1892–1896}. Hg. und mit einem Nachwort von Leo A. Lensing. Göttingen: \emph{Wallstein} 2009, S. 38.} }\toendnotes[C]{\smallbreak}\pstart{}{\pb}Lieber Arthur Schnitzler.\pend\pstart
           Nehme herzlich Theil an ihrem Erfolge. Habe mit Spannung die Morgenblätter von
                  \label{K_L00503_1v}\edtext{heute Donnerſtag}{\lemma{\textnormal{\emph{heute Donnerſtag}}}\Cendnote{\textnormal{Zusammen mit der Datierung \textcolor{blue}{Schnitzler}s auf »Oct. 95« lässt sich als
                  Datum für diesen Brief der 10. 10. 1895 ermitteln, der Tag nach der
                  Uraufführung der \emph{\textcolor{green}{Liebelei}}.}}}\label{K_L00503_1h} (3 Uhr
                  Nachmittag) erwartet.\pend
           \pstart
           Hier iſt herrliche dicke Ruhe, Herbſt-Friede. Schreiben Sie mir doch einmal.\pend
           \pstart
           Ich leſe »\textcolor{green}{\textsc{en route}}{}\ledrightnote{\textcolor{green}{En route}}« von \textcolor{blue}{\textsc{Huysmans}}{}\ledrightnote{\textcolor{blue}{Joris-Karl Huysmans}}.\pend
           \pstart
           Sie haben hoffentlich die \textsc{C}\label{T_L00503_1v}\edtext{..........}{\lemma{\textnormal{\emph{XXXX Lemmafehler}}}\Cendnote{\textnormal{Von \textcolor{blue}{Schnitzler} wurden die fehlenden Buchstaben mit Bleistift in lateinischer Schreibschrift
                  ergänzt: »\textsc{igaretten}«, wobei hier die Schrift darauf
                  hindeutet, dass diese Ergänzung erst nach dem Wechsel seiner Schreibschrift und
                  mithin erst nach 1906 anzusetzen ist.}}}\label{T_L00503_1h} unter »Baumwollwaare«
               vom 16./8 erhalten?!\pend
           \pstart
           Adieu, ihr{\\[\baselineskip]}\spacefill\mbox{Richard Engländer.}\pend
           \leftskip=0em{}\pstart
           \noindent{}\textcolor{pink}{Goldener Brunnen}{}\ledrightnote{\textcolor{pink}{Goldener Brunnen}}.\pend
           \endnumbering\briefempfaengerindex{Schnitzler, Arthur@\textsc{Schnitzler, Arthur}!zzzAltenberg, Peter@\emph{von Peter Altenberg}!1895-10-102@{{[}10. 10. 1895{]}}|)be}\mylabel{h}  \normalsize

\doendnotes{C}
\bigskip
\vfill

\clearpage

\footnotesize

\lohead{\textsc{register}}

% Definiere theindex-Environment komplett neu ohne reledmac
\makeatletter
\renewenvironment{theindex}{%
  \section*{\indexname}%
  \setlength{\parindent}{0pt}%
  \setlength{\parskip}{0pt plus 0.3pt}%
  \let\item\@idxitem
}{%
  \clearpage
}
\makeatother

\IfFileExists{\jobname-pw.ind}{\input{\jobname-pw.ind}}{}

\end{document}

      