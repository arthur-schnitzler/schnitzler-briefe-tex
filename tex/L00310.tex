%% latex-korrekturansicht-vorspann.tex
%% Vorspann für die Korrekturansicht.
%% Lädt die gemeinsame Datei latex-vorspann.tex mit gesetztem Schalter.

\newif\ifkorrekturansicht
\korrekturansichttrue

\input{../tex-inputs/latex-vorspann}


               \section[Hermann Bahr an Arthur Schnitzler, 2. 4. 1894]{ Hermann Bahr an Arthur Schnitzler, 2. 4. 1894}\nopagebreak\mylabel{v}\rehead{ }\normalsize\beginnumbering\briefempfaengerindex{Schnitzler, Arthur@\textsc{Schnitzler, Arthur}!zzzBahr, Hermann@\emph{von Hermann Bahr}!1894-04-022@{2. 4. 1894}|(be} \toendnotes[C]{\smallbreak\pagebreak[2]} \Standort{TMW, HS AM 39930 Ba.}
\physDesc{maschinelle Abschrift
\newline{}Schreibmaschine\newline{}Zusatz: Original nicht nachweisbar; es wurde von \textcolor{blue}{Heinrich Schnitzler} am
                                    22. 8. 1937 dem \textcolor{pink}{Wolf-Museum} in \textcolor{pink}{Eisenstadt} geschenkt. \textcolor{blue}{Sándor
                                    Wolf} emigrierte 1938 nach \textcolor{pink}{Israel}, wohin seine Bibliothek nachzuholen ihm
                                 möglicherweise gelang. Nach seinem Tod im Jahr 1946
                                 ließ seine Schwester \textcolor{blue}{Frieda
                                    Löwy} einen Teil der Sammlung 1958 in \textcolor{pink}{Luzern} versteigern, der Brief
                                 dürfte sich nicht darunter befunden haben. }\buchAbdrucke{\weitereDrucke{Hermann Bahr, Arthur Schnitzler: \emph{Briefwechsel, Aufzeichnungen, Dokumente (1891–1931)}. Hg. Kurt Ifkovits und Martin Anton Müller. Göttingen: \emph{Wallstein} 2018, S. 68.} }\pstart
           \raggedleft{}{\pb}2. 4. 1894\pend
           \pstart{}Lieber Schnitzler,\pend\pstart
           ich habe mir die Geschichte mit dem Bicycle doch anders überlegt – lieber nicht. Der
               Gedanke, da umständlich zu lernen und mich mit einem fremden Instrument zu peinigen,
               macht mich nur nervöse. Sei deswegen nicht böse \pend
           \pstart
           Deinem treuen{\\[\baselineskip]}\spacefill\mbox{Bahr}\pend
           \leftskip=0em{}\endnumbering\briefempfaengerindex{Schnitzler, Arthur@\textsc{Schnitzler, Arthur}!zzzBahr, Hermann@\emph{von Hermann Bahr}!1894-04-022@{2. 4. 1894}|)be}\mylabel{h}  \normalsize

\doendnotes{C}
\bigskip
\vfill

\clearpage

\footnotesize

\lohead{\textsc{register}}

% Definiere theindex-Environment komplett neu ohne reledmac
\makeatletter
\renewenvironment{theindex}{%
  \section*{\indexname}%
  \setlength{\parindent}{0pt}%
  \setlength{\parskip}{0pt plus 0.3pt}%
  \let\item\@idxitem
}{%
  \clearpage
}
\makeatother

\IfFileExists{\jobname-pw.ind}{\input{\jobname-pw.ind}}{}

\end{document}

      