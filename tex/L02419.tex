%% latex-korrekturansicht-vorspann.tex
%% Vorspann für die Korrekturansicht.
%% Lädt die gemeinsame Datei latex-vorspann.tex mit gesetztem Schalter.

\newif\ifkorrekturansicht
\korrekturansichttrue

\input{../tex-inputs/latex-vorspann}


               \section[Arthur Schnitzler an Hugo Hofmannsthal, {[}5?.{]} 11. 1924]{ Arthur Schnitzler an Hugo Hofmannsthal,
               {[}5?.{]} 11. 1924}\nopagebreak\mylabel{v}\rehead{ }\normalsize\beginnumbering\briefempfaengerindex{Hofmannsthal, Hugo von@\textsc{Hofmannsthal, Hugo von}!zzzSchnitzler, Arthur@\emph{von Arthur Schnitzler}!1924-11-051@{{[}5?.{]} 11. 1924}|(be} \toendnotes[C]{\smallbreak\pagebreak[2]} \Standort{FDH, Hs-30885,151.}
\physDesc{Postkarte
\newline{}Handschrift: schwarze Tinte, lateinische Kurrent\newline{}Versand: Stempel: »\nobreak{}\oindex{XVIII., Waehring@\textbf{XVIII., Währing}, \emph{Bezirk (A.BZK)}|pwk}\textcolor{gray}{18/1} Wien, \textcolor{gray}{5} XI 24, 6\nobreak{}«.  }\buchAbdrucke{\weitereDrucke{Hugo von Hofmannsthal, Arthur Schnitzler: \emph{Briefwechsel}. Hg. Therese Nickl und Heinrich Schnitzler. Frankfurt am Main: \emph{S. Fischer} 1964, S. 300.} }\toendnotes[C]{\smallbreak}\pstart{}{\pb}\label{T_L02419-1v}\edtext{\textcolor{gray}{\textbf{A. S.}}}{\lemma{\textnormal{\emph{A. S.}}}\Cendnote{\textnormal{ovaler Absenderkleber}}}\label{T_L02419-1h}\pend{}\pstart{}\textcolor{pink}{\textcolor{gray}{\textbf{WIEN, XVIII.}}}{}\ledrightnote{\textcolor{pink}{XVIII., Währing}}\pend{}\pstart{}\textcolor{pink}{\textcolor{gray}{\textbf{STERNWARTESTR. 71}}}{}\ledrightnote{\textcolor{pink}{Sternwartestraße}}\pend{}{\bigskip}\pstart{}an Hr Hugo v Hofma{\geminationn}sthal\pend{}\pstart{}\textcolor{pink}{Bad Aussee}{}\ledrightnote{\textcolor{pink}{Bad Aussee}}\pend{}\pstart{}\textcolor{pink}{Steiermark}{}\ledrightnote{\textcolor{pink}{Steiermark}}.\pend{}{\bigskip}\pstart
           \raggedleft{}{\pb}\textcolor{pink}{Wien}{}\ledrightnote{\textcolor{pink}{Wien}}, 6. 11. 24\pend
           \pstart
           mein lieber Hugo – schönen Dank für Ihren Gruſs aus \textcolor{pink}{Aussee}{}\ledrightnote{\textcolor{pink}{Bad Aussee}}. Über das \textcolor{green}{Frl. Else}{}\ledrightnote{\textcolor{green}{Fräulein Else}} hör
               ich und les ich von allen Seiten so viel gutes, dſs ich sie im ganzen beinah
               überschätzt finden muſs – ebenso wie die \textcolor{green}{K. d. V.}{}\ledrightnote{\textcolor{green}{Komödie der Verführung. In drei Akten}} –
                  we{\geminationn} auch vielfach gewürdigt, – doch noch in höherm
               Maſs (und nicht immer reinen Herzens) misverstanden. Nun es ist das alte Lied – wir
               müssen es alle singen. Ich freue mich, dſs Ihr Stück vollendet ist. Wohl »\textcolor{green}{Der Thurm}{}\ledrightnote{\textcolor{green}{Der Turm. Ein Trauerspiel}}«? Und die neue Arbeit –? Wa{\geminationn} werden Sie vorlesen? Wa{\geminationn}
               kommen Sie nach \textcolor{pink}{Wien}{}\ledrightnote{\textcolor{pink}{Wien}}? Was haben Sie für Winterpläne?
               – Ich bleibe wohl vorläufig hier; im Jänner{ }soll {\pb}ich in der \textcolor{pink}{Schweiz}{}\ledrightnote{\textcolor{pink}{Schweiz}} lesen, – was ich hauptsächlich thun will,
               um mir eine \textcolor{pink}{Engadin}{}\ledrightnote{\textcolor{pink}{Engadin}}er Schnee- u Sonnenwoche \strikeout{\textcolor{gray}{ver}} »mit gutem Gewissen« vergönnen
               zu dürfen. – Ich dictire novellistisch und arbeite vorwiegend
               aphoristisch-fragmentistisch. Schreiben Sie bald wieder, und wärs nur ein Wort! Es
               ist so schön, von Ihnen was direct zu wissen! \pend
           \pstart
           Alles Herzliche. Ihr \spacefill\mbox{A.}\pend
           \endnumbering\briefempfaengerindex{Hofmannsthal, Hugo von@\textsc{Hofmannsthal, Hugo von}!zzzSchnitzler, Arthur@\emph{von Arthur Schnitzler}!1924-11-051@{{[}5?.{]} 11. 1924}|)be}\mylabel{h}  \normalsize

\doendnotes{C}
\bigskip
\vfill

\clearpage

\footnotesize

\lohead{\textsc{register}}

% Definiere theindex-Environment komplett neu ohne reledmac
\makeatletter
\renewenvironment{theindex}{%
  \section*{\indexname}%
  \setlength{\parindent}{0pt}%
  \setlength{\parskip}{0pt plus 0.3pt}%
  \let\item\@idxitem
}{%
  \clearpage
}
\makeatother

\IfFileExists{\jobname-pw.ind}{\input{\jobname-pw.ind}}{}

\end{document}

      