%% latex-korrekturansicht-vorspann.tex
%% Vorspann für die Korrekturansicht.
%% Lädt die gemeinsame Datei latex-vorspann.tex mit gesetztem Schalter.

\newif\ifkorrekturansicht
\korrekturansichttrue

\input{../tex-inputs/latex-vorspann}


               \section[Hugo von Hofmannsthal an Arthur Schnitzler, 17. 10. {[}1895{]}]{ Hugo von Hofmannsthal an Arthur Schnitzler, 17. 10. {[}1895{]}}\nopagebreak\mylabel{v}\rehead{ }\normalsize\beginnumbering\briefempfaengerindex{Schnitzler, Arthur@\textsc{Schnitzler, Arthur}!zzzHofmannsthal, Hugo von@\emph{von Hugo von Hofmannsthal}!1895-10-171@{17. 10. {[}1895{]}}|(be} \toendnotes[C]{\smallbreak\pagebreak[2]} \Standort{CUL, Schnitzler, B 43.}
\physDesc{Brief, 1 Blatt, 3 Seiten
\newline{}Handschrift: schwarze Tinte, deutsche Kurrent
\newline{}Schnitzler: mit Bleistift die Jahreszahl ergänzt: »95« und nummeriert: »76« }\buchAbdrucke{\weitereDrucke{Hugo von Hofmannsthal, Arthur Schnitzler: \emph{Briefwechsel}. Hg. Therese Nickl und Heinrich Schnitzler. Frankfurt am Main: \emph{S. Fischer} 1964, S. 63.} }\toendnotes[C]{\smallbreak}\pstart
           \raggedleft{}{\pb}\textcolor{pink}{Venedig}{}\ledrightnote{\textcolor{pink}{Venedig}}{ }17. October\pend
           \pstart
           am Sonntag{ }Früh hab ich Sie beſucht, aber nur 3 Frauen mit Beſen gefunden. Ich
                    wollte Ihnen ſagen, daſs ich nach den Zeitungen und dem Reden der Leute wirklich
                    glaube, daſs Sie jetzt dieſes unberechenbare und ſchwer zu definierende erworben
                    haben, womit man Aufmerkſamkeit und Bewunderung erzwingen kann. Ich glaube, Sie
                    dürfen ſich jetzt erlauben, für die Darſtellung {\pb}tiefer und kühner Dinge auf
                    mehreren Beifall zu rechnen als bloß auf den von 3 oder 4 Freunden.\pend
           \pstart
           \textcolor{blue}{Richard}{}\ledrightnote{\textcolor{blue}{Richard Beer-Hofmann}} hat mir die geſcheidte \textcolor{green}{Kritik}{}\ledrightnote{→\textcolor{green}{Burgtheater [Rechte der Seele/Liebelei]}} von \textcolor{blue}{Berger}{}\ledrightnote{\textcolor{blue}{Alfred von Berger}} geſchickt und die \label{K_L00508_1v}\edtext{\textcolor{green}{Verſpottung}{}\ledrightnote{→\textcolor{green}{Jung-Wiener Dichter. (Zur Burgtheater-Première.)}}}{\lemma{\textnormal{\emph{Verſpottung}}}\Cendnote{\textnormal{Der Text geht nicht nur auf die
                            \emph{\textcolor{green}{Liebelei}} ein, sondern auch auf
                            \textcolor{blue}{Hofmannsthal} und \textcolor{blue}{Beer-Hofmann}.}}}\label{K_L00508_1h} von dem \textcolor{blue}{Anonymen}{}\ledrightnote{\textcolor{blue}{Der Reporter}}. Iſt es der kleine \textcolor{blue}{Kraus}{}\ledrightnote{\textcolor{blue}{Karl Kraus}}? Es hat mich unterhalten, ich wäre froh,
                    wenn ſolche Sachen viel öfter geſchrieben würden und auch Caricaturen von uns
                    gezeichnet. {\pb}Das wird ſich
                    auch immer ſteigern je mutiger und beſſer wir werden; ich denke, von der
                    Generation von Philologen und Dilettanten, die vor uns war, wirds nicht viel
                    Verhöhnungen geben.\pend
           \pstart
           Hier arbeit ich nicht, aber werds wohl nachher.\pend
           \pstart
           Adieu. Herzlich Ihr{\\[\baselineskip]}\spacefill\mbox{Hugo.}\pend
           \leftskip=0em{}\endnumbering\briefempfaengerindex{Schnitzler, Arthur@\textsc{Schnitzler, Arthur}!zzzHofmannsthal, Hugo von@\emph{von Hugo von Hofmannsthal}!1895-10-171@{17. 10. {[}1895{]}}|)be}\mylabel{h}  \normalsize

\doendnotes{C}
\bigskip
\vfill

\clearpage

\footnotesize

\lohead{\textsc{register}}

% Definiere theindex-Environment komplett neu ohne reledmac
\makeatletter
\renewenvironment{theindex}{%
  \section*{\indexname}%
  \setlength{\parindent}{0pt}%
  \setlength{\parskip}{0pt plus 0.3pt}%
  \let\item\@idxitem
}{%
  \clearpage
}
\makeatother

\IfFileExists{\jobname-pw.ind}{\input{\jobname-pw.ind}}{}

\end{document}

      