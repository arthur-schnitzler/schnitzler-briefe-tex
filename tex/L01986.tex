%% latex-korrekturansicht-vorspann.tex
%% Vorspann für die Korrekturansicht.
%% Lädt die gemeinsame Datei latex-vorspann.tex mit gesetztem Schalter.

\newif\ifkorrekturansicht
\korrekturansichttrue

\input{../tex-inputs/latex-vorspann}


               \section[Thomas Mann an Arthur Schnitzler, 22. 11. 1910]{ Thomas Mann an Arthur Schnitzler, 22. 11. 1910}\nopagebreak\mylabel{v}\rehead{ }\normalsize\beginnumbering\briefempfaengerindex{Schnitzler, Arthur@\textsc{Schnitzler, Arthur}!zzzMann, Thomas@\emph{von Thomas Mann}!1910-11-223@{22. 11. 1910}|(be} \toendnotes[C]{\smallbreak\pagebreak[2]} \Standort{CUL, Schnitzler, B 67.}
\physDesc{Brief, 1 Blatt, 2 Seiten
\newline{}Handschrift: schwarze Tinte, deutsche Kurrent
\newline{}Schnitzler: mit Bleistift beschriftet: »\textsc{Mann}« }\buchAbdrucke{\weitereDrucke{Hertha Krotkoff: \emph{Arthur Schnitzler – Thomas Mann: Briefe.} In: \emph{Modern Austrian Literature}, Jg. 7 (1974) Nr. 1/2, S. 14.} }\toendnotes[C]{\smallbreak}\pstart
           \noindent{}\raggedleft{}{\pb}\textcolor{gray}{\textbf{\textcolor{pink}{\textsc{München}}{}\ledrightnote{\textcolor{pink}{München}}\textsc{, den}}}{ }22. XI. 1910.\pend
           \pstart
           \noindent{}\raggedleft{}\textcolor{gray}{\textbf{\textcolor{pink}{FRANZ JOSEPH-STRASSE 2}{}\ledrightnote{\textcolor{pink}{Franz-Joseph-Straße}}.}}\pend
           \pstart{}Sehr verehrter Herr Doctor:\pend\pstart
           Der Verlag \textcolor{brown}{S. Fiſcher}{}\ledrightnote{\textcolor{brown}{S. Fischer Verlag}}{ }ſendet mir in Ihrem gütigen Auftrage Ihr neues
                        \textcolor{green}{Werk}{}\ledrightnote{→\textcolor{green}{Der junge Medardus. Dramatische Historie in einem Vorspiel und fünf Aufzügen}}. Ich brauche Ihnen
                    nicht zu ſagen, mit welcher Freude ich es in Empfang genommen habe. Das \label{K_L01986_1v}\edtext{\textcolor{green}{Bruchstück}{}\ledrightnote{→\textcolor{green}{Vorspiel zu einem Drama »Der junge Medardus«}}}{\lemma{\textnormal{\emph{Bruchstück}}}\Cendnote{\textnormal{\textcolor{blue}{Arthur Schnitzler}: \emph{\textcolor{green}{Vorspiel zu einem Drama »Der junge Medardus«}}. In:
                                \emph{\textcolor{green}{Die neue Rundschau}}, Jg. 21, H. 10,
                                1. 10. 1910, S. 1385–1415.}}}\label{K_L01986_1h}, das Sie in
                    der \textcolor{green}{Neuen Rundſchau}{}\ledrightnote{\textcolor{green}{Die neue Rundschau}} daraus veroeffentlichten,
                    kannte ich ſchon. Nun {\pb}iſt es mir ein
                    Bedürfnis, Ihnen aus der Lektüre des kunſt- und lebensvollen Ganzen heraus,
                    meinen herzlichen Dank und Glückwunſch darzubringen.\pend
           \pstart
           Ihr ſehr ergebener{\\[\baselineskip]}\spacefill\mbox{Thomas Mann.}\pend
           \leftskip=0em{}\endnumbering\briefempfaengerindex{Schnitzler, Arthur@\textsc{Schnitzler, Arthur}!zzzMann, Thomas@\emph{von Thomas Mann}!1910-11-223@{22. 11. 1910}|)be}\mylabel{h}  \normalsize

\doendnotes{C}
\bigskip
\vfill

\clearpage

\footnotesize

\lohead{\textsc{register}}

% Definiere theindex-Environment komplett neu ohne reledmac
\makeatletter
\renewenvironment{theindex}{%
  \section*{\indexname}%
  \setlength{\parindent}{0pt}%
  \setlength{\parskip}{0pt plus 0.3pt}%
  \let\item\@idxitem
}{%
  \clearpage
}
\makeatother

\IfFileExists{\jobname-pw.ind}{\input{\jobname-pw.ind}}{}

\end{document}

      