%% latex-korrekturansicht-vorspann.tex
%% Vorspann für die Korrekturansicht.
%% Lädt die gemeinsame Datei latex-vorspann.tex mit gesetztem Schalter.

\newif\ifkorrekturansicht
\korrekturansichttrue

\input{../tex-inputs/latex-vorspann}


               \section[Arthur Schnitzler an Albert Ehrenstein, 19. 12. 1906]{ Arthur Schnitzler an Albert Ehrenstein, 19. 12. 1906}\nopagebreak\mylabel{v}\rehead{ }\normalsize\beginnumbering\briefempfaengerindex{Ehrenstein, Albert@\textsc{Ehrenstein, Albert}!zzzSchnitzler, Arthur@\emph{von Arthur Schnitzler}!1906-12-191@{19. 12. 1906}|(be} \toendnotes[C]{\smallbreak\pagebreak[2]} \Standort{Jerusalem, The National Library of Israel, ARC. Ms. Var. 306 1 118.}
\physDesc{Kartenbrief
\newline{}Handschrift: schwarze Tinte, deutsche Kurrent\newline{}Versand: 1) Stempel: »\nobreak{}Wien 68, 1\textcolor{gray}{9}. XII. 0\textcolor{gray}{6}, 12\nobreak{}«.  2) Stempel: »\nobreak{}\oindex{XVI., Ottakring@\textbf{XVI., Ottakring}, \emph{Bezirk (A.BZK)}|pwk}16/\textsubscript{1}
                                        Wien, 20. XII. 0{[}6{]}, Best\textcolor{gray}{ellt}\nobreak{}«. }\pstart{}{\pb}\textcolor{gray}{\textbf{Dr. Arthur Schnitzler}}\pend{}\pstart{}\textcolor{gray}{\textbf{\textcolor{pink}{Wien, XVIII. Spoettelgasse 7}{}\ledrightnote{\textcolor{pink}{Edmund-Weiß-Gasse}}.}}\pend{}{\bigskip}\pstart{}Herrn \textsc{Albert Ehrenstein,}\pend{}\pstart{}\textcolor{pink}{Wien}{}\ledrightnote{\textcolor{pink}{Wien}}\pend{}\pstart{}\textcolor{pink}{\textsc{Ottakringerstr. 114}}{}\ledrightnote{\textcolor{pink}{Ottakringerstraße}}.\pend{}{\bigskip}\pstart
           \raggedleft{}{\pb}19/12 906.\pend
           \pstart{}lieber Herr Ehrenstein,\pend\pstart
           Wenn Sie ſich Freitag Na halb vier die \textcolor{green}{Helena}{}\ledrightnote{\textcolor{green}{Helena}} von mir abholen wollen, wird es mir angenehm ſein, Sie bei
                    dieſer Gelegenheit zu ſprechen.\pend
           \pstart
           beſtens grüß\textcolor{gray}{end}{\\[\baselineskip]}Ihr \textcolor{gray}{e}rgeb \spacefill\mbox{A. S.}\pend
           \leftskip=0em{}\endnumbering\briefempfaengerindex{Ehrenstein, Albert@\textsc{Ehrenstein, Albert}!zzzSchnitzler, Arthur@\emph{von Arthur Schnitzler}!1906-12-191@{19. 12. 1906}|)be}\mylabel{h}  \normalsize

\doendnotes{C}
\bigskip
\vfill

\clearpage

\footnotesize

\lohead{\textsc{register}}

% Definiere theindex-Environment komplett neu ohne reledmac
\makeatletter
\renewenvironment{theindex}{%
  \section*{\indexname}%
  \setlength{\parindent}{0pt}%
  \setlength{\parskip}{0pt plus 0.3pt}%
  \let\item\@idxitem
}{%
  \clearpage
}
\makeatother

\IfFileExists{\jobname-pw.ind}{\input{\jobname-pw.ind}}{}

\end{document}

      