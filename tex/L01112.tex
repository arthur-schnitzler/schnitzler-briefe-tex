%% latex-korrekturansicht-vorspann.tex
%% Vorspann für die Korrekturansicht.
%% Lädt die gemeinsame Datei latex-vorspann.tex mit gesetztem Schalter.

\newif\ifkorrekturansicht
\korrekturansichttrue

\input{../tex-inputs/latex-vorspann}


               \section[Arthur Schnitzler: Widmungsexemplar Frau Bertha Garlan für Hermann Bahr, {[}24.?{]} 4. 1901]{ Arthur Schnitzler: Widmungsexemplar Frau Bertha Garlan für Hermann Bahr,
               {[}24.?{]} 4. 1901}\nopagebreak\mylabel{v}\rehead{ }\normalsize\beginnumbering\briefempfaengerindex{Bahr, Hermann@\textsc{Bahr, Hermann}!zzzSchnitzler, Arthur@\emph{von Arthur Schnitzler}!1901-04-241@{{[}24.?{]} 4. 1901}|(be} \toendnotes[C]{\smallbreak\pagebreak[2]} \Standort{Salzburg, Universitätsbibliothek, 32334-I.}
\physDesc{Widmung am Vortitel
\newline{}Handschrift: schwarze Tinte, deutsche Kurrent}\buchAbdrucke{\weitereDrucke{Hermann Bahr, Arthur Schnitzler: \emph{Briefwechsel, Aufzeichnungen, Dokumente (1891–1931)}. Hg. Kurt Ifkovits und Martin Anton Müller. Göttingen: \emph{Wallstein} 2018, S. 203.} }\pstart
           \noindent{}{\pb}Seinem lieben Hermann Bahr{\\}herzlichſt\pend
           \pstart \spacefill\mbox{Arthur Schnitzl\damage{\textcolor{gray}{er}}}\pend{}\pstart
           \noindent{}\textcolor{pink}{Wien}{}\ledrightnote{\textcolor{pink}{Wien}}, April 901.\pend
           {\bigskip}\pstart
           \noindent{}\centering{}\textcolor{gray}{\textbf{\textcolor{green}{Frau Bertha Garlan}{}\ledrightnote{\textcolor{green}{Frau Bertha Garlan. Roman}}}}\pend
           {\bigskip}\pstart
           \noindent{}\centering{}{\pb}\textcolor{gray}{\textbf{\textcolor{green}{Frau Bertha Garlan}{}\ledrightnote{\textcolor{green}{Frau Bertha Garlan. Roman}}}}\pend
           \pstart
           \noindent{}\centering{}\textcolor{gray}{\textbf{\so{Roman}}}\pend
           \pstart
           \noindent{}\centering{}\textcolor{gray}{\textbf{von}}\pend
           \pstart
           \noindent{}\centering{}\textcolor{gray}{\textbf{Arthur Schnitzler}}\pend
           {\bigskip}\pstart
           \noindent{}\centering{}\textcolor{gray}{\textbf{\textcolor{pink}{\textbf{Berlin}}{}\ledrightnote{\textcolor{pink}{Berlin}}}}\pend
           \pstart
           \noindent{}\centering{}\textcolor{gray}{\textbf{\textcolor{brown}{\so{S. Fiſcher, Verlag}}{}\ledrightnote{\textcolor{brown}{S. Fischer Verlag}}}}\pend
           \pstart
           \noindent{}\centering{}\textcolor{gray}{\textbf{1901}}\pend
           \endnumbering\briefempfaengerindex{Bahr, Hermann@\textsc{Bahr, Hermann}!zzzSchnitzler, Arthur@\emph{von Arthur Schnitzler}!1901-04-241@{{[}24.?{]} 4. 1901}|)be}\mylabel{h}  \normalsize

\doendnotes{C}
\bigskip
\vfill

\clearpage

\footnotesize

\lohead{\textsc{register}}

% Definiere theindex-Environment komplett neu ohne reledmac
\makeatletter
\renewenvironment{theindex}{%
  \section*{\indexname}%
  \setlength{\parindent}{0pt}%
  \setlength{\parskip}{0pt plus 0.3pt}%
  \let\item\@idxitem
}{%
  \clearpage
}
\makeatother

\IfFileExists{\jobname-pw.ind}{\input{\jobname-pw.ind}}{}

\end{document}

      