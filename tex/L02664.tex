\documentclass[twoside=false,titlepage=false,open=any, parskip=never, fontsize=12pt, headings=small, chapterprefix=false, appendixprefix=false]{scrbook}
\addtolength{\oddsidemargin}{\evensidemargin}
\setlength{\oddsidemargin}{.5\oddsidemargin}
\setlength{\evensidemargin}{\oddsidemargin}

\usepackage[{textwidth=13cm,textheight=23cm,marginpar=3cm, left=2cm}]{geometry}
%\usepackage[textwidth=80mm, layoutwidth=170mm, paperheight =297mm, paperwidth  =210mm, layoutvoffset= 20mm,layouthoffset= 20mm]{geometry}
%\usepackage[paperheight =297mm, paperwidth  =210mm, layoutheight=230mm, layoutwidth=158mm, layoutvoffset= 20mm, layouthoffset= 20mm, textwidth=150mm, textheight=185mm, showcrop=false]{geometry}
%sepackage[paperheight=230mm, paperwidth=138mm, textwidth=100mm, textheight=185mm]{geometry}
 \usepackage[usenames, dvipsnames]{xcolor}
\usepackage{scrlayer-scrpage}
\usepackage{hyphenat}
\usepackage{fontspec}
\usepackage{moresize}
\usepackage[english, french, greek, ngerman]{babel}
%\usepackage{ipa}  für das Seitenwechselzeichens
\usepackage[babel]{microtype}
\usepackage[dash, dot]{dashundergaps}
\usepackage{soul}
\usepackage{ragged2e}
\usepackage[makeindex, protected]{splitidx}
\usepackage[itemlayout=abshang,hangindent=0.85em, subindent=0em, subsubindent=1em, justific=RaggedRight, columns=1, columnsep=0pt, indentunit=1em, totoc=false]{idxlayout}
\usepackage{scrhack}
\usepackage{xpatch}
\usepackage{reledmac}
\usepackage{refcount} % Für die Seitenverweise 1–3 etc. 
\usepackage{etoolbox}
\usepackage{framed}
\usepackage[export]{adjustbox} % loads also graphicx, für Bildgröße autom. maximal
\usepackage{float} %ermöglicht exakte Bildpositionierung
\usepackage{mdframed}
\usepackage{enumitem}
\usepackage{relsize}
\usepackage{longtable}
\usepackage{chngcntr} % Sectionnummern durchgehend
\usepackage{hanging} % Für hängende Absätze
\usepackage[rightmargin=0em, leftmargin=1em, indentfirst=false]{quoting} % Für die geänderte quote-Umgebung in den Hrsg-Texten
%\usepackage{fontawesome}
\usepackage{ellipsis}
\RequirePackage{hyphsubst}%
\HyphSubstIfExists{ngerman-x-latest}{\HyphSubstLet{ngerman}{ngerman-x-latest}}{} 
\listfiles
\usepackage[noadjust]{marginnote}

\KOMAoptions{toc=chapterentrydotfill, toc=flat}
\addtokomafont{chapterentrypagenumber}{\mdseries}
\setkomafont{chapterentry}{\normalfont\mdseries}
\setkomafont{partentry}{\normalfont\mdseries}
\RedeclareSectionCommand[tocbeforeskip=0pt]{chapter}

\setlength{\skip\footins}{4mm plus 2mm} % Abstand Fussnote Text
\interfootnotelinepenalty=10000 % Kein Seitenwechsel in Fuss

%\DeclareTextFontCommand{\emph}{\textit}

% Der Befehl erlaubt rechtsbündig bei Unterschriften, die nicht mehr in die Zeile passen
\def\spacefill{\hspace{\fill}\mbox{}\linebreak[0]\hspace*{\fill}}
\usepackage{atbegshi}
\usepackage{zref-abspage}
\usepackage{perpage}
\usepackage{zref-user}
\usepackage{tikz}
\usepackage{ulem}
\usetikzlibrary{calc,decorations.pathmorphing}

\PassOptionsToPackage{gray}{xcolor}
\definecolor{gray}{gray}{0.6}

\doublehyphendemerits=1000000 % das hier verhindert zu viele aufeinanderfolgende Trennstriche am Zeilenende


\usepackage{zref-abspage}
\usepackage{zref-user}
\usepackage{tikz}
\usepackage{atbegshi}
\usepackage{ulem}
\usetikzlibrary{calc,decorations.pathmorphing}

\PassOptionsToPackage{gray}{xcolor}
\definecolor{gray}{gray}{0.6}

\doublehyphendemerits=1000000 % das hier verhindert zu viele aufeinanderfolgende Trennstriche am Zeilenende

\makeatletter
\newcommand{\currentsidemargin}{%
  \ifodd\zref@extract{textarea-\thetextarea}{abspage}%
    \oddsidemargin%
  \else%
    \evensidemargin%
  \fi%
}

\newcounter{textarea}
\newcommand{\settextarea}{%
   \stepcounter{textarea}%
   \zlabel{textarea-\thetextarea}%
   \begin{tikzpicture}[overlay,remember picture]
    % Helper nodes
    \path (current page.north west) ++(\hoffset, -\voffset)
        node[anchor=north west, shape=rectangle, inner sep=0, minimum width=\paperwidth, minimum height=\paperheight]
        (pagearea) {};
    \path (pagearea.north west) ++(1in+\currentsidemargin,-1in-\topmargin-\headheight-\headsep)
        node[anchor=north west, shape=rectangle, inner sep=0, minimum width=\textwidth, minimum height=7pt]
        (textarea) {};
  \end{tikzpicture}%
}

\tikzset{tikzul/.style={yshift=-.75\dp\strutbox}}

\newcounter{tikzul}%
\newcommand\tikzul[1][]{%
    \begingroup
    \global\tikzullinewidth\linewidth
    \def\tikzulsetting{[#1]}%
    \stepcounter{tikzul}%
    \settextarea
    \zlabel{tikzul-begin-\thetikzul}%
    \tikz[overlay,remember picture,tikzul] \coordinate (tikzul-\thetikzul) at (0,0);% Modified \tikzmark macro
    \ifnum\zref@extract{tikzul-begin-\thetikzul}{abspage}=\zref@extract{tikzul-end-\thetikzul}{abspage}
    \else
        \AtBeginShipoutNext{\tikzul@endpage{#1}}%
    \fi
    \bgroup
    \def\par{\ifhmode\unskip\fi\egroup\par\@ifnextchar\noindent{\noindent\tikzul[#1]}{\tikzul[#1]\bgroup}}%
    \aftergroup\endtikzul
    \let\@let@token=%
}
\newlength\tikzullinewidth


\def\tikzul@endpage#1{%
\setbox\AtBeginShipoutBox\hbox{%
\box\AtBeginShipoutBox
\hbox{%
\begin{tikzpicture}[overlay,remember picture,tikzul]
\draw[#1]
    let \p1 = (tikzul-\thetikzul), \p2 = ([xshift=\tikzullinewidth+\@totalleftmargin]textarea.south west) in
    \ifdim\dimexpr\y1-\y2<.5\baselineskip
        (\x1,\y1) -- (\x2,\y1)
    \else
        let \p3 = ([xshift=\@totalleftmargin]textarea.west) in
        (\x1,\y1) -- +(\tikzullinewidth-\x1+\x3,0)
        % (\x3,\y2) -- (\x2,\y2)
        (\x3,\y1)
       \myloop{\y1-\y2+.5\baselineskip}{%
           ++(0,-\baselineskip) -- +(\tikzullinewidth,0)
       }%
    \fi
;
\end{tikzpicture}%
}}%
}%


\def\endtikzul{%
    \zlabel{tikzul-end-\thetikzul}%
    \ifnum\zref@extract{tikzul-begin-\thetikzul}{abspage}=\zref@extract{tikzul-end-\thetikzul}{abspage}
    \begin{tikzpicture}[overlay,remember picture,tikzul]
        \expandafter\draw\tikzulsetting
            let \p1 = (tikzul-\thetikzul), \p2 = (0,0) in
            \ifdim\y1=\y2
                (\x1,\y1) -- (\x2,\y2)
            \else
                let \p3 = ([xshift=\@totalleftmargin]textarea.west), \p4 = ([xshift=-\rightmargin]textarea.east) in
                (\x1,\y1) -- +(\tikzullinewidth-\x1+\x3,0)
                (\x3,\y2) -- (\x2,\y2)
                (\x3,\y1)
                \myloop{\y1-\y2}{%
                    ++(0,-\baselineskip) -- +(\tikzullinewidth,0)
                }%
            \fi
        ;
    \end{tikzpicture}%
    \else
    \settextarea
    \begin{tikzpicture}[overlay,remember picture,tikzul]
        \expandafter\draw\tikzulsetting
            let \p1 = ([xshift=\@totalleftmargin,yshift=-.5\baselineskip]textarea.north west), \p2 = (0,0) in
            \ifdim\dimexpr\y1-\y2<.5\baselineskip
                (\x1,\y2) -- (\x2,\y2)
            \else
                let \p3 = ([xshift=\@totalleftmargin]textarea.west), \p4 = ([xshift=-\rightmargin]textarea.east) in
                (\x3,\y2) -- (\x2,\y2)
                (\x3,\y2)
                \myloop{\y1-\y2}{%
                    ++(0,+\baselineskip) -- +(\tikzullinewidth,0)
                }
            \fi
        ;
    \end{tikzpicture}%
    \fi
    \endgroup
}

% -------------------------------------------------------------- Additions by Peter Grill

\tikzset{tikzst/.style={yshift=0.5\dp\strutbox}}

\newcounter{tikzst}%
\newcommand\tikzst[1][]{%
    \begingroup
    \global\tikzstlinewidth\linewidth
    \def\tikzstsetting{[#1]}%
    \stepcounter{tikzst}%
    \settextarea
    \zlabel{tikzst-begin-\thetikzst}%
    \tikz[overlay,remember picture,tikzst] \coordinate (tikzst-\thetikzst) at (0,0);% Modified \tikzmark macro
    \ifnum\zref@extract{tikzst-begin-\thetikzst}{abspage}=\zref@extract{tikzst-end-\thetikzst}{abspage}
    \else
        \AtBeginShipoutNext{\tikzst@endpage{#1}}%
    \fi
    \bgroup
    \def\par{\ifhmode\unskip\fi\egroup\par\@ifnextchar\noindent{\noindent\tikzst[#1]}{\tikzst[#1]\bgroup}}%
    \aftergroup\endtikzst
    \let\@let@token=%
}
\newlength\tikzstlinewidth


\def\tikzst@endpage#1{%
\setbox\AtBeginShipoutBox\hbox{%
\box\AtBeginShipoutBox
\hbox{%
\begin{tikzpicture}[overlay,remember picture,tikzst]
\draw[#1]
    let \p1 = (tikzst-\thetikzst), \p2 = ([xshift=\tikzstlinewidth+\@totalleftmargin]textarea.south west) in
    \ifdim\dimexpr\y1-\y2<.5\baselineskip
        (\x1,\y1) -- (\x2,\y1)
    \else
        let \p3 = ([xshift=\@totalleftmargin]textarea.west) in
        (\x1,\y1) -- +(\tikzstlinewidth-\x1+\x3,0)
        % (\x3,\y2) -- (\x2,\y2)
        (\x3,\y1)
       \myloop{\y1-\y2+.5\baselineskip}{%
           ++(0,-\baselineskip) -- +(\tikzstlinewidth,0)
       }%
    \fi
;
\end{tikzpicture}%
}}%
}%


\def\endtikzst{%
    \zlabel{tikzst-end-\thetikzst}%
    \ifnum\zref@extract{tikzst-begin-\thetikzst}{abspage}=\zref@extract{tikzst-end-\thetikzst}{abspage}
    \begin{tikzpicture}[overlay,remember picture,tikzst]
        \expandafter\draw\tikzstsetting
            let \p1 = (tikzst-\thetikzst), \p2 = (0,0) in
            \ifdim\y1=\y2
                (\x1,\y1) -- (\x2,\y2)
            \else
                let \p3 = ([xshift=\@totalleftmargin]textarea.west), \p4 = ([xshift=-\rightmargin]textarea.east) in
                (\x1,\y1) -- +(\tikzstlinewidth-\x1+\x3,0)
                (\x3,\y2) -- (\x2,\y2)
                (\x3,\y1)
                \myloop{\y1-\y2}{%
                    ++(0,-\baselineskip) -- +(\tikzstlinewidth,0)
                }%
            \fi
        ;
    \end{tikzpicture}%
    \else
    \settextarea
    \begin{tikzpicture}[overlay,remember picture,tikzst]
        \expandafter\draw\tikzstsetting
            let \p1 = ([xshift=\@totalleftmargin,yshift=-.5\baselineskip]textarea.north west), \p2 = (0,0) in
            \ifdim\dimexpr\y1-\y2<.5\baselineskip
                (\x1,\y2) -- (\x2,\y2)
            \else
                let \p3 = ([xshift=\@totalleftmargin]textarea.west), \p4 = ([xshift=-\rightmargin]textarea.east) in
                (\x3,\y2) -- (\x2,\y2)
                (\x3,\y2)
                \myloop{\y1-\y2}{%
                    ++(0,+\baselineskip) -- +(\tikzstlinewidth,0)
                }
            \fi
        ;
    \end{tikzpicture}%
    \fi
    \endgroup
}
% --------------------------------------------------------------

\def\myloop#1#2#3{%
    #3%
    \ifdim\dimexpr#1>1.1\baselineskip
        #2%
        \expandafter\myloop\expandafter{\the\dimexpr#1-\baselineskip\relax}{#2}%
    \fi
}

\makeatother






\def\myloop#1#2#3{%
    #3%
    \ifdim\dimexpr#1>1.1\baselineskip
        #2%
        \expandafter\myloop\expandafter{\the\dimexpr#1-\baselineskip\relax}{#2}%
    \fi
}

\makeatother
%\newcommand{\damage}[1]{\tikzul[gray,line width=0.15\ht\strutbox,semitransparent]{#1}}
%\newcommand{\strikeout}[1]{\tikzst[black]{#1}}

\newcommand{\damage}[1]{\textcolor{orange}{#1}}
\newcommand{\strikeout}[1]{\sout{#1}}


\setlength{\parindent}{1em}

% Mehr als drei Auslassungspunkte 

\newcommand{\dotsseven}{%
.\kern\ellipsisgap 
.\kern\ellipsisgap
.\kern\ellipsisgap 
.\kern\ellipsisgap
.\kern\ellipsisgap
.\kern\ellipsisgap 
.\kern\ellipsisgap 	
\relax}

\newcommand{\dotssix}{%
.\kern\ellipsisgap 
.\kern\ellipsisgap
.\kern\ellipsisgap
.\kern\ellipsisgap
.\kern\ellipsisgap 
.\kern\ellipsisgap 
\relax}

\newcommand{\dotsfive}{%
.\kern\ellipsisgap 
.\kern\ellipsisgap
.\kern\ellipsisgap
.\kern\ellipsisgap 
.\kern\ellipsisgap 
\relax}

\newcommand{\dotsfour}{%
.\kern\ellipsisgap 
.\kern\ellipsisgap
.\kern\ellipsisgap
.\kern\ellipsisgap 
\relax}

\newcommand{\dotstwo}{%
.\kern\ellipsisgap 
.\kern\ellipsisgap
\relax}


% Silbentrennung
\selectlanguage{ngerman}
\hyphenation{Re-kours EP-STEIN Her-vay-vor-les-ung Steu-er-sa-chen Öst-reich Burck-hard Keuch-hus-ten Oedi-pus-auf-führ-un-gen Hi-obs-post Kärnt-ner-ring Vei-tlis-sen-gas-se Franck-gas-se Rath-hau-se Sechs-schg Stu-bai-thal Tha-deusz Volks-th Halb-mo-nats-schrift JAHR-ES-ZEI-TEN Te-le-phon mit-ge-theilt Ge-schäfts-ver-bin-dung hoch-müth-ig Ueber-zeu-gung bis-chen Au-tor-rech-te Hof-manns-thal Nor-deijk Irre-seins Tschap-perl mit-zu-thei-len Aeu-ße-rung be-thö-ren Kü-ni-gel Be-ur-thei-lung Kuenst-lern ko-moe-di-sche hae-mor-rha-gi-scher Doer-mann Wash-burn flei-ssig haute Buddh-ist Preu-ssen Lin-den-café Mit-theil-un-gen An-theil Lieu-te-nant oes-terr Rieg-ner Oes-ter-reich gro-ssem Fran-zo-sen-thum Roche Lili Ent-schlie-ssun-gen äu-ssert wuen-sche Trans-ac-tio-nen Ue-ber-win-dung Eu-gene Stra-ssen-dir-ne qua-tre Deutsch-öst-er-reich Deutsch-öst-er-reichs Bjørn-stjer-ne noth-ing Edit-ed Olga Ar-naud Mer-gent-heim Léon-tine Polla-czek Brion Barre Hoch-sin-ger Ka-tha-rina Arouet Va-len-ci-ennes Ueber-win-dung Type-writer-in Tolstoi-buch Schnitzler Copier-buche Schiller Intel-lek-tuell-en-as-so-zi-a-tion Salten Devrient Grien-steidl Ge-sell-ſchaft ein-ge-ſchloſ-ſen Fort-ſetz-un-gen Bor-dell-ſtück fort-ſchrei-ten wirk-ſam-es ſchrift-ſtel-ler-i-ſchen hin-weg-ſe-hen Gerichts-saal-be-richt-er-ſtat-ter}



% Sonderbefehl für .–
\def\dotdash{\nobreak\hspace{0pt}.–}  %ACHTUNG BEIM ERSETZEN: LEERZEICHEN DANACH 
\def\commadash{\nobreak\hspace{0pt},–}
\def\excdash{\nobreak\hspace{0pt}!–}
\def\semicolondash{\nobreak\hspace{0pt};–}
\def\parentdotdash{\nobreak\hspace{0pt}).–}
\def\slashislash{\,\slash\,\allowbreak\hspace{0pt}}

\newcommand{\strich}{\makebox[1em][l]{– }}


% Seite einrichten

% Farbe definieren
%\setmainfont[RawFeature={-liga}, 
%SmallCapsFont=WSVgara-Caps, 
%ItalicFont=WSVgara-Italic, 
%BoldFont=WSVgara-Bold,
%BoldItalicFont=WSVgara-BoldItalic
%]{WSVgara}
%\setsansfont[RawFeature={-liga}, 
%SmallCapsFont=WSVgara-Caps, 
%ItalicFont=WSVgara-Italic, 
%BoldFont=WSVgara-Bold,
%BoldItalicFont=WSVgara-BoldItalic
%]{WSVgara}

%\setmainfont{Brill}
%\setsansfont{Brill}

%\setmainfont[ItalicFont=SinaNova-Italic, 
%BoldFont=SinaNova-Bold,
%BoldItalicFont=SinaNova-BoldItalic
%]{SinaNova-Regular}
%\setsansfont[ItalicFont=SinaNova-Italic, 
%BoldFont=SinaNova-Bold,
%BoldItalicFont=SinaNova-BoldItalic
%]{SinaNova-Regular}



\def\labelitemi{--}

% Geminationsstrich, U-Strich

 \newcommand{\overbar}[1]{$\overline{\hbox{#1}}$}


% Ausrufezeichen in den Index kriegen
\newcommand{\rufezeichen}{"!}

% Griechisch
	
%\newfontfamily\greekfont{GaramondPremrPro}
%\newcommand\griechisch[1]{\greekfont{}#1{}\normalfont}

%\newfontfamily\sansseriffont[HyphenChar=None, RawFeature={-liga}, Scale=1.03]{TheSans-Regular}
%\newfontfamily\sansseriffont{uarial}


%\newfontfamily\sansseriffont[HyphenChar=None, LetterSpace=1.0, RawFeature={-liga}]{TheSans-SemiBold}
%\newcommand\sansseriff[1]{\sffamily{}#1{}\normalfont}

\newcommand{\mini}{\,}


\newcommand{\key}{\textsuperscript{\textcolor{red}{KEY}}}


%% Sperrung (Package Soul)
%% Hier ist die Sperrung definiert. Sperrung erreicht man mit \so{gesperrtes Wort}
\sodef\so{}{.14em}{.4em plus.1em minus .1em}{.4em plus.1em minus .1em}

% SCHRIFTEN
\setkomafont{disposition}{}
\addtokomafont{caption}{\small}
\addtokomafont{captionlabel}{\small}

%% Schrift der Kopf und Fußzeile
\renewcommand*{\headfont}{\normalfont}
\setkomafont{pagehead}{\footnotesize\addfontfeature{LetterSpace=10.0}}
\setkomafont{pagenumber}{\normalfont\normalsize}
\ohead[]{\pagemark}% Seitenzahl (c = centered) 
\ofoot[]{}


 
% Flatterndes Seitenende
\raggedbottom

% Fussnoten neu Anfangen

\makeatletter
\pretocmd{\@schapter}{\setcounter{footnote}{0}}{}{}
\pretocmd{\@chapter}{\setcounter{footnote}{0}}{}{}
\pretocmd{\@section}{\setcounter{footnote}{0}}{}{}
\makeatother


% Section Nummern durchgehend

\RedeclareSectionCommand[
  counterwithout=chapter
]{section}

% Section Punkt

\renewcommand*{\sectionformat}{}
\renewcommand*{\partformat}{}


% Marginpar Schrift

\newkomafont{margin}{\footnotesize} 
\makeatletter 
\let\MarginParOriginal\marginpar 
\renewcommand*{\marginpar}{\@dblarg\@marginpar} 
\newcommand{\@marginpar}[2][]{% 
  \MarginParOriginal[\usekomafont{margin}{#1\par}]{\usekomafont{margin}{#2\par}} 
} 
\makeatother 



\let\oldbeginnumbering\beginnumbering

\def\beginnumbering{\oldbeginnumbering\par\nopagebreak}


% Fußnoten linksbündig
\deffootnote{1.5em}{1em}{% 
\makebox[1.5em][l]{\thefootnotemark}%
}


% Fussnotenlineal (wobei für reledmac wohl was anderes gilt)
\let\normalfootnoterule\footnoterule
\setfootnoterule{0pt}
\let\normalfootnoterule\footnoterule


\setlength{\skip\footins}{8mm plus 2mm} % Abstand Fussnote Text
\interfootnotelinepenalty=10000 % Kein Seitenwechsel in Fuss

%% Kapitelüberschriften
\renewcommand*{\raggedchapter}{\centering} 
\renewcommand*{\raggedsection}{%
 \CenteringLeftskip=1cm plus 1em\relax 
 \CenteringRightskip=1cm plus 1em\relax 
 \Centering\footnotesize\thesection{}.\ }
\setkomafont{section}{\footnotesize}
\setkomafont{chapter}{\normalfont\Large}
\renewcommand{\chapterpagestyle}{empty}%The first page in each chapter won't have any heading or footer, especially no page number

% section ohne führende Kapitelnummer
\renewcommand*\thesection{\arabic{section}}

% Bildunterschrift ohne Nummer
\renewcommand*{\figureformat}{}
\renewcommand*{\captionformat}{}

% Abstand Bild
\setlength{\textfloatsep}{\baselineskip}

%% Zeilennummern
\firstlinenum{0} \linenumincrement{5}
\lineation{section} %Jeder Abschnitt wird durchnummeriert
\renewcommand{\numlabfont}{\ssmall} %Schriftgröße Zeilennummern

%\AtBeginEnvironment{multicols}{\RaggedRight} % Linksbündig in Spalten


% SEITENUMBRÜCHE IM TEXT MARKIEREN

%% Seitenumbrüche


\newcommand{\Theight}{\dimexpr\fontcharht\font`W}
\newcommand{\pbposition}{\depth}
\newcommand{\pb}{\nobreak\hspace{0pt}\raisebox{-0.1em}{\raisebox{\pbposition}{\textnormal{|}}}\nobreak\hspace{0pt}}

% EINFÜGUNGEN IM TEXT MARKIEREN

\renewcaptionname{ngerman}{\contentsname}{Inhalt}           %Table of contents


\newcommand{\introOben}{\textnormal{\raisebox{\Theight}{\raisebox{-\height}{\small{v}\normalsize}}}}
\newcommand{\introUnten}{\textnormal{\raisebox{\Theight}{\raisebox{-\height}{\small{v}\normalsize}}}}
\newcommand{\introMitteVorne}{\textnormal{\raisebox{\Theight}{\raisebox{-\height}{\small{v}\normalsize}}}}
\newcommand{\introMitteHinten}{\textnormal{\raisebox{\Theight}{\raisebox{-\height}{\small{v}\normalsize}}}}
\newcommand{\substVorne}{\textnormal{\raisebox{\Theight}{\raisebox{-\height}{\rotatebox[origin=c]{180}{v}\normalsize}}}}
\newcommand{\substDazwischen}{}
\newcommand{\substHinten}{\textnormal{\raisebox{\Theight}{\raisebox{-\height}{\small{v}\normalsize}}}}


% MARGINALSPALTE
\setlength\ledrsnotewidth{1.5cm}


% FUSSNOTE
%% Im Apparat f. und ff.
\Xtwolines{f.}
\Xtwolinesbutnotmore

%% Sperrungen bei Lemmas im Apparat
%\pretocmd{\so}{\null}{}{}
% Hab ich auskommentiert: Hat einen Fehler ergeben, denn plötzlich war ein Abstand vor Absätzen, die mit einer Sperrung beginnen

%% Zeilennummerierung Abstand zum Lemma
\Xboxlinenum{5mm}

%% Bei zwei Apparateinträgen in einer Zeile wird nur beim ersten Mal die Zeile gezählt
\Xnumberonlyfirstinline
\Xnumberonlyfirstintwolines
\Xinplaceofnumber{1em}
\Xhangindent{1em}

% ENDNOTEN
\Xendlemmadisablefontselection[A]
\renewcommand*{\printnpnum}[1]{{\noindent}\tiny}
\Xendparagraph[A] % Endnoten in einem Absatz
%\Xendtwolines{\tiny{f.}}
\Xendbeforepagenumber{} 
\Xendnotenumfont[A]{\tiny}
\Xendboxlinenum[A]{0em}
\Xendlemmaseparator{$\rbracket$}
\Xendnotefontsize[A]{\footnotesize}
\Xendhangindent[A]{1em}
\Xendlemmafont[A]{\itshape}
\Xendlemmafont[B]{\bfseries}
\Xendnotefontsize[B]{\footnotesize}
\Xendnotenumfont{\footnotesize}
\Xendlineprefixsingle[C]{\tiny}
\Xendlineprefixmore[C]{\tiny}
\Xendlemmadisablefontselection
\Xendlemmafont{\itshape}
\Xendlinerangeseparator{\tiny{--}}
\Xendhangindent{4em}
\Xendboxlinenum{3.6em}
\Xendafternumber{0.4em}
\Xendboxlinenumalign{R}

%\Xendboxstartlinenum{3.5em}
%\Xendboxendlinenum{1em}


%% Kaufmanns-Und (=)
            
            

\newcommand{\kaufmannsund}{\&} 

%% Tabelle Zellensprung
% Ein weiterer Anlass, das Kaufmannsund in der Übergabe zu vermeiden:

\newcommand{\zellensprung}{ \& }

%% INDEX
    
    \makeindex 
    \newcommand*\lettergroup[1]{}
    
        \newcommand{\pw}[1]{#1}
        \newcommand{\pwt}[1]{\textbf{#1}}
        \newcommand{\pws}[1]{\upshape{\textbf{#1}}}
            
        \newcommand{\pwe}[1]{\textbf{\emph{#1}}}
             
    \newcommand{\pwk}[1]{#1\textsuperscript{\tiny{K}}}
    \newcommand{\pwv}[1]{\emph{#1}}
     \newcommand{\pwkv}[1]{\emph{#1}\textsuperscript{\tiny{K}}}
               \newcommand{\pwuv}[1]{\emph{#1}?}
               \newcommand{\pwu}[1]{#1?}
 \newcommand{\range}[2]{{\def\pw##1{##1}#1}--#2}

\newcommand{\buch}[1]{#1}


%% MEHRERE INDIZES

\newindex[Register]{pw}
%\newindex[Institutionen Organisationen Periodika und Unternehmen]{org}
%\newindex[Institutionen und Orte]{o}
\newindex[Korrespondenzpartner]{briefe-out}
\newindex[Gedruckte Quellen]{buch-abdruck}

\newcommand\briefsenderindex[1]{\sindex[briefe-out]{#1}}
\newcommand\briefempfaengerindex[1]{\sindex[briefe-out]{#1}}

\newcommand\buchabdruck[1]{\sindex[buch-abdruck]{#1}}
\renewcommand\buchabdruck[1]{}



%% Symbole

%\newcommand{\symaddr}{\includegraphics[height=6pt]{symbol/noun_637366.png}}
%\newcommand{\symweiteredrucke}{\includegraphics[height=6pt]{symbol/noun_634729.png}}
%\newcommand{\symdruckvorlage}{\includegraphics[height=6pt]{symbol/noun_637409.png}}
%\newcommand{\symstandort}{\includegraphics[height=6pt]{symbol/noun_634216.png}}
%\newcommand{\symhead}{\includegraphics[height=6pt]{symbol/noun_1162030_cc.png}}


\newcommand{\symaddr}{A}
\newcommand{\symweiteredrucke}{D}
\newcommand{\symdruckvorlage}{V}
\newcommand{\symstandort}{O}
\newcommand{\symhead}{H}



\newcommand\anhangTitel[2]{\toendnotes[C]{\hangpara{4em}{1}{\makebox[4em][l]{\textbf{#1}}\textbf{#2}}\endgraf}}
\newcommand\Adresse[1]{\toendnotes[C]{\hangpara{4em}{1}{\makebox[4em][l]{\makebox[3.6em][r]{\symaddr}}}#1\endgraf}}

\newcommand\buchAlsQuelle[1]{\toendnotes[C]{\footnotesize\par\hangpara{4em}{1}{\makebox[4em][l]{\makebox[3.6em][r]{\symdruckvorlage}}}#1\endgraf}}
\newcommand\buchAbdrucke[1]{\toendnotes[C]{\footnotesize\par\hangpara{4em}{1}{\makebox[4em][l]{\makebox[3.6em][r]{\symweiteredrucke}}}#1\endgraf}}
\newcommand\Standort[1]{\toendnotes[C]{\footnotesize\hangpara{4em}{1}{\makebox[4em][l]{\makebox[3.6em][r]{\symstandort}}}#1\endgraf}}
\newcommand\biographical[1]{\toendnotes[C]{\footnotesize\hangpara{4em}{1}{\makebox[4em][l]{\makebox[3.6em][r]{\symhead}}}#1\endgraf}}
\newcommand\biographicalOhne[1]{\toendnotes[C]{\footnotesize\hangpara{4em}{1}{\makebox[4em][l]{\makebox[3.6em][r]{}}}#1\endgraf}}



\newcommand\datumImAnhang[1]{\toendnotes[C]{#1}}

\let\newcell&

\newcommand\physDesc[1]{\toendnotes[C]{\hangpara{4em}{0}#1\endgraf}}
\newcommand\weitereDrucke[1]{#1}


% Schnitzler Tagebuch Auszüge
\newcommand{\prgrph}[1]{\endgraf\medskip\noindent\textbf{#1}\newline}


%% VERWEISE
% Dieser Befehl vom Typ
% \verweis{FW_V_schwn_A}{FW_V_schwn_E} 
% dient den Verweisen auf den Text von Kommentar und Herausgebereingriffen. Ihm werden die Namen der beiden Labels – Anfang und Ende – übergeben und er setzt den Anfang und entscheidet ob f. oder ff. folgt 


\newcounter{mystart}
\newcounter{mystop}
\newcounter{phantom}

\newcommand*\myrangeref[2]{%
  \setcounterpageref{mystart}{#1}%
  \setcounterpageref{mystop}{#2}%
  \ifnum\value{mystop}<\value{mystart}%
    \typeout{[myrangeref] Strange...stop (#2) before start (#1).}%
    \pageref{#2}--\pageref{#1}%
  \else
    \pageref{#1}%
    \ifnum\value{mystart}<\value{mystop}%
      \addtocounter{mystop}{-1}%
      \ifnum\value{mystart}<\value{mystop}%
        \,ff.
        %--\pageref{#2}%%
      \else
        \,f.
         %%--\pageref{#2}%
              \fi
    \fi
  \fi
}
            
\newcommand*\myrangerefkasten[2]{%
  \setcounterpageref{mystart}{#1}%
  \setcounterpageref{mystop}{#2}%
  \ifnum\value{mystop}<\value{mystart}%
    \typeout{[myrangeref] Strange...stop (#2) before start (#1).}%
    \pageref{#2}--\pageref{#1}%
  \else
    \makebox[12pt][r]{\pageref{#1}}%
    \ifnum\value{mystart}<\value{mystop}%
      \addtocounter{mystop}{-1}%
      \ifnum\value{mystart}<\value{mystop}%
        --\pageref{#2}%%
      \else
         --\pageref{#2}%
         % alternativ hierher: f.
      \fi
    \fi
  \fi
}


\newcommand*\mylabel[1]{%
  \refstepcounter{phantom}%
  \label{#1}%
}

\newenvironment{anhang}{\vspace{1cm}
}{}

\emfontdeclare{\itshape}

%% RAHMEN SEITLICH

\newlength{\leftbarwidth}
\setlength{\leftbarwidth}{3pt}
\newlength{\leftbarsep}
\setlength{\leftbarsep}{10pt}

\renewenvironment{leftbar}[1][\hsize]
{% 
\def\FrameCommand 
{%
{\hspace{-7pt} \color{black} \vrule width 0.5pt}%
\hspace{0pt}%must no space.
\fboxsep=\FrameSep\colorbox{white}%
}%
\MakeFramed{\hsize#1\advance\hsize-\width\FrameRestore}%
}
{\endMakeFramed}
\setlength{\FrameSep}{5pt}

\newmdenv[topline=false, leftline=true, rightline=true, bottomline=false,%
  linewidth=0.5pt, leftmargin=30pt, rightmargin=30pt, %
  skipabove=8pt, skipbelow=8pt]{mdbar}

% Überstreichung (OVERLINE)

\makeatletter
\newcommand*{\textoverline}[1]{$\overline{\hbox{#1}}\m@th$}
\makeatother

% Rahmen für Hintergrundfarbe
\fboxsep0mm

% Befehl für gekürzte Texte

\newcommand{\kuerzung}{, Auszug}

% Verse 

\setlength{\stanzaindentbase}{20pt} %Play with it later.
\setstanzaindents{5,1,1}
\setcounter{stanzaindentsrepetition}{2}
\newcommand{\stanzaend}{\&}
\sethangingsymbol{\protect\hfill}
\AtEveryStopStanza{\vspace{0.25\baselineskip}} %Abstand zwischen Strophen


% Versuch eines Grid

\RedeclareSectionCommand[
  beforeskip=3\baselineskip,
  afterskip=\baselineskip
]{chapter}
\RedeclareSectionCommand[
  beforeskip=2\baselineskip,
  afterskip=\baselineskip
]{section}

\newcommand\adjacent[2][]{%
  \bgroup
  \RedeclareSectionCommand[
    beforeskip=2\baselineskip,
    afterskip=\baselineskip,
  ]{chapter}%
  \if\relax\detokenize{#1}\relax
    \addchap{#2}%
  \else
    \addchap[#1]{#2}%
  \fi
  \egroup
  \section
}


%change the part format in table of contents
\renewcaptionname{ngerman}{\contentsname}{Inhalt} 


% Inhaltsverzeichnis

\AtBeginDocument{%
  \addtocontents{toc}{\protect\label{toc}}%
}

\renewcaptionname{ngerman}{\contentsname}{Verzeichnis der Dokumente} 
 
 
   \DeclareTOCStyleEntry[
  beforeskip=15pt,
  entryformat=\normalsize\normalfont\centering,
  pagenumberformat=\nullfont,
  linefill={},
  raggedentrytext=true
]{part}{part}

  \DeclareTOCStyleEntry[
  beforeskip=5pt,
  entryformat=\normalsize\normalfont\centering,
  pagenumberformat=\nullfont,
  linefill={},
  raggedentrytext=true
]{chapter}{chapter}

\DeclareTOCStyleEntry[
  onstarthigherlevel=\vspace*{0.5\baselineskip}\nobreak,
  indent=0pt,
  entryformat=\normalsize\def\autodot{.},
  pagenumberformat=\normalsize,
  raggedentrytext=true
]{section}{section}



 
% Das folgende auskommentiert, funktionierte nicht mehr, ging aber in Bahr/Schnitzler. Sollte eigentlich dazu dienen, beim Inhaltsverzeichnis die Nummern rechtsbündig zu setzen

 \iffalse
 
  \DeclareTOCStyleEntry[
  beforeskip=5pt,
  entryformat=\normalsize\normalfont\centering,
  pagenumberformat=\nullfont,
  linefill={},
  raggedentrytext=true
]{chapter}{chapter}

\DeclareTOCStyleEntry[
  onstarthigherlevel=\vspace*{0.5\baselineskip}\nobreak,
  indent=0pt,
  entryformat=\normalsize\def\autodot{.},
  pagenumberformat=\normalsize,
  raggedentrytext=true
]{section}{section}
 
 
  \newcommand*\sectionnumberbox[1]{\hfill #1\hspace{.6em}}

\newlength{\zweiziffern}
\newlength{\dreiziffern}
\newlength{\vierziffern}
\settowidth{\zweiziffern}{9999}
\settowidth{\dreiziffern}{99999}
\settowidth{\vierziffern}{99999999}
 
\BeforeStartingTOC[toc]{\value{tocdepth}=\sectiontocdepth}


\DeclareTOCStyleEntry[
  onstarthigherlevel=\vspace*{0.5\baselineskip}\nobreak,
  indent=0pt,
  entryformat=\normalsize\def\autodot{.},
  entrynumberformat=\sectionnumberbox,
  pagenumberformat=\normalsize,
  numwidth=\zweiziffern,
  raggedentrytext=true
]{section}{section}

\newcommand{\toccheck}{\ifnum \value{section}=76 \addtocontents{toc}{\protect\DeclareTOCStyleEntry[numwidth=\dreiziffern]{section}{section}} \else \ifnum \value{section}=990 \addtocontents{toc}{\protect\DeclareTOCStyleEntry[numwidth=\vierziffern]{section}{section}} \fi \fi}
\fi



% Längen für Tabellen
\newlength{\longeste}
\newlength{\longestz}
\newlength{\longestd}
\newlength{\longestv}
\newlength{\longestf}

\newcommand\halbtextwidth{0.9\textwidth}

\newcommand\pwindex[1]{{\sindex[pw]{#1}}}
\newcommand\oindex[1]{{\sindex[pw]{#1}}}
\newcommand\orgindex[1]{{\sindex[pw]{#1}}}

\renewcommand\oindex[1]{{{\sindex[pw]{#1}}}}
\renewcommand\orgindex[1]{{{\sindex[pw]{#1}}}}



% INDEX

%\renewcommand\pwindex[1]{}
%\renewcommand\oindex[1]{}
%\renewcommand\orgindex[1]{}
%\renewcommand\buchabdruck[1]{}


\newcommand\url[1]{\mbox{#1}}
\renewcommand\ngermanhyphenmins{33}

\makeatletter
\newcommand*{\geminationm}{$\overline{\hbox{m}}\m@th$}
\newcommand*{\geminationn}{$\overline{\hbox{n}}\m@th$}
\makeatother

%part
\renewcommand{\partmarkformat}{}
\renewcommand{\partheadmidvskip}{\enskip}
\renewcommand{\partformat}{}
\setkomafont{partnumber}{\usekomafont{part}}


%\geometry{headsep=8pt} % Abstand Kopfzeile - Text
%% DOKUMENT

\begin{document}

% Section ohne Nummer
\renewcommand*{\raggedsection}{%
 \CenteringLeftskip=1cm plus 1em\relax 
 \CenteringRightskip=1cm plus 1em\relax 
 \Centering\normalsize}



\widowpenalty=10000         % avoid widows
\clubpenalty=10000          % avoid orphans

\sloppy
\setlength{\parindent}{0em}

\setlength{\ledlsnotewidth}{4cm}
\setlength{\ledrsnotewidth}{4cm}
\renewcommand*{\ledlsnotefontsetup}{\scriptsize\sffamily}% left
\renewcommand*{\ledrsnotefontsetup}{\scriptsize\sffamily}% left
\thispagestyle{empty} 

               \section[Paul Goldmann an Arthur Schnitzler, 9. 6. {[}1891{]}]{ Paul Goldmann an Arthur Schnitzler, 9. 6. {[}1891{]}}\nopagebreak\mylabel{v}\rehead{ }\normalsize\beginnumbering\briefempfaengerindex{Schnitzler, Arthur@\textsc{Schnitzler, Arthur}!zzzGoldmann, Paul@\emph{von Paul Goldmann}!1891-06-091@{9. 6. {[}1891{]}}|(be} \toendnotes[C]{\smallbreak\pagebreak[2]} \Standort{DLA, A:Schnitzler, HS.NZ85.1.3162.}
\physDesc{Brief, 3 Blätter, 7 Seiten
\newline{}Handschrift: schwarze Tinte, deutsche Kurrent
\newline{}Schnitzler: mit rotem Buntstift eine Unterstreichung }\toendnotes[C]{\smallbreak}\pstart
           \raggedleft{}{\pb}\textsc{\textcolor{pink}{Brüssel}{}\ledrightnote{\textcolor{pink}{Brüssel}}}, 9. Juni.\pend
           \pstart\center{}Mein lieber Arthur!\pend\pstart
           Das iſt der Unterſchied zwiſchen Freundſchafts- und Liebescorreſpondenz: die Liebe
               will Gefühle, und die Freundſchaft wird durch Gefühle auf die Dauer gelangweilt und
               will Thatſachen. Dieſe Wochen, in denen ich Dir nicht geſchrieben, ſollten alſo eine
               kleine Thatſachen-Sparbüchſe ſein; und jetzt, wo ich meine Erſparniſſe in dieſer
               Beziehung nachſehe, finde ich nichts und kann Dir wieder nichts bieten als ein Paar
               ſchäbige Stimmungen und Empfindungen. Der Grund für den Thatſachenmangel iſt vor
               Allem der, daß ich die Hautpzeit des Tages allein auf meinem Zimmer und mit meiner
               Arbeit verbringe. Meine Empfehlungen habe ich wohl abgegeben, aber ſie haben zu
               nichts geführt; ausgeſuchte Höflichkeit überall; aber die Höflichkeit iſt ein gar
               matter Wärmeſpender; ſie erwärmt nicht mehr als ein flüchtiger Händedruck, und das
               Herz kann dabei erfrieren. Da und dort hat man mich zum Diner eingeladen, und war
               froh, als der eigenthümliche Geiſt, dem man Alles Zweimal ſagen mußte, um von ihm
               verſtanden zu werden, und der ſelbſt ein jämmerliches Stottern vorführte, die {\pb}Thür hinter ſich zumachte. Ein klein wenig näher –
               aber auch nichts weniger als intim – verkehr ich mit einem jungen \label{K_L02664-5v}\edtext{\textcolor{blue}{Manne}{}\ledrightnote{→\textcolor{blue}{?? [Leiter einer Glasfabrik]}}}{\lemma{\textnormal{\emph{Manne}}}\Cendnote{\textnormal{nicht identifiziert}}}\label{K_L02664-5h} (22 Jahre),
                  \textcolor{blue}{Erbe}{}\ledrightnote{→\textcolor{blue}{?? [Leiter einer Glasfabrik]}} und \textcolor{blue}{Leiter}{}\ledrightnote{→\textcolor{blue}{?? [Leiter einer Glasfabrik]}} einer großen \label{K_L02664-6v}\edtext{\textcolor{brown}{Glasfabrik}{}\ledrightnote{→\textcolor{brown}{?? [Glasfabrik in Belgien]}}}{\lemma{\textnormal{\emph{Glasfabrik}}}\Cendnote{\textnormal{nicht ermittelt}}}\label{K_L02664-6h}; demgemäß ein
               wenig ſtolz und \label{K_L02664-1v}\edtext{\textsc{\begin{otherlanguage}{french}hautain\end{otherlanguage}}}{\lemma{\textnormal{\emph{hautain}}}\Cendnote{\textnormal{französisch: hochmütig, unnahbar}}}\label{K_L02664-1h},
               aber wohlerzogen genug, um das dem ihm warm empfohlenen Fremden nicht zu zeigen. Im
               Allgemeinen ein ſehr hübſcher, \strikeout{\textcolor{gray}{aſ}} äſthetiſch angenehmer \textcolor{blue}{Menſch}{}\ledrightnote{→\textcolor{blue}{?? [Leiter einer Glasfabrik]}} – eine Art \textsc{\textcolor{blue}{Boris Fanjung}{}\ledrightnote{\textcolor{blue}{Boris Van-Jung}}}, nur viel feiner und hochſtehender. Ein wenig \textcolor{blue}{Kunſtdilettant}{}\ledrightnote{→\textcolor{blue}{?? [Leiter einer Glasfabrik]}} und reizend, wenn er ſeine
               naiven Pläne entwickelt »\label{K_L02664-2v}\edtext{\textsc{\begin{otherlanguage}{french}de joindre l’art à l’industrie\end{otherlanguage}}}{\lemma{\textnormal{\emph{de … l’industrie}}}\Cendnote{\textnormal{französisch: die Kunst mit der Industrie
                  zu verbinden}}}\label{K_L02664-2h}«. Vor Allem aber – ſtrenggläubiger \textcolor{blue}{Katholik}{}\ledrightnote{→\textcolor{blue}{?? [Leiter einer Glasfabrik]}}, der allſonntäglich zur Meſſe geht
               und ſich auf nichts in der Welt mehr freut, als auf ſein Fortleben nach dem Tode.
               Dazu eine blonde, äußerlich unbedeutende, ſehr fromme, \strikeout{und} ſehr ſanfte und ſehr kurzſichtige \label{K_L02664-7v}\edtext{\textcolor{blue}{Schweſter}{}\ledrightnote{→\textcolor{blue}{?? [Schwester eines Glasfabrikanten]}}}{\lemma{\textnormal{\emph{Schweſter}}}\Cendnote{\textnormal{nicht identifiziert}}}\label{K_L02664-7h} mit einem
               ewigen \label{K_L02664-3v}\edtext{Lorgnon}{\lemma{\textnormal{\emph{Lorgnon}}}\Cendnote{\textnormal{Brille mit Haltestiel}}}\label{K_L02664-3h} und mit Redensarten wie »\label{K_L02664-4v}\edtext{\textsc{\begin{otherlanguage}{french}Jésus es mon ami intime\end{otherlanguage}}}{\lemma{\textnormal{\emph{Jésus es mon ami intime}}}\Cendnote{\textnormal{französisch: Jesus ist mein enger
                  Freund}}}\label{K_L02664-4h}«. Fürſtlicher Haushalt, nicht ohne Stimmung das Ganze – aber doch
               ohne rechte Wärme{\dots} Außerdem iſt da in \textcolor{pink}{Brüſſel}{}\ledrightnote{\textcolor{pink}{Brüssel}} der \textcolor{blue}{Chefredacteur}{}\ledrightnote{→\textcolor{blue}{Charles Tardieu}} der »\textsc{\textcolor{brown}{Indépendance belge}{}\ledrightnote{\textcolor{brown}{L’Indépendance Belge}}}« (Geograph wie Du biſt, wirſt Du fragen, {\pb}wieſo \textcolor{pink}{Brüſſel}{}\ledrightnote{\textcolor{pink}{Brüssel}} zu \textcolor{pink}{Belgien}{}\ledrightnote{\textcolor{pink}{Belgien}} kommt, aber ich kann Dir verrathen, daß es die \textcolor{pink}{Hauptſtadt}{}\ledrightnote{→\textcolor{pink}{Brüssel}} davon iſt). Dieſer also, \textsc{\textcolor{blue}{M. Tardieu}{}\ledrightnote{\textcolor{blue}{Charles Tardieu}}}, iſt ein durchaus charmanter \textcolor{blue}{Menſch}{}\ledrightnote{→\textcolor{blue}{Charles Tardieu}}, der einzige echte \textcolor{blue}{Franzoſe}{}\ledrightnote{→\textcolor{blue}{Charles Tardieu}}, den ich bisher kennen gelernt, \textcolor{blue}{Cavalier}{}\ledrightnote{→\textcolor{blue}{Charles Tardieu}}, unermüdlicher und geiſtſprühender
                  \textcolor{blue}{Plauderer}{}\ledrightnote{→\textcolor{blue}{Charles Tardieu}} und profunder
                  \textcolor{blue}{Kunſtkenner}{}\ledrightnote{→\textcolor{blue}{Charles Tardieu}}, \textcolor{blue}{Specialist}{}\ledrightnote{→\textcolor{blue}{Charles Tardieu}} für \textcolor{pink}{niederländiſche}{}\ledrightnote{\textcolor{pink}{Niederlande}} Malerei und \label{K_L02664-10v}\edtext{enragirter}{\lemma{\textnormal{\emph{enragirter}}}\Cendnote{\textnormal{begeisteter}}}\label{K_L02664-10h}\textcolor{blue}{Wagner}{}\ledrightnote{\textcolor{blue}{Richard Wagner}}ianer. Der \textcolor{blue}{Chefredacteur}{}\ledrightnote{→\textcolor{blue}{Charles Tardieu}} der »\textsc{\textcolor{brown}{Indépendance}{}\ledrightnote{\textcolor{brown}{L’Indépendance Belge}}}« iſt natürlich in \textcolor{pink}{Brüſſel}{}\ledrightnote{\textcolor{pink}{Brüssel}} ein großer \textcolor{blue}{Mann}{}\ledrightnote{→\textcolor{blue}{Charles Tardieu}} – wenn \introOben{}er\introOben{} auch von dem Größenwahn der \textcolor{pink}{Wien}{}\ledrightnote{\textcolor{pink}{Wien}}er Zeitungsſ\textcolor{gray}{a}ujuden keine Spur beſitzt – und hat
               Beſſeres zu thun, als mit dem Correſpondenten der »\textcolor{brown}{Frankfurter Zeitung}{}\ledrightnote{\textcolor{brown}{Frankfurter Zeitung}}« zu verkehren; aber alle 8 Tage ergibt ſich doch eine
               Plauder-Viertelſtunde auf ſeiner \textcolor{brown}{Redaction}{}\ledrightnote{→\textcolor{brown}{L’Indépendance Belge}}sſtube, die ich dann immer höchlich angeregt verlaſſe. Und dann iſt
                  \textcolor{pink}{Brüſſel}{}\ledrightnote{\textcolor{pink}{Brüssel}} ſelbſt – elegante und ſympathiſche \textcolor{pink}{Stadt}{}\ledrightnote{→\textcolor{pink}{Brüssel}}. Schöne Leute. Und vor
               Allem eine große hiſtoriſche Vergangenheit – die gewiſſe gothiſche Bettdecke, die man
               ſich über die \strikeout{Ohr} Ohren zieht, wenn man von der
               Gegenwart nichts wiſſen will. {\pb}Viel Kunst –
               herrliche alte und elende neue: Ein \textcolor{brown}{Muſeum}{}\ledrightnote{→\textcolor{brown}{Musées royaux des Beaux-Arts de Belgique}} mit \textsc{\textcolor{blue}{Rubens}{}\ledrightnote{\textcolor{blue}{Peter Paul Rubens}}} und \textsc{\textcolor{blue}{Jordaens}{}\ledrightnote{\textcolor{blue}{Jacob Jordaens}}}, wie ich ſie ſo ſchön noch nirgend geſehen und die mich gründlich v\substVorne{}\textsuperscript{\textcolor{gray}{or}}\substDazwischen{}om\substHinten{} »Modernen« kurirt haben, ſo daß ich allmälig anfange, mir die Gegenwart
               abzugewöhnen. Kurzum: Eindrücke genug; aber doch der ewig wiederkehrende Grundton,
               der in Alles hineinſummt: ſremd, fremd und fremd! Ach, mein liebes \textcolor{pink}{Wien}{}\ledrightnote{\textcolor{pink}{Wien}}! {\dotsfive}\pend
           \pstart
           Und zu thun habe ich! Du ſelbſt wirſt zwar kaum meine Arbeiten verfolgen können, was
               ich im Übrigen ganz begreiflich finde. Soviel ich mich erinnere, haſt Du nie eine
               beſondere Vorliebe für \textcolor{pink}{belgiſche}{}\ledrightnote{\textcolor{pink}{Belgien}} Politik
               beſeſſen. Und was die Feuilletons anlangt, die ich ſchreibe, die ſollſt Du erſt nicht
               leſen, weil ſie eh’ nichts taugen. Aber immerhin, es gibt gewaltige Arbeit. Allein
               die Lectüre der 14 freitäglich erſcheinenden großen Blätter nimmt mir vier bis fünf
               Stunden pro Tag. Aber die Arbeit iſt gut – Du weißt ja, nicht? – und jetzt beſonders,
               denn ſie richtet ſich als eine \label{K_L02664-8v}\edtext{\textcolor{pink}{ſpaniſche}{}\ledrightnote{\textcolor{pink}{Spanien}} Wand}{\lemma{\textnormal{\emph{ſpaniſche Wand}}}\Cendnote{\textnormal{bewegliche Wand zur Raumtrennung}}}\label{K_L02664-8h} auf, die mir das {\pb}ewig unzufriedene, traurige und hoffnungsloſe
               Geſicht eines eigenen Selbſt verbirgt {\dots} Fürchterliche
               Schwierigkeiten macht mir die Sprache. Seit ich hier bin, habe ich nicht eine Sylbe
               zugelernt. Und wenn man in der Regel ſagt, man ſolle in ein fremdes Land gehen, um
               die fremde Sprache zu lernen, ſo ſage ich dementgegen aus eigener Erfahrung, daß der
               Aufenthalt im fremden Land nur dazu nütze iſt, Einen von Woche zu Woche mehr zu
               überzeugen, daß man von der fremden Sprache keinen Dunſt hat und nie einen bekommen
                  wird{\dotsfour}\pend
           \pstart
           Ja richtig, der \label{K_L02664-9v}\edtext{Koffer}{\lemma{\textnormal{\emph{Koffer}}}\Cendnote{\textnormal{\textcolor{blue}{Goldmann} dürfte bei \textcolor{blue}{Schnitzler} für die Reise nach \textcolor{pink}{Frankfurt} einen Koffer ausgeliehen haben.}}}\label{K_L02664-9h}! Damit iſt
               es mir gut gegangen. Ich laſſe ihn in \textcolor{pink}{Frankfurt}{}\ledrightnote{\textcolor{pink}{Frankfurt am Main}}
               und bitte meine \textcolor{blue}{Mutter}{}\ledrightnote{→\textcolor{blue}{Clementine Goldmann}}, ihn
               Dir zu überſenden. Meine \textcolor{blue}{Mutter}{}\ledrightnote{→\textcolor{blue}{Clementine Goldmann}}, die in’s Land geht, vergißt im Eifer der Reiſe. Und mein \textcolor{blue}{Onkel}{}\ledrightnote{→\textcolor{blue}{Fedor Mamroth}} ſchreibt mir dieſer
               Tage: er habe mir den Koffer, den ich in \textcolor{pink}{Frankfurt}{}\ledrightnote{\textcolor{pink}{Frankfurt am Main}} gelaſſen, nach \textcolor{pink}{Brüſſel}{}\ledrightnote{\textcolor{pink}{Brüssel}}
               nachgeſchickt. Ich muß alſo wohl oder übel warten {\pb}bis der Koffer hier ankommt, und dann werde ich den Vielgereiſten ſofort nach \textcolor{pink}{Wien}{}\ledrightnote{\textcolor{pink}{Wien}} ſpediren. Sei mir nicht böſe, bitte, deswegen!
               Haſt Du irgend einen Wunſch, bezüglich irgend eines Gegenſtandes, den man bei dieſer
               Gelegenheit in \textcolor{pink}{Brüſſel}{}\ledrightnote{\textcolor{pink}{Brüssel}} erwerben und mitſchicken
               könnte? Litteratur, Kunſt, Muſik, Crawatten, Eßwaren oder ſo etwas? Bitte, denke
               nach. Mir iſt leid darum, den Koffer leer zu expediren{\dotsfour}\pend
           \pstart
           Und nun bekomme ich wohl einen recht langen Brief? Befinden, Arbeiten, Verkehr,
               Stimmung, Sommerpläne, Tages- und Abendeintheilung \textsc{etc}. Ich
               bin heißhungrig nach jedem Biſſen Neuigkeit von Dir, von \textcolor{pink}{Wien}{}\ledrightnote{\textcolor{pink}{Wien}} und den anderen Freunden. »\textsc{\textcolor{blue}{Es}{}\ledrightnote{→\textcolor{blue}{Marie Glümer}}}« iſt in \textcolor{pink}{Brünn}{}\ledrightnote{\textcolor{pink}{Brünn}}? Und \textsc{Madame \textcolor{blue}{Olga}{}\ledrightnote{\textcolor{blue}{Olga Waissnix}}}? Ich kann Dir ſagen, die echten \textsc{\begin{otherlanguage}{french}Mondainen\end{otherlanguage}}, die man hier sieht, ſehen doch noch ganz anders aus{\dots} Bitte grüße vielma{[}l{]}s \textsc{\textcolor{blue}{Kapper}{}\ledrightnote{\textcolor{blue}{Friedrich Kapper}}}, \textsc{\textcolor{blue}{Beer-Hofmann}{}\ledrightnote{\textcolor{blue}{Richard Beer-Hofmann}}} und \textsc{\textcolor{blue}{Loris}{}\ledrightnote{\textcolor{blue}{Hugo von Hofmannsthal}}}. Und ſei Du ſelbſt gegrüßt, von Herzen und in Treue!\pend
           \pstart
           Dein {\\[\baselineskip]}\spacefill\mbox{Paul Goldmann.}\pend
           \leftskip=0em{}\pstart
           \noindent{}\uline{Adreſſe umſtehend:}\pend
           \pstart
           {\pb}\textsc{\textcolor{pink}{Brüssel – St. Josse ten Noode, 21. rue des
                        plantes}{}\ledrightnote{\textcolor{pink}{rue des Plantes}}}.\pend
           \pstart
           Meine ergebenen Empfehlungen an die Deinen!\pend
           \endnumbering\briefempfaengerindex{Schnitzler, Arthur@\textsc{Schnitzler, Arthur}!zzzGoldmann, Paul@\emph{von Paul Goldmann}!1891-06-091@{9. 6. {[}1891{]}}|)be}\mylabel{h}  
         \normalsize

\newenvironment{esempio}[3]%
{
    \vspace{1.5ex}
    \rlap{\underline{#1}}
    \par
    \setlength{\parindent}{0cm}
    \nopagebreak
    \leftskip=#2cm
    \rightskip=#3cm
}
{
    \par
}

\doendnotes{C}
\bigskip

\printindex[pw]


\end{document}
      