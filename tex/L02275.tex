%% latex-korrekturansicht-vorspann.tex
%% Vorspann für die Korrekturansicht.
%% Lädt die gemeinsame Datei latex-vorspann.tex mit gesetztem Schalter.

\newif\ifkorrekturansicht
\korrekturansichttrue

\input{../tex-inputs/latex-vorspann}


               \section[Robert Adam an Arthur Schnitzler, 12. 10. 1917]{ Robert Adam an Arthur Schnitzler, 12. 10. 1917}\nopagebreak\mylabel{v}\rehead{ }\normalsize\beginnumbering\briefempfaengerindex{Schnitzler, Arthur@\textsc{Schnitzler, Arthur}!zzzAdam, Robert@\emph{von Robert Adam}!1917-10-121@{12. 10. 1917}|(be} \toendnotes[C]{\smallbreak\pagebreak[2]} \Standort{DLA, A:Schnitzler, HS.NZ85.1.4230,21.}
\physDesc{Brief, 1 Blatt, 3 Seiten
\newline{}Handschrift: schwarze Tinte, deutsche Kurrent
\newline{}Schnitzler: 1) mit Bleistift beschriftet: »\textsc{Adam}« 2) mit rotem Buntstift eine Unterstreichung}\Standort{Wien, Österreichische Nationalbibliothek, Cod.ser. 52.263, 202.}
\physDesc{Brief, maschinelle Abschrift
\newline{}Schreibmaschine}\pstart
           \raggedleft{}{\pb}\textcolor{pink}{Wien}{}\ledrightnote{\textcolor{pink}{Wien}}, am 12. Oktober
                        1917.\pend
           \pstart{}Hochverehrter Herr Doktor!\pend\pstart
           Ich überſende Ihnen (da ich glaube, daß Sie es mir geſtatten) meine jüngſte
                    Tragikomödie, »\textcolor{green}{Juda}{}\ledrightnote{\textcolor{green}{Das Ende des Judas}}«, die ſoeben
                    fertiggewordene Arbeit des letzten Halbjahres, mit der Bitte, ſie zu leſen, und
                    mit der Bitte um Rat, was ich damit anfangen ſoll. Ich habe das Gefühl, daß es
                    das erſte Theaterſtück iſt, das ich geſchrieben habe; ob es, mit meinen anderen
                    Arbeiten verglichen, einen Fortſchritt bedeutet oder aber einen Rückſchritt, das
                    kann ich ſelbſt, und gar jetzt ſchon, nicht beurteilen. Bühnenwirkſam dürfte es
                    ſein, wenigſtens in ſeiner zweiten Hälfte; aber ob nicht mein Stoff {\pb}knabenhaft-töricht iſt, fragen immer wieder
                    nicht zu widerlegende Skrupel (denen allerdings eine dem Milieu des Stückes
                    gemäße Gegenfrage zu antworten weiß: welcher Theaterſtoff iſt nicht kindiſch?)
                    Mit einem Worte: ich ſtehe meiner Arbeit nun, da ſie vollendet iſt, mit ſehr
                    ſchwankenden Gefühlen und urteilslos gegenüber.\pend
           \pstart
           So bin ich auf den erſten Eindruck, den ſie auf Sie, hochverehrter Herr Doktor,
                    machen wird, ſehr geſpannt und ſehe Ihrem Urteil, das Sie mir ja wohl nicht
                    weigern werden, mit Angſt und Beben entgegen. Iſt das Ganze als Ganzes etwas
                    wert oder nicht? Daß mir gewiſſe Einzelheiten nicht mißlungen ſind, glaube ich
                    allerdings. –\pend
           \pstart
           Und wenn das Stück etwas wert {\pb}ſein ſollte: ſoll
                    ich’s dem \textcolor{brown}{Burgtheater}{}\ledrightnote{\textcolor{brown}{Burgtheater}} und dem \textcolor{brown}{Münchner Hoftheater}{}\ledrightnote{\textcolor{brown}{Königliche Hof- und Nationaltheater München}} einreichen? oder ſoll ich mein Heil bei
                    akatholiſchen Theatern ſuchen?\pend
           \pstart
           Wenn ich wenigſtens zur »jungen Generation« gehörte! Aber ach! ich darf mich
                    nicht mehr zu ihr zählen (und Gott möge mich vor ſolchem bewahren!) und zur
                    »alten Generation« habe ich auch nicht mehr gehört. Wo ſoll ich ein Plätzlein an
                    der Sonne ſuchen? –\pend
           \pstart
           Indem ich Sie bitte, mir die 180 Seiten lange Einſendung nicht zu verübeln,
                    verbleibe ich mit den ergeben ten Grüßen Ihr{\\[\baselineskip]}\spacefill\mbox{Robert Adam}\pend
           \leftskip=0em{}\endnumbering\briefempfaengerindex{Schnitzler, Arthur@\textsc{Schnitzler, Arthur}!zzzAdam, Robert@\emph{von Robert Adam}!1917-10-121@{12. 10. 1917}|)be}\mylabel{h}  \normalsize

\doendnotes{C}
\bigskip
\vfill

\clearpage

\footnotesize

\lohead{\textsc{register}}

% Definiere theindex-Environment komplett neu ohne reledmac
\makeatletter
\renewenvironment{theindex}{%
  \section*{\indexname}%
  \setlength{\parindent}{0pt}%
  \setlength{\parskip}{0pt plus 0.3pt}%
  \let\item\@idxitem
}{%
  \clearpage
}
\makeatother

\IfFileExists{\jobname-pw.ind}{\input{\jobname-pw.ind}}{}

\end{document}

      