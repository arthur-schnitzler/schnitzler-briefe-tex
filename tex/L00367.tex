%% latex-korrekturansicht-vorspann.tex
%% Vorspann für die Korrekturansicht.
%% Lädt die gemeinsame Datei latex-vorspann.tex mit gesetztem Schalter.

\newif\ifkorrekturansicht
\korrekturansichttrue

\input{../tex-inputs/latex-vorspann}


               \section[Richard Beer-Hofmann an Arthur Schnitzler, 7. 9. 1894]{ Richard Beer-Hofmann an Arthur Schnitzler, 7. 9. 1894}\nopagebreak\mylabel{v}\rehead{ }\normalsize\beginnumbering\briefempfaengerindex{Schnitzler, Arthur@\textsc{Schnitzler, Arthur}!zzzBeer-Hofmann, Richard@\emph{von Richard Beer-Hofmann}!1894-09-071@{7. 9. 1894}|(be} \toendnotes[C]{\smallbreak\pagebreak[2]} \Standort{CUL, Schnitzler, B 8.}
\physDesc{Brief, 1 Blatt, 2 Seiten
\newline{}Handschrift: Bleistift, lateinische Kurrent
\newline{}Schnitzler: mit Bleistift nummeriert: »32« }\buchAbdrucke{\weitereDrucke{1) Arthur Schnitzler, Richard Beer-Hofmann: \emph{Briefwechsel 1891–1931}. Hg. Konstanze Fliedl. Wien, Zürich: \emph{Europaverlag} 1992, S. 58–59.} \weitereDrucke{2) Hermann Bahr, Arthur Schnitzler: \emph{Briefwechsel, Aufzeichnungen, Dokumente
                                (1891–1931)}. Hg. Kurt Ifkovits und Martin Anton Müller. Göttingen: \emph{Wallstein} 2018.} }\toendnotes[C]{\smallbreak}\pstart
           \noindent{}{\pb}Lieber Arthur!
                    Ich habe eine Menge Bitten an Sie.\pend
           \pstart
           I. Senden Sie mir unter Kreuzband den \textcolor{blue}{\textcolor{green}{Bolgar}{}\ledrightnote{→\textcolor{green}{Die Regeln des Duells}}}{}\ledrightnote{\textcolor{blue}{Franz von Bolgár}}, ich nehme ihn auf die Reise mit.\pend
           \pstart
           II. Fragen Sie telefonisch bei \textcolor{blue}{Paul Horn}{}\ledrightnote{\textcolor{blue}{Paul Horn}} an
                    ob es geht daß ich \strikeout{Dinge an} falls ich
                    zollpflichtige Sachen \strikeout{an} von \textcolor{pink}{Italien}{}\ledrightnote{\textcolor{pink}{Italien}} herübersenden sollte ich sie adressiren
                    kann an Herrn \uline{\textcolor{blue}{Paul
                            Horn}{}\ledrightnote{\textcolor{blue}{Paul Horn}}} p. Adr. \textcolor{brown}{\uline{Schenker u.} Co}{}\ledrightnote{\textcolor{brown}{Schenker {\kaufmannsund} Co.}} und ob dann \textcolor{brown}{Schenkers}{}\ledrightnote{\textcolor{brown}{Schenker {\kaufmannsund} Co.}} die Verzollung\introOben{}sarbeiten\introOben{} etc. \strikeout{er} übernehmen. Weil
                    ich nicht wegen meines \textcolor{blue}{Papa}{}\ledrightnote{→\textcolor{blue}{Hermann Beer}}’s die Sachen
                    (Moritz gehste herunter vom Bock) an mich adressiren kann, und ich denke daß es
                    ihm \introOben{}\textcolor{blue}{Paul Horn}{}\ledrightnote{\textcolor{blue}{Paul Horn}} od
                            \textcolor{blue}{Schenker}{}\ledrightnote{\textcolor{blue}{Gottfried Schenker}}\introOben{} eben weniger
                    Scherereien macht. Wie ist die Adresse von \textcolor{blue}{Paul
                        Horn}{}\ledrightnote{\textcolor{blue}{Paul Horn}} und wie die der \uline{Firma}{ }\textcolor{brown}{Schenker}{}\ledrightnote{\textcolor{brown}{Schenker {\kaufmannsund} Co.}}? –\pend
           \pstart
           {\pb}III. Grüße à
                    Discretion.\pend
           \pstart
           IV. Bitten Sie \textcolor{blue}{Bahr}{}\ledrightnote{\textcolor{blue}{Hermann Bahr}} er möchte die Nummern der
                    »\textcolor{green}{Zeit}{}\ledrightnote{\textcolor{green}{Die Zeit. Wiener Wochenschrift}}« mir nachsenden ich werde meine
                    Adresse ihm bekannt geben. Ich abonnire natürlich.\pend
           \pstart
           V. Danke ich für alle Scherereien die Sie mit mir haben.\pend
           \pstart
           Genaue Route, Tag der Abreise gebe ich Ihnen noch bekannt.\pend
           \pstart
           Herzlichst Ihr{\\[\baselineskip]}\spacefill\mbox{Richard}\pend
           \leftskip=0em{}\pstart
           7 Sept 94{ }\textcolor{pink}{Ischl}{}\ledrightnote{\textcolor{pink}{Bad Ischl}}\pend
           \pstart
           Wie ist die \uline{Adresse} der \textcolor{blue}{\introOben{}Adele\introOben{}
                     Sandrock}{}\ledrightnote{\textcolor{blue}{Adele Sandrock}}? \pend
           \endnumbering\briefempfaengerindex{Schnitzler, Arthur@\textsc{Schnitzler, Arthur}!zzzBeer-Hofmann, Richard@\emph{von Richard Beer-Hofmann}!1894-09-071@{7. 9. 1894}|)be}\mylabel{h}  \normalsize

\doendnotes{C}
\bigskip
\vfill

\clearpage

\footnotesize

\lohead{\textsc{register}}

% Definiere theindex-Environment komplett neu ohne reledmac
\makeatletter
\renewenvironment{theindex}{%
  \section*{\indexname}%
  \setlength{\parindent}{0pt}%
  \setlength{\parskip}{0pt plus 0.3pt}%
  \let\item\@idxitem
}{%
  \clearpage
}
\makeatother

\IfFileExists{\jobname-pw.ind}{\input{\jobname-pw.ind}}{}

\end{document}

      