%% latex-korrekturansicht-vorspann.tex
%% Vorspann für die Korrekturansicht.
%% Lädt die gemeinsame Datei latex-vorspann.tex mit gesetztem Schalter.

\newif\ifkorrekturansicht
\korrekturansichttrue

\input{../tex-inputs/latex-vorspann}


               \section[Gertrud Rung an Arthur Schnitzler, 24. 5. 1925]{ Gertrud Rung an Arthur Schnitzler, 24. 5. 1925}\nopagebreak\mylabel{v}\rehead{ }\normalsize\beginnumbering\briefempfaengerindex{Schnitzler, Arthur@\textsc{Schnitzler, Arthur}!zzzRung, Gertrud@\emph{von Gertrud Rung}!1925-05-241@{24. 5. 1925}|(be} \toendnotes[C]{\smallbreak\pagebreak[2]} \Standort{CUL, Schnitzler, B 17.}
\physDesc{Brief, 1 Blatt, 2 Seiten
\newline{}Handschrift: schwarze Tinte, lateinische Kurrent
\newline{}Schnitzler: mit Bleistift beschriftet: »\noindent{}\textsc{Brandes}{ / }\textsc{(Rung}{[}){]}« \newline{}Ordnung: mit Bleistift von unbekannter Hand nummeriert:
                                        »58« }\buchAbdrucke{\weitereDrucke{Georg Brandes, Arthur Schnitzler: \emph{Ein Briefwechsel}. Hg. Kurt Bergel. Bern: \emph{Francke} 1956, S. 146.} }\pstart
           \raggedleft{}{\pb}\textcolor{pink}{Oesterreichischer Hof, Salzburg}{}\ledrightnote{\textcolor{pink}{Österreichischer Hof}}{\\}24/5. 25\pend
           \pstart{}Hochverehrter Herr Dr Schnitzler.\pend\pstart
           Dr \textcolor{blue}{Brandes}{}\ledrightnote{\textcolor{blue}{Georg Brandes}} dankt Ihnen ergebenst für Ihren
                    freundlichen Brief. Wie Sie wahrscheinlich aus den Zeitungen erfahren haben,
                    erkrankte Dr \textcolor{blue}{Brandes}{}\ledrightnote{\textcolor{blue}{Georg Brandes}} gleich nach seiner
                    Ankunft hier an Bronchitis, und es sah für ein paar Tage recht ernst aus, aber
                    glücklicherweise ist es gut gegangen, die Krankheit ist beinahe vorüber und
                    Morgen {\pb}wird er, wenn das
                    Wetter schön bleibt, spazieren fahren.\pend
           \pstart
           Mit Ausnahme der ersten Woche hat die Sonne jeden Tag von einem wolkenlosen
                    Himmel niedergeschienen, und \textcolor{pink}{Salzburg}{}\ledrightnote{\textcolor{pink}{Salzburg}} hat sich
                    in aller ihrer Schönheit dargeboten; die Stadt ist ja entzückend und ich hoffe,
                    daß Dr \textcolor{blue}{Brandes}{}\ledrightnote{\textcolor{blue}{Georg Brandes}} bald im Stande sein wird
                    kleinere Ausflüge zu machen und etwas von der Schönheit zu genießen.\pend
           \pstart
           Dr Brandes beauftragt mich Sie {\pb}zu sagen, daß auch für ihn war das Zusammensein mit Ihnen, hochverehrter Herr
                    Doktor, eine große Freude, und daß er sich bei Ihnen außerordentlich wohl
                    befunden habe. Er würde sich sehr freuen wenn Sie, wie Sie andeuteten, im Herbst
                    nach \textcolor{pink}{Kopenhagen}{}\ledrightnote{\textcolor{pink}{Kopenhagen}} kämen.\pend
           \pstart
           Ich möchte gern die Gelegenheit benützen und Ihnen, verehrter und lieber Herr
                    Doktor, vom Herzen danken für die schönen Stunden die ich bei Ihnen
                    verbrachte.\pend
           \pstart
           Mit besten Grüßen von Dr \textcolor{blue}{Brandes}{}\ledrightnote{\textcolor{blue}{Georg Brandes}}
                        und\hspace*{2.5em}Ihrer{\\[\baselineskip]}\spacefill\mbox{Gertrud Rung}\pend
           \leftskip=0em{}\endnumbering\briefempfaengerindex{Schnitzler, Arthur@\textsc{Schnitzler, Arthur}!zzzRung, Gertrud@\emph{von Gertrud Rung}!1925-05-241@{24. 5. 1925}|)be}\mylabel{h}  \normalsize

\doendnotes{C}
\bigskip
\vfill

\clearpage

\footnotesize

\lohead{\textsc{register}}

% Definiere theindex-Environment komplett neu ohne reledmac
\makeatletter
\renewenvironment{theindex}{%
  \section*{\indexname}%
  \setlength{\parindent}{0pt}%
  \setlength{\parskip}{0pt plus 0.3pt}%
  \let\item\@idxitem
}{%
  \clearpage
}
\makeatother

\IfFileExists{\jobname-pw.ind}{\input{\jobname-pw.ind}}{}

\end{document}

      