%% latex-korrekturansicht-vorspann.tex
%% Vorspann für die Korrekturansicht.
%% Lädt die gemeinsame Datei latex-vorspann.tex mit gesetztem Schalter.

\newif\ifkorrekturansicht
\korrekturansichttrue

\input{../tex-inputs/latex-vorspann}


               \section[Richard Beer-Hofmann an Arthur Schnitzler, 20. 10. 1894]{ Richard Beer-Hofmann an Arthur Schnitzler, 20. 10. 1894}\nopagebreak\mylabel{v}\rehead{ }\normalsize\beginnumbering\briefempfaengerindex{Schnitzler, Arthur@\textsc{Schnitzler, Arthur}!zzzBeer-Hofmann, Richard@\emph{von Richard Beer-Hofmann}!1894-10-202@{20. 10. 1894}|(be} \toendnotes[C]{\smallbreak\pagebreak[2]} \Standort{CUL, Schnitzler, B 8.}
\physDesc{Brief, 1 Blatt, 4 Seiten
\newline{}Handschrift: Bleistift, lateinische Kurrent
\newline{}Schnitzler: mit Bleistift beschriftet: »\textsc{Bajae} 20 Oct 94« und nummeriert:
            »50« }\buchAbdrucke{\weitereDrucke{Arthur Schnitzler, Richard Beer-Hofmann: \emph{Briefwechsel 1891–1931}. Hg. Konstanze Fliedl. Wien, Zürich: \emph{Europaverlag} 1992, S. 65–66.} }\toendnotes[C]{\smallbreak}\pstart
           \noindent{}{\pb}Lieber Arthur!
                    Gerade, wie ich in den Wagen steige, bekomme ich Ihre Karte. Meinen
                    Brief \strikeout{ha} und Karte haben Sie wohl?\pend
           \pstart
           \uline{Das}
               schreibe ich beim schwarzen Kaffee auf einer
                    Terrasse am Meer in \uline{\textcolor{pink}{Bajae}{}\ledrightnote{\textcolor{pink}{Baia}}} – (Bitte lesen Sie zu Hause über \textcolor{pink}{Bajae}{}\ledrightnote{\textcolor{pink}{Baia}}
                    nach.) Abends bin ich wieder in \textcolor{pink}{Neapel}{}\ledrightnote{\textcolor{pink}{Neapel}}, dann
                    morgen und die nächsten Tage \textcolor{pink}{Capri}{}\ledrightnote{\textcolor{pink}{Capri}},
                        \textcolor{pink}{Sorrent}{}\ledrightnote{\textcolor{pink}{Sorrent}} dann \textcolor{pink}{Venedig}{}\ledrightnote{\textcolor{pink}{Venedig}}. Adressiren Sie bitte Briefe und die 4. Nr. der
                        \textcolor{brown}{Zeit}{}\ledrightnote{\textcolor{brown}{Die Zeit. Wiener Wochenschrift}} nach \textcolor{pink}{Venedig,
                            \uline{Bauer und Grünwald}}{}\ledrightnote{\textcolor{pink}{Grand Hotel Bauer-Grünwald}}. – Die 1te und 2. Nu{\geminationm}er habe ich; 3\textsuperscript{te} erwarte ich. {\pb}À propos (warum à propos,
                    warum fällt mir das jetzt ein?) was stand auf den in Verlust gerathenen
                        \textcolor{pink}{Pallanza}{}\ledrightnote{\textcolor{pink}{Pallanza}}er Karten? \textcolor{blue}{Bahr}{}\ledrightnote{\textcolor{blue}{Hermann Bahr}} bitte grüßen Sie herzlich, und der »\label{K_L00388_1v}\edtext{\textcolor{green}{Abonnent}{}\ledrightnote{\textcolor{green}{Der Abonnent}}}{\lemma{\textnormal{\emph{Abonnent}}}\Cendnote{\textnormal{\textcolor{blue}{Caph [= Hermann Bahr]}: \emph{\textcolor{green}{Der Abonnent}}. In: \emph{\textcolor{green}{Die Zeit}}, Bd. 1, Nr. 1, 6. 10. 1894,
                            S. 6–7.}}}\label{K_L00388_1h}« hat mir »\uline{wol}
                    getan«, und das »\label{K_L00388_2v}\edtext{\textcolor{green}{Burgtheater}{}\ledrightnote{\textcolor{green}{Der Abonnent}}}{\lemma{\textnormal{\emph{Burgtheater}}}\Cendnote{\textnormal{\textcolor{blue}{Hermann Bahr}: \emph{\textcolor{green}{Burgtheater}}. In: \emph{\textcolor{green}{Die Zeit}}, Bd. 1, Nr. 1, 6. 10. 1894,
                            S. 9–10.}}}\label{K_L00388_2h}« (\textcolor{blue}{Burkhard}{}\ledrightnote{\textcolor{blue}{Max Eugen Burckhard}})
                    war gescheidt \uline{und} diplomatisch. Und die
                        »\textcolor{green}{Schmetterlingsschlacht}{}\ledrightnote{\textcolor{green}{Die Schmetterlingsschlacht}}« hat er sich
                    teilweise eingeredet – ich kenne \strikeout{S}sie nicht, –
                    aber ich mißbillige \strikeout{S}sie. Kleine Probleme von
                    kleinen Warten und anstatt tiefster Auffassung des {\pb}Lebens bürgerlich-ideale
                    Moral auf dem Grunde; und die Belohnung \textcolor{gray}{×}\-\textcolor{gray}{×}\-\textcolor{gray}{×} guter Sitten in reicher Heirath, und die Versorgung, –
                    der Blick in die Zukunft.\pend
           \pstart
           Das Meer ist viel schöner. Und viele andere, viel kleinere Dinge auch. Lieber
                    Arthur, bitte schreiben Sie mir \uline{sehr sicher} nach
                        \textcolor{pink}{Venedig}{}\ledrightnote{\textcolor{pink}{Venedig}}, und viel; denn Sie würden
                    unendlich leiden unter dem Gedanken, wie peinlich ich es empfinden müsste in
                        \textcolor{pink}{Venedig}{}\ledrightnote{\textcolor{pink}{Venedig}} keinen Brief {\pb}zu finden, nachdem auf der
                    ganzen Fahrt dahin mich drauf gefreut habe.\pend
           \pstart
           Es gibt Studenten des jus in \textcolor{pink}{Prag}{}\ledrightnote{\textcolor{pink}{Prag}} die sehr
                    gut Lawn-Tennis spielen, nicht antisemitisch, gegen den \textcolor{brown}{deutschen Schulverein}{}\ledrightnote{\textcolor{brown}{Deutscher Schulverein}} und die Politik, und insbesondere den
                    Liberalismus sind; \textcolor{blue}{Maupassant}{}\ledrightnote{\textcolor{blue}{Guy de Maupassant}} lesen, den
                        \textcolor{blue}{Bahr}{}\ledrightnote{\textcolor{blue}{Hermann Bahr}} teilweise (\textcolor{green}{Dora}{}\ledrightnote{\textcolor{green}{Dora}}) kennen, und freudig erschauern wenn ich sage daß ich
                        \textcolor{blue}{Bahr}{}\ledrightnote{\textcolor{blue}{Hermann Bahr}} kenne (\uline{einen} gibt es \uline{sicher}). Die Leute die
                    heute 17 u. 19 sind, werden die sein die in 10 Jahren sich uns neigen werden –
                    oder früher? Das »uns« nehme ich \uline{principiell}
                    zurück. \spacefill\mbox{Richard.}\pend
           \endnumbering\briefempfaengerindex{Schnitzler, Arthur@\textsc{Schnitzler, Arthur}!zzzBeer-Hofmann, Richard@\emph{von Richard Beer-Hofmann}!1894-10-202@{20. 10. 1894}|)be}\mylabel{h}  \normalsize

\doendnotes{C}
\bigskip
\vfill

\clearpage

\footnotesize

\lohead{\textsc{register}}

% Definiere theindex-Environment komplett neu ohne reledmac
\makeatletter
\renewenvironment{theindex}{%
  \section*{\indexname}%
  \setlength{\parindent}{0pt}%
  \setlength{\parskip}{0pt plus 0.3pt}%
  \let\item\@idxitem
}{%
  \clearpage
}
\makeatother

\IfFileExists{\jobname-pw.ind}{\input{\jobname-pw.ind}}{}

\end{document}

      