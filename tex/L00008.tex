%% latex-korrekturansicht-vorspann.tex
%% Vorspann für die Korrekturansicht.
%% Lädt die gemeinsame Datei latex-vorspann.tex mit gesetztem Schalter.

\newif\ifkorrekturansicht
\korrekturansichttrue

\input{../tex-inputs/latex-vorspann}


               \section[Michael Georg Conrad an Arthur Schnitzler, 14. 11. 1890]{ Michael Georg Conrad an Arthur Schnitzler,
                    14. 11. 1890}\nopagebreak\mylabel{v}\rehead{ }\normalsize\beginnumbering\briefempfaengerindex{Schnitzler, Arthur@\textsc{Schnitzler, Arthur}!zzzConrad, Michael Georg@\emph{von Michael Georg Conrad}!1890-11-141@{14. 11. 1890}|(be} \toendnotes[C]{\smallbreak\pagebreak[2]} \Standort{CUL, Schnitzler, B 22.}
\physDesc{Postkarte
\newline{}Handschrift: schwarze Tinte, deutsche Kurrent\newline{}Versand: 1) Stempel: »\nobreak{}\oindex{Muenchen@\textbf{München}, \emph{https://www.geonames.org/ontologyP.PPLA}|pwk}München I, 14 Nov 90, 4–5 N\nobreak{}«.  2) Stempel: »\nobreak{}Wien, 15/11 90\nobreak{}«. \newline{}Ordnung: mit rotem Buntstift von unbekannter Hand nummeriert: »1« }\toendnotes[C]{\smallbreak}\pstart{}{\pb}Herrn Dr. Arthur
                        Schnitzler\pend{}\pstart{}\textcolor{pink}{Wien I.}{}\ledrightnote{\textcolor{pink}{I., Innere Stadt}}\pend{}\pstart{}\textcolor{pink}{Giſelaſtr. 11}{}\ledrightnote{\textcolor{pink}{Bösendorferstraße}}\pend{}{\bigskip}\pstart
           {\pb}\textcolor{pink}{München}{}\ledrightnote{\textcolor{pink}{München}},
                            14. 11. 90.\pend
           \pstart
           Das \textcolor{green}{Gedicht}{}\ledrightnote{→\textcolor{green}{Morgenandacht}} wird in der »\textcolor{green}{Geſellſchaft}{}\ledrightnote{\textcolor{green}{Die Gesellschaft. Monatsschrift}}« abgedruckt. Dank und Gruß!\pend
           \pstart \spacefill\mbox{Dr. Conrad.}\pend{}\endnumbering\briefempfaengerindex{Schnitzler, Arthur@\textsc{Schnitzler, Arthur}!zzzConrad, Michael Georg@\emph{von Michael Georg Conrad}!1890-11-141@{14. 11. 1890}|)be}\mylabel{h}  \normalsize

\doendnotes{C}
\bigskip
\vfill

\clearpage

\footnotesize

\lohead{\textsc{register}}

% Definiere theindex-Environment komplett neu ohne reledmac
\makeatletter
\renewenvironment{theindex}{%
  \section*{\indexname}%
  \setlength{\parindent}{0pt}%
  \setlength{\parskip}{0pt plus 0.3pt}%
  \let\item\@idxitem
}{%
  \clearpage
}
\makeatother

\IfFileExists{\jobname-pw.ind}{\input{\jobname-pw.ind}}{}

\end{document}

      