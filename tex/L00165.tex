%% latex-korrekturansicht-vorspann.tex
%% Vorspann für die Korrekturansicht.
%% Lädt die gemeinsame Datei latex-vorspann.tex mit gesetztem Schalter.

\newif\ifkorrekturansicht
\korrekturansichttrue

\input{../tex-inputs/latex-vorspann}


               \section[Hugo von Hofmannsthal an Arthur Schnitzler, {[}30.? 1. 1893{]}]{ Hugo von Hofmannsthal an Arthur Schnitzler, {[}30.? 1. 1893{]}}\nopagebreak\mylabel{v}\rehead{ }\normalsize\beginnumbering\briefempfaengerindex{Schnitzler, Arthur@\textsc{Schnitzler, Arthur}!zzzHofmannsthal, Hugo von@\emph{von Hugo von Hofmannsthal}!1893-01-302@{{[}30.? 1. 1893{]}}|(be} \toendnotes[C]{\smallbreak\pagebreak[2]} \Standort{CUL, Schnitzler, B 43.}
\physDesc{Briefkarte mit aufgeprägtem Wappen
\newline{}Handschrift: schwarze Tinte, deutsche Kurrent
\newline{}Schnitzler: mit Bleistift nummeriert: »37« }\buchAbdrucke{\weitereDrucke{Hugo von Hofmannsthal, Arthur Schnitzler: \emph{Briefwechsel}. Hg. Therese Nickl und Heinrich Schnitzler. Frankfurt am Main: \emph{S. Fischer} 1964, S. 33–34.} }\toendnotes[C]{\smallbreak}\pstart
           \raggedleft{}{\pb}\label{K_L00165_1v}\edtext{Montag}{\lemma{\textnormal{\emph{Montag}}}\Cendnote{\textnormal{Der 30. 1. 1893 war ein
                     Montag. Die Einordnung erfolgt durch das Antwortschreiben \textcolor{blue}{Schnitzler}s.}}}\label{K_L00165_1h}.\pend
           \pstart{}lieber Arthur.\pend\pstart
           Die Empfehlung \textcolor{blue}{Engländer}{}\ledrightnote{\textcolor{blue}{Peter Altenberg}}s ſehr gern beim nächſten
               Zuſammentreffen mit \textcolor{blue}{Berger}{}\ledrightnote{\textcolor{blue}{Alfred von Berger}}, was für eine Arbeit
               iſt es denn?\pend
           \pstart
           Über \textcolor{blue}{Fels}{}\ledrightnote{\textcolor{blue}{Friedrich Michael Fels}} höre ich unbeſtimmt erſchreckendes; ich
               werde Ihnen in den nächſten Tagen etwas ſchicken, eventuell ein paar Freunde ohne
               Namennennung um Mithilfe bitten; ſagen Sie mir doch, was wahr iſt. »\textcolor{green}{\uline{Familie}}{}\ledrightnote{\textcolor{green}{Familie}}«?!!\pend
           \pstart
           Ein herausgegriffenes Kapitel aus dem »\textcolor{green}{Kind}{}\ledrightnote{\textcolor{green}{Age of Innocence}}« hat
               mir einen ſtarken Eindruck gemacht; ich freue mich ſehr auf die Vollendung.\pend
           \pstart
           Das \textcolor{green}{Exemplar}{}\ledrightnote{→\textcolor{green}{Anatol}} für die \textcolor{brown}{akademiſche Vereinigung}{}\ledrightnote{\textcolor{brown}{Wiener Akademische Vereinigung}}{ }ſchicken Sie am tactvollſten in das \textcolor{pink}{Hôtel Wandel}{}\ledrightnote{\textcolor{pink}{Hotel Wandl}}{ }{\pb}mit der Weiſung, es am
                  Samstagabend dem \textcolor{blue}{Präſidenten}{}\ledrightnote{\textcolor{blue}{?? [Präsident der Akademischen Vereinigung]}} zu
               übergeben.\pend
           \pstart
           Der kleine \textcolor{blue}{\textsc{Teltſch}}{}\ledrightnote{\textcolor{blue}{Ede Telcs}} möchte auch gern eins haben. Vor einer Woche hat mir eine \label{K_L00165_2v}\edtext{\textcolor{blue}{\textcolor{pink}{Ruſſin}{}\ledrightnote{\textcolor{pink}{Russland}}}{}\ledrightnote{→\textcolor{blue}{?? [Russin]}}}{\lemma{\textnormal{\emph{Ruſſin}}}\Cendnote{\textnormal{vgl.: »Sonntag 22.{ / }Die beiden Russinnen.« (\textcolor{blue}{Hofmannsthal}: \emph{Aufzeichnungen}, S. 204).}}}\label{K_L00165_2h}, meine \textsc{Souper}nachbarin, ſehr von den »\textsc{proverbes de ce
                  monsieur, qui est en même temps médecin}«, \strikeout{gerſch} geſchwärmt.\pend
           \pstart
           Wann ſoll denn \textcolor{blue}{Salten}{}\ledrightnote{\textcolor{blue}{Felix Salten}} fortkommen?\pend
           \pstart
           Herzlichſt{\\[\baselineskip]}\spacefill\mbox{Loris.}\pend
           \leftskip=0em{}\endnumbering\briefempfaengerindex{Schnitzler, Arthur@\textsc{Schnitzler, Arthur}!zzzHofmannsthal, Hugo von@\emph{von Hugo von Hofmannsthal}!1893-01-302@{{[}30.? 1. 1893{]}}|)be}\mylabel{h}  \normalsize

\doendnotes{C}
\bigskip
\vfill

\clearpage

\footnotesize

\lohead{\textsc{register}}

% Definiere theindex-Environment komplett neu ohne reledmac
\makeatletter
\renewenvironment{theindex}{%
  \section*{\indexname}%
  \setlength{\parindent}{0pt}%
  \setlength{\parskip}{0pt plus 0.3pt}%
  \let\item\@idxitem
}{%
  \clearpage
}
\makeatother

\IfFileExists{\jobname-pw.ind}{\input{\jobname-pw.ind}}{}

\end{document}

      