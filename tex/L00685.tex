%% latex-korrekturansicht-vorspann.tex
%% Vorspann für die Korrekturansicht.
%% Lädt die gemeinsame Datei latex-vorspann.tex mit gesetztem Schalter.

\newif\ifkorrekturansicht
\korrekturansichttrue

\input{../tex-inputs/latex-vorspann}


               \section[Arthur Schnitzler an Richard Beer-Hofmann, 12. 6. 1897]{ Arthur Schnitzler an Richard Beer-Hofmann,
               12. 6. 1897}\nopagebreak\mylabel{v}\rehead{ }\normalsize\beginnumbering\briefempfaengerindex{Beer-Hofmann, Richard@\textsc{Beer-Hofmann, Richard}!zzzSchnitzler, Arthur@\emph{von Arthur Schnitzler}!1897-06-121@{12. 6. 1897}|(be} \toendnotes[C]{\smallbreak\pagebreak[2]} \Standort{YCGL, MSS 31.}
\physDesc{Brief, 2 Blätter, 8 Seiten, Umschlag
\newline{}Handschrift: Bleistift, deutsche Kurrent\newline{}Versand: 1) Stempel: »\nobreak{}\oindex{IX., Alsergrund@\textbf{IX., Alsergrund}, \emph{Bezirk (A.BZK)}|pwk}Wien 9/3, 12. 6. 97, 5–6N\nobreak{}«.  2) Stempel: »\nobreak{}\oindex{Bad Ischl@\textbf{Bad Ischl}, \emph{Besiedelter Ort (A.BSO)}|pwk}Ischl, 13. 6. 97, 7–8V\nobreak{}«. }\buchAbdrucke{\weitereDrucke{Arthur Schnitzler, Richard Beer-Hofmann: \emph{Briefwechsel 1891–1931}. Hg. Konstanze Fliedl. Wien, Zürich: \emph{Europaverlag} 1992, S. 108–109.} }\toendnotes[C]{\smallbreak}\pstart{}{\pb}\textsc{Dr. Richard Beer-Hofmann}\pend{}\pstart{}\textsc{\textcolor{pink}{Ischl}{}\ledrightnote{\textcolor{pink}{Bad Ischl}}}\pend{}\pstart{}\textcolor{pink}{\textsc{Egelmoos 22}}{}\ledrightnote{\textcolor{pink}{Eglmoosgasse}}\pend{}\pstart{}\textcolor{pink}{\textsc{Ober-Oesterreich}}{}\ledrightnote{\textcolor{pink}{Oberösterreich}}\pend{}{\bigskip}\pstart
           \raggedleft{}{\pb}12. 6. 97\pend
           \pstart
           Mein lieber Richard. Ich danke ſehr für Ihre Bemühung bei \textcolor{pink}{\textsc{Leopold}}{}\ledrightnote{\textcolor{pink}{Hotel und Pension Rudolfshöhe (Leopold Petter)}}. Wahrſcheinlich ko{\geminationm} ich früher, ſo gegen
                  27, 28. Bitte ſchaun Sie ſich da{\geminationn} im Vorüberradeln das Zi{\geminationm}er an, ob nicht alles wackelt, was in dieſem Wirtshaus {\pb}immer vorauszuſetzen iſt. Notwendig ein großer
               Tiſch (zum Schreiben.) Da meine \textcolor{blue}{Mama}{}\ledrightnote{→\textcolor{blue}{Louise Schnitzler}} eine kleine Couſine, \textcolor{blue}{Grethel}{}\ledrightnote{\textcolor{blue}{Margarethe Manassewitsch}}, zur
               Begleitg hat, brauch ich gar nicht nah von ihr zu ſein. –\pend
           \pstart
           Nun, wegen \textcolor{pink}{\textsc{Bayreuth}}{}\ledrightnote{\textcolor{pink}{Bayreuth}}, da müſſen Sie ſich rasch {\pb}entſchließen,
               aber nicht gleich Nein ſagen, weil es raſch ſein muſs. \textcolor{green}{\textsc{Parsifal}}{}\ledrightnote{\textcolor{green}{Parsifal}} iſt am 27., 28. und 30. \uline{Juli}{ }ſoweit es für mich in Betracht kommt. \uline{Ein}{ }Sitz 12 Gulden. Ich habe auch an \textcolor{blue}{Paul}{}\ledrightnote{\textcolor{blue}{Paul Goldmann}}
               geſchrieben. Soll ich ei{\pb}nen Sitz für Sie nehmen?
               Am liebſten 28. Man bringt ihn auch i{\geminationm}er
               wieder los, da ein großes Geriſs iſt; alſo riskirt iſt nicht viel. Überhaupt!
               12 Gulden – Zwei Gulden – und noch vier – – Und noch ſechs – Man {\pb}hält es und hat vier achter gegen vier zehner, da
               iſt doch die \textsc{\textcolor{green}{Parsifal}{}\ledrightnote{\textcolor{green}{Parsifal}}-Chance} eher werth. –\pend
           \pstart
           – Ich ſpiele mich mit einem \textcolor{green}{Komödienplan}{}\ledrightnote{→\textcolor{green}{Der Weg ins Freie. Roman}} herum {\dotsfour} aber ich fang nicht an,
               bevor die Sache von der 1. bis zur letzten Scene abſolut feſtſteht und alle {\pb}Perſonen zu einander eine wirkliche ſowohl
               äußerliche als innerliche Beziehung haben. Ich habe keine Luſt, wieder ein Stück zu
               ſchreiben, wo man Perſonen nach Belieben entfernen und dazu thun kann. – \textcolor{green}{Freiwild}{}\ledrightnote{\textcolor{green}{Freiwild. Schauspiel in 3 Akten}} in \textcolor{pink}{Prag}{}\ledrightnote{\textcolor{pink}{Prag}}
                  frei{\pb}gegeben – für den Fall, daſs \textcolor{pink}{Bayern}{}\ledrightnote{\textcolor{pink}{Bayern}}. Man räth mir ſehr, besonders \textcolor{blue}{Gustav Schwk}{}\ledrightnote{\textcolor{blue}{Gustav Schwarzkopf}}. Habe noch nicht geantwortet. –\pend
           \pstart
           – Ängſtigt Sie’s »\textcolor{green}{mit ahnungsvoller
                  Gegenwart}{}\ledrightnote{→\textcolor{green}{Faust}}«? – Ich ſpüre noch garnichts. –\pend
           \pstart
           Ich freu mich ſehr auf Sie. We{\geminationn}{ }{\pb}Sie »\textsc{\so{fesch}}« ſind, ſo ko{\geminationm}en
               Sie mir nach \textcolor{pink}{Lambach}{}\ledrightnote{\textcolor{pink}{Lambach}}, oder, billiger, nach \textcolor{pink}{Gmunden}{}\ledrightnote{\textcolor{pink}{Gmunden}} entgegen auf dem Rad und wir fahren zuſa{\geminationm}en u. ſ. w.\pend
           \pstart
           Antworten Sie mir gleich.\pend
           \pstart
           Herzlich Ihr{\\}\spacefill\mbox{Arthur.}\pend
           \endnumbering\briefempfaengerindex{Beer-Hofmann, Richard@\textsc{Beer-Hofmann, Richard}!zzzSchnitzler, Arthur@\emph{von Arthur Schnitzler}!1897-06-121@{12. 6. 1897}|)be}\mylabel{h}  \normalsize

\doendnotes{C}
\bigskip
\vfill

\clearpage

\footnotesize

\lohead{\textsc{register}}

% Definiere theindex-Environment komplett neu ohne reledmac
\makeatletter
\renewenvironment{theindex}{%
  \section*{\indexname}%
  \setlength{\parindent}{0pt}%
  \setlength{\parskip}{0pt plus 0.3pt}%
  \let\item\@idxitem
}{%
  \clearpage
}
\makeatother

\IfFileExists{\jobname-pw.ind}{\input{\jobname-pw.ind}}{}

\end{document}

      