%% latex-korrekturansicht-vorspann.tex
%% Vorspann für die Korrekturansicht.
%% Lädt die gemeinsame Datei latex-vorspann.tex mit gesetztem Schalter.

\newif\ifkorrekturansicht
\korrekturansichttrue

\input{../tex-inputs/latex-vorspann}


               \section[Olga Schnitzler an Richard Beer-Hofmann, {[}17. 1. 1909?{]}]{ Olga Schnitzler an Richard Beer-Hofmann, {[}17. 1. 1909?{]}}\nopagebreak\mylabel{v}\rehead{ }\normalsize\beginnumbering\briefempfaengerindex{Beer-Hofmann, Richard@\textsc{Beer-Hofmann, Richard}!zzzSchnitzler, Olga@\emph{von Olga Schnitzler}!1909-01-171@{{[}17. 1. 1909?{]}}|(be} \toendnotes[C]{\smallbreak\pagebreak[2]} \Standort{YCGL, MSS 31.}
\physDesc{Briefkarte, Umschlag
\newline{}Handschrift: schwarze Tinte, lateinische Kurrent\newline{}Versand: ohne postalischen Übermittlungsvermerk }\toendnotes[C]{\smallbreak}\pstart{}{\pb}Herrn D\textsuperscript{r} Richard
                  Beer-Hofmann\pend{}{\bigskip}\pstart
           \noindent{}{\pb}Lieber Herr Doctor, ich danke sehr für
               die schönen \label{K_L02564-1v}\edtext{Blumen}{\lemma{\textnormal{\emph{Blumen}}}\Cendnote{\textnormal{Die Karte ist undatiert, aber unter den
                  Korrespondenzstücken des Jahres 1909 überliefert. \textcolor{blue}{Olga Schnitzler} feierte am 17. 1. 1909 ihren 27. Geburtstag. Auch der in
                  Folge angesprochene, mögliche Besuch von \textcolor{blue}{Artur}
                  und \textcolor{blue}{Erna Fleischer} stützt die zeitliche
                  Datierung, insofern das Ehepaar \textcolor{blue}{Schnitzler} nur wenige Tage zuvor, am 12. 1. 1909, bei diesen zuhause gewesen war. 
                  Ein Gegenbesuch ist also wahrscheinlicher als in den anderen Monaten, in denen der
                  Kontakt unterbrochen gewesen sein dürfte.}}}\label{K_L02564-1h}, ich habe mich sehr damit
               gefreut.\pend
           \pstart
           \textcolor{blue}{Fleischer}{}\ledrightnote{\textcolor{blue}{Artur Fleischer}{\newline}\textcolor{blue}{Erna Fleischer}}s kommen heute nicht, wir bleiben
               zuhause und würden uns sehr freuen, wenn Ihr \textcolor{blue}{Beide}{}\ledrightnote{→\textcolor{blue}{Paula Beer-Hofmann}} mit uns nachtmahlen wolltet. Backhaendel sind {\pb}auf jeden Fall versorgt. (Mit \textcolor{blue}{Fleischer}{}\ledrightnote{\textcolor{blue}{Artur Fleischer}{\newline}\textcolor{blue}{Erna Fleischer}}s wollten wir im \textcolor{pink}{Türkenschanzpark}{}\ledrightnote{\textcolor{pink}{Türkenschanzpark}} essen – ma soll sehen, die Backhaendel sind Euch zuliebe
               da.)\pend
           \pstart
           Herzliche Grüsse!{\\[\baselineskip]}\spacefill\mbox{OlgaS.}\pend
           \leftskip=0em{}\endnumbering\briefempfaengerindex{Beer-Hofmann, Richard@\textsc{Beer-Hofmann, Richard}!zzzSchnitzler, Olga@\emph{von Olga Schnitzler}!1909-01-171@{{[}17. 1. 1909?{]}}|)be}\mylabel{h}  \normalsize

\doendnotes{C}
\bigskip
\vfill

\clearpage

\footnotesize

\lohead{\textsc{register}}

% Definiere theindex-Environment komplett neu ohne reledmac
\makeatletter
\renewenvironment{theindex}{%
  \section*{\indexname}%
  \setlength{\parindent}{0pt}%
  \setlength{\parskip}{0pt plus 0.3pt}%
  \let\item\@idxitem
}{%
  \clearpage
}
\makeatother

\IfFileExists{\jobname-pw.ind}{\input{\jobname-pw.ind}}{}

\end{document}

      