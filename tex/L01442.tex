%% latex-korrekturansicht-vorspann.tex
%% Vorspann für die Korrekturansicht.
%% Lädt die gemeinsame Datei latex-vorspann.tex mit gesetztem Schalter.

\newif\ifkorrekturansicht
\korrekturansichttrue

\input{../tex-inputs/latex-vorspann}


               \section[Arthur Schnitzler an Richard Beer-Hofmann, 10. 9. 1904]{ Arthur Schnitzler an Richard Beer-Hofmann, 10. 9. 1904}\nopagebreak\mylabel{v}\rehead{ }\normalsize\beginnumbering\briefempfaengerindex{Beer-Hofmann, Richard@\textsc{Beer-Hofmann, Richard}!zzzSchnitzler, Arthur@\emph{von Arthur Schnitzler}!1904-09-101@{10. 9. 1904}|(be} \toendnotes[C]{\smallbreak\pagebreak[2]} \Standort{YCGL, MSS 31.}
\physDesc{Brief, 1 Blatt, 3 Seiten, Umschlag
\newline{}Handschrift: Bleistift, deutsche Kurrent\newline{}Versand: 1) Stempel: »\nobreak{}\oindex{St. Gilgen@\textbf{St. Gilgen}, \emph{Besiedelter Ort (A.BSO)}|pwk}St. Gilge\textcolor{gray}{n}, 10{[}. 9. 1904{]}\nobreak{}«.  2) Stempel: »\nobreak{}\oindex{Bad Aussee@\textbf{Bad Aussee}, \emph{Besiedelter Ort (A.BSO)}|pwk}{\pb}Aussee in
                                       Steiermark, 11/9 04\nobreak{}«. }\toendnotes[C]{\smallbreak}\pstart{}{\pb}\textsc{Herrn Dr Richard Beer-Hofmann}\pend{}\pstart{}\textsc{\textcolor{pink}{Markt Aussee}{}\ledrightnote{\textcolor{pink}{Bad Aussee}}}\pend{}\pstart{}\textcolor{pink}{\textsc{Villa Frühling}.}{}\ledrightnote{\textcolor{pink}{Villa Frühling}}\pend{}{\bigskip}\pstart
           \raggedleft{}{\pb}Samſtag 10. 9. 904\pend
           \pstart
           lieber Richard, ich ſchlage Ihnen vor: Verlaſſen Sie etwa
                  Mittwoch{ }\textcolor{pink}{Auſſee}{}\ledrightnote{\textcolor{pink}{Bad Aussee}}, kommen Sie hieher, bleiben Sie 3–4 Tage,
               leſen Sie mir Ihr \textcolor{green}{Stück}{}\ledrightnote{→\textcolor{green}{Der Graf von Charolais. Ein Trauerspiel}} vor; wir
               fahren da{\geminationn} mit Ihnen z. E. Montag nach \textcolor{pink}{Salzburg}{}\ledrightnote{\textcolor{pink}{Salzburg}}, woſelbſt wir einige Tage ver{\pb}bringen. \textcolor{pink}{Auſſee}{}\ledrightnote{\textcolor{pink}{Bad Aussee}} würde
               mich ja ſehr reizen, we{\geminationn} ich nicht ein ziemlich
               ausgeſprochenes Ruhebedürfnis und einige Scheu vor Hin u Herfahrerei, Ein-
               Auspackerei hätte. \textcolor{blue}{Olga}{}\ledrightnote{\textcolor{blue}{Olga Schnitzler}} geht es ungefähr
               ebenſo.\pend
           \pstart
           Theilen Sie mir bitte Ihren {\pb}Entſchluſs ev.
               telegraphiſch mit.\pend
           \pstart
           Herzlichſt{\\[\baselineskip]}Ihr{\\[\baselineskip]}\spacefill\mbox{A.}\pend
           \leftskip=0em{}\endnumbering\briefempfaengerindex{Beer-Hofmann, Richard@\textsc{Beer-Hofmann, Richard}!zzzSchnitzler, Arthur@\emph{von Arthur Schnitzler}!1904-09-101@{10. 9. 1904}|)be}\mylabel{h}  \normalsize

\doendnotes{C}
\bigskip
\vfill

\clearpage

\footnotesize

\lohead{\textsc{register}}

% Definiere theindex-Environment komplett neu ohne reledmac
\makeatletter
\renewenvironment{theindex}{%
  \section*{\indexname}%
  \setlength{\parindent}{0pt}%
  \setlength{\parskip}{0pt plus 0.3pt}%
  \let\item\@idxitem
}{%
  \clearpage
}
\makeatother

\IfFileExists{\jobname-pw.ind}{\input{\jobname-pw.ind}}{}

\end{document}

      