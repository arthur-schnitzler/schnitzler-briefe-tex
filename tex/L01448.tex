%% latex-korrekturansicht-vorspann.tex
%% Vorspann für die Korrekturansicht.
%% Lädt die gemeinsame Datei latex-vorspann.tex mit gesetztem Schalter.

\newif\ifkorrekturansicht
\korrekturansichttrue

\input{../tex-inputs/latex-vorspann}


               \section[Hugo von Hofmannsthal an Arthur Schnitzler, 22. 9. 1904]{ Hugo von Hofmannsthal an Arthur Schnitzler, 22. 9. 1904}\nopagebreak\mylabel{v}\rehead{ }\normalsize\beginnumbering\briefempfaengerindex{Schnitzler, Arthur@\textsc{Schnitzler, Arthur}!zzzHofmannsthal, Hugo von@\emph{von Hugo von Hofmannsthal}!1904-09-221@{22. 9. 1904}|(be} \toendnotes[C]{\smallbreak\pagebreak[2]} \Standort{CUL, Schnitzler, B 43.}
\physDesc{Postkarte
\newline{}Handschrift: schwarze Tinte, deutsche Kurrent\newline{}Versand: 1) Stempel: »\nobreak{}\oindex{Bahnhof@\textbf{Bahnhof}, \emph{Bahnhofsgebäude (K.BHF)}|pwk}Venezia {[}Ferrovia{]}, 22 9 \textcolor{gray}{0}4 , 10S\nobreak{}«.  2) Stempel: »\nobreak{}\oindex{XVIII., Waehring@\textbf{XVIII., Währing}, \emph{Bezirk (A.BZK)}|pwk}18/1 Wien 110, 24. 9. 04, 2.V, Bestellt\nobreak{}«. 
\newline{}Schnitzler: mit Bleistift Monat und Jahreszahl ergänzt: »/9 904« \newline{}Ordnung: 1) mit Bleistift von unbekannter Hand nummeriert:
                                    »224« 2) mit Bleistift von unbekannter Hand nummeriert:
                                    »255«}\buchAbdrucke{\weitereDrucke{Hugo von Hofmannsthal, Arthur Schnitzler: \emph{Briefwechsel}. Hg. Therese Nickl und Heinrich Schnitzler. Frankfurt am Main: \emph{S. Fischer} 1964, S. 202.} }\toendnotes[C]{\smallbreak}\pstart{}{\pb}\textsc{D\textsuperscript{r} Arthur Schnitzler}\pend{}\pstart{}\textcolor{pink}{\textsc{Wien}}{}\ledrightnote{\textcolor{pink}{Wien}}\pend{}\pstart{}\textcolor{pink}{\textsc{XVIII Spöttelgasse 7}}{}\ledrightnote{\textcolor{pink}{Edmund-Weiß-Gasse}}\pend{}\pstart{}\textcolor{pink}{\textsc{Austria}}{}\ledrightnote{\textcolor{pink}{Österreich}}\pend{}{\bigskip}\pstart
           \raggedleft{}{\pb}22.\pend
           \pstart
           lieber, bin wohl und recht fleißig, bei hellem aber ſehr kühlem
               Wetter. Bitte vielmals ſchicken Sie mir recht bald hieher – ich habe in den
               Abendſtunden gar nichts zu leſen – womöglich: \textsc{\textcolor{blue}{H. Mann}{}\ledrightnote{\textcolor{blue}{Heinrich Mann}}, \textcolor{green}{Herzogin}{}\ledrightnote{\textcolor{green}{Die Göttinnen oder Die drei Romane der Herzogin von Assy}}}, I u. II (\textsc{Bd III \textcolor{green}{Venus}{}\ledrightnote{\textcolor{green}{Die Göttinnen oder Die drei Romane der Herzogin von Assy}}} habe ich) und das Heft der \textcolor{green}{Zukunft}{}\ledrightnote{\textcolor{green}{Die Zukunft}}, worin \textcolor{blue}{\textsc{H}.}{}\ledrightnote{\textcolor{blue}{Heinrich Mann}} über \textcolor{green}{\textsc{Elektra}}{}\ledrightnote{\textcolor{green}{Elektra. Tragödie in einem Aufzug}}{ }\textcolor{green}{ſchrieb}{}\ledrightnote{→\textcolor{green}{Elektra}}. Wenn das nicht möglich,
               ſo vielleicht »\textcolor{green}{\textsc{Jagd nach Liebe}}{}\ledrightnote{\textcolor{green}{Die Jagd nach Liebe}}«. Voraus dankend, von Herzen\pend
           \pstart \spacefill\mbox{Hugo.}\pend{}\pstart
           \noindent{}\label{T_L01448_1v}\edtext{\textsc{P. S.} Eben ko{\geminationm}t die »\textcolor{green}{Zukunft}{}\ledrightnote{\textcolor{green}{Die Zukunft}}«, alſo die nicht.}{\lemma{\textnormal{\emph{P. S. … nicht.}}}\Cendnote{\textnormal{quer am rechten Rand}}}\label{T_L01448_1h}\pend
           \endnumbering\briefempfaengerindex{Schnitzler, Arthur@\textsc{Schnitzler, Arthur}!zzzHofmannsthal, Hugo von@\emph{von Hugo von Hofmannsthal}!1904-09-221@{22. 9. 1904}|)be}\mylabel{h}  \normalsize

\doendnotes{C}
\bigskip
\vfill

\clearpage

\footnotesize

\lohead{\textsc{register}}

% Definiere theindex-Environment komplett neu ohne reledmac
\makeatletter
\renewenvironment{theindex}{%
  \section*{\indexname}%
  \setlength{\parindent}{0pt}%
  \setlength{\parskip}{0pt plus 0.3pt}%
  \let\item\@idxitem
}{%
  \clearpage
}
\makeatother

\IfFileExists{\jobname-pw.ind}{\input{\jobname-pw.ind}}{}

\end{document}

      