%% latex-korrekturansicht-vorspann.tex
%% Vorspann für die Korrekturansicht.
%% Lädt die gemeinsame Datei latex-vorspann.tex mit gesetztem Schalter.

\newif\ifkorrekturansicht
\korrekturansichttrue

\input{../tex-inputs/latex-vorspann}


               \section[Arthur Schnitzler an Richard Beer-Hofmann, 13. 7. 1903]{ Arthur Schnitzler an Richard Beer-Hofmann,
               13. 7. 1903}\nopagebreak\mylabel{v}\rehead{ }\normalsize\beginnumbering\briefempfaengerindex{Beer-Hofmann, Richard@\textsc{Beer-Hofmann, Richard}!zzzSchnitzler, Arthur@\emph{von Arthur Schnitzler}!1903-07-132@{13. 7. 1903}|(be} \toendnotes[C]{\smallbreak\pagebreak[2]} \Standort{YCGL, MSS 31.}
\physDesc{Postkarte
\newline{}Handschrift: roter Buntstift, deutsche Kurrent\newline{}Versand: 1) Stempel: »\nobreak{}\oindex{IX., Alsergrund@\textbf{IX., Alsergrund}, \emph{Bezirk (A.BZK)}|pwk}Wien 9/3, 13. 7. 03, 11–12V\nobreak{}«.  2) Stempel: »\nobreak{}\oindex{Rodaun@\textbf{Rodaun}, \emph{Teil eines besiedelten Ortes (A.BSOX)}|pwk}Rodaun, 13. 7. 03, 2–4N\nobreak{}«. }\buchAbdrucke{\weitereDrucke{Arthur Schnitzler, Richard Beer-Hofmann: \emph{Briefwechsel 1891–1931}. Hg. Konstanze Fliedl. Wien, Zürich: \emph{Europaverlag} 1992, S. 163.} }\pstart{}{\pb}Herrn \textsc{Dr Rich.}\pend{}\pstart{}\textsc{Beer-Hofmann}\pend{}\pstart{}\textcolor{pink}{\textsc{Rodaun}}{}\ledrightnote{\textcolor{pink}{Rodaun}}\pend{}\pstart{}\textsc{\textcolor{pink}{Liesinger Straße 1}{}\ledrightnote{\textcolor{pink}{Liesingerstraße}}.}\pend{}{\bigskip}\pstart
           \noindent{}{\pb}Lieber, radeln Sie ſchon fleißig?\pend
           \pstart
           Herzlichſt Ihr{\\[\baselineskip]}\spacefill\mbox{A.}\pend
           \leftskip=0em{}\pstart
           13. 7. 903.\pend
           \endnumbering\briefempfaengerindex{Beer-Hofmann, Richard@\textsc{Beer-Hofmann, Richard}!zzzSchnitzler, Arthur@\emph{von Arthur Schnitzler}!1903-07-132@{13. 7. 1903}|)be}\mylabel{h}  \normalsize

\doendnotes{C}
\bigskip
\vfill

\clearpage

\footnotesize

\lohead{\textsc{register}}

% Definiere theindex-Environment komplett neu ohne reledmac
\makeatletter
\renewenvironment{theindex}{%
  \section*{\indexname}%
  \setlength{\parindent}{0pt}%
  \setlength{\parskip}{0pt plus 0.3pt}%
  \let\item\@idxitem
}{%
  \clearpage
}
\makeatother

\IfFileExists{\jobname-pw.ind}{\input{\jobname-pw.ind}}{}

\end{document}

      