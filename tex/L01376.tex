%% latex-korrekturansicht-vorspann.tex
%% Vorspann für die Korrekturansicht.
%% Lädt die gemeinsame Datei latex-vorspann.tex mit gesetztem Schalter.

\newif\ifkorrekturansicht
\korrekturansichttrue

\input{../tex-inputs/latex-vorspann}


               \section[Arthur Schnitzler an Hermann Bahr, 22. 2. 1904]{ Arthur Schnitzler an Hermann Bahr, 22. 2. 1904}\nopagebreak\mylabel{v}\rehead{ }\normalsize\beginnumbering\briefempfaengerindex{Bahr, Hermann@\textsc{Bahr, Hermann}!zzzSchnitzler, Arthur@\emph{von Arthur Schnitzler}!1904-02-221@{22. 2. 1904}|(be} \toendnotes[C]{\smallbreak\pagebreak[2]} \Standort{TMW, HS AM 23367 Ba.}
\physDesc{Brief, 2 Blätter, 7 Seiten
\newline{}Handschrift: schwarze Tinte, deutsche Kurrent}\buchAbdrucke{\weitereDrucke{1) \emph{22. 2. 1904.} In: Arthur Schnitzler: \emph{The Letters of Arthur Schnitzler to Hermann Bahr}. Edited, annotated, and with an introduction, by Donald G.
                        Daviau. Chapel Hill: \emph{The University of North Carolina Press} 1978, S. 84–85 (University of North Carolina studies in the Germanic languages
                        and literatures, 89).} \weitereDrucke{2) Hermann Bahr, Arthur Schnitzler: \emph{Briefwechsel, Aufzeichnungen, Dokumente (1891–1931)}. Hg. Kurt Ifkovits und Martin Anton Müller. Göttingen: \emph{Wallstein} 2018, S. 303–304.} }\toendnotes[C]{\smallbreak}\pstart
           \raggedleft{}{\pb}\textcolor{pink}{Wien}{}\ledrightnote{\textcolor{pink}{Wien}}, 22. 2. 904\pend
           \pstart
           mein lieber Hermann, wir waren eben in \textcolor{pink}{Hietzing}{}\ledrightnote{\textcolor{pink}{XIII., Hietzing}}, mit \textcolor{blue}{Hugo’s}{}\ledrightnote{\textcolor{blue}{Gertrude von Hofmannsthal}{\newline}\textcolor{blue}{Hugo von Hofmannsthal}} u \textcolor{blue}{Richards}{}\ledrightnote{\textcolor{blue}{Paula Beer-Hofmann}{\newline}\textcolor{blue}{Richard Beer-Hofmann}} u \textcolor{blue}{Karg}{}\ledrightnote{\textcolor{blue}{Edgar von Karg-Bebenburg}} zuſammen, u da hab ich mit großer Freude gehört, daſs du dich viel
               wohler befindeſt. Nun möchte ich aber gern recht bald ein Wort von dir ſelbſt
               vernehmen, und wiſſen, wie es mit deinen Plänen für die nächſte Zeit ſteht. Ich bin
               seit \label{K_L01376_1v}\edtext{Freitag Abend}{\lemma{\textnormal{\emph{Freitag Abend}}}\Cendnote{\textnormal{eigentlich schon seit dem
                     19. 2. 1904 abends (einem Donnerstag)}}}\label{K_L01376_1h} wieder in \textcolor{pink}{Wien}{}\ledrightnote{\textcolor{pink}{Wien}}; wir (\textcolor{blue}{Olga}{}\ledrightnote{\textcolor{blue}{Olga Schnitzler}} u ich) waren
                  {\pb}auf der Rückreise
               einen Tag in \textcolor{pink}{Dresden}{}\ledrightnote{\textcolor{pink}{Dresden}} und haben allzukurze Stunden in
               der \textcolor{pink}{Galerie}{}\ledrightnote{\textcolor{pink}{Gemäldegalerie Alte Meister}} verbracht.\pend
           \pstart
           Über den \textcolor{green}{Einſamen Weg}{}\ledrightnote{\textcolor{green}{Der einsame Weg. Schauspiel in fünf Akten}} haſt du wohl, ſoweit es ſich
               um den äußerlichen Verlauf des erſten Abends handelt, das weſentliche geleſen. Es war
               ein leidlicher Abfall, Huſten und Unruhe von Anbeginn, matter Beifall nach 2. u 3.
               Akt mit Widerſpruch; Gelächter und ſtarker Beifall nach dem 4. Akt, viel Applaus und
               viel Ziſchen am {\pb}Schluſs. Der 2. Abend, ausverkauft, ging beträchtlich beſſer – und nun ſcheint
               ſich, wie ich aus \textcolor{pink}{Berlin}{}\ledrightnote{\textcolor{pink}{Berlin}} höre, das Stück, das bei
               einem Theil der Kritik ſehr lebhafte Anerkennung fand, doch einige Zeit halten zu
               wollen. In \textcolor{pink}{Wien}{}\ledrightnote{\textcolor{pink}{Wien}} war eigentlich nur das \textcolor{blue}{Goldmann}{}\ledrightnote{\textcolor{blue}{Paul Goldmann}}’ſche \label{K_L01376_2v}\edtext{\textcolor{green}{Telegra{\geminationm}}{}\ledrightnote{→\textcolor{green}{Schnitzlers »Einsamer Weg« (Telegramm der »Neuen Freien Presse«)}}}{\lemma{\textnormal{\emph{Telegra}}}\Cendnote{\textnormal{[O. V.:] \emph{\textcolor{green}{Schnitzlers »Einsamer Weg« (Telegramm
                        der »Neuen Freien Presse«)}}. In: \emph{\textcolor{green}{Neue Freie
                        Presse}}, Nr. 14178, 14. 2. 1904, S. 12.}}}\label{K_L01376_2h}
               wirklich ſchlecht – was er mir perſönlich über das \textcolor{green}{Stück}{}\ledrightnote{→\textcolor{green}{Der einsame Weg. Schauspiel in fünf Akten}} zu ſagen wußte, waren nur die folgenden Worte, als
               ich ihn ein paar Tage nach der Première zum Abſchied {\pb}beſuchte\substVorne{}\textsuperscript{,}\substDazwischen{}:\substHinten{} »Ich ſchreibe eben das \label{K_L01376_3v}\edtext{\textcolor{green}{Feuillet}{}\ledrightnote{→\textcolor{green}{Berliner Theater. »Der einsame Weg«. Von Arthur Schnitzler}}}{\lemma{\textnormal{\emph{Feuillet}}}\Cendnote{\textnormal{\textcolor{blue}{Paul Goldmann}: \emph{\textcolor{green}{Berliner Theater. »Der einsame Weg«. Von Arthur Schnitzler}}. In: \emph{\textcolor{green}{Neue Freie Presse}}, Nr. 14187,
                        23. 2. 1904, S. 1–3.}}}\label{K_L01376_3h} über den \textcolor{green}{E. W.}{}\ledrightnote{\textcolor{green}{Der einsame Weg. Schauspiel in fünf Akten}} – Du wirſt keine Freude daran haben.« – Die Fehler des \textcolor{green}{Stücks}{}\ledrightnote{→\textcolor{green}{Der einsame Weg. Schauspiel in fünf Akten}}{ }ſpür ich jetzt wie mir vorko{\geminationm}t ſehr genau: Das Verhältnis zwiſchen \textcolor{green}{Sala}{}\ledrightnote{→\textcolor{green}{Der einsame Weg. Schauspiel in fünf Akten}} u \textsc{Johann\textcolor{gray}{a}} müßte ſchon zu Beginn völlig declarirt sein – das iſt ein techniſcher Fehler, de\substVorne{}\textsuperscript{r}\substDazwischen{}n\substHinten{} gutzumachen in meinen Kräften ſtände. Andres aber dürfte in den Mängeln
               meiner Begabung begründet ſein – ſo insbeſondre eine gewiſſe Steifigkeit im Weſen {\pb}Julians. Immerhin
               bleibt es eine ſchwierige Sache von einer Perſon die Meinung verbreiten zu wollen –
               ſie ſei einmal ein Genie geweſen. Ja we{\geminationn} man das Bild
               ins Foyer hängen könnte, das \textcolor{green}{Julian}{}\ledrightnote{→\textcolor{green}{Der einsame Weg. Schauspiel in fünf Akten}} vor 25 Jahren gemalt und das ihn berühmt gemacht hat! Übrigens –
               vielleicht wäre es auch im Augenblick vergeſſen, da man ſich wieder ins Parket
               begibt.\pend
           \pstart
           Was ich ſelbſt an dem \textcolor{green}{Stück}{}\ledrightnote{→\textcolor{green}{Der einsame Weg. Schauspiel in fünf Akten}}
               wirklich liebe, iſt der fünfte Akt und die {\pb}Geſtalt des \textcolor{green}{Sala}{}\ledrightnote{→\textcolor{green}{Der einsame Weg. Schauspiel in fünf Akten}}, der gegenüber ich mich,
               eigentlich das erſte Mal in meinem Leben, als eine Art von Schöpfer fühle. Und der
               fünfte Akt bedeutet mir zuweilen etwas mehr als der Abschluſs eines Dramas – ja nicht
               viel weniger als der Abschluſs von 42 ſelbſt gelebten Jahren. \introOben{}–\introOben{} Nun ſeh ich mancherlei vor mir, was mir, wenn ich etwas weniger faul,
               etwas weniger zerſtreut, und mit \strikeout{\textcolor{gray}{×}\-\textcolor{gray}{×}\-\textcolor{gray}{×}\-\textcolor{gray}{×}-} wahrer Intenſität
               begabt wäre, nach dem ſonſtigen Stande meines Innern, eigentlich gelingen
               müßte. –\pend
           \pstart
           {\pb}– Wir haben in \textcolor{pink}{Berlin}{}\ledrightnote{\textcolor{pink}{Berlin}} oft von dir geſprochen und alle Leute die du
               kennſt laſſen dich grüßen. Meine \textcolor{pink}{sicilianiſchen}{}\ledrightnote{\textcolor{pink}{Sizilien}} und
                  \textcolor{pink}{korfioliſchen}{}\ledrightnote{\textcolor{pink}{Korfu}} Pläne weben weiter – wirſt du
               auch  ſüdlicher wandern und werden wir uns ſehen? Meine \textcolor{blue}{Frau}{}\ledrightnote{→\textcolor{blue}{Olga Schnitzler}} grüßt dich herzlich, ich desgleichen und wir wären
               ſehr froh, wenn wir bald noch beſſeres, ganz gutes von dir hörten.\pend
           \pstart
           Dein{\\[\baselineskip]}\spacefill\mbox{Arthur}\pend
           \leftskip=0em{}\endnumbering\briefempfaengerindex{Bahr, Hermann@\textsc{Bahr, Hermann}!zzzSchnitzler, Arthur@\emph{von Arthur Schnitzler}!1904-02-221@{22. 2. 1904}|)be}\mylabel{h}  \normalsize

\doendnotes{C}
\bigskip
\vfill

\clearpage

\footnotesize

\lohead{\textsc{register}}

% Definiere theindex-Environment komplett neu ohne reledmac
\makeatletter
\renewenvironment{theindex}{%
  \section*{\indexname}%
  \setlength{\parindent}{0pt}%
  \setlength{\parskip}{0pt plus 0.3pt}%
  \let\item\@idxitem
}{%
  \clearpage
}
\makeatother

\IfFileExists{\jobname-pw.ind}{\input{\jobname-pw.ind}}{}

\end{document}

      