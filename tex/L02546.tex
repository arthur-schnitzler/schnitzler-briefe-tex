%% latex-korrekturansicht-vorspann.tex
%% Vorspann für die Korrekturansicht.
%% Lädt die gemeinsame Datei latex-vorspann.tex mit gesetztem Schalter.

\newif\ifkorrekturansicht
\korrekturansichttrue

\input{../tex-inputs/latex-vorspann}


               \section[Arthur Schnitzler an Hermann Bahr, 5. 9. 1931]{ Arthur Schnitzler an Hermann Bahr, 5. 9. 1931}\nopagebreak\mylabel{v}\rehead{ }\normalsize\beginnumbering\briefempfaengerindex{Bahr, Hermann@\textsc{Bahr, Hermann}!zzzSchnitzler, Arthur@\emph{von Arthur Schnitzler}!1931-09-051@{5. 9. 1931}|(be} \toendnotes[C]{\smallbreak\pagebreak[2]} \Standort{TMW, HS AM 23400 Ba.}
\physDesc{Brief, 1 Blatt, 1 Seite
\newline{}Schreibmaschine
\newline{}Handschrift: Bleistift, lateinische Kurrent (\noindent{}Unterschrift und Grußformel)
\newline{}Bahr: mit rotem Buntstift ergänzt: »Unmittelbar vor Fahrt nach
                                       \textcolor{pink}{Garmisch}« \newline{}Ordnung: mit Bleistift von unbekannter Hand beschriftet:
                                    »erledigt« }\Standort{DLA, A:Schnitzler, 85.1.294/8.}
\physDesc{Brief, 1 Blatt, 1 Seite, maschineller Durchschlag
\newline{}Schreibmaschine}\buchAbdrucke{\weitereDrucke{1) \emph{5. 9. 1931.} In: Arthur Schnitzler: \emph{The Letters of Arthur Schnitzler to Hermann Bahr}. Edited, annotated, and with an introduction, by Donald G.
                        Daviau. Chapel Hill: \emph{The University of North Carolina Press} 1978, S. 118 (University of North Carolina studies in the Germanic languages
                        and literatures, 89).} \weitereDrucke{2) Hermann Bahr, Arthur Schnitzler: \emph{Briefwechsel, Aufzeichnungen, Dokumente (1891–1931)}. Hg. Kurt Ifkovits und Martin Anton Müller. Göttingen: \emph{Wallstein} 2018, S. 598–599.} }\toendnotes[C]{\smallbreak}\pstart
           \noindent{}{\pb}\textcolor{gray}{\textbf{D\textsuperscript{r} Arthur Schnitzler}}\hfill 5. 9. 1931.\pend
           \pstart
           \textcolor{gray}{\textbf{\textcolor{pink}{Wien. XVIII. Sternwartestrasse 71}{}\ledrightnote{\textcolor{pink}{Sternwartestraße}}.}}\pend
           \pstart{}Lieber Hermann.\pend\pstart
           Ich lese, dass Dein »\textcolor{green}{Konzert}{}\ledrightnote{\textcolor{green}{Das Konzert. Lustspiel in drei Akten}}« jetzt als \label{K_L02546_1v}\edtext{\textcolor{green}{Tonfilm}{}\ledrightnote{→\textcolor{green}{Das Konzert}} erscheint}{\lemma{\textnormal{\emph{Tonfilm erscheint}}}\Cendnote{\textnormal{Die Verfilmung durch \textcolor{blue}{Leo Mittler} lief bereits seit 28. 8. 1931 in den \textcolor{pink}{Wien}er
               Kinos.}}}\label{K_L02546_1h}, nachdem es vorher, so weit ich mich erinnere, auch schon als \textcolor{green}{\label{K_L02546_2v}\edtext{stummer Film}{\lemma{\textnormal{\emph{stummer Film}}}\Cendnote{\textnormal{\emph{\textcolor{green}{The Concert}} (1921), Regie \textcolor{blue}{Victor Schertzinger}, von
                     demselben neuerlich unter dem Titel \emph{\textcolor{green}{Fashions in
                        Love}} (1929) als Tonfilm realisiert.}}}\label{K_L02546_2h}}{}\ledrightnote{→\textcolor{green}{The Concert}} zu sehen war. Ich möchte nun gern wissen – falls es Dir nicht unbequem ist mir
               darauf zu antworten – ob, resp. welche Ansprüche die seinerzeitigen Verfertiger des
               stummen Films an Dich gestellt haben. Ich erlebe es in jedem einzelnen Fall, so mit
                  »\label{K_L02546_3v}\edtext{\textcolor{green}{Liebelei}{}\ledrightnote{\textcolor{green}{Liebelei. Schauspiel in drei Akten}}}{\lemma{\textnormal{\emph{Liebelei}}}\Cendnote{\textnormal{Erstmals verfilmt 1914 (\emph{\textcolor{green}{Elskovsleg}}, Regie \textcolor{blue}{Holger-Madsen} und \textcolor{blue}{August Blom}). Ab
                     1921 Verhandlungen über eine Neuverfilmung, vgl. \textcolor{blue}{Arthur Schnitzler}: \emph{Filmarbeiten. Drehbücher, Entwürfe, Skizzen}. Hg. Achim Aurnhammer,
                     Hans Peter Buohler, Philipp Gresser, Julia Ilgner, Carolin Maikler, Lea
                     Marquart. Würzburg: \emph{Ergon}{ }2015, S. 101–103. Neuerliche \textcolor{green}{Verfilmung}{ }1927 (Regie \textcolor{blue}{Jakob} und \textcolor{blue}{Luise Fleck}).}}}\label{K_L02546_3h}«, »\label{K_L02546_4v}\edtext{\textcolor{green}{Anatol}{}\ledrightnote{\textcolor{green}{Anatol}}}{\lemma{\textnormal{\emph{Anatol}}}\Cendnote{\textnormal{\emph{\textcolor{green}{The Affairs of Anatol}} (1921), Regie
                     \textcolor{blue}{Cecil B. DeMille}. Zu dem Plan einer
                  neuerlichen Verfilmung, die nicht realisiert wurde, gibt es Hinweise in \textcolor{blue}{Schnitzler}s \emph{\textcolor{green}{Tagebuch}} zwischen 3. 11. 1930 und 4. 5. 1931.}}}\label{K_L02546_4h}«, »\label{K_L02546_5v}\edtext{\textcolor{green}{Fräulein Else}{}\ledrightnote{\textcolor{green}{Fräulein Else}}}{\lemma{\textnormal{\emph{Fräulein Else}}}\Cendnote{\textnormal{\emph{\textcolor{green}{Fräulein Else}} (1929), Regie \textcolor{blue}{Paul Czinner}}}}\label{K_L02546_5h}«, dass sich die seinerzeitigen Verfertiger der stummen Fassung
               freundlich-erpresserisch gebärden, in welcher Haltung die Leute durch allerlei
               Gesetze, Auffassungen, Bestimmungen – auch insoweit sie nicht vorhanden sind – mehr
               oder weniger unterstützt werden.\pend
           \pstart
           Wolltest Du mir bei dieser Gelegenheit auch sonst ein Wort über Dich und Dein
               Befinden sagen, so wird es mich herzlich freuen.\pend
           \pstart
           {[}hs.:{]} Mit vielen Grüßen und der Bitte mich deiner verehrten \textcolor{blue}{Gattin}{}\ledrightnote{→\textcolor{blue}{Anna Bahr-Mildenburg}} zu empfehlen{\\[\baselineskip]}Dein{\\[\baselineskip]}\spacefill\mbox{Arth}\pend
           \leftskip=0em{}\pstart
           \noindent{}{[}ms.:{]}  Herrn Hermann Bahr,{\\}\textcolor{pink}{München}{}\ledrightnote{\textcolor{pink}{München}}.\pend
           \endnumbering\briefempfaengerindex{Bahr, Hermann@\textsc{Bahr, Hermann}!zzzSchnitzler, Arthur@\emph{von Arthur Schnitzler}!1931-09-051@{5. 9. 1931}|)be}\mylabel{h}  \normalsize

\doendnotes{C}
\bigskip
\vfill

\clearpage

\footnotesize

\lohead{\textsc{register}}

% Definiere theindex-Environment komplett neu ohne reledmac
\makeatletter
\renewenvironment{theindex}{%
  \section*{\indexname}%
  \setlength{\parindent}{0pt}%
  \setlength{\parskip}{0pt plus 0.3pt}%
  \let\item\@idxitem
}{%
  \clearpage
}
\makeatother

\IfFileExists{\jobname-pw.ind}{\input{\jobname-pw.ind}}{}

\end{document}

      