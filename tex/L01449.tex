%% latex-korrekturansicht-vorspann.tex
%% Vorspann für die Korrekturansicht.
%% Lädt die gemeinsame Datei latex-vorspann.tex mit gesetztem Schalter.

\newif\ifkorrekturansicht
\korrekturansichttrue

\input{../tex-inputs/latex-vorspann}


               \section[Arthur Schnitzler an Hugo von Hofmannsthal, 24. 9. 1904]{ Arthur Schnitzler an Hugo von Hofmannsthal, 24. 9. 1904}\nopagebreak\mylabel{v}\rehead{ }\normalsize\beginnumbering\briefempfaengerindex{Hofmannsthal, Hugo von@\textsc{Hofmannsthal, Hugo von}!zzzSchnitzler, Arthur@\emph{von Arthur Schnitzler}!1904-09-241@{24. 9. 1904}|(be} \toendnotes[C]{\smallbreak\pagebreak[2]} \Standort{FDH, Hs-30885,115.}
\physDesc{Kartenbrief
\newline{}Handschrift: schwarze Tinte, deutsche Kurrent\newline{}Versand: 1) Stempel: »\nobreak{}\oindex{XVIII., Waehring@\textbf{XVIII., Währing}, \emph{Bezirk (A.BZK)}|pwk}18/1 Wien, 24{[}. 09.{]} 04, 1\nobreak{}«.  2) Stempel: »\nobreak{}\oindex{Thueringen@\textbf{Thüringen}, \emph{Teil eines Landes (A.LNDX)}|pwk}Venezia\nobreak{}«. 3) Stempel: »\nobreak{}\oindex{Thueringen@\textbf{Thüringen}, \emph{Teil eines Landes (A.LNDX)}|pwk}Venezia, 25. {[}9. 0{]}4, 12M\nobreak{}«. 4) Stempel: »\nobreak{}\oindex{Rodaun@\textbf{Rodaun}, \emph{Teil eines besiedelten Ortes (A.BSOX)}|pwk}{[}Rod{]}aun, \textcolor{gray}{27}. {[}9.{]}
                                                  IV\nobreak{}«. 5) mit Bleistift von unbekannter Hand die originale Adressierung 
                                            geändert zu: »\textsc{\textcolor{pink}{5 Badgasse} / \textcolor{pink}{Rodaun presso Vienna}}«}\buchAbdrucke{\weitereDrucke{1) Hugo von Hofmannsthal, Arthur Schnitzler: \emph{Briefwechsel}. Hg. Therese Nickl und Heinrich Schnitzler. Frankfurt am Main: \emph{S. Fischer} 1964, S. 202–203.} \weitereDrucke{2) Hermann Bahr, Arthur Schnitzler: \emph{Briefwechsel, Aufzeichnungen, Dokumente
                                (1891–1931)}. Hg. Kurt Ifkovits und Martin Anton Müller. Göttingen: \emph{Wallstein} 2018, S. 322.} }\toendnotes[C]{\smallbreak}\pstart{}{\pb}\textsc{Herrn Hugo von Hofmannsthal}\pend{}\pstart{}\textsc{\textcolor{pink}{Venedig}{}\ledrightnote{\textcolor{pink}{Thüringen}}}\pend{}\pstart{}\textsc{\textcolor{pink}{Hotel Europe}{}\ledrightnote{\textcolor{pink}{Hotel de l’Europe}}}\pend{}{\bigskip}\pstart
           \raggedleft{}{\pb}24. 9. 904\pend
           \pstart
           lieber Hugo, \textcolor{green}{Jagd nach Liebe}{}\ledrightnote{\textcolor{green}{Die Jagd nach Liebe}} ist bei \textsc{\textcolor{blue}{Wassermann}{}\ledrightnote{\textcolor{blue}{Jakob Wassermann}}}, ich habe ihm geſchrieben, er möge Ihnen das Buch ſenden. – \textsc{\textcolor{green}{Assy}{}\ledrightnote{\textcolor{green}{Die Göttinnen oder Die drei Romane der Herzogin von Assy}}} beſitz ich gar nicht. –\pend
           \pstart
           Ich fange erſt in den nächſten Tagen ordentlich zu arbeiten an. Hatte viel
                    Kopfweh. Wir ſind ſeit 20. Abend hier, waren in \textcolor{pink}{Salzburg}{}\ledrightnote{\textcolor{pink}{Salzburg}} mit \textcolor{blue}{Richard}{}\ledrightnote{\textcolor{blue}{Richard Beer-Hofmann}} u
                        \textcolor{blue}{Bahr}{}\ledrightnote{\textcolor{blue}{Hermann Bahr}} zuſammen; ſahen auch \textcolor{blue}{Karg}{}\ledrightnote{\textcolor{blue}{Edgar von Karg-Bebenburg}} ein paar Mal. –\pend
           \pstart
           Vielleicht kann uns \textsc{\textcolor{blue}{Gerty}{}\ledrightnote{\textcolor{blue}{Gertrude von Hofmannsthal}}} die Adreſſe der \textcolor{blue}{Italienerin}{}\ledrightnote{→\textcolor{blue}{?? [Italienischlehrerin in Wien]}}{ }ſagen, bei der ſie einmal Stunden genommen
                    hat. Adreſſe u Namen. Einmal war ſie bei mir, einer Überſetzung wegen, wohnte
                    damals \textsc{\textcolor{pink}{Hammerand}{}\ledrightnote{\textcolor{pink}{Hotel Hammerand}}}. –\pend
           \pstart
           Geſtern bin ich geradelt, \textcolor{pink}{Hütteldorf}{}\ledrightnote{\textcolor{pink}{Hütteldorf}}, \textcolor{pink}{Neuwaldegg}{}\ledrightnote{\textcolor{pink}{Neuwaldegg}}; es iſt ſchon ſo herbſtlich. Mein
                    Rad hat ſsich auf der Reiſe recht erholt.\pend
           \pstart
           Herzliche Grüße, an Sie, \textsc{\textcolor{blue}{Gerty}{}\ledrightnote{\textcolor{blue}{Gertrude von Hofmannsthal}}}, \textcolor{blue}{Hans}{}\ledrightnote{\textcolor{blue}{Hans Bernhard Schlesinger}} von \textcolor{blue}{uns beiden}{}\ledrightnote{→\textcolor{blue}{Olga Schnitzler}}\pend
           \pstart
           Ihr{\\[\baselineskip]}\spacefill\mbox{A.}\pend
           \leftskip=0em{}\endnumbering\briefempfaengerindex{Hofmannsthal, Hugo von@\textsc{Hofmannsthal, Hugo von}!zzzSchnitzler, Arthur@\emph{von Arthur Schnitzler}!1904-09-241@{24. 9. 1904}|)be}\mylabel{h}  \normalsize

\doendnotes{C}
\bigskip
\vfill

\clearpage

\footnotesize

\lohead{\textsc{register}}

% Definiere theindex-Environment komplett neu ohne reledmac
\makeatletter
\renewenvironment{theindex}{%
  \section*{\indexname}%
  \setlength{\parindent}{0pt}%
  \setlength{\parskip}{0pt plus 0.3pt}%
  \let\item\@idxitem
}{%
  \clearpage
}
\makeatother

\IfFileExists{\jobname-pw.ind}{\input{\jobname-pw.ind}}{}

\end{document}

      