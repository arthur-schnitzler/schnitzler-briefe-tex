%% latex-korrekturansicht-vorspann.tex
%% Vorspann für die Korrekturansicht.
%% Lädt die gemeinsame Datei latex-vorspann.tex mit gesetztem Schalter.

\newif\ifkorrekturansicht
\korrekturansichttrue

\input{../tex-inputs/latex-vorspann}


               \section[Hugo von Hofmannsthal an Arthur Schnitzler, 4. 12. 1891]{ Hugo von Hofmannsthal an Arthur Schnitzler, 4. 12. 1891}\nopagebreak\mylabel{v}\rehead{ }\normalsize\beginnumbering\briefempfaengerindex{Schnitzler, Arthur@\textsc{Schnitzler, Arthur}!zzzHofmannsthal, Hugo von@\emph{von Hugo von Hofmannsthal}!1891-12-041@{4. 12. 1891}|(be} \toendnotes[C]{\smallbreak\pagebreak[2]} \Standort{CUL, Schnitzler, B 43.}
\physDesc{Postkarte
\newline{}Handschrift: schwarze Tinte, lateinische Kurrent\newline{}Versand: Stempel: »\nobreak{}Wien, 4/12 91, 10 A\nobreak{}«.  
\newline{}Schnitzler: mit Bleistift jeweils auf der Text- und auf der Anschriftenseite
                                 datiert: »\strikeout{2}4 12 91.«
                               \newline{}Ordnung: mit Bleistift von unbekannter Hand nummeriert:
                                 »2« }\buchAbdrucke{\weitereDrucke{1) Hugo von Hofmannsthal, Arthur Schnitzler: \emph{Briefwechsel}. Hg. Therese Nickl und Heinrich Schnitzler. Frankfurt am Main: \emph{S. Fischer} 1964, S. 7.} \weitereDrucke{2) Hermann Bahr, Arthur Schnitzler: \emph{Briefwechsel, Aufzeichnungen, Dokumente (1891–1931)}. Hg. Kurt Ifkovits und Martin Anton Müller. Göttingen: \emph{Wallstein} 2018, S. 14.} }\toendnotes[C]{\smallbreak}\pstart{}{\pb}D\textsuperscript{r}
                  Arthur Schnitzler\pend{}\pstart{}\textcolor{pink}{Wien}{}\ledrightnote{\textcolor{pink}{Wien}}\pend{}\pstart{}\textcolor{pink}{I Kärnthnerring 12}{}\ledrightnote{\textcolor{pink}{Kärntnerring}}\pend{}{\bigskip}\pstart
           \noindent{}{\pb}\textcolor{blue}{Bahr}{}\ledrightnote{\textcolor{blue}{Hermann Bahr}} wohnt \textcolor{pink}{Heumarkt 9, 3 Stiege, 3. Stock Thür 37}{}\ledrightnote{\textcolor{pink}{Am Heumarkt}}. Kommt aber, wenn sie ihm nichts
               anderes schreiben, ebenso wie ich \label{K_L00046_1v}\edtext{Sonntag}{\lemma{\textnormal{\emph{Sonntag}}}\Cendnote{\textnormal{Am Sonntag, den 6. 12. 1891, besuchten \textcolor{blue}{Schnitzler} und \textcolor{blue}{Hofmannsthal} am Nachmittag den erkrankten \textcolor{blue}{Bahr}.}}}\label{K_L00046_1h} um
                  5 zu ihnen.\pend
           \pstart \spacefill\mbox{Loris.}\pend{}\endnumbering\briefempfaengerindex{Schnitzler, Arthur@\textsc{Schnitzler, Arthur}!zzzHofmannsthal, Hugo von@\emph{von Hugo von Hofmannsthal}!1891-12-041@{4. 12. 1891}|)be}\mylabel{h}  \normalsize

\doendnotes{C}
\bigskip
\vfill

\clearpage

\footnotesize

\lohead{\textsc{register}}

% Definiere theindex-Environment komplett neu ohne reledmac
\makeatletter
\renewenvironment{theindex}{%
  \section*{\indexname}%
  \setlength{\parindent}{0pt}%
  \setlength{\parskip}{0pt plus 0.3pt}%
  \let\item\@idxitem
}{%
  \clearpage
}
\makeatother

\IfFileExists{\jobname-pw.ind}{\input{\jobname-pw.ind}}{}

\end{document}

      