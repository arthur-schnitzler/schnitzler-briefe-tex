%% latex-korrekturansicht-vorspann.tex
%% Vorspann für die Korrekturansicht.
%% Lädt die gemeinsame Datei latex-vorspann.tex mit gesetztem Schalter.

\newif\ifkorrekturansicht
\korrekturansichttrue

\input{../tex-inputs/latex-vorspann}


               \section[Richard Beer-Hofmann an Arthur Schnitzler, 20. 7. 1910]{ Richard Beer-Hofmann an Arthur Schnitzler, 20. 7. 1910}\nopagebreak\mylabel{v}\rehead{ }\normalsize\beginnumbering\briefempfaengerindex{Schnitzler, Arthur@\textsc{Schnitzler, Arthur}!zzzBeer-Hofmann, Richard@\emph{von Richard Beer-Hofmann}!1910-07-201@{20. 7. 1910}|(be} \toendnotes[C]{\smallbreak\pagebreak[2]} \Standort{CUL, Schnitzler, B 8.}
\physDesc{Bildpostkarte
\newline{}Handschrift: Bleistift, lateinische Kurrent\newline{}Versand: 1) Stempel: »\nobreak{}\oindex{Bad Aussee@\textbf{Bad Aussee}, \emph{Besiedelter Ort (A.BSO)}|pwk}Aussee in Steiermark, 20. VII. 10\nobreak{}«.  2) mutmaßlich vom Briefträger die Straßenangabe in der Adressierung
                                 gestrichen
\newline{}Schnitzler: mit Bleistift beschriftet: »\textsc{BH}« \newline{}Ordnung: mit Bleistift von unbekannter Hand nummeriert:
                                    »235« }\buchAbdrucke{\weitereDrucke{Arthur Schnitzler, Richard Beer-Hofmann: \emph{Briefwechsel 1891–1931}. Hg. Konstanze Fliedl. Wien, Zürich: \emph{Europaverlag} 1992, S. 211.} }\toendnotes[C]{\smallbreak}\pstart{}{\pb}Herrn\pend{}\pstart{}Arthur Schnitzler\pend{}\pstart{}\textcolor{pink}{Wien}{}\ledrightnote{\textcolor{pink}{Wien}}\pend{}\pstart{}\textcolor{pink}{XVIII Spöttelgasse 7}{}\ledrightnote{\textcolor{pink}{Edmund-Weiß-Gasse}}\pend{}{\bigskip}\pstart
           \noindent{}\centering{}{\pb}\textcolor{gray}{\textbf{\textcolor{pink}{Aussee}{}\ledrightnote{\textcolor{pink}{Bad Aussee}} von der \textcolor{pink}{Tauscherin}{}\ledrightnote{\textcolor{pink}{Tauscherin}}.}}\pend
           \pstart
           \noindent{}{\pb}Lieber Arthur, wir
               sind auf 2 Stunden – ohne irgendwelche Besuchsabsichten hier. Das Erste was wir
               erblicken: Rothe Zettel »Das Repertoirestück des \textcolor{pink}{k.k.
                  Hofburgtheaters}{}\ledrightnote{\textcolor{pink}{Burgtheater}} ›\textcolor{green}{Liebelei}{}\ledrightnote{\textcolor{green}{Liebelei. Schauspiel in drei Akten}}‹«.\pend
           \pstart
           \label{T_L01949-1v}\edtext{Herzliche Grüße, Sie
                  Repertoirebeherrscher!}{\lemma{\textnormal{\emph{Herzliche … Repertoirebeherrscher!}}}\Cendnote{\textnormal{von der rechten
                  Seite weg gegen den Uhrzeigersinn an zwei Seiten entlang des Textes
                  geschrieben, die Unterschrift auf der dritten}}}\label{T_L01949-1h}\pend
           \pstart \spacefill\mbox{Richard}\pend{}\endnumbering\briefempfaengerindex{Schnitzler, Arthur@\textsc{Schnitzler, Arthur}!zzzBeer-Hofmann, Richard@\emph{von Richard Beer-Hofmann}!1910-07-201@{20. 7. 1910}|)be}\mylabel{h}  \normalsize

\doendnotes{C}
\bigskip
\vfill

\clearpage

\footnotesize

\lohead{\textsc{register}}

% Definiere theindex-Environment komplett neu ohne reledmac
\makeatletter
\renewenvironment{theindex}{%
  \section*{\indexname}%
  \setlength{\parindent}{0pt}%
  \setlength{\parskip}{0pt plus 0.3pt}%
  \let\item\@idxitem
}{%
  \clearpage
}
\makeatother

\IfFileExists{\jobname-pw.ind}{\input{\jobname-pw.ind}}{}

\end{document}

      