%% latex-korrekturansicht-vorspann.tex
%% Vorspann für die Korrekturansicht.
%% Lädt die gemeinsame Datei latex-vorspann.tex mit gesetztem Schalter.

\newif\ifkorrekturansicht
\korrekturansichttrue

\input{../tex-inputs/latex-vorspann}


               \section[Robert Adam an Arthur Schnitzler, 17. 7. 1918]{ Robert Adam an Arthur Schnitzler, 17. 7. 1918}\nopagebreak\mylabel{v}\rehead{ }\normalsize\beginnumbering\briefempfaengerindex{Schnitzler, Arthur@\textsc{Schnitzler, Arthur}!zzzAdam, Robert@\emph{von Robert Adam}!1918-07-171@{17. 7. 1918}|(be} \toendnotes[C]{\smallbreak\pagebreak[2]} \Standort{CUL, Schnitzler, B 1.}
\physDesc{Brief, 1 Blatt, 4 Seiten
\newline{}Handschrift: schwarze Tinte, deutsche Kurrent
\newline{}Schnitzler: 1) mit Bleistift beschriftet: »\textsc{Adam}« 2) mit rotem Buntstift zwei Unterstreichungen\newline{}Ordnung: von unbekannter Hand nummeriert: »4« }\Standort{Wien, Österreichische Nationalbibliothek, Cod.ser. 52.263, 217.}
\physDesc{Brief, maschinelle Abschrift
\newline{}Schreibmaschine}\toendnotes[C]{\smallbreak}\pstart
           \raggedleft{}{\pb}\textcolor{pink}{Andorf}{}\ledrightnote{\textcolor{pink}{Andorf}}, 17. Juli 1918.\pend
           \pstart{}Hochverehrter Herr Doktor!\pend\pstart
           Ich bin auf meiner Suche nach einem einſamen Erholungsorte – infolge einer während
               der Eiſenbahnfahrt vernommenen Äußerung einer Mitreiſenden – in dieſen kleinen
               bäuerlichen Ort des \textcolor{pink}{Innviertel}{}\ledrightnote{\textcolor{pink}{Innviertel}}s, nicht weit von \textcolor{pink}{Schärding}{}\ledrightnote{\textcolor{pink}{Schärding}} entfernt, geraten und habe das gefunden,
               was ich geſucht hatte: ungeſtörte Einſamkeit – nur manchmal verſucht ſich die ältere
                  \textcolor{blue}{Wirtstochter}{}\ledrightnote{→\textcolor{blue}{Hintermayer}} oder ein
               ſtrebſamer Jüngling der Nachbarſchaft im Klavierüben; letzteres hat ſeinen Grund
               darin, daß mein \textcolor{blue}{Wirt}{}\ledrightnote{→\textcolor{blue}{Hintermayer}} im Beſitze
               des Ortsklaviers iſt –, wundervolle fruchtbare Wieſen und Felder ringsum im
               Hügelland, weite Strec{\pb}ken
               abwechslungsreicher Nadelwälder, in denen es außer vielem Wild, das jetzt für mich
               leider nicht in Betracht kommt, Beeren und Schwämme gibt und endlich eine ſehr gute,
               reichliche und nach \textcolor{pink}{Wien}{}\ledrightnote{\textcolor{pink}{Wien}}er Begriffen äußerſt
               wohlfeile Friedenskoſt; denn man verfügt hier noch über Nahrungsmittel, deren
               Exiſtenz in \textcolor{pink}{Wien}{}\ledrightnote{\textcolor{pink}{Wien}} längſt zur Sage geworden iſt, vor
               allem reichlich über Mehl, Butter und Milch. Dieſes Phänomen iſt zum Teil darauf
               zurückzuführen, daß man Sommergäſte mit wenigen Ausnahmen rückſichtslos abweiſt und
               ſich Hamſterverſuchen gegenüber ſehr ſpröde zeigt; weshalb man mit mir eine Ausnahme
               gemacht hat, weiß ich eigentlich nicht recht, aber es geſchah – nach urſprünglicher
               Abweiſung – und ich bin dem Schickſal dafür ſehr dankbar. Ich glaube bereits die
               günſtigen Wirkungen der unſparſamen {\pb}Verköſtigung nicht nur auf meinen körperlichen, ſondern auch auf meinen geiſtigen
               Zuſtand wahrzunehmen, eine gewiſſe Fähigkeit, freier und ungenierter Gedankengängen
               nachzugehen, ohne beſorgen zu müſſen, daß ſie plötzlich – wie es in \textcolor{pink}{Wien}{}\ledrightnote{\textcolor{pink}{Wien}}{ }ſo oft geſchah – in die Sackgaſſe der Nahrungsfrage
               einzulaufen: dies Kriegsthema des Eſſens ſchien mir in Geſpräch und Denken ſchon ſo
               unvermeidlich wie der Kopf \textcolor{blue}{Karls I.}{}\ledrightnote{\textcolor{blue}{Charles I von England}} in den \textcolor{green}{Promemorien des armen \textsc{Dick}}{}\ledrightnote{→\textcolor{green}{David Copperfield}} im \textcolor{green}{\textsc{David Copperfield}}{}\ledrightnote{\textcolor{green}{David Copperfield}}.\pend
           \pstart
           Meine Lebensweiſe hier iſt von äußerſter Einfachheit: ich gehe nach dem Frühſtück in
               den Wald, laufe und liege drin bis zum Mittageſſen; bis zur Jauſe ſitze oder liege
               ich in oder beim Hauſe; dann gehe ich wieder in den Wald und verlaſſe ihn erſt, um
               zum Nachtmahl zu gehen; nach dem Nachtmahl ſpaziere ich ein wenig auf den Feldern
               umher und ſitze dann mit Bauern und Schul{\pb}lehrer beim \label{K_L02289_1v}\edtext{Moſt}{\lemma{\textnormal{\emph{Moſt}}}\Cendnote{\textnormal{gegärter Fruchtsaft}}}\label{K_L02289_1h}. Ich habe in
               zwei Wochen – außer der Zeitung – keine 20 Seiten im »\textcolor{green}{Siebenkäs}{}\ledrightnote{\textcolor{green}{Siebenkäs}}« geleſen und nur ſehr wenig geſchrieben. Trotzdem bin ich mit
               jener \textcolor{green}{Kriegstragödie}{}\ledrightnote{→\textcolor{green}{Robert}}, von der\strikeout{n} ich Ihnen erzählte, (der Kannibalengeſchichte)
               ziemlich weit gekommen; zum Niederſchreiben bin ich nur viel zu faul. Aber dieſes
               läßt ſich hoffentlich in \textcolor{pink}{Wien}{}\ledrightnote{\textcolor{pink}{Wien}} nachholen.\pend
           \pstart
           Die Kriegsſtimmung der hieſigen Bevölkerung, die durch die letzte Niederlage
               ſchreckliche Verluſte erlitten hat, iſt nicht viel beſſer als die der \textcolor{pink}{Wien}{}\ledrightnote{\textcolor{pink}{Wien}}er; vor Äußerungen der Erregung bewahrt ſie wohl nur ihre
               felſenhafte Zuverſicht, demnächſt zu \textcolor{pink}{Baiern}{}\ledrightnote{\textcolor{pink}{Bayern}} zu
               gehören: – worauf dieſer Glaube beruht, iſt nicht zu eruieren.\pend
           \pstart
           Mein Urlaub endet leider ſchon in 10 Tagen.\pend
           \pstart
           Mit den herzlichſten Grüßen\pend
           \pstart
           Ihr ergebener{\\[\baselineskip]}\spacefill\mbox{Robert Adam}\pend
           \leftskip=0em{}\endnumbering\briefempfaengerindex{Schnitzler, Arthur@\textsc{Schnitzler, Arthur}!zzzAdam, Robert@\emph{von Robert Adam}!1918-07-171@{17. 7. 1918}|)be}\mylabel{h}  \normalsize

\doendnotes{C}
\bigskip
\vfill

\clearpage

\footnotesize

\lohead{\textsc{register}}

% Definiere theindex-Environment komplett neu ohne reledmac
\makeatletter
\renewenvironment{theindex}{%
  \section*{\indexname}%
  \setlength{\parindent}{0pt}%
  \setlength{\parskip}{0pt plus 0.3pt}%
  \let\item\@idxitem
}{%
  \clearpage
}
\makeatother

\IfFileExists{\jobname-pw.ind}{\input{\jobname-pw.ind}}{}

\end{document}

      