%% latex-korrekturansicht-vorspann.tex
%% Vorspann für die Korrekturansicht.
%% Lädt die gemeinsame Datei latex-vorspann.tex mit gesetztem Schalter.

\newif\ifkorrekturansicht
\korrekturansichttrue

\input{../tex-inputs/latex-vorspann}


               \section[Richard Beer-Hofmann an Arthur Schnitzler, 9. 7. 1895]{ Richard Beer-Hofmann an Arthur Schnitzler, 9. 7. 1895}\nopagebreak\mylabel{v}\rehead{ }\normalsize\beginnumbering\briefempfaengerindex{Schnitzler, Arthur@\textsc{Schnitzler, Arthur}!zzzBeer-Hofmann, Richard@\emph{von Richard Beer-Hofmann}!1895-07-091@{9. 7. 1895}|(be} \toendnotes[C]{\smallbreak\pagebreak[2]} \Standort{CUL, Schnitzler, B 8.}
\physDesc{Briefkarte
\newline{}Handschrift: Bleistift, lateinische Kurrent
\newline{}Schnitzler: mit Bleistift nummeriert: »63« }\buchAbdrucke{\weitereDrucke{Arthur Schnitzler, Richard Beer-Hofmann: \emph{Briefwechsel 1891–1931}. Hg. Konstanze Fliedl. Wien, Zürich: \emph{Europaverlag} 1992, S. 78.} }\toendnotes[C]{\smallbreak}\pstart
           \raggedleft{}{\pb}\textcolor{pink}{Ischl}{}\ledrightnote{\textcolor{pink}{Bad Ischl}}{ }9/VII 95\pend
           \pstart
           Lieber Arthur! Natürlich hab ich Ihnen nicht geschrieben, und ebenso
               natürlich hab ich Gewissensbisse. \textcolor{blue}{Blumenthal}{}\ledrightnote{\textcolor{blue}{Oskar Blumenthal}} ist
               hier – in eigener \textcolor{pink}{Villa}{}\ledrightnote{→\textcolor{pink}{Villa Blumenthal}}–. \textcolor{blue}{Jarno}{}\ledrightnote{\textcolor{blue}{Josef Jarno}} hat heute die
               Première \uline{seines}
                stückes »\textcolor{green}{der Rabenvater}{}\ledrightnote{\textcolor{green}{Der Rabenvater. Schwank in drei Akten}}« (noch irgend ein \textcolor{blue}{Compagnon}{}\ledrightnote{→\textcolor{blue}{Hanns Friedrich Fischer}} ist dabei). Es lebe der neue \textcolor{blue}{Kadelburg}{}\ledrightnote{\textcolor{blue}{Gustav Kadelburg}}!\pend
           \pstart
           {\pb}Er hatte die ungeheuerliche Idee »\textcolor{green}{Liebelei}{}\ledrightnote{\textcolor{green}{Liebelei. Schauspiel in drei Akten}}« hier
               aufführen zu wollen. In \textcolor{pink}{Berlin}{}\ledrightnote{\textcolor{pink}{Berlin}}
                soll er darin
               mitspielen. \textcolor{blue}{Nhil}{}\ledrightnote{\textcolor{blue}{Robert Nhil}} war, – ist möglicherweise
               noch hier. Der kleine \textcolor{blue}{Kraus}{}\ledrightnote{\textcolor{blue}{Karl Kraus}} hat bereits
               3 mal mit tiefer Herzlichkeit mir die Hand geschüttelt. Es waren i{\geminationm}er andere
               dabei. Er ist köstlich verlegen, nur ich schweige was ihn sehr beunruhigt. Sie ko{\geminationm}en bald?\pend
           \pstart Herzlichst Ihr \spacefill\mbox{R.}\pend{}\endnumbering\briefempfaengerindex{Schnitzler, Arthur@\textsc{Schnitzler, Arthur}!zzzBeer-Hofmann, Richard@\emph{von Richard Beer-Hofmann}!1895-07-091@{9. 7. 1895}|)be}\mylabel{h}  \normalsize

\doendnotes{C}
\bigskip
\vfill

\clearpage

\footnotesize

\lohead{\textsc{register}}

% Definiere theindex-Environment komplett neu ohne reledmac
\makeatletter
\renewenvironment{theindex}{%
  \section*{\indexname}%
  \setlength{\parindent}{0pt}%
  \setlength{\parskip}{0pt plus 0.3pt}%
  \let\item\@idxitem
}{%
  \clearpage
}
\makeatother

\IfFileExists{\jobname-pw.ind}{\input{\jobname-pw.ind}}{}

\end{document}

      