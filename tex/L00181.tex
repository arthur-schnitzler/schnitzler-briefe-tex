%% latex-korrekturansicht-vorspann.tex
%% Vorspann für die Korrekturansicht.
%% Lädt die gemeinsame Datei latex-vorspann.tex mit gesetztem Schalter.

\newif\ifkorrekturansicht
\korrekturansichttrue

\input{../tex-inputs/latex-vorspann}


               \section[Friedrich M. Fels an Arthur Schnitzler, 23. 2. 1893]{ Friedrich M. Fels an Arthur Schnitzler, 23. 2. 1893}\nopagebreak\mylabel{v}\rehead{ }\normalsize\beginnumbering\briefempfaengerindex{Schnitzler, Arthur@\textsc{Schnitzler, Arthur}!zzzFels, Friedrich Michael@\emph{von Friedrich Michael Fels}!1893-02-231@{23. 2.1893}|(be} \toendnotes[C]{\smallbreak\pagebreak[2]} \Standort{DLA, A:Schnitzler, HS.NZ85.1.2956.}
\physDesc{Brief, 1 Blatt, 3 Seiten
\newline{}Handschrift: schwarze Tinte, lateinische Kurrent
\newline{}Schnitzler: mit Bleistift nummeriert: »7« }\toendnotes[C]{\smallbreak}\pstart
           \raggedleft{}{\pb}\textcolor{pink}{Meran-Obermais}{}\ledrightnote{\textcolor{pink}{Obermais}}, den 23. Februar
                     1893\pend
           \pstart\center{}Lieber Dr. Schnitzler!\pend\pstart
           Soeben empfange ich Ihren Brief und beeile mich, ihn zu beantworten. Seien Sie jetzt
               nur nicht so boshaft, diese Schnelligkeit allein meiner Langeweile
               zuzuschreiben! –\pend
           \pstart
           Allerdings setze ich jetzt mehr Vertrauen in \textcolor{pink}{Meran}{}\ledrightnote{\textcolor{pink}{Meran}}
               und seine Heilkraft und zwar weil ich ich letztere an meinem eigenen Leichnam
               verspürt habe; de{\geminationn} entschieden geht es mir schon etwas,
                  we{\geminationn} auch noch nicht viel, beſser. Ich fühle mich im
               Kopf wohler, und meine Füſse schmerzen mich nicht mehr so sehr. Die beiden letzten
               Tage habe ich sogar einen kleinen Spaziergang, ohne Rollwagen, versucht; und heute
               will ich es unternehmen, wenigstens nach \textcolor{pink}{Meran}{}\ledrightnote{\textcolor{pink}{Meran}}{ }\uline{hinunter} zu gehen.\pend
           \pstart
           Freilich pflege ich mich auch genügend. Ich ruhe sehr viel, und im Eſsen bilde ich
               mich zum Wetteſser aus. Ein hiesiger Arzt pflegt zu derartigen Kranken zu sagen
               »Eſsen Sie so, daſs man Sie im ganzen Hotel nur den ›Freſser‹ ne{\geminationn}t«, und an diese Weisung halte ich mich auch, obwol es
               nicht mein Arzt ist. Mit dem Wein ist die Sache etwas unangenehm. {\pb}Der leichte rote Tyroler, den ich zu trinken pflege,
               ist sehr \label{K_L00181_1v}\edtext{tanin}{\lemma{\textnormal{\emph{tanin}}}\Cendnote{\textnormal{französisch: Gerbstoff}}}\label{K_L00181_1h}haltig und
               bereitet mir Unterleibsbeschwerden. Weiſswein soll ich nicht trinken, und die anderen
               Rotweine sind furchtbar teuer. Ich habe mir jetzt so geholfen, daſs ich mittags roten
               nehme, in den Ihre Medizin ko{\geminationm}t, abends weiſser: das
               reine Gewebe der Penelope. – Dreimal täglich nehme ich jetzt auch \textcolor{brown}{Gude}{}\ledrightnote{\textcolor{brown}{Dr. A. Gude GmbH.}}’s Mangan-Eisen-Pepton-Essenz. Wollen Sie sich, bitte, darnach
               erkundigen, und mir schreiben, was man davon hält. Da sie nämlich in der hiesigen
               Apotheke nicht vorrätig war und erst aus \textcolor{pink}{Leipzig}{}\ledrightnote{\textcolor{pink}{Leipzig}}
               verschrieben werden muſste, sowie aus anderen Gründen glaube ich, daſs sie ein ganz
               neues Mittel ist und ich dem Dr \textcolor{blue}{Schreiber}{}\ledrightnote{\textcolor{blue}{Joseph Schreiber}} als
               Versuchskanichen diene. Es würde mich intereſsieren, etwas zu erfahren.\pend
           \pstart
           Das Wetter ist nicht andauernd schön: einen Tag hat es geregnet; und am folgenden
               Morgen lag sogar etwas Schnee, aber schon mittags nahm ihn die So{\geminationn}e hinweg. Jetzt ist’s wieder; aber heizen muſs ich mir
               doch noch morgens und abends laſsen. Natürlich trage ich Winterkleider und gehe nie
               ohne Mantel aus.\pend
           \pstart
           Meine Gelder sind riesig zusa{\geminationm}engeschmolzen. Unter den
                  \textcolor{pink}{Wien}{}\ledrightnote{\textcolor{pink}{Wien}}er Auslagen, die ich Ihnen angab, vergaſs {\pb}ich noch die Rechnung meiner \textcolor{blue}{Wirtin}{}\ledrightnote{→\textcolor{blue}{?? [Vermieterin von F. M. Fels]}}, die auch gegen 10 fl betrug. So kam
               ich mit 38 fl hier an. Davon habe ich in die Apotheke fl 7.40 und dem Badediener fl 4
               (für 2 Wochen Baden und Frottieren) bezahlt; Sie kö{\geminationn}en
               Sich denken, wie ich finanziell stehe. Auch habe ich in der ersten Woche, bei meiner
                  Unbeka{\geminationn}tschaft mit hiesigen Verhältniſsen, im Hotel
               eine ziemlich groſse Rechnung gemacht, so daſs ich auf Eingang von \textcolor{blue}{Gelber}{}\ledrightnote{\textcolor{blue}{Ludwig Gelber}}s und \textcolor{blue}{Steinbach}{}\ledrightnote{\textcolor{blue}{Josef Steinbach}}s Sa{\geminationm}lung mit Sicherheit rechnen muſs: sonst bin ich
               verloren. Beide sind übrigens bereits moniert. –\pend
           \pstart
           Bitte, richten Sie allen lieben Beka{\geminationn}ten herzliche
               Grüſse aus: \textcolor{blue}{Beer-Hofma{\geminationn}}{}\ledrightnote{\textcolor{blue}{Richard Beer-Hofmann}}, \textcolor{blue}{Loris}{}\ledrightnote{\textcolor{blue}{Hugo von Hofmannsthal}}, \textcolor{blue}{Salten}{}\ledrightnote{\textcolor{blue}{Felix Salten}}, \textcolor{blue}{Engländer}{}\ledrightnote{\textcolor{blue}{Peter Altenberg}} und we{\geminationn} Sie sonst noch jemanden treffen, und sagen Sie ihnen,
               es möge mir der eine oder andere auch einmal schreiben. Ich schreibe ihnen nicht,
               weil ich annehme, daſs meine Briefe an Sie ihnen mitgeteilt werden. Für Ihre Wünsche
               zu meiner Genesung dankend, verbleibe ich\pend
           \pstart
           Ihr{\\[\baselineskip]}dankbar ergebener{\\[\baselineskip]}\spacefill\mbox{Fels}\pend
           \leftskip=0em{}\endnumbering\briefempfaengerindex{Schnitzler, Arthur@\textsc{Schnitzler, Arthur}!zzzFels, Friedrich Michael@\emph{von Friedrich Michael Fels}!1893-02-231@{23. 2.1893}|)be}\mylabel{h}  \normalsize

\doendnotes{C}
\bigskip
\vfill

\clearpage

\footnotesize

\lohead{\textsc{register}}

% Definiere theindex-Environment komplett neu ohne reledmac
\makeatletter
\renewenvironment{theindex}{%
  \section*{\indexname}%
  \setlength{\parindent}{0pt}%
  \setlength{\parskip}{0pt plus 0.3pt}%
  \let\item\@idxitem
}{%
  \clearpage
}
\makeatother

\IfFileExists{\jobname-pw.ind}{\input{\jobname-pw.ind}}{}

\end{document}

      