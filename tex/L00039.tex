%% latex-korrekturansicht-vorspann.tex
%% Vorspann für die Korrekturansicht.
%% Lädt die gemeinsame Datei latex-vorspann.tex mit gesetztem Schalter.

\newif\ifkorrekturansicht
\korrekturansichttrue

\input{../tex-inputs/latex-vorspann}


               \section[Arthur Schnitzler an Hugo von Hofmannsthal, 11. 9. 1891]{ Arthur Schnitzler an Hugo von Hofmannsthal, 11. 9. 1891}\nopagebreak\mylabel{v}\rehead{ }\normalsize\beginnumbering\briefempfaengerindex{Hofmannsthal, Hugo von@\textsc{Hofmannsthal, Hugo von}!zzzSchnitzler, Arthur@\emph{von Arthur Schnitzler}!1891-09-111@{11. 9. 1891}|(be} \toendnotes[C]{\smallbreak\pagebreak[2]} \Standort{FDH, Hs-30885,15.}
\physDesc{Brief, 1 Blatt, 3 Seiten
\newline{}Handschrift: schwarze Tinte, deutsche Kurrent\newline{}Ordnung: auf der ersten Seite wurde von Schnitzler mutmaßlich bei der
                                 Durchsicht der Korrespondenz 1929 mit Bleistift das
                                 Datum falsch ergänzt: »11/7 91« }\buchAbdrucke{\weitereDrucke{Hugo von Hofmannsthal, Arthur Schnitzler: \emph{Briefwechsel}. Hg. Therese Nickl und Heinrich Schnitzler. Frankfurt am Main: \emph{S. Fischer} 1964, S. 13.} }\toendnotes[C]{\smallbreak}\pstart\center{}{\pb}Lieber Freund,\pend\pstart
           der Anfang von \textcolor{green}{Reichtum}{}\ledrightnote{\textcolor{green}{Reichtum. Erzählung}} iſt abſcheulich – Sie
               kennen ja die \textcolor{green}{Moderne Rundſchau}{}\ledrightnote{\textcolor{green}{Moderne Rundschau}}! – plötzlich wurde
               das Ding geſetzt, obwohl es ausgemacht war, daß die erſten Kapitel vorher verändert
               werden müſſten. Jedenfalls änder’ ich für den \label{K_L00039_1v}\edtext{\textcolor{green}{Separatabdruck}{}\ledrightnote{→\textcolor{green}{Reichtum. Erzählung}}}{\lemma{\textnormal{\emph{Separatabdruck}}}\Cendnote{\textnormal{\emph{\textcolor{green}{Reichtum}}. Erzählung von \textcolor{blue}{Arthur Schnitzler}. Separat-Abdruck aus der »\emph{\textcolor{green}{Modernen Rundschau}}«. Druck von \emph{\textcolor{brown}{Carl Steinhardt { }{\kaufmannsund} Cie.}}{ }{[}1891{]}.}}}\label{K_L00039_1h}. Die Fortſetzung iſt beſſer. Vorläufig {\pb}werd ich in den weiteſten Kreiſen verachtet. –\pend
           \pstart
           Wann kommen Sie? Durch wen hab ich Sie grüßen laſſen? \textcolor{blue}{\textsc{Salten}}{}\ledrightnote{\textcolor{blue}{Felix Salten}} iſt in \textcolor{pink}{Miskolcz}{}\ledrightnote{\textcolor{pink}{Miskolc}}, das wiſſen Sie wohl. Von
                  \textcolor{blue}{\textsc{Beer-Hofma{\geminationn}}}{}\ledrightnote{\textcolor{blue}{Richard Beer-Hofmann}} hab ich keine Nachricht. Das \textcolor{green}{Mährchen}{}\ledrightnote{\textcolor{green}{Das Märchen. Schauspiel in drei Aufzügen}} reich
               ich der \textcolor{pink}{Burg}{}\ledrightnote{\textcolor{pink}{Burgtheater}} ein, laſs es vorher als \label{K_L00039_2v}\edtext{Manuscript}{\lemma{\textnormal{\emph{Manuscript}}}\Cendnote{\textnormal{\textcolor{blue}{Arthur Schnitzler}: \emph{\textcolor{green}{Das Märchen. Schauspiel in drei Aufzügen}}. Wien: \emph{\textcolor{brown}{Carl Steinhardt}}{ }1891.}}}\label{K_L00039_2h} drucken. {\pb}Bringen Sie was mit? Bringen Sie was
               mit! –\pend
           \pstart
           Leben Sie wohl, ich freu mich ſehr Sie bald wiederzuſehen. Ganz der Ihre{\\[\baselineskip]}\spacefill\mbox{Arth Sch}\pend
           \leftskip=0em{}\pstart
           \textcolor{pink}{Wien}{}\ledrightnote{\textcolor{pink}{Wien}}{ }11. Sept. 91.\pend
           \endnumbering\briefempfaengerindex{Hofmannsthal, Hugo von@\textsc{Hofmannsthal, Hugo von}!zzzSchnitzler, Arthur@\emph{von Arthur Schnitzler}!1891-09-111@{11. 9. 1891}|)be}\mylabel{h}  \normalsize

\doendnotes{C}
\bigskip
\vfill

\clearpage

\footnotesize

\lohead{\textsc{register}}

% Definiere theindex-Environment komplett neu ohne reledmac
\makeatletter
\renewenvironment{theindex}{%
  \section*{\indexname}%
  \setlength{\parindent}{0pt}%
  \setlength{\parskip}{0pt plus 0.3pt}%
  \let\item\@idxitem
}{%
  \clearpage
}
\makeatother

\IfFileExists{\jobname-pw.ind}{\input{\jobname-pw.ind}}{}

\end{document}

      