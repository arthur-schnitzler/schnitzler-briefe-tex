%% latex-korrekturansicht-vorspann.tex
%% Vorspann für die Korrekturansicht.
%% Lädt die gemeinsame Datei latex-vorspann.tex mit gesetztem Schalter.

\newif\ifkorrekturansicht
\korrekturansichttrue

\input{../tex-inputs/latex-vorspann}


               \section[Richard Beer-Hofmann an Arthur Schnitzler, {[}29. 5. 1895{]}]{ Richard Beer-Hofmann an Arthur Schnitzler,
               {[}29. 5. 1895{]}}\nopagebreak\mylabel{v}\rehead{ }\normalsize\beginnumbering\briefempfaengerindex{Schnitzler, Arthur@\textsc{Schnitzler, Arthur}!zzzBeer-Hofmann, Richard@\emph{von Richard Beer-Hofmann}!1895-05-291@{{[}29. 5. 1895{]}}|(be} \toendnotes[C]{\smallbreak\pagebreak[2]} \Standort{CUL, Schnitzler, B 8.}
\physDesc{Brief, 1 Blatt, 3 Seiten
\newline{}Handschrift: Bleistift, lateinische Kurrent
\newline{}Schnitzler: mit Bleistift nummeriert: »57« und datiert »29/5 95« }\buchAbdrucke{\weitereDrucke{Arthur Schnitzler, Richard Beer-Hofmann: \emph{Briefwechsel 1891–1931}. Hg. Konstanze Fliedl. Wien, Zürich: \emph{Europaverlag} 1992, S. 72.} }\pstart
           \noindent{}{\pb}Lieber Arthur! Dr. \textcolor{blue}{G. N.}{}\ledrightnote{\textcolor{blue}{Gabor Nobl}} hätte
               gestern zu mir ko{\geminationm}en sollen; er war aber weder gestern
               noch heute bei mir: Haben Sie die Güte ihm beiliegende 20 fl zu übermitteln. Er
               gibt Ihnen wol auch Auskunft über den wirklichen Tatbestand, den er ja inzwischen
               erhoben {\pb}haben dürfte. Meine
               Adresse ist\pend
           \pstart
           \uline{n. a. Lieut. im k-k. Landw. Inf. Rgmt. \textcolor{pink}{Caslau}{}\ledrightnote{\textcolor{pink}{Caslau}} – N\textsuperscript{o} 12}. Bitte
               schreiben Sie mir. Grüßen Sie bitte \textcolor{blue}{Salten}{}\ledrightnote{\textcolor{blue}{Felix Salten}}, auch
                  D\textsuperscript{r.}{ }\textcolor{blue}{G. N.}{}\ledrightnote{\textcolor{blue}{Gabor Nobl}} Empfehlung und besten Dank.\pend
           \pstart
           {\pb}Mir ist mis.\pend
           \pstart
           Herzlichst Ihr{\\[\baselineskip]}\spacefill\mbox{Richard.}\pend
           \leftskip=0em{}\endnumbering\briefempfaengerindex{Schnitzler, Arthur@\textsc{Schnitzler, Arthur}!zzzBeer-Hofmann, Richard@\emph{von Richard Beer-Hofmann}!1895-05-291@{{[}29. 5. 1895{]}}|)be}\mylabel{h}  \normalsize

\doendnotes{C}
\bigskip
\vfill

\clearpage

\footnotesize

\lohead{\textsc{register}}

% Definiere theindex-Environment komplett neu ohne reledmac
\makeatletter
\renewenvironment{theindex}{%
  \section*{\indexname}%
  \setlength{\parindent}{0pt}%
  \setlength{\parskip}{0pt plus 0.3pt}%
  \let\item\@idxitem
}{%
  \clearpage
}
\makeatother

\IfFileExists{\jobname-pw.ind}{\input{\jobname-pw.ind}}{}

\end{document}

      