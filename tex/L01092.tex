%% latex-korrekturansicht-vorspann.tex
%% Vorspann für die Korrekturansicht.
%% Lädt die gemeinsame Datei latex-vorspann.tex mit gesetztem Schalter.

\newif\ifkorrekturansicht
\korrekturansichttrue

\input{../tex-inputs/latex-vorspann}


               \section[Hugo von Hofmannsthal an Arthur Schnitzler, {[}17. 1. 1901{]}]{ Hugo von Hofmannsthal an Arthur Schnitzler, {[}17. 1. 1901{]}}\nopagebreak\mylabel{v}\rehead{ }\normalsize\beginnumbering\briefempfaengerindex{Schnitzler, Arthur@\textsc{Schnitzler, Arthur}!zzzHofmannsthal, Hugo von@\emph{von Hugo von Hofmannsthal}!1901-01-171@{17. 1. 1901}|(be} \toendnotes[C]{\smallbreak\pagebreak[2]} \Standort{CUL, Schnitzler, B 43.}
\physDesc{Brief, 1 Blatt, 3 Seiten
\newline{}Handschrift: schwarze Tinte, deutsche Kurrent
\newline{}Schnitzler: mit Bleistift datiert: »17/1 901.« \newline{}Ordnung: 1) mit Bleistift von unbekannter Hand nummeriert: »\strikeout{190}« 2) mit Bleistift von unbekannter Hand nummeriert: »183«}\buchAbdrucke{\weitereDrucke{Hugo von Hofmannsthal, Arthur Schnitzler: \emph{Briefwechsel}. Hg. Therese Nickl und Heinrich Schnitzler. Frankfurt am Main: \emph{S. Fischer} 1964, S. 146.} }\toendnotes[C]{\smallbreak}\pstart{}{\pb}lieber,\pend\pstart
           falls Sie dem kranken Schriftſteller \textcolor{blue}{Hans Wagner}{}\ledrightnote{\textcolor{blue}{Hans Wagner}}
               keins von Ihren Büchern geſchickt haben, so thuen Sie es bitte doch noch; er hat mir
               einen ſo merkwürdigen ergreifenden Dankbrief geſchrieben, Geld will er abſolut nicht,
               aber die Freude, die er über Bücher hat, {\pb}iſt ſehr rührend und man kann ſich
               ſeinen Zuſtand ganz gut vorſtellen.\pend
           \pstart
           Er ist gewiſs ein Dichter, d. h. ein Menſch mit einem Fieber der Phantaſie, ſowie
                  »\textcolor{green}{mein Freund Y.}{}\ledrightnote{\textcolor{green}{Mein Freund Ypsilon}}«\pend
           \pstart
           Wahrſcheinlich iſt natürlich das was er ſchreibt, gar nichts werth. Auf
               Wiederſehen!\pend
           \pstart
           {\pb}Von Herzen Ihr{\\[\baselineskip]}\spacefill\mbox{Hugo}\pend
           \leftskip=0em{}\pstart
           \noindent{}An die \textcolor{green}{Frau Berthe{ }\textsc{Garlan}}{}\ledrightnote{\textcolor{green}{Frau Bertha Garlan. Roman}} hab ich mich gleich beim Aufwachen mit Freude erinnert.\pend
           \pstart
           Der arme \textcolor{blue}{Menſch}{}\ledrightnote{→\textcolor{blue}{Hans Wagner}} iſt im \textcolor{pink}{Eliſabethſpital}{}\ledrightnote{\textcolor{pink}{Kaiserin-Elisabeth-Spital}}{\\}Pavillon III{\\}Saal 3{\\}Bett 26.\pend
           \endnumbering\briefempfaengerindex{Schnitzler, Arthur@\textsc{Schnitzler, Arthur}!zzzHofmannsthal, Hugo von@\emph{von Hugo von Hofmannsthal}!1901-01-171@{17. 1. 1901}|)be}\mylabel{h}  \normalsize

\doendnotes{C}
\bigskip
\vfill

\clearpage

\footnotesize

\lohead{\textsc{register}}

% Definiere theindex-Environment komplett neu ohne reledmac
\makeatletter
\renewenvironment{theindex}{%
  \section*{\indexname}%
  \setlength{\parindent}{0pt}%
  \setlength{\parskip}{0pt plus 0.3pt}%
  \let\item\@idxitem
}{%
  \clearpage
}
\makeatother

\IfFileExists{\jobname-pw.ind}{\input{\jobname-pw.ind}}{}

\end{document}

      