%% latex-korrekturansicht-vorspann.tex
%% Vorspann für die Korrekturansicht.
%% Lädt die gemeinsame Datei latex-vorspann.tex mit gesetztem Schalter.

\newif\ifkorrekturansicht
\korrekturansichttrue

\input{../tex-inputs/latex-vorspann}


               \section[Arthur Schnitzler an Georg Brandes, 18. 7. 1897]{ Arthur Schnitzler an Georg Brandes, 18. 7. 1897}\nopagebreak\mylabel{v}\rehead{ }\normalsize\beginnumbering\briefempfaengerindex{Brandes, Georg@\textsc{Brandes, Georg}!zzzSchnitzler, Arthur@\emph{von Arthur Schnitzler}!1897-07-181@{18. 7. 1897}|(be} \toendnotes[C]{\smallbreak\pagebreak[2]} \Standort{Kopenhagen, Det Kongelige Bibliotek, Georg Brandes Arkiv, box 125.}
\physDesc{Brief, 2 Blätter, 7 Seiten
\newline{}Handschrift: schwarze Tinte, deutsche Kurrent\newline{}Ordnung: mit Bleistift von unbekannter Hand nummeriert: »9. Schnitzler«, das zweite
                                    Blatt mit »18/7 97« gekennzeichnet }\buchAbdrucke{\weitereDrucke{Georg Brandes, Arthur Schnitzler: \emph{Ein Briefwechsel}. Hg. Kurt Bergel. Bern: \emph{Francke} 1956, S. 64–65.} }\pstart
           \raggedleft{}{\pb}\textcolor{pink}{\textsc{Ischl}}{}\ledrightnote{\textcolor{pink}{Bad Ischl}}, 18. 7. 97. \pend
           \pstart{}Verehrteſter Herr Brandes,\pend\pstart
           Ich danke Ihnen herzlich, dſs Sie mir ſo ſchnell eine Nachricht haben zugehen
                    laſſen. Vor allem entnehme ich ihr, daſs jede Gefahr vorüber iſt, und das iſt ja
                    das weſentliche. Auch ſcheint es, dſs Sie ſchon wieder arbeiten dürfen – und
                    ſogar ſich aergern – we{\geminationn} das mit aerztlicher {\pb}Erlaubnis geſchieht? Aber mir ſcheint
                    wirklich, Sie ſind mit den deutſchen Überſetzungen ein bischen gar zu ſtreng –
                    die Leute, die nicht das Glück haben, Überſetzungen Ihrer Bücher mit dem Urtext
                    vergleichen zu können, finden auch in dieſen Überſetzungen irgend was und ſogar
                    ſehr viel, das \introOben{}ihnen\introOben{} trotz Misverſtändniſſen u
                    Flüchtigkeiten (die ja uns \introOben{}großentheils\introOben{} entgehen) der
                    ganze Georg Brandes zu ſein ſcheint. {\pb}Freilich ahnt man oft, daſs hier ein Zauber verloren gegangen iſt, der
                    unwiederbringlich iſt; – aber glauben Sie mir, es bleibt noch i{\geminationm}er ſo viel Zauber übrig, daſs die meiſten gar
                    nicht dazu ko{\geminationm}en, den fehlenden zu vermiſſen. Ich
                    gehöre ja leider auch zu denen, die nicht \textcolor{pink}{däniſch}{}\ledrightnote{\textcolor{pink}{Dänemark}} verſtehn – und Sie haben mir noch jedesmal, durch die
                    ſchwächſten Übertragungen hindurch, wahrhaftig {\pb}viel gegeben!\pend
           \pstart
           Ich wuſste nicht, dſs \textcolor{blue}{Paul Goldmann}{}\ledrightnote{\textcolor{blue}{Paul Goldmann}} Ihnen
                    ſchon lange Zeit nicht geſchrieben hat. Aber Sie können kaum ahnen, was dieſer
                    Mann zu thun hat. Ich bin im Frühjahr in \textcolor{pink}{Paris}{}\ledrightnote{\textcolor{pink}{Paris}}
                    geweſen, und habe manche Tage mit ihm verbracht; er ko{\geminationm}t überhaupt kaum je eine Viertelſtunde zur Ruhe.
                    Allerdings hat er etwas zu viel Gewiſſen und opfert meiner An{\pb}ſicht nach der \textcolor{brown}{Frankf. Zeitg}{}\ledrightnote{\textcolor{brown}{Frankfurter Zeitung}} mehr von dem beſten ſeines Lebens auf, als ſie ihm je
                    danken wird. Da der Gruſs an meine Freunde wohl ihm und Dr. \textcolor{blue}{\textsc{Beer-Hofma{\geminationn}}}{}\ledrightnote{\textcolor{blue}{Richard Beer-Hofmann}} gilt, hab ich ihn beiden mitgetheilt. Dr \textcolor{blue}{\textsc{B. H.}}{}\ledrightnote{\textcolor{blue}{Richard Beer-Hofmann}} iſt hier und dankt Ihnen vielmals; er
                    verbindet ſeine beſten Wünſche für Ihre baldige vollko{\geminationm}ene Geneſung mit den meinen.\pend
           \pstart
           {\pb}Eine Frage an Sie hatte ich mir ſchon
                    neulich vorgenommen: Haben Sie die Skizzen von \textcolor{blue}{\textsc{Altenberg}}{}\ledrightnote{\textcolor{blue}{Peter Altenberg}} geleſen? (Es iſt ein Buch: »\textcolor{green}{Wie ich es ſehe}{}\ledrightnote{\textcolor{green}{Wie ich es sehe}},« der Autor hat es Ihnen wohl
                    geſchickt.)\pend
           \pstart
           Ich ſchreibe jetzt, nach einigen kleinern Erzählungen, wieder ein Stück und habe
                    mehr Freude daran als von meinem letzten. Ob es beſſer wird, \strikeout{f} weiſs ich freilich {\pb}noch nicht. Aber das Freudhaben iſt ja doch
                    das wichtigere. –\pend
           \pstart
           In wenigen Tagen fahre ich wieder nach \textcolor{pink}{Wien}{}\ledrightnote{\textcolor{pink}{Wien}}
                    zurück; vielleicht erfreuen Sie mich bald wieder durch ein Wort; und wär es auch
                    nur das eine »Geſundheit.«\pend
           \pstart Ich grüße Sie, hochverehrter Herr Brandes, in herzlichſter Ergebenheit.
                        \spacefill\mbox{ArthurSchnitzler}\pend{}\endnumbering\briefempfaengerindex{Brandes, Georg@\textsc{Brandes, Georg}!zzzSchnitzler, Arthur@\emph{von Arthur Schnitzler}!1897-07-181@{18. 7. 1897}|)be}\mylabel{h}  \normalsize

\doendnotes{C}
\bigskip
\vfill

\clearpage

\footnotesize

\lohead{\textsc{register}}

% Definiere theindex-Environment komplett neu ohne reledmac
\makeatletter
\renewenvironment{theindex}{%
  \section*{\indexname}%
  \setlength{\parindent}{0pt}%
  \setlength{\parskip}{0pt plus 0.3pt}%
  \let\item\@idxitem
}{%
  \clearpage
}
\makeatother

\IfFileExists{\jobname-pw.ind}{\input{\jobname-pw.ind}}{}

\end{document}

      