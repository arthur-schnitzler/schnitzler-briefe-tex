%% latex-korrekturansicht-vorspann.tex
%% Vorspann für die Korrekturansicht.
%% Lädt die gemeinsame Datei latex-vorspann.tex mit gesetztem Schalter.

\newif\ifkorrekturansicht
\korrekturansichttrue

\input{../tex-inputs/latex-vorspann}


               \section[Max Mell an Arthur Schnitzler, 15. 7. 1907]{ Max Mell an Arthur Schnitzler, 15. 7. 1907}\nopagebreak\mylabel{v}\rehead{ }\normalsize\beginnumbering\briefempfaengerindex{Schnitzler, Arthur@\textsc{Schnitzler, Arthur}!zzzMell, Max@\emph{von Max Mell}!1907-07-151@{15. 7. 1907}|(be} \toendnotes[C]{\smallbreak\pagebreak[2]} \Standort{CUL, Schnitzler, B 70.}
\physDesc{Brief, 1 Blatt, 1 Seite
\newline{}Handschrift: schwarze Tinte, deutsche Kurrent
\newline{}Schnitzler: mit Bleistift beschriftet: »\textsc{Mell}« }\toendnotes[C]{\smallbreak}\pstart
           \noindent{}{\pb}15/VII.\hfill 1907\pend
           \pstart
           \centering{}\textcolor{gray}{\textbf{WW WIENER}}\pend
           \pstart
           \noindent{}\centering{}\textcolor{gray}{\textbf{\textcolor{brown}{WERKSTÆTTE}{}\ledrightnote{\textcolor{brown}{Wiener Werkstätte}}}}\pend
           \pstart
           \noindent{}\centering{}\textcolor{gray}{\textbf{\textcolor{pink}{7}{}\ledrightnote{\textcolor{pink}{VII., Neubau}}}}\pend
           \pstart
           \noindent{}\centering{}\textcolor{gray}{\textbf{\textcolor{pink}{NEUSTIFTGASSE}{}\ledrightnote{\textcolor{pink}{Neustiftgasse}}}}\pend
           \pstart
           \noindent{}\centering{}\textcolor{gray}{\textbf{32}}\pend
           \pstart{}Sehr verehrter Herr Doktor,\pend\pstart
           im Herbſt will die »\textcolor{brown}{Wiener Werkſtätte}{}\ledrightnote{\textcolor{brown}{Wiener Werkstätte}}« einen
                        \label{K_L01692_1v}\edtext{Almanach}{\lemma{\textnormal{\emph{Almanach}}}\Cendnote{\textnormal{In der hier präsentierten Form kam
                        der Almanach nicht zustande. Erst 1911 erschien ein solcher Almanach.}}}\label{K_L01692_1h} »\textcolor{green}{Die Frau}{}\ledrightnote{→\textcolor{green}{Almanach der Wiener Werkstätte}}« herausgeben, ich
                    bin mit der Redaktion betraut und bitte Sie nun, mich mit einem Beitrag zu
                    unterſtützen. Hoffentlich können Sie mir dieſe Freude machen! Ich ſoll die
                    Einſendungen bis Anfang September beiſammen haben, was ſchon etwas knapp iſt,
                    aber \textcolor{blue}{Waerndorfer}{}\ledrightnote{\textcolor{blue}{Friedrich Wärndorfer}} und \textcolor{blue}{Hoffmann}{}\ledrightnote{\textcolor{blue}{Josef Hoffmann}} konnten ſich ſolange nicht entſchließen. Es iſt
                    ſelbſtverſtändlig, daß Sie nur in die beſte Geſellschaft kommen.\pend
           \pstart
           Es war mir ſehr leid, Sie nicht mehr geſehen zu haben. So wünſch ich Ihnen und
                    Ihrer verehrten \textcolor{blue}{Frau}{}\ledrightnote{→\textcolor{blue}{Olga Schnitzler}}
                    ſchriftlich, aber nicht minder herzlich recht angenehmen Sommer. – Ich bleib
                    noch da, \textcolor{blue}{Mary}{}\ledrightnote{\textcolor{blue}{Maria Mell}} iſt in \textcolor{pink}{Ungarn}{}\ledrightnote{\textcolor{pink}{Ungarn}}.\pend
           \pstart
           Mit den beſten Empfehlungen{\\[\baselineskip]}Ihr ſehr ergebener{\\[\baselineskip]}\spacefill\mbox{Max Mell.}\pend
           \leftskip=0em{}\pstart
           \textcolor{pink}{II. Wittelsbachſtr. 5}{}\ledrightnote{\textcolor{pink}{Wittelsbachstraße}}.\pend
           \endnumbering\briefempfaengerindex{Schnitzler, Arthur@\textsc{Schnitzler, Arthur}!zzzMell, Max@\emph{von Max Mell}!1907-07-151@{15. 7. 1907}|)be}\mylabel{h}  \normalsize

\doendnotes{C}
\bigskip
\vfill

\clearpage

\footnotesize

\lohead{\textsc{register}}

% Definiere theindex-Environment komplett neu ohne reledmac
\makeatletter
\renewenvironment{theindex}{%
  \section*{\indexname}%
  \setlength{\parindent}{0pt}%
  \setlength{\parskip}{0pt plus 0.3pt}%
  \let\item\@idxitem
}{%
  \clearpage
}
\makeatother

\IfFileExists{\jobname-pw.ind}{\input{\jobname-pw.ind}}{}

\end{document}

      