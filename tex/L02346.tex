%% latex-korrekturansicht-vorspann.tex
%% Vorspann für die Korrekturansicht.
%% Lädt die gemeinsame Datei latex-vorspann.tex mit gesetztem Schalter.

\newif\ifkorrekturansicht
\korrekturansichttrue

\input{../tex-inputs/latex-vorspann}


               \section[Richard Beer-Hofmann an Arthur Schnitzler, 9. 7. 1920]{ Richard Beer-Hofmann an Arthur Schnitzler, 9. 7. 1920}\nopagebreak\mylabel{v}\rehead{ }\normalsize\beginnumbering\briefempfaengerindex{Schnitzler, Arthur@\textsc{Schnitzler, Arthur}!zzzBeer-Hofmann, Richard@\emph{von Richard Beer-Hofmann}!1920-07-091@{9. 7. 1920}|(be} \toendnotes[C]{\smallbreak\pagebreak[2]} \Standort{CUL, Schnitzler, B 8.}
\physDesc{Bildpostkarte
\newline{}Handschrift: Bleistift, lateinische Kurrent\newline{}Versand: Stempel: »\nobreak{}\oindex{Bad Aussee@\textbf{Bad Aussee}, \emph{Besiedelter Ort (A.BSO)}|pwk}Bad Aussee, 10. VII. 20\nobreak{}«.  \newline{}Ordnung: mit Bleistift von unbekannter Hand nummeriert: »269« }\toendnotes[C]{\smallbreak}\pstart{}{\pb}Herrn\pend{}\pstart{}D\textsuperscript{r} Arthur Schnitzler\pend{}\pstart{}\textcolor{pink}{Wien XVIII}{}\ledrightnote{\textcolor{pink}{XVIII., Währing}}\pend{}\pstart{}\textcolor{pink}{Sternwartstrasse 71}{}\ledrightnote{\textcolor{pink}{Sternwartestraße}}\pend{}{\bigskip}\pstart
           \noindent{}\centering{}{\pb}\textcolor{gray}{\textbf{\textcolor{pink}{Salzkammergut}{}\ledrightnote{\textcolor{pink}{Salzkammergut}}. \textcolor{pink}{Bad Aussee}{}\ledrightnote{\textcolor{pink}{Bad Aussee}}.}}\pend
           \pstart
           \centering{}{\pb}9/VII. 20\pend
           \pstart
           Lieber Arthur! Ich kam nicht mehr dazu – in der Hetze der Abreise –
               noch zu Ihnen zu kommen. Alles Gute für den So{\geminationm}er –
               vielleicht doch \textcolor{pink}{Salzka{\geminationm}ergut}{}\ledrightnote{\textcolor{pink}{Salzkammergut}} statt \textcolor{pink}{Reichenau}{}\ledrightnote{\textcolor{pink}{Reichenau an der Rax}}? Es ist hier doch
               sehr schön! Herzlichst\pend
           \pstart \label{TLL02346_AS-1v}\edtext{Ihr\spacefill\mbox{Richard}}{\lemma{\textnormal{\emph{IhrRichard}}}\Cendnote{\textnormal{seitlich entlang des Textes}}}\label{TLL02346_AS-1h}\pend{}\endnumbering\briefempfaengerindex{Schnitzler, Arthur@\textsc{Schnitzler, Arthur}!zzzBeer-Hofmann, Richard@\emph{von Richard Beer-Hofmann}!1920-07-091@{9. 7. 1920}|)be}\mylabel{h}  \normalsize

\doendnotes{C}
\bigskip
\vfill

\clearpage

\footnotesize

\lohead{\textsc{register}}

% Definiere theindex-Environment komplett neu ohne reledmac
\makeatletter
\renewenvironment{theindex}{%
  \section*{\indexname}%
  \setlength{\parindent}{0pt}%
  \setlength{\parskip}{0pt plus 0.3pt}%
  \let\item\@idxitem
}{%
  \clearpage
}
\makeatother

\IfFileExists{\jobname-pw.ind}{\input{\jobname-pw.ind}}{}

\end{document}

      