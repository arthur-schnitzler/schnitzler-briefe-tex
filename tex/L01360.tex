%% latex-korrekturansicht-vorspann.tex
%% Vorspann für die Korrekturansicht.
%% Lädt die gemeinsame Datei latex-vorspann.tex mit gesetztem Schalter.

\newif\ifkorrekturansicht
\korrekturansichttrue

\input{../tex-inputs/latex-vorspann}


               \section[Arthur Schnitzler an Franz Blei, 8. 1. 1904]{ Arthur Schnitzler an Franz Blei, 8. 1. 1904}\nopagebreak\mylabel{v}\rehead{ }\normalsize\beginnumbering\briefempfaengerindex{Blei, Franz@\textsc{Blei, Franz}!zzzSchnitzler, Arthur@\emph{von Arthur Schnitzler}!1904-01-082@{8. 1. 1904}|(be} \toendnotes[C]{\smallbreak\pagebreak[2]} \Standort{DLA, A:Schnitzler, HS.NZ85.1.403.}
\physDesc{Brief, 1 Blatt, 1 Seite, Durchschlag (am linken Textrand Textverlust des ersten, teilweise der ersten zwei Buchstabens einer Zeile durch fehlerhafte Verwendung des Durchschlagpapiers entstanden)
\newline{}Schreibmaschine
\newline{}Handschrift: roter Buntstift, lateinische Kurrent (\noindent{}»Fr Blei« und vier
                                            Unterstreichungen)\newline{}Editorischer Hinweis: Die Zeichen des Textverlusts werden stillschweigend
                                            ergänzt, sofern sie inhaltlich verlässlich zu
                                            erschließen sind. }\toendnotes[C]{\smallbreak}\pstart
           \raggedleft{}{\pb}\textcolor{pink}{Wien}{}\ledrightnote{\textcolor{pink}{Wien}}, 8. Januar 1904.{\\}\textcolor{pink}{XVIII. Spöttelg. 7}{}\ledrightnote{\textcolor{pink}{Edmund-Weiß-Gasse}}.\pend
           \pstart{}Sehr geehrter Herr Blei!\pend\pstart
           Für Ihre freundlichen Nachrichten danke ich sehr. Könnte ich nicht wissen, warum
                    mein \textcolor{pink}{englischer}{}\ledrightnote{\textcolor{pink}{England}}{ }\textcolor{blue}{Verleger}{}\ledrightnote{→\textcolor{blue}{Alfred Bates}} »\begin{otherlanguage}{english}distinctly shady\end{otherlanguage}« sein soll? Jedenfalls habe ich bis
                        1. Juli 1906 in Hinsicht auf den »\textcolor{green}{Kakadu}{}\ledrightnote{\textcolor{green}{Der grüne Kakadu. Groteske in einem Akt}}« Vertrag, der mich bindet.\pend
           \pstart
           In Betreff eventuellen Verlags meiner Novellen bei \textcolor{brown}{Heinemann}{}\ledrightnote{\textcolor{brown}{William Heinemann Ltd}} erwarte ich gern präzisere Anträge.\pend
           \pstart
           Dass ich das Honorar von den \textcolor{brown}{Scharfrichtern}{}\ledrightnote{\textcolor{brown}{Die elf Scharfrichter}} noch
                    immer nicht bekommen habe, kann ich Ihnen bei dem besten Willen nicht
                    verhehlen.\pend
           \pstart
           Mir hat es recht leid getan, Sie in \textcolor{pink}{Wien}{}\ledrightnote{\textcolor{pink}{Wien}} nicht
                    gesehen \damage{zu} haben; bei den \label{K_L01360_1v}\edtext{\textcolor{brown}{Scharfrichtern}{}\ledrightnote{\textcolor{brown}{Die elf Scharfrichter}} im \textcolor{pink}{Savoy}{}\ledrightnote{\textcolor{pink}{Hotel Savoy}}}{\lemma{\textnormal{\emph{Scharfrichtern im Savoy}}}\Cendnote{\textnormal{am
                        10. 12. 1903.}}}\label{K_L01360_1h} hat es mir sehr behagt.\pend
           \pstart
           Mit verbindlichem Gruss{\\[\baselineskip]} Ihr aufrichtig ergebener\pend
           \leftskip=0em{}\endnumbering\briefempfaengerindex{Blei, Franz@\textsc{Blei, Franz}!zzzSchnitzler, Arthur@\emph{von Arthur Schnitzler}!1904-01-082@{8. 1. 1904}|)be}\mylabel{h}  \normalsize

\doendnotes{C}
\bigskip
\vfill

\clearpage

\footnotesize

\lohead{\textsc{register}}

% Definiere theindex-Environment komplett neu ohne reledmac
\makeatletter
\renewenvironment{theindex}{%
  \section*{\indexname}%
  \setlength{\parindent}{0pt}%
  \setlength{\parskip}{0pt plus 0.3pt}%
  \let\item\@idxitem
}{%
  \clearpage
}
\makeatother

\IfFileExists{\jobname-pw.ind}{\input{\jobname-pw.ind}}{}

\end{document}

      