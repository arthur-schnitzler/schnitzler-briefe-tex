%% latex-korrekturansicht-vorspann.tex
%% Vorspann für die Korrekturansicht.
%% Lädt die gemeinsame Datei latex-vorspann.tex mit gesetztem Schalter.

\newif\ifkorrekturansicht
\korrekturansichttrue

\input{../tex-inputs/latex-vorspann}


               \section[Arthur Schnitzler an Richard Beer-Hofmann, 17. 10. 1896]{ Arthur Schnitzler an Richard Beer-Hofmann,
               17. 10. 1896}\nopagebreak\mylabel{v}\rehead{ }\normalsize\beginnumbering\briefempfaengerindex{Beer-Hofmann, Richard@\textsc{Beer-Hofmann, Richard}!zzzSchnitzler, Arthur@\emph{von Arthur Schnitzler}!1896-10-171@{17. 10. 1896}|(be} \toendnotes[C]{\smallbreak\pagebreak[2]} \Standort{YCGL, MSS 31.}
\physDesc{Postkarte
\newline{}Handschrift: Bleistift, deutsche Kurrent\newline{}Versand: 1) Rohrpost 2) Stempel: »\nobreak{}\oindex{IX., Alsergrund@\textbf{IX., Alsergrund}, \emph{Bezirk (A.BZK)}|pwk}Wien 9/1, 19 X 96, 2 10N\nobreak{}«. 3) Stempel: »\nobreak{}\oindex{I., Innere Stadt@\textbf{I., Innere Stadt}, \emph{Bezirk (A.BZK)}|pwk}Wien 1/1, 17 X 96, 2 40N\nobreak{}«. }\buchAbdrucke{\weitereDrucke{Arthur Schnitzler, Richard Beer-Hofmann: \emph{Briefwechsel 1891–1931}. Hg. Konstanze Fliedl. Wien, Zürich: \emph{Europaverlag} 1992, S. 99.} }\toendnotes[C]{\smallbreak}\pstart{}{\pb}\textsc{Dr Rich Beer-Hofmann}\pend{}\pstart{}\textcolor{pink}{Wien}{}\ledrightnote{\textcolor{pink}{Wien}}\pend{}\pstart{}\textcolor{pink}{\textsc{I Wollzeile 15}}{}\ledrightnote{\textcolor{pink}{Wollzeile}}\pend{}\pstart{}4. Stock.\pend{}{\bigskip}\pstart
           \noindent{}{\pb}Lieber Richard, morgen So{\geminationn}tag Abend \uline{nicht} bei Ihnen, ſondern \textcolor{pink}{\textsc{Imperial}}{}\ledrightnote{\textcolor{pink}{Café Imperial}}. – Näheres \uline{heut} Abe\textcolor{gray}{nd}{ }\uline{nach} den \textcolor{green}{Müttern}{}\ledrightnote{\textcolor{green}{Die Mütter. Schauspiel in vier Acten}}. Wir (\textcolor{blue}{Hugo}{}\ledrightnote{\textcolor{blue}{Hugo von Hofmannsthal}}{ }\textsc{etc}) im \textcolor{pink}{Riedhof}{}\ledrightnote{\textcolor{pink}{Riedhof}}. Ich ko{\geminationm}e mit \textcolor{blue}{Brahm}{}\ledrightnote{\textcolor{blue}{Otto Brahm}}{ }\textcolor{blue}{Hirſchf}{}\ledrightnote{\textcolor{blue}{Georg Hirschfeld}}
                  nach{[}.{]}{ }ſehe Sie übrigens \textcolor{green}{Zwischenakt}{}\ledrightnote{→\textcolor{green}{Die Mütter. Schauspiel in vier Acten}}\pend
           \pstart Herzlich Ihr \spacefill\mbox{A.}\pend{}\endnumbering\briefempfaengerindex{Beer-Hofmann, Richard@\textsc{Beer-Hofmann, Richard}!zzzSchnitzler, Arthur@\emph{von Arthur Schnitzler}!1896-10-171@{17. 10. 1896}|)be}\mylabel{h}  \normalsize

\doendnotes{C}
\bigskip
\vfill

\clearpage

\footnotesize

\lohead{\textsc{register}}

% Definiere theindex-Environment komplett neu ohne reledmac
\makeatletter
\renewenvironment{theindex}{%
  \section*{\indexname}%
  \setlength{\parindent}{0pt}%
  \setlength{\parskip}{0pt plus 0.3pt}%
  \let\item\@idxitem
}{%
  \clearpage
}
\makeatother

\IfFileExists{\jobname-pw.ind}{\input{\jobname-pw.ind}}{}

\end{document}

      