%% latex-korrekturansicht-vorspann.tex
%% Vorspann für die Korrekturansicht.
%% Lädt die gemeinsame Datei latex-vorspann.tex mit gesetztem Schalter.

\newif\ifkorrekturansicht
\korrekturansichttrue

\input{../tex-inputs/latex-vorspann}


               \section[Arthur Schnitzler an Hugo von Hofmannsthal, 11. 8. 1891]{ Arthur Schnitzler an Hugo von Hofmannsthal, 11. 8. 1891}\nopagebreak\mylabel{v}\rehead{ }\normalsize\beginnumbering\briefempfaengerindex{Hofmannsthal, Hugo von@\textsc{Hofmannsthal, Hugo von}!zzzSchnitzler, Arthur@\emph{von Arthur Schnitzler}!1891-08-111@{11. 8. 1891}|(be} \toendnotes[C]{\smallbreak\pagebreak[2]} \Standort{FDH, Hs-30885,10.}
\physDesc{Brief, 1 Blatt, 4 Seiten
\newline{}Handschrift: schwarze Tinte, deutsche Kurrent}\buchAbdrucke{\weitereDrucke{1) Hugo von Hofmannsthal, Arthur Schnitzler: \emph{Briefwechsel}. Hg. Therese Nickl und Heinrich Schnitzler. Frankfurt am Main: \emph{S. Fischer} 1964, S. 11–12.} \weitereDrucke{2) Hermann Bahr, Arthur Schnitzler: \emph{Briefwechsel, Aufzeichnungen, Dokumente (1891–1931)}. Hg. Kurt Ifkovits und Martin Anton Müller. Göttingen: \emph{Wallstein} 2018, S. 7.} }\toendnotes[C]{\smallbreak}\pstart
           \raggedleft{}{\pb}\textcolor{pink}{Wien}{}\ledrightnote{\textcolor{pink}{Wien}}, 11. Aug. 91\pend
           \pstart
           Lieber Freund,\hspace*{2em}es iſt ſehr wahrſcheinlich, daß ich die \label{K_L00028_1v}\edtext{beiden Feiertage}{\lemma{\textnormal{\emph{beiden Feiertage}}}\Cendnote{\textnormal{Der 15. 8. 1891 – Mariä Himmelfahrt –, war ein
                  Samstag. Dienstag, der 18. 8. war Geburtstag des Kaisers \textcolor{blue}{Franz Joseph}.}}}\label{K_L00028_1h} in \textcolor{pink}{Iſchl}{}\ledrightnote{\textcolor{pink}{Bad Ischl}} bei meinen Leuten verbringe. Bei dieſer Gelegenheit möcht
               ich ſehr gerne mit Ihnen zuſa{\geminationm}en sein. Nicht wahr, Sie
               theilen mir gleich in 2 Zeilen mit, ob Sie am 15. u.
                  16. Auguſt in \textcolor{pink}{\textsc{Strobl}}{}\ledrightnote{\textcolor{pink}{Strobl}}{ }ſind, ob Sie eventuell {\pb}nach \textcolor{pink}{Iſchl}{}\ledrightnote{\textcolor{pink}{Bad Ischl}} herüber kommen wollen \textsc{etc}. Von meiner Ankunft verſtändige ich Sie jedenfalls. Ich
               will auch dem \textcolor{blue}{\textsc{Beer Hofmann}}{}\ledrightnote{\textcolor{blue}{Richard Beer-Hofmann}} nach \textcolor{pink}{\textsc{Aussee}}{}\ledrightnote{\textcolor{pink}{Bad Aussee}}{ }ſchreiben (im übrigen hab \label{T_L00028_1v}\edtext{auch \uline{ich}}{\lemma{\textnormal{\emph{auch ich}}}\Cendnote{\textnormal{durch Austauschzeichen die
                  Wortreihenfolge von »\uline{ich} auch« geändert.}}}\label{T_L00028_1h} noch keine
               Zeile von ihm erhalten) – vielleicht ſind wir alle drei zusa{\geminationm}en, {\pb}ſpielen Feiertagspöbel,
               und fühlen uns wohl. –\pend
           \pstart
           Ihr \label{K_L00028_2v}\edtext{\textcolor{green}{\textcolor{pink}{Salzburger}{}\ledrightnote{\textcolor{pink}{Salzburg}} Artikel}{}\ledrightnote{→\textcolor{green}{Die Mozart-Centenarfeier in Salzburg}}}{\lemma{\textnormal{\emph{Salzburger Artikel}}}\Cendnote{\textnormal{\textcolor{blue}{Loris}: \emph{\textcolor{green}{Die
                        Mozart-Centenarfeier in Salzburg}}. In: \emph{\textcolor{green}{Allgemeine Kunst-Chronik}}, Bd. 15, Nr. 16, 1. August-Heft,
                        1. 8. 1891, S. 423–433.}}}\label{K_L00028_2h} war wunderſchön; wohl
               Ihnen, der ſo was im »Halbſchlaf« aufs Papier träumen kann. Ich bin wach, vielleicht
               ſogar überwach; aber es iſt ein verlogener Herbſtmorgen mit einer Barbierbeckensonne!
               – Haben Sie \textcolor{blue}{\textsc{Salten}}{}\ledrightnote{\textcolor{blue}{Felix Salten}}{ }{\pb}\label{K_L00028_3v}\edtext{über \textcolor{green}{\textcolor{blue}{\textsc{Bahr}}{}\ledrightnote{\textcolor{blue}{Hermann Bahr}}}{}\ledrightnote{→\textcolor{green}{Die Überwindung des Naturalismus}}}{\lemma{\textnormal{\emph{über Bahr}}}\Cendnote{\textnormal{\textcolor{blue}{Salten}: \emph{\textcolor{green}{Die
                        Überwindung des Naturalismus}}. In: \emph{\textcolor{green}{Allgemeine Kunst-Chronik}}, Bd. 15, Nr. 16, 1. August-Heft,
                        1. 8. 1891, S. 446–447.}}}\label{K_L00028_3h} geleſen? Ich finde –
               vortrefflich! –\pend
           \pstart
           Leben Sie wohl, hoffentlich plaudern wir bald.{\\[\baselineskip]}Ihr\spacefill\mbox{Arth
                  Schnitz}\pend
           \leftskip=0em{}\endnumbering\briefempfaengerindex{Hofmannsthal, Hugo von@\textsc{Hofmannsthal, Hugo von}!zzzSchnitzler, Arthur@\emph{von Arthur Schnitzler}!1891-08-111@{11. 8. 1891}|)be}\mylabel{h}  \normalsize

\doendnotes{C}
\bigskip
\vfill

\clearpage

\footnotesize

\lohead{\textsc{register}}

% Definiere theindex-Environment komplett neu ohne reledmac
\makeatletter
\renewenvironment{theindex}{%
  \section*{\indexname}%
  \setlength{\parindent}{0pt}%
  \setlength{\parskip}{0pt plus 0.3pt}%
  \let\item\@idxitem
}{%
  \clearpage
}
\makeatother

\IfFileExists{\jobname-pw.ind}{\input{\jobname-pw.ind}}{}

\end{document}

      