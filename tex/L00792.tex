%% latex-korrekturansicht-vorspann.tex
%% Vorspann für die Korrekturansicht.
%% Lädt die gemeinsame Datei latex-vorspann.tex mit gesetztem Schalter.

\newif\ifkorrekturansicht
\korrekturansichttrue

\input{../tex-inputs/latex-vorspann}


               \section[Hugo von Hofmannsthal an Arthur Schnitzler, {[}19. 4. 1898{]}]{ Hugo von Hofmannsthal an Arthur Schnitzler, {[}19. 4. 1898{]}}\nopagebreak\mylabel{v}\rehead{ }\normalsize\beginnumbering\briefempfaengerindex{Schnitzler, Arthur@\textsc{Schnitzler, Arthur}!zzzHofmannsthal, Hugo von@\emph{von Hugo von Hofmannsthal}!1898-04-192@{{[}19. 4. 1898{]}}|(be} \toendnotes[C]{\smallbreak\pagebreak[2]} \Standort{CUL, Schnitzler, B 43b/1.}
\physDesc{Brief, 1 Blatt (Briefkopf mit Möwen und einem Segelschiff), 3 Seiten
\newline{}Handschrift: schwarze Tinte, deutsche Kurrent
\newline{}Schnitzler: mit Bleistift datiert: »19/4/98« \newline{}Ordnung: 1) mit Bleistift von unbekannter Hand nummeriert: »\strikeout{113}« 2) mit Bleistift von unbekannter Hand nummeriert: »111«}\buchAbdrucke{\weitereDrucke{Hugo von Hofmannsthal, Arthur Schnitzler: \emph{Briefwechsel}. Hg. Therese Nickl und Heinrich Schnitzler. Frankfurt am Main: \emph{S. Fischer} 1964, S. 100–101.} }\toendnotes[C]{\smallbreak}\pstart{}{\pb}lieber Arthur\pend\pstart
           möchten Sie am \label{K_L00792_1v}\edtext{Donnerstag}{\lemma{\textnormal{\emph{Donnerstag}}}\Cendnote{\textnormal{Die angesprochene Radpartie fand am
                            21. 4. 1898 – dem besagten Donnerstag – unter Teilnahme
                            \textcolor{blue}{Schnitzler}s statt.}}}\label{K_L00792_1h} eine
                        Rad-Tages-partie{ }\strikeout{nach} machen nämlich mit mir, \textcolor{blue}{Mutter}{}\ledrightnote{→\textcolor{blue}{Franziska Schlesinger}} und \textcolor{blue}{Tochter}{}\ledrightnote{→\textcolor{blue}{Gertrude von Hofmannsthal}}{ }\textcolor{blue}{Schleſinger}{}\ledrightnote{\textcolor{blue}{Franziska Schlesinger}{\newline}\textcolor{blue}{Gertrude von Hofmannsthal}} und den beiden \textcolor{blue}{Franckenſteins}{}\ledrightnote{\textcolor{blue}{Clemens von Franckenstein}{\newline}\textcolor{blue}{Georg von Franckenstein}}. Natürlich eine \uline{kleine} Partie {\pb}z. B. \textcolor{pink}{\textsc{Pressbaum}}{}\ledrightnote{\textcolor{pink}{Pressbaum}}–\textcolor{pink}{Baden}{}\ledrightnote{\textcolor{pink}{Baden bei Wien}}.\pend
           \pstart
           Den Weg müſſten Sie wiſſen, wir wiſſen alle nichts aber man hat ja Karten. Bitte
                    antworten Sie mir umgehend aber ſehr ungeniert natürlich, wenn Sie keine Luſt
                    haben braucht es ja keinen anderen Grund. – Ich danke vielmals {\pb}für Ihr Geſpräch mit
                        \textcolor{blue}{Schlenther}{}\ledrightnote{\textcolor{blue}{Paul Schlenther}}. Ich wär natürlich rieſig
                    froh, wenn etwas daraus würde, beſonders in \uline{der}
                    Beſetzung.\pend
           \pstart
           Geſtern abend war ich mit \textcolor{blue}{Richard}{}\ledrightnote{\textcolor{blue}{Richard Beer-Hofmann}} 1 Stunde im
                        \textcolor{pink}{\textsc{Europe}}{}\ledrightnote{\textcolor{pink}{Café de l’Europe}}.\pend
           \pstart
           Morgen nach 11\textsuperscript{h} werd ich ins \textcolor{pink}{Kaiſerhof}{}\ledrightnote{\textcolor{pink}{Café Kaiserhof (Inh. Johann Wortner)}}{ }ſchauen,
                        \uline{ohne} gegenſeitige Bindung. Adieu.\pend
           \pstart \spacefill\mbox{Hugo.}\pend{}\endnumbering\briefempfaengerindex{Schnitzler, Arthur@\textsc{Schnitzler, Arthur}!zzzHofmannsthal, Hugo von@\emph{von Hugo von Hofmannsthal}!1898-04-192@{{[}19. 4. 1898{]}}|)be}\mylabel{h}  \normalsize

\doendnotes{C}
\bigskip
\vfill

\clearpage

\footnotesize

\lohead{\textsc{register}}

% Definiere theindex-Environment komplett neu ohne reledmac
\makeatletter
\renewenvironment{theindex}{%
  \section*{\indexname}%
  \setlength{\parindent}{0pt}%
  \setlength{\parskip}{0pt plus 0.3pt}%
  \let\item\@idxitem
}{%
  \clearpage
}
\makeatother

\IfFileExists{\jobname-pw.ind}{\input{\jobname-pw.ind}}{}

\end{document}

      