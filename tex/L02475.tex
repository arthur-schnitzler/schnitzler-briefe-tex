%% latex-korrekturansicht-vorspann.tex
%% Vorspann für die Korrekturansicht.
%% Lädt die gemeinsame Datei latex-vorspann.tex mit gesetztem Schalter.

\newif\ifkorrekturansicht
\korrekturansichttrue

\input{../tex-inputs/latex-vorspann}


               \section[Stefan Großmann an Arthur Schnitzler, 25. 5. 1926]{ Stefan Großmann an Arthur Schnitzler, 25. 5. 1926}\nopagebreak\mylabel{v}\rehead{ }\normalsize\beginnumbering\briefempfaengerindex{Schnitzler, Arthur@\textsc{Schnitzler, Arthur}!zzzGrossmann, Stefan@\emph{von Stefan Großmann}!1926-05-251@{25. 5. 1926}|(be} \toendnotes[C]{\smallbreak\pagebreak[2]} \Standort{DLA, A:Schnitzler, HS.NZ85.1.3232.}
\physDesc{Brief, 1 Blatt, 2 Seiten
\newline{}Schreibmaschine
\newline{}Handschrift: roter Buntstift (\noindent{}Unterschrift, eine Streichung)
\newline{}Schnitzler: mit Bleistift handschriftliche, nicht verlässlich zu
                                 entziffernde Antwortskizze auf der 2. Seite }\toendnotes[C]{\smallbreak}\pstart
           \noindent{}\centering{}{\pb}\textcolor{gray}{\textbf{\textcolor{brown}{Das Tage-Buch}{}\ledrightnote{\textcolor{brown}{Das Tage-Buch}}}}\pend
           \pstart
           \noindent{}\centering{}\textcolor{gray}{\textbf{\emph{Herausgeber: Stefan Großmann und \textcolor{blue}{Leopold Schwarzschild}{}\ledrightnote{\textcolor{blue}{Leopold Schwarzschild}}}}}\pend
           \pstart
           \noindent{}\centering{}\textcolor{gray}{\textbf{Tagebuchverlag m. b. H., \textcolor{pink}{Berlin
                        SW 19}{}\ledrightnote{\textcolor{pink}{Berlin}}}}\pend
           \pstart
           \noindent{}\centering{}\textcolor{gray}{\textbf{\textcolor{pink}{BEUTHSTRASSE 19}{}\ledrightnote{\textcolor{pink}{Beuthstrasse}}}}\pend
           \pstart
           \noindent{}\centering{}\textcolor{gray}{\textbf{\emph{Telegramm-Adresse: Tagebuch \textcolor{pink}{Berlin}{}\ledrightnote{\textcolor{pink}{Berlin}} ⋅ Fernsprecher: Merkur 8790–8792}}}\pend
           \pstart
           \noindent{}\centering{}\textcolor{gray}{\textbf{\emph{\so{Sprechstunde der Redaktion: 12–1 Uhr}}}}\pend
           \pstart
           \noindent{}\centering{}\textcolor{gray}{\textbf{*}}\pend
           \pstart
           \noindent{}Tgb./Gr./Schl.\pend
           \pstart
           \raggedleft{}\textcolor{pink}{Berlin}{}\ledrightnote{\textcolor{pink}{Berlin}}, den 25. Mai 1926.\pend
           \pstart
           \raggedleft{}Herrn\pend
           \pstart
           \noindent{}\raggedleft{}Dr. Arthur \so{Schnitzler}\pend
           \pstart
           \noindent{}\raggedleft{}\textcolor{pink}{\so{Wien} XVIII}{}\ledrightnote{\textcolor{pink}{XVIII., Währing}}\pend
           \pstart
           \noindent{}\raggedleft{}\textcolor{pink}{Sternwartestr. 71.}{}\ledrightnote{\textcolor{pink}{Sternwartestraße}}\pend
           \pstart\center{}Verehrter Herr Doktor Schnitzler!\pend\pstart
           Mit der Beharrlichkeit eines unangenehmen Menschen und eines guten Redakteurs stelle
               ich mich wieder bei Ihnen ein.\pend
           \pstart
           Ich las in der »\textcolor{green}{Neuen Freien Presse}{}\ledrightnote{\textcolor{green}{Neue Freie Presse}}« die \label{K_L02475_1v}\edtext{\textcolor{green}{Bruchstücke}{}\ledrightnote{→\textcolor{green}{Bemerkungen. (Aus dem noch unveröffentlichten „Buch der Sprüche und Bedenken“.)}}}{\lemma{\textnormal{\emph{Bruchstücke}}}\Cendnote{\textnormal{\textcolor{blue}{Arthur Schnitzler}: \emph{\textcolor{green}{Bemerkungen. (Aus dem noch unveröffentlichten »Buch der Sprüche
                        und Bedenken«.)}} In: \emph{\textcolor{green}{Neue Freie
                     Presse}}, Nr. 22158, 23. 5. 1923, S. 33.}}}\label{K_L02475_1h} aus
               Ihrem unveröffentlichten \textcolor{green}{Buch}{}\ledrightnote{→\textcolor{green}{Buch der Sprüche und Bedenken}},
               die Sie in der Pfingstnummer publizieren liessen und dachte dabei, das sind doch
               eigentlich die Beiträge, um die ich Arthur Schnitzler seit langem bedränge, und die
               er mir prinzipiell \strikeout{eigentlich} zugesagt hat. Es ist in
               diesen \textcolor{green}{Bemerkungen}{}\ledrightnote{\textcolor{green}{Bemerkungen. (Aus dem noch unveröffentlichten „Buch der Sprüche und Bedenken“.)}} so viel Weisheit und so viel
               verborgener Humor enthalten, dass Sie mir sicher gestatten werden, dass ich vorerst
               diese \textcolor{green}{Bemerkungen}{}\ledrightnote{\textcolor{green}{Bemerkungen. (Aus dem noch unveröffentlichten „Buch der Sprüche und Bedenken“.)}} im \textcolor{green}{TAGE-BUCH}{}\ledrightnote{\textcolor{green}{Das Tage-Buch}}{ }\label{K_L02475_2v}\edtext{nachdrucke}{\lemma{\textnormal{\emph{nachdrucke}}}\Cendnote{\textnormal{\textcolor{blue}{Arthur Schnitzler}: \emph{\textcolor{green}{Bemerkungen}}. In: \emph{\textcolor{green}{Das
                        Tage-Buch}}, Jg. 7, H. 22, 29. 5. 1926,
                  S. 747–748.}}}\label{K_L02475_2h}. Die »\textcolor{green}{Neue Freie
                  Presse}{}\ledrightnote{\textcolor{green}{Neue Freie Presse}}« wird ja seit dem Umsturz in \textcolor{pink}{Deutschland}{}\ledrightnote{\textcolor{pink}{Deutschland}} wenig gelesen und selbst wenn es ein Leser ein zweites Mal im
                  \textcolor{green}{TAGE-BUCH}{}\ledrightnote{\textcolor{green}{Das Tage-Buch}} findet, so kann {\pb}er aus einem zweiten Lesen nur
               noch weiteren Gewinn ziehen.\pend
           \pstart
           Ich wäre sehr glücklich, wenn Sie mir aus diesen unveröffentlichten Reichtümern noch
               einiges anderes zum Erstdruck anvertrauen wollten und begrüsse Sie in dieser Hoffnung
               als\pend
           \pstart
           Ihr dankbar ergebener{\\[\baselineskip]}\spacefill\mbox{{[}hs.:{]} Stefan Großmann}\pend
           \leftskip=0em{}\endnumbering\briefempfaengerindex{Schnitzler, Arthur@\textsc{Schnitzler, Arthur}!zzzGrossmann, Stefan@\emph{von Stefan Großmann}!1926-05-251@{25. 5. 1926}|)be}\mylabel{h}  \normalsize

\doendnotes{C}
\bigskip
\vfill

\clearpage

\footnotesize

\lohead{\textsc{register}}

% Definiere theindex-Environment komplett neu ohne reledmac
\makeatletter
\renewenvironment{theindex}{%
  \section*{\indexname}%
  \setlength{\parindent}{0pt}%
  \setlength{\parskip}{0pt plus 0.3pt}%
  \let\item\@idxitem
}{%
  \clearpage
}
\makeatother

\IfFileExists{\jobname-pw.ind}{\input{\jobname-pw.ind}}{}

\end{document}

      