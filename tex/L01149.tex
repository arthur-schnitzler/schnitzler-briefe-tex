%% latex-korrekturansicht-vorspann.tex
%% Vorspann für die Korrekturansicht.
%% Lädt die gemeinsame Datei latex-vorspann.tex mit gesetztem Schalter.

\newif\ifkorrekturansicht
\korrekturansichttrue

\input{../tex-inputs/latex-vorspann}


               \section[Hugo von Hofmannsthal an Arthur Schnitzler, 18. 7. {[}1901{]}]{ Hugo von Hofmannsthal an Arthur Schnitzler, 18. 7. {[}1901{]}}\nopagebreak\mylabel{v}\rehead{ }\normalsize\beginnumbering\briefempfaengerindex{Schnitzler, Arthur@\textsc{Schnitzler, Arthur}!zzzHofmannsthal, Hugo von@\emph{von Hugo von Hofmannsthal}!1901-07-181@{18. 7. {[}1901{]}}|(be} \toendnotes[C]{\smallbreak\pagebreak[2]} \Standort{CUL, Schnitzler, B 43b/1.}
\physDesc{Brief, 1 Blatt, 2 Seiten
\newline{}Handschrift: schwarze Tinte, deutsche Kurrent
\newline{}Schnitzler: mit Bleistift die Jahreszahl »901« ergänzt \newline{}Ordnung: mit Bleistift von unbekannter Hand nummeriert:
                                    »177« }\buchAbdrucke{\weitereDrucke{1) Hugo von Hofmannsthal, Arthur Schnitzler: \emph{Briefwechsel}. Hg. Therese Nickl und Heinrich Schnitzler. Frankfurt am Main: \emph{S. Fischer} 1964, S. 149–150.} \weitereDrucke{2) Hermann Bahr, Arthur Schnitzler: \emph{Briefwechsel, Aufzeichnungen, Dokumente (1891–1931)}. Hg. Kurt Ifkovits und Martin Anton Müller. Göttingen: \emph{Wallstein} 2018, S. 213.} }\toendnotes[C]{\smallbreak}\pstart
           \raggedleft{}{\pb}18. Juli.\hspace*{1.5em}\textcolor{pink}{Rodaun}{}\ledrightnote{\textcolor{pink}{Rodaun}},\pend
           \pstart{}mein guter lieber Arthur\pend\pstart
           ſchon gleich beim Betreten dieses Hauſes am 1\textsuperscript{ten}\label{K_L01149_1v}\edtext{Juni}{\lemma{\textnormal{\emph{Juni}}}\Cendnote{\textnormal{Von Schnitzler mit Bleistift zu »Juli«
                     korrigiert.}}}\label{K_L01149_1h} habe ich mit herzlicher Freude Ihren lieben Brief gefunden, und es iſt mir
               faſt unbegreiflich, daſs 17 Tage vergehen konnten, wo ich wirklich jeden Tag daran
               dachte, Ihnen zu ſchreiben, und immer wieder die eine Viertelſtunde ſich wegrückte.
               Allerdings hab ich in dieſen Tagen mit ziemlicher Haſt und ziemlich viel Einfällen
               den letzten Act des \textcolor{green}{Ballets}{}\ledrightnote{→\textcolor{green}{Der Triumph der Zeit}}
               endlich ausgeführt, ſo daſs von nun an dieſes ziemlich umfangreiche Ding, deſſen
               Werth oder Unwerth ich abſolut nicht abſchätzen kann, unter meinen Arbeiten exiſtiren
               wird. Hoffentlich kann ichs Ihnen im Herbſt vorleſen und es miſsfällt Ihnen
               nicht.\pend
           \pstart
           Dieſes Aneinander-vorüber-ſchweben in \textcolor{pink}{Innsbruck}{}\ledrightnote{\textcolor{pink}{Innsbruck}} hat
               mir damals recht leid gethan. Hätte man nicht ein paar Stunden zuſammen ſein können?
               ich glaube daſs wäre für alle vier ein freundlicher Eindruck geweſen.
               Auch iſt doch von \textcolor{blue}{Gerty}{}\ledrightnote{\textcolor{blue}{Gertrude von Hofmannsthal}} eine Indiscretion eben ſo
               wenig zu fürchten wie von mir und überdies hätte man ihr den Familiennamen der \textcolor{blue}{andern}{}\ledrightnote{→\textcolor{blue}{Olga Schnitzler}} gar nicht zu ſagen
               gebraucht. Wir ſind an dieſem Abend noch ins \textcolor{pink}{Hofgartengaſthaus}{}\ledrightnote{\textcolor{pink}{Hofgartengasthaus}} nachtmahlen gegangen, dem einzigen Ort, wo man »im Freien
               nachtmahlt« und ich habe ſehr gehofft, {\pb}daſs wir uns dort begegnen würden,
               es iſt aber leider nicht der Fall geweſen.\hspace*{2.5em}Mit dem
               Haus und dem Leben hier bin ich ſehr zufrieden, ich will aber nicht viel darüber
               ſagen, ſondern freue mich darauf, es Ihnen zu zeigen. Jetzt wüſste ich ſchon gerne,
               wo ich mir vorſtellen ſoll, daſs Sie ſind.\hspace*{2.5em}Ich will
               nun möglichſt bald anfangen, das große figurenreiche und tragiſche \textcolor{green}{Stück}{}\ledrightnote{→\textcolor{green}{Pompilia oder das Leben}} zu ſchreiben, deſſen Stoff mir von \textcolor{blue}{Browning}{}\ledrightnote{\textcolor{blue}{Robert Browning}} überliefert iſt.\pend
           \pstart
           Von Menſchen ſehe ich \textcolor{blue}{Bahr}{}\ledrightnote{\textcolor{blue}{Hermann Bahr}}, der öfter \label{K_L01149_2v}\edtext{herüberkommt}{\lemma{\textnormal{\emph{herüberkommt}}}\Cendnote{\textnormal{Das neu bezogene Haus \textcolor{blue}{Hofmannsthal}s lag etwa acht Kilometer von dem \textcolor{blue}{Bahrs} entfernt.}}}\label{K_L01149_2h}, und erwarte nächſtens \textcolor{blue}{Andrian}{}\ledrightnote{\textcolor{blue}{Leopold von Andrian-Werburg}} für einige Tage.\pend
           \pstart
           Ich freue mich ſehr auf einen Brief von Ihnen.\pend
           \pstart
           Von Herzen Ihr{\\[\baselineskip]}\spacefill\mbox{Hugo.}\pend
           \leftskip=0em{}\endnumbering\briefempfaengerindex{Schnitzler, Arthur@\textsc{Schnitzler, Arthur}!zzzHofmannsthal, Hugo von@\emph{von Hugo von Hofmannsthal}!1901-07-181@{18. 7. {[}1901{]}}|)be}\mylabel{h}  \normalsize

\doendnotes{C}
\bigskip
\vfill

\clearpage

\footnotesize

\lohead{\textsc{register}}

% Definiere theindex-Environment komplett neu ohne reledmac
\makeatletter
\renewenvironment{theindex}{%
  \section*{\indexname}%
  \setlength{\parindent}{0pt}%
  \setlength{\parskip}{0pt plus 0.3pt}%
  \let\item\@idxitem
}{%
  \clearpage
}
\makeatother

\IfFileExists{\jobname-pw.ind}{\input{\jobname-pw.ind}}{}

\end{document}

      