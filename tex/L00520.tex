%% latex-korrekturansicht-vorspann.tex
%% Vorspann für die Korrekturansicht.
%% Lädt die gemeinsame Datei latex-vorspann.tex mit gesetztem Schalter.

\newif\ifkorrekturansicht
\korrekturansichttrue

\input{../tex-inputs/latex-vorspann}


               \section[Lou Andreas-Salomé an Arthur Schnitzler, 8. 12. 1895]{ Lou Andreas-Salomé an Arthur Schnitzler, 8. 12. 1895}\nopagebreak\mylabel{v}\rehead{ }\normalsize\beginnumbering\briefempfaengerindex{Schnitzler, Arthur@\textsc{Schnitzler, Arthur}!zzzAndreas-Salome, Lou@\emph{von Lou Andreas-Salomé}!1895-12-081@{8. 12. 1895}|(be} \toendnotes[C]{\smallbreak\pagebreak[2]} \Standort{CUL, Schnitzler, B 3.}
\physDesc{Postkarte
\newline{}Handschrift: schwarze Tinte, deutsche Kurrent\newline{}Versand: 1) Stempel: »\nobreak{}\oindex{I., Innere Stadt@\textbf{I., Innere Stadt}, \emph{Bezirk (A.BZK)}|pwk}Wien 1/1, 8. 12. 95, 8–9V\nobreak{}«.  2) Stempel: »\nobreak{}\oindex{IX., Alsergrund@\textbf{IX., Alsergrund}, \emph{Bezirk (A.BZK)}|pwk}Wien 9/3, 8. 12. 95, 11.V\nobreak{}«. 
\newline{}Schnitzler: mit Bleistift datiert: »8/12 95« \newline{}Ordnung: mit Bleistift von unbekannter Hand nummeriert:
                                        »12« }\pstart{}{\pb}Herrn \textsc{D\textsuperscript{r}}\pend{}\pstart{}\textsc{Arthur Schnitzler}\pend{}\pstart{}\textsc{\textcolor{pink}{Wien}{}\ledrightnote{\textcolor{pink}{Wien}}}\pend{}\pstart{}\textcolor{pink}{Frankgasse 1}{}\ledrightnote{\textcolor{pink}{Frankgasse}}.
                    \pend{}{\bigskip}\pstart
           \noindent{}{\pb}Lieber Herr \textsc{D\textsuperscript{r}}, ich kann heute Abend nicht mehr in’s \textcolor{pink}{\textsc{Griensteidl}}{}\ledrightnote{\textcolor{pink}{Café Griensteidl}}, weil ich zu ſpät nach Hauſe gekommen und außerdem nicht recht wohl
                    bin. Gern würde ich aber durch eine Karte erfahren, ob ich in der \damage{\textcolor{gray}{näch}}ſten Woche einen Tag freihalten ſoll, ſei es für Theater oder ſonſt
                    was. Vielleicht frage ich \strikeout{Sie} Montag in Ihrer
                    Sprechſtunde an, weiß aber nicht gewiß ob es ſich ſo macht.\pend
           \pstart
           Mit herzlichen Grüßen Ihre{\\[\baselineskip]}\spacefill\mbox{LouAS.}\pend
           \leftskip=0em{}\endnumbering\briefempfaengerindex{Schnitzler, Arthur@\textsc{Schnitzler, Arthur}!zzzAndreas-Salome, Lou@\emph{von Lou Andreas-Salomé}!1895-12-081@{8. 12. 1895}|)be}\mylabel{h}  \normalsize

\doendnotes{C}
\bigskip
\vfill

\clearpage

\footnotesize

\lohead{\textsc{register}}

% Definiere theindex-Environment komplett neu ohne reledmac
\makeatletter
\renewenvironment{theindex}{%
  \section*{\indexname}%
  \setlength{\parindent}{0pt}%
  \setlength{\parskip}{0pt plus 0.3pt}%
  \let\item\@idxitem
}{%
  \clearpage
}
\makeatother

\IfFileExists{\jobname-pw.ind}{\input{\jobname-pw.ind}}{}

\end{document}

      