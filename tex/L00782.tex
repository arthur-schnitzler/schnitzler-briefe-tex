%% latex-korrekturansicht-vorspann.tex
%% Vorspann für die Korrekturansicht.
%% Lädt die gemeinsame Datei latex-vorspann.tex mit gesetztem Schalter.

\newif\ifkorrekturansicht
\korrekturansichttrue

\input{../tex-inputs/latex-vorspann}


               \section[Hugo von Hofmannsthal an Arthur Schnitzler, {[}10. 3. 1898?{]}]{ Hugo von Hofmannsthal an Arthur Schnitzler, {[}10. 3. 1898?{]}}\nopagebreak\mylabel{v}\rehead{ }\normalsize\beginnumbering\briefempfaengerindex{Schnitzler, Arthur@\textsc{Schnitzler, Arthur}!zzzHofmannsthal, Hugo von@\emph{von Hugo von Hofmannsthal}!1898-03-101@{{[}10. 3. 1898?{]}}|(be} \toendnotes[C]{\smallbreak\pagebreak[2]} \Standort{CUL, Schnitzler, B 43b/1.}
\physDesc{Brief, 1 Blatt, 2 Seiten
\newline{}Handschrift: schwarze Tinte, deutsche Kurrent
\newline{}Schnitzler: mit Bleistift datiert: »März 98« \newline{}Ordnung: 1) mit Bleistift von unbekannter Hand nummeriert: »\strikeout{108}« 2) mit Bleistift von unbekannter Hand nummeriert:
                                    »109«}\buchAbdrucke{\weitereDrucke{Hugo von Hofmannsthal, Arthur Schnitzler: \emph{Briefwechsel}. Hg. Therese Nickl und Heinrich Schnitzler. Frankfurt am Main: \emph{S. Fischer} 1964, S. 100.} }\toendnotes[C]{\smallbreak}\pstart
           \noindent{}{\pb}\textcolor{gray}{\textbf{\label{T_L00782-1v}\edtext{hvH}{\lemma{\textnormal{\emph{hvH}}}\Cendnote{\textnormal{gedrucktes Monogramm mit Krone in blauer Farbe}}}\label{T_L00782-1h}}}\pend
           \pstart
           \raggedleft{}Donnerstag.\pend
           \pstart{}lieber Arthur\pend\pstart
           entſchuldigen Sie daſs ich Sie wegen einer Dummheit beläſtige.\pend
           \pstart
           Am \label{K_L00782_1v}\edtext{zweiten Jänner}{\lemma{\textnormal{\emph{zweiten Jänner}}}\Cendnote{\textnormal{Das Gastspiel hatte bereits von
                     25.–28. 11. 1897 stattgefunden. Bei der erwähnten
                  Aufführung an einem Sonntag dürfte es sich um die Schlussvorstellung am
                  28. 11. 1897 handeln.}}}\label{K_L00782_1h} oder einem dieſem Datum ſehr nahen Sonn oder
               Feiertag hat die \textcolor{blue}{\textsc{Réjane}}{}\ledrightnote{\textcolor{blue}{Gabrielle-Charlotte Réju}} im \textcolor{pink}{Carltheater}{}\ledrightnote{\textcolor{pink}{Carl-Theater}}{ }\uline{nachmittag} die \textcolor{green}{\textsc{Madame Sans Gêne}}{}\ledrightnote{\textcolor{green}{Madame Sans-Gêne}} geſpielt. Ich wär ſehr froh, wenn ich den {\pb}Theaterzettel von dieſer
               Vorſtellung haben könnt, den ſicher noch irgend ein Diener{[},{]}
               Beamter oder ſo jemand im \textcolor{pink}{Carltheater}{}\ledrightnote{\textcolor{pink}{Carl-Theater}} beſitzt.
               Vielleicht könnten Sie mir durch die \textcolor{blue}{\textsc{Glümer}}{}\ledrightnote{\textcolor{blue}{Marie Glümer}} oder ſo mir einen verſchaffen. Das wäre sehr lieb.\pend
           \pstart
           Ihr{\\[\baselineskip]}\spacefill\mbox{Hugo.}\pend
           \leftskip=0em{}\endnumbering\briefempfaengerindex{Schnitzler, Arthur@\textsc{Schnitzler, Arthur}!zzzHofmannsthal, Hugo von@\emph{von Hugo von Hofmannsthal}!1898-03-101@{{[}10. 3. 1898?{]}}|)be}\mylabel{h}  \normalsize

\doendnotes{C}
\bigskip
\vfill

\clearpage

\footnotesize

\lohead{\textsc{register}}

% Definiere theindex-Environment komplett neu ohne reledmac
\makeatletter
\renewenvironment{theindex}{%
  \section*{\indexname}%
  \setlength{\parindent}{0pt}%
  \setlength{\parskip}{0pt plus 0.3pt}%
  \let\item\@idxitem
}{%
  \clearpage
}
\makeatother

\IfFileExists{\jobname-pw.ind}{\input{\jobname-pw.ind}}{}

\end{document}

      