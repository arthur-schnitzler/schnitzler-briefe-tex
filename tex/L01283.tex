%% latex-korrekturansicht-vorspann.tex
%% Vorspann für die Korrekturansicht.
%% Lädt die gemeinsame Datei latex-vorspann.tex mit gesetztem Schalter.

\newif\ifkorrekturansicht
\korrekturansichttrue

\input{../tex-inputs/latex-vorspann}


               \section[Hermann Bahr an Arthur Schnitzler, {[}30. 3. 1903{]}]{ Hermann Bahr an Arthur Schnitzler, {[}30. 3. 1903{]}}\nopagebreak\mylabel{v}\rehead{ }\normalsize\beginnumbering\briefempfaengerindex{Schnitzler, Arthur@\textsc{Schnitzler, Arthur}!zzzBahr, Hermann@\emph{von Hermann Bahr}!1903-03-302@{{[}30. 3. 1903{]}}|(be} \toendnotes[C]{\smallbreak\pagebreak[2]} \Standort{CUL, Schnitzler, B 5b.}
\physDesc{Brief, 1 Blatt, 2 Seiten
\newline{}Handschrift: schwarze Tinte, deutsche Kurrent
\newline{}Schnitzler: mit Bleistift datiert: »Ende März 903.« \newline{}Ordnung: mit Bleistift von unbekannter Hand nummeriert: »97« }\buchAbdrucke{\weitereDrucke{Hermann Bahr, Arthur Schnitzler: \emph{Briefwechsel, Aufzeichnungen, Dokumente (1891–1931)}. Hg. Kurt Ifkovits und Martin Anton Müller. Göttingen: \emph{Wallstein} 2018, S. 258.} }\pstart
           \raggedleft{}{\pb}Montag\pend
           \pstart\center{}Lieber Arthur!\pend\pstart
           Ich hatte ſogleich bei \textcolor{blue}{Pötzl}{}\ledrightnote{\textcolor{blue}{Eduard Pötzl}} (ſchriftlich,
               damit er es nicht ableugnen kann) ein Feuilleton über den \textcolor{green}{Reigen}{}\ledrightnote{\textcolor{green}{Reigen. Zehn Dialoge}} angemeldet, um es ihm wenigstens zu erſchweren, daß er von
               anderer Seite etwas über das Buch bringt. Darauf erhalte ich eben folgende Antwort,
               die ich mir gelegentlich zurückerbitte. Ich gehe nun heute oder morgen mit der Sache
               zu \textcolor{blue}{Wilhelm Singer}{}\ledrightnote{\textcolor{blue}{Wilhelm Singer}}, der mir Recht geben, über
                  \textcolor{blue}{P.}{}\ledrightnote{\textcolor{blue}{Eduard Pötzl}} wahnſinnig ſchimpfen und zuletzt
               entſcheiden wird, daß Leute wie wir – nemlich {\pb}Er,
               Ich und Du – viel zu hoch ſtehen, um uns mit ſolchen Burſchen einzulaſſen, das heißt
               daß es alſo bei \textcolor{blue}{P}{}\ledrightnote{\textcolor{blue}{Eduard Pötzl}}’s Entſcheidung bleibt.\pend
           \pstart
           Jedenfalls aber bitte ich Dich nochmals mir baldigſt ein Exemplar zu ſchicken.\pend
           \pstart
           Herzlichſt{\\[\baselineskip]}Dein{\\[\baselineskip]}\spacefill\mbox{Hermann}\pend
           \leftskip=0em{}\endnumbering\briefempfaengerindex{Schnitzler, Arthur@\textsc{Schnitzler, Arthur}!zzzBahr, Hermann@\emph{von Hermann Bahr}!1903-03-302@{{[}30. 3. 1903{]}}|)be}\mylabel{h}  \normalsize

\doendnotes{C}
\bigskip
\vfill

\clearpage

\footnotesize

\lohead{\textsc{register}}

% Definiere theindex-Environment komplett neu ohne reledmac
\makeatletter
\renewenvironment{theindex}{%
  \section*{\indexname}%
  \setlength{\parindent}{0pt}%
  \setlength{\parskip}{0pt plus 0.3pt}%
  \let\item\@idxitem
}{%
  \clearpage
}
\makeatother

\IfFileExists{\jobname-pw.ind}{\input{\jobname-pw.ind}}{}

\end{document}

      