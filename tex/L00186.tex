%% latex-korrekturansicht-vorspann.tex
%% Vorspann für die Korrekturansicht.
%% Lädt die gemeinsame Datei latex-vorspann.tex mit gesetztem Schalter.

\newif\ifkorrekturansicht
\korrekturansichttrue

\input{../tex-inputs/latex-vorspann}


               \section[Fedor Mamroth an Arthur Schnitzler, 5. 3. 1893]{ Fedor Mamroth an Arthur Schnitzler, 5. 3. 1893}\nopagebreak\mylabel{v}\rehead{ }\normalsize\beginnumbering\briefempfaengerindex{Schnitzler, Arthur@\textsc{Schnitzler, Arthur}!zzzMamroth, Fedor@\emph{von Fedor Mamroth}!1893-03-052@{5. 3. 1893}|(be} \toendnotes[C]{\smallbreak\pagebreak[2]} \Standort{CUL, Schnitzler, B 68.}
\physDesc{Brief, 1 Blatt, 2 Seiten
\newline{}Handschrift: blaue Tinte, deutsche Kurrent
\newline{}Schnitzler: 1) mit Bleistift nummeriert: »4.« 2) mit rotem Buntstift eine Unterstreichung}\toendnotes[C]{\smallbreak}\pstart
           \noindent{}{\pb}\textcolor{brown}{\textcolor{gray}{\textbf{\textsc{Frankfurter Zeitung}}}{\\}\textsc{\textcolor{gray}{\textbf{und}}}{\\}\textcolor{gray}{\textbf{\textsc{Handelsblatt.}}}}{}\ledrightnote{\textcolor{brown}{Frankfurter Zeitung}}\pend
           \pstart
           \textcolor{gray}{\textbf{\textsc{Redaction.}}}\hfill \textcolor{gray}{\textbf{\textsc{\textcolor{pink}{Frankfurt a. M.}{}\ledrightnote{\textcolor{pink}{Frankfurt am Main}},}}}{ }5. März \textsc{\textcolor{gray}{\textbf{189}}}3\pend
           \pstart
           \textcolor{gray}{\textbf{\textsc{Telegramm-Adresse:}}}\pend
           \pstart
           \textcolor{gray}{\textbf{\textsc{Zeitung Frankfurt Main.}}}\pend
           \pstart{}Mein ſehr verehrter Herr Doctor!\pend\pstart
           Ich habe letzten Sonntag – heute vor 8 Tagen – Ihren \textcolor{green}{Roman}{}\ledrightnote{→\textcolor{green}{Sterben. Novelle}} in einem Zuge ausgeleſen, was mir bei einem
                    Manuſcript ſchon lange nicht paſſiert iſt, und darüber ſogar das Theater
                    verſäumt, was mir noch nie paſſiert iſt. Die ganze Woche über kam ich nicht
                    dazu, Ihnen zu ſchreiben, u. erſt heute vermag ich Ihnen mitzutheilen, daß ich
                    die Erzählung \uline{nicht} acceptiere.\pend
           \pstart
           Warum? Nicht mit Rückſicht auf die Prüderie des Publikums, denn die paar Stellen,
                    die als bedenklich in Betracht kämen, ließen ſich leicht beſeitigen. Nein, aus
                    einem Grunde, den Sie von Ihrem Standpunkt aus gar nicht verſtehen dürften: Der
                        \textcolor{green}{Roman}{}\ledrightnote{→\textcolor{green}{Sterben. Novelle}} ist mir viel zu ernſt u. düſter,
                    mir, dem man beſtändig den Vorwurf macht, daß unſer Roman-Feuilleton »viel zu
                    ernſt u. düſter« ſei. Berückſichtigen Sie gefälligſt, daß ich nichts weiter bin
                    als ein Knecht \label{T_L00186_1v}\edtext{und}{\lemma{\textnormal{\emph{und}}}\Cendnote{\textnormal{Er schreibt »und und«.}}}\label{T_L00186_1h} daß ich aus tiefſter
                    Knechts-Überzeugung ablehnen muß, unſer Publikum mit einer ſo wenig fröhlichen
                    und erbaulichen Erzählung, ſchon in aller Frühe beim Morgenkaffee zu
                    verſtimmen.\pend
           \pstart
           Alſo ich nehme Ihren \textcolor{green}{Roman}{}\ledrightnote{→\textcolor{green}{Sterben. Novelle}} nicht, und das iſt
                    wohl die Hauptſache, für Sie, aber nicht für mich; denn ich muß Ihnen noch etwas
                    ſagen, was an u. für ſich ſehr gleichgiltig iſt, Ihnen, aber nicht mir, nämlich
                    daß {\pb}\uline{ich} der Lektüre Ihrer \textcolor{green}{Erzählung}{}\ledrightnote{→\textcolor{green}{Sterben. Novelle}} eine große Freude verdanke, – nein, das iſt
                        \damage{wohl} nicht das richtige Wort: eine zunehmende Aufregung, eine innige
                    Antheilnahme, eine ſtarke Erſchütterung. Es iſt eine glänzende Arbeit, mit der
                    Sie einen ſchönen Erfolg haben werden, nicht in einer Zeitung, ſondern im Buche.
                    Ich würde mir an Ihrer Stelle erſt keine Mühe geben, ſie bei einer Redaction
                    einzureichen; wenn \uline{ich}{ }ſie nicht nehme, nimmt
                    ſie Niemand; ſoweit glaube ich den Geiſt der \textcolor{pink}{deutſchen}{}\ledrightnote{\textcolor{pink}{Deutschland}} u. \textcolor{pink}{öſterreichiſchen}{}\ledrightnote{\textcolor{pink}{Österreich}} Preſſe zu kennen. Alſo im Buche u. ich wäre
                    glücklich, Ihnen, falls dies nötig wäre, in irgend einer Weiſe dabei behilflich
                    ſein zu können. Und mit einem anderen Titel. »\textcolor{green}{Der
                        ſterbende Herr}{}\ledrightnote{\textcolor{green}{Sterben. Novelle}}« iſt gar nichts. Da müſſen Sie ſchon etwas anderes
                    finden. Aber um auf die Qualität der Arbeit zurückzukommen: ich müßte außer
                    Landes gehen, um einen Vergleich zu finden. Erinnern Sie ſich des Todes des
                    Fürſten Andrej in »\textcolor{green}{Krieg und Frieden}{}\ledrightnote{\textcolor{green}{Krieg und Frieden}}«? Das
                    hat ein \textcolor{blue}{Dichter}{}\ledrightnote{→\textcolor{blue}{Leo N. von Tolstoi}} geſchrieben, der kein Arzt
                    war. Ihren Roman hat ein Arzt geſchrieben, der ein Dichter iſt. Es iſt die erſte
                    zugleich künſtleriſche und wahrheitstreue Darſtellung des Grundverhältniſſes
                    zwiſchen Tod u. Leben einerſeits u. der phyſiſchen Auflöſung andrerſeits, die
                    ich kenne. Welche Fülle von Beobachtungen u. welche überzeugende Richtigkeit in
                    Auffaſſung und Entwicklung zweier einfacher Menſchenſchickſale! Ich
                    beglückwünſche Sie aufrichtig zu dieſer Arbeit, mein ſehr verehrter Herr Doctor,
                    jetzt weiß ich ganz genau, wer Sie ſind, und jetzt bin ich der Erſte, der für
                    Ihren Beruf mit Freuden Zeugniß ablegt.\pend
           \pstart
           Ihr\hspace*{1.5em}ergebener{\\[\baselineskip]}\spacefill\mbox{FMamroth}\pend
           \leftskip=0em{}\endnumbering\briefempfaengerindex{Schnitzler, Arthur@\textsc{Schnitzler, Arthur}!zzzMamroth, Fedor@\emph{von Fedor Mamroth}!1893-03-052@{5. 3. 1893}|)be}\mylabel{h}  \normalsize

\doendnotes{C}
\bigskip
\vfill

\clearpage

\footnotesize

\lohead{\textsc{register}}

% Definiere theindex-Environment komplett neu ohne reledmac
\makeatletter
\renewenvironment{theindex}{%
  \section*{\indexname}%
  \setlength{\parindent}{0pt}%
  \setlength{\parskip}{0pt plus 0.3pt}%
  \let\item\@idxitem
}{%
  \clearpage
}
\makeatother

\IfFileExists{\jobname-pw.ind}{\input{\jobname-pw.ind}}{}

\end{document}

      