%% latex-korrekturansicht-vorspann.tex
%% Vorspann für die Korrekturansicht.
%% Lädt die gemeinsame Datei latex-vorspann.tex mit gesetztem Schalter.

\newif\ifkorrekturansicht
\korrekturansichttrue

\input{../tex-inputs/latex-vorspann}


               \section[Hugo von Hofmannsthal an Arthur Schnitzler, 30. 4. {[}1917{]}]{ Hugo von Hofmannsthal an Arthur Schnitzler, 30. 4. {[}1917{]}}\nopagebreak\mylabel{v}\rehead{ }\normalsize\beginnumbering\briefempfaengerindex{Schnitzler, Arthur@\textsc{Schnitzler, Arthur}!zzzHofmannsthal, Hugo von@\emph{von Hugo von Hofmannsthal}!1917-04-301@{30. 4. {[}1917{]}}|(be} \toendnotes[C]{\smallbreak\pagebreak[2]} \Standort{CUL, Schnitzler, B 43.}
\physDesc{Briefkarte
\newline{}Handschrift: schwarze Tinte, deutsche Kurrent
\newline{}Schnitzler: mit Bleistift die Jahreszahl ergänzt: »17« und beschriftet: »Hugo« \newline{}Ordnung: 1) mit Bleistift von \textcolor{blue}{Frieda Pollak} (?) mit dem Buchstaben »A« (Abgeschrieben/Abschrift) gekennzeichnet 2) mit Bleistift von unbekannter Hand nummeriert: »\strikeout{347}«3) mit Bleistift von unbekannter Hand nummeriert: »358«}\buchAbdrucke{\weitereDrucke{Hugo von Hofmannsthal, Arthur Schnitzler: \emph{Briefwechsel}. Hg. Therese Nickl und Heinrich Schnitzler. Frankfurt am Main: \emph{S. Fischer} 1964, S. 281.} }\toendnotes[C]{\smallbreak}\pstart
           \raggedleft{}{\pb}\textcolor{pink}{R.}{}\ledrightnote{\textcolor{pink}{Rodaun}}{ }30 IV.\pend
           \pstart{}mein lieber Arthur \pend\pstart
           ich weiß nicht, ob Sie nicht vielleicht ohnedies die Abſicht haben, zu der \label{K_L02259_1v}\edtext{\introOben{}\textcolor{brown}{Concordia-}{}\ledrightnote{\textcolor{brown}{Concordia}}\introOben{}Veranſtaltung}{\lemma{\textnormal{\emph{Concordia-Veranſtaltung}}}\Cendnote{\textnormal{vgl. A. S.: \emph{Tagebuch}, 3. 5. 1917}}}\label{K_L02259_1h} für die \textcolor{pink}{Schweiz}{}\ledrightnote{\textcolor{pink}{Schweiz}}er zuzuſagen u. zu ko{\geminationm}en – jedenfalls
               finde ich es – abgeſehen von meiner perſönlichen Freude, Sie dann dort zu ſehen und
               in einem gewiſſen Sinn, nicht \uline{allein} zu ſein – ſo
               überaus nützlich und \uline{richtig} wenn Sie {\pb}kämen, denn es handelt ſich ja
               nicht ſo ſehr um den mehr minder trivialen Abend, den wir da verbringen werden,
               ſondern um die Rückwirkung nach der \textcolor{pink}{Schweiz}{}\ledrightnote{\textcolor{pink}{Schweiz}} hin,
               und es iſt doch nur natürlich, wenn da Ihre Gegenwart ſehr ins Gewicht fällt, mehr
               als jede andere, da Sie ja eigentlich von allen deutſch ſchreibenden Bühnendichtern
               der einzige \introOben{}im Ausland\introOben{} nicht nur bekannte, ſondern wirklich
               populäre ſind.\pend
           \pstart
           Herzlich Ihr{\\[\baselineskip]}\spacefill\mbox{Hugo.}\pend
           \leftskip=0em{}\endnumbering\briefempfaengerindex{Schnitzler, Arthur@\textsc{Schnitzler, Arthur}!zzzHofmannsthal, Hugo von@\emph{von Hugo von Hofmannsthal}!1917-04-301@{30. 4. {[}1917{]}}|)be}\mylabel{h}  \normalsize

\doendnotes{C}
\bigskip
\vfill

\clearpage

\footnotesize

\lohead{\textsc{register}}

% Definiere theindex-Environment komplett neu ohne reledmac
\makeatletter
\renewenvironment{theindex}{%
  \section*{\indexname}%
  \setlength{\parindent}{0pt}%
  \setlength{\parskip}{0pt plus 0.3pt}%
  \let\item\@idxitem
}{%
  \clearpage
}
\makeatother

\IfFileExists{\jobname-pw.ind}{\input{\jobname-pw.ind}}{}

\end{document}

      