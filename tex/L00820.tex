%% latex-korrekturansicht-vorspann.tex
%% Vorspann für die Korrekturansicht.
%% Lädt die gemeinsame Datei latex-vorspann.tex mit gesetztem Schalter.

\newif\ifkorrekturansicht
\korrekturansichttrue

\input{../tex-inputs/latex-vorspann}


               \section[Arthur Schnitzler an Hugo von Hofmannsthal, 15. 7. 1898]{ Arthur Schnitzler an Hugo von Hofmannsthal, 15. 7. 1898}\nopagebreak\mylabel{v}\rehead{ }\normalsize\beginnumbering\briefempfaengerindex{Hofmannsthal, Hugo von@\textsc{Hofmannsthal, Hugo von}!zzzSchnitzler, Arthur@\emph{von Arthur Schnitzler}!1898-07-152@{15. 7. 1898}|(be} \toendnotes[C]{\smallbreak\pagebreak[2]} \Standort{FDH, Hs-30885,70.}
\physDesc{Brief, 1 Blatt, 3 Seiten
\newline{}Handschrift: Bleistift, deutsche Kurrent}\buchAbdrucke{\weitereDrucke{Hugo von Hofmannsthal, Arthur Schnitzler: \emph{Briefwechsel}. Hg. Therese Nickl und Heinrich Schnitzler. Frankfurt am Main: \emph{S. Fischer} 1964, S. 105106.} }\toendnotes[C]{\smallbreak}\pstart
           \raggedleft{}{\pb}\textcolor{pink}{Graz}{}\ledrightnote{\textcolor{pink}{Graz}}, Freitag{\\}15/7 98\pend
           \pstart
           Mein lieber Hugo, meine Abſicht iſt, So{\geminationn}tag von hier fortzureiſen; dann zu
                    Bahn, Rad, Wagen weiter, vielleicht ko{\geminationm} ich in die
                        \textcolor{pink}{Fuſch}{}\ledrightnote{\textcolor{pink}{Bad Fusch}}, da ſeh ich wohl noch Ihre \textcolor{blue}{Eltern}{}\ledrightnote{→\textcolor{blue}{Hugo August von Hofmannsthal}{\newline}→\textcolor{blue}{Anna von Hofmannsthal}}, Do{\geminationn}erſtag 21.{ }\textcolor{pink}{\introOben{}Bad\introOben{} Gaſtein, \textsc{Villa
                            Wassing}}{}\ledrightnote{\textcolor{pink}{Villa Dr. Wassing}}, \uline{dort treffen mich Nachrichten bis
                            23.} (Bei meiner \textcolor{blue}{Mama}{}\ledrightnote{→\textcolor{blue}{Louise Schnitzler}}). \introOben{}(Alſo nicht offne Karte!)\introOben{} – Da{\geminationn}{ }ſchlängle ich mich allmählich nach \textcolor{pink}{Salzburg}{}\ledrightnote{\textcolor{pink}{Salzburg}} – und weiteres hören Sie noch. – Die Zeit
                    hier vergeht leidlich, wenn auch nicht ganz nach meiner Laune; zum
                    Familienleben, {\pb}ſelbſt in mäßigem Umfang bin ich
                    nicht geboren. Auch ſind jetzt die Zuſtände durch die merkwürdige Vermengung von
                    illegitimem und anerkanntem, Einſicht und Halbheit, ganz unruhig.\pend
           \pstart
           Zum Arbeiten bin ich gar nicht geko{\geminationm}en; mit einer
                    ſehr lebhaften Sehnſucht ruft es mich zu meinem neuen \textcolor{green}{Stück}{}\ledrightnote{→\textcolor{green}{Der Schleier der Beatrice. Schauspiel in fünf Akten}} – und doch werd ich vorher wahrſcheinlich was
                    anderes ſchreiben. Die alte Skizze vom »\textcolor{green}{Sohn}{}\ledrightnote{\textcolor{green}{Der Sohn. Aus den Papieren eines Arztes}}« (Muttermörder) geſtaltet ſich in mir zu irgendwas aus, was
                    beinah {\pb}ein \textcolor{green}{Roman}{}\ledrightnote{→\textcolor{green}{Therese. Chronik eines Frauenlebens}}{ }ſein könnte. – Daſs ich von \textcolor{pink}{Wien}{}\ledrightnote{\textcolor{pink}{Wien}} fort bin, iſt mir recht; daſs es von hier aus bald weiter
                    geht, nicht minder. Das Radeln macht mir Freude.\pend
           \pstart
           Warum ſchreiben Sie mir in Ihrem letzten \introOben{}(vom
                        12.)\introOben{} nicht, wie’s Ihnen geht? Das hoff ich, wenn auch nur
                    mit ein paar Zeilen, in \textcolor{pink}{Gaſtein}{}\ledrightnote{\textcolor{pink}{Bad Gastein}} zu erfahren.
                        \textcolor{blue}{Richard}{}\ledrightnote{\textcolor{blue}{Richard Beer-Hofmann}}{ }ſchrieb mir kurz, ohne beſti{\geminationm}te Zuſage, nicht wohlgelaunt.\pend
           \pstart
           Laſſen Sie uns auf ein ſchönes Wiederſehen hoffen. Von Herzen Ihr
                        \spacefill\mbox{Arthur}\pend
           \endnumbering\briefempfaengerindex{Hofmannsthal, Hugo von@\textsc{Hofmannsthal, Hugo von}!zzzSchnitzler, Arthur@\emph{von Arthur Schnitzler}!1898-07-152@{15. 7. 1898}|)be}\mylabel{h}  \normalsize

\doendnotes{C}
\bigskip
\vfill

\clearpage

\footnotesize

\lohead{\textsc{register}}

% Definiere theindex-Environment komplett neu ohne reledmac
\makeatletter
\renewenvironment{theindex}{%
  \section*{\indexname}%
  \setlength{\parindent}{0pt}%
  \setlength{\parskip}{0pt plus 0.3pt}%
  \let\item\@idxitem
}{%
  \clearpage
}
\makeatother

\IfFileExists{\jobname-pw.ind}{\input{\jobname-pw.ind}}{}

\end{document}

      