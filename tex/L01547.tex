%% latex-korrekturansicht-vorspann.tex
%% Vorspann für die Korrekturansicht.
%% Lädt die gemeinsame Datei latex-vorspann.tex mit gesetztem Schalter.

\newif\ifkorrekturansicht
\korrekturansichttrue

\input{../tex-inputs/latex-vorspann}


               \section[Arthur Schnitzler an Hermann Bahr, 17. 9. 1905]{ Arthur Schnitzler an Hermann Bahr, 17. 9. 1905}\nopagebreak\mylabel{v}\rehead{ }\normalsize\beginnumbering\briefempfaengerindex{Bahr, Hermann@\textsc{Bahr, Hermann}!zzzSchnitzler, Arthur@\emph{von Arthur Schnitzler}!1905-09-171@{17. 9. 1905}|(be} \toendnotes[C]{\smallbreak\pagebreak[2]} \Standort{TMW, HS AM 23376 Ba.}
\physDesc{Brief, 1 Blatt, 3 Seiten
\newline{}Handschrift: schwarze Tinte, deutsche Kurrent\newline{}Ordnung: Lochung }\buchAbdrucke{\weitereDrucke{1) Arthur Schnitzler: \emph{Briefe 1875–1912}. Hg. Therese Nickl und Heinrich Schnitzler. Frankfurt am Main: \emph{S. Fischer} 1981, S. 516–517.} \weitereDrucke{2) \emph{17. 9. 1905.} In: Arthur Schnitzler: \emph{The Letters of Arthur Schnitzler to Hermann Bahr}. Edited, annotated, and with an introduction, by Donald G.
                        Daviau. Chapel Hill: \emph{The University of North Carolina Press} 1978, S. 90–91 (University of North Carolina studies in the Germanic languages
                        and literatures, 89).} \weitereDrucke{3) Hermann Bahr, Arthur Schnitzler: \emph{Briefwechsel, Aufzeichnungen, Dokumente (1891–1931)}. Hg. Kurt Ifkovits und Martin Anton Müller. Göttingen: \emph{Wallstein} 2018, S. 351.} }\toendnotes[C]{\smallbreak}\pstart
           \raggedleft{}{\pb}17. 9. 905\pend
           \pstart
           lieber Hermann, für den Fall, dſs ich dich nicht zu Hauſe treffe,
               ſchreibe ich \damage{d}ir gleich.\pend
           \pstart
           Das gedruckte Stück »\label{K_L01547_1v}\edtext{\textcolor{green}{Zwiſchenſpiel}{}\ledrightnote{\textcolor{green}{Zwischenspiel. Komödie in drei Akten}}}{\lemma{\textnormal{\emph{Zwiſchenſpiel}}}\Cendnote{\textnormal{Entsprechend dürfte die erste Buchausgabe
                  auf 1906 vordatiert sein: \textcolor{blue}{Arthur Schnitzler}: \emph{\textcolor{green}{Das Zwischenspiel. Komödie in drei Akten}}. Berlin: \emph{\textcolor{brown}{S. Fischer}}{ }1906.}}}\label{K_L01547_1h}« und »\textcolor{green}{Der Ruf des Lebens}{}\ledrightnote{\textcolor{green}{Zwischenspiel. Komödie in drei Akten}}« liegen
               hier bei.\pend
           \pstart
           Über das erſtere iſt weiter nichts zu ſagen; lies es bitte und betrachte es im
               übrigen vorläufig ſorgfältg als \textsc{\uline{Mscrpt}}.\pend
           \pstart
           Am »\textcolor{green}{Ruf des Lebens}{}\ledrightnote{\textcolor{green}{Der Ruf des Lebens. Schauspiel in drei Akten}}« ist noch einiges weniges zu
               machen. Ich bring es {\pb}dir ſchon
               heute, weil ich die Frage an dich richten möchte, ob du die \uline{Widmung} des Buches annehmen willſt? Es iſt vielleicht in dem Stück eine
               Ahnung von dem \label{K_L01547_2v}\edtext{Wunsch
               erfüllſt, den du anläßlich des \textcolor{green}{Puppenſpielers}{}\ledrightnote{\textcolor{green}{Der Puppenspieler}} oeffentlich ausſprachſt}{\lemma{\textnormal{\emph{Wunsch … ausſprachſt}}}\Cendnote{\textnormal{Vgl. Arthur Schnitzler an Hermann Bahr, 14. 12. 1904 und
                        \emph{Briefwechsel} Bahr/Schnitzler 332}}}\label{K_L01547_2h}. –\pend
           \pstart
           Schreib mir bitte ein Wort, wa{\geminationn} wir zuſa{\geminationm}en ſein könnten. Möchteſ\damage{t} du nicht einmal bei uns nachtmahlen? Auch meine \textcolor{blue}{Frau}{}\ledrightnote{→\textcolor{blue}{Olga Schnitzler}} würde ſich ſoſehr freuen. Oder wenn dir
               die \textcolor{pink}{Spöttelgaſſe}{}\ledrightnote{\textcolor{pink}{Edmund-Weiß-Gasse}} unbe{\pb}quem, \textcolor{pink}{Hietzing}{}\ledrightnote{\textcolor{pink}{XIII., Hietzing}}? Man ſieht einander doch gar zu wenig! Ich grüße dich
               herzlich.\pend
           \pstart
           Dein{\\[\baselineskip]}\spacefill\mbox{A.}\pend
           \leftskip=0em{}\endnumbering\briefempfaengerindex{Bahr, Hermann@\textsc{Bahr, Hermann}!zzzSchnitzler, Arthur@\emph{von Arthur Schnitzler}!1905-09-171@{17. 9. 1905}|)be}\mylabel{h}  \normalsize

\doendnotes{C}
\bigskip
\vfill

\clearpage

\footnotesize

\lohead{\textsc{register}}

% Definiere theindex-Environment komplett neu ohne reledmac
\makeatletter
\renewenvironment{theindex}{%
  \section*{\indexname}%
  \setlength{\parindent}{0pt}%
  \setlength{\parskip}{0pt plus 0.3pt}%
  \let\item\@idxitem
}{%
  \clearpage
}
\makeatother

\IfFileExists{\jobname-pw.ind}{\input{\jobname-pw.ind}}{}

\end{document}

      