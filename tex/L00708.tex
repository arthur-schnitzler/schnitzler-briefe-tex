%% latex-korrekturansicht-vorspann.tex
%% Vorspann für die Korrekturansicht.
%% Lädt die gemeinsame Datei latex-vorspann.tex mit gesetztem Schalter.

\newif\ifkorrekturansicht
\korrekturansichttrue

\input{../tex-inputs/latex-vorspann}


               \section[Hugo von Hofmannsthal an Arthur Schnitzler, {[}20. 7. 1897{]}]{ Hugo von Hofmannsthal an Arthur Schnitzler, {[}20. 7. 1897{]}}\nopagebreak\mylabel{v}\rehead{ }\normalsize\beginnumbering\briefempfaengerindex{Schnitzler, Arthur@\textsc{Schnitzler, Arthur}!zzzHofmannsthal, Hugo von@\emph{von Hugo von Hofmannsthal}!1897-07-202@{{[}20. 7. 1897{]}}|(be} \toendnotes[C]{\smallbreak\pagebreak[2]} \Standort{CUL, Schnitzler, B 43.}
\physDesc{Brief, 1 Blatt, 3 Seiten
\newline{}Handschrift: Bleistift, deutsche Kurrent
\newline{}Schnitzler: mit Bleistift datiert: »et\textcolor{gray}{w}{ }20 Juli 97« \newline{}Ordnung: 1) mit Bleistift von unbekannter Hand nummeriert: »\strikeout{99}« 2) mit Bleistift von unbekannter Hand nummeriert: »101«}\buchAbdrucke{\weitereDrucke{Hugo von Hofmannsthal, Arthur Schnitzler: \emph{Briefwechsel}. Hg. Therese Nickl und Heinrich Schnitzler. Frankfurt am Main: \emph{S. Fischer} 1964, S. 93.} }\pstart
           \raggedleft{}{\pb}Dienstag\pend
           \pstart{}lieber Arthur\pend\pstart
           bitte ſeien Sie noch vor Ihrer Abreiſe ſo gut mir hierher den Namen und die
                    Adreſſe des \textcolor{pink}{Iſchl}{}\ledrightnote{\textcolor{pink}{Bad Ischl}}er Arztes zu ſchreiben, den
                    Sie für den beſten halten (\uline{neben}{ }\textcolor{blue}{Widerhofer}{}\ledrightnote{\textcolor{blue}{Hermann Widerhofer}}.)
                        \textcolor{blue}{Poldy}{}\ledrightnote{\textcolor{blue}{Leopold von Andrian-Werburg}}’s Nervoſität hat ſich nämlich in
                    eine unausgeſetzte martervolle Angſt vor Schwindſucht {\pb}verwandelt, zum Theil
                    hervorgerufen durch eine unvorſichtige aber gar nicht wirklich beängſtigende
                    Äußerung \textcolor{blue}{Schrötter}{}\ledrightnote{\textcolor{blue}{Leopold Schrötter von Kristelli}}s. Er muſs alſo von \textcolor{pink}{Auſſee}{}\ledrightnote{\textcolor{pink}{Altaussee}} aus die Möglichkeit haben, ſooft er
                    will einen Arzt zu ſehen, der ihm die Unſchädlichkeit {\pb}des betreffenden Symptomes,
                    das er ſich von Tag zu Tag wechſelnd einredet, nachweist.\pend
           \pstart
           Im voraus dankt Ihnen{\\[\baselineskip]} Ihr\spacefill\mbox{Hugo.}\pend
           \leftskip=0em{}\endnumbering\briefempfaengerindex{Schnitzler, Arthur@\textsc{Schnitzler, Arthur}!zzzHofmannsthal, Hugo von@\emph{von Hugo von Hofmannsthal}!1897-07-202@{{[}20. 7. 1897{]}}|)be}\mylabel{h}  \normalsize

\doendnotes{C}
\bigskip
\vfill

\clearpage

\footnotesize

\lohead{\textsc{register}}

% Definiere theindex-Environment komplett neu ohne reledmac
\makeatletter
\renewenvironment{theindex}{%
  \section*{\indexname}%
  \setlength{\parindent}{0pt}%
  \setlength{\parskip}{0pt plus 0.3pt}%
  \let\item\@idxitem
}{%
  \clearpage
}
\makeatother

\IfFileExists{\jobname-pw.ind}{\input{\jobname-pw.ind}}{}

\end{document}

      