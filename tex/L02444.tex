%% latex-korrekturansicht-vorspann.tex
%% Vorspann für die Korrekturansicht.
%% Lädt die gemeinsame Datei latex-vorspann.tex mit gesetztem Schalter.

\newif\ifkorrekturansicht
\korrekturansichttrue

\input{../tex-inputs/latex-vorspann}


               \section[Arthur Schnitzler an Georg Brandes, 7. 7. 1925]{ Arthur Schnitzler an Georg Brandes, 7. 7. 1925}\nopagebreak\mylabel{v}\rehead{ }\normalsize\beginnumbering\briefempfaengerindex{Brandes, Georg@\textsc{Brandes, Georg}!zzzSchnitzler, Arthur@\emph{von Arthur Schnitzler}!1925-07-071@{7. 7. 1925}|(be} \toendnotes[C]{\smallbreak\pagebreak[2]} \Standort{Kopenhagen, Det Kongelige Bibliotek, Georg Brandes Arkiv, box 125.}
\physDesc{Brief, 6 Blätter, 6 Seiten (Von Schnitzler paginiert: »II« bis »VI«)
\newline{}Handschrift: schwarze Tinte, lateinische Kurrent\newline{}Ordnung: mit Bleistift von unbekannter Hand nummeriert:
                                    »52.« und die weiteren Blätter datiert: »7/7 25« }\buchAbdrucke{\weitereDrucke{1) Georg Brandes, Arthur Schnitzler: \emph{Ein Briefwechsel}. Hg. Kurt Bergel. Bern: \emph{Francke} 1956, S. 147–149.} \weitereDrucke{2) Arthur Schnitzler: \emph{Briefe 1913–1931}. Hg. Peter Michael Braunwarth, Richard Miklin, Susanne Pertlik und Heinrich Schnitzler. Frankfurt am Main: \emph{S. Fischer} 1984, S. 411–414.} }\toendnotes[C]{\smallbreak}\pstart
           \raggedleft{}{\pb}\textcolor{pink}{Wien}{}\ledrightnote{\textcolor{pink}{Wien}}, 7. Juli 1925\pend
           \pstart
           mein lieber und verehrter Freund, Sie haben mich während Ihres
               diesmaligen Aufenthalts in \textcolor{pink}{Wien}{}\ledrightnote{\textcolor{pink}{Wien}} »nicht heiter«
               gefunden, – und so muß ich fast befürchten, daß Sie nicht ganz bemerkt haben, wie
               glücklich mich Ihre Anwesenheit gemacht hat und wie froh ich war, daß Sie mir Ihre
               Sympathie – eines der Geschenke, für die ich dem Schicksal besonders dankbar bin –
               all die Jahre hindurch, die wir einander schon kennen, ungehindert erhalten haben.
               Darf ich Ihnen heute in diesen Zeilen zum Ausdruck bringen, was von Angesicht zu
               Angesicht auszusprechen, was in meinem Betragen zu verdeutlichen ich, mehr meinem
               ganzen Wesen nach, als aus vorübergehenden Sti{\geminationm}ungen
               heraus, nicht so recht im Stande war und bin? Es ist richtig, (und es bewegt mich
               sehr, daſs Sie es empfunden haben, we{\geminationn} es mir auch ein
               bischen leid thut), daſs ich {\pb}zuweilen ein wenig
               melancholisch bin, oder doch bedrückt. Hauptanlaß wohl mein Ohrenleiden, an dem nicht
               nur die langsam aber sicher zunehmende Schwerhörigkeit, sondern, mehr noch, die
               ununterbrochenen subjectiven Geräusche, ein Klingen, ein Sausen, und ein \strikeout{\textcolor{gray}{nicht}} stetes Vogelzwitschern (das sich bis zu einem mäßigen Papageiengekreisch
               verstärken kann) recht quälend sind. Und, sonderbar genug, es gibt doch Stunden, ja
               Tage, an denen mir diese Geräusche, – so continuirlich sie i{\geminationm}er (seit bald dreißig Jahren!) kaum zu Bewußtsein ko{\geminationm}en. Im ganzen verläuft ja die Sache etwas langsamer,
               als ich zu Beginn der Erkrankung gefürchtet habe – man gewöhnts auch allmälig \label{T_L02444_1v}\edtext{(}{\lemma{\textnormal{\emph{(}}}\Cendnote{\textnormal{öffnende Klammer am Zeilenende gestrichen und in der neuen Zeile erneut
                  ausgeführt}}}\label{T_L02444_1h}zu mindesten manchmal) aber es ist doch schli{\geminationm}, daſs mir insbesondere der Theaterbesuch schon
               ziemlich vergällt ist und auch bei musikalischen Genüssen viel, sehr viel entgeht.
               Und schli{\geminationm}, {\pb}daß es
               eine eigentliche »Stille« für mich längst nicht mehr gibt. Glücklicherweise werd ich
               im Schlafen nicht gestört, – wenn auch diese Geräusche auf mancherlei, oft ganz
               phantastische Art sich in meine Träume drängen.\pend
           \pstart
           Auch meine persönliche Existenz ist ja nicht ganz einfach, wie Sie wissen; aber es
               würde zu weit führen, da in Einzelheiten einzugehen; – an Conflicten seelischer Art
               mangelt es ja in diesen Grenzjahren (es ist vielleicht kühn, mit 63 noch von
               Grenzjahren zu reden, aber gerade Sie werden mich verstehen) nie.\pend
           \pstart
           Dabei fühl ich doch, daſs ich im Grunde nicht klagen dürfte (ich thu’s auch
               selten), – besonders darum weil meine beiden \textcolor{blue}{Kinder}{}\ledrightnote{→\textcolor{blue}{Heinrich Schnitzler}{\newline}→\textcolor{blue}{Lili Schnitzler}} sehr wohl gerathen sind (auch steh ich jetzt mit
               meiner früheren \textcolor{blue}{Gattin}{}\ledrightnote{→\textcolor{blue}{Olga Schnitzler}}, die in
                  \textcolor{pink}{Baden-Baden}{}\ledrightnote{\textcolor{pink}{Baden-Baden}} lebt, in sehr freundschaftlichen,
               natürlich nicht immer unge{\pb}trübten Beziehungen),
               und ferner weil ich mich in meiner Schaffenslust eher noch wachsen als abnehmen
               fühle. Auch an äußeren Erfolgen fehlt es nicht; und nach einer Periode, die sich ein
               wenig bedenklich anließ, glaub ich auch materiell – ach nicht durch das Vorhandensein
               eines Vermögens – wer besitzt de{\geminationn} jetzt etwas!, – aber
               durch das Ansteigen meiner Einnahmen, – mit Ruhe in die Zukunft blicken zu dürfen.
               Und blasirt bin ich ja nicht – mir macht eigentlich alles mehr Freude als es mir in
               meiner Jugend gemacht hat, – jede Blume, jeder Spaziergang, jedes schöne Buch und
               Herzlichkeit mancher Art, die mir entgegengebracht wird. »So wollen wirs de{\geminationn} noch eine Weile weiter treiben« wie ein sehr Großer
               gesagt haben kö{\geminationn}te und wahrscheinlich irgendwo gesagt
               hat – und Sie sollen wissen, liebster Freund, daſs ich, we{\geminationn} auch gelegentlich ein wenig verdüstert, {\pb}mich gar nicht übel befinde; – und hoffentlich
               mach ich auch Ihnen einen vergnügtem Eindruck, we{\geminationn} wir
               uns wiedersehen.\pend
           \pstart
           Wie gut begreife ich, daſs Sie nicht nach »\textcolor{pink}{Leningrad}{}\ledrightnote{\textcolor{pink}{Sankt Petersburg}}« gehen wollen – auch ich, (selbst we{\geminationn}
               ich dort nicht reden müßte,) hätte nicht die geringste Lust dazu. Kennen Sie das \introOben{}(kleine)\introOben{}{ }\textcolor{green}{Buch}{}\ledrightnote{→\textcolor{green}{Das ABC des Kommunismus}{\newline}→\textcolor{green}{Das ABC des Kommunismus}} von \textcolor{blue}{Bucharin}{}\ledrightnote{\textcolor{blue}{Nikolaj Ivanovič Bucharin}} über den Bolschewismus? Wenn die deutsche Übersetzung
               nicht \label{T_L02444_2v}\edtext{etwa}{\lemma{\textnormal{\emph{etwa}}}\Cendnote{\textnormal{unsicher zu lesen; wohl zur Verdeutlichung gestrichen und über
                  der Zeile wiederholt}}}\label{T_L02444_2h} zu dem Zwecke gefälscht ist, um die Idee – (die
               Idee!!) des Bolschewismus zu compromittieren, da{\geminationn} hat es
                  \textcolor{blue}{Bucharin}{}\ledrightnote{\textcolor{blue}{Nikolaj Ivanovič Bucharin}} selbst in unübertrefflicher Weise
               gethan. –\pend
           \pstart
           Ihren Brief hab ich in \textcolor{pink}{Bozen}{}\ledrightnote{\textcolor{pink}{Bozen}} erhalten, (\textcolor{pink}{Bolzano}{}\ledrightnote{\textcolor{pink}{Bozen}}) von wo ich erst vor ein paar Tagen nach
                  \textcolor{pink}{Wien}{}\ledrightnote{\textcolor{pink}{Wien}} zurückgekehrt bin. Ich bleibe nur den
                  Juli über hier, und fahre im August wahrscheinlich
               wieder in die Dolomiten. Für den Herbst steht
               mir allerlei bevor: in {\pb}\label{K_L02444_1v}\edtext{\textcolor{pink}{Berlin}{}\ledrightnote{\textcolor{pink}{Berlin}}}{\lemma{\textnormal{\emph{Berlin}}}\Cendnote{\textnormal{Die geplante Inszenierung von \textcolor{blue}{Victor Barnowsky} wird nicht realisiert
                     (Vgl. \emph{Briefe} II, 468).}}}\label{K_L02444_1h} die
               Aufführung der \textcolor{green}{Komödie der Verführung}{}\ledrightnote{\textcolor{green}{Komödie der Verführung. In drei Akten}}, – in \label{K_L02444_2v}\edtext{\textcolor{pink}{Wien}{}\ledrightnote{\textcolor{pink}{Wien}} Reprisen}{\lemma{\textnormal{\emph{Wien Reprisen}}}\Cendnote{\textnormal{\emph{\textcolor{green}{Das weite Land}} wurde ab 4. 9. 1925
                  am \textcolor{pink}{Deutschen Volkstheater} gegeben, wo auch \emph{\textcolor{green}{Der einsame Weg}} am 14. 11. 1925 im
                  Zuge eines Gastspiels von \textcolor{blue}{Albert Bassermann}
                  aufgeführt wurde.}}}\label{K_L02444_2h} von »\textcolor{green}{Das weite Land}{}\ledrightnote{\textcolor{green}{Das weite Land. Tragikomödie in fünf Akten}}«
               und »\textcolor{green}{der einsame Weg}{}\ledrightnote{\textcolor{green}{Der einsame Weg. Schauspiel in fünf Akten}}«, vielleicht auch ein neues
                  \textcolor{green}{Stück}{}\ledrightnote{→\textcolor{green}{Der Gang zum Weiher. Dramatische Dichtung}} (in Versen). Ein paar
                  \textcolor{green}{Novellen}{}\ledrightnote{→\textcolor{green}{Die Frau des Richters. Novelle}{\newline}→\textcolor{green}{Traumnovelle}} sind auch
               fertig. In \label{K_L02444_3v}\edtext{\textcolor{pink}{Paris}{}\ledrightnote{\textcolor{pink}{Paris}}}{\lemma{\textnormal{\emph{Paris}}}\Cendnote{\textnormal{Nicht realisiert, das Stück wurde erst
                     1931 gegeben.}}}\label{K_L02444_3h} wird vielleicht »\textcolor{green}{das weite Land}{}\ledrightnote{\textcolor{green}{Das weite Land. Tragikomödie in fünf Akten}}« gespielt werden; und nach \textcolor{pink}{Amerika}{}\ledrightnote{\textcolor{pink}{Amerika}} bin ich zur Premiere des »\textcolor{green}{einsamen
                  Wegs}{}\ledrightnote{\textcolor{green}{Der einsame Weg. Schauspiel in fünf Akten}}« im \label{K_L02444_4v}\edtext{\textcolor{brown}{Guild Theater}{}\ledrightnote{\textcolor{brown}{Guild Theatre}}}{\lemma{\textnormal{\emph{Guild Theater}}}\Cendnote{\textnormal{Das \textcolor{green}{Stück} wurde erst 1931 auf den Spielplan
                  genommen.}}}\label{K_L02444_4h} u. des »\textcolor{green}{Ruf des Lebens}{}\ledrightnote{\textcolor{green}{Der Ruf des Lebens. Schauspiel in drei Akten}}« am
                  \label{K_L02444_5v}\edtext{\textcolor{brown}{Astor Theater}{}\ledrightnote{\textcolor{brown}{Astor Theatre}}}{\lemma{\textnormal{\emph{Astor Theater}}}\Cendnote{\textnormal{Produziert vom \emph{\textcolor{brown}{Astor Theater}} wurde \emph{\textcolor{green}{The Call of
                     Life}} im \textcolor{pink}{Comedy Theatre} am
                     9. 10. 1925 zum ersten Mal und in Folge 19 Mal gegeben. Die
                  Bearbeitung stammte von \textcolor{blue}{Dorothy
                  Donnelli}.}}}\label{K_L02444_5h} eingeladen (Ich werde aber kaum hinreisen.) –\pend
           \pstart
           – Ich lese immer noch, aufs stärkste angeregt, Ihren wunderbaren \textcolor{green}{Julius Caesar}{}\ledrightnote{\textcolor{green}{Gaius Julius Cæsar}}. Und erwarte Ihr »\textcolor{green}{Hellas}{}\ledrightnote{\textcolor{green}{Hellas}}«. –\pend
           \pstart
           Bleiben Sie mir weiter, und lange noch der Freund, der Sie mir immer waren; es ist
               schön zu wissen, daß Sie auf der Welt sind! Ich grüße Sie von Herzen!\pend
           \pstart
           Ihr{\\[\baselineskip]}\spacefill\mbox{Arthur Schnitzler}\pend
           \leftskip=0em{}\endnumbering\briefempfaengerindex{Brandes, Georg@\textsc{Brandes, Georg}!zzzSchnitzler, Arthur@\emph{von Arthur Schnitzler}!1925-07-071@{7. 7. 1925}|)be}\mylabel{h}  \normalsize

\doendnotes{C}
\bigskip
\vfill

\clearpage

\footnotesize

\lohead{\textsc{register}}

% Definiere theindex-Environment komplett neu ohne reledmac
\makeatletter
\renewenvironment{theindex}{%
  \section*{\indexname}%
  \setlength{\parindent}{0pt}%
  \setlength{\parskip}{0pt plus 0.3pt}%
  \let\item\@idxitem
}{%
  \clearpage
}
\makeatother

\IfFileExists{\jobname-pw.ind}{\input{\jobname-pw.ind}}{}

\end{document}

      