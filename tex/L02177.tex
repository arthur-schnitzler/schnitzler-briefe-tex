%% latex-korrekturansicht-vorspann.tex
%% Vorspann für die Korrekturansicht.
%% Lädt die gemeinsame Datei latex-vorspann.tex mit gesetztem Schalter.

\newif\ifkorrekturansicht
\korrekturansichttrue

\input{../tex-inputs/latex-vorspann}


               \section[Arthur und Olga Schnitzler an Richard und Paula Beer-Hofmann, 5. 5. 1914]{ Arthur und Olga Schnitzler an Richard und Paula Beer-Hofmann,
                    5. 5. 1914}\nopagebreak\mylabel{v}\rehead{ }\normalsize\beginnumbering\briefempfaengerindex{Beer-Hofmann, Paula@\textsc{Beer-Hofmann, Paula}!zzzSchnitzler, Olga@\emph{von Olga Schnitzler}!1914-05-051@{5. 5. 1914}|(be}\briefempfaengerindex{Beer-Hofmann, Paula@\textsc{Beer-Hofmann, Paula}!zzzSchnitzler, Arthur@\emph{von Arthur Schnitzler}!1914-05-051@{5. 5. 1914}|(be}\briefempfaengerindex{Beer-Hofmann, Richard@\textsc{Beer-Hofmann, Richard}!zzzSchnitzler, Olga@\emph{von Olga Schnitzler}!1914-05-051@{5. 5. 1914}|(be}\briefempfaengerindex{Beer-Hofmann, Richard@\textsc{Beer-Hofmann, Richard}!zzzSchnitzler, Arthur@\emph{von Arthur Schnitzler}!1914-05-051@{5. 5. 1914}|(be} \toendnotes[C]{\smallbreak\pagebreak[2]} \Standort{YCGL, MSS 31.}
\physDesc{Bildpostkarte
\newline{}Handschrift Arthur Schnitzler: Bleistift, deutsche Kurrent\newline{}Handschrift Olga Schnitzler: Bleistift\newline{}Versand: Stempel: »\nobreak{}\oindex{Santa Maria Novella@\textbf{Santa Maria Novella}, \emph{Bahnhofsgebäude (K.BHF)}|pwk}Firenze Ferrovia, 5. V 1914, 20–21\nobreak{}«.  }\pstart{}{\pb}Hrn \textsc{Dr. Richard
                                BeerHofma{\geminationn}}\pend{}\pstart{}u Frau\pend{}\pstart{}\textsc{\textcolor{pink}{Wien XVIII}{}\ledrightnote{\textcolor{pink}{XVIII., Währing}}}\pend{}\pstart{}\textcolor{pink}{\textsc{Hasenauerstr. 59}}{}\ledrightnote{\textcolor{pink}{Hasenauerstraße}}\pend{}{\bigskip}\pstart
           \noindent{}\centering{}{\pb}\textcolor{gray}{\textbf{\textcolor{pink}{Firenze – R. Galleria Uffizi}{}\ledrightnote{\textcolor{pink}{Uffizien}}}}\pend
           \pstart
           \noindent{}\centering{}\textcolor{gray}{\textbf{\textcolor{green}{L’adorazione dei Re Magi}{}\ledrightnote{\textcolor{green}{Die Anbetung der heiligen drei Könige}} (\textcolor{blue}{Botticelli}{}\ledrightnote{\textcolor{blue}{Sandro Botticelli}})}}\pend
           \pstart
           \centering{}{\pb}\textcolor{pink}{Florenz}{}\ledrightnote{\textcolor{pink}{Florenz}}{ }5. 5. 914\pend
           \pstart
           Herzliche Grüße!\pend
           \pstart
           Wir haben das ſchönſte Wetter und befinden uns ſehr wohl.\pend
           \pstart
           \spacefill\mbox{Arthur.}{\\[\baselineskip]}\spacefill\mbox{{[}hs. O. Schnitzler:{]} Olga.}\pend
           \leftskip=0em{}\endnumbering\briefempfaengerindex{Beer-Hofmann, Paula@\textsc{Beer-Hofmann, Paula}!zzzSchnitzler, Olga@\emph{von Olga Schnitzler}!1914-05-051@{5. 5. 1914}|)be}\briefempfaengerindex{Beer-Hofmann, Paula@\textsc{Beer-Hofmann, Paula}!zzzSchnitzler, Arthur@\emph{von Arthur Schnitzler}!1914-05-051@{5. 5. 1914}|)be}\briefempfaengerindex{Beer-Hofmann, Richard@\textsc{Beer-Hofmann, Richard}!zzzSchnitzler, Olga@\emph{von Olga Schnitzler}!1914-05-051@{5. 5. 1914}|)be}\briefempfaengerindex{Beer-Hofmann, Richard@\textsc{Beer-Hofmann, Richard}!zzzSchnitzler, Arthur@\emph{von Arthur Schnitzler}!1914-05-051@{5. 5. 1914}|)be}\mylabel{h}  \normalsize

\doendnotes{C}
\bigskip
\vfill

\clearpage

\footnotesize

\lohead{\textsc{register}}

% Definiere theindex-Environment komplett neu ohne reledmac
\makeatletter
\renewenvironment{theindex}{%
  \section*{\indexname}%
  \setlength{\parindent}{0pt}%
  \setlength{\parskip}{0pt plus 0.3pt}%
  \let\item\@idxitem
}{%
  \clearpage
}
\makeatother

\IfFileExists{\jobname-pw.ind}{\input{\jobname-pw.ind}}{}

\end{document}

      