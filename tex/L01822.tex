%% latex-korrekturansicht-vorspann.tex
%% Vorspann für die Korrekturansicht.
%% Lädt die gemeinsame Datei latex-vorspann.tex mit gesetztem Schalter.

\newif\ifkorrekturansicht
\korrekturansichttrue

\input{../tex-inputs/latex-vorspann}


               \section[Albert Ehrenstein an Arthur Schnitzler, 15. 1. 1909]{ Albert Ehrenstein an Arthur Schnitzler, 15. 1. 1909}\nopagebreak\mylabel{v}\rehead{ }\normalsize\beginnumbering\briefempfaengerindex{Schnitzler, Arthur@\textsc{Schnitzler, Arthur}!zzzEhrenstein, Albert@\emph{von Albert Ehrenstein}!1909-01-152@{15. 1. 1909}|(be} \toendnotes[C]{\smallbreak\pagebreak[2]} \Standort{CUL, Schnitzler, B 30.}
\physDesc{Brief, 1 Blatt, 2 Seiten
\newline{}Handschrift: schwarze Tinte, deutsche Kurrent
\newline{}Schnitzler: mit Bleistift beschriftet: »\textsc{Ehrenstein}« }\buchAbdrucke{\weitereDrucke{Albert Ehrenstein: \emph{Briefe}. Hg. Hanni Mittelmann. München: \emph{Boer} 1989, S. 24 (Werke, 1).} }\toendnotes[C]{\smallbreak}\pstart
           \raggedleft{}{\pb}15. Jan. 09.\pend
           \pstart{}\textsc{Sehr geehrter Herr Doktor!}\pend\pstart
           Die hiſtoriſche \textcolor{green}{Novellette}{}\ledrightnote{→\textcolor{green}{Tod des Zehir eddin Muhammed Baber}} zu ſchreiben, von der ich das letztemal Ihnen, ſehr geehrter Herr
               Doktor, ſprach, iſt mir vorläufig mißlungen. Die Langeweile, welche mir die
               Beſchäftigung mit ihr verurſachte, war ſo enorm, daß ich mich nicht dazu haben konnte
               ſie zu vollenden, trotzdem der bereits von heftigem Fieber gequälte Held nur noch
               binnen drei Seiten zu ſterben hatte. Glücklicherweiſe \label{K_L01822_1v}\edtext{träumte mir im vorigen Monat}{\lemma{\textnormal{\emph{träumte … Monat}}}\Cendnote{\textnormal{am 7. 12. 1908, vgl. \textcolor{blue}{Ehrenstein}: \emph{Briefe},
                  S. 24.}}}\label{K_L01822_1h} ein \textcolor{green}{Märchen}{}\ledrightnote{→\textcolor{green}{Tai-Gin}}, das den Vorzug hat, für die \textcolor{brown}{Öſterreichiſche Rundschau}{}\ledrightnote{\textcolor{brown}{Österreichische Rundschau}} nicht ganz ungeeignet zu ſcheinen. Wenn nun Sie,
                  {\pb}ſehr geehrter Herr Doktor, dieſes \textcolor{green}{Opusculum}{}\ledrightnote{→\textcolor{green}{Tai-Gin}} einer geneigten
               Durchſicht zu unterziehen die Güte hätten, würde mir das eine große Freude bereiten.
               Denn bei dem nicht geringen Volumen des von mir für die \textcolor{green}{Diſſertation}{}\ledrightnote{→\textcolor{green}{Die Lage in Ungarn (Siebenbürgen und Serbien ausgenommen) im Jahre 1790}} zu bearbeitenden Aktenmaterials, würde mir
               eine neuerliche Hingabe an zeitraubend-wertloſe literariſche Experimente gegenwärtig
               recht schwer fallen.\pend
           \pstart
           Mit der Bitte, die kaum leichtfertige Inanſpruchnahme Ihrer koſtbaren Zeit nicht
               allzu ungünſtig beurteilen zu wollen, verbleibe ich ergebenſt Ihr Sie verehrender\pend
           \pstart \spacefill\mbox{Albert Ehrenstein.}\pend{}\endnumbering\briefempfaengerindex{Schnitzler, Arthur@\textsc{Schnitzler, Arthur}!zzzEhrenstein, Albert@\emph{von Albert Ehrenstein}!1909-01-152@{15. 1. 1909}|)be}\mylabel{h}  \normalsize

\doendnotes{C}
\bigskip
\vfill

\clearpage

\footnotesize

\lohead{\textsc{register}}

% Definiere theindex-Environment komplett neu ohne reledmac
\makeatletter
\renewenvironment{theindex}{%
  \section*{\indexname}%
  \setlength{\parindent}{0pt}%
  \setlength{\parskip}{0pt plus 0.3pt}%
  \let\item\@idxitem
}{%
  \clearpage
}
\makeatother

\IfFileExists{\jobname-pw.ind}{\input{\jobname-pw.ind}}{}

\end{document}

      