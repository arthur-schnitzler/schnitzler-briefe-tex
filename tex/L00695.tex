%% latex-korrekturansicht-vorspann.tex
%% Vorspann für die Korrekturansicht.
%% Lädt die gemeinsame Datei latex-vorspann.tex mit gesetztem Schalter.

\newif\ifkorrekturansicht
\korrekturansichttrue

\input{../tex-inputs/latex-vorspann}


               \section[Hermann Bahr an Arthur Schnitzler, 8. 7. 1897]{ Hermann Bahr an Arthur Schnitzler, 8. 7. 1897}\nopagebreak\mylabel{v}\rehead{ }\normalsize\beginnumbering\briefempfaengerindex{Schnitzler, Arthur@\textsc{Schnitzler, Arthur}!zzzBahr, Hermann@\emph{von Hermann Bahr}!1897-07-082@{8. 7. 1897}|(be} \toendnotes[C]{\smallbreak\pagebreak[2]} \Standort{CUL, Schnitzler, B 5b.}
\physDesc{Brief, 1 Blatt, 4 Seiten
\newline{}Handschrift: schwarze Tinte, deutsche Kurrent\newline{}Ordnung: mit Bleistift von unbekannter Hand nummeriert: »53« }\buchAbdrucke{\weitereDrucke{Hermann Bahr, Arthur Schnitzler: \emph{Briefwechsel, Aufzeichnungen, Dokumente (1891–1931)}. Hg. Kurt Ifkovits und Martin Anton Müller. Göttingen: \emph{Wallstein} 2018, S. 148.} }\toendnotes[C]{\smallbreak}\pstart
           \noindent{}{\pb}\textcolor{gray}{\textbf{»\textcolor{brown}{Die
                        Zeit}{}\ledrightnote{\textcolor{brown}{Die Zeit. Wiener Wochenschrift}}«}}\hfill \textcolor{gray}{\textbf{\textbf{\textcolor{pink}{Wien}{}\ledrightnote{\textcolor{pink}{Wien}}}, den }}8. Juli \textcolor{gray}{\textbf{189}}7\pend
           \pstart
           \textcolor{gray}{\textbf{Wiener Wochenſchrift}}\hfill \textcolor{gray}{\textbf{\textcolor{pink}{IX/3, Günthergaſſe 1}{}\ledrightnote{\textcolor{pink}{Günthergasse}}.}}\pend
           \pstart
           \textcolor{gray}{\textbf{\textbf{Herausgeber}:}}{\\}\textcolor{gray}{\textbf{Profeſſor Dr. \textcolor{blue}{I. Singer}{}\ledrightnote{\textcolor{blue}{Isidor Singer}}, \textcolor{blue}{Hermann Bahr}{}\ledrightnote{\textcolor{blue}{Hermann Bahr}},
                        Dr. \textcolor{blue}{Heinrich Kanner}{}\ledrightnote{\textcolor{blue}{Heinrich Kanner}}.}}\pend
           \pstart
           \textcolor{gray}{\textbf{Telephon Nr. 6415.}}\pend
           \pstart\center{}Lieber Freund!\pend\pstart
           \textcolor{blue}{Neumann-Hofers}{}\ledrightnote{\textcolor{blue}{Gilbert Otto Neumann-Hofer}} Drängen nachgebend, der mich
               noch immer mit Dir plagt, frage ich noch einmal bei Dir an, ob Du denn nicht doch
               irgendwie zu beſtimmen wäreſt, einen Vertrag mit ihm einzugehen, der Dich für drei
               oder fünf Jahre an ſein \textcolor{pink}{Theater}{}\ledrightnote{→\textcolor{pink}{Lessing-Theater}}
               bindet. Ich habe Dir schon geſagt: er bietet Dir 12{\%}
               Tantièmen an, oder wenn Du es vorziehſt, ein Einreichungs{\pb}honorar; eventuell ließe er ſich wohl zu beidem
               bereden. Es iſt ihm ſehr wichtig, gerade Dich zu haben. Stelle Deine Forderungen; ich
               habe neulich in den paar Minuten Dir nicht ſo recht zureden können u. weiß nicht, ob
               ich Dich in \textcolor{pink}{Iſchl}{}\ledrightnote{\textcolor{pink}{Bad Ischl}}{ }ſehen werde. Ich bitte Dich
               alſo brieflich, Dir die Sache doch noch einmal zu überlegen. Sie hat gewiß ihre
               Bedenken. Aber überlege Dir, ob ſie ſich nicht ſo drehen läßt, daß ſie die größten
               Vorzüge für Dich hat. Suche Dir etwa Termine aus, wie Du ſie ſonſt an keinem Theater
                  {\pb}kriegſt, oder was ſonſt etwa in Deinen Wünſchen
               liegt. Ich weiß ja nicht, worauf Du am meiſten Werth legſt. Schreib mir das dann. Ich
               würde ſehr wünſchen, daß Du doch irgendwie mit \textcolor{blue}{Neumannhofer}{}\ledrightnote{\textcolor{blue}{Gilbert Otto Neumann-Hofer}} zuſammen kommſt: denn ich hoffe ſo diesen allmälig dahin zu
               bringen, daß er aus dem \textcolor{brown}{Leſſingtheater}{}\ledrightnote{\textcolor{brown}{Lessing-Theater}}
               eine gut \textcolor{pink}{öſtreichiſche}{}\ledrightnote{\textcolor{pink}{Österreich}} Bühne macht. Dies würde
               ich von Herzen wünſchen.\pend
           \pstart
           In der Hoffnung, daß es Dir immer gut geht, bin ich, mit vielen Grüßen {\pb}an \textcolor{blue}{Richard}{}\ledrightnote{\textcolor{blue}{Richard Beer-Hofmann}},\pend
           \pstart
           Dein alter treuer{\\[\baselineskip]}\spacefill\mbox{Hermann}\pend
           \leftskip=0em{}\pstart
           \textcolor{gray}{\textbf{\label{T_L00695_1v}\edtext{Alle für »\textcolor{brown}{Die Zeit}{}\ledrightnote{\textcolor{brown}{Die Zeit. Wiener Wochenschrift}}« beſtimmten Zuſchriften und Sendungen ſind an
                  die Redaction der »\textcolor{brown}{Zeit}{}\ledrightnote{\textcolor{brown}{Die Zeit. Wiener Wochenschrift}}« und \textbf{nicht} an die Perſon eines der Herausgeber zu richten.}{\lemma{\textnormal{\emph{Alle … richten.}}}\Cendnote{\textnormal{am unteren Rand der ersten Seite}}}\label{T_L00695_1h}}}\pend
           \endnumbering\briefempfaengerindex{Schnitzler, Arthur@\textsc{Schnitzler, Arthur}!zzzBahr, Hermann@\emph{von Hermann Bahr}!1897-07-082@{8. 7. 1897}|)be}\mylabel{h}  \normalsize

\doendnotes{C}
\bigskip
\vfill

\clearpage

\footnotesize

\lohead{\textsc{register}}

% Definiere theindex-Environment komplett neu ohne reledmac
\makeatletter
\renewenvironment{theindex}{%
  \section*{\indexname}%
  \setlength{\parindent}{0pt}%
  \setlength{\parskip}{0pt plus 0.3pt}%
  \let\item\@idxitem
}{%
  \clearpage
}
\makeatother

\IfFileExists{\jobname-pw.ind}{\input{\jobname-pw.ind}}{}

\end{document}

      