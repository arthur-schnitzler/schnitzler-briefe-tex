%% latex-korrekturansicht-vorspann.tex
%% Vorspann für die Korrekturansicht.
%% Lädt die gemeinsame Datei latex-vorspann.tex mit gesetztem Schalter.

\newif\ifkorrekturansicht
\korrekturansichttrue

\input{../tex-inputs/latex-vorspann}


               \section[Gerty von Hofmannsthal an Arthur Schnitzler, 7. 3. 1925]{ Gerty von Hofmannsthal an Arthur Schnitzler, 7. 3. 1925}\nopagebreak\mylabel{v}\rehead{ }\normalsize\beginnumbering\briefempfaengerindex{Schnitzler, Arthur@\textsc{Schnitzler, Arthur}!zzzHofmannsthal, Gertrude von@\emph{von Gertrude von Hofmannsthal}!1925-03-071@{7. 3. 1925}|(be} \toendnotes[C]{\smallbreak\pagebreak[2]} \Standort{CUL, Schnitzler, B 43.}
\physDesc{Postkarte
\newline{}Handschrift: schwarze Tinte, lateinische Kurrent\newline{}Versand: Stempel: »\nobreak{}\oindex{I., Innere Stadt@\textbf{I., Innere Stadt}, \emph{Bezirk (A.BZK)}|pwk}1/1 Wien 15, 8. III. 25, VIII\nobreak{}«.  
\newline{}Schnitzler: mit Bleistift beschriftet »\textsc{Gerty Hofmannst}« und die Jahreszahl beim Datum ergänzt: »25« \newline{}Ordnung: 1) mit Bleistift von unbekannter Hand nummeriert: »\strikeout{386}« 2) mit Bleistift von unbekannter Hand nummeriert: »\strikeout{389}«}\buchAbdrucke{\weitereDrucke{Hugo von Hofmannsthal, Arthur Schnitzler: \emph{Briefwechsel}. Hg. Therese Nickl und Heinrich Schnitzler. Frankfurt am Main: \emph{S. Fischer} 1964, S. 394.} }\toendnotes[C]{\smallbreak}\pstart{}{\pb}S. H.\pend{}\pstart{}Herrn Dr. Arthur Schnitzler\pend{}\pstart{}\textcolor{pink}{Wien XVIII}{}\ledrightnote{\textcolor{pink}{XVIII., Währing}}\pend{}\pstart{}\textcolor{pink}{Sternwartestrasse 71}{}\ledrightnote{\textcolor{pink}{Sternwartestraße}}\pend{}{\bigskip}\pstart
           \raggedleft{}{\pb}7/III\pend
           \pstart
           Lieber Arthur, ich verdanke Ihnen den schönen Abend \label{K_L02437_1v}\edtext{neulich}{\lemma{\textnormal{\emph{neulich}}}\Cendnote{\textnormal{vgl. A. S.: \emph{Tagebuch}, 1. 3. 1925}}}\label{K_L02437_1h} und habe mich wirklich wunderbar unterhalten. \textcolor{blue}{Waldau}{}\ledrightnote{\textcolor{blue}{Gustav Waldau}} war doch \uline{ganz} reizend!\pend
           \pstart
           Da Sie neulich so rührend waren mir zu helfen so will ich Ihnen noch sagen, dass
               leider meine Depesche \textcolor{blue}{Hugo}{}\ledrightnote{\textcolor{blue}{Hugo von Hofmannsthal}} nicht mehr erreicht
               hat. Ich verschiebe jetzt die ganze \label{K_L02437_2v}\edtext{Auseinandersetzung}{\lemma{\textnormal{\emph{Auseinandersetzung}}}\Cendnote{\textnormal{Sie sollte einen
                  Vortrag \textcolor{blue}{Hofmannsthal}s verschieben, aber der
                  Veranstalter hatte mit einer Strafzahlung gedroht.}}}\label{K_L02437_2h} bis nach \textcolor{blue}{Hugo}{}\ledrightnote{\textcolor{blue}{Hugo von Hofmannsthal}}s {\pb}Rückkunft. Auch würden weitere Briefe von mir (ohne Hilfe) die Sache nur
               abschwächen. Ein bisschen schien er schon »kleiner« in seiner Antwort!\pend
           \pstart
           Von \textcolor{blue}{Hugo}{}\ledrightnote{\textcolor{blue}{Hugo von Hofmannsthal}} das erste Telegr. auf dem Meer dass er
               sehr zufrieden ist.\pend
           \pstart Viele herzliche Grüsse und nochmals Dank Ihre \spacefill\mbox{Gerty}\pend{}\endnumbering\briefempfaengerindex{Schnitzler, Arthur@\textsc{Schnitzler, Arthur}!zzzHofmannsthal, Gertrude von@\emph{von Gertrude von Hofmannsthal}!1925-03-071@{7. 3. 1925}|)be}\mylabel{h}  \normalsize

\doendnotes{C}
\bigskip
\vfill

\clearpage

\footnotesize

\lohead{\textsc{register}}

% Definiere theindex-Environment komplett neu ohne reledmac
\makeatletter
\renewenvironment{theindex}{%
  \section*{\indexname}%
  \setlength{\parindent}{0pt}%
  \setlength{\parskip}{0pt plus 0.3pt}%
  \let\item\@idxitem
}{%
  \clearpage
}
\makeatother

\IfFileExists{\jobname-pw.ind}{\input{\jobname-pw.ind}}{}

\end{document}

      