%% latex-korrekturansicht-vorspann.tex
%% Vorspann für die Korrekturansicht.
%% Lädt die gemeinsame Datei latex-vorspann.tex mit gesetztem Schalter.

\newif\ifkorrekturansicht
\korrekturansichttrue

\input{../tex-inputs/latex-vorspann}


               \section[Hermann Bahr: Widmungsexemplar Theater. Roman für Arthur Schnitzler, {[}nach dem 20. 3. 1897{]}]{ Hermann Bahr: Widmungsexemplar Theater. Roman für Arthur Schnitzler, {[}nach dem
               20. 3. 1897{]}}\nopagebreak\mylabel{v}\rehead{ }\normalsize\beginnumbering\briefempfaengerindex{Schnitzler, Arthur@\textsc{Schnitzler, Arthur}!zzzBahr, Hermann@\emph{von Hermann Bahr}!1897-03-201@{{[}nach dem
                  20. 3. 1897{]}}|(be} \toendnotes[C]{\smallbreak\pagebreak[2]} \Standort{DLA, G:Schnitzler, Arthur (Sammlung Heinrich Schnitzler).}
\physDesc{Widmung am Vortitel
\newline{}Handschrift: schwarze Tinte, deutsche Kurrent\newline{}Ordnung: bei der Enteignung des Exemplars 1938 von unbekannter Hand mit Bleistift ergänzte
                           Informationen: »\noindent{}1. Ausgabe 0{ / }handſchriftliche Widm. d. Verf.« }\buchAbdrucke{\weitereDrucke{Hermann Bahr, Arthur Schnitzler: \emph{Briefwechsel, Aufzeichnungen, Dokumente (1891–1931)}. Hg. Kurt Ifkovits und Martin Anton Müller. Göttingen: \emph{Wallstein} 2018, S. 139.} }\pstart
           \noindent{}{\pb}Meinem lieben Arthur\pend
           \pstart \spacefill\mbox{Hermann Bahr}\pend{}\pstart
           \noindent{}März 1897\pend
           {\bigskip}\pstart
           \noindent{}\centering{}\textcolor{gray}{\textbf{\textcolor{green}{Theater}{}\ledrightnote{\textcolor{green}{Theater. Ein Wiener Roman}}.}}\pend
           {\bigskip}\pstart
           \noindent{}\centering{}{\pb}\textcolor{gray}{\textbf{HERMANN BAHR.}}\pend
           \pstart
           \noindent{}\centering{}\textcolor{gray}{\textbf{\textcolor{green}{Theater}{}\ledrightnote{\textcolor{green}{Theater. Ein Wiener Roman}}.}}\pend
           \pstart
           \noindent{}\centering{}\textcolor{gray}{\textbf{\textbf{Ein \textcolor{pink}{Wien}{}\ledrightnote{\textcolor{pink}{Wien}}er
                     Roman.}}}\pend
           {\bigskip}\pstart
           \noindent{}\centering{}\textcolor{gray}{\textbf{\textcolor{pink}{BERLIN}{}\ledrightnote{\textcolor{pink}{Berlin}}}}\pend
           \pstart
           \noindent{}\centering{}\textcolor{gray}{\textbf{\textcolor{brown}{\so{S. Fiſcher, Verlag}}{}\ledrightnote{\textcolor{brown}{S. Fischer Verlag}}}}\pend
           \pstart
           \noindent{}\centering{}\textcolor{gray}{\textbf{1897.}}\pend
           \endnumbering\briefempfaengerindex{Schnitzler, Arthur@\textsc{Schnitzler, Arthur}!zzzBahr, Hermann@\emph{von Hermann Bahr}!1897-03-201@{{[}nach dem
                  20. 3. 1897{]}}|)be}\mylabel{h}  \normalsize

\doendnotes{C}
\bigskip
\vfill

\clearpage

\footnotesize

\lohead{\textsc{register}}

% Definiere theindex-Environment komplett neu ohne reledmac
\makeatletter
\renewenvironment{theindex}{%
  \section*{\indexname}%
  \setlength{\parindent}{0pt}%
  \setlength{\parskip}{0pt plus 0.3pt}%
  \let\item\@idxitem
}{%
  \clearpage
}
\makeatother

\IfFileExists{\jobname-pw.ind}{\input{\jobname-pw.ind}}{}

\end{document}

      