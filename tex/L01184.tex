%% latex-korrekturansicht-vorspann.tex
%% Vorspann für die Korrekturansicht.
%% Lädt die gemeinsame Datei latex-vorspann.tex mit gesetztem Schalter.

\newif\ifkorrekturansicht
\korrekturansichttrue

\input{../tex-inputs/latex-vorspann}


               \section[Hermann Bahr an Arthur Schnitzler, 27. 10. {[}1901{]}]{ Hermann Bahr an Arthur Schnitzler, 27. 10. {[}1901{]}}\nopagebreak\mylabel{v}\rehead{ }\normalsize\beginnumbering\briefempfaengerindex{Schnitzler, Arthur@\textsc{Schnitzler, Arthur}!zzzBahr, Hermann@\emph{von Hermann Bahr}!1901-10-271@{27. 10. 1901}|(be} \toendnotes[C]{\smallbreak\pagebreak[2]} \Standort{CUL, Schnitzler, B 5b.}
\physDesc{Brief, 1 Blatt, 3 Seiten
\newline{}Handschrift: blaue Tinte, deutsche Kurrent
\newline{}Schnitzler: mit Bleistift die Jahreszahl »901« ergänzt \newline{}Ordnung: mit Bleistift von unbekannter Hand nummeriert: »82« }\buchAbdrucke{\weitereDrucke{Hermann Bahr, Arthur Schnitzler: \emph{Briefwechsel, Aufzeichnungen, Dokumente (1891–1931)}. Hg. Kurt Ifkovits und Martin Anton Müller. Göttingen: \emph{Wallstein} 2018, S. 216–217.} }\toendnotes[C]{\smallbreak}\pstart
           \raggedleft{}{\pb}27. 10.\pend
           \pstart\center{}Lieber Arthur!\pend\pstart
           Für Deinen lieben Brief danke ich Dir ſehr. – Die \textcolor{green}{Pantomime}{}\ledrightnote{→\textcolor{green}{Die Pantomime vom braven Manne}} finde ich ſehr, ſehr ſchlecht; ich habe ſie nur
               abgedruckt, um den \textcolor{pink}{Berlin}{}\ledrightnote{\textcolor{pink}{Berlin}}ern mitzutheilen, daß
               ich ſchon 1892\textsc{en plein naturalisme} Pantomimen gemacht habe (wie übrigens
               Du und \textcolor{blue}{Hugo}{}\ledrightnote{\textcolor{blue}{Hugo von Hofmannsthal}} und \textcolor{blue}{Richard}{}\ledrightnote{\textcolor{blue}{Richard Beer-Hofmann}} auch).\pend
           \pstart
           {\pb}Mit Baron \textcolor{blue}{\textsc{Berger}}{}\ledrightnote{\textcolor{blue}{Alfred von Berger}} habe ich lange über Deine Stücke geſprochen:
               er hält die »\textcolor{green}{letzten Maſken}{}\ledrightnote{\textcolor{green}{Die letzten Masken}}« und »\textcolor{green}{Literatur}{}\ledrightnote{\textcolor{green}{Literatur}}« für »Meiſterwerke erſten Ranges«,
               während er für das Sceniſche der »\textcolor{green}{Frau mit dem
                  Dolch}{}\ledrightnote{\textcolor{green}{Die Frau mit dem Dolche}}« Angſt zu haben ſcheint.\pend
           \pstart
           Wenn Du mit \textcolor{blue}{\textsc{Bukovics}}{}\ledrightnote{\textcolor{blue}{Emerich von Bukovics}}
               nicht energiſcher biſt, ſage ich Dir {\pb}voraus, daß
               Du in dieser Saiſon nicht mehr dran kommſt.\pend
           \pstart
           \label{K_L01184_1v}\edtext{Raſend}{\lemma{\textnormal{\emph{Raſend}}}\Cendnote{\textnormal{In seiner Besprechung der Inszenierung von \textcolor{blue}{Gerhart Hauptmann}s \textcolor{green}{Stück}, \emph{\textcolor{green}{Berliner Theater.
                     »Einsame Menschen« im Deutschen Theater}} (\emph{\textcolor{brown}{Neue Freie Presse}}, Nr. 13345,
                        19. 10. 1901, S. 1–3), nennt \textcolor{blue}{Goldmann} die jüngeren Bühnenschriftsteller
                  unfähig zum Dramatischem; diese hätten ihre Schwäche zum Ideal erhoben und dabei
                  das Theater langweilig gemacht.}}}\label{K_L01184_1h} war ich über \textcolor{blue}{Goldmanns}{}\ledrightnote{\textcolor{blue}{Paul Goldmann}}{ }\textcolor{green}{Feuilleton »\textcolor{green}{Einſame Menſchen}{}\ledrightnote{\textcolor{green}{Einsame Menschen}}«}{}\ledrightnote{→\textcolor{green}{Berliner Theater. »Einsame Menschen« im Deutschen Theater}}. Das ſollte wirklich polizeilich
               verboten ſein.\pend
           \pstart
           Herzlichſt{\\[\baselineskip]}Dein{\\[\baselineskip]}\spacefill\mbox{Hermann}\pend
           \leftskip=0em{}\endnumbering\briefempfaengerindex{Schnitzler, Arthur@\textsc{Schnitzler, Arthur}!zzzBahr, Hermann@\emph{von Hermann Bahr}!1901-10-271@{27. 10. 1901}|)be}\mylabel{h}  \normalsize

\doendnotes{C}
\bigskip
\vfill

\clearpage

\footnotesize

\lohead{\textsc{register}}

% Definiere theindex-Environment komplett neu ohne reledmac
\makeatletter
\renewenvironment{theindex}{%
  \section*{\indexname}%
  \setlength{\parindent}{0pt}%
  \setlength{\parskip}{0pt plus 0.3pt}%
  \let\item\@idxitem
}{%
  \clearpage
}
\makeatother

\IfFileExists{\jobname-pw.ind}{\input{\jobname-pw.ind}}{}

\end{document}

      