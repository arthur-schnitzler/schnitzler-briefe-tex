%% latex-korrekturansicht-vorspann.tex
%% Vorspann für die Korrekturansicht.
%% Lädt die gemeinsame Datei latex-vorspann.tex mit gesetztem Schalter.

\newif\ifkorrekturansicht
\korrekturansichttrue

\input{../tex-inputs/latex-vorspann}


               \section[Arthur Schnitzler an Hermann Bahr, {[}14. 3.? 1901{]}]{ Arthur Schnitzler an Hermann Bahr, {[}14. 3.? 1901{]}}\nopagebreak\mylabel{v}\rehead{ }\normalsize\beginnumbering\briefempfaengerindex{Bahr, Hermann@\textsc{Bahr, Hermann}!zzzSchnitzler, Arthur@\emph{von Arthur Schnitzler}!1901-03-141@{{[}14. 3.? 1901{]}}|(be} \toendnotes[C]{\smallbreak\pagebreak[2]} \Standort{TMW, HS AM 23339 Ba.}
\physDesc{Brief, 1 Blatt, 3 Seiten
\newline{}Handschrift: Bleistift, deutsche Kurrent\newline{}Ordnung: Lochung }\buchAbdrucke{\weitereDrucke{1) \emph{[September 1901?].} In: Arthur Schnitzler: \emph{The Letters of Arthur Schnitzler to Hermann Bahr}. Edited, annotated, and with an introduction, by Donald G.
                        Daviau. Chapel Hill: \emph{The University of North Carolina Press} 1978, S. 69–70 (University of North Carolina studies in the Germanic languages
                        and literatures, 89).} \weitereDrucke{2) Hermann Bahr, Arthur Schnitzler: \emph{Briefwechsel, Aufzeichnungen, Dokumente (1891–1931)}. Hg. Kurt Ifkovits und Martin Anton Müller. Göttingen: \emph{Wallstein} 2018, S. 202.} }\toendnotes[C]{\smallbreak}\pstart
           \noindent{}{\pb}mein lieber Hermann, es handelt ſich um nichts wichtiges; vielleicht
                  ka{\geminationn} ich also Dienſtg Vormittg zu dir –
               ohne dich im geringſten zu binden. Eines ka{\geminationn} ich dir
               vielleicht gleich hier ſagen, wobei ich dich bitte, gelegentlich zu \textcolor{blue}{\textsc{Bukovis}}{}\ledrightnote{\textcolor{blue}{Emerich von Bukovics}} davon zu reden.\pend
           \pstart
           \label{K_L01103_1v}\edtext{Mein Einakterabend wird beſtehen}{\lemma{\textnormal{\emph{Mein … beſtehen}}}\Cendnote{\textnormal{Zur Vorgeschichte, die sich Ende
                     Februar ereignete, vgl. den Brief \textcolor{blue}{Schnitzler}s an \textcolor{blue}{Emerich von
                     Bukovics}, 11. 12. 1901, in \emph{Briefwechsel} Bahr/Schnitzler 219–220}}}\label{K_L01103_1h} aus »\textcolor{green}{Literatur}{}\ledrightnote{\textcolor{green}{Literatur}}«, einem \label{K_L01103_2v}\edtext{\textcolor{green}{andern}{}\ledrightnote{→\textcolor{green}{Die Frau mit dem Dolche}}, der halb fertig iſt
               ziemlich phantaſtiſch}{\lemma{\textnormal{\emph{andern, … phantaſtiſch}}}\Cendnote{\textnormal{Durch
                  »phantastisch« scheint auf \emph{\textcolor{green}{Die Frau mit dem
                     Dolche}} Bezug genommen zu sein, wobei die Niederschrift erst zwischen
                     Mai und August datierbar ist.}}}\label{K_L01103_2h} und {\pb}einem \label{K_L01103_3v}\edtext{\textcolor{green}{dritten}{}\ledrightnote{→\textcolor{green}{Die letzten Masken}}}{\lemma{\textnormal{\emph{dritten}}}\Cendnote{\textnormal{\label{LKommKL038-3v}Vermutlich \emph{\textcolor{green}{Die letzten Masken}}. Seit 12. 3. 1901 lag der Stoff
                     als Novelle abgeschlossen vor, und am »24. 4.?« (\emph{Cambridge University Library}, Schnitzler,
                        A 80) versuchte \textcolor{blue}{Schnitzler},
                     ihn dramatisch zu bearbeiten.\label{LKommKL038-3h}}}}\label{K_L01103_3h} – den ich noch nicht begonnen habe. –\pend
           \pstart
           Dagegen ſoll \textcolor{green}{Marionetten}{}\ledrightnote{\textcolor{green}{Marionetten. Drei Einakter}} (das hier beſti{\geminationm}t gut wirken wird, in guter Darſtellung) da es doch als
               ſagen wir Literaturſatire nur einen kleinen Kreis intereſſiren kann) lieber an dem
               Abend gegeben werden, wo der \textcolor{green}{Kakadu}{}\ledrightnote{\textcolor{green}{Der grüne Kakadu. Groteske in einem Akt}} aufgeführt
               wird. Alſo irgend was von einem andern (man {\pb}ſprach mir von »\textcolor{green}{\textsc{Fast}nacht}{}\ledrightnote{\textcolor{green}{Fastnacht}}«) dann \textcolor{green}{Kakadu}{}\ledrightnote{\textcolor{green}{Der grüne Kakadu. Groteske in einem Akt}}, am Schluſs \textcolor{green}{\textsc{Marionetten}}{}\ledrightnote{\textcolor{green}{Marionetten. Drei Einakter}}.\pend
           \pstart
           Nun, darüber und \introOben{}über\introOben{} einiges andere nächſtens.\pend
           \pstart
           Viele herzliche Grüße{\\[\baselineskip]}dein{\\[\baselineskip]}\spacefill\mbox{ArthurSch}\pend
           \leftskip=0em{}\endnumbering\briefempfaengerindex{Bahr, Hermann@\textsc{Bahr, Hermann}!zzzSchnitzler, Arthur@\emph{von Arthur Schnitzler}!1901-03-141@{{[}14. 3.? 1901{]}}|)be}\mylabel{h}  \normalsize

\doendnotes{C}
\bigskip
\vfill

\clearpage

\footnotesize

\lohead{\textsc{register}}

% Definiere theindex-Environment komplett neu ohne reledmac
\makeatletter
\renewenvironment{theindex}{%
  \section*{\indexname}%
  \setlength{\parindent}{0pt}%
  \setlength{\parskip}{0pt plus 0.3pt}%
  \let\item\@idxitem
}{%
  \clearpage
}
\makeatother

\IfFileExists{\jobname-pw.ind}{\input{\jobname-pw.ind}}{}

\end{document}

      