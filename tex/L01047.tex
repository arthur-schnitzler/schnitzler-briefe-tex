%% latex-korrekturansicht-vorspann.tex
%% Vorspann für die Korrekturansicht.
%% Lädt die gemeinsame Datei latex-vorspann.tex mit gesetztem Schalter.

\newif\ifkorrekturansicht
\korrekturansichttrue

\input{../tex-inputs/latex-vorspann}


               \section[Arthur Schnitzler an Richard Beer-Hofmann, 21. 6. 1900]{ Arthur Schnitzler an Richard Beer-Hofmann, 21. 6. 1900}\nopagebreak\mylabel{v}\rehead{ }\normalsize\beginnumbering\briefempfaengerindex{Beer-Hofmann, Richard@\textsc{Beer-Hofmann, Richard}!zzzSchnitzler, Arthur@\emph{von Arthur Schnitzler}!1900-06-211@{21. 6. 1900}|(be} \toendnotes[C]{\smallbreak\pagebreak[2]} \Standort{YCGL, MSS 31.}
\physDesc{Bildpostkarte
\newline{}Handschrift: Bleistift, deutsche Kurrent\newline{}Versand: 1) Stempel: »\nobreak{}Wien 54, 22. 6. 00, 7–8V\nobreak{}«.  2) Stempel: »\nobreak{}\oindex{Altaussee@\textbf{Altaussee}, \emph{http://www.geonames.org/ontologyA.ADM3}|pwk}Alt-Aussee, 23 6 00\nobreak{}«. }\buchAbdrucke{\weitereDrucke{Arthur Schnitzler, Richard Beer-Hofmann: \emph{Briefwechsel 1891–1931}. Hg. Konstanze Fliedl. Wien, Zürich: \emph{Europaverlag} 1992, S. 146.} }\toendnotes[C]{\smallbreak}\pstart{}{\pb}Herrn \textsc{Dr. Richard
                     Beer-Hofmann}\pend{}\pstart{}\textsc{\textcolor{pink}{Altaussee}{}\ledrightnote{\textcolor{pink}{Altaussee}}}\pend{}{\bigskip}\pstart
           \noindent{}\centering{}{\pb}\textcolor{gray}{\textbf{\textcolor{pink}{WIEN}{}\ledrightnote{\textcolor{pink}{Wien}}, \textcolor{pink}{K. u.  k. kunsthistorisches Museum}{}\ledrightnote{\textcolor{pink}{Kunsthistorisches Museum}},
                     Stiegenhaus.}}\pend
           \pstart
           \noindent{}Ein \label{K_L01047_1v}\edtext{\textcolor{green}{Citat}{}\ledrightnote{→\textcolor{green}{Theseus besiegt den Kentaur}}}{\lemma{\textnormal{\emph{Citat}}}\Cendnote{\textnormal{Auf dem Bild ist \emph{\textcolor{green}{Theseus besiegt den Kentaur}} zu sehen. In \emph{\textcolor{green}{Frau Bertha Garlan}} erwähnt \textcolor{blue}{Schnitzler} aber das heute in \textcolor{pink}{London}
                     befindliche \emph{\textcolor{green}{Theseus und der Minotaurus}}.}}}\label{K_L01047_1h} aus der neuen \textcolor{green}{Novelle}{}\ledrightnote{→\textcolor{green}{Frau Bertha Garlan. Roman}}. –\pend
           \pstart
           21. 6. 900\pend
           \pstart
           Im \label{K_L01047_2v}\edtext{Vorſtadtbeiſel}{\lemma{\textnormal{\emph{Vorſtadtbeiſel}}}\Cendnote{\textnormal{österreichisch Beisl: einfaches Lokal}}}\label{K_L01047_2h}
               mit den geſchmackloſen Tapeten und den kleinen Beamten am Nebentiſch.\pend
           \pstart
           Herzlichſt{\\[\baselineskip]}Ihr \spacefill\mbox{Arth.}\pend
           \leftskip=0em{}\endnumbering\briefempfaengerindex{Beer-Hofmann, Richard@\textsc{Beer-Hofmann, Richard}!zzzSchnitzler, Arthur@\emph{von Arthur Schnitzler}!1900-06-211@{21. 6. 1900}|)be}\mylabel{h}  \normalsize

\doendnotes{C}
\bigskip
\vfill

\clearpage

\footnotesize

\lohead{\textsc{register}}

% Definiere theindex-Environment komplett neu ohne reledmac
\makeatletter
\renewenvironment{theindex}{%
  \section*{\indexname}%
  \setlength{\parindent}{0pt}%
  \setlength{\parskip}{0pt plus 0.3pt}%
  \let\item\@idxitem
}{%
  \clearpage
}
\makeatother

\IfFileExists{\jobname-pw.ind}{\input{\jobname-pw.ind}}{}

\end{document}

      