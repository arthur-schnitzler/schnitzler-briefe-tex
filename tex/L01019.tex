%% latex-korrekturansicht-vorspann.tex
%% Vorspann für die Korrekturansicht.
%% Lädt die gemeinsame Datei latex-vorspann.tex mit gesetztem Schalter.

\newif\ifkorrekturansicht
\korrekturansichttrue

\input{../tex-inputs/latex-vorspann}


               \section[Richard Beer-Hofmann an Arthur Schnitzler, 7. 3. 1900]{ Richard Beer-Hofmann an Arthur Schnitzler, 7. 3. 1900}\nopagebreak\mylabel{v}\rehead{ }\normalsize\beginnumbering\briefempfaengerindex{Schnitzler, Arthur@\textsc{Schnitzler, Arthur}!zzzBeer-Hofmann, Richard@\emph{von Richard Beer-Hofmann}!1900-03-071@{7. 3. 1900}|(be} \toendnotes[C]{\smallbreak\pagebreak[2]} \Standort{CUL, Schnitzler, B 8.}
\physDesc{Bildpostkarte
\newline{}Handschrift: schwarze Tinte, lateinische Kurrent\newline{}Versand: 1) Stempel: »\nobreak{}\oindex{Bahnhof@\textbf{Bahnhof}, \emph{Bahnhofsgebäude (K.BHF)}|pwk}Fi\textcolor{gray}{renze}
                                 Fer\textcolor{gray}{rovia}, 8 3 00, 11 M\nobreak{}«.  2) Stempel: »\nobreak{}\oindex{IX., Alsergrund@\textbf{IX., Alsergrund}, \emph{Bezirk (A.BZK)}|pwk}Wien 9/3, 10. 3. 00, 8.V, Bestellt\nobreak{}«. \newline{}Ordnung: mit Bleistift von unbekannter Hand nummeriert:
                              »153« }\pstart{}{\pb}D\textsuperscript{r} Arthur Schnitzler\pend{}\pstart{}\textcolor{pink}{Wien}{}\ledrightnote{\textcolor{pink}{Wien}}\pend{}\pstart{}\textcolor{pink}{IX. Frankgasse 1}{}\ledrightnote{\textcolor{pink}{Frankgasse}}\pend{}\pstart{}\textcolor{pink}{Austria}{}\ledrightnote{\textcolor{pink}{Österreich}}\pend{}{\bigskip}\pstart
           \noindent{}\centering{}\textcolor{gray}{\textbf{{\pb}\textcolor{pink}{Venezia}{}\ledrightnote{\textcolor{pink}{Venedig}}{ }\textcolor{brown}{Accademia di Belle Arti}{}\ledrightnote{\textcolor{brown}{Accademia di belle arti di Venezia}}.}}\pend
           \pstart
           \noindent{}\centering{}\textcolor{gray}{\textbf{\textcolor{blue}{Andrea Mantegna}{}\ledrightnote{\textcolor{blue}{Andrea Mantegna}}, \textcolor{green}{San Giorgio}{}\ledrightnote{\textcolor{green}{Sankt Georg}}.}}\pend
           \pstart
           \raggedleft{}\textcolor{pink}{Florenz}{}\ledrightnote{\textcolor{pink}{Florenz}}{\\}7/III 1900\pend
           \pstart{}Lieber Arthur\pend\pstart
           ich hoffe Sonntag in \textcolor{pink}{Wien}{}\ledrightnote{\textcolor{pink}{Wien}} zu sein\pend
           \pstart
           Sind Sie Abends im \textcolor{brown}{Schach-Klub}{}\ledrightnote{\textcolor{brown}{Wiener Schachclub}}?\pend
           \pstart
           Herzlich{\\[\baselineskip]}Ihr\spacefill\mbox{R.}\pend
           \leftskip=0em{}\endnumbering\briefempfaengerindex{Schnitzler, Arthur@\textsc{Schnitzler, Arthur}!zzzBeer-Hofmann, Richard@\emph{von Richard Beer-Hofmann}!1900-03-071@{7. 3. 1900}|)be}\mylabel{h}  \normalsize

\doendnotes{C}
\bigskip
\vfill

\clearpage

\footnotesize

\lohead{\textsc{register}}

% Definiere theindex-Environment komplett neu ohne reledmac
\makeatletter
\renewenvironment{theindex}{%
  \section*{\indexname}%
  \setlength{\parindent}{0pt}%
  \setlength{\parskip}{0pt plus 0.3pt}%
  \let\item\@idxitem
}{%
  \clearpage
}
\makeatother

\IfFileExists{\jobname-pw.ind}{\input{\jobname-pw.ind}}{}

\end{document}

      