%% latex-korrekturansicht-vorspann.tex
%% Vorspann für die Korrekturansicht.
%% Lädt die gemeinsame Datei latex-vorspann.tex mit gesetztem Schalter.

\newif\ifkorrekturansicht
\korrekturansichttrue

\input{../tex-inputs/latex-vorspann}


               \section[Wilhelm Bölsche an Arthur Schnitzler, {[}24. 7. 1892{]}]{ Wilhelm Bölsche an Arthur Schnitzler, {[}24. 7. 1892{]}}\nopagebreak\mylabel{v}\rehead{ }\normalsize\beginnumbering\briefempfaengerindex{Schnitzler, Arthur@\textsc{Schnitzler, Arthur}!zzzBoelsche, Wilhelm@\emph{von Wilhelm Bölsche}!1892-07-241@{{[}24. 7. 1892{]}}|(be} \toendnotes[C]{\smallbreak\pagebreak[2]} \Standort{DLA, A:Schnitzler, HS.NZ85.1.2577,6.}
\physDesc{Brief, 1 Blatt, 2 Seiten
\newline{}Handschrift: schwarze Tinte, deutsche Kurrent
\newline{}Schnitzler: mit Bleistift datiert: »24/7 92« \newline{}Ordnung: mit rotem Buntstift von unbekannter Hand nummeriert:
                                        »7« }\buchAbdrucke{\weitereDrucke{Wilhelm Bölsche: \emph{Briefwechsel. Mit Autoren der Freien Bühne}. Hg. Gerd-Hermann Susen. Berlin: \emph{Weidler} 2010, S. 682 (Werke und Briefe. Wissenschaftliche Ausgabe, Briefe I).} }\toendnotes[C]{\smallbreak}\pstart
           \raggedleft{}{\pb}\textcolor{pink}{Friedrichshagen}{}\ledrightnote{\textcolor{pink}{Friedrichshagen}}{\\}b. \textcolor{pink}{Berlin}{}\ledrightnote{\textcolor{pink}{Berlin}}.{\\}\textcolor{pink}{Wilhelmſtr. 72}{}\ledrightnote{\textcolor{pink}{Peter-Hille-Straße}}.\pend
           \pstart\center{}Hochverehrter Herr Doktor!\pend\pstart
           Zu meinem Erſtaunen erſehe ich aus Ihrem Briefe, daß ein vor längerer Zeit ſchon
                    an Sie abgeſandter Brief Sie offenbar nicht erreicht hat. Ich ſchrieb damals,
                    daß ich betreffs Ihrer \textcolor{green}{Novelle}{}\ledrightnote{→\textcolor{green}{Das Himmelbett}} etwas \introOben{}in\introOben{} Zweifel ſei, ob ſie ſich für
                    eine Zeitſchrift eigne – des Motivs wegen – und ſtellte Ihnen anheim, ob Sie mir
                    nicht lieber eine andere dafür geben wollten. Glücklicher Weiſe – wie ich jetzt
                    ſagen muß – legte ich in {\pb}meiner Unſchlüſſigkeit
                    das Manuſkript nicht bei, – ich wollte es erſt noch von eine\substVorne{}\textsuperscript{m}\substDazwischen{}n\substHinten{} Andern leſen laſſen, um \strikeout{d} zu ſehen, ob
                    ich mich nicht über die bedenkliche Wirkung täuſche. Es iſt alſo noch hier, und
                    ich lege es heute bei – zugleich unter Wiederholung der Bitte um etwas Anderes.
                    Der \textcolor{green}{Stoff}{}\ledrightnote{→\textcolor{green}{Das Himmelbett}} iſt wirklich
                    »zeitſchriftlich« unmöglich!\pend
           \pstart
           Mit herzlichem Gruß{\\[\baselineskip]}Ihr{\\[\baselineskip]}\spacefill\mbox{W. Bölsche}\pend
           \leftskip=0em{}\endnumbering\briefempfaengerindex{Schnitzler, Arthur@\textsc{Schnitzler, Arthur}!zzzBoelsche, Wilhelm@\emph{von Wilhelm Bölsche}!1892-07-241@{{[}24. 7. 1892{]}}|)be}\mylabel{h}  \normalsize

\doendnotes{C}
\bigskip
\vfill

\clearpage

\footnotesize

\lohead{\textsc{register}}

% Definiere theindex-Environment komplett neu ohne reledmac
\makeatletter
\renewenvironment{theindex}{%
  \section*{\indexname}%
  \setlength{\parindent}{0pt}%
  \setlength{\parskip}{0pt plus 0.3pt}%
  \let\item\@idxitem
}{%
  \clearpage
}
\makeatother

\IfFileExists{\jobname-pw.ind}{\input{\jobname-pw.ind}}{}

\end{document}

      