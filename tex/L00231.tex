%% latex-korrekturansicht-vorspann.tex
%% Vorspann für die Korrekturansicht.
%% Lädt die gemeinsame Datei latex-vorspann.tex mit gesetztem Schalter.

\newif\ifkorrekturansicht
\korrekturansichttrue

\input{../tex-inputs/latex-vorspann}


               \section[Arthur Schnitzler an Richard Beer-Hofmann, {[}zwischen 3. und 15. 7. 1893{]}]{ Arthur Schnitzler an Richard Beer-Hofmann, {[}zwischen 3. und
               15. 7. 1893{]}}\nopagebreak\mylabel{v}\rehead{ }\normalsize\beginnumbering\briefempfaengerindex{Beer-Hofmann, Richard@\textsc{Beer-Hofmann, Richard}!zzzSchnitzler, Arthur@\emph{von Arthur Schnitzler}!1893-07-031@{{[}zwischen 3. und
                  15. 7. 1893{]}}|(be} \toendnotes[C]{\smallbreak\pagebreak[2]} \Standort{YCGL, MSS 31.}
\physDesc{Briefkarte mit Trauerrand, Umschlag mit Trauerrand
\newline{}Handschrift: Bleistift, deutsche Kurrent\newline{}Versand: ohne postalischen Übermittlungsvermerk }\pstart{}{\pb}\textsc{Hrn. Dr. Rich. Beer-Hofmann}\pend{}\pstart{}\textsc{\textcolor{pink}{Ischl}{}\ledrightnote{\textcolor{pink}{Bad Ischl}}}\pend{}\pstart{}\textsc{\textcolor{pink}{Schulgasse 8}{}\ledrightnote{\textcolor{pink}{Schulgasse}}}.\pend{}{\bigskip}\pstart{}{\pb}Lieber Richard,\pend\pstart
           ich ka{\geminationn} heute leider nicht zu Ihnen {\pb}ko{\geminationm}en.\pend
           \pstart
           Grüß\textcolor{gray}{en} Sie \textsc{\textcolor{blue}{Loris}{}\ledrightnote{\textcolor{blue}{Hugo von Hofmannsthal}}}.\pend
           \pstart
           We{\geminationn} ich aber \uline{doch} ka{\geminationn}{ }ſo ko{\geminationm} ich erſt nach
               zehn.\pend
           \pstart
           Unwahrſcheinlich.\pend
           \pstart
           Ihr{\\[\baselineskip]}\spacefill\mbox{Arthur}\pend
           \leftskip=0em{}\endnumbering\briefempfaengerindex{Beer-Hofmann, Richard@\textsc{Beer-Hofmann, Richard}!zzzSchnitzler, Arthur@\emph{von Arthur Schnitzler}!1893-07-031@{{[}zwischen 3. und
                  15. 7. 1893{]}}|)be}\mylabel{h}  \normalsize

\doendnotes{C}
\bigskip
\vfill

\clearpage

\footnotesize

\lohead{\textsc{register}}

% Definiere theindex-Environment komplett neu ohne reledmac
\makeatletter
\renewenvironment{theindex}{%
  \section*{\indexname}%
  \setlength{\parindent}{0pt}%
  \setlength{\parskip}{0pt plus 0.3pt}%
  \let\item\@idxitem
}{%
  \clearpage
}
\makeatother

\IfFileExists{\jobname-pw.ind}{\input{\jobname-pw.ind}}{}

\end{document}

      