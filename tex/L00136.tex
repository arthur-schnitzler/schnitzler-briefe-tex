%% latex-korrekturansicht-vorspann.tex
%% Vorspann für die Korrekturansicht.
%% Lädt die gemeinsame Datei latex-vorspann.tex mit gesetztem Schalter.

\newif\ifkorrekturansicht
\korrekturansichttrue

\input{../tex-inputs/latex-vorspann}


               \section[Richard Beer-Hofmann an Arthur Schnitzler, {[}1892–1894?{]}]{ Richard Beer-Hofmann an Arthur Schnitzler, {[}1892–1894?{]}}\nopagebreak\mylabel{v}\rehead{ }\normalsize\beginnumbering\briefempfaengerindex{Schnitzler, Arthur@\textsc{Schnitzler, Arthur}!zzzBeer-Hofmann, Richard@\emph{von Richard Beer-Hofmann}!1892-11-203@{{[}1892–1894?{]}}|(be} \toendnotes[C]{\smallbreak\pagebreak[2]} \Standort{CUL, Schnitzler, B 8.}
\physDesc{Brief, 1 Blatt, 1 Seite
\newline{}Handschrift: schwarze Tinte, lateinische Kurrent
\newline{}Schnitzler: mit Bleistift nummeriert: »17« }\toendnotes[C]{\smallbreak}\pstart{}{\pb}Lieber Arthur!\pend\pstart
           \textcolor{blue}{Specht}{}\ledrightnote{\textcolor{blue}{Richard Specht}} liest \label{K_L00136_1v}\edtext{Samstag}{\lemma{\textnormal{\emph{Samstag}}}\Cendnote{\textnormal{Die mit \textcolor{blue}{Schnitzler}s \emph{\textcolor{green}{Tagebuch}} nachweisbaren
                  Lesungen \textcolor{blue}{Specht}s fanden entweder nicht an einem
                  Samstag oder nicht bei \textcolor{blue}{Beer-Hofmann} statt. Die
                  erste war am 20. 11. 1892, die letzte am 29. 3. 1894.
                  Dementsprechend dürfte auch dieses Korrespondenzstück in diesen Zeitraum
                  fallen.}}}\label{K_L00136_1h}{ }6 Uhr bei mir; bitte pünktlich, wir soupiren dann auswärts zusa{\geminationm}en.\pend
           \pstart
           Herzlichst{\\[\baselineskip]}\spacefill\mbox{Richard.}\pend
           \leftskip=0em{}\pstart
           \noindent{}Bitte Sonntag für um \uline{4}. frei zu halten.\pend
           \endnumbering\briefempfaengerindex{Schnitzler, Arthur@\textsc{Schnitzler, Arthur}!zzzBeer-Hofmann, Richard@\emph{von Richard Beer-Hofmann}!1892-11-203@{{[}1892–1894?{]}}|)be}\mylabel{h}  \normalsize

\doendnotes{C}
\bigskip
\vfill

\clearpage

\footnotesize

\lohead{\textsc{register}}

% Definiere theindex-Environment komplett neu ohne reledmac
\makeatletter
\renewenvironment{theindex}{%
  \section*{\indexname}%
  \setlength{\parindent}{0pt}%
  \setlength{\parskip}{0pt plus 0.3pt}%
  \let\item\@idxitem
}{%
  \clearpage
}
\makeatother

\IfFileExists{\jobname-pw.ind}{\input{\jobname-pw.ind}}{}

\end{document}

      