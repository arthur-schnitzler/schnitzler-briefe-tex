%% latex-korrekturansicht-vorspann.tex
%% Vorspann für die Korrekturansicht.
%% Lädt die gemeinsame Datei latex-vorspann.tex mit gesetztem Schalter.

\newif\ifkorrekturansicht
\korrekturansichttrue

\input{../tex-inputs/latex-vorspann}


               \section[Hermann Bahr an Arthur Schnitzler, 21. 12. {[}1900{]}]{ Hermann Bahr an Arthur Schnitzler, 21. 12. {[}1900{]}}\nopagebreak\mylabel{v}\rehead{ }\normalsize\beginnumbering\briefempfaengerindex{Schnitzler, Arthur@\textsc{Schnitzler, Arthur}!zzzBahr, Hermann@\emph{von Hermann Bahr}!1900-12-211@{21. 12. {[}1900{]}}|(be} \toendnotes[C]{\smallbreak\pagebreak[2]} \Standort{CUL, Schnitzler, B 5b.}
\physDesc{Brief, 1 Blatt, 3 Seiten
\newline{}Handschrift: schwarze Tinte, deutsche Kurrent
\newline{}Schnitzler: mit Bleistift die Jahreszahl »900« ergänzt \newline{}Ordnung: mit Bleistift von unbekannter Hand nummeriert:
                                    »71« }\buchAbdrucke{\weitereDrucke{Hermann Bahr, Arthur Schnitzler: \emph{Briefwechsel, Aufzeichnungen, Dokumente (1891–1931)}. Hg. Kurt Ifkovits und Martin Anton Müller. Göttingen: \emph{Wallstein} 2018, S. 191.} }\toendnotes[C]{\smallbreak}\pstart
           \raggedleft{}{\pb}2\substVorne{}\textsuperscript{2}\substDazwischen{}1\substHinten{}/12\pend
           \pstart\center{}Lieber Arthur!\pend\pstart
           \textcolor{blue}{\textsc{Bukovics}}{}\ledrightnote{\textcolor{blue}{Emerich von Bukovics}}{ }ſagt mir, es ſei über den \label{K_L01086_1v}\edtext{\textcolor{pink}{Volkstheater}{}\ledrightnote{\textcolor{pink}{Volkstheater}}abend}{\lemma{\textnormal{\emph{Volkstheaterabend}}}\Cendnote{\textnormal{Ein jährlich stattfindender Abend in einem angemieteten
                  Veranstaltungssaal mit speziellem Programm. 1901 fand er am
                     9. 3. in den \textcolor{pink}{Sophiensälen} statt.
                  Vor der Eröffnung der Tanzfläche wurden Lieder gesungen und das Mimodrama \emph{\textcolor{green}{Die Hand}} von \textcolor{blue}{Henri
                     Berény} gegeben.}}}\label{K_L01086_1h} dieſes Jahr noch nichts beschloſſen. Ich mache Dich
               nur aufmerkſam, daß bei dem ſpäten Anfang (½ 11), der elenden Bühne (meiſtens \textcolor{pink}{Ronacher}{}\ledrightnote{\textcolor{pink}{Etablissement Ronacher}}) u. der kaum zu bändigenden Tanzluſt hier nur
                  {\pb}ganz einfache u. rohe Sachen wirken.\pend
           \pstart
           Für die lieben Worte Deines Briefes danke ich Dir ſehr und bin, Dir das Beſte
               wünſchend,\pend
           \pstart
           Dein alter{\\[\baselineskip]}\spacefill\mbox{Hermann}\pend
           \leftskip=0em{}\pstart
           \noindent{}Hofrath \textcolor{blue}{\textsc{Burckhard}}{}\ledrightnote{\textcolor{blue}{Max Eugen Burckhard}} möchte ſehr gern ein Exemplar der {\pb}\textcolor{green}{Beatrice}{}\ledrightnote{\textcolor{green}{Der Schleier der Beatrice. Schauspiel in fünf Akten}} haben; kannſt Du ihm nicht eins
                  ſchicken?\pend
           \endnumbering\briefempfaengerindex{Schnitzler, Arthur@\textsc{Schnitzler, Arthur}!zzzBahr, Hermann@\emph{von Hermann Bahr}!1900-12-211@{21. 12. {[}1900{]}}|)be}\mylabel{h}  \normalsize

\doendnotes{C}
\bigskip
\vfill

\clearpage

\footnotesize

\lohead{\textsc{register}}

% Definiere theindex-Environment komplett neu ohne reledmac
\makeatletter
\renewenvironment{theindex}{%
  \section*{\indexname}%
  \setlength{\parindent}{0pt}%
  \setlength{\parskip}{0pt plus 0.3pt}%
  \let\item\@idxitem
}{%
  \clearpage
}
\makeatother

\IfFileExists{\jobname-pw.ind}{\input{\jobname-pw.ind}}{}

\end{document}

      