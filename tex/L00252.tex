%% latex-korrekturansicht-vorspann.tex
%% Vorspann für die Korrekturansicht.
%% Lädt die gemeinsame Datei latex-vorspann.tex mit gesetztem Schalter.

\newif\ifkorrekturansicht
\korrekturansichttrue

\input{../tex-inputs/latex-vorspann}


               \section[Hermann Bahr an Arthur Schnitzler, {[}12. 8. 1893{]}]{ Hermann Bahr an Arthur Schnitzler, {[}12. 8. 1893{]}}\nopagebreak\mylabel{v}\rehead{ }\normalsize\beginnumbering\briefempfaengerindex{Schnitzler, Arthur@\textsc{Schnitzler, Arthur}!zzzBahr, Hermann@\emph{von Hermann Bahr}!1893-08-121@{{[}12. 8. 1893{]}}|(be} \toendnotes[C]{\smallbreak\pagebreak[2]} \Standort{CUL, Schnitzler, B 5b.}
\physDesc{Brief, 1 Blatt, 3 Seiten
\newline{}Handschrift: schwarze Tinte, deutsche Kurrent
\newline{}Schnitzler: mit Bleistift datiert: »Mitte Aug 93« \newline{}Ordnung: 1) mit rotem Buntstift von unbekannter Hand nummeriert:
                                    »12« 2) mit Bleistift von unbekannter Hand nummeriert:
                                    »12«}\buchAbdrucke{\weitereDrucke{Hermann Bahr, Arthur Schnitzler: \emph{Briefwechsel, Aufzeichnungen, Dokumente (1891–1931)}. Hg. Kurt Ifkovits und Martin Anton Müller. Göttingen: \emph{Wallstein} 2018, S. 36.} }\toendnotes[C]{\smallbreak}\pstart\center{}{\pb}Lieber Freund!\pend\pstart
           Ich bin verzweifelt. Ihr Brief trifft mich im Packen – \label{K_L00252_1v}\edtext{ich verreiſe}{\lemma{\textnormal{\emph{ich verreiſe}}}\Cendnote{\textnormal{an
                  seinen \textcolor{blue}{Vater},
                     12. 8. 1893: »Ich verreise heute Abend auf einige Tage nach
                        \textcolor{pink}{Böhmen} und kann keine Adresse angeben, da
                     ich sie selber noch nicht weiß und mich auch nirgends länger als ein paar
                     Stunden aufhalten werde.« (\emph{Theatermuseum Wien}, AM 50775 Ba)}}}\label{K_L00252_1h} heute auf ein paar Tage.
               Ich fange alſo ſofort zu suchen an – denn irgendwo habe ich ja dieſes verruchte
                  \label{K_L00252_2v}\edtext{\textcolor{green}{Amerika}{}\ledrightnote{\textcolor{green}{Amerika}}}{\lemma{\textnormal{\emph{Amerika}}}\Cendnote{\textnormal{\textcolor{blue}{Arthur Schnitzler}: \emph{\textcolor{green}{Amerika}}. In: \emph{\textcolor{green}{An der schönen
                        blauen Donau}}, Jg. 4, H. 9, {[}1. 5.{]} 1889,
                  S. 197.}}}\label{K_L00252_2h}, aber wo? Ich habe alles von unterſt zu oberſt gekehrt –
               bisher umſonſt. Mittwoch komme ich {\pb}auf ein oder
               zwei Tage zurück u. will dann wie ein Sträfling ſuchen. Sind Sie ſehr böſe, we{\geminationn} ich Sie bis dahin vertröſte?\pend
           \pstart
           Ich muß dann ohnehin zu Ihnen um Ihnen wegen des \textcolor{blue}{Regimentsarztes}{}\ledrightnote{→\textcolor{blue}{?? [Regimentsarzt]}} zu danken u. Sie zu fragen, in welcher
               Weiſe es für mich angemeſſen ist, mich bei dem Herrn zu \textsc{revan{\pb}chieren}.\pend
           \pstart
           In großer Haſt{\\[\baselineskip]}Ihr treuer{\\[\baselineskip]}\spacefill\mbox{Bahr}\pend
           \leftskip=0em{}\pstart
           \noindent{}Schreiben Sie uns doch einmal ein Feuilleton!\pend
           \endnumbering\briefempfaengerindex{Schnitzler, Arthur@\textsc{Schnitzler, Arthur}!zzzBahr, Hermann@\emph{von Hermann Bahr}!1893-08-121@{{[}12. 8. 1893{]}}|)be}\mylabel{h}  \normalsize

\doendnotes{C}
\bigskip
\vfill

\clearpage

\footnotesize

\lohead{\textsc{register}}

% Definiere theindex-Environment komplett neu ohne reledmac
\makeatletter
\renewenvironment{theindex}{%
  \section*{\indexname}%
  \setlength{\parindent}{0pt}%
  \setlength{\parskip}{0pt plus 0.3pt}%
  \let\item\@idxitem
}{%
  \clearpage
}
\makeatother

\IfFileExists{\jobname-pw.ind}{\input{\jobname-pw.ind}}{}

\end{document}

      