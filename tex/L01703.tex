%% latex-korrekturansicht-vorspann.tex
%% Vorspann für die Korrekturansicht.
%% Lädt die gemeinsame Datei latex-vorspann.tex mit gesetztem Schalter.

\newif\ifkorrekturansicht
\korrekturansichttrue

\input{../tex-inputs/latex-vorspann}


               \section[Richard Beer-Hofmann an Arthur Schnitzler, 29. 8. 1907]{ Richard Beer-Hofmann an Arthur Schnitzler, 29. 8. 1907}\nopagebreak\mylabel{v}\rehead{ }\normalsize\beginnumbering\briefempfaengerindex{Schnitzler, Arthur@\textsc{Schnitzler, Arthur}!zzzBeer-Hofmann, Richard@\emph{von Richard Beer-Hofmann}!1907-08-291@{29. 8. 1907}|(be} \toendnotes[C]{\smallbreak\pagebreak[2]} \Standort{CUL, Schnitzler, B 8.}
\physDesc{Brief, 1 Blatt, 2 Seiten
\newline{}Handschrift: Bleistift, lateinische Kurrent\newline{}Ordnung: mit Bleistift von unbekannter Hand nummeriert: »212« }\buchAbdrucke{\weitereDrucke{Arthur Schnitzler, Richard Beer-Hofmann: \emph{Briefwechsel 1891–1931}. Hg. Konstanze Fliedl. Wien, Zürich: \emph{Europaverlag} 1992, S. 184.} }\toendnotes[C]{\smallbreak}\pstart
           \raggedleft{}{\pb}\textcolor{pink}{Velden}{}\ledrightnote{\textcolor{pink}{Velden}}{ }29/VIII 07\pend
           \pstart
           Lieber Arthur! Wir haben überlegt: Es wäre mit drei \textcolor{blue}{Kindern}{}\ledrightnote{→\textcolor{blue}{Naëmah Beer-Hofmann}{\newline}→\textcolor{blue}{Mirjam Beer-Hofmann}{\newline}→\textcolor{blue}{Gabriel Beer-Hofmann}} u. der \textcolor{blue}{Christine}{}\ledrightnote{\textcolor{blue}{Christine}} – (6 in einem Wagen) nicht schön 4 Tage
               im Wagen bis \textcolor{pink}{Bozen}{}\ledrightnote{\textcolor{pink}{Bozen}} zu fahren. Auch für das täglich
               Aus und Einpacken – täglich wo anders übernachten – sind bessere Nerven nötig, als
                  \textcolor{blue}{Paula}{}\ledrightnote{\textcolor{blue}{Paula Beer-Hofmann}} augenblicklich hat. Sie hat nur den
               Wunsch viel zu schlafen, ruhig zu sitzen, und in sehr heisser Sonne zu braten. So
               drängt Alles nach dem \textcolor{pink}{Lido}{}\ledrightnote{\textcolor{pink}{Lido}}, den wir in nicht ganz
               sieben Stunden von hier, erreichen können.\pend
           \pstart
           {\pb}Ich reise also Samstag hier ab –
               bin es – wenn Sie dies lesen hoffentlich schon – übernachte in \textcolor{pink}{Villach}{}\ledrightnote{\textcolor{pink}{Villach}} und fahre Sonntag Früh nach \textcolor{pink}{Venedig}{}\ledrightnote{\textcolor{pink}{Venedig}}, – vorläufig \textcolor{pink}{Bauer-Grünwald}{}\ledrightnote{\textcolor{pink}{Grand Hotel Bauer-Grünwald}}, bis
               wir Zi{\geminationm}er auf dem \textcolor{pink}{Lido}{}\ledrightnote{\textcolor{pink}{Lido}} beko{\geminationm}en. So werde ich Sie erst wieder in \textcolor{pink}{Wien}{}\ledrightnote{\textcolor{pink}{Wien}} sehen, ausser Sie wählen den Rückweg über \textcolor{pink}{Venedig}{}\ledrightnote{\textcolor{pink}{Venedig}} – was ja auch einiges für sich hätte. Im
                  Herbst erhoffe ich mir \strikeout{so} ein paar
               schöne Tage mit Spaziergängen mit Ihnen – hier folgt eine Schilderung \textcolor{pink}{Wien}{}\ledrightnote{\textcolor{pink}{Wien}}s im Herbst – von Ihnen besser besorgt als von mir. Von
               Herzen\pend
           \pstart Ihr \spacefill\mbox{Richard}\pend{}\pstart
           Grüsse an Frau \textcolor{blue}{Olga}{}\ledrightnote{\textcolor{blue}{Olga Schnitzler}} von \textcolor{blue}{Paula}{}\ledrightnote{\textcolor{blue}{Paula Beer-Hofmann}} u mir\pend
           \endnumbering\briefempfaengerindex{Schnitzler, Arthur@\textsc{Schnitzler, Arthur}!zzzBeer-Hofmann, Richard@\emph{von Richard Beer-Hofmann}!1907-08-291@{29. 8. 1907}|)be}\mylabel{h}  \normalsize

\doendnotes{C}
\bigskip
\vfill

\clearpage

\footnotesize

\lohead{\textsc{register}}

% Definiere theindex-Environment komplett neu ohne reledmac
\makeatletter
\renewenvironment{theindex}{%
  \section*{\indexname}%
  \setlength{\parindent}{0pt}%
  \setlength{\parskip}{0pt plus 0.3pt}%
  \let\item\@idxitem
}{%
  \clearpage
}
\makeatother

\IfFileExists{\jobname-pw.ind}{\input{\jobname-pw.ind}}{}

\end{document}

      