%% latex-korrekturansicht-vorspann.tex
%% Vorspann für die Korrekturansicht.
%% Lädt die gemeinsame Datei latex-vorspann.tex mit gesetztem Schalter.

\newif\ifkorrekturansicht
\korrekturansichttrue

\input{../tex-inputs/latex-vorspann}


               \section[Richard Beer-Hofmann an Arthur Schnitzler, {[}9. 5. 1895{]}]{ Richard Beer-Hofmann an Arthur Schnitzler, {[}9. 5. 1895{]}}\nopagebreak\mylabel{v}\rehead{ }\normalsize\beginnumbering\briefempfaengerindex{Schnitzler, Arthur@\textsc{Schnitzler, Arthur}!zzzBeer-Hofmann, Richard@\emph{von Richard Beer-Hofmann}!1895-05-091@{{[}9. 5. 1895{]}}|(be} \toendnotes[C]{\smallbreak\pagebreak[2]} \Standort{CUL, Schnitzler, B 8.}
\physDesc{Brief, 1 Blatt (Auf der Rückseite von Beer-Hofmann mit Bleistift: »DrJosef HeidenthallerWohnung Johannesgasse 3. u. 5Berlin! Dr. F. C. Andreas.vor. 10 Tagen.«), 1 Seite
\newline{}Handschrift: Bleistift, lateinische Kurrent
\newline{}Schnitzler: mit Bleistift datiert: »\substVorne{}\textsuperscript{2}\substDazwischen{}9\substHinten{} /5 95« und nummeriert: »58« }\toendnotes[C]{\smallbreak}\pstart
           \noindent{}{\pb}\textcolor{blue}{Wir}{}\ledrightnote{→\textcolor{blue}{Lou Andreas-Salomé}} sind die Strasse längs des
               Hauses (\textcolor{pink}{Stelzer}{}\ledrightnote{\textcolor{pink}{Gasthaus Stelzer}}) (\textcolor{pink}{Badgasse}{}\ledrightnote{\textcolor{pink}{Badgasse}}) geradeaus in den Wald gegangen und halten uns i{\geminationm}er an der Mauer des \textcolor{pink}{Kalksburger Convicts}{}\ledrightnote{\textcolor{pink}{Kollegium Kalksburg}} –\pend
           \pstart \spacefill\mbox{Richard}\pend{}\pstart
           \noindent{}Herrn D\textsuperscript{r} Arthur Schnitzler\pend
           \endnumbering\briefempfaengerindex{Schnitzler, Arthur@\textsc{Schnitzler, Arthur}!zzzBeer-Hofmann, Richard@\emph{von Richard Beer-Hofmann}!1895-05-091@{{[}9. 5. 1895{]}}|)be}\mylabel{h}  \normalsize

\doendnotes{C}
\bigskip
\vfill

\clearpage

\footnotesize

\lohead{\textsc{register}}

% Definiere theindex-Environment komplett neu ohne reledmac
\makeatletter
\renewenvironment{theindex}{%
  \section*{\indexname}%
  \setlength{\parindent}{0pt}%
  \setlength{\parskip}{0pt plus 0.3pt}%
  \let\item\@idxitem
}{%
  \clearpage
}
\makeatother

\IfFileExists{\jobname-pw.ind}{\input{\jobname-pw.ind}}{}

\end{document}

      