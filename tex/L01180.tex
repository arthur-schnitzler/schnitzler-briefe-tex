%% latex-korrekturansicht-vorspann.tex
%% Vorspann für die Korrekturansicht.
%% Lädt die gemeinsame Datei latex-vorspann.tex mit gesetztem Schalter.

\newif\ifkorrekturansicht
\korrekturansichttrue

\input{../tex-inputs/latex-vorspann}


               \section[Hermann Bahr an Arthur Schnitzler, 7. 10. {[}1901{]}]{ Hermann Bahr an Arthur Schnitzler, 7. 10. {[}1901{]}}\nopagebreak\mylabel{v}\rehead{ }\normalsize\beginnumbering\briefempfaengerindex{Schnitzler, Arthur@\textsc{Schnitzler, Arthur}!zzzBahr, Hermann@\emph{von Hermann Bahr}!1901-10-071@{7. 10. 1901}|(be} \toendnotes[C]{\smallbreak\pagebreak[2]} \Standort{CUL, Schnitzler, B 5b.}
\physDesc{Brief, 1 Blatt, 2 Seiten
\newline{}Handschrift: blaue Tinte, deutsche Kurrent
\newline{}Schnitzler: mit Bleistift die Jahreszahl »901« ergänzt \newline{}Ordnung: mit Bleistift von unbekannter Hand nummeriert:
                                    »80« }\buchAbdrucke{\weitereDrucke{Hermann Bahr, Arthur Schnitzler: \emph{Briefwechsel, Aufzeichnungen, Dokumente (1891–1931)}. Hg. Kurt Ifkovits und Martin Anton Müller. Göttingen: \emph{Wallstein} 2018, S. 215.} }\toendnotes[C]{\smallbreak}\pstart
           \raggedleft{}{\pb}7. October\pend
           \pstart\center{}Lieber Arthur!\pend\pstart
           Morgen kann ich leider nicht, da ich zur \label{K_L01180_1v}\edtext{Probe des »\textcolor{green}{neuen Simſon}{}\ledrightnote{\textcolor{green}{Der neue Simson}}«}{\lemma{\textnormal{\emph{Probe des »neuen Simſon«}}}\Cendnote{\textnormal{Das \textcolor{green}{Schauspiel} von \textcolor{blue}{C.
                     Karlweis} hatte die Uraufführung am 19. 10. 1901 im \textcolor{pink}{Deutschen Volkstheater}.}}}\label{K_L01180_1h} muß, und eben um dieſes
               Stückes willen kann ich auch über die nächſten Tage nicht wohl disponieren. Dagegen
               bin ich \label{K_L01180_2v}\edtext{Samſtag, Sonntag, Montag}{\lemma{\textnormal{\emph{Samſtag, Sonntag, Montag}}}\Cendnote{\textnormal{14., 15., 16. 10. 1901.}}}\label{K_L01180_2h} immer mit
               Vergnügen bereit und, wenn es ſchön iſt, könnte ich Dir dann auch die {\pb}hübſchen kleinen Spazierwege zeigen, die es hier
               gibt. Übrigens hoffe ich Dich noch vorher zu ſehen, da ich in den nächſten Tagen
               einmal zu Dir springen will.\pend
           \pstart
           Herzlichſt{\\[\baselineskip]}Dein{\\[\baselineskip]}\spacefill\mbox{Hermann}\pend
           \leftskip=0em{}\endnumbering\briefempfaengerindex{Schnitzler, Arthur@\textsc{Schnitzler, Arthur}!zzzBahr, Hermann@\emph{von Hermann Bahr}!1901-10-071@{7. 10. 1901}|)be}\mylabel{h}  \normalsize

\doendnotes{C}
\bigskip
\vfill

\clearpage

\footnotesize

\lohead{\textsc{register}}

% Definiere theindex-Environment komplett neu ohne reledmac
\makeatletter
\renewenvironment{theindex}{%
  \section*{\indexname}%
  \setlength{\parindent}{0pt}%
  \setlength{\parskip}{0pt plus 0.3pt}%
  \let\item\@idxitem
}{%
  \clearpage
}
\makeatother

\IfFileExists{\jobname-pw.ind}{\input{\jobname-pw.ind}}{}

\end{document}

      