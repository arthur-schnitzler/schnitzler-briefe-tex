%% latex-korrekturansicht-vorspann.tex
%% Vorspann für die Korrekturansicht.
%% Lädt die gemeinsame Datei latex-vorspann.tex mit gesetztem Schalter.

\newif\ifkorrekturansicht
\korrekturansichttrue

\input{../tex-inputs/latex-vorspann}


               \section[Arthur Schnitzler an Therese Rie-Andro, 12. 2. 1912]{ Arthur Schnitzler an Therese Rie-Andro, 12. 2. 1912}\nopagebreak\mylabel{v}\rehead{ }\normalsize\beginnumbering\briefempfaengerindex{Rie, Therese@\textsc{Rie, Therese}!zzzSchnitzler, Arthur@\emph{von Arthur Schnitzler}!1912-02-122@{12. 2. 1912}|(be} \toendnotes[C]{\smallbreak\pagebreak[2]} \Standort{DLA, A:Schnitzler, HS1985.1.253.}
\physDesc{Brief, 1 Blatt, 2 Seiten, maschineller Durchschlag
\newline{}Schreibmaschine
\newline{}Handschrift Arthur Schnitzler: roter Buntstift, lateinische Kurrent (\noindent{}Beschriftung mit »Andro« in der linken,
                                            mit »Ri« in rechten oberen Ecke. Oberhalb
                                            von »musikalische Legende« der Name des
                                            Werks: »(\textcolor{green}{Palestrina})« und zwei
                                            Unterstreichungen)\newline{}Handschrift  : roter Buntstift, lateinische Kurrent (\noindent{}in der rechten oberen Ecke Vermerk, dass es sich um einen
                                            Durchschlag (Kopie) handelt: »K«)}\buchAbdrucke{\weitereDrucke{Arthur Schnitzler: \emph{Briefe 1875–1912}. Hg. Therese Nickl und Heinrich Schnitzler. Frankfurt am Main: \emph{S. Fischer} 1981, S. 690–691.} }\toendnotes[C]{\smallbreak}\pstart
           \centering{}{\pb}12. \strikeout{1}2. 1912.\pend
           \pstart\center{}Sehr verehrte Frau.\pend\pstart
           Die \textcolor{green}{musikalische Legende}{}\ledrightnote{→\textcolor{green}{Palestrina. Musikalische Legende in drei Akten}} von
                        \textcolor{blue}{Hans Pfitzner}{}\ledrightnote{\textcolor{blue}{Hans Pfitzner}} habe ich mit grösstem
                    Interesse gelesen; als Grundlage für musikalische Bearbeitung scheint mir das
                        \textcolor{green}{Buch}{}\ledrightnote{→\textcolor{green}{Palestrina. Musikalische Legende in drei Akten}} sehr glücklich
                    entworfen, aber \strikeout{auch} dichterische und
                    theatralische Qualitäten selbständiger Art würden für Einfall und Durchführung
                    auch bei solchen Lesern Anteilnahme werben, die nicht, wie es mir begegnet ist,
                    schon während der Lektüre immerfort Musik mitklingen hörten, leider noch nicht
                    die von \textcolor{blue}{Pfitzner}{}\ledrightnote{\textcolor{blue}{Hans Pfitzner}}, der ich mich diesmal ganz
                    besonders entgegenfreue. Vielleicht gebricht es dem \textcolor{green}{zweiten Akt}{}\ledrightnote{→\textcolor{green}{Palestrina. Musikalische Legende in drei Akten}} ein wenig an innerer Klarheit, doch denke
                    ich mir wird die Musik hier manches zu entwirren imstande sein, was die
                    Knappheit des Textes allzu dicht verknotet hat. Eine Kleinigkeit noch. Im
                    letzten \textcolor{green}{Akt}{}\ledrightnote{→\textcolor{green}{Palestrina. Musikalische Legende in drei Akten}} sollten die
                    Leute auf der Strasse nicht »\label{K_L02574-1v}\edtext{Eviva}{\lemma{\textnormal{\emph{Eviva}}}\Cendnote{\textnormal{Das monierte Detail
                        wurde von \textcolor{blue}{Pfitzner} nicht
                    geändert.}}}\label{K_L02574-1h}!« rufen; man muss ja annehmen, dass das Ganze aus dem \textcolor{pink}{Italienischen}{}\ledrightnote{\textcolor{pink}{Italien}} ins Deutsche über{\pb}tragen ist und so wirkt es etwas unlogisch, dass gerade
                    dieses eine populäre Wort \textcolor{pink}{italienisch}{}\ledrightnote{\textcolor{pink}{Italien}} stehen
                    geblieben ist.\pend
           \pstart
           Bitte, verehrte Frau, \textcolor{blue}{Hans Pfitzner}{}\ledrightnote{\textcolor{blue}{Hans Pfitzner}} in meinem
                    Namen für sein Vertrauen aufs Herzlichste zu danken{[}.{]} Ich
                    hoffe es bald persönlich tun zu können, da er ja im Frühjahr nach \textcolor{pink}{Wien}{}\ledrightnote{\textcolor{pink}{Wien}} kommen dürfte. Von Ihnen hoffe ich bald
                    wieder etwas zu lesen; ich irre mich ja nicht, wenn ich Sie mit der Verfasserin
                    eines Novellenbuches (hiess es nicht die »\textcolor{green}{Augen des
                            Hy\strikeout{e}ronimus}{}\ledrightnote{\textcolor{green}{Die Augen des Hieronymus}}«) identifiziere, das ich vor
                    einer Reihe von Jahren mit Vergnügen kennen gelernt habe.\pend
           \pstart Mit verbildlichem Gruss\pend{}{\bigskip}\pstart
           \noindent{}Frau L. Andro, \textcolor{pink}{Wien}{}\ledrightnote{\textcolor{pink}{Wien}}.\pend
           \endnumbering\briefempfaengerindex{Rie, Therese@\textsc{Rie, Therese}!zzzSchnitzler, Arthur@\emph{von Arthur Schnitzler}!1912-02-122@{12. 2. 1912}|)be}\mylabel{h}  \normalsize

\doendnotes{C}
\bigskip
\vfill

\clearpage

\footnotesize

\lohead{\textsc{register}}

% Definiere theindex-Environment komplett neu ohne reledmac
\makeatletter
\renewenvironment{theindex}{%
  \section*{\indexname}%
  \setlength{\parindent}{0pt}%
  \setlength{\parskip}{0pt plus 0.3pt}%
  \let\item\@idxitem
}{%
  \clearpage
}
\makeatother

\IfFileExists{\jobname-pw.ind}{\input{\jobname-pw.ind}}{}

\end{document}

      