%% latex-korrekturansicht-vorspann.tex
%% Vorspann für die Korrekturansicht.
%% Lädt die gemeinsame Datei latex-vorspann.tex mit gesetztem Schalter.

\newif\ifkorrekturansicht
\korrekturansichttrue

\input{../tex-inputs/latex-vorspann}


               \section[Georg Brandes an Arthur Schnitzler, 22. 1. 1899]{ Georg Brandes an Arthur Schnitzler, 22. 1. 1899}\nopagebreak\mylabel{v}\rehead{ }\normalsize\beginnumbering\briefempfaengerindex{Schnitzler, Arthur@\textsc{Schnitzler, Arthur}!zzzBrandes, Georg@\emph{von Georg Brandes}!1899-01-221@{22. 1. 1899}|(be} \toendnotes[C]{\smallbreak\pagebreak[2]} \Standort{CUL, Schnitzler, B 17.}
\physDesc{Postkarte
\newline{}Handschrift: blaue Tinte, lateinische Kurrent\newline{}Versand: 1) Stempel: »\nobreak{}\oindex{Kopenhagen@\textbf{Kopenhagen}, \emph{Besiedelter Ort (A.BSO)}|pwk}Kobenhavn, 22. 1. 99, 3–4 E\nobreak{}«.  2) Stempel: »\nobreak{}\oindex{IX., Alsergrund@\textbf{IX., Alsergrund}, \emph{Bezirk (A.BZK)}|pwk}Wien 9/3, 24. 1. 99, 8. V, Bestellt\nobreak{}«. \newline{}Ordnung: mit Bleistift von unbekannter Hand nummeriert: »13« }\buchAbdrucke{\weitereDrucke{Georg Brandes, Arthur Schnitzler: \emph{Ein Briefwechsel}. Hg. Kurt Bergel. Bern: \emph{Francke} 1956, S. 72–73.} }\toendnotes[C]{\smallbreak}\pstart{}{\pb}Herrn Dr. Arthur
                        Schnitzler\pend{}\pstart{}\textcolor{pink}{Frankgasse 1}{}\ledrightnote{\textcolor{pink}{Frankgasse}}\pend{}\pstart{}\textcolor{pink}{Wien IX}{}\ledrightnote{\textcolor{pink}{IX., Alsergrund}}\pend{}{\bigskip}\pstart
           \raggedleft{}{\pb}22 Januar 99\pend
           \pstart
           Lieber Herr Doctor! Es war ein Fehler von mir dass ich nicht
                    für die \textcolor{green}{Novellensammlung}{}\ledrightnote{→\textcolor{green}{Die Frau des Weisen. Novelletten}}
                    dankte. ich habe sie mit grosser Aufmerksamkeit gelesen. Für mich ist die \textcolor{green}{Novelle}{}\ledrightnote{→\textcolor{green}{Die Toten schweigen}} die zuerst in \textcolor{green}{Cosmopolis}{}\ledrightnote{\textcolor{green}{Cosmopolis}} stand – ich erinnere mich nicht des
                    Titels – ein \uline{Meisterwerk} erstaunlich wahr und
                    packend; nur ein (sehr kleiner) Fehler gegen den Schluss, dass die Frau zuletzt
                    alles gesteht. Als ob Frauen je geständen, wenn keine Beweise vorliegen, und
                    wenn sie keinem absolut überlegenen Mann gegenüber stehen! Ein wahres
                    Meisterwerk ist es dennoch.\pend
           \pstart
           Meine \textcolor{green}{Gedichte}{}\ledrightnote{→\textcolor{green}{Ungdomsvers [Jugendgedichte]}}! Was soll ich
                    darüber sagen. Lesen Sie \textcolor{pink}{Dänisch}{}\ledrightnote{\textcolor{pink}{Dänemark}}, so werden
                    Sie einräumen dass zwei oder drei sehr gut sind, »\textcolor{green}{Reconvalescent-Besuch}{}\ledrightnote{\textcolor{green}{Reconvalescent-Besuch}}« und »\textcolor{green}{Harald
                        Haarfager in Finmarken}{}\ledrightnote{\textcolor{green}{Harald Haarfager in Finmarken}}«. Es ist eine Art Jugend-Tagebuch. – Ich liege
                    noch immer zu Bett, schon 5 Wochen, Sie wissen ja was Venenentzündung ist. Doch
                    ist es diesmal anscheinend nicht schlimm. Beste Grüsse \spacefill\mbox{G. B.}\pend
           \pstart
           \noindent{}\label{T_L00882_1v}\edtext{Sie haben wohl meinen \textcolor{green}{Protest gegen die
                            Ausweisungen der Dänen}{}\ledrightnote{→\textcolor{green}{Köllers Erfolge}} gelesen, oder auch nicht. 100 Zeitungen
                        aller Länder haben ihn abgedruckt aber die \textcolor{brown}{Neue
                            Freie}{}\ledrightnote{\textcolor{brown}{Neue Freie Presse}} ist ja \textcolor{pink}{preussisch}{}\ledrightnote{\textcolor{pink}{Preußen}}.}{\lemma{\textnormal{\emph{Sie … preussisch.}}}\Cendnote{\textnormal{am linken Rand}}}\label{T_L00882_1h}\pend
           \endnumbering\briefempfaengerindex{Schnitzler, Arthur@\textsc{Schnitzler, Arthur}!zzzBrandes, Georg@\emph{von Georg Brandes}!1899-01-221@{22. 1. 1899}|)be}\mylabel{h}  \normalsize

\doendnotes{C}
\bigskip
\vfill

\clearpage

\footnotesize

\lohead{\textsc{register}}

% Definiere theindex-Environment komplett neu ohne reledmac
\makeatletter
\renewenvironment{theindex}{%
  \section*{\indexname}%
  \setlength{\parindent}{0pt}%
  \setlength{\parskip}{0pt plus 0.3pt}%
  \let\item\@idxitem
}{%
  \clearpage
}
\makeatother

\IfFileExists{\jobname-pw.ind}{\input{\jobname-pw.ind}}{}

\end{document}

      