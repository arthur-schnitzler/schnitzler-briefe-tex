%% latex-korrekturansicht-vorspann.tex
%% Vorspann für die Korrekturansicht.
%% Lädt die gemeinsame Datei latex-vorspann.tex mit gesetztem Schalter.

\newif\ifkorrekturansicht
\korrekturansichttrue

\input{../tex-inputs/latex-vorspann}


               \section[Arthur Schnitzler an Hugo von Hofmannsthal, 20. 8. 1904]{ Arthur Schnitzler an Hugo von Hofmannsthal, 20. 8. 1904}\nopagebreak\mylabel{v}\rehead{ }\normalsize\beginnumbering\briefempfaengerindex{Hofmannsthal, Hugo von@\textsc{Hofmannsthal, Hugo von}!zzzSchnitzler, Arthur@\emph{von Arthur Schnitzler}!1904-08-201@{20. 8. 1904}|(be} \toendnotes[C]{\smallbreak\pagebreak[2]} \Standort{FDH, Hs-30885,112.}
\physDesc{Brief, 2 Blätter, 6 Seiten
\newline{}Handschrift: schwarze Tinte, deutsche Kurrent\newline{}Ordnung: mit Bleistift von Schnitzler mutmaßlich bei der Durchsicht der
                                 Korrespondenz 1929 das zweite Blatt beschrieben:
                                    »II 20/8 904« }\buchAbdrucke{\weitereDrucke{1) Hugo von Hofmannsthal, Arthur Schnitzler: \emph{Briefwechsel}. Hg. Therese Nickl und Heinrich Schnitzler. Frankfurt am Main: \emph{S. Fischer} 1964, S. 197–199.} \weitereDrucke{2) Hermann Bahr, Arthur Schnitzler: \emph{Briefwechsel, Aufzeichnungen, Dokumente (1891–1931)}. Hg. Kurt Ifkovits und Martin Anton Müller. Göttingen: \emph{Wallstein} 2018, S. 316.} }\toendnotes[C]{\smallbreak}\pstart
           \raggedleft{}{\pb}\textcolor{pink}{Wien}{}\ledrightnote{\textcolor{pink}{Wien}}{ }20. 8. 904\pend
           \pstart
           lieber Hugo, mit der \textcolor{pink}{Salzk.gut}{}\ledrightnote{\textcolor{pink}{Salzkammergut}}reiſe
               ſteht es wie folgt: in dieſen Tagen beende ich die erſte flüchtige Niederſchrift
               eines neuen dreiaktigen \textcolor{green}{Stücks}{}\ledrightnote{→\textcolor{green}{Zwischenspiel. Komödie in drei Akten}};
               die \textcolor{blue}{Grünwald}{}\ledrightnote{\textcolor{blue}{Ida Grünwald}} ko{\geminationm}t
               etwa 25., 26., und dann muſs ich es, um es überſichtlich
               vor mir zu haben, und weil das überhaupt zu den Etappen meiner Arbeitsweiſe gehört u
               mich ſehr fördert, dictiren. Nun ka{\geminationn} ich, auch weil der
               Anfangstag der \textcolor{blue}{Grünwald}{}\ledrightnote{\textcolor{blue}{Ida Grünwald}}{ }\substVorne{}\textsuperscript{\textcolor{gray}{sich}}\substDazwischen{}noch nicht feſtſteht\substHinten{} (ich bin ohne Nachricht, \textsc{resp} Antwort von ihr),
               nicht {\pb}auf den Tag beſtimmen, wann ich fertig bin. Ich
                  \uline{hoffe}, es wird ſich fügen, daſs wir schon am
                  3.{ }\textcolor{pink}{Wien}{}\ledrightnote{\textcolor{pink}{Wien}} verlaſſen können; wird aber \textsc{\textcolor{blue}{Gerty}{}\ledrightnote{\textcolor{blue}{Gertrude von Hofmannsthal}}} auch warten, wenn der 4. oder gar der 5. September
               draus wird? Wir möchten natürlich auch ſehr gern mit ihr zuſammen fahren; ich ka{\geminationn} nur heute mich zur Beſti{\geminationm}ung des Tages nicht verpflichten. Immerhin werde ich am erſten Dictirtag ſchon
               wiſſen können, wa{\geminationn} wir bereit ſind. Ich hoffe ja ſehr,
               daſs es der 3.{ }ſein wird. Sie erſehen daraus {\pb}jedenfalls, daſs wir zu \textcolor{pink}{Iſchl}{}\ledrightnote{\textcolor{pink}{Goldenes Kreuz}} entſchloſſen iſt, wo wir fürs erſte Quartier nehmen, Ausflüge machen
                  (\textcolor{blue}{Olga}{}\ledrightnote{\textcolor{blue}{Olga Schnitzler}} kennt das \textcolor{pink}{Salzka{\geminationm}ergut}{}\ledrightnote{\textcolor{pink}{Salzkammergut}} gar nicht), und ich ſehne mich
               auch ſehr nach ein paar ſchönen Radtouren mit Ihnen. Auch zu einer Fußpartie
               (Ruckſack!) wär ich zu haben. Nicht unmöglich iſt es, daſs ich da{\geminationn} auch noch mit \textcolor{blue}{Olga}{}\ledrightnote{\textcolor{blue}{Olga Schnitzler}}
               weiterfahre, \textcolor{pink}{Tirol}{}\ledrightnote{\textcolor{pink}{Tirol}}, \textcolor{pink}{Bozner}{}\ledrightnote{\textcolor{pink}{Bozen}} Gegend, und falls das Wetter allzu herbſtlich wird, \textcolor{pink}{München}{}\ledrightnote{\textcolor{pink}{München}}. Wir ſehen uns ja jedenfalls ſchon am ersten {\pb}\textcolor{pink}{Iſchl}{}\ledrightnote{\textcolor{pink}{Bad Ischl}}er Tag, aber ſagen Sie mir doch gleich, wa{\geminationn}{ }Sie wieder in \textcolor{pink}{Rodaun}{}\ledrightnote{\textcolor{pink}{Rodaun}} zurück ſein müſſen oder wollen. Wohnen wollen wir in der \textcolor{pink}{Kaiſerkrone}{}\ledrightnote{\textcolor{pink}{Hotel Kaiserkrone}}. –\pend
           \pstart
           Sind Sie mit dem »\textcolor{green}{geretteten}{}\ledrightnote{\textcolor{green}{Das gerettete Venedig. Trauerspiel in fünf Aufzügen}}« fertig? Mir geht es
               mit dem Arbeiten nicht übel und ginge mir gewiſs noch beſſer, we{\geminationn} nicht mein Widerwillen gegen den phyſ. Akt des
               Schreibens immer beträchtlicher würde und ſich oft genug in leichten Schreibkrämpfen
               äußerte.\pend
           \pstart
           Danke ſehr betreffs \textcolor{blue}{V. S.}{}\ledrightnote{\textcolor{blue}{Robert Gilbert Vansittart}}, mein Aerger hat ſich
               natürlich ſchon gelegt – natürlich würde es mich aber {\pb}ſehr freuen, wenn Ordnung in die ganze Angelegenheit gebracht werden könnte und ich
               von \textcolor{pink}{England}{}\ledrightnote{\textcolor{pink}{England}}, \textcolor{pink}{Irland}{}\ledrightnote{\textcolor{pink}{Irland}}
               u \textcolor{pink}{Schottland}{}\ledrightnote{\textcolor{pink}{Schottland}} nicht länger misverſtanden \introOben{}verfolgt u geächtet\introOben{} würde. –\pend
           \pstart
           – \textsc{\textcolor{green}{Vehse}{}\ledrightnote{→\textcolor{green}{Geschichte der deutschen Höfe seit der Reformation}}} iſt und bleibt ein koſtbares Buch. Zudem studier ich, des Überblickes halber,
               Geſchichte \introOben{}wie\introOben{} zur Matura. Ich wäre weiter als ich bin, we{\geminationn} ich ein gebildeter Menſch wäre!\pend
           \pstart
           Was iſts mit \textcolor{blue}{Richard}{}\ledrightnote{\textcolor{blue}{Richard Beer-Hofmann}}? Seine Karte mit \textcolor{blue}{Paula}{}\ledrightnote{\textcolor{blue}{Paula Beer-Hofmann}}{ }\textcolor{gray}{wie} den \textcolor{gray}{\textcolor{blue}{Kindern}{}\ledrightnote{→\textcolor{blue}{Naëmah Beer-Hofmann}{\newline}→\textcolor{blue}{Mirjam Beer-Hofmann}{\newline}→\textcolor{blue}{Gabriel Beer-Hofmann}}}{ }\textcolor{gray}{an}{ }\textcolor{gray}{×}\-\textcolor{gray}{×}\-\textcolor{gray}{×} hab ich bekommen. Von ſich
               ſchreibt er nichts. Grüßen Sie alle, die mir lieb ſind.\pend
           \pstart Herzlichſt Ihr\spacefill\mbox{A.{\pb}}\pend{}\pstart
           \noindent{}{\pb}\textsc{\textcolor{blue}{Gerty}{}\ledrightnote{\textcolor{blue}{Gertrude von Hofmannsthal}}} wird wohl auch am liebſten mit dem Zehn Uhr Früh Zug fahren?{\\}A.\pend
           \pstart
           Geſtern Abend waren wir mit \textcolor{blue}{Bahr}{}\ledrightnote{\textcolor{blue}{Hermann Bahr}}, (\textcolor{pink}{Hietzing}{}\ledrightnote{\textcolor{pink}{XIII., Hietzing}}) dem’s recht gut, und was das
                  weſentlichſte iſt, hoffnungsvoll zu gehen ſcheint.\pend
           \endnumbering\briefempfaengerindex{Hofmannsthal, Hugo von@\textsc{Hofmannsthal, Hugo von}!zzzSchnitzler, Arthur@\emph{von Arthur Schnitzler}!1904-08-201@{20. 8. 1904}|)be}\mylabel{h}  \normalsize

\doendnotes{C}
\bigskip
\vfill

\clearpage

\footnotesize

\lohead{\textsc{register}}

% Definiere theindex-Environment komplett neu ohne reledmac
\makeatletter
\renewenvironment{theindex}{%
  \section*{\indexname}%
  \setlength{\parindent}{0pt}%
  \setlength{\parskip}{0pt plus 0.3pt}%
  \let\item\@idxitem
}{%
  \clearpage
}
\makeatother

\IfFileExists{\jobname-pw.ind}{\input{\jobname-pw.ind}}{}

\end{document}

      