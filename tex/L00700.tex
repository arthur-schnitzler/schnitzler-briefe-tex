%% latex-korrekturansicht-vorspann.tex
%% Vorspann für die Korrekturansicht.
%% Lädt die gemeinsame Datei latex-vorspann.tex mit gesetztem Schalter.

\newif\ifkorrekturansicht
\korrekturansichttrue

\input{../tex-inputs/latex-vorspann}


               \section[Hugo von Hofmannsthal an Arthur Schnitzler, {[}12.? 7. 1897{]}]{ Hugo von Hofmannsthal an Arthur Schnitzler, {[}12.? 7. 1897{]}}\nopagebreak\mylabel{v}\rehead{ }\normalsize\beginnumbering\briefempfaengerindex{Schnitzler, Arthur@\textsc{Schnitzler, Arthur}!zzzHofmannsthal, Hugo von@\emph{von Hugo von Hofmannsthal}!1897-07-121@{{[}12.? 7. 1897{]}}|(be} \toendnotes[C]{\smallbreak\pagebreak[2]} \Standort{CUL, Schnitzler, B 43.}
\physDesc{Brief, 1 Blatt, 2 Seiten
\newline{}Handschrift: schwarze Tinte, deutsche Kurrent
\newline{}Schnitzler: mit Bleistift datiert: »Mitte Juli 97« \newline{}Ordnung: 1) mit Bleistift von unbekannter Hand nummeriert: »\strikeout{96}« 2) mit Bleistift von unbekannter Hand nummeriert: »95«}\buchAbdrucke{\weitereDrucke{Hugo von Hofmannsthal, Arthur Schnitzler: \emph{Briefwechsel}. Hg. Therese Nickl und Heinrich Schnitzler. Frankfurt am Main: \emph{S. Fischer} 1964, S. 91.} }\pstart{}{\pb}mein lieber
                        Arthur\pend\pstart
           herzlichen Dank für den Brief. \textcolor{blue}{\textsc{Poldy}}{}\ledrightnote{\textcolor{blue}{Leopold von Andrian-Werburg}}, dem es fortgeſetzt ſehr ſchlecht geht, kommt auf
                        \textcolor{blue}{Widerhofer}{}\ledrightnote{\textcolor{blue}{Hermann Widerhofer}}s dringenden Rat hieher zu
                    mir. Ich muſs daher natürlich, um ihm Zeit zur Erholung zu geben, meinen
                    Aufenthalt hier um mindeſt 10–12 Tage verlängern. Bitte gleich Antwort ob \introOben{}für\introOben{}
               Sie und \textcolor{blue}{Richard}{}\ledrightnote{\textcolor{blue}{Richard Beer-Hofmann}}
                    das \textcolor{pink}{Salzburg}{}\ledrightnote{\textcolor{pink}{Salzburg}}er {\pb}\textsc{Rendezvous} in den erſten Tagen des Auguſt{ }ſich eintheilen läſst.\pend
           \pstart
           Herzlich Ihr{\\[\baselineskip]}\spacefill\mbox{Hugo.}\pend
           \leftskip=0em{}\pstart
           \textcolor{pink}{Bad Fusch}{}\ledrightnote{\textcolor{pink}{Bad Fusch}}, Montag.\pend
           \endnumbering\briefempfaengerindex{Schnitzler, Arthur@\textsc{Schnitzler, Arthur}!zzzHofmannsthal, Hugo von@\emph{von Hugo von Hofmannsthal}!1897-07-121@{{[}12.? 7. 1897{]}}|)be}\mylabel{h}  \normalsize

\doendnotes{C}
\bigskip
\vfill

\clearpage

\footnotesize

\lohead{\textsc{register}}

% Definiere theindex-Environment komplett neu ohne reledmac
\makeatletter
\renewenvironment{theindex}{%
  \section*{\indexname}%
  \setlength{\parindent}{0pt}%
  \setlength{\parskip}{0pt plus 0.3pt}%
  \let\item\@idxitem
}{%
  \clearpage
}
\makeatother

\IfFileExists{\jobname-pw.ind}{\input{\jobname-pw.ind}}{}

\end{document}

      