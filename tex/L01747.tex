%% latex-korrekturansicht-vorspann.tex
%% Vorspann für die Korrekturansicht.
%% Lädt die gemeinsame Datei latex-vorspann.tex mit gesetztem Schalter.

\newif\ifkorrekturansicht
\korrekturansichttrue

\input{../tex-inputs/latex-vorspann}


               \section[Max Burckhard an Arthur Schnitzler, 31. 12. 1907]{ Max Burckhard an Arthur Schnitzler, 31. 12. 1907}\nopagebreak\mylabel{v}\rehead{ }\normalsize\beginnumbering\briefempfaengerindex{Schnitzler, Arthur@\textsc{Schnitzler, Arthur}!zzzBurckhard, Max Eugen@\emph{von Max Eugen Burckhard}!1907-12-311@{31. 12. 1907}|(be} \toendnotes[C]{\smallbreak\pagebreak[2]} \Standort{CUL, Schnitzler, B 20.}
\physDesc{Brief, 1 Blatt, 2 Seiten
\newline{}Handschrift: schwarze Tinte, deutsche Kurrent\newline{}Ordnung: mit Bleistift von unbekannter Hand nummeriert: »21« }\toendnotes[C]{\smallbreak}\pstart
           \noindent{}{\pb}\textcolor{gray}{\textbf{D\textsuperscript{r.} Max Burckhard}}\hfill \textcolor{gray}{\textbf{\textcolor{pink}{Wien, IX. Porzellangasse 48}{}\ledrightnote{\textcolor{pink}{Porzellangasse}}}}{ }31. 12. 07\pend
           \pstart
           \raggedleft{}\textcolor{gray}{\textbf{\strikeout{St. Gilgen}}}\hspace*{3.5em}\pend
           \pstart{}Lieber, verehrter Herr Doctor!\pend\pstart
           Ich kann Ihnen gar nicht ſagen wie ſehr mich die Nachricht von der Erkrankung
                    Ihrer Frau \textcolor{blue}{Gemahlin}{}\ledrightnote{→\textcolor{blue}{Olga Schnitzler}} betrübt
                    und erſchreckt hat, und ich freue mich nur von ganzem Herzen zu hören, daſs ſie
                    ſich ſchon auf dem Wege der Geneſung befindet. Das ſind wohl jetzt ſchwere
                    Zeiten {\pb}für Sie gewesen. Mögen um ſo
                    beſſere und frohere nun kommen.\pend
           \pstart
           Ich danke Ihnen vielmals für Ihren lieben Brief und wünſche Ihnen und der
                    gnädigen \textcolor{blue}{Frau}{}\ledrightnote{→\textcolor{blue}{Olga Schnitzler}} von ganzem
                    Herzen das allerbeſte für das neue Jahr und für immerdar.\pend
           \pstart
           Ihr getreuer{\\[\baselineskip]}\spacefill\mbox{D\textsuperscript{r}Burckhard}\pend
           \leftskip=0em{}\endnumbering\briefempfaengerindex{Schnitzler, Arthur@\textsc{Schnitzler, Arthur}!zzzBurckhard, Max Eugen@\emph{von Max Eugen Burckhard}!1907-12-311@{31. 12. 1907}|)be}\mylabel{h}  \normalsize

\doendnotes{C}
\bigskip
\vfill

\clearpage

\footnotesize

\lohead{\textsc{register}}

% Definiere theindex-Environment komplett neu ohne reledmac
\makeatletter
\renewenvironment{theindex}{%
  \section*{\indexname}%
  \setlength{\parindent}{0pt}%
  \setlength{\parskip}{0pt plus 0.3pt}%
  \let\item\@idxitem
}{%
  \clearpage
}
\makeatother

\IfFileExists{\jobname-pw.ind}{\input{\jobname-pw.ind}}{}

\end{document}

      