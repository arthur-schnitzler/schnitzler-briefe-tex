%% latex-korrekturansicht-vorspann.tex
%% Vorspann für die Korrekturansicht.
%% Lädt die gemeinsame Datei latex-vorspann.tex mit gesetztem Schalter.

\newif\ifkorrekturansicht
\korrekturansichttrue

\input{../tex-inputs/latex-vorspann}


               \section[Hugo von Hofmannsthal an Olga Schnitzler, 26. 12. 1909]{ Hugo von Hofmannsthal an Olga Schnitzler, 26. 12. 1909}\nopagebreak\mylabel{v}\rehead{ }\normalsize\beginnumbering\briefempfaengerindex{Schnitzler, Olga@\textsc{Schnitzler, Olga}!zzzHofmannsthal, Hugo von@\emph{von Hugo von Hofmannsthal}!1909-12-261@{26. 12. 1909}|(be} \toendnotes[C]{\smallbreak\pagebreak[2]} \Standort{CUL, Schnitzler, B 43.}
\physDesc{Brief, 1 Blatt, 1 Seite
\newline{}Handschrift: schwarze Tinte, lateinische Kurrent\newline{}Ordnung: 1) mit Bleistift von unbekannter Hand nummeriert: »\strikeout{306}« 2) mit Bleistift von unbekannter Hand nummeriert: »313«}\buchAbdrucke{\weitereDrucke{Hugo von Hofmannsthal, Arthur Schnitzler: \emph{Briefwechsel}. Hg. Therese Nickl und Heinrich Schnitzler. Frankfurt am Main: \emph{S. Fischer} 1964, S. 380–381.} }\toendnotes[C]{\smallbreak}\stanza{}{\pb}Seit Olga uns ein \textcolor{blue}{Zweites}{}\ledrightnote{→\textcolor{blue}{Lili Schnitzler}} bracht\newverse{}Wird sie noch doppelt hochgeacht\newverse{}und gar \uline{noch}{ }\uline{schöner} sie zu machen\newverse{}schenkt man ihr nette \label{K_L01910_1v}\edtext{Siebensachen}{\lemma{\textnormal{\emph{Siebensachen}}}\Cendnote{\textnormal{Sie bekam ein
                     Medaillon aus dem Atelier der \emph{\textcolor{brown}{Wiener
                        Werkstätten}} geschenkt.}}}\label{K_L01910_1h}.\newverse{}Worauf sie fröhlich sich bespiegelt\newverse{}und seufzt: Ach ist der Hugo frech!\newverse{}{\dotsfour}\newverse{}Das Schächtelchen ist nicht –  –»\textcolor{green}{versiegelt}{}\ledrightnote{→\textcolor{green}{Versiegelt. Komische Oper}}«\newverse{}und was darin ist – nicht von \textcolor{blue}{Blech}{}\ledrightnote{\textcolor{blue}{Leo Blech}}.\stanzaend{}\pstart
           An Olga.\hspace*{1.5em}26. XII. 1909.\pend
           \endnumbering\briefempfaengerindex{Schnitzler, Olga@\textsc{Schnitzler, Olga}!zzzHofmannsthal, Hugo von@\emph{von Hugo von Hofmannsthal}!1909-12-261@{26. 12. 1909}|)be}\mylabel{h}  \normalsize

\doendnotes{C}
\bigskip
\vfill

\clearpage

\footnotesize

\lohead{\textsc{register}}

% Definiere theindex-Environment komplett neu ohne reledmac
\makeatletter
\renewenvironment{theindex}{%
  \section*{\indexname}%
  \setlength{\parindent}{0pt}%
  \setlength{\parskip}{0pt plus 0.3pt}%
  \let\item\@idxitem
}{%
  \clearpage
}
\makeatother

\IfFileExists{\jobname-pw.ind}{\input{\jobname-pw.ind}}{}

\end{document}

      