%% latex-korrekturansicht-vorspann.tex
%% Vorspann für die Korrekturansicht.
%% Lädt die gemeinsame Datei latex-vorspann.tex mit gesetztem Schalter.

\newif\ifkorrekturansicht
\korrekturansichttrue

\input{../tex-inputs/latex-vorspann}


               \section[Arthur Schnitzler an Stefan Großmann, 7. 11. 1927]{ Arthur Schnitzler an Stefan Großmann, 7. 11. 1927}\nopagebreak\mylabel{v}\rehead{ }\normalsize\beginnumbering\briefempfaengerindex{Grossmann, Stefan@\textsc{Großmann, Stefan}!zzzSchnitzler, Arthur@\emph{von Arthur Schnitzler}!1927-11-071@{7. 11. 1927}|(be} \toendnotes[C]{\smallbreak\pagebreak[2]} \Standort{DLA, A:Schnitzler, HS.NZ85.1.896.}
\physDesc{Brief, 1 Blatt, 1 Seite, maschineller Durchschlag
\newline{}Handschrift: roter Buntstift, lateinische Kurrent (\noindent{}»Grossmann«, »\textcolor{pink}{Berlin}«, Unterstreichungen)}\toendnotes[C]{\smallbreak}\pstart
           \raggedleft{}{\pb}7. 11. 1927.\pend
           \pstart{}Verehrter Herr Stefan Grossmann.\pend\pstart
           Ende dieses Monats wird mein \textcolor{green}{Aphorismenbuch}{}\ledrightnote{→\textcolor{green}{Buch der Sprüche und Bedenken}} erscheinen und wenn Sie ihren
                    freundlichen Wunsch von früher her noch aufrecht erhalten, so würde ich Ihnen
                    gerne etliches (noch Ungedrucktes) aus dem \textcolor{green}{Buch}{}\ledrightnote{→\textcolor{green}{Buch der Sprüche und Bedenken}}{ }\strikeout{zur Verfügung stellen} zum Vorabdruck zur
                    Verfügung stellen.\pend
           \pstart
           Stimmt es, dass in Ihrem »\textcolor{green}{Tagebuch}{}\ledrightnote{\textcolor{green}{Das Tage-Buch}}« im
                        Sommer dieses Jahres wieder einige meiner Aphorismen (entweder
                    aus der »\textcolor{green}{\textcolor{green}{Neuen Freien Presse}{}\ledrightnote{\textcolor{green}{Neue Freie Presse}}}{}\ledrightnote{→\textcolor{green}{Bemerkungen. Aus dem noch unveröffentlichten »Buch der Sprüche und Bedenken«.}}« oder einer \textcolor{green}{\textcolor{green}{Dresdner Zeitung}{}\ledrightnote{→\textcolor{green}{Dresdner Neueste Nachrichten}}}{}\ledrightnote{→\textcolor{green}{Bemerkungen. (Aus dem noch unveröffentlichten »Buch der Sprüche und Bedenken«)}} \label{K_L02492_1v}\edtext{abgedruckt waren}{\lemma{\textnormal{\emph{abgedruckt waren}}}\Cendnote{\textnormal{Es waren 1927 keine
                        Aphorismen \textcolor{blue}{Schnitzler}s abgedruckt. Erst
                        in Folge dieses Briefes erschienen am 19. 11. 1927{ }\emph{\textcolor{green}{Bemerkungen}} (Jg. 8, H. 47,
                            S. 1879–1881).}}}\label{K_L02492_1h}? Dies frage ich nur der Ordnung wegen.\pend
           \pstart
           Mit verbindlichen Grüssen{\\[\baselineskip]}Ihr sehr ergebener\pend
           \leftskip=0em{}{\bigskip}\pstart
           \noindent{}Herrn Stefan Grossmann{\\}Herausgeber des »\textcolor{brown}{Tagebuch}{}\ledrightnote{\textcolor{brown}{Das Tage-Buch}}«,{\\}\textcolor{pink}{Berlin SW. 19, Beuthstr. 19}{}\ledrightnote{\textcolor{pink}{Beuthstrasse}}.\pend
           \endnumbering\briefempfaengerindex{Grossmann, Stefan@\textsc{Großmann, Stefan}!zzzSchnitzler, Arthur@\emph{von Arthur Schnitzler}!1927-11-071@{7. 11. 1927}|)be}\mylabel{h}  \normalsize

\doendnotes{C}
\bigskip
\vfill

\clearpage

\footnotesize

\lohead{\textsc{register}}

% Definiere theindex-Environment komplett neu ohne reledmac
\makeatletter
\renewenvironment{theindex}{%
  \section*{\indexname}%
  \setlength{\parindent}{0pt}%
  \setlength{\parskip}{0pt plus 0.3pt}%
  \let\item\@idxitem
}{%
  \clearpage
}
\makeatother

\IfFileExists{\jobname-pw.ind}{\input{\jobname-pw.ind}}{}

\end{document}

      