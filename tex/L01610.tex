%% latex-korrekturansicht-vorspann.tex
%% Vorspann für die Korrekturansicht.
%% Lädt die gemeinsame Datei latex-vorspann.tex mit gesetztem Schalter.

\newif\ifkorrekturansicht
\korrekturansichttrue

\input{../tex-inputs/latex-vorspann}


               \section[Arthur Schnitzler an Georg Brandes, 13. 7. 1906]{ Arthur Schnitzler an Georg Brandes, 13. 7. 1906}\nopagebreak\mylabel{v}\rehead{ }\normalsize\beginnumbering\briefempfaengerindex{Brandes, Georg@\textsc{Brandes, Georg}!zzzSchnitzler, Arthur@\emph{von Arthur Schnitzler}!1906-07-131@{13. 7. 1906}|(be} \toendnotes[C]{\smallbreak\pagebreak[2]} \Standort{Kopenhagen, Det Kongelige Bibliotek, Georg Brandes Arkiv, box 125.}
\physDesc{Briefkarte
\newline{}Handschrift: schwarze Tinte, lateinische Kurrent\newline{}Ordnung: mit schwarzer Tinte von unbekannter Hand in der linken
                                            oberen Ecke: »\textsc{S}« vermerkt, womöglich zur archivalischen
                                            Einordnung }\buchAbdrucke{\weitereDrucke{Georg Brandes, Arthur Schnitzler: \emph{Ein Briefwechsel}. Hg. Kurt Bergel. Bern: \emph{Francke} 1956, S. 94.} }\pstart
           \noindent{}{\pb}\textcolor{gray}{\textbf{Dr. Arthur Schnitzler}}\hfill 13. Juli 906\pend
           \pstart
           \textcolor{gray}{\textbf{\textcolor{pink}{Wien, XVIII. Spoettelgasse 7}{}\ledrightnote{\textcolor{pink}{Edmund-Weiß-Gasse}}.}}\pend
           \pstart{}verehrtester Herr Brandes,\pend\pstart
           entschuldigen Sie, dass ich neulich gar so miserabel schrieb. Der Sie grüßen
                    liess, ist \textcolor{blue}{Brahm}{}\ledrightnote{\textcolor{blue}{Otto Brahm}} (der übrigens
                    möglicherweise in diese Gegend ko{\geminationm}t.) Dass Sie
                    schon aus Bett und Spital heraus sind, freut mich sehr. Aber glauben Sie um
                    Gotteswillen nicht, dass ich auf »Gegenbesuche« od. dergl. Anspruch mache.
                    Freilich möchte ich Sie sehr gerne noch einmal sehen, ehe ich \textcolor{pink}{Daenemark}{}\ledrightnote{\textcolor{pink}{Dänemark}} verlasse (was kaum vor 3–4 Wochen der Fall sein
                    wird), aber wenn Ihnen \textcolor{pink}{Marienlyst}{}\ledrightnote{\textcolor{pink}{Marienlyst}}{ }{\pb}die geringste Unbequemlichkeit macht, so
                    erlauben Sie mir vielleicht wieder einmal, Sie in \textcolor{pink}{Kopenhagen}{}\ledrightnote{\textcolor{pink}{Kopenhagen}} heimzusuchen. Jedenfalls werd ich mich melden, we{\geminationn} ich auf der Rückreise ein paar Tage Aufenthalt
                    mache. Aber wenn Sie hieher ko{\geminationm}en (es ist wirklich
                    wunderschön da), haben Sie die Güte, mich vorher wissen zu lassen. Ich möchte
                    doch nicht gern in \textcolor{pink}{Schweden}{}\ledrightnote{\textcolor{pink}{Schweden}} drüben, in \textcolor{pink}{Skodsborg}{}\ledrightnote{\textcolor{pink}{Skodsborg}} oder – in \textcolor{pink}{Kopenhagen}{}\ledrightnote{\textcolor{pink}{Kopenhagen}} sein, wenn Sie in \textcolor{pink}{Marienlyst}{}\ledrightnote{\textcolor{pink}{Marienlyst}} erscheinen.\pend
           \pstart
           Herzlichen Gruß. Ihr sehr ergebener{\\[\baselineskip]}\spacefill\mbox{Arthur Schnitzler}\pend
           \leftskip=0em{}\endnumbering\briefempfaengerindex{Brandes, Georg@\textsc{Brandes, Georg}!zzzSchnitzler, Arthur@\emph{von Arthur Schnitzler}!1906-07-131@{13. 7. 1906}|)be}\mylabel{h}  \normalsize

\doendnotes{C}
\bigskip
\vfill

\clearpage

\footnotesize

\lohead{\textsc{register}}

% Definiere theindex-Environment komplett neu ohne reledmac
\makeatletter
\renewenvironment{theindex}{%
  \section*{\indexname}%
  \setlength{\parindent}{0pt}%
  \setlength{\parskip}{0pt plus 0.3pt}%
  \let\item\@idxitem
}{%
  \clearpage
}
\makeatother

\IfFileExists{\jobname-pw.ind}{\input{\jobname-pw.ind}}{}

\end{document}

      