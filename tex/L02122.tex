%% latex-korrekturansicht-vorspann.tex
%% Vorspann für die Korrekturansicht.
%% Lädt die gemeinsame Datei latex-vorspann.tex mit gesetztem Schalter.

\newif\ifkorrekturansicht
\korrekturansichttrue

\input{../tex-inputs/latex-vorspann}


               \section[Arthur Schnitzler an Hermann Bahr, 18. 4. 1913]{ Arthur Schnitzler an Hermann Bahr, 18. 4. 1913}\nopagebreak\mylabel{v}\rehead{ }\normalsize\beginnumbering\briefempfaengerindex{Bahr, Hermann@\textsc{Bahr, Hermann}!zzzSchnitzler, Arthur@\emph{von Arthur Schnitzler}!1913-04-181@{18. 4. 1913}|(be} \toendnotes[C]{\smallbreak\pagebreak[2]} \Standort{TMW, HS AM 60160 Ba.}
\physDesc{Briefkarte
\newline{}Schreibmaschine
\newline{}Handschrift: schwarze Tinte, deutsche Kurrent (\noindent{}Grußformel und Unterschrift)\newline{}Ordnung: Lochung }\buchAbdrucke{\weitereDrucke{1) \emph{18. 4. 1913, Abschrift.} In: Arthur Schnitzler: \emph{The Letters of Arthur Schnitzler to Hermann Bahr}. Edited, annotated, and with an introduction, by Donald G.
                        Daviau. Chapel Hill: \emph{The University of North Carolina Press} 1978, S. 110 (University of North Carolina studies in the Germanic languages
                        and literatures, 89).} \weitereDrucke{2) Hermann Bahr, Arthur Schnitzler: \emph{Briefwechsel, Aufzeichnungen, Dokumente (1891–1931)}. Hg. Kurt Ifkovits und Martin Anton Müller. Göttingen: \emph{Wallstein} 2018, S. 482.} }\toendnotes[C]{\smallbreak}\pstart
           \noindent{}{\pb}\textcolor{gray}{\textbf{Dr. Arthur Schnitzler}}\hfill 18. 4. 1913. \pend
           \pstart
           \textcolor{gray}{\textbf{\textcolor{pink}{Wien XVIII. Sternwartestrasse 71}{}\ledrightnote{\textcolor{pink}{Sternwartestraße}}}}\pend
           \pstart{}Lieber Hermann.\pend\pstart
           Auch ich habe einen Brief von \textcolor{blue}{Altenberg}{}\ledrightnote{\textcolor{blue}{Peter Altenberg}}{ }\introOben{}(\introOben{}offenbar ähnlichen Inhalts wie der an Dich\introOben{})\introOben{} erhalten; sein \textcolor{blue}{Bruder}{}\ledrightnote{→\textcolor{blue}{Georg Engländer}} hat ihn mir überschickt. Diesem habe ich nun
               geantwortet, er möge mir sagen, was ich seiner Ansicht nach in der Angelegenheit tun
               könne; ich sei natürlich gerne bereit in die Anstalt zu gehen und dort mit dem
               behandelnden \textcolor{blue}{Arzt}{}\ledrightnote{→\textcolor{blue}{Karl Richter}} Rücksprache
               zu nehmen. Ich selbst habe \textcolor{blue}{Altenberg}{}\ledrightnote{\textcolor{blue}{Peter Altenberg}} schon über
               ein Jahr nicht gesehen und stehe trotz allem, was mir selbst von ärztlicher Seite
               berichtet wird, der absoluten Echtheit von \textcolor{blue}{P. A.}{}\ledrightnote{\textcolor{blue}{Peter Altenberg}}’s
                  Irr{\pb}sinn – es ist ja
               vielleicht dumm – mit einer seit fast drei Jahrzehnten bewährten Skepsis gegenüber.
               Dass an \textcolor{blue}{P. A.}{}\ledrightnote{\textcolor{blue}{Peter Altenberg}}’s Einschliessung nicht etwa böser
               Wille schuld sein kann ist selbstverständlich. Also, wenn eine Entlassung überhaupt
               möglich (was ich aus vielen Gründen für höchst wahrscheinlich halte) wird dazu weder
               Skandal noch Entführung notwendig sein. Du hörst bald mehr von mir. Wann kommst Du
               nach \textcolor{pink}{Wien}{}\ledrightnote{\textcolor{pink}{Wien}}? Man sieht Dich nun doch nicht trotzdem Du
               in \textcolor{pink}{Salzburg}{}\ledrightnote{\textcolor{pink}{Salzburg}} wohnst.\pend
           \pstart
           Herzliche Grüsse von Haus zu Haus{\\[\baselineskip]}Dein{\\[\baselineskip]}\spacefill\mbox{{[}hs.:{]} Arthur}\pend
           \leftskip=0em{}\endnumbering\briefempfaengerindex{Bahr, Hermann@\textsc{Bahr, Hermann}!zzzSchnitzler, Arthur@\emph{von Arthur Schnitzler}!1913-04-181@{18. 4. 1913}|)be}\mylabel{h}  \normalsize

\doendnotes{C}
\bigskip
\vfill

\clearpage

\footnotesize

\lohead{\textsc{register}}

% Definiere theindex-Environment komplett neu ohne reledmac
\makeatletter
\renewenvironment{theindex}{%
  \section*{\indexname}%
  \setlength{\parindent}{0pt}%
  \setlength{\parskip}{0pt plus 0.3pt}%
  \let\item\@idxitem
}{%
  \clearpage
}
\makeatother

\IfFileExists{\jobname-pw.ind}{\input{\jobname-pw.ind}}{}

\end{document}

      