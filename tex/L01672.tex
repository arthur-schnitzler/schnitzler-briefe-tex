%% latex-korrekturansicht-vorspann.tex
%% Vorspann für die Korrekturansicht.
%% Lädt die gemeinsame Datei latex-vorspann.tex mit gesetztem Schalter.

\newif\ifkorrekturansicht
\korrekturansichttrue

\input{../tex-inputs/latex-vorspann}


               \section[Hugo von Hofmannsthal an Arthur Schnitzler, 29. 4. 1907]{ Hugo von Hofmannsthal an Arthur Schnitzler, 29. 4. 1907}\nopagebreak\mylabel{v}\rehead{ }\normalsize\beginnumbering\briefempfaengerindex{Schnitzler, Arthur@\textsc{Schnitzler, Arthur}!zzzHofmannsthal, Hugo von@\emph{von Hugo von Hofmannsthal}!1907-04-291@{29. 4. 1907}|(be} \toendnotes[C]{\smallbreak\pagebreak[2]} \Standort{CUL, Schnitzler, B 43.}
\physDesc{Brief, 1 Blatt, 1 Seite
\newline{}Schreibmaschine
\newline{}Handschrift: Bleistift (\noindent{}Unterschrift)\newline{}Ordnung: mit Bleistift von unbekannter Hand Nummerierungen angebracht:
                                    »274« respektive »275« }\buchAbdrucke{\weitereDrucke{Hugo von Hofmannsthal, Arthur Schnitzler: \emph{Briefwechsel}. Hg. Therese Nickl und Heinrich Schnitzler. Frankfurt am Main: \emph{S. Fischer} 1964, S. 228.} }\pstart
           \raggedleft{}{\pb}\textcolor{pink}{Rodaun}{}\ledrightnote{\textcolor{pink}{Rodaun}} d 29 IV 07\pend
           \pstart{}Mein lieber Arthur, \pend\pstart
           bitte seien Sie so gut und schreiben mir gleich in aller Kürze auf einer offenen
               Karte ob Sie von \textcolor{blue}{Fischer}{}\ledrightnote{\textcolor{blue}{Samuel Fischer}} für Novellen und
               ähnliches gleich von der ersten Auflage an 25{\%} bekommen oder
               von den ersten Auflagen nur 20{\%} und erst von einer höheren an
                  25{\%}.\pend
           \pstart
           Letzteres wäre mein bisheriger Fall.\pend
           \pstart Herzlichchst Ihr\spacefill\mbox{{[}hs.:{]} Hugo}\pend{}\endnumbering\briefempfaengerindex{Schnitzler, Arthur@\textsc{Schnitzler, Arthur}!zzzHofmannsthal, Hugo von@\emph{von Hugo von Hofmannsthal}!1907-04-291@{29. 4. 1907}|)be}\mylabel{h}  \normalsize

\doendnotes{C}
\bigskip
\vfill

\clearpage

\footnotesize

\lohead{\textsc{register}}

% Definiere theindex-Environment komplett neu ohne reledmac
\makeatletter
\renewenvironment{theindex}{%
  \section*{\indexname}%
  \setlength{\parindent}{0pt}%
  \setlength{\parskip}{0pt plus 0.3pt}%
  \let\item\@idxitem
}{%
  \clearpage
}
\makeatother

\IfFileExists{\jobname-pw.ind}{\input{\jobname-pw.ind}}{}

\end{document}

      