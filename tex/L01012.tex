%% latex-korrekturansicht-vorspann.tex
%% Vorspann für die Korrekturansicht.
%% Lädt die gemeinsame Datei latex-vorspann.tex mit gesetztem Schalter.

\newif\ifkorrekturansicht
\korrekturansichttrue

\input{../tex-inputs/latex-vorspann}


               \section[Arthur Schnitzler an Georg Brandes, 11. 2. 1900]{ Arthur Schnitzler an Georg Brandes, 11. 2. 1900}\nopagebreak\mylabel{v}\rehead{ }\normalsize\beginnumbering\briefempfaengerindex{Brandes, Georg@\textsc{Brandes, Georg}!zzzSchnitzler, Arthur@\emph{von Arthur Schnitzler}!1900-02-111@{11. 2. 1900}|(be} \toendnotes[C]{\smallbreak\pagebreak[2]} \Standort{Kopenhagen, Det Kongelige Bibliotek, Georg Brandes Arkiv, box 125.}
\physDesc{Briefkarte
\newline{}Handschrift: schwarze Tinte, deutsche Kurrent\newline{}Ordnung: mit Bleistift von unbekannter Hand nummeriert: »19. Schnitzler« }\buchAbdrucke{\weitereDrucke{Georg Brandes, Arthur Schnitzler: \emph{Ein Briefwechsel}. Hg. Kurt Bergel. Bern: \emph{Francke} 1956, S. 79.} }\toendnotes[C]{\smallbreak}\pstart
           \raggedleft{}{\pb}\textcolor{pink}{Wien}{}\ledrightnote{\textcolor{pink}{Wien}}, 11. 2. 1900.\pend
           \pstart
           \textcolor{pink}{IX. Frankgaſſe 1.}{}\ledrightnote{\textcolor{pink}{Frankgasse}}\pend
           \pstart
           Verehrteſter Herr Brandes, Sie haben dieſer Tage ein kleines
                        \textcolor{green}{Novellenbuch}{}\ledrightnote{→\textcolor{green}{Der Hinterbliebene. Kurze Novellen}} von \textcolor{blue}{Felix Salten}{}\ledrightnote{\textcolor{blue}{Felix Salten}} zugeſchickt erhalten. Der
                    Verfaſſer (den Sie bei mir \label{K_L01012_1v}\edtext{einmal
                        ſahn}{\lemma{\textnormal{\emph{einmal
                        ſahn}}}\Cendnote{\textnormal{vgl. A. S.: \emph{Tagebuch}, 28. 1. 1898}}}\label{K_L01012_1h}) wäre natürlich ſehr froh, wenn Sie Zeit fänden, ſein Buch gelegentlich
                    zu leſen, und auch ich bitte Sie darum.\pend
           \pstart
           Von mir hören Sie bald mehr, bei Gelegenheit einer {\pb}\textcolor{green}{Dialogſa{\geminationm}lung}{}\ledrightnote{→\textcolor{green}{Reigen. Zehn Dialoge}}, die ich nur drucken, aber nicht
                    erſcheinen laſſe, da die Menſchheit zu ſittlich iſt, um es zu dulden.\pend
           \pstart
           Ich ſehne das Frühjahr herbei; der Winter iſt für mich wie ein Gefängnis. Warum
                    ich nicht in den Süden fliehe? Das hat allerlei Gründe – vielleicht auch gar
                    keinen rechten. Ihre Geſundheit hoff ich iſt jetzt vollko{\geminationm}en gefeſtigt. Von Herzen Ihr
                        \spacefill\mbox{ArthurSchnitzler}\pend
           \endnumbering\briefempfaengerindex{Brandes, Georg@\textsc{Brandes, Georg}!zzzSchnitzler, Arthur@\emph{von Arthur Schnitzler}!1900-02-111@{11. 2. 1900}|)be}\mylabel{h}  \normalsize

\doendnotes{C}
\bigskip
\vfill

\clearpage

\footnotesize

\lohead{\textsc{register}}

% Definiere theindex-Environment komplett neu ohne reledmac
\makeatletter
\renewenvironment{theindex}{%
  \section*{\indexname}%
  \setlength{\parindent}{0pt}%
  \setlength{\parskip}{0pt plus 0.3pt}%
  \let\item\@idxitem
}{%
  \clearpage
}
\makeatother

\IfFileExists{\jobname-pw.ind}{\input{\jobname-pw.ind}}{}

\end{document}

      