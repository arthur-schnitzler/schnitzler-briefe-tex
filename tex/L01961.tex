%% latex-korrekturansicht-vorspann.tex
%% Vorspann für die Korrekturansicht.
%% Lädt die gemeinsame Datei latex-vorspann.tex mit gesetztem Schalter.

\newif\ifkorrekturansicht
\korrekturansichttrue

\input{../tex-inputs/latex-vorspann}


               \section[Hugo von Hofmannsthal an Arthur Schnitzler, 30. 9. {[}1910{]}]{ Hugo von Hofmannsthal an Arthur Schnitzler, 30. 9. {[}1910{]}}\nopagebreak\mylabel{v}\rehead{ }\normalsize\beginnumbering\briefempfaengerindex{Schnitzler, Arthur@\textsc{Schnitzler, Arthur}!zzzHofmannsthal, Hugo von@\emph{von Hugo von Hofmannsthal}!1910-09-301@{30. 9. {[}1910{]}}|(be} \toendnotes[C]{\smallbreak\pagebreak[2]} \Standort{CUL, Schnitzler, B 43.}
\physDesc{Brief, 1 Blatt, 4 Seiten
\newline{}Handschrift: schwarze Tinte, deutsche Kurrent
\newline{}Schnitzler: mit Bleistift die Jahreszahl ergänzt: »1910« und beschriftet: »\textsc{Hugo}« \newline{}Ordnung: 1) mit Bleistift von unbekannter Hand nummeriert:
                                    »315« 2) mit Bleistift von unbekannter Hand nummeriert: »322«}\buchAbdrucke{\weitereDrucke{Hugo von Hofmannsthal, Arthur Schnitzler: \emph{Briefwechsel}. Hg. Therese Nickl und Heinrich Schnitzler. Frankfurt am Main: \emph{S. Fischer} 1964, S. 253.} }\toendnotes[C]{\smallbreak}\pstart
           \noindent{}{\pb}30 IX.\hfill \textcolor{pink}{München, Hotel Marienbad}{}\ledrightnote{\textcolor{pink}{Hotel Marienbad}}\pend
           \pstart
           mein lieber, wenn Ihnen auch wie mir, inliegender \label{K_L01961_1v}\edtext{Beſetzungsvorſchlag}{\lemma{\textnormal{\emph{Beſetzungsvorſchlag}}}\Cendnote{\textnormal{Es handelt sich um die Trauerfeier für \textcolor{blue}{Kainz}, die am 23. 10. 1910
                  stattfinden sollte und bei der – neben anderem – der \emph{\textcolor{green}{Der Tor und der Tod}} gegeben werden sollte. \textcolor{blue}{Gerasch} bekam die ihm hier zugedachte Rolle, die Rolle des \textcolor{green}{Tod}s sollte \textcolor{blue}{Albert Heine} spielen.}}}\label{K_L01961_1h} absurd erſcheint und die
               Beſetzung \textsc{\textcolor{green}{Claudio}{}\ledrightnote{→\textcolor{green}{Der Thor und der Tod}} – \textcolor{blue}{Gerasch}{}\ledrightnote{\textcolor{blue}{Alfred Gerasch}}} / \textsc{\textcolor{green}{Tod}{}\ledrightnote{→\textcolor{green}{Der Thor und der Tod}} – \textcolor{blue}{Reimers}{}\ledrightnote{\textcolor{blue}{Georg Reimers}}} als die richtigere, ſo tun Sie mir den großen Gefallen und bringen dieſe meine
               und Ihre Auffassung bei \textsc{Berger}{ }{\pb}\textsc{telephonisch} in meinem Namen unter Berufung auf dieſen
               Brief vor.\pend
           \pstart
           Ich finde den Gedanken, \textsc{Tressler} eine geiſtige Geſtalt
               agieren zu ſehen, ſcheußlich und möchte das Ganze faſt lieber inhibieren, ſcheue aber
               dann wieder den {\pb}überflüſſigen
               Rummel. O ekelhaftes \textcolor{pink}{Wien}{}\ledrightnote{\textcolor{pink}{Wien}}! ekelhafteres \textcolor{brown}{Burgtheater}{}\ledrightnote{\textcolor{brown}{Burgtheater}}! ekelhaft wenn es einen nicht ſpielt und
               noch fühlbar ekelhafter, wenn es Miene macht, einen zu ſpielen! (Gilt für mich, und
               nicht für Sie). Bitte depeſchieren Sie mir {\pb}hieher was Sie getan oder nicht
               getan haben.\pend
           \pstart
           Freute mich ſehr über den ſo \label{K_L01961_2v}\edtext{ſtarken
                  Erfolg}{\lemma{\textnormal{\emph{ſtarken
                  Erfolg}}}\Cendnote{\textnormal{Diese war am
                     15. 9. 1910 im \emph{\textcolor{brown}{Burgtheater}}
                  neuerlich inszeniert worden. \textcolor{blue}{Schnitzler} weilte
                  zu der Zeit in \textcolor{pink}{Frankfurt am Main}, um der
                  Uraufführung der \textcolor{green}{Opernfassung}
                  am 18. 9. 1910 beizuwohnen.}}}\label{K_L01961_2h} der braven alten »\textcolor{green}{Liebelei}{}\ledrightnote{\textcolor{green}{Liebelei. Schauspiel in drei Akten}}«.\hspace*{1.5em}Wenn Sie ein
               überflüſſiges Exemplar vom »\textcolor{green}{Weiten Land}{}\ledrightnote{\textcolor{green}{Das weite Land. Tragikomödie in fünf Akten}}« haben, ſo
               trifft es mich von Dienſtag an auf \textsc{Schloss
                  Neubeuern am Inn} und macht mir große Freude.\pend
           \pstart Ihr \spacefill\mbox{Hugo.}\pend{}\endnumbering\briefempfaengerindex{Schnitzler, Arthur@\textsc{Schnitzler, Arthur}!zzzHofmannsthal, Hugo von@\emph{von Hugo von Hofmannsthal}!1910-09-301@{30. 9. {[}1910{]}}|)be}\mylabel{h}  \normalsize

\doendnotes{C}
\bigskip
\vfill

\clearpage

\footnotesize

\lohead{\textsc{register}}

% Definiere theindex-Environment komplett neu ohne reledmac
\makeatletter
\renewenvironment{theindex}{%
  \section*{\indexname}%
  \setlength{\parindent}{0pt}%
  \setlength{\parskip}{0pt plus 0.3pt}%
  \let\item\@idxitem
}{%
  \clearpage
}
\makeatother

\IfFileExists{\jobname-pw.ind}{\input{\jobname-pw.ind}}{}

\end{document}

      