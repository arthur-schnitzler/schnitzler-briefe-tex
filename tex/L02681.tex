%% latex-korrekturansicht-vorspann.tex
%% Vorspann für die Korrekturansicht.
%% Lädt die gemeinsame Datei latex-vorspann.tex mit gesetztem Schalter.

\newif\ifkorrekturansicht
\korrekturansichttrue

\input{../tex-inputs/latex-vorspann}


               \section[Paul Goldmann an Arthur Schnitzler, 9. 6. {[}1899{]}]{ Paul Goldmann an Arthur Schnitzler, 9. 6. {[}1899{]}}\nopagebreak\mylabel{v}\rehead{ }\normalsize\beginnumbering\briefempfaengerindex{Schnitzler, Arthur@\textsc{Schnitzler, Arthur}!zzzGoldmann, Paul@\emph{von Paul Goldmann}!1899-06-091@{9. 6. {[}1899{]}}|(be} \toendnotes[C]{\smallbreak\pagebreak[2]} \Standort{DLA, A:Schnitzler, HS.NZ85.1.3169.}
\physDesc{Telegramm1 Blatt, 1 Seite
\newline{}maschinell
\newline{}Schnitzler: mit Bleistift datiert: »9/6 99« \newline{}Ordnung: beschnitten }\pstart
           \noindent{}\centering{}{\pb}fr \textcolor{pink}{frankfurtm}{}\ledrightnote{\textcolor{pink}{Frankfurt am Main}}
               898 23 9/6{ }9 50 m =\pend
           \pstart
           \noindent{}jch war verreist. kenne \textcolor{blue}{antoine}{}\ledrightnote{\textcolor{blue}{André Antoine}}s adresse nicht. du bittest
               am besten \textcolor{blue}{thorel}{}\ledrightnote{\textcolor{blue}{Jean Thorel}} um uebermittlung briefes =\pend
           \pstart \spacefill\mbox{goldmann. +}\pend{}\endnumbering\briefempfaengerindex{Schnitzler, Arthur@\textsc{Schnitzler, Arthur}!zzzGoldmann, Paul@\emph{von Paul Goldmann}!1899-06-091@{9. 6. {[}1899{]}}|)be}\mylabel{h}\begin{anhang}\end{anhang}\normalsize

\doendnotes{C}
\bigskip
\vfill

\clearpage

\footnotesize

\lohead{\textsc{register}}

% Definiere theindex-Environment komplett neu ohne reledmac
\makeatletter
\renewenvironment{theindex}{%
  \section*{\indexname}%
  \setlength{\parindent}{0pt}%
  \setlength{\parskip}{0pt plus 0.3pt}%
  \let\item\@idxitem
}{%
  \clearpage
}
\makeatother

\IfFileExists{\jobname-pw.ind}{\input{\jobname-pw.ind}}{}

\end{document}

      