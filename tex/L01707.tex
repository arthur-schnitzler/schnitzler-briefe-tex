%% latex-korrekturansicht-vorspann.tex
%% Vorspann für die Korrekturansicht.
%% Lädt die gemeinsame Datei latex-vorspann.tex mit gesetztem Schalter.

\newif\ifkorrekturansicht
\korrekturansichttrue

\input{../tex-inputs/latex-vorspann}


               \section[Arthur Schnitzler an Hugo von Hofmannsthal, 16. 9. 1907]{ Arthur Schnitzler an Hugo von Hofmannsthal, 16. 9. 1907}\nopagebreak\mylabel{v}\rehead{ }\normalsize\beginnumbering\briefempfaengerindex{Hofmannsthal, Hugo von@\textsc{Hofmannsthal, Hugo von}!zzzSchnitzler, Arthur@\emph{von Arthur Schnitzler}!1907-09-161@{16. 9. 1907}|(be} \toendnotes[C]{\smallbreak\pagebreak[2]} \Standort{FDH, Hs-30885,129.}
\physDesc{Brief, 1 Blatt, 1 Seite, maschineller Durchschlag
\newline{}Schreibmaschine
\newline{}Handschrift: 1) Bleistift, deutsche Kurrent (\noindent{}Beschriftung: »\textsc{Hofmsthal}«)\hspace{1em}2) roter Buntstift, deutsche Kurrent (\noindent{}Unterstreichungen)\hspace{1em}\newline{}Ordnung: 1) Lochung 2) mit Bleistift von unbekannter Hand nummeriert:
                                                »129«\newline{}Zusatz: Zusammen mit der fehlenden Unterschrift scheint es
                                            unwahrscheinlich, dass dies das tatsächlich übermittelte
                                            Korrespondenzstück darstellt, obzwar es im Nachlass
                                            Hofmannsthals aufbewahrt ist. Mit großer
                                            Wahrscheinlichkeit dürfte es bei der Durchsicht der
                                            Briefe nach Hofmannsthals Tod 1929
                                            hinzugefügt worden sein. }\buchAbdrucke{\weitereDrucke{Hugo von Hofmannsthal, Arthur Schnitzler: \emph{Briefwechsel}. Hg. Therese Nickl und Heinrich Schnitzler. Frankfurt am Main: \emph{S. Fischer} 1964, S. 231.} }\toendnotes[C]{\smallbreak}\pstart
           \raggedleft{}{\pb}16. Sept. 07.\pend
           \pstart{}Lieber Hugo,\pend\pstart
           Ich danke Ihnen noch sehr für Ihr Telegramm. Der »\textcolor{brown}{Morgen}{}\ledrightnote{\textcolor{brown}{Morgen. Wochenschrift für deutsche Kultur}}« scheint über meine Forderung nicht angenehm überrascht gewesen
                    zu sein. Sie bieten die \label{T_L01707_1v}\edtext{Hälfte,}{\lemma{\textnormal{\emph{Hälfte,}}}\Cendnote{\textnormal{Fehler: »Hälfte.,«}}}\label{T_L01707_1h} scheinen aber entschlossen,
                    wenn sie auch das \textcolor{green}{Buch}{}\ledrightnote{→\textcolor{green}{Der Weg ins Freie. Roman}}
                    kriegen, höher gehen zu wollen{\dotstwo} Ich habe eigentlich
                    nicht den Eindruck, dass aus der Sache was werden wird. Dieser Schreibebrief hat
                    übrigens einen besonderen Zweck. Ich muss Sie etwas meinen \textcolor{green}{Roman}{}\ledrightnote{→\textcolor{green}{Der Weg ins Freie. Roman}} betreffend fragen. Ist es nicht
                    höchst unwahrscheinlich, dass ein Mensch erst mit acht–neunundzwanzig Jahren
                    seine Diplomatenprüfung ablegt? Wär es aber nicht möglich, dass ein junger
                    Mensch eine \label{T_L01707_2v}\edtext{Staatskarriere}{\lemma{\textnormal{\emph{Staatskarriere}}}\Cendnote{\textnormal{Fehler:
                        »Staastkarriere«}}}\label{T_L01707_2h} einschlägt, Statthalterei zum
                    Beispiel und dass er dann zur Diplomatie übergeht? Ferner: Muss jemand, der die
                    Diplomatenprüfung macht vorher die \textcolor{brown}{orientalische
                        Akademie}{}\ledrightnote{\textcolor{brown}{Orientalische Akademie}} besucht haben, oder genügt die Universität?\pend
           \endnumbering\briefempfaengerindex{Hofmannsthal, Hugo von@\textsc{Hofmannsthal, Hugo von}!zzzSchnitzler, Arthur@\emph{von Arthur Schnitzler}!1907-09-161@{16. 9. 1907}|)be}\mylabel{h}  \normalsize

\doendnotes{C}
\bigskip
\vfill

\clearpage

\footnotesize

\lohead{\textsc{register}}

% Definiere theindex-Environment komplett neu ohne reledmac
\makeatletter
\renewenvironment{theindex}{%
  \section*{\indexname}%
  \setlength{\parindent}{0pt}%
  \setlength{\parskip}{0pt plus 0.3pt}%
  \let\item\@idxitem
}{%
  \clearpage
}
\makeatother

\IfFileExists{\jobname-pw.ind}{\input{\jobname-pw.ind}}{}

\end{document}

      