%% latex-korrekturansicht-vorspann.tex
%% Vorspann für die Korrekturansicht.
%% Lädt die gemeinsame Datei latex-vorspann.tex mit gesetztem Schalter.

\newif\ifkorrekturansicht
\korrekturansichttrue

\input{../tex-inputs/latex-vorspann}


               \section[Arthur Schnitzler an Hermann Bahr, 9. 10. 1903]{ Arthur Schnitzler an Hermann Bahr, 9. 10. 1903}\nopagebreak\mylabel{v}\rehead{ }\normalsize\beginnumbering\briefempfaengerindex{Bahr, Hermann@\textsc{Bahr, Hermann}!zzzSchnitzler, Arthur@\emph{von Arthur Schnitzler}!1903-10-091@{9. 10. 1903}|(be} \toendnotes[C]{\smallbreak\pagebreak[2]} \Standort{TMW, HS AM 23358 Ba.}
\physDesc{Kartenbrief
\newline{}Handschrift: schwarze Tinte, deutsche Kurrent\newline{}Versand: 1) Stempel: »\nobreak{}\oindex{IX., Alsergrund@\textbf{IX., Alsergrund}, \emph{Bezirk (A.BZK)}|pwk}Wien 9, 9. 10. {[}1903{]}, 11–12 V\nobreak{}«.  2) Stempel: »\nobreak{}\oindex{XIII., Hietzing@\textbf{XIII., Hietzing}, \emph{Bezirk (A.BZK)}|pwk}Wien 13, 9 10 03\nobreak{}«. }\buchAbdrucke{\weitereDrucke{1) \emph{9. 10. 1903.} In: Arthur Schnitzler: \emph{The Letters of Arthur Schnitzler to Hermann Bahr}. Edited, annotated, and with an introduction, by Donald G.
                        Daviau. Chapel Hill: \emph{The University of North Carolina Press} 1978, S. 80 (University of North Carolina studies in the Germanic languages
                        and literatures, 89).} \weitereDrucke{2) Hermann Bahr, Arthur Schnitzler: \emph{Briefwechsel, Aufzeichnungen, Dokumente (1891–1931)}. Hg. Kurt Ifkovits und Martin Anton Müller. Göttingen: \emph{Wallstein} 2018, S. 272.} }\toendnotes[C]{\smallbreak}\pstart{}{\pb}Herrn Hermann
                  Bahr\pend{}\pstart{}\textcolor{pink}{Wien-Ob-St Veit}{}\ledrightnote{\textcolor{pink}{Ober Sankt Veit}}\pend{}\pstart{}\textcolor{pink}{Veitliſſengaſſe.}{}\ledrightnote{\textcolor{pink}{Veitlissengasse}}\pend{}{\bigskip}\pstart
           \noindent{}\raggedleft{}{\pb}\textsc{\uline{\textcolor{pink}{XVIII Spöttelgasse 7}{}\ledrightnote{\textcolor{pink}{Edmund-Weiß-Gasse}}}}\pend
           \pstart
           \textcolor{pink}{Wien}{}\ledrightnote{\textcolor{pink}{Wien}}{ }9. 10. 903.\pend
           \pstart
           lieber Hermann, \textcolor{green}{Reigen}{}\ledrightnote{\textcolor{green}{Reigen. Zehn Dialoge}} laſs ich dir ſofort ſchicken. Ich bin
               neugierig was die Cenſur ſagt. Dann werden wir über die Anzahl der Sitze reden, die
               du ſo gütig biſt mir in Ausſicht zu ſtellen. In \textcolor{pink}{Berlin}{}\ledrightnote{\textcolor{pink}{Berlin}} grüße mir, wenn du ſie ſiehſt, \textcolor{blue}{Brahm}{}\ledrightnote{\textcolor{blue}{Otto Brahm}}, \textcolor{blue}{Baſſermann}{}\ledrightnote{\textcolor{blue}{Albert Bassermann}}, \textcolor{blue}{Rittner}{}\ledrightnote{\textcolor{blue}{Rudolf Rittner}}, \textcolor{blue}{Sauer}{}\ledrightnote{\textcolor{blue}{Oskar Sauer}}; – es handelt
               ſich wohl um dein neues \label{K_L01326_1v}\edtext{\textcolor{green}{Stück}{}\ledrightnote{→\textcolor{green}{Der Meister}}}{\lemma{\textnormal{\emph{Stück}}}\Cendnote{\textnormal{\textcolor{blue}{Hermann Bahr}: \emph{\textcolor{green}{Der Meister. Komödie in drei Akten}}. Berlin: \emph{\textcolor{brown}{S. Fischer}}{ }1904.}}}\label{K_L01326_1h}? Hoffentlich ſeh ich dich ab\damage{er} noch vor deiner Abreiſe. Entweder komm ich auf eine viertel Stunde zu dir
               nach \textcolor{pink}{Ob Veit}{}\ledrightnote{\textcolor{pink}{Ober Sankt Veit}} – oder, man könnte ſich, ev. mit \textcolor{blue}{Hugo’s}{}\ledrightnote{\textcolor{blue}{Hugo von Hofmannsthal}{\newline}\textcolor{blue}{Gertrude von Hofmannsthal}} in \textcolor{pink}{Hietzing}{}\ledrightnote{\textcolor{pink}{XIII., Hietzing}} zu Abend u Abendeſſen treffen?\pend
           \pstart
           Herzlichſt dein{\\[\baselineskip]}\spacefill\mbox{Arthur.}\pend
           \leftskip=0em{}\endnumbering\briefempfaengerindex{Bahr, Hermann@\textsc{Bahr, Hermann}!zzzSchnitzler, Arthur@\emph{von Arthur Schnitzler}!1903-10-091@{9. 10. 1903}|)be}\mylabel{h}  \normalsize

\doendnotes{C}
\bigskip
\vfill

\clearpage

\footnotesize

\lohead{\textsc{register}}

% Definiere theindex-Environment komplett neu ohne reledmac
\makeatletter
\renewenvironment{theindex}{%
  \section*{\indexname}%
  \setlength{\parindent}{0pt}%
  \setlength{\parskip}{0pt plus 0.3pt}%
  \let\item\@idxitem
}{%
  \clearpage
}
\makeatother

\IfFileExists{\jobname-pw.ind}{\input{\jobname-pw.ind}}{}

\end{document}

      