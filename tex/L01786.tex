%% latex-korrekturansicht-vorspann.tex
%% Vorspann für die Korrekturansicht.
%% Lädt die gemeinsame Datei latex-vorspann.tex mit gesetztem Schalter.

\newif\ifkorrekturansicht
\korrekturansichttrue

\input{../tex-inputs/latex-vorspann}


               \section[Arthur Schnitzler an Hugo von Hofmannsthal, 6. 8. 1908]{ Arthur Schnitzler an Hugo von Hofmannsthal, 6. 8. 1908}\nopagebreak\mylabel{v}\rehead{ }\normalsize\beginnumbering\briefempfaengerindex{Hofmannsthal, Hugo von@\textsc{Hofmannsthal, Hugo von}!zzzSchnitzler, Arthur@\emph{von Arthur Schnitzler}!1908-08-061@{6. 8. 1908}|(be} \toendnotes[C]{\smallbreak\pagebreak[2]} \Standort{FDH, Hs-30885,132.}
\physDesc{Brief, 1 Blatt, 4 Seiten
\newline{}Handschrift: schwarze Tinte, lateinische Kurrent}\buchAbdrucke{\weitereDrucke{Hugo von Hofmannsthal, Arthur Schnitzler: \emph{Briefwechsel}. Hg. Therese Nickl und Heinrich Schnitzler. Frankfurt am Main: \emph{S. Fischer} 1964, S. 239.} }\toendnotes[C]{\smallbreak}\pstart
           {\pb}\textcolor{gray}{\textbf{Dr. Arthur Schnitzler}}\hfill \textcolor{pink}{Seis am Schlern}{}\ledrightnote{\textcolor{pink}{Seis am Schlern}},\pend
           \pstart
           \textcolor{gray}{\textbf{\textcolor{pink}{Wien XVIII. Spoettelgasse 7}{}\ledrightnote{\textcolor{pink}{Edmund-Weiß-Gasse}}.}}\hfill 6. 8. 08\pend
           \pstart{}lieber Hugo, \pend\pstart
           Sie sehen, wir sind noch i{\geminationm}er da, und wahrscheinlich
               bleiben wir bis ungefähr 20. we{\geminationn} nicht
               länger. Seit 10 Tagen ist \textcolor{blue}{Wasserma{\geminationn}}{}\ledrightnote{\textcolor{blue}{Jakob Wassermann}} hier, \textcolor{blue}{Agnes Speyer}{}\ledrightnote{\textcolor{blue}{Agnes Ulmann}}, Doctor \textcolor{blue}{Kaufmann}{}\ledrightnote{\textcolor{blue}{Arthur Kaufmann}}, und gestern sind wir von einer sehr schönen Partie
                  zurückgeko{\geminationm}en: – \textcolor{pink}{Seis}{}\ledrightnote{\textcolor{pink}{Seis am Schlern}} – \textcolor{pink}{Weisslahnbad}{}\ledrightnote{\textcolor{pink}{Weisslahnbad}} – \textcolor{pink}{Jungbru{\geminationn}thal}{}\ledrightnote{\textcolor{pink}{Jungbrunntal}} – \textcolor{pink}{Schlern}{}\ledrightnote{\textcolor{pink}{Schlern}} – \textcolor{pink}{Seis}{}\ledrightnote{\textcolor{pink}{Seis am Schlern}}; – besonders der
               (etwa 5stündg Spaziergang von hier nach \textcolor{pink}{Weisslahnbad}{}\ledrightnote{\textcolor{pink}{Weisslahnbad}} gehört zu {\pb}den schönsten, die man
               träumen ka{\geminationn}, und ist, wie die ganze Gegend, nicht
               berühmt genug. Vor 8 Tagen ist \textcolor{blue}{Brahm}{}\ledrightnote{\textcolor{blue}{Otto Brahm}} abgereist,
               der sich nicht weniger als drei Wochen lang hier aufgehalten hat, und, trotz allerlei
               mehr oder weniger fundirten Hypochondrien, in guter Laune und ebensolchem
               Wohlbefinden.\pend
           \pstart
           Von hier aus mach ich mit \textcolor{blue}{Olga}{}\ledrightnote{\textcolor{blue}{Olga Schnitzler}} eine kleine Reise;
               wohin steht noch nicht fest – \textcolor{pink}{Martino}{}\ledrightnote{\textcolor{pink}{San Martino di Castrozza}} oder \textcolor{pink}{Campiglio}{}\ledrightnote{\textcolor{pink}{Madonna di Campiglio}}, event. \textcolor{pink}{München}{}\ledrightnote{\textcolor{pink}{München}} zum Schluss. – Dass Sie zu {\pb}meinem \textcolor{green}{Roman}{}\ledrightnote{→\textcolor{green}{Der Weg ins Freie. Roman}} kein glückliches Verhältnis
               gefunden haben, war in der That nicht schwer zu merken. Und so sehr ich Ihrem
               Ausspruch beisti{\geminationm}e, dass Sie zwischen mir und meinen
               Arbeiten keine Grenze ziehen können, ich empfinde ihn als doppelt u. zehnfach wahr
               gegenüber einem Werk, das mich in Gedanke u Ausführung durch manches reife und \introOben{}höchst\introOben{} bewußte Jahr meines Lebens vornehmlich beschäftigt hat.
               Als so wahr erweist es sich, was Sie selbst zu spüren scheinen, wie es kaum denkbar
               ist, zum Dichter eines Werks, das für eine {\pb}ganze
               Entwicklungsperiode \substVorne{}\textsuperscript{eines}\substDazwischen{}\label{T_L01786_1v}\edtext{dieses}{\lemma{\textnormal{\emph{dieses}}}\Cendnote{\textnormal{In der ersten Schicht schrieb er »dieses«,
                        ersetzte es dann durch »eines«, um dann wieder zu
                           »dieses« zurückzukehren.}}}\label{T_L01786_1h}\substHinten{} Dichters bedeutend ist, in einem glücklichern Verhältnis zu stehen als zu
               der Dichtung selbst und dass ich Ihnen für den Takt dankbar bin, der es Sie als
               richtig erkennen liess, jedes weitre Wort über ein \textcolor{green}{Werk}{}\ledrightnote{\textcolor{green}{Der Weg ins Freie. Roman}} zu unterlassen, das nichts vermocht hat als Sie zu verstören und von
               dem mir ein unverlierbar und untrüglich starkes \introOben{}\strikeout{\textcolor{gray}{×}\-\textcolor{gray}{×}\-\textcolor{gray}{×}\-\textcolor{gray}{×}\-\textcolor{gray}{×}\-\textcolor{gray}{×}\-\textcolor{gray}{×}\-\textcolor{gray}{×}\-\textcolor{gray}{×}\-\textcolor{gray}{×}}\introOben{} Nachgefühl in der Seele geblieben ist. –\pend
           \pstart
           Auf Wiedersehen im Herbst; Sie bringen hoffentlich viel schönes zum
               vorlesen mit, – lassen Sie sichs beide in \textcolor{pink}{Sils}{}\ledrightnote{\textcolor{pink}{Sils im Engadin}}
               wohlergehen.\pend
           \pstart Wir grüßen herzlichst.\spacefill\mbox{Arthur.}\pend{}\endnumbering\briefempfaengerindex{Hofmannsthal, Hugo von@\textsc{Hofmannsthal, Hugo von}!zzzSchnitzler, Arthur@\emph{von Arthur Schnitzler}!1908-08-061@{6. 8. 1908}|)be}\mylabel{h}  \normalsize

\doendnotes{C}
\bigskip
\vfill

\clearpage

\footnotesize

\lohead{\textsc{register}}

% Definiere theindex-Environment komplett neu ohne reledmac
\makeatletter
\renewenvironment{theindex}{%
  \section*{\indexname}%
  \setlength{\parindent}{0pt}%
  \setlength{\parskip}{0pt plus 0.3pt}%
  \let\item\@idxitem
}{%
  \clearpage
}
\makeatother

\IfFileExists{\jobname-pw.ind}{\input{\jobname-pw.ind}}{}

\end{document}

      