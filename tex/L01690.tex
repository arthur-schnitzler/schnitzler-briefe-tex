%% latex-korrekturansicht-vorspann.tex
%% Vorspann für die Korrekturansicht.
%% Lädt die gemeinsame Datei latex-vorspann.tex mit gesetztem Schalter.

\newif\ifkorrekturansicht
\korrekturansichttrue

\input{../tex-inputs/latex-vorspann}


               \section[Hugo von Hofmannsthal an Arthur Schnitzler, 11. 7. 1907]{ Hugo von Hofmannsthal an Arthur Schnitzler, 11. 7. 1907}\nopagebreak\mylabel{v}\rehead{ }\normalsize\beginnumbering\briefempfaengerindex{Schnitzler, Arthur@\textsc{Schnitzler, Arthur}!zzzHofmannsthal, Hugo von@\emph{von Hugo von Hofmannsthal}!1907-07-112@{11. 7. 1907}|(be} \toendnotes[C]{\smallbreak\pagebreak[2]} \Standort{CUL, Schnitzler, B 43.}
\physDesc{Postkarte
\newline{}Handschrift: schwarze Tinte, deutsche Kurrent\newline{}Versand: 1) Stempel: »\nobreak{}\oindex{Cortina d'Ampezzo@\textbf{Cortina d'Ampezzo}, \emph{Besiedelter Ort (A.BSO)}|pwk}Cortina, 11. VII. 07\nobreak{}«.  2) Stempel: »\nobreak{}\oindex{Welsberg-Taisten@\textbf{Welsberg-Taisten}, \emph{Besiedelter Ort (A.BSO)}|pwk}Welsbe{[}rg{]}, \textcolor{gray}{12.}{[} 7. 1907{]}\nobreak{}«. 
\newline{}Schnitzler: mit Bleistift datiert: »11/7 90\textcolor{gray}{7}« \newline{}Ordnung: 1) mit Bleistift von unbekannter Hand nummeriert: »\strikeout{281}« 2) mit Bleistift von unbekannter Hand nummeriert:
                                    »283«}\buchAbdrucke{\weitereDrucke{Hugo von Hofmannsthal, Arthur Schnitzler: \emph{Briefwechsel}. Hg. Therese Nickl und Heinrich Schnitzler. Frankfurt am Main: \emph{S. Fischer} 1964, S. 230.} }\toendnotes[C]{\smallbreak}\pstart{}{\pb}\textsc{Herrn D\textsuperscript{r} Arthur Schnitzler}\pend{}\pstart{}\textsc{\textcolor{pink}{Wildbad Waldbrunn}{}\ledrightnote{\textcolor{pink}{Wildbad Waldbrunn}}}\pend{}\pstart{}\textsc{\textcolor{pink}{Welsberg}{}\ledrightnote{\textcolor{pink}{Welsberg-Taisten}}}\pend{}\pstart{}\textsc{\textcolor{pink}{Pusterthal}{}\ledrightnote{\textcolor{pink}{Pustertal}}}\pend{}{\bigskip}\pstart
           {\pb}\textcolor{pink}{\textsc{Cort.}}{}\ledrightnote{\textcolor{pink}{Cortina d'Ampezzo}}\hfill Donnerstag\pend
           \pstart
           Sie arbeiten von 2–5? Gut. Ich werde von \label{K_L01690_1v}\edtext{¼ 3 bis ¾ 5}{\lemma{\textnormal{\emph{¼ 3 bis ¾ 5}}}\Cendnote{\textnormal{von 14:15 Uhr bis 16:45 Uhr}}}\label{K_L01690_1h}{ }\textcolor{green}{arbeiten}{}\ledrightnote{→\textcolor{green}{Silvia im »Stern«}} und dafür das doppelte
               Honorar verlangen.\pend
           \pstart
           Wir sind \uline{Sonntag}{ }1\textsuperscript{h}10 nachmittags bei Ihnen. Freuen uns
               ſehr.\pend
           \pstart
           Von Herzen{\\[\baselineskip]}\spacefill\mbox{Hugo.}\pend
           \leftskip=0em{}\endnumbering\briefempfaengerindex{Schnitzler, Arthur@\textsc{Schnitzler, Arthur}!zzzHofmannsthal, Hugo von@\emph{von Hugo von Hofmannsthal}!1907-07-112@{11. 7. 1907}|)be}\mylabel{h}  \normalsize

\doendnotes{C}
\bigskip
\vfill

\clearpage

\footnotesize

\lohead{\textsc{register}}

% Definiere theindex-Environment komplett neu ohne reledmac
\makeatletter
\renewenvironment{theindex}{%
  \section*{\indexname}%
  \setlength{\parindent}{0pt}%
  \setlength{\parskip}{0pt plus 0.3pt}%
  \let\item\@idxitem
}{%
  \clearpage
}
\makeatother

\IfFileExists{\jobname-pw.ind}{\input{\jobname-pw.ind}}{}

\end{document}

      