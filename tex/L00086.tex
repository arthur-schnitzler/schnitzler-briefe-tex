%% latex-korrekturansicht-vorspann.tex
%% Vorspann für die Korrekturansicht.
%% Lädt die gemeinsame Datei latex-vorspann.tex mit gesetztem Schalter.

\newif\ifkorrekturansicht
\korrekturansichttrue

\input{../tex-inputs/latex-vorspann}


               \section[Hermann Bahr an Arthur Schnitzler, 20. 3. 1892]{ Hermann Bahr an Arthur Schnitzler, 20. 3. 1892}\nopagebreak\mylabel{v}\rehead{ }\normalsize\beginnumbering\briefempfaengerindex{Schnitzler, Arthur@\textsc{Schnitzler, Arthur}!zzzBahr, Hermann@\emph{von Hermann Bahr}!1892-03-201@{20. 3. 1892}|(be} \toendnotes[C]{\smallbreak\pagebreak[2]} \Standort{CUL, Schnitzler, B 5b.}
\physDesc{Postkarte
\newline{}Handschrift: schwarze Tinte, deutsche Kurrent\newline{}Versand: 1) Stempel: »\nobreak{}\oindex{I., Innere Stadt@\textbf{I., Innere Stadt}, \emph{Bezirk (A.BZK)}|pwk}Wien 1/1, 20. 3{[}.{]} 92, 7–8 N\nobreak{}«.  2) Stempel: »\nobreak{}Bestellt, \oindex{I., Innere Stadt@\textbf{I., Innere Stadt}, \emph{Bezirk (A.BZK)}|pwk}Wien 1/1, 21. 3. 92, 9–10½ V.\nobreak{}«. \newline{}Ordnung: mit rotem Buntstift von unbekannter Hand
                           und mit Bleistift jeweils nummeriert: »9« }\buchAbdrucke{\weitereDrucke{Hermann Bahr, Arthur Schnitzler: \emph{Briefwechsel, Aufzeichnungen, Dokumente (1891–1931)}. Hg. Kurt Ifkovits und Martin Anton Müller. Göttingen: \emph{Wallstein} 2018, S. 27.} }\pstart{}{\pb}Herrn \textsc{D\textsuperscript{r} Arthur Schnitzler}\pend{}\pstart{}Schriftsteller\pend{}\pstart{}\textcolor{pink}{\textsc{Wien I}}{}\ledrightnote{\textcolor{pink}{I., Innere Stadt}}\pend{}\pstart{}\textcolor{pink}{Kärntnerring 12}{}\ledrightnote{\textcolor{pink}{Kärntnerring}}\pend{}{\bigskip}\pstart
           \noindent{}{\pb}Lieber Freund! Ich habe ſofort an D\textsuperscript{r}{ }\textcolor{blue}{H.}{}\ledrightnote{\textcolor{blue}{H.}} in der angegebenen Weiſe geſchrieben u. wäre
               ſehr glücklich, wenn meine Bitte Erfolg haben ſollte.\pend
           \pstart
           Mit beſten Grüßen{\\[\baselineskip]}\spacefill\mbox{HermBahr}\pend
           \leftskip=0em{}\endnumbering\briefempfaengerindex{Schnitzler, Arthur@\textsc{Schnitzler, Arthur}!zzzBahr, Hermann@\emph{von Hermann Bahr}!1892-03-201@{20. 3. 1892}|)be}\mylabel{h}  \normalsize

\doendnotes{C}
\bigskip
\vfill

\clearpage

\footnotesize

\lohead{\textsc{register}}

% Definiere theindex-Environment komplett neu ohne reledmac
\makeatletter
\renewenvironment{theindex}{%
  \section*{\indexname}%
  \setlength{\parindent}{0pt}%
  \setlength{\parskip}{0pt plus 0.3pt}%
  \let\item\@idxitem
}{%
  \clearpage
}
\makeatother

\IfFileExists{\jobname-pw.ind}{\input{\jobname-pw.ind}}{}

\end{document}

      