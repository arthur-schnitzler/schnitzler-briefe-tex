%% latex-korrekturansicht-vorspann.tex
%% Vorspann für die Korrekturansicht.
%% Lädt die gemeinsame Datei latex-vorspann.tex mit gesetztem Schalter.

\newif\ifkorrekturansicht
\korrekturansichttrue

\input{../tex-inputs/latex-vorspann}


               \section[Arthur Schnitzler an Richard Beer-Hofmann, 1. 6. 1899]{ Arthur Schnitzler an Richard Beer-Hofmann, 1. 6. 1899}\nopagebreak\mylabel{v}\rehead{ }\normalsize\beginnumbering\briefempfaengerindex{Beer-Hofmann, Richard@\textsc{Beer-Hofmann, Richard}!zzzSchnitzler, Arthur@\emph{von Arthur Schnitzler}!1899-06-011@{1. 6. 1899}|(be} \toendnotes[C]{\smallbreak\pagebreak[2]} \Standort{YCGL, MSS 31.}
\physDesc{Brief, 2 Blätter, 8 Seiten, Umschlag
\newline{}Handschrift: 1) Bleistift, deutsche Kurrent\hspace{1em}2) schwarze Tinte, deutsche Kurrent (\noindent{}Umschlag)\hspace{1em}\newline{}Versand: 1) Stempel: »\nobreak{}\oindex{IX., Alsergrund@\textbf{IX., Alsergrund}, \emph{Bezirk (A.BZK)}|pwk}Wien 9/3, 2. 6. 99, 9–10V\nobreak{}«.  2) Stempel: »\nobreak{}\oindex{Seeboden@\textbf{Seeboden}, \emph{http://www.geonames.org/ontologyA.ADM3}|pwk}{\pb}See{[}boden{]}, 3. 6. {[}1899{]}\nobreak{}«. }\buchAbdrucke{\weitereDrucke{Arthur Schnitzler, Richard Beer-Hofmann: \emph{Briefwechsel 1891–1931}. Hg. Konstanze Fliedl. Wien, Zürich: \emph{Europaverlag} 1992, S. 128–129.} }\toendnotes[C]{\smallbreak}\pstart{}{\pb}\textsc{Herrn Dr Richard Beer-Hofmann }\pend{}\pstart{}\textcolor{pink}{\textsc{Kärnthen}}{}\ledrightnote{\textcolor{pink}{Kärnten}}\pend{}\pstart{}\textcolor{pink}{\textsc{Seeboden}}{}\ledrightnote{\textcolor{pink}{Seeboden}}\pend{}\pstart{}am \textcolor{pink}{\textsc{Millstätter}ſee}{}\ledrightnote{\textcolor{pink}{Millstätter See}}\pend{}\pstart{}\textsc{\textcolor{pink}{Villa Platzer}{}\ledrightnote{\textcolor{pink}{Villa Platzer}}}\pend{}{\bigskip}\pstart
           \raggedleft{}{\pb}1. 6. 99. \pend
           \pstart
           Mein lieber Richard,\pend
           \pstart
           die \label{K_L00921_1v}\edtext{Rieſenkarte}{\lemma{\textnormal{\emph{Rieſenkarte}}}\Cendnote{\textnormal{Die Karte vom 29. 5. 1899 ist größer als eine normale
                  Postkarte.}}}\label{K_L00921_1h} hab ich beko{\geminationm}en und danke für den
               lieben \label{K_L00921_2v}\edtext{Frozelgruſs}{\lemma{\textnormal{\emph{Frozelgruſs}}}\Cendnote{\textnormal{frotzeln, umgangssprachlich für:
                  necken}}}\label{K_L00921_2h}. – Hier iſt es traurig – immer trauriger – Frühling und einſam –
               und ich weiſs nicht was ich mit mir beginnen ſoll –\pend
           \pstart
           Jetzt eben, \label{K_L00921_3v}\edtext{Feiertag}{\lemma{\textnormal{\emph{Feiertag}}}\Cendnote{\textnormal{Fronleichnam}}}\label{K_L00921_3h}, Nachmittg, ſehr ſchön
               – und der Abend vor mir – und nebſtbei das »ganze« Leben – vollko{\geminationm}en {\pb}überflüſſig. –\pend
           \pstart
           \label{K_L00921_4v}\edtext{Neulich}{\lemma{\textnormal{\emph{Neulich}}}\Cendnote{\textnormal{siehe A. S.: \emph{Tagebuch}, 28. 5. 1899}}}\label{K_L00921_4h} war ich mit \textcolor{blue}{Hugo}{}\ledrightnote{\textcolor{blue}{Hugo von Hofmannsthal}}{ }\textcolor{pink}{Kampthal}{}\ledrightnote{\textcolor{pink}{Kamptal}} und \textcolor{pink}{Wachau}{}\ledrightnote{\textcolor{pink}{Wachau}}, die Abende auf dem Land ſind ſchauerlich – was da alles in der Luft
               ſchwebt – da verſtummen die Worte und verſiegen die Thränen. Ich habe Angſt vor dem
               Sommer, beſonders vor den Abenden, vor den Abenden am See –\pend
           \pstart
           – Zuckungen, als we{\geminationn} ich {\pb}arbeiten wollte hab ich ſchon zuweilen, aber weiter noch nichts. Vorläufig ſteht es
               noch immer ſo, daſs nur der \uline{eine} Gedanke mildert –
               nun, Sie wiſſen ja.\pend
           \pstart
           Nebstbei, ganz nebſtbei bringt mich auch das Ohrenſauſen langſam um – es iſt wahrhaft
               gräßlich, nicht eine Sekunde Ruhe zu haben und jeden Tag ein wenig nur {\pb}ein ganz klein wenig ſchlechter zu hören. –\pend
           \pstart
           Sie wiſſen ſchon, dſs der Direktor \textcolor{blue}{Schleſinger}{}\ledrightnote{\textcolor{blue}{Emil Schlesinger}}
               geſtern geſtorben ist. \label{K_L00921_5v}\edtext{Morgen vor 14
                  Tagen}{\lemma{\textnormal{\emph{Morgen vor 14
                  Tagen}}}\Cendnote{\textnormal{siehe A. S.: \emph{Tagebuch}, 19. 5. 1899}}}\label{K_L00921_5h} waren \textcolor{blue}{Hugo}{}\ledrightnote{\textcolor{blue}{Hugo von Hofmannsthal}} und ich mit \textcolor{blue}{ihm}{}\ledrightnote{→\textcolor{blue}{Emil Schlesinger}} auf der \textcolor{pink}{Rohrerhütte}{}\ledrightnote{\textcolor{pink}{Rohrerhütte}} zuſammen; er war heiſer und ſonſt »ganz geſund«. –\pend
           \pstart
           \label{K_L00921_6v}\edtext{Geſtern war \introOben{}auch\introOben{} das »\textcolor{green}{Vermächtnis}{}\ledrightnote{\textcolor{green}{Das Vermächtnis. Schauspiel in drei Akten}}«}{\lemma{\textnormal{\emph{Geſtern … »Vermächtnis«}}}\Cendnote{\textnormal{Es stand am \emph{\textcolor{brown}{Burgtheater}} noch immer am Spielplan.}}}\label{K_L00921_6h}. Kein gutes Klima, unſre
               Stücke. – \label{K_L00921_7v}\edtext{Zweimal}{\lemma{\textnormal{\emph{Zweimal}}}\Cendnote{\textnormal{am 25. 5. 1899 und am 30. 5. 1899}}}\label{K_L00921_7h} war ich {\pb}in \textcolor{pink}{Kaltenleutgeben}{}\ledrightnote{\textcolor{pink}{Kaltenleutgeben}}, bei \textcolor{blue}{Brahm}{}\ledrightnote{\textcolor{blue}{Otto Brahm}}. Er iſt ein
               nahezu wohlthuender Menſch. –\pend
           \pstart
           \label{K_L00921_8v}\edtext{Samſtag}{\lemma{\textnormal{\emph{Samſtag}}}\Cendnote{\textnormal{vgl. A. S.: \emph{Tagebuch}, 27. 5. 1899}}}\label{K_L00921_8h} beim »\textcolor{green}{Richter von Zalamea}{}\ledrightnote{\textcolor{green}{Der Richter von Zalamea}}«. \textcolor{blue}{Baumeiſter}{}\ledrightnote{\textcolor{blue}{Bernhard Baumeister}} unbeſchreiblich. Und das Stück! \textcolor{blue}{Hugo}{}\ledrightnote{\textcolor{blue}{Hugo von Hofmannsthal}} findet, daſs Sie noch am eheſten ſo eins ſchreiben könnten
               (er meint, unter »uns«, alſo: Sie, er, ich, \textcolor{blue}{Leo
                  Hirſchfeld}{}\ledrightnote{\textcolor{blue}{Leo Feld}}, \textcolor{blue}{Oskar Friedmann}{}\ledrightnote{\textcolor{blue}{Oskar Friedmann}}, \textcolor{blue}{Karlweis}{}\ledrightnote{\textcolor{blue}{Carl Karlweis}}) – ich hoffe {\pb}Sie laſſen ihn nicht in dem Glauben, – ſondern
               ſchreiben wirklich ein Stück.\pend
           \pstart
           Hören Sie: Ein jüdiſcher Selcher will \introOben{}im\introOben{}{ }So{\geminationm}er einmal auf ein
               paar Augenblicke ſein Local verlaſſen – die Thür iſt offen, wie er hinaustritt –
               liegt ein großer Hund da. Der Selcher denkt: Mach ich jetzt die Thür zu, ſo merkt
               doch jenner (der Hund) daſs {\pb}ich fort bin und ſpringt
               ſich durch die Glasſcheiben in mein Geſchäft und friſſt ſich meine Würſtel – ich
               laſſe doch lieber die Thür offen, werd er glauben, ich bin gar nicht eweg
               gegangen. –\pend
           \pstart
           – Er geht, ko{\geminationm}t nach einer Weile zurück, der Hund iſt im
               Geschäft und hat ſich richtig alle Würſtel aufgefreſſen. Der Selcher schüttelt {\pb}den Kopf und ſagt: »A ſo ä Dreh von dem Hund!«\pend
           \pstart
           – Schöneres ka{\geminationn} ich Ihnen heut nicht mehr \substVorne{}\textsuperscript{ſagen}\substDazwischen{}erzählen\substHinten{}! –\pend
           \pstart
           – Grüß Sie Gott. Schreiben Sie mir bald.\pend
           \pstart Ihr \spacefill\mbox{Arthur}\pend{}\endnumbering\briefempfaengerindex{Beer-Hofmann, Richard@\textsc{Beer-Hofmann, Richard}!zzzSchnitzler, Arthur@\emph{von Arthur Schnitzler}!1899-06-011@{1. 6. 1899}|)be}\mylabel{h}  \normalsize

\doendnotes{C}
\bigskip
\vfill

\clearpage

\footnotesize

\lohead{\textsc{register}}

% Definiere theindex-Environment komplett neu ohne reledmac
\makeatletter
\renewenvironment{theindex}{%
  \section*{\indexname}%
  \setlength{\parindent}{0pt}%
  \setlength{\parskip}{0pt plus 0.3pt}%
  \let\item\@idxitem
}{%
  \clearpage
}
\makeatother

\IfFileExists{\jobname-pw.ind}{\input{\jobname-pw.ind}}{}

\end{document}

      