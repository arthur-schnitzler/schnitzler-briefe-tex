%% latex-korrekturansicht-vorspann.tex
%% Vorspann für die Korrekturansicht.
%% Lädt die gemeinsame Datei latex-vorspann.tex mit gesetztem Schalter.

\newif\ifkorrekturansicht
\korrekturansichttrue

\input{../tex-inputs/latex-vorspann}


               \section[Hugo von Hofmannsthal und Hermine Benedict an Arthur Schnitzler, 21. {[}8. 1896{]}]{ Hugo von Hofmannsthal und Hermine Benedict an Arthur Schnitzler,
                    21. {[}8. 1896{]}}\nopagebreak\mylabel{v}\rehead{ }\normalsize\beginnumbering\briefempfaengerindex{Schnitzler, Arthur@\textsc{Schnitzler, Arthur}!zzzSchaffgotsch, Hermine von@\emph{von Hermine von Schaffgotsch}!1896-08-211@{21. {[}8. 1896{]}}|(be}\briefempfaengerindex{Schnitzler, Arthur@\textsc{Schnitzler, Arthur}!zzzHofmannsthal, Hugo von@\emph{von Hugo von Hofmannsthal}!1896-08-211@{21. {[}8. 1896{]}}|(be} \toendnotes[C]{\smallbreak\pagebreak[2]} \Standort{CUL, Schnitzler, B 43.}
\physDesc{Brief, 1 Blatt, 4 Seiten
\newline{}Handschrift Hugo von Hofmannsthal: schwarze Tinte, deutsche Kurrent\newline{}Handschrift Hermine von Schaffgotsch: schwarze Tinte, deutsche Kurrent
\newline{}Schnitzler: mit Bleistift Monat und Jahr ergänzt: »Aug. 96« \newline{}Ordnung: mit Bleistift von unbekannter Hand nummeriert:
                                        »79« }\buchAbdrucke{\weitereDrucke{Hugo von Hofmannsthal, Arthur Schnitzler: \emph{Briefwechsel}. Hg. Therese Nickl und Heinrich Schnitzler. Frankfurt am Main: \emph{S. Fischer} 1964, S. 72–74.} }\toendnotes[C]{\smallbreak}\pstart
           \raggedleft{}{\pb}\textcolor{pink}{Alt.auſſee}{}\ledrightnote{\textcolor{pink}{Altaussee}}{ }21\textsuperscript{ten}\pend
           \pstart{}lieber Arthur!\pend\pstart
           {[}hs. Schaffgotsch:{]} Ihre erſtaunten Augen beim Eröffnen dieſes \label{K_L00580_1v}\edtext{Briefes}{\lemma{\textnormal{\emph{Briefes}}}\Cendnote{\textnormal{vgl. A. S.: \emph{Tagebuch}, 26. 8. 1896}}}\label{K_L00580_1h}{\\}{[}hs. Hofmannsthal:{]} zu ſehen intereſſiert mich weniger als zu erfahren,
                    wie Ihr \textcolor{blue}{vier}{}\ledrightnote{→\textcolor{blue}{Richard Beer-Hofmann}{\newline}→\textcolor{blue}{Paula Beer-Hofmann}{\newline}→\textcolor{blue}{Paul Goldmann}}
                    Menſchen {\\}{[}hs. Schaffgotsch:{]} beſonders \textcolor{blue}{Richard}{}\ledrightnote{\textcolor{blue}{Richard Beer-Hofmann}} und \textcolor{blue}{Paula}{}\ledrightnote{\textcolor{blue}{Paula Beer-Hofmann}}, von der man
                    nicht recht weiß, {\\}{[}hs. Hofmannsthal:{]} ob ſie außer der Seekrankheit noch
                    etwas merkwürdiges in \textcolor{pink}{Dänemark}{}\ledrightnote{\textcolor{pink}{Dänemark}} erlebt hat
                        {\\}{[}hs. Schaffgotsch:{]}  (und ob das Mädchen mit dem Loch im Strumpf
                    ſchon »die Epiſode« gena{\geminationn}t werden darf {\\}{[}hs. Hofmannsthal:{]} weiß man ja auch nicht) Euch befindet.\pend
           \pstart
           Von \textcolor{blue}{Paul}{}\ledrightnote{\textcolor{blue}{Paul Goldmann}} hab ich immer die Empfindung, er
                        {\\}{[}hs. Schaffgotsch:{]} erinnert ſich auch ſo gut an die Heroinenzeit
                    beim »\textcolor{pink}{\textsc{Leopold}}{}\ledrightnote{\textcolor{pink}{Hotel und Pension Rudolfshöhe (Leopold Petter)}}« in \textcolor{pink}{\textsc{Ischl}}{}\ledrightnote{\textcolor{pink}{Bad Ischl}} vor 2 Jahren \pend
           \pstart
           {\pb}{[}hs. Hofmannsthal:{]} wie wir alle, aber gar nicht mehr ordentlich an mich
                    und ich hab ihn wirklich {\\}{[}hs. Schaffgotsch:{]} nur einmal geſehen und ka{\geminationn} da- her unmöglich ſo warm empfinden wie jener
                    Dichter.\pend
           \pstart
           {[}hs. Hofmannsthal:{]} Ich verlange mir ſehr zu wiſſen, ob das was wir
                    einmal in der Nacht nach der \textsc{Soirée}{\\}{[}hs. Schaffgotsch:{]} beſprochen, auf Wahrheit beruht – mir will ſcheinen
                    – nein – 3mal Nein!! {\\}{[}hs. Hofmannsthal:{]} ich hoffe ja!: daſs Sie einmal
                    für ein paar Wochen von allen inneren Gewöhnungen losgeko{\geminationm}en, {\\}{[}hs. Schaffgotsch:{]} iſt für Sie
                    wahrſcheinlich ſehr gut, aber \introOben{}für\introOben{} das, was Sie früher
                    beſchäftigt, recht traurig. {\\}{[}hs. Hofmannsthal:{]} Umſo beſſer! – Daſs Sie
                    in dem zweiten \textcolor{green}{Act}{}\ledrightnote{→\textcolor{green}{Freiwild. Schauspiel in 3 Akten}} dem \textcolor{green}{Mädel}{}\ledrightnote{→\textcolor{green}{Freiwild. Schauspiel in 3 Akten}} mehr Leben gegeben
                    haben, wird ſicher {\\}{[}hs. Schaffgotsch:{]} eine große Wirkung haben, denn
                    wir haben ja ſchon oft beſprochen, daß die \textcolor{green}{Christine}{}\ledrightnote{→\textcolor{green}{Liebelei. Schauspiel in drei Akten}} davon nicht genug habe {\\}{[}hs. Hofmannsthal:{]} und das \textcolor{green}{Stück}{}\ledrightnote{→\textcolor{green}{Freiwild. Schauspiel in 3 Akten}} braucht Rührung, ſonſt wird es trocken und revoltierend. Meine
                        {\\}{[}hs. Schaffgotsch:{]} Neugierde, es zu leſen, kennt keine Grenzen,
                    denn wenn man Leute nicht oft ſieht, muſs man in ihren Zeilen leſen \pend
           \pstart
           {\pb}{[}hs. Hofmannsthal:{]} und das iſt ſchwer, denn leider drücken immer nur
                    einzelne kleine Sachen das Wirkliche aus, {\\}{[}hs. Schaffgotsch:{]} während
                    große Thaten und große Züge, die darauf angelegt ſind, charakteriſtiſch zu
                    wirken, eine ganze Welt von Mißverſtändniſſen hervorrufen.\pend
           \pstart
           {[}hs. Hofmannsthal:{]} Werden wir heuer endlich theaterſpielen? { }ſind wir zu jung oder zu alt dazu? Oder zu ernſt,
                    oder {\\}{[}hs. Schaffgotsch:{]} »zu alt, um nur zu ſpielen«? Jedenfalls müſste
                    die weibliche Hauptrolle diesmal nicht von Ihnen geſchrieben ſein, {\\}{[}hs. Hofmannsthal:{]} (warum?). Meine \textcolor{green}{Novelle}{}\ledrightnote{\textcolor{green}{Geschichte der beiden Liebespaare}}
                    werden Sie nie ſehen. Nie heißt nie. Weil ſie ſo ſchlecht iſt. {\\}{[}hs. Schaffgotsch:{]} Er zeigt nicht einmal die guten Sachen herzu. Doch \uline{müſste} man ihn manchmal leſen, we{\geminationn} die Perſon undeutlich wird. {\\}{[}hs. Hofmannsthal:{]} Freilich haben meine Sachen wieder das Häßliche, daſs alles
                    allzudeutlich geſagt iſt. Ob der \textcolor{blue}{Richard}{}\ledrightnote{\textcolor{blue}{Richard Beer-Hofmann}}{\\}{[}hs. Schaffgotsch:{]} wieder etwas ſchreibt, iſt, wie ich reumüthig
                    bekenne, für uns \textcolor{pink}{\textsc{Altausseer}}{}\ledrightnote{\textcolor{pink}{Altaussee}} ganz intereſſant, {\\}{[}hs. Hofmannsthal:{]} ich verſuche mir manchmal
                        vor\introOben{}zu\introOben{}ſtellen wie es wäre, wenn Sie hier wären
                        {\\}{[}hs. Schaffgotsch:{]} und ob wir alle Drei dabei nicht \uline{viel} netter herauskämen, was ich ganz beſtimmt glaube;
                    ſeien Sie \pend
           \pstart
           {\pb}{[}hs. Hofmannsthal:{]} nicht bös, aber ich bin ſicher wir würden uns { }ſchrecklich nervös machen und beinahe ſtreiten,
                    denn {\\}{[}hs. Schaffgotsch:{]} zwei noch ſo gute, gleichgeartete, männliche
                    Naturen haben nicht die Größe nett neben einander einherzugehen {\\}{[}hs. Hofmannsthal:{]} wenn zwiſchen ihnen etwas Halbwahres beunruhigend
                    herumwimmelt. Deswegen {\\}{[}hs. Schaffgotsch:{]} werden Sie doch herkommen,
                    ſchon allein um \strikeout{J}dieſe jugendliche Behauptung von
                        »\uuline{Halb}\uline{wahr}« zu widerlegen, {\\}{[}hs. Hofmannsthal:{]} wozu Sie ja durch Ihre oft beſprochene Überſchätzung der weiblichen
                    »Individualitäten« ſo geeignet ſind. {\\}{[}hs. Schaffgotsch:{]} Glücklich der,
                    welcher imſtande iſt, Geſtalten zu ſchaffen, an die er glaubt, drum laſſen Sie
                    ſich nicht hetzen, {\\}{[}hs. Hofmannsthal:{]} ſondern glauben Sie ruhig weiter,
                    auf das Wirkliche kommt’s nicht an, denn vielleicht exiſtiert es gar nicht.
                        {\\}{[}hs. Schaffgotsch:{]} Ich glaube, wir brauchen Sie darüber nicht
                    aufzuklären, Sie haben ein ſo ſtarkes Wahrheitsgefühl, {\\}{[}hs. Hofmannsthal:{]} daſs Sie auch den dreifachen Sinn dieſes Briefes erkannt
                    haben werden, worüber Sie nächſtens in \textcolor{pink}{Wien}{}\ledrightnote{\textcolor{pink}{Wien}} mir
                    (nur hier) Auskunft geben können.\pend
           \pstart
           Herzlich Ihr{\\[\baselineskip]}\spacefill\mbox{Hugo.}\pend
           \leftskip=0em{}\endnumbering\briefempfaengerindex{Schnitzler, Arthur@\textsc{Schnitzler, Arthur}!zzzSchaffgotsch, Hermine von@\emph{von Hermine von Schaffgotsch}!1896-08-211@{21. {[}8. 1896{]}}|)be}\briefempfaengerindex{Schnitzler, Arthur@\textsc{Schnitzler, Arthur}!zzzHofmannsthal, Hugo von@\emph{von Hugo von Hofmannsthal}!1896-08-211@{21. {[}8. 1896{]}}|)be}\mylabel{h}  \normalsize

\doendnotes{C}
\bigskip
\vfill

\clearpage

\footnotesize

\lohead{\textsc{register}}

% Definiere theindex-Environment komplett neu ohne reledmac
\makeatletter
\renewenvironment{theindex}{%
  \section*{\indexname}%
  \setlength{\parindent}{0pt}%
  \setlength{\parskip}{0pt plus 0.3pt}%
  \let\item\@idxitem
}{%
  \clearpage
}
\makeatother

\IfFileExists{\jobname-pw.ind}{\input{\jobname-pw.ind}}{}

\end{document}

      