%% latex-korrekturansicht-vorspann.tex
%% Vorspann für die Korrekturansicht.
%% Lädt die gemeinsame Datei latex-vorspann.tex mit gesetztem Schalter.

\newif\ifkorrekturansicht
\korrekturansichttrue

\input{../tex-inputs/latex-vorspann}


               \section[Arthur Schnitzler an Robert Adam, 4. 6. 1912]{ Arthur Schnitzler an Robert Adam, 4. 6. 1912}\nopagebreak\mylabel{v}\rehead{ }\normalsize\beginnumbering\briefempfaengerindex{Adam, Robert@\textsc{Adam, Robert}!zzzSchnitzler, Arthur@\emph{von Arthur Schnitzler}!1912-06-041@{4. 6. 1912}|(be} \toendnotes[C]{\smallbreak\pagebreak[2]} \Standort{DLA, 96.34.1/8.}
\physDesc{Postkarte, Umschlag
\newline{}Umschlag mit Schreibmaschine
\newline{}Handschrift: schwarze Tinte, deutsche Kurrent\newline{}Versand: Stempel: »\nobreak{}\oindex{XVIII., Waehring@\textbf{XVIII., Währing}, \emph{Bezirk (A.BZK)}|pwk}18/1 Wien, 4. VI. {[}1912{]}\nobreak{}«.  \newline{}Ordnung: mit Bleistift von unbekannter Hand Kuvert datiert: »Mai 1912« }\pstart{}{\pb}{[}ms.:{]} Herrn Dr. R. A. Pollak\pend{}\pstart{}Bezirksrichter\pend{}\pstart{}\textcolor{pink}{Zistersdorf}{}\ledrightnote{\textcolor{pink}{Zistersdorf}}.\pend{}\pstart{}\textcolor{pink}{N.Oe.}{}\ledrightnote{\textcolor{pink}{Niederösterreich}}\pend{}{\bigskip}\pstart
           \noindent{}{\pb}Herzlichſten Dank\pend
           \pstart \spacefill\mbox{Arthur Schnitzler}\pend{}\pstart
           \textcolor{pink}{Wien}{}\ledrightnote{\textcolor{pink}{Wien}}, im Mai 1912\pend
           \endnumbering\briefempfaengerindex{Adam, Robert@\textsc{Adam, Robert}!zzzSchnitzler, Arthur@\emph{von Arthur Schnitzler}!1912-06-041@{4. 6. 1912}|)be}\mylabel{h}  \normalsize

\doendnotes{C}
\bigskip
\vfill

\clearpage

\footnotesize

\lohead{\textsc{register}}

% Definiere theindex-Environment komplett neu ohne reledmac
\makeatletter
\renewenvironment{theindex}{%
  \section*{\indexname}%
  \setlength{\parindent}{0pt}%
  \setlength{\parskip}{0pt plus 0.3pt}%
  \let\item\@idxitem
}{%
  \clearpage
}
\makeatother

\IfFileExists{\jobname-pw.ind}{\input{\jobname-pw.ind}}{}

\end{document}

      