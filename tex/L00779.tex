%% latex-korrekturansicht-vorspann.tex
%% Vorspann für die Korrekturansicht.
%% Lädt die gemeinsame Datei latex-vorspann.tex mit gesetztem Schalter.

\newif\ifkorrekturansicht
\korrekturansichttrue

\input{../tex-inputs/latex-vorspann}


               \section[Richard Beer-Hofmann an Arthur Schnitzler, Februar 1898]{ Richard Beer-Hofmann an Arthur Schnitzler, Februar 1898}\nopagebreak\mylabel{v}\rehead{ }\normalsize\beginnumbering\briefempfaengerindex{Schnitzler, Arthur@\textsc{Schnitzler, Arthur}!zzzBeer-Hofmann, Richard@\emph{von Richard Beer-Hofmann}!1898-02-011@{Februar 1898}|(be} \toendnotes[C]{\smallbreak\pagebreak[2]} \Standort{CUL, Schnitzler, B 8.}
\physDesc{Notiz auf Konzeptpapier2 Blätter, 3 Seiten
\newline{}Handschrift: Bleistift, deutsche Kurrent
\newline{}Schnitzler: mit Bleistift datiert: »Feber 98« }\buchAbdrucke{\weitereDrucke{Arthur Schnitzler, Richard Beer-Hofmann: \emph{Briefwechsel 1891–1931}. Hg. Konstanze Fliedl. Wien, Zürich: \emph{Europaverlag} 1992, S. 115–116.} }\toendnotes[C]{\smallbreak}\pstart
           \noindent{}{\pb}\uline{\label{K_L00779-1v}\edtext{\textcolor{green}{Der Andere}{}\ledrightnote{\textcolor{green}{Der Andere. Aus dem Tagebuch eines Hinterbliebenen}}}{\lemma{\textnormal{\emph{Der Andere}}}\Cendnote{\textnormal{hier und in Folge Überlegungen,
                     welche Texte \textcolor{blue}{Schnitzler} für seine erste
                     Sammlung von Novellen verwenden solle, die wenige Wochen später als \emph{\textcolor{green}{Die Frau des Weisen}} erschien}}}\label{K_L00779-1h}.}\pend
           \pstart
           Wenn möglich anstatt »\textcolor{green}{Gattin}{}\ledrightnote{→\textcolor{green}{Der Andere. Aus dem Tagebuch eines Hinterbliebenen}}«
               etwas anderes.\pend
           \pstart
           »\textcolor{green}{\uline{blonde}{ }\uline{junge}{ }\uline{schöne} Mann}{}\ledrightnote{→\textcolor{green}{Der Andere. Aus dem Tagebuch eines Hinterbliebenen}}«\pend
           \pstart
           Jedenfalls aufnehmen in die Sa{\geminationm}lung\pend
           \pstart
           \numberlinefalse{}\centering{}–\numberlinetrue{}\pend
           \pstart
           \noindent{}\uline{\textcolor{green}{Amerika}{}\ledrightnote{\textcolor{green}{Amerika}}}\pend
           \pstart
           »\textcolor{green}{Sie liegt mir zu Füßen den Lockenkopf
                  an mein Knie gelehnt}{}\ledrightnote{→\textcolor{green}{Amerika}}«. Das ist aber schrecklich.\pend
           \pstart
           »\textcolor{green}{die süße weiße Hautstelle hinter dem
                  Ohr}{}\ledrightnote{→\textcolor{green}{Amerika}}«\pend
           \pstart
           »\textcolor{green}{Eine Fülle von Erinnerungen steigt in
                  mir auf}{}\ledrightnote{→\textcolor{green}{Amerika}}«\pend
           \pstart
           vorher noch »\textcolor{green}{und stille ist’s im
                  Gemach}{}\ledrightnote{→\textcolor{green}{Amerika}}«\pend
           \pstart
           {\pb}\textcolor{green}{\strikeout{Die kleine}}{}\ledrightnote{\textcolor{green}{Die kleine Komödie}}\pend
           \pstart
           {\pb}Könnte aufgenommen werden wenn
                  \introOben{}stark\introOben{}{ }\uline{überarbeitet}. Aber es sind so viele Sachen drinn die
               wegmüssten.\pend
           \pstart
           »\textcolor{green}{wie von ihren rothen Lippen der Ruf
                  erschallte.}{}\ledrightnote{→\textcolor{green}{Amerika}}« u. s. w.\pend
           \noindent\rule{\textwidth}{0.5pt}\pstart
           Bei »\textcolor{green}{mein \uline{Freund Ypsilon}}{}\ledrightnote{\textcolor{green}{Mein Freund Ypsilon}}« ist sehr schad um die Idee. \uline{Aber gewiss} nicht
               aufnehmen\pend
           \noindent\rule{\textwidth}{0.5pt}\pstart
           »\textcolor{green}{Die kleine Komödie}{}\ledrightnote{\textcolor{green}{Die kleine Komödie}}«\pend
           \pstart
           etwas kürzen – nicht viel und aufnehmen. Sie ist anspruchslos und hat keinen
               pretentiösen Ton\pend
           \endnumbering\briefempfaengerindex{Schnitzler, Arthur@\textsc{Schnitzler, Arthur}!zzzBeer-Hofmann, Richard@\emph{von Richard Beer-Hofmann}!1898-02-011@{Februar 1898}|)be}\mylabel{h}  \normalsize

\doendnotes{C}
\bigskip
\vfill

\clearpage

\footnotesize

\lohead{\textsc{register}}

% Definiere theindex-Environment komplett neu ohne reledmac
\makeatletter
\renewenvironment{theindex}{%
  \section*{\indexname}%
  \setlength{\parindent}{0pt}%
  \setlength{\parskip}{0pt plus 0.3pt}%
  \let\item\@idxitem
}{%
  \clearpage
}
\makeatother

\IfFileExists{\jobname-pw.ind}{\input{\jobname-pw.ind}}{}

\end{document}

      