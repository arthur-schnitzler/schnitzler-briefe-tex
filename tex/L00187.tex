%% latex-korrekturansicht-vorspann.tex
%% Vorspann für die Korrekturansicht.
%% Lädt die gemeinsame Datei latex-vorspann.tex mit gesetztem Schalter.

\newif\ifkorrekturansicht
\korrekturansichttrue

\input{../tex-inputs/latex-vorspann}


               \section[Eduard Michael Kafka an Arthur Schnitzler, 7. 3. 1893]{ Eduard Michael Kafka an Arthur Schnitzler, 7. 3. 1893}\nopagebreak\mylabel{v}\rehead{ }\normalsize\beginnumbering\briefempfaengerindex{Schnitzler, Arthur@\textsc{Schnitzler, Arthur}!zzzKafka, Eduard Michael@\emph{von Eduard Michael Kafka}!1893-03-071@{7. 3. 1893}|(be} \toendnotes[C]{\smallbreak\pagebreak[2]} \Standort{DLA, A:Schnitzler, HS.NZ85.1.3604.}
\physDesc{Brief, 1 Blatt, 1 Seite
\newline{}Handschrift: schwarze Tinte, deutsche Kurrent
\newline{}Schnitzler: mit rotem Buntstift eine Unterstreichung }\pstart
           \noindent{}\centering{}{\pb}\textcolor{gray}{\textbf{\textcolor{blue}{Wilh. Sundermeyer}{}\ledrightnote{\textcolor{blue}{Wilhelm Sundermeyer}}}}\pend
           \pstart
           \noindent{}\centering{}\textcolor{gray}{\textbf{\textcolor{pink}{Bahnhof Kreiensen}{}\ledrightnote{\textcolor{pink}{Bahnhof}}.}}\pend
           \pstart
           \raggedleft{}\textcolor{gray}{\textbf{\textcolor{pink}{Kreiensen}{}\ledrightnote{\textcolor{pink}{Kreiensen}}, den }}7/III \textcolor{gray}{\textbf{189}}3.\pend
           \pstart{}Lieber Schnitzler,\pend\pstart
           bitte, wollen Sie die Güte haben, mir ein Ex. »\textcolor{green}{Anatol}{}\ledrightnote{\textcolor{green}{Anatol}}« möglichſt umgehend nach \textcolor{pink}{München}{}\ledrightnote{\textcolor{pink}{München}}, oder beſſer nach \uline{\textcolor{pink}{Mannheim}{}\ledrightnote{\textcolor{pink}{Mannheim}}} (\textcolor{pink}{Pfälzer Hof}{}\ledrightnote{\textcolor{pink}{Pfälzer Hof}}) ſenden. –\pend
           \pstart
           Es that mir ſehr leid, Sie vor einigen Tagen, als ich über \textcolor{pink}{Brünn}{}\ledrightnote{\textcolor{pink}{Brünn}} u. \textcolor{pink}{Prag}{}\ledrightnote{\textcolor{pink}{Prag}}, ein paar
                    Stunden in \textcolor{pink}{Wien}{}\ledrightnote{\textcolor{pink}{Wien}} weilte, nicht getroffen zu
                    haben.\pend
           \pstart
           Man erzählte mir Trauriges von \textcolor{blue}{Fels}{}\ledrightnote{\textcolor{blue}{Friedrich Michael Fels}}; es war
                    mir eine warme Freude, zu hören, daſs Sie ſich ſeiner nach Kräften annehmen.
                    Bitte, ſchreiben Sie mir doch gütigſt ein paar Zeilen, wie es ihm geht, – oder,
                    lieber, ſenden Sie mir seine Adreſſe; ich will, da ich ihm nun doch wol kaum
                    mehr werde beſuchen können – vor meiner \textcolor{pink}{ſchwediſch}{}\ledrightnote{\textcolor{pink}{Schweden}}-\textcolor{pink}{norwegiſchen}{}\ledrightnote{\textcolor{pink}{Norwegen}} Reiſe –
                    gerne ein paar Zeilen an ihn richten.\pend
           \pstart
           Leben Sie recht wohl, lieber Freund, u. ſeien Sie herzlichſt gegrüßt\pend
           \pstart
           von Ihrem getreuen{\\[\baselineskip]}\spacefill\mbox{EMKafka}\pend
           \leftskip=0em{}\endnumbering\briefempfaengerindex{Schnitzler, Arthur@\textsc{Schnitzler, Arthur}!zzzKafka, Eduard Michael@\emph{von Eduard Michael Kafka}!1893-03-071@{7. 3. 1893}|)be}\mylabel{h}  \normalsize

\doendnotes{C}
\bigskip
\vfill

\clearpage

\footnotesize

\lohead{\textsc{register}}

% Definiere theindex-Environment komplett neu ohne reledmac
\makeatletter
\renewenvironment{theindex}{%
  \section*{\indexname}%
  \setlength{\parindent}{0pt}%
  \setlength{\parskip}{0pt plus 0.3pt}%
  \let\item\@idxitem
}{%
  \clearpage
}
\makeatother

\IfFileExists{\jobname-pw.ind}{\input{\jobname-pw.ind}}{}

\end{document}

      