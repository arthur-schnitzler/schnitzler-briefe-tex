%% latex-korrekturansicht-vorspann.tex
%% Vorspann für die Korrekturansicht.
%% Lädt die gemeinsame Datei latex-vorspann.tex mit gesetztem Schalter.

\newif\ifkorrekturansicht
\korrekturansichttrue

\input{../tex-inputs/latex-vorspann}


               \section[Gerhard Wiese an Arthur Schnitzler, 21. 3. 1894]{ Gerhard Wiese an Arthur Schnitzler, 21. 3. 1894}\nopagebreak\mylabel{v}\rehead{ }\normalsize\beginnumbering\briefempfaengerindex{Schnitzler, Arthur@\textsc{Schnitzler, Arthur}!zzzWiese, Gerhard@\emph{von Gerhard Wiese}!1894-03-212@{21. 3. 1894}|(be} \toendnotes[C]{\smallbreak\pagebreak[2]} \Standort{CUL, Schnitzler, B 15.}
\physDesc{Brief, 1 Blatt, 2 Seiten
\newline{}Handschrift: schwarze Tinte, deutsche Kurrent
\newline{}Schnitzler: 1) mit Bleistift auf der Rückseite beschriftet: »\textsc{(\textcolor{blue}{Blumenthal})}« 2) mit rotem Buntstift nummeriert: »6«\newline{}Ordnung: mit Bleistift von unbekannter Hand nummeriert:
                                                »6« }\toendnotes[C]{\smallbreak}\pstart
           \noindent{}\centering{}{\pb}\textcolor{gray}{\textbf{\textcolor{brown}{LESSING-THEATER}{}\ledrightnote{\textcolor{brown}{Lessing-Theater}}}}\pend
           \pstart
           \noindent{}\centering{}\textcolor{gray}{\textbf{Director:}}{\\}\textcolor{gray}{\textbf{Dr. Oscar Blumenthal.}}\pend
           \pstart
           \noindent{}\raggedleft{}\textcolor{gray}{\textbf{\textcolor{pink}{Berlin N.W.}{}\ledrightnote{\textcolor{pink}{Berlin}}, den}}{ }21. März \textcolor{gray}{\textbf{189}}4.{\\}\textcolor{gray}{\textbf{\textcolor{pink}{Friedrich-Carl-Ufer}{}\ledrightnote{\textcolor{pink}{Kapelle-Ufer}}}}.\pend
           \pstart\center{}Sehr geehrter Herr!\pend\pstart
           Wie Sie aus beiliegendem Wochenſpielplan erſehen, iſt die Frage, welcher \textcolor{green}{Einakter}{}\ledrightnote{→\textcolor{green}{Der Eisenfresser}} nach »\textcolor{green}{Niobe}{}\ledrightnote{\textcolor{green}{Niobe}}« gegeben werden ſoll, bereits
                    entſchieden. Herr \textsc{Dr. \textcolor{blue}{Oscar
                            Blumenthal}{}\ledrightnote{\textcolor{blue}{Oskar Blumenthal}}} weilt zur Zeit in \textsc{\textcolor{pink}{Moscau}{}\ledrightnote{\textcolor{pink}{Moskau}}} und kehrt vorausſichtlich erſt Ende April nach \textcolor{pink}{Berlin}{}\ledrightnote{\textcolor{pink}{Berlin}} zurück. Wir ſtellen Ihnen ergebenſt anheim, alsdann
                    auf den Inhalt Ihres jüngſten Schreibens zurückzukommen.\pend
           \pstart
           Hochachtungsvoll\pend
           \pstart
           \raggedleft{}\textcolor{gray}{\textbf{\textit{Die Direction}}}{\\}\textcolor{gray}{\textbf{\textit{des}}}{\\}\textcolor{gray}{\textbf{\textit{\textcolor{brown}{Lessing-Theaters}{}\ledrightnote{\textcolor{brown}{Lessing-Theater}}.}}}\pend
           \pstart \spacefill\mbox{Wieſe}\pend{}\endnumbering\briefempfaengerindex{Schnitzler, Arthur@\textsc{Schnitzler, Arthur}!zzzWiese, Gerhard@\emph{von Gerhard Wiese}!1894-03-212@{21. 3. 1894}|)be}\mylabel{h}  \normalsize

\doendnotes{C}
\bigskip
\vfill

\clearpage

\footnotesize

\lohead{\textsc{register}}

% Definiere theindex-Environment komplett neu ohne reledmac
\makeatletter
\renewenvironment{theindex}{%
  \section*{\indexname}%
  \setlength{\parindent}{0pt}%
  \setlength{\parskip}{0pt plus 0.3pt}%
  \let\item\@idxitem
}{%
  \clearpage
}
\makeatother

\IfFileExists{\jobname-pw.ind}{\input{\jobname-pw.ind}}{}

\end{document}

      