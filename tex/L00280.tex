%% latex-korrekturansicht-vorspann.tex
%% Vorspann für die Korrekturansicht.
%% Lädt die gemeinsame Datei latex-vorspann.tex mit gesetztem Schalter.

\newif\ifkorrekturansicht
\korrekturansichttrue

\input{../tex-inputs/latex-vorspann}


               \section[Arthur Schnitzler an Hermann Bahr, 7. 11. 1893]{ Arthur Schnitzler an Hermann Bahr, 7. 11. 1893}\nopagebreak\mylabel{v}\rehead{ }\normalsize\beginnumbering\briefempfaengerindex{Bahr, Hermann@\textsc{Bahr, Hermann}!zzzSchnitzler, Arthur@\emph{von Arthur Schnitzler}!1893-11-071@{7. 11. 1893}|(be} \toendnotes[C]{\smallbreak\pagebreak[2]} \Standort{TMW, HS AM 23323 Ba.}
\physDesc{Brief, 1 Blatt (Briefpapier mit Trauerrand), 3 Seiten
\newline{}Handschrift: schwarze Tinte, deutsche Kurrent\newline{}Ordnung: Lochung }\buchAbdrucke{\weitereDrucke{1) \emph{7. 11. 1893.} In: Arthur Schnitzler: \emph{The Letters of Arthur Schnitzler to Hermann Bahr}. Edited, annotated, and with an introduction, by Donald G.
                        Daviau. Chapel Hill: \emph{The University of North Carolina Press} 1978, S. 57–58 (University of North Carolina studies in the Germanic languages
                        and literatures, 89).} \weitereDrucke{2) Hermann Bahr, Arthur Schnitzler: \emph{Briefwechsel, Aufzeichnungen, Dokumente (1891–1931)}. Hg. Kurt Ifkovits und Martin Anton Müller. Göttingen: \emph{Wallstein} 2018, S. 47.} }\toendnotes[C]{\smallbreak}\pstart{}{\pb}Lieber
                  Freund,\pend\pstart
           hier iſt also etwas, was ſich möglicherweiſe als Eingangsfeuilleton eignet. Ich habe
               ihm vorläufig keinen Namen gegeben – eventuell könnte man das Ding »\label{K_L00280_1v}\edtext{\textcolor{green}{Abendſpaziergang}{}\ledrightnote{\textcolor{green}{Spaziergang}}}{\lemma{\textnormal{\emph{Abendſpaziergang}}}\Cendnote{\textnormal{Am Vortag hatte Schnitzler den Text
                  vollendet, am 15. 11. 1893 liest er ihn \textcolor{blue}{Beer-Hofmann} und \textcolor{blue}{Hofmannsthal} vor, »der viel getadelt
                     wurde«. Am selben Tag korrigierte er ihn noch. Am
                     6. 12. 1893 erscheint der Text als \emph{\textcolor{green}{Spaziergang}}.}}}\label{K_L00280_1h}« heißen. Vortheilhaft erſcheint mir, daſs
               in den vier Freunden \label{LL001-2v}Typen\label{LL001-2h} angedeutet
               ſind, die ſich vielleicht {\pb}weiterhin für die Reihe
               noch irgendwie werden verwenden laſſen. –\pend
           \pstart
           Ich ſchicke Ihnen da gleich auch eine andre kleine \label{K_L00280_2v}\edtext{Geſchichte}{\lemma{\textnormal{\emph{Geſchichte}}}\Cendnote{\textnormal{eventuell \emph{\textcolor{green}{Die Braut}}}}}\label{K_L00280_2h} mit, die, wenn ſie
               nicht am Ende zu »frivol« iſt, ganz ohne Praetenſion gelegentlich unter den Skizzen
               gebracht werden könnte.\pend
           \pstart
           Ich hoffe Ihnen nun aber bald was vernünftiges ſchicken {\pb}zu können. \label{LL001-1v}Schließlich werde ich doch wohl auch das Feuilleton
                  ſchreiben lernen – vorläufig fehlt mir noch manches dazu\label{LL001-1h}.\pend
           \pstart
           – Mit herzlichen Grüßen{\\[\baselineskip]}Ihr ſehr ergebner{\\[\baselineskip]}\spacefill\mbox{Arthur Schnitzler}\pend
           \leftskip=0em{}\pstart
           \textcolor{pink}{Wien}{}\ledrightnote{\textcolor{pink}{Wien}}, 7. November
                     93.\pend
           \endnumbering\briefempfaengerindex{Bahr, Hermann@\textsc{Bahr, Hermann}!zzzSchnitzler, Arthur@\emph{von Arthur Schnitzler}!1893-11-071@{7. 11. 1893}|)be}\mylabel{h}  \normalsize

\doendnotes{C}
\bigskip
\vfill

\clearpage

\footnotesize

\lohead{\textsc{register}}

% Definiere theindex-Environment komplett neu ohne reledmac
\makeatletter
\renewenvironment{theindex}{%
  \section*{\indexname}%
  \setlength{\parindent}{0pt}%
  \setlength{\parskip}{0pt plus 0.3pt}%
  \let\item\@idxitem
}{%
  \clearpage
}
\makeatother

\IfFileExists{\jobname-pw.ind}{\input{\jobname-pw.ind}}{}

\end{document}

      