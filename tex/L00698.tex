%% latex-korrekturansicht-vorspann.tex
%% Vorspann für die Korrekturansicht.
%% Lädt die gemeinsame Datei latex-vorspann.tex mit gesetztem Schalter.

\newif\ifkorrekturansicht
\korrekturansichttrue

\input{../tex-inputs/latex-vorspann}


               \section[Arthur Schnitzler an Hugo von Hofmannsthal, 9. 7. 1897]{ Arthur Schnitzler an Hugo von Hofmannsthal,
                    9. 7. 1897}\nopagebreak\mylabel{v}\rehead{ }\normalsize\beginnumbering\briefempfaengerindex{Hofmannsthal, Hugo von@\textsc{Hofmannsthal, Hugo von}!zzzSchnitzler, Arthur@\emph{von Arthur Schnitzler}!1897-07-093@{9. 7. 1897}|(be} \toendnotes[C]{\smallbreak\pagebreak[2]} \Standort{FDH, Hs-30885,60.}
\physDesc{Brief, 1 Blatt, 4 Seiten
\newline{}Handschrift: schwarze Tinte, deutsche Kurrent}\buchAbdrucke{\weitereDrucke{Hugo von Hofmannsthal, Arthur Schnitzler: \emph{Briefwechsel}. Hg. Therese Nickl und Heinrich Schnitzler. Frankfurt am Main: \emph{S. Fischer} 1964, S. 90–91.} }\pstart
           \raggedleft{}{\pb}\textcolor{pink}{\textsc{Ischl}}{}\ledrightnote{\textcolor{pink}{Bad Ischl}}, 9. 7. 97.\pend
           \pstart
           Mein lieber Hugo, überallher ko{\geminationm}en
                    nur ärgerliche Nachrichten, insbeſonders dieſe Schwierigkeiten mit der \textcolor{pink}{Wien}{}\ledrightnote{\textcolor{pink}{Wien}}er Wohnung ſtören mich ſehr. Ich werde
                    wohl früher nach \textcolor{pink}{Wien}{}\ledrightnote{\textcolor{pink}{Wien}} fahren u gleich
                    definitiv in \textcolor{pink}{Wien}{}\ledrightnote{\textcolor{pink}{Wien}} bleiben.\pend
           \pstart
           Jetzt ka{\geminationn} ich nicht weg von hier, es wäre auch eine
                    wahrſcheinlich nutzloſe Hin u Herhetzerei. {\pb}Bitte
                    lieber Hugo, ginge das, daſs wir unſer \textcolor{pink}{Salzburg}{}\ledrightnote{\textcolor{pink}{Salzburg}}er Zuſa{\geminationm}enſein um ein paar Tage
                    früher hätten? Daſs Sie ſtatt am 23.{ }ſchon am 22. oder
                    noch lieber am 21. in \textcolor{pink}{S.}{}\ledrightnote{\textcolor{pink}{Salzburg}} wären,
                        \textsc{resp.} ich Sie in \textcolor{pink}{\textsc{Bruck}}{}\ledrightnote{\textcolor{pink}{Bruck an der Großglocknerstraße}}-\textcolor{pink}{\textsc{Fusch}}{}\ledrightnote{\textcolor{pink}{Fusch an der Großglocknerstraße}} abholte? –\pend
           \pstart
           Mit \textcolor{blue}{Poldi Andrian}{}\ledrightnote{\textcolor{blue}{Leopold von Andrian-Werburg}} wirds hoffentlich
                    (dieſes »hoffentlich« kommt nicht nur aus Bequemlichkeit ſondern auch aus
                    »ärztlicher Einſicht« her) bald {\pb}wieder beſſer
                    ſein. Jetzt gleich nach \textcolor{pink}{Wien}{}\ledrightnote{\textcolor{pink}{Wien}} zu fahren wäre
                    mir eine rechte Unannehmlichkeit, und wirklich nöthig iſt’s ja gewiſs nicht.
                    Schreiben Sie mir aber doch, wenn Sie können, näheres! –\pend
           \pstart
           – Könnten Einem doch nur alle äußeren Sachen abgenommen werden. Es gibt ja ſoviel
                    Leute, denen das ſo viel Freude macht und die \textcolor{gray}{nur} dadurch,
                    daß ich es äußere, ich {\pb}meine{[},{]} adminiſtrative Sachen gibt, die ſie zu
                    beſorgen haben, zum Bewußtſein ihrer Exiſtenz kommen; – ließe ſich das
                    nicht irgendwie vertheilen? Ich ſtelle mir ein Secretariat, eine Agentur im
                    großen Stile vor, wo man alles findet, we{\geminationn} man nur
                    in zehn Worten mittheilt: dieſe oder jene Schwierigkeit habe ich.\pend
           \pstart – Auf Wiederſehen. Herzliche Grüße! Ihr \spacefill\mbox{Arthur.}\pend{}\endnumbering\briefempfaengerindex{Hofmannsthal, Hugo von@\textsc{Hofmannsthal, Hugo von}!zzzSchnitzler, Arthur@\emph{von Arthur Schnitzler}!1897-07-093@{9. 7. 1897}|)be}\mylabel{h}  \normalsize

\doendnotes{C}
\bigskip
\vfill

\clearpage

\footnotesize

\lohead{\textsc{register}}

% Definiere theindex-Environment komplett neu ohne reledmac
\makeatletter
\renewenvironment{theindex}{%
  \section*{\indexname}%
  \setlength{\parindent}{0pt}%
  \setlength{\parskip}{0pt plus 0.3pt}%
  \let\item\@idxitem
}{%
  \clearpage
}
\makeatother

\IfFileExists{\jobname-pw.ind}{\input{\jobname-pw.ind}}{}

\end{document}

      