%% latex-korrekturansicht-vorspann.tex
%% Vorspann für die Korrekturansicht.
%% Lädt die gemeinsame Datei latex-vorspann.tex mit gesetztem Schalter.

\newif\ifkorrekturansicht
\korrekturansichttrue

\input{../tex-inputs/latex-vorspann}


               \section[Hermann Bahr an Arthur Schnitzler, 28. 9. 1912]{ Hermann Bahr an Arthur Schnitzler, 28. 9. 1912}\nopagebreak\mylabel{v}\rehead{ }\normalsize\beginnumbering\briefempfaengerindex{Schnitzler, Arthur@\textsc{Schnitzler, Arthur}!zzzBahr, Hermann@\emph{von Hermann Bahr}!1912-09-281@{28. 9. 1912}|(be} \toendnotes[C]{\smallbreak\pagebreak[2]} \Standort{CUL, Schnitzler, B 5b.}
\physDesc{Bildpostkarte
\newline{}Handschrift: schwarze Tinte, deutsche Kurrent\newline{}Versand: Stempel: »\nobreak{}\oindex{Semmering@\textbf{Semmering}, \emph{Besiedelter Ort (A.BSO)}|pwk}Semmering, 29. IX. 12, XII\nobreak{}«.  \newline{}Ordnung: mit Bleistift von unbekannter Hand nummeriert:
                                    »174« }\buchAbdrucke{\weitereDrucke{Hermann Bahr, Arthur Schnitzler: \emph{Briefwechsel, Aufzeichnungen, Dokumente (1891–1931)}. Hg. Kurt Ifkovits und Martin Anton Müller. Göttingen: \emph{Wallstein} 2018, S. 478.} }\toendnotes[C]{\smallbreak}\pstart{}{\pb}Herrn \textsc{D\textsuperscript{r} Artur Schnitzler}\pend{}\pstart{}\textcolor{pink}{XVIII Sternwarteſtr 71}{}\ledrightnote{\textcolor{pink}{Sternwartestraße}}\pend{}\pstart{}\textsc{\textcolor{pink}{Wien XVIII}{}\ledrightnote{\textcolor{pink}{XVIII., Währing}}}\pend{}{\bigskip}\pstart
           \noindent{}\centering{}\textcolor{gray}{\textbf{{\pb}\textcolor{pink}{Venezia – Cortile Palazzo Ducale}{}\ledrightnote{\textcolor{pink}{Palazzo Ducale}}}}\pend
           \pstart
           \noindent{}{\pb}\textcolor{pink}{Semmering Villa Mautner}{}\ledrightnote{\textcolor{pink}{Villa Mauthner-Markhof}}\pend
           \pstart
           \raggedleft{}28. 9. 12\pend
           \pstart
           Herzlichen Dank, lieber Artur, für Deine Karte und die eben
               einlangenden \label{K_L02092_1v}\edtext{\textcolor{green}{vier Bände Theater}{}\ledrightnote{→\textcolor{green}{Gesammelte Werke}}}{\lemma{\textnormal{\emph{vier Bände Theater}}}\Cendnote{\textnormal{Die \emph{\textcolor{green}{Gesammelten Werke}} erschienen 1912 bei \emph{\textcolor{brown}{S. Fischer}} in sieben Bänden, vier enthielten die \emph{\textcolor{green}{Theaterstücke}}, drei die \emph{\textcolor{green}{Erzählenden Schriften}}.}}}\label{K_L02092_1h}. Ich \label{K_L02092_2v}\edtext{vagabundiere durch die Welt}{\lemma{\textnormal{\emph{vagabundiere … Welt}}}\Cendnote{\textnormal{Die Vortragstourneen in \textcolor{pink}{Deutschland} fanden vom 4. 11. 1912 bis
                  zum 27. 11. 1912 und vom 13. 12. 1912 bis zum
                     30. 1. 1913 statt. Ob er den für Dezember geplanten
                  Aufenthalt in \textcolor{pink}{Rom} verwirklichte, ist nicht
                  geklärt.}}}\label{K_L02092_2h} (zunächſt von hier nach \textcolor{pink}{Deutſchland}{}\ledrightnote{\textcolor{pink}{Deutschland}}, ſechs Wochen Vorleſungen, dann nach \textcolor{pink}{Rom}{}\ledrightnote{\textcolor{pink}{Rom}}, Januar und Februar wieder Vorleſungen), bis ich am 1. März
               in \textcolor{pink}{Salzburg, Arenbergſchloß}{}\ledrightnote{\textcolor{pink}{Schloss Arenberg}} zu landen hoffe.\pend
           \pstart Herzlichſt immer Dein \spacefill\mbox{Hermann}\pend{}\pstart
           \noindent{}Schönſte Grüße an Frau \textcolor{blue}{Olga}{}\ledrightnote{\textcolor{blue}{Olga Schnitzler}}!\pend
           \endnumbering\briefempfaengerindex{Schnitzler, Arthur@\textsc{Schnitzler, Arthur}!zzzBahr, Hermann@\emph{von Hermann Bahr}!1912-09-281@{28. 9. 1912}|)be}\mylabel{h}  \normalsize

\doendnotes{C}
\bigskip
\vfill

\clearpage

\footnotesize

\lohead{\textsc{register}}

% Definiere theindex-Environment komplett neu ohne reledmac
\makeatletter
\renewenvironment{theindex}{%
  \section*{\indexname}%
  \setlength{\parindent}{0pt}%
  \setlength{\parskip}{0pt plus 0.3pt}%
  \let\item\@idxitem
}{%
  \clearpage
}
\makeatother

\IfFileExists{\jobname-pw.ind}{\input{\jobname-pw.ind}}{}

\end{document}

      