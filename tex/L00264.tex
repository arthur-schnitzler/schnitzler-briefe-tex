%% latex-korrekturansicht-vorspann.tex
%% Vorspann für die Korrekturansicht.
%% Lädt die gemeinsame Datei latex-vorspann.tex mit gesetztem Schalter.

\newif\ifkorrekturansicht
\korrekturansichttrue

\input{../tex-inputs/latex-vorspann}


               \section[Arthur Schnitzler und Paul Goldmann an Richard Beer-Hofmann, 18. 9. 1893]{ Arthur Schnitzler und Paul Goldmann an Richard Beer-Hofmann,
                    18. 9. 1893}\nopagebreak\mylabel{v}\rehead{ }\normalsize\beginnumbering\briefempfaengerindex{Beer-Hofmann, Richard@\textsc{Beer-Hofmann, Richard}!zzzGoldmann, Paul@\emph{von Paul Goldmann}!1893-09-181@{18. 9. 1893}|(be}\briefempfaengerindex{Beer-Hofmann, Richard@\textsc{Beer-Hofmann, Richard}!zzzSchnitzler, Arthur@\emph{von Arthur Schnitzler}!1893-09-181@{18. 9. 1893}|(be} \toendnotes[C]{\smallbreak\pagebreak[2]} \Standort{YCGL, MSS 31.}
\physDesc{Brief, 1 Blatt (Briefpapier mit Trauerrand), 2 Seiten, Umschlag mit Trauerrand
\newline{}Handschrift Arthur Schnitzler: Bleistift, deutsche Kurrent\newline{}Handschrift Paul Goldmann: Bleistift, deutsche Kurrent\newline{}Versand: 1) Stempel: »\nobreak{}\oindex{Salzburg@\textbf{Salzburg}, \emph{Besiedelter Ort (A.BSO)}|pwk}{\pb}Salzburg
                                                  Stadt, 18/9 93, 2N\nobreak{}«.  2) Stempel: »\nobreak{}\oindex{Znaim@\textbf{Znaim}, \emph{Besiedelter Ort (A.BSO)}|pwk}Znaim, 25/\textcolor{gray}{9} 93, 8–10V\nobreak{}«. 3) Stempel: »\nobreak{}\oindex{I., Innere Stadt@\textbf{I., Innere Stadt}, \emph{Bezirk (A.BZK)}|pwk}Wien 1/1, 25 9. 93, 5–6½ N, Bestellt\nobreak{}«. 4) mit Tinte von unbekannter Hand Empfängeradresse geändert zu: »\textsc{\textcolor{pink}{Wollzeile Nro. 15}{ }\textcolor{pink}{Wien}}«}\pstart{}{\pb}\textsc{Herrn Dr. Richard Beer-Hofmann}\pend{}\pstart{}kk. Lieutenant a d Reſ. des Kuk Infanterie-Regim. \textsc{Nr. 99}\pend{}\pstart{}\textsc{\textcolor{pink}{Znaim}{}\ledrightnote{\textcolor{pink}{Znaim}}}\pend{}{\bigskip}\pstart
           \raggedleft{}{\pb}\textcolor{pink}{\textsc{Salzburg}}{}\ledrightnote{\textcolor{pink}{Salzburg}} 18. 9. 93\pend
           \pstart{}Lieber Richard,\pend\pstart
           wir ſitzen im \textcolor{pink}{\textsc{Café Tomaselli}}{}\ledrightnote{\textcolor{pink}{Café Tomaselli}} und grüßen Sie herzlich.\pend
           \pstart \spacefill\mbox{Arthur}\pend{}\pstart
           \noindent{}{[}hs. Goldmann:{]} Liebſter Freund!\pend
           \pstart
           Wir feiern ſeit geſtern das große Erinnerungsfeſt. Ich weiß nun alles – bis auf
                    Deinen Hund und Deine Cravatten. Es iſt ſo ſchön, \strikeout{bei}{ }{\pb}beiſammen zu ſein!\pend
           \pstart
           Ich kann leider nicht nach \textcolor{pink}{Wien}{}\ledrightnote{\textcolor{pink}{Wien}}, aber Du mußt
                    nach \textcolor{pink}{\textsc{Paris}}{}\ledrightnote{\textcolor{pink}{Paris}}. Du wirſt mir darauf, wie gewöhnlich, nicht antworten. Das macht nichts.
                    Aber ich \strikeout{er} erwarte Dich in \textcolor{pink}{\textsc{Paris}}{}\ledrightnote{\textcolor{pink}{Paris}}, nächſtens, ſo nächſtens als möglich. Ja? Treuen Gruß!\pend
           \pstart
           Dein{\\[\baselineskip]}\spacefill\mbox{Paul Goldmann.}\pend
           \leftskip=0em{}\endnumbering\briefempfaengerindex{Beer-Hofmann, Richard@\textsc{Beer-Hofmann, Richard}!zzzGoldmann, Paul@\emph{von Paul Goldmann}!1893-09-181@{18. 9. 1893}|)be}\briefempfaengerindex{Beer-Hofmann, Richard@\textsc{Beer-Hofmann, Richard}!zzzSchnitzler, Arthur@\emph{von Arthur Schnitzler}!1893-09-181@{18. 9. 1893}|)be}\mylabel{h}  \normalsize

\doendnotes{C}
\bigskip
\vfill

\clearpage

\footnotesize

\lohead{\textsc{register}}

% Definiere theindex-Environment komplett neu ohne reledmac
\makeatletter
\renewenvironment{theindex}{%
  \section*{\indexname}%
  \setlength{\parindent}{0pt}%
  \setlength{\parskip}{0pt plus 0.3pt}%
  \let\item\@idxitem
}{%
  \clearpage
}
\makeatother

\IfFileExists{\jobname-pw.ind}{\input{\jobname-pw.ind}}{}

\end{document}

      