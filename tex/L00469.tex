%% latex-korrekturansicht-vorspann.tex
%% Vorspann für die Korrekturansicht.
%% Lädt die gemeinsame Datei latex-vorspann.tex mit gesetztem Schalter.

\newif\ifkorrekturansicht
\korrekturansichttrue

\input{../tex-inputs/latex-vorspann}


               \section[Arthur Schnitzler an Richard Beer-Hofmann, {[}5. 8. 1895?{]}]{ Arthur Schnitzler an Richard Beer-Hofmann, {[}5. 8. 1895?{]}}\nopagebreak\mylabel{v}\rehead{ }\normalsize\beginnumbering\briefempfaengerindex{Beer-Hofmann, Richard@\textsc{Beer-Hofmann, Richard}!zzzSchnitzler, Arthur@\emph{von Arthur Schnitzler}!1895-08-051@{{[}5. 8. 1895?{]}}|(be} \toendnotes[C]{\smallbreak\pagebreak[2]} \Standort{YCGL, MSS 31.}
\physDesc{Briefkarte
\newline{}Handschrift: Bleistift, deutsche Kurrent}\toendnotes[C]{\smallbreak}\pstart
           \noindent{}{\pb}Lieber Richard! \label{K_L00469_1v}\edtext{\textcolor{blue}{\textsc{Salten}}{}\ledrightnote{\textcolor{blue}{Felix Salten}}}{\lemma{\textnormal{\emph{Salten}}}\Cendnote{\textnormal{Das verwendete Papier (und die
                  Einordnung zwischen die anderen Korrespondenzstücke im Archiv) deuten auf
                     1895. Aus dem Inhalt geht hervor, dass die Kommunikation
                  außerhalb von \textcolor{pink}{Wien}{ }stattfindet (»angeko{\geminationm}en«). Das reduziert die durch das \emph{\textcolor{green}{Tagebuch}} möglichen Daten auf 5. 8. 1895 und 16. 8. 1895. Beim zweiten
                  Termin kündigt \textcolor{blue}{Salten} aber an, einen späteren
                  Zug zu nehmen. Auch dürfte sich \textcolor{blue}{Beer-Hofmann}
                  zu diesem Zeitpunkt nicht in \textcolor{pink}{Ischl} aufgehalten
                  haben, was den 5. 8. 1895 wahrscheinlich macht. Im Theater sieht \textcolor{blue}{Schnitzler} an diesem Tag \emph{\textcolor{green}{Zwei glückliche Tage}}.}}}\label{K_L00469_1h} iſt erſt kurz vor 1
               hier angeko{\geminationm}en. – Haben Sie ſchon einen Sitz für mich
                  geno{\geminationm}en, ſo geh ich natürlich ins Theater – nicht –
               nicht. – Für alle Fälle laſſen Sie mir was ſagen. {\pb}Iſts Ihnen recht, ko{\geminationm} ich mit \textcolor{blue}{\textsc{S.}}{}\ledrightnote{\textcolor{blue}{Felix Salten}} zwiſchen 5 u 6 zu Ihnen.\pend
           \pstart
           Herzlich{\\[\baselineskip]}Ihr \spacefill\mbox{Arth}\pend
           \leftskip=0em{}\endnumbering\briefempfaengerindex{Beer-Hofmann, Richard@\textsc{Beer-Hofmann, Richard}!zzzSchnitzler, Arthur@\emph{von Arthur Schnitzler}!1895-08-051@{{[}5. 8. 1895?{]}}|)be}\mylabel{h}  \normalsize

\doendnotes{C}
\bigskip
\vfill

\clearpage

\footnotesize

\lohead{\textsc{register}}

% Definiere theindex-Environment komplett neu ohne reledmac
\makeatletter
\renewenvironment{theindex}{%
  \section*{\indexname}%
  \setlength{\parindent}{0pt}%
  \setlength{\parskip}{0pt plus 0.3pt}%
  \let\item\@idxitem
}{%
  \clearpage
}
\makeatother

\IfFileExists{\jobname-pw.ind}{\input{\jobname-pw.ind}}{}

\end{document}

      