%% latex-korrekturansicht-vorspann.tex
%% Vorspann für die Korrekturansicht.
%% Lädt die gemeinsame Datei latex-vorspann.tex mit gesetztem Schalter.

\newif\ifkorrekturansicht
\korrekturansichttrue

\input{../tex-inputs/latex-vorspann}


               \section[Peter Altenberg an Arthur Schnitzler, {[}zwischen April und Oktober 1912{]}]{ Peter Altenberg an Arthur Schnitzler, {[}zwischen April und Oktober
                    1912{]}}\nopagebreak\mylabel{v}\rehead{ }\normalsize\beginnumbering\briefempfaengerindex{Schnitzler, Arthur@\textsc{Schnitzler, Arthur}!zzzAltenberg, Peter@\emph{von Peter Altenberg}!1912-04-091@{{[}zwischen April und
                        Oktober 1912{]}}|(be} \toendnotes[C]{\smallbreak\pagebreak[2]} \Standort{DLA, A:Schnitzler, HS.NZ85.1.2342, S. 9.}
\physDesc{maschinelle Abschrift}\toendnotes[C]{\smallbreak}\pstart
           \raggedleft{}{\pb}\textcolor{pink}{Semmering}{}\ledrightnote{\textcolor{pink}{Semmering}}.\pend
           \pstart
           Lieber Dr. Arthur Schnitzler, bitte, schenken Sie mir für meine
                    12-jährige Heilige \label{K_L02061_1v}\edtext{\textcolor{blue}{Klara Panhans}{}\ledrightnote{\textcolor{blue}{Klara Panhans}}}{\lemma{\textnormal{\emph{Klara Panhans}}}\Cendnote{\textnormal{Ab 9. 4. 1912 war die
                        Hotelierstochter 12 Jahre alt, am 17. 10. 1912 fand der Bruch
                        statt, so dass dieses Korrespondenzstück in der Zwischenzeit abgefasst sein
                        muss.}}}\label{K_L02061_1h} ein Autogramm! Ihr\pend
           \pstart \spacefill\mbox{P. A.}\pend{}\endnumbering\briefempfaengerindex{Schnitzler, Arthur@\textsc{Schnitzler, Arthur}!zzzAltenberg, Peter@\emph{von Peter Altenberg}!1912-04-091@{{[}zwischen April und
                        Oktober 1912{]}}|)be}\mylabel{h}  \normalsize

\doendnotes{C}
\bigskip
\vfill

\clearpage

\footnotesize

\lohead{\textsc{register}}

% Definiere theindex-Environment komplett neu ohne reledmac
\makeatletter
\renewenvironment{theindex}{%
  \section*{\indexname}%
  \setlength{\parindent}{0pt}%
  \setlength{\parskip}{0pt plus 0.3pt}%
  \let\item\@idxitem
}{%
  \clearpage
}
\makeatother

\IfFileExists{\jobname-pw.ind}{\input{\jobname-pw.ind}}{}

\end{document}

      