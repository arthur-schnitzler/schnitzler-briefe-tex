%% latex-korrekturansicht-vorspann.tex
%% Vorspann für die Korrekturansicht.
%% Lädt die gemeinsame Datei latex-vorspann.tex mit gesetztem Schalter.

\newif\ifkorrekturansicht
\korrekturansichttrue

\input{../tex-inputs/latex-vorspann}


               \section[Hugo von Hofmannsthal an Arthur Schnitzler, {[}5. 2. 1917{]}]{ Hugo von Hofmannsthal an Arthur Schnitzler, {[}5. 2. 1917{]}}\nopagebreak\mylabel{v}\rehead{ }\normalsize\beginnumbering\briefempfaengerindex{Schnitzler, Arthur@\textsc{Schnitzler, Arthur}!zzzHofmannsthal, Hugo von@\emph{von Hugo von Hofmannsthal}!1917-02-051@{{[}5. 2. 1917{]}}|(be} \toendnotes[C]{\smallbreak\pagebreak[2]} \Standort{CUL, Schnitzler, B 43.}
\physDesc{Brief, 1 Blatt, 2 Seiten
\newline{}Handschrift: schwarze Tinte, deutsche Kurrent
\newline{}Schnitzler: 1) mit Bleistift datiert: »5/2 917« und beschriftet: »\textsc{Hugo}« 2) mit rotem Buntstift eine Unterstreichung\newline{}Ordnung: 1) mit Bleistift von unbekannter Hand nummeriert: »\strikeout{343}« 2) mit Bleistift von unbekannter Hand nummeriert: »356«}\buchAbdrucke{\weitereDrucke{Hugo von Hofmannsthal, Arthur Schnitzler: \emph{Briefwechsel}. Hg. Therese Nickl und Heinrich Schnitzler. Frankfurt am Main: \emph{S. Fischer} 1964, S. 280.} }\toendnotes[C]{\smallbreak}\pstart
           \raggedleft{}{\pb}Montag\pend
           \pstart{}mein lieber Arthur\pend\pstart
           heute abend iſt es leider nicht gegangen, weil \textcolor{blue}{Gerty}{}\ledrightnote{\textcolor{blue}{Gertrude von Hofmannsthal}} mit den \textcolor{blue}{Kindern}{}\ledrightnote{→\textcolor{blue}{Christiane von Hofmannsthal}{\newline}→\textcolor{blue}{Raimund von Hofmannsthal}{\newline}→\textcolor{blue}{Franz von Hofmannsthal}} zur \textcolor{blue}{Wieſenthal}{}\ledrightnote{\textcolor{blue}{Grethe Wiesenthal}} geht und ich
               etwas mit \textcolor{blue}{Andrian}{}\ledrightnote{\textcolor{blue}{Leopold von Andrian-Werburg}}{ }ſprechen muſs, der i{\geminationm}er erst von 9\textsuperscript{h} abends an frei iſt.\pend
           \pstart
           Euer Herko{\geminationm}en Mittwoch iſt ein lieber
               Gedanke, aber ſo weit ſind wir noch nicht. Es iſt ja noch längſt \label{K_L02254_1v}\edtext{keine \textcolor{pink}{Wohnung}{}\ledrightnote{→\textcolor{pink}{Stallburggasse}}}{\lemma{\textnormal{\emph{keine Wohnung}}}\Cendnote{\textnormal{Gemeint ist die Wohnung in der \textcolor{pink}{Stallburggasse 2}, die sie sich
                  herrichteten.}}}\label{K_L02254_1h}, die Handwerker liefern nichts, und ich habe auch, unter
               immer neuen Sorgen u. Verdüſterungen, gar nicht den Kopf, {\pb}die Leute zu drängen.\pend
           \pstart
           Es ſcheint jetzt daſs ich erſt Ende der Woche abreiſen kann, ſo könnten wir \label{K_L02254_2v}\edtext{Mittwoch}{\lemma{\textnormal{\emph{Mittwoch}}}\Cendnote{\textnormal{vgl. A. S.: \emph{Tagebuch}, 7. 2. 1917}}}\label{K_L02254_2h}{ }Abends zu Euch ko{\geminationm}en: Vorausſetzung ein
               wirklich der Situation gemäßes Nachtmahl, Brot bringen wir mit.\pend
           \pstart
           Paſst es Euch nicht, bitten wir um Abſage morgen Dienstag{ }vormittags an \uline{229}.\pend
           \pstart
           Ihr{\\[\baselineskip]}\spacefill\mbox{Hugo.}\pend
           \leftskip=0em{}\endnumbering\briefempfaengerindex{Schnitzler, Arthur@\textsc{Schnitzler, Arthur}!zzzHofmannsthal, Hugo von@\emph{von Hugo von Hofmannsthal}!1917-02-051@{{[}5. 2. 1917{]}}|)be}\mylabel{h}  \normalsize

\doendnotes{C}
\bigskip
\vfill

\clearpage

\footnotesize

\lohead{\textsc{register}}

% Definiere theindex-Environment komplett neu ohne reledmac
\makeatletter
\renewenvironment{theindex}{%
  \section*{\indexname}%
  \setlength{\parindent}{0pt}%
  \setlength{\parskip}{0pt plus 0.3pt}%
  \let\item\@idxitem
}{%
  \clearpage
}
\makeatother

\IfFileExists{\jobname-pw.ind}{\input{\jobname-pw.ind}}{}

\end{document}

      