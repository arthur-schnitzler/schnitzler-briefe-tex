%% latex-korrekturansicht-vorspann.tex
%% Vorspann für die Korrekturansicht.
%% Lädt die gemeinsame Datei latex-vorspann.tex mit gesetztem Schalter.

\newif\ifkorrekturansicht
\korrekturansichttrue

\input{../tex-inputs/latex-vorspann}


               \section[Paul Goldmann an Arthur Schnitzler, 24. 12. {[}1891{]}]{ Paul Goldmann an Arthur Schnitzler, 24. 12. {[}1891{]}}\nopagebreak\mylabel{v}\rehead{ }\normalsize\beginnumbering\briefempfaengerindex{Schnitzler, Arthur@\textsc{Schnitzler, Arthur}!zzzGoldmann, Paul@\emph{von Paul Goldmann}!1891-12-241@{24. 12. {[}1891{]}}|(be} \toendnotes[C]{\smallbreak\pagebreak[2]} \Standort{DLA, A:Schnitzler, HS.NZ85.1.3162.}
\physDesc{Brief, 1 Blatt, 1 Seite
\newline{}Handschrift: schwarze Tinte, deutsche Kurrent
\newline{}Schnitzler: mit Bleistift das Jahr »1891« vermerkt }\toendnotes[C]{\smallbreak}\pstart
           \raggedleft{}{\pb}24. December –\pend
           \pstart
           Weihnachtsabend. \label{K_L02677-1v}\edtext{Buden}{\lemma{\textnormal{\emph{Buden}}}\Cendnote{\textnormal{Schaubuden,
                  Verkaufsstände}}}\label{K_L02677-1h} auf den \textsc{Boulevards}, und eine
               dichte Menge an ihnen vorbei auf dem \label{K_L02677-2v}\edtext{\begin{otherlanguage}{french}Trottoir\end{otherlanguage}}{\lemma{\textnormal{\emph{Trottoir}}}\Cendnote{\textnormal{österreichisch: Bürgersteig,
                  Gehsteig}}}\label{K_L02677-2h}. Brauſen, Rauſchen, Frauenduft, Lichterglanz, \textcolor{pink}{Paris}{}\ledrightnote{\textcolor{pink}{Paris}}. Und ich, zur Straße verurtheilt, und ſelbſt auf der
               Straße ein Fremder. Sorgenberg, gedehmüthigt, zukunftverzweifelnd, von einer Dirne
               beſchmutzt. Ein Zufall führt mich am Hauſe vorüber. Die \textcolor{green}{Zeitung}{}\ledrightnote{→\textcolor{green}{Frankfurter Zeitung}}, »\label{K_L02677-3v}\edtext{\textcolor{green}{Weihnachtseinkäufe}{}\ledrightnote{\textcolor{green}{Weihnachts-Einkäufe}}}{\lemma{\textnormal{\emph{Weihnachtseinkäufe}}}\Cendnote{\textnormal{Arthur Schnitzler: \emph{\textcolor{green}{Weihnachts-Einkäufe}}.
                     In: \emph{\textcolor{green}{Frankfurter Zeitung}}, Jg. 36, Nr. 358,
                        24. 12. 1891, S. 1–2.}}}\label{K_L02677-3h}«. Mein lieber, lieber
               Freund, wie danke ich Dir für dieſen Weihnachtsgruß, der nicht beabſichtigt war und
               doch in’s tiefſte Herz traf. Ich gehe ſchlafen, mit ein paar Thränen in den Augen.
               Was für ein großer Künſtler biſt Du, mein Sohn!\pend
           \pstart
           Gute Nacht!\pend
           \endnumbering\briefempfaengerindex{Schnitzler, Arthur@\textsc{Schnitzler, Arthur}!zzzGoldmann, Paul@\emph{von Paul Goldmann}!1891-12-241@{24. 12. {[}1891{]}}|)be}\mylabel{h}  \normalsize

\doendnotes{C}
\bigskip
\vfill

\clearpage

\footnotesize

\lohead{\textsc{register}}

% Definiere theindex-Environment komplett neu ohne reledmac
\makeatletter
\renewenvironment{theindex}{%
  \section*{\indexname}%
  \setlength{\parindent}{0pt}%
  \setlength{\parskip}{0pt plus 0.3pt}%
  \let\item\@idxitem
}{%
  \clearpage
}
\makeatother

\IfFileExists{\jobname-pw.ind}{\input{\jobname-pw.ind}}{}

\end{document}

      