%% latex-korrekturansicht-vorspann.tex
%% Vorspann für die Korrekturansicht.
%% Lädt die gemeinsame Datei latex-vorspann.tex mit gesetztem Schalter.

\newif\ifkorrekturansicht
\korrekturansichttrue

\input{../tex-inputs/latex-vorspann}


               \section[Adolf Treibl an Arthur Schnitzler, 15. 1. 1906]{ Adolf Treibl an Arthur Schnitzler, 15. 1. 1906}\nopagebreak\mylabel{v}\rehead{ }\normalsize\beginnumbering\briefempfaengerindex{Schnitzler, Arthur@\textsc{Schnitzler, Arthur}!zzzTreibl, Adolf@\emph{von Adolf Treibl}!1906-01-151@{15 1. 1906}|(be} \toendnotes[C]{\smallbreak\pagebreak[2]} \Standort{DLA, A:Schnitzler, HS.NZ85.1.4815,3.}
\physDesc{Brief, 1 Blatt, 4 Seiten
\newline{}Handschrift: schwarze Tinte, deutsche Kurrent
\newline{}Schnitzler: mit Bleistift beschriftet: »\textsc{Treibl (Ehrenstein}« }\toendnotes[C]{\smallbreak}\pstart
           \noindent{}{\pb}\textsc{Euer Hochwohlgeboren}\pend
           \pstart{}\textsc{Hochverehrter Herr Doctor}\pend\pstart
           Namens meines \label{K_L01572_1v}\edtext{Schwagers}{\lemma{\textnormal{\emph{Schwagers}}}\Cendnote{\textnormal{Treibl war mit einer Tante
                        mütterlicherseits von \textcolor{blue}{Albert Ehrenstein}
                        verheiratet.}}}\label{K_L01572_1h} Herrn \textcolor{blue}{\textsc{Alex Ehrenstein}}{}\ledrightnote{\textcolor{blue}{Alexander Ehrenstein}} und ſeiner \textcolor{blue}{Frau}{}\ledrightnote{→\textcolor{blue}{Charlotte Ehrenstein}}
                    beehre ich mich den verbindlichſten Dank für die warme Teilnahme auszudrücken,
                    die Euer Hochwohlgeboren dem lieben \textcolor{blue}{\textsc{Albert}}{}\ledrightnote{\textcolor{blue}{Albert Ehrenstein}} zuteil werden laſſen. {\pb}Dem Opfer, das Sie
                    mit Ihrem \label{K_L01572_2v}\edtext{geſtrigen Beſuch}{\lemma{\textnormal{\emph{geſtrigen Beſuch}}}\Cendnote{\textnormal{vgl. A. S.: \emph{Tagebuch}, 14. 1. 1906}}}\label{K_L01572_2h} nicht nur dem Patienten ſondern auch ſeinen mitleidenden \textcolor{blue}{Eltern}{}\ledrightnote{→\textcolor{blue}{Alexander Ehrenstein}{\newline}→\textcolor{blue}{Charlotte Ehrenstein}} gebracht
                    haben, wird, deſſen können hochverehrter Herr Doktor ſich verſichert halten, ein
                    treueſt und dankbarest Gedenken immer bewahrt werden.\pend
           \pstart
           Der Zuſtand des lieben \textcolor{blue}{\textsc{Albert}}{}\ledrightnote{\textcolor{blue}{Albert Ehrenstein}} iſt über Nacht wohl ruhiger geworden, doch lautet {\pb}die Auskunft des zu Rate gezogenen Arztes \textsc{D\textsuperscript{r}{ }\textcolor{blue}{Alfred Adler}{}\ledrightnote{\textcolor{blue}{Alfred Adler}}}, den ich als \textsc{Psychologen} und \textsc{Diagnostiker} hochschätze nichts weniger als
                    befriedigend. Er ſchließt auf \textsc{acute Paranoia} und
                    empfiehlt die Abgabe in ein Sanatorium.\pend
           \pstart
           Während ich dies ſchreibe iſt die \textcolor{blue}{Schwägerin}{}\ledrightnote{→\textcolor{blue}{Charlotte Ehrenstein}} in \textcolor{pink}{\textsc{Ob. Döbling}}{}\ledrightnote{\textcolor{pink}{Oberdöbling}} um die Aufnahme in das \textcolor{pink}{Sanatorium \textsc{Obersteiner}}{}\ledrightnote{\textcolor{pink}{Sanatorium Obersteiner}} vorzubereiten.\pend
           \pstart
           Indem ich unſeren herzlichſten Dank wiederhole {\pb}bitte ich dem lieben \textcolor{blue}{\textsc{Albert}}{}\ledrightnote{\textcolor{blue}{Albert Ehrenstein}} die \textsc{Sympathien} gütigſt zu bewahren, die, wie ich
                    begreife, ihn mit gerechtem Stolz erfüllen.\pend
           \pstart
           In vollkommener Hochachtung{\\[\baselineskip]}ergebenſt{\\[\baselineskip]}\spacefill\mbox{Adolf Treibl}\pend
           \leftskip=0em{}\pstart
           \textcolor{pink}{Wien}{}\ledrightnote{\textcolor{pink}{Wien}}{ }15/I 06\pend
           \endnumbering\briefempfaengerindex{Schnitzler, Arthur@\textsc{Schnitzler, Arthur}!zzzTreibl, Adolf@\emph{von Adolf Treibl}!1906-01-151@{15 1. 1906}|)be}\mylabel{h}  \normalsize

\doendnotes{C}
\bigskip
\vfill

\clearpage

\footnotesize

\lohead{\textsc{register}}

% Definiere theindex-Environment komplett neu ohne reledmac
\makeatletter
\renewenvironment{theindex}{%
  \section*{\indexname}%
  \setlength{\parindent}{0pt}%
  \setlength{\parskip}{0pt plus 0.3pt}%
  \let\item\@idxitem
}{%
  \clearpage
}
\makeatother

\IfFileExists{\jobname-pw.ind}{\input{\jobname-pw.ind}}{}

\end{document}

      