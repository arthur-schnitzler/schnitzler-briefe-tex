%% latex-korrekturansicht-vorspann.tex
%% Vorspann für die Korrekturansicht.
%% Lädt die gemeinsame Datei latex-vorspann.tex mit gesetztem Schalter.

\newif\ifkorrekturansicht
\korrekturansichttrue

\input{../tex-inputs/latex-vorspann}


               \section[Albert Ehrenstein an Arthur Schnitzler, 16. 1. 1908]{ Albert Ehrenstein an Arthur Schnitzler, 16. 1. 1908}\nopagebreak\mylabel{v}\rehead{ }\normalsize\beginnumbering\briefempfaengerindex{Schnitzler, Arthur@\textsc{Schnitzler, Arthur}!zzzEhrenstein, Albert@\emph{von Albert Ehrenstein}!1908-01-162@{16. 1. 1908}|(be} \toendnotes[C]{\smallbreak\pagebreak[2]} \Standort{CUL, Schnitzler, B 30.}
\physDesc{Brief, 1 Blatt, 4 Seiten
\newline{}Handschrift: schwarze Tinte, deutsche Kurrent
\newline{}Schnitzler: mit Bleistift Vermerk: »\textsc{A. Ehrenstein}« und neben das Datum die richtige Jahreszahl »08« geschrieben }\buchAbdrucke{\weitereDrucke{Albert Ehrenstein: \emph{Briefe}. Hg. Hanni Mittelmann. München: \emph{Boer} 1989, S. 21 (Werke, 1).} }\toendnotes[C]{\smallbreak}\pstart
           \raggedleft{}{\pb}\textsc{\textcolor{pink}{Wien, XVI. Ottakringerstr 114}{}\ledrightnote{\textcolor{pink}{Ottakringerstraße}}}\pend
           \pstart
           \textsc{16. Januar \label{T_L01751_1v}\edtext{07}{\lemma{\textnormal{\emph{07}}}\Cendnote{\textnormal{Schreibirrtum}}}\label{T_L01751_1h}}\pend
           \pstart\center{}\textsc{Sehr geehrter Herr Doktor!}\pend\pstart
           Zu den vielen \label{K_L01751_1v}\edtext{Glückwünſchen}{\lemma{\textnormal{\emph{Glückwünſchen}}}\Cendnote{\textnormal{zur Zuerkennung des \emph{\textcolor{brown}{Grillparzer-Preises}} für das \emph{\textcolor{green}{Zwischenspiel}} am 15. 1. 1908}}}\label{K_L01751_1h}, die Sie, ſehr verehrter Herr Doktor, in dieſen
               Tagen überfliegen werden, auch meine beſcheidene Gratulation.\pend
           \pstart
           Dürfte doch dieſe \textcolor{pink}{öſterreichiſch}{}\ledrightnote{\textcolor{pink}{Österreich}}{ }ſo unverzeihlich lang hinausgezögerte Ehrung, die
               nun, ſchwer vermeidbar geworden, nicht einmal auf deren Urheber zurückfällt,
               geſchweige denn ihren Zweck erreicht, manchen, und unter ihren auch mich,
               möglicherweiſe mehr und inniger gefreut haben als den Geehrten ſelbſt, dem die jetzt
               mit üblicher Rückſichtsloſigkeit hereinbrechende Briefflut vielleicht beſchwerlich
               fällt {\pb}und die Freude verkümmert. Aber
               auch ſo muß man einigermaßen froh ſein, daß ſich die Dinge etwas gebeſſert haben,
               indem ſich auch bei uns ſogar akademiſche Preisrichter dem längſt feſtſtehenden
               Urteil der Verſtändigen bequemten. Denn gewiß: hätte es zu \textcolor{blue}{Grillparzer}{}\ledrightnote{\textcolor{blue}{Franz Grillparzer}}s Zeiten etwa einen \textcolor{blue}{Walther von der Vogelweide}{}\ledrightnote{\textcolor{blue}{Walther von der Vogelweide}}-Preis gegeben, alle möglichen \textcolor{blue}{Halme}{}\ledrightnote{\textcolor{blue}{Friedrich Halm}} und \textcolor{blue}{Gutzkows}{}\ledrightnote{\textcolor{blue}{Karl Gutzkow}} hätten
               ihn erbuckelt, nur nicht den \textcolor{pink}{Wien}{}\ledrightnote{\textcolor{pink}{Wien}}er \textcolor{blue}{Dichter}{}\ledrightnote{→\textcolor{blue}{Franz Grillparzer}} hätte man durch ihn zu neuem Leben
               aufgerufen.\pend
           \pstart
           Jedenfalls, der Wunſch, ſolche und ähnliche Auszeichnung durch wiederholte {\pb}Verleihung an den ihrer Würdigſten ebenſo
               lächerlicher als trauriger Parteilichkeit entzogen zu ſehen, kommt mir aus dem
               Herzen. Habe ich doch Ihnen, ſehr geehrter Herr Doktor, nichts Kleines zu danken:
               Troſt in der Krankheit, Ermunterung im Trübſinn, Anregung aus Ihren Werken –
               namentlich dem prämierten \textcolor{green}{Stücke}{}\ledrightnote{→\textcolor{green}{Zwischenspiel. Komödie in drei Akten}}.
               Und wenn es mir gegönnt war, bloß den \label{K_L01751_2v}\edtext{Anfang}{\lemma{\textnormal{\emph{Anfang}}}\Cendnote{\textnormal{Der erste von sechs Teilen
                  des Vorabdrucks von \emph{\textcolor{green}{Der Weg ins Freie}} wurde im
                  Anfang des Monats ausgegebenen Januar-Heft der \emph{\textcolor{green}{Neuen Rundschau}} (S. 31–71) gedruckt.}}}\label{K_L01751_2h} Ihres
               neuen \textcolor{green}{Romans}{}\ledrightnote{→\textcolor{green}{Der Weg ins Freie. Roman}} mehrmals mit ſtets
               erneutem Entzücken zu leſen, haben Sie, ſehr geehrter Herr Doktor, daran keinen
               geringen Anteil.\pend
           \pstart
           {\pb}Indem ich noch für dieſe Beläſtigung um
               Entſchuldigung bitte, verbleibe ich\pend
           \pstart
           Hochachtungsvoll{\\[\baselineskip]}Ihr Ergebenſter{\\[\baselineskip]}\spacefill\mbox{Albert Ehrenstein}\pend
           \leftskip=0em{}\endnumbering\briefempfaengerindex{Schnitzler, Arthur@\textsc{Schnitzler, Arthur}!zzzEhrenstein, Albert@\emph{von Albert Ehrenstein}!1908-01-162@{16. 1. 1908}|)be}\mylabel{h}  \normalsize

\doendnotes{C}
\bigskip
\vfill

\clearpage

\footnotesize

\lohead{\textsc{register}}

% Definiere theindex-Environment komplett neu ohne reledmac
\makeatletter
\renewenvironment{theindex}{%
  \section*{\indexname}%
  \setlength{\parindent}{0pt}%
  \setlength{\parskip}{0pt plus 0.3pt}%
  \let\item\@idxitem
}{%
  \clearpage
}
\makeatother

\IfFileExists{\jobname-pw.ind}{\input{\jobname-pw.ind}}{}

\end{document}

      