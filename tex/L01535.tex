%% latex-korrekturansicht-vorspann.tex
%% Vorspann für die Korrekturansicht.
%% Lädt die gemeinsame Datei latex-vorspann.tex mit gesetztem Schalter.

\newif\ifkorrekturansicht
\korrekturansichttrue

\input{../tex-inputs/latex-vorspann}


               \section[Hermann Bahr an Arthur Schnitzler, 3. 8. 1905]{ Hermann Bahr an Arthur Schnitzler, 3. 8. 1905}\nopagebreak\mylabel{v}\rehead{ }\normalsize\beginnumbering\briefempfaengerindex{Schnitzler, Arthur@\textsc{Schnitzler, Arthur}!zzzBahr, Hermann@\emph{von Hermann Bahr}!1905-08-031@{3. 8. 1905}|(be} \toendnotes[C]{\smallbreak\pagebreak[2]} \Standort{CUL, Schnitzler, B 5b.}
\physDesc{Bildpostkarte
\newline{}Handschrift: Bleistift, deutsche Kurrent\newline{}Versand: Stempel: »\nobreak{}\oindex{Wuerzburg@\textbf{Würzburg}, \emph{Besiedelter Ort (A.BSO)}|pwk}Wuerzburg, 3 Aug 05, 8 Nm.\nobreak{}«.  \newline{}Ordnung: mit Bleistift von unbekannter Hand
                           nummeriert: »129« }\buchAbdrucke{\weitereDrucke{Hermann Bahr, Arthur Schnitzler: \emph{Briefwechsel, Aufzeichnungen, Dokumente (1891–1931)}. Hg. Kurt Ifkovits und Martin Anton Müller. Göttingen: \emph{Wallstein} 2018, S. 349.} }\toendnotes[C]{\smallbreak}\pstart{}{\pb}\textsc{Arthur Schnitzler}\pend{}\pstart{}\textcolor{pink}{\textsc{Wien XVIII}}{}\ledrightnote{\textcolor{pink}{XVIII., Währing}}\pend{}\pstart{}\textcolor{pink}{\textsc{Spöttlgasse 7}}{}\ledrightnote{\textcolor{pink}{Edmund-Weiß-Gasse}}\pend{}{\bigskip}\pstart
           \noindent{}\centering{}\textcolor{gray}{\textbf{{\pb}\textcolor{pink}{Platz’scher Garten}{}\ledrightnote{\textcolor{pink}{Platz’scher Garten}}}}\pend
           \pstart
           \noindent{}\centering{}\textcolor{gray}{\textbf{Bes. Fr. \textcolor{blue}{Kneuer}{}\ledrightnote{\textcolor{blue}{Franz Kneuer}}}}\pend
           \pstart
           \noindent{}\centering{}\textcolor{gray}{\textbf{Gruss aus \textcolor{pink}{Würzburg}{}\ledrightnote{\textcolor{pink}{Würzburg}}}}\pend
           \pstart
           \raggedleft{}{\pb}3. \label{T_L01535_1v}\edtext{7.}{\lemma{\textnormal{\emph{7.}}}\Cendnote{\textnormal{Schreibirrtum}}}\label{T_L01535_1h} 05\pend
           \pstart
           Dich u Deine liebe \textcolor{blue}{Frau}{}\ledrightnote{→\textcolor{blue}{Olga Schnitzler}} grüßt
               herzlichſt\pend
           \pstart \spacefill\mbox{HermannB.}\pend{}\endnumbering\briefempfaengerindex{Schnitzler, Arthur@\textsc{Schnitzler, Arthur}!zzzBahr, Hermann@\emph{von Hermann Bahr}!1905-08-031@{3. 8. 1905}|)be}\mylabel{h}  \normalsize

\doendnotes{C}
\bigskip
\vfill

\clearpage

\footnotesize

\lohead{\textsc{register}}

% Definiere theindex-Environment komplett neu ohne reledmac
\makeatletter
\renewenvironment{theindex}{%
  \section*{\indexname}%
  \setlength{\parindent}{0pt}%
  \setlength{\parskip}{0pt plus 0.3pt}%
  \let\item\@idxitem
}{%
  \clearpage
}
\makeatother

\IfFileExists{\jobname-pw.ind}{\input{\jobname-pw.ind}}{}

\end{document}

      