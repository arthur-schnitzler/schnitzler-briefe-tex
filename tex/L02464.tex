%% latex-korrekturansicht-vorspann.tex
%% Vorspann für die Korrekturansicht.
%% Lädt die gemeinsame Datei latex-vorspann.tex mit gesetztem Schalter.

\newif\ifkorrekturansicht
\korrekturansichttrue

\input{../tex-inputs/latex-vorspann}


               \section[Stefan Großmann an Arthur Schnitzler, 2. 1. 192{[}6?{]}]{ Stefan Großmann an Arthur Schnitzler, 2. 1. 192{[}6?{]}}\nopagebreak\mylabel{v}\rehead{ }\normalsize\beginnumbering\briefempfaengerindex{Schnitzler, Arthur@\textsc{Schnitzler, Arthur}!zzzGrossmann, Stefan@\emph{von Stefan Großmann}!1926-01-021@{2. 1. 192{[}6?{]}}|(be} \toendnotes[C]{\smallbreak\pagebreak[2]} \Standort{DLA, A:Schnitzler, HS.NZ85.1.3232.}
\physDesc{Brief, 1 Blatt, 1 Seite
\newline{}Schreibmaschine
\newline{}Handschrift: blaue Tinte (\noindent{}Unterschrift)
\newline{}Schnitzler: mit rotem Buntstift beschriftet: »\textsc{Großma{\geminationn}}« }\toendnotes[C]{\smallbreak}\pstart
           \noindent{}\centering{}{\pb}\textcolor{gray}{\textbf{Das \textcolor{brown}{Tage-Buch}{}\ledrightnote{\textcolor{brown}{Das Tage-Buch}}}}\pend
           \pstart
           \noindent{}\centering{}\textcolor{gray}{\textbf{\emph{Herausgeber: Stefan Großmann und \textcolor{blue}{Leopold Schwarzschild}{}\ledrightnote{\textcolor{blue}{Leopold Schwarzschild}}}}}\pend
           \pstart
           \noindent{}\centering{}\textcolor{gray}{\textbf{Tagebuchverlag m. b. H., \textcolor{pink}{Berlin
                        SW 19}{}\ledrightnote{\textcolor{pink}{Berlin}}}}\pend
           \pstart
           \noindent{}\centering{}\textcolor{gray}{\textbf{\textcolor{pink}{BEUTHSTRASSE 19}{}\ledrightnote{\textcolor{pink}{Beuthstrasse}}}}\pend
           \pstart
           \noindent{}\centering{}\textcolor{gray}{\textbf{\emph{Telegramm-Adresse: Tagebuch \textcolor{pink}{Berlin}{}\ledrightnote{\textcolor{pink}{Berlin}} ⋅ Fernsprecher: Merkur 8790–8792}}}\pend
           \pstart
           \noindent{}\centering{}\textcolor{gray}{\textbf{\emph{\so{Sprechstunde der Redaktion: 12–1 Uhr}}}}\pend
           \pstart
           \noindent{}\centering{}\textcolor{gray}{\textbf{*}}\pend
           \pstart
           \noindent{}Tgb./Gr./Schl.\hfill \textcolor{pink}{Berlin}{}\ledrightnote{\textcolor{pink}{Berlin}}, den \label{K_L02464_1v}\edtext{2. Januar 1925}{\lemma{\textnormal{\emph{2. Januar 1925}}}\Cendnote{\textnormal{Es dürfte sich bei der Jahreszahl
                        um einen Irrtum handeln, da ansonsten mehrere Korrespondenzstücke als
                        verlustig zu gelten hätten.}}}\label{K_L02464_1h}.\pend
           {\bigskip}\pstart
           \noindent{}\raggedleft{}Herrn\pend
           \pstart
           \noindent{}\raggedleft{}Dr. Arthur \so{Schnitzler}\pend
           \pstart
           \noindent{}\raggedleft{}\textcolor{pink}{\so{Wien XVIII}}{}\ledrightnote{\textcolor{pink}{XVIII., Währing}}\pend
           \pstart
           \noindent{}\raggedleft{}\textcolor{pink}{Sternwartestr. 71}{}\ledrightnote{\textcolor{pink}{Sternwartestraße}}.\pend
           \pstart\center{}Verehrter Herr Doktor!\pend\pstart
           Ich muss Ihrem liebenswürdigen Brief vom 24. v. Mts. widersprechen, wenn
               ich ihn etwa als eine Zurücknahme Ihrer freundlichen Zusage, einen Beitrag fürs \textcolor{brown}{TAGE-BUCH}{}\ledrightnote{\textcolor{brown}{Das Tage-Buch}} zu schicken, ansehen soll.\pend
           \pstart
           Sie würden mich zu grossem Dank verpflichten, wenn Sie recht bald Ihre Zusage
               erfüllen wollten.\pend
           \pstart
           Ich begrüsse Sie mit ausgezeichneter Hochschätzung{\\[\baselineskip]}Ihr sehr ergebener{\\[\baselineskip]}\spacefill\mbox{{[}hs.:{]} Großmann}\pend
           \leftskip=0em{}\endnumbering\briefempfaengerindex{Schnitzler, Arthur@\textsc{Schnitzler, Arthur}!zzzGrossmann, Stefan@\emph{von Stefan Großmann}!1926-01-021@{2. 1. 192{[}6?{]}}|)be}\mylabel{h}  \normalsize

\doendnotes{C}
\bigskip
\vfill

\clearpage

\footnotesize

\lohead{\textsc{register}}

% Definiere theindex-Environment komplett neu ohne reledmac
\makeatletter
\renewenvironment{theindex}{%
  \section*{\indexname}%
  \setlength{\parindent}{0pt}%
  \setlength{\parskip}{0pt plus 0.3pt}%
  \let\item\@idxitem
}{%
  \clearpage
}
\makeatother

\IfFileExists{\jobname-pw.ind}{\input{\jobname-pw.ind}}{}

\end{document}

      