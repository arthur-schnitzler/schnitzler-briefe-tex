%% latex-korrekturansicht-vorspann.tex
%% Vorspann für die Korrekturansicht.
%% Lädt die gemeinsame Datei latex-vorspann.tex mit gesetztem Schalter.

\newif\ifkorrekturansicht
\korrekturansichttrue

\input{../tex-inputs/latex-vorspann}


               \section[Arthur Schnitzler an Richard Beer-Hofmann, 4. 1. 1894]{ Arthur Schnitzler an Richard Beer-Hofmann, 4. 1. 1894}\nopagebreak\mylabel{v}\rehead{ }\normalsize\beginnumbering\briefempfaengerindex{Beer-Hofmann, Richard@\textsc{Beer-Hofmann, Richard}!zzzSchnitzler, Arthur@\emph{von Arthur Schnitzler}!1894-01-041@{4. 1. 1894}|(be} \toendnotes[C]{\smallbreak\pagebreak[2]} \Standort{YCGL, MSS 31.}
\physDesc{Briefkarte mit Trauerrand
\newline{}Handschrift: schwarze Tinte, deutsche Kurrent\newline{}Ordnung: mit Bleistift von unbekannter Hand unterhalb der Monatsangabe
                                 die alternative Datierung »5.« vermerkt }\toendnotes[C]{\smallbreak}\pstart
           \noindent{}{\pb}Lieber Richard, bitte ſenden Sie dem \textcolor{blue}{\textsc{Fels}}{}\ledrightnote{\textcolor{blue}{Friedrich Michael Fels}} möglichſt bald die beſprochenen Sachen; – auch das Geld können Sie direct an
               ihn ſenden; ich habe mich vergewiſſert, dß es ihn nicht beleidigen wird. –\pend
           \pstart
           Es iſt traurig, dß wir uns ſo ſelten ſehn. –\pend
           \pstart
           Morgen will ich entweder zur \textcolor{green}{böſen Nacht}{}\ledrightnote{\textcolor{green}{Eine böse Nacht. Lustspiel in 3 Acten}} oder zum
                  \label{K_L00289_1v}\edtext{\textcolor{green}{Bild des Signorelli}{}\ledrightnote{\textcolor{green}{Das Bild des Signorelli. Schauspiel in 4 Acten}}}{\lemma{\textnormal{\emph{Bild des Signorelli}}}\Cendnote{\textnormal{Er entschied sich für dafür und ging in
                  die Uraufführung ins \textcolor{pink}{Raimund-Theater}.}}}\label{K_L00289_1h}:
               Jedenfalls {\pb}könnten wir uns alle wieder einmal gegen
               eilf im \textcolor{pink}{Central}{}\ledrightnote{\textcolor{pink}{Café Central}} finden.\pend
           \pstart
           Herzliche Grüße{\\[\baselineskip]}Ihr{\\[\baselineskip]}\spacefill\mbox{Arthur}\pend
           \leftskip=0em{}\pstart
           4. 1. 94.\pend
           \endnumbering\briefempfaengerindex{Beer-Hofmann, Richard@\textsc{Beer-Hofmann, Richard}!zzzSchnitzler, Arthur@\emph{von Arthur Schnitzler}!1894-01-041@{4. 1. 1894}|)be}\mylabel{h}  \normalsize

\doendnotes{C}
\bigskip
\vfill

\clearpage

\footnotesize

\lohead{\textsc{register}}

% Definiere theindex-Environment komplett neu ohne reledmac
\makeatletter
\renewenvironment{theindex}{%
  \section*{\indexname}%
  \setlength{\parindent}{0pt}%
  \setlength{\parskip}{0pt plus 0.3pt}%
  \let\item\@idxitem
}{%
  \clearpage
}
\makeatother

\IfFileExists{\jobname-pw.ind}{\input{\jobname-pw.ind}}{}

\end{document}

      