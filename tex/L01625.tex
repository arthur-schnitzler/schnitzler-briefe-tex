%% latex-korrekturansicht-vorspann.tex
%% Vorspann für die Korrekturansicht.
%% Lädt die gemeinsame Datei latex-vorspann.tex mit gesetztem Schalter.

\newif\ifkorrekturansicht
\korrekturansichttrue

\input{../tex-inputs/latex-vorspann}


               \section[Arthur Schnitzler an Hugo von Hofmannsthal, 8. 9. 1906]{ Arthur Schnitzler an Hugo von Hofmannsthal, 8. 9. 1906}\nopagebreak\mylabel{v}\rehead{ }\normalsize\beginnumbering\briefempfaengerindex{Hofmannsthal, Hugo von@\textsc{Hofmannsthal, Hugo von}!zzzSchnitzler, Arthur@\emph{von Arthur Schnitzler}!1906-09-081@{8. 9. 1906}|(be} \toendnotes[C]{\smallbreak\pagebreak[2]} \Standort{FDH, Hs-30885,125.}
\physDesc{Brief, 1 Blatt, 4 Seiten
\newline{}Handschrift: schwarze Tinte, deutsche Kurrent}\buchAbdrucke{\weitereDrucke{Hugo von Hofmannsthal, Arthur Schnitzler: \emph{Briefwechsel}. Hg. Therese Nickl und Heinrich Schnitzler. Frankfurt am Main: \emph{S. Fischer} 1964, S. 221–222.} }\toendnotes[C]{\smallbreak}\pstart
           \raggedleft{}{\pb}\textcolor{pink}{Wien}{}\ledrightnote{\textcolor{pink}{Wien}}, 8. 9. 906\pend
           \pstart{}mein lieber Hugo, \pend\pstart
           auch unſer Sommer war gut. In \textcolor{pink}{\textsc{Marienlyst}}{}\ledrightnote{\textcolor{pink}{Marienlyst}} waren wir volle ſechs Wochen. Schöne Seebäder, höchſt
               anmuthige Waldſpaziergänge, ein angenehmes Hotel. Schrieb ein fünfactiges \textcolor{green}{Stück}{}\ledrightnote{→\textcolor{green}{Das Wort. Tragikomödie in fünf Akten}}, das natürlich vorläufig
               nicht zu brauchen iſt und von dem ich noch nicht weiſs, wa{\geminationn} ich es vollende. Auch einen \textcolor{green}{Einakter}{}\ledrightnote{→\textcolor{green}{Komtesse Mizzi oder Der Familientag}} hab ich ausführlich ſkizzirt. \textcolor{blue}{Salten}{}\ledrightnote{\textcolor{blue}{Felix Salten}} und \textcolor{blue}{Frau}{}\ledrightnote{→\textcolor{blue}{Ottilie Salten}} war \label{K_L01625_1v}\edtext{einen Nachmittag}{\lemma{\textnormal{\emph{einen Nachmittag}}}\Cendnote{\textnormal{siehe A. S.: \emph{Tagebuch}, 2. 8. 1906}}}\label{K_L01625_1h} bei uns, mit
                  \textcolor{blue}{Verwandten}{}\ledrightnote{→\textcolor{blue}{Richard Metzl}{\newline}→\textcolor{blue}{Wladimir Metzl}{\newline}→\textcolor{blue}{Metzl}{\newline}→\textcolor{blue}{Emil Salzmann}}. Schon nach Erledigung der \label{K_L01625_2v}\edtext{Umzugsfrage}{\lemma{\textnormal{\emph{Umzugsfrage}}}\Cendnote{\textnormal{Sie übersiedelten zum
                     15. 9. 1906 aus \textcolor{pink}{Berlin}
                  nach \textcolor{pink}{Wien}.}}}\label{K_L01625_2h}\strikeout{.}
               und daher in guter Sti{\geminationm}ung. Ich freu mich ſehr, daſs er
               wieder zu uns kommt. Frau \textcolor{blue}{Fulda}{}\ledrightnote{\textcolor{blue}{Ida d’Albert}} war ein paar
               Wochen in \textcolor{pink}{\textsc{Marienlyst}}{}\ledrightnote{\textcolor{pink}{Marienlyst}} und
               blieb noch nach unſrer Abreiſe. {\pb}Meine \textcolor{blue}{Schwägerin}{}\ledrightnote{→\textcolor{blue}{Elisabeth Steinrück}} war in \textcolor{pink}{\textsc{Gilleleje}}{}\ledrightnote{\textcolor{pink}{Gilleleje}}, nördlich von \textcolor{pink}{\textsc{Marienlyst}}{}\ledrightnote{\textcolor{pink}{Marienlyst}}, am offnen Meer, kam dann
               auf ein paar Tage, mit \textcolor{blue}{Steinrück}{}\ledrightnote{\textcolor{blue}{Albert Steinrück}} zu uns, wir
               fuhren gemeinſchaftlich nach \textsc{\textcolor{pink}{Kopenhagen}{}\ledrightnote{\textcolor{pink}{Kopenhagen}}}. Sie iſt jetzt in \textcolor{pink}{\textsc{Görbersdorf}}{}\ledrightnote{\textcolor{pink}{Görbersdorf}}, es geht ihr recht gut. Von \textcolor{pink}{\textsc{Kopenhagen}}{}\ledrightnote{\textcolor{pink}{Kopenhagen}} aus wurde \textcolor{blue}{Heini}{}\ledrightnote{\textcolor{blue}{Heinrich Schnitzler}}, dem das Meer ſehr imponirt hat und der
               jetzt wo er kann, mit ſeinen Reiſeerlebniſſen protzt, mit dem \textcolor{blue}{Fräulein}{}\ledrightnote{→\textcolor{blue}{Anna Loew}} nach \textcolor{pink}{Wien}{}\ledrightnote{\textcolor{pink}{Wien}}{ }ſpedirt. Wir zwei fuhren nach \textcolor{pink}{Weimar}{}\ledrightnote{\textcolor{pink}{Weimar}}, das uns aufs tiefſte ergriff. \textcolor{blue}{Fred}{}\ledrightnote{\textcolor{blue}{W. Fred}}, äußerſt ſympathiſch, aber recht leidend, war ein paar
               Tage mit uns zuſa{\geminationm}en. Von \textcolor{pink}{Weimar}{}\ledrightnote{\textcolor{pink}{Weimar}} nach \textcolor{pink}{Ilmenau}{}\ledrightnote{\textcolor{pink}{Ilmenau}}, auf den \textcolor{pink}{\textsc{Kickelhahn}}{}\ledrightnote{\textcolor{pink}{Kickelhahn}}; von \textcolor{pink}{\textsc{Ilmenau}}{}\ledrightnote{\textcolor{pink}{Ilmenau}} zu Wagen, {\pb}durch den reizvollen \textcolor{pink}{Thüringerwald}{}\ledrightnote{\textcolor{pink}{Thüringer Wald}}, über die \textcolor{pink}{Schmücke}{}\ledrightnote{\textcolor{pink}{Schmücke}}, nach
                  \textcolor{pink}{Oberhof}{}\ledrightnote{\textcolor{pink}{Oberhof}}, das ſich ganz alpenhaft geberdet,
               gleich weiter nach \textcolor{pink}{Eiſenach}{}\ledrightnote{\textcolor{pink}{Eisenach}}, nach \textcolor{pink}{Nürnberg}{}\ledrightnote{\textcolor{pink}{Nürnberg}}, wo wir das hübſche Marionettentheater von \textcolor{blue}{Brann}{}\ledrightnote{\textcolor{blue}{Paul Brann}}{ }ſahen, und von da nach \textcolor{pink}{Wien}{}\ledrightnote{\textcolor{pink}{Wien}}. Hier ſind wir ſeit beinah drei Wochen. \textcolor{blue}{Olga}{}\ledrightnote{\textcolor{blue}{Olga Schnitzler}} ließ ſich von \textcolor{blue}{Julius}{}\ledrightnote{\textcolor{blue}{Julius Schnitzler}} eine
               Kleinigkeit an den Füßen \label{T_L01625_1v}\edtext{operiren}{\lemma{\textnormal{\emph{operiren}}}\Cendnote{\textnormal{geschrieben:
                  »operirte«}}}\label{T_L01625_1h}, ſo dſs ſie noch nicht Tennis ſpielen kann. Ich
               hingegen ſehr fleißig, beinah täglich. Mit \textsc{\textcolor{blue}{Wassermann}{}\ledrightnote{\textcolor{blue}{Jakob Wassermann}}, \textcolor{blue}{Agnes
                  Speyer}{}\ledrightnote{\textcolor{blue}{Agnes Ulmann}}, \textcolor{blue}{Speidel}{}\ledrightnote{\textcolor{blue}{Felix Speidel}}} u \textcolor{blue}{Frau}{}\ledrightnote{→\textcolor{blue}{Else Speidel-Haeberle}}. Arbeite wenig. Beſchäftigt mit einem \textcolor{green}{Stück}{}\ledrightnote{→\textcolor{green}{Fink und Fliederbusch. Komödie in drei Akten}}, das ich ſchon
               vor 3 Jahren begonnen habe (modern.) – Morgen fahren wir alle auf den \textcolor{pink}{Semmering}{}\ledrightnote{\textcolor{pink}{Semmering}}, für etwa {\pb}acht Tage. Es
               wäre nicht unmöglich, dſs ich für meinen Theil von dort aus noch weiterwandere oder
               radle, vielleicht mit \textcolor{blue}{Waſſermann}{}\ledrightnote{\textcolor{blue}{Jakob Wassermann}}, ins \textcolor{pink}{Salzka{\geminationm}ergut}{}\ledrightnote{\textcolor{pink}{Salzkammergut}}. Laſſen Sie
               mich jedenfalls wiſſen (\textcolor{pink}{Südbahnhotel}{}\ledrightnote{\textcolor{pink}{Südbahnhotel}}) wie lange Sie
               noch in \textcolor{pink}{Lueg}{}\ledrightnote{\textcolor{pink}{Lueg am Wolfgangsee}} bleiben. Hiemit wäre das äußerliche der
               letzten Monate und der nächſten Zukunft in Kürze mitgetheilt; es gab im übrigen recht
               viele gute Stunden aber mehr hypochondriſche als mit Ruhe zu tragen wären.
               Künſtleriſche Intenſitäten wurden \substVorne{}\textsuperscript{mehr}\substDazwischen{}häufiger\substHinten{} auf Spaziergängen durchlebt als am Schreibtiſch, und die neueſten Geſtalten
               laſſen ſich wohl bis ins tiefſte erkennen aber nicht bis ins letzte regieren. Ich
               freue mich auf unſer nächſtes Zuſa{\geminationm}enſein und erhoffe es
               bald.\pend
           \pstart
           Herzlichst Ihr{\\[\baselineskip]}\spacefill\mbox{A.}\pend
           \leftskip=0em{}\endnumbering\briefempfaengerindex{Hofmannsthal, Hugo von@\textsc{Hofmannsthal, Hugo von}!zzzSchnitzler, Arthur@\emph{von Arthur Schnitzler}!1906-09-081@{8. 9. 1906}|)be}\mylabel{h}  \normalsize

\doendnotes{C}
\bigskip
\vfill

\clearpage

\footnotesize

\lohead{\textsc{register}}

% Definiere theindex-Environment komplett neu ohne reledmac
\makeatletter
\renewenvironment{theindex}{%
  \section*{\indexname}%
  \setlength{\parindent}{0pt}%
  \setlength{\parskip}{0pt plus 0.3pt}%
  \let\item\@idxitem
}{%
  \clearpage
}
\makeatother

\IfFileExists{\jobname-pw.ind}{\input{\jobname-pw.ind}}{}

\end{document}

      