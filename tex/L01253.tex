%% latex-korrekturansicht-vorspann.tex
%% Vorspann für die Korrekturansicht.
%% Lädt die gemeinsame Datei latex-vorspann.tex mit gesetztem Schalter.

\newif\ifkorrekturansicht
\korrekturansichttrue

\input{../tex-inputs/latex-vorspann}


               \section[Gerhart Hauptmann an Arthur Schnitzler, {[}29. 11. 1902{]}]{ Gerhart Hauptmann an Arthur Schnitzler, {[}29. 11. 1902{]}}\nopagebreak\mylabel{v}\rehead{ }\normalsize\beginnumbering\briefempfaengerindex{Schnitzler, Arthur@\textsc{Schnitzler, Arthur}!zzzHauptmann, Gerhart@\emph{von Gerhart Hauptmann}!1902-11-291@{{[}29. 11. 1902{]}}|(be} \toendnotes[C]{\smallbreak\pagebreak[2]} \Standort{Staatsbibliothek Berlin – Preußischer Kulturbesitz, Autogr. I/4486.}
\physDesc{Brief, 1 Blatt, 1 Seite
\newline{}Handschrift: Bleistift, lateinische Kurrent
\newline{}Schnitzler: mit Bleistift datiert: »29. 11. 902« }\Standort{DLA, A:Schnitzler, HS.NZ85.1.3362.}
\physDesc{1 Blatt, 1 Seite, Fotokopie}\toendnotes[C]{\smallbreak}\pstart
           \noindent{}{\pb}\textcolor{gray}{\textbf{\textcolor{pink}{\label{T_L01253-1v}\edtext{HOTEL SACHER}{\lemma{\textnormal{\emph{Hotel Sacher}}}\Cendnote{\textnormal{Briefkopf mit gedrucktem
                                 Wappen}}}\label{T_L01253-1h}}{}\ledrightnote{\textcolor{pink}{Hotel Sacher}}}}\hfill \textcolor{gray}{\textbf{\textcolor{pink}{WIEN}{}\ledrightnote{\textcolor{pink}{Wien}}, .......... 19{\dots}}}\pend
           \pstart
           Herzlichen Gruss, lieber Herr Schnitzler, ich freue mich, dass Sie
               doch noch kommen.\pend
           \pstart
           Ihr{\\[\baselineskip]}\spacefill\mbox{G. Hptm}\pend
           \leftskip=0em{}\endnumbering\briefempfaengerindex{Schnitzler, Arthur@\textsc{Schnitzler, Arthur}!zzzHauptmann, Gerhart@\emph{von Gerhart Hauptmann}!1902-11-291@{{[}29. 11. 1902{]}}|)be}\mylabel{h}  \normalsize

\doendnotes{C}
\bigskip
\vfill

\clearpage

\footnotesize

\lohead{\textsc{register}}

% Definiere theindex-Environment komplett neu ohne reledmac
\makeatletter
\renewenvironment{theindex}{%
  \section*{\indexname}%
  \setlength{\parindent}{0pt}%
  \setlength{\parskip}{0pt plus 0.3pt}%
  \let\item\@idxitem
}{%
  \clearpage
}
\makeatother

\IfFileExists{\jobname-pw.ind}{\input{\jobname-pw.ind}}{}

\end{document}

      