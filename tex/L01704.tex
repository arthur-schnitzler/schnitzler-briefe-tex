%% latex-korrekturansicht-vorspann.tex
%% Vorspann für die Korrekturansicht.
%% Lädt die gemeinsame Datei latex-vorspann.tex mit gesetztem Schalter.

\newif\ifkorrekturansicht
\korrekturansichttrue

\input{../tex-inputs/latex-vorspann}


               \section[Hugo von Hofmannsthal an Arthur Schnitzler, {[}Anfang September 1907{]}]{ Hugo von Hofmannsthal an Arthur Schnitzler, {[}Anfang
               September 1907{]}}\nopagebreak\mylabel{v}\rehead{ }\normalsize\beginnumbering\briefempfaengerindex{Schnitzler, Arthur@\textsc{Schnitzler, Arthur}!zzzHofmannsthal, Hugo von@\emph{von Hugo von Hofmannsthal}!1907-09-011@{{[}Anfang
                  September 1907{]}}|(be} \toendnotes[C]{\smallbreak\pagebreak[2]} \Standort{CUL, Schnitzler, B 43.}
\physDesc{Telegramm
\newline{}maschinell
\newline{}Schnitzler: mit Bleistift datiert »Anf Sep 907« \newline{}Ordnung: 1) beschnitten 2) mit Bleistift von unbekannter Hand nummeriert:
                              »288«}\buchAbdrucke{\weitereDrucke{Hugo von Hofmannsthal, Arthur Schnitzler: \emph{Briefwechsel}. Hg. Therese Nickl und Heinrich Schnitzler. Frankfurt am Main: \emph{S. Fischer} 1964, S. 231.} }\toendnotes[C]{\smallbreak}\pstart
           {\pb}\textcolor{pink}{m{[}e{]}ran}{}\ledrightnote{\textcolor{pink}{Meran}} fr \textcolor{pink}{semmering}{}\ledrightnote{\textcolor{pink}{Semmering}} 1+ 359 20 10 40 m\pend
           \pstart
           mindestens \label{K_L01704_1v}\edtext{fuenfzehn}{\lemma{\textnormal{\emph{fuenfzehn}}}\Cendnote{\textnormal{Es
                  handelt sich um Überlegungen, den Vorabdruck von \emph{\textcolor{green}{Der
                     Weg ins Freie}} der Zeitschrift \emph{\textcolor{brown}{Morgen}} zu
                  geben, bei der \textcolor{blue}{Hofmannsthal} beteiligt
                  war.}}}\label{K_L01704_1h} viellejcht zwanzig da ja sonst \textcolor{green}{abdruck}{}\ledrightnote{→\textcolor{green}{Der Weg ins Freie. Roman}}{ }\textcolor{brown}{rundschau}{}\ledrightnote{\textcolor{brown}{Neue Rundschau, Neue Deutsche Rundschau, Freie Bühne}} viel erfreulicher wuerde zwanzig
               vorschlagen herzlich\pend
           \pstart \spacefill\mbox{hugo}\pend{}\endnumbering\briefempfaengerindex{Schnitzler, Arthur@\textsc{Schnitzler, Arthur}!zzzHofmannsthal, Hugo von@\emph{von Hugo von Hofmannsthal}!1907-09-011@{{[}Anfang
                  September 1907{]}}|)be}\mylabel{h}  \normalsize

\doendnotes{C}
\bigskip
\vfill

\clearpage

\footnotesize

\lohead{\textsc{register}}

% Definiere theindex-Environment komplett neu ohne reledmac
\makeatletter
\renewenvironment{theindex}{%
  \section*{\indexname}%
  \setlength{\parindent}{0pt}%
  \setlength{\parskip}{0pt plus 0.3pt}%
  \let\item\@idxitem
}{%
  \clearpage
}
\makeatother

\IfFileExists{\jobname-pw.ind}{\input{\jobname-pw.ind}}{}

\end{document}

      