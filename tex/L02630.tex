%% latex-korrekturansicht-vorspann.tex
%% Vorspann für die Korrekturansicht.
%% Lädt die gemeinsame Datei latex-vorspann.tex mit gesetztem Schalter.

\newif\ifkorrekturansicht
\korrekturansichttrue

\input{../tex-inputs/latex-vorspann}


               \section[Paul Goldmann an Arthur Schnitzler, 31. 12. {[}1894{]}]{ Paul Goldmann an Arthur Schnitzler, 31. 12. {[}1894{]}}\nopagebreak\mylabel{v}\rehead{ }\normalsize\beginnumbering\briefempfaengerindex{Schnitzler, Arthur@\textsc{Schnitzler, Arthur}!zzzGoldmann, Paul@\emph{von Paul Goldmann}!1894-12-311@{31. 12. {[}1894{]}}|(be} \toendnotes[C]{\smallbreak\pagebreak[2]} \Standort{DLA, A:Schnitzler, HS.NZ85.1.3164.}
\physDesc{Brief, 3 Blätter, 11 Seiten
\newline{}Handschrift: schwarze Tinte, deutsche Kurrent
\newline{}Schnitzler: 1) mit Bleistift auf dem ersten Blatt die Jahreszahl
                                       »94« vermerkt 2) mit rotem Buntstift sieben Unterstreichungen}\toendnotes[C]{\smallbreak}\pstart
           \noindent{}{\pb}\textcolor{gray}{\textbf{\textcolor{brown}{Frankfurter Zeitung}{}\ledrightnote{\textcolor{brown}{Frankfurter Zeitung}}.}}\hfill \textsc{\textcolor{pink}{Paris}{}\ledrightnote{\textcolor{pink}{Paris}}},
                        31. December.\pend
           \pstart
           \textcolor{gray}{\textbf{(\textcolor{brown}{Gazette de
                  Francfort}{}\ledrightnote{\textcolor{brown}{Frankfurter Zeitung}}.)}}\pend
           \pstart
           \textcolor{gray}{\textbf{Fondateur \textbf{M. \textcolor{blue}{L. Sonnemann}{}\ledrightnote{\textcolor{blue}{Leopold Sonnemann}}}.}}\pend
           \pstart
           \textcolor{gray}{\textbf{\begin{otherlanguage}{french}Journal politique,
                        financier,\end{otherlanguage}}}\pend
           \pstart
           \textcolor{gray}{\textbf{\begin{otherlanguage}{french}commercial et
                     littéraire.\end{otherlanguage}}}\pend
           \pstart
           \textcolor{gray}{\textbf{\begin{otherlanguage}{french}\textbf{Paraissant trois fois
                           par jour.}\end{otherlanguage}}}\pend
           \pstart
           \textcolor{gray}{\textbf{–}}\pend
           \pstart
           \textcolor{gray}{\textbf{\begin{otherlanguage}{french}\textbf{Bureau à \textcolor{pink}{Paris}{}\ledrightnote{\textcolor{pink}{Paris}}:}\end{otherlanguage}}}\pend
           \pstart
           \textcolor{gray}{\textbf{\begin{otherlanguage}{french}\textcolor{pink}{24. Rue
                           Feydeau}{}\ledrightnote{\textcolor{pink}{rue Feydeau}}.\end{otherlanguage}}}\pend
           \pstart\center{}Mein lieber Freund,\pend\pstart
           das ſind recht erfreuliche Nachrichten, – unberufen! – die Dein Brief bringt. \label{K_L02630-1v}\edtext{\textsc{\textcolor{blue}{Speidel}{}\ledrightnote{\textcolor{blue}{Ludwig Speidel}}}}{\lemma{\textnormal{\emph{Speidel}}}\Cendnote{\textnormal{Zum
                  positiven Urteil \textcolor{blue}{Ludwig Speidel}s über die \emph{\textcolor{green}{Liebelei}}{ }vgl. A. S.: \emph{Tagebuch}, 14. 12. 1894, 17. 12. 1894 und 18. 12. 1894}}}\label{K_L02630-1h}
               beſonders iſt eine förmliche Überraſchung. Der Mann, der \substVorne{}\textsuperscript{\textcolor{gray}{×}\-\textcolor{gray}{×}}\substDazwischen{}bei\substHinten{} der Lampe nach Mitternacht über Deinem \textcolor{green}{Stücke}{}\ledrightnote{→\textcolor{green}{Liebelei. Schauspiel in drei Akten}} ſitzt, wird mir beinahe ſympathiſch. \strikeout{H} Sollten wir ihm vielleicht Unrecht gethan haben? Er
               war gegen das Neue; aber hat es denn viel Neues gegeben? Und haben wir nicht am Ende
               das Neue mit uns verwechſelt, die wir neu waren? Das Urtheil, das er über Dich fällt,
               ſpricht ſehr zu Ehren {\pb}ſeines Kunſtverſtändniſſes.
               Nun kann es doch unmöglich mehr fehlen. Wo ſoviel Mächtige dafür ſind, wird das
               Theater-Geſindel nichts mehr ausrichten können. Daß \textcolor{blue}{B.}{}\ledrightnote{\textcolor{blue}{Max Eugen Burckhard}} Dich \label{K_L02630-2v}\edtext{beſucht}{\lemma{\textnormal{\emph{beſucht}}}\Cendnote{\textnormal{vgl. A. S.: \emph{Tagebuch}, 18. 12. 1894}}}\label{K_L02630-2h}, imponirt mir beſonders. Welchen
               Weg haſt Du durchlaufen \strikeout{zwiſchen} von drei Jahren bis
               auf heut! Mir kommt ſo vor, als ſei jetzt nur noch ein tüchtiger Ruck zu
               geben, und dann am Ziel! Wenn ſich die \textsc{\textcolor{blue}{Sandrock}{}\ledrightnote{\textcolor{blue}{Adele Sandrock}}} vom \label{K_mets_Goldmann_94-partII-5v}\edtext{\textcolor{brown}{Volkstheater}{}\ledrightnote{\textcolor{brown}{Volkstheater}} jetzt ſchon losmachen}{\lemma{\textnormal{\emph{Volkstheater … losmachen}}}\Cendnote{\textnormal{\textcolor{blue}{Adele
                     Sandrock} war für die Rolle der \textcolor{green}{Christine} vorgesehen. Der Wechsel ans \emph{\textcolor{brown}{Burgtheater}} war schon im Sommer 1894 für die
                  Saison 1895/1896 ausgemacht. Durch neuerliche
                  Verhandlungen fand der Übertritt bereits zum 1. 2. 1895
                  statt.}}}\label{K_mets_Goldmann_94-partII-5h} könnte, ſo wäre es wohl gut (Warum ſpielt übrigens die \textsc{\textcolor{blue}{Hohenfels}{}\ledrightnote{\textcolor{blue}{Stella Hohenfels}}} nicht die
               Rolle?). Wenn nicht, ſo warteſt Du ruhig bis zum nächſten Jahr. Der Titel »\textcolor{green}{Liebelei}{}\ledrightnote{\textcolor{green}{Liebelei. Schauspiel in drei Akten}}« mißfällt mir. {\pb}Er klingt maniriert, unliterariſch und verkleinert
               die Arbeit. Ich möchte, daß Du auf die kleine \textsc{Nuance}
               verzichteſt und einfach gerade heraus »Eine Liebſchaft« ſagſt. Das klingt mehr nach
               bürgerlichem Drama. Und nun werde ich endlich ungeduldig. Alle Welt hat ſchon über
               dem \textcolor{green}{Stücke}{}\ledrightnote{→\textcolor{green}{Liebelei. Schauspiel in drei Akten}} geſeſſen, mit \strikeout{B} Bangen und ohne. Ich weiß allerlei Urtheile und kenne
               es ſelber noch nicht. Könnteſt Du es mir nicht auf wenige Tage zugänglich machen? Ich
               leſe es in einem Tage aus und ſchicke es ſofort zurück. Bitte, bitte, mach’ es
               irgendwie möglich; Du kannſt Dir denken, wie geſpannt {\pb}ich bin. Die Spannung wächſt mit jeder neuen
               Nachricht. Nun muß ichs endlich kennen lernen, zum Teufel auch! Und, nicht wahr,
               ſobald Cenſur und Intendanz geſprochen haben, theilſt Du mir ſofort das Reſultat mit?
               Schreib’ mir auch, ob die \label{K_L02630-3v}\edtext{\textcolor{brown}{Frankf. Ztg.}{}\ledrightnote{\textcolor{brown}{Frankfurter Zeitung}}}{\lemma{\textnormal{\emph{Frankf. Ztg.}}}\Cendnote{\textnormal{XXXX}}}\label{K_L02630-3h} etwas darüber bringen ſoll. Einſtweilen
               beglückwünſche ich Dich von Herzen zu den bisherigen guten
                  Reſultaten{[}.{]}{ }\textsc{\textcolor{blue}{Speidel}{}\ledrightnote{\textcolor{blue}{Ludwig Speidel}}} iſt bereits der
               halbe Erfolg. Ich freue mich ſehr{\dotsfive}\pend
           \pstart
           In einem der nächſten Hefte des »\textsc{\textcolor{brown}{Mercure de France}{}\ledrightnote{\textcolor{brown}{Mercure de France}}}« kommt ein \label{K_L02630-12v}\edtext{\textcolor{green}{Aufſatz}{}\ledrightnote{→\textcolor{green}{Les Jeunes Viennois}}}{\lemma{\textnormal{\emph{Aufſatz}}}\Cendnote{\textnormal{Der Text erschien mit
                  einer gewissen Verzögerung in einer anderen Zeitschrift: \textcolor{blue}{Henri Albert}: \emph{\textcolor{green}{Les Jeunes
                        Viennois}}. In: \emph{\textcolor{green}{Revue des revues}},
                     Bd. 13, 1. 4. 1895, S. 8–13.}}}\label{K_L02630-12h} von
                  \textsc{\textcolor{blue}{Albert}{}\ledrightnote{\textcolor{blue}{Henri Albert}}} über Euch.
               Leider hat er mich nicht um Rath {\pb}beim Schreiben
               gefragt. Es ſtehen alſo offenbar einige Stiefel drin. Aber die Haupttache iſt doch,
               daß etwas geſchrieben wird. Auch will er nächſtens \label{K_L02630-4v}\edtext{etwas}{\lemma{\textnormal{\emph{etwas}}}\Cendnote{\textnormal{nicht
                  ermittelt}}}\label{K_L02630-4h} von Dir überſetzen. Wie macht ſich der literariſche und
               buchhändleriſche Erfolg von »\textcolor{green}{Sterben}{}\ledrightnote{\textcolor{green}{Sterben. Novelle}}«?\pend
           \pstart
           Was hört man von der »\textcolor{brown}{Zeit}{}\ledrightnote{\textcolor{brown}{Die Zeit. Wiener Wochenschrift}}«? Wie geht ſie und wie
               gefällt ſie?\pend
           \pstart
           Gern will ich Dir die \textcolor{brown}{Frankf. Ztg.}{}\ledrightnote{\textcolor{brown}{Frankfurter Zeitung}} ſchicken, wenn
               ich etwas darin habe. Aber ich habe kaum mehr etwas drin. Kann {\pb}mich nicht mehr zum Schreiben aufraffen. Es liegen
               Centnerlaſten auf mir. Die Krankheit, die nicht heilen will – Ihr Ärzte ſeid nichts
               als menſchenfreundliche Lügner – die Vereinſamung, die Heimatloſigkeit, das Gefühl
               des Zurückbleibens, die Verlotterung. Wie ich aus \textsc{\textcolor{pink}{Ischl}{}\ledrightnote{\textcolor{pink}{Bad Ischl}}} zurückkam, wollte ich eine Rieſen-Anſtrengung
               machen. Die iſt mißlungen, und nun laſſe ich mich ſinken und leiſte nur mehr wenig
               Widerſtand. Ich leſe nicht ein Mal mehr ein Buch zu Ende; und wenn die Reue kommt, ſo
               ſlüchte ich mich in Politik und Depeſchen hinein.\pend
           \pstart
           {\pb}Den Brief an Frl. \textsc{\textcolor{blue}{Sandrock}{}\ledrightnote{\textcolor{blue}{Adele Sandrock}}} habe ich endlich geſchrieben. Es
               war keine Kleinigkeit. Ich ſollte meine Anſicht über das Leben mittheilen. Das iſt
               nicht leicht, wenn man viel zu thun hat. Ich habe ein idiotiſches Zeug abgeſchickt,
                  \textsc{\label{K_mets_Goldmann_94-partII-66v}\edtext{mais enfin}{\lemma{\textnormal{\emph{mais enfin}}}\Cendnote{\textnormal{französisch: aber zuletzt}}}\label{K_mets_Goldmann_94-partII-66h}},
               ich habe geantwortet.\pend
           \pstart
           Ich möchte ein wenig wiſſen, wie Du lebſt? Geſellſchaft? Freundſchaſt? Abenteuer?\pend
           \pstart
           \label{K_mets_Goldmann_94-partII-77v}\edtext{\textsc{\textcolor{blue}{Bahr}{}\ledrightnote{\textcolor{blue}{Hermann Bahr}}} hat mich neulich in ſehr
               liebenswürdiger Weiſe citirt}{\lemma{\textnormal{\emph{Bahr … citirt}}}\Cendnote{\textnormal{Sein \textcolor{green}{Text} beginnt mit: »Als ich diesen Mai in
                        \textcolor{pink}{Paris} mit \textcolor{blue}{Paul Goldmann}, dem \textcolor{blue}{Correspondenten} der \textcolor{brown}{Frankfurter
                        Zeitung}, plauderte und um jeden Preis ein neues Talent wissen wollte,
                     sagte er mir: ›Ein Talent? Ein neues Talent? Ein ernstes, sicheres, wirkliches
                     Talent? Nicht bloß so eine geschwinde und vergängliche Erfindung der Journale
                     von heute auf morgen? Das ist schwer. Da ist jetzt wohl niemand als \textcolor{blue}{Camille Mauclair}. Sonst wüßte ich keinen. Er
                     hat freilich eigentlich noch nichts geschrieben; aber alle hoffen viel von ihm.
                     Er verspricht mehr, als er bis jetzt gehalten hätte; aber er scheint mir
                     sicher. Stellen Sie sich etwa, ins \textcolor{pink}{Paris}erische übersetzt, Ihren kleinen \textcolor{blue}{Hofmannsthal} vor.‹« (\textcolor{blue}{Hermann Bahr}: \emph{\textcolor{green}{Camille Mauclair}}. In:
                        \emph{\textcolor{green}{Die Zeit}}, Bd. 1, H. 10,
                        8. 12. 1894, S. 154–155.)}}}\label{K_mets_Goldmann_94-partII-77h}. Warum
               hat er das gethan?\pend
           \pstart
           Ich mache mir Vorwürſe, daß ich Dich zum Abonnement auf das {\pb}»\textcolor{brown}{Journal}{}\ledrightnote{→\textcolor{brown}{Le Journal}}« aufgefordert habe. Es wird niederträchtig ſchlecht. Vielleicht
               verſuchſt Du es fortan mit der Abendausgabe des »\textsc{\textcolor{brown}{Journal des Débats}{}\ledrightnote{\textcolor{brown}{Journal des débats}}}«. Die politiſchen Artikel
               brauchſt Du ja nicht zu leſen; aber es ſind köſtliche \textsc{\label{K_mets_Goldmann_94-partII-666v}\edtext{chroniqueurs}{\lemma{\textnormal{\emph{chroniqueurs}}}\Cendnote{\textnormal{französisch:
                     Kolumnisten}}}\label{K_mets_Goldmann_94-partII-666h}} darin, höhere literariſche Leute: \textsc{\textcolor{blue}{Hallays}{}\ledrightnote{\textcolor{blue}{André Hallays}}, \textcolor{blue}{Bazin}{}\ledrightnote{\textcolor{blue}{René Bazin}}, \textcolor{blue}{Filon}{}\ledrightnote{\textcolor{blue}{Augustin Filon}}, \textcolor{blue}{Lemaître}{}\ledrightnote{\textcolor{blue}{Jules Lemaître}}{ }}\textsc{etc}. Willſt Du, daß ichs Dir abonnire? Noch habe ich \textsc{30 Francs 30 ct.}, die Du beharrlich todtſchweigſt. Hat \textsc{\textcolor{blue}{Richard}{}\ledrightnote{\textcolor{blue}{Richard Beer-Hofmann}}} den »\label{K_L02630-66v}\edtext{\textcolor{green}{Courrier
                  Français}{}\ledrightnote{\textcolor{green}{Le Courrier français}}}{\lemma{\textnormal{\emph{Courrier
                  Français}}}\Cendnote{\textnormal{illustrierte \textcolor{green}{Satirezeitschrift}, die zwischen
                     1884 und 1914 erschien}}}\label{K_L02630-66h}« abonnirt? Sonſt
               ſchicke ich ihn Dir. Anbei ſchicke ich Dir wieder ein paar \label{K_mets_Goldmann_94-partII-55v}\edtext{Artikel}{\lemma{\textnormal{\emph{Artikel}}}\Cendnote{\textnormal{Die Beilagen sind nicht
               überliefert.}}}\label{K_mets_Goldmann_94-partII-55h}, Kraut und Rüben durcheinander. \textsc{\textcolor{blue}{Drumont}{}\ledrightnote{\textcolor{blue}{Édouard Drumont}}} iſt ein großer {\pb}Polemiſt, nur ſtark irrſinnig. In Bezug auf Juden
               und Deutſche leidet er an Verfolgungswahn. Aber in erſterer Beziehung beginnt der
               Irrſinn doch erſt nach einer weiten Grenze; Vieles Unglaubliche, was er über jüdiſche
               Corruption ſchreibt, iſt wahr. Auch iſt er größenwahnſinnig und kommt ſich
               thatſächlich als gottgeſandter Meſſias vor. Anderſeits gibt ihm aber gerade nur
               dieſer Wahnſinn die ungeheure Kraft, mit der er manchmal dreinſchlägt.\pend
           \pstart
           {\pb}\textsc{\textcolor{blue}{Sokal}{}\ledrightnote{\textcolor{blue}{Clemens Sokal}}} war bei mir; er
               gefällt mir gut. Scheint ein geſcheiter und ernſter Menſch zu ſein{\dotsfour}\pend
           \pstart
           Ich wünſche Dir von Herzen Glück zum neuen Jahr. Mir ahnt, daß das Jahr
                  1895 wichtig für Dich werden wird. Sieht es nicht vertrauenerweckend
               aus? Mit ſeiner runden Fünfheiten!\pend
           \pstart
           Was aber auch geſchehen mag, Gutes oder Allerbeſtes, wir bleiben die Alten, nicht
               wahr?\pend
           \pstart
           Herzlichſt und in Treue Dein{\\[\baselineskip]}\spacefill\mbox{Paul Goldmann\textcolor{gray}{.}}\pend
           \leftskip=0em{}\pstart
           \noindent{}{\pb}Bitte, empfiehl’ mich Deiner Frau \textcolor{blue}{Mutter}{}\ledrightnote{→\textcolor{blue}{Louise Schnitzler}} und richte \label{T_mets_Goldmann_94-partII-22v}\edtext{ihr}{\lemma{\textnormal{\emph{ihr}}}\Cendnote{\textnormal{er schreibt
                        »Ihr«}}}\label{T_mets_Goldmann_94-partII-22h} meine ergebenſten Neujahrs-Wünſche aus.\pend
           \pstart
           Was lieſt Du jetzt?\pend
           \endnumbering\briefempfaengerindex{Schnitzler, Arthur@\textsc{Schnitzler, Arthur}!zzzGoldmann, Paul@\emph{von Paul Goldmann}!1894-12-311@{31. 12. {[}1894{]}}|)be}\mylabel{h}  \normalsize

\doendnotes{C}
\bigskip
\vfill

\clearpage

\footnotesize

\lohead{\textsc{register}}

% Definiere theindex-Environment komplett neu ohne reledmac
\makeatletter
\renewenvironment{theindex}{%
  \section*{\indexname}%
  \setlength{\parindent}{0pt}%
  \setlength{\parskip}{0pt plus 0.3pt}%
  \let\item\@idxitem
}{%
  \clearpage
}
\makeatother

\IfFileExists{\jobname-pw.ind}{\input{\jobname-pw.ind}}{}

\end{document}

      