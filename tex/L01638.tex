%% latex-korrekturansicht-vorspann.tex
%% Vorspann für die Korrekturansicht.
%% Lädt die gemeinsame Datei latex-vorspann.tex mit gesetztem Schalter.

\newif\ifkorrekturansicht
\korrekturansichttrue

\input{../tex-inputs/latex-vorspann}


               \section[Arthur Schnitzler an Hugo von Hofmannsthal, 27. 11. 1906]{ Arthur Schnitzler an Hugo von Hofmannsthal, 27. 11. 1906}\nopagebreak\mylabel{v}\rehead{ }\normalsize\beginnumbering\briefempfaengerindex{Hofmannsthal, Hugo von@\textsc{Hofmannsthal, Hugo von}!zzzSchnitzler, Arthur@\emph{von Arthur Schnitzler}!1906-11-271@{27. 11. 1906}|(be} \toendnotes[C]{\smallbreak\pagebreak[2]} \Standort{FDH, Hs-30885,126.}
\physDesc{Brief, 1 Blatt, 4 Seiten
\newline{}Handschrift: schwarze Tinte, deutsche Kurrent}\buchAbdrucke{\weitereDrucke{Hugo von Hofmannsthal, Arthur Schnitzler: \emph{Briefwechsel}. Hg. Therese Nickl und Heinrich Schnitzler. Frankfurt am Main: \emph{S. Fischer} 1964, S. 224.} }\toendnotes[C]{\smallbreak}\pstart
           \raggedleft{}{\pb}\textcolor{pink}{Wien}{}\ledrightnote{\textcolor{pink}{Wien}}, 27. Nov 906\pend
           \pstart
           lieber Hugo, ſchönen Dank für das \label{K_L01638_1v}\edtext{Buch}{\lemma{\textnormal{\emph{Buch}}}\Cendnote{\textnormal{unklar; die
                  kurze Erwähnung deutet auf kein bedeutenderes Werk hin. Zwar könnte es sich um den
                  ersten Band der zwölfbändigen Ausgabe von \emph{\textcolor{green}{Tausendundeine Nacht}} in der Übersetzung von \textcolor{blue}{Felix Paul Greve} (\emph{\textcolor{brown}{Insel-Verlag}}, Ausgabe ab November 1906) handeln, dessen \textcolor{green}{Vorrede} in Folge erwähnt wird,
                  doch ist diese auch unmittelbar vor dem Brief am 25. 11. 1906 in \emph{\textcolor{green}{Der Tag}} erschienen.}}}\label{K_L01638_1h}. Außerordentlich habe
               ich Ihre \textcolor{green}{Vorrede}{}\ledrightnote{→\textcolor{green}{Vorrede}} zu »\textcolor{green}{Tauſend und eine Nacht}{}\ledrightnote{\textcolor{green}{Tausendundeine Nacht}}«, dann Ihren \textcolor{green}{Artikel}{}\ledrightnote{→\textcolor{green}{Die unvergleichliche Tänzerin}} über die Tänzerin \textcolor{blue}{Ruth}{}\ledrightnote{\textcolor{blue}{Ruth Saint Denis}} gefunden. In früherer Zeit war in ſolchen Aufſätzen von
               Ihnen zuweilen ein oder das andere Wort enthalten, das ſich zu hoch davonſchwang, ſo
               daſs \substVorne{}\textsuperscript{zuweilen}{\allowbreak}\substDazwischen{}manchmal\substHinten{} gerade eine beſondere Schönheit mir den Rythmus des ganzen ein wenig ſtörte.
               Jetzt iſt Gleichmaß und {\pb}Flügelhaftigkeit auch dieſen
               Aufſätzen ſo vollkommen eigen, \strikeout{daſs man} und die
               Eigenart \strikeout{iſt} Ihres Proſaſtils iſt zugleich ſo gewahrt
               und ſo erhöht worden, daſs man für dieſe Produkte am liebſten einen eignen Namen
               erſinnen möchte. Sehr ſchön waren auch die \textcolor{green}{Dialoge}{}\ledrightnote{→\textcolor{green}{Unterhaltungen über ein neues Buch}} über die »\textcolor{green}{Schweſtern}{}\ledrightnote{\textcolor{green}{Die Schwestern. Drei Novellen}}«, beſonders der zweite Artikel. Wunderbar iſt es Ihnen gelungen,
               den Widerſtreit der Empfindungen auszudrücken, mit dem man dem ganzen Problem {\pb}\textcolor{blue}{Waſſermann}{}\ledrightnote{\textcolor{blue}{Jakob Wassermann}} gegenüberſteht, indem Sie, wohl auch
               zu eigner Beruhigung, Ihre Seele dialogiſch aufgelöſt und ſich dazu bekannt haben,
               daſs wir nicht nur der Welt, den Erlebniſſen, den Menſchen, ſondern auch jener
               einzigen Einheitlichkeit die wir Kunſtwerk nennen, durchaus nicht einheitlich,
               ſondern zugleich onkel- majors- mädchen- gutsbeſitzer- träumerhaft ins Auge ſchauen.
               Gewöhnlich ſchreibt über die Dinge Einer, der nur ein Onkel, {\pb}nur ein Träumer, nur ein Mädchen iſt. All dies ließe ſich
               richtiger ausdrücken, wozu mir die Sa{\geminationm}lung in dieſem
               Augenblicke fehlt.\pend
           \pstart
           Hoffentlich ſieht man ſich wieder we{\geminationn} Sie \label{K_L01638_2v}\edtext{zurückkehren}{\lemma{\textnormal{\emph{zurückkehren}}}\Cendnote{\textnormal{Er ist von 28. 11. bis 16. 12. 1906 in
                     \textcolor{pink}{Deutschland} unterwegs.}}}\label{K_L01638_2h}, aus \textcolor{pink}{München}{}\ledrightnote{\textcolor{pink}{München}}, \textcolor{pink}{Göttingen}{}\ledrightnote{\textcolor{pink}{Göttingen}}, \textcolor{pink}{Berlin}{}\ledrightnote{\textcolor{pink}{Berlin}}. Laſſen Sie gelegentlich
               was von ſich hören.\pend
           \pstart
           Herzlichst{\\[\baselineskip]}Ihr{\\[\baselineskip]}\spacefill\mbox{Arthur.}\pend
           \leftskip=0em{}\endnumbering\briefempfaengerindex{Hofmannsthal, Hugo von@\textsc{Hofmannsthal, Hugo von}!zzzSchnitzler, Arthur@\emph{von Arthur Schnitzler}!1906-11-271@{27. 11. 1906}|)be}\mylabel{h}  \normalsize

\doendnotes{C}
\bigskip
\vfill

\clearpage

\footnotesize

\lohead{\textsc{register}}

% Definiere theindex-Environment komplett neu ohne reledmac
\makeatletter
\renewenvironment{theindex}{%
  \section*{\indexname}%
  \setlength{\parindent}{0pt}%
  \setlength{\parskip}{0pt plus 0.3pt}%
  \let\item\@idxitem
}{%
  \clearpage
}
\makeatother

\IfFileExists{\jobname-pw.ind}{\input{\jobname-pw.ind}}{}

\end{document}

      