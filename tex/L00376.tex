%% latex-korrekturansicht-vorspann.tex
%% Vorspann für die Korrekturansicht.
%% Lädt die gemeinsame Datei latex-vorspann.tex mit gesetztem Schalter.

\newif\ifkorrekturansicht
\korrekturansichttrue

\input{../tex-inputs/latex-vorspann}


               \section[Arthur Schnitzler an Richard Beer-Hofmann, 5. 10. 1894]{ Arthur Schnitzler an Richard Beer-Hofmann, 5. 10. 1894}\nopagebreak\mylabel{v}\rehead{ }\normalsize\beginnumbering\briefempfaengerindex{Beer-Hofmann, Richard@\textsc{Beer-Hofmann, Richard}!zzzSchnitzler, Arthur@\emph{von Arthur Schnitzler}!1894-10-051@{5. 10. 1894}|(be} \toendnotes[C]{\smallbreak\pagebreak[2]} \Standort{YCGL, MSS 31.}
\physDesc{Brief, 2 Blätter, 5 Seiten, Umschlag
\newline{}Handschrift: Bleistift, deutsche Kurrent\newline{}Versand: 1) Stempel: »\nobreak{}\oindex{I., Innere Stadt@\textbf{I., Innere Stadt}, \emph{Bezirk (A.BZK)}|pwk}Wien 1/1, 5. 10. 94, 8–9 V\nobreak{}«.  2) Stempel: »\nobreak{}\oindex{Rom@\textbf{Rom}, \emph{Besiedelter Ort (A.BSO)}|pwk}Rom, 7 10-94, 2 S\nobreak{}«. 3) nachgesandt nach »\textcolor{pink}{\textsc{Hôtel Quirinal}}«}\buchAbdrucke{\weitereDrucke{1) Arthur Schnitzler: \emph{Briefe 1875–1912}. Hg. Therese Nickl und Heinrich Schnitzler. Frankfurt am Main: \emph{S. Fischer} 1981, S. 229–230.} \weitereDrucke{2) Arthur Schnitzler, Richard Beer-Hofmann: \emph{Briefwechsel 1891–1931}. Hg. Konstanze Fliedl. Wien, Zürich: \emph{Europaverlag} 1992, S. 62–63.} }\toendnotes[C]{\smallbreak}\pstart{}{\pb}\textsc{Dr. Arthur Schnitzler}, \textcolor{pink}{Wien,
                     IX. Frankg. 1.}{}\ledrightnote{\textcolor{pink}{Frankgasse}}\pend{}{\bigskip}\pstart{}{\pb}Herrn \textsc{Dr. Richard
                     Beer-Hofmann}\pend{}\pstart{}\textsc{\textcolor{pink}{Rom}{}\ledrightnote{\textcolor{pink}{Rom}}}\pend{}\pstart{}\textsc{a posta ferma}\pend{}\pstart{}\textsc{\textcolor{pink}{Italien}{}\ledrightnote{\textcolor{pink}{Italien}}}\pend{}{\bigskip}\pstart
           \raggedleft{}{\pb}\textcolor{pink}{Wien}{}\ledrightnote{\textcolor{pink}{Wien}}, 5. Oct
                  94.\pend
           \pstart{}Lieber Bekannter!\pend\pstart
           Das einzige, was Sie mir von Ihrer \textcolor{pink}{italien}{}\ledrightnote{\textcolor{pink}{Italien}}. Reiſe
               mittheilen, iſt daſs mein \textcolor{green}{\textcolor{blue}{\textsc{Guercino}}{}\ledrightnote{\textcolor{blue}{Guercino}}}{}\ledrightnote{→\textcolor{green}{Die Verstoßung der Hagar}} in \textcolor{pink}{Mailand}{}\ledrightnote{\textcolor{pink}{Mailand}} hängt. Das ſteht aber ſchon im »\textcolor{blue}{\textcolor{green}{\textsc{Lübke}}{}\ledrightnote{→\textcolor{green}{Grundriß der Kunstgeschichte}}}{}\ledrightnote{\textcolor{blue}{Wilhelm Lübke}}« – ich muſs Sie alſo, we{\geminationn} Sie überhaupt die
               Abſicht haben, Neuigkeiten aus \textcolor{pink}{Italien}{}\ledrightnote{\textcolor{pink}{Italien}} an mich zu
               ſchreiben, um sorgfältigere Auswahl bitten. Laſſen Sie ſich nicht etwa einfallen, mir
               aus \textcolor{pink}{Rom}{}\ledrightnote{\textcolor{pink}{Rom}} zu ſchreiben, daſs dort \textcolor{blue}{\textsc{Julius Caesar}}{}\ledrightnote{\textcolor{blue}{Gaius Iulius Caesar}} ermordet wurde – es ſteht im \textcolor{blue}{\textcolor{green}{Ploetz}{}\ledrightnote{→\textcolor{green}{Auszug aus der alten, mittleren und neueren Geschichte}}}{}\ledrightnote{\textcolor{blue}{Karl Ploetz}}! – Dagegen bin ich gern {\pb}bereit, perſönlicheres
               von Ihnen zu erfahren – haben Sie keine von den \label{K_L00376_1v}\edtext{\textcolor{green}{Schweſtern Rondoli}{}\ledrightnote{→\textcolor{green}{Die Schwestern Rondoli}}}{\lemma{\textnormal{\emph{Schweſtern Rondoli}}}\Cendnote{\textnormal{In der Novelle von \textcolor{blue}{Maupassant} hat die männliche Hauptfigur auf einer Reise eine
                  Liebschaft mit einer Frau, im Folgejahr mit ihrer Schwester.}}}\label{K_L00376_1h} getroffen? –
               Beantworten Sie mir auch gütigſt einige Fragen. 1.) Wa{\geminationn}
                  ko{\geminationm}en Sie zurück? 2.) Wie weit werden Sie Ihre Reiſe
               ausdehnen. 3) Haben Sie was geſchrieben?\pend
           \pstart
           Einige Thatſachen: \textcolor{blue}{\uline{Ludaßy}}{}\ledrightnote{\textcolor{blue}{Julius von Gans-Ludassy}} iſt Chefred. der \textcolor{brown}{Wr. Allg. Ztg.}{}\ledrightnote{\textcolor{brown}{Wiener Allgemeine Zeitung}} (mit einem
               nicht übeln Gehalt) worden. Er rechnet auf das ganze junge \textcolor{pink}{Wien}{}\ledrightnote{\textcolor{pink}{Wien}}; »alſo« auch auf Sie. (Die Gänſefüße ſind 17gradig.) –\pend
           \pstart
           Morgen iſt die »\textcolor{green}{Schmetterlingsſchlacht}{}\ledrightnote{\textcolor{green}{Die Schmetterlingsschlacht}}« – ich hab
                  {\pb}noch keinen Sitz, was mich geradezu aufregt. –\pend
           \pstart
           »Man sagt« iſt durchgefallen. –\pend
           \pstart
           Mein \textcolor{green}{Stück}{}\ledrightnote{→\textcolor{green}{Liebelei. Schauspiel in drei Akten}} (gefährliche
               Nachbarſchaft der Thatſachen – Sie ſehen, ich bin nicht abergläubiſch, oder erſt
               recht, oder erſt recht gar nicht, oder gar nicht erſt recht gar nicht – ) ist {\dots} hier stock’ ich ſchon — vollendet? {\dotstwo} Nein. Beendet? Nein. Fertig? – Nein. – Ich habe »nur mehr« dran zu feilen.
               Hab ich Ihnen den Titel ſchon geſchrieben?{\dotstwo} »\textcolor{green}{Liebelei}{}\ledrightnote{\textcolor{green}{Liebelei. Schauspiel in drei Akten}}«. – Anfangs wird er ihnen wahrſcheinlich
               nicht {\pb}gefallen; aber er iſt gut, – auch praktiſch
                  geno{\geminationm}en. –\pend
           \pstart
           Ich lese: \textsc{\textcolor{blue}{Rosenkranz}{}\ledrightnote{\textcolor{blue}{Karl Rosenkranz}}, \textcolor{green}{\textcolor{blue}{Diderot}{}\ledrightnote{\textcolor{blue}{Denis Diderot}}}{}\ledrightnote{\textcolor{green}{Diderots Leben und Werke}}; – \textcolor{blue}{Keller}{}\ledrightnote{\textcolor{blue}{Otto Keller}}}, \textcolor{green}{Musikgeschichte}{}\ledrightnote{\textcolor{green}{Geschichte der Musik}} u. a. –\pend
           \pstart
           Vorgeleſen wurde mir – ein fünfaktiges Drama in Verſen, in dem aber gewiſs Talent
               ſteckt; \textcolor{green}{\textsc{Phryne}}{}\ledrightnote{\textcolor{green}{Die Athenerin}} von \textcolor{blue}{\textsc{Leo Ebermann}}{}\ledrightnote{\textcolor{blue}{Leo Ebermann}}, der mich aber als Menſch und beſonders als Vorleſer ſehr nervös macht: er
               poſirt auf guten Sprecher{\dots}\pend
           \pstart
           Phrrryne{\dotstwo}\pend
           \pstart
           Gawiſs {\dotstwo} du darrrfſt nicht länger lebohn{\dots}\pend
           \pstart
           Meine Gerechtigkeit hat Orgien {\pb}gefeiert; eigentlich
               wollte ich ihm ununterbrochen Ihre Büſte »in’ \strikeout{den} Kop\substVorne{}\textsuperscript{f}\substDazwischen{}p\substHinten{} hereinhaun«. – (Lachen Sie nicht; der Kellner beobachtet Sie. –)\pend
           \pstart
           Leben Sie wohl, ſchreiben Sie mir, und ſeien Sie herzlichſt gegrüßt.\pend
           \pstart Ihr \spacefill\mbox{Arthur}\pend{}\endnumbering\briefempfaengerindex{Beer-Hofmann, Richard@\textsc{Beer-Hofmann, Richard}!zzzSchnitzler, Arthur@\emph{von Arthur Schnitzler}!1894-10-051@{5. 10. 1894}|)be}\mylabel{h}  \normalsize

\doendnotes{C}
\bigskip
\vfill

\clearpage

\footnotesize

\lohead{\textsc{register}}

% Definiere theindex-Environment komplett neu ohne reledmac
\makeatletter
\renewenvironment{theindex}{%
  \section*{\indexname}%
  \setlength{\parindent}{0pt}%
  \setlength{\parskip}{0pt plus 0.3pt}%
  \let\item\@idxitem
}{%
  \clearpage
}
\makeatother

\IfFileExists{\jobname-pw.ind}{\input{\jobname-pw.ind}}{}

\end{document}

      