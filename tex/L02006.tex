%% latex-korrekturansicht-vorspann.tex
%% Vorspann für die Korrekturansicht.
%% Lädt die gemeinsame Datei latex-vorspann.tex mit gesetztem Schalter.

\newif\ifkorrekturansicht
\korrekturansichttrue

\input{../tex-inputs/latex-vorspann}


               \section[Arthur Schnitzler an Stefan Großmann, {[}8. 2. 1911{]}]{ Arthur Schnitzler an Stefan Großmann, {[}8. 2. 1911{]}}\nopagebreak\mylabel{v}\rehead{ }\normalsize\beginnumbering\briefempfaengerindex{Grossmann, Stefan@\textsc{Großmann, Stefan}!zzzSchnitzler, Arthur@\emph{von Arthur Schnitzler}!1911-02-081@{{[}8. 2. 1911{]}}|(be} \toendnotes[C]{\smallbreak\pagebreak[2]} \Standort{CUL, Schnitzler, B 34.}
\physDesc{Brief, 1 Blatt, 1 Seite, Entwurf
\newline{}Handschrift: Bleistift, deutsche Kurrent\newline{}Ordnung: von unbekannter Hand nummeriert:
                                            »9a« \newline{}Zusatz: Die Existenz des Briefes, um dessen Entwurf es sich
                                            hier handelt, wird durch die Karte vom selben Tag
                                            gestützt. }\pstart{}{\pb}S. g. H.\pend\pstart
           auf Ihre Anfrage theile ich Ihnen mit, daſs ich dem von Ihnen
                        genannt\textcolor{gray}{en} Herr\textcolor{gray}{n}{ }\textcolor{blue}{E.}{}\ledrightnote{\textcolor{blue}{Albert Ehrenstein}} niemals bestätg habe, Sie
                        benützt\textcolor{gray}{en} Ihre Macht als Kritiker zu erot Erpreſſungen –
                    erſtens weil ich dergleich über Sie überhaupt \uline{niemals} verno{\geminationm}\textcolor{gray}{en} hab u 2. weil ich Klat\textcolor{gray}{ſ}ch wn ich ſchon
                    nicht vermei\textcolor{gray}{d} ka{\geminationn}, ihn
                        z\textcolor{gray}{u} hören, wed rede, verbreite
                    \textcolor{gray}{gflg}.\pend
           \endnumbering\briefempfaengerindex{Grossmann, Stefan@\textsc{Großmann, Stefan}!zzzSchnitzler, Arthur@\emph{von Arthur Schnitzler}!1911-02-081@{{[}8. 2. 1911{]}}|)be}\mylabel{h}  \normalsize

\doendnotes{C}
\bigskip
\vfill

\clearpage

\footnotesize

\lohead{\textsc{register}}

% Definiere theindex-Environment komplett neu ohne reledmac
\makeatletter
\renewenvironment{theindex}{%
  \section*{\indexname}%
  \setlength{\parindent}{0pt}%
  \setlength{\parskip}{0pt plus 0.3pt}%
  \let\item\@idxitem
}{%
  \clearpage
}
\makeatother

\IfFileExists{\jobname-pw.ind}{\input{\jobname-pw.ind}}{}

\end{document}

      