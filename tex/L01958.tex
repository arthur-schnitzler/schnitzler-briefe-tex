%% latex-korrekturansicht-vorspann.tex
%% Vorspann für die Korrekturansicht.
%% Lädt die gemeinsame Datei latex-vorspann.tex mit gesetztem Schalter.

\newif\ifkorrekturansicht
\korrekturansichttrue

\input{../tex-inputs/latex-vorspann}


               \section[Hermann Bahr an Arthur Schnitzler, 26. 9. 1910]{ Hermann Bahr an Arthur Schnitzler, 26. 9. 1910}\nopagebreak\mylabel{v}\rehead{ }\normalsize\beginnumbering\briefempfaengerindex{Schnitzler, Arthur@\textsc{Schnitzler, Arthur}!zzzBahr, Hermann@\emph{von Hermann Bahr}!1910-09-261@{26. 9. 1910}|(be} \toendnotes[C]{\smallbreak\pagebreak[2]} \Standort{CUL, Schnitzler, B 5b.}
\physDesc{Brief, 1 Blatt, 3 Seiten
\newline{}Handschrift Lisa Clarus: schwarze Tinte, lateinische Kurrent\newline{}Handschrift Hermann Bahr: schwarze Tinte, deutsche Kurrent (\noindent{}Unterschrift, Nachschrift)
\newline{}Schnitzler: mit Bleistift ergänzt »Bahr« \newline{}Ordnung: mit Bleistift von unbekannter Hand nummeriert:
                                    »166« }\buchAbdrucke{\weitereDrucke{Hermann Bahr, Arthur Schnitzler: \emph{Briefwechsel, Aufzeichnungen, Dokumente (1891–1931)}. Hg. Kurt Ifkovits und Martin Anton Müller. Göttingen: \emph{Wallstein} 2018, S. 437–438.} }\toendnotes[C]{\smallbreak}\pstart
           \raggedleft{}{\pb}\textcolor{pink}{Wien XIII/\textsubscript{7}}{}\ledrightnote{\textcolor{pink}{Ober Sankt Veit}}\hspace*{1.5em}26. 9. 10.\pend
           \pstart\center{}Lieber Arthur!\pend\pstart
           Ich fahre \label{K_L01958_1v}\edtext{Samstag für vier Wochen nach
                  \textcolor{pink}{London}{}\ledrightnote{\textcolor{pink}{London}}}{\lemma{\textnormal{\emph{Samstag … London}}}\Cendnote{\textnormal{Am 1. 10. 1910 begleitete
                     \textcolor{blue}{Bahr} seine \textcolor{blue}{Frau} nach \textcolor{pink}{London}, am
                     28. 10. 1910 waren sie wieder in \textcolor{pink}{Wien}.}}}\label{K_L01958_1h} und so werde ich leider \label{LL327-1v}bei Deiner \textcolor{green}{Première}{}\ledrightnote{→\textcolor{green}{Der junge Medardus. Dramatische Historie in einem Vorspiel und fünf Aufzügen}} nicht in
                     \textcolor{pink}{Wien}{}\ledrightnote{\textcolor{pink}{Wien}}\label{LL327-1h}{ }ſein. Es wäre mir aber lieb, wenn Du mir (entweder
               noch vor Samstag hieher oder dann nach \textcolor{pink}{London E. C.
                  Victoria Embankment, De Keysers Hotel}{}\ledrightnote{\textcolor{pink}{De Keysers Royal Hotel}}) ein \label{K_L01958_2v}\edtext{Buch Deines neuen \textcolor{green}{Stückes}{}\ledrightnote{→\textcolor{green}{Der junge Medardus. Dramatische Historie in einem Vorspiel und fünf Aufzügen}}}{\lemma{\textnormal{\emph{Buch … Stückes}}}\Cendnote{\textnormal{\textcolor{blue}{Arthur Schnitzler}: \emph{\textcolor{green}{Der junge Medardus. Dramatische Historie in einem Vorspiel und
                        fünf Aufzügen}}. Berlin: \emph{\textcolor{brown}{S. Fischer}}{ }1910 (am 26. 10. 1910 vom \emph{\textcolor{green}{Börsenblatt für den deutschen Buchhandel}} als
                     Neuerscheinung gemeldet).}}}\label{K_L01958_2h}{ }ſchicken könntest. Macht es mir einen starken
               Eindruck und habe ich darüber wirklich etwas zu sagen, so würde ich das für das \textcolor{green}{Wiener Journal}{}\ledrightnote{\textcolor{green}{Neues Wiener Journal}} von \textcolor{pink}{London}{}\ledrightnote{\textcolor{pink}{London}} aus tun und veranlassen, dass von irgend einem der Herren {\pb}der Redaktion eine kurze Notiz über die Aufführung
               und Aufnahme angehängt werde. Kann ich aber in kein rechtes inneres Verhältnis dazu
               kommen, woran ja ebenso der Autor wie der Kritiker schuld sein kann, so ist es uns
               beiden besser, wenn ich die Gelegenheit zu schweigen ausnütze, statt mich um das
               Stück herum zu reden, was mir, je älter ich werde, immer unleidlicher wird.\pend
           \pstart
           Im November komme ich nur auf ein paar Tage zurück, weil ich gleich
               wieder an den Rhein, auf eine \label{K_L01958_3v}\edtext{Vorlesungstournée}{\lemma{\textnormal{\emph{Vorlesungstournée}}}\Cendnote{\textnormal{Vom 17. 11. bis 3. 12. 1910 war \textcolor{blue}{Bahr} auf einer umfangreichen Tournee durch \textcolor{pink}{Deutschland}.}}}\label{K_L01958_3h} muss. Aber im
                  Dezember wird es uns dann doch einmal vergönnt sein, in Ruhe zu \textcolor{blue}{Euch}{}\ledrightnote{→\textcolor{blue}{Olga Schnitzler}} zu kommen.\pend
           \pstart
           {\pb}Mit herzlichen Grüssen von \textcolor{blue}{uns}{}\ledrightnote{→\textcolor{blue}{Anna Bahr-Mildenburg}} beiden, auch an Deine liebe \textcolor{blue}{Frau}{}\ledrightnote{→\textcolor{blue}{Olga Schnitzler}},{\\[\baselineskip]}Dein alter{\\[\baselineskip]}\spacefill\mbox{{[}hs. Bahr:{]} Hermann}\pend
           \leftskip=0em{}\pstart
           \noindent{}viel zu nervös, um ſelbſt ſchreiben zu können.\pend
           \endnumbering\briefempfaengerindex{Schnitzler, Arthur@\textsc{Schnitzler, Arthur}!zzzBahr, Hermann@\emph{von Hermann Bahr}!1910-09-261@{26. 9. 1910}|)be}\mylabel{h}  \normalsize

\doendnotes{C}
\bigskip
\vfill

\clearpage

\footnotesize

\lohead{\textsc{register}}

% Definiere theindex-Environment komplett neu ohne reledmac
\makeatletter
\renewenvironment{theindex}{%
  \section*{\indexname}%
  \setlength{\parindent}{0pt}%
  \setlength{\parskip}{0pt plus 0.3pt}%
  \let\item\@idxitem
}{%
  \clearpage
}
\makeatother

\IfFileExists{\jobname-pw.ind}{\input{\jobname-pw.ind}}{}

\end{document}

      