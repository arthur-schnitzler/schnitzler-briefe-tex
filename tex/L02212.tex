%% latex-korrekturansicht-vorspann.tex
%% Vorspann für die Korrekturansicht.
%% Lädt die gemeinsame Datei latex-vorspann.tex mit gesetztem Schalter.

\newif\ifkorrekturansicht
\korrekturansichttrue

\input{../tex-inputs/latex-vorspann}


               \section[Arthur Schnitzler an Richard Beer-Hofmann, {[}30.? 6. 1915{]}]{ Arthur Schnitzler an Richard Beer-Hofmann, {[}30.? 6. 1915{]}}\nopagebreak\mylabel{v}\rehead{ }\normalsize\beginnumbering\briefempfaengerindex{Beer-Hofmann, Richard@\textsc{Beer-Hofmann, Richard}!zzzSchnitzler, Arthur@\emph{von Arthur Schnitzler}!1915-06-301@{{[}30.? 6. 1915{]}}|(be} \toendnotes[C]{\smallbreak\pagebreak[2]} \Standort{CUL, Schnitzler, B 8.1, S. 148.}
\physDesc{maschinelle Abschrift
\newline{}Schreibmaschine\newline{}Zusatz: von unbekannter Hand als Briefnummer »335«
                                 gekennzeichnet }\buchAbdrucke{\weitereDrucke{Arthur Schnitzler, Richard Beer-Hofmann: \emph{Briefwechsel 1891–1931}. Hg. Konstanze Fliedl. Wien, Zürich: \emph{Europaverlag} 1992, S. 221.} }\toendnotes[C]{\smallbreak}\pstart
           \raggedleft{}{\pb}\textcolor{pink}{Wien}{}\ledrightnote{\textcolor{pink}{Wien}}, ? 1915.\pend
           \pstart
           Lieber Richard, Dr. \textcolor{blue}{Reik}{}\ledrightnote{\textcolor{blue}{Theodor Reik}} wohnt
               in \textcolor{pink}{Wien XVIII. Lazaristengasse 2}{}\ledrightnote{\textcolor{pink}{Lazaristengasse}}. – Auf dem \textcolor{pink}{Semmering}{}\ledrightnote{\textcolor{pink}{Semmering}} wars sehr schön (das Essen leider
               unmöglich, und lächerlich theuer), – seit \label{K_L02212_1v}\edtext{Montag}{\lemma{\textnormal{\emph{Montag}}}\Cendnote{\textnormal{28. 6. 1915, was eine
                  unmittelbare Antwort an \textcolor{blue}{Beer-Hofmann}
                  wahrscheinlich macht.}}}\label{K_L02212_1h} sind wir zurück und seither regnet’s. Dass wir
                  Ende Juli nach \textcolor{pink}{Ischl}{}\ledrightnote{\textcolor{pink}{Bad Ischl}} kommen (auf
               circa 10–14 Tage) ist nicht unwahrscheinlich. Lassen Sie sichs wohl ergehn, und all
               den Ihren. Wir grüssen herzlichst. Ihr \spacefill\mbox{Arthur.}\pend
           \pstart
           \noindent{}(nach \textcolor{pink}{Ischl, Doctor-Sterz-Weg 14}{}\ledrightnote{\textcolor{pink}{Doktor-Sterz-Weg}})\pend
           \endnumbering\briefempfaengerindex{Beer-Hofmann, Richard@\textsc{Beer-Hofmann, Richard}!zzzSchnitzler, Arthur@\emph{von Arthur Schnitzler}!1915-06-301@{{[}30.? 6. 1915{]}}|)be}\mylabel{h}  \normalsize

\doendnotes{C}
\bigskip
\vfill

\clearpage

\footnotesize

\lohead{\textsc{register}}

% Definiere theindex-Environment komplett neu ohne reledmac
\makeatletter
\renewenvironment{theindex}{%
  \section*{\indexname}%
  \setlength{\parindent}{0pt}%
  \setlength{\parskip}{0pt plus 0.3pt}%
  \let\item\@idxitem
}{%
  \clearpage
}
\makeatother

\IfFileExists{\jobname-pw.ind}{\input{\jobname-pw.ind}}{}

\end{document}

      