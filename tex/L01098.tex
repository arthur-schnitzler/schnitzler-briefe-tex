%% latex-korrekturansicht-vorspann.tex
%% Vorspann für die Korrekturansicht.
%% Lädt die gemeinsame Datei latex-vorspann.tex mit gesetztem Schalter.

\newif\ifkorrekturansicht
\korrekturansichttrue

\input{../tex-inputs/latex-vorspann}


               \section[Hermann Bahr an Arthur Schnitzler, 20. 2. {[}1901{]}]{ Hermann Bahr an Arthur Schnitzler, 20. 2. {[}1901{]}}\nopagebreak\mylabel{v}\rehead{ }\normalsize\beginnumbering\briefempfaengerindex{Schnitzler, Arthur@\textsc{Schnitzler, Arthur}!zzzBahr, Hermann@\emph{von Hermann Bahr}!1901-02-201@{20. 2. 1901}|(be} \toendnotes[C]{\smallbreak\pagebreak[2]} \Standort{CUL, Schnitzler, B 5b.}
\physDesc{Brief, 1 Blatt, 2 Seiten
\newline{}Handschrift: schwarze Tinte, deutsche Kurrent
\newline{}Schnitzler: mit Bleistift die Jahreszahl »901.« ergänzt \newline{}Ordnung: mit Bleistift von unbekannter Hand nummeriert:
                                    »75« }\buchAbdrucke{\weitereDrucke{Hermann Bahr, Arthur Schnitzler: \emph{Briefwechsel, Aufzeichnungen, Dokumente (1891–1931)}. Hg. Kurt Ifkovits und Martin Anton Müller. Göttingen: \emph{Wallstein} 2018, S. 197.} }\toendnotes[C]{\smallbreak}\pstart
           \noindent{}\centering{}{\pb}\textcolor{gray}{\textbf{\textcolor{brown}{Redaktion des Neuen Wiener Tagblatt}{}\ledrightnote{\textcolor{brown}{Neues Wiener Tagblatt}}}}\pend
           \pstart
           \noindent{}\centering{}\textcolor{gray}{\textbf{\textsc{\textcolor{pink}{Wien, I., Rothenturmstrasse,
                        Steyrerhof}{}\ledrightnote{\textcolor{pink}{Steyrerhof}}.}}}\pend
           \pstart
           \noindent{}\centering{}\textcolor{gray}{\textbf{Telegramm-Adresse: \textcolor{brown}{Tagblatt}{}\ledrightnote{\textcolor{brown}{Neues Wiener Tagblatt}},
                        \textcolor{pink}{Steyrerhof, Wien}{}\ledrightnote{\textcolor{pink}{Steyrerhof}}. – Telephon Nr. 384.
                     Staats-Telephon Nr. 36.}}\pend
           \pstart
           20. Febr.\pend
           \pstart\center{}Lieber Arthur!\pend\pstart
           Ich habe, in einer zu meinem \textcolor{blue}{Kraus}{}\ledrightnote{\textcolor{blue}{Karl Kraus}}-Proceß
               gehörenden Angelegenheit, \uline{dringendſt} mit Dir, ſo bald
               als irgend möglich, \substVorne{}\textsuperscript{\textcolor{gray}{mi}}\substDazwischen{}zu\substHinten{}{ }ſprechen und bitte Dich deshalb, mich morgen, ſo
               bald Du aufgeſtanden biſt, telephoniſch (an \textcolor{blue}{\textsc{Bukovics}}{}\ledrightnote{\textcolor{blue}{Emerich von Bukovics}}, \label{K_L01098_1v}\edtext{\textcolor{pink}{Ober St. Veiter}{}\ledrightnote{\textcolor{pink}{Ober Sankt Veit}} Wohnung}{\lemma{\textnormal{\emph{Ober St. Veiter Wohnung}}}\Cendnote{\textnormal{Eigentlich ein Haus. Auf einem Teil des ursprünglich zu diesem Haus
                  gehörenden Grundstücks hatte \textcolor{blue}{Bahr} sein Haus
                  errichtet.}}}\label{K_L01098_1h}) wiſſen zu laſſen, wann und wo ich Dich treffen kann. Ich bin
               auf Dein Aviſo parat, ſofort \label{K_L01098_2v}\edtext{nach \textcolor{pink}{Wien}{}\ledrightnote{\textcolor{pink}{Wien}} zu fahren}{\lemma{\textnormal{\emph{nach Wien zu fahren}}}\Cendnote{\textnormal{\textcolor{pink}{Ober Sankt Veit} war bis zur Eingliederung in \textcolor{pink}{Wien}{ }1892 eine eigenständige Gemeinde, was sich in dieser Aussage
                  offensichtlich tradiert.}}}\label{K_L01098_2h} u. eine Stunde ſpäter überall zu ſein, wo es Dir
               paßt. Nur bitte, beſtimmt vor vier {\pb}Uhr und, wenn
               es irgendwie früher angeht, je früher, deſto beſſer.\pend
           \pstart
           Verzeih die Störung{\\[\baselineskip]}Deinem{\\[\baselineskip]}herzlich grüßenden{\\[\baselineskip]}\spacefill\mbox{HermannBahr}\pend
           \leftskip=0em{}\endnumbering\briefempfaengerindex{Schnitzler, Arthur@\textsc{Schnitzler, Arthur}!zzzBahr, Hermann@\emph{von Hermann Bahr}!1901-02-201@{20. 2. 1901}|)be}\mylabel{h}  \normalsize

\doendnotes{C}
\bigskip
\vfill

\clearpage

\footnotesize

\lohead{\textsc{register}}

% Definiere theindex-Environment komplett neu ohne reledmac
\makeatletter
\renewenvironment{theindex}{%
  \section*{\indexname}%
  \setlength{\parindent}{0pt}%
  \setlength{\parskip}{0pt plus 0.3pt}%
  \let\item\@idxitem
}{%
  \clearpage
}
\makeatother

\IfFileExists{\jobname-pw.ind}{\input{\jobname-pw.ind}}{}

\end{document}

      