%% latex-korrekturansicht-vorspann.tex
%% Vorspann für die Korrekturansicht.
%% Lädt die gemeinsame Datei latex-vorspann.tex mit gesetztem Schalter.

\newif\ifkorrekturansicht
\korrekturansichttrue

\input{../tex-inputs/latex-vorspann}


               \section[Arthur Schnitzler an Hugo von Hofmannsthal, 22. 7. 1897]{ Arthur Schnitzler an Hugo von Hofmannsthal, 22. 7. 1897}\nopagebreak\mylabel{v}\rehead{ }\normalsize\beginnumbering\briefempfaengerindex{Hofmannsthal, Hugo von@\textsc{Hofmannsthal, Hugo von}!zzzSchnitzler, Arthur@\emph{von Arthur Schnitzler}!1897-07-221@{22. 7. 1897}|(be} \toendnotes[C]{\smallbreak\pagebreak[2]} \Standort{FDH, Hs-30885,63.}
\physDesc{Briefkarte
\newline{}Handschrift: schwarze Tinte, deutsche Kurrent\newline{}Ordnung: von Schnitzler – wohl im Zuge der
                                    Durchsicht 1929 – die Jahreszahl ergänzt: »1898?« }\buchAbdrucke{\weitereDrucke{Hugo von Hofmannsthal, Arthur Schnitzler: \emph{Briefwechsel}. Hg. Therese Nickl und Heinrich Schnitzler. Frankfurt am Main: \emph{S. Fischer} 1964, S. 94.} }\toendnotes[C]{\smallbreak}\pstart
           \noindent{}{\pb}Mein lieber Hugo. Mit den Aerzten
                    ſieht’s hier ſchlecht aus; am liebſten empfehle ich Ihnen Doctor \textcolor{blue}{Herſchmann}{}\ledrightnote{\textcolor{blue}{Herschmann}}, der wohl der geſcheidteſte iſt,
                    ſelbſt einmal mit ſeiner Lunge zu thun hatte u. jetzt ganz geſund iſt. – Es tut
                    mir leid, dſs ich \textcolor{blue}{Poldy Andrian}{}\ledrightnote{\textcolor{blue}{Leopold von Andrian-Werburg}} nicht in
                    der nächſten Zeit ſehen kann; ich denke doch, dſs ihm manches {\pb}auszureden wäre. –\pend
           \pstart
           \label{K_L00710_1v}\edtext{Heute}{\lemma{\textnormal{\emph{Heute}}}\Cendnote{\textnormal{Das erlaubt die Datierung des Korrespondenzstücks, da die
                        angesprochene Aufführung am \textcolor{pink}{Saison-Theater} in \textcolor{pink}{Gmunden} am
                            22. 7. 1897 stattfand. \textcolor{blue}{Schnitzler} und \textcolor{blue}{Beer-Hofmann}
                        nahmen teil.}}}\label{K_L00710_1h} fahre ich vielleicht mit \textcolor{blue}{Richard}{}\ledrightnote{\textcolor{blue}{Richard Beer-Hofmann}} nach \textcolor{pink}{Gmund\textcolor{gray}{en}}{}\ledrightnote{\textcolor{pink}{Gmunden}}, wo \textcolor{green}{Freiwild}{}\ledrightnote{\textcolor{green}{Freiwild. Schauspiel in 3 Akten}} iſt; morgen nach \textcolor{pink}{Salzburg}{}\ledrightnote{\textcolor{pink}{Salzburg}}; übermorgen Früh beginnt die bereits angedeutete
                    Radtour. Zwei kleine \label{K_L00710_2v}\edtext{\textcolor{blue}{Schwäger}{}\ledrightnote{→\textcolor{blue}{Carl Reinhard}{\newline}→\textcolor{blue}{Franz Reinhard}}}{\lemma{\textnormal{\emph{Schwäger}}}\Cendnote{\textnormal{Die Radtour fand nicht statt. Die
                        Edition von Heinrich Schnitzler/Nickl gibt im Kommentar an,
                        dass mit dem »kleinen Schwager« des Briefes vom 21. 7. 1897 ein
                            \textcolor{blue}{Bruder} von
                            \textcolor{blue}{Marie Reinhard} gemeint sei.
                        Entsprechend könnten es sich hier um die beiden Brüder \textcolor{blue}{Karl} und \textcolor{blue}{Franz}
                        handeln. Zu der Radreise kam es aber nicht, da \textcolor{blue}{Schnitzler} nach \textcolor{pink}{Wien} zurückkehrte,
                        um ein Haus für eine versteckte Geburt des gemeinsamen Kindes mit \textcolor{blue}{Marie Reinhard} zu suchen.}}}\label{K_L00710_2h} und
                    wahrſcheinlich \textcolor{blue}{Wolzogen}{}\ledrightnote{\textcolor{blue}{Ernst von Wolzogen}} (\textcolor{green}{Lumpengeſindel}{}\ledrightnote{\textcolor{green}{Das Lumpengesindel}}) ſind mit mir.\pend
           \pstart
           Herzlichen Gruß,{\\[\baselineskip]} Ihr \spacefill\mbox{Arthur}\pend
           \leftskip=0em{}\endnumbering\briefempfaengerindex{Hofmannsthal, Hugo von@\textsc{Hofmannsthal, Hugo von}!zzzSchnitzler, Arthur@\emph{von Arthur Schnitzler}!1897-07-221@{22. 7. 1897}|)be}\mylabel{h}  \normalsize

\doendnotes{C}
\bigskip
\vfill

\clearpage

\footnotesize

\lohead{\textsc{register}}

% Definiere theindex-Environment komplett neu ohne reledmac
\makeatletter
\renewenvironment{theindex}{%
  \section*{\indexname}%
  \setlength{\parindent}{0pt}%
  \setlength{\parskip}{0pt plus 0.3pt}%
  \let\item\@idxitem
}{%
  \clearpage
}
\makeatother

\IfFileExists{\jobname-pw.ind}{\input{\jobname-pw.ind}}{}

\end{document}

      