%% latex-korrekturansicht-vorspann.tex
%% Vorspann für die Korrekturansicht.
%% Lädt die gemeinsame Datei latex-vorspann.tex mit gesetztem Schalter.

\newif\ifkorrekturansicht
\korrekturansichttrue

\input{../tex-inputs/latex-vorspann}


               \section[Arthur Schnitzler an Richard Beer-Hofmann, 18. 3. 1897]{ Arthur Schnitzler an Richard Beer-Hofmann, 18. 3. 1897}\nopagebreak\mylabel{v}\rehead{ }\normalsize\beginnumbering\briefempfaengerindex{Beer-Hofmann, Richard@\textsc{Beer-Hofmann, Richard}!zzzSchnitzler, Arthur@\emph{von Arthur Schnitzler}!1897-03-181@{18. 3. 1897}|(be} \toendnotes[C]{\smallbreak\pagebreak[2]} \Standort{YCGL, MSS 31.}
\physDesc{Postkarte
\newline{}Handschrift: Bleistift, deutsche Kurrent\newline{}Versand: 1) Rohrpost 2) Stempel: »\nobreak{}\oindex{I., Innere Stadt@\textbf{I., Innere Stadt}, \emph{Bezirk (A.BZK)}|pwk}Wien 1/1, 18 {[}3. 1897{]}, 7 30 V\nobreak{}«. 3) Stempel: »\nobreak{}\oindex{I., Innere Stadt@\textbf{I., Innere Stadt}, \emph{Bezirk (A.BZK)}|pwk}Wien 1/1, 18 III 97, 7 40 V\nobreak{}«. }\toendnotes[C]{\smallbreak}\pstart{}{\pb}Herrn Dr \textsc{Richard
                     Beer-Hofmann}\pend{}\pstart{}\textcolor{pink}{Wien}{}\ledrightnote{\textcolor{pink}{Wien}}\pend{}\pstart{}\textsc{\textcolor{pink}{I. Wollzeile 15}{}\ledrightnote{\textcolor{pink}{Wollzeile}}.}\pend{}{\bigskip}\pstart
           \noindent{}{\pb}\textcolor{pink}{Raimundtheater}{}\ledrightnote{\textcolor{pink}{Raimund-Theater}}!\pend
           \pstart
           Vergeſſen Sie nicht!{\\}2 \label{K_L00654_1v}\edtext{\textcolor{green}{Sitze}{}\ledrightnote{→\textcolor{green}{Die Sklavin. Schauspiel in vier Aufzügen}}}{\lemma{\textnormal{\emph{Sitze}}}\Cendnote{\textnormal{\textcolor{blue}{Schnitzler} besuchte die Premiere von \emph{\textcolor{green}{Die Sklavin}}. (\emph{Cambridge
                        University Library}, A 179a)}}}\label{K_L00654_1h}! Mir ſchicken!\pend
           \pstart
           \label{T_L00654_1v}\edtext{\label{K_L00654_2v}\edtext{Von mir keine Grüße}{\lemma{\textnormal{\emph{Von mir keine Grüße}}}\Cendnote{\textnormal{In der Handschrift von 
                  \textcolor{blue}{Beer-Hofmann} steht mit Bleistift in lateinischer Kurrentschrift auf der
                  Karte geschrieben:
                     »Herzliche Grüße von Richard«. Die Reaktion \textcolor{blue}{Schnitzler}s bezieht
                  sich darauf, wobei zwei Abläufe denkbar sind: Der Gruß befand sich auf der Karte, als \textcolor{blue}{Schnitzler}
                  beschloss, sie wiederzuverwenden. Oder \textcolor{blue}{Beer-Hofmann} ergänzte den Gruß, als er die
                  gewünschten Theaterkarten zusammen mit dieser Karte retournierte, woraufhin \textcolor{blue}{Schnitzler}
                  seine Reaktion notierte und erneut zurücksandte.}}}\label{K_L00654_2h}}{\lemma{\textnormal{\emph{Von mir keine Grüße}}}\Cendnote{\textnormal{am oberen Rand auf dem Kopf}}}\label{T_L00654_1h}\spacefill\mbox{Arth}\pend
           \endnumbering\briefempfaengerindex{Beer-Hofmann, Richard@\textsc{Beer-Hofmann, Richard}!zzzSchnitzler, Arthur@\emph{von Arthur Schnitzler}!1897-03-181@{18. 3. 1897}|)be}\mylabel{h}  \normalsize

\doendnotes{C}
\bigskip
\vfill

\clearpage

\footnotesize

\lohead{\textsc{register}}

% Definiere theindex-Environment komplett neu ohne reledmac
\makeatletter
\renewenvironment{theindex}{%
  \section*{\indexname}%
  \setlength{\parindent}{0pt}%
  \setlength{\parskip}{0pt plus 0.3pt}%
  \let\item\@idxitem
}{%
  \clearpage
}
\makeatother

\IfFileExists{\jobname-pw.ind}{\input{\jobname-pw.ind}}{}

\end{document}

      