%% latex-korrekturansicht-vorspann.tex
%% Vorspann für die Korrekturansicht.
%% Lädt die gemeinsame Datei latex-vorspann.tex mit gesetztem Schalter.

\newif\ifkorrekturansicht
\korrekturansichttrue

\input{../tex-inputs/latex-vorspann}


               \section[Hugo von Hofmannsthal an Arthur Schnitzler, 2. 3. 1903]{ Hugo von Hofmannsthal an Arthur Schnitzler, 2. 3. 1903}\nopagebreak\mylabel{v}\rehead{ }\normalsize\beginnumbering\briefempfaengerindex{Schnitzler, Arthur@\textsc{Schnitzler, Arthur}!zzzHofmannsthal, Hugo von@\emph{von Hugo von Hofmannsthal}!1903-03-021@{2. 3. 1903}|(be} \toendnotes[C]{\smallbreak\pagebreak[2]} \Standort{CUL, Schnitzler, B 43.}
\physDesc{Bildpostkarte
\newline{}Handschrift: schwarze Tinte, lateinische Kurrent\newline{}Versand: 1) Stempel: »\nobreak{}\oindex{Garmisch-Partenkirchen@\textbf{Garmisch-Partenkirchen}, \emph{Besiedelter Ort (A.BSO)}|pwk}Breslau, 2. 3. 03, 1–2V\nobreak{}«.  2) Stempel: »\nobreak{}\oindex{Berlin@\textbf{Berlin}, \emph{https://www.geonames.org/ontologyP.PPLC}|pwk}Berlin, 3. 3. 03, Bestellt vom Postamte 6\nobreak{}«. 
\newline{}Schnitzler: mit Bleistift die Jahreszahl ergänzt: »902« \newline{}Ordnung: 1) mit Bleistift von unbekannter Hand nummeriert: »\strikeout{225}« 2) mit Bleistift von unbekannter Hand nummeriert: »195«}\buchAbdrucke{\weitereDrucke{Hugo von Hofmannsthal, Arthur Schnitzler: \emph{Briefwechsel}. Hg. Therese Nickl und Heinrich Schnitzler. Frankfurt am Main: \emph{S. Fischer} 1964, S. 168.} }\toendnotes[C]{\smallbreak}\pstart{}{\pb}Herrn D\textsuperscript{r} Arthur Schnitzler\pend{}\pstart{}\textcolor{pink}{Berlin}{}\ledrightnote{\textcolor{pink}{Berlin}}\pend{}\pstart{}\textcolor{pink}{Schuhmannstrasse}{}\ledrightnote{\textcolor{pink}{Schumannstraße}}, \textcolor{pink}{Deutsches Theater}{}\ledrightnote{\textcolor{pink}{Deutsches Theater Berlin}}\pend{}{\bigskip}\pstart
           \noindent{}\centering{}\textcolor{gray}{\textbf{{\pb}\textcolor{pink}{Breslau}{}\ledrightnote{\textcolor{pink}{Breslau}}{ }\textcolor{pink}{Südpark-Restaurant}{}\ledrightnote{\textcolor{pink}{Südpark-Restaurant}}.}}\pend
           \pstart
           Gruss an Sie, \textcolor{blue}{Brahm}{}\ledrightnote{\textcolor{blue}{Otto Brahm}} und den wackern alten \label{K_L01274_1v}\edtext{\textcolor{green}{Chiaveluzzi}{}\ledrightnote{→\textcolor{green}{Der Schleier der Beatrice. Schauspiel in fünf Akten}}}{\lemma{\textnormal{\emph{Chiaveluzzi}}}\Cendnote{\textnormal{Figur aus \emph{\textcolor{green}{Der Schleier der Beatrice}}; bei der \textcolor{pink}{Berlin}er Inszenierung wurde er von \textcolor{blue}{Adolf
                     Kurth} gespielt.}}}\label{K_L01274_1h}.\pend
           \pstart
           2 März\pend
           \endnumbering\briefempfaengerindex{Schnitzler, Arthur@\textsc{Schnitzler, Arthur}!zzzHofmannsthal, Hugo von@\emph{von Hugo von Hofmannsthal}!1903-03-021@{2. 3. 1903}|)be}\mylabel{h}  \normalsize

\doendnotes{C}
\bigskip
\vfill

\clearpage

\footnotesize

\lohead{\textsc{register}}

% Definiere theindex-Environment komplett neu ohne reledmac
\makeatletter
\renewenvironment{theindex}{%
  \section*{\indexname}%
  \setlength{\parindent}{0pt}%
  \setlength{\parskip}{0pt plus 0.3pt}%
  \let\item\@idxitem
}{%
  \clearpage
}
\makeatother

\IfFileExists{\jobname-pw.ind}{\input{\jobname-pw.ind}}{}

\end{document}

      