%% latex-korrekturansicht-vorspann.tex
%% Vorspann für die Korrekturansicht.
%% Lädt die gemeinsame Datei latex-vorspann.tex mit gesetztem Schalter.

\newif\ifkorrekturansicht
\korrekturansichttrue

\input{../tex-inputs/latex-vorspann}


               \section[Arthur Schnitzler an Richard Beer-Hofmann, {[}3. 12. 1908?{]}]{ Arthur Schnitzler an Richard Beer-Hofmann, {[}3. 12. 1908?{]}}\nopagebreak\mylabel{v}\rehead{ }\normalsize\beginnumbering\briefempfaengerindex{Beer-Hofmann, Richard@\textsc{Beer-Hofmann, Richard}!zzzSchnitzler, Arthur@\emph{von Arthur Schnitzler}!1908-12-031@{{[}3. 12. 1908?{]}}|(be} \toendnotes[C]{\smallbreak\pagebreak[2]} \Standort{YCGL, MSS 31.}
\physDesc{Brief, 1 Blatt, 2 Seiten, Umschlag
\newline{}Handschrift: Bleistift, deutsche Kurrent\newline{}Versand: ohne postalischen Übermittlungsvermerk }\toendnotes[C]{\smallbreak}\pstart{}{\pb}\textcolor{gray}{\textbf{Dr. Arthur Schnitzler}}\pend{}\pstart{}\textcolor{gray}{\textbf{\textcolor{pink}{Wien XVIII. Spoettelgasse 7}{}\ledrightnote{\textcolor{pink}{Edmund-Weiß-Gasse}}.}}\pend{}{\bigskip}\pstart{}{\pb}\textsc{Herrn Dr Richard Beer-Hofmann}\pend{}\pstart{}\textcolor{pink}{Wien XIX}{}\ledrightnote{\textcolor{pink}{XIX., Döbling}}\pend{}{\bigskip}\pstart
           \noindent{}{\pb}\textcolor{gray}{\textbf{Dr. Arthur Schnitzler}}{\\}\textcolor{gray}{\textbf{\textcolor{pink}{Wien XVIII. Spoettelgasse 7}{}\ledrightnote{\textcolor{pink}{Edmund-Weiß-Gasse}}.}}\pend
           \pstart{}lieber Richard\pend\pstart
           wir fahren \label{K_L01818_1v}\edtext{morgen Freitag}{\lemma{\textnormal{\emph{morgen Freitag}}}\Cendnote{\textnormal{Das Korrespondenzstück ist undatiert.
                  Unter der Annahme, dass die hier geplante \textcolor{pink}{Semmering}-Reise auf die hier besprochene Weise stattfand, trifft das für
                  den Aufenthalt von 4. 12. 1908 bis 6. 12. 1908 zu.}}}\label{K_L01818_1h} (11.25) auf den \textcolor{pink}{Semmering}{}\ledrightnote{\textcolor{pink}{Semmering}} (bis So{\geminationn}tag{[}){]}. We{\geminationn} Sie und \textcolor{blue}{Paula}{}\ledrightnote{\textcolor{blue}{Paula Beer-Hofmann}} mitfahren wollten würd es uns beſonders {\pb}freuen. Ja wir würden Sie ſogar mit dem Auto um
                  ¾ 11 abholen.\pend
           \pstart
           Der ich wohl zu leben wünſche,\pend
           \pstart herzlichſt\spacefill\mbox{A.}\pend{}\endnumbering\briefempfaengerindex{Beer-Hofmann, Richard@\textsc{Beer-Hofmann, Richard}!zzzSchnitzler, Arthur@\emph{von Arthur Schnitzler}!1908-12-031@{{[}3. 12. 1908?{]}}|)be}\mylabel{h}  \normalsize

\doendnotes{C}
\bigskip
\vfill

\clearpage

\footnotesize

\lohead{\textsc{register}}

% Definiere theindex-Environment komplett neu ohne reledmac
\makeatletter
\renewenvironment{theindex}{%
  \section*{\indexname}%
  \setlength{\parindent}{0pt}%
  \setlength{\parskip}{0pt plus 0.3pt}%
  \let\item\@idxitem
}{%
  \clearpage
}
\makeatother

\IfFileExists{\jobname-pw.ind}{\input{\jobname-pw.ind}}{}

\end{document}

      