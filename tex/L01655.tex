%% latex-korrekturansicht-vorspann.tex
%% Vorspann für die Korrekturansicht.
%% Lädt die gemeinsame Datei latex-vorspann.tex mit gesetztem Schalter.

\newif\ifkorrekturansicht
\korrekturansichttrue

\input{../tex-inputs/latex-vorspann}


               \section[Hugo von Hofmannsthal an Arthur Schnitzler, {[}zwischen 3.–7. 2. 1907{]}]{ Hugo von Hofmannsthal an Arthur Schnitzler, {[}zwischen
               3.–7. 2. 1907{]}}\nopagebreak\mylabel{v}\rehead{ }\normalsize\beginnumbering\briefempfaengerindex{Schnitzler, Arthur@\textsc{Schnitzler, Arthur}!zzzHofmannsthal, Hugo von@\emph{von Hugo von Hofmannsthal}!1907-02-031@{{[}zwischen
                  3.–7. 2. 1907{]}}|(be} \toendnotes[C]{\smallbreak\pagebreak[2]} \Standort{CUL, Schnitzler, B 43.}
\physDesc{Brief, 1 Blatt, 1 Seite, Fragment
\newline{}Handschrift: schwarze Tinte, deutsche Kurrent\newline{}Ordnung: 1) mit Bleistift von unbekannter Hand nummeriert: »\strikeout{269}« 2) mit Bleistift von unbekannter Hand nummeriert:
                                    »269« und beschriftet: »lacking
                                    Sheet 1?«}\buchAbdrucke{\weitereDrucke{Hugo von Hofmannsthal, Arthur Schnitzler: \emph{Briefwechsel}. Hg. Therese Nickl und Heinrich Schnitzler. Frankfurt am Main: \emph{S. Fischer} 1964, S. 227.} }\toendnotes[C]{\smallbreak}\pstart{}{\pb}lieber, \pend\pstart
           man ſieht ſich \uline{nie}. Momentan ſind wieder \textcolor{blue}{Gerty}{}\ledrightnote{\textcolor{blue}{Gertrude von Hofmannsthal}} und ich nicht recht wohl, können nicht in die
               Stadt. Ich habe \label{K_L01655_1v}\edtext{böſes Aug, ſchlechten
               Hals, wehen Fuß}{\lemma{\textnormal{\emph{böſes … Fuß}}}\Cendnote{\textnormal{Offensichtlich um diese
                  auszuheilen, reist \textcolor{blue}{Hofmannsthal} am
                     12. 2. 1907 ins \textcolor{pink}{Südbahnhotel} am
                     \textcolor{pink}{Semmering}, während seine \textcolor{blue}{Frau} zuhause bleibt. Das Schreiben kann demnach nur mit
                  nötigem Abstand zum einzig verbleibenden Wochenende im Februar 1907
                  davor entstanden sein.}}}\label{K_L01655_1h}. Kann nicht ſingen, nicht ſtehen, nicht ſchauen.
               Wünſche mir ſehr Geſellſchaft. Seid doch einmal im Leben nett (zum Unterſchied von
               dem † † † \textcolor{blue}{Bärenviehzeug}{}\ledrightnote{→\textcolor{blue}{Richard Beer-Hofmann}{\newline}→\textcolor{blue}{Paula Beer-Hofmann}}). Es iſt jetzt ſo hübſch hier, Schnee und hübſch und dabei mild,
               alſo {\pb}kommt einmal her, oder Samstag oder Sonntag; oder zum Eſſen, oder
               zum Nachmittag oder zum Nachtmahl oder alles zugleich.\pend
           \pstart
           Depeſchiert ſchön gleich Eure werte Antwort.\pend
           \pstart
           Euer unvergleichlicher und ergebenſter Diener{\\[\baselineskip]}\spacefill\mbox{Hugo}\pend
           \leftskip=0em{}\endnumbering\briefempfaengerindex{Schnitzler, Arthur@\textsc{Schnitzler, Arthur}!zzzHofmannsthal, Hugo von@\emph{von Hugo von Hofmannsthal}!1907-02-031@{{[}zwischen
                  3.–7. 2. 1907{]}}|)be}\mylabel{h}  \normalsize

\doendnotes{C}
\bigskip
\vfill

\clearpage

\footnotesize

\lohead{\textsc{register}}

% Definiere theindex-Environment komplett neu ohne reledmac
\makeatletter
\renewenvironment{theindex}{%
  \section*{\indexname}%
  \setlength{\parindent}{0pt}%
  \setlength{\parskip}{0pt plus 0.3pt}%
  \let\item\@idxitem
}{%
  \clearpage
}
\makeatother

\IfFileExists{\jobname-pw.ind}{\input{\jobname-pw.ind}}{}

\end{document}

      