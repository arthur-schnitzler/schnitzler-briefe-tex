%% latex-korrekturansicht-vorspann.tex
%% Vorspann für die Korrekturansicht.
%% Lädt die gemeinsame Datei latex-vorspann.tex mit gesetztem Schalter.

\newif\ifkorrekturansicht
\korrekturansichttrue

\input{../tex-inputs/latex-vorspann}


               \section[Hugo von Hofmannsthal an Arthur Schnitzler, 13. 3. 1896]{ Hugo von Hofmannsthal an Arthur Schnitzler, 13. 3. 1896}\nopagebreak\mylabel{v}\rehead{ }\normalsize\beginnumbering\briefempfaengerindex{Schnitzler, Arthur@\textsc{Schnitzler, Arthur}!zzzHofmannsthal, Hugo von@\emph{von Hugo von Hofmannsthal}!1896-03-131@{13. 3. 1896}|(be} \toendnotes[C]{\smallbreak\pagebreak[2]} \Standort{DLA, A:Schnitzler, 76.740.}
\physDesc{Brief, 1 Blatt, 1 Seite
\newline{}Handschrift: schwarze Tinte, deutsche Kurrent}\buchAbdrucke{\weitereDrucke{1) Hugo von Hofmannsthal, Arthur Schnitzler: \emph{Briefwechsel}. Hg. Therese Nickl und Heinrich Schnitzler. Frankfurt am Main: \emph{S. Fischer} 1964, S. 65.} \weitereDrucke{2) Hans-Ulrich Lindken: \emph{Arthur Schnitzler. Aspekte und Akzente. Materialien zu
                            Leben und Werk}. Frankfurt am Main, Bern, Göttingen: \emph{Peter Lang} 1984, S. 173 (Europäische Hochschulschriften, Reihe 1, Deutsche Sprache
                            und Literatur, 754).} }\toendnotes[C]{\smallbreak}\pstart
           \noindent{}{\pb}Freitag 13. 3. 96.
                        \hfill \textcolor{gray}{\textbf{\textcolor{pink}{III Salesianergasse 12}{}\ledrightnote{\textcolor{pink}{Salesianergasse}}}}\pend
           \pstart{}lieber Arthur\pend\pstart
           Wir ſehen uns zu ſelten. Könnte ich nicht \label{K_L00538_1v}\edtext{Sonntag}{\lemma{\textnormal{\emph{Sonntag}}}\Cendnote{\textnormal{vgl. A. S.: \emph{Tagebuch}, 15. 3. 1896}}}\label{K_L00538_1h} nach
                        7 zu Ihnen kommen. ich denke wir gehen dann zusammen um
                        ½ 10 Uhr in dieſe Geſellſchaft. Oder ſtört es Sie beim
                    Anziehen, wenn ich dabei bin?\pend
           \pstart
           Bitte Antwort. Herzlich Ihr{\\[\baselineskip]}\spacefill\mbox{Hugo.}\pend
           \leftskip=0em{}\endnumbering\briefempfaengerindex{Schnitzler, Arthur@\textsc{Schnitzler, Arthur}!zzzHofmannsthal, Hugo von@\emph{von Hugo von Hofmannsthal}!1896-03-131@{13. 3. 1896}|)be}\mylabel{h}  \normalsize

\doendnotes{C}
\bigskip
\vfill

\clearpage

\footnotesize

\lohead{\textsc{register}}

% Definiere theindex-Environment komplett neu ohne reledmac
\makeatletter
\renewenvironment{theindex}{%
  \section*{\indexname}%
  \setlength{\parindent}{0pt}%
  \setlength{\parskip}{0pt plus 0.3pt}%
  \let\item\@idxitem
}{%
  \clearpage
}
\makeatother

\IfFileExists{\jobname-pw.ind}{\input{\jobname-pw.ind}}{}

\end{document}

      