%% latex-korrekturansicht-vorspann.tex
%% Vorspann für die Korrekturansicht.
%% Lädt die gemeinsame Datei latex-vorspann.tex mit gesetztem Schalter.

\newif\ifkorrekturansicht
\korrekturansichttrue

\input{../tex-inputs/latex-vorspann}


               \section[Arthur Schnitzler an Albert Ehrenstein, 20. 1. 1909]{ Arthur Schnitzler an Albert Ehrenstein, 20. 1. 1909}\nopagebreak\mylabel{v}\rehead{ }\normalsize\beginnumbering\briefempfaengerindex{Ehrenstein, Albert@\textsc{Ehrenstein, Albert}!zzzSchnitzler, Arthur@\emph{von Arthur Schnitzler}!1909-01-201@{20. 1. 1909}|(be} \toendnotes[C]{\smallbreak\pagebreak[2]} \Standort{Jerusalem, The National Library of Israel, ARC. Ms. Var. 306 1 118.}
\physDesc{Brief, 1 Blatt, 1 Seite
\newline{}Schreibmaschine
\newline{}Handschrift: schwarze Tinte (\noindent{}Unterschrift)}\Standort{DLA, A:Schnitzler, 85.1.642,1.}
\physDesc{Brief, maschineller Durchschlag
\newline{}Schreibmaschine
\newline{}Handschrift: roter Buntstift, lateinische Kurrent (\noindent{}»Ehrenstein«)}\toendnotes[C]{\smallbreak}\pstart
           \noindent{}{\pb}\textcolor{gray}{\textbf{Dr. Arthur Schnitzler}}{\\}\textcolor{gray}{\textbf{\textcolor{pink}{Wien XVIII. Spoettelgasse 7}{}\ledrightnote{\textcolor{pink}{Edmund-Weiß-Gasse}}.}}\pend
           \pstart
           \raggedleft{}20. Jänner 09.\pend
           \pstart{}Lieber Herr Ehrenstein,\pend\pstart
           Mit dem von Ihnen genannten \textcolor{brown}{Blatt}{}\ledrightnote{→\textcolor{brown}{Österreichische Rundschau}} stehe ich aus Gründen, deren Erörterung zu weit führen würde in
                    keinerlei Verbindung mehr. Ich erlaube mir also Ihnen das \textcolor{green}{Manuskript}{}\ledrightnote{→\textcolor{green}{Tai-Gin}} zurückzusenden, das übrigens
                    meiner persönlichen Empfindung nach nicht zu Ihren gelungensten Arbeiten
                    gehört.\pend
           \pstart
           Mit verbindlichem Gruss{\\[\baselineskip]} Ihr{\\[\baselineskip]}\spacefill\mbox{{[}hs.:{]} Arthur Schnitzler}\pend
           \leftskip=0em{}\pstart
           \noindent{}{[}ms.:{]} Herrn Albert Ehrenstein.\pend
           \endnumbering\briefempfaengerindex{Ehrenstein, Albert@\textsc{Ehrenstein, Albert}!zzzSchnitzler, Arthur@\emph{von Arthur Schnitzler}!1909-01-201@{20. 1. 1909}|)be}\mylabel{h}  \normalsize

\doendnotes{C}
\bigskip
\vfill

\clearpage

\footnotesize

\lohead{\textsc{register}}

% Definiere theindex-Environment komplett neu ohne reledmac
\makeatletter
\renewenvironment{theindex}{%
  \section*{\indexname}%
  \setlength{\parindent}{0pt}%
  \setlength{\parskip}{0pt plus 0.3pt}%
  \let\item\@idxitem
}{%
  \clearpage
}
\makeatother

\IfFileExists{\jobname-pw.ind}{\input{\jobname-pw.ind}}{}

\end{document}

      