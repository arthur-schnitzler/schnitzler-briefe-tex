%% latex-korrekturansicht-vorspann.tex
%% Vorspann für die Korrekturansicht.
%% Lädt die gemeinsame Datei latex-vorspann.tex mit gesetztem Schalter.

\newif\ifkorrekturansicht
\korrekturansichttrue

\input{../tex-inputs/latex-vorspann}


               \section[Max Burckhard an Arthur Schnitzler, {[}23. 6. 1897{]}]{ Max Burckhard an Arthur Schnitzler, {[}23. 6. 1897{]}}\nopagebreak\mylabel{v}\rehead{ }\normalsize\beginnumbering\briefempfaengerindex{Schnitzler, Arthur@\textsc{Schnitzler, Arthur}!zzzBurckhard, Max Eugen@\emph{von Max Eugen Burckhard}!1897-06-232@{{[}23. 6. 1897{]}}|(be} \toendnotes[C]{\smallbreak\pagebreak[2]} \Standort{CUL, Schnitzler, B 20.}
\physDesc{Brief, 1 Blatt, 1 Seite
\newline{}Handschrift: schwarze Tinte, deutsche Kurrent
\newline{}Schnitzler: mit Bleistift datiert: »23/6 97« \newline{}Ordnung: mit Bleistift von unbekannter Hand nummeriert:
                                        »9« }\toendnotes[C]{\smallbreak}\pstart{}{\pb}Sehr verehrter lieber Herr
                        Doctor!\pend\pstart
           Ich bin ſo frei im Auftrage des Autors beiliegendes \label{K_L00691_1v}\edtext{\textcolor{green}{Stück}{}\ledrightnote{→\textcolor{green}{’s Katherl. Volksstück in fünf Aufzügen}{\newline}→\textcolor{green}{Die Bürgermeisterwahl. Ländliche Comödie in vier Acten}}}{\lemma{\textnormal{\emph{Stück}}}\Cendnote{\textnormal{Es könnte sich um \emph{\textcolor{green}{Die Bürgermeisterwahl}} oder \emph{\textcolor{green}{s’Katherl}} von \textcolor{blue}{Max Burckhard}
                        handeln, die am 20. respektive 25. 11. 1897 in \textcolor{pink}{Wien} ihre Uraufführung erlebten.}}}\label{K_L00691_1h} zu
                    überſenden.\pend
           \pstart
           In herzlicher Verehrung{\\[\baselineskip]}\spacefill\mbox{D\textsuperscript{r}Burckhard}\pend
           \leftskip=0em{}\endnumbering\briefempfaengerindex{Schnitzler, Arthur@\textsc{Schnitzler, Arthur}!zzzBurckhard, Max Eugen@\emph{von Max Eugen Burckhard}!1897-06-232@{{[}23. 6. 1897{]}}|)be}\mylabel{h}  \normalsize

\doendnotes{C}
\bigskip
\vfill

\clearpage

\footnotesize

\lohead{\textsc{register}}

% Definiere theindex-Environment komplett neu ohne reledmac
\makeatletter
\renewenvironment{theindex}{%
  \section*{\indexname}%
  \setlength{\parindent}{0pt}%
  \setlength{\parskip}{0pt plus 0.3pt}%
  \let\item\@idxitem
}{%
  \clearpage
}
\makeatother

\IfFileExists{\jobname-pw.ind}{\input{\jobname-pw.ind}}{}

\end{document}

      