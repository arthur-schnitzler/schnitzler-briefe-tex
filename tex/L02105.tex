%% latex-korrekturansicht-vorspann.tex
%% Vorspann für die Korrekturansicht.
%% Lädt die gemeinsame Datei latex-vorspann.tex mit gesetztem Schalter.

\newif\ifkorrekturansicht
\korrekturansichttrue

\input{../tex-inputs/latex-vorspann}


               \section[Hugo von Hofmannsthal an Arthur Schnitzler, {[}29. 11. 1912{]}]{ Hugo von Hofmannsthal an Arthur Schnitzler, {[}29. 11. 1912{]}}\nopagebreak\mylabel{v}\rehead{ }\normalsize\beginnumbering\briefempfaengerindex{Schnitzler, Arthur@\textsc{Schnitzler, Arthur}!zzzHofmannsthal, Hugo von@\emph{von Hugo von Hofmannsthal}!1912-11-292@{{[}29. 11. 1912{]}}|(be} \toendnotes[C]{\smallbreak\pagebreak[2]} \Standort{CUL, Schnitzler, B 43.}
\physDesc{Telegramm
\newline{}maschinell\newline{}Versand: mit schwarzer Tinte auf der Rückseite der
                           postalische Vermerk des Telegrammboten: »\noindent{}{\pb}Adr.
                                    wohn{[}t nicht{]}{ }\textsc{\textcolor{pink}{Esplanade}}, nach Aussage des Poſt-Chefs ſoll Adr. im \textsc{\textcolor{pink}{Hotel Adlon}} wohnen?{ / }\textcolor{blue}{Geier}{ }11/9.« 
\newline{}Schnitzler: mit Bleistift datiert: »29/11 912« \newline{}Ordnung: 1) beschnitten 2) mit Bleistift von unbekannter Hand nummeriert: »241«}\buchAbdrucke{\weitereDrucke{Hugo von Hofmannsthal, Arthur Schnitzler: \emph{Briefwechsel}. Hg. Therese Nickl und Heinrich Schnitzler. Frankfurt am Main: \emph{S. Fischer} 1964, S. 270.} }\pstart
           \noindent{}{\pb}tieftraurig um guten lieben nie
               wieder zufindenden \textcolor{blue}{brahm}{}\ledrightnote{\textcolor{blue}{Otto Brahm}} bitte ihm auch fuer mich
               blumen bringen von herzen ihr \spacefill\mbox{hugo +}\pend
           \endnumbering\briefempfaengerindex{Schnitzler, Arthur@\textsc{Schnitzler, Arthur}!zzzHofmannsthal, Hugo von@\emph{von Hugo von Hofmannsthal}!1912-11-292@{{[}29. 11. 1912{]}}|)be}\mylabel{h}  \normalsize

\doendnotes{C}
\bigskip
\vfill

\clearpage

\footnotesize

\lohead{\textsc{register}}

% Definiere theindex-Environment komplett neu ohne reledmac
\makeatletter
\renewenvironment{theindex}{%
  \section*{\indexname}%
  \setlength{\parindent}{0pt}%
  \setlength{\parskip}{0pt plus 0.3pt}%
  \let\item\@idxitem
}{%
  \clearpage
}
\makeatother

\IfFileExists{\jobname-pw.ind}{\input{\jobname-pw.ind}}{}

\end{document}

      