%% latex-korrekturansicht-vorspann.tex
%% Vorspann für die Korrekturansicht.
%% Lädt die gemeinsame Datei latex-vorspann.tex mit gesetztem Schalter.

\newif\ifkorrekturansicht
\korrekturansichttrue

\input{../tex-inputs/latex-vorspann}


               \section[Arthur Schnitzler an Richard Beer-Hofmann, 18. 2. 1908]{ Arthur Schnitzler an Richard Beer-Hofmann, 18. 2. 1908}\nopagebreak\mylabel{v}\rehead{ }\normalsize\beginnumbering\briefempfaengerindex{Beer-Hofmann, Richard@\textsc{Beer-Hofmann, Richard}!zzzSchnitzler, Arthur@\emph{von Arthur Schnitzler}!1908-02-181@{18. 2. 1908}|(be} \toendnotes[C]{\smallbreak\pagebreak[2]} \Standort{YCGL, MSS 31.}
\physDesc{Bildpostkarte
\newline{}Handschrift: Bleistift, deutsche Kurrent\newline{}Versand: Stempel: »\nobreak{}\oindex{Semmering@\textbf{Semmering}, \emph{Besiedelter Ort (A.BSO)}|pwk}Semmering, 18. II. \textcolor{gray}{08}, 3\nobreak{}«.  }\buchAbdrucke{\weitereDrucke{Arthur Schnitzler, Richard Beer-Hofmann: \emph{Briefwechsel 1891–1931}. Hg. Konstanze Fliedl. Wien, Zürich: \emph{Europaverlag} 1992, S. 189.} }\toendnotes[C]{\smallbreak}\pstart{}{\pb}\textsc{Dr Richard Beer-Hofma{\geminationn}}\pend{}\pstart{}\textcolor{pink}{\textsc{Wien XVIII}}{}\ledrightnote{\textcolor{pink}{XVIII., Währing}}\pend{}\pstart{}\textcolor{pink}{\textsc{Hasenauerstr 59}}{}\ledrightnote{\textcolor{pink}{Hasenauerstraße}}.\pend{}{\bigskip}\pstart
           \noindent{}\centering{}\textcolor{gray}{\textbf{{\pb}\textcolor{pink}{Semmering}{}\ledrightnote{\textcolor{pink}{Semmering}}. \textcolor{pink}{Südbahnhotel}{}\ledrightnote{\textcolor{pink}{Südbahnhotel}}.}}\pend
           \pstart
           \raggedleft{}{\pb}18. 2.\pend
           \pstart
           lieber Richard, wir wollen \label{K_L01761_1v}\edtext{Donnerſtag}{\lemma{\textnormal{\emph{Donnerſtag}}}\Cendnote{\textnormal{Es verzögerte sich. Vgl. A. S.: \emph{Tagebuch}, 22. 2. 1908}}}\label{K_L01761_1h} in \textcolor{pink}{Wien}{}\ledrightnote{\textcolor{pink}{Wien}} ſein. Schade dſs Sie nicht
                  heraufgeko{\geminationm}en ſind. Ich bitte Sie dringend, leſen Sie
               nun nicht am Ende den »\textcolor{green}{Weg}{}\ledrightnote{\textcolor{green}{Der Weg ins Freie. Roman}}« in \label{K_L01761_2v}\edtext{\textcolor{green}{Fortſetzungen}{}\ledrightnote{→\textcolor{green}{Die neue Rundschau}}}{\lemma{\textnormal{\emph{Fortſetzungen}}}\Cendnote{\textnormal{Der Roman erschien in sechs Teilen von
                     Januar bis Juni 1908 in der \emph{\textcolor{green}{Neuen Rundschau}}.}}}\label{K_L01761_2h} weiter, ſondern warten auf das
               Buch.\pend
           \pstart
           Herzlichſt{\\[\baselineskip]}Ihr{\\[\baselineskip]}\spacefill\mbox{A.}\pend
           \leftskip=0em{}\endnumbering\briefempfaengerindex{Beer-Hofmann, Richard@\textsc{Beer-Hofmann, Richard}!zzzSchnitzler, Arthur@\emph{von Arthur Schnitzler}!1908-02-181@{18. 2. 1908}|)be}\mylabel{h}  \normalsize

\doendnotes{C}
\bigskip
\vfill

\clearpage

\footnotesize

\lohead{\textsc{register}}

% Definiere theindex-Environment komplett neu ohne reledmac
\makeatletter
\renewenvironment{theindex}{%
  \section*{\indexname}%
  \setlength{\parindent}{0pt}%
  \setlength{\parskip}{0pt plus 0.3pt}%
  \let\item\@idxitem
}{%
  \clearpage
}
\makeatother

\IfFileExists{\jobname-pw.ind}{\input{\jobname-pw.ind}}{}

\end{document}

      