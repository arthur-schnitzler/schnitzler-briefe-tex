%% latex-korrekturansicht-vorspann.tex
%% Vorspann für die Korrekturansicht.
%% Lädt die gemeinsame Datei latex-vorspann.tex mit gesetztem Schalter.

\newif\ifkorrekturansicht
\korrekturansichttrue

\input{../tex-inputs/latex-vorspann}


               \section[Hugo von Hofmannsthal an Arthur Schnitzler, 27. 9. 1904]{ Hugo von Hofmannsthal an Arthur Schnitzler, 27. 9. 1904}\nopagebreak\mylabel{v}\rehead{ }\normalsize\beginnumbering\briefempfaengerindex{Schnitzler, Arthur@\textsc{Schnitzler, Arthur}!zzzHofmannsthal, Hugo von@\emph{von Hugo von Hofmannsthal}!1904-09-271@{27. 9. 1904}|(be} \toendnotes[C]{\smallbreak\pagebreak[2]} \Standort{CUL, Schnitzler, B 43.}
\physDesc{Postkarte
\newline{}Handschrift: schwarze Tinte, deutsche Kurrent\newline{}Versand: 1) Stempel: »\nobreak{}\oindex{X., Favoriten@\textbf{X., Favoriten}, \emph{Bezirk (A.BZK)}|pwk}10/2 Wien 78, 27 IX 04, 10\nobreak{}«.  2) Stempel: »\nobreak{}\oindex{XVIII., Waehring@\textbf{XVIII., Währing}, \emph{Bezirk (A.BZK)}|pwk}18/1 Wien 110, 28. 9. 04, 8.V, Bestellt\nobreak{}«. \newline{}Ordnung: 1) mit Bleistift von unbekannter Hand nummeriert: »\strikeout{225}« 2) mit Bleistift von unbekannter Hand  nummeriert: »256«}\buchAbdrucke{\weitereDrucke{Hugo von Hofmannsthal, Arthur Schnitzler: \emph{Briefwechsel}. Hg. Therese Nickl und Heinrich Schnitzler. Frankfurt am Main: \emph{S. Fischer} 1964, S. 203.} }\pstart{}{\pb}\textsc{Herrn D\textsuperscript{r} Arthur Schnitzler}\pend{}\pstart{}\textcolor{pink}{\textsc{Wien}}{}\ledrightnote{\textcolor{pink}{Wien}}\pend{}\pstart{}\textcolor{pink}{\textsc{XVIII Spöttelgasse 7}}{}\ledrightnote{\textcolor{pink}{Edmund-Weiß-Gasse}}\pend{}{\bigskip}\pstart
           \noindent{}{\pb}Lieber, erbitte
               1.) wenn Sie es haben, per Poſt: \textsc{\textcolor{blue}{Kierkegaard}{}\ledrightnote{\textcolor{blue}{Søren Aabye Kierkegaard}}{ }\textcolor{green}{Entweder – oder}{}\ledrightnote{\textcolor{green}{Entweder – Oder}}}\pend
           \pstart
           2.) gelegentlich, wenn Sie es nicht mehr brauchen, mein gebundenes »\textsc{\textcolor{green}{Kunst und Künstler}{}\ledrightnote{\textcolor{green}{Kunst und Künstler}}}{[}«{]}\pend
           \pstart
           3.) ſobald Wetter erträglich, ein abendliches oder ſonſtiges \textsc{rendez vous}.\pend
           \pstart
           Herzlich{\\[\baselineskip]}\spacefill\mbox{Hugo.}\pend
           \leftskip=0em{}\pstart
           27.{\\}IX.\pend
           \endnumbering\briefempfaengerindex{Schnitzler, Arthur@\textsc{Schnitzler, Arthur}!zzzHofmannsthal, Hugo von@\emph{von Hugo von Hofmannsthal}!1904-09-271@{27. 9. 1904}|)be}\mylabel{h}  \normalsize

\doendnotes{C}
\bigskip
\vfill

\clearpage

\footnotesize

\lohead{\textsc{register}}

% Definiere theindex-Environment komplett neu ohne reledmac
\makeatletter
\renewenvironment{theindex}{%
  \section*{\indexname}%
  \setlength{\parindent}{0pt}%
  \setlength{\parskip}{0pt plus 0.3pt}%
  \let\item\@idxitem
}{%
  \clearpage
}
\makeatother

\IfFileExists{\jobname-pw.ind}{\input{\jobname-pw.ind}}{}

\end{document}

      