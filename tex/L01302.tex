%% latex-korrekturansicht-vorspann.tex
%% Vorspann für die Korrekturansicht.
%% Lädt die gemeinsame Datei latex-vorspann.tex mit gesetztem Schalter.

\newif\ifkorrekturansicht
\korrekturansichttrue

\input{../tex-inputs/latex-vorspann}


               \section[Arthur Schnitzler an Hermann Bahr, 1{[}3{]}. 7. 1903]{ Arthur Schnitzler an Hermann Bahr, 1{[}3{]}. 7. 1903}\nopagebreak\mylabel{v}\rehead{ }\normalsize\beginnumbering\briefempfaengerindex{Bahr, Hermann@\textsc{Bahr, Hermann}!zzzSchnitzler, Arthur@\emph{von Arthur Schnitzler}!1903-07-131@{1{[}3{]}. 7. 1903}|(be} \toendnotes[C]{\smallbreak\pagebreak[2]} \Standort{TMW, HS AM 60165 Ba.}
\physDesc{Briefkarte
\newline{}Handschrift: schwarze Tinte, deutsche Kurrent\newline{}Ordnung: Lochung }\buchAbdrucke{\weitereDrucke{1) \emph{13. 7. 1903, Abschrift.} In: Arthur Schnitzler: \emph{The Letters of Arthur Schnitzler to Hermann Bahr}. Edited, annotated, and with an introduction, by Donald G.
                        Daviau. Chapel Hill: \emph{The University of North Carolina Press} 1978, S. 79 (University of North Carolina studies in the Germanic languages
                        and literatures, 89).} \weitereDrucke{2) Hermann Bahr, Arthur Schnitzler: \emph{Briefwechsel, Aufzeichnungen, Dokumente (1891–1931)}. Hg. Kurt Ifkovits und Martin Anton Müller. Göttingen: \emph{Wallstein} 2018, S. 267.} }\toendnotes[C]{\smallbreak}\pstart
           {\pb}1\damage{3}. 7. 903.\pend
           \pstart
           lieber Hermann, \textcolor{blue}{Salten}{}\ledrightnote{\textcolor{blue}{Felix Salten}}{ }\damage{ü}bermittelt mir deine freundliche Frage, ob ich was dagegen hätte, we{\geminationn} du den \textcolor{green}{Reigen}{}\ledrightnote{\textcolor{green}{Reigen. Zehn Dialoge}}
               öffentlich vorzuleſen versuchteſt. Im Gegentheil, es wird {\pb}mir \damage{ſe}hr angenehm ſein. Nu\damage{r} werde ich zum erſten Mal bedauern – daſs ich der Verfaſſer bin – weil ich
               nemlich nicht als Zuhörer meiner eigenen Sachen unter dem Publikum ſitzen kann! Auf
                  Wiederſehen\hspace*{1.5em}dein getreuer{\\}\spacefill\mbox{A. S.}\pend
           \pstart
           \noindent{}\label{T_L01302-1v}\edtext{Prächtig war dein \textcolor{green}{Dialog}{}\ledrightnote{\textcolor{green}{Dialog vom Tragischen}} in der \textcolor{green}{\textsc{N. D. R}}{}\ledrightnote{\textcolor{green}{Die neue Rundschau}}! –}{\lemma{\textnormal{\emph{Prächtig … N. D. R! –}}}\Cendnote{\textnormal{auf der ersten Seite, am unteren Seitenrand, verkehrt zum Text}}}\label{T_L01302-1h}\pend
           \endnumbering\briefempfaengerindex{Bahr, Hermann@\textsc{Bahr, Hermann}!zzzSchnitzler, Arthur@\emph{von Arthur Schnitzler}!1903-07-131@{1{[}3{]}. 7. 1903}|)be}\mylabel{h}  \normalsize

\doendnotes{C}
\bigskip
\vfill

\clearpage

\footnotesize

\lohead{\textsc{register}}

% Definiere theindex-Environment komplett neu ohne reledmac
\makeatletter
\renewenvironment{theindex}{%
  \section*{\indexname}%
  \setlength{\parindent}{0pt}%
  \setlength{\parskip}{0pt plus 0.3pt}%
  \let\item\@idxitem
}{%
  \clearpage
}
\makeatother

\IfFileExists{\jobname-pw.ind}{\input{\jobname-pw.ind}}{}

\end{document}

      