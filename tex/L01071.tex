%% latex-korrekturansicht-vorspann.tex
%% Vorspann für die Korrekturansicht.
%% Lädt die gemeinsame Datei latex-vorspann.tex mit gesetztem Schalter.

\newif\ifkorrekturansicht
\korrekturansichttrue

\input{../tex-inputs/latex-vorspann}


               \section[Richard Beer-Hofmann an Arthur Schnitzler, 6. 9. 1900]{ Richard Beer-Hofmann an Arthur Schnitzler, 6. 9. 1900}\nopagebreak\mylabel{v}\rehead{ }\normalsize\beginnumbering\briefempfaengerindex{Schnitzler, Arthur@\textsc{Schnitzler, Arthur}!zzzBeer-Hofmann, Richard@\emph{von Richard Beer-Hofmann}!1900-09-061@{6. 9. 1900}|(be} \toendnotes[C]{\smallbreak\pagebreak[2]} \Standort{CUL, Schnitzler, B 8.}
\physDesc{Brief, 1 Blatt, 1 Seite
\newline{}Handschrift: Bleistift, lateinische Kurrent\newline{}Ordnung: mit Bleistift von unbekannter Hand nummeriert: »158« }\pstart
           \raggedleft{}{\pb}\textcolor{pink}{Alt-Aussee}{}\ledrightnote{\textcolor{pink}{Altaussee}}{ }6/IX 1900\pend
           \pstart
           Lieber Arthur! In Eile: Ich bleibe noch bis ungefähr 18. hier u.
                  ko{\geminationm}e dann nach \textcolor{pink}{Wien}{}\ledrightnote{\textcolor{pink}{Wien}}. Dh: ich \uline{will} das thun. Werde ich \textcolor{blue}{Paul}{}\ledrightnote{\textcolor{blue}{Paul Goldmann}} dann noch in \textcolor{pink}{Wien}{}\ledrightnote{\textcolor{pink}{Wien}} treffen?\pend
           \pstart
           Schreiben Sie mir, bitte, zwei Zeilen.\pend
           \pstart Von Herzen Ihr \spacefill\mbox{R.}\pend{}\endnumbering\briefempfaengerindex{Schnitzler, Arthur@\textsc{Schnitzler, Arthur}!zzzBeer-Hofmann, Richard@\emph{von Richard Beer-Hofmann}!1900-09-061@{6. 9. 1900}|)be}\mylabel{h}  \normalsize

\doendnotes{C}
\bigskip
\vfill

\clearpage

\footnotesize

\lohead{\textsc{register}}

% Definiere theindex-Environment komplett neu ohne reledmac
\makeatletter
\renewenvironment{theindex}{%
  \section*{\indexname}%
  \setlength{\parindent}{0pt}%
  \setlength{\parskip}{0pt plus 0.3pt}%
  \let\item\@idxitem
}{%
  \clearpage
}
\makeatother

\IfFileExists{\jobname-pw.ind}{\input{\jobname-pw.ind}}{}

\end{document}

      