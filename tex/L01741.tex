%% latex-korrekturansicht-vorspann.tex
%% Vorspann für die Korrekturansicht.
%% Lädt die gemeinsame Datei latex-vorspann.tex mit gesetztem Schalter.

\newif\ifkorrekturansicht
\korrekturansichttrue

\input{../tex-inputs/latex-vorspann}


               \section[Arthur Schnitzler an Hermann Bahr, 16. 12. 1907]{ Arthur Schnitzler an Hermann Bahr, 16. 12. 1907}\nopagebreak\mylabel{v}\rehead{ }\normalsize\beginnumbering\briefempfaengerindex{Bahr, Hermann@\textsc{Bahr, Hermann}!zzzSchnitzler, Arthur@\emph{von Arthur Schnitzler}!1907-12-161@{16. 12. 1907}|(be} \toendnotes[C]{\smallbreak\pagebreak[2]} \Standort{TMW, HS AM 23388 Ba.}
\physDesc{Brief, 1 Blatt, 3 Seiten
\newline{}Handschrift: schwarze Tinte, lateinische Kurrent\newline{}Ordnung: Lochung }\buchAbdrucke{\weitereDrucke{1) \emph{16. 12. 1907.} In: Arthur Schnitzler: \emph{The Letters of Arthur Schnitzler to Hermann Bahr}. Edited, annotated, and with an introduction, by Donald G.
                        Daviau. Chapel Hill: \emph{The University of North Carolina Press} 1978, S. 100 (University of North Carolina studies in the Germanic languages
                        and literatures, 89).} \weitereDrucke{2) Hermann Bahr, Arthur Schnitzler: \emph{Briefwechsel, Aufzeichnungen, Dokumente (1891–1931)}. Hg. Kurt Ifkovits und Martin Anton Müller. Göttingen: \emph{Wallstein} 2018, S. 398.} }\toendnotes[C]{\smallbreak}\pstart
           \noindent{}\raggedleft{}{\pb}\uuline{Vertraulich}\pend
           \pstart
           \noindent{}\textcolor{gray}{\textbf{Dr. Arthur Schnitzler}}\hfill 16/12 907\pend
           \pstart
           \textcolor{gray}{\textbf{\textcolor{pink}{Wien XVIII. Spoettelgasse 7}{}\ledrightnote{\textcolor{pink}{Edmund-Weiß-Gasse}}.}}\pend
           \pstart{}lieber Hermann,\pend\pstart
           ich weiss nicht, ob du noch \label{K_L01741_1v}\edtext{in \textcolor{pink}{Wien}{}\ledrightnote{\textcolor{pink}{Wien}}}{\lemma{\textnormal{\emph{in Wien}}}\Cendnote{\textnormal{\textcolor{blue}{Bahr} war nicht mehr in \textcolor{pink}{Berlin}, doch möglicherweise auf dem \textcolor{pink}{Semmering}.}}}\label{K_L01741_1h} bist – schreibe dir jedenfalls an deine \textcolor{pink}{Wr}{}\ledrightnote{\textcolor{pink}{Wien}} Adresse, aufsuchen kö{\geminationn}t ich dich keineswegs, weil meine \textcolor{blue}{Frau}{}\ledrightnote{→\textcolor{blue}{Olga Schnitzler}} sich eben in Reconvalescenz von einem Scharlach befindet – (doch schon
               gekräftigt genug, um dich herzlich zu grüßen und dir mit mir zu dem \label{K_L01741_2v}\edtext{\textcolor{green}{nachtigalligen Erfolg}{}\ledrightnote{→\textcolor{green}{Die gelbe Nachtigall}}}{\lemma{\textnormal{\emph{nachtigalligen Erfolg}}}\Cendnote{\textnormal{Uraufführung von \emph{\textcolor{green}{Die gelbe Nachtigall}} am 10. 12. 1907 am \textcolor{pink}{Deutschen Theater}}}}\label{K_L01741_2h} schönstens zu gratuliren)
               – Also \uuline{unter uns}{ }{\pb}formeller Antrag des
                  \textcolor{pink}{Hebbeltheater}{}\ledrightnote{\textcolor{pink}{Hebbel-Theater}} liegt mir vor: \textcolor{green}{Beatrice}{}\ledrightnote{\textcolor{green}{Der Schleier der Beatrice. Schauspiel in fünf Akten}} nächste Saison, \textcolor{blue}{Ritscher}{}\ledrightnote{\textcolor{blue}{Helene Ritscher}} als \textcolor{green}{Beatrice}{}\ledrightnote{→\textcolor{green}{Der Schleier der Beatrice. Schauspiel in fünf Akten}}.
               Meine Frage an dich: hältst dus 1) für wahrscheinlich, dass \textcolor{blue}{Reinhardt}{}\ledrightnote{\textcolor{blue}{Max Reinhardt}} auf die \textcolor{green}{Beatrice}{}\ledrightnote{\textcolor{green}{Der Schleier der Beatrice. Schauspiel in fünf Akten}}
                  reflectirt\strikeout{e}? 2) hältst du, im Jafalle \textcolor{pink}{Deutsches Theater}{}\ledrightnote{\textcolor{pink}{Deutsches Theater Berlin}} für praktischer als \strikeout{für}{ }\textcolor{pink}{Hebbeltheater}{}\ledrightnote{\textcolor{pink}{Hebbel-Theater}}? 3) Zu welcher Zeit wäre \textcolor{blue}{Reinhardt}{}\ledrightnote{\textcolor{blue}{Max Reinhardt}} zu einer fixen Entscheidg zu
                  veranlassen?\strikeout{)}\pend
           \pstart
           {\pb}– Du bist nicht böse,
               wenn ich dich nochmals um vollkommen \uline{vertrauliche}
               Behandlg der Angelegenheit ersuche.\pend
           \pstart
           herzlichst der Deine,{\\[\baselineskip]}\spacefill\mbox{Arthur}\pend
           \leftskip=0em{}\endnumbering\briefempfaengerindex{Bahr, Hermann@\textsc{Bahr, Hermann}!zzzSchnitzler, Arthur@\emph{von Arthur Schnitzler}!1907-12-161@{16. 12. 1907}|)be}\mylabel{h}  \normalsize

\doendnotes{C}
\bigskip
\vfill

\clearpage

\footnotesize

\lohead{\textsc{register}}

% Definiere theindex-Environment komplett neu ohne reledmac
\makeatletter
\renewenvironment{theindex}{%
  \section*{\indexname}%
  \setlength{\parindent}{0pt}%
  \setlength{\parskip}{0pt plus 0.3pt}%
  \let\item\@idxitem
}{%
  \clearpage
}
\makeatother

\IfFileExists{\jobname-pw.ind}{\input{\jobname-pw.ind}}{}

\end{document}

      