%% latex-korrekturansicht-vorspann.tex
%% Vorspann für die Korrekturansicht.
%% Lädt die gemeinsame Datei latex-vorspann.tex mit gesetztem Schalter.

\newif\ifkorrekturansicht
\korrekturansichttrue

\input{../tex-inputs/latex-vorspann}


               \section[Arthur Schnitzler an Georg Brandes, 4. 1. 1899]{ Arthur Schnitzler an Georg Brandes, 4. 1. 1899}\nopagebreak\mylabel{v}\rehead{ }\normalsize\beginnumbering\briefempfaengerindex{Brandes, Georg@\textsc{Brandes, Georg}!zzzSchnitzler, Arthur@\emph{von Arthur Schnitzler}!1899-01-041@{4. 1. 1899}|(be} \toendnotes[C]{\smallbreak\pagebreak[2]} \Standort{Kopenhagen, Det Kongelige Bibliotek, Georg Brandes Arkiv, box 125.}
\physDesc{Briefkarte
\newline{}Handschrift: schwarze Tinte, deutsche Kurrent\newline{}Ordnung: von unbekannter Hand nummeriert:
                                        »12.« }\buchAbdrucke{\weitereDrucke{Georg Brandes, Arthur Schnitzler: \emph{Ein Briefwechsel}. Hg. Kurt Bergel. Bern: \emph{Francke} 1956, S. 69.} }\toendnotes[C]{\smallbreak}\pstart
           \noindent{}{\pb}Verehrteſter Herr Brandes, aus der \label{K_L00874_1v}\edtext{Zeitung}{\lemma{\textnormal{\emph{Zeitung}}}\Cendnote{\textnormal{In der
                            \emph{\textcolor{brown}{Neuen Freien Presse}} findet sich die
                        Meldung am 3. 1. 1899 ([O. V.:] \emph{\textcolor{green}{Ein Nachruf für Frau \textcolor{blue}{Brandes}}}, Nr. 12344, S. 5–6).}}}\label{K_L00874_1h} erfahre ich, dſs Ihre \textcolor{blue}{Mutter}{}\ledrightnote{→\textcolor{blue}{Emilie Brandes}} geſtorben iſt. In herzlicher
                    Theilnahme drücke ich Ihnen die Hand.\pend
           \pstart
           Ihr Ihnen wahrhaft ergebener{\\[\baselineskip]}\spacefill\mbox{Arthur Schnitzler}\pend
           \leftskip=0em{}\pstart
           \textcolor{pink}{Wien}{}\ledrightnote{\textcolor{pink}{Wien}}{\\}4. 1. 99.\pend
           \endnumbering\briefempfaengerindex{Brandes, Georg@\textsc{Brandes, Georg}!zzzSchnitzler, Arthur@\emph{von Arthur Schnitzler}!1899-01-041@{4. 1. 1899}|)be}\mylabel{h}  \normalsize

\doendnotes{C}
\bigskip
\vfill

\clearpage

\footnotesize

\lohead{\textsc{register}}

% Definiere theindex-Environment komplett neu ohne reledmac
\makeatletter
\renewenvironment{theindex}{%
  \section*{\indexname}%
  \setlength{\parindent}{0pt}%
  \setlength{\parskip}{0pt plus 0.3pt}%
  \let\item\@idxitem
}{%
  \clearpage
}
\makeatother

\IfFileExists{\jobname-pw.ind}{\input{\jobname-pw.ind}}{}

\end{document}

      