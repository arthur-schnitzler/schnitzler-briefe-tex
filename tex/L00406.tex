%% latex-korrekturansicht-vorspann.tex
%% Vorspann für die Korrekturansicht.
%% Lädt die gemeinsame Datei latex-vorspann.tex mit gesetztem Schalter.

\newif\ifkorrekturansicht
\korrekturansichttrue

\input{../tex-inputs/latex-vorspann}


               \section[Hugo August von Hofmannsthal an Arthur Schnitzler, 26. 11. 1894]{ Hugo August von Hofmannsthal an Arthur Schnitzler,
                    26. 11. 1894}\nopagebreak\mylabel{v}\rehead{ }\normalsize\beginnumbering\briefempfaengerindex{Schnitzler, Arthur@\textsc{Schnitzler, Arthur}!zzzHofmannsthal, Hugo August von@\emph{von Hugo August von Hofmannsthal}!1894-11-261@{26. 11. 1894}|(be} \toendnotes[C]{\smallbreak\pagebreak[2]} \Standort{DLA, A:Schnitzler, HS.NZ85.1.3483.}
\physDesc{Briefkarte
\newline{}Handschrift: blaue Tinte, deutsche Kurrent}\toendnotes[C]{\smallbreak}\pstart{}{\pb}Lieber Freund!\pend\pstart
           Wenn der verſt. Dombaumeiſter \textcolor{blue}{\textsc{Schmid}}{}\ledrightnote{\textcolor{blue}{Friedrich Schmidt}} einem Kunſtwerke uneingeſchränktes Lob zollen wollte, pflegte er einfach
                    zu ſagen: Das iſt einmal was Wirkliches! Das Wort ſprang mir auf die Lippen als
                    ich Ihr neues \textcolor{green}{Buch}{}\ledrightnote{→\textcolor{green}{Sterben. Novelle}} geleſen
                    hatte u ich weiß wirklich nichts beßeres darüber zu ſagen! Ich gratuliere Ihnen
                    herzlichſt {\pb}dazu und freue mich aufrichtig über
                    Ihr Können.\pend
           \pstart
           Mit den freundlichſten Grüßen Ihr{\\[\baselineskip]}ergebenſter{\\[\baselineskip]}\spacefill\mbox{D\textsuperscript{r} vHofmannsthal}\pend
           \leftskip=0em{}\pstart
           26/11 94.\pend
           \endnumbering\briefempfaengerindex{Schnitzler, Arthur@\textsc{Schnitzler, Arthur}!zzzHofmannsthal, Hugo August von@\emph{von Hugo August von Hofmannsthal}!1894-11-261@{26. 11. 1894}|)be}\mylabel{h}  \normalsize

\doendnotes{C}
\bigskip
\vfill

\clearpage

\footnotesize

\lohead{\textsc{register}}

% Definiere theindex-Environment komplett neu ohne reledmac
\makeatletter
\renewenvironment{theindex}{%
  \section*{\indexname}%
  \setlength{\parindent}{0pt}%
  \setlength{\parskip}{0pt plus 0.3pt}%
  \let\item\@idxitem
}{%
  \clearpage
}
\makeatother

\IfFileExists{\jobname-pw.ind}{\input{\jobname-pw.ind}}{}

\end{document}

      