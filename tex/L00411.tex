%% latex-korrekturansicht-vorspann.tex
%% Vorspann für die Korrekturansicht.
%% Lädt die gemeinsame Datei latex-vorspann.tex mit gesetztem Schalter.

\newif\ifkorrekturansicht
\korrekturansichttrue

\input{../tex-inputs/latex-vorspann}


               \section[Jakob Julius David an Arthur Schnitzler, 23. 12. 1894]{ Jakob Julius David an Arthur Schnitzler, 23. 12. 1894}\nopagebreak\mylabel{v}\rehead{ }\normalsize\beginnumbering\briefempfaengerindex{Schnitzler, Arthur@\textsc{Schnitzler, Arthur}!zzzDavid, Jakob Julius@\emph{von Jakob Julius David}!1894-12-231@{23. 12. 1894}|(be} \toendnotes[C]{\smallbreak\pagebreak[2]} \Standort{CUL, Schnitzler, B 25.}
\physDesc{Brief, 1 Blatt, 1 Seite
\newline{}Handschrift: schwarze Tinte, lateinische Kurrent
\newline{}Schnitzler: 1) mit rotem Buntstift beschriftet: »\textsc{David}« und der Buchtitel unterstrichen 2) mit Bleistift nummeriert: »1.«}\buchAbdrucke{\weitereDrucke{Josef Körner: \emph{Herman Groeneweg, J. J. David in seinem Verhältnis zur Heimat, Geschichte, Gesellschaft und Literatur.} In: \emph{Literaturblatt für germanische und romanische
                        Philologie}, Jg. 52 (1931), Sp. 33.} }\toendnotes[C]{\smallbreak}\pstart
           \raggedleft{}{\pb}23/12 94.\pend
           \pstart\center{}Werther Herr Doctor!\pend\pstart
           Ich habe \textcolor{green}{Sterben}{}\ledrightnote{\textcolor{green}{Sterben. Novelle}} bis nun zwei mal gelesen, und
               werde wohl noch darauf zurückkommen. Es ist eine höchst tüchtige und eine wirklich
               merkwürdige Arbeit; in der Analyse von wirksamster Feinheit und Tiefe.
               Bewundernswerth ist die Kunst, mit welcher Sie den zeitlich so knappen und doch für
               die Vorgänge fast zu weitgesteckten Rahmen mit Leben zu erfüllen wißen. Es ist ein
               vollkommen zielbewußtes Schlendern; was Abschweifung erscheinen könnte, führt nur
               desto sicherer zum letzten Ende. Manchmal möcht’ ich mir mehr Leidenschaftlichkeit
               verlangen; besonders am Schluße könnte ein stärkeres Temperament durchbrennen. Aber:
               Sie haben in dieser \textcolor{green}{Arbeit}{}\ledrightnote{→\textcolor{green}{Sterben. Novelle}} einen
               mächtigen Ruck vorwärts gethan und will ich Ihnen sagen, in wie ferne mir Arbeit das
               Höchste dünkt: im Sinne der Arbeit an sich selbst. Da nun sind Sie tüchtig und
               ehrlich am Werke und darum rücken Sie vor in schönen Erfolgen und zu einer ersten
               Stellung, auf die Sie heute schon Anspruch haben.\pend
           \pstart
           Es grüßt und begrüßt Sie herzlichst{\\[\baselineskip]}Ihr{\\[\baselineskip]}\spacefill\mbox{David}\pend
           \leftskip=0em{}\endnumbering\briefempfaengerindex{Schnitzler, Arthur@\textsc{Schnitzler, Arthur}!zzzDavid, Jakob Julius@\emph{von Jakob Julius David}!1894-12-231@{23. 12. 1894}|)be}\mylabel{h}  \normalsize

\doendnotes{C}
\bigskip
\vfill

\clearpage

\footnotesize

\lohead{\textsc{register}}

% Definiere theindex-Environment komplett neu ohne reledmac
\makeatletter
\renewenvironment{theindex}{%
  \section*{\indexname}%
  \setlength{\parindent}{0pt}%
  \setlength{\parskip}{0pt plus 0.3pt}%
  \let\item\@idxitem
}{%
  \clearpage
}
\makeatother

\IfFileExists{\jobname-pw.ind}{\input{\jobname-pw.ind}}{}

\end{document}

      