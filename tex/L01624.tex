%% latex-korrekturansicht-vorspann.tex
%% Vorspann für die Korrekturansicht.
%% Lädt die gemeinsame Datei latex-vorspann.tex mit gesetztem Schalter.

\newif\ifkorrekturansicht
\korrekturansichttrue

\input{../tex-inputs/latex-vorspann}


               \section[Hugo von Hofmannsthal an Arthur Schnitzler, 4. {[}9. 1906{]}]{ Hugo von Hofmannsthal an Arthur Schnitzler,
               4. {[}9. 1906{]}}\nopagebreak\mylabel{v}\rehead{ }\normalsize\beginnumbering\briefempfaengerindex{Schnitzler, Arthur@\textsc{Schnitzler, Arthur}!zzzHofmannsthal, Hugo von@\emph{von Hugo von Hofmannsthal}!1906-09-041@{4. {[}9. 1906{]}}|(be} \toendnotes[C]{\smallbreak\pagebreak[2]} \Standort{CUL, Schnitzler, B 43.}
\physDesc{Brief, 1 Blatt, 4 Seiten
\newline{}Handschrift: schwarze Tinte, deutsche Kurrent
\newline{}Schnitzler: mit Bleistift zum Datum eine mutmaßliche Monatsangabe ergänzt: »7(?).« \newline{}Ordnung: 1) mit Bleistift von unbekannter Hand nummeriert: »\strikeout{214}« 2) mit Bleistift von unbekannter Hand  nummeriert: »197«}\buchAbdrucke{\weitereDrucke{Hugo von Hofmannsthal, Arthur Schnitzler: \emph{Briefwechsel}. Hg. Therese Nickl und Heinrich Schnitzler. Frankfurt am Main: \emph{S. Fischer} 1964, S. 220.} }\toendnotes[C]{\smallbreak}\pstart
           \raggedleft{}{\pb}\textcolor{pink}{\textsc{Lueg}}{}\ledrightnote{\textcolor{pink}{Lueg am Wolfgangsee}}{ }4\textsuperscript{ten}\pend
           \pstart{}mein lieber Arthur \pend\pstart
           ich habe rechtes Verlangen, von Ihnen ein bischen ausführlicher zu hören.\hspace*{1.5em}Von mir (und \textcolor{blue}{Gerty}{}\ledrightnote{\textcolor{blue}{Gertrude von Hofmannsthal}})
               kann ich, was Stimmung, Laune, Genießen des Sommers betrifft, nur Gutes berichten,
               von einer größeren Arbeit iſt freilich noch nichts zu ſagen, manchmal {\pb}ſcheint dergleichen recht nahe,
               dann iſt es wieder, als ob es untertauchte und ſich verbärge, aber nicht in Waſſer,
               ſondern in einer viel härteren undurchſichtigen Subſtanz, doch halte ich gar nicht
               für unmöglich, daſs der Herbſt, der mir oft günſtig war, auch diesmal plötzlich und
               ſpringquellhaft wieder etwas {\pb}hervortreibt – das Gefühl der Armut hatte ich jedesfalls nicht, vieles größere und
               kleinere mehr Gedankenhafte hat ſich geordnet, aufgeſchrieben hab ich auch gar nicht
               weniges und eine gewiſſe Möglichkeit, epiſches (kürzeres zunächſt) in mir auszubilden
               fühle ich auch, mehr als ein Vorgefühl {\pb}allerdings. Unſeres letzten Zuſa{\geminationm}enſeins, des Spaziergangs bei drohenden Wolken und des
               ſchönen leichten und inhaltsvollen Redens denke ich auch – auf ein paar Tage \textcolor{pink}{Semmering}{}\ledrightnote{\textcolor{pink}{Semmering}} (vielleicht mit \textcolor{blue}{Brahm}{}\ledrightnote{\textcolor{blue}{Otto Brahm}}) möchte ich jedenfalls rechnen.\pend
           \pstart
           Ich weiß nicht, (da es ſo wunderſchön iſt) ob ich nicht noch 10–14 Tage hier bleibe,
               die \textcolor{blue}{Kinder}{}\ledrightnote{→\textcolor{blue}{Christiane von Hofmannsthal}{\newline}→\textcolor{blue}{Raimund von Hofmannsthal}{\newline}→\textcolor{blue}{Franz von Hofmannsthal}}{ }ſind ſchon in \textcolor{pink}{Rodaun}{}\ledrightnote{\textcolor{pink}{Rodaun}}.\pend
           \pstart
            Schreiben Sie.\hspace*{1.5em}Von Herzen\pend
           \pstart \spacefill\mbox{Hugo.}\pend{}\endnumbering\briefempfaengerindex{Schnitzler, Arthur@\textsc{Schnitzler, Arthur}!zzzHofmannsthal, Hugo von@\emph{von Hugo von Hofmannsthal}!1906-09-041@{4. {[}9. 1906{]}}|)be}\mylabel{h}  \normalsize

\doendnotes{C}
\bigskip
\vfill

\clearpage

\footnotesize

\lohead{\textsc{register}}

% Definiere theindex-Environment komplett neu ohne reledmac
\makeatletter
\renewenvironment{theindex}{%
  \section*{\indexname}%
  \setlength{\parindent}{0pt}%
  \setlength{\parskip}{0pt plus 0.3pt}%
  \let\item\@idxitem
}{%
  \clearpage
}
\makeatother

\IfFileExists{\jobname-pw.ind}{\input{\jobname-pw.ind}}{}

\end{document}

      