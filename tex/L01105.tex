%% latex-korrekturansicht-vorspann.tex
%% Vorspann für die Korrekturansicht.
%% Lädt die gemeinsame Datei latex-vorspann.tex mit gesetztem Schalter.

\newif\ifkorrekturansicht
\korrekturansichttrue

\input{../tex-inputs/latex-vorspann}


               \section[Arthur Schnitzler an Richard Beer-Hofmann, 3. 4. 1901]{ Arthur Schnitzler an Richard Beer-Hofmann, 3. 4. 1901}\nopagebreak\mylabel{v}\rehead{ }\normalsize\beginnumbering\briefempfaengerindex{Beer-Hofmann, Richard@\textsc{Beer-Hofmann, Richard}!zzzSchnitzler, Arthur@\emph{von Arthur Schnitzler}!1901-04-031@{3. 4. 1901}|(be} \toendnotes[C]{\smallbreak\pagebreak[2]} \Standort{YCGL, MSS 31.}
\physDesc{Bildpostkarte
\newline{}Handschrift: schwarze Tinte, deutsche Kurrent\newline{}Versand: Stempel: »\nobreak{}\oindex{Rom@\textbf{Rom}, \emph{Besiedelter Ort (A.BSO)}|pwk}Roma Ferrovia, 3 4 {[}1901{]}, 10\nobreak{}«.  \newline{}Ordnung: mit Bleistift von unbekannter Hand datiert: »3. 4.« }\pstart{}{\pb}Herrn Dr. \textsc{Richard
                            Beer-Hofmann}\pend{}\pstart{}\textcolor{pink}{Wien}{}\ledrightnote{\textcolor{pink}{Wien}}\pend{}\pstart{}\textcolor{pink}{\textsc{I. Wollzeile 15}}{}\ledrightnote{\textcolor{pink}{Wollzeile}}.\pend{}\pstart{}\textcolor{pink}{\textsc{Austria}}{}\ledrightnote{\textcolor{pink}{Österreich}}\pend{}{\bigskip}\pstart
           \noindent{}\centering{}{\pb}\textcolor{gray}{\textbf{\textcolor{pink}{Porta S. Paolo – Piramide di Caio
                                Cestio}{}\ledrightnote{\textcolor{pink}{Cestius-Pyramide}}.}}\pend
           \pstart
           lieber Richard, bitte, welches Hotel in \textcolor{pink}{Florenz}{}\ledrightnote{\textcolor{pink}{Florenz}} haben Sie mir empfohlen? Und wie geht es Ihnen u bei
                    Ihnen? Von Herzen Ihr\pend
           \pstart \spacefill\mbox{Arthur}\pend{}\pstart
           \textcolor{pink}{Rom}{}\ledrightnote{\textcolor{pink}{Rom}}, \textsc{\uline{post restante}}{\\}\substVorne{}\textsuperscript{2}\substDazwischen{}3\substHinten{}. 4. 90\textcolor{gray}{1}.\pend
           \endnumbering\briefempfaengerindex{Beer-Hofmann, Richard@\textsc{Beer-Hofmann, Richard}!zzzSchnitzler, Arthur@\emph{von Arthur Schnitzler}!1901-04-031@{3. 4. 1901}|)be}\mylabel{h}  \normalsize

\doendnotes{C}
\bigskip
\vfill

\clearpage

\footnotesize

\lohead{\textsc{register}}

% Definiere theindex-Environment komplett neu ohne reledmac
\makeatletter
\renewenvironment{theindex}{%
  \section*{\indexname}%
  \setlength{\parindent}{0pt}%
  \setlength{\parskip}{0pt plus 0.3pt}%
  \let\item\@idxitem
}{%
  \clearpage
}
\makeatother

\IfFileExists{\jobname-pw.ind}{\input{\jobname-pw.ind}}{}

\end{document}

      