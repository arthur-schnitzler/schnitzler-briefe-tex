%% latex-korrekturansicht-vorspann.tex
%% Vorspann für die Korrekturansicht.
%% Lädt die gemeinsame Datei latex-vorspann.tex mit gesetztem Schalter.

\newif\ifkorrekturansicht
\korrekturansichttrue

\input{../tex-inputs/latex-vorspann}


               \section[Paul Goldmann an Arthur Schnitzler, 1. 6. {[}1894{]}]{ Paul Goldmann an Arthur Schnitzler, 1. 6. {[}1894{]}}\nopagebreak\mylabel{v}\rehead{ }\normalsize\beginnumbering\briefempfaengerindex{Schnitzler, Arthur@\textsc{Schnitzler, Arthur}!zzzGoldmann, Paul@\emph{von Paul Goldmann}!1894-06-011@{1. 6. {[}1894{]}}|(be} \toendnotes[C]{\smallbreak\pagebreak[2]} \Standort{DLA, A:Schnitzler, HS.NZ85.1.3164.}
\physDesc{Brief, 2 Blätter, 6 Seiten
\newline{}Handschrift: schwarze Tinte, deutsche Kurrent
\newline{}Schnitzler: 1) mit Bleistift auf dem ersten Blatt die Jahreszahl »94« vermerkt 2) mit rotem Buntstift vier Unterstreichungen}\toendnotes[C]{\smallbreak}\pstart
           \noindent{}{\pb}\textcolor{gray}{\textbf{\textcolor{brown}{Frankfurter Zeitung}{}\ledrightnote{\textcolor{brown}{Frankfurter Zeitung}}.}}\hfill \textsc{\textcolor{pink}{Paris}{}\ledrightnote{\textcolor{pink}{Paris}}}, 1. Juni.\pend
           \pstart
           \textcolor{gray}{\textbf{(\textcolor{brown}{Gazette de
                     Francfort}{}\ledrightnote{\textcolor{brown}{Frankfurter Zeitung}}.)}}\pend
           \pstart
           \textcolor{gray}{\textbf{Fondateur \textbf{M. \textcolor{blue}{L. Sonnemann}{}\ledrightnote{\textcolor{blue}{Leopold Sonnemann}}}.}}\pend
           \pstart
           \textcolor{gray}{\textbf{\begin{otherlanguage}{french}Journal politique, financier,\end{otherlanguage}}}\pend
           \pstart
           \textcolor{gray}{\textbf{\begin{otherlanguage}{french}commercial et littéraire.\end{otherlanguage}}}\pend
           \pstart
           \textcolor{gray}{\textbf{\begin{otherlanguage}{french}\textbf{Paraissant trois fois par jour.}\end{otherlanguage}}}\pend
           \pstart
           \textcolor{gray}{\textbf{–}}\pend
           \pstart
           \textcolor{gray}{\textbf{\begin{otherlanguage}{french}\textbf{Bureau à \textcolor{pink}{Paris}{}\ledrightnote{\textcolor{pink}{Paris}}:}\end{otherlanguage}}}\pend
           \pstart
           \textcolor{gray}{\textbf{\begin{otherlanguage}{french}\textcolor{pink}{24. Rue Feydeau}{}\ledrightnote{\textcolor{pink}{rue Feydeau}}.\end{otherlanguage}}}\pend
           \pstart\center{}Mein lieber Freund,\pend\pstart
           \textcolor{blue}{\textsc{Hermann Bahr}}{}\ledrightnote{\textcolor{blue}{Hermann Bahr}} iſt alſo doch bei mir geweſen; aber ich wünſchte, es wäre lieber nicht
               geſchehen. Er hat mir einen abſcheulichen Eindruck gemacht, – ein Intriguant, ein
               Jeſuit – und wenn, wie dies wahrſcheinlich, ſeine Geſinnung der meinigen gleicht, ſo
               ſind wir, mit einem herzlichen Händedruck, als erklärte Feinde geſchieden. Der \textcolor{blue}{Mann}{}\ledrightnote{→\textcolor{blue}{Hermann Bahr}} hat mir in der kurzen
               Zeit ſeines Hier-Seins mehr Stänkereien angerichtet, als ſonſt irgend Einer, hat mich
               aus meiner Sicherheit {\pb}gebracht und mich durch
               allerlei Perfidie erregt und verſtimmt. Es wäre zu weitläufig, das hier zu erzählen;
               der \textcolor{blue}{Mensch}{}\ledrightnote{→\textcolor{blue}{Hermann Bahr}}, der hier mit
               einem infamen Pack von Reportern niedrigſter Sorte verkehrt, hat ſich dort allerlei
               Verleumdungen über mich geholt, die er mir, mit liebenswürdigem Wohlwollen, wieder
               erzählt hat. Ich berühre das nur, um Dich davor zu warnen, irgendwelchen
               freundſchaftlichen Referaten aus dieſer Quelle Glauben zu ſchenken. Der Grund,
               weshalb ich mich heut an Dich wende, iſt ein \strikeout{b}
               anderer. Er liegt in Einigem, was mir der \textcolor{blue}{Herr}{}\ledrightnote{→\textcolor{blue}{Hermann Bahr}} über Euch geſagt hat. Zunächſt ſelbſtverſtändlich ſpielt
               er ſich als den eigentlichen Förderer und {\pb}Inſpirator der \label{K_L02623-1v}\edtext{\textcolor{pink}{Wien}{}\ledrightnote{\textcolor{pink}{Wien}}er Literatur-Strömung}{\lemma{\textnormal{\emph{Wiener Literatur-Strömung}}}\Cendnote{\textnormal{Bei »Jung Wien« handelte
                  es sich um eine losen Verbund von Autoren ohne gemeinsames Programm. Unter diesem Namen agierte kurze Zeit ein Verein, der sich
                  zumindest zwischen 17. 3. 1891 und
                  5. 5. 1891 wöchentlich traf.
                  Einen
                  Anspruch auf Popularisierung der neuen Strömung und damit auch auf eine Rolle als
                  ihr Ausformer konnte \textcolor{blue}{Bahr} damit begründen,
                  dass er in einem dreiteiligen Feuilleton, \emph{\textcolor{green}{Das
                     junge Österreich}}, das zuerst am 20. 9. 1893, am
                     27. 9. 1893 und am 7. 10. 1893 in der \emph{\textcolor{green}{Deutschen Zeitung}} erschien, erstmals eine gemeinsame
                  Sichtung unternahm (Jg. 23, Nr. 7806, Morgen-Ausgabe, S. 1–2; Nr. 7813,
                     Morgen-Ausgabe, S. 1–3; Nr. 7823, Morgen-Ausgabe, S. 1–3). Im Folgejahr
                  nahm er es in die Zusammenstellung von Texten \emph{\textcolor{green}{Studien zur Kritik der Moderne}} (Frankfurt am Main: \emph{\textcolor{brown}{Literarische Anstalt Rütten {\kaufmannsund} Loening}}) auf. Das »Euch« dürfte dabei auf die bleibendsten dieser Autoren gemünzt
                  sein, die privat in regelmäßigem Umgang mit \textcolor{blue}{Schnitzler} standen, vor allem \textcolor{blue}{Richard
                     Beer-Hofmann}, \textcolor{blue}{Hugo von Hofmannsthal}
                  und \textcolor{blue}{Felix Salten}.}}}\label{K_L02623-1h} auf. Zu gleicher
               Zeit hat er über jeden von Euch bei aller ſcheinbaren Anerkennung irgend ein
               herabſetzendes Wort, ſo daß von der \textcolor{pink}{Wien}{}\ledrightnote{\textcolor{pink}{Wien}}er
               Literatur eigentlich als vollgiltig nur \textcolor{blue}{Hermann \textsc{Bahr}}{}\ledrightnote{\textcolor{blue}{Hermann Bahr}} übrig bleibt. Selbſt die Leute ſeiner eigenen \textcolor{green}{Revüe}{}\ledrightnote{→\textcolor{green}{Die Zeit. Wiener Wochenschrift}} drückt er herunter. \textsc{\textcolor{blue}{Kanner}{}\ledrightnote{\textcolor{blue}{Heinrich Kanner}}}{ }\strikeout{iſt} wird ſich nach ſeiner Darſtellung mit der
               Adminiſtration befaſſen; und wenn \strikeout{n} man \textsc{\textcolor{blue}{Kanner}{}\ledrightnote{\textcolor{blue}{Heinrich Kanner}}} nur aus ſeinen Reden kennt, ſo muß man ihn für nichts als für einen Kaſſier
               halten, während doch in Wahrheit \textsc{\textcolor{blue}{Kanner}{}\ledrightnote{\textcolor{blue}{Heinrich Kanner}}} der \strikeout{Ein} Einzige iſt, der für die {\pb}\textsc{\textcolor{green}{Revue}{}\ledrightnote{→\textcolor{green}{Die Zeit. Wiener Wochenschrift}}} Zukunfts-Hoffnungen rechtfertigt. Nun aber zu Euch zurück. Ich möchte Dich
               bitten, mir mit ein paar Worten etwas über das Verhältniß von \textsc{\textcolor{blue}{Hermann Bahr}{}\ledrightnote{\textcolor{blue}{Hermann Bahr}}} zu Eurem Kreiſe zu ſagen. Insbeſondere möchte ich wiſſen, ob zwiſchen ihm und
                  \label{K_L02623-44v}\edtext{\textsc{\textcolor{blue}{Loris}{}\ledrightnote{\textcolor{blue}{Hugo von Hofmannsthal}}} wirklich jene intime Freundſchaft}{\lemma{\textnormal{\emph{Loris … Freundſchaft}}}\Cendnote{\textnormal{Ohne \textcolor{blue}{Schnitzler}s Antwort zu kennen, finden
                  sich in seinem \emph{\textcolor{green}{Tagebuch}} doch mehrfach
                  Aussagen, die die bestehende Nähe zwischen \textcolor{blue}{Bahr} und \textcolor{blue}{Hofmannsthal} kritisch
                  beurteilen, beispielsweise A. S.: \emph{Tagebuch}, 6. 11. 1895, aber
                  auch \textcolor{blue}{Goldmann} beschäftigt das Thema länger,
                     vgl. A. S.: \emph{Tagebuch}, 26. 8. 1895. }}}\label{K_L02623-44h}
               beſteht, \strikeout{die} wie er vorgibt; ob er wirklich
               berechtigt iſt, ſich als den »\textcolor{blue}{Erzieher}{}\ledrightnote{→\textcolor{blue}{Hermann Bahr}}« von \textsc{\textcolor{blue}{Loris}{}\ledrightnote{\textcolor{blue}{Hugo von Hofmannsthal}}} aufzuſpielen, wie er das thut \textsc{etc}. Bitte, ſchreib’
               mir bald; denn das Alles quält mich ſehr ſeit geſtern{ }Abend. Ich will Dir nicht ſagen, warum, ſondern Deine Antwort
               abwarten.\pend
           \pstart
           Herzlichſt und in Treue {\\[\baselineskip]}Dein \spacefill\mbox{Paul Goldmann.}\pend
           \leftskip=0em{}\pstart
           \noindent{}{\pb}Ja ſo, entſchuldige, in meiner Erregung hätte ich
                  beinahe Deine Angelegenheiten vergeſſen. Der Verleger \textcolor{blue}{\textsc{Albert Langen}}{}\ledrightnote{\textcolor{blue}{Albert Langen}} iſt ein reicher junger Menſch, der ſich zum Verleger gemacht hat, um mit
                  Literatur protzen zu können. Der \textcolor{blue}{Menſch}{}\ledrightnote{→\textcolor{blue}{Albert Langen}} iſt idiotiſch urtheilslos, \strikeout{und} verlogen und betrügeriſch. Er iſt von dem halb wahnſinnigen \textsc{\textcolor{blue}{Gretor}{}\ledrightnote{\textcolor{blue}{Willy Grétor}}} beeinflußt, von dem ich Dir im \label{K_L02623-4v}\edtext{vorigen Sommer erzählt}{\lemma{\textnormal{\emph{vorigen Sommer erzählt}}}\Cendnote{\textnormal{XXXX}}}\label{K_L02623-4h}.
                  Ich rathe Dir dringend, Dich \label{K_L02623-2v}\edtext{mit
                  dem Burſchen in nichts {\pb}einzulaſſen}{\lemma{\textnormal{\emph{mit … einzulaſſen}}}\Cendnote{\textnormal{In \textcolor{blue}{Langen}s \emph{\textcolor{green}{Simplicissimus}} erschien
                     nur knapp zwei Jahre später, am 18. 4. 1896, Schnitzlers Einakter
                        \emph{\textcolor{green}{Die überspannte Person}}.}}}\label{K_L02623-2h}.\pend
           \pstart
           Deine \label{K_L02623-5v}\edtext{\textcolor{green}{Novelle}{}\ledrightnote{→\textcolor{green}{Sterben. Novelle}}}{\lemma{\textnormal{\emph{Novelle}}}\Cendnote{\textnormal{Es dürfte sich um die Buchausgabe von
                        \emph{\textcolor{green}{Sterben}} handeln. \textcolor{blue}{Fedor Mamroth} hatte im Vorjahr den Abdruck abgelehnt,
                        vgl. Fedor Mamroth an Arthur Schnitzler, 4. 6. 1893. Am
                        4. 12. 1894 wurde die Novelle in der \emph{\textcolor{green}{Frankfurter Zeitung}}{ }\textcolor{green}{rezensiert}, vgl. Arthur Schnitzler an Fedor Mamroth, 7. 12. 1894.}}}\label{K_L02623-5h} ſollſt Du natürlich
                  ſofort der \textcolor{brown}{Frankf. Ztg.}{}\ledrightnote{\textcolor{brown}{Frankfurter Zeitung}} ſchicken.\pend
           \pstart
           Wenn Du nur eine Ahnung hätteſt, wie mich alle »äußeren Umſtände Deiner Exiſtenz«
                  intereſſieren. Vor Allem: haſt Du materielle Sorgen?\pend
           \pstart
           Glückliche Reiſe und frohe Stimmung für die Reiſe! Such’ Dir in \label{K_L02623-3v}\edtext{\textsc{\textcolor{pink}{Muenchen}{}\ledrightnote{\textcolor{pink}{München}}}}{\lemma{\textnormal{\emph{Muenchen}}}\Cendnote{\textnormal{Von 2. 6. 1894 bis 8. 6. 1894 hielt sich Schnitzler in \textcolor{pink}{München} auf.}}}\label{K_L02623-3h} in einem der kleinen
                  Seiten-Cabinete der \textsc{\textcolor{pink}{Pinakothek}{}\ledrightnote{\textcolor{pink}{Alte Pinakothek}}} den kleinen \textsc{\textcolor{green}{\textcolor{blue}{Altdorfer}{}\ledrightnote{\textcolor{blue}{Albrecht Altdorfer}}}{}\ledrightnote{→\textcolor{green}{Laubwald mit dem heiligen Georg}}}{ }\strikeout{de} auf, welcher einen grünen, grünen Wald
                  darſtellt, worin ein putziger kleiner Ritter einen Drachen bekämpft! Das iſt eines
                  meiner Lieblingsbilder: Deutſch und märchenhaft.\pend
           \endnumbering\briefempfaengerindex{Schnitzler, Arthur@\textsc{Schnitzler, Arthur}!zzzGoldmann, Paul@\emph{von Paul Goldmann}!1894-06-011@{1. 6. {[}1894{]}}|)be}\mylabel{h}  \normalsize

\doendnotes{C}
\bigskip
\vfill

\clearpage

\footnotesize

\lohead{\textsc{register}}

% Definiere theindex-Environment komplett neu ohne reledmac
\makeatletter
\renewenvironment{theindex}{%
  \section*{\indexname}%
  \setlength{\parindent}{0pt}%
  \setlength{\parskip}{0pt plus 0.3pt}%
  \let\item\@idxitem
}{%
  \clearpage
}
\makeatother

\IfFileExists{\jobname-pw.ind}{\input{\jobname-pw.ind}}{}

\end{document}

      