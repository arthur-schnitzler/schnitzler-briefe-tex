%% latex-korrekturansicht-vorspann.tex
%% Vorspann für die Korrekturansicht.
%% Lädt die gemeinsame Datei latex-vorspann.tex mit gesetztem Schalter.

\newif\ifkorrekturansicht
\korrekturansichttrue

\input{../tex-inputs/latex-vorspann}


               \section[Arthur Schnitzler an Hermann Bahr, 11. 3. 1907]{ Arthur Schnitzler an Hermann Bahr, 11. 3. 1907}\nopagebreak\mylabel{v}\rehead{ }\normalsize\beginnumbering\briefempfaengerindex{Bahr, Hermann@\textsc{Bahr, Hermann}!zzzSchnitzler, Arthur@\emph{von Arthur Schnitzler}!1907-03-111@{11. 3. 1907}|(be} \toendnotes[C]{\smallbreak\pagebreak[2]} \Standort{TMW, HS AM 23384 Ba.}
\physDesc{Brief, 1 Blatt, 1 Seite
\newline{}Schreibmaschine
\newline{}Handschrift: 1) blaue Tinte, deutsche Kurrent (\noindent{}Unterschrift und Nachschrift, Korrekturen)\hspace{1em}2) Bleistift, deutsche Kurrent (\noindent{}Unterschrift und Nachschrift, Korrekturen)\hspace{1em}}\Standort{DLA, A:Schnitzler, 85.1.294/1.}
\physDesc{Brief, 1 Blatt, 1 Seite, maschineller Durchschlag
\newline{}Schreibmaschine}\buchAbdrucke{\weitereDrucke{1) \emph{11. 3. 1907.} In: Arthur Schnitzler: \emph{The Letters of Arthur Schnitzler to Hermann Bahr}. Edited, annotated, and with an introduction, by Donald G.
                        Daviau. Chapel Hill: \emph{The University of North Carolina Press} 1978, S. 97 (University of North Carolina studies in the Germanic languages
                        and literatures, 89).} \weitereDrucke{2) Hermann Bahr, Arthur Schnitzler: \emph{Briefwechsel, Aufzeichnungen, Dokumente (1891–1931)}. Hg. Kurt Ifkovits und Martin Anton Müller. Göttingen: \emph{Wallstein} 2018, S. 390.} }\toendnotes[C]{\smallbreak}\pstart
           \noindent{}\raggedleft{}{\pb}\textcolor{pink}{XVIII Spoettelgasse 7}{}\ledrightnote{\textcolor{pink}{Edmund-Weiß-Gasse}}\pend
           \pstart
           \raggedleft{}\textcolor{pink}{Wien}{}\ledrightnote{\textcolor{pink}{Wien}} am 11. März 07.\pend
           \pstart{}Lieber Hermann,\pend\pstart
           Da ich nichts weiter von Dir gehört habe scheint es, dass das Projekt der \textcolor{pink}{Kammer}{}\ledrightnote{\textcolor{pink}{Kammerspiele Berlin}}\textcolor{green}{liebelei}{}\ledrightnote{\textcolor{green}{Liebelei. Schauspiel in drei Akten}} vorläufig zurückgelegt worden ist. Nun
               fällt mir etwas ein, dass ich Dir zu gelegentlicher Ueberlegung mitteilen möchte.
                  \label{LL097-1v}Wie wärs, wenn die \textcolor{pink}{Kammerspiele}{}\ledrightnote{\textcolor{pink}{Kammerspiele Berlin}} in der nächsten Saison einen Versuch mit dem
                     »\textcolor{green}{Märchen}{}\ledrightnote{\textcolor{green}{Das Märchen. Schauspiel in drei Aufzügen}}« wagten. Du weisst, dass das Stück
                  über \textcolor{pink}{Wien}{}\ledrightnote{\textcolor{pink}{Wien}} nie hinausgekommen ist, dass es hingegen
                  – in \textcolor{pink}{Russland}{}\ledrightnote{\textcolor{pink}{Russland}} – einen meiner stärksten und
                  dauerndsten Erfolge bedeutet hat.\label{LL097-1h} Es ist wirklich geradezu lächerlich,
               dass sich in \textcolor{pink}{Deutschland}{}\ledrightnote{\textcolor{pink}{Deutschland}} noch kein Theater an das
               Stück gewagt hat. Die \textcolor{pink}{Kammerspiele}{}\ledrightnote{\textcolor{pink}{Kammerspiele Berlin}}, die das \label{K_L01662_1v}\edtext{\textcolor{green}{Friedensfest}{}\ledrightnote{\textcolor{green}{Das Friedensfest}} aufgeführt}{\lemma{\textnormal{\emph{Friedensfest aufgeführt}}}\Cendnote{\textnormal{\textcolor{blue}{Hauptmann}s \emph{\textcolor{green}{Das
                     Friedensfest}} hatte am 7. 1. 1907 Premiere.}}}\label{K_L01662_1h} haben,
               wären vielleicht am ehesten dazu geeignet, eine Aufführung dieses Stücks | mit der
                  \textcolor{blue}{Höflich}{}\ledrightnote{\textcolor{blue}{Lucie Höflich}} | zu versuchen, womit wenig riskiert
               und möglicherweise einiges zu gewinnen wäre. Dass der Schluss des dritten Aktes
               geändert ist dürfte Dir bekannt sein.\pend
           \pstart
           Wenn Du glaubst, dass die Sache nicht ganz aus{[}s{]}ichts{\pb}los ist, so sprichst Du
               vielleicht bei irgend einer Gelegenheit in diesem Sinn mit \textcolor{blue}{Reinhart}{}\ledrightnote{\textcolor{blue}{Max Reinhardt}}.\pend
           \pstart
           Sei herzlich gegrüsst und lass jedenfalls recht bald etwas von Dir hören. Wann kommst
               Du zurück? Du häl{[}t{]}st Dich doch vor \label{K_L01662_2v}\edtext{\textcolor{pink}{Ragusa}{}\ledrightnote{\textcolor{pink}{Dubrovnik}}}{\lemma{\textnormal{\emph{Ragusa}}}\Cendnote{\textnormal{\label{LKommKL097-1v}Vom 1. bis zum
                        8. 5. 1907 urlaubt \textcolor{blue}{Bahr} an
                     der oberen Adria, nach \textcolor{pink}{Dubrovnik} kommt er nicht.\label{LKommKL097-1h}}}}\label{K_L01662_2h} einige Zeit in \textcolor{pink}{Wien}{}\ledrightnote{\textcolor{pink}{Wien}} auf? \pend
           \pstart
           {[}hs.:{]} Dein{\\[\baselineskip]}\spacefill\mbox{Arthur}\pend
           \leftskip=0em{}\pstart
           \noindent{}viele Grüße von meiner \textcolor{blue}{Frau}{}\ledrightnote{→\textcolor{blue}{Olga Schnitzler}}.\pend
           \endnumbering\briefempfaengerindex{Bahr, Hermann@\textsc{Bahr, Hermann}!zzzSchnitzler, Arthur@\emph{von Arthur Schnitzler}!1907-03-111@{11. 3. 1907}|)be}\mylabel{h}  \normalsize

\doendnotes{C}
\bigskip
\vfill

\clearpage

\footnotesize

\lohead{\textsc{register}}

% Definiere theindex-Environment komplett neu ohne reledmac
\makeatletter
\renewenvironment{theindex}{%
  \section*{\indexname}%
  \setlength{\parindent}{0pt}%
  \setlength{\parskip}{0pt plus 0.3pt}%
  \let\item\@idxitem
}{%
  \clearpage
}
\makeatother

\IfFileExists{\jobname-pw.ind}{\input{\jobname-pw.ind}}{}

\end{document}

      