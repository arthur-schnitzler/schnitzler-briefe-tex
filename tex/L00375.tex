%% latex-korrekturansicht-vorspann.tex
%% Vorspann für die Korrekturansicht.
%% Lädt die gemeinsame Datei latex-vorspann.tex mit gesetztem Schalter.

\newif\ifkorrekturansicht
\korrekturansichttrue

\input{../tex-inputs/latex-vorspann}


               \section[Richard Beer-Hofmann an Arthur Schnitzler, {[}2. 10. 1894{]}]{ Richard Beer-Hofmann an Arthur Schnitzler, {[}2. 10. 1894{]}}\nopagebreak\mylabel{v}\rehead{ }\normalsize\beginnumbering\briefempfaengerindex{Schnitzler, Arthur@\textsc{Schnitzler, Arthur}!zzzBeer-Hofmann, Richard@\emph{von Richard Beer-Hofmann}!1894-10-021@{{[}2. 10. 1894{]}}|(be} \toendnotes[C]{\smallbreak\pagebreak[2]} \Standort{CUL, Schnitzler, B 8.}
\physDesc{Brief, 1 Blatt, 4 Seiten
\newline{}Handschrift: Bleistift, lateinische Kurrent
\newline{}Schnitzler: mit Bleistift beschriftet »\textcolor{pink}{Florenz}, 2/10 94« und nummeriert: »48« }\buchAbdrucke{\weitereDrucke{1) Arthur Schnitzler, Richard Beer-Hofmann: \emph{Briefwechsel 1891–1931}. Hg. Konstanze Fliedl. Wien, Zürich: \emph{Europaverlag} 1992, S. 61–62.} \weitereDrucke{2) Hermann Bahr, Arthur Schnitzler: \emph{Briefwechsel, Aufzeichnungen, Dokumente
                                (1891–1931)}. Hg. Kurt Ifkovits und Martin Anton Müller. Göttingen: \emph{Wallstein} 2018.} }\toendnotes[C]{\smallbreak}\pstart
           \noindent{}{\pb}Lieber Arthur! Mit Ihrem Brief hab ich mich sehr gefreut. Wenn
                    man Tagelang stu{\geminationm} unter schönen Sachen herum geht
                    freut einen eine – na wie soll ich sagen, – na eine \uline{bekannte} sti{\geminationm}e wieder –\pend
           \pstart
           Ich bin von den \textcolor{pink}{Uffizien}{}\ledrightnote{\textcolor{pink}{Uffizien}} geko{\geminationm}en u. habe auf dem Wege ins Restaurant {\pb}Ihren Brief von der Post
                    geholt und ihn dann mit Behagen während des Speisens gelesen. Ich habe Aufsehen
                    erregt weil ich fortwährend, auch nachher geschmunzelt habe, schließlich hat der
                    Kellner auch geschmunzelt und mich für eine heitere joviale Natur gehalten.\pend
           \pstart
           Sie schreiben i{\geminationm}er schlechter; d. h. ich kann sehr
                    schwer {\pb}Ihre Zeilen
                    entziffern, höchstens die Unterschrift, und die heisst dann »Richard«. Wenn Sie
                    mich nach der
                    »Madonna«
                    fragen, und noch dazu so nebenher im Postscriptum ({\{}2, 4, 6, 8 – – – – ∞?{\}}gradig?) so beweist dies nur daß »sie«
                    Ihre sexuelle Phantasie stark erregt. Bitte. – Bitte tun Sie wie wenn ich nicht
                    zu Hause wäre. – Sie können auch nach meiner Adresse fragen, – mehrmals – {\pb}und dabei findet sich
                    Gelegenheit.\pend
           \pstart
           Bitte: \textcolor{blue}{Bahr}{}\ledrightnote{\textcolor{blue}{Hermann Bahr}} soll die »\textcolor{brown}{Zeit}{}\ledrightnote{\textcolor{brown}{Die Zeit. Wiener Wochenschrift}}« (die erste Nu{\geminationm}er) \uline{a posta ferma}{ }\uline{\textcolor{pink}{Rom}{}\ledrightnote{\textcolor{pink}{Rom}}} senden – ja? Von Donnerstag an, bitte adressiren Sie auch die
                    Briefe u. Karten an mich, dorthin. Und schreiben Sie mir öfters: Ich werde jeden
                    Tag vor Tisch mir etwas von Ihnen abholen gehen. Ihr »\textcolor{green}{\textcolor{blue}{Guercino}{}\ledrightnote{\textcolor{blue}{Guercino}}}{}\ledrightnote{→\textcolor{green}{Die Verstoßung der Hagar}}« hängt in \textcolor{pink}{Mailand}{}\ledrightnote{\textcolor{pink}{Mailand}}. Grüße bitte richten
                    Sie ein für allemal \uline{à discretion} aus, wissen
                    Sie, so als Belohnung. Herzlichst Ihr –\pend
           \pstart \spacefill\mbox{Richard}\pend{}\pstart
           Dienstag{ }\introOben{}(½ 11)\introOben{}{ }früh,! \textcolor{pink}{Florenz}{}\ledrightnote{\textcolor{pink}{Florenz}}\pend
           \endnumbering\briefempfaengerindex{Schnitzler, Arthur@\textsc{Schnitzler, Arthur}!zzzBeer-Hofmann, Richard@\emph{von Richard Beer-Hofmann}!1894-10-021@{{[}2. 10. 1894{]}}|)be}\mylabel{h}  \normalsize

\doendnotes{C}
\bigskip
\vfill

\clearpage

\footnotesize

\lohead{\textsc{register}}

% Definiere theindex-Environment komplett neu ohne reledmac
\makeatletter
\renewenvironment{theindex}{%
  \section*{\indexname}%
  \setlength{\parindent}{0pt}%
  \setlength{\parskip}{0pt plus 0.3pt}%
  \let\item\@idxitem
}{%
  \clearpage
}
\makeatother

\IfFileExists{\jobname-pw.ind}{\input{\jobname-pw.ind}}{}

\end{document}

      