%% latex-korrekturansicht-vorspann.tex
%% Vorspann für die Korrekturansicht.
%% Lädt die gemeinsame Datei latex-vorspann.tex mit gesetztem Schalter.

\newif\ifkorrekturansicht
\korrekturansichttrue

\input{../tex-inputs/latex-vorspann}


               \section[Arthur Schnitzler an Hermann Bahr, 13. 1. 1908]{ Arthur Schnitzler an Hermann Bahr, 13. 1. 1908}\nopagebreak\mylabel{v}\rehead{ }\normalsize\beginnumbering\briefempfaengerindex{Bahr, Hermann@\textsc{Bahr, Hermann}!zzzSchnitzler, Arthur@\emph{von Arthur Schnitzler}!1908-01-131@{13. 1. 1908}|(be} \toendnotes[C]{\smallbreak\pagebreak[2]} \Standort{TMW, HS AM 60171 Ba.}
\physDesc{Briefkarte
\newline{}Handschrift: schwarze Tinte, lateinische Kurrent\newline{}Ordnung: Lochung }\buchAbdrucke{\weitereDrucke{1) \emph{13. 1. 1908, Abschrift.} In: Arthur Schnitzler: \emph{The Letters of Arthur Schnitzler to Hermann Bahr}. Edited, annotated, and with an introduction, by Donald G.
                        Daviau. Chapel Hill: \emph{The University of North Carolina Press} 1978, S. 101 (University of North Carolina studies in the Germanic languages
                        and literatures, 89).} \weitereDrucke{2) Hermann Bahr, Arthur Schnitzler: \emph{Briefwechsel, Aufzeichnungen, Dokumente (1891–1931)}. Hg. Kurt Ifkovits und Martin Anton Müller. Göttingen: \emph{Wallstein} 2018, S. 401.} }\toendnotes[C]{\smallbreak}\pstart
           {\pb}13\damage{. 1}. 908\pend
           \pstart
           \textcolor{gray}{\textbf{Dr. Arthur Schnitzler}}\pend
           \pstart
           \textcolor{gray}{\textbf{\textcolor{pink}{Wien XVIII. Spoettelgasse 7}{}\ledrightnote{\textcolor{pink}{Edmund-Weiß-Gasse}}.}}\pend
           \pstart
           mein lieber Hermann, erst heut dank ich dir für deinen guten Brief
               vom 23. v. M. Mit \label{K_L01750_1v}\edtext{\textcolor{pink}{Hebbelth}{}\ledrightnote{\textcolor{pink}{Hebbel-Theater}} hab ich abgeschlossen}{\lemma{\textnormal{\emph{Hebbelth … abgeschlossen}}}\Cendnote{\textnormal{vgl. Arthur Schnitzler an Hermann Bahr, 16. 12. 1907; Hermann Bahr an Arthur Schnitzler, 18. 12. 1907; Arthur Schnitzler an Hermann Bahr, 20. 12. 1907}}}\label{K_L01750_1h} – doch hör ich von \textcolor{blue}{Valentin}{}\ledrightnote{\textcolor{blue}{Richard Vallentin}}s \label{K_L01750_2v}\edtext{Gesundheitszustand}{\lemma{\textnormal{\emph{Gesundheitszustand}}}\Cendnote{\textnormal{\textcolor{blue}{Richard Vallentin} starb am
                     14. 1. 1908.}}}\label{K_L01750_2h} ungünstiges. (Und über das Theater selbst\substVorne{}\textsuperscript{, }\substDazwischen{} (\substHinten{}unter uns) nichts sehr hoffnungsreiches.) Meine \textcolor{blue}{Frau}{}\ledrightnote{→\textcolor{blue}{Olga Schnitzler}} liegt noch, die Contumaz dauert etwa noch 10–14 Tage.
               Schreib mir {\pb}ein Wort,
                  \label{K_L01750_3v}\edtext{wa{\geminationn} du
               nach \textcolor{pink}{Berlin}{}\ledrightnote{\textcolor{pink}{Berlin}} fährst}{\lemma{\textnormal{\emph{wa du
               nach Berlin fährst}}}\Cendnote{\textnormal{\textcolor{blue}{Bahr} begann am 18. 1. 1908 den vierten (und letzten) zweimonatigen Aufenthalt bei \textcolor{blue}{Max Reinhardt} in \textcolor{pink}{Berlin}.}}}\label{K_L01750_3h}. Wie gern spräch ich dich bald wieder. Herzliche
               Grüße.\pend
           \pstart
           Dein{\\[\baselineskip]}\spacefill\mbox{Arthur}\pend
           \leftskip=0em{}\endnumbering\briefempfaengerindex{Bahr, Hermann@\textsc{Bahr, Hermann}!zzzSchnitzler, Arthur@\emph{von Arthur Schnitzler}!1908-01-131@{13. 1. 1908}|)be}\mylabel{h}  \normalsize

\doendnotes{C}
\bigskip
\vfill

\clearpage

\footnotesize

\lohead{\textsc{register}}

% Definiere theindex-Environment komplett neu ohne reledmac
\makeatletter
\renewenvironment{theindex}{%
  \section*{\indexname}%
  \setlength{\parindent}{0pt}%
  \setlength{\parskip}{0pt plus 0.3pt}%
  \let\item\@idxitem
}{%
  \clearpage
}
\makeatother

\IfFileExists{\jobname-pw.ind}{\input{\jobname-pw.ind}}{}

\end{document}

      