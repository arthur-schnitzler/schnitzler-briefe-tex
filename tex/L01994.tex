%% latex-korrekturansicht-vorspann.tex
%% Vorspann für die Korrekturansicht.
%% Lädt die gemeinsame Datei latex-vorspann.tex mit gesetztem Schalter.

\newif\ifkorrekturansicht
\korrekturansichttrue

\input{../tex-inputs/latex-vorspann}


               \section[Arthur Schnitzler an Albert Ehrenstein, 25. 12. 1910]{ Arthur Schnitzler an Albert Ehrenstein, 25. 12. 1910}\nopagebreak\mylabel{v}\rehead{ }\normalsize\beginnumbering\briefempfaengerindex{Ehrenstein, Albert@\textsc{Ehrenstein, Albert}!zzzSchnitzler, Arthur@\emph{von Arthur Schnitzler}!1910-12-251@{25. 12. 1910}|(be} \toendnotes[C]{\smallbreak\pagebreak[2]} \Standort{Jerusalem, The National Library of Israel, ARC. Ms. Var. 306 1 118.}
\physDesc{Bildpostkarte
\newline{}Handschrift: schwarze Tinte, deutsche Kurrent\newline{}Versand: Stempel: »\nobreak{}Wien, 25. XII. 10, 4–5\nobreak{}«.  \newline{}Ordnung: mit Bleistift von unbekannter Hand nummeriert:
                                    »11« }\toendnotes[C]{\smallbreak}\pstart{}{\pb}Hrn Dr. \textsc{Albert}\pend{}\pstart{}\textsc{Ehrenstein}\pend{}\pstart{}\textsc{\textcolor{pink}{Wien XIV}{}\ledrightnote{\textcolor{pink}{XIV., Penzing}}}\pend{}\pstart{}\textsc{\textcolor{pink}{Ottakringer Hptstr 114}{}\ledrightnote{\textcolor{pink}{Ottakringerstraße}}.}\pend{}{\bigskip}\pstart
           \noindent{}\centering{}{\pb}\textcolor{gray}{\textbf{Wasserfall im \textcolor{pink}{Türkenschanzpark.\hspace*{1.5em}Wien
                     XVIII/I}{}\ledrightnote{\textcolor{pink}{Türkenschanzpark}}.}}\pend
           \pstart
           {\pb}\label{K_L01994_1v}\edtext{Gratulire}{\lemma{\textnormal{\emph{Gratulire}}}\Cendnote{\textnormal{zur Promotion, die am 21. 12. 1910 stattgefunden
                  hatte}}}\label{K_L01994_1h} herzlichst.\pend
           \pstart \spacefill\mbox{ArthSch}\pend{}\pstart
           25. 12. 910.\pend
           \endnumbering\briefempfaengerindex{Ehrenstein, Albert@\textsc{Ehrenstein, Albert}!zzzSchnitzler, Arthur@\emph{von Arthur Schnitzler}!1910-12-251@{25. 12. 1910}|)be}\mylabel{h}  \normalsize

\doendnotes{C}
\bigskip
\vfill

\clearpage

\footnotesize

\lohead{\textsc{register}}

% Definiere theindex-Environment komplett neu ohne reledmac
\makeatletter
\renewenvironment{theindex}{%
  \section*{\indexname}%
  \setlength{\parindent}{0pt}%
  \setlength{\parskip}{0pt plus 0.3pt}%
  \let\item\@idxitem
}{%
  \clearpage
}
\makeatother

\IfFileExists{\jobname-pw.ind}{\input{\jobname-pw.ind}}{}

\end{document}

      