%% latex-korrekturansicht-vorspann.tex
%% Vorspann für die Korrekturansicht.
%% Lädt die gemeinsame Datei latex-vorspann.tex mit gesetztem Schalter.

\newif\ifkorrekturansicht
\korrekturansichttrue

\input{../tex-inputs/latex-vorspann}


               \section[Auguste Hauschner an Arthur Schnitzler, 2. 2. 1909]{ Auguste Hauschner an Arthur Schnitzler, 2. 2. 1909}\nopagebreak\mylabel{v}\rehead{ }\normalsize\beginnumbering\briefempfaengerindex{Schnitzler, Arthur@\textsc{Schnitzler, Arthur}!zzzHauschner, Auguste@\emph{von Auguste Hauschner}!1909-02-021@{2. 2. 1909}|(be} \toendnotes[C]{\smallbreak\pagebreak[2]} \Standort{DLA, A:Schnitzler, HS1985.1.3363.}
\physDesc{Brief, 1 Blatt, 2 Seiten
\newline{}Handschrift: schwarze Tinte, lateinische Kurrent
\newline{}Schnitzler: mit Bleistift Vermerk »\textsc{Hauschner}« und
                                       »\textcolor{pink}{Am
                                       Karlsbad 25} }\toendnotes[C]{\smallbreak}\pstart
           \noindent{}{\pb}Das wäre mir freilich eine grosse Freude, geehrter Herr
               Doctor Schnitzler, wenn Sie mich in \textcolor{pink}{Berlin}{}\ledrightnote{\textcolor{pink}{Berlin}} aufsuchen
               und etwas von Ihrem Schaffen mit mir sprechen würden. Und da es doch nicht zum
               Unmöglichen gehört, dass ich das erlebe, so will ich Ihnen sagen, dass ich, leider,
               leider, das Heim, in dem ich seit fast zwanzig Jahren lebe, im April
               verlassen muss, und dann \textcolor{pink}{Am {\pb}Karlsbad 25}{}\ledrightnote{\textcolor{pink}{Am Karlsbad}} wohnen werde.\pend
           \pstart
           Es wäre schön, wenn mir diese Freude durch ein so glückliches geistiges Erlebniss
               heimischer gerecht würde, wie Ihre \label{K_L02588-1v}\edtext{persönliche Bekanntschaft}{\lemma{\textnormal{\emph{persönliche Bekanntschaft}}}\Cendnote{\textnormal{Es dürfte
                  weder zu einem solchen Besuch, noch zu einer persönlichen Bekanntschaft gekommen
                  sein. }}}\label{K_L02588-1h} es für mich wäre.\pend
           \pstart
           Mit verbindlichen Grüssen und vielen Dank für Ihren Brief{\\[\baselineskip]}\spacefill\mbox{Auguste Hauschner}\pend
           \leftskip=0em{}\pstart
           \textcolor{pink}{Berlin}{}\ledrightnote{\textcolor{pink}{Berlin}}{ }2. 2. 09\pend
           \endnumbering\briefempfaengerindex{Schnitzler, Arthur@\textsc{Schnitzler, Arthur}!zzzHauschner, Auguste@\emph{von Auguste Hauschner}!1909-02-021@{2. 2. 1909}|)be}\mylabel{h}  \normalsize

\doendnotes{C}
\bigskip
\vfill

\clearpage

\footnotesize

\lohead{\textsc{register}}

% Definiere theindex-Environment komplett neu ohne reledmac
\makeatletter
\renewenvironment{theindex}{%
  \section*{\indexname}%
  \setlength{\parindent}{0pt}%
  \setlength{\parskip}{0pt plus 0.3pt}%
  \let\item\@idxitem
}{%
  \clearpage
}
\makeatother

\IfFileExists{\jobname-pw.ind}{\input{\jobname-pw.ind}}{}

\end{document}

      