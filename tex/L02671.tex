%% latex-korrekturansicht-vorspann.tex
%% Vorspann für die Korrekturansicht.
%% Lädt die gemeinsame Datei latex-vorspann.tex mit gesetztem Schalter.

\newif\ifkorrekturansicht
\korrekturansichttrue

\input{../tex-inputs/latex-vorspann}


               \section[Paul Goldmann an Arthur Schnitzler, 22. 11. {[}1891{]}]{ Paul Goldmann an Arthur Schnitzler, 22. 11. {[}1891{]}}\nopagebreak\mylabel{v}\rehead{ }\normalsize\beginnumbering\briefempfaengerindex{Schnitzler, Arthur@\textsc{Schnitzler, Arthur}!zzzGoldmann, Paul@\emph{von Paul Goldmann}!1891-11-221@{22. 11. {[}1891{]}}|(be} \toendnotes[C]{\smallbreak\pagebreak[2]} \Standort{DLA, A:Schnitzler, HS.NZ85.1.3162.}
\physDesc{Brief, 1 Blatt, 3 Seiten
\newline{}Handschrift: blaue Tinte, deutsche Kurrent
\newline{}Schnitzler: 1) mit rotem Buntstift eine Unterstreichung 2) mit Bleistift datiert: »9\textcolor{gray}{1}«}\toendnotes[C]{\smallbreak}\pstart
           \noindent{}\centering{}{\pb}\textcolor{gray}{\textbf{Dr. jur. Paul Goldmann}}\pend
           \pstart
           \noindent{}\centering{}\textcolor{gray}{\textbf{\begin{otherlanguage}{french}Correspondant de la »\textcolor{brown}{Gazette de Francfort}{}\ledrightnote{\textcolor{brown}{Frankfurter Zeitung}}«\end{otherlanguage}}}\pend
           \pstart
           \noindent{}\centering{}\textcolor{gray}{\textbf{\textcolor{pink}{\begin{otherlanguage}{french}Bruxelles, 21, rue des Plantes\end{otherlanguage}}{}\ledrightnote{\textcolor{pink}{rue des Plantes}}.}}\pend
           \pstart
           \raggedleft{}\textcolor{pink}{Brüſſel}{}\ledrightnote{\textcolor{pink}{Brüssel}}, 22. November.\pend
           \pstart\center{}Mein lieber Arthur!\pend\pstart
           Im Fluge: vielen, vielen, vielen Dank für den lieben Brief und die heutige Sendung.
               Ich ſchleppe das \label{K_L02671-4v}\edtext{\textcolor{green}{Büchlein}{}\ledrightnote{→\textcolor{green}{Das Märchen. Schauspiel in drei Aufzügen}}}{\lemma{\textnormal{\emph{Büchlein}}}\Cendnote{\textnormal{Es dürfte sich noch nicht um das
                  Bühnenmanuskript von \emph{\textcolor{green}{Das Märchen}} handeln, das
                     \textcolor{blue}{Schnitzler} erst am 5. 12. 1891 geliefert
                  bekam. Wahrscheinlich hatte er eine Abschrift geschickt, die dadurch verfügbar
                  wurde, dass sich das Manuskript in Druck befand.}}}\label{K_L02671-4h} den ganzen Tag mit mir
               herum, getraue mich aber nicht hineinzublicken, weil \strikeout{h\textcolor{gray}{eut}}{ }heut wieder einmal die \textcolor{pink}{Wien}{}\ledrightnote{\textcolor{pink}{Wien}}-Wunde offen iſt und mir jede Beſchäftigung mit dem, was mir dort lieb
               und theuer iſt, wüthendes Herz- und Heimweh verurſacht. Nächſtens hoffentlich eine
               ausführliche Antwort. Das heutige nur als Thatbeſtandaufnahme meiner Freude und
               meines Dankes{\dotsfour}\pend
           \pstart
           Die Fäden! Die Fäden! In \textcolor{pink}{Paris}{}\ledrightnote{\textcolor{pink}{Paris}} hat die \textcolor{brown}{Frkf. Ztg.}{}\ledrightnote{\textcolor{brown}{Frankfurter Zeitung}} auch{\pb}
               einen neuen \textcolor{blue}{Correſpondenten}{}\ledrightnote{→\textcolor{blue}{Leopold Spitzer}}
               für den finanziellen Theil ernannt, der mein engerer \textcolor{blue}{College}{}\ledrightnote{→\textcolor{blue}{Leopold Spitzer}}{ }\strikeout{\textcolor{gray}{w}} und zugleich ein wenig mein \textcolor{blue}{Mitarbeiter}{}\ledrightnote{→\textcolor{blue}{Leopold Spitzer}} werden ſoll. Weißt Du wer? Dein Freund \textsc{\textcolor{blue}{Spitzer}{}\ledrightnote{\textcolor{blue}{Leopold Spitzer}}}, von dem Du mir erſt kürzlich ſchriebſt, daß er Dich in \textcolor{pink}{Wien}{}\ledrightnote{\textcolor{pink}{Wien}}{ }\label{K_L02671-1v}\edtext{ beſucht }{\lemma{\textnormal{\emph{ beſucht }}}\Cendnote{\textnormal{nicht bekannt}}}\label{K_L02671-1h}{ }\textsc{etc}. Wir werden eine \textsc{Schnitzler}-Gemeinde in \strikeout{\textcolor{gray}{Wi}{ }}\textcolor{pink}{Paris}{}\ledrightnote{\textcolor{pink}{Paris}} begründen. Und von nun an werden die zwei
                  \textcolor{pink}{Pariſ}{}\ledrightnote{\textcolor{pink}{Paris}}er \textcolor{blue}{Correſpondenten}{}\ledrightnote{→\textcolor{blue}{Leopold Spitzer}} eines der größten deutschen \textcolor{brown}{Blätter}{}\ledrightnote{→\textcolor{brown}{Frankfurter Zeitung}}{ }\strikeout{v\textcolor{gray}{on}} mit vereinten Kräften »an Dich glauben«, was gewiß ein ganzes Publicum
               aufwiegt. Kind, das Du biſt, mit Deinen Zweifeln, die doch übrigens für den
               Eingeweihten eine ſo deutliche Beſtätigung Deines Talentes bilden{\dotsfour}\pend
           \pstart
           {\pb}Dein nächſtjähriger \label{K_L02671-2v}\edtext{Reiſeplan}{\lemma{\textnormal{\emph{Reiſeplan}}}\Cendnote{\textnormal{Schnitzler kam das nächste Mal erst am 12. 4. 1897 nach \textcolor{pink}{Paris}.}}}\label{K_L02671-2h} enthält doch \textcolor{pink}{Paris}{}\ledrightnote{\textcolor{pink}{Paris}}? Ich
               halte das übrigens für ſo ſelbſtverſtändlich, daß ich gar nicht danach frage. Ich
               ſehe nur eine Schwierigkeit: nämlich daß ich bis zu Deiner Ankunft nicht etwa bereits
               wieder entlaſſen bin.\pend
           \pstart
           Das gehört übrigens Alles bereits in den nächſten großen Brief. Gott grüße Dich, mein
               lieber Alter!\pend
           \pstart
           Dein {\\[\baselineskip]}treuer {\\[\baselineskip]}\spacefill\mbox{Paul.}\pend
           \leftskip=0em{}\pstart
           \noindent{}Grüße an {\dots}{ }
                  Du weißt
                     ſchon{\dots}\pend
           \endnumbering\briefempfaengerindex{Schnitzler, Arthur@\textsc{Schnitzler, Arthur}!zzzGoldmann, Paul@\emph{von Paul Goldmann}!1891-11-221@{22. 11. {[}1891{]}}|)be}\mylabel{h}  \normalsize

\doendnotes{C}
\bigskip
\vfill

\clearpage

\footnotesize

\lohead{\textsc{register}}

% Definiere theindex-Environment komplett neu ohne reledmac
\makeatletter
\renewenvironment{theindex}{%
  \section*{\indexname}%
  \setlength{\parindent}{0pt}%
  \setlength{\parskip}{0pt plus 0.3pt}%
  \let\item\@idxitem
}{%
  \clearpage
}
\makeatother

\IfFileExists{\jobname-pw.ind}{\input{\jobname-pw.ind}}{}

\end{document}

      