%% latex-korrekturansicht-vorspann.tex
%% Vorspann für die Korrekturansicht.
%% Lädt die gemeinsame Datei latex-vorspann.tex mit gesetztem Schalter.

\newif\ifkorrekturansicht
\korrekturansichttrue

\input{../tex-inputs/latex-vorspann}


               \section[Arthur Schnitzler an Hugo von Hofmannsthal, 17. 7. 1900]{ Arthur Schnitzler an Hugo von Hofmannsthal, 17. 7. 1900}\nopagebreak\mylabel{v}\rehead{ }\normalsize\beginnumbering\briefempfaengerindex{Hofmannsthal, Hugo von@\textsc{Hofmannsthal, Hugo von}!zzzSchnitzler, Arthur@\emph{von Arthur Schnitzler}!1900-07-171@{17. 7. 1900}|(be} \toendnotes[C]{\smallbreak\pagebreak[2]} \Standort{FDH, Hs-30885,93.}
\physDesc{Brief, 2 Blätter, 8 Seiten
\newline{}Handschrift: schwarze Tinte, deutsche Kurrent\newline{}Ordnung: mit Bleistift von Schnitzler mutmaßlich bei der
                                 Durchsicht der Korrespondenz 1929 das zweite Blatt datiert: »17/7 900« }\buchAbdrucke{\weitereDrucke{1) Hugo von Hofmannsthal, Arthur Schnitzler: \emph{Briefwechsel}. Hg. Therese Nickl und Heinrich Schnitzler. Frankfurt am Main: \emph{S. Fischer} 1964, S. 141.} \weitereDrucke{2) Arthur Schnitzler: \emph{Briefe 1875–1912}. Hg. Therese Nickl und Heinrich Schnitzler. Frankfurt am Main: \emph{S. Fischer} 1981, S. 387–388.} }\toendnotes[C]{\smallbreak}\pstart
           \raggedleft{}{\pb}\textcolor{pink}{Reichenau b/Payerbach{\\}Curhaus}{}\ledrightnote{\textcolor{pink}{Kurhaus Rudolfsbad}}.
                     17. 7. 900. \pend
           \pstart
           mein lieber Hugo, wenn Sie dieſen Brief beko{\geminationm}en, ſind Sie ſchon wieder zurück von Ihrem kleinen
               Ausflug und haben hoffentlich \introOben{}alle\introOben{} Verdroſſenheit verloren.
                  \uline{Ich} wüßte wirklich nicht, was ich jetzt ohne
               Arbeit beginnen würde. Komme ich durch äußere Umſtände, unruhige Verhältniſſe durch
               einige Tage nicht dazu, wenigſtens ein paar kurze Stunden zu ſchreiben, ſo verſinke
               ich in eine wahre Schwermuth. Hier bin ich nun im ganzen {\pb}gut dran. Ob viel dabei herausko{\geminationm}en wird, bei dem
               nämlich was ich jetzt ſchreibe, iſt ja noch nicht ſicher, aber das weſentliche liegt
               ja wo anders. Nachher gibts ja beinah nur Aerger, ob einem was gelungen iſt oder
               nicht. Ich habe hier ein kleines \textcolor{green}{Luſtſpiel}{}\ledrightnote{→\textcolor{green}{Die Quellen des Nil}} neu geſchrieben (deſſen erſte Faſſung \label{K_L01057_1v}\edtext{vor 2 Jahren in \textcolor{pink}{Tegernſee}{}\ledrightnote{\textcolor{pink}{Tegernsee}}}{\lemma{\textnormal{\emph{vor … Tegernſee}}}\Cendnote{\textnormal{siehe A. S.: \emph{Tagebuch}, 2. 8. 1898}}}\label{K_L01057_1h} unter glücklichern Umſtänden entſtan\textcolor{gray}{d}) und bin jetzt mit
               einer ziemlich ſonderbaren \textcolor{green}{Novelle}{}\ledrightnote{→\textcolor{green}{Lieutenant Gustl. Novelle}} beſchäftigt, die mir viele Freude macht. Von dieſer {\pb}hoff ich zuverſichtlich, daſs ſie auch Ihnen andern
               Freude machen wird. Meine große \textcolor{green}{Novelle}{}\ledrightnote{→\textcolor{green}{Frau Bertha Garlan. Roman}} hab ich der \textcolor{brown}{\textsc{N. Dtsch. Rundschau}}{}\ledrightnote{\textcolor{brown}{Neue Rundschau, Neue Deutsche Rundschau, Freie Bühne}} gegeben; ſie iſt nicht übel ausgefallen; bisher kennen ſie \textcolor{blue}{Salten}{}\ledrightnote{\textcolor{blue}{Felix Salten}} u \textcolor{blue}{Schwarzkopf}{}\ledrightnote{\textcolor{blue}{Gustav Schwarzkopf}}, die
               beide ſehr zufrieden ſcheinen. – Wie lange ich noch hier bleibe weiſs ich nicht
               genau; in etwa 8–10 Tagen dürfte ich jedenfalls in \textcolor{pink}{Wien}{}\ledrightnote{\textcolor{pink}{Wien}}{ }ſein; aber über die erſte Auguſthälfte
               herrſcht noch große Unklarheit. Mitte Auguſt{ }ſoll eine Fußtour bego{\geminationn}en werden, die {\pb}ich in \textcolor{pink}{\textsc{Altaussee}}{}\ledrightnote{\textcolor{pink}{Altaussee}} mit \textcolor{blue}{Richard}{}\ledrightnote{\textcolor{blue}{Richard Beer-Hofmann}} ausgeheckt habe. \textcolor{blue}{Paul Goldmann}{}\ledrightnote{\textcolor{blue}{Paul Goldmann}}, \textcolor{blue}{Kerr}{}\ledrightnote{\textcolor{blue}{Alfred Kerr}}, \textcolor{blue}{Oskar Meyer}{}\ledrightnote{\textcolor{blue}{Oskar Mayer}}{ }ſchließen ſich vielleicht an. Am Ende auch \textcolor{blue}{Georg Hirſchfeld}{}\ledrightnote{\textcolor{blue}{Georg Hirschfeld}} (\textcolor{blue}{Elly}{}\ledrightnote{\textcolor{blue}{Elly Petersen}} dürfte wegen \textcolor{blue}{Kerr}{}\ledrightnote{\textcolor{blue}{Alfred Kerr}} u \textcolor{blue}{Goldmann}{}\ledrightnote{\textcolor{blue}{Paul Goldmann}}{ }ſehr dafür ſein.) –\pend
           \pstart
           Ein paar Stunden täglich plaudere ich mit einer angehenden nicht hübſchen \textcolor{blue}{Schauſpielerin}{}\ledrightnote{→\textcolor{blue}{Olga Schnitzler}}, die für ihre
               18 Jahre von einer unglaublichen Klugheit iſt. Sie wohnt hier mit ihrer \textcolor{blue}{Schweſter}{}\ledrightnote{→\textcolor{blue}{Elisabeth Steinrück}}, die ein \label{K_L01057_2v}\edtext{16jähriges}{\lemma{\textnormal{\emph{16jähriges}}}\Cendnote{\textnormal{Sie war zu dem Zeitpunkt erst 14.}}}\label{K_L01057_2h} keckes aber geſcheidtes
               Judenmädl iſt; ſtets {\pb}iſt auch ein junges blondes \textcolor{blue}{Ding}{}\ledrightnote{→\textcolor{blue}{Bertha Schimitschek}} mit ihnen, die
               wahrſcheinlich verrückt werden wird. Geſtern hab ich mit denen allen in ihrem kleinen
               Garten genachtmahlt. Die \textcolor{blue}{Schauſpielerin}{}\ledrightnote{→\textcolor{blue}{Olga Schnitzler}} hatte Nachmittags die \textcolor{green}{\textsc{Madonna Dianora}}{}\ledrightnote{\textcolor{green}{Die Frau im Fenster}}{ }ſtudirt; der kleinen \textcolor{blue}{Schweſter}{}\ledrightnote{→\textcolor{blue}{Elisabeth Steinrück}} hatte ein 20jähriger \textcolor{blue}{Verehrer}{}\ledrightnote{→\textcolor{blue}{?? [Verehrer von Elisabeth Steinrück]}} »\textcolor{green}{Geſtern}{}\ledrightnote{\textcolor{green}{Gestern. Dramatische Studie in einem Akt in Versen}}« aus \textcolor{pink}{Wien}{}\ledrightnote{\textcolor{pink}{Wien}} mitgebracht. Ich finde den
               Zufall hübſch, der es macht, daſs Sie das gleich erfahren können; nichts beruhigt
               mehr über die Vielheit u Verwirrtheit des Lebens, als we{\geminationn} man Fäden {\pb}irgendwo zuſa{\geminationm}en laufen ſieht. –\pend
           \pstart
           Sonſt hab ich hier noch \textsc{Dr} \textcolor{blue}{\textsc{Redlich}}{}\ledrightnote{\textcolor{blue}{Josef Redlich}} und ſeine \textcolor{blue}{Frau}{}\ledrightnote{→\textcolor{blue}{Alice Leo}} (die \textcolor{pink}{Königsberg}{}\ledrightnote{\textcolor{pink}{Kaliningrad}}erin) geſprochen; meine \textcolor{blue}{Mama}{}\ledrightnote{→\textcolor{blue}{Louise Schnitzler}} u meine \textcolor{blue}{Schweſter}{}\ledrightnote{→\textcolor{blue}{Gisela Hajek}} wohnen hier, \textcolor{blue}{Schwägerin}{}\ledrightnote{→\textcolor{blue}{Helene Schnitzler}} u Familie in \textcolor{pink}{Edlach}{}\ledrightnote{\textcolor{pink}{Edlach}}. Den Vormittg verbu{\geminationm}l ich und
               verſpazier’ ich; nur nach Tiſch arbeite ich. – Wie denken Sie den Reſt des Sommers zu
               verbringen? Es iſt ſehr wahrſcheinlich, dſs ich Anfangs Auguſt in \textcolor{pink}{Iſchl}{}\ledrightnote{\textcolor{pink}{Bad Ischl}}{ }ſein werde; ſollte man ſich nicht {\pb}irgendwo, in \textcolor{pink}{Salzburg}{}\ledrightnote{\textcolor{pink}{Salzburg}}
               z. B. begegnen können? – \textcolor{blue}{Richard}{}\ledrightnote{\textcolor{blue}{Richard Beer-Hofmann}} arbeitet. Als
               ich bei ihm war, befan\textcolor{gray}{d} ſich ſeine \textcolor{blue}{Frau}{}\ledrightnote{→\textcolor{blue}{Paula Beer-Hofmann}} nicht ſehr wohl, doch ſcheint es jetzt viel beſſer
               oder ganz gut zu gehn. Schreiben Sie mir\strikeout{h} recht bald
               wieder, iſts kein Brief, ſo ſei es eine Karte. Aber verlieren wir uns keineswegs,
               auch nicht auf Tage, ganz aus den Augen.\pend
           \pstart
           Ich hoffe Ihr \textcolor{blue}{Papa}{}\ledrightnote{→\textcolor{blue}{Hugo August von Hofmannsthal}} iſt ganz
               geſund. Grüßen Sie ihn, Ihre \textcolor{blue}{Mama}{}\ledrightnote{→\textcolor{blue}{Anna von Hofmannsthal}}, und {\pb}die Familie \textcolor{blue}{Speyer}{}\ledrightnote{\textcolor{blue}{Nanette Speyer}{\newline}\textcolor{blue}{Albert Speyer}} mehr oder weniger.\pend
           \pstart
           Herzlichſt der Ihrige{\\[\baselineskip]}\spacefill\mbox{Arthur.}\pend
           \leftskip=0em{}\pstart
           Benützen Sie nur meine \textcolor{pink}{Wien}{}\ledrightnote{\textcolor{pink}{Wien}}er Adreſſe, das iſt am
               ſicherſten. Ich habe vergeſſen, daſs ich Sie von der \textcolor{blue}{Schauſpielerin}{}\ledrightnote{→\textcolor{blue}{Olga Schnitzler}}{ }ſehr herzlich grüßen ſoll.\pend
           \endnumbering\briefempfaengerindex{Hofmannsthal, Hugo von@\textsc{Hofmannsthal, Hugo von}!zzzSchnitzler, Arthur@\emph{von Arthur Schnitzler}!1900-07-171@{17. 7. 1900}|)be}\mylabel{h}  \normalsize

\doendnotes{C}
\bigskip
\vfill

\clearpage

\footnotesize

\lohead{\textsc{register}}

% Definiere theindex-Environment komplett neu ohne reledmac
\makeatletter
\renewenvironment{theindex}{%
  \section*{\indexname}%
  \setlength{\parindent}{0pt}%
  \setlength{\parskip}{0pt plus 0.3pt}%
  \let\item\@idxitem
}{%
  \clearpage
}
\makeatother

\IfFileExists{\jobname-pw.ind}{\input{\jobname-pw.ind}}{}

\end{document}

      