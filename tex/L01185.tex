%% latex-korrekturansicht-vorspann.tex
%% Vorspann für die Korrekturansicht.
%% Lädt die gemeinsame Datei latex-vorspann.tex mit gesetztem Schalter.

\newif\ifkorrekturansicht
\korrekturansichttrue

\input{../tex-inputs/latex-vorspann}


               \section[Arthur Schnitzler an Hermann Bahr, 28. 10. 1901]{ Arthur Schnitzler an Hermann Bahr, 28. 10. 1901}\nopagebreak\mylabel{v}\rehead{ }\normalsize\beginnumbering\briefempfaengerindex{Bahr, Hermann@\textsc{Bahr, Hermann}!zzzSchnitzler, Arthur@\emph{von Arthur Schnitzler}!1901-10-281@{28. 10. 1901}|(be} \toendnotes[C]{\smallbreak\pagebreak[2]} \Standort{TMW, HS AM 23347 Ba.}
\physDesc{Brief, 1 Blatt, 4 Seiten
\newline{}Handschrift: schwarze Tinte, deutsche Kurrent\newline{}Ordnung: 1) Lochung 2) mit Bleistift von unbekannter Hand datiert: »26 X. 01«}\buchAbdrucke{\weitereDrucke{1) \emph{28. 10. 1901.} In: Arthur Schnitzler: \emph{The Letters of Arthur Schnitzler to Hermann Bahr}. Edited, annotated, and with an introduction, by Donald G.
                        Daviau. Chapel Hill: \emph{The University of North Carolina Press} 1978, S. 72 (University of North Carolina studies in the Germanic languages
                        and literatures, 89).} \weitereDrucke{2) Hermann Bahr, Arthur Schnitzler: \emph{Briefwechsel, Aufzeichnungen, Dokumente (1891–1931)}. Hg. Kurt Ifkovits und Martin Anton Müller. Göttingen: \emph{Wallstein} 2018, S. 217.} }\toendnotes[C]{\smallbreak}\pstart{}{\pb}lieber Hermann,
               \pend\pstart
           aus deinem lieben Brief entnehme ich u. a. dſs \textcolor{blue}{Berger}{}\ledrightnote{\textcolor{blue}{Alfred von Berger}} hier war. \uline{Iſt} er noch in \textcolor{pink}{Wien}{}\ledrightnote{\textcolor{pink}{Wien}}? (Er schrieb mir eine \label{K_L01185_1v}\edtext{Karte}{\lemma{\textnormal{\emph{Karte}}}\Cendnote{\textnormal{»Hochgeehrter Herr Doctor!{ / }Nächste Woche spreche ich Sie in \textcolor{pink}{Wien}. Ich
                        bin von den ›\textcolor{green}{letzten Stunden}‹ entzückt, so
                        entzückt, als die \textcolor{pink}{Hamburg}er darüber empört
                        sein werden. Alles Nähere mündlich.\hspace*{1.5em}Herzlich grüßt{ / }\textcolor{blue}{Alfred v. Berger}{ / }18/10 1901« (gedruckter Kopf: »\textcolor{pink}{Deutsches Schauspielhaus in Hamburg}«, \emph{Cambridge University Library}, Schnitzler, B 10).}}}\label{K_L01185_1h}{ }\introOben{}(aus \textcolor{pink}{Hamburg}{}\ledrightnote{\textcolor{pink}{Hamburg}})\introOben{}, dſs er mich
               perſönlich ſprechen wollte, in Angelegenheit der \textcolor{green}{Stücke}{}\ledrightnote{→\textcolor{green}{Die letzten Masken}{\newline}→\textcolor{green}{Literatur}{\newline}→\textcolor{green}{Die Frau mit dem Dolche}}.) –\pend
           \pstart
           Die \textcolor{green}{Dolchdame}{}\ledrightnote{\textcolor{green}{Die Frau mit dem Dolche}} iſt gewiſs ein ſchweres ſcenisches
                  {\pb}Ding; aber ſo weit
               ſind wir heute doch ſchon in dieſen Sachen, dſs es unbedingt gehen muſs. –\pend
           \pstart
           \textsc{\textcolor{blue}{Bukovics}{}\ledrightnote{\textcolor{blue}{Emerich von Bukovics}}} hat mich neulich mit der Ausſicht entlaſſen, dſs er über die Beſetz nachdenken
               werde. Du haſt ja recht; ich muſs energiſcher mit ihm ſein, aber mir fehlt die rechte
               Begeiſterung für die vorausſichtliche Volks{\pb}theateraufführg. Nun es
               bleibt mir ja nichts andres übrig. Ich werde nächſtens »ſtürmiſch« einen \label{K_L01185_2v}\edtext{Contract mit einer Million Poenale
               verlangen}{\lemma{\textnormal{\emph{Contract … verlangen}}}\Cendnote{\textnormal{Vgl. den Brief \textcolor{blue}{Schnitzler}s an \textcolor{blue}{Emerich von Bukovics}, 11. 12. 1901, in \emph{Briefwechsel} Bahr/Schnitzler 219–220.}}}\label{K_L01185_2h}.\pend
           \pstart
           – Wie man die »\textcolor{green}{Literatur}{}\ledrightnote{\textcolor{green}{Literatur}}« ſo beſonders gut finden
               kann, verſteh ich abſolut nicht; mein \textsc{faible}{ }ſind die »\textcolor{green}{lebendigen
                  Stunden}{}\ledrightnote{\textcolor{green}{Lebendige Stunden}}.«\pend
           \pstart
           \textcolor{blue}{Kainz}{}\ledrightnote{\textcolor{blue}{Josef Kainz}} wollte am 5. den \textcolor{green}{Gustl}{}\ledrightnote{\textcolor{green}{Lieutenant Gustl. Novelle}}{ }{\pb}leſen; aber \introOben{}–\introOben{} Herr \label{K_L01185_3v}\edtext{\textcolor{blue}{Gutmann}{}\ledrightnote{\textcolor{blue}{Albert Gutmann}}}{\lemma{\textnormal{\emph{Gutmann}}}\Cendnote{\textnormal{Betreiber einer Konzertagentur, die im \textcolor{pink}{Bösendorfer-Saal} Veranstaltungen
                  organisierte.}}}\label{K_L01185_3h} hat Angſt gehabt. Ich werde anfangen, die militäriſche
               Verachtg gegen das Civil zu theilen.\pend
           \pstart
           Herzlichſt dein{\\[\baselineskip]}\spacefill\mbox{Arthur}\pend
           \leftskip=0em{}\pstart
           28. X. 901.\pend
           \endnumbering\briefempfaengerindex{Bahr, Hermann@\textsc{Bahr, Hermann}!zzzSchnitzler, Arthur@\emph{von Arthur Schnitzler}!1901-10-281@{28. 10. 1901}|)be}\mylabel{h}  \normalsize

\doendnotes{C}
\bigskip
\vfill

\clearpage

\footnotesize

\lohead{\textsc{register}}

% Definiere theindex-Environment komplett neu ohne reledmac
\makeatletter
\renewenvironment{theindex}{%
  \section*{\indexname}%
  \setlength{\parindent}{0pt}%
  \setlength{\parskip}{0pt plus 0.3pt}%
  \let\item\@idxitem
}{%
  \clearpage
}
\makeatother

\IfFileExists{\jobname-pw.ind}{\input{\jobname-pw.ind}}{}

\end{document}

      