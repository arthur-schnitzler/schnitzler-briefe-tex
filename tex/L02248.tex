%% latex-korrekturansicht-vorspann.tex
%% Vorspann für die Korrekturansicht.
%% Lädt die gemeinsame Datei latex-vorspann.tex mit gesetztem Schalter.

\newif\ifkorrekturansicht
\korrekturansichttrue

\input{../tex-inputs/latex-vorspann}


               \section[Robert Adam an Arthur Schnitzler, 30. 11. 1916]{ Robert Adam an Arthur Schnitzler, 30. 11. 1916}\nopagebreak\mylabel{v}\rehead{ }\normalsize\beginnumbering\briefempfaengerindex{Schnitzler, Arthur@\textsc{Schnitzler, Arthur}!zzzAdam, Robert@\emph{von Robert Adam}!1916-11-301@{30. 11. 1916}|(be} \toendnotes[C]{\smallbreak\pagebreak[2]} \Standort{DLA, A:Schnitzler, HS.NZ85.1.4230,16.}
\physDesc{Brief, 1 Blatt, 2 Seiten
\newline{}Handschrift: schwarze Tinte, deutsche Kurrent
\newline{}Schnitzler: 1) mit Bleistift beschriftet: »\textsc{Adam}« 2) mit rotem Buntstift zwei Unterstreichungen}\pstart
           \raggedleft{}{\pb}\textcolor{pink}{Wien}{}\ledrightnote{\textcolor{pink}{Wien}}, am 30. November 1916\pend
           \pstart{}Hochgeehrter Herr Doktor!\pend\pstart
           Ich mache von Ihrer gütigen Erlaubnis Gebrauch und überſende Ihnen eine Probe der
                    Alexandriner (daß es ſo ſpät geſchieht, bitte ich mit den ſtarken Amtsgeſchäften
                    zu entſchuldigen, die mir in den letzten Tagen keine freie Stunde übrigließen;
                    etwas ausfeilen mußte ich die Verſe ja doch und ſo nahm das Abſchreiben einige
                    Zeit in Anſpruch). Von Knittelverſen habe ich nur eine ganz kurze Probe
                    angefügt; ſie ſind Ihnen ja in der mir geläufi{\pb}gen Art aus meinen früheren Arbeiten bekannt. –\pend
           \pstart
           Mit herzlichſten Grüßen Ihr\pend
           \pstart
           ergebener{\\[\baselineskip]}\spacefill\mbox{Robert Adam}\pend
           \leftskip=0em{}\endnumbering\briefempfaengerindex{Schnitzler, Arthur@\textsc{Schnitzler, Arthur}!zzzAdam, Robert@\emph{von Robert Adam}!1916-11-301@{30. 11. 1916}|)be}\mylabel{h}  \normalsize

\doendnotes{C}
\bigskip
\vfill

\clearpage

\footnotesize

\lohead{\textsc{register}}

% Definiere theindex-Environment komplett neu ohne reledmac
\makeatletter
\renewenvironment{theindex}{%
  \section*{\indexname}%
  \setlength{\parindent}{0pt}%
  \setlength{\parskip}{0pt plus 0.3pt}%
  \let\item\@idxitem
}{%
  \clearpage
}
\makeatother

\IfFileExists{\jobname-pw.ind}{\input{\jobname-pw.ind}}{}

\end{document}

      