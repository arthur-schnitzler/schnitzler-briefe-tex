%% latex-korrekturansicht-vorspann.tex
%% Vorspann für die Korrekturansicht.
%% Lädt die gemeinsame Datei latex-vorspann.tex mit gesetztem Schalter.

\newif\ifkorrekturansicht
\korrekturansichttrue

\input{../tex-inputs/latex-vorspann}


               \section[Hugo von Hofmannsthal an Arthur Schnitzler, 7. 6. 1901]{ Hugo von Hofmannsthal an Arthur Schnitzler, 7. 6. 1901}\nopagebreak\mylabel{v}\rehead{ }\normalsize\beginnumbering\briefempfaengerindex{Schnitzler, Arthur@\textsc{Schnitzler, Arthur}!zzzHofmannsthal, Hugo von@\emph{von Hugo von Hofmannsthal}!1901-06-072@{7. 6. 1901}|(be} \toendnotes[C]{\smallbreak\pagebreak[2]} \Standort{CUL, Schnitzler, B 43.}
\physDesc{Brief, 1 Blatt, 3 Seiten
\newline{}Handschrift: schwarze Tinte, deutsche Kurrent
\newline{}Schnitzler: mit Bleistift die Jahreszahl ergänzt: »901« \newline{}Ordnung: 1) mit Bleistift von unbekannter Hand nummeriert: »\strikeout{1\textcolor{gray}{8}3}« 2) mit Bleistift von unbekannter Hand nummeriert: »173«}\buchAbdrucke{\weitereDrucke{Hugo von Hofmannsthal, Arthur Schnitzler: \emph{Briefwechsel}. Hg. Therese Nickl und Heinrich Schnitzler. Frankfurt am Main: \emph{S. Fischer} 1964, S. 147.} }\pstart
           \raggedleft{}{\pb}7 VI.\pend
           \pstart{}mein lieber Arthur, \pend\pstart
           es iſt ſo lieb von Ihnen, daſs Sie ſchon damals daran gedacht haben, mir etwas
               Schönes zu ſchenken; ich freue mich ſehr damit und freue mich darauf, die ſchöne
               Truhe irgendwo in dem Haus aufzuſtellen.\pend
           \pstart
           {\pb}Es iſt mir wie eine Art Schmerz,
               daſs ich im Beginn eines Sommers nicht die Ausſicht habe, Sie irgendwo
               zu ſehen, hoffentlich wird es im Herbſt{ }ſein. Schreiben Sie nicht zu ſelten, ich meine
               antworten Sie nicht nach zu großen Zwiſchenräumen.\pend
           \pstart
           {\pb}Gott behüte Sie.\pend
           \pstart
           Von Herzen Ihr{\\[\baselineskip]}\spacefill\mbox{Hugo.}\pend
           \leftskip=0em{}\endnumbering\briefempfaengerindex{Schnitzler, Arthur@\textsc{Schnitzler, Arthur}!zzzHofmannsthal, Hugo von@\emph{von Hugo von Hofmannsthal}!1901-06-072@{7. 6. 1901}|)be}\mylabel{h}  \normalsize

\doendnotes{C}
\bigskip
\vfill

\clearpage

\footnotesize

\lohead{\textsc{register}}

% Definiere theindex-Environment komplett neu ohne reledmac
\makeatletter
\renewenvironment{theindex}{%
  \section*{\indexname}%
  \setlength{\parindent}{0pt}%
  \setlength{\parskip}{0pt plus 0.3pt}%
  \let\item\@idxitem
}{%
  \clearpage
}
\makeatother

\IfFileExists{\jobname-pw.ind}{\input{\jobname-pw.ind}}{}

\end{document}

      