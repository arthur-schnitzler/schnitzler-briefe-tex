%% latex-korrekturansicht-vorspann.tex
%% Vorspann für die Korrekturansicht.
%% Lädt die gemeinsame Datei latex-vorspann.tex mit gesetztem Schalter.

\newif\ifkorrekturansicht
\korrekturansichttrue

\input{../tex-inputs/latex-vorspann}


               \section[Adolf Treibl an Arthur Schnitzler, 18. 1. 1906]{ Adolf Treibl an Arthur Schnitzler, 18. 1. 1906}\nopagebreak\mylabel{v}\rehead{ }\normalsize\beginnumbering\briefempfaengerindex{Schnitzler, Arthur@\textsc{Schnitzler, Arthur}!zzzTreibl, Adolf@\emph{von Adolf Treibl}!1906-01-181@{18. 1. 1906}|(be} \toendnotes[C]{\smallbreak\pagebreak[2]} \Standort{DLA, A:Schnitzler, HS.NZ85.1.4815,1.}
\physDesc{Brief, 1 Blatt, 4 Seiten
\newline{}Handschrift: schwarze Tinte, deutsche Kurrent
\newline{}Schnitzler: mit Bleistift beschriftet: »\textsc{Ehrenstein (Treibl}« }\toendnotes[C]{\smallbreak}\pstart
           \noindent{}{\pb}\textsc{Euer Hochwohlgeboren}\pend
           \pstart{}Hochverehrter Herr \textsc{Doctor}.\pend\pstart
           Es iſt halt ein großes \textsc{Kreuz}! Noch einmal appellieren die
                  \textcolor{blue}{Eltern}{}\ledrightnote{→\textcolor{blue}{Charlotte Ehrenstein}{\newline}→\textcolor{blue}{Alexander Ehrenstein}} des erkrankten
                  \textsc{\textcolor{blue}{Albert Ehrenstein}{}\ledrightnote{\textcolor{blue}{Albert Ehrenstein}}} an die Opferwilligkeit von \textsc{Euer Hochwohlgeboren}.
               Bisher haben drei Ärzte: \textsc{D\textsuperscript{r}{ }\textcolor{blue}{Adler}{}\ledrightnote{\textcolor{blue}{Alfred Adler}}}, \textsc{der Hausarzt D\textsuperscript{r}}{ }\label{K_L01574_1v}\edtext{\textsc{\textcolor{blue}{Jellenik}{}\ledrightnote{\textcolor{blue}{Edmund Jelinek}}}}{\lemma{\textnormal{\emph{Jellenik}}}\Cendnote{\textnormal{Ein Arzt mit Namen »Jellenik« ist in \textcolor{pink}{Wien} nicht nachweisbar. Es dürfte sich um \textcolor{blue}{Edmund Jelinek} handeln (vgl. A. S.: \emph{Tagebuch}, 18. 1. 1906).}}}\label{K_L01574_1h} u ein von \textcolor{pink}{Brünn}{}\ledrightnote{\textcolor{pink}{Brünn}} berufener Onkel des Patienten \textsc{D\textsuperscript{r}{ }\textcolor{blue}{Jakob Ehrenstein}{}\ledrightnote{\textcolor{blue}{Jakob Ehrenstein}}}{ }ſich ziemlich einhellig \strikeout{über} für ein Sanatorium aus{\pb}geſprochen.
               Allerdings \strikeout{über} der Grad der Notwendigkeit dieſer
               Verfügung wurde nicht gleichmäßig betont. Der Kranke ſelbſt hält aber an einer Reiſe
               nach \textsc{\textcolor{pink}{Meran}{}\ledrightnote{\textcolor{pink}{Meran}}} feſt, weil Euer Hochwohlgeboren eine ſolche ſeinerzeit empfohlen haben.\pend
           \pstart
           Heute{ }nachmittags (18/I) treten um ¼ 5\textsuperscript{h} noch einmal der \textcolor{blue}{Hausarzt}{}\ledrightnote{→\textcolor{blue}{Edmund Jelinek}} und ein Spezialiſt: \textsc{D\textsuperscript{r}{ }\textcolor{blue}{Kornfeld}{}\ledrightnote{\textcolor{blue}{Sigmund Kornfeld}}} zu einem Konzilium zuſammen. Namens und im Auftrag der \textcolor{blue}{Eltern}{}\ledrightnote{→\textcolor{blue}{Charlotte Ehrenstein}{\newline}→\textcolor{blue}{Alexander Ehrenstein}} erlaube ich mir nun die Bitte, Euer
               Hochwohlgeboren mögen die ganz beſondere Güte haben, {\pb}dieſem Konzilium beizuwohnen und den \textcolor{blue}{Patienten}{}\ledrightnote{→\textcolor{blue}{Albert Ehrenstein}} im Sinne der zu treffenden Maßnahmen
               beeinflußen.\pend
           \pstart
           Euer Hochwohlgeboren können verſichert ſein wir wiſſen die Schwere der Opfer, die in
               dieser \textsc{Affaire} Euer Hochwohlgeboren bringen, wohl zu
               würdigen und es iſt nicht Selbſtſucht oder Rückſichtsloſigkeit, die uns neuerlich an
               Herrn \textsc{Doktor} mit dieſer geradezu anmaßlichen Bitte
               herantreten läßt. Wenn der \textcolor{blue}{Patient}{}\ledrightnote{→\textcolor{blue}{Albert Ehrenstein}} irgend welchen anderen Einflüſſen, als denen die von Euer
               Hochwohlgeboren ausgehen, zugängig wäre, hätten wir es gewiß nicht {\pb}gewagt, neuerlich zu beläſtigen.\pend
           \pstart
           Mit der Bitte, um des leidenden \textcolor{blue}{Menſchen}{}\ledrightnote{→\textcolor{blue}{Albert Ehrenstein}} willen, dem ausgeſprochenen Wunſche zu willfahren verharret in
               vollkommener Hochachtung\pend
           \pstart
           Euer Hochwohlgeboren ganz ergebſter{\\[\baselineskip]}\spacefill\mbox{Ad. Treibl}\pend
           \leftskip=0em{}\pstart
           \noindent{}Adreſſe: \textcolor{blue}{\textsc{Alex Ehrenstein}}{}\ledrightnote{\textcolor{blue}{Alexander Ehrenstein}}\pend
           \pstart
           \textcolor{pink}{Wien XVI}{}\ledrightnote{\textcolor{pink}{XVI., Ottakring}}\pend
           \pstart
           \textcolor{pink}{\textsc{Ottakringerstr} 114}{}\ledrightnote{\textcolor{pink}{Ottakringerstraße}}\pend
           \pstart
           \textcolor{pink}{Wien}{}\ledrightnote{\textcolor{pink}{Wien}}, 18/I 1906\pend
           \endnumbering\briefempfaengerindex{Schnitzler, Arthur@\textsc{Schnitzler, Arthur}!zzzTreibl, Adolf@\emph{von Adolf Treibl}!1906-01-181@{18. 1. 1906}|)be}\mylabel{h}  \normalsize

\doendnotes{C}
\bigskip
\vfill

\clearpage

\footnotesize

\lohead{\textsc{register}}

% Definiere theindex-Environment komplett neu ohne reledmac
\makeatletter
\renewenvironment{theindex}{%
  \section*{\indexname}%
  \setlength{\parindent}{0pt}%
  \setlength{\parskip}{0pt plus 0.3pt}%
  \let\item\@idxitem
}{%
  \clearpage
}
\makeatother

\IfFileExists{\jobname-pw.ind}{\input{\jobname-pw.ind}}{}

\end{document}

      