%% latex-korrekturansicht-vorspann.tex
%% Vorspann für die Korrekturansicht.
%% Lädt die gemeinsame Datei latex-vorspann.tex mit gesetztem Schalter.

\newif\ifkorrekturansicht
\korrekturansichttrue

\input{../tex-inputs/latex-vorspann}


               \section[Arthur Schnitzler an Richard Beer-Hofmann, 30. 9. 1899]{ Arthur Schnitzler an Richard Beer-Hofmann, 30. 9. 1899}\nopagebreak\mylabel{v}\rehead{ }\normalsize\beginnumbering\briefempfaengerindex{Beer-Hofmann, Richard@\textsc{Beer-Hofmann, Richard}!zzzSchnitzler, Arthur@\emph{von Arthur Schnitzler}!1899-09-301@{30. 9. 1899}|(be} \toendnotes[C]{\smallbreak\pagebreak[2]} \Standort{CUL, Schnitzler, B 8.}
\physDesc{Bildpostkarte
\newline{}Handschrift: Bleistift, deutsche Kurrent\newline{}Versand: 1) Stempel: »\nobreak{}30. 9. 99, Bahnpost Cöln (RH) – Niederlahnst – Frankfurt (M) Zug
                                       32\nobreak{}«.  2) Stempel: »\nobreak{}\oindex{Sankt Michael@\textbf{Sankt Michael}, \emph{Bezirk (A.BZK)}|pwk}St. Michael in Eppan, 1 10 99\nobreak{}«. }\toendnotes[C]{\smallbreak}\pstart{}{\pb}Herrn \textsc{Dr. Richard
                     Beer-Hofmann}\pend{}\pstart{}\textcolor{pink}{\textsc{St. Michael in Eppan}.}{}\ledrightnote{\textcolor{pink}{Sankt Michael}}\pend{}{\bigskip}\pstart
           \noindent{}\centering{}\textcolor{gray}{\textbf{{\pb}Gruss vom \textcolor{pink}{National-Denkmal}{}\ledrightnote{\textcolor{pink}{Niederwalddenkmal}}.}}\pend
           \pstart
           \centering{}\textsc{\label{K_L00984_1v}\edtext{Heilô}{\lemma{\textnormal{\emph{Heilô}}}\Cendnote{\textnormal{aus dem Althochdeutschen entnommene Begrüßung, die
                        1899 besonders bei völkisch gesinnten Personen in Mode war;
                     hier satirisch in Zusammenhang mit dem Besuch beim \textcolor{pink}{Niederwalddenkmal} aufzufassen.}}}\label{K_L00984_1h}!}\pend
           \endnumbering\briefempfaengerindex{Beer-Hofmann, Richard@\textsc{Beer-Hofmann, Richard}!zzzSchnitzler, Arthur@\emph{von Arthur Schnitzler}!1899-09-301@{30. 9. 1899}|)be}\mylabel{h}  \normalsize

\doendnotes{C}
\bigskip
\vfill

\clearpage

\footnotesize

\lohead{\textsc{register}}

% Definiere theindex-Environment komplett neu ohne reledmac
\makeatletter
\renewenvironment{theindex}{%
  \section*{\indexname}%
  \setlength{\parindent}{0pt}%
  \setlength{\parskip}{0pt plus 0.3pt}%
  \let\item\@idxitem
}{%
  \clearpage
}
\makeatother

\IfFileExists{\jobname-pw.ind}{\input{\jobname-pw.ind}}{}

\end{document}

      