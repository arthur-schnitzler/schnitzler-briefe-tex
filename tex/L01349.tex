%% latex-korrekturansicht-vorspann.tex
%% Vorspann für die Korrekturansicht.
%% Lädt die gemeinsame Datei latex-vorspann.tex mit gesetztem Schalter.

\newif\ifkorrekturansicht
\korrekturansichttrue

\input{../tex-inputs/latex-vorspann}


               \section[Hugo von Hofmannsthal an Arthur Schnitzler, 11. 12. {[}1903?{]}]{ Hugo von Hofmannsthal an Arthur Schnitzler, 11. 12. {[}1903?{]}}\nopagebreak\mylabel{v}\rehead{ }\normalsize\beginnumbering\briefempfaengerindex{Schnitzler, Arthur@\textsc{Schnitzler, Arthur}!zzzHofmannsthal, Hugo von@\emph{von Hugo von Hofmannsthal}!1903-12-111@{11. 12. {[}1903?{]}}|(be} \toendnotes[C]{\smallbreak\pagebreak[2]} \Standort{CUL, Schnitzler, B 43.}
\physDesc{Briefkarte
\newline{}Handschrift: schwarze Tinte, deutsche Kurrent\newline{}Ordnung: 1) mit Bleistift von unbekannter Hand nummeriert: »\strikeout{221}« 2) mit Bleistift von unbekannter Hand nummeriert:
                                    »207«}\buchAbdrucke{\weitereDrucke{Hugo von Hofmannsthal, Arthur Schnitzler: \emph{Briefwechsel}. Hg. Therese Nickl und Heinrich Schnitzler. Frankfurt am Main: \emph{S. Fischer} 1964, S. 181.} }\toendnotes[C]{\smallbreak}\pstart
           \raggedleft{}{\pb}11. XII.\pend
           \pstart
           Das \textcolor{green}{Buch}{}\ledrightnote{→\textcolor{green}{Das gerettete Venedig. Trauerspiel in fünf Aufzügen}} gehört dem
               Arthur, das Meſſer der \textcolor{blue}{Olga}{}\ledrightnote{\textcolor{blue}{Olga Schnitzler}}. Hie und da darf aber
               auch die \textcolor{blue}{Olga}{}\ledrightnote{\textcolor{blue}{Olga Schnitzler}} in dem ſchönen \textcolor{green}{Buch}{}\ledrightnote{→\textcolor{green}{Das gerettete Venedig. Trauerspiel in fünf Aufzügen}} leſen, \uline{nie} aber der Arthur {\pb}mit dem ſchönen Meſſer
               aufſchneiden.\pend
           \pstart
           Auf Wiederſehen alſo \label{K_L01349_1v}\edtext{Montag}{\lemma{\textnormal{\emph{Montag}}}\Cendnote{\textnormal{14. 12. 1903}}}\label{K_L01349_1h}{ }abend in \textcolor{pink}{Kuffners \label{K_L01349_2v}\edtext{B.h.}{\lemma{\textnormal{\emph{B.h.}}}\Cendnote{\textnormal{Bierhalle}}}\label{K_L01349_2h}}{}\ledrightnote{\textcolor{pink}{Ottakringer Bräu}}\pend
           \pstart
           \textcolor{blue}{Gerty}{}\ledrightnote{\textcolor{blue}{Gertrude von Hofmannsthal}} wird auch mitkommen. Nicht nach 8\textsuperscript{h}.\pend
           \pstart
           Herzlich{\\[\baselineskip]}\spacefill\mbox{Hugo.}\pend
           \leftskip=0em{}\endnumbering\briefempfaengerindex{Schnitzler, Arthur@\textsc{Schnitzler, Arthur}!zzzHofmannsthal, Hugo von@\emph{von Hugo von Hofmannsthal}!1903-12-111@{11. 12. {[}1903?{]}}|)be}\mylabel{h}  \normalsize

\doendnotes{C}
\bigskip
\vfill

\clearpage

\footnotesize

\lohead{\textsc{register}}

% Definiere theindex-Environment komplett neu ohne reledmac
\makeatletter
\renewenvironment{theindex}{%
  \section*{\indexname}%
  \setlength{\parindent}{0pt}%
  \setlength{\parskip}{0pt plus 0.3pt}%
  \let\item\@idxitem
}{%
  \clearpage
}
\makeatother

\IfFileExists{\jobname-pw.ind}{\input{\jobname-pw.ind}}{}

\end{document}

      