%% latex-korrekturansicht-vorspann.tex
%% Vorspann für die Korrekturansicht.
%% Lädt die gemeinsame Datei latex-vorspann.tex mit gesetztem Schalter.

\newif\ifkorrekturansicht
\korrekturansichttrue

\input{../tex-inputs/latex-vorspann}


               \section[Hugo von Hofmannsthal an Arthur Schnitzler, {[}21. 2. 1913{]}]{ Hugo von Hofmannsthal an Arthur Schnitzler, {[}21. 2. 1913{]}}\nopagebreak\mylabel{v}\rehead{ }\normalsize\beginnumbering\briefempfaengerindex{Schnitzler, Arthur@\textsc{Schnitzler, Arthur}!zzzHofmannsthal, Hugo von@\emph{von Hugo von Hofmannsthal}!1913-02-211@{{[}21. 2. 1913{]}}|(be} \toendnotes[C]{\smallbreak\pagebreak[2]} \Standort{CUL, Schnitzler, B 43.}
\physDesc{Briefkarte
\newline{}Handschrift: schwarze Tinte, deutsche Kurrent
\newline{}Schnitzler: mit Bleistift datiert: »21/2 913« und beschriftet: »\textsc{Hugo}« \newline{}Ordnung: 1) mit Bleistift von unbekannter Hand nummeriert: »\strikeout{334}« 2) mit Bleistift von unbekannter Hand nummeriert: »347«}\buchAbdrucke{\weitereDrucke{Hugo von Hofmannsthal, Arthur Schnitzler: \emph{Briefwechsel}. Hg. Therese Nickl und Heinrich Schnitzler. Frankfurt am Main: \emph{S. Fischer} 1964, S. 272.} }\toendnotes[C]{\smallbreak}\pstart
           \raggedleft{}{\pb}\textcolor{pink}{Rodaun}{}\ledrightnote{\textcolor{pink}{Rodaun}}{ }Freitg\pend
           \pstart{}mein lieber Arthur \pend\pstart
           ganz gewiſs werde ich Montag um ¾ 6 bei Ihnen ſein – weil
               es mir eine der größten und reinſten Freuden iſt, eine neue Ihrer \textcolor{green}{Arbeiten}{}\ledrightnote{→\textcolor{green}{Frau Beate und ihr Sohn. Novelle}} von Ihrer eigenen Stimme zuerſt zu
               hören – und weil ich überhaupt beſtändig {\pb}traurig darüber bin, daſs ich Sie
               ſo wenig ſehe, daſs in dieſem Einander-ſehen gar keine Improviſation möglich iſt, gar
               keine Begegnung, kein Miteinander-ausgehen, ſondern allmählich nur dieſe einzige Form
               des Nachtmahls, faſt ein wenig starr, ſich herausgebildet hat, was vielleicht –
               bedenkt man wie kurz das Leben und wie unerſchöpflich das Individuum iſt – nicht ſo
                  \label{T_L02113_1v}\edtext{ſein müßte
               und ſollte}{\lemma{\textnormal{\emph{ſein müßte
               und ſollte}}}\Cendnote{\textnormal{weiter
                  quer am linken Rand}}}\label{T_L02113_1h}.\pend
           \pstart Von Herzen Ihr\spacefill\mbox{Hugo}\pend{}\endnumbering\briefempfaengerindex{Schnitzler, Arthur@\textsc{Schnitzler, Arthur}!zzzHofmannsthal, Hugo von@\emph{von Hugo von Hofmannsthal}!1913-02-211@{{[}21. 2. 1913{]}}|)be}\mylabel{h}  \normalsize

\doendnotes{C}
\bigskip
\vfill

\clearpage

\footnotesize

\lohead{\textsc{register}}

% Definiere theindex-Environment komplett neu ohne reledmac
\makeatletter
\renewenvironment{theindex}{%
  \section*{\indexname}%
  \setlength{\parindent}{0pt}%
  \setlength{\parskip}{0pt plus 0.3pt}%
  \let\item\@idxitem
}{%
  \clearpage
}
\makeatother

\IfFileExists{\jobname-pw.ind}{\input{\jobname-pw.ind}}{}

\end{document}

      