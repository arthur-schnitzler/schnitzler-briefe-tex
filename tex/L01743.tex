%% latex-korrekturansicht-vorspann.tex
%% Vorspann für die Korrekturansicht.
%% Lädt die gemeinsame Datei latex-vorspann.tex mit gesetztem Schalter.

\newif\ifkorrekturansicht
\korrekturansichttrue

\input{../tex-inputs/latex-vorspann}


               \section[Arthur Schnitzler an Hermann Bahr, 20. 12. 1907]{ Arthur Schnitzler an Hermann Bahr, 20. 12. 1907}\nopagebreak\mylabel{v}\rehead{ }\normalsize\beginnumbering\briefempfaengerindex{Bahr, Hermann@\textsc{Bahr, Hermann}!zzzSchnitzler, Arthur@\emph{von Arthur Schnitzler}!1907-12-201@{20. 12. 1907}|(be} \toendnotes[C]{\smallbreak\pagebreak[2]} \Standort{TMW, HS AM 23389 Ba.}
\physDesc{Brief, 1 Blatt, 3 Seiten
\newline{}Handschrift: schwarze Tinte, deutsche Kurrent\newline{}Ordnung: Lochung }\buchAbdrucke{\weitereDrucke{1) \emph{20. 12. 1907.} In: Arthur Schnitzler: \emph{The Letters of Arthur Schnitzler to Hermann Bahr}. Edited, annotated, and with an introduction, by Donald G.
                        Daviau. Chapel Hill: \emph{The University of North Carolina Press} 1978, S. 100–101 (University of North Carolina studies in the Germanic languages
                        and literatures, 89).} \weitereDrucke{2) Hermann Bahr, Arthur Schnitzler: \emph{Briefwechsel, Aufzeichnungen, Dokumente (1891–1931)}. Hg. Kurt Ifkovits und Martin Anton Müller. Göttingen: \emph{Wallstein} 2018, S. 399–400.} }\toendnotes[C]{\smallbreak}\pstart
           \noindent{}{\pb}\textcolor{gray}{\textbf{Dr. Arthur Schnitzler}}\hfill 20. 12. 907\pend
           \pstart
           \textcolor{gray}{\textbf{\textcolor{pink}{Wien XVIII. Spoettelgasse 7}{}\ledrightnote{\textcolor{pink}{Edmund-Weiß-Gasse}}.}}\pend
           \pstart{}lieber Hermann,\pend\pstart
           ich danke dir herzlich. So ungefähr hab ich mir \textcolor{blue}{\textsc{Reinh}}{}\ledrightnote{\textcolor{blue}{Max Reinhardt}}.s Verhältnis zur \textsc{\textcolor{green}{Beatrice}{}\ledrightnote{\textcolor{green}{Der Schleier der Beatrice. Schauspiel in fünf Akten}}} (u Verfaſſer) vorgeſtellt. Ich werde also mit \label{LL102-1v}\substVorne{}\textsuperscript{V}\substDazwischen{}H\substHinten{}\textcolor{pink}{ebbel}{}\ledrightnote{\textcolor{pink}{Hebbel-Theater}} abſchließen\label{LL102-1h} – und darf wohl
               ausſprechen, daſs der Gedanke du und die \textcolor{blue}{\textsc{Mildenburg}}{}\ledrightnote{\textcolor{blue}{Anna Bahr-Mildenburg}} wollten ſich der \textcolor{blue}{\textsc{Ritscher}}{}\ledrightnote{\textcolor{blue}{Helene Ritscher}} und der \textcolor{green}{\textsc{Beatrice}}{}\ledrightnote{\textcolor{green}{Der Schleier der Beatrice. Schauspiel in fünf Akten}} annehmen, mich höchſt wohlthuend berührt. In den Delirien meiner {\pb}\textcolor{blue}{Frau}{}\ledrightnote{→\textcolor{blue}{Olga Schnitzler}} kam es übrigens öfters vor,
               daſs du und die \textcolor{blue}{\textsc{Mildenburg}}{}\ledrightnote{\textcolor{blue}{Anna Bahr-Mildenburg}} oben auf dem Kaſten ſaßen. Dieser Pl\damage{atz} war Euch reſervirt; die übrigen Geſtalten trieben ſich in tieferen Regionen
               herum. Jetzt ſcherzt man darüber! So gut es \textcolor{blue}{Olga}{}\ledrightnote{\textcolor{blue}{Olga Schnitzler}}
               im ganzen ſchon geht – wir müſſen noch längere Zeit \label{K_L01743_1v}\edtext{contumazirt}{\lemma{\textnormal{\emph{contumazirt}}}\Cendnote{\textnormal{in
                  Quarantäne}}}\label{K_L01743_1h} bleiben. (Unser \textcolor{blue}{Bub}{}\ledrightnote{→\textcolor{blue}{Heinrich Schnitzler}} wohnt ſeit 14 Tagen bei ſeiner \textcolor{blue}{Großmama}{}\ledrightnote{→\textcolor{blue}{Louise Schnitzler}}). Alſo ob ich dich noch vor Deiner
               Abreiſe ſehen werde? Mir wärs natürlich ſehr lieb. (für alle Fälle ſei’s geſagt: ich
               bin ſorg{\pb}fältig desinfizirt eh
               ich Briefe ſchreibe)\pend
           \pstart
           Vielleicht haſt du Zeit mir, wenigſtens in ein paar Zeilen etwas über dich zu ſagen;
               ich weiſs so gut wie nichts von dir. – \pend
           \pstart
           Herzlichſt grüßt dich (u meine \textcolor{blue}{Frau}{}\ledrightnote{→\textcolor{blue}{Olga Schnitzler}} thut desgleichen){\\[\baselineskip]}dein{\\[\baselineskip]}\spacefill\mbox{Arthur}\pend
           \leftskip=0em{}\endnumbering\briefempfaengerindex{Bahr, Hermann@\textsc{Bahr, Hermann}!zzzSchnitzler, Arthur@\emph{von Arthur Schnitzler}!1907-12-201@{20. 12. 1907}|)be}\mylabel{h}  \normalsize

\doendnotes{C}
\bigskip
\vfill

\clearpage

\footnotesize

\lohead{\textsc{register}}

% Definiere theindex-Environment komplett neu ohne reledmac
\makeatletter
\renewenvironment{theindex}{%
  \section*{\indexname}%
  \setlength{\parindent}{0pt}%
  \setlength{\parskip}{0pt plus 0.3pt}%
  \let\item\@idxitem
}{%
  \clearpage
}
\makeatother

\IfFileExists{\jobname-pw.ind}{\input{\jobname-pw.ind}}{}

\end{document}

      