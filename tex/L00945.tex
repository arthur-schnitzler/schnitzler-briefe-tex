%% latex-korrekturansicht-vorspann.tex
%% Vorspann für die Korrekturansicht.
%% Lädt die gemeinsame Datei latex-vorspann.tex mit gesetztem Schalter.

\newif\ifkorrekturansicht
\korrekturansichttrue

\input{../tex-inputs/latex-vorspann}


               \section[Arthur Schnitzler an Richard Beer-Hofmann, 16. 7. 1899]{ Arthur Schnitzler an Richard Beer-Hofmann, 16. 7. 1899}\nopagebreak\mylabel{v}\rehead{ }\normalsize\beginnumbering\briefempfaengerindex{Beer-Hofmann, Richard@\textsc{Beer-Hofmann, Richard}!zzzSchnitzler, Arthur@\emph{von Arthur Schnitzler}!1899-07-161@{16. 7. 1899}|(be} \toendnotes[C]{\smallbreak\pagebreak[2]} \Standort{YCGL, MSS 31.}
\physDesc{Brief, 1 Blatt, 4 Seiten, Umschlag
\newline{}Handschrift: Bleistift, deutsche Kurrent\newline{}Versand: 1) Stempel: »\nobreak{}Wien, 16. 7. {[}1899{]}, 5–6N\nobreak{}«.  2) Stempel: »\nobreak{}\oindex{Seeboden@\textbf{Seeboden}, \emph{http://www.geonames.org/ontologyA.ADM3}|pwk}{\pb}\textcolor{gray}{Seeb}oden, 17. 7. 99\nobreak{}«. }\buchAbdrucke{\weitereDrucke{Arthur Schnitzler, Richard Beer-Hofmann: \emph{Briefwechsel 1891–1931}. Hg. Konstanze Fliedl. Wien, Zürich: \emph{Europaverlag} 1992, S. 132–133.} }\pstart{}{\pb}\textsc{Herrn Dr Rich Beer-Hofmann}\pend{}\pstart{}\textcolor{pink}{\textsc{Seeboden am Millstätter}ſee}{}\ledrightnote{\textcolor{pink}{Seeboden}}\pend{}\pstart{}\textcolor{pink}{\textsc{Villa Platzer}}{}\ledrightnote{\textcolor{pink}{Villa Platzer}}\pend{}\pstart{}\textcolor{pink}{\textsc{Kärnthen}}{}\ledrightnote{\textcolor{pink}{Kärnten}}\pend{}{\bigskip}\pstart
           \raggedleft{}{\pb}16/7 99\pend
           \pstart
           Lieber Richard, ich will Dinſtg{ }Früh in \textcolor{pink}{\textsc{Velden}}{}\ledrightnote{\textcolor{pink}{Velden}}, \textcolor{pink}{\textsc{Pension Pundschu}}{}\ledrightnote{\textcolor{pink}{Pension Pundschu}} eintreffen. Schreiben Sie mir dann, wa{\geminationn} Sie zu mir oder ich zu Ihnen ko{\geminationm}en ſoll. {\pb}Wollen Sie
               früher mit Ihrer Arbeit fertig ſein, ſo ſchreiben Sie mir eben, wann Sie fertig
               sind.\pend
           \pstart
           \textcolor{pink}{\textsc{Bayreuth}}{}\ledrightnote{\textcolor{pink}{Bayreuth}} wird kaum {\pb}was zu bekommen ſein.\pend
           \pstart
           Bin ich Ende Juli{ }ſchon in jener Gegend, ſo ko{\geminationm} ich kaum mehr nach \textcolor{pink}{Kärnthen}{}\ledrightnote{\textcolor{pink}{Kärnten}}, \textsc{resp}. \textcolor{pink}{Tirol}{}\ledrightnote{\textcolor{pink}{Tirol}} zurück. Im übrigen all das läßt sich mündlich {\pb}beſſer beſprechen.\pend
           \pstart Herzlich Ihr \spacefill\mbox{Arth}\pend{}\pstart
           \textcolor{blue}{\textsc{Wasserma{\geminationn}}}{}\ledrightnote{\textcolor{blue}{Jakob Wassermann}} kommt Mittwoch nach \textcolor{pink}{Velden}{}\ledrightnote{\textcolor{pink}{Velden}}.\pend
           \endnumbering\briefempfaengerindex{Beer-Hofmann, Richard@\textsc{Beer-Hofmann, Richard}!zzzSchnitzler, Arthur@\emph{von Arthur Schnitzler}!1899-07-161@{16. 7. 1899}|)be}\mylabel{h}  \normalsize

\doendnotes{C}
\bigskip
\vfill

\clearpage

\footnotesize

\lohead{\textsc{register}}

% Definiere theindex-Environment komplett neu ohne reledmac
\makeatletter
\renewenvironment{theindex}{%
  \section*{\indexname}%
  \setlength{\parindent}{0pt}%
  \setlength{\parskip}{0pt plus 0.3pt}%
  \let\item\@idxitem
}{%
  \clearpage
}
\makeatother

\IfFileExists{\jobname-pw.ind}{\input{\jobname-pw.ind}}{}

\end{document}

      