%% latex-korrekturansicht-vorspann.tex
%% Vorspann für die Korrekturansicht.
%% Lädt die gemeinsame Datei latex-vorspann.tex mit gesetztem Schalter.

\newif\ifkorrekturansicht
\korrekturansichttrue

\input{../tex-inputs/latex-vorspann}


               \section[Max Mell an Arthur Schnitzler, 14. 5. 1912]{ Max Mell an Arthur Schnitzler, 14. 5. 1912}\nopagebreak\mylabel{v}\rehead{ }\normalsize\beginnumbering\briefempfaengerindex{Schnitzler, Arthur@\textsc{Schnitzler, Arthur}!zzzMell, Max@\emph{von Max Mell}!1912-05-142@{14. 5. 1912}|(be} \toendnotes[C]{\smallbreak\pagebreak[2]} \Standort{DLA, A:Schnitzler, HS.NZ85.1.5556.}
\physDesc{Brief, 1 Blatt (Briefpapier mit Trauerrand), 1 Seite
\newline{}Handschrift: schwarze Tinte, deutsche Kurrent
\newline{}Schnitzler: 1) mit rotem Buntstift ein Strich etwas versetzt zur Datumsangabe 2) mit Bleistift die Absenderadresse unterhalb des Brieftexts: »\textsc{\textcolor{pink}{II. Wittelsbachg. 5}.}«}\pstart
           \raggedleft{}{\pb}\textcolor{pink}{Wien}{}\ledrightnote{\textcolor{pink}{Wien}}, 14. Mai 1912.\pend
           \pstart{}Sehr verehrter Herr Doktor!\pend\pstart
           Das ſchöne Feſt, das Sie heute begehn, ſcheint mir eine ſchickliche Gelegenheit, Ihnen
               dankbar zu bekennen, daß ich mich vor dem Phänomen Ihres Werkes immer berührt,
               forſchend, ſtudierend, erkennend, bewundernd ſtehen fühle. Ich ſage das, weil ich
               meine, geiſtigen Beſitz zu geben, das iſt ja das, weshalb man ſchafft, und was die
               Freude an dem erledigten, innerlich abgelöſten Werk noch immer weiter fortzuſsetzen
               vermag. Ich fühle mich Ihnen tief verpflichtet und darf, in Erinnerung vieler
               Freundlichkeit, die Sie mir erwieſen, zu dieſen Worten vielleicht noch meine
               herzlichen Wünſche für heute und immer hinzufügen:\pend
           \pstart
           als Ihr{\\[\baselineskip]}\spacefill\mbox{Max Mell.}\pend
           \leftskip=0em{}\endnumbering\briefempfaengerindex{Schnitzler, Arthur@\textsc{Schnitzler, Arthur}!zzzMell, Max@\emph{von Max Mell}!1912-05-142@{14. 5. 1912}|)be}\mylabel{h}  \normalsize

\doendnotes{C}
\bigskip
\vfill

\clearpage

\footnotesize

\lohead{\textsc{register}}

% Definiere theindex-Environment komplett neu ohne reledmac
\makeatletter
\renewenvironment{theindex}{%
  \section*{\indexname}%
  \setlength{\parindent}{0pt}%
  \setlength{\parskip}{0pt plus 0.3pt}%
  \let\item\@idxitem
}{%
  \clearpage
}
\makeatother

\IfFileExists{\jobname-pw.ind}{\input{\jobname-pw.ind}}{}

\end{document}

      