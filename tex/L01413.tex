%% latex-korrekturansicht-vorspann.tex
%% Vorspann für die Korrekturansicht.
%% Lädt die gemeinsame Datei latex-vorspann.tex mit gesetztem Schalter.

\newif\ifkorrekturansicht
\korrekturansichttrue

\input{../tex-inputs/latex-vorspann}


               \section[Richard Beer-Hofmann an Arthur Schnitzler, 1. 7. 1904]{ Richard Beer-Hofmann an Arthur Schnitzler, 1. 7. 1904}\nopagebreak\mylabel{v}\rehead{ }\normalsize\beginnumbering\briefempfaengerindex{Schnitzler, Arthur@\textsc{Schnitzler, Arthur}!zzzBeer-Hofmann, Richard@\emph{von Richard Beer-Hofmann}!1904-07-011@{1. 7. 1904}|(be} \toendnotes[C]{\smallbreak\pagebreak[2]} \Standort{CUL, Schnitzler, B 8.}
\physDesc{Bildpostkarte
\newline{}Handschrift: schwarze Tinte, lateinische Kurrent\newline{}Versand: 1) Stempel: »\nobreak{}\oindex{Bad Aussee@\textbf{Bad Aussee}, \emph{Besiedelter Ort (A.BSO)}|pwk}Auss\textcolor{gray}{ee} in
                                       Steiermark, 1 7 {[}04{]}\nobreak{}«.  2) Stempel: »\nobreak{}\oindex{XVIII., Waehring@\textbf{XVIII., Währing}, \emph{Bezirk (A.BZK)}|pwk}18/1 Wien 110, 2. 7. 04, 10.V, Bestellt\nobreak{}«. \newline{}Ordnung: mit Bleistift von unbekannter Hand nummeriert:
                                    »183« }\buchAbdrucke{\weitereDrucke{Arthur Schnitzler, Richard Beer-Hofmann: \emph{Briefwechsel 1891–1931}. Hg. Konstanze Fliedl. Wien, Zürich: \emph{Europaverlag} 1992, S. 164.} }\toendnotes[C]{\smallbreak}\pstart{}{\pb}Herrn\pend{}\pstart{}Arthur Schnitzler\pend{}\pstart{}\textcolor{pink}{Wien}{}\ledrightnote{\textcolor{pink}{Wien}}\pend{}\pstart{}\textcolor{pink}{XVIII. Spöttelgasse 7}{}\ledrightnote{\textcolor{pink}{Edmund-Weiß-Gasse}}.\pend{}{\bigskip}\pstart
           \noindent{}\centering{}{\pb}\textcolor{gray}{\textbf{\textcolor{pink}{Aussee}{}\ledrightnote{\textcolor{pink}{Bad Aussee}} von \textcolor{pink}{Sixleithen}{}\ledrightnote{\textcolor{pink}{Sixleitengasse}}.}}\pend
           \pstart
           \raggedleft{}1/VII 04\pend
           \pstart
           Herzliche Grüße! Der arme Baron \textcolor{green}{L.}{}\ledrightnote{→\textcolor{green}{Das Schicksal des Freiherrn von Leisenbohg. Novellette}}! \textcolor{green}{Sigurd}{}\ledrightnote{→\textcolor{green}{Das Schicksal des Freiherrn von Leisenbohg. Novellette}} hat auf
                  »\label{K_L01413_1v}\edtext{Schlag treffen}{\lemma{\textnormal{\emph{Schlag treffen}}}\Cendnote{\textnormal{Eine Eröffnung \textcolor{green}{Sigurd}s bewirkt in \emph{\textcolor{green}{Das
                     Schicksal des Freiherrn von Leisenbohg}}, dass sein Konkurrent
                     \textcolor{green}{Leisenbohg} einen Herzinfarkt erleidet. \textcolor{blue}{Beer-Hofmann} sagt, dass dies \textcolor{green}{Sigurd} mit Absicht tat.}}}\label{K_L01413_1h} gespielt«! Und werden Sie
               gesund.\pend
           \pstart Ihr \spacefill\mbox{Richard}\pend{}\pstart
           \noindent{}\label{T_L01413_1v}\edtext{unsere Wohnung}{\lemma{\textnormal{\emph{unsere Wohnung}}}\Cendnote{\textnormal{Verweis auf Markierung im Bild}}}\label{T_L01413_1h}\pend
           \endnumbering\briefempfaengerindex{Schnitzler, Arthur@\textsc{Schnitzler, Arthur}!zzzBeer-Hofmann, Richard@\emph{von Richard Beer-Hofmann}!1904-07-011@{1. 7. 1904}|)be}\mylabel{h}  \normalsize

\doendnotes{C}
\bigskip
\vfill

\clearpage

\footnotesize

\lohead{\textsc{register}}

% Definiere theindex-Environment komplett neu ohne reledmac
\makeatletter
\renewenvironment{theindex}{%
  \section*{\indexname}%
  \setlength{\parindent}{0pt}%
  \setlength{\parskip}{0pt plus 0.3pt}%
  \let\item\@idxitem
}{%
  \clearpage
}
\makeatother

\IfFileExists{\jobname-pw.ind}{\input{\jobname-pw.ind}}{}

\end{document}

      