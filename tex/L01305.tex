%% latex-korrekturansicht-vorspann.tex
%% Vorspann für die Korrekturansicht.
%% Lädt die gemeinsame Datei latex-vorspann.tex mit gesetztem Schalter.

\newif\ifkorrekturansicht
\korrekturansichttrue

\input{../tex-inputs/latex-vorspann}


               \section[Arthur Schnitzler an Richard Beer-Hofmann, 28. 7. 1903]{ Arthur Schnitzler an Richard Beer-Hofmann, 28. 7. 1903}\nopagebreak\mylabel{v}\rehead{ }\normalsize\beginnumbering\briefempfaengerindex{Beer-Hofmann, Richard@\textsc{Beer-Hofmann, Richard}!zzzSchnitzler, Arthur@\emph{von Arthur Schnitzler}!1903-07-281@{28. 7. 1903}|(be} \toendnotes[C]{\smallbreak\pagebreak[2]} \Standort{YCGL, MSS 31.}
\physDesc{Bildpostkarte
\newline{}Handschrift: Bleistift, deutsche Kurrent\newline{}Versand: 1) Stempel: »\nobreak{}\oindex{Hochschneeberg@\textbf{Hochschneeberg}, \emph{Berg (N.BRG)}|pwk}Hochschneeberg, 29. 7. \textcolor{gray}{03}, 2–4N\nobreak{}«.  2) Stempel: »\nobreak{}\oindex{Rodaun@\textbf{Rodaun}, \emph{Teil eines besiedelten Ortes (A.BSOX)}|pwk}{[}Rod{]}\textcolor{gray}{a}un, 30/7 {[}03{]}\nobreak{}«. \newline{}Ordnung: mit Bleistift von unbekannter Hand datiert: »28. 7.« }\toendnotes[C]{\smallbreak}\pstart{}{\pb}Herrn \textsc{Dr Richard
                     Beer-Hofmann}\pend{}\pstart{}\textcolor{pink}{\textsc{Rodaun \textsuperscript{b}/ Wien}}{}\ledrightnote{\textcolor{pink}{Rodaun}}\pend{}\pstart{}\textcolor{pink}{\textsc{Liesinger Hptstr 2}}{}\ledrightnote{\textcolor{pink}{Liesingerstraße}}.\pend{}{\bigskip}\pstart
           \noindent{}\centering{}{\pb}\textcolor{gray}{\textbf{\textcolor{pink}{Fischerhütte am Schneeberg}{}\ledrightnote{\textcolor{pink}{Fischerhütte}} mit dem \textcolor{pink}{Klosterwappen}{}\ledrightnote{\textcolor{pink}{Klosterwappen}} 2075 Mtr.}}\pend
           \pstart
           \raggedleft{}\label{K_L01305_1v}\edtext{28. 7. 903}{\lemma{\textnormal{\emph{28. 7. 903}}}\Cendnote{\textnormal{Das Besondere an dieser Karte ist,
                        dass der Empfänger unmittelbar nachdem sie abgeschickt wurde, ebenfalls am
                        Berg erschien und über Nacht blieb.}}}\label{K_L01305_1h}.\pend
           \pstart
           Herzliche Grüße!\pend
           \pstart
           Ihr{\\[\baselineskip]}\spacefill\mbox{A.}\pend
           \leftskip=0em{}\endnumbering\briefempfaengerindex{Beer-Hofmann, Richard@\textsc{Beer-Hofmann, Richard}!zzzSchnitzler, Arthur@\emph{von Arthur Schnitzler}!1903-07-281@{28. 7. 1903}|)be}\mylabel{h}  \normalsize

\doendnotes{C}
\bigskip
\vfill

\clearpage

\footnotesize

\lohead{\textsc{register}}

% Definiere theindex-Environment komplett neu ohne reledmac
\makeatletter
\renewenvironment{theindex}{%
  \section*{\indexname}%
  \setlength{\parindent}{0pt}%
  \setlength{\parskip}{0pt plus 0.3pt}%
  \let\item\@idxitem
}{%
  \clearpage
}
\makeatother

\IfFileExists{\jobname-pw.ind}{\input{\jobname-pw.ind}}{}

\end{document}

      