%% latex-korrekturansicht-vorspann.tex
%% Vorspann für die Korrekturansicht.
%% Lädt die gemeinsame Datei latex-vorspann.tex mit gesetztem Schalter.

\newif\ifkorrekturansicht
\korrekturansichttrue

\input{../tex-inputs/latex-vorspann}


               \section[Richard Beer-Hofmann an Arthur Schnitzler, {[}8. 11. 1894{]}]{ Richard Beer-Hofmann an Arthur Schnitzler, {[}8. 11. 1894{]}}\nopagebreak\mylabel{v}\rehead{ }\normalsize\beginnumbering\briefempfaengerindex{Schnitzler, Arthur@\textsc{Schnitzler, Arthur}!zzzBeer-Hofmann, Richard@\emph{von Richard Beer-Hofmann}!1894-11-082@{{[}8. 11. 1894{]}}|(be} \toendnotes[C]{\smallbreak\pagebreak[2]} \Standort{CUL, Schnitzler, B 8.}
\physDesc{Brief, 1 Blatt, 1 Seite
\newline{}Handschrift: Bleistift, lateinische Kurrent
\newline{}Schnitzler: mit Bleistift datiert: »8/11 94« und nummeriert »52« }\toendnotes[C]{\smallbreak}\pstart
           \noindent{}{\pb}Lieber! Wenn Sie also
               für \textcolor{green}{morgen}{}\ledrightnote{→\textcolor{green}{Ein Pelikan. Schauspiel in fünf Aufzügen}} noch nichts haben, nehmen Sie bitte
               auch \label{K_L00399_1v}\edtext{nichts für mich}{\lemma{\textnormal{\emph{nichts für mich}}}\Cendnote{\textnormal{Auch \textcolor{blue}{Schnitzler} dürfte die Vorstellung nicht besucht haben.}}}\label{K_L00399_1h}. Ich
               bin voraussichtlich verhindert.\pend
           \pstart Herzlichst\spacefill\mbox{Rich}\pend{}\endnumbering\briefempfaengerindex{Schnitzler, Arthur@\textsc{Schnitzler, Arthur}!zzzBeer-Hofmann, Richard@\emph{von Richard Beer-Hofmann}!1894-11-082@{{[}8. 11. 1894{]}}|)be}\mylabel{h}  \normalsize

\doendnotes{C}
\bigskip
\vfill

\clearpage

\footnotesize

\lohead{\textsc{register}}

% Definiere theindex-Environment komplett neu ohne reledmac
\makeatletter
\renewenvironment{theindex}{%
  \section*{\indexname}%
  \setlength{\parindent}{0pt}%
  \setlength{\parskip}{0pt plus 0.3pt}%
  \let\item\@idxitem
}{%
  \clearpage
}
\makeatother

\IfFileExists{\jobname-pw.ind}{\input{\jobname-pw.ind}}{}

\end{document}

      