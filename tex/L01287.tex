%% latex-korrekturansicht-vorspann.tex
%% Vorspann für die Korrekturansicht.
%% Lädt die gemeinsame Datei latex-vorspann.tex mit gesetztem Schalter.

\newif\ifkorrekturansicht
\korrekturansichttrue

\input{../tex-inputs/latex-vorspann}


               \section[Arthur Schnitzler an Hermann Bahr, 6. 4. 1903]{ Arthur Schnitzler an Hermann Bahr, 6. 4. 1903}\nopagebreak\mylabel{v}\rehead{ }\normalsize\beginnumbering\briefempfaengerindex{Bahr, Hermann@\textsc{Bahr, Hermann}!zzzSchnitzler, Arthur@\emph{von Arthur Schnitzler}!1903-04-061@{6. 4. 1903}|(be} \toendnotes[C]{\smallbreak\pagebreak[2]} \Standort{TMW, HS AM 23354 Ba.}
\physDesc{Brief, 2 Blätter, 7 Seiten
\newline{}Handschrift: schwarze Tinte, deutsche Kurrent\newline{}Ordnung: Lochung }\buchAbdrucke{\weitereDrucke{1) Arthur Schnitzler: \emph{Briefe 1875–1912}. Hg. Therese Nickl und Heinrich Schnitzler. Frankfurt am Main: \emph{S. Fischer} 1981, S. 458–460.} \weitereDrucke{2) \emph{6. 4. 1903.} In: Arthur Schnitzler: \emph{The Letters of Arthur Schnitzler to Hermann Bahr}. Edited, annotated, and with an introduction, by Donald G.
                        Daviau. Chapel Hill: \emph{The University of North Carolina Press} 1978, S. 77–78 (University of North Carolina studies in the Germanic languages
                        and literatures, 89).} \weitereDrucke{3) Hermann Bahr, Arthur Schnitzler: \emph{Briefwechsel, Aufzeichnungen, Dokumente (1891–1931)}. Hg. Kurt Ifkovits und Martin Anton Müller. Göttingen: \emph{Wallstein} 2018, S. 264–265.} }\toendnotes[C]{\smallbreak}\pstart
           \raggedleft{}{\pb}\textcolor{pink}{Wien}{}\ledrightnote{\textcolor{pink}{Wien}}, 6. 4. 903.\pend
           \pstart{}lieber Hermann,\pend\pstart
           ich glaube wir befinden uns beide in einer ſehr ähnlichen Situation der
               Oeffentlichkeit gegenüber: was immer wir thun oder unterlaſſen werden – eine
               compact-vertrackte Majorität wird ſchimpfen. Es wird alſo immer notwendiger find ich
               ſich ausſchließlich nach dem zu richten, was wir ſelbſt für das vernünftige halten –
               auf die Gefahr hin dſs wir uns ge{\pb}legentlich irren.
               Willſt du mir \textcolor{green}{deinen neuen Band}{}\ledrightnote{→\textcolor{green}{Rezensionen. Wiener Theater 1901 bis 1903}}
               widmen, ſo ſeh ich darin nichts andres als den neueſten Ausdruck für die Herzlichkeit
               unsrer Beziehungen, zu der wir uns ja wahrhaftig ſchwer genug durchgerungen haben.
               Ich freu mich nun umſo mehr, daſs wir ſo weit ſind daſs wir einander wirklich
               verſtehen und – was in dieſen Jahren {\pb}doch eigentlich recht
               ſelten vorko{\geminationm}t, uns – ich ſchließe von mir wohl nicht
               ganz verfehlt auf dich – einander jenſeits von Literatur und allerlei Getriebe – gern
               haben. \uline{Ich} für meinen Theil nehme alſo die Gefahr auf
               mich, neuerdings als mit dir vercliquet angeſehen zu werden, \introOben{}–\introOben{} (ob\substVorne{}\textsuperscript{ſ}\substDazwischen{}z\substHinten{}war ich nachweiſen könnte, daſs ich nie eine lobende Kritik über dich
               geſchrieben habe) – und {\pb}mehr als das – ich danke dir aufrichtg für deine liebenswürdg Abſicht. Eine Bitte
               füg ich bei, obwohl ſie recht überflüſſig ſein dürfte: ſage mir nichts »freundliches«
               oder »ſchönes« in deinem \textcolor{green}{Widmungswort}{}\ledrightnote{→\textcolor{green}{Rezensionen. Wiener Theater 1901 bis 1903}}. Die Thatſache der Zu\substVorne{}\textsuperscript{n}\substDazwischen{}ei\substHinten{}gnung allein iſt mir Freude genug.\pend
           \pstart
           Eben erſt merke ich, daſs du mir auf einer Extraſeite den Wortlaut der Widmung schon
               mitgetheilt haſt. Sie iſt einfach und ſchön. Ich danke dir.\pend
           \pstart
           {\pb}Die Nachricht des \textcolor{green}{N. Wr. Journ}{}\ledrightnote{\textcolor{green}{Neues Wiener Journal}} ist unwahr, mindeſtens um ſehr geraume
               Zeit verfrüht. Erinnerſt du dich, dſs wir gerade am Tag vorher mit einem \textcolor{blue}{Herrn}{}\ledrightnote{→\textcolor{blue}{Alfred Deutsch-German}} des \textcolor{brown}{N. Wr. J.}{}\ledrightnote{\textcolor{brown}{Neues Wiener Journal}} über die Büberei geſprochen haben, die \substVorne{}\textsuperscript{die}\substDazwischen{}durch\substHinten{}{ }\substVorne{}\textsuperscript{\textcolor{gray}{den}}\substDazwischen{}die\substHinten{} journaliſtiſchen Einmiſchung ins Privatleben verübt werden? – In meinem Fall
               war es ja zufällig gleichgiltig; aber es hätte ebenſo gut eine freche Indiscretion
               ſein {[}können.{]}\pend
           \pstart
           \damage{–} Wie ſteht es mit deinen \label{K_L01287_1v}\edtext{Reiſe-
               u Erholungsplänen}{\lemma{\textnormal{\emph{Reiſe-
               u Erholungsplänen}}}\Cendnote{\textnormal{\textcolor{blue}{Bahr} hielt sich vom 18. bis
                     25. 5. in der \textcolor{pink}{Kuranstalt Konried in
                     Reichenau an der Rax} auf, \textcolor{blue}{Schnitzler}
                  war zu der Zeit vor allem in \textcolor{pink}{Wien}.}}}\label{K_L01287_1h}? Ich
               hoffe dich {\pb}jedenfalls
               ſehr bald zu ſehen; i{\geminationm}erhin verſtändige mich; denn ich
               möchte we{\geminationn}’s dir nicht unangenehm iſt, auch ganz gern
               ein paar Tage in die \textcolor{pink}{Reichenauer Gegend}{}\ledrightnote{\textcolor{pink}{Reichenau an der Rax}}.\pend
           \pstart
           Zum Cap. \textcolor{green}{Reigen}{}\ledrightnote{\textcolor{green}{Reigen. Zehn Dialoge}}: \textcolor{blue}{Salten}{}\ledrightnote{\textcolor{blue}{Felix Salten}} hat ſein Feuill. vorläufig in der \textcolor{green}{Zeit}{}\ledrightnote{\textcolor{green}{Die Zeit}} auch noch nicht unterbringen können. Warum?{\dotstwo}
               Mein – \textcolor{blue}{Schwager}{}\ledrightnote{→\textcolor{blue}{Markus Hajek}} war entſetzt,
               als er durch \label{K_L01287_2v}\edtext{\textcolor{blue}{Singer}{}\ledrightnote{\textcolor{blue}{Isidor Singer}}}{\lemma{\textnormal{\emph{Singer}}}\Cendnote{\textnormal{\textcolor{blue}{Isidor Singer}, der Herausgeber der
                  Wochenschrift und der gleichnamigen Tageszeitung \emph{\textcolor{brown}{Die Zeit}}.}}}\label{K_L01287_2h} erfuhr, daſs von dieſem verderblichen Buch an
                  her{\pb}vorragender
               Stelle Notiz genommen werden ſolle u rieth ihm dringend ab. \textcolor{blue}{Singer}{}\ledrightnote{\textcolor{blue}{Isidor Singer}}: »Sehn Sie, ſogar der \textcolor{blue}{Schwager}{}\ledrightnote{→\textcolor{blue}{Markus Hajek}}{\dots}«\pend
           \pstart
           Man ernenne doch endlich den Storch zum Ehrenbürger der Menſchheit.\pend
           \pstart
           herzlichen Gruſs{\\[\baselineskip]}dein getreuer{\\[\baselineskip]}\spacefill\mbox{Arthur}\pend
           \leftskip=0em{}\endnumbering\briefempfaengerindex{Bahr, Hermann@\textsc{Bahr, Hermann}!zzzSchnitzler, Arthur@\emph{von Arthur Schnitzler}!1903-04-061@{6. 4. 1903}|)be}\mylabel{h}  \normalsize

\doendnotes{C}
\bigskip
\vfill

\clearpage

\footnotesize

\lohead{\textsc{register}}

% Definiere theindex-Environment komplett neu ohne reledmac
\makeatletter
\renewenvironment{theindex}{%
  \section*{\indexname}%
  \setlength{\parindent}{0pt}%
  \setlength{\parskip}{0pt plus 0.3pt}%
  \let\item\@idxitem
}{%
  \clearpage
}
\makeatother

\IfFileExists{\jobname-pw.ind}{\input{\jobname-pw.ind}}{}

\end{document}

      