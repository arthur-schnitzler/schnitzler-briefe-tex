%% latex-korrekturansicht-vorspann.tex
%% Vorspann für die Korrekturansicht.
%% Lädt die gemeinsame Datei latex-vorspann.tex mit gesetztem Schalter.

\newif\ifkorrekturansicht
\korrekturansichttrue

\input{../tex-inputs/latex-vorspann}


               \section[Hugo Hofmannsthal an Arthur Schnitzler, 16. 1. 1923]{ Hugo Hofmannsthal an Arthur Schnitzler, 16. 1. 1923}\nopagebreak\mylabel{v}\rehead{ }\normalsize\beginnumbering\briefempfaengerindex{Schnitzler, Arthur@\textsc{Schnitzler, Arthur}!zzzHofmannsthal, Hugo von@\emph{von Hugo von Hofmannsthal}!1923-01-161@{16. 1. 1923}|(be} \toendnotes[C]{\smallbreak\pagebreak[2]} \Standort{CUL, Schnitzler, B 43.}
\physDesc{Brief, 1 Blatt, 2 Seiten
\newline{}Handschrift: schwarze Tinte, lateinische Kurrent
\newline{}Schnitzler: 1) mit Bleistift beschriftet: »\textsc{Hugo}« 2) mit rotem Buntstift mehrere Unterstreichungen\newline{}Ordnung: 1) mit Bleistift von \textcolor{blue}{Frieda Pollak} (?) mit dem Buchstaben »A« (Abgeschrieben/Abschrift) gekennzeichnet 2) mit Bleistift von unbekannter Hand nummeriert: »\strikeout{368}«3) mit Bleistift von unbekannter Hand nummeriert: »372«}\buchAbdrucke{\weitereDrucke{Hugo von Hofmannsthal, Arthur Schnitzler: \emph{Briefwechsel}. Hg. Therese Nickl und Heinrich Schnitzler. Frankfurt am Main: \emph{S. Fischer} 1964, S. 297–298.} }\toendnotes[C]{\smallbreak}\pstart
           \raggedleft{}{\pb}\textcolor{pink}{Rodaun}{}\ledrightnote{\textcolor{pink}{Rodaun}}{ }16 I 23\pend
           \pstart{}mein lieber Arthur\pend\pstart
           es freut mich so, dass ich wieder einmal von Ihnen einen Brief beko{\geminationm}e. – Zuletzt habe ich Sie im
                        September gesehen – aber Sie mich nicht – bei der \label{K_L02396_1v}\edtext{Première der \textcolor{green}{Dame Kobold}{}\ledrightnote{\textcolor{green}{Dame Kobold}}}{\lemma{\textnormal{\emph{Première der Dame Kobold}}}\Cendnote{\textnormal{siehe A. S.: \emph{Tagebuch}, 16. 9. 1922}}}\label{K_L02396_1h}. Sie standen neben Ihrer kleinen großen \textcolor{blue}{Tochter}{}\ledrightnote{→\textcolor{blue}{Lili Schnitzler}}, mir zugekehrt. Ich war auf der Gallerie und
                    ich sah Sie mit dem Opernglas an. Wie inhaltsvoll und freundlich war mir Ihr
                    Gesicht! Wie wenn ich ein Buch von tausend Seiten, dessen Inhalt ich aber gut
                    kenne – in einem Augenblick überblättert hätte.\pend
           \pstart
           Wie gerne würde ich Sie manchmal sehen, lieber Arthur. In die Stadt ko{\geminationm}e ich fast nie. Ich behalte das kleine
                    Absteigquartier so lange man mirs lässt, aber ich beheize die Wohnung nicht
                    mehr, \uline{betreibe} sie nicht mehr, halte dort keine
                    Bedienerin. Ich kann das alles nicht mehr. Ich bin durch den Marksturz in eine
                    fast unhaltbare materielle Situation geraten. Aber davon will {\pb}ich Sie durchaus nicht
                    unterhalten. – Wenn es im März freundlich ist, dann möchte ich
                    einmal vormittag zu Ihnen ko{\geminationm}en, mit Ihnen
                    spazierengehen u. bei Ihnen essen. Ich weiss ja da\textcolor{gray}{s}s es Sie beschwert, hier
                    herüber zu fahren! –\pend
           \pstart
           Mit \textcolor{blue}{Strauss}{}\ledrightnote{\textcolor{blue}{Richard Strauss}} würde ich sehr ungerne über die
                        \textcolor{green}{Opernsache}{}\ledrightnote{→\textcolor{green}{Der Schleier der Beatrice. Schauspiel in fünf Akten}} reden – aber
                    mit \textcolor{blue}{Schalk}{}\ledrightnote{\textcolor{blue}{Franz Schalk}} gerne wenn Sie wollen (obwohl es
                    eben so aussichtslos ist da ich den Standpunkt kenne und die enormen Argumente
                    die man für ihn geltend machen kann) – nur möchte ich abwarten, bis \textcolor{blue}{Schalk}{}\ledrightnote{\textcolor{blue}{Franz Schalk}} die schwere Sorge um seine \textcolor{blue}{Frau}{}\ledrightnote{→\textcolor{blue}{Lili Schalk}} los ist, die seit
                    Wochen höchst elend darniederliegt mit einer Gelenksentzündung.\pend
           \pstart
           Adieu, lieber Arthur.\pend
           \pstart
           Von Herzen, wie immer,{\\[\baselineskip]}Ihr{\\[\baselineskip]}\spacefill\mbox{Hugo.}\pend
           \leftskip=0em{}\endnumbering\briefempfaengerindex{Schnitzler, Arthur@\textsc{Schnitzler, Arthur}!zzzHofmannsthal, Hugo von@\emph{von Hugo von Hofmannsthal}!1923-01-161@{16. 1. 1923}|)be}\mylabel{h}  \normalsize

\doendnotes{C}
\bigskip
\vfill

\clearpage

\footnotesize

\lohead{\textsc{register}}

% Definiere theindex-Environment komplett neu ohne reledmac
\makeatletter
\renewenvironment{theindex}{%
  \section*{\indexname}%
  \setlength{\parindent}{0pt}%
  \setlength{\parskip}{0pt plus 0.3pt}%
  \let\item\@idxitem
}{%
  \clearpage
}
\makeatother

\IfFileExists{\jobname-pw.ind}{\input{\jobname-pw.ind}}{}

\end{document}

      