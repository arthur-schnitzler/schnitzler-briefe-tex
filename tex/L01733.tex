%% latex-korrekturansicht-vorspann.tex
%% Vorspann für die Korrekturansicht.
%% Lädt die gemeinsame Datei latex-vorspann.tex mit gesetztem Schalter.

\newif\ifkorrekturansicht
\korrekturansichttrue

\input{../tex-inputs/latex-vorspann}


               \section[Hugo von Hofmannsthal an Arthur Schnitzler, 16. {[}11. 1907?{]}]{ Hugo von Hofmannsthal an Arthur Schnitzler, 16. {[}11. 1907?{]}}\nopagebreak\mylabel{v}\rehead{ }\normalsize\beginnumbering\briefempfaengerindex{Schnitzler, Arthur@\textsc{Schnitzler, Arthur}!zzzHofmannsthal, Hugo von@\emph{von Hugo von Hofmannsthal}!1907-11-162@{16. {[}11. 1907?{]}}|(be} \toendnotes[C]{\smallbreak\pagebreak[2]} \Standort{CUL, Schnitzler, B 43.}
\physDesc{Postkarte
\newline{}Handschrift: schwarze Tinte, deutsche Kurrent\newline{}Versand: Stempel: »\nobreak{}\oindex{Rodaun@\textbf{Rodaun}, \emph{Teil eines besiedelten Ortes (A.BSOX)}|pwk}Rodaun, 16. X\textcolor{gray}{I}. 0{[}7{]}, 12\nobreak{}«.  
\newline{}Schnitzler: mit Bleistift datiert: »16/10 907« \newline{}Ordnung: 1) mit Bleistift von unbekannter Hand nummeriert:
                                    »285« 2) mit Bleistift von unbekannter Hand nummeriert:
                                    »285«}\buchAbdrucke{\weitereDrucke{Hugo von Hofmannsthal, Arthur Schnitzler: \emph{Briefwechsel}. Hg. Therese Nickl und Heinrich Schnitzler. Frankfurt am Main: \emph{S. Fischer} 1964, S. 232.} }\toendnotes[C]{\smallbreak}\pstart{}{\pb}\textsc{Herrn D\textsuperscript{r} Arthur Schnitzler}\pend{}\pstart{}\textsc{\textcolor{pink}{Wien}{}\ledrightnote{\textcolor{pink}{Wien}}}\pend{}\pstart{}\textsc{\textcolor{pink}{XVII Spöttelgasse 7}{}\ledrightnote{\textcolor{pink}{Edmund-Weiß-Gasse}}}\pend{}\pstart{}\textsc{neben \textcolor{pink}{Türkenschanzstrasse}{}\ledrightnote{\textcolor{pink}{Türkenschanzstrasse}}.}\pend{}{\bigskip}\pstart
           \noindent{}{\pb}Alſo wir ko{\geminationm}en ganz beſtimmt \label{K_L01733_1v}\edtext{Montag}{\lemma{\textnormal{\emph{Montag}}}\Cendnote{\textnormal{Die
                  handschriftliche Datierung Schnitzlers dürfte auf einer falschen Entzifferung des
                  Stempels beruhen. Nachdem aber die angesprochenen Details sich nicht mit den
                  sonstigen Dokumenten ein Einklang bringen lassen, ist ein kleiner Punkt beim
                  Stempel als Überrest eines »I« zu deuten und die Karte in den November zu
                  verlegen.}}}\label{K_L01733_1h} ſchon gegen \label{K_L01733_2v}\edtext{¾ 7}{\lemma{\textnormal{\emph{¾ 7}}}\Cendnote{\textnormal{18 Uhr 45}}}\label{K_L01733_2h}. \label{K_L01733_3v}\edtext{Dienstag}{\lemma{\textnormal{\emph{Dienstag}}}\Cendnote{\textnormal{Er reist am Mittwoch, den
                     20. 11. 1907 zuerst nach \textcolor{pink}{Dresden} und dann, nach
                  drei Tagen, weiter nach \textcolor{pink}{Berlin} und \textcolor{pink}{Weimar}. Am 17. 12. 1907 ist er
                  zurück.}}}\label{K_L01733_3h} reiſe ich.\pend
           \pstart
           Herzlich{\\[\baselineskip]}\spacefill\mbox{Hugo.}\pend
           \leftskip=0em{}\endnumbering\briefempfaengerindex{Schnitzler, Arthur@\textsc{Schnitzler, Arthur}!zzzHofmannsthal, Hugo von@\emph{von Hugo von Hofmannsthal}!1907-11-162@{16. {[}11. 1907?{]}}|)be}\mylabel{h}  \normalsize

\doendnotes{C}
\bigskip
\vfill

\clearpage

\footnotesize

\lohead{\textsc{register}}

% Definiere theindex-Environment komplett neu ohne reledmac
\makeatletter
\renewenvironment{theindex}{%
  \section*{\indexname}%
  \setlength{\parindent}{0pt}%
  \setlength{\parskip}{0pt plus 0.3pt}%
  \let\item\@idxitem
}{%
  \clearpage
}
\makeatother

\IfFileExists{\jobname-pw.ind}{\input{\jobname-pw.ind}}{}

\end{document}

      