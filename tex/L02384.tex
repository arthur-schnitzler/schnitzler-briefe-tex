%% latex-korrekturansicht-vorspann.tex
%% Vorspann für die Korrekturansicht.
%% Lädt die gemeinsame Datei latex-vorspann.tex mit gesetztem Schalter.

\newif\ifkorrekturansicht
\korrekturansichttrue

\input{../tex-inputs/latex-vorspann}


               \section[Gerhart Hauptmann an Arthur Schnitzler, 15. 5. 1922]{ Gerhart Hauptmann an Arthur Schnitzler, 15. 5. 1922}\nopagebreak\mylabel{v}\rehead{ }\normalsize\beginnumbering\briefempfaengerindex{Schnitzler, Arthur@\textsc{Schnitzler, Arthur}!zzzHauptmann, Gerhart@\emph{von Gerhart Hauptmann}!1922-05-153@{15. 5. 1922}|(be} \toendnotes[C]{\smallbreak\pagebreak[2]} \Standort{DLA, A:Schnitzler, 85.1.557.}
\physDesc{Telegramm
\newline{}maschinell\newline{}Versand: »\textcolor{gray}{\textbf{Aufgenommen von}}{ }\textcolor{gray}{\textbf{\textit{BL}}}{ }\textcolor{gray}{\textbf{auf Ltg.Nr.}}{ }\textcolor{gray}{\textbf{\textit{SI,25}}}{ }\textcolor{gray}{\textbf{am}}{ }\textcolor{gray}{\textbf{\textit{15. MAI}}} \textcolor{gray}{\textbf{192{\dots}}}{ }\textcolor{gray}{\textbf{um {\dots} Uhr {\dots} M. {\dots}
                                       Mitt. durch:}}{ }\textcolor{gray}{\textbf{\textit{\textcolor{blue}{OTTO}}}}« 
\newline{}Schnitzler: mit rotem Buntstift eine Unterstreichung \newline{}Zusatz: umseitig eine Werbung für einen
                                    Opalograph-Vervielfältiger }\buchAbdrucke{\weitereDrucke{Hans-Ulrich Lindken: \emph{Arthur Schnitzler. Aspekte und Akzente. Materialien zu Leben
                        und Werk}. Frankfurt am Main, Bern, Göttingen: \emph{Peter Lang} 1984, S. 416 (Europäische Hochschulschriften, Reihe 1, Deutsche Sprache und
                        Literatur, 754).} }\toendnotes[C]{\smallbreak}\pstart{}{\pb}arthur schmitzler\pend{}\pstart{}\textcolor{pink}{sertnwartestr 71 wien}{}\ledrightnote{\textcolor{pink}{Sternwartestraße}}\pend{}{\bigskip}\pstart
           {\pb}178/15 \textcolor{pink}{agnetendorf}{}\ledrightnote{\textcolor{pink}{Agnetendorf}} sp 97 41 15/5{ }11,5m\pend
           \pstart
           seien sie herzlichst begruesst und nehmen sie meene und meiner \textcolor{blue}{frau}{}\ledrightnote{→\textcolor{blue}{Margarete Hauptmann}} innige wuensche zum heutigen tage und
               besonders fuer kommende schoene arbeitreiche jahre dankbar fuer genossene gaben
               wollen wir mehr mehr in alter ergebenheit \pend
           \pstart \spacefill\mbox{= gerhart hauptmann +}\pend{}\endnumbering\briefempfaengerindex{Schnitzler, Arthur@\textsc{Schnitzler, Arthur}!zzzHauptmann, Gerhart@\emph{von Gerhart Hauptmann}!1922-05-153@{15. 5. 1922}|)be}\mylabel{h}  \normalsize

\doendnotes{C}
\bigskip
\vfill

\clearpage

\footnotesize

\lohead{\textsc{register}}

% Definiere theindex-Environment komplett neu ohne reledmac
\makeatletter
\renewenvironment{theindex}{%
  \section*{\indexname}%
  \setlength{\parindent}{0pt}%
  \setlength{\parskip}{0pt plus 0.3pt}%
  \let\item\@idxitem
}{%
  \clearpage
}
\makeatother

\IfFileExists{\jobname-pw.ind}{\input{\jobname-pw.ind}}{}

\end{document}

      