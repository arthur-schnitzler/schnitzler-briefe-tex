%% latex-korrekturansicht-vorspann.tex
%% Vorspann für die Korrekturansicht.
%% Lädt die gemeinsame Datei latex-vorspann.tex mit gesetztem Schalter.

\newif\ifkorrekturansicht
\korrekturansichttrue

\input{../tex-inputs/latex-vorspann}


               \section[Frank Wedekind an Arthur Schnitzler, 24. 12. 1909]{ Frank Wedekind an Arthur Schnitzler, 24. 12. 1909}\nopagebreak\mylabel{v}\rehead{ }\normalsize\beginnumbering\briefempfaengerindex{Schnitzler, Arthur@\textsc{Schnitzler, Arthur}!zzzWedekind, Frank@\emph{von Frank Wedekind}!1909-12-242@{24. 12. 1909}|(be} \toendnotes[C]{\smallbreak\pagebreak[2]} \Standort{CUL, Schnitzler, B 111.}
\physDesc{Brief, 1 Blatt, 4 Seiten
\newline{}Handschrift: schwarze Tinte, deutsche Kurrent
\newline{}Schnitzler: mit Bleistift beschriftet: »\textsc{Wedekind}« }\toendnotes[C]{\smallbreak}\pstart{}{\pb}Sehr verehrter Herr Doctor!\pend\pstart
           Darf ich Sie aufrichtig und herzlich bitten, es nur nicht als Theilnahmsloſigkeit
               auszulegen, daß wir nicht zu Ihnen kamen. Am Tage als wir zu ſpielen aufhörten, bekam
               meine \textcolor{blue}{Frau}{}\ledrightnote{→\textcolor{blue}{Tilly Wedekind}} die Nachricht, daß
               unſere \textcolor{blue}{Kleine}{}\ledrightnote{→\textcolor{blue}{Pamela Wedekind}}, die in \textcolor{pink}{Graz}{}\ledrightnote{\textcolor{pink}{Graz}} war, arg erkältet ſei. {\pb}Meine \textcolor{blue}{Frau}{}\ledrightnote{→\textcolor{blue}{Tilly Wedekind}} reiſte Hals über Kopf ohne ſich einen Augenblick Ruhe
               zu gönnen hin, um \label{T_L01909_1v}\edtext{\textcolor{blue}{ſie}{}\ledrightnote{→\textcolor{blue}{Pamela Wedekind}}}{\lemma{\textnormal{\emph{ſie}}}\Cendnote{\textnormal{\textcolor{blue}{Wedekind}{ }schreibt: »Sie«.}}}\label{T_L01909_1h} zu holen
               und als ſie mit ihr nach \textcolor{pink}{Wien}{}\ledrightnote{\textcolor{pink}{Wien}} kam fand ich es für
               dringend geboten, ohne Aufenthalt \textcolor{pink}{nach Hauſe}{}\ledrightnote{→\textcolor{pink}{München}} zurückzukehren. Am Dienſtag hoffte ich Sie
               wenigſtens allein noch aufſuchen zu können, aber auch dazu fehlte mir buchſtäblich
               die Zeit. So muß ich Ihnen meinen herzlichen Dank für die liebenswürdige
               Aufmerkſamkeit {\pb}die Sie für meine Arbeit
               übrig hatten, nun ſchriftlich ausſprechen. Dieſe Gelegenheit kann ich aber nicht
               vorbeiziehen laſſen ohne Ihnen zu ſagen, daß ich Ihnen die reichſten, künſtleriſch
               höchſten Genüſſe verdanke, die uns die deutſche Sprache ſeit zwanzig Jahren bietet,
               und daß ich für viele Ihrer Werke die bedingungsloſe Verehrung fühle, die ich ſonſt
               nur für Vergangenes aufbringen kann. So weit ich weiß kennen wir uns ſeit \label{K_L01909_1v}\edtext{bald zehn Jahren}{\lemma{\textnormal{\emph{bald zehn Jahren}}}\Cendnote{\textnormal{vgl. A. S.: \emph{Tagebuch}, 16. 11. 1901}}}\label{K_L01909_1h} und haben
               uns in dieſen zehn Jahren {\pb}\label{K_L01909_2v}\edtext{zwei mal geſehen}{\lemma{\textnormal{\emph{zwei mal geſehen}}}\Cendnote{\textnormal{siehe A. S.: \emph{Tagebuch}, 1. 5. 1907, siehe A. S.: \emph{Tagebuch}, 15. 9. 1909}}}\label{K_L01909_2h}. Sie werden es mir daher nicht
               verdenken, daß ich die Gelegenheit wahrneme, Ihnen mein Herz auszuſchütten. An mir
               ſoll es doch gewiß nicht liegen, daß wir uns nicht öfter begegnen.\pend
           \pstart
           Wollen Sie bitte Ihrer verehrten \textcolor{blue}{Frau
                  Gemahlin}{}\ledrightnote{→\textcolor{blue}{Olga Schnitzler}} meiner \textcolor{blue}{Frau}{}\ledrightnote{→\textcolor{blue}{Tilly Wedekind}} und
               meine ergebenſten Empfehlungen ausſprechen.\pend
           \pstart
           Ihr ergebener{\\[\baselineskip]}\spacefill\mbox{FrankWedekind.}\pend
           \leftskip=0em{}\pstart
           Heiliger Abend 1909.\pend
           \endnumbering\briefempfaengerindex{Schnitzler, Arthur@\textsc{Schnitzler, Arthur}!zzzWedekind, Frank@\emph{von Frank Wedekind}!1909-12-242@{24. 12. 1909}|)be}\mylabel{h}  \normalsize

\doendnotes{C}
\bigskip
\vfill

\clearpage

\footnotesize

\lohead{\textsc{register}}

% Definiere theindex-Environment komplett neu ohne reledmac
\makeatletter
\renewenvironment{theindex}{%
  \section*{\indexname}%
  \setlength{\parindent}{0pt}%
  \setlength{\parskip}{0pt plus 0.3pt}%
  \let\item\@idxitem
}{%
  \clearpage
}
\makeatother

\IfFileExists{\jobname-pw.ind}{\input{\jobname-pw.ind}}{}

\end{document}

      