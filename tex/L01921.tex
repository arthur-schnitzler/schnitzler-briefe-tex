%% latex-korrekturansicht-vorspann.tex
%% Vorspann für die Korrekturansicht.
%% Lädt die gemeinsame Datei latex-vorspann.tex mit gesetztem Schalter.

\newif\ifkorrekturansicht
\korrekturansichttrue

\input{../tex-inputs/latex-vorspann}


               \section[Stefan Großmann an Arthur Schnitzler, 2. 4. 1910]{ Stefan Großmann an Arthur Schnitzler, 2. 4. 1910}\nopagebreak\mylabel{v}\rehead{ }\normalsize\beginnumbering\briefempfaengerindex{Schnitzler, Arthur@\textsc{Schnitzler, Arthur}!zzzGrossmann, Stefan@\emph{von Stefan Großmann}!1910-04-021@{2. 4. 1910}|(be} \toendnotes[C]{\smallbreak\pagebreak[2]} \Standort{CUL, Schnitzler, B 34.}
\physDesc{Brief
\newline{}Handschrift: schwarze Tinte, lateinische Kurrent
\newline{}Schnitzler: 1) mit Bleistift beschriftet: »Großmann« 2) mit rotem Buntstift zwei Unterstreichungen\newline{}Ordnung: mit Bleistift von unbekannter Hand nummeriert: »8« }\toendnotes[C]{\smallbreak}\pstart
           \noindent{}{\pb}\textcolor{gray}{\textbf{\textcolor{brown}{ARBEITER-ZEITUNG}{}\ledrightnote{\textcolor{brown}{Arbeiter-Zeitung}}}}\pend
           \pstart
           \textcolor{gray}{\textbf{\textcolor{pink}{Wien}{}\ledrightnote{\textcolor{pink}{Wien}}, VI/1, \textcolor{pink}{Mariahilferstrasse 89}{}\ledrightnote{\textcolor{pink}{Mariahilferstraße}}}}\hfill \textcolor{gray}{\textbf{\textcolor{pink}{Wien}{}\ledrightnote{\textcolor{pink}{Wien}}, am}} 2. IV \textcolor{gray}{\textbf{19}}10\pend
           \pstart
           \textcolor{gray}{\textbf{Telephon 880, 900}}\pend
           \pstart
           \textcolor{gray}{\textbf{Postsparkassen-Scheck-Konto Nr. 19.210}}\pend
           \pstart\center{}Verehrter Herr\pend\pstart
           Verzeihen Sie \strikeout{Einem} mir, dass ich Ihren Brief erst
               heute beantworte.\pend
           \pstart
           Die Schauspieler baten mich, Sie erst zur \textcolor{green}{Première}{}\ledrightnote{→\textcolor{green}{Literatur}{\newline}→\textcolor{green}{Die letzten Masken}{\newline}→\textcolor{green}{Die Frage an das Schicksal}} zu laden, heute wurde noch irrsinnig gearbeitet.
               Sie wollten nicht im Rohzustande vor Sie hintreten.\pend
           \pstart
           Die letzte Probe fand heute nachmittag statt und endete um ¼ 7
               abends.\pend
           \pstart
           Leider wird Sie »\textcolor{green}{Literatur}{}\ledrightnote{\textcolor{green}{Literatur}}« nicht voll erfreun. Ich
               war krank vor Ärger, weil die Leiter des Theaters das willigste \strikeout{erf} freudigste Publikum der \textcolor{brown}{Freien Volksbühne}{}\ledrightnote{\textcolor{brown}{Wiener Freie Volksbühne}} kennen und, seine Milde missbrauchend, sagen: Da brauchen
               wir uns nicht anzustrengen.\pend
           \pstart
           {\pb}Ich war gestern im Ärger des Tags schon
               willig Sie zu bitten, lieber zu einer späteren Aufführung zu kommen. Jedenfalls wird
               die Qualität unserer Vorstellungen durch den »\textcolor{green}{halben
                  Held}{}\ledrightnote{\textcolor{green}{Ein halber Held. Tragödie}}« besser repräsentirt.\pend
           \pstart
           Ich sage das zornknirschend, aber ich will Sie lieber nicht irreführen. Wenn ich
               unser Theater selbst leiten werde, werde ich jene {\pb}Commandogewalt über die Schauspieler haben,
               die unerlässlich ist.\pend
           \pstart
           Um Ihnen nach diesen verdriesslichen Mittheilungen zu zeigen, wie sehr mir (der
               einmal als junger Esel sehr dumm vor Ihnen stand) an Ihrem Ja und Nein gelegen ist,
               müssen Sie mir gestatten, Ihnen meine \textcolor{green}{Besprechung}{}\ledrightnote{→\textcolor{green}{Arthur Schnitzler: Der Ruf des Lebens. Zur ersten Aufführung im Deutschen Volkstheater}} des »\textcolor{green}{Ruf des Lebens}{}\ledrightnote{\textcolor{green}{Der Ruf des Lebens. Schauspiel in drei Akten}}«
               vorzulegen. Ihnen liegt selbstverständlich nichts an {\pb}meiner Huldigung. Ich will Ihnen nur zeigen,
               einen wie \uline{andächtigen} Abend ich Ihnen verdankte.\pend
           \pstart
           \textcolor{blue}{S. Fischer}{}\ledrightnote{\textcolor{blue}{Samuel Fischer}} wurde verständigt. Seine Zustimmung
               ist zweifellos.\pend
           \pstart
           \uuline{Dank} und ergebensten Gruß:{\\[\baselineskip]}\spacefill\mbox{Stefan
                  Großmann}\pend
           \leftskip=0em{}\endnumbering\briefempfaengerindex{Schnitzler, Arthur@\textsc{Schnitzler, Arthur}!zzzGrossmann, Stefan@\emph{von Stefan Großmann}!1910-04-021@{2. 4. 1910}|)be}\mylabel{h}  \normalsize

\doendnotes{C}
\bigskip
\vfill

\clearpage

\footnotesize

\lohead{\textsc{register}}

% Definiere theindex-Environment komplett neu ohne reledmac
\makeatletter
\renewenvironment{theindex}{%
  \section*{\indexname}%
  \setlength{\parindent}{0pt}%
  \setlength{\parskip}{0pt plus 0.3pt}%
  \let\item\@idxitem
}{%
  \clearpage
}
\makeatother

\IfFileExists{\jobname-pw.ind}{\input{\jobname-pw.ind}}{}

\end{document}

      