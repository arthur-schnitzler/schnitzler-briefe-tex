%% latex-korrekturansicht-vorspann.tex
%% Vorspann für die Korrekturansicht.
%% Lädt die gemeinsame Datei latex-vorspann.tex mit gesetztem Schalter.

\newif\ifkorrekturansicht
\korrekturansichttrue

\input{../tex-inputs/latex-vorspann}


               \section[Friedrich M. Fels an Arthur Schnitzler, 6. 11. 1894]{ Friedrich M. Fels an Arthur Schnitzler, 6. 11. 1894}\nopagebreak\mylabel{v}\rehead{ }\normalsize\beginnumbering\briefempfaengerindex{Schnitzler, Arthur@\textsc{Schnitzler, Arthur}!zzzFels, Friedrich Michael@\emph{von Friedrich Michael Fels}!1894-11-061@{6. 11. 1894}|(be} \toendnotes[C]{\smallbreak\pagebreak[2]} \Standort{DLA, A:Schnitzler, HS.NZ85.1.2956.}
\physDesc{Brief, 1 Blatt, 2 Seiten
\newline{}Handschrift: schwarze Tinte, lateinische Kurrent
\newline{}Schnitzler: 1) mit Bleistift nummeriert: »18« 2) mit rotem Buntstift eine Unterstreichung}\buchAbdrucke{\weitereDrucke{Hermann Bahr, Arthur Schnitzler: \emph{Briefwechsel, Aufzeichnungen, Dokumente (1891–1931)}. Hg. Kurt Ifkovits und Martin Anton Müller. Göttingen: \emph{Wallstein} 2018, S. 86.} }\toendnotes[C]{\smallbreak}\pstart
           \raggedleft{}{\pb}\textcolor{pink}{Wien XVIII, Gürtelstr. 90}{}\ledrightnote{\textcolor{pink}{Währinger Gürtel}}{\\}6. Nov. 94\pend
           \pstart{}Lieber Doktor Schnitzler!\pend\pstart
           \textcolor{blue}{Herma{\geminationn} Bahr}{}\ledrightnote{\textcolor{blue}{Hermann Bahr}} hat den
               Artikel »\textcolor{green}{Skandinavien in Deutschland}{}\ledrightnote{\textcolor{green}{Skandinavien in Deutschland}}« abgelehnt,
               weil er nicht aktuell genug sei und deshalb vor 3–4 Monaten nicht erscheinen kö{\geminationn}e. Da er selbstredend! gar nicht annahm, daſs ich so
               lange warten werde, habe ich auch nichts gesagt, obgleich ich herzlich froh gewesen
               wäre, we{\geminationn} er da{\geminationn} erschienen
               wäre; ich werde froh sein müſsen, we{\geminationn} er anderswo so
               bald erscheint. Aber man muſs den Leuten \introOben{}die\introOben{} Ausreden nicht
               zu schwer machen. Von Artikeln war keine Rede mehr; dagegen sagte \textcolor{blue}{Bahr}{}\ledrightnote{\textcolor{blue}{Hermann Bahr}}, er werde mir Buchbesprechungen und zwar von
               literarhistorischen Werken – von andern verstehe ich wohl zu wenig – übertragen; ich
               nahm mit Dank an und habe nun die Hoffnung, we{\geminationn}s sehr
               gut geht, in einem Jahr drei Rezensionen schreiben zu dürfen und damit {\pb}5 fl zu verdienen. Hingehen werde ich wohl kaum mehr,
               da er, als ich gemeldet wurde, obgleich ich auf heute 4 Uhr von ihm bestellt war,
               laut aufseufzte und \textcolor{gray}{vernehmlich}{ }ſagte »So lassen Sie ihn in Gottes Namen
               herein.« –\pend
           \pstart
           Den \textcolor{green}{Artikel}{}\ledrightnote{→\textcolor{green}{Skandinavien in Deutschland}} werde ich morgen nach
                  \textcolor{pink}{Berlin}{}\ledrightnote{\textcolor{pink}{Berlin}}{ }ſchicken, den beka{\geminationn}ten
               Weg: zuerst \textcolor{brown}{Zukunft}{}\ledrightnote{\textcolor{brown}{Die Zukunft}}, da{\geminationn}{ }\textcolor{brown}{Nation}{}\ledrightnote{\textcolor{brown}{Die Nation}}, da{\geminationn}{ }\textcolor{brown}{Tante Voss}{}\ledrightnote{→\textcolor{brown}{Vossische Zeitung}}, da{\geminationn}{ }\textcolor{brown}{Gegenwart}{}\ledrightnote{\textcolor{brown}{Die Gegenwart}}, da{\geminationn}{ }{\dots} wer weiss, wohin noch. Den von \label{K_L00397_1v}\edtext{\textcolor{blue}{David}{}\ledrightnote{\textcolor{blue}{Jakob Julius David}}}{\lemma{\textnormal{\emph{David}}}\Cendnote{\textnormal{von der \emph{\textcolor{brown}{Wiener Allgemeinen Zeitung}}}}}\label{K_L00397_1h} refusierten \label{K_L00397_2v}\edtext{\textcolor{blue}{Sealsfield}{}\ledrightnote{\textcolor{blue}{Charles Sealsfield}}artikel}{\lemma{\textnormal{\emph{Sealsfieldartikel}}}\Cendnote{\textnormal{möglicherweise die zur Einleitung von \textcolor{blue}{Charles Sealsfield}: \emph{\textcolor{green}{Das Kajütenbuch oder nationale Charakteristiken}}. Hg. und
                     eingel. von \textcolor{blue}{Friedrich M. Fels}. Stuttgart:
                        \emph{\textcolor{brown}{Philipp Reclam Jun.}} [o. J.]}}}\label{K_L00397_2h} bringe ich \label{K_L00397_3v}\edtext{\textcolor{blue}{Uhl}{}\ledrightnote{\textcolor{blue}{Friedrich Uhl}}}{\lemma{\textnormal{\emph{Uhl}}}\Cendnote{\textnormal{der \emph{\textcolor{brown}{Wiener
                     Zeitung}}}}}\label{K_L00397_3h}, da{\geminationn}{ }\label{K_L00397_4v}\edtext{\textcolor{blue}{Pötzl}{}\ledrightnote{\textcolor{blue}{Eduard Pötzl}}}{\lemma{\textnormal{\emph{Pötzl}}}\Cendnote{\textnormal{dem \emph{\textcolor{brown}{Neuen
                     Wiener Tagblatt}}}}}\label{K_L00397_4h}, da{\geminationn}{ }\label{K_L00397_5v}\edtext{\textcolor{blue}{Schönthan}{}\ledrightnote{\textcolor{blue}{Paul von Schönthan-Pernwald}}}{\lemma{\textnormal{\emph{Schönthan}}}\Cendnote{\textnormal{dem \emph{\textcolor{brown}{Wiener
                     Tagblatt}}}}}\label{K_L00397_5h}, da{\geminationn}{ }\label{K_L00397_6v}\edtext{\textcolor{blue}{Granichstädten}{}\ledrightnote{\textcolor{blue}{Emil Granichstaedten}}}{\lemma{\textnormal{\emph{Granichstädten}}}\Cendnote{\textnormal{der \emph{\textcolor{brown}{Presse}}}}}\label{K_L00397_6h}{ }{\dots} da{\geminationn} gehe ich in die
               Provinz, nach \textcolor{pink}{Brü{\geminationn}}{}\ledrightnote{\textcolor{pink}{Brünn}} und \textcolor{pink}{Olmütz}{}\ledrightnote{\textcolor{pink}{Olomouc}}; vielleicht, dass man ihn in \textcolor{blue}{Sealsfield}{}\ledrightnote{\textcolor{blue}{Charles Sealsfield}}s Heimat ni{\geminationm}t, und 3 fl sind besser als nichts.\pend
           \pstart
           Besten Gruſs{\\[\baselineskip]}\spacefill\mbox{Fels}\pend
           \leftskip=0em{}\pstart
           \noindent{}Ich merke eben, dass ich die ekelhafte Gewohnheit angeno{\geminationm}en habe, Ihnen mein Leid, wenn ich nicht ko{\geminationm}en ka{\geminationn}, weil ich an dem
                  Tag schon bei Ihnen war, – schriftlich zu klagen. Seien Sie mir nicht böse!\pend
           \endnumbering\briefempfaengerindex{Schnitzler, Arthur@\textsc{Schnitzler, Arthur}!zzzFels, Friedrich Michael@\emph{von Friedrich Michael Fels}!1894-11-061@{6. 11. 1894}|)be}\mylabel{h}  \normalsize

\doendnotes{C}
\bigskip
\vfill

\clearpage

\footnotesize

\lohead{\textsc{register}}

% Definiere theindex-Environment komplett neu ohne reledmac
\makeatletter
\renewenvironment{theindex}{%
  \section*{\indexname}%
  \setlength{\parindent}{0pt}%
  \setlength{\parskip}{0pt plus 0.3pt}%
  \let\item\@idxitem
}{%
  \clearpage
}
\makeatother

\IfFileExists{\jobname-pw.ind}{\input{\jobname-pw.ind}}{}

\end{document}

      