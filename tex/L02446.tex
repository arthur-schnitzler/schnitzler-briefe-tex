%% latex-korrekturansicht-vorspann.tex
%% Vorspann für die Korrekturansicht.
%% Lädt die gemeinsame Datei latex-vorspann.tex mit gesetztem Schalter.

\newif\ifkorrekturansicht
\korrekturansichttrue

\input{../tex-inputs/latex-vorspann}


               \section[Arthur Schnitzler an Georg Brandes, 2{[}7?{]}. 7. 1925]{ Arthur Schnitzler an Georg Brandes, 2{[}7?{]}. 7. 1925}\nopagebreak\mylabel{v}\rehead{ }\normalsize\beginnumbering\briefempfaengerindex{Brandes, Georg@\textsc{Brandes, Georg}!zzzSchnitzler, Arthur@\emph{von Arthur Schnitzler}!1925-07-271@{2{[}7?{]}. 7. 1925}|(be} \toendnotes[C]{\smallbreak\pagebreak[2]} \Standort{Kopenhagen, Det Kongelige Bibliotek, Georg Brandes Arkiv, box 125.}
\physDesc{Bildpostkarte
\newline{}Handschrift: schwarze Tinte, lateinische Kurrent\newline{}Versand: 1) Stempel: »\nobreak{}Wien\nobreak{}«.  2) Stempel: »\nobreak{}\oindex{Kopenhagen@\textbf{Kopenhagen}, \emph{Besiedelter Ort (A.BSO)}|pwk}\textcolor{gray}{Kjobenhav}n, 29. 7. {[}1925{]}, 20M\nobreak{}«. \newline{}Ordnung: mit Bleistift von unbekannter Hand nummeriert:
                                    »53« und datiert: »29-7-25 (?)« }\buchAbdrucke{\weitereDrucke{1) Georg Brandes, Arthur Schnitzler: \emph{Ein Briefwechsel}. Hg. Kurt Bergel. Bern: \emph{Francke} 1956, S. 150.} \weitereDrucke{2) Arthur Schnitzler: \emph{Briefe 1913–1931}. Hg. Peter Michael Braunwarth, Richard Miklin, Susanne Pertlik und Heinrich Schnitzler. Frankfurt am Main: \emph{S. Fischer} 1984, S. 417.} }\toendnotes[C]{\smallbreak}\pstart{}{\pb}\label{T_L02446-1v}\edtext{\textcolor{gray}{\textbf{A. S.}}}{\lemma{\textnormal{\emph{A. S.}}}\Cendnote{\textnormal{ovaler Absenderkleber über die
                     Kartenkante, teilweise über den Text}}}\label{T_L02446-1h}\pend{}\pstart{}\textcolor{pink}{\textcolor{gray}{\textbf{WIEN, XVIII.}}}{}\ledrightnote{\textcolor{pink}{XVIII., Währing}}\pend{}\pstart{}\textcolor{pink}{\textcolor{gray}{\textbf{STERNWARTESTR. 71}}}{}\ledrightnote{\textcolor{pink}{Sternwartestraße}}\pend{}{\bigskip}\pstart{}{\pb}Herrn\pend{}\pstart{}Georg Brandes\pend{}\pstart{}\textcolor{pink}{Kopenhagen}{}\ledrightnote{\textcolor{pink}{Kopenhagen}}\pend{}\pstart{}\textcolor{pink}{Daenemark}{}\ledrightnote{\textcolor{pink}{Dänemark}}\pend{}{\bigskip}\pstart
           \noindent{}\centering{}{\pb}{[}\textcolor{pink}{Sternwartestraße 71}{}\ledrightnote{\textcolor{pink}{Sternwartestraße}}{]}\pend
           \pstart
           {\pb}Herzlichen \damage{Dank für} Ihre liebe Karte. Ihre Bitte es niemandem zu sagen, daß die Menschheit eine
               abscheuliche Bande, ko{\geminationm}t leider verspätet. Weiſs der
               Teufel durch welche Indiscretion – die Sache hat sich herumgesprochen!\pend
           \pstart
           – Ich bin noch in \textcolor{pink}{Wien}{}\ledrightnote{\textcolor{pink}{Wien}}, arbeite allerlei, denke Ihrer
               in alter inniger Freundschaft und bitte Sie, mich und {\pb}dieses Haus in gütiger Erinnerung zu behalten\pend
           \pstart
           Mit tausend Grüßen{\\[\baselineskip]}Ihr getreuer\spacefill\mbox{Arthur Schnitzler}\pend
           \leftskip=0em{}\endnumbering\briefempfaengerindex{Brandes, Georg@\textsc{Brandes, Georg}!zzzSchnitzler, Arthur@\emph{von Arthur Schnitzler}!1925-07-271@{2{[}7?{]}. 7. 1925}|)be}\mylabel{h}  \normalsize

\doendnotes{C}
\bigskip
\vfill

\clearpage

\footnotesize

\lohead{\textsc{register}}

% Definiere theindex-Environment komplett neu ohne reledmac
\makeatletter
\renewenvironment{theindex}{%
  \section*{\indexname}%
  \setlength{\parindent}{0pt}%
  \setlength{\parskip}{0pt plus 0.3pt}%
  \let\item\@idxitem
}{%
  \clearpage
}
\makeatother

\IfFileExists{\jobname-pw.ind}{\input{\jobname-pw.ind}}{}

\end{document}

      