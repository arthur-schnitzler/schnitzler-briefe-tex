%% latex-korrekturansicht-vorspann.tex
%% Vorspann für die Korrekturansicht.
%% Lädt die gemeinsame Datei latex-vorspann.tex mit gesetztem Schalter.

\newif\ifkorrekturansicht
\korrekturansichttrue

\input{../tex-inputs/latex-vorspann}


               \section[Richard Beer-Hofmann an Arthur Schnitzler, {[}22. 12. 1910{]}]{ Richard Beer-Hofmann an Arthur Schnitzler, {[}22. 12. 1910{]}}\nopagebreak\mylabel{v}\rehead{ }\normalsize\beginnumbering\briefempfaengerindex{Schnitzler, Arthur@\textsc{Schnitzler, Arthur}!zzzBeer-Hofmann, Richard@\emph{von Richard Beer-Hofmann}!1910-12-221@{{[}22. 12. 1910{]}}|(be} \toendnotes[C]{\smallbreak\pagebreak[2]} \Standort{CUL, Schnitzler, B 8.}
\physDesc{Kartenbrief, 1 Blatt, 2 Seiten
\newline{}Handschrift: blauer Buntstift, lateinische Kurrent\newline{}Versand: ohne postalischen Übermittlungsvermerk 
\newline{}Schnitzler: mit Bleistift datiert: »22/12 910« \newline{}Ordnung: mit Bleistift von unbekannter Hand nummeriert:
                                    »239« }\buchAbdrucke{\weitereDrucke{Arthur Schnitzler, Richard Beer-Hofmann: \emph{Briefwechsel 1891–1931}. Hg. Konstanze Fliedl. Wien, Zürich: \emph{Europaverlag} 1992, S. 213.} }\toendnotes[C]{\smallbreak}\pstart{}{\pb}Herrn\pend{}\pstart{}Arthur Schnitzler\pend{}{\bigskip}\pstart{}{\pb}Lieber Arthur!\pend\pstart
           \textcolor{blue}{Berger}{}\ledrightnote{\textcolor{blue}{Josef Berger}} – der \label{KLL01992_AS-1v}\edtext{ehrliche}{\lemma{\textnormal{\emph{ehrliche}}}\Cendnote{\textnormal{gemeint
                  in Abgrenzung zum \emph{\textcolor{brown}{Burgtheaterdirektor}}{ }\textcolor{blue}{Alfred von Berger}}}}\label{KLL01992_AS-1h} – fragt teleph. an, ob Sie auf die kleine Roccococo{\geminationm}ode reflectiren, da \strikeout{S}
               sonst eine Dame – die auf Antwort wartet – sie möchte. Bitte um Antwort, u. Trauer
                  \label{T_L01992_1v}\edtext{dass}{\lemma{\textnormal{\emph{dass}}}\Cendnote{\textnormal{geschrieben: das}}}\label{T_L01992_1h} Sie uns heute Vorm. verfehlten. Ihr
                  \spacefill\mbox{R}\pend
           \endnumbering\briefempfaengerindex{Schnitzler, Arthur@\textsc{Schnitzler, Arthur}!zzzBeer-Hofmann, Richard@\emph{von Richard Beer-Hofmann}!1910-12-221@{{[}22. 12. 1910{]}}|)be}\mylabel{h}  \normalsize

\doendnotes{C}
\bigskip
\vfill

\clearpage

\footnotesize

\lohead{\textsc{register}}

% Definiere theindex-Environment komplett neu ohne reledmac
\makeatletter
\renewenvironment{theindex}{%
  \section*{\indexname}%
  \setlength{\parindent}{0pt}%
  \setlength{\parskip}{0pt plus 0.3pt}%
  \let\item\@idxitem
}{%
  \clearpage
}
\makeatother

\IfFileExists{\jobname-pw.ind}{\input{\jobname-pw.ind}}{}

\end{document}

      