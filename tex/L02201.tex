%% latex-korrekturansicht-vorspann.tex
%% Vorspann für die Korrekturansicht.
%% Lädt die gemeinsame Datei latex-vorspann.tex mit gesetztem Schalter.

\newif\ifkorrekturansicht
\korrekturansichttrue

\input{../tex-inputs/latex-vorspann}


               \section[Georg Brandes an Arthur Schnitzler, 23. 12. 1914]{ Georg Brandes an Arthur Schnitzler, 23. 12. 1914}\nopagebreak\mylabel{v}\rehead{ }\normalsize\beginnumbering\briefempfaengerindex{Schnitzler, Arthur@\textsc{Schnitzler, Arthur}!zzzBrandes, Georg@\emph{von Georg Brandes}!1914-12-231@{23. 12. 1914}|(be} \toendnotes[C]{\smallbreak\pagebreak[2]} \Standort{CUL, Schnitzler, B 17.}
\physDesc{Postkarte
\newline{}Handschrift: schwarze Tinte, lateinische Kurrent\newline{}Versand: Stempel: »\nobreak{}\oindex{Kopenhagen@\textbf{Kopenhagen}, \emph{Besiedelter Ort (A.BSO)}|pwk}Kjøbenhavn, 23. 12. 14, 2–3E\nobreak{}«.  
\newline{}Schnitzler: mit Bleistift beschriftet: »\textsc{Brandes}« \newline{}Ordnung: mit Bleistift von unbekannter Hand nummeriert:
                                        »44« }\buchAbdrucke{\weitereDrucke{Georg Brandes, Arthur Schnitzler: \emph{Ein Briefwechsel}. Hg. Kurt Bergel. Bern: \emph{Francke} 1956, S. 113–114.} }\toendnotes[C]{\smallbreak}\pstart{}{\pb}Herrn Dr. Arthur
                        Schnitzler\pend{}\pstart{}\textcolor{pink}{Sternwartestrasse 71}{}\ledrightnote{\textcolor{pink}{Sternwartestraße}}\pend{}\pstart{}\textcolor{pink}{Wien XVIII}{}\ledrightnote{\textcolor{pink}{XVIII., Währing}}\pend{}{\bigskip}\pstart
           \raggedleft{}{\pb}\textcolor{pink}{Kopenhagen}{}\ledrightnote{\textcolor{pink}{Kopenhagen}}{ }23 Dec 14\pend
           \pstart{}Verehrter und lieber Freund\pend\pstart
           Es freute mich ein Lebenszeichen von Ihnen zu sehen. Es freut mich noch mehr,
                    dass Sie und die Ihrigen in guter und ruhiger Stimmung sind. Meine einzige \textcolor{blue}{Tochter}{}\ledrightnote{→\textcolor{blue}{Edith Philipp}} ist in \textcolor{pink}{Berlin}{}\ledrightnote{\textcolor{pink}{Berlin}} verheirathet. Ihr junger \textcolor{blue}{Mann}{}\ledrightnote{→\textcolor{blue}{Reinhold Philipp}} ist Fabrikant und
                    Gardelieutenant der Artillerie, er wurde schon im September zum
                    Oberlieutenant befördert und bekam im November das eiserne Kreuz. Aber er ist in
                    steter Lebensgefahr. Meine \textcolor{blue}{Tochter}{}\ledrightnote{→\textcolor{blue}{Edith Philipp}} war mehrere Monate hier mit zwei \textcolor{blue}{Kleinen}{}\ledrightnote{→\textcolor{blue}{Gerda Philipp}{\newline}→\textcolor{blue}{Georg Philipp}}, einer \textcolor{blue}{Tochter}{}\ledrightnote{→\textcolor{blue}{Gerda Philipp}} von 7 Jahren und einem \textcolor{blue}{Jungen}{}\ledrightnote{→\textcolor{blue}{Georg Philipp}} von 2 Jahren, beide
                    sehr hübsch; sie ist jetzt in \textcolor{pink}{Berlin}{}\ledrightnote{\textcolor{pink}{Berlin}} und
                    natürlich recht unruhig und mitgenommen von der ewigen Spannung. Ich arbeite
                    viel, schreibe im Augenblick ein \textcolor{green}{Buch über \textcolor{blue}{Goethe}{}\ledrightnote{\textcolor{blue}{Johann Wolfgang von Goethe}}}{}\ledrightnote{→\textcolor{green}{Wolfgang Goethe}}, parallel zu dem, ich einmal über \textcolor{green}{\textcolor{blue}{Shspeare}{}\ledrightnote{\textcolor{blue}{William Shakespeare}}}{}\ledrightnote{→\textcolor{green}{William Shakespeare}}
               schrieb. Ausserdem habe ich fast jeden Monat ein grosses Essay
                    veröffentlicht.\pend
           \pstart Grüssen Sie Ihre Frau \textcolor{blue}{Gemahlin}{}\ledrightnote{→\textcolor{blue}{Olga Schnitzler}} und \textcolor{blue}{Beer-Hoffmanns}{}\ledrightnote{\textcolor{blue}{Richard Beer-Hofmann}{\newline}\textcolor{blue}{Paula Beer-Hofmann}}. Ihr \spacefill\mbox{G. B.}\pend{}\endnumbering\briefempfaengerindex{Schnitzler, Arthur@\textsc{Schnitzler, Arthur}!zzzBrandes, Georg@\emph{von Georg Brandes}!1914-12-231@{23. 12. 1914}|)be}\mylabel{h}  \normalsize

\doendnotes{C}
\bigskip
\vfill

\clearpage

\footnotesize

\lohead{\textsc{register}}

% Definiere theindex-Environment komplett neu ohne reledmac
\makeatletter
\renewenvironment{theindex}{%
  \section*{\indexname}%
  \setlength{\parindent}{0pt}%
  \setlength{\parskip}{0pt plus 0.3pt}%
  \let\item\@idxitem
}{%
  \clearpage
}
\makeatother

\IfFileExists{\jobname-pw.ind}{\input{\jobname-pw.ind}}{}

\end{document}

      