%% latex-korrekturansicht-vorspann.tex
%% Vorspann für die Korrekturansicht.
%% Lädt die gemeinsame Datei latex-vorspann.tex mit gesetztem Schalter.

\newif\ifkorrekturansicht
\korrekturansichttrue

\input{../tex-inputs/latex-vorspann}


               \section[Arthur Schnitzler an Richard Beer-Hofmann, 22. 9. 1891]{ Arthur Schnitzler an Richard Beer-Hofmann, 22. 9. 1891}\nopagebreak\mylabel{v}\rehead{ }\normalsize\beginnumbering\briefempfaengerindex{Beer-Hofmann, Richard@\textsc{Beer-Hofmann, Richard}!zzzSchnitzler, Arthur@\emph{von Arthur Schnitzler}!1891-09-221@{22. 9. 1891}|(be} \toendnotes[C]{\smallbreak\pagebreak[2]} \Standort{YCGL, MSS 31.}
\physDesc{Kartenbrief
\newline{}Handschrift: Bleistift, deutsche Kurrent\newline{}Versand: 1) Stempel: »\nobreak{}\oindex{Halle an der Saale@\textbf{Halle an der Saale}, \emph{Besiedelter Ort (A.BSO)}|pwk}Halle Saale 2, 22. 9. 91, 9–10 N\nobreak{}«.  2) Stempel: »\nobreak{}\oindex{III., Landstrasse@\textbf{III., Landstraße}, \emph{Bezirk (A.BZK)}|pwk}Wien 3/2, 24 9 91, 8 10. V, Bestellt\nobreak{}«. }\buchAbdrucke{\weitereDrucke{1) Arthur Schnitzler: \emph{Briefe 1875–1912}. Hg. Therese Nickl und Heinrich Schnitzler. Frankfurt am Main: \emph{S. Fischer} 1981, S. 121.} \weitereDrucke{2) Arthur Schnitzler, Richard Beer-Hofmann: \emph{Briefwechsel 1891–1931}. Hg. Konstanze Fliedl. Wien, Zürich: \emph{Europaverlag} 1992, S. 32.} }\toendnotes[C]{\smallbreak}\pstart{}{\pb}\textsc{Herrn Dr. Rich. Beer-Hofmann}\pend{}\pstart{}\textsc{\textcolor{pink}{Wien}{}\ledrightnote{\textcolor{pink}{Wien}}}\pend{}\pstart{}\textcolor{pink}{\textsc{III Seidlgasse 30}}{}\ledrightnote{\textcolor{pink}{Seidlgasse}}\pend{}{\bigskip}\pstart
           \noindent{}{\pb}Lieber Richard, das muſs man erleben, dieſes \textcolor{pink}{Halle}{}\ledrightnote{\textcolor{pink}{Halle an der Saale}}! Tramways, die an die Ehrlichkeit der Menſchen glauben –
               im Waggon ſind Käſtchen, wo {\pb}man ſein Fahrgeld
               hineinwirft. – Und dieſe Menſchen ſelbſt – I{\geminationm}erfort ſ\substVorne{}\textsuperscript{\textcolor{gray}{in}}\substDazwischen{}chr\substHinten{}eien ſie und ſind ſtolz auf \label{K_L00041_1v}\edtext{das geeinte \textcolor{pink}{deutſche Reich}{}\ledrightnote{\textcolor{pink}{Deutschland}}}{\lemma{\textnormal{\emph{das … Reich}}}\Cendnote{\textnormal{Am 2. 9. 1891 hatte sich
                  zum 20. Mal der Tag von \textcolor{pink}{Sedan} (Ende des \textcolor{pink}{Deutsch}-\textcolor{pink}{Französischen} Krieges von 1870/1871) gejährt,
                  der im \textcolor{pink}{Deutschen Reich} als Tag der Einheit galt.
                     Vgl. Arthur Schnitzler an Hugo von Hofmannsthal, 1. 9. 1895}}}\label{K_L00041_1h}. Lauter Nationalparvenus. – Ich ko{\geminationm}e bald.
                  Ihr\spacefill\mbox{Arthur}\pend
           \endnumbering\briefempfaengerindex{Beer-Hofmann, Richard@\textsc{Beer-Hofmann, Richard}!zzzSchnitzler, Arthur@\emph{von Arthur Schnitzler}!1891-09-221@{22. 9. 1891}|)be}\mylabel{h}  \normalsize

\doendnotes{C}
\bigskip
\vfill

\clearpage

\footnotesize

\lohead{\textsc{register}}

% Definiere theindex-Environment komplett neu ohne reledmac
\makeatletter
\renewenvironment{theindex}{%
  \section*{\indexname}%
  \setlength{\parindent}{0pt}%
  \setlength{\parskip}{0pt plus 0.3pt}%
  \let\item\@idxitem
}{%
  \clearpage
}
\makeatother

\IfFileExists{\jobname-pw.ind}{\input{\jobname-pw.ind}}{}

\end{document}

      