%% latex-korrekturansicht-vorspann.tex
%% Vorspann für die Korrekturansicht.
%% Lädt die gemeinsame Datei latex-vorspann.tex mit gesetztem Schalter.

\newif\ifkorrekturansicht
\korrekturansichttrue

\input{../tex-inputs/latex-vorspann}


               \section[Hugo Hofmannsthal an Arthur Schnitzler, 9. 3. {[}1926{]}]{ Hugo Hofmannsthal an Arthur Schnitzler, 9. 3. {[}1926{]}}\nopagebreak\mylabel{v}\rehead{ }\normalsize\beginnumbering\briefempfaengerindex{Schnitzler, Arthur@\textsc{Schnitzler, Arthur}!zzzHofmannsthal, Hugo von@\emph{von Hugo von Hofmannsthal}!1926-03-091@{9. 3. {[}1926{]}}|(be} \toendnotes[C]{\smallbreak\pagebreak[2]} \Standort{CUL, Schnitzler, B 43.}
\physDesc{Brief, 2 Blätter (das zweite Blatt nummeriert mit: »II.«), 4 Seiten
\newline{}Handschrift: schwarze Tinte, lateinische Kurrent
\newline{}Schnitzler: mit Bleistift die Jahreszahl ergänzt: »26« und beschriftet: »\textsc{Hugo}«. Datiert: »9/3 26« \newline{}Ordnung: 1) mit Bleistift von unbekannter Hand auf der zweiten Seite der
                           Vermerk »Abgeschrieben« und auf der vierten, ansonsten
                           unbeschriebenen Seite der Name: »{\pb}Hofmannsthal« 2) mit Bleistift von unbekannter Hand
                           nummeriert: »370«3) mit Bleistift von unbekannter Hand nummeriert: »379«}\buchAbdrucke{\weitereDrucke{Hugo von Hofmannsthal, Arthur Schnitzler: \emph{Briefwechsel}. Hg. Therese Nickl und Heinrich Schnitzler. Frankfurt am Main: \emph{S. Fischer} 1964, S. 305.} }\pstart
           {\pb}\textcolor{pink}{Rodaun}{}\ledrightnote{\textcolor{pink}{Rodaun}}{ }9 III\pend
           \pstart{}mein lieber Arthur\pend\pstart
           \textcolor{blue}{Lili}{}\ledrightnote{\textcolor{blue}{Lili Schnitzler}}, das hübsche, Hüte wechselnde, schwer
               wiederzuerkennende Wesen sagt mir, dass Sie schon eine ganze Weile zurück sind, in
               dessen ich Sie noch in \textcolor{pink}{Deutschland}{}\ledrightnote{\textcolor{pink}{Deutschland}} glaubte.\pend
           \pstart
           Sie soll mir nur freundlich verzeihen und mich immer etwas vertraulich anlächeln.
               Denn ich habe nicht etwa ein schlechtes Physiognomieengedächtnis, sondern etwas viel
               Sonderbareres. Meine Phantasie verändert mir das Erinnerungsbild, sie gestaltet um,
               verschärft einen besti{\geminationm}ten Zug, und tritt dann das
               Original vor mich, so weigert sich die Phantasie, die Identität anzuerke{\geminationn}en. Ich grüße infolgedessen in einem Theater oder auf
               der Gasse fast nur fremde Menschen, deren Gesichter ich mit vermeintlichen Gesichtern
               in einen plausiblen {\pb}Zusammenhang
               bringe. Außerdem aber habe ich schlechte Augen.\hspace*{1.5em}Soviel nun von \textcolor{blue}{Lili}{}\ledrightnote{\textcolor{blue}{Lili Schnitzler}} u. meinen schwierigen, durch
               wechselnde Hüte und wechselnden Ausdruck noch erschwerten Begegnungen mit ihr. Jetzt
               aber eine Bitte, eine Quälerei, eine mehr zu den vielen die jede Post bringt. Aber
               ich wage es, denn es handelt sich darum, einem ordentlichen, in die schwierigste Lage
               geratenen Menschen zu helfen. Der Verleger \textcolor{blue}{Erich
                  Reiss}{}\ledrightnote{\textcolor{blue}{Erich Reiss}} (Verleger von \textcolor{blue}{Brandes}{}\ledrightnote{\textcolor{blue}{Georg Brandes}} und anderen,
               fast lauter guter Sachen) ist zusammengebrochen. Es wäre ihm vom größten Nutzen, vor
               allem moralisch, wenn Sie (ebenso wie ich) die Güte haben wollten, ein paar Zeilen in
               Maschinschrift zu dictieren, worin Sie bekunden dass der Verlag \textcolor{brown}{Erich Reiss}{}\ledrightnote{\textcolor{brown}{Erich Reiß}}{ }{\pb}ein Unternehmen von culturellem
               Wert war.\pend
           \pstart
           Bitte tun Sie es auch mir zu lieb, ich kenne den Menschen seit vielen Jahren, und
               durchaus im Guten.\pend
           \pstart
           Ein paar überaus liebe Zeilen, die Sie mir vor vielen Wochen schrieben, klingen immer
               in mir nach.\hspace*{1.5em}Soll ich, wenn es freundlicher wird, zu
               einem Vormittagsspaziergang hinüber ko{\geminationm}en?\hspace*{1.5em}Oder gibts eine andere Form der Begegnung, die Ihnen
               nicht beschwerend ist? \pend
           \pstart In Freundschaft Ihr\spacefill\mbox{Hugo.}\pend{}\pstart
           \noindent{}PS Die Sache mit \textcolor{blue}{E. Reiss}{}\ledrightnote{\textcolor{blue}{Erich Reiss}} ist, soviel ich
                  verstehe, dringend eilig!\pend
           \endnumbering\briefempfaengerindex{Schnitzler, Arthur@\textsc{Schnitzler, Arthur}!zzzHofmannsthal, Hugo von@\emph{von Hugo von Hofmannsthal}!1926-03-091@{9. 3. {[}1926{]}}|)be}\mylabel{h}  \normalsize

\doendnotes{C}
\bigskip
\vfill

\clearpage

\footnotesize

\lohead{\textsc{register}}

% Definiere theindex-Environment komplett neu ohne reledmac
\makeatletter
\renewenvironment{theindex}{%
  \section*{\indexname}%
  \setlength{\parindent}{0pt}%
  \setlength{\parskip}{0pt plus 0.3pt}%
  \let\item\@idxitem
}{%
  \clearpage
}
\makeatother

\IfFileExists{\jobname-pw.ind}{\input{\jobname-pw.ind}}{}

\end{document}

      