%% latex-korrekturansicht-vorspann.tex
%% Vorspann für die Korrekturansicht.
%% Lädt die gemeinsame Datei latex-vorspann.tex mit gesetztem Schalter.

\newif\ifkorrekturansicht
\korrekturansichttrue

\input{../tex-inputs/latex-vorspann}


               \section[Georg Brandes an Arthur Schnitzler, 10. 3. 1899]{ Georg Brandes an Arthur Schnitzler, 10. 3. 1899}\nopagebreak\mylabel{v}\rehead{ }\normalsize\beginnumbering\briefempfaengerindex{Schnitzler, Arthur@\textsc{Schnitzler, Arthur}!zzzBrandes, Georg@\emph{von Georg Brandes}!1899-03-101@{10. 3. 1899}|(be} \toendnotes[C]{\smallbreak\pagebreak[2]} \Standort{CUL, Schnitzler, B 17.}
\physDesc{Brief, 1 Blatt, 4 Seiten
\newline{}Handschrift: Bleistift, lateinische Kurrent\newline{}Ordnung: mit Bleistift von unbekannter Hand nummeriert:
                                    »14« }\buchAbdrucke{\weitereDrucke{Georg Brandes, Arthur Schnitzler: \emph{Ein Briefwechsel}. Hg. Kurt Bergel. Bern: \emph{Francke} 1956, S. 73–74.} }\toendnotes[C]{\smallbreak}\pstart
           \raggedleft{}{\pb}\textcolor{pink}{Kopenhagen}{}\ledrightnote{\textcolor{pink}{Kopenhagen}}{ }10 März 99\pend
           \pstart{}Liebster Dr. Schnitzler\pend\pstart
           Ich bin leider noch im Bett; bald sind jetzt 3 Monate so vergangen. Ich schreibe
               Ihnen nur heute weil ich \textcolor{blue}{Jemand}{}\ledrightnote{→\textcolor{blue}{Karl Larsen}}
               gestern eine Karte für Sie gab und nicht will, dass Sie sich dadurch im Geringsten
               verpflichtet glauben sollen. Es war mir nicht möglich Nein zu sagen. Es ist der \textcolor{pink}{dänische}{}\ledrightnote{\textcolor{pink}{Dänemark}} Schriftsteller \textcolor{blue}{Karl Larsen}{}\ledrightnote{\textcolor{blue}{Karl Larsen}}, ein talentvoller Mensch, gewissenhafter Psycholog,
               sehr feinhörig in allem Sprachlichen, ein wahrer Phonograph, aber langweilig, weil er
               immer nur von \uline{sich} spricht, immer nur an seinen
               litterarischen \uline{Vortheil} denkt und Kritiken und
               öffentliches Lob haben will. Sie kennen den Typus.\pend
           \pstart
           Aber er kann Ihnen jedenfalls einen {\pb}Gruss aus \textcolor{pink}{Kopenhagen}{}\ledrightnote{\textcolor{pink}{Kopenhagen}} bringen.\pend
           \pstart
           So entzückt ich war über Ihr letztes grösseres \textcolor{green}{Schaupiel}{}\ledrightnote{→\textcolor{green}{Das Vermächtnis. Schauspiel in drei Akten}} – ich entsinne mich des Titels nicht – wo der
               junge Mann im ersten Akt stirbt – so fremd ist mir der kl. \textcolor{green}{Einakter}{}\ledrightnote{→\textcolor{green}{Der grüne Kakadu. Groteske in einem Akt}} den Sie mir kürzlich schickten. Ich
               weiss ja nicht ob irgend eine historische Notiz zu Grunde liegt, sonst aber kommt die
               Idee mir sonderbar vor, dass vornehme Leute – seien sie auch noch so abgespannt –
               eine Kneipe besuchen sollten um sich von \uline{Schauspielern
                  revolutionäre Scenen vorspielen zu lassen}. Es ist so verdammt künstlich, so
                  »\label{K_L00905_1v}\edtext{ausklamüstirt}{\lemma{\textnormal{\emph{ausklamüstirt}}}\Cendnote{\textnormal{ausklamüsern: (zu sehr) im Detail
                  ausgedacht}}}\label{K_L00905_1h}«, wie die \textcolor{pink}{Norddeutschen}{}\ledrightnote{\textcolor{pink}{Deutschland}}
               sagen.\pend
           \pstart
           Sonst wissen Sie, dass ich in Sie verliebt bin und alles was Sie machen {\pb}gut finde und jede Gelegenheit
               ergreife Sie mündlich und schriftlich zu preisen.\pend
           \pstart
           Ist es nicht sonderbar? Mein so ruhiges und würdiges \label{K_L00905_2v}\edtext{\textcolor{green}{Manifest}{}\ledrightnote{→\textcolor{green}{Das Dänentum in Südjütland}}}{\lemma{\textnormal{\emph{Manifest}}}\Cendnote{\textnormal{Es erschien zuerst als \emph{\textcolor{green}{Danskheden i Sønderjylland}} In: \emph{\textcolor{green}{Tilskueren}}, Jg. 16, März 1899,
                     S. 185–199, dann als \emph{\textcolor{green}{Das Dänenthum in Südjütland}}. In: \emph{\textcolor{green}{Die Zukunft}}, Bd. 27, 8. 4. 1899,
                     S. 58–71.}}}\label{K_L00905_2h} an die Deutschen haben sowohl die \textcolor{brown}{Neue freie
                  Presse}{}\ledrightnote{\textcolor{brown}{Neue Freie Presse}} wie die \textcolor{brown}{Frankfurter Zeitung}{}\ledrightnote{\textcolor{brown}{Frankfurter Zeitung}}
               abgewiesen. Nun versuche ich mein Glück bei \textcolor{blue}{Barth}{}\ledrightnote{\textcolor{blue}{Theodor Barth}}’s \textcolor{brown}{Die Nation}{}\ledrightnote{\textcolor{brown}{Die Nation}}. Ich lasse es in allen
               Sprachen sonst erscheinen. Es ist ein Bogen gross über die \textcolor{pink}{schleswigsche}{}\ledrightnote{\textcolor{pink}{Südschleswig}} Sache.\pend
           \pstart
           Ich habe sonst wenig arbeiten können. Nur \textcolor{blue}{Annie
                  Vivanti}{}\ledrightnote{\textcolor{blue}{Annie Vivanti}} aus dem \textcolor{pink}{Italiänischen}{}\ledrightnote{\textcolor{pink}{Italien}} in \label{K_L00905_3v}\edtext{\textcolor{pink}{dänische}{}\ledrightnote{\textcolor{pink}{Dänemark}}{ }\textcolor{green}{Verse}{}\ledrightnote{→\textcolor{green}{[Gedichte]}}}{\lemma{\textnormal{\emph{dänische Verse}}}\Cendnote{\textnormal{\textcolor{blue}{Georg Brandes}: \emph{\textcolor{green}{Annie Vivanti}}. In: \emph{\textcolor{green}{Tilskueren}},
                     Jg. 16, Februar 1899, S. 107–124.}}}\label{K_L00905_3h} gebracht.\pend
           \pstart
           Sie liebenswürdiger fragten mich einmal in einem Brief: Wie sind Ihre \textcolor{green}{Verse}{}\ledrightnote{→\textcolor{green}{Ungdomsvers [Jugendgedichte]}}, sind sie gut? \textcolor{blue}{Nansen}{}\ledrightnote{\textcolor{blue}{Peter Nansen}} findet sie akademisch, ein Urtheil, das ich ein bischen
               komisch finde, denn sie {\pb}sind sehr
               persönlich, aber als \textcolor{green}{Verse}{}\ledrightnote{→\textcolor{green}{Ungdomsvers [Jugendgedichte]}} sind
               sie gut. Das Einzige auf der Welt was ich kann ist \textcolor{pink}{dänisch}{}\ledrightnote{\textcolor{pink}{Dänemark}} schreiben.\pend
           \pstart
           Ich drücke Ihre Hände. Kürzlich erfuhr ich, dass \textcolor{blue}{Goldmann}{}\ledrightnote{\textcolor{blue}{Paul Goldmann}} wieder in \textcolor{pink}{Europa}{}\ledrightnote{\textcolor{pink}{Europa}} ist. Das freut
               mich.\pend
           \pstart
           Ihr ganz ergebener{\\[\baselineskip]}\spacefill\mbox{Georg Brandes}\pend
           \leftskip=0em{}\pstart
           \noindent{}Man fängt in nächster Woche hier an, meine \textcolor{green}{Gesammelte
                     Schriften}{}\ledrightnote{\textcolor{green}{Samlede Skrifter [Gesammelte Werke]}} (!) herauszugeben und glaubt an einen Erfolg.\pend
           \endnumbering\briefempfaengerindex{Schnitzler, Arthur@\textsc{Schnitzler, Arthur}!zzzBrandes, Georg@\emph{von Georg Brandes}!1899-03-101@{10. 3. 1899}|)be}\mylabel{h}  \normalsize

\doendnotes{C}
\bigskip
\vfill

\clearpage

\footnotesize

\lohead{\textsc{register}}

% Definiere theindex-Environment komplett neu ohne reledmac
\makeatletter
\renewenvironment{theindex}{%
  \section*{\indexname}%
  \setlength{\parindent}{0pt}%
  \setlength{\parskip}{0pt plus 0.3pt}%
  \let\item\@idxitem
}{%
  \clearpage
}
\makeatother

\IfFileExists{\jobname-pw.ind}{\input{\jobname-pw.ind}}{}

\end{document}

      