%% latex-korrekturansicht-vorspann.tex
%% Vorspann für die Korrekturansicht.
%% Lädt die gemeinsame Datei latex-vorspann.tex mit gesetztem Schalter.

\newif\ifkorrekturansicht
\korrekturansichttrue

\input{../tex-inputs/latex-vorspann}


               \section[Paul Goldmann an Arthur Schnitzler, 14. 8. 1897]{ Paul Goldmann an Arthur Schnitzler, 14. 8. 1897}\nopagebreak\mylabel{v}\rehead{ }\normalsize\beginnumbering\briefempfaengerindex{Schnitzler, Arthur@\textsc{Schnitzler, Arthur}!zzzGoldmann, Paul@\emph{von Paul Goldmann}!1897-08-141@{14. 8. 1897}|(be} \toendnotes[C]{\smallbreak\pagebreak[2]} \Standort{DLA, A:Schnitzler, HS.NZ85.1.3167.}
\physDesc{Telegramm
\newline{}maschinell\newline{}Versand: 1) Stempel: »\nobreak{}14/8, M21 \textcolor{blue}{Wölfer}\nobreak{}«.  2) Stempel: »\nobreak{}\oindex{IX., Alsergrund@\textbf{IX., Alsergrund}, \emph{Bezirk (A.BZK)}|pwk}Wien 9/2 71, 14 VIII 97, 12 50N\nobreak{}«. \newline{}Ordnung: 1) mit Bleistift von unbekannter Hand datiert:
                                       »14. 8. 1897« 2) mit Bleistift von unbekannter Hand Vermerk:
                                 »71«}\toendnotes[C]{\smallbreak}\pstart{}{\pb}doctor schnitzler \textcolor{pink}{win}{}\ledrightnote{\textcolor{pink}{Wien}}\pend{}\pstart{}\textcolor{pink}{neunt bezirk frankgasze 1}{}\ledrightnote{\textcolor{pink}{Frankgasse}}.\pend{}{\bigskip}\pstart
           \noindent{}\centering{}{\pb}\textcolor{pink}{win}{}\ledrightnote{\textcolor{pink}{Wien}} fr \textcolor{pink}{muenchen}{}\ledrightnote{\textcolor{pink}{München}}
               8915 15 14/8{ }11. \pend
           \pstart
           \noindent{}montag oder dinstag{ }\label{K_L02632-1v}\edtext{komme ich}{\lemma{\textnormal{\emph{komme ich}}}\Cendnote{\textnormal{Dieses Telegramm wurde als Beilage zu Arthur Schnitzler an Richard Beer-Hofmann, 14. 8. 1897 übermittelt.}}}\label{K_L02632-1h} nach \textcolor{pink}{salzburg}{}\ledrightnote{\textcolor{pink}{Salzburg}} grusz \spacefill\mbox{= goldmann.}\pend
           \endnumbering\briefempfaengerindex{Schnitzler, Arthur@\textsc{Schnitzler, Arthur}!zzzGoldmann, Paul@\emph{von Paul Goldmann}!1897-08-141@{14. 8. 1897}|)be}\mylabel{h}  \normalsize

\doendnotes{C}
\bigskip
\vfill

\clearpage

\footnotesize

\lohead{\textsc{register}}

% Definiere theindex-Environment komplett neu ohne reledmac
\makeatletter
\renewenvironment{theindex}{%
  \section*{\indexname}%
  \setlength{\parindent}{0pt}%
  \setlength{\parskip}{0pt plus 0.3pt}%
  \let\item\@idxitem
}{%
  \clearpage
}
\makeatother

\IfFileExists{\jobname-pw.ind}{\input{\jobname-pw.ind}}{}

\end{document}

      