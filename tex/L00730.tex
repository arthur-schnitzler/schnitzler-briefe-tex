%% latex-korrekturansicht-vorspann.tex
%% Vorspann für die Korrekturansicht.
%% Lädt die gemeinsame Datei latex-vorspann.tex mit gesetztem Schalter.

\newif\ifkorrekturansicht
\korrekturansichttrue

\input{../tex-inputs/latex-vorspann}


               \section[Arthur Schnitzler an Hugo von Hofmannsthal, 4. 10. 1897]{ Arthur Schnitzler an Hugo von Hofmannsthal, 4. 10. 1897}\nopagebreak\mylabel{v}\rehead{ }\normalsize\beginnumbering\briefempfaengerindex{Hofmannsthal, Hugo von@\textsc{Hofmannsthal, Hugo von}!zzzSchnitzler, Arthur@\emph{von Arthur Schnitzler}!1897-10-042@{4. 10. 1897}|(be} \toendnotes[C]{\smallbreak\pagebreak[2]} \Standort{FDH, Hs-30885,64.}
\physDesc{Brief, 1 Blatt, 3 Seiten
\newline{}Handschrift: schwarze Tinte, deutsche Kurrent\newline{}Ordnung: von Schnitzler mutmaßlich bei der Durchsicht der Korrespondenz 1929 mit
                                    Bleistift datiert: »4/10 97« }\buchAbdrucke{\weitereDrucke{Hugo von Hofmannsthal, Arthur Schnitzler: \emph{Briefwechsel}. Hg. Therese Nickl und Heinrich Schnitzler. Frankfurt am Main: \emph{S. Fischer} 1964, S. 96.} }\toendnotes[C]{\smallbreak}\pstart
           \noindent{}{\pb}Mein lieber Hugo, ich danke Ihnen ſehr; Sie wiſſen ja, dſs es
                        i{\geminationm}er ſehr wohlthuend auf mich wirkt, we{\geminationn} mich irgendwas die Herzlichkeit unſres
                    Verhältniſſes lebhaft empfinden läßt. – Es iſt ſehr ſchrecklich geweſen; im
                    Anfang ſo ſchrecklich, {\pb}dſs ich es garnicht
                    begriffen habe. In den letzten Tagen hat es ſich raſch gemildert; beſonders ſeit
                    dem Augenblick wo ich erfahren, dſs auch Sie zwiſchen Tod und Leben
                    war. –\pend
           \pstart
           Ich habe auch zu arbeiten angefangen; d. h. ich leſe mein neues \textcolor{green}{Stück}{}\ledrightnote{→\textcolor{green}{Das Vermächtnis. Schauspiel in drei Akten}} durch und bin noch nicht drauf gekommen, wo der
                    Hauptfehler ſteckt. –\pend
           \pstart
           {\pb}Das \textcolor{green}{neue}{}\ledrightnote{→\textcolor{green}{Die Frau im Fenster}{\newline}→\textcolor{green}{Die Hochzeit der Sobeide}} was Sie geſchrieben haben möcht ich natürlich ſehr bald hören.
                    Nicht wahr, ich weiſs es gleich, wenn Sie in \textcolor{pink}{Wien}{}\ledrightnote{\textcolor{pink}{Wien}} angeko{\geminationm}en ſind? Wie lange hab
                    ich ſchon nicht mit Ihnen geſprochen!\pend
           \pstart
           Das was Sie über die \textcolor{green}{Rede von \textsc{D’Annunzio}}{}\ledrightnote{\textcolor{green}{Die Rede Gabriele d’Annunzios. Notizen von einer Reise im oberen Italien}} geſagt haben, iſt ſehr ſchön. –\pend
           \pstart
           Leben Sie wohl.\pend
           \pstart Von Herzen Ihr \spacefill\mbox{Arthur}\pend{}\pstart
           \textcolor{pink}{Wien}{}\ledrightnote{\textcolor{pink}{Wien}}{ }4. 10. 97.\pend
           \endnumbering\briefempfaengerindex{Hofmannsthal, Hugo von@\textsc{Hofmannsthal, Hugo von}!zzzSchnitzler, Arthur@\emph{von Arthur Schnitzler}!1897-10-042@{4. 10. 1897}|)be}\mylabel{h}  \normalsize

\doendnotes{C}
\bigskip
\vfill

\clearpage

\footnotesize

\lohead{\textsc{register}}

% Definiere theindex-Environment komplett neu ohne reledmac
\makeatletter
\renewenvironment{theindex}{%
  \section*{\indexname}%
  \setlength{\parindent}{0pt}%
  \setlength{\parskip}{0pt plus 0.3pt}%
  \let\item\@idxitem
}{%
  \clearpage
}
\makeatother

\IfFileExists{\jobname-pw.ind}{\input{\jobname-pw.ind}}{}

\end{document}

      