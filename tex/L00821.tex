%% latex-korrekturansicht-vorspann.tex
%% Vorspann für die Korrekturansicht.
%% Lädt die gemeinsame Datei latex-vorspann.tex mit gesetztem Schalter.

\newif\ifkorrekturansicht
\korrekturansichttrue

\input{../tex-inputs/latex-vorspann}


               \section[Richard Beer-Hofmann an Arthur Schnitzler, 17. 7. 1898]{ Richard Beer-Hofmann an Arthur Schnitzler, 17. 7. 1898}\nopagebreak\mylabel{v}\rehead{ }\normalsize\beginnumbering\briefempfaengerindex{Schnitzler, Arthur@\textsc{Schnitzler, Arthur}!zzzBeer-Hofmann, Richard@\emph{von Richard Beer-Hofmann}!1898-07-171@{17. 7. 1898}|(be} \toendnotes[C]{\smallbreak\pagebreak[2]} \Standort{CUL, Schnitzler, B 8.}
\physDesc{Brief, 1 Blatt, 3 Seiten
\newline{}Handschrift: Bleistift, lateinische Kurrent\newline{}Ordnung: mit Bleistift von unbekannter Hand nummeriert: »121« }\buchAbdrucke{\weitereDrucke{Arthur Schnitzler, Richard Beer-Hofmann: \emph{Briefwechsel 1891–1931}. Hg. Konstanze Fliedl. Wien, Zürich: \emph{Europaverlag} 1992, S. 123.} }\pstart
           \raggedleft{}{\pb}\textcolor{pink}{Steindorf}{}\ledrightnote{\textcolor{pink}{Steindorf am Ossiacher See}}{ }Sonntag 17/VII 98\pend
           \pstart
           Lieber Arthur! Brief aus \textcolor{pink}{Graz}{}\ledrightnote{\textcolor{pink}{Graz}}
               erhalten. Weiß noch gar nichts Besti{\geminationm}tes\pend
           \pstart
           \textcolor{blue}{Hugo}{}\ledrightnote{\textcolor{blue}{Hugo von Hofmannsthal}} will daß ich die 10 Tage mitmache, und dann
               mit ihm in \textcolor{pink}{Ober-Italien}{}\ledrightnote{\textcolor{pink}{Italien}}{ }{\pb}dh. an einem der Seen bleibe. Die
               10 Tage unwahrscheinlich. Eher das letztere nur wäre mir \textcolor{pink}{Venedig}{}\ledrightnote{\textcolor{pink}{Venedig}} – Seebad lieber, da \textcolor{pink}{Venedig}{}\ledrightnote{\textcolor{pink}{Venedig}}{ }\uline{6} Stunden die Seen mindestens 15–16 {\pb}Stunden weit sind\pend
           \pstart
           Bitte geben Sie mir bis zum letzten \textcolor{pink}{Salzburg}{}\ledrightnote{\textcolor{pink}{Salzburg}}er Tag
                  i{\geminationm}er Nachricht wo Sie Brief oder Telegr.
               erreicht.\pend
           \pstart
           \textcolor{blue}{Paula}{}\ledrightnote{\textcolor{blue}{Paula Beer-Hofmann}} u. \textcolor{blue}{Mirjam}{}\ledrightnote{\textcolor{blue}{Mirjam Beer-Hofmann}} dan{\pb}ken für d. Gruß
               u. erwiedern ihn. \textcolor{blue}{Mirjam}{}\ledrightnote{\textcolor{blue}{Mirjam Beer-Hofmann}} freut sich riesig wenn
               ich ihr Ihre Briefe vorlese. Schreiben Sie also oft.\pend
           \pstart
           Von Herzen{\\[\baselineskip]}\spacefill\mbox{Richard}\pend
           \leftskip=0em{}\endnumbering\briefempfaengerindex{Schnitzler, Arthur@\textsc{Schnitzler, Arthur}!zzzBeer-Hofmann, Richard@\emph{von Richard Beer-Hofmann}!1898-07-171@{17. 7. 1898}|)be}\mylabel{h}  \normalsize

\doendnotes{C}
\bigskip
\vfill

\clearpage

\footnotesize

\lohead{\textsc{register}}

% Definiere theindex-Environment komplett neu ohne reledmac
\makeatletter
\renewenvironment{theindex}{%
  \section*{\indexname}%
  \setlength{\parindent}{0pt}%
  \setlength{\parskip}{0pt plus 0.3pt}%
  \let\item\@idxitem
}{%
  \clearpage
}
\makeatother

\IfFileExists{\jobname-pw.ind}{\input{\jobname-pw.ind}}{}

\end{document}

      