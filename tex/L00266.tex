%% latex-korrekturansicht-vorspann.tex
%% Vorspann für die Korrekturansicht.
%% Lädt die gemeinsame Datei latex-vorspann.tex mit gesetztem Schalter.

\newif\ifkorrekturansicht
\korrekturansichttrue

\input{../tex-inputs/latex-vorspann}


               \section[Hermann Bahr an Arthur Schnitzler, 20. 9. 1893]{ Hermann Bahr an Arthur Schnitzler, 20. 9. 1893}\nopagebreak\mylabel{v}\rehead{ }\normalsize\beginnumbering\briefempfaengerindex{Schnitzler, Arthur@\textsc{Schnitzler, Arthur}!zzzBahr, Hermann@\emph{von Hermann Bahr}!1893-09-201@{20. 9. 1893}|(be} \toendnotes[C]{\smallbreak\pagebreak[2]} \Standort{CUL, Schnitzler, B 5b.}
\physDesc{Visitenkarte
\newline{}Handschrift: Bleistift, deutsche Kurrent
\newline{}Schnitzler: 1) mit Bleistift datiert: »20/1 \textcolor{gray}{93}« 2) mit rotem Buntstift die Monatsangabe der Bleistiftdatierung mit
                                    »9« überschrieben und nummeriert:
                                    »14« sowie ein Strich seitlich der
                                 Anrede\newline{}Ordnung: mit Bleistift von unbekannter Hand die Nummerierung mit Rotstift
                                 verdeutlicht und neuerlich nummeriert: »14« }\buchAbdrucke{\weitereDrucke{Hermann Bahr, Arthur Schnitzler: \emph{Briefwechsel, Aufzeichnungen, Dokumente (1891–1931)}. Hg. Kurt Ifkovits und Martin Anton Müller. Göttingen: \emph{Wallstein} 2018, S. 37.} }\pstart
           \noindent{}\centering{}{\pb}\textcolor{gray}{\textbf{Hermann Bahr}}\pend
           \pstart
           \noindent{}\centering{}\textcolor{gray}{\textbf{Redacteur der »\textcolor{brown}{Deutschen
                        Zeitung}{}\ledrightnote{\textcolor{brown}{Deutsche Zeitung}}«}}\pend
           \pstart
           \noindent{}\raggedleft{}\textcolor{gray}{\textbf{\textcolor{pink}{Wien, III., Salesianergasse 12}{}\ledrightnote{\textcolor{pink}{Salesianergasse}}.}}\pend
           \pstart{}{\pb}Lieber Freund!\pend\pstart
           Ich konnte leider heute vor 4 nicht frei werden, doch hoffe ich Sie
                  beſti{\geminationm}t morgen um 3 am \textcolor{pink}{Burgring}{}\ledrightnote{\textcolor{pink}{Burgring}} zu ſehen.\pend
           \pstart
           Herzlichſt\pend
           \pstart
           \raggedleft{}Ihr\pend
           \endnumbering\briefempfaengerindex{Schnitzler, Arthur@\textsc{Schnitzler, Arthur}!zzzBahr, Hermann@\emph{von Hermann Bahr}!1893-09-201@{20. 9. 1893}|)be}\mylabel{h}  \normalsize

\doendnotes{C}
\bigskip
\vfill

\clearpage

\footnotesize

\lohead{\textsc{register}}

% Definiere theindex-Environment komplett neu ohne reledmac
\makeatletter
\renewenvironment{theindex}{%
  \section*{\indexname}%
  \setlength{\parindent}{0pt}%
  \setlength{\parskip}{0pt plus 0.3pt}%
  \let\item\@idxitem
}{%
  \clearpage
}
\makeatother

\IfFileExists{\jobname-pw.ind}{\input{\jobname-pw.ind}}{}

\end{document}

      