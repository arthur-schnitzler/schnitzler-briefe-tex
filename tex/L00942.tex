%% latex-korrekturansicht-vorspann.tex
%% Vorspann für die Korrekturansicht.
%% Lädt die gemeinsame Datei latex-vorspann.tex mit gesetztem Schalter.

\newif\ifkorrekturansicht
\korrekturansichttrue

\input{../tex-inputs/latex-vorspann}


               \section[Richard Beer-Hofmann an Arthur Schnitzler, 14. 7. 1899]{ Richard Beer-Hofmann an Arthur Schnitzler, 14. 7. 1899}\nopagebreak\mylabel{v}\rehead{ }\normalsize\beginnumbering\briefempfaengerindex{Schnitzler, Arthur@\textsc{Schnitzler, Arthur}!zzzBeer-Hofmann, Richard@\emph{von Richard Beer-Hofmann}!1899-07-142@{14. 7. 1899}|(be} \toendnotes[C]{\smallbreak\pagebreak[2]} \Standort{CUL, Schnitzler, B 8.}
\physDesc{Brief, 1 Blatt, 4 Seiten
\newline{}Handschrift: Bleistift, lateinische Kurrent\newline{}Ordnung: mit Bleistift von unbekannter Hand nummeriert: »132« }\buchAbdrucke{\weitereDrucke{Arthur Schnitzler, Richard Beer-Hofmann: \emph{Briefwechsel 1891–1931}. Hg. Konstanze Fliedl. Wien, Zürich: \emph{Europaverlag} 1992, S. 132.} }\toendnotes[C]{\smallbreak}\pstart
           \centering{}{\pb}\textcolor{pink}{Seeboden}{}\ledrightnote{\textcolor{pink}{Seeboden}}{ }14/VII 99\pend
           \pstart
           Lieber Arthur! Das »Vielleicht« konnte sich doch selbstverständlich
               nur auf die gemeinschaftliche Tour beziehen. Ich wünsche – aber das ist ja
               selbstverständlich, – ich hoffe mit einer Wahrscheinlichkeit von 75{\%} daß wir in den letzten Julitagen eine gemeinschaftliche
               Tour machen können. Vielleicht daß wir von hier aus {\pb}am 25 od.
                  26 über die \textcolor{pink}{Tauern}{}\ledrightnote{\textcolor{pink}{Hohe Tauern}} nach \textcolor{pink}{Salzburg}{}\ledrightnote{\textcolor{pink}{Salzburg}}{ }\strikeout{m} gehen – dort 2 Tage bleiben (1 Tag davon muß ich nach
                  \textcolor{pink}{Ischl}{}\ledrightnote{\textcolor{pink}{Bad Ischl}}{ }\introOben{}od.\introOben{}{ }\textcolor{pink}{Aussee}{}\ledrightnote{\textcolor{pink}{Bad Aussee}}) dann nach \textcolor{pink}{Bayreuth}{}\ledrightnote{\textcolor{pink}{Bayreuth}} am 31 – und von dort \textcolor{pink}{München}{}\ledrightnote{\textcolor{pink}{München}}{ }\textcolor{pink}{Innsbruck}{}\ledrightnote{\textcolor{pink}{Innsbruck}}{ }\textcolor{pink}{Franzensfeste}{}\ledrightnote{\textcolor{pink}{Franzensfeste}}\substVorne{}\textsuperscript{–}\substDazwischen{}(\substHinten{}eventuell begleite ich Sie nach \textcolor{pink}{Bozen}{}\ledrightnote{\textcolor{pink}{Bozen}}\introOben{})\introOben{} zurück. Vorher möchte ich Sie gewiß gerne hier oder in \textcolor{pink}{Millstatt}{}\ledrightnote{\textcolor{pink}{Millstatt}} haben.\pend
           \pstart
           Meine ganze Reserve im Ausdruck datirt nur aus der Nervosi{\pb}tät Pläne zu machen, und aus der
                  zweiten\strikeout{,} Nervosität ob ich bis zu Ihrer Ankunft
                  \textcolor{green}{fertig}{}\ledrightnote{→\textcolor{green}{Der Tod Georgs}} sein werde. Ihre
               Adresse in \textcolor{pink}{Velden}{}\ledrightnote{\textcolor{pink}{Velden}} haben Sie mir noch nicht
               angegeben. Von Herzen\pend
           \pstart
           Ihr{\\[\baselineskip]}\spacefill\mbox{Richard}\pend
           \leftskip=0em{}\pstart
           \noindent{}Bitte sagen Sie \textcolor{blue}{Schwarzkopf}{}\ledrightnote{\textcolor{blue}{Gustav Schwarzkopf}} daß ich zu
                     versti{\geminationm}t war um ihm zu schreiben – ich weiß schon,
                  er wird sagen: »u wenn er nicht {\pb}versti{\geminationm}t ist schreibt er?« Aber ich lasse \substVorne{}\textsuperscript{I}\substDazwischen{}i\substHinten{}hn herzlich grüßen und ich würde mich mehr – als er glaubt – freuen wenn
                  er hieher käme.\pend
           \pstart
           – Ich \uline{habe} geschrieben »versti{\geminationm}t war«. Diese Vergangenheit ist unberechtigt.\pend
           \endnumbering\briefempfaengerindex{Schnitzler, Arthur@\textsc{Schnitzler, Arthur}!zzzBeer-Hofmann, Richard@\emph{von Richard Beer-Hofmann}!1899-07-142@{14. 7. 1899}|)be}\mylabel{h}  \normalsize

\doendnotes{C}
\bigskip
\vfill

\clearpage

\footnotesize

\lohead{\textsc{register}}

% Definiere theindex-Environment komplett neu ohne reledmac
\makeatletter
\renewenvironment{theindex}{%
  \section*{\indexname}%
  \setlength{\parindent}{0pt}%
  \setlength{\parskip}{0pt plus 0.3pt}%
  \let\item\@idxitem
}{%
  \clearpage
}
\makeatother

\IfFileExists{\jobname-pw.ind}{\input{\jobname-pw.ind}}{}

\end{document}

      