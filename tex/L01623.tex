%% latex-korrekturansicht-vorspann.tex
%% Vorspann für die Korrekturansicht.
%% Lädt die gemeinsame Datei latex-vorspann.tex mit gesetztem Schalter.

\newif\ifkorrekturansicht
\korrekturansichttrue

\input{../tex-inputs/latex-vorspann}


               \section[Arthur Schnitzler an Richard Beer-Hofmann, 13. 8. 1906]{ Arthur Schnitzler an Richard Beer-Hofmann, 13. 8. 1906}\nopagebreak\mylabel{v}\rehead{ }\normalsize\beginnumbering\briefempfaengerindex{Beer-Hofmann, Richard@\textsc{Beer-Hofmann, Richard}!zzzSchnitzler, Arthur@\emph{von Arthur Schnitzler}!1906-08-131@{13. 8. 1906}|(be} \toendnotes[C]{\smallbreak\pagebreak[2]} \Standort{YCGL, MSS 31.}
\physDesc{Bildpostkarte
\newline{}Handschrift: Bleistift, deutsche Kurrent\newline{}Versand: 1) Stempel: »\nobreak{}\oindex{Weimar@\textbf{Weimar}, \emph{Besiedelter Ort (A.BSO)}|pwk}Wei{[}mar{]}, 13. \textcolor{gray}{8}. 0\textcolor{gray}{6}, 12\nobreak{}«.  2) Stempel: »\nobreak{}\oindex{Rodaun@\textbf{Rodaun}, \emph{Teil eines besiedelten Ortes (A.BSOX)}|pwk}{[}Ro{]}daun, 14{[}. 8. 1906{]}\nobreak{}«. \newline{}Ordnung: mit Bleistift von unbekannter Hand datiert: »13. 8.« }\toendnotes[C]{\smallbreak}\pstart{}{\pb}\textsc{Dr. Richard Beer-Hofmann}\pend{}\pstart{}\textcolor{pink}{\textsc{Rodaun bei Wien}}{}\ledrightnote{\textcolor{pink}{Rodaun}}\pend{}\pstart{}\textcolor{pink}{\textsc{Liesingerstraße 1}}{}\ledrightnote{\textcolor{pink}{Liesingerstraße}}.\pend{}{\bigskip}\pstart
           \noindent{}\centering{}{\pb}\textcolor{gray}{\textbf{\textcolor{pink}{WEIMAR, SCHLOSS BELVEDÈRE,{ }\label{K_L01623_1v}\edtext{\uline{NATUR}THEATER}{\lemma{\textnormal{\emph{Naturtheater}}}\Cendnote{\textnormal{Unterstreichung des Wortteiles
                           und Markierung des ganzen Wortes von Schnitzler.}}}\label{K_L01623_1h}}{}\ledrightnote{\textcolor{pink}{Belvedere}}.}}\pend
           \pstart
           \raggedleft{}\textcolor{pink}{\textsc{Weimar}}{}\ledrightnote{\textcolor{pink}{Weimar}}, 13. 8. 906.\pend
           \pstart
           War wahrſcheinlich auch keines.\pend
           \endnumbering\briefempfaengerindex{Beer-Hofmann, Richard@\textsc{Beer-Hofmann, Richard}!zzzSchnitzler, Arthur@\emph{von Arthur Schnitzler}!1906-08-131@{13. 8. 1906}|)be}\mylabel{h}  \normalsize

\doendnotes{C}
\bigskip
\vfill

\clearpage

\footnotesize

\lohead{\textsc{register}}

% Definiere theindex-Environment komplett neu ohne reledmac
\makeatletter
\renewenvironment{theindex}{%
  \section*{\indexname}%
  \setlength{\parindent}{0pt}%
  \setlength{\parskip}{0pt plus 0.3pt}%
  \let\item\@idxitem
}{%
  \clearpage
}
\makeatother

\IfFileExists{\jobname-pw.ind}{\input{\jobname-pw.ind}}{}

\end{document}

      