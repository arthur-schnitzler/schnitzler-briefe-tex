%% latex-korrekturansicht-vorspann.tex
%% Vorspann für die Korrekturansicht.
%% Lädt die gemeinsame Datei latex-vorspann.tex mit gesetztem Schalter.

\newif\ifkorrekturansicht
\korrekturansichttrue

\input{../tex-inputs/latex-vorspann}


               \section[Robert Adam an Arthur Schnitzler, 24. 9. 1916]{ Robert Adam an Arthur Schnitzler, 24. 9. 1916}\nopagebreak\mylabel{v}\rehead{ }\normalsize\beginnumbering\briefempfaengerindex{Schnitzler, Arthur@\textsc{Schnitzler, Arthur}!zzzAdam, Robert@\emph{von Robert Adam}!1916-09-241@{24. 9. 1916}|(be} \toendnotes[C]{\smallbreak\pagebreak[2]} \Standort{DLA, A:Schnitzler, HS.NZ85.1.4230,14.}
\physDesc{Brief, 1 Blatt, 4 Seiten
\newline{}Handschrift: schwarze Tinte, deutsche Kurrent
\newline{}Schnitzler: 1) mit Bleistift beschriftet: »\textsc{Adam}« 2) mit rotem Buntstift mehrere Unterstreichungen}\Standort{Wien, Österreichische Nationalbibliothek, Cod.ser. 52.263, 177.}
\physDesc{Brief, maschinelle Abschrift
\newline{}Schreibmaschine}\toendnotes[C]{\smallbreak}\pstart
           \raggedleft{}{\pb}\textcolor{pink}{Wien}{}\ledrightnote{\textcolor{pink}{Wien}}, am 24. September 1916\pend
           \pstart{}Hochverehrter Herr Doktor!\pend\pstart
           Ich vermute Sie, nach einem ſchönen und erholungsreichen Sommer, ſchon wieder
                    nach \textcolor{pink}{Wien}{}\ledrightnote{\textcolor{pink}{Wien}} zurückgekehrt und bin, Ihrer
                    liebenswürdigen Erlaubnis eingedenk, auch schon unbeſcheiden genug, anzufragen,
                    ob ich Sie einmal durch einen Beſuch ſtören darf?\pend
           \pstart
           Mir iſt die Zeit ſeit Ende meines Urlaubs unter unausgeſetzter und ſehr
                    anſtrengender Amtsarbeit vergangen, und wenn Sie mich fragen ſollten, was ich in
                    dieſen Monaten Dichteriſches geleiſtet, ſo müßte ich ſehr kleinlaut werden. Ich
                    habe allerdings an einer {\pb}ſonderbaren \textcolor{green}{Märchenkomödie}{}\ledrightnote{→\textcolor{green}{Märchenkomödie}} zu ſchreiben
                    begonnen, aber kraft- und zuglos, gewiſſermaßen \strikeout{unter
                        de} im drückenden Bewußtſein der Unterernährtheit, nur an freien
                    Sonntagnachmittagen: und daß dabei nichts Erſprießliches herausſchauen konnte,
                    iſt gewiß klar.\pend
           \pstart
           (Dafür habe ich in den letzten Tagen ein leibliches \textcolor{blue}{Kind}{}\ledrightnote{→\textcolor{blue}{Viktor Franz Patzner}} gekriegt, einen Buben, der
                    anſcheinend gut gedeiht, und damit darf ich mich tröſten).\pend
           \pstart
           Ich bin Ihnen für viele Bücher, die Sie mir anrieten, großen Dank ſchuldig: vor
                    allem für den \textcolor{blue}{\textsc{Coster}}{}\ledrightnote{\textcolor{blue}{Charles de Coster}}’ſchen \textcolor{green}{\textsc{Uhlenspiegel}}{}\ledrightnote{\textcolor{green}{Tyll Ulenspiegel und Lamm Goedzak}} und den \textcolor{green}{\textsc{Jean-Christophe}}{}\ledrightnote{\textcolor{green}{Jean Christophe}} (ich halte ſchon beim erſten Bande). Auch den »\textcolor{green}{Deutſchen Krieg}{}\ledrightnote{\textcolor{green}{Der große Krieg in Deutschland}}« der \textcolor{blue}{\textsc{Ricarda Huch}}{}\ledrightnote{\textcolor{blue}{Ricarda Huch}} habe ich zu zwei Dritteln geleſen, mit großer Hochachtung für den
                    phantaſievollen Geiſt, der den Canvas der pragmatiſchen Geſchichtsſchreibung mit
                        {\pb}farbigen Bildern gediegenſter Ausführung
                    beſtickt hat; aber ich kann mir halt nicht helfen, ich komme über den Eindruck
                    einer – gewiß vorzüglichen und nie geſchmackloſen – Handarbeit nicht \strikeout{hinaus} hinweg, allerdings der umfangreichſten und
                    mühevollſten Handarbeit, die ich noch je geleſen habe; ich muß hinzufügen: auch
                    der originellſten.\pend
           \pstart
           Eines der Bücher von \textcolor{blue}{\textsc{Lenotre}}{}\ledrightnote{\textcolor{blue}{G. Lenotre}} (deſſen Bekanntſchaft ich auch Ihnen verdanke) leſe ich gerade: \textcolor{green}{\textsc{Bleus, Blancs + Rouges}}{}\ledrightnote{\textcolor{green}{Bleus, Blancs et Rouges}} und werde gewiß auch die andern leſen; in dem Zitierten iſt ein
                    wunderſchöner Komödienſtoff zu finden (\textcolor{green}{\textsc{Le mariage de Monsieur de Bréchard}}{}\ledrightnote{\textcolor{green}{Le mariage de Monsieur de Bréchard}}). Unangenehm berührt mich nur die prononzierte Parteinahme des Autors, der
                    ein erzkatholiſcher Royaliſt ſein muß, für jeden Antirevolutionär und gegen
                    jeden Terroriſten: die zur Folge hat, daß ſeine hiſtoriſchen Novellen nur Engel
                    und Teu{\pb}fel zu Helden haben.\pend
           \pstart
           Wegen der \textcolor{green}{Memoiren}{}\ledrightnote{→\textcolor{green}{Meine Memoiren}} von \textcolor{blue}{\textsc{Alexandre Dumas Père}}{}\ledrightnote{\textcolor{blue}{Alexandre père Dumas}} habe ich vergeblich die \textcolor{pink}{Wien}{}\ledrightnote{\textcolor{pink}{Wien}}er
                    Buchhandlungen beſucht; ich weiß ſicher, daß ich ein Exemplar bei Sommerbeginn
                    in einer Auslage ſah; es muß ſeither verkauft worden ſein. Selbſtverſtändlich
                    ſteht Ihnen, hochverehrter Herr Doktor, mein Exemplar jederzeit zur Verfügung.
                    Darf ich es Ihnen ſchicken?\pend
           \pstart
           Ich freue mich ſchon ungemein darauf, Sie wiederzuſehen: ohne Ihre Teilnahme, das
                    fühle ich, wäre ich ſchon längſt entmutigt von allen Dichterplänen abgekommen
                    und zum einfachen \textcolor{pink}{Wien}{}\ledrightnote{\textcolor{pink}{Wien}}er Bezirksrichter mit
                    einigen Gelehrſamkeitsaſpirationen geworden. Und vielleicht bringe ich, wenn nur
                    erſt dieſer Krieg vorüber iſt, doch noch etwas Anſtändiges zuwege.\pend
           \pstart
           Mit den freundlichſten Grüßen Ihr ergebener\pend
           \pstart \spacefill\mbox{Robert Adam}\pend{}\endnumbering\briefempfaengerindex{Schnitzler, Arthur@\textsc{Schnitzler, Arthur}!zzzAdam, Robert@\emph{von Robert Adam}!1916-09-241@{24. 9. 1916}|)be}\mylabel{h}  \normalsize

\doendnotes{C}
\bigskip
\vfill

\clearpage

\footnotesize

\lohead{\textsc{register}}

% Definiere theindex-Environment komplett neu ohne reledmac
\makeatletter
\renewenvironment{theindex}{%
  \section*{\indexname}%
  \setlength{\parindent}{0pt}%
  \setlength{\parskip}{0pt plus 0.3pt}%
  \let\item\@idxitem
}{%
  \clearpage
}
\makeatother

\IfFileExists{\jobname-pw.ind}{\input{\jobname-pw.ind}}{}

\end{document}

      