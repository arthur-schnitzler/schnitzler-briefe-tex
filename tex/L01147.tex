%% latex-korrekturansicht-vorspann.tex
%% Vorspann für die Korrekturansicht.
%% Lädt die gemeinsame Datei latex-vorspann.tex mit gesetztem Schalter.

\newif\ifkorrekturansicht
\korrekturansichttrue

\input{../tex-inputs/latex-vorspann}


               \section[Edith Brandes an Arthur Schnitzler, 15. 7. 1901]{ Edith Brandes an Arthur Schnitzler, 15. 7. 1901}\nopagebreak\mylabel{v}\rehead{ }\normalsize\beginnumbering\briefempfaengerindex{Schnitzler, Arthur@\textsc{Schnitzler, Arthur}!zzzPhilipp, Edith@\emph{von Edith Philipp}!1901-07-151@{15. 7. 1901}|(be} \toendnotes[C]{\smallbreak\pagebreak[2]} \Standort{CUL, Schnitzler, B 17.}
\physDesc{Brief, 1 Blatt (Briefpapier mit aufgedruckten Tauben), 3 Seiten
\newline{}Handschrift: schwarze Tinte, lateinische Kurrent\newline{}Ordnung: mit Bleistift von unbekannter Hand
                                    nummeriert: »27« }\Standort{DLA, A:Schnitzler, HS.NZ85.1.2595.}
\physDesc{1 Blatt, 1 Seite, maschinelle Abschrift}\buchAbdrucke{\weitereDrucke{Georg Brandes, Arthur Schnitzler: \emph{Ein Briefwechsel}. Hg. Kurt Bergel. Bern: \emph{Francke} 1956, S. 90.} }\toendnotes[C]{\smallbreak}\pstart
           \raggedleft{}{\pb}\textcolor{pink}{Hotel Øresund. Skodsborg}{}\ledrightnote{\textcolor{pink}{Hotel Øresund}}{\\}15-7-1901\pend
           \pstart\center{}Verehrter Herr Schnitzler!\pend\pstart
           Mit unendlicher Mühe habe ich Ihre freundlichen Zeilen dechiffrirt. Ich schäme
                    mich ein bischen mich so als Stammbuchsdame Ihnen präsentirt zu haben; aber Sie
                    nehmen die Aufgabe {\pb}zu
                    feierlich. Sie brauchen nicht Ihre Bücher zu verschreiben, auch nicht
                    geistreicher zu sein als wie Sie jeden Tag ohne Anstrengung sind. In meinem
                    Album finden sich so spirituelle Sachen, wie »Willkommen noch einmal«! und
                    ähnliches. Für eine beliebige Zeile bin ich dankbar. Es würde mir schwer fallen
                    Ihnen zu sagen, welches von Ihren Büchern mir am besten gefällt\substVorne{}\textsuperscript{,}\substDazwischen{}. –\substHinten{}{ }\substVorne{}\textsuperscript{i}\substDazwischen{}I\substHinten{}n jedem findet sich so viel Schönes.\pend
           \pstart
           {\pb}Mit besten Grüssen von
                    meinem \textcolor{blue}{Papa}{}\ledrightnote{→\textcolor{blue}{Georg Brandes}} und mir{\\[\baselineskip]}\spacefill\mbox{Edith Brandes.}\pend
           \leftskip=0em{}\endnumbering\briefempfaengerindex{Schnitzler, Arthur@\textsc{Schnitzler, Arthur}!zzzPhilipp, Edith@\emph{von Edith Philipp}!1901-07-151@{15. 7. 1901}|)be}\mylabel{h}  \normalsize

\doendnotes{C}
\bigskip
\vfill

\clearpage

\footnotesize

\lohead{\textsc{register}}

% Definiere theindex-Environment komplett neu ohne reledmac
\makeatletter
\renewenvironment{theindex}{%
  \section*{\indexname}%
  \setlength{\parindent}{0pt}%
  \setlength{\parskip}{0pt plus 0.3pt}%
  \let\item\@idxitem
}{%
  \clearpage
}
\makeatother

\IfFileExists{\jobname-pw.ind}{\input{\jobname-pw.ind}}{}

\end{document}

      