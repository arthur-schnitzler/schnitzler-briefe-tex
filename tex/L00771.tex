%% latex-korrekturansicht-vorspann.tex
%% Vorspann für die Korrekturansicht.
%% Lädt die gemeinsame Datei latex-vorspann.tex mit gesetztem Schalter.

\newif\ifkorrekturansicht
\korrekturansichttrue

\input{../tex-inputs/latex-vorspann}


               \section[Hugo von Hofmannsthal an Arthur Schnitzler, {[}30. 1. 1898{]}]{ Hugo von Hofmannsthal an Arthur Schnitzler, {[}30. 1. 1898{]}}\nopagebreak\mylabel{v}\rehead{ }\normalsize\beginnumbering\briefempfaengerindex{Schnitzler, Arthur@\textsc{Schnitzler, Arthur}!zzzHofmannsthal, Hugo von@\emph{von Hugo von Hofmannsthal}!1898-01-301@{{[}30. 1. 1898{]}}|(be} \toendnotes[C]{\smallbreak\pagebreak[2]} \Standort{CUL, Schnitzler, B 43b/1.}
\physDesc{Briefkarte
\newline{}Handschrift: Bleistift, deutsche Kurrent
\newline{}Schnitzler: mit Bleistift datiert: »30/1 98« \newline{}Ordnung: 1) mit Bleistift von unbekannter Hand nummeriert: »\strikeout{108}« 2) mit Bleistift von unbekannter Hand nummeriert:
                                    »107«}\buchAbdrucke{\weitereDrucke{Hugo von Hofmannsthal, Arthur Schnitzler: \emph{Briefwechsel}. Hg. Therese Nickl und Heinrich Schnitzler. Frankfurt am Main: \emph{S. Fischer} 1964, S. 99.} }\toendnotes[C]{\smallbreak}\pstart
           \noindent{}{\pb}lieber, ſeien {[}Sie{]} nicht bös. Sie müſſen
               miſsverſtanden haben, ich hab meinen Sitz zur \label{K_L00771_1v}\edtext{\textcolor{blue}{Landi}{}\ledrightnote{\textcolor{blue}{Camilla Landi}}}{\lemma{\textnormal{\emph{Landi}}}\Cendnote{\textnormal{\textcolor{blue}{Camilla Landi} trat am 11. 2. 1898
                  im \textcolor{pink}{Bösendorfersaal} auf. \textcolor{blue}{Schnitzler} war zu dem Zeitpunkt nicht in \textcolor{pink}{Wien} und besuchte die Vorstellung nicht.}}}\label{K_L00771_1h}{ }ſchon ſeit 10 Tagen. Ich glaube \textcolor{blue}{Richard}{}\ledrightnote{\textcolor{blue}{Richard Beer-Hofmann}} hat Sie gebeten, ich nur um 3 Sitze zur \label{K_L00771_2v}\edtext{\textcolor{green}{\textsc{première}}{}\ledrightnote{→\textcolor{green}{Freiwild. Schauspiel in 3 Akten}}}{\lemma{\textnormal{\emph{première}}}\Cendnote{\textnormal{von \emph{\textcolor{green}{Freiwild}} am 4. 2. 1898 im \textcolor{pink}{Carl-Theater}}}}\label{K_L00771_2h}.\pend
           \pstart
           {\pb}Die \textcolor{blue}{Brandes}{}\ledrightnote{\textcolor{blue}{Georg Brandes}}abende waren ſehr hübſch und haben mir ſehr viel Freude
               gemacht. Ich hoff, ich ſeh Sie bald wieder.\pend
           \pstart Ihr \spacefill\mbox{Hugo}\pend{}\endnumbering\briefempfaengerindex{Schnitzler, Arthur@\textsc{Schnitzler, Arthur}!zzzHofmannsthal, Hugo von@\emph{von Hugo von Hofmannsthal}!1898-01-301@{{[}30. 1. 1898{]}}|)be}\mylabel{h}  \normalsize

\doendnotes{C}
\bigskip
\vfill

\clearpage

\footnotesize

\lohead{\textsc{register}}

% Definiere theindex-Environment komplett neu ohne reledmac
\makeatletter
\renewenvironment{theindex}{%
  \section*{\indexname}%
  \setlength{\parindent}{0pt}%
  \setlength{\parskip}{0pt plus 0.3pt}%
  \let\item\@idxitem
}{%
  \clearpage
}
\makeatother

\IfFileExists{\jobname-pw.ind}{\input{\jobname-pw.ind}}{}

\end{document}

      