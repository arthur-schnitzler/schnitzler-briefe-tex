%% latex-korrekturansicht-vorspann.tex
%% Vorspann für die Korrekturansicht.
%% Lädt die gemeinsame Datei latex-vorspann.tex mit gesetztem Schalter.

\newif\ifkorrekturansicht
\korrekturansichttrue

\input{../tex-inputs/latex-vorspann}


               \section[Arthur Schnitzler und Leopold Kramer an Richard Beer-Hofmann, 5. 6. 1898]{ Arthur Schnitzler und Leopold Kramer an Richard Beer-Hofmann,
               5. 6. 1898}\nopagebreak\mylabel{v}\rehead{ }\normalsize\beginnumbering\briefempfaengerindex{Beer-Hofmann, Richard@\textsc{Beer-Hofmann, Richard}!zzzKramer, Leopold@\emph{von Leopold Kramer}!1898-06-051@{5. 6. 1898}|(be}\briefempfaengerindex{Beer-Hofmann, Richard@\textsc{Beer-Hofmann, Richard}!zzzSchnitzler, Arthur@\emph{von Arthur Schnitzler}!1898-06-051@{5. 6. 1898}|(be} \toendnotes[C]{\smallbreak\pagebreak[2]} \Standort{YCGL, MSS 31.}
\physDesc{Bildpostkarte
\newline{}Handschrift Arthur Schnitzler: Bleistift, deutsche Kurrent\newline{}Handschrift Leopold Kramer: Bleistift, lateinische Kurrent\newline{}Versand: Stempel: »\nobreak{}\oindex{Steindorf am Ossiacher See@\textbf{Steindorf am Ossiacher See}, \emph{http://www.geonames.org/ontologyA.ADM3}|pwk}Steindorf am Ossiacher See, {[}5{]} 6 {[}98{]}\nobreak{}«.  }\buchAbdrucke{\weitereDrucke{Arthur Schnitzler, Richard Beer-Hofmann: \emph{Briefwechsel 1891–1931}. Hg. Konstanze Fliedl. Wien, Zürich: \emph{Europaverlag} 1992, S. 117.} }\pstart{}{\pb}\textsc{Dr. Richard Beer-Hofmann}\pend{}\pstart{}\textsc{\textcolor{pink}{Steindorf}{}\ledrightnote{\textcolor{pink}{Steindorf am Ossiacher See}}}\pend{}\pstart{}\textsc{am \textcolor{pink}{Ossiacher See}{}\ledrightnote{\textcolor{pink}{Ossiacher See}}}\pend{}\pstart{}\textsc{\textcolor{pink}{Kärnthen}{}\ledrightnote{\textcolor{pink}{Kärnten}}}\pend{}{\bigskip}\pstart
           \noindent{}\centering{}\textcolor{gray}{\textbf{{\pb}Gruss aus \textcolor{pink}{Krieglach}{}\ledrightnote{\textcolor{pink}{Krieglach}}. \textcolor{blue}{P. K. Rosegger}{}\ledrightnote{\textcolor{blue}{Peter Rosegger}}’s
                     Geburtshaus. \textcolor{blue}{P. K. Rosegger}{}\ledrightnote{\textcolor{blue}{Peter Rosegger}}’s Villa}}\pend
           \pstart
           \raggedleft{}{\pb}So{\geminationn}tag.\pend
           \pstart
           Hier iſt das erſte Nachtquartier. I{\geminationm}er näher. Herzlichſt
               Ihr \spacefill\mbox{Arth}\pend
           \pstart
           \noindent{}{[}hs. Kramer:{]} D\textsuperscript{r} Schnitzler maltraitirt
               mich schrecklich\pend
           \pstart
           Ich komme luftleer nach \textcolor{pink}{Ossiach}{}\ledrightnote{\textcolor{pink}{Ossiach}} –
                  \spacefill\mbox{Kramer}\pend
           \endnumbering\briefempfaengerindex{Beer-Hofmann, Richard@\textsc{Beer-Hofmann, Richard}!zzzKramer, Leopold@\emph{von Leopold Kramer}!1898-06-051@{5. 6. 1898}|)be}\briefempfaengerindex{Beer-Hofmann, Richard@\textsc{Beer-Hofmann, Richard}!zzzSchnitzler, Arthur@\emph{von Arthur Schnitzler}!1898-06-051@{5. 6. 1898}|)be}\mylabel{h}  \normalsize

\doendnotes{C}
\bigskip
\vfill

\clearpage

\footnotesize

\lohead{\textsc{register}}

% Definiere theindex-Environment komplett neu ohne reledmac
\makeatletter
\renewenvironment{theindex}{%
  \section*{\indexname}%
  \setlength{\parindent}{0pt}%
  \setlength{\parskip}{0pt plus 0.3pt}%
  \let\item\@idxitem
}{%
  \clearpage
}
\makeatother

\IfFileExists{\jobname-pw.ind}{\input{\jobname-pw.ind}}{}

\end{document}

      