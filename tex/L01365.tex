%% latex-korrekturansicht-vorspann.tex
%% Vorspann für die Korrekturansicht.
%% Lädt die gemeinsame Datei latex-vorspann.tex mit gesetztem Schalter.

\newif\ifkorrekturansicht
\korrekturansichttrue

\input{../tex-inputs/latex-vorspann}


               \section[Arthur Schnitzler an Hermann Bahr, 27. 1. 1904]{ Arthur Schnitzler an Hermann Bahr, 27. 1. 1904}\nopagebreak\mylabel{v}\rehead{ }\normalsize\beginnumbering\briefempfaengerindex{Bahr, Hermann@\textsc{Bahr, Hermann}!zzzSchnitzler, Arthur@\emph{von Arthur Schnitzler}!1904-01-271@{27. 1. 1904}|(be} \toendnotes[C]{\smallbreak\pagebreak[2]} \Standort{TMW, HS AM 23364 Ba.}
\physDesc{Kartenbrief
\newline{}Handschrift: schwarze Tinte, deutsche Kurrent\newline{}Versand: 1) Stempel: »\nobreak{}\oindex{XVIII., Waehring@\textbf{XVIII., Währing}, \emph{Bezirk (A.BZK)}|pwk}18/1 Wien, 27. 1. 04, 11–12 V\nobreak{}«.  2) Stempel: »\nobreak{}\oindex{Wangen im Allgaeu@\textbf{Wangen im Allgäu}, \emph{Besiedelter Ort (A.BSO)}|pwk}Wangen, 29./1. 04, 9–10 V\nobreak{}«. \newline{}Ordnung: Lochung }\buchAbdrucke{\weitereDrucke{1) \emph{27. 1. 1904.} In: Arthur Schnitzler: \emph{The Letters of Arthur Schnitzler to Hermann Bahr}. Edited, annotated, and with an introduction, by Donald G.
                        Daviau. Chapel Hill: \emph{The University of North Carolina Press} 1978, S. 83 (University of North Carolina studies in the Germanic languages
                        and literatures, 89).} \weitereDrucke{2) Hermann Bahr, Arthur Schnitzler: \emph{Briefwechsel, Aufzeichnungen, Dokumente (1891–1931)}. Hg. Kurt Ifkovits und Martin Anton Müller. Göttingen: \emph{Wallstein} 2018, S. 292.} }\pstart{}{\pb}Herrn \textsc{Hermann Bahr}\pend{}\pstart{}\textsc{\textcolor{pink}{Marbach (Sanatorium)}{}\ledrightnote{\textcolor{pink}{Sanatorium Schloss Marbach am Bodensee}}}\pend{}\pstart{}\textsc{\textcolor{pink}{Radolfzell}{}\ledrightnote{\textcolor{pink}{Radolfzell am Bodensee}} am Bodensee}\pend{}{\bigskip}\pstart{}{\pb}mein lieber
                  Hermann,\pend\pstart
           möchteſt du mir ein Wort ſchreiben, wie’s dir geht? wie lang du in \textcolor{pink}{Marbach}{}\ledrightnote{\textcolor{pink}{Marbach am Bodensee}} bleiben wirſt? –\pend
           \pstart
           Anfang Feber fahre ich nach \textcolor{pink}{Berlin}{}\ledrightnote{\textcolor{pink}{Berlin}}, den \textcolor{green}{Einſamen Weg}{}\ledrightnote{\textcolor{green}{Der einsame Weg. Schauspiel in fünf Akten}} hab ich dir durch \textcolor{blue}{Fiſcher}{}\ledrightnote{\textcolor{blue}{Samuel Fischer}}{ }ſchicken laſſen!\pend
           \pstart
           Herzliche Grüße!{\\[\baselineskip]}Dein getreuer{\\[\baselineskip]}\spacefill\mbox{Arthur}\pend
           \leftskip=0em{}\pstart
           27. 1. 904.\pend
           \endnumbering\briefempfaengerindex{Bahr, Hermann@\textsc{Bahr, Hermann}!zzzSchnitzler, Arthur@\emph{von Arthur Schnitzler}!1904-01-271@{27. 1. 1904}|)be}\mylabel{h}  \normalsize

\doendnotes{C}
\bigskip
\vfill

\clearpage

\footnotesize

\lohead{\textsc{register}}

% Definiere theindex-Environment komplett neu ohne reledmac
\makeatletter
\renewenvironment{theindex}{%
  \section*{\indexname}%
  \setlength{\parindent}{0pt}%
  \setlength{\parskip}{0pt plus 0.3pt}%
  \let\item\@idxitem
}{%
  \clearpage
}
\makeatother

\IfFileExists{\jobname-pw.ind}{\input{\jobname-pw.ind}}{}

\end{document}

      