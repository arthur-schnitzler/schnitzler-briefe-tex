%% latex-korrekturansicht-vorspann.tex
%% Vorspann für die Korrekturansicht.
%% Lädt die gemeinsame Datei latex-vorspann.tex mit gesetztem Schalter.

\newif\ifkorrekturansicht
\korrekturansichttrue

\input{../tex-inputs/latex-vorspann}


               \section[Arthur Schnitzler an Hermann Bahr, 17. 3. 1930]{ Arthur Schnitzler an Hermann Bahr, 17. 3. 1930}\nopagebreak\mylabel{v}\rehead{ }\normalsize\beginnumbering\briefempfaengerindex{Bahr, Hermann@\textsc{Bahr, Hermann}!zzzSchnitzler, Arthur@\emph{von Arthur Schnitzler}!1930-03-171@{17. 3. 1930}|(be} \toendnotes[C]{\smallbreak\pagebreak[2]} \Standort{TMW, HS AM 23399 Ba.}
\physDesc{Brief, 1 Blatt, 2 Seiten
\newline{}Handschrift: schwarze Tinte, lateinische Kurrent
\newline{}Bahr: 1) mit rotem Buntstift ergänzt: »\uline{Schnitzler}« 2) mit blauem Buntstift im Text »bindet« unterstrichen}\buchAbdrucke{\weitereDrucke{1) \emph{17. 3. 1930.} In: Arthur Schnitzler: \emph{The Letters of Arthur Schnitzler to Hermann Bahr}. Edited, annotated, and with an introduction, by Donald G.
                        Daviau. Chapel Hill: \emph{The University of North Carolina Press} 1978, S. 117–118 (University of North Carolina studies in the Germanic languages
                        and literatures, 89).} \weitereDrucke{2) Hermann Bahr, Arthur Schnitzler: \emph{Briefwechsel, Aufzeichnungen, Dokumente (1891–1931)}. Hg. Kurt Ifkovits und Martin Anton Müller. Göttingen: \emph{Wallstein} 2018, S. 596.} }\toendnotes[C]{\smallbreak}\pstart
           \raggedleft{}{\pb}\textcolor{pink}{Wien}{}\ledrightnote{\textcolor{pink}{Wien}}, 17. 3. 1930.\pend
           \pstart
           Mein lieber Hermann, dein Heimweh nach \textcolor{pink}{Wien}{}\ledrightnote{\textcolor{pink}{Wien}} und das deiner verehrten \textcolor{blue}{Gattin}{}\ledrightnote{→\textcolor{blue}{Anna Bahr-Mildenburg}} hat auch mir ans Herz gegriffen, und der \textcolor{blue}{Hofrätin}{}\ledrightnote{→\textcolor{blue}{Berta Zuckerkandl}}, mit der ich \label{K_L02533_1v}\edtext{neulich}{\lemma{\textnormal{\emph{neulich}}}\Cendnote{\textnormal{am 28. 2. 1930}}}\label{K_L02533_1h} davon sprach. Aber so wenig ich den \textcolor{brown}{Nobelpreis}{}\ledrightnote{\textcolor{brown}{Bauernfeld-Preis}} kriegen werde, so wenig hab ich in \textcolor{pink}{Oesterreich}{}\ledrightnote{\textcolor{pink}{Österreich}} zu sagen, sonst hätt ich dich längst wieder ans \textcolor{pink}{Burgtheater}{}\ledrightnote{\textcolor{pink}{Burgtheater}} berufen (auf die Gefahr hin, daſs du mich wieder
               nicht aufführst, auch ohne \textcolor{blue}{Poldi}{}\ledrightnote{\textcolor{blue}{Leopold von Andrian-Werburg}}) – und wie erst
               Frau \textcolor{blue}{Mildenburg}{}\ledrightnote{\textcolor{blue}{Anna Bahr-Mildenburg}} an die \textcolor{pink}{Oper}{}\ledrightnote{\textcolor{pink}{Oper}} oder wohin sie sonst möchte, – und in der Musik geht ja meine
               Objectivität noch weiter als in der Literatur. Aber je weniger man versteht und je
               mehr man liebt, um so gerechter ist man.\pend
           \pstart
           Aber Scherz beiseite, was bindet dich eigentlich an \textcolor{pink}{München}{}\ledrightnote{\textcolor{pink}{München}}? Ich habe das Gefühl, daſs deine Leiden und – entschuldige – deine
               Hypochondrien sich hier zumindest lindern würden. Es würde viele freuen auch manche
               die nicht in allem deines Sinnes sind, Dich wieder hier zu wissen. Denn wissen wir
               überhaupt {\pb}welchen
               Sinnes wir sind. Kaum welchen Herzens. Beziehungen, auch unterbrochene, auch
               gestörte, sind das einzige reale in der seelischen Oekonomie. \label{LL141-1v}\label{LL141-1h}Wenn mir meine Vergangenheit erscheint, bist du mir immer Einer
               der nächsten, und so ka{\geminationn} es auch in der Gegenwart nicht
               anders sein. \pend
           \pstart
           \label{K_L02533_2v}\edtext{Klingt das nicht ein bischen nach fünfter
               Akt, erste Scene?}{\lemma{\textnormal{\emph{Klingt … Scene?}}}\Cendnote{\textnormal{vgl. das ähnliche Bild
                     \emph{Briefwechsel} Bahr/Schnitzler 577}}}\label{K_L02533_2h} Sagen wir: Vierter,
                  \textcolor{gray}{vor}letzte. Wir wollen nicht sentimental \introOben{}werden.\introOben{} Ich bemerke mit angemessener Kühle: Hoffentlich sieht man sich
               einmal wieder. Es wäre schön.\pend
           \pstart
           Von Herzen Dein{\\[\baselineskip]}\spacefill\mbox{Arthur}\pend
           \leftskip=0em{}\endnumbering\briefempfaengerindex{Bahr, Hermann@\textsc{Bahr, Hermann}!zzzSchnitzler, Arthur@\emph{von Arthur Schnitzler}!1930-03-171@{17. 3. 1930}|)be}\mylabel{h}  \normalsize

\doendnotes{C}
\bigskip
\vfill

\clearpage

\footnotesize

\lohead{\textsc{register}}

% Definiere theindex-Environment komplett neu ohne reledmac
\makeatletter
\renewenvironment{theindex}{%
  \section*{\indexname}%
  \setlength{\parindent}{0pt}%
  \setlength{\parskip}{0pt plus 0.3pt}%
  \let\item\@idxitem
}{%
  \clearpage
}
\makeatother

\IfFileExists{\jobname-pw.ind}{\input{\jobname-pw.ind}}{}

\end{document}

      