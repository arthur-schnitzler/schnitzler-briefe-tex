%% latex-korrekturansicht-vorspann.tex
%% Vorspann für die Korrekturansicht.
%% Lädt die gemeinsame Datei latex-vorspann.tex mit gesetztem Schalter.

\newif\ifkorrekturansicht
\korrekturansichttrue

\input{../tex-inputs/latex-vorspann}


               \section[Georg Brandes an Arthur Schnitzler, 24. 7. 1906]{ Georg Brandes an Arthur Schnitzler, 24. 7. 1906}\nopagebreak\mylabel{v}\rehead{ }\normalsize\beginnumbering\briefempfaengerindex{Schnitzler, Arthur@\textsc{Schnitzler, Arthur}!zzzBrandes, Georg@\emph{von Georg Brandes}!1906-07-242@{24. 7. 1906}|(be} \toendnotes[C]{\smallbreak\pagebreak[2]} \Standort{CUL, Schnitzler, B 17.}
\physDesc{Kartenbrief
\newline{}Handschrift: schwarze Tinte, lateinische Kurrent\newline{}Versand: 1) Stempel: »\nobreak{}\oindex{Kopenhagen@\textbf{Kopenhagen}, \emph{Besiedelter Ort (A.BSO)}|pwk}Kobenhavn, 25. 7. 06, 5–6F\nobreak{}«.  2) Stempel: »\nobreak{}\oindex{Marienlyst@\textbf{Marienlyst}, \emph{Schloss (K.SLS)}|pwk}Helsingør, 25. 7. 06, 7–9F\nobreak{}«. \newline{}Ordnung: mit Bleistift von unbekannter Hand nummeriert: »32« }\buchAbdrucke{\weitereDrucke{Georg Brandes, Arthur Schnitzler: \emph{Ein Briefwechsel}. Hg. Kurt Bergel. Bern: \emph{Francke} 1956, S. 94.} }\pstart{}{\pb}Hr. Dr. Arthur
                        Schnitzler\pend{}\pstart{}\textcolor{pink}{Marienlyst}{}\ledrightnote{\textcolor{pink}{Marienlyst}}\pend{}\pstart{}ver\pend{}\pstart{} Helsingør \pend{}{\bigskip}\pstart
           \raggedleft{}{\pb}\textcolor{pink}{Kopenhagen}{}\ledrightnote{\textcolor{pink}{Kopenhagen}}{ }24 Juli\pend
           \pstart{}Verehrter Freund\pend\pstart
           Ich möchte Sie einen Tag ein paar Stunden besuchen, wenn Sie noch in \textcolor{pink}{Marienlyst}{}\ledrightnote{\textcolor{pink}{Marienlyst}}
               sind. Ich glaube, der beste Zug
                    von hier ist der, der circa um 6 Uhr in \textcolor{pink}{Helsingør}{}\ledrightnote{\textcolor{pink}{Helsingør}} ist. Passt Ihnen das? Etwa
                    Samstag?\pend
           \pstart
           Ihr ergebener{\\[\baselineskip]}\spacefill\mbox{Georg Brandes}\pend
           \leftskip=0em{}\endnumbering\briefempfaengerindex{Schnitzler, Arthur@\textsc{Schnitzler, Arthur}!zzzBrandes, Georg@\emph{von Georg Brandes}!1906-07-242@{24. 7. 1906}|)be}\mylabel{h}  \normalsize

\doendnotes{C}
\bigskip
\vfill

\clearpage

\footnotesize

\lohead{\textsc{register}}

% Definiere theindex-Environment komplett neu ohne reledmac
\makeatletter
\renewenvironment{theindex}{%
  \section*{\indexname}%
  \setlength{\parindent}{0pt}%
  \setlength{\parskip}{0pt plus 0.3pt}%
  \let\item\@idxitem
}{%
  \clearpage
}
\makeatother

\IfFileExists{\jobname-pw.ind}{\input{\jobname-pw.ind}}{}

\end{document}

      