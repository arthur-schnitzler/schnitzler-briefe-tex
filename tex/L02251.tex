%% latex-korrekturansicht-vorspann.tex
%% Vorspann für die Korrekturansicht.
%% Lädt die gemeinsame Datei latex-vorspann.tex mit gesetztem Schalter.

\newif\ifkorrekturansicht
\korrekturansichttrue

\input{../tex-inputs/latex-vorspann}


               \section[Arthur Schnitzler an Robert Adam, 12. 12. 1916]{ Arthur Schnitzler an Robert Adam, 12. 12. 1916}\nopagebreak\mylabel{v}\rehead{ }\normalsize\beginnumbering\briefempfaengerindex{Adam, Robert@\textsc{Adam, Robert}!zzzSchnitzler, Arthur@\emph{von Arthur Schnitzler}!1916-12-121@{12. 12. 1916}|(be} \toendnotes[C]{\smallbreak\pagebreak[2]} \Standort{DLA, 96.34.1/25.}
\physDesc{Postkarte
\newline{}Handschrift: schwarze Tinte, deutsche Kurrent\newline{}Versand: Stempel: »\nobreak{}12 XI 16, \oindex{XVIII., Waehring@\textbf{XVIII., Währing}, \emph{Bezirk (A.BZK)}|pwk}18/1 Wien, 10\nobreak{}«.  }\pstart{}{\pb}\textcolor{gray}{\textbf{Dr. Arthur Schnitzler}}\pend{}\pstart{}\textcolor{gray}{\textbf{\textcolor{pink}{Wien XVIII. Sternwartestrasse 71}{}\ledrightnote{\textcolor{pink}{Sternwartestraße}}}}\pend{}{\bigskip}\pstart{}Hrn \textsc{Dr. Rob. Adam Pollak}\pend{}\pstart{}\textcolor{pink}{Wien XII}{}\ledrightnote{\textcolor{pink}{XII., Meidling}}.\pend{}\pstart{}\textcolor{pink}{\textsc{Meidlinger Hptstr} 56}{}\ledrightnote{\textcolor{pink}{Meidlinger Hauptstraße}}\pend{}{\bigskip}\pstart
           \raggedleft{}{\pb}12. 12. 916\pend
           \pstart{}Verehrter Herr Doktor,\pend\pstart
           ich freue mich über die guten Nachrichten aus \textcolor{pink}{München}{}\ledrightnote{\textcolor{pink}{München}}; hoffentlich haben ſie auch gute Folge. Für morgen
                        Mittwoch muſs ich Ihnen leider abſagen. Die Abende der ganzen
                    Woche hab ich beſetzt und ſchreibe Ihnen ſobald als möglich, wa{\geminationn} ich Sie bei mir ſehen kann.\pend
           \pstart Herzlichſt grüßend Ihr ergebner \spacefill\mbox{A. S.}\pend{}\endnumbering\briefempfaengerindex{Adam, Robert@\textsc{Adam, Robert}!zzzSchnitzler, Arthur@\emph{von Arthur Schnitzler}!1916-12-121@{12. 12. 1916}|)be}\mylabel{h}  \normalsize

\doendnotes{C}
\bigskip
\vfill

\clearpage

\footnotesize

\lohead{\textsc{register}}

% Definiere theindex-Environment komplett neu ohne reledmac
\makeatletter
\renewenvironment{theindex}{%
  \section*{\indexname}%
  \setlength{\parindent}{0pt}%
  \setlength{\parskip}{0pt plus 0.3pt}%
  \let\item\@idxitem
}{%
  \clearpage
}
\makeatother

\IfFileExists{\jobname-pw.ind}{\input{\jobname-pw.ind}}{}

\end{document}

      