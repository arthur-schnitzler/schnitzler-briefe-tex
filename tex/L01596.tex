%% latex-korrekturansicht-vorspann.tex
%% Vorspann für die Korrekturansicht.
%% Lädt die gemeinsame Datei latex-vorspann.tex mit gesetztem Schalter.

\newif\ifkorrekturansicht
\korrekturansichttrue

\input{../tex-inputs/latex-vorspann}


               \section[Hugo von Hofmannsthal an Arthur Schnitzler, 7. 5. 1906]{ Hugo von Hofmannsthal an Arthur Schnitzler, 7. 5. 1906}\nopagebreak\mylabel{v}\rehead{ }\normalsize\beginnumbering\briefempfaengerindex{Schnitzler, Arthur@\textsc{Schnitzler, Arthur}!zzzHofmannsthal, Hugo von@\emph{von Hugo von Hofmannsthal}!1906-05-071@{7. 5. 1906}|(be} \toendnotes[C]{\smallbreak\pagebreak[2]} \Standort{CUL, Schnitzler, B 43.}
\physDesc{Postkarte
\newline{}Handschrift: schwarze Tinte, deutsche Kurrent\newline{}Versand: 1) Stempel: »\nobreak{}\oindex{Rodaun@\textbf{Rodaun}, \emph{Teil eines besiedelten Ortes (A.BSOX)}|pwk}Rodaun\nobreak{}«.  2) Stempel: »\nobreak{}\oindex{XVIII., Waehring@\textbf{XVIII., Währing}, \emph{Bezirk (A.BZK)}|pwk}18/1 Wien 110, 8. V. 06, VIII, Bestellt\nobreak{}«. 3) mit Bleistift von unbekannter Hand die verwischte Bezirksnummer in der Adressierung daneben ein weiteres Mal geschrieben
\newline{}Schnitzler: mit Bleistift datiert: »7/5 906« \newline{}Ordnung: 1) mit Bleistift von unbekannter Hand nummeriert:
                                       »16\textcolor{gray}{6}« 2) mit Bleistift von unbekannter Hand nummeriert:
                                    »162«}\buchAbdrucke{\weitereDrucke{Hugo von Hofmannsthal, Arthur Schnitzler: \emph{Briefwechsel}. Hg. Therese Nickl und Heinrich Schnitzler. Frankfurt am Main: \emph{S. Fischer} 1964, S. 219.} }\pstart{}{\pb}\textsc{Herrn D\textsuperscript{r} Arthur Schnitzler}\pend{}\pstart{}\textsc{\textcolor{pink}{Wien}{}\ledrightnote{\textcolor{pink}{Wien}}}\pend{}\pstart{}\textsc{\textcolor{pink}{\damage{\textcolor{gray}{XVIII}} Spöttelgasse 7}{}\ledrightnote{\textcolor{pink}{Edmund-Weiß-Gasse}}}\pend{}\pstart{}nächſt der \textsc{\textcolor{pink}{Türkenschanzstrasse}{}\ledrightnote{\textcolor{pink}{Türkenschanzstrasse}}}\pend{}{\bigskip}\pstart
           \raggedleft{}{\pb}Montag\pend
           \pstart
           Wollte nur ſagen: das wäre abſcheulich wenn Ihr vielleicht in der \textcolor{pink}{Brühl}{}\ledrightnote{\textcolor{pink}{Brühl}}{ }ſitzt, und man wüßte es nicht. Überhaupt: sollte
               ich ein Wort auf ſie prägen – ſo wäre es: Nervenkaſperle.\pend
           \pstart
           Die \textcolor{blue}{Olga}{}\ledrightnote{\textcolor{blue}{Olga Schnitzler}} ist eine singende \textcolor{blue}{Trieſch}{}\ledrightnote{\textcolor{blue}{Irene Triesch}}, zufällig ohne Hände geboren.\pend
           \pstart Ihr\spacefill\mbox{Hugo.}\pend{}\endnumbering\briefempfaengerindex{Schnitzler, Arthur@\textsc{Schnitzler, Arthur}!zzzHofmannsthal, Hugo von@\emph{von Hugo von Hofmannsthal}!1906-05-071@{7. 5. 1906}|)be}\mylabel{h}  \normalsize

\doendnotes{C}
\bigskip
\vfill

\clearpage

\footnotesize

\lohead{\textsc{register}}

% Definiere theindex-Environment komplett neu ohne reledmac
\makeatletter
\renewenvironment{theindex}{%
  \section*{\indexname}%
  \setlength{\parindent}{0pt}%
  \setlength{\parskip}{0pt plus 0.3pt}%
  \let\item\@idxitem
}{%
  \clearpage
}
\makeatother

\IfFileExists{\jobname-pw.ind}{\input{\jobname-pw.ind}}{}

\end{document}

      