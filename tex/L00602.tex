%% latex-korrekturansicht-vorspann.tex
%% Vorspann für die Korrekturansicht.
%% Lädt die gemeinsame Datei latex-vorspann.tex mit gesetztem Schalter.

\newif\ifkorrekturansicht
\korrekturansichttrue

\input{../tex-inputs/latex-vorspann}


               \section[Arthur Schnitzler an Georg Brandes, 8. 10. 1896]{ Arthur Schnitzler an Georg Brandes, 8. 10. 1896}\nopagebreak\mylabel{v}\rehead{ }\normalsize\beginnumbering\briefempfaengerindex{Brandes, Georg@\textsc{Brandes, Georg}!zzzSchnitzler, Arthur@\emph{von Arthur Schnitzler}!1896-10-081@{8. 10. 1896}|(be} \toendnotes[C]{\smallbreak\pagebreak[2]} \Standort{Kopenhagen, Det Kongelige Bibliotek, Georg Brandes Arkiv, box 125.}
\physDesc{Brief, 1 Blatt, 3 Seiten
\newline{}Handschrift: schwarze Tinte, deutsche Kurrent\newline{}Ordnung: mit Bleistift von unbekannter Hand auf der ersten Seite: »Schnitzler« vermerkt und nummeriert: »5« }\buchAbdrucke{\weitereDrucke{Georg Brandes, Arthur Schnitzler: \emph{Ein Briefwechsel}. Hg. Kurt Bergel. Bern: \emph{Francke} 1956, S. 58.} }\toendnotes[C]{\smallbreak}\pstart
           \raggedleft{}{\pb}8. X\damage{.} 96. \textcolor{pink}{Wien}{}\ledrightnote{\textcolor{pink}{Wien}}.\pend
           \pstart\center{}Verehrteſter Herr Brandes,\pend\pstart
           der vollſtändige Titel des Buches lautet:\pend
           \pstart
           Georg Brandes, \textcolor{green}{Aus dem Reiche des
                            Abſolutismus{[}.{]} Charakterbilder aus \strikeout{dem} Leben, Politik, Sitten, Kunſt, Literatur
                        Rußlands}{}\ledrightnote{\textcolor{green}{Eindrücke aus Russland}}. Überſetzt von \textcolor{blue}{\textsc{Alfred Forster}}{}\ledrightnote{\textcolor{blue}{Alfred Forster}}.\pend
           \pstart
           \textsc{Leipzig}, bei \textcolor{brown}{\textsc{Siegismund u Volkening}}{}\ledrightnote{\textcolor{brown}{Siegismund u Volkening}}.\pend
           \pstart
           Was den \textcolor{green}{Artikel über die Cenſur in
                            \textcolor{pink}{Polen}{}\ledrightnote{\textcolor{pink}{Polen}}}{}\ledrightnote{→\textcolor{green}{Censur in Polen}} anbelangt, ſo werden
                    freilich wenige auf die Vermuthung ko{\geminationm}en, daſs er
                    aus einem {\pb}zehn Jahre alten Buch
                    herausgeſchrieben iſt, – und ich möchte annehmen, daſs das auch der Redaction
                    der \textcolor{brown}{Zeit}{}\ledrightnote{\textcolor{brown}{Die Zeit. Wiener Wochenschrift}} nicht bekannt war, von der Sie
                    übrigens \label{K_L00602_1v}\edtext{perſönlich Aufklärung}{\lemma{\textnormal{\emph{perſönlich Aufklärung}}}\Cendnote{\textnormal{Der Brief \textcolor{blue}{Hermann Bahr}s an \textcolor{blue}{Brandes} ist
                        abgedruckt in \textcolor{blue}{Hermann Bahr}, \textcolor{blue}{Arthur Schnitzler}: \emph{Briefwechsel,
                                Aufzeichnungen, Dokumente}. Hg. Kurt Ifkovits und Martin
                            Anton Müller. Göttingen: \emph{Wallstein}{ }2018, S. 127.}}}\label{K_L00602_1h} beko{\geminationm}en ſollen. Ich ſagte Ihnen ſchon im Sommer, daſs
                    man bei uns u. wohl auch in \textcolor{pink}{Deutſchland}{}\ledrightnote{\textcolor{pink}{Deutschland}} keine
                    rechte Vorſtellung davon hat, in welcher Art Überſetzungen Ihrer Werke
                    verfertigt und in welcher Art ſie ausgenutzt werden. Vielfach iſt ſogar die
                    Anſicht verbreitet, daſs Sie selbſt auch deutſche Artikel ſchreiben und manche
                    Ihrer Sachen ſelbſt aus dem \textcolor{pink}{däniſchen}{}\ledrightnote{\textcolor{pink}{Dänemark}} ins
                    deutſche übertragen.\pend
           \pstart
           {\pb}All dies ſcheint Ihnen zuweilen doch
                    ärgerlich zu ſein; aber ich erinnere mich nicht, daſs Sie ſich irgend einmal
                    dagegen öffentlich verwahrt haben.\pend
           \pstart
           Wäre es nicht doch ſchön und gut, wenn Sie das gelegentlich einmal thäten – nicht
                    um Ihretwillen – aber um der allgemeinen Bedeutung willen, welche Fragen des
                    literariſchen Rechts und des literariſchen Anſtands zukommt. –\pend
           \pstart
           Verfügen Sie jederzeit über mich und ſeien Sie versichert, daſs ich dem Künſtler
                    und dem Menſchen gleich herzlich ergeben bin.\pend
           \pstart Der Ihre mit vielen Grüßen \spacefill\mbox{ArtSchnitzler}\pend{}\endnumbering\briefempfaengerindex{Brandes, Georg@\textsc{Brandes, Georg}!zzzSchnitzler, Arthur@\emph{von Arthur Schnitzler}!1896-10-081@{8. 10. 1896}|)be}\mylabel{h}  \normalsize

\doendnotes{C}
\bigskip
\vfill

\clearpage

\footnotesize

\lohead{\textsc{register}}

% Definiere theindex-Environment komplett neu ohne reledmac
\makeatletter
\renewenvironment{theindex}{%
  \section*{\indexname}%
  \setlength{\parindent}{0pt}%
  \setlength{\parskip}{0pt plus 0.3pt}%
  \let\item\@idxitem
}{%
  \clearpage
}
\makeatother

\IfFileExists{\jobname-pw.ind}{\input{\jobname-pw.ind}}{}

\end{document}

      