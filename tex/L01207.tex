%% latex-korrekturansicht-vorspann.tex
%% Vorspann für die Korrekturansicht.
%% Lädt die gemeinsame Datei latex-vorspann.tex mit gesetztem Schalter.

\newif\ifkorrekturansicht
\korrekturansichttrue

\input{../tex-inputs/latex-vorspann}


               \section[Hermann Bahr an Arthur Schnitzler, 8. 3. {[}1902{]}]{ Hermann Bahr an Arthur Schnitzler, 8. 3. {[}1902{]}}\nopagebreak\mylabel{v}\rehead{ }\normalsize\beginnumbering\briefempfaengerindex{Schnitzler, Arthur@\textsc{Schnitzler, Arthur}!zzzBahr, Hermann@\emph{von Hermann Bahr}!1902-03-081@{8. 3. {[}1902{]}}|(be} \toendnotes[C]{\smallbreak\pagebreak[2]} \Standort{CUL, Schnitzler, B 5b.}
\physDesc{Brief, 1 Blatt, 1 Seite
\newline{}Handschrift: schwarze Tinte, deutsche Kurrent\newline{}Ordnung: mit Bleistift von unbekannter Hand nummeriert: »86« }\buchAbdrucke{\weitereDrucke{Hermann Bahr, Arthur Schnitzler: \emph{Briefwechsel, Aufzeichnungen, Dokumente (1891–1931)}. Hg. Kurt Ifkovits und Martin Anton Müller. Göttingen: \emph{Wallstein} 2018, S. 227.} }\pstart
           \noindent{}\centering{}{\pb}\textcolor{gray}{\textbf{\textcolor{brown}{Redaktion des Neuen Wiener
                           Tagblatt}{}\ledrightnote{\textcolor{brown}{Neues Wiener Tagblatt}}}}\pend
           \pstart
           \noindent{}\centering{}\textcolor{gray}{\textbf{\textsc{\textcolor{pink}{Wien, I., Rothenturmstrasse,
                        Steyrerhof}{}\ledrightnote{\textcolor{pink}{Steyrerhof}}.}}}\pend
           \pstart
           \noindent{}\centering{}\textcolor{gray}{\textbf{Telegramm-Adresse: \textcolor{brown}{Tagblatt}{}\ledrightnote{\textcolor{brown}{Neues Wiener Tagblatt}}, \textcolor{pink}{Steyrerhof, Wien}{}\ledrightnote{\textcolor{pink}{Steyrerhof}}. –
                     Telephon Nr. 384. Staats-Telephon Nr. 36.}}\pend
           \pstart
           8. März\pend
           \pstart\center{}Lieber Arthur!\pend\pstart
           Ein paar Mädeln quälen mich um eine Photographie von Dir mit Unterſchrift oder »ein
               paar Bücher mit Widmung«, als Gewinnſte bei einer Tombola im \textcolor{brown}{Concordiaclub}{}\ledrightnote{\textcolor{brown}{Concordia}} am 16. d.{ }ſchick ſie mir oder
               ſchreib mir nein, damit ich nur das Gezwitſcher los bin.\pend
           \pstart
           Herzlichſt{\\[\baselineskip]}Dein{\\[\baselineskip]}\spacefill\mbox{HermannBahr}\pend
           \leftskip=0em{}\endnumbering\briefempfaengerindex{Schnitzler, Arthur@\textsc{Schnitzler, Arthur}!zzzBahr, Hermann@\emph{von Hermann Bahr}!1902-03-081@{8. 3. {[}1902{]}}|)be}\mylabel{h}  \normalsize

\doendnotes{C}
\bigskip
\vfill

\clearpage

\footnotesize

\lohead{\textsc{register}}

% Definiere theindex-Environment komplett neu ohne reledmac
\makeatletter
\renewenvironment{theindex}{%
  \section*{\indexname}%
  \setlength{\parindent}{0pt}%
  \setlength{\parskip}{0pt plus 0.3pt}%
  \let\item\@idxitem
}{%
  \clearpage
}
\makeatother

\IfFileExists{\jobname-pw.ind}{\input{\jobname-pw.ind}}{}

\end{document}

      