%% latex-korrekturansicht-vorspann.tex
%% Vorspann für die Korrekturansicht.
%% Lädt die gemeinsame Datei latex-vorspann.tex mit gesetztem Schalter.

\newif\ifkorrekturansicht
\korrekturansichttrue

\input{../tex-inputs/latex-vorspann}


               \section[Arthur Schnitzler an Richard Beer-Hofmann, 21. 1. 1924]{ Arthur Schnitzler an Richard Beer-Hofmann, 21. 1. 1924}\nopagebreak\mylabel{v}\rehead{ }\normalsize\beginnumbering\briefempfaengerindex{Beer-Hofmann, Richard@\textsc{Beer-Hofmann, Richard}!zzzSchnitzler, Arthur@\emph{von Arthur Schnitzler}!1924-01-211@{21. 1. 1924}|(be} \toendnotes[C]{\smallbreak\pagebreak[2]} \Standort{YCGL, MSS 31.}
\physDesc{Brief, 1 Blatt, 1 Seite
\newline{}Schreibmaschine
\newline{}Handschrift: Bleistift, lateinische Kurrent (\noindent{}Anrede und Schlussformel)\newline{}Beilagen: 1) maschinschriftlicher Durchschlag: 1 Blatt, 2 Seiten 2) maschinschriftlicher Durchschlag: 1 Blatt, 2 Seiten, mit
                                 handschriftlichen Korrekturen in schwarzer Tinte\newline{}Ordnung: 1) mit Bleistift von unbekannter Hand auf der ersten Beilage die
                                 Zugehörigkeit festgehalten: »(zu
                                    21. 1. 24)« 2) mit Bleistift von unbekannter Hand auf der zweiten Beilage die
                                 Zugehörigkeit festgehalten: »Beilage zum Brief an
                                    Beer-Hofmann (21. 1. 24)« und die
                                 fehlendende Unterschrift in eckiger Klammer ergänzt: »Arthur
                                    Schnitzler«}\toendnotes[C]{\smallbreak}\pstart
           \raggedleft{}{\pb}\textcolor{pink}{Wien}{}\ledrightnote{\textcolor{pink}{Wien}}, 21. 1. 1924.\pend
           \pstart\center{}{[}hs.:{]} lieber Richard,\pend\pstart
           {[}ms.:{]} Beifolgend lege ich zwei Blätter bei, das eine
               enthält die Antwort des \textcolor{blue}{Bundestheaterkommissärs}{}\ledrightnote{→\textcolor{blue}{Albert Renkin}} auf unser erstes an die \textcolor{brown}{Staatstheaterkasse}{}\ledrightnote{\textcolor{brown}{Bundestheaterkassen}} gerichtetes Schreiben, das zweite Blatt die Erwiderung,
               die ich dem \textcolor{blue}{Bundestheaterkommissär}{}\ledrightnote{→\textcolor{blue}{Albert Renkin}} zuzusenden vorschlage. Wenn Sie sich damit
               einverstanden erklären, ersuche ich um Unterzeichnung und Rücksendung an mich.\pend
           \pstart
           {[}hs.:{]} Herzlichst{\\[\baselineskip]}Ihr{\\[\baselineskip]}\spacefill\mbox{A.}\pend
           \leftskip=0em{}{\bigskip}\pstart
           \noindent{}\raggedleft{}{\pb}\textcolor{brown}{Bundesministerium für Unterricht}{}\ledrightnote{\textcolor{brown}{Ministerium für Unterricht}}\pend
           \pstart
           \noindent{}\raggedleft{}\textcolor{blue}{Bundestheater-Kommissär}{}\ledrightnote{→\textcolor{blue}{Albert Renkin}}\pend
           \pstart
           \noindent{}Zahl 2831/1923\hfill 3. Jänner 1924.\pend
           \pstart
           Herrn\pend
           \leftskip=3em{}\pstart
           \noindent{}Dr. \textcolor{blue}{Raoul Auernheimer}{}\ledrightnote{\textcolor{blue}{Raoul Auernheimer}}\pend
           \leftskip=0em{}\leftskip=3em{}\pstart
           \uline{\textcolor{pink}{Wien}{}\ledrightnote{\textcolor{pink}{Wien}}.}\pend
           \leftskip=0em{}\pstart
           Auf die Zuschrift vom 20. Dezember 1923 beehrt sich der \textcolor{blue}{Bundestheater-Kommissär}{}\ledrightnote{→\textcolor{blue}{Albert Renkin}}
               mitzuteilen, dass die gesonderte Aufstellung von Kasseneinnahmen und
               Abonnementsquoten in den Tantiemenabrechnungen der früheren Jahre auf Grund damals
               üblicher Tantiemenverträge erfolgte. Eine derartige Trennung macht jedoch dermalen
               einerseits der gegenwärtig in Verwendung stehende Tantiemenvertrag, nach welchem der
               Tantiemenberechnung einheitlich die aus den Tageseingängen und den
               Abonnementsvergütungen sich ergebende Summe zu Grunde gelegt wird, andererseits die
               gegenüber früher geänderte Art der Verrechnung der Abonnementsbeträge überflüssig,
                  {\pb}indem sie nicht mehr eine den Durchschnitt
               darstellende \so{fixe} Abonnementsquote, sondern die
               Abonnementsbeträge in ihrer vollen Höhe in die Einnahmen einbezogen werden.\pend
           \pstart
           Betreffs der Frage bezüglich der Lustbarkeitssteuer und eventueller sonstiger Abgaben
               wolle zur Kenntnis genommen werden, dass von den Gesammteinnahmen die
               Pensionszuschläge und die Lustbarkeitssteuer in Abzug gebracht und von der so
               verbleibenden Einnahmensumme die Tantiemen berechnet werden. Andere Abzüge finden
               nicht statt.\pend
           \pstart
           Der \textcolor{blue}{Bundestheater-Kommissär}{}\ledrightnote{→\textcolor{blue}{Albert Renkin}}
               ersucht, die Herren Dr. Beer-Hofmann, Dr. ARTHUR Schnitzler,
               Dr. \textcolor{blue}{Karl Schönherr}{}\ledrightnote{\textcolor{blue}{Karl Schönherr}} und \textcolor{blue}{Franz Werfel}{}\ledrightnote{\textcolor{blue}{Franz Werfel}} hievon in Kenntnis zu setzen.\pend
           \pstart
           Für den \textcolor{blue}{Bundestheater-Kommissär}{}\ledrightnote{→\textcolor{blue}{Albert Renkin}}: \strikeout{Dr Ernst}{\\}\spacefill\mbox{Dr. \textcolor{blue}{Eckmann}{}\ledrightnote{\textcolor{blue}{Alfred Eckmann}}}\pend
           {\bigskip}\pstart
           \raggedleft{}{\pb}21. 1. 1924.\pend
           \pstart
           An den\pend
           \leftskip=3em{}\pstart
           \noindent{}\textcolor{blue}{Bundestheater-Kommissär}{}\ledrightnote{→\textcolor{blue}{Albert Renkin}}\pend
           \leftskip=0em{}\leftskip=3em{}\pstart
           \textcolor{brown}{Bundesministerium für Unterricht}{}\ledrightnote{\textcolor{brown}{Ministerium für Unterricht}}\pend
           \leftskip=0em{}\pstart
           \noindent{}Zahl 2831/1923.\hfill \uline{\textcolor{pink}{Wien}{}\ledrightnote{\textcolor{pink}{Wien}}.}\pend
           \pstart
           Die Beantwortung unserer an die \textcolor{brown}{Staatstheaterkasse}{}\ledrightnote{\textcolor{brown}{Bundestheaterkassen}}
               gerichtete Anfrage bestätigen wir dankend und erlauben uns Folgendes zu bemerken.\pend
           \pstart
           Die Bestimmungen über die Tantiemenauszahlung resp. -Verrechnung erscheinen in den
               gegenwärtigen Verträgen gegenüber den früheren, die keine eigentlichen Verträge,
               sondern sogenannte Tantiemenreverse waren, kaum geändert. Doch da nach jenen früheren
               Verträgen eine fixe Abonnementsquote galt, jetzt aber, wie der Herr \textcolor{blue}{Bundestheaterkommissär}{}\ledrightnote{→\textcolor{blue}{Albert Renkin}} schreibt, die
               Abonnementsbeträge in ihrer vollen Höhe in die Einnahmen einbezogen werden, so wäre
               gerade jetzt eine getrennte Aufstellung von Tageseinnahmen und Abonnementsquote
               vorzuziehen; wie ja auch früher in den Tantiemenabrechnungen für den Autor bei jeder
               Vorstellung die Tageseinnahme und die fixe Abonnementsquote getrennt figurierten.\pend
           \pstart
           Da ja auch der \textcolor{brown}{Burgtheaterdirektion}{}\ledrightnote{\textcolor{brown}{Burgtheater}} allabendlich
               eine nach Abonnementsquote und Tageseinnahme getrennte Verrechnung vorgelegt wird,
               erwächst für die Kassagebahrung nicht die geringste Schwierigkeit oder Mühe dadurch
               dass sie, wie es eben früher der Fall war, den Autoren die gleiche Verrechnung zu{\pb}gänglich machte.\pend
           \pstart
           Zu der Frage \substVorne{}\textsuperscript{der}\substDazwischen{}eines\substHinten{} Pensionsabz\substVorne{}\textsuperscript{ü}\substDazwischen{}u\substHinten{}g\substVorne{}\textsuperscript{e}\substDazwischen{}es\substHinten{} von den Tantiemen\strikeout{von den Tantiemen}, \substVorne{}\textsuperscript{die}\substDazwischen{}der\substHinten{} unseres Wissens an anderen Theatern \strikeout{von den
                  Tantiemen} nicht stattfinde\substVorne{}\textsuperscript{n}\substDazwischen{}t\substHinten{}, behalten wir uns eine Aeusserung vor, sobald wir über die Höhe der
               Pensionszuschläge und Höhe der Lustbarkeitssteuer den bereits in unserem vorigen
               Schreiben erbetenen Aufschluss erhalten haben.\pend
           \endnumbering\briefempfaengerindex{Beer-Hofmann, Richard@\textsc{Beer-Hofmann, Richard}!zzzSchnitzler, Arthur@\emph{von Arthur Schnitzler}!1924-01-211@{21. 1. 1924}|)be}\mylabel{h}  \normalsize

\doendnotes{C}
\bigskip
\vfill

\clearpage

\footnotesize

\lohead{\textsc{register}}

% Definiere theindex-Environment komplett neu ohne reledmac
\makeatletter
\renewenvironment{theindex}{%
  \section*{\indexname}%
  \setlength{\parindent}{0pt}%
  \setlength{\parskip}{0pt plus 0.3pt}%
  \let\item\@idxitem
}{%
  \clearpage
}
\makeatother

\IfFileExists{\jobname-pw.ind}{\input{\jobname-pw.ind}}{}

\end{document}

      