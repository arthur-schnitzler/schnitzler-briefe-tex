%% latex-korrekturansicht-vorspann.tex
%% Vorspann für die Korrekturansicht.
%% Lädt die gemeinsame Datei latex-vorspann.tex mit gesetztem Schalter.

\newif\ifkorrekturansicht
\korrekturansichttrue

\input{../tex-inputs/latex-vorspann}


               \section[Hugo von Hofmannsthal an Arthur Schnitzler, 23. 2. 1892]{ Hugo von Hofmannsthal an Arthur Schnitzler, 23. 2. 1892}\nopagebreak\mylabel{v}\rehead{ }\normalsize\beginnumbering\briefempfaengerindex{Schnitzler, Arthur@\textsc{Schnitzler, Arthur}!zzzHofmannsthal, Hugo von@\emph{von Hugo von Hofmannsthal}!1892-02-231@{23. 2. 1892}|(be} \toendnotes[C]{\smallbreak\pagebreak[2]} \Standort{CUL, Schnitzler, B 43.}
\physDesc{Postkarte
\newline{}Handschrift: Bleistift, deutsche Kurrent\newline{}Versand: 1) Stempel: »\nobreak{}Wien 3/3 40, 24. 2. 92, 7–8V\nobreak{}«.  2) Stempel: »\nobreak{}Wien, 24. 2. 92, 10½–12V\nobreak{}«. 
\newline{}Schnitzler: mit Bleistift auf der Anschriftenseite: »24/2 92« und auf der Textseite datiert: »2\strikeout{4}3. 2. 92« \newline{}Ordnung: von unbekannter Hand nummeriert: »18« }\buchAbdrucke{\weitereDrucke{Hugo von Hofmannsthal, Arthur Schnitzler: \emph{Briefwechsel}. Hg. Therese Nickl und Heinrich Schnitzler. Frankfurt am Main: \emph{S. Fischer} 1964, S. 16.} }\toendnotes[C]{\smallbreak}\pstart{}{\pb}Herrn \textsc{D\textsuperscript{r} Arthur Schnitzler}\pend{}\pstart{}\textsc{\textcolor{pink}{Wien}{}\ledrightnote{\textcolor{pink}{Wien}}}\pend{}\pstart{}\textsc{\textcolor{pink}{I Kärnthner\strikeout{strasse}ring 12}{}\ledrightnote{\textcolor{pink}{Kärntnerring}}}\pend{}{\bigskip}\pstart
           \raggedleft{}{\pb}\label{K_L00075_1v}\edtext{Dienstag}{\lemma{\textnormal{\emph{Dienstag}}}\Cendnote{\textnormal{\textcolor{blue}{Hofmannsthal} schrieb die Karte unmittelbar
                     nach dem Besuch von \emph{\textcolor{green}{Feodora}}, dem zweiten
                     Auftritt von \textcolor{blue}{Eleonora Duse} bei ihrem ersten
                        \textcolor{pink}{Wien}er Gastspiel. Entgegen seiner
                     Ankündigung, auch noch \emph{\textcolor{green}{Fernande}} sehen zu
                     wollen, wurden bis zum 26. 2. 1892 nur \emph{\textcolor{green}{Nora
                        oder Ein Puppenheim}} und die \emph{\textcolor{green}{Kameliendame}} gegeben. \textcolor{blue}{Schnitzler}
                     erlebte sie erst zwei Monate später, bei ihrem zweiten Gastspiel: am
                        17. 5. 1892 und 24. 5. 1892 sah er \emph{\textcolor{green}{Nora}} und \emph{\textcolor{green}{Fernande}}. (\emph{Cambridge University Library}, A 179a).}}}\label{K_L00075_1h}{ }11 Uhr nachts\pend
           \pstart
           Wenn Sie ſich die \textcolor{blue}{\textsc{Duse}}{}\ledrightnote{\textcolor{blue}{Eleonora Duse}} nicht anſehen, wenn auch auf der letzten Gallerie und ſtehend, verſäumen Sie
               mehr, als Sie ſich vorſtellen können.\pend
           \pstart \spacefill\mbox{Loris.}\pend{}\pstart
           \noindent{}Ich gehe zu \textcolor{green}{\textsc{Nora}}{}\ledrightnote{\textcolor{green}{Nora oder ein Puppenheim}} und \textcolor{green}{\textsc{Fernande}}{}\ledrightnote{\textcolor{green}{Fernande}}\pend
           \pstart
           Alles andere iſt jetzt gleichgiltig.\pend
           \endnumbering\briefempfaengerindex{Schnitzler, Arthur@\textsc{Schnitzler, Arthur}!zzzHofmannsthal, Hugo von@\emph{von Hugo von Hofmannsthal}!1892-02-231@{23. 2. 1892}|)be}\mylabel{h}  \normalsize

\doendnotes{C}
\bigskip
\vfill

\clearpage

\footnotesize

\lohead{\textsc{register}}

% Definiere theindex-Environment komplett neu ohne reledmac
\makeatletter
\renewenvironment{theindex}{%
  \section*{\indexname}%
  \setlength{\parindent}{0pt}%
  \setlength{\parskip}{0pt plus 0.3pt}%
  \let\item\@idxitem
}{%
  \clearpage
}
\makeatother

\IfFileExists{\jobname-pw.ind}{\input{\jobname-pw.ind}}{}

\end{document}

      