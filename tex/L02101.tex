%% latex-korrekturansicht-vorspann.tex
%% Vorspann für die Korrekturansicht.
%% Lädt die gemeinsame Datei latex-vorspann.tex mit gesetztem Schalter.

\newif\ifkorrekturansicht
\korrekturansichttrue

\input{../tex-inputs/latex-vorspann}


               \section[Arthur Schnitzler an Georg Brandes, 20. 11. 1912]{ Arthur Schnitzler an Georg Brandes, 20. 11. 1912}\nopagebreak\mylabel{v}\rehead{ }\normalsize\beginnumbering\briefempfaengerindex{Brandes, Georg@\textsc{Brandes, Georg}!zzzSchnitzler, Arthur@\emph{von Arthur Schnitzler}!1912-11-201@{20. 11. 1912}|(be} \toendnotes[C]{\smallbreak\pagebreak[2]} \Standort{Kopenhagen, Det Kongelige Bibliotek, Georg Brandes Arkiv, box 125.}
\physDesc{Brief
\newline{}Schreibmaschine
\newline{}Handschrift: schwarze Tinte, deutsche Kurrent (\noindent{}eine Korrektur, Unterschrift, Nachschrift)\newline{}Ordnung: mit Bleistift von unbekannter Hand nummeriert:
                                    »33.« }\buchAbdrucke{\weitereDrucke{Georg Brandes, Arthur Schnitzler: \emph{Ein Briefwechsel}. Hg. Kurt Bergel. Bern: \emph{Francke} 1956, S. 105.} }\toendnotes[C]{\smallbreak}\pstart
           \noindent{}{\pb}\textcolor{gray}{\textbf{Dr. Arthur Schnitzler}}{\\}\textcolor{gray}{\textbf{\textcolor{pink}{Wien XVIII. Sternwartestrasse 71}{}\ledrightnote{\textcolor{pink}{Sternwartestraße}}}}\pend
           \pstart
           \raggedleft{}20. 11. 1912. \pend
           \pstart\center{}Lieber und verehrter Herr Brandes.\pend\pstart
           Da ich leider nicht weiss, wo Sie abgestiegen sind, sende ich Ihnen diesen Brief in
               die \textcolor{pink}{Urania}{}\ledrightnote{\textcolor{pink}{Urania}}. Ich frage vor allem bei Ihnen an, ob
               Sie uns das Vergnügen machen wollen am Freitag Abend gegen acht bei uns
               zu essen. Es wäre sehr liebenswürdig von Ihnen mir gleich nach Empfang dieser Zeilen
               pneumatisch eine zusagende Antwort zu senden. Morgen Abend, Donnerstag,
               werde ich Ihnen nach Ihrer \label{K_L02101_1v}\edtext{Vorlesung}{\lemma{\textnormal{\emph{Vorlesung}}}\Cendnote{\textnormal{In seinem zweiten Vortrag
                  sprach er am 21. 11. 1912 um ½ 8 im Volksbildungshaus
                     \textcolor{pink}{Urania} über »\textcolor{blue}{Goethe} und die Zeitalter«. Am Vortrag hatte er bereits über »\textcolor{blue}{Jeanne d’Arc}« gesprochen, die dritte und letzte Vorlesung war \textcolor{blue}{Strindberg} gewidmet.}}}\label{K_L02101_1h} endlich wieder \label{K_L02101_2v}\edtext{die Hand drücken}{\lemma{\textnormal{\emph{die Hand drücken}}}\Cendnote{\textnormal{vgl. A. S.: \emph{Tagebuch}, 21. 11. 1912}}}\label{K_L02101_2h} können. Seien Sie willkommen in \textcolor{pink}{Wien}{}\ledrightnote{\textcolor{pink}{Wien}} und
               herzliche Grüsse.\pend
           \pstart
           Ihr sehr ergebener{\\[\baselineskip]}\spacefill\mbox{{[}hs.:{]} ArthurSchnitzler}\pend
           \leftskip=0em{}\pstart
           {[}ms.:{]} Samstag{ }Abend fahre ich nach \textcolor{pink}{Berlin}{}\ledrightnote{\textcolor{pink}{Berlin}} zu den
                  Proben meines neuen \textcolor{green}{Stückes}{}\ledrightnote{→\textcolor{green}{Professor Bernhardi. Komödie in fünf Akten}}.
                  Sollten Sie den Freitag Abend schon vergeben haben, so schenken Sie uns den \introOben{}Freitag\introOben{} Mittag gegen ½ 2.\pend
           \pstart
           Herrn Georg Brandes, \textcolor{pink}{Wien}{}\ledrightnote{\textcolor{pink}{Wien}}.\pend
           \pstart
           {[}hs.:{]} Erfahre eben Ihre Adreſſe – ſchicke alſo den Brief ans \textcolor{pink}{Continental}{}\ledrightnote{\textcolor{pink}{Hotel Continental}}.\pend
           \endnumbering\briefempfaengerindex{Brandes, Georg@\textsc{Brandes, Georg}!zzzSchnitzler, Arthur@\emph{von Arthur Schnitzler}!1912-11-201@{20. 11. 1912}|)be}\mylabel{h}  \normalsize

\doendnotes{C}
\bigskip
\vfill

\clearpage

\footnotesize

\lohead{\textsc{register}}

% Definiere theindex-Environment komplett neu ohne reledmac
\makeatletter
\renewenvironment{theindex}{%
  \section*{\indexname}%
  \setlength{\parindent}{0pt}%
  \setlength{\parskip}{0pt plus 0.3pt}%
  \let\item\@idxitem
}{%
  \clearpage
}
\makeatother

\IfFileExists{\jobname-pw.ind}{\input{\jobname-pw.ind}}{}

\end{document}

      