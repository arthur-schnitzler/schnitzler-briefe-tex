%% latex-korrekturansicht-vorspann.tex
%% Vorspann für die Korrekturansicht.
%% Lädt die gemeinsame Datei latex-vorspann.tex mit gesetztem Schalter.

\newif\ifkorrekturansicht
\korrekturansichttrue

\input{../tex-inputs/latex-vorspann}


               \section[Arthur Schnitzler an Richard Beer-Hofmann, 5. 8. 1912]{ Arthur Schnitzler an Richard Beer-Hofmann, 5. 8. 1912}\nopagebreak\mylabel{v}\rehead{ }\normalsize\beginnumbering\briefempfaengerindex{Beer-Hofmann, Richard@\textsc{Beer-Hofmann, Richard}!zzzSchnitzler, Arthur@\emph{von Arthur Schnitzler}!1912-08-051@{5. 8. 1912}|(be} \toendnotes[C]{\smallbreak\pagebreak[2]} \Standort{YCGL, MSS 31.}
\physDesc{Bildpostkarte
\newline{}Handschrift: Bleistift, deutsche Kurrent\newline{}Versand: Stempel: »\nobreak{}\oindex{Brijuni@\textbf{Brijuni}, \emph{https://www.geonames.org/ontologyP.PPL}|pwk}Brioni, 5. 8. 1\textcolor{gray}{2}\nobreak{}«.  }\toendnotes[C]{\smallbreak}\pstart{}{\pb}\textsc{Herrn Dr.}\pend{}\pstart{}\textsc{Richard Beerhofmann}\pend{}\pstart{}\textsc{\textcolor{pink}{St Moritz}{}\ledrightnote{\textcolor{pink}{Sankt Moritz}}}\pend{}\pstart{}\textsc{im \textcolor{pink}{Engadin}{}\ledrightnote{\textcolor{pink}{Engadin}}}\pend{}\pstart{}\textsc{\textcolor{pink}{Waldhaus}{}\ledrightnote{\textcolor{pink}{Hotel Waldhaus}}.}\pend{}{\bigskip}\pstart
           \noindent{}\centering{}{\pb}\textcolor{gray}{\textbf{\textcolor{pink}{Insel Brioni i. d. Adria}{}\ledrightnote{\textcolor{pink}{Brijuni}}.}}\pend
           \pstart
           \noindent{}\centering{}\textcolor{gray}{\textbf{\textcolor{pink}{Val Catena}{}\ledrightnote{\textcolor{pink}{Val Catena}}.}}\pend
           \pstart
           {\pb}lieber Richard, unſer Plan iſt am 20 od
                  21. über die \textcolor{pink}{ital Seen}{}\ledrightnote{\textcolor{pink}{Italien}} nach \textcolor{pink}{\textsc{Sils Maria}}{}\ledrightnote{\textcolor{pink}{Sils im Engadin}}, dort bis Ende Auguſt, da{\geminationn}{ }\textcolor{pink}{\textsc{München}}{}\ledrightnote{\textcolor{pink}{München}}, (\textcolor{pink}{\textsc{Tutzing}}{}\ledrightnote{\textcolor{pink}{Tutzing}}) \textsc{circa} 8 Tage; u \textsc{direct} von
               dort \textcolor{pink}{Wien}{}\ledrightnote{\textcolor{pink}{Wien}}. Vielleicht trifft man ſich in \textcolor{pink}{München}{}\ledrightnote{\textcolor{pink}{München}} (\label{KLL02082_Beer-Hofmann-1v}\edtext{Sitze beſorgt?}{\lemma{\textnormal{\emph{Sitze beſorgt?}}}\Cendnote{\textnormal{siehe Arthur Schnitzler an Richard Beer-Hofmann, 27. 8. 1895, siehe Arthur Schnitzler an Richard Beer-Hofmann, 28. 7. 1922}}}\label{KLL02082_Beer-Hofmann-1h}) Wie lange bleibt \textcolor{blue}{Kaufma{\geminationn}}{}\ledrightnote{\textcolor{blue}{Arthur Kaufmann}} im \textcolor{pink}{Engadin}{}\ledrightnote{\textcolor{pink}{Engadin}}?\pend
           \pstart
           Herzlichſt{\\[\baselineskip]}Ihr \spacefill\mbox{A.}\pend
           \leftskip=0em{}\endnumbering\briefempfaengerindex{Beer-Hofmann, Richard@\textsc{Beer-Hofmann, Richard}!zzzSchnitzler, Arthur@\emph{von Arthur Schnitzler}!1912-08-051@{5. 8. 1912}|)be}\mylabel{h}  \normalsize

\doendnotes{C}
\bigskip
\vfill

\clearpage

\footnotesize

\lohead{\textsc{register}}

% Definiere theindex-Environment komplett neu ohne reledmac
\makeatletter
\renewenvironment{theindex}{%
  \section*{\indexname}%
  \setlength{\parindent}{0pt}%
  \setlength{\parskip}{0pt plus 0.3pt}%
  \let\item\@idxitem
}{%
  \clearpage
}
\makeatother

\IfFileExists{\jobname-pw.ind}{\input{\jobname-pw.ind}}{}

\end{document}

      