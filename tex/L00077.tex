%% latex-korrekturansicht-vorspann.tex
%% Vorspann für die Korrekturansicht.
%% Lädt die gemeinsame Datei latex-vorspann.tex mit gesetztem Schalter.

\newif\ifkorrekturansicht
\korrekturansichttrue

\input{../tex-inputs/latex-vorspann}


               \section[Oscar Blumenthal an Arthur Schnitzler, {[}6. 3. 1892{]}]{ Oscar Blumenthal an Arthur Schnitzler, {[}6. 3. 1892{]}}\nopagebreak\mylabel{v}\rehead{ }\normalsize\beginnumbering\briefempfaengerindex{Schnitzler, Arthur@\textsc{Schnitzler, Arthur}!zzzBlumenthal, Oskar@\emph{von Oskar Blumenthal}!1892-03-061@{{[}6. 3. 1892{]}}|(be} \toendnotes[C]{\smallbreak\pagebreak[2]} \Standort{CUL, Schnitzler, B 15.}
\physDesc{Visitenkarte
\newline{}Handschrift: schwarze Tinte, deutsche Kurrent
\newline{}Schnitzler: 1) mit Bleistift datiert: »März 92« 2) mit rotem Buntstift nummeriert:
                                 »3«\newline{}Ordnung: mit Bleistift von unbekannter Hand »a« an die
                                 Nummerierung angehängt }\toendnotes[C]{\smallbreak}\pstart
           \noindent{}\centering{}{\pb}\textcolor{gray}{\textbf{Dr. Oscar Blumenthal}}{\\}\textcolor{gray}{\textbf{Direktor des \textcolor{brown}{Lessingtheaters}{}\ledrightnote{\textcolor{brown}{Lessing-Theater}}.}}\pend
           \pstart
           \noindent{}bittet \label{K_L00077_1v}\edtext{morgen, Montag}{\lemma{\textnormal{\emph{morgen, Montag}}}\Cendnote{\textnormal{Das Treffen fand am
                  7. 3. 1892 statt.}}}\label{K_L00077_1h}, 4 Uhr um Ihren freundlichen Beſuch \textcolor{pink}{\textsc{Hôtel}{ }Sacher}{}\ledrightnote{\textcolor{pink}{Hotel Sacher}}\pend
           \endnumbering\briefempfaengerindex{Schnitzler, Arthur@\textsc{Schnitzler, Arthur}!zzzBlumenthal, Oskar@\emph{von Oskar Blumenthal}!1892-03-061@{{[}6. 3. 1892{]}}|)be}\mylabel{h}  \normalsize

\doendnotes{C}
\bigskip
\vfill

\clearpage

\footnotesize

\lohead{\textsc{register}}

% Definiere theindex-Environment komplett neu ohne reledmac
\makeatletter
\renewenvironment{theindex}{%
  \section*{\indexname}%
  \setlength{\parindent}{0pt}%
  \setlength{\parskip}{0pt plus 0.3pt}%
  \let\item\@idxitem
}{%
  \clearpage
}
\makeatother

\IfFileExists{\jobname-pw.ind}{\input{\jobname-pw.ind}}{}

\end{document}

      