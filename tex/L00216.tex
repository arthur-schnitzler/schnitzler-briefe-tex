%% latex-korrekturansicht-vorspann.tex
%% Vorspann für die Korrekturansicht.
%% Lädt die gemeinsame Datei latex-vorspann.tex mit gesetztem Schalter.

\newif\ifkorrekturansicht
\korrekturansichttrue

\input{../tex-inputs/latex-vorspann}


               \section[Fedor Mamroth an Arthur Schnitzler, 4. 6. 1893]{ Fedor Mamroth an Arthur Schnitzler, 4. 6. 1893}\nopagebreak\mylabel{v}\rehead{ }\normalsize\beginnumbering\briefempfaengerindex{Schnitzler, Arthur@\textsc{Schnitzler, Arthur}!zzzMamroth, Fedor@\emph{von Fedor Mamroth}!1893-06-041@{4. 6. 1893}|(be} \toendnotes[C]{\smallbreak\pagebreak[2]} \Standort{CUL, Schnitzler, B 68.}
\physDesc{Brief, 1 Blatt, 2 Seiten
\newline{}Handschrift: blaue Tinte, deutsche Kurrent
\newline{}Schnitzler: mit rotem Buntstift eine Unterstreichung \newline{}Ordnung: mit Bleistift von unbekannter Hand nummeriert: »5« }\toendnotes[C]{\smallbreak}\pstart
           \noindent{}{\pb}\textcolor{brown}{\textcolor{gray}{\textbf{\textsc{Frankfurter Zeitung}}}{\\}\textsc{\textcolor{gray}{\textbf{und}}}{\\}\textcolor{gray}{\textbf{\textsc{Handelsblatt.}}}}{}\ledrightnote{\textcolor{brown}{Frankfurter Zeitung}}\pend
           \pstart
           \textcolor{gray}{\textbf{\textsc{Redaktion.\footnote{\noindent{}\textcolor{gray}{\textbf{\textsc{Für die Redaktion bestimmte
                                                  Briefe und Sendungen wolle man \so{nicht} an die Person eines
                                                  Redakteurs, sondern stets \textbf{an die
                                                  Redaktion der Frankfurter Zeitung}
                                                  adressiren}}}.}}}}\hfill \textcolor{gray}{\textbf{\textsc{\textcolor{pink}{Frankfurt a. M.}{}\ledrightnote{\textcolor{pink}{Frankfurt am Main}},}}}{ }4. Juni \textsc{\textcolor{gray}{\textbf{189}}}3\pend
           \pstart
           \textcolor{gray}{\textbf{\textsc{Telegramm-Adresse:}}}\pend
           \pstart
           \textcolor{gray}{\textbf{\textsc{Zeitung Frankfurt Main.}}}\pend
           \pstart{}Sehr verehrter Herr Doctor!\pend\pstart
           Ich habe Ihren Roman »\textcolor{green}{Der ſterbende Herr}{}\ledrightnote{\textcolor{green}{Sterben. Novelle}}« mit
                    einer Theilnahme geleſen, die mir noch ſelten eine eingereichte Arbeit
                    eingeflößt hat. Ich beglückwünſche Sie zu dieſer Dichtung, in der ſie
                    den feinen Geiſt eines Poeten und \introOben{}die\introOben{}{ }ſcharfe Beobachtungsgabe des Arztes mit merkwürdiger Ergänzungskunſt
                    verſchmolzen haben. Allein »\textcolor{green}{Der ſterbende
                        Herr}{}\ledrightnote{\textcolor{green}{Sterben. Novelle}}« iſt kein Zeitungs- ſondern ein Buchroman; erſtens nicht aus
                    Gründen, die ich an dieſer Stelle nicht zu erörtern vermag. Darf ich mir
                    erlauben, Ihnen einen Rath zu ertheilen, ſo würde ich Ihnen dringend empfehlen,
                    für die Veröffentlichung Ihrer ſchönen Arbeit, die Ihnen einen verdienten Erfolg
                    einbringen wird, ohne Verzug einen Verleger zu ſuchen. Mein Intereſſe daran iſt
                    ein ſo aufrichtiges, daß es mir ein Vergnügen wäre, Ihnen auch perſönlich in
                    dieſer Richtung zu dienen, wenn ich dem Kreiſe der deutſchen Verleger leider
                    nicht völlig fernſtünde. Aber ich kann mir nicht denken, daß Ihnen eine
                    Placirung der Arbeit Schwierigkeiten bereiten ſollte. Es gibt doch gewiß
                    Unternehmer von Urtheil u. Geſchmack, die den Werth einer ſo hervorragenden
                    Compoſition zu ſchätzen wiſſen! Eine Änderung des Titels würde ich Ihnen
                    ernſtlich {\pb}in Vorſchlag bringen. Wie
                    denken Sie über »Das letzte Jahr« oder »Ende« oder »Ein Todesurtheil« oder »Der
                    Wille zum Leben« u. ſ. w. All das heißt auch nicht viel, aber es ſcheint mir
                    doch beſſer als der gewählte Titel.\pend
           \pstart
           Verſäumen Sie nicht, mir Nachricht zu geben, ſobald der \textcolor{green}{Roman}{}\ledrightnote{→\textcolor{green}{Sterben. Novelle}} unter Dach u. Fach gelangt.\pend
           \pstart
           Hochachtungsvoll{\\[\baselineskip]}Ihr{\\[\baselineskip]}ergebener{\\[\baselineskip]}\spacefill\mbox{D\textsuperscript{r} FMamroth.}\pend
           \leftskip=0em{}\endnumbering\briefempfaengerindex{Schnitzler, Arthur@\textsc{Schnitzler, Arthur}!zzzMamroth, Fedor@\emph{von Fedor Mamroth}!1893-06-041@{4. 6. 1893}|)be}\mylabel{h}  \normalsize

\doendnotes{C}
\bigskip
\vfill

\clearpage

\footnotesize

\lohead{\textsc{register}}

% Definiere theindex-Environment komplett neu ohne reledmac
\makeatletter
\renewenvironment{theindex}{%
  \section*{\indexname}%
  \setlength{\parindent}{0pt}%
  \setlength{\parskip}{0pt plus 0.3pt}%
  \let\item\@idxitem
}{%
  \clearpage
}
\makeatother

\IfFileExists{\jobname-pw.ind}{\input{\jobname-pw.ind}}{}

\end{document}

      