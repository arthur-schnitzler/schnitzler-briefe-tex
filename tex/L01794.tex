%% latex-korrekturansicht-vorspann.tex
%% Vorspann für die Korrekturansicht.
%% Lädt die gemeinsame Datei latex-vorspann.tex mit gesetztem Schalter.

\newif\ifkorrekturansicht
\korrekturansichttrue

\input{../tex-inputs/latex-vorspann}


               \section[Arthur Schnitzler an Albert Ehrenstein, 20. 10. 1908]{ Arthur Schnitzler an Albert Ehrenstein, 20. 10. 1908}\nopagebreak\mylabel{v}\rehead{ }\normalsize\beginnumbering\briefempfaengerindex{Ehrenstein, Albert@\textsc{Ehrenstein, Albert}!zzzSchnitzler, Arthur@\emph{von Arthur Schnitzler}!1908-10-202@{20. 10. 1908}|(be} \toendnotes[C]{\smallbreak\pagebreak[2]} \Standort{Jerusalem, The National Library of Israel, ARC. Ms. Var. 306 1 118.}
\physDesc{Briefkarte
\newline{}Handschrift: schwarze Tinte, deutsche Kurrent}\toendnotes[C]{\smallbreak}\pstart
           \noindent{}{\pb}\textcolor{gray}{\textbf{Dr. Arthur Schnitzler}}\hfill 20. X. 908\pend
           \pstart
           \textcolor{gray}{\textbf{\textcolor{pink}{Wien XVIII. Spoettelgasse 7}{}\ledrightnote{\textcolor{pink}{Edmund-Weiß-Gasse}}.}}\pend
           \pstart
           Werther Herr Ehrenſtein, das \textcolor{green}{\textsc{Mcrpt}}{}\ledrightnote{→\textcolor{green}{Seltene Gäste}} iſt wohlverwahrt bei mir. Aber ich war in der letzten Zeit ſo beschäftigt,
                    daſs ich erſt vorgeſtern nun die Skizzen geleſen hab. Ich hoffe {\pb}es eilt Ihnen nicht ſehr und Sie geſatten mir, Ihren
                    freundlichen Besuch erſt in etwa 10–14 Tagen zu erhalten, – zu einer Stunde, die
                    ich Ihnen mittheilen werde.\pend
           \pstart
           mit beſtem Gruſs Ihr ſehr ergeb\textcolor{gray}{ener}{\\[\baselineskip]}\spacefill\mbox{A. S.}\pend
           \leftskip=0em{}\endnumbering\briefempfaengerindex{Ehrenstein, Albert@\textsc{Ehrenstein, Albert}!zzzSchnitzler, Arthur@\emph{von Arthur Schnitzler}!1908-10-202@{20. 10. 1908}|)be}\mylabel{h}  \normalsize

\doendnotes{C}
\bigskip
\vfill

\clearpage

\footnotesize

\lohead{\textsc{register}}

% Definiere theindex-Environment komplett neu ohne reledmac
\makeatletter
\renewenvironment{theindex}{%
  \section*{\indexname}%
  \setlength{\parindent}{0pt}%
  \setlength{\parskip}{0pt plus 0.3pt}%
  \let\item\@idxitem
}{%
  \clearpage
}
\makeatother

\IfFileExists{\jobname-pw.ind}{\input{\jobname-pw.ind}}{}

\end{document}

      