%% latex-korrekturansicht-vorspann.tex
%% Vorspann für die Korrekturansicht.
%% Lädt die gemeinsame Datei latex-vorspann.tex mit gesetztem Schalter.

\newif\ifkorrekturansicht
\korrekturansichttrue

\input{../tex-inputs/latex-vorspann}


               \section[Arthur Schnitzler an Hugo von Hofmannsthal, 5.–6. 8. 1904]{ Arthur Schnitzler an Hugo von Hofmannsthal, 5.–6. 8. 1904}\nopagebreak\mylabel{v}\rehead{ }\normalsize\beginnumbering\briefempfaengerindex{Hofmannsthal, Hugo von@\textsc{Hofmannsthal, Hugo von}!zzzSchnitzler, Arthur@\emph{von Arthur Schnitzler}!1904-08-061@{5.–6. 8. 1904}|(be} \toendnotes[C]{\smallbreak\pagebreak[2]} \Standort{FDH, Hs-30885,110.}
\physDesc{Brief, 2 Blätter, 8 Seiten
\newline{}Handschrift: schwarze Tinte, deutsche Kurrent\newline{}Ordnung: mit Bleistift von Schnitzler mutmaßlich bei der
                                 Durchsicht der Korrespondenz 1929 das zweite Blatt nummeriert: »II« und datiert: »5/8 904« }\buchAbdrucke{\weitereDrucke{Hugo von Hofmannsthal, Arthur Schnitzler: \emph{Briefwechsel}. Hg. Therese Nickl und Heinrich Schnitzler. Frankfurt am Main: \emph{S. Fischer} 1964, S. 192–193.} }\toendnotes[C]{\smallbreak}\pstart
           \raggedleft{}{\pb}\textcolor{pink}{Wien}{}\ledrightnote{\textcolor{pink}{Wien}}, 5. 8. 904\pend
           \pstart
           lieber Hugo, Ihr Brief aus der \textcolor{pink}{Fuſch}{}\ledrightnote{\textcolor{pink}{Bad Fusch}} hat mich ſehr erfreut und ich bin begierig was Sie nun eigentlich
               alles außer dem \textcolor{green}{geretteten Venedig}{}\ledrightnote{\textcolor{green}{Das gerettete Venedig. Trauerspiel in fünf Aufzügen}} von dieſem So{\geminationm}er nach Hauſe bringen werden. In der Wärme die uns
               umfließt, in der Beſo{\geminationn}theit der ganzen Atmosphäre muſs
               doch etwas ſeltſam befruchtendes liegen, denn auch mir geht es ſo gut wie lange
               nicht. Es hat begonnen an einem der erſten \label{K_L01422_1v}\edtext{Tage}{\lemma{\textnormal{\emph{Tage}}}\Cendnote{\textnormal{vgl. A. S.: \emph{Tagebuch}, 3. 7. 1904}}}\label{K_L01422_1h}, da ich von
               meinem Unwohlſein wieder aufgeſtanden war – wo ich \introOben{}Nachmittags\introOben{} eine ganze \textcolor{green}{Novellette}{}\ledrightnote{→\textcolor{green}{Das neue Lied}}
               niederſchrieb, die mir (der Einfall beſtand ſchon ſeit {\pb}lange) Vormittags auf einem Spaziergang aufgegangen war. Dann
               arbeitete ich an dem \textcolor{green}{Roman}{}\ledrightnote{→\textcolor{green}{Der Weg ins Freie. Roman}}
               weiter, deſſen Fülle ich nur mehr möchte beherrſchen können. Vom
                  12.–24 (ungefähr) waren wir in \textcolor{pink}{Reichenau}{}\ledrightnote{\textcolor{pink}{Reichenau an der Rax}}, wo ich auch in guter Sti{\geminationm}ung weiterſchrieb. Ausflüge \textcolor{pink}{Naßwald}{}\ledrightnote{\textcolor{pink}{Nasswald}}, \textcolor{pink}{Rax}{}\ledrightnote{\textcolor{pink}{Rax}}. Rad beinah gar nicht – die vielen müheloſen
               Dahinraſer im Automobil verderben einem die naive Freude. Aber es wird ſchon
                  wiederko{\geminationm}en, in fremdem Gegenden.\pend
           \pstart
           Nun ſind wir ſeit etwa 12 Tagen wieder in \textcolor{pink}{Wien}{}\ledrightnote{\textcolor{pink}{Wien}} und in
               unſerer {\pb}angenehmen Wohnung gefällt es uns ſehr gut und
               wir finden uns alle Vater, \textcolor{blue}{Mutter}{}\ledrightnote{→\textcolor{blue}{Olga Schnitzler}} und \textcolor{blue}{Kind}{}\ledrightnote{→\textcolor{blue}{Heinrich Schnitzler}}
               behaglich. Seit der \textcolor{blue}{Julius}{}\ledrightnote{\textcolor{blue}{Julius Schnitzler}} auf Ferien iſt ſteht
               uns ſein Fiaker zur Verfügung \strikeout{iſt}, und ſo fahr ich
               mit \textcolor{blue}{Olga}{}\ledrightnote{\textcolor{blue}{Olga Schnitzler}} jeden Abend aufs Land, immer aufs neue u
               immer mehr entzückt von dieſen \textcolor{pink}{Wiener Wald}{}\ledrightnote{\textcolor{pink}{Wienerwald}}
               Landſchaften – die mich beinah immer ſo ergreifen als käme ich nach langen Jahren von
               irgendwoher in dieſe heimatliche Wunderſamkeit zurück. Geſtern Abend fuhren wir an
               dem verwaiſten \textcolor{pink}{Ro{\pb}daun}{}\ledrightnote{\textcolor{pink}{Rodaun}}
               ganz nah vorüber, von \textcolor{pink}{Mauer}{}\ledrightnote{\textcolor{pink}{Mauer}} über \textcolor{pink}{Kalksburg}{}\ledrightnote{\textcolor{pink}{Kalksburg}} (eine Waldſtraße, \textcolor{pink}{Klauſenſtraße}{}\ledrightnote{\textcolor{pink}{Kalksburger Straße}} glaub ich, die ich noch gar nicht kannte) nach dem \textcolor{pink}{rothen Stadel}{}\ledrightnote{\textcolor{pink}{Der rothe Stadl}}, und haben Ihrer und \textcolor{blue}{Richard}{}\ledrightnote{\textcolor{blue}{Richard Beer-Hofmann}}s herzlich gedacht. (Es war ſozuſagen eine
               ungeſchriebene Anſichtskarte, die ſich abſpielte) –\pend
           \pstart
           Vor ein \label{K_L01422_2v}\edtext{paar Tagen}{\lemma{\textnormal{\emph{paar Tagen}}}\Cendnote{\textnormal{vgl. A. S.: \emph{Tagebuch}, 31. 7. 1904}}}\label{K_L01422_2h}, in \textcolor{pink}{Mauerbach}{}\ledrightnote{\textcolor{pink}{Mauerbach}},
               entwickelte ſich plötzlich aus einer kleinen Notiz, die ich in mein Büchel
               eingetragen hatte, im Geſpräch mit \textcolor{blue}{Olga}{}\ledrightnote{\textcolor{blue}{Olga Schnitzler}}, ein
               völliges \textcolor{green}{Luſtſpielſujet}{}\ledrightnote{→\textcolor{green}{Zwischenspiel. Komödie in drei Akten}}, am
               nächſten Tag ent{\pb}warf ich das \textsc{Scenarium}, am übernächſten ſtanden die Geſtalten ſchon ſo klar vor mir, daſs
               ich mich berechtigt fühlte, die erſte ſchlamperte Niederſchrift zu beginnen, die mich
               wohl nicht lange in Anſpruch nehmen wird. Es ka{\geminationn}, we{\geminationn} die Laune bleibt, ein graziöſes Ding werden. Ein
               andres Stück, eine \textcolor{green}{5aktige
                  Komödie}{}\ledrightnote{→\textcolor{green}{Ritterlichkeit}}, von der in \textcolor{pink}{Taormina}{}\ledrightnote{\textcolor{pink}{Taormina}} 3 Akte ganz
               flüchtig und zum Theil blödſinnig hingeſchmiſſen wurden, die ſich aber hier,
               wenigſtens im Plan, zu etwas ſehr möglichem entwickelte, {\pb}bleibt nun bis auf weiteres liegen. Von dem phantaſtiſch hiſtoriſchen \textcolor{green}{Stück}{}\ledrightnote{→\textcolor{green}{Der junge Medardus. Dramatische Historie in einem Vorspiel und fünf Aufzügen}} und manchem andern, das in
               zweiter Reihe und dritter ſteht, will ich vorläufig nicht reden; ich möchte nur das
               ſtrategiſche Talent haben, die Truppen, die ich vorläufig nicht brauche, mit der
               nöthigen Autorität in die Reſerve oder wenigſtens hinter die Schlachtlinie zu
               verweiſen (Hören Sie den ehemaligen k. u. k. Oberarzt aus dieſen Worten trompeten?)
               Außerdem {\pb}möcht ich allerdings noch manches andre: vor
               allem mehr Fleiſs{\dots}\pend
           \pstart
           \raggedleft{}6. 8\pend
           \pstart
           wurde geſtern unterbrochen und will heute nur noch viele ſchöne Grüße hinzuſetzen.
               Heute (es iſt Nachmittg) waren wir ſchon am Vormittag auf
               der \textcolor{pink}{Sophienalpe}{}\ledrightnote{\textcolor{pink}{Sophienalpe}}, und das iſt die Gegend, wo ich von
               den Geſtalten des \textcolor{green}{Romans}{}\ledrightnote{→\textcolor{green}{Der Weg ins Freie. Roman}} am
               härteſten bedrängt werde. –\pend
           \pstart
           Wir bleiben nun denk ich bis Anfang September hier in \textcolor{pink}{Wien}{}\ledrightnote{\textcolor{pink}{Wien}}, und dann möchten wir, auf etwa 14 Tage nicht allzu weit,
                  \textcolor{pink}{Iſchl}{}\ledrightnote{\textcolor{pink}{Bad Ischl}} etwa. Es {\pb}wäre
               nicht undenkbar, daſs die \textcolor{blue}{Fanny Mütter}{}\ledrightnote{\textcolor{blue}{Franziska Mütter}} mitkommt;
               aber ich halt es für unwahrſcheinlich. Kämen Sie da{\geminationn}
               event. auch mit \textcolor{blue}{\textsc{Gerty}}{}\ledrightnote{\textcolor{blue}{Gertrude von Hofmannsthal}}, ſo könnten wir zwei ein paar unſrer ſchönen Radtouren vollführen? – Jedenfalls
               treffen wir uns im Herbſt, nicht wahr? –\pend
           \pstart
           Grüßen Sie was Sie in \textcolor{pink}{Auſſee}{}\ledrightnote{\textcolor{pink}{Bad Aussee}} von erfreulichen
               Menſchen ſehen und antworten mir raſcher als ich Ihnen diesmal geantwortet habe.\pend
           \pstart
           Herzlichſt\hspace*{1.5em}Ihr{\\[\baselineskip]}\spacefill\mbox{A.}\pend
           \leftskip=0em{}\endnumbering\briefempfaengerindex{Hofmannsthal, Hugo von@\textsc{Hofmannsthal, Hugo von}!zzzSchnitzler, Arthur@\emph{von Arthur Schnitzler}!1904-08-061@{5.–6. 8. 1904}|)be}\mylabel{h}  \normalsize

\doendnotes{C}
\bigskip
\vfill

\clearpage

\footnotesize

\lohead{\textsc{register}}

% Definiere theindex-Environment komplett neu ohne reledmac
\makeatletter
\renewenvironment{theindex}{%
  \section*{\indexname}%
  \setlength{\parindent}{0pt}%
  \setlength{\parskip}{0pt plus 0.3pt}%
  \let\item\@idxitem
}{%
  \clearpage
}
\makeatother

\IfFileExists{\jobname-pw.ind}{\input{\jobname-pw.ind}}{}

\end{document}

      