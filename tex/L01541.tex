%% latex-korrekturansicht-vorspann.tex
%% Vorspann für die Korrekturansicht.
%% Lädt die gemeinsame Datei latex-vorspann.tex mit gesetztem Schalter.

\newif\ifkorrekturansicht
\korrekturansichttrue

\input{../tex-inputs/latex-vorspann}


               \section[Hugo von Hofmannsthal an Arthur Schnitzler, 11. {[}8.?{]} 1905]{ Hugo von Hofmannsthal an Arthur Schnitzler, 11. {[}8.?{]} 1905}\nopagebreak\mylabel{v}\rehead{ }\normalsize\beginnumbering\briefempfaengerindex{Schnitzler, Arthur@\textsc{Schnitzler, Arthur}!zzzHofmannsthal, Hugo von@\emph{von Hugo von Hofmannsthal}!1905-08-111@{11. {[}8.?{]} 1905}|(be} \toendnotes[C]{\smallbreak\pagebreak[2]} \Standort{CUL, Schnitzler, B 43.}
\physDesc{Postkarte
\newline{}Handschrift: schwarze Tinte, deutsche Kurrent\newline{}Versand: 1) Rohrpost 2) Stempel: »\nobreak{}\oindex{I., Innere Stadt@\textbf{I., Innere Stadt}, \emph{Bezirk (A.BZK)}|pwk}Wien 1/1, 11 V\textcolor{gray}{III} 05, 11 10V\nobreak{}«. 3) Stempel: »\nobreak{}\oindex{XVIII., Waehring@\textbf{XVIII., Währing}, \emph{Bezirk (A.BZK)}|pwk}18/1 Wien 111, 11 V\textcolor{gray}{III} 05, XII\textsuperscript{50}\nobreak{}«. \newline{}Ordnung: 1) mit Bleistift von unbekannter Hand nummeriert: »\strikeout{209}« 2) mit Bleistift von unbekannter Hand nummeriert: »\strikeout{258}«3) mit Bleistift von unbekannter Hand nummeriert:
                                    »257«}\buchAbdrucke{\weitereDrucke{Hugo von Hofmannsthal, Arthur Schnitzler: \emph{Briefwechsel}. Hg. Therese Nickl und Heinrich Schnitzler. Frankfurt am Main: \emph{S. Fischer} 1964, S. 213.} }\toendnotes[C]{\smallbreak}\pstart{}{\pb}\textsc{Herrn D\textsuperscript{r} Arthur Schnitzler}\pend{}\pstart{}\textcolor{pink}{\textsc{Wien}}{}\ledrightnote{\textcolor{pink}{Wien}}\pend{}\pstart{}\textcolor{pink}{\textsc{XVIII Spöttelgasse 7}}{}\ledrightnote{\textcolor{pink}{Edmund-Weiß-Gasse}}\pend{}\pstart{}\textsc{pneumatisch}\pend{}{\bigskip}\pstart
           \noindent{}{\pb}lieb\textcolor{gray}{en}
                  Leuten, bitte ko{\geminationm}t nicht ſpäter als mit dem
               Zug der ½ 6{ }\textcolor{pink}{Hietzing}{}\ledrightnote{\textcolor{pink}{XIII., Hietzing}} weggeht, weil wir einen \label{K_L01541_1v}\edtext{Wagen beſtellt}{\lemma{\textnormal{\emph{Wagen beſtellt}}}\Cendnote{\textnormal{Die Poststempel sind bei der Monatsangabe undeutlich. Im
                     Juli 1905 ist \textcolor{blue}{Hofmannsthal} auf
                  Waffenübung, so dass das als möglicher Termin ausscheidet. Am 11. 8. 1905 kommt es zum
                  Treffen in \textcolor{pink}{Rodaun}.}}}\label{K_L01541_1h} zum ein bisl
               Ausfahren.\pend
           \pstart \spacefill\mbox{Hugo.}\pend{}\endnumbering\briefempfaengerindex{Schnitzler, Arthur@\textsc{Schnitzler, Arthur}!zzzHofmannsthal, Hugo von@\emph{von Hugo von Hofmannsthal}!1905-08-111@{11. {[}8.?{]} 1905}|)be}\mylabel{h}  \normalsize

\doendnotes{C}
\bigskip
\vfill

\clearpage

\footnotesize

\lohead{\textsc{register}}

% Definiere theindex-Environment komplett neu ohne reledmac
\makeatletter
\renewenvironment{theindex}{%
  \section*{\indexname}%
  \setlength{\parindent}{0pt}%
  \setlength{\parskip}{0pt plus 0.3pt}%
  \let\item\@idxitem
}{%
  \clearpage
}
\makeatother

\IfFileExists{\jobname-pw.ind}{\input{\jobname-pw.ind}}{}

\end{document}

      