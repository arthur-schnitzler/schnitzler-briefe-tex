%% latex-korrekturansicht-vorspann.tex
%% Vorspann für die Korrekturansicht.
%% Lädt die gemeinsame Datei latex-vorspann.tex mit gesetztem Schalter.

\newif\ifkorrekturansicht
\korrekturansichttrue

\input{../tex-inputs/latex-vorspann}


               \section[Georg Brandes an Arthur Schnitzler, 10. 12. 1924]{ Georg Brandes an Arthur Schnitzler, 10. 12. 1924}\nopagebreak\mylabel{v}\rehead{ }\normalsize\beginnumbering\briefempfaengerindex{Schnitzler, Arthur@\textsc{Schnitzler, Arthur}!zzzBrandes, Georg@\emph{von Georg Brandes}!1924-12-101@{10. 12. 1924}|(be} \toendnotes[C]{\smallbreak\pagebreak[2]} \Standort{CUL, Schnitzler, B 17.}
\physDesc{Brief, 1 Blatt, 4 Seiten
\newline{}Handschrift: schwarze Tinte, lateinische Kurrent
\newline{}Schnitzler: mit rotem Buntstift mehrere Unterstreichungen \newline{}Ordnung: mit Bleistift von unbekannter Hand nummeriert: »55« }\buchAbdrucke{\weitereDrucke{Georg Brandes, Arthur Schnitzler: \emph{Ein Briefwechsel}. Hg. Kurt Bergel. Bern: \emph{Francke} 1956, S. 140–141.} }\toendnotes[C]{\smallbreak}\pstart
           \raggedleft{}{\pb}\textcolor{pink}{Kopenhagen}{}\ledrightnote{\textcolor{pink}{Kopenhagen}}{ }10 December 24\pend
           \pstart
           Mein liebster Schnitzler\hspace*{3.5em}Viel Arbeit und lang dauernde wenn auch nicht
                    schwere Krankheit, die noch nicht vorüber ist, haben mich verhindert, Ihnen in
                    Dank mein Herz auszuschütten. Irgend jemand, der von Ihnen kam oder auf Sie sich
                    berief, war neulich bei mir. Wie er hiess, habe ich vergessen.\pend
           \pstart
           Ich habe Ihnen für zwei \textcolor{green}{Bücher}{}\ledrightnote{→\textcolor{green}{Fräulein Else}{\newline}→\textcolor{green}{Komödie der Verführung. In drei Akten}} zu danken. Besonders das erstere die \textcolor{green}{Komoedie der Verführung}{}\ledrightnote{\textcolor{green}{Komödie der Verführung. In drei Akten}} gibt viel zu denken über den
                    Reichtum und die Tiefe Ihrer Erfahrungen, vielleicht noch mehr über die Fülle
                    und Geschmeidigkeit Ihrer Erfindungskraft, die ich am meisten bewundere, weil
                    sie mir völlig fehlt. Man bewundert {\pb}wol immer am meisten
                    Fähigkeiten, die uns verweigert sind.\pend
           \pstart
           Ich habe mit Ueberraschung gesehen dass Ihre paar kurzen Aufenthalte in unserem
                    kleinen langweiligen \textcolor{pink}{Land}{}\ledrightnote{→\textcolor{pink}{Dänemark}}
                    Ihre Phantasie in Bewegung gesetzt hat, und dass sogar die \textcolor{pink}{Nordküste von \textcolor{pink}{Seeland}{}\ledrightnote{\textcolor{pink}{Seeland}}}{}\ledrightnote{→\textcolor{pink}{Gilleleje}} unter Ihren Händen einen Zauberschimmer erhalten hat.\pend
           \pstart
           Sie sind ein grosser Menschenkenner, besonders ein Frauenkenner wie wenige. Meine
                    Erfahrungen stimmen nicht immer mit den Ihrigen überein. Aber der Menschenschlag
                    war verschieden, ich habe meistens \textcolor{pink}{Skandinavinnen}{}\ledrightnote{\textcolor{pink}{Skandinavien}} und \textcolor{pink}{Russinnen}{}\ledrightnote{\textcolor{pink}{Russland}}
                    gekannt, nie \textcolor{pink}{Oesterreicherinnen}{}\ledrightnote{\textcolor{pink}{Österreich}}. Die wenigen
                    dieser Nation, die ich getroffen habe, waren sehr prosaisch; alle Ihre Frauen
                    haben eine poetische Aureole.\pend
           \pstart
           Das \textcolor{green}{andere Buch}{}\ledrightnote{→\textcolor{green}{Fräulein Else}} dessen
                    erzählende Form an Ihr Meisterwerk über den {\pb}\textcolor{green}{Lieutenant Gustel}{}\ledrightnote{\textcolor{green}{Lieutenant Gustl. Novelle}} erinnert, ist ganz einfach
                    aufgebaut, durch traurige Wahrheit \uline{ergreifend}.
                    Sie haben den tragischen Ausgang gewollt, haben dem armen Mädchen die Auswege
                    versperrt. Am feinsten scheint mir in der Erzählung die Lebenslust, die das
                    junge \textcolor{green}{Mädchen}{}\ledrightnote{→\textcolor{green}{Fräulein Else}} an den \textcolor{green}{Vetter}{}\ledrightnote{→\textcolor{green}{Fräulein Else}} und an den \textcolor{green}{Fred}{}\ledrightnote{→\textcolor{green}{Fräulein Else}} zieht. Warum sind Sie
                    so hart gewesen, sie sterben zu lassen! – –\pend
           \pstart
           Sie werden bemerkt haben, dass die Jahre zwischen 80 und 90 nicht die Blüthezeit
                    der Weiber ist. Sie ist ja leider auch nicht die der Männer, wenn man sich auch
                    gern Illusionen macht.\pend
           \pstart
           Ich habe ein paar \textcolor{green}{Bücher}{}\ledrightnote{→\textcolor{green}{Hertuginden af Dino og Fyrsten af Talleyrand}{\newline}→\textcolor{green}{Uimodstaaelige: (Attende aarhundrede, Frankrig)}} über das 18. Jahrhundert in \textcolor{pink}{Frankreich}{}\ledrightnote{\textcolor{pink}{Frankreich}} herausgegeben über \label{K_L02422_1v}\edtext{\textcolor{blue}{Talleyrand}{}\ledrightnote{\textcolor{blue}{Charles Maurice de Talleyrand-Périgord}}}{\lemma{\textnormal{\emph{Talleyrand}}}\Cendnote{\textnormal{\textcolor{blue}{Georg Brandes}: \emph{\textcolor{green}{Hertuginden af Dino og Fyrsten af Talleyrand}}.
                            Kopenhagen: \emph{Gyldendalske Boghandel, Nordisk
                                forlag}{ }1923.}}}\label{K_L02422_1h}, über \label{K_L02422_2v}\edtext{\textcolor{blue}{Lauzun}{}\ledrightnote{\textcolor{blue}{Armand-Louis de Gontaut de Biron}}}{\lemma{\textnormal{\emph{Lauzun}}}\Cendnote{\textnormal{\textcolor{blue}{Georg Brandes}: \emph{\textcolor{green}{Uimodstaaelige. Attende aarhundrede. Frankrig}}.
                            Kopenhagen: \emph{Gyldendalske Boghandel, Nordisk
                                forlag}{ }1924.}}}\label{K_L02422_2h} etc. aber ich habe bisher die Uebersetzung verhindert da die
                    Form noch nicht endgültig ist.\pend
           \pstart
           {\pb}In der letzten Zeit habe ich
                    ein \textcolor{green}{Buch}{}\ledrightnote{→\textcolor{green}{Urkristendom}} auf dem Stapel,
                        \strikeout{von} das beweisen will dass das Leben \textcolor{blue}{Jesu}{}\ledrightnote{\textcolor{blue}{Jesus}} (ungefähr wie das Leben Wilhelm Tells)
                    nur Sage ist. Ich habe ein paar Kapitel schon veröffentlicht und werde bald
                    damit zu Ende sein, erwarte nur Rückkehr der Gesundheit. Es wird leider viel
                    Geheul verursachen.\pend
           \pstart
           Dieser Brief ist ein sehr schwacher Ausdruck meiner freundschaftlichen Gefühle.
                    Mit den Jahren blieben wenige zurück, denen man sich geistig verwandt fühlt und
                    von denen man etwas lernt. Sie sind einer von diesen ganz wenigen für mich.\pend
           \pstart
           Jemand sagte mir, ein \textcolor{green}{Buch}{}\ledrightnote{→\textcolor{green}{Gaius Julius Cæsar}}
                    das ich 1918 über \textcolor{blue}{\uline{Cäsar}}{}\ledrightnote{\textcolor{blue}{Gaius Iulius Caesar}} schrieb sei \label{K_L02422_3v}\edtext{deutsch
                        erschienen}{\lemma{\textnormal{\emph{deutsch
                        erschienen}}}\Cendnote{\textnormal{\textcolor{blue}{Georg Brandes}: \emph{\textcolor{green}{Cajus Julius Caesar}}. Autorisierte Übersetzung von
                                \textcolor{blue}{Erwin Magnus}. Berlin: \emph{\textcolor{brown}{Erich Reiss}}{ }1925 (erschienen September
                        1924).}}}\label{K_L02422_3h}. Ich habe weder ein Exemplar noch ein Honorar
                    gesehen.\pend
           \pstart Ihr\hspace*{3.5em}\spacefill\mbox{Georg Brandes}\pend{}\endnumbering\briefempfaengerindex{Schnitzler, Arthur@\textsc{Schnitzler, Arthur}!zzzBrandes, Georg@\emph{von Georg Brandes}!1924-12-101@{10. 12. 1924}|)be}\mylabel{h}  \normalsize

\doendnotes{C}
\bigskip
\vfill

\clearpage

\footnotesize

\lohead{\textsc{register}}

% Definiere theindex-Environment komplett neu ohne reledmac
\makeatletter
\renewenvironment{theindex}{%
  \section*{\indexname}%
  \setlength{\parindent}{0pt}%
  \setlength{\parskip}{0pt plus 0.3pt}%
  \let\item\@idxitem
}{%
  \clearpage
}
\makeatother

\IfFileExists{\jobname-pw.ind}{\input{\jobname-pw.ind}}{}

\end{document}

      