%% latex-korrekturansicht-vorspann.tex
%% Vorspann für die Korrekturansicht.
%% Lädt die gemeinsame Datei latex-vorspann.tex mit gesetztem Schalter.

\newif\ifkorrekturansicht
\korrekturansichttrue

\input{../tex-inputs/latex-vorspann}


               \section[Stefan Großmann an Arthur Schnitzler, 23. 11. 1925]{ Stefan Großmann an Arthur Schnitzler, 23. 11. 1925}\nopagebreak\mylabel{v}\rehead{ }\normalsize\beginnumbering\briefempfaengerindex{Schnitzler, Arthur@\textsc{Schnitzler, Arthur}!zzzGrossmann, Stefan@\emph{von Stefan Großmann}!1925-11-231@{23. 11. 1925}|(be} \toendnotes[C]{\smallbreak\pagebreak[2]} \Standort{DLA, A:Schnitzler, HS.NZ85.1.3232.}
\physDesc{Brief, 1 Blatt, 1 Seite
\newline{}Schreibmaschine
\newline{}Handschrift: schwarze Tinte (\noindent{}Unterschrift)
\newline{}Schnitzler: mit rotem Buntstift zwei Unterstreichungen }\pstart
           \noindent{}\centering{}{\pb}\textcolor{gray}{\textbf{\textcolor{brown}{Das Tage-Buch}{}\ledrightnote{\textcolor{brown}{Das Tage-Buch}}}}\pend
           \pstart
           \noindent{}\centering{}\textcolor{gray}{\textbf{\emph{Herausgeber: Stefan Großmann und \textcolor{blue}{Leopold Schwarzschild}{}\ledrightnote{\textcolor{blue}{Leopold Schwarzschild}}}}}\pend
           \pstart
           \noindent{}\centering{}\textcolor{gray}{\textbf{Tagebuchverlag m. b. H., \textcolor{pink}{Berlin SW 19}{}\ledrightnote{\textcolor{pink}{Berlin}}}}\pend
           \pstart
           \noindent{}\centering{}\textcolor{gray}{\textbf{\textcolor{pink}{BEUTHSTRASSE 19}{}\ledrightnote{\textcolor{pink}{Beuthstrasse}}}}\pend
           \pstart
           \noindent{}\centering{}\textcolor{gray}{\textbf{\emph{Telegramm-Adresse: Tagebuch \textcolor{pink}{Berlin}{}\ledrightnote{\textcolor{pink}{Berlin}} ⋅ Fernsprecher: Merkur
                     8790–8792}}}\pend
           \pstart
           \noindent{}\centering{}\textcolor{gray}{\textbf{\emph{\so{Sprechstunde der Redaktion: 12–1 Uhr}}}}\pend
           \pstart
           \noindent{}\centering{}\textcolor{gray}{\textbf{*}}\pend
           \pstart
           \noindent{}Tgb./Gr./Schl.\hfill \textcolor{pink}{Berlin}{}\ledrightnote{\textcolor{pink}{Berlin}}, den
                        23. November 1925.\pend
           \pstart
           \raggedleft{}Herrn\pend
           \pstart
           \noindent{}\raggedleft{}Dr. Arthur \so{Schnitzler}\pend
           \pstart
           \noindent{}\raggedleft{}\textcolor{pink}{\so{Wien} XVIII}{}\ledrightnote{\textcolor{pink}{XVIII., Währing}}\pend
           \pstart
           \noindent{}\raggedleft{}\textcolor{pink}{Sternwartestr. 71}{}\ledrightnote{\textcolor{pink}{Sternwartestraße}}. \pend
           \pstart\center{}Verehrter Herr Doktor Schnitzler!\pend\pstart
           Sie waren so freundlich, mir im Prinzip einen Beitrag füs \textcolor{brown}{TAGE-BUCH}{}\ledrightnote{\textcolor{brown}{Das Tage-Buch}} zu versprechen. Sie würden mich zu grossem Dank verpflichten, wenn
               Sie mir den Beitrag jetzt schicken wollten; ich würde ihn dann um die Jahreswende
               veröffentlichen und gerade diese Hefte, die in verstärkter Auflage erscheinen, sind
               für uns von grösster Wichtigkeit.\pend
           \pstart
           In der Hoffnung, recht bald von Ihnen zu hören, bin ich mit dem Ausdruck der
               vorzüglichsten\pend
           \pstart
           Hochachtung{\\[\baselineskip]}ganz ergebenst{\\[\baselineskip]}\spacefill\mbox{{[}hs.:{]} Stefan Großmann}\pend
           \leftskip=0em{}\endnumbering\briefempfaengerindex{Schnitzler, Arthur@\textsc{Schnitzler, Arthur}!zzzGrossmann, Stefan@\emph{von Stefan Großmann}!1925-11-231@{23. 11. 1925}|)be}\mylabel{h}  \normalsize

\doendnotes{C}
\bigskip
\vfill

\clearpage

\footnotesize

\lohead{\textsc{register}}

% Definiere theindex-Environment komplett neu ohne reledmac
\makeatletter
\renewenvironment{theindex}{%
  \section*{\indexname}%
  \setlength{\parindent}{0pt}%
  \setlength{\parskip}{0pt plus 0.3pt}%
  \let\item\@idxitem
}{%
  \clearpage
}
\makeatother

\IfFileExists{\jobname-pw.ind}{\input{\jobname-pw.ind}}{}

\end{document}

      