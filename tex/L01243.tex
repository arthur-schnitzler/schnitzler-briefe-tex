%% latex-korrekturansicht-vorspann.tex
%% Vorspann für die Korrekturansicht.
%% Lädt die gemeinsame Datei latex-vorspann.tex mit gesetztem Schalter.

\newif\ifkorrekturansicht
\korrekturansichttrue

\input{../tex-inputs/latex-vorspann}


               \section[Arthur Schnitzler an Hugo von Hofmannsthal, 21. 10. 1902]{ Arthur Schnitzler an Hugo von Hofmannsthal, 21. 10. 1902}\nopagebreak\mylabel{v}\rehead{ }\normalsize\beginnumbering\briefempfaengerindex{Hofmannsthal, Hugo von@\textsc{Hofmannsthal, Hugo von}!zzzSchnitzler, Arthur@\emph{von Arthur Schnitzler}!1902-10-211@{21. 10. 1902}|(be} \toendnotes[C]{\smallbreak\pagebreak[2]} \Standort{FDH, Hs-30885,99.}
\physDesc{Postkarte
\newline{}Handschrift: schwarze Tinte, deutsche Kurrent\newline{}Versand: Stempel: »\nobreak{}\oindex{IX., Alsergrund@\textbf{IX., Alsergrund}, \emph{Bezirk (A.BZK)}|pwk}9/3 Wien 72, 21. 10. 02, 8N\nobreak{}«.  \newline{}Ordnung: von Schnitzler mutmaßlich
                           bei der Durchsicht der Korrespondenz 1929 mit Bleistift beschriftet:
                                 »\textcolor{pink}{Rom}{ }1903.« }\buchAbdrucke{\weitereDrucke{Hugo von Hofmannsthal, Arthur Schnitzler: \emph{Briefwechsel}. Hg. Therese Nickl und Heinrich Schnitzler. Frankfurt am Main: \emph{S. Fischer} 1964, S. 162.} }\toendnotes[C]{\smallbreak}\pstart{}{\pb}\textsc{Hrn Hugo v. Hofmannsthal}\pend{}\pstart{}\textcolor{pink}{\textsc{Rom}}{}\ledrightnote{\textcolor{pink}{Rom}}\pend{}\pstart{}\textsc{\textcolor{pink}{Hotel Hassler}{}\ledrightnote{\textcolor{pink}{Hôtel Hassler}}}\pend{}\pstart{}\textcolor{pink}{\textsc{Italia}}{}\ledrightnote{\textcolor{pink}{Italien}}\pend{}{\bigskip}\pstart
           \noindent{}{\pb}lieber, die \textcolor{blue}{Sandrock}{}\ledrightnote{\textcolor{blue}{Adele Sandrock}} möchte den
                  \textcolor{green}{Tod des Tizian}{}\ledrightnote{\textcolor{green}{Der Tod des Tizian}}, wohl um ihn vorzuleſen; – bitte
               ſehr laſſen Sie ihr ein Exemplar ſenden.\pend
           \pstart
           – Ich bin heute Früh aus \textcolor{pink}{\textsc{Agnetendorf}}{}\ledrightnote{\textcolor{pink}{Agnetendorf}} gekommen, wo ich nach 6tägigem \textcolor{pink}{Berlin}{}\ledrightnote{\textcolor{pink}{Berlin}}er
               Aufenthalt, 1 Tag mit \textcolor{blue}{Brahm}{}\ledrightnote{\textcolor{blue}{Otto Brahm}} bei \textcolor{blue}{Hauptmann}{}\ledrightnote{\textcolor{blue}{Gerhart Hauptmann}}{ }ſehr angenehm
               verbrachte. –\pend
           \pstart
           \textcolor{green}{\textsc{Beatrice}}{}\ledrightnote{\textcolor{green}{Der Schleier der Beatrice. Schauspiel in fünf Akten}} dürfte im Feber am \textcolor{pink}{Dtſch. Th.}{}\ledrightnote{\textcolor{pink}{Deutsches Theater Berlin}}
               geſpielt werden. –\pend
           \pstart
           \textcolor{green}{\textsc{M. Vanna}}{}\ledrightnote{\textcolor{green}{Monna Vanna}} iſt ein außerordentlicher Kaſſenerfolg. Die \label{K_L01243_1v}\edtext{Aufführung}{\lemma{\textnormal{\emph{Aufführung}}}\Cendnote{\textnormal{Er besuchte die Vorstellung am 14. 10. 1902. Zum Urteil
                     Vgl. A. S.: \emph{Tagebuch}, 19. 10. 1902.}}}\label{K_L01243_1h} läßt zu
               wünſchen übrig. Haben Sie meinen Brief erhalten? – Schreiben Sie ein Wort, wie’s
               Ihnen geht.\pend
           \pstart Herzlichſt Ihr \spacefill\mbox{A.}\pend{}\endnumbering\briefempfaengerindex{Hofmannsthal, Hugo von@\textsc{Hofmannsthal, Hugo von}!zzzSchnitzler, Arthur@\emph{von Arthur Schnitzler}!1902-10-211@{21. 10. 1902}|)be}\mylabel{h}  \normalsize

\doendnotes{C}
\bigskip
\vfill

\clearpage

\footnotesize

\lohead{\textsc{register}}

% Definiere theindex-Environment komplett neu ohne reledmac
\makeatletter
\renewenvironment{theindex}{%
  \section*{\indexname}%
  \setlength{\parindent}{0pt}%
  \setlength{\parskip}{0pt plus 0.3pt}%
  \let\item\@idxitem
}{%
  \clearpage
}
\makeatother

\IfFileExists{\jobname-pw.ind}{\input{\jobname-pw.ind}}{}

\end{document}

      