%% latex-korrekturansicht-vorspann.tex
%% Vorspann für die Korrekturansicht.
%% Lädt die gemeinsame Datei latex-vorspann.tex mit gesetztem Schalter.

\newif\ifkorrekturansicht
\korrekturansichttrue

\input{../tex-inputs/latex-vorspann}


               \section[Richard Beer-Hofmann an Arthur Schnitzler, 6. {[}7. 1908{]}]{ Richard Beer-Hofmann an Arthur Schnitzler, 6. {[}7. 1908{]}}\nopagebreak\mylabel{v}\rehead{ }\normalsize\beginnumbering\briefempfaengerindex{Schnitzler, Arthur@\textsc{Schnitzler, Arthur}!zzzBeer-Hofmann, Richard@\emph{von Richard Beer-Hofmann}!1908-07-062@{6. {[}7. 1908{]}}|(be} \toendnotes[C]{\smallbreak\pagebreak[2]} \Standort{CUL, Schnitzler, B 8.}
\physDesc{Bildpostkarte
\newline{}Handschrift: schwarze Tinte, lateinische Kurrent\newline{}Versand: Stempel: »\nobreak{}\oindex{IX., Alsergrund@\textbf{IX., Alsergrund}, \emph{Bezirk (A.BZK)}|pwk}9/4 W{[}ien{]}, 6.V\textcolor{gray}{II.08}, \textcolor{gray}{8}\nobreak{}«.  
\newline{}Schnitzler: mit Bleistift datiert: »6/7 (?) 08« und beschriftet: »\textsc{Beerhfm}« \newline{}Ordnung: 1) mit Bleistift von unbekannter Hand nummeriert: »\strikeout{215}« 2) mit Bleistift von unbekannter Hand nummeriert: »216«}\pstart{}{\pb}Herrn\pend{}\pstart{} D\textsuperscript{r} Arthur Schnitzler\pend{}\pstart{}\textcolor{pink}{Seis a. Schlern}{}\ledrightnote{\textcolor{pink}{Seis am Schlern}}\pend{}\pstart{}\textcolor{pink}{Villa Heufler}{}\ledrightnote{\textcolor{pink}{Villa Heufler}}\pend{}\pstart{}\textcolor{pink}{Tirol}{}\ledrightnote{\textcolor{pink}{Südtirol}}\pend{}{\bigskip}\pstart
           \noindent{}\centering{}{\pb}\textcolor{gray}{\textbf{\textcolor{pink}{Wien}{}\ledrightnote{\textcolor{pink}{Wien}}. Panorama vom \textcolor{pink}{Türkenschanzpark}{}\ledrightnote{\textcolor{pink}{Türkenschanzpark}} aus.}}\pend
           \pstart{}{\pb}Lieber Arthur!\pend\pstart
           Wir wollen von 14 \introOben{}Juli\introOben{} an in \uline{\textcolor{pink}{Strobl}{}\ledrightnote{\textcolor{pink}{Strobl}}}{ }\uline{\textcolor{pink}{Hôtel a See}{}\ledrightnote{\textcolor{pink}{Hotel am See}}} sein.\pend
           \pstart
           Herzliche Grüsse Ihnen u. den Ihren.{\\[\baselineskip]}\spacefill\mbox{Richard}\pend
           \leftskip=0em{}\endnumbering\briefempfaengerindex{Schnitzler, Arthur@\textsc{Schnitzler, Arthur}!zzzBeer-Hofmann, Richard@\emph{von Richard Beer-Hofmann}!1908-07-062@{6. {[}7. 1908{]}}|)be}\mylabel{h}  \normalsize

\doendnotes{C}
\bigskip
\vfill

\clearpage

\footnotesize

\lohead{\textsc{register}}

% Definiere theindex-Environment komplett neu ohne reledmac
\makeatletter
\renewenvironment{theindex}{%
  \section*{\indexname}%
  \setlength{\parindent}{0pt}%
  \setlength{\parskip}{0pt plus 0.3pt}%
  \let\item\@idxitem
}{%
  \clearpage
}
\makeatother

\IfFileExists{\jobname-pw.ind}{\input{\jobname-pw.ind}}{}

\end{document}

      