%% latex-korrekturansicht-vorspann.tex
%% Vorspann für die Korrekturansicht.
%% Lädt die gemeinsame Datei latex-vorspann.tex mit gesetztem Schalter.

\newif\ifkorrekturansicht
\korrekturansichttrue

\input{../tex-inputs/latex-vorspann}


               \section[Hermann Bahr an Arthur Schnitzler, {[}22. 12. 1891{]}]{ Hermann Bahr an Arthur Schnitzler, {[}22. 12. 1891{]}}\nopagebreak\mylabel{v}\rehead{ }\normalsize\beginnumbering\briefempfaengerindex{Schnitzler, Arthur@\textsc{Schnitzler, Arthur}!zzzBahr, Hermann@\emph{von Hermann Bahr}!1891-12-221@{22. 12. 1891}|(be} \toendnotes[C]{\smallbreak\pagebreak[2]} \Standort{CUL, Schnitzler, B 5b.}
\physDesc{Brief, 1 Blatt, 1 Seite
\newline{}Handschrift: Bleistift, deutsche Kurrent
\newline{}Schnitzler: 1) mit Bleistift datiert: »22/12 91.
                                 « 2) mit rotem Buntstift nummeriert: »1.«\newline{}Ordnung: mit Bleistift von unbekannter Hand nummeriert:
                              »1.« und verso »\textsc{Bahr}« beschriftet }\buchAbdrucke{\weitereDrucke{Hermann Bahr, Arthur Schnitzler: \emph{Briefwechsel, Aufzeichnungen, Dokumente (1891–1931)}. Hg. Kurt Ifkovits und Martin Anton Müller. Göttingen: \emph{Wallstein} 2018, S. 16.} }\toendnotes[C]{\smallbreak}\pstart{}{\pb}Lieber Herr Dr!\pend\pstart
           Bitte, teilen Sie mir we{\geminationn} möglich mit, ob es Ihnen paßt,
               daß uns morgen \introOben{}Mittwoch\introOben{} Abend von 6–8 (ſei es bei Ihnen,
               oder bei mir) \textcolor{blue}{\textsc{Bératon}}{}\ledrightnote{\textcolor{blue}{Ferry Bératon}}{ }\label{K_L00056_1v}\edtext{ſein Stück}{\lemma{\textnormal{\emph{ſein Stück}}}\Cendnote{\textnormal{Unklar. Nachdem am 2. 5. 1892{ }\emph{\textcolor{green}{L’intruse}} von \textcolor{blue}{Maurice
                     Maeterlinck} in \textcolor{blue}{Bératons}
                  Übersetzung gegeben wurde und zuvor weitere Dramen des Autors zur Inszenierung
                  angedacht waren, könnte es sich um eine Übertragung von \emph{\textcolor{green}{La Princesse Maleine}} handeln.}}}\label{K_L00056_1h} vorlieſt. Ich möchte Sie
               bitten, mich etwa bis 5 zu verſtändigen, da ich noch zu \textcolor{blue}{\textsc{Loris}}{}\ledrightnote{\textcolor{blue}{Hugo von Hofmannsthal}}{ }ſchicken u \textcolor{blue}{\textsc{Beraton}}{}\ledrightnote{\textcolor{blue}{Ferry Bératon}} Antwort ſagen muß.\pend
           \pstart
           \substVorne{}\textsuperscript{\textcolor{gray}{M}}\substDazwischen{}Im\substHinten{} übrigen bitte größte Discretion! \textcolor{blue}{B.}{}\ledrightnote{\textcolor{blue}{Ferry Bératon}} will
               nicht, daß »die Welt« etwas von ſr Miſſetat erfahre.\pend
           \pstart
           Herzlichſt{\\[\baselineskip]}\spacefill\mbox{Bahr.}\pend
           \leftskip=0em{}\endnumbering\briefempfaengerindex{Schnitzler, Arthur@\textsc{Schnitzler, Arthur}!zzzBahr, Hermann@\emph{von Hermann Bahr}!1891-12-221@{22. 12. 1891}|)be}\mylabel{h}  \normalsize

\doendnotes{C}
\bigskip
\vfill

\clearpage

\footnotesize

\lohead{\textsc{register}}

% Definiere theindex-Environment komplett neu ohne reledmac
\makeatletter
\renewenvironment{theindex}{%
  \section*{\indexname}%
  \setlength{\parindent}{0pt}%
  \setlength{\parskip}{0pt plus 0.3pt}%
  \let\item\@idxitem
}{%
  \clearpage
}
\makeatother

\IfFileExists{\jobname-pw.ind}{\input{\jobname-pw.ind}}{}

\end{document}

      