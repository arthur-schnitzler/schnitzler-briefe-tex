%% latex-korrekturansicht-vorspann.tex
%% Vorspann für die Korrekturansicht.
%% Lädt die gemeinsame Datei latex-vorspann.tex mit gesetztem Schalter.

\newif\ifkorrekturansicht
\korrekturansichttrue

\input{../tex-inputs/latex-vorspann}


               \section[Arthur Schnitzler an Hermann Bahr, 11. 10. 1907]{ Arthur Schnitzler an Hermann Bahr, 11. 10. 1907}\nopagebreak\mylabel{v}\rehead{ }\normalsize\beginnumbering\briefempfaengerindex{Bahr, Hermann@\textsc{Bahr, Hermann}!zzzSchnitzler, Arthur@\emph{von Arthur Schnitzler}!1907-10-111@{11. 10. 1907}|(be} \toendnotes[C]{\smallbreak\pagebreak[2]} \Standort{TMW, HS AM 23387 Ba.}
\physDesc{Brief, 2 Blätter, 2 Seiten
\newline{}Schreibmaschine
\newline{}Handschrift: schwarze Tinte, deutsche Kurrent (\noindent{}Unterschrift, Korrekturen und Nachschrift)}\Standort{DLA, A:Schnitzler, 85.1.294/2.}
\physDesc{Brief, 2 Blätter, 2 Seiten, maschineller Durchschlag
\newline{}Schreibmaschine
\newline{}Handschrift: Bleistift, deutsche Kurrent (\noindent{}Ergänzung: »bold«)}\buchAbdrucke{\weitereDrucke{1) Arthur Schnitzler: \emph{Briefe 1875–1912}. Hg. Therese Nickl und Heinrich Schnitzler. Frankfurt am Main: \emph{S. Fischer} 1981, S. 564–565.} \weitereDrucke{2) \emph{11. 10. 1907.} In: Arthur Schnitzler: \emph{The Letters of Arthur Schnitzler to Hermann Bahr}. Edited, annotated, and with an introduction, by Donald G.
                        Daviau. Chapel Hill: \emph{The University of North Carolina Press} 1978, S. 99–101 (University of North Carolina studies in the Germanic languages
                        and literatures, 89).} \weitereDrucke{3) Hermann Bahr, Arthur Schnitzler: \emph{Briefwechsel, Aufzeichnungen, Dokumente (1891–1931)}. Hg. Kurt Ifkovits und Martin Anton Müller. Göttingen: \emph{Wallstein} 2018, S. 397–398.} }\toendnotes[C]{\smallbreak}\pstart
           \noindent{}{\pb}\textcolor{gray}{\textbf{Dr. Arthur Schnitzler}}\hfill 11. Okt. 07\pend
           \pstart
           \textcolor{gray}{\textbf{\textcolor{pink}{Wien XVIII. Spoettelgasse 7}{}\ledrightnote{\textcolor{pink}{Edmund-Weiß-Gasse}}.}}\pend
           \pstart{}Lieber Hermann,\pend\pstart
           Ich danke Dir sehr, dass Du mir ermöglicht hast Dein neues \textcolor{green}{Stück}{}\ledrightnote{→\textcolor{green}{Die gelbe Nachtigall}} zu lesen. Dass Du es kurzweg als Scherz
               bezeichnest nehme ich als Koketterie. Ich habe durchaus Vergnügen und sehr oft Freude
               daran gehabt. Man wünschte sich vielleicht Gestalten wie \textcolor{green}{Korz}{}\ledrightnote{→\textcolor{green}{Die gelbe Nachtigall}} und \textcolor{green}{Fanny}{}\ledrightnote{→\textcolor{green}{Die gelbe Nachtigall}}, auch \textcolor{green}{Jason}{}\ledrightnote{→\textcolor{green}{Die gelbe Nachtigall}} und
               die geringern, in einer ernster bewegten Welt wiederzufinden, wie ich überhaupt die
               Charakteristik und Karikaturistik in dem Stück noch höher werten möchte, als das
               anekdotische Element. Ich hoffe | aus praktischen Gründen | Du bereitest das Publikum
               durch einen glücklichen \label{K_L01721_1v}\edtext{Untertitel}{\lemma{\textnormal{\emph{Untertitel}}}\Cendnote{\textnormal{Das Werk erschien ohne
                  Untertitel.}}}\label{K_L01721_1h}{ }\label{T_L01721_1v}\edtext{ein wenig}{\lemma{\textnormal{\emph{ein wenig}}}\Cendnote{\textnormal{korrigiert aus: »einwenig«}}}\label{T_L01721_1h} vor, wie es seine Augen
               einzustellen hat, um mit ungestörter Lust schauen und geniessen zu dürfen. Nennst das
               Ganze vielleicht burleske Komödie oder so ähnlich. Ferner, wenn mir ein bescheidener
               Rat gestattet ist, würde ich die \label{K_L01721_2v}\edtext{Schlussscene des zweiten Aktes | den schwarzen Kuss | streichen}{\lemma{\textnormal{\emph{Schlussscene … streichen}}}\Cendnote{\textnormal{Auch die gedruckte Fassung enthält an
                  dieser Stelle eine Szene mit schmatzendem Kuss, dürfte also nicht geändert worden
                  sein.}}}\label{K_L01721_2h}, da mir ihr Humor zu \textcolor{blue}{Kadlbürgerlich}{}\ledrightnote{\textcolor{blue}{Gustav Kadelburg}} scheint im Verhältnis zu der grotesken Laune, die sonst durch
               die Komödie fegt. Ob es den Leuten möglich sein wird sich ganz nach Deinem {\pb}Willen in die
               gemässigtere Haltung des Schlusses zu finden, wag ich nicht vorher zu sagen. Für das,
               was den »Witz« in Deinem Stücke vorstellt, reicht natürlich auch das aus, was am Ende
               die »Pointe« wird, \introOben{}–\introOben{} und das burleske widersetzt sich seiner
               ganzen Natur nach jeder entgiltigen Erledigung. Es ist gleichsam aus dem Chaos selbst
               geboren, während der Witz doch immer ein Spross des Tages ist, in einer Art von
               festem Verhältnis zu unsern Sitten, unserer Ordnung, unserer Tradition steht, auch
               wenn er sich über sie lustig zu machen scheint{[}.{]} Der Witz\introOben{}bold\introOben{} besieht sich die Erde von einem Fesselballon aus, der
               Burleskant schwebt frei in den Lüften. In ihm steckt so sicher ein Anarchist, wie im
               Witzbold ein Pedant. Dies nur nebenbei | wie es das Los der allgemeinen Bemerkungen
               nun ist. | Im übrigen glaub ich, dass sich die Leute bei Deinem Stück sehr amüsieren
               werden, selbst wenn sie es verstehen sollten.\pend
           \pstart
           {[}hs.:{]} herzlichſte Grüße, auch von meiner \textcolor{blue}{Frau}{}\ledrightnote{→\textcolor{blue}{Olga Schnitzler}}.{\\[\baselineskip]}laß doch, von Zeit zu Zeit ein Wort von dir hören.{\\[\baselineskip]}Dein{\\[\baselineskip]}\spacefill\mbox{Arthur}\pend
           \leftskip=0em{}\endnumbering\briefempfaengerindex{Bahr, Hermann@\textsc{Bahr, Hermann}!zzzSchnitzler, Arthur@\emph{von Arthur Schnitzler}!1907-10-111@{11. 10. 1907}|)be}\mylabel{h}  \normalsize

\doendnotes{C}
\bigskip
\vfill

\clearpage

\footnotesize

\lohead{\textsc{register}}

% Definiere theindex-Environment komplett neu ohne reledmac
\makeatletter
\renewenvironment{theindex}{%
  \section*{\indexname}%
  \setlength{\parindent}{0pt}%
  \setlength{\parskip}{0pt plus 0.3pt}%
  \let\item\@idxitem
}{%
  \clearpage
}
\makeatother

\IfFileExists{\jobname-pw.ind}{\input{\jobname-pw.ind}}{}

\end{document}

      