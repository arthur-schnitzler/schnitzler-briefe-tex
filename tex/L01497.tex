%% latex-korrekturansicht-vorspann.tex
%% Vorspann für die Korrekturansicht.
%% Lädt die gemeinsame Datei latex-vorspann.tex mit gesetztem Schalter.

\newif\ifkorrekturansicht
\korrekturansichttrue

\input{../tex-inputs/latex-vorspann}


               \section[Hugo und Gerty von Hofmannsthal, Gerhart und Margarete Hauptmann an Arthur Schnitzler, 29. 1. 1905]{ Hugo und Gerty von Hofmannsthal, Gerhart und Margarete Hauptmann an
               Arthur Schnitzler, 29. 1. 1905}\nopagebreak\mylabel{v}\rehead{ }\normalsize\beginnumbering\briefempfaengerindex{Schnitzler, Arthur@\textsc{Schnitzler, Arthur}!zzzHauptmann, Margarete@\emph{von Margarete Hauptmann}!1905-01-292@{29. 1. 1905}|(be}\briefempfaengerindex{Schnitzler, Arthur@\textsc{Schnitzler, Arthur}!zzzHauptmann, Gerhart@\emph{von Gerhart Hauptmann}!1905-01-292@{29. 1. 1905}|(be}\briefempfaengerindex{Schnitzler, Arthur@\textsc{Schnitzler, Arthur}!zzzHofmannsthal, Gertrude von@\emph{von Gertrude von Hofmannsthal}!1905-01-292@{29. 1. 1905}|(be}\briefempfaengerindex{Schnitzler, Arthur@\textsc{Schnitzler, Arthur}!zzzHofmannsthal, Hugo von@\emph{von Hugo von Hofmannsthal}!1905-01-292@{29. 1. 1905}|(be} \toendnotes[C]{\smallbreak\pagebreak[2]} \Standort{CUL, Schnitzler, B 43.}
\physDesc{Bildpostkarte
\newline{}Handschrift Hugo von Hofmannsthal: schwarze Tinte, lateinische Kurrent\newline{}Handschrift Gertrude von Hofmannsthal: schwarze Tinte, lateinische Kurrent\newline{}Handschrift Gerhart Hauptmann: schwarze Tinte, lateinische Kurrent\newline{}Handschrift Margarete Hauptmann: schwarze Tinte, lateinische Kurrent\newline{}Versand: Stempel: »\nobreak{}\oindex{Agnetendorf@\textbf{Agnetendorf}, \emph{Besiedelter Ort (A.BSO)}|pwk}Agnetendorf, 30. 1. 05, 6–\textcolor{gray}{7}N\nobreak{}«.  \newline{}Ordnung: 1) mit Bleistift von unbekannter Hand nummeriert: »\strikeout{222}« 2) mit Bleistift von unbekannter Hand nummeriert:
                                    »248«}\buchAbdrucke{\weitereDrucke{Hugo von Hofmannsthal, Arthur Schnitzler: \emph{Briefwechsel}. Hg. Therese Nickl und Heinrich Schnitzler. Frankfurt am Main: \emph{S. Fischer} 1964, S. 210.} }\toendnotes[C]{\smallbreak}\pstart{}{\pb}Herrn D\textsuperscript{r} Arthur Schnitzler\pend{}\pstart{}\textcolor{pink}{Wien}{}\ledrightnote{\textcolor{pink}{Wien}}\pend{}\pstart{}\textcolor{pink}{XVIII Spöttelgasse 7}{}\ledrightnote{\textcolor{pink}{Edmund-Weiß-Gasse}}.\pend{}{\bigskip}\pstart
           \noindent{}\centering{}\textcolor{gray}{\textbf{{\pb}\textcolor{pink}{Agnetendorf}{}\ledrightnote{\textcolor{pink}{Agnetendorf}}}}\pend
           \pstart
           \noindent{}\centering{}\textcolor{gray}{\textbf{\label{T_L01497-1v}\edtext{Besitzung Gerhart Hauptmann}{\lemma{\textnormal{\emph{Besitzung … Hauptmann}}}\Cendnote{\textnormal{handschriftlich
                        gestrichen oder unterstrichen?}}}\label{T_L01497-1h}}}\pend
           \pstart
           \noindent{}\centering{}\textcolor{gray}{\textbf{Der \textcolor{pink}{Wiesenstein}{}\ledrightnote{\textcolor{pink}{Haus Wiesenstein}}}}\pend
           \pstart
           \raggedleft{}29 I. 05\pend
           \pstart
           wünschen wohl zu dichten!\pend
           \pstart
           \spacefill\mbox{Hugo}{\\[\baselineskip]}{[}hs. Hauptmann:{]} u{\\[\baselineskip]}\spacefill\mbox{Gerhart}{\\[\baselineskip]}\spacefill\mbox{{[}hs. G. Hofmannsthal:{]} Gerty}{\\[\baselineskip]}\spacefill\mbox{{[}hs. Hauptmann:{]} Grete H.}\pend
           \leftskip=0em{}\endnumbering\briefempfaengerindex{Schnitzler, Arthur@\textsc{Schnitzler, Arthur}!zzzHauptmann, Margarete@\emph{von Margarete Hauptmann}!1905-01-292@{29. 1. 1905}|)be}\briefempfaengerindex{Schnitzler, Arthur@\textsc{Schnitzler, Arthur}!zzzHauptmann, Gerhart@\emph{von Gerhart Hauptmann}!1905-01-292@{29. 1. 1905}|)be}\briefempfaengerindex{Schnitzler, Arthur@\textsc{Schnitzler, Arthur}!zzzHofmannsthal, Gertrude von@\emph{von Gertrude von Hofmannsthal}!1905-01-292@{29. 1. 1905}|)be}\briefempfaengerindex{Schnitzler, Arthur@\textsc{Schnitzler, Arthur}!zzzHofmannsthal, Hugo von@\emph{von Hugo von Hofmannsthal}!1905-01-292@{29. 1. 1905}|)be}\mylabel{h}  \normalsize

\doendnotes{C}
\bigskip
\vfill

\clearpage

\footnotesize

\lohead{\textsc{register}}

% Definiere theindex-Environment komplett neu ohne reledmac
\makeatletter
\renewenvironment{theindex}{%
  \section*{\indexname}%
  \setlength{\parindent}{0pt}%
  \setlength{\parskip}{0pt plus 0.3pt}%
  \let\item\@idxitem
}{%
  \clearpage
}
\makeatother

\IfFileExists{\jobname-pw.ind}{\input{\jobname-pw.ind}}{}

\end{document}

      