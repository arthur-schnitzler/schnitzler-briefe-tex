%% latex-korrekturansicht-vorspann.tex
%% Vorspann für die Korrekturansicht.
%% Lädt die gemeinsame Datei latex-vorspann.tex mit gesetztem Schalter.

\newif\ifkorrekturansicht
\korrekturansichttrue

\input{../tex-inputs/latex-vorspann}


               \section[Hugo von Hofmannsthal an Arthur Schnitzler, 13. 10. 1893]{ Hugo von Hofmannsthal an Arthur Schnitzler, 13. 10. 1893}\nopagebreak\mylabel{v}\rehead{ }\normalsize\beginnumbering\briefempfaengerindex{Schnitzler, Arthur@\textsc{Schnitzler, Arthur}!zzzHofmannsthal, Hugo von@\emph{von Hugo von Hofmannsthal}!1893-10-132@{13. 10. 1893}|(be} \toendnotes[C]{\smallbreak\pagebreak[2]} \Standort{CUL, Schnitzler, B 43.}
\physDesc{Postkarte
\newline{}Handschrift: Bleistift, deutsche Kurrent\newline{}Versand: 1) Rohrpost 2) Stempel: »\nobreak{}Wien 3/1, 13 X 93, 6 20N\nobreak{}«. 3) Stempel: »\nobreak{}Wien 1/1, 13 X 93, 6 50N\nobreak{}«. 
\newline{}Schnitzler: mit Bleistift nummeriert: »53« und datiert: »13/X 93« }\buchAbdrucke{\weitereDrucke{Hugo von Hofmannsthal, Arthur Schnitzler: \emph{Briefwechsel}. Hg. Therese Nickl und Heinrich Schnitzler. Frankfurt am Main: \emph{S. Fischer} 1964, S. 47.} }\toendnotes[C]{\smallbreak}\pstart{}{\pb}\textsc{Herrn D\textsuperscript{r} Arthur
                            Schnitzler}\pend{}\pstart{}\textsc{\textcolor{pink}{I. Grillparzerstrasse 7}{}\ledrightnote{\textcolor{pink}{Grillparzerstraße}}}\pend{}\pstart{}\textsc{\textcolor{pink}{Wien}{}\ledrightnote{\textcolor{pink}{Wien}}}\pend{}{\bigskip}\pstart{}{\pb}lieber Arthur!\pend\pstart
           Der arme \label{K_L00271_1v}\edtext{\textcolor{blue}{Rudolf Schwarzk.}{}\ledrightnote{\textcolor{blue}{Rudolf Schwarzkopf}}}{\lemma{\textnormal{\emph{Rudolf Schwarzk.}}}\Cendnote{\textnormal{Die Todesanzeige (\emph{\textcolor{brown}{Neue Freie Presse}}, Nr. 10470,
                                15. 10. 1893, S. 14) belegt die
                        Datierung.}}}\label{K_L00271_1h} iſt heute früh in \textcolor{pink}{Meran}{}\ledrightnote{\textcolor{pink}{Meran}} geſtorben. \textcolor{blue}{Guſtav}{}\ledrightnote{\textcolor{blue}{Gustav Schwarzkopf}}
                    begräbt ihn dort. \textcolor{blue}{Max}{}\ledrightnote{\textcolor{blue}{Max Schwarzkopf}} iſt hier. Es iſt ſehr
                    traurig.\pend
           \pstart \spacefill\mbox{Hugo.}\pend{}\endnumbering\briefempfaengerindex{Schnitzler, Arthur@\textsc{Schnitzler, Arthur}!zzzHofmannsthal, Hugo von@\emph{von Hugo von Hofmannsthal}!1893-10-132@{13. 10. 1893}|)be}\mylabel{h}  \normalsize

\doendnotes{C}
\bigskip
\vfill

\clearpage

\footnotesize

\lohead{\textsc{register}}

% Definiere theindex-Environment komplett neu ohne reledmac
\makeatletter
\renewenvironment{theindex}{%
  \section*{\indexname}%
  \setlength{\parindent}{0pt}%
  \setlength{\parskip}{0pt plus 0.3pt}%
  \let\item\@idxitem
}{%
  \clearpage
}
\makeatother

\IfFileExists{\jobname-pw.ind}{\input{\jobname-pw.ind}}{}

\end{document}

      