%% latex-korrekturansicht-vorspann.tex
%% Vorspann für die Korrekturansicht.
%% Lädt die gemeinsame Datei latex-vorspann.tex mit gesetztem Schalter.

\newif\ifkorrekturansicht
\korrekturansichttrue

\input{../tex-inputs/latex-vorspann}


               \section[Hugo von Hofmannsthal an Arthur Schnitzler, {[}8. 9. 1899{]}]{ Hugo von Hofmannsthal an Arthur Schnitzler, {[}8. 9. 1899{]}}\nopagebreak\mylabel{v}\rehead{ }\normalsize\beginnumbering\briefempfaengerindex{Schnitzler, Arthur@\textsc{Schnitzler, Arthur}!zzzHofmannsthal, Hugo von@\emph{von Hugo von Hofmannsthal}!1899-09-081@{{[}8. 9. 1899{]}}|(be} \toendnotes[C]{\smallbreak\pagebreak[2]} \Standort{CUL, Schnitzler, B 43.}
\physDesc{Brief, 1 Blatt, 4 Seiten
\newline{}Handschrift: schwarze Tinte, deutsche Kurrent
\newline{}Schnitzler: mit Bleistift datiert: »7/9. 99.« \newline{}Ordnung: 1) mit Bleistift von unbekannter Hand nummeriert: »\strikeout{160}« 2) mit Bleistift von unbekannter Hand nummeriert:
                                    »157«}\buchAbdrucke{\weitereDrucke{Hugo von Hofmannsthal, Arthur Schnitzler: \emph{Briefwechsel}. Hg. Therese Nickl und Heinrich Schnitzler. Frankfurt am Main: \emph{S. Fischer} 1964, S. 129–130.} }\toendnotes[C]{\smallbreak}\pstart{}{\pb}mein lieber Arthur\pend\pstart
           ſeien Sie nicht bös ich hab in meinen Kopfſchmerzen \label{K_L00969_1v}\edtext{geſtern}{\lemma{\textnormal{\emph{geſtern}}}\Cendnote{\textnormal{In \textcolor{blue}{Schnitzler}s \emph{\textcolor{green}{Tagebuch}} ist die Abreise am 7. 9. 1899 vermerkt. Entsprechend ist dieses
                  Korrespondenzstück auf den Folgetag zu datieren.}}}\label{K_L00969_1h} verſchiedenes in \textcolor{pink}{Iſchl}{}\ledrightnote{\textcolor{pink}{Bad Ischl}} liegen laſſen. Bitte ſeien Sie ſo lieb und
               verſchaffen mirs wieder. Erſtens hab ich in meinem Bett mein Nachthemd liegen laſſen.
               Bitte vielmals laſſens {\pb}Sie mirs
               durch den \textcolor{blue}{\textsc{Petter}}{}\ledrightnote{\textcolor{blue}{Leopold Petter}}{ }ſchicken, als Poſtpacket. Das zweite tut mir aber
               noch viel mehr leid. Ich hab auf der Bahn durch Schlamperei des Trägers (\uuline{\textsc{N\textsuperscript{o}} 1}) mein von Ihnen bewundertes dunkles Schirmfutteral mit einem {\pb}ſchönen Schirm von \textcolor{brown}{Rodeck}{}\ledrightnote{\textcolor{brown}{Gebrüder Rodeck}} und grauem Naturſtock vergeſſen. Bitte
               vielmals gehen Sie zum \textcolor{blue}{Stationschef}{}\ledrightnote{→\textcolor{blue}{Ferdinand Miliczek}} und Sie werdens gewiſs beko{\geminationm}en.
               Bitte vielmals ſchicken Sie mir dann das Packet (das ist das wenigſt mühſame für Sie)
                  {\pb}in die große \textcolor{pink}{\textsc{Gassner-Villa}}{}\ledrightnote{\textcolor{pink}{Villa Gassner}} mit der Weiſung, Gehört Hofmannsthal, ſoll liegen bleiben.\pend
           \pstart Nicht bös ſein. Ihr \spacefill\mbox{Hugo.}\pend{}\endnumbering\briefempfaengerindex{Schnitzler, Arthur@\textsc{Schnitzler, Arthur}!zzzHofmannsthal, Hugo von@\emph{von Hugo von Hofmannsthal}!1899-09-081@{{[}8. 9. 1899{]}}|)be}\mylabel{h}  \normalsize

\doendnotes{C}
\bigskip
\vfill

\clearpage

\footnotesize

\lohead{\textsc{register}}

% Definiere theindex-Environment komplett neu ohne reledmac
\makeatletter
\renewenvironment{theindex}{%
  \section*{\indexname}%
  \setlength{\parindent}{0pt}%
  \setlength{\parskip}{0pt plus 0.3pt}%
  \let\item\@idxitem
}{%
  \clearpage
}
\makeatother

\IfFileExists{\jobname-pw.ind}{\input{\jobname-pw.ind}}{}

\end{document}

      