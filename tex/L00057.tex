%% latex-korrekturansicht-vorspann.tex
%% Vorspann für die Korrekturansicht.
%% Lädt die gemeinsame Datei latex-vorspann.tex mit gesetztem Schalter.

\newif\ifkorrekturansicht
\korrekturansichttrue

\input{../tex-inputs/latex-vorspann}


               \section[Arthur Schnitzler an Richard Beer-Hofmann, {[}zwischen 1892 und Mitte 1893?{]}]{ Arthur Schnitzler an Richard Beer-Hofmann, {[}zwischen 1892 und Mitte
               1893?{]}}\nopagebreak\mylabel{v}\rehead{ }\normalsize\beginnumbering\briefempfaengerindex{Beer-Hofmann, Richard@\textsc{Beer-Hofmann, Richard}!zzzSchnitzler, Arthur@\emph{von Arthur Schnitzler}!1892-01-011@{{[}zwischen 1892 und Mitte
                  1893?{]}}|(be} \toendnotes[C]{\smallbreak\pagebreak[2]} \Standort{YCGL, MSS 31.}
\physDesc{Briefkarte
\newline{}Handschrift: Bleistift, deutsche Kurrent}\toendnotes[C]{\smallbreak}\pstart
           \noindent{}{\pb}Lieber Richard; \label{K_L00057_1v}\edtext{\textcolor{blue}{\textsc{Loris}}{}\ledrightnote{\textcolor{blue}{Hugo von Hofmannsthal}}}{\lemma{\textnormal{\emph{Loris}}}\Cendnote{\textnormal{Dies ist der einzige Hinweis, der
                  erlaubt, das undatierte Korrespondenzstück zumindest irgendwie zeitlich
                  einzugrenzen, da \textcolor{blue}{Hofmannsthal} das Pseudonym
                  nur bis Mitte 1893 regelmäßig verwendete, danach aber auch \textcolor{blue}{Schnitzler} zunehmend dazu überging, den
                  Vornamen zu verwenden. Der erhaltene Briefwechsel \textcolor{blue}{Hofmannsthal}/\textcolor{blue}{Beer-Hofmann} legt nahe, dass erst 1892 ein vertraulicher
                  Umgang zwischen den beiden aufkam, der Mittagessen beim anderen zu Hause
                  involvierte.}}}\label{K_L00057_1h} ſpeiſt nicht bei Ihnen – wir treffen uns alle um \uline{12 Uhr Mittags} im \textcolor{pink}{\textsc{Griensteidl}}{}\ledrightnote{\textcolor{pink}{Café Griensteidl}}; alle {\pb}ſind verſtändigt.\pend
           \pstart
           Herzlichſt Ihr{\\[\baselineskip]}\spacefill\mbox{Arthur}\pend
           \leftskip=0em{}\endnumbering\briefempfaengerindex{Beer-Hofmann, Richard@\textsc{Beer-Hofmann, Richard}!zzzSchnitzler, Arthur@\emph{von Arthur Schnitzler}!1892-01-011@{{[}zwischen 1892 und Mitte
                  1893?{]}}|)be}\mylabel{h}  \normalsize

\doendnotes{C}
\bigskip
\vfill

\clearpage

\footnotesize

\lohead{\textsc{register}}

% Definiere theindex-Environment komplett neu ohne reledmac
\makeatletter
\renewenvironment{theindex}{%
  \section*{\indexname}%
  \setlength{\parindent}{0pt}%
  \setlength{\parskip}{0pt plus 0.3pt}%
  \let\item\@idxitem
}{%
  \clearpage
}
\makeatother

\IfFileExists{\jobname-pw.ind}{\input{\jobname-pw.ind}}{}

\end{document}

      