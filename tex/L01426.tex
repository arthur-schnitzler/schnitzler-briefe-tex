%% latex-korrekturansicht-vorspann.tex
%% Vorspann für die Korrekturansicht.
%% Lädt die gemeinsame Datei latex-vorspann.tex mit gesetztem Schalter.

\newif\ifkorrekturansicht
\korrekturansichttrue

\input{../tex-inputs/latex-vorspann}


               \section[Hugo von Hofmannsthal an Arthur Schnitzler, 12. 8. 1904]{ Hugo von Hofmannsthal an Arthur Schnitzler, 12. 8. 1904}\nopagebreak\mylabel{v}\rehead{ }\normalsize\beginnumbering\briefempfaengerindex{Schnitzler, Arthur@\textsc{Schnitzler, Arthur}!zzzHofmannsthal, Hugo von@\emph{von Hugo von Hofmannsthal}!1904-08-121@{12. 8. 1904}|(be} \toendnotes[C]{\smallbreak\pagebreak[2]} \Standort{CUL, Schnitzler, B 43.}
\physDesc{Brief, 2 Blätter, 8 Seiten
\newline{}Handschrift: schwarze Tinte, deutsche Kurrent
\newline{}Schnitzler: mit Bleistift die Jahreszahl ergänzt: »904« \newline{}Ordnung: 1) mit Bleistift von unbekannter Hand nummeriert: »\strikeout{254}« 2) mit Bleistift von unbekannter Hand nummeriert: »232.1« bzw. »232.2«}\buchAbdrucke{\weitereDrucke{Hugo von Hofmannsthal, Arthur Schnitzler: \emph{Briefwechsel}. Hg. Therese Nickl und Heinrich Schnitzler. Frankfurt am Main: \emph{S. Fischer} 1964, S. 196.} }\toendnotes[C]{\smallbreak}\pstart
           \raggedleft{}{\pb}\textcolor{pink}{Markt Auſſee, Ramgut}{}\ledrightnote{\textcolor{pink}{Ramgut}}{\\}12 VIII.\pend
           \pstart{}Lieber,\pend\pstart
           Ich ging gegen Abend vom \textcolor{pink}{Markt}{}\ledrightnote{\textcolor{pink}{Bad Aussee}} herauf, begegnete
               drei Frauen deren Geſichter ich nicht ſehen konnte. Hinter mir ſagte eine davon, ihr
               Gespräch fortſetzend: »und dann ſind wir mit ihnen auseinandergekommen, das war zu
               der Zeit wie ſie mit dem Arthur Schnitzler verlobt war«{\dots}
               und die andere ſagte beſtätigend: »ja, zu der Zeit war ſie mit {\pb}dem Arthur Schnitzler verlobt«.
               Von wem kann da die Rede geweſen ſein? Vielleicht von der ewigen \textcolor{blue}{Minnie}{}\ledrightnote{\textcolor{blue}{Hermine von Schaffgotsch}}?\pend
           \pstart
           \numberlinefalse{}\centering{}–\numberlinetrue{}\pend
           \pstart
           \noindent{}Eine Stunde ſpäter ſoupierte ich mit Leuten: da hörte ich mir gegenüber einen \strikeout{zus} zu ſeinem Nachbar ſagen, auf engliſch: »und dann
               hat mir der Manager geſagt, wenn Schnitzler fortfährt, {\pb}ſolche Sachen zu machen, wird man
               ihn als einen litterariſchen Pariah behandeln (wörtlich.)« Das intereſſierte mich
               doch ſehr und ich habe nach Tisch den Betreffenden angeredet: es iſt der \textsc{attaché} bei der \textcolor{pink}{engliſchen
                  Botſchaft in Paris}{}\ledrightnote{\textcolor{pink}{Botschaft von Großbritannien in Paris}}{ }\textsc{Mr.
                     \textcolor{blue}{van Sittard}{}\ledrightnote{\textcolor{blue}{Robert Gilbert Vansittart}}}, ein ungewöhnlicher junger
               Menſch, ganz jung, 23, ein Spieler, ſehr elegant, hat die beſte Prüfung gemacht, die
               in {\pb}der engliſchen Diplomatie ſeit
               vielen Jahren vorgekommen iſt, war \textsc{head-boy} von \textcolor{brown}{\textsc{Eton}}{}\ledrightnote{\textcolor{brown}{Eton College}}, ſchreibt auf
               franzöſiſch Theaterſtücke und hat was das netteſte iſt, eine unglaublich intenſive
               Liebe für Ihre Sachen. Er findet ſie weit beſſer als alles was auf allen engliſchen
               und franzöſiſchen Theatern zuſa{\geminationm}en aufgeführt wird,
               worin er ja Recht haben dürfte.\hspace*{1.5em}Als ich ihn beſuchte
               (er {\pb}iſt bis 23\textsuperscript{ten}{ }\textcolor{pink}{Altauſſee, \textsc{Villa
                     Franckenſtein}}{}\ledrightnote{\textcolor{pink}{Villa Franckenstein}}) lag auf dem Tisch \textcolor{green}{Vermächtnis}{}\ledrightnote{\textcolor{green}{Das Vermächtnis. Schauspiel in drei Akten}}, \textcolor{green}{Beatrice}{}\ledrightnote{\textcolor{green}{Der Schleier der Beatrice. Schauspiel in fünf Akten}}, \textcolor{green}{Sterben}{}\ledrightnote{\textcolor{green}{Sterben. Novelle}}. Diese 3 waren das einzige was er nicht kannte und
               nachzuholen hatte. Er ſagt alſo: es geſchieht ihm nun ſchon das zweitemal das er ganz
               auf dem Punkt iſt, ſeine von Ihnen autoriſierte Überſetzung von 3-4 \textcolor{green}{Anatol}{}\ledrightnote{\textcolor{green}{Anatol}}ſachen auf eine {\pb}gute Bühne zu bringen und daſs im
               letzten Moment Einſpruch erhoben wird von Leuten, denen Sie \uline{auch} die Autoriſation erteilt haben. Sonderbarerweiſe kam \uline{während} ich mit ihm redete ein Brief, in dem abermals ein
                  \textcolor{blue}{Regiſſeur}{}\ledrightnote{→\textcolor{blue}{Christopher St. John}}{ }ſchreibt: »wenn \textsc{Mr.
                  Schnitzler} fortfährt, ſich ſo \uline{außerordentlich}
               zu benehmen, wird niemand in \textcolor{pink}{England}{}\ledrightnote{\textcolor{pink}{England}} mehr etwas von
               ihm wiſſen {\pb}wollen.« Was liegt da
               vor? ich kenne Ihre ungewöhnliche Exactheit und habe \textcolor{blue}{\textsc{van Sittard}}{}\ledrightnote{\textcolor{blue}{Robert Gilbert Vansittart}} verſichert, es muſs da ein
               Schwindel vorliegen. Bitte klären Sie ſogleich ihn oder mich auf, damit er
               nöthigenfalls durch einen Proceſs da Klarheit ſchafft und ſeinen ſo ſchönen und
               ziemlich ungewöhnlichen Eifer nicht verliert. Es iſt ein recht intereſſanter
               Mensch.\pend
           \pstart
           \centering{}{\pb}–\pend
           \pstart
           \noindent{}Ich bin alſo von der Waffenübung befreit, d.h. ſie iſt auf den November
               verſchoben, wo ſie mich nicht ſehr geniert.\hspace*{1.5em}So
               treffen wir uns hoffentlich. Wo? \textcolor{pink}{Iſchl}{}\ledrightnote{\textcolor{pink}{Bad Ischl}}, ich meine
               der Fleck \textcolor{pink}{Iſchl}{}\ledrightnote{\textcolor{pink}{Bad Ischl}}{ }ſelbſt, wird mir vielleicht dadurch unmöglich, daſs
               meine \textcolor{blue}{Schwiegermutter}{}\ledrightnote{→\textcolor{blue}{Franziska Schlesinger}} hingeht.
               Da käme ich eventuell an den \textcolor{pink}{Wolfgangſee}{}\ledrightnote{\textcolor{pink}{Wolfgangsee}},
               jedenfalls rechne ich auf Zuſa{\geminationm}enſein, d.h. für den Fall
               daſs Sie die \textcolor{blue}{Mutter}{}\ledrightnote{→\textcolor{blue}{Louise Schnitzler}}{ }\uline{nicht} mithaben.\pend
           \pstart
           Von Herzen Ihr{\\[\baselineskip]}\spacefill\mbox{Hugo}\pend
           \leftskip=0em{}\endnumbering\briefempfaengerindex{Schnitzler, Arthur@\textsc{Schnitzler, Arthur}!zzzHofmannsthal, Hugo von@\emph{von Hugo von Hofmannsthal}!1904-08-121@{12. 8. 1904}|)be}\mylabel{h}  \normalsize

\doendnotes{C}
\bigskip
\vfill

\clearpage

\footnotesize

\lohead{\textsc{register}}

% Definiere theindex-Environment komplett neu ohne reledmac
\makeatletter
\renewenvironment{theindex}{%
  \section*{\indexname}%
  \setlength{\parindent}{0pt}%
  \setlength{\parskip}{0pt plus 0.3pt}%
  \let\item\@idxitem
}{%
  \clearpage
}
\makeatother

\IfFileExists{\jobname-pw.ind}{\input{\jobname-pw.ind}}{}

\end{document}

      