%% latex-korrekturansicht-vorspann.tex
%% Vorspann für die Korrekturansicht.
%% Lädt die gemeinsame Datei latex-vorspann.tex mit gesetztem Schalter.

\newif\ifkorrekturansicht
\korrekturansichttrue

\input{../tex-inputs/latex-vorspann}


               \section[Hugo von Hofmannsthal an Arthur Schnitzler, 21. 8. 1898]{ Hugo von Hofmannsthal an Arthur Schnitzler, 21. 8. 1898}\nopagebreak\mylabel{v}\rehead{ }\normalsize\beginnumbering\briefempfaengerindex{Schnitzler, Arthur@\textsc{Schnitzler, Arthur}!zzzHofmannsthal, Hugo von@\emph{von Hugo von Hofmannsthal}!1898-08-211@{21. 8. 1898}|(be} \toendnotes[C]{\smallbreak\pagebreak[2]} \Standort{CUL, Schnitzler, B 43.}
\physDesc{Postkarte
\newline{}Handschrift: schwarze Tinte, deutsche Kurrent\newline{}Versand: 1) Stempel: »\nobreak{}\oindex{Lugano@\textbf{Lugano}, \emph{Besiedelter Ort (A.BSO)}|pwk}Lugano, 21. VIII. 98, 1\nobreak{}«.  2) Stempel: »\nobreak{}\oindex{Luzern@\textbf{Luzern}, \emph{Besiedelter Ort (A.BSO)}|pwk}Luzern Brf. Dist., 21. VIII. 98, 8\nobreak{}«. 
\newline{}Schnitzler: mit Bleistift datiert: »21/8 98« \newline{}Ordnung: 1) mit Bleistift von unbekannter Hand nummeriert »\strikeout{126}« 2) mit Bleistift von unbekannter Hand nummeriert
                                    »120«}\buchAbdrucke{\weitereDrucke{Hugo von Hofmannsthal, Arthur Schnitzler: \emph{Briefwechsel}. Hg. Therese Nickl und Heinrich Schnitzler. Frankfurt am Main: \emph{S. Fischer} 1964, S. 110.} }\toendnotes[C]{\smallbreak}\pstart{}{\pb}\textsc{Herrn D\textsuperscript{r} Arthur Schnitzler}\pend{}\pstart{}\textcolor{pink}{\textsc{Lucerne}}{}\ledrightnote{\textcolor{pink}{Luzern}}\pend{}\pstart{}\textsc{poste rest.}\pend{}\pstart{}\textcolor{pink}{\textsc{Suisse}}{}\ledrightnote{\textcolor{pink}{Schweiz}}\pend{}{\bigskip}\pstart
           {\pb}\textcolor{pink}{Lugano, du Parc}{}\ledrightnote{\textcolor{pink}{Hôtel du Parc}}, Sonntag{ }Früh.
               \pend
           \pstart
           Bin über \textcolor{pink}{\textsc{Zermatt}}{}\ledrightnote{\textcolor{pink}{Zermatt}} und \textcolor{pink}{\textsc{Simplon}}{}\ledrightnote{\textcolor{pink}{Simplon}} gut angeko{\geminationm}en, \textcolor{pink}{wohne}{}\ledrightnote{→\textcolor{pink}{Hôtel du Parc}}{ }ſchön und angenehm. Hoffe ſehr auf Nachricht von
               Ihnen und bitte vielmals um Recepiſſe der Taſche, das bis jetzt nicht in meinen
               Händen.\pend
           \pstart Ihr \spacefill\mbox{Hugo.}\pend{}\endnumbering\briefempfaengerindex{Schnitzler, Arthur@\textsc{Schnitzler, Arthur}!zzzHofmannsthal, Hugo von@\emph{von Hugo von Hofmannsthal}!1898-08-211@{21. 8. 1898}|)be}\mylabel{h}  \normalsize

\doendnotes{C}
\bigskip
\vfill

\clearpage

\footnotesize

\lohead{\textsc{register}}

% Definiere theindex-Environment komplett neu ohne reledmac
\makeatletter
\renewenvironment{theindex}{%
  \section*{\indexname}%
  \setlength{\parindent}{0pt}%
  \setlength{\parskip}{0pt plus 0.3pt}%
  \let\item\@idxitem
}{%
  \clearpage
}
\makeatother

\IfFileExists{\jobname-pw.ind}{\input{\jobname-pw.ind}}{}

\end{document}

      