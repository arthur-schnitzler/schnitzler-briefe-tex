%% latex-korrekturansicht-vorspann.tex
%% Vorspann für die Korrekturansicht.
%% Lädt die gemeinsame Datei latex-vorspann.tex mit gesetztem Schalter.

\newif\ifkorrekturansicht
\korrekturansichttrue

\input{../tex-inputs/latex-vorspann}


               \section[Paul Goldmann an Arthur Schnitzler, 17. 2. {[}1894{]}]{ Paul Goldmann an Arthur Schnitzler, 17. 2. {[}1894{]}}\nopagebreak\mylabel{v}\rehead{ }\normalsize\beginnumbering\briefempfaengerindex{Schnitzler, Arthur@\textsc{Schnitzler, Arthur}!zzzGoldmann, Paul@\emph{von Paul Goldmann}!1894-02-172@{17. 2. {[}1894{]}}|(be} \toendnotes[C]{\smallbreak\pagebreak[2]} \Standort{DLA, A:Schnitzler, HS.NZ85.1.3164.}
\physDesc{Brief, 1 Blatt, 4 Seiten
\newline{}Handschrift: schwarze Tinte, deutsche Kurrent
\newline{}Schnitzler: 1) mit Bleistift auf dem ersten Blatt die Jahreszahl »94« vermerkt 2) mit rotem Buntstift zwei Unterstreichungen}\toendnotes[C]{\smallbreak}\pstart
           \raggedleft{}{\pb}\textsc{\textcolor{pink}{Paris}{}\ledrightnote{\textcolor{pink}{Paris}}}, 17. Februar.\pend
           \pstart\center{}Mein lieber Freund,\pend\pstart
           Es iſt nur der Zeitmangel. Ich denke oft an Dich. Stelle Dir ſehr oft vor und es iſt
               doch noch mehr. Spreche auch viel von Dir. Aber ſchreiben? Unmöglich. Und was auch?
               Was ich thue, ſiehst Du aus der \textcolor{green}{Zeitung}{}\ledrightnote{→\textcolor{green}{Frankfurter Zeitung}}, wo Du meine Arbeiten mit einer Treue verfolgſt, die mich rührt.
               Nebenher keinen Strich. \textsc{\label{K_L02609-12v}\edtext{Improductivitas absoluta}{\lemma{\textnormal{\emph{Improductivitas absoluta}}}\Cendnote{\textnormal{lateinisch: völlige Unproduktivität}}}\label{K_L02609-12h}}. Schädel
               leer, Herz leer. Verkommene Exiſtenz. Scheußlicher bürgerlicher Zuſtand, ſeeliſcher
               desgleichen. {\pb}Das iſt immer dieſelbe Geſchichte. Was
               willſt Du alſo von mir hören? Mir iſt lieber, ich höre von Dir. Das iſt doch
               wenigſtens eine Freude.\pend
           \pstart
           Und doch ein kleiner Lichtblick. Einen \textcolor{blue}{Menſchen}{}\ledrightnote{→\textcolor{blue}{Henri Albert}} gefunden, den Erſten ſeit \textcolor{pink}{Wien}{}\ledrightnote{\textcolor{pink}{Wien}}. Heißt \textsc{\textcolor{blue}{Henri Albert}{}\ledrightnote{\textcolor{blue}{Henri Albert}}}, Mitte zwanzig. Dasjenige, was wir ſeinerzeit impertinent genug waren, eine
               Wir-Natur zu nennen. Noch mehr: ich glaube beinahe, daß er ein viertes Exemplar iſt
               von der \textsc{Species}{ }\textsc{Arthur} – \textsc{\textcolor{blue}{Richard}{}\ledrightnote{\textcolor{blue}{Richard Beer-Hofmann}}} – \textsc{\textcolor{blue}{Loris}{}\ledrightnote{\textcolor{blue}{Hugo von Hofmannsthal}}}. Noch weiß ichs nicht genau; denn ich habe die Aufrichtigkeit-Diagnoſe noch
               nicht ſtellen können. Alles {\pb}Übrige ſcheint zu
               ſtimmen. Und, oh Wunder, er kennt \textcolor{blue}{Euch}{}\ledrightnote{→\textcolor{blue}{Richard Beer-Hofmann}{\newline}→\textcolor{blue}{Hugo von Hofmannsthal}} Alle, hat von \textcolor{blue}{Allen}{}\ledrightnote{→\textcolor{blue}{Richard Beer-Hofmann}{\newline}→\textcolor{blue}{Hugo von Hofmannsthal}} geleſen. Nun kennt er \textcolor{blue}{Euch}{}\ledrightnote{→\textcolor{blue}{Richard Beer-Hofmann}{\newline}→\textcolor{blue}{Hugo von Hofmannsthal}} natürlich erſt
               recht. Ich habe ihn – auf Widerruf – zum auswärtigen Mitglied unſeres Kreiſes
               ernannt, weil ich ihn lieb gewonnen und dies \strikeout{das} der
               höchſte Orden iſt, das Goldene Vließ, das ich zu vergeben habe. Wenn das keine
               Enttäuſchung iſt – in \textsc{\textcolor{pink}{Paris}{}\ledrightnote{\textcolor{pink}{Paris}}} haben die Naturen ſolche Untiefen! – ſo iſts ein wahrer Fund geweſen. Er
               correſpondirt von \textcolor{pink}{hier}{}\ledrightnote{→\textcolor{pink}{Paris}} für die
               »\textcolor{brown}{Freie Bühne}{}\ledrightnote{\textcolor{brown}{Freie Bühne}}«, ſchreibt außerdem viel in den
               jungen franzöſiſchen Revüen. Als \textcolor{blue}{Elſäſſer}{}\ledrightnote{→\textcolor{blue}{Henri Albert}} ſpricht und ſchreibt er deutſch wie franzöſiſch. {\pb}Ich bin hinter ihm her, daß er mir \label{K_L02609-2v}\edtext{über \textcolor{blue}{Euch}{}\ledrightnote{→\textcolor{blue}{Richard Beer-Hofmann}{\newline}→\textcolor{blue}{Hugo von Hofmannsthal}} einen Artikel in den »\textsc{\textcolor{green}{Mercure de France}{}\ledrightnote{\textcolor{green}{Mercure de France}}}« oder die »\textsc{\textcolor{green}{Société Nouvelle}{}\ledrightnote{\textcolor{green}{La Société Nouvelle. Revue internationale. Sociologie, Arts, Sciences, Lettres}}}«}{\lemma{\textnormal{\emph{über … Nouvelle«}}}\Cendnote{\textnormal{Bereits wenig später erschien die
                  \textcolor{green}{Rezension} des \emph{\textcolor{green}{Modernen Musenalmanach auf das Jahr 1894}} im
                  \emph{\textcolor{green}{Mercure de France}}, in der die Beiträge \textcolor{blue}{Schnitzler} und \textcolor{blue}{Hofmannsthal} hervorgehoben wurden: \textcolor{blue}{Henri Albert}: \emph{\textcolor{green}{Le nouvel almanach de M. Bierbaum}}. In: \emph{\textcolor{green}{Mercure de France}}, Jg. 10, Nr. 51,
                     März 1894, S. 233–246, hier: S. 244–245.}}}\label{K_L02609-2h} macht,
               daß er \textcolor{green}{etwas}{}\ledrightnote{→\textcolor{green}{Weihnachts-Einkäufe}} von Dir \label{K_L02609-1v}\edtext{überſetzt}{\lemma{\textnormal{\emph{überſetzt}}}\Cendnote{\textnormal{\textcolor{blue}{Arthur Schnitzler}: \emph{\textcolor{green}{Les Emplettes de Noël}}. Übersetzung \textcolor{blue}{Henri Albert}. In: \emph{\textcolor{green}{L’Idée libre. Revue mensuelle de Littérature et d'Art}}, Jg. 3,
                     Nr. 5–6, Mai–Juni 1894, S. 215–225.}}}\label{K_L02609-1h}{ }\textsc{etc.} Hoffen wir!\pend
           \pstart
           Wann kommt endlich Einer von \textcolor{blue}{Euch}{}\ledrightnote{→\textcolor{blue}{Richard Beer-Hofmann}{\newline}→\textcolor{blue}{Hugo von Hofmannsthal}} her?\pend
           \pstart
           Deine Zukunfts-Zuverſicht betreffend Deine Production für dieſes
               Jahr hat mich unendlich erfreut. Aber was? Und wie gehts Dir ſonſt?
               Perſönliches, perſönliches, mein theurer Freund!\pend
           \pstart
           Über \textsc{\textcolor{blue}{Niemann}{}\ledrightnote{\textcolor{blue}{August Niemann}}} bin ich ganz anderer Anſicht. Mich hat das \textcolor{green}{Ding}{}\ledrightnote{→\textcolor{green}{Der Junggesell. Humoreske}} hoch entzückt gerade wegen ſeiner
               Abſichtsloſigkeit, gerade, weil ich in ihm ein einfaches, humorvolles, \strikeout{,}{ }zierliches Kunſtwerk
               gefunden, von der Höhe des intellectuellen Standpunktes abgeſehen. Wer von uns hat da
               Recht? Und \textsc{\textcolor{blue}{Duerer}{}\ledrightnote{\textcolor{blue}{Emil Dürer}}}? Schreib’ mir über \textsc{\textcolor{blue}{Duerer}{}\ledrightnote{\textcolor{blue}{Emil Dürer}}}! Herzlichſt und in Treue Dein \spacefill\mbox{Paul Goldmann}\pend
           \pstart
           \noindent{}{\pb}\label{T_L02609-1v}\edtext{viele herzliche Grüße an die \textcolor{blue}{Freunde}{}\ledrightnote{→\textcolor{blue}{Richard Beer-Hofmann}{\newline}→\textcolor{blue}{Hugo von Hofmannsthal}}. Schreib
                  mir bald einen \uline{langen} Brief}{\lemma{\textnormal{\emph{viele … Brief}}}\Cendnote{\textnormal{am oberen Rand auf der ersten Seite}}}\label{T_L02609-1h}\pend
           \endnumbering\briefempfaengerindex{Schnitzler, Arthur@\textsc{Schnitzler, Arthur}!zzzGoldmann, Paul@\emph{von Paul Goldmann}!1894-02-172@{17. 2. {[}1894{]}}|)be}\mylabel{h}  \normalsize

\doendnotes{C}
\bigskip
\vfill

\clearpage

\footnotesize

\lohead{\textsc{register}}

% Definiere theindex-Environment komplett neu ohne reledmac
\makeatletter
\renewenvironment{theindex}{%
  \section*{\indexname}%
  \setlength{\parindent}{0pt}%
  \setlength{\parskip}{0pt plus 0.3pt}%
  \let\item\@idxitem
}{%
  \clearpage
}
\makeatother

\IfFileExists{\jobname-pw.ind}{\input{\jobname-pw.ind}}{}

\end{document}

      