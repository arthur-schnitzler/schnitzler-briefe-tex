%% latex-korrekturansicht-vorspann.tex
%% Vorspann für die Korrekturansicht.
%% Lädt die gemeinsame Datei latex-vorspann.tex mit gesetztem Schalter.

\newif\ifkorrekturansicht
\korrekturansichttrue

\input{../tex-inputs/latex-vorspann}


               \section[Hugo von Hofmannsthal an Arthur Schnitzler, 6. 5. {[}1900{]}]{ Hugo von Hofmannsthal an Arthur Schnitzler, 6. 5. {[}1900{]}}\nopagebreak\mylabel{v}\rehead{ }\normalsize\beginnumbering\briefempfaengerindex{Schnitzler, Arthur@\textsc{Schnitzler, Arthur}!zzzHofmannsthal, Hugo von@\emph{von Hugo von Hofmannsthal}!1900-05-062@{6. 5. {[}1900{]}}|(be} \toendnotes[C]{\smallbreak\pagebreak[2]} \Standort{CUL, Schnitzler, B 43.}
\physDesc{Brief, 1 Blatt, 4 Seiten
\newline{}Handschrift: schwarze Tinte, deutsche Kurrent
\newline{}Schnitzler: mit Bleistift die Jahreszahl ergänzt: »900« \newline{}Ordnung: mit Bleistift von unbekannter Hand nummeriert:
                              »161« }\buchAbdrucke{\weitereDrucke{Hugo von Hofmannsthal, Arthur Schnitzler: \emph{Briefwechsel}. Hg. Therese Nickl und Heinrich Schnitzler. Frankfurt am Main: \emph{S. Fischer} 1964, S. 138–139.} }\toendnotes[C]{\smallbreak}\pstart
           \raggedleft{}{\pb}\textcolor{pink}{\textsc{Brighton}}{}\ledrightnote{\textcolor{pink}{Brighton}},
                     6 V.\pend
           \pstart{}mein lieber Arthur\pend\pstart
           ich war ſehr froh darüber daſs Sie in der Zeit von \textcolor{blue}{Papa}{}\ledrightnote{→\textcolor{blue}{Hugo August von Hofmannsthal}}s Krankheit meine \textcolor{blue}{Eltern}{}\ledrightnote{→\textcolor{blue}{Anna von Hofmannsthal}{\newline}→\textcolor{blue}{Hugo August von Hofmannsthal}} oft beſucht und \introOben{}mir\introOben{}{ }ſo gut und beruhigend darüber geſchrieben
               haben.\pend
           \pstart
           Ein Zufall hat mich veranlaſst, für kurze Zeit hierher zu gehen und ſo werde ich auch
               noch mit einer etwas traumhaften {\pb}Flüchtigkeit \textcolor{pink}{London}{}\ledrightnote{\textcolor{pink}{London}}{ }ſehen.\pend
           \pstart
           Wenn ich auch nicht gar ſo viel Fertiges mitbringe, ſo dafür um ſo mehr angefangenes
               und entworfenes.\pend
           \pstart
           Hier iſt mir nach einer langen Zeit zuerſt die \textcolor{brown}{N. Fr.
                  Preſſe}{}\ledrightnote{\textcolor{brown}{Neue Freie Presse}} wieder in die Hände gekommen. Das ſtrömt eine kleinliche, ordinäre,
               herabgekommene Atmoſphäre {\pb}aus, in
               welcher man \uline{niemals wirklich} zu leben trachten
               muſs.\pend
           \pstart
           Warum \textcolor{green}{ſchreibt}{}\ledrightnote{→\textcolor{green}{Berliner Theater. (»Der König von Rom.«)}} ein anſtändiger
               Menſch wie \textcolor{blue}{Goldmann}{}\ledrightnote{\textcolor{blue}{Paul Goldmann}} 6 Spalten voll mit
               Nichts, dieſes Nichts in dem unbeſchreiblich widerwärtigen witzelnden jüdiſchen Ton,
               der nirgends auf der Welt exiſtiert als im Feuilleton \textcolor{pink}{deutſcher}{}\ledrightnote{\textcolor{pink}{Deutschland}} u. \textcolor{pink}{oeſterr.}{}\ledrightnote{\textcolor{pink}{Österreich}} Zeitungen? \pend
           \pstart
           {\pb}Ungefähr den 18\textsuperscript{ten} werde ich in \textcolor{pink}{Wien}{}\ledrightnote{\textcolor{pink}{Wien}}{ }ſein und freue mich ſehr auf Sie und \textcolor{blue}{Richard}{}\ledrightnote{\textcolor{blue}{Richard Beer-Hofmann}}, auf den Frühling in \textcolor{pink}{Niederöſterreich}{}\ledrightnote{\textcolor{pink}{Niederösterreich}} und aufs Radfahren.\pend
           \pstart Von Herzen Ihr \spacefill\mbox{Hugo.}\pend{}\endnumbering\briefempfaengerindex{Schnitzler, Arthur@\textsc{Schnitzler, Arthur}!zzzHofmannsthal, Hugo von@\emph{von Hugo von Hofmannsthal}!1900-05-062@{6. 5. {[}1900{]}}|)be}\mylabel{h}  \normalsize

\doendnotes{C}
\bigskip
\vfill

\clearpage

\footnotesize

\lohead{\textsc{register}}

% Definiere theindex-Environment komplett neu ohne reledmac
\makeatletter
\renewenvironment{theindex}{%
  \section*{\indexname}%
  \setlength{\parindent}{0pt}%
  \setlength{\parskip}{0pt plus 0.3pt}%
  \let\item\@idxitem
}{%
  \clearpage
}
\makeatother

\IfFileExists{\jobname-pw.ind}{\input{\jobname-pw.ind}}{}

\end{document}

      