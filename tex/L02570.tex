%% latex-korrekturansicht-vorspann.tex
%% Vorspann für die Korrekturansicht.
%% Lädt die gemeinsame Datei latex-vorspann.tex mit gesetztem Schalter.

\newif\ifkorrekturansicht
\korrekturansichttrue

\input{../tex-inputs/latex-vorspann}


               \section[Therese Rie-Andro an Arthur Schnitzler, 29. 11. 1912]{ Therese Rie-Andro an Arthur Schnitzler, 29. 11. 1912}\nopagebreak\mylabel{v}\rehead{ }\normalsize\beginnumbering\briefempfaengerindex{Schnitzler, Arthur@\textsc{Schnitzler, Arthur}!zzzRie, Therese@\emph{von Therese Rie}!1912-11-293@{29. 11. 1912}|(be} \toendnotes[C]{\smallbreak\pagebreak[2]} \Standort{DLA, A:Schnitzler, 85.1.4310.}
\physDesc{Brief, 1 Blatt, 3 Seiten
\newline{}Handschrift: blaue Tinte, lateinische Kurrent
\newline{}Schnitzler: 1) mit Bleistift beschriftet: »\textsc{Andro}« 2) mit rotem Buntstift mehrere Unterstreichungen}\toendnotes[C]{\smallbreak}\pstart
           \raggedleft{}{\pb}\textcolor{pink}{Wien}{}\ledrightnote{\textcolor{pink}{Wien}}, d. 29. Nov. 12.\pend
           \pstart{}Sehr geehrter Herr,\pend\pstart
           Sie haben mir vor einigen Monaten einen Brief geschrieben, der mich sehr sehr erfreut
               hat; dennoch würde ich Ihnen gewiß nicht schreiben, wenn ich nicht unter einem
               ungeheuer starken künstlerischen Eindruck stünde: es ist der »\textcolor{green}{Professor Bernhardi}{}\ledrightnote{\textcolor{green}{Professor Bernhardi. Komödie in fünf Akten}}«, den ich (durch dessen \label{K_L02570-1v}\edtext{Vorlesung}{\lemma{\textnormal{\emph{Vorlesung}}}\Cendnote{\textnormal{In \textcolor{pink}{Wien} wurde am
                     28. 11. 1912 – dem Tag der \textcolor{pink}{Berlin}er Uraufführung – eine Lesung durch \textcolor{blue}{Ferdinand Onno} veranstaltet. Gelesen wurden der 1. Akt, das Gespräch
                  von Bernhardi und Flint im 2. Akt, der 3. Akt, das Gespräch von
                  Bernhardi und dem Pfarrer im 4. Akt und der 5. Akt. Um die Lücken zu
                  überbrücken, schrieb \textcolor{blue}{Schnitzler} kurze
                  Verbindungstexte, die im Nachlass als Durchschläge erhalten sind
                        (\emph{Cambridge}, A 117,2, freundliche
                  Auskunft von Judith Beniston).}}}\label{K_L02570-1h}) kennen lernte. Sie werden ja jetzt soviel
               Schönes drüber hören und lesen, daß ich es wol kaum wagen kann, Ihnen etwas zu sagen;
               ich versuch’s auch gar nicht erſt. Aber diese in milder Heiterkeit sich lösende
               Tragödie des aufrechten Menschen, dieser wunderbar in \textcolor{blue}{Goethe}{}\ledrightnote{\textcolor{blue}{Johann Wolfgang von Goethe}}’sche Sti{\geminationm}ung ausklingende Schluß: »\textcolor{green}{Selig wer sich vor der Welt \uline{ohne Haß verschließt}}{}\ledrightnote{→\textcolor{green}{An den Mond}}« – {\pb}die gehen mir selbst in diesen trüben
               ahnungsschweren Kriegszeiten i{\geminationm}er noch nach.\pend
           \pstart
           Aber noch anderes war es mir und mehr: die Erläuterung längst entschwundener
               Kindheitserlebnisse, das Emportauchen von damals kaum begriffenen und doch erfaßten
               Dingen. Mein \textcolor{blue}{Vater}{}\ledrightnote{→\textcolor{blue}{Maximilian Herz}} war
               Abteilungsvorstand an der \textcolor{brown}{Poliklinik}{}\ledrightnote{\textcolor{brown}{Allgemeine Poliklinik}}, als Ihr \textcolor{blue}{Vater}{}\ledrightnote{→\textcolor{blue}{Johann Schnitzler}} (den ich gekannt und
               geliebt habe) Direktor war. Oft iſt er heiß und erregt nach Hause geko{\geminationm}en, hat vor mir, dem kleinen Kinde, auf das niemand
               achtete, gesprochen. Es war ein Kampf, den die rechtlichen Leute alle dort führten,
               vornehmlich gegen \label{K_L02570-2v}\edtext{\textcolor{blue}{Einen}{}\ledrightnote{→\textcolor{blue}{Julius von Hochenegg}}}{\lemma{\textnormal{\emph{Einen}}}\Cendnote{\textnormal{Sie könnte \textcolor{blue}{Julius Hochenegg} meinen, den \textcolor{blue}{Schnitzler} als Vorlage für die Figur des \textcolor{green}{Professor Ebenwald} verwendet hat (vgl. A. S.: \emph{Tagebuch}, 23. 10. 1922).}}}\label{K_L02570-2h} führten,
               der, glaube ich, \introOben{}leider\introOben{} Vize-Direktor war. Ich weiß, daß Ihr
                  \textcolor{green}{Stück}{}\ledrightnote{→\textcolor{green}{Professor Bernhardi. Komödie in fünf Akten}} nicht an Geschehnisse
               anknüpft, aber an innere Erlebnisse, an Sti{\geminationm}ungen, die
               damals in der Luft gelegen haben müßen und ich kann Ihnen nicht beschreiben, wie es
               mich durchschauert hat, als ich diese Atmosphäre emportauchen fühlte, in der mein \textcolor{blue}{Vater}{}\ledrightnote{→\textcolor{blue}{Maximilian Herz}} (er starb
                  1890, als ich noch ein Kind war) gelebt hat, mitgekämpft und
               mitgelitten hat. Obgleich er in Ihrem \textcolor{green}{Stück}{}\ledrightnote{→\textcolor{green}{Professor Bernhardi. Komödie in fünf Akten}}{ }\strikeout{sicherlich} nicht »vorko{\geminationm}t« (um den banalen Ausdruck der Leute zu gebrauchen, die dem dichterischen Schaffen
               ganz ferne stehen) war es mir einen Augenblick, als wäre mir etwas von ihm, an dem
               ich mit meiner ganzen Kinderleidenschaft hing, zurückgekehrt: so sehr hat Ihr \textcolor{green}{Stück}{}\ledrightnote{→\textcolor{green}{Professor Bernhardi. Komödie in fünf Akten}} das Schicksal des Arztes ins
               Typische erhöht, stilisiert. Und darum müßen Sie begreifen, wie sehr ergriffen ich
               von Ihrem \textcolor{green}{Stück}{}\ledrightnote{→\textcolor{green}{Professor Bernhardi. Komödie in fünf Akten}} war, wie ich {\pb}es mit der ganz tiefen Dankbarkeit in mich aufgeno{\geminationm}en habe, als sei mir ein unbekanntes Stück meines
               eigenen Lebens gedeutet worden. Und deshalb sind Sie mir, verehrter Herr Doctor, auch
               nicht böse, wenn ich – ungeru\textcolor{gray}{f}en, und still wieder gehend – ko{\geminationm}e, um Ihnen das zu sagen!\pend
           \pstart \spacefill\mbox{L. Andro.}\pend{}\endnumbering\briefempfaengerindex{Schnitzler, Arthur@\textsc{Schnitzler, Arthur}!zzzRie, Therese@\emph{von Therese Rie}!1912-11-293@{29. 11. 1912}|)be}\mylabel{h}  \normalsize

\doendnotes{C}
\bigskip
\vfill

\clearpage

\footnotesize

\lohead{\textsc{register}}

% Definiere theindex-Environment komplett neu ohne reledmac
\makeatletter
\renewenvironment{theindex}{%
  \section*{\indexname}%
  \setlength{\parindent}{0pt}%
  \setlength{\parskip}{0pt plus 0.3pt}%
  \let\item\@idxitem
}{%
  \clearpage
}
\makeatother

\IfFileExists{\jobname-pw.ind}{\input{\jobname-pw.ind}}{}

\end{document}

      