%% latex-korrekturansicht-vorspann.tex
%% Vorspann für die Korrekturansicht.
%% Lädt die gemeinsame Datei latex-vorspann.tex mit gesetztem Schalter.

\newif\ifkorrekturansicht
\korrekturansichttrue

\input{../tex-inputs/latex-vorspann}


               \section[Hermann Bahr an Arthur Schnitzler, {[}15. 3. 1903{]}]{ Hermann Bahr an Arthur Schnitzler, {[}15. 3. 1903{]}}\nopagebreak\mylabel{v}\rehead{ }\normalsize\beginnumbering\briefempfaengerindex{Schnitzler, Arthur@\textsc{Schnitzler, Arthur}!zzzBahr, Hermann@\emph{von Hermann Bahr}!1903-03-152@{{[}15. 3. 1903{]}}|(be} \toendnotes[C]{\smallbreak\pagebreak[2]} \Standort{CUL, Schnitzler, B 5b.}
\physDesc{Brief, 1 Blatt, 2 Seiten
\newline{}Handschrift: schwarze Tinte, deutsche Kurrent
\newline{}Schnitzler: mit Bleistift datiert: »15/3 903« \newline{}Ordnung: mit Bleistift von unbekannter Hand nummeriert:
                              »95« }\buchAbdrucke{\weitereDrucke{Hermann Bahr, Arthur Schnitzler: \emph{Briefwechsel, Aufzeichnungen, Dokumente (1891–1931)}. Hg. Kurt Ifkovits und Martin Anton Müller. Göttingen: \emph{Wallstein} 2018, S. 254.} }\pstart
           \raggedleft{}{\pb}Sonntag\pend
           \pstart\center{}Lieber Arthur!\pend\pstart
           Ich gratuliere Dir herzlichſt zu dem, wie ich von Herrn \textcolor{blue}{Epſtein}{}\ledrightnote{\textcolor{blue}{Moritz Epstein}} erfahre, außergewöhnlich ſtarken Erfolge der »\textcolor{green}{L. St.}{}\ledrightnote{\textcolor{green}{Lebendige Stunden. Vier Einakter}}«, der mich nicht blos um Deinetwillen,
               ſondern auch deswegen ſo freut, weil die Gelehrten des \textcolor{pink}{Deutſchen Volkstheaters}{}\ledrightnote{\textcolor{pink}{Volkstheater}} wieder einmal ſo zu Schanden geworden ſind.\pend
           \pstart
           Mir gehts heute wieder gut, nur habe ich nach den Erfahrungen der letzten Wochen
               ſchon gar nicht mehr recht den Mut zu hoffen, daß ich noch einmal wirklich gesund {\pb}werden ſollte.\pend
           \pstart
           Herzlichſt{\\[\baselineskip]}Dein{\\[\baselineskip]}\spacefill\mbox{Hermann}\pend
           \leftskip=0em{}\endnumbering\briefempfaengerindex{Schnitzler, Arthur@\textsc{Schnitzler, Arthur}!zzzBahr, Hermann@\emph{von Hermann Bahr}!1903-03-152@{{[}15. 3. 1903{]}}|)be}\mylabel{h}  \normalsize

\doendnotes{C}
\bigskip
\vfill

\clearpage

\footnotesize

\lohead{\textsc{register}}

% Definiere theindex-Environment komplett neu ohne reledmac
\makeatletter
\renewenvironment{theindex}{%
  \section*{\indexname}%
  \setlength{\parindent}{0pt}%
  \setlength{\parskip}{0pt plus 0.3pt}%
  \let\item\@idxitem
}{%
  \clearpage
}
\makeatother

\IfFileExists{\jobname-pw.ind}{\input{\jobname-pw.ind}}{}

\end{document}

      