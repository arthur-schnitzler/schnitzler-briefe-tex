%% latex-korrekturansicht-vorspann.tex
%% Vorspann für die Korrekturansicht.
%% Lädt die gemeinsame Datei latex-vorspann.tex mit gesetztem Schalter.

\newif\ifkorrekturansicht
\korrekturansichttrue

\input{../tex-inputs/latex-vorspann}


               \section[Arthur Schnitzler: Widmungsexemplar Der Schleier der Beatrice für Richard Dehmel, {[}1. 1. 1902?{]}]{ Arthur Schnitzler: Widmungsexemplar Der Schleier der Beatrice für
               Richard Dehmel, {[}1. 1. 1902?{]}}\nopagebreak\mylabel{v}\rehead{ }\normalsize\beginnumbering\briefempfaengerindex{Dehmel, Richard@\textsc{Dehmel, Richard}!zzzSchnitzler, Arthur@\emph{von Arthur Schnitzler}!1902-01-012@{{[}1. 1. 1902?{]}}|(be} \toendnotes[C]{\smallbreak\pagebreak[2]} \Standort{Hamburg, Staats- und Universitätsbibliothek, Dehmel, NL DA Bib:478.}
\physDesc{Widmung am Vorsatzblatt
\newline{}Handschrift: schwarze Tinte, deutsche Kurrent
\newline{}Dehmel: an der inneren vorderen Umschlagseite Exlibris }\toendnotes[C]{\smallbreak}\pstart
           \noindent{}{\pb}Richard Dehmel{\\}mit verehrungsvollem \label{K_L02550-1v}\edtext{Neujahrsgruſs}{\lemma{\textnormal{\emph{Neujahrsgruſs}}}\Cendnote{\textnormal{Da die Antwort auf die Zusendung ebenfalls am 1. 1. 1902 verfasst ist, ist auch
                  eine frühere Übermittlung des Buches denkbar.}}}\label{K_L02550-1h}{ }1902\pend
           \pstart \spacefill\mbox{ArthurSchnitzler}\pend{}{\bigskip}\pstart
           \noindent{}\centering{}{\pb}\textcolor{gray}{\textbf{\textcolor{green}{Der Schleier der Beatrice}{}\ledrightnote{\textcolor{green}{Der Schleier der Beatrice. Schauspiel in fünf Akten}}}}\pend
           \pstart
           \noindent{}\centering{}\textcolor{gray}{\textbf{Schauſpiel in fünf Akten}}\pend
           \pstart
           \noindent{}\centering{}\textcolor{gray}{\textbf{von}}\pend
           \pstart
           \noindent{}\centering{}\textcolor{gray}{\textbf{\textbf{Arthur Schnitzler}}}\pend
           \pstart
           \noindent{}\centering{}\textcolor{gray}{\textbf{Zweite Auflage}}\pend
           {\bigskip}\pstart
           \noindent{}\centering{}\textcolor{gray}{\textbf{\textcolor{pink}{\textbf{Berlin}}{}\ledrightnote{\textcolor{pink}{Berlin}}}}\pend
           \pstart
           \noindent{}\centering{}\textcolor{gray}{\textbf{\textcolor{brown}{S. Fiſcher, Verlag}{}\ledrightnote{\textcolor{brown}{S. Fischer Verlag}}}}\pend
           \pstart
           \noindent{}\centering{}\textcolor{gray}{\textbf{1901}}\pend
           \endnumbering\briefempfaengerindex{Dehmel, Richard@\textsc{Dehmel, Richard}!zzzSchnitzler, Arthur@\emph{von Arthur Schnitzler}!1902-01-012@{{[}1. 1. 1902?{]}}|)be}\mylabel{h}  \normalsize

\doendnotes{C}
\bigskip
\vfill

\clearpage

\footnotesize

\lohead{\textsc{register}}

% Definiere theindex-Environment komplett neu ohne reledmac
\makeatletter
\renewenvironment{theindex}{%
  \section*{\indexname}%
  \setlength{\parindent}{0pt}%
  \setlength{\parskip}{0pt plus 0.3pt}%
  \let\item\@idxitem
}{%
  \clearpage
}
\makeatother

\IfFileExists{\jobname-pw.ind}{\input{\jobname-pw.ind}}{}

\end{document}

      