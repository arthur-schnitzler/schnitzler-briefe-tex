%% latex-korrekturansicht-vorspann.tex
%% Vorspann für die Korrekturansicht.
%% Lädt die gemeinsame Datei latex-vorspann.tex mit gesetztem Schalter.

\newif\ifkorrekturansicht
\korrekturansichttrue

\input{../tex-inputs/latex-vorspann}


               \section[Arthur Schnitzler an Richard Beer-Hofmann, 24. 8. 1892]{ Arthur Schnitzler an Richard Beer-Hofmann, 24. 8. 1892}\nopagebreak\mylabel{v}\rehead{ }\normalsize\beginnumbering\briefempfaengerindex{Beer-Hofmann, Richard@\textsc{Beer-Hofmann, Richard}!zzzSchnitzler, Arthur@\emph{von Arthur Schnitzler}!1892-08-241@{24. 8. 1892}|(be} \toendnotes[C]{\smallbreak\pagebreak[2]} \Standort{YCGL, MSS 31.}
\physDesc{Brief, 1 Blatt, 3 Seiten, Umschlag
\newline{}Handschrift: Bleistift, deutsche Kurrent\newline{}Versand: 1) Stempel: »\nobreak{}{\pb}Wien
                                          \textcolor{gray}{4}/1, 24 8 92, 7–8N\nobreak{}«.  2) Stempel: »\nobreak{}\oindex{Bad Ischl@\textbf{Bad Ischl}, \emph{Besiedelter Ort (A.BSO)}|pwk}Ischl, 25 8 9{[}2{]}, 10 F\nobreak{}«. }\buchAbdrucke{\weitereDrucke{Arthur Schnitzler, Richard Beer-Hofmann: \emph{Briefwechsel 1891–1931}. Hg. Konstanze Fliedl. Wien, Zürich: \emph{Europaverlag} 1992, S. 38.} }\pstart{}{\pb}Herrn Dr \textsc{Richard Beer
                     Hofmann}\pend{}\pstart{}\textsc{\textcolor{pink}{Ischl}{}\ledrightnote{\textcolor{pink}{Bad Ischl}}.}\pend{}\pstart{}\textsc{\textcolor{pink}{Grazerstraße 6}{}\ledrightnote{\textcolor{pink}{Grazer Straße}}}.\pend{}{\bigskip}\pstart{}{\pb}Lieber Richard,\pend\pstart
           ich theile Ihnen mit, dß ich Samſtag in \textcolor{pink}{Iſchl}{}\ledrightnote{\textcolor{pink}{Bad Ischl}} eintreffen werde; wo ich wohne, iſt noch nicht beſtimmt – \textcolor{pink}{\textsc{Leopold}}{}\ledrightnote{\textcolor{pink}{Hotel und Pension Rudolfshöhe (Leopold Petter)}} wahrſcheinlich – möglich \textcolor{pink}{\textsc{Elisabeth}}{}\ledrightnote{\textcolor{pink}{Hotel Kaiserin Elisabeth}}. –\pend
           \pstart
           {\pb}Viele herzliche Grüße bis dahin! – \pend
           \pstart
           Meine Abſicht iſt es, Touren zu machen; jawohl, lachen Sie nicht; ich brauche
               nothwendig phyſiſche Bewegung, vielleicht ſogar Abmattung, um mich aus einer {\pb}unerträglichen Dumpfheit des Seeliſchen zu retten.\pend
           \pstart
           Ich freue mich auf Sie, ich hoffe ſogar auf Sie.\pend
           \pstart Ihr \spacefill\mbox{Arthur}\pend{}\pstart
           24. 8. 92{ }\textcolor{pink}{Wien}{}\ledrightnote{\textcolor{pink}{Wien}}\pend
           \endnumbering\briefempfaengerindex{Beer-Hofmann, Richard@\textsc{Beer-Hofmann, Richard}!zzzSchnitzler, Arthur@\emph{von Arthur Schnitzler}!1892-08-241@{24. 8. 1892}|)be}\mylabel{h}  \normalsize

\doendnotes{C}
\bigskip
\vfill

\clearpage

\footnotesize

\lohead{\textsc{register}}

% Definiere theindex-Environment komplett neu ohne reledmac
\makeatletter
\renewenvironment{theindex}{%
  \section*{\indexname}%
  \setlength{\parindent}{0pt}%
  \setlength{\parskip}{0pt plus 0.3pt}%
  \let\item\@idxitem
}{%
  \clearpage
}
\makeatother

\IfFileExists{\jobname-pw.ind}{\input{\jobname-pw.ind}}{}

\end{document}

      