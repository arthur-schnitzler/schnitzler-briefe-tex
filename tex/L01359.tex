%% latex-korrekturansicht-vorspann.tex
%% Vorspann für die Korrekturansicht.
%% Lädt die gemeinsame Datei latex-vorspann.tex mit gesetztem Schalter.

\newif\ifkorrekturansicht
\korrekturansichttrue

\input{../tex-inputs/latex-vorspann}


               \section[Arthur Schnitzler an Hermann Bahr, 8. 1. 1904]{ Arthur Schnitzler an Hermann Bahr, 8. 1. 1904}\nopagebreak\mylabel{v}\rehead{ }\normalsize\beginnumbering\briefempfaengerindex{Bahr, Hermann@\textsc{Bahr, Hermann}!zzzSchnitzler, Arthur@\emph{von Arthur Schnitzler}!1904-01-081@{8. 1. 1904}|(be} \toendnotes[C]{\smallbreak\pagebreak[2]} \Standort{TMW, HS AM 23363 Ba.}
\physDesc{Brief, 1 Blatt, 1 Seite
\newline{}Schreibmaschine
\newline{}Handschrift: schwarze Tinte, deutsche Kurrent (\noindent{}Schlussformel, Unterschrift und Einfügung von
                                 »ev.«)\newline{}Ordnung: Lochung }\buchAbdrucke{\weitereDrucke{1) \emph{8. 1. 1904.} In: Arthur Schnitzler: \emph{The Letters of Arthur Schnitzler to Hermann Bahr}. Edited, annotated, and with an introduction, by Donald G.
                        Daviau. Chapel Hill: \emph{The University of North Carolina Press} 1978, S. 83 (University of North Carolina studies in the Germanic languages
                        and literatures, 89).} \weitereDrucke{2) Hermann Bahr, Arthur Schnitzler: \emph{Briefwechsel, Aufzeichnungen, Dokumente (1891–1931)}. Hg. Kurt Ifkovits und Martin Anton Müller. Göttingen: \emph{Wallstein} 2018, S. 288.} }\toendnotes[C]{\smallbreak}\pstart
           \raggedleft{}{\pb}\textcolor{pink}{Wien}{}\ledrightnote{\textcolor{pink}{Wien}}, 8. Januar 1904.\pend
           \pstart
           \raggedleft{}\textcolor{pink}{XVIII. Spöttelg. 7.}{}\ledrightnote{\textcolor{pink}{Edmund-Weiß-Gasse}}\pend
           \pstart{}Lieber Hermann!\pend\pstart
           Die Adresse des Dr. \textcolor{blue}{Stephan Epstein}{}\ledrightnote{\textcolor{blue}{Stephan Epstein}} ist: \textcolor{pink}{Paris, 78, Rue de l’Assomption}{}\ledrightnote{\textcolor{pink}{rue de l’Assomption}}. Er hat dir wol auch
               über das \introOben{}ev.\introOben{}{ }\label{K_L01359_1v}\edtext{Gastspiel}{\lemma{\textnormal{\emph{Gastspiel}}}\Cendnote{\textnormal{1904 trat \textcolor{blue}{Antoine} nicht in \textcolor{pink}{Wien} auf.}}}\label{K_L01359_1h}{ }\textcolor{blue}{Antoine}{}\ledrightnote{\textcolor{blue}{André Antoine}} geschrieben. Seine \textcolor{blue}{Frau}{}\ledrightnote{→\textcolor{blue}{Henriette Estelle Epstein}}, die \label{K_L01359_2v}\edtext{neulich}{\lemma{\textnormal{\emph{neulich}}}\Cendnote{\textnormal{siehe A. S.: \emph{Tagebuch}, 28. 12. 1903}}}\label{K_L01359_2h} in \textcolor{pink}{Wien}{}\ledrightnote{\textcolor{pink}{Wien}} war, fragte mich, auf welche Weise es möglich
               wäre, die \textcolor{pink}{Sezession}{}\ledrightnote{\textcolor{pink}{Secession}} zu veranlassen, einen in \textcolor{pink}{Paris}{}\ledrightnote{\textcolor{pink}{Paris}} lebenden Künstler, \textcolor{blue}{Bernhard Hoetger}{}\ledrightnote{\textcolor{blue}{Bernhard Hoetger}}, zu einer Ausstellung seiner Werke einzuladen.
               Sie schickt Dir nächstens irgend ein französisches Journal, in welchem \textcolor{blue}{Hoetgerische}{}\ledrightnote{\textcolor{blue}{Bernhard Hoetger}} Arbeiten abgebildet sind.\pend
           \pstart
           Morgen fahre ich auf einige Tage auf den \textcolor{pink}{Semmering}{}\ledrightnote{\textcolor{pink}{Semmering}},
               komme gleich, wenn ich zurück bin, mit deiner freundlichen Erlaubnis zu dir, und
               hoffe, dich wohl zu finden.\pend
           \pstart
           {[}hs.:{]} Herzliche Grüße, auch von meiner \textcolor{blue}{Frau}{}\ledrightnote{→\textcolor{blue}{Olga Schnitzler}}{\\[\baselineskip]}dein \spacefill\mbox{Arthu\damage{r}}\pend
           \leftskip=0em{}\endnumbering\briefempfaengerindex{Bahr, Hermann@\textsc{Bahr, Hermann}!zzzSchnitzler, Arthur@\emph{von Arthur Schnitzler}!1904-01-081@{8. 1. 1904}|)be}\mylabel{h}  \normalsize

\doendnotes{C}
\bigskip
\vfill

\clearpage

\footnotesize

\lohead{\textsc{register}}

% Definiere theindex-Environment komplett neu ohne reledmac
\makeatletter
\renewenvironment{theindex}{%
  \section*{\indexname}%
  \setlength{\parindent}{0pt}%
  \setlength{\parskip}{0pt plus 0.3pt}%
  \let\item\@idxitem
}{%
  \clearpage
}
\makeatother

\IfFileExists{\jobname-pw.ind}{\input{\jobname-pw.ind}}{}

\end{document}

      