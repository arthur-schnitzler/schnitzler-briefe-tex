%% latex-korrekturansicht-vorspann.tex
%% Vorspann für die Korrekturansicht.
%% Lädt die gemeinsame Datei latex-vorspann.tex mit gesetztem Schalter.

\newif\ifkorrekturansicht
\korrekturansichttrue

\input{../tex-inputs/latex-vorspann}


               \section[Hugo von Hofmannsthal an Arthur Schnitzler, {[}9. 3. 1904{]}]{ Hugo von Hofmannsthal an Arthur Schnitzler, {[}9. 3. 1904{]}}\nopagebreak\mylabel{v}\rehead{ }\normalsize\beginnumbering\briefempfaengerindex{Schnitzler, Arthur@\textsc{Schnitzler, Arthur}!zzzHofmannsthal, Hugo von@\emph{von Hugo von Hofmannsthal}!1904-03-092@{{[}9. 3. 1904{]}}|(be} \toendnotes[C]{\smallbreak\pagebreak[2]} \Standort{CUL, Schnitzler, B 43.}
\physDesc{Brief, 2 Blätter, 6 Seiten
\newline{}Handschrift: schwarze Tinte, deutsche Kurrent
\newline{}Schnitzler: mit Bleistift datiert: »9/3 904.« \newline{}Ordnung: 1) mit Bleistift von unbekannter Hand nummeriert: »\strikeout{293}« 2) mit Bleistift von unbekannter Hand nummeriert: »216.1« bzw. »216.2«}\buchAbdrucke{\weitereDrucke{Hugo von Hofmannsthal, Arthur Schnitzler: \emph{Briefwechsel}. Hg. Therese Nickl und Heinrich Schnitzler. Frankfurt am Main: \emph{S. Fischer} 1964, S. 183.} }\toendnotes[C]{\smallbreak}\pstart
           \raggedleft{}{\pb}Mittwoch\pend
           \pstart{}mein lieber Arthur\pend\pstart
           das Befinden meiner armen \textcolor{blue}{Mutter}{}\ledrightnote{→\textcolor{blue}{Anna von Hofmannsthal}} hat einen Punkt erreicht wo – ohne daſs vielleicht eine acute Gefahr
               vorliegt, wenigſtens weiß ich darüber nichts beſti{\geminationm}tes –
               die Combination von eingeſtellten Functionen der Gedärme, von unaufhörlichen
               Schmerzen und von einer kaum glaublichen Nerven{\pb}ſchwäche die zu fortwährenden
               Üblichkeiten führt – 12–15mal Brechanfälle im Tag – die Exiſtenz buchſtäblich \uline{unerträglich} macht, nicht nur für ſie, ſondern auch
               für meinen armen \textcolor{blue}{Papa}{}\ledrightnote{→\textcolor{blue}{Hugo August von Hofmannsthal}}, den \textcolor{blue}{Mama}{}\ledrightnote{→\textcolor{blue}{Anna von Hofmannsthal}}s verzweifelte nervöſe Angſt
               buchſtäblich nicht aus dem Zimmer läſst, mit Ausnahme der Bureauſtunden.\pend
           \pstart
           Ich ſage mir jetzt: es muſs {\pb}etwas
               geſchehen, es iſt nicht möglich, ſo das Leben von 2 alternden Menſchen hinzufriſten,
               mit gelegentlichen Beſuchen von Ärzten, und täglichem Besuch eines \textcolor{blue}{Hausarztes}{}\ledrightnote{→\textcolor{blue}{Hans Schandlbauer}}, der am Rand der
               Verzweiflung über das alles iſt.\pend
           \pstart
           Nun denke ich, daſs Sie vielleicht von Ihrem \textcolor{blue}{Bruder}{}\ledrightnote{→\textcolor{blue}{Julius Schnitzler}} zum Teil über \textcolor{blue}{Mama}{}\ledrightnote{→\textcolor{blue}{Anna von Hofmannsthal}} orientiert ſind, wenn aber auch nicht, bitte {\pb}beſuchen Sie mit mir einmal meine
                  \textcolor{blue}{Mutter}{}\ledrightnote{→\textcolor{blue}{Anna von Hofmannsthal}} auf eine Stunde, ich
               meine es nicht im ärztlichen Sinn, ſondern mehr menſchlich, pſychiſch, ihr thut ſchon
               abſolut noth, daſs ein neuer Menſch – (ſie hat Sie ſehr gern) zu ihr ſympathiſch und
               aufmunternd ſpricht, vielleicht können Sie ihr etwas rathen, nicht ſpeciell, ſondern
               allgemein ihr furchtbares Nervenbefinden betreffend.\pend
           \pstart
           Nicht wahr, Sie thun mir {\pb}das
               zulieb?\pend
           \pstart
           Sie machen alles lieber an Vormittagen ab, alſo wollen Sie Samstag gegen
                     11\textsuperscript{h} oder 11½ in die \textcolor{pink}{Saleſianergaſſe}{}\ledrightnote{\textcolor{pink}{Salesianergasse}}
                  ko{\geminationm}en?\pend
           \pstart
           Ich würde Sie dort erwarten. Nur wenn Sie \uline{nicht}
               können und lieber Sonntag oder Montag wählen, brauchen Sie
               mir zu antworten, dann {\pb}aber
               telegraphiſch, bitte.\pend
           \pstart
           Von Herzen Ihr{\\[\baselineskip]}\spacefill\mbox{Hugo}\pend
           \leftskip=0em{}\endnumbering\briefempfaengerindex{Schnitzler, Arthur@\textsc{Schnitzler, Arthur}!zzzHofmannsthal, Hugo von@\emph{von Hugo von Hofmannsthal}!1904-03-092@{{[}9. 3. 1904{]}}|)be}\mylabel{h}  \normalsize

\doendnotes{C}
\bigskip
\vfill

\clearpage

\footnotesize

\lohead{\textsc{register}}

% Definiere theindex-Environment komplett neu ohne reledmac
\makeatletter
\renewenvironment{theindex}{%
  \section*{\indexname}%
  \setlength{\parindent}{0pt}%
  \setlength{\parskip}{0pt plus 0.3pt}%
  \let\item\@idxitem
}{%
  \clearpage
}
\makeatother

\IfFileExists{\jobname-pw.ind}{\input{\jobname-pw.ind}}{}

\end{document}

      