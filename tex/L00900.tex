%% latex-korrekturansicht-vorspann.tex
%% Vorspann für die Korrekturansicht.
%% Lädt die gemeinsame Datei latex-vorspann.tex mit gesetztem Schalter.

\newif\ifkorrekturansicht
\korrekturansichttrue

\input{../tex-inputs/latex-vorspann}


               \section[Arthur Schnitzler an Julius Rodenberg, 7. 3. 1899]{ Arthur Schnitzler an Julius Rodenberg, 7. 3. 1899 }\nopagebreak\mylabel{v}\rehead{ }\normalsize\beginnumbering\briefempfaengerindex{Rodenberg, Julius@\textsc{Rodenberg, Julius}!zzzSchnitzler, Arthur@\emph{von Arthur Schnitzler}!1899-03-072@{7. 3. 1899}|(be} \toendnotes[C]{\smallbreak\pagebreak[2]} \Standort{Weimar, Klassik Stiftung, 81/X,2,10.}
\physDesc{Brief, 1 Blatt, 2 Seiten
\newline{}Handschrift: schwarze Tinte, deutsche Kurrent}\pstart{}{\pb}Sehr geehrter Herr Doktor,\pend\pstart
           noch immer ko{\geminationm}e ich mit keiner Novelle; – ich habe
                    noch immer keine geſchrieben. Hingegen möchte ich Ihnen gern meinen in der \textcolor{pink}{Burg}{}\ledrightnote{\textcolor{pink}{Burgtheater}} aufgeführten Einakter »\textcolor{green}{Die Gefährtin}{}\ledrightnote{\textcolor{green}{Die Gefährtin. Schauspiel in einem Akt}}« für die »\textcolor{brown}{Deutſche Rundſchau}{}\ledrightnote{\textcolor{brown}{Deutsche Rundschau}}« überreichen, und bitte Sie mir freundlichſt zu
                    ſagen, erſtens, ob {\pb}Sie überhaupt dramatiſches
                    bringen, zweitens ob Sie einen Einakter von mir haben wollen, drittens \uline{wann}{ }Sie das kleine Stück bringen könnten, wenn Sie
                    es nehmen.\pend
           \pstart
           Ihr hochachtungsvoll ergebener{\\[\baselineskip]}\spacefill\mbox{ArthurSchnitzler}\pend
           \leftskip=0em{}\pstart
           \textcolor{pink}{Wien}{}\ledrightnote{\textcolor{pink}{Wien}}{ }7. 3. 99.\pend
           \endnumbering\briefempfaengerindex{Rodenberg, Julius@\textsc{Rodenberg, Julius}!zzzSchnitzler, Arthur@\emph{von Arthur Schnitzler}!1899-03-072@{7. 3. 1899}|)be}\mylabel{h}  \normalsize

\doendnotes{C}
\bigskip
\vfill

\clearpage

\footnotesize

\lohead{\textsc{register}}

% Definiere theindex-Environment komplett neu ohne reledmac
\makeatletter
\renewenvironment{theindex}{%
  \section*{\indexname}%
  \setlength{\parindent}{0pt}%
  \setlength{\parskip}{0pt plus 0.3pt}%
  \let\item\@idxitem
}{%
  \clearpage
}
\makeatother

\IfFileExists{\jobname-pw.ind}{\input{\jobname-pw.ind}}{}

\end{document}

      