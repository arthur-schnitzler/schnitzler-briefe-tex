%% latex-korrekturansicht-vorspann.tex
%% Vorspann für die Korrekturansicht.
%% Lädt die gemeinsame Datei latex-vorspann.tex mit gesetztem Schalter.

\newif\ifkorrekturansicht
\korrekturansichttrue

\input{../tex-inputs/latex-vorspann}


               \section[Arthur Schnitzler an Richard Beer-Hofmann, 17. 8. 1897]{ Arthur Schnitzler an Richard Beer-Hofmann, 17. 8. 1897}\nopagebreak\mylabel{v}\rehead{ }\normalsize\beginnumbering\briefempfaengerindex{Beer-Hofmann, Richard@\textsc{Beer-Hofmann, Richard}!zzzSchnitzler, Arthur@\emph{von Arthur Schnitzler}!1897-08-171@{17. 8. 1897}|(be} \toendnotes[C]{\smallbreak\pagebreak[2]} \Standort{YCGL, MSS 31.}
\physDesc{Brief, 1 Blatt, 1 Seite, Umschlag
\newline{}Handschrift: Bleistift, deutsche Kurrent\newline{}Versand: 1) Stempel: »\nobreak{}Wien, 17 8 97, 11–12 N\nobreak{}«.  2) Stempel: »\nobreak{}\oindex{Salzburg@\textbf{Salzburg}, \emph{Besiedelter Ort (A.BSO)}|pwk}Salzburg Stadt, 18 8 97, 11–F\nobreak{}«. 3) Stempel: »\nobreak{}\oindex{Salzburg@\textbf{Salzburg}, \emph{Besiedelter Ort (A.BSO)}|pwk}Salzburg Stadt, 18 8 97, 1–F\nobreak{}«. 4) Stempel: »\nobreak{}\oindex{Bad Ischl@\textbf{Bad Ischl}, \emph{Besiedelter Ort (A.BSO)}|pwk}Ischl, 18. 8. 97, 7–8 N\nobreak{}«. 5) die drei Adresszeilen
                                    durchgestrichen und darunter von unbekannter Hand mit Bleistift: »\noindent{}\textcolor{pink}{Ischl, Eglmoos 22}.«}\pstart{}{\pb}Herrn Dr. \textsc{Richard Beer-Hofmann}\pend{}\pstart{}\textcolor{pink}{\textsc{Salzburg}}{}\ledrightnote{\textcolor{pink}{Salzburg}}\pend{}\pstart{}\textsc{\textcolor{pink}{Hotel Oesterreichischer Hof}{}\ledrightnote{\textcolor{pink}{Österreichischer Hof}}.}\pend{}{\bigskip}\pstart
           \raggedleft{}{\pb}Dinſtag\pend
           \pstart
           Lieber Richard. Do{\geminationn}erſtag{ }Abend
                    oder Freitg{ }früh bin ich in \textcolor{pink}{Iſchl}{}\ledrightnote{\textcolor{pink}{Bad Ischl}}. Das Zi{\geminationm}er für \textcolor{blue}{Paul}{}\ledrightnote{\textcolor{blue}{Paul Goldmann}}
                    bei \textcolor{blue}{Petter}{}\ledrightnote{\textcolor{blue}{Leopold Petter}} beſtellt. Trifft Sie dieſer Brief
                    überhaupt noch in \textcolor{pink}{Salzburg}{}\ledrightnote{\textcolor{pink}{Salzburg}}? –\pend
           \pstart
           Grüßen Sie \textcolor{blue}{Paul}{}\ledrightnote{\textcolor{blue}{Paul Goldmann}} herzlich; auch ſich
                        ſelbſt.\pend
           \pstart Ihr \spacefill\mbox{Arthur}\pend{}\endnumbering\briefempfaengerindex{Beer-Hofmann, Richard@\textsc{Beer-Hofmann, Richard}!zzzSchnitzler, Arthur@\emph{von Arthur Schnitzler}!1897-08-171@{17. 8. 1897}|)be}\mylabel{h}  \normalsize

\doendnotes{C}
\bigskip
\vfill

\clearpage

\footnotesize

\lohead{\textsc{register}}

% Definiere theindex-Environment komplett neu ohne reledmac
\makeatletter
\renewenvironment{theindex}{%
  \section*{\indexname}%
  \setlength{\parindent}{0pt}%
  \setlength{\parskip}{0pt plus 0.3pt}%
  \let\item\@idxitem
}{%
  \clearpage
}
\makeatother

\IfFileExists{\jobname-pw.ind}{\input{\jobname-pw.ind}}{}

\end{document}

      