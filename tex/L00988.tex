%% latex-korrekturansicht-vorspann.tex
%% Vorspann für die Korrekturansicht.
%% Lädt die gemeinsame Datei latex-vorspann.tex mit gesetztem Schalter.

\newif\ifkorrekturansicht
\korrekturansichttrue

\input{../tex-inputs/latex-vorspann}


               \section[Richard Beer-Hofmann an Arthur Schnitzler, 3. 10. 1899]{ Richard Beer-Hofmann an Arthur Schnitzler, 3. 10. 1899}\nopagebreak\mylabel{v}\rehead{ }\normalsize\beginnumbering\briefempfaengerindex{Schnitzler, Arthur@\textsc{Schnitzler, Arthur}!zzzBeer-Hofmann, Richard@\emph{von Richard Beer-Hofmann}!1899-10-032@{3. 10. 1899}|(be} \toendnotes[C]{\smallbreak\pagebreak[2]} \Standort{CUL, Schnitzler, B 8.}
\physDesc{Brief, 1 Blatt, 2 Seiten
\newline{}Handschrift: schwarze Tinte, lateinische Kurrent\newline{}Ordnung: mit Bleistift von unbekannter Hand nummeriert: »143« }\buchAbdrucke{\weitereDrucke{Arthur Schnitzler, Richard Beer-Hofmann: \emph{Briefwechsel 1891–1931}. Hg. Konstanze Fliedl. Wien, Zürich: \emph{Europaverlag} 1992, S. 138–139.} }\toendnotes[C]{\smallbreak}\pstart
           \raggedleft{}{\pb}\textcolor{pink}{St. Michael in Eppan}{}\ledrightnote{\textcolor{pink}{Sankt Michael}}{ }3 X 1899\pend
           \pstart
           Lieber Arthur 1.) Von \textcolor{pink}{Vahrn}{}\ledrightnote{\textcolor{pink}{Vahrn}} bin ich
               fort weil es in dieser Höhe circa 670\textsuperscript{m} schon zu kühl
               ist.\pend
           \pstart
           2.) Dieses \textcolor{pink}{St. Michael}{}\ledrightnote{\textcolor{pink}{Sankt Michael}} liegt an der heuer
               eröffneten \textcolor{brown}{Überetscher Bahn}{}\ledrightnote{\textcolor{brown}{Überetscher Bahn}} – \textcolor{pink}{Bozen}{}\ledrightnote{\textcolor{pink}{Bozen}} – \textcolor{pink}{Kaltern}{}\ledrightnote{\textcolor{pink}{Caldaro sulla strada del vino}} –, nur eine
               Wagenstunde von \textcolor{pink}{Bozen}{}\ledrightnote{\textcolor{pink}{Bozen}}. Meistens ko{\geminationm}en hier nur die Leute die auf die \textcolor{pink}{Mendel}{}\ledrightnote{\textcolor{pink}{Mendelpass}} fahren durch; ständig wohnen hier wenig Fremde. In
               unserem »\textcolor{pink}{Hôtel}{}\ledrightnote{\textcolor{pink}{Eppaner Hof}}« außer uns Niemand. 3.) Auf die
               Idee hieherzuko{\geminationm}en hat mich ein Eisenbahnplakat
               gebracht. 4.) Ich dürfte nicht länger als 2 Wochen noch hierbleiben. \substVorne{}\textsuperscript{4}\substDazwischen{}5\substHinten{}.) Ich bin im \textcolor{green}{I Akt}{}\ledrightnote{→\textcolor{green}{Der Graf von Charolais. Ein Trauerspiel}} (der
               drei Abtheilungen hat) in der ersten Abtheilung im 5ten Versehundert. \textcolor{green}{433 Verse}{}\ledrightnote{→\textcolor{green}{Der Graf von Charolais. Ein Trauerspiel}} hats gebraucht bis ich den Helden
               auf die Bühne gelassen habe. \substVorne{}\textsuperscript{5}\substDazwischen{}6\substHinten{}.) Meine Laune wäre besser {\pb}wenn ich mehr schlafen würde. Im übrigen hängt sie von der Arbeit ab. Viele Verse –
               gute Laune; wenig Verse – schlechte Laune. O Gott! Was wird mir nicht Alles
               gestrichen werden. »Die Brillanten werden sie mer stehn lassen«! Antworten sie
               höflich: »Also Alles«!.\pend
           \pstart
            Ich grüße Sie herzlich{\\[\baselineskip]}Ihr \spacefill\mbox{Richard}\pend
           \leftskip=0em{}\pstart
           \noindent{}Grüßen Sie \textcolor{blue}{Brahm}{}\ledrightnote{\textcolor{blue}{Otto Brahm}} und Kerr. Dem \textcolor{blue}{Brahm}{}\ledrightnote{\textcolor{blue}{Otto Brahm}} bringen Sie um Gotteswillen keine bessere
                  Meinung von mir bei! Bis auf Weiteres laßen Sie mich für ihn »Ein Herr mit einem
                  Monocle« sein.\pend
           \endnumbering\briefempfaengerindex{Schnitzler, Arthur@\textsc{Schnitzler, Arthur}!zzzBeer-Hofmann, Richard@\emph{von Richard Beer-Hofmann}!1899-10-032@{3. 10. 1899}|)be}\mylabel{h}  \normalsize

\doendnotes{C}
\bigskip
\vfill

\clearpage

\footnotesize

\lohead{\textsc{register}}

% Definiere theindex-Environment komplett neu ohne reledmac
\makeatletter
\renewenvironment{theindex}{%
  \section*{\indexname}%
  \setlength{\parindent}{0pt}%
  \setlength{\parskip}{0pt plus 0.3pt}%
  \let\item\@idxitem
}{%
  \clearpage
}
\makeatother

\IfFileExists{\jobname-pw.ind}{\input{\jobname-pw.ind}}{}

\end{document}

      