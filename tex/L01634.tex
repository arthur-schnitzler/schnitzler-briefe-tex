%% latex-korrekturansicht-vorspann.tex
%% Vorspann für die Korrekturansicht.
%% Lädt die gemeinsame Datei latex-vorspann.tex mit gesetztem Schalter.

\newif\ifkorrekturansicht
\korrekturansichttrue

\input{../tex-inputs/latex-vorspann}


               \section[Arthur Schnitzler an Hermann Bahr, 18. 10. 1906]{ Arthur Schnitzler an Hermann Bahr, 18. 10. 1906 }\nopagebreak\mylabel{v}\rehead{ }\normalsize\beginnumbering\briefempfaengerindex{Bahr, Hermann@\textsc{Bahr, Hermann}!zzzSchnitzler, Arthur@\emph{von Arthur Schnitzler}!1906-10-181@{18. 10. 1906}|(be} \toendnotes[C]{\smallbreak\pagebreak[2]} \Standort{TMW, HS AM 23383 Ba.}
\physDesc{Brief, 1 Blatt, 2 Seiten
\newline{}Handschrift: schwarze Tinte, deutsche Kurrent\newline{}Ordnung: Lochung }\buchAbdrucke{\weitereDrucke{1) \emph{18. 10. 1906.} In: Arthur Schnitzler: \emph{The Letters of Arthur Schnitzler to Hermann Bahr}. Edited, annotated, and with an introduction, by Donald G.
                        Daviau. Chapel Hill: \emph{The University of North Carolina Press} 1978, S. 95–96 (University of North Carolina studies in the Germanic languages
                        and literatures, 89).} \weitereDrucke{2) Hermann Bahr, Arthur Schnitzler: \emph{Briefwechsel, Aufzeichnungen, Dokumente (1891–1931)}. Hg. Kurt Ifkovits und Martin Anton Müller. Göttingen: \emph{Wallstein} 2018, S. 383–384.} }\toendnotes[C]{\smallbreak}\pstart
           \raggedleft{}{\pb}\textcolor{pink}{Wien}{}\ledrightnote{\textcolor{pink}{Wien}}, 18. X. 906\pend
           \pstart{}lieber Hermann, \pend\pstart
           eine Aehnlichkeit zwiſchen \textcolor{green}{deinem
                  Akt}{}\ledrightnote{→\textcolor{green}{Die tiefe Natur. Ein Akt}} und dem \textcolor{green}{Abſchiedſouper}{}\ledrightnote{\textcolor{green}{Abschiedssouper}} wäre höchſtens
               irgendwo im äußerlich ſtofflichen zu finden, im innerlich stofflichen ſchon nicht
               mehr, und gewiſs nicht im \strikeout{eigentlich} »ſeeliſch
               geſtaltlichen« – \introOben{}(\introOben{}um zu i{\geminationm}er
               grauenhafteren Worten auf- oder niederzuſteigen). Dein Problem ist viel verzwickter,
               der Fortgang der Handlung gedrehter, ſpiraliger, jüdiſcher gegenüber der naiv \textcolor{pink}{\textsc{gauloisen}}{}\ledrightnote{\textcolor{pink}{Frankreich}} Fabel des braven alten \textcolor{green}{Anatolſtückls}{}\ledrightnote{→\textcolor{green}{Anatol}}, außerdem wird bei mir ſoupirt und bei dir doch eigentlich nur
                  \label{K_L01634_1v}\edtext{gejauſnet}{\lemma{\textnormal{\emph{gejauſnet}}}\Cendnote{\textnormal{österreichisch Jause: Zwischenmahlzeit}}}\label{K_L01634_1h}. Die Atmosphäre
               deines Stücks ist dünner, ſchärfer; das ganze brutaler (für {\pb}meinen Geſchmack im
               Beginn beſonders bis zum Abſtoßenden brutal) angepackt. Wenn du mir, oder dem guten
                  \textcolor{green}{Anatol}{}\ledrightnote{→\textcolor{green}{Anatol}}, dieſen \textcolor{green}{intereſſanten Einakter}{}\ledrightnote{→\textcolor{green}{Die tiefe Natur. Ein Akt}} widmen willſt, ſo nehm
               ich s natürlich mit Dank u Rührung an, nur mußt du mir erlauben, deine Erinnerung
               nicht als Anregungsqui\damage{tt}irung und Ausdruck einer Gewiſſensſchuld ſondern als ein neues und daher mir
                  willko{\geminationm}enes Zeichen unſerer guten Zuſa{\geminationm}engehörigkeit zu empfinden u zu empfangen.\pend
           \pstart
           Hoffentlich fügt es ſich, dſs wir einander vor deiner Abreiſe noch einmal ſehen.
               (Gern möcht ich auch etwas, \label{K_L01634_2v}\edtext{\textcolor{blue}{\textsc{Reinhardt}}{}\ledrightnote{\textcolor{blue}{Max Reinhardt}} betreffendes}{\lemma{\textnormal{\emph{Reinhardt betreffendes}}}\Cendnote{\textnormal{eine Aufführung von
                     \emph{\textcolor{green}{Der Schleier der Beatrice}}, vgl. A. S.: \emph{Tagebuch}, 29. 10. 1906 und vgl. den
                     Brief von \textcolor{blue}{Schnitzler} an \textcolor{blue}{Max Reinhardt}, 24. 12. 1909 in A. S. \emph{Briefe} I,613–621.}}}\label{K_L01634_2h}, aber hauptſächlich in
                  \damage{mei}nem Intereſſe liegendes\strikeout{)} mit dir
               beſprechen.)\pend
           \pstart
           Herzlichſt, mit Grüßen von{\\[\baselineskip]}meiner \textcolor{blue}{Frau}{}\ledrightnote{→\textcolor{blue}{Olga Schnitzler}} u mir{\\[\baselineskip]}dein{\\[\baselineskip]}\spacefill\mbox{Arthur}\pend
           \leftskip=0em{}\endnumbering\briefempfaengerindex{Bahr, Hermann@\textsc{Bahr, Hermann}!zzzSchnitzler, Arthur@\emph{von Arthur Schnitzler}!1906-10-181@{18. 10. 1906}|)be}\mylabel{h}  \normalsize

\doendnotes{C}
\bigskip
\vfill

\clearpage

\footnotesize

\lohead{\textsc{register}}

% Definiere theindex-Environment komplett neu ohne reledmac
\makeatletter
\renewenvironment{theindex}{%
  \section*{\indexname}%
  \setlength{\parindent}{0pt}%
  \setlength{\parskip}{0pt plus 0.3pt}%
  \let\item\@idxitem
}{%
  \clearpage
}
\makeatother

\IfFileExists{\jobname-pw.ind}{\input{\jobname-pw.ind}}{}

\end{document}

      