%% latex-korrekturansicht-vorspann.tex
%% Vorspann für die Korrekturansicht.
%% Lädt die gemeinsame Datei latex-vorspann.tex mit gesetztem Schalter.

\newif\ifkorrekturansicht
\korrekturansichttrue

\input{../tex-inputs/latex-vorspann}


               \section[Arthur Schnitzler: Widmungsexemplar Marionetten für Hugo von Hofmannsthal, {[}23.?{]} 3. 1906]{ Arthur Schnitzler: Widmungsexemplar Marionetten für Hugo von
                    Hofmannsthal, {[}23.?{]} 3. 1906}\nopagebreak\mylabel{v}\rehead{ }\normalsize\beginnumbering\briefempfaengerindex{Hofmannsthal, Hugo von@\textsc{Hofmannsthal, Hugo von}!zzzSchnitzler, Arthur@\emph{von Arthur Schnitzler}!1906-03-232@{{[}23.?{]} 3. 1906}|(be} \toendnotes[C]{\smallbreak\pagebreak[2]} \Standort{FDH, FDH 1936.}
\physDesc{Widmung am Vorsatzblatt
\newline{}Handschrift: schwarze Tinte, deutsche Kurrent
\newline{}Hofmannsthal: handschriftliche Notiz im Buchinneren: »\noindent{}Und wenn ich Sie vor mir stehen sehe, bereit dem ehrfurchtgebietenden Willen Ihres
                  Vaters zu trotzen mit wem, mit wem vergleiche ich Sie treffender als mit jenem \textcolor{blue}{Xerxes} der \introOben{}Éstultissima
                     furia jactantia in der Raserei der Selbstüberhebung\introOben{} sich anschickte die
                  Wogen des Hellespont zu peitschen und dem majestätischen Meeresgott Fesseln
                  anzulegen?{ / }ein weiblicher Bruder jenes \textcolor{blue}{Commodus} (beim
                     II\textsuperscript{ten} Mal){ / }Schluss der II\textsuperscript{ten} Scene \textcolor{green}{Jourdain – Lucile}{ / }L. Es gibt nichts was Sie erweichen könnte{ / }J Nein{ / }L. Nun denn (lächelt){ / }J. klopft sie auf die Backen.{ / }Menschen meiner Art u mein Ranges« }\buchAbdrucke{\weitereDrucke{Hugo von Hofmannsthal: \emph{Bibliothek}. Hg. Ellen Ritter † in Zusammenarbeit mit Dalia Bukauskaité und
                                Konrad Heumann. Frankfurt am Main: \emph{S. Fischer} 2011, S. 605 (Sämtliche Werke. Kritische Ausgabe, XL).} }\toendnotes[C]{\smallbreak}\pstart
           \noindent{}{\pb}Meinem lieben Hugo\pend
           \pstart \spacefill\mbox{Arthur}\pend{}\pstart
           \noindent{}\textcolor{pink}{Wien}{}\ledrightnote{\textcolor{pink}{Wien}}{ }\label{K_L01593_1v}\edtext{März 906}{\lemma{\textnormal{\emph{März 906}}}\Cendnote{\textnormal{Die Datierung folgt der
                                Widmung an \textcolor{blue}{Bahr}, 23. 3. 1906.}}}\label{K_L01593_1h}.\pend
           {\bigskip}\pstart
           \noindent{}\centering{}{\pb}\textcolor{gray}{\textbf{\textcolor{green}{MARIONETTEN}{}\ledrightnote{\textcolor{green}{Marionetten. Drei Einakter}}}}\pend
           \pstart
           \noindent{}\centering{}\textcolor{gray}{\textbf{Drei Einakter von}}{\\}\textcolor{gray}{\textbf{\so{Arthur Schnitzler}}}\pend
           {\bigskip}\pstart
           \noindent{}\centering{}\textcolor{gray}{\textbf{\textcolor{brown}{S. Fischer, Verlag}{}\ledrightnote{\textcolor{brown}{S. Fischer Verlag}}{ }\textcolor{pink}{Berlin}{}\ledrightnote{\textcolor{pink}{Berlin}}}}\pend
           \pstart
           \noindent{}\centering{}\textcolor{gray}{\textbf{1906}}\pend
           \endnumbering\briefempfaengerindex{Hofmannsthal, Hugo von@\textsc{Hofmannsthal, Hugo von}!zzzSchnitzler, Arthur@\emph{von Arthur Schnitzler}!1906-03-232@{{[}23.?{]} 3. 1906}|)be}\mylabel{h}  \normalsize

\doendnotes{C}
\bigskip
\vfill

\clearpage

\footnotesize

\lohead{\textsc{register}}

% Definiere theindex-Environment komplett neu ohne reledmac
\makeatletter
\renewenvironment{theindex}{%
  \section*{\indexname}%
  \setlength{\parindent}{0pt}%
  \setlength{\parskip}{0pt plus 0.3pt}%
  \let\item\@idxitem
}{%
  \clearpage
}
\makeatother

\IfFileExists{\jobname-pw.ind}{\input{\jobname-pw.ind}}{}

\end{document}

      