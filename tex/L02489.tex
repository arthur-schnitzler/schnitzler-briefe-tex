%% latex-korrekturansicht-vorspann.tex
%% Vorspann für die Korrekturansicht.
%% Lädt die gemeinsame Datei latex-vorspann.tex mit gesetztem Schalter.

\newif\ifkorrekturansicht
\korrekturansichttrue

\input{../tex-inputs/latex-vorspann}


               \section[Arthur Schnitzler an Richard Beer-Hofmann, 17. 7. 1927]{ Arthur Schnitzler an Richard Beer-Hofmann, 17. 7. 1927}\nopagebreak\mylabel{v}\rehead{ }\normalsize\beginnumbering\briefempfaengerindex{Beer-Hofmann, Richard@\textsc{Beer-Hofmann, Richard}!zzzSchnitzler, Arthur@\emph{von Arthur Schnitzler}!1927-07-171@{17. 7. 1927}|(be} \toendnotes[C]{\smallbreak\pagebreak[2]} \Standort{YCGL, MSS 31.}
\physDesc{Brief, 1 Blatt (Briefpapier mit Trauerrand), 2 Seiten, Umschlag
\newline{}Handschrift: Bleistift, lateinische Kurrent\newline{}Versand: ohne postalischen Übermittlungsvermerk }\buchAbdrucke{\weitereDrucke{Arthur Schnitzler, Richard Beer-Hofmann: \emph{Briefwechsel 1891–1931}. Hg. Konstanze Fliedl. Wien, Zürich: \emph{Europaverlag} 1992, S. 230.} }\toendnotes[C]{\smallbreak}\pstart{}{\pb}\label{T_L02489-1v}\edtext{\textcolor{gray}{\textbf{A. S.}}}{\lemma{\textnormal{\emph{A. S.}}}\Cendnote{\textnormal{ovaler Absenderkleber}}}\label{T_L02489-1h}\pend{}\pstart{}\textcolor{pink}{\textcolor{gray}{\textbf{WIEN, XVIII.}}}{}\ledrightnote{\textcolor{pink}{XVIII., Währing}}\pend{}\pstart{}\textcolor{pink}{\textcolor{gray}{\textbf{STERNWARTESTR. 71}}}{}\ledrightnote{\textcolor{pink}{Sternwartestraße}}\pend{}{\bigskip}\pstart{}{\pb}Hrn Dr Richard Beer Hofma{\geminationn}\pend{}\pstart{}\textcolor{pink}{Wien XVIII}{}\ledrightnote{\textcolor{pink}{Wien}}\pend{}\pstart{}\textcolor{pink}{Hasenauerstr 59}{}\ledrightnote{\textcolor{pink}{Hasenauerstraße}}\pend{}{\bigskip}\pstart
           \raggedleft{}{\pb}\textcolor{pink}{Wien}{}\ledrightnote{\textcolor{pink}{Wien}}, 17. 7  927\pend
           \pstart
           lieber Richard, wollen Sie heute mit \textcolor{blue}{Paula}{}\ledrightnote{\textcolor{blue}{Paula Beer-Hofmann}} bei mir nachtmahlen? \introOben{}(8 Uhr)\introOben{} Ich
               würde mich sehr freuen. »\uline{Nach dem} Nachtmahl« wird
               nicht acceptirt, da ich in diesem Falle selbst außer Haus ginge. Sagen Sie ja, so
               würde ich auch Frau Dr \textcolor{blue}{Menczel}{}\ledrightnote{\textcolor{blue}{Rosa Menczel}} bitten lassen.
                  (\textcolor{blue}{Er}{}\ledrightnote{→\textcolor{blue}{Philipp Menczel}} ist schon fort, nicht
               wahr\substVorne{}\textsuperscript{{\dotstwo}}\substDazwischen{}?\substHinten{})\pend
           \pstart
           {\pb}Auf Wiedersehen hoffentlich,\pend
           \pstart
           herzlichst\hspace*{1.5em}Ihr{\\[\baselineskip]}\spacefill\mbox{Arthur}\pend
           \leftskip=0em{}\endnumbering\briefempfaengerindex{Beer-Hofmann, Richard@\textsc{Beer-Hofmann, Richard}!zzzSchnitzler, Arthur@\emph{von Arthur Schnitzler}!1927-07-171@{17. 7. 1927}|)be}\mylabel{h}  \normalsize

\doendnotes{C}
\bigskip
\vfill

\clearpage

\footnotesize

\lohead{\textsc{register}}

% Definiere theindex-Environment komplett neu ohne reledmac
\makeatletter
\renewenvironment{theindex}{%
  \section*{\indexname}%
  \setlength{\parindent}{0pt}%
  \setlength{\parskip}{0pt plus 0.3pt}%
  \let\item\@idxitem
}{%
  \clearpage
}
\makeatother

\IfFileExists{\jobname-pw.ind}{\input{\jobname-pw.ind}}{}

\end{document}

      