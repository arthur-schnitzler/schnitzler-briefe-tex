%% latex-korrekturansicht-vorspann.tex
%% Vorspann für die Korrekturansicht.
%% Lädt die gemeinsame Datei latex-vorspann.tex mit gesetztem Schalter.

\newif\ifkorrekturansicht
\korrekturansichttrue

\input{../tex-inputs/latex-vorspann}


               \section[Gerty von Hofmannsthal an Olga Schnitzler, 23. {[}9. 1911{]}]{ Gerty von Hofmannsthal an Olga Schnitzler, 23. {[}9. 1911{]}}\nopagebreak\mylabel{v}\rehead{ }\normalsize\beginnumbering\briefempfaengerindex{Schnitzler, Olga@\textsc{Schnitzler, Olga}!zzzHofmannsthal, Gertrude von@\emph{von Gertrude von Hofmannsthal}!1911-09-231@{23. {[}9. 1911{]}}|(be} \toendnotes[C]{\smallbreak\pagebreak[2]} \Standort{CUL, Schnitzler, B 43.}
\physDesc{Bildpostkarte
\newline{}Handschrift: schwarze Tinte, lateinische Kurrent\newline{}Versand: Stempel: »\nobreak{}23. {[}9. 1911{]}\nobreak{}«.  \newline{}Ordnung: 1) mit Bleistift von unbekannter Hand nummeriert: »\strikeout{360}« 2) mit Bleistift von unbekannter Hand nummeriert: »246«}\pstart{}{\pb}Frau Olga
                  Schnitzler\pend{}\pstart{}\textcolor{pink}{Wien}{}\ledrightnote{\textcolor{pink}{Wien}}\pend{}\pstart{}\textcolor{pink}{XVIII Sternwartestrasse}{}\ledrightnote{\textcolor{pink}{Sternwartestraße}}\pend{}{\bigskip}\pstart
           \noindent{}\centering{}\textcolor{gray}{\textbf{{\pb}\textcolor{pink}{København, Nyhavn}{}\ledrightnote{\textcolor{pink}{Nyhavn}}}}\pend
           \pstart
           {\pb}In dieser für Euch bekannten
               Umgebung gedenken wir Euch besonders oft und herzlichst. Wie schwere und schmerzliche
               Tage mögen für \textcolor{blue}{Arthur}{}\ledrightnote{} und Sie jetzt gewesen sein {\pb}und wie oft sind meine Gedanken
               mit Euch\pend
           \pstart
           Herzlichst{\\[\baselineskip]}\spacefill\mbox{Gerty}\pend
           \leftskip=0em{}\endnumbering\briefempfaengerindex{Schnitzler, Olga@\textsc{Schnitzler, Olga}!zzzHofmannsthal, Gertrude von@\emph{von Gertrude von Hofmannsthal}!1911-09-231@{23. {[}9. 1911{]}}|)be}\mylabel{h}  \normalsize

\doendnotes{C}
\bigskip
\vfill

\clearpage

\footnotesize

\lohead{\textsc{register}}

% Definiere theindex-Environment komplett neu ohne reledmac
\makeatletter
\renewenvironment{theindex}{%
  \section*{\indexname}%
  \setlength{\parindent}{0pt}%
  \setlength{\parskip}{0pt plus 0.3pt}%
  \let\item\@idxitem
}{%
  \clearpage
}
\makeatother

\IfFileExists{\jobname-pw.ind}{\input{\jobname-pw.ind}}{}

\end{document}

      