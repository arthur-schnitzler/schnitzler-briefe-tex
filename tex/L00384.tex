%% latex-korrekturansicht-vorspann.tex
%% Vorspann für die Korrekturansicht.
%% Lädt die gemeinsame Datei latex-vorspann.tex mit gesetztem Schalter.

\newif\ifkorrekturansicht
\korrekturansichttrue

\input{../tex-inputs/latex-vorspann}


               \section[Richard Beer-Hofmann an Arthur Schnitzler, 18. 10. 1894]{ Richard Beer-Hofmann an Arthur Schnitzler, 18. 10. 1894}\nopagebreak\mylabel{v}\rehead{ }\normalsize\beginnumbering\briefempfaengerindex{Schnitzler, Arthur@\textsc{Schnitzler, Arthur}!zzzBeer-Hofmann, Richard@\emph{von Richard Beer-Hofmann}!1894-10-181@{18. 10. 1894}|(be} \toendnotes[C]{\smallbreak\pagebreak[2]} \Standort{CUL, Schnitzler, B 8.}
\physDesc{Brief, 2 Blätter, 8 Seiten
\newline{}Handschrift: Bleistift, lateinische Kurrent
\newline{}Schnitzler: mit Bleistift datiert: »17/10 94« und nummeriert: »49«, Datum auf dem
                                 zweiten Blatt wiederholt }\buchAbdrucke{\weitereDrucke{Arthur Schnitzler, Richard Beer-Hofmann: \emph{Briefwechsel 1891–1931}. Hg. Konstanze Fliedl. Wien, Zürich: \emph{Europaverlag} 1992, S. 64–65.} }\toendnotes[C]{\smallbreak}\pstart
           \noindent{}{\pb}Lieber Arthur! Ich verdiene es nicht – aber schreiben Sie – ich
               meine Briefe an mich. Ich bin furchtbar neugierig auf Ihr Stück. Sie werden es mir
               separat vorlesen müssen, und \textcolor{blue}{Salten}{}\ledrightnote{\textcolor{blue}{Felix Salten}} und \textcolor{blue}{Hugo}{}\ledrightnote{\textcolor{blue}{Hugo von Hofmannsthal}} werden bitten es nochmal hören zu dürfen.
               Wenn \textcolor{blue}{Kraus}{}\ledrightnote{\textcolor{blue}{Karl Kraus}} sich überni{\geminationm}t, sagen Sie {\pb}ihm die Worte: »\textcolor{green}{\uline{Musenalmanach}}{}\ledrightnote{\textcolor{green}{Moderner Musen-Almanach auf das Jahr 1894}} – \textcolor{blue}{\uline{Herodot}}{}\ledrightnote{\textcolor{blue}{Herodot}}« und er wird \label{K_L00384_1v}\edtext{erbleichen}{\lemma{\textnormal{\emph{erbleichen}}}\Cendnote{\textnormal{Anspielung darauf, dass \textcolor{blue}{Kraus} in seiner \textcolor{green}{Rezension} »autoritätsgläubiger« ist als \textcolor{blue}{Herodot}, der die Zeitbedingtheit von Ruhm
                  thematisierte?}}}\label{K_L00384_1h}.\pend
           \pstart
           Ich habe gestern eine Karte an Sie geschrieben. Wegen »\textcolor{brown}{Saubermänner}{}\ledrightnote{\textcolor{brown}{Saubermänner}}«, suchen Sie es zu vereiteln, daß \textcolor{blue}{Schönthan}{}\ledrightnote{\textcolor{blue}{Paul von Schönthan-Pernwald}} an mich eine Aufforderung richtet beizutreten. Refus
               wäre Beleidigung, und es ist genug, daß Sie beitre{\pb}ten mussten. »\textcolor{green}{Ikarus Ikarus, Ja{\geminationm}ers
                  genug}{}\ledrightnote{→\textcolor{green}{Faust}}« – (Mir ko{\geminationm}t vor ich citire \label{K_L00384_2v}\edtext{ungenau}{\lemma{\textnormal{\emph{ungenau}}}\Cendnote{\textnormal{richtig: »Jammer«}}}\label{K_L00384_2h} – oder genau – oder – ungenau sagt
               A. S.)\pend
           \pstart
           Denken Sie, ich erhalte gleichzeitig mit \uline{Ihrem} Brief
               einen von \textcolor{blue}{S. Fischer}{}\ledrightnote{\textcolor{blue}{Samuel Fischer}}, der vor kurzem wie er
               schreibt meine Novellen gelesen hat und er hegt »seit jener Zeit den lebhaften Wunsch
                  {\pb}falls Sie betreffs Ihrer
               zukünftigen Production mit einem andern Verlag noch nichts vereinbart haben Ihre
               Werke in meinem Verlage zu publiciren« folgt eine Schilderung seines Verlages und die
               inhaltsschwere Phrase: »mannigfache Vorteile bieten zu können«. Zum Schluss
               Aufforderung eine Novelle bei ihm zu publiciren (\textcolor{green}{freie
                  Bühne}{}\ledrightnote{\textcolor{green}{Neue Deutsche Rundschau}}). »Sollten Sie {\pb}etwas
               fertig haben, so würden Sie uns durch die Einsendung sehr erfreuen«: Dem »erfreuten
               u. lebhaftwünschenden« Verlag werde ich natürlich furchtbar frech antworten, oder
               besser vornehm reservirt – schon weil ich – (ich weiss es ist peinlich, für meine
               Freunde, ich fange an lächerliche Figur zu werden, ich soll doch was fer{\pb}tig machen, –  oder nein ich soll
               mir Zeit lassen) nichts fertig habe. –\pend
           \pstart
           Ich bin längstens 5ten Nov. in \textcolor{pink}{Wien}{}\ledrightnote{\textcolor{pink}{Wien}}. Ich
               fange an meine Aufnahmsfähigkeit zu verlieren – \uline{zu
                  viel}, – zu viel stürmt auf einen, Landschaft Kunst und manchmal {\pb}auch eigne Gedanken über all das,
               und über anderes, – durch Associationen verrücktester Art hervorgerufen.\pend
           \pstart
           Ich freue mich sehr auf Euch und \textcolor{pink}{Wien}{}\ledrightnote{\textcolor{pink}{Wien}}. Hier in \textcolor{pink}{\uline{Italien}}{}\ledrightnote{\textcolor{pink}{Italien}} – in \textcolor{pink}{\uline{Rom}}{}\ledrightnote{\textcolor{pink}{Rom}} in \textcolor{pink}{\uline{Neapel}}{}\ledrightnote{\textcolor{pink}{Neapel}} empfinde ich es daß die einzige Stadt wo ich leben {\pb}und – bitte nicht zu lachen –
               arbeiten kann doch nur \textcolor{pink}{Wien}{}\ledrightnote{\textcolor{pink}{Wien}} ist. Was aber kein
               Coupletrefrain sein soll. Schreiben Sie mir bald, – \textcolor{pink}{\uline{Neapel}}{}\ledrightnote{\textcolor{pink}{Neapel}}.\pend
           \pstart
           Herzlichst Ihr{\\[\baselineskip]}\spacefill\mbox{Richard}\pend
           \leftskip=0em{}\pstart
           Donnerstag\hspace*{1.5em}\textcolor{pink}{\uline{Neapel}}{}\ledrightnote{\textcolor{pink}{Neapel}}\pend
           \pstart
           18/10 94.\pend
           \endnumbering\briefempfaengerindex{Schnitzler, Arthur@\textsc{Schnitzler, Arthur}!zzzBeer-Hofmann, Richard@\emph{von Richard Beer-Hofmann}!1894-10-181@{18. 10. 1894}|)be}\mylabel{h}  \normalsize

\doendnotes{C}
\bigskip
\vfill

\clearpage

\footnotesize

\lohead{\textsc{register}}

% Definiere theindex-Environment komplett neu ohne reledmac
\makeatletter
\renewenvironment{theindex}{%
  \section*{\indexname}%
  \setlength{\parindent}{0pt}%
  \setlength{\parskip}{0pt plus 0.3pt}%
  \let\item\@idxitem
}{%
  \clearpage
}
\makeatother

\IfFileExists{\jobname-pw.ind}{\input{\jobname-pw.ind}}{}

\end{document}

      