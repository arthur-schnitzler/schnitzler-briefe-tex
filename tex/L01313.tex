%% latex-korrekturansicht-vorspann.tex
%% Vorspann für die Korrekturansicht.
%% Lädt die gemeinsame Datei latex-vorspann.tex mit gesetztem Schalter.

\newif\ifkorrekturansicht
\korrekturansichttrue

\input{../tex-inputs/latex-vorspann}


               \section[Arthur Schnitzler an Richard Beer-Hofmann, 23. 8. 1903]{ Arthur Schnitzler an Richard Beer-Hofmann, 23. 8. 1903}\nopagebreak\mylabel{v}\rehead{ }\normalsize\beginnumbering\briefempfaengerindex{Beer-Hofmann, Richard@\textsc{Beer-Hofmann, Richard}!zzzSchnitzler, Arthur@\emph{von Arthur Schnitzler}!1903-08-232@{23. 8. 1903}|(be} \toendnotes[C]{\smallbreak\pagebreak[2]} \Standort{YCGL, MSS 31.}
\physDesc{Brief, 1 Blatt, 3 Seiten, Umschlag
\newline{}Handschrift: Bleistift, deutsche Kurrent\newline{}Versand: Stempel: »\nobreak{}\oindex{IX., Alsergrund@\textbf{IX., Alsergrund}, \emph{Bezirk (A.BZK)}|pwk}Wien \textcolor{gray}{9/3}, 24. \textcolor{gray}{8}. 0\textcolor{gray}{3}, 7–\textcolor{gray}{9}V\nobreak{}«.  }\buchAbdrucke{\weitereDrucke{Arthur Schnitzler, Richard Beer-Hofmann: \emph{Briefwechsel 1891–1931}. Hg. Konstanze Fliedl. Wien, Zürich: \emph{Europaverlag} 1992, S. 163–164.} }\toendnotes[C]{\smallbreak}\pstart{}{\pb}Herrn \textsc{Dr Richard }\pend{}\pstart{}\textsc{Beer-Hofma{\geminationn}}\pend{}\pstart{}\textcolor{pink}{\textsc{Rodaun}}{}\ledrightnote{\textcolor{pink}{Rodaun}}{ }\textsuperscript{b}/\textcolor{pink}{\textsc{Wien}}{}\ledrightnote{\textcolor{pink}{Wien}}\pend{}\pstart{}\textsc{\textcolor{pink}{Liesinger Straße 2}{}\ledrightnote{\textcolor{pink}{Liesingerstraße}}.}\pend{}{\bigskip}\pstart
           \raggedleft{}{\pb}23. 8. 903. \pend
           \pstart
           lieber Richard, mein Telegr. iſt eben an Sie abgegangen; ich füge
               brieflich den Vorſchlag bei, dſs Sie dann gleich bei uns in der \textcolor{pink}{Gentzgaſſe}{}\ledrightnote{\textcolor{pink}{Gentzgasse}} eſſen mögen. Vielleicht hat Ihre \textcolor{blue}{Frau}{}\ledrightnote{→\textcolor{blue}{Paula Beer-Hofmann}} am gleichen Tag etwas in \textcolor{pink}{Wien}{}\ledrightnote{\textcolor{pink}{Wien}} zu thun, u da{\geminationn} gilt das
               gleiche, ebenſo herzlich, für ſie. –\pend
           \pstart
           Möchten Sie mir auch in Kürze mittheilen, wie {\pb}Sie das
               ſ. Z. in Ihrem Fall mit Honoraren und Trinkgeldern (von den Taxen abgeſehen) gehalten
               haben?\pend
           \pstart
           Ich verſtändige niemanden von dem Vorgang, ehe meine \textcolor{blue}{Mama}{}\ledrightnote{→\textcolor{blue}{Louise Schnitzler}} wieder zurück iſt der ich auch erſt dann Mittheilg
               machen werde. Alſo ſagen Sie bitte auch niemandem was davon. –\pend
           \pstart
           Meine Reiſe war ſehr ſchön; das neue \textcolor{pink}{Hotel}{}\ledrightnote{→\textcolor{pink}{Palast Hotel Lido}} in \textcolor{pink}{Riva}{}\ledrightnote{\textcolor{pink}{Riva del Garda}} ſcheint angenehm zu ſein; ich
               denke {\pb}mit \textcolor{blue}{Olga}{}\ledrightnote{\textcolor{blue}{Olga Schnitzler}}{ }Mitte September dorthin zu reiſen. Vielleicht ſpäter \textcolor{pink}{Meran}{}\ledrightnote{\textcolor{pink}{Meran}}.\pend
           \pstart
           Herzlichſt Ihr{\\[\baselineskip]}\spacefill\mbox{Arthur}\pend
           \leftskip=0em{}\endnumbering\briefempfaengerindex{Beer-Hofmann, Richard@\textsc{Beer-Hofmann, Richard}!zzzSchnitzler, Arthur@\emph{von Arthur Schnitzler}!1903-08-232@{23. 8. 1903}|)be}\mylabel{h}  \normalsize

\doendnotes{C}
\bigskip
\vfill

\clearpage

\footnotesize

\lohead{\textsc{register}}

% Definiere theindex-Environment komplett neu ohne reledmac
\makeatletter
\renewenvironment{theindex}{%
  \section*{\indexname}%
  \setlength{\parindent}{0pt}%
  \setlength{\parskip}{0pt plus 0.3pt}%
  \let\item\@idxitem
}{%
  \clearpage
}
\makeatother

\IfFileExists{\jobname-pw.ind}{\input{\jobname-pw.ind}}{}

\end{document}

      