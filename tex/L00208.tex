%% latex-korrekturansicht-vorspann.tex
%% Vorspann für die Korrekturansicht.
%% Lädt die gemeinsame Datei latex-vorspann.tex mit gesetztem Schalter.

\newif\ifkorrekturansicht
\korrekturansichttrue

\input{../tex-inputs/latex-vorspann}


               \section[Arthur Schnitzler an Richard Beer-Hofmann, 3. 5. 1893]{ Arthur Schnitzler an Richard Beer-Hofmann, 3. 5. 1893}\nopagebreak\mylabel{v}\rehead{ }\normalsize\beginnumbering\briefempfaengerindex{Beer-Hofmann, Richard@\textsc{Beer-Hofmann, Richard}!zzzSchnitzler, Arthur@\emph{von Arthur Schnitzler}!1893-05-031@{3. 5. 1893}|(be} \toendnotes[C]{\smallbreak\pagebreak[2]} \Standort{YCGL, MSS 31.}
\physDesc{gedruckte Todesanzeige, Umschlag mit Trauerrand
\newline{}Druck: »\textcolor{brown}{M. ENGEL{ }{\kaufmannsund}{ }SÖHNE}{ }\textcolor{pink}{WIEN, 1.,
                                       LICHTENFELSGASSE 9}«
\newline{}Handschrift: schwarze Tinte, deutsche Kurrent (\noindent{}Umschlag)\newline{}Versand: Stempel: »\nobreak{}Wien 1/1, 3. 5. 93, 3–4 N\nobreak{}«.  }\pstart{}{\pb}Herrn \textsc{Dr. Rich.
                     Beer-Hofmann}\pend{}\pstart{}\textsc{\textcolor{pink}{Wien}{}\ledrightnote{\textcolor{pink}{Wien}}}\pend{}\pstart{}\textsc{\textcolor{pink}{I Wollzeile 15}{}\ledrightnote{\textcolor{pink}{Wollzeile}}}.\pend{}{\bigskip}\pstart
           \noindent{}{\pb}Tieferschüttert geben die Unterzeichneten hiemit im
               eigenen und im Namen der Familie Nachricht von dem Hinscheiden ihres innigstgeliebten
               Gatten, resp. Vaters, Bruders und Schwiegervaters, des Herrn\pend
           \pstart
           \centering{}Dr. \textcolor{blue}{Johann Schnitzler}{}\ledrightnote{\textcolor{blue}{Johann Schnitzler}}\pend
           \pstart
           \noindent{}\centering{}k. k. Regierungsrath, k. k. a. o. Universitäts-Professor, Direktor der
                  \textcolor{brown}{allgemeinen Poliklinik}{}\ledrightnote{\textcolor{brown}{Allgemeine Poliklinik}}, Commandeur des \textcolor{pink}{dän.}{}\ledrightnote{\textcolor{pink}{Dänemark}}{ }\textcolor{brown}{Danebrog-Orden}{}\ledrightnote{\textcolor{brown}{Dannebrogorden}}s etc. etc.\pend
           \pstart
           \noindent{}welcher nach kurzem Leiden am 2. Mai 1893, Nachmittags ½ 2
                  Uhr, im 59. Lebensjahre verschieden ist.\pend
           \pstart
           Die irdische Hülle des theuren Verblichenen wird Donnerstag, den 4. Mai,
                  ½ 10 Uhr Vormittags vom Trauerhause \textcolor{pink}{I.,
                  Burgring 1}{}\ledrightnote{\textcolor{pink}{Burgring}}, auf den \textcolor{pink}{Central-Friedhof (israel.
                  Abtheilung)}{}\ledrightnote{\textcolor{pink}{Wiener Zentralfriedhof}} überführt und dort zur ewigen Ruhe bestattet.\pend
           \leftskip=3em{}\pstart
           \noindent{}\textcolor{pink}{\so{Wien}}{}\ledrightnote{\textcolor{pink}{Wien}}, 3. Mai 1893.\pend
           \leftskip=0em{}\pstart
           \noindent{}\centering{}\textcolor{blue}{Louise Schnitzler}{}\ledrightnote{\textcolor{blue}{Louise Schnitzler}}{\\}geb. \textcolor{blue}{Markbreiter}{}\ledrightnote{\textcolor{blue}{Louise Schnitzler}}{\\}als Gattin.\pend
           \pstart
           \noindent{}Dr. Arthur Schnitzler\hfill \textcolor{blue}{Johanna Willheim}{}\ledrightnote{\textcolor{blue}{Johanna Willheim}}\pend
           \pstart
           Dr. \textcolor{blue}{Julius Schnitzler}{}\ledrightnote{\textcolor{blue}{Julius Schnitzler}}\hfill geb. Schnitzler\pend
           \pstart
           \textcolor{blue}{Gisela Hajek}{}\ledrightnote{\textcolor{blue}{Gisela Hajek}}\hfill als Schwester.\pend
           \pstart
           als Kinder.\hfill Dr. \textcolor{blue}{Marcus Hajek}{}\ledrightnote{\textcolor{blue}{Markus Hajek}}\pend
           \pstart
           \raggedleft{}als Schwiegersohn\pend
           \endnumbering\briefempfaengerindex{Beer-Hofmann, Richard@\textsc{Beer-Hofmann, Richard}!zzzSchnitzler, Arthur@\emph{von Arthur Schnitzler}!1893-05-031@{3. 5. 1893}|)be}\mylabel{h}  \normalsize

\doendnotes{C}
\bigskip
\vfill

\clearpage

\footnotesize

\lohead{\textsc{register}}

% Definiere theindex-Environment komplett neu ohne reledmac
\makeatletter
\renewenvironment{theindex}{%
  \section*{\indexname}%
  \setlength{\parindent}{0pt}%
  \setlength{\parskip}{0pt plus 0.3pt}%
  \let\item\@idxitem
}{%
  \clearpage
}
\makeatother

\IfFileExists{\jobname-pw.ind}{\input{\jobname-pw.ind}}{}

\end{document}

      