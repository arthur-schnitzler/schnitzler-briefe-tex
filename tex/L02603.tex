%% latex-korrekturansicht-vorspann.tex
%% Vorspann für die Korrekturansicht.
%% Lädt die gemeinsame Datei latex-vorspann.tex mit gesetztem Schalter.

\newif\ifkorrekturansicht
\korrekturansichttrue

\input{../tex-inputs/latex-vorspann}


               \section[Arthur Schnitzler an Karin Michaëlis, 8. 11. 1916]{ Arthur Schnitzler an Karin Michaëlis, 8. 11. 1916}\nopagebreak\mylabel{v}\rehead{ }\normalsize\beginnumbering\briefempfaengerindex{Michaelis, Karin@\textsc{Michaëlis, Karin}!zzzSchnitzler, Arthur@\emph{von Arthur Schnitzler}!1912-10-211@{8. 11. 1916}|(be} \toendnotes[C]{\smallbreak\pagebreak[2]} \Standort{Kopenhagen, Det Kongelige Bibliotek, Palsbo Ac.}
\physDesc{Postkarte
\newline{}Handschrift: schwarze Tinte, lateinische Kurrent\newline{}Versand: 1) Stempel: »\nobreak{}Wien, 8. 11. 16, 4\nobreak{}«.  2) Stempel: »\nobreak{}Zensuriert \textcolor{brown}{K. u. k.
                                          Zensurstelle}\nobreak{}«. 3) Stempel: »\nobreak{}\oindex{Svendborg@\textbf{Svendborg}, \emph{https://www.geonames.org/ontologyP.PPLA2}|pwk}Svendborg, 14. 11. 16, 7–9F\nobreak{}«. 4) Stempel: »\nobreak{}\oindex{Thurø@\textbf{Thurø}, \emph{Insel (N.INS)}|pwk}Thurø\nobreak{}«. 5) ursprüngliche Adressierung überklebt und von unbekannter Hand
                                 mit schwarzer Tinte neue Empfangsadresse vermerkt: »\noindent{}adr{ / }Fru \textcolor{blue}{Herdis Bergstrøm}{ / }\textcolor{pink}{Dosseringen 304 {\\}Kobenhamn}« }\buchAbdrucke{\weitereDrucke{Arthur Schnitzler: \emph{Arthur Schnitzlers Briefe nach Dänemark}. Hg. Ernst-Ulrich Pinkert. Roskilde: \emph{Center for Østrigsk-Nordiske Kulturstudier} 2006, S. 18.} }\toendnotes[C]{\smallbreak}\pstart{}{\pb}\textcolor{gray}{\textbf{Dr. Arthur Schnitzler}}\pend{}\pstart{}\textcolor{gray}{\textbf{\textcolor{pink}{Wien XVIII. Sternwartestrasse 71}{}\ledrightnote{\textcolor{pink}{Sternwartestraße}}}}\pend{}{\bigskip}\pstart{}Frau Karin Michaelis\pend{}\pstart{}{[}\textcolor{pink}{\label{K_L02603-11v}\edtext{Thurø}{\lemma{\textnormal{\emph{Thurø}}}\Cendnote{\textnormal{Bei dieser Adresszeile handelt es sich um eine
                           Rekonstruktion, da der betreffende Teil auf der Karte abgeklebt ist. Da
                           sich aber der Stempel von \textcolor{pink}{Thurø} auf
                           der Karte findet und hier Karin Michaëlis einen Wohnsitz hatte, kann die
                           ursprüngliche Adressangabe, zumindest soweit es um die Ortsangabe geht,
                           erschlossen werden.}}}\label{K_L02603-11h}}{}\ledrightnote{\textcolor{pink}{Thurø}}{]}\pend{}\pstart{}\damage{\textcolor{gray}{\textcolor{pink}{Dänemark}{}\ledrightnote{\textcolor{pink}{Dänemark}}}}\pend{}{\bigskip}\pstart
           \raggedleft{}{\pb}8. 11. 916\pend
           \pstart
           verehrte Frau Karin Michaelis – es freut mich sehr,
               daß Ihnen die \textcolor{green}{Beate}{}\ledrightnote{\textcolor{green}{Frau Beate und ihr Sohn. Novelle}} gefallen hat, eins meiner
               Werke, das vielfach und mit besondrer Vorliebe \label{K_L02603-1v}\edtext{misverstanden}{\lemma{\textnormal{\emph{misverstanden}}}\Cendnote{\textnormal{Kritisch begutachtet wurden vor allem die erotischen Inhalte, ganz besonders die
                  inzestuös deutbaren Momente in der Novelle \emph{\textcolor{green}{Frau
                     Beate und ihr Sohn}} bzw. die »Unsittlichkeit« (vgl. A. S.: \emph{Tagebuch}, 14. 9. 1913) der Protagonistin
                  Beate.}}}\label{K_L02603-1h} wird. Der Schluss scheint ja (offenbar aus künstlerischen – nicht
               dramatischen – Gründen) – wie mir der Zweifel auch \label{K_L02603-2v}\edtext{Wohlwollender}{\lemma{\textnormal{\emph{Wohlwollender}}}\Cendnote{\textnormal{Am
                     24. 2. 1913 las \textcolor{blue}{Schnitzler}{ }\emph{\textcolor{green}{Frau Beate und ihr Sohn}}{ }\textcolor{blue}{Richard Beer-Hofmann}, \textcolor{blue}{Hugo von Hofmannsthal}, \textcolor{blue}{Leo Van-Jung}, \textcolor{blue}{Felix Salten}, \textcolor{blue}{Jakob Wassermann}, \textcolor{blue}{Gustav Schwarzkopf} und seiner Frau \textcolor{blue}{Olga} vor und erntete vor allem für den Schluss Kritik. vgl. A. S.: \emph{Tagebuch}, 23. 2. 1913}}}\label{K_L02603-2h} zu bedenken gibt – nicht durchaus überzeugend zu sein. – Ich schreibe Ihnen
               meinen Dank und Gruß auf einer Karte – die nach meiner Erfahrung \label{K_L02603-3v}\edtext{sichrer ins neutrale \textcolor{pink}{Ausland}{}\ledrightnote{→\textcolor{pink}{Dänemark}} gelangt}{\lemma{\textnormal{\emph{sichrer … gelangt}}}\Cendnote{\textnormal{Postalisch versandte Korrespondenzstücke wurden von der \emph{\textcolor{brown}{K. u. k. Zensurstelle}} gelesen, egal ob Brief
                  oder Postkarte. Bei letzterer wurde aber, da sie offen versandt wurde, eher davon
                  ausgegangen, dass auf ihr keine Geheimnisse stehen konnten.}}}\label{K_L02603-3h} als Briefe –
                  {\pb}auf die Gefahr
               hin, daß Sie mich für minder correct (aber gerade zu langweilig) halten wie früher. \pend
           \pstart
           {\pb}Auf Wiedersehen
               hoffentlich, und sch\textcolor{gray}{oene} Grüße, auch von meiner \textcolor{blue}{Frau}{}\ledrightnote{→\textcolor{blue}{Olga Schnitzler}}. Ihr sehr ergebner{\\[\baselineskip]}\spacefill\mbox{Arthur Schnitzler}\pend
           \leftskip=0em{}\endnumbering\briefempfaengerindex{Michaelis, Karin@\textsc{Michaëlis, Karin}!zzzSchnitzler, Arthur@\emph{von Arthur Schnitzler}!1912-10-211@{8. 11. 1916}|)be}\mylabel{h}  \normalsize

\doendnotes{C}
\bigskip
\vfill

\clearpage

\footnotesize

\lohead{\textsc{register}}

% Definiere theindex-Environment komplett neu ohne reledmac
\makeatletter
\renewenvironment{theindex}{%
  \section*{\indexname}%
  \setlength{\parindent}{0pt}%
  \setlength{\parskip}{0pt plus 0.3pt}%
  \let\item\@idxitem
}{%
  \clearpage
}
\makeatother

\IfFileExists{\jobname-pw.ind}{\input{\jobname-pw.ind}}{}

\end{document}

      