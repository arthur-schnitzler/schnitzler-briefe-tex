%% latex-korrekturansicht-vorspann.tex
%% Vorspann für die Korrekturansicht.
%% Lädt die gemeinsame Datei latex-vorspann.tex mit gesetztem Schalter.

\newif\ifkorrekturansicht
\korrekturansichttrue

\input{../tex-inputs/latex-vorspann}


               \section[Arthur Schnitzler an Hermann Bahr, 25. 9. 1912]{ Arthur Schnitzler an Hermann Bahr, 25. 9. 1912}\nopagebreak\mylabel{v}\rehead{ }\normalsize\beginnumbering\briefempfaengerindex{Bahr, Hermann@\textsc{Bahr, Hermann}!zzzSchnitzler, Arthur@\emph{von Arthur Schnitzler}!1912-09-251@{{[}25./26.{]} 9. 1912}|(be} \toendnotes[C]{\smallbreak\pagebreak[2]} \Standort{TMW, HS AM 60162 Ba.}
\physDesc{Bildpostkarte
\newline{}Handschrift: schwarze Tinte, deutsche Kurrent\newline{}Versand: 1) Stempel: »\nobreak{}\oindex{I., Innere Stadt@\textbf{I., Innere Stadt}, \emph{Bezirk (A.BZK)}|pwk}1/1 Wien 8, 25. IX 12, {[}3{]}–4\nobreak{}«.  2) mit Bleistift von unbekannter Hand die ursprüngliche
                                 Adressierung gestrichen und ausgebessert zu: »\textcolor{pink}{Semmering Villa Mauthner}«\newline{}Ordnung: Lochung \newline{}Zusatz: Postkartenmotiv mit \textcolor{blue}{Olga} und
                                    \textcolor{blue}{Heinrich} links vor dem Haus
                                 und Schnitzler und \textcolor{blue}{Lili} auf dem
                                 Söller }\buchAbdrucke{\weitereDrucke{1) \emph{26. 9. 1912, Abschrift.} In: Arthur Schnitzler: \emph{The Letters of Arthur Schnitzler to Hermann Bahr}. Edited, annotated, and with an introduction, by Donald G.
                        Daviau. Chapel Hill: \emph{The University of North Carolina Press} 1978, S. 109 (University of North Carolina studies in the Germanic languages
                        and literatures, 89).} \weitereDrucke{2) Hermann Bahr, Arthur Schnitzler: \emph{Briefwechsel, Aufzeichnungen, Dokumente (1891–1931)}. Hg. Kurt Ifkovits und Martin Anton Müller. Göttingen: \emph{Wallstein} 2018, S. 477.} }\toendnotes[C]{\smallbreak}\pstart{}{\pb}Herrn \pend{}\pstart{}\textsc{Hermann Bahr}\pend{}\pstart{}\textcolor{pink}{Wien Ober St. Veit}{}\ledrightnote{\textcolor{pink}{Ober Sankt Veit}}\pend{}\pstart{}\textcolor{pink}{Veitliſſengaſſe}{}\ledrightnote{\textcolor{pink}{Veitlissengasse}}\pend{}{\bigskip}\pstart
           \noindent{}\centering{}\textcolor{gray}{\textbf{{\pb}\textcolor{pink}{Wien, XVIII, Sternwartestr. 71}{}\ledrightnote{\textcolor{pink}{Sternwartestraße}}.}}\pend
           \pstart
           herzlichen Dank, lieber Hermann für dein neues \label{K_L02090_1v}\edtext{\textcolor{green}{B\damage{uc}h}{}\ledrightnote{→\textcolor{green}{Inventur}}}{\lemma{\textnormal{\emph{Buch}}}\Cendnote{\textnormal{\textcolor{blue}{Hermann Bahr}: \emph{\textcolor{green}{Inventur}}. Berlin: \emph{\textcolor{brown}{S. Fischer}}{ }1912.}}}\label{K_L02090_1h} u viele Grüße. Ob die dich \damage{tre}ffen werden, weiſs ich nicht – de{\geminationn} niemand
               weiſs \label{K_L02090_2v}\edtext{wo du biſt}{\lemma{\textnormal{\emph{wo du biſt}}}\Cendnote{\textnormal{\textcolor{blue}{Bahr} war zumindest seit Mitte
                     September mehrere Wochen in der \textcolor{pink}{Villa
                     Mautner}.}}}\label{K_L02090_2h}. So ſei denn der Findigkeit der Poſt {\pb}vertraut. Au\damage{f} bald!\pend
           \pstart Dein \spacefill\mbox{Arthur }\pend{}\pstart
           \label{K_L02090_3v}\edtext{26/9 1912}{\lemma{\textnormal{\emph{26/9 1912}}}\Cendnote{\textnormal{Der Poststempel widerspricht der
                        Datierung.}}}\label{K_L02090_3h}\pend
           \endnumbering\briefempfaengerindex{Bahr, Hermann@\textsc{Bahr, Hermann}!zzzSchnitzler, Arthur@\emph{von Arthur Schnitzler}!1912-09-251@{{[}25./26.{]} 9. 1912}|)be}\mylabel{h}  \normalsize

\doendnotes{C}
\bigskip
\vfill

\clearpage

\footnotesize

\lohead{\textsc{register}}

% Definiere theindex-Environment komplett neu ohne reledmac
\makeatletter
\renewenvironment{theindex}{%
  \section*{\indexname}%
  \setlength{\parindent}{0pt}%
  \setlength{\parskip}{0pt plus 0.3pt}%
  \let\item\@idxitem
}{%
  \clearpage
}
\makeatother

\IfFileExists{\jobname-pw.ind}{\input{\jobname-pw.ind}}{}

\end{document}

      