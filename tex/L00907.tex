%% latex-korrekturansicht-vorspann.tex
%% Vorspann für die Korrekturansicht.
%% Lädt die gemeinsame Datei latex-vorspann.tex mit gesetztem Schalter.

\newif\ifkorrekturansicht
\korrekturansichttrue

\input{../tex-inputs/latex-vorspann}


               \section[Hugo von Hofmannsthal an Arthur Schnitzler, {[}20. 3. 1899{]}]{ Hugo von Hofmannsthal an Arthur Schnitzler,
                    {[}20. 3. 1899{]}}\nopagebreak\mylabel{v}\rehead{ }\normalsize\beginnumbering\briefempfaengerindex{Schnitzler, Arthur@\textsc{Schnitzler, Arthur}!zzzHofmannsthal, Hugo von@\emph{von Hugo von Hofmannsthal}!1899-03-201@{{[}20. 3. 1899{]}}|(be} \toendnotes[C]{\smallbreak\pagebreak[2]} \Standort{CUL, Schnitzler, B 43.}
\physDesc{Brief, 1 Blatt, 1 Seite
\newline{}Handschrift: Bleistift, deutsche Kurrent
\newline{}Schnitzler: mit Bleistift datiert: »am 20 März
            99.« \newline{}Ordnung: 1) mit Bleistift von unbekannter Hand nummeriert: »\strikeout{142}« 2) mit Bleistift von unbekannter Hand nummeriert: »139«}\buchAbdrucke{\weitereDrucke{Hugo von Hofmannsthal, Arthur Schnitzler: \emph{Briefwechsel}. Hg. Therese Nickl und Heinrich Schnitzler. Frankfurt am Main: \emph{S. Fischer} 1964, S. 119.} }\toendnotes[C]{\smallbreak}\pstart{}{\pb}mein guter lieber
                        Arthur\pend\pstart
           es thut mir ſo unausſprechlich \label{K_L00907_1v}\edtext{leid
                    um Sie}{\lemma{\textnormal{\emph{leid
                    um Sie}}}\Cendnote{\textnormal{\textcolor{blue}{Schnitzler} trauerte um seine langjährige
                        Partnerin \textcolor{blue}{Marie Reinhard}, die am
                            18. 3. 1899 an Sepsis gestorben war.}}}\label{K_L00907_1h}, und ich kann
                    nicht einmal ein biſſl um Sie ſein, ich denk faſt den ganzen Tag an Sie. Heut
                    war meine \label{K_L00907_2v}\edtext{\textsc{Promotion}}{\lemma{\textnormal{\emph{Promotion}}}\Cendnote{\textnormal{Die Arbeit war betitelt: \emph{\textcolor{green}{Über den Sprachgebrauch bei den Dichtern der
                            Pléjade}}.}}}\label{K_L00907_2h}, von morgen bin ich in \textsc{\textcolor{pink}{Berlin}{}\ledrightnote{\textcolor{pink}{Berlin}}}\pend
           \pstart
           \centering{}\textsc{\textcolor{pink}{Hotel Windsor}{}\ledrightnote{\textcolor{pink}{Hotel Windsor}}\hspace*{1em}\textcolor{pink}{Behrenstrasse}{}\ledrightnote{\textcolor{pink}{Behrenstraße}}}.\pend
           \pstart
           \noindent{}Bitte \uuline{bitte}{ }ſchreiben Sie mir und arbeiten Sie, zwingen
                    Sie ſich.\pend
           \pstart
           Ihr alter{\\[\baselineskip]}\spacefill\mbox{Hugo}\pend
           \leftskip=0em{}\endnumbering\briefempfaengerindex{Schnitzler, Arthur@\textsc{Schnitzler, Arthur}!zzzHofmannsthal, Hugo von@\emph{von Hugo von Hofmannsthal}!1899-03-201@{{[}20. 3. 1899{]}}|)be}\mylabel{h}  \normalsize

\doendnotes{C}
\bigskip
\vfill

\clearpage

\footnotesize

\lohead{\textsc{register}}

% Definiere theindex-Environment komplett neu ohne reledmac
\makeatletter
\renewenvironment{theindex}{%
  \section*{\indexname}%
  \setlength{\parindent}{0pt}%
  \setlength{\parskip}{0pt plus 0.3pt}%
  \let\item\@idxitem
}{%
  \clearpage
}
\makeatother

\IfFileExists{\jobname-pw.ind}{\input{\jobname-pw.ind}}{}

\end{document}

      