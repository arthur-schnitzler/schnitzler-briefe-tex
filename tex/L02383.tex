%% latex-korrekturansicht-vorspann.tex
%% Vorspann für die Korrekturansicht.
%% Lädt die gemeinsame Datei latex-vorspann.tex mit gesetztem Schalter.

\newif\ifkorrekturansicht
\korrekturansichttrue

\input{../tex-inputs/latex-vorspann}


               \section[Georg Brandes an Arthur Schnitzler, 15. 5. 1922]{ Georg Brandes an Arthur Schnitzler, 15. 5. 1922}\nopagebreak\mylabel{v}\rehead{ }\normalsize\beginnumbering\briefempfaengerindex{Schnitzler, Arthur@\textsc{Schnitzler, Arthur}!zzzBrandes, Georg@\emph{von Georg Brandes}!1922-05-152@{15. 5. 1922}|(be} \toendnotes[C]{\smallbreak\pagebreak[2]} \buchAlsQuelle{Georg Brandes, Arthur Schnitzler: \emph{Ein Briefwechsel}. Hg. Kurt Bergel. Bern: \emph{Francke} 1956, S. 136.}\toendnotes[C]{\smallbreak}\pstart
           \raggedleft{}{\pb}\textcolor{pink}{Kopenhagen}{}\ledrightnote{\textcolor{pink}{Kopenhagen}}, 15. Mai 1922\pend
           \pstart{}Mein lieber Freund\pend\pstart
           Im Jahre 1898 saß ich an diesem Tag an Ihrem Tisch in einem kleinen
               Kreis; auch Ihre Frau \textcolor{blue}{Mutter}{}\ledrightnote{→\textcolor{blue}{Louise Schnitzler}}
               war damals anwesend. Ich sagte das wenig geistvolle Wort: »Sie sind also gerade
               20 Jahre jünger als ich« und Sie antworteten lächelnd: »Und ich denke, wir werden
               auch in der Zukunft denselben Abstand von einander innehalten.«\pend
           \pstart
           Wir haben es also noch 20 Jahren gethan. Daß ich Ihnen Glück wünsche, versteht sich
               von selbst, wenn dieser mythologische Begriff sonst einen Sinn hat; ich wünsche Ihnen
               jedenfalls alles Gute, und ich danke Ihnen von Herzen für das, was Sie 30 Jahre
               hindurch mir gewesen sind, eine stets rinnende Quelle geistiger Genüsse, ja mehr als
               das: Sie haben mir das so seltene Gefühl gegeben, \emph{\label{K_L02383_1v}\edtext{in der Ferne einen congenialen Freund}{\lemma{\textnormal{\emph{in … Freund}}}\Cendnote{\textnormal{Das Original dieses Korrespondenzstücks
                  ist verschollen. Es fehlt auch in den Abschriften, die vom Briefwechsel gemacht
                  wurden. Der Text folgt der Buchausgabe, die diese Unterstreichung \textcolor{blue}{Schnitzler} zuschreibt, zugleich aber durch
                  Kursivsetzung in den edierten Text übernimmt.}}}\label{K_L02383_1h}} zu haben.\pend
           \pstart
           Als ich von meiner vierteljährigen Abwesenheit hier ankam, wurde mir allgemein
               gesagt, Sie würden am 11. Mai hier sein und hier einen Vortrag halten.
               Ich hatte schon gründlich überlegt, ob meine \textcolor{blue}{Köchin}{}\ledrightnote{→\textcolor{blue}{?? [Köchin von Georg Brandes]}} gut genug sei und welches Restaurant wir für Sie und
               mich und einige Freunde die beste vorkäme, und nun sind Sie nicht da und ich weiß
               nicht den Grund. Es ist eine arge Enttäuschung. Weshalb sind Sie nicht gekommen?
               Unsere Zeitungen sagen es nicht.\pend
           \pstart
           Ich war in \textcolor{pink}{Griechenland}{}\ledrightnote{\textcolor{pink}{Griechenland}}. Mir wurde in \textcolor{pink}{Athen}{}\ledrightnote{\textcolor{pink}{Athen}} viel Freundlichkeit erwiesen. Was nicht \textcolor{blue}{Sokrates}{}\ledrightnote{\textcolor{blue}{Sokrates}} gelang, geschah mir; ich wurde im \label{K_L02383_2v}\edtext{\textcolor{pink}{Prytaneion}{}\ledrightnote{→\textcolor{pink}{Griechisches Parlament}}}{\lemma{\textnormal{\emph{Prytaneion}}}\Cendnote{\textnormal{im antiken \textcolor{pink}{Griechenland}: Regierungssitz.}}}\label{K_L02383_2h} versorgt. Da die Regierung erfuhr,
               ich sei in \textcolor{pink}{Athen}{}\ledrightnote{\textcolor{pink}{Athen}} – in den ersten Tagen kannte ich
               keinen Menschen – ließ sie mich wissen, sie räume mir drei schöne Zimmer mit
               Badezimmer ein; ich darf weder für Essen noch für Wein das geringste zahlen. Sogar
               meine Wäsche werden bezahlt, meine Wagen etc. Und in großer öffentlicher Sitzung wo
               schöne Reden gehalten wurden, machte die \textcolor{pink}{Universität}{}\ledrightnote{\textcolor{pink}{Nationale und Kapodistrias-Universität Athen}} in Gegenwart der Minister, der Prinzen, der Professoren und
               Studenten mich zum Ehrendoctor. Die jungen Studentinnen (meistens aus \textcolor{pink}{Smyrna}{}\ledrightnote{\textcolor{pink}{Izmir}}) warfen Rosenblätter über mich. Das war ein
               südländischer Feier. Glücklicherweise redete ich ganz gut – die anderen
               sprechen \textcolor{pink}{neugriechisch}{}\ledrightnote{\textcolor{pink}{Griechenland}} und \textcolor{pink}{altgriechisch}{}\ledrightnote{\textcolor{pink}{Griechenland}}, ich \textcolor{pink}{französisch}{}\ledrightnote{\textcolor{pink}{Frankreich}}.\pend
           \pstart
           Nun bin ich einsam hier, erwartete Sie, und Sie kommen nicht.\pend
           \pstart Ihr \spacefill\mbox{Georg Brandes}\pend{}\endnumbering\briefempfaengerindex{Schnitzler, Arthur@\textsc{Schnitzler, Arthur}!zzzBrandes, Georg@\emph{von Georg Brandes}!1922-05-152@{15. 5. 1922}|)be}\mylabel{h}  \normalsize

\doendnotes{C}
\bigskip
\vfill

\clearpage

\footnotesize

\lohead{\textsc{register}}

% Definiere theindex-Environment komplett neu ohne reledmac
\makeatletter
\renewenvironment{theindex}{%
  \section*{\indexname}%
  \setlength{\parindent}{0pt}%
  \setlength{\parskip}{0pt plus 0.3pt}%
  \let\item\@idxitem
}{%
  \clearpage
}
\makeatother

\IfFileExists{\jobname-pw.ind}{\input{\jobname-pw.ind}}{}

\end{document}

      