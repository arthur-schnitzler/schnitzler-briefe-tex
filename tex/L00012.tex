%% latex-korrekturansicht-vorspann.tex
%% Vorspann für die Korrekturansicht.
%% Lädt die gemeinsame Datei latex-vorspann.tex mit gesetztem Schalter.

\newif\ifkorrekturansicht
\korrekturansichttrue

\input{../tex-inputs/latex-vorspann}


               \section[Hugo von Hofmannsthal an Arthur Schnitzler, {[}6. 5. 1891?{]}]{ Hugo von Hofmannsthal an Arthur Schnitzler, {[}6. 5. 1891?{]}}\nopagebreak\mylabel{v}\rehead{ }\normalsize\beginnumbering\briefempfaengerindex{Schnitzler, Arthur@\textsc{Schnitzler, Arthur}!zzzHofmannsthal, Hugo von@\emph{von Hugo von Hofmannsthal}!1891-05-061@{{[}6. 5. 1891?{]}}|(be} \toendnotes[C]{\smallbreak\pagebreak[2]} \Standort{CUL, Schnitzler, B 43.}
\physDesc{Visitenkarte des Vaters Hugo August von Hofmannsthal
\newline{}Handschrift: Bleistift, deutsche Kurrent\newline{}Ordnung: 1) mit Bleistift von \textcolor{blue}{Frieda Pollak} (?) mit dem Buchstaben »A« (Abgeschrieben/Abschrift) gekennzeichnet 2) mit Bleistift von unbekannter Hand datiert: »9\strikeout{1.}0«}\buchAbdrucke{\weitereDrucke{Hugo von Hofmannsthal, Arthur Schnitzler: \emph{Briefwechsel}. Hg. Therese Nickl und Heinrich Schnitzler. Frankfurt am Main: \emph{S. Fischer} 1964, S. 7.} }\toendnotes[C]{\smallbreak}\pstart
           \noindent{}\centering{}{\pb}\textcolor{gray}{\textbf{Hugo von Hofmannsthal}}\pend
           \pstart
           \noindent{}dankt beſchämt und warm für \textcolor{green}{Alkandis Lied}{}\ledrightnote{\textcolor{green}{Alkandi’s Lied}}, die
               5 Worte auf dem Titelblatt {\pb}und
               den hübſchen Gedanken, aus einer Höflichkeit der Form eine Höflichkeit des Herzens zu
               machen. Sehen wir uns, falls ich heute den \label{K_L00012_1v}\edtext{Naturaliſtennaturausflug}{\lemma{\textnormal{\emph{Naturaliſtennaturausflug}}}\Cendnote{\textnormal{Hierbei handelt es sich womöglich um die »Landpartie der
                  Naturalisten«, die Schnitzler am 6. 5. 1891 in einer Stoffnotiz
                     erwähnt. (Vgl. \emph{Briefwechsel} Bahr/Schnitzler, S. 7.)}}}\label{K_L00012_1h}
               mitmache? Müßige Frage, gleichviel\hspace*{1.5em}\textsc{à bientôt}\pend
           \endnumbering\briefempfaengerindex{Schnitzler, Arthur@\textsc{Schnitzler, Arthur}!zzzHofmannsthal, Hugo von@\emph{von Hugo von Hofmannsthal}!1891-05-061@{{[}6. 5. 1891?{]}}|)be}\mylabel{h}  \normalsize

\doendnotes{C}
\bigskip
\vfill

\clearpage

\footnotesize

\lohead{\textsc{register}}

% Definiere theindex-Environment komplett neu ohne reledmac
\makeatletter
\renewenvironment{theindex}{%
  \section*{\indexname}%
  \setlength{\parindent}{0pt}%
  \setlength{\parskip}{0pt plus 0.3pt}%
  \let\item\@idxitem
}{%
  \clearpage
}
\makeatother

\IfFileExists{\jobname-pw.ind}{\input{\jobname-pw.ind}}{}

\end{document}

      