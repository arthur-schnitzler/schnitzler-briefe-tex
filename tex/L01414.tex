%% latex-korrekturansicht-vorspann.tex
%% Vorspann für die Korrekturansicht.
%% Lädt die gemeinsame Datei latex-vorspann.tex mit gesetztem Schalter.

\newif\ifkorrekturansicht
\korrekturansichttrue

\input{../tex-inputs/latex-vorspann}


               \section[Hugo von Hofmannsthal an Arthur Schnitzler, 1. 7. 1904]{ Hugo von Hofmannsthal an Arthur Schnitzler, 1. 7. 1904}\nopagebreak\mylabel{v}\rehead{ }\normalsize\beginnumbering\briefempfaengerindex{Schnitzler, Arthur@\textsc{Schnitzler, Arthur}!zzzHofmannsthal, Hugo von@\emph{von Hugo von Hofmannsthal}!1904-07-012@{1. 7. 1904}|(be} \toendnotes[C]{\smallbreak\pagebreak[2]} \Standort{CUL, Schnitzler, B 43.}
\physDesc{Postkarte
\newline{}Handschrift: schwarze Tinte, deutsche Kurrent\newline{}Versand: 1) Stempel: »\nobreak{}\oindex{Rodaun@\textbf{Rodaun}, \emph{Teil eines besiedelten Ortes (A.BSOX)}|pwk}Rodaun, \textcolor{gray}{1. 7. 04}\nobreak{}«.  2) Stempel: »\nobreak{}\oindex{XVIII., Waehring@\textbf{XVIII., Währing}, \emph{Bezirk (A.BZK)}|pwk}18/1 Wien, 2. 7. 04, 8.V, Bestellt\nobreak{}«. 
\newline{}Schnitzler: mit Bleistift datiert: »2. 7 904« \newline{}Ordnung: 1) mit Bleistift von unbekannter Hand nummeriert: »\strikeout{236}« 2) mit Bleistift von unbekannter Hand nummeriert:
                                    »227«}\buchAbdrucke{\weitereDrucke{Hugo von Hofmannsthal, Arthur Schnitzler: \emph{Briefwechsel}. Hg. Therese Nickl und Heinrich Schnitzler. Frankfurt am Main: \emph{S. Fischer} 1964, S. 190.} }\toendnotes[C]{\smallbreak}\pstart{}{\pb}\textsc{Herrn D\textsuperscript{r} Arthur Schnitzler}\pend{}\pstart{}\textcolor{pink}{\textsc{Wien}}{}\ledrightnote{\textcolor{pink}{Wien}}\pend{}\pstart{}\textcolor{pink}{\textsc{XVIII Spöttelgasse 7}}{}\ledrightnote{\textcolor{pink}{Edmund-Weiß-Gasse}}\pend{}{\bigskip}\pstart
           \raggedleft{}{\pb}\label{K_L01414_1v}\edtext{Samstag}{\lemma{\textnormal{\emph{Samstag}}}\Cendnote{\textnormal{Schreibirrtum, nachdem die Karte an
                        einem Samstag um 8 Uhr früh zugestellt wurde.}}}\label{K_L01414_1h}.\pend
           \pstart
           Also Mittwoch, \label{K_L01414_2v}\edtext{\textsc{cher jaune}}{\lemma{\textnormal{\emph{cher jaune}}}\Cendnote{\textnormal{französisch: lieber Gelber}}}\label{K_L01414_2h}, wenn
               es nicht abſurdes Wetter macht.\pend
           \pstart
           \textcolor{blue}{O.}{}\ledrightnote{\textcolor{blue}{Olga Schnitzler}}{ }ſoll ſchön üben. \textcolor{green}{\textsc{\label{K_L01414_3v}\edtext{Leisenbogh}{\lemma{\textnormal{\emph{Leisenbogh}}}\Cendnote{\textnormal{Er bezieht sich bereits auf den
                        Erstdruck, \emph{\textcolor{green}{Die neue Rundschau}}, Jg. 15, H. 7,
                              Juli 1904, S. 829–842. Am 11. 4. 1904 hatte er es bereits mündlich
                        vorgetragen bekommen.}}}\label{K_L01414_3h}}}{}\ledrightnote{\textcolor{green}{Das Schicksal des Freiherrn von Leisenbohg. Novellette}} iſt gut, durchaus angenehm, durchaus fein, ſollte nur um ein Etwas mehr
               Intenſität in der Groteskerie haben.\pend
           \pstart Ihr \spacefill\mbox{Hugo}\pend{}\endnumbering\briefempfaengerindex{Schnitzler, Arthur@\textsc{Schnitzler, Arthur}!zzzHofmannsthal, Hugo von@\emph{von Hugo von Hofmannsthal}!1904-07-012@{1. 7. 1904}|)be}\mylabel{h}  \normalsize

\doendnotes{C}
\bigskip
\vfill

\clearpage

\footnotesize

\lohead{\textsc{register}}

% Definiere theindex-Environment komplett neu ohne reledmac
\makeatletter
\renewenvironment{theindex}{%
  \section*{\indexname}%
  \setlength{\parindent}{0pt}%
  \setlength{\parskip}{0pt plus 0.3pt}%
  \let\item\@idxitem
}{%
  \clearpage
}
\makeatother

\IfFileExists{\jobname-pw.ind}{\input{\jobname-pw.ind}}{}

\end{document}

      