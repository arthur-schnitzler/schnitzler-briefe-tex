%% latex-korrekturansicht-vorspann.tex
%% Vorspann für die Korrekturansicht.
%% Lädt die gemeinsame Datei latex-vorspann.tex mit gesetztem Schalter.

\newif\ifkorrekturansicht
\korrekturansichttrue

\input{../tex-inputs/latex-vorspann}


               \section[Georg Brandes an Arthur Schnitzler, 16. 3. 1898]{ Georg Brandes an Arthur Schnitzler, 16. 3. 1898}\nopagebreak\mylabel{v}\rehead{ }\normalsize\beginnumbering\briefempfaengerindex{Schnitzler, Arthur@\textsc{Schnitzler, Arthur}!zzzBrandes, Georg@\emph{von Georg Brandes}!1898-03-161@{16. 3. 1898}|(be} \toendnotes[C]{\smallbreak\pagebreak[2]} \Standort{CUL, Schnitzler, B 17.}
\physDesc{Brief, 1 Blatt, 4 Seiten
\newline{}Handschrift: schwarze Tinte, lateinische Kurrent\newline{}Ordnung: von unbekannter Hand nummeriert: »9« }\buchAbdrucke{\weitereDrucke{Georg Brandes, Arthur Schnitzler: \emph{Ein Briefwechsel}. Hg. Kurt Bergel. Bern: \emph{Francke} 1956, S. 66–67.} }\toendnotes[C]{\smallbreak}\pstart
           \raggedleft{}{\pb}\textcolor{pink}{Taormina (Sicilia) Hotel Timeo}{}\ledrightnote{\textcolor{pink}{Hotel Timeo}}{\\}16 März 98\pend
           \pstart\center{}Liebster Dr. Schnitzler\pend\pstart
           Ich fühle mich Ihnen und Herrn Dr. \textcolor{blue}{Beer-Hofmann}{}\ledrightnote{\textcolor{blue}{Richard Beer-Hofmann}} gegenüber wirklich wie ein Schweinehund. Ich nehme
                    in \textcolor{pink}{Wien}{}\ledrightnote{\textcolor{pink}{Wien}} Ihre Zeit in Anspruch; Sie sehen
                    täglich, wie es mir geht, Sie beide sind die letzten, die mich in \textcolor{pink}{Wien}{}\ledrightnote{\textcolor{pink}{Wien}} besuchen; ich reise fort und lasse nicht von
                    mir hören, danke Ihnen nicht einmal. Nur hoffe ich dass Sie einen stummen Gruss
                    von mir bekommen haben, da ich meine \textcolor{blue}{Tochter}{}\ledrightnote{→\textcolor{blue}{Edith Philipp}} bat, Ihnen einige alten Drucksachen zu senden.\pend
           \pstart
           Der Anfang meiner Reise in \textcolor{pink}{Italien}{}\ledrightnote{\textcolor{pink}{Italien}} war durch die Krankheit meiner \textcolor{blue}{Mutter}{}\ledrightnote{→\textcolor{blue}{Emilie Brandes}}
               sehr {\pb}verdüstert. Indes, sie lebt. Sie liegt zwar noch zu Bette aber es geht ihr
                    besser; sie kann täglich eine Stunde aus dem Bette sein.\pend
           \pstart
           Ich las irgendwo, in \textcolor{pink}{Florenz}{}\ledrightnote{\textcolor{pink}{Florenz}} glaub’ ich,
                    etwas über die Aufführung Ihres \textcolor{green}{Stückes}{}\ledrightnote{→\textcolor{green}{Freiwild. Schauspiel in 3 Akten}} in
                    einem deutschen Blatt, konnte aber nicht daraus klug werden. Sind Sie mit dem
                    Resultat zufrieden gewesen?\pend
           \pstart
           Ich ging von \textcolor{pink}{Florenz}{}\ledrightnote{\textcolor{pink}{Florenz}} nach \textcolor{pink}{Rom}{}\ledrightnote{\textcolor{pink}{Rom}}, wo die Studenten der \textcolor{pink}{philosophischen Fakultät}{}\ledrightnote{\textcolor{pink}{Universität La Sapienza}} artig genug waren mich mit einer sehr
                    netten Adresse zu begrüssen. Es war dort bald kalt, bald warm, doch trocken,
                    aber in \textcolor{pink}{Neapel}{}\ledrightnote{\textcolor{pink}{Neapel}} wurde ich von argem
                    Regenwetter verfolgt. Dort sah ich curios genug die ganze Aristokratie, da man
                    mich {\pb}viel in diesen Kreisen
                    einlud, obwohl ich nicht einmal Empfehlungsschreiben hatte.\pend
           \pstart
           Hier in diesem gesegneten und verhungernden Land hatte ich wieder fast immer
                    Regen. Ich bin schon mehr als 14 Tage hier. Aber wenn es bisweilen schön ist,
                    dann ist es hier am Fusse des Etna in der
                    starken herrlichen Wärme am Ufer des Meeres wahrlich sehr schön. Hier hat jeder
                    Fleck ihre Geschichte, hier haben Araber und Normannen usw. Spuren hinterlassen,
                    hier hat \textcolor{blue}{Heine}{}\ledrightnote{\textcolor{blue}{Heinrich Heine}}’s \textcolor{blue}{Platen}{}\ledrightnote{\textcolor{blue}{August von Platen}} gelebt, und noch giebt es hier in \textcolor{pink}{Taormina}{}\ledrightnote{\textcolor{pink}{Taormina}} nicht wenige deutsche Herren mit seinen
                    Leidenschaften.\pend
           \pstart
           Ich lebe hier gesellig am Tage, einsam {\pb}von 5 Uhr ab, lese und
                    schreibe viel, oder so viel ich vermag, denn alt und dumm bin ich.\pend
           \pstart
           Ich danke Herrn \textcolor{blue}{Beer Hofmann}{}\ledrightnote{\textcolor{blue}{Richard Beer-Hofmann}} viel für das \textcolor{green}{Buch}{}\ledrightnote{→\textcolor{green}{Lust}} von \textcolor{blue}{d’Annunzio}{}\ledrightnote{\textcolor{blue}{Gabriele D’Annunzio}}, das ich zwischen \textcolor{pink}{Wien}{}\ledrightnote{\textcolor{pink}{Wien}} und \textcolor{pink}{Florenz}{}\ledrightnote{\textcolor{pink}{Florenz}} las; es
                    war mir eigentlich zuwider, und ich mag auch das Uebrige von \textcolor{blue}{d’Annunzio}{}\ledrightnote{\textcolor{blue}{Gabriele D’Annunzio}} nur wenig. Uebrigens war die
                    Uebersetzung sehr stark gekürzt, als ich sie mit dem Original verglich. Grüssen
                    Sie mir sehr herzlich den weisen \textcolor{blue}{Mann}{}\ledrightnote{→\textcolor{blue}{Richard Beer-Hofmann}},
                        \textcolor{pink}{Wollzeile 15}{}\ledrightnote{\textcolor{pink}{Wollzeile}}, I\pend
           \pstart
           Ich bitte Sie mich auch Ihrer Frau \textcolor{blue}{Mutter}{}\ledrightnote{→\textcolor{blue}{Louise Schnitzler}}
                    bestens zu empfehlen.\pend
           \pstart
           Ihr ergebener{\\[\baselineskip]}\spacefill\mbox{Georg Brandes}\pend
           \leftskip=0em{}\endnumbering\briefempfaengerindex{Schnitzler, Arthur@\textsc{Schnitzler, Arthur}!zzzBrandes, Georg@\emph{von Georg Brandes}!1898-03-161@{16. 3. 1898}|)be}\mylabel{h}  \normalsize

\doendnotes{C}
\bigskip
\vfill

\clearpage

\footnotesize

\lohead{\textsc{register}}

% Definiere theindex-Environment komplett neu ohne reledmac
\makeatletter
\renewenvironment{theindex}{%
  \section*{\indexname}%
  \setlength{\parindent}{0pt}%
  \setlength{\parskip}{0pt plus 0.3pt}%
  \let\item\@idxitem
}{%
  \clearpage
}
\makeatother

\IfFileExists{\jobname-pw.ind}{\input{\jobname-pw.ind}}{}

\end{document}

      