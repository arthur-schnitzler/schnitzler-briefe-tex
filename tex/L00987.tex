%% latex-korrekturansicht-vorspann.tex
%% Vorspann für die Korrekturansicht.
%% Lädt die gemeinsame Datei latex-vorspann.tex mit gesetztem Schalter.

\newif\ifkorrekturansicht
\korrekturansichttrue

\input{../tex-inputs/latex-vorspann}


               \section[Arthur Schnitzler an Richard Beer-Hofmann, 3. 10. 1899]{ Arthur Schnitzler an Richard Beer-Hofmann, 3. 10. 1899}\nopagebreak\mylabel{v}\rehead{ }\normalsize\beginnumbering\briefempfaengerindex{Beer-Hofmann, Richard@\textsc{Beer-Hofmann, Richard}!zzzSchnitzler, Arthur@\emph{von Arthur Schnitzler}!1899-10-031@{3. 10. 1899}|(be} \toendnotes[C]{\smallbreak\pagebreak[2]} \Standort{CUL, Schnitzler, B 8.}
\physDesc{Klappkarte
\newline{}Handschrift Arthur Schnitzler: Bleistift, deutsche Kurrent\newline{}Handschrift  : blaue Tinte, lateinische Kurrent (\noindent{}Speisenfolge)\newline{}Versand: 1) Stempel: »\nobreak{}\oindex{Wiesbaden@\textbf{Wiesbaden}, \emph{Besiedelter Ort (A.BSO)}|pwk}Wiesbaden, 3. 10. 99, 3–4N\nobreak{}«.  2) Stempel: »\nobreak{}6. {[}10.{]} 99, \oindex{Sankt Michael@\textbf{Sankt Michael}, \emph{Bezirk (A.BZK)}|pwk}St. Michael Eppan\nobreak{}«. \newline{}Ordnung: mit Bleistift von unbekannter Hand datiert: »3. 10.« }\toendnotes[C]{\smallbreak}\pstart{}{\pb}\textsc{Dr. Richard Beer-Hofmann}\pend{}\pstart{}\textcolor{pink}{\textsc{St. Michael in Eppan}}{}\ledrightnote{\textcolor{pink}{Sankt Michael}}\pend{}{\bigskip}\pstart
           \noindent{}\centering{}\textcolor{gray}{\textbf{{\pb}\textcolor{pink}{Wiesbaden}{}\ledrightnote{\textcolor{pink}{Wiesbaden}}. Blick aus dem \textcolor{pink}{Hotel du Parc et Bristol}{}\ledrightnote{\textcolor{pink}{Hôtel du Parc {\kaufmannsund} Bristol}}}}\pend
           \pstart
           {\pb}Heute Abd fahr ich nach \textcolor{pink}{Berlin}{}\ledrightnote{\textcolor{pink}{Berlin}}. – Will mein \textcolor{green}{Stück}{}\ledrightnote{→\textcolor{green}{Der Schleier der Beatrice. Schauspiel in fünf Akten}} nochmals umarbeiten. – Bleibe in \textcolor{pink}{Berlin}{}\ledrightnote{\textcolor{pink}{Berlin}} wahrſcheinlich bis Sonntag. Wohne dort \textcolor{pink}{\textsc{Hotel Savoy}}{}\ledrightnote{\textcolor{pink}{Hotel Savoy}}. Viele herzl Grüße. Ich freue mich über Ihre 420 \textcolor{green}{Verſe}{}\ledrightnote{\textcolor{green}{Der Graf von Charolais. Ein Trauerspiel}}.\pend
           \pstart \spacefill\mbox{A.}\pend{}\pstart
           \noindent{}{\pb}gleichfalls hiſtorisches\pend
           \pstart
           \centering{}\textcolor{gray}{\textbf{Menu.}}{ }{[}hs. ?? [Schreibkraft der Menükarte 3.10.1899]:{]} du 3. Oct. 1899\pend
           \pstart
           \noindent{}\centering{}Consommé pâtés d’Italie\pend
           \pstart
           \noindent{}\centering{}\textcolor{gray}{Canape à la meuni}ère – Pommes\pend
           \pstart
           \noindent{}\centering{}Roastbeef garni\pend
           \pstart
           \noindent{}\centering{}Haricots verts – Hareng\pend
           \pstart
           \noindent{}\centering{}Chapon rôti – Comp. – Salade\pend
           \pstart
           \noindent{}\centering{}Bavarois à la romaine\pend
           \pstart
           \noindent{}\centering{}Fruits – Dessert.\pend
           \pstart
           \noindent{}\textcolor{gray}{\textbf{\textsc{\textcolor{pink}{Hotel du Parc et Bristol}{}\ledrightnote{\textcolor{pink}{Hôtel du Parc {\kaufmannsund} Bristol}}}}}\pend
           \endnumbering\briefempfaengerindex{Beer-Hofmann, Richard@\textsc{Beer-Hofmann, Richard}!zzzSchnitzler, Arthur@\emph{von Arthur Schnitzler}!1899-10-031@{3. 10. 1899}|)be}\mylabel{h}  \normalsize

\doendnotes{C}
\bigskip
\vfill

\clearpage

\footnotesize

\lohead{\textsc{register}}

% Definiere theindex-Environment komplett neu ohne reledmac
\makeatletter
\renewenvironment{theindex}{%
  \section*{\indexname}%
  \setlength{\parindent}{0pt}%
  \setlength{\parskip}{0pt plus 0.3pt}%
  \let\item\@idxitem
}{%
  \clearpage
}
\makeatother

\IfFileExists{\jobname-pw.ind}{\input{\jobname-pw.ind}}{}

\end{document}

      