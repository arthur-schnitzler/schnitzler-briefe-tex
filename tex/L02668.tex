%% latex-korrekturansicht-vorspann.tex
%% Vorspann für die Korrekturansicht.
%% Lädt die gemeinsame Datei latex-vorspann.tex mit gesetztem Schalter.

\newif\ifkorrekturansicht
\korrekturansichttrue

\input{../tex-inputs/latex-vorspann}


               \section[Paul Goldmann an Arthur Schnitzler, 4. 8. 1891]{ Paul Goldmann an Arthur Schnitzler, 4. 8. 1891}\nopagebreak\mylabel{v}\rehead{ }\normalsize\beginnumbering\briefempfaengerindex{Schnitzler, Arthur@\textsc{Schnitzler, Arthur}!zzzGoldmann, Paul@\emph{von Paul Goldmann}!1891-08-041@{4. 8. 1891}|(be} \toendnotes[C]{\smallbreak\pagebreak[2]} \Standort{DLA, A:Schnitzler, HS.NZ85.1.3162.}
\physDesc{Brief, 3 Blätter, 12 Seiten
\newline{}Handschrift: schwarze Tinte, deutsche Kurrent
\newline{}Schnitzler: mit Bleistift das Jahr »1891« vermerkt }\toendnotes[C]{\smallbreak}\pstart
           \centering{}{\pb}\textcolor{pink}{Brüſſel}{}\ledrightnote{\textcolor{pink}{Brüssel}}, 4. Auguſt.\pend
           \pstart\center{}Mein lieber Arthur!\pend\pstart
           Der Himmel allein weiß, wieviele Briefe ich Dir inzwiſchen geſchrieben habe. Das
               Unglück wollte nur, daß ich nie dazu kam, einen davon auf’s Papier zu bringen. Daß
                  \strikeout{ich} ſeit meinem Fortgang aus \textcolor{pink}{Wien}{}\ledrightnote{\textcolor{pink}{Wien}} auch nicht ein Tag vorübergezogen iſt, an dem ich Deiner
               nicht gedacht, iſt ebenſo buchſtäblich wahr, als es phraſenhaft erſcheint. Das Maß
               meiner Berufsarbeit iſt mehr als menſchlich; aber ich \strikeout{\textcolor{gray}{×}} freue mich deſſen und ſuche eher zu mehren als zu mindern; ich bedarf wahrer
                  Arbeits\label{K_L02668-10v}\edtext{b\substVorne{}\textsuperscript{\textcolor{gray}{ac}}\substDazwischen{}ac\substHinten{}hanale}{\lemma{\textnormal{\emph{bacachanale}}}\Cendnote{\textnormal{Bacchusfeste}}}\label{K_L02668-10h}, um
               an mich ſelbſt zu vergeſſen, was mir trotzdem nicht völlig gelingt. I\substVorne{}\textsuperscript{m}\substDazwischen{}n\substHinten{} Familien- und Geſchäftsangelegenheiten habe ich vor \substVorne{}\textsuperscript{acht}\substDazwischen{}14\substHinten{} Tagen nach \textcolor{pink}{Frankfurt}{}\ledrightnote{\textcolor{pink}{Frankfurt am Main}} reiſen müſſen; und
               da mir der \textcolor{blue}{Chef}{}\ledrightnote{→\textcolor{blue}{Leopold Sonnemann}} des \textcolor{brown}{Blattes}{}\ledrightnote{→\textcolor{brown}{Frankfurter Zeitung}} die Aufgabe zuertheilte,
               über die dortige \label{K_L02668-1v}\edtext{elektriſche
                  Ausſtellung}{\lemma{\textnormal{\emph{elektriſche
                  Ausſtellung}}}\Cendnote{\textnormal{Die \emph{\textcolor{brown}{Internationale Elektrotechnische Ausstellung}} fand von
                     16. 5. 1891 bis 19. 10. 1891 in \textcolor{pink}{Frankfurt am Main} statt. \textcolor{blue}{Goldmann} schrieb darüber: \emph{\textcolor{green}{XXXX}}. In: \emph{\textcolor{green}{Frankfurter Zeitung}}, Jg. ZZ, Nr. ZZZZ, ZZ. ZZ. 1891,
                     S. ZZZZ.}}}\label{K_L02668-1h} zu ſchreiben – ſtell’ Dir vor! – gingen mit dieſer
               widerlichen Arbeit auch noch die acht Tage nach der Rückkehr zum Teufel. Heut iſt ein Tag nach einer auf Poſten durchwachten Nacht
               (die \textcolor{blue}{Königin}{}\ledrightnote{→\textcolor{blue}{Marie Henriette von Österreich}} iſt erkrankt
               und man erwartete ſtündlich die \label{K_L02668-77v}\edtext{Todesnachricht}{\lemma{\textnormal{\emph{Todesnachricht}}}\Cendnote{\textnormal{\textcolor{blue}{Marie Henriette von Österreich}, die Ehefrau
                  von \textcolor{blue}{Leopold II. von Belgien}, wurde zwar von
                  der Presse kurzfristig in Lebensgefahr geglaubt, war aber nur kurz indisponiert
                  und lebte bis zum Jahr 1902.}}}\label{K_L02668-77h}). Zum Schlafen bin ich zu nervös,
               zum Arbeiten zu müde, {\pb}und nachdem ich mich ſoeben
               eine Stunde in tauſend qualvollen Gedanken auf dem Ruhebett gewälzt, flüchte ich mich
               vor meinen Dämonen in Deine Nähe, die ſie ſo oft gebannt hat. Und ſo wird denn der
               längſt geſchriebene Brief nunmehr wirklich geſchrieben{\dotsfive}\pend
           \pstart
           Keine Spur von Wohlbefinden hier, mein lieber Arthur! Äußerlich freilich ſieht ſich
               die Sache recht gut an. Ich habe Erfolg und Zufriedenheit von meinen Vorgeſetzten
               her; und ich bin in guten Beziehungen zur officiellen Welt, zu Miniſtern, Deputirten
               und allerlei ſonſtigem hohen Gethier. Aber es iſt klar, daß \strikeout{\textcolor{gray}{d}} es nicht genügt, um de\substVorne{}\textsuperscript{m}\substDazwischen{}n\substHinten{} Wärmebedarf eines weichen Herzens herzuſtellen, wenn man von
               Miniſterpräſidenten empfangen wird. Alles Übrige aber, was ich von der \textcolor{pink}{Brüſſel}{}\ledrightnote{\textcolor{pink}{Brüssel}}er Bevölkerung kennen gelernt, iſt eiskalt
               und abweiſend dem Fremden, zumal dem Deutſchen gegenüber. Die Leute haben zwar \strikeout{Alle} insgeſammt vollendete Formen; aber ich habe in
               meinem Leben nicht ſo erkannt, was die Höflichkeit für eine unbeſiegliche {\pb}Waffe iſt \uline{gegen} den,
               demgegenüber man ſie anwendet. Die Leute hier verſtehen die Kunſt, ſich Einem mit
               Händeſchütteln vom \strikeout{Leibe} Leibe zu halten. Das gilt
               ganz im Speciellen von den journaliſtiſchen Collegen. Es ſind zwar vollendete
               Gentlemen im Äußern – wie Tag und Nacht gegenüber dem \textcolor{pink}{Wien}{}\ledrightnote{\textcolor{pink}{Wien}}er Geſindel – aber falſch, unverläßlich, verlogen ſind ſie zu gleicher
               Zeit. Ich bin demgemäß nach wie vor völlig iſolirt. Ein paar äußerliche Beziehungen
               dienen eher dazu, mir meine Einſamkeit noch fühlbarer zu machen, als ſie
               abzuſchwächen. Meine Abende verbringe ich meiſt allein, meine Sonntage gleichfalls –
               in der Regel trifft man mich zu jeder Tageszeit an meinem Schreibtiſch. Deine Frage
               nach »intereſſanten Frauen« übergehe ich mit ſtiller Heiterkeit. Straßendirnen, die,
               weil ſie kein Anderer mag, mit dem häßlichen und \label{K_L02668-13v}\edtext{ungeſchlachten}{\lemma{\textnormal{\emph{ungeſchlachten}}}\Cendnote{\textnormal{massig, klobig}}}\label{K_L02668-13h} Fremden gehen müſſen und die ihn dafür ausplündern, wie ein
               Heuſchreckenſchwarm, der einen Acker überfällt – das iſt meine {\pb}weibliche Welt. Liebelos und freudlos – das iſt die
               Firma, unter der mein Leben ſein Geſchäft fortführt. Ich ſehne mich namenlos nach \textcolor{pink}{Wien}{}\ledrightnote{\textcolor{pink}{Wien}} und nach Dir und dem andern, was mir dort
               theuer iſt, zurück – namenlos! Und ich habe eine \textcolor{gray}{Z}eit der heftigen
               Empörung gegen das Schickſal gehabt und an den Stäben des Käfigs gerüttelt. Ich habe
               in \textcolor{pink}{Frankfurt}{}\ledrightnote{\textcolor{pink}{Frankfurt am Main}} erklärt, daß ich unter allen
               Umſtänden nach \textcolor{pink}{Wien}{}\ledrightnote{\textcolor{pink}{Wien}} zurück will. Aber keine
               Ausſicht. Unſer \textcolor{blue}{Chefredacteur}{}\ledrightnote{→\textcolor{blue}{Leopold Sonnemann}} verachtet \textcolor{pink}{Wien}{}\ledrightnote{\textcolor{pink}{Wien}} und \textcolor{pink}{Öſterreich}{}\ledrightnote{\textcolor{pink}{Österreich}} aufs Tiefſte und hält es nicht der
               Mühe für werth, dort einen anſtändigen Correſpondenten-Poſten zu etabliren. Und dann
               kam mein \textcolor{blue}{Onkel}{}\ledrightnote{\textcolor{blue}{Fedor Mamroth}} mit ſeiner harten Pflichtlogik:
               man iſt in \textcolor{pink}{Wien}{}\ledrightnote{\textcolor{pink}{Wien}} glücklich, zugegeben! aber der
               Mann, der für ſein und ſeiner Familie Fortkommen ſorgen ſoll, hat nicht das Recht,
               glücklich zu ſein. {\dots} Dabei ſällt mir etwas ein: der \strikeout{\textcolor{gray}{W}}{ }\label{K_L02668-11v}\edtext{\textcolor{pink}{Pariſ}{}\ledrightnote{\textcolor{pink}{Paris}}er Correſpondentenpoſten}{\lemma{\textnormal{\emph{Pariſer Correſpondentenpoſten}}}\Cendnote{\textnormal{Vgl. dazu den Brief, den \textcolor{blue}{Hermann Bahr} am 7. 8. 1891 an \textcolor{blue}{Hugo von Hofmannsthal} schrieb: »Sehr eilig: haben Sie Bekannte in der \uline{Direktion} der \textcolor{brown}{Neuen Freien
                           Presse}? Wissen Sie überhaupt, wer von den \textcolor{blue}{Herausgeber}n eigentlich
                        die geschäftlichen Entscheidungen trifft? Können Sie mir etwa eine
                        Empfehlung an irgendsowen verschaffen?{ / }Es handelt sich nemlich darum, daß \textcolor{blue}{Wilhelm
                           Singer}{ }\textcolor{blue}{Herausgeber} des \textcolor{brown}{Wiener Tagblatt} geworden ist, und daß es
                        famos wäre, wenn ich statt seiner \textcolor{pink}{Paris}er Correspondent der \textcolor{brown}{Neuen
                           Freien} würde. Die Politik ist mir so wurst, daß ich sicherlich
                        leicht zum Wohlgefallen der ganzen \textcolor{brown}{Redaktion} schreiben könnte, und von Literatur u.
                        Malerei verstehe ich vielleicht ebensoviel als Herr \textcolor{blue}{Singer}.« (\emph{Briefwechsel 1891–1934}. Hg. Elsbeth Dangel-Pelloquin.
                     Göttingen: \emph{Wallstein} 2013, S. 10). Die Stelle
                  wurde mit \textcolor{blue}{Theodor Herzl} besetzt.}}}\label{K_L02668-11h} der
                  »\textcolor{brown}{Neuen Freien Preſſe}{}\ledrightnote{\textcolor{brown}{Neue Freie Presse}}« iſt durch \textsc{\textcolor{blue}{Singer}{}\ledrightnote{\textcolor{blue}{Wilhelm Singer}}}’s Berufung nach {\pb}\textcolor{pink}{Wien}{}\ledrightnote{\textcolor{pink}{Wien}} freigeworden; man hat es mir hier nahegelegt,
               mich darum zu bewerben; aber ich habe es nicht gethan. Wenn Du aber am Ende irgendwie
               – ohne daß natürlich Jemand eine Ahnung von meiner Bewerbung haben dürfte! – in
               dieſer Richtung etwas wirken könnteſt, ſo wäre ich wohl recht einverſtanden; das wäre
               immerhin ein Schritt in der Richtung nach \textcolor{pink}{Wien}{}\ledrightnote{\textcolor{pink}{Wien}}.
               Aber das iſt nur ſo eine Idee! Fällt Dir nicht gleich etwas Wirkſames
                  diesbezüglich\strikeout{\textcolor{gray}{er}} ein, ſo gib’ Dich, bitte, nicht weiter damit ab! {\dotsfour}
               Dein lieber Brief, der meine Arbeiten lobt, hat mich unendlich gefreut. Ich danke Dir
               für die Minute des Stolzes, die Du mir damit bereitet. Du weißt, ich rechne Dich zu
               meinen ſtrengſten und unfehlbarſten Richtern. Habe ich wirklich etwas Gutes
               geſchrieben, ſo war es kein Kunſtſtück. Jene Tage in \textcolor{pink}{Holland}{}\ledrightnote{\textcolor{pink}{Niederlande}} waren von unvergeßlicher Schönheit und brachten eine Fülle von
               Eindrücken, die tief, \strikeout{\textcolor{gray}{aber} tief} aber tief ſich in’s Herz gruben. Ich glaube, in
               dieſen Tagen iſt mir zum erſten Mal das Licht darüber aufgegangen, was die Malerei
               iſt. Die Wärme freilich, mit der Du ſchreibſt, iſt \strikeout{\textcolor{gray}{fie}} viel mehr {\pb}ein Compliment für Dich als für
               mich. Treue Herzen wie das Deinige ſind ſolche, die in der Welt wohl noch hie und da
               vorhanden ſein mögen, die man aber nur einmal findet{\dotsfour} Und
               dann das zweite Brieflein! Am Morgen um vier Uhr kam ich \strikeout{\textcolor{gray}{au}s} von \textcolor{pink}{Frankfurt}{}\ledrightnote{\textcolor{pink}{Frankfurt am Main}}
               heim – mit fieberndem Kopfe und brennenden Augen, nach einer ſchlafloſen Nachtfahrt.
               Und in dem grauen Morgenzwielicht, beim Schein einer blinzelnden Kerze las ich Deinen
               Brief. Mein Herz war eiskalt vor Verlaſſenheit und ſchrie förmlich vor Sehnſucht, als
               aus dieſen mit Bleiſtift gekritzelten Zeilen die ſüße Viſion des \textcolor{pink}{Wien}{}\ledrightnote{\textcolor{pink}{Wien}}er Sommerabends mit Frauen- und Blumenduft aufſtieg. Es war
               vielleicht ein vom Champagner geſchaffener Einfall, der dieſen Brief geſchrieben.
               Aber in dieſem troſtloſen Morgen, in dieſem Zimmer eines Verbannten wurde daraus eine
               Offenbarung von Freundestreue und holder Frauengüte. Küſſe die kleine \textcolor{blue}{Goldelſe}{}\ledrightnote{→\textcolor{blue}{Else Singer}} für
               mich auf Mund und Augen! {\dots}\pend
           \pstart
           {\pb}Und nun zu Dir, mein lieber Arthur! Von ganzem
               Herzen habe ich mich über den \label{K_L02668-3v}\edtext{im
               Freundeskreiſe errungenen Erfolg}{\lemma{\textnormal{\emph{im … Erfolg}}}\Cendnote{\textnormal{Am 25. 6. 1891 hatte \textcolor{blue}{Schnitzler} mehreren Freunden \emph{\textcolor{green}{Das Märchen}} vorgelesen und eine positive Aufnahme im \emph{\textcolor{green}{Tagebuch}} festgehalten.}}}\label{K_L02668-3h} Deines \textcolor{green}{Stückes}{}\ledrightnote{→\textcolor{green}{Das Märchen. Schauspiel in drei Aufzügen}} gefreut. Dein letzter
               längerer Brief, in dem Du mir das mittheilteſt, ſchien mir auch die ſchönſte Frucht
               dieſes Erfolges bereits zu enthalten: nämlich Luſt zum Produciren. Dabei fällt mir
               ein, daß mir mein \textcolor{blue}{Onkel}{}\ledrightnote{→\textcolor{blue}{Fedor Mamroth}}
               erzählte, Du habeſt ihm eine \textcolor{green}{Geſchichte}{}\ledrightnote{→\textcolor{green}{Die drei Elixire}} von »ſeltener Schönheit« (wirklich!) \label{K_L02668-4v}\edtext{geſchickt}{\lemma{\textnormal{\emph{geſchickt}}}\Cendnote{\textnormal{siehe Fedor Mamroth an Arthur Schnitzler, 21. 6. 1891}}}\label{K_L02668-4h}, er habe ſie aber leider aus Sittlichkeits-Gründen nicht veröffentlichen
               können. \strikeout{Du} Ich habe ſerner während meines \textcolor{pink}{Frankfurt}{}\ledrightnote{\textcolor{pink}{Frankfurt am Main}}er Aufenthalts Gelegenheit genommen, mit
               dem \label{K_L02668-5v}\edtext{\textsc{\textcolor{blue}{spiritus rector}{}\ledrightnote{→\textcolor{blue}{Karl Schönfeld}}}}{\lemma{\textnormal{\emph{spiritus rector}}}\Cendnote{\textnormal{lateinisch: geistiger Leiter}}}\label{K_L02668-5h} des
                  \textcolor{brown}{Frankfurter Theater}{}\ledrightnote{\textcolor{brown}{Frankfurter Stadt-Theater}}s, Herrn \textsc{\textcolor{blue}{Schönfeld}{}\ledrightnote{\textcolor{blue}{Karl Schönfeld}}}, von Dir zu ſprechen. Ich habe Dich, diplomatiſch, als einen Mann geſchildert,
               der die herrlichſten Werke ſchafft, um nichts in der Welt aber dazu zu bringen iſt,
               dieſelben herauszugeben, ſo daß er ganz begierig wurde, etwas von Dir zu ſehen.
               Willſt Du ihm etwas \label{K_L02668-6v}\edtext{ſchicken}{\lemma{\textnormal{\emph{ſchicken}}}\Cendnote{\textnormal{nicht bekannt}}}\label{K_L02668-6h}, ſo biſt Du
               eingeführt; freilich iſt der genannte \textcolor{blue}{Herr}{}\ledrightnote{→\textcolor{blue}{Karl Schönfeld}} ein jämmerlicher {\pb}\textcolor{blue}{Banauſe}{}\ledrightnote{→\textcolor{blue}{Karl Schönfeld}}. An \label{K_L02668-7v}\edtext{\textsc{\textcolor{blue}{Burckhard}{}\ledrightnote{\textcolor{blue}{Max Eugen Burckhard}}}}{\lemma{\textnormal{\emph{Burckhard}}}\Cendnote{\textnormal{Dieser leitete seit dem Vorjahr das \emph{\textcolor{brown}{Burgtheater}} in \textcolor{pink}{Wien}; \textcolor{blue}{Schnitzler} hatte sich längst
                  an ihn gewandt gehabt und ihm \emph{\textcolor{green}{Alkandi’s Lied}}
                  geschickt (vgl. Arthur Schnitzler an Max Burckhard, [20.] 5. 1891) und auch schon
                  eine freundliche Ablehnung erhalten (vgl. Max Burckhard an Arthur Schnitzler, 14. 7. 1891). }}}\label{K_L02668-7h} aber ſolltest Du Dich abſolut wenden – noch nicht mit
               dem großen \textcolor{green}{Drama}{}\ledrightnote{→\textcolor{green}{Das Märchen. Schauspiel in drei Aufzügen}}, ſondern vorerſt mit dem \textsc{\textcolor{green}{Alkandi}{}\ledrightnote{\textcolor{green}{Alkandi’s Lied}}}\textcolor{gray}{!} Willſt Du, ſo ſchreibe ich von hier aus an ihn und erbitte mir
               als einzige Gefälligkeit für die erwieſenen Dienſte, daß er Dir ſeine Aufmerkſamkeit
               zuwendet; das kann er mir nicht abſchlagen. An meinen \textcolor{blue}{Onkel}{}\ledrightnote{\textcolor{blue}{Fedor Mamroth}} ſollteſt Du baldmöglichſt etwas wieder ſchicken; er
               wünſcht nichts Beſſeres, als Dich drucken zu können. Die \label{K_L02668-12v}\edtext{Novelle}{\lemma{\textnormal{\emph{Novelle}}}\Cendnote{\textnormal{Es
                  dürfte sich um \textcolor{blue}{Schnitzler}s Plan handeln,
                  gemeinsam mit Freunden unter dem Titel »Aus der Kaffeehausecke« eine
                  Novellensammlung zu verfassen, vgl. Arthur Schnitzler an Richard Beer-Hofmann, 6. 6. 1891.}}}\label{K_L02668-12h} möchte ich gar gern mit Dir ſchreiben; aber für’s Erſte habe ich keine
               Zeit; wenn Du alſo irgendeine Luſt haſt, ſie allein zu machen, ſo warte nicht mehr
               auf mich. Die Gründung der »\textcolor{brown}{Freien Bühne}{}\ledrightnote{\textcolor{brown}{»Freie Bühne« Verein für moderne Literatur}}« mit dem
               Streber \label{K_L02668-9v}\edtext{\textsc{\textcolor{blue}{Wengraf}{}\ledrightnote{\textcolor{blue}{Edmund Wengraf}}} an der Spitze}{\lemma{\textnormal{\emph{Wengraf an der Spitze}}}\Cendnote{\textnormal{Am 7. 7. 1891 fand die
                  Gründungssitzung von \emph{\textcolor{brown}{Freie Bühne, Verein für
                     moderne Literatur}} statt. Zum \textcolor{blue}{Obmann} wurde \textcolor{blue}{Friedrich
                     Michael Fels} gewählt, \textcolor{blue}{Stellvertreter} wurden \textcolor{blue}{Edmund
                     Wengraf} und \textcolor{blue}{Hermann Fürst}. \textcolor{blue}{Schnitzler} wurde \textcolor{blue}{Ausschuss-Mitglied}.}}}\label{K_L02668-9h} mißfällt mir
               durchaus; an die Stelle des Vicepräſidenten hätte Niemand Anderer gehört als Du; und
               wäre ich in \textcolor{pink}{Wien}{}\ledrightnote{\textcolor{pink}{Wien}} geweſen, ſo würde ich auch dafür
               geſorgt haben, daß die Sache {\pb}ſo gekommen wäre.
               Offen geſtanden – wie die Sache ſich jetzt ausnimmt, habe ich kein großes Zutrauen;
               es ſind zuviel kleine perſönliche Ehrgeize dabei, die befriedigt werden wollen, als
               daß für die Idee Platz wäre. Du weißt ja: ein kleiner Ehrgeiz iſt immer ſtärker als
               eine große Idee; und wenn die Zwei ſich verbinden, ſo wird die Letztere \substVorne{}\textsuperscript{\textcolor{gray}{×}\-\textcolor{gray}{×}\-\textcolor{gray}{×}\-\textcolor{gray}{×}\-\textcolor{gray}{×}\-\textcolor{gray}{×}\-\textcolor{gray}{×}\-\textcolor{gray}{×}}\substDazwischen{}ſtets\substHinten{} betrogen. Immerhin, wenn das \textcolor{brown}{Unternehmen}{}\ledrightnote{→\textcolor{brown}{»Freie Bühne« Verein für moderne Literatur}} wenigſtens Dir eine größere Publicität bringt,
               wenn es Dich der großen Menge zuführt, ſo bin ich’s zufrieden. Vor Allem aber
               ſchreibe, ſchreibe und ſchreibe und ſchaffe Vorrath für den Tag, da man kommen wird,
               Dich ſuchen. Den dritten \textcolor{green}{Act}{}\ledrightnote{→\textcolor{green}{Das Märchen. Schauspiel in drei Aufzügen}} möchte ich für mein Leben gern leſen. Aber es iſt Dir
               wohl zu umſtändlich, mir ihn über die hundert Meilen herüber zu ſchicken? Wenn \textsc{\label{K_L02668-33v}\edtext{\textcolor{blue}{Schwarzkopf}{}\ledrightnote{\textcolor{blue}{Gustav Schwarzkopf}}}{\lemma{\textnormal{\emph{Schwarzkopf}}}\Cendnote{\textnormal{Die überlieferte Korrespondenz setzt
                     später ein, es dürfte sich also um eine mündliche Aussage handeln, die \textcolor{blue}{Schnitzler} in seinem Brief wiedergab. Ein
                     Treffen von \textcolor{blue}{Schnitzler} und \textcolor{blue}{Schwarzkopf} ist in der Zeit nicht im \emph{\textcolor{green}{Tagebuch}} erwähnt.}}}\label{K_L02668-33h}} ſagt: zum Mindeſten eine \label{K_L02668-8v}\edtext{\textcolor{green}{literariſche Arbeit}{}\ledrightnote{→\textcolor{green}{Das Märchen. Schauspiel in drei Aufzügen}}}{\lemma{\textnormal{\emph{literariſche Arbeit}}}\Cendnote{\textnormal{Siehe A. S.: \emph{Tagebuch}, 25. 6. 1891}}}\label{K_L02668-8h}, ſo bin \uline{ich} damit \uline{nicht} zufrieden; ich ſtelle höhere Anſprüche an Dich; Du kannſt, wie ich
               weiß, und darum ſollſt Du {\pb}lebendige Dramen
               ſchreiben und keine Buch-Theaterſtücke. Ich pfeife auf den literariſchen Werth. In
               Dir ſteckt echtes Bühnenleben; und ſo lange Du das nicht voll aus Dir
               herausgeſchaffen haſt, ſo lange haſt Du kein Recht, ſtillzuſtehen und auszuruhen.
               Auch möchte ich mir die Sache an Deiner Stelle anderſeits nicht leicht machen durch
               die Erfindung der Dramen nach den neuen Geſetzen. Von \textsc{\textcolor{blue}{Sophokles}{}\ledrightnote{\textcolor{blue}{Sophokles}}} bis \textsc{\textcolor{blue}{Sardou}{}\ledrightnote{\textcolor{blue}{Victorien Sardou}}} gibt es nur eine Art der dramatiſchen Wirkung; und jede Wirkung die anders iſt,
               iſt eben keine dramatiſche. Folg’ mir, gehe den geraden, von den großen Meiſtern
               gezeigten Weg und ſuche keine neuen Pfade, die nur in die Irre führen; wenn irgend
               Einer auf dieſem Wege zum großen Erfolg zu gelangen die Kunſt hat – und auf all’
               dieſen Seitenwegen gibt es das nicht, den großen Erfolg – ſo biſt Du es. Alſo falle
               nicht in {\pb}die Verſuchungen des Guten, die vom Beſten
                  ableiten{\dotsfive}\pend
           \pstart
           Dein\strikeout{e} Gefühlsleben – ich bitte um einen kleinen Abriß
               davon. Beſonders über Deine Liebe (das banalſte Wort iſt doch hier das wenigſt
               verletzende). Wo iſt das \substVorne{}\textsuperscript{\textcolor{gray}{Mädel}}\substDazwischen{}\textcolor{blue}{Fräulein}{}\ledrightnote{→\textcolor{blue}{Marie Glümer}}\substHinten{} jetzt? Wo ſiehſt Du ſie und wie oft? Was macht die \label{K_L02668-44v}\edtext{Eiferſucht auf die Vergangenheit}{\lemma{\textnormal{\emph{Eiferſucht … Vergangenheit}}}\Cendnote{\textnormal{Dies das Thema von \textcolor{blue}{Schnitzler}s \emph{\textcolor{green}{Märchen}}, in dem er die
                  Schwierigkeiten thematisierte, die ein Mann empfand, wenn seine Partnerin bereits
                  zuvor in Beziehungen gewesen war.}}}\label{K_L02668-44h}? Und iſt – aber ganz ehrlich! – noch
               keine Abnahme der Leidenſchaft zu ſpüren? – Was macht \textsc{\label{K_L02668-99v}\edtext{\textcolor{blue}{Madame la Mondaine}{}\ledrightnote{→\textcolor{blue}{Olga Waissnix}}}{\lemma{\textnormal{\emph{Madame la Mondaine}}}\Cendnote{\textnormal{französisch: Frau von Welt. Hier
                     hantiert \textcolor{blue}{Goldmann} mit einer
                     Typologisierung der beiden aktuellen Liebesbeziehungen \textcolor{blue}{Schnitzler}s, wobei \textcolor{blue}{Marie Glümer} die Rolle »Fräulein/süßes Mädel« zufällt, \textcolor{blue}{Olga Waissnix} die der eleganten Frau der
                     Gesellschaft. Wenige Wochen später, Ende November 1891, griff \textcolor{blue}{Schnitzler} bei der Abfassung des \textcolor{green}{Dialog}s \emph{\textcolor{green}{Weihnachts-Einkäufe}} die Unterscheidung auf: »\so{Er:} Es ist ja nichts Beleidigendes – durchaus
                           nicht! – Ich bin ja auch ein Typus!{ / }\so{Sie:} Und was für einer denn?{ / }\so{Er:}{\dots} Leichtsinniger Melancholiker! { / }\so{Sie:}{\dotstwo} Und {\dotstwo} und
                           ich?{ / }\so{Er:} Sie? – ganz einfach: Mondaine! { / }\so{Sie:} So{\dots}!{\dotstwo} Und \so{sie}!?{ / }\so{Er:} Sie{\dotstwo}? Sie{\dotstwo}, das süße Mäd’l!{ / }\so{Sie:} Süß! Gleich »süß«? – Und ich – die
                           »Mondaine« schlechtweg –{ / }\so{Er:} Böse Mondaine – wenn Sie durchaus wollen
                              {\dots}« (\textcolor{blue}{Arthur Schnitzler}: \emph{\textcolor{green}{Weihnachts-Einkäufe}}. In: \emph{\textcolor{green}{Frankfurter Zeitung}}, Jg. 36, Nr. 358,
                           24. 12. 1891, S. 1–2.) In der Buchausgabe bekommen
                     die beiden Dialogisierenden Namen: »Anatol« und »Gabriele«. Letzterer ist eine
                     doppelte Chiffre für \textcolor{blue}{Olga Waissnix}.
                     Einerseits ist er der Name der weiblichen Protagonistin in \textcolor{blue}{Paul Heyse}s Novelle \emph{\textcolor{green}{Die
                        guten Kameraden}}, in der \textcolor{blue}{Olga} und
                        \textcolor{blue}{Schnitzler} ihre Beziehung präfiguriert
                     sahen. (Vgl. Martin Anton Müller: \emph{Reconstructing Arthur
                           Schnitzler’s Library: Literary and Biographical Sources for ›Die Frau des
                           Weisen‹}. In: \emph{Austrian Studies}, Bd. 27,
                           2019, S. 44–57, hier S. 51–57) Andererseits ist
                     »Gabriele« der Vorname von \textcolor{blue}{Olga}s \textcolor{blue}{Schwester}, die
                     zeitweise eine Botenfunktion in der Beziehung innehatte.}}}\label{K_L02668-99h}}?\pend
           \pstart
           Sag’ mir, liebſter Freund: kannſt Du deine \strikeout{So\textcolor{gray}{mm}} Sommerpläne nicht ſo entwerfen, daß Du auf ein – zwei Wochen an’s Meer kommſt?
               Iſt gar keine Möglichkeit vorhanden, daß ich Dich in \introOben{}den\introOben{}
               folgenden Monaten irgendwo \label{K_L02668-111v}\edtext{ſehen}{\lemma{\textnormal{\emph{ſehen}}}\Cendnote{\textnormal{1891 kam es zu keinem persönlichen Treffen zwischen \textcolor{blue}{Goldmann} und \textcolor{blue}{Schnitzler}. Sie begegneten sich erst am 17. 9. 1893 wieder persönlich.}}}\label{K_L02668-111h}
               kann?\pend
           \pstart
           Schreib’ mir ferner, mit wem Du jetzt verkehrſt, wo Du Deine Abende zubringſt, was
                  {\pb}die Freunde machen, wie es bei Dir zu Hauſe geht
               und was es ſonſt Neues gibt?\pend
           \pstart
           Ich danke Dir tauſendmal für all’ das Liebe, womit Du mich hier in meiner Einſamkeit
               erfreut haſt, und grüße Dich von ganzem Herzen\pend
           \pstart
           Dein \damage{\textcolor{gray}{treuer}}{\\[\baselineskip]}\spacefill\mbox{Paul Goldmann.}\pend
           \leftskip=0em{}\pstart
           \noindent{}Mit dem Franzöſiſchen geht es mir elend; ich mache abſolut keine Fortſchritte.\pend
           \pstart
           Empfiehl’ mich den Deinen, grüße mir \textsc{\textcolor{blue}{Kapper}{}\ledrightnote{\textcolor{blue}{Friedrich Kapper}}} und Deinen \textcolor{blue}{Bruder}{}\ledrightnote{→\textcolor{blue}{Julius Schnitzler}}.\pend
           \endnumbering\briefempfaengerindex{Schnitzler, Arthur@\textsc{Schnitzler, Arthur}!zzzGoldmann, Paul@\emph{von Paul Goldmann}!1891-08-041@{4. 8. 1891}|)be}\mylabel{h}  \normalsize

\doendnotes{C}
\bigskip
\vfill

\clearpage

\footnotesize

\lohead{\textsc{register}}

% Definiere theindex-Environment komplett neu ohne reledmac
\makeatletter
\renewenvironment{theindex}{%
  \section*{\indexname}%
  \setlength{\parindent}{0pt}%
  \setlength{\parskip}{0pt plus 0.3pt}%
  \let\item\@idxitem
}{%
  \clearpage
}
\makeatother

\IfFileExists{\jobname-pw.ind}{\input{\jobname-pw.ind}}{}

\end{document}

      