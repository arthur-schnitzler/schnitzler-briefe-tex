%% latex-korrekturansicht-vorspann.tex
%% Vorspann für die Korrekturansicht.
%% Lädt die gemeinsame Datei latex-vorspann.tex mit gesetztem Schalter.

\newif\ifkorrekturansicht
\korrekturansichttrue

\input{../tex-inputs/latex-vorspann}


               \section[Max Burckhard an Arthur Schnitzler, 30. 11. 1905]{ Max Burckhard an Arthur Schnitzler, 30. 11. 1905}\nopagebreak\mylabel{v}\rehead{ }\normalsize\beginnumbering\briefempfaengerindex{Schnitzler, Arthur@\textsc{Schnitzler, Arthur}!zzzBurckhard, Max Eugen@\emph{von Max Eugen Burckhard}!1905-11-301@{30. 11. 1905}|(be} \toendnotes[C]{\smallbreak\pagebreak[2]} \Standort{CUL, Schnitzler, B 20.}
\physDesc{Brief, 1 Blatt, 2 Seiten
\newline{}Handschrift: schwarze Tinte, deutsche Kurrent
\newline{}Schnitzler: mit Bleistift beschriftet: »B« und datiert: »1905?« \newline{}Ordnung: mit Bleistift von unbekannter Hand nummeriert: »15« }\toendnotes[C]{\smallbreak}\pstart
           \raggedleft{}{\pb}\textcolor{pink}{St. Gilgen}{}\ledrightnote{\textcolor{pink}{St. Gilgen}}{ }30/11 05\pend
           \pstart{}Sehr verehrter lieber Herr Doctor!\pend\pstart
           Herzlichſten Dank für das »\textcolor{green}{Zwischenſpiel}{}\ledrightnote{\textcolor{green}{Zwischenspiel. Komödie in drei Akten}}«, das
                    ich noch nicht gekannt hatte und das einen außerordentlich tiefen Eindruck auf
                    mich gemacht hat – beſonders dadurch vielleicht, daſs die eigenthümliche Sti{\geminationm}ung, {\pb}mit der es ſchon einſetzt, ſo außerordentlich feſtgehalten iſt bis zum letzten
                    Augenblick.\pend
           \pstart
           Auf baldiges Wiederſehen, denn jetzt geht der Sommer zur Neige.\pend
           \pstart
           Mit Handkuſs an Ihre verehrte \textcolor{blue}{Gattin}{}\ledrightnote{→\textcolor{blue}{Olga Schnitzler}} u herzlichſte Grüße\pend
           \pstart
           Ihr getreuer{\\[\baselineskip]}\spacefill\mbox{D\textsuperscript{r}Burckhard}\pend
           \leftskip=0em{}\pstart
           \noindent{}Ich gratuliere noch zum \label{K_L01568_1v}\edtext{\textcolor{pink}{Berlin}{}\ledrightnote{\textcolor{pink}{Berlin}}er Erfolg}{\lemma{\textnormal{\emph{Berliner Erfolg}}}\Cendnote{\textnormal{Am 25. 11. 1905 fand die Aufführung von
                                \emph{\textcolor{green}{Zwischenspiel}} am \textcolor{pink}{Deutschen Theater}
                      statt, etwas über einen Monat
                            nach der \textcolor{pink}{Wien}er Uraufführung.}}}\label{K_L01568_1h}\pend
           \endnumbering\briefempfaengerindex{Schnitzler, Arthur@\textsc{Schnitzler, Arthur}!zzzBurckhard, Max Eugen@\emph{von Max Eugen Burckhard}!1905-11-301@{30. 11. 1905}|)be}\mylabel{h}  \normalsize

\doendnotes{C}
\bigskip
\vfill

\clearpage

\footnotesize

\lohead{\textsc{register}}

% Definiere theindex-Environment komplett neu ohne reledmac
\makeatletter
\renewenvironment{theindex}{%
  \section*{\indexname}%
  \setlength{\parindent}{0pt}%
  \setlength{\parskip}{0pt plus 0.3pt}%
  \let\item\@idxitem
}{%
  \clearpage
}
\makeatother

\IfFileExists{\jobname-pw.ind}{\input{\jobname-pw.ind}}{}

\end{document}

      