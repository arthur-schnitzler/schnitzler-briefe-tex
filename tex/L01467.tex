%% latex-korrekturansicht-vorspann.tex
%% Vorspann für die Korrekturansicht.
%% Lädt die gemeinsame Datei latex-vorspann.tex mit gesetztem Schalter.

\newif\ifkorrekturansicht
\korrekturansichttrue

\input{../tex-inputs/latex-vorspann}


               \section[Arthur Schnitzler an Richard Beer-Hofmann, 8. 11. 1904]{ Arthur Schnitzler an Richard Beer-Hofmann, 8. 11. 1904}\nopagebreak\mylabel{v}\rehead{ }\normalsize\beginnumbering\briefempfaengerindex{Beer-Hofmann, Richard@\textsc{Beer-Hofmann, Richard}!zzzSchnitzler, Arthur@\emph{von Arthur Schnitzler}!1904-11-081@{8. 11. 1904}|(be} \toendnotes[C]{\smallbreak\pagebreak[2]} \Standort{YCGL, MSS 31.}
\physDesc{Kartenbrief
\newline{}Handschrift: schwarze Tinte, deutsche Kurrent\newline{}Versand: 1) Stempel: »\nobreak{}\textcolor{gray}{Wien}, 8. XI. 04, 6\nobreak{}«.  2) Stempel: »\nobreak{}\oindex{Rodaun@\textbf{Rodaun}, \emph{Teil eines besiedelten Ortes (A.BSOX)}|pwk}\textcolor{gray}{Ro}daun\nobreak{}«. 
\newline{}Beer-Hofmann: mit Tinte das Datum der Beantwortung vermerkt: »9/XI\strikeout{I} b.« }\buchAbdrucke{\weitereDrucke{Arthur Schnitzler, Richard Beer-Hofmann: \emph{Briefwechsel 1891–1931}. Hg. Konstanze Fliedl. Wien, Zürich: \emph{Europaverlag} 1992, S. 169.} }\toendnotes[C]{\smallbreak}\pstart{}{\pb}\textsc{Herrn Dr Richard Beer-Hofmann}\pend{}\pstart{}\textsc{\textcolor{pink}{Rodaun}{}\ledrightnote{\textcolor{pink}{Rodaun}}}\pend{}\pstart{}\textcolor{pink}{\textsc{Liesingerstraße} 2}{}\ledrightnote{\textcolor{pink}{Liesingerstraße}}\pend{}{\bigskip}\pstart
           \noindent{}\raggedleft{}{\pb}\textcolor{pink}{XVIII \textsc{Spoettel} 7}{}\ledrightnote{\textcolor{pink}{Edmund-Weiß-Gasse}}.\pend
           \pstart
           \raggedleft{}8. 11. 904.\pend
           \pstart
           lieber Richard, ich fahre vorausſichtlich Samſtag nach
                  \textcolor{pink}{Berlin}{}\ledrightnote{\textcolor{pink}{Berlin}}. Soll ich Ihnen dort irgendwas beſorgen,
               ſo ſchreiben Sie mir ein Wort.\pend
           \pstart
           Meine »\textcolor{green}{\textsc{Première}}{}\ledrightnote{→\textcolor{green}{Der grüne Kakadu. Groteske in einem Akt}{\newline}→\textcolor{green}{Der tapfere Cassian. Puppenspiel in einem Akt}}« ſoll am 19. ſein. –\pend
           \pstart
           – Hörte von dem echt jüdiſchen Vorgehen Ihres \textcolor{blue}{Hausherrn}{}\ledrightnote{→\textcolor{blue}{Rudolf Berger}}. Immerhin wäre es eine »fertige Sach« –.\pend
           \pstart
           Wie gehts Ihnen denn? Ich kann die Bemerkung nicht unterdrücken, daſs es mir lieb wär
                  we{\geminationn} wir nicht ſo weit von einander wohnten. –
               Herzlichſt Ihr \spacefill\mbox{A.}\pend
           \endnumbering\briefempfaengerindex{Beer-Hofmann, Richard@\textsc{Beer-Hofmann, Richard}!zzzSchnitzler, Arthur@\emph{von Arthur Schnitzler}!1904-11-081@{8. 11. 1904}|)be}\mylabel{h}  \normalsize

\doendnotes{C}
\bigskip
\vfill

\clearpage

\footnotesize

\lohead{\textsc{register}}

% Definiere theindex-Environment komplett neu ohne reledmac
\makeatletter
\renewenvironment{theindex}{%
  \section*{\indexname}%
  \setlength{\parindent}{0pt}%
  \setlength{\parskip}{0pt plus 0.3pt}%
  \let\item\@idxitem
}{%
  \clearpage
}
\makeatother

\IfFileExists{\jobname-pw.ind}{\input{\jobname-pw.ind}}{}

\end{document}

      