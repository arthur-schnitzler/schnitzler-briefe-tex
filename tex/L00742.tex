%% latex-korrekturansicht-vorspann.tex
%% Vorspann für die Korrekturansicht.
%% Lädt die gemeinsame Datei latex-vorspann.tex mit gesetztem Schalter.

\newif\ifkorrekturansicht
\korrekturansichttrue

\input{../tex-inputs/latex-vorspann}


               \section[Arthur Schnitzler an Hugo von Hofmannsthal, {[}16. 11. 1897{]}]{ Arthur Schnitzler an Hugo von Hofmannsthal, {[}16. 11. 1897{]}}\nopagebreak\mylabel{v}\rehead{ }\normalsize\beginnumbering\briefempfaengerindex{Hofmannsthal, Hugo von@\textsc{Hofmannsthal, Hugo von}!zzzSchnitzler, Arthur@\emph{von Arthur Schnitzler}!1897-11-161@{{[}16. 11. 1897{]}}|(be} \toendnotes[C]{\smallbreak\pagebreak[2]} \Standort{FDH, Hs-30885,65.}
\physDesc{Brief, 1 Blatt, 3 Seiten
\newline{}Handschrift: schwarze Tinte, deutsche Kurrent
\newline{}Hofmannsthal: mit Bleistift die 4. (leere) Seite beschriftet: »\noindent{}{\pb}\strikeout{\textcolor{blue}{Lutz}}{ / }\textcolor{blue}{Poldy}{ / }\textcolor{blue}{B\textsuperscript{\textcolor{gray}{rn}} Hess}{ / }\textcolor{blue}{Bodenhausen}{ / }\strikeout{\textcolor{blue}{Hansl}}« \newline{}Ordnung: von Schnitzler mutmaßlich bei der Durchsicht der Korrespondenz 1929 mit
                                    Bleistift beschriftet: »Datum? 92?
                                            96?« }\buchAbdrucke{\weitereDrucke{Hugo von Hofmannsthal, Arthur Schnitzler: \emph{Briefwechsel}. Hg. Therese Nickl und Heinrich Schnitzler. Frankfurt am Main: \emph{S. Fischer} 1964, S. 97–98.} }\toendnotes[C]{\smallbreak}\pstart
           \raggedleft{}{\pb}Dinstag{ }Früh.\pend
           \pstart
           Lieber Hugo, ich vergaſs Ihnen zu ſchreiben, dſs heute
                        Dinſtag{ }Abend{ }\uline{nichts} bei mir iſt. – Ihre Antwort \substVorne{}\textsuperscript{hatte}\substDazwischen{}geſtern\substHinten{} Früh hatte ich wohl erwartet; aber ich konnte den Verſuch nicht
                    weigern. Im übrigen mußte auch ich abſagen und hätte auch Ihnen abgeſagt, da ich
                    ſchrecklich verkühlt bin. –\pend
           \pstart
           Hier ſind Ihre drei \textcolor{green}{Stücke}{}\ledrightnote{→\textcolor{green}{Der weiße Fächer. Ein Zwischenspiel}{\newline}→\textcolor{green}{Die Hochzeit der Sobeide}{\newline}→\textcolor{green}{Die Schwestern}}. Ich habe mich {\pb}beim Leſen ſehr
                    gefreut. Am reinſten hat der \textcolor{green}{weiße Fächer}{}\ledrightnote{\textcolor{green}{Der weiße Fächer. Ein Zwischenspiel}} auf
                    mich gewirkt; käme es zwiſchen \textcolor{green}{Fortunio}{}\ledrightnote{→\textcolor{green}{Der weiße Fächer. Ein Zwischenspiel}} und \textcolor{green}{Miranda}{}\ledrightnote{→\textcolor{green}{Der weiße Fächer. Ein Zwischenspiel}} irgendwo, am beſten wohl am Schluſs, zu einem lebhaften
                    Sichſelber und Einanderverſtehn – ganz kurz, aber ſtark, ſo wäre das \textcolor{green}{Stück}{}\ledrightnote{→\textcolor{green}{Der weiße Fächer. Ein Zwischenspiel}} etwas vollko{\geminationm}enes. Bei der \textcolor{green}{jungen
                        Frau}{}\ledrightnote{\textcolor{green}{Die Hochzeit der Sobeide}} hab ich zum Schluſs meinen lieben \textcolor{green}{Kaufmann}{}\ledrightnote{→\textcolor{green}{Die Hochzeit der Sobeide}} wieder herbeigeſehnt. Hoffentlich laſſen Sie
                    ihn erſcheinen, bei welcher Gelegenheit {\pb}er
                    vielleicht auch aufklären könnte, wieſo die junge \textcolor{green}{Frau}{}\ledrightnote{→\textcolor{green}{Die Hochzeit der Sobeide}}{ }ſich über den \textcolor{green}{Sohn des Teppichhändlers}{}\ledrightnote{→\textcolor{green}{Die Hochzeit der Sobeide}} in ſo furchtbarer Weiſe
                    durch viele Jahre täuſchen konnte.\pend
           \pstart
           Meine Karte mit dem Brief von \textcolor{blue}{Andrian}{}\ledrightnote{\textcolor{blue}{Leopold von Andrian-Werburg}} haben
                    Sie bekommen? –\pend
           \pstart
           Herzlichen Gruſs.{\\[\baselineskip]}Ihr \spacefill\mbox{Arthur}\pend
           \leftskip=0em{}\endnumbering\briefempfaengerindex{Hofmannsthal, Hugo von@\textsc{Hofmannsthal, Hugo von}!zzzSchnitzler, Arthur@\emph{von Arthur Schnitzler}!1897-11-161@{{[}16. 11. 1897{]}}|)be}\mylabel{h}  \normalsize

\doendnotes{C}
\bigskip
\vfill

\clearpage

\footnotesize

\lohead{\textsc{register}}

% Definiere theindex-Environment komplett neu ohne reledmac
\makeatletter
\renewenvironment{theindex}{%
  \section*{\indexname}%
  \setlength{\parindent}{0pt}%
  \setlength{\parskip}{0pt plus 0.3pt}%
  \let\item\@idxitem
}{%
  \clearpage
}
\makeatother

\IfFileExists{\jobname-pw.ind}{\input{\jobname-pw.ind}}{}

\end{document}

      