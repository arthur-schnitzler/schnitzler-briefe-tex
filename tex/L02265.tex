%% latex-korrekturansicht-vorspann.tex
%% Vorspann für die Korrekturansicht.
%% Lädt die gemeinsame Datei latex-vorspann.tex mit gesetztem Schalter.

\newif\ifkorrekturansicht
\korrekturansichttrue

\input{../tex-inputs/latex-vorspann}


               \section[Arthur Schnitzler an Richard Beer-Hofmann, 29. 6. 1917]{ Arthur Schnitzler an Richard Beer-Hofmann, 29. 6. 1917}\nopagebreak\mylabel{v}\rehead{ }\normalsize\beginnumbering\briefempfaengerindex{Beer-Hofmann, Richard@\textsc{Beer-Hofmann, Richard}!zzzSchnitzler, Arthur@\emph{von Arthur Schnitzler}!1917-06-291@{29. 6. 1917}|(be} \toendnotes[C]{\smallbreak\pagebreak[2]} \Standort{YCGL, MSS 31.}
\physDesc{Kartenbrief
\newline{}Handschrift: Bleistift, lateinische Kurrent\newline{}Versand: Stempel: »\nobreak{}Wien, 30 VI 17\nobreak{}«.  
\newline{}Beer-Hofmann: mit blauem Buntstift Empfang und Beantwortung vermerkt:
                                    »E. B. 19./VII 17« }\buchAbdrucke{\weitereDrucke{Arthur Schnitzler, Richard Beer-Hofmann: \emph{Briefwechsel 1891–1931}. Hg. Konstanze Fliedl. Wien, Zürich: \emph{Europaverlag} 1992, S. 223.} }\toendnotes[C]{\smallbreak}\pstart{}{\pb}Abſ. Schnitzler, \textcolor{pink}{Wien XVIII Sternwartestr 71}{}\ledrightnote{\textcolor{pink}{Sternwartestraße}}.\pend{}{\bigskip}\pstart{}Herrn Doctor Richard Beer\substVorne{}\textsuperscript{h}\substDazwischen{}-H\substHinten{}ofmann\pend{}\pstart{}\textcolor{pink}{Bad Ischl}{}\ledrightnote{\textcolor{pink}{Bad Ischl}}\pend{}\pstart{}\textcolor{pink}{Grazerstr. 56}{}\ledrightnote{\textcolor{pink}{Grazer Straße}}\pend{}{\bigskip}\pstart
           \raggedleft{}{\pb}\textcolor{pink}{Wien}{}\ledrightnote{\textcolor{pink}{Wien}}, 29. 6. 1917\pend
           \pstart
           lieber Richard, ich nehme an es wird Sie interessiren, \label{KLL02265_Beer-Hofmann-1v}\edtext{näheres über \textcolor{blue}{Arthur Kfm.}{}\ledrightnote{\textcolor{blue}{Arthur Kaufmann}}}{\lemma{\textnormal{\emph{näheres über Arthur Kfm.}}}\Cendnote{\textnormal{vgl. A. S.: \emph{Tagebuch}, 24. 6. 1917}}}\label{KLL02265_Beer-Hofmann-1h} zu erfahren. Vorgestern war \introOben{}Prof.\introOben{}{ }\textcolor{blue}{Redlich}{}\ledrightnote{\textcolor{blue}{Emil Redlich}} bei ihm; er stellte die Diagnose \introOben{}(ich wohnte bei)\introOben{}, die wir schon nach den 2 Briefen, die ich
               von \textcolor{blue}{A. K.}{}\ledrightnote{\textcolor{blue}{Arthur Kaufmann}} nach \textcolor{pink}{Gastein}{}\ledrightnote{\textcolor{pink}{Bad Gastein}} erhalten hatte höchst wahrscheinlich war: (acute \introOben{}Manie\introOben{}) \uline{Manie}, »Hypomanie« wie er hinzu
               setzte – eine leichtere Form \introOben{}(Paranoia – keine Spur!)\introOben{}. Im
               19. Lebensjahr hat \textcolor{blue}{K.}{}\ledrightnote{\textcolor{blue}{Arthur Kaufmann}} einen ähnlichen Anfall
               gehabt, – damals trat die Krankheit als schwere Melancholie auf; – da der
               Zwischenraum ein so langer war – ist die Prognose günstig – we{\geminationn}{ }\introOben{}auch\introOben{} natürlich eine Wiederkehr in absehbarer Zeit keineswegs
               ausgeschlossen erscheint. Subjectiv befindet sich \textcolor{blue}{A.}{}\ledrightnote{\textcolor{blue}{Arthur Kaufmann}} wohl – nicht mehr \uline{zu wohl} – wie uns beim
               ersten Besuch \introOben{}in \textcolor{pink}{Purkersdorf}{}\ledrightnote{\textcolor{pink}{Purkersdorf}}\introOben{} beinah vorkam; kein zwanghaftes Denken mehr, kein Grübeln, – er \uline{will} gesund werden, möglichst bald und vollko{\geminationm}en, – vor allem um sein \textcolor{green}{Werk}{}\ledrightnote{→\textcolor{green}{[Philosophisches Werk]}} in aller Ruhe schreiben zu können. Wir wollen hoffen –
               und ich halte es für sehr möglich – daß er gerade in der Hauptsache gar nicht
               verrückt war – denn wer sollte die Philosophie weiter bringen können als er –
               insbesondre, da er die schöne Absicht hat sie überflüssig zu machen. Uns gehts
                  \label{TLL02265_Beer-Hofmann-1v}\edtext{recht gut, \textcolor{pink}{Gastein}{}\ledrightnote{\textcolor{pink}{Bad Gastein}} war sehr erholend, ich arbeite und wünschte
               ähnliches und andres auch von Ihnen zu hören. Wir grüßen}{\lemma{\textnormal{\emph{recht … grüßen}}}\Cendnote{\textnormal{am Seitenkopf, verkehrt zum
                  Text}}}\label{TLL02265_Beer-Hofmann-1h}{ }\label{TLL02265_Beer-Hofmann-2v}\edtext{Sie herzlichst Ihr
                  \spacefill\mbox{Arthur}}{\lemma{\textnormal{\emph{Sie … Arthur}}}\Cendnote{\textnormal{weiter am Seitenrand}}}\label{TLL02265_Beer-Hofmann-2h}\pend
           \endnumbering\briefempfaengerindex{Beer-Hofmann, Richard@\textsc{Beer-Hofmann, Richard}!zzzSchnitzler, Arthur@\emph{von Arthur Schnitzler}!1917-06-291@{29. 6. 1917}|)be}\mylabel{h}  \normalsize

\doendnotes{C}
\bigskip
\vfill

\clearpage

\footnotesize

\lohead{\textsc{register}}

% Definiere theindex-Environment komplett neu ohne reledmac
\makeatletter
\renewenvironment{theindex}{%
  \section*{\indexname}%
  \setlength{\parindent}{0pt}%
  \setlength{\parskip}{0pt plus 0.3pt}%
  \let\item\@idxitem
}{%
  \clearpage
}
\makeatother

\IfFileExists{\jobname-pw.ind}{\input{\jobname-pw.ind}}{}

\end{document}

      