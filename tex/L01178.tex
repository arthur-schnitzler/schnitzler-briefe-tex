%% latex-korrekturansicht-vorspann.tex
%% Vorspann für die Korrekturansicht.
%% Lädt die gemeinsame Datei latex-vorspann.tex mit gesetztem Schalter.

\newif\ifkorrekturansicht
\korrekturansichttrue

\input{../tex-inputs/latex-vorspann}


               \section[Hermann Bahr: Widmungsexemplar Wirkung in die Ferne für Arthur Schnitzler, Oktober 1901]{ Hermann Bahr: Widmungsexemplar Wirkung in die Ferne für Arthur
               Schnitzler, Oktober 1901}\nopagebreak\mylabel{v}\rehead{ }\normalsize\beginnumbering\briefempfaengerindex{Schnitzler, Arthur@\textsc{Schnitzler, Arthur}!zzzBahr, Hermann@\emph{von Hermann Bahr}!1901-10-011@{Oktober 1901}|(be} \toendnotes[C]{\smallbreak\pagebreak[2]} \Standort{DLA, G:Schnitzler, Arthur (Sammlung Heinrich Schnitzler).}
\physDesc{Widmung am Titelblatt
\newline{}Handschrift: blaue Tinte, deutsche Kurrent}\pstart
           \noindent{}{\pb}Arthur Schnitzler\pend
           \pstart
           herzlichſt{\\[\baselineskip]}\spacefill\mbox{Hermann Bahr}\pend
           \leftskip=0em{}\pstart
           \noindent{}Okt. 1901. \pend
           {\bigskip}\pstart
           \noindent{}\centering{}\textcolor{gray}{\textbf{\textbf{Hermann Bahr}}}\pend
           \pstart
           \noindent{}\centering{}\textcolor{gray}{\textbf{\textcolor{green}{\so{Wirkung in die Ferne}}{}\ledrightnote{\textcolor{green}{Wirkung in die Ferne und Anderes}}.}}\pend
           \pstart
           \noindent{}\centering{}\textcolor{gray}{\textbf{und Anderes}}\pend
           {\bigskip}\pstart
           \noindent{}\centering{}\textcolor{gray}{\textbf{\textcolor{brown}{Wiener Verlag}{}\ledrightnote{\textcolor{brown}{Wiener Verlag}}}}\pend
           \pstart
           \noindent{}\centering{}\textcolor{gray}{\textbf{1902}}\pend
           \endnumbering\briefempfaengerindex{Schnitzler, Arthur@\textsc{Schnitzler, Arthur}!zzzBahr, Hermann@\emph{von Hermann Bahr}!1901-10-011@{Oktober 1901}|)be}\mylabel{h}  \normalsize

\doendnotes{C}
\bigskip
\vfill

\clearpage

\footnotesize

\lohead{\textsc{register}}

% Definiere theindex-Environment komplett neu ohne reledmac
\makeatletter
\renewenvironment{theindex}{%
  \section*{\indexname}%
  \setlength{\parindent}{0pt}%
  \setlength{\parskip}{0pt plus 0.3pt}%
  \let\item\@idxitem
}{%
  \clearpage
}
\makeatother

\IfFileExists{\jobname-pw.ind}{\input{\jobname-pw.ind}}{}

\end{document}

      