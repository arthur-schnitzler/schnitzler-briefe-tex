%% latex-korrekturansicht-vorspann.tex
%% Vorspann für die Korrekturansicht.
%% Lädt die gemeinsame Datei latex-vorspann.tex mit gesetztem Schalter.

\newif\ifkorrekturansicht
\korrekturansichttrue

\input{../tex-inputs/latex-vorspann}


               \section[Arthur Schnitzler an Stefan Großmann, 24. 9. 1925]{ Arthur Schnitzler an Stefan Großmann, 24. 9. 1925}\nopagebreak\mylabel{v}\rehead{ }\normalsize\beginnumbering\briefempfaengerindex{Grossmann, Stefan@\textsc{Großmann, Stefan}!zzzSchnitzler, Arthur@\emph{von Arthur Schnitzler}!1925-09-241@{24. 9. 1925}|(be} \toendnotes[C]{\smallbreak\pagebreak[2]} \Standort{DLA, A:Schnitzler, HS.NZ85.1.896.}
\physDesc{Brief, maschineller Durchschlag
\newline{}Schreibmaschine
\newline{}Handschrift: roter Buntstift, deutsche Kurrent (\noindent{}Beschriftung: »\textsc{Großma{\geminationn}}« und zwei Unterstreichungen)}\pstart
           \raggedleft{}{\pb}24. 9. 1925. \pend
           \pstart{}Verehrter Herr Grossmann.\pend\pstart
           Vielen Dank für Ihre freundliche Einladung, der ich sehr gerne Folge leisten
                    werde, ohne mich aber in diesem Augenblick für einen bestimmten Termin
                    verpflichten zu können.\pend
           \pstart
           Mit den verbindlichsten Grüssen{\\[\baselineskip]}Ihr sehr ergebener\pend
           \leftskip=0em{}{\bigskip}\pstart
           \noindent{}Herrn Stefan Grossmann,{\\}\textcolor{brown}{Tagebuchverlag}{}\ledrightnote{\textcolor{brown}{Das Tage-Buch}}, \textcolor{pink}{Berlin SW 19}{}\ledrightnote{\textcolor{pink}{Berlin}}.{\\}\textcolor{pink}{Beuthstrasse 19}{}\ledrightnote{\textcolor{pink}{Beuthstrasse}}.\pend
           \endnumbering\briefempfaengerindex{Grossmann, Stefan@\textsc{Großmann, Stefan}!zzzSchnitzler, Arthur@\emph{von Arthur Schnitzler}!1925-09-241@{24. 9. 1925}|)be}\mylabel{h}  \normalsize

\doendnotes{C}
\bigskip
\vfill

\clearpage

\footnotesize

\lohead{\textsc{register}}

% Definiere theindex-Environment komplett neu ohne reledmac
\makeatletter
\renewenvironment{theindex}{%
  \section*{\indexname}%
  \setlength{\parindent}{0pt}%
  \setlength{\parskip}{0pt plus 0.3pt}%
  \let\item\@idxitem
}{%
  \clearpage
}
\makeatother

\IfFileExists{\jobname-pw.ind}{\input{\jobname-pw.ind}}{}

\end{document}

      