%% latex-korrekturansicht-vorspann.tex
%% Vorspann für die Korrekturansicht.
%% Lädt die gemeinsame Datei latex-vorspann.tex mit gesetztem Schalter.

\newif\ifkorrekturansicht
\korrekturansichttrue

\input{../tex-inputs/latex-vorspann}


               \section[Arthur Schnitzler an Stefan Großmann, 17. 2. 1921]{ Arthur Schnitzler an Stefan Großmann, 17. 2. 1921}\nopagebreak\mylabel{v}\rehead{ }\normalsize\beginnumbering\briefempfaengerindex{Grossmann, Stefan@\textsc{Großmann, Stefan}!zzzSchnitzler, Arthur@\emph{von Arthur Schnitzler}!1921-02-171@{17. 2. 1921}|(be} \toendnotes[C]{\smallbreak\pagebreak[2]} \Standort{DLA, A:Schnitzler, HS.NZ85.1.896.}
\physDesc{Brief, maschineller Durchschlag
\newline{}Schreibmaschine
\newline{}Handschrift: roter Buntstift, deutsche Kurrent (\noindent{}Beschriftung: »K{[}opie{]}«,
                                 Unterstreichungen)}\buchAbdrucke{\weitereDrucke{1) \pwindex{Reigen der Gassenjungen@\emph{Der Reigen der Gassenjungen}|pwk}\pwindex{Tage-Buch@\emph{Das Tage-Buch}|pwk}Stefan Großmann: \emph{Der Reigen der Gassenjungen.} In: \emph{Das Tage-Buch}, Jg. 2, Nr. 8, 26. 2. 1921, S. 252–253.} \weitereDrucke{2) Arthur Schnitzler: \emph{Briefe 1913–1931}. Hg. Peter Michael Braunwarth, Richard Miklin, Susanne Pertlik und Heinrich Schnitzler. Frankfurt am Main: \emph{S. Fischer} 1984, S. 234–235.} }\toendnotes[C]{\smallbreak}\pstart
           \raggedleft{}{\pb}17. 2. 1921.\pend
           \pstart{}Sehr verehrter Herr Grossmann.\pend\pstart
           Vielen Dank für Ihr freundliches Interesse. Sie haben indess wohl meine Karte
               erhalten, in der ich Ihnen sagte, wie sehr mich Ihr parodistischer \textcolor{green}{Dialog}{}\ledrightnote{→\textcolor{green}{Hänischs Reigen. Eine unsittliche Szenenfolge}} amüsiert hat. Ich habe vorläufig keine
               Absicht mich über den »\textcolor{green}{Reigen}{}\ledrightnote{\textcolor{green}{Reigen. Zehn Dialoge}}« und die sogenannnte
                  \textcolor{green}{Reigen}{}\ledrightnote{\textcolor{green}{Reigen. Zehn Dialoge}}-Affaire in der Oeffentlichkeit weiter zu
               äussern. Was ich {\pb}Herrn \textcolor{green}{\textcolor{blue}{Maximilian Harden}{}\ledrightnote{\textcolor{blue}{Maximilian Harden}}}{}\ledrightnote{→\textcolor{green}{Reigen}}{ }\textcolor{green}{erwidert}{}\ledrightnote{→\textcolor{green}{Berichtigung. Ein paar Worte zum Gutachten Maximilian Hardens über den »Reigen«}} habe, ersehen Sie aus
               beiliegendem \label{K_L02363_1v}\edtext{Zeitungsblatt}{\lemma{\textnormal{\emph{Zeitungsblatt}}}\Cendnote{\textnormal{\textcolor{blue}{Arthur
                        Schnitzler}: \emph{\textcolor{green}{Berichtigung. Ein paar Worte
                        zum Gutachten Maximilian Hardens über den »Reigen«}} in: \emph{\textcolor{green}{Neues Wiener Journal}}, Jg. 29, Nr. 9782,
                        30. 1. 1921, S. 6.}}}\label{K_L02363_1h}. Die Berichtigung war übrigens
               in einigen \textcolor{pink}{Berlin}{}\ledrightnote{\textcolor{pink}{Berlin}}er Blättern abgedruckt. Von den
               hiesigen Skandalen, insbesondere von dem \label{K_L02363_2v}\edtext{gestrigen}{\lemma{\textnormal{\emph{gestrigen}}}\Cendnote{\textnormal{am 16. 2. 1921}}}\label{K_L02363_2h}, werden Sie wohl indess gelesen
               haben. Was soll man dazu sagen? Ich käme mir unsäglich komisch vor, wollte ich mit
               den Herren \textcolor{blue}{Kuntschak}{}\ledrightnote{\textcolor{blue}{Leopold Kunschak}} oder \textcolor{blue}{Seipel}{}\ledrightnote{\textcolor{blue}{Ignaz Seipel}} oder mit dem Schusterlehrling polemisieren, der das
               Theater stürmt, mit dem begeisterten Ruf: Nieder mit dem Reigen! Man schändet unsere
               Frauen! Nieder mit den Sozialdemokraten! (Es kann übrigens auch ein Stud. med.
               gewesen sein oder ein Tapezierergehilfe, – wobei meine Sympathie immerhin noch mehr
               bei dem Tapezierergehilfen ist als bei den Herren \textcolor{blue}{Seipel}{}\ledrightnote{\textcolor{blue}{Ignaz Seipel}} und \textcolor{blue}{Kuntschak}{}\ledrightnote{\textcolor{blue}{Leopold Kunschak}}.{[}){]} Ich habe ja schon einige ähnliche Sachen
               erlebt, wenn auch in bescheideneren Dimensionen. Erinnern Sie sich nur an den »\textcolor{green}{Leutnant Gustl}{}\ledrightnote{\textcolor{green}{Lieutenant Gustl. Novelle}}« und den »\textcolor{green}{Professor Bernhardi}{}\ledrightnote{\textcolor{green}{Professor Bernhardi. Komödie in fünf Akten}}«. Nach einigen Jahren bleibt von all dem
               Lärm nichts weiter übrig als die Bücher, die ich geschrieben und eine dunkle
               Erinnerung an die Blamage meiner Gegner. In diesem Fall wird es nicht anders
               sein.\pend
           \pstart
           Mit herzlichem Gruss{\\[\baselineskip]}Ihr sehr ergebener{\\[\baselineskip]}\pend
           \leftskip=0em{}\endnumbering\briefempfaengerindex{Grossmann, Stefan@\textsc{Großmann, Stefan}!zzzSchnitzler, Arthur@\emph{von Arthur Schnitzler}!1921-02-171@{17. 2. 1921}|)be}\mylabel{h}  \normalsize

\doendnotes{C}
\bigskip
\vfill

\clearpage

\footnotesize

\lohead{\textsc{register}}

% Definiere theindex-Environment komplett neu ohne reledmac
\makeatletter
\renewenvironment{theindex}{%
  \section*{\indexname}%
  \setlength{\parindent}{0pt}%
  \setlength{\parskip}{0pt plus 0.3pt}%
  \let\item\@idxitem
}{%
  \clearpage
}
\makeatother

\IfFileExists{\jobname-pw.ind}{\input{\jobname-pw.ind}}{}

\end{document}

      