%% latex-korrekturansicht-vorspann.tex
%% Vorspann für die Korrekturansicht.
%% Lädt die gemeinsame Datei latex-vorspann.tex mit gesetztem Schalter.

\newif\ifkorrekturansicht
\korrekturansichttrue

\input{../tex-inputs/latex-vorspann}


               \section[Hugo von Hofmannsthal an Arthur Schnitzler, 24. 7. {[}1916{]}]{ Hugo von Hofmannsthal an Arthur Schnitzler, 24. 7. {[}1916{]}}\nopagebreak\mylabel{v}\rehead{ }\normalsize\beginnumbering\briefempfaengerindex{Schnitzler, Arthur@\textsc{Schnitzler, Arthur}!zzzHofmannsthal, Hugo von@\emph{von Hugo von Hofmannsthal}!1916-07-241@{24. 7. {[}1916{]}}|(be} \toendnotes[C]{\smallbreak\pagebreak[2]} \Standort{CUL, Schnitzler, B 43.}
\physDesc{Briefkarte
\newline{}Handschrift: schwarze Tinte, deutsche Kurrent
\newline{}Schnitzler: mit Bleistift Jahreszahl und Ort ergänzt: »1916{ }\textsc{\textcolor{pink}{Altaussee}}« \newline{}Ordnung: 1) mit Bleistift von \textcolor{blue}{Frieda Pollak} (?) mit dem Buchstaben »A« (Abgeschrieben/Abschrift) gekennzeichnet 2) mit Bleistift von unbekannter Hand nummeriert: »\strikeout{346}«3) mit Bleistift von unbekannter Hand nummeriert: »355«}\buchAbdrucke{\weitereDrucke{Hugo von Hofmannsthal, Arthur Schnitzler: \emph{Briefwechsel}. Hg. Therese Nickl und Heinrich Schnitzler. Frankfurt am Main: \emph{S. Fischer} 1964, S. 278.} }\toendnotes[C]{\smallbreak}\pstart
           \raggedleft{}{\pb}24 VII.\pend
           \pstart{}mein lieber Arthur\pend\pstart
           ich freue mich zu denken daſs Sie \textcolor{blue}{Olga}{}\ledrightnote{\textcolor{blue}{Olga Schnitzler}} u. die \textcolor{blue}{Kinder}{}\ledrightnote{→\textcolor{blue}{Heinrich Schnitzler}{\newline}→\textcolor{blue}{Lili Schnitzler}} hier in der Nähe
               ſind und, wie ich denke, zufrieden.\hspace*{1.5em}Ich hoffe daſs
               ich eine Zeitlang hier bleiben u. vielleicht etwas für mich arbeiten kann – es iſt
               freilich immer ungewiſs.\hspace*{1.5em}Die \textcolor{blue}{Kinder}{}\ledrightnote{→\textcolor{blue}{Heinrich Schnitzler}{\newline}→\textcolor{blue}{Lili Schnitzler}}{ }ſagen mir, Sie hätten {\pb}geſagt,
               Ihre Arbeitszeit wäre nachmittag bis gegen 6\textsuperscript{h}.\hspace*{1.5em}So würde ich gerne morgen etwas nach 6\textsuperscript{h} zu Ihnen ko{\geminationm}en, \textcolor{blue}{Gerty}{}\ledrightnote{\textcolor{blue}{Gertrude von Hofmannsthal}} auch (außer \textcolor{blue}{Olga}{}\ledrightnote{\textcolor{blue}{Olga Schnitzler}} läſst anderes
                  ſagen)\hspace*{1.5em}Man könnte dann vielleicht zuſa{\geminationm}en herumgehen u zuſa{\geminationm}en
               beim \textcolor{pink}{\textsc{Seewirth}}{}\ledrightnote{\textcolor{pink}{Seewirt}} nachtmahlen. Wenn es paſst bedarf es keiner Antwort.\pend
           \pstart
           Der Ihre, herzlich{\\[\baselineskip]}\spacefill\mbox{Hugo.}\pend
           \leftskip=0em{}\endnumbering\briefempfaengerindex{Schnitzler, Arthur@\textsc{Schnitzler, Arthur}!zzzHofmannsthal, Hugo von@\emph{von Hugo von Hofmannsthal}!1916-07-241@{24. 7. {[}1916{]}}|)be}\mylabel{h}  \normalsize

\doendnotes{C}
\bigskip
\vfill

\clearpage

\footnotesize

\lohead{\textsc{register}}

% Definiere theindex-Environment komplett neu ohne reledmac
\makeatletter
\renewenvironment{theindex}{%
  \section*{\indexname}%
  \setlength{\parindent}{0pt}%
  \setlength{\parskip}{0pt plus 0.3pt}%
  \let\item\@idxitem
}{%
  \clearpage
}
\makeatother

\IfFileExists{\jobname-pw.ind}{\input{\jobname-pw.ind}}{}

\end{document}

      