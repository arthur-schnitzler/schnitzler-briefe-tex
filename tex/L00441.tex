%% latex-korrekturansicht-vorspann.tex
%% Vorspann für die Korrekturansicht.
%% Lädt die gemeinsame Datei latex-vorspann.tex mit gesetztem Schalter.

\newif\ifkorrekturansicht
\korrekturansichttrue

\input{../tex-inputs/latex-vorspann}


               \section[Laura Marholm an Arthur Schnitzler, 15. 5. 1895]{ Laura Marholm an Arthur Schnitzler, 15. 5. 1895}\nopagebreak\mylabel{v}\rehead{ }\normalsize\beginnumbering\briefempfaengerindex{Schnitzler, Arthur@\textsc{Schnitzler, Arthur}!zzzMarholm, Laura@\emph{von Laura Marholm}!1895-05-151@{15. 5. 1895}|(be} \toendnotes[C]{\smallbreak\pagebreak[2]} \Standort{CUL, Schnitzler, B 69.}
\physDesc{Brief, 1 Blatt, 1 Seite
\newline{}Handschrift: schwarze Tinte, lateinische Kurrent}\toendnotes[C]{\smallbreak}\pstart
           \noindent{}\raggedleft{}{\pb}\textcolor{pink}{Schliersee}{}\ledrightnote{\textcolor{pink}{Schliersee}}, \textcolor{pink}{Oberbaiern}{}\ledrightnote{\textcolor{pink}{Oberbayern}}\pend
           \pstart
           \raggedleft{}15. Mai 95.\pend
           \pstart{}Sehr geehrter Herr Doktor.\pend\pstart
           Den \textcolor{green}{Musenalmanach von 94}{}\ledrightnote{\textcolor{green}{Moderner Musen-Almanach auf das Jahr 1894}} hab ich noch nicht
                    finden können, aber ich muß ihn haben und finde ihn schon. Das, was ich meine,
                    ist vielleicht nur ein Erzeugniß der Einsamkeit, wo das Leben Einem zu dicht und
                    stark an den Ohren klopft. Es ist sehr merkwürdig, daß ich es grade am stärksten
                    in Glücksmomenten empfinde.\pend
           \pstart
           Ich freue mich auf ihre weiteren Bücher!\pend
           \pstart
           Heute nur eine Bitte: haben Sie nicht bemerkt, ob in der letzten Zeit von mir das
                    eine oder andere Feuilleton: »\textcolor{green}{Der Dichter des
                        Weibmysteriums}{}\ledrightnote{\textcolor{green}{Der Dichter des Weibmysteriums}}« oder »\textcolor{green}{Weisse
                            Fläche}{}\ledrightnote{\textcolor{green}{Weiße Fläche}}« in der \textcolor{green}{N. freien Presse}{}\ledrightnote{\textcolor{green}{Neue Freie Presse}}
               sichtbar gewesen ist? Man erfährt niemals was direct von daher. Und ich habe
                    Niemanden in \textcolor{pink}{Wien}{}\ledrightnote{\textcolor{pink}{Wien}}, der mir darüber Auskunft
                    gäbe. Sie sind doch Leser der \textcolor{green}{N. fr. Presse}{}\ledrightnote{\textcolor{green}{Neue Freie Presse}}
                    und ich wäre Ihnen sehr dankbar für die Nachricht, ob das eine oder andere schon
                    erschienen ist, oder bis Ende Mai erscheint, da ich das erstere
                    \textcolor{green}{Feuilleton}{}\ledrightnote{→\textcolor{green}{Der Dichter des Weibmysteriums}} bald in ein \label{K_L00441_1v}\edtext{\textcolor{green}{Buch}{}\ledrightnote{→\textcolor{green}{Wir Frauen und unsere Dichter}}}{\lemma{\textnormal{\emph{Buch}}}\Cendnote{\textnormal{Da der Text über \textcolor{blue}{Barbey d’Aurevilly} erst am
                        2. 11. 1895 in der \emph{\textcolor{green}{Zukunft}} (Bd. 13, S. 219–226) erschien, fehlt er
                        in der 1. Auflage von \emph{\textcolor{green}{Wir Frauen und unsere
                            Dichter}} (Wien, Leipzig: \emph{Verlag der Wiener
                                Mode}{ }1895), wurde aber in die »Zweite umgearbeitete und wesentlich
                            vermehrte Ausgabe mit 8 Portraits« aufgenommen (Berlin:
                                \emph{Carl Duncker}{ }{[}1900{]}, S. 271–289).}}}\label{K_L00441_1h} aufnehmen will.\pend
           \pstart
           \label{T_L00441_1v}\edtext{Also}{\lemma{\textnormal{\emph{Also}}}\Cendnote{\textnormal{weiter quer am linken Rand}}}\label{T_L00441_1h} beste Grüße für diesmal. Kommt bald was von Ihnen?\pend
           \pstart Ihre ergeb.\hspace*{1.5em}\spacefill\mbox{Laura Marholm.}\pend{}\endnumbering\briefempfaengerindex{Schnitzler, Arthur@\textsc{Schnitzler, Arthur}!zzzMarholm, Laura@\emph{von Laura Marholm}!1895-05-151@{15. 5. 1895}|)be}\mylabel{h}  \normalsize

\doendnotes{C}
\bigskip
\vfill

\clearpage

\footnotesize

\lohead{\textsc{register}}

% Definiere theindex-Environment komplett neu ohne reledmac
\makeatletter
\renewenvironment{theindex}{%
  \section*{\indexname}%
  \setlength{\parindent}{0pt}%
  \setlength{\parskip}{0pt plus 0.3pt}%
  \let\item\@idxitem
}{%
  \clearpage
}
\makeatother

\IfFileExists{\jobname-pw.ind}{\input{\jobname-pw.ind}}{}

\end{document}

      