%% latex-korrekturansicht-vorspann.tex
%% Vorspann für die Korrekturansicht.
%% Lädt die gemeinsame Datei latex-vorspann.tex mit gesetztem Schalter.

\newif\ifkorrekturansicht
\korrekturansichttrue

\input{../tex-inputs/latex-vorspann}


               \section[Max Burckhard an Arthur Schnitzler, {[}1. 12. 1900?{]}]{ Max Burckhard an Arthur Schnitzler, {[}1. 12. 1900?{]}}\nopagebreak\mylabel{v}\rehead{ }\normalsize\beginnumbering\briefempfaengerindex{Schnitzler, Arthur@\textsc{Schnitzler, Arthur}!zzzBurckhard, Max Eugen@\emph{von Max Eugen Burckhard}!1900-12-012@{{[}1. 12. 1900?{]}}|(be} \toendnotes[C]{\smallbreak\pagebreak[2]} \Standort{CUL, Schnitzler, B 20.}
\physDesc{Telegramm
\newline{}maschinell
\newline{}Schnitzler: mit Bleistift datiert: »99? 900?
                                 902?« \newline{}Ordnung: beschnitten }\toendnotes[C]{\smallbreak}\pstart
           \noindent{}{\pb}\label{T_L01083_1v}\edtext{hiesige}{\lemma{\textnormal{\emph{hiesige}}}\Cendnote{\textnormal{korrigiert aus: »hisige«}}}\label{T_L01083_1h} theaterpflichten
               hielten mich leider fest. meine aufrichtigsten und besten wuensche fuer \label{K_L01083_1v}\edtext{heute abend}{\lemma{\textnormal{\emph{heute abend}}}\Cendnote{\textnormal{Das Telegramm könnte am 29. 4. 1899,
                     1. 12. 1900 oder 4. 1. 1902 verfasst sein – alles
                  Tage, an denen Schnitzler zu Uraufführungen im Ausland weilte. Am
                  wahrscheinlichsten ist die Uraufführung von \emph{\textcolor{green}{Der
                     Schleier der Beatrice}} am 1. 12. 1900 in \textcolor{pink}{Breslau}, da viele \textcolor{pink}{Wien}er
                  speziell dafür anreisten. Bei den Theaterpflichten \textcolor{blue}{Burckhard}s dürfte es sich um seine Tätigkeit als Theaterkritiker handeln.
                  Die Abschrift des Briefwechsels datiert ausschließlich auf
                  »1902?«.}}}\label{K_L01083_1h}. gruesse \spacefill\mbox{doctor burckhard +}\pend
           \endnumbering\briefempfaengerindex{Schnitzler, Arthur@\textsc{Schnitzler, Arthur}!zzzBurckhard, Max Eugen@\emph{von Max Eugen Burckhard}!1900-12-012@{{[}1. 12. 1900?{]}}|)be}\mylabel{h}  \normalsize

\doendnotes{C}
\bigskip
\vfill

\clearpage

\footnotesize

\lohead{\textsc{register}}

% Definiere theindex-Environment komplett neu ohne reledmac
\makeatletter
\renewenvironment{theindex}{%
  \section*{\indexname}%
  \setlength{\parindent}{0pt}%
  \setlength{\parskip}{0pt plus 0.3pt}%
  \let\item\@idxitem
}{%
  \clearpage
}
\makeatother

\IfFileExists{\jobname-pw.ind}{\input{\jobname-pw.ind}}{}

\end{document}

      