%% latex-korrekturansicht-vorspann.tex
%% Vorspann für die Korrekturansicht.
%% Lädt die gemeinsame Datei latex-vorspann.tex mit gesetztem Schalter.

\newif\ifkorrekturansicht
\korrekturansichttrue

\input{../tex-inputs/latex-vorspann}


               \section[Arthur Schnitzler an Richard Beer-Hofmann, 26. 2. 1905]{ Arthur Schnitzler an Richard Beer-Hofmann, 26. 2. 1905}\nopagebreak\mylabel{v}\rehead{ }\normalsize\beginnumbering\briefempfaengerindex{Beer-Hofmann, Richard@\textsc{Beer-Hofmann, Richard}!zzzSchnitzler, Arthur@\emph{von Arthur Schnitzler}!1905-02-261@{26. 2. 1905}|(be} \toendnotes[C]{\smallbreak\pagebreak[2]} \Standort{YCGL, MSS 31.}
\physDesc{Kartenbrief
\newline{}Handschrift: schwarze Tinte, deutsche Kurrent\newline{}Versand: 1) Stempel: »\nobreak{}Wien 68, 26. 2. 05, 5–6N\nobreak{}«.  2) Stempel: »\nobreak{}\oindex{Rodaun@\textbf{Rodaun}, \emph{Teil eines besiedelten Ortes (A.BSOX)}|pwk}Rodaun, 27. 2. \textcolor{gray}{05}, 7–9V\nobreak{}«. }\toendnotes[C]{\smallbreak}\pstart{}{\pb}Herrn \textsc{Dr. Richard
                     Beer-Hofmann}\pend{}\pstart{}\textsc{\textcolor{pink}{Rodaun}{}\ledrightnote{\textcolor{pink}{Rodaun}}}\pend{}\pstart{}\textcolor{pink}{\textsc{Liesinger Straße 2}}{}\ledrightnote{\textcolor{pink}{Liesingerstraße}}. \pend{}{\bigskip}\pstart
           \raggedleft{}{\pb}So{\geminationn}tag 26. 2. 905.\pend
           \pstart
           lieber Richard, ich reiſe am \label{K_L01502_1v}\edtext{Freitag 3.{ }}{\lemma{\textnormal{\emph{Freitag 3. }}}\Cendnote{\textnormal{siehe A. S.: \emph{Tagebuch}, 3. 3. 1905}}}\label{K_L01502_1h} \textcolor{pink}{Genua}{}\ledrightnote{\textcolor{pink}{Genua}} zu \textcolor{pink}{Mittelmeer}{}\ledrightnote{\textcolor{pink}{Mittelmeer}}zwecken; und, unter günſtigen Umſtänden bin ich erſt \label{K_L01502_2v}\edtext{gegen den 20.}{\lemma{\textnormal{\emph{gegen den 20.}}}\Cendnote{\textnormal{vgl. A. S.: \emph{Tagebuch}, 18. 3. 1905}}}\label{K_L01502_2h} wieder hier\substVorne{}\textsuperscript{?}\substDazwischen{}.\substHinten{}\pend
           \pstart
           Könnte man ſich nicht vorher doch einmal ſehen? Den \textcolor{blue}{Hugo’s}{}\ledrightnote{\textcolor{blue}{Gertrude von Hofmannsthal}{\newline}\textcolor{blue}{Hugo von Hofmannsthal}} hab ich für \label{K_L01502_3v}\edtext{Mittwoch}{\lemma{\textnormal{\emph{Mittwoch}}}\Cendnote{\textnormal{Das Treffen fand, ohne das Ehepaar \textcolor{blue}{Hofmannsthal}, am Donnerstag statt; siehe A. S.: \emph{Tagebuch}, 2. 3. 1905}}}\label{K_L01502_3h}{ }Abend, \textcolor{pink}{Hietzing}{}\ledrightnote{\textcolor{pink}{Ottakringer Bräu}} geſchrieben; kommen Sie
               etwa auch mit \textcolor{blue}{Paula}{}\ledrightnote{\textcolor{blue}{Paula Beer-Hofmann}}? Oder wollen Sie nicht
               endlich einmal bei uns eſſen?\pend
           \pstart
           Laſſen Sie jedenfalls ein Wort hören.\pend
           \pstart
           Herzlichſt Ihr{\\[\baselineskip]}\spacefill\mbox{A.}\pend
           \leftskip=0em{}\endnumbering\briefempfaengerindex{Beer-Hofmann, Richard@\textsc{Beer-Hofmann, Richard}!zzzSchnitzler, Arthur@\emph{von Arthur Schnitzler}!1905-02-261@{26. 2. 1905}|)be}\mylabel{h}  \normalsize

\doendnotes{C}
\bigskip
\vfill

\clearpage

\footnotesize

\lohead{\textsc{register}}

% Definiere theindex-Environment komplett neu ohne reledmac
\makeatletter
\renewenvironment{theindex}{%
  \section*{\indexname}%
  \setlength{\parindent}{0pt}%
  \setlength{\parskip}{0pt plus 0.3pt}%
  \let\item\@idxitem
}{%
  \clearpage
}
\makeatother

\IfFileExists{\jobname-pw.ind}{\input{\jobname-pw.ind}}{}

\end{document}

      