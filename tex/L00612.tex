%% latex-korrekturansicht-vorspann.tex
%% Vorspann für die Korrekturansicht.
%% Lädt die gemeinsame Datei latex-vorspann.tex mit gesetztem Schalter.

\newif\ifkorrekturansicht
\korrekturansichttrue

\input{../tex-inputs/latex-vorspann}


               \section[Peter Altenberg an Arthur Schnitzler, {[}30.? 10. 1896{]}]{ Peter Altenberg an Arthur Schnitzler, {[}30.? 10. 1896{]}}\nopagebreak\mylabel{v}\rehead{ }\normalsize\beginnumbering\briefempfaengerindex{Schnitzler, Arthur@\textsc{Schnitzler, Arthur}!zzzAltenberg, Peter@\emph{von Peter Altenberg}!1896-10-301@{{[}30.? 10. 1896{]}}|(be} \toendnotes[C]{\smallbreak\pagebreak[2]} \Standort{CUL, Schnitzler, B 2.}
\physDesc{Brief, 1 Blatt, 3 Seiten
\newline{}Handschrift: schwarze Tinte, deutsche Kurrent
\newline{}Schnitzler: 1) mit Bleistift auf das falsche Jahr datiert: »Nov 97« 2) mit rotem Buntstift eine Unterstreichung\newline{}Ordnung: mit Bleistift von unbekannter Hand nummeriert:
                                 »6« }\buchAbdrucke{\weitereDrucke{1) Kurt Bergel: \emph{Arthur Schnitzlers unveröffentlichte Tragikomödie Das Wort.} In: \emph{Studies in Arthur Schnitzler. Centennial Commemorative
                        Volume}. Hg. Herbert W. Reichert und Herman Salinger. Chapel Hill: \emph{University of North Carolina Press} 1963, S. 20 (UNC Studies in the Germanic Languages and Literatures, 42).} \weitereDrucke{2) Arthur Schnitzler: \emph{Das Wort. Tragikomödie in fünf Akten. Fragment}. Aus dem Nachlaß hg. und eingel. von Kurt Bergel. Frankfurt am Main: \emph{S. Fischer} 1966, S. 8–9.} \weitereDrucke{3) Peter Altenberg: \emph{Die Selbsterfindung eines Dichters. Briefe und Dokumente
                        1892–1896}. Hg. und mit einem Nachwort von Leo A. Lensing. Göttingen: \emph{Wallstein} 2009, S. 77.} }\toendnotes[C]{\smallbreak}\pstart{}{\pb}Lieber \textsc{D\textsuperscript{r.}} Arthur Schnitzler:\pend\pstart
           Sie können ſich gar nicht vorſtellen, wie tief mich ihre wunderbare Aufmerkſamkeit
               ergriffen hat.\pend
           \pstart
           Sie haben einem Bankrottirer des Lebens zu ſeinen ſparſamen Augenblicken des Glückes
               einen heiligen Augenblick hinzugefügt.\pend
           \pstart
           Mögen Sie, edler Sieger im Leben, nicht ſich wundern, wenn Einer, der durch
               körperliche, ſeeliſche und ökonomiſche Leiden beſiegt und zerdrückt \introOben{}iſt\introOben{}, manchesmal mit Verwunderung auf Jene blickt, {\pb}welchen das Schickſal freundlicher
               lächelt. Mögen Sie mir es verzeihen, der ich die »\uline{ewige
                  Bewegung}«, das »\uline{innere Stürmen}« für das
               Schönſte halte, wenn ich mit Verwunderung auf ihren innigeren Freundeskreis blicke,
               in welchem uralte Greiſe wie \textcolor{blue}{Leo Ebermann}{}\ledrightnote{\textcolor{blue}{Leo Ebermann}} und \textcolor{blue}{Gustav Schwarzkopf}{}\ledrightnote{\textcolor{blue}{Gustav Schwarzkopf}}{ }Stammſitze haben.\pend
           \pstart
           Merkwürdig, Sie waren der Erſte, der mir über meine Manuſkripte erlöſende Worte
               ſagte. Nun bringen Sie mir ein wundervolles Urtheil{ }{\pb}von \textcolor{blue}{G. Hauptmann}{}\ledrightnote{\textcolor{blue}{Gerhart Hauptmann}}.\pend
           \pstart
           Sie haben ſich i{\geminationm}er fein und zart gegen mich
               benommen.\pend
           \pstart
           Möge in kommender Zeit ein freundſchaftliche\strikeout{s}res
               Zuſammenleben mir Gelegenheit geben, meine keimenden Neigungen auswachſsen zu laſſen.
               Das wünſche ich mir!\pend
           \pstart
           Schreiben Sie mir aus \textcolor{pink}{Berlin}{}\ledrightnote{\textcolor{pink}{Berlin}}. Sie erleben dort
               gewiſs ſehr viel. Ich ſelbſt lebe in Sehnſucht nach meiner ſchwarzen Freundin \label{K_L00612_1v}\edtext{\textcolor{blue}{\uline{\textsc{Nahbadûh}}}{}\ledrightnote{\textcolor{blue}{Nah-Badû}}}{\lemma{\textnormal{\emph{Nahbadûh}}}\Cendnote{\textnormal{Dabei handelt es sich um eine der
                  Schaustellerinnen des in \textcolor{pink}{Wien} errichteten \textcolor{pink}{Afrika-Dorfes}, das \textcolor{blue}{Altenberg} frequentierte. Seine Liebe zu
                  derselben kommt im Buch \emph{\textcolor{green}{Ashantee}} (Berlin:
                        \emph{\textcolor{brown}{S. Fischer}}1897) mehrfach zum Ausdruck. Es handelt sich dabei aber nicht um eine
                  literarische Figur, sondern um die Literarisierung einer Leidenschaft, wie \textcolor{blue}{Georg Hirschfeld} andeutet (\textcolor{blue}{Georg Hirschfeld}: \emph{\textcolor{green}{Wiener Erinnerungen}}. In: \emph{\textcolor{green}{Neue
                        Freie Presse}}, Nr. 24163, 20. 12. 1931,
                  S. 31).}}}\label{K_L00612_1h}, dieſem »\label{K_L00612_2v}\edtext{letzten Wahnſinne meiner Seele}{\lemma{\textnormal{\emph{letzten … Seele}}}\Cendnote{\textnormal{Sofern
                  es als Zitat gemeint ist, könnte es auf \textcolor{blue}{Lord
                     Byron} (\emph{\textcolor{green}{The Giaour}}: »\textcolor{green}{The cherish’d madness of my heart}«, deutsch
                     »Geliebter Wahnsinn meiner Seele«, \emph{\textcolor{green}{\textcolor{blue}{Lord Byron}’s sämmtliche Werke}}. Nach
                     den Anforderungen unserer Zeit neu übersetzt von Mehreren. Siebenter Band.
                     Stuttgart: \emph{Hoffmann’sche Verlags-Buchhandlung}{ }1839, S. 96) oder \textcolor{blue}{Friedrich
                     Halm} (»\textcolor{green}{O Wahnsinn meiner
                     Seele, / Der Wirklichkeit in leerem Traum vermengt!}«, \emph{\textcolor{green}{Griseldis}}. Dramatisches Gedicht von \textcolor{blue}{Friedrich Halm}. Wien: \emph{Carl
                        Gerold}{ }1837, S. 109) zurückgehen.}}}\label{K_L00612_2h}«!\pend
           \pstart Ihr \spacefill\mbox{Peter Altenberg}\pend{}\endnumbering\briefempfaengerindex{Schnitzler, Arthur@\textsc{Schnitzler, Arthur}!zzzAltenberg, Peter@\emph{von Peter Altenberg}!1896-10-301@{{[}30.? 10. 1896{]}}|)be}\mylabel{h}  \normalsize

\doendnotes{C}
\bigskip
\vfill

\clearpage

\footnotesize

\lohead{\textsc{register}}

% Definiere theindex-Environment komplett neu ohne reledmac
\makeatletter
\renewenvironment{theindex}{%
  \section*{\indexname}%
  \setlength{\parindent}{0pt}%
  \setlength{\parskip}{0pt plus 0.3pt}%
  \let\item\@idxitem
}{%
  \clearpage
}
\makeatother

\IfFileExists{\jobname-pw.ind}{\input{\jobname-pw.ind}}{}

\end{document}

      