%% latex-korrekturansicht-vorspann.tex
%% Vorspann für die Korrekturansicht.
%% Lädt die gemeinsame Datei latex-vorspann.tex mit gesetztem Schalter.

\newif\ifkorrekturansicht
\korrekturansichttrue

\input{../tex-inputs/latex-vorspann}


               \section[Lou Andreas-Salomé an Arthur Schnitzler, {[}vor dem 17. 8. 1895{]}]{ Lou Andreas-Salomé an Arthur Schnitzler, {[}vor dem
                    17. 8. 1895{]}}\nopagebreak\mylabel{v}\rehead{ }\normalsize\beginnumbering\briefempfaengerindex{Schnitzler, Arthur@\textsc{Schnitzler, Arthur}!zzzAndreas-Salome, Lou@\emph{von Lou Andreas-Salomé}!1895-08-161@{{[}vor dem 17. 8. 1895{]}}|(be} \toendnotes[C]{\smallbreak\pagebreak[2]} \Standort{CUL, Schnitzler, B 3.}
\physDesc{Brief, 1 Blatt, 1 Seite
\newline{}Handschrift: schwarze Tinte, deutsche Kurrent
\newline{}Schnitzler: mit Bleistift datiert mit »August 95« \newline{}Ordnung: mit rotem Buntstift von unbekannter Hand nummeriert mit
                                        »6« }\pstart{}{\pb}Lieber Herr \textsc{D\textsuperscript{r}},\pend\pstart
           danke für Ihren Brief! ich werde alſo am 20\textsuperscript{ten}, Dienstag, im »\textcolor{pink}{oeſterreichiſchen
                    Hof}{}\ledrightnote{\textcolor{pink}{Österreichischer Hof}}« in \textcolor{pink}{\textsc{Salzburg}}{}\ledrightnote{\textcolor{pink}{Salzburg}} nach Ihnen fragen. Möglicherweiſe ſteige \introOben{}auch\introOben{} ich dort ab, wenn etwas frei iſt. Bis dahin lautet meine
                    Adreſſe: \textcolor{pink}{\textsc{Diessen am Ammersee, »Klosterbräu«}}{}\ledrightnote{\textcolor{pink}{Klosterbräu}}.\pend
           \pstart
           Gruß Ihnen Beiden und auf fröhliches Zuſammenſein in \textcolor{pink}{\textsc{Salzburg}}{}\ledrightnote{\textcolor{pink}{Salzburg}}!{\\[\baselineskip]}\spacefill\mbox{LouAS.}\pend
           \leftskip=0em{}\endnumbering\briefempfaengerindex{Schnitzler, Arthur@\textsc{Schnitzler, Arthur}!zzzAndreas-Salome, Lou@\emph{von Lou Andreas-Salomé}!1895-08-161@{{[}vor dem 17. 8. 1895{]}}|)be}\mylabel{h}  \normalsize

\doendnotes{C}
\bigskip
\vfill

\clearpage

\footnotesize

\lohead{\textsc{register}}

% Definiere theindex-Environment komplett neu ohne reledmac
\makeatletter
\renewenvironment{theindex}{%
  \section*{\indexname}%
  \setlength{\parindent}{0pt}%
  \setlength{\parskip}{0pt plus 0.3pt}%
  \let\item\@idxitem
}{%
  \clearpage
}
\makeatother

\IfFileExists{\jobname-pw.ind}{\input{\jobname-pw.ind}}{}

\end{document}

      