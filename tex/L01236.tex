%% latex-korrekturansicht-vorspann.tex
%% Vorspann für die Korrekturansicht.
%% Lädt die gemeinsame Datei latex-vorspann.tex mit gesetztem Schalter.

\newif\ifkorrekturansicht
\korrekturansichttrue

\input{../tex-inputs/latex-vorspann}


               \section[Adalbert Seligmann an Arthur Schnitzler, 30. 9. {[}1902?{]}]{ Adalbert Seligmann an Arthur Schnitzler,
                    30. 9. {[}1902?{]}}\nopagebreak\mylabel{v}\rehead{ }\normalsize\beginnumbering\briefempfaengerindex{Schnitzler, Arthur@\textsc{Schnitzler, Arthur}!zzzSeligmann, Adalbert Franz@\emph{von Adalbert Franz Seligmann}!1902-09-302@{30. 9. {[}1902?{]}}|(be} \toendnotes[C]{\smallbreak\pagebreak[2]} \Standort{CUL, Schnitzler, B 97.}
\physDesc{Briefkarte
\newline{}Handschrift: schwarze Tinte, deutsche Kurrent}\toendnotes[C]{\smallbreak}\pstart
           \noindent{}{\pb}Verehrter Freund!
                    Ueberbringer dieſes, ein unverſchuldet in Not geratener \label{K_L01236_1v}\edtext{\textcolor{blue}{Schriftſteller}{}\ledrightnote{→\textcolor{blue}{Ferency}}}{\lemma{\textnormal{\emph{Schriftſteller}}}\Cendnote{\textnormal{Der Karte fehlt
                        die Jahresangabe. Sofern die Person im \emph{\textcolor{green}{Tagebuch}} erwähnt ist, könnte es sich um einen nicht näher
                        bestimmten \textcolor{blue}{Ferency} handeln, der \textcolor{blue}{Schnitzler} am 30. 9. 1902 besucht.}}}\label{K_L01236_1h}, von \textcolor{blue}{\textsc{Jacobsen}}{}\ledrightnote{\textcolor{blue}{Siegfried Jacobsohn}} (\textcolor{pink}{Berlin}{}\ledrightnote{\textcolor{pink}{Berlin}}) \textcolor{blue}{\textsc{Polgar}}{}\ledrightnote{\textcolor{blue}{Alfred Polgar}} u. \textcolor{blue}{\textsc{Glücksmann}}{}\ledrightnote{\textcolor{blue}{Heinrich Glücksmann}} warm empfohlen, erſucht mich um
                    einige Worte an einen \textcolor{pink}{München}{}\ledrightnote{\textcolor{pink}{München}}er Verlag. Da
                    ich aber dort keine Beziehungen habe, wäre es Ihnen vielleicht möglich, ihm ein
                        {\pb}paar Zeilen mitzugeben. Es
                    handelt ſich ihm nur darum, daß ſeine Sachen in dem betreffenden Verlag bald
                    geleſen werden u. er in kurzer Zeit einen zuſagenden oder ablehnenden Beſcheid
                    erhält. Verzeihen Sie die Beläſtigung.\pend
           \pstart Ihr ergebenſter\spacefill\mbox{A. F. Seligmann}\pend{}\pstart
           30/IX.\pend
           \endnumbering\briefempfaengerindex{Schnitzler, Arthur@\textsc{Schnitzler, Arthur}!zzzSeligmann, Adalbert Franz@\emph{von Adalbert Franz Seligmann}!1902-09-302@{30. 9. {[}1902?{]}}|)be}\mylabel{h}  \normalsize

\doendnotes{C}
\bigskip
\vfill

\clearpage

\footnotesize

\lohead{\textsc{register}}

% Definiere theindex-Environment komplett neu ohne reledmac
\makeatletter
\renewenvironment{theindex}{%
  \section*{\indexname}%
  \setlength{\parindent}{0pt}%
  \setlength{\parskip}{0pt plus 0.3pt}%
  \let\item\@idxitem
}{%
  \clearpage
}
\makeatother

\IfFileExists{\jobname-pw.ind}{\input{\jobname-pw.ind}}{}

\end{document}

      