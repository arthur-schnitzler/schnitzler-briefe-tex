%% latex-korrekturansicht-vorspann.tex
%% Vorspann für die Korrekturansicht.
%% Lädt die gemeinsame Datei latex-vorspann.tex mit gesetztem Schalter.

\newif\ifkorrekturansicht
\korrekturansichttrue

\input{../tex-inputs/latex-vorspann}


               \section[Georg Brandes an Arthur Schnitzler, {[}3. 4. 1900?{]}]{ Georg Brandes an Arthur Schnitzler, {[}3. 4. 1900?{]}}\nopagebreak\mylabel{v}\rehead{ }\normalsize\beginnumbering\briefempfaengerindex{Schnitzler, Arthur@\textsc{Schnitzler, Arthur}!zzzBrandes, Georg@\emph{von Georg Brandes}!1900-04-033@{{[}3. 4. 1900?{]}}|(be} \toendnotes[C]{\smallbreak\pagebreak[2]} \Standort{CUL, Schnitzler, B 17.}
\physDesc{Telegramm
\newline{}maschinell\newline{}Ordnung: 1) beschnitten 2) mit Bleistift von unbekannter Hand
                                    nummeriert: »81«}\buchAbdrucke{\weitereDrucke{Georg Brandes, Arthur Schnitzler: \emph{Ein Briefwechsel}. Hg. Kurt Bergel. Bern: \emph{Francke} 1956, S. 79.} }\toendnotes[C]{\smallbreak}\pstart
           \noindent{}{\pb}\textcolor{pink}{ragusa}{}\ledrightnote{\textcolor{pink}{Opatija}} fr \textcolor{pink}{budapest}{}\ledrightnote{\textcolor{pink}{Budapest}} 6 245 21 5 10+\pend
           \pstart
           wie lange blejben{ }sie liebster in \label{K_L01028_1v}\edtext{\textcolor{pink}{abbazia}{}\ledrightnote{\textcolor{pink}{Opatija}}}{\lemma{\textnormal{\emph{abbazia}}}\Cendnote{\textnormal{\textcolor{blue}{Schnitzler} war vom
                            4. 4. 1900 bis 7. 4. 1900 in \textcolor{pink}{Opatija}. Da der Brief vom
                            3. 4. 1900 bereits einem gemeinsamen Treffen auf der Reise
                        eine definitive Absage erteilt, dürfte das Telegramm unmittelbar zuvor,
                        vermutlich ebenfalls am 3. 4. 1900 gelaufen sein.}}}\label{K_L01028_1h}?
                    fuerchte naehmlich hier zu lange auf gehalten zu werden 1
                        \spacefill\mbox{brandes}\pend
           \endnumbering\briefempfaengerindex{Schnitzler, Arthur@\textsc{Schnitzler, Arthur}!zzzBrandes, Georg@\emph{von Georg Brandes}!1900-04-033@{{[}3. 4. 1900?{]}}|)be}\mylabel{h}  \normalsize

\doendnotes{C}
\bigskip
\vfill

\clearpage

\footnotesize

\lohead{\textsc{register}}

% Definiere theindex-Environment komplett neu ohne reledmac
\makeatletter
\renewenvironment{theindex}{%
  \section*{\indexname}%
  \setlength{\parindent}{0pt}%
  \setlength{\parskip}{0pt plus 0.3pt}%
  \let\item\@idxitem
}{%
  \clearpage
}
\makeatother

\IfFileExists{\jobname-pw.ind}{\input{\jobname-pw.ind}}{}

\end{document}

      