%% latex-korrekturansicht-vorspann.tex
%% Vorspann für die Korrekturansicht.
%% Lädt die gemeinsame Datei latex-vorspann.tex mit gesetztem Schalter.

\newif\ifkorrekturansicht
\korrekturansichttrue

\input{../tex-inputs/latex-vorspann}


               \section[Oscar Blumenthal an Arthur Schnitzler, 16. 1. 1892]{ Oscar Blumenthal an Arthur Schnitzler, 16. 1. 1892}\nopagebreak\mylabel{v}\rehead{ }\normalsize\beginnumbering\briefempfaengerindex{Schnitzler, Arthur@\textsc{Schnitzler, Arthur}!zzzBlumenthal, Oskar@\emph{von Oskar Blumenthal}!1892-01-161@{16. 1. 1892}|(be} \toendnotes[C]{\smallbreak\pagebreak[2]} \Standort{CUL, Schnitzler, B 15.}
\physDesc{Brief, 1 Blatt, 1 Seite
\newline{}Handschrift  : schwarze Tinte, deutsche Kurrent\newline{}Handschrift Oskar Blumenthal: schwarze Tinte, deutsche Kurrent
\newline{}Schnitzler: mit rotem Buntstift nummeriert: »2« \newline{}Ordnung: mit Bleistift von unbekannter Hand nummeriert:
                                 »2« }\toendnotes[C]{\smallbreak}\pstart
           \noindent{}\centering{}{\pb}\textcolor{gray}{\textbf{\textcolor{brown}{LESSING-THEATER}{}\ledrightnote{\textcolor{brown}{Lessing-Theater}}}}\pend
           \pstart
           \noindent{}\centering{}\textcolor{gray}{\textbf{Director:}}{\\}\textcolor{gray}{\textbf{Dr. Oscar Blumenthal.}}\pend
           \pstart
           \noindent{}\raggedleft{}\textcolor{gray}{\textbf{\textcolor{pink}{Berlin N.W.}{}\ledrightnote{\textcolor{pink}{Berlin}}, den}}{ }16. Januar \textcolor{gray}{\textbf{189}}2.{\\}\textcolor{gray}{\textbf{\textcolor{pink}{Friedrich-Carl-Ufer}{}\ledrightnote{\textcolor{pink}{Kapelle-Ufer}}}}.\pend
           \pstart\center{}Sehr geehrter Herr Doktor!\pend\pstart
           Die von Ihnen gewünſchte kritiſche Gloſſirung Ihres intereſſanten \textcolor{green}{Schauſpiels}{}\ledrightnote{→\textcolor{green}{Das Märchen. Schauspiel in drei Aufzügen}} muß ich mir für den
                  Sommer aufſparen, da ich gegenwärtig durch eine Fülle von anderen
               dringenden Arbeiten zu ſehr in Anſpruch genommen bin. Jedenfalls rathe ich Ihnen
               nochmals, ſich \label{K_L00062_1v}\edtext{mit Herrn \textsc{\textcolor{blue}{Emanuel Reicher}{}\ledrightnote{\textcolor{blue}{Emanuel Reicher}}}}{\lemma{\textnormal{\emph{mit … Reicher}}}\Cendnote{\textnormal{Laut \emph{\textcolor{green}{Tagebuch}} schrieb \textcolor{blue}{Schnitzler} am 24. 1. 1892 an \textcolor{blue}{Reicher}.}}}\label{K_L00062_1h} (\textcolor{pink}{\textsc{Berlin O.,} Alexanderſtraße 30}{}\ledrightnote{\textcolor{pink}{Alexanderstraße}}) wegen der Aufnahme
               des \textcolor{green}{Werks}{}\ledrightnote{→\textcolor{green}{Das Märchen. Schauspiel in drei Aufzügen}} in ſein \textcolor{brown}{Ausſtellungsrepertoire}{}\ledrightnote{→\textcolor{brown}{Emanuel Reicher’s Deutsche Gastspielgesellschaft}} in Verbindung zu
               ſetzen.\pend
           \pstart
           Hochachtungsvoll{\\[\baselineskip]}\spacefill\mbox{{[}hs. Blumenthal:{]} Dr. Osc. Blumenthal}\pend
           \leftskip=0em{}\endnumbering\briefempfaengerindex{Schnitzler, Arthur@\textsc{Schnitzler, Arthur}!zzzBlumenthal, Oskar@\emph{von Oskar Blumenthal}!1892-01-161@{16. 1. 1892}|)be}\mylabel{h}  \normalsize

\doendnotes{C}
\bigskip
\vfill

\clearpage

\footnotesize

\lohead{\textsc{register}}

% Definiere theindex-Environment komplett neu ohne reledmac
\makeatletter
\renewenvironment{theindex}{%
  \section*{\indexname}%
  \setlength{\parindent}{0pt}%
  \setlength{\parskip}{0pt plus 0.3pt}%
  \let\item\@idxitem
}{%
  \clearpage
}
\makeatother

\IfFileExists{\jobname-pw.ind}{\input{\jobname-pw.ind}}{}

\end{document}

      