%% latex-korrekturansicht-vorspann.tex
%% Vorspann für die Korrekturansicht.
%% Lädt die gemeinsame Datei latex-vorspann.tex mit gesetztem Schalter.

\newif\ifkorrekturansicht
\korrekturansichttrue

\input{../tex-inputs/latex-vorspann}


               \section[Arthur Schnitzler an Wilhelm Bölsche, 27. 3. 1892]{ Arthur Schnitzler an Wilhelm Bölsche, 27. 3. 1892}\nopagebreak\mylabel{v}\rehead{ }\normalsize\beginnumbering\briefempfaengerindex{Boelsche, Wilhelm@\textsc{Bölsche, Wilhelm}!zzzSchnitzler, Arthur@\emph{von Arthur Schnitzler}!1892-03-271@{27. 3. 1892}|(be} \toendnotes[C]{\smallbreak\pagebreak[2]} \Standort{Wrocław, Biblioteka Uniwersytecka, Böl.Pis 1763.}
\physDesc{Brief, 1 Blatt, 2 Seiten
\newline{}Handschrift: schwarze Tinte, deutsche Kurrent
\newline{}Bölsche: als »Erl{[}edigt{]}« gezeichnet }\buchAbdrucke{\weitereDrucke{1) Alois Woldan: \emph{Arthur Schnitzler – Briefe an Wilhelm Bölsche.} In: \emph{Germanica Wratislaviensia} (1987) Nr. 77, S. 460.} \weitereDrucke{2) Wilhelm Bölsche: \emph{Briefwechsel. Mit Autoren der Freien Bühne}. Hg. Gerd-Hermann Susen. Berlin: \emph{Weidler} 2010, S. 678–679 (Werke und Briefe. Wissenschaftliche Ausgabe, Briefe I).} }\toendnotes[C]{\smallbreak}\pstart
           {\pb}\textcolor{pink}{\textsc{Wien I Giselastraße 11}}{}\ledrightnote{\textcolor{pink}{Bösendorferstraße}}.\hfill 27. 3. 92.\pend
           \pstart{}Sehr geehrter Herr,\pend\pstart
           beſten Dank für Ihre freundliche Antwort. Und nun wieder eine Frage, die aber
                    ohne jede Mühe in Kürze mit einem Ja oder Nein zu beantworten iſt. Ich möchte
                    Ihnen gerne eine kleine \textcolor{green}{Geſchichte}{}\ledrightnote{→\textcolor{green}{Das Himmelbett}}{ }ſtatt der \textcolor{green}{Elixire}{}\ledrightnote{\textcolor{green}{Die drei Elixire}}{ }ſchicken, die Ihnen nicht zu gefallen
                    ſcheinen, \introOben{}eine \textcolor{green}{Geſchichte}{}\ledrightnote{→\textcolor{green}{Das Himmelbett}}\introOben{}, die wohl auch beſſer in den Rahmen Ihres \textcolor{green}{Blattes}{}\ledrightnote{→\textcolor{green}{Freie Bühne für den Entwickelungskampf der Zeit}} paſſen dürfte. Nur läge mir aber ſehr viel
                    daran, daß ſie ſchon im \uline{Maiheft} der \textcolor{green}{Freien Bühne}{}\ledrightnote{\textcolor{green}{Freie Bühne für modernes Leben}}{ }{\pb}erſchiene. (Sie faſſt im ganzen 3–4 Seiten.) Wäre dies
                    – im Fall natürlich, daß Ihnen die kleine \textcolor{green}{Arbeit}{}\ledrightnote{→\textcolor{green}{Das Himmelbett}}{ }ſonſt convenirt – möglich, ſo theilen Sie mir
                    das freundlichſt durch ein \uline{Ja} mit. 2 Tage drauf
                    ſind Sie im Beſitz des \textcolor{green}{Manuscriptes}{}\ledrightnote{→\textcolor{green}{Das Himmelbett}}, das ja in einer viertel Stunde geleſen iſt.\pend
           \pstart
           Für die Erfüllung meines Erſuchens wäre ich Ihnen herzlichſt verbunden.\pend
           \pstart
           Mit aufrichtiger Hochachtung{\\[\baselineskip]}Ihr ergebner\spacefill\mbox{DrArthurSchnitzler}\pend
           \leftskip=0em{}\endnumbering\briefempfaengerindex{Boelsche, Wilhelm@\textsc{Bölsche, Wilhelm}!zzzSchnitzler, Arthur@\emph{von Arthur Schnitzler}!1892-03-271@{27. 3. 1892}|)be}\mylabel{h}  \normalsize

\doendnotes{C}
\bigskip
\vfill

\clearpage

\footnotesize

\lohead{\textsc{register}}

% Definiere theindex-Environment komplett neu ohne reledmac
\makeatletter
\renewenvironment{theindex}{%
  \section*{\indexname}%
  \setlength{\parindent}{0pt}%
  \setlength{\parskip}{0pt plus 0.3pt}%
  \let\item\@idxitem
}{%
  \clearpage
}
\makeatother

\IfFileExists{\jobname-pw.ind}{\input{\jobname-pw.ind}}{}

\end{document}

      