%% latex-korrekturansicht-vorspann.tex
%% Vorspann für die Korrekturansicht.
%% Lädt die gemeinsame Datei latex-vorspann.tex mit gesetztem Schalter.

\newif\ifkorrekturansicht
\korrekturansichttrue

\input{../tex-inputs/latex-vorspann}


               \section[Arthur Schnitzler an Hugo von Hofmannsthal, 7. 1. 1893]{ Arthur Schnitzler an Hugo von Hofmannsthal, 7. 1. 1893}\nopagebreak\mylabel{v}\rehead{ }\normalsize\beginnumbering\briefempfaengerindex{Hofmannsthal, Hugo von@\textsc{Hofmannsthal, Hugo von}!zzzSchnitzler, Arthur@\emph{von Arthur Schnitzler}!1893-01-071@{7. 1. 1893}|(be} \toendnotes[C]{\smallbreak\pagebreak[2]} \Standort{FDH, Hs-30885,32.}
\physDesc{Briefkarte
\newline{}Handschrift: schwarze Tinte, deutsche Kurrent\newline{}Ordnung: von Schnitzler mutmaßlich bei der Durchsicht der Korrespondenz
                                    1929 mit Bleistift datiert: »7. 1. 93« }\buchAbdrucke{\weitereDrucke{Hugo von Hofmannsthal, Arthur Schnitzler: \emph{Briefwechsel}. Hg. Therese Nickl und Heinrich Schnitzler. Frankfurt am Main: \emph{S. Fischer} 1964, S. 33.} }\toendnotes[C]{\smallbreak}\pstart{}{\pb}Lieber Hugo,\pend\pstart
           verſpäteten Dank für die liebenswürdige Überſendung der Ballkarten. – Morgen iſt
               nichts bei mir; alſo Dienſtag im \textcolor{pink}{\textsc{Pfob}}{}\ledrightnote{\textcolor{pink}{Café Pfob}} oder we{\geminationn} da nicht, Mittwoch auf dem \label{K_L00155_1v}\edtext{Ball}{\lemma{\textnormal{\emph{Ball}}}\Cendnote{\textnormal{Am 11. 1. 1893 fand der Juristenball statt.}}}\label{K_L00155_1h}.
               Aber da{\geminationn} werden wir gefälligſt wieder vernünftig, –
               entſchuldigen Sie das »wir«.\pend
           \pstart
           »\label{K_L00155_2v}\edtext{\textcolor{green}{\textsc{Swinburne}}{}\ledrightnote{\textcolor{green}{Algernon Charles Swinburne}}}{\lemma{\textnormal{\emph{Swinburne}}}\Cendnote{\textnormal{\textcolor{blue}{Loris}: \emph{\textcolor{green}{Charles
                        Algernon Swinburne}}. In: \emph{\textcolor{green}{Deutsche
                        Zeitung}}, Nr. 7551, 5. 1. 1893, Morgen-Ausgabe,
                     S. 1–2.}}}\label{K_L00155_2h}« war wunderſchön, eins Ihrer ſchönſten meiner Anſicht
               nach. – \pend
           \pstart
           \textcolor{blue}{\textsc{Fels}}{}\ledrightnote{\textcolor{blue}{Friedrich Michael Fels}} bereits wohler; von Ihrer Güte wird gelegentlich Gebrauch gemacht werden; ich
               ſprach mit ihm viertgradig über alles. – Waren Sie mit der \textcolor{green}{So{\geminationn}- u {\pb}\textsc{Montagszeitung}}{}\ledrightnote{\textcolor{green}{Wiener Sonn- und Montagszeitung}}{ }\label{K_L00155_3v}\edtext{\textcolor{green}{zufrieden}{}\ledrightnote{→\textcolor{green}{»Anatol« von Arthur Schnitzler}}}{\lemma{\textnormal{\emph{zufrieden}}}\Cendnote{\textnormal{\textcolor{blue}{l.a.t. [=Robert Hirschfeld]}: \emph{\textcolor{green}{»\textcolor{green}{Anatol}« von Arthur Schnitzler}}. In: \emph{\textcolor{green}{Wiener Sonn- und Montagszeitung}}, Jg. 31, Nr. 1,
                        2. 1. 1893, S. 2–3.}}}\label{K_L00155_3h}? – Nicht unmöglich iſt es,
               daß ich morgen So{\geminationn}tag nach etwelchen
               Beſuchen um 7 ins \textcolor{pink}{\textsc{Griensteidl}}{}\ledrightnote{\textcolor{pink}{Café Griensteidl}} ko{\geminationm}e. –\pend
           \pstart
           Herzlichſt der Ihre{\\[\baselineskip]}\spacefill\mbox{Arthur.}\pend
           \leftskip=0em{}\pstart
           Samſtag 7. 1. 93.\pend
           \endnumbering\briefempfaengerindex{Hofmannsthal, Hugo von@\textsc{Hofmannsthal, Hugo von}!zzzSchnitzler, Arthur@\emph{von Arthur Schnitzler}!1893-01-071@{7. 1. 1893}|)be}\mylabel{h}  \normalsize

\doendnotes{C}
\bigskip
\vfill

\clearpage

\footnotesize

\lohead{\textsc{register}}

% Definiere theindex-Environment komplett neu ohne reledmac
\makeatletter
\renewenvironment{theindex}{%
  \section*{\indexname}%
  \setlength{\parindent}{0pt}%
  \setlength{\parskip}{0pt plus 0.3pt}%
  \let\item\@idxitem
}{%
  \clearpage
}
\makeatother

\IfFileExists{\jobname-pw.ind}{\input{\jobname-pw.ind}}{}

\end{document}

      