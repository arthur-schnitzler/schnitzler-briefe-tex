%% latex-korrekturansicht-vorspann.tex
%% Vorspann für die Korrekturansicht.
%% Lädt die gemeinsame Datei latex-vorspann.tex mit gesetztem Schalter.

\newif\ifkorrekturansicht
\korrekturansichttrue

\input{../tex-inputs/latex-vorspann}


               \section[Max Burckhard an Arthur Schnitzler, 23. 8. 1911]{ Max Burckhard an Arthur Schnitzler, 23. 8. 1911}\nopagebreak\mylabel{v}\rehead{ }\normalsize\beginnumbering\briefempfaengerindex{Schnitzler, Arthur@\textsc{Schnitzler, Arthur}!zzzBurckhard, Max Eugen@\emph{von Max Eugen Burckhard}!1911-08-231@{23. 8. 1911}|(be} \toendnotes[C]{\smallbreak\pagebreak[2]} \Standort{CUL, Schnitzler, B 20.}
\physDesc{Brief, 1 Blatt, 1 Seite
\newline{}Schreibmaschine
\newline{}Handschrift: schwarze Tinte (\noindent{}Unterschrift)
\newline{}Schnitzler: mit rotem Buntstift eine Unterstreichung \newline{}Ordnung: mit Bleistift von unbekannter Hand nummeriert: »28« }\toendnotes[C]{\smallbreak}\pstart
           \noindent{}{\pb}\textcolor{gray}{\textbf{\textsc{D\textsuperscript{r.} Max
                                    Burckhard}}}\hfill \textcolor{gray}{\textbf{\textcolor{pink}{Wien, I. Lichtenfelsgasse 7}{}\ledrightnote{\textcolor{pink}{Lichtenfelsgasse}}}}\pend
           \pstart
           \raggedleft{}\textcolor{gray}{\textbf{\textcolor{pink}{St. Gilgen}{}\ledrightnote{\textcolor{pink}{St. Gilgen}}}}{ }23. 8. 11.\pend
           \pstart{}Sehr verehrter lieber Herr Doctor!\pend\pstart
           Herzlichsten Dank für die Zusendung des »\textcolor{green}{weiten
                        Landes}{}\ledrightnote{\textcolor{green}{Das weite Land. Tragikomödie in fünf Akten}}«, das mich natürlich, wie alles von Ihnen sehr interessiert hat
                    und das auch durch die Personen sehr stark auf mich gewirkt hat. Freilich hat es
                    mich jetzt sehr traurig ergriffen, da das \textcolor{blue}{Vorbild}{}\ledrightnote{→\textcolor{blue}{Theodor Christomannos}} Dr. \textcolor{green}{Aigners}{}\ledrightnote{→\textcolor{green}{Das weite Land. Tragikomödie in fünf Akten}} inzwischen von uns gegangen ist, und ich
                    diesem prächtigen Menschen von Herzen zugethan war. Ich habe übrigens zufällig
                    noch eine andere gute Bekannte in dem Stück gefunden (wenn auch Sie sie
                    vielleicht gar nicht als dieselbe Person kennen); im Leben hat sich nemlich die
                    »kritische Scene« zwischen \textcolor{green}{Erna}{}\ledrightnote{→\textcolor{green}{Das weite Land. Tragikomödie in fünf Akten}} und \textcolor{green}{Türk}{}\ledrightnote{→\textcolor{green}{Das weite Land. Tragikomödie in fünf Akten}}
                    (unter welchem Spitznamen Ihnen wol \textcolor{blue}{Christomanos}{}\ledrightnote{\textcolor{blue}{Theodor Christomannos}} auch bekannt worden sein wird) abgespielt. Jedenfalls
                    glich sie \textcolor{green}{Erna}{}\ledrightnote{→\textcolor{green}{Das weite Land. Tragikomödie in fünf Akten}}{ }sehr in ihrer Art und obwol wir uns nur sehr
                    selten sprachen, waren wir doch sehr gut (»im guten Sinne«). Inzwischen wird sie
                    wol auch älter geworden sein, was ja bekanntlich den Menschen gewöhnlich nicht
                    zum Vorteil gereicht.\pend
           \pstart
           Sehr leid war es mir, daß ich heuer nicht mehr von Ihrer Anwesenheit haben
                    konnte. Mit Handkuss an die verehrte gnädige \textcolor{blue}{Frau}{}\ledrightnote{→\textcolor{blue}{Olga Schnitzler}} und herzlichsten Grüßen Ihr treu ergebener\pend
           \pstart \spacefill\mbox{{[}hs.:{]} D\textsuperscript{r}Burckhard}\pend{}\endnumbering\briefempfaengerindex{Schnitzler, Arthur@\textsc{Schnitzler, Arthur}!zzzBurckhard, Max Eugen@\emph{von Max Eugen Burckhard}!1911-08-231@{23. 8. 1911}|)be}\mylabel{h}  \normalsize

\doendnotes{C}
\bigskip
\vfill

\clearpage

\footnotesize

\lohead{\textsc{register}}

% Definiere theindex-Environment komplett neu ohne reledmac
\makeatletter
\renewenvironment{theindex}{%
  \section*{\indexname}%
  \setlength{\parindent}{0pt}%
  \setlength{\parskip}{0pt plus 0.3pt}%
  \let\item\@idxitem
}{%
  \clearpage
}
\makeatother

\IfFileExists{\jobname-pw.ind}{\input{\jobname-pw.ind}}{}

\end{document}

      