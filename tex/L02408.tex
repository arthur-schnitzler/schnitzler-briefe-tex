%% latex-korrekturansicht-vorspann.tex
%% Vorspann für die Korrekturansicht.
%% Lädt die gemeinsame Datei latex-vorspann.tex mit gesetztem Schalter.

\newif\ifkorrekturansicht
\korrekturansichttrue

\input{../tex-inputs/latex-vorspann}


               \section[Hugo Hofmannsthal an Arthur Schnitzler, 18. 1. 1924]{ Hugo Hofmannsthal an Arthur Schnitzler, 18. 1. 1924}\nopagebreak\mylabel{v}\rehead{ }\normalsize\beginnumbering\briefempfaengerindex{Schnitzler, Arthur@\textsc{Schnitzler, Arthur}!zzzHofmannsthal, Hugo von@\emph{von Hugo von Hofmannsthal}!1924-01-181@{18. 1. 1924}|(be} \toendnotes[C]{\smallbreak\pagebreak[2]} \Standort{CUL, Schnitzler, B 43.}
\physDesc{Postkarte
\newline{}Handschrift: schwarze Tinte, lateinische Kurrent\newline{}Versand: Stempel: »\nobreak{}\oindex{Rodaun@\textbf{Rodaun}, \emph{Teil eines besiedelten Ortes (A.BSOX)}|pwk}\textcolor{gray}{Rod}{[}aun{]}\nobreak{}«.  \newline{}Ordnung: 1) mit Bleistift von unbekannter Hand nummeriert: »\strikeout{384}« 2) mit Bleistift von unbekannter Hand nummeriert:
                                    »373«}\buchAbdrucke{\weitereDrucke{Hugo von Hofmannsthal, Arthur Schnitzler: \emph{Briefwechsel}. Hg. Therese Nickl und Heinrich Schnitzler. Frankfurt am Main: \emph{S. Fischer} 1964, S. 298.} }\toendnotes[C]{\smallbreak}\pstart{}{\pb}Herrn D\textsuperscript{r} Arthur Schnitzler\pend{}\pstart{}\textcolor{pink}{Wien}{}\ledrightnote{\textcolor{pink}{Wien}}\pend{}\pstart{}\textcolor{pink}{XVIII Sternwartestrasse 71}{}\ledrightnote{\textcolor{pink}{Sternwartestraße}}\pend{}{\bigskip}\pstart
           \noindent{}\textcolor{gray}{\textbf{{\pb}\textcolor{pink}{\textsc{Rodaun}}{}\ledrightnote{\textcolor{pink}{Rodaun}}}}\hfill 18 I 24.\pend
           \pstart
           \textcolor{gray}{\textbf{B. \textcolor{pink}{WIEN}{}\ledrightnote{\textcolor{pink}{Wien}}}}\pend
           \pstart{}mein lieber Arthur\pend\pstart
           um unser \label{K_L02408_1v}\edtext{Gespräch}{\lemma{\textnormal{\emph{Gespräch}}}\Cendnote{\textnormal{vgl. A. S.: \emph{Tagebuch}, 11. 1. 1924}}}\label{K_L02408_1h} noch für mich allein zu verlängern, wollte ich gestern abends die »\textcolor{green}{Große Scene}{}\ledrightnote{\textcolor{green}{Große Szene}}« lesen – aber ich muss durch ein
               Versehen seinerzeit diesen Band (\textcolor{green}{Comödie der Worte}{}\ledrightnote{\textcolor{green}{Komödie der Worte. Drei Einakter}})
                  {\pb}nicht beko{\geminationm}en haben! Haben Sie vielleicht ein entbehrliches
               Exemplar? Nämlich auch in meinen Bänden \introOben{}Ihrer\introOben{}{ }\label{K_L02408_2v}\edtext{\textcolor{green}{ges. Theaterstücke}{}\ledrightnote{→\textcolor{green}{Gesammelte Werke}}}{\lemma{\textnormal{\emph{ges. Theaterstücke}}}\Cendnote{\textnormal{1912 erschien \emph{\textcolor{green}{Die gesammelten Werke}}
                  mit vier Bänden \emph{\textcolor{green}{Die Theaterstücke}}. Anlässlich des
                  60. Geburtstages wurde 1922 die Ausgabe um einen \textcolor{green}{Ergänzungsband} erweitert, der die Stücke
                  seit 1912 umfasste. \emph{\textcolor{green}{Die gesammelten
                     Werke}} sind nicht in \textcolor{blue}{Hofmannsthal}s
                  Nachlass mit seiner Bibliothek überliefert.}}}\label{K_L02408_2h} deren ich 4 habe, finde ich
               diese Einacterreihe nicht! – Zum Ersatz habe ich da{\geminationn} das
                  »\textcolor{green}{Weite Land}{}\ledrightnote{\textcolor{green}{Das weite Land. Tragikomödie in fünf Akten}}« gelesen u. mit sehr großem
               Eindruck. Sie haben damals offenbar alles \uline{Detail} sehr
                  \label{K_L02408_3v}\edtext{eindrucksvoll vorgelesen}{\lemma{\textnormal{\emph{eindrucksvoll vorgelesen}}}\Cendnote{\textnormal{Sofern sie stattgefunden hat, lässt sich
                  diese Lesung nicht datieren.}}}\label{K_L02408_3h}, auf der Bühne habe ich es nie gesehen, u. so
               war mir nicht gegenwärtig gewesen, wie sehr dieses complexe Ganze durch die
               erstaunliche Gestalt des \textcolor{green}{Hofreiter}{}\ledrightnote{→\textcolor{green}{Das weite Land. Tragikomödie in fünf Akten}}{ }\uline{zusa{\geminationm}engehalten} wird. –
               Das \label{T_L02408_1v}\edtext{ganze genre gehört nur Ihnen, u.
               ist höchst interressant}{\lemma{\textnormal{\emph{ganze … interressant}}}\Cendnote{\textnormal{quer am linken
                  Rand}}}\label{T_L02408_1h}\pend
           \pstart
           \label{T_L02408_2v}\edtext{Von Herzen Ihr{\\[\baselineskip]}\spacefill\mbox{Hugo.}}{\lemma{\textnormal{\emph{Von Herzen IhrHugo.}}}\Cendnote{\textnormal{Grußformel quer am rechten Rand}}}\label{T_L02408_2h}\pend
           \leftskip=0em{}\pstart
           \noindent{}\label{T_L02408_3v}\edtext{PS. Eben finde ich \textcolor{green}{B\textsuperscript{d} V}{}\ledrightnote{\textcolor{green}{Die Theaterstücke. Ergänzungsband V}} der Theaterstücke! Er war
                     verstellt.}{\lemma{\textnormal{\emph{PS. … verstellt.}}}\Cendnote{\textnormal{quer am linken Rand der
                     ersten Seite}}}\label{T_L02408_3h}\pend
           \endnumbering\briefempfaengerindex{Schnitzler, Arthur@\textsc{Schnitzler, Arthur}!zzzHofmannsthal, Hugo von@\emph{von Hugo von Hofmannsthal}!1924-01-181@{18. 1. 1924}|)be}\mylabel{h}  \normalsize

\doendnotes{C}
\bigskip
\vfill

\clearpage

\footnotesize

\lohead{\textsc{register}}

% Definiere theindex-Environment komplett neu ohne reledmac
\makeatletter
\renewenvironment{theindex}{%
  \section*{\indexname}%
  \setlength{\parindent}{0pt}%
  \setlength{\parskip}{0pt plus 0.3pt}%
  \let\item\@idxitem
}{%
  \clearpage
}
\makeatother

\IfFileExists{\jobname-pw.ind}{\input{\jobname-pw.ind}}{}

\end{document}

      