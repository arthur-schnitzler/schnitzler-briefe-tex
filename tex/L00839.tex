%% latex-korrekturansicht-vorspann.tex
%% Vorspann für die Korrekturansicht.
%% Lädt die gemeinsame Datei latex-vorspann.tex mit gesetztem Schalter.

\newif\ifkorrekturansicht
\korrekturansichttrue

\input{../tex-inputs/latex-vorspann}


               \section[Hermann Bahr und Therese Strisower an Arthur Schnitzler, {[}26.?{]} 8. 1898]{ Hermann Bahr und Therese Strisower an Arthur Schnitzler,
               {[}26.?{]} 8. 1898}\nopagebreak\mylabel{v}\rehead{ }\normalsize\beginnumbering\briefempfaengerindex{Schnitzler, Arthur@\textsc{Schnitzler, Arthur}!zzzHorn, Therese@\emph{von Therese Horn}!1898-08-261@{{[}26.?{]} 8. 1898}|(be}\briefempfaengerindex{Schnitzler, Arthur@\textsc{Schnitzler, Arthur}!zzzBahr, Hermann@\emph{von Hermann Bahr}!1898-08-261@{{[}26.?{]} 8. 1898}|(be} \toendnotes[C]{\smallbreak\pagebreak[2]} \Standort{CUL, Schnitzler, B 5b.}
\physDesc{Bildpostkarte
\newline{}Handschrift Hermann Bahr: Bleistift, deutsche Kurrent\newline{}Handschrift Therese Horn: Bleistift, lateinische Kurrent\newline{}Versand: 1) Stempel: »\nobreak{}\oindex{Carbonin@\textbf{Carbonin}, \emph{https://www.geonames.org/ontologyP.PPL}|pwk}Schluderb{[}ach{]}, 2\textcolor{gray}{×} 8 98\nobreak{}«.  2) Stempel: »\nobreak{}\oindex{IX., Alsergrund@\textbf{IX., Alsergrund}, \emph{Bezirk (A.BZK)}|pwk}Wien 9/3, 28. 8. 98, 9.V, Bestellt\nobreak{}«. \newline{}Ordnung: mit Bleistift von unbekannter Hand nummeriert: »59« }\buchAbdrucke{\weitereDrucke{Hermann Bahr, Arthur Schnitzler: \emph{Briefwechsel, Aufzeichnungen, Dokumente (1891–1931)}. Hg. Kurt Ifkovits und Martin Anton Müller. Göttingen: \emph{Wallstein} 2018, S. 161.} }\toendnotes[C]{\smallbreak}\pstart{}{\pb}Herrn \textsc{D\textsuperscript{r} Arthur Schnitzler}\pend{}\pstart{}\textsc{\textcolor{pink}{Wien IX}{}\ledrightnote{\textcolor{pink}{Wien}}}\pend{}\pstart{}\textsc{\textcolor{pink}{Frankgasse 1}{}\ledrightnote{\textcolor{pink}{Frankgasse}}}\pend{}{\bigskip}\pstart
           \noindent{}\centering{}\textcolor{gray}{\textbf{{\pb}\textcolor{pink}{Landro}{}\ledrightnote{\textcolor{pink}{Höhlenstein}} mit \textcolor{pink}{Monte Cristallo}{}\ledrightnote{\textcolor{pink}{Monte Cristallo}}.}}\pend
           \pstart
           {\pb}Warum biſt Du nicht hier? Telegrafiere ſofort\pend
           \pstart Deinem \spacefill\mbox{Hermann}\pend{}\pstart
           {[}hs. Horn:{]} Warum waren Sie nicht hier? \label{T_L00839_1v}\edtext{Telegrafieren Sie sofort Ihrer Risa,}{\lemma{\textnormal{\emph{Telegrafieren … Risa,}}}\Cendnote{\textnormal{quer am rechten Rand}}}\label{T_L00839_1h}{ }\label{T_L00839_2v}\edtext{aber schon nach \textcolor{pink}{Unterach}{}\ledrightnote{\textcolor{pink}{Unterach am Attersee}}.}{\lemma{\textnormal{\emph{aber … Unterach.}}}\Cendnote{\textnormal{am oberen Rand auf dem
                  Kopf}}}\label{T_L00839_2h}\pend
           \endnumbering\briefempfaengerindex{Schnitzler, Arthur@\textsc{Schnitzler, Arthur}!zzzHorn, Therese@\emph{von Therese Horn}!1898-08-261@{{[}26.?{]} 8. 1898}|)be}\briefempfaengerindex{Schnitzler, Arthur@\textsc{Schnitzler, Arthur}!zzzBahr, Hermann@\emph{von Hermann Bahr}!1898-08-261@{{[}26.?{]} 8. 1898}|)be}\mylabel{h}  \normalsize

\doendnotes{C}
\bigskip
\vfill

\clearpage

\footnotesize

\lohead{\textsc{register}}

% Definiere theindex-Environment komplett neu ohne reledmac
\makeatletter
\renewenvironment{theindex}{%
  \section*{\indexname}%
  \setlength{\parindent}{0pt}%
  \setlength{\parskip}{0pt plus 0.3pt}%
  \let\item\@idxitem
}{%
  \clearpage
}
\makeatother

\IfFileExists{\jobname-pw.ind}{\input{\jobname-pw.ind}}{}

\end{document}

      