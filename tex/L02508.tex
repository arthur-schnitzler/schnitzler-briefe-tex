%% latex-korrekturansicht-vorspann.tex
%% Vorspann für die Korrekturansicht.
%% Lädt die gemeinsame Datei latex-vorspann.tex mit gesetztem Schalter.

\newif\ifkorrekturansicht
\korrekturansichttrue

\input{../tex-inputs/latex-vorspann}


               \section[Thomas Mann an Arthur Schnitzler, 1. 2. 1929]{ Thomas Mann an Arthur Schnitzler, 1. 2. 1929}\nopagebreak\mylabel{v}\rehead{ }\normalsize\beginnumbering\briefempfaengerindex{Schnitzler, Arthur@\textsc{Schnitzler, Arthur}!zzzMann, Thomas@\emph{von Thomas Mann}!1929-02-011@{1. 2. 1929}|(be} \toendnotes[C]{\smallbreak\pagebreak[2]} \Standort{CUL, Schnitzler, B 67.}
\physDesc{Brief, 1 Blatt, 2 Seiten
\newline{}Schreibmaschine
\newline{}Handschrift: schwarze Tinte (\noindent{}Unterschrift)
\newline{}Schnitzler: 1) mit Datum einer nicht erhaltenen Antwort mit rotem Buntstift
                                 versehen oder eine Wiederholung von Monat und Jahr mit
                                 vorangestellter Jahreszahl: »29 II–« und beschrieben: »\textsc{\textcolor{pink}{Amerika}, Liste}« 2) mit rotem Buntstift mehrere Unterstreichungen}\buchAbdrucke{\weitereDrucke{Hertha Krotkoff: \emph{Arthur Schnitzler – Thomas Mann: Briefe.} In: \emph{Modern Austrian Literature}, Jg. 7 (1974) Nr. 1/2, S. 26.} }\toendnotes[C]{\smallbreak}\pstart
           \noindent{}{\pb}\textcolor{gray}{\textbf{DR. THOMAS MANN}}\hfill \textcolor{gray}{\textbf{\textcolor{pink}{MÜNCHEN}{}\ledrightnote{\textcolor{pink}{München}} den}}{ }1. II. 29.\pend
           \pstart
           \raggedleft{}\textcolor{gray}{\textbf{\textcolor{pink}{POSCHINGERSTR. 1}{}\ledrightnote{\textcolor{pink}{Poschingerstraße}}}}\pend
           \pstart{}Lieber, verehrter Herr Doktor Schnitzler,\pend\pstart
           Haben Sie Dank für Ihre Zeilen! Ich war glücklich, sie zu erhalten, denn ich hatte
               ein schlechtes Gewissen, weil ich in \textcolor{pink}{Wien}{}\ledrightnote{\textcolor{pink}{Wien}} nicht
               versucht habe, mich mit Ihnen in Verbindung zu setzen, Sie zu sehen, zu sprechen. Ich
               brauche kaum zu sagen, warum es nicht geschah. Es war Scheu vor dem \label{K_L02508_1v}\edtext{schrecklichen Kummer}{\lemma{\textnormal{\emph{schrecklichen Kummer}}}\Cendnote{\textnormal{der Tod der Tochter \textcolor{blue}{Lili} am 26. 7. 1928}}}\label{K_L02508_1h}, den das Schicksal
               Ihnen kürzlich zugefügt hat, und von dem wir alle mit Ihnen so tief erschüttert
               wurden. Ich wusste nicht, ob Sie aufgelegt seien, meinen Besuch oder irgend welchen
               anderen zu empfangen. Aber Ihre Zeilen lassen mich hoffen, dass ich bald wieder
               einmal die Freude haben werde, Ihnen die Hand zu drücken.\pend
           \pstart
           Nun zu Ihrer Anfrage. Ich habe die Aufforderung des »\textcolor{brown}{Book
                  of the Month Club}{}\ledrightnote{\textcolor{brown}{Book of The Month Club}}« erhalten und zustimmend beantwortet. Die Leute stellen
               sich ein präsentables Komitee zusammen, und da sie notorisch viel Geld haben, finde
               ich nichts Böses darin, mir meinen Namen honorieren zu lassen, zumal es ja in der Tat
               nicht ganz ausschliesslich der Name ist, sondern ich durchaus gesonnen bin, ihnen von
               Zeit zu Zeit einen Brief zu schreiben und sie auf deutsche Bücher hinzuweisen, die
               für ihre Veröffentlichungen in Betracht kommen. Das ist eine geistige Teilnahme, die
               sie belohnen {\pb}dürfen. Natürlich würde ich
               mich freuen, wenn Sie diesen Gesichtspunkt anerkennen würden und auch dabei
               wären.\pend
           \pstart
           Seien Sie recht herzlich und verehrungsvoll begrüsst{\\[\baselineskip]} von Ihrem ergebenen{\\[\baselineskip]}\spacefill\mbox{{[}hs.:{]} Thomas Mann}\pend
           \leftskip=0em{}\endnumbering\briefempfaengerindex{Schnitzler, Arthur@\textsc{Schnitzler, Arthur}!zzzMann, Thomas@\emph{von Thomas Mann}!1929-02-011@{1. 2. 1929}|)be}\mylabel{h}  \normalsize

\doendnotes{C}
\bigskip
\vfill

\clearpage

\footnotesize

\lohead{\textsc{register}}

% Definiere theindex-Environment komplett neu ohne reledmac
\makeatletter
\renewenvironment{theindex}{%
  \section*{\indexname}%
  \setlength{\parindent}{0pt}%
  \setlength{\parskip}{0pt plus 0.3pt}%
  \let\item\@idxitem
}{%
  \clearpage
}
\makeatother

\IfFileExists{\jobname-pw.ind}{\input{\jobname-pw.ind}}{}

\end{document}

      