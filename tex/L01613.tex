%% latex-korrekturansicht-vorspann.tex
%% Vorspann für die Korrekturansicht.
%% Lädt die gemeinsame Datei latex-vorspann.tex mit gesetztem Schalter.

\newif\ifkorrekturansicht
\korrekturansichttrue

\input{../tex-inputs/latex-vorspann}


               \section[Arthur Schnitzler an Richard Beer-Hofmann, 17. 7. 1906]{ Arthur Schnitzler an Richard Beer-Hofmann, 17. 7. 1906}\nopagebreak\mylabel{v}\rehead{ }\normalsize\beginnumbering\briefempfaengerindex{Beer-Hofmann, Richard@\textsc{Beer-Hofmann, Richard}!zzzSchnitzler, Arthur@\emph{von Arthur Schnitzler}!1906-07-171@{17. 7. 1906}|(be} \toendnotes[C]{\smallbreak\pagebreak[2]} \Standort{YCGL, MSS 31.}
\physDesc{Bildpostkarte
\newline{}Handschrift: Bleistift, deutsche Kurrent\newline{}Versand: Stempel: »\nobreak{}\oindex{Helsingør@\textbf{Helsingør}, \emph{Besiedelter Ort (A.BSO)}|pwk}Hels{[}ingør{]}, 17. 7. 06, 2–5E\nobreak{}«.  \newline{}Ordnung: mit Bleistift von unbekannter Hand datiert:
                                    »17. 7.« }\buchAbdrucke{\weitereDrucke{Arthur Schnitzler, Richard Beer-Hofmann: \emph{Briefwechsel 1891–1931}. Hg. Konstanze Fliedl. Wien, Zürich: \emph{Europaverlag} 1992, S. 179.} }\pstart{}{\pb}\textsc{Dr. Richard}\pend{}\pstart{}\textsc{Beer-Hofmann}\pend{}\pstart{}\textcolor{pink}{\textsc{Rodaun}}{}\ledrightnote{\textcolor{pink}{Rodaun}}\pend{}\pstart{}\textsc{bei \textcolor{pink}{Wien}{}\ledrightnote{\textcolor{pink}{Wien}}}. \pend{}\pstart{}\textcolor{pink}{\textsc{Liesingerstraße 1}}{}\ledrightnote{\textcolor{pink}{Liesingerstraße}}.\pend{}\pstart{}\textcolor{pink}{Austria}{}\ledrightnote{\textcolor{pink}{Österreich}}\pend{}{\bigskip}\pstart
           \noindent{}{\pb}\textcolor{gray}{\textbf{\textcolor{green}{Hamlets}{}\ledrightnote{\textcolor{green}{Hamlet}} Statue}}\pend
           \pstart
           {\pb}Ihre \textcolor{pink}{Heiligenkreuz}{}\ledrightnote{\textcolor{pink}{Heiligenkreuz}}erkarte beko{\geminationm}en. Herzlichen Dank u
               Gruſs. Wohin gehn Sie? Oder bleiben Sie? Wir dürften noch 3 Wochen hier verweilen;
               ich arbeite. Sind mit Natur, Kunſt und Küche ſehr zufrieden. \textcolor{blue}{Heini}{}\ledrightnote{\textcolor{blue}{Heinrich Schnitzler}} zeichnet täglich einen Block voll. Laſſen Sie was von
               ſich hören. Eine Karte, auf der die Hälfte von mir iſt, genügt mir nicht. Herzlichſt
               Ihr \spacefill\mbox{A.}\pend
           \pstart
           \noindent{}{\pb}Viele Grüße von uns allen an Sie alle.\pend
           \endnumbering\briefempfaengerindex{Beer-Hofmann, Richard@\textsc{Beer-Hofmann, Richard}!zzzSchnitzler, Arthur@\emph{von Arthur Schnitzler}!1906-07-171@{17. 7. 1906}|)be}\mylabel{h}  \normalsize

\doendnotes{C}
\bigskip
\vfill

\clearpage

\footnotesize

\lohead{\textsc{register}}

% Definiere theindex-Environment komplett neu ohne reledmac
\makeatletter
\renewenvironment{theindex}{%
  \section*{\indexname}%
  \setlength{\parindent}{0pt}%
  \setlength{\parskip}{0pt plus 0.3pt}%
  \let\item\@idxitem
}{%
  \clearpage
}
\makeatother

\IfFileExists{\jobname-pw.ind}{\input{\jobname-pw.ind}}{}

\end{document}

      