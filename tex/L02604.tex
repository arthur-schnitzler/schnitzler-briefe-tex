%% latex-korrekturansicht-vorspann.tex
%% Vorspann für die Korrekturansicht.
%% Lädt die gemeinsame Datei latex-vorspann.tex mit gesetztem Schalter.

\newif\ifkorrekturansicht
\korrekturansichttrue

\input{../tex-inputs/latex-vorspann}


               \section[Paul Goldmann an Arthur Schnitzler, {[}Mitte? August 1894{]}]{ Paul Goldmann an Arthur Schnitzler, {[}Mitte? August 1894{]}}\nopagebreak\mylabel{v}\rehead{ }\normalsize\beginnumbering\briefempfaengerindex{Schnitzler, Arthur@\textsc{Schnitzler, Arthur}!zzzGoldmann, Paul@\emph{von Paul Goldmann}!1894-08-151@{{[}Mitte? August 1894{]}}|(be} \toendnotes[C]{\smallbreak\pagebreak[2]} \Standort{DLA, A:Schnitzler, HS.NZ85.1.3164.}
\physDesc{Telegramm
\newline{}maschinell
\newline{}Schnitzler: mit Bleistift Vermerk »\textsc{August
                                          94}« \newline{}Ordnung: beschnitten }\toendnotes[C]{\smallbreak}\pstart
           \noindent{}{\pb}ich \label{K_L02604-1v}\edtext{komme \textcolor{pink}{ischl}{}\ledrightnote{\textcolor{pink}{Bad Ischl}}}{\lemma{\textnormal{\emph{komme ischl}}}\Cendnote{\textnormal{Der Brief \textcolor{blue}{Schnitzler}s an
                     \textcolor{blue}{Beer-Hofmann} vom [18. 8. 1894] dürfte in unmittelbarer zeitlicher Nähe
                  zu diesem Telegramm verfasst sein, weil die Antwort auf das Telegramm skizziert
                  wird.}}}\label{K_L02604-1h} erbitte letztes einverstaendniss telegra\damage{m}{ }\textcolor{pink}{genf}{}\ledrightnote{\textcolor{pink}{Genf}} poste restante\pend
           \pstart \spacefill\mbox{= goldmann +}\pend{}\endnumbering\briefempfaengerindex{Schnitzler, Arthur@\textsc{Schnitzler, Arthur}!zzzGoldmann, Paul@\emph{von Paul Goldmann}!1894-08-151@{{[}Mitte? August 1894{]}}|)be}\mylabel{h}  \normalsize

\doendnotes{C}
\bigskip
\vfill

\clearpage

\footnotesize

\lohead{\textsc{register}}

% Definiere theindex-Environment komplett neu ohne reledmac
\makeatletter
\renewenvironment{theindex}{%
  \section*{\indexname}%
  \setlength{\parindent}{0pt}%
  \setlength{\parskip}{0pt plus 0.3pt}%
  \let\item\@idxitem
}{%
  \clearpage
}
\makeatother

\IfFileExists{\jobname-pw.ind}{\input{\jobname-pw.ind}}{}

\end{document}

      