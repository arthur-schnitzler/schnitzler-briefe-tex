%% latex-korrekturansicht-vorspann.tex
%% Vorspann für die Korrekturansicht.
%% Lädt die gemeinsame Datei latex-vorspann.tex mit gesetztem Schalter.

\newif\ifkorrekturansicht
\korrekturansichttrue

\input{../tex-inputs/latex-vorspann}


               \section[Robert Adam an Arthur Schnitzler, 13. 2. 1920]{ Robert Adam an Arthur Schnitzler, 13. 2. 1920}\nopagebreak\mylabel{v}\rehead{ }\normalsize\beginnumbering\briefempfaengerindex{Schnitzler, Arthur@\textsc{Schnitzler, Arthur}!zzzAdam, Robert@\emph{von Robert Adam}!1920-02-131@{13. 2. 1920}|(be} \toendnotes[C]{\smallbreak\pagebreak[2]} \Standort{CUL, Schnitzler, B 1.}
\physDesc{Brief, 1 Blatt, 4 Seiten
\newline{}Handschrift: blaue Tinte, deutsche Kurrent
\newline{}Schnitzler: 1) mit Bleistift beschriftet: »\textsc{Adam}« 2) mit rotem Buntstift vereinzelte Unterstreichungen\newline{}Ordnung: mit Bleistift von unbekannter Hand nummeriert:
                                        »14« }\Standort{Wien, Österreichische Nationalbibliothek, Cod.ser. 52.268, 57 recto, 61.}
\physDesc{handschriftliche Abschrift
\newline{}Handschrift: schwarze Tinte, Gabelsberger Kurzschrift}\Standort{Wien, Österreichische Nationalbibliothek, Cod.ser. 52.268, 57 recto, 61.}
\physDesc{maschinelle Abschrift
\newline{}Schreibmaschine}\toendnotes[C]{\smallbreak}\pstart
           \raggedleft{}{\pb}\textcolor{pink}{Wien}{}\ledrightnote{\textcolor{pink}{Wien}}, am 13. Februar 1920\pend
           \pstart\center{}Hochverehrter Herr Doktor!\pend\pstart
           Für Ihre »\textcolor{green}{Schweſtern}{}\ledrightnote{\textcolor{green}{Die Schwestern oder Casanova in Spa. Lustspiel in Versen}}«, die mir geſtern zukamen,
                    meinen beſten Dank! Ich habe ſie ſofort geleſen, ſehr begierig, Sie wieder, nach
                    langer Zeit, in Versen reden zu hören. Der Vers iſt mir, dem Mann der alten
                    Schule, doch immer das Berufsgewand des Dichters, nicht ein Salonanzug, und mir
                    will ſcheinen, daß man im Berufsgewand am freieſten und förderlichſten Arbeit
                    leiſtet. Ihre Verſe fließen wundervoll und leihen Ihren Gedanken neuen Reiz,
                    ohne ihnen die charakteriſtiſchen Eigentümlichkeiten Ihrer Proſa zu nehmen. Ich
                    sage dies, obwohl ich den Blankvers, der nur der einſilbigen \textcolor{pink}{engliſchen}{}\ledrightnote{\textcolor{pink}{England}} oder noch der verſchleifenden \textcolor{pink}{italieniſchen}{}\ledrightnote{\textcolor{pink}{Italien}}{ }Sprache
                    angemeſſen iſt, im Deutſchen ſonſt herzlich haſſe (was ich Ihnen ſchon geſagt
                    habe); denn der deutſche Blankvers, mei{\pb}ſterhaft gehandhabt, das iſt
                    gemeiſtert, das iſt oft gebrochen, gezerrt, gepreßt, iſt ein unerträgliches
                    Geſchöpf, eine endloſe Melodie, ein ſtätiges Meeresrauſchen. (Mir iſt dies jetzt
                    wieder klar geworden, da ich ein gerade erſchienenes Buch eines Vetters: »\textcolor{green}{Träume auf der Aſphodelosinſel}{}\ledrightnote{\textcolor{green}{Träume auf der Asphodelosinsel}}«, ein
                    philoſophiſches Troſtbüchlein in Verſen von \textcolor{blue}{\textsc{Otto Fürth}}{}\ledrightnote{\textcolor{blue}{Otto Fürth}}, leſe, ein ſehr klar und geiſtvoll
                    geſchriebenes Buch, deſſen Blankvers blitz und blank iſt und deshalb endlos wogt
                    und flutet: was ja im konkreten Falle vielleicht nicht übel iſt, da es zur
                    Stärkung der Illuſion, man ſei auf einer Inſel, gewiß beiträgt. Der deutſche
                    Vers \textsc{par excellence}{ }ſcheint mir doch der Knittelvers zu ſein.)\pend
           \pstart
           Ich bewundere Ihre großartige Charakteriſierung des \textcolor{green}{Caſanova}{}\ledrightnote{→\textcolor{green}{Die Schwestern oder Casanova in Spa. Lustspiel in Versen}}-Milieus; jede der Geſtalten der Komödie iſt
                    auf \textcolor{green}{Caſanova}{}\ledrightnote{→\textcolor{green}{Die Schwestern oder Casanova in Spa. Lustspiel in Versen}} abgeſtellt,
                    dazu geboren, einmal mit ihm zuſammenzutreffen, ohne das Abenteuer \textcolor{green}{Caſanova}{}\ledrightnote{→\textcolor{green}{Die Schwestern oder Casanova in Spa. Lustspiel in Versen}} nicht zu denken.
                    Und dabei tragen die meiſten einen oder den andern Zug, den \textcolor{green}{Caſanova}{}\ledrightnote{→\textcolor{green}{Die Schwestern oder Casanova in Spa. Lustspiel in Versen}} gezeigt hat oder dereinſt
                    zeigen wird; wie \textcolor{green}{\textsc{Gudar}}{}\ledrightnote{→\textcolor{green}{Die Schwestern oder Casanova in Spa. Lustspiel in Versen}} einmal etwas wie \textcolor{green}{Caſanova}{}\ledrightnote{→\textcolor{green}{Die Schwestern oder Casanova in Spa. Lustspiel in Versen}}{ }{\pb}geweſen iſt, wird \textsc{Tito} wohl ſeinerzeit zu einem werden; und in \textcolor{green}{\textsc{Santis}}{}\ledrightnote{→\textcolor{green}{Die Schwestern oder Casanova in Spa. Lustspiel in Versen}}{ }ſammeln ſich jene üblen Eigenſchaften, die der alternde \textcolor{green}{Caſanova}{}\ledrightnote{→\textcolor{green}{Die Schwestern oder Casanova in Spa. Lustspiel in Versen}} in \strikeout{\textcolor{gray}{geeigne}} Panne-Situationen hervorkehrt, zu \strikeout{eigner}
                    einer eigenen, aber gutmütig-ſchäbigen Geſtalt. Nur mit dem \textcolor{green}{\textsc{Andrea}}{}\ledrightnote{→\textcolor{green}{Die Schwestern oder Casanova in Spa. Lustspiel in Versen}} bin ich, um aufrichtig zu ſein, nicht
                    ganz einverſtanden; ich hätte ihn um ein gut Teil mehr \textsc{Bourgeois} gewünſcht; daß er das Mädel, mit dem er durchgeht, heiraten
                    will, daß er nur einmal ſpielt und daß er darob trotz Gewinns Reue empfindet,
                    macht dem Sohn ehrbarer Bürger alle Ehre; aber ich meine, er müßte die Dukaten
                    noch mit viel ſchwererem Herzen hergeben und nicht 1050, ſondern ſagen wir 950.
                    Auch im \textsc{Problema}-Streit iſt er mir zu freiſinnig, zu
                    großzügig, zu amoraliſch; mag dies auch gewiß dem Zeitalter entſprechen, ſo
                    entgeht doch, ſcheint mir, dem Drama dadurch ein ſcharfer Kontraſt. Hingegen
                    sind die zwei, nein drei \textcolor{green}{Caſanova}{}\ledrightnote{→\textcolor{green}{Die Schwestern oder Casanova in Spa. Lustspiel in Versen}}-Damen herrlich, \textcolor{green}{\textsc{Flaminia}}{}\ledrightnote{→\textcolor{green}{Die Schwestern oder Casanova in Spa. Lustspiel in Versen}} wie \textcolor{green}{\textsc{Anina}}{}\ledrightnote{→\textcolor{green}{Die Schwestern oder Casanova in Spa. Lustspiel in Versen}} und \textcolor{green}{\textsc{Theresa}}{}\ledrightnote{→\textcolor{green}{Die Schwestern oder Casanova in Spa. Lustspiel in Versen}}. Daß die große Szene zwiſchen \textcolor{green}{\textsc{Flaminia}}{}\ledrightnote{→\textcolor{green}{Die Schwestern oder Casanova in Spa. Lustspiel in Versen}} und \textcolor{green}{\textsc{Anina}}{}\ledrightnote{→\textcolor{green}{Die Schwestern oder Casanova in Spa. Lustspiel in Versen}} im zweiten Akte bei der Aufführung
                    etwas – für Moraliſch-Imprägnierte – Bedenk{\pb}liches haben dürfte, kann ich nicht
                    verkennen; zu fein geſpielt dürften die beiden Damen zu viel von ihrer
                    Schweſterſchaft einbüßen, und eine Vergröberung aus der fein gedachten und
                    geformten Szene eine ſehr unangenehme jenes Neides machen, für den der \textcolor{pink}{Wien}{}\ledrightnote{\textcolor{pink}{Wien}}er einen nicht wiederzugebenden Ausdruck
                    hat. –\pend
           \pstart
           Daß ich mich nie mit etwas Gedrucktem revanchieren kann, betrübt mich tief. Aber
                    die Zeiten wollen daran nichts ändern. Ich ſchreibe gar nichts und vertiefe
                    mich, wenn ich nicht an Akten arbeite, in die alten \textcolor{pink}{Italiener}{}\ledrightnote{\textcolor{pink}{Italien}} und – das iſt meine letzte Leidenſchaft –
                    Lateiner: \textcolor{blue}{\textsc{Vergil}}{}\ledrightnote{\textcolor{blue}{Otto Fürth}} (den ich erſt jetzt auf's Höchſte
                    verehren lernte), \textcolor{blue}{\textsc{Plautus}}{}\ledrightnote{\textcolor{blue}{Titus Maccius Plautus}}, \textcolor{blue}{\textsc{Valerius Flaccus}}{}\ledrightnote{\textcolor{blue}{Gaius Valerius Flaccus}}, \textcolor{blue}{\textsc{Florus}}{}\ledrightnote{\textcolor{blue}{Florus}} und andere. Ich habe ſchon einen
                    ganzen Stoß römiſcher Autoren zuſammengekauft; es iſt ein Lichtblick in
                    ſchwarzen Tagen, daß die Valutaentwertung auf das klaſſiſche Altertum nur mit
                    ungefähr 50 {\%} rückwirkt. –\pend
           \pstart
           Nochmals vielen Dank und die ergebenſten Grüße!\hspace*{3.5em}Ihr\pend
           \pstart \spacefill\mbox{D\textsuperscript{r}RAdam}\pend{}\endnumbering\briefempfaengerindex{Schnitzler, Arthur@\textsc{Schnitzler, Arthur}!zzzAdam, Robert@\emph{von Robert Adam}!1920-02-131@{13. 2. 1920}|)be}\mylabel{h}  \normalsize

\doendnotes{C}
\bigskip
\vfill

\clearpage

\footnotesize

\lohead{\textsc{register}}

% Definiere theindex-Environment komplett neu ohne reledmac
\makeatletter
\renewenvironment{theindex}{%
  \section*{\indexname}%
  \setlength{\parindent}{0pt}%
  \setlength{\parskip}{0pt plus 0.3pt}%
  \let\item\@idxitem
}{%
  \clearpage
}
\makeatother

\IfFileExists{\jobname-pw.ind}{\input{\jobname-pw.ind}}{}

\end{document}

      