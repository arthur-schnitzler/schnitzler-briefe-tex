%% latex-korrekturansicht-vorspann.tex
%% Vorspann für die Korrekturansicht.
%% Lädt die gemeinsame Datei latex-vorspann.tex mit gesetztem Schalter.

\newif\ifkorrekturansicht
\korrekturansichttrue

\input{../tex-inputs/latex-vorspann}


               \section[Arthur Schnitzler an Richard Beer-Hofmann, 31. 1. 1893]{ Arthur Schnitzler an Richard Beer-Hofmann, 31. 1. 1893}\nopagebreak\mylabel{v}\rehead{ }\normalsize\beginnumbering\briefempfaengerindex{Beer-Hofmann, Richard@\textsc{Beer-Hofmann, Richard}!zzzSchnitzler, Arthur@\emph{von Arthur Schnitzler}!1893-01-312@{31. 1. 1893}|(be} \toendnotes[C]{\smallbreak\pagebreak[2]} \Standort{YCGL, MSS 31.}
\physDesc{Briefkarte, Umschlag
\newline{}Handschrift: Bleistift, deutsche Kurrent\newline{}Versand: ohne postalischen Übermittlungsvermerk }\buchAbdrucke{\weitereDrucke{Arthur Schnitzler, Richard Beer-Hofmann: \emph{Briefwechsel 1891–1931}. Hg. Konstanze Fliedl. Wien, Zürich: \emph{Europaverlag} 1992, S. 41.} }\toendnotes[C]{\smallbreak}\pstart{}{\pb}\textsc{Hrn Dr. Rich. Beer H}\damage{ofmann}\pend{}\pstart{}\textcolor{pink}{\textsc{Wollzeile 15}}{}\ledrightnote{\textcolor{pink}{Wollzeile}}\pend{}\pstart{}{\pb}Dſtm. bez.\pend{}{\bigskip}\pstart
           \noindent{}{\pb}\strikeout{\textcolor{gray}{\textbf{PROFESSOR{ }\textcolor{blue}{SCHNITZLER}{}\ledrightnote{\textcolor{blue}{Johann Schnitzler}}}}}\hfill \textcolor{pink}{\textsc{\uline{I Grillparzerstraße 7}}}{}\ledrightnote{\textcolor{pink}{Grillparzerstraße}}. \pend
           \pstart
           Lieber Richard! Voilà – aber was?! Sie \uuline{vergaßen} mir die Karte zu ſenden!! Bitte entweder um Aufklärg oder um die
               Karte! Ja? {\pb}Dem \textcolor{blue}{Löbl}{}\ledrightnote{\textcolor{blue}{Emil Löbl}} hab ich um eine Redoutekarte geſchrieben. Sollt ich ſie kriegen, ſo geh
               ich! Sie erfahrens rechtzeitig! Vorher \substVorne{}\textsuperscript{\textcolor{gray}{bitt}}\substDazwischen{}geh\substHinten{} ich \substVorne{}\textsuperscript{eine}\substDazwischen{}zu\substHinten{}{ }\textcolor{green}{Mongodin}{}\ledrightnote{\textcolor{green}{Madame Mongodin}}\pend
           \pstart
           – Alſo bitte die Karte!\pend
           \pstart
           Herzlich{\\[\baselineskip]}\label{T_L00167_1v}\edtext{Ihr \spacefill\mbox{Arthur}}{\lemma{\textnormal{\emph{XXXX Lemmafehler}}}\Cendnote{\textnormal{am Papier links von
                     »Herzlich«, aber durch den Bleistiftdruck als zwei Schreibakte
                  zu erkennen}}}\label{K_L00167_1h}\pend
           \leftskip=0em{}\endnumbering\briefempfaengerindex{Beer-Hofmann, Richard@\textsc{Beer-Hofmann, Richard}!zzzSchnitzler, Arthur@\emph{von Arthur Schnitzler}!1893-01-312@{31. 1. 1893}|)be}\mylabel{h}  \normalsize

\doendnotes{C}
\bigskip
\vfill

\clearpage

\footnotesize

\lohead{\textsc{register}}

% Definiere theindex-Environment komplett neu ohne reledmac
\makeatletter
\renewenvironment{theindex}{%
  \section*{\indexname}%
  \setlength{\parindent}{0pt}%
  \setlength{\parskip}{0pt plus 0.3pt}%
  \let\item\@idxitem
}{%
  \clearpage
}
\makeatother

\IfFileExists{\jobname-pw.ind}{\input{\jobname-pw.ind}}{}

\end{document}

      