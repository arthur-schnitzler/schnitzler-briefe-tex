%% latex-korrekturansicht-vorspann.tex
%% Vorspann für die Korrekturansicht.
%% Lädt die gemeinsame Datei latex-vorspann.tex mit gesetztem Schalter.

\newif\ifkorrekturansicht
\korrekturansichttrue

\input{../tex-inputs/latex-vorspann}


               \section[Arthur Schnitzler an Robert Adam, 29. 5. 1921]{ Arthur Schnitzler an Robert Adam, 29. 5. 1921}\nopagebreak\mylabel{v}\rehead{ }\normalsize\beginnumbering\briefempfaengerindex{Adam, Robert@\textsc{Adam, Robert}!zzzSchnitzler, Arthur@\emph{von Arthur Schnitzler}!1921-05-291@{29. 5. 1921}|(be} \toendnotes[C]{\smallbreak\pagebreak[2]} \Standort{DLA, 96.34.2/27.}
\physDesc{Postkarte
\newline{}Handschrift: Bleistift, lateinische Kurrent\newline{}Versand: Stempel: »\nobreak{}\textcolor{gray}{Wien}, 30. V. 21, VII\textsuperscript{50}\nobreak{}«.  }\pstart{}{\pb}A. S.\pend{}\pstart{}\textcolor{pink}{Wien XVIII}{}\ledrightnote{\textcolor{pink}{XVIII., Währing}}\pend{}\pstart{}\textcolor{pink}{Sternwartestr \textcolor{gray}{71}}{}\ledrightnote{\textcolor{pink}{Sternwartestraße}}\pend{}{\bigskip}\pstart{}Hrn Ob.L.R\pend{}\pstart{}Dr. Robert Ad. Pollak\pend{}\pstart{}\textcolor{pink}{Wien XII}{}\ledrightnote{\textcolor{pink}{XII., Meidling}}\pend{}\pstart{}\textcolor{pink}{58 Meidlinger Hptstr 58}{}\ledrightnote{\textcolor{pink}{Meidlinger Hauptstraße}}.\pend{}{\bigskip}\pstart
           \raggedleft{}{\pb}29/5 1921\pend
           \pstart{}Verehrter Herr Doktor,\pend\pstart
           eben erhalt ich dringende Einladg zu einer Sitzg eines Autorenverbands für
                        Dinstag – dürft ich Sie vielleicht statt Dinstag –
                    am {\pb}Donnerstag, 2/6{ }gegen ½ 7 erwarten?\pend
           \pstart
           Mit verbindl Gruß{\\[\baselineskip]}Ihr sehr ergeb \spacefill\mbox{A. S.}\pend
           \leftskip=0em{}\endnumbering\briefempfaengerindex{Adam, Robert@\textsc{Adam, Robert}!zzzSchnitzler, Arthur@\emph{von Arthur Schnitzler}!1921-05-291@{29. 5. 1921}|)be}\mylabel{h}  \normalsize

\doendnotes{C}
\bigskip
\vfill

\clearpage

\footnotesize

\lohead{\textsc{register}}

% Definiere theindex-Environment komplett neu ohne reledmac
\makeatletter
\renewenvironment{theindex}{%
  \section*{\indexname}%
  \setlength{\parindent}{0pt}%
  \setlength{\parskip}{0pt plus 0.3pt}%
  \let\item\@idxitem
}{%
  \clearpage
}
\makeatother

\IfFileExists{\jobname-pw.ind}{\input{\jobname-pw.ind}}{}

\end{document}

      