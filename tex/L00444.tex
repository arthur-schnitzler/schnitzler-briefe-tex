%% latex-korrekturansicht-vorspann.tex
%% Vorspann für die Korrekturansicht.
%% Lädt die gemeinsame Datei latex-vorspann.tex mit gesetztem Schalter.

\newif\ifkorrekturansicht
\korrekturansichttrue

\input{../tex-inputs/latex-vorspann}


               \section[Friedrich M. Fels an Arthur Schnitzler, 20. 5. 1895]{ Friedrich M. Fels an Arthur Schnitzler, 20. 5. 1895}\nopagebreak\mylabel{v}\rehead{ }\normalsize\beginnumbering\briefempfaengerindex{Schnitzler, Arthur@\textsc{Schnitzler, Arthur}!zzzFels, Friedrich Michael@\emph{von Friedrich Michael Fels}!1895-05-201@{20. 5. 1895}|(be} \toendnotes[C]{\smallbreak\pagebreak[2]} \Standort{DLA, A:Schnitzler, HS.NZ85.1.2956.}
\physDesc{Kartenbrief
\newline{}Handschrift: schwarze Tinte, lateinische Kurrent\newline{}Versand: 1) Stempel: »\nobreak{}\oindex{I., Innere Stadt@\textbf{I., Innere Stadt}, \emph{Bezirk (A.BZK)}|pwk}Wien 1/1, 20. 5. 95, 1–2N\nobreak{}«.  2) Stempel: »\nobreak{}\oindex{IX., Alsergrund@\textbf{IX., Alsergrund}, \emph{Bezirk (A.BZK)}|pwk}Wien 9/3, 20. 5. 95, 3.N, Bestellt\nobreak{}«. 
\newline{}Schnitzler: mit Bleistift datiert: »23/4 95« und nummeriert: »22« }\toendnotes[C]{\smallbreak}\pstart{}{\pb}Herrn\pend{}\pstart{}Dr. Arthur Schnitzler\pend{}\pstart{}\textcolor{pink}{Wien}{}\ledrightnote{\textcolor{pink}{Wien}}\pend{}\pstart{}\textcolor{pink}{IX, Frankgaſse \damage{1}}{}\ledrightnote{\textcolor{pink}{Frankgasse}}\pend{}{\bigskip}\pstart
           \noindent{}{\pb}Lieber Dr Schnitzler!  Sie sagten mir neulich, Sie wollten mit \textcolor{blue}{Beer-Hofma{\geminationn}}{}\ledrightnote{\textcolor{blue}{Richard Beer-Hofmann}} reden wegen eines Anzugs; falls Sie es nicht gethan haben, darf ich jetzt wohl
               daran eri{\geminationn}ern. Es ist sehr langweilig, seine Hose jeden
               Morgen, da man sie anzieht, flicken zu müſsen. – Haben Sie das \textcolor{green}{Buch}{}\ledrightnote{→\textcolor{green}{Adhimukti}} der \textcolor{blue}{Fa{\geminationn}y Gröger}{}\ledrightnote{\textcolor{blue}{Fanny Gröger}}{ }ſchon gesehen, oder besitzen Sie es gar? We{\geminationn} ja, darf ich Sie später auf ein paar Tage darum
                  \damage{bi}tten? – Mit \textcolor{blue}{Hirschfeld}{}\ledrightnote{\textcolor{blue}{Robert Hirschfeld}} habe ich nicht
               gesprochen. Doch werde ich dieser Tage zu ihm gehen, um ihm ein neues Feuilleton zu
               bringen; da{\geminationn} erfahre ich wohl auch, ob aus \textcolor{pink}{Ossiacher See}{}\ledrightnote{\textcolor{pink}{Ossiacher See}} etwas wird. – Beiläufig: Sie müſsen
               ja ganz hochmütig geworden sein. \label{K_L00444_1v}\edtext{150 frcs für \textcolor{green}{Übersetzungsrecht}{}\ledrightnote{→\textcolor{green}{Sterben. Novelle}}}{\lemma{\textnormal{\emph{150 frcs für Übersetzungsrecht}}}\Cendnote{\textnormal{Für die französische Übersetzung von \emph{\textcolor{green}{Sterben}} vgl. den Antrag durch \textcolor{blue}{Raoul Bourse} (A. S.: \emph{Tagebuch}, 1. 5. 1895), die Übersetzung erfolgte durch \textcolor{blue}{Gaspard Vallette}.}}}\label{K_L00444_1h} – so was hätten Sie sich so bald
               nicht träumen laſsen.\pend
           \pstart
           Herzl. Gruſs und Dank{\\[\baselineskip]}\spacefill\mbox{F.}\pend
           \leftskip=0em{}\pstart
           \noindent{}\textcolor{pink}{Wien XVIII, Währinger-Gürtel 154 part. Th. 9}{}\ledrightnote{\textcolor{pink}{Währinger Gürtel}}\pend
           \endnumbering\briefempfaengerindex{Schnitzler, Arthur@\textsc{Schnitzler, Arthur}!zzzFels, Friedrich Michael@\emph{von Friedrich Michael Fels}!1895-05-201@{20. 5. 1895}|)be}\mylabel{h}  \normalsize

\doendnotes{C}
\bigskip
\vfill

\clearpage

\footnotesize

\lohead{\textsc{register}}

% Definiere theindex-Environment komplett neu ohne reledmac
\makeatletter
\renewenvironment{theindex}{%
  \section*{\indexname}%
  \setlength{\parindent}{0pt}%
  \setlength{\parskip}{0pt plus 0.3pt}%
  \let\item\@idxitem
}{%
  \clearpage
}
\makeatother

\IfFileExists{\jobname-pw.ind}{\input{\jobname-pw.ind}}{}

\end{document}

      