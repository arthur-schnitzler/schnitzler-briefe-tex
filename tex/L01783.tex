%% latex-korrekturansicht-vorspann.tex
%% Vorspann für die Korrekturansicht.
%% Lädt die gemeinsame Datei latex-vorspann.tex mit gesetztem Schalter.

\newif\ifkorrekturansicht
\korrekturansichttrue

\input{../tex-inputs/latex-vorspann}


               \section[Max Burckhard an Arthur Schnitzler, 14. 7. 1908]{ Max Burckhard an Arthur Schnitzler, 14. 7. 1908}\nopagebreak\mylabel{v}\rehead{ }\normalsize\beginnumbering\briefempfaengerindex{Schnitzler, Arthur@\textsc{Schnitzler, Arthur}!zzzBurckhard, Max Eugen@\emph{von Max Eugen Burckhard}!1908-07-141@{14. 7. 1908}|(be} \toendnotes[C]{\smallbreak\pagebreak[2]} \Standort{CUL, Schnitzler, B 20.}
\physDesc{Brief, 1 Blatt, 3 Seiten
\newline{}Handschrift: schwarze Tinte, deutsche Kurrent\newline{}Ordnung: mit Bleistift von unbekannter Hand nummeriert: »23« }\toendnotes[C]{\smallbreak}\pstart
           \noindent{}{\pb}\textcolor{gray}{\textbf{D\textsuperscript{r.} Max Burckhard}}\hfill \textcolor{gray}{\textbf{\strikeout{\textcolor{pink}{Wien, IX. Porzellangasse 48}{}\ledrightnote{\textcolor{pink}{Porzellangasse}}}{ }..........}}\pend
           \pstart
           \raggedleft{}\textcolor{gray}{\textbf{\textcolor{pink}{St. Gilgen}{}\ledrightnote{\textcolor{pink}{St. Gilgen}}}}{ }14. 7. 08\pend
           \pstart{}Sehr verehrter lieber Herr Doctor!\pend\pstart
           Ich beglückwünſche Sie ſehr \strikeout{für} zu Ihrem
                    Aufenthalt, den mir Ihre liebe Karte meldet. Ich war einmal wenige Tage auf der
                        \textcolor{pink}{Seiſeralm}{}\ledrightnote{\textcolor{pink}{Seiser Alm}} – allerdings zur Schnittzeit. Es
                    war dort nicht nur wunderſchön, ſondern auch anſonſt außerordentlich erheiternd;
                    es war damals das einzigemal, daſs ich Gelegenheit hatte, das \textcolor{pink}{ſüdtiroliſche}{}\ledrightnote{\textcolor{pink}{Südtirol}} Volksleben (von ſeiner angenehmſten Seite)
                    kennen zu lernen. Freilich hatte ich mich mit großen Vorräthen an feſtem und
                    flüßigem Proviant eingeführt und hatte ſchon vorher die Bekanntſchaft einiger
                    Theilnehmerinnen auf demSchlern gemacht.\pend
           \pstart
           {\pb}Nun, und ſind Sie uns \textcolor{pink}{St. Gilgnern}{}\ledrightnote{\textcolor{pink}{St. Gilgen}} ganz untreu geworden? Da es anfängt, Momente
                    zu geben, in denen ich mir einbilden kann, daſs ich mich noch einmal
                    zuſammenklaube, bilde ich mir ein, daſs ich davon etwas davon haben würde, wenn
                    Sie mit Ihrer verehrten Frau \textcolor{blue}{Gemahlin}{}\ledrightnote{→\textcolor{blue}{Olga Schnitzler}} hier wieder einmal in die heimiſchen Berge zukehren. Wie
                    herrliche Spaziergänge es hier gibt, das habe ich eigentlich erſt entdeckt, ſeit
                    die Facultät sich ablehnend gegen größere Spaziergänge ausgeſprochen hat.\pend
           \pstart
           {\pb}In herzlicher Verehrung mit Handkuſs
                    an Ihre liebe \textcolor{blue}{Frau}{}\ledrightnote{→\textcolor{blue}{Olga Schnitzler}} und
                    herzlichſtem Gruß\pend
           \pstart
           Ihr getreu ergebener{\\[\baselineskip]}\spacefill\mbox{D\textsuperscript{r}Burckhard}\pend
           \leftskip=0em{}\endnumbering\briefempfaengerindex{Schnitzler, Arthur@\textsc{Schnitzler, Arthur}!zzzBurckhard, Max Eugen@\emph{von Max Eugen Burckhard}!1908-07-141@{14. 7. 1908}|)be}\mylabel{h}  \normalsize

\doendnotes{C}
\bigskip
\vfill

\clearpage

\footnotesize

\lohead{\textsc{register}}

% Definiere theindex-Environment komplett neu ohne reledmac
\makeatletter
\renewenvironment{theindex}{%
  \section*{\indexname}%
  \setlength{\parindent}{0pt}%
  \setlength{\parskip}{0pt plus 0.3pt}%
  \let\item\@idxitem
}{%
  \clearpage
}
\makeatother

\IfFileExists{\jobname-pw.ind}{\input{\jobname-pw.ind}}{}

\end{document}

      