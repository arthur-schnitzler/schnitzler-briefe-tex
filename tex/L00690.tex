%% latex-korrekturansicht-vorspann.tex
%% Vorspann für die Korrekturansicht.
%% Lädt die gemeinsame Datei latex-vorspann.tex mit gesetztem Schalter.

\newif\ifkorrekturansicht
\korrekturansichttrue

\input{../tex-inputs/latex-vorspann}


               \section[Arthur Schnitzler an Richard Beer-Hofmann, 23. 6. 1897]{ Arthur Schnitzler an Richard Beer-Hofmann,
               23. 6. 1897}\nopagebreak\mylabel{v}\rehead{ }\normalsize\beginnumbering\briefempfaengerindex{Beer-Hofmann, Richard@\textsc{Beer-Hofmann, Richard}!zzzSchnitzler, Arthur@\emph{von Arthur Schnitzler}!1897-06-231@{23. 6. 1897}|(be} \toendnotes[C]{\smallbreak\pagebreak[2]} \Standort{YCGL, MSS 31.}
\physDesc{Brief, 1 Blatt, 4 Seiten, Umschlag
\newline{}Handschrift: Bleistift, deutsche Kurrent\newline{}Versand: 1) Stempel: »\nobreak{}\oindex{IX., Alsergrund@\textbf{IX., Alsergrund}, \emph{Bezirk (A.BZK)}|pwk}Wien 9/3, 23. 6. 97, 5–6N\nobreak{}«.  2) Stempel: »\nobreak{}\oindex{Bad Ischl@\textbf{Bad Ischl}, \emph{Besiedelter Ort (A.BSO)}|pwk}Ischl, 24. 6. 97, 7–8{[}V{]}\nobreak{}«. }\buchAbdrucke{\weitereDrucke{Arthur Schnitzler, Richard Beer-Hofmann: \emph{Briefwechsel 1891–1931}. Hg. Konstanze Fliedl. Wien, Zürich: \emph{Europaverlag} 1992, S. 110–111.} }\toendnotes[C]{\smallbreak}\pstart{}{\pb}Herrn \textsc{Dr. Rich.
                     Beer-Hofmann}\pend{}\pstart{}\textsc{\textcolor{pink}{Ischl}{}\ledrightnote{\textcolor{pink}{Bad Ischl}}}\pend{}\pstart{}\textcolor{pink}{\textsc{Egelmoos 22}}{}\ledrightnote{\textcolor{pink}{Eglmoosgasse}}\pend{}\pstart{}\textsc{\strikeout{N}\textcolor{pink}{O.Oe.}{}\ledrightnote{\textcolor{pink}{Oberösterreich}}}\pend{}{\bigskip}\pstart
           \raggedleft{}{\pb}23. 6. 97. \pend
           \pstart
           Lieber Richard. In den letzten Tagen war ich vielfach beſchäftigt
               und beunruhigt; Wohnung ſuchen für »\label{K_L00690_1v}\edtext{ſpäter}{\lemma{\textnormal{\emph{ſpäter}}}\Cendnote{\textnormal{\textcolor{blue}{Marie Reinhard} und er erwarteten ein gemeinsames
               Kind.}}}\label{K_L00690_1h}«, und die \textcolor{blue}{\textsc{inconnue}}{}\ledrightnote{\textcolor{blue}{Marie Glümer}} (Sie wiſſen ja wer das iſt) – ich hab Ihnen manchmal
               ſchreiben wollen, litt aber an »Überfülle des Stoffes«. Laſſe mir alles aufs
               mündliche. Daſs Ihr letzter Brief ſehr ſchön {\pb}war, wiſſen Sie ja ſelbſt; es iſt recht ſchmachvoll dſs ich mir überlegen mußte, ob
               ich das ſagen ſoll. Ich mein übrigens Ihren vorletzten. Ihr letzter iſt heut geko{\geminationm}en.\pend
           \pstart
           Alles ſoll beſorgt werden, ſelbſt dasjenige, womit Sie der Vorſehung in die Speichen
               fallen wollen, u. womit ich nicht das Vogel{\pb}futter meine.\pend
           \pstart
           Ich komme \uline{Samſtag}, vielleicht schon Samſtag früh an.
               Bitte, we{\geminationn}’s Ihnen nicht unbequem, beſtellen Sie \uline{mir} (nicht für meine \textcolor{blue}{Mama}{}\ledrightnote{→\textcolor{blue}{Louise Schnitzler}}, die ſpäter ko{\geminationm}t) das
               Zimmer; iſt’s Ihnen unbequem, ſo ſchreiben Sie dem \textsc{\textcolor{blue}{Petter}{}\ledrightnote{\textcolor{blue}{Leopold Petter}}} eine {\pb}Karte. – Ich ſage nichts näheres
               über das Zimmer, \uline{Sie} haben die ganze
               Verantwortung.\pend
           \pstart
           \textcolor{blue}{Schwkopf}{}\ledrightnote{\textcolor{blue}{Gustav Schwarzkopf}} noch nicht entſchieden, ſchreiben Sie
               ihm zuredend.\pend
           \pstart
           Ich freue mich ſehr auf Sie, beinah ſehn’ ich mich.\pend
           \pstart Herzlich Ihr \spacefill\mbox{Arthur}\pend{}\endnumbering\briefempfaengerindex{Beer-Hofmann, Richard@\textsc{Beer-Hofmann, Richard}!zzzSchnitzler, Arthur@\emph{von Arthur Schnitzler}!1897-06-231@{23. 6. 1897}|)be}\mylabel{h}  \normalsize

\doendnotes{C}
\bigskip
\vfill

\clearpage

\footnotesize

\lohead{\textsc{register}}

% Definiere theindex-Environment komplett neu ohne reledmac
\makeatletter
\renewenvironment{theindex}{%
  \section*{\indexname}%
  \setlength{\parindent}{0pt}%
  \setlength{\parskip}{0pt plus 0.3pt}%
  \let\item\@idxitem
}{%
  \clearpage
}
\makeatother

\IfFileExists{\jobname-pw.ind}{\input{\jobname-pw.ind}}{}

\end{document}

      