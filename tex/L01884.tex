%% latex-korrekturansicht-vorspann.tex
%% Vorspann für die Korrekturansicht.
%% Lädt die gemeinsame Datei latex-vorspann.tex mit gesetztem Schalter.

\newif\ifkorrekturansicht
\korrekturansichttrue

\input{../tex-inputs/latex-vorspann}


               \section[Albert Ehrenstein an Arthur Schnitzler, 6. 11. 1909]{ Albert Ehrenstein an Arthur Schnitzler, 6. 11. 1909}\nopagebreak\mylabel{v}\rehead{ }\normalsize\beginnumbering\briefempfaengerindex{Schnitzler, Arthur@\textsc{Schnitzler, Arthur}!zzzEhrenstein, Albert@\emph{von Albert Ehrenstein}!1909-11-062@{06. 11. 1909}|(be} \toendnotes[C]{\smallbreak\pagebreak[2]} \Standort{CUL, Schnitzler, B 30.}
\physDesc{Brief, 1 Blatt, 3 Seiten
\newline{}Handschrift: schwarze Tinte, deutsche Kurrent}\buchAbdrucke{\weitereDrucke{Albert Ehrenstein: \emph{Briefe}. Hg. Hanni Mittelmann. München: \emph{Boer} 1989, S. 34–35 (Werke, 1).} }\pstart
           {\pb}\textsc{Albert Ehrenstein}\hfill 6. XI. 09.
                        \pend
           \pstart
           \textsc{\textcolor{pink}{XVI. Ottakringerstr 114}{}\ledrightnote{\textcolor{pink}{Ottakringerstraße}}.}\pend
           \pstart{}\textsc{Sehr geehrter Herr Doktor,}\pend\pstart
           nun habe ich auf meiner Tournee durch die Schattenſeiten des Metiers zu meiner
                    nicht ganz gelinden Verzweiflung auch noch die kennen gelernt, welche ſich in
                    Maſchinenfräuleins und deren Schreibfehlern verkörpert. Von den Arbeiten, die
                    ſich in dieſer Neugeſtaltung bei Ihnen, ſehr geehrter Herr Doktor, einfinden,
                    ſind Ihnen nur »\textcolor{green}{Mitgefühl}{}\ledrightnote{\textcolor{green}{Mitgefühl}}« und »\textcolor{green}{Saccumum}{}\ledrightnote{\textcolor{green}{Saccumum}}« unbekannt.\pend
           \pstart
           Da ich keine Ahnung habe, was für Sachen einem Verlegerherzen goldhaltig ſcheinen
                    können, habe ich keine beſondere Auswahl {\pb}unter meinen
                    Produkten getroffen – wahrſcheinlich iſt ſo etwas wie eine Sichtung auch kaum
                    durchführbar. Ich wenigſtens habe nicht herausfinden können, welches die
                    langweiligſten ſind – es tut einem wirklich die Wahl weh. Gäben die Götter, daß
                    der Herr Ko{\geminationm}erzienrat \textcolor{blue}{Fiſcher}{}\ledrightnote{\textcolor{blue}{Samuel Fischer}} dieſen angeblichen Novellenzyklus akzeptiert oder
                    – was ihn ja nichts koſten würde – irgendetwas in der \textcolor{green}{Rundſchau}{}\ledrightnote{\textcolor{green}{Die neue Rundschau}} bringt. Es wäre das für mich eine kleine
                    Verſicherung gegen gewiſſe Stupiditäten der Außenwelt, die ſich demnächſt in
                    zudringlichen Fragen hiſtoriſchen Charakters manifeſtieren dürften.\pend
           \pstart
           {\pb}Und ein etwaiger Mißerfolg wäre im Vorhinein
                    kompenſiert.\pend
           \pstart
           Sollte eine Art von grauſamem, aber vielleicht logiſchem und gerechtem
                    Parallelismus mich auf beiden Seiten zuſchanden werden laſſen, meinen
                    Erfahrungen gemäß nicht bloß auf Ihren Empfehlungen, ſondern auch auf meinen
                    Leiſtungen ſo etwas wie ein Fluch liegen, bleibe ich Ihnen, ſehr geehrter Herr
                    Doktor, noch immer äußerſt dankbar für ſo manches frühere, namentlich für Ihre
                    harten Worte über mein Übelwollen – denn auch eine derartige Frottierung hatte
                    äußerſt nötig Ihr ergebenſter\pend
           \pstart \spacefill\mbox{Albert Ehrenstein.}\pend{}\endnumbering\briefempfaengerindex{Schnitzler, Arthur@\textsc{Schnitzler, Arthur}!zzzEhrenstein, Albert@\emph{von Albert Ehrenstein}!1909-11-062@{06. 11. 1909}|)be}\mylabel{h}  \normalsize

\doendnotes{C}
\bigskip
\vfill

\clearpage

\footnotesize

\lohead{\textsc{register}}

% Definiere theindex-Environment komplett neu ohne reledmac
\makeatletter
\renewenvironment{theindex}{%
  \section*{\indexname}%
  \setlength{\parindent}{0pt}%
  \setlength{\parskip}{0pt plus 0.3pt}%
  \let\item\@idxitem
}{%
  \clearpage
}
\makeatother

\IfFileExists{\jobname-pw.ind}{\input{\jobname-pw.ind}}{}

\end{document}

      