%% latex-korrekturansicht-vorspann.tex
%% Vorspann für die Korrekturansicht.
%% Lädt die gemeinsame Datei latex-vorspann.tex mit gesetztem Schalter.

\newif\ifkorrekturansicht
\korrekturansichttrue

\input{../tex-inputs/latex-vorspann}


               \section[Arthur Schnitzler an Hermann Bahr, 29. 1. 1906]{ Arthur Schnitzler an Hermann Bahr, 29. 1. 1906}\nopagebreak\mylabel{v}\rehead{ }\normalsize\beginnumbering\briefempfaengerindex{Bahr, Hermann@\textsc{Bahr, Hermann}!zzzSchnitzler, Arthur@\emph{von Arthur Schnitzler}!1906-01-292@{29. 1. 1906}|(be} \toendnotes[C]{\smallbreak\pagebreak[2]} \Standort{TMW, HS AM 23378 Ba.}
\physDesc{Brief, 1 Blatt, 2 Seiten
\newline{}Handschrift: schwarze Tinte, deutsche Kurrent\newline{}Ordnung: Lochung }\buchAbdrucke{\weitereDrucke{1) \emph{29. 1. 1906.} In: Arthur Schnitzler: \emph{The Letters of Arthur Schnitzler to Hermann Bahr}. Edited, annotated, and with an introduction, by Donald G.
                        Daviau. Chapel Hill: \emph{The University of North Carolina Press} 1978, S. 93 (University of North Carolina studies in the Germanic languages
                        and literatures, 89).} \weitereDrucke{2) Hermann Bahr, Arthur Schnitzler: \emph{Briefwechsel, Aufzeichnungen, Dokumente (1891–1931)}. Hg. Kurt Ifkovits und Martin Anton Müller. Göttingen: \emph{Wallstein} 2018, S. 372.} }\toendnotes[C]{\smallbreak}\pstart
           \noindent{}{\pb}\textcolor{gray}{\textbf{Dr. Arthur Schnitzler}}\hfill 29. 1. 906.\pend
           \pstart
           \textcolor{gray}{\textbf{\textcolor{pink}{Wien, XVIII. Spoettelgasse 7}{}\ledrightnote{\textcolor{pink}{Edmund-Weiß-Gasse}}.}}\pend
           \pstart{}lieber Hermann, \pend\pstart
           es thut mir natürlich rieſig leid, daſs man nun auch mein \textcolor{green}{Stück}{}\ledrightnote{→\textcolor{green}{Der Ruf des Lebens. Schauspiel in drei Akten}} benützt, um dir was unangenehmes
               anzuthun, aber ich bitte dich ja nicht dieſen Fall als Cabinetsfrage zwiſchen dir und
               der \textcolor{brown}{Intendanz}{}\ledrightnote{→\textcolor{brown}{Königliche Hof- und Nationaltheater München}} zu behandeln.
               Intereſſiren wird dich unter dieſen Umſtänden vielleicht daſs mir das \label{K_L01578_1v}\edtext{\textcolor{pink}{Petersburger \uline{kaiser}{\pb}\uline{liche} Theater}{}\ledrightnote{\textcolor{pink}{Alexandrinski-Theater}} telegrafiſch tauſend Rubel}{\lemma{\textnormal{\emph{Petersburger … Rubel}}}\Cendnote{\textnormal{vgl. A. S.: \emph{Tagebuch}, 26. 1. 1906}}}\label{K_L01578_1h} Garantie
               bieten lieſs, wenn ich das Erſcheinen des \textcolor{green}{\uline{Buches}}{}\ledrightnote{→\textcolor{green}{Der Ruf des Lebens. Schauspiel in drei Akten}}{ }\introOben{}in deutſcher Sprache\introOben{} bis \label{K_L01578_2v}\edtext{Oktober hinausſchieben}{\lemma{\textnormal{\emph{Oktober hinausſchieben}}}\Cendnote{\textnormal{Es erscheint im März 1906.}}}\label{K_L01578_2h} wollte.\pend
           \pstart
           Herzlichſt dein{\\[\baselineskip]}\spacefill\mbox{A.}\pend
           \leftskip=0em{}\pstart
           \noindent{}Kann man dich nicht d\damage{oc}h vielleicht einmal ſehen? –\pend
           \pstart
           Viele Grüße von meiner \textcolor{blue}{Frau}{}\ledrightnote{→\textcolor{blue}{Olga Schnitzler}}.\pend
           \endnumbering\briefempfaengerindex{Bahr, Hermann@\textsc{Bahr, Hermann}!zzzSchnitzler, Arthur@\emph{von Arthur Schnitzler}!1906-01-292@{29. 1. 1906}|)be}\mylabel{h}  \normalsize

\doendnotes{C}
\bigskip
\vfill

\clearpage

\footnotesize

\lohead{\textsc{register}}

% Definiere theindex-Environment komplett neu ohne reledmac
\makeatletter
\renewenvironment{theindex}{%
  \section*{\indexname}%
  \setlength{\parindent}{0pt}%
  \setlength{\parskip}{0pt plus 0.3pt}%
  \let\item\@idxitem
}{%
  \clearpage
}
\makeatother

\IfFileExists{\jobname-pw.ind}{\input{\jobname-pw.ind}}{}

\end{document}

      