%% latex-korrekturansicht-vorspann.tex
%% Vorspann für die Korrekturansicht.
%% Lädt die gemeinsame Datei latex-vorspann.tex mit gesetztem Schalter.

\newif\ifkorrekturansicht
\korrekturansichttrue

\input{../tex-inputs/latex-vorspann}


               \section[Max Burckhard: Widmungsexemplar Franz Stelzhamer Charakterbilder für Arthur Schnitzler, 27. 10. 1905]{ Max Burckhard: Widmungsexemplar Franz Stelzhamer Charakterbilder für Arthur Schnitzler,
                    27. 10. 1905}\nopagebreak\mylabel{v}\rehead{ }\normalsize\beginnumbering\briefempfaengerindex{Schnitzler, Arthur@\textsc{Schnitzler, Arthur}!zzzBurckhard, Max Eugen@\emph{von Max Eugen Burckhard}!1905-10-271@{27. 10. 1905}|(be} \toendnotes[C]{\smallbreak\pagebreak[2]} \Standort{DLA, G:Schnitzler, Arthur (Sammlung Heinrich Schnitzler).}
\physDesc{Widmung am Vortitel
\newline{}Handschrift: schwarze Tinte, deutsche Kurrent\newline{}Ordnung: bei der Enteignung des Exemplars 1938 von unbekannter Hand mit Bleistift
                                    ergänzte Informationen: »\noindent{}= 440752-B{ / }s{[}ine{]}.a{[}nno{]}.!« }\toendnotes[C]{\smallbreak}\pstart
           \noindent{}{\pb}Arthur Schnitzler\pend
           \pstart herzlichſt d. Herausgeber\spacefill\mbox{DrBurckhard}\pend{}\pstart
           27. 10. 05\pend
           {\bigskip}\pstart
           \noindent{}\centering{}\textcolor{gray}{\textbf{\textcolor{green}{Charakterbilder aus Oberöſterreich}{}\ledrightnote{\textcolor{green}{Charakterbilder aus Oberösterreich}}}}\pend
           {\bigskip}\pstart
           \noindent{}\centering{}{\pb}\textcolor{gray}{\textbf{\textcolor{blue}{Franz Stelzhamer}{}\ledrightnote{\textcolor{blue}{Franz Stelzhamer}}.}}\pend
           \pstart
           \noindent{}\centering{}\textcolor{gray}{\textbf{\textcolor{green}{Charakterbilder aus{\\}Oberoeſterreich}{}\ledrightnote{\textcolor{green}{Charakterbilder aus Oberösterreich}}.–
                        Mit}}{\\}\textcolor{gray}{\textbf{einem Geleitspruch}}{\\}\textcolor{gray}{\textbf{von}}\pend
           \pstart
           \noindent{}\centering{}\textcolor{gray}{\textbf{\textcolor{blue}{Gerhart Hauptmann}{}\ledrightnote{\textcolor{blue}{Gerhart Hauptmann}}.}}\pend
           {\bigskip}\pstart
           \noindent{}\centering{}\textcolor{gray}{\textbf{\label{K_L01565_1v}\edtext{\textcolor{brown}{Wiener Verlag}{}\ledrightnote{\textcolor{brown}{Wiener Verlag}}}{\lemma{\textnormal{\emph{Wiener Verlag}}}\Cendnote{\textnormal{am 13. 11. 1905 vom \emph{\textcolor{green}{Börsenblatt für den deutschen
                     Buchhandel}} als Neuerscheinung gemeldet}}}\label{K_L01565_1h}{ }\textcolor{pink}{Wien}{}\ledrightnote{\textcolor{pink}{Wien}}{ }u.{ }\textcolor{pink}{Leipzig}{}\ledrightnote{\textcolor{pink}{Leipzig}}}}\pend
           \endnumbering\briefempfaengerindex{Schnitzler, Arthur@\textsc{Schnitzler, Arthur}!zzzBurckhard, Max Eugen@\emph{von Max Eugen Burckhard}!1905-10-271@{27. 10. 1905}|)be}\mylabel{h}  \normalsize

\doendnotes{C}
\bigskip
\vfill

\clearpage

\footnotesize

\lohead{\textsc{register}}

% Definiere theindex-Environment komplett neu ohne reledmac
\makeatletter
\renewenvironment{theindex}{%
  \section*{\indexname}%
  \setlength{\parindent}{0pt}%
  \setlength{\parskip}{0pt plus 0.3pt}%
  \let\item\@idxitem
}{%
  \clearpage
}
\makeatother

\IfFileExists{\jobname-pw.ind}{\input{\jobname-pw.ind}}{}

\end{document}

      