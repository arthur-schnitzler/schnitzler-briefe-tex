%% latex-korrekturansicht-vorspann.tex
%% Vorspann für die Korrekturansicht.
%% Lädt die gemeinsame Datei latex-vorspann.tex mit gesetztem Schalter.

\newif\ifkorrekturansicht
\korrekturansichttrue

\input{../tex-inputs/latex-vorspann}


               \section[Arthur Schnitzler an Hugo von Hofmannsthal, {[}1.? 2. 1892{]}]{ Arthur Schnitzler an Hugo von Hofmannsthal, {[}1.? 2. 1892{]}}\nopagebreak\mylabel{v}\rehead{ }\normalsize\beginnumbering\briefempfaengerindex{Hofmannsthal, Hugo von@\textsc{Hofmannsthal, Hugo von}!zzzSchnitzler, Arthur@\emph{von Arthur Schnitzler}!1892-02-013@{{[}1.? 2. 1892{]}}|(be} \toendnotes[C]{\smallbreak\pagebreak[2]} \Standort{FDH, Hs-30885,17.}
\physDesc{Briefkarte
\newline{}Handschrift: Bleistift, deutsche Kurrent\newline{}Ordnung: 1) von Schnitzler mutmaßlich bei der Durchsicht der Korrespondenz 1929 mit Bleistift datiert: »9/\substVorne{}\textsuperscript{3}\substDazwischen{}4\substHinten{}?{ }90?« 2) mit Bleistift von unbekannter Hand nummeriert: »17«}\buchAbdrucke{\weitereDrucke{Hugo von Hofmannsthal, Arthur Schnitzler: \emph{Briefwechsel}. Hg. Therese Nickl und Heinrich Schnitzler. Frankfurt am Main: \emph{S. Fischer} 1964, S. 15.} }\toendnotes[C]{\smallbreak}\pstart
           \noindent{}{\pb}Lieber Freund, hier ſind die \textcolor{green}{Bücher}{}\ledrightnote{→\textcolor{green}{Die Blinden}{\newline}→\textcolor{green}{Der Garten der Bérenice}{\newline}→\textcolor{green}{Die sieben Prinzessinnen}}. So{\geminationn}tag ist \label{K_L00068_1v}\edtext{\textcolor{blue}{\textsc{Goldschmidt}}{}\ledrightnote{\textcolor{blue}{Adalbert von Goldschmidt}}}{\lemma{\textnormal{\emph{Goldschmidt}}}\Cendnote{\textnormal{Am 7. 2. 1892 fand eine Matinée mit \textcolor{blue}{Emanuel Reicher} im Haus von \textcolor{blue}{Adalbert von Goldschmidt} statt, an der
                            \textcolor{blue}{Schnitzler} teilnahm.}}}\label{K_L00068_1h} von
                        3 an, alſo wohl bis 6. Und am Abend bin ich
                    eingeladen. Ich fände es hübſch, we{\geminationn} wir an irgend
                    einem Wochentagsabend die Zuſa{\geminationm}enkunft arrangirten.
                    Z. B. Samſtag{ }{\pb}Abend um 7 Uhr bei mir? Oder Anfangs nächſter Woche?
                        \label{K_L00068_2v}\edtext{Montag z. B. – Doch da ist \textcolor{green}{\textsc{Crampton}}{}\ledrightnote{\textcolor{green}{College Crampton. Komödie in fünf Akten}}}{\lemma{\textnormal{\emph{Montag … Crampton}}}\Cendnote{\textnormal{\textcolor{blue}{Schnitzler} besuchte die Premiere von \textcolor{blue}{Gerhart Hauptmann}s \emph{\textcolor{green}{College Crampton}} im \textcolor{pink}{Burgtheater} am 8. 2. 1892 (\emph{Cambridge University Library}, A 179a).}}}\label{K_L00068_2h}.
                        Mittwoch? –\pend
           \pstart
           Herzlichſt Ihr{\\[\baselineskip]}\spacefill\mbox{Arthur}\pend
           \leftskip=0em{}\endnumbering\briefempfaengerindex{Hofmannsthal, Hugo von@\textsc{Hofmannsthal, Hugo von}!zzzSchnitzler, Arthur@\emph{von Arthur Schnitzler}!1892-02-013@{{[}1.? 2. 1892{]}}|)be}\mylabel{h}  \normalsize

\doendnotes{C}
\bigskip
\vfill

\clearpage

\footnotesize

\lohead{\textsc{register}}

% Definiere theindex-Environment komplett neu ohne reledmac
\makeatletter
\renewenvironment{theindex}{%
  \section*{\indexname}%
  \setlength{\parindent}{0pt}%
  \setlength{\parskip}{0pt plus 0.3pt}%
  \let\item\@idxitem
}{%
  \clearpage
}
\makeatother

\IfFileExists{\jobname-pw.ind}{\input{\jobname-pw.ind}}{}

\end{document}

      