%% latex-korrekturansicht-vorspann.tex
%% Vorspann für die Korrekturansicht.
%% Lädt die gemeinsame Datei latex-vorspann.tex mit gesetztem Schalter.

\newif\ifkorrekturansicht
\korrekturansichttrue

\input{../tex-inputs/latex-vorspann}


               \section[Thomas Mann an Arthur Schnitzler, 22. 5. 1913]{ Thomas Mann an Arthur Schnitzler, 22. 5. 1913}\nopagebreak\mylabel{v}\rehead{ }\normalsize\beginnumbering\briefempfaengerindex{Schnitzler, Arthur@\textsc{Schnitzler, Arthur}!zzzMann, Thomas@\emph{von Thomas Mann}!1913-05-221@{22. 5. 1913}|(be} \toendnotes[C]{\smallbreak\pagebreak[2]} \Standort{CUL, Schnitzler, B 67.}
\physDesc{Brief, 1 Blatt, 3 Seiten
\newline{}Handschrift: schwarze Tinte, deutsche Kurrent
\newline{}Schnitzler: 1) mit Bleistift beschriftet: »\textsc{Thomas Mann}« 2) mit rotem Buntstift eine Unterstreichung}\buchAbdrucke{\weitereDrucke{1) Thomas Mann: \emph{Briefe 1889–1936}. Mann, Erika. Frankfurt am Main: \emph{S. Fischer} 1961, S. 102.} \weitereDrucke{2) Hertha Krotkoff: \emph{Arthur Schnitzler – Thomas Mann: Briefe.} In: \emph{Modern Austrian Literature}, Jg. 7 (1974) Nr. 1/2, S. 16–17.} }\toendnotes[C]{\smallbreak}\pstart
           \noindent{}\raggedleft{}{\pb}\textcolor{gray}{\textbf{\textcolor{pink}{BAD TÖLZ}{}\ledrightnote{\textcolor{pink}{Bad Tölz}}, DEN}}{ }22. Mai 1913.\pend
           \pstart
           \noindent{}\raggedleft{}\textcolor{gray}{\textbf{\textcolor{pink}{LANDHAUS THOMAS MANN.}{}\ledrightnote{\textcolor{pink}{Thomas Mann Villa}}}}\pend
           \pstart{}Verehrter Herr Doctor:\pend\pstart
           Ihre wundervolle \textcolor{green}{Sommergeſchichte}{}\ledrightnote{→\textcolor{green}{Frau Beate und ihr Sohn. Novelle}}, von der mir ein Exemplar in Ihrem gütigen Auftrage
                    zugeſandt wurde, habe ich geſtern Abend in großer Bewegung beendigt. Sie wird
                    mich noch lange feſthalten und beſchäftigen. Die heutige Kunſt verſteht ſich ja
                    im Ganzen nicht ſchlecht auf »Stimmung«; aber einen Fall, wo Stimmung ſich
                    dermaßen unerbittlich, fürchterlich, verhängnishaft verdichtet, wie hier bei
                    Ihnen, – den gibt es, glaube ich, auch heute {\pb}nicht zum zweiten Mal. Ich werde
                    nicht müde, auch bei geſchloſſenem Buche die Dichtigkeit und magiſche
                    Unzerreißbarkeit dieſes erotiſchen Kunſt- und Schickſalsgeſpinſtes zu prüfen und
                    zu bewundern und bitte Ihnen meinen tiefen Reſpekt ausdrücken zu dürfen vor
                    Ihrer großen Zaubermacht. Der Schluß geht mir beſtändig nach. Trotz feinſter,
                    vielfältigſter Vorbereitung – iſt er möglich ſo oder iſt er es nicht? Auf jeden
                    Fall iſt er überwältigend ſchön.\pend
           \pstart
           Ich habe die Überraſchung, zu ſehen, daß mein »\textcolor{green}{Tod
                        in Venedig}{}\ledrightnote{\textcolor{green}{Der Tod in Venedig}}«, bei deſſen Herſtellung ich {\pb}auf garnichts hoffte, ſehr warm
                    aufgenommen wird. Bis auf einen giftigen \textcolor{green}{Angriff}{}\ledrightnote{→\textcolor{green}{Tagebuch}} des Herrn \textcolor{blue}{Kerr}{}\ledrightnote{\textcolor{blue}{Alfred Kerr}}, hinter deſſen tänzeriſchem \textcolor{green}{Pamphletchen}{}\ledrightnote{→\textcolor{green}{Tagebuch}} gegen mich ſich freilich viel
                    Charakter-Elend verbirgt, habe ich faſt nur ſehr Ehrenvolles darüber gehört. Und
                    daß die erſte Beruhigung vom Autor der »\textcolor{green}{Frau
                        Beate}{}\ledrightnote{\textcolor{green}{Frau Beate und ihr Sohn. Novelle}}« kam, darüber bin ich nun wieder beſonders glücklich.\pend
           \pstart
           Mit den beſten Empfehlungen an Sie und Ihre \textcolor{blue}{Gattin}{}\ledrightnote{→\textcolor{blue}{Olga Schnitzler}}, verehrter Herr Doctor,{\\[\baselineskip]}Ihr ergebenſter{\\[\baselineskip]}\spacefill\mbox{Thomas Mann.}\pend
           \leftskip=0em{}\endnumbering\briefempfaengerindex{Schnitzler, Arthur@\textsc{Schnitzler, Arthur}!zzzMann, Thomas@\emph{von Thomas Mann}!1913-05-221@{22. 5. 1913}|)be}\mylabel{h}  \normalsize

\doendnotes{C}
\bigskip
\vfill

\clearpage

\footnotesize

\lohead{\textsc{register}}

% Definiere theindex-Environment komplett neu ohne reledmac
\makeatletter
\renewenvironment{theindex}{%
  \section*{\indexname}%
  \setlength{\parindent}{0pt}%
  \setlength{\parskip}{0pt plus 0.3pt}%
  \let\item\@idxitem
}{%
  \clearpage
}
\makeatother

\IfFileExists{\jobname-pw.ind}{\input{\jobname-pw.ind}}{}

\end{document}

      