%% latex-korrekturansicht-vorspann.tex
%% Vorspann für die Korrekturansicht.
%% Lädt die gemeinsame Datei latex-vorspann.tex mit gesetztem Schalter.

\newif\ifkorrekturansicht
\korrekturansichttrue

\input{../tex-inputs/latex-vorspann}


               \section[Arthur Schnitzler an Richard Beer-Hofmann, 17. 6. 1898]{ Arthur Schnitzler an Richard Beer-Hofmann, 17. 6. 1898}\nopagebreak\mylabel{v}\rehead{ }\normalsize\beginnumbering\briefempfaengerindex{Beer-Hofmann, Richard@\textsc{Beer-Hofmann, Richard}!zzzSchnitzler, Arthur@\emph{von Arthur Schnitzler}!1898-06-171@{17. 6. 1898}|(be} \toendnotes[C]{\smallbreak\pagebreak[2]} \Standort{CUL, Schnitzler, B 8.1, S. 71.}
\physDesc{maschinelle Abschrift
\newline{}Schreibmaschine\newline{}Ordnung: von unbekannter Hand nummeriert: »119« }\Standort{YCGL, MSS 31.}
\physDesc{Korrekturen zu Schlaflied für Mirjam1 Blatt, 1 Seite
\newline{}Handschrift: Bleistift, deutsche Kurrent\newline{}Ordnung: 1) mit Bleistift von unbekannter
                                 Hand beschriftet »Schnitzler: Korrekturen zu Beer-Hofmanns
                                    ›Schlaflied für Mirjam‹« 2) mit Tinte von unbekannter Hand zur Zeile 6 der 2. Strophe:
                                    »doch«}\Standort{CUL, Schnitzler, B 8.}
\physDesc{1 Blatt, 2 Seiten, Gedichtabschrift
\newline{}Handschrift: Bleistift, deutsche Kurrent\newline{}Ordnung: mit Bleistift von unbekannter Hand nummeriert:
                                    »116« }\buchAbdrucke{\weitereDrucke{Arthur Schnitzler, Richard Beer-Hofmann: \emph{Briefwechsel 1891–1931}. Hg. Konstanze Fliedl. Wien, Zürich: \emph{Europaverlag} 1992, S. 119–120, 118–119.} }\toendnotes[C]{\smallbreak}\pstart
           \raggedleft{}{\pb}\textcolor{pink}{Wien}{}\ledrightnote{\textcolor{pink}{Wien}}, 17. 6. 98.\pend
           \pstart
           Lieber Richard, beiliegend mein Interpunktionsgefühl. Im
               wesentlichen liegt ja nicht viel dran. \textcolor{blue}{Hugo}{}\ledrightnote{\textcolor{blue}{Hugo von Hofmannsthal}} ist
               in der \textcolor{pink}{Brühl}{}\ledrightnote{\textcolor{pink}{Brühl}}, ich wollte gestern zu ihm; aber es
               regnete. Am Tag meiner Abfahrt hatte ich Regen bis \textcolor{pink}{Wr.
                  Neustadt}{}\ledrightnote{\textcolor{pink}{Wiener Neustadt}} – dann war es schön und blieb so bis gestern. Meine Sommerpläne sind
               verpfuscht. Man lässt \textcolor{blue}{sie}{}\ledrightnote{→\textcolor{blue}{Marie Reinhard}} nicht
               mit mir reisen, so wird ein enervirendes Hin und Her herauskommen. Ich bleibe vor
               allem einmal bis Mitte Juli in \textcolor{pink}{Wien}{}\ledrightnote{\textcolor{pink}{Wien}}; bin
               dann ein paar Tage mit \textcolor{blue}{ihr}{}\ledrightnote{→\textcolor{blue}{Marie Reinhard}} und
               ihrer \textcolor{blue}{Schwester}{}\ledrightnote{→\textcolor{blue}{Caroline Burger}} sowie \textcolor{blue}{Schwager}{}\ledrightnote{→\textcolor{blue}{Rudolf Burger}} in \textcolor{pink}{Gr.}{}\ledrightnote{\textcolor{pink}{Graz}} zusammen – wohin ich vom
                  20.–27. Juli gehe, weiss ich nicht. (Wollen Sie irgendwo
               mit mir zusammen sein? Aber nicht in \textcolor{pink}{Steindorf}{}\ledrightnote{\textcolor{pink}{Steindorf am Ossiacher See}}) Dann per Rad mit \textcolor{blue}{ihr}{}\ledrightnote{→\textcolor{blue}{Marie Reinhard}} und den \textcolor{blue}{Ihren}{}\ledrightnote{\textcolor{blue}{Caroline Burger}{\newline}\textcolor{blue}{Carl Reinhard}{\newline}\textcolor{blue}{Therese Reinhard}} nach \textcolor{pink}{Tegernsee}{}\ledrightnote{\textcolor{pink}{Tegernsee}}. – Von dort verschwind ich sofort; –
               wahrscheinlich in die Schweiz. Da werd ich eine Zeitlang mit der \textcolor{blue}{Mama}{}\ledrightnote{→\textcolor{blue}{Louise Schnitzler}} zusammen sein. (\textcolor{pink}{Vierwaldstädtersee}{}\ledrightnote{\textcolor{pink}{Vierwaldstättersee}}). Die letzte Augustwoche
               wahrscheinlich in \textcolor{pink}{Tegernsee}{}\ledrightnote{\textcolor{pink}{Tegernsee}} – dann in den ersten
                  Septembertagen wenns geht, durchs \textcolor{pink}{Ampezzo}{}\ledrightnote{\textcolor{pink}{Ampezzo}} per Rad nach \textcolor{pink}{Venedig}{}\ledrightnote{\textcolor{pink}{Venedig}}. –\pend
           \pstart
           Im übrigen arbeite ich und fühl mich aus den bekannten Ursachen nicht wohl. – (Milder
               Ausdruck.)\pend
           \pstart
           Brief und Carton hab ich erhalten, danke sehr. Wie gehts Ihnen? Machen Sie was? \textcolor{blue}{Paul G.}{}\ledrightnote{\textcolor{blue}{Paul Goldmann}} hat Recht, sag ich Ihnen! – \textcolor{blue}{Gustav Schw.}{}\ledrightnote{\textcolor{blue}{Gustav Schwarzkopf}} und \textcolor{blue}{Leo V.}{}\ledrightnote{\textcolor{blue}{Leo Van-Jung}} werden sicher Ihre Grüsse erwidern, sobald ich sie ihnen ausgerichtet
               habe. – Das gleiche nehm ich von \textcolor{blue}{Paula}{}\ledrightnote{\textcolor{blue}{Paula Beer-Hofmann}}, ja beinah
               von \textcolor{blue}{Mirjam}{}\ledrightnote{\textcolor{blue}{Mirjam Beer-Hofmann}} an. Sie wird einmal sehr gerührt sein,
               wenn sie als alte Frau ihrer Enkelin das \textcolor{green}{Gedicht}{}\ledrightnote{→\textcolor{green}{Schlaflied für Mirjam}} vom Urgrosspapa vorlesen wird. Und auch Ihrer Urenkelin
               werden vielleicht Thränen ins Auge kommen. Auf Wiedersehen, womöglich noch
               vorher.\pend
           \pstart Herzlich Ihr \spacefill\mbox{Arthur.}\pend{}\pstart
           \noindent{}(nach \textcolor{pink}{Steindorf}{}\ledrightnote{\textcolor{pink}{Steindorf am Ossiacher See}})\pend
           {\bigskip}\pstart
           \noindent{}{\pb}\textcolor{green}{Strophe I}{}\ledrightnote{→\textcolor{green}{Schlaflied für Mirjam}}\pend
           \settowidth{\longeste}{Zeile}\settowidth{\longestz}{5}\settowidth{\longestd}{nach ; ein –}\settowidth{\longestv}{}\settowidth{\longestf}{}\addtolength\longeste{1em}
        \addtolength\longestz{1em}
        \addtolength\longestd{1em}
      \pstart\noindent\makebox[\the\longeste][l]{Zeile}\makebox[\the\longestz][l]{2}
                  \makebox[\the\longestd][l]{nach Sieh \uuline{,}}\pend\pstart\noindent\makebox[\the\longeste][l]{Zeile}\makebox[\the\longestz][l]{3}
                  \makebox[\the\longestd][l]{– fort!}\pend\pstart\noindent\makebox[\the\longeste][l]{Zeile}\makebox[\the\longestz][l]{5}
                  \makebox[\the\longestd][l]{nach ; ein –}\pend\pstart
           \textcolor{green}{Strophe II}{}\ledrightnote{→\textcolor{green}{Schlaflied für Mirjam}}\pend
           \settowidth{\longeste}{Zeile}\settowidth{\longestz}{6,}\settowidth{\longestd}{das auch stört nicht.}\settowidth{\longestv}{}\settowidth{\longestf}{}\addtolength\longeste{1em}
        \addtolength\longestz{1em}
        \addtolength\longestd{1em}
      \pstart\noindent\makebox[\the\longeste][l]{Zeile}\makebox[\the\longestz][l]{2}
                  \makebox[\the\longestd][l]{ſtatt – lieber ,}\pend\pstart\noindent\makebox[\the\longeste][l]{}\makebox[\the\longestz][l]{4}
                  \makebox[\the\longestd][l]{das \uline{auch} stört nicht.}\pend\pstart\noindent\makebox[\the\longeste][l]{Zeile}\makebox[\the\longestz][l]{6,}
                  \makebox[\the\longestd][l]{lieber kein –}\pend\pstart
           \uline{\textcolor{green}{Strophe III}{}\ledrightnote{→\textcolor{green}{Schlaflied für Mirjam}}}\pend
           \settowidth{\longeste}{Zeile}\settowidth{\longestz}{7}\settowidth{\longestd}{iſt ein Beiſtrich; an den gleichen Stellen Str I u II fehlt er –}\settowidth{\longestv}{}\settowidth{\longestf}{}\addtolength\longeste{1em}
        \addtolength\longestz{1em}
        \addtolength\longestd{1em}
      \pstart\noindent\makebox[\the\longeste][l]{Zeile}\makebox[\the\longestz][l]{1}
                  \makebox[\the\longestd][l]{– fort!}\pend\pstart\noindent\makebox[\the\longeste][l]{Zeile}\makebox[\the\longestz][l]{2}
                  \makebox[\the\longestd][l]{ebenſo}\pend\pstart\noindent\makebox[\the\longeste][l]{Zeile}\makebox[\the\longestz][l]{7}
                  \makebox[\the\longestd][l]{iſt ein Beiſtrich; an den gleichen Stellen Str I u II fehlt er –}\pend\pstart\noindent\makebox[\the\longeste][l]{}\makebox[\the\longestz][l]{}
                  \makebox[\the\longestd][l]{eins von beiden! –}\pend\pstart
           \textcolor{green}{Strophe IV}{}\ledrightnote{→\textcolor{green}{Schlaflied für Mirjam}}\pend
           \settowidth{\longeste}{Zeile}\settowidth{\longestz}{6,}\settowidth{\longestd}{der erſte – fort}\settowidth{\longestv}{}\settowidth{\longestf}{}\addtolength\longeste{1em}
        \addtolength\longestz{1em}
        \addtolength\longestd{1em}
      \pstart\noindent\makebox[\the\longeste][l]{Zeile}\makebox[\the\longestz][l]{4}
                  \makebox[\the\longestd][l]{lieber \uline{,} statt –}\pend\pstart\noindent\makebox[\the\longeste][l]{Zeile}\makebox[\the\longestz][l]{6,}
                  \makebox[\the\longestd][l]{der erſte – fort}\pend\pstart\noindent\makebox[\the\longeste][l]{Zeile}\makebox[\the\longestz][l]{7}
                  \makebox[\the\longestd][l]{der letzte –}\pend{\bigskip}\pstart
           \noindent{}{\pb}Schlaflied für \textcolor{blue}{Mirjam}{}\ledrightnote{\textcolor{blue}{Mirjam Beer-Hofmann}}\pend
           {\bigskip}\stanza{}Schlaf mein Kind – schlaf, es iſt spät.\newverse{}Sieh, wie die Sonne zur Ruh dort geht;\newverse{}Hinter den Bergen ſtirbt ſie im Roth.\newverse{}Du, – du weißt nichts von Sonne und Tod,\newverse{}Wendeſt die Augen zum Licht und zum Schein\newverse{}Schlaf – es ſind ſo viel Sonnen noch dein,\newverse{}Schlaf mein Kind – mein Kind, ſchlaf ein.\stanzaend{}\stanza{}– Schlaf mein Kind – der Abendwind weht\newverse{}Weiß man, woher er ko{\geminationm}t – wohin er geht?\newverse{}Dunkel, verborgen die Wege hier ſind\newverse{}Dir, und mir, und uns allen mein Kind.\newverse{}Blinde ſo geh’n wir, und gehen allein\newverse{}Keiner kann Keinem Gefährte hier ſein –\newverse{}Schlaf mein Kind {[}–{]} mein Kind ſchlaf ein\stanzaend{}\stanza{}{\pb}Schlaf mein Kind – und horch
                     nicht auf mich;\newverse{}Sinn hat’s für mich nur – und Schall iſts für dich.\newverse{}Schall nur, wie Windeswehn, Waſſergerinn,\newverse{}Worte – vielleicht eines Lebens Gewinn.\newverse{}Was ich gewonnen, gräbt mit mir man ein,\newverse{}Keiner ka{\geminationn} Keinem ein Erbe hier sein,\newverse{}Schlaf mein Kind – mein Kind ſchlaf ein.\stanzaend{}\stanza{}Schläfſt du \textcolor{blue}{Mirjam}{}\ledrightnote{\textcolor{blue}{Mirjam Beer-Hofmann}}? – \textcolor{blue}{Mirjam}{}\ledrightnote{\textcolor{blue}{Mirjam Beer-Hofmann}} mein Kind,\newverse{}Ufer nur ſind wir, und tief in uns rinnt\newverse{}Blut von Geweſ’nen – zu Ko{\geminationm}enden rollt’s;\newverse{}Blut unſrer Väter, voll Unruh und Stolz.\newverse{}In uns sind alle; wer fühlt ſich allein?\newverse{}Du biſt ihr Leben – ihr Leben iſt dein,\newverse{}\textcolor{blue}{Mirjam}{}\ledrightnote{\textcolor{blue}{Mirjam Beer-Hofmann}} mein Leben – mein Kind ſchlaf
                     ein.\stanzaend{}\pstart
           \spacefill\mbox{Richard Beer-Hofmann}\pend
           \endnumbering\briefempfaengerindex{Beer-Hofmann, Richard@\textsc{Beer-Hofmann, Richard}!zzzSchnitzler, Arthur@\emph{von Arthur Schnitzler}!1898-06-171@{17. 6. 1898}|)be}\mylabel{h}  \normalsize

\doendnotes{C}
\bigskip
\vfill

\clearpage

\footnotesize

\lohead{\textsc{register}}

% Definiere theindex-Environment komplett neu ohne reledmac
\makeatletter
\renewenvironment{theindex}{%
  \section*{\indexname}%
  \setlength{\parindent}{0pt}%
  \setlength{\parskip}{0pt plus 0.3pt}%
  \let\item\@idxitem
}{%
  \clearpage
}
\makeatother

\IfFileExists{\jobname-pw.ind}{\input{\jobname-pw.ind}}{}

\end{document}

      