%% latex-korrekturansicht-vorspann.tex
%% Vorspann für die Korrekturansicht.
%% Lädt die gemeinsame Datei latex-vorspann.tex mit gesetztem Schalter.

\newif\ifkorrekturansicht
\korrekturansichttrue

\input{../tex-inputs/latex-vorspann}


               \section[Arthur Schnitzler an Hermann Bahr, 16. 5. 1902]{ Arthur Schnitzler an Hermann Bahr, 16. 5. 1902}\nopagebreak\mylabel{v}\rehead{ }\normalsize\beginnumbering\briefempfaengerindex{Bahr, Hermann@\textsc{Bahr, Hermann}!zzzSchnitzler, Arthur@\emph{von Arthur Schnitzler}!1902-05-161@{16. 5. 1902}|(be} \toendnotes[C]{\smallbreak\pagebreak[2]} \Standort{TMW, HS AM 23351 Ba.}
\physDesc{Brief, 1 Blatt, 3 Seiten
\newline{}Handschrift: schwarze Tinte, deutsche Kurrent\newline{}Ordnung: Lochung }\buchAbdrucke{\weitereDrucke{1) \emph{16. 5. 1902, Abschrift.} In: Arthur Schnitzler: \emph{The Letters of Arthur Schnitzler to Hermann Bahr}. Edited, annotated, and with an introduction, by Donald G.
                        Daviau. Chapel Hill: \emph{The University of North Carolina Press} 1978, S. 75 (University of North Carolina studies in the Germanic languages
                        and literatures, 89).} \weitereDrucke{2) Hermann Bahr, Arthur Schnitzler: \emph{Briefwechsel, Aufzeichnungen, Dokumente (1891–1931)}. Hg. Kurt Ifkovits und Martin Anton Müller. Göttingen: \emph{Wallstein} 2018, S. 238.} }\toendnotes[C]{\smallbreak}\pstart{}{\pb}mein lieber Hermann,\pend\pstart
           bevor ich zu dir hinausko{\geminationm}e, dir für deinen guten
               ſchönen Brief zu danken, wollte ich dir heute ſchon ſagen, wie herzlich er mich
               gefreut hat – und daſs die Blumen, d\damage{ie} du mir \introOben{}ge\introOben{}ſchickt haſt, mindeſtens ebenſo wohl u
               herrlich duften als wenn ſie von einem weiblichen Weſen kämen – und je{\pb}denfalls zu den freundlichſten Enttäuſchungen gehören, die
               mir geworden ſind – Noch mehreres wollte ich dir ſchreiben, was aber zu leſen dir
               heute die Sti{\geminationm}ung fehlen wird, denn eben leſe ich daſs
               deine \label{K_L01220_1v}\edtext{\textcolor{blue}{Mutter}{}\ledrightnote{→\textcolor{blue}{Wilhelmine Bahr}} geſtorben}{\lemma{\textnormal{\emph{Mutter geſtorben}}}\Cendnote{\textnormal{\textcolor{blue}{Mina Bahr}
                   starb am
                     15. 5. 1902 in \textcolor{pink}{Salzburg}. Eine Meldung brachte etwa die \emph{\textcolor{brown}{Neue Freie Presse}}, Nr. 13551,
                        16. 5. 1902, Abendblatt, S. 2.}}}\label{K_L01220_1h}
               iſt, und ſo ka{\geminationn} ich für heute nichts anderes mehr ſagen,
               als daſs ich dich bitte, an die innigſte {\pb}Theilnahme eines
               Menſchen zu glauben, der dein Freund \uline{geworden} iſt.
               Und was man ſo allmälig wurde, bleibt man – beſonders in unſeren Jahren. Nicht mehr
               für heute. Ich hoffe dich bald zu ſehen.\pend
           \pstart
           In Treue dein{\\[\baselineskip]}\spacefill\mbox{Arthur}\pend
           \leftskip=0em{}\pstart
           \textcolor{pink}{Wien}{}\ledrightnote{\textcolor{pink}{Wien}}{ }16. 5. 902\pend
           \endnumbering\briefempfaengerindex{Bahr, Hermann@\textsc{Bahr, Hermann}!zzzSchnitzler, Arthur@\emph{von Arthur Schnitzler}!1902-05-161@{16. 5. 1902}|)be}\mylabel{h}  \normalsize

\doendnotes{C}
\bigskip
\vfill

\clearpage

\footnotesize

\lohead{\textsc{register}}

% Definiere theindex-Environment komplett neu ohne reledmac
\makeatletter
\renewenvironment{theindex}{%
  \section*{\indexname}%
  \setlength{\parindent}{0pt}%
  \setlength{\parskip}{0pt plus 0.3pt}%
  \let\item\@idxitem
}{%
  \clearpage
}
\makeatother

\IfFileExists{\jobname-pw.ind}{\input{\jobname-pw.ind}}{}

\end{document}

      