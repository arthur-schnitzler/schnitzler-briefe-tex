%% latex-korrekturansicht-vorspann.tex
%% Vorspann für die Korrekturansicht.
%% Lädt die gemeinsame Datei latex-vorspann.tex mit gesetztem Schalter.

\newif\ifkorrekturansicht
\korrekturansichttrue

\input{../tex-inputs/latex-vorspann}


               \section[Hermann Bahr an Arthur Schnitzler, 27. 4. 1907]{ Hermann Bahr an Arthur Schnitzler, 27. 4. 1907}\nopagebreak\mylabel{v}\rehead{ }\normalsize\beginnumbering\briefempfaengerindex{Schnitzler, Arthur@\textsc{Schnitzler, Arthur}!zzzBahr, Hermann@\emph{von Hermann Bahr}!1907-04-271@{27. 4. 1907}|(be} \toendnotes[C]{\smallbreak\pagebreak[2]} \Standort{CUL, Schnitzler, B 5b.}
\physDesc{Kartenbrief
\newline{}Handschrift: Bleistift, deutsche Kurrent\newline{}Versand: 1) Stempel: »\nobreak{}\oindex{X., Favoriten@\textbf{X., Favoriten}, \emph{Bezirk (A.BZK)}|pwk}Wien 10/2, 27. IV. 07, 8\nobreak{}«.  2) Stempel: »\nobreak{}\oindex{XVIII., Waehring@\textbf{XVIII., Währing}, \emph{Bezirk (A.BZK)}|pwk}18/1 Wien, 28. IV. 07\nobreak{}«. \newline{}Ordnung: mit Bleistift von unbekannter Hand nummeriert: »149« }\buchAbdrucke{\weitereDrucke{Hermann Bahr, Arthur Schnitzler: \emph{Briefwechsel, Aufzeichnungen, Dokumente (1891–1931)}. Hg. Kurt Ifkovits und Martin Anton Müller. Göttingen: \emph{Wallstein} 2018, S. 392.} }\toendnotes[C]{\smallbreak}\pstart{}{\pb}\textsc{D\textsuperscript{r} Artur
                  Schnitzler}\pend{}\pstart{}\textsc{\textcolor{pink}{Wien XVIII}{}\ledrightnote{\textcolor{pink}{XVIII., Währing}}}\pend{}\pstart{}\textsc{\textcolor{pink}{Spöttelgasse 7}{}\ledrightnote{\textcolor{pink}{Edmund-Weiß-Gasse}}}\pend{}{\bigskip}\pstart
           \raggedleft{}{\pb}\textcolor{pink}{Südbahnhof}{}\ledrightnote{\textcolor{pink}{Südbahnhof}}{\\}27. 4\pend
           \pstart
           Ich fahre eben auf den \textcolor{pink}{Semmering (Südbahnhotel)}{}\ledrightnote{\textcolor{pink}{Südbahnhotel}}, um
                  \label{K_L01671_1v}\edtext{Dienſtag}{\lemma{\textnormal{\emph{Dienſtag}}}\Cendnote{\textnormal{den 30. 4. 1907}}}\label{K_L01671_1h} Früh von dort mit dem Frühſchnellzug nach \textcolor{pink}{Trieſt}{}\ledrightnote{\textcolor{pink}{Triest}} zu fahren. \label{K_L01671_2v}\edtext{Komm mit!}{\lemma{\textnormal{\emph{Komm mit!}}}\Cendnote{\textnormal{Einen
                  inhaltlich und sprachlich beinahe identischen Brief schreibt \textcolor{blue}{Bahr} am selben Tag an \textcolor{blue}{Beer-Hofmann}.}}}\label{K_L01671_2h} In dieſem Fall erwartet bis Montag Mittag
               telegrafiſche Nachricht\pend
           \pstart
           Dein{\\[\baselineskip]}\spacefill\mbox{Hermann B}\pend
           \leftskip=0em{}\endnumbering\briefempfaengerindex{Schnitzler, Arthur@\textsc{Schnitzler, Arthur}!zzzBahr, Hermann@\emph{von Hermann Bahr}!1907-04-271@{27. 4. 1907}|)be}\mylabel{h}  \normalsize

\doendnotes{C}
\bigskip
\vfill

\clearpage

\footnotesize

\lohead{\textsc{register}}

% Definiere theindex-Environment komplett neu ohne reledmac
\makeatletter
\renewenvironment{theindex}{%
  \section*{\indexname}%
  \setlength{\parindent}{0pt}%
  \setlength{\parskip}{0pt plus 0.3pt}%
  \let\item\@idxitem
}{%
  \clearpage
}
\makeatother

\IfFileExists{\jobname-pw.ind}{\input{\jobname-pw.ind}}{}

\end{document}

      