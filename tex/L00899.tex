%% latex-korrekturansicht-vorspann.tex
%% Vorspann für die Korrekturansicht.
%% Lädt die gemeinsame Datei latex-vorspann.tex mit gesetztem Schalter.

\newif\ifkorrekturansicht
\korrekturansichttrue

\input{../tex-inputs/latex-vorspann}


               \section[Arthur Schnitzler an Hermann Bahr, 7. 3. 1899]{ Arthur Schnitzler an Hermann Bahr, 7. 3. 1899}\nopagebreak\mylabel{v}\rehead{ }\normalsize\beginnumbering\briefempfaengerindex{Bahr, Hermann@\textsc{Bahr, Hermann}!zzzSchnitzler, Arthur@\emph{von Arthur Schnitzler}!1899-03-071@{7. 3. 1899}|(be} \toendnotes[C]{\smallbreak\pagebreak[2]} \Standort{TMW, HS AM 23335 Ba.}
\physDesc{Brief, 1 Blatt (Briefpapier mit Trauerrand), 2 Seiten
\newline{}Handschrift: schwarze Tinte, deutsche Kurrent\newline{}Ordnung: Lochung }\buchAbdrucke{\weitereDrucke{1) \emph{7. 3. 1899, Abschrift.} In: Arthur Schnitzler: \emph{The Letters of Arthur Schnitzler to Hermann Bahr}. Edited, annotated, and with an introduction, by Donald G.
                        Daviau. Chapel Hill: \emph{The University of North Carolina Press} 1978, S. 65–66 (University of North Carolina studies in the Germanic languages
                        and literatures, 89).} \weitereDrucke{2) Hermann Bahr, Arthur Schnitzler: \emph{Briefwechsel, Aufzeichnungen, Dokumente (1891–1931)}. Hg. Kurt Ifkovits und Martin Anton Müller. Göttingen: \emph{Wallstein} 2018, S. 169.} }\toendnotes[C]{\smallbreak}\pstart{}{\pb}Lieber
                  Bahr,\pend\pstart
           als meine \textcolor{green}{3
                  Einakter}{}\ledrightnote{→\textcolor{green}{Paracelsus. Versspiel in einem Akt}{\newline}→\textcolor{green}{Die Gefährtin. Schauspiel in einem Akt}{\newline}→\textcolor{green}{Der grüne Kakadu. Groteske in einem Akt}} angekündigt wurden wünſchteſt du einen davon. Ich \label{K_L00899_1v}\edtext{verſprach dir bald darauf die »\textcolor{green}{Gefährtin}{}\ledrightnote{\textcolor{green}{Die Gefährtin. Schauspiel in einem Akt}}«, du nahmſt an}{\lemma{\textnormal{\emph{verſprach … an}}}\Cendnote{\textnormal{Arthur Schnitzler an Hermann Bahr, 1. 12. 1898}}}\label{K_L00899_1h}. Du fragteſt wieder; ich ſagte dir
               das \textsc{Manuscript} nach der Aufführg zu. Damit band ich mich
               und beantwortete Aufforderungen von andrer Seite \label{K_L00899_2v}\edtext{abſchlägig}{\lemma{\textnormal{\emph{abſchlägig}}}\Cendnote{\textnormal{Es erschien,
                  nach der Absage \textcolor{blue}{Bahrs}, in keinem anderen
                  Organ.}}}\label{K_L00899_2h}. Nun ſteckſt du plötzlich »ſo tief in alten Verpflichtungen«, daſs
               du das Stück {\pb}nicht
               bringen kannſt. – Tr\damage{otz}dem Du durch den Aufſchub der \textcolor{green}{Sobeïde}{}\ledrightnote{\textcolor{green}{Die Hochzeit der Sobeide}} 2
               oder 3 Nummern freibekommen haſt! – \pend
           \pstart
           Dieſer Sachverhalt ſei hiemit conſtatirt. Jede weitere Discuſſion darüber lehne ich
               ab.\pend
           \pstart
           Besten Gruſs. Dein ergebner{\\[\baselineskip]}\spacefill\mbox{Arthur Schnitzler}\pend
           \leftskip=0em{}\pstart
           \textcolor{pink}{Wien}{}\ledrightnote{\textcolor{pink}{Wien}}{ }7. 3. 99.\pend
           \endnumbering\briefempfaengerindex{Bahr, Hermann@\textsc{Bahr, Hermann}!zzzSchnitzler, Arthur@\emph{von Arthur Schnitzler}!1899-03-071@{7. 3. 1899}|)be}\mylabel{h}  \normalsize

\doendnotes{C}
\bigskip
\vfill

\clearpage

\footnotesize

\lohead{\textsc{register}}

% Definiere theindex-Environment komplett neu ohne reledmac
\makeatletter
\renewenvironment{theindex}{%
  \section*{\indexname}%
  \setlength{\parindent}{0pt}%
  \setlength{\parskip}{0pt plus 0.3pt}%
  \let\item\@idxitem
}{%
  \clearpage
}
\makeatother

\IfFileExists{\jobname-pw.ind}{\input{\jobname-pw.ind}}{}

\end{document}

      