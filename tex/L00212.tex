%% latex-korrekturansicht-vorspann.tex
%% Vorspann für die Korrekturansicht.
%% Lädt die gemeinsame Datei latex-vorspann.tex mit gesetztem Schalter.

\newif\ifkorrekturansicht
\korrekturansichttrue

\input{../tex-inputs/latex-vorspann}


               \section[Karl Kraus an Arthur Schnitzler, 5. 5. 1893]{ Karl Kraus an Arthur Schnitzler, 5. 5. 1893}\nopagebreak\mylabel{v}\rehead{ }\normalsize\beginnumbering\briefempfaengerindex{Schnitzler, Arthur@\textsc{Schnitzler, Arthur}!zzzKraus, Karl@\emph{von Karl Kraus}!1893-05-051@{5. 5. 1893}|(be} \toendnotes[C]{\smallbreak\pagebreak[2]} \Standort{DLA, A:Schnitzler, HS.NZ85.1.3790, S. 11–12.}
\physDesc{maschinelle Abschrift}\buchAbdrucke{\weitereDrucke{\emph{Karl Kraus und Arthur Schnitzler. Eine Dokumentation.} Hg. Reinhard Urbach. In: \emph{Literatur und Kritik}, Bd. 49, Oktober 1970, S. 518.} }\toendnotes[C]{\smallbreak}\pstart
           \raggedleft{}{\pb}\textcolor{pink}{Wien}{}\ledrightnote{\textcolor{pink}{Wien}}, 5. Mai 1893. \pend
           \pstart
           Liebster Herr Doctor! Beiliegend sende ich Ihnen den \textcolor{green}{Kritikausschnitt}{}\ledrightnote{→\textcolor{green}{Wiener Dichter}} aus Nr. 18
                    des \textcolor{green}{Magazin}{}\ledrightnote{\textcolor{green}{Magazin für die Literatur des Auslandes}} (6. Mai), das mir
                    eben zuging. – Leider konnte ich gestern{ }½ 10 nicht im \textcolor{pink}{Trauerhause}{}\ledrightnote{→\textcolor{pink}{Burgring}} erscheinen, da ich die Parte erst vormit{\pb}tags
                    gestern erhielt. Nochmals auf diesem Wege mein herzlichstes \textcolor{blue}{Beileid}{}\ledrightnote{→\textcolor{blue}{Johann Schnitzler}} und viel Grüsse von Ihrem\pend
           \pstart treuen \spacefill\mbox{Karl Kraus}\pend{}\endnumbering\briefempfaengerindex{Schnitzler, Arthur@\textsc{Schnitzler, Arthur}!zzzKraus, Karl@\emph{von Karl Kraus}!1893-05-051@{5. 5. 1893}|)be}\mylabel{h}  \normalsize

\doendnotes{C}
\bigskip
\vfill

\clearpage

\footnotesize

\lohead{\textsc{register}}

% Definiere theindex-Environment komplett neu ohne reledmac
\makeatletter
\renewenvironment{theindex}{%
  \section*{\indexname}%
  \setlength{\parindent}{0pt}%
  \setlength{\parskip}{0pt plus 0.3pt}%
  \let\item\@idxitem
}{%
  \clearpage
}
\makeatother

\IfFileExists{\jobname-pw.ind}{\input{\jobname-pw.ind}}{}

\end{document}

      