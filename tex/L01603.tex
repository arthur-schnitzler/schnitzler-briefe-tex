%% latex-korrekturansicht-vorspann.tex
%% Vorspann für die Korrekturansicht.
%% Lädt die gemeinsame Datei latex-vorspann.tex mit gesetztem Schalter.

\newif\ifkorrekturansicht
\korrekturansichttrue

\input{../tex-inputs/latex-vorspann}


               \section[Arthur Schnitzler an Richard Beer-Hofmann, 24. 6. 1906]{ Arthur Schnitzler an Richard Beer-Hofmann, 24. 6. 1906}\nopagebreak\mylabel{v}\rehead{ }\normalsize\beginnumbering\briefempfaengerindex{Beer-Hofmann, Richard@\textsc{Beer-Hofmann, Richard}!zzzSchnitzler, Arthur@\emph{von Arthur Schnitzler}!1906-06-241@{24. 6. 1906}|(be} \toendnotes[C]{\smallbreak\pagebreak[2]} \Standort{YCGL, MSS 31.}
\physDesc{Kartenbrief
\newline{}Handschrift: schwarze Tinte, deutsche Kurrent\newline{}Versand: 1) Stempel: »\nobreak{}\oindex{IX., Alsergrund@\textbf{IX., Alsergrund}, \emph{Bezirk (A.BZK)}|pwk}9/4 Wien 68, 24. VI. 06, XII\nobreak{}«.  2) Stempel: »\nobreak{}\oindex{Rodaun@\textbf{Rodaun}, \emph{Teil eines besiedelten Ortes (A.BSOX)}|pwk}R{[}odaun{]}\nobreak{}«. }\pstart{}{\pb}\textsc{Dr. Richard Beer-Hofmann}\pend{}\pstart{}\textcolor{pink}{\textsc{Rodaun}}{}\ledrightnote{\textcolor{pink}{Rodaun}}\pend{}\pstart{}\textcolor{pink}{\textsc{Liesingerstraße} 1}{}\ledrightnote{\textcolor{pink}{Liesingerstraße}}.\pend{}{\bigskip}\pstart
           \raggedleft{}{\pb}24. 6. 906\pend
           \pstart{}lieber Richard, \pend\pstart
           wir reiſen ſchon morgen Montag ab. \textcolor{pink}{Berlin}{}\ledrightnote{\textcolor{pink}{Berlin}}. \textcolor{pink}{Kopenhagen}{}\ledrightnote{\textcolor{pink}{Kopenhagen}}. \textcolor{pink}{\textsc{Marienlyst}}{}\ledrightnote{\textcolor{pink}{Marienlyst}}, wo wir ein paar Wochen zu beiben hoffen. Ich weiſs nicht, ob Sie ſchon aus \textcolor{pink}{Velden}{}\ledrightnote{\textcolor{pink}{Velden}} zurück ſind, keinesfalls können wir vor
               unſerer Abreiſe noch nach \textcolor{pink}{Rodaun}{}\ledrightnote{\textcolor{pink}{Rodaun}} ko{\geminationm}en – alſo hoffentlich, nach gutem Sommer ein gutes
               Wiederſehen im Herbſt. \pend
           \pstart
           Von \damage{\textcolor{gray}{Herzen}}, mit Grüßen von uns an Sie alle\pend
           \pstart
           Ihr{\\[\baselineskip]}\spacefill\mbox{A.}\pend
           \leftskip=0em{}\endnumbering\briefempfaengerindex{Beer-Hofmann, Richard@\textsc{Beer-Hofmann, Richard}!zzzSchnitzler, Arthur@\emph{von Arthur Schnitzler}!1906-06-241@{24. 6. 1906}|)be}\mylabel{h}  \normalsize

\doendnotes{C}
\bigskip
\vfill

\clearpage

\footnotesize

\lohead{\textsc{register}}

% Definiere theindex-Environment komplett neu ohne reledmac
\makeatletter
\renewenvironment{theindex}{%
  \section*{\indexname}%
  \setlength{\parindent}{0pt}%
  \setlength{\parskip}{0pt plus 0.3pt}%
  \let\item\@idxitem
}{%
  \clearpage
}
\makeatother

\IfFileExists{\jobname-pw.ind}{\input{\jobname-pw.ind}}{}

\end{document}

      