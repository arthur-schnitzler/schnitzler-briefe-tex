%% latex-korrekturansicht-vorspann.tex
%% Vorspann für die Korrekturansicht.
%% Lädt die gemeinsame Datei latex-vorspann.tex mit gesetztem Schalter.

\newif\ifkorrekturansicht
\korrekturansichttrue

\input{../tex-inputs/latex-vorspann}


               \section[Richard Beer-Hofmann an Arthur Schnitzler, 8. 9. 1904]{ Richard Beer-Hofmann an Arthur Schnitzler, 8. 9. 1904}\nopagebreak\mylabel{v}\rehead{ }\normalsize\beginnumbering\briefempfaengerindex{Schnitzler, Arthur@\textsc{Schnitzler, Arthur}!zzzBeer-Hofmann, Richard@\emph{von Richard Beer-Hofmann}!1904-09-081@{8. 9. 1904}|(be} \toendnotes[C]{\smallbreak\pagebreak[2]} \Standort{CUL, Schnitzler, B 8.}
\physDesc{Brief, 1 Blatt, 1 Seite
\newline{}Handschrift: blaue Tinte, lateinische Kurrent\newline{}Ordnung: mit Bleistift von unbekannter Hand nummeriert: »188« }\toendnotes[C]{\smallbreak}\pstart
           \centering{}{\pb}\textcolor{pink}{Aussee}{}\ledrightnote{\textcolor{pink}{Bad Aussee}}{ }8/IX. 04.\pend
           \pstart
           Lieber Arthur! \textcolor{blue}{Zauner}{}\ledrightnote{\textcolor{blue}{Franz Zauner}}
               (nicht der \textcolor{blue}{Bildhauer}{}\ledrightnote{→\textcolor{blue}{Franz Anton von Zauner}} – \label{K_L01440_1v}\edtext{der ist todt}{\lemma{\textnormal{\emph{der ist todt}}}\Cendnote{\textnormal{Unter der Annahme,
                  dass es sich bei \textcolor{blue}{Zauner} ebenfalls um einen
                  »Franz Zauner« handelt, könnte es sich um einen Stukkateur handeln, der
                     1904 in \textcolor{pink}{Wien} tätig war.}}}\label{K_L01440_1h})
               sendet an Sie das Gewünschte.\pend
           \pstart
           Wenn es so fortregnet bleiben Sie ja wol kaum in \textcolor{pink}{Lueg}{}\ledrightnote{\textcolor{pink}{Lueg am Wolfgangsee}}.
               Schreiben Sie mir, ob Sie nicht doch lieber ko{\geminationm}en, wo
               das Wetter durch meine Anwesenheit sich ja wesentlich mildert. Wenn Sie nicht ko{\geminationm}en – verständigen Sie mich, \strikeout{was} ob und wann Sie nach \textcolor{pink}{Salzburg}{}\ledrightnote{\textcolor{pink}{Salzburg}} gehen.
               Ich möchte nächste Woche – Beginn – auf 2 Tage hin.\pend
           \pstart
           Herzlichst{\\[\baselineskip]}Ihr{\\[\baselineskip]}\spacefill\mbox{Richard.}\pend
           \leftskip=0em{}\endnumbering\briefempfaengerindex{Schnitzler, Arthur@\textsc{Schnitzler, Arthur}!zzzBeer-Hofmann, Richard@\emph{von Richard Beer-Hofmann}!1904-09-081@{8. 9. 1904}|)be}\mylabel{h}  \normalsize

\doendnotes{C}
\bigskip
\vfill

\clearpage

\footnotesize

\lohead{\textsc{register}}

% Definiere theindex-Environment komplett neu ohne reledmac
\makeatletter
\renewenvironment{theindex}{%
  \section*{\indexname}%
  \setlength{\parindent}{0pt}%
  \setlength{\parskip}{0pt plus 0.3pt}%
  \let\item\@idxitem
}{%
  \clearpage
}
\makeatother

\IfFileExists{\jobname-pw.ind}{\input{\jobname-pw.ind}}{}

\end{document}

      