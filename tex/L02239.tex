%% latex-korrekturansicht-vorspann.tex
%% Vorspann für die Korrekturansicht.
%% Lädt die gemeinsame Datei latex-vorspann.tex mit gesetztem Schalter.

\newif\ifkorrekturansicht
\korrekturansichttrue

\input{../tex-inputs/latex-vorspann}


               \section[Arthur Schnitzler an Richard Beer-Hofmann, 23. 8. 1916]{ Arthur Schnitzler an Richard Beer-Hofmann, 23. 8. 1916}\nopagebreak\mylabel{v}\rehead{ }\normalsize\beginnumbering\briefempfaengerindex{Beer-Hofmann, Richard@\textsc{Beer-Hofmann, Richard}!zzzSchnitzler, Arthur@\emph{von Arthur Schnitzler}!1916-08-231@{23. 8. 1916}|(be} \toendnotes[C]{\smallbreak\pagebreak[2]} \Standort{YCGL, MSS 31.}
\physDesc{Kartenbrief
\newline{}Handschrift: Bleistift, deutsche Kurrent\newline{}Versand: Stempel: »\nobreak{}\oindex{Altaussee@\textbf{Altaussee}, \emph{http://www.geonames.org/ontologyA.ADM3}|pwk}Alt Aussee, 23. VIII. 1\textcolor{gray}{6}\nobreak{}«.  
\newline{}Beer-Hofmann: mit blauem Buntstift den Empfang vermerkt:
                                 »E« }\buchAbdrucke{\weitereDrucke{Arthur Schnitzler, Richard Beer-Hofmann: \emph{Briefwechsel 1891–1931}. Hg. Konstanze Fliedl. Wien, Zürich: \emph{Europaverlag} 1992, S. 222.} }\toendnotes[C]{\smallbreak}\pstart{}{\pb}\textsc{Schnitzler}.\pend{}\pstart{}\textsc{\textcolor{pink}{Altaussee}{}\ledrightnote{\textcolor{pink}{Altaussee}}}\pend{}\pstart{}\textsc{\textcolor{pink}{Fischerndorf 79}{}\ledrightnote{\textcolor{pink}{Fischerndorf}}}\pend{}{\bigskip}\pstart{}\textsc{Herrn Dr. Richard Beer-Hofmann}\pend{}\pstart{}\textcolor{pink}{\textsc{Bad Ischl}}{}\ledrightnote{\textcolor{pink}{Bad Ischl}}\pend{}\pstart{}\textcolor{pink}{\textsc{Grazerstr 52}}{}\ledrightnote{\textcolor{pink}{Grazer Straße}}.\pend{}{\bigskip}\pstart
           \raggedleft{}{\pb}\textcolor{pink}{Altaussee}{}\ledrightnote{\textcolor{pink}{Altaussee}},{\\}23. 8. 1916\pend
           \pstart
           lieber Richard, vielen Dank für Ihre Bemühungen und das Telegra{\geminationm} – nun ko{\geminationm}en wir doch
               nicht nach \textcolor{pink}{Iſchl}{}\ledrightnote{\textcolor{pink}{Bad Ischl}} (\uline{dem \textcolor{pink}{Kreuz}{}\ledrightnote{\textcolor{pink}{Goldenes Kreuz}} hab ich natürlich ſchon abtelegrafirt}) –
               nicht ſo ſehr wegen des Wetters, als weil sich \textcolor{blue}{\textsc{Steiners}}{}\ledrightnote{\textcolor{blue}{Franz Steiner}{\newline}\textcolor{blue}{Margit Steiner}} gerade für \label{K_L02239-1v}\edtext{Freitag}{\lemma{\textnormal{\emph{Freitag}}}\Cendnote{\textnormal{siehe A. S.: \emph{Tagebuch}, 25. 8. 1916}}}\label{K_L02239-1h} bei uns
               angeſagt haben.\pend
           \pstart
           – Von meiner \textcolor{blue}{Schwägerin}{}\ledrightnote{→\textcolor{blue}{Elisabeth Steinrück}} ko{\geminationm}en etwas bedenkliche Nachrichten; es iſt ſehr möglich,
               daſs \textcolor{blue}{Olga}{}\ledrightnote{\textcolor{blue}{Olga Schnitzler}} (wenn ſie das Paſsviſum bekommt) auf
               8–12 Tage nach \textcolor{pink}{Partenkirchen}{}\ledrightnote{\textcolor{pink}{Partenkirchen}} fährt – auch ich
               bemühe mich um ein Viſum, – warte aber jedenfalls, wenn \textcolor{blue}{Olga}{}\ledrightnote{\textcolor{blue}{Olga Schnitzler}}{ }\strikeout{\textcolor{gray}{×}\-\textcolor{gray}{×}\-\textcolor{gray}{×}\-\textcolor{gray}{×}} reiſt, ein Telegra{\geminationm} von ihr aus \textcolor{pink}{Partenk.}{}\ledrightnote{\textcolor{pink}{Partenkirchen}} ab, ehe auch ich hinführe. So wäre es alſo denkbar,
               daſs wir gegen Ende des Monats in \textcolor{pink}{Salzburg}{}\ledrightnote{\textcolor{pink}{Salzburg}} wären,
               wohin ich \textcolor{blue}{O.}{}\ledrightnote{\textcolor{blue}{Olga Schnitzler}} jedenfalls begleiten würde;
               vielleicht haben Sie auch noch einen \textcolor{pink}{Salzb.}{}\ledrightnote{\textcolor{pink}{Salzburg}} Abſtecher
               vor, und man könnte dort zuſa{\geminationm}en ſein? Nach \textcolor{pink}{Iſchl}{}\ledrightnote{\textcolor{pink}{Bad Ischl}} alſo ko{\geminationm}en wir in
               den nächſten Tagen kaum. Von allem weitern verſtändige ich Sie. Hören Sie was von \textcolor{blue}{\textsc{Arthur Kaufma{\geminationn}}}{}\ledrightnote{\textcolor{blue}{Arthur Kaufmann}}? Ko{\geminationm}t er nach \textcolor{pink}{Iſchl}{}\ledrightnote{\textcolor{pink}{Bad Ischl}}?\pend
           \pstart
           Herzlichſt Ihr{\\[\baselineskip]}\spacefill\mbox{Arthur}\pend
           \leftskip=0em{}\endnumbering\briefempfaengerindex{Beer-Hofmann, Richard@\textsc{Beer-Hofmann, Richard}!zzzSchnitzler, Arthur@\emph{von Arthur Schnitzler}!1916-08-231@{23. 8. 1916}|)be}\mylabel{h}  \normalsize

\doendnotes{C}
\bigskip
\vfill

\clearpage

\footnotesize

\lohead{\textsc{register}}

% Definiere theindex-Environment komplett neu ohne reledmac
\makeatletter
\renewenvironment{theindex}{%
  \section*{\indexname}%
  \setlength{\parindent}{0pt}%
  \setlength{\parskip}{0pt plus 0.3pt}%
  \let\item\@idxitem
}{%
  \clearpage
}
\makeatother

\IfFileExists{\jobname-pw.ind}{\input{\jobname-pw.ind}}{}

\end{document}

      