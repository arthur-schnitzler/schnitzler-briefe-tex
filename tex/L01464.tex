%% latex-korrekturansicht-vorspann.tex
%% Vorspann für die Korrekturansicht.
%% Lädt die gemeinsame Datei latex-vorspann.tex mit gesetztem Schalter.

\newif\ifkorrekturansicht
\korrekturansichttrue

\input{../tex-inputs/latex-vorspann}


               \section[Arthur Schnitzler an Richard Beer-Hofmann, 2. 11. 1904]{ Arthur Schnitzler an Richard Beer-Hofmann, 2. 11. 1904}\nopagebreak\mylabel{v}\rehead{ }\normalsize\beginnumbering\briefempfaengerindex{Beer-Hofmann, Richard@\textsc{Beer-Hofmann, Richard}!zzzSchnitzler, Arthur@\emph{von Arthur Schnitzler}!1904-11-021@{2. 11. 1904}|(be} \toendnotes[C]{\smallbreak\pagebreak[2]} \Standort{YCGL, MSS 31.}
\physDesc{Brief, 1 Blatt, 3 Seiten, Umschlag
\newline{}Handschrift: Bleistift, deutsche Kurrent\newline{}Versand: 1) Stempel: »\nobreak{}Wien, 2. XI. 04, 7\nobreak{}«.  2) Stempel: »\nobreak{}\oindex{Rodaun@\textbf{Rodaun}, \emph{Teil eines besiedelten Ortes (A.BSOX)}|pwk}Rodau\textcolor{gray}{n}, 3 \textcolor{gray}{11 04}\nobreak{}«. 
\newline{}Beer-Hofmann: mit Tinte den Zeitpunkt der Beantwortung notiert: »4/XI. b.« }\buchAbdrucke{\weitereDrucke{Arthur Schnitzler, Richard Beer-Hofmann: \emph{Briefwechsel 1891–1931}. Hg. Konstanze Fliedl. Wien, Zürich: \emph{Europaverlag} 1992, S. 167.} }\toendnotes[C]{\smallbreak}\pstart{}{\pb}\textcolor{pink}{Wien}{}\ledrightnote{\textcolor{pink}{Wien}}\pend{}\pstart{}\textsc{Arthur Schnitzler XIII \textcolor{pink}{Spoettelg}{}\ledrightnote{\textcolor{pink}{Edmund-Weiß-Gasse}}}\pend{}{\bigskip}\pstart{}{\pb}\textsc{Herrn Dr Rich. Beer-Hofmann}\pend{}\pstart{}\textsc{\textcolor{pink}{Rodaun}{}\ledrightnote{\textcolor{pink}{Rodaun}}}\pend{}\pstart{}L\pend{}\pstart{}\textcolor{pink}{\textsc{Liesingerstraße 2}}{}\ledrightnote{\textcolor{pink}{Liesingerstraße}}.\pend{}{\bigskip}\pstart
           \raggedleft{}{\pb}2. 11. 904\pend
           \pstart
           lieber Richard, ich bekomme heute beiliegendes \label{K_L01464_1v}\edtext{Telegra{\geminationm}}{\lemma{\textnormal{\emph{Telegra}}}\Cendnote{\textnormal{Im Telegramm vom
                     1. 11. 1904 schreibt \textcolor{blue}{Max
                     Reinhardt}, dass sich die Inszenierung von \emph{\textcolor{green}{Der
                     grüne Kakadu}}, \emph{\textcolor{green}{Der tapfere Cassian}} und
                     \emph{\textcolor{green}{Das Haus Delorme}} wegen Erkrankung von \textcolor{blue}{Agnes Sorma} weiter verzögere. (\emph{Der Briefwechsel Arthur Schnitzlers mit Max Reinhardt und
                        dessen Mitarbeitern}. Hg. Renate Wagner. Salzburg: \emph{Otto
                        Müller Verlag}{ }1971, S. 44.)}}}\label{K_L01464_1h}. Mir ſehr ärgerlich, weil auf mein
               Erſuchen im \textcolor{brown}{Volkstheater}{}\ledrightnote{\textcolor{brown}{Volkstheater}}{ }\label{K_L01464_2v}\edtext{\textcolor{green}{\textsc{Freiwild}{ }\textsc{Premiere}}{}\ledrightnote{\textcolor{green}{Freiwild. Schauspiel in 3 Akten}}}{\lemma{\textnormal{\emph{Freiwild Premiere}}}\Cendnote{\textnormal{Diese fand letztlich am
                     28. 1. 1905 statt.}}}\label{K_L01464_2h} wegen meiner \textcolor{pink}{Berlin}{}\ledrightnote{\textcolor{pink}{Berlin}}er \label{K_L01464_3v}\edtext{\textcolor{green}{\textsc{Premiere}}{}\ledrightnote{→\textcolor{green}{Der grüne Kakadu. Groteske in einem Akt}{\newline}→\textcolor{green}{Der tapfere Cassian. Puppenspiel in einem Akt}}}{\lemma{\textnormal{\emph{Premiere}}}\Cendnote{\textnormal{Die Uraufführung von \emph{\textcolor{green}{Der tapfere Cassian}} zusammen mit einer Neueinstudierung von
                     \emph{\textcolor{green}{Der grüne Kakadu}} ging letztlich am
                     22. 11. 1904 vonstatten.}}}\label{K_L01464_3h} hinausgeſchoben wurde u es jetzt
               erſt recht zu einer Colliſion kommen dürfte. Ich {\pb}nehme an, daſs nun der \textcolor{green}{Graf v \textsc{Charolais}}{}\ledrightnote{\textcolor{green}{Der Graf von Charolais. Ein Trauerspiel}} gleich (\label{K_L01464_4v}\edtext{vor \textcolor{blue}{\textcolor{green}{\textsc{Ruederer}}{}\ledrightnote{→\textcolor{green}{Die Morgenröthe. Komödie aus dem Jahre 1848}}}{}\ledrightnote{\textcolor{blue}{Josef Ruederer}}}{\lemma{\textnormal{\emph{vor Ruederer}}}\Cendnote{\textnormal{\emph{\textcolor{green}{Vor Morgenröthe}} erlebte am
                     15. 11. 1904 die Uraufführung.}}}\label{K_L01464_4h}) \label{K_L01464_5v}\edtext{drankommt}{\lemma{\textnormal{\emph{drankommt}}}\Cendnote{\textnormal{\emph{\textcolor{green}{Der Graf von Charolais}} hatte am
                     23. 12. 1904 Uraufführung.}}}\label{K_L01464_5h} (wobei ich allerdings noch immer
               nicht verſtehe, weshalb \textcolor{blue}{er}{}\ledrightnote{→\textcolor{blue}{Otto Brahm}}
               plötzlich meine Sachen nicht beſetzen kann) – jedenfalls bitte ich Sie mir ein Wort
               zu ſchrei{\pb}ben ſobald Sie aus \textcolor{pink}{Berlin}{}\ledrightnote{\textcolor{pink}{Berlin}} eine Nachricht haben u mir auch dieſes Telegr.
               zurückzuſchicken.\pend
           \pstart
           Herzlichſt Ihr{\\[\baselineskip]}\spacefill\mbox{A.}\pend
           \leftskip=0em{}\endnumbering\briefempfaengerindex{Beer-Hofmann, Richard@\textsc{Beer-Hofmann, Richard}!zzzSchnitzler, Arthur@\emph{von Arthur Schnitzler}!1904-11-021@{2. 11. 1904}|)be}\mylabel{h}  \normalsize

\doendnotes{C}
\bigskip
\vfill

\clearpage

\footnotesize

\lohead{\textsc{register}}

% Definiere theindex-Environment komplett neu ohne reledmac
\makeatletter
\renewenvironment{theindex}{%
  \section*{\indexname}%
  \setlength{\parindent}{0pt}%
  \setlength{\parskip}{0pt plus 0.3pt}%
  \let\item\@idxitem
}{%
  \clearpage
}
\makeatother

\IfFileExists{\jobname-pw.ind}{\input{\jobname-pw.ind}}{}

\end{document}

      