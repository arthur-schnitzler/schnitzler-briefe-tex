%% latex-korrekturansicht-vorspann.tex
%% Vorspann für die Korrekturansicht.
%% Lädt die gemeinsame Datei latex-vorspann.tex mit gesetztem Schalter.

\newif\ifkorrekturansicht
\korrekturansichttrue

\input{../tex-inputs/latex-vorspann}


               \section[Stefan Großmann an Arthur Schnitzler, 19. 2. 1912]{ Stefan Großmann an Arthur Schnitzler, 19. 2. 1912}\nopagebreak\mylabel{v}\rehead{ }\normalsize\beginnumbering\briefempfaengerindex{Schnitzler, Arthur@\textsc{Schnitzler, Arthur}!zzzGrossmann, Stefan@\emph{von Stefan Großmann}!1912-02-191@{19. 2. 1912}|(be} \toendnotes[C]{\smallbreak\pagebreak[2]} \Standort{CUL, Schnitzler, B 34.}
\physDesc{Brief, 1 Blatt, 1 Seite
\newline{}Handschrift: schwarze Tinte, lateinische Kurrent\newline{}Ordnung: mit Bleistift von unbekannter Hand nummeriert:
                                    »12« }\toendnotes[C]{\smallbreak}\pstart
           \noindent{}{\pb}\textcolor{gray}{\textbf{Stefan Großmann}}\hfill \textcolor{gray}{\textbf{\textcolor{pink}{WIEN VI, Ufergasse 18}{}\ledrightnote{\textcolor{pink}{Linke Wienzeile}}}}\pend
           \pstart
           \textcolor{gray}{\textbf{Korrespondent des »\textcolor{brown}{Berliner Tageblatt}{}\ledrightnote{\textcolor{brown}{Berliner Tageblatt}}«}}\hfill \textcolor{gray}{\textbf{Telephon 1326/II}}\pend
           \pstart
           \centering{}19. II. 1912\pend
           \pstart\center{}Verehrter Herr Doktor!\pend\pstart
           Sie haben ganz recht und ich werde bei der nächsten Gelegenheit mein vorschnelles
                    \textcolor{green}{Urteil}{}\ledrightnote{→\textcolor{green}{Schnitzlers »Weites Land«. Erste Aufführung im Burgtheater}}
                     über das \textcolor{green}{weite
                        Land}{}\ledrightnote{\textcolor{green}{Das weite Land. Tragikomödie in fünf Akten}} öffentlich rektifizieren.\pend
           \pstart
           Sehr ergeben:{\\[\baselineskip]}\spacefill\mbox{Stefan Großmann}\pend
           \leftskip=0em{}\endnumbering\briefempfaengerindex{Schnitzler, Arthur@\textsc{Schnitzler, Arthur}!zzzGrossmann, Stefan@\emph{von Stefan Großmann}!1912-02-191@{19. 2. 1912}|)be}\mylabel{h}  \normalsize

\doendnotes{C}
\bigskip
\vfill

\clearpage

\footnotesize

\lohead{\textsc{register}}

% Definiere theindex-Environment komplett neu ohne reledmac
\makeatletter
\renewenvironment{theindex}{%
  \section*{\indexname}%
  \setlength{\parindent}{0pt}%
  \setlength{\parskip}{0pt plus 0.3pt}%
  \let\item\@idxitem
}{%
  \clearpage
}
\makeatother

\IfFileExists{\jobname-pw.ind}{\input{\jobname-pw.ind}}{}

\end{document}

      