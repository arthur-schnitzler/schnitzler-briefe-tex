%% latex-korrekturansicht-vorspann.tex
%% Vorspann für die Korrekturansicht.
%% Lädt die gemeinsame Datei latex-vorspann.tex mit gesetztem Schalter.

\newif\ifkorrekturansicht
\korrekturansichttrue

\input{../tex-inputs/latex-vorspann}


               \section[Hugo von Hofmannsthal an Arthur Schnitzler, {[}20. 10. 1903?{]}]{ Hugo von Hofmannsthal an Arthur Schnitzler, {[}20. 10. 1903?{]}}\nopagebreak\mylabel{v}\rehead{ }\normalsize\beginnumbering\briefempfaengerindex{Schnitzler, Arthur@\textsc{Schnitzler, Arthur}!zzzHofmannsthal, Hugo von@\emph{von Hugo von Hofmannsthal}!1903-10-201@{{[}20. 10. 1903?{]}}|(be} \toendnotes[C]{\smallbreak\pagebreak[2]} \buchAlsQuelle{Hugo von Hofmannsthal, Arthur Schnitzler: \emph{Briefwechsel}. Hg. Therese Nickl und Heinrich Schnitzler. Frankfurt am Main: \emph{S. Fischer} 1964, S. 178.}\toendnotes[C]{\smallbreak}\pstart
           \noindent{}{\pb}Haben Sie \label{K_L01330_1v}\edtext{in \textcolor{pink}{Brünn}{}\ledrightnote{\textcolor{pink}{Brünn}} gelesen}{\lemma{\textnormal{\emph{in Brünn gelesen}}}\Cendnote{\textnormal{\textcolor{blue}{Schnitzler} las am 19. 10. 1903 für
                  die \emph{\textcolor{brown}{Neue akademische Vereinigung}} im kleinen
                  Festsaal des \textcolor{pink}{Deutschen Hauses}. Da sich \textcolor{blue}{Schnitzler} und \textcolor{blue}{Hofmannsthal} am 21. 10. 1903 bei \textcolor{blue}{Felix Salten}
                  getroffen haben, muss das Telegramm zwischen der Lesung und dieser Begegnung
                  abgeschickt worden sein. \textcolor{blue}{Hofmannsthal} sollte
                  am 22. 2. 1904 ebenfalls in \textcolor{pink}{Brünn}
                  lesen, die Veranstaltung musste jedoch wegen einer Erkrankung \textcolor{blue}{Hofmannsthal}s verschoben werden.}}}\label{K_L01330_1h}\hspace*{1em}Wieviel Minuten\hspace*{1em}Welches Honorar\hspace*{1em}Ist es angenehmes Lokal\pend
           \pstart \spacefill\mbox{Hugo}\pend{}\endnumbering\briefempfaengerindex{Schnitzler, Arthur@\textsc{Schnitzler, Arthur}!zzzHofmannsthal, Hugo von@\emph{von Hugo von Hofmannsthal}!1903-10-201@{{[}20. 10. 1903?{]}}|)be}\mylabel{h}  \normalsize

\doendnotes{C}
\bigskip
\vfill

\clearpage

\footnotesize

\lohead{\textsc{register}}

% Definiere theindex-Environment komplett neu ohne reledmac
\makeatletter
\renewenvironment{theindex}{%
  \section*{\indexname}%
  \setlength{\parindent}{0pt}%
  \setlength{\parskip}{0pt plus 0.3pt}%
  \let\item\@idxitem
}{%
  \clearpage
}
\makeatother

\IfFileExists{\jobname-pw.ind}{\input{\jobname-pw.ind}}{}

\end{document}

      