%% latex-korrekturansicht-vorspann.tex
%% Vorspann für die Korrekturansicht.
%% Lädt die gemeinsame Datei latex-vorspann.tex mit gesetztem Schalter.

\newif\ifkorrekturansicht
\korrekturansichttrue

\input{../tex-inputs/latex-vorspann}


               \section[Arthur Schnitzler an Richard Beer-Hofmann, 23. 7. 1910]{ Arthur Schnitzler an Richard Beer-Hofmann, 23. 7. 1910}\nopagebreak\mylabel{v}\rehead{ }\normalsize\beginnumbering\briefempfaengerindex{Beer-Hofmann, Richard@\textsc{Beer-Hofmann, Richard}!zzzSchnitzler, Arthur@\emph{von Arthur Schnitzler}!1910-07-231@{23. 7. 1910}|(be} \toendnotes[C]{\smallbreak\pagebreak[2]} \Standort{YCGL, MSS 31.}
\physDesc{Brief, 1 Blatt, 4 Seiten, Umschlag
\newline{}Adresse mit Schreibmaschine
\newline{}Handschrift: Bleistift, deutsche Kurrent\newline{}Versand: Stempel: »\nobreak{}\oindex{XVIII., Waehring@\textbf{XVIII., Währing}, \emph{Bezirk (A.BZK)}|pwk}18/\textcolor{gray}{3} Wien
                                          1\textcolor{gray}{14}, 23. VII. 10, 3\nobreak{}«.  \newline{}Ordnung: mit Bleistift von unbekannter Hand am Umschlag datiert: »23. 7.« }\buchAbdrucke{\weitereDrucke{Arthur Schnitzler, Richard Beer-Hofmann: \emph{Briefwechsel 1891–1931}. Hg. Konstanze Fliedl. Wien, Zürich: \emph{Europaverlag} 1992, S. 211–212.} }\toendnotes[C]{\smallbreak}\pstart{}{\pb}\textcolor{gray}{\textbf{Dr. Arthur Schnitzler}}\pend{}\pstart{}\textcolor{gray}{\textbf{\textcolor{pink}{Wien XVIII. Spoettelgasse 7}{}\ledrightnote{\textcolor{pink}{Edmund-Weiß-Gasse}}.}}\pend{}{\bigskip}\pstart{}{\pb}Herrn\pend{}\pstart{}Dr. Richard Beer-Hofmann\pend{}\pstart{}\textcolor{pink}{\so{Ischl}}{}\ledrightnote{\textcolor{pink}{Bad Ischl}}\pend{}\pstart{}\textcolor{pink}{Steinfeld 6}{}\ledrightnote{\textcolor{pink}{Steinfeld}}\pend{}{\bigskip}\pstart
           \noindent{}{\pb}\textcolor{gray}{\textbf{Dr. Arthur Schnitzler}}\hfill \textcolor{pink}{XVIII Sternwartestr 71}{}\ledrightnote{\textcolor{pink}{Sternwartestraße}}\pend
           \pstart
           \textcolor{gray}{\textbf{\textcolor{pink}{Wien XVIII. Spoettelgasse 7}{}\ledrightnote{\textcolor{pink}{Edmund-Weiß-Gasse}}.}}\pend
           \pstart{}mein lieber Richard,\pend\pstart
           hier ſende ich Ihnen Ihr \textcolor{green}{Gedicht}{}\ledrightnote{→\textcolor{green}{Schlaflied für Mirjam}}{ }ſammt Abſchrift, von der \textcolor{blue}{So{\geminationm}erremplacantin}{}\ledrightnote{→\textcolor{blue}{Grethe Hoffmann}}
               der braven \textcolor{blue}{Frieda}{}\ledrightnote{\textcolor{blue}{Frieda Pollak}}. –\pend
           \pstart
           Wir ſind leidlich in Ordnung und {\pb}freuen uns des neuen
               Heims. Ich fahre Dinſtag wieder auf ein paar Tage auf den \textcolor{pink}{Semmering}{}\ledrightnote{\textcolor{pink}{Semmering}}, zu \textcolor{blue}{Brahm}{}\ledrightnote{\textcolor{blue}{Otto Brahm}} u \textcolor{blue}{Kainz}{}\ledrightnote{\textcolor{blue}{Josef Kainz}}, der vom \textcolor{green}{Hofreiter}{}\ledrightnote{→\textcolor{green}{Das weite Land. Tragikomödie in fünf Akten}} ſehr angethan iſt und ihn {\pb}gleich ſpielen will.\pend
           \pstart
           Erſter Beſuch in dieſem Hauſe: Baron \textcolor{blue}{Berger}{}\ledrightnote{\textcolor{blue}{Alfred von Berger}}, aus
               ſolchem Grund. Aber die Sache iſt, aus mannigfachen Gründen noch nicht ganz ſicher. –
               Ins \textcolor{pink}{Salzka{\geminationm}er{\pb}gut}{}\ledrightnote{\textcolor{pink}{Salzkammergut}}, we{\geminationn} alles in
               Ordnung hoffen wir nach 20. Auguſt zu reiſen.\pend
           \pstart
           Ich hoffe es geht Ihnen allen ſo wie wirs wünſchen.\pend
           \pstart
           Von Herzen Ihr{\\[\baselineskip]}\spacefill\mbox{A.}\pend
           \leftskip=0em{}\endnumbering\briefempfaengerindex{Beer-Hofmann, Richard@\textsc{Beer-Hofmann, Richard}!zzzSchnitzler, Arthur@\emph{von Arthur Schnitzler}!1910-07-231@{23. 7. 1910}|)be}\mylabel{h}  \normalsize

\doendnotes{C}
\bigskip
\vfill

\clearpage

\footnotesize

\lohead{\textsc{register}}

% Definiere theindex-Environment komplett neu ohne reledmac
\makeatletter
\renewenvironment{theindex}{%
  \section*{\indexname}%
  \setlength{\parindent}{0pt}%
  \setlength{\parskip}{0pt plus 0.3pt}%
  \let\item\@idxitem
}{%
  \clearpage
}
\makeatother

\IfFileExists{\jobname-pw.ind}{\input{\jobname-pw.ind}}{}

\end{document}

      