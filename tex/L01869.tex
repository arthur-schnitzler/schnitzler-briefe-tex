%% latex-korrekturansicht-vorspann.tex
%% Vorspann für die Korrekturansicht.
%% Lädt die gemeinsame Datei latex-vorspann.tex mit gesetztem Schalter.

\newif\ifkorrekturansicht
\korrekturansichttrue

\input{../tex-inputs/latex-vorspann}


               \section[Arthur Schnitzler an Hermann Bahr, 28. 8. 1909]{ Arthur Schnitzler an Hermann Bahr, 28. 8. 1909}\nopagebreak\mylabel{v}\rehead{ }\normalsize\beginnumbering\briefempfaengerindex{Bahr, Hermann@\textsc{Bahr, Hermann}!zzzSchnitzler, Arthur@\emph{von Arthur Schnitzler}!1909-08-281@{28. 8. 1909}|(be} \toendnotes[C]{\smallbreak\pagebreak[2]} \Standort{TMW, HS AM 60170 Ba.}
\physDesc{Bildpostkarte
\newline{}Handschrift: Bleistift, deutsche Kurrent\newline{}Versand: Stempel: »\nobreak{}\oindex{Muenchen@\textbf{München}, \emph{https://www.geonames.org/ontologyP.PPLA}|pwk}München, 28 Aug 09, 3–4 N\nobreak{}«.  \newline{}Ordnung: Lochung \newline{}Zusatz: Postkartenmotiv von \textcolor{blue}{Heinrich
                           Kley.} }\buchAbdrucke{\weitereDrucke{1) \emph{28. 8. 1909, Abschrift.} In: Arthur Schnitzler: \emph{The Letters of Arthur Schnitzler to Hermann Bahr}. Edited, annotated, and with an introduction, by Donald G.
                        Daviau. Chapel Hill: \emph{The University of North Carolina Press} 1978, S. 104 (University of North Carolina studies in the Germanic languages
                        and literatures, 89).} \weitereDrucke{2) Hermann Bahr, Arthur Schnitzler: \emph{Briefwechsel, Aufzeichnungen, Dokumente (1891–1931)}. Hg. Kurt Ifkovits und Martin Anton Müller. Göttingen: \emph{Wallstein} 2018, S. 424.} }\toendnotes[C]{\smallbreak}\pstart{}{\pb}\textsc{Abs.
                     Schnitzler \textcolor{pink}{Wien}{}\ledrightnote{\textcolor{pink}{Wien}}}\pend{}\pstart{}\textsc{\textcolor{pink}{XVIII Spoettelg. 7.}{}\ledrightnote{\textcolor{pink}{Edmund-Weiß-Gasse}}}\pend{}{\bigskip}\pstart{}\textsc{Herrn Hermann Bahr}\pend{}\pstart{}aus \textcolor{pink}{Wien}{}\ledrightnote{\textcolor{pink}{Wien}} d. Z.\pend{}\pstart{}\textsc{\textcolor{pink}{Zell im Zillerthal}{}\ledrightnote{\textcolor{pink}{Zell am Ziller}}}\pend{}\pstart{}\textsc{\textcolor{pink}{Tirol}{}\ledrightnote{\textcolor{pink}{Tirol}}}\pend{}{\bigskip}\pstart
           \noindent{}\centering{}{\pb}\textcolor{gray}{\textbf{\textcolor{pink}{Alte Mariensäule}{}\ledrightnote{\textcolor{pink}{Alte Mariensäule}}}}\pend
           \pstart
           \textcolor{pink}{München}{}\ledrightnote{\textcolor{pink}{München}}{\\}28. 8. 09.\pend
           \pstart
           Herzlichen Gruſs, auch deiner verehrten \textcolor{blue}{Gattin}{}\ledrightnote{→\textcolor{blue}{Anna Bahr-Mildenburg}}.\pend
           \pstart I{\geminationm}er dein \spacefill\mbox{Arthur}\pend{}\endnumbering\briefempfaengerindex{Bahr, Hermann@\textsc{Bahr, Hermann}!zzzSchnitzler, Arthur@\emph{von Arthur Schnitzler}!1909-08-281@{28. 8. 1909}|)be}\mylabel{h}  \normalsize

\doendnotes{C}
\bigskip
\vfill

\clearpage

\footnotesize

\lohead{\textsc{register}}

% Definiere theindex-Environment komplett neu ohne reledmac
\makeatletter
\renewenvironment{theindex}{%
  \section*{\indexname}%
  \setlength{\parindent}{0pt}%
  \setlength{\parskip}{0pt plus 0.3pt}%
  \let\item\@idxitem
}{%
  \clearpage
}
\makeatother

\IfFileExists{\jobname-pw.ind}{\input{\jobname-pw.ind}}{}

\end{document}

      