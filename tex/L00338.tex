%% latex-korrekturansicht-vorspann.tex
%% Vorspann für die Korrekturansicht.
%% Lädt die gemeinsame Datei latex-vorspann.tex mit gesetztem Schalter.

\newif\ifkorrekturansicht
\korrekturansichttrue

\input{../tex-inputs/latex-vorspann}


               \section[Arthur Schnitzler an Hugo von Hofmannsthal, {[}15. 6. 1894?{]}]{ Arthur Schnitzler an Hugo von Hofmannsthal, {[}15. 6. 1894?{]}}\nopagebreak\mylabel{v}\rehead{ }\normalsize\beginnumbering\briefempfaengerindex{Hofmannsthal, Hugo von@\textsc{Hofmannsthal, Hugo von}!zzzSchnitzler, Arthur@\emph{von Arthur Schnitzler}!1894-06-151@{{[}15. 6. 1894?{]}}|(be} \toendnotes[C]{\smallbreak\pagebreak[2]} \Standort{FDH, Hs-30885,29.}
\physDesc{Briefkarte
\newline{}Handschrift: schwarze Tinte, deutsche Kurrent}\buchAbdrucke{\weitereDrucke{1) Hugo von Hofmannsthal, Arthur Schnitzler: \emph{Briefwechsel}. Hg. Therese Nickl und Heinrich Schnitzler. Frankfurt am Main: \emph{S. Fischer} 1964, S. 17.} \weitereDrucke{2) Hermann Bahr, Arthur Schnitzler: \emph{Briefwechsel, Aufzeichnungen, Dokumente
                                (1891–1931)}. Hg. Kurt Ifkovits und Martin Anton Müller. Göttingen: \emph{Wallstein} 2018.} }\toendnotes[C]{\smallbreak}\pstart
           \noindent{}{\pb}Lieber Hugo, faſt ſicher ſeh’ ich
                    morgen \textcolor{blue}{Salten}{}\ledrightnote{\textcolor{blue}{Felix Salten}}, faſt ſicher alſo wird er
                    Sonntag mit uns ſein. Nun war ich geſtern bei \textcolor{blue}{Bahr}{}\ledrightnote{\textcolor{blue}{Hermann Bahr}}, der auch was von So{\geminationn}tag
                    redete, und ich überlaſſe Ihnen die Sache einzurichten wie’s Ihnen lieb iſt.
                        Jeden{\pb}falls erwarte ich Sie So{\geminationn}tag ½ 4.\pend
           \pstart
           Mit vielen herzlichen Grüßen.{\\[\baselineskip]}Ihr{\\[\baselineskip]}\spacefill\mbox{Arthur.}\pend
           \leftskip=0em{}\pstart
           \noindent{}Eventuell ſchreiben Sie mir noch eine Zeile.\pend
           \pstart
           \label{T_L00338_1v}\edtext{\uline{Freitag.}}{\lemma{\textnormal{\emph{Freitag.}}}\Cendnote{\textnormal{undatiert. Ein Treffen mit
                                    \textcolor{blue}{Bahr} am Donnerstag und
                                    \textcolor{blue}{Salten} am Samstag lässt sich
                                mit \textcolor{blue}{Schnitzler}s \emph{\textcolor{green}{Tagebuch}} zu keinem anderen Zeitpunkt nachweisen, zudem deckt sich die Uhrzeit.}}}\label{T_L00338_1h}\pend
           \endnumbering\briefempfaengerindex{Hofmannsthal, Hugo von@\textsc{Hofmannsthal, Hugo von}!zzzSchnitzler, Arthur@\emph{von Arthur Schnitzler}!1894-06-151@{{[}15. 6. 1894?{]}}|)be}\mylabel{h}  \normalsize

\doendnotes{C}
\bigskip
\vfill

\clearpage

\footnotesize

\lohead{\textsc{register}}

% Definiere theindex-Environment komplett neu ohne reledmac
\makeatletter
\renewenvironment{theindex}{%
  \section*{\indexname}%
  \setlength{\parindent}{0pt}%
  \setlength{\parskip}{0pt plus 0.3pt}%
  \let\item\@idxitem
}{%
  \clearpage
}
\makeatother

\IfFileExists{\jobname-pw.ind}{\input{\jobname-pw.ind}}{}

\end{document}

      