%% latex-korrekturansicht-vorspann.tex
%% Vorspann für die Korrekturansicht.
%% Lädt die gemeinsame Datei latex-vorspann.tex mit gesetztem Schalter.

\newif\ifkorrekturansicht
\korrekturansichttrue

\input{../tex-inputs/latex-vorspann}


               \section[Karl Kraus an Arthur Schnitzler, 26. 1. 1893]{ Karl Kraus an Arthur Schnitzler, 26. 1. 1893}\nopagebreak\mylabel{v}\rehead{ }\normalsize\beginnumbering\briefempfaengerindex{Schnitzler, Arthur@\textsc{Schnitzler, Arthur}!zzzKraus, Karl@\emph{von Karl Kraus}!1893-01-261@{26. 1. 1893}|(be} \toendnotes[C]{\smallbreak\pagebreak[2]} \Standort{CUL, Schnitzler, B 55.}
\physDesc{Kartenbrief
\newline{}Handschrift: schwarze Tinte, deutsche Kurrent\newline{}Versand: 1) Stempel: »\nobreak{}Wien 1/1, 26. 1 {[}93{]}, 11–12V\nobreak{}«.  2) Stempel: »\nobreak{}Wien 1/1, 26/1. 93, 1–2½ N\nobreak{}«. 
\newline{}Schnitzler: mit Bleistift auf der Textseite beschriftet: »\textcolor{pink}{Wienerh}{ }\textcolor{pink}{Mauerst 20}« }\buchAbdrucke{\weitereDrucke{\emph{Karl Kraus und Arthur Schnitzler. Eine Dokumentation.} Hg. Reinhard Urbach. In: \emph{Literatur und Kritik}, Bd. 49, Oktober 1970, S. 515.} }\toendnotes[C]{\smallbreak}\pstart{}{\pb}Herrn Schriftſteller\pend{}\pstart{}D\textsuperscript{r} med Arthur Schnitzler\pend{}\pstart{}\textcolor{pink}{Grillparzerstr. 7}{}\ledrightnote{\textcolor{pink}{Grillparzerstraße}}\pend{}\pstart{}\textcolor{pink}{Wien I.}{}\ledrightnote{\textcolor{pink}{I., Innere Stadt}}\pend{}{\bigskip}\pstart{}{\pb}Lieber Doctor
                        Schnitzler!\pend\pstart
           \textcolor{blue}{Otto Julius Bierbaum}{}\ledrightnote{\textcolor{blue}{Otto Julius Bierbaum}} fordert Sie durch mich
                    auf, ihm was für ſeinen \textcolor{green}{Mod. Muſen-Almanach
                        1894}{}\ledrightnote{\textcolor{green}{Moderner Musen-Almanach auf das Jahr 1894}} zukommen zu laſſen. Der Almanach erſcheint
                        1. Septemb. 93. Endtermin für die \textcolor{green}{Einſendung}{}\ledrightnote{→\textcolor{green}{Die drei Elixire}}{ }\uline{1. Juli}. Adreſſe: \textcolor{blue}{O. J. Bierbaum}{}\ledrightnote{\textcolor{blue}{Otto Julius Bierbaum}}, \textcolor{pink}{\uline{Oberbayern}}{}\ledrightnote{\textcolor{pink}{Oberbayern}}: \uline{Post \textcolor{pink}{Beuerberg}{}\ledrightnote{\textcolor{pink}{Beuerberg}}}; \textcolor{pink}{\uline{Auf der Öd}}{}\ledrightnote{\textcolor{pink}{Auf der Öd}}.\pend
           \pstart
           Über Ihren \textcolor{green}{Anatol}{}\ledrightnote{\textcolor{green}{Anatol}}{ }ſchreibe ich einige \label{K_L00164_1v}\edtext{Zeilen}{\lemma{\textnormal{\emph{Zeilen}}}\Cendnote{\textnormal{nicht erschienen}}}\label{K_L00164_1h} für\strikeout{’s}{ }\textcolor{brown}{N. l. Bl.}{}\ledrightnote{\textcolor{brown}{Neue litterarische Blätter}} (\textcolor{pink}{Bremen}{}\ledrightnote{\textcolor{pink}{Bremen}}) 1. März, welche N\textsuperscript{r.} in
                    4–5000 Ex. erſcheinen wird. Demnächſt erhalten Sie von mir Druckſorte:
                    Aufforderung zur Satirenanthologie.\pend
           \pstart Gruß u. Handſchlag. Ihr \spacefill\mbox{Karl Kraus.}\pend{}\endnumbering\briefempfaengerindex{Schnitzler, Arthur@\textsc{Schnitzler, Arthur}!zzzKraus, Karl@\emph{von Karl Kraus}!1893-01-261@{26. 1. 1893}|)be}\mylabel{h}  \normalsize

\doendnotes{C}
\bigskip
\vfill

\clearpage

\footnotesize

\lohead{\textsc{register}}

% Definiere theindex-Environment komplett neu ohne reledmac
\makeatletter
\renewenvironment{theindex}{%
  \section*{\indexname}%
  \setlength{\parindent}{0pt}%
  \setlength{\parskip}{0pt plus 0.3pt}%
  \let\item\@idxitem
}{%
  \clearpage
}
\makeatother

\IfFileExists{\jobname-pw.ind}{\input{\jobname-pw.ind}}{}

\end{document}

      