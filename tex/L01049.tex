%% latex-korrekturansicht-vorspann.tex
%% Vorspann für die Korrekturansicht.
%% Lädt die gemeinsame Datei latex-vorspann.tex mit gesetztem Schalter.

\newif\ifkorrekturansicht
\korrekturansichttrue

\input{../tex-inputs/latex-vorspann}


               \section[Julius Rodenberg an Arthur Schnitzler, 23. 6. 1900]{ Julius Rodenberg an Arthur Schnitzler, 23. 6. 1900}\nopagebreak\mylabel{v}\rehead{ }\normalsize\beginnumbering\briefempfaengerindex{Schnitzler, Arthur@\textsc{Schnitzler, Arthur}!zzzRodenberg, Julius@\emph{von Julius Rodenberg}!1900-06-231@{23. 6. 1900}|(be} \toendnotes[C]{\smallbreak\pagebreak[2]} \Standort{TMW, HS Schn 4/27/1.}
\physDesc{Brief, 1 Blatt, 2 Seiten
\newline{}Handschrift: schwarze Tinte, deutsche Kurrent
\newline{}Schnitzler: mit rotem Buntstift eine Unterstreichung }\toendnotes[C]{\smallbreak}\pstart
           \noindent{}\centering{}{\pb}\textcolor{gray}{\textbf{\textcolor{brown}{Deutsche Rundschau}{}\ledrightnote{\textcolor{brown}{Deutsche Rundschau}}}}\pend
           \pstart
           \noindent{}\textcolor{gray}{\textbf{Expedition u. Redaction:}}\hfill \textcolor{gray}{\textbf{Herausgeber:}}\pend
           \pstart
           \textcolor{gray}{\textbf{\textcolor{brown}{Gebrüder Paetel}{}\ledrightnote{\textcolor{brown}{Gebrüder Paetel Verlag}} in \textcolor{pink}{Berlin}{}\ledrightnote{\textcolor{pink}{Berlin}}}}\hfill \textcolor{gray}{\textbf{Julius Rodenberg in \textcolor{pink}{Berlin}{}\ledrightnote{\textcolor{pink}{Berlin}}}}\pend
           \pstart
           \textcolor{gray}{\textbf{\textcolor{pink}{W., Lützowstr. 7}{}\ledrightnote{\textcolor{pink}{Lützowstraße}}.}}\hfill \textcolor{gray}{\textbf{\textcolor{pink}{W., Margarethenstr. 1}{}\ledrightnote{\textcolor{pink}{Margaretenstraße}}.}}\pend
           \pstart
           \raggedleft{}\textbf{\textcolor{gray}{\textbf{\textcolor{pink}{Berlin W.}{}\ledrightnote{\textcolor{pink}{Berlin}},}} den}{ }23. Juni 1900.\pend
           \pstart{}Hochgeehrter Herr Doctor!\pend\pstart
           Empfangen Sie meinen verbindlichſten Dank für Ihr freundliches Schreiben vom
                        21. d M. u. das darin enthaltene Anerbieten. Ich brauche Ihnen
                    nicht zu ſagen, welchen Werth es für mich hat, Sie wißen es, wie ſehr ich mich
                    freuen würde, endlich einmal eine \textcolor{green}{Novelle}{}\ledrightnote{→\textcolor{green}{Frau Bertha Garlan. Roman}} von Ihnen bringen zu können u. wie froh ich jede Hoffnung dazu
                    begrüßt habe. Zu meinem größten Bedauern aber, indem Sie jetzt eben wieder mir
                    eine ſolche Hoffnung machen, deuten Sie ſelber an, daß Sie auch diesmal an ihrer
                    Erfüllung zweifeln. Sie kennen ja die Haltung der »\textcolor{brown}{\textsc{Rundschau}}{}\ledrightnote{\textcolor{brown}{Deutsche Rundschau}}« u. wenn Sie das von Ihnen behandelte Sujet für »bedenklich« halten, ſo
                    kann ich kaum glauben, daß ich darin anderer Meinung ſein werde als Sie, u. wage
                    deshalb {\pb}gar
                    nicht, Sie um Einſendung Ihrer \textcolor{green}{Arbeit}{}\ledrightnote{→\textcolor{green}{Frau Bertha Garlan. Roman}} zu bitten. Denn eine Ablehnung würde peinlich für mich ſein u.
                    einen Zeitverluſt für Sie bedeuten. Alſo, ſehr geehrter Herr Doctor, bewahren
                    Sie mir Ihren freundlichen guten Willen, u. ſobald Sie eine Novelle ſchreiben,
                    die nach Ihrem eigenen Dafürhalten mehr in den Rahmen der »\textcolor{brown}{\textsc{Rundschau}}{}\ledrightnote{\textcolor{brown}{Deutsche Rundschau}}« paßt, ſenden Sie ſie und ſeien Sie überzeugt, daß ſie uns herzlich
                        willko{\geminationm}en ſein wird.\pend
           \pstart
           In aufrichtiger Hochachtung{\\[\baselineskip]}ergebenſt{\\[\baselineskip]}Ihr{\\[\baselineskip]}\spacefill\mbox{Dr Julius Rodenberg.}\pend
           \leftskip=0em{}\endnumbering\briefempfaengerindex{Schnitzler, Arthur@\textsc{Schnitzler, Arthur}!zzzRodenberg, Julius@\emph{von Julius Rodenberg}!1900-06-231@{23. 6. 1900}|)be}\mylabel{h}  \normalsize

\doendnotes{C}
\bigskip
\vfill

\clearpage

\footnotesize

\lohead{\textsc{register}}

% Definiere theindex-Environment komplett neu ohne reledmac
\makeatletter
\renewenvironment{theindex}{%
  \section*{\indexname}%
  \setlength{\parindent}{0pt}%
  \setlength{\parskip}{0pt plus 0.3pt}%
  \let\item\@idxitem
}{%
  \clearpage
}
\makeatother

\IfFileExists{\jobname-pw.ind}{\input{\jobname-pw.ind}}{}

\end{document}

      