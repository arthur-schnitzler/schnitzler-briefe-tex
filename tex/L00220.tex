%% latex-korrekturansicht-vorspann.tex
%% Vorspann für die Korrekturansicht.
%% Lädt die gemeinsame Datei latex-vorspann.tex mit gesetztem Schalter.

\newif\ifkorrekturansicht
\korrekturansichttrue

\input{../tex-inputs/latex-vorspann}


               \section[Arthur Schnitzler an Wilhelm Bölsche, 11. 6. 1893]{ Arthur Schnitzler an Wilhelm Bölsche, 11. 6. 1893}\nopagebreak\mylabel{v}\rehead{ }\normalsize\beginnumbering\briefempfaengerindex{Boelsche, Wilhelm@\textsc{Bölsche, Wilhelm}!zzzSchnitzler, Arthur@\emph{von Arthur Schnitzler}!1893-06-111@{11. 6. 1893}|(be} \toendnotes[C]{\smallbreak\pagebreak[2]} \Standort{Wrocław, Biblioteka Uniwersytecka, Böl.Pis 1768.}
\physDesc{Brief, 1 Blatt (Briefpapier mit Trauerrand), 4 Seiten
\newline{}Handschrift: schwarze Tinte, deutsche Kurrent
\newline{}Bölsche: als »Erl{[}edigt{]}« gezeichnet }\buchAbdrucke{\weitereDrucke{1) Alois Woldan: \emph{Arthur Schnitzler – Briefe an Wilhelm Bölsche.} In: \emph{Germanica Wratislaviensia} (1987) Nr. 77, S. 462–463.} \weitereDrucke{2) Wilhelm Bölsche: \emph{Briefwechsel. Mit Autoren der Freien Bühne}. Hg. Gerd-Hermann Susen. Berlin: \emph{Weidler} 2010, S. 686–687 (Werke und Briefe. Wissenschaftliche Ausgabe, Briefe I).} }\toendnotes[C]{\smallbreak}\pstart
           {\pb}\textsc{\textcolor{pink}{Wien}{}\ledrightnote{\textcolor{pink}{Wien}}}{ }11. 6. 93.\hfill \textcolor{pink}{\textsc{I. Grillparzerstr 7}}{}\ledrightnote{\textcolor{pink}{Grillparzerstraße}}.\pend
           \pstart{}Sehr geehrter Herr Doktor!\pend\pstart
           Vor mehr als 2 Monaten hab ich Ihnen eine \textcolor{green}{Skizze}{}\ledrightnote{→\textcolor{green}{Die Braut}} zur eventuellen Veröffentlichung eingeſandt
                        »\textcolor{green}{\uline{Die Braut}}{}\ledrightnote{\textcolor{green}{Die Braut}}«. – Vor \uline{ca} 2 Wochen hab ich die Frage an
                    Sie gerichtet, ob Sie geneigt wären, mein 3 aktiges für die nächſte Saiſon am
                        \textcolor{brown}{Leſſingtheater}{}\ledrightnote{\textcolor{brown}{Lessing-Theater}} zur Aufführung beſti{\geminationm}tes Schauſpiel »\textcolor{green}{\uline{Das Märchen}}{}\ledrightnote{\textcolor{green}{Das Märchen. Schauspiel in drei Aufzügen}}« {\pb}in der \textcolor{green}{\textsc{Freien Bühne}}{}\ledrightnote{\textcolor{green}{Freie Bühne für den Entwickelungskampf der Zeit}} zu veröffentlichen. Warum, erlaube ich mir zu fragen, laſſen Sie mich denn
                    ſo lange auf Antwort warten? Meine \textcolor{green}{Skizze}{}\ledrightnote{→\textcolor{green}{Die Braut}} iſt in einer viertel Stunde geleſen, und was
                    nun gar mein \textcolor{green}{Stück}{}\ledrightnote{→\textcolor{green}{Das Märchen. Schauspiel in drei Aufzügen}} anlangt,
                    ſo bedarf es ja vorläufig nur eines principiellen Ja oder Nein. Sie,
                    verehrteſter Herr Doktor, {\pb}der Sie ſelbſt
                    Schriftſteller ſind, Sie wiſſen ja, wie nervös das Warten macht; und ich, der
                    ſelbſt Redakteur einer (mediz.) \textcolor{brown}{Zeitſchrift}{}\ledrightnote{\textcolor{brown}{Internationale klinische Rundschau}}
                    bin, beantworte jeden Einlauf in ſpäteſtens 8 Tagen. Es mag ja Leute geben,
                    deren Briefe man unberückſichtigt zur Seite werfen kann; ich gehöre {\pb}nicht zu dieſen, wovon Sie verehrteſter Herr Doktor,
                    gewiß ſelbſt überzeugt ſind. –\pend
           \pstart
           – Ich wiederhole alſo meine beiden Fragen: Nehmen Sie die »\textcolor{green}{Die \uline{Braut}}{}\ledrightnote{\textcolor{green}{Die Braut}}«? – Und zweitens, wollen Sie das \textcolor{green}{Das
                        Märchen}{}\ledrightnote{\textcolor{green}{Das Märchen. Schauspiel in drei Aufzügen}} im Laufe dieſes So{\geminationm}ers drucken? –\pend
           \pstart
           Ich bin mit ausgezeichneter Hochachtung{\\[\baselineskip]}Ihr ſehr ergebner{\\[\baselineskip]}\spacefill\mbox{Dr. Arthur Schnitzler}\pend
           \leftskip=0em{}\endnumbering\briefempfaengerindex{Boelsche, Wilhelm@\textsc{Bölsche, Wilhelm}!zzzSchnitzler, Arthur@\emph{von Arthur Schnitzler}!1893-06-111@{11. 6. 1893}|)be}\mylabel{h}  \normalsize

\doendnotes{C}
\bigskip
\vfill

\clearpage

\footnotesize

\lohead{\textsc{register}}

% Definiere theindex-Environment komplett neu ohne reledmac
\makeatletter
\renewenvironment{theindex}{%
  \section*{\indexname}%
  \setlength{\parindent}{0pt}%
  \setlength{\parskip}{0pt plus 0.3pt}%
  \let\item\@idxitem
}{%
  \clearpage
}
\makeatother

\IfFileExists{\jobname-pw.ind}{\input{\jobname-pw.ind}}{}

\end{document}

      