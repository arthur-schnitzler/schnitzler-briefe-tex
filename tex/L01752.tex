%% latex-korrekturansicht-vorspann.tex
%% Vorspann für die Korrekturansicht.
%% Lädt die gemeinsame Datei latex-vorspann.tex mit gesetztem Schalter.

\newif\ifkorrekturansicht
\korrekturansichttrue

\input{../tex-inputs/latex-vorspann}


               \section[Richard Beer-Hofmann an Arthur Schnitzler, {[}17. 1. 1908{]}]{ Richard Beer-Hofmann an Arthur Schnitzler, {[}17. 1. 1908{]}}\nopagebreak\mylabel{v}\rehead{ }\normalsize\beginnumbering\briefempfaengerindex{Schnitzler, Arthur@\textsc{Schnitzler, Arthur}!zzzBeer-Hofmann, Richard@\emph{von Richard Beer-Hofmann}!1908-01-171@{{[}17. 1. 1908{]}}|(be} \toendnotes[C]{\smallbreak\pagebreak[2]} \Standort{CUL, Schnitzler, B 8.}
\physDesc{Brief, 1 Blatt (Briefpapier mit Trauerrand), 3 Seiten
\newline{}Handschrift: blauer Buntstift, lateinische Kurrent
\newline{}Schnitzler: mit Bleistift datiert: »17/1 908« \newline{}Ordnung: mit Bleistift von unbekannter Hand nummeriert:
                              »218« }\buchAbdrucke{\weitereDrucke{Arthur Schnitzler, Richard Beer-Hofmann: \emph{Briefwechsel 1891–1931}. Hg. Konstanze Fliedl. Wien, Zürich: \emph{Europaverlag} 1992, S. 188.} }\toendnotes[C]{\smallbreak}\pstart
           \raggedleft{}{\pb}Freitag\pend
           \pstart
           Lieber Arthur! Ich \textcolor{brown}{freue mich}{}\ledrightnote{→\textcolor{brown}{Franz-Grillparzer-Preis}} von ganzem Herzen – besonders nach dem letzten Gespräch das wir
               hierüber hatten. Ob \label{K_L01752_1v}\edtext{\textcolor{blue}{Minor}{}\ledrightnote{\textcolor{blue}{Jakob Minor}}s Motivirung}{\lemma{\textnormal{\emph{Minors Motivirung}}}\Cendnote{\textnormal{Der Einigung auf \textcolor{blue}{Schnitzler}
                  lag ein Kompromiss innerhalb der Jury zugrunde. \textcolor{blue}{Jakob Minor}, der Vorsitzende der Kommission, begründete die Entscheidung
                  so: »Für das Votum des Preisrichterkollegiums kam, wie Hofrat \textcolor{blue}{Minor} in seinem Referat ausführte, in erster
                     Linie das \textcolor{green}{Stück}, das den
                     Preis erhielt, in Betracht und erst in zweiter Linie der Dichter.«
                     ([O. V.;] \emph{\textcolor{green}{Die Verleihung des
                        Grillparzer-Preises}}. In: \emph{\textcolor{green}{Neue Freie
                        Presse}}, Nr. 15590, 16. 1. 1908, Morgenblatt,
                  S. 8).}}}\label{K_L01752_1h} eine Perfidie, oder ein ungeschicktes Compliment war wird
               sich kaum feststellen lassen.\pend
           \pstart
           Auch die \textcolor{brown}{N. Fr. Pr.}{}\ledrightnote{\textcolor{brown}{Neue Freie Presse}} war wieder einmal recht
               herzig.\pend
           \pstart
           {\pb}Bitte lassen Sie doch von sich
               hören – hören, wörtlich geno{\geminationm}en – ich kann nichts
               dazutun. \textcolor{blue}{Naëmah}{}\ledrightnote{\textcolor{blue}{Naëmah Beer-Hofmann}}, \textcolor{blue}{Bubi}{}\ledrightnote{\textcolor{blue}{Gabriel Beer-Hofmann}} haben {\pb}Influenza
               gehabt, sind noch zu Bett, wir Andern noch nicht. Herzlichst\pend
           \pstart
           Ihr{\\[\baselineskip]}\spacefill\mbox{Richard}\pend
           \leftskip=0em{}\endnumbering\briefempfaengerindex{Schnitzler, Arthur@\textsc{Schnitzler, Arthur}!zzzBeer-Hofmann, Richard@\emph{von Richard Beer-Hofmann}!1908-01-171@{{[}17. 1. 1908{]}}|)be}\mylabel{h}  \normalsize

\doendnotes{C}
\bigskip
\vfill

\clearpage

\footnotesize

\lohead{\textsc{register}}

% Definiere theindex-Environment komplett neu ohne reledmac
\makeatletter
\renewenvironment{theindex}{%
  \section*{\indexname}%
  \setlength{\parindent}{0pt}%
  \setlength{\parskip}{0pt plus 0.3pt}%
  \let\item\@idxitem
}{%
  \clearpage
}
\makeatother

\IfFileExists{\jobname-pw.ind}{\input{\jobname-pw.ind}}{}

\end{document}

      