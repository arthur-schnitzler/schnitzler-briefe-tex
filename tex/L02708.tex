%% latex-korrekturansicht-vorspann.tex
%% Vorspann für die Korrekturansicht.
%% Lädt die gemeinsame Datei latex-vorspann.tex mit gesetztem Schalter.

\newif\ifkorrekturansicht
\korrekturansichttrue

\input{../tex-inputs/latex-vorspann}


               \section[Paul Goldmann an Arthur Schnitzler, 22. 5. {[}1893{]}]{ Paul Goldmann an Arthur Schnitzler, 22. 5. {[}1893{]}}\nopagebreak\mylabel{v}\rehead{ }\normalsize\beginnumbering\briefempfaengerindex{Schnitzler, Arthur@\textsc{Schnitzler, Arthur}!zzzGoldmann, Paul@\emph{von Paul Goldmann}!1893-05-221@{22. 5. {[}1893{]}}|(be} \toendnotes[C]{\smallbreak\pagebreak[2]} \Standort{DLA, A:Schnitzler, HS.NZ85.1.3163.}
\physDesc{Brief, 1 Blatt, 4 Seiten
\newline{}Handschrift: schwarze Tinte, deutsche Kurrent
\newline{}Schnitzler: 1) mit Bleistift das Jahr »93« vermerkt 2) mit rotem Buntstift drei Unterstreichungen}\toendnotes[C]{\smallbreak}\pstart
           \noindent{}{\pb}\textcolor{gray}{\textbf{\textbf{\textcolor{brown}{Frankfurter Zeitung}{}\ledrightnote{\textcolor{brown}{Frankfurter Zeitung}}.}}}\pend
           \pstart
           \textcolor{gray}{\textbf{\textbf{(\textcolor{brown}{\begin{otherlanguage}{french}Gazette de Francfort\end{otherlanguage}}{}\ledrightnote{\textcolor{brown}{Frankfurter Zeitung}}.)}}}\pend
           \pstart
           \textcolor{gray}{\textbf{\begin{otherlanguage}{french}\textcolor{blue}{Directeur}{}\ledrightnote{→\textcolor{blue}{Leopold Sonnemann}}\end{otherlanguage}{ }\textbf{M. \textcolor{blue}{L. Sonnemann}{}\ledrightnote{\textcolor{blue}{Leopold Sonnemann}}.}}}\hfill \textsc{\textcolor{pink}{Paris}{}\ledrightnote{\textcolor{pink}{Paris}}}, 22. Mai.\pend
           \pstart
           \begin{otherlanguage}{french}\textcolor{gray}{\textbf{\textcolor{green}{Journal}{}\ledrightnote{\textcolor{green}{Frankfurter Zeitung}} politique, financier,}}\end{otherlanguage}\pend
           \pstart
           \begin{otherlanguage}{french}\textcolor{gray}{\textbf{commercial et litteraire.}}\end{otherlanguage}\pend
           \pstart
           \begin{otherlanguage}{french}\textcolor{gray}{\textbf{\textbf{Paraissant trois fois par jour}}}\end{otherlanguage}\pend
           \pstart
           \begin{otherlanguage}{french}\textcolor{gray}{\textbf{\textbf{Bureaux à \textcolor{pink}{Paris}{}\ledrightnote{\textcolor{pink}{Paris}}:}}}\end{otherlanguage}\pend
           \pstart
           \begin{otherlanguage}{french}\textcolor{gray}{\textbf{\textbf{\textcolor{pink}{rue Richelieu 75}{}\ledrightnote{\textcolor{pink}{rue Richelieu}}.}}}\end{otherlanguage}\pend
           \pstart
           Mein lieber Arthur!\pend
           \pstart
           Dein lieber Brief, für den ich Dir herzlichſt danke, hat mich im Weſentlichen
               beruhigt. Die Hauptſache iſt, daß Dir die niedrigen \label{K_L02708-1v}\edtext{Brodſorgen fern bleiben}{\lemma{\textnormal{\emph{Brodſorgen fern bleiben}}}\Cendnote{\textnormal{\textcolor{blue}{Schnitzler}s Anteil am Erbe seines \textcolor{blue}{Vater}s ermöglichte ihm
                  einige Zeit finanzielle Sicherheit.}}}\label{K_L02708-1h}. Alles übrige Weh’, das ich tief
               beklage, ſoweit es Dich als Menſchen betrifft, wird Dir vielleicht doch zum \label{T_L02708-1v}\edtext{Z{[}ie{]}le}{\lemma{\textnormal{\emph{Ziele}}}\Cendnote{\textnormal{\textcolor{blue}{Goldmann} schreibt
                  »Zeile«}}}\label{T_L02708-1h} ſein. Und mit jenem künſtleriſchen Egoismus, der
               Alles unter dem Geſichtspunkte ſeiner eigenſten Zwecke ſieht, denke ich mir, daß ein
               wenig Härtung und Hämmerung von Seiten des Lebens Deiner ſchönen Begabung gar
               herrlich zuſtatten kommen wird. Auch \textsc{\textcolor{blue}{Herzl}{}\ledrightnote{\textcolor{blue}{Theodor Herzl}}{ }}{\pb}iſt dierſer Anficht, der Dich jetzt \label{K_L02708-2v}\edtext{zu lieben und zu verſtehen begonnen}{\lemma{\textnormal{\emph{zu … begonnen}}}\Cendnote{\textnormal{\textcolor{blue}{Schnitzler} und \textcolor{blue}{Theodor Herzl} korrespondierten korrespondierten zwischen
                     Mai und September{ }1893 auch häufig miteinander. Siehe \textcolor{blue}{Theodor Herzl}. Briefe und Tagebücher.
                     Hg. v. Alex Bein, Hermann Greive, Moshe Schaerf und Julius H. Schoeps. Bd. 1.:
                        \textcolor{blue}{Theodor Herzl}. Briefe und
                     autobiographische Notizen. 1866–1895. Bearb. v. Johannes Wachten. In Zusammenarbeit m. Chaya Harel,
                     Daisy Tycho und Manfred Winkler. \textcolor{pink}{Berlin}/\textcolor{pink}{Frankfurt a. M.}/\textcolor{pink}{Wien}: \emph{\textcolor{brown}{Ullstein}}/\emph{\textcolor{brown}{Propyläen}} 1983,
                     S. 526–541.}}}\label{K_L02708-2h} hat und mit dem ich oft über Dich ſpreche. Hier und da erfahre ich auf dieſem
               Wege etwas über Dein Ergehen, wenn er einen Brief von Dir bekommen hat. Und dann
               denke ich mir: »Der hat aber ein Glück.« Auch \textsc{\textcolor{blue}{Isidor Fuchs}{}\ledrightnote{\textcolor{blue}{Isidor Fuchs}}} hat mir viel über \textcolor{pink}{Wien}{}\ledrightnote{\textcolor{pink}{Wien}} erzählt. Und ſo \strikeout{h\textcolor{gray}{a}} bin ich denn durch fleißiges \strikeout{Euch} Betreiben
               dieſes Nachrichtendienſtes ein wenig auf dem Laufenden der Veränderungen, die ſich in
               den äußeren \textcolor{pink}{Wien}{}\ledrightnote{\textcolor{pink}{Wien}}er Dingen vollzogen, und weiß vor
               allen Dingen von Deinen Erfolgen, die mich mit wahrer Freude {\pb}erfüllt. Immerhin gibt es in meinem Wiſſen gewaltige
               Lücken. Und wenn Du mir nur ein wenig Näheres über die inneren Dinge ſchreiben
               könnteſt – über die Natur der \label{K_L02708-3v}\edtext{Unfälle}{\lemma{\textnormal{\emph{Unfälle}}}\Cendnote{\textnormal{Nicht nur mit dem Tod des
                     \textcolor{blue}{Vater}s am 2. 5. 1893 hatte \textcolor{blue}{Schnitzler} seit Anfang des Jahres 1893 zu kämpfen, auch sein Liebesleben gestaltete sich
                  unverhofft schwierig, erhielt er doch am 28. 1. 1893 erste Hinweise auf \textcolor{blue}{Marie Glümer}s Untreue.}}}\label{K_L02708-3h}, die Dich
               betroffen, über Stimmungen und Pläne – ein wenig, ein ganz klein wenig, damit ich
               wieder Dein liebes Bild etwas klarer vor Augen habe und damit ich nicht blos auf die
               Erinnerungen angewieſen bin, um es mir zu verdeutlichen, – ſo wäre ich Dir recht ſehr
               dankbar.\pend
           \pstart
           Auch ein Paar Nachrichten über die \textcolor{blue}{Freunde}{}\ledrightnote{→\textcolor{blue}{Richard Beer-Hofmann}{\newline}→\textcolor{blue}{Hugo von Hofmannsthal}}, von denen ich kein Wort mehr weiß, über
                  \textsc{\textcolor{blue}{Richard}{}\ledrightnote{\textcolor{blue}{Richard Beer-Hofmann}}} und {\pb}\textsc{\textcolor{blue}{Loris}{}\ledrightnote{\textcolor{blue}{Hugo von Hofmannsthal}}}, würden mir hochwillkommen ſein, ſowie über dieſen \label{K_L02708-4v}\edtext{\textcolor{blue}{Tauſendkünſtler}{}\ledrightnote{→\textcolor{blue}{Hermann Bahr}}}{\lemma{\textnormal{\emph{Tauſendkünſtler}}}\Cendnote{\textnormal{Anspielung auf \textcolor{blue}{Hermann Bahr}s vielseitige journalistische und literarische
                  Betätigung}}}\label{K_L02708-4h}{ }\textsc{\textcolor{blue}{Hermann Bahr}{}\ledrightnote{\textcolor{blue}{Hermann Bahr}}}, der \strikeout{\textcolor{gray}{×}} es alſo doch fertig gebracht zu haben ſcheint, in \textcolor{pink}{Wien}{}\ledrightnote{\textcolor{pink}{Wien}}{ }\textsc{Carrière} zu machen, worum ich ihn aufrichtig beneide.\pend
           \pstart
           Daran, Dir meine Dienſte in den ſchwierigen Zeiten, die Du jetzt durch machſt,
               anzubieten, habe ich \strikeout{\textcolor{gray}{×}} gedacht, aber ich habe \strikeout{mich} auch gemeint, daß
               Du mich leider kaum wirſt brauchen können. Iſt Dir aber doch zu etwas eine
               bedingungsloſe Ergebenheit nützlich, ſo denke daran, daß es für mich keine größere
               Freude geben könnte, als ſie Dir zu beweiſen.\pend
           \pstart In Treue Dein \spacefill\mbox{Paul Goldm}\pend{}\endnumbering\briefempfaengerindex{Schnitzler, Arthur@\textsc{Schnitzler, Arthur}!zzzGoldmann, Paul@\emph{von Paul Goldmann}!1893-05-221@{22. 5. {[}1893{]}}|)be}\mylabel{h}\begin{anhang}\end{anhang}\normalsize

\doendnotes{C}
\bigskip
\vfill

\clearpage

\footnotesize

\lohead{\textsc{register}}

% Definiere theindex-Environment komplett neu ohne reledmac
\makeatletter
\renewenvironment{theindex}{%
  \section*{\indexname}%
  \setlength{\parindent}{0pt}%
  \setlength{\parskip}{0pt plus 0.3pt}%
  \let\item\@idxitem
}{%
  \clearpage
}
\makeatother

\IfFileExists{\jobname-pw.ind}{\input{\jobname-pw.ind}}{}

\end{document}

      