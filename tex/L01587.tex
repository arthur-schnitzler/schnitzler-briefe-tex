%% latex-korrekturansicht-vorspann.tex
%% Vorspann für die Korrekturansicht.
%% Lädt die gemeinsame Datei latex-vorspann.tex mit gesetztem Schalter.

\newif\ifkorrekturansicht
\korrekturansichttrue

\input{../tex-inputs/latex-vorspann}


               \section[Arthur Schnitzler an Hugo von Hofmannsthal, 6. 3. 1906]{ Arthur Schnitzler an Hugo von Hofmannsthal, 6. 3. 1906}\nopagebreak\mylabel{v}\rehead{ }\normalsize\beginnumbering\briefempfaengerindex{Hofmannsthal, Hugo von@\textsc{Hofmannsthal, Hugo von}!zzzSchnitzler, Arthur@\emph{von Arthur Schnitzler}!1906-03-061@{6. 3. 1906}|(be} \toendnotes[C]{\smallbreak\pagebreak[2]} \Standort{FDH, Hs-30885,124.}
\physDesc{Brief, 2 Blätter, 8 Seiten
\newline{}Handschrift: schwarze Tinte, deutsche Kurrent\newline{}Ordnung: mit Bleistift von Schnitzler mutmaßlich bei der
                                 Durchsicht der Korrespondenz 1929 das zweite Blatt datiert: »6/3 906« und nummeriert: »II.« }\buchAbdrucke{\weitereDrucke{Hugo von Hofmannsthal, Arthur Schnitzler: \emph{Briefwechsel}. Hg. Therese Nickl und Heinrich Schnitzler. Frankfurt am Main: \emph{S. Fischer} 1964, S. 218.} }\toendnotes[C]{\smallbreak}\pstart
           {\pb}\textcolor{gray}{\textbf{Dr. Arthur Schnitzler}}\hfill 6. 3. 906\pend
           \pstart
           \textcolor{gray}{\textbf{\textcolor{pink}{Wien XVIII. Spoettelgasse 7}{}\ledrightnote{\textcolor{pink}{Edmund-Weiß-Gasse}}.}}\pend
           \pstart{}mein lieber Hugo, \pend\pstart
           aus verſchiedenen Gründen ſind wir erſt Samſtag Abend frei u Ihnen zur
               Verfügung und fragen Sie, ob Sie lieber bei uns nachtmahlen \strikeout{wollen} oder ob wir einander in \textcolor{pink}{Hietzing}{}\ledrightnote{\textcolor{pink}{XIII., Hietzing}}
               treffen wollen? Es wäre ſehr nett von Ihnen \textcolor{blue}{beiden}{}\ledrightnote{→\textcolor{blue}{Gertrude von Hofmannsthal}}, wenn Sie die Reiſe in die \textcolor{pink}{Spöttelgaſſe}{}\ledrightnote{\textcolor{pink}{Edmund-Weiß-Gasse}} nicht ſcheuten. –\pend
           \pstart
           {\pb}\textcolor{blue}{Harden}{}\ledrightnote{\textcolor{blue}{Maximilian Harden}} hat mich nur mäßig irritirt. Erſtens weil
               ich auf alles mögliche gefaſſt war, da man mir ja gleich (Theater\textcolor{pink}{berlin}{}\ledrightnote{\textcolor{pink}{Berlin}} iſt ja ein Tratſchneſt) von ſeinem albern taktloſen
               Benehmen im Theater bei der \textcolor{green}{\textsc{Première}}{}\ledrightnote{→\textcolor{green}{Der Ruf des Lebens. Schauspiel in drei Akten}} erzählt hatte. Ferner iſt mir ſeine Erſcheinung als die eines Politikers, eines
               großen u amuſanten Politikers in allen Dingen dieſer Welt alſo auch in der Kunſt (und
               ſogar in der Politik) ſeit lange ſo feſtſtehend, {\pb}daſs
               mir alle ſeine Emanationen auch nur in dieſem Sinne wirklich intereſſant ſind. Daſs
               er trotzdem manchmal höchſt vorzügliches \substVorne{}\textsuperscript{mit}\substDazwischen{}und\substHinten{}{ }\strikeout{über} ſogar treffendes über Menſchen, Künſtler,
               Bücher, Stücke ſagt – insbeſondere wenn er vom »politiſchen« abſehen kann, und noch
               öfter, wenn ſein Geſchmack und ſeine Parteiſtellung in einer ihm ſelbſt unbewußten
               Weiſe ineinanderfließen – würd ich nicht leugnen, auch we{\geminationn} er noch lächerlicher über mich \textcolor{green}{geſchrie{\pb}ben}{}\ledrightnote{→\textcolor{green}{Theater}} hätte. Im übrigen hab ich nicht
               einmal die Empfindung, daſs er mich hat treffen wollen, und käme der Fall vor
               Gericht, ſo würd ich ihn vielleicht wegen momentaner Si{\geminationn}esverwirrung freiſprechen. Ja we{\geminationn} ich alle die
               vielfältigen Elemente meines heutigen Verhältniſſes zu ihm unterſuche, ſo möcht ich
               faſt glauben, dſs auch irgend ein Hauch von Mitleid dabei iſt.\pend
           \pstart
           Nun was das \textcolor{green}{Stück}{}\ledrightnote{→\textcolor{green}{Der Ruf des Lebens. Schauspiel in drei Akten}}{ }ſelbſt anbelangt ſo iſt ja beim beſten Willen nicht
               zu überſehen, daſs im \textcolor{green}{3. Akt}{}\ledrightnote{→\textcolor{green}{Der Ruf des Lebens. Schauspiel in drei Akten}} ein
                  {\pb}tiefer Fehler ſteckt – der damit nicht geringer
               erklärt wird, daſs man ihn \introOben{}im\introOben{} architektoniſchen am
               deutlichſten entdeckt. Auf einem Spaziergang heute, an dieſem ſchönen Frühlingstag,
               durch den \textcolor{pink}{Dornbacherpark}{}\ledrightnote{\textcolor{pink}{Dornbacher Park}}, hab ich mir den »\textcolor{green}{Ruf}{}\ledrightnote{\textcolor{green}{Der Ruf des Lebens. Schauspiel in drei Akten}}« neu entworfen (ſchreiben werd ich ihn wohl nie)
               in fünf Akten und glaube an den Wurzeln geweſen zu ſein. So klug wie meine klügſten
               Kritiker bin ich lange noch: ich müßte {\pb}nur noch um
               einiges mehr Dichter ſein und die Welt \substVorne{}\textsuperscript{könnte}{\allowbreak}\substDazwischen{}dürfte\substHinten{} Dramen von mir erwarten, die weder durch die Talentloſigkeit des Fräulein
                  \textcolor{blue}{Schiff}{}\ledrightnote{\textcolor{blue}{Else Bassermann}} noch durch die Boſheit des Herrn \textcolor{blue}{Rittner}{}\ledrightnote{\textcolor{blue}{Rudolf Rittner}} umzubringen wären.\pend
           \pstart
           Im \textcolor{green}{Oedipus}{}\ledrightnote{\textcolor{green}{Oedipus und die Sphinx. Tragödie in drei Aufzügen}} haben die \textcolor{blue}{\textsc{Sandrock}}{}\ledrightnote{\textcolor{blue}{Adele Sandrock}} und \textcolor{blue}{\textsc{Moissi}}{}\ledrightnote{\textcolor{blue}{Alexander Moissi}} am ſtärkſten auf mich gewirkt (\label{K_L01587_1v}\edtext{Dinſtag den 24. Feber}{\lemma{\textnormal{\emph{Dinſtag den 24. Feber}}}\Cendnote{\textnormal{Er war am 26. 2. 1906, einem Montag, in der
                  Vorführung.}}}\label{K_L01587_1h}), die \textcolor{blue}{\textsc{Sorma}}{}\ledrightnote{\textcolor{blue}{Agnes Sorma}} bei aller edeln Süßigkeit ſchien mir nicht ohne Manier. Was mit dem Chor \introOben{}(von \textcolor{blue}{Reinhardt}{}\ledrightnote{\textcolor{blue}{Max Reinhardt}})\introOben{} intendirt
               war, hat mich mächtig ergriffen, in der Ausführung ſtörte mich zuweilen bildlich {\pb}geſprochen die überdeutliche Arbeit der Maſchinerie. Was
               mich aus dem dritten Akt des \textcolor{green}{Werkes}{}\ledrightnote{→\textcolor{green}{Oedipus und die Sphinx. Tragödie in drei Aufzügen}}, das ich bewundere, etwas kühl angeweht hat, weiſs ich mir ſelbſt
               noch nicht recht zu deuten – vielleicht war es nichts andres, als daſs ich nach Hauſe
               geſchickt wurde, während ich, in höherm Sinn, nur in einen Zwiſchenakt entlaſſen
               werden durfte. Um was ich Sie diesmal beſonders beneide, iſt, daſs Sie mit einem \textcolor{blue}{Regiſſeur}{}\ledrightnote{→\textcolor{blue}{Max Reinhardt}} arbeiten konnten, der
               an Ihr Werk glaubte. Die \substVorne{}\textsuperscript{Mischung}{\allowbreak}\substDazwischen{}Atmosphäre\substHinten{} von Pflichttreue und künſtleriſcher Feindſeligkeit, in der \strikeout{mich} mein \textcolor{green}{Werk}{}\ledrightnote{→\textcolor{green}{Der Ruf des Lebens. Schauspiel in drei Akten}} zum Bühnenleben erwuchs, hatte {\pb}etwas niederdrückendes.\pend
           \pstart
           Herzlichſt{\\[\baselineskip]}Ihr{\\[\baselineskip]}\spacefill\mbox{A.}\pend
           \leftskip=0em{}\endnumbering\briefempfaengerindex{Hofmannsthal, Hugo von@\textsc{Hofmannsthal, Hugo von}!zzzSchnitzler, Arthur@\emph{von Arthur Schnitzler}!1906-03-061@{6. 3. 1906}|)be}\mylabel{h}  \normalsize

\doendnotes{C}
\bigskip
\vfill

\clearpage

\footnotesize

\lohead{\textsc{register}}

% Definiere theindex-Environment komplett neu ohne reledmac
\makeatletter
\renewenvironment{theindex}{%
  \section*{\indexname}%
  \setlength{\parindent}{0pt}%
  \setlength{\parskip}{0pt plus 0.3pt}%
  \let\item\@idxitem
}{%
  \clearpage
}
\makeatother

\IfFileExists{\jobname-pw.ind}{\input{\jobname-pw.ind}}{}

\end{document}

      