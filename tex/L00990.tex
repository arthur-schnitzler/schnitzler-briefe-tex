%% latex-korrekturansicht-vorspann.tex
%% Vorspann für die Korrekturansicht.
%% Lädt die gemeinsame Datei latex-vorspann.tex mit gesetztem Schalter.

\newif\ifkorrekturansicht
\korrekturansichttrue

\input{../tex-inputs/latex-vorspann}


               \section[Arthur Schnitzler an Hugo von Hofmannsthal, 8. 10. 1899]{ Arthur Schnitzler an Hugo von Hofmannsthal, 8. 10. 1899}\nopagebreak\mylabel{v}\rehead{ }\normalsize\beginnumbering\briefempfaengerindex{Hofmannsthal, Hugo von@\textsc{Hofmannsthal, Hugo von}!zzzSchnitzler, Arthur@\emph{von Arthur Schnitzler}!1899-10-082@{8. 10. 1899}|(be} \toendnotes[C]{\smallbreak\pagebreak[2]} \Standort{FDH, Hs-30885,88.}
\physDesc{Brief, 1 Blatt, 4 Seiten
\newline{}Handschrift: schwarze Tinte, deutsche Kurrent}\buchAbdrucke{\weitereDrucke{1) Hugo von Hofmannsthal, Arthur Schnitzler: \emph{Briefwechsel}. Hg. Therese Nickl und Heinrich Schnitzler. Frankfurt am Main: \emph{S. Fischer} 1964, S. 132–133.} \weitereDrucke{2) Hermann Bahr, Arthur Schnitzler: \emph{Briefwechsel, Aufzeichnungen, Dokumente
                                (1891–1931)}. Hg. Kurt Ifkovits und Martin Anton Müller. Göttingen: \emph{Wallstein} 2018, S. 172.} }\toendnotes[C]{\smallbreak}\pstart
           \raggedleft{}{\pb}\textsc{\textcolor{pink}{Berlin}{}\ledrightnote{\textcolor{pink}{Berlin}}}, 8. 10. 99.\pend
           \pstart
           mein lieber Hugo, geſtern Abend hab ich die \textsc{\textcolor{green}{Beatrice}{}\ledrightnote{\textcolor{green}{Der Schleier der Beatrice. Schauspiel in fünf Akten}}} dem \textcolor{blue}{Brahm}{}\ledrightnote{\textcolor{blue}{Otto Brahm}} vorgeleſen; mir ſcheint, ſie
                    hat auf ihn gewirkt, eigentlich hatte er keine Einwendungen, und jedenfalls kam
                    ihm die Sache fertiger vor als mir, der ich ſie keinesfalls vorläufig aus der
                    Hand gebe. Ich weiſs ſehr genau was noch daran zu machen iſt; und einiges wird
                    auch gelingen. Die entſchiedenſte {\pb}Einwendg von \textcolor{blue}{Brahm}{}\ledrightnote{\textcolor{blue}{Otto Brahm}} war eigentlich der Monolog oder beſſer
                    die Anrede des \textsc{Andrea} – das einzige Stückl, das Sie
                    kennen, – das er ganz hinaus haben möchte. Ich las, mit einer Souper
                    Unterbrechung von 7–12; ſo lang würde die Sache ungeſtrichen mindeſtens
                    ſpielen!\pend
           \pstart
           Ich werde wahrſcheinlich Donnerſtag in \textcolor{pink}{Wien}{}\ledrightnote{\textcolor{pink}{Wien}}{ }ſein;
                        \textcolor{blue}{Paul Goldmann}{}\ledrightnote{\textcolor{blue}{Paul Goldmann}} ko{\geminationm}t auch und wird etwa acht {\pb}Tage bei mir wohnen. Wann ſind Sie wieder in \textcolor{pink}{Wien}{}\ledrightnote{\textcolor{pink}{Wien}}? Es wäre ſchön, wenn \textcolor{blue}{G.}{}\ledrightnote{\textcolor{blue}{Paul Goldmann}}{ }Sie noch zu ſehen bekäme. –\pend
           \pstart
           Über das äußere Leben hier lieber mündlich. –\pend
           \pstart
           Ich weiſs nicht, ob Sie dieſes \label{K_L00990_1v}\edtext{\textcolor{green}{Anfangsfeuilleton}{}\ledrightnote{→\textcolor{green}{Die Entdeckung der Provinz}}}{\lemma{\textnormal{\emph{Anfangsfeuilleton}}}\Cendnote{\textnormal{\emph{\textcolor{green}{Die Entdeckung der Provinz}} ist Bahrs
                        erstes Feuilleton für das \emph{\textcolor{brown}{Neue Wiener
                            Tagblatt}}.}}}\label{K_L00990_1h} von \textcolor{blue}{Bahr}{}\ledrightnote{\textcolor{blue}{Hermann Bahr}}
                    geleſen haben. Ich schicks Ihnen hier. \label{LL439-1v}Er
                        iſt gewiſs nicht nur ein Aff, ſondern auch ein boshafter Aff. –\label{LL439-1h}\pend
           \pstart
           Wie geht’s Ihnen? Fließt die Arbeit {\pb}munter fort? –
                    Daſs Ihnen das \textcolor{green}{Stück}{}\ledrightnote{→\textcolor{green}{Das Bergwerk zu Falun}}{ }ſich
                    verſagen könnte, iſt ganz unmöglich; es geht in ſo reiner Linie vorwärts, daſs
                    es nur mehr auf die rechte Sti{\geminationm}ung ankommt. Am Ende
                    bringen Sie’s ſchon vollendet nach \textcolor{pink}{Wien}{}\ledrightnote{\textcolor{pink}{Wien}}? –\pend
           \pstart
           Das \textcolor{pink}{Deutſche Theater}{}\ledrightnote{\textcolor{pink}{Deutsches Theater Berlin}} braucht ungeheuer notwendig
                    ein oder mehrere Stücke. \textcolor{blue}{Br.}{}\ledrightnote{\textcolor{blue}{Otto Brahm}} hat ſo gut wie
                    gar nichts. Meines will ich in jedem Fall zuerſt in \textcolor{pink}{Wien}{}\ledrightnote{\textcolor{pink}{Wien}}{ }ſpielen laſſen; aber es eilt nicht. Ich habe viel vor und möchte
                    wohler, möchte ganz geſund ſein.\pend
           \pstart Von Herzen Ihr \spacefill\mbox{Arthur}\pend{}\endnumbering\briefempfaengerindex{Hofmannsthal, Hugo von@\textsc{Hofmannsthal, Hugo von}!zzzSchnitzler, Arthur@\emph{von Arthur Schnitzler}!1899-10-082@{8. 10. 1899}|)be}\mylabel{h}  \normalsize

\doendnotes{C}
\bigskip
\vfill

\clearpage

\footnotesize

\lohead{\textsc{register}}

% Definiere theindex-Environment komplett neu ohne reledmac
\makeatletter
\renewenvironment{theindex}{%
  \section*{\indexname}%
  \setlength{\parindent}{0pt}%
  \setlength{\parskip}{0pt plus 0.3pt}%
  \let\item\@idxitem
}{%
  \clearpage
}
\makeatother

\IfFileExists{\jobname-pw.ind}{\input{\jobname-pw.ind}}{}

\end{document}

      