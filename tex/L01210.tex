%% latex-korrekturansicht-vorspann.tex
%% Vorspann für die Korrekturansicht.
%% Lädt die gemeinsame Datei latex-vorspann.tex mit gesetztem Schalter.

\newif\ifkorrekturansicht
\korrekturansichttrue

\input{../tex-inputs/latex-vorspann}


               \section[Hugo von Hofmannsthal an Arthur Schnitzler, {[}28. 3. 1902{]}]{ Hugo von Hofmannsthal an Arthur Schnitzler, {[}28. 3. 1902{]}}\nopagebreak\mylabel{v}\rehead{ }\normalsize\beginnumbering\briefempfaengerindex{Schnitzler, Arthur@\textsc{Schnitzler, Arthur}!zzzHofmannsthal, Hugo von@\emph{von Hugo von Hofmannsthal}!1902-03-281@{28. 3. 1902}|(be} \toendnotes[C]{\smallbreak\pagebreak[2]} \Standort{CUL, Schnitzler, B 43.}
\physDesc{Brief, 3 Blätter, 12 Seiten
\newline{}Handschrift: schwarze Tinte, deutsche Kurrent
\newline{}Schnitzler: mit Bleistift datiert: »28/3 902« \newline{}Ordnung: 1) mit Bleistift von unbekannter Hand nummeriert:
                              »\strikeout{193}« 2) mit Bleistift von unbekannter Hand nummeriert: »186« und die folgenden Blätter mit
                                 »186.2.« beziehungsweise »186.3.« beschriftet}\buchAbdrucke{\weitereDrucke{Hugo von Hofmannsthal, Arthur Schnitzler: \emph{Briefwechsel}. Hg. Therese Nickl und Heinrich Schnitzler. Frankfurt am Main: \emph{S. Fischer} 1964, S. 154–155.} }\toendnotes[C]{\smallbreak}\pstart{}{\pb}mein lieber guter
                  Arthur,\pend\pstart
           ich will Ihnen aufrichtig ſagen, daſs mich Ihr Telegramm ſehr verletzt hat. Ich will
               es deswegen lieber ausſprechen als verſchweigen, weil ich glaube, daſs das, was an
               ſolchen Dingen für mich ſo verletzend iſt, von Ihnen, als höchſt unwichtig, kaum {\pb}bemerkt wird und
                  da{[}ſ{]}s das Ganze in dem Moment vermieden wäre, wo Sie überhaupt
               zum Bewuſstſein davon kämen.\pend
           \pstart
           In den 10 Jahren, ſeit wir uns kennen, hab ich die unaufhörliche Freude eines intimen
               Verkehrs mit Ihnen immer unter ſolchen Formen {\pb}genießen können, die Ihre
               Bequemlichkeit in Bezug auf Ort und Stunde des Zuſammentreffens etc nie tangiert
               haben. Es war nicht nur für Sie, ſondern auch für mich bequemer, es war durch alle
               Umſtände gegeben, daſs Sie faſt nie zu mir gekommen ſind und ich oft zu Ihnen etc.
               etc.\pend
           \pstart
           {\pb}Und andererſeits haben Sie in
               dieſer langen Zeit wohl auch bemerken können, daſs mir ziemlich fern liegt Sie irgend
               wie durch Bekanntmachen mit Leuten etc in Anſpruch zu nehmen.\pend
           \pstart
           Nun ereignet ſich ein beſonderer ganz vereinzelter Fall: eine \textcolor{blue}{Frau}{}\ledrightnote{→\textcolor{blue}{Christiane von Thun-Hohenstein-Salm-Reifferscheidt}}, mit der ich ziemlich befreundet bin, {\pb}und die wirklich eine merkwürdige
                  \textcolor{blue}{Frau}{}\ledrightnote{→\textcolor{blue}{Christiane von Thun-Hohenstein-Salm-Reifferscheidt}} iſt, durch eine ſeltene
               Übereinſtimmung von Güte, Vornehmheit und wirklichem Geiſt, dabei von der äußerſten
               Zurückhaltung, iſoliert und faſt menſchenſcheu, dieſe \textcolor{blue}{Frau}{}\ledrightnote{→\textcolor{blue}{Christiane von Thun-Hohenstein-Salm-Reifferscheidt}} erfreut mich (ich gebrauche das Wort in ſeiner
               wirklichen Bedeutung) ſeit jeher durch ihre warme {\pb}und kluge Auffaſſung aller Ihrer
               Arbeiten. Und dieſe \textcolor{blue}{Frau}{}\ledrightnote{→\textcolor{blue}{Christiane von Thun-Hohenstein-Salm-Reifferscheidt}}\strikeout{,}{ }ſpricht mir, ganz ausnahmsweiſe, ihrer Art gar
               nicht entſprechend, lebhaft und mehrmals den Wunſch aus, Sie einmal zu ſehen. Ich
               antworte: ganz gern, ganz leicht, einmal bei mir draußen. Es vergeht der Herbſt, der
               Winter, es {\pb}kommt das
               unfreundliche Frühjahr und da ſie furchtbar an Neuralgien leidet, ſagt ſie: ſo werde
               ich wieder nicht nach \textcolor{pink}{Rodaun}{}\ledrightnote{\textcolor{pink}{Rodaun}} kommen, und ich füge
               hinzu: und das mit dem Schnitzler wird nicht zuſammengehen. Im Augenblick fällt uns
               ein, daſs ſie in ihrer Wohnung {\pb}ganz allein iſt, ihre \textcolor{blue}{Söhne}{}\ledrightnote{→\textcolor{blue}{Josef Oswald Thun-Hohenstein-Salm-Reifferscheid}{\newline}→\textcolor{blue}{Paul von Thun-Hohenstein}{\newline}→\textcolor{blue}{Adolf Thun-Hohenstein-Salm-Reifferscheid}} in \textcolor{pink}{Prag}{}\ledrightnote{\textcolor{pink}{Prag}}, ihr \textcolor{blue}{Mann}{}\ledrightnote{→\textcolor{blue}{Oswald Thun-Hohenstein-Salm-Reifferscheid}} an der \textcolor{pink}{Riviera}{}\ledrightnote{\textcolor{pink}{Riviera}}, und es kommt uns, mit der halb kindiſchen Freude, etwas
               ungewöhnliches zu arrangieren, der Gedanke an dieſes Frühſtück. Aus Beſcheidenheit
               fügt ſie hinzu, man ſollte, damit Sie ſich nicht langweilen, noch {\pb}jemand Geſcheidten einladen der
               Ihnen neu und unterhaltend ſein könnte, ich ſchlage \textcolor{blue}{Kaſſner}{}\ledrightnote{\textcolor{blue}{Rudolf Kassner}} vor, den ich Ihnen ſchon lange bekannt machen wollte, man wählt die
               Stunde des Frühſtücks, die Sie in nichts ſtören kann, weil {\pb}ich weiß daſs Sie nachmittags gern
               arbeiten und Ruhe haben, es iſt eine \textcolor{pink}{Wohnung}{}\ledrightnote{→\textcolor{pink}{Palais Thun-Salm}} in der inneren Stadt,\pend
           \pstart
           \numberlinefalse{}–\numberlinetrue{}\pend
           \pstart
           ich überſchreite eine ſeit 10 Jahren geübte Zurückhaltung und trage Ihnen dieſe Sache
               als herzlichen Wunſch oder Bitte von {\pb}mir vor, und Sie antworten, daſs
               Ihnen Mittagseinladungen in der nächſten Zeit unbequem ſind!\pend
           \pstart
           Ich kann wirklich nicht weiterſchreiben, weil ich zu erregt bin, und die Thränen in
               den Augen {\pb}habe, natürlich nicht
               vor Rührung ſondern vor Zorn.\pend
           \pstart
           Da Sie aus dieſer Heftigkeit vielleicht gerade bemerken, wie herzlich gern ich Sie
               habe, ſo hoffe ich, daſs dieſer Brief Sie in keiner häſslichen Art ärgern wird.\pend
           \pstart
           Von Herzen Ihr{\\[\baselineskip]}\spacefill\mbox{Hugo.}\pend
           \leftskip=0em{}\endnumbering\briefempfaengerindex{Schnitzler, Arthur@\textsc{Schnitzler, Arthur}!zzzHofmannsthal, Hugo von@\emph{von Hugo von Hofmannsthal}!1902-03-281@{28. 3. 1902}|)be}\mylabel{h}  \normalsize

\doendnotes{C}
\bigskip
\vfill

\clearpage

\footnotesize

\lohead{\textsc{register}}

% Definiere theindex-Environment komplett neu ohne reledmac
\makeatletter
\renewenvironment{theindex}{%
  \section*{\indexname}%
  \setlength{\parindent}{0pt}%
  \setlength{\parskip}{0pt plus 0.3pt}%
  \let\item\@idxitem
}{%
  \clearpage
}
\makeatother

\IfFileExists{\jobname-pw.ind}{\input{\jobname-pw.ind}}{}

\end{document}

      