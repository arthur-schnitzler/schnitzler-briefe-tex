%% latex-korrekturansicht-vorspann.tex
%% Vorspann für die Korrekturansicht.
%% Lädt die gemeinsame Datei latex-vorspann.tex mit gesetztem Schalter.

\newif\ifkorrekturansicht
\korrekturansichttrue

\input{../tex-inputs/latex-vorspann}


               \section[Hugo von Hofmannsthal an Arthur Schnitzler, {[}9. 4. 1904{]}]{ Hugo von Hofmannsthal an Arthur Schnitzler, {[}9. 4. 1904{]}}\nopagebreak\mylabel{v}\rehead{ }\normalsize\beginnumbering\briefempfaengerindex{Schnitzler, Arthur@\textsc{Schnitzler, Arthur}!zzzHofmannsthal, Hugo von@\emph{von Hugo von Hofmannsthal}!1904-04-091@{{[}9. 4. 1904{]}}|(be} \toendnotes[C]{\smallbreak\pagebreak[2]} \Standort{CUL, Schnitzler, B 43.}
\physDesc{Brief, 1 Blatt, 1 Seite
\newline{}Handschrift: schwarze Tinte, deutsche Kurrent
\newline{}Schnitzler: mit Bleistift den mutmaßlichen Empfangstag vermerkt: »11/4 904.« \newline{}Ordnung: 1) mit Bleistift von unbekannter Hand nummeriert: »\strikeout{214}« 2) mit Bleistift von unbekannter Hand nummeriert:
                                    »220«}\buchAbdrucke{\weitereDrucke{Hugo von Hofmannsthal, Arthur Schnitzler: \emph{Briefwechsel}. Hg. Therese Nickl und Heinrich Schnitzler. Frankfurt am Main: \emph{S. Fischer} 1964, S. 186.} }\toendnotes[C]{\smallbreak}\pstart
           \noindent{}{\pb}lieber, \textcolor{blue}{Papa}{}\ledrightnote{→\textcolor{blue}{Hugo August von Hofmannsthal}} freut ſich ſehr über Ihre
               Freundlichkeit und wird mit Freude \label{K_L01390_1v}\edtext{Montag abends}{\lemma{\textnormal{\emph{Montag abends}}}\Cendnote{\textnormal{vgl. A. S.: \emph{Tagebuch}, 11. 4. 1904}}}\label{K_L01390_1h} mitko{\geminationm}en.\hspace*{1.5em}Allenfalls ſagen
               Sie es vielleicht auch \textcolor{blue}{S.kopf}{}\ledrightnote{\textcolor{blue}{Gustav Schwarzkopf}}, mit dem \textcolor{blue}{Papa}{}\ledrightnote{→\textcolor{blue}{Hugo August von Hofmannsthal}}{ }ſehr gut ſteht {\dots} ganz
               wie Sie gelaunt ſind.\pend
           \pstart
           Von Herzen Ihr{\\[\baselineskip]}\spacefill\mbox{Hugo.}\pend
           \leftskip=0em{}\pstart
           Samstg{ }abend.\pend
           \endnumbering\briefempfaengerindex{Schnitzler, Arthur@\textsc{Schnitzler, Arthur}!zzzHofmannsthal, Hugo von@\emph{von Hugo von Hofmannsthal}!1904-04-091@{{[}9. 4. 1904{]}}|)be}\mylabel{h}  \normalsize

\doendnotes{C}
\bigskip
\vfill

\clearpage

\footnotesize

\lohead{\textsc{register}}

% Definiere theindex-Environment komplett neu ohne reledmac
\makeatletter
\renewenvironment{theindex}{%
  \section*{\indexname}%
  \setlength{\parindent}{0pt}%
  \setlength{\parskip}{0pt plus 0.3pt}%
  \let\item\@idxitem
}{%
  \clearpage
}
\makeatother

\IfFileExists{\jobname-pw.ind}{\input{\jobname-pw.ind}}{}

\end{document}

      