%% latex-korrekturansicht-vorspann.tex
%% Vorspann für die Korrekturansicht.
%% Lädt die gemeinsame Datei latex-vorspann.tex mit gesetztem Schalter.

\newif\ifkorrekturansicht
\korrekturansichttrue

\input{../tex-inputs/latex-vorspann}


               \section[Karl Kraus an Arthur Schnitzler, 22. 1. 1893]{ Karl Kraus an Arthur Schnitzler, 22. 1. 1893}\nopagebreak\mylabel{v}\rehead{ }\normalsize\beginnumbering\briefempfaengerindex{Schnitzler, Arthur@\textsc{Schnitzler, Arthur}!zzzKraus, Karl@\emph{von Karl Kraus}!1893-01-221@{22. 1. 1893}|(be} \toendnotes[C]{\smallbreak\pagebreak[2]} \Standort{CUL, Schnitzler, B 55.}
\physDesc{Briefkarte
\newline{}Handschrift: schwarze Tinte, deutsche Kurrent}\buchAbdrucke{\weitereDrucke{1) \emph{Karl Kraus und Arthur Schnitzler. Eine Dokumentation.} Hg. Reinhard Urbach. In: \emph{Literatur und Kritik}, Bd. 49, Oktober 1970, S. 514–515.} \weitereDrucke{2) Hermann Bahr, Arthur Schnitzler: \emph{Briefwechsel, Aufzeichnungen, Dokumente
                                (1891–1931)}. Hg. Kurt Ifkovits und Martin Anton Müller. Göttingen: \emph{Wallstein} 2018, S. 32.} }\toendnotes[C]{\smallbreak}\pstart
           {\pb}\textcolor{pink}{Wien}{}\ledrightnote{\textcolor{pink}{Wien}}, 22/\textsubscript{I} 93.\pend
           \pstart
           Lieber Herr Doctor! Bin grade in einer Hochzeit drin; beeile
                    mich aber trotzdem Ihren lieben Brief, den ich eben erhielt, zu beantworten; ich
                    hatte nämlich gleich nachm. für Sie Kritikauschnitt vorbereitet u. dazu ein
                    Briefchen geſchrieben, welches ich nun freilich nicht benutzen kann.\pend
           \pstart
           Alſo ich bin in der angenehmen Lage, Ihnen einen Ausschnitt bereits heute
                    verschaffen zu können. Anbei ist er.\pend
           \pstart
           {\pb}Haben Sie zufällig \textcolor{green}{Fr. Bühne}{}\ledrightnote{\textcolor{green}{Freie Bühne für den Entwickelungskampf der Zeit}} Januarheft in die Hand bekommen?\pend
           \pstart
           Leſen Sie den \label{K_L00161_1v}\edtext{\textcolor{green}{Artikel}{}\ledrightnote{→\textcolor{green}{Von Hermann Bahr und seiner Bücherei}}}{\lemma{\textnormal{\emph{Artikel}}}\Cendnote{\textnormal{\textcolor{blue}{Felix Hollaender}: \emph{\textcolor{green}{Von Hermann Bahr und seiner Bücherei}}. In: \emph{\textcolor{green}{Freie Bühne}}, Jg. 4, Nr. 1,
                                1. 1. 1893, S. 82–89.}}}\label{K_L00161_1h} von \introOben{}F.\introOben{}{ }\textcolor{blue}{Holländer}{}\ledrightnote{\textcolor{blue}{Felix Hollaender}} über \textcolor{blue}{\uline{Hermann Bahr}}{}\ledrightnote{\textcolor{blue}{Hermann Bahr}}, den er in geradezu dummer Weiſe in den Himmel hebt. Dort finden Sie bei
                    der Stelle über \textcolor{blue}{Bahr}{}\ledrightnote{\textcolor{blue}{Hermann Bahr}}’s \label{K_L00161_2v}\edtext{\textcolor{green}{Dora}{}\ledrightnote{\textcolor{green}{Dora}}-Schmarren}{\lemma{\textnormal{\emph{Dora-Schmarren}}}\Cendnote{\textnormal{\textcolor{blue}{Hermann Bahr}: \emph{\textcolor{green}{Dora}}. Berlin: \emph{\textcolor{brown}{S. Fischer}}{ }1893 (erschienen November 1892).
                        Schmarren, hier: Unsinn.}}}\label{K_L00161_2h}, den \textcolor{blue}{Holl.}{}\ledrightnote{\textcolor{blue}{Felix Hollaender}} für das größte psycholog. Kunſtwerk hält (!!!!), eine \uline{ſehr, ſehr}{ }\label{K_L00161_3v}\edtext{ſchmeichelhafte Bemerkung}{\lemma{\textnormal{\emph{ſchmeichelhafte Bemerkung}}}\Cendnote{\textnormal{S. 88: »Ich weiß bei uns
                            Niemanden, der nach diesem \textcolor{green}{Büchlein}
                     sich mit \textcolor{blue}{Bahr} messen könnte; in \textcolor{pink}{Oesterreich} käme nur noch \textcolor{blue}{Arthur Schnitzler} in Betracht.«}}}\label{K_L00161_3h}
                    über einen gewiſſen Arthur Schnitzler. Verzeihen Sie mir, Liebster, den \label{K_L00161_4v}\edtext{\textcolor{green}{Franz Moor}{}\ledrightnote{→\textcolor{green}{Die Räuber}}}{\lemma{\textnormal{\emph{Franz Moor}}}\Cendnote{\textnormal{vgl. A. S.: \emph{Tagebuch}, 14. 1. 1893}}}\label{K_L00161_4h}. Soll gewiss nimmer
                    vorkommen! bitte, bitte! Viele Grüße\hspace*{3.5em}Ihr ſehr
                    ergeb. \spacefill\mbox{Karl Kraus}.\pend
           \endnumbering\briefempfaengerindex{Schnitzler, Arthur@\textsc{Schnitzler, Arthur}!zzzKraus, Karl@\emph{von Karl Kraus}!1893-01-221@{22. 1. 1893}|)be}\mylabel{h}  \normalsize

\doendnotes{C}
\bigskip
\vfill

\clearpage

\footnotesize

\lohead{\textsc{register}}

% Definiere theindex-Environment komplett neu ohne reledmac
\makeatletter
\renewenvironment{theindex}{%
  \section*{\indexname}%
  \setlength{\parindent}{0pt}%
  \setlength{\parskip}{0pt plus 0.3pt}%
  \let\item\@idxitem
}{%
  \clearpage
}
\makeatother

\IfFileExists{\jobname-pw.ind}{\input{\jobname-pw.ind}}{}

\end{document}

      