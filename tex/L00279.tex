%% latex-korrekturansicht-vorspann.tex
%% Vorspann für die Korrekturansicht.
%% Lädt die gemeinsame Datei latex-vorspann.tex mit gesetztem Schalter.

\newif\ifkorrekturansicht
\korrekturansichttrue

\input{../tex-inputs/latex-vorspann}


               \section[Arthur Schnitzler an Hermann Bahr, 3. 11. 1893]{ Arthur Schnitzler an Hermann Bahr, 3. 11. 1893}\nopagebreak\mylabel{v}\rehead{ }\normalsize\beginnumbering\briefempfaengerindex{Bahr, Hermann@\textsc{Bahr, Hermann}!zzzSchnitzler, Arthur@\emph{von Arthur Schnitzler}!1893-11-032@{3. 11. 1893}|(be} \toendnotes[C]{\smallbreak\pagebreak[2]} \Standort{TMW, HS AM 23322 Ba.}
\physDesc{Brief, 1 Blatt (Briefpapier mit Trauerrand), 3 Seiten
\newline{}Handschrift: schwarze Tinte, deutsche Kurrent\newline{}Ordnung: Lochung }\buchAbdrucke{\weitereDrucke{1) \emph{3. 11. 1893.} In: Arthur Schnitzler: \emph{The Letters of Arthur Schnitzler to Hermann Bahr}. Edited, annotated, and with an introduction, by Donald G.
                        Daviau. Chapel Hill: \emph{The University of North Carolina Press} 1978, S. 57 (University of North Carolina studies in the Germanic languages
                        and literatures, 89).} \weitereDrucke{2) Hermann Bahr, Arthur Schnitzler: \emph{Briefwechsel, Aufzeichnungen, Dokumente (1891–1931)}. Hg. Kurt Ifkovits und Martin Anton Müller. Göttingen: \emph{Wallstein} 2018, S. 46.} }\toendnotes[C]{\smallbreak}\pstart{}{\pb}Lieber
                  Freund,\pend\pstart
           ich beiße bereits ſeit einigen Tagen in den ſauren Apfel, und werde mein Verſprechen
               halten. Es iſt nur wie ein Verhängnis, daſs mir nichts nach Wunsch gelingen will. Es
               iſt, wie we{\geminationn} mich die Empfindung: »man erwartet es von
               Dir« lähmte. –\pend
           \pstart
           – Seit ich Feuilletons ſchreiben ſoll, hab ich eine ewige unbezwingliche Luſt,
               fünfactige Trauer{\pb}ſpiele zu ſchreiben. Wirken Sie dahin, dſs \textsc{\textcolor{blue}{Burkhardt}{}\ledrightnote{\textcolor{blue}{Max Eugen Burckhard}}} eines von mir fordert – ich werde die ſchönste \textcolor{pink}{Wien}{}\ledrightnote{\textcolor{pink}{Wien}}er Geſchichte ſchreiben.\pend
           \pstart
           Im übrigen haben Sie Dinſtag oder ſpätestens Mittwoch das bewußte \textcolor{green}{Eingangsfeuilleton}{}\ledrightnote{→\textcolor{green}{Spaziergang}}. Eventuell werden Sie das
               Bedürfnis haben es zu ändern, wogegen ich principiell nichts einzuwenden habe. – (Nur
               müßt’ ich natürlich wiſſen, wie, wo, \textsc{etc.})\pend
           \pstart
           {\pb}Vielleicht werd ich
               auch noch im Stande ſein, Ihnen ſtatt des \textsc{\label{K_L00279_1v}\edtext{\textcolor{green}{Artifex}{}\ledrightnote{\textcolor{green}{Artifex}}}{\lemma{\textnormal{\emph{Artifex}}}\Cendnote{\textnormal{\emph{\textcolor{green}{Artifex}}, allegorisches Gedicht in Jamben,
                     entstanden im Sommer 1893, unveröffentlicht (\emph{Cambridge University Library}, Schnitzler, A 49). Eine Überarbeitung
                     fand am 19. 11. 1893 statt.}}}\label{K_L00279_1h}} was geſcheidteres zu geben. Wollen Sie mir ihn nicht vorläufig zurückleihen,
               damit ich zum mindeſten die böſeſten Verſe in ein behaglicheres Deutſch übertrage? – \pend
           \pstart
           – Herzlichen Gruſs{\\[\baselineskip]}Ihr sehr ergebner{\\[\baselineskip]}\spacefill\mbox{Arthur Schnitzler.}\pend
           \leftskip=0em{}\pstart
           \textcolor{pink}{Wien}{}\ledrightnote{\textcolor{pink}{Wien}}{ }3. XI. 93.\pend
           \endnumbering\briefempfaengerindex{Bahr, Hermann@\textsc{Bahr, Hermann}!zzzSchnitzler, Arthur@\emph{von Arthur Schnitzler}!1893-11-032@{3. 11. 1893}|)be}\mylabel{h}  \normalsize

\doendnotes{C}
\bigskip
\vfill

\clearpage

\footnotesize

\lohead{\textsc{register}}

% Definiere theindex-Environment komplett neu ohne reledmac
\makeatletter
\renewenvironment{theindex}{%
  \section*{\indexname}%
  \setlength{\parindent}{0pt}%
  \setlength{\parskip}{0pt plus 0.3pt}%
  \let\item\@idxitem
}{%
  \clearpage
}
\makeatother

\IfFileExists{\jobname-pw.ind}{\input{\jobname-pw.ind}}{}

\end{document}

      