%% latex-korrekturansicht-vorspann.tex
%% Vorspann für die Korrekturansicht.
%% Lädt die gemeinsame Datei latex-vorspann.tex mit gesetztem Schalter.

\newif\ifkorrekturansicht
\korrekturansichttrue

\input{../tex-inputs/latex-vorspann}


               \section[Hugo von Hofmannsthal an Arthur Schnitzler, 16. 4. {[}1914{]}]{ Hugo von Hofmannsthal an Arthur Schnitzler, 16. 4. {[}1914{]}}\nopagebreak\mylabel{v}\rehead{ }\normalsize\beginnumbering\briefempfaengerindex{Schnitzler, Arthur@\textsc{Schnitzler, Arthur}!zzzHofmannsthal, Hugo von@\emph{von Hugo von Hofmannsthal}!1914-04-162@{16. 4. {[}1914{]}}|(be} \toendnotes[C]{\smallbreak\pagebreak[2]} \Standort{CUL, Schnitzler, B 43.}
\physDesc{Briefkarte
\newline{}Handschrift: schwarze Tinte, deutsche Kurrent
\newline{}Schnitzler: mit Bleistift die Jahreszahl ergänzt: »914« und beschriftet: »Hofm« \newline{}Ordnung: 1) mit Bleistift von unbekannter Hand nummeriert: »\strikeout{336}« 2) mit Bleistift von unbekannter Hand nummeriert: »349«}\buchAbdrucke{\weitereDrucke{Hugo von Hofmannsthal, Arthur Schnitzler: \emph{Briefwechsel}. Hg. Therese Nickl und Heinrich Schnitzler. Frankfurt am Main: \emph{S. Fischer} 1964, S. 274–275.} }\toendnotes[C]{\smallbreak}\pstart
           \raggedleft{}{\pb}\textcolor{pink}{Rodaun}{}\ledrightnote{\textcolor{pink}{Rodaun}}{ }16 IV.\pend
           \pstart{}mein lieber Arthur \pend\pstart
           auch mir iſt das Notwendige, das Conſtante in allem Menſchlichen mit reifenden Jahren
               immer ſtärker vor Augen und in der Seele – und es war nichts anderes als was Sie
               bezeichnen: »leiſe Wehmut« – was mich hatte dieſe Zeilen vom \textcolor{pink}{Semmering}{}\ledrightnote{\textcolor{pink}{Semmering}}{ }ſchreiben laſſen.\hspace*{1.5em}Inzwiſchen war ich ein wenig in \textcolor{pink}{Nieder-}{}\ledrightnote{\textcolor{pink}{Niederösterreich}} und \textcolor{pink}{Oberoeſterreich}{}\ledrightnote{\textcolor{pink}{Oberösterreich}}, {\pb}per Auto, ganz im Flug: \textcolor{pink}{Amſtetten}{}\ledrightnote{\textcolor{pink}{Amstetten}} – \textcolor{pink}{Iſchl}{}\ledrightnote{\textcolor{pink}{Bad Ischl}} – \textcolor{pink}{Salzburg}{}\ledrightnote{\textcolor{pink}{Salzburg}} – dann zurück nach \textcolor{pink}{Wels}{}\ledrightnote{\textcolor{pink}{Wels}} – \textcolor{pink}{Enns}{}\ledrightnote{\textcolor{pink}{Enns}}, bei
                  \textcolor{pink}{\textsc{Wallsee}}{}\ledrightnote{\textcolor{pink}{Wallsee}} über die \textsc{Donau}, am nördlichen Ufer weiter, eine Nacht in \textcolor{pink}{\textsc{Dürnstein}}{}\ledrightnote{\textcolor{pink}{Dürnstein}}: dies alles, nächſte Landſchaft, wird mir immer ergreifender, immer
               abgrundtiefer – auch mein eigenes Verhältnis dazu, durch Blut und Nicht-Blut,
               Verbundenheit und Sehnſucht, Nah-ſein und Fern-ſein. Wenn dies ſo fortgeht, ſo muſs
               ja das Alter eine wehrhafte zitternde, leicht fiebernde Jugend ſein. – Wir erwarten in
               dieſen Tagen \textcolor{blue}{\textsc{Schroeder}}{}\ledrightnote{\textcolor{blue}{Rudolf Alexander Schröder}}; ko{\geminationm}t er nicht, was auch leicht möglich, ſo ſind
               wir in allernächſter Zeit \label{T_L02176_1v}\edtext{bei Euch. Von Herzen Ihr}{\lemma{\textnormal{\emph{bei Euch. Von Herzen Ihr}}}\Cendnote{\textnormal{weiter quer am linken
                  Rand}}}\label{T_L02176_1h}\pend
           \pstart \spacefill\mbox{Hugo.}\pend{}\endnumbering\briefempfaengerindex{Schnitzler, Arthur@\textsc{Schnitzler, Arthur}!zzzHofmannsthal, Hugo von@\emph{von Hugo von Hofmannsthal}!1914-04-162@{16. 4. {[}1914{]}}|)be}\mylabel{h}  \normalsize

\doendnotes{C}
\bigskip
\vfill

\clearpage

\footnotesize

\lohead{\textsc{register}}

% Definiere theindex-Environment komplett neu ohne reledmac
\makeatletter
\renewenvironment{theindex}{%
  \section*{\indexname}%
  \setlength{\parindent}{0pt}%
  \setlength{\parskip}{0pt plus 0.3pt}%
  \let\item\@idxitem
}{%
  \clearpage
}
\makeatother

\IfFileExists{\jobname-pw.ind}{\input{\jobname-pw.ind}}{}

\end{document}

      