%% latex-korrekturansicht-vorspann.tex
%% Vorspann für die Korrekturansicht.
%% Lädt die gemeinsame Datei latex-vorspann.tex mit gesetztem Schalter.

\newif\ifkorrekturansicht
\korrekturansichttrue

\input{../tex-inputs/latex-vorspann}


               \section[Hugo und Gerty von Hofmannsthal an Arthur Schnitzler, 31. 10. 1908]{ Hugo und Gerty von Hofmannsthal an Arthur Schnitzler,
               31. 10. 1908}\nopagebreak\mylabel{v}\rehead{ }\normalsize\beginnumbering\briefempfaengerindex{Schnitzler, Arthur@\textsc{Schnitzler, Arthur}!zzzHofmannsthal, Hugo von@\emph{von Hugo von Hofmannsthal}!1908-10-311@{31. 10. 1908}|(be} \toendnotes[C]{\smallbreak\pagebreak[2]} \Standort{CUL, Schnitzler, B 43.}
\physDesc{Brief, 1 Blatt, 1 Seite
\newline{}Schreibmaschine\newline{}Beilage: \textcolor{blue}{Camillo Müller}: eigenhändiger
                                 Brief, 2 Blätter, 7 Seiten, schwarze Tinte \newline{}Ordnung: 1) Die Abschrift dürfte nach dem Tod Hofmannsthals von seiner Witwe oder
                                 seiner \textcolor{blue}{Tochter}
                                 erstellt worden sein. Warum sie sich in Schnitzlers Nachlass befindet und wo das
                                 Original verblieben ist, bleibt ungeklärt 2) mit Bleistift von unbekannter Hand nummeriert: »\strikeout{296}«3) mit Bleistift von unbekannter Hand nummeriert:
                                    »303«4) mit Bleistift von unbekannter Hand
                                    nummeriert: »302«}\buchAbdrucke{\weitereDrucke{Hugo von Hofmannsthal, Arthur Schnitzler: \emph{Briefwechsel}. Hg. Therese Nickl und Heinrich Schnitzler. Frankfurt am Main: \emph{S. Fischer} 1964, S. 241–242.} }\toendnotes[C]{\smallbreak}\pstart
           \raggedleft{}{\pb}\textcolor{pink}{Rodaun}{}\ledrightnote{\textcolor{pink}{Rodaun}} d 31 X 08\pend
           \pstart{}Mein lieber Arthur,\pend\pstart
           wegen des Schreibers danke ich sehr aber ich möchte lieber ein Frauenzimmer von
               weiblichem Geschlecht. Um mir das nachzutragen, dürften Sie nicht der berüchtigte
               Erotiker sein! \pend
           \pstart
           Was den »\textcolor{brown}{Morgen}{}\ledrightnote{\textcolor{brown}{Morgen. Wochenschrift für deutsche Kultur}}{[}«{]} betrifft, so hänge ich mit diesem schönen Unternehmen
               ausschliesslich nur mehr durch einen Process zusammen, werde aber gern das nächste
               Mal bei Ihnen die \textcolor{green}{Gedichte}{}\ledrightnote{→\textcolor{green}{[Gedichte]}} von
                  \textcolor{blue}{Winterstein}{}\ledrightnote{\textcolor{blue}{Alfred von Winterstein}} anschauen, vielleicht kann man sie
               an \textcolor{blue}{Blei}{}\ledrightnote{\textcolor{blue}{Franz Blei}} für seine \textcolor{brown}{Zeitschrift}{}\ledrightnote{→\textcolor{brown}{Hyperion}} schicken oder sonst wo hin. Drittens bitte ich
               Sie recht herzlich den eingelegten Brief mir zuliebe durchzusehen und wenn Sie keinen
               Grund dagegen haben demgemäss dieses Fräulein \textcolor{blue}{Braun}{}\ledrightnote{\textcolor{blue}{Thekla Maria Braun}} vom \textcolor{brown}{Volkstheater}{}\ledrightnote{\textcolor{brown}{Volkstheater}}, das sich auch schon
               direct an Sie gewandt hat, bei sich zu empfangen. Denn ich sage mir dass es einem so
               anständigen Menschen wie Dr. \textcolor{blue}{Camillo Müller}{}\ledrightnote{\textcolor{blue}{Camillo Müller}}, der
               mich ausserdem nur sehr oberflächlich kennt, gewiss schwer gefallen ist so
               ausführlich deswegen an mich zu schreiben und vielleicht hängt für die arme Person
               wirklich unberechenbar viel daran, dass man ihr hilft. Und es ist ja sehr möglich,
               dass sich Herr \textcolor{blue}{Weisse}{}\ledrightnote{\textcolor{blue}{Adolf Weisse}} hier wieder einmal wie ein
               Schwein gegen jemanden benimmt etc.\pend
           \pstart
           Ich wurschtle mich weiter gegen das Ende meines vierten \textcolor{green}{Aktes}{}\ledrightnote{→\textcolor{green}{Der Rosenkavalier}} und bin \pend
           \pstart
           von Herzen Ihr{\\[\baselineskip]}\spacefill\mbox{Hugo.}\pend
           \leftskip=0em{}\pstart
           \noindent{}\label{K_L01796_1v}\edtext{Gruss von der Schreiberin }{\lemma{\textnormal{\emph{Gruss … Schreiberin}}}\Cendnote{\textnormal{Das dürfte so zu lesen sein, dass das
                     nicht überlieferte Original von \textcolor{blue}{Gerty von
                        Hofmannsthal} geschrieben worden war.}}}\label{K_L01796_1h}.\pend
           {\bigskip}\pstart
           \raggedleft{}{\pb}{[}hs. Müller:{]} \textcolor{pink}{Wien}{}\ledrightnote{\textcolor{pink}{Wien}}, 29. Okt. 1908.\pend
           \pstart{}\textsc{Sehr geehrter Herr!}\pend\pstart
           Nehmen Sie es mir, bitte, nicht übel, wenn ich Sie mit einem Anliegen beläſtige, das
               Ihnen etwas ſonderbar erſcheinen mag.\pend
           \pstart
           Sie ſind, ſoviel ich weiß, mit Hr. D\textsuperscript{r}{ }\textsc{Schnitzler} befreundet, den ich leider perſönlich nicht
               kenne. Wenigſtens habe ich Sie ſeinerzeit in Geſellſchaft des Hr. \textsc{Schnitzler} in \textcolor{pink}{\textsc{St. Gilgen}}{}\ledrightnote{\textcolor{pink}{St. Gilgen}} geſehen.\pend
           \pstart
           Nun ſoll demnächſt im \textcolor{brown}{Deutſchen Volkstheater}{}\ledrightnote{\textcolor{brown}{Volkstheater}}{ }\textsc{Schnitzler}’s »\textcolor{green}{\textsc{Liebelei}}{}\ledrightnote{\textcolor{green}{Liebelei. Schauspiel in drei Akten}}« zur Aufführung ge{\pb}langen,
               ſobald nur erſt die Beſetzung der Rolle der »\textcolor{green}{\textsc{\uline{Mizi Schlager}}}{}\ledrightnote{→\textcolor{green}{Liebelei. Schauspiel in drei Akten}}« feſtgeſetzt. Und hier iſt der Punkt, wo ich Ihre gütige Intervention in
               Anſpruch nehmen will.\pend
           \pstart
           Für dieſe Rolle war nämlich urſprünglich ein Frl. \textcolor{blue}{Thekla \textsc{Braun}}{}\ledrightnote{\textcolor{blue}{Thekla Maria Braun}} in Ausſicht geno{\geminationm}en, die erſt ſeit Beginn dieſer
                  \textsc{Saison} dem \textcolor{brown}{Volkstheater}{}\ledrightnote{\textcolor{brown}{Volkstheater}}
               angehört. Frl. \textcolor{blue}{\textsc{Braun}}{}\ledrightnote{\textcolor{blue}{Thekla Maria Braun}} war früher beim \textcolor{brown}{Opernballet}{}\ledrightnote{\textcolor{brown}{Opernballett}}, dann zwei Jahre
               in \textcolor{pink}{Graz}{}\ledrightnote{\textcolor{pink}{Graz}} als Schauſpielerin – und hier eben ſah ſie
               Dir. \textcolor{blue}{\textsc{Weisse}}{}\ledrightnote{\textcolor{blue}{Adolf Weisse}} in der Rolle der »\textcolor{green}{\textsc{Schlager Mizi}}{}\ledrightnote{→\textcolor{green}{Liebelei. Schauspiel in drei Akten}}« u. engagierte ſie vom Fleck weg fürs \textcolor{brown}{Deutſche
                  Volkstheater}{}\ledrightnote{\textcolor{brown}{Volkstheater}}. Er verſicherte ſie, daſs er die »\textcolor{green}{\textsc{Liebelei}}{}\ledrightnote{\textcolor{green}{Liebelei. Schauspiel in drei Akten}}« fürs \textcolor{brown}{Volkstheater}{}\ledrightnote{\textcolor{brown}{Volkstheater}}{ }{\pb}mit Hilfe des Autors – das Stück
               gehörte dem \textcolor{brown}{Burgtheater}{}\ledrightnote{\textcolor{brown}{Burgtheater}} – freimachen werde, denn er
               könne das Stück speziell in der Rolle der »\textcolor{green}{\textsc{Schlager}}{}\ledrightnote{→\textcolor{green}{Liebelei. Schauspiel in drei Akten}}« beſſer beſetzen als Dir. \textcolor{blue}{\textsc{Schlenther}}{}\ledrightnote{\textcolor{blue}{Paul Schlenther}} u. dgl. m. Da Frl. \textcolor{blue}{\textsc{Braun}}{}\ledrightnote{\textcolor{blue}{Thekla Maria Braun}}, die ich ſeit 10 Jahren kenne – ſie war damals ein 15jähriger Backfiſch u. kam
               in die Tanzſtunden zu \textcolor{blue}{\textsc{Hassreiter}}{}\ledrightnote{\textcolor{blue}{Josef Hassreiter}}, die ich alter Esel beſuchte – auf meinen Rat das Engagement am \textcolor{brown}{Volkstheater}{}\ledrightnote{\textcolor{brown}{Volkstheater}} angeno{\geminationm}en hat, obwohl
               ſie verlockendere Anträge anderer \textcolor{pink}{W\textsuperscript{r}}{}\ledrightnote{\textcolor{pink}{Wien}} Bühnen beſaß, ſo bin ein bischen engagiert in dieſer Sache u. möchte \substVorne{}\textsuperscript{ihr}\substDazwischen{}ſie\substHinten{} nun in ihrer Leidenbahn {\pb}– das war nämlich bis nun ihr \textsc{Engagement} – nicht ganz im
               Stiche laſſen. Frl. \textcolor{blue}{\textsc{Braun}}{}\ledrightnote{\textcolor{blue}{Thekla Maria Braun}}, die für erſte Rollen mit einer \textsc{Anfangsgage} von
               5000 K engagiert worden war, kam vorläufig zu keiner einzigen. Meist ſtand ihr Frau
                  \textcolor{blue}{\textsc{Glöckner}}{}\ledrightnote{\textcolor{blue}{Josefine Glöckner}} im Wege. Nun würde ſie i{\geminationm}er wieder auf die »\textcolor{green}{\textsc{Liebelei}}{}\ledrightnote{\textcolor{green}{Liebelei. Schauspiel in drei Akten}}« vertröſtet, die ja noch in \label{K_L01796_2v}\edtext{dieſem Jahre erſcheinen}{\lemma{\textnormal{\emph{dieſem Jahre erſcheinen}}}\Cendnote{\textnormal{Die Aufführung
                  verzögerte sich bis 5. 1. 1909. \textcolor{blue}{Thekla Braun} wurde
                  nicht eingesetzt, die zweite weibliche Hauptrolle spielte \textcolor{blue}{Charlotte Waldow}.}}}\label{K_L01796_2h}, und in der ſie »ſich machen werde.«
               Siehe da – die »\textcolor{green}{\textsc{Liebelei}}{}\ledrightnote{\textcolor{green}{Liebelei. Schauspiel in drei Akten}}« kam, aber Frl. \textcolor{blue}{Braun}{}\ledrightnote{\textcolor{blue}{Thekla Maria Braun}} ſoll die \uline{Rolle nicht ſpielen}. \uline{Wer} ſie ſpielen wird, ſteht allerdings noch nicht feſt, u. es ſcheint die
               Beſetzung einige Schwierigkeiten {\pb}zu machen, ſofern man der nageliegendſten, der mit Frl. \textsc{\textcolor{blue}{Braun}{}\ledrightnote{\textcolor{blue}{Thekla Maria Braun}}} gefliſſentlich aus dem Wege geht. Frl. \textcolor{blue}{\textsc{Braun}}{}\ledrightnote{\textcolor{blue}{Thekla Maria Braun}} hat daher an Hr. D\textsuperscript{r}{ }\textsc{Schnitzler} die ſchriftliche Bitte gerichtet, ihr zu
               geſtatten, daſs ſie ihm die Rolle der der »\textcolor{green}{\textsc{Mizi Schlager}}{}\ledrightnote{→\textcolor{green}{Liebelei. Schauspiel in drei Akten}}« vorſpreche, damit ſich der Autor ſelbſt, der gewiſs das eminenteſte Intereſſe
               an einer richtigen Beſetzung hat, ein entſprechendes Urteil über die Fähigkeiten des
               Fräuleins bilden kann.\pend
           \pstart
           Ich möchte nun meinerſeits an Sie, verehrter Herr, die ergebenſte Bitte richten, das
                  {\pb}Anſuchen des Frl. \textcolor{blue}{\textsc{Braun}}{}\ledrightnote{\textcolor{blue}{Thekla Maria Braun}} bei Herrn D\textsuperscript{r}{ }\textsc{Schnitzler} auf meine Empfehlung hin zu befürworten. Die
               Direktion hat ja dann noch immer freie Hand, und es iſt wenigſtens alles geſchehen,
               um einem allfälligen Miſsgriff vorzubeugen u. auch ein ſtarkes, ſtrebſsames Talent
               vor unverdienter Kränkung zu ſchützen.\pend
           \pstart
           Falls Sie dem Fräulein \textcolor{blue}{\textsc{Braun}}{}\ledrightnote{\textcolor{blue}{Thekla Maria Braun}} geſtatten wollten, Sie zu beſuchen, ſo bitte ich um zeitige Bekanntgabe von Tag
               und Stunde, die Ihnen {\pb}genehm
               wären. Jedesfalls wiederhole ich aber meine Bitte um Befürwortung jenes Erſuchens,
               des Frl. \textcolor{blue}{\textsc{Braun}}{}\ledrightnote{\textcolor{blue}{Thekla Maria Braun}} an D\textsuperscript{r} \textsc{Schnitzler}
               richtete. –\pend
           \pstart
           Und zum Schluſſe bitte ich nochmals, mir dieſe langweilige, Sie wohl empflindlich
               ſtörende Epiſtel zu verzeihen – ich komm gewiſs kein zweitesmal!\pend
           \pstart
           In aufrichtiger Verehrung{\\[\baselineskip]}Ihr{\\[\baselineskip]}\spacefill\mbox{Camillo Müller.}\pend
           \leftskip=0em{}\pstart
           \noindent{}\textcolor{pink}{I. Wipplingerstraſſe 33}{}\ledrightnote{\textcolor{pink}{Wipplingerstraße}}, T. 14048.\pend
           \pstart
           \label{T_L01796_1v}\edtext{Bitte der gnädigen \textcolor{blue}{Frau}{}\ledrightnote{→\textcolor{blue}{Gertrude von Hofmannsthal}} meine Handküſſe zu übermitteln!
                     W. O.}{\lemma{\textnormal{\emph{Bitte … W. O.}}}\Cendnote{\textnormal{in drei Zeilen seitlich zu
                     Schlussformel, Unterschrift und Adresse}}}\label{T_L01796_1h}\pend
           \endnumbering\briefempfaengerindex{Schnitzler, Arthur@\textsc{Schnitzler, Arthur}!zzzHofmannsthal, Hugo von@\emph{von Hugo von Hofmannsthal}!1908-10-311@{31. 10. 1908}|)be}\mylabel{h}  \normalsize

\doendnotes{C}
\bigskip
\vfill

\clearpage

\footnotesize

\lohead{\textsc{register}}

% Definiere theindex-Environment komplett neu ohne reledmac
\makeatletter
\renewenvironment{theindex}{%
  \section*{\indexname}%
  \setlength{\parindent}{0pt}%
  \setlength{\parskip}{0pt plus 0.3pt}%
  \let\item\@idxitem
}{%
  \clearpage
}
\makeatother

\IfFileExists{\jobname-pw.ind}{\input{\jobname-pw.ind}}{}

\end{document}

      