%% latex-korrekturansicht-vorspann.tex
%% Vorspann für die Korrekturansicht.
%% Lädt die gemeinsame Datei latex-vorspann.tex mit gesetztem Schalter.

\newif\ifkorrekturansicht
\korrekturansichttrue

\input{../tex-inputs/latex-vorspann}


               \section[Leo Van-Jung, Fanny Mütter, Richard Beer-Hofmann an Arthur und Olga Schnitzler, {[}24. 9.?{]} 1905]{ Leo Van-Jung, Fanny Mütter, Richard Beer-Hofmann an Arthur und Olga
               Schnitzler, {[}24. 9.?{]} 1905}\nopagebreak\mylabel{v}\rehead{ }\normalsize\beginnumbering\briefempfaengerindex{Schnitzler, Olga@\textsc{Schnitzler, Olga}!zzzBeer-Hofmann, Richard@\emph{von Richard Beer-Hofmann}!1905-09-241@{{[}24. 9.?{]} 1905}|(be}\briefempfaengerindex{Schnitzler, Olga@\textsc{Schnitzler, Olga}!zzzMuetter, Franziska@\emph{von Franziska Mütter}!1905-09-241@{{[}24. 9.?{]} 1905}|(be}\briefempfaengerindex{Schnitzler, Olga@\textsc{Schnitzler, Olga}!zzzVan-Jung, Leo@\emph{von Leo Van-Jung}!1905-09-241@{{[}24. 9.?{]} 1905}|(be}\briefempfaengerindex{Schnitzler, Arthur@\textsc{Schnitzler, Arthur}!zzzBeer-Hofmann, Richard@\emph{von Richard Beer-Hofmann}!1905-09-241@{{[}24. 9.?{]} 1905}|(be}\briefempfaengerindex{Schnitzler, Arthur@\textsc{Schnitzler, Arthur}!zzzMuetter, Franziska@\emph{von Franziska Mütter}!1905-09-241@{{[}24. 9.?{]} 1905}|(be}\briefempfaengerindex{Schnitzler, Arthur@\textsc{Schnitzler, Arthur}!zzzVan-Jung, Leo@\emph{von Leo Van-Jung}!1905-09-241@{{[}24. 9.?{]} 1905}|(be} \toendnotes[C]{\smallbreak\pagebreak[2]} \Standort{DLA, A:Schnitzler, 85.1.4821.}
\physDesc{Bildpostkarte
\newline{}Handschrift Leo Van-Jung: Bleistift, lateinische Kurrent\newline{}Handschrift Franziska Mütter: Bleistift, deutsche Kurrent\newline{}Handschrift Richard Beer-Hofmann: Bleistift, lateinische Kurrent\newline{}Versand: Stempel: »\nobreak{}\oindex{Lido@\textbf{Lido}, \emph{Teil eines besiedelten Ortes (A.BSOX)}|pwk}S. Elisabett{[}a di Lido
                                          (Venezia){]}, 2\textcolor{gray}{4} {[}9{]} 05\nobreak{}«.  }\toendnotes[C]{\smallbreak}\pstart{}{\pb}Herrn D\textsuperscript{r} Arthur
                  Schnitzler\pend{}\pstart{}\textcolor{pink}{Wien XVIII. Spöttelgasse 7}{}\ledrightnote{\textcolor{pink}{Edmund-Weiß-Gasse}}.\pend{}\pstart{}\textcolor{pink}{Austria}{}\ledrightnote{\textcolor{pink}{Österreich}}\pend{}\pstart{}\textcolor{pink}{Vienna}{}\ledrightnote{\textcolor{pink}{Wien}}\pend{}{\bigskip}\pstart
           \noindent{}\centering{}{\pb}\textcolor{gray}{\textbf{\textcolor{pink}{VENEZIA}{}\ledrightnote{\textcolor{pink}{Venedig}} – \textcolor{brown}{Accademia di Belle Arti}{}\ledrightnote{\textcolor{brown}{Accademia di belle arti di Venezia}} – \textcolor{green}{La Presentazione della Vergine}{}\ledrightnote{\textcolor{green}{La presentazione della Vergine al Tempio}} – \textcolor{blue}{Tiziano}{}\ledrightnote{\textcolor{blue}{Tizian}}}}\pend
           \pstart
           Lieber Arthur, \uline{erst} heute schreib ich Ihnen, aber nicht weil ich an
               Sie vergessen habe, sondern weil ich mich freue Sie bald wieder zu sehen und von den
                  »\label{K_L02579-1v}\edtext{\textcolor{green}{Sünderinnen}{}\ledrightnote{→\textcolor{green}{Zwischenspiel. Komödie in drei Akten}{\newline}→\textcolor{green}{Der Ruf des Lebens. Schauspiel in drei Akten}}}{\lemma{\textnormal{\emph{Sünderinnen}}}\Cendnote{\textnormal{Es dürfte sich um eine gemeinsame
                  Bezeichnung für die zwei Stücke \emph{\textcolor{green}{Zwischenspiel}}
                  und \emph{\textcolor{green}{Der Ruf des Lebens}} handeln, die, noch ohne
                  finalen Titel weitgehend fertig gestellt waren, was in mehreren Zeitungen gemeldet
                  worden war. \textcolor{blue}{Van-Jung} kannte sie beide, da \textcolor{blue}{Schnitzler} sie ihm am 12. 8. 1905 vorgelesen
                  hatte.}}}\label{K_L02579-1h}« zu hören. Einige Zeitungsnotizen haben mich sehr neugierig gemacht.
               Handkuss der lieben Frau Olga und die allerherzlichsten Grüsse Ihnen von Ihrem\pend
           \pstart \spacefill\mbox{Leo.}\pend{}\pstart
           \noindent{}{[}hs. Mütter:{]} Lieber Dr. und liebſte Olga! Ich bleibe noch einige Tage hier und
               werde den lieben Brief Olga’s morgen beantworten. Für heute tauſend Grüße von Ihrer
               alten \spacefill\mbox{Fanny Mütter}\pend
           \pstart
           \noindent{}{[}hs. Beer-Hofmann:{]} Lieber Arthur! Wir sind – hoffe ich \label{K_L02579-2v}\edtext{Mittwoch oder Donnerstag}{\lemma{\textnormal{\emph{Mittwoch oder Donnerstag}}}\Cendnote{\textnormal{Der Poststempel dieser Karte ist nur bei
                  der Jahresangabe verlässlich zu entziffern. Eine grobe Einordnung lässt sich mit
                     \textcolor{blue}{Beer-Hofmann}s Zusammentstellung seiner
                  Lebensdaten treffen: »Ende August, über \textcolor{pink}{Bozen} an den \textcolor{pink}{Lido} (\textcolor{pink}{Hôtel des Bains}), \textcolor{blue}{Bella
                        Vengerova}, \textcolor{blue}{Arthur Kaufmann}, \textcolor{blue}{Leo Van Jung} kommen nach.« Die
                  Tagesangabe des Poststempels ist zweistellig und beginnt mit einer »2«, so dass
                  die Karte Ende August oder Ende September anzusiedeln
                  ist. Letzteres wiederum ist wahrscheinlicher, da es bis zum [7. 10. 1905] zu keinem gemeinsamen Treffen kam.}}}\label{K_L02579-2h}
               in \textcolor{pink}{Rodaun}{}\ledrightnote{\textcolor{pink}{Rodaun}}\pend
           \pstart
           Freuen uns Sie bald zu sehen.\pend
           \pstart Ihr \spacefill\mbox{Richard}\pend{}\endnumbering\briefempfaengerindex{Schnitzler, Olga@\textsc{Schnitzler, Olga}!zzzBeer-Hofmann, Richard@\emph{von Richard Beer-Hofmann}!1905-09-241@{{[}24. 9.?{]} 1905}|)be}\briefempfaengerindex{Schnitzler, Olga@\textsc{Schnitzler, Olga}!zzzMuetter, Franziska@\emph{von Franziska Mütter}!1905-09-241@{{[}24. 9.?{]} 1905}|)be}\briefempfaengerindex{Schnitzler, Olga@\textsc{Schnitzler, Olga}!zzzVan-Jung, Leo@\emph{von Leo Van-Jung}!1905-09-241@{{[}24. 9.?{]} 1905}|)be}\briefempfaengerindex{Schnitzler, Arthur@\textsc{Schnitzler, Arthur}!zzzBeer-Hofmann, Richard@\emph{von Richard Beer-Hofmann}!1905-09-241@{{[}24. 9.?{]} 1905}|)be}\briefempfaengerindex{Schnitzler, Arthur@\textsc{Schnitzler, Arthur}!zzzMuetter, Franziska@\emph{von Franziska Mütter}!1905-09-241@{{[}24. 9.?{]} 1905}|)be}\briefempfaengerindex{Schnitzler, Arthur@\textsc{Schnitzler, Arthur}!zzzVan-Jung, Leo@\emph{von Leo Van-Jung}!1905-09-241@{{[}24. 9.?{]} 1905}|)be}\mylabel{h}  \normalsize

\doendnotes{C}
\bigskip
\vfill

\clearpage

\footnotesize

\lohead{\textsc{register}}

% Definiere theindex-Environment komplett neu ohne reledmac
\makeatletter
\renewenvironment{theindex}{%
  \section*{\indexname}%
  \setlength{\parindent}{0pt}%
  \setlength{\parskip}{0pt plus 0.3pt}%
  \let\item\@idxitem
}{%
  \clearpage
}
\makeatother

\IfFileExists{\jobname-pw.ind}{\input{\jobname-pw.ind}}{}

\end{document}

      