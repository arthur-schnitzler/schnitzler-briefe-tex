%% latex-korrekturansicht-vorspann.tex
%% Vorspann für die Korrekturansicht.
%% Lädt die gemeinsame Datei latex-vorspann.tex mit gesetztem Schalter.

\newif\ifkorrekturansicht
\korrekturansichttrue

\input{../tex-inputs/latex-vorspann}


               \section[Wilhelm Bölsche an Arthur Schnitzler, 24. 3. 1892]{ Wilhelm Bölsche an Arthur Schnitzler, 24. 3. 1892}\nopagebreak\mylabel{v}\rehead{ }\normalsize\beginnumbering\briefempfaengerindex{Schnitzler, Arthur@\textsc{Schnitzler, Arthur}!zzzBoelsche, Wilhelm@\emph{von Wilhelm Bölsche}!1892-03-241@{24. 3. 1892}|(be} \toendnotes[C]{\smallbreak\pagebreak[2]} \Standort{DLA, A:Schnitzler, HS.NZ85.1.2577,4.}
\physDesc{Brief, 1 Blatt, 2 Seiten
\newline{}Handschrift: schwarze Tinte, deutsche Kurrent\newline{}Ordnung: mit rotem Buntstift von unbekannter Hand nummeriert: »5« }\buchAbdrucke{\weitereDrucke{Wilhelm Bölsche: \emph{Briefwechsel. Mit Autoren der Freien Bühne}. Hg. Gerd-Hermann Susen. Berlin: \emph{Weidler} 2010, S. 677 (Werke und Briefe. Wissenschaftliche Ausgabe, Briefe I).} }\toendnotes[C]{\smallbreak}\pstart
           \raggedleft{}{\pb}\textcolor{pink}{Friedrichshagen}{}\ledrightnote{\textcolor{pink}{Friedrichshagen}}{\\}24. III. 92. \pend
           \pstart\center{}Hochgeehrter Herr Doktor!\pend\pstart
           Verzeihen Sie, daß ich noch nicht geantwortet. Aber die Arbeitslaſt iſt für mich
                    enorm in dieſen Momenten des \label{K_L00087_1v}\edtext{Neubaus!}{\lemma{\textnormal{\emph{Neubaus!}}}\Cendnote{\textnormal{Seit 1892 erschien die \emph{\textcolor{green}{Freie Bühne}} nicht mehr als Wochen-,
                        sondern als Monatsschrift.}}}\label{K_L00087_1h}\pend
           \pstart
           Ihre »\textcolor{green}{Elixire}{}\ledrightnote{\textcolor{green}{Die drei Elixire}}« bringe ich, ſobald es ſich
                    machen läßt. Offen geſtanden, ſind ſie mir nicht ſo lieb wie die erſte \textcolor{green}{Novelle}{}\ledrightnote{→\textcolor{green}{Hermann Bahrs Querulant}}, ſie ſind lange
                    nicht ſo aktuell. Aber ſie kommen doch!\pend
           \pstart
           Mit den Gedichten iſt’s eine böſe Sache. Ich habe jetzt ein \textcolor{green}{\textcolor{blue}{Lilienkron}{}\ledrightnote{\textcolor{blue}{Detlev von Liliencron}}’ſches}{}\ledrightnote{→\textcolor{green}{Der Kartäusermönch}} probeweiſe einmal
                    in’s nächſte Heft geſtreut {\pb}aber ich denke mir, es
                    wird doch nur ſelten ſich auch nach dieſer Seite hin grade die »\textcolor{green}{Freie Bühne}{}\ledrightnote{\textcolor{green}{Freie Bühne für den Entwickelungskampf der Zeit}}« ausbauen laſſen. Lyriſche Zeitſchriften gibt’s
                    ja genug, unſer Schwerpunkt muß unbedingt anderswo liegen. Wollen Sie’s indeſſen
                    wagen, ſo ſenden Sie mir etwas, das Obige ſoll keine prinzipielle Ablehnung
                    ſein!\pend
           \pstart
           Mit beſtem Gruß{\\[\baselineskip]}Ihr{\\[\baselineskip]}\spacefill\mbox{Wilhelm Bölsche}\pend
           \leftskip=0em{}\endnumbering\briefempfaengerindex{Schnitzler, Arthur@\textsc{Schnitzler, Arthur}!zzzBoelsche, Wilhelm@\emph{von Wilhelm Bölsche}!1892-03-241@{24. 3. 1892}|)be}\mylabel{h}  \normalsize

\doendnotes{C}
\bigskip
\vfill

\clearpage

\footnotesize

\lohead{\textsc{register}}

% Definiere theindex-Environment komplett neu ohne reledmac
\makeatletter
\renewenvironment{theindex}{%
  \section*{\indexname}%
  \setlength{\parindent}{0pt}%
  \setlength{\parskip}{0pt plus 0.3pt}%
  \let\item\@idxitem
}{%
  \clearpage
}
\makeatother

\IfFileExists{\jobname-pw.ind}{\input{\jobname-pw.ind}}{}

\end{document}

      