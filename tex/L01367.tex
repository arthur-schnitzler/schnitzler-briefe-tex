%% latex-korrekturansicht-vorspann.tex
%% Vorspann für die Korrekturansicht.
%% Lädt die gemeinsame Datei latex-vorspann.tex mit gesetztem Schalter.

\newif\ifkorrekturansicht
\korrekturansichttrue

\input{../tex-inputs/latex-vorspann}


               \section[Hermann Bahr an Arthur Schnitzler, 29. 1. 1904]{ Hermann Bahr an Arthur Schnitzler, 29. 1. 1904}\nopagebreak\mylabel{v}\rehead{ }\normalsize\beginnumbering\briefempfaengerindex{Schnitzler, Arthur@\textsc{Schnitzler, Arthur}!zzzBahr, Hermann@\emph{von Hermann Bahr}!1904-01-291@{29. 1. 1904}|(be} \toendnotes[C]{\smallbreak\pagebreak[2]} \Standort{CUL, Schnitzler, B 5b.}
\physDesc{Brief, 1 Blatt, 2 Seiten
\newline{}Handschrift: schwarze Tinte, deutsche Kurrent\newline{}Ordnung: mit Bleistift von unbekannter Hand nummeriert: »108« }\buchAbdrucke{\weitereDrucke{Hermann Bahr, Arthur Schnitzler: \emph{Briefwechsel, Aufzeichnungen, Dokumente (1891–1931)}. Hg. Kurt Ifkovits und Martin Anton Müller. Göttingen: \emph{Wallstein} 2018, S. 292–293.} }\toendnotes[C]{\smallbreak}\pstart
           \raggedleft{}{\pb}29. 1. 04\pend
           \pstart\center{}Lieber Arthur!\pend\pstart
           Ich »ſoll« nach \textcolor{blue}{Ortner}{}\ledrightnote{\textcolor{blue}{Norbert von Ortner-Rodenstätt}} zwei bis drei Monate hier
               bleiben, glaube aber nicht es ſo lang auszuhalten. Es iſt hier ſehr unangenehm und
               ich überlege hin und her, \damage{ob} es nicht viel geſcheiter wäre, in \textcolor{pink}{Taormina}{}\ledrightnote{\textcolor{pink}{Taormina}}
               oder \textcolor{pink}{Kairo}{}\ledrightnote{\textcolor{pink}{Kairo}}{ }z\damage{u}{ }ſitzen. Ich tue übrigens nichts, ohne vorher \textcolor{blue}{Ortner}{}\ledrightnote{\textcolor{blue}{Norbert von Ortner-Rodenstätt}} zu ſchreiben.\pend
           \pstart
           Der »\textcolor{green}{einſame Weg}{}\ledrightnote{\textcolor{green}{Der einsame Weg. Schauspiel in fünf Akten}}« kam geſtern an und wurde sogleich
               geleſen. Wunderbar finde ich, wie Du da von der Peripherie der Menſchheit, an welcher
               ſich die meiſten Stücke ſonſt herumbewegen, in die Mitte ihres geiſtigen Lebens
               kommſt, nemlich unſeres geiſtigen Lebens, der Sachen, um die wir uns heute allein
               noch kümmern können. (Wobei ich mich an einen Satz \textcolor{blue}{Maeterlincks}{}\ledrightnote{\textcolor{blue}{Maurice Maeterlinck}} von dem \label{K_L01367_1v}\edtext{ſtill an
               ſeinem Tiſche ſitzenden Greiſe}{\lemma{\textnormal{\emph{ſtill … Greiſe}}}\Cendnote{\textnormal{In \emph{\textcolor{green}{À propos de Solness le Constructeur}} (\emph{\textcolor{brown}{Le Figaro}}, Jg. 40, Ser. 3, Nr. 92,
                        2. 4. 1894, S. 1, späterer Titel
                        \emph{\textcolor{green}{Le Tragique quotidien}}) schreibt \textcolor{blue}{Maeterlinck} über das
                  »tiefere Leben« eines Alten, der in seinem Stuhl versucht, seine Umgebung zu
                  begreifen, im Vergleich beispielsweise zu einem Liebhaber, der die Geliebte
                  erwürgt.}}}\label{K_L01367_1h} und an manches erinnere, was in meinem \textcolor{green}{Dialog vom Tragiſchen}{}\ledrightnote{\textcolor{green}{Dialog vom Tragischen}} gefordert wird). Allerdings {\pb}hat mir geſtern, beim erſten eiligen Leſen und in
               meiner jetzigen geiſtigen Trübung, im dramatiſchen Ductus etwas gefehlt, ich kann es
               nicht anders sagen, als daß mir die Bewegung des Stückes einige Male abzubrechen und
               ſich dann auf eine mir nicht gleich verſtändliche Art wieder zu ſammeln oder zu
               erſetzen ſchien. Ich leſe es nun aber in ein paar Tagen wieder und mit dieſen
               Bemerkungen iſt wol überhaupt mehr mein elender Zuſtand als das Stück kritiſiert.\pend
           \pstart
           Grüß \textcolor{blue}{Brahm}{}\ledrightnote{\textcolor{blue}{Otto Brahm}} und wen ich ſonſt in \textcolor{pink}{Berlin}{}\ledrightnote{\textcolor{pink}{Berlin}} kenne, empfiel mich Deiner \textcolor{blue}{Frau}{}\ledrightnote{→\textcolor{blue}{Olga Schnitzler}} und ſei herzlichſt gegrüßt
               von{\\[\baselineskip]}Deinem alten{\\[\baselineskip]}\spacefill\mbox{Hermann}\pend
           \leftskip=0em{}\endnumbering\briefempfaengerindex{Schnitzler, Arthur@\textsc{Schnitzler, Arthur}!zzzBahr, Hermann@\emph{von Hermann Bahr}!1904-01-291@{29. 1. 1904}|)be}\mylabel{h}  \normalsize

\doendnotes{C}
\bigskip
\vfill

\clearpage

\footnotesize

\lohead{\textsc{register}}

% Definiere theindex-Environment komplett neu ohne reledmac
\makeatletter
\renewenvironment{theindex}{%
  \section*{\indexname}%
  \setlength{\parindent}{0pt}%
  \setlength{\parskip}{0pt plus 0.3pt}%
  \let\item\@idxitem
}{%
  \clearpage
}
\makeatother

\IfFileExists{\jobname-pw.ind}{\input{\jobname-pw.ind}}{}

\end{document}

      