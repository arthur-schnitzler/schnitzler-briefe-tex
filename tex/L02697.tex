%% latex-korrekturansicht-vorspann.tex
%% Vorspann für die Korrekturansicht.
%% Lädt die gemeinsame Datei latex-vorspann.tex mit gesetztem Schalter.

\newif\ifkorrekturansicht
\korrekturansichttrue

\input{../tex-inputs/latex-vorspann}


               \section[Paul Goldmann an Arthur Schnitzler, 23. 4. {[}1892{]}]{ Paul Goldmann an Arthur Schnitzler, 23. 4. {[}1892{]}}\nopagebreak\mylabel{v}\rehead{ }\normalsize\beginnumbering\briefempfaengerindex{Schnitzler, Arthur@\textsc{Schnitzler, Arthur}!zzzGoldmann, Paul@\emph{von Paul Goldmann}!1892-04-231@{23. 4. {[}1892{]}}|(be} \toendnotes[C]{\smallbreak\pagebreak[2]} \Standort{DLA, A:Schnitzler, HS.NZ85.1.3163.}
\physDesc{Brief, 1 Blatt, 4 Seiten
\newline{}Handschrift: schwarze Tinte, deutsche Kurrent
\newline{}Schnitzler: mit Bleistift zwei Mal das Jahr »92« vermerkt }\toendnotes[C]{\smallbreak}\pstart
           \noindent{}{\pb}\textcolor{gray}{\textbf{\textcolor{brown}{Frankfurter Zeitung}{}\ledrightnote{\textcolor{brown}{Frankfurter Zeitung}}.}}\pend
           \pstart
           \textcolor{gray}{\textbf{(\textcolor{brown}{Gazette de Francfort}{}\ledrightnote{\textcolor{brown}{Frankfurter Zeitung}}.)}}\pend
           \pstart
           \textcolor{gray}{\textbf{\begin{otherlanguage}{french}Directeur\end{otherlanguage}: \textbf{M. \textcolor{blue}{L. Sonnemann}{}\ledrightnote{\textcolor{blue}{Leopold Sonnemann}}}.}}\pend
           \pstart
           \textcolor{gray}{\textbf{\begin{otherlanguage}{french}Journal politique, financier,\end{otherlanguage}}}\pend
           \pstart
           \textcolor{gray}{\textbf{\begin{otherlanguage}{french}commercial et litteraire.\end{otherlanguage}}}\pend
           \pstart
           \textcolor{gray}{\textbf{\begin{otherlanguage}{french}\textbf{Paraissant trois fois par jour}\end{otherlanguage}}}\hfill \textsc{\textcolor{pink}{Paris}{}\ledrightnote{\textcolor{pink}{Paris}}}, 23. April.\pend
           \pstart
           \textcolor{gray}{\textbf{–}}\pend
           \pstart
           \textcolor{gray}{\textbf{\begin{otherlanguage}{french}\textbf{Bureaux à \textcolor{pink}{Paris}{}\ledrightnote{\textcolor{pink}{Paris}}:}\end{otherlanguage}}}\pend
           \pstart
           \textcolor{gray}{\textbf{\begin{otherlanguage}{french}\textbf{\textcolor{pink}{rue Richelieu 75}{}\ledrightnote{\textcolor{pink}{rue Richelieu}}.}\end{otherlanguage}}}\pend
           \pstart
           \centering{}Mein lieber \textsc{Arthur}!\pend
           \pstart
           \noindent{}Ich ſehe, es geht nicht. Seit Wochen und Wochen warte ich, um zwei freie Stunden zu
               haben für den Brief an Dich. Denn ich mag Dir nicht ſchreiben, vierzig Zeilen
               flüchtig hingeſchmiert, wie man aller Welt ſchreibt. Und es geht nicht, die freien
               Stunden wollen nicht kommen. Seit ich meinen Dienſt angetreten hocke ich im Büreau
               von 8 Uhr früh bis 8 Uhr Abends, den Sonntag inbegriffen. Draußen und rings um mich
               iſt \textsc{\textcolor{pink}{Paris}{}\ledrightnote{\textcolor{pink}{Paris}}}. Ich bin einſam, elend, zerdrückt, verekelt, lebensunluſtig und kämpfe den
               ſchweren Kampf, in dem es keinen Sieg gibt und in dem der einzige Erfolg darin
               beſteht, die {\pb}unabwendbare Niederlage um ein paar
               Jahre länger hinauszuſchieben. Ich will Dir das Alles im Einzelnen erzählen und
               begründen. Ich habe Dir eigentlich ſchon hundert Mal geſchrieben, nur nicht mit Tinte
               auf Papier. Ich denke mit unſäglichem Heimweh an Dich zurück. Und jeder Deiner lieben
               Briefe, all’ Deine lieben treuen Worte, haben mich innig erfreut und mir ſo
               wohlgethan, wie Du es Dir nicht denken kannſt. Ich nehme heut nur die Feder zur Hand, weil ich es unmöglich länger aufſchieben
               kann, Dir zu danken. Ich glaube zwar nicht, daß zwiſchen uns Mißverſtändniſſe möglich
               ſind; aber die Entfernung iſt eine ſolche Fälſcherin! Und ſo ſchreibe ich Dir heut, nur um \strikeout{d}
               auszudrücken, daß ich Dir ſeit Langem, ich kann ruhig ſagen täglich {\pb}ſchreiben will, und daß ich Dir doch nächſtens, bald
               ſchreiben werde – trotz Allem{\dotsfive}\pend
           \pstart
           Nur das \label{K_L02697-v}\edtext{Gedicht}{\lemma{\textnormal{\emph{Gedicht}}}\Cendnote{\textnormal{Es dürfte sich um \emph{\textcolor{green}{Anfang vom Ende}} handeln, das am 3. 3. 1892 beim
                  Vereinsabend des \emph{\textcolor{brown}{Vereins für modernes Leben}} v
                  und das gedruckt am 15. 7. 1892 in der \emph{\textcolor{green}{Deutschen Dichtung}} erschien (\emph{\textcolor{green}{Deutsche Dichtung}}, Bd. 12, Nr. 8,
                        15. 7. 1892, S. 192).}}}\label{K_L02697-h} ſoll gleich hier hinein.
               Tauſend Dank dafür. Ich verſtehe. Mir iſt ſo, als ſtündeſt Du von einem Steine auf,
               auf dem Du unterwegs geruht, und begünneſt nun rüſtig nach oben zu ſteigen. Aber auf
               der andern Seite geht auch ein \strikeout{Leid} Leid aus Deinem
               Leben weg. Und ich war mit dieſem Leid befreundet. Das Glück, oder die Kunſt, die an
               deſſen Stelle treten, kennen mich nicht. Bedenken eines unheilbaren
               Selbſtſüchtlers.\pend
           \pstart
           Die Verſe – deliciös.\pend
           \pstart
           Ich umarme Dich von Herzen und in Treue, mein lieber Arthur!\pend
           \pstart
           Dein {\\[\baselineskip]}\spacefill\mbox{Paul Goldmann}\pend
           \leftskip=0em{}\pstart
           \noindent{}{\pb}Bitte, bitte, bitte: Komm im Sommer nach \textcolor{pink}{Paris}{}\ledrightnote{\textcolor{pink}{Paris}} oder ſei im Auguſt 14 Tage mit mir \label{K_L02697-1v}\edtext{zuſammen}{\lemma{\textnormal{\emph{zuſammen}}}\Cendnote{\textnormal{Der Wunsch erfüllte sich
                     nicht.}}}\label{K_L02697-1h}! Bitte!!!\pend
           \endnumbering\briefempfaengerindex{Schnitzler, Arthur@\textsc{Schnitzler, Arthur}!zzzGoldmann, Paul@\emph{von Paul Goldmann}!1892-04-231@{23. 4. {[}1892{]}}|)be}\mylabel{h}  \normalsize

\doendnotes{C}
\bigskip
\vfill

\clearpage

\footnotesize

\lohead{\textsc{register}}

% Definiere theindex-Environment komplett neu ohne reledmac
\makeatletter
\renewenvironment{theindex}{%
  \section*{\indexname}%
  \setlength{\parindent}{0pt}%
  \setlength{\parskip}{0pt plus 0.3pt}%
  \let\item\@idxitem
}{%
  \clearpage
}
\makeatother

\IfFileExists{\jobname-pw.ind}{\input{\jobname-pw.ind}}{}

\end{document}

      