%% latex-korrekturansicht-vorspann.tex
%% Vorspann für die Korrekturansicht.
%% Lädt die gemeinsame Datei latex-vorspann.tex mit gesetztem Schalter.

\newif\ifkorrekturansicht
\korrekturansichttrue

\input{../tex-inputs/latex-vorspann}


               \section[Arthur Schnitzler an Albert Ehrenstein, 23. 11. 1909]{ Arthur Schnitzler an Albert Ehrenstein, 23. 11. 1909}\nopagebreak\mylabel{v}\rehead{ }\normalsize\beginnumbering\briefempfaengerindex{Ehrenstein, Albert@\textsc{Ehrenstein, Albert}!zzzSchnitzler, Arthur@\emph{von Arthur Schnitzler}!1909-11-231@{23. 11. 1909}|(be} \toendnotes[C]{\smallbreak\pagebreak[2]} \Standort{Jerusalem, The National Library of Israel, ARC. Ms. Var. 306 1 118.}
\physDesc{Brief, 1 Blatt, 1 Seite
\newline{}Schreibmaschine
\newline{}Handschrift: schwarze Tinte, lateinische Kurrent (\noindent{}Schlussformel, Unterschrift, eine Korrektur)}\Standort{DLA, A:Schnitzler, 85.1.642,2.}
\physDesc{Brief, maschineller Durchschlag
\newline{}Schreibmaschine
\newline{}Handschrift: roter Buntstift, lateinische Kurrent (\noindent{}Beschriftung: »Ehrenstein«)}\toendnotes[C]{\smallbreak}\pstart
           \noindent{}{\pb}\textcolor{gray}{\textbf{Dr. Arthur Schnitzler}}\pend
           \pstart
           \textcolor{gray}{\textbf{\textcolor{pink}{Wien XVIII. Spoettelgasse 7}{}\ledrightnote{\textcolor{pink}{Edmund-Weiß-Gasse}}.}}\hfill 23. 11. 1909.\pend
           \pstart{}Lieber Herr Ehrenstein! \pend\pstart
           Meine \textcolor{pink}{Berlin}{}\ledrightnote{\textcolor{pink}{Berlin}}er Reise dürfte erst im
                        Jänner oder Februar stattfinden. Ich bin noch
                    nicht dazugekommen Ihre neuen Manuskripte zu lesen, will es aber in den
                    allernächsten Tagen tun{[}.{]} Hoffentlich
                     wird die
                        \textcolor{blue}{Polgar}{}\ledrightnote{\textcolor{blue}{Alfred Polgar}}’sche Empfehlung an \textcolor{blue}{Bie}{}\ledrightnote{\textcolor{blue}{Oskar Bie}} von Nutzen sein. Vielleicht wäre es nun das Beste,
                    wenn ich an \textcolor{blue}{Fischer}{}\ledrightnote{\textcolor{blue}{Samuel Fischer}} oder \textcolor{blue}{Bie}{}\ledrightnote{\textcolor{blue}{Oskar Bie}} schriebe, dass ich die Absicht hatte persönlich mit
                    dem \textcolor{brown}{Verlag}{}\ledrightnote{→\textcolor{brown}{S. Fischer Verlag}} oder der \textcolor{brown}{Redaktion}{}\ledrightnote{→\textcolor{brown}{Neue Rundschau, Neue Deutsche Rundschau, Freie Bühne}} über Ihre Sachen zu
                    sprechen und dass ich nur wegen Verzögerung meiner Reise auf schriftlichem Wege
                    die Aufmerksamkeit darauf zu lenken genötigt sei. Mehr Erfolg scheint mir ja
                    allerdings der persönliche Weg zu versprechen. Hat es bis nächste Woche Zeit, so
                    können wir mündlich darüber reden.\pend
           \pstart
           Bestens grüßend{\\[\baselineskip]}{[}hs. Schnitzler:{]} Ihr{\\[\baselineskip]}\spacefill\mbox{ArthSchnitzler}\pend
           \leftskip=0em{}\endnumbering\briefempfaengerindex{Ehrenstein, Albert@\textsc{Ehrenstein, Albert}!zzzSchnitzler, Arthur@\emph{von Arthur Schnitzler}!1909-11-231@{23. 11. 1909}|)be}\mylabel{h}  \normalsize

\doendnotes{C}
\bigskip
\vfill

\clearpage

\footnotesize

\lohead{\textsc{register}}

% Definiere theindex-Environment komplett neu ohne reledmac
\makeatletter
\renewenvironment{theindex}{%
  \section*{\indexname}%
  \setlength{\parindent}{0pt}%
  \setlength{\parskip}{0pt plus 0.3pt}%
  \let\item\@idxitem
}{%
  \clearpage
}
\makeatother

\IfFileExists{\jobname-pw.ind}{\input{\jobname-pw.ind}}{}

\end{document}

      