%% latex-korrekturansicht-vorspann.tex
%% Vorspann für die Korrekturansicht.
%% Lädt die gemeinsame Datei latex-vorspann.tex mit gesetztem Schalter.

\newif\ifkorrekturansicht
\korrekturansichttrue

\input{../tex-inputs/latex-vorspann}


               \section[Hugo von Hofmannsthal an Arthur Schnitzler, 21. 8. {[}1904{]}]{ Hugo von Hofmannsthal an Arthur Schnitzler, 21. 8. {[}1904{]}}\nopagebreak\mylabel{v}\rehead{ }\normalsize\beginnumbering\briefempfaengerindex{Schnitzler, Arthur@\textsc{Schnitzler, Arthur}!zzzHofmannsthal, Hugo von@\emph{von Hugo von Hofmannsthal}!1904-08-211@{21. 8. 1904}|(be} \toendnotes[C]{\smallbreak\pagebreak[2]} \Standort{CUL, Schnitzler, B 43.}
\physDesc{Brief, 1 Blatt, 4 Seiten
\newline{}Handschrift: schwarze Tinte, deutsche Kurrent
\newline{}Schnitzler: mit Bleistift die Jahreszahl ergänzt: »904« \newline{}Ordnung: mit Bleistift von unbekannter Hand nummeriert:
                              »234« }\buchAbdrucke{\weitereDrucke{Hugo von Hofmannsthal, Arthur Schnitzler: \emph{Briefwechsel}. Hg. Therese Nickl und Heinrich Schnitzler. Frankfurt am Main: \emph{S. Fischer} 1964, S. 199.} }\toendnotes[C]{\smallbreak}\pstart
           \raggedleft{}{\pb}\textcolor{pink}{Ramgut}{}\ledrightnote{\textcolor{pink}{Ramgut}}{ }21 VIII.\pend
           \pstart{}lieber, \pend\pstart
           das ſcheint ſich ja ſehr ſchön zu treffen. \textcolor{blue}{Gerty}{}\ledrightnote{\textcolor{blue}{Gertrude von Hofmannsthal}}
               iſt auf jeden Fall ſehr froh mit Ihnen zu fahren und würde dafür eventuell bis zum
                     5\textsuperscript{ten} warten. Viel lieber wäre es ihr freilich, den 2\textsuperscript{ten} oder 3\textsuperscript{ten} zu fahren, was auch wohl möglich ſein wird, da mir \textcolor{blue}{Idchen Grünwald}{}\ledrightnote{\textcolor{blue}{Ida Grünwald}}{ }{\pb}heute aus \textcolor{pink}{\textsc{Haarlem}}{}\ledrightnote{\textcolor{pink}{Haarlem}} anzeigt daſs ſie pünktlich den 26\textsuperscript{ten} zurück ſein wird.\hspace*{1.5em}So werden wir dann
               hoffentlich eine ſchöne Woche zuſammen haben. Nur dürfte ich mich kaum in \textcolor{pink}{Iſchl}{}\ledrightnote{\textcolor{pink}{Bad Ischl}}{ }ſelber niederlaſſen, wo ich mit Sicherheit \textsc{Migraine} bekomme, ſondern nahe davon, etwa am \textcolor{pink}{Wolfgangſee}{}\ledrightnote{\textcolor{pink}{Wolfgangsee}}. Wie ſchön {\pb}aber wenn wir doch ein paar Tage
               im gleichen Hôtel wären. Nur \textcolor{pink}{Iſchl}{}\ledrightnote{\textcolor{pink}{Bad Ischl}} iſt mir abſolut
               unerträglich, wegen des Klimas und wegen der Geſichter der Leute die ich immer
               weniger vertrage.\pend
           \pstart
           Mein Aufenthalt iſt nicht durch die Rückkehr nach \textcolor{pink}{Rodaun}{}\ledrightnote{\textcolor{pink}{Rodaun}} begrenzt, ſondern durch den Wunſch, ungefähr {\pb}15\textsuperscript{ten} oder 16\textsuperscript{ten} September für einen
               ruhigen mehrwöchentlichen Aufenthalt in \textcolor{pink}{Venedig}{}\ledrightnote{\textcolor{pink}{Venedig}}
               einzutreffen. Denn das iſt die Stadt meiner arbeitſamſten Arbeit, meiner
               concentrierteſten Concentration und meiner einfältigſten Einfälle, und ſo hoffe ich
               denn dort wieder ein nicht ganz ſterbliches \textcolor{green}{Drama}{}\ledrightnote{→\textcolor{green}{Oedipus und die Sphinx. Tragödie in drei Aufzügen}} aufs erbleichende Papier zu ſchleudern. Wir nehmen den Weg
               dorthin etwa über \textcolor{pink}{\textsc{Trient}}{}\ledrightnote{\textcolor{pink}{Trient}} und durchs \textcolor{pink}{\textsc{val sugana}}{}\ledrightnote{\textcolor{pink}{Val Sugana}}, und ſo iſt man etwa bis \textcolor{pink}{Bozen}{}\ledrightnote{\textcolor{pink}{Bozen}} zuſa{\geminationm}en. Ei, niedlich!\pend
           \pstart Ihr\spacefill\mbox{Hugo}\pend{}\endnumbering\briefempfaengerindex{Schnitzler, Arthur@\textsc{Schnitzler, Arthur}!zzzHofmannsthal, Hugo von@\emph{von Hugo von Hofmannsthal}!1904-08-211@{21. 8. 1904}|)be}\mylabel{h}  \normalsize

\doendnotes{C}
\bigskip
\vfill

\clearpage

\footnotesize

\lohead{\textsc{register}}

% Definiere theindex-Environment komplett neu ohne reledmac
\makeatletter
\renewenvironment{theindex}{%
  \section*{\indexname}%
  \setlength{\parindent}{0pt}%
  \setlength{\parskip}{0pt plus 0.3pt}%
  \let\item\@idxitem
}{%
  \clearpage
}
\makeatother

\IfFileExists{\jobname-pw.ind}{\input{\jobname-pw.ind}}{}

\end{document}

      