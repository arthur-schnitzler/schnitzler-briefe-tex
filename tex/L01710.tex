%% latex-korrekturansicht-vorspann.tex
%% Vorspann für die Korrekturansicht.
%% Lädt die gemeinsame Datei latex-vorspann.tex mit gesetztem Schalter.

\newif\ifkorrekturansicht
\korrekturansichttrue

\input{../tex-inputs/latex-vorspann}


               \section[Richard Beer-Hofmann an Arthur Schnitzler, {[}27. 9. 1907{]}]{ Richard Beer-Hofmann an Arthur Schnitzler, {[}27. 9. 1907{]}}\nopagebreak\mylabel{v}\rehead{ }\normalsize\beginnumbering\briefempfaengerindex{Schnitzler, Arthur@\textsc{Schnitzler, Arthur}!zzzBeer-Hofmann, Richard@\emph{von Richard Beer-Hofmann}!1907-09-271@{{[}27. 9. 1907{]}}|(be} \toendnotes[C]{\smallbreak\pagebreak[2]} \Standort{CUL, Schnitzler, B 8.}
\physDesc{Manuskript1 Blatt, 2 Seiten
\newline{}Handschrift: Bleistift, lateinische Kurrent
\newline{}Schnitzler: mit Bleistift datiert: »Oct 907« \newline{}Ordnung: 1) mit Bleistift von \textcolor{blue}{Olga Schnitzler} (?) betitelt:
               »Auf das \textcolor{green}{Feuilleton} von
                  \textcolor{blue}{Berger} über Arthur.« 2) mit Bleistift von unbekannter Hand nummeriert:
                                    »278a«}\buchAbdrucke{\weitereDrucke{Arthur Schnitzler, Richard Beer-Hofmann: \emph{Briefwechsel 1891–1931}. Hg. Konstanze Fliedl. Wien, Zürich: \emph{Europaverlag} 1992, S. 185.} }\toendnotes[C]{\smallbreak}\stanza{}{\pb}Wie das Schicksal es auch
                     füge, –\newverse{}\textcolor{blue}{Alfred}{}\ledrightnote{\textcolor{blue}{Alfred von Berger}} kann nichts mehr passieren!\newverse{}Wahrheit mischt er hold mit Lüge –\newverse{}\label{K_L01710-1v}\edtext{Schreibt Kritik}{\lemma{\textnormal{\emph{Schreibt Kritik}}}\Cendnote{\textnormal{In seinem Feuilleton \emph{\textcolor{green}{Arthur Schnitzler}} (\emph{\textcolor{green}{Neue Freie Presse}}, Nr. 15467,
                              22. 9. 1907, S. 1–2.) schreibt \textcolor{blue}{Alfred von Berger}, \textcolor{blue}{Schnitzler}s ganzes Werk bestehe nur aus drei Dingen, Sex, Tod und
                        (Schau-)Spiel.}}}\label{K_L01710-1h} mit Hintertüren.\stanzaend{}\stanza{}Vorn ist’s eine Ruhmespforte\newverse{}Hinten wirds ein Hochgericht,\newverse{}Rückversichert sind die Worte –\newverse{}\uline{Alles} sagt er – und sagt’s nicht!\stanzaend{}\stanza{}Wird es eine Ehrenkette?\newverse{}Flicht er Ihnen einen Strick?\newverse{}Selber weiss er’s nicht – ich wette –\newverse{}Dieser Janus der Kritik.\stanzaend{}\stanza{}{\pb}Doch im ganzen,
                     ungefährlich\newverse{}wird die Sache – wie mir scheint –\newverse{}Danken Sie ihm nur \uline{so} ehrlich,\newverse{}Als er’s selbst mit Ihnen meint.\stanzaend{}\stanza{}\textcolor{blue}{Alfred}{}\ledrightnote{\textcolor{blue}{Alfred von Berger}}s Lob, und \textcolor{blue}{Alfred}{}\ledrightnote{\textcolor{blue}{Alfred von Berger}}s Tadel\newverse{}Rührt Sie ja nicht! – Gott sei Dank!\newverse{}– Doch – welch hoher Seelenadel,\newverse{}Spricht aus \textcolor{blue}{Alfred}{}\ledrightnote{\textcolor{blue}{Alfred von Berger}}s Lotterbank!\stanzaend{}\pstart \spacefill\mbox{R. B-H.}\pend{}\endnumbering\briefempfaengerindex{Schnitzler, Arthur@\textsc{Schnitzler, Arthur}!zzzBeer-Hofmann, Richard@\emph{von Richard Beer-Hofmann}!1907-09-271@{{[}27. 9. 1907{]}}|)be}\mylabel{h}  \normalsize

\doendnotes{C}
\bigskip
\vfill

\clearpage

\footnotesize

\lohead{\textsc{register}}

% Definiere theindex-Environment komplett neu ohne reledmac
\makeatletter
\renewenvironment{theindex}{%
  \section*{\indexname}%
  \setlength{\parindent}{0pt}%
  \setlength{\parskip}{0pt plus 0.3pt}%
  \let\item\@idxitem
}{%
  \clearpage
}
\makeatother

\IfFileExists{\jobname-pw.ind}{\input{\jobname-pw.ind}}{}

\end{document}

      