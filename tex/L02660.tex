%% latex-korrekturansicht-vorspann.tex
%% Vorspann für die Korrekturansicht.
%% Lädt die gemeinsame Datei latex-vorspann.tex mit gesetztem Schalter.

\newif\ifkorrekturansicht
\korrekturansichttrue

\input{../tex-inputs/latex-vorspann}


               \section[Paul Goldmann an Arthur Schnitzler, 6. 4. 1891]{ Paul Goldmann an Arthur Schnitzler, 6. 4. 1891}\nopagebreak\mylabel{v}\rehead{ }\normalsize\beginnumbering\briefempfaengerindex{Schnitzler, Arthur@\textsc{Schnitzler, Arthur}!zzzGoldmann, Paul@\emph{von Paul Goldmann}!1891-04-061@{6. 4. 1891}|(be} \toendnotes[C]{\smallbreak\pagebreak[2]} \Standort{DLA, A:Schnitzler, HS.NZ85.1.3162.}
\physDesc{Brief, 1 Blatt, 4 Seiten
\newline{}Handschrift: schwarze Tinte, deutsche Kurrent
\newline{}Schnitzler: mit rotem Buntstift eine Unterstreichung }\toendnotes[C]{\smallbreak}\pstart
           \noindent{}{\pb}\textcolor{brown}{\textcolor{gray}{\textbf{\textsc{Frankfurter Zeitung}}}}{}\ledrightnote{\textcolor{brown}{Frankfurter Zeitung}}\pend
           \pstart
           \textcolor{gray}{\textbf{\textsc{und}}}\pend
           \pstart
           \textcolor{gray}{\textbf{\textsc{Handelsblatt.}}}\hfill \textcolor{gray}{\textbf{\textcolor{pink}{Frankfurt a. M.}{}\ledrightnote{\textcolor{pink}{Frankfurt am Main}}, }}6. April \textcolor{gray}{\textbf{189}}1.\pend
           \pstart
           \textcolor{gray}{\textbf{\textbf{\textsc{Redaction.}}}}\pend
           \pstart
           \textcolor{gray}{\textbf{\textbf{\textsc{Telegramm-Adresse:}}}}\pend
           \pstart
           \textcolor{gray}{\textbf{\textbf{\textsc{Zeitung \textcolor{pink}{Frankfurt
                              Main}{}\ledrightnote{\textcolor{pink}{Frankfurt am Main}}.}}}}\pend
           \pstart\center{}Mein lieber Arthur!\pend\pstart
           Die Geſchichte von den Grenzen der menſchlichen Empfindungsfähigkeit iſt wohl
               richtig; aber es bleibt Einem doch nicht erſpart, die ganze Größe des Schmerzes zu
               empfinden, nicht auf einmal zwar, aber ratenweis, in einzelnen Attaquen. Ich habe
                  heut{ }Nacht wieder ſo ein wildes \label{K_L02660-22v}\edtext{Heimwehfieber}{\lemma{\textnormal{\emph{Heimwehfieber}}}\Cendnote{\textnormal{Im \emph{\textcolor{green}{Tagebuch}} fasste \textcolor{blue}{Schnitzler} den Brief zusammen: »Heute von \textcolor{blue}{Goldmann} der erste Brief, fühlt sich in
                        \textcolor{pink}{Frankfurt} sehr unglücklich.«
                     (8. 4. 1891)}}}\label{K_L02660-22h}
               durchgemacht; und wenn ich feig wäre, möchte ich den nächſten Zug benutzen und in der
               geliebten \textcolor{pink}{Stadt}{}\ledrightnote{→\textcolor{pink}{Wien}} mich in irgend
               einen Winkel verkriechen und nimmer daraus hervorkommen. Weiß der Himmmel – es kommt
               mir vor, als hätte ich die größte Dummheit gemacht, da ich von \textcolor{pink}{Wien}{}\ledrightnote{\textcolor{pink}{Wien}} wegging. Hier iſt es öde und troſtlos: die kleine \textcolor{pink}{Stadt}{}\ledrightnote{→\textcolor{pink}{Frankfurt am Main}}, die unſympathiſchen
               Menſchen und Langweile an allen Ecken und Enden; man kommt ſich vor wie im Gefängniß,
               und der Ruck, mit dem {\pb}die ſchwere Thür hinter Einem
               in’s Schloß gefallen, zittert in allen Nerven nach. Meinen \textcolor{blue}{Onkel}{}\ledrightnote{→\textcolor{blue}{Fedor Mamroth}} finde ich stumpf, gedrückt, reſignirt
               wieder, halb erſtickt von der Kleinſtadtatmoſphäre, mit einer tollen Sehnſucht nach
               der Welt draußen und, ich glaube auch, nach \textcolor{pink}{Wien}{}\ledrightnote{\textcolor{pink}{Wien}}
               im Herzen. Meine \textcolor{blue}{Mutter}{}\ledrightnote{→\textcolor{blue}{Clementine Goldmann}}
               krank, gealtert, ſorgenvoll, tief unglücklich. Was ich von den Verhältniſſen in der
               deutſchen Journaliſtik bisher gehört habe, lautet höchſt unerquicklich und läßt die
                  \textcolor{pink}{Wien}{}\ledrightnote{\textcolor{pink}{Wien}}er Zuſtände eher günſtiger erſcheinen. Die
               hieſigen Collegen empfingen mich freundlich aber kühl, wie es ſchon in \textcolor{pink}{Preußen}{}\ledrightnote{\textcolor{pink}{Preußen}} Brauch iſt. Zum \textcolor{blue}{Chefredacteur}{}\ledrightnote{→\textcolor{blue}{Leopold Sonnemann}} vorzudringen iſt mir noch
               nicht gelungen. Vorläufig heißt es, daß ich bis 1. Juni hierbleiben ſoll; Näheres iſt noch nicht verfügt. Was daraus
               werden ſoll, weiß ich nicht. Mir ſcheint, ich hätte beſſer gethan, als {\pb}Stiefelputzer bei irgendwem in \textcolor{pink}{Wien}{}\ledrightnote{\textcolor{pink}{Wien}} zu bleiben. Hier draußen iſt das \textcolor{pink}{Sibirien}{}\ledrightnote{\textcolor{pink}{Sibirien}} und die Verbannung.\pend
           \pstart
           Dir und allen Freunden danke ich noch von ganzem Herzen für alles Liebe, das Ihr \strikeout{mich} mir bis zum Schluß gethan. Beim \label{K_L02660-2v}\edtext{Abſchied}{\lemma{\textnormal{\emph{Abſchied}}}\Cendnote{\textnormal{\textcolor{blue}{Goldmann} war am 1. 4. 1891 abgereist.}}}\label{K_L02660-2h} hätte ich Euch
               gern noch ein Paar innige Worte geſagt, ſand aber nur – wie gewöhnlich – ein Paar
               dumme Witze. Auch jetzt finde ich den rechten Ausdruck nicht; ich mag auch nach
               keiner ſtylvollen Redewendung ſuchen. Mir brennt im Herzen die Trauer um Euch Alle, –
               die Überzeugung, daß ich es nie mehr wieder ſo gut haben werde wie bei Euch – und der
               eitle Schmerz, daß ich jetzt schon ganz erſetzt und halb vergeſſen bin.\pend
           \pstart
           Schreib’ mir bald, grüß’ mir Alle – beſonders \textsc{\textcolor{blue}{Richard}{}\ledrightnote{\textcolor{blue}{Richard Beer-Hofmann}}}, \textsc{\textcolor{blue}{Loris}{}\ledrightnote{\textcolor{blue}{Hugo von Hofmannsthal}}} und die \textsc{\textcolor{blue}{Fanjungs}{}\ledrightnote{→\textcolor{blue}{Leo Van-Jung}{\newline}→\textcolor{blue}{Boris Van-Jung}}} – und wenn Du Dich {\pb}ſelbſt erwiſcheſt, ſo
               grüß’ Dich, ſo oft Du kannſt (Briefkaſtenwitz!).\pend
           \pstart
           Dein treuer {\\[\baselineskip]}\spacefill\mbox{Paul Goldmann.}\pend
           \leftskip=0em{}\pstart
           \noindent{}Zeige dieſen Brief, wenn Du willſt, dem kleinen \textsc{\textcolor{blue}{Richard}{}\ledrightnote{\textcolor{blue}{Richard Beer-Hofmann}}}, ſonſt aber Niemandem.\pend
           \pstart
           Empfehlungen an Deine Familie.\pend
           \endnumbering\briefempfaengerindex{Schnitzler, Arthur@\textsc{Schnitzler, Arthur}!zzzGoldmann, Paul@\emph{von Paul Goldmann}!1891-04-061@{6. 4. 1891}|)be}\mylabel{h}  \normalsize

\doendnotes{C}
\bigskip
\vfill

\clearpage

\footnotesize

\lohead{\textsc{register}}

% Definiere theindex-Environment komplett neu ohne reledmac
\makeatletter
\renewenvironment{theindex}{%
  \section*{\indexname}%
  \setlength{\parindent}{0pt}%
  \setlength{\parskip}{0pt plus 0.3pt}%
  \let\item\@idxitem
}{%
  \clearpage
}
\makeatother

\IfFileExists{\jobname-pw.ind}{\input{\jobname-pw.ind}}{}

\end{document}

      