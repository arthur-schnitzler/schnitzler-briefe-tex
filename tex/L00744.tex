%% latex-korrekturansicht-vorspann.tex
%% Vorspann für die Korrekturansicht.
%% Lädt die gemeinsame Datei latex-vorspann.tex mit gesetztem Schalter.

\newif\ifkorrekturansicht
\korrekturansichttrue

\input{../tex-inputs/latex-vorspann}


               \section[Max Burckhard an Arthur Schnitzler, {[}19. 11.? 1897{]}]{ Max Burckhard an Arthur Schnitzler, {[}19. 11.? 1897{]}}\nopagebreak\mylabel{v}\rehead{ }\normalsize\beginnumbering\briefempfaengerindex{Schnitzler, Arthur@\textsc{Schnitzler, Arthur}!zzzBurckhard, Max Eugen@\emph{von Max Eugen Burckhard}!1897-11-191@{{[}19. 11.? 1897{]}}|(be} \toendnotes[C]{\smallbreak\pagebreak[2]} \Standort{CUL, Schnitzler, B 20.}
\physDesc{Visitenkarte
\newline{}Handschrift: schwarze Tinte, deutsche Kurrent
\newline{}Schnitzler: mit Bleistift ergänzte Jahreszahl: »97« \newline{}Ordnung: mit Bleistift von unbekannter Hand nummeriert:
                                    »30« }\Standort{DLA, A:Schnitzler, HS.NZ85.1.2665, S.  [12].}
\physDesc{maschinelle Abschrift}\toendnotes[C]{\smallbreak}\pstart
           \noindent{}{\pb}\textcolor{gray}{\textbf{\textsc{D\textsuperscript{r.} Burckhard}}}\pend
           \pstart
           \textcolor{gray}{\textbf{\textsc{\textcolor{pink}{IX. Frankgasse 1}{}\ledrightnote{\textcolor{pink}{Frankgasse}}.}}}\pend
           \pstart{}{\pb}Lieber verehrter Herr Doctor!\pend\pstart
           Ich war Ihrer \substVorne{}\textsuperscript{\textcolor{gray}{×}\-\textcolor{gray}{×}\-\textcolor{gray}{×}\-\textcolor{gray}{×}}\substDazwischen{}freund\substHinten{}ſchaftlichen Geſinnung vertrauend bereits heute Vormittag ſo frei Ihnen eine
                  \label{K_L00744_1v}\edtext{Gaſtkarte}{\lemma{\textnormal{\emph{Gaſtkarte}}}\Cendnote{\textnormal{Das Korrespondenzstück ist undatiert. Im Herbst
                     1897 wurden zwei Theaterstücke \textcolor{blue}{Burckhards} uraufgeführt. Bei der Uraufführung von \emph{\textcolor{green}{’s Katherl}} am 25. 11. 1897 war \textcolor{blue}{Schnitzler} verreist. Von \emph{\textcolor{green}{Die Bürgermeisterwahl}} besuchte er die erste
                  Vorstellung am 20. 11. 1897 im \textcolor{pink}{Deutschen Volkstheater}, so dass dieses
                  Korrespondenzstück am Vorabend der Premiere gelaufen sein könnte.}}}\label{K_L00744_1h} für
               morgen zu ſenden, die jedenfalls im Lauf des Nachmittags in Ihren Händen ſein wird.
               Ich danke Ihnen herzlich für Ihre liebenswürdigen Zeilen.\pend
           \pstart
           Herzlichst{\\[\baselineskip]}\spacefill\mbox{DrBurc}\pend
           \leftskip=0em{}\endnumbering\briefempfaengerindex{Schnitzler, Arthur@\textsc{Schnitzler, Arthur}!zzzBurckhard, Max Eugen@\emph{von Max Eugen Burckhard}!1897-11-191@{{[}19. 11.? 1897{]}}|)be}\mylabel{h}  \normalsize

\doendnotes{C}
\bigskip
\vfill

\clearpage

\footnotesize

\lohead{\textsc{register}}

% Definiere theindex-Environment komplett neu ohne reledmac
\makeatletter
\renewenvironment{theindex}{%
  \section*{\indexname}%
  \setlength{\parindent}{0pt}%
  \setlength{\parskip}{0pt plus 0.3pt}%
  \let\item\@idxitem
}{%
  \clearpage
}
\makeatother

\IfFileExists{\jobname-pw.ind}{\input{\jobname-pw.ind}}{}

\end{document}

      