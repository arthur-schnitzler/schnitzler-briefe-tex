%% latex-korrekturansicht-vorspann.tex
%% Vorspann für die Korrekturansicht.
%% Lädt die gemeinsame Datei latex-vorspann.tex mit gesetztem Schalter.

\newif\ifkorrekturansicht
\korrekturansichttrue

\input{../tex-inputs/latex-vorspann}


               \section[Hugo von Hofmannsthal an Arthur Schnitzler, 6. 8. 1899]{ Hugo von Hofmannsthal an Arthur Schnitzler, 6. 8. 1899}\nopagebreak\mylabel{v}\rehead{ }\normalsize\beginnumbering\briefempfaengerindex{Schnitzler, Arthur@\textsc{Schnitzler, Arthur}!zzzHofmannsthal, Hugo von@\emph{von Hugo von Hofmannsthal}!1899-08-061@{6. 8. 1899}|(be} \toendnotes[C]{\smallbreak\pagebreak[2]} \Standort{CUL, Schnitzler, B 43.}
\physDesc{Postkarte
\newline{}Handschrift: schwarze Tinte, deutsche Kurrent\newline{}Versand: 1) Stempel: »\nobreak{}\oindex{Altaussee@\textbf{Altaussee}, \emph{http://www.geonames.org/ontologyA.ADM3}|pwk}Alt-Aussee, 6 8 99\nobreak{}«.  2) Stempel: »\nobreak{}\oindex{San Martino di Castrozza@\textbf{San Martino di Castrozza}, \emph{https://www.geonames.org/ontologyP.PPL}|pwk}San Martin{[}o di Castrozza{]}, 8{[}. 8. 99{]}\nobreak{}«. 
\newline{}Schnitzler: mit Bleistift datiert: »6/8 99« \newline{}Ordnung: mit Bleistift von unbekannter Hand nummeriert:
                              »154« }\buchAbdrucke{\weitereDrucke{Hugo von Hofmannsthal, Arthur Schnitzler: \emph{Briefwechsel}. Hg. Therese Nickl und Heinrich Schnitzler. Frankfurt am Main: \emph{S. Fischer} 1964, S. 129.} }\pstart{}{\pb}\textsc{Herrn D\textsuperscript{r} Arthur Schnitzler}\pend{}\pstart{}\textsc{\textcolor{pink}{San Martino di Castrozzo}{}\ledrightnote{\textcolor{pink}{San Martino di Castrozza}}}\pend{}\pstart{}\textsc{\textcolor{pink}{Tirol}{}\ledrightnote{\textcolor{pink}{Tirol}}}\pend{}{\bigskip}\pstart
           \noindent{}{\pb}Ich freu mich, zu denken daſs Ihr
               alle beiſa{\geminationm}en ſeid und dieſe ſchönen Tage und
               Sternennächte genießt. \textcolor{blue}{\textsc{Frankenſtein}}{}\ledrightnote{\textcolor{blue}{Clemens von Franckenstein}} freut ſich ſehr auf \textcolor{blue}{Waſſermann}{}\ledrightnote{\textcolor{blue}{Jakob Wassermann}}. Ich erbitte
               von \textcolor{blue}{Richard}{}\ledrightnote{\textcolor{blue}{Richard Beer-Hofmann}} noch nähere Nachrichten wo er
                     13\textsuperscript{ten} bis 16\textsuperscript{ten} iſt, ebenſo von Ihnen\pend
           \pstart
           \textsc{\textcolor{pink}{Alt-Aussee Gasthaus Bru{\geminationn}thaler}{}\ledrightnote{\textcolor{pink}{Gasthaus Brunnthaler}}}.\pend
           \pstart
           Bin ſehr erholt und wohl.\pend
           \pstart
           Herzlich Euer{\\[\baselineskip]}\spacefill\mbox{Hugo.}\pend
           \leftskip=0em{}\endnumbering\briefempfaengerindex{Schnitzler, Arthur@\textsc{Schnitzler, Arthur}!zzzHofmannsthal, Hugo von@\emph{von Hugo von Hofmannsthal}!1899-08-061@{6. 8. 1899}|)be}\mylabel{h}  \normalsize

\doendnotes{C}
\bigskip
\vfill

\clearpage

\footnotesize

\lohead{\textsc{register}}

% Definiere theindex-Environment komplett neu ohne reledmac
\makeatletter
\renewenvironment{theindex}{%
  \section*{\indexname}%
  \setlength{\parindent}{0pt}%
  \setlength{\parskip}{0pt plus 0.3pt}%
  \let\item\@idxitem
}{%
  \clearpage
}
\makeatother

\IfFileExists{\jobname-pw.ind}{\input{\jobname-pw.ind}}{}

\end{document}

      