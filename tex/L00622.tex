%% latex-korrekturansicht-vorspann.tex
%% Vorspann für die Korrekturansicht.
%% Lädt die gemeinsame Datei latex-vorspann.tex mit gesetztem Schalter.

\newif\ifkorrekturansicht
\korrekturansichttrue

\input{../tex-inputs/latex-vorspann}


               \section[Oscar Blumenthal an Arthur Schnitzler, 14. 11. 1896]{ Oscar Blumenthal an Arthur Schnitzler, 14. 11. 1896}\nopagebreak\mylabel{v}\rehead{ }\normalsize\beginnumbering\briefempfaengerindex{Schnitzler, Arthur@\textsc{Schnitzler, Arthur}!zzzBlumenthal, Oskar@\emph{von Oskar Blumenthal}!1896-11-142@{14. 11. 1896}|(be} \toendnotes[C]{\smallbreak\pagebreak[2]} \Standort{CUL, Schnitzler, B 15.}
\physDesc{Brief, 1 Blatt, 2 Seiten
\newline{}Schreibmaschine
\newline{}Handschrift: schwarze Tinte (\noindent{}Unterschrift)
\newline{}Schnitzler: 1) mit Bleistift auf der leeren Rückseite beschriftet: »\textsc{(Blumenthal)}« 2) mit rotem Buntstift zwei Unterstreichungen\newline{}Ordnung: mit Bleistift von unbekannter Hand nummeriert:
                                    »7« }\pstart
           \noindent{}\centering{}{\pb}\textcolor{gray}{\textbf{\textcolor{brown}{\textsc{Lessing-Theater}}{}\ledrightnote{\textcolor{brown}{Lessing-Theater}}}}\pend
           \pstart
           \noindent{}\centering{}\textcolor{gray}{\textbf{\textsc{Director}:}}{ }\textcolor{gray}{\textbf{\textsc{Dr.}{ }OSCAR BLUMENTHAL.}}\pend
           \pstart
           \raggedleft{}\textcolor{gray}{\textbf{\textcolor{pink}{Berlin N.W. (40)}{}\ledrightnote{\textcolor{pink}{Berlin}}, den}}{ }14. November 1896.\pend
           \pstart\center{}Werther Herr Doctor!\pend\pstart
           Während meiner Anwesenheit in \textcolor{pink}{Wien}{}\ledrightnote{\textcolor{pink}{Wien}} habe ich
                    leider keine Gelegenheit gefunden, Sie zu sehen, und möchte Ihnen deshalb auf
                    diesem Wege eine Idee unterbreiten, die ich zunächst mit \textcolor{blue}{Friedrich Mitterwurzer}{}\ledrightnote{\textcolor{blue}{Friedrich Mitterwurzer}} besprochen habe, und zwar mit
                    begeisterter Zustimmung von seiner Seite. Da bei dem Einacter-Cyclus »\textcolor{green}{MORITURI}{}\ledrightnote{\textcolor{green}{Morituri}}« das
                    Publikum sich geneigt gefunden hat, eine Reihe von einactigen dramatischen
                    Genrebildern für ein Ganzes zu nehmen, wenn sie auch nur durch einen losen Faden
                    mit einander verknüpft sind, so ist mir der Gedanke gekommen, ob nicht Ihr
                    prächtiger »\textcolor{green}{ANATOL}{}\ledrightnote{\textcolor{green}{Anatol}}«
                    in ähnlicher Weise für das Theater erobert werden könnte. Ich denke mir unter
                    dem Gesammt-Titel »\textcolor{green}{ANATOL}{}\ledrightnote{\textcolor{green}{Anatol}}«, fünf Capitel aus einem Liebesleben von ARTHUR SCHNITZLER, eine Zusammenfassung etwa der fünf {\pb}einactigen Plaudereien aus Ihrem
                    Buch: »\textcolor{green}{EINE FRAGE AN DAS
                            SCHICKSAL}{}\ledrightnote{\textcolor{green}{Die Frage an das Schicksal}}«, — »\textcolor{green}{WEIHNACHTS-AUSVERKAUF}{}\ledrightnote{\textcolor{green}{Weihnachts-Einkäufe}}«, — »\textcolor{green}{EPISODE}{}\ledrightnote{\textcolor{green}{Episode}}« — {[}»{]}\textcolor{green}{DAS ABSCHIEDSSOUPER AM
                            HOCHZEITSMORGEN}{}\ledrightnote{\textcolor{green}{Abschiedssouper}}«, — und glaube, dass es leicht gelingen
                    könnte, durch Hinzufügung einzelner Sätze, besonders in das erste und letzte
                    Stück dieser Serie einen inneren Halt und volle Abrundung zu geben. \textcolor{blue}{MITTERWURZER}{}\ledrightnote{\textcolor{blue}{Friedrich Mitterwurzer}} ist
                    mit Begeisterung bereit, den \textcolor{green}{ANATOL}{}\ledrightnote{\textcolor{green}{Anatol}} bei seinem, den ganzen Monat April
                    umfassenden, Gastspiel zur Darstellung zu bringen, und ich bitte freundlichst um
                    Nachricht, wie Sie sich zu dieser Idee stellen würden.\pend
           \pstart
           Mit besten Grüssen{\\[\baselineskip]} Ihr ergebener{\\[\baselineskip]}\spacefill\mbox{{[}hs. Blumenthal:{]} Dr. Osc. Blumenthal}\pend
           \leftskip=0em{}\endnumbering\briefempfaengerindex{Schnitzler, Arthur@\textsc{Schnitzler, Arthur}!zzzBlumenthal, Oskar@\emph{von Oskar Blumenthal}!1896-11-142@{14. 11. 1896}|)be}\mylabel{h}  \normalsize

\doendnotes{C}
\bigskip
\vfill

\clearpage

\footnotesize

\lohead{\textsc{register}}

% Definiere theindex-Environment komplett neu ohne reledmac
\makeatletter
\renewenvironment{theindex}{%
  \section*{\indexname}%
  \setlength{\parindent}{0pt}%
  \setlength{\parskip}{0pt plus 0.3pt}%
  \let\item\@idxitem
}{%
  \clearpage
}
\makeatother

\IfFileExists{\jobname-pw.ind}{\input{\jobname-pw.ind}}{}

\end{document}

      