%% latex-korrekturansicht-vorspann.tex
%% Vorspann für die Korrekturansicht.
%% Lädt die gemeinsame Datei latex-vorspann.tex mit gesetztem Schalter.

\newif\ifkorrekturansicht
\korrekturansichttrue

\input{../tex-inputs/latex-vorspann}


               \section[Karl Kraus an Arthur Schnitzler, 31. 12. 1892]{ Karl Kraus an Arthur Schnitzler, 31. 12. 1892}\nopagebreak\mylabel{v}\rehead{ }\normalsize\beginnumbering\briefempfaengerindex{Schnitzler, Arthur@\textsc{Schnitzler, Arthur}!zzzKraus, Karl@\emph{von Karl Kraus}!1892-12-311@{31. 12. 1892}|(be} \toendnotes[C]{\smallbreak\pagebreak[2]} \Standort{CUL, Schnitzler, B 55.}
\physDesc{Postkarte
\newline{}Handschrift: schwarze Tinte, deutsche Kurrent\newline{}Versand: Stempel: »\nobreak{}\oindex{I., Innere Stadt@\textbf{I., Innere Stadt}, \emph{Bezirk (A.BZK)}|pwk}Wien 1/1, 31. 12. 92, 7–8 N\nobreak{}«.  }\buchAbdrucke{\weitereDrucke{\emph{Karl Kraus und Arthur Schnitzler. Eine Dokumentation.} Hg. Reinhard Urbach. In: \emph{Literatur und Kritik}, Bd. 49, Oktober 1970, S. 514.} }\toendnotes[C]{\smallbreak}\pstart{}{\pb}Herrn
                        Schriftsteller\pend{}\pstart{}D\textsuperscript{r.} Arthur Schnitzler,\pend{}\pstart{}\textcolor{pink}{Wien I}{}\ledrightnote{\textcolor{pink}{I., Innere Stadt}}\pend{}\pstart{}\textcolor{pink}{Grillparzerstr. 7}{}\ledrightnote{\textcolor{pink}{Grillparzerstraße}}.\pend{}{\bigskip}\pstart{}{\pb}Mein lieber Herr
                        Doctor!\pend\pstart
           Die \label{K_L00150_1v}\edtext{\textcolor{green}{Kritik}{}\ledrightnote{→\textcolor{green}{Arthur Schnitzler, Anatol}}}{\lemma{\textnormal{\emph{Kritik}}}\Cendnote{\textnormal{\textcolor{blue}{Karl Kraus}: \emph{\textcolor{green}{Arthur
                                    Schnitzler, Anatol}}. In: \emph{\textcolor{green}{Die
                                Gesellschaft}}, Jg. 9, Nr. 1,
                                1. 1. 1893, S. 109–110.}}}\label{K_L00150_1h}
                    über »\textcolor{green}{Anatol}{}\ledrightnote{\textcolor{green}{Anatol}}« (2 Spalten) iſt im Jännerheft
                    der »\textcolor{brown}{Geſellſch.}{}\ledrightnote{\textcolor{brown}{Die Gesellschaft}}« erſchienen. Beleg wird die
                    Schriftleitung an den \textcolor{brown}{Verlag}{}\ledrightnote{→\textcolor{brown}{Bibliographisches Bureau}} nach
                        \textcolor{pink}{Berlin}{}\ledrightnote{\textcolor{pink}{Berlin}}{ }ſchicken. Warum kommen Sie nicht
                    mehr ins \textcolor{pink}{Grienſteidl}{}\ledrightnote{\textcolor{pink}{Café Griensteidl}}? Wie geht’s?\pend
           \pstart
           Herzlichſte Grüße!{\\[\baselineskip]}Prost Neujahr!{\\[\baselineskip]}Ihr sehr ergeb.{\\[\baselineskip]}\spacefill\mbox{Karl
                        Kraus,}\pend
           \leftskip=0em{}\pstart
           \noindent{}\textcolor{pink}{I Maximilianstr. 13}{}\ledrightnote{\textcolor{pink}{Mahlerstraße}}.\pend
           \endnumbering\briefempfaengerindex{Schnitzler, Arthur@\textsc{Schnitzler, Arthur}!zzzKraus, Karl@\emph{von Karl Kraus}!1892-12-311@{31. 12. 1892}|)be}\mylabel{h}  \normalsize

\doendnotes{C}
\bigskip
\vfill

\clearpage

\footnotesize

\lohead{\textsc{register}}

% Definiere theindex-Environment komplett neu ohne reledmac
\makeatletter
\renewenvironment{theindex}{%
  \section*{\indexname}%
  \setlength{\parindent}{0pt}%
  \setlength{\parskip}{0pt plus 0.3pt}%
  \let\item\@idxitem
}{%
  \clearpage
}
\makeatother

\IfFileExists{\jobname-pw.ind}{\input{\jobname-pw.ind}}{}

\end{document}

      