%% latex-korrekturansicht-vorspann.tex
%% Vorspann für die Korrekturansicht.
%% Lädt die gemeinsame Datei latex-vorspann.tex mit gesetztem Schalter.

\newif\ifkorrekturansicht
\korrekturansichttrue

\input{../tex-inputs/latex-vorspann}


               \section[Fedor Mamroth an Arthur Schnitzler, 2. 8. 1889]{ Fedor Mamroth an Arthur Schnitzler, 2. 8. 1889}\nopagebreak\mylabel{v}\rehead{ }\normalsize\beginnumbering\briefempfaengerindex{Schnitzler, Arthur@\textsc{Schnitzler, Arthur}!zzzMamroth, Fedor@\emph{von Fedor Mamroth}!1889-08-021@{2. 8. 1889}|(be} \toendnotes[C]{\smallbreak\pagebreak[2]} \Standort{CUL, Schnitzler, B 68.}
\physDesc{Brief, 1 Blatt, 1 Seite
\newline{}Handschrift Paul Goldmann: blaue Tinte, deutsche Kurrent
\newline{}Schnitzler: 1) mit Bleistift nummeriert: »1.« 2) mit rotem Buntstift eine Unterstreichung}\toendnotes[C]{\smallbreak}\pstart
           \noindent{}{\pb}\textcolor{brown}{\textcolor{gray}{\textbf{\textsc{Frankfurter Zeitung}}}{\\}\textcolor{gray}{\textbf{\textsc{und}}}{\\}\textcolor{gray}{\textbf{\textsc{Handelsblatt.}}}}{}\ledrightnote{\textcolor{brown}{Frankfurter Zeitung}}\pend
           \pstart
           \textcolor{gray}{\textbf{\textsc{Redaction.}}}\hfill \textcolor{gray}{\textbf{\textsc{\textcolor{pink}{Frankfurt a. M.}{}\ledrightnote{\textcolor{pink}{Frankfurt am Main}},}}}{ }2. Aug. \textcolor{gray}{\textbf{\textsc{188}}}9\pend
           \pstart
           \textcolor{gray}{\textbf{\textsc{Telegramm-Adresse:}}}\pend
           \pstart
           \textcolor{gray}{\textbf{\textsc{Zeitung Frankfurt Main}}}\pend
           \pstart{}Hochgeehrter Herr Doctor!\pend\pstart
           \label{K_L00001_1v}\edtext{»\textcolor{green}{Der
                  Sohn}{}\ledrightnote{\textcolor{green}{Der Sohn. Aus den Papieren eines Arztes}}«}{\lemma{\textnormal{\emph{»Der
                  Sohn«}}}\Cendnote{\textnormal{Die Erzählung entstand im
                     Sommer 1889 (A. S.: \emph{Tagebuch}, 8. 9. 1889).}}}\label{K_L00001_1h} iſt leider auch mir zu düſter, ſo kunſtvoll das pſychologiſche Motiv
               immer entwickelt iſt.\pend
           \pstart
           Seien Sie mir nicht böſe, wenn ich Ihnen das \textcolor{green}{Ms}{}\ledrightnote{→\textcolor{green}{Der Sohn. Aus den Papieren eines Arztes}} zurückſende, erfreuen Sie mich bald durch \label{K_L00001_2v}\edtext{einen anderen Beitrag}{\lemma{\textnormal{\emph{einen anderen Beitrag}}}\Cendnote{\textnormal{Erst am 24. 12. 1891 erschien
                  ein erster Beitrag \textcolor{blue}{Schnitzler}s in der \emph{\textcolor{green}{Frankfurter Zeitung}}, die \emph{\textcolor{green}{Weihnachts-Einkäufe}} (Nr. 358, S. 1–2).}}}\label{K_L00001_2h} u. empfangen Sie
               meine höflichſten Grüße.\pend
           \pstart
           Ihr{\\[\baselineskip]}ergebener{\\[\baselineskip]}\spacefill\mbox{D\textsuperscript{r} FMamroth}\pend
           \leftskip=0em{}\endnumbering\briefempfaengerindex{Schnitzler, Arthur@\textsc{Schnitzler, Arthur}!zzzMamroth, Fedor@\emph{von Fedor Mamroth}!1889-08-021@{2. 8. 1889}|)be}\mylabel{h}  \normalsize

\doendnotes{C}
\bigskip
\vfill

\clearpage

\footnotesize

\lohead{\textsc{register}}

% Definiere theindex-Environment komplett neu ohne reledmac
\makeatletter
\renewenvironment{theindex}{%
  \section*{\indexname}%
  \setlength{\parindent}{0pt}%
  \setlength{\parskip}{0pt plus 0.3pt}%
  \let\item\@idxitem
}{%
  \clearpage
}
\makeatother

\IfFileExists{\jobname-pw.ind}{\input{\jobname-pw.ind}}{}

\end{document}

      