%% latex-korrekturansicht-vorspann.tex
%% Vorspann für die Korrekturansicht.
%% Lädt die gemeinsame Datei latex-vorspann.tex mit gesetztem Schalter.

\newif\ifkorrekturansicht
\korrekturansichttrue

\input{../tex-inputs/latex-vorspann}


               \section[Arthur Schnitzler an Robert Adam, 18. 6. 1915]{ Arthur Schnitzler an Robert Adam, 18. 6. 1915}\nopagebreak\mylabel{v}\rehead{ }\normalsize\beginnumbering\briefempfaengerindex{Adam, Robert@\textsc{Adam, Robert}!zzzSchnitzler, Arthur@\emph{von Arthur Schnitzler}!1915-06-181@{18. 6. 1915}|(be} \toendnotes[C]{\smallbreak\pagebreak[2]} \Standort{DLA, 96.34.1/12.}
\physDesc{Briefkarte, Umschlag
\newline{}Handschrift: schwarze Tinte, lateinische Kurrent\newline{}Versand: Stempel: »\nobreak{}Wien\nobreak{}«.  }\toendnotes[C]{\smallbreak}\pstart{}{\pb}\textcolor{gray}{\textbf{Dr. Arthur Schnitzler}}\pend{}\pstart{}\textcolor{gray}{\textbf{\textcolor{pink}{Wien XVIII. Sternwartestrasse 71}{}\ledrightnote{\textcolor{pink}{Sternwartestraße}}}}\pend{}{\bigskip}\pstart{}{\pb}Herrn Dr. Rob. Ad. Pollak\pend{}\pstart{}k.k.-Bezirksrichter\pend{}\pstart{}\textcolor{pink}{Zistersdorf}{}\ledrightnote{\textcolor{pink}{Zistersdorf}}.\pend{}{\bigskip}\pstart
           \noindent{}{\pb}\textcolor{gray}{\textbf{Dr. Arthur Schnitzler}}\hfill 18. 6. 15.\pend
           \pstart
           \textcolor{gray}{\textbf{\textcolor{pink}{Wien XVIII. Sternwartestrasse 71}{}\ledrightnote{\textcolor{pink}{Sternwartestraße}}}}\pend
           \pstart{}Verehrter Herr Adam,\pend\pstart
           mit besonderm Vergnügen habe ich Ihre freundliche \textcolor{green}{Manuscriptsendung}{}\ledrightnote{→\textcolor{green}{Der Fremde}} empfangen, mit wirklichem,
                    innersten Interesse die sechs Scenen gelesen, und wüßte nicht, was Sie davon
                    abhalten sollte, diese vornehme we{\geminationn} auch nicht in
                    allen Theilen gleich starke, und in manchen rhythmischen Eigenheiten nicht
                    durchaus einleuchtende Dichtung dem Publikum oder auch den Theatern vorzulegen.
                    Gewiß werden viele (und nicht die urtheilselosesten) {\pb}\introOben{}Leute\introOben{} mit gleichem Antheil und zuweilen mit tieferer
                    Bewegung die Scenen auf sich wirken lassen – in denen manchen nun auch eine
                    Theaterwirkung zu stecken scheint. Freilich werden nicht viele Bühnen für diese
                    eigenartige Sache in Betracht kommen. We{\geminationn} Sie im
                    Laufe der nächsten Zeit nach \textcolor{pink}{Wien}{}\ledrightnote{\textcolor{pink}{Wien}} kämen, lassen
                    Sie michs vielleicht wissen; es wäre mir ein Vergnügen, Sie wieder zu sprechen –
                    eventuell auch zu dem problematischen Capitel der praktischen Möglichkeiten
                    Ihrer \textcolor{green}{Arbeit}{}\ledrightnote{→\textcolor{green}{Der Fremde}}.\pend
           \pstart
           Verbindlich grüßend u dankend{\\[\baselineskip]}Ihr sehr ergebner{\\[\baselineskip]}\spacefill\mbox{Arthur Schnitzler}\pend
           \leftskip=0em{}\endnumbering\briefempfaengerindex{Adam, Robert@\textsc{Adam, Robert}!zzzSchnitzler, Arthur@\emph{von Arthur Schnitzler}!1915-06-181@{18. 6. 1915}|)be}\mylabel{h}  \normalsize

\doendnotes{C}
\bigskip
\vfill

\clearpage

\footnotesize

\lohead{\textsc{register}}

% Definiere theindex-Environment komplett neu ohne reledmac
\makeatletter
\renewenvironment{theindex}{%
  \section*{\indexname}%
  \setlength{\parindent}{0pt}%
  \setlength{\parskip}{0pt plus 0.3pt}%
  \let\item\@idxitem
}{%
  \clearpage
}
\makeatother

\IfFileExists{\jobname-pw.ind}{\input{\jobname-pw.ind}}{}

\end{document}

      