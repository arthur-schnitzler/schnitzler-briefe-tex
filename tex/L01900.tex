\documentclass[twoside=false,titlepage=false,open=any, parskip=never, fontsize=12pt, headings=small, chapterprefix=false, appendixprefix=false]{scrbook}
\addtolength{\oddsidemargin}{\evensidemargin}
\setlength{\oddsidemargin}{.5\oddsidemargin}
\setlength{\evensidemargin}{\oddsidemargin}

\usepackage[{textwidth=13cm,textheight=23cm,marginpar=3cm, left=2cm}]{geometry}
%\usepackage[textwidth=80mm, layoutwidth=170mm, paperheight =297mm, paperwidth  =210mm, layoutvoffset= 20mm,layouthoffset= 20mm]{geometry}
%\usepackage[paperheight =297mm, paperwidth  =210mm, layoutheight=230mm, layoutwidth=158mm, layoutvoffset= 20mm, layouthoffset= 20mm, textwidth=150mm, textheight=185mm, showcrop=false]{geometry}
%sepackage[paperheight=230mm, paperwidth=138mm, textwidth=100mm, textheight=185mm]{geometry}
 \usepackage[usenames, dvipsnames]{xcolor}
\usepackage{scrlayer-scrpage}
\usepackage{hyphenat}
\usepackage{fontspec}
\usepackage{moresize}
\usepackage[english, french, greek, ngerman]{babel}
%\usepackage{ipa}  für das Seitenwechselzeichens
\usepackage[babel]{microtype}
\usepackage[dash, dot]{dashundergaps}
\usepackage{soul}
\usepackage{ragged2e}
\usepackage[makeindex, protected]{splitidx}
\usepackage[itemlayout=abshang,hangindent=0.85em, subindent=0em, subsubindent=1em, justific=RaggedRight, columns=1, columnsep=0pt, indentunit=1em, totoc=false]{idxlayout}
\usepackage{scrhack}
\usepackage{xpatch}
\usepackage{reledmac}
\usepackage{refcount} % Für die Seitenverweise 1–3 etc. 
\usepackage{etoolbox}
\usepackage{framed}
\usepackage[export]{adjustbox} % loads also graphicx, für Bildgröße autom. maximal
\usepackage{float} %ermöglicht exakte Bildpositionierung
\usepackage{mdframed}
\usepackage{enumitem}
\usepackage{relsize}
\usepackage{longtable}
\usepackage{chngcntr} % Sectionnummern durchgehend
\usepackage{hanging} % Für hängende Absätze
\usepackage[rightmargin=0em, leftmargin=1em, indentfirst=false]{quoting} % Für die geänderte quote-Umgebung in den Hrsg-Texten
%\usepackage{fontawesome}
\usepackage{ellipsis}
\RequirePackage{hyphsubst}%
\HyphSubstIfExists{ngerman-x-latest}{\HyphSubstLet{ngerman}{ngerman-x-latest}}{} 
\listfiles
\usepackage[noadjust]{marginnote}

\KOMAoptions{toc=chapterentrydotfill, toc=flat}
\addtokomafont{chapterentrypagenumber}{\mdseries}
\setkomafont{chapterentry}{\normalfont\mdseries}
\setkomafont{partentry}{\normalfont\mdseries}
\RedeclareSectionCommand[tocbeforeskip=0pt]{chapter}

\setlength{\skip\footins}{4mm plus 2mm} % Abstand Fussnote Text
\interfootnotelinepenalty=10000 % Kein Seitenwechsel in Fuss

%\DeclareTextFontCommand{\emph}{\textit}

% Der Befehl erlaubt rechtsbündig bei Unterschriften, die nicht mehr in die Zeile passen
\def\spacefill{\hspace{\fill}\mbox{}\linebreak[0]\hspace*{\fill}}
\usepackage{atbegshi}
\usepackage{zref-abspage}
\usepackage{perpage}
\usepackage{zref-user}
\usepackage{tikz}
\usepackage{ulem}
\usetikzlibrary{calc,decorations.pathmorphing}
\setmainfont[Path=../fonts/,
  Extension=.otf,
  UprightFont=*-Regular,
  ItalicFont=*-Italic]{EBGaramond12}


\PassOptionsToPackage{gray}{xcolor}
\definecolor{gray}{gray}{0.6}

\doublehyphendemerits=1000000 % das hier verhindert zu viele aufeinanderfolgende Trennstriche am Zeilenende


\usepackage{zref-abspage}
\usepackage{zref-user}
\usepackage{tikz}
\usepackage{atbegshi}
\usepackage{ulem}
\usetikzlibrary{calc,decorations.pathmorphing}

\PassOptionsToPackage{gray}{xcolor}
\definecolor{gray}{gray}{0.6}

\doublehyphendemerits=1000000 % das hier verhindert zu viele aufeinanderfolgende Trennstriche am Zeilenende

\makeatletter
\newcommand{\currentsidemargin}{%
  \ifodd\zref@extract{textarea-\thetextarea}{abspage}%
    \oddsidemargin%
  \else%
    \evensidemargin%
  \fi%
}

\newcounter{textarea}
\newcommand{\settextarea}{%
   \stepcounter{textarea}%
   \zlabel{textarea-\thetextarea}%
   \begin{tikzpicture}[overlay,remember picture]
    % Helper nodes
    \path (current page.north west) ++(\hoffset, -\voffset)
        node[anchor=north west, shape=rectangle, inner sep=0, minimum width=\paperwidth, minimum height=\paperheight]
        (pagearea) {};
    \path (pagearea.north west) ++(1in+\currentsidemargin,-1in-\topmargin-\headheight-\headsep)
        node[anchor=north west, shape=rectangle, inner sep=0, minimum width=\textwidth, minimum height=7pt]
        (textarea) {};
  \end{tikzpicture}%
}

\tikzset{tikzul/.style={yshift=-.75\dp\strutbox}}

\newcounter{tikzul}%
\newcommand\tikzul[1][]{%
    \begingroup
    \global\tikzullinewidth\linewidth
    \def\tikzulsetting{[#1]}%
    \stepcounter{tikzul}%
    \settextarea
    \zlabel{tikzul-begin-\thetikzul}%
    \tikz[overlay,remember picture,tikzul] \coordinate (tikzul-\thetikzul) at (0,0);% Modified \tikzmark macro
    \ifnum\zref@extract{tikzul-begin-\thetikzul}{abspage}=\zref@extract{tikzul-end-\thetikzul}{abspage}
    \else
        \AtBeginShipoutNext{\tikzul@endpage{#1}}%
    \fi
    \bgroup
    \def\par{\ifhmode\unskip\fi\egroup\par\@ifnextchar\noindent{\noindent\tikzul[#1]}{\tikzul[#1]\bgroup}}%
    \aftergroup\endtikzul
    \let\@let@token=%
}
\newlength\tikzullinewidth


\def\tikzul@endpage#1{%
\setbox\AtBeginShipoutBox\hbox{%
\box\AtBeginShipoutBox
\hbox{%
\begin{tikzpicture}[overlay,remember picture,tikzul]
\draw[#1]
    let \p1 = (tikzul-\thetikzul), \p2 = ([xshift=\tikzullinewidth+\@totalleftmargin]textarea.south west) in
    \ifdim\dimexpr\y1-\y2<.5\baselineskip
        (\x1,\y1) -- (\x2,\y1)
    \else
        let \p3 = ([xshift=\@totalleftmargin]textarea.west) in
        (\x1,\y1) -- +(\tikzullinewidth-\x1+\x3,0)
        % (\x3,\y2) -- (\x2,\y2)
        (\x3,\y1)
       \myloop{\y1-\y2+.5\baselineskip}{%
           ++(0,-\baselineskip) -- +(\tikzullinewidth,0)
       }%
    \fi
;
\end{tikzpicture}%
}}%
}%


\def\endtikzul{%
    \zlabel{tikzul-end-\thetikzul}%
    \ifnum\zref@extract{tikzul-begin-\thetikzul}{abspage}=\zref@extract{tikzul-end-\thetikzul}{abspage}
    \begin{tikzpicture}[overlay,remember picture,tikzul]
        \expandafter\draw\tikzulsetting
            let \p1 = (tikzul-\thetikzul), \p2 = (0,0) in
            \ifdim\y1=\y2
                (\x1,\y1) -- (\x2,\y2)
            \else
                let \p3 = ([xshift=\@totalleftmargin]textarea.west), \p4 = ([xshift=-\rightmargin]textarea.east) in
                (\x1,\y1) -- +(\tikzullinewidth-\x1+\x3,0)
                (\x3,\y2) -- (\x2,\y2)
                (\x3,\y1)
                \myloop{\y1-\y2}{%
                    ++(0,-\baselineskip) -- +(\tikzullinewidth,0)
                }%
            \fi
        ;
    \end{tikzpicture}%
    \else
    \settextarea
    \begin{tikzpicture}[overlay,remember picture,tikzul]
        \expandafter\draw\tikzulsetting
            let \p1 = ([xshift=\@totalleftmargin,yshift=-.5\baselineskip]textarea.north west), \p2 = (0,0) in
            \ifdim\dimexpr\y1-\y2<.5\baselineskip
                (\x1,\y2) -- (\x2,\y2)
            \else
                let \p3 = ([xshift=\@totalleftmargin]textarea.west), \p4 = ([xshift=-\rightmargin]textarea.east) in
                (\x3,\y2) -- (\x2,\y2)
                (\x3,\y2)
                \myloop{\y1-\y2}{%
                    ++(0,+\baselineskip) -- +(\tikzullinewidth,0)
                }
            \fi
        ;
    \end{tikzpicture}%
    \fi
    \endgroup
}

% -------------------------------------------------------------- Additions by Peter Grill

\tikzset{tikzst/.style={yshift=0.5\dp\strutbox}}

\newcounter{tikzst}%
\newcommand\tikzst[1][]{%
    \begingroup
    \global\tikzstlinewidth\linewidth
    \def\tikzstsetting{[#1]}%
    \stepcounter{tikzst}%
    \settextarea
    \zlabel{tikzst-begin-\thetikzst}%
    \tikz[overlay,remember picture,tikzst] \coordinate (tikzst-\thetikzst) at (0,0);% Modified \tikzmark macro
    \ifnum\zref@extract{tikzst-begin-\thetikzst}{abspage}=\zref@extract{tikzst-end-\thetikzst}{abspage}
    \else
        \AtBeginShipoutNext{\tikzst@endpage{#1}}%
    \fi
    \bgroup
    \def\par{\ifhmode\unskip\fi\egroup\par\@ifnextchar\noindent{\noindent\tikzst[#1]}{\tikzst[#1]\bgroup}}%
    \aftergroup\endtikzst
    \let\@let@token=%
}
\newlength\tikzstlinewidth


\def\tikzst@endpage#1{%
\setbox\AtBeginShipoutBox\hbox{%
\box\AtBeginShipoutBox
\hbox{%
\begin{tikzpicture}[overlay,remember picture,tikzst]
\draw[#1]
    let \p1 = (tikzst-\thetikzst), \p2 = ([xshift=\tikzstlinewidth+\@totalleftmargin]textarea.south west) in
    \ifdim\dimexpr\y1-\y2<.5\baselineskip
        (\x1,\y1) -- (\x2,\y1)
    \else
        let \p3 = ([xshift=\@totalleftmargin]textarea.west) in
        (\x1,\y1) -- +(\tikzstlinewidth-\x1+\x3,0)
        % (\x3,\y2) -- (\x2,\y2)
        (\x3,\y1)
       \myloop{\y1-\y2+.5\baselineskip}{%
           ++(0,-\baselineskip) -- +(\tikzstlinewidth,0)
       }%
    \fi
;
\end{tikzpicture}%
}}%
}%


\def\endtikzst{%
    \zlabel{tikzst-end-\thetikzst}%
    \ifnum\zref@extract{tikzst-begin-\thetikzst}{abspage}=\zref@extract{tikzst-end-\thetikzst}{abspage}
    \begin{tikzpicture}[overlay,remember picture,tikzst]
        \expandafter\draw\tikzstsetting
            let \p1 = (tikzst-\thetikzst), \p2 = (0,0) in
            \ifdim\y1=\y2
                (\x1,\y1) -- (\x2,\y2)
            \else
                let \p3 = ([xshift=\@totalleftmargin]textarea.west), \p4 = ([xshift=-\rightmargin]textarea.east) in
                (\x1,\y1) -- +(\tikzstlinewidth-\x1+\x3,0)
                (\x3,\y2) -- (\x2,\y2)
                (\x3,\y1)
                \myloop{\y1-\y2}{%
                    ++(0,-\baselineskip) -- +(\tikzstlinewidth,0)
                }%
            \fi
        ;
    \end{tikzpicture}%
    \else
    \settextarea
    \begin{tikzpicture}[overlay,remember picture,tikzst]
        \expandafter\draw\tikzstsetting
            let \p1 = ([xshift=\@totalleftmargin,yshift=-.5\baselineskip]textarea.north west), \p2 = (0,0) in
            \ifdim\dimexpr\y1-\y2<.5\baselineskip
                (\x1,\y2) -- (\x2,\y2)
            \else
                let \p3 = ([xshift=\@totalleftmargin]textarea.west), \p4 = ([xshift=-\rightmargin]textarea.east) in
                (\x3,\y2) -- (\x2,\y2)
                (\x3,\y2)
                \myloop{\y1-\y2}{%
                    ++(0,+\baselineskip) -- +(\tikzstlinewidth,0)
                }
            \fi
        ;
    \end{tikzpicture}%
    \fi
    \endgroup
}
% --------------------------------------------------------------

\def\myloop#1#2#3{%
    #3%
    \ifdim\dimexpr#1>1.1\baselineskip
        #2%
        \expandafter\myloop\expandafter{\the\dimexpr#1-\baselineskip\relax}{#2}%
    \fi
}

\makeatother






\def\myloop#1#2#3{%
    #3%
    \ifdim\dimexpr#1>1.1\baselineskip
        #2%
        \expandafter\myloop\expandafter{\the\dimexpr#1-\baselineskip\relax}{#2}%
    \fi
}

\makeatother
%\newcommand{\damage}[1]{\tikzul[gray,line width=0.15\ht\strutbox,semitransparent]{#1}}
%\newcommand{\strikeout}[1]{\tikzst[black]{#1}}

\newcommand{\damage}[1]{\textcolor{orange}{#1}}
\newcommand{\strikeout}[1]{\sout{#1}}


\setlength{\parindent}{1em}

% Mehr als drei Auslassungspunkte 

\newcommand{\dotsseven}{%
.\kern\ellipsisgap 
.\kern\ellipsisgap
.\kern\ellipsisgap 
.\kern\ellipsisgap
.\kern\ellipsisgap
.\kern\ellipsisgap 
.\kern\ellipsisgap 	
\relax}

\newcommand{\dotssix}{%
.\kern\ellipsisgap 
.\kern\ellipsisgap
.\kern\ellipsisgap
.\kern\ellipsisgap
.\kern\ellipsisgap 
.\kern\ellipsisgap 
\relax}

\newcommand{\dotsfive}{%
.\kern\ellipsisgap 
.\kern\ellipsisgap
.\kern\ellipsisgap
.\kern\ellipsisgap 
.\kern\ellipsisgap 
\relax}

\newcommand{\dotsfour}{%
.\kern\ellipsisgap 
.\kern\ellipsisgap
.\kern\ellipsisgap
.\kern\ellipsisgap 
\relax}

\newcommand{\dotstwo}{%
.\kern\ellipsisgap 
.\kern\ellipsisgap
\relax}


% Silbentrennung
\selectlanguage{ngerman}
\hyphenation{Re-kours EP-STEIN Her-vay-vor-les-ung Steu-er-sa-chen Öst-reich Burck-hard Keuch-hus-ten Oedi-pus-auf-führ-un-gen Hi-obs-post Kärnt-ner-ring Vei-tlis-sen-gas-se Franck-gas-se Rath-hau-se Sechs-schg Stu-bai-thal Tha-deusz Volks-th Halb-mo-nats-schrift JAHR-ES-ZEI-TEN Te-le-phon mit-ge-theilt Ge-schäfts-ver-bin-dung hoch-müth-ig Ueber-zeu-gung bis-chen Au-tor-rech-te Hof-manns-thal Nor-deijk Irre-seins Tschap-perl mit-zu-thei-len Aeu-ße-rung be-thö-ren Kü-ni-gel Be-ur-thei-lung Kuenst-lern ko-moe-di-sche hae-mor-rha-gi-scher Doer-mann Wash-burn flei-ssig haute Buddh-ist Preu-ssen Lin-den-café Mit-theil-un-gen An-theil Lieu-te-nant oes-terr Rieg-ner Oes-ter-reich gro-ssem Fran-zo-sen-thum Roche Lili Ent-schlie-ssun-gen äu-ssert wuen-sche Trans-ac-tio-nen Ue-ber-win-dung Eu-gene Stra-ssen-dir-ne qua-tre Deutsch-öst-er-reich Deutsch-öst-er-reichs Bjørn-stjer-ne noth-ing Edit-ed Olga Ar-naud Mer-gent-heim Léon-tine Polla-czek Brion Barre Hoch-sin-ger Ka-tha-rina Arouet Va-len-ci-ennes Ueber-win-dung Type-writer-in Tolstoi-buch Schnitzler Copier-buche Schiller Intel-lek-tuell-en-as-so-zi-a-tion Salten Devrient Grien-steidl Ge-sell-ſchaft ein-ge-ſchloſ-ſen Fort-ſetz-un-gen Bor-dell-ſtück fort-ſchrei-ten wirk-ſam-es ſchrift-ſtel-ler-i-ſchen hin-weg-ſe-hen Gerichts-saal-be-richt-er-ſtat-ter}



% Sonderbefehl für .–
\def\dotdash{\nobreak\hspace{0pt}.–}  %ACHTUNG BEIM ERSETZEN: LEERZEICHEN DANACH 
\def\commadash{\nobreak\hspace{0pt},–}
\def\excdash{\nobreak\hspace{0pt}!–}
\def\semicolondash{\nobreak\hspace{0pt};–}
\def\parentdotdash{\nobreak\hspace{0pt}).–}
\def\slashislash{\,\slash\,\allowbreak\hspace{0pt}}

\newcommand{\strich}{\makebox[1em][l]{– }}


% Seite einrichten

% Farbe definieren
%\setmainfont[RawFeature={-liga}, 
%SmallCapsFont=WSVgara-Caps, 
%ItalicFont=WSVgara-Italic, 
%BoldFont=WSVgara-Bold,
%BoldItalicFont=WSVgara-BoldItalic
%]{WSVgara}
%\setsansfont[RawFeature={-liga}, 
%SmallCapsFont=WSVgara-Caps, 
%ItalicFont=WSVgara-Italic, 
%BoldFont=WSVgara-Bold,
%BoldItalicFont=WSVgara-BoldItalic
%]{WSVgara}

%\setmainfont{Brill}
%\setsansfont{Brill}

%\setmainfont[ItalicFont=SinaNova-Italic, 
%BoldFont=SinaNova-Bold,
%BoldItalicFont=SinaNova-BoldItalic
%]{SinaNova-Regular}
%\setsansfont[ItalicFont=SinaNova-Italic, 
%BoldFont=SinaNova-Bold,
%BoldItalicFont=SinaNova-BoldItalic
%]{SinaNova-Regular}



\def\labelitemi{--}

% Geminationsstrich, U-Strich

 \newcommand{\overbar}[1]{$\overline{\hbox{#1}}$}


% Ausrufezeichen in den Index kriegen
\newcommand{\rufezeichen}{"!}

% Griechisch
	
%\newfontfamily\greekfont{GaramondPremrPro}
%\newcommand\griechisch[1]{\greekfont{}#1{}\normalfont}
\newcommand\griechisch[1]{#1}


%\newfontfamily\sansseriffont[HyphenChar=None, RawFeature={-liga}, Scale=1.03]{TheSans-Regular}
%\newfontfamily\sansseriffont{uarial}


%\newfontfamily\sansseriffont[HyphenChar=None, LetterSpace=1.0, RawFeature={-liga}]{TheSans-SemiBold}
%\newcommand\sansseriff[1]{\sffamily{}#1{}\normalfont}

\newcommand{\mini}{\,}


\newcommand{\key}{\textsuperscript{\textcolor{red}{KEY}}}


%% Sperrung (Package Soul)
%% Hier ist die Sperrung definiert. Sperrung erreicht man mit \so{gesperrtes Wort}
\sodef\so{}{.14em}{.4em plus.1em minus .1em}{.4em plus.1em minus .1em}

% SCHRIFTEN
\setkomafont{disposition}{}
\addtokomafont{caption}{\small}
\addtokomafont{captionlabel}{\small}

%% Schrift der Kopf und Fußzeile
\renewcommand*{\headfont}{\normalfont}
\setkomafont{pagehead}{\footnotesize\addfontfeature{LetterSpace=10.0}}
\setkomafont{pagenumber}{\normalfont\normalsize}
\ohead[]{\pagemark}% Seitenzahl (c = centered) 
\ofoot[]{}


 
% Flatterndes Seitenende
\raggedbottom

% Fussnoten neu Anfangen

\makeatletter
\pretocmd{\@schapter}{\setcounter{footnote}{0}}{}{}
\pretocmd{\@chapter}{\setcounter{footnote}{0}}{}{}
\pretocmd{\@section}{\setcounter{footnote}{0}}{}{}
\makeatother


% Section Nummern durchgehend

\RedeclareSectionCommand[
  counterwithout=chapter
]{section}

% Section Punkt

\renewcommand*{\sectionformat}{}
\renewcommand*{\partformat}{}


% Marginpar Schrift

\newkomafont{margin}{\footnotesize} 
\makeatletter 
\let\MarginParOriginal\marginpar 
\renewcommand*{\marginpar}{\@dblarg\@marginpar} 
\newcommand{\@marginpar}[2][]{% 
  \MarginParOriginal[\usekomafont{margin}{#1\par}]{\usekomafont{margin}{#2\par}} 
} 
\makeatother 



\let\oldbeginnumbering\beginnumbering

\def\beginnumbering{\oldbeginnumbering\par\nopagebreak}


% Fußnoten linksbündig
\deffootnote{1.5em}{1em}{% 
\makebox[1.5em][l]{\thefootnotemark}%
}


% Fussnotenlineal (wobei für reledmac wohl was anderes gilt)
\let\normalfootnoterule\footnoterule
\setfootnoterule{0pt}
\let\normalfootnoterule\footnoterule


\setlength{\skip\footins}{8mm plus 2mm} % Abstand Fussnote Text
\interfootnotelinepenalty=10000 % Kein Seitenwechsel in Fuss

%% Kapitelüberschriften
\renewcommand*{\raggedchapter}{\centering} 
\renewcommand*{\raggedsection}{%
 \CenteringLeftskip=1cm plus 1em\relax 
 \CenteringRightskip=1cm plus 1em\relax 
 \Centering\footnotesize\thesection{}.\ }
\setkomafont{section}{\footnotesize}
\setkomafont{chapter}{\normalfont\Large}
\renewcommand{\chapterpagestyle}{empty}%The first page in each chapter won't have any heading or footer, especially no page number

% section ohne führende Kapitelnummer
\renewcommand*\thesection{\arabic{section}}

% Bildunterschrift ohne Nummer
\renewcommand*{\figureformat}{}
\renewcommand*{\captionformat}{}

% Abstand Bild
\setlength{\textfloatsep}{\baselineskip}

%% Zeilennummern
\firstlinenum{0} \linenumincrement{5}
\lineation{section} %Jeder Abschnitt wird durchnummeriert
\renewcommand{\numlabfont}{\ssmall} %Schriftgröße Zeilennummern

%\AtBeginEnvironment{multicols}{\RaggedRight} % Linksbündig in Spalten


% SEITENUMBRÜCHE IM TEXT MARKIEREN

%% Seitenumbrüche


\newcommand{\Theight}{\dimexpr\fontcharht\font`W}
\newcommand{\pbposition}{\depth}
\newcommand{\pb}{\nobreak\hspace{0pt}\raisebox{-0.1em}{\raisebox{\pbposition}{\textnormal{|}}}\nobreak\hspace{0pt}}

% EINFÜGUNGEN IM TEXT MARKIEREN

\renewcaptionname{ngerman}{\contentsname}{Inhalt}           %Table of contents


\newcommand{\introOben}{\textnormal{\raisebox{\Theight}{\raisebox{-\height}{\small{v}\normalsize}}}}
\newcommand{\introUnten}{\textnormal{\raisebox{\Theight}{\raisebox{-\height}{\small{v}\normalsize}}}}
\newcommand{\introMitteVorne}{\textnormal{\raisebox{\Theight}{\raisebox{-\height}{\small{v}\normalsize}}}}
\newcommand{\introMitteHinten}{\textnormal{\raisebox{\Theight}{\raisebox{-\height}{\small{v}\normalsize}}}}
\newcommand{\substVorne}{\textnormal{\raisebox{\Theight}{\raisebox{-\height}{\rotatebox[origin=c]{180}{v}\normalsize}}}}
\newcommand{\substDazwischen}{}
\newcommand{\substHinten}{\textnormal{\raisebox{\Theight}{\raisebox{-\height}{\small{v}\normalsize}}}}


% MARGINALSPALTE
\setlength\ledrsnotewidth{1.5cm}


% FUSSNOTE
%% Im Apparat f. und ff.
\Xtwolines{f.}
\Xtwolinesbutnotmore

%% Sperrungen bei Lemmas im Apparat
%\pretocmd{\so}{\null}{}{}
% Hab ich auskommentiert: Hat einen Fehler ergeben, denn plötzlich war ein Abstand vor Absätzen, die mit einer Sperrung beginnen

%% Zeilennummerierung Abstand zum Lemma
\Xboxlinenum{5mm}

%% Bei zwei Apparateinträgen in einer Zeile wird nur beim ersten Mal die Zeile gezählt
\Xnumberonlyfirstinline
\Xnumberonlyfirstintwolines
\Xinplaceofnumber{1em}
\Xhangindent{1em}

% ENDNOTEN
\Xendlemmadisablefontselection[A]
\renewcommand*{\printnpnum}[1]{{\noindent}\tiny}
\Xendparagraph[A] % Endnoten in einem Absatz
%\Xendtwolines{\tiny{f.}}
\Xendbeforepagenumber{} 
\Xendnotenumfont[A]{\tiny}
\Xendboxlinenum[A]{0em}
\Xendlemmaseparator{$\rbracket$}
\Xendnotefontsize[A]{\footnotesize}
\Xendhangindent[A]{1em}
\Xendlemmafont[A]{\itshape}
\Xendlemmafont[B]{\bfseries}
\Xendnotefontsize[B]{\footnotesize}
\Xendnotenumfont{\footnotesize}
\Xendlineprefixsingle[C]{\tiny}
\Xendlineprefixmore[C]{\tiny}
\Xendlemmadisablefontselection
\Xendlemmafont{\itshape}
\Xendlinerangeseparator{\tiny{--}}
\Xendhangindent{4em}
\Xendboxlinenum{3.6em}
\Xendafternumber{0.4em}
\Xendboxlinenumalign{R}

%\Xendboxstartlinenum{3.5em}
%\Xendboxendlinenum{1em}


%% Kaufmanns-Und (=)
            
            

\newcommand{\kaufmannsund}{\&} 

%% Tabelle Zellensprung
% Ein weiterer Anlass, das Kaufmannsund in der Übergabe zu vermeiden:

\newcommand{\zellensprung}{ \& }

%% INDEX
    
    \makeindex 
    \newcommand*\lettergroup[1]{}
    
        \newcommand{\pw}[1]{#1}
        \newcommand{\pwt}[1]{\textbf{#1}}
        \newcommand{\pws}[1]{\upshape{\textbf{#1}}}
            
        \newcommand{\pwe}[1]{\textbf{\emph{#1}}}
             
    \newcommand{\pwk}[1]{#1\textsuperscript{\tiny{K}}}
    \newcommand{\pwv}[1]{\emph{#1}}
     \newcommand{\pwkv}[1]{\emph{#1}\textsuperscript{\tiny{K}}}
               \newcommand{\pwuv}[1]{\emph{#1}?}
               \newcommand{\pwu}[1]{#1?}
 \newcommand{\range}[2]{{\def\pw##1{##1}#1}--#2}

\newcommand{\buch}[1]{#1}


%% MEHRERE INDIZES

\newindex[Register]{pw}
%\newindex[Institutionen Organisationen Periodika und Unternehmen]{org}
%\newindex[Institutionen und Orte]{o}
\newindex[Korrespondenzpartner]{briefe-out}
\newindex[Gedruckte Quellen]{buch-abdruck}

\newcommand\briefsenderindex[1]{\sindex[briefe-out]{#1}}
\newcommand\briefempfaengerindex[1]{\sindex[briefe-out]{#1}}

\newcommand\buchabdruck[1]{\sindex[buch-abdruck]{#1}}
\renewcommand\buchabdruck[1]{}



%% Symbole

%\newcommand{\symaddr}{\includegraphics[height=6pt]{symbol/noun_637366.png}}
%\newcommand{\symweiteredrucke}{\includegraphics[height=6pt]{symbol/noun_634729.png}}
%\newcommand{\symdruckvorlage}{\includegraphics[height=6pt]{symbol/noun_637409.png}}
%\newcommand{\symstandort}{\includegraphics[height=6pt]{symbol/noun_634216.png}}
%\newcommand{\symhead}{\includegraphics[height=6pt]{symbol/noun_1162030_cc.png}}


\newcommand{\symaddr}{A}
\newcommand{\symweiteredrucke}{D}
\newcommand{\symdruckvorlage}{V}
\newcommand{\symstandort}{O}
\newcommand{\symhead}{H}



\newcommand\anhangTitel[2]{\toendnotes[C]{\hangpara{4em}{1}{\makebox[4em][l]{\textbf{#1}}\textbf{#2}}\endgraf}}
\newcommand\Adresse[1]{\toendnotes[C]{\hangpara{4em}{1}{\makebox[4em][l]{\makebox[3.6em][r]{\symaddr}}}#1\endgraf}}

\newcommand\buchAlsQuelle[1]{\toendnotes[C]{\footnotesize\par\hangpara{4em}{1}{\makebox[4em][l]{\makebox[3.6em][r]{\symdruckvorlage}}}#1\endgraf}}
\newcommand\buchAbdrucke[1]{\toendnotes[C]{\footnotesize\par\hangpara{4em}{1}{\makebox[4em][l]{\makebox[3.6em][r]{\symweiteredrucke}}}#1\endgraf}}
\newcommand\Standort[1]{\toendnotes[C]{\footnotesize\hangpara{4em}{1}{\makebox[4em][l]{\makebox[3.6em][r]{\symstandort}}}#1\endgraf}}
\newcommand\biographical[1]{\toendnotes[C]{\footnotesize\hangpara{4em}{1}{\makebox[4em][l]{\makebox[3.6em][r]{\symhead}}}#1\endgraf}}
\newcommand\biographicalOhne[1]{\toendnotes[C]{\footnotesize\hangpara{4em}{1}{\makebox[4em][l]{\makebox[3.6em][r]{}}}#1\endgraf}}



\newcommand\datumImAnhang[1]{\toendnotes[C]{#1}}

\let\newcell&

\newcommand\physDesc[1]{\toendnotes[C]{\hangpara{4em}{0}#1\endgraf}}
\newcommand\weitereDrucke[1]{#1}


% Schnitzler Tagebuch Auszüge
\newcommand{\prgrph}[1]{\endgraf\medskip\noindent\textbf{#1}\newline}


%% VERWEISE
% Dieser Befehl vom Typ
% \verweis{FW_V_schwn_A}{FW_V_schwn_E} 
% dient den Verweisen auf den Text von Kommentar und Herausgebereingriffen. Ihm werden die Namen der beiden Labels – Anfang und Ende – übergeben und er setzt den Anfang und entscheidet ob f. oder ff. folgt 


\newcounter{mystart}
\newcounter{mystop}
\newcounter{phantom}

\newcommand*\myrangeref[2]{%
  \setcounterpageref{mystart}{#1}%
  \setcounterpageref{mystop}{#2}%
  \ifnum\value{mystop}<\value{mystart}%
    \typeout{[myrangeref] Strange...stop (#2) before start (#1).}%
    \pageref{#2}--\pageref{#1}%
  \else
    \pageref{#1}%
    \ifnum\value{mystart}<\value{mystop}%
      \addtocounter{mystop}{-1}%
      \ifnum\value{mystart}<\value{mystop}%
        \,ff.
        %--\pageref{#2}%%
      \else
        \,f.
         %%--\pageref{#2}%
              \fi
    \fi
  \fi
}
            
\newcommand*\myrangerefkasten[2]{%
  \setcounterpageref{mystart}{#1}%
  \setcounterpageref{mystop}{#2}%
  \ifnum\value{mystop}<\value{mystart}%
    \typeout{[myrangeref] Strange...stop (#2) before start (#1).}%
    \pageref{#2}--\pageref{#1}%
  \else
    \makebox[12pt][r]{\pageref{#1}}%
    \ifnum\value{mystart}<\value{mystop}%
      \addtocounter{mystop}{-1}%
      \ifnum\value{mystart}<\value{mystop}%
        --\pageref{#2}%%
      \else
         --\pageref{#2}%
         % alternativ hierher: f.
      \fi
    \fi
  \fi
}


\newcommand*\mylabel[1]{%
  \refstepcounter{phantom}%
  \label{#1}%
}

\newenvironment{anhang}{\vspace{1cm}
}{}

\emfontdeclare{\itshape}

%% RAHMEN SEITLICH

\newlength{\leftbarwidth}
\setlength{\leftbarwidth}{3pt}
\newlength{\leftbarsep}
\setlength{\leftbarsep}{10pt}

\renewenvironment{leftbar}[1][\hsize]
{% 
\def\FrameCommand 
{%
{\hspace{-7pt} \color{black} \vrule width 0.5pt}%
\hspace{0pt}%must no space.
\fboxsep=\FrameSep\colorbox{white}%
}%
\MakeFramed{\hsize#1\advance\hsize-\width\FrameRestore}%
}
{\endMakeFramed}
\setlength{\FrameSep}{5pt}

\newmdenv[topline=false, leftline=true, rightline=true, bottomline=false,%
  linewidth=0.5pt, leftmargin=30pt, rightmargin=30pt, %
  skipabove=8pt, skipbelow=8pt]{mdbar}

% Überstreichung (OVERLINE)

\makeatletter
\newcommand*{\textoverline}[1]{$\overline{\hbox{#1}}\m@th$}
\makeatother

% Rahmen für Hintergrundfarbe
\fboxsep0mm

% Befehl für gekürzte Texte

\newcommand{\kuerzung}{, Auszug}

% Verse 

\setlength{\stanzaindentbase}{20pt} %Play with it later.
\setstanzaindents{5,1,1}
\setcounter{stanzaindentsrepetition}{2}
\newcommand{\stanzaend}{\&}
\sethangingsymbol{\protect\hfill}
\AtEveryStopStanza{\vspace{0.25\baselineskip}} %Abstand zwischen Strophen


% Versuch eines Grid

\RedeclareSectionCommand[
  beforeskip=3\baselineskip,
  afterskip=\baselineskip
]{chapter}
\RedeclareSectionCommand[
  beforeskip=2\baselineskip,
  afterskip=\baselineskip
]{section}

\newcommand\adjacent[2][]{%
  \bgroup
  \RedeclareSectionCommand[
    beforeskip=2\baselineskip,
    afterskip=\baselineskip,
  ]{chapter}%
  \if\relax\detokenize{#1}\relax
    \addchap{#2}%
  \else
    \addchap[#1]{#2}%
  \fi
  \egroup
  \section
}


%change the part format in table of contents
\renewcaptionname{ngerman}{\contentsname}{Inhalt} 


% Inhaltsverzeichnis

\AtBeginDocument{%
  \addtocontents{toc}{\protect\label{toc}}%
}

\renewcaptionname{ngerman}{\contentsname}{Verzeichnis der Dokumente} 
 
 
   \DeclareTOCStyleEntry[
  beforeskip=15pt,
  entryformat=\normalsize\normalfont\centering,
  pagenumberformat=\nullfont,
  linefill={},
  raggedentrytext=true
]{part}{part}

  \DeclareTOCStyleEntry[
  beforeskip=5pt,
  entryformat=\normalsize\normalfont\centering,
  pagenumberformat=\nullfont,
  linefill={},
  raggedentrytext=true
]{chapter}{chapter}

\DeclareTOCStyleEntry[
  onstarthigherlevel=\vspace*{0.5\baselineskip}\nobreak,
  indent=0pt,
  entryformat=\normalsize\def\autodot{.},
  pagenumberformat=\normalsize,
  raggedentrytext=true
]{section}{section}



 
% Das folgende auskommentiert, funktionierte nicht mehr, ging aber in Bahr/Schnitzler. Sollte eigentlich dazu dienen, beim Inhaltsverzeichnis die Nummern rechtsbündig zu setzen

 \iffalse
 
  \DeclareTOCStyleEntry[
  beforeskip=5pt,
  entryformat=\normalsize\normalfont\centering,
  pagenumberformat=\nullfont,
  linefill={},
  raggedentrytext=true
]{chapter}{chapter}

\DeclareTOCStyleEntry[
  onstarthigherlevel=\vspace*{0.5\baselineskip}\nobreak,
  indent=0pt,
  entryformat=\normalsize\def\autodot{.},
  pagenumberformat=\normalsize,
  raggedentrytext=true
]{section}{section}
 
 
  \newcommand*\sectionnumberbox[1]{\hfill #1\hspace{.6em}}

\newlength{\zweiziffern}
\newlength{\dreiziffern}
\newlength{\vierziffern}
\settowidth{\zweiziffern}{9999}
\settowidth{\dreiziffern}{99999}
\settowidth{\vierziffern}{99999999}
 
\BeforeStartingTOC[toc]{\value{tocdepth}=\sectiontocdepth}


\DeclareTOCStyleEntry[
  onstarthigherlevel=\vspace*{0.5\baselineskip}\nobreak,
  indent=0pt,
  entryformat=\normalsize\def\autodot{.},
  entrynumberformat=\sectionnumberbox,
  pagenumberformat=\normalsize,
  numwidth=\zweiziffern,
  raggedentrytext=true
]{section}{section}

\newcommand{\toccheck}{\ifnum \value{section}=76 \addtocontents{toc}{\protect\DeclareTOCStyleEntry[numwidth=\dreiziffern]{section}{section}} \else \ifnum \value{section}=990 \addtocontents{toc}{\protect\DeclareTOCStyleEntry[numwidth=\vierziffern]{section}{section}} \fi \fi}
\fi



% Längen für Tabellen
\newlength{\longeste}
\newlength{\longestz}
\newlength{\longestd}
\newlength{\longestv}
\newlength{\longestf}

\newcommand\halbtextwidth{0.9\textwidth}

\newcommand\pwindex[1]{{\sindex[pw]{#1}}}
\newcommand\oindex[1]{{\sindex[pw]{#1}}}
\newcommand\orgindex[1]{{\sindex[pw]{#1}}}

\renewcommand\oindex[1]{{{\sindex[pw]{#1}}}}
\renewcommand\orgindex[1]{{{\sindex[pw]{#1}}}}



% INDEX

%\renewcommand\pwindex[1]{}
%\renewcommand\oindex[1]{}
%\renewcommand\orgindex[1]{}
%\renewcommand\buchabdruck[1]{}


\newcommand\url[1]{\mbox{#1}}
\renewcommand\ngermanhyphenmins{33}

\makeatletter
\newcommand*{\geminationm}{$\overline{\hbox{m}}\m@th$}
\newcommand*{\geminationn}{$\overline{\hbox{n}}\m@th$}
\makeatother

%part
\renewcommand{\partmarkformat}{}
\renewcommand{\partheadmidvskip}{\enskip}
\renewcommand{\partformat}{}
\setkomafont{partnumber}{\usekomafont{part}}


%\geometry{headsep=8pt} % Abstand Kopfzeile - Text
%% DOKUMENT

\begin{document}

% Section ohne Nummer
\renewcommand*{\raggedsection}{%
 \CenteringLeftskip=1cm plus 1em\relax 
 \CenteringRightskip=1cm plus 1em\relax 
 \Centering\normalsize}



\widowpenalty=10000         % avoid widows
\clubpenalty=10000          % avoid orphans

\sloppy
\setlength{\parindent}{0em}

\setlength{\ledlsnotewidth}{4cm}
\setlength{\ledrsnotewidth}{4cm}
\renewcommand*{\ledlsnotefontsetup}{\scriptsize\sffamily}% left
\renewcommand*{\ledrsnotefontsetup}{\scriptsize\sffamily}% left
\thispagestyle{empty} 

               \section[Richard Beer-Hofmann an Arthur Schnitzler, {[}13. 12. 1909?{]}]{ Richard Beer-Hofmann an Arthur Schnitzler, {[}13. 12. 1909?{]}}\nopagebreak\mylabel{v}\rehead{ }\normalsize\beginnumbering\briefempfaengerindex{Schnitzler, Arthur@\textsc{Schnitzler, Arthur}!zzzBeer-Hofmann, Richard@\emph{von Richard Beer-Hofmann}!1909-12-131@{{[}13. 12. 1909?{]}}|(be} \toendnotes[C]{\smallbreak\pagebreak[2]} \Standort{CUL, Schnitzler, B 8.}
\physDesc{Manuskript13 Blätter, 13 Seiten (Paginierung mit Schreibmaschine)
\newline{}Schreibmaschine
\newline{}Handschrift: Bleistift, deutsche Kurrent (\noindent{}Korrekturen)}\buchAbdrucke{\weitereDrucke{Arthur Schnitzler, Richard Beer-Hofmann: \emph{Briefwechsel 1891–1931}. Hg. Konstanze Fliedl. Wien, Zürich: \emph{Europaverlag} 1992, S. 198–206.} }\toendnotes[C]{\smallbreak}\pstart
           \noindent{}\centering{}{\pb}\textcolor{green}{\uline{DAS ECHO DES
								LEBENS.}}{}\ledrightnote{\textcolor{green}{Das Echo des Lebens}}\pend
           \pstart
           \noindent{}\centering{}Ein Epilog zur \label{KLL01900_AS-1v}\edtext{Generalprobe}{\lemma{\textnormal{\emph{Generalprobe}}}\Cendnote{\textnormal{Der Text ist
						undatiert, wurde \textcolor{blue}{Schnitzler} aber am 13. 12. 1909 von \textcolor{blue}{Beer-Hofmann} vorgelesen. Damit ist
						anzunehmen, dass er ihn zu diesem Zeitpunkt erhalten hat.}}}\label{KLL01900_AS-1h} des
						Stückes{\\}»\textcolor{green}{Der Ruf des Lebens}{}\ledrightnote{\textcolor{green}{Der Ruf des Lebens. Schauspiel in drei Akten}}«.\pend
           {\bigskip}\pstart
           \noindent{}Am Tage der Aufführung. Vier Uhr Nachmittags. Da es der 11. Dezember
					ist, dämmert es bereits merklich. Der Vorhang ist hochgezogen. Die Bühne trägt
					die Dekoration des \textcolor{green}{2. Aktes}{}\ledrightnote{→\textcolor{green}{Der Ruf des Lebens. Schauspiel in drei Akten}}.
					Die Figuren des Stückes, die noch vor kurzem, in der unmateriellen Wirklichkeit,
					die ihnen die Worte ihres Schöpfers gaben, bewegt aufrecht standen – lehnen nun,
					in der materiellen Unwirklichkeit, die ihnen gestern – bei der Generalprobe –
					die Schauspieler gaben, etwas blass und müde an den Wänden umher. Nur »der alte
					Moser« liegt von rechts nach links, die ganze Bühne überquerend, wie ein
					Schlagbaum am Boden. Auf dem Fensterbrett scheint der Oberleib des Obersten zu
					stehen. Man kann augenblicklich nicht erkennen, ob er einen Unterleib besitzt.
					Irene, die man befragen könnte, liegt neben dem alten Moser auf dem Boden. Falls
					Max sie fragen sollte, wird sie es verneinen.\pend
           \pstart
           Vorne, hart am Souffleurkasten, ist eine dünne, frisch gestrichene grüne Barriere
					aufgestellt, wohl, um zu verhindern, dass die Person\introOben{}en\introOben{}
					des Stückes dem Publikum zu nahe gehen. Der Souffleurkasten scheint besetzt –
					nach der Unruhe, die in ihm herrscht; (als sässe jemand darinn, dem
					er zu eng ist).\pend
           \pstart
           Eine Pause.\pend
           \pstart
           Dann, eine ungeduldige Stimme aus dem Souffleurkasten:\pend
           \pstart
           {\pb}So fangen Sie doch an!\pend
           \pstart
           Marie: (mit etwas starren Augen, leise, und ein wenig verlegen) Verzeihen Sie,
					Herr – – – ich weiss gar nicht, wie ich Sie nennen soll – –\pend
           \pstart
           \substVorne{}\textsuperscript{Stimme}{\allowbreak}\substDazwischen{}Souffleur\substHinten{}: Souffleur! Nennen Sie mich nur so. Für Sie bin ich es augenblicklich –
					was ich sonst bin, kommt hier nicht in Betracht. Fangen Sie doch an!\pend
           \pstart
           Marie: Verzeihen Sie, Herr Souffleur – aber – – ich bin vielleicht nicht
					ganz berechtigt, Sie das zu fragen – aber wieso sind wir da?\pend
           \pstart
           Katharina: Ja! Wieso sind wir da?\pend
           \pstart
           Der Oberst: Sie fragen nach den letzten Dingen – liebe Marie! Nach unserem
					Dasein.\pend
           \pstart
           Max: (zu Albrecht leise) Der Oberst ist ein gar zu witziger Kopf!\pend
           \pstart
           Der Oberst: Die \strikeout{ewig} Fragen nach den letzten
					Dingen, für den letzten Akt, liebe Marie! Vorher, ist jede Tiefe, eine Grube,
					die sich der Dichter gräbt.\pend
           \pstart
           \substVorne{}\textsuperscript{Stimme}{\allowbreak}\substDazwischen{}Souffleur\substHinten{}: (ärgerlich) Dann graben Sie doch nicht, Herr Oberst!\pend
           \pstart
           \label{T_L01900-1v}\edtext{Marie: Aber}{\lemma{\textnormal{\emph{Marie: Aber}}}\Cendnote{\textnormal{geändert aus: »Marie; Aber«}}}\label{T_L01900-1h} ich
					habe ja nur ganz unschuldig gefragt – – –\pend
           \pstart
           Die Oberstin (hebt den Kopf, sehr hart) \introOben{}»\introOben{}Unschuldig\introOben{}«\introOben{}? Sie? (Sie lacht auf, und lässt den Kopf wieder
					sinken.)\pend
           \pstart
           Katharina: Auch ich habe nur leichthin – – –\pend
           \pstart
           Die Oberstin: (wie vorhin) »Leichthin«? \uline{Das} passt
					für Sie. »Leichthin«.\pend
           \pstart
           \substVorne{}\textsuperscript{Stimme}{\allowbreak}\substDazwischen{}Souffleur\substHinten{}: (ärgerlich zur Oberstin) Fangen Sie nicht wieder an – –!\pend
           \pstart
           Unteroffizier Sebastian (sehr militärisch, aber mit eingefetteter Stimme): Melde
					gehorsamst, wir sollen doch anfangen!\pend
           \pstart
           \substVorne{}\textsuperscript{Stimme}{\allowbreak}\substDazwischen{}Souffleur\substHinten{}: Reden Sie nichts drein – Sie – wie heissen \substVorne{}\textsuperscript{s}\substDazwischen{}S\substHinten{}ie, Unteroffizier, ich habe Ihren Namen vergessen.\pend
           \pstart
           {\pb}Sebastian: Melde gehorsamst,
					Herr Soffleur, Sebastian!\pend
           \pstart
           \substVorne{}\textsuperscript{Stimme}{\allowbreak}\substDazwischen{}Souffleur\substHinten{}: Anfangen!\pend
           \pstart
           Der Arzt: Herr Souffleur, ich kann Fräulein Marie nicht Unrecht geben
					– – –\pend
           \pstart
           Oberst: \uline{Das} haben wir gemerkt!\pend
           \pstart
           Der Arzt: – – es ist doch für uns alle – sozusagen – eine Lebensfrage, zu
					wissen, wieso wir da sind?\pend
           \pstart
           Der Adjunkt / Max (gleichzeitig) Fräulein Marie hat Recht.\pend
           \pstart
           \substVorne{}\textsuperscript{Stimme}{\allowbreak}\substDazwischen{}Souffleur\substHinten{}: Gut, ich will versuchen – – –\pend
           \pstart
           Der Oberst: Der Dichter ist unser Schöpfer! Sie, mein Herr, sind der
					Versucher!\pend
           \pstart
           \substVorne{}\textsuperscript{Stimme}{\allowbreak}\substDazwischen{}Souffleur\substHinten{}: Graben Sie nicht, Herr Oberst! Ich will versuchen es Ihnen zu sagen.
					Geben Sie Acht!\pend
           \pstart
           Der alte Moser (hebt den Kopf): Acht? Nein, neunundsiebzig Jahre bin ich
					alt – –\pend
           \pstart
           Der Arzt: Sie irren sich, Herr Moser, Sie sind jetzt gar nicht mehr alt; Sie sind
					ganz jung tot.\pend
           \pstart
           Der alte Moser: Ich will nicht tot sein.\pend
           \pstart
           Der Arzt: Herr Moser! Ich bin Ihr Arzt – Sie müssen tot sein! Sie sind dazu
					verpflichtet. Nicht nur sich selbst gegenüber – – –\pend
           \pstart
           \introOben{}\strikeout{Stimme}: \introOben{}Souffleur: Passen Sie auf, Sie
					werden mich nicht verstehen! Wissen Sie, wie lange die kleinen Teilchen im
					Resonnanzboden einer Violine nicht zur Ruhe kommen und noch
					fortschwingen, wenn für unser Ohr der Bogenstrich, der sie erschütterte, längst
					verklungen ist?\pend
           \pstart
           Albrecht: Wir reiten morgen in den Tod, Herr Souffleur – und Sie – prüfen uns
					Physik?\pend
           \pstart
           {\pb}Souffleur: Es war nur eine
					rhetorische Frage – – –\pend
           \pstart
           (Der Hintergrund wird durchsichtig; im hellen Sonnenlicht erblickt man ein Dorf,
					lieblich an Hängen gelagert. Eine tiefsinnige Stimme sagt:\pend
           \pstart
           Rhetorisch! Ja!\pend
           \pstart
           Eine Frauenstimme (wiederholt in kurzen Atemstössen, ekstatisch): Rhetorisch! Ja,
					das ist er!\pend
           \pstart
           (Die Landschaft entschwindet.)\pend
           \pstart
           Arzt: War das Grünau, Herr Adjunkt?\pend
           \pstart
           Adjunkt: Nein, lieber Doktor: Grinzing!\pend
           \pstart
           Souffleur: Nun sehen Sie: Von allen Worten, die Sie gestern auf der Generalprobe
					sprachen, schwingt noch die Luft; sie hallen noch von den Mauern und den
					gemalten Leinenwänden wieder, und wer feine Ohren hat, kann sie hören.\pend
           \pstart
           Arzt: Aber Herr Souffleur, das würde ja nur erklären, wieso unsere Stimmen da
					sind; aber woher nehmen Sie denn unsere Leiber?\pend
           \pstart
           Souffleur: Ich könnte sagen: »Von die zwei Gulden!« Aber Sie würden mich nicht
					verstehen, und es würde, überdies, vielleicht mein Inkognito lüften.\pend
           \pstart
           Der alte Moser: (hebt den Kopf) Nicht lüften! Sind Sie toll, ich kann den Tod
					davon haben. (er legt sich wieder hin).\pend
           \pstart
           Arzt: (energisch) Herr Moser, ich ordiniere Ihnen tot zu sein. Wenn Sie meine
					Verordnungen nicht befolgen, stehe ich für nichts!\pend
           \pstart
           Souffleur: Ihre Leiber also, Herr Arzt? Nun denn: Wenn Sie einen Gegenstand, zum
					Beispiel das Fensterkreuz dort – – –\pend
           \pstart
           Max: Hier ist keines, Herr Souffleur! Sonst bleibt der Herr Oberst beim
					Hereinspringen hängen!\pend
           \pstart
           {\pb}Souffleur: Lassen Sie mich
					ausreden!\pend
           \pstart
           Oberstin: (höhnisch) \uline{Wir} – \uline{Sie}? Haha!\pend
           \pstart
           Souffleur: Wenn Sie ein Fensterkreuz fest ins Auge fassen, und dann den Blick von
					ihm abwenden, so werden Sie – wenn Sie genau beobachten – noch einen Augenblick
					lang, vor sich in der Luft, das Bild des Fensterkreuzes sehen. Ich sage: »Das
					Bild«! Aber wissen wir, ob es mehr oder weniger Bild ist, als das Fensterkreuz,
					das wir vorhin sahen – nicht? – – Sagen Sie doch: »Jo«! – Von der
					Leiblichkeit, die gestern auf der Probe Schauspieler den Worten des Dichters
					liehen, schweben die Formen noch durcheinander in diesem Raum, und haben – eine
					Weile – eine Art von Leiblichkeit.\pend
           \pstart
           Sebastian: Herr Souffleur, melde gehorsamst, dass doch seither die gestrige
					Abendvorstellung – ganz ausverkauft – hier war!\pend
           \pstart
           Souffleur: (nach einer Pause) Sie haben recht – aber ich fange an zu zweifeln, ob
					Sie wirklich »Sebastian« heissen!\pend
           \pstart
           Sebastian: Melde gehorsamst: So soll ich le – – (abbrechend zu Max) Fürchten
					Sie nichts, Herr Leutnant – auch wir haben einander zugeschworen, auch wir sind
					totgeweiht!\pend
           \pstart
           Oberst: Wie kommt es, Unteroffizier, dass ich gar niemanden vom Regiment
					erblicke?\pend
           \pstart
           Oberstin: (höhnisch) Weil Du mit dem Rücken gegen den Kasernhof stehst!\pend
           \pstart
           Oberst: Aber es sollten doch Kürrassiere unseres Regimentes vorbeigehen und
					grüssen, und ich sollte »Gute Nacht« wünschen – – – wo sind denn Alle
					– – Unteroffizier!\pend
           \pstart
           Sebastian: Melde gehorsamst: Das ganze Regiment ist nach Hause gefahren um
					Abschied zu nehmen, und wir haben einander {\pb}zugeschworen, dass keiner
					zurückkommt. Wir sind Alle blaue Kürrassiere.\pend
           \pstart
           Katharina: (gerührt zu Sebastian) Geben Sie mir Ihre Hand, Sebastian\pend
           \pstart
           Sebastian: (gibt die Hand nicht)\pend
           \pstart
           Katharina: Abschied nehmen ist süss!\pend
           \pstart
           Sebastian: Melde gehorsamst, es kann auch eine Woche dauern!\pend
           \pstart
           Katharina: Ich dachte, Du seist ein lustiger Bursche?\pend
           \pstart
           Sebastian: Zu Befehl! Auch lustig bin ich, aber dann sag ich nicht »Abschied
					nehmen«.\pend
           \pstart
           Marie: (kommt nach vorne, bückt sich zum Souffleurkasten) Ich danke Ihnen, Herr
					Souffleur – jetzt versteh ich mein Dasein! (sie wendet sich und tritt der
					Oberstin auf ihr Kleid)\pend
           \pstart
           Oberstin: (wütend) Jetzt treten Sie mir noch die Schleppe ab! \uline{Das} ist zu viel! Sie – Sie – Hyäne des
					Schlachtfeldes!\pend
           \pstart
           Maria: (sanft) Was habe ich Ihnen getan, Frau Oberst?\pend
           \pstart
           Oberstin: Sie fragen noch? Mein Mann erschiesst mich, und an meiner Leiche hängen
					Sie sich an den Hals meines Geliebten?\pend
           \pstart
           Albrecht: (zu Max) Das sind \so{Deine} Zusammenhänge,
					Max!\pend
           \pstart
           Marie: Und \so{mein} Schicksal vergessen Sie? Ich habe mich
					ihm hingegeben, und aus meinen Armen ist er gegangen, sich umbringen für eine
					Andere? \label{K_L01900-1v}\edtext{Was bin \strikeout{ich} denn dann ich ihm gewesen?}{\lemma{\textnormal{\emph{Was … gewesen?}}}\Cendnote{\textnormal{Anspielung auf die Rede von Christine
						am Ende von \emph{\textcolor{green}{Liebelei}}: »Und ich {\dots} was bin denn ich? was bin denn ich ihm
								gewesen{\dots}«.}}}\label{K_L01900-1h}\pend
           \pstart
           Oberst: (hebt den Kopf wie ein altes Schlachtross, das bekannte Signale hört) Was
					sind \uline{das} für Töne?! Herr Leutnant!\pend
           \pstart
           Max: Zu Befehl, Herr Oberst!\pend
           \pstart
           Oberst: Ich komme um eine Kleinigkeit zu holen.\pend
           \pstart
           Max: Herr Oberst?\pend
           \pstart
           {\pb}Oberst: Meine Frau hat sich
					gestern bei Ihnen vergessen!\pend
           \pstart
           Max: Herr Oberst scherzen.\pend
           \pstart
           Oberst: (stark) Sie \uline{hat} sich vergessen.\pend
           \pstart
           Sebastian: Melde gehorsamst, Herr Oberst, da liegt sie \uline{noch}. (er will sie aufheben.)\pend
           \pstart
           Oberst: Lassen Sie! (zu Max) Ich will \strikeout{nicht}, dass
					man sie später bei Ihnen finde!\pend
           \pstart
           Marie: (will auf Max zueilen) Max!\pend
           \pstart
           Der alte Moser: (hält sie am Fuss fest) Geh nicht weg von mir, Marie! Ich hab
					dich gequält; ich bin ein alter kranker Mann von neunundsiebzig Jahren! Vergib
					mir!\pend
           \pstart
           Marie: (sanft) Ich vergebe Dir!\pend
           \pstart
           Oberstin: (wild) Sie vergibt \so{Ihnen}, Sie vergibt \so{Sie}, sie vergibt in allen Fällen! Ein sanftes
					Mädchen, Ihre Tochter! Und Sie, Herr Rittmeister, sollten sich auch schämen!
					Hören Sie auf mich zu zwicken! Sonst steh ich auf!\pend
           \pstart
           Oberst: Irene!? Welcher Rittmeister zwickt dich?\pend
           \pstart
           Max: Herr Oberst, wir sind beide vom Regiment der Geweihten.\pend
           \pstart
           Oberstin: Der Herr Moser!\pend
           \pstart
           Oberst: Woher \label{T_L01900_1v}\edtext{weisst}{\lemma{\textnormal{\emph{weisst}}}\Cendnote{\textnormal{korrigiert aus:
						»wiesst« der Vorlage.}}}\label{T_L01900_1h} Du Irene, dass er Rittmeister
					ist?\pend
           \pstart
           Sebastian: Melde gehorsamst: Am Zwick, Herr Oberst!\pend
           \pstart
           Katharina: Stecken Sie mir die Locken auf, Sebastian.!\pend
           \pstart
           Sebastian: (fängt an, sie zu frisieren.)\pend
           \pstart
           Oberst: Herr Rittmeister, Sie werden diesen Mord (auf Irene weisend) auf sich
					nehmen und sich morgen Früh standrechtlich erschiessen lassen. (bitter höhnend)
					Es wird Ihnen nicht schwer fallen, Sie sind ja das Sterben gewöhnt!\pend
           \pstart
           Der alte Moser: Ich bin ein alter Mann von neunundsiebzig Jahren – –\pend
           \pstart
           {\pb}Oberst: Und erst
					Rittmeister?\pend
           \pstart
           Katharina: Noch diese Locke, Sebastian!\pend
           \pstart
           Oberst: Wo haben Sie gedient, Herr Rittmeister?\pend
           \pstart
           Der alte Moser: Wir sind alle blaue Kürrassiere!\pend
           \pstart
           Oberst: Oh, über die verschlungenen Schicksalswege! So sind Sie der Rittmeister
					Moser, den ich erfunden habe?\pend
           \pstart
           Katharina: Noch eine Locke hier, Sebastian!\pend
           \pstart
           Sebastian: (eine Locke aufsteckend) \so{Noch} ein Dreh!\pend
           \pstart
           Oberst: Da wären Sie ja an allem Schuld, was uns heute trifft!\pend
           \pstart
           \introOben{}\strikeout{Rittmeister}\introOben{}Der alte Moser: Wenn Sie mich aber erfunden haben!\pend
           \pstart
           Oberst: (bitter) Das hat Sie nicht gehindert wirklich zu sein.\pend
           \pstart
           Arzt: Jetzt erkennen wir es, Herr Moser! Sie sind an allem schuld! Und an Ihrem
					eigenen Tod trifft Ihre arme Tochter kein Verschulden. Sie selbst haben sich
					umgebracht. Wären Sie damals, anstatt feige davonzulaufen, den Heldentod
					gestorben – Sie hätten nie geheiratet, hätten nie eine Tochter (die Sie
					notgedrungen vergiften musste) gehabt – und wären heute, – mit Ausnahme kleiner
					Altersbeschwerden frisch und gesund!\pend
           \pstart
           Sebastian: (eine neue Locke aufsteckend, sehr begeistert) \uline{Noch} ein Dreh!\pend
           \pstart
           Oberstin: Um Ihretwillen, Sie alter Feigling, muss mein Ma\substVorne{}\textsuperscript{nn}\substDazwischen{}x\substHinten{} sterben!\pend
           \pstart
           Oberst: Um Ihretwillen, Herr Rittmeister, ist ein ganzes Regiment totgeweiht!\pend
           \pstart
           Der Arzt: Sie sind an allem schuld, Herr Moser!\pend
           \pstart
           Der alte Moser: Jetzt ist’s zu viel und vor allem, Herr Doktor, sagen Sie nicht
					Herr Moser, sondern Rittmeister zu mir. \so{Mein}
					Davonlaufen ist an allem schuld? Ja, {\pb}dann ist mein Vater daran
					schuld, weil er meine Mutter geheiratet hat, und so fort, bis auf Adam und Eva!
					Muss ich alter Mann von neunundsiebzig Jahren Ihnen sagen, dass man sich nicht
					auf Kausalitäten einlassen soll, weil sonst eine Konfusion herauskommt!\pend
           \pstart
           Souffleur: Es \so{giebt} keine Kausalitäten!\pend
           \pstart
           Der alte Moser: Lassen Sie mich ausreden!\pend
           \pstart
           Souffleur: Sie \substVorne{}\textsuperscript{werden}{\allowbreak}\substDazwischen{}wären\substHinten{} der Erste, dem \uline{ich} das gestattet hätte!
					Es gibt keine Ursachen, und keine Wirkungen! Eine Wirkung ist eine Ursache, die
					noch lebt, sonst könnte sie nicht mehr wirken!\pend
           \pstart
           Sebastian: (selig fri\strikeout{e}sierend): \so{Noch} ein Dreh!\pend
           \pstart
           Der alte Moser: Das versteh’ ich nicht! Aber, zum Teufel, merken Sie denn nicht,
					dass Sie Alle meinem Davonlaufen Ihr Leben verdanken? Glauben Sie, mein
					Heldentod hatte den Dichter interessiert?\pend
           \pstart
           Sebastian: (jubilierend) \so{Noch} ein Dreh!\pend
           \pstart
           Oberst: Ihr Davonlaufen, ist doch überhaupt eine witzige Erfindung von \so{mir}! \so{Ich} kann darauf stolz
					sein, dass Sie davongelaufen sind, aber doch nicht \so{Sie}.\pend
           \pstart
           Sebastian: (dem Wahnsinn nah) \so{Noch} ein Dreh!\pend
           \pstart
           Arzt: Nun, Herr Oberst, wenn Sie sich aber gar so viel einbilden auf das Elend,
					das Sie mit Ihrer witzigen Erfindung angerichtet haben, so muss ich Ihnen schon
					sagen: \so{wir} haben keinen Grund Ihnen dankbar zu sein.
					Ertrinkende sind keine Menschen für ein Drama. Und in dem Stück sind fast alle
					am Ersaufen. Katharina und der Herr Moser und die beiden Herrn Leutnants und Sie
					auch, Herr Oberst. Für das, was einer tut, der weiss, dass er morgen {\pb}sterben soll, kann man ihn
					nicht mehr zur Rechenschaft ziehen; das ist kein Lebender mehr. Der hat nicht
					mehr freien Willen, den kommandiert nur seine Todesangst, und wo es nicht freien
					Willen gibt – gibt es kein Drama!\pend
           \pstart
           Oberst: Schade, dass Ihre Weisheit so kurzen Atem hat, wie der Herr Moser. Es \label{T_L01900-3v}\edtext{\so{gab}}{\lemma{\textnormal{\emph{gab}}}\Cendnote{\textnormal{zusätzlich noch handschriftlich
							unterstrichen}}}\label{T_L01900-3h} keines – aber es \label{T_L01900-2v}\edtext{\so{gibt}}{\lemma{\textnormal{\emph{gibt}}}\Cendnote{\textnormal{zusätzlich noch handschriftlich
							unterstrichen}}}\label{T_L01900-2h} eines: Sie selbst spielen ja darin!\pend
           \pstart
           Sebastian: (entrüstet) \so{Den} Dreh mach’ ich nicht
					mit!\pend
           \pstart
           Arzt: So werde ich Ihnen nach der Vorstellung sagen – – –\pend
           \pstart
           Oberst: Das können Sie nicht! Sie sind nur \so{in} der
					Vorstellung!\pend
           \pstart
           Sebastian: (schreit) Aufhören!\pend
           \pstart
           Arzt: So werde ich Ihnen sagen, dass Tod und Leben Voraussetzungen sind – nicht
					Stoffe. Tod und Leben sind unverantwortlich und stehen niemandem Rede; und dass
					einer Dichter ist, heisst nur, dass er manchmal – nicht zu oft – nach ihnen
					fragen darf. Fragen! Ohne Antwort zu bekommen! Und schön muss er fragen – sehr
					schön!\pend
           \pstart
           Oberst: Sie wollen mich wohl belehren, Herr Doktor? Schweigen Sie endlich!\pend
           \pstart
           Arzt: Herr Oberst haben Ihrem Regiment zu befehlen – nicht mir!\pend
           \pstart
           Oberst: Sie irren, mein Lieber! Auch Ihnen! Sie sind – wie ich selbst – von \uline{meinen} Gnaden! (er springt ins Zimmer, die Maske
					fällt ab – der Dichter steht da).\pend
           \pstart
           Sebastian: (wimmernd in die Knie sinkend) \so{Kein} Dreh
					mehr! Keiner mehr! Wenn \so{ich} schon nicht mehr mit
					kann!\pend
           \pstart
           Dichter: (nach vorne kommend, zündet sich eine Virginia an, und sagt zum
					Souffleur, in den Kasten hinunter, scharf): {\pb}Die letzte Rede des Arztes
					haben doch \so{Sie} souffliert!\pend
           \pstart
           Souffleur: Ja mein Lieber – ebenso wie Ihre Reden!\pend
           \pstart
           Sebastian: (wimmert auf, der Arzt unterstützt ihn und fühlt ihm den Puls).\pend
           \pstart
           Dichter: Wieso? Ja so! Natürlich! Aber hören Sie auf, Sie sehen doch, der Mann
					stirbt bereits an Ihren Drehs!\pend
           \pstart
           Souffleur: An meinen? Doch auch an Ihren!\pend
           \pstart
           Dichter: Ja, ja! Aber nach dem Nachtmahl ist mir das zu anstrengend. (Er will
					sich auf die Barriere setzen).\pend
           \pstart
           Souffleur: (aufschreiend) Nicht auf die Barriere! Das ist keine, das ist der
					letzte Akt: grün und stark gestrichen! Auf den wird sich die Kritik setzen!\pend
           \pstart
           Dichter: Keine Witze jetzt! Sagen Sie übrigens: Es ist ja sehr ehrenvoll für
					mich, aber – haben Sie wirklich nichts anderes zu tun als sich meine Figuren
					herzunehmen, und mit ihnen Schindluder zu treiben? \textcolor{blue}{Fischer}{}\ledrightnote{\textcolor{blue}{Samuel Fischer}} würde sagen: »Ein ausgeruhter Kopp!« Bei Ihnen
					kann man sich ja nicht einmal revanchieren. Bis zu Ihrer Generalprobe bin ich
					bestenfalls so alt wie der alte Moser! Warum verschwenden Sie so viel
					Geist – –\pend
           \pstart
           Arzt: (um Sebastian bemüht) Ein Glas Wasser, bitte.\pend
           \pstart
           Dichter: (fortfahrend) – – da sehen Sie – so viel Geist an fertige Figuren,
					die ihn wirklich nicht brauchen? Geben Sie Ihren davon – ich meine den noch
					unfertigen! Bei einer Pentalogie kann man nie davon genug haben! Uebrigens, mir
					fällt ein: Sie könnten eine Familienfideikomisstiftung aus der Pent\strikeout{h}alogie machen. Immer der älteste Sohn hat daran
					zu schreiben, und wenn einer Ihrer Nachkommen sie wirklich fertig macht, so
						wird{ }{\pb}er enterbt – weil er aus der
					Art geschlagen ist! Wissen Sie: \so{wenn} Sie schon das
					Stück kritisieren wollen, gestalten Sie nicht an meinen Gestalten herum –
					sondern schreiben Sie einen Essay für die »\textcolor{brown}{Rundschau}{}\ledrightnote{\textcolor{brown}{Neue Rundschau, Neue Deutsche Rundschau, Freie Bühne}}« – das wird wenigstens klarer und deutlicher sein.\pend
           \pstart
           Souffleur: Deutlicher vielleicht – aber das war nicht meine Absicht!\pend
           \pstart
           Dichter: Hetzen Sie doch nicht den Satz zu Tode – er ist sehr gut!\pend
           \pstart
           Souffleur: Eigentlich, lieber Arthur, ist es recht unfreundlich von Ihnen, mir
					das von der Pentalogie – wenn auch im Scherze – so vor allen den Leuten zu
					sagen! Sie hätten mir das auch unter vier Augen sagen können – das wäre
					liebenswürdiger gewesen.\pend
           \pstart
           Dichter: Liebenswürdiger vielleicht! Aber das war nicht meine Absicht!\pend
           \pstart
           Souffleur: Hetzen Sie doch den Satz nicht zu Tode! Er ist sehr gut!\pend
           \pstart
           Dichter: Uebrigens, das, was Sie den Arzt da sagen lassen, von der Kausalität –
					ist recht couragiert von Ihnen. Sie spielen den Krieg in Feindesland! \so{Sie} – als Verteidiger der Kausalität! Wissen Sie, was
					Sie sind??\pend
           \pstart
           Souffleur: Meiner Bescheiden\introOben{}heit\introOben{}, lieber Arthur, ist es
					wohl zuzutrauen, dass ich weiss, was ich bin!\pend
           \pstart
           Dichter: Sie sind: »\label{K_L01900_1v}\edtext{Grachi de
					seditione quaerentes}{\lemma{\textnormal{\emph{Grachi … quaerentes}}}\Cendnote{\textnormal{Umwandlung
						einer lateinischen Redewendung in den Singular: »Quis tulerit Gracchos de
						seditione quaerentes« (»Wer ertrüge es, wenn die Gracchen sich über Aufruhr
						beklagen«).}}}\label{K_L01900_1h}«! Bombenwerfer, die über Knallbonbons sich beklagen! \so{Sie} verlangen Kausalität in einem Drama! Ich krieg
					ordentlich eine Wut, wenn ich mir das vorstelle! (er bricht erbittert ein Stück
					von seiner Virginia, die nicht brennt, ab) Ausgerechnet \so{Sie} machen mir Vorwürfe! {\pb}\so{Sie}, der Sie – – Sie – (wütend auflachend) \so{Sie}, \so{Sie}: »Es geschah«
					Sie!\pend
           \pstart
           Sebastian: (interessiert aufhorchend) Eschkenasi?? Von welchem Eschkenasi sind
					Sie – –\pend
           \pstart
           Souffleur: (milde) Bestehen Sie noch immer darauf, dass Sie \strikeout{»Sesbas} »Sebastian« heissen?\pend
           \pstart
           Sebastian: (hat sich aufgerichtet; respektlos, in herzlicher Gemütlichkeit,
					fraternisierend) Sind Sie nicht bös, Herr Dichter – und Herr Eschkenasi – wir
					sind Alle blaue Kürrassiere!\pend
           \pstart
           \centering{}\uline{Der Vorhang fällt.}\pend
           \endnumbering\briefempfaengerindex{Schnitzler, Arthur@\textsc{Schnitzler, Arthur}!zzzBeer-Hofmann, Richard@\emph{von Richard Beer-Hofmann}!1909-12-131@{{[}13. 12. 1909?{]}}|)be}\mylabel{h}  
         \normalsize

\newenvironment{esempio}[3]%
{
    \vspace{1.5ex}
    \rlap{\underline{#1}}
    \par
    \setlength{\parindent}{0cm}
    \nopagebreak
    \leftskip=#2cm
    \rightskip=#3cm
}
{
    \par
}

\doendnotes{C}
\bigskip

\printindex[pw]


\end{document}
      