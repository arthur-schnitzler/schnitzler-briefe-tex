%% latex-korrekturansicht-vorspann.tex
%% Vorspann für die Korrekturansicht.
%% Lädt die gemeinsame Datei latex-vorspann.tex mit gesetztem Schalter.

\newif\ifkorrekturansicht
\korrekturansichttrue

\input{../tex-inputs/latex-vorspann}


               \section[Hugo Hofmannsthal an Arthur Schnitzler, 2. 11. 1919]{ Hugo Hofmannsthal an Arthur Schnitzler, 2. 11. 1919}\nopagebreak\mylabel{v}\rehead{ }\normalsize\beginnumbering\briefempfaengerindex{Schnitzler, Arthur@\textsc{Schnitzler, Arthur}!zzzHofmannsthal, Hugo von@\emph{von Hugo von Hofmannsthal}!1919-11-021@{2. 11. 1919}|(be} \toendnotes[C]{\smallbreak\pagebreak[2]} \Standort{CUL, Schnitzler, B 43.}
\physDesc{Brief, 1 Blatt, 4 Seiten
\newline{}Handschrift: schwarze Tinte, deutsche Kurrent
\newline{}Schnitzler: 1) mit Bleistift die Jahreszahl »19« ergänzt 2) mit rotem Buntstift einzelne Unterstreichungen\newline{}Ordnung: 1) mit Bleistift von \textcolor{blue}{Frieda
                                    Pollak} (?) mit dem Buchstaben »A«
                                 (Abgeschrieben/Abschrift) gekennzeichnet 2) mit Bleistift von unbekannter Hand nummeriert: »\strikeout{354}«3) mit Bleistift von unbekannter Hand nummeriert: »?
                                    383«, bei der von Schnitzler ergänzten Jahreszahl
                                 ebenfalls ein Fragezeichen hinzugefügt}\buchAbdrucke{\weitereDrucke{Hugo von Hofmannsthal, Arthur Schnitzler: \emph{Briefwechsel}. Hg. Therese Nickl und Heinrich Schnitzler. Frankfurt am Main: \emph{S. Fischer} 1964, S. 287.} }\toendnotes[C]{\smallbreak}\pstart
           \raggedleft{}{\pb}\textcolor{pink}{Bad Aussee}{}\ledrightnote{\textcolor{pink}{Bad Aussee}}{ }2 XI 19\pend
           \pstart{}mein lieber Arthur\pend\pstart
           Sie haben mir vor mehr als einem Monat einen ſo lieben ſchönen Brief hierher
               geſchrieben – ich dank Ihnen vielmals dafür.\hspace*{1.5em}Über
               unſere Vorleſungen denk ich ſo wie Sie: ſie ſind mir auch als Feſte ganz beſonderer
               Art in der Erinnerung, und am ſtärkſten und beſonderſten von allen \label{K_L02331_1v}\edtext{die des »\textcolor{green}{Märchens}{}\ledrightnote{\textcolor{green}{Das Märchen. Schauspiel in drei Aufzügen}}«}{\lemma{\textnormal{\emph{die des »Märchens«}}}\Cendnote{\textnormal{am
                     25. 6. 1891}}}\label{K_L02331_1h} in \textcolor{blue}{Richard}{}\ledrightnote{\textcolor{blue}{Richard Beer-Hofmann}}s verhängter u. nach Naphtalin
               riechender Wohnung in der \label{K_L02331_2v}\edtext{\textcolor{pink}{Gärtnergaſſe}{}\ledrightnote{\textcolor{pink}{Gärtnergasse}}}{\lemma{\textnormal{\emph{Gärtnergaſſe}}}\Cendnote{\textnormal{Vermutlich eine Verwechslung, er dürfte
                  eine Parallelstraße meinen, die \textcolor{pink}{Seidlgasse}.}}}\label{K_L02331_2h} – aber auch manche Andere, ſo \label{K_L02331_3v}\edtext{ein Abend}{\lemma{\textnormal{\emph{ein Abend}}}\Cendnote{\textnormal{am 11. 4. 1904, in
                  Anwesenheit von \textcolor{blue}{Schwarzkopf}}}}\label{K_L02331_3h} wo Sie mir
               ganz allein – oder mir und \textcolor{blue}{Schwarzkopf}{}\ledrightnote{\textcolor{blue}{Gustav Schwarzkopf}} – in der
               Wohnung, die Sie vor dieſer jetzigen zuletzt bewohnten – die Geſchichte des \textcolor{green}{Freiherrn von \textsc{Leisenbogh}}{}\ledrightnote{\textcolor{green}{Das Schicksal des Freiherrn von Leisenbohg. Novellette}} vorlaſen, die ich ſo beſonders liebe.\pend
           \pstart
           \label{T_L02331_1v}\edtext{Wenn}{\lemma{\textnormal{\emph{Wenn}}}\Cendnote{\textnormal{Absatztrennmarkierung nachträglich mit Bleistift eingefügt}}}\label{T_L02331_1h}{ }{\pb}ich das \textcolor{green}{Geſellſchaftsluſtſpiel}{}\ledrightnote{→\textcolor{green}{Der Schwierige. Lustspiel in drei Akten}} fertig habe, an dem ich
               immer noch im Einzelnen herumbeſſere, ſo freue ich mich recht ſehr, es Ihnen, ſei es
               Ihnen allein oder mit noch ein paar Menſchen, zu leſen. Vielleicht hätte ich die
               Geſellſchaft, die es darſtellt, die Oeſterreichiſche \strikeout{arſtr} ariſtokratiſche Geſellſchaft, nie mit ſo viel Liebe in ihrem \textsc{charme} und ihrer Qualität darſtellen können als in dem
               hiſtoriſchen Augenblick wo ſie, die bis vor kurzem eine Gegebenheit, ja eine Macht
               war, ſich leiſe u. geiſterhaft ins Nichts auflöst, wie {\pb}ein übriggebliebenes Nebelwölkchen
               am Morgen.\pend
           \pstart
           Inzwiſchen iſt das Märchen von der \textcolor{green}{Frau ohne
                  Schatten}{}\ledrightnote{\textcolor{green}{Die Frau ohne Schatten. Erzählung}} zu Ihnen gewandert, und, hoffentlich, ſeit langem in Ihren
               Händen.\pend
           \pstart
           Ich habe, in faſt ſieben Jahren, unſäglich viel Mühe an dieſe kleine Arbeit gewandt –
               hoffentlich merkt man ihr dies nicht an. Wenn ſie Ihnen und \textcolor{blue}{Olga}{}\ledrightnote{\textcolor{blue}{Olga Schnitzler}} ein bischen Vergnügen gemacht hat, ſo ſchreiben Sie mir
               ein paar Zeilen darüber – weſſen Beifall ſollte man denn wünſchen u. ſuchen, als der
               paar Menſchen mit {\pb}denen und durch
               die man das Leben gelebt hat.\pend
           \pstart
           Adieu, Arthur.\hspace*{1.5em}Im Vorübergehen möcht ich Sie auf ein
               ſehr kluges, zu vielem Denken anregendes Buch aufmerkſam machen, das mir dieſe
               letzten etwas unproductiveren Föhntage ſehr bereichert hat: \textsc{\textcolor{blue}{Keyserling}{}\ledrightnote{\textcolor{blue}{Hermann von Keyserling}}s}{ }\textcolor{green}{Reiſetagebuch eines
                  Philoſophen}{}\ledrightnote{\textcolor{green}{Das Reisetagebuch eines Philosophen}}.{\\}Ihr\spacefill\mbox{Hugo}\pend
           \pstart
           \noindent{}PS. Iſt es denn richtig daſs ein abſurdes Geſetz einem \textcolor{blue}{Händler}{}\ledrightnote{→\textcolor{blue}{Karl Ernst Henrici}} der \textcolor{blue}{Brahm}{}\ledrightnote{\textcolor{blue}{Otto Brahm}}s ganzen Briefwechſel gekauft hat, jetzt das Recht gibt, unſere ſo
                  ganz vertraulichen Briefe an den Todten, ob wir wollen oder nicht, zu publicieren?
               \pend
           \endnumbering\briefempfaengerindex{Schnitzler, Arthur@\textsc{Schnitzler, Arthur}!zzzHofmannsthal, Hugo von@\emph{von Hugo von Hofmannsthal}!1919-11-021@{2. 11. 1919}|)be}\mylabel{h}  \normalsize

\doendnotes{C}
\bigskip
\vfill

\clearpage

\footnotesize

\lohead{\textsc{register}}

% Definiere theindex-Environment komplett neu ohne reledmac
\makeatletter
\renewenvironment{theindex}{%
  \section*{\indexname}%
  \setlength{\parindent}{0pt}%
  \setlength{\parskip}{0pt plus 0.3pt}%
  \let\item\@idxitem
}{%
  \clearpage
}
\makeatother

\IfFileExists{\jobname-pw.ind}{\input{\jobname-pw.ind}}{}

\end{document}

      