%% latex-korrekturansicht-vorspann.tex
%% Vorspann für die Korrekturansicht.
%% Lädt die gemeinsame Datei latex-vorspann.tex mit gesetztem Schalter.

\newif\ifkorrekturansicht
\korrekturansichttrue

\input{../tex-inputs/latex-vorspann}


               \section[Arthur und Olga Schnitzler u. a. an Richard Beer-Hofmann, 15. 7. 1914]{ Arthur und Olga Schnitzler u. a. an Richard Beer-Hofmann,
               15. 7. 1914}\nopagebreak\mylabel{v}\rehead{ }\normalsize\beginnumbering\briefempfaengerindex{Beer-Hofmann, Richard@\textsc{Beer-Hofmann, Richard}!zzzSchmidl, Paula@\emph{von Paula Schmidl}!1914-07-151@{15. 7. 1914}|(be}\briefempfaengerindex{Beer-Hofmann, Richard@\textsc{Beer-Hofmann, Richard}!zzzSchmidl, Hugo@\emph{von Hugo Schmidl}!1914-07-151@{15. 7. 1914}|(be}\briefempfaengerindex{Beer-Hofmann, Richard@\textsc{Beer-Hofmann, Richard}!zzzSchnitzler, Olga@\emph{von Olga Schnitzler}!1914-07-151@{15. 7. 1914}|(be}\briefempfaengerindex{Beer-Hofmann, Richard@\textsc{Beer-Hofmann, Richard}!zzzSchnitzler, Arthur@\emph{von Arthur Schnitzler}!1914-07-151@{15. 7. 1914}|(be} \toendnotes[C]{\smallbreak\pagebreak[2]} \Standort{YCGL, MSS 31.}
\physDesc{Bildpostkarte
\newline{}Handschrift Arthur Schnitzler: Bleistift, deutsche Kurrent\newline{}Handschrift Paula Schmidl: Bleistift\newline{}Handschrift Olga Schnitzler: Bleistift, lateinische Kurrent\newline{}Handschrift Hugo Schmidl: Bleistift, lateinische Kurrent\newline{}Versand: Stempel: »\nobreak{}1\textcolor{gray}{5}. VII. 14\nobreak{}«.  
\newline{}Beer-Hofmann: mit blauem Buntstift Vermerk: »E« (für
                                 »Erhalt«?) }\toendnotes[C]{\smallbreak}\pstart{}{\pb}Hrn \textsc{Dr. Richard Beer-Hofmann}\pend{}\pstart{}\textsc{\textcolor{pink}{Weissenbach}{}\ledrightnote{\textcolor{pink}{Weißenbach am Attersee}}}\pend{}\pstart{}\textsc{Am \textcolor{pink}{Attersee}{}\ledrightnote{\textcolor{pink}{Attersee}}}\pend{}\pstart{}\textcolor{pink}{\textsc{Ob.Oe.}}{}\ledrightnote{\textcolor{pink}{Oberösterreich}}\pend{}{\bigskip}\pstart
           \noindent{}\centering{}{\pb}\textcolor{gray}{\textbf{\textcolor{pink}{Mariazell}{}\ledrightnote{\textcolor{pink}{Mariazell}}, 862 m Seehöhe, \textcolor{pink}{Steiermark}{}\ledrightnote{\textcolor{pink}{Steiermark}} gegen \textcolor{pink}{Hochschwab}{}\ledrightnote{\textcolor{pink}{Hochschwab}}.}}\pend
           \pstart
           {\pb}Herzliche Grüße Ihnen Allen und vielen Dank für den
                  \label{K_L02186-1v}\edtext{Schachtelkäſe}{\lemma{\textnormal{\emph{Schachtelkäſe}}}\Cendnote{\textnormal{hier wohl in seiner weiter gefassten
                  Bedeutung eines in einer Schachtel verkauften Käses}}}\label{K_L02186-1h}.\pend
           \pstart
           Von einer Autopartie \textcolor{pink}{\textsc{Lunz}}{}\ledrightnote{\textcolor{pink}{Lunz am See}}–\textcolor{pink}{\textsc{Mariazell}}{}\ledrightnote{\textcolor{pink}{Mariazell}}, 15/7 914.\pend
           \pstart \spacefill\mbox{Arthur}\pend{}\pstart
           \noindent{}{[}hs. O. Schnitzler:{]} Herzlichst\pend
           \pstart \spacefill\mbox{Olga.}\pend{}\pstart \spacefill\mbox{{[}hs. Schmidl:{]} Paula Schmidl}\pend{}\pstart
           \noindent{}{[}hs. Schmidl:{]} Gruss!\pend
           \pstart \spacefill\mbox{Hugo Schmidl}\pend{}\endnumbering\briefempfaengerindex{Beer-Hofmann, Richard@\textsc{Beer-Hofmann, Richard}!zzzSchmidl, Paula@\emph{von Paula Schmidl}!1914-07-151@{15. 7. 1914}|)be}\briefempfaengerindex{Beer-Hofmann, Richard@\textsc{Beer-Hofmann, Richard}!zzzSchmidl, Hugo@\emph{von Hugo Schmidl}!1914-07-151@{15. 7. 1914}|)be}\briefempfaengerindex{Beer-Hofmann, Richard@\textsc{Beer-Hofmann, Richard}!zzzSchnitzler, Olga@\emph{von Olga Schnitzler}!1914-07-151@{15. 7. 1914}|)be}\briefempfaengerindex{Beer-Hofmann, Richard@\textsc{Beer-Hofmann, Richard}!zzzSchnitzler, Arthur@\emph{von Arthur Schnitzler}!1914-07-151@{15. 7. 1914}|)be}\mylabel{h}  \normalsize

\doendnotes{C}
\bigskip
\vfill

\clearpage

\footnotesize

\lohead{\textsc{register}}

% Definiere theindex-Environment komplett neu ohne reledmac
\makeatletter
\renewenvironment{theindex}{%
  \section*{\indexname}%
  \setlength{\parindent}{0pt}%
  \setlength{\parskip}{0pt plus 0.3pt}%
  \let\item\@idxitem
}{%
  \clearpage
}
\makeatother

\IfFileExists{\jobname-pw.ind}{\input{\jobname-pw.ind}}{}

\end{document}

      