%% latex-korrekturansicht-vorspann.tex
%% Vorspann für die Korrekturansicht.
%% Lädt die gemeinsame Datei latex-vorspann.tex mit gesetztem Schalter.

\newif\ifkorrekturansicht
\korrekturansichttrue

\input{../tex-inputs/latex-vorspann}


               \section[Hermann Bahr an Arthur Schnitzler, 23. 7. 1895]{ Hermann Bahr an Arthur Schnitzler, 23. 7. 1895}\nopagebreak\mylabel{v}\rehead{ }\normalsize\beginnumbering\briefempfaengerindex{Schnitzler, Arthur@\textsc{Schnitzler, Arthur}!zzzBahr, Hermann@\emph{von Hermann Bahr}!1895-07-231@{23. 7. 1895}|(be} \toendnotes[C]{\smallbreak\pagebreak[2]} \Standort{CUL, Schnitzler, B 5b.}
\physDesc{Brief, 1 Blatt, 2 Seiten
\newline{}Handschrift: schwarze Tinte, deutsche Kurrent\newline{}Ordnung: 1) mit rotem Buntstift von unbekannter Hand nummeriert: »30« 2) mit Bleistift von unbekannter Hand nummeriert: »30«}\buchAbdrucke{\weitereDrucke{Hermann Bahr, Arthur Schnitzler: \emph{Briefwechsel, Aufzeichnungen, Dokumente (1891–1931)}. Hg. Kurt Ifkovits und Martin Anton Müller. Göttingen: \emph{Wallstein} 2018, S. 104.} }\toendnotes[C]{\smallbreak}\pstart
           \noindent{}{\pb}\textcolor{gray}{\textbf{»\textcolor{brown}{Die
                        Zeit}{}\ledrightnote{\textcolor{brown}{Die Zeit. Wiener Wochenschrift}}«}}\hfill \textcolor{gray}{\textbf{\textbf{\textcolor{pink}{Wien}{}\ledrightnote{\textcolor{pink}{Wien}}}, den }}23. Juli \textcolor{gray}{\textbf{189}}5\pend
           \pstart
           \textcolor{gray}{\textbf{Wiener Wochenſchrift}}\hfill \textcolor{gray}{\textbf{\textcolor{pink}{IX/3, Günthergaſſe 1}{}\ledrightnote{\textcolor{pink}{Günthergasse}}.}}\pend
           \pstart
           \textcolor{gray}{\textbf{\textbf{Herausgeber}:}}{\\}\textcolor{gray}{\textbf{Profeſſor Dr. \textcolor{blue}{I. Singer}{}\ledrightnote{\textcolor{blue}{Isidor Singer}}, \textcolor{blue}{Hermann Bahr}{}\ledrightnote{\textcolor{blue}{Hermann Bahr}},
                        Dr. \textcolor{blue}{Heinrich Kanner}{}\ledrightnote{\textcolor{blue}{Heinrich Kanner}}.}}\pend
           \pstart
           \textcolor{gray}{\textbf{Telephon Nr. 6415.}}\pend
           \pstart{}Lieber Freund!\pend\pstart
           Ich habe die »\textcolor{green}{Geſchichte von einem greiſen
                  Dichter}{}\ledrightnote{\textcolor{green}{Später Ruhm}}« ſofort geleſen und dann, nachdem ich ſie einige Tage bei mir
               erwogen, noch einmal. Als Redacteur muß ich nun ſagen, daß ich eine ſo lange, dabei
               doch dünne Geſchichte von ſchwacher Handlung und nicht ſehr deutlichen Geſtalten
               durch Zerſtückelung in etwa acht Partieen, mit Pauſen von acht Tagen, ſchädigen und
               um jede Wirkung bringen würde. Wenn ich auch als Kritiker reden darf, ſo möchte ich
               nicht verhehlen, daß mir die Novelle von unmäßiger Länge und {\pb}einer gewiſſen, nicht in der Sache liegenden
               Schwere scheint, indem ein heiterer, aber nur bei Kürze und Leichtigkeit wirkſamer
               Gedanke allzu gewaltſam hinausgezogen wird. Davon hoffe ich mit Dir anfangs Auguſt in
                  \textcolor{pink}{\textsc{Ischl}}{}\ledrightnote{\textcolor{pink}{Bad Ischl}} zu ſprechen und dem Redacteur wäre es lieb, wenn Du Dich entſchließen könnteſt,
               es auf ein Drittel zu kürzen, was der Kritiker auch aus inneren Gründen billigen, ja
               fordern müßte. Jedenfalls danke ich Dir für die Sendung des \textsc{Mnscr}. ſehr und grüße Dich wie \textcolor{blue}{Richard}{}\ledrightnote{\textcolor{blue}{Richard Beer-Hofmann}} herzlich\pend
           \pstart
           als Dein treuer{\\[\baselineskip]}\spacefill\mbox{HermBahr}\pend
           \leftskip=0em{}\pstart
           \noindent{}Herrn \textsc{Dr Arthur Schnitzler, \textcolor{pink}{Ischl}{}\ledrightnote{\textcolor{pink}{Bad Ischl}}}\pend
           \leftskip=3em{}\pstart
           \noindent{}reco{\geminationm}andieren.\pend
           \leftskip=0em{}\pstart
           \textcolor{gray}{\textbf{\label{T_L00465_1v}\edtext{Alle für »\textcolor{brown}{Die Zeit}{}\ledrightnote{\textcolor{brown}{Die Zeit. Wiener Wochenschrift}}« beſtimmten Zuſchriften und Sendungen ſind an
                  die Redaction der »\textcolor{brown}{Zeit}{}\ledrightnote{\textcolor{brown}{Die Zeit. Wiener Wochenschrift}}« und \textbf{nicht} an die Perſon eines der Herausgeber zu richten.}{\lemma{\textnormal{\emph{Alle … richten.}}}\Cendnote{\textnormal{am unteren Rand der ersten Seite}}}\label{T_L00465_1h}}}\pend
           \endnumbering\briefempfaengerindex{Schnitzler, Arthur@\textsc{Schnitzler, Arthur}!zzzBahr, Hermann@\emph{von Hermann Bahr}!1895-07-231@{23. 7. 1895}|)be}\mylabel{h}  \normalsize

\doendnotes{C}
\bigskip
\vfill

\clearpage

\footnotesize

\lohead{\textsc{register}}

% Definiere theindex-Environment komplett neu ohne reledmac
\makeatletter
\renewenvironment{theindex}{%
  \section*{\indexname}%
  \setlength{\parindent}{0pt}%
  \setlength{\parskip}{0pt plus 0.3pt}%
  \let\item\@idxitem
}{%
  \clearpage
}
\makeatother

\IfFileExists{\jobname-pw.ind}{\input{\jobname-pw.ind}}{}

\end{document}

      