%% latex-korrekturansicht-vorspann.tex
%% Vorspann für die Korrekturansicht.
%% Lädt die gemeinsame Datei latex-vorspann.tex mit gesetztem Schalter.

\newif\ifkorrekturansicht
\korrekturansichttrue

\input{../tex-inputs/latex-vorspann}


               \section[Arthur Schnitzler an Richard Beer-Hofmann, 24. 7. 1896]{ Arthur Schnitzler an Richard Beer-Hofmann,
               24. 7. 1896}\nopagebreak\mylabel{v}\rehead{ }\normalsize\beginnumbering\briefempfaengerindex{Beer-Hofmann, Richard@\textsc{Beer-Hofmann, Richard}!zzzSchnitzler, Arthur@\emph{von Arthur Schnitzler}!1896-07-241@{24. 7. 1896}|(be} \toendnotes[C]{\smallbreak\pagebreak[2]} \Standort{YCGL, MSS 31.}
\physDesc{Briefkarte, Umschlag
\newline{}Handschrift: Bleistift, deutsche Kurrent\newline{}Versand: 1) Stempel: »\nobreak{}\oindex{Oslo@\textbf{Oslo}, \emph{http://www.geonames.org/ontologyP.PPLC}|pwk}Christiania, 24 VII {[}96{]}\nobreak{}«.  2) Stempel: »\nobreak{}\oindex{Kopenhagen@\textbf{Kopenhagen}, \emph{Besiedelter Ort (A.BSO)}|pwk}Kjøbenhavn, 25 7 96\nobreak{}«. }\buchAbdrucke{\weitereDrucke{Arthur Schnitzler, Richard Beer-Hofmann: \emph{Briefwechsel 1891–1931}. Hg. Konstanze Fliedl. Wien, Zürich: \emph{Europaverlag} 1992, S. 93.} }\pstart{}{\pb}Herrn \textsc{Dr. Richard
                     Beer-Hofmann}\pend{}\pstart{}\textsc{\textcolor{pink}{Kopenhagen}{}\ledrightnote{\textcolor{pink}{Kopenhagen}}}\pend{}\pstart{}\textsc{post rest.}\pend{}{\bigskip}\pstart
           \noindent{}{\pb}Lieber Richard, ich dank für Ihr Telegra{\geminationm}, das ich geſtern in \textcolor{pink}{\textsc{Tr}.}{}\ledrightnote{\textcolor{pink}{Trondheim}} vorgefunden; hoffe weitre Nachrichten; vielleicht gar Sie ſelbſt in
                  \textcolor{pink}{Stockholm}{}\ledrightnote{\textcolor{pink}{Stockholm}}. Bin um 12 nach 17ſtündg
               ziemlich ſchlafloſer Fahrt hier angelangt – es gibt keine {\pb}Tagſchnellzüge. – Habe die \textcolor{pink}{Nordcap}{}\ledrightnote{\textcolor{pink}{Nordkap}}reiſe im ganzen ſehr wohlgelaunt, nur arg durch Kopfweh
               geſtört, durchgemacht, viel ſehr ſchönes geſehn, aber nur wenige Augenblicke tiefen
               Genießens erlebt. Freue mich auf Sie. Herzlich der Ihre \spacefill\mbox{Arthur}\pend
           \endnumbering\briefempfaengerindex{Beer-Hofmann, Richard@\textsc{Beer-Hofmann, Richard}!zzzSchnitzler, Arthur@\emph{von Arthur Schnitzler}!1896-07-241@{24. 7. 1896}|)be}\mylabel{h}  \normalsize

\doendnotes{C}
\bigskip
\vfill

\clearpage

\footnotesize

\lohead{\textsc{register}}

% Definiere theindex-Environment komplett neu ohne reledmac
\makeatletter
\renewenvironment{theindex}{%
  \section*{\indexname}%
  \setlength{\parindent}{0pt}%
  \setlength{\parskip}{0pt plus 0.3pt}%
  \let\item\@idxitem
}{%
  \clearpage
}
\makeatother

\IfFileExists{\jobname-pw.ind}{\input{\jobname-pw.ind}}{}

\end{document}

      