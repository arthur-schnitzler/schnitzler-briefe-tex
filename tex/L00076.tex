%% latex-korrekturansicht-vorspann.tex
%% Vorspann für die Korrekturansicht.
%% Lädt die gemeinsame Datei latex-vorspann.tex mit gesetztem Schalter.

\newif\ifkorrekturansicht
\korrekturansichttrue

\input{../tex-inputs/latex-vorspann}


               \section[Arthur Schnitzler an Wilhelm Bölsche, 24. 2. 1892]{ Arthur Schnitzler an Wilhelm Bölsche, 24. 2. 1892}\nopagebreak\mylabel{v}\rehead{ }\normalsize\beginnumbering\briefempfaengerindex{Boelsche, Wilhelm@\textsc{Bölsche, Wilhelm}!zzzSchnitzler, Arthur@\emph{von Arthur Schnitzler}!1892-02-241@{24. 2. 1892}|(be} \toendnotes[C]{\smallbreak\pagebreak[2]} \Standort{Wrocław, Biblioteka Uniwersytecka, Böl.Pis 1762.}
\physDesc{Brief, 1 Blatt, 2 Seiten
\newline{}Handschrift: schwarze Tinte, deutsche Kurrent}\buchAbdrucke{\weitereDrucke{1) Alois Woldan: \emph{Arthur Schnitzler – Briefe an Wilhelm Bölsche.} In: \emph{Germanica Wratislaviensia} (1987) Nr. 77, S. 459.} \weitereDrucke{2) Wilhelm Bölsche: \emph{Briefwechsel. Mit Autoren der Freien Bühne}. Hg. Gerd-Hermann Susen. Berlin: \emph{Weidler} 2010, S. 676 (Werke und Briefe. Wissenschaftliche Ausgabe, Briefe I).} }\pstart
           \noindent{}{\pb}\textsc{\textcolor{pink}{Wien I Giselastraße 11}{}\ledrightnote{\textcolor{pink}{Bösendorferstraße}}}\pend
           \pstart
           \raggedleft{}24/2 92.\pend
           \pstart{}Verehrteſter Herr,\pend\pstart
           erlauben Sie mir, zwei Fragen an Sie zu richten, für deren Beantwortung ich Ihnen
                    ſehr dankbar wäre.\pend
           \pstart
           1.) Wa{\geminationn} gedenken Sie meine »\textcolor{green}{\textsc{Elixire}}{}\ledrightnote{\textcolor{green}{Die drei Elixire}}« in der \textcolor{green}{Freien Bühne}{}\ledrightnote{\textcolor{green}{Freie Bühne für den Entwickelungskampf der Zeit}} zum Abdruck zu
                    bringen?\pend
           \pstart
           2) Veröffentlichen Sie in den nächſten Heften vielleicht auch Gedichte? Ich
                    möchte {\pb}Ihnen für dieſen Fall ſehr gern welche
                    ſenden.\pend
           \pstart
           Entſchuldigen Sie, verehrteſter Herr, die verurſachte Mühe und ſeien Sie
                    meiner ausgezeichneten Hochachtung verſichert.{\\[\baselineskip]}\spacefill\mbox{Dr Arthur Schnitzler.}\pend
           \leftskip=0em{}\endnumbering\briefempfaengerindex{Boelsche, Wilhelm@\textsc{Bölsche, Wilhelm}!zzzSchnitzler, Arthur@\emph{von Arthur Schnitzler}!1892-02-241@{24. 2. 1892}|)be}\mylabel{h}  \normalsize

\doendnotes{C}
\bigskip
\vfill

\clearpage

\footnotesize

\lohead{\textsc{register}}

% Definiere theindex-Environment komplett neu ohne reledmac
\makeatletter
\renewenvironment{theindex}{%
  \section*{\indexname}%
  \setlength{\parindent}{0pt}%
  \setlength{\parskip}{0pt plus 0.3pt}%
  \let\item\@idxitem
}{%
  \clearpage
}
\makeatother

\IfFileExists{\jobname-pw.ind}{\input{\jobname-pw.ind}}{}

\end{document}

      