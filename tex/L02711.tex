%% latex-korrekturansicht-vorspann.tex
%% Vorspann für die Korrekturansicht.
%% Lädt die gemeinsame Datei latex-vorspann.tex mit gesetztem Schalter.

\newif\ifkorrekturansicht
\korrekturansichttrue

\input{../tex-inputs/latex-vorspann}


               \section[Paul Goldmann an Arthur Schnitzler, 8. 8. 1893]{ Paul Goldmann an Arthur Schnitzler, 8. 8. 1893}\nopagebreak\mylabel{v}\rehead{ }\normalsize\beginnumbering\briefempfaengerindex{Schnitzler, Arthur@\textsc{Schnitzler, Arthur}!zzzGoldmann, Paul@\emph{von Paul Goldmann}!1893-08-081@{8. 8. 1893}|(be} \toendnotes[C]{\smallbreak\pagebreak[2]} \Standort{DLA, A:Schnitzler, HS.NZ85.1.3163.}
\physDesc{Brief, 2 Blätter, 8 Seiten
\newline{}Handschrift: schwarze Tinte, deutsche Kurrent
\newline{}Schnitzler: mit rotem Buntstift eine Unterstreichung }\toendnotes[C]{\smallbreak}\pstart
           \noindent{}\textcolor{gray}{\textbf{{\pb}\textbf{\textcolor{brown}{Frankfurter Zeitung}{}\ledrightnote{\textcolor{brown}{Frankfurter Zeitung}}.}}}\hfill \textsc{\textcolor{pink}{Paris}{}\ledrightnote{\textcolor{pink}{Paris}}}, 8. August.\pend
           \pstart
           \textcolor{gray}{\textbf{\textbf{(\textcolor{brown}{\begin{otherlanguage}{french}Gazette de Francfort\end{otherlanguage}}{}\ledrightnote{\textcolor{brown}{Frankfurter Zeitung}}.)}}}\hfill 93.\pend
           \pstart
           \textcolor{gray}{\textbf{\begin{otherlanguage}{french}\textcolor{blue}{Directeur}{}\ledrightnote{→\textcolor{blue}{Leopold Sonnemann}}\end{otherlanguage}{ }\textbf{M. \textcolor{blue}{L. Sonnemann}{}\ledrightnote{\textcolor{blue}{Leopold Sonnemann}}.}}}\pend
           \pstart
           \begin{otherlanguage}{french}\textcolor{gray}{\textbf{\textcolor{green}{Journal}{}\ledrightnote{\textcolor{green}{Frankfurter Zeitung}} politique, financier,}}\end{otherlanguage}\pend
           \pstart
           \begin{otherlanguage}{french}\textcolor{gray}{\textbf{commercial et litteraire.}}\end{otherlanguage}\pend
           \pstart
           \begin{otherlanguage}{french}\textcolor{gray}{\textbf{\textbf{Paraissant trois fois par jour}}}\end{otherlanguage}\pend
           \pstart
           \begin{otherlanguage}{french}\textcolor{gray}{\textbf{\textbf{Bureaux à \textcolor{pink}{Paris}{}\ledrightnote{\textcolor{pink}{Paris}}:}}}\end{otherlanguage}\pend
           \pstart
           \begin{otherlanguage}{french}\textcolor{gray}{\textbf{\textbf{\textcolor{pink}{rue Richelieu 75}{}\ledrightnote{\textcolor{pink}{rue Richelieu}}.}}}\end{otherlanguage}\pend
           \pstart
           Mein lieber Arthur!\pend
           \pstart
           Nicht ohne Bangen habe ich diesmal Deinen lieben Brief eröffnet. Ich war mir einer
               großen Schuld bewußt, und fürchtete \textcolor{gray}{Vorwürfe}. Die bekam ich nun
               nicht direct – ich kenne Deine Güte und Nachſicht – wohl gibt es aber da ein Wort,
               das ich nicht verſtehe: »Mißtrauen«. Wirklich, ich habe keine Ahnung, worauf Du damit
               anſpielſt, und befürchte irgend eine verleumderiſche Klatſcherei. Mißtrauen? Aber
               wenn es irgend einen Menſchen gibt, den ich mit ruhigem Herzen bis in den letzten
               Winkel meines Weſens hineinſehen laße, ſo {\pb}biſt Du
               es, und das weißt Du ſehr wohl. Ich traue Dir ebenſo wie mir ſelbſt – nicht ideal,
               ſchwärmeriſch, \textcolor{gray}{pen}ſionsmädchenhaft, ſondern auf Grund kühler
               Manneserfahrung, mit der ich Dich als den Beſten und Treueſten erprobt habe. Was
               willſt Du alſo mit dem kurioſen Wort? Es klingt wie eine falſche Note und zeigt mir,
               daß Zeit und Entfernung auch zwiſchen uns die übliche Arbeit gethan.\pend
           \pstart
           Ich habe mich mit Deinem letzten Briefe unendlich gefreut, wochenlang! Und doch habe
               ich Dir nicht geantwortet. Warum? Weil ich gelähmt bin – moraliſch und geiſtig, weil
                  dieſe\strikeout{s} grauenhafte \label{K_L02711-1v}\edtext{Krankheit}{\lemma{\textnormal{\emph{Krankheit}}}\Cendnote{\textnormal{Siehe Paul Goldmann an Arthur Schnitzler, 6. 2. [1893]}}}\label{K_L02711-1h} mein ganzes Sein in einen Nebel hüllt, weil ich am Leben und an der Zukunft
               verzweifle, weil mein Leben {\pb}in zwei Abſchnitte
               zerfällt, die geſunde und die kranke Zeit, weil ich an die geſunde Zeit kein Anrecht
               mehr habe und weil Alles, was mir daher kommt, Alles Liebe und Hoffnungsreiche, mir
               als verloren erſcheint. Mir kommt es vor, als hätte ich kein Recht mehr, mitzuleben.
               Darum konnt’ ich den alten Ton nicht finden, nicht einmal die Energie, eine Feder in
               die Hand zu nehmen, und darum habe ich Dir nicht geantwortet. Mir geht es
               gottsſchlecht trotz aller Kuren. Das Übel greift um ſich, und ich weiß nicht, was aus
               mir wird. Da klammere ich mich denn an die Arbeit und pflüge jeden Tag mein
               abgeſtecktes Stück Feld ab. Bis ich aber fertig, ſo kommen alle Geſpenſter {\pb}wieder. Sehr ſtark bin ich nie geweſen, nun bin ich
               weinerlich wie eine alte Frau, und kaum ein Abend vergeht ohne Thränen. Dabei glaubt
               man nun doch nicht und hat nicht einmal den Troſt, daß Einem Gott das zur Prüfung
               geſchickt hat. Man weiß nur, daß man ein ſchädliches Exemplar der Race geworden,
               deſſen Mitthunwollen ein Verſtoß gegen alle Geſetze der Hygiene iſt. Dann kommt
               natürlich der gute Selbtmord. Aber es iſt unmöglich, des Leben zu verlaſſen, das man
               jetzt erſt zu verſtehen beginnt, das ſo mannigfaltig und ſo farbig iſt. So bleibt
               Einem nichts als Händeringen und Haarausraufen.\pend
           \pstart
           Ich habe bisher nicht einmal den Entschluß faſſen könnnen, auf Urlaub zu gehen. {\pb}Ich fürchte mich vor der arbeitsloſen Zeit. Von
               Hauſe drängen ſie mich aber. Mein \textcolor{blue}{Onkel}{}\ledrightnote{→\textcolor{blue}{Fedor Mamroth}} iſt im September in \textsc{\textcolor{pink}{Salzburg}{}\ledrightnote{\textcolor{pink}{Salzburg}}}, und ich ſoll durchaus \label{K_L02711-2v}\edtext{hinkommen}{\lemma{\textnormal{\emph{hinkommen}}}\Cendnote{\textnormal{\textcolor{blue}{Goldmann} reiste im September 1893 tatsächlich nach \textcolor{pink}{Salzburg}. Vom 17. 9. 1893 ist ein gemeinsamer Abend in \textcolor{pink}{Hellbrunn} mit \textcolor{blue}{Schnitzler} und \textcolor{blue}{Fedor Mamroth}
                  bekannt, vom 18. 9. 1893
                  ein Konzertbesuch mit \textcolor{blue}{Schnitzler}.}}}\label{K_L02711-2h}. Er
               malt mir all’ die Herrlichkeiten von \textcolor{pink}{\textsc{Salzburg}}{}\ledrightnote{\textcolor{pink}{Salzburg}} aus, wie man einem paniſchen Kinde zuredet. Da iſt beſonders eine Verheißung:
                  \textsc{Arthur Schnitzler}. Ach, ich habe ein ſolches Heimweh
               nach Dir, mein theurer Freund. Vielleicht reiße ich mich doch heraus und komme. Thu’
               mir jedenfalls die Liebe und halte Dir im September ein
               paar Tage für mich frei. Wenn ich reiſen ſollte, verſtändige ich Dich {\pb}in den letzten Tagen des Auguſt. Schreib’ mir, ob Dich nur dieſe Zeit eine Nachricht in \textcolor{pink}{Wien}{}\ledrightnote{\textcolor{pink}{Wien}} trifft. Aber bereite Dich vor, mich ſehr zum
               Nachtheil verändert, zu finden, und geh’ nicht zu ſtreng mit mir in’s Gericht.\pend
           \pstart
           Dann ſprechen wir auch über alles Übrige. Ich halte zum Beiſpiel eine \label{K_L02711-3v}\edtext{Reiſe nach \textcolor{pink}{Berlin}{}\ledrightnote{\textcolor{pink}{Berlin}}}{\lemma{\textnormal{\emph{Reiſe nach Berlin}}}\Cendnote{\textnormal{nicht geschehen}}}\label{K_L02711-3h} zur Betreibung
               Deiner Dramatiſchen Angelegenheiten für unerläßlich. Ebenſo ließe ſich vielleicht
               hier etwas mit \textsc{\textcolor{blue}{Antoine}{}\ledrightnote{\textcolor{blue}{André Antoine}}} machen, wenn Du eines der \textsc{\textcolor{green}{Anatol}{}\ledrightnote{\textcolor{green}{Anatol}}}-Stücke ins Franzöſiſche überſetzen könnteſt und ſelbſt hierherkämeſt, um die
               Sache zu betreiben. Seit dem \label{K_L02711-4v}\edtext{Erfolge
                  \textsc{\textcolor{blue}{Gerhart Hauptmann}{}\ledrightnote{\textcolor{blue}{Gerhart Hauptmann}}s}}{\lemma{\textnormal{\emph{Erfolge … Hauptmanns}}}\Cendnote{\textnormal{\textcolor{blue}{Gerhart Hauptmann} erreichte seinen ersten
                  großen Bühnenerfolg mit der fünfaktigen \textcolor{green}{Komödie}{ }\emph{\textcolor{green}{College Crampton}}, die am 16. 1. 1892 am \emph{\textcolor{brown}{Deutschen Theater}} in \textcolor{pink}{Berlin}
                  uraufgeführt wurde und am 8. 2. 1892 unter Anwesenheit \textcolor{blue}{Schnitzler}s im \emph{\textcolor{brown}{Burgtheater}} Premiere
                  feierte.}}}\label{K_L02711-4h} ſind ſie dort wie ich höre nicht unzugänglich { }{\pb}für \textcolor{pink}{Deutſch}{}\ledrightnote{\textcolor{pink}{Deutschland}}es
               und \textcolor{pink}{Öſterreich}{}\ledrightnote{\textcolor{pink}{Österreich}}iſches. Mit dem, was Trottel in
               Saublättern \label{K_L02711-5v}\edtext{über Dich ſchreiben}{\lemma{\textnormal{\emph{über Dich ſchreiben}}}\Cendnote{\textnormal{Erst wenige Tage zuvor, am 3. 8. 1893, erschien unter dem Pseudonym \textcolor{blue}{Bruno Walden} eine äußerst
                  negative \emph{\textcolor{green}{Kritik}} über den \emph{\textcolor{green}{Anatol-Zyklus}} von \textcolor{blue}{Florentine Galliny} in der \emph{\textcolor{green}{Wiener
                     Abendpost}}: \textcolor{blue}{Bruno Walden} [= \textcolor{blue}{Florentine Galliny}]: \emph{\textcolor{green}{Feuilleton. Literatur}}. In: \emph{\textcolor{green}{Wiener Abendpost}}. \textcolor{green}{Beilage} zur \emph{\textcolor{green}{Wiener
                        Zeitung}}., Jg. 190, Nr. 176, 3. 8. 1893,
                     S. 1–2 »\textcolor{blue}{Walden}«
                  schreibt: »Bei \textcolor{blue}{Arthur Schnitzler}s
                        »\textcolor{green}{Anatol}« hat ganz und gar die »\textsc{Vie \textcolor{pink}{Paris}ienne}«
                     Pathin geſtanden, und hier tritt das Nachtreterthum noch viel unangenehmer und
                     plumper zu Tage [{\dots}]. Was dem \textcolor{pink}{Pariſ}er Blatte petillante Frivolität, iſt hier crüder
                     Cynismus, der ſich in der Schlußſzene zum Höhenpunkte des Unwidernden
                     potencirt.« Über \textcolor{blue}{Hugo von
                     Hofmannsthal}s einleitende \textcolor{green}{Verse} schreibt »\textcolor{blue}{Walden}« außerdem: »Die Leichtbeſchwingheit dieſer
                        \textcolor{green}{Verse} gebricht der
                     vorgeführten \textcolor{green}{Scenenreihe}, und damit entfällt die »hübſche Formel böſer Dinge«, deren
                     Abſtoßendes in Folge deſſen ungemildert bleibt, was, wenn auch ethisch ganz
                     nützlich, doch kaum beabſichtigt geweſen ſein dürfte. Die introſpectiven
                     Grübeleien – ein echt deutſcher Zug – dieſes Anatol, der ſich ſo ver – –
                     zweifelt intereſſant vorkommt, ſind es, die einer Leichtfertigkeit, welche
                     einzig in unbewßter Lebensüberſchäumung eine \textsc{\begin{otherlanguage}{french}\label{K_L02711-8v}\edtext{Raison d’être}{\lemma{\textnormal{\emph{Raison d’être}}}\Cendnote{\textnormal{französisch: Rechtfertigung,
                              Daseinsbereichtigung}}}\label{K_L02711-8h}\end{otherlanguage}} aufzuweiſen vermag, einen ſo anwidernd perverſen Zug aufdrücken. Das
                     entrüſtete Freundeswort ſeines ſo langmüthig verſtändnißvollen Vertrauten in
                     der \textcolor{green}{Schlußſcene} »\textcolor{green}{Anatols Hochzeitstag}«:
                     »So was thut man nicht!« läßt sich für dieſelbe dahin variiren: So was ſchreibt
                     man nicht.« (S. 2) Am 4. 8. 1893 notierte sich \textcolor{blue}{Schnitzler} dazu im \emph{\textcolor{green}{Tagebuch}}:
                     »In der \textcolor{green}{Abendpost} von \textcolor{blue}{Bruno Walden} eine
                     alberne und niederträchtige \textcolor{green}{Kritik} über
                        \textcolor{green}{Anatol}, die mich
                  verstimmte.«}}}\label{K_L02711-5h}, ſollſt Du Dir dein \textsc{cabinet}
               tapezieren und ruhig weiter ſchaffen, auch von vorübergehenden Muthloſigkeiten
               unbeirrt, wie ſie die alltäglichen Erſcheinungsformen aller \strikeout{p\textcolor{gray}{rh}} producirenden Thätigkeit ſind, wenn etwas zuviel Gehirnſchmalz verbraucht iſt.
               Das dumme Gethue, das Dir heute in die Beine kläfft, wird Dir morgen die Hand
               ſchlecken, wenn erſt der \uline{Erfolg} da ſein wird, das
               einzige Beweisſtück in den Augen des Geſindels. Den aber wirſt Du haben, aus dem
               einfachen Grunde, weil Du von der\strikeout{\textcolor{gray}{n}} jungen ſchreibenden {\pb}Generation eines der
               größten und glänzendſten Talente biſt. Du biſt viel mehr als \textsc{\textcolor{blue}{Herzl}{}\ledrightnote{\textcolor{blue}{Theodor Herzl}}}, denn dieſer iſt – ſo erſtaunlich Dir das klingen mag, – ein enger \textcolor{blue}{Geiſt}{}\ledrightnote{→\textcolor{blue}{Theodor Herzl}}, kein \textcolor{blue}{Dichter}{}\ledrightnote{→\textcolor{blue}{Theodor Herzl}}, und nur eine \textcolor{blue}{F\textcolor{gray}{ern}begabung}{}\ledrightnote{→\textcolor{blue}{Theodor Herzl}}. Ich kenne
               nur Einen, mit dem ich Dich ernſtlich vergleiche, das iſt \textsc{\textcolor{blue}{Gerhart Hauptmann}{}\ledrightnote{\textcolor{blue}{Gerhart Hauptmann}}}. Du biſt im Weichen das, was er im Starken iſt – ich urtheile nach den »\textcolor{green}{Weber}{}\ledrightnote{\textcolor{green}{Die Weber}}n« – und dieſe Überzeugung werden mir alle
               kritiſirenden Pinſel nicht erſchüttern. Deine letzten Werke kenn ich nicht. Mein \textcolor{blue}{Onkel}{}\ledrightnote{→\textcolor{blue}{Fedor Mamroth}} nennt Deinen \textcolor{green}{Roman}{}\ledrightnote{→\textcolor{green}{Sterben. Novelle}} »bedeutend«. Das iſt ein
                  \label{K_L02711-6v}\edtext{\textsc{Epitheton}}{\lemma{\textnormal{\emph{Epitheton}}}\Cendnote{\textnormal{Attribut}}}\label{K_L02711-6h}, das ich von ihm nur auf
               die bewunderten Meiſter bisher anwenden gehört und ich nehme es als erfreuliches
               Zeugniß.\pend
           \pstart Sei von Herzen gegrüßt, mein lieber Arthur! Dein \spacefill\mbox{Paul Goldm}\pend{}\endnumbering\briefempfaengerindex{Schnitzler, Arthur@\textsc{Schnitzler, Arthur}!zzzGoldmann, Paul@\emph{von Paul Goldmann}!1893-08-081@{8. 8. 1893}|)be}\mylabel{h}\begin{anhang}\end{anhang}\normalsize

\doendnotes{C}
\bigskip
\vfill

\clearpage

\footnotesize

\lohead{\textsc{register}}

% Definiere theindex-Environment komplett neu ohne reledmac
\makeatletter
\renewenvironment{theindex}{%
  \section*{\indexname}%
  \setlength{\parindent}{0pt}%
  \setlength{\parskip}{0pt plus 0.3pt}%
  \let\item\@idxitem
}{%
  \clearpage
}
\makeatother

\IfFileExists{\jobname-pw.ind}{\input{\jobname-pw.ind}}{}

\end{document}

      