%% latex-korrekturansicht-vorspann.tex
%% Vorspann für die Korrekturansicht.
%% Lädt die gemeinsame Datei latex-vorspann.tex mit gesetztem Schalter.

\newif\ifkorrekturansicht
\korrekturansichttrue

\input{../tex-inputs/latex-vorspann}


               \section[Arthur Schnitzler an Hugo Hofmannsthal, 1. 10. 1919]{ Arthur Schnitzler an Hugo Hofmannsthal, 1. 10. 1919}\nopagebreak\mylabel{v}\rehead{ }\normalsize\beginnumbering\briefempfaengerindex{Hofmannsthal, Hugo von@\textsc{Hofmannsthal, Hugo von}!zzzSchnitzler, Arthur@\emph{von Arthur Schnitzler}!1919-10-011@{1. 10. 1919}|(be} \toendnotes[C]{\smallbreak\pagebreak[2]} \Standort{FDH, Hs-30885,149.}
\physDesc{Brief, 2 Blätter, 3 Seiten
\newline{}Handschrift: Bleistift, lateinische Kurrent}\buchAbdrucke{\weitereDrucke{1) Hugo von Hofmannsthal, Arthur Schnitzler: \emph{Briefwechsel}. Hg. Therese Nickl und Heinrich Schnitzler. Frankfurt am Main: \emph{S. Fischer} 1964, S. 285–286.} \weitereDrucke{2) Arthur Schnitzler: \emph{Briefe 1913–1931}. Hg. Peter Michael Braunwarth, Richard Miklin, Susanne Pertlik und Heinrich Schnitzler. Frankfurt am Main: \emph{S. Fischer} 1984, S. 195–197.} }\toendnotes[C]{\smallbreak}\pstart
           \raggedleft{}{\pb}1. 10. 19{\\}\textcolor{pink}{Wien}{}\ledrightnote{\textcolor{pink}{Wien}}\pend
           \pstart
           mein lieber Hugo, vor ein paar Wochen schon hat mir die \textcolor{blue}{Hofrätin}{}\ledrightnote{\textcolor{blue}{Berta Zuckerkandl}} gesagt, Sie seien auf einen Brief an mich
               ohne Antwort geblieben; ich will Ihnen nur mittheilen, dſs Ihr Brief vom
                  19. 9. der erste ist, den ich seit vielen Monaten von Ihnen erhielt –
               der letzte berichtete von Ihrem leidenden Zustand und ich schrieb Ihnen darauf, dſs
               ich gern einmal zu Ihnen nach \textcolor{pink}{Rodaun}{}\ledrightnote{\textcolor{pink}{Rodaun}} käme, aber
               darauf hatt ich von Ihnen nichts weiter gehört. Nun freuts mich sehr dſs die neueste
               Kunde so arbeitsfroh und hoffnungsvoll klingt und es wäre wahrhaftig schön, we{\geminationn} man wieder einmal einer jener feiertäglichen
               Vorlesestunden entgegensehen dürfte – die nur im Lauf der Jahre um so viel seltener
               geworden sind als selbst die seltensten Feiertage. Und was für eine Reihe von
               festlich ergreifenden Abenden – von jenem ersten an, an dem Sie mir, an einem warmen
                  \label{K_L02327_1v}\edtext{Juniabend}{\lemma{\textnormal{\emph{Juniabend}}}\Cendnote{\textnormal{siehe A. S.: \emph{Tagebuch}, 7. 10. 1891}}}\label{K_L02327_1h} war es, in der \textcolor{pink}{Giselastraße}{}\ledrightnote{\textcolor{pink}{Bösendorferstraße}}, »\textcolor{green}{Gestern}{}\ledrightnote{\textcolor{green}{Gestern. Dramatische Studie in einem Akt in Versen}}« vorlasen – oder war ich es, der mit dem
                  »\label{K_L02327_2v}\edtext{\textcolor{green}{Märchen}{}\ledrightnote{\textcolor{green}{Das Märchen. Schauspiel in drei Aufzügen}}}{\lemma{\textnormal{\emph{Märchen}}}\Cendnote{\textnormal{Diese Lesung fand am 25. 6. 1891 in der \textcolor{pink}{Seidlgasse} statt. Aber bereits früher lassen sich
                  solche Lesungen im privaten Kreis nachweisen.}}}\label{K_L02327_2h}« anfing, in der \textcolor{pink}{Seidlgasse}{}\ledrightnote{\textcolor{pink}{Seidlgasse}}, bei \textcolor{blue}{Richard}{}\ledrightnote{\textcolor{blue}{Richard Beer-Hofmann}} – ich weiß nicht mehr? Es kam wirklich wenig darauf an, ob das Werk
               als solches mehr oder weniger vollendet war – der Beifall geringer oder größer – im
               Rückblick bleiben es durchaus Stunden der kräftigsten, belebtesten Atmosphäre –
               bessere, reinere: als wenn man dasselbe Werk zum ersten Mal der Oeffentlich{\pb}keit zu praesentiren hatte. Ich bin höchst gespannt was
               Sie aus \textcolor{pink}{Altaussee}{}\ledrightnote{\textcolor{pink}{Altaussee}} mitbringen werden. Mit meiner
               Arbeit (\textcolor{green}{Stück}{}\ledrightnote{→\textcolor{green}{Der Gang zum Weiher. Dramatische Dichtung}}) geht es so langsam
               vorwärts, dſs ich fast von einem Stillstand sprechen kann – obzwar ich die
               Continuität zum mindesten durch beharrliches Anstarren unbeschriebener Papierblätter
               oder Ausstreichen des Geschriebenen festzuhalten versuche. Das letzte, was ich fertig
               gemacht \introOben{}habe\introOben{}, sind die »\textcolor{green}{Schwestern}{}\ledrightnote{\textcolor{green}{Die Schwestern oder Casanova in Spa. Lustspiel in Versen}}«, die bei \textcolor{blue}{Reinhardt}{}\ledrightnote{\textcolor{blue}{Max Reinhardt}} kommen
               sollen; – mir selbst ist selten was von mir so lieb gewesen. Ich hab allerlei vor,
               manches aus den letzten Jahren ist sogar recht weit gediehen; aber meine Arbeitskraft
               ist – wohl unter dem Einfluss dieses grauenhaften Weltzustandes – so tief herunter
               wie noch nie. Zu einer größern Reise hab ich mich nicht entschließen können, nun lädt
               mich meine \textcolor{blue}{Schwägerin}{}\ledrightnote{→\textcolor{blue}{Elisabeth Steinrück}}{ }sehr dringend nach \textcolor{pink}{Partenkirchen}{}\ledrightnote{\textcolor{pink}{Partenkirchen}} (wohin auch \textcolor{blue}{Olga}{}\ledrightnote{\textcolor{blue}{Olga Schnitzler}} im
               Anschluss an ein \textcolor{pink}{Münchn}{}\ledrightnote{\textcolor{pink}{München}}er Concert\strikeout{)} gehen wird); aber mich graut vor Wartesälen,
               Bahncoupés, Zollvisitationen, Gepäckaufgeben; und so wird auch daraus kaum was
               werden. Ich bin in diesem Sommer {\pb}nur in \textcolor{pink}{Reichenau}{}\ledrightnote{\textcolor{pink}{Reichenau an der Rax}} gewesen, \label{K_L02327_3v}\edtext{einmal zehn Tage}{\lemma{\textnormal{\emph{einmal zehn Tage}}}\Cendnote{\textnormal{vom 7. 8. 1919 bis zum 20. 8. 1919}}}\label{K_L02327_3h} (mit
               all den Meinen) einmal \label{K_L02327_4v}\edtext{drei Tage}{\lemma{\textnormal{\emph{drei Tage}}}\Cendnote{\textnormal{vom 8. 9. 1919 bis zum 12. 9. 1919}}}\label{K_L02327_4h}; – das ist für mich ein Ort
               so erfüllt von Erinnerungen der mannigfachsten Art, dſs ich ihnen, in der schweren
                  Sti{\geminationm}ung dieser So{\geminationm}ertage, kaum gewachsen war. Immerhin wurden mir in tausend und mehr Metern Höhe,
               auf Wiesen, an Waldesrand, ein paar gute Stunden.\pend
           \pstart
           – We{\geminationn} nicht früher mein lieber Hugo so sehe ich Sie wohl
               bei der \label{K_L02327_5v}\edtext{Generalprobe}{\lemma{\textnormal{\emph{Generalprobe}}}\Cendnote{\textnormal{vgl. A. S.: \emph{Tagebuch}, 8. 10. 1919}}}\label{K_L02327_5h} der \textcolor{green}{sonnigen Frau}{}\ledrightnote{→\textcolor{green}{Die Frau ohne Schatten. Oper in drei Akten}} (ich
               habe \textcolor{blue}{Strauß}{}\ledrightnote{\textcolor{blue}{Richard Strauss}} um Einlaß gebeten, auch für \textcolor{blue}{Olga}{}\ledrightnote{\textcolor{blue}{Olga Schnitzler}}, hoffentlich gehts) – ich kenne schon allerlei
               daraus vom Clavier her und freu mich ganz besonders. Haben Sie de{\geminationn} nun auch die \textcolor{green}{Märchen-Erzählung}{}\ledrightnote{→\textcolor{green}{Die Frau ohne Schatten. Erzählung}}, von der Sie mir öfters sprachen – die
               denselben Stoff behandelt, fertig gemacht? \pend
           \pstart
           – Ich schicke diese Zeilen noch nach \textcolor{pink}{Aussee}{}\ledrightnote{\textcolor{pink}{Bad Aussee}}. Haben
               Sie weiterhin gute, reiche Tage! \pend
           \pstart
           Von Herzen Ihr{\\[\baselineskip]}\spacefill\mbox{Arth}\pend
           \leftskip=0em{}\endnumbering\briefempfaengerindex{Hofmannsthal, Hugo von@\textsc{Hofmannsthal, Hugo von}!zzzSchnitzler, Arthur@\emph{von Arthur Schnitzler}!1919-10-011@{1. 10. 1919}|)be}\mylabel{h}  \normalsize

\doendnotes{C}
\bigskip
\vfill

\clearpage

\footnotesize

\lohead{\textsc{register}}

% Definiere theindex-Environment komplett neu ohne reledmac
\makeatletter
\renewenvironment{theindex}{%
  \section*{\indexname}%
  \setlength{\parindent}{0pt}%
  \setlength{\parskip}{0pt plus 0.3pt}%
  \let\item\@idxitem
}{%
  \clearpage
}
\makeatother

\IfFileExists{\jobname-pw.ind}{\input{\jobname-pw.ind}}{}

\end{document}

      