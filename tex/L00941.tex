%% latex-korrekturansicht-vorspann.tex
%% Vorspann für die Korrekturansicht.
%% Lädt die gemeinsame Datei latex-vorspann.tex mit gesetztem Schalter.

\newif\ifkorrekturansicht
\korrekturansichttrue

\input{../tex-inputs/latex-vorspann}


               \section[Arthur Schnitzler an Hugo von Hofmannsthal, 14. 7. 1899]{ Arthur Schnitzler an Hugo von Hofmannsthal, 14. 7. 1899}\nopagebreak\mylabel{v}\rehead{ }\normalsize\beginnumbering\briefempfaengerindex{Hofmannsthal, Hugo von@\textsc{Hofmannsthal, Hugo von}!zzzSchnitzler, Arthur@\emph{von Arthur Schnitzler}!1899-07-141@{14. 7. 1899}|(be} \toendnotes[C]{\smallbreak\pagebreak[2]} \Standort{FDH, Hs-30885,83.}
\physDesc{Briefkarte
\newline{}Handschrift: Bleistift, deutsche Kurrent}\buchAbdrucke{\weitereDrucke{Hugo von Hofmannsthal, Arthur Schnitzler: \emph{Briefwechsel}. Hg. Therese Nickl und Heinrich Schnitzler. Frankfurt am Main: \emph{S. Fischer} 1964, S. 125.} }\toendnotes[C]{\smallbreak}\pstart
           \raggedleft{}{\pb}14/7 99\pend
           \pstart
           mein lieber Hugo. Montag reiſe ich wahrſcheinlich ab. Adresse: \textsc{\textcolor{pink}{Velden}{}\ledrightnote{\textcolor{pink}{Velden}}, \textcolor{pink}{Pension Pundschu}{}\ledrightnote{\textcolor{pink}{Pension Pundschu}}}. Bin dort mit \textcolor{blue}{Mama}{}\ledrightnote{→\textcolor{blue}{Louise Schnitzler}} u
                        \textcolor{blue}{Schweſter}{}\ledrightnote{→\textcolor{blue}{Gisela Hajek}}. \textcolor{blue}{Waſſermann}{}\ledrightnote{\textcolor{blue}{Jakob Wassermann}} geht vielleicht mit. Von \textcolor{blue}{Richard}{}\ledrightnote{\textcolor{blue}{Richard Beer-Hofmann}} hör ich wenig; eben eine Karte; ich
                    hab nicht den Eindruck, dſs er in guter Sti{\geminationm}ung
                    ist. – Wie lang ich in \textcolor{pink}{V.}{}\ledrightnote{\textcolor{pink}{Velden}} bleibe? – 8–14 Tage.
                    Möchte gern dann höher. Es bleibt hoffentlich bei Mitte Auguſt für
                        {\pb}uns 2; bitte ſchieben Sie’s nicht viel weiter
                    hinaus, we{\geminationn} es geht. – Was für eine Art 5actiges
                        \textcolor{green}{Stück}{}\ledrightnote{→\textcolor{green}{Das Bergwerk zu Falun}} iſt das, was Sie
                    ſchreiben? \pend
           \pstart
           – Über alles, was ich innerlich durchmache, iſt ſchwer zu ſchreiben. Es iſt wie
                    wenn die Wolken i{\geminationm}er tiefer und ſchwerer herabſänken, je aufrechter man
                    geht.\pend
           \pstart Herzlich der Ihre \spacefill\mbox{Arth}\pend{}\pstart
           \noindent{}Grüßen Sie \textcolor{blue}{Minnie}{}\ledrightnote{\textcolor{blue}{Hermine von Schaffgotsch}}.\pend
           \endnumbering\briefempfaengerindex{Hofmannsthal, Hugo von@\textsc{Hofmannsthal, Hugo von}!zzzSchnitzler, Arthur@\emph{von Arthur Schnitzler}!1899-07-141@{14. 7. 1899}|)be}\mylabel{h}  \normalsize

\doendnotes{C}
\bigskip
\vfill

\clearpage

\footnotesize

\lohead{\textsc{register}}

% Definiere theindex-Environment komplett neu ohne reledmac
\makeatletter
\renewenvironment{theindex}{%
  \section*{\indexname}%
  \setlength{\parindent}{0pt}%
  \setlength{\parskip}{0pt plus 0.3pt}%
  \let\item\@idxitem
}{%
  \clearpage
}
\makeatother

\IfFileExists{\jobname-pw.ind}{\input{\jobname-pw.ind}}{}

\end{document}

      