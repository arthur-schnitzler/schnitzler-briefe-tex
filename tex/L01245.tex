%% latex-korrekturansicht-vorspann.tex
%% Vorspann für die Korrekturansicht.
%% Lädt die gemeinsame Datei latex-vorspann.tex mit gesetztem Schalter.

\newif\ifkorrekturansicht
\korrekturansichttrue

\input{../tex-inputs/latex-vorspann}


               \section[Ferdinand von Saar an Arthur Schnitzler, 25. 10. 1902]{ Ferdinand von Saar an Arthur Schnitzler, 25. 10. 1902}\nopagebreak\mylabel{v}\rehead{ }\normalsize\beginnumbering\briefempfaengerindex{Schnitzler, Arthur@\textsc{Schnitzler, Arthur}!zzzSaar, Ferdinand von@\emph{von Ferdinand von Saar}!1902-10-251@{25. 10. 1902}|(be} \toendnotes[C]{\smallbreak\pagebreak[2]} \Standort{CUL, Schnitzler, B 88.}
\physDesc{Briefkarte
\newline{}Handschrift: schwarze Tinte, deutsche Kurrent
\newline{}Schnitzler: mit Bleistift nummeriert: »6« }\toendnotes[C]{\smallbreak}\pstart
           \noindent{}\textcolor{gray}{\textbf{{\pb}Bei meinem Eintritt in das
                        70. Lebensjahr sind mir so zahlreiche Beweise der Anerkennung und Zuneigung
                        geworden, dass ich nur in dieser Weise meinen wärmsten Dank darbringen kann.
                        Mögen Alle, die mich am späten Abend meines Lebens durch Ehrungen
                        ausgezeichnet, mir Liebes und Gutes gesagt oder bezeigt, die Versicherung
                        entgegen nehmen, dass ich mich durch all diese Kundgebungen im tiefsten
                        beglückt fühle. Bin ich doch jetzt von dem erhebenden Bewusstsein
                        durchdrungen, den Besten meiner Zeit genug gethan zu haben.}}\pend
           \pstart
           \textcolor{gray}{\textbf{\textcolor{pink}{Wien-Döbling}{}\ledrightnote{\textcolor{pink}{XIX., Döbling}}.}}{ }25/10. 1902\pend
           \pstart
           mit herzlichem Dichtergruß{\\[\baselineskip]}und beſonderem Danke{\\[\baselineskip]}für die collegial
                    anerkennende »\label{K_L01245_1v}\edtext{\textcolor{green}{Widmung.}{}\ledrightnote{→\textcolor{green}{Liebelei. Erstes Bild}}}{\lemma{\textnormal{\emph{Widmung.}}}\Cendnote{\textnormal{Gemeint ist \textcolor{blue}{Schnitzler}s Beitrag für eine Festschrift: \emph{\textcolor{green}{Liebelei. Erstes Bild}}. In: \emph{\textcolor{green}{Widmungen zur Feier des siebzigsten
                                Geburtstages Ferdinand von Saar’s}}. Hg. v. \textcolor{blue}{Richard Specht}. Buchschmuck v. \textcolor{blue}{A. F. Seligmann}. Wien: \emph{\textcolor{brown}{Wiener Verlag}}{ }1903 (vordatiert von 14. 11. 1902),
                            S. 175–196.}}}\label{K_L01245_1h}«{\\[\baselineskip]}\spacefill\mbox{Ferdinand von Saar.}\pend
           \leftskip=0em{}\endnumbering\briefempfaengerindex{Schnitzler, Arthur@\textsc{Schnitzler, Arthur}!zzzSaar, Ferdinand von@\emph{von Ferdinand von Saar}!1902-10-251@{25. 10. 1902}|)be}\mylabel{h}  \normalsize

\doendnotes{C}
\bigskip
\vfill

\clearpage

\footnotesize

\lohead{\textsc{register}}

% Definiere theindex-Environment komplett neu ohne reledmac
\makeatletter
\renewenvironment{theindex}{%
  \section*{\indexname}%
  \setlength{\parindent}{0pt}%
  \setlength{\parskip}{0pt plus 0.3pt}%
  \let\item\@idxitem
}{%
  \clearpage
}
\makeatother

\IfFileExists{\jobname-pw.ind}{\input{\jobname-pw.ind}}{}

\end{document}

      