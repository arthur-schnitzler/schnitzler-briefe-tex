%% latex-korrekturansicht-vorspann.tex
%% Vorspann für die Korrekturansicht.
%% Lädt die gemeinsame Datei latex-vorspann.tex mit gesetztem Schalter.

\newif\ifkorrekturansicht
\korrekturansichttrue

\input{../tex-inputs/latex-vorspann}


               \section[Anna von Hofmannsthal und Arthur Schnitzler an Hugo von Hofmannsthal, {[}19. 7. 1898{]}]{ Anna von Hofmannsthal und Arthur Schnitzler an Hugo von Hofmannsthal,
               {[}19. 7. 1898{]}}\nopagebreak\mylabel{v}\rehead{ }\normalsize\beginnumbering\briefempfaengerindex{Hofmannsthal, Hugo von@\textsc{Hofmannsthal, Hugo von}!zzzSchnitzler, Arthur@\emph{von Arthur Schnitzler}!1898-07-191@{{[}19. 7. 1898{]}}|(be}\briefempfaengerindex{Hofmannsthal, Hugo von@\textsc{Hofmannsthal, Hugo von}!zzzHofmannsthal, Anna von@\emph{von Anna von Hofmannsthal}!1898-07-191@{{[}19. 7. 1898{]}}|(be} \toendnotes[C]{\smallbreak\pagebreak[2]} \Standort{FDH, Hofmannsthal, M8.}
\physDesc{Brief, 1 Blatt, 4 Seiten
\newline{}Handschrift Anna von Hofmannsthal: schwarze Tinte, deutsche Kurrent\newline{}Handschrift Arthur Schnitzler: schwarze Tinte, deutsche Kurrent}\buchAbdrucke{\weitereDrucke{Arthur Schnitzler: \emph{Briefe 1875–1912}. Hg. Therese Nickl und Heinrich Schnitzler. Frankfurt am Main: \emph{S. Fischer} 1981, S. 351.} }\toendnotes[C]{\smallbreak}\pstart
           \raggedleft{}{\pb}\textcolor{pink}{\textsc{Fusch}}{}\ledrightnote{\textcolor{pink}{Fusch an der Großglocknerstraße}} den 19/7.\pend
           \pstart\center{}Mein lieber kleiner \textsc{Hugi}!\pend\pstart
           Heute ein prachtvoller \textsc{So{\geminationm}ertag}! der gute \textcolor{blue}{\textsc{Papa}}{}\ledrightnote{→\textcolor{blue}{Hugo August von Hofmannsthal}} iſt mit \textsc{Arthur}, der geſtern nach unſerem \textsc{Souper} angefahren kam, nämlich \textsc{D\textsuperscript{r} Schnitzler} iſt dieſer \textsc{Arthur} in \textcolor{pink}{\textsc{Ferleithen}}{}\ledrightnote{\textcolor{pink}{Ferleiten}} von wo ſie \substVorne{}\textsuperscript{nach}\substDazwischen{}vor\substHinten{} Tiſch zurück kehren wollen. Die liebe kleine \textcolor{blue}{\textsc{Dora}}{}\ledrightnote{\textcolor{blue}{Dora Michaelis}}, die einer Erkältung wegen mit ihrer Familie die auch nach \textcolor{pink}{\textsc{Ferleithen}}{}\ledrightnote{\textcolor{pink}{Ferleiten}} iſt nicht mit konnte, ſitzt neben mir auf der \textsc{Veranda}
               und kocht mit den \textcolor{blue}{2 Flatſcherkindern}{}\ledrightnote{\textcolor{blue}{Martin Flatscher}{\newline}\textcolor{blue}{Maria Anna Flatscher}}. \textcolor{blue}{\textsc{Papa}}{}\ledrightnote{→\textcolor{blue}{Hugo August von Hofmannsthal}} hat ein ſehr hübſches Flanellhemd und ſeinen ſchwarzen Gürtel angezogen, eine
                  \textsc{affectirte} ſchottiſche Kappe aufgeſetzt, und iſt mit der
                  »\textcolor{green}{\textsc{Liebelei}}{}\ledrightnote{→\textcolor{green}{Liebelei. Schauspiel in drei Akten}}« die ich nicht ſah, weil ich noch im Bette lag, friſchen Muthes um ½ 8
                  Uhr früh ab.\pend
           \pstart
           Seit es ſchön iſt, fühlt ſich \textcolor{blue}{\textsc{Papa}}{}\ledrightnote{→\textcolor{blue}{Hugo August von Hofmannsthal}} unberufen ſehr wohl, iſt luſtig und zieht ſich ſehr gepflegt an. Über Alles das
               ſind wir froh, nicht wahr lieber Hugi.\pend
           \pstart
           {\pb}Sehr ſtolz bin ich darauf, daß Du mit meinem Brief ſo
               zufrieden biſt!\pend
           \pstart
           \textsc{Amusantes} kann ich Dir eigentlich nichts ſchreiben, aber
               von alldem was hier vorgeht, und wie uns zu Muthe iſt, davon weißt Du immer! –\pend
           \pstart
           Geſtern war ich faſt den ganzen Nachmittag im Wald oben, und habe ſo recht nach
               Herzensluſt mit den \textcolor{blue}{\textsc{Speyermädeln}}{}\ledrightnote{\textcolor{blue}{Paula Schmidl}{\newline}\textcolor{blue}{Julie Wassermann}{\newline}\textcolor{blue}{Agnes Ulmann}{\newline}\textcolor{blue}{Emilie Sgal}{\newline}\textcolor{blue}{Dora Michaelis}{\newline}\textcolor{blue}{Sophie Knepler}} geplauſcht. Dann bin ich mit \textcolor{blue}{\textsc{Papa}}{}\ledrightnote{→\textcolor{blue}{Hugo August von Hofmannsthal}} auf der Anna Bank gemüthlich geſeßen, und bei \textsc{Arthur’s
                  Souper assistirten} wir auch. Wir ſind mit ihm unter den Bäumen vor dem
                  Fliegen\textsc{salon} geſeßen. Alſo 12 Stunden in der beſten
               Luft, die es überhaupt giebt. Ich ſeh ſchon, wie Du jetzt lachſt, daß ich die \textcolor{pink}{\textsc{Fusch}}{}\ledrightnote{\textcolor{pink}{Fusch an der Großglocknerstraße}} ſchon wieder ſo lobe! –\pend
           \pstart
           Während ich mit Dir plaudere, kommt abwechſelnd die kleine \textcolor{blue}{\textsc{Nani}}{}\ledrightnote{\textcolor{blue}{Maria Anna Flatscher}} und der \textcolor{blue}{\textsc{Martin}}{}\ledrightnote{\textcolor{blue}{Maria Anna Flatscher}}, und zeigen mir die ſchönen Sachen, die ſie am Tiſch neben an, in dem Geſchirrl
               das {\pb}wir ihnen mitbrachten, kochten. Sie ſind wirklich
               liebe Fratzen, und machen mir viel Spaß, und ko{\geminationm}e ich
               mir um Vieles jünger vor wenn ich mit Kindern oder jungen \textsc{Mädeln} bin. Du weißt, daß mich die Frauen in meinem Alter nur mäßig anregen.
               Eigentlich verſti{\geminationm}en ſie mich mehr, und fühle ich dann
               mein Alter! es iſt das eine Schwäche von mir deren ich mich aufrichtig geſagt aber
               nicht ſchäme.\pend
           \pstart
           Abends wollen wir heute wieder zu \textcolor{pink}{\textsc{Weilguni}}{}\ledrightnote{\textcolor{pink}{Hotel Weilguni}} gehen, ſchöne Muſick hören. ich freue mich ſehr darauf, denn das iſt mir ein
               großer Genuß für mich.\pend
           \pstart
           Damit die Schreiberei noch \textsc{animirter} wird, werfen die
               Kinder über unter und neben mich den Ballen. Unglaublich, was ſie heute treiben, aber
               mich ſtört es nicht und ſpiele ich immer wieder ſelbſt mit ihnen. \pend
           {\bigskip}\pstart
           \noindent{}{\pb}{[}hs. Schnitzler:{]} mein lieber Hugo, aus \textcolor{pink}{Ferleiten}{}\ledrightnote{\textcolor{pink}{Ferleiten}} haben Sie ſchon meinen gedruckten Gruſs beko{\geminationm}en, nehmen Sie noch einen geſchriebnen aus der \textcolor{pink}{Fuſch}{}\ledrightnote{\textcolor{pink}{Fusch an der Großglocknerstraße}}. Ich freue mich ſehr hiehergeko{\geminationm}en zu ſein; vor zwanzig Jahren oder mehr bin ich zum
               letzten Mal hier geweſen. Jetzt eben ko{\geminationm} ich mit Ihrem
                  \textcolor{blue}{Papa}{}\ledrightnote{→\textcolor{blue}{Hugo August von Hofmannsthal}} aus \textcolor{pink}{Ferleiten}{}\ledrightnote{\textcolor{pink}{Ferleiten}} zurück und Ihre Mama offerirt mir dieſe leere Seite.
               So werd ich mit Liebenswürdigkeiten überſchüttet.\pend
           \pstart
           Auf Wiederſehen!\pend
           \pstart Von Herzen Ihr \spacefill\mbox{Arthur.}\pend{}\endnumbering\briefempfaengerindex{Hofmannsthal, Hugo von@\textsc{Hofmannsthal, Hugo von}!zzzSchnitzler, Arthur@\emph{von Arthur Schnitzler}!1898-07-191@{{[}19. 7. 1898{]}}|)be}\briefempfaengerindex{Hofmannsthal, Hugo von@\textsc{Hofmannsthal, Hugo von}!zzzHofmannsthal, Anna von@\emph{von Anna von Hofmannsthal}!1898-07-191@{{[}19. 7. 1898{]}}|)be}\mylabel{h}  \normalsize

\doendnotes{C}
\bigskip
\vfill

\clearpage

\footnotesize

\lohead{\textsc{register}}

% Definiere theindex-Environment komplett neu ohne reledmac
\makeatletter
\renewenvironment{theindex}{%
  \section*{\indexname}%
  \setlength{\parindent}{0pt}%
  \setlength{\parskip}{0pt plus 0.3pt}%
  \let\item\@idxitem
}{%
  \clearpage
}
\makeatother

\IfFileExists{\jobname-pw.ind}{\input{\jobname-pw.ind}}{}

\end{document}

      