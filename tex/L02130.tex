%% latex-korrekturansicht-vorspann.tex
%% Vorspann für die Korrekturansicht.
%% Lädt die gemeinsame Datei latex-vorspann.tex mit gesetztem Schalter.

\newif\ifkorrekturansicht
\korrekturansichttrue

\input{../tex-inputs/latex-vorspann}


               \section[Hermann Bahr an Arthur Schnitzler, 23. 4. 1913]{ Hermann Bahr an Arthur Schnitzler, 23. 4. 1913}\nopagebreak\mylabel{v}\rehead{ }\normalsize\beginnumbering\briefempfaengerindex{Schnitzler, Arthur@\textsc{Schnitzler, Arthur}!zzzBahr, Hermann@\emph{von Hermann Bahr}!1913-04-231@{23. 4. 1913}|(be} \toendnotes[C]{\smallbreak\pagebreak[2]} \Standort{CUL, Schnitzler, B 5b.}
\physDesc{Brief, 1 Blatt, 2 Seiten
\newline{}Handschrift: schwarze Tinte, deutsche Kurrent
\newline{}Schnitzler: mit Bleistift ergänzt »Bahr« \newline{}Ordnung: mit Bleistift von unbekannter Hand nummeriert:
                                    »177« }\buchAbdrucke{\weitereDrucke{Hermann Bahr, Arthur Schnitzler: \emph{Briefwechsel, Aufzeichnungen, Dokumente (1891–1931)}. Hg. Kurt Ifkovits und Martin Anton Müller. Göttingen: \emph{Wallstein} 2018, S. 485.} }\toendnotes[C]{\smallbreak}\pstart
           \raggedleft{}{\pb}23. 4. 13\pend
           \pstart{}Lieber Arthur,\pend\pstart
           herzlichen Dank! Ich bin ſehr froh, den armen \textcolor{blue}{Peter}{}\ledrightnote{\textcolor{blue}{Peter Altenberg}} bald wieder »draußen« zu wiſſen.\pend
           \pstart
           Und nun noch was. Ich \label{K_L02130_1v}\edtext{ſchrieb Dir im
                  Dezember}{\lemma{\textnormal{\emph{ſchrieb Dir im
                  Dezember}}}\Cendnote{\textnormal{Hermann Bahr an Arthur Schnitzler, 7. 12. 1912}}}\label{K_L02130_1h}, daß ich keine Luſt habe, Geld für ihn herzugeben. Ich glaube nemlich
               beſtimmt zu wiſſen, daß er es nicht braucht und daß ich es alſo beſſer verwenden
               kann. Sollteſt Du aber einmal den Eindruck haben, daß es \uline{notwendig} iſt, ſo bitte ſchreib mir das, da geb ich natürlich gleich, was
               ich entbehren kann. Aber bitte {\pb}dies ganz unter
               uns.\pend
           \pstart
           Ich erfuhr jetzt erſt, daß Du einem \label{K_L02130_2v}\edtext{»Comité« für meinen 50. Geburtstag}{\lemma{\textnormal{\emph{»Comité« … Geburtstag}}}\Cendnote{\textnormal{vgl. \emph{Briefwechsel}
                  Bahr/Schnitzler 483.}}}\label{K_L02130_2h} uſw. Ich danke Dir dafür ſehr.\pend
           \pstart
           Zur \label{K_L02130_3v}\edtext{»\textcolor{green}{Götterdämmerung}{}\ledrightnote{\textcolor{green}{Götterdämmerung}}« war ich neulich in \textcolor{pink}{Wien}{}\ledrightnote{\textcolor{pink}{Wien}},
               komme wol zum »\textcolor{green}{Triſtan}{}\ledrightnote{\textcolor{green}{Tristan und Isolde}}« wieder}{\lemma{\textnormal{\emph{»Götterdämmerung« … wieder}}}\Cendnote{\textnormal{ Die \textcolor{pink}{Hofoper} gab \textcolor{blue}{Wagner}s \emph{\textcolor{green}{Götterdämmerung}} am 13. 4. 1913, \emph{\textcolor{green}{Tristan und Isolde}} am 5. 5. 1913, beide Male mit
                     \textcolor{blue}{Anna Bahr-Mildenburg}.}}}\label{K_L02130_3h}, aber immer
               knapp zur Vorſtellung und nachher in aller Früh wieder weg, denn ich
                  bi\textcolor{gray}{n} mitten in einem neuen \label{K_L02130_4v}\edtext{\textcolor{green}{Stück}{}\ledrightnote{→\textcolor{green}{Das Phantom}}}{\lemma{\textnormal{\emph{Stück}}}\Cendnote{\textnormal{\emph{\textcolor{green}{Das Phantom}} (Komödie in drei Akten. Mit
                     Dekorationsskizzen von \textcolor{blue}{Koloman Moser}.
                     Berlin: \emph{\textcolor{brown}{S. Fischer}}{ }1913).}}}\label{K_L02130_4h}. Aber, wohin Du ſommers auch gehſt, Du kommſt doch über \textcolor{pink}{Salzburg}{}\ledrightnote{\textcolor{pink}{Salzburg}} und wir freuen uns \textcolor{blue}{Beide}{}\ledrightnote{→\textcolor{blue}{Anna Bahr-Mildenburg}}{ }ſehr, ſehr, ſehr darauf, \textcolor{blue}{Euch}{}\ledrightnote{→\textcolor{blue}{Olga Schnitzler}} dann hier zu haben und einmal ausgiebig
               mit \textcolor{blue}{Euch}{}\ledrightnote{→\textcolor{blue}{Olga Schnitzler}} zuſammen zu ſein.\pend
           \pstart
           \label{K_L02130_5v}\edtext{Immer derſelbe}{\lemma{\textnormal{\emph{Immer derſelbe}}}\Cendnote{\textnormal{Hier lässt sich eine Verbindung zu einem
                  zentralen Motto \textcolor{blue}{Bahrs} herstellen, das er
                     1911{ }so begründete: »In ein Stammbuch schrieb
                     einer stolz: Immer derselbe! Ich darunter keck: Niemals derselbe! Spät erst
                     ging mir auf, das Rechte wäre wohl Beides: Niemals derselbe und eben darin doch
                     immer derselbe zu sein!« (\emph{\textcolor{green}{[Stammbuch-Spruch]}} In: \emph{Musen-Almanach 1911.} Berlin: \emph{Verein Berliner
                        Presse} 1910, S. 39) Im Jahr darauf knüpfte er im Text \emph{\textcolor{green}{Selbstinventur}} (\emph{\textcolor{green}{Die neue Rundschau}}, Jg. 23, H. 9,
                     S. 1287–1303) längere Überlegungen daran an.}}}\label{K_L02130_5h}{\\[\baselineskip]}\spacefill\mbox{Hermann}\pend
           \leftskip=0em{}\endnumbering\briefempfaengerindex{Schnitzler, Arthur@\textsc{Schnitzler, Arthur}!zzzBahr, Hermann@\emph{von Hermann Bahr}!1913-04-231@{23. 4. 1913}|)be}\mylabel{h}  \normalsize

\doendnotes{C}
\bigskip
\vfill

\clearpage

\footnotesize

\lohead{\textsc{register}}

% Definiere theindex-Environment komplett neu ohne reledmac
\makeatletter
\renewenvironment{theindex}{%
  \section*{\indexname}%
  \setlength{\parindent}{0pt}%
  \setlength{\parskip}{0pt plus 0.3pt}%
  \let\item\@idxitem
}{%
  \clearpage
}
\makeatother

\IfFileExists{\jobname-pw.ind}{\input{\jobname-pw.ind}}{}

\end{document}

      