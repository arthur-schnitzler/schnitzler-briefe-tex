%% latex-korrekturansicht-vorspann.tex
%% Vorspann für die Korrekturansicht.
%% Lädt die gemeinsame Datei latex-vorspann.tex mit gesetztem Schalter.

\newif\ifkorrekturansicht
\korrekturansichttrue

\input{../tex-inputs/latex-vorspann}


               \section[Arthur Schnitzler an Hermann Bahr, 30. 3. 1902]{ Arthur Schnitzler an Hermann Bahr, 30. 3. 1902}\nopagebreak\mylabel{v}\rehead{ }\normalsize\beginnumbering\briefempfaengerindex{Bahr, Hermann@\textsc{Bahr, Hermann}!zzzSchnitzler, Arthur@\emph{von Arthur Schnitzler}!1902-03-301@{30. 3. 1902}|(be} \toendnotes[C]{\smallbreak\pagebreak[2]} \Standort{TMW, HS AM 23350 Ba.}
\physDesc{Brief, 1 Blatt, 3 Seiten
\newline{}Handschrift: schwarze Tinte, deutsche Kurrent\newline{}Ordnung: 1) Lochung 2) mit Bleistift von unbekannter Hand ergänzt: »\textsc{Charfreitag}«}\buchAbdrucke{\weitereDrucke{1) \emph{[28. 3.] 1902.} In: Arthur Schnitzler: \emph{The Letters of Arthur Schnitzler to Hermann Bahr}. Edited, annotated, and with an introduction, by Donald G.
                        Daviau. Chapel Hill: \emph{The University of North Carolina Press} 1978, S. 74–75 (University of North Carolina studies in the Germanic languages
                        and literatures, 89).} \weitereDrucke{2) Hermann Bahr, Arthur Schnitzler: \emph{Briefwechsel, Aufzeichnungen, Dokumente (1891–1931)}. Hg. Kurt Ifkovits und Martin Anton Müller. Göttingen: \emph{Wallstein} 2018, S. 227–228.} }\toendnotes[C]{\smallbreak}\pstart
           \raggedleft{}{\pb}Oſterſo{\geminationn}tag 1902\pend
           \pstart{}lieber Hermann,\pend\pstart
           eine \label{K_L01212_1v}\edtext{\textcolor{blue}{Dame}{}\ledrightnote{→\textcolor{blue}{Aur. St.}}}{\lemma{\textnormal{\emph{Dame}}}\Cendnote{\textnormal{Vgl. A. S.: \emph{Tagebuch}, 30. 3. 1902: »\textcolor{blue}{Aur.
                     St.}«.}}}\label{K_L01212_1h} bringt mir beiliegende 2 Skizzen{[},{]} der
                  \textcolor{blue}{Verfaſſer}{}\ledrightnote{→\textcolor{blue}{Gustav Modry}} hat die Abſicht
               Journaliſt zu werden. Ich ſoll ihn protegiren. Was anders ſoll er noch nicht
               geſchrieben haben. Auf dich hab ich ſo viel Einfluſs, ich ſoll’s dir doch einfach
               ſchicken.\pend
           \pstart
           Ich thue das, nicht ohne mich für diese Inanſpruchnahme deiner Zeit gebührend zu
               entſchuldigen. Aber ich denke, in 3 Minuten haſt du die Werke des jungen Manns
               geleſen, und wir ſind \introOben{}(bis auf weiteres)\introOben{} von dem Verdacht
               befreit, {\pb}die »Jungen« zu unterdrücken.\pend
           \pstart
           Wenn du mir überdies in 3 Worten dein Urtheil über die Leiſtungen dieſes Herrn
               kundgibſt, in einem Brief, den ich der Dame gleich zeigen ka{\geminationn}, u. mit \substVorne{}\textsuperscript{g}\substDazwischen{}d\substHinten{}einer \introOben{}ganzen\introOben{} Aufrichtigkeit, die in dieſem Fall
               beſonders nützlich, ja nothwendig wäre, ſo bin ich dir ſehr verbunden. –\pend
           \pstart
           In \textcolor{pink}{Venedig}{}\ledrightnote{\textcolor{pink}{Venedig}}{ }ſollen die Blattern ſein. Man müßte ſich
               für alle Fälle impfen laſſen, eh man hinunter{\pb}radelt.\pend
           \pstart
           Ich ſeh dich übrigens bei der \label{K_L01212_2v}\edtext{\textcolor{green}{»Kraft«probe}{}\ledrightnote{\textcolor{green}{Über unsere Kraft}}}{\lemma{\textnormal{\emph{»Kraft«probe}}}\Cendnote{\textnormal{\emph{\textcolor{green}{Über unsere Kraft}} von \textcolor{blue}{Bjørnson} wurde im \textcolor{pink}{Deutschen
                     Volkstheater} in zwei Teilen gegeben, der erste am 4., der zweite am 5. 4. 1902. Ob auch
                  die Generalprobe auf zwei Tage aufgeteilt war, ist unklar.}}}\label{K_L01212_2h}.\pend
           \pstart
           Herzlichſt der Deine{\\[\baselineskip]}\spacefill\mbox{Arth Sch}\pend
           \leftskip=0em{}\endnumbering\briefempfaengerindex{Bahr, Hermann@\textsc{Bahr, Hermann}!zzzSchnitzler, Arthur@\emph{von Arthur Schnitzler}!1902-03-301@{30. 3. 1902}|)be}\mylabel{h}  \normalsize

\doendnotes{C}
\bigskip
\vfill

\clearpage

\footnotesize

\lohead{\textsc{register}}

% Definiere theindex-Environment komplett neu ohne reledmac
\makeatletter
\renewenvironment{theindex}{%
  \section*{\indexname}%
  \setlength{\parindent}{0pt}%
  \setlength{\parskip}{0pt plus 0.3pt}%
  \let\item\@idxitem
}{%
  \clearpage
}
\makeatother

\IfFileExists{\jobname-pw.ind}{\input{\jobname-pw.ind}}{}

\end{document}

      