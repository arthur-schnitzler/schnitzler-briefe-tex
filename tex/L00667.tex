%% latex-korrekturansicht-vorspann.tex
%% Vorspann für die Korrekturansicht.
%% Lädt die gemeinsame Datei latex-vorspann.tex mit gesetztem Schalter.

\newif\ifkorrekturansicht
\korrekturansichttrue

\input{../tex-inputs/latex-vorspann}


               \section[Richard Beer-Hofmann an Arthur Schnitzler, 21. 4. 1897]{ Richard Beer-Hofmann an Arthur Schnitzler,
               21. 4. 1897}\nopagebreak\mylabel{v}\rehead{ }\normalsize\beginnumbering\briefempfaengerindex{Schnitzler, Arthur@\textsc{Schnitzler, Arthur}!zzzBeer-Hofmann, Richard@\emph{von Richard Beer-Hofmann}!1897-04-211@{21. 4. 1897}|(be} \toendnotes[C]{\smallbreak\pagebreak[2]} \Standort{CUL, Schnitzler, B 8.}
\physDesc{Brief, 1 Blatt, 2 Seiten
\newline{}Handschrift: Bleistift, lateinische Kurrent
\newline{}Schnitzler: mit Bleistift die Jahreszahl ergänzt: »97« \newline{}Ordnung: mit Bleistift von unbekannter Hand
                           nummeriert: »94« }\buchAbdrucke{\weitereDrucke{Arthur Schnitzler, Richard Beer-Hofmann: \emph{Briefwechsel 1891–1931}. Hg. Konstanze Fliedl. Wien, Zürich: \emph{Europaverlag} 1992, S. 102.} }\pstart
           \raggedleft{}{\pb}\textcolor{pink}{Wien}{}\ledrightnote{\textcolor{pink}{Wien}}{ }21/IV{\\}½ 12 Nachts{\\}im Caffée.\pend
           \pstart{}Lieber Arthur!\pend\pstart
           Ich hab heute Ihren Brief beko{\geminationm}en. Ich habe noch nie
               einen Menschen gesehen, der sich so sehr schämt sich einzugestehn {\pb}daß er sich wolfühlt. No ja – es
               geht Ihnen eben gut; sagen Sie »Unberufen« und gestehen Sie es sich ein.\pend
           \pstart
           Hier nichts Neues; nur \textcolor{blue}{Zaccone}{}\ledrightnote{\textcolor{blue}{Ermete Zacconi}} – ein Schauspieler
               den ich von \textcolor{pink}{Rom}{}\ledrightnote{\textcolor{pink}{Rom}} aus kannte. {\pb}Ein ganz Großer. »Techniker«
               schreien die Leute die nicht einmal Technik haben\pend
           \pstart
           Ich arbeite. \textcolor{blue}{Salten}{}\ledrightnote{\textcolor{blue}{Felix Salten}} ist seit Tagen ich weiß nicht
               wo mit ich weiß nicht wem. \textcolor{blue}{Georg Hirschfeld}{}\ledrightnote{\textcolor{blue}{Georg Hirschfeld}}
               unsichtbar. Schreiben {\pb}Sie bald den
               verheißenen »wirklichen Brief«. Ich grüße von Herzen \textcolor{blue}{Paul}{}\ledrightnote{\textcolor{blue}{Paul Goldmann}}; er soll aus der Tatsache daß ich Ihnen schreibe keine Folgerungen für
               mein schreibfaules Verhältniß zu ihm ableiten. Herzlichst\pend
           \pstart \spacefill\mbox{Richard}\pend{}\endnumbering\briefempfaengerindex{Schnitzler, Arthur@\textsc{Schnitzler, Arthur}!zzzBeer-Hofmann, Richard@\emph{von Richard Beer-Hofmann}!1897-04-211@{21. 4. 1897}|)be}\mylabel{h}  \normalsize

\doendnotes{C}
\bigskip
\vfill

\clearpage

\footnotesize

\lohead{\textsc{register}}

% Definiere theindex-Environment komplett neu ohne reledmac
\makeatletter
\renewenvironment{theindex}{%
  \section*{\indexname}%
  \setlength{\parindent}{0pt}%
  \setlength{\parskip}{0pt plus 0.3pt}%
  \let\item\@idxitem
}{%
  \clearpage
}
\makeatother

\IfFileExists{\jobname-pw.ind}{\input{\jobname-pw.ind}}{}

\end{document}

      