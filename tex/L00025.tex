%% latex-korrekturansicht-vorspann.tex
%% Vorspann für die Korrekturansicht.
%% Lädt die gemeinsame Datei latex-vorspann.tex mit gesetztem Schalter.

\newif\ifkorrekturansicht
\korrekturansichttrue

\input{../tex-inputs/latex-vorspann}


               \section[Arthur Schnitzler an Hugo von Hofmannsthal, 27. 7. 1891]{ Arthur Schnitzler an Hugo von Hofmannsthal, 27. 7. 1891}\nopagebreak\mylabel{v}\rehead{ }\normalsize\beginnumbering\briefempfaengerindex{Hofmannsthal, Hugo von@\textsc{Hofmannsthal, Hugo von}!zzzSchnitzler, Arthur@\emph{von Arthur Schnitzler}!1891-07-271@{27. 7. 1891}|(be} \toendnotes[C]{\smallbreak\pagebreak[2]} \Standort{FDH, Hs-30885,9.}
\physDesc{Brief, 2 Blätter, 6 Seiten
\newline{}Handschrift: schwarze Tinte, deutsche Kurrent}\buchAbdrucke{\weitereDrucke{1) Hugo von Hofmannsthal, Arthur Schnitzler: \emph{Briefwechsel}. Hg. Therese Nickl und Heinrich Schnitzler. Frankfurt am Main: \emph{S. Fischer} 1964, S. 9–10.} \weitereDrucke{2) Arthur Schnitzler: \emph{Briefe 1875–1912}. Hg. Therese Nickl und Heinrich Schnitzler. Frankfurt am Main: \emph{S. Fischer} 1981, S. 119–120.} }\toendnotes[C]{\smallbreak}\pstart
           \raggedleft{}{\pb}\textcolor{pink}{Wien}{}\ledrightnote{\textcolor{pink}{Wien}}, 27. Juli
                  1891.\pend
           \pstart
           Verehrter Freund, eine \label{K_L00025-3v}\edtext{Karte}{\lemma{\textnormal{\emph{Karte}}}\Cendnote{\textnormal{siehe Paul Goldmann an Arthur Schnitzler, 25. 7. 1891}}}\label{K_L00025-3h}, die ich eben von \textcolor{blue}{Paul Goldma{\geminationn}}{}\ledrightnote{\textcolor{blue}{Paul Goldmann}} beko{\geminationm}e, eri{\geminationn}ert mich,
               wie üblich es iſt, Briefe zu beantworten, und wie ich Ihnen ſchon längſt hätte
               ſchreiben ſollen, ja, wie ich Ihnen ſogar hätte ſchreiben wollen, we{\geminationn} mein Gehirn nicht die ganze letzte Zeit über todte
               Stellen hätte hinwegko{\geminationm}en müſſen. In zweierlei Perioden
               bietet einem das Leben was, in der der Anfänge, wo tauſenderlei über einen ko{\geminationm}t, und man {\pb}jeden Tag ein
               neues Blatt herzunehmen hat und nur drauflos zu begi{\geminationn}en.
                  Da{\geminationn} die andre Periode, wo man das Bedürfnis des
               Abſchließens hat – wo man die alten Blätter ni{\geminationm}t und
               einem alle möglichen Worte, Punkte u Gedankenſtriche einfallen, – die man verg\substVorne{}\textsuperscript{eſſen}\substDazwischen{}aß\substHinten{}{ }\strikeout{hat}. Die erſte Periode: wo man ſich an ſich
               berauſcht, die zweite: wo man ſich an ſich beruhigt. Ich bin jetzt in keiner von
               beiden, alſo arm und blöd. Nervös, ſehr. \textcolor{blue}{Beer-{\pb}Hofma{\geminationn}}{}\ledrightnote{\textcolor{blue}{Richard Beer-Hofmann}} iſt auch ſchon weg, das wiſſen Sie ja. – In die \textcolor{pink}{\textsc{Kugel}}{}\ledrightnote{\textcolor{pink}{Café Kugel}} ko{\geminationm} ich ſelten, es waren ſchon ein paar \textcolor{brown}{Ausſchuſsſitzungen}{}\ledrightnote{→\textcolor{brown}{»Freie Bühne« Verein für moderne Literatur}};
                  Specialcomités{ }ſind gew\textcolor{gray}{ä}hlt worden; ich ſitze im Theatercomité
               zuſammen mit \textcolor{blue}{\textsc{Pernerstorfer}}{}\ledrightnote{\textcolor{blue}{Engelbert Pernerstorfer}}, \textcolor{blue}{\textsc{Wengraf}}{}\ledrightnote{\textcolor{blue}{Edmund Wengraf}}, \textcolor{blue}{\textsc{Osten}}{}\ledrightnote{\textcolor{blue}{Heinrich Osten}}, \textcolor{blue}{\textsc{Kafka}}{}\ledrightnote{\textcolor{blue}{Eduard Michael Kafka}}, \textcolor{blue}{\textsc{Kulka}}{}\ledrightnote{\textcolor{blue}{Julius Kulka}}. –\hspace*{2.5em}Bis jetzt iſt noch nicht viel geſcheidtes
                  herausgeko{\geminationm}en. – Mit \textcolor{blue}{\textsc{Salten}}{}\ledrightnote{\textcolor{blue}{Felix Salten}} bin ich viel zuſa{\geminationm}en, auch auf dem »Land« des
               Abends. \textcolor{blue}{\textsc{Burckhard}}{}\ledrightnote{\textcolor{blue}{Max Eugen Burckhard}} hat mir den \textcolor{green}{Alkandi}{}\ledrightnote{\textcolor{green}{Alkandi’s Lied}} mit einigen
               ſchmeichelhaften Worten {\pb}zurückgeſandt – ich hab’ ihn
                  angeno{\geminationm}en. Mein \textcolor{green}{Stück}{}\ledrightnote{→\textcolor{green}{Das Märchen. Schauspiel in drei Aufzügen}} ruht und iſt mir zuwider. – Wie geht es Ihrem \textcolor{green}{hi{\geminationm}elblauen Einakter}{}\ledrightnote{→\textcolor{green}{Gestern. Dramatische Studie in einem Akt in Versen}}? Und wollen Sie mir nichts von Ihren Sachen ſchicken? Sie
               würden mir eine wirkliche Freude machen, ſeien Sie erſter oder ſiebenter Grad! –
               Geleſen wird mancherlei \textcolor{blue}{\textsc{Burckhardt}}{}\ledrightnote{\textcolor{blue}{Jacob Burckhardt}}, \textcolor{green}{Cultur der Renaiſſance}{}\ledrightnote{\textcolor{green}{Die Cultur der Renaissance in Italien. Ein Versuch}}, \textcolor{blue}{\textsc{Goethe}}{}\ledrightnote{\textcolor{blue}{Johann Wolfgang von Goethe}}, \textcolor{green}{Annalen}{}\ledrightnote{\textcolor{green}{Tag- und Jahreshefte}}, \textcolor{blue}{\textsc{Lessing}}{}\ledrightnote{\textcolor{blue}{Gotthold Ephraim Lessing}}s \textcolor{green}{Drama\strikeout{turgie}
                  Entwürfe}{}\ledrightnote{\textcolor{green}{Dramatische Entwürfe und Pläne}}, \textcolor{blue}{\textsc{Jonas Lie}}{}\ledrightnote{\textcolor{blue}{Jonas Lie}}{ }\textsc{etc.} Beſonders \textcolor{blue}{\textsc{Nietz}’ſche}{}\ledrightnote{\textcolor{blue}{Friedrich Nietzsche}} – zuletzt {\pb}hat mich ſein Schluſscapitel und das \textcolor{green}{Schlußgedicht}{}\ledrightnote{→\textcolor{green}{Nachgesang. Aus den hohen Bergen}} zu \textcolor{green}{\textsc{Jenseits von Gut u Böse}}{}\ledrightnote{\textcolor{green}{Jenseits von Gut und Böse}} ergriffen. – Eri{\geminationn}ern Sie ſich? \textcolor{blue}{\textsc{Nietz}’ſche}{}\ledrightnote{\textcolor{blue}{Friedrich Nietzsche}}{ }Sentimentalität! – Weinender Marmor! Stellen, die
               ſogar auf Weiber wirken, ohne daß man den Stellen oder den Weibern bös werden
               müßte. – Werden Sie mir bald wieder ſchreiben? Arbeiten Sie viel? Erleben {\pb}Sie was? Spielen Sie aber lieber \textsc{lawn-tennis}, ſtatt ſich zu verlieben, oder nehmen Sie wenigſtens, we{\geminationn} beides über Sie geko{\geminationm}en,
               das erſtere ernſter.\pend
           \pstart
           Herzlichen Gruſs. Den Ihrigen meine Empfehlungen. Iſt \textcolor{blue}{\textsc{Schwarzkopf}}{}\ledrightnote{\textcolor{blue}{Gustav Schwarzkopf}}{ }ſchon bei Ihnen? Ich ſah ihn ſchon Wochen lang
               nicht. –\pend
           \pstart
           Alſo nochmals, viele Grüße{\\[\baselineskip]}Ihr \spacefill\mbox{Arthur Sch}\pend
           \leftskip=0em{}\endnumbering\briefempfaengerindex{Hofmannsthal, Hugo von@\textsc{Hofmannsthal, Hugo von}!zzzSchnitzler, Arthur@\emph{von Arthur Schnitzler}!1891-07-271@{27. 7. 1891}|)be}\mylabel{h}  \normalsize

\doendnotes{C}
\bigskip
\vfill

\clearpage

\footnotesize

\lohead{\textsc{register}}

% Definiere theindex-Environment komplett neu ohne reledmac
\makeatletter
\renewenvironment{theindex}{%
  \section*{\indexname}%
  \setlength{\parindent}{0pt}%
  \setlength{\parskip}{0pt plus 0.3pt}%
  \let\item\@idxitem
}{%
  \clearpage
}
\makeatother

\IfFileExists{\jobname-pw.ind}{\input{\jobname-pw.ind}}{}

\end{document}

      