%% latex-korrekturansicht-vorspann.tex
%% Vorspann für die Korrekturansicht.
%% Lädt die gemeinsame Datei latex-vorspann.tex mit gesetztem Schalter.

\newif\ifkorrekturansicht
\korrekturansichttrue

\input{../tex-inputs/latex-vorspann}


               \section[Richard Beer-Hofmann an Arthur Schnitzler, 30. 11. 1909]{ Richard Beer-Hofmann an Arthur Schnitzler, 30. 11. 1909}\nopagebreak\mylabel{v}\rehead{ }\normalsize\beginnumbering\briefempfaengerindex{Schnitzler, Arthur@\textsc{Schnitzler, Arthur}!zzzBeer-Hofmann, Richard@\emph{von Richard Beer-Hofmann}!1909-11-301@{30. 11. 1909}|(be} \toendnotes[C]{\smallbreak\pagebreak[2]} \Standort{CUL, Schnitzler, B 8.}
\physDesc{Brief, 1 Blatt, 4 Seiten
\newline{}Handschrift: Bleistift, lateinische Kurrent
\newline{}Schnitzler: mit Bleistift beschriftet: »\textsc{R. Beerhofm}« \newline{}Ordnung: 1) mit Bleistift von unbekannter Hand nummeriert: »\strikeout{219}« 2) mit Bleistift von unbekannter Hand nummeriert:
                                    »225«}\buchAbdrucke{\weitereDrucke{Arthur Schnitzler, Richard Beer-Hofmann: \emph{Briefwechsel 1891–1931}. Hg. Konstanze Fliedl. Wien, Zürich: \emph{Europaverlag} 1992, S. 195–196.} }\toendnotes[C]{\smallbreak}\pstart
           \raggedleft{}{\pb}30/XI 09{\\}10 ¾ Nachts\pend
           \pstart
           Lieber Arthur! \textcolor{blue}{Poldi Andrian}{}\ledrightnote{\textcolor{blue}{Leopold von Andrian-Werburg}} geht eben weg; er ist – \textcolor{blue}{Felix Oppenheimer}{}\ledrightnote{\textcolor{blue}{Felix von Oppenheimer}} ist vor dem \label{K_L01890-1v}\edtext{Leichenbegängnis}{\lemma{\textnormal{\emph{Leichenbegängnis}}}\Cendnote{\textnormal{Die Überführung aus dem Trauerhaus in der \textcolor{pink}{Reisnerstraße 28} auf den Friedhof fand am
                     30. 11. 1909 statt.}}}\label{K_L01890-1h} seines \textcolor{blue}{Vaters}{}\ledrightnote{→\textcolor{blue}{Ludwig von Oppenheimer}} – \textcolor{blue}{Hugo}{}\ledrightnote{\textcolor{blue}{Hugo von Hofmannsthal}} auf dem \textcolor{pink}{Se{\geminationm}ering}{}\ledrightnote{\textcolor{pink}{Semmering}} – von der Bahn aus – ohne in einem Hôtel gewesen zu sein, zu mir {\pb}gefahren. Irgend eine – hoffentlich
               – wiederum nur hypochondrische Sache — diesmals Zungenkrebs – hat ihn ganz verstört.
               Er möchte dass Sie ihm rathen zu {\pb}wem er gehen soll – vielleicht sogar mit ihm hingehen. Er will – um Sie sicher zu
               treffen – morgen – Mittwoch – um 10\textsuperscript{h}.
                  Vorm. zu Ihnen ko{\geminationm}en, und bat mich Sie zu
               verständigen – was {\pb}ich hiemit
               tue –\pend
           \pstart
           Herzlichst Ihr{\\[\baselineskip]}\spacefill\mbox{Richard}\pend
           \leftskip=0em{}\pstart
           \noindent{}\textcolor{blue}{Lili}{}\ledrightnote{\textcolor{blue}{Lili Schnitzler}} die bei uns vorfuhr hat die \textcolor{blue}{Kinder}{}\ledrightnote{→\textcolor{blue}{Naëmah Beer-Hofmann}{\newline}→\textcolor{blue}{Mirjam Beer-Hofmann}{\newline}→\textcolor{blue}{Gabriel Beer-Hofmann}} – durch
                  ihr elegantes und energisches Lutschen – sehr entzückt.\pend
           \endnumbering\briefempfaengerindex{Schnitzler, Arthur@\textsc{Schnitzler, Arthur}!zzzBeer-Hofmann, Richard@\emph{von Richard Beer-Hofmann}!1909-11-301@{30. 11. 1909}|)be}\mylabel{h}  \normalsize

\doendnotes{C}
\bigskip
\vfill

\clearpage

\footnotesize

\lohead{\textsc{register}}

% Definiere theindex-Environment komplett neu ohne reledmac
\makeatletter
\renewenvironment{theindex}{%
  \section*{\indexname}%
  \setlength{\parindent}{0pt}%
  \setlength{\parskip}{0pt plus 0.3pt}%
  \let\item\@idxitem
}{%
  \clearpage
}
\makeatother

\IfFileExists{\jobname-pw.ind}{\input{\jobname-pw.ind}}{}

\end{document}

      