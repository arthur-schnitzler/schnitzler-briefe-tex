%% latex-korrekturansicht-vorspann.tex
%% Vorspann für die Korrekturansicht.
%% Lädt die gemeinsame Datei latex-vorspann.tex mit gesetztem Schalter.

\newif\ifkorrekturansicht
\korrekturansichttrue

\input{../tex-inputs/latex-vorspann}


               \section[Arthur Schnitzler an Richard Beer-Hofmann, 29. 9. 1899]{ Arthur Schnitzler an Richard Beer-Hofmann, 29. 9. 1899}\nopagebreak\mylabel{v}\rehead{ }\normalsize\beginnumbering\briefempfaengerindex{Beer-Hofmann, Richard@\textsc{Beer-Hofmann, Richard}!zzzSchnitzler, Arthur@\emph{von Arthur Schnitzler}!1899-09-291@{29. 9. 1899}|(be} \toendnotes[C]{\smallbreak\pagebreak[2]} \Standort{YCGL, MSS 31.}
\physDesc{Brief, 1 Blatt, 4 Seiten, Umschlag
\newline{}Handschrift: Bleistift, deutsche Kurrent\newline{}Versand: 1) Stempel: »\nobreak{}\oindex{Wiesbaden@\textbf{Wiesbaden}, \emph{Besiedelter Ort (A.BSO)}|pwk}Wiesbaden, 29. 9. 99, 9–10N\nobreak{}«.  2) Stempel: »\nobreak{}\oindex{Sankt Michael@\textbf{Sankt Michael}, \emph{Bezirk (A.BZK)}|pwk}St. Michael in Eppan, 2 10 99\nobreak{}«. }\buchAbdrucke{\weitereDrucke{Arthur Schnitzler, Richard Beer-Hofmann: \emph{Briefwechsel 1891–1931}. Hg. Konstanze Fliedl. Wien, Zürich: \emph{Europaverlag} 1992, S. 138.} }\toendnotes[C]{\smallbreak}\pstart{}{\pb}\textsc{Arthur Schnitzler}{ }\textcolor{pink}{Wien IX.}{}\ledrightnote{\textcolor{pink}{IX., Alsergrund}}\pend{}\pstart{}\textsc{\textcolor{pink}{Frankgasse}{}\ledrightnote{\textcolor{pink}{Frankgasse}}}\pend{}{\bigskip}\pstart{}{\pb}Herrn \textsc{Dr. Richard
                     Beer-Hofmann}\pend{}\pstart{}\textsc{\textcolor{pink}{St. Michael im Eppan}{}\ledrightnote{\textcolor{pink}{Sankt Michael}}}\pend{}{\bigskip}\pstart
           \noindent{}{\pb}Mein lieber Richard, wo iſt das, \textcolor{pink}{\textsc{St Michael im Eppan}}{}\ledrightnote{\textcolor{pink}{Sankt Michael}}? – Wie ſind Sie auf die Idee gekommen? Wie lang bleiben Sie dort? – In welchem
                  \textcolor{green}{Akt}{}\ledrightnote{→\textcolor{green}{Der Graf von Charolais. Ein Trauerspiel}}{ }ſind Sie? Wie iſt Ihre Laune? Warum {\pb}sind Sie von \textsc{\textcolor{pink}{Vahrn}{}\ledrightnote{\textcolor{pink}{Vahrn}}} fort? – \pend
           \pstart
           – \textcolor{blue}{Paul}{}\ledrightnote{\textcolor{blue}{Paul Goldmann}} iſt beſſer geſti{\geminationm}t als je (um Gotteswillen ſagen oder ſchreiben Sie’s
               ihm nicht). – Weil \textcolor{pink}{Wiesbdn}{}\ledrightnote{\textcolor{pink}{Wiesbaden}} grad in der Näh von \textcolor{pink}{Frankfurt}{}\ledrightnote{\textcolor{pink}{Frankfurt am Main}}, bin ich hergegangen, find es »eher«
               angenehm, würde {\pb}\textcolor{blue}{Hugo}{}\ledrightnote{\textcolor{blue}{Hugo von Hofmannsthal}}{ }ſagen. Das \textcolor{green}{Stück}{}\ledrightnote{→\textcolor{green}{Der Schleier der Beatrice. Schauspiel in fünf Akten}} wird wieder einmal »vorläufig« fertig. – Ich arbeite
               nicht wenig, aber nicht eben viel – »wir« haben doch wenig Arbeitskraft im ganzen und
               großen. »Trotzdem« freu ich mich auf Ihr \textcolor{green}{Stück}{}\ledrightnote{→\textcolor{green}{Der Graf von Charolais. Ein Trauerspiel}}. – Schreiben {\pb}Sie mir
               nach \textcolor{pink}{Berlin}{}\ledrightnote{\textcolor{pink}{Berlin}}{ }\textcolor{pink}{\textsc{Hotel Savoy}}{}\ledrightnote{\textcolor{pink}{Hotel Savoy}}, ich denke dſs ich vom nächſten Dinſtag 3. – bis So{\geminationn}tag dort ſein werde.\pend
           \pstart
           Grüßen Sie \textcolor{blue}{Frau}{}\ledrightnote{→\textcolor{blue}{Paula Beer-Hofmann}} und \textcolor{blue}{Kinder}{}\ledrightnote{→\textcolor{blue}{Naëmah Beer-Hofmann}{\newline}→\textcolor{blue}{Mirjam Beer-Hofmann}}.\pend
           \pstart
           Leben Sie wohl.\pend
           \pstart
           Herzlichſt Ihr{\\[\baselineskip]}\spacefill\mbox{Arthur}\pend
           \leftskip=0em{}\pstart
           \textcolor{pink}{\textsc{Wsbn}}{}\ledrightnote{\textcolor{pink}{Wiesbaden}}{ }29. 9. 99. \pend
           \endnumbering\briefempfaengerindex{Beer-Hofmann, Richard@\textsc{Beer-Hofmann, Richard}!zzzSchnitzler, Arthur@\emph{von Arthur Schnitzler}!1899-09-291@{29. 9. 1899}|)be}\mylabel{h}  \normalsize

\doendnotes{C}
\bigskip
\vfill

\clearpage

\footnotesize

\lohead{\textsc{register}}

% Definiere theindex-Environment komplett neu ohne reledmac
\makeatletter
\renewenvironment{theindex}{%
  \section*{\indexname}%
  \setlength{\parindent}{0pt}%
  \setlength{\parskip}{0pt plus 0.3pt}%
  \let\item\@idxitem
}{%
  \clearpage
}
\makeatother

\IfFileExists{\jobname-pw.ind}{\input{\jobname-pw.ind}}{}

\end{document}

      