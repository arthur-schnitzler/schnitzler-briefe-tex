%% latex-korrekturansicht-vorspann.tex
%% Vorspann für die Korrekturansicht.
%% Lädt die gemeinsame Datei latex-vorspann.tex mit gesetztem Schalter.

\newif\ifkorrekturansicht
\korrekturansichttrue

\input{../tex-inputs/latex-vorspann}


               \section[Arthur Schnitzler an Samuel Fischer, 25. 7. 1893]{ Arthur Schnitzler an Samuel Fischer, 25. 7. 1893}\nopagebreak\mylabel{v}\rehead{ }\normalsize\beginnumbering\briefempfaengerindex{Fischer, Samuel@\textsc{Fischer, Samuel}!zzzSchnitzler, Arthur@\emph{von Arthur Schnitzler}!1893-07-251@{25. 7. 1893}|(be} \toendnotes[C]{\smallbreak\pagebreak[2]} \Standort{Wrocław, Biblioteka Uniwersytecka, Böl.Nau 417.}
\physDesc{Brief, 1 Blatt (Briefpapier mit Trauerrand), 4 Seiten
\newline{}Handschrift: schwarze Tinte, deutsche Kurrent}\buchAbdrucke{\weitereDrucke{Wilhelm Bölsche: \emph{Briefwechsel. Mit Autoren der Freien Bühne}. Hg. Gerd-Hermann Susen. Berlin: \emph{Weidler} 2010, S. 693 (Werke und Briefe. Wissenschaftliche Ausgabe, Briefe I).} }\toendnotes[C]{\smallbreak}\pstart{}{\pb}Sehr geehrter Herr,\pend\pstart
           über \label{K_L00242_1v}\edtext{Aufforderung}{\lemma{\textnormal{\emph{Aufforderung}}}\Cendnote{\textnormal{Dieser Brief ist im Nachlass \textcolor{blue}{Bölsche}s überliefert, \textcolor{blue}{S. Fischer} hat ihn also an diesen
                        weitergegeben.}}}\label{K_L00242_1h} des Herrn \textcolor{blue}{\textsc{Dr. W. Bölsche}}{}\ledrightnote{\textcolor{blue}{Wilhelm Bölsche}}{ }ſende ich Ihnen \textcolor{green}{\uline{Das Märchen}}{}\ledrightnote{\textcolor{green}{Das Märchen. Schauspiel in drei Aufzügen}} zu. Wollen Sie mir gütigſt bald mittheilen, wann eine eventuelle
                    Veröffentlichung in der »\textcolor{green}{\textsc{Freien Bühne}}{}\ledrightnote{\textcolor{green}{Freie Bühne für den Entwickelungskampf der Zeit}}« {\pb}beginnen kann. Ich ſende Ihnen das \textcolor{green}{Manuscript}{}\ledrightnote{→\textcolor{green}{Das Märchen. Schauspiel in drei Aufzügen}}, ſa{\geminationm}t den
                    Zuſätzen und Anmerkungen, wie ich ſie für eine bevorſtehende Aufführg am \textcolor{brown}{Leſſing Theater}{}\ledrightnote{\textcolor{brown}{Lessing-Theater}} angebracht habe. Nur wünſchte
                    ich, daſs die Schilderungen der Perſonen, wie ſie ſich auf den erſten 2
                    beigefügten Blättern befinden, im Druck wegbleiben.\pend
           \pstart
           {\pb}Um Correcturen erſuche ich dringend.\pend
           \pstart
           Ich ſehe Ihrer werthen Entſcheidung ſowie der Angabe der Bedingungen, unter
                    welchen Sie das \textcolor{green}{Stück}{}\ledrightnote{\textcolor{green}{Das Märchen. Schauspiel in drei Aufzügen}} nehmen wollen, mit
                    lebhaftem Intereſſe entgegen, und möchte auch gern Ihre Äußerung über eine
                    event. Buchausgabe vernehmen.\pend
           \pstart
           – In der Hoffnung, daſs {\pb}Sie mich nicht zu lange auf
                    Antwort warten laſſen, bin ich in beſonderer Hochachtg\pend
           \pstart
           Ihr erg\textcolor{gray}{ebener}{\\[\baselineskip]}\spacefill\mbox{Dr. Arthur Schnitzler}\pend
           \leftskip=0em{}\pstart
           \textsc{\textcolor{pink}{Wien}{}\ledrightnote{\textcolor{pink}{Wien}}}, 25. Juli 93{\\}\textcolor{pink}{\textsc{I. Grillparzerstraße 7}}{}\ledrightnote{\textcolor{pink}{Grillparzerstraße}}\pend
           \endnumbering\briefempfaengerindex{Fischer, Samuel@\textsc{Fischer, Samuel}!zzzSchnitzler, Arthur@\emph{von Arthur Schnitzler}!1893-07-251@{25. 7. 1893}|)be}\mylabel{h}  \normalsize

\doendnotes{C}
\bigskip
\vfill

\clearpage

\footnotesize

\lohead{\textsc{register}}

% Definiere theindex-Environment komplett neu ohne reledmac
\makeatletter
\renewenvironment{theindex}{%
  \section*{\indexname}%
  \setlength{\parindent}{0pt}%
  \setlength{\parskip}{0pt plus 0.3pt}%
  \let\item\@idxitem
}{%
  \clearpage
}
\makeatother

\IfFileExists{\jobname-pw.ind}{\input{\jobname-pw.ind}}{}

\end{document}

      