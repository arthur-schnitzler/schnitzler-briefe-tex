%% latex-korrekturansicht-vorspann.tex
%% Vorspann für die Korrekturansicht.
%% Lädt die gemeinsame Datei latex-vorspann.tex mit gesetztem Schalter.

\newif\ifkorrekturansicht
\korrekturansichttrue

\input{../tex-inputs/latex-vorspann}


               \section[Arthur Schnitzler an Hermann Bahr, 11. 10. 1900]{ Arthur Schnitzler an Hermann Bahr, 11. 10. 1900}\nopagebreak\mylabel{v}\rehead{ }\normalsize\beginnumbering\briefempfaengerindex{Bahr, Hermann@\textsc{Bahr, Hermann}!zzzSchnitzler, Arthur@\emph{von Arthur Schnitzler}!1900-10-111@{11. 10. 1900}|(be} \toendnotes[C]{\smallbreak\pagebreak[2]} \Standort{TMW, HS AM 60152 Ba.}
\physDesc{Briefkarte
\newline{}Handschrift: schwarze Tinte, deutsche Kurrent\newline{}Ordnung: Lochung }\buchAbdrucke{\weitereDrucke{1) \emph{1. 10. 1900, Abschrift.} In: Arthur Schnitzler: \emph{The Letters of Arthur Schnitzler to Hermann Bahr}. Edited, annotated, and with an introduction, by Donald G.
                        Daviau. Chapel Hill: \emph{The University of North Carolina Press} 1978, S. 66–67 (University of North Carolina studies in the Germanic languages
                        and literatures, 89).} \weitereDrucke{2) Hermann Bahr, Arthur Schnitzler: \emph{Briefwechsel, Aufzeichnungen, Dokumente (1891–1931)}. Hg. Kurt Ifkovits und Martin Anton Müller. Göttingen: \emph{Wallstein} 2018, S. 182.} }\toendnotes[C]{\smallbreak}\pstart
           \noindent{}{\pb}Lieber Hermann, ich danke dir vielmals für den »\label{K_L01076_1v}\edtext{\textcolor{green}{Franzl}{}\ledrightnote{\textcolor{green}{Der Franzl. Fünf Bilder aus dem Leben eines guten Mannes}}}{\lemma{\textnormal{\emph{Franzl}}}\Cendnote{\textnormal{\textcolor{blue}{Hermann Bahr}: \emph{\textcolor{green}{Der Franzl. Fünf Bilder eines guten Mannes}}.}}}\label{K_L01076_1h}«, den ich mir auf einen kurzen Landaufenthalt mitnehme, um ihn mit
               Muße u Vergnügen zu leſen. Ich will dich gleich was fragen. Im Sommer hab ich eine
               mäßig {\pb}lange \textcolor{green}{Geſchichte}{}\ledrightnote{→\textcolor{green}{Lieutenant Gustl. Novelle}} geſchrieben, die ſich ausnehmend zum Vorleſen
               eignet, und die niemand beſſer vorleſen könnte als du. Bevor ich dir das \textsc{\textcolor{green}{Mscrpt}{}\ledrightnote{→\textcolor{green}{Lieutenant Gustl. Novelle}}}{ }ſchicke (\textsc{typewritten}) möchte ich nur dein \uline{principielles} Einverſtändnis haben. Herzlichen Gruß.
               Dein{\\}\spacefill\mbox{Arthur Schnitzler}\pend
           \pstart
           \damage{1}1. 10. 900.\pend
           \endnumbering\briefempfaengerindex{Bahr, Hermann@\textsc{Bahr, Hermann}!zzzSchnitzler, Arthur@\emph{von Arthur Schnitzler}!1900-10-111@{11. 10. 1900}|)be}\mylabel{h}  \normalsize

\doendnotes{C}
\bigskip
\vfill

\clearpage

\footnotesize

\lohead{\textsc{register}}

% Definiere theindex-Environment komplett neu ohne reledmac
\makeatletter
\renewenvironment{theindex}{%
  \section*{\indexname}%
  \setlength{\parindent}{0pt}%
  \setlength{\parskip}{0pt plus 0.3pt}%
  \let\item\@idxitem
}{%
  \clearpage
}
\makeatother

\IfFileExists{\jobname-pw.ind}{\input{\jobname-pw.ind}}{}

\end{document}

      