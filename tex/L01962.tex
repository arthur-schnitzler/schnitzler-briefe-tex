%% latex-korrekturansicht-vorspann.tex
%% Vorspann für die Korrekturansicht.
%% Lädt die gemeinsame Datei latex-vorspann.tex mit gesetztem Schalter.

\newif\ifkorrekturansicht
\korrekturansichttrue

\input{../tex-inputs/latex-vorspann}


               \section[Arthur Schnitzler an Hugo von Hofmannsthal, 4. 10. 1910]{ Arthur Schnitzler an Hugo von Hofmannsthal, 4. 10. 1910}\nopagebreak\mylabel{v}\rehead{ }\normalsize\beginnumbering\briefempfaengerindex{Hofmannsthal, Hugo von@\textsc{Hofmannsthal, Hugo von}!zzzSchnitzler, Arthur@\emph{von Arthur Schnitzler}!1910-10-041@{4. 10. 1910}|(be} \toendnotes[C]{\smallbreak\pagebreak[2]} \Standort{FDH, Hs-30885,139.}
\physDesc{Brief, 1 Blatt, 2 Seiten
\newline{}Handschrift: schwarze Tinte, deutsche Kurrent (\noindent{}Schlußformel und
                                        Unterschrift)}\buchAbdrucke{\weitereDrucke{Hugo von Hofmannsthal, Arthur Schnitzler: \emph{Briefwechsel}. Hg. Therese Nickl und Heinrich Schnitzler. Frankfurt am Main: \emph{S. Fischer} 1964, S. 253–254.} }\toendnotes[C]{\smallbreak}\pstart
           {\pb}\textcolor{gray}{\textbf{Dr. Arthur Schnitzler}}\hfill 4. 10. 1910.\pend
           \pstart
           \textcolor{gray}{\textbf{\textcolor{pink}{Wien XVIII. Sternwartestrasse 71}{}\ledrightnote{\textcolor{pink}{Sternwartestraße}}}}\pend
           \pstart\center{}Mein lieber Hugo.\pend\pstart
           Mein Telegramm hat Sie hoffentlich noch in \textcolor{pink}{München}{}\ledrightnote{\textcolor{pink}{München}} erreicht. Es war mir nicht möglich eine telephonische
                    Verbindung mit \textcolor{blue}{Rosenbaum}{}\ledrightnote{\textcolor{blue}{Richard Rosenbaum}} zu bekommen. Bald
                    war er auf der Probe, bald hat sich überhaupt niemand gemeldet. \textcolor{blue}{Berger}{}\ledrightnote{\textcolor{blue}{Alfred von Berger}} selbst war verreist und bis gestern noch nicht
                    zurückgekehrt. So habe ich also Ihren Besetzungsvorschlag an die \textcolor{pink}{Direktion}{}\ledrightnote{→\textcolor{pink}{Burgtheater}} schriftlich
                    mitgeteilt und mich zugleich damit sehr einverstanden erklärt. Im übrigen lag
                    Ihrem Brief kein Besetzungsvorschlag des \textcolor{pink}{Burgtheater}{}\ledrightnote{\textcolor{pink}{Burgtheater}}s bei; Sie schreiben von \textcolor{blue}{Tressler}{}\ledrightnote{\textcolor{blue}{Otto Tressler}} für den \textcolor{green}{Claudio}{}\ledrightnote{→\textcolor{green}{Der Thor und der Tod}}, was wirklich lächerlich wäre. Wie sonst die Rollen hätten
                    verteilt werden sollen, weiss ich nicht, nur dass die \textcolor{blue}{Bleibtreu}{}\ledrightnote{\textcolor{blue}{Hedwig Bleibtreu}} für den \textcolor{green}{Tod}{}\ledrightnote{→\textcolor{green}{Der Thor und der Tod}} in Aussicht genommen war, hatte ich schon früher
                    gehört, ohne für diese Idee sehr eingenommen zu sein. Ich hoffe übrigens Sie
                    haben sich auch persönlich an die Direktion gewandt, was ich doch jedenfalls
                    viel wirksamer fände als meine Intervention, so gern ich immer dazu
                        {[}bereit{]}{ }{\pb}war und bin. An dem \label{K_L01962_1v}\edtext{\textcolor{green}{Oedipus}{}\ledrightnote{\textcolor{green}{König Ödipus. Übersetzt und für die neuere Bühne eingerichtet}} haben Sie hoffentlich in \textcolor{pink}{München}{}\ledrightnote{\textcolor{pink}{München}}}{\lemma{\textnormal{\emph{Oedipus … München}}}\Cendnote{\textnormal{Die Premiere von \emph{\textcolor{green}{König Ödipus}} (Regie: \textcolor{blue}{Max
                            Reinhardt}) in der Übersetzung von \textcolor{blue}{Hofmannsthal} hatte am 25. 9. 1910 in der \textcolor{pink}{Neuen Musik-Festhalle}
                        stattgefunden.}}}\label{K_L01962_1h} viel Freude gehabt. Hier schicke ich Ihnen also »\textcolor{green}{das weite Land}{}\ledrightnote{\textcolor{green}{»Das weite Land«. (Tragikomödie in fünf Akten von Artur Schnitzler. Zum erstenmal aufgeführt am 14. Oktober 1911)}}«, das ich bitte noch durchaus
                    als Manuscript zu behandeln.\pend
           \pstart
           {[}hs.:{]} Auf baldiges Wiederſehen.{\\[\baselineskip]}Herzlichſt Ihr{\\[\baselineskip]}\spacefill\mbox{Arthur}\pend
           \leftskip=0em{}\endnumbering\briefempfaengerindex{Hofmannsthal, Hugo von@\textsc{Hofmannsthal, Hugo von}!zzzSchnitzler, Arthur@\emph{von Arthur Schnitzler}!1910-10-041@{4. 10. 1910}|)be}\mylabel{h}  \normalsize

\doendnotes{C}
\bigskip
\vfill

\clearpage

\footnotesize

\lohead{\textsc{register}}

% Definiere theindex-Environment komplett neu ohne reledmac
\makeatletter
\renewenvironment{theindex}{%
  \section*{\indexname}%
  \setlength{\parindent}{0pt}%
  \setlength{\parskip}{0pt plus 0.3pt}%
  \let\item\@idxitem
}{%
  \clearpage
}
\makeatother

\IfFileExists{\jobname-pw.ind}{\input{\jobname-pw.ind}}{}

\end{document}

      