%% latex-korrekturansicht-vorspann.tex
%% Vorspann für die Korrekturansicht.
%% Lädt die gemeinsame Datei latex-vorspann.tex mit gesetztem Schalter.

\newif\ifkorrekturansicht
\korrekturansichttrue

\input{../tex-inputs/latex-vorspann}


               \section[Arthur Schnitzler an Thomas Mann, 18. 11. 1923]{ Arthur Schnitzler an Thomas Mann, 18. 11. 1923}\nopagebreak\mylabel{v}\rehead{ }\normalsize\beginnumbering\briefempfaengerindex{Mann, Thomas@\textsc{Mann, Thomas}!zzzSchnitzler, Arthur@\emph{von Arthur Schnitzler}!1923-11-181@{18. 11. 1923}|(be} \toendnotes[C]{\smallbreak\pagebreak[2]} \Standort{Zürich, Thomas-Mann-Archiv, B-II-SCHNM-2.}
\physDesc{Brief, 1 Blatt, 2 Seiten, Umschlag
\newline{}Handschrift: schwarze Tinte, lateinische Kurrent}\toendnotes[C]{\smallbreak}\pstart{}{\pb}\label{T_L02404-1v}\edtext{\textcolor{gray}{\textbf{A. S.}}}{\lemma{\textnormal{\emph{A. S.}}}\Cendnote{\textnormal{ovaler Absenderkleber}}}\label{T_L02404-1h}\pend{}\pstart{}\textcolor{pink}{\textcolor{gray}{\textbf{WIEN, XVIII.}}}{}\ledrightnote{\textcolor{pink}{XVIII., Währing}}\pend{}\pstart{}\textcolor{pink}{\textcolor{gray}{\textbf{STERNWARTESTR. 71}}}{}\ledrightnote{\textcolor{pink}{Sternwartestraße}}\pend{}{\bigskip}\pstart{}{\pb}Herrn Thomas Mann\pend{}\pstart{}\textcolor{pink}{München}{}\ledrightnote{\textcolor{pink}{München}}\pend{}\pstart{}\strikeout{Puch}{ }{[}ms.:{]} \textcolor{pink}{Puschingerstr. 1}{}\ledrightnote{\textcolor{pink}{Poschingerstraße}}. \pend{}{\bigskip}\pstart
           \raggedleft{}{\pb}\textcolor{pink}{Wien}{}\ledrightnote{\textcolor{pink}{Wien}}, 18. 11. 923\pend
           \pstart{}lieber und verehrter Herr Thomas Mann,\pend\pstart
           dürft ich mir im geringsten das Recht und die Kraft zugestehen, Sie zu
                    Fortführung u Beendigung des \textcolor{green}{Felix Krull}{}\ledrightnote{\textcolor{green}{Bekenntnisse des Hochstaplers Felix Krull}}
                    anzuspornen, ich thät es, we{\geminationn} man so sagen darf,
                    aus vollen Stiefeln. Das \textcolor{green}{Fragment}{}\ledrightnote{→\textcolor{green}{Bekenntnisse des Hochstaplers Felix Krull}}, das vorliegt, find ich köstlich und kostbar. Ich weiß nicht,
                    ob Sie selbst (verzeihen Sie die Anmaßung) die völlige Einzigartigkeit Ihrer
                    Stimme so zu spüren im Stande sind, wie der Leser – aber ich wünschte, daß Sie
                    das »Buch der Kindheit« einmal nur als Kenner und Genießer, nicht nebstbei als
                    der Verfasser sich zu Gemüthe führten, – Sie hätten die reinste Freude und
                    empfänden die Pflicht und den Drang zu »\textcolor{gray}{erinnern}«, – wie ich
                    sie empfand.\pend
           \pstart
           Ich wünschte zum Beschluss dieser Zeilen {\pb}nicht von der \textcolor{pink}{Stadt}{}\ledrightnote{→\textcolor{pink}{München}}
                    reden, in der Sie leben, von der Welt, in der wir alle leben – nur die Hoffnung
                    aussprechen, daß Sie mit den Ihren sich so wohl befinden, als es überhaupt
                    möglich. Man erzählt sich, dß Sie bald nach \textcolor{pink}{Wien}{}\ledrightnote{\textcolor{pink}{Wien}}
                    kommen wollen. Wir sehen einander hoffentlich gewiss wieder.\pend
           \pstart
           Seien Sie, mit Ihrer verehrten \textcolor{blue}{Gattin}{}\ledrightnote{→\textcolor{blue}{Katia Mann}}{\\[\baselineskip]}sehr herzlich gegrüßt von Ihrem{\\[\baselineskip]}freundschaftlich ergebenen{\\[\baselineskip]}\spacefill\mbox{Arthur Schnitzler}\pend
           \leftskip=0em{}\pstart
           \noindent{}{[}({]}Darf ich vielleicht auch noch erwähnen, daß mein
                        21jähriger \textcolor{blue}{Sohn}{}\ledrightnote{→\textcolor{blue}{Heinrich Schnitzler}}, wie
                        meine 14jährige \textcolor{blue}{Tochter}{}\ledrightnote{→\textcolor{blue}{Lili Schnitzler}}
                        (die ein bischen über ihre Jahre hinaus ist) von Ihrem \textcolor{green}{Fragment}{}\ledrightnote{→\textcolor{green}{Bekenntnisse des Hochstaplers Felix Krull}} in gleicher Weise entzückt
                        waren?) \pend
           \endnumbering\briefempfaengerindex{Mann, Thomas@\textsc{Mann, Thomas}!zzzSchnitzler, Arthur@\emph{von Arthur Schnitzler}!1923-11-181@{18. 11. 1923}|)be}\mylabel{h}  \normalsize

\doendnotes{C}
\bigskip
\vfill

\clearpage

\footnotesize

\lohead{\textsc{register}}

% Definiere theindex-Environment komplett neu ohne reledmac
\makeatletter
\renewenvironment{theindex}{%
  \section*{\indexname}%
  \setlength{\parindent}{0pt}%
  \setlength{\parskip}{0pt plus 0.3pt}%
  \let\item\@idxitem
}{%
  \clearpage
}
\makeatother

\IfFileExists{\jobname-pw.ind}{\input{\jobname-pw.ind}}{}

\end{document}

      