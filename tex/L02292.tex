%% latex-korrekturansicht-vorspann.tex
%% Vorspann für die Korrekturansicht.
%% Lädt die gemeinsame Datei latex-vorspann.tex mit gesetztem Schalter.

\newif\ifkorrekturansicht
\korrekturansichttrue

\input{../tex-inputs/latex-vorspann}


               \section[Arthur Schnitzler an Georg Brandes, 2. 8. 1918]{ Arthur Schnitzler an Georg Brandes, 2. 8. 1918}\nopagebreak\mylabel{v}\rehead{ }\normalsize\beginnumbering\briefempfaengerindex{Brandes, Georg@\textsc{Brandes, Georg}!zzzSchnitzler, Arthur@\emph{von Arthur Schnitzler}!1918-08-021@{2. 8. 1918}|(be} \toendnotes[C]{\smallbreak\pagebreak[2]} \Standort{Kopenhagen, Det Kongelige Bibliotek, Georg Brandes Arkiv, box 125.}
\physDesc{Brief, 2 Blätter, 4 Seiten
\newline{}Handschrift: schwarze Tinte, lateinische Kurrent\newline{}Ordnung: mit Bleistift von unbekannter Hand nummeriert: »40.«
                                    und mit »Schnitzler« beschriftet }\buchAbdrucke{\weitereDrucke{1) Georg Brandes, Arthur Schnitzler: \emph{Ein Briefwechsel}. Hg. Kurt Bergel. Bern: \emph{Francke} 1956, S. 122–123.} \weitereDrucke{2) Arthur Schnitzler: \emph{Briefe 1913–1931}. Hg. Peter Michael Braunwarth, Richard Miklin, Susanne Pertlik und Heinrich Schnitzler. Frankfurt am Main: \emph{S. Fischer} 1984, S. 165–166.} }\toendnotes[C]{\smallbreak}\pstart
           \raggedleft{}{\pb}2. 8 1918{\\}\textcolor{pink}{Wien XVII. Sternwartestr. 71}{}\ledrightnote{\textcolor{pink}{Sternwartestraße}}\pend
           \pstart{}mein lieber und verehrter Herr Brandes,\pend\pstart
           ich lese vom Tode \textcolor{blue}{Peter Nansen}{}\ledrightnote{\textcolor{blue}{Peter Nansen}}s, und habe das
                    Bedürfnis irgend jemandem zu sagen, wie tief mich das Hinscheiden dieses
                    liebenswerthen Menschen bewegt, den ich zuletzt kurz vor Ausbruch des Kriegs bei
                    mir in \textcolor{pink}{Wien}{}\ledrightnote{\textcolor{pink}{Wien}} gesehen habe – schon recht verändert, ja irgendwie gezeichnet – aber
                    doch noch von dem ganzen Zauber seines Wesens umwittert, den ich, fast mehr als
                    aus seinen reizvollen Büchern, aus seinem Gehaben, seiner Art zu sprechen,
                    seinem Schweigen, seinen Blicken zu spüren vermeinte. Nun fügt es der Zufall,
                    daß ich mir gerade in der letzten Zeit Ihre Briefe, lieber und verehrter Freund
                    abschreiben ließ – einige, mit Bleistift geschrieben, waren fast unlesbar
                    geworden, – und nun, da ich sie, \uline{vom ersten} bis
                    zum letzten, \strikeout{alle} – mit welchem Vergnügen! –
                    wieder durchnahm, fand ich öfters \textcolor{blue}{Peter
                        Nansen}{}\ledrightnote{\textcolor{blue}{Peter Nansen}}s Namen wiederkehren; auch von seinem Kranksein ist die Rede
                    darin, und da liegt es nahe mich mit meinem Beileid, – meinem Leid an Sie zu
                    wenden, der \textcolor{blue}{Nansen}{}\ledrightnote{\textcolor{blue}{Peter Nansen}}s Freund war und für mich
                    zugleich, und für die meisten Mitlebenden, {\pb}der repraesentative Mann \textcolor{pink}{Daenemarks}{}\ledrightnote{\textcolor{pink}{Dänemark}} ist. Und
                    ich benutze die Gelegenheit Ihnen wieder einmal, über diese zerrissene und
                    stöhnende Welt hin\substVorne{}\textsuperscript{über}\substDazwischen{}weg\substHinten{}, die Hand zu drücken um Ihnen zu sagen, mit welcher Sympathie, ja darf
                    ich es etwas sentimental ausdrücken –: mit welcher Sehnsucht ich Ihrer gedenke!
                    Von Ihren letzten Büchern haben Sie mir geschrieben;– vom \textcolor{green}{\textcolor{blue}{Goethe}{}\ledrightnote{\textcolor{blue}{Johann Wolfgang von Goethe}}}{}\ledrightnote{→\textcolor{green}{Wolfgang Goethe}} und \textcolor{green}{\textcolor{blue}{Voltaire}{}\ledrightnote{\textcolor{blue}{Voltaire}}}{}\ledrightnote{→\textcolor{green}{Voltaire und sein Jahrhundert}};– sie existiren noch nicht in deutscher Sprache, – und nun werden Sie wohl
                    auch Ihren \textcolor{green}{\textcolor{blue}{Julius Caesar}{}\ledrightnote{\textcolor{blue}{Gaius Iulius Caesar}}}{}\ledrightnote{→\textcolor{green}{Gaius Julius Cæsar}} bald abschliessen. Aber wa{\geminationn} werd ich
                    Ignorant, der nicht daenisch versteht, sie endlich lesen dürfen? – Auch ich hab
                    allerlei gemacht – nicht so bedeutungsvolles! – und nach meiner alten
                    zudringlichen Gewohnheit werd ich Ihnen ein \textcolor{green}{Stück}{}\ledrightnote{→\textcolor{green}{Die Schwestern oder Casanova in Spa. Lustspiel in Versen}} und eine \textcolor{green}{Novelle}{}\ledrightnote{→\textcolor{green}{Casanovas Heimfahrt}} zusenden, sobald sie gedruckt sind. – Aber
                    wann werden wir einander wiedersehen? Lassen Sie mich doch bald wieder – und
                    wärs nur mit einem Wort, wissen, daß Sie sich wohl befinden und Ihre edle Stirn
                    über den Dunst und Dampf dieser Jammerwelt in {\pb}reinere Lüfte emporzurecken vermögen. Ihnen im neutralen \textcolor{pink}{Land}{}\ledrightnote{→\textcolor{pink}{Dänemark}} ist es doch immerhin leichter als
                    uns. In meiner Familie geht es ganz leidlich; mein \textcolor{blue}{Bub}{}\ledrightnote{→\textcolor{blue}{Heinrich Schnitzler}} (wird 16) meine \textcolor{blue}{Tochter}{}\ledrightnote{→\textcolor{blue}{Lili Schnitzler}} (wird 9) entwickeln sich in jeder
                    Hinsicht gut; meine \textcolor{blue}{Frau}{}\ledrightnote{→\textcolor{blue}{Olga Schnitzler}}
                    hat wohl unter den häuslichen Kriegswirtschaftssorgen wie jede u jeder etwas
                    gelitten, trotzdem aber ihre Kunst nicht vernachlässigt, ihre Stimme entwickelt
                    sich aufs schönste. Nun ist sie bei ihrer \textcolor{blue}{Schwester}{}\ledrightnote{→\textcolor{blue}{Elisabeth Steinrück}} in \textcolor{pink}{Bayern}{}\ledrightnote{\textcolor{pink}{Bayern}}
                        (\textcolor{pink}{Partenkirchen}{}\ledrightnote{\textcolor{pink}{Garmisch-Partenkirchen}}) wohin ich Mitte dieses
                    Monats auch zu fahren gedenke. Über politisches ka{\geminationn}
                    ich mich in einem Brief nicht so ausführlich äußern als ich möchte – wie
                    complicirt gerade bei uns all diese Probleme sind, ersehen Sie aus jeder
                    Zeitung, selbst aus dem censurirtesten \textcolor{pink}{Wien}{}\ledrightnote{\textcolor{pink}{Wien}}er
                    Blatt. Und trotz aller Schwierigkeiten – Misslichkeiten – Unsicherheiten: wie
                    viel Auftrieb, Sti{\geminationm}ungskraft, Talent – welche
                    positive Möglichkeiten in diesem \textcolor{pink}{Land}{}\ledrightnote{→\textcolor{pink}{Österreich}}, das vielleicht nicht {\pb}alle
                    seine Bewohner als »Vaterland« aber jeder als »Heimat« liebt. Ich muß hier
                    innehalten – trotzdem ich daran bin, viel freundlicheres über \textcolor{pink}{Oesterreich}{}\ledrightnote{\textcolor{pink}{Österreich}} zu sagen, als es \introOben{}selbst\introOben{} unsere officiösen Zeitungen zu thun pflegen.\pend
           \pstart
           Bitte bestätigen Sie mir bald den Empfang dieses Briefes und erhalten Sie mir und
                    den Meinen Ihre Freundschaft.\pend
           \pstart
           Von Herzen{\\[\baselineskip]}Ihr{\\[\baselineskip]}\spacefill\mbox{Arthur Schnitzler}\pend
           \leftskip=0em{}\endnumbering\briefempfaengerindex{Brandes, Georg@\textsc{Brandes, Georg}!zzzSchnitzler, Arthur@\emph{von Arthur Schnitzler}!1918-08-021@{2. 8. 1918}|)be}\mylabel{h}  \normalsize

\doendnotes{C}
\bigskip
\vfill

\clearpage

\footnotesize

\lohead{\textsc{register}}

% Definiere theindex-Environment komplett neu ohne reledmac
\makeatletter
\renewenvironment{theindex}{%
  \section*{\indexname}%
  \setlength{\parindent}{0pt}%
  \setlength{\parskip}{0pt plus 0.3pt}%
  \let\item\@idxitem
}{%
  \clearpage
}
\makeatother

\IfFileExists{\jobname-pw.ind}{\input{\jobname-pw.ind}}{}

\end{document}

      