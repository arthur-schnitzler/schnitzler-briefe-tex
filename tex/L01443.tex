%% latex-korrekturansicht-vorspann.tex
%% Vorspann für die Korrekturansicht.
%% Lädt die gemeinsame Datei latex-vorspann.tex mit gesetztem Schalter.

\newif\ifkorrekturansicht
\korrekturansichttrue

\input{../tex-inputs/latex-vorspann}


               \section[Richard Beer-Hofmann an Arthur Schnitzler, 13. 9. 1904]{ Richard Beer-Hofmann an Arthur Schnitzler,
               13. 9. 1904}\nopagebreak\mylabel{v}\rehead{ }\normalsize\beginnumbering\briefempfaengerindex{Schnitzler, Arthur@\textsc{Schnitzler, Arthur}!zzzBeer-Hofmann, Richard@\emph{von Richard Beer-Hofmann}!1904-09-131@{13. 9. 1904}|(be} \toendnotes[C]{\smallbreak\pagebreak[2]} \Standort{CUL, Schnitzler, B 8.}
\physDesc{Brief, 1 Blatt, 2 Seiten
\newline{}Handschrift: blaue Tinte, lateinische Kurrent\newline{}Ordnung: mit Bleistift von unbekannter Hand nummeriert: »189« }\buchAbdrucke{\weitereDrucke{Arthur Schnitzler, Richard Beer-Hofmann: \emph{Briefwechsel 1891–1931}. Hg. Konstanze Fliedl. Wien, Zürich: \emph{Europaverlag} 1992, S. 166.} }\toendnotes[C]{\smallbreak}\pstart
           \centering{}{\pb}\textcolor{pink}{Aussee}{}\ledrightnote{\textcolor{pink}{Bad Aussee}}{ }13/IX 04\pend
           \pstart
           Lieber Arthur! Wir sind \textcolor{blue}{beide}{}\ledrightnote{→\textcolor{blue}{Paula Beer-Hofmann}} stark erkältet und haben nun keine anderen Reisepläne
               als \textcolor{pink}{Rodaun}{}\ledrightnote{\textcolor{pink}{Rodaun}}. Vielleicht – nicht wahrscheinlich –
               passiren wir – wenn wir über \textcolor{pink}{Salzburg}{}\ledrightnote{\textcolor{pink}{Salzburg}} gehen – \textcolor{pink}{Lueg}{}\ledrightnote{\textcolor{pink}{Lueg am Wolfgangsee}}. – Bitte verständigen Sie mich wann Sie in \textcolor{pink}{Wien}{}\ledrightnote{\textcolor{pink}{Wien}} sind. Ich will {\pb}am 18. oder
                  19. zu Hause sein.\pend
           \pstart
           Herzliche Grüße an Sie \textcolor{blue}{Beide}{}\ledrightnote{→\textcolor{blue}{Olga Schnitzler}}.\pend
           \pstart
           Ihr{\\[\baselineskip]}\spacefill\mbox{Richard}\pend
           \leftskip=0em{}\endnumbering\briefempfaengerindex{Schnitzler, Arthur@\textsc{Schnitzler, Arthur}!zzzBeer-Hofmann, Richard@\emph{von Richard Beer-Hofmann}!1904-09-131@{13. 9. 1904}|)be}\mylabel{h}  \normalsize

\doendnotes{C}
\bigskip
\vfill

\clearpage

\footnotesize

\lohead{\textsc{register}}

% Definiere theindex-Environment komplett neu ohne reledmac
\makeatletter
\renewenvironment{theindex}{%
  \section*{\indexname}%
  \setlength{\parindent}{0pt}%
  \setlength{\parskip}{0pt plus 0.3pt}%
  \let\item\@idxitem
}{%
  \clearpage
}
\makeatother

\IfFileExists{\jobname-pw.ind}{\input{\jobname-pw.ind}}{}

\end{document}

      