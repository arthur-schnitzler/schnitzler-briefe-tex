%% latex-korrekturansicht-vorspann.tex
%% Vorspann für die Korrekturansicht.
%% Lädt die gemeinsame Datei latex-vorspann.tex mit gesetztem Schalter.

\newif\ifkorrekturansicht
\korrekturansichttrue

\input{../tex-inputs/latex-vorspann}


               \section[Hermann Bahr an Arthur Schnitzler, 5. 8. 1905]{ Hermann Bahr an Arthur Schnitzler, 5. 8. 1905}\nopagebreak\mylabel{v}\rehead{ }\normalsize\beginnumbering\briefempfaengerindex{Schnitzler, Arthur@\textsc{Schnitzler, Arthur}!zzzBahr, Hermann@\emph{von Hermann Bahr}!1905-08-051@{5. 8. 1905}|(be} \toendnotes[C]{\smallbreak\pagebreak[2]} \Standort{CUL, Schnitzler, B 5b.}
\physDesc{Bildpostkarte
\newline{}Handschrift: Bleistift, deutsche Kurrent\newline{}Versand: Stempel: »\nobreak{}\oindex{Muenchen@\textbf{München}, \emph{https://www.geonames.org/ontologyP.PPLA}|pwk}München–Glaspalast, 5 Aug 05, 12–1\nobreak{}«.  \newline{}Ordnung: mit Bleistift von unbekannter Hand nummeriert:
                              »131« }\buchAbdrucke{\weitereDrucke{Hermann Bahr, Arthur Schnitzler: \emph{Briefwechsel, Aufzeichnungen, Dokumente (1891–1931)}. Hg. Kurt Ifkovits und Martin Anton Müller. Göttingen: \emph{Wallstein} 2018, S. 349.} }\toendnotes[C]{\smallbreak}\pstart{}{\pb}\textsc{D\textsuperscript{r} Artur Schnitzler}\pend{}\pstart{}\textcolor{pink}{\textsc{Wien XVIII}}{}\ledrightnote{\textcolor{pink}{XVIII., Währing}}\pend{}\pstart{}\textcolor{pink}{\textsc{Spöttelgasse 7}}{}\ledrightnote{\textcolor{pink}{Edmund-Weiß-Gasse}}\pend{}{\bigskip}\pstart
           \noindent{}\centering{}\textcolor{gray}{\textbf{{\pb}\textcolor{pink}{München. Glaspalast}{}\ledrightnote{\textcolor{pink}{Glaspalast}}}}\pend
           \pstart
           {\pb}5. 8.\pend
           \pstart
           Einſtweilen herzlichſten Dank für Deinen lieben Brief. Mit allem anderen magſt Du
               recht haben, mit \textcolor{green}{\textsc{Besenius}}{}\ledrightnote{→\textcolor{green}{Die Andere}} nicht. Für mich müßte das \textcolor{green}{Stück}{}\ledrightnote{→\textcolor{green}{Die Andere}} eigentlich \textcolor{green}{\textsc{Besenius}}{}\ledrightnote{→\textcolor{green}{Die Andere}} heißen, da ſein Thema
               iſt: 1) Was kann ein wirklicher Menſch heute werden? Antwort: \textcolor{green}{\textsc{Besenius}}{}\ledrightnote{→\textcolor{green}{Die Andere}}. 2) Wie
               wird man \textcolor{green}{\textsc{Besenius}}{}\ledrightnote{→\textcolor{green}{Die Andere}}? Wenn man \textcolor{green}{Heinrich}{}\ledrightnote{→\textcolor{green}{Die Andere}} iſt und dies erlebt.\pend
           \pstart Herzlichſt \spacefill\mbox{H.}\pend{}\pstart
           \noindent{}Viele Grüße Deiner \textcolor{blue}{Frau}{}\ledrightnote{→\textcolor{blue}{Olga Schnitzler}}\pend
           \endnumbering\briefempfaengerindex{Schnitzler, Arthur@\textsc{Schnitzler, Arthur}!zzzBahr, Hermann@\emph{von Hermann Bahr}!1905-08-051@{5. 8. 1905}|)be}\mylabel{h}  \normalsize

\doendnotes{C}
\bigskip
\vfill

\clearpage

\footnotesize

\lohead{\textsc{register}}

% Definiere theindex-Environment komplett neu ohne reledmac
\makeatletter
\renewenvironment{theindex}{%
  \section*{\indexname}%
  \setlength{\parindent}{0pt}%
  \setlength{\parskip}{0pt plus 0.3pt}%
  \let\item\@idxitem
}{%
  \clearpage
}
\makeatother

\IfFileExists{\jobname-pw.ind}{\input{\jobname-pw.ind}}{}

\end{document}

      