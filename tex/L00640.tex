%% latex-korrekturansicht-vorspann.tex
%% Vorspann für die Korrekturansicht.
%% Lädt die gemeinsame Datei latex-vorspann.tex mit gesetztem Schalter.

\newif\ifkorrekturansicht
\korrekturansichttrue

\input{../tex-inputs/latex-vorspann}


               \section[Hugo von Hofmannsthal an Arthur Schnitzler, {[}16. 1. 1897{]}]{ Hugo von Hofmannsthal an Arthur Schnitzler, {[}16. 1. 1897{]}}\nopagebreak\mylabel{v}\rehead{ }\normalsize\beginnumbering\briefempfaengerindex{Schnitzler, Arthur@\textsc{Schnitzler, Arthur}!zzzHofmannsthal, Hugo von@\emph{von Hugo von Hofmannsthal}!1897-01-163@{{[}16. 1. 1897{]}}|(be} \toendnotes[C]{\smallbreak\pagebreak[2]} \Standort{CUL, Schnitzler, B 43.}
\physDesc{Brief, 1 Blatt, 3 Seiten
\newline{}Handschrift: schwarze Tinte, deutsche Kurrent
\newline{}Schnitzler: mit Bleistift datiert: »16/1 97« \newline{}Ordnung: mit Bleistift von unbekannter Hand nummeriert:
                                        »85« }\buchAbdrucke{\weitereDrucke{Hugo von Hofmannsthal, Arthur Schnitzler: \emph{Briefwechsel}. Hg. Therese Nickl und Heinrich Schnitzler. Frankfurt am Main: \emph{S. Fischer} 1964, S. 77.} }\toendnotes[C]{\smallbreak}\pstart
           \noindent{}{\pb}\textcolor{gray}{\textbf{\label{T_L00640-1v}\edtext{hvH}{\lemma{\textnormal{\emph{hvH}}}\Cendnote{\textnormal{gedrucktes Monogramm mit Krone in blauer Farbe}}}\label{T_L00640-1h}}}\pend
           \pstart
           \raggedleft{}\label{K_L00640_1v}\edtext{Samstag}{\lemma{\textnormal{\emph{Samstag}}}\Cendnote{\textnormal{Am Samstag,
                                    16. 1. 1897 erschien der dritte und letzte Teil des
                                Erstdrucks von \emph{\textcolor{green}{Die Frau des Weisen.
                                    Erzählung}} in der Wochenschrift \emph{\textcolor{brown}{Die Zeit}} (Bd. 10, Nr. 118,
                                        2. 1. 1897, S. 15–16; Nr. 119,
                                        9. 1. 1897, S. 31–32; Nr. 120,
                                        16. 1. 1897, S. 47–48).}}}\label{K_L00640_1h}.\pend
           \pstart{}mein lieber Arthur\pend\pstart
           ich ſehe Sie, glaub ich, weder heute im Café noch \label{K_L00640_2v}\edtext{morgen}{\lemma{\textnormal{\emph{morgen}}}\Cendnote{\textnormal{Am
                            17. 1. 1897 ist \textcolor{blue}{Hofmannsthal} bei \textcolor{blue}{Louis}
                    und \textcolor{blue}{Regina Loeb} (\textcolor{blue}{Hugo von Hofmannsthal}: \emph{Aufzeichnungen}. Hg. Rudolf Hirsch † und Ellen Ritter † in Zusammenarbeit
                        mit Konrad Heumann und Peter Michael Braunwarth. Frankfurt am Main: \emph{\textcolor{brown}{S. Fischer}}{ }2013, S. 378 (\emph{Sämtliche Werke},
                        XXXIX)).}}}\label{K_L00640_2h} bei
                        \textcolor{blue}{L.}{}\ledrightnote{\textcolor{blue}{Louis Loeb}{\newline}\textcolor{blue}{Regina Loeb}} und möchte Ihnen doch ſagen,
                    daſs die »\textcolor{green}{Frau des Weiſen}{}\ledrightnote{\textcolor{green}{Die Frau des Weisen. Erzählung}}« eine ſehr ſchöne
                    Novelle iſt. Ich war von der Führung des Schluſſes überraſcht wie von einer
                    völlig unerwarteten und {\pb}doch
                    unendlich einfachen naheliegenden Löſung einer Rechenaufgabe, das was man in der
                    Mathematik eine »ſchöne Löſung« nennt. Auch iſt alles Äußerliche, das den
                    Fortgang der Handlung unterſtützt, wunderſchön ſparſam und durchſichtig. Man
                    ſieht die Landſchaft nicht, man glaubt ſich in ihr zu bewegen\strikeout{d}, und {\pb}fühlt \uline{unmittelbar} ihre Wirkung auf’s Gemüth der handelnden
                    Perſonen.\pend
           \pstart
           Ich bin ſchläfrig, und kann mich nicht gut ausdrücken. Sie waren übrigens in den
                    letzten Tagen beſonders lieb und nett gegen mich.\pend
           \pstart
           Herzlich Ihr{\\[\baselineskip]}\spacefill\mbox{Hugo.}\pend
           \leftskip=0em{}\endnumbering\briefempfaengerindex{Schnitzler, Arthur@\textsc{Schnitzler, Arthur}!zzzHofmannsthal, Hugo von@\emph{von Hugo von Hofmannsthal}!1897-01-163@{{[}16. 1. 1897{]}}|)be}\mylabel{h}  \normalsize

\doendnotes{C}
\bigskip
\vfill

\clearpage

\footnotesize

\lohead{\textsc{register}}

% Definiere theindex-Environment komplett neu ohne reledmac
\makeatletter
\renewenvironment{theindex}{%
  \section*{\indexname}%
  \setlength{\parindent}{0pt}%
  \setlength{\parskip}{0pt plus 0.3pt}%
  \let\item\@idxitem
}{%
  \clearpage
}
\makeatother

\IfFileExists{\jobname-pw.ind}{\input{\jobname-pw.ind}}{}

\end{document}

      