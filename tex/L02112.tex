%% latex-korrekturansicht-vorspann.tex
%% Vorspann für die Korrekturansicht.
%% Lädt die gemeinsame Datei latex-vorspann.tex mit gesetztem Schalter.

\newif\ifkorrekturansicht
\korrekturansichttrue

\input{../tex-inputs/latex-vorspann}


               \section[Hugo von Hofmannsthal an Arthur Schnitzler, 3. 1. 1913]{ Hugo von Hofmannsthal an Arthur Schnitzler, 3. 1. 1913}\nopagebreak\mylabel{v}\rehead{ }\normalsize\beginnumbering\briefempfaengerindex{Schnitzler, Arthur@\textsc{Schnitzler, Arthur}!zzzHofmannsthal, Hugo von@\emph{von Hugo von Hofmannsthal}!1913-01-031@{3. 1. 1913}|(be} \toendnotes[C]{\smallbreak\pagebreak[2]} \Standort{CUL, Schnitzler, B 43.}
\physDesc{Brief, 1 Blatt, 3 Seiten
\newline{}Handschrift: schwarze Tinte, deutsche Kurrent
\newline{}Schnitzler: mit Bleistift beschriftet: »\textsc{Hofmannsthal}« \newline{}Ordnung: 1) mit Bleistift von unbekannter Hand nummeriert: »\strikeout{333}« 2) mit Bleistift von unbekannter Hand nummeriert: »346«}\buchAbdrucke{\weitereDrucke{Hugo von Hofmannsthal, Arthur Schnitzler: \emph{Briefwechsel}. Hg. Therese Nickl und Heinrich Schnitzler. Frankfurt am Main: \emph{S. Fischer} 1964, S. 271–272.} }\toendnotes[C]{\smallbreak}\pstart
           \noindent{}{\pb}\textcolor{gray}{\textbf{Schloss \textcolor{pink}{Neubeuern}{}\ledrightnote{\textcolor{pink}{Neubeuern}}{ }\textsuperscript{a}/Inn}}\pend
           \pstart
           \textcolor{gray}{\textbf{Oberbayern}}\pend
           \pstart
           \raggedleft{}3 I 13.\pend
           \pstart{}mein lieber Arthur \pend\pstart
           Dr. \textcolor{blue}{Eger}{}\ledrightnote{\textcolor{blue}{Paul Eger}} hat am 28. XII. die Sache
               durch ein directes Geſpräch mit \textcolor{blue}{Thimig}{}\ledrightnote{\textcolor{blue}{Hugo Thimig}} recht gut
               eingeleitet so daſs ich nun ganz ausnahmsweiſe die \strikeout{directe} Bitte an Sie ſtellen möchte, eine Begegnung mit dem gleichen Mann
               mir zu Liebe und mit directem Hinweis auf meine Perſon und meine an Sie gerichtete
               Bitte in der allernächſten Zeit zu ſuchen, nicht mehr ihre Herbeiführung dem Zufall
               zu überlaſſen. Denn es liegt mir doch recht viel an der Sache und ſie hat
               einigermaßen Eile, weil der einzig mögliche Termin vor Oſtern iſt, und
               zwar 8–10 Tage \uline{vor}{ }Oſtern mindeſtens, und Oſtern fällt ſchon auf den \label{K_L02112_1v}\edtext{22\textsuperscript{ten} März}{\lemma{\textnormal{\emph{22ten März}}}\Cendnote{\textnormal{Ostersonntag war der
                     23. 3. 1913.}}}\label{K_L02112_1h}.\pend
           \pstart
           \textcolor{blue}{Thimig}{}\ledrightnote{\textcolor{blue}{Hugo Thimig}}s einziges Bedenken war, die Kritik könne
               die \textcolor{blue}{Reinhardt}{}\ledrightnote{\textcolor{blue}{Max Reinhardt}}ſche Aufführung gegen ihn
               ausſpielen, worauf schon \textcolor{blue}{Eger}{}\ledrightnote{\textcolor{blue}{Paul Eger}} erwiderte:
               1.) ſchreibe gerade in den großen Blättern ein anderer Referent als {\pb}der über \textcolor{blue}{R.}{}\ledrightnote{\textcolor{blue}{Max Reinhardt}} geſchrieben habe, 2\textsuperscript{t\textcolor{gray}{ens}}: ſei, mit
               geringen Ausnahmen, immer noch eine reſpectvolle Prädispoſition für das \textcolor{brown}{Burgtheater}{}\ledrightnote{\textcolor{brown}{Burgtheater}} vorhanden und 3\textsuperscript{\textcolor{gray}{tens}} könne die Vorſtellung gerade dieſes \textcolor{green}{Stückes}{}\ledrightnote{→\textcolor{green}{Jedermann. Das Spiel vom Sterben des reichen Mannes}} ganz vortrefflich \label{T_L02112_1v}\edtext{werden und werde (wenn man von dem
               einzigen \textcolor{blue}{\textsc{Moissi}}{}\ledrightnote{\textcolor{blue}{Alexander Moissi}} absehe) den Vergleich}{\lemma{\textnormal{\emph{werden … Vergleich}}}\Cendnote{\textnormal{durch Umstellung korrigiert aus:
                     »werden (wenn man von dem einzigen \textcolor{blue}{\textsc{Moissi}} absehe) und werde den Vergleich«.}}}\label{T_L02112_1h} nicht zu ſcheuen
               haben.\pend
           \pstart
           Ich bin in \uline{dieſem} Falle auch ſicher, dem Regiſſeur
               ſehr erfolgreich zur Seite ſein zu können, da mir nach \textcolor{blue}{Reinhardt}{}\ledrightnote{\textcolor{blue}{Max Reinhardt}} und nach \textcolor{pink}{Dresden}{}\ledrightnote{\textcolor{pink}{Dresden}}, jedes Detail
               des Sceniſchen und Schauſpieleriſchen mit ungewöhnlicher Präciſion innerlich zur
               Verfügung iſt.\hspace*{1.5em}Ich würde als Regiſſeur \textcolor{blue}{Thimig}{}\ledrightnote{\textcolor{blue}{Hugo Thimig}}{ }ſelbst oder \textcolor{blue}{Heine}{}\ledrightnote{\textcolor{blue}{Albert Heine}} zur Bedingung machen.\pend
           \pstart
           Ich wäre Ihnen herzlich dankbar, lieber Arthur. Ich bin etwa den 8\textsuperscript{ten} wieder in \textcolor{pink}{Rodaun}{}\ledrightnote{\textcolor{pink}{Rodaun}}, vielleicht finde ich da
               ein Wort von Ihnen.\pend
           \pstart
           Ihr{\\[\baselineskip]}\spacefill\mbox{Hugo.}\pend
           \leftskip=0em{}\endnumbering\briefempfaengerindex{Schnitzler, Arthur@\textsc{Schnitzler, Arthur}!zzzHofmannsthal, Hugo von@\emph{von Hugo von Hofmannsthal}!1913-01-031@{3. 1. 1913}|)be}\mylabel{h}  \normalsize

\doendnotes{C}
\bigskip
\vfill

\clearpage

\footnotesize

\lohead{\textsc{register}}

% Definiere theindex-Environment komplett neu ohne reledmac
\makeatletter
\renewenvironment{theindex}{%
  \section*{\indexname}%
  \setlength{\parindent}{0pt}%
  \setlength{\parskip}{0pt plus 0.3pt}%
  \let\item\@idxitem
}{%
  \clearpage
}
\makeatother

\IfFileExists{\jobname-pw.ind}{\input{\jobname-pw.ind}}{}

\end{document}

      