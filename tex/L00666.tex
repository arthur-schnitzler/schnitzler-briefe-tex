%% latex-korrekturansicht-vorspann.tex
%% Vorspann für die Korrekturansicht.
%% Lädt die gemeinsame Datei latex-vorspann.tex mit gesetztem Schalter.

\newif\ifkorrekturansicht
\korrekturansichttrue

\input{../tex-inputs/latex-vorspann}


               \section[Arthur Schnitzler an Richard Beer-Hofmann, 19. 4. 1897]{ Arthur Schnitzler an Richard Beer-Hofmann, 19. 4. 1897}\nopagebreak\mylabel{v}\rehead{ }\normalsize\beginnumbering\briefempfaengerindex{Beer-Hofmann, Richard@\textsc{Beer-Hofmann, Richard}!zzzSchnitzler, Arthur@\emph{von Arthur Schnitzler}!1897-04-191@{19. 4. 1897}|(be} \toendnotes[C]{\smallbreak\pagebreak[2]} \Standort{YCGL, MSS 31.}
\physDesc{Brief, 1 Blatt, 4 Seiten, Umschlag
\newline{}Handschrift: schwarze Tinte, deutsche Kurrent\newline{}Versand: 1) Stempel: »\nobreak{}\oindex{rue La Fayette@\textbf{rue La Fayette}, \emph{Straße (K.STR)}|pwk}Paris 51 R. Lafayette, 19 Avril 97, 5\textsuperscript{E}\nobreak{}«.  2) Stempel: »\nobreak{}\oindex{I., Innere Stadt@\textbf{I., Innere Stadt}, \emph{Bezirk (A.BZK)}|pwk}Wien 1/1, 21 4. 97, 6–8½V., Bestellt\nobreak{}«. }\buchAbdrucke{\weitereDrucke{Arthur Schnitzler, Richard Beer-Hofmann: \emph{Briefwechsel 1891–1931}. Hg. Konstanze Fliedl. Wien, Zürich: \emph{Europaverlag} 1992, S. 101.} }\toendnotes[C]{\smallbreak}\pstart{}{\pb}Herrn \textsc{Dr. Rich.
                     Beer-Hofmann}\pend{}\pstart{}\textcolor{pink}{Wien}{}\ledrightnote{\textcolor{pink}{Wien}}\pend{}\pstart{}\textcolor{pink}{\textsc{I. Wollzeile 15}}{}\ledrightnote{\textcolor{pink}{Wollzeile}}.\pend{}\pstart{}\textsc{\textcolor{pink}{Autriche}{}\ledrightnote{\textcolor{pink}{Österreich}}}\pend{}{\bigskip}\pstart
           \raggedleft{}{\pb}Oſtermontag, 19. 4. 97.\pend
           \pstart
           Lieber Richard, ich weiſss ja doch nicht, wa{\geminationn} ich endlich Luſt zu einem wirklichen Brief beko{\geminationm}en werde; ſo ſchreib ich Ihnen lieber dieſe paar Worte,
               um Ihnen zu ſagen, daſs ich an \textcolor{pink}{Wien}{}\ledrightnote{\textcolor{pink}{Wien}} mit heftigem
               Widerwillen, aber an \substVorne{}\textsuperscript{p}\substDazwischen{}e\substHinten{}in paar Menſchen, die ich nicht zu ne{\geminationn}en
               brauche, mit einer Art \introOben{}von\introOben{} nicht beſonders {\pb}ſchmerzlicher Sehnſucht denke. Es geht mir ganz gut;
               aber es iſt eine verwickelte Art von Wohlbefinden, ſo daſs ich durchaus nicht
               verwundert bin, mich zu Zeiten ſehr miſerabel zu befinden. Ich bin natürlich nicht
               allein und doch viel allein; bin im weſentlichen frei und doch zuweilen gebunden;
               freue mich ſehr hier zu ſein, weiſs aber nicht wieviel auf Rechnung der {\pb}Freude ko{\geminationm}t, nicht in \textcolor{pink}{Wien}{}\ledrightnote{\textcolor{pink}{Wien}} zu ſein. Viel hier intereſſirt mich – und doch
               hab ich bei den allgemeinern Eindrücken nicht das Gefühl, neues zu erfahren; es
               beſtätigt ſich nur das meiſte. Ich glaube daſs ich gerne hier leben würde; man
               verſchwindet und iſt durchaus nicht beleidigt. Daſs Verkehr etwas ſehr großes
               bedeuten kann, ſpürt man hier; nicht {\pb}durch
               Multiplicationen ka{\geminationn} man das mit \textcolor{pink}{Wien}{}\ledrightnote{\textcolor{pink}{Wien}} vergleichen; es iſt was andres; brutaler, ſchöner und
               gemeiner. –\pend
           \pstart
           \textcolor{blue}{Paul}{}\ledrightnote{\textcolor{blue}{Paul Goldmann}} iſt auf ein paar Tage nach \textcolor{pink}{Frankfurt}{}\ledrightnote{\textcolor{pink}{Frankfurt am Main}}. Mir schreiben Sie nur weiter (nur weiter iſt gut) an
               die Adreſſe \textcolor{blue}{Pauls}{}\ledrightnote{\textcolor{blue}{Paul Goldmann}}, die ist jetzt \textsc{\textcolor{pink}{10 rue de la Bourse}{}\ledrightnote{\textcolor{pink}{rue de la Bourse}}}. – Ich wohne woanders, angenehm. Schreiben Sie mir was es Neues gibt. Aber
               ſicher, bitte. Grüßen Sie \textcolor{blue}{Hugo}{}\ledrightnote{\textcolor{blue}{Hugo von Hofmannsthal}}, \textcolor{blue}{Leo}{}\ledrightnote{\textcolor{blue}{Leo Van-Jung}}, \textcolor{blue}{Salten}{}\ledrightnote{\textcolor{blue}{Felix Salten}}, \textcolor{blue}{Schwarzk}{}\ledrightnote{\textcolor{blue}{Gustav Schwarzkopf}}, \textcolor{blue}{Paula}{}\ledrightnote{\textcolor{blue}{Paula Beer-Hofmann}} und \label{T_L00666_1v}\edtext{andere \textsc{a discrétion}. Ihr
                  \spacefill\mbox{Arthur.}}{\lemma{\textnormal{\emph{andere … Arthur.}}}\Cendnote{\textnormal{auf der ersten
                  Seite unter dem Text.}}}\label{T_L00666_1h}\pend
           \endnumbering\briefempfaengerindex{Beer-Hofmann, Richard@\textsc{Beer-Hofmann, Richard}!zzzSchnitzler, Arthur@\emph{von Arthur Schnitzler}!1897-04-191@{19. 4. 1897}|)be}\mylabel{h}  \normalsize

\doendnotes{C}
\bigskip
\vfill

\clearpage

\footnotesize

\lohead{\textsc{register}}

% Definiere theindex-Environment komplett neu ohne reledmac
\makeatletter
\renewenvironment{theindex}{%
  \section*{\indexname}%
  \setlength{\parindent}{0pt}%
  \setlength{\parskip}{0pt plus 0.3pt}%
  \let\item\@idxitem
}{%
  \clearpage
}
\makeatother

\IfFileExists{\jobname-pw.ind}{\input{\jobname-pw.ind}}{}

\end{document}

      