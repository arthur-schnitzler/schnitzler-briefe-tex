%% latex-korrekturansicht-vorspann.tex
%% Vorspann für die Korrekturansicht.
%% Lädt die gemeinsame Datei latex-vorspann.tex mit gesetztem Schalter.

\newif\ifkorrekturansicht
\korrekturansichttrue

\input{../tex-inputs/latex-vorspann}


               \section[Arthur Schnitzler an Stefan Großmann, 8. 2. 1911]{ Arthur Schnitzler an Stefan Großmann, 8. 2. 1911}\nopagebreak\mylabel{v}\rehead{ }\normalsize\beginnumbering\briefempfaengerindex{Grossmann, Stefan@\textsc{Großmann, Stefan}!zzzSchnitzler, Arthur@\emph{von Arthur Schnitzler}!1911-02-082@{8. 2. 1911}|(be} \toendnotes[C]{\smallbreak\pagebreak[2]} \Standort{Wien, Österreichische Nationalbibliothek, 98/B65/1 LIT MAG.}
\physDesc{Briefkarte
\newline{}Handschrift: schwarze Tinte, deutsche Kurrent}\pstart
           \noindent{}{\pb}Zuſatz zu dem Brief von heute morgens.\pend
           \pstart
           \textcolor{gray}{\textbf{Dr. Arthur Schnitzler}}\hfill 8. II. 911.\pend
           \pstart
           \textcolor{gray}{\textbf{\textcolor{pink}{Wien XVIII. Sternwartestrasse 71}{}\ledrightnote{\textcolor{pink}{Sternwartestraße}}}}\pend
           \pstart
           Sehr geehrter Herr, um jedes Misverſtändnis auszuſchließen:
                    auch von Herrn \textcolor{blue}{Ehrenſtein}{}\ledrightnote{\textcolor{blue}{Albert Ehrenstein}} hab ich das ihm
                    durch mich angeblich beſtätigte \uline{nie} gehört (ſo
                    daſs Sie alſo der abſolut \uline{erſte} Menſch ſind, der
                    mir zu dieſem Klatſch Mittheilung macht.)\pend
           \pstart
           Hochachtungsvoll{\\[\baselineskip]}\spacefill\mbox{A. S.}\pend
           \leftskip=0em{}\endnumbering\briefempfaengerindex{Grossmann, Stefan@\textsc{Großmann, Stefan}!zzzSchnitzler, Arthur@\emph{von Arthur Schnitzler}!1911-02-082@{8. 2. 1911}|)be}\mylabel{h}  \normalsize

\doendnotes{C}
\bigskip
\vfill

\clearpage

\footnotesize

\lohead{\textsc{register}}

% Definiere theindex-Environment komplett neu ohne reledmac
\makeatletter
\renewenvironment{theindex}{%
  \section*{\indexname}%
  \setlength{\parindent}{0pt}%
  \setlength{\parskip}{0pt plus 0.3pt}%
  \let\item\@idxitem
}{%
  \clearpage
}
\makeatother

\IfFileExists{\jobname-pw.ind}{\input{\jobname-pw.ind}}{}

\end{document}

      