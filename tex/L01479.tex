%% latex-korrekturansicht-vorspann.tex
%% Vorspann für die Korrekturansicht.
%% Lädt die gemeinsame Datei latex-vorspann.tex mit gesetztem Schalter.

\newif\ifkorrekturansicht
\korrekturansichttrue

\input{../tex-inputs/latex-vorspann}


               \section[Hugo von Hofmannsthal an Arthur Schnitzler, 14. 12. 1904]{ Hugo von Hofmannsthal an Arthur Schnitzler, 14. 12. 1904}\nopagebreak\mylabel{v}\rehead{ }\normalsize\beginnumbering\briefempfaengerindex{Schnitzler, Arthur@\textsc{Schnitzler, Arthur}!zzzHofmannsthal, Hugo von@\emph{von Hugo von Hofmannsthal}!1904-12-143@{14. 12. 1904}|(be} \toendnotes[C]{\smallbreak\pagebreak[2]} \Standort{CUL, Schnitzler, B 43.}
\physDesc{Postkarte
\newline{}Handschrift: schwarze Tinte, deutsche Kurrent\newline{}Versand: 1) Stempel: »\nobreak{}\oindex{I., Innere Stadt@\textbf{I., Innere Stadt}, \emph{Bezirk (A.BZK)}|pwk}Wien 1/1, 14{[}.{]} 12. 04, 12\textcolor{gray}{–}1N\nobreak{}«.  2) Stempel: »\nobreak{}\oindex{XVIII., Waehring@\textbf{XVIII., Währing}, \emph{Bezirk (A.BZK)}|pwk}18/1 Wien 110, 14. 12. 04, 5.N, Bestellt\nobreak{}«. 
\newline{}Schnitzler: mit Bleistift datiert: »14/12 904« \newline{}Ordnung: 1) mit Bleistift von unbekannter Hand nummeriert: »\strikeout{218}« 2) mit Bleistift von unbekannter Hand nummeriert: »243«}\buchAbdrucke{\weitereDrucke{Hugo von Hofmannsthal, Arthur Schnitzler: \emph{Briefwechsel}. Hg. Therese Nickl und Heinrich Schnitzler. Frankfurt am Main: \emph{S. Fischer} 1964, S. 207.} }\toendnotes[C]{\smallbreak}\pstart{}{\pb}\textsc{Herrn D\textsuperscript{r} Arthur Schnitzler}\pend{}\pstart{}\textcolor{pink}{\textsc{Wien}}{}\ledrightnote{\textcolor{pink}{Wien}}\pend{}\pstart{}\textcolor{pink}{\textsc{XVIII. Spöttelgasse 7}.}{}\ledrightnote{\textcolor{pink}{Edmund-Weiß-Gasse}}\pend{}{\bigskip}\pstart
           \noindent{}{\pb}lieber, unbedingt
               möchten wir den Abend des 20\textsuperscript{ten} oder 21\textsuperscript{ten} oder 22\textsuperscript{ten} bei Euch verbringen. \textcolor{blue}{Papa}{}\ledrightnote{→\textcolor{blue}{Hugo August von Hofmannsthal}} bittet mitko{\geminationm}en zu dürfen und würde es als
               ſeine Geburtstagsfeier betrachten (ſein Geburtstag iſt am 21\textsuperscript{ten}.).\pend
           \pstart
           Wir freuen uns ſehr darauf und hoffen auf Muſik, \label{K_L01479_1v}\edtext{\textsc{croc–en-bouche}}{\lemma{\textnormal{\emph{croc–en-bouche}}}\Cendnote{\textnormal{auch: Croquembouche; eine Pyramide aus
                  übereinander gestapelten und mit Creme gefüllten Windbeuteln}}}\label{K_L01479_1h} und
               Kaiſerbirnſchnaps. \textcolor{blue}{Bär}{}\ledrightnote{\textcolor{blue}{Richard Beer-Hofmann}}s \label{K_L01479_2v}\edtext{Schickſale}{\lemma{\textnormal{\emph{Schickſale}}}\Cendnote{\textnormal{Dürfte sich auf die Schwierigkeiten beziehen, die sich bei der
                  Vorbereitung der Uraufführung von \emph{\textcolor{green}{Der Graf von
                     Charolais}} am 23. 12. 1904 aufgetan hatten.}}}\label{K_L01479_2h} ſind
               furchtbar.\pend
           \pstart Ihr \spacefill\mbox{Hugo.}\pend{}\pstart
           \noindent{}\label{T_L01479_1v}\edtext{Bitte \uline{welcher} Tag!!}{\lemma{\textnormal{\emph{Bitte welcher Tag!!}}}\Cendnote{\textnormal{quer am linken
                     Rand}}}\label{T_L01479_1h}\pend
           \endnumbering\briefempfaengerindex{Schnitzler, Arthur@\textsc{Schnitzler, Arthur}!zzzHofmannsthal, Hugo von@\emph{von Hugo von Hofmannsthal}!1904-12-143@{14. 12. 1904}|)be}\mylabel{h}  \normalsize

\doendnotes{C}
\bigskip
\vfill

\clearpage

\footnotesize

\lohead{\textsc{register}}

% Definiere theindex-Environment komplett neu ohne reledmac
\makeatletter
\renewenvironment{theindex}{%
  \section*{\indexname}%
  \setlength{\parindent}{0pt}%
  \setlength{\parskip}{0pt plus 0.3pt}%
  \let\item\@idxitem
}{%
  \clearpage
}
\makeatother

\IfFileExists{\jobname-pw.ind}{\input{\jobname-pw.ind}}{}

\end{document}

      