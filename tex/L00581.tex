%% latex-korrekturansicht-vorspann.tex
%% Vorspann für die Korrekturansicht.
%% Lädt die gemeinsame Datei latex-vorspann.tex mit gesetztem Schalter.

\newif\ifkorrekturansicht
\korrekturansichttrue

\input{../tex-inputs/latex-vorspann}


               \section[Richard Beer-Hofmann an Arthur Schnitzler, 30. 8. 1896]{ Richard Beer-Hofmann an Arthur Schnitzler, 30. 8. 1896}\nopagebreak\mylabel{v}\rehead{ }\normalsize\beginnumbering\briefempfaengerindex{Schnitzler, Arthur@\textsc{Schnitzler, Arthur}!zzzBeer-Hofmann, Richard@\emph{von Richard Beer-Hofmann}!1896-08-301@{30. 8. 1896}|(be} \toendnotes[C]{\smallbreak\pagebreak[2]} \Standort{CUL, Schnitzler, B 8.}
\physDesc{Bildpostkarte
\newline{}Handschrift: Bleistift, lateinische Kurrent\newline{}Versand: 1) Stempel: »\nobreak{}\oindex{Leipzig@\textbf{Leipzig}, \emph{Besiedelter Ort (A.BSO)}|pwk}Leipzig, 30. 8. 96, 10–11 N\nobreak{}«.  2) Stempel: »\nobreak{}\oindex{IX., Alsergrund@\textbf{IX., Alsergrund}, \emph{Bezirk (A.BZK)}|pwk}Wien 9/3, 31. 8. 96, 7.N, Bestellt\nobreak{}«. \newline{}Ordnung: mit Bleistift von unbekannter Hand nummeriert: »81« }\pstart{}{\pb}D\textsuperscript{r}
                  Arthur Schnitzler\pend{}\pstart{}\textcolor{pink}{Wien}{}\ledrightnote{\textcolor{pink}{Wien}}\pend{}\pstart{}\textcolor{pink}{IX. Frankgasse 1}{}\ledrightnote{\textcolor{pink}{Frankgasse}}\pend{}{\bigskip}\pstart
           \noindent{}\centering{}\textcolor{gray}{\textbf{{\pb}Gruss aus dem \textcolor{pink}{Zoolog. Garten}{}\ledrightnote{\textcolor{pink}{Zoo}} in \textcolor{pink}{Leipzig}{}\ledrightnote{\textcolor{pink}{Leipzig}}, d.}}{ }30/VIII 96\pend
           \pstart
           \noindent{}\centering{}\textcolor{gray}{\textbf{Besitzer \textcolor{blue}{E.
                        Pinkert}{}\ledrightnote{\textcolor{blue}{Ernst Pinkert}}}}\pend
           \pstart
           \noindent{}\centering{}\textcolor{gray}{\textbf{Bären-Zwinger.}}\hspace*{1.5em}\textcolor{gray}{\textbf{Raubthierhaus.}}\hspace*{1.5em}\textcolor{gray}{\textbf{Antilopen-Haus.}}\hspace*{2em}\textcolor{gray}{\textbf{Teich m. Büffel u. Kameel-Haus.}}\pend
           \pstart
           Lieber! Da man den »\textcolor{green}{Doppeladler}{}\ledrightnote{\textcolor{green}{Unter dem Doppel-Adler}}«
               spielt \uline{muß} ich doch Ihnen schreiben. – Ich bin
                  Donnerstag in \textcolor{pink}{Baden}{}\ledrightnote{\textcolor{pink}{Baden bei Wien}}.\pend
           \pstart
           Herzlichst{\\[\baselineskip]}Ihr \spacefill\mbox{Richard}\pend
           \leftskip=0em{}\endnumbering\briefempfaengerindex{Schnitzler, Arthur@\textsc{Schnitzler, Arthur}!zzzBeer-Hofmann, Richard@\emph{von Richard Beer-Hofmann}!1896-08-301@{30. 8. 1896}|)be}\mylabel{h}  \normalsize

\doendnotes{C}
\bigskip
\vfill

\clearpage

\footnotesize

\lohead{\textsc{register}}

% Definiere theindex-Environment komplett neu ohne reledmac
\makeatletter
\renewenvironment{theindex}{%
  \section*{\indexname}%
  \setlength{\parindent}{0pt}%
  \setlength{\parskip}{0pt plus 0.3pt}%
  \let\item\@idxitem
}{%
  \clearpage
}
\makeatother

\IfFileExists{\jobname-pw.ind}{\input{\jobname-pw.ind}}{}

\end{document}

      