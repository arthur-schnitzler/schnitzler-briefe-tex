%% latex-korrekturansicht-vorspann.tex
%% Vorspann für die Korrekturansicht.
%% Lädt die gemeinsame Datei latex-vorspann.tex mit gesetztem Schalter.

\newif\ifkorrekturansicht
\korrekturansichttrue

\input{../tex-inputs/latex-vorspann}


               \section[Arthur Schnitzler an Albert Ehrenstein, 9. 2. 1911]{ Arthur Schnitzler an Albert Ehrenstein, 9. 2. 1911}\nopagebreak\mylabel{v}\rehead{ }\normalsize\beginnumbering\briefempfaengerindex{Ehrenstein, Albert@\textsc{Ehrenstein, Albert}!zzzSchnitzler, Arthur@\emph{von Arthur Schnitzler}!1911-02-091@{9. 2. 1911}|(be} \toendnotes[C]{\smallbreak\pagebreak[2]} \Standort{Jerusalem, The National Library of Israel, ARC. Ms. Var. 306 1 118.}
\physDesc{Brief, 1 Blatt, 2 Seiten
\newline{}Schreibmaschine
\newline{}Handschrift: schwarze Tinte, lateinische Kurrent (\noindent{}Korrekturen, Unterschrift)}\buchAbdrucke{\weitereDrucke{Arthur Schnitzler: \emph{Briefe 1875–1912}. Hg. Therese Nickl und Heinrich Schnitzler. Frankfurt am Main: \emph{S. Fischer} 1981, S. 656–657.} }\pstart
           {\pb}\textcolor{gray}{\textbf{Dr. Arthur Schnitzler}}\hfill 9. 2. 1911.\pend
           \pstart
           \textcolor{gray}{\textbf{\textcolor{pink}{Wien XVIII. Sternwartestrasse 71}{}\ledrightnote{\textcolor{pink}{Sternwartestraße}}}}\pend
           \pstart{}Sehr geehrter Herr Doktor.\pend\pstart
           Gestern erhielt ich einen Brief von \textcolor{blue}{Stefan
                        Grossmann}{}\ledrightnote{\textcolor{blue}{Stefan Großmann}}, der unter anderem folgende Stelle enthält: »Ein junger
                    Literat \introOben{}(\introOben{}von Talent\introOben{})\introOben{}{ }\introOben{}\textsc{Herr Ehrenstein}\introOben{} erzählt verschiedenen Leuten unter anderm auch dem \textcolor{brown}{Fackel}{}\ledrightnote{\textcolor{brown}{Die Fackel}}-\textcolor{blue}{Kraus}{}\ledrightnote{\textcolor{blue}{Karl Kraus}}, Sie hätten
                    ihm ›bestätigt‹, dass ich meine Macht als Kritiker zu erotischen Erpressungen an
                    Schauspielerinnen ausgenützt hätte.« Zugleich bittet er mich um eine Silbe
                    darüber, dass ich eine solche Bestätigung nicht gab, \introOben{}»\introOben{}wie ich sie ja auch nicht geben konnte.\introOben{}«\introOben{}\pend
           \pstart
           Ich habe Herrn \textcolor{blue}{Grossmann}{}\ledrightnote{\textcolor{blue}{Stefan Großmann}} wie natürlich den
                    Tatsachen entsprechend geantwortet, dass ich Ihnen ein solches Gerücht nicht
                    bestätigt habe und nicht bestätigen konnte, da ich es von keine Seite, auch von
                    Ihnen selbst nicht –, jemals vernommen hatte. Hiemit wäre die Sache nach der
                    einen Seite abgetan. Was aber aus der Geschichte leider hervorgeht ist, dass Sie
                    sich befugt finden Privatge{\pb}spräche zwischen mir und Ihnen
                    weiter zu tragen – in Kreise, die mir äusserlich und innerlich ferne sind und
                    bleiben sollen. Dem gegenüber kommt ja meine\introOben{}r\introOben{}
                    Auffassung \introOben{}nach\introOben{} kaum \substVorne{}\textsuperscript{mehr}\substDazwischen{}sonderlich\substHinten{} in Betracht, dass Sie wie dieser Fall beweist, bei solcher Gelegenheit
                    Ihre Phantasie in entstellender ja wie es scheint in erfindender Richtung walten
                    lassen. Denn wenn ich hier auch die Möglichkeit von Missverständnissen im
                    weitesten Ausmass zugestehen wollte, es ist jedenfalls total ausgeschlossen,
                    dass sich \textcolor{blue}{Grossmann}{}\ledrightnote{\textcolor{blue}{Stefan Großmann}} und \textcolor{blue}{Kraus}{}\ledrightnote{\textcolor{blue}{Karl Kraus}} diese Fabel einfach aus den Fingern gesogen hätten.
                    Dass ich bei meinem Ihnen bekannten Ekel vor Literatengezänk – und Geklatsch
                    mich unter diesen Umständen genötigt sehe auf die Fortsetzung eines persönlichen
                    Verkehrs mit Ihnen zu verzichten, werden Sie ohneweiters einsehen, mit welcher
                    Erklärung die leidige Angelegenheit für mich, der ich Wichtigeres zu tun habe,
                    ein für alle Mal erledigt ist.\pend
           \pstart
           Hochachtungsvoll{\\[\baselineskip]}\spacefill\mbox{{[}hs.:{]} Dr Arthur Schnitzler}\pend
           \leftskip=0em{}\pstart
           \noindent{}{[}ms.:{]} Herrn Dr. Albert Ehrenstein, \textcolor{pink}{Wien}{}\ledrightnote{\textcolor{pink}{Wien}}.\pend
           \endnumbering\briefempfaengerindex{Ehrenstein, Albert@\textsc{Ehrenstein, Albert}!zzzSchnitzler, Arthur@\emph{von Arthur Schnitzler}!1911-02-091@{9. 2. 1911}|)be}\mylabel{h}  \normalsize

\doendnotes{C}
\bigskip
\vfill

\clearpage

\footnotesize

\lohead{\textsc{register}}

% Definiere theindex-Environment komplett neu ohne reledmac
\makeatletter
\renewenvironment{theindex}{%
  \section*{\indexname}%
  \setlength{\parindent}{0pt}%
  \setlength{\parskip}{0pt plus 0.3pt}%
  \let\item\@idxitem
}{%
  \clearpage
}
\makeatother

\IfFileExists{\jobname-pw.ind}{\input{\jobname-pw.ind}}{}

\end{document}

      