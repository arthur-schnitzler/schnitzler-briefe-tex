%% latex-korrekturansicht-vorspann.tex
%% Vorspann für die Korrekturansicht.
%% Lädt die gemeinsame Datei latex-vorspann.tex mit gesetztem Schalter.

\newif\ifkorrekturansicht
\korrekturansichttrue

\input{../tex-inputs/latex-vorspann}


               \section[Stefan Großmann an Arthur Schnitzler, 10. 2. 1921]{ Stefan Großmann an Arthur Schnitzler, 10. 2. 1921}\nopagebreak\mylabel{v}\rehead{ }\normalsize\beginnumbering\briefempfaengerindex{Schnitzler, Arthur@\textsc{Schnitzler, Arthur}!zzzGrossmann, Stefan@\emph{von Stefan Großmann}!1921-02-101@{10. 2. 1921}|(be} \toendnotes[C]{\smallbreak\pagebreak[2]} \Standort{CUL, Schnitzler, B 34.}
\physDesc{Brief, 1 Blatt, 1 Seite
\newline{}Schreibmaschine
\newline{}Handschrift: blaue Tinte, deutsche Kurrent (\noindent{}Unterschrift)
\newline{}Schnitzler: 1) mit rotem Buntstift zwei Unterstreichungen 2) mit Bleistift auf der Rückseite das Antwortschreiben in
                                 Lateinschrift skizziert: »\noindent{}{\pb}Viel\textcolor{gray}{en} Dank für Ihre freund
                                       Zeilen.{ / }Sich\textcolor{gray}{er} keine Absicht –{ / }Gra mit Herr \textcolor{blue}{Hard\textcolor{gray}{en}} {\dotstwo}{ / }Üb hiesige\textcolor{gray}{s}{\dotstwo} haben Sie wohl geles{ / }Ich käme mir nur komisch vor sollt ich und Herr \textcolor{blue}{Kunsch} od nur der
                                       Schusterlehrling, polemis, der das Theater stürmt {\dots} in d\textcolor{gray}{em} Rufe
                                       »Man schändet uns Frauen« (u das Stück \strikeout{\textcolor{gray}{imm}} das \textcolor{gray}{er} kannte.{ / }Wobei m\textcolor{gray}{eine} Sympathie noch im mehr bei d
                                       Schusterlehrlg als bei den »\textcolor{blue}{Seipel} u \textcolor{blue}{Kun}{ / }– Aehnlich\textcolor{gray}{es} ist \textcolor{gray}{im
                                          wieder} einem{ / }passirt, \textcolor{green}{Gustl} – \textcolor{green}{Bernha}.{ / }Die \strikeout{Stücke dank}{ }\textcolor{gray}{von} meine Stüc u d\textcolor{gray}{ie}
                                       Blamage mein\textcolor{gray}{er} Gegne\textcolor{gray}{r}{ / }\textcolor{gray}{Unerhörtes}!{ / }Herzl«\newline{}Ordnung: mit Bleistift von unbekannter Hand nummeriert:
                                    »15« }\toendnotes[C]{\smallbreak}\pstart
           \noindent{}\centering{}{\pb}\textcolor{gray}{\textbf{\textcolor{brown}{Das Tage-Buch}{}\ledrightnote{\textcolor{brown}{Das Tage-Buch}}}}\pend
           \pstart
           \noindent{}\centering{}\textcolor{gray}{\textbf{Erscheint jeden Sonnabend ⋅ Herausgeber: Stefan Großmann}}\pend
           \pstart
           \noindent{}\centering{}\textcolor{gray}{\textbf{\textcolor{brown}{Ernst Rowohlt Verlag}{}\ledrightnote{\textcolor{brown}{Ernst Rowohlt Verlag}} ⋅ \textcolor{pink}{Berlin W 35}{}\ledrightnote{\textcolor{pink}{Berlin}}}}\pend
           \pstart
           \noindent{}\centering{}\textcolor{gray}{\textbf{\textcolor{pink}{POTSDAMER STRASSE 123\textsuperscript{B}
                        ⋅ AN DER POTSDAMER BRÜCKE}{}\ledrightnote{\textcolor{pink}{Potsdamerstraße}}}}\pend
           \pstart
           \noindent{}\centering{}\textcolor{gray}{\textbf{TELEGRAMM-ADRESSE: \textcolor{brown}{TAGEBUCH
                        BERLIN}{}\ledrightnote{\textcolor{brown}{Das Tage-Buch}} ⋅ FERNSPRECHER: \textcolor{brown}{AMT LÜTZOW}{}\ledrightnote{\textcolor{brown}{Fernsprechamt Lietzow}}
                     Nr. 4931}}\pend
           \pstart
           \noindent{}\centering{}\textcolor{gray}{\textbf{SPRECHSTUNDE DER REDAKTION: 12–1 UHR}}\pend
           \pstart
           \noindent{}Gr/Sch\pend
           \pstart
           \centering{}10. Februar 1921\pend
           \pstart
           \textcolor{gray}{\textbf{\emph{REDAKTION}}}\pend
           \pstart
           Herrn\pend
           \leftskip=3em{}\pstart
           \noindent{}Dr.med. Arthur \so{Schnitzler}\pend
           \leftskip=0em{}\leftskip=3em{}\pstart
           \textcolor{pink}{\so{Wien}}{}\ledrightnote{\textcolor{pink}{Wien}}\pend
           \leftskip=0em{}\leftskip=3em{}\pstart
           \textcolor{pink}{Sternwartstr. 71}{}\ledrightnote{\textcolor{pink}{Sternwartestraße}}\pend
           \leftskip=0em{}\pstart{}Verehrter lieber Herr Dr. Schnitzler!\pend\pstart
           Ich übersende Ihnen heute einige Nummern des »\textcolor{brown}{Tage-Buch}{}\ledrightnote{\textcolor{brown}{Das Tage-Buch}}«, in denen ich die etwas heuchlerische Hetze gegen den »\textcolor{green}{Reigen}{}\ledrightnote{\textcolor{green}{Reigen. Zehn Dialoge}}« \textcolor{green}{satyrisch behandelt}{}\ledrightnote{→\textcolor{green}{Tilla zürnt der Zeit}{\newline}→\textcolor{green}{Hänischs Reigen. Eine unsittliche Szenenfolge}} habe. Es ist mir bekannt, dass Sie
               niemals zu Ihrem Schaffen selbst das Wort nehmen wollten. Wenn Sie aber bedenken, in
               wie unangenehmer Form \textcolor{blue}{Harden}{}\ledrightnote{\textcolor{blue}{Maximilian Harden}} jetzt gegen die »\textcolor{green}{Reigen}{}\ledrightnote{\textcolor{green}{Reigen. Zehn Dialoge}}«-Aufführung \textcolor{green}{geschrieben}{}\ledrightnote{\textcolor{green}{Reigen}} hat, wäre es vielleicht doch von Wert und Nutzen, wenn Sie sich
               entschliessen könnten, im »\textcolor{brown}{Tage-Buch}{}\ledrightnote{\textcolor{brown}{Das Tage-Buch}}« selbst das
               Wort zu ergreifen und sich zur öffentlichen Aufführung des »\label{T_L02362_1v}\edtext{\textcolor{green}{Reigen}{}\ledrightnote{\textcolor{green}{Reigen. Zehn Dialoge}}}{\lemma{\textnormal{\emph{Reigen}}}\Cendnote{\textnormal{geschrieben Reiegn}}}\label{T_L02362_1h}« zu äussern.
               Jedenfalls bitte ich Sie, über meine Zeitschrift zu verfügen. Das »\textcolor{brown}{Tage-Buch}{}\ledrightnote{\textcolor{brown}{Das Tage-Buch}}« hat sich in den fünfviertel Jahren seines Bestehens in
                  \textcolor{pink}{Deutschland}{}\ledrightnote{\textcolor{pink}{Deutschland}} vollkommen durchgesetzt und Sie
               sprechen durch mein »\textcolor{brown}{Tage-Buch}{}\ledrightnote{\textcolor{brown}{Das Tage-Buch}}« zu dem gebildeten
                  \textcolor{pink}{Deutschland}{}\ledrightnote{\textcolor{pink}{Deutschland}}, das ehedem die »\textcolor{green}{Zukunft}{}\ledrightnote{\textcolor{green}{Die Zukunft}}« gelesen hat. Ich würde mich freuen und das Gefühl haben,
               einer gerechten Sache zu dienen, wenn Sie sich entschliessen wollten, durch das »\textcolor{brown}{Tage-Buch}{}\ledrightnote{\textcolor{brown}{Das Tage-Buch}}« zu sprechen.\pend
           \pstart
           Mit herzlichen Grüssen{\\[\baselineskip]} Ihr sehr ergebener{\\[\baselineskip]}\spacefill\mbox{{[}hs.:{]} Stefan Großmann}\pend
           \leftskip=0em{}\endnumbering\briefempfaengerindex{Schnitzler, Arthur@\textsc{Schnitzler, Arthur}!zzzGrossmann, Stefan@\emph{von Stefan Großmann}!1921-02-101@{10. 2. 1921}|)be}\mylabel{h}  \normalsize

\doendnotes{C}
\bigskip
\vfill

\clearpage

\footnotesize

\lohead{\textsc{register}}

% Definiere theindex-Environment komplett neu ohne reledmac
\makeatletter
\renewenvironment{theindex}{%
  \section*{\indexname}%
  \setlength{\parindent}{0pt}%
  \setlength{\parskip}{0pt plus 0.3pt}%
  \let\item\@idxitem
}{%
  \clearpage
}
\makeatother

\IfFileExists{\jobname-pw.ind}{\input{\jobname-pw.ind}}{}

\end{document}

      