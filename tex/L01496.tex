%% latex-korrekturansicht-vorspann.tex
%% Vorspann für die Korrekturansicht.
%% Lädt die gemeinsame Datei latex-vorspann.tex mit gesetztem Schalter.

\newif\ifkorrekturansicht
\korrekturansichttrue

\input{../tex-inputs/latex-vorspann}


               \section[Gerhart Hauptmann an Arthur Schnitzler, 29. {[}1. 1905?{]}]{ Gerhart Hauptmann an Arthur Schnitzler, 29. {[}1. 1905?{]}}\nopagebreak\mylabel{v}\rehead{ }\normalsize\beginnumbering\briefempfaengerindex{Schnitzler, Arthur@\textsc{Schnitzler, Arthur}!zzzHauptmann, Gerhart@\emph{von Gerhart Hauptmann}!1905-01-291@{{[}29. 1. 1905?{]}}|(be} \toendnotes[C]{\smallbreak\pagebreak[2]} \Standort{CUL, Schnitzler, B 36.}
\physDesc{Telegramm
\newline{}maschinell\newline{}Versand: Stempel des Telegrafenbeamten, der Telegrafenbeamtin: »\textcolor{blue}{Fischer}« \newline{}Ordnung: beschnitten }\toendnotes[C]{\smallbreak}\pstart
           {\pb}fr \textcolor{pink}{agnetendorf}{}\ledrightnote{\textcolor{pink}{Agnetendorf}} 128 28 29{ }12.25 n\pend
           \pstart
           lieber herr schnitzler ich werde gern den gewuenschten \label{K_L01496_1v}\edtext{\textcolor{green}{prolog}{}\ledrightnote{→\textcolor{green}{Prolog einer musikalischen Feier zum Gedächtnisse Schillers}}}{\lemma{\textnormal{\emph{prolog}}}\Cendnote{\textnormal{Das undatierte Telegramm dürfte am 29.
                  eines Monats versandt sein. Es dürfte in Zusammenhang mit dem von \textcolor{blue}{Hauptmann} verfassten \textcolor{green}{Prolog} stehen, der am 22. 3. 1905 bei der
                     \textcolor{blue}{Schiller}feier des \emph{\textcolor{brown}{Wiener Konzertvereins}} vorgetragen wurde. Nachdem der
                     29. 2. 1905 zu kurzfristig für eine solche Zusage erscheint,
                  könnte es am 29. 1. 1905 geschickt worden sein. Das wiederum würde es
                  nahelegen, dass \textcolor{blue}{Hofmannsthal} mit der
                  Kommission betraut war, die Anfrage zu stellen.}}}\label{K_L01496_1h} so gut es geht verfaszen.
               herzliche gruesze von haus zu haus ihr \spacefill\mbox{gerhart hauptmann +}\pend
           \endnumbering\briefempfaengerindex{Schnitzler, Arthur@\textsc{Schnitzler, Arthur}!zzzHauptmann, Gerhart@\emph{von Gerhart Hauptmann}!1905-01-291@{{[}29. 1. 1905?{]}}|)be}\mylabel{h}  \normalsize

\doendnotes{C}
\bigskip
\vfill

\clearpage

\footnotesize

\lohead{\textsc{register}}

% Definiere theindex-Environment komplett neu ohne reledmac
\makeatletter
\renewenvironment{theindex}{%
  \section*{\indexname}%
  \setlength{\parindent}{0pt}%
  \setlength{\parskip}{0pt plus 0.3pt}%
  \let\item\@idxitem
}{%
  \clearpage
}
\makeatother

\IfFileExists{\jobname-pw.ind}{\input{\jobname-pw.ind}}{}

\end{document}

      