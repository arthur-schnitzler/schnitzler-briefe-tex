%% latex-korrekturansicht-vorspann.tex
%% Vorspann für die Korrekturansicht.
%% Lädt die gemeinsame Datei latex-vorspann.tex mit gesetztem Schalter.

\newif\ifkorrekturansicht
\korrekturansichttrue

\input{../tex-inputs/latex-vorspann}


               \section[Arthur Schnitzler an Robert Adam, 5. 10. 1918]{ Arthur Schnitzler an Robert Adam, 5. 10. 1918}\nopagebreak\mylabel{v}\rehead{ }\normalsize\beginnumbering\briefempfaengerindex{Adam, Robert@\textsc{Adam, Robert}!zzzSchnitzler, Arthur@\emph{von Arthur Schnitzler}!1918-10-051@{5. 10. 1918}|(be} \toendnotes[C]{\smallbreak\pagebreak[2]} \Standort{DLA, 96.34.2/13.}
\physDesc{Postkarte
\newline{}Handschrift: Bleistift, lateinische Kurrent\newline{}Versand: Stempel: »\nobreak{}Wien, 5. X. 18, 8\nobreak{}«.  }\toendnotes[C]{\smallbreak}\pstart{}{\pb}Dr. Arthur Schnitzler\pend{}\pstart{}\textcolor{pink}{Wien XVIII. Sternwartestr 71}{}\ledrightnote{\textcolor{pink}{Sternwartestraße}}\pend{}{\bigskip}\pstart{}Hrn Dr. Robert Adam Pollak\pend{}\pstart{}\textcolor{pink}{Wien XII}{}\ledrightnote{\textcolor{pink}{XII., Meidling}}.\pend{}\pstart{}\textcolor{pink}{Meidlinger Hauptstr 58}{}\ledrightnote{\textcolor{pink}{Meidlinger Hauptstraße}}\pend{}{\bigskip}\pstart
           \raggedleft{}{\pb}5. X. 18\pend
           \pstart{}Verehrter Herr Doktor,\pend\pstart
           haben Sie Montag{ }Abend gegen 7 Zeit so wird es mich sehr freuen Sie bei mir zu
                    sehen. Bitte noch keins der \textcolor{green}{Bücher}{}\ledrightnote{→\textcolor{green}{Geistesstörung und Verbrechen im Kindesalter}{\newline}→\textcolor{green}{Minderjährige Verbrecher. (Versuch einer strafgerichtlichen Psychologie) mit Original-Gutachten von Berenini – Brusa – Colajanni – Negri – Nordau – Pierantoni}} mitzubringen; – ich komm jetzt nicht dazu es
                    zu lesen.\pend
           \pstart
           Herzlich grüßend{\\[\baselineskip]}Ihr ergeb\textcolor{gray}{ner}\spacefill\mbox{ArtSch}\pend
           \leftskip=0em{}\endnumbering\briefempfaengerindex{Adam, Robert@\textsc{Adam, Robert}!zzzSchnitzler, Arthur@\emph{von Arthur Schnitzler}!1918-10-051@{5. 10. 1918}|)be}\mylabel{h}  \normalsize

\doendnotes{C}
\bigskip
\vfill

\clearpage

\footnotesize

\lohead{\textsc{register}}

% Definiere theindex-Environment komplett neu ohne reledmac
\makeatletter
\renewenvironment{theindex}{%
  \section*{\indexname}%
  \setlength{\parindent}{0pt}%
  \setlength{\parskip}{0pt plus 0.3pt}%
  \let\item\@idxitem
}{%
  \clearpage
}
\makeatother

\IfFileExists{\jobname-pw.ind}{\input{\jobname-pw.ind}}{}

\end{document}

      