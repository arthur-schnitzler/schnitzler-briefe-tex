%% latex-korrekturansicht-vorspann.tex
%% Vorspann für die Korrekturansicht.
%% Lädt die gemeinsame Datei latex-vorspann.tex mit gesetztem Schalter.

\newif\ifkorrekturansicht
\korrekturansichttrue

\input{../tex-inputs/latex-vorspann}


               \section[Hermann Bahr an Arthur Schnitzler, 21. 6. 1906]{ Hermann Bahr an Arthur Schnitzler, 21. 6. 1906}\nopagebreak\mylabel{v}\rehead{ }\normalsize\beginnumbering\briefempfaengerindex{Schnitzler, Arthur@\textsc{Schnitzler, Arthur}!zzzBahr, Hermann@\emph{von Hermann Bahr}!1906-06-211@{21. 6. 1906}|(be} \toendnotes[C]{\smallbreak\pagebreak[2]} \Standort{CUL, Schnitzler, B 5b.}
\physDesc{Brief, 1 Blatt, 2 Seiten
\newline{}Handschrift: blaue Tinte, deutsche Kurrent\newline{}Ordnung: mit Bleistift von unbekannter Hand nummeriert:
                                    »139« }\buchAbdrucke{\weitereDrucke{Hermann Bahr, Arthur Schnitzler: \emph{Briefwechsel, Aufzeichnungen, Dokumente (1891–1931)}. Hg. Kurt Ifkovits und Martin Anton Müller. Göttingen: \emph{Wallstein} 2018, S. 378–379.} }\toendnotes[C]{\smallbreak}\pstart
           \raggedleft{}{\pb}\textcolor{pink}{Wien XIII/\textsubscript{7}}{}\ledrightnote{\textcolor{pink}{Ober Sankt Veit}}{\\}21. 6. 06\pend
           \pstart\center{}Lieber Artur!\pend\pstart
           Ich wollte immer noch zu Dir, war aber die letzte Zeit ſo gehetzt, daß es nie ging.
               Den »\label{K_L01601_1v}\edtext{\textcolor{green}{Faun}{}\ledrightnote{\textcolor{green}{Der Faun}}}{\lemma{\textnormal{\emph{Faun}}}\Cendnote{\textnormal{fertiggestellt am
                     5. 6. 1906 (Bahr: \emph{Tagebücher, Skizzenhefte,
                        Notizbücher} V,16)}}}\label{K_L01601_1h}« haſt Du wol bekommen. Ich möchte
               gern gelegentlich ein durchaus aufrichtiges, rückſichtsloſes Wort von Dir darüber
               hören. Und dann bitte ich Dich, es, wenn Dus geleſen haſt, an \textcolor{blue}{Salten}{}\ledrightnote{\textcolor{blue}{Felix Salten}} nach \textcolor{pink}{Berlin}{}\ledrightnote{\textcolor{pink}{Berlin}} zu
               ſchicken. Ich fahre \label{K_L01601_2v}\edtext{morgen nach \textcolor{pink}{Venedig}{}\ledrightnote{\textcolor{pink}{Venedig}}}{\lemma{\textnormal{\emph{morgen nach Venedig}}}\Cendnote{\textnormal{\textcolor{blue}{Bahr} fuhr am 23. 6. 1906 und blieb bis
                     Ende Juli.}}}\label{K_L01601_2h}. Nachrichten an meine \textcolor{pink}{Wiener {\pb}Adresse}{}\ledrightnote{\textcolor{pink}{Veitlissengasse}} kommen mir
               immer nach. Vielleicht könnten wir uns im Auguſt irgendwo treffen. Grüß Deine \textcolor{blue}{Frau}{}\ledrightnote{→\textcolor{blue}{Olga Schnitzler}} herzlichſt und nimm die
               beſten Wünſche für einen frohen Sommer von \pend
           \pstart
           Deinem alten{\\[\baselineskip]}\spacefill\mbox{Hermann}\pend
           \leftskip=0em{}\endnumbering\briefempfaengerindex{Schnitzler, Arthur@\textsc{Schnitzler, Arthur}!zzzBahr, Hermann@\emph{von Hermann Bahr}!1906-06-211@{21. 6. 1906}|)be}\mylabel{h}  \normalsize

\doendnotes{C}
\bigskip
\vfill

\clearpage

\footnotesize

\lohead{\textsc{register}}

% Definiere theindex-Environment komplett neu ohne reledmac
\makeatletter
\renewenvironment{theindex}{%
  \section*{\indexname}%
  \setlength{\parindent}{0pt}%
  \setlength{\parskip}{0pt plus 0.3pt}%
  \let\item\@idxitem
}{%
  \clearpage
}
\makeatother

\IfFileExists{\jobname-pw.ind}{\input{\jobname-pw.ind}}{}

\end{document}

      