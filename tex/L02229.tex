%% latex-korrekturansicht-vorspann.tex
%% Vorspann für die Korrekturansicht.
%% Lädt die gemeinsame Datei latex-vorspann.tex mit gesetztem Schalter.

\newif\ifkorrekturansicht
\korrekturansichttrue

\input{../tex-inputs/latex-vorspann}


               \section[Robert Adam an Arthur Schnitzler, 21. 5. 1916]{ Robert Adam an Arthur Schnitzler, 21. 5. 1916}\nopagebreak\mylabel{v}\rehead{ }\normalsize\beginnumbering\briefempfaengerindex{Schnitzler, Arthur@\textsc{Schnitzler, Arthur}!zzzAdam, Robert@\emph{von Robert Adam}!1916-05-211@{21. 5. 1916}|(be} \toendnotes[C]{\smallbreak\pagebreak[2]} \Standort{Wien, Österreichische Nationalbibliothek, Cod. ser. 52.263.}
\physDesc{maschinelle Abschrift
\newline{}Handschrift: Bleistift (\noindent{}geringfügige Korrekturen)\newline{}Ordnung: von unbekannter Hand nummeriert:
                                            »176« }\pstart
           \raggedleft{}{\pb}21. Mai 1916\pend
           \pstart{}Hochverehrter Herr Doktor!\pend\pstart
           Ich sende Ihnen, wie Sie wünschten, ein Exemplar des »\textcolor{green}{Ali ibn Bekkâr}{}\ledrightnote{\textcolor{green}{Die Geschichte des Alî ibn Bekkâr mit Schams an-Nahâr}}« und eines von »\textcolor{green}{In aeternum}{}\ledrightnote{\textcolor{green}{In aeternum. Eine Phantasie}}«.\pend
           \pstart
           Dies mein Erstlingsprodukt hiess ursprünglich »\textcolor{green}{Götterdämmerung}{}\ledrightnote{\textcolor{green}{In aeternum. Eine Phantasie}}« und wurde um das Jahr 1900 geschrieben.
                    Soviel zur Entschuldigung. Der Schluss dürfte Ihnen übrigens bekannt
                    vorkommen! –\pend
           \pstart Mit ergebensten Grüssen Ihr \spacefill\mbox{RA}\pend{}\endnumbering\briefempfaengerindex{Schnitzler, Arthur@\textsc{Schnitzler, Arthur}!zzzAdam, Robert@\emph{von Robert Adam}!1916-05-211@{21. 5. 1916}|)be}\mylabel{h}  \normalsize

\doendnotes{C}
\bigskip
\vfill

\clearpage

\footnotesize

\lohead{\textsc{register}}

% Definiere theindex-Environment komplett neu ohne reledmac
\makeatletter
\renewenvironment{theindex}{%
  \section*{\indexname}%
  \setlength{\parindent}{0pt}%
  \setlength{\parskip}{0pt plus 0.3pt}%
  \let\item\@idxitem
}{%
  \clearpage
}
\makeatother

\IfFileExists{\jobname-pw.ind}{\input{\jobname-pw.ind}}{}

\end{document}

      