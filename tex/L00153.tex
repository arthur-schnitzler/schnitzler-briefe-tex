%% latex-korrekturansicht-vorspann.tex
%% Vorspann für die Korrekturansicht.
%% Lädt die gemeinsame Datei latex-vorspann.tex mit gesetztem Schalter.

\newif\ifkorrekturansicht
\korrekturansichttrue

\input{../tex-inputs/latex-vorspann}


               \section[Friedrich M. Fels an Arthur Schnitzler, {[}1. 1. 1893?{]}]{ Friedrich M. Fels an Arthur Schnitzler, {[}1. 1. 1893?{]}}\nopagebreak\mylabel{v}\rehead{ }\normalsize\beginnumbering\briefempfaengerindex{Schnitzler, Arthur@\textsc{Schnitzler, Arthur}!zzzFels, Friedrich Michael@\emph{von Friedrich Michael Fels}!1893-01-012@{{[}1. 1. 1893?{]}}|(be} \toendnotes[C]{\smallbreak\pagebreak[2]} \Standort{DLA, A:Schnitzler, HS.NZ85.1.2956.}
\physDesc{Brief, 1 Blatt, 1 Seite
\newline{}Handschrift: schwarze Tinte, lateinische Kurrent
\newline{}Schnitzler: mit Bleistift nummeriert: »3« }\toendnotes[C]{\smallbreak}\pstart
           \noindent{}{\pb}Lieber Dr Arthur Schnitzler! Gestern bald als Sie gingen, brachte
               mir der Diener zwei Wohnungen: 1. \textcolor{pink}{Reisnerstraſse}{}\ledrightnote{\textcolor{pink}{Reisnerstraße}}
               wenig vom \label{K_L00153_1v}\edtext{Bureau}{\lemma{\textnormal{\emph{Bureau}}}\Cendnote{\textnormal{\textcolor{blue}{Fels} dürfte bei der \emph{\textcolor{brown}{Allgemeinen Kunst-Chronik}} in der \textcolor{pink}{Reisnerstrasse 3} angestellt gewesen sein.}}}\label{K_L00153_1h} c. 16 fl und
                  \label{K_L00153_2v}\edtext{\textcolor{pink}{Strohgaſse}{}\ledrightnote{\textcolor{pink}{Strohgasse}}}{\lemma{\textnormal{\emph{Strohgaſse}}}\Cendnote{\textnormal{Im Brief \textcolor{blue}{Hofmannsthal}s an \textcolor{blue}{Schnitzler} vom [9. 9. 1893] wird diese Wohnung
                  erwähnt. Damit kann dieses Korrespondenzstück zeitlich zumindest nach hinten
                  eingegrenzt werden.}}}\label{K_L00153_2h}{ }\uline{12 fl} – letztere angesehen, geno{\geminationm}en. Das Kabinet gut ausgestattet, die Verhältniſse
               scheinen ganz ordentlich zu sein; nur eines: auſserordentlich pünktlich im
               Bezahlen!\pend
           \pstart
           Lieber Doktor! Sie thäten mir wirklich einen Gefallen, \uline{nein}, Sie \uline{müſsen} mich heute noch aufsuchen,
               im Bureau, da{\geminationn} Wohnung. Ich habe Ihnen manches zu sagen,
               was gegen meine Beſserung spricht. Also Sie \uline{müſsen}
               heute ko{\geminationm}en.\pend
           \pstart
           Herzl.{\\[\baselineskip]}\spacefill\mbox{Fels}\pend
           \leftskip=0em{}\endnumbering\briefempfaengerindex{Schnitzler, Arthur@\textsc{Schnitzler, Arthur}!zzzFels, Friedrich Michael@\emph{von Friedrich Michael Fels}!1893-01-012@{{[}1. 1. 1893?{]}}|)be}\mylabel{h}  \normalsize

\doendnotes{C}
\bigskip
\vfill

\clearpage

\footnotesize

\lohead{\textsc{register}}

% Definiere theindex-Environment komplett neu ohne reledmac
\makeatletter
\renewenvironment{theindex}{%
  \section*{\indexname}%
  \setlength{\parindent}{0pt}%
  \setlength{\parskip}{0pt plus 0.3pt}%
  \let\item\@idxitem
}{%
  \clearpage
}
\makeatother

\IfFileExists{\jobname-pw.ind}{\input{\jobname-pw.ind}}{}

\end{document}

      