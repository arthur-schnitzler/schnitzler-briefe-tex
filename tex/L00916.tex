%% latex-korrekturansicht-vorspann.tex
%% Vorspann für die Korrekturansicht.
%% Lädt die gemeinsame Datei latex-vorspann.tex mit gesetztem Schalter.

\newif\ifkorrekturansicht
\korrekturansichttrue

\input{../tex-inputs/latex-vorspann}


               \section[Georg Brandes an Arthur Schnitzler, 11. 5. 1899]{ Georg Brandes an Arthur Schnitzler, 11. 5. 1899}\nopagebreak\mylabel{v}\rehead{ }\normalsize\beginnumbering\briefempfaengerindex{Schnitzler, Arthur@\textsc{Schnitzler, Arthur}!zzzBrandes, Georg@\emph{von Georg Brandes}!1899-05-111@{11. 5. 1899}|(be} \toendnotes[C]{\smallbreak\pagebreak[2]} \Standort{CUL, Schnitzler, B 17.}
\physDesc{Brief, 1 Blatt, 4 Seiten
\newline{}Handschrift: blaue Tinte, lateinische Kurrent\newline{}Ordnung: mit Bleistift von unbekannter Hand nummeriert:
                                    »15« }\buchAbdrucke{\weitereDrucke{Georg Brandes, Arthur Schnitzler: \emph{Ein Briefwechsel}. Hg. Kurt Bergel. Bern: \emph{Francke} 1956, S. 75–76.} }\toendnotes[C]{\smallbreak}\pstart
           \raggedleft{}{\pb}\textcolor{pink}{Kopenhagen}{}\ledrightnote{\textcolor{pink}{Kopenhagen}}{ }11 Mai 99\pend
           \pstart
           Liebster\hspace*{3.5em}Sie haben mich sehr geehrt, indem Sie mir
                    Ihren Schmerz gesagt haben. Sie wünschen, dass ich darüber nichts sage, ich
                    antworte \strikeout{denn} nur: ich habe selbst viel
                    erfahren, Verluste gelitten, bisweilen recht Hartes ausgestanden; Sie sind jung,
                    ich \introOben{}bin\introOben{} alt, \strikeout{de}ich wage
                    deshalb sonst keinen Vergleich, ich glaube aber, wir haben \strikeout{e}Eins gemeinsam, den inneren Born, den
                    unversiegbaren Lebenstrieb, dem das Leben immer wieder werth wird.\pend
           \pstart
           Ich kann dies sagen, denn meine Lage scheint meine Worte zu verspotten. Seit
                    5 Monaten liege ich zu Bett. Ich heile nicht. Eine Entzündung der Venen folgt
                    bei mir immer der anderen, bisweilen bricht die Entzündung auf ein Mal an drei
                    Stellen aus. 5 Monate im Gefängnis machen eine lange öde Zeit. Ich erhalte mir
                    das Leben {\pb}durch Lesen und
                    Schreiben, erhalte auch bisweilen Besuche. Man hat hier eine \textcolor{green}{Volksausgabe}{}\ledrightnote{→\textcolor{green}{Samlede Skrifter [Gesammelte Werke]}} meiner Schriften angefangen
                        (\textcolor{blue}{Peter Nansen}{}\ledrightnote{\textcolor{blue}{Peter Nansen}} Ihr guter Bekannter ist der
                    Urheber) und sie scheint Erfolg zu haben. Man hat circa 5000 Subscribenten und
                    druckt 6000 Exemplare. Es erscheinen alle 14 Tage 10 Bogen, und es wird etwas
                    über 3 Jahre dauern. Dennoch gehören einige meiner grösseren Schriften nicht
                    diesem Verlag. So viel Papier habe ich armer geschwärzt.\pend
           \pstart
           Madame \textcolor{blue}{Marni}{}\ledrightnote{\textcolor{blue}{Jeanne Marni}}, die ich übrigens nie gesehen
                    habe, schrieb mir, dass \textcolor{blue}{Goldmann}{}\ledrightnote{\textcolor{blue}{Paul Goldmann}} bei ihr
                    gewesen war und sich mit Freundschaft meiner erinnert hatte, was mich erfreute.
                        \textcolor{blue}{Richard Beer Hofmann}{}\ledrightnote{\textcolor{blue}{Richard Beer-Hofmann}} gibt mir nie {\pb}ein Lebenszeichen.\pend
           \pstart
           Wie gut dass Sie nicht von jenem Schriftsteller heimgesucht wurden! Lasen Sie den
                    kl. \textcolor{green}{Aufsatz}{}\ledrightnote{→\textcolor{green}{Das Dänentum in Südjütland}} pro patria den
                    ich in der \textcolor{brown}{\uline{Zukunft}}{}\ledrightnote{\textcolor{brown}{Die Zukunft}} vom 7 April hatte?
                        \textcolor{brown}{\uline{Neue fr. Presse}}{}\ledrightnote{\textcolor{brown}{Neue Freie Presse}} und \textcolor{brown}{\uline{Die Zeit}}{}\ledrightnote{\textcolor{brown}{Die Zeit. Wiener Wochenschrift}} verweigerten, ihn zu drucken.
                    Die \textcolor{pink}{Oesterreicher}{}\ledrightnote{\textcolor{pink}{Österreich}}
               sind \textcolor{pink}{preussischer}{}\ledrightnote{\textcolor{pink}{Preußen}} als die \textcolor{pink}{Preussen}{}\ledrightnote{\textcolor{pink}{Preußen}}. Das arme \textcolor{pink}{Skandinavien}{}\ledrightnote{\textcolor{pink}{Skandinavien}}, man peinigt im Süden die \textcolor{pink}{Schleswiger}{}\ledrightnote{\textcolor{pink}{Südschleswig}}, im Norden die \textcolor{pink}{Finnländer}{}\ledrightnote{\textcolor{pink}{Finnland}}.\pend
           \pstart
           Ich erhalte Gottlob täglich von den meisten Gegenden \textcolor{pink}{Europas}{}\ledrightnote{\textcolor{pink}{Europa}} Briefe und Bücher, sonst wäre ich in meinem Elend zu Grunde
                    gegangen. Ich lese stetig \textcolor{brown}{\uline{L’Aurore}}{}\ledrightnote{\textcolor{brown}{L’Aurore}} und \textcolor{brown}{\uline{Le Siècle}}{}\ledrightnote{\textcolor{brown}{Le Siècle}}, folge so von Tag zu Tag dem
                    Verlauf der Begebenheiten in \textcolor{pink}{Frankreich}{}\ledrightnote{\textcolor{pink}{Frankreich}}.
                    Welches Stück Seelenlehre! Ich habe in meinem {\pb}Leben wenig so Lehrreiches
                    gelesen.\pend
           \pstart
           Ihr \textcolor{green}{Buch}{}\ledrightnote{→\textcolor{green}{Der grüne Kakadu – Paracelsus – Die Gefährtin. Drei Einakter}} habe ich noch nicht
                    erhalten; ich werde es mit derselben ernsten Aufmerksamkeit lesen, womit ich
                    Ihnen immer folge. Ich las kürzlich das \textcolor{green}{\uline{Vermächtnis}}{}\ledrightnote{\textcolor{green}{Das Vermächtnis. Schauspiel in drei Akten}} wieder; es verdient, dass
                    man dazu zurückkehrt. Ein kleiner dummer \textcolor{pink}{schwedischer}{}\ledrightnote{\textcolor{pink}{Schweden}}{ }\textcolor{blue}{Journalist}{}\ledrightnote{→\textcolor{blue}{Adolf Paul}} hatte Sie vor
                    einigen Tagen in einem \textcolor{brown}{\textcolor{pink}{Stockholm}{}\ledrightnote{\textcolor{pink}{Stockholm}}erblatt}{}\ledrightnote{→\textcolor{brown}{Svensk Dagbladet}}, das mir zugeschickt
                    wird, \label{K_L00916_1v}\edtext{\textcolor{green}{angegriffen}{}\ledrightnote{→\textcolor{green}{Från Berlins teatrar}}}{\lemma{\textnormal{\emph{angegriffen}}}\Cendnote{\textnormal{\textcolor{blue}{Adolf Paul}: \emph{\textcolor{green}{Från Berlins teatrar}}. In: \emph{\textcolor{green}{Svensk Dagbladet}}, Nr. 142, 8. 5. 1899,
                            S. 2.}}}\label{K_L00916_1h}; es brannte mir die Finger, dagegen zu schreiben,
                    habe es nicht gethan, weil ich ein wenig müde bin und soviele Correcturen
                    täglich zu besorgen habe, thue es vielleicht noch. Doch ich kann Ihnen
                    vielleicht einmal auf bessere Weise nützlich sein.\pend
           \pstart
           Ich drücke Ihnen die Hand.\pend
           \pstart Ihr \spacefill\mbox{Georg Brandes}\pend{}\endnumbering\briefempfaengerindex{Schnitzler, Arthur@\textsc{Schnitzler, Arthur}!zzzBrandes, Georg@\emph{von Georg Brandes}!1899-05-111@{11. 5. 1899}|)be}\mylabel{h}  \normalsize

\doendnotes{C}
\bigskip
\vfill

\clearpage

\footnotesize

\lohead{\textsc{register}}

% Definiere theindex-Environment komplett neu ohne reledmac
\makeatletter
\renewenvironment{theindex}{%
  \section*{\indexname}%
  \setlength{\parindent}{0pt}%
  \setlength{\parskip}{0pt plus 0.3pt}%
  \let\item\@idxitem
}{%
  \clearpage
}
\makeatother

\IfFileExists{\jobname-pw.ind}{\input{\jobname-pw.ind}}{}

\end{document}

      