%% latex-korrekturansicht-vorspann.tex
%% Vorspann für die Korrekturansicht.
%% Lädt die gemeinsame Datei latex-vorspann.tex mit gesetztem Schalter.

\newif\ifkorrekturansicht
\korrekturansichttrue

\input{../tex-inputs/latex-vorspann}


               \section[Arthur Schnitzler an Georg Brandes, 31. 12. 1897]{ Arthur Schnitzler an Georg Brandes, 31. 12. 1897}\nopagebreak\mylabel{v}\rehead{ }\normalsize\beginnumbering\briefempfaengerindex{Brandes, Georg@\textsc{Brandes, Georg}!zzzSchnitzler, Arthur@\emph{von Arthur Schnitzler}!1897-12-311@{31. 12. 1897}|(be} \toendnotes[C]{\smallbreak\pagebreak[2]} \Standort{Kopenhagen, Det Kongelige Bibliotek, Georg Brandes Arkiv, box 125.}
\physDesc{Brief, 1 Blatt, 3 Seiten
\newline{}Handschrift: schwarze Tinte, deutsche Kurrent\newline{}Ordnung: mit Bleistift von unbekannter Hand
                                    nummeriert: »10. Schnitzler« }\buchAbdrucke{\weitereDrucke{Georg Brandes, Arthur Schnitzler: \emph{Ein Briefwechsel}. Hg. Kurt Bergel. Bern: \emph{Francke} 1956, S. 66.} }\pstart
           \raggedleft{}{\pb}\textcolor{pink}{\textsc{Wien}}{}\ledrightnote{\textcolor{pink}{Wien}}, 31. 12. 97{\\}\textcolor{pink}{IX. Frankgaſſe 1}{}\ledrightnote{\textcolor{pink}{Frankgasse}}\pend
           \pstart{}Verehrteſter Herr Brandes,\pend\pstart
           was für eine erfreuliche Nachricht als erſte nach ſo langer Zeit! Sowohl
                        \textcolor{blue}{\textsc{Beer-Hofma{\geminationn}}}{}\ledrightnote{\textcolor{blue}{Richard Beer-Hofmann}} als ich ſind in \textcolor{pink}{Wien}{}\ledrightnote{\textcolor{pink}{Wien}} und freuen uns ſehr, Sie ſobald wiederzuſehen. Als Hotel wird mir
                    in {\pb}der letzten Zeit das »\textcolor{pink}{Reſidenz-Hotel}{}\ledrightnote{\textcolor{pink}{Residenzhotel}}« in der \textcolor{pink}{\textsc{Teinfaltstraße}}{}\ledrightnote{\textcolor{pink}{Teinfaltstraße}}, ſehr gut gelegen, empfohlen; es iſt nicht abſolut erſten Ranges,
                    ſcheint mir aber angenehmer als die großen Hotels, \textcolor{pink}{\textsc{Imperial}}{}\ledrightnote{\textcolor{pink}{Hotel Imperial}}, \textcolor{pink}{\textsc{Grand Hotel}}{}\ledrightnote{\textcolor{pink}{Grand Hotel Wien}}, \textcolor{pink}{\textsc{Bristol}}{}\ledrightnote{\textcolor{pink}{Hotel Bristol}}. Vielleicht ſchreiben Sie mir noch näheres {\pb}über Ihre Wünſche; auf eine weitere
                    Nachricht von Ihrem Kommen dürfen wir ja hoffen?\pend
           \pstart
           Mit herzlichen Grüßen{\\[\baselineskip]}Ihr ergebenſter{\\[\baselineskip]}\spacefill\mbox{Arthur Schnitzler.}\pend
           \leftskip=0em{}\endnumbering\briefempfaengerindex{Brandes, Georg@\textsc{Brandes, Georg}!zzzSchnitzler, Arthur@\emph{von Arthur Schnitzler}!1897-12-311@{31. 12. 1897}|)be}\mylabel{h}  \normalsize

\doendnotes{C}
\bigskip
\vfill

\clearpage

\footnotesize

\lohead{\textsc{register}}

% Definiere theindex-Environment komplett neu ohne reledmac
\makeatletter
\renewenvironment{theindex}{%
  \section*{\indexname}%
  \setlength{\parindent}{0pt}%
  \setlength{\parskip}{0pt plus 0.3pt}%
  \let\item\@idxitem
}{%
  \clearpage
}
\makeatother

\IfFileExists{\jobname-pw.ind}{\input{\jobname-pw.ind}}{}

\end{document}

      