%% latex-korrekturansicht-vorspann.tex
%% Vorspann für die Korrekturansicht.
%% Lädt die gemeinsame Datei latex-vorspann.tex mit gesetztem Schalter.

\newif\ifkorrekturansicht
\korrekturansichttrue

\input{../tex-inputs/latex-vorspann}


               \section[Albert Ehrenstein an Arthur Schnitzler, 12. 7. 1910]{ Albert Ehrenstein an Arthur Schnitzler, 12. 7. 1910}\nopagebreak\mylabel{v}\rehead{ }\normalsize\beginnumbering\briefempfaengerindex{Schnitzler, Arthur@\textsc{Schnitzler, Arthur}!zzzEhrenstein, Albert@\emph{von Albert Ehrenstein}!1910-07-122@{12. 7. 1910}|(be} \toendnotes[C]{\smallbreak\pagebreak[2]} \Standort{CUL, Schnitzler, B 30.}
\physDesc{Brief, 2 Blätter, 7 Seiten
\newline{}Handschrift: schwarze Tinte, deutsche Kurrent
\newline{}Schnitzler: mit Bleistift beschriftet: »\textsc{Ehrenstein}« }\buchAbdrucke{\weitereDrucke{Albert Ehrenstein: \emph{Briefe}. Hg. Hanni Mittelmann. München: \emph{Boer} 1989, S. 45–48 (Werke, 1).} }\toendnotes[C]{\smallbreak}\pstart
           {\pb}\textsc{\textcolor{pink}{Vradist bei Holics}{}\ledrightnote{\textcolor{pink}{Vrádište}}, }\hfill \textsc{12. Juli 1910}\pend
           \pstart
           \textsc{\textcolor{pink}{Ungarn}{}\ledrightnote{\textcolor{pink}{Ungarn}}}\pend
           \pstart{}\textsc{Hochverehrter Herr Doktor,}\pend\pstart
           ich glaube, es wird, Sie vielleicht intereſſieren, wenn ich wieder einmal über
                    meine literariſchen Miß- und Erfolge Nachricht gebe. \textcolor{blue}{Kraus}{}\ledrightnote{\textcolor{blue}{Karl Kraus}}, mit dem ich übrigens bereits ſehr ſchlecht ſtehe,
                    weil wir beide, wie Sie wiſſen, recht unverträglich ſind, hat einmal ein \label{K_L01946_1v}\edtext{\textcolor{green}{Gedicht}{}\ledrightnote{→\textcolor{green}{Wanderers Lied}}}{\lemma{\textnormal{\emph{Gedicht}}}\Cendnote{\textnormal{\textcolor{blue}{Albert Ehrenstein}: \emph{\textcolor{green}{Wanderers Lied}}. In: \emph{\textcolor{green}{Die Fackel}}, Jg. 11, Nr. 296–297, 18. 2. 1910,
                            S. 36.}}}\label{K_L01946_1h} von mir gebracht, ein anderes akzeptiert, der
                    honorarfeindliche \textcolor{pink}{Berlin}{}\ledrightnote{\textcolor{pink}{Berlin}}er »\textcolor{brown}{Sturm}{}\ledrightnote{\textcolor{brown}{Der Sturm}}« zwei minderwertige \textcolor{green}{Skizzen}{}\ledrightnote{→\textcolor{green}{Die Parasiten der Parasiten}{\newline}→\textcolor{green}{Tod eines Seebären}}. Im übrigen ein Debacle
                    auf der ganzen Linie. Die Verlage \textcolor{brown}{Reiß}{}\ledrightnote{\textcolor{brown}{Erich Reiß}}, \textcolor{brown}{Fleiſchel}{}\ledrightnote{\textcolor{brown}{Egon Fleischel {\kaufmannsund} Co.}}, \textcolor{brown}{Langen}{}\ledrightnote{\textcolor{brown}{Albert Langen}}, \textcolor{brown}{\textcolor{blue}{v. Weber}{}\ledrightnote{\textcolor{blue}{Hans von Weber}}}{}\ledrightnote{→\textcolor{brown}{Hyperion}} haben meine Sachen ohne weitere Begründung refuſiert, \textcolor{blue}{Georg Müller}{}\ledrightnote{\textcolor{blue}{Georg Müller}} iſt trotz der Intervention der Herren \textcolor{blue}{Alfred Kubin}{}\ledrightnote{\textcolor{blue}{Alfred Kubin}} und \textcolor{blue}{A. Halbert}{}\ledrightnote{\textcolor{blue}{Abraham Halbert}} zu einer höflichen Ablehnung geſchritten, der
                        \textcolor{brown}{Inſelverlag}{}\ledrightnote{\textcolor{brown}{Insel-Verlag}} reagierte nach einer Empfehlung
                    durch \label{K_L01946_2v}\edtext{\textcolor{blue}{Paul Ernſt}{}\ledrightnote{\textcolor{blue}{Paul Ernst}}}{\lemma{\textnormal{\emph{Paul Ernſt}}}\Cendnote{\textnormal{Vgl. den Brief \textcolor{blue}{Ehrensteins} an \textcolor{blue}{Paul Ernst} vom
                                16. 5. 1910, abgedruckt in: \textcolor{blue}{A. E.}: \emph{Briefe},
                            S. 39.}}}\label{K_L01946_2h}{ }{\pb}ähnlich ſauer. An komiſchen
                    Werturteilen fehlte es nicht, \textcolor{blue}{Soyka}{}\ledrightnote{\textcolor{blue}{Otto Soyka}}{ }ſchimpfte mich ein Genie, \textcolor{blue}{Paul Ernſt}{}\ledrightnote{\textcolor{blue}{Paul Ernst}} gab zuerſt reichliches Lob von ſich, um
                    ſchließlich bei dem \textsc{Cliché} »frühreifes \textcolor{pink}{Wien}{}\ledrightnote{\textcolor{pink}{Wien}}er Talent, das längſtens in fünf Jahren abgeſtorben
                    ſein wird« zu enden. Angeſichts Ihrer Anſicht, vieles bei mir ſei noch unreif,
                    erinnert mich dieſer Widerſpruch lebhaft daran, daß \textcolor{blue}{Auernheimer}{}\ledrightnote{\textcolor{blue}{Raoul Auernheimer}} meine \textcolor{blue}{Th.
                        Mann}{}\ledrightnote{\textcolor{blue}{Thomas Mann}}-kritik dithyrambiſch nannte, \textcolor{blue}{Polgar}{}\ledrightnote{\textcolor{blue}{Alfred Polgar}}{ }ſie für ein abſcheuliches Pamphlet erklärte,
                    jener mich als phantaſtiſchen Schriftſteller rubrizierte, \textcolor{blue}{Großmann}{}\ledrightnote{\textcolor{blue}{Stefan Großmann}}{ }ſich durch meinen Realismus abgeſtoßen fühlte.
                    Die Prognoſe des D\textsuperscript{r} \textcolor{blue}{Ernſt}{}\ledrightnote{\textcolor{blue}{Paul Ernst}}{ }ſcheint mir \introOben{}jedenfalls\introOben{} unzutreffend: nach fünfjähriger Stagnation ſind mir meine
                    lyriſchen Fähigkeiten heuer wiedergekehrt. Immerhin hat eine \textcolor{green}{Ballade}{}\ledrightnote{→\textcolor{green}{Graf Cilli}}, die ich im Mai
                    fabrizierte, bereits den Rekord von zwölf Retournierungen. Ich möchte ſie mit
                    einigen anderen kleinen Arbeiten {\pb}Ihnen unterbreiten: Ich halte die Sachen nämlich nicht für ſo ſchlecht wie die
                    vereinigten Redaktionsphiliſter, deren Autogramme zu ſammeln mein Schickſal zu
                    ſein ſcheint. Die Herren \textcolor{blue}{Heſſe}{}\ledrightnote{\textcolor{blue}{Hermann Hesse}}, \label{K_L01946_3v}\edtext{\textcolor{blue}{Gumppenberg}{}\ledrightnote{\textcolor{blue}{Hanns von Gumppenberg}}}{\lemma{\textnormal{\emph{Gumppenberg}}}\Cendnote{\textnormal{Vgl. den Brief \textcolor{blue}{Ehrensteins} an \textcolor{blue}{Hanns von Gumppenberg} vom
                                16. 5. 1910, abgedruckt in: \textcolor{blue}{A. E.}: \emph{Briefe},
                            S. 38.}}}\label{K_L01946_3h}, \textcolor{blue}{K. B. Heinrich}{}\ledrightnote{\textcolor{blue}{Karl Borromäus Heinrich}},
                        \textcolor{blue}{Scheerbart}{}\ledrightnote{\textcolor{blue}{Paul Scheerbart}}, \textcolor{blue}{Lang-}{}\ledrightnote{\textcolor{blue}{Philipp Langmann}}, \textcolor{blue}{Wid-}{}\ledrightnote{\textcolor{blue}{Joseph Victor Widmann}}, \textcolor{blue}{Hoff-}{}\ledrightnote{\textcolor{blue}{Camill Hoffmann}} und \textcolor{blue}{Großmann}{}\ledrightnote{\textcolor{blue}{Stefan Großmann}}
                    behaupten einhellig eine intenſive Nichteignung meiner Arbeiten für Ihre
                    reſpektiven Blätter. \textcolor{blue}{Bie}{}\ledrightnote{\textcolor{blue}{Oskar Bie}} verwechſelt mich
                    konſtant mit \textcolor{blue}{R. Auernheimer}{}\ledrightnote{\textcolor{blue}{Raoul Auernheimer}}, \textcolor{pink}{Wien III}{}\ledrightnote{\textcolor{pink}{III., Landstraße}}, und verlangt immer wieder duftige \textcolor{pink}{Wien}{}\ledrightnote{\textcolor{pink}{Wien}}er Ware, die ich natürlich nicht herſtellen kann.
                    Kurz, es dürfte kein namhaftes Organ in \textcolor{pink}{Öſterreich}{}\ledrightnote{\textcolor{pink}{Österreich}} und \textcolor{pink}{Deutſchland}{}\ledrightnote{\textcolor{pink}{Deutschland}} geben, das
                    mich nicht mit ſeinen nichtsſagenden Ablehnungsformularen beglückt hätte. — Ein
                    Herr \textcolor{blue}{König}{}\ledrightnote{\textcolor{blue}{Otto König}} vom »\textcolor{brown}{Merker}{}\ledrightnote{\textcolor{brown}{Der Merker}}« möchte für den Spätherbſt eine kritiſche Studie über Sie, den
                    Dramatiker, von mir haben, aber ſein Blatt zahlt ſpät und ſchlecht, und mit
                    meiner Betrachtungsweiſe wäre wohl eher noch der Autor als der päpſtliche \textcolor{brown}{Merker}{}\ledrightnote{\textcolor{brown}{Der Merker}}{ }{\pb}einverſtanden. Ich würde Sie
                    nämlich, trotzdem Ihre Stücke oftmals von der Bühne her auf mich ſtark gewirkt
                    haben, ebenſowenig einen Dramatiker nennen wie etwa \textcolor{blue}{Grillparzer}{}\ledrightnote{\textcolor{blue}{Franz Grillparzer}} oder irgend einen anderen \textcolor{pink}{öſterreichiſchen}{}\ledrightnote{\textcolor{pink}{Österreich}} Dichter. Ich würde ſagen, Sie ſeien im
                    Grunde genommen ein Lyriker, ein Stimmungsdichter, der ſich zu\introOben{}r\introOben{}{ }\strikeout{ſeiner} Erreichung ſeiner Zwecke oft des
                    Dialoges, noch häufiger der epiſchen Form bedient. »\textcolor{green}{Der einſame Weg}{}\ledrightnote{\textcolor{green}{Der einsame Weg. Schauspiel in fünf Akten}}« zum Beiſpiel iſt nichts \introOben{}anderes\introOben{} als eine wunderſchöne, dialogiſierte Novelle, in der ebenſo
                    wie in den ähnlichen \textcolor{green}{Wahlverwandtſchaften}{}\ledrightnote{\textcolor{green}{Die Wahlverwandtschaften}}
                    (aber auch bei \textcolor{blue}{Homer}{}\ledrightnote{\textcolor{blue}{Homer}} und den \textcolor{green}{Buddenbrooks}{}\ledrightnote{\textcolor{green}{Buddenbrooks}}) ein Ausſterben der feiner organiſierten
                    Individuen, ein \substVorne{}\textsuperscript{Überleben}{\allowbreak}\substDazwischen{}Amlebenbleiben\substHinten{} der gangbareren Typen zu regiſtrieren iſt. Jene unerbittliche Logik,
                    jene unabwendbaren Reſultate ineinanderwachſender Motive, zu denen \textcolor{blue}{Shakeſpeare}{}\ledrightnote{\textcolor{blue}{William Shakespeare}} kam, hat von deutſchen \substVorne{}\textsuperscript{Dichtern}{\allowbreak}\substDazwischen{}Dramatikern\substHinten{} nicht einmal \textcolor{blue}{Kleiſt}{}\ledrightnote{\textcolor{blue}{Heinrich von Kleist}}; \textcolor{blue}{Hebbel}{}\ledrightnote{\textcolor{blue}{Friedrich Hebbel}} und \textcolor{blue}{Schiller}{}\ledrightnote{\textcolor{blue}{Friedrich von Schiller}}{ }ſind Dialektiker, {\pb}\textcolor{blue}{Goethe}{}\ledrightnote{\textcolor{blue}{Johann Wolfgang von Goethe}} iſt – ich weiß kein höheres Lob für
                    Ihren muſikaliſchen, ſtets melodiſchen Stil – Lyriker. Diejenigen Ihrer Werke,
                    die auf den Einfall und Einfälle geſtellt ſind, wie die meiſten Ihrer Einakter
                    und Dialoge, wüßte ich nicht zu beſprechen. Mit Mathematik befaſſe ich mich
                    nicht gern, und wenn, ſo würde ich den »\textcolor{green}{Reigen}{}\ledrightnote{\textcolor{green}{Reigen. Zehn Dialoge}}« als Vertreter hinſtellen und beklopfen. Behaupten, es gebräche der
                    Compoſition an Vollſtändigkeit, ſei man ſchon Algebraiker genug, die Prinzipien
                    der Combination und Permutation anzuwenden, hätte der Cirkus komplett ſein
                    müſſen, die Dörfer Sodom und Gomorrha nicht außer Betracht bleiben dürfen.\pend
           \pstart
           Über die Vollkommenheit wieder, repräſentiert durch den »\textcolor{green}{einſamen Weg}{}\ledrightnote{\textcolor{green}{Der einsame Weg. Schauspiel in fünf Akten}}«, »\textcolor{green}{großen
                        Wurſtel}{}\ledrightnote{\textcolor{green}{Zum großen Wurstel}}« und »\textcolor{green}{Schleier der Beatrice}{}\ledrightnote{\textcolor{green}{Der Schleier der Beatrice. Schauspiel in fünf Akten}}«
                    (deſſen Helden übrigens \introOben{}\strikeout{der unlogiſchere, ſentimentalere}\introOben{}\textcolor{blue}{Altenberg}{}\ledrightnote{\textcolor{blue}{Peter Altenberg}} nicht zum Selbſtmord hätten
                    ſchreiten laſſen, \introOben{}bloß\introOben{} weil die Vertreterin der {\pb}\strikeout{der} Weiblichkeit von einem anderen Mann träumte)
                    – über das Vollendete läßt ſich wenig ſagen. Vor allem aber gebricht es mir an
                    Material, ich kenne nicht jenen \textcolor{green}{Schauſpielereinakter}{}\ledrightnote{→\textcolor{green}{Das Haus Delorme. Eine Familienszene}}, der in \textcolor{pink}{Berlin}{}\ledrightnote{\textcolor{pink}{Berlin}}
                    zu einem \label{K_L01946_4v}\edtext{Skandal}{\lemma{\textnormal{\emph{Skandal}}}\Cendnote{\textnormal{\emph{\textcolor{green}{Das Haus Delorme}} wurde kurz vor der
                        Premiere im März 1904 zurückgezogen, wobei \textcolor{blue}{Schnitzler} selbst als Grund nannte, die Schauspieler
                        hätten ihr eigenes Milieu nicht darstellen mögen (\emph{Briefe} I,488–489).}}}\label{K_L01946_4h} führte, und was
                    mich noch mehr intereſſierte: ich kenne bis auf das Bruchſtück in einem \label{K_L01946_5v}\edtext{\textcolor{green}{Widmungsbuche}{}\ledrightnote{→\textcolor{green}{Widmungen zur Feier des siebzigsten Geburtstages Ferdinand von Saar’s.}}}{\lemma{\textnormal{\emph{Widmungsbuche}}}\Cendnote{\textnormal{\textcolor{blue}{Arthur Schnitzler}: \emph{\textcolor{green}{Liebelei. Erstes Bild}}. In: \emph{\textcolor{green}{Widmungen zur Feier des siebzigsten Geburtstages \textcolor{blue}{Ferdinand von Saar}’s}}. Hg.
                                \textcolor{blue}{Richard Specht}. Buchschmuck \textcolor{blue}{A. F. Seligmann}. Wien: \emph{\textcolor{brown}{Wiener Verlag}} 1903,
                            S. 175–196.}}}\label{K_L01946_5h} die erſte Faſſung der »\textcolor{green}{Liebelei}{}\ledrightnote{\textcolor{green}{Liebelei. Schauspiel in drei Akten}}« nicht, die mir in dieſer Form, nach dem Fragment
                    beurteilt, viel höheren Wert zu beſitzen ſcheint. (Dieſelbe legere Technik fand
                    ich in den in der »\textcolor{brown}{N. Fr. Preſſe}{}\ledrightnote{\textcolor{brown}{Neue Freie Presse}}«
                    veröffentlichten \label{K_L01946_6v}\edtext{Szenen}{\lemma{\textnormal{\emph{Szenen}}}\Cendnote{\textnormal{\textcolor{blue}{Arthur Schnitzler}: \emph{\textcolor{green}{Bastei-Szene. Erste Szene des dritten Aufzuges aus der
                                dramatischen Historie: »\so{Der junge Medardus}}}. In: \emph{\textcolor{green}{Neue Freie Presse}},
                            Nr. 16378, 27. 3. 1910, S. 32–39.}}}\label{K_L01946_6h} aus dem
                        »\textcolor{green}{Medardus}{}\ledrightnote{\textcolor{green}{Der junge Medardus. Dramatische Historie in einem Vorspiel und fünf Aufzügen}}« wieder, die andererſeits wieder
                    eine gewiſſe und vielleicht luſtige Ähnlichkeit mit dem »\textcolor{green}{Kakadu}{}\ledrightnote{\textcolor{green}{Der grüne Kakadu. Groteske in einem Akt}}« beſitzen.) \textsc{Summa summarum}
                    möchte ich ſehr gern ein Eſſay über Sie ſchreiben (ſchon weil ich Ihnen
                    womöglich jedes Gefallen an der vorliegenden Form des »\textcolor{green}{Wegs ins Freie}{}\ledrightnote{\textcolor{green}{Der Weg ins Freie. Roman}}« benehmen will), aber weder ſcheint mir {\pb}der »\textcolor{brown}{Merker}{}\ledrightnote{\textcolor{brown}{Der Merker}}« das geeignete Blatt, noch könnte ich ohne einiges
                    biographiſche und entwicklungsgeſchichtliche Material ſo ſchnell etwa Ihrer und
                    meiner Würdiges zu Tage befördern. Wenigſtens kaum vor März 1911,
                    denn meine Studien machen nur langſame Fortſchritte. Zwar ſind die
                    geographiſch-hiſtoriſchen Arbeiten bereits approbiert, das kleine philoſophiſche
                    Rigoroſum bereits hinter mir und ſo ſteht zu befürchten, daß ich im
                        Oktober zum Dr. phil. degradiert werde. Aber ich \substVorne{}\textsuperscript{fürchte,}{\allowbreak}\substDazwischen{}beſorge\substHinten{} nicht über genügend ſtarke Protektion zu verfügen, um ins \textcolor{brown}{Miniſterium des Unterrichts}{}\ledrightnote{\textcolor{brown}{Ministerium für Unterricht}} oder \textcolor{brown}{Inneren}{}\ledrightnote{\textcolor{brown}{Ministerium für Inneres}} kommen zu können und es müßte alſo im
                        Jänner{ }ſchreckliche, überdies nicht gerade viel
                    Chancen bietende Lehramtsprüfungen ablegen\pend
           \pstart
           Ihr Hochachtungsvoll und ergebenſt grüßender{\\[\baselineskip]}\spacefill\mbox{Albert Ehrenstein.}\pend
           \leftskip=0em{}\endnumbering\briefempfaengerindex{Schnitzler, Arthur@\textsc{Schnitzler, Arthur}!zzzEhrenstein, Albert@\emph{von Albert Ehrenstein}!1910-07-122@{12. 7. 1910}|)be}\mylabel{h}  \normalsize

\doendnotes{C}
\bigskip
\vfill

\clearpage

\footnotesize

\lohead{\textsc{register}}

% Definiere theindex-Environment komplett neu ohne reledmac
\makeatletter
\renewenvironment{theindex}{%
  \section*{\indexname}%
  \setlength{\parindent}{0pt}%
  \setlength{\parskip}{0pt plus 0.3pt}%
  \let\item\@idxitem
}{%
  \clearpage
}
\makeatother

\IfFileExists{\jobname-pw.ind}{\input{\jobname-pw.ind}}{}

\end{document}

      