%% latex-korrekturansicht-vorspann.tex
%% Vorspann für die Korrekturansicht.
%% Lädt die gemeinsame Datei latex-vorspann.tex mit gesetztem Schalter.

\newif\ifkorrekturansicht
\korrekturansichttrue

\input{../tex-inputs/latex-vorspann}


               \section[Arthur Schnitzler an Richard Beer-Hofmann, 6. 8. 1903]{ Arthur Schnitzler an Richard Beer-Hofmann, 6. 8. 1903}\nopagebreak\mylabel{v}\rehead{ }\normalsize\beginnumbering\briefempfaengerindex{Beer-Hofmann, Richard@\textsc{Beer-Hofmann, Richard}!zzzSchnitzler, Arthur@\emph{von Arthur Schnitzler}!1903-08-061@{6. 8. 1903}|(be} \toendnotes[C]{\smallbreak\pagebreak[2]} \Standort{YCGL, MSS 31.}
\physDesc{Brief, 1 Blatt, 3 Seiten, Umschlag
\newline{}Handschrift: Bleistift, deutsche Kurrent\newline{}Versand: 1) Stempel: »\nobreak{}\oindex{XIII., Hietzing@\textbf{XIII., Hietzing}, \emph{Bezirk (A.BZK)}|pwk}Wien 13/2 89, 1\textcolor{gray}{1}–12V\nobreak{}«.  2) Stempel: »\nobreak{}\oindex{Rodaun@\textbf{Rodaun}, \emph{Teil eines besiedelten Ortes (A.BSOX)}|pwk}{\pb}Rodaun, 6. 8. 03, 2{[}–{]}4N\nobreak{}«. \newline{}Ordnung: mit Bleistift von unbekannter Hand datiert: »6. 8.« }\toendnotes[C]{\smallbreak}\pstart{}{\pb}Herrn \textsc{Dr Richard
                            Beer-Hofmann}\pend{}\pstart{}\textcolor{pink}{\textsc{Rodaun bei Liesing}}{}\ledrightnote{\textcolor{pink}{Rodaun}}\pend{}\pstart{}\textcolor{pink}{\textsc{Liesinger Straße 2}}{}\ledrightnote{\textcolor{pink}{Liesingerstraße}}.\pend{}{\bigskip}\pstart
           \raggedleft{}{\pb}\uline{Donnerſtag}.
                    \pend
           \pstart{}lieber Richard,\pend\pstart
           könnte man ſich vielleicht \label{K_L01306_1v}\edtext{Samſtag}{\lemma{\textnormal{\emph{Samſtag}}}\Cendnote{\textnormal{siehe A. S.: \emph{Tagebuch}, 8. 8. 1903}}}\label{K_L01306_1h} zum Nachtmahl in dem \textcolor{pink}{Hietzinger \textsc{Restaurant}}{}\ledrightnote{\textcolor{pink}{Ottakringer Bräu}} treffen? Vorher \textcolor{pink}{Schönbrunn}{}\ledrightnote{\textcolor{pink}{Schloß Schönbrunn}}? \textsc{\textcolor{blue}{Hugo}{}\ledrightnote{\textcolor{blue}{Hugo von Hofmannsthal}}} desgleichen\textcolor{gray}{?}\pend
           \pstart
           Schreiben Sie mir bitte ein Wort {\pb}Stund und Ort zu
                    fixiren (\textcolor{pink}{Glashaus}{}\ledrightnote{\textcolor{pink}{Palmenhaus Schönbrunn}}? Sieben?)
                    Schlechtes Wetter ſollte (in Hinſicht aufs \textsc{Rest.}) kein
                    Hindernis ſein.\pend
           \pstart
           Herzlichſt{\\[\baselineskip]}Ihr{\\[\baselineskip]}\spacefill\mbox{Arthur}\pend
           \leftskip=0em{}\pstart
           \noindent{}{\pb}Wenn Sie telephoniren bitte am beſten
                        2–3.\pend
           \endnumbering\briefempfaengerindex{Beer-Hofmann, Richard@\textsc{Beer-Hofmann, Richard}!zzzSchnitzler, Arthur@\emph{von Arthur Schnitzler}!1903-08-061@{6. 8. 1903}|)be}\mylabel{h}  \normalsize

\doendnotes{C}
\bigskip
\vfill

\clearpage

\footnotesize

\lohead{\textsc{register}}

% Definiere theindex-Environment komplett neu ohne reledmac
\makeatletter
\renewenvironment{theindex}{%
  \section*{\indexname}%
  \setlength{\parindent}{0pt}%
  \setlength{\parskip}{0pt plus 0.3pt}%
  \let\item\@idxitem
}{%
  \clearpage
}
\makeatother

\IfFileExists{\jobname-pw.ind}{\input{\jobname-pw.ind}}{}

\end{document}

      