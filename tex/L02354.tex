%% latex-korrekturansicht-vorspann.tex
%% Vorspann für die Korrekturansicht.
%% Lädt die gemeinsame Datei latex-vorspann.tex mit gesetztem Schalter.

\newif\ifkorrekturansicht
\korrekturansichttrue

\input{../tex-inputs/latex-vorspann}


               \section[Georg Brandes an Arthur Schnitzler, 17. 8. 1920]{ Georg Brandes an Arthur Schnitzler, 17. 8. 1920}\nopagebreak\mylabel{v}\rehead{ }\normalsize\beginnumbering\briefempfaengerindex{Schnitzler, Arthur@\textsc{Schnitzler, Arthur}!zzzBrandes, Georg@\emph{von Georg Brandes}!1920-08-171@{17. 8. 1920}|(be} \toendnotes[C]{\smallbreak\pagebreak[2]} \Standort{CUL, Schnitzler, B 17.}
\physDesc{Postkarte
\newline{}Handschrift: schwarze Tinte, lateinische Kurrent\newline{}Versand: Stempel: »\nobreak{}\oindex{Kopenhagen@\textbf{Kopenhagen}, \emph{Besiedelter Ort (A.BSO)}|pwk}Kjøbenhavn, 17. 8. 20, 6–7 E\nobreak{}«.  
\newline{}Schnitzler: mit Bleistift beschriftet: »\textsc{Brandes}« \newline{}Ordnung: mit Bleistift von unbekannter Hand nummeriert:
                                    »51« }\buchAbdrucke{\weitereDrucke{Georg Brandes, Arthur Schnitzler: \emph{Ein Briefwechsel}. Hg. Kurt Bergel. Bern: \emph{Francke} 1956, S. 130–131.} }\toendnotes[C]{\smallbreak}\pstart{}{\pb}Herrn Dr. Arthur
                  Schnitzler\pend{}\pstart{}\textcolor{pink}{Sternwartestraße 71}{}\ledrightnote{\textcolor{pink}{Sternwartestraße}}\pend{}\pstart{}\textcolor{pink}{Wien \textsubscript{XVIII}}{}\ledrightnote{\textcolor{pink}{XVIII., Währing}}\pend{}{\bigskip}\pstart
           \raggedleft{}{\pb}\textcolor{pink}{Kopenhagen}{}\ledrightnote{\textcolor{pink}{Kopenhagen}}{ }17 August 20\pend
           \pstart{}Verehrtester Freund\pend\pstart
           Am 13 Juni schrieb ich Ihnen nach langem Schweigen einen sehr langen und
               ausführlichen Brief in der Hoffnung ein wenig über Sie, die Ihrigen und gemeinsame
               Freunde zu hören.\pend
           \pstart
           Ich erhielt nie eine Zeile Antwort, und da es immerhin möglich ist, dass mein Brief
               Sie nicht erreicht hat, erlaube ich mir die Anfrage, ob Sie ihn erhalten haben, ob
               Sie zum Antworten – was ich höchst natürlich finde – nicht aufgelegt waren. Ein
               Vorwurf würde Sie wahrlich nicht treffen. Aber in früherer Zeit antworteten Sie
               willig, obwol {\pb}die Correspondenz
               uns Allen ein \label{K_L02354_1v}\edtext{corvée}{\lemma{\textnormal{\emph{corvée}}}\Cendnote{\textnormal{französisch: Mühsal}}}\label{K_L02354_1h} geworden
               ist.\pend
           \pstart
           Die Verhältnisse sind ja in \textcolor{pink}{Wien}{}\ledrightnote{\textcolor{pink}{Wien}} besonders schwierig
               und traurig. Ich denke mir, dass Sie überhaupt nicht den Sommer in \textcolor{pink}{Wien}{}\ledrightnote{\textcolor{pink}{Wien}} verbringen.\pend
           \pstart Ihr in alter Freundschaft ergebener \spacefill\mbox{Georg Brandes}\pend{}\endnumbering\briefempfaengerindex{Schnitzler, Arthur@\textsc{Schnitzler, Arthur}!zzzBrandes, Georg@\emph{von Georg Brandes}!1920-08-171@{17. 8. 1920}|)be}\mylabel{h}  \normalsize

\doendnotes{C}
\bigskip
\vfill

\clearpage

\footnotesize

\lohead{\textsc{register}}

% Definiere theindex-Environment komplett neu ohne reledmac
\makeatletter
\renewenvironment{theindex}{%
  \section*{\indexname}%
  \setlength{\parindent}{0pt}%
  \setlength{\parskip}{0pt plus 0.3pt}%
  \let\item\@idxitem
}{%
  \clearpage
}
\makeatother

\IfFileExists{\jobname-pw.ind}{\input{\jobname-pw.ind}}{}

\end{document}

      