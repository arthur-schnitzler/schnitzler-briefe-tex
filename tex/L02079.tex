%% latex-korrekturansicht-vorspann.tex
%% Vorspann für die Korrekturansicht.
%% Lädt die gemeinsame Datei latex-vorspann.tex mit gesetztem Schalter.

\newif\ifkorrekturansicht
\korrekturansichttrue

\input{../tex-inputs/latex-vorspann}


               \section[Arthur Schnitzler an Richard Beer-Hofmann, 19. 7. 1912]{ Arthur Schnitzler an Richard Beer-Hofmann, 19. 7. 1912}\nopagebreak\mylabel{v}\rehead{ }\normalsize\beginnumbering\briefempfaengerindex{Beer-Hofmann, Richard@\textsc{Beer-Hofmann, Richard}!zzzSchnitzler, Arthur@\emph{von Arthur Schnitzler}!1912-07-191@{19. 7. 1912}|(be} \toendnotes[C]{\smallbreak\pagebreak[2]} \Standort{YCGL, MSS 31.}
\physDesc{Bildpostkarte
\newline{}Handschrift: schwarze Tinte, deutsche Kurrent\newline{}Versand: Stempel: »\nobreak{}\oindex{IX., Alsergrund@\textbf{IX., Alsergrund}, \emph{Bezirk (A.BZK)}|pwk}9/4 \textcolor{gray}{Wien 53}, 19 VI\textcolor{gray}{I} 12, 12\nobreak{}«.  \newline{}Zusatz: Postkartenmotiv mit \textcolor{blue}{Olga} und
                                    \textcolor{blue}{Heinrich} links vor dem Haus
                                 und Schnitzler und \textcolor{blue}{Lili} auf dem
                                 Söller }\buchAbdrucke{\weitereDrucke{Arthur Schnitzler, Richard Beer-Hofmann: \emph{Briefwechsel 1891–1931}. Hg. Konstanze Fliedl. Wien, Zürich: \emph{Europaverlag} 1992, S. 216–217.} }\pstart{}{\pb}Herrn \textsc{Dr. Richard Beer
                     hofmann}\pend{}\pstart{}aus \textcolor{pink}{Wien}{}\ledrightnote{\textcolor{pink}{Wien}}\pend{}\pstart{}\textsc{d. z. Z. \textcolor{pink}{Hotel Waldhaus}{}\ledrightnote{\textcolor{pink}{Hotel Waldhaus}}}\pend{}\pstart{}\textsc{\textcolor{pink}{St. Moritz
                     }{}\ledrightnote{\textcolor{pink}{Sankt Moritz}}}\pend{}\pstart{}\textsc{im \textcolor{pink}{Engadin}{}\ledrightnote{\textcolor{pink}{Engadin}}}\pend{}{\bigskip}\pstart
           \noindent{}\centering{}{\pb}\textcolor{gray}{\textbf{\textcolor{pink}{Wien, XVIII, Sternwartestr. 71}{}\ledrightnote{\textcolor{pink}{Sternwartestraße}}.}}\pend
           \pstart
           \raggedleft{}19. 7. 912.\pend
           \pstart
           (\textsc{My house is my Nachtkastl})\pend
           \pstart
           {\pb}lieber Richard, viel Dank für Ihre Karten aus denen ich allerdings
               nicht mehr entnehme als daſs Sie dort ſind u wieder wegreiſen werden. Aber ich hoffe
               Sie ſind alle wohl und (verzeihen Sie) vergnügt und wir hören mehr von Ihnen nach \textcolor{pink}{\uline{Brioni}}{}\ledrightnote{\textcolor{pink}{Brijuni}} bei \textcolor{pink}{Pola}{}\ledrightnote{\textcolor{pink}{Pola}}, wohin wir morgen abſegeln.
               Herzlichſt mit allſeitigen Grüßen \pend
           \pstart Ihr \spacefill\mbox{A.}\pend{}\endnumbering\briefempfaengerindex{Beer-Hofmann, Richard@\textsc{Beer-Hofmann, Richard}!zzzSchnitzler, Arthur@\emph{von Arthur Schnitzler}!1912-07-191@{19. 7. 1912}|)be}\mylabel{h}  \normalsize

\doendnotes{C}
\bigskip
\vfill

\clearpage

\footnotesize

\lohead{\textsc{register}}

% Definiere theindex-Environment komplett neu ohne reledmac
\makeatletter
\renewenvironment{theindex}{%
  \section*{\indexname}%
  \setlength{\parindent}{0pt}%
  \setlength{\parskip}{0pt plus 0.3pt}%
  \let\item\@idxitem
}{%
  \clearpage
}
\makeatother

\IfFileExists{\jobname-pw.ind}{\input{\jobname-pw.ind}}{}

\end{document}

      