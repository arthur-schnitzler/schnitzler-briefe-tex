%% latex-korrekturansicht-vorspann.tex
%% Vorspann für die Korrekturansicht.
%% Lädt die gemeinsame Datei latex-vorspann.tex mit gesetztem Schalter.

\newif\ifkorrekturansicht
\korrekturansichttrue

\input{../tex-inputs/latex-vorspann}


               \section[Richard Beer-Hofmann an Arthur Schnitzler, 22. 2. 1900]{ Richard Beer-Hofmann an Arthur Schnitzler, 22. 2. 1900}\nopagebreak\mylabel{v}\rehead{ }\normalsize\beginnumbering\briefempfaengerindex{Schnitzler, Arthur@\textsc{Schnitzler, Arthur}!zzzBeer-Hofmann, Richard@\emph{von Richard Beer-Hofmann}!1900-02-221@{22. 2. 1900}|(be} \toendnotes[C]{\smallbreak\pagebreak[2]} \Standort{CUL, Schnitzler, B 8.}
\physDesc{Brief, 2 Blätter, 4 Seiten
\newline{}Handschrift: schwarze Tinte, lateinische Kurrent\newline{}Ordnung: mit Bleistift von unbekannter Hand nummeriert: »151« }\buchAbdrucke{\weitereDrucke{Arthur Schnitzler, Richard Beer-Hofmann: \emph{Briefwechsel 1891–1931}. Hg. Konstanze Fliedl. Wien, Zürich: \emph{Europaverlag} 1992, S. 142–143.} }\toendnotes[C]{\smallbreak}\pstart
           \raggedleft{}{\pb}\textcolor{pink}{Sanremo}{}\ledrightnote{\textcolor{pink}{Sanremo}}{ }22/II 1900\pend
           \pstart
           Mein lieber Arthur! »Beneiden«! Mein Gott! Wissen Sie was »beneiden«
               heißt? »Das Andere nicht wissen.« Im übrigen, dieser demonstrative »Süden« mit
               »Nachtkastel-Palmen«, der um 5 Uhr Abends die Maske abwirft, ist recht traurig.
               Überhaupt versagt \textcolor{pink}{Italien}{}\ledrightnote{\textcolor{pink}{Italien}} zum erstenmal bei mir;
               vielleicht wirds in \textcolor{pink}{Florenz}{}\ledrightnote{\textcolor{pink}{Florenz}} besser. Ich vertrage es
               offenbar nicht irgendwohin direkt des schönen Wetters halber zu gehen. Sofort fang
               ich an aufs Wetter aufzupassen, bemerke wenn es blufft, und finde schließlich daß es,
               wie alle Dinge wenn man ihnen auf die Finger sieht, auch »in seinem Fach ein Esel«
               ist, und gar nicht weiß wie schönes Wetter eigentlich sein soll. Man darf gar nichts
               genau ansehen wollen; {\pb}Vielleicht
               heisst das große Geheimniß eines erträglichen Daseins: Oberflächlichkeit. Unsereiner,
               der einmal zu graben begonnen hat, kann freilich nicht mehr zurück; aber vielleicht
               geht es an so tief zu graben bis man auf der anderen Seite wieder herausko{\geminationm}t; das ist dann unsere »Oberflächlichkeit«. Der nächste
               Weg ist das nicht! »\label{K_L01016_1v}\edtext{Pollak wo hast Du Dein linkes
                  Ohr?}{\lemma{\textnormal{\emph{Pollak … Ohr?}}}\Cendnote{\textnormal{Stehende
                  Redewendung für den Griff mit der rechten Hand über den Kopf zum linken Ohr. Ein
                  (jüdischer) Junge, der vom Lehrer gefragt wurde, wo er sein linkes Ohr habe, soll
                  diese umständliche Geste gemacht haben. Vgl. Arthur Schnitzler an Richard Beer-Hofmann, 15. 10. 1894}}}\label{K_L01016_1h}«\pend
           \pstart
           Meine \textcolor{blue}{Frau}{}\ledrightnote{→\textcolor{blue}{Paula Beer-Hofmann}} hat sich bisher
               nicht erholt, ich habe hier einen Husten beko{\geminationm}en, die
               Einzige die sich wol fühlt ist \textcolor{blue}{Mirjam}{}\ledrightnote{\textcolor{blue}{Mirjam Beer-Hofmann}}; bis sie
               größer sein wird, wirds schon besser werden. Frau Professor \label{K_L01016_2v}\edtext{\textcolor{blue}{Döppler}{}\ledrightnote{\textcolor{blue}{Berta Doepler}}}{\lemma{\textnormal{\emph{Döppler}}}\Cendnote{\textnormal{\textcolor{blue}{Berta Doepler} ist am 25. 7. 1895
                  auf der Kurliste von \textcolor{pink}{Bad Ischl} verzeichnet,
                  wodurch eine frühere Bekanntschaft anzunehmen ist.}}}\label{K_L01016_2h} habe ich hier getroffen
               und mir von ihr vortratschen lassen, was sie amüsant und eifrig hat;
               Ideenassociation: \textcolor{blue}{Elly H.}{}\ledrightnote{\textcolor{blue}{Elly Petersen}} hat sich richtig, wie
               ich herzloser Weise schon vorher zu \textcolor{blue}{Meyer}{}\ledrightnote{\textcolor{blue}{Oskar Mayer}} sagte, mit ihrer Krankheit eine {\pb}Position bei uns gemacht; man kann
               nicht sagen daß es mit wenig Einsatz geschehen ist. Wenn ihr \textcolor{blue}{Mann}{}\ledrightnote{→\textcolor{blue}{Georg Hirschfeld}} jetzt noch kein Geld verdienen würde,
               wäre er ein Dichter – für uns – nur um nicht roh zu sein. Frau Professor \textcolor{blue}{D.}{}\ledrightnote{\textcolor{blue}{Berta Doepler}} hat ihn – sie findet ihn überschätzt – mit dem
               Zeichner \uline{\textcolor{blue}{Allers}{}\ledrightnote{\textcolor{blue}{Christian Wilhelm Allers}}} verglichen; wer von \textcolor{blue}{H.}{}\ledrightnote{\textcolor{blue}{Georg Hirschfeld}}s Freunden ihr das
               beigebracht haben mag? Auf ihrem eigenen Mist ist das nicht gewachsen; ich glaube
               übrigens sie hat überhaupt keinen eigenen Mist. Daß Sie sich die Lektüre von \textcolor{green}{Georgs Tod}{}\ledrightnote{\textcolor{green}{Der Tod Georgs}} für einen Frühlingstag auf dem Land
               aufheben ist sicher für das Buch gut; ob auch für den Tag? Wenn Sie mir durchaus das
                  \textcolor{green}{Buch}{}\ledrightnote{→\textcolor{green}{Wiener Bummelgeschichten}} des »\textcolor{blue}{dampfenden Jünglings}{}\ledrightnote{→\textcolor{blue}{Max Messer}}« schicken wollen, schicken
               Sie es nach \textcolor{pink}{Florenz}{}\ledrightnote{\textcolor{pink}{Florenz}}, poste {\pb}restante. Nicht vielleicht deshalb
               weil ich hier bin, sondern weil ich am 27. dort sein will.\pend
           \pstart
           Ich arbeite natürlich nichts. Von \textcolor{blue}{Hugo}{}\ledrightnote{\textcolor{blue}{Hugo von Hofmannsthal}} habe ich
               keinerlei Nachricht. An \textcolor{blue}{Brandes}{}\ledrightnote{\textcolor{blue}{Georg Brandes}} habe ich heute
               mein \textcolor{green}{Buch}{}\ledrightnote{→\textcolor{green}{Der Tod Georgs}} geschickt Ich glaube nicht, daß er was damit anfangen kann. Auch \textcolor{blue}{Robert Hirschfeld}{}\ledrightnote{\textcolor{blue}{Robert Hirschfeld}} der mich vor meiner Abreise
               becomplimentirte scheint keine Ahnung zu haben was der Inhalt des \textcolor{green}{Buches}{}\ledrightnote{→\textcolor{green}{Der Tod Georgs}} ist. Was macht \textcolor{blue}{Gustav}{}\ledrightnote{\textcolor{blue}{Gustav Schwarzkopf}}; während ich seinen Vornamen niederschreibe werde ich so
               verlegen, als sähe ich sein ungläubiges Lächeln zu dieser Intimität. Grüßen Sie ihn,
               dann à discretion die Übrigen, aber in gemessenen Distanzen.\pend
           \pstart
           Wie ich meinen Brief überlese, finde ich daß er »\uline{witzig}« ist. »\textcolor{green}{Gott sei Dank er wird witzig}{}\ledrightnote{→\textcolor{green}{Kabale und Liebe}}«! Aber der
               Hofmarschall \textcolor{green}{Kalb}{}\ledrightnote{→\textcolor{green}{Kabale und Liebe}}, der das sagt
               weiß nicht daß das für den \textcolor{green}{Ferdinand}{}\ledrightnote{→\textcolor{green}{Kabale und Liebe}} ein schlechtes Symptom ist. Für mich auch.\pend
           \pstart
           Von Herzen\hspace*{1.5em}Ihr{\\[\baselineskip]}\spacefill\mbox{Richard}\pend
           \leftskip=0em{}\endnumbering\briefempfaengerindex{Schnitzler, Arthur@\textsc{Schnitzler, Arthur}!zzzBeer-Hofmann, Richard@\emph{von Richard Beer-Hofmann}!1900-02-221@{22. 2. 1900}|)be}\mylabel{h}  \normalsize

\doendnotes{C}
\bigskip
\vfill

\clearpage

\footnotesize

\lohead{\textsc{register}}

% Definiere theindex-Environment komplett neu ohne reledmac
\makeatletter
\renewenvironment{theindex}{%
  \section*{\indexname}%
  \setlength{\parindent}{0pt}%
  \setlength{\parskip}{0pt plus 0.3pt}%
  \let\item\@idxitem
}{%
  \clearpage
}
\makeatother

\IfFileExists{\jobname-pw.ind}{\input{\jobname-pw.ind}}{}

\end{document}

      