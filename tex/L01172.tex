%% latex-korrekturansicht-vorspann.tex
%% Vorspann für die Korrekturansicht.
%% Lädt die gemeinsame Datei latex-vorspann.tex mit gesetztem Schalter.

\newif\ifkorrekturansicht
\korrekturansichttrue

\input{../tex-inputs/latex-vorspann}


               \section[Hermann Bahr an Arthur Schnitzler, 12. 9. 1901]{ Hermann Bahr an Arthur Schnitzler, 12. 9. 1901}\nopagebreak\mylabel{v}\rehead{ }\normalsize\beginnumbering\briefempfaengerindex{Schnitzler, Arthur@\textsc{Schnitzler, Arthur}!zzzBahr, Hermann@\emph{von Hermann Bahr}!1901-09-122@{12. 9. 1901}|(be} \toendnotes[C]{\smallbreak\pagebreak[2]} \Standort{CUL, Schnitzler, B 5b.}
\physDesc{Brief, 1 Blatt, 2 Seiten
\newline{}Handschrift: schwarze Tinte, deutsche Kurrent
\newline{}Schnitzler: mit Bleistift die Jahreszahl »901« ergänzt \newline{}Ordnung: mit Bleistift von unbekannter Hand nummeriert:
                                    »79« }\buchAbdrucke{\weitereDrucke{Hermann Bahr, Arthur Schnitzler: \emph{Briefwechsel, Aufzeichnungen, Dokumente (1891–1931)}. Hg. Kurt Ifkovits und Martin Anton Müller. Göttingen: \emph{Wallstein} 2018, S. 214.} }\toendnotes[C]{\smallbreak}\pstart
           \noindent{}\centering{}{\pb}\textcolor{gray}{\textbf{\textcolor{brown}{Redaktion des Neuen Wiener Tagblatt}{}\ledrightnote{\textcolor{brown}{Neues Wiener Tagblatt}}}}\pend
           \pstart
           \noindent{}\centering{}\textcolor{gray}{\textbf{\textsc{\textcolor{pink}{Wien, I., Rothenturmstrasse,
                        Steyrerhof}{}\ledrightnote{\textcolor{pink}{Steyrerhof}}.}}}\pend
           \pstart
           \noindent{}\centering{}\textcolor{gray}{\textbf{Telegramm-Adresse: \textcolor{brown}{Tagblatt}{}\ledrightnote{\textcolor{brown}{Neues Wiener Tagblatt}},
                        \textcolor{pink}{Steyrerhof, Wien}{}\ledrightnote{\textcolor{pink}{Steyrerhof}}. – Telephon Nr. 384.
                     Staats-Telephon Nr. 36.}}\pend
           \pstart
           \raggedleft{}12. 9.\pend
           \pstart\center{}Lieber Arthur!\pend\pstart
           Ich habe Deine \textcolor{green}{Stücke}{}\ledrightnote{→\textcolor{green}{Lebendige Stunden. Vier Einakter}{\newline}→\textcolor{green}{Die Frau mit dem Dolche}}
               geſtern abends bekommen, nachts geleſen und heute früh dem \textcolor{blue}{\textsc{Bukovics}}{}\ledrightnote{\textcolor{blue}{Emerich von Bukovics}} gegeben. Die Idee, die Du in ihnen mit Deiner wunderbaren, ja ganz einzigen
               Technik ausführſt, geht mir ſehr nahe und berührt mich ſehr; in \label{K_L01172_1v}\edtext{\textcolor{green}{einer}{}\ledrightnote{→\textcolor{green}{Das schöne Mädchen. Pantomime}} der »\textcolor{green}{Exiſtenzen}{}\ledrightnote{→\textcolor{green}{Das schöne Mädchen. Pantomime}}«, für \textcolor{blue}{Salten}{}\ledrightnote{\textcolor{blue}{Felix Salten}}}{\lemma{\textnormal{\emph{einer … Salten}}}\Cendnote{\textnormal{\emph{\textcolor{green}{Das schöne Mädchen}}, verfasst für das von \textcolor{blue}{Salten} geleitete Kabarett \textcolor{pink}{Zum lieben Augustin} (veröffentlicht in: \emph{\textcolor{green}{Schwarz auf Weiss}}. Wien: \emph{Comité für
                        das Fest der Kunstgewerbeschüler}{ }1902, S. 23–32).}}}\label{K_L01172_1h}, iſt was ähnliches gemeint, nur
               pantomimiſch und ſchon deshalb roher dargeſtellt. In den »\textcolor{green}{Lebendigen Stunden}{}\ledrightnote{\textcolor{green}{Lebendige Stunden. Vier Einakter}}« möchte ich die Verſtorbene deutlicher {\pb}zu ſehen kriegen. Im »\textcolor{green}{Dolch}{}\ledrightnote{\textcolor{green}{Die Frau mit dem Dolche}}« fürchte ich die Dummheit unſerer Premièren-Idioten; auch macht mir
               Sorge, ob die zweite Verwandlung rapid genug geſchehen kann. Aber von alledem
               mündlich und \introOben{}in\introOben{} Ruhe, wenn ich nicht gerade auf dem Sprung
               zur \label{K_L01172_2v}\edtext{\textcolor{green}{Stuart}{}\ledrightnote{\textcolor{green}{Maria Stuart}}}{\lemma{\textnormal{\emph{Stuart}}}\Cendnote{\textnormal{im \textcolor{pink}{Deutschen
                     Volkstheater}; keine Premiere.}}}\label{K_L01172_2h} bin.\pend
           \pstart
           Herzlichſt{\\[\baselineskip]}Dein{\\[\baselineskip]}\spacefill\mbox{Hermann}\pend
           \leftskip=0em{}\endnumbering\briefempfaengerindex{Schnitzler, Arthur@\textsc{Schnitzler, Arthur}!zzzBahr, Hermann@\emph{von Hermann Bahr}!1901-09-122@{12. 9. 1901}|)be}\mylabel{h}  \normalsize

\doendnotes{C}
\bigskip
\vfill

\clearpage

\footnotesize

\lohead{\textsc{register}}

% Definiere theindex-Environment komplett neu ohne reledmac
\makeatletter
\renewenvironment{theindex}{%
  \section*{\indexname}%
  \setlength{\parindent}{0pt}%
  \setlength{\parskip}{0pt plus 0.3pt}%
  \let\item\@idxitem
}{%
  \clearpage
}
\makeatother

\IfFileExists{\jobname-pw.ind}{\input{\jobname-pw.ind}}{}

\end{document}

      