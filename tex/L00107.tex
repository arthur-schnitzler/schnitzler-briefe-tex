%% latex-korrekturansicht-vorspann.tex
%% Vorspann für die Korrekturansicht.
%% Lädt die gemeinsame Datei latex-vorspann.tex mit gesetztem Schalter.

\newif\ifkorrekturansicht
\korrekturansichttrue

\input{../tex-inputs/latex-vorspann}


               \section[Arthur Schnitzler an Hugo von Hofmannsthal, {[}25.? 7. 1892{]}]{ Arthur Schnitzler an Hugo von Hofmannsthal, {[}25.? 7. 1892{]}}\nopagebreak\mylabel{v}\rehead{ }\normalsize\beginnumbering\briefempfaengerindex{Hofmannsthal, Hugo von@\textsc{Hofmannsthal, Hugo von}!zzzSchnitzler, Arthur@\emph{von Arthur Schnitzler}!1892-07-251@{{[}25.? 7. 1892{]}}|(be} \toendnotes[C]{\smallbreak\pagebreak[2]} \Standort{FDH, Hs-30885,23.}
\physDesc{Brief, 1 Blatt, 3 Seiten
\newline{}Handschrift: Bleistift, deutsche Kurrent\newline{}Ordnung: von unbekannter Hand datiert: »So{\geminationm}er 92« }\buchAbdrucke{\weitereDrucke{Hugo von Hofmannsthal, Arthur Schnitzler: \emph{Briefwechsel}. Hg. Therese Nickl und Heinrich Schnitzler. Frankfurt am Main: \emph{S. Fischer} 1964, S. 24.} }\toendnotes[C]{\smallbreak}\pstart
           \noindent{}{\pb}Lieber Loris! Nächſtens mehr! Heute nur
                    eine Frage. – Mein \textcolor{green}{Anatol Cyclus}{}\ledrightnote{\textcolor{green}{Anatol}} erſcheint im
                        October im \textcolor{brown}{\textsc{Bibl. Bureau}}{}\ledrightnote{\textcolor{brown}{Bibliographisches Bureau}} (nächſtens näheres). – Ihr \textcolor{green}{Gedicht}{}\ledrightnote{→\textcolor{green}{Einleitung}} leitet die Sa{\geminationm}lung ein; wollen
                    Sie ihm irgend einen Namen geben; haben Sie ſonſt irgendwelche Wünſche? Möchten
                    Sie im {\pb}Inhalt verzeichnet ſein? –\pend
           \pstart
           – In ein paar Tagen beginnt die Drucklegung.\pend
           \pstart
           Auf Ihren erfreulichen Brief muſs ich Ihnen noch antworten. – Bitte baldige
                    Auskunft. – Haben Sie ſchon bemerkt, wie miſerabel die »\textcolor{green}{Agonie}{}\ledrightnote{\textcolor{green}{Agonie}}« iſt? – Gut iſt nur {\pb}\textcolor{green}{Frage an das Schickſal}{}\ledrightnote{\textcolor{green}{Die Frage an das Schicksal}} wie \textcolor{green}{Epiſode}{}\ledrightnote{\textcolor{green}{Episode}}.\pend
           \pstart
           Wie gehts Ihrem Stück? –\pend
           \pstart
           Meine \textcolor{green}{Novelle}{}\ledrightnote{→\textcolor{green}{Sterben. Novelle}} iſt in 2, 3
                    Tagen beendet – ich habe nemlich Zeit, während der Ordinationsſtunde zu
                    ſchreiben!\pend
           \pstart Ihr \spacefill\mbox{Arthur}\pend{}\endnumbering\briefempfaengerindex{Hofmannsthal, Hugo von@\textsc{Hofmannsthal, Hugo von}!zzzSchnitzler, Arthur@\emph{von Arthur Schnitzler}!1892-07-251@{{[}25.? 7. 1892{]}}|)be}\mylabel{h}  \normalsize

\doendnotes{C}
\bigskip
\vfill

\clearpage

\footnotesize

\lohead{\textsc{register}}

% Definiere theindex-Environment komplett neu ohne reledmac
\makeatletter
\renewenvironment{theindex}{%
  \section*{\indexname}%
  \setlength{\parindent}{0pt}%
  \setlength{\parskip}{0pt plus 0.3pt}%
  \let\item\@idxitem
}{%
  \clearpage
}
\makeatother

\IfFileExists{\jobname-pw.ind}{\input{\jobname-pw.ind}}{}

\end{document}

      