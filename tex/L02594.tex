%% latex-korrekturansicht-vorspann.tex
%% Vorspann für die Korrekturansicht.
%% Lädt die gemeinsame Datei latex-vorspann.tex mit gesetztem Schalter.

\newif\ifkorrekturansicht
\korrekturansichttrue

\input{../tex-inputs/latex-vorspann}


               \section[Marie Herzfeld an Arthur Schnitzler, 30. 3. 1913]{ Marie Herzfeld an Arthur Schnitzler, 30. 3. 1913}\nopagebreak\mylabel{v}\rehead{ }\normalsize\beginnumbering\briefempfaengerindex{Schnitzler, Arthur@\textsc{Schnitzler, Arthur}!zzzHerzfeld, Marie@\emph{von Marie Herzfeld}!1913-03-301@{30. 3. 1913}|(be} \toendnotes[C]{\smallbreak\pagebreak[2]} \Standort{DLA, A:Schnitzler, HS.1985.1.03436,5.}
\physDesc{Brief, 1 Blatt, 2 Seiten
\newline{}Handschrift: schwarze Tinte, lateinische Kurrent
\newline{}Schnitzler: 1) mit Bleistift Vermerk »\textsc{\uline{Herzfel{[}d{]}.}}« 2) mit rotem Buntstift Vermerk »\textsc{\textcolor{blue}{Seybel}, \textcolor{blue}{Barbi}}« und eine
                                 Unterstreichung}\toendnotes[C]{\smallbreak}\pstart
           \noindent{}\raggedleft{}{\pb}\textcolor{pink}{Wien II/\textsubscript{1}, Lichtenauerg. 5}{}\ledrightnote{\textcolor{pink}{Lichtenauergasse}}\pend
           \pstart
           30/III 1913\pend
           \pstart\center{}Geehrter Herr Doktor!\pend\pstart
           D\textsuperscript{r}{ }\textcolor{blue}{Georg von Seybel}{}\ledrightnote{\textcolor{blue}{Georg von Seybel}} hat eine \label{K_L02594-1v}\edtext{Adresse an die \textcolor{blue}{Barbi}{}\ledrightnote{\textcolor{blue}{Alice Barbi}}
               verfasst, um sie zu bitten, dass sie nach \textcolor{pink}{Wien}{}\ledrightnote{\textcolor{pink}{Wien}}
                  komme\strikeout{,} und als letzte im \textcolor{pink}{Bösendorfersal}{}\ledrightnote{\textcolor{pink}{Bösendorfer-Saal}}{ }\strikeout{zu} singe\strikeout{n}.}{\lemma{\textnormal{\emph{Adresse … singen.}}}\Cendnote{\textnormal{Am 2. 5. 1913 wurde der \textcolor{pink}{Bösendorfer-Saal} für immer geschlossen. Davor
                  sollten, nach Plan von \textcolor{blue}{Hugo Knepler}, dem
                  Inhaber der \emph{\textcolor{brown}{Konzertdirektion Gutmann}}, vier
                     »Abschiedskonzerte« stattfinden ([O. V.:] \emph{\textcolor{green}{Abschiedskonzerte im Bösendorfer-Saale}}. In:
                        \emph{\textcolor{green}{Fremden-Blatt}}, Jg. 67, Nr. 86,
                        30. 3. 1913, S. 10). Kurz vor der
                  Schließung wird von der hier angesprochenen »Adresse« berichtet und
                  dass die Sängerin \textcolor{blue}{Alice Barbi} diese Einladung
                  ablehnte ([O. V.:] \emph{\textcolor{green}{Abschiedskonzerte im
                        Bösendorfersaale}}. In: \emph{\textcolor{green}{Neue Wiener
                        Tagblatt}}, Jg. 47, Nr. 104,
                     17. 4. 1913, S. 16).}}}\label{K_L02594-1h} Warum diese
               Sache als Geheimnis behandelt wird, weiß ich nicht; Faktum ist, dass nur
               »Auserwählte« unterzeichnen sollen – und dass alles mit feierlicher {\pb}Langsamkeit vor sich geht –, da der \textcolor{blue}{Verf.}{}\ledrightnote{→\textcolor{blue}{Georg von Seybel}} des Schriftstückes verreist. Von morgen an wird die Adresse, die bisher \label{T_L02594-1v}\edtext{von Haus zu Haus}{\lemma{\textnormal{\emph{von Haus zu Haus}}}\Cendnote{\textnormal{sie schreibt: »zu Haus zu Haus«}}}\label{T_L02594-1h}
               getragen wurde, bei \textcolor{brown}{Gutmann}{}\ledrightnote{\textcolor{brown}{Gutmann (Konzertdirektion)}} zur Unterzeichnung
               aufliegen und da ich weiß, wie hoch die \textcolor{blue}{Barbi}{}\ledrightnote{\textcolor{blue}{Alice Barbi}} ihre
               Arbeiten schätzt und umgekehrt weiß, wieviel Genuss Sie ihr danken, so hoffe ich, Sie
               setzen Ihren Namen auf die Blätter. Ob die Adresse im \label{K_L02594-11v}\edtext{\textcolor{pink}{Opernhaus}{}\ledrightnote{\textcolor{pink}{Oper}} oder in der \textcolor{pink}{Schellingg.}{}\ledrightnote{\textcolor{pink}{Schellinggasse}}}{\lemma{\textnormal{\emph{Opernhaus … Schellingg.}}}\Cendnote{\textnormal{Die \emph{\textcolor{brown}{Konzertdirektion Gutmann}}
                        betrieb ein Kartenbüro in der \textcolor{pink}{Oper}, hatte aber ihren Haupsitz in der \textcolor{pink}{Schellinggasse}.}}}\label{K_L02594-11h} sein wird, lasse ich Ihnen morgen telephonieren.\pend
           \pstart
           Wärmstens {\\[\baselineskip]}\spacefill\mbox{Marie Herzfeld}\pend
           \leftskip=0em{}\endnumbering\briefempfaengerindex{Schnitzler, Arthur@\textsc{Schnitzler, Arthur}!zzzHerzfeld, Marie@\emph{von Marie Herzfeld}!1913-03-301@{30. 3. 1913}|)be}\mylabel{h}  \normalsize

\doendnotes{C}
\bigskip
\vfill

\clearpage

\footnotesize

\lohead{\textsc{register}}

% Definiere theindex-Environment komplett neu ohne reledmac
\makeatletter
\renewenvironment{theindex}{%
  \section*{\indexname}%
  \setlength{\parindent}{0pt}%
  \setlength{\parskip}{0pt plus 0.3pt}%
  \let\item\@idxitem
}{%
  \clearpage
}
\makeatother

\IfFileExists{\jobname-pw.ind}{\input{\jobname-pw.ind}}{}

\end{document}

      