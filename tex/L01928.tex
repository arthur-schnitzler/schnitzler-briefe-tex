%% latex-korrekturansicht-vorspann.tex
%% Vorspann für die Korrekturansicht.
%% Lädt die gemeinsame Datei latex-vorspann.tex mit gesetztem Schalter.

\newif\ifkorrekturansicht
\korrekturansichttrue

\input{../tex-inputs/latex-vorspann}


               \section[Hugo von Hofmannsthal an Arthur Schnitzler, {[}9. 5. 1910{]}]{ Hugo von Hofmannsthal an Arthur Schnitzler, {[}9. 5. 1910{]}}\nopagebreak\mylabel{v}\rehead{ }\normalsize\beginnumbering\briefempfaengerindex{Schnitzler, Arthur@\textsc{Schnitzler, Arthur}!zzzHofmannsthal, Hugo von@\emph{von Hugo von Hofmannsthal}!1910-05-091@{{[}9. 5. 1910{]}}|(be} \toendnotes[C]{\smallbreak\pagebreak[2]} \Standort{CUL, Schnitzler, B 43.}
\physDesc{Telegramm
\newline{}maschinell\newline{}Versand: Stempel des Telegrafenbeamten: »\textcolor{blue}{Sedlacek}« 
\newline{}Schnitzler: mit Bleistift datiert: »9/5 10« \newline{}Ordnung: mit Bleistift von unbekannter Hand nummeriert:
                                    »318« }\buchAbdrucke{\weitereDrucke{Hugo von Hofmannsthal, Arthur Schnitzler: \emph{Briefwechsel}. Hg. Therese Nickl und Heinrich Schnitzler. Frankfurt am Main: \emph{S. Fischer} 1964, S. 249.} }\toendnotes[C]{\smallbreak}\pstart
           {\pb}\textcolor{pink}{budapest}{}\ledrightnote{\textcolor{pink}{Budapest}} 51-786 22/21 9{ }5 50\pend
           \pstart
           verzweifelt ueber ungeschicklichkeit versuche telephonisch \label{K_L01928_1v}\edtext{ordnen}{\lemma{\textnormal{\emph{ordnen}}}\Cendnote{\textnormal{Es bezieht sich auf die bevorstehende Aufführung von \emph{\textcolor{green}{Cristinas Heimreise}} am 13. 5. 1910 in \textcolor{pink}{Wien}, bei der \textcolor{blue}{Hofmannsthal} zuerst keine (Frei-)Karten zur Disposition bekommen hatte.}}}\label{K_L01928_1h} habe selber anscheinend keinen platz im \textcolor{pink}{haus}{}\ledrightnote{→\textcolor{pink}{Theater an der Wien}}\pend
           \pstart herzlichstes =\spacefill\mbox{hugo .+=}\pend{}\endnumbering\briefempfaengerindex{Schnitzler, Arthur@\textsc{Schnitzler, Arthur}!zzzHofmannsthal, Hugo von@\emph{von Hugo von Hofmannsthal}!1910-05-091@{{[}9. 5. 1910{]}}|)be}\mylabel{h}  \normalsize

\doendnotes{C}
\bigskip
\vfill

\clearpage

\footnotesize

\lohead{\textsc{register}}

% Definiere theindex-Environment komplett neu ohne reledmac
\makeatletter
\renewenvironment{theindex}{%
  \section*{\indexname}%
  \setlength{\parindent}{0pt}%
  \setlength{\parskip}{0pt plus 0.3pt}%
  \let\item\@idxitem
}{%
  \clearpage
}
\makeatother

\IfFileExists{\jobname-pw.ind}{\input{\jobname-pw.ind}}{}

\end{document}

      