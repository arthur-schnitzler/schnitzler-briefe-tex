%% latex-korrekturansicht-vorspann.tex
%% Vorspann für die Korrekturansicht.
%% Lädt die gemeinsame Datei latex-vorspann.tex mit gesetztem Schalter.

\newif\ifkorrekturansicht
\korrekturansichttrue

\input{../tex-inputs/latex-vorspann}


               \section[Karl Kraus an Arthur Schnitzler, 21. 7. 1893]{ Karl Kraus an Arthur Schnitzler, 21. 7. 1893}\nopagebreak\mylabel{v}\rehead{ }\normalsize\beginnumbering\briefempfaengerindex{Schnitzler, Arthur@\textsc{Schnitzler, Arthur}!zzzKraus, Karl@\emph{von Karl Kraus}!1893-07-211@{21. 7. 1893}|(be} \toendnotes[C]{\smallbreak\pagebreak[2]} \Standort{CUL, Schnitzler, B 55.}
\physDesc{Brief, 1 Blatt, 4 Seiten
\newline{}Handschrift: schwarze Tinte, deutsche Kurrent\newline{}Beilage: Manuskript auf dem gleichen Briefpapier, 1 Blatt, 1 Seite,
                                 schwarze Tinte }\buchAbdrucke{\weitereDrucke{\emph{Karl Kraus und Arthur Schnitzler. Eine Dokumentation.} Hg. Reinhard Urbach. In: \emph{Literatur und Kritik}, Bd. 49, Oktober 1970, S. 518–519.} }\toendnotes[C]{\smallbreak}\pstart{}{\pb}Schnitzler\pend{}{\bigskip}\pstart
           \noindent{}{\pb}\textcolor{gray}{\textbf{KARL KRAUS}}\hfill \substVorne{}\textsuperscript{\textcolor{gray}{\textbf{\textcolor{pink}{Wien I., Maximilianstrasse 13}{}\ledrightnote{\textcolor{pink}{Mahlerstraße}}.}}}{\allowbreak}\substDazwischen{}\textcolor{pink}{Ischl}{}\ledrightnote{\textcolor{pink}{Bad Ischl}}\substHinten{}{ }21. Juli, \textcolor{gray}{\textbf{189}}3\pend
           \pstart{}Mein liebſter, verehrter Herr Doctor!\pend\pstart
           Daſs Sie ſo »ſpurlos« ſich auch dem Staube gemacht haben, thut mir ſehr leid. Seit
               Ihrer Vorſtellung haben wir uns ja gar nicht gesprochen.\pend
           \pstart
           »Sieh’ſt du, \uline{das}{ }\uuline{\edtext{hätt}{\Cendnote{siebenfach unterstrichen}}}’ (!!!!) ich dir \introOben{}doch\introOben{}
               nicht geſagt!« – ich werde dieſen genialen Zug in Frl. \textcolor{blue}{Falkner}{}\ledrightnote{\textcolor{blue}{Julie Falkner}}’s Darſtellung nie vergeſſen. Und darauf noch dröhnender
               Abgangsapplaus, der \strikeout{d} auch die \uline{zweite}{ }Schluſspointe (»\textcolor{green}{Es iſt ja leicht gegangen etc}{}\ledrightnote{→\textcolor{green}{Abschiedssouper}}«) unmöglich machte! Von dem
               »Bordellſtück« »\textcolor{green}{Abſchiedsouper}{}\ledrightnote{\textcolor{green}{Abschiedssouper}}« wird hier viel
               geſprochen.\pend
           \pstart
           Meine herzlichſte Gratulation zur \label{K_L00239_1v}\edtext{\textcolor{green}{Kritik}{}\ledrightnote{→\textcolor{green}{Aus Ischl, 14. Juli, schreibt man uns: …}}}{\lemma{\textnormal{\emph{Kritik}}}\Cendnote{\textnormal{[O. V.:] \emph{\textcolor{green}{[Aus Ischl, 14. Juli, schreibt man
                        uns]}}. In: \emph{\textcolor{green}{Neue Freie Presse}},
                     Nr. 10.381, 18. 7. 1893, S. 6.}}}\label{K_L00239_1h} in \textcolor{green}{N. Fr. Preſſe}{}\ledrightnote{\textcolor{green}{Neue Freie Presse}} (und \label{K_L00239_2v}\edtext{\textcolor{green}{\textcolor{blue}{Bauer}{}\ledrightnote{\textcolor{blue}{Julius Bauer}}}{}\ledrightnote{→\textcolor{green}{[Abschiedsouper in Ischl]}}}{\lemma{\textnormal{\emph{Bauer}}}\Cendnote{\textnormal{[O. V. = \textcolor{blue}{Julius Bauer}:] \emph{\textcolor{green}{[Abschiedssouper in Ischl]}}. In: \emph{\textcolor{green}{Illustrirtes Wiener Extrablatt}}, Jg. 22, Nr. 196,
                        18. 7. 1893, S. 5.}}}\label{K_L00239_2h} im \textcolor{brown}{Extrablatt}{}\ledrightnote{\textcolor{brown}{Illustrirtes Wiener Extrablatt}})! Sehr dämlich hat ſich Herr \label{K_L00239_3v}\edtext{\textcolor{blue}{Skrein}{}\ledrightnote{\textcolor{blue}{Stefan Skrein}}}{\lemma{\textnormal{\emph{Skrein}}}\Cendnote{\textnormal{\textcolor{blue}{Stefan}: \emph{\textcolor{green}{Ischler Brief}}. In: \emph{\textcolor{green}{Wiener Allgemeine
                        Zeitung}}, Jg. 14, Nr. 4593, 18. 7. 1893,
                  S. 2.}}}\label{K_L00239_3h} in der »\textcolor{brown}{Allgemeinen}{}\ledrightnote{\textcolor{brown}{Wiener Allgemeine Zeitung}}«
                  \textcolor{green}{geäußert}{}\ledrightnote{→\textcolor{green}{Ischler Brief}}.\pend
           \pstart
           Dies mal haben \textcolor{brown}{N. Fr. Pr.}{}\ledrightnote{\textcolor{brown}{Neue Freie Presse}} u. \textcolor{brown}{Allgemeine}{}\ledrightnote{\textcolor{brown}{Wiener Allgemeine Zeitung}} die Rollen getauſcht.\pend
           \pstart
           {\pb}Ich habe eine \textcolor{green}{Notiz}{}\ledrightnote{→\textcolor{green}{Ischler Sommertheater}} an das \textcolor{brown}{Wiener Tagblatt}{}\ledrightnote{\textcolor{brown}{Wiener Tagblatt}}{ }\label{K_L00239_4v}\edtext{geſchickt}{\lemma{\textnormal{\emph{geſchickt}}}\Cendnote{\textnormal{[O. V.:] \emph{\textcolor{green}{Ischler Sommertheater}}. In: \emph{\textcolor{green}{Wiener Abendblatt}}, Jg. 29, Nr. 199,
                        21. 7. 1893, S. 4.}}}\label{K_L00239_4h}; \uline{hoffentlich}{ }\uline{wird} (oder, wenn Sie dieſen Brief erhalten) \uline{wurde} es gedruckt. Im \textcolor{brown}{\uline{Magazin}}{}\ledrightnote{\textcolor{brown}{Magazin für die Literatur des Auslandes}} wird nichts erſcheinen. Allerdings bin ich nicht ſchuld. Damit Sie meinen guten
               Willen ſehen, ſende ich Ihnen beiliegend meine \substVorne{}\textsuperscript{Kritik}{\allowbreak}\substDazwischen{}Notiz\substHinten{}, die mir heute \textcolor{blue}{Neumann-Hofer}{}\ledrightnote{\textcolor{blue}{Gilbert Otto Neumann-Hofer}}
               zurückſandte – mit der Bemerkung:\pend
           \pstart
           »Eine Vorſtellung in \textcolor{pink}{Ischl}{}\ledrightnote{\textcolor{pink}{Bad Ischl}} kann in einem
               Wochenblatte nicht beſprochen werden. Solche gelegentlichen Ereigniſſe ſind auf die
               Notiznahme ſeitens der Tagesblätter beſchränkt.« Na, alſo! –\pend
           \pstart
           \textcolor{blue}{Devrient}{}\ledrightnote{\textcolor{blue}{Max Devrient}}’s Vorleſung war famos: namentlich \textcolor{blue}{Fontane}{}\ledrightnote{\textcolor{blue}{Theodor Fontane}}.\pend
           \pstart
            Ich habe ihm gleich nach unſerer ſeinerzeit. Unterredung nach \textcolor{pink}{Wien}{}\ledrightnote{\textcolor{pink}{Wien}} geſchrieben, er ſolle {\pb}\textcolor{blue}{Liliencron}{}\ledrightnote{\textcolor{blue}{Detlev von Liliencron}} leſen. Nun hat er mich – ſelbſt
               aufgeſucht. Liebenswürdig, was? Wie gedruckt; \textcolor{blue}{Liliencron}{}\ledrightnote{\textcolor{blue}{Detlev von Liliencron}}, den er ſich gleich kaufte, hat ihn \uline{entzückt} u. er wird ihn beſtimmt in \textcolor{pink}{\uline{Wien}}{}\ledrightnote{\textcolor{pink}{Wien}} vorleſen. Er fragte mich auch, ob ich Gedichte von \uuline{Ihnen} hätte; er wollte ſie nämlich in \textcolor{pink}{Marienbad}{}\ledrightnote{\textcolor{pink}{Marienbad}}, wohin er ſich noch am Tage des Beſuches begab, vorleſen. Da nun
               aber die Vorleſung gleich auf den nächſten Tag angeſetzt war, lehnte er auch eine
               eventuelles Telegramm an Sie (zu dem ich mich bereit erklärte) ab. Aber im
                  Winter will er’s nachholen.\pend
           \pstart
           Leben Sie wohl, bitte beſte Grüße an \textcolor{blue}{Loris}{}\ledrightnote{\textcolor{blue}{Hugo von Hofmannsthal}} u \textcolor{blue}{Salten}{}\ledrightnote{\textcolor{blue}{Felix Salten}} auszurichten!\pend
           \pstart
           Herzlichſst Ihr ſehr ergebener{\\[\baselineskip]}\spacefill\mbox{KarlKraus}\pend
           \leftskip=0em{}\pstart
           \noindent{}N.B. Was ſagen Sie zur »\textcolor{brown}{Freien Bühne}{}\ledrightnote{\textcolor{brown}{»Freie Bühne« Verein für moderne Literatur}}« in \textcolor{pink}{Wien}{}\ledrightnote{\textcolor{pink}{Wien}}, die – \textcolor{blue}{Elbogen}{}\ledrightnote{\textcolor{blue}{Friedrich Elbogen}} aufführt. Ist das nicht zum Todtlachen? Die Veranstalter ſind
                  Revolverjournalisten.\pend
           \pstart
           \noindent{}{\pb}\textcolor{gray}{\textbf{KARL KRAUS}}\hfill \substVorne{}\textsuperscript{\textcolor{gray}{\textbf{\textcolor{pink}{Wien I., Maximilianstrasse 13}{}\ledrightnote{\textcolor{pink}{Mahlerstraße}}.}}}{\allowbreak}\substDazwischen{}\textcolor{pink}{Ischl}{}\ledrightnote{\textcolor{pink}{Bad Ischl}}\substHinten{}{ }15. VII \textcolor{gray}{\textbf{189}}3\pend
           \pstart
           \uline{Arthur Schnitzlers} einaktige Komödie »\textcolor{green}{Abſchiedssouper}{}\ledrightnote{\textcolor{green}{Abschiedssouper}}« fand im \textcolor{pink}{Ischler Stadttheater}{}\ledrightnote{\textcolor{pink}{Stadttheater (Bad Ischl)}} ihre Probeaufführung. Das kleine \textcolor{pink}{oberöſterreichiſche}{}\ledrightnote{\textcolor{pink}{Oberösterreich}} Curorttheater iſt die erſte
               Bühne, die ſich des prächtigen Stückleins angenommen hat.\pend
           \pstart
           Der überaus lebendige, geiſtreiche Einakter, der eine geradezu bravouröſe Technik
               aufweist, iſt die wirkſamſte der ſieben »\textcolor{green}{Anatol}{}\ledrightnote{\textcolor{green}{Anatol}}«studien (siehe \label{K_L00239_5v}\edtext{\textcolor{green}{Besprechung}{}\ledrightnote{→\textcolor{green}{Wiener Dichter}}}{\lemma{\textnormal{\emph{Besprechung}}}\Cendnote{\textnormal{[O. V.:] \emph{\textcolor{green}{Arthur Schnitzler}}. In: \emph{\textcolor{green}{Das Magazin für Litteratur}}, Jg. 62, Nr. 18,
                        6. 5. 1893, S. 294.}}}\label{K_L00239_5h} in N\textsuperscript{r.} 18) und fand den lebhafteſten Beifall, den nur einige »verſchämte«, in
               ihren heiligſten Gefühlen verletzte Curgäſte im Intereſſe der \substVorne{}\textsuperscript{publiken und privaten}{\allowbreak}\substDazwischen{}privaten und publiken\substHinten{}{ }Sicherheit abwehren zu müſſen glaubten. Geſpielt
               wurde recht brav; namentlich zeichnete ſich der treffliche \textcolor{blue}{\uline{Jarno}}{}\ledrightnote{\textcolor{blue}{Josef Jarno}} vom \textcolor{pink}{berliner Reſidenztheater}{}\ledrightnote{\textcolor{pink}{Wallnertheater}} als \textcolor{green}{Max}{}\ledrightnote{→\textcolor{green}{Anatol}} aus. Die famoſe Schluſspointe
               gieng leider wirkungslos, weil unverſtanden, vorüber. –\pend
           \pstart
           Arthur Schnitzler, neben \textcolor{blue}{Loris}{}\ledrightnote{\textcolor{blue}{Richard Beer-Hofmann}} der talentvollſte
               unter den wenigen talentierten \textcolor{pink}{Wien}{}\ledrightnote{\textcolor{pink}{Wien}}ern, \strikeout{muſste} hat an dieſem Abend die Concurrenz – der Herren
                  \textcolor{blue}{Moſer}{}\ledrightnote{\textcolor{blue}{Gustav von Moser}}{ }{\kaufmannsund}{ }\textcolor{blue}{Miſch}{}\ledrightnote{\textcolor{blue}{Robert Misch}} aushalten müſſen, deren \introOben{}dreiaktiger\introOben{}{ }Schwank »\textcolor{green}{Fräulein
                  Frau}{}\ledrightnote{\textcolor{green}{Fräulein Frau}}« gegeben wurde. Nach dem grobkörnigen Schablonenmachwerk das graziöſe
               Kunſtwerkchen! Das war denn nun ein beſchämend leichter Sieg für Arthur Schnitzler.
               Daſs ſich gleichwohl die beiden \textcolor{blue}{Schwankherren}{}\ledrightnote{→\textcolor{blue}{Gustav von Moser}{\newline}→\textcolor{blue}{Robert Misch}} mit ihrem »\textcolor{green}{Fräulein Frau}{}\ledrightnote{\textcolor{green}{Fräulein Frau}}« die Bühnen früher erobert haben als Schnitzler, der ja doch
               zu den böſen Modernen i. e. »Unſittlichen« gehört, mit irgend einem ſeiner Werke, iſt
               bei der Einſichtsloſigkeit unſerer Bühnenleiter begreiflich. \spacefill\mbox{(K.K.)}\pend
           \endnumbering\briefempfaengerindex{Schnitzler, Arthur@\textsc{Schnitzler, Arthur}!zzzKraus, Karl@\emph{von Karl Kraus}!1893-07-211@{21. 7. 1893}|)be}\mylabel{h}  \normalsize

\doendnotes{C}
\bigskip
\vfill

\clearpage

\footnotesize

\lohead{\textsc{register}}

% Definiere theindex-Environment komplett neu ohne reledmac
\makeatletter
\renewenvironment{theindex}{%
  \section*{\indexname}%
  \setlength{\parindent}{0pt}%
  \setlength{\parskip}{0pt plus 0.3pt}%
  \let\item\@idxitem
}{%
  \clearpage
}
\makeatother

\IfFileExists{\jobname-pw.ind}{\input{\jobname-pw.ind}}{}

\end{document}

      