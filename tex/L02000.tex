%% latex-korrekturansicht-vorspann.tex
%% Vorspann für die Korrekturansicht.
%% Lädt die gemeinsame Datei latex-vorspann.tex mit gesetztem Schalter.

\newif\ifkorrekturansicht
\korrekturansichttrue

\input{../tex-inputs/latex-vorspann}


               \section[Arthur Schnitzler an Georg Brandes, 19. 1. 1911]{ Arthur Schnitzler an Georg Brandes, 19. 1. 1911}\nopagebreak\mylabel{v}\rehead{ }\normalsize\beginnumbering\briefempfaengerindex{Brandes, Georg@\textsc{Brandes, Georg}!zzzSchnitzler, Arthur@\emph{von Arthur Schnitzler}!1911-01-191@{19. 1. 1911}|(be} \toendnotes[C]{\smallbreak\pagebreak[2]} \Standort{Kopenhagen, Det Kongelige Bibliotek, Georg Brandes Arkiv, box 125.}
\physDesc{Brief, 3 Blätter (Zählung der Blätter 2 und 3), 5 Seiten
\newline{}Handschrift: schwarze Tinte, lateinische Kurrent\newline{}Ordnung: 1) auf der letzten Seite von unbekannter Hand
                                    mit Bleistift geschrieben: »\textsc{Schnitzler}« 2) mit Bleistift von unbekannter Hand
                                    nummeriert »30.«, Blatt 2 und 3 mit Datum »19/1 11« versehen und zwei Unterstreichungen}\buchAbdrucke{\weitereDrucke{1) Georg Brandes, Arthur Schnitzler: \emph{Ein Briefwechsel}. Hg. Kurt Bergel. Bern: \emph{Francke} 1956, S. 99–100.} \weitereDrucke{2) Arthur Schnitzler: \emph{Briefe 1875–1912}. Hg. Therese Nickl und Heinrich Schnitzler. Frankfurt am Main: \emph{S. Fischer} 1981, S. 649–651.} }\toendnotes[C]{\smallbreak}\pstart
           \raggedleft{}{\pb}\textcolor{pink}{XVIII. Sternwartestr 71}{}\ledrightnote{\textcolor{pink}{Sternwartestraße}}{\\}\textcolor{pink}{Wien}{}\ledrightnote{\textcolor{pink}{Wien}}, 19. I. 911. \pend
           \pstart{}Verehrter Herr Brandes,\pend\pstart
           \noindent{}mit Ihrem Brief über den \textcolor{green}{Medardus}{}\ledrightnote{\textcolor{green}{Der junge Medardus. Dramatische Historie in einem Vorspiel und fünf Aufzügen}} hab ich mich
                    sehr gefreut. Der Erfolg hier dauert an; das \textcolor{pink}{Burgtheater}{}\ledrightnote{\textcolor{pink}{Burgtheater}} hatte seit Jahren nicht eine solche Reihe von
                    ausverkauften Häusern; übrigens ist es eine vortreffliche Aufführung, und es
                    wäre mir eine wirkliche Genugthuung we{\geminationn}
               Sie sie
                    einmal sehen könnten. Natürlich ist unendlich viel gestrichen; darunter Scenen
                    von bedeutender Wichtigkeit – und ich selbst war der Streicher; von der alten
                    Theatererfahrung ausgehend, daß das Publikum gegen Längen empfindlicher ist als
                    gegen Lücken. Ich hatte das Stück geschrieben, ohne die Eventualität einer
                    Aufführung überhaupt in Betracht zu ziehen, ließ meine Phantasie und meine Feder
                    laufen, wie es ihnen beliebte, {\pb}hatte aber
                    natürlich immer die lebendigen Bühnenbilder vor mir, ohne recht zu glauben, daß
                    es mir vergö{\geminationn}t sein würde, sie je in Wirklichkeit
                    zu erblicken. Schon \textcolor{blue}{Schlenther}{}\ledrightnote{\textcolor{blue}{Paul Schlenther}} nahm das \textcolor{green}{Stück}{}\ledrightnote{→\textcolor{green}{Der junge Medardus. Dramatische Historie in einem Vorspiel und fünf Aufzügen}} an, konnte sich aber,
                    in bekannter Weise nicht entschliessen, seine Absicht zur That zu machen; erst
                    dem Baron \textcolor{blue}{Berger}{}\ledrightnote{\textcolor{blue}{Alfred von Berger}} verdankt \strikeout{ich} das \textcolor{green}{Stück}{}\ledrightnote{→\textcolor{green}{Der junge Medardus. Dramatische Historie in einem Vorspiel und fünf Aufzügen}}
               sein Erwachen zum Bühnenleben. Seither ist
                    schon manches andre fertig geworden und Sie, verehrter Freund, der allen meinen
                    Arbeiten mit so wohlthuendem Interesse entgegenkommt, werden natürlich auch in
                    den neuen und neuesten Fällen die Consequenzen zu tragen haben. –\pend
           \pstart
           Denken Sie nicht dran, nach langer Zeit endlich wieder nach \textcolor{pink}{Wien}{}\ledrightnote{\textcolor{pink}{Wien}} zu kommen? Wie gern möchte ich mit Ihnen reden, Sie
                    in meinem Hause begrüßen – »Mein Haus« sag ich, denn im vergangenen {\pb}Sommer hab ich von Frau \textcolor{blue}{Bleibtreu}{}\ledrightnote{\textcolor{blue}{Hedwig Bleibtreu}}, der Wittwe des Schauspielers \textcolor{blue}{Römpler}{}\ledrightnote{\textcolor{blue}{Alexander Römpler}} – (sie spielt die Frau \textcolor{green}{Klaehr}{}\ledrightnote{→\textcolor{green}{Der junge Medardus. Dramatische Historie in einem Vorspiel und fünf Aufzügen}} im \textcolor{green}{Medardus}{}\ledrightnote{\textcolor{green}{Der junge Medardus. Dramatische Historie in einem Vorspiel und fünf Aufzügen}}), eine kleine Villa im \textcolor{pink}{Cottage}{}\ledrightnote{\textcolor{pink}{Währinger Cottage}} gekauft die ich mit \textcolor{blue}{Frau}{}\ledrightnote{→\textcolor{blue}{Olga Schnitzler}} und \textcolor{blue}{Kindern}{}\ledrightnote{→\textcolor{blue}{Heinrich Schnitzler}{\newline}→\textcolor{blue}{Lili Schnitzler}} – (den \textcolor{blue}{Buben}{}\ledrightnote{→\textcolor{blue}{Heinrich Schnitzler}}, der jetzt 8 Jahre ist, kennen Sie von \textcolor{pink}{Marienlyst}{}\ledrightnote{\textcolor{pink}{Marienlyst}} her, das \textcolor{blue}{Mädchen}{}\ledrightnote{→\textcolor{blue}{Lili Schnitzler}} ist kaum anderthalb Jahre alt)
                    bewohne.\pend
           \pstart
           So darf ich \introOben{}mich\introOben{} mancher inneren wie äußeren Erfolge
                    erfreuen, und empfinde das viele Gute, das mir vom Schicksal beschieden,
                    zuweilen so stark, daß ich jenes stetig fortschreitende Ohrenleiden, von dem ich
                    seit 15 Jahren geplagt bin, gern als einen Polykratesring ansehen möchte – we{\geminationn}{ }\strikeout{\textcolor{gray}{ich}} auch als einen allzu werthvollen – und jedenfalls als einen, den kein
                    Fischer der Welt mir jemals zurückbringen wird. –\pend
           \pstart
           {\pb}\textcolor{blue}{Beer Hofmann}{}\ledrightnote{\textcolor{blue}{Richard Beer-Hofmann}} mit seiner \textcolor{blue}{Frau}{}\ledrightnote{→\textcolor{blue}{Paula Beer-Hofmann}} und seinen drei \textcolor{blue}{Kindern}{}\ledrightnote{→\textcolor{blue}{Mirjam Beer-Hofmann}{\newline}→\textcolor{blue}{Gabriel Beer-Hofmann}{\newline}→\textcolor{blue}{Naëmah Beer-Hofmann}} wohnt ganz nahe von
                    mir, in einem sehr schönen \textcolor{pink}{Haus}{}\ledrightnote{→\textcolor{pink}{Hasenauerstraße}}, das ihm der Architekt \textcolor{blue}{Josef
                        Hoffmann}{}\ledrightnote{\textcolor{blue}{Josef Hoffmann}} gebaut hat, und arbeitet nicht so viel, als er seinem Talent
                    nach verpflichtet oder verurtheilt wäre. Sie sollten wieder einmal herkommen, –
                    womöglich im Mai – man könnte einander so vieles erzählen; – in einer Stunde
                    etwa zehn Mal so viel, als in zwei Briefen steht; das beste, was man von
                    Menschen hat, die einem werth sind, bleiben doch die zwanglosen Unterhaltungen,
                    die von der ganzen Atmosphäre der Persönlichkeit umgeben sind – was ist dagegen
                    die gewollte Condensation und Praecision eines noch so herzlich intendirten
                    Schreibens? {\pb}In Briefen will man was besti{\geminationm}tes sagen; – man dankt, man berichtet – man
                    bezweckt; – in Gesprächen läßt man sich und den andern viel reiner leben, – man
                    mag mit hundert Geheimnissen voneinander scheiden; – die Stimme, der Tonfall,
                    die Geste geben selbst Befangenheiten, ja Unaufrichtigkeiten (die zwischen uns
                    nicht zu befürchten sind) jene beste und einzige Wahrheit, an der wir uns erl\substVorne{}\textsuperscript{\textcolor{gray}{e}}\substDazwischen{}a\substHinten{}ben dürfen: Gegenwart.\pend
           \pstart
           Dies soll Sie natürlich nur bestimmen (o welche Kraft traue ich schiefen
                    Aphorismen zu!) nach \textcolor{pink}{Wien}{}\ledrightnote{\textcolor{pink}{Wien}} zu reisen – aber Sie
                    ja nicht abhalten, mich bald wieder durch ein paar geschriebene Worte zu
                    erfreuen. In herzlicher Verehrung\pend
           \pstart
           Ihr{\\[\baselineskip]}\spacefill\mbox{Arthur Schnitzler}\pend
           \leftskip=0em{}\endnumbering\briefempfaengerindex{Brandes, Georg@\textsc{Brandes, Georg}!zzzSchnitzler, Arthur@\emph{von Arthur Schnitzler}!1911-01-191@{19. 1. 1911}|)be}\mylabel{h}  \normalsize

\doendnotes{C}
\bigskip
\vfill

\clearpage

\footnotesize

\lohead{\textsc{register}}

% Definiere theindex-Environment komplett neu ohne reledmac
\makeatletter
\renewenvironment{theindex}{%
  \section*{\indexname}%
  \setlength{\parindent}{0pt}%
  \setlength{\parskip}{0pt plus 0.3pt}%
  \let\item\@idxitem
}{%
  \clearpage
}
\makeatother

\IfFileExists{\jobname-pw.ind}{\input{\jobname-pw.ind}}{}

\end{document}

      