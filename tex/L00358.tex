%% latex-korrekturansicht-vorspann.tex
%% Vorspann für die Korrekturansicht.
%% Lädt die gemeinsame Datei latex-vorspann.tex mit gesetztem Schalter.

\newif\ifkorrekturansicht
\korrekturansichttrue

\input{../tex-inputs/latex-vorspann}


               \section[Hermann Bahr an Arthur Schnitzler, 29. 7. 1894]{ Hermann Bahr an Arthur Schnitzler, 29. 7. 1894}\nopagebreak\mylabel{v}\rehead{ }\normalsize\beginnumbering\briefempfaengerindex{Schnitzler, Arthur@\textsc{Schnitzler, Arthur}!zzzBahr, Hermann@\emph{von Hermann Bahr}!1894-07-291@{29. 7. 1894}|(be} \toendnotes[C]{\smallbreak\pagebreak[2]} \Standort{CUL, Schnitzler, B 5b.}
\physDesc{Postkarte
\newline{}Handschrift: schwarze Tinte, deutsche Kurrent\newline{}Versand: Stempel: »\nobreak{}\oindex{VIII., Josefstadt@\textbf{VIII., Josefstadt}, \emph{Bezirk (A.BZK)}|pwk}Wien 8/1, 29 7 94, 3–4 N\nobreak{}«.  \newline{}Ordnung: 1) mit rotem Buntstift von unbekannter Hand nummeriert: »25« 2) mit Bleistift von unbekannter Hand nummeriert: »25«}\buchAbdrucke{\weitereDrucke{Hermann Bahr, Arthur Schnitzler: \emph{Briefwechsel, Aufzeichnungen, Dokumente (1891–1931)}. Hg. Kurt Ifkovits und Martin Anton Müller. Göttingen: \emph{Wallstein} 2018, S. 77.} }\toendnotes[C]{\smallbreak}\pstart{}{\pb}Herrn \textsc{D\textsuperscript{r} Arthur Schnitzler}\pend{}\pstart{}Schriftsteller\pend{}\pstart{}\textcolor{pink}{Wien IX}{}\ledrightnote{\textcolor{pink}{IX., Alsergrund}}\pend{}\pstart{}\textcolor{pink}{\textsc{Franckgasse 1}}{}\ledrightnote{\textcolor{pink}{Frankgasse}}\pend{}{\bigskip}\pstart
           \noindent{}{\pb}Mein Telephon iſt \label{K_L00358_1v}\edtext{6415}{\lemma{\textnormal{\emph{6415}}}\Cendnote{\textnormal{Die Nummer der
                  Redaktion der \emph{\textcolor{brown}{Zeit}}. Privat war er am
                     8. 5. 1894 in die \textcolor{pink}{Lammgasse 3} umgezogen. Hier weist ihn das Adressverzeichnis
                     \emph{\textcolor{green}{Lehmann}}{ }1895 ebenfalls als »Telephonabonnent« aus, Nr. 4531.}}}\label{K_L00358_1h}.\pend
           \pstart
           Herzlichſt{\\[\baselineskip]}\spacefill\mbox{Bahr}\pend
           \leftskip=0em{}\pstart
           \noindent{}\label{LL171-1v}\textcolor{blue}{D.}{}\ledrightnote{\textcolor{blue}{Adele Sandrock}}{ }ſchreibt mir heute, daß ſie am 5.
                     »auf zwei Minuten« nach \textcolor{pink}{Wien}{}\ledrightnote{\textcolor{pink}{Wien}} kommt.\label{LL171-1h}\pend
           \endnumbering\briefempfaengerindex{Schnitzler, Arthur@\textsc{Schnitzler, Arthur}!zzzBahr, Hermann@\emph{von Hermann Bahr}!1894-07-291@{29. 7. 1894}|)be}\mylabel{h}  \normalsize

\doendnotes{C}
\bigskip
\vfill

\clearpage

\footnotesize

\lohead{\textsc{register}}

% Definiere theindex-Environment komplett neu ohne reledmac
\makeatletter
\renewenvironment{theindex}{%
  \section*{\indexname}%
  \setlength{\parindent}{0pt}%
  \setlength{\parskip}{0pt plus 0.3pt}%
  \let\item\@idxitem
}{%
  \clearpage
}
\makeatother

\IfFileExists{\jobname-pw.ind}{\input{\jobname-pw.ind}}{}

\end{document}

      