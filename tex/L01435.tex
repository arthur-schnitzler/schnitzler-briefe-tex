%% latex-korrekturansicht-vorspann.tex
%% Vorspann für die Korrekturansicht.
%% Lädt die gemeinsame Datei latex-vorspann.tex mit gesetztem Schalter.

\newif\ifkorrekturansicht
\korrekturansichttrue

\input{../tex-inputs/latex-vorspann}


               \section[Arthur Schnitzler an Richard Beer-Hofmann, 3. 9. 1904]{ Arthur Schnitzler an Richard Beer-Hofmann, 3. 9. 1904}\nopagebreak\mylabel{v}\rehead{ }\normalsize\beginnumbering\briefempfaengerindex{Beer-Hofmann, Richard@\textsc{Beer-Hofmann, Richard}!zzzSchnitzler, Arthur@\emph{von Arthur Schnitzler}!1904-09-032@{3. 9. 1904}|(be} \toendnotes[C]{\smallbreak\pagebreak[2]} \Standort{YCGL, MSS 31.}
\physDesc{Brief, 1 Blatt, 2 Seiten, Umschlag
\newline{}Handschrift: Bleistift, deutsche Kurrent\newline{}Versand: 1) Stempel: »\nobreak{}\oindex{Bad Ischl@\textbf{Bad Ischl}, \emph{Besiedelter Ort (A.BSO)}|pwk}Ischl, \textcolor{gray}{4}. 9. 04, 6–8V\nobreak{}«.  2) Stempel: »\nobreak{}\oindex{Bad Aussee@\textbf{Bad Aussee}, \emph{Besiedelter Ort (A.BSO)}|pwk}{\pb}Aussee in
                              Steiermark, \textcolor{gray}{5} 9 04\nobreak{}«. }\buchAbdrucke{\weitereDrucke{Arthur Schnitzler, Richard Beer-Hofmann: \emph{Briefwechsel 1891–1931}. Hg. Konstanze Fliedl. Wien, Zürich: \emph{Europaverlag} 1992, S. 166.} }\pstart{}{\pb}\textcolor{gray}{\textbf{\textcolor{pink}{Hôtel Kaiserkrone, Bad Ischl}{}\ledrightnote{\textcolor{pink}{Hotel Kaiserkrone}}.}}\pend{}\pstart{}\textcolor{blue}{\textcolor{gray}{\textbf{J. G. Haager jun.}}}{}\ledrightnote{\textcolor{blue}{Johann Georg Haager}}\pend{}{\bigskip}\pstart{}\textsc{Herrn Dr Richard }\pend{}\pstart{}\textsc{Beer-Hofmann}\pend{}\pstart{}\textsc{\textcolor{pink}{Markt Aussee}{}\ledrightnote{\textcolor{pink}{Bad Aussee}}}\pend{}\pstart{}\textcolor{pink}{\textsc{Villa Frühling}}{}\ledrightnote{\textcolor{pink}{Villa Frühling}}\pend{}{\bigskip}\pstart
           \noindent{}{\pb}\textcolor{gray}{\textbf{\textbf{\textcolor{pink}{Hotel Kaiserkrone}{}\ledrightnote{\textcolor{pink}{Hotel Kaiserkrone}}}}}\pend
           \pstart
           \centering{}\textcolor{gray}{\textbf{\textbf{\textcolor{pink}{Bad Ischl}{}\ledrightnote{\textcolor{pink}{Bad Ischl}}}}}\pend
           \pstart
           \noindent{}\textcolor{gray}{\textbf{\textbf{Centrale Lage}}}\pend
           \pstart
           \textcolor{gray}{\textbf{mit schattigem Restaurationsgarten an dem Ischlflusse gegenüber der \textcolor{pink}{Kaiserlichen Villa}{}\ledrightnote{\textcolor{pink}{Kaiservilla}}}}. \pend
           \pstart
           \textcolor{gray}{\textbf{Lese-Salon}}\hfill \textcolor{gray}{\textbf{\textcolor{pink}{Bad Ischl}{}\ledrightnote{\textcolor{pink}{Bad Ischl}}, am}}{ }3. 9. 904\pend
           \pstart
           \textbf{\textcolor{gray}{\textbf{Badezimmer}}}\pend
           \pstart
           \textbf{\textcolor{gray}{\textbf{Telephon}}}\pend
           \pstart
           \textbf{\textcolor{gray}{\textbf{Omnibus am Bahnhofe.}}}\pend
           \pstart
           \textcolor{gray}{\textbf{Bier vom Fass aus \textcolor{brown}{\textbf{A. Dreher’s Brauerei}}{}\ledrightnote{\textcolor{brown}{Anton Drehers Brauereien}} in \textcolor{pink}{Schwechat}{}\ledrightnote{\textcolor{pink}{Schwechat}}.}}\pend
           \pstart
           lieber Richard, vor allem gratulir ich Ihnen herzlich zum
               vollendeten \textcolor{green}{\textsc{Charolais}}{}\ledrightnote{\textcolor{green}{Der Graf von Charolais. Ein Trauerspiel}}. Ferner: wir fahren Montag nach \textcolor{pink}{\textsc{Lueg}}{}\ledrightnote{\textcolor{pink}{Lueg am Wolfgangsee}} und bleiben dort bis etwa Donnerſtag Früh. Ich beſprach heute eben mit \textcolor{blue}{Hugo}{}\ledrightnote{\textcolor{blue}{Hugo von Hofmannsthal}}, wie hübſch das wäre, we{\geminationn} Sie auch herüber kämen. Thuen Sie’s doch jedenfalls.
               \textcolor{blue}{Hugo’s}{}\ledrightnote{\textcolor{blue}{Hugo von Hofmannsthal}{\newline}\textcolor{blue}{Gertrude von Hofmannsthal}} fahren Mittwoch Abend nach
                  \textcolor{pink}{Salzburg}{}\ledrightnote{\textcolor{pink}{Salzburg}}; \textcolor{blue}{Olga}{}\ledrightnote{\textcolor{blue}{Olga Schnitzler}}
               u ich würden {\pb}dann \introOben{}von \textsc{Lueg} aus\introOben{} mit Ihnen nach \textcolor{pink}{Auſſee}{}\ledrightnote{\textcolor{pink}{Bad Aussee}} fahren, wo wir etwa 2–3 Tage (\textcolor{pink}{\textsc{Hotel Elisabeth}}{}\ledrightnote{\textcolor{pink}{Bade-Hotel Elisabeth}} wie man uns räth) wohnen wollen. (Unſer weiteres Progra{\geminationm} iſt dann einige Tage \textcolor{pink}{Iſchl}{}\ledrightnote{\textcolor{pink}{Bad Ischl}}, einige Tage \textcolor{pink}{Salzburg}{}\ledrightnote{\textcolor{pink}{Salzburg}})\pend
           \pstart
           Jedenfalls, we{\geminationn} Sie nicht ſelbſt kommen, bitte um ein
               Wort \textcolor{pink}{Gaſthof \textsc{Lueg}}{}\ledrightnote{\textcolor{pink}{Hotel und Pension Lueg}}, bei \textcolor{pink}{\textsc{St. Gilgen}}{}\ledrightnote{\textcolor{pink}{St. Gilgen}}.\pend
           \pstart
           Aber kommen Sie.\pend
           \pstart
           Herzlichſt Ihr{\\[\baselineskip]}\spacefill\mbox{A.}\pend
           \leftskip=0em{}\endnumbering\briefempfaengerindex{Beer-Hofmann, Richard@\textsc{Beer-Hofmann, Richard}!zzzSchnitzler, Arthur@\emph{von Arthur Schnitzler}!1904-09-032@{3. 9. 1904}|)be}\mylabel{h}  \normalsize

\doendnotes{C}
\bigskip
\vfill

\clearpage

\footnotesize

\lohead{\textsc{register}}

% Definiere theindex-Environment komplett neu ohne reledmac
\makeatletter
\renewenvironment{theindex}{%
  \section*{\indexname}%
  \setlength{\parindent}{0pt}%
  \setlength{\parskip}{0pt plus 0.3pt}%
  \let\item\@idxitem
}{%
  \clearpage
}
\makeatother

\IfFileExists{\jobname-pw.ind}{\input{\jobname-pw.ind}}{}

\end{document}

      