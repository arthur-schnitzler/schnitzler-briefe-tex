%% latex-korrekturansicht-vorspann.tex
%% Vorspann für die Korrekturansicht.
%% Lädt die gemeinsame Datei latex-vorspann.tex mit gesetztem Schalter.

\newif\ifkorrekturansicht
\korrekturansichttrue

\input{../tex-inputs/latex-vorspann}


               \section[Felix Braun an Arthur Schnitzler, 15. 3. 1926]{ Felix Braun an Arthur Schnitzler, 15. 3. 1926}\nopagebreak\mylabel{v}\rehead{ }\normalsize\beginnumbering\briefempfaengerindex{Schnitzler, Arthur@\textsc{Schnitzler, Arthur}!zzzBraun, Felix@\emph{von Felix Braun}!1926-03-151@{15. 3. 1926}|(be} \toendnotes[C]{\smallbreak\pagebreak[2]} \Standort{DLA, A:Schnitzler, HS.NZ85.1.2604,7.}
\physDesc{Brief, 2 Blätter, 4 Seiten
\newline{}Handschrift: schwarze Tinte, deutsche Kurrent
\newline{}Schnitzler: 1) mit Bleistift beschriftet: »\strikeout{\textcolor{gray}{×}\-\textcolor{gray}{×}}{ }\textsc{Braun}« 2) mit rotem Buntstift mehrere Unterstreichungen}\toendnotes[C]{\smallbreak}\pstart
           \centering{}{\pb}\textcolor{pink}{Wien}{}\ledrightnote{\textcolor{pink}{Wien}}, den 15. III. 26\pend
           \pstart{}Verehrter Herr Doktor!\pend\pstart
           Tief ergriffen und bewegt hat mich Ihr »\textcolor{green}{Gang zum
                        Weiher}{}\ledrightnote{\textcolor{green}{Der Gang zum Weiher. Dramatische Dichtung}}« und nicht nur dieſe Wirkung, zu der die ſchönſte äſthetiſche
                    tritt, auch eine innerſt-perſönliche fühle ich auf mich ausgeübt, Antwort auf
                    manche Frage, Qual und Furcht gegeben, und ſo kann ich nur ſagen, daß ich Ihnen
                    für dieſe Dichtung als Leſer, als Schriftſteller und nicht zuletzt als Menſch
                    aufs Innigſte verbunden bin.\pend
           \pstart
           Es iſt eine Dichtung der Weisheit und der ſpäten Einſamkeit, von der die Jugend,
                    die Einſamkeit ſo leidenſchaftlich ſucht, nichts {\pb}ahnt. Wie ſchon im »\textcolor{green}{Einſamen Weg}{}\ledrightnote{\textcolor{green}{Der einsame Weg. Schauspiel in fünf Akten}}« und
                    neuerdings in der »\textcolor{green}{Komoedie der Verführung}{}\ledrightnote{\textcolor{green}{Komödie der Verführung. In drei Akten}}«
                    iſt hier Einſamkeitsluft um die Geſtalten von Männern, die aus der Jugend
                    getreten ſind. Das iſt ſehr erregend und ergreifend. \uline{Dieſe} Tragoedie des Mannes haben Sie wohl als Erſter gedichtet. Und
                    dies Älterwerden beginnt vielleicht weit früher, als es ſich Jugend träumen
                    läßt. Das Erbarmungsloſe, das in ſolchem Kampf jeden, aber auch jeden Vorzug zu
                    nichte macht, iſt noch nie ſo erkannt, ſo gewieſen worden.\pend
           \pstart
           Schön ſind die Verſe, Ihre ſchönſten bisher. Dieſelbe hohe, klare Luft ſchwebt
                    über ihnen. Ein \textcolor{blue}{Goethe}{}\ledrightnote{\textcolor{blue}{Johann Wolfgang von Goethe}}’ſcher Hauch,
                    überhaupt Atem unſerer klaſſiſchen Dramendichtung {\pb}beglückt darin mit. Daß Sie durch das neue \textcolor{green}{Werk}{}\ledrightnote{→\textcolor{green}{Der Gang zum Weiher. Dramatische Dichtung}} an unſere große Tradition anſchließen, iſt mir
                    beſonders, der ich ich mich immer darum bemüht habe, erwünſcht und wertvoll.\pend
           \pstart
           Nicht ganz überzeugend finde ich die Geſtalt des \textcolor{green}{Mädchens}{}\ledrightnote{→\textcolor{green}{Der Gang zum Weiher. Dramatische Dichtung}}. Soll ſie nur eine Idee ſein? Die der
                    Jugend? Die des weiblichen Naturweſens? Sie verſagt nach meinem Gefühl ſowohl
                    gegen \textcolor{green}{Konrad}{}\ledrightnote{→\textcolor{green}{Der Gang zum Weiher. Dramatische Dichtung}} wie gegen \textcolor{green}{Sylveſter}{}\ledrightnote{→\textcolor{green}{Der Gang zum Weiher. Dramatische Dichtung}}. Sie iſt nicht
                    weiblich und nicht menſchlich genug. Andererſeits wüßte ich freilich ſelbſt
                    keine beſſere Löſung.\pend
           \pstart
           Ich ſchreibe in Eile, denn ich bin vor der \label{K_L02468_1v}\edtext{Abreiſe}{\lemma{\textnormal{\emph{Abreiſe}}}\Cendnote{\textnormal{Die Uraufführung fand am 27. 3. 1926 im
                            \textcolor{pink}{Badischen Landestheater Karlsruhe}
                        statt.}}}\label{K_L02468_1h}: in \textcolor{pink}{Karlsruhe}{}\ledrightnote{\textcolor{pink}{Karlsruhe}} wird mein »\textcolor{green}{\textsc{Tantalos}}{}\ledrightnote{\textcolor{green}{Tantalos}}« geſpielt und ich will bei den {\pb}Proben
                    dabei ſein. Es iſt zum erſten Mal, daß ich das erlebe.\pend
           \pstart
           Seien Sie von Herzen bedankt, verehrter Arthur Schnitzler! Wie glücklich müſſen
                    Sie beim Schreiben dieſes \textcolor{green}{Werks}{}\ledrightnote{→\textcolor{green}{Der Gang zum Weiher. Dramatische Dichtung}} geweſen ſein! Ich halte es für Ihr größtes!\pend
           \pstart
           Wie immer verharrend\hspace*{1.5em}Ihr{\\[\baselineskip]}\spacefill\mbox{Felix Braun.}\pend
           \leftskip=0em{}\endnumbering\briefempfaengerindex{Schnitzler, Arthur@\textsc{Schnitzler, Arthur}!zzzBraun, Felix@\emph{von Felix Braun}!1926-03-151@{15. 3. 1926}|)be}\mylabel{h}  \normalsize

\doendnotes{C}
\bigskip
\vfill

\clearpage

\footnotesize

\lohead{\textsc{register}}

% Definiere theindex-Environment komplett neu ohne reledmac
\makeatletter
\renewenvironment{theindex}{%
  \section*{\indexname}%
  \setlength{\parindent}{0pt}%
  \setlength{\parskip}{0pt plus 0.3pt}%
  \let\item\@idxitem
}{%
  \clearpage
}
\makeatother

\IfFileExists{\jobname-pw.ind}{\input{\jobname-pw.ind}}{}

\end{document}

      