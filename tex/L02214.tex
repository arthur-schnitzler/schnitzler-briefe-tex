%% latex-korrekturansicht-vorspann.tex
%% Vorspann für die Korrekturansicht.
%% Lädt die gemeinsame Datei latex-vorspann.tex mit gesetztem Schalter.

\newif\ifkorrekturansicht
\korrekturansichttrue

\input{../tex-inputs/latex-vorspann}


               \section[Arthur Schnitzler an Robert Adam, 13. {[}7. 1915?{]}]{ Arthur Schnitzler an Robert Adam, 13. {[}7. 1915?{]}}\nopagebreak\mylabel{v}\rehead{ }\normalsize\beginnumbering\briefempfaengerindex{Adam, Robert@\textsc{Adam, Robert}!zzzSchnitzler, Arthur@\emph{von Arthur Schnitzler}!1915-07-131@{13. {[}7. 1915?{]}}|(be} \toendnotes[C]{\smallbreak\pagebreak[2]} \Standort{DLA, 96.34.1–2.}
\physDesc{Umschlag
\newline{}Handschrift: 1) schwarze Tinte, deutsche Kurrent\hspace{1em}2) Bleistift, deutsche Kurrent (\noindent{}»Druckſache«)\hspace{1em}\newline{}Versand: 1) Stempel: »\nobreak{}\oindex{XVIII., Waehring@\textbf{XVIII., Währing}, \emph{Bezirk (A.BZK)}|pwk}18/\textsubscript{1} Wien
                                        111, 13 \textcolor{gray}{VII.} {[}1915{]}\nobreak{}«.  2) das erhöhte Porto
                                    und der Kleber »R« (für »Rekommandiert«, eingeschrieben) deuten
                                    darauf hin, dass damit das Manuskript retour gesandt wurde}\pstart{}{\pb}Abſender\pend{}\pstart{}\textsc{Dr Arthur Schnitzler.}{ }\textcolor{pink}{Wien XVIII}{}\ledrightnote{\textcolor{pink}{XVIII., Währing}}\pend{}\pstart{}\textsc{\textcolor{pink}{Sternwartestr 71.}{}\ledrightnote{\textcolor{pink}{Sternwartestraße}}}\pend{}{\bigskip}\pstart{}Herrn Dr. \textsc{Robert Adam Pollak}\pend{}\pstart{}Bezirksrichter in\pend{}\pstart{}\textsc{\textcolor{pink}{Zistersdorf}{}\ledrightnote{\textcolor{pink}{Zistersdorf}}}\pend{}\pstart{}\textcolor{pink}{\textsc{N. Oe.}}{}\ledrightnote{\textcolor{pink}{Niederösterreich}} –
                    \pend{}\pstart{}Druckſache\pend{}{\bigskip}\endnumbering\briefempfaengerindex{Adam, Robert@\textsc{Adam, Robert}!zzzSchnitzler, Arthur@\emph{von Arthur Schnitzler}!1915-07-131@{13. {[}7. 1915?{]}}|)be}\mylabel{h}  \normalsize

\doendnotes{C}
\bigskip
\vfill

\clearpage

\footnotesize

\lohead{\textsc{register}}

% Definiere theindex-Environment komplett neu ohne reledmac
\makeatletter
\renewenvironment{theindex}{%
  \section*{\indexname}%
  \setlength{\parindent}{0pt}%
  \setlength{\parskip}{0pt plus 0.3pt}%
  \let\item\@idxitem
}{%
  \clearpage
}
\makeatother

\IfFileExists{\jobname-pw.ind}{\input{\jobname-pw.ind}}{}

\end{document}

      