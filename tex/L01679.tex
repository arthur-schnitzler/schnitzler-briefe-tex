%% latex-korrekturansicht-vorspann.tex
%% Vorspann für die Korrekturansicht.
%% Lädt die gemeinsame Datei latex-vorspann.tex mit gesetztem Schalter.

\newif\ifkorrekturansicht
\korrekturansichttrue

\input{../tex-inputs/latex-vorspann}


               \section[Arthur Schnitzler an Gerty von Hofmannsthal, 26. 5. 1907]{ Arthur Schnitzler an Gerty von Hofmannsthal, 26. 5. 1907}\nopagebreak\mylabel{v}\rehead{ }\normalsize\beginnumbering\briefempfaengerindex{Hofmannsthal, Gertrude von@\textsc{Hofmannsthal, Gertrude von}!zzzSchnitzler, Arthur@\emph{von Arthur Schnitzler}!1907-05-261@{26. 5. 1907}|(be} \toendnotes[C]{\smallbreak\pagebreak[2]} \Standort{FDH, Hs-30997,127.}
\physDesc{Briefkarte
\newline{}Handschrift: schwarze Tinte, deutsche Kurrent}\buchAbdrucke{\weitereDrucke{Hugo von Hofmannsthal, Arthur Schnitzler: \emph{Briefwechsel}. Hg. Therese Nickl und Heinrich Schnitzler. Frankfurt am Main: \emph{S. Fischer} 1964, S. 375.} }\pstart
           \noindent{}{\pb}\textcolor{gray}{\textbf{Dr. Arthur Schnitzler}}\hfill 26. 5. 907\pend
           \pstart
           \textcolor{gray}{\textbf{\textcolor{pink}{Wien XVIII. Spoettelgasse 7}{}\ledrightnote{\textcolor{pink}{Edmund-Weiß-Gasse}}.}}\pend
           \pstart
           liebe Gerty, \textcolor{blue}{Hugo}{}\ledrightnote{\textcolor{blue}{Hugo von Hofmannsthal}} hat mir geſchrieben, daſs er geſtern
               verreiſt iſt, aber nicht die Adreſſe angegeben, wo ihn Briefe treffen. Wollen Sie mir
               ein Wort in die \textcolor{pink}{Hinterbrühl \textsc{Radetzky}}{}\ledrightnote{\textcolor{pink}{Hotel Radetzky}}{ }ſchreiben? Auch wie es der Gräfin \textcolor{blue}{Thun}{}\ledrightnote{\textcolor{blue}{Christiane von Thun-Hohenstein-Salm-Reifferscheidt}} geht, ob ſie ſchon außer Gefahr iſt. Und ſehr nett {\pb}wärs, we{\geminationn} Sie einmal
               hinüber kämen und eventuell zu einer \textsc{Tennisparti} bereit
               wären? –\pend
           \pstart
           Herzlichſt mit Grüßen von \textcolor{blue}{Olga}{}\ledrightnote{\textcolor{blue}{Olga Schnitzler}} und mir\pend
           \pstart
           Ihr{\\[\baselineskip]}\spacefill\mbox{Arthur}\pend
           \leftskip=0em{}\endnumbering\briefempfaengerindex{Hofmannsthal, Gertrude von@\textsc{Hofmannsthal, Gertrude von}!zzzSchnitzler, Arthur@\emph{von Arthur Schnitzler}!1907-05-261@{26. 5. 1907}|)be}\mylabel{h}  \normalsize

\doendnotes{C}
\bigskip
\vfill

\clearpage

\footnotesize

\lohead{\textsc{register}}

% Definiere theindex-Environment komplett neu ohne reledmac
\makeatletter
\renewenvironment{theindex}{%
  \section*{\indexname}%
  \setlength{\parindent}{0pt}%
  \setlength{\parskip}{0pt plus 0.3pt}%
  \let\item\@idxitem
}{%
  \clearpage
}
\makeatother

\IfFileExists{\jobname-pw.ind}{\input{\jobname-pw.ind}}{}

\end{document}

      