%% latex-korrekturansicht-vorspann.tex
%% Vorspann für die Korrekturansicht.
%% Lädt die gemeinsame Datei latex-vorspann.tex mit gesetztem Schalter.

\newif\ifkorrekturansicht
\korrekturansichttrue

\input{../tex-inputs/latex-vorspann}


               \section[Arthur Schnitzler an Hermann Bahr, 3. 1. 1902]{ Arthur Schnitzler an Hermann Bahr, 3. 1. 1902}\nopagebreak\mylabel{v}\rehead{ }\normalsize\beginnumbering\briefempfaengerindex{Bahr, Hermann@\textsc{Bahr, Hermann}!zzzSchnitzler, Arthur@\emph{von Arthur Schnitzler}!1902-01-031@{3. 1. 1902}|(be} \toendnotes[C]{\smallbreak\pagebreak[2]} \Standort{TMW, HS AM 23348 Ba.}
\physDesc{Brief, 2 Blätter, 7 Seiten
\newline{}Handschrift: Bleistift, deutsche Kurrent\newline{}Ordnung: Lochung }\buchAbdrucke{\weitereDrucke{1) \emph{3. 1. 1902.} In: Arthur Schnitzler: \emph{The Letters of Arthur Schnitzler to Hermann Bahr}. Edited, annotated, and with an introduction, by Donald G.
                        Daviau. Chapel Hill: \emph{The University of North Carolina Press} 1978, S. 73–74 (University of North Carolina studies in the Germanic languages
                        and literatures, 89).} \weitereDrucke{2) Hermann Bahr, Arthur Schnitzler: \emph{Briefwechsel, Aufzeichnungen, Dokumente (1891–1931)}. Hg. Kurt Ifkovits und Martin Anton Müller. Göttingen: \emph{Wallstein} 2018, S. 222–223.} }\toendnotes[C]{\smallbreak}\pstart
           \raggedleft{}{\pb}3. 1. 902{\\}\textcolor{pink}{\textsc{Berlin}}{}\ledrightnote{\textcolor{pink}{Berlin}}\pend
           \pstart
           lieber Hermann, ich habe \textcolor{blue}{Brahm}{}\ledrightnote{\textcolor{blue}{Otto Brahm}}
               geſprochen, er äußerte ſich anerkennend über den \textcolor{green}{Krampus}{}\ledrightnote{\textcolor{green}{Der Krampus}}, findet nur, daſs gerade das \textcolor{pink}{Deutſche
                  Theater}{}\ledrightnote{\textcolor{pink}{Deutsches Theater Berlin}} nicht der rechte Boden für das Stück sei. Ich glaube alſo nicht, daſs
               er zu der Aufführg nach \textcolor{pink}{Hamburg}{}\ledrightnote{\textcolor{pink}{Hamburg}} fahren wird, hielte
               es aber doch für ganz gut, we{\geminationn} du ihn unverbindlich mit
               ein paar {\pb}Worten dazu
               einladen möchteſt. Gegen deine Bemerkung über den literar. Stempel, den doch erſt das
                  \textcolor{pink}{Deutſche Theater}{}\ledrightnote{\textcolor{pink}{Deutsches Theater Berlin}} verleihe (die ihm mitzutheilen ich
               mich wohl für befugt halten durfte?) ſchien er nicht unempfindlich zu ſein, und ich
               zweifle nicht daran, daſs er deine nächſten Stücke ohne vorgefaſſte Meinung leſen
               wird. Ich bin übrigens mor{\pb}gen Nachmittag bei ihm
               und habe ſicher Gelegenheit, nochmals in deinem Sinne zu reden. Er gehört doch, bei
               allen Begrenztheiten und Eigenſinnigkeiten zu den weitaus verſtändigſten
               Theatermenſchen \introOben{}(vielleicht auch Menſchen ſchlichtweg –)\introOben{},
               die es gibt, und iſt derjenige, mit dem man am gradlinigſten und verläßlichſten
               verkehren kann. Man darf von ihm ſagen, daſs {\pb}er nie lügt. Du
               sollteſt dich einmal perſönlich mit ihm ausſprechen. We{\geminationn}
               er nicht nach \textcolor{pink}{Hamburg}{}\ledrightnote{\textcolor{pink}{Hamburg}} ko{\geminationm}t, vielleicht beſuchst du ihn auf der Hin- oder Rückfahrt? – \pend
           \pstart
           \label{K_L01195_1v}\edtext{Dieſer Tage}{\lemma{\textnormal{\emph{Dieſer Tage}}}\Cendnote{\textnormal{vgl. A. S.: \emph{Tagebuch}, 1. 1. 1902}}}\label{K_L01195_1h}{ }ſprach ich \textsc{\textcolor{blue}{Harden}{}\ledrightnote{\textcolor{blue}{Maximilian Harden}}}\damage{,} der jetzt ſehr gegen den kleinen \textcolor{blue}{Kraus}{}\ledrightnote{\textcolor{blue}{Karl Kraus}}
                  eingeno{\geminationm}en iſt und findet, daſs ein ſolches \textcolor{green}{Blatt}{}\ledrightnote{→\textcolor{green}{Die Fackel}} in \textcolor{pink}{Berlin}{}\ledrightnote{\textcolor{pink}{Berlin}}{ }ſich nicht halten kö{\geminationn}te. {\pb}Anläßlich der
                  \label{K_L01195_2v}\edtext{\textcolor{green}{Krausiſchen Kritik}{}\ledrightnote{→\textcolor{green}{[Wie mich Herr Bahr beneidet]}} über die \textsc{\textcolor{green}{veine}{}\ledrightnote{\textcolor{green}{Das Glück}}}}{\lemma{\textnormal{\emph{Krausiſchen … veine}}}\Cendnote{\textnormal{\textcolor{blue}{Kraus}{ }schreibt in der \emph{\textcolor{green}{Fackel}} (\textcolor{green}{Bd. 10, H. 82, Anfang October, S. 19}): »Herr \textcolor{blue}{Bahr}, der wiederum das
                     Referat über das \textcolor{pink}{Deutsche Volkstheater}
                     übernommen hat, berichtet, dass in dem neuen \textcolor{green}{Stücke} von \textcolor{blue}{Capus}
                     ein ›mit zwei Strichen wunderbar gezeichneter‹ Journalist vorkomme, der sich
                     nicht verkauft, weil ›ihm das nie so viel tragen kann wie seine
                     Unbestechlichkeit‹. Man versichert mir – ich kann die Mittheilung leider nicht
                     überprüfen –, dass diese Stelle, die Herr \textcolor{blue}{Bahr} mit so munterem Behagen citiert, nachträglich in die Uebersetzung
                     der \textcolor{pink}{französischen} Comödie hineingeflickt
                     worden sei und dass Herr \textcolor{blue}{Bahr}{ }sich selbst citiere.« \textcolor{blue}{Bahrs} Besprechung, in der sich das Zitat
                  findet: \emph{\textcolor{green}{Das Glück. (La veine. Komödie in vier Aufzügen von
                           \textcolor{blue}{Alfred Capus}. Deutsch von \textcolor{blue}{Theodor Wolff}. Zum erstenmal aufgeführt
                        im \textcolor{pink}{Deutschen Volkstheater} am 28.
                           September 1901)}}. In: \emph{\textcolor{green}{Neues
                        Wiener Tagblatt}}, Jg. 35, Nr. 267, 29. 9. 1901,
                     S. 2–4.}}}\label{K_L01195_2h}, in der \textcolor{blue}{Kr.}{}\ledrightnote{\textcolor{blue}{Karl Kraus}} von einer
               angeblich extra von dir \introOben{}(?)\introOben{} gegen ihn hineingedichteten
               Stelle erzählte, hat er ihm (\textsc{\textcolor{blue}{Harden}{}\ledrightnote{\textcolor{blue}{Maximilian Harden}}} dem \textcolor{blue}{Kraus}{}\ledrightnote{\textcolor{blue}{Karl Kraus}}) eine Karte geſchrieben, er müſſe
               gelegentlich diesen Irrthum richtigſtellen, da die betreffende Stelle ſich \label{K_L01195_3v}\edtext{im Original}{\lemma{\textnormal{\emph{im Original}}}\Cendnote{\textnormal{»Car pourquoi se vendrait-il? Ça ne lui rapporterait
                     jamais autant que d’être incorruptible.« \textcolor{blue}{Alfred Capus}: \emph{\textcolor{green}{La veine. Comédie en quatre actes}}. Paris: \emph{Éditions de la
                        Revue Blanche}{ }{[}1901?{]}, S. 149 (III, 9).}}}\label{K_L01195_3h} fände; – \textcolor{blue}{Kraus}{}\ledrightnote{\textcolor{blue}{Karl Kraus}}{ }ſoll es auch zugeſagt \strikeout{\textcolor{gray}{haben}, aber} bisher nicht {\pb}gethan haben. – \pend
           \pstart
           Heute war Generalprobe der \textcolor{green}{Lebendigen Stunden}{}\ledrightnote{\textcolor{green}{Lebendige Stunden. Vier Einakter}}. Sie
               fiel günſtig – für abergläubiſch\damage{e} Gemüther zu günſtig \substVorne{}\textsuperscript{\textcolor{gray}{ohne}}\substDazwischen{}aus\substHinten{}. –\pend
           \pstart
           Ganz entzückt bin ich von \textcolor{blue}{\textsc{Bassermann}}{}\ledrightnote{\textcolor{blue}{Albert Bassermann}}. \label{K_L01195_4v}\edtext{Neulich ſah ich ihn als \textsc{\textcolor{green}{H\introOben{}j\introOben{}a\substVorne{}\textsuperscript{\textcolor{gray}{jm}}\substDazwischen{}lm\substHinten{}ar}{}\ledrightnote{→\textcolor{green}{Die Wildente}}}, \textcolor{blue}{\textsc{Sauer}}{}\ledrightnote{\textcolor{blue}{Oskar Sauer}} als \textcolor{green}{\textsc{Gregers Werle}}{}\ledrightnote{→\textcolor{green}{Die Wildente}}}{\lemma{\textnormal{\emph{Neulich … Werle}}}\Cendnote{\textnormal{Am 30. 12. 1901 spielte er
                  in \textcolor{blue}{Ibsens}{ }\emph{\textcolor{green}{Wildente}}.}}}\label{K_L01195_4h}; ich habe ſelten ſo ſtarke
               ſchauſpieleriſche Eindrücke erlebt. Die \textcolor{blue}{Trieſch}{}\ledrightnote{\textcolor{blue}{Irene Triesch}}{ }{\pb}kann überraſchend
               viel. –\pend
           \pstart
           – Ich ſeh dich hoffentlich bald wieder. Herzlichen Gruſs. Dein \pend
           \pstart \spacefill\mbox{Arth Sch}\pend{}\endnumbering\briefempfaengerindex{Bahr, Hermann@\textsc{Bahr, Hermann}!zzzSchnitzler, Arthur@\emph{von Arthur Schnitzler}!1902-01-031@{3. 1. 1902}|)be}\mylabel{h}  \normalsize

\doendnotes{C}
\bigskip
\vfill

\clearpage

\footnotesize

\lohead{\textsc{register}}

% Definiere theindex-Environment komplett neu ohne reledmac
\makeatletter
\renewenvironment{theindex}{%
  \section*{\indexname}%
  \setlength{\parindent}{0pt}%
  \setlength{\parskip}{0pt plus 0.3pt}%
  \let\item\@idxitem
}{%
  \clearpage
}
\makeatother

\IfFileExists{\jobname-pw.ind}{\input{\jobname-pw.ind}}{}

\end{document}

      