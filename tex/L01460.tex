%% latex-korrekturansicht-vorspann.tex
%% Vorspann für die Korrekturansicht.
%% Lädt die gemeinsame Datei latex-vorspann.tex mit gesetztem Schalter.

\newif\ifkorrekturansicht
\korrekturansichttrue

\input{../tex-inputs/latex-vorspann}


               \section[Arthur Schnitzler an Gerhart Hauptmann, 24. 10. 1904]{ Arthur Schnitzler an Gerhart Hauptmann, 24. 10. 1904}\nopagebreak\mylabel{v}\rehead{ }\normalsize\beginnumbering\briefempfaengerindex{Hauptmann, Gerhart@\textsc{Hauptmann, Gerhart}!zzzSchnitzler, Arthur@\emph{von Arthur Schnitzler}!1904-10-241@{24. 10. 1904}|(be} \toendnotes[C]{\smallbreak\pagebreak[2]} \Standort{Staatsbibliothek Berlin – Preußischer Kulturbesitz, GHBrBl A:Schnitzler (9).}
\physDesc{Telegramm
\newline{}maschinell\newline{}Versand: 1) auf der Rückseite ein handschriftlicher Vermerk mit Tinte:
                                    »Adrſ. Im \textcolor{brown}{Lessing-Theater} nicht anweſend, Nachſ. nach \textcolor{pink}{Hotel de Rome}. \textcolor{blue}{\textcolor{gray}{Litmica}}« 2) Stempel: »\nobreak{}\oindex{Berlin@\textbf{Berlin}, \emph{https://www.geonames.org/ontologyP.PPLC}|pwk}Berlin N.W., 24 V 04, 1\textcolor{gray}{8}\textsuperscript{20}N\nobreak{}«. 3) Stempel: »\nobreak{}Ausgefertigt, 24 Oct., 1\textcolor{gray}{×}\textsuperscript{\textcolor{gray}{×}2}\nobreak{}«. 4) »\textcolor{gray}{\textbf{\textbf{Aufgenommen} von}}{ }\textcolor{gray}{IW}{ }\textcolor{gray}{\textbf{den}}{ }24\textcolor{gray}{\textbf{/}}11{ }\textcolor{gray}{\textbf{um}}{ }11 \textcolor{gray}{\textbf{Uhr}} 19 \textcolor{gray}{\textbf{M.}}m{ }\textcolor{gray}{\textbf{durch}}{ }\textcolor{gray}{Grm}«5) mit Bleistift unterhalb des Empfängers: »\textcolor{pink}{Hotel Rom{ }U d Linden 39}«\newline{}Ordnung: Lochung }\pstart{}{\pb}= {[}g{]}erhard hauptmann
                     \textcolor{pink}{berlin}{}\ledrightnote{\textcolor{pink}{Berlin}}\pend{}\pstart{}\textcolor{pink}{lessingtheater}{}\ledrightnote{\textcolor{pink}{Lessing-Theater}} +\pend{}{\bigskip}\pstart
           {\pb}\textcolor{gray}{\textbf{Telegramm}} de \textcolor{pink}{wien}{}\ledrightnote{\textcolor{pink}{Wien}}
                  111.+723 21 24{ }10 40m{ }\textcolor{gray}{\textbf{W.}}{ }\textcolor{gray}{\textbf{190}}4{ }\pend
           \pstart
           jch beglueckwuensche sie und \textcolor{blue}{brahm}{}\ledrightnote{\textcolor{blue}{Otto Brahm}} herzlich zur
               ruhmreichen auferstehung des \textcolor{green}{florian geyer}{}\ledrightnote{\textcolor{green}{Florian Geyer. Die Tragödie des Bauernkrieges}} –
               herzlichst gruessend jhr \spacefill\mbox{arthur schnitzler +}\pend
           \endnumbering\briefempfaengerindex{Hauptmann, Gerhart@\textsc{Hauptmann, Gerhart}!zzzSchnitzler, Arthur@\emph{von Arthur Schnitzler}!1904-10-241@{24. 10. 1904}|)be}\mylabel{h}  \normalsize

\doendnotes{C}
\bigskip
\vfill

\clearpage

\footnotesize

\lohead{\textsc{register}}

% Definiere theindex-Environment komplett neu ohne reledmac
\makeatletter
\renewenvironment{theindex}{%
  \section*{\indexname}%
  \setlength{\parindent}{0pt}%
  \setlength{\parskip}{0pt plus 0.3pt}%
  \let\item\@idxitem
}{%
  \clearpage
}
\makeatother

\IfFileExists{\jobname-pw.ind}{\input{\jobname-pw.ind}}{}

\end{document}

      