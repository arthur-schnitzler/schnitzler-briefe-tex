%% latex-korrekturansicht-vorspann.tex
%% Vorspann für die Korrekturansicht.
%% Lädt die gemeinsame Datei latex-vorspann.tex mit gesetztem Schalter.

\newif\ifkorrekturansicht
\korrekturansichttrue

\input{../tex-inputs/latex-vorspann}


               \section[Arthur Schnitzler an Gerhart Hauptmann, 8. 3. 1903]{ Arthur Schnitzler an Gerhart Hauptmann, 8. 3. 1903}\nopagebreak\mylabel{v}\rehead{ }\normalsize\beginnumbering\briefempfaengerindex{Hauptmann, Gerhart@\textsc{Hauptmann, Gerhart}!zzzSchnitzler, Arthur@\emph{von Arthur Schnitzler}!1903-03-081@{8. 3. 1903}|(be} \toendnotes[C]{\smallbreak\pagebreak[2]} \Standort{Staatsbibliothek Berlin – Preußischer Kulturbesitz, GHBrBl A:Schnitzler (8).}
\physDesc{Telegramm
\newline{}Handschrift  Weichert: blauer Buntstift, deutsche Kurrent\newline{}Versand: »\noindent{}\textcolor{gray}{\textbf{\textbf{Aufgenommen} von}}{ }\textcolor{gray}{Il}{ }\textcolor{gray}{\textbf{den}}{ }8\textcolor{gray}{\textbf{/}}3{ }\textcolor{gray}{\textbf{um}}{ }9 \textcolor{gray}{\textbf{Uhr}} 35\textcolor{gray}{\textbf{M}}{ }\textcolor{gray}{17}{ }\textcolor{gray}{\textbf{durch}}{ }\textcolor{blue}{Weichert}.« \newline{}Ordnung: Lochung }\toendnotes[C]{\smallbreak}\pstart{}{\pb}Gerhart Hauptmann\pend{}\pstart{}\textcolor{pink}{Agnetendorf}{}\ledrightnote{\textcolor{pink}{Agnetendorf}}\pend{}{\bigskip}\pstart
           {\pb}\textcolor{gray}{\textbf{Telegramm aus}}{ }\textcolor{pink}{Berlin 9}{}\ledrightnote{\textcolor{pink}{Berlin}}\hfill 24 \textcolor{gray}{\textbf{W.}}{ }\textcolor{gray}{\textbf{190}}3 \textcolor{gray}{\textbf{den}} 8\textcolor{gray}{\textbf{\textsuperscript{ten}}} 3{ }\textcolor{gray}{\textbf{um}} 9 \textcolor{gray}{\textbf{Uhr}} 3 \textcolor{gray}{\textbf{Min.}} m\pend
           \pstart
           Hätten Ihre lieben Wünſche ſo viel Kraft gehabt als Sie mich erfreuten \textcolor{green}{es}{}\ledrightnote{→\textcolor{green}{Der Schleier der Beatrice. Schauspiel in fünf Akten}} wäre ein großer Erfolg
               geworden. Ich grüße Sie in herzlicher Freundſchaft\pend
           \pstart \spacefill\mbox{Arthur Schnitzler}\pend{}\endnumbering\briefempfaengerindex{Hauptmann, Gerhart@\textsc{Hauptmann, Gerhart}!zzzSchnitzler, Arthur@\emph{von Arthur Schnitzler}!1903-03-081@{8. 3. 1903}|)be}\mylabel{h}  \normalsize

\doendnotes{C}
\bigskip
\vfill

\clearpage

\footnotesize

\lohead{\textsc{register}}

% Definiere theindex-Environment komplett neu ohne reledmac
\makeatletter
\renewenvironment{theindex}{%
  \section*{\indexname}%
  \setlength{\parindent}{0pt}%
  \setlength{\parskip}{0pt plus 0.3pt}%
  \let\item\@idxitem
}{%
  \clearpage
}
\makeatother

\IfFileExists{\jobname-pw.ind}{\input{\jobname-pw.ind}}{}

\end{document}

      