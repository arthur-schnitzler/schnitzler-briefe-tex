%% latex-korrekturansicht-vorspann.tex
%% Vorspann für die Korrekturansicht.
%% Lädt die gemeinsame Datei latex-vorspann.tex mit gesetztem Schalter.

\newif\ifkorrekturansicht
\korrekturansichttrue

\input{../tex-inputs/latex-vorspann}


               \section[Arthur Schnitzler an Richard Beer-Hofmann, 15. 6. 1895]{ Arthur Schnitzler an Richard Beer-Hofmann, 15. 6. 1895}\nopagebreak\mylabel{v}\rehead{ }\normalsize\beginnumbering\briefempfaengerindex{Beer-Hofmann, Richard@\textsc{Beer-Hofmann, Richard}!zzzSchnitzler, Arthur@\emph{von Arthur Schnitzler}!1895-06-151@{15. 6. 1895}|(be} \toendnotes[C]{\smallbreak\pagebreak[2]} \Standort{YCGL, MSS 31.}
\physDesc{Brief, 1 Blatt, 4 Seiten, Umschlag
\newline{}Handschrift: 1) Bleistift, deutsche Kurrent\hspace{1em}2) schwarze Tinte, deutsche Kurrent (\noindent{}Umschlag)\hspace{1em}\newline{}Versand: 1) Stempel: »\nobreak{}\oindex{I., Innere Stadt@\textbf{I., Innere Stadt}, \emph{Bezirk (A.BZK)}|pwk}Wien 1/1, 15. 6. 95, 7–8 N\nobreak{}«.  2) Stempel: »\nobreak{}\oindex{Caslau@\textbf{Caslau}, \emph{Besiedelter Ort (A.BSO)}|pwk}Časlau, 16 6 95\nobreak{}«. }\buchAbdrucke{\weitereDrucke{1) Arthur Schnitzler: \emph{Briefe 1875–1912}. Hg. Therese Nickl und Heinrich Schnitzler. Frankfurt am Main: \emph{S. Fischer} 1981, S. 260–261.} \weitereDrucke{2) Arthur Schnitzler, Richard Beer-Hofmann: \emph{Briefwechsel 1891–1931}. Hg. Konstanze Fliedl. Wien, Zürich: \emph{Europaverlag} 1992, S. 74–75.} }\toendnotes[C]{\smallbreak}\pstart{}{\pb}Herrn KuK u. u. \textsc{Lieutenant}\pend{}\pstart{}\textsc{Dr. Richard Beer-Hofmann}\pend{}\pstart{}im \textsc{Kh. Landw.-Inf}-Regmt\pend{}\pstart{}\textsc{»\textcolor{pink}{Caslau}{}\ledrightnote{\textcolor{pink}{Caslau}}«
                  Nr 12}.\pend{}{\bigskip}\pstart
           \raggedleft{}{\pb}15. Juni 95\pend
           \pstart
           Lieber Richard, heut bin ich ſo ſchlecht aufgelegt, als wär ich in
                  \textcolor{pink}{\textsc{Caslau}}{}\ledrightnote{\textcolor{pink}{Caslau}}. – Einer der Gründe: ſchiefe Stellung in der Familie; Bemerkungen, daſs
               ich »ohne einen Kreuzer Geld zu haben« im So{\geminationm}er nach
                  \textcolor{pink}{\textsc{Kopenhagen}}{}\ledrightnote{\textcolor{pink}{Kopenhagen}} fahren will – Bemerkungen, die mir von dritter, nein vierter Seite
               zurückkommen. –\pend
           \pstart
           \textsc{\textcolor{blue}{Dörmann}{}\ledrightnote{\textcolor{blue}{Felix Dörmann}}} iſt da und erzählt viele Dinge von ſich – er hat 3 Stücke geſchrieben und hat
                  \introOben{}in \textcolor{pink}{Berlin}{}\ledrightnote{\textcolor{pink}{Berlin}}\introOben{} 65 Verhältniſſe gehabt. Ich übertreibe nicht. Er aber ja {\dots} a {\dots} a –\pend
           \pstart
           – Die \textcolor{green}{Kritik}{}\ledrightnote{→\textcolor{green}{Fanny Gröger, »Adhimukti«}} vom kleinen \textcolor{blue}{Kraus}{}\ledrightnote{\textcolor{blue}{Karl Kraus}} in dem {\pb}Abendblatt der
                  \textcolor{brown}{N. Fr. Pr.}{}\ledrightnote{\textcolor{brown}{Neue Freie Presse}} über die \textcolor{blue}{Gröger}{}\ledrightnote{\textcolor{blue}{Fanny Gröger}} haben Sie geleſen? Er benützt die Gelegenheit,
               uns (Sie, \textcolor{blue}{\textsc{Loris}}{}\ledrightnote{\textcolor{blue}{Hugo von Hofmannsthal}}{ }\introOben{}\textcolor{blue}{\textsc{Salten}}{}\ledrightnote{\textcolor{blue}{Felix Salten}}\introOben{} mich) in die Waden zu beißen.\strikeout{)} Wir werden
               noch ſchmerzlicheres zu überleben haben. –\pend
           \pstart
           \textcolor{green}{\textsc{Frauenlob}}{}\ledrightnote{\textcolor{green}{Frauenlob. Lustspiel in drei Aufzügen}} von Hrn. \textsc{\textcolor{blue}{Lothar}{}\ledrightnote{\textcolor{blue}{Rudolf Lothar}}} an der \textcolor{pink}{Burg}{}\ledrightnote{\textcolor{pink}{Burgtheater}}{ }\label{K_L00454_1v}\edtext{angenommen}{\lemma{\textnormal{\emph{angenommen}}}\Cendnote{\textnormal{Zu einer Aufführung kam es aber nicht.}}}\label{K_L00454_1h}. – Gerücht über
                  »\textcolor{green}{Liebelei}{}\ledrightnote{\textcolor{green}{Liebelei. Schauspiel in drei Akten}}«: es werde überhaupt nicht an der
                  \textcolor{pink}{Burg}{}\ledrightnote{\textcolor{pink}{Burgtheater}} zur Aufführung kommen. Entſtehung liegt
               nahe; werde \textcolor{blue}{Burckh.}{}\ledrightnote{\textcolor{blue}{Max Eugen Burckhard}} aufſuchen.\pend
           \pstart
           – Für den Abdruck der \textcolor{green}{\textsc{Kl. Komödie}}{}\ledrightnote{\textcolor{green}{Die kleine Komödie}}{ }{\pb}in
               der \textcolor{green}{\textsc{Freien Bühne}}{}\ledrightnote{\textcolor{green}{Neue Deutsche Rundschau}} will \textcolor{blue}{\textsc{Fischer}}{}\ledrightnote{\textcolor{blue}{Samuel Fischer}} mir 25, bitte, 25 Mark bezahlen. Ich hab ihm einen groben Brief
               geſchrieben – da mir ja nichts dran liegt. Was haben Sie gegen \textsc{\textcolor{blue}{Zasche}{}\ledrightnote{\textcolor{blue}{Theodor Zasche}}}? Er wird das ganz hübſch machen. – Die \textcolor{green}{Novelle}{}\ledrightnote{→\textcolor{green}{Die kleine Komödie}} zu datiren hat keinen Sinn; es kü{\geminationm}ert
               ſich doch keiner drum und ſieht aus wie eine Entſchuldigung. –\pend
           \pstart
           Ich ſchreibe an meinem \textcolor{green}{Stück}{}\ledrightnote{→\textcolor{green}{Freiwild. Schauspiel in 3 Akten}} – vorläufig ohne an eine
                  Aufführungs{\pb}möglichkeit zu denken. –\pend
           \pstart
           Meine Abſicht iſt, Anfang Juli in die \textcolor{pink}{böhm.}{}\ledrightnote{\textcolor{pink}{Böhmen}} Bäder zu reiſen und vor Mitte Juli in \textcolor{pink}{Iſchl}{}\ledrightnote{\textcolor{pink}{Bad Ischl}} zu ſein. – Wann wollen Sie nach \textcolor{pink}{München}{}\ledrightnote{\textcolor{pink}{München}} gehn? – Wie ſtehn Sie zu \textcolor{pink}{Kopenhagen}{}\ledrightnote{\textcolor{pink}{Kopenhagen}}? Beantworten Sie gütigſt. – \textcolor{blue}{Goldmann}{}\ledrightnote{\textcolor{blue}{Paul Goldmann}} wird im Auguſt Urlaub nehmen,
               genaueres unbekannt.\pend
           \pstart
           – Mein rechtes Ohr laß ich behandeln, das macht mich auch recht nervös. –\pend
           \pstart
           Leben Sie wohl, ſeien Sie herzlich gegrüßt.\pend
           \pstart Ihr \spacefill\mbox{Arthur.}\pend{}\endnumbering\briefempfaengerindex{Beer-Hofmann, Richard@\textsc{Beer-Hofmann, Richard}!zzzSchnitzler, Arthur@\emph{von Arthur Schnitzler}!1895-06-151@{15. 6. 1895}|)be}\mylabel{h}  \normalsize

\doendnotes{C}
\bigskip
\vfill

\clearpage

\footnotesize

\lohead{\textsc{register}}

% Definiere theindex-Environment komplett neu ohne reledmac
\makeatletter
\renewenvironment{theindex}{%
  \section*{\indexname}%
  \setlength{\parindent}{0pt}%
  \setlength{\parskip}{0pt plus 0.3pt}%
  \let\item\@idxitem
}{%
  \clearpage
}
\makeatother

\IfFileExists{\jobname-pw.ind}{\input{\jobname-pw.ind}}{}

\end{document}

      