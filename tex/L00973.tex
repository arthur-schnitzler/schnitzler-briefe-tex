%% latex-korrekturansicht-vorspann.tex
%% Vorspann für die Korrekturansicht.
%% Lädt die gemeinsame Datei latex-vorspann.tex mit gesetztem Schalter.

\newif\ifkorrekturansicht
\korrekturansichttrue

\input{../tex-inputs/latex-vorspann}


               \section[Richard Beer-Hofmann an Arthur Schnitzler, 12. 9. 1899]{ Richard Beer-Hofmann an Arthur Schnitzler, 12. 9. 1899}\nopagebreak\mylabel{v}\rehead{ }\normalsize\beginnumbering\briefempfaengerindex{Schnitzler, Arthur@\textsc{Schnitzler, Arthur}!zzzBeer-Hofmann, Richard@\emph{von Richard Beer-Hofmann}!1899-09-121@{12. 9. 1899}|(be} \toendnotes[C]{\smallbreak\pagebreak[2]} \Standort{CUL, Schnitzler, B 8.}
\physDesc{Brief, 1 Blatt, 2 Seiten
\newline{}Handschrift: schwarze Tinte, lateinische Kurrent\newline{}Ordnung: mit Bleistift von unbekannter Hand nummeriert:
                                    »141« }\buchAbdrucke{\weitereDrucke{Arthur Schnitzler, Richard Beer-Hofmann: \emph{Briefwechsel 1891–1931}. Hg. Konstanze Fliedl. Wien, Zürich: \emph{Europaverlag} 1992, S. 136–137.} }\toendnotes[C]{\smallbreak}\pstart
           \raggedleft{}{\pb}\textcolor{pink}{Vahrn}{}\ledrightnote{\textcolor{pink}{Vahrn}}{ }12/IX 99\pend
           \pstart
           Lieber Arthur! Ihre Karte gestern, heute Ihren Brief vom
                  9. erhalten. Ich habe ihn mehr errathen als gelesen; was heisst
                  \textcolor{gray}{durch allerlei.}{ }\textcolor{blue}{Hugo}{}\ledrightnote{\textcolor{blue}{Hugo von Hofmannsthal}}s Brief vom 7. daß er herko{\geminationm}en will habe ich gestern erhalten, und ihm telegrafirt
               er möge nur kommen. Ich arbeite täglich, und komme – wenn auch langsam vorwärts. In
               der »\textcolor{brown}{Zeit}{}\ledrightnote{\textcolor{brown}{Die Zeit. Wiener Wochenschrift}}« werden voraussichtlich nur die ersten
                  \label{K_L00973_1v}\edtext{\textcolor{green}{2. Cap.}{}\ledrightnote{→\textcolor{green}{Der Tod Georgs. Fragment}}}{\lemma{\textnormal{\emph{2. Cap.}}}\Cendnote{\textnormal{Es erschien nur das gekürzte \textcolor{green}{zweite Kapitel} in vier Teilen
                  zwischen 4. und 25. 11. 1899.}}}\label{K_L00973_1h} erscheinen. Das
               Ganze würden sie in \uline{10} Fortsetz. tranchiren müssen,
               und das Buch könnte erst Mitte Dez. erscheinen. Das wäre zu langweilig. Wer wird also
               auf dem Titel figuriren? Schon entschieden? Ich {\pb}mache Sie aufmerksam: In \textcolor{pink}{München}{}\ledrightnote{\textcolor{pink}{München}} geht um 9.10 Nachts ein Zug
               ab, der um 4.36 Früh in \textcolor{pink}{Brixen}{}\ledrightnote{\textcolor{pink}{Brixen}} ist. Von da 20
               Minuten Wagen nach \strikeout{V}{ }\textcolor{pink}{Vahrn}{}\ledrightnote{\textcolor{pink}{Vahrn}}. Außerdem ein N. S. Express, der um
                  9.55{ }\substVorne{}\textsuperscript{Früh}\substDazwischen{}Vorm\substHinten{} von \textcolor{pink}{München}{}\ledrightnote{\textcolor{pink}{München}} abgeht, um 3.02
                  Nachm. in \textcolor{pink}{Franzensfeste}{}\ledrightnote{\textcolor{pink}{Franzensfeste}} ist; \strikeout{von} (in \textcolor{pink}{Brixen}{}\ledrightnote{\textcolor{pink}{Brixen}} hält
               er nicht). Von \textcolor{pink}{Franzensfeste}{}\ledrightnote{\textcolor{pink}{Franzensfeste}} mit dem Wagen circa
               9–10 Kilom. hieher. Es ist hier angenehm, ruhig, bei der table d’hôte nur \textcolor{blue}{Paula}{}\ledrightnote{\textcolor{blue}{Paula Beer-Hofmann}} und ich inbegriffen \uline{4} Personen. Abends, wie bei \textcolor{pink}{Petter}{}\ledrightnote{\textcolor{pink}{Hotel und Pension Rudolfshöhe (Leopold Petter)}}, an
               separaten Tischen. Lärchen und Edelkastanienwald. Gegenüber Weingelände. Vielleicht
                  ko{\geminationm}en Sie? Man soll ja doch so spät als möglich nach
                  \textcolor{pink}{Wien}{}\ledrightnote{\textcolor{pink}{Wien}}?\pend
           \pstart
           Herzlichst{\\[\baselineskip]}Ihr{\\[\baselineskip]}\spacefill\mbox{Richard}\pend
           \leftskip=0em{}\endnumbering\briefempfaengerindex{Schnitzler, Arthur@\textsc{Schnitzler, Arthur}!zzzBeer-Hofmann, Richard@\emph{von Richard Beer-Hofmann}!1899-09-121@{12. 9. 1899}|)be}\mylabel{h}  \normalsize

\doendnotes{C}
\bigskip
\vfill

\clearpage

\footnotesize

\lohead{\textsc{register}}

% Definiere theindex-Environment komplett neu ohne reledmac
\makeatletter
\renewenvironment{theindex}{%
  \section*{\indexname}%
  \setlength{\parindent}{0pt}%
  \setlength{\parskip}{0pt plus 0.3pt}%
  \let\item\@idxitem
}{%
  \clearpage
}
\makeatother

\IfFileExists{\jobname-pw.ind}{\input{\jobname-pw.ind}}{}

\end{document}

      