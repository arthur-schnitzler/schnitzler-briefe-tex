%% latex-korrekturansicht-vorspann.tex
%% Vorspann für die Korrekturansicht.
%% Lädt die gemeinsame Datei latex-vorspann.tex mit gesetztem Schalter.

\newif\ifkorrekturansicht
\korrekturansichttrue

\input{../tex-inputs/latex-vorspann}


               \section[Arthur Schnitzler an Richard Beer-Hofmann, 22. 7. 1893]{ Arthur Schnitzler an Richard Beer-Hofmann, 22. 7. 1893}\nopagebreak\mylabel{v}\rehead{ }\normalsize\beginnumbering\briefempfaengerindex{Beer-Hofmann, Richard@\textsc{Beer-Hofmann, Richard}!zzzSchnitzler, Arthur@\emph{von Arthur Schnitzler}!1893-07-221@{22. 7. 1893}|(be} \toendnotes[C]{\smallbreak\pagebreak[2]} \Standort{YCGL, MSS 31.}
\physDesc{Brief, 1 Blatt (Briefpapier mit Trauerrand), 4 Seiten, Umschlag mit Trauerrand
\newline{}Handschrift: schwarze Tinte, deutsche Kurrent\newline{}Versand: 1) Stempel: »\nobreak{}Wien 9/3, 22. 7. 93, 2–3 M\nobreak{}«.  2) Stempel: »\nobreak{}\oindex{Salzburg@\textbf{Salzburg}, \emph{Besiedelter Ort (A.BSO)}|pwk}Salzburg Stadt, 23 7 93, 2 N\nobreak{}«. 3) mit schwarzer Tinte von unbekannter Hand die beiden Adresszeilen
                                 gestrichen und ersetzt durch: »\noindent{}\textsc{Post Restante}{ / }\textsc{\textcolor{pink}{Salzburg}}«}\buchAbdrucke{\weitereDrucke{Arthur Schnitzler, Richard Beer-Hofmann: \emph{Briefwechsel 1891–1931}. Hg. Konstanze Fliedl. Wien, Zürich: \emph{Europaverlag} 1992, S. 47.} }\toendnotes[C]{\smallbreak}\pstart{}{\pb}Herrn \textsc{Dr. Richard
                     Beer-Hofmann}\pend{}\pstart{}\textsc{\textcolor{pink}{Ischl}{}\ledrightnote{\textcolor{pink}{Bad Ischl}}}\pend{}\pstart{}\textsc{\textcolor{pink}{Schulgasse 8}{}\ledrightnote{\textcolor{pink}{Schulgasse}}}.\pend{}{\bigskip}\pstart
           \raggedleft{}{\pb}\textcolor{pink}{Wien}{}\ledrightnote{\textcolor{pink}{Wien}}{ }22. 7. 93\pend
           \pstart{}Lieber Richard,\pend\pstart
           die Abſchrift Ihrer \textcolor{green}{Novelle}{}\ledrightnote{→\textcolor{green}{Das Kind}}
               dürfte Montag oder Dinſtag beendet \strikeout{wurde} werden, obwohl ſie erſt heute begonnen wird. Mein
               deſignirter Abſchreiber war ausgezogen – und ſchreibt nicht mehr; ein zweiter, den er
               mir empfahl, refuſirte gleichfalls und empfahl mir einen \textcolor{blue}{dritten}{}\ledrightnote{→\textcolor{blue}{?? [Schreibkraft für Arthur Schnitzler]}}, welcher heute bei mir war, einen {\pb}guten Eindruck auf mich machte, u dem ich endlich \textcolor{green}{Das Kind}{}\ledrightnote{\textcolor{green}{Das Kind}} übergab. –\pend
           \pstart
           War \textcolor{green}{was}{}\ledrightnote{→\textcolor{green}{Aus Ischl}} in der alten \textcolor{green}{Preſſe}{}\ledrightnote{\textcolor{green}{Die Presse}} über \textcolor{green}{Abſch.\textsc{s}.}{}\ledrightnote{\textcolor{green}{Abschiedssouper}}? – Was ſagen Sie zu der \textcolor{green}{\textcolor{green}{Allgem. Zeitung}{}\ledrightnote{\textcolor{green}{Wiener Allgemeine Zeitung}}}{}\ledrightnote{→\textcolor{green}{Ischler Brief}}? Champagner – alſo \textcolor{blue}{\textsc{Murger}}{}\ledrightnote{\textcolor{blue}{Henri Murger}} – weil ſie beim \textcolor{blue}{\textsc{Murger}}{}\ledrightnote{\textcolor{blue}{Henri Murger}} verhungern. Soll ich mich bei \textcolor{blue}{\textsc{Osten}}{}\ledrightnote{\textcolor{blue}{Heinrich Osten}} bedanken? – War im \textcolor{brown}{\textsc{Börsencourier}}{}\ledrightnote{\textcolor{brown}{Berliner Börsen-Courier}} was? Den krieg’ ich auch nie zu Geſichte. –\pend
           \pstart
           Neulich machte ich mit \textcolor{blue}{\textsc{Salten}}{}\ledrightnote{\textcolor{blue}{Felix Salten}} eine wunderſchöne \textsc{Bicycletour} von \textcolor{pink}{\textsc{Klosterneubg}}{}\ledrightnote{\textcolor{pink}{Klosterneuburg}} nach \textcolor{pink}{\textsc{Tulln}}{}\ledrightnote{\textcolor{pink}{Tulln an der Donau}}{ }{\pb}am Donauufer. Ihr
               müſſt unbedingt fahren lernen –\pend
           \pstart
           – Meine Sti{\geminationm}ung iſt recht ſchlecht; die Luft iſt
               drückend und unausſtehlich, und manche \textsc{Hypochondrien} quälen
               mich. Geſchrieben – noch nichts, die Zeit iſt ſo zerſplittert; ein ewiges Hin
                  un\textcolor{gray}{d} Her von der Klinik auf die Druckerei – in die \textcolor{pink}{Grillparzerſtr.}{}\ledrightnote{\textcolor{pink}{Grillparzerstraße}} – auf den \textcolor{pink}{Burgring}{}\ledrightnote{\textcolor{pink}{Burgring}} – zu meinem \textcolor{blue}{Schwager}{}\ledrightnote{→\textcolor{blue}{Markus Hajek}} – auf den \textcolor{pink}{Kahlenberg}{}\ledrightnote{\textcolor{pink}{Kahlenberg}}
               u. ſ. w. –\pend
           \pstart
           Was gibts \substVorne{}\textsuperscript{aus}\substDazwischen{}in\substHinten{}{ }\textcolor{pink}{\textsc{Ischl}}{}\ledrightnote{\textcolor{pink}{Bad Ischl}}? – Sprachen {\pb}Sie \textcolor{blue}{Benedikt}{}\ledrightnote{\textcolor{blue}{Markus Benedict}{\newline}\textcolor{blue}{Marianne Benedict}}’s häufig? – Was macht der \textcolor{green}{Götterliebling}{}\ledrightnote{\textcolor{green}{Der Tod Georgs}}? – Hat \textcolor{blue}{Freund}{}\ledrightnote{\textcolor{blue}{Carl Freund}}{ }ſchon der \textcolor{blue}{\textsc{Fl.}}{}\ledrightnote{\textcolor{blue}{Bertha Flegmann}} geantwortet? – Wird noch viel über das \textcolor{green}{Stück}{}\ledrightnote{→\textcolor{green}{Abschiedssouper}} geſchimpft? – Wirds noch einmal aufgeführt? –
               Sprechen Sie \textcolor{blue}{\textsc{Jarno}}{}\ledrightnote{\textcolor{blue}{Josef Jarno}}? – Wie gehts der kleinen \textcolor{blue}{\textsc{Wreden}}{}\ledrightnote{\textcolor{blue}{Grethe Wreden}}? – Sie werden allerdings keine Luſt haben, es zu erforſchen. – Iſt die \textcolor{blue}{\textsc{Griebl}}{}\ledrightnote{\textcolor{blue}{Karoline Gribl}} und die alte \textcolor{blue}{\textsc{Friese}}{}\ledrightnote{\textcolor{blue}{Josefine Skura}}{ }ſchon ins Kloſter gegangen?\pend
           \pstart
           Schreiben Sie bald, we{\geminationn} auch wenig\pend
           \pstart Herzlich Ihr \spacefill\mbox{ArthurSch}\pend{}\pstart
           \noindent{}\label{T_L00240_1v}\edtext{Senden Sie mir das \textcolor{green}{Iſchler Wochenblatt}{}\ledrightnote{\textcolor{green}{Ischler Wochenblatt}} mit der \textcolor{green}{Kritik}{}\ledrightnote{→\textcolor{green}{?? [Kritik im Ischler Wochenblatt]}}}{\lemma{\textnormal{\emph{Senden … Kritik}}}\Cendnote{\textnormal{auf der ersten Seite neben dem Datum
                     auf dem Kopf.}}}\label{T_L00240_1h}\pend
           \endnumbering\briefempfaengerindex{Beer-Hofmann, Richard@\textsc{Beer-Hofmann, Richard}!zzzSchnitzler, Arthur@\emph{von Arthur Schnitzler}!1893-07-221@{22. 7. 1893}|)be}\mylabel{h}  \normalsize

\doendnotes{C}
\bigskip
\vfill

\clearpage

\footnotesize

\lohead{\textsc{register}}

% Definiere theindex-Environment komplett neu ohne reledmac
\makeatletter
\renewenvironment{theindex}{%
  \section*{\indexname}%
  \setlength{\parindent}{0pt}%
  \setlength{\parskip}{0pt plus 0.3pt}%
  \let\item\@idxitem
}{%
  \clearpage
}
\makeatother

\IfFileExists{\jobname-pw.ind}{\input{\jobname-pw.ind}}{}

\end{document}

      