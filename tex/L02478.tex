%% latex-korrekturansicht-vorspann.tex
%% Vorspann für die Korrekturansicht.
%% Lädt die gemeinsame Datei latex-vorspann.tex mit gesetztem Schalter.

\newif\ifkorrekturansicht
\korrekturansichttrue

\input{../tex-inputs/latex-vorspann}


               \section[Georg Brandes an Arthur Schnitzler, 28. 8. 1926]{ Georg Brandes an Arthur Schnitzler, 28. 8. 1926}\nopagebreak\mylabel{v}\rehead{ }\normalsize\beginnumbering\briefempfaengerindex{Schnitzler, Arthur@\textsc{Schnitzler, Arthur}!zzzBrandes, Georg@\emph{von Georg Brandes}!1926-08-281@{28. 8. 1926}|(be} \toendnotes[C]{\smallbreak\pagebreak[2]} \Standort{CUL, Schnitzler, B 17.}
\physDesc{Postkarte
\newline{}Handschrift: Bleistift, lateinische Kurrent\newline{}Versand: Stempel: »\nobreak{}\oindex{Kopenhagen@\textbf{Kopenhagen}, \emph{Besiedelter Ort (A.BSO)}|pwk}Københaven, 2\textcolor{gray}{8}. VIII. 1926\nobreak{}«.  
\newline{}Schnitzler: 1) mit Bleistift datiert: »28/8« 2) mit rotem Buntstift vereinzelte Unterstreichungen\newline{}Ordnung: mit Bleistift von unbekannter Hand nummeriert:
                                        »63« }\buchAbdrucke{\weitereDrucke{Georg Brandes, Arthur Schnitzler: \emph{Ein Briefwechsel}. Hg. Kurt Bergel. Bern: \emph{Francke} 1956, S. 153.} }\toendnotes[C]{\smallbreak}\pstart{}{\pb}Herrn Dr. Arthur
                        Schnitzler\pend{}\pstart{}\textcolor{pink}{Sternwartestrasse 71}{}\ledrightnote{\textcolor{pink}{Sternwartestraße}}\pend{}\pstart{}\textcolor{pink}{Wien XVIII}{}\ledrightnote{\textcolor{pink}{VIII., Josefstadt}}\pend{}{\bigskip}\pstart
           \raggedleft{}{\pb}\textcolor{pink}{Kopenhagen}{}\ledrightnote{\textcolor{pink}{Kopenhagen}}{ }\textcolor{blue}{Goethe}{}\ledrightnote{\textcolor{blue}{Johann Wolfgang von Goethe}}s Geburtstag 1926
                    \pend
           \pstart
           Verehrter Freund Seit April 1925 hab ich Sie nicht
                    gesehen, und es ist mir, als sah ich Sie gestern. So lebhaft stehen Sie mir vor
                    Augen. Seitdem haben Sie eine weite Reise nach den \textcolor{pink}{canarischen Inseln}{}\ledrightnote{\textcolor{pink}{Gran Canaria}} gemacht, sich freundlich meiner erinnert, mir die
                    sonderbar tiefsinnige \textcolor{green}{Traumnovelle}{}\ledrightnote{\textcolor{green}{Traumnovelle}} zugesandt,
                    vermutlich noch anderes hervorgebracht. Ich bitte nur, mich nicht zu vergessen;
                    ich war in \textcolor{pink}{Karlsbad}{}\ledrightnote{\textcolor{pink}{Karlsbad}}, \textcolor{pink}{Prag}{}\ledrightnote{\textcolor{pink}{Prag}}, \textcolor{pink}{Schandau}{}\ledrightnote{\textcolor{pink}{Bad Schandau}}, meiner
                    Gesundheit halber, und bin nicht krank, arbeite weiter mit Forschungen über \textcolor{blue}{Petrus}{}\ledrightnote{\textcolor{blue}{Simon Petrus}} u. \textcolor{blue}{Paulus}{}\ledrightnote{\textcolor{blue}{Paulus}}. Ueber \textcolor{blue}{\uline{Petrus}}{}\ledrightnote{\textcolor{blue}{Simon Petrus}} erschien vor langer Zeit ein \textcolor{green}{Büchlein}{}\ledrightnote{→\textcolor{green}{Petrus}}, aber da mein \textcolor{blue}{Verleger}{}\ledrightnote{→\textcolor{blue}{Erich Reiss}} in \textcolor{pink}{Berlin}{}\ledrightnote{\textcolor{pink}{Berlin}}
                    bankerot ist, wurde es nicht deutsch publicirt.\pend
           \pstart
           Es war schön, daß ich in \textcolor{pink}{Wien}{}\ledrightnote{\textcolor{pink}{Wien}} Ihr Gast sein
                    durfte. Ihre junge \textcolor{blue}{Tochter}{}\ledrightnote{→\textcolor{blue}{Lili Schnitzler}}
                    war {\pb}war Schmuck des
                    Hauses.\pend
           \pstart
           Ich bitte, gelegentlich \textcolor{blue}{Beer-Hofmann}{}\ledrightnote{\textcolor{blue}{Richard Beer-Hofmann}} und
                    seine \textcolor{blue}{Gemahlin}{}\ledrightnote{→\textcolor{blue}{Paula Beer-Hofmann}} sehr
                    herzlich von mir zu grüssen.\pend
           \pstart
           Ich weiss nicht, ob Sie Zeit zum Lesen haben. Sonst würde ich Ihnen \textcolor{green}{Kyra Kyralina}{}\ledrightnote{\textcolor{green}{Kyra Kyralina}} von dem \textcolor{pink}{Rumänen}{}\ledrightnote{\textcolor{pink}{Rumänien}}{ }\textcolor{blue}{Panit Istrati}{}\ledrightnote{\textcolor{blue}{Panaït Istrati}} empfehlen. Er schreibt
                    französisch und hat grosse Frische.\pend
           \pstart Ihr getreuer Freund \spacefill\mbox{Georg B}\pend{}\endnumbering\briefempfaengerindex{Schnitzler, Arthur@\textsc{Schnitzler, Arthur}!zzzBrandes, Georg@\emph{von Georg Brandes}!1926-08-281@{28. 8. 1926}|)be}\mylabel{h}  \normalsize

\doendnotes{C}
\bigskip
\vfill

\clearpage

\footnotesize

\lohead{\textsc{register}}

% Definiere theindex-Environment komplett neu ohne reledmac
\makeatletter
\renewenvironment{theindex}{%
  \section*{\indexname}%
  \setlength{\parindent}{0pt}%
  \setlength{\parskip}{0pt plus 0.3pt}%
  \let\item\@idxitem
}{%
  \clearpage
}
\makeatother

\IfFileExists{\jobname-pw.ind}{\input{\jobname-pw.ind}}{}

\end{document}

      