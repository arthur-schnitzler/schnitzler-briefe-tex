%% latex-korrekturansicht-vorspann.tex
%% Vorspann für die Korrekturansicht.
%% Lädt die gemeinsame Datei latex-vorspann.tex mit gesetztem Schalter.

\newif\ifkorrekturansicht
\korrekturansichttrue

\input{../tex-inputs/latex-vorspann}


               \section[Arthur Schnitzler an Richard Beer-Hofmann, 26. 2. 1902]{ Arthur Schnitzler an Richard Beer-Hofmann, 26. 2. 1902}\nopagebreak\mylabel{v}\rehead{ }\normalsize\beginnumbering\briefempfaengerindex{Beer-Hofmann, Richard@\textsc{Beer-Hofmann, Richard}!zzzSchnitzler, Arthur@\emph{von Arthur Schnitzler}!1902-02-261@{26. 2. 1902}|(be} \toendnotes[C]{\smallbreak\pagebreak[2]} \Standort{YCGL, MSS 31.}
\physDesc{Brief, 1 Blatt, 3 Seiten, Umschlag
\newline{}Handschrift: Bleistift, deutsche Kurrent\newline{}Versand: 1) Stempel: »\nobreak{}Wien, 27. \textcolor{gray}{2}. 0\textcolor{gray}{2}, \textcolor{gray}{11}V\nobreak{}«.  2) Stempel: »\nobreak{}\oindex{Rodaun@\textbf{Rodaun}, \emph{Teil eines besiedelten Ortes (A.BSOX)}|pwk}{\pb}Rodaun, 27 2 02\nobreak{}«. }\buchAbdrucke{\weitereDrucke{Arthur Schnitzler, Richard Beer-Hofmann: \emph{Briefwechsel 1891–1931}. Hg. Konstanze Fliedl. Wien, Zürich: \emph{Europaverlag} 1992, S. 157.} }\toendnotes[C]{\smallbreak}\pstart{}{\pb}Herrn \textsc{Dr Richard
                     Beer-Hofmann}\pend{}\pstart{}\textcolor{pink}{\textsc{Rodaun} bei Lieſing}{}\ledrightnote{\textcolor{pink}{Rodaun}}\pend{}\pstart{}\textcolor{pink}{\textsc{Liesinger}ſtraße 1}{}\ledrightnote{\textcolor{pink}{Liesingerstraße}}. \pend{}{\bigskip}\pstart
           \raggedleft{}{\pb}\uline{Mittwoch}.\pend
           \pstart
           lieber Richard, ich bin vorläufig wieder nach \textcolor{pink}{Wien}{}\ledrightnote{\textcolor{pink}{Wien}} gezogen, habe noch \label{K_L01206_1v}\edtext{keine »Villa«}{\lemma{\textnormal{\emph{keine »Villa«}}}\Cendnote{\textnormal{Er war auf Suche nach einer
                  gemeinsamen Unterkunft für seine Familie.}}}\label{K_L01206_1h}, bin ſehr enervirt. Die {\pb}\textcolor{blue}{Damen}{}\ledrightnote{→\textcolor{blue}{Olga Schnitzler}{\newline}→\textcolor{blue}{Elisabeth Steinrück}}{ }ſind zu \textcolor{pink}{\textsc{Schönberger}}{}\ledrightnote{\textcolor{pink}{Zum goldenen Stern}} gezogen, da es im \textcolor{pink}{Kurhaus}{}\ledrightnote{\textcolor{pink}{Cur- und Wasserheilanstalt Hinterbrühl}} auf die Dauer
               abſolut nicht auszuhalten war.\pend
           \pstart
           Ich weiſs noch nicht, wa{\geminationn} ich nach \textcolor{pink}{Rodaun}{}\ledrightnote{\textcolor{pink}{Rodaun}}{ }{\pb}komme, grüßen Sie bitte auch \textcolor{blue}{Hugo}{}\ledrightnote{\textcolor{blue}{Hugo von Hofmannsthal}}.\pend
           \pstart
           Herzlichſt{\\[\baselineskip]}Ihr{\\[\baselineskip]}\spacefill\mbox{A.}\pend
           \leftskip=0em{}\endnumbering\briefempfaengerindex{Beer-Hofmann, Richard@\textsc{Beer-Hofmann, Richard}!zzzSchnitzler, Arthur@\emph{von Arthur Schnitzler}!1902-02-261@{26. 2. 1902}|)be}\mylabel{h}  \normalsize

\doendnotes{C}
\bigskip
\vfill

\clearpage

\footnotesize

\lohead{\textsc{register}}

% Definiere theindex-Environment komplett neu ohne reledmac
\makeatletter
\renewenvironment{theindex}{%
  \section*{\indexname}%
  \setlength{\parindent}{0pt}%
  \setlength{\parskip}{0pt plus 0.3pt}%
  \let\item\@idxitem
}{%
  \clearpage
}
\makeatother

\IfFileExists{\jobname-pw.ind}{\input{\jobname-pw.ind}}{}

\end{document}

      