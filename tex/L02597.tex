%% latex-korrekturansicht-vorspann.tex
%% Vorspann für die Korrekturansicht.
%% Lädt die gemeinsame Datei latex-vorspann.tex mit gesetztem Schalter.

\newif\ifkorrekturansicht
\korrekturansichttrue

\input{../tex-inputs/latex-vorspann}


               \section[Arthur Schnitzler an Marie Herzfeld, 20. 4. 1909]{ Arthur Schnitzler an Marie Herzfeld, 20. 4. 1909}\nopagebreak\mylabel{v}\rehead{ }\normalsize\beginnumbering\briefempfaengerindex{Herzfeld, Marie@\textsc{Herzfeld, Marie}!zzzSchnitzler, Arthur@\emph{von Arthur Schnitzler}!1909-04-201@{20. 4. 1909}|(be} \toendnotes[C]{\smallbreak\pagebreak[2]} \Standort{DLA, A:Schnitzler, HS.1985.1.993.}
\physDesc{Brief, 1 Blatt, 1 Seite, maschineller Durchschlag
\newline{}Schreibmaschine
\newline{}Handschrift: 1) Bleistift, lateinische Kurrent (\noindent{}Vermerk »\textsc{Herzfeld}«)\hspace{1em}2) roter Buntstift (\noindent{}mit rotem Buntstift drei Unterstreichungen)\hspace{1em}}\toendnotes[C]{\smallbreak}\pstart
           \raggedleft{}{\pb}20. April 09.\pend
           \pstart{}Verehrtes Fräulein,\pend\pstart
           \textcolor{blue}{Frau Tesi}{}\ledrightnote{\textcolor{blue}{Anna Rotenstern-Tesi}} wird von ihrem Gedächtnis getäuscht,
               wenn Sie Ihnen sagte, dass ich ihr von der \textcolor{green}{Revolutionshochzeit}{}\ledrightnote{\textcolor{green}{Revolutionsbryllup. Skuespil i tre Akter}} gesprochen hätte. Ich habe von dem \textcolor{green}{Stück}{}\ledrightnote{→\textcolor{green}{Revolutionsbryllup. Skuespil i tre Akter}} schon das beste gehört, habe es aber
               bisher weder gelesen noch gesehen. Dass \textcolor{blue}{Frau Tesi}{}\ledrightnote{\textcolor{blue}{Anna Rotenstern-Tesi}}
               einiges von mir übersetzt hat stimmt. Meine direkten Verhandlungen fandem mit ihrem
               Gatten Herrn \textcolor{blue}{Rottenstern Swestitsch}{}\ledrightnote{\textcolor{blue}{Peter Rotenstern}} statt. \textcolor{blue}{Beide}{}\ledrightnote{→\textcolor{blue}{Anna Rotenstern-Tesi}{\newline}→\textcolor{blue}{Peter Rotenstern}} scheinen mir, soweit
               es die Konventionsverhältnisse zwischen \textcolor{pink}{Oesterreich}{}\ledrightnote{\textcolor{pink}{Österreich}} und \textcolor{pink}{Russland}{}\ledrightnote{\textcolor{pink}{Russland}} zulassen,
               verlässliche Menschen. \label{K_L02597-2v}\edtext{Ich habe von
               ihnen, sowohl für \textcolor{green}{Zwischenspiel}{}\ledrightnote{\textcolor{green}{Zwischenspiel. Komödie in drei Akten}} als für \textcolor{green}{Ruf des Lebens}{}\ledrightnote{\textcolor{green}{Der Ruf des Lebens. Schauspiel in drei Akten}}, wenn ich mich recht erinnere auch
               für den \textcolor{green}{einsamen Weg}{}\ledrightnote{\textcolor{green}{Der einsame Weg. Schauspiel in fünf Akten}}}{\lemma{\textnormal{\emph{Ich … Weg}}}\Cendnote{\textnormal{Die \emph{\textcolor{green}{Übersetzung}} des \emph{\textcolor{green}{Zwischenspiels}} erschien 1905, \textcolor{green}{jene} von \emph{\textcolor{green}{Der Ruf des Lebens}}{ }1906 und \textcolor{green}{jene} von
                     \emph{\textcolor{green}{Der einsame Weg}}{ }1904.}}}\label{K_L02597-2h} einige recht minimale Summen, / je 300 Kronen/ als
               Tantiemengarantie erhalten. Weitere Gelder flossen mir nie zu., was aber wie gesagt
               an den traurigen Rechtsverhältnissen zwischen \textcolor{pink}{Russland}{}\ledrightnote{\textcolor{pink}{Russland}} und \textcolor{pink}{Oesterreich}{}\ledrightnote{\textcolor{pink}{Österreich}} liegen mag. Wie
               es scheint haben andre \textcolor{pink}{österr.}{}\ledrightnote{\textcolor{pink}{Österreich}} und \textcolor{pink}{deutsche}{}\ledrightnote{\textcolor{pink}{Deutschland}} Autoren auch keine bessern Erfahrungen
               gemacht.\pend
           \endnumbering\briefempfaengerindex{Herzfeld, Marie@\textsc{Herzfeld, Marie}!zzzSchnitzler, Arthur@\emph{von Arthur Schnitzler}!1909-04-201@{20. 4. 1909}|)be}\mylabel{h}  \normalsize

\doendnotes{C}
\bigskip
\vfill

\clearpage

\footnotesize

\lohead{\textsc{register}}

% Definiere theindex-Environment komplett neu ohne reledmac
\makeatletter
\renewenvironment{theindex}{%
  \section*{\indexname}%
  \setlength{\parindent}{0pt}%
  \setlength{\parskip}{0pt plus 0.3pt}%
  \let\item\@idxitem
}{%
  \clearpage
}
\makeatother

\IfFileExists{\jobname-pw.ind}{\input{\jobname-pw.ind}}{}

\end{document}

      