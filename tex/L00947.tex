%% latex-korrekturansicht-vorspann.tex
%% Vorspann für die Korrekturansicht.
%% Lädt die gemeinsame Datei latex-vorspann.tex mit gesetztem Schalter.

\newif\ifkorrekturansicht
\korrekturansichttrue

\input{../tex-inputs/latex-vorspann}


               \section[Arthur Schnitzler an Hugo von Hofmannsthal, 18. 7. 1899]{ Arthur Schnitzler an Hugo von Hofmannsthal,
                    18. 7. 1899}\nopagebreak\mylabel{v}\rehead{ }\normalsize\beginnumbering\briefempfaengerindex{Hofmannsthal, Hugo von@\textsc{Hofmannsthal, Hugo von}!zzzSchnitzler, Arthur@\emph{von Arthur Schnitzler}!1899-07-182@{18. 7. 1899}|(be} \toendnotes[C]{\smallbreak\pagebreak[2]} \Standort{FDH, Hs-30885,84.}
\physDesc{Briefkarte
\newline{}Handschrift: Bleistift, deutsche Kurrent\newline{}Ordnung: mit Bleistift die Jahreszahl ergänzt: »99« wahrscheinlich
                                erst bei der Durchsicht der Briefe 1929 ergänzt }\buchAbdrucke{\weitereDrucke{Hugo von Hofmannsthal, Arthur Schnitzler: \emph{Briefwechsel}. Hg. Therese Nickl und Heinrich Schnitzler. Frankfurt am Main: \emph{S. Fischer} 1964, S. 126.} }\toendnotes[C]{\smallbreak}\pstart
           \raggedleft{}{\pb}18. 7.\pend
           \pstart
           lieber Hugo, ich bin heut Früh hier angeko{\geminationm}en. \introOben{}Meine\introOben{}{ }\textcolor{blue}{Mutter}{}\ledrightnote{→\textcolor{blue}{Louise Schnitzler}} und \textcolor{blue}{Schweſter}{}\ledrightnote{→\textcolor{blue}{Gisela Hajek}} wohnen hier. – Habe
                    Nachmitt\textcolor{gray}{ag} mit \textcolor{blue}{Schwager}{}\ledrightnote{→\textcolor{blue}{Rudolf Burger}} u
                        \textcolor{blue}{Schweſter}{}\ledrightnote{→\textcolor{blue}{Caroline Burger}} (von \textcolor{blue}{\uline{ihr}}{}\ledrightnote{→\textcolor{blue}{Marie Reinhard}}) am See ein Rendezvous. – Heut iſt der \uline{18}. – – Warte auf Nachricht von \textcolor{blue}{Richard}{}\ledrightnote{\textcolor{blue}{Richard Beer-Hofmann}}, ob er nicht arbeitet (eine Karte deutet es an) – bevor ich
                    ihn beſuche. – Bleibe mindeſtens 8 Tage hier. – Ob ich meine Radtour
                    bis 1. Sept. hinausſchiebe, fraglich. – Auch \textcolor{blue}{Salten}{}\ledrightnote{\textcolor{blue}{Felix Salten}} wollte ſie mitmachen. – Keiner
                    bindet {\pb}den andern. Im Auguſt{ }ſehn wir
                    uns jedenfalls, ko{\geminationm}e ins \textcolor{pink}{Salzka{\geminationm}ergut}{}\ledrightnote{\textcolor{pink}{Salzkammergut}} – wäre ſchön,
                        we{\geminationn} wir zusa{\geminationm}en
                    wären u jeder arbeitete.\pend
           \pstart
           – Will jetzt gleich, in dieſer Minute, mein \textcolor{green}{Stück}{}\ledrightnote{→\textcolor{green}{Der Schleier der Beatrice. Schauspiel in fünf Akten}} hervornehmen. – Was iſt das \textcolor{green}{Ihre}{}\ledrightnote{→\textcolor{green}{Das Bergwerk zu Falun}}? Historisch? Was neues?
                    Neue Idee? Ich freue mich dſs Sie in Sti{\geminationm}ung ſind.
                    Bitte gleich wieder eine Zeile.\pend
           \pstart Von Herzen Ihr \spacefill\mbox{Arth}\pend{}\pstart
           \noindent{}\textcolor{pink}{\textsc{Velden, Pension
                                Pundschu}}{}\ledrightnote{\textcolor{pink}{Pension Pundschu}}\pend
           \endnumbering\briefempfaengerindex{Hofmannsthal, Hugo von@\textsc{Hofmannsthal, Hugo von}!zzzSchnitzler, Arthur@\emph{von Arthur Schnitzler}!1899-07-182@{18. 7. 1899}|)be}\mylabel{h}  \normalsize

\doendnotes{C}
\bigskip
\vfill

\clearpage

\footnotesize

\lohead{\textsc{register}}

% Definiere theindex-Environment komplett neu ohne reledmac
\makeatletter
\renewenvironment{theindex}{%
  \section*{\indexname}%
  \setlength{\parindent}{0pt}%
  \setlength{\parskip}{0pt plus 0.3pt}%
  \let\item\@idxitem
}{%
  \clearpage
}
\makeatother

\IfFileExists{\jobname-pw.ind}{\input{\jobname-pw.ind}}{}

\end{document}

      