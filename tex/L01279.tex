%% latex-korrekturansicht-vorspann.tex
%% Vorspann für die Korrekturansicht.
%% Lädt die gemeinsame Datei latex-vorspann.tex mit gesetztem Schalter.

\newif\ifkorrekturansicht
\korrekturansichttrue

\input{../tex-inputs/latex-vorspann}


               \section[Richard Beer-Hofmann an Arthur Schnitzler, 26. 3.  1903]{ Richard Beer-Hofmann an Arthur Schnitzler,
               26. 3.  1903}\nopagebreak\mylabel{v}\rehead{ }\normalsize\beginnumbering\briefempfaengerindex{Schnitzler, Arthur@\textsc{Schnitzler, Arthur}!zzzBeer-Hofmann, Richard@\emph{von Richard Beer-Hofmann}!1903-03-261@{26. 3. 1903}|(be} \toendnotes[C]{\smallbreak\pagebreak[2]} \Standort{CUL, Schnitzler, B 8.}
\physDesc{Postkarte
\newline{}Handschrift: schwarze Tinte, lateinische Kurrent\newline{}Versand: 1) Stempel: »\nobreak{}\oindex{Perchtoldsdorf@\textbf{Perchtoldsdorf}, \emph{https://www.geonames.org/ontologyA.ADM3}|pwk}Perchtoldsdorf, 26. 3. 03, 10-12V\nobreak{}«.  2) Stempel: »\nobreak{}\oindex{IX., Alsergrund@\textbf{IX., Alsergrund}, \emph{Bezirk (A.BZK)}|pwk}9/3 Wien 72, 26. 3. 03, 7.N, Bestellt\nobreak{}«. 3) die Adresse von \textcolor{blue}{Hofmannsthal} geschrieben\newline{}Ordnung: mit Bleistift von unbekannter Hand nummeriert: »178« }\buchAbdrucke{\weitereDrucke{Arthur Schnitzler, Richard Beer-Hofmann: \emph{Briefwechsel 1891–1931}. Hg. Konstanze Fliedl. Wien, Zürich: \emph{Europaverlag} 1992, S. 162.} }\toendnotes[C]{\smallbreak}\pstart{}{\pb}Herrn Dr. Arthur
                  Schnitzler\pend{}\pstart{}\textcolor{green}{\textcolor{blue}{Schleiermacher}{}\ledrightnote{→\textcolor{blue}{Friedrich Schleiermacher}}}{}\ledrightnote{→\textcolor{green}{Der Schleier der Beatrice. Schauspiel in fünf Akten}}\pend{}\pstart{}\textcolor{pink}{Wien}{}\ledrightnote{\textcolor{pink}{Wien}}\pend{}\pstart{}\textcolor{pink}{IX. Franckgasse 1}{}\ledrightnote{\textcolor{pink}{Frankgasse}}.\pend{}{\bigskip}\pstart
           \noindent{}Lieber Arthur! Nr\textsuperscript{o} 12 rechts, II. Stock
                  \textcolor{pink}{Burg}{}\ledrightnote{\textcolor{pink}{Burgtheater}}, \textcolor{green}{\uline{Lear}}{}\ledrightnote{\textcolor{green}{König Lear}}, sind \uline{wir} (\label{K_L01279_1v}\edtext{Vers}{\lemma{\textnormal{\emph{Vers}}}\Cendnote{\textnormal{Gemeint ist der Reim »Lear« /
                  »wir«.}}}\label{K_L01279_1h}) \textcolor{blue}{Hugo}{}\ledrightnote{\textcolor{blue}{Hugo von Hofmannsthal}}{ }\introOben{}mit \textcolor{blue}{Frau}{}\ledrightnote{→\textcolor{blue}{Gertrude von Hofmannsthal}}\introOben{} u. ich \introOben{}mit \textcolor{blue}{Frau}{}\ledrightnote{→\textcolor{blue}{Paula Beer-Hofmann}}\introOben{} am Samstag. Vielleicht können Sie – als Gast – mitko{\geminationm}en?\pend
           \pstart
           Man sieht und spricht Sie ohnehin vielzuwenig. (Für mich – und Sie.).\pend
           \pstart
           Herzlich{\\[\baselineskip]}Ihr{\\[\baselineskip]}\spacefill\mbox{Richard}\pend
           \leftskip=0em{}\endnumbering\briefempfaengerindex{Schnitzler, Arthur@\textsc{Schnitzler, Arthur}!zzzBeer-Hofmann, Richard@\emph{von Richard Beer-Hofmann}!1903-03-261@{26. 3. 1903}|)be}\mylabel{h}  \normalsize

\doendnotes{C}
\bigskip
\vfill

\clearpage

\footnotesize

\lohead{\textsc{register}}

% Definiere theindex-Environment komplett neu ohne reledmac
\makeatletter
\renewenvironment{theindex}{%
  \section*{\indexname}%
  \setlength{\parindent}{0pt}%
  \setlength{\parskip}{0pt plus 0.3pt}%
  \let\item\@idxitem
}{%
  \clearpage
}
\makeatother

\IfFileExists{\jobname-pw.ind}{\input{\jobname-pw.ind}}{}

\end{document}

      