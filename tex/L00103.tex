%% latex-korrekturansicht-vorspann.tex
%% Vorspann für die Korrekturansicht.
%% Lädt die gemeinsame Datei latex-vorspann.tex mit gesetztem Schalter.

\newif\ifkorrekturansicht
\korrekturansichttrue

\input{../tex-inputs/latex-vorspann}


               \section[Hugo von Hofmannsthal an Arthur Schnitzler, {[}2. 7. 1892{]}]{ Hugo von Hofmannsthal an Arthur Schnitzler, {[}2. 7. 1892{]}}\nopagebreak\mylabel{v}\rehead{ }\normalsize\beginnumbering\briefempfaengerindex{Schnitzler, Arthur@\textsc{Schnitzler, Arthur}!zzzHofmannsthal, Hugo von@\emph{von Hugo von Hofmannsthal}!1892-07-021@{{[}2. 7. 1892{]}}|(be} \toendnotes[C]{\smallbreak\pagebreak[2]} \Standort{CUL, Schnitzler, B 43.}
\physDesc{Brief, 1 Blatt, 1 Seite
\newline{}Handschrift: schwarze Tinte, deutsche Kurrent
\newline{}Schnitzler: mit Bleistift datiert »2/7 92« und nummeriert: »25« }\buchAbdrucke{\weitereDrucke{Hugo von Hofmannsthal, Arthur Schnitzler: \emph{Briefwechsel}. Hg. Therese Nickl und Heinrich Schnitzler. Frankfurt am Main: \emph{S. Fischer} 1964, S. 21–22.} }\toendnotes[C]{\smallbreak}\pstart{}{\pb}Lieber Arthur.\pend\pstart
           Beſten Dank. Mittwoch{ }abend bin ich \label{K_L00103_1v}\edtext{fertig}{\lemma{\textnormal{\emph{fertig}}}\Cendnote{\textnormal{mit der Matura
                        (Reifeprüfung)}}}\label{K_L00103_1h}. Ich möchte ſehr gern den Donnerstag-
                    oder Freitagabend mit Ihnen und \textcolor{blue}{Salten}{}\ledrightnote{\textcolor{blue}{Felix Salten}}
                    zubringen, incluſive \textsc{Souper}, (\textcolor{pink}{Ausſtellung}{}\ledrightnote{\textcolor{pink}{Internationales Ausstellungstheater im k.k. Prater}}?) und bitte um baldige freundliche
                    Entſcheidung, damit ich mir das übrige danach einrichten kann.\pend
           \pstart \spacefill\mbox{Loris.}\pend{}\endnumbering\briefempfaengerindex{Schnitzler, Arthur@\textsc{Schnitzler, Arthur}!zzzHofmannsthal, Hugo von@\emph{von Hugo von Hofmannsthal}!1892-07-021@{{[}2. 7. 1892{]}}|)be}\mylabel{h}  \normalsize

\doendnotes{C}
\bigskip
\vfill

\clearpage

\footnotesize

\lohead{\textsc{register}}

% Definiere theindex-Environment komplett neu ohne reledmac
\makeatletter
\renewenvironment{theindex}{%
  \section*{\indexname}%
  \setlength{\parindent}{0pt}%
  \setlength{\parskip}{0pt plus 0.3pt}%
  \let\item\@idxitem
}{%
  \clearpage
}
\makeatother

\IfFileExists{\jobname-pw.ind}{\input{\jobname-pw.ind}}{}

\end{document}

      