%% latex-korrekturansicht-vorspann.tex
%% Vorspann für die Korrekturansicht.
%% Lädt die gemeinsame Datei latex-vorspann.tex mit gesetztem Schalter.

\newif\ifkorrekturansicht
\korrekturansichttrue

\input{../tex-inputs/latex-vorspann}


               \section[Lou Andreas-Salomé an Arthur Schnitzler, {[}1. 12. 1895{]}]{ Lou Andreas-Salomé an Arthur Schnitzler, {[}1. 12. 1895{]}}\nopagebreak\mylabel{v}\rehead{ }\normalsize\beginnumbering\briefempfaengerindex{Schnitzler, Arthur@\textsc{Schnitzler, Arthur}!zzzAndreas-Salome, Lou@\emph{von Lou Andreas-Salomé}!1895-12-011@{{[}1. 12. 1895{]}}|(be} \toendnotes[C]{\smallbreak\pagebreak[2]} \Standort{CUL, Schnitzler, B 3.}
\physDesc{Brief, 1 Blatt, 1 Seite
\newline{}Handschrift: schwarze Tinte, deutsche Kurrent
\newline{}Schnitzler: mit Bleistift datiert: »1/12 95« \newline{}Ordnung: mit rotem Buntstift von unbekannter Hand nummeriert:
                                        »11.« }\pstart
           \raggedleft{}{\pb}Sonntag.\pend
           \pstart{}Lieber Herr \textsc{D\textsuperscript{r}},\pend\pstart
           am liebſten wäre es mir, wenn \introOben{}am Dienstag\introOben{} Jemand von
                    Ihnen \uline{nach} Ihrem Theaterbeſuch mich vom
                        \textcolor{pink}{\textsc{Hôtel Royal}}{}\ledrightnote{\textcolor{pink}{Hotel Royal}} zum Nachtmahl im \textcolor{pink}{\textsc{Griensteidl}}{}\ledrightnote{\textcolor{pink}{Café Griensteidl}} abholen könnte. Aber ich habe keine Ahnung ob das ein großer Umweg
                    für Sie wäre, in dem Fall wage ich mich auch allein in’s \textcolor{pink}{\textsc{Griensteidl}}{}\ledrightnote{\textcolor{pink}{Café Griensteidl}}, \strikeout{f} wenn Sie mich wiſſen laſſen
                    wollen um welche Zeit ich es thun ſoll.\pend
           \pstart
           Mit herzlichen Grüßen Ihre{\\[\baselineskip]}\spacefill\mbox{LouAS.}\pend
           \leftskip=0em{}\endnumbering\briefempfaengerindex{Schnitzler, Arthur@\textsc{Schnitzler, Arthur}!zzzAndreas-Salome, Lou@\emph{von Lou Andreas-Salomé}!1895-12-011@{{[}1. 12. 1895{]}}|)be}\mylabel{h}  \normalsize

\doendnotes{C}
\bigskip
\vfill

\clearpage

\footnotesize

\lohead{\textsc{register}}

% Definiere theindex-Environment komplett neu ohne reledmac
\makeatletter
\renewenvironment{theindex}{%
  \section*{\indexname}%
  \setlength{\parindent}{0pt}%
  \setlength{\parskip}{0pt plus 0.3pt}%
  \let\item\@idxitem
}{%
  \clearpage
}
\makeatother

\IfFileExists{\jobname-pw.ind}{\input{\jobname-pw.ind}}{}

\end{document}

      