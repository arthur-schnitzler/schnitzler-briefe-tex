%% latex-korrekturansicht-vorspann.tex
%% Vorspann für die Korrekturansicht.
%% Lädt die gemeinsame Datei latex-vorspann.tex mit gesetztem Schalter.

\newif\ifkorrekturansicht
\korrekturansichttrue

\input{../tex-inputs/latex-vorspann}


               \section[Arthur Schnitzler an Robert Adam, 20. 7. 1915]{ Arthur Schnitzler an Robert Adam, 20. 7. 1915}\nopagebreak\mylabel{v}\rehead{ }\normalsize\beginnumbering\briefempfaengerindex{Adam, Robert@\textsc{Adam, Robert}!zzzSchnitzler, Arthur@\emph{von Arthur Schnitzler}!1915-07-201@{20. 7. 1915}|(be} \toendnotes[C]{\smallbreak\pagebreak[2]} \Standort{DLA, 96.34.1/15.}
\physDesc{Briefkarte, Umschlag
\newline{}Handschrift: schwarze Tinte, lateinische Kurrent\newline{}Versand: Stempel: »\nobreak{}\oindex{XVIII., Waehring@\textbf{XVIII., Währing}, \emph{Bezirk (A.BZK)}|pwk}18/1 Wien 110, 21. VII. 15, 3\nobreak{}«.  }\pstart{}{\pb}\textcolor{gray}{\textbf{Dr. Arthur Schnitzler}}\pend{}\pstart{}\textcolor{gray}{\textbf{\textcolor{pink}{Wien XVIII. Sternwartestrasse 71}{}\ledrightnote{\textcolor{pink}{Sternwartestraße}}}}\pend{}{\bigskip}\pstart{}{\pb}Herrn Dr. Robert Adam
                        Pollak\pend{}\pstart{}Bezirksrichter in\pend{}\pstart{}\textcolor{pink}{Zistersdorf}{}\ledrightnote{\textcolor{pink}{Zistersdorf}}.\pend{}\pstart{}\textcolor{pink}{N. Oe.}{}\ledrightnote{\textcolor{pink}{Niederösterreich}}\pend{}{\bigskip}\pstart
           \noindent{}{\pb}\textcolor{gray}{\textbf{Dr. Arthur Schnitzler}}\hfill 20/7 1915\pend
           \pstart
           \textcolor{gray}{\textbf{\textcolor{pink}{Wien XVIII. Sternwartestrasse 71}{}\ledrightnote{\textcolor{pink}{Sternwartestraße}}}}\pend
           \pstart
           verehrter Herr Doctor, es freut mich, daß Sie meine nicht
                    durchaus freundlichen Worte über die »\textcolor{green}{Gesellschaft}{}\ledrightnote{\textcolor{green}{Gesellschaft [Eine Gaunerkomödie]}}« so liebenswürdg aufgeno{\geminationm}en
                    haben und ich möchte nur nochmals darauf hinweisen, daß ich eine Art von
                    Bühnenwirkung durchaus nicht ausgeschlossen halte{[}.{]} Was das
                    »gelegentliche Hinschmeißen« anbelangt, so bin ich übrigens ganz Ihrer Ansicht –
                    nur weiß man nicht im voraus, was der »Welt« gefallen wird – und die Nachwelt
                    (die bisweilen sehr früh anfängt) ent{\pb}scheidet nach ziemlich
                    geheimnisvollen Gesetzen, gerechter – aber im Sinne der Selbstkritik – die einem
                    gewissen Niveau des Talents continuierlich waltet (auch we{\geminationn} wir versuchen wegzuhören).\pend
           \pstart
           So sehe ich Ihrer »\textcolor{green}{Rechtsphilosophie}{}\ledrightnote{\textcolor{green}{Rechtsphilosophie}}«, Ihrer
                    neuen Komödie und einer baldigen Wiederbegegnung mit Vergnügen entgegegen.\pend
           \pstart
           herzlich grüßend\hspace*{1.5em}Ihr sehr ergebner{\\[\baselineskip]}\spacefill\mbox{Arthur Schnitzler}\pend
           \leftskip=0em{}\endnumbering\briefempfaengerindex{Adam, Robert@\textsc{Adam, Robert}!zzzSchnitzler, Arthur@\emph{von Arthur Schnitzler}!1915-07-201@{20. 7. 1915}|)be}\mylabel{h}  \normalsize

\doendnotes{C}
\bigskip
\vfill

\clearpage

\footnotesize

\lohead{\textsc{register}}

% Definiere theindex-Environment komplett neu ohne reledmac
\makeatletter
\renewenvironment{theindex}{%
  \section*{\indexname}%
  \setlength{\parindent}{0pt}%
  \setlength{\parskip}{0pt plus 0.3pt}%
  \let\item\@idxitem
}{%
  \clearpage
}
\makeatother

\IfFileExists{\jobname-pw.ind}{\input{\jobname-pw.ind}}{}

\end{document}

      