%% latex-korrekturansicht-vorspann.tex
%% Vorspann für die Korrekturansicht.
%% Lädt die gemeinsame Datei latex-vorspann.tex mit gesetztem Schalter.

\newif\ifkorrekturansicht
\korrekturansichttrue

\input{../tex-inputs/latex-vorspann}


               \section[Karl Kraus an Arthur Schnitzler, 19. 3. 1893]{ Karl Kraus an Arthur Schnitzler, 19. 3. 1893}\nopagebreak\mylabel{v}\rehead{ }\normalsize\beginnumbering\briefempfaengerindex{Schnitzler, Arthur@\textsc{Schnitzler, Arthur}!zzzKraus, Karl@\emph{von Karl Kraus}!1893-03-191@{19. 3. 1893}|(be} \toendnotes[C]{\smallbreak\pagebreak[2]} \Standort{CUL, Schnitzler, B 55.}
\physDesc{Brief, 1 Blatt, 4 Seiten
\newline{}Handschrift: schwarze Tinte, deutsche Kurrent}\buchAbdrucke{\weitereDrucke{\emph{Karl Kraus und Arthur Schnitzler. Eine Dokumentation.} Hg. Reinhard Urbach. In: \emph{Literatur und Kritik}, Bd. 49, Oktober 1970, S. 516–517.} }\toendnotes[C]{\smallbreak}\pstart
           \noindent{}{\pb}\textcolor{gray}{\textbf{Karl Kraus}}\hfill \textcolor{gray}{\textbf{\textcolor{pink}{Wien}{}\ledrightnote{\textcolor{pink}{Wien}}, am}}{ }19. 3. \textcolor{gray}{\textbf{189}}3\pend
           \pstart
           \textcolor{gray}{\textbf{\textcolor{pink}{Wien}{}\ledrightnote{\textcolor{pink}{Wien}}}}\pend
           \pstart
           \textcolor{gray}{\textbf{\textcolor{pink}{I., Maximilianstrasse 13}{}\ledrightnote{\textcolor{pink}{Mahlerstraße}}.}}\pend
           \pstart{}Sehr verehrter Herr Doctor!\pend\pstart
           Leider ſehe ich mich genöthigt, mich in einer Angelegenheit an Sie zu wenden, mit
                    der Sie gewiss nicht gerne belästigt werden. Aber, da ich \uline{Sie}, lieber Herr, ſtets hochgeſchätzt und geachtet habe, ſo will
                    ich \introOben{}mich\introOben{} auch Ihnen \strikeout{mich} ganz
                    offenbaren. Sie können ermeſſen, wie ſehr es mich kränkten muſste, daſs Sie mir
                    vorgeſtern im \textcolor{pink}{Grienſteidl}{}\ledrightnote{\textcolor{pink}{Café Griensteidl}}, nachdem wir uns
                    4 Wochen nicht geſehen hatten, mit ſichtlicher Kälte und – ich möchte ſagen –
                    »ceremonieller« Höflichkeit begegneten.\pend
           \pstart
           Und weil es mir nun ganz enorm furchtbar und rieſig daran liegt, daſs \uline{Sie}, liebſter Herr D\textsuperscript{r.} Schnitzler, von mir \uline{gut} denken oder
                    ſo denken, wie über mich zu denken iſt, ſo will ich \uline{Ihnen}, damit \uline{Sie}{ }ſich \introOben{}nicht\introOben{} durch
                    nichtige Redereien beſtimmen laſſen, mir böſe zu ſein und mich quasi für einen
                    »Ausſätzigen« anzuſehen, folgende Thatſachen mittheilen:\pend
           \pstart
           Meine in N\textsuperscript{o} 8 des »\textcolor{green}{Magazin}{}\ledrightnote{\textcolor{green}{Magazin für die Literatur des Auslandes}}« enthaltene »\textcolor{green}{\textcolor{blue}{Dörmann}{}\ledrightnote{\textcolor{blue}{Felix Dörmann}}–\textcolor{blue}{Specht}{}\ledrightnote{\textcolor{blue}{Richard Specht}}«-Recenſion}{}\ledrightnote{→\textcolor{green}{Wiener Lyriker}} iſt \uline{in dieſer
                        Form} bereits vor Monaten entſtanden. Herr \textcolor{blue}{Richard Specht}{}\ledrightnote{\textcolor{blue}{Richard Specht}}{ }ſandte mir im November od.
                        December, (ich weiß nicht genau, wann) ſeine \textcolor{green}{Gedichte}{}\ledrightnote{\textcolor{green}{Gedichte}}. Ich ſchrieb ſofort (nach 2–3 Tagen) eine Kritik,
                        \uline{diese \textcolor{green}{Kritik}{}\ledrightnote{→\textcolor{green}{Wiener Lyriker}}} (mit \textcolor{blue}{Dörmann}{}\ledrightnote{\textcolor{blue}{Felix Dörmann}} zuſammen beſprach ich ihn;
                        \textcolor{blue}{F. D.}{}\ledrightnote{\textcolor{blue}{Felix Dörmann}} »\textcolor{green}{Senſationen}{}\ledrightnote{\textcolor{green}{Sensationen}}« ſandte mir gerade vorher \textcolor{blue}{L.
                        Weiß}{}\ledrightnote{\textcolor{blue}{Leopold Weiß}} zur Recenſion). \textcolor{blue}{Dörmann}{}\ledrightnote{\textcolor{blue}{Felix Dörmann}}{ }\uline{kannte ich damals} noch nicht; den lernte ich
                    erſt ſpäter durch Vermittelung D\textsuperscript{r.} \textcolor{blue}{Beer-Hofmann}{}\ledrightnote{\textcolor{blue}{Richard Beer-Hofmann}}’s perſönlich kennen.\pend
           \pstart
           Die Kritik gab ich dem »\textcolor{brown}{Tagblatt}{}\ledrightnote{\textcolor{brown}{Wiener Tagblatt}}«. \textcolor{blue}{Alexander Landesberg}{}\ledrightnote{\textcolor{blue}{Alexander Landesberg}} behielt ſie volle
                    2 Monate bei ſich, ohne ſich zu entſcheiden. Endlich gieng ich hin. Er erklärte,
                    dieſer Sache keinen ſo breiten Raum gewähren zu können. Er ſuchte sie heraus,
                    fand ſie nach langem Suchen und gab ſie mir – {\pb}Nun ſchickte ich die Arbeit \introOben{}(\uline{Dieſelbe!! In dieſer
                        Form!!})\introOben{} – auf’s Geratewohl – an’s »\textcolor{brown}{Magazin}{}\ledrightnote{\textcolor{brown}{Magazin für die Literatur des Auslandes}}«. Nach 8 Tagen ſchrieb mir \textcolor{blue}{Paul
                            Sch\strikeout{l}ettler}{}\ledrightnote{\textcolor{blue}{Paul Schettler}} für die Redaction: »Ihre
                    Besprechung der beiden \textcolor{pink}{Wien}{}\ledrightnote{\textcolor{pink}{Wien}}er ›Neurotiker‹
                    acceptiert das ›\textcolor{brown}{Magazin}{}\ledrightnote{\textcolor{brown}{Magazin für die Literatur des Auslandes}}‹ mit Vergnügen.«\pend
           \pstart
           Als ich nach \textcolor{pink}{Berlin}{}\ledrightnote{\textcolor{pink}{Berlin}} kam, machte man mich auf die
                    bereits erſchienene \textcolor{green}{Kritik}{}\ledrightnote{→\textcolor{green}{Wiener Lyriker}}
                    aufmerkſam. Ich war dem \textcolor{brown}{Tgbl.}{}\ledrightnote{\textcolor{brown}{Wiener Tagblatt}} vom Herzen
                    dankbar, daſs es die \textcolor{green}{Kritik}{}\ledrightnote{→\textcolor{green}{Wiener Lyriker}}{ }retournierte. Denn durch dieſe \textcolor{green}{Kritik}{}\ledrightnote{→\textcolor{green}{Wiener Lyriker}}, die \textcolor{blue}{Otto Neumann-Hofer}{}\ledrightnote{\textcolor{blue}{Gilbert Otto Neumann-Hofer}} und die andern Herren \introOben{}(\uline{auch Baron \textcolor{blue}{Liliencron}{}\ledrightnote{\textcolor{blue}{Detlev von Liliencron}}})\introOben{} außerordentlich lobten, ſchuf ich mir feſte Position im »\textcolor{brown}{Magazin}{}\ledrightnote{\textcolor{brown}{Magazin für die Literatur des Auslandes}}«. Die Sache wurde ſofort honoriert und
                    weitere Artikel (über \textcolor{pink}{Wien}{}\ledrightnote{\textcolor{pink}{Wien}}er Litteratur,
                    »Decadence« etc) – ſozuſagen – »beſtellt«.\pend
           \pstart
           Ich glaube, es ſind ſchon 4 Monate her, daſs mir Herr \textcolor{blue}{Specht}{}\ledrightnote{\textcolor{blue}{Richard Specht}}{ }ſein \textcolor{green}{Büchlein}{}\ledrightnote{→\textcolor{green}{Gedichte}}{ }ſchickte, circa{ }\uline{4 Monate} alſo ſeit Abfaſſung des vor 2–3 Wochen
                    erſchienenen \textcolor{green}{Artikels}{}\ledrightnote{→\textcolor{green}{Wiener Lyriker}}!!
                    Deshalb iſt entſtanden\strikeout{,}{ }\uline{lange, lange}, bevor ich Herrn \textcolor{blue}{Specht}{}\ledrightnote{\textcolor{blue}{Richard Specht}} den wirklich mit Müh und Not beſchafften
                    »Sündentraum«beleg ſchickte und da\substVorne{}\textsuperscript{bei}\substDazwischen{}zu\substHinten{} jenen ominösen, aber durch und durch freundlichen Brief ſchrieb, der
                    den harmloſen Witz (»\textcolor{blue}{Dör-mannbar}{}\ledrightnote{→\textcolor{blue}{Felix Dörmann}}« enthielt) ſie iſt entſtanden, \uline{lange} bevor ich Herrn \textcolor{blue}{Dörmann}{}\ledrightnote{\textcolor{blue}{Felix Dörmann}}
                    perſönlich kennen lernte, ſo daſs alſo weder von einem perſönlichen Gefühle {\pb}Herrn \textcolor{blue}{Specht}{}\ledrightnote{\textcolor{blue}{Richard Specht}} gegenüber noch von einer »Beeinfluſſung durch \textcolor{blue}{Dörmann}{}\ledrightnote{\textcolor{blue}{Felix Dörmann}}« die Rede ſein kann!\pend
           \pstart
           \uuline{Das beſchwöre ich}!\pend
           \pstart
           \textcolor{blue}{\uline{Alexander Landesberg}}{}\ledrightnote{\textcolor{blue}{Alexander Landesberg}}, \textcolor{blue}{Alexander Engel}{}\ledrightnote{\textcolor{blue}{Alexander Engel}}, \textcolor{blue}{Anton Lindner}{}\ledrightnote{\textcolor{blue}{Anton Lindner}} etc etc andere Freunde ſind Zeugen!!\pend
           \pstart
           Die \textcolor{green}{Kritik}{}\ledrightnote{→\textcolor{green}{Wiener Lyriker}} (\uline{ganz} in der jetzigen Geſtalt!!) iſt – vor
                    Monaten – aus einer ehrlichen, vollſten, ureigenſten Überzeugung heraus
                    entſtanden. Nichts liegt mir ferner als Unehrlichkeit, als »Rachegefühl« und
                    jüdiſches Tagſschreiberthum. Man hüte ſich, mich in dieſer niederträchtigen
                    Weise zu verleumden!!\pend
           \pstart
           Ich haſſe und haſste diese falſche, erlogene »Decadence«, die artig mit ſich
                    ſelbst coquettiert; ich bekämpfe und werde immer bekämpfen: die posierte,
                    krankhafte, onanierte Poeſie! {\pb}Und
                        \uline{dieſer Haſs} war das Kritikmotiv!\pend
           \pstart
           \strikeout{Glauben} Sie werden vielleicht, verehrter Herr
                        D\textsuperscript{r.}, ſich denken: Aha, wer ſich \uline{ſo} vertheidigt, \uline{muſs}{ }ſich wohl verteidigen!? \strikeout{und} Nein, ſeien Sie versichert, die ganze Litanei
                    hab ich auch nur \uline{Ihnen}\footnote{\noindent{}Auch dem verehrten Herrn D\textsuperscript{r.}{ }\textcolor{blue}{B-Hofmann}{}\ledrightnote{\textcolor{blue}{Richard Beer-Hofmann}} hätte ich’s geſagt!} hergeſagt, weil mir an \uline{Ihrer} Meinung
                        \strikeout{etw} viel liegt. Den andern gegenüber hab’
                    ich es Gottſseidank nicht nöthig, mich zu vertheidigen!\pend
           \pstart
           Wenn ich Sie beläſtigt habe, verzeihen Sie.\pend
           \pstart
           \textcolor{blue}{Otto Erich Hartleben}{}\ledrightnote{\textcolor{blue}{Otto Erich Hartleben}} grüßt Sie durch
                    mich.\pend
           \pstart
           Für »\textcolor{brown}{Neue litt. Bl}{}\ledrightnote{\textcolor{brown}{Neue litterarische Blätter}}« \introOben{}(\textcolor{pink}{Bremen}{}\ledrightnote{\textcolor{pink}{Bremen}})\introOben{} wäre ich mit \strikeout{mit}{ }\textcolor{green}{Anatol}{}\ledrightnote{\textcolor{green}{Anatol}} zu ſpät gekommen, da das dort in
                        \label{K_L00191_1v}\edtext{Einläufe}{\lemma{\textnormal{\emph{Einläufe}}}\Cendnote{\textnormal{\emph{\textcolor{green}{Neue litterarische Blätter}}, Jg. 1,
                            H. 5/6, 1. 3. 1893, S. 66.}}}\label{K_L00191_1h} verzeichnete \textcolor{green}{Buch}{}\ledrightnote{→\textcolor{green}{Anatol}} bereits an einen andern
                        \textcolor{blue}{Mitarbeiter}{}\ledrightnote{→\textcolor{blue}{Josef Schmid-Braunfels}} zur \textcolor{green}{Recension}{}\ledrightnote{→\textcolor{green}{Arthur Schnitzler: Anatol}} abgegeben
                    wurde.\pend
           \pstart
           Sonſt ſtehe ich Ihnen mit aufrichtigem Vergnügen ſtets zu Dienſten u bin (Sie
                    noch \uline{um paar Zeilen bittend}!) Ihr \uline{Sie vollkommen hochachtender}\pend
           \pstart
           Herzlichſt grüſſend{\\[\baselineskip]}\spacefill\mbox{Karl Kraus}\pend
           \leftskip=0em{}\endnumbering\briefempfaengerindex{Schnitzler, Arthur@\textsc{Schnitzler, Arthur}!zzzKraus, Karl@\emph{von Karl Kraus}!1893-03-191@{19. 3. 1893}|)be}\mylabel{h}  \normalsize

\doendnotes{C}
\bigskip
\vfill

\clearpage

\footnotesize

\lohead{\textsc{register}}

% Definiere theindex-Environment komplett neu ohne reledmac
\makeatletter
\renewenvironment{theindex}{%
  \section*{\indexname}%
  \setlength{\parindent}{0pt}%
  \setlength{\parskip}{0pt plus 0.3pt}%
  \let\item\@idxitem
}{%
  \clearpage
}
\makeatother

\IfFileExists{\jobname-pw.ind}{\input{\jobname-pw.ind}}{}

\end{document}

      