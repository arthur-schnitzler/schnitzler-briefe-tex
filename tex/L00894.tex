%% latex-korrekturansicht-vorspann.tex
%% Vorspann für die Korrekturansicht.
%% Lädt die gemeinsame Datei latex-vorspann.tex mit gesetztem Schalter.

\newif\ifkorrekturansicht
\korrekturansichttrue

\input{../tex-inputs/latex-vorspann}


               \section[Jakob Julius David an Arthur Schnitzler, 27. 2. 1899]{ Jakob Julius David an Arthur Schnitzler, 27. 2. 1899}\nopagebreak\mylabel{v}\rehead{ }\normalsize\beginnumbering\briefempfaengerindex{Schnitzler, Arthur@\textsc{Schnitzler, Arthur}!zzzDavid, Jakob Julius@\emph{von Jakob Julius David}!1899-02-271@{27. 2. 1899}|(be} \toendnotes[C]{\smallbreak\pagebreak[2]} \Standort{TMW, HS Schn 1/93/1.}
\physDesc{Postkarte
\newline{}Handschrift: schwarze Tinte, lateinische Kurrent\newline{}Versand: 1) Rohrpost 2) Stempel: »\nobreak{}\oindex{I., Innere Stadt@\textbf{I., Innere Stadt}, \emph{Bezirk (A.BZK)}|pwk}Wien 1/1, 27 II 99, 1 20V\nobreak{}«. 3) Stempel: »\nobreak{}\oindex{IX., Alsergrund@\textbf{IX., Alsergrund}, \emph{Bezirk (A.BZK)}|pwk}Wien 9/2, 27 II 99, 1 50N\nobreak{}«. }\toendnotes[C]{\smallbreak}\pstart{}{\pb}Herrn D\textsuperscript{r}. Arthur Schnitzler\pend{}\pstart{}\textcolor{pink}{IX.}{}\ledrightnote{\textcolor{pink}{IX., Alsergrund}}\pend{}\pstart{}\textcolor{pink}{Franckgaße 1}{}\ledrightnote{\textcolor{pink}{Frankgasse}}\pend{}{\bigskip}\pstart\center{}{\pb}Werther Herr!\pend\pstart
           Ich habe heute im Theater vergeblich versucht, mir Ihre drei \textcolor{green}{Einacter}{}\ledrightnote{→\textcolor{green}{Der grüne Kakadu – Paracelsus – Die Gefährtin. Drei Einakter}} zu verschaffen. Ohne Ansicht
                    des Buches ka{\geminationn}{ }\uline{ich} nicht \label{K_L00894_1v}\edtext{schreiben}{\lemma{\textnormal{\emph{schreiben}}}\Cendnote{\textnormal{In
                        Folge entstand: \textcolor{blue}{J. J. David}: \emph{\textcolor{green}{Aus ungleichen Tagen}}. In: \emph{\textcolor{green}{Neues Wiener Journal}}, Jg. 7, Nr. 1925,
                                2. 3. 1899, S. 1–2.}}}\label{K_L00894_1h}; ich bitte Sie also,
                    mir die \textcolor{green}{Stücke}{}\ledrightnote{→\textcolor{green}{Der grüne Kakadu – Paracelsus – Die Gefährtin. Drei Einakter}} auf einige
                    Stunden, nur über Nacht, es sei von heute oder morgen zu leihen. Sie sollen sie
                        Dienstag oder Mitwoch zu Ihrer paßenden Stunde
                    dort finden, wo Sie wollen. Unter allen Umständen erbitte ich um Nachricht.\pend
           \pstart
           Bestens Ihr{\\[\baselineskip]}\spacefill\mbox{David}\pend
           \leftskip=0em{}\pstart
           \noindent{}\textcolor{pink}{II. Ob. Donaustraße 59}{}\ledrightnote{\textcolor{pink}{Obere Donaustraße}}\pend
           \endnumbering\briefempfaengerindex{Schnitzler, Arthur@\textsc{Schnitzler, Arthur}!zzzDavid, Jakob Julius@\emph{von Jakob Julius David}!1899-02-271@{27. 2. 1899}|)be}\mylabel{h}  \normalsize

\doendnotes{C}
\bigskip
\vfill

\clearpage

\footnotesize

\lohead{\textsc{register}}

% Definiere theindex-Environment komplett neu ohne reledmac
\makeatletter
\renewenvironment{theindex}{%
  \section*{\indexname}%
  \setlength{\parindent}{0pt}%
  \setlength{\parskip}{0pt plus 0.3pt}%
  \let\item\@idxitem
}{%
  \clearpage
}
\makeatother

\IfFileExists{\jobname-pw.ind}{\input{\jobname-pw.ind}}{}

\end{document}

      