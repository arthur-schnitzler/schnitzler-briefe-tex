%% latex-korrekturansicht-vorspann.tex
%% Vorspann für die Korrekturansicht.
%% Lädt die gemeinsame Datei latex-vorspann.tex mit gesetztem Schalter.

\newif\ifkorrekturansicht
\korrekturansichttrue

\input{../tex-inputs/latex-vorspann}


               \section[Joseph Victor Widmann an Arthur Schnitzler, 26. 3. 1893]{ Joseph Victor Widmann an Arthur Schnitzler,
                    26. 3. 1893}\nopagebreak\mylabel{v}\rehead{ }\normalsize\beginnumbering\briefempfaengerindex{Schnitzler, Arthur@\textsc{Schnitzler, Arthur}!zzzWidmann, Joseph Victor@\emph{von Joseph Victor Widmann}!1893-03-261@{26. 3. 1893}|(be} \toendnotes[C]{\smallbreak\pagebreak[2]} \Standort{CUL, Schnitzler, B 113.}
\physDesc{Postkarte
\newline{}Handschrift: schwarze Tinte, deutsche Kurrent\newline{}Versand: Stempel: »\nobreak{}\oindex{Bern@\textbf{Bern}, \emph{Besiedelter Ort (A.BSO)}|pwk}Bern Brf. Exp., 25 III. 93., 1\nobreak{}«.  }\toendnotes[C]{\smallbreak}\pstart{}{\pb}\textsc{Herrn
                                D\textsuperscript{r} Arthur Schnitzler}\pend{}\pstart{}in\pend{}\pstart{}\textcolor{pink}{\textsc{Wien} I}{}\ledrightnote{\textcolor{pink}{I., Innere Stadt}}. \pend{}\pstart{}\textsc{\textcolor{pink}{I.
                                Grillparzerstraſse 7}{}\ledrightnote{\textcolor{pink}{Grillparzerstraße}}.}\pend{}{\bigskip}\pstart
           \raggedleft{}{\pb}\textcolor{pink}{Bern}{}\ledrightnote{\textcolor{pink}{Bern}}, d.
                            26. März 1893.\pend
           \pstart{}Verehrteſter Herr!\pend\pstart
           Die \textcolor{green}{Beſprechung}{}\ledrightnote{→\textcolor{green}{Kunst und Litteratur}} Ihres \textcolor{green}{Anatol}{}\ledrightnote{\textcolor{green}{Anatol}} war von mir ſelbſt, wie ich überhaupt die meiſten
                    literariſchen Referate des »\textcolor{brown}{Bund}{}\ledrightnote{\textcolor{brown}{Der Bund}}« ſchreibe. Es
                    freut mich, Ihrer vom 14. d. an die Redaktion gerichteten Zuſchrift zu
                    entnehmen, daß ſie Ihnen Spaß machte. \strikeout{A} Ihren
                        \textcolor{green}{Anatol}{}\ledrightnote{\textcolor{green}{Anatol}} habe ich ſehr wohl gelobt.\pend
           \pstart Hochachtungsvoll\spacefill\mbox{J. V. Widmann}\pend{}\pstart
           \raggedleft{}liter. Redakteur des »\textcolor{brown}{Bund}{}\ledrightnote{\textcolor{brown}{Der Bund}}«\pend
           \endnumbering\briefempfaengerindex{Schnitzler, Arthur@\textsc{Schnitzler, Arthur}!zzzWidmann, Joseph Victor@\emph{von Joseph Victor Widmann}!1893-03-261@{26. 3. 1893}|)be}\mylabel{h}  \normalsize

\doendnotes{C}
\bigskip
\vfill

\clearpage

\footnotesize

\lohead{\textsc{register}}

% Definiere theindex-Environment komplett neu ohne reledmac
\makeatletter
\renewenvironment{theindex}{%
  \section*{\indexname}%
  \setlength{\parindent}{0pt}%
  \setlength{\parskip}{0pt plus 0.3pt}%
  \let\item\@idxitem
}{%
  \clearpage
}
\makeatother

\IfFileExists{\jobname-pw.ind}{\input{\jobname-pw.ind}}{}

\end{document}

      