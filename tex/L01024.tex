%% latex-korrekturansicht-vorspann.tex
%% Vorspann für die Korrekturansicht.
%% Lädt die gemeinsame Datei latex-vorspann.tex mit gesetztem Schalter.

\newif\ifkorrekturansicht
\korrekturansichttrue

\input{../tex-inputs/latex-vorspann}


               \section[Arthur Schnitzler an Hugo von Hofmannsthal, 23. 3. 1900]{ Arthur Schnitzler an Hugo von Hofmannsthal, 23. 3. 1900}\nopagebreak\mylabel{v}\rehead{ }\normalsize\beginnumbering\briefempfaengerindex{Hofmannsthal, Hugo von@\textsc{Hofmannsthal, Hugo von}!zzzSchnitzler, Arthur@\emph{von Arthur Schnitzler}!1900-03-231@{23. 3. 1900}|(be} \toendnotes[C]{\smallbreak\pagebreak[2]} \Standort{FDH, Hs-30885,90.}
\physDesc{Brief, 2 Blätter, 8 Seiten
\newline{}Handschrift: schwarze Tinte, deutsche Kurrent\newline{}Ordnung: auch das zweite Blatt von Schnitzler mutmaßlich bei der
                                 Durchsicht der Korrespondenz 1929 mit »23. 3. 900.« datiert }\buchAbdrucke{\weitereDrucke{Hugo von Hofmannsthal, Arthur Schnitzler: \emph{Briefwechsel}. Hg. Therese Nickl und Heinrich Schnitzler. Frankfurt am Main: \emph{S. Fischer} 1964, S. 135–136.} }\toendnotes[C]{\smallbreak}\pstart
           \raggedleft{}{\pb}23. 3. 900.\pend
           \pstart
           mein lieber Hugo, Sie haben mich recht lang warten laſſen, aber was
               Sie mir ſchreiben iſt alles erfreulich und ſchön, und so hab ich es erwartet. Der
               kleine Ort heißt \textcolor{pink}{\textsc{Vilennes}}{}\ledrightnote{\textcolor{pink}{Villenes-sur-Seine}} oder \textcolor{pink}{\textsc{Vilaines}}{}\ledrightnote{\textcolor{pink}{Villenes-sur-Seine}} – bei \textcolor{pink}{\textsc{Poissy}}{}\ledrightnote{\textcolor{pink}{Poissy}}, we{\geminationn} mich nicht die Erinnerg trügt, an der
                  \textsc{Marne}. Ich ka{\geminationn} nie an jene Stunde zurückdenken,
               ohne daſs ſich mein ganzes Weſen mit einem unbegreiflichen Schauer füllt, ſo als we{\geminationn} ich dort es eigentlich ſchon hätte wiſſen müssen – {\pb}oder gar – es gewußt hätte – (»\textcolor{green}{dort – wo wir an lichten Tagen nicht
                  hineinſchaun!}{}\ledrightnote{→\textcolor{green}{Der Schleier der Beatrice. Schauspiel in fünf Akten}}«) – Ihr Brief kam grad am Morgen des \label{K_L01024_1v}\edtext{18. März}{\lemma{\textnormal{\emph{18. März}}}\Cendnote{\textnormal{\textcolor{blue}{Maria Reinhard}s erster Todestag.}}}\label{K_L01024_1h}. –\pend
           \pstart
           Ihr kleines \textcolor{green}{Vorſpiel}{}\ledrightnote{→\textcolor{green}{Vorspiel zur Antigone des Sophokles}}, das ich
               ſehr einfach und ſchön finde, hab ich gleich an \textcolor{blue}{Paul
                     Goldma{\geminationn}}{}\ledrightnote{\textcolor{blue}{Paul Goldmann}} (\textcolor{pink}{\textsc{Berlin}}{}\ledrightnote{\textcolor{pink}{Berlin}}, \textcolor{pink}{\textsc{Dessauer}ſtraße 19}{}\ledrightnote{\textcolor{pink}{Dessauer Straße}}) geſchickt, vielleicht ſchreiben
               Sie ihm auch ein Wort? \pend
           \pstart
           – Wir leben hier noch im ewigen {\pb}Winter. Schnee heut
               Nacht! – Und Wind, Regen, Koth. Es ist abſcheulich. Ich will in den nächſten Tagen
               ein bischen in den Süden fahren, bis \textcolor{pink}{Raguſa}{}\ledrightnote{\textcolor{pink}{Dubrovnik}}. Nicht
               mit rechter Freude. Aber ich hab auch i{\geminationm}er Katarrhe,
               jetzt noch dazu dumme Geſchichten mit plombirten Zähnen, dazu alles andre, kurz, ich
                  ka{\geminationn}{ }{\pb}mich kaum je eine viertel Stunde wohl fühlen.
                  Anfang März war ich ein paar Tage in \textcolor{pink}{Edlach}{}\ledrightnote{\textcolor{pink}{Edlach}}; habe dort den Frühling finden wollen, aber Eis und 10 Grad Kälte,
               ſowie \textcolor{blue}{\textsc{Dora Speyer}}{}\ledrightnote{\textcolor{blue}{Dora Michaelis}} gefunden, die übrigens lieb iſt.\pend
           \pstart
           – Jetzt iſt \textcolor{blue}{\textsc{Brandes}}{}\ledrightnote{\textcolor{blue}{Georg Brandes}} hier, erzählt ſehr amüſant, und iſt gewiſs was ſehr beſondres. Und {\pb}doch (warum »und doch«?) hab ich eher ein Gefühl der
               Entfremdung diesmal ihm gegenüber. Liegt wohl an meiner Sti{\geminationm}ung. –\pend
           \pstart
           Ich arbeite an nichts als an der langen \textcolor{green}{Novelle}{}\ledrightnote{→\textcolor{green}{Frau Bertha Garlan. Roman}}, die wohl (ſtofflich) ſo eine Art Seitenstück zur
                  \textcolor{green}{\textsc{Femme de 30 ans}}{}\ledrightnote{\textcolor{green}{Eine Frau von dreißig Jahren}} wird, eine \label{K_L01024_2v}\edtext{\textsc{veuve de 30 ans}}{\lemma{\textnormal{\emph{veuve de 30 ans}}}\Cendnote{\textnormal{französisch: Witwe von 30 Jahren}}}\label{K_L01024_2h}
               – viel{\pb}leicht ſchließ ich ſie auf der \textcolor{pink}{dalmatiniſchen}{}\ledrightnote{\textcolor{pink}{Dalmatien}} Küſtenfahrt ab. –\pend
           \pstart
           Eben telephonirt mir \textcolor{blue}{Richard}{}\ledrightnote{\textcolor{blue}{Richard Beer-Hofmann}} ich möge in den \textcolor{brown}{Schachclub}{}\ledrightnote{\textcolor{brown}{Wiener Schachclub}} ko{\geminationm}en – Iſt
               das nicht ganz unwahrſcheinlich in \textcolor{pink}{Paris}{}\ledrightnote{\textcolor{pink}{Paris}} zu hören,
               daſs hier weiter telephonirt wird – in den \textcolor{brown}{Schachclub}{}\ledrightnote{\textcolor{brown}{Wiener Schachclub}} gegangen –? So iſt es mir gewiſſermaßen räthſelhaft, daſs gewiſs
               das Haus {\pb}in der \textcolor{pink}{\textsc{rue Maubeuge Nr. 5}}{}\ledrightnote{\textcolor{pink}{rue de Maubeuge}}{ }ſteht – ja daſs noch die Zi{\geminationm}er exiſtiren, die Fenſter – die Waſchtiſche – –\pend
           \pstart
           Ich ka{\geminationn} Ihnen gar nicht ſagen wie mir iſt, während ich
               dieſen Brief ende. Als hätt ich’s noch i{\geminationm}er nicht ganz
                  \textcolor{blue}{verſtanden}{}\ledrightnote{→\textcolor{blue}{Marie Reinhard}} – denn in dieſem
               Augenblick ſind mir Dinge eingefallen, an die ich ſeitdem nicht gedacht.\pend
           \pstart
           {\pb}leben Sie wohl. Wann kommen Sie wieder? Werden wir
                  zuſa{\geminationm}en radeln? Ich bin neugierig auf das, was Sie
               mir von den Namenloſen erzählen werden.\pend
           \pstart
           Von Herzen{\\[\baselineskip]}Ihr{\\[\baselineskip]}\spacefill\mbox{Arthur.}\pend
           \leftskip=0em{}\pstart
           Grüßen Sie \textcolor{blue}{Hans Schleſinger}{}\ledrightnote{\textcolor{blue}{Hans Bernhard Schlesinger}} u. \textcolor{blue}{Bubi Franckenſtein}{}\ledrightnote{\textcolor{blue}{Georg von Franckenstein}}.\pend
           \endnumbering\briefempfaengerindex{Hofmannsthal, Hugo von@\textsc{Hofmannsthal, Hugo von}!zzzSchnitzler, Arthur@\emph{von Arthur Schnitzler}!1900-03-231@{23. 3. 1900}|)be}\mylabel{h}  \normalsize

\doendnotes{C}
\bigskip
\vfill

\clearpage

\footnotesize

\lohead{\textsc{register}}

% Definiere theindex-Environment komplett neu ohne reledmac
\makeatletter
\renewenvironment{theindex}{%
  \section*{\indexname}%
  \setlength{\parindent}{0pt}%
  \setlength{\parskip}{0pt plus 0.3pt}%
  \let\item\@idxitem
}{%
  \clearpage
}
\makeatother

\IfFileExists{\jobname-pw.ind}{\input{\jobname-pw.ind}}{}

\end{document}

      