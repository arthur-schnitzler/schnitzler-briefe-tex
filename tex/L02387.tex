%% latex-korrekturansicht-vorspann.tex
%% Vorspann für die Korrekturansicht.
%% Lädt die gemeinsame Datei latex-vorspann.tex mit gesetztem Schalter.

\newif\ifkorrekturansicht
\korrekturansichttrue

\input{../tex-inputs/latex-vorspann}


               \section[Arthur Schnitzler an Georg Brandes, 7. 6. 1922]{ Arthur Schnitzler an Georg Brandes, 7. 6. 1922}\nopagebreak\mylabel{v}\rehead{ }\normalsize\beginnumbering\briefempfaengerindex{Brandes, Georg@\textsc{Brandes, Georg}!zzzSchnitzler, Arthur@\emph{von Arthur Schnitzler}!1922-06-071@{7. 6. 1922}|(be} \toendnotes[C]{\smallbreak\pagebreak[2]} \Standort{Kopenhagen, Det Kongelige Bibliotek, Georg Brandes Arkiv, box 125.}
\physDesc{Brief, 2 Blätter, 4 Seiten
\newline{}Handschrift: schwarze Tinte, lateinische Kurrent\newline{}Ordnung: mit Bleistift von unbekannter Hand beschriftet:
                                                »Schnitzler« und nummeriert:
                                                »45.«, das zweite Blatt mit ergänztem
                                            Datum: »7/6 22« }\buchAbdrucke{\weitereDrucke{Georg Brandes, Arthur Schnitzler: \emph{Ein Briefwechsel}. Hg. Kurt Bergel. Bern: \emph{Francke} 1956, S. 137–138.} }\toendnotes[C]{\smallbreak}\pstart
           \raggedleft{}{\pb}\textcolor{pink}{Wien}{}\ledrightnote{\textcolor{pink}{Wien}}, 7. 6. 22\pend
           \pstart
           Mein lieber und verehrter Freund, daß ich nicht nach \textcolor{pink}{Kopenhagen}{}\ledrightnote{\textcolor{pink}{Kopenhagen}} gekommen bin, war niemandem
                    aergerlicher als mir, aber niemand hatte weniger Schuld daran. Hören Sie wie es
                    war: Ein sehr netter junger Mann aus \textcolor{pink}{Daenemark}{}\ledrightnote{\textcolor{pink}{Dänemark}}, Herr \textcolor{blue}{Axel Fraenckel}{}\ledrightnote{\textcolor{blue}{Axel Fraenckel}},
                    Literat, forderte mich im Namen eines »radicalen« Studentenbundes auf, in \textcolor{pink}{Kopenhagen}{}\ledrightnote{\textcolor{pink}{Kopenhagen}} zu lesen. Ich war mit Vergnügen
                    bereit – ja ich spielte mit dem Gedanken gerade den 15 Mai in \textcolor{pink}{Kopenhagen}{}\ledrightnote{\textcolor{pink}{Kopenhagen}} und womöglich mit Ihnen
               zuzubringen. Ich erklärte, daß ich im \textcolor{pink}{Haag}{}\ledrightnote{\textcolor{pink}{Den Haag}}, (wo
                    ich, wie in \textcolor{pink}{Amsterdam}{}\ledrightnote{\textcolor{pink}{Amsterdam}} u \textcolor{pink}{Rotterdam}{}\ledrightnote{\textcolor{pink}{Rotterdam}} aus meinen Werken vorlas) \uline{definitive}{ }\strikeout{Aus} Nachrichten abwarten wolle u. zw. bis
                    spaetestens 30. April. Ich war bis zum 8. Mai in \textcolor{pink}{Holland}{}\ledrightnote{\textcolor{pink}{Niederlande}} – es kam keine Zeile, – und ich
                    selbst konnte mich nicht an den Studentenbund wenden – schon darum, weil mir
                    weder der officielle Name, noch die Adresse noch der Name des Obmanns {\pb}bekannt war – so dacht ich man habe in \textcolor{pink}{Kopenhagen}{}\ledrightnote{\textcolor{pink}{Kopenhagen}} auf mein Kommen verzichtet, – fuhr
               nach \textcolor{pink}{Berlin}{}\ledrightnote{\textcolor{pink}{Berlin}}, – wo mir – über \textcolor{pink}{Haag}{}\ledrightnote{\textcolor{pink}{Den Haag}}, – und \textcolor{pink}{Wien}{}\ledrightnote{\textcolor{pink}{Wien}} – (die
                    kürzeste Verbindung) ein Telegramm nachgesandt wurde – von dem Studentenbund –
                    ich möge meinen Ankunftstag melden. Nun aber hatt ich meine Dispositionen schon
                    total geändert u. es war zu spät, wieder in den Norden zu reisen; – auch hatt
                    ich einigermaßen die Lust verloren. So verbracht ich meinen Geburtstag –
                        vollko{\geminationm}en allein – in \textcolor{pink}{Nürnberg}{}\ledrightnote{\textcolor{pink}{Nürnberg}} und fuhr von da nach \textcolor{pink}{München}{}\ledrightnote{\textcolor{pink}{München}} und \textcolor{pink}{Wien}{}\ledrightnote{\textcolor{pink}{Wien}}. Entweder
               ist ein Brief in den \textcolor{pink}{Haag}{}\ledrightnote{\textcolor{pink}{Den Haag}} verloren gegangen
                    oder die Herren vom Studentenbund haben die Angelegenheit etwas zu lax behandelt
                    – aber ich hoffe, ein nächstes Mal – vielleicht im nächsten Frühling (freilich –
                    schon »am nächsten Tag« ist ein kühnes Wort) – wird die Sache zu Stande kommen.
                        {\pb}Morgen fahr ich nach \textcolor{pink}{Graz}{}\ledrightnote{\textcolor{pink}{Graz}}, wo ich zweimal vorlese – ein etwas ärmlicher Ersatz
                    für \textcolor{pink}{Kopenhagen}{}\ledrightnote{\textcolor{pink}{Kopenhagen}} und Sie.\pend
           \pstart
           Und für Ihre lieben Worte, mein verehrter Georg Brandes, ka{\geminationn} ich Ihnen nur schriftlich danken. (Haben Sie
                        de{\geminationn} auch meinen Brief zu Ihrem soundsovielten
                    Geburtstag bekommen?)\pend
           \pstart
           Anfang Juli bring ich meine \textcolor{blue}{Kinder}{}\ledrightnote{→\textcolor{blue}{Heinrich Schnitzler}{\newline}→\textcolor{blue}{Lili Schnitzler}} an den Starnbergersee zu ihrer \textcolor{blue}{Mutter}{}\ledrightnote{→\textcolor{blue}{Olga Schnitzler}}. (Mein \textcolor{blue}{Sohn}{}\ledrightnote{→\textcolor{blue}{Heinrich Schnitzler}}, bald zwanzig, ist für die nächste Saison schon
                    hier am \textcolor{pink}{Raimundtheater}{}\ledrightnote{\textcolor{pink}{Raimund-Theater}} engagirt; er studirt auch
                    Philosophie an der \textcolor{pink}{Universität}{}\ledrightnote{→\textcolor{pink}{Universität Wien}}, arbeitet auch theatergeschichtlich, macht
                    Inszenierungspläne, zeichnet u malt Figurinen, treibt viel Musik; meine \textcolor{blue}{Tochter}{}\ledrightnote{→\textcolor{blue}{Lili Schnitzler}}, bald dreizehn,
                    geht ins Gymnasium.); meine Sommer{\pb}pläne sind
                    noch etwas unsicher; – ich wünschte sehr, nach ziemlich unruhigen und verwirrten
                    Zeiten, ins geordnete Arbeiten zu gelangen – und, insbesondre ein \textcolor{green}{Stück}{}\ledrightnote{→\textcolor{green}{Komödie der Verführung. In drei Akten}} zu vollenden, dessen
                    letzter Akt an der \textcolor{pink}{daenischen}{}\ledrightnote{\textcolor{pink}{Dänemark}} Küste spielen
                    soll. Ich baue dort ein köstliches Hotel hin wie ich es seinerzeit am \textcolor{pink}{Völser Weiher}{}\ledrightnote{\textcolor{pink}{Völser Weiher}} (\textcolor{green}{im weiten Land}{}\ledrightnote{\textcolor{green}{Das weite Land. Tragikomödie in fünf Akten}}) gethan – mögen mir die Gestalten auch so gelingen, wie
                    das Hotel – es ist ersten Ranges.\pend
           \pstart
           Erhalten Sie mir Ihre Freundschaft und seien Sie von Herzen gegrüßt.\pend
           \pstart
           Von Ihren \textcolor{pink}{athen}{}\ledrightnote{\textcolor{pink}{Athen}}iensischen Abenteuern hatt ich
                    hier schon in der Zeitung gelesen. Mein Garten steht voll Rosen; – bin ich auch
                    kein \textcolor{pink}{griechischer}{}\ledrightnote{\textcolor{pink}{Griechenland}} Student – ich streue sie
                    alle im Geiste auf Ihr theures Haupt!\pend
           \pstart
           In Treue{\\[\baselineskip]}Ihr \spacefill\mbox{Arthur Schnitzler}\pend
           \leftskip=0em{}\endnumbering\briefempfaengerindex{Brandes, Georg@\textsc{Brandes, Georg}!zzzSchnitzler, Arthur@\emph{von Arthur Schnitzler}!1922-06-071@{7. 6. 1922}|)be}\mylabel{h}  \normalsize

\doendnotes{C}
\bigskip
\vfill

\clearpage

\footnotesize

\lohead{\textsc{register}}

% Definiere theindex-Environment komplett neu ohne reledmac
\makeatletter
\renewenvironment{theindex}{%
  \section*{\indexname}%
  \setlength{\parindent}{0pt}%
  \setlength{\parskip}{0pt plus 0.3pt}%
  \let\item\@idxitem
}{%
  \clearpage
}
\makeatother

\IfFileExists{\jobname-pw.ind}{\input{\jobname-pw.ind}}{}

\end{document}

      