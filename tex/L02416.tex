%% latex-korrekturansicht-vorspann.tex
%% Vorspann für die Korrekturansicht.
%% Lädt die gemeinsame Datei latex-vorspann.tex mit gesetztem Schalter.

\newif\ifkorrekturansicht
\korrekturansichttrue

\input{../tex-inputs/latex-vorspann}


               \section[Arthur Schnitzler an Felix Braun, 19. 10. 1924]{ Arthur Schnitzler an Felix Braun, 19. 10. 1924}\nopagebreak\mylabel{v}\rehead{ }\normalsize\beginnumbering\briefempfaengerindex{Braun, Felix@\textsc{Braun, Felix}!zzzSchnitzler, Arthur@\emph{von Arthur Schnitzler}!1924-10-191@{19. 10. 1924}|(be} \toendnotes[C]{\smallbreak\pagebreak[2]} \Standort{Wienbibliothek im Rathaus, H.I.N.-198.046.}
\physDesc{Postkarte
\newline{}Handschrift: schwarze Tinte, lateinische Kurrent\newline{}Versand: Stempel: »\nobreak{}\oindex{XVIII., Waehring@\textbf{XVIII., Währing}, \emph{Bezirk (A.BZK)}|pwk}18/1 Wien 110, 20. X. 24, 8\nobreak{}«.  }\toendnotes[C]{\smallbreak}\pstart{}{\pb}\label{T_L02416-1v}\edtext{\textcolor{gray}{\textbf{A. S.}}}{\lemma{\textnormal{\emph{A. S.}}}\Cendnote{\textnormal{ovaler Absenderkleber}}}\label{T_L02416-1h}\pend{}\pstart{}\textcolor{pink}{\textcolor{gray}{\textbf{WIEN, XVIII.}}}{}\ledrightnote{\textcolor{pink}{XVIII., Währing}}\pend{}\pstart{}\textcolor{pink}{\textcolor{gray}{\textbf{STERNWARTESTR. 71}}}{}\ledrightnote{\textcolor{pink}{Sternwartestraße}}\pend{}{\bigskip}\pstart{}Hrn Felix Braun\pend{}\pstart{}\textcolor{pink}{Wien XIX}{}\ledrightnote{\textcolor{pink}{XIX., Döbling}}\pend{}\pstart{}\textcolor{pink}{Sieveringerstr 191}{}\ledrightnote{\textcolor{pink}{Sieveringer Straße}}\pend{}{\bigskip}\pstart
           \raggedleft{}{\pb}\textcolor{pink}{Wien}{}\ledrightnote{\textcolor{pink}{Wien}}, 19. 10. 924\pend
           \pstart
           Verehrter und lieber Herr Felix Braun,  für Ihren schönen Brief
                    seien Sie sehr herzlich bedankt, ebenso wie für die beiden \textcolor{green}{Bücher}{}\ledrightnote{→\textcolor{green}{Wunderstunden. Drei Erzählungen}{\newline}→\textcolor{green}{Der unsichtbare Gast}}, \substVorne{}\textsuperscript{die}\substDazwischen{}von denen\substHinten{} ich eben das eine, die »\textcolor{green}{Wunderstunden}{}\ledrightnote{\textcolor{green}{Wunderstunden. Drei Erzählungen}}« mit innigstem Vergnügen gelesen habe. Wir begegnen einander
                    hoffentlich beide einmal wieder – ich wünschte sehr Sie fühlten meine
                    aufrichtige Sympathie auch aus diesen paar geschrie{\pb}benen Worten, wie ich mich der
                    Ihrigen in wohlthuender Weise gewiſs zu fühlen glaube. Ich drücke Ihnen die
                    Hand als Ihr herzlich ergebner\pend
           \pstart \spacefill\mbox{Arthur Schnitzler}\pend{}\endnumbering\briefempfaengerindex{Braun, Felix@\textsc{Braun, Felix}!zzzSchnitzler, Arthur@\emph{von Arthur Schnitzler}!1924-10-191@{19. 10. 1924}|)be}\mylabel{h}  \normalsize

\doendnotes{C}
\bigskip
\vfill

\clearpage

\footnotesize

\lohead{\textsc{register}}

% Definiere theindex-Environment komplett neu ohne reledmac
\makeatletter
\renewenvironment{theindex}{%
  \section*{\indexname}%
  \setlength{\parindent}{0pt}%
  \setlength{\parskip}{0pt plus 0.3pt}%
  \let\item\@idxitem
}{%
  \clearpage
}
\makeatother

\IfFileExists{\jobname-pw.ind}{\input{\jobname-pw.ind}}{}

\end{document}

      