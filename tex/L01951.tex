%% latex-korrekturansicht-vorspann.tex
%% Vorspann für die Korrekturansicht.
%% Lädt die gemeinsame Datei latex-vorspann.tex mit gesetztem Schalter.

\newif\ifkorrekturansicht
\korrekturansichttrue

\input{../tex-inputs/latex-vorspann}


               \section[Hugo von Hofmannsthal u. a. an Arthur Schnitzler, 23. 7. 1910]{ Hugo von Hofmannsthal u. a. an Arthur Schnitzler, 23. 7. 1910}\nopagebreak\mylabel{v}\rehead{ }\normalsize\beginnumbering\briefempfaengerindex{Schnitzler, Arthur@\textsc{Schnitzler, Arthur}!zzzFriedmann, Rose@\emph{von Rose Friedmann}!1910-07-232@{23. 7. 1910}|(be}\briefempfaengerindex{Schnitzler, Arthur@\textsc{Schnitzler, Arthur}!zzzHofmannsthal, Gertrude von@\emph{von Gertrude von Hofmannsthal}!1910-07-232@{23. 7. 1910}|(be}\briefempfaengerindex{Schnitzler, Arthur@\textsc{Schnitzler, Arthur}!zzzFriedmann, Louis Philipp@\emph{von Louis Philipp Friedmann}!1910-07-232@{23. 7. 1910}|(be}\briefempfaengerindex{Schnitzler, Arthur@\textsc{Schnitzler, Arthur}!zzzHofmannsthal, Hugo von@\emph{von Hugo von Hofmannsthal}!1910-07-232@{23. 7. 1910}|(be} \toendnotes[C]{\smallbreak\pagebreak[2]} \Standort{CUL, Schnitzler, B 43.}
\physDesc{Bildpostkarte
\newline{}Handschrift Hugo von Hofmannsthal: schwarze Tinte, deutsche Kurrent\newline{}Handschrift Louis Philipp Friedmann: Bleistift\newline{}Handschrift Gertrude von Hofmannsthal: schwarze Tinte\newline{}Handschrift Rose Friedmann: schwarze Tinte\newline{}Versand: Stempel: »\nobreak{}\oindex{Muenchen@\textbf{München}, \emph{https://www.geonames.org/ontologyP.PPLA}|pwk}München, 23. 7. 10, 4–5N\nobreak{}«.  
\newline{}Schnitzler: mit Bleistift beschriftet: »\textsc{Hugo}« \newline{}Ordnung: 1) mit Bleistift von unbekannter Hand nummeriert:
                                    »321« 2) mit Bleistift von unbekannter Hand nummeriert: »376«}\buchAbdrucke{\weitereDrucke{Hugo von Hofmannsthal, Arthur Schnitzler: \emph{Briefwechsel}. Hg. Therese Nickl und Heinrich Schnitzler. Frankfurt am Main: \emph{S. Fischer} 1964, S. 251.} }\pstart{}{\pb}\textsc{Herrn D\textsuperscript{r}}\pend{}\pstart{}\textsc{Arthur Schnitzler}\pend{}\pstart{}\textcolor{pink}{\textsc{Wien}}{}\ledrightnote{\textcolor{pink}{Wien}}\pend{}\pstart{}\textsc{\textcolor{pink}{XVIII. Spöttelgasse 7}{}\ledrightnote{\textcolor{pink}{Edmund-Weiß-Gasse}}.}\pend{}{\bigskip}\pstart
           \noindent{}\centering{}\textcolor{gray}{\textbf{{\pb}\textcolor{blue}{M. von Schwind}{}\ledrightnote{\textcolor{blue}{Moritz von Schwind}}}}\pend
           \pstart
           \noindent{}\centering{}\textcolor{gray}{\textbf{\textcolor{green}{Die Landpartie – La partie de campagne – The rural
                        excursion}{}\ledrightnote{\textcolor{green}{Die Landpartie. Schwind und Bauernfeld auf einem Leiterwagen}} (\textcolor{blue}{Schwind}{}\ledrightnote{\textcolor{blue}{Moritz von Schwind}} u. \textcolor{blue}{Bauernfeld}{}\ledrightnote{\textcolor{blue}{Eduard von Bauernfeld}}).}}\pend
           \pstart
           \centering{}{\pb}\textcolor{pink}{München}{}\ledrightnote{\textcolor{pink}{München}}. 23 VII.\pend
           \pstart
           Ganz ähnlich ſind wir geſtern u. vorgeſtern über \textcolor{pink}{Sa\substVorne{}\textsuperscript{z}\substDazwischen{}lz\substHinten{}burg}{}\ledrightnote{\textcolor{pink}{Salzburg}} hierher gefahren und doch auch wieder unähnlich. Es iſt eine
               Preisaufgabe für \textcolor{blue}{\textsc{Heini}}{}\ledrightnote{\textcolor{blue}{Heinrich Schnitzler}}, a. die Ähnlichkeit – b.) die Unterſchiede herauszufinden.\hspace*{1.5em}Viele Grüße Euch beiden.\pend
           \pstart
           \spacefill\mbox{Hugo.}{\\[\baselineskip]}\spacefill\mbox{{[}hs. Friedmann:{]} L. Friedmann}{\\[\baselineskip]}\spacefill\mbox{{[}hs. G. Hofmannsthal:{]} Gerty}{\\[\baselineskip]}\spacefill\mbox{{[}hs. Friedmann:{]} Rose Friedmann}\pend
           \leftskip=0em{}\endnumbering\briefempfaengerindex{Schnitzler, Arthur@\textsc{Schnitzler, Arthur}!zzzFriedmann, Rose@\emph{von Rose Friedmann}!1910-07-232@{23. 7. 1910}|)be}\briefempfaengerindex{Schnitzler, Arthur@\textsc{Schnitzler, Arthur}!zzzHofmannsthal, Gertrude von@\emph{von Gertrude von Hofmannsthal}!1910-07-232@{23. 7. 1910}|)be}\briefempfaengerindex{Schnitzler, Arthur@\textsc{Schnitzler, Arthur}!zzzFriedmann, Louis Philipp@\emph{von Louis Philipp Friedmann}!1910-07-232@{23. 7. 1910}|)be}\briefempfaengerindex{Schnitzler, Arthur@\textsc{Schnitzler, Arthur}!zzzHofmannsthal, Hugo von@\emph{von Hugo von Hofmannsthal}!1910-07-232@{23. 7. 1910}|)be}\mylabel{h}  \normalsize

\doendnotes{C}
\bigskip
\vfill

\clearpage

\footnotesize

\lohead{\textsc{register}}

% Definiere theindex-Environment komplett neu ohne reledmac
\makeatletter
\renewenvironment{theindex}{%
  \section*{\indexname}%
  \setlength{\parindent}{0pt}%
  \setlength{\parskip}{0pt plus 0.3pt}%
  \let\item\@idxitem
}{%
  \clearpage
}
\makeatother

\IfFileExists{\jobname-pw.ind}{\input{\jobname-pw.ind}}{}

\end{document}

      