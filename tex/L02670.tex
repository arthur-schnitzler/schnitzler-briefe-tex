%% latex-korrekturansicht-vorspann.tex
%% Vorspann für die Korrekturansicht.
%% Lädt die gemeinsame Datei latex-vorspann.tex mit gesetztem Schalter.

\newif\ifkorrekturansicht
\korrekturansichttrue

\input{../tex-inputs/latex-vorspann}


               \section[Paul Goldmann an Arthur Schnitzler, 15. 11. 1891]{ Paul Goldmann an Arthur Schnitzler, 15. 11. 1891}\nopagebreak\mylabel{v}\rehead{ }\normalsize\beginnumbering\briefempfaengerindex{Schnitzler, Arthur@\textsc{Schnitzler, Arthur}!zzzGoldmann, Paul@\emph{von Paul Goldmann}!1891-11-151@{15. 11. 1891}|(be} \toendnotes[C]{\smallbreak\pagebreak[2]} \Standort{DLA, A:Schnitzler, HS.NZ85.1.3162.}
\physDesc{Brief, 2 Blätter, 8 Seiten
\newline{}Handschrift: blaue Tinte, deutsche Kurrent
\newline{}Schnitzler: mit rotem Buntstift eine Unterstreichung und eine seitliche
                                 Markierung }\toendnotes[C]{\smallbreak}\pstart
           \noindent{}\centering{}{\pb}\textcolor{gray}{\textbf{Dr. jur. Paul Goldmann}}\pend
           \pstart
           \noindent{}\centering{}\textcolor{gray}{\textbf{\begin{otherlanguage}{french}Correspondant de la »\textcolor{brown}{Gazette de Francfort}{}\ledrightnote{\textcolor{brown}{Frankfurter Zeitung}}«\end{otherlanguage}}}\pend
           \pstart
           \noindent{}\centering{}\textcolor{gray}{\textbf{\textcolor{pink}{\begin{otherlanguage}{french}Bruxelles, 21, rue des Plantes\end{otherlanguage}}{}\ledrightnote{\textcolor{pink}{rue des Plantes}}.}}\pend
           \pstart
           \raggedleft{}\textcolor{pink}{Brüſſel}{}\ledrightnote{\textcolor{pink}{Brüssel}},
                  15. November 1891.\pend
           \pstart\center{}Mein lieber Arthur!\pend\pstart
           Der Dank für Deine lieben Briefe, die mich unendlich erfreut haben, brennt mir ſchon
               lange auf dem Herzen. Aber eine große Affaire, die ſeit ein paar Wochen im Zuge iſt,
               hat mir bisher die Hände gebunden. Heut iſt es entſchieden: in 14 Tagen gehe ich nach
                  \textcolor{pink}{Paris}{}\ledrightnote{\textcolor{pink}{Paris}} als politiſcher und literariſcher
               Correſpondent der »\textcolor{brown}{Frankfurter Zeitung}{}\ledrightnote{\textcolor{brown}{Frankfurter Zeitung}}«.
               Äußerlich recht ehrenvoll. Innerlich, unter uns, nur ein Verſuch ſeitens des \textcolor{brown}{Blattes}{}\ledrightnote{→\textcolor{brown}{Frankfurter Zeitung}}, eine billige junge
               Kraft in zehnfachem Maße auszubeuten als bisher. Die Arbeit in \textcolor{pink}{Paris}{}\ledrightnote{\textcolor{pink}{Paris}} wächſt in’s Unendliche, desgleichen die
               Verantwortlichkeit; keiner der früheren Correſpondenten {\pb}hat ſich noch länger als drei Jahre halten können.
               In Bezug auf den Gehalt werde ich wahrſcheinlich betrogen werden; die kleine Erhöhung
               gegen bisher wird durch die theuren Lebensverhältniſſe aufgewogen; von meinem
               einzigen Ziel, zur Selbſtändigkeit zu \strikeout{g\textcolor{gray}{l}} gelangen, bin ich alſo ferner als je. Und bei meinem Ekel vor der Politik, der
               ſich hier noch \strikeout{ac} accentuirt hat, bei meiner Ignoranz
               in der franzöſiſchen Sprache, bei meinem Hang zur ruhigen, \strikeout{\textcolor{gray}{ſt}} friedlichen, langſamen Arbeit habe ich alle Ausſichten, mich nicht zu bewähren
               und nicht zum Wohlbehagen zu gelangen. Ich gehe morgen
               von hier fort. Die \textcolor{pink}{Stadt}{}\ledrightnote{→\textcolor{pink}{Brüssel}} iſt
               mir in den letzten Wochen lieb geworden; ich war im Begriff, mein \textsc{Milieu} zu finden. Und im Augenblick, wo ich mich hübſch
               behaglich in eine warme Ecke drücken will, {\pb}\strikeout{reißt} reißt das Leben die Thür auf, zwingt mir wieder
               den Wanderſtab \strikeout{heraus} in die Hand und ſtößt mich in
               den Sturm und Regen der Landſtraße hinaus. Gott weiß allein, was er mit mir
               vorhat.\pend
           \pstart
           Vielleicht finde ich vor meiner Abreiſe von hier noch Zeit, Dir ausführlich zu
               ſchreiben. Einſtweilen laß’ Dir mit einem innigen Dankwort genügen für den
               Wärmeſtrom, den Du mit Deinen lieben Briefen in mein Herz geleitet. Was mich im
               Beſonderen für Dich erfreut, das iſt ein gewiſſer Hauch von Arbeitsfreude, der daraus
               hervorweht. Wenn das keine vorübergehende Stimmung, ſondern ein bleibender
               Seelenzuſtand iſt, ſo gibt es kein noch ſo hohes Ziel, deſſen Erreichung ich für Dich
               nicht erhoffe. Einer Sorge möchte ich gleich hier Ausdruck verleihen: \strikeout{ich} die \label{K_L02670-1v}\edtext{Bedenken}{\lemma{\textnormal{\emph{Bedenken}}}\Cendnote{\textnormal{siehe Paul Goldmann an Arthur Schnitzler, 4. 8. 1891. Am 28. 10. 1891 hatte der erste (und letzte)
                  »gesellige Abend« stattgefunden, bei dem von \textcolor{blue}{Schnitzler} zwei Gedichte vorgetragen worden waren und von dem \textcolor{blue}{Schnitzler}{ }\textcolor{blue}{Goldmann} berichtet haben dürfte.}}}\label{K_L02670-1h},
               welche {\pb}ich gegen das Bodenfaſſen der »\textcolor{brown}{Freien-Bühne}{}\ledrightnote{\textcolor{brown}{»Freie Bühne« Verein für moderne Literatur}}«-Bewegung gehabt, ſind jetzt in mir
               faſt zur negativen Gewißheit erwachſen. Die \label{K_L02670-2v}\edtext{Macher der \textcolor{brown}{Bewegung}{}\ledrightnote{→\textcolor{brown}{»Freie Bühne« Verein für moderne Literatur}}}{\lemma{\textnormal{\emph{Macher der Bewegung}}}\Cendnote{\textnormal{Am 7. 7. 1891 fand die Gründungssitzung von \emph{\textcolor{brown}{Freie Bühne, Verein für moderne Literatur}}
                  statt. Zum \textcolor{blue}{Obmann} wurde
                     \textcolor{blue}{Friedrich Michael Fels} gewählt, \textcolor{blue}{Stellvertreter} wurden \textcolor{blue}{Edmund Wengraf} und \textcolor{blue}{Hermann Fürst}. \textcolor{blue}{Schnitzler} war selbst Ausschussmitglied des \textcolor{brown}{Vereins}. Vgl. Paul Goldmann an Arthur Schnitzler, 4. 8. 1891}}}\label{K_L02670-2h} ſind \strikeout{zu} theils zu wenig erfahren, theils zu
               wenig begabt, theils zu wenig ehrlich; und der blöde Widerſtand des Publicums wie
               ſeiner Lakaien, der »Kritiker«, iſt auf dieſe Weiſe nicht zu brechen. Die \textsc{\textcolor{blue}{Wengraf}{}\ledrightnote{\textcolor{blue}{Edmund Wengraf}}s etc.} ſind die Schlauen,
               welche Wind \strikeout{h} davon haben und beizeiten ihren Einſatz
               aus dem Spiele ziehen. Denen werden wahrſcheinlich noch Andere folgen. Nun möchte ich
               um Alles in der Welt nicht, daß Du das Opfer Deiner makelloſen Ehrlichkeit wirſt und
               Deinen guten Namen an eine Sache hefteſt, die ihn bei ihrem Zuſammenbruch ſchwer
               compromittiren könnte. Ein Martyrium für die gute Sache – {\pb}meinetwegen! Aber die Sache iſt nicht gut – dieſe
               Sache der \textsc{\textcolor{blue}{Joachim}{}\ledrightnote{\textcolor{blue}{Jaques Joachim}}s}, \textsc{\textcolor{blue}{Kafka}{}\ledrightnote{\textcolor{blue}{Eduard Michael Kafka}}s }\textsc{etc.} Und darum meine ich: wenn die \textcolor{brown}{Unternehmung}{}\ledrightnote{→\textcolor{brown}{»Freie Bühne« Verein für moderne Literatur}} nicht unbedingte Ausſicht auf
                  \label{K_L02670-3v}\edtext{Gedeihen}{\lemma{\textnormal{\emph{Gedeihen}}}\Cendnote{\textnormal{Tatsächlich kriselte es in der \emph{\textcolor{brown}{Freien Bühne}} bereits wenige Wochen nach der Gründung. In einem \textcolor{green}{Theaterbrief} begründete \textcolor{blue}{Friedrich Michael Fels} das Scheitern des \textcolor{brown}{Verein}s damit, dass zu wenig
                  der geplanten Vorhaben umgesetzt wurden und außer dem einen »geselligen Abend«
                  nichts zustande kam. (\textcolor{blue}{Friedrich Michael Fels}: \emph{\textcolor{green}{Wiener Brief}}. In: \emph{\textcolor{green}{Freie
                        Bühne für den Entwickelungskampf der Zeit}}, Jg. 3, H. 1, Februar 1892, S. 197–201.)}}}\label{K_L02670-3h} bietet;
               wenn Du nicht ſelbſt unumſchränkt leiten kannſt – ſo zieh’ auch Du Dich ein wenig
               zurück. Du brauchſt, weiß Gott, keine \textcolor{brown}{Partei}{}\ledrightnote{→\textcolor{brown}{»Freie Bühne« Verein für moderne Literatur}} und biſt ſtark genug, deine eigenen Wege zu gehen.
               Eine \label{K_L02670-4v}\edtext{Aufführung des »\textcolor{green}{Märchen}{}\ledrightnote{\textcolor{green}{Das Märchen. Schauspiel in drei Aufzügen}}«}{\lemma{\textnormal{\emph{Aufführung des »Märchen«}}}\Cendnote{\textnormal{\emph{\textcolor{green}{Das Märchen}} wurde eine Zeit lang – und
                  offenbar bis zur Gegenwart dieses Briefes – als Inszenierung der \emph{\textcolor{brown}{Freien Bühne}} erwogen (vgl. A. S.: \emph{Tagebuch}, 13. 7. 1891). \textcolor{blue}{Schnitzler} selbst lehnte dies jedoch ab und wollte das \textcolor{green}{Drama} am \emph{\textcolor{brown}{Burgtheater}} aufgeführt wissen.}}}\label{K_L02670-4h} durch die »\textcolor{brown}{Freie Bühne}{}\ledrightnote{\textcolor{brown}{»Freie Bühne« Verein für moderne Literatur}}«, wenn nicht ganz vorzügliche ſchauſpieleriſche
               Kräfte geſichert ſind, hielte ich für eine große Gefahr. Das Publicum iſt zu dumm, um
               das \textcolor{green}{Stück}{}\ledrightnote{→\textcolor{green}{Das Märchen. Schauspiel in drei Aufzügen}} zu begreifen; und
               auf der andern Seite mangelt der »\textcolor{brown}{Freien Bühne}{}\ledrightnote{\textcolor{brown}{»Freie Bühne« Verein für moderne Literatur}}«
                  {\pb}in \textcolor{pink}{Wien}{}\ledrightnote{\textcolor{pink}{Wien}} die
               Autorität, welche, als Surrogat des Verſtändniſſes, das dumme Volk zum Beifall
               zwingt. Nach dem von den »führenden Geiſtern« der Preſſe ausgehenden Loſungswort wird
               jeder Lausbub ſich berechtigt glauben, Kritik zu üben; und die Zeitungen werden Dich
               zerreißen oder mit, \strikeout{\textcolor{gray}{g}} vernichtendem Wohlwollen behandeln. (\label{K_L02670-55v}\edtext{\textsc{N. B.}}{\lemma{\textnormal{\emph{N. B.}}}\Cendnote{\textnormal{nota bene, lateinisch: merke
                  wohl}}}\label{K_L02670-55h}{ }\label{K_L02670-5v}\edtext{\textsc{\textcolor{blue}{Hugo Klein}{}\ledrightnote{\textcolor{blue}{Hugo Klein}}s}{ }\textcolor{green}{Artikel}{}\ledrightnote{\textcolor{green}{»Freie Bühne«}}}{\lemma{\textnormal{\emph{Hugo Kleins Artikel}}}\Cendnote{\textnormal{\textcolor{blue}{h. k.}: \emph{\textcolor{green}{»Freie Bühne«}}. In: XXXXX\textcolor{red}{\textsuperscript{\textbf{KEY}}}, Jg. YY, Nr. YYY,
                        30. 10. 1891, S. YY. \textcolor{blue}{Klein} schreibt darin satirisch-kritisch über den ersten Vortragsabend der
                     \emph{\textcolor{brown}{Freien Bühne}} am 28. 10. 1891 und erwähnt dabei \textcolor{blue}{Schnitzler} folgendermaßen: »zwei
                     Gedichte von \textcolor{blue}{Arthur Schnitzler}, von
                     welchen besonders das eine: ›Am Flügel‹, unverkennbar den Einfluß \textcolor{blue}{Baumbach}’s widerspiegelt«. vgl. A. S.: \emph{Tagebuch}, 30. 10. 1891}}}\label{K_L02670-5h} habe ich
               geleſen; wäre ich in \textcolor{pink}{Wien}{}\ledrightnote{\textcolor{pink}{Wien}} geweſen, ich hätte den
                  \textcolor{blue}{Burſchen}{}\ledrightnote{→\textcolor{blue}{Hugo Klein}} geohrfeigt,
               allein wegen der Stelle über Dich!). Etwas Anderes wäre die Aufführung in \textcolor{pink}{Berlin}{}\ledrightnote{\textcolor{pink}{Berlin}}. Kein ſicherer Erfolg freilich; aber dort
               wirſt Du wenigſtens von Einigen ſo ernſt genormmen werden, als Du es verdienſt. Ich
               halte es für das Beſte, die \strikeout{\textcolor{gray}{×}\-\textcolor{gray}{×}\-\textcolor{gray}{×}\-\textcolor{gray}{×}\-\textcolor{gray}{×}\-\textcolor{gray}{×}\-\textcolor{gray}{×}}{ }\label{K_L02670-6v}\edtext{Antwort \textsc{\textcolor{blue}{Blumenthal}{}\ledrightnote{\textcolor{blue}{Oskar Blumenthal}}s}}{\lemma{\textnormal{\emph{Antwort Blumenthals}}}\Cendnote{\textnormal{Siehe Oscar Blumenthal an Arthur Schnitzler, 15. 12. 1891}}}\label{K_L02670-6h} abzuwarten und {\pb}vorher in \textcolor{pink}{Wien}{}\ledrightnote{\textcolor{pink}{Wien}} nicht einen Schritt zu thun. In \textsc{\textcolor{blue}{Burckhard}{}\ledrightnote{\textcolor{blue}{Max Eugen Burckhard}}s}{ }\label{K_L02670-7v}\edtext{Antwort}{\lemma{\textnormal{\emph{Antwort}}}\Cendnote{\textnormal{\textcolor{blue}{Schnitzler} hatte die Nachricht, dass \textcolor{blue}{Max Burckhard}{ }\emph{\textcolor{green}{Das Märchen}} nicht am \emph{\textcolor{brown}{Burgtheater}} inszenieren werde, am 28. 10. 1891 erhalten. Sie dürfte eher mündlich
                  als schriftlich mitgeteilt worden sein. Jedenfalls hat sich kein entsprechendes
                  Korrespondenzstück erhalten. Als Begründung notierte sich \textcolor{blue}{Schnitzler} im \emph{\textcolor{green}{Tagebuch}}: »zu viel Rede, zu wenig Handlung«.}}}\label{K_L02670-7h} liegt,
               trotz der \label{K_L02670-78v}\edtext{literariſch-ungebildeten
                  Form}{\lemma{\textnormal{\emph{literariſch-ungebildeten Form}}}\Cendnote{\textnormal{Anspielung darauf, dass \textcolor{blue}{Burckhard} Jurist war und ohne
                  künstlerisch-artistische Vorerfahrung die Leitung des \emph{\textcolor{brown}{Burgtheater}}s überantwortet bekommen hatte.}}}\label{K_L02670-78h},
               vielleicht ein geſunder Inſtinct. Du hätteſt ihm unter allen Umſtänden \label{K_L02670-56v}\edtext{zuerſt den \textsc{\textcolor{green}{Alkandi}{}\ledrightnote{\textcolor{green}{Alkandi’s Lied}}}}{\lemma{\textnormal{\emph{zuerſt den Alkandi}}}\Cendnote{\textnormal{Diesen Einakter hatte \textcolor{blue}{Max Burckhard} bereits am 14. 7. 1891 abgelehnt (Max Burckhard an Arthur Schnitzler, 14. 7. 1891).}}}\label{K_L02670-56h} geben ſollen; und ich rathe Dir entſchieden, es auch
               jetzt noch zu thun. Bringt er das \textcolor{green}{Stück}{}\ledrightnote{→\textcolor{green}{Alkandi’s Lied}} und gefällt es, ſo wäre es gar nicht unmöglich, daß er noch auf das
                  »\textcolor{green}{Märchen}{}\ledrightnote{\textcolor{green}{Das Märchen. Schauspiel in drei Aufzügen}}« zurückkäme. Im Übrigen behalte ich
               mir alle näheren Urtheile bis nach der Lectüre vor, die ich aufrichtigſt
               herbeiwünſche.\pend
           \pstart
           Dies für heut. Tauſend Dank noch für die Beantwortung
               meiner Fragen, die ausführlichen Mittheilungen über die Lieben in \textcolor{pink}{Wien}{}\ledrightnote{\textcolor{pink}{Wien}}, und all’ das Gütige und Freundſchaftliche, das Deine {\pb}Briefe ſonſt noch enthalten haben. Sie waren mir
               eine Art Feſtgeſchenk. Ehe ich von hier ſcheide (ich fahre etwa am 30. November) höre ich wohl noch ein Wort von Dir? Viele,
               viele Grüße an die \textcolor{pink}{Wien}{}\ledrightnote{\textcolor{pink}{Wien}}er Freunde, vor Allem \textsc{\textcolor{blue}{Richard}{}\ledrightnote{\textcolor{blue}{Richard Beer-Hofmann}}} und \textsc{\textcolor{blue}{Loris}{}\ledrightnote{\textcolor{blue}{Hugo von Hofmannsthal}}} und \textsc{\textcolor{blue}{Kapper}{}\ledrightnote{\textcolor{blue}{Friedrich Kapper}}}. Einen herzlichen Händedruck an \textsc{\textcolor{blue}{Salten}{}\ledrightnote{\textcolor{blue}{Felix Salten}}}, der mein ſeeliger \label{K_L02670-555v}\edtext{\textcolor{blue}{Erbe auf dem gewiſſen mit
                  Kiſſen}{}\ledrightnote{→\textcolor{blue}{?? [Partnerin von Paul Goldmann und später Felix Salten]}}}{\lemma{\textnormal{\emph{Erbe … Kiſſen}}}\Cendnote{\textnormal{nicht ermittelt}}}\label{K_L02670-555h} weich drapirten
               Sopha geworden zu ſein ſcheint. Ergebene Empfehlungen an die Deinen. Vielen Dank und
               Gruß an »\label{K_L02670-77v}\edtext{\textcolor{blue}{es}{}\ledrightnote{→\textcolor{blue}{Marie Glümer}}}{\lemma{\textnormal{\emph{es}}}\Cendnote{\textnormal{das »süße Mädel«, \textcolor{blue}{Marie Glümer}}}}\label{K_L02670-77h}«, das meiner ſo treulich gedenkt. Und, um im Austheilen der Gnaden
               fortzufahren, Dir, mein lieber Alter, das goldene Vließ meines Erbhauſes: eine
               herzliche Umarmung!\pend
           \pstart
           Dein {\\[\baselineskip]}treuer {\\[\baselineskip]}\spacefill\mbox{Paul Goldmann.}\pend
           \leftskip=0em{}\pstart
           \noindent{}\textsc{À propos}: Kennſt Du wen in \textcolor{pink}{Paris}{}\ledrightnote{\textcolor{pink}{Paris}}, an den Du mich empfehlen könnteſt?\pend
           \endnumbering\briefempfaengerindex{Schnitzler, Arthur@\textsc{Schnitzler, Arthur}!zzzGoldmann, Paul@\emph{von Paul Goldmann}!1891-11-151@{15. 11. 1891}|)be}\mylabel{h}  \normalsize

\doendnotes{C}
\bigskip
\vfill

\clearpage

\footnotesize

\lohead{\textsc{register}}

% Definiere theindex-Environment komplett neu ohne reledmac
\makeatletter
\renewenvironment{theindex}{%
  \section*{\indexname}%
  \setlength{\parindent}{0pt}%
  \setlength{\parskip}{0pt plus 0.3pt}%
  \let\item\@idxitem
}{%
  \clearpage
}
\makeatother

\IfFileExists{\jobname-pw.ind}{\input{\jobname-pw.ind}}{}

\end{document}

      