%% latex-korrekturansicht-vorspann.tex
%% Vorspann für die Korrekturansicht.
%% Lädt die gemeinsame Datei latex-vorspann.tex mit gesetztem Schalter.

\newif\ifkorrekturansicht
\korrekturansichttrue

\input{../tex-inputs/latex-vorspann}


               \section[Felix Braun an Arthur Schnitzler, 31. 12. 1924]{ Felix Braun an Arthur Schnitzler, 31. 12. 1924}\nopagebreak\mylabel{v}\rehead{ }\normalsize\beginnumbering\briefempfaengerindex{Schnitzler, Arthur@\textsc{Schnitzler, Arthur}!zzzBraun, Felix@\emph{von Felix Braun}!1924-12-311@{31. 12. 1924}|(be} \toendnotes[C]{\smallbreak\pagebreak[2]} \Standort{DLA, A:Schnitzler, HS.NZ85.1.2604,5.}
\physDesc{Postkarte
\newline{}Handschrift: schwarze Tinte, deutsche Kurrent\newline{}Versand: Stempel: »\nobreak{}\oindex{XVI., Ottakring@\textbf{XVI., Ottakring}, \emph{Bezirk (A.BZK)}|pwk}\textcolor{gray}{16/1} Wien, 31. XII. 2\textcolor{gray}{4}, 9\nobreak{}«.  }\toendnotes[C]{\smallbreak}\pstart{}{\pb}\textsc{Felix Braun}\pend{}\pstart{}\textcolor{pink}{\textsc{Wien – Sievering}}{}\ledrightnote{\textcolor{pink}{Sievering}}\pend{}{\bigskip}\pstart{}\textsc{Herrn Dr.}\pend{}\pstart{}\textsc{Arthur Schnitzler}\pend{}\pstart{}\textcolor{pink}{\textsc{Wien XVIII}}{}\ledrightnote{\textcolor{pink}{XVIII., Währing}}\pend{}\pstart{}\textcolor{pink}{\textsc{Sternwartestraße 71}}{}\ledrightnote{\textcolor{pink}{Sternwartestraße}}\pend{}{\bigskip}\pstart
           \centering{}{\pb}\textcolor{pink}{Wien}{}\ledrightnote{\textcolor{pink}{Wien}} / 31. XII. 24\pend
           \pstart{}Verehrter Herr Doktor!\pend\pstart
           Vielen, herzlichen Dank für Ihre ſehr liebe Karte! Inzwiſchen iſt wohl auch mein
                    Brief mit dem Dank für das \textcolor{green}{Geſchenk}{}\ledrightnote{→\textcolor{green}{Fräulein Else}} gekommen, das Sie mir ſo freundlich gemacht haben. Möchte
                    Ihnen, verehrter Herr Doktor, das neue Jahr viel Gutes bringen!\pend
           \pstart
           In herzlicher Ergebenheit Ihr{\\[\baselineskip]}\spacefill\mbox{Felix Braun.}\pend
           \leftskip=0em{}\endnumbering\briefempfaengerindex{Schnitzler, Arthur@\textsc{Schnitzler, Arthur}!zzzBraun, Felix@\emph{von Felix Braun}!1924-12-311@{31. 12. 1924}|)be}\mylabel{h}  \normalsize

\doendnotes{C}
\bigskip
\vfill

\clearpage

\footnotesize

\lohead{\textsc{register}}

% Definiere theindex-Environment komplett neu ohne reledmac
\makeatletter
\renewenvironment{theindex}{%
  \section*{\indexname}%
  \setlength{\parindent}{0pt}%
  \setlength{\parskip}{0pt plus 0.3pt}%
  \let\item\@idxitem
}{%
  \clearpage
}
\makeatother

\IfFileExists{\jobname-pw.ind}{\input{\jobname-pw.ind}}{}

\end{document}

      