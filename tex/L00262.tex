%% latex-korrekturansicht-vorspann.tex
%% Vorspann für die Korrekturansicht.
%% Lädt die gemeinsame Datei latex-vorspann.tex mit gesetztem Schalter.

\newif\ifkorrekturansicht
\korrekturansichttrue

\input{../tex-inputs/latex-vorspann}


               \section[Arthur Schnitzler an Richard Beer-Hofmann, 13. 9. 1893]{ Arthur Schnitzler an Richard Beer-Hofmann, 13. 9. 1893}\nopagebreak\mylabel{v}\rehead{ }\normalsize\beginnumbering\briefempfaengerindex{Beer-Hofmann, Richard@\textsc{Beer-Hofmann, Richard}!zzzSchnitzler, Arthur@\emph{von Arthur Schnitzler}!1893-09-131@{13. 9. 1893}|(be} \toendnotes[C]{\smallbreak\pagebreak[2]} \Standort{YCGL, MSS 31.}
\physDesc{Brief, 1 Blatt (Briefpapier mit Trauerrand), 3 Seiten, Umschlag mit Trauerrand
\newline{}Handschrift: Bleistift, deutsche Kurrent\newline{}Versand: 1) Stempel: »\nobreak{}Wien 1/1, 1\textcolor{gray}{3}. 9. 9\textcolor{gray}{3}, 11–12 N\nobreak{}«.  2) Stempel: »\nobreak{}\oindex{Znaim@\textbf{Znaim}, \emph{Besiedelter Ort (A.BSO)}|pwk}\textcolor{gray}{Zn}ojmo, 14 9 93, 10{[}–12{]} \textcolor{gray}{N}\nobreak{}«. }\buchAbdrucke{\weitereDrucke{Arthur Schnitzler, Richard Beer-Hofmann: \emph{Briefwechsel 1891–1931}. Hg. Konstanze Fliedl. Wien, Zürich: \emph{Europaverlag} 1992, S. 53.} }\pstart{}{\pb}Herrn \textsc{Dr. Richard
                     Beer-Hofmann}\pend{}\pstart{}k. u. k. Lieutenant in der Ref. beim k. k. Inf. Regimente Nr. 99\pend{}\pstart{}in \textcolor{pink}{Znaim}{}\ledrightnote{\textcolor{pink}{Znaim}}. \pend{}{\bigskip}\pstart{}{\pb}Lieber Richard,\pend\pstart
           Ihre Karte fand ich Montag, als ich von \textcolor{pink}{Reichenau}{}\ledrightnote{\textcolor{pink}{Reichenau an der Rax}} zurück kam; habe ſehr bedauert, dß ich Sie verſäumen mußte. –\pend
           \pstart
           Samſtag fahre ich auf 2–3 Tage nach \textcolor{pink}{Salzburg}{}\ledrightnote{\textcolor{pink}{Salzburg}}, wo ſich
                  \textcolor{blue}{Goldma{\geminationn}}{}\ledrightnote{\textcolor{blue}{Paul Goldmann}} be{\pb}findet. –\pend
           \pstart
           Geſtern hab ich den Vertrag mit dem \textcolor{pink}{\textsc{Dtsch. Volksth}.}{}\ledrightnote{\textcolor{pink}{Volkstheater}} unterſchrieben, nach welchem das
                  \textcolor{green}{M.}{}\ledrightnote{\textcolor{green}{Das Märchen. Schauspiel in drei Aufzügen}} vor 1. Dezember 93 in Scene
               gehen müſſte, – »in würdiger Aufführung« wie es im Vertrag heißt. –\pend
           \pstart
           {\pb}Laſſen Sie was von sich hören, ko{\geminationm}en Sie in guter Sti{\geminationm}ung
               zurück und ſeien Sie herzlich gegrüßt!\pend
           \pstart Ihr\spacefill\mbox{Arthur}\pend{}\pstart
           \textcolor{pink}{Wien}{}\ledrightnote{\textcolor{pink}{Wien}}{ }13. 9 93.\pend
           \endnumbering\briefempfaengerindex{Beer-Hofmann, Richard@\textsc{Beer-Hofmann, Richard}!zzzSchnitzler, Arthur@\emph{von Arthur Schnitzler}!1893-09-131@{13. 9. 1893}|)be}\mylabel{h}  \normalsize

\doendnotes{C}
\bigskip
\vfill

\clearpage

\footnotesize

\lohead{\textsc{register}}

% Definiere theindex-Environment komplett neu ohne reledmac
\makeatletter
\renewenvironment{theindex}{%
  \section*{\indexname}%
  \setlength{\parindent}{0pt}%
  \setlength{\parskip}{0pt plus 0.3pt}%
  \let\item\@idxitem
}{%
  \clearpage
}
\makeatother

\IfFileExists{\jobname-pw.ind}{\input{\jobname-pw.ind}}{}

\end{document}

      