%% latex-korrekturansicht-vorspann.tex
%% Vorspann für die Korrekturansicht.
%% Lädt die gemeinsame Datei latex-vorspann.tex mit gesetztem Schalter.

\newif\ifkorrekturansicht
\korrekturansichttrue

\input{../tex-inputs/latex-vorspann}


               \section[Arthur Schnitzler an Hugo von Hofmannsthal, 12. 7. 1893]{ Arthur Schnitzler an Hugo von Hofmannsthal, 12. 7. 1893}\nopagebreak\mylabel{v}\rehead{ }\normalsize\beginnumbering\briefempfaengerindex{Hofmannsthal, Hugo von@\textsc{Hofmannsthal, Hugo von}!zzzSchnitzler, Arthur@\emph{von Arthur Schnitzler}!1893-07-121@{12. 7. 1893}|(be} \toendnotes[C]{\smallbreak\pagebreak[2]} \Standort{FDH, Hs-30885,36.}
\physDesc{Brief, 1 Blatt (Briefpapier mit Trauerrand), 4 Seiten
\newline{}Handschrift: schwarze Tinte, deutsche Kurrent\newline{}Ordnung: von Schnitzler mutmaßlich bei der Durchsicht der Korrespondenz 1929 mit
                                    Bleistift datiert: »12. 7. 93« }\buchAbdrucke{\weitereDrucke{Hugo von Hofmannsthal, Arthur Schnitzler: \emph{Briefwechsel}. Hg. Therese Nickl und Heinrich Schnitzler. Frankfurt am Main: \emph{S. Fischer} 1964, S. 40.} }\toendnotes[C]{\smallbreak}\pstart{}{\pb}Lieber Loris,\pend\pstart
           meine \label{K_L00236_1v}\edtext{\textcolor{green}{Einakter}{}\ledrightnote{→\textcolor{green}{Abschiedssouper}{\newline}→\textcolor{green}{Die Frage an das Schicksal}}}{\lemma{\textnormal{\emph{Einakter}}}\Cendnote{\textnormal{Nur \emph{\textcolor{green}{Abschiedssouper}} wurde gegeben.}}}\label{K_L00236_1h}{ }ſind
                        Freitag. Erſte Probe geſtern – \textcolor{green}{Anatol}{}\ledrightnote{→\textcolor{green}{Abschiedssouper}} (Herr \textcolor{blue}{\textsc{Hoefer}}{}\ledrightnote{\textcolor{blue}{Emil Höfer}}) erſchien einfach nicht. – Ich nahm mit \textcolor{blue}{\textsc{Jarno}}{}\ledrightnote{\textcolor{blue}{Josef Jarno}} die \textcolor{green}{Stücke}{}\ledrightnote{→\textcolor{green}{Abschiedssouper}{\newline}→\textcolor{green}{Die Frage an das Schicksal}} durch;
                    Inſcenierung, Stellung etc. – Die \textcolor{blue}{\textsc{Griebl}}{}\ledrightnote{\textcolor{blue}{Marie Griebl}} gibt die \textcolor{green}{\textsc{Annie}}{}\ledrightnote{→\textcolor{green}{Abschiedssouper}}. – \pend
           \pstart
           Urtheil \textcolor{blue}{\textsc{Friese}}{}\ledrightnote{\textcolor{blue}{Carl Adolph Friese}}’s: Es iſt {\pb}ein Skandal, ſo was
                    aufzuführen. – Frau \textcolor{blue}{\textsc{Friese}}{}\ledrightnote{\textcolor{blue}{Josefine Skura}} (dieſe alte Stabscanaille, wie \textcolor{blue}{\textsc{Jarno}}{}\ledrightnote{\textcolor{blue}{Josef Jarno}}{ }ſagt) hat ſich \uline{geſchämt}, wie ſie das
                        \textcolor{green}{Abſch.-\textsc{souper}}{}\ledrightnote{\textcolor{green}{Abschiedssouper}} geleſen. –\pend
           \pstart
           Die Cenſur ſtrich: \uline{am Buſen geruht} u ſetzte dafür
                        \uline{gekoſt}. –\pend
           \pstart
           – Ob mir die Geſchichte für \textcolor{pink}{Berlin}{}\ledrightnote{\textcolor{pink}{Berlin}} nützen
                    wird, iſt nicht abzuſehen – da \textcolor{blue}{\textsc{Jarno}}{}\ledrightnote{\textcolor{blue}{Josef Jarno}} höchſt un{\pb}verläßlich zu ſein ſcheint.
               Ihm hat die \textcolor{green}{Frage a. d. Sch.}{}\ledrightnote{\textcolor{green}{Die Frage an das Schicksal}}{ }ſchon 150 Mark
                    getragen – ſo viel bekam jeder der Mitwirkenden bei \label{K_L00236_2v}\edtext{\textcolor{blue}{\textsc{Grelling}}{}\ledrightnote{\textcolor{blue}{Richard Grelling}}}{\lemma{\textnormal{\emph{Grelling}}}\Cendnote{\textnormal{Privataufführung bei \textcolor{blue}{Richard Grelling} kurz vor dem
                            14. 1. 1891.}}}\label{K_L00236_2h}. –\pend
           \pstart
           Gearbeitet hab ich beinah nichts; alles ungewiſſe, ſo nichtig es ſein mag,
                    beſchäftigt nach außen hin u macht daher nervös, – Hoffentlich haben {\pb}Sie Ihre glückliche Verſeſti{\geminationm}ung wiedergefunden. – Schade, daſs Sie Freitag
                    nicht da ſind.\pend
           \pstart
           Herzlichen Gruß{\\[\baselineskip]}Ihr\spacefill\mbox{Arth.}\pend
           \leftskip=0em{}\pstart
           \textcolor{pink}{\textsc{Ischl}}{}\ledrightnote{\textcolor{pink}{Bad Ischl}}, 12. 7. 93.\pend
           \endnumbering\briefempfaengerindex{Hofmannsthal, Hugo von@\textsc{Hofmannsthal, Hugo von}!zzzSchnitzler, Arthur@\emph{von Arthur Schnitzler}!1893-07-121@{12. 7. 1893}|)be}\mylabel{h}  \normalsize

\doendnotes{C}
\bigskip
\vfill

\clearpage

\footnotesize

\lohead{\textsc{register}}

% Definiere theindex-Environment komplett neu ohne reledmac
\makeatletter
\renewenvironment{theindex}{%
  \section*{\indexname}%
  \setlength{\parindent}{0pt}%
  \setlength{\parskip}{0pt plus 0.3pt}%
  \let\item\@idxitem
}{%
  \clearpage
}
\makeatother

\IfFileExists{\jobname-pw.ind}{\input{\jobname-pw.ind}}{}

\end{document}

      