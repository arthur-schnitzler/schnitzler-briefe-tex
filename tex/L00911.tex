%% latex-korrekturansicht-vorspann.tex
%% Vorspann für die Korrekturansicht.
%% Lädt die gemeinsame Datei latex-vorspann.tex mit gesetztem Schalter.

\newif\ifkorrekturansicht
\korrekturansichttrue

\input{../tex-inputs/latex-vorspann}


               \section[Hugo von Hofmannsthal an Arthur Schnitzler, 26. 3. 1899]{ Hugo von Hofmannsthal an Arthur Schnitzler, 26. 3. 1899}\nopagebreak\mylabel{v}\rehead{ }\normalsize\beginnumbering\briefempfaengerindex{Schnitzler, Arthur@\textsc{Schnitzler, Arthur}!zzzHofmannsthal, Hugo von@\emph{von Hugo von Hofmannsthal}!1899-03-261@{26. 3. 1899}|(be} \toendnotes[C]{\smallbreak\pagebreak[2]} \Standort{CUL, Schnitzler, B 43.}
\physDesc{Brief, 1 Blatt, 2 Seiten
\newline{}Handschrift: schwarze Tinte, deutsche Kurrent\newline{}Beilage: maschinelles Telegramm nach \textcolor{pink}{Berlin} 
\newline{}Schnitzler: mit Bleistift datiert: »2\strikeout{9}6/3 99« \newline{}Ordnung: 1) mit Bleistift von unbekannter Hand nummeriert: »\strikeout{143}« 2) mit Bleistift von unbekannter Hand nummeriert: »140«}\buchAbdrucke{\weitereDrucke{Hugo von Hofmannsthal, Arthur Schnitzler: \emph{Briefwechsel}. Hg. Therese Nickl und Heinrich Schnitzler. Frankfurt am Main: \emph{S. Fischer} 1964, S. 121.} }\pstart
           \raggedleft{}{\pb}\textcolor{pink}{Berlin}{}\ledrightnote{\textcolor{pink}{Berlin}}{ }Sonntg\pend
           \pstart
           lieber, eben bekomm ich dieſes Telegra{\geminationm} von dem armen \textcolor{blue}{Poldy}{}\ledrightnote{\textcolor{blue}{Leopold von Andrian-Werburg}}. Er bildet ſich dieſmal ein, daſs er wahnſinnig wird. Vielleicht
                    können Sie irgendwas machen.\pend
           \pstart
           Ich ko{\geminationm}e, da Sie nicht herko{\geminationm}en, ſchon ſpäteſtens Samstag nach
                        \textcolor{pink}{Wien}{}\ledrightnote{\textcolor{pink}{Wien}}.\pend
           \pstart
           Ich ſehe viele Menſchen: \textcolor{blue}{Hauptmann}{}\ledrightnote{\textcolor{blue}{Gerhart Hauptmann}}, \textcolor{blue}{Ludwig von Hofmann}{}\ledrightnote{\textcolor{blue}{Ludwig von Hofmann}}, \textcolor{blue}{\textsc{Kessler}}{}\ledrightnote{\textcolor{blue}{Harry von Kessler}}, \textcolor{blue}{Bodenhauſen}{}\ledrightnote{\textcolor{blue}{Eberhard von Bodenhausen}}, \textcolor{blue}{Kainz}{}\ledrightnote{\textcolor{blue}{Josef Kainz}}, die \textcolor{blue}{Dumont}{}\ledrightnote{\textcolor{blue}{Louise Dumont}}{ }\textsc{etc. etc.} auch viele gute Vorſtellungen, wie \textcolor{green}{Fuhrmann Henſchel}{}\ledrightnote{\textcolor{green}{Fuhrmann Henschel}}. {\pb}Bin aber nicht im Stand
                    einen Brief zu ſchreiben.\pend
           \pstart
           Von Herzen Ihr{\\[\baselineskip]}\spacefill\mbox{Hugo.}\pend
           \leftskip=0em{}{\bigskip}\pstart
           \noindent{}{\pb}v \textcolor{pink}{insbruck}{}\ledrightnote{\textcolor{pink}{Innsbruck}} 3747 31 26/3{ }9 40m\pend
           \pstart
           {[}bef{]}uerchtungen geisteszustand fast eingetroffen bin sofort
                        \textcolor{pink}{insbruck}{}\ledrightnote{\textcolor{pink}{Innsbruck}} gefahren
                    {[}prof{]}essor \textcolor{blue}{meyer}{}\ledrightnote{\textcolor{blue}{Karl Mayer}}
                    consultiren dieser verreist. bitte wenn kannst sofort herkommen wo ist \textcolor{blue}{schnitzler}{}\ledrightnote{}? = \textcolor{blue}{poldi}{}\ledrightnote{\textcolor{blue}{Leopold von Andrian-Werburg}}{ }\textcolor{pink}{goldner adler}{}\ledrightnote{\textcolor{pink}{Hotel Goldener Adler}}.+=\pend
           \endnumbering\briefempfaengerindex{Schnitzler, Arthur@\textsc{Schnitzler, Arthur}!zzzHofmannsthal, Hugo von@\emph{von Hugo von Hofmannsthal}!1899-03-261@{26. 3. 1899}|)be}\mylabel{h}  \normalsize

\doendnotes{C}
\bigskip
\vfill

\clearpage

\footnotesize

\lohead{\textsc{register}}

% Definiere theindex-Environment komplett neu ohne reledmac
\makeatletter
\renewenvironment{theindex}{%
  \section*{\indexname}%
  \setlength{\parindent}{0pt}%
  \setlength{\parskip}{0pt plus 0.3pt}%
  \let\item\@idxitem
}{%
  \clearpage
}
\makeatother

\IfFileExists{\jobname-pw.ind}{\input{\jobname-pw.ind}}{}

\end{document}

      