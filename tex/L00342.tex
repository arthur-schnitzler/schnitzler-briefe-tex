%% latex-korrekturansicht-vorspann.tex
%% Vorspann für die Korrekturansicht.
%% Lädt die gemeinsame Datei latex-vorspann.tex mit gesetztem Schalter.

\newif\ifkorrekturansicht
\korrekturansichttrue

\input{../tex-inputs/latex-vorspann}


               \section[Richard Beer-Hofmann an Arthur Schnitzler, 30. 6. 1894]{ Richard Beer-Hofmann an Arthur Schnitzler,
                    30. 6. 1894}\nopagebreak\mylabel{v}\rehead{ }\normalsize\beginnumbering\briefempfaengerindex{Schnitzler, Arthur@\textsc{Schnitzler, Arthur}!zzzBeer-Hofmann, Richard@\emph{von Richard Beer-Hofmann}!1894-06-301@{30. 6. 1894}|(be} \toendnotes[C]{\smallbreak\pagebreak[2]} \Standort{CUL, Schnitzler, B 8.}
\physDesc{Brief, 2 Blätter, 7 Seiten
\newline{}Handschrift: blauer Buntstift, lateinische Kurrent
\newline{}Schnitzler: mit Bleistift datiert: »30/6 94« und nummeriert: »33« }\buchAbdrucke{\weitereDrucke{Arthur Schnitzler, Richard Beer-Hofmann: \emph{Briefwechsel 1891–1931}. Hg. Konstanze Fliedl. Wien, Zürich: \emph{Europaverlag} 1992, S. 55–56.} }\toendnotes[C]{\smallbreak}\pstart{}{\pb}Lieber
                        Arthur!\pend\pstart
           An \textcolor{blue}{F.}{}\ledrightnote{\textcolor{blue}{Friedrich Michael Fels}} hatte ich natürlich vergessen, ordnete
                    aber die Sache sofort nach Erhalt Ihres Briefes. –\pend
           \pstart
           Unter welcher Adresse \label{K_L00342_1v}\edtext{gratulirt}{\lemma{\textnormal{\emph{gratulirt}}}\Cendnote{\textnormal{Dessen Hochzeit
                        stand unmittelbar bevor: Am 8. 7. 1894 heirateten \textcolor{blue}{Julius Schnitzler} und \textcolor{blue}{Helene Altmann}.}}}\label{K_L00342_1h} man Ihrem \textcolor{blue}{Bruder}{}\ledrightnote{→\textcolor{blue}{Julius Schnitzler}}?\pend
           \pstart
           Bitte Sie um Folgendes: Ich brauche ein \label{K_L00342_2v}\edtext{Cachenez}{\lemma{\textnormal{\emph{Cachenez}}}\Cendnote{\textnormal{ein Schal}}}\label{K_L00342_2h} welches so groß ist, daß {\pb}man es falten und als
                    Schärpe binden kann. Es soll ganz \uline{schwarz}
                sein
                    und zwar \uline{schwerer}{ }\uline{weicher}{ }\uline{matter}
                seidenstoff – nicht
                    Atlas – womöglich schwarz in schwarz gemustert, vielleicht brokatartig. Wenn Sie
                    es bei \textcolor{brown}{Stoll + Uhlig}{}\ledrightnote{\textcolor{brown}{Stoll {\kaufmannsund} Uhlig}}{ }{\pb}beko{\geminationm}en, dann lassen Sie es mir direkt zusenden ohne
                    zu bezahlen, beko{\geminationm}en Sie es dort nicht, oder sehen
                    Sie irgendwo etwas Passendes, so lassen Sie es mir zusenden und bezahlen
                    unterdessen. Es kann übrigens auch {\pb}\uline{wenn es das giebt} (?) schwarze glatte \uline{Roh}seide sein.\pend
           \pstart
           \textcolor{blue}{Bahr}{}\ledrightnote{\textcolor{blue}{Hermann Bahr}} war vorgestern zwei Stunden in \textcolor{pink}{Ischl}{}\ledrightnote{\textcolor{pink}{Bad Ischl}}.\pend
           \pstart
           \textcolor{blue}{Kappers}{}\ledrightnote{\textcolor{blue}{Friedrich Kapper}{\newline}\textcolor{blue}{Adele Kapper}}
                sind hier, ich predige \textcolor{blue}{ihm}{}\ledrightnote{→\textcolor{blue}{Friedrich Kapper}} Unmoral und beweise
                    ihm wie bescheiden {\pb}er
                    sein müsste. \textcolor{blue}{Paul Schulz}{}\ledrightnote{\textcolor{blue}{Paul Schulz}}
                sprach ich; was hat
                    der wieder gegen Sie? Oder vielmehr gegen das »\textcolor{green}{Abschiedssouper}{}\ledrightnote{\textcolor{green}{Abschiedssouper}}«? Übrigens liebt er auch den Styl \textcolor{blue}{J. Opp{\dots}}{}\ledrightnote{\textcolor{blue}{Josef Oppenheim}} und mag den \textcolor{blue}{Th. Herzl}{}\ledrightnote{\textcolor{blue}{Theodor Herzl}} nicht.\pend
           \pstart
           {\pb}Ko{\geminationm}en Sie bald nach der Hochzeit Ihres \textcolor{blue}{Bruders}{}\ledrightnote{→\textcolor{blue}{Julius Schnitzler}}? \textcolor{pink}{Leopold}{}\ledrightnote{\textcolor{pink}{Hotel und Pension Rudolfshöhe (Leopold Petter)}}?\pend
           \pstart
           Grüßen Sie \textcolor{blue}{Hugo}{}\ledrightnote{\textcolor{blue}{Hugo von Hofmannsthal}}, zeigen Sie ihm aber nicht
                    den Brief, er macht mir sonst Vorwürfe daß zuviel »Tatsächliches« {\pb}drinnen steht. \textcolor{blue}{Salten}{}\ledrightnote{\textcolor{blue}{Felix Salten}} auch.\pend
           \pstart
           Herzlichst{\\[\baselineskip]}Ihr \spacefill\mbox{Richard}\pend
           \leftskip=0em{}\pstart
           \textcolor{pink}{Ischl}{}\ledrightnote{\textcolor{pink}{Bad Ischl}}{ }30/VI 94\pend
           \pstart
           Ich freu mich aufs Siegeln\pend
           \endnumbering\briefempfaengerindex{Schnitzler, Arthur@\textsc{Schnitzler, Arthur}!zzzBeer-Hofmann, Richard@\emph{von Richard Beer-Hofmann}!1894-06-301@{30. 6. 1894}|)be}\mylabel{h}  \normalsize

\doendnotes{C}
\bigskip
\vfill

\clearpage

\footnotesize

\lohead{\textsc{register}}

% Definiere theindex-Environment komplett neu ohne reledmac
\makeatletter
\renewenvironment{theindex}{%
  \section*{\indexname}%
  \setlength{\parindent}{0pt}%
  \setlength{\parskip}{0pt plus 0.3pt}%
  \let\item\@idxitem
}{%
  \clearpage
}
\makeatother

\IfFileExists{\jobname-pw.ind}{\input{\jobname-pw.ind}}{}

\end{document}

      