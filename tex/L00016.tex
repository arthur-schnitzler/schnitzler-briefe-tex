%% latex-korrekturansicht-vorspann.tex
%% Vorspann für die Korrekturansicht.
%% Lädt die gemeinsame Datei latex-vorspann.tex mit gesetztem Schalter.

\newif\ifkorrekturansicht
\korrekturansichttrue

\input{../tex-inputs/latex-vorspann}


               \section[Richard Beer-Hofmann an Arthur Schnitzler, 30. 5. 1891]{ Richard Beer-Hofmann an Arthur Schnitzler, 30. 5. 1891}\nopagebreak\mylabel{v}\rehead{ }\normalsize\beginnumbering\briefempfaengerindex{Schnitzler, Arthur@\textsc{Schnitzler, Arthur}!zzzBeer-Hofmann, Richard@\emph{von Richard Beer-Hofmann}!1891-05-301@{30. 5. 1891}|(be} \toendnotes[C]{\smallbreak\pagebreak[2]} \Standort{CUL, Schnitzler, B 8.}
\physDesc{Briefkarte
\newline{}Handschrift: schwarze Tinte, lateinische Kurrent
\newline{}Schnitzler: mit Bleistift datiert: »30/5 91« und nummeriert: »2.« }\buchAbdrucke{\weitereDrucke{Arthur Schnitzler, Richard Beer-Hofmann: \emph{Briefwechsel 1891–1931}. Hg. Konstanze Fliedl. Wien, Zürich: \emph{Europaverlag} 1992, S. 30.} }\toendnotes[C]{\smallbreak}\pstart\center{}{\pb}Lieber Arthur!\pend\pstart
           Denken Sie mein \textcolor{blue}{Cousin}{}\ledrightnote{→\textcolor{blue}{Victor Carl Wolf}{\newline}→\textcolor{blue}{Emil Wolf}} hat auf mein Anrathen \label{K_L00016_1v}\edtext{die alten Jahrgänge}{\lemma{\textnormal{\emph{die alten Jahrgänge}}}\Cendnote{\textnormal{\emph{\textcolor{green}{An der schönen blauen Donau}}, ein
                  »Unterhaltungsblatt für die Familie«, erschien seit dem 15. 1. 1886
                  alle 14 Tage. Die von \textcolor{blue}{Beer-Hofmann}
                  angesprochenen Texte finden sich in den Jahrgängen 1888 bis 1890.}}}\label{K_L00016_1h} der »\textcolor{green}{blauen Donau}{}\ledrightnote{\textcolor{green}{An der schönen blauen Donau}}« gekauft und an Sonntag Nachmittagen,
               wenn ich frei bin lese ich Einzelnes daraus vor; Philisterpublikum zum größten Theil
               aber Publikum. \textcolor{blue}{Loris}{}\ledrightnote{\textcolor{blue}{Hugo von Hofmannsthal}}{ }Gedichte, von \textcolor{blue}{Paul}{}\ledrightnote{\textcolor{blue}{Paul Goldmann}} die \textcolor{green}{Bleisoldaten}{}\ledrightnote{\textcolor{green}{Bleisoldaten. Novellette}} und noch einige
               Kleinigkeiten, von Ihnen Gedichte, »\textcolor{green}{Episode}{}\ledrightnote{\textcolor{green}{Episode}}« und
                  »\textcolor{green}{Alkandi}{}\ledrightnote{\textcolor{green}{Alkandi’s Lied}}«. Die »\textcolor{green}{Lieder eines Nervösen}{}\ledrightnote{\textcolor{green}{Lieder eines Nervösen}}« kannte ich nicht{[}.{]} sie haben mir
               nie was von ihnen gesagt, und sie stehen auch nicht auf der Höhe der anderen. \textcolor{green}{Episode}{}\ledrightnote{\textcolor{green}{Episode}} ist merkwürdigerweise begriffen worden und
               hat gefallen {\pb}was ich zwei \textcolor{blue}{Cousins}{}\ledrightnote{→\textcolor{blue}{Victor Carl Wolf}{\newline}→\textcolor{blue}{Emil Wolf}} die Publicum
               waren nicht zugetraut hätte. \textcolor{green}{Alkandi}{}\ledrightnote{\textcolor{green}{Alkandi’s Lied}} las ich spät
               Abends, und als meine \textcolor{blue}{Tante}{}\ledrightnote{→\textcolor{blue}{Charlotte Wolf}}
               mich erinnerte daß es spät sei war mein \textcolor{blue}{Cousin}{}\ledrightnote{→\textcolor{blue}{Victor Carl Wolf}{\newline}→\textcolor{blue}{Emil Wolf}} derart wüthend über die Störung daß er
               einen halben Jahrgang »\textcolor{green}{blaue Donau}{}\ledrightnote{\textcolor{green}{An der schönen blauen Donau}}« zu Boden warf!
                  »\uline{Die Macht der Poesie}«. Wenn Sie glauben ich hätte
               viel Zeit zum Schreiben irren Sie; heute habe ich Kaserninspection und muß hier in
               der Kaserne sitzen, und übernachten, sonst käme ich nicht zum Schreiben. Wenn sie
               Lust haben schreiben Sie Ihrem \spacefill\mbox{Richard}\pend
           \pstart
           30 Mai 91\pend
           \pstart
           \label{T_L00016_1v}\edtext{Daß Sie mir als Adresse}{\lemma{\textnormal{\emph{Daß Sie mir als Adresse}}}\Cendnote{\textnormal{weiter quer am linken Rand}}}\label{T_L00016_1h}{ }\label{K_L00016_2v}\edtext{\textcolor{pink}{Giselastrasse}{}\ledrightnote{\textcolor{pink}{Bösendorferstraße}} und nicht \textcolor{pink}{Ring}{}\ledrightnote{\textcolor{pink}{Burgring}}}{\lemma{\textnormal{\emph{Giselastrasse … Ring}}}\Cendnote{\textnormal{Das Haus hatte zwei Eingänge, wobei
                     die letztere Adresse die repräsentativere darstellt.}}}\label{K_L00016_2h} angaben ist sehr
                  hübsch von Ihnen; ich danke. Mein Brief \label{T_L00016_2v}\edtext{und »Sie« werden sich auf der Stiege treffen.}{\lemma{\textnormal{\emph{und … treffen.}}}\Cendnote{\textnormal{am oberen Rand auf dem Kopf}}}\label{T_L00016_2h}\pend
           \endnumbering\briefempfaengerindex{Schnitzler, Arthur@\textsc{Schnitzler, Arthur}!zzzBeer-Hofmann, Richard@\emph{von Richard Beer-Hofmann}!1891-05-301@{30. 5. 1891}|)be}\mylabel{h}  \normalsize

\doendnotes{C}
\bigskip
\vfill

\clearpage

\footnotesize

\lohead{\textsc{register}}

% Definiere theindex-Environment komplett neu ohne reledmac
\makeatletter
\renewenvironment{theindex}{%
  \section*{\indexname}%
  \setlength{\parindent}{0pt}%
  \setlength{\parskip}{0pt plus 0.3pt}%
  \let\item\@idxitem
}{%
  \clearpage
}
\makeatother

\IfFileExists{\jobname-pw.ind}{\input{\jobname-pw.ind}}{}

\end{document}

      