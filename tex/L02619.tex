%% latex-korrekturansicht-vorspann.tex
%% Vorspann für die Korrekturansicht.
%% Lädt die gemeinsame Datei latex-vorspann.tex mit gesetztem Schalter.

\newif\ifkorrekturansicht
\korrekturansichttrue

\input{../tex-inputs/latex-vorspann}


               \section[Paul Goldmann an Arthur Schnitzler, 1. 5. {[}1894{]}]{ Paul Goldmann an Arthur Schnitzler, 1. 5. {[}1894{]}}\nopagebreak\mylabel{v}\rehead{ }\normalsize\beginnumbering\briefempfaengerindex{Schnitzler, Arthur@\textsc{Schnitzler, Arthur}!zzzGoldmann, Paul@\emph{von Paul Goldmann}!1894-05-011@{1. 5. {[}1894{]}}|(be} \toendnotes[C]{\smallbreak\pagebreak[2]} \Standort{DLA, A:Schnitzler, HS.NZ85.1.3164.}
\physDesc{Brief, 2 Blätter, 8 Seiten
\newline{}Handschrift: schwarze Tinte, deutsche Kurrent
\newline{}Schnitzler: 1) mit Bleistift auf dem ersten Blatt die Jahreszahl »94« vermerkt 2) mit rotem Buntstift fünf Unterstreichungen}\toendnotes[C]{\smallbreak}\pstart
           \noindent{}{\pb}\textcolor{gray}{\textbf{\textcolor{brown}{Frankfurter Zeitung}{}\ledrightnote{\textcolor{brown}{Frankfurter Zeitung}}.}}\hfill \textsc{\textcolor{pink}{Paris}{}\ledrightnote{\textcolor{pink}{Paris}}}, 1. Mai.\pend
           \pstart
           \textcolor{gray}{\textbf{(\textcolor{brown}{Gazette de
                     Francfort}{}\ledrightnote{\textcolor{brown}{Frankfurter Zeitung}}.)}}\pend
           \pstart
           \textcolor{gray}{\textbf{\begin{otherlanguage}{french}Fondateur\end{otherlanguage}{ }\textbf{M. \textcolor{blue}{L. Sonnemann}{}\ledrightnote{\textcolor{blue}{Leopold Sonnemann}}}.}}\pend
           \pstart
           \textcolor{gray}{\textbf{\begin{otherlanguage}{french}Journal politique, financier,\end{otherlanguage}}}\pend
           \pstart
           \textcolor{gray}{\textbf{\begin{otherlanguage}{french}commercial et littéraire.\end{otherlanguage}}}\pend
           \pstart
           \textcolor{gray}{\textbf{\begin{otherlanguage}{french}\textbf{Paraissant trois fois par jour}\end{otherlanguage}}}.\pend
           \pstart
           \textcolor{gray}{\textbf{–}}\pend
           \pstart
           \textcolor{gray}{\textbf{\begin{otherlanguage}{french}\textbf{Bureaux à \textcolor{pink}{Paris}{}\ledrightnote{\textcolor{pink}{Paris}}:}\end{otherlanguage}}}\pend
           \pstart
           \textcolor{gray}{\textbf{\begin{otherlanguage}{french}\textcolor{pink}{\textbf{24. Rue Feydeau}}{}\ledrightnote{\textcolor{pink}{rue Feydeau}}.\end{otherlanguage}}}\pend
           \pstart\center{}Mein lieber Arthur,\pend\pstart
           Anbei erhälſt Du den \label{K_L02619-1v}\edtext{»\textsc{\textcolor{green}{Mercure de France}{}\ledrightnote{\textcolor{green}{Mercure de France}}}«, wo \textsc{\textcolor{blue}{Henri Albert}{}\ledrightnote{\textcolor{blue}{Henri Albert}}}{ }\strikeout{Dich} gelegentlich wieder von Deinem Talente ſpricht
                  (S.{ }92)}{\lemma{\textnormal{\emph{»Mercure … 92)}}}\Cendnote{\textnormal{\textcolor{blue}{Henri Albert}: \emph{\textcolor{green}{Journaux et Revues}}. In: \emph{\textcolor{green}{Mercure de France}}, Jg. 11, Nr. 53, Mai 1894, S. 87–92, hier:
                     S. 92.}}}\label{K_L02619-1h}. Was zahlſt Du uns eigentlich für die Reklame?\pend
           \pstart
           Ich danke Dir herzlichſt für die Überſendung der beiden \textcolor{green}{Skizzen}{}\ledrightnote{→\textcolor{green}{Die überspannte Person}{\newline}→\textcolor{green}{Halb Zwei}}, komme erſt
               Ende der Woche dazu, ſie in Ruhe zu leſen, und \label{K_L02619-7v}\edtext{ſchreibe Dir dann ſofort darüber.}{\lemma{\textnormal{\emph{ſchreibe … darüber.}}}\Cendnote{\textnormal{siehe Paul Goldmann an Arthur Schnitzler, 29. 5. [1894]}}}\label{K_L02619-7h}{ }\textsc{\textcolor{blue}{Albert}{}\ledrightnote{\textcolor{blue}{Henri Albert}}} ſehe ich morgen und werde Dir dann berichten,
               wie es mit Deiner \textcolor{green}{Überſetzung}{}\ledrightnote{→\textcolor{green}{Les Emplettes de Noël}}
               ſteht. Schicke ihm das Honorar, wenn Du kannſt, gleich, an ſeine {\pb}\textcolor{pink}{Adreſſe}{}\ledrightnote{→\textcolor{pink}{rue Jacob}}, ohne weitere
               Bemerkung. \strikeout{\textcolor{gray}{E}} Ich beſorge ſchon den nöthigen Commentar. Ich denke 10 bis 12 Gulden, wenn Dir
               das nicht zu viel iſt. Kannſt Du jetzt nicht, ſo warte ruhig, bis Du von ihm etwas
               Poſitives über den Ausgang der Arbeit erfährſt. Ich veranlaſſe ihn jedenfalls,
               demnächſt an Dich zu ſchreiben{\dotsfour}\pend
           \pstart
           Bitte, dementire auf das Energiſcheſte das Gerücht von meiner Candidatur auf \label{K_L02619-4v}\edtext{\textsc{\textcolor{blue}{Herzl}{}\ledrightnote{\textcolor{blue}{Theodor Herzl}}s} Nachfolge}{\lemma{\textnormal{\emph{Herzls Nachfolge}}}\Cendnote{\textnormal{als Korrespondent der \emph{\textcolor{brown}{Neuen Freien Presse}} in \textcolor{pink}{Paris}. \textcolor{blue}{Herzl} hatte die Stellung
                  von Oktober 1891 bis Juli 1895 inne.}}}\label{K_L02619-4h}. Es iſt
               nicht ein wahres Wort daran, und wenn es meiner \textcolor{brown}{Redaction}{}\ledrightnote{→\textcolor{brown}{Frankfurter Zeitung}} zu Ohren kommt, kann es nur meine jetzige Stellung
               gefährden. Daß \textsc{\textcolor{blue}{Herzl}{}\ledrightnote{\textcolor{blue}{Theodor Herzl}}} weggeht {\pb}iſt möglich. Aber \label{K_L02619-2v}\edtext{niemals wird man mich zur »\textcolor{brown}{Neuen Fr. Preſſe}{}\ledrightnote{\textcolor{brown}{Neue Freie Presse}}« nehmen}{\lemma{\textnormal{\emph{niemals … nehmen}}}\Cendnote{\textnormal{Zwischen 1890 und 1892 hatte \textcolor{blue}{Goldmann} bereits für die \emph{\textcolor{brown}{Neue Freie Presse}} geschrieben. Ab 1902 wurde
                  er als ihr Theaterkorrespondent tätig.}}}\label{K_L02619-2h}. Zwiſchen dem \textcolor{brown}{Blatte}{}\ledrightnote{→\textcolor{brown}{Neue Freie Presse}} und meinem \textcolor{blue}{Onkel}{}\ledrightnote{→\textcolor{blue}{Fedor Mamroth}} beſteht, wie Du wohl weißt, eine
                  \label{K_L02619-5v}\edtext{tödtliche Feindſchaft}{\lemma{\textnormal{\emph{tödtliche Feindſchaft}}}\Cendnote{\textnormal{\textcolor{blue}{Mamroth} hatte seine Laufbahn
                     1873 bei der \emph{\textcolor{brown}{Neuen Freien
                     Presse}} begonnen, wechselte dann in Folge aber zu anderen \textcolor{pink}{Wien}er Zeitschriften und Zeitungen, bevor er mit
                     1. 4. 1889 das Feuilleton der \emph{\textcolor{brown}{Frankfurter Zeitung}} betreute.}}}\label{K_L02619-5h}. Und dieſe Leute mit ihren
               Börſenjobber-Seelen haſſen bis ins ſiebente Glied. Als \textsc{\textcolor{blue}{Benedict}{}\ledrightnote{\textcolor{blue}{Moriz Benedikt}}} vor einigen Monaten hier war, hat er es abgelehnt, daß ich ihm vorgeſtellt
               werde! Dazu kommt, daß \textsc{\textcolor{blue}{Herzl}{}\ledrightnote{\textcolor{blue}{Theodor Herzl}}} ſelbſt keinen Finger rühren wird, um meine Candidatur zu ſtützen, eher das
               Gegentheil. Ich habe ihn hier genau kennen gelernt. Er iſt {\pb}eine ſeltſame Miſchung von Künſtler und jüdiſchem
               Journaliſten. Auf der einen, der Künſtler-Seite, charmant, glänzend, ſympathiſch; auf
               der andern Seite: kleinlich, eiferſüchtig, \strikeout{b\textcolor{gray}{er}} geheimnißthueriſch, berechnend und größenwahnſinnig. Ich will ja nicht ſagen,
               daß er gegen meine Candidatur intriguiren würde – obwohl es mich nicht erſtaunen
               würde, wenn ers thäte – aber er wird ſicher nicht das Mindeſte thun, um mich, vor
               deſſen Nebenbuhlerſchaft er ſich fürchtet – der Dummkopf! – an ſeine Stelle zu
               bringen. Das Alles hindert aber {\pb}nicht, daß er jetzt
               einen \textcolor{green}{Einakter}{}\ledrightnote{\textcolor{green}{Die Glosse. Lustspiel in einem Act}} in Verſen
               geſchrieben, der ein Stück köſtlicher und großer Kunſt iſt. Zu Niemandem ein Wort von
               alledem, nicht wahr? Noch eins: \textsc{\textcolor{blue}{Dr. Schwitzer}{}\ledrightnote{\textcolor{blue}{Ludwig Schwitzer}}}, früheres Mitglied der volkswirthſchaftlichen Redaction der \textcolor{brown}{N. Fr. Pr.}{}\ledrightnote{\textcolor{brown}{Neue Freie Presse}}, iſt plötzlich \textcolor{pink}{hier}{}\ledrightnote{→\textcolor{pink}{Paris}} aufgetaucht und ich glaube, \label{K_L02619-3v}\edtext{\begin{otherlanguage}{french}\textsc{c’est pour recueillir la succession}\end{otherlanguage}}{\lemma{\textnormal{\emph{c’est … succession}}}\Cendnote{\textnormal{französisch: um die Nachfolge zu
                  besorgen}}}\label{K_L02619-3h}.\pend
           \pstart
           \textsc{\textcolor{blue}{Rudolf Lothar}{}\ledrightnote{\textcolor{blue}{Rudolf Lothar}}} iſt auf einer ſeiner literariſchen Handlungsreiſen auch \textcolor{pink}{hier}{}\ledrightnote{→\textcolor{pink}{Paris}} eingetroffen. Er will alle {\pb}möglichen Leute interviewen, \textsc{\textcolor{blue}{Pailleron}{}\ledrightnote{\textcolor{blue}{Édouard Pailleron}}} und \textsc{\textcolor{blue}{Verlaine}{}\ledrightnote{\textcolor{blue}{Paul Verlaine}}}, Kraut und Rüben durcheinander. Er hat ſich an \textsc{\textcolor{blue}{Henri Albert}{}\ledrightnote{\textcolor{blue}{Henri Albert}}} herangedrängt, um \label{K_L02619-9v}\edtext{im »\textsc{\textcolor{green}{Mercure}{}\ledrightnote{\textcolor{green}{Mercure de France}}}« genannt}{\lemma{\textnormal{\emph{im »Mercure« genannt}}}\Cendnote{\textnormal{nicht ermittelt}}}\label{K_L02619-9h} zu
               werden \textsc{etc}. Ich habe einen grämlichen Haß gegen dieſen \textcolor{blue}{Burſchen}{}\ledrightnote{→\textcolor{blue}{Rudolf Lothar}}, der im führenden \textcolor{green}{Blatte}{}\ledrightnote{→\textcolor{green}{Neue Freie Presse}} Literaturmeinung macht
               und deſſen Stücke als die Blüthe des jungen Geiſtes \strikeout{\textcolor{gray}{×}\-\textcolor{gray}{×}} auf allen Jahrmärkten angeprieſen werden, während Du vorläufig nur von einer
               Elite gekannt und gewürdigt biſt. Ich finde, er hat Dir direct ſeine Celebrität
               geſtohlen. Und als ich dieſen geſchäftigten {\pb}\textcolor{blue}{Barbiergeſellen}{}\ledrightnote{→\textcolor{blue}{Rudolf Lothar}} neulich im
               Theater traf, drehte ich ihm einfach den Rücken. Das war wohl exceſſiv, aber ich kann
               nichts gegen mein Temperament.\pend
           \pstart
           Ein grünes einſames windſtilles Land! Wie, wenn Du auch nach \textsc{\textcolor{pink}{Hamburg}{}\ledrightnote{\textcolor{pink}{Hamburg}}} kämeſt, wo ich wahrscheinlich meinen Uraub werde verbringen müſſen. Und wann,
               wann endlich werde ich Dich \label{K_L02619-6v}\edtext{in \textsc{\textcolor{pink}{Paris}{}\ledrightnote{\textcolor{pink}{Paris}}} ſehen}{\lemma{\textnormal{\emph{in Paris ſehen}}}\Cendnote{\textnormal{Erst 1897 reiste
                  Schnitzler nach \textcolor{pink}{Paris}.}}}\label{K_L02619-6h}? Komm doch
               wenigſtens auf 14 Tage! Wenn Du nicht ſo ein verwöhnter Prinz wäreſt, könnteſt Du
               ſogar bei mir wohnen, aber ohne jeden Comfort!\pend
           \pstart
           {\pb}Tauſend Dank auch für alles Liebe, das Du mir ſonſt
               ſagſt. Es iſt immer Feſttag bei mir, wenn ein Brief von Dir ankommt. Wie kann ich Dir
               das Alles lohnen\substVorne{}\textsuperscript{?}\substDazwischen{}!\substHinten{}\pend
           \pstart
           Möchte gern etwas Näheres über die große \label{K_L02619-444v}\edtext{\textcolor{green}{Erzählung}{}\ledrightnote{→\textcolor{green}{Sterben. Novelle}}}{\lemma{\textnormal{\emph{Erzählung}}}\Cendnote{\textnormal{Die Novelle \emph{\textcolor{green}{Sterben}} war im Frühjahr 1894 vom \emph{\textcolor{brown}{S. Fischer-Verlag}} akzeptiert worden. Der
                  Erstdruck erschien zwischen Oktober und Dezember in drei
                  Teilen in der \emph{\textcolor{green}{Neuen Deutschen
                  Rundschau}}.}}}\label{K_L02619-444h} wiſſen.\pend
           \pstart
           Weißt Du, daß deine Schrift immer ſchlechter wird? Ich kann ſie zur Noth noch
               entziffern, weil ich die hiſtorische Entwickelung mitgemacht habe. Aber die Andern?
               Dein zukünftiger Biograph? Der Sammler deiner nachgelaſſenen Schriften?{\dotsfour}\pend
           \pstart
           Grüß’ Dich Gott, mein theurer Freund, und ſchreib’ mir bald. Auch von den Andern, \textsc{\textcolor{blue}{Loris}{}\ledrightnote{\textcolor{blue}{Hugo von Hofmannsthal}}} u. \textsc{\textcolor{blue}{Richard}{}\ledrightnote{\textcolor{blue}{Richard Beer-Hofmann}}}.\pend
           \pstart
           Dein treuer {\\[\baselineskip]}\spacefill\mbox{Paul Goldmann}\pend
           \leftskip=0em{}\endnumbering\briefempfaengerindex{Schnitzler, Arthur@\textsc{Schnitzler, Arthur}!zzzGoldmann, Paul@\emph{von Paul Goldmann}!1894-05-011@{1. 5. {[}1894{]}}|)be}\mylabel{h}  \normalsize

\doendnotes{C}
\bigskip
\vfill

\clearpage

\footnotesize

\lohead{\textsc{register}}

% Definiere theindex-Environment komplett neu ohne reledmac
\makeatletter
\renewenvironment{theindex}{%
  \section*{\indexname}%
  \setlength{\parindent}{0pt}%
  \setlength{\parskip}{0pt plus 0.3pt}%
  \let\item\@idxitem
}{%
  \clearpage
}
\makeatother

\IfFileExists{\jobname-pw.ind}{\input{\jobname-pw.ind}}{}

\end{document}

      