%% latex-korrekturansicht-vorspann.tex
%% Vorspann für die Korrekturansicht.
%% Lädt die gemeinsame Datei latex-vorspann.tex mit gesetztem Schalter.

\newif\ifkorrekturansicht
\korrekturansichttrue

\input{../tex-inputs/latex-vorspann}


               \section[Arthur Schnitzler an Hermann Bahr, 18. 5. 1907]{ Arthur Schnitzler an Hermann Bahr, 18. 5. 1907}\nopagebreak\mylabel{v}\rehead{ }\normalsize\beginnumbering\briefempfaengerindex{Bahr, Hermann@\textsc{Bahr, Hermann}!zzzSchnitzler, Arthur@\emph{von Arthur Schnitzler}!1907-05-181@{18. 5. 1907}|(be} \toendnotes[C]{\smallbreak\pagebreak[2]} \Standort{TMW, HS AM 60173 Ba.}
\physDesc{Postkarte
\newline{}Handschrift: schwarze Tinte, deutsche Kurrent\newline{}Versand: Stempel: »\nobreak{}\oindex{XVIII., Waehring@\textbf{XVIII., Währing}, \emph{Bezirk (A.BZK)}|pwk}18/1 Wien, 18. V. 07, XII\nobreak{}«.  \newline{}Ordnung: Lochung }\buchAbdrucke{\weitereDrucke{1) \emph{18. 5. 1907, Abschrift.} In: Arthur Schnitzler: \emph{The Letters of Arthur Schnitzler to Hermann Bahr}. Edited, annotated, and with an introduction, by Donald G.
                        Daviau. Chapel Hill: \emph{The University of North Carolina Press} 1978, S. 98 (University of North Carolina studies in the Germanic languages
                        and literatures, 89).} \weitereDrucke{2) Hermann Bahr, Arthur Schnitzler: \emph{Briefwechsel, Aufzeichnungen, Dokumente (1891–1931)}. Hg. Kurt Ifkovits und Martin Anton Müller. Göttingen: \emph{Wallstein} 2018, S. 393.} }\toendnotes[C]{\smallbreak}\pstart{}{\pb}\textcolor{gray}{\textbf{Dr. Arthur Schnitzler}}\pend{}\pstart{}\textcolor{gray}{\textbf{\textcolor{pink}{Wien, XVIII. Spoettelgasse 7}{}\ledrightnote{\textcolor{pink}{Edmund-Weiß-Gasse}}.}}\pend{}{\bigskip}\pstart{}{\pb}\textsc{Herrn Hermann Bahr,}\pend{}\pstart{}\textcolor{pink}{Wien Ober St. Veit}{}\ledrightnote{\textcolor{pink}{Ober Sankt Veit}}\pend{}\pstart{}\textcolor{pink}{Veitliſſengaſſe.}{}\ledrightnote{\textcolor{pink}{Veitlissengasse}}\pend{}{\bigskip}\pstart
           \raggedleft{}{\pb}18. 5. 907\pend
           \pstart
           lieber Hermann; \textcolor{green}{Band 1}{}\ledrightnote{→\textcolor{green}{Brehms Tierleben}} mit Dank erhalten. (Du
               haſt doch hoffentlich \textcolor{green}{Band 2}{}\ledrightnote{→\textcolor{green}{Brehms Tierleben}}, den
               ich dir noch vor deiner Abreiſe \textsc{per} Poſt nach \textsc{\textcolor{pink}{Ob St Veit}{}\ledrightnote{\textcolor{pink}{Ober Sankt Veit}}}{ }ſenden lieſs\substVorne{}\textsuperscript{)}\substDazwischen{},\substHinten{} richtig erhalten?)\pend
           \pstart
           Ka{\geminationn} ich nächſtens einmal vormittag zu dir hinaus kommen?\pend
           \pstart herzlichſt dein \spacefill\mbox{Arthur.}\pend{}\endnumbering\briefempfaengerindex{Bahr, Hermann@\textsc{Bahr, Hermann}!zzzSchnitzler, Arthur@\emph{von Arthur Schnitzler}!1907-05-181@{18. 5. 1907}|)be}\mylabel{h}  \normalsize

\doendnotes{C}
\bigskip
\vfill

\clearpage

\footnotesize

\lohead{\textsc{register}}

% Definiere theindex-Environment komplett neu ohne reledmac
\makeatletter
\renewenvironment{theindex}{%
  \section*{\indexname}%
  \setlength{\parindent}{0pt}%
  \setlength{\parskip}{0pt plus 0.3pt}%
  \let\item\@idxitem
}{%
  \clearpage
}
\makeatother

\IfFileExists{\jobname-pw.ind}{\input{\jobname-pw.ind}}{}

\end{document}

      