%% latex-korrekturansicht-vorspann.tex
%% Vorspann für die Korrekturansicht.
%% Lädt die gemeinsame Datei latex-vorspann.tex mit gesetztem Schalter.

\newif\ifkorrekturansicht
\korrekturansichttrue

\input{../tex-inputs/latex-vorspann}


               \section[Wilhelm Bölsche an Arthur Schnitzler, 6. 10. 1891]{ Wilhelm Bölsche an Arthur Schnitzler, 6. 10. 1891}\nopagebreak\mylabel{v}\rehead{ }\normalsize\beginnumbering\briefempfaengerindex{Schnitzler, Arthur@\textsc{Schnitzler, Arthur}!zzzBoelsche, Wilhelm@\emph{von Wilhelm Bölsche}!1891-10-061@{6. 10. 1891}|(be} \toendnotes[C]{\smallbreak\pagebreak[2]} \Standort{DLA, A:Schnitzler, HS.NZ85.1.2577,2.}
\physDesc{Brief, 1 Blatt, 1 Seite
\newline{}Handschrift: schwarze Tinte, deutsche Kurrent\newline{}Ordnung: mit rotem Buntstift von unbekannter Hand nummeriert: »2« }\buchAbdrucke{\weitereDrucke{Wilhelm Bölsche: \emph{Briefwechsel. Mit Autoren der Freien Bühne}. Hg. Gerd-Hermann Susen. Berlin: \emph{Weidler} 2010, S. 672 (Werke und Briefe. Wissenschaftliche Ausgabe, Briefe I).} }\toendnotes[C]{\smallbreak}\pstart
           \raggedleft{}{\pb}\textcolor{pink}{Friedrichshagen}{}\ledrightnote{\textcolor{pink}{Friedrichshagen}}{\\}b. \textcolor{pink}{Berlin}{}\ledrightnote{\textcolor{pink}{Berlin}}.{\\}\textcolor{pink}{Wilhelmſtr 72}{}\ledrightnote{\textcolor{pink}{Peter-Hille-Straße}}.{\\}6. X. 91.\pend
           \pstart\center{}Hochgeehrter Herr Doktor!\pend\pstart
           Ich ſehe eben mit Bedauern, daß \label{K_L00042_1v}\edtext{mein \textcolor{blue}{Stellvertreter}{}\ledrightnote{→\textcolor{blue}{Julius Hart}}}{\lemma{\textnormal{\emph{mein Stellvertreter}}}\Cendnote{\textnormal{\textcolor{blue}{Julius Hart} betreute die Redaktion der
                            \emph{\textcolor{green}{Freien Bühne}} vom
                            26. 8. 1891 bis zum 23. 9. 1891.}}}\label{K_L00042_1h}
                    während meiner mehrmonatlichen Abweſenheit Sie nicht benachrichtigt hat, daß
                    Ihre Novelle »\textcolor{green}{Der Sohn}{}\ledrightnote{\textcolor{green}{Der Sohn. Aus den Papieren eines Arztes}}« von mir angenommen
                    worden war. Nur etwas warten muß ſie leider, das \label{K_L00042_2v}\edtext{\textcolor{green}{Drama}{}\ledrightnote{→\textcolor{green}{Das Lumpengesindel}}}{\lemma{\textnormal{\emph{Drama}}}\Cendnote{\textnormal{\textcolor{blue}{Ernst von Wolzogen}: \emph{\textcolor{green}{Das Lumpengesindel. Komödie in 5 Aufzügen}}. In: \emph{\textcolor{green}{Freie Bühne für modernes Leben}}, Jg. 2,
                            H. 40–52, 7. 10. 1891 – 30. 10. 1891 (13
                            Teile).}}}\label{K_L00042_2h}, das wir jetzt abdrucken, ſchiebt alle Novellen
                    zurück.\pend
           \pstart
           Mit vorzüglicher Hochachtung{\\[\baselineskip]}\spacefill\mbox{Wilhelm Bölsche}\pend
           \leftskip=0em{}\endnumbering\briefempfaengerindex{Schnitzler, Arthur@\textsc{Schnitzler, Arthur}!zzzBoelsche, Wilhelm@\emph{von Wilhelm Bölsche}!1891-10-061@{6. 10. 1891}|)be}\mylabel{h}  \normalsize

\doendnotes{C}
\bigskip
\vfill

\clearpage

\footnotesize

\lohead{\textsc{register}}

% Definiere theindex-Environment komplett neu ohne reledmac
\makeatletter
\renewenvironment{theindex}{%
  \section*{\indexname}%
  \setlength{\parindent}{0pt}%
  \setlength{\parskip}{0pt plus 0.3pt}%
  \let\item\@idxitem
}{%
  \clearpage
}
\makeatother

\IfFileExists{\jobname-pw.ind}{\input{\jobname-pw.ind}}{}

\end{document}

      