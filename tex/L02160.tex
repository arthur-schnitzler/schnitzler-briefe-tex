%% latex-korrekturansicht-vorspann.tex
%% Vorspann für die Korrekturansicht.
%% Lädt die gemeinsame Datei latex-vorspann.tex mit gesetztem Schalter.

\newif\ifkorrekturansicht
\korrekturansichttrue

\input{../tex-inputs/latex-vorspann}


               \section[Stefan Großmann an Arthur Schnitzler, 5. 12. 1913]{ Stefan Großmann an Arthur Schnitzler, 5. 12. 1913}\nopagebreak\mylabel{v}\rehead{ }\normalsize\beginnumbering\briefempfaengerindex{Schnitzler, Arthur@\textsc{Schnitzler, Arthur}!zzzGrossmann, Stefan@\emph{von Stefan Großmann}!1913-12-051@{5. 12. 1913}|(be} \toendnotes[C]{\smallbreak\pagebreak[2]} \Standort{CUL, Schnitzler, B 34.}
\physDesc{Brief, 1 Blatt, 1 Seite
\newline{}Handschrift: schwarze Tinte, deutsche Kurrent
\newline{}Schnitzler: mit rotem Buntstift eine Unterstreichung \newline{}Ordnung: mit Bleistift von unbekannter Hand nummeriert: »14« }\pstart
           \noindent{}{\pb}\textcolor{gray}{\textbf{STEFAN GROSSMANN}}\hfill \textcolor{gray}{\textbf{\textcolor{pink}{WIEN,}{}\ledrightnote{\textcolor{pink}{Wien}}}}{ }5. \substVorne{}\textsuperscript{\textsc{Janner}}{\allowbreak}\substDazwischen{}\textsc{December}\substHinten{} 1913\pend
           \pstart
           \raggedleft{}\textcolor{pink}{I. \textsc{Dominikanerbastei} 5}{}\ledrightnote{\textcolor{pink}{Dominikanerbastei}}\pend
           \pstart\center{}Sehr verehrter Herr Schnitzler\pend\pstart
           Der Verlag \textcolor{brown}{Ullstein}{}\ledrightnote{\textcolor{brown}{Ullstein Verlag}} theilt mir mit, daſs er bis
                    zum 2. oder 3. Jänner warten will, wenn er mit
                    einiger Beſtimmtheit auf einen Beitrag von ihnen rechnen kann. Nur würden wir
                    bitten, uns ungefähr das Ausmaß Ihres Beitrages vorausſagen zu wollen, wenn Sie
                    das können.\pend
           \pstart
           Selbſtverſtändlich wäre es dem Redakteur eine große Erleichterung, Ihren Beitrag
                    früher zu erhalten.\pend
           \pstart
           Jedenfalls danken wir Ihnen für Ihre Bereitwilligkeit und hoffen ſehr auf Ihre
                    Gabe.\pend
           \pstart
           Sehr ergeben{\\[\baselineskip]} Ihr \spacefill\mbox{Stefan Großmann}\pend
           \leftskip=0em{}\endnumbering\briefempfaengerindex{Schnitzler, Arthur@\textsc{Schnitzler, Arthur}!zzzGrossmann, Stefan@\emph{von Stefan Großmann}!1913-12-051@{5. 12. 1913}|)be}\mylabel{h}  \normalsize

\doendnotes{C}
\bigskip
\vfill

\clearpage

\footnotesize

\lohead{\textsc{register}}

% Definiere theindex-Environment komplett neu ohne reledmac
\makeatletter
\renewenvironment{theindex}{%
  \section*{\indexname}%
  \setlength{\parindent}{0pt}%
  \setlength{\parskip}{0pt plus 0.3pt}%
  \let\item\@idxitem
}{%
  \clearpage
}
\makeatother

\IfFileExists{\jobname-pw.ind}{\input{\jobname-pw.ind}}{}

\end{document}

      