%% latex-korrekturansicht-vorspann.tex
%% Vorspann für die Korrekturansicht.
%% Lädt die gemeinsame Datei latex-vorspann.tex mit gesetztem Schalter.

\newif\ifkorrekturansicht
\korrekturansichttrue

\input{../tex-inputs/latex-vorspann}


               \section[Hugo von Hofmannsthal an Arthur Schnitzler, {[}1. 1. 1899{]}]{ Hugo von Hofmannsthal an Arthur Schnitzler, {[}1. 1. 1899{]}}\nopagebreak\mylabel{v}\rehead{ }\normalsize\beginnumbering\briefempfaengerindex{Schnitzler, Arthur@\textsc{Schnitzler, Arthur}!zzzHofmannsthal, Hugo von@\emph{von Hugo von Hofmannsthal}!1899-01-011@{{[}1. 1. 1899{]}}|(be} \toendnotes[C]{\smallbreak\pagebreak[2]} \Standort{CUL, Schnitzler, B 43.}
\physDesc{Brief, 1 Blatt, 4 Seiten
\newline{}Handschrift: schwarze Tinte, deutsche Kurrent
\newline{}Schnitzler: mit Bleistift datiert: »Jänner? 99« \newline{}Ordnung: 1) mit Bleistift von unbekannter Hand nummeriert: »\strikeout{138}« 2) mit Bleistift von unbekannter Hand nummeriert:
                                    »130«}\buchAbdrucke{\weitereDrucke{Hugo von Hofmannsthal, Arthur Schnitzler: \emph{Briefwechsel}. Hg. Therese Nickl und Heinrich Schnitzler. Frankfurt am Main: \emph{S. Fischer} 1964, S. 115–116.} }\toendnotes[C]{\smallbreak}\pstart
           \raggedleft{}{\pb}\label{K_L00873_1v}\edtext{\textcolor{pink}{Baden, Julienhof}{}\ledrightnote{\textcolor{pink}{Julienhof}}}{\lemma{\textnormal{\emph{Baden, Julienhof}}}\Cendnote{\textnormal{\textcolor{blue}{Hofmannsthal} hielt sich vom
                        28. 12. 1898 bis 9. 1. 1899 in der Pension \textcolor{pink}{Julienhof} in \textcolor{pink}{Baden} auf.}}}\label{K_L00873_1h}\pend
           \pstart
           lieber Arthur, mir gehts hier gut und ich hab am
                  Silveſterabend in der ſchönſten Stille die neue 2\textsuperscript{te}{ }\textcolor{green}{Verwandlung}{}\ledrightnote{→\textcolor{green}{Die Hochzeit der Sobeide}} vollendet. \label{K_L00873_2v}\edtext{Heut}{\lemma{\textnormal{\emph{Heut}}}\Cendnote{\textnormal{Die genauere Datierung des Briefes gelingt durch den Brief an
                     \textcolor{blue}{Franziska Schlesinger} vom
                     4. 1. 1899, worin er berichtet, am ersten Tag des Jahres kurz in
                     \textcolor{pink}{Wien} gewesen zu sein und dort ihren Brief
                  vorgefunden zu haben.}}}\label{K_L00873_2h} war ich wenige Stunden in der \textcolor{pink}{Stadt}{}\ledrightnote{\textcolor{pink}{Wien}}, habs dem \textcolor{blue}{Richard}{}\ledrightnote{\textcolor{blue}{Richard Beer-Hofmann}}
               vorgeleſen der es nun in Ordnung findet, ſo daſs ich’s nicht mehr zu Ihnen ſondern
               zum {\pb}Typieren getragen habe.\pend
           \pstart
           Habe auch \textcolor{blue}{Schlenther}{}\ledrightnote{\textcolor{blue}{Paul Schlenther}} geſprochen. Haben Sie
               Nachrichten über den »\textcolor{green}{Kakadu}{}\ledrightnote{\textcolor{green}{Der grüne Kakadu. Groteske in einem Akt}}«?\hspace*{2em}Neulich hab ich mir von \textcolor{blue}{2 geſcheiten Leuten}{}\ledrightnote{→\textcolor{blue}{?? [Gesprächspartner von Hofmannsthal 1]}{\newline}→\textcolor{blue}{?? [Gesprächspartner von Hofmannsthal 2]}} unſre ſchöne Juniradpartie durch \textcolor{pink}{Mitteldeutſchland}{}\ledrightnote{\textcolor{pink}{Deutschland}} aufſchreiben laſſen. Wir kommen am
                  \textcolor{pink}{Hörſelberg}{}\ledrightnote{\textcolor{pink}{Hörselberge}} und vielen ſchönen Sachen vorbei,
                  {\pb}fahren über \textcolor{pink}{Ilmenau}{}\ledrightnote{\textcolor{pink}{Ilmenau}} in \textcolor{pink}{Weimar}{}\ledrightnote{\textcolor{pink}{Weimar}} ein, wohnen
               4 Tage im »\textcolor{pink}{Erbprinzen}{}\ledrightnote{\textcolor{pink}{Erbprinz}}« und ſind – hoffentlich –
               brav und luſtig.\pend
           \pstart
           Ich hab heut in \textcolor{pink}{Wien}{}\ledrightnote{\textcolor{pink}{Wien}} mit \label{K_L00873_3v}\edtext{\textcolor{blue}{jemand}{}\ledrightnote{→\textcolor{blue}{Gertrude von Hofmannsthal}}}{\lemma{\textnormal{\emph{jemand}}}\Cendnote{\textnormal{Wenngleich nicht mit Sicherheit zu
                  belegen, liegt es nahe, dass er seinen \textcolor{blue}{Eltern} ein Treffen mit seiner späteren Frau \textcolor{blue}{Gerty} verheimlichte.}}}\label{K_L00873_3h} gegeſſen und dann
               zuhaus geſagt, ich hab bei Ihnen gegeſſen. Da ich ſolche Lügen ſehr ungern hab {\pb}und auch dieſe nur halb in
               Zerſtreutheit geſagt habe, bitte dementieren Sie nicht, falls Sie zufällig meine \textcolor{blue}{Eltern}{}\ledrightnote{→\textcolor{blue}{Hugo August von Hofmannsthal}{\newline}→\textcolor{blue}{Anna von Hofmannsthal}}{ }ſehen.\pend
           \pstart
           Von Herzen Ihr{\\[\baselineskip]}\spacefill\mbox{Hugo.}\pend
           \leftskip=0em{}\endnumbering\briefempfaengerindex{Schnitzler, Arthur@\textsc{Schnitzler, Arthur}!zzzHofmannsthal, Hugo von@\emph{von Hugo von Hofmannsthal}!1899-01-011@{{[}1. 1. 1899{]}}|)be}\mylabel{h}  \normalsize

\doendnotes{C}
\bigskip
\vfill

\clearpage

\footnotesize

\lohead{\textsc{register}}

% Definiere theindex-Environment komplett neu ohne reledmac
\makeatletter
\renewenvironment{theindex}{%
  \section*{\indexname}%
  \setlength{\parindent}{0pt}%
  \setlength{\parskip}{0pt plus 0.3pt}%
  \let\item\@idxitem
}{%
  \clearpage
}
\makeatother

\IfFileExists{\jobname-pw.ind}{\input{\jobname-pw.ind}}{}

\end{document}

      