%% latex-korrekturansicht-vorspann.tex
%% Vorspann für die Korrekturansicht.
%% Lädt die gemeinsame Datei latex-vorspann.tex mit gesetztem Schalter.

\newif\ifkorrekturansicht
\korrekturansichttrue

\input{../tex-inputs/latex-vorspann}


               \section[Oscar Blumenthal an Arthur Schnitzler, 1. 8. 1892]{ Oscar Blumenthal an Arthur Schnitzler, 1. 8. 1892}\nopagebreak\mylabel{v}\rehead{ }\normalsize\beginnumbering\briefempfaengerindex{Schnitzler, Arthur@\textsc{Schnitzler, Arthur}!zzzBlumenthal, Oskar@\emph{von Oskar Blumenthal}!1892-08-011@{1. 8. 1892}|(be} \toendnotes[C]{\smallbreak\pagebreak[2]} \Standort{CUL, Schnitzler, B 15.}
\physDesc{Brief, 1 Blatt, 1 Seite
\newline{}Handschrift  : schwarze Tinte, deutsche Kurrent\newline{}Handschrift Oskar Blumenthal: schwarze Tinte, deutsche Kurrent (\noindent{}Unterschrift)
\newline{}Schnitzler: mit rotem Buntstift eine Unterstreichung und
            nummeriert: »3« }\pstart
           \noindent{}\centering{}{\pb}\textcolor{gray}{\textbf{\textcolor{brown}{LESSING-THEATER}{}\ledrightnote{\textcolor{brown}{Lessing-Theater}}}}\pend
           \pstart
           \noindent{}\centering{}\textcolor{gray}{\textbf{Director:}}{\\}\textcolor{gray}{\textbf{Dr. Oscar Blumenthal.}}\pend
           \pstart
           \noindent{}\raggedleft{}\textcolor{gray}{\textbf{\textcolor{pink}{Berlin N.W.}{}\ledrightnote{\textcolor{pink}{Berlin}}, den}}{ }1. August \textcolor{gray}{\textbf{189}}2.{\\}\textcolor{gray}{\textbf{\textcolor{pink}{Friedrich-Carl-Ufer}{}\ledrightnote{\textcolor{pink}{Kapelle-Ufer}}}}.\pend
           \pstart\center{}Werther Herr Doktor!\pend\pstart
           Ueber den Aufführungstermin von »\textcolor{green}{Das Märchen}{}\ledrightnote{\textcolor{green}{Das Märchen. Schauspiel in drei Aufzügen}}«
                    kann ich Ihnen im Augenblick eine beſtimmte Zuſage nicht machen, da ſich die
                    Dispoſitionen für die neue Saiſon noch nicht klar genug überblicken laſſen. Doch
                    wird jedenfalls erſt im zweiten Quartal die Aufführung ſtattfinden
                    können, da ich für die Monate Oktober, November,
                        Dezember theils durch die abgeſchloſſenen Verträge, theils
                    durch das Gaſtſpiel der \textcolor{blue}{\textsc{Duse}}{}\ledrightnote{\textcolor{blue}{Eleonora Duse}}{ }ſehr eingeengt bin.\pend
           \pstart
           Mit freundlichen Grüßen\hspace*{2.5em}Ihr{\\[\baselineskip]}\spacefill\mbox{{[}hs. Blumenthal:{]} Dr. Osc. Blumenthal}\pend
           \leftskip=0em{}\pstart
           \noindent{}{[}hs.:{]} Herrn{\\}\textsc{Dr Arthur Schnitzler}{\\}\textsc{\textcolor{pink}{Wien I.}{}\ledrightnote{\textcolor{pink}{I., Innere Stadt}}}\pend
           \pstart
           cop.\pend
           \endnumbering\briefempfaengerindex{Schnitzler, Arthur@\textsc{Schnitzler, Arthur}!zzzBlumenthal, Oskar@\emph{von Oskar Blumenthal}!1892-08-011@{1. 8. 1892}|)be}\mylabel{h}  \normalsize

\doendnotes{C}
\bigskip
\vfill

\clearpage

\footnotesize

\lohead{\textsc{register}}

% Definiere theindex-Environment komplett neu ohne reledmac
\makeatletter
\renewenvironment{theindex}{%
  \section*{\indexname}%
  \setlength{\parindent}{0pt}%
  \setlength{\parskip}{0pt plus 0.3pt}%
  \let\item\@idxitem
}{%
  \clearpage
}
\makeatother

\IfFileExists{\jobname-pw.ind}{\input{\jobname-pw.ind}}{}

\end{document}

      