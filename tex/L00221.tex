%% latex-korrekturansicht-vorspann.tex
%% Vorspann für die Korrekturansicht.
%% Lädt die gemeinsame Datei latex-vorspann.tex mit gesetztem Schalter.

\newif\ifkorrekturansicht
\korrekturansichttrue

\input{../tex-inputs/latex-vorspann}


               \section[Wilhelm Bölsche an Arthur Schnitzler, 12. 6. 1893]{ Wilhelm Bölsche an Arthur Schnitzler, 12. 6. 1893}\nopagebreak\mylabel{v}\rehead{ }\normalsize\beginnumbering\briefempfaengerindex{Schnitzler, Arthur@\textsc{Schnitzler, Arthur}!zzzBoelsche, Wilhelm@\emph{von Wilhelm Bölsche}!1893-06-121@{12. 6. 1893}|(be} \toendnotes[C]{\smallbreak\pagebreak[2]} \Standort{DLA, A:Schnitzler, HS.NZ85.1.2577,7.}
\physDesc{Brief, 1 Blatt, 3 Seiten
\newline{}Handschrift: schwarze Tinte, deutsche Kurrent
\newline{}Schnitzler: mit rotem Buntstift nummeriert: »8« und eine
                                 Unterstreichung }\buchAbdrucke{\weitereDrucke{Wilhelm Bölsche: \emph{Briefwechsel. Mit Autoren der Freien Bühne}. Hg. Gerd-Hermann Susen. Berlin: \emph{Weidler} 2010, S. 687–688 (Werke und Briefe. Wissenschaftliche Ausgabe, Briefe I).} }\toendnotes[C]{\smallbreak}\pstart
           {\pb}\textcolor{gray}{\textbf{\textit{Wilhelm Bölsche}}}\hfill 12. VI. 93\pend
           \pstart
           \textcolor{gray}{\textbf{\textit{\textcolor{pink}{Friedrichshagen}{}\ledrightnote{\textcolor{pink}{Friedrichshagen}}.}}}\pend
           \pstart{}Hochgeehrter Herr Dr!\pend\pstart
           Sie haben ein Recht, ungehalten zu ſein, aber ich wünſchte Sie in meine Lage, um dann
               Ihr Urteil zu hören. Ihr Mahnbrief iſt bis jetzt unbeantwortet geblieben, weil ich
               verreiſt war, – eine äußerſt notwendige Ruhepauſe! Daß Ihre \textcolor{green}{Novelle}{}\ledrightnote{\textcolor{green}{Die Braut}} nicht vorher erledigt war, iſt ja eine redaktionelle
               Sünde. Bei der Maſſe der Einſendung und in Anbetracht des Umſtandes, daß ich die
               Redaktion bis in jede Couvertadreſſe hinein ganz allein zu beſorgen habe, iſt es mir
               allerdings noch nicht einmal als »Ideal« aufgetaucht, ſpäteſtens in 8 Tagen {\pb}jede Einſendung erledigen zu können, zumal da ¾ der
               Einſender ſelbſt bei dicken Romanen und Dramen nicht bloß redaktionelle, ſondern auch
               noch »wirkliche« Urteile verlangen.\pend
           \pstart
           Was Ihre \textcolor{green}{Novelle}{}\ledrightnote{\textcolor{green}{Die Braut}} anbetrifft, ſo iſt ſie mir
               pſychologiſch nicht recht durchdringlich: in dieſer fragmentariſchen Form lieſt ſie
               ſich bloß wie eine Umſchreibung des \textcolor{blue}{Lombroſo}{}\ledrightnote{\textcolor{blue}{Cesare Lombroso}}’ſchen Dogma’s von der gleichſam \label{K_L00221_1v}\edtext{prädeſtinierten Dirne}{\lemma{\textnormal{\emph{prädeſtinierten Dirne}}}\Cendnote{\textnormal{In seinem Werk \emph{\textcolor{green}{La donna
                     delinquente. La prostituta e la donna normale}} (1893, deutsch
                     \emph{\textcolor{green}{Das Weib als Verbrecherin und Prostituierte}},
                     1894) vertrat \textcolor{blue}{Cesare Lombroso}
                  die These, dass die Prostitution mancher Frauen aus ihren ›natürlichen‹ Anlagen
                  erklärbar sei und stellte eine Analogie zu den Männern her, die durch biologische
                  Anlagen zu Verbrechern würden.}}}\label{K_L00221_1h}, aber nicht wie eine Dichtung. Entſchieden
               verlangt dieſer Stoff viel mehr Fleiſch und Blut, und vielleicht bearbeiten Sie ihn
               ſo noch einmal. Die \textcolor{green}{Szene}{}\ledrightnote{→\textcolor{green}{Die Braut}}, {\pb}wie das Mädchen dem Bräutigam ihre Gefühle bekennt,
               halte ich für pſychologiſch ſehr unwahrſcheinlich!\pend
           \pstart
           Mit herzlichem Gruß{\\[\baselineskip]} Ihr{\\[\baselineskip]}\spacefill\mbox{W. Bölsche}\pend
           \leftskip=0em{}\endnumbering\briefempfaengerindex{Schnitzler, Arthur@\textsc{Schnitzler, Arthur}!zzzBoelsche, Wilhelm@\emph{von Wilhelm Bölsche}!1893-06-121@{12. 6. 1893}|)be}\mylabel{h}  \normalsize

\doendnotes{C}
\bigskip
\vfill

\clearpage

\footnotesize

\lohead{\textsc{register}}

% Definiere theindex-Environment komplett neu ohne reledmac
\makeatletter
\renewenvironment{theindex}{%
  \section*{\indexname}%
  \setlength{\parindent}{0pt}%
  \setlength{\parskip}{0pt plus 0.3pt}%
  \let\item\@idxitem
}{%
  \clearpage
}
\makeatother

\IfFileExists{\jobname-pw.ind}{\input{\jobname-pw.ind}}{}

\end{document}

      