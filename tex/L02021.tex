%% latex-korrekturansicht-vorspann.tex
%% Vorspann für die Korrekturansicht.
%% Lädt die gemeinsame Datei latex-vorspann.tex mit gesetztem Schalter.

\newif\ifkorrekturansicht
\korrekturansichttrue

\input{../tex-inputs/latex-vorspann}


               \section[Hugo von Hofmannsthal an Arthur Schnitzler, 2. 6. {[}1911{]}]{ Hugo von Hofmannsthal an Arthur Schnitzler, 2. 6. {[}1911{]}}\nopagebreak\mylabel{v}\rehead{ }\normalsize\beginnumbering\briefempfaengerindex{Schnitzler, Arthur@\textsc{Schnitzler, Arthur}!zzzHofmannsthal, Hugo von@\emph{von Hugo von Hofmannsthal}!1911-06-021@{2. 6. {[}1911{]}}|(be} \toendnotes[C]{\smallbreak\pagebreak[2]} \Standort{CUL, Schnitzler, B 43.}
\physDesc{Brief, 1 Blatt, 2 Seiten
\newline{}Handschrift: schwarze Tinte, deutsche Kurrent
\newline{}Schnitzler: mit Bleistift die Jahreszahl ergänzt: »911« und beschriftet: »Hugo« \newline{}Ordnung: 1) mit Bleistift von unbekannter Hand nummeriert: »\strikeout{321}« 2) mit Bleistift von unbekannter Hand nummeriert:
                                    »330«}\buchAbdrucke{\weitereDrucke{Hugo von Hofmannsthal, Arthur Schnitzler: \emph{Briefwechsel}. Hg. Therese Nickl und Heinrich Schnitzler. Frankfurt am Main: \emph{S. Fischer} 1964, S. 261.} }\toendnotes[C]{\smallbreak}\pstart
           \raggedleft{}{\pb}2. VI.{ }\textcolor{pink}{R}{}\ledrightnote{\textcolor{pink}{Rodaun}}\pend
           \pstart{}mein lieber Arthur\pend\pstart
           ich war minder lang in \textcolor{pink}{Paris}{}\ledrightnote{\textcolor{pink}{Paris}} als ich zu bleiben mir
               vorgeſetzt hatte – beim \label{K_L02021_1v}\edtext{Zurückkommen}{\lemma{\textnormal{\emph{Zurückkommen}}}\Cendnote{\textnormal{am
                     11. 5. 1911}}}\label{K_L02021_1h} war meine Vorfreude groß, Sie nun bald zu
               ſehen, ausgiebig zu ſehen und mehr als einmal, die vielen Fäden fortzuſpinnen, die
               uns verbinden und von denen ja niemals einer abgeriſſen ist, freute mich {\pb}darauf, Euch hier zu ſehen, ehe
               das Haus und die \textcolor{blue}{Kinder}{}\ledrightnote{→\textcolor{blue}{Christiane von Hofmannsthal}{\newline}→\textcolor{blue}{Raimund von Hofmannsthal}{\newline}→\textcolor{blue}{Franz von Hofmannsthal}}{ }ſich Euch ganz entfremden – kam und hörte, nun
               wäret wieder Ihr im Fortgehen, da war ich wirklich ganz traurig. Doch kommt Ihr
               wieder und ſo wird dieſer Brief Sie bald finden und man wird dann nicht mehr lang
               ſein, ohne ſich zu ſehen.\pend
           \pstart
           Vieles Gute Liebe an \textcolor{blue}{Olga}{}\ledrightnote{\textcolor{blue}{Olga Schnitzler}}.{\\[\baselineskip]}Ihr{\\[\baselineskip]}\spacefill\mbox{Hugo}\pend
           \leftskip=0em{}\endnumbering\briefempfaengerindex{Schnitzler, Arthur@\textsc{Schnitzler, Arthur}!zzzHofmannsthal, Hugo von@\emph{von Hugo von Hofmannsthal}!1911-06-021@{2. 6. {[}1911{]}}|)be}\mylabel{h}  \normalsize

\doendnotes{C}
\bigskip
\vfill

\clearpage

\footnotesize

\lohead{\textsc{register}}

% Definiere theindex-Environment komplett neu ohne reledmac
\makeatletter
\renewenvironment{theindex}{%
  \section*{\indexname}%
  \setlength{\parindent}{0pt}%
  \setlength{\parskip}{0pt plus 0.3pt}%
  \let\item\@idxitem
}{%
  \clearpage
}
\makeatother

\IfFileExists{\jobname-pw.ind}{\input{\jobname-pw.ind}}{}

\end{document}

      