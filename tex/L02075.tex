%% latex-korrekturansicht-vorspann.tex
%% Vorspann für die Korrekturansicht.
%% Lädt die gemeinsame Datei latex-vorspann.tex mit gesetztem Schalter.

\newif\ifkorrekturansicht
\korrekturansichttrue

\input{../tex-inputs/latex-vorspann}


               \section[Arthur Schnitzler an Hugo von Hofmannsthal, 1{[}3?{]}. 6. 1912]{ Arthur Schnitzler an Hugo von Hofmannsthal, 1{[}3?{]}. 6. 1912}\nopagebreak\mylabel{v}\rehead{ }\normalsize\beginnumbering\briefempfaengerindex{Hofmannsthal, Hugo von@\textsc{Hofmannsthal, Hugo von}!zzzSchnitzler, Arthur@\emph{von Arthur Schnitzler}!1912-06-131@{1{[}3?{]}. 6. 1912}|(be} \toendnotes[C]{\smallbreak\pagebreak[2]} \Standort{FDH, Hs-30885,145.}
\physDesc{Brief, 1 Blatt (Trauerrand), 3 Seiten
\newline{}Handschrift: schwarze Tinte, deutsche Kurrent}\buchAbdrucke{\weitereDrucke{Hugo von Hofmannsthal, Arthur Schnitzler: \emph{Briefwechsel}. Hg. Therese Nickl und Heinrich Schnitzler. Frankfurt am Main: \emph{S. Fischer} 1964, S. 268.} }\toendnotes[C]{\smallbreak}\pstart
           {\pb}\textcolor{gray}{\textbf{A. S.}}\hfill \textcolor{pink}{Wien}{}\ledrightnote{\textcolor{pink}{Wien}}, 12. 6. 912\pend
           \pstart
           Mein lieber Hugo, für Ihren ſchönen Brief, der mir ans Herz
               gegriffen hat, muß ich Ihnen gleich danken. Zu erwidern hab ich nur mit dem Wunſch,
               daſs es zwiſchen uns bleibe, wie es war und iſt, was die unzerſtörbare innere
               Verknüpfg anbelangt – daſs aber die äußern Verknüpfungen ſich etwas {\pb}häufiger ergeben ſollten, als bisher. Denn das
               »Umeinanderwiſſen« iſt zwar ein edles und ſchmackhaftes aber doch ein magers Brod für
               die Seele. Und um gleich den Anfang zu machen, wir möchten gerne \label{K_L02075_1v}\edtext{nächſte Woche}{\lemma{\textnormal{\emph{nächſte Woche}}}\Cendnote{\textnormal{siehe A. S.: \emph{Tagebuch}, 20. 6. 1912}}}\label{K_L02075_1h} bei Euch angefahren kommen, in den frühen Abendſtunden; gegen Ende, ich
               ſchreibe oder telegrafire den Tag \introOben{}am\introOben{} Montag oder Dinſtag,
                  {\pb}jetzt mach ich mich eben fertig, um \label{K_L02075_2v}\edtext{nach \textcolor{pink}{Prag}{}\ledrightnote{\textcolor{pink}{Prag}}}{\lemma{\textnormal{\emph{nach Prag}}}\Cendnote{\textnormal{Nachdem er erst am 13. 6. 1912 im \emph{\textcolor{green}{Tagebuch}}
                  festhält, zu packen und abzureisen, ohnedies nur einen Tag in \textcolor{pink}{Prag} bleibt und am 15. 6. 1912 bereits retour fährt, dürfte die
                  Datierung \textcolor{blue}{Schnitzler}s nicht stimmen. Am 14. 6. 1912 wurde \emph{\textcolor{green}{Der einsame Weg}} am \textcolor{pink}{Neuen Deutschen Theater} aufgeführt. Laut Ankündigung war es der 12. Teil
                  des »Arthur Schnitzler-Zyklus«.}}}\label{K_L02075_2h} zu fahren, wo ich gezy\substVorne{}\textsuperscript{c}\substDazwischen{}k\substHinten{}elt werde. Ich ſoll mir den \textcolor{green}{Eins. Weg}{}\ledrightnote{\textcolor{green}{Der einsame Weg. Schauspiel in fünf Akten}}
               vorſpielen laſſen.\pend
           \pstart
           Wir grüßen Euch herzlichſt{\\[\baselineskip]} Ihr{\\[\baselineskip]}\spacefill\mbox{Arthur}\pend
           \leftskip=0em{}\endnumbering\briefempfaengerindex{Hofmannsthal, Hugo von@\textsc{Hofmannsthal, Hugo von}!zzzSchnitzler, Arthur@\emph{von Arthur Schnitzler}!1912-06-131@{1{[}3?{]}. 6. 1912}|)be}\mylabel{h}  \normalsize

\doendnotes{C}
\bigskip
\vfill

\clearpage

\footnotesize

\lohead{\textsc{register}}

% Definiere theindex-Environment komplett neu ohne reledmac
\makeatletter
\renewenvironment{theindex}{%
  \section*{\indexname}%
  \setlength{\parindent}{0pt}%
  \setlength{\parskip}{0pt plus 0.3pt}%
  \let\item\@idxitem
}{%
  \clearpage
}
\makeatother

\IfFileExists{\jobname-pw.ind}{\input{\jobname-pw.ind}}{}

\end{document}

      