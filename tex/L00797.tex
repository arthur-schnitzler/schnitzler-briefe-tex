%% latex-korrekturansicht-vorspann.tex
%% Vorspann für die Korrekturansicht.
%% Lädt die gemeinsame Datei latex-vorspann.tex mit gesetztem Schalter.

\newif\ifkorrekturansicht
\korrekturansichttrue

\input{../tex-inputs/latex-vorspann}


               \section[Arthur Schnitzler an Hugo von Hofmannsthal, {[}29. 5. 1898?{]}]{ Arthur Schnitzler an Hugo von Hofmannsthal, {[}29. 5. 1898?{]}}\nopagebreak\mylabel{v}\rehead{ }\normalsize\beginnumbering\briefempfaengerindex{Hofmannsthal, Hugo von@\textsc{Hofmannsthal, Hugo von}!zzzSchnitzler, Arthur@\emph{von Arthur Schnitzler}!1898-05-291@{{[}29. 5. 1898?{]}}|(be} \toendnotes[C]{\smallbreak\pagebreak[2]} \Standort{FDH, Hs-30885,48.}
\physDesc{Brief, 1 Blatt, 2 Seiten
\newline{}Handschrift: Bleistift, deutsche Kurrent\newline{}Ordnung: von Schnitzler mutmaßlich bei der Durchsicht der Korrespondenz 1929 mit Bleistift
                                    beschriftet: »Datum? 95?« }\buchAbdrucke{\weitereDrucke{Hugo von Hofmannsthal, Arthur Schnitzler: \emph{Briefwechsel}. Hg. Therese Nickl und Heinrich Schnitzler. Frankfurt am Main: \emph{S. Fischer} 1964, S. 64.} }\toendnotes[C]{\smallbreak}\pstart
           \noindent{}{\pb}Lieber Hugo, ich höre eben, wir haben eine Loge zu \label{K_L00797_1v}\edtext{\textcolor{green}{\textsc{\uline{Norma}}}{}\ledrightnote{\textcolor{green}{Norma}}}{\lemma{\textnormal{\emph{Norma}}}\Cendnote{\textnormal{Der einzige belegbare Besuch \textcolor{blue}{Schnitzler}s in \emph{\textcolor{green}{Norma}} war am 29. 5. 1898 (\emph{Cambridge University Library} A 179). Am selben Abend vermerkt das
                            \emph{\textcolor{green}{Tagebuch}} ein
                        Abendessen mit \textcolor{blue}{Hofmannsthal}.}}}\label{K_L00797_1h} (\textcolor{blue}{\textsc{Lehmann}}{}\ledrightnote{\textcolor{blue}{Lilli Lehmann}}); bitte kommen Sie vielleicht ſtatt ins \textcolor{pink}{Reſidenzhotel}{}\ledrightnote{\textcolor{pink}{Residenzhotel}} um ½ 9 oder wann Sie wollen in die Loge
                    (2. Stock, \strikeout{links, 9} rechts, 9). – Wenn Sie keine
                        {\pb}Luſt haben (was mir leid thäte), so kommen Sie
                    ins \textcolor{pink}{Reſidenzhotel}{}\ledrightnote{\textcolor{pink}{Residenzhotel}}, aber etwas ſpäter.\pend
           \pstart
           Herzlichen Gruſs{\\[\baselineskip]}Ihr \spacefill\mbox{Arthur}\pend
           \leftskip=0em{}\endnumbering\briefempfaengerindex{Hofmannsthal, Hugo von@\textsc{Hofmannsthal, Hugo von}!zzzSchnitzler, Arthur@\emph{von Arthur Schnitzler}!1898-05-291@{{[}29. 5. 1898?{]}}|)be}\mylabel{h}  \normalsize

\doendnotes{C}
\bigskip
\vfill

\clearpage

\footnotesize

\lohead{\textsc{register}}

% Definiere theindex-Environment komplett neu ohne reledmac
\makeatletter
\renewenvironment{theindex}{%
  \section*{\indexname}%
  \setlength{\parindent}{0pt}%
  \setlength{\parskip}{0pt plus 0.3pt}%
  \let\item\@idxitem
}{%
  \clearpage
}
\makeatother

\IfFileExists{\jobname-pw.ind}{\input{\jobname-pw.ind}}{}

\end{document}

      