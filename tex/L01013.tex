%% latex-korrekturansicht-vorspann.tex
%% Vorspann für die Korrekturansicht.
%% Lädt die gemeinsame Datei latex-vorspann.tex mit gesetztem Schalter.

\newif\ifkorrekturansicht
\korrekturansichttrue

\input{../tex-inputs/latex-vorspann}


               \section[Richard Beer-Hofmann an Arthur Schnitzler, 15. 2. 1900]{ Richard Beer-Hofmann an Arthur Schnitzler, 15. 2. 1900}\nopagebreak\mylabel{v}\rehead{ }\normalsize\beginnumbering\briefempfaengerindex{Schnitzler, Arthur@\textsc{Schnitzler, Arthur}!zzzBeer-Hofmann, Richard@\emph{von Richard Beer-Hofmann}!1900-02-151@{15. 2. 1900}|(be} \toendnotes[C]{\smallbreak\pagebreak[2]} \Standort{CUL, Schnitzler, B 8.}
\physDesc{Bildpostkarte
\newline{}Handschrift: blauer Buntstift, lateinische Kurrent\newline{}Versand: 1) Stempel: »\nobreak{}\oindex{Pegli@\textbf{Pegli}, \emph{http://www.geonames.org/ontologyP.PPLX}|pwk}Pegli (\textcolor{pink}{Genova}), 15 2 00\nobreak{}«.  2) Stempel: »\nobreak{}\oindex{IX., Alsergrund@\textbf{IX., Alsergrund}, \emph{Bezirk (A.BZK)}|pwk}Wien 9/3, 17. 2. 00, 8.V\nobreak{}«. \newline{}Ordnung: mit Bleistift von unbekannter Hand nummeriert: »149« }\buchAbdrucke{\weitereDrucke{Arthur Schnitzler, Richard Beer-Hofmann: \emph{Briefwechsel 1891–1931}. Hg. Konstanze Fliedl. Wien, Zürich: \emph{Europaverlag} 1992, S. 140.} }\pstart{}{\pb}\textcolor{gray}{\textbf{A}}n D\textsuperscript{r} Arthur
                  Schnitzler\pend{}\pstart{}\textcolor{pink}{Wien}{}\ledrightnote{\textcolor{pink}{Wien}}\pend{}\pstart{}\textcolor{pink}{IX Frankgasse 1}{}\ledrightnote{\textcolor{pink}{Frankgasse}}\pend{}\pstart{}\textcolor{pink}{Austria}{}\ledrightnote{\textcolor{pink}{Österreich}}\pend{}{\bigskip}\pstart
           \noindent{}\centering{}\textcolor{gray}{\textbf{{\pb}\textcolor{green}{La Vergine col Figlio}{}\ledrightnote{\textcolor{green}{Madonna mit den Cherubin}} (\textcolor{blue}{A. Mantegna}{}\ledrightnote{\textcolor{blue}{Andrea Mantegna}})}}\pend
           \pstart
           \noindent{}{\pb}\textcolor{gray}{\textbf{\textcolor{pink}{Milano}{}\ledrightnote{\textcolor{pink}{Mailand}}}}\hfill \textcolor{pink}{Pegli}{}\ledrightnote{\textcolor{pink}{Pegli}}{ }15/II 1900\pend
           \pstart
           Lieber Arthur! Ich möchte wissen, I. Was Sie machen, II. Wie \textcolor{blue}{Paul}{}\ledrightnote{\textcolor{blue}{Paul Goldmann}}s Adresse ist. III Ob sein Onkel \textcolor{blue}{Fedor M.}{}\ledrightnote{\textcolor{blue}{Fedor Mamroth}} heißt. IV Ob Sie nach \textcolor{pink}{Italien}{}\ledrightnote{\textcolor{pink}{Italien}} gehen. Ich grüße Sie herzlich\pend
           \pstart Ihr\spacefill\mbox{Richard.}\pend{}\endnumbering\briefempfaengerindex{Schnitzler, Arthur@\textsc{Schnitzler, Arthur}!zzzBeer-Hofmann, Richard@\emph{von Richard Beer-Hofmann}!1900-02-151@{15. 2. 1900}|)be}\mylabel{h}  \normalsize

\doendnotes{C}
\bigskip
\vfill

\clearpage

\footnotesize

\lohead{\textsc{register}}

% Definiere theindex-Environment komplett neu ohne reledmac
\makeatletter
\renewenvironment{theindex}{%
  \section*{\indexname}%
  \setlength{\parindent}{0pt}%
  \setlength{\parskip}{0pt plus 0.3pt}%
  \let\item\@idxitem
}{%
  \clearpage
}
\makeatother

\IfFileExists{\jobname-pw.ind}{\input{\jobname-pw.ind}}{}

\end{document}

      