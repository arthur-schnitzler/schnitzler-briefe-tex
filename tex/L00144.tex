%% latex-korrekturansicht-vorspann.tex
%% Vorspann für die Korrekturansicht.
%% Lädt die gemeinsame Datei latex-vorspann.tex mit gesetztem Schalter.

\newif\ifkorrekturansicht
\korrekturansichttrue

\input{../tex-inputs/latex-vorspann}


               \section[Hugo von Hofmannsthal an Arthur Schnitzler, 23. 12. {[}1892{]}]{ Hugo von Hofmannsthal an Arthur Schnitzler, 23. 12. {[}1892{]}}\nopagebreak\mylabel{v}\rehead{ }\normalsize\beginnumbering\briefempfaengerindex{Schnitzler, Arthur@\textsc{Schnitzler, Arthur}!zzzHofmannsthal, Hugo von@\emph{von Hugo von Hofmannsthal}!1892-12-231@{23. 12. {[}1892{]}}|(be} \toendnotes[C]{\smallbreak\pagebreak[2]} \Standort{CUL, Schnitzler, B 43.}
\physDesc{Brief, 1 Blatt (Briefpapier mit aufgeprägtem Wappen), 2 Seiten
\newline{}Handschrift: schwarze Tinte, deutsche Kurrent
\newline{}Schnitzler: mit Bleistift nummeriert: »35« und mit einer
                                 Jahreszahl versehen: »92« }\buchAbdrucke{\weitereDrucke{Hugo von Hofmannsthal, Arthur Schnitzler: \emph{Briefwechsel}. Hg. Therese Nickl und Heinrich Schnitzler. Frankfurt am Main: \emph{S. Fischer} 1964, S. 32–33.} }\toendnotes[C]{\smallbreak}\pstart
           \raggedleft{}{\pb}23 December.\pend
           \pstart{}mein lieber Arthur.\pend\pstart
           Ich glaube, ich werde beſſer nicht über \textcolor{green}{Anatol}{}\ledrightnote{\textcolor{green}{Anatol}}{ }ſchreiben. Die Mühe, beinahe Überwindung, die es
               mich koſtet, macht mich ſtutzig. Sich dem Vorwurf der tactloſen Camaraderie ausſetzen
               und nichts dabei erzielen als eine gequälte mühſam gedehnte Beſprechung?\pend
           \pstart
           Ich weiß offenbar zu viel von dem \textcolor{green}{Buch}{}\ledrightnote{→\textcolor{green}{Anatol}} und ſehe daher nicht klar. Oder Gott weiß, was es ſonſt iſt. Vielleicht
               erlauben {\pb}Sie mir, Ihnen nächſtens
               die 50 Zeilen mitzubringen, die ich zuſammengebracht habe; vielleicht können wir die
               Kritik der Kritik machen und dabei etwas lernen. Wann in der Weihnachtswoche werden
               wir uns ausgiebig ſehen? und was machen die Proben mit \textcolor{blue}{Paul Horn}{}\ledrightnote{\textcolor{blue}{Paul Horn}} und \label{K_L00144_1v}\edtext{\textsc{\textcolor{green}{Aspasia}{}\ledrightnote{\textcolor{green}{Aspasia}}–\textcolor{blue}{Dora}{}\ledrightnote{\textcolor{blue}{Dorothea Kohnberger}}}}{\lemma{\textnormal{\emph{Aspasia–Dora}}}\Cendnote{\textnormal{Bei \emph{\textcolor{green}{Aspasia}} könnte es sich um die gleichnamige Oper von \textcolor{blue}{Carl Schroeder} handeln, die am 3. 3. 1892
                  uraufgeführt worden war. Möglicherweise wurden Partien aus ihr von \textcolor{blue}{Dora Kohnberger} im Zuge einer Privataufführung
                  bei \textcolor{blue}{Bertha Flegmann} einstudiert.}}}\label{K_L00144_1h}?\pend
           \pstart
           Allerherzlichſt Ihr immer dankbar und aufrichtig ergebener (4\textsuperscript{ter} Grad){\\[\baselineskip]}\spacefill\mbox{Loris}\pend
           \leftskip=0em{}\endnumbering\briefempfaengerindex{Schnitzler, Arthur@\textsc{Schnitzler, Arthur}!zzzHofmannsthal, Hugo von@\emph{von Hugo von Hofmannsthal}!1892-12-231@{23. 12. {[}1892{]}}|)be}\mylabel{h}  \normalsize

\doendnotes{C}
\bigskip
\vfill

\clearpage

\footnotesize

\lohead{\textsc{register}}

% Definiere theindex-Environment komplett neu ohne reledmac
\makeatletter
\renewenvironment{theindex}{%
  \section*{\indexname}%
  \setlength{\parindent}{0pt}%
  \setlength{\parskip}{0pt plus 0.3pt}%
  \let\item\@idxitem
}{%
  \clearpage
}
\makeatother

\IfFileExists{\jobname-pw.ind}{\input{\jobname-pw.ind}}{}

\end{document}

      