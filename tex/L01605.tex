%% latex-korrekturansicht-vorspann.tex
%% Vorspann für die Korrekturansicht.
%% Lädt die gemeinsame Datei latex-vorspann.tex mit gesetztem Schalter.

\newif\ifkorrekturansicht
\korrekturansichttrue

\input{../tex-inputs/latex-vorspann}


               \section[Hugo von Hofmannsthal an Arthur Schnitzler, {[}zwischen 24. und 27. 6. 1906{]}]{ Hugo von Hofmannsthal an Arthur Schnitzler, {[}zwischen 24. und
               27. 6. 1906{]}}\nopagebreak\mylabel{v}\rehead{ }\normalsize\beginnumbering\briefempfaengerindex{Schnitzler, Arthur@\textsc{Schnitzler, Arthur}!zzzHofmannsthal, Hugo von@\emph{von Hugo von Hofmannsthal}!1906-06-243@{{[}zwischen 24. und 27. 6. 1906{]}}|(be} \toendnotes[C]{\smallbreak\pagebreak[2]} \Standort{CUL, Schnitzler, B 43.}
\physDesc{Brief, 1 Blatt, 2 Seiten
\newline{}Handschrift: schwarze Tinte, deutsche Kurrent
\newline{}Schnitzler: mit Bleistift (falsch) datiert: »Ende Juni 901« \newline{}Ordnung: 1) mit Bleistift von unbekannter Hand nummeriert: »\strikeout{180}« 2) mit Bleistift von unbekannter Hand nummeriert:
                                    »176«}\buchAbdrucke{\weitereDrucke{Hugo von Hofmannsthal, Arthur Schnitzler: \emph{Briefwechsel}. Hg. Therese Nickl und Heinrich Schnitzler. Frankfurt am Main: \emph{S. Fischer} 1964, S. 148.} }\toendnotes[C]{\smallbreak}\pstart
           \noindent{}\centering{}{\pb}\textcolor{gray}{\textbf{\textcolor{pink}{Südbahn-Hôtel Semmering}{}\ledrightnote{\textcolor{pink}{Südbahnhotel}}.}}\pend
           \pstart
           \noindent{}\textcolor{gray}{\textbf{TELEGRAMME:}}\pend
           \pstart
           \textcolor{gray}{\textbf{\textcolor{pink}{SÜDBAHNHÔTEL SEMMERING}{}\ledrightnote{\textcolor{pink}{Südbahnhotel}}}}. \pend
           \pstart
           \textcolor{gray}{\textbf{TELEPHON:}}\pend
           \pstart
           \textcolor{gray}{\textbf{HÔTEL{ }{\dotsfour} Nr. 5.}}. \pend
           \pstart
           \textcolor{gray}{\textbf{DEPENDANCE Nr. 6.}}\pend
           \pstart
           lieber\hspace*{1.5em}bitte ſchicken Sie mir \label{K_L01605_1v}\edtext{nach \textcolor{pink}{Rodaun}{}\ledrightnote{\textcolor{pink}{Rodaun}}}{\lemma{\textnormal{\emph{nach Rodaun}}}\Cendnote{\textnormal{Der Aufenthalt am \textcolor{pink}{Semmering} fand von 23.–27. 6. 1906
                  statt, was die Datierung ermöglicht.}}}\label{K_L01605_1h} die \textcolor{green}{Selbſtbiografie}{}\ledrightnote{→\textcolor{green}{Memoiren meines Lebens}} von \textcolor{blue}{Caſtelli}{}\ledrightnote{\textcolor{blue}{Ignaz Franz Castelli}}. Ferner wenn Sie eine gute Biographie von \textcolor{blue}{\uline{Raimund}}{}\ledrightnote{\textcolor{blue}{Ferdinand Raimund}} haben, ſowie Briefe oder Tagebücher von \textcolor{blue}{Raimund}{}\ledrightnote{\textcolor{blue}{Ferdinand Raimund}}. Ferner wenn Sie etwas dergleichen das näheres über \textcolor{blue}{Raimund}{}\ledrightnote{\textcolor{blue}{Ferdinand Raimund}} enthält, nicht haben aber \uline{wiſſen}, ſo ſchreiben Sie mir bitte den Titel gleich. Bitte ſchicken Sie
               alles möglichſt {\pb}bald. Ich bin
               herzlich dankbar dafür.\pend
           \pstart
           Den \textcolor{green}{\textcolor{blue}{Pöhnl}{}\ledrightnote{\textcolor{blue}{Hans Pöhnl}}}{}\ledrightnote{→\textcolor{green}{Deutsche Volksbühnenspiele}}{ }ſchick ich per Poſt an Sie zurück.\pend
           \pstart
           Wie lange ſind Sie noch da?{\\[\baselineskip]}Ihr\spacefill\mbox{Hugo.}\pend
           \leftskip=0em{}\endnumbering\briefempfaengerindex{Schnitzler, Arthur@\textsc{Schnitzler, Arthur}!zzzHofmannsthal, Hugo von@\emph{von Hugo von Hofmannsthal}!1906-06-243@{{[}zwischen 24. und 27. 6. 1906{]}}|)be}\mylabel{h}  \normalsize

\doendnotes{C}
\bigskip
\vfill

\clearpage

\footnotesize

\lohead{\textsc{register}}

% Definiere theindex-Environment komplett neu ohne reledmac
\makeatletter
\renewenvironment{theindex}{%
  \section*{\indexname}%
  \setlength{\parindent}{0pt}%
  \setlength{\parskip}{0pt plus 0.3pt}%
  \let\item\@idxitem
}{%
  \clearpage
}
\makeatother

\IfFileExists{\jobname-pw.ind}{\input{\jobname-pw.ind}}{}

\end{document}

      