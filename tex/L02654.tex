%% latex-korrekturansicht-vorspann.tex
%% Vorspann für die Korrekturansicht.
%% Lädt die gemeinsame Datei latex-vorspann.tex mit gesetztem Schalter.

\newif\ifkorrekturansicht
\korrekturansichttrue

\input{../tex-inputs/latex-vorspann}


               \section[Paul Lasker-Schüler an Arthur Schnitzler, 28. 4. {[}1925?{]}]{ Paul Lasker-Schüler an Arthur Schnitzler, 28. 4. {[}1925?{]}}\nopagebreak\mylabel{v}\rehead{ }\normalsize\beginnumbering\briefempfaengerindex{Schnitzler, Arthur@\textsc{Schnitzler, Arthur}!zzzLasker-Schueler, Paul@\emph{von Paul Lasker-Schüler}!1925-04-281@{28. 4. {[}1925?{]}}|(be} \toendnotes[C]{\smallbreak\pagebreak[2]} \Standort{DLA, A:Schnitzler, HS.1985.1.3876.}
\physDesc{Brief, 1 Blatt, 1 Seite, maschinelle Abschrift
\newline{}Schreibmaschine}\toendnotes[C]{\smallbreak}\pstart
           {\pb}\label{K_L02654-1v}\edtext{28. IV.}{\lemma{\textnormal{\emph{28. IV.}}}\Cendnote{\textnormal{Es ist kein Besuch \textcolor{blue}{Paul Lasker-Schüler}s bei \textcolor{blue}{Schnitzler} bekannt, durch den das Datum des Briefes gesichert bestimmt
                     werden könnte. Da der Brief nur in Abschrift vorliegt, lässt sich nicht 
                     mit Gewissheit ausschließen, dass der Abschreiber, die Abschreiberin bei der Entzifferung der Monatsangabe keinen
                     Fehler gemacht hat. Mit Hilfe der freundlichen Auskunft von Karl Jürgen
                     Skrodzki lässt sich folgende Argumentation führen, warum der Brief 1925
                     entstanden sein muss. Unter der Annahme, dass die Monatsangabe stimmt, kommen nur
                     die Jahre 1924 und 1925 in Betracht, da sich
                     hier \textcolor{blue}{Paul Lasker-Schüler} im April in \textcolor{pink}{Wien}
                     aufhielt. Der Brief wurde mit großer Wahrscheinlichkeit nicht vor jenem \textcolor{blue}{Else Lasker-Schüler}s an \textcolor{blue}{Schnitzler}
                     (10. 12. 1924) verfasst. Paul Lasker-Schüler hätte ohne
                     diese Vorarbeit seiner \textcolor{blue}{Mutter} vermutlich nicht an \textcolor{blue}{Schnitzler} geschrieben.}}}\label{K_L02654-1h}\pend
           \pstart
           \centering{}Paul Lasker-Schüler\pend
           {\bigskip}\pstart
           \noindent{}bittet vielmals darum, empfangen zu werden, wenn es irgend
               möglich vielleicht noch heute, denn es handelt sich um einen \label{K_L02654-2v}\edtext{medizinischen Ratschlag}{\lemma{\textnormal{\emph{medizinischen Ratschlag}}}\Cendnote{\textnormal{Im Dezember 1925 erkrankte Paul Lasker-Schüler an Tuberkulose.}}}\label{K_L02654-2h}.\pend
           \pstart
           Ich bitte Sie vielmals Herr Doktor, mir meine Aufdringlichkeit nicht übel zu nehmen.
               Meine Adresse ist \textcolor{pink}{Pension Bleckmann}{}\ledrightnote{\textcolor{pink}{Pension Bleckmann}}{ }{\\}Thelephon 26 206.\pend
           \endnumbering\briefempfaengerindex{Schnitzler, Arthur@\textsc{Schnitzler, Arthur}!zzzLasker-Schueler, Paul@\emph{von Paul Lasker-Schüler}!1925-04-281@{28. 4. {[}1925?{]}}|)be}\mylabel{h}  \normalsize

\doendnotes{C}
\bigskip
\vfill

\clearpage

\footnotesize

\lohead{\textsc{register}}

% Definiere theindex-Environment komplett neu ohne reledmac
\makeatletter
\renewenvironment{theindex}{%
  \section*{\indexname}%
  \setlength{\parindent}{0pt}%
  \setlength{\parskip}{0pt plus 0.3pt}%
  \let\item\@idxitem
}{%
  \clearpage
}
\makeatother

\IfFileExists{\jobname-pw.ind}{\input{\jobname-pw.ind}}{}

\end{document}

      