%% latex-korrekturansicht-vorspann.tex
%% Vorspann für die Korrekturansicht.
%% Lädt die gemeinsame Datei latex-vorspann.tex mit gesetztem Schalter.

\newif\ifkorrekturansicht
\korrekturansichttrue

\input{../tex-inputs/latex-vorspann}


               \section[Arthur Schnitzler an Richard Beer-Hofmann, {[}zwischen 5. 5. 1893 und 2. 5. 1894{]}]{ Arthur Schnitzler an Richard Beer-Hofmann, {[}zwischen 5. 5. 1893 und
               2. 5. 1894{]}}\nopagebreak\mylabel{v}\rehead{ }\normalsize\beginnumbering\briefempfaengerindex{Beer-Hofmann, Richard@\textsc{Beer-Hofmann, Richard}!zzzSchnitzler, Arthur@\emph{von Arthur Schnitzler}!1893-05-042@{{[}zwischen 5. 5. 1893 und
                  2. 5. 1894{]}}|(be} \toendnotes[C]{\smallbreak\pagebreak[2]} \Standort{YCGL, MSS 31.}
\physDesc{Brief, 1 Blatt (Briefpapier mit Trauerrand), 2 Seiten
\newline{}Handschrift: Bleistift, deutsche Kurrent}\toendnotes[C]{\smallbreak}\pstart
           \noindent{}{\pb}Herzlichen Gruſs, ich freue mich ſehr Sie \label{K_L00211_1v}\edtext{heute}{\lemma{\textnormal{\emph{heute}}}\Cendnote{\textnormal{Das undatierte Korrespondenzstück lässt sich durch den
                  Trauerrand in die Zeit nach der Beerdigung und in das Jahr nach dem Tod des Vaters \textcolor{blue}{Johann Schnitzler} einordnen.}}}\label{K_L00211_1h}{ }Abend zu ſehen.\pend
           \pstart Ihr \spacefill\mbox{Arthur}\pend{}\pstart
           \noindent{}{\pb}Vielleicht erfahre ich noch, wo Sie vorher
                  ſind?\pend
           \pstart
           Es iſt aber nicht nothwendig\pend
           \endnumbering\briefempfaengerindex{Beer-Hofmann, Richard@\textsc{Beer-Hofmann, Richard}!zzzSchnitzler, Arthur@\emph{von Arthur Schnitzler}!1893-05-042@{{[}zwischen 5. 5. 1893 und
                  2. 5. 1894{]}}|)be}\mylabel{h}  \normalsize

\doendnotes{C}
\bigskip
\vfill

\clearpage

\footnotesize

\lohead{\textsc{register}}

% Definiere theindex-Environment komplett neu ohne reledmac
\makeatletter
\renewenvironment{theindex}{%
  \section*{\indexname}%
  \setlength{\parindent}{0pt}%
  \setlength{\parskip}{0pt plus 0.3pt}%
  \let\item\@idxitem
}{%
  \clearpage
}
\makeatother

\IfFileExists{\jobname-pw.ind}{\input{\jobname-pw.ind}}{}

\end{document}

      