%% latex-korrekturansicht-vorspann.tex
%% Vorspann für die Korrekturansicht.
%% Lädt die gemeinsame Datei latex-vorspann.tex mit gesetztem Schalter.

\newif\ifkorrekturansicht
\korrekturansichttrue

\input{../tex-inputs/latex-vorspann}


               \section[Paul Goldmann an Arthur Schnitzler, 30. 6. 1891]{ Paul Goldmann an Arthur Schnitzler, 30. 6. 1891}\nopagebreak\mylabel{v}\rehead{ }\normalsize\beginnumbering\briefempfaengerindex{Schnitzler, Arthur@\textsc{Schnitzler, Arthur}!zzzGoldmann, Paul@\emph{von Paul Goldmann}!1891-06-301@{30. 6. 1891}|(be} \toendnotes[C]{\smallbreak\pagebreak[2]} \Standort{DLA, A:Schnitzler, HS.NZ85.1.3162.}
\physDesc{Postkarte
\newline{}Handschrift: 1) schwarze Tinte, deutsche Kurrent\hspace{1em}2) schwarze Tinte, lateinische Kurrent (\noindent{}Adresse)\hspace{1em}\newline{}Versand: 1) Stempel: »\nobreak{}\oindex{Bruessel@\textbf{Brüssel}, \emph{Besiedelter Ort (A.BSO)}|pwk}\begin{otherlanguage}{french}Bruxelles\end{otherlanguage}, \begin{otherlanguage}{french}30 Juin 1891\end{otherlanguage}, 9-S\nobreak{}«.  2) Stempel: »\nobreak{}Wien 1/1, 2{[}.{]} 7. 91, Bestellt\nobreak{}«. 
\newline{}Schnitzler: mit Bleistift das Datum »30/ 6 91« vermerkt }\toendnotes[C]{\smallbreak}\pstart{}{\pb}\textcolor{pink}{Autriche!}{}\ledrightnote{\textcolor{pink}{Österreich}}\pend{}\pstart{}\begin{otherlanguage}{french}\textcolor{gray}{\textbf{M}}onsieur le docteur Arthur Schnitzler\end{otherlanguage}\pend{}\pstart{}\textcolor{pink}{\begin{otherlanguage}{french}Vienne\end{otherlanguage}}{}\ledrightnote{\textcolor{pink}{Wien}}\pend{}\pstart{}\textcolor{pink}{I. Giselastraße 11.}{}\ledrightnote{\textcolor{pink}{Bösendorferstraße}}\pend{}{\bigskip}\pstart
           \noindent{}{\pb}\label{K_L02665-1v}\edtext{Alſo doch?!}{\lemma{\textnormal{\emph{Alſo doch?!}}}\Cendnote{\textnormal{Bezug unklar}}}\label{K_L02665-1h}\pend
           \endnumbering\briefempfaengerindex{Schnitzler, Arthur@\textsc{Schnitzler, Arthur}!zzzGoldmann, Paul@\emph{von Paul Goldmann}!1891-06-301@{30. 6. 1891}|)be}\mylabel{h}  \normalsize

\doendnotes{C}
\bigskip
\vfill

\clearpage

\footnotesize

\lohead{\textsc{register}}

% Definiere theindex-Environment komplett neu ohne reledmac
\makeatletter
\renewenvironment{theindex}{%
  \section*{\indexname}%
  \setlength{\parindent}{0pt}%
  \setlength{\parskip}{0pt plus 0.3pt}%
  \let\item\@idxitem
}{%
  \clearpage
}
\makeatother

\IfFileExists{\jobname-pw.ind}{\input{\jobname-pw.ind}}{}

\end{document}

      