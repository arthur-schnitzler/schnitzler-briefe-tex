%% latex-korrekturansicht-vorspann.tex
%% Vorspann für die Korrekturansicht.
%% Lädt die gemeinsame Datei latex-vorspann.tex mit gesetztem Schalter.

\newif\ifkorrekturansicht
\korrekturansichttrue

\input{../tex-inputs/latex-vorspann}


               \section[Olga Schnitzler an Richard und Paula Beer-Hofmann, {[}12. 12. 1909{]}]{ Olga Schnitzler an Richard und Paula Beer-Hofmann,
               {[}12. 12. 1909{]}}\nopagebreak\mylabel{v}\rehead{ }\normalsize\beginnumbering\briefempfaengerindex{Beer-Hofmann, Paula@\textsc{Beer-Hofmann, Paula}!zzzSchnitzler, Olga@\emph{von Olga Schnitzler}!1909-12-121@{{[}12. 12. 1909{]}}|(be}\briefempfaengerindex{Beer-Hofmann, Richard@\textsc{Beer-Hofmann, Richard}!zzzSchnitzler, Olga@\emph{von Olga Schnitzler}!1909-12-121@{{[}12. 12. 1909{]}}|(be} \toendnotes[C]{\smallbreak\pagebreak[2]} \Standort{YCGL, MSS 31.}
\physDesc{Briefkarte, oben rechts mit Bleistift beschriftet: »S.«
\newline{}Handschrift: schwarze Tinte, lateinische Kurrent\newline{}Versand: ohne postalischen Übermittlungsvermerk }\buchAbdrucke{\weitereDrucke{Arthur Schnitzler, Richard Beer-Hofmann: \emph{Briefwechsel 1891–1931}. Hg. Konstanze Fliedl. Wien, Zürich: \emph{Europaverlag} 1992, S. 206.} }\toendnotes[C]{\smallbreak}\pstart{}{\pb}Herrn u. Frau\pend{}\pstart{}D\textsuperscript{r} Richard Beer-Hofmann\pend{}\pstart{}\textcolor{pink}{Hasenauerstrasse 59}{}\ledrightnote{\textcolor{pink}{Hasenauerstraße}}. \pend{}{\bigskip}\pstart
           \noindent{}{\pb}Liebe Beer-Hofmanns, \label{K_L01898-1v}\edtext{morgen}{\lemma{\textnormal{\emph{morgen}}}\Cendnote{\textnormal{siehe A. S.: \emph{Tagebuch}, 13. 12. 1909}}}\label{K_L01898-1h}{ }Abend kommen \textcolor{blue}{Kaufmann}{}\ledrightnote{\textcolor{blue}{Arthur Kaufmann}}, \textcolor{blue}{Leo}{}\ledrightnote{\textcolor{blue}{Leo Van-Jung}} und wahrscheinlich auch \textcolor{blue}{Saltens}{}\ledrightnote{\textcolor{blue}{Felix Salten}{\newline}\textcolor{blue}{Ottilie Salten}} zu uns, wir würden uns sehr freuen und das Gefühl
               von 2 aufgesetzten Kronen haben, wenn Ihr auch kämet.\pend
           \pstart
           Herzliche Grüsse!{\\[\baselineskip]}\spacefill\mbox{Olga.}\pend
           \leftskip=0em{}\pstart
           \noindent{}\label{KLL01898_Beer-Hofmann-1v}\edtext{Sonntag}{\lemma{\textnormal{\emph{Sonntag}}}\Cendnote{\textnormal{Die Datierung folgt dem
                     Tagebucheintrag vom 13. 12. 1909}}}\label{KLL01898_Beer-Hofmann-1h}.\hspace*{1.5em}Bitte früh kommen!\pend
           \endnumbering\briefempfaengerindex{Beer-Hofmann, Paula@\textsc{Beer-Hofmann, Paula}!zzzSchnitzler, Olga@\emph{von Olga Schnitzler}!1909-12-121@{{[}12. 12. 1909{]}}|)be}\briefempfaengerindex{Beer-Hofmann, Richard@\textsc{Beer-Hofmann, Richard}!zzzSchnitzler, Olga@\emph{von Olga Schnitzler}!1909-12-121@{{[}12. 12. 1909{]}}|)be}\mylabel{h}  \normalsize

\doendnotes{C}
\bigskip
\vfill

\clearpage

\footnotesize

\lohead{\textsc{register}}

% Definiere theindex-Environment komplett neu ohne reledmac
\makeatletter
\renewenvironment{theindex}{%
  \section*{\indexname}%
  \setlength{\parindent}{0pt}%
  \setlength{\parskip}{0pt plus 0.3pt}%
  \let\item\@idxitem
}{%
  \clearpage
}
\makeatother

\IfFileExists{\jobname-pw.ind}{\input{\jobname-pw.ind}}{}

\end{document}

      