%% latex-korrekturansicht-vorspann.tex
%% Vorspann für die Korrekturansicht.
%% Lädt die gemeinsame Datei latex-vorspann.tex mit gesetztem Schalter.

\newif\ifkorrekturansicht
\korrekturansichttrue

\input{../tex-inputs/latex-vorspann}


               \section[Hugo von Hofmannsthal an Arthur Schnitzler, 18. 10. 1904]{ Hugo von Hofmannsthal an Arthur Schnitzler, 18. 10. 1904}\nopagebreak\mylabel{v}\rehead{ }\normalsize\beginnumbering\briefempfaengerindex{Schnitzler, Arthur@\textsc{Schnitzler, Arthur}!zzzHofmannsthal, Hugo von@\emph{von Hugo von Hofmannsthal}!1904-10-181@{18. 10. 1904}|(be} \toendnotes[C]{\smallbreak\pagebreak[2]} \Standort{CUL, Schnitzler, B 43.}
\physDesc{Postkarte
\newline{}Handschrift Gertrude von Hofmannsthal: schwarze Tinte, lateinische Kurrent\newline{}Versand: 1) Stempel: »\nobreak{}\oindex{X., Favoriten@\textbf{X., Favoriten}, \emph{Bezirk (A.BZK)}|pwk}Wien 10/2, 18. X. 04, 10\nobreak{}«.  2) Stempel: »\nobreak{}\oindex{VII., Neubau@\textbf{VII., Neubau}, \emph{Bezirk (A.BZK)}|pwk}Wien 7/3, 19. 10. 04, 8–9V, Bestellt\nobreak{}«. 3) Stempel: »\nobreak{}\oindex{VII., Neubau@\textbf{VII., Neubau}, \emph{Bezirk (A.BZK)}|pwk}18/1 Wien 110, 19. 10. 04, 10V, Bestellt\nobreak{}«. 
\newline{}Schnitzler: mit Bleistift datiert: »18/10 904« \newline{}Ordnung: 1) mit Bleistift von unbekannter Hand nummeriert: »\strikeout{227}« 2) mit Bleistift von unbekannter Hand nummeriert:
                                    »240«}\buchAbdrucke{\weitereDrucke{Hugo von Hofmannsthal, Arthur Schnitzler: \emph{Briefwechsel}. Hg. Therese Nickl und Heinrich Schnitzler. Frankfurt am Main: \emph{S. Fischer} 1964, S. 207.} }\toendnotes[C]{\smallbreak}\pstart{}{\pb}Herrn Dr Arthur
                  Schnitzler\pend{}\pstart{}\textcolor{pink}{Wien}{}\ledrightnote{\textcolor{pink}{Wien}}\pend{}\pstart{}\textcolor{pink}{XVIII Spöttlgasse 7}{}\ledrightnote{\textcolor{pink}{Edmund-Weiß-Gasse}}. \pend{}{\bigskip}\pstart
           \noindent{}{\pb}Mit Freude \label{K_L01457_1v}\edtext{Mittwoch}{\lemma{\textnormal{\emph{Mittwoch}}}\Cendnote{\textnormal{siehe A. S.: \emph{Tagebuch}, 19. 10. 1904}}}\label{K_L01457_1h}{ }abends{ }\textcolor{pink}{Hietzing}{}\ledrightnote{\textcolor{pink}{XIII., Hietzing}}\pend
           \pstart Herzlichst \spacefill\mbox{Hugo.}\pend{}\endnumbering\briefempfaengerindex{Schnitzler, Arthur@\textsc{Schnitzler, Arthur}!zzzHofmannsthal, Hugo von@\emph{von Hugo von Hofmannsthal}!1904-10-181@{18. 10. 1904}|)be}\mylabel{h}  \normalsize

\doendnotes{C}
\bigskip
\vfill

\clearpage

\footnotesize

\lohead{\textsc{register}}

% Definiere theindex-Environment komplett neu ohne reledmac
\makeatletter
\renewenvironment{theindex}{%
  \section*{\indexname}%
  \setlength{\parindent}{0pt}%
  \setlength{\parskip}{0pt plus 0.3pt}%
  \let\item\@idxitem
}{%
  \clearpage
}
\makeatother

\IfFileExists{\jobname-pw.ind}{\input{\jobname-pw.ind}}{}

\end{document}

      