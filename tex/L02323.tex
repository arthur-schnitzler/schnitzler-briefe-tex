%% latex-korrekturansicht-vorspann.tex
%% Vorspann für die Korrekturansicht.
%% Lädt die gemeinsame Datei latex-vorspann.tex mit gesetztem Schalter.

\newif\ifkorrekturansicht
\korrekturansichttrue

\input{../tex-inputs/latex-vorspann}


               \section[Hugo Hofmannsthal an Arthur Schnitzler, 20. 4. 1919]{ Hugo Hofmannsthal an Arthur Schnitzler, 20. 4. 1919}\nopagebreak\mylabel{v}\rehead{ }\normalsize\beginnumbering\briefempfaengerindex{Schnitzler, Arthur@\textsc{Schnitzler, Arthur}!zzzHofmannsthal, Hugo von@\emph{von Hugo von Hofmannsthal}!1919-04-201@{20. 4. 1919}|(be} \toendnotes[C]{\smallbreak\pagebreak[2]} \Standort{CUL, Schnitzler, B 43.}
\physDesc{Brief, 1 Blatt, 2 Seiten
\newline{}Handschrift: schwarze Tinte, deutsche Kurrent
\newline{}Schnitzler: mit Bleistift beschriftet: »\textsc{Hugo}« \newline{}Ordnung: 1) mit Bleistift von \textcolor{blue}{Frieda Pollak} (?) mit dem Buchstaben »A« (Abgeschrieben/Abschrift) gekennzeichnet 2) mit Bleistift von unbekannter Hand nummeriert: »352«}\buchAbdrucke{\weitereDrucke{Hugo von Hofmannsthal, Arthur Schnitzler: \emph{Briefwechsel}. Hg. Therese Nickl und Heinrich Schnitzler. Frankfurt am Main: \emph{S. Fischer} 1964, S. 283.} }\toendnotes[C]{\smallbreak}\pstart
           \raggedleft{}{\pb}\textcolor{pink}{Rodaun}{}\ledrightnote{\textcolor{pink}{Rodaun}}, Osterſonntag 19.\pend
           \pstart{}mein lieber Arthur\pend\pstart
           grüß Sie Gott. Wie gehts Ihnen denn immer?\pend
           \pstart
           Ich bin ſchon ſeit 3 Wochen krank, muſs jetzt liegen wegen einer
                    Rippenfellreizung. Sie waren ja auch in dieſem Winter \label{K_L02323_1v}\edtext{einmal recht krank}{\lemma{\textnormal{\emph{einmal recht krank}}}\Cendnote{\textnormal{siehe A. S.: \emph{Tagebuch}, 20. 1. 1919}}}\label{K_L02323_1h} u. ich
                    hab es gar nicht gewuſst!\pend
           \pstart
           Ich bitte Sie Arthur, wegen dieſer \textcolor{brown}{Autorenorganiſation}{}\ledrightnote{→\textcolor{brown}{Deutschösterreichischer Autorenverband}}, daſs Sie eventuell den Leuten von mir ſagen, daſs
                    ich krank bin, und dann autoriſiere ich Sie, alles was Ihnen zu {\pb}beſchließen oder wozu
                        zuzuſti{\geminationm}en Ihnen richtig erſcheint, dies auch
                    in meinem Namen zu tun.\pend
           \pstart
           Ich wundere mich nur wie man eine ſpecielle \textcolor{brown}{Organiſation}{}\ledrightnote{→\textcolor{brown}{Deutschösterreichischer Autorenverband}} in \textcolor{pink}{Oeſterreich}{}\ledrightnote{\textcolor{pink}{Österreich}}{ }ſchaffen will, da wir doch alle an dem
                    deutſchen geſa{\geminationm}ten Bühnenweſen beteiligt ſind, –
                    aber ſei dem wie immer.\pend
           \pstart
           In alter Liebe{\\[\baselineskip]}Ihr{\\[\baselineskip]}\spacefill\mbox{Hugo.}\pend
           \leftskip=0em{}\pstart
           \noindent{}\textsc{PS}. Alles Gute an \textcolor{blue}{Olga}{}\ledrightnote{\textcolor{blue}{Olga Schnitzler}}. Wie ſchön war man früher oft zuſa{\geminationm}en. Im Bett liegend, genieße ich manches
                        Freundliche in der Erinnerung.\pend
           \endnumbering\briefempfaengerindex{Schnitzler, Arthur@\textsc{Schnitzler, Arthur}!zzzHofmannsthal, Hugo von@\emph{von Hugo von Hofmannsthal}!1919-04-201@{20. 4. 1919}|)be}\mylabel{h}  \normalsize

\doendnotes{C}
\bigskip
\vfill

\clearpage

\footnotesize

\lohead{\textsc{register}}

% Definiere theindex-Environment komplett neu ohne reledmac
\makeatletter
\renewenvironment{theindex}{%
  \section*{\indexname}%
  \setlength{\parindent}{0pt}%
  \setlength{\parskip}{0pt plus 0.3pt}%
  \let\item\@idxitem
}{%
  \clearpage
}
\makeatother

\IfFileExists{\jobname-pw.ind}{\input{\jobname-pw.ind}}{}

\end{document}

      