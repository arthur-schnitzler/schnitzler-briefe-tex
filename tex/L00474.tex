%% latex-korrekturansicht-vorspann.tex
%% Vorspann für die Korrekturansicht.
%% Lädt die gemeinsame Datei latex-vorspann.tex mit gesetztem Schalter.

\newif\ifkorrekturansicht
\korrekturansichttrue

\input{../tex-inputs/latex-vorspann}


               \section[Arthur Schnitzler an Hugo von Hofmannsthal, 17. 8. 1895]{ Arthur Schnitzler an Hugo von Hofmannsthal, 17. 8. 1895}\nopagebreak\mylabel{v}\rehead{ }\normalsize\beginnumbering\briefempfaengerindex{Hofmannsthal, Hugo von@\textsc{Hofmannsthal, Hugo von}!zzzSchnitzler, Arthur@\emph{von Arthur Schnitzler}!1895-08-171@{17. 8. 1895}|(be} \toendnotes[C]{\smallbreak\pagebreak[2]} \Standort{FDH, Hs-30885,45.}
\physDesc{Brief, 1 Blatt, 4 Seiten
\newline{}Handschrift: schwarze Tinte, deutsche Kurrent}\buchAbdrucke{\weitereDrucke{Hugo von Hofmannsthal, Arthur Schnitzler: \emph{Briefwechsel}. Hg. Therese Nickl und Heinrich Schnitzler. Frankfurt am Main: \emph{S. Fischer} 1964, S. 59–60.} }\toendnotes[C]{\smallbreak}\pstart
           \raggedleft{}{\pb}\textcolor{pink}{\textsc{Ischl}}{}\ledrightnote{\textcolor{pink}{Bad Ischl}}, \uline{17. 8. 95.}\pend
           \pstart
           Mein Lieber Hugo, Ihren Brief habe ich beim Zurückko{\geminationm}en aus \textcolor{pink}{Wien}{}\ledrightnote{\textcolor{pink}{Wien}}
                    gefunden. Dort bin ich 2 Tage geweſen und habe die Marionetten in \textcolor{pink}{\textsc{Venedig}}{}\ledrightnote{\textcolor{pink}{Venedig in Wien}} u \textcolor{green}{\textsc{Hänsel u Grethel}}{}\ledrightnote{\textcolor{green}{Hänsel und Grethel}} geſehen. An einzelne von dieſen Marionetten denke ich zurück wie an
                    lebendige Schauſpieler, die ſich auch an mich erinnern müſſen. Im übrigen iſt
                        \textcolor{pink}{Wien}{}\ledrightnote{\textcolor{pink}{Wien}} jetzt dumpf und übelriechend und es
                    iſt gut, daſs ich wieder weg konnte. In \textcolor{pink}{Iſchl}{}\ledrightnote{\textcolor{pink}{Bad Ischl}} bleib ich nur noch bis Montag. Dann fahr ich per Rad nach
                        \textcolor{pink}{Salzburg}{}\ledrightnote{\textcolor{pink}{Salzburg}}, mit \textcolor{blue}{Salten}{}\ledrightnote{\textcolor{blue}{Felix Salten}}. {\pb}Auch \textcolor{blue}{Richard}{}\ledrightnote{\textcolor{blue}{Richard Beer-Hofmann}}, dem ich Ihre Kränkung beſtellt habe, ko{\geminationm}t wohl hin, und die Frau \textcolor{blue}{Lou}{}\ledrightnote{\textcolor{blue}{Lou Andreas-Salomé}} wird ſchon dort ſein. Wenn Sie mir gleich zwei Zeilen
                    ſchreiben, ſo kann ich ſie mir noch in \textcolor{pink}{Salzburg}{}\ledrightnote{\textcolor{pink}{Salzburg}}{ }\textsc{post restante} abholen u hätte eine große Freude.
                        Donnerſtag radle ich nämlich weiter, auf einem bisher noch
                    nicht definitiv feſtgeſtellten Weg nach \textcolor{pink}{\textsc{München}}{}\ledrightnote{\textcolor{pink}{München}}, wo das Rendezvous mit
                        \textcolor{blue}{Goldma{\geminationn}}{}\ledrightnote{\textcolor{blue}{Paul Goldmann}} iſt. In \textcolor{pink}{M.}{}\ledrightnote{\textcolor{pink}{München}} bin ich mindeſtens
                    bis 3. September (Briefe dahin auch \textsc{post
                        restante}. Aber ich {\pb}werd Ihnen von
                    meiner Radtour noch öfters ein paar Worte ſchreiben)\pend
           \pstart
           – Ich hab hier den erſten \textcolor{green}{Akt}{}\ledrightnote{→\textcolor{green}{Freiwild. Schauspiel in 3 Akten}} zu Ende
                    geſchrieben, und ein paar kleine \textcolor{green}{Geſchichten}{}\ledrightnote{→\textcolor{green}{Die Frau des Weisen. Erzählung}{\newline}→\textcolor{green}{Ein Abschied}}, an denen mir vielleicht ſchon manches gelungen iſt. Sie
                    wiſſen ja, meine große Sehnſucht: die ſehr einfache Geſchichte, die in ſich
                    ſelbſt ganz fertig iſt. Eine Flaſche, die man ausgießt, ohne daſs es
                    nachtröpfeln darf und ohne daſs was zurückbleibt. – Auch geht es mir heuer
                    innerlich gut – es gelingt mir faſt jedesmal kleine Eitelkeiten und große {\pb}Hypochondrien davon zujagen, wenn ſie ſich
                    melden wollen. Im ganzen fühl ich mich in dieſem Jahre um fünf Jahre jünger als
                    im vorigen, was darin begründet iſt, daſs ich in weniger falſchen Verhältniſſen
                    lebe als damals. Was Sie einmal von der Seele, die i{\geminationm}er eine kindliche bleibt, ſagten, fällt mir ein. Es mag ſein, daſs Altwerden
                    wirklich nur eine Schwäche iſt, von der man ſich befreien kann{\dotsfour}{ }ſolang man eben doch eigentlich nur 33 Jahre alt
                    iſt.\pend
           \pstart
           Leben Sie wohl, ſeien Sie herzlich gegrüßt. Und ſchreiben Sie eine Zeile nach
                        \textcolor{pink}{Salzb.}{}\ledrightnote{\textcolor{pink}{Salzburg}}\pend
           \pstart Ihr \spacefill\mbox{Arthur}\pend{}\pstart
           \noindent{}\label{T_L00474_1v}\edtext{Ich habe an \textcolor{blue}{Goldm.}{}\ledrightnote{\textcolor{blue}{Paul Goldmann}} wegen \textcolor{blue}{Mamroth}{}\ledrightnote{\textcolor{blue}{Fedor Mamroth}} geſchrieben.}{\lemma{\textnormal{\emph{Ich … geſchrieben.}}}\Cendnote{\textnormal{Das Postscript befindet sich neben der Ortsangabe auf
                            der ersten Seite auf dem Kopf.}}}\label{T_L00474_1h}\pend
           \endnumbering\briefempfaengerindex{Hofmannsthal, Hugo von@\textsc{Hofmannsthal, Hugo von}!zzzSchnitzler, Arthur@\emph{von Arthur Schnitzler}!1895-08-171@{17. 8. 1895}|)be}\mylabel{h}  \normalsize

\doendnotes{C}
\bigskip
\vfill

\clearpage

\footnotesize

\lohead{\textsc{register}}

% Definiere theindex-Environment komplett neu ohne reledmac
\makeatletter
\renewenvironment{theindex}{%
  \section*{\indexname}%
  \setlength{\parindent}{0pt}%
  \setlength{\parskip}{0pt plus 0.3pt}%
  \let\item\@idxitem
}{%
  \clearpage
}
\makeatother

\IfFileExists{\jobname-pw.ind}{\input{\jobname-pw.ind}}{}

\end{document}

      