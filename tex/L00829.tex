%% latex-korrekturansicht-vorspann.tex
%% Vorspann für die Korrekturansicht.
%% Lädt die gemeinsame Datei latex-vorspann.tex mit gesetztem Schalter.

\newif\ifkorrekturansicht
\korrekturansichttrue

\input{../tex-inputs/latex-vorspann}


               \section[Hugo von Hofmannsthal an Arthur Schnitzler, 3. 8. {[}1898{]}]{ Hugo von Hofmannsthal an Arthur Schnitzler, 3. 8. {[}1898{]}}\nopagebreak\mylabel{v}\rehead{ }\normalsize\beginnumbering\briefempfaengerindex{Schnitzler, Arthur@\textsc{Schnitzler, Arthur}!zzzHofmannsthal, Hugo von@\emph{von Hugo von Hofmannsthal}!1898-08-031@{3. 8. {[}1898{]}}|(be} \toendnotes[C]{\smallbreak\pagebreak[2]} \Standort{CUL, Schnitzler, B 43.}
\physDesc{Brief, 1 Blatt, 4 Seiten
\newline{}Handschrift: schwarze Tinte, deutsche Kurrent
\newline{}Schnitzler: mit Bleistift die Jahreszahl ergänzt: »98« \newline{}Ordnung: mit Bleistift von unbekannter Hand nummeriert:
                                    »119« }\buchAbdrucke{\weitereDrucke{Hugo von Hofmannsthal, Arthur Schnitzler: \emph{Briefwechsel}. Hg. Therese Nickl und Heinrich Schnitzler. Frankfurt am Main: \emph{S. Fischer} 1964, S. 108.} }\toendnotes[C]{\smallbreak}\pstart
           \raggedleft{}{\pb}\textcolor{pink}{Hinterbrühl}{}\ledrightnote{\textcolor{pink}{Hinterbrühl}}{\\}3 VIII.\pend
           \pstart{}mein lieber Arthur\pend\pstart
           ich bin ſehr froh, ſchreiben zu können, daſs es ja nun faſt ſicher zu dem ko{\geminationm}en wird, was wir uns beide gewünſcht haben und
                    woran ich noch in \textcolor{pink}{\textsc{Czortków}}{}\ledrightnote{\textcolor{pink}{Tschortkiw}} nicht ſehr feſt geglaubt habe.\pend
           \pstart
           Bitte ſchreiben Sie mir jetzt {\pb}aber gleich hierher welchen Weg durch die \textcolor{pink}{Schweiz}{}\ledrightnote{\textcolor{pink}{Schweiz}} wir eigentlich vorhaben, damit ichs meinen \textcolor{blue}{Eltern}{}\ledrightnote{→\textcolor{blue}{Hugo August von Hofmannsthal}{\newline}→\textcolor{blue}{Anna von Hofmannsthal}}{ }ſagen kann. Ich hab gar keinen Wunſch als daſs die Tour
                    ungefähr am 20\textsuperscript{\textsc{ten}} in der Gegend von \textcolor{pink}{Chur}{}\ledrightnote{\textcolor{pink}{Chur}} aufhören ſoll
                    von wo man dann leicht über \textcolor{pink}{\textsc{Maloja}}{}\ledrightnote{\textcolor{pink}{Maloja}} oder anders {\pb}in meine \textcolor{pink}{oberitalieniſche}{}\ledrightnote{\textcolor{pink}{Italien}}{ }Seengegend kommt. Dort möchte ich 14–20 Tage an einem Ort ruhig bleiben.
                    Wunderſchön wäre es natürlich wenn Sie mit mir bleiben könnten, wir die
                    Mahlzeiten und Abende und hie und da einen Unterbrechungstag {\pb}zuſa{\geminationm}en verbrächten.\pend
           \pstart
           Ich denke am vormittag des 11\textsuperscript{\textsc{ten}} in \textcolor{pink}{Innsbruck}{}\ledrightnote{\textcolor{pink}{Innsbruck}} zu ſein, höchſtens etwa um
                        \uline{einen} Tag ſpäter. Bitte antworten Sie auf
                    dieſen Brief recht ſchnell, ob Ihnen alles recht iſt.\pend
           \pstart
           Von Herzen Ihr{\\[\baselineskip]}\spacefill\mbox{Hugo.}\pend
           \leftskip=0em{}\endnumbering\briefempfaengerindex{Schnitzler, Arthur@\textsc{Schnitzler, Arthur}!zzzHofmannsthal, Hugo von@\emph{von Hugo von Hofmannsthal}!1898-08-031@{3. 8. {[}1898{]}}|)be}\mylabel{h}  \normalsize

\doendnotes{C}
\bigskip
\vfill

\clearpage

\footnotesize

\lohead{\textsc{register}}

% Definiere theindex-Environment komplett neu ohne reledmac
\makeatletter
\renewenvironment{theindex}{%
  \section*{\indexname}%
  \setlength{\parindent}{0pt}%
  \setlength{\parskip}{0pt plus 0.3pt}%
  \let\item\@idxitem
}{%
  \clearpage
}
\makeatother

\IfFileExists{\jobname-pw.ind}{\input{\jobname-pw.ind}}{}

\end{document}

      