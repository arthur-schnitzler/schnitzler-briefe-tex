%% latex-korrekturansicht-vorspann.tex
%% Vorspann für die Korrekturansicht.
%% Lädt die gemeinsame Datei latex-vorspann.tex mit gesetztem Schalter.

\newif\ifkorrekturansicht
\korrekturansichttrue

\input{../tex-inputs/latex-vorspann}


               \section[Arthur und Olga Schnitzler an Richard Beer-Hofmann, 30. 7. 1912]{ Arthur und Olga Schnitzler an Richard Beer-Hofmann, 30. 7. 1912}\nopagebreak\mylabel{v}\rehead{ }\normalsize\beginnumbering\briefempfaengerindex{Beer-Hofmann, Richard@\textsc{Beer-Hofmann, Richard}!zzzSchnitzler, Olga@\emph{von Olga Schnitzler}!1912-07-301@{30. 7. 1912}|(be}\briefempfaengerindex{Beer-Hofmann, Richard@\textsc{Beer-Hofmann, Richard}!zzzSchnitzler, Arthur@\emph{von Arthur Schnitzler}!1912-07-301@{30. 7. 1912}|(be} \toendnotes[C]{\smallbreak\pagebreak[2]} \Standort{YCGL, MSS 31.}
\physDesc{Bildpostkarte
\newline{}Handschrift Arthur Schnitzler: Bleistift, deutsche Kurrent\newline{}Handschrift Olga Schnitzler: Bleistift, lateinische Kurrent\newline{}Versand: 1) Stempel: »\nobreak{}\oindex{Brijuni@\textbf{Brijuni}, \emph{https://www.geonames.org/ontologyP.PPL}|pwk}Brioni, 30. 7. \textcolor{gray}{12}\nobreak{}«.  2) Stempel: »\nobreak{}\oindex{Sankt Moritz@\textbf{Sankt Moritz}, \emph{https://www.geonames.org/ontologyP.PPL}|pwk}St. Moritz-\textcolor{gray}{Dorf}, 1. \textcolor{gray}{VIII. 12}, XII\nobreak{}«. 
\newline{}Beer-Hofmann: mit Bleistift das Datum der Beantwortung vermerkt: »B
                                       2/VIII 12« }\toendnotes[C]{\smallbreak}\pstart{}{\pb}Herrn \textsc{Dr. Richard
                     Beer-Hofmann –}\pend{}\pstart{}aus \textcolor{pink}{Wien}{}\ledrightnote{\textcolor{pink}{Wien}}\pend{}\pstart{}\textsc{\textcolor{pink}{St. Morit\textcolor{gray}{z}}{}\ledrightnote{\textcolor{pink}{Sankt Moritz}}}\pend{}\pstart{}\textsc{im \textcolor{pink}{Engadin}{}\ledrightnote{\textcolor{pink}{Engadin}}}\pend{}\pstart{}\textsc{\textcolor{pink}{Waldhaus}{}\ledrightnote{\textcolor{pink}{Hotel Waldhaus}}}\pend{}{\bigskip}\pstart
           \noindent{}\centering{}{\pb}\textcolor{gray}{\textbf{\textcolor{pink}{Insel Brioni i. d. Adria}{}\ledrightnote{\textcolor{pink}{Brijuni}}.}}\pend
           \pstart
           \noindent{}\centering{}\textcolor{gray}{\textbf{Straße in \textcolor{pink}{Val Madonna}{}\ledrightnote{\textcolor{pink}{Val Madonna}}}}\pend
           \pstart
           {\pb}Herzliche Grüße Ihnen \textcolor{blue}{Allen}{}\ledrightnote{→\textcolor{blue}{Naëmah Beer-Hofmann}{\newline}→\textcolor{blue}{Paula Beer-Hofmann}{\newline}→\textcolor{blue}{Gabriel Beer-Hofmann}{\newline}→\textcolor{blue}{Mirjam Beer-Hofmann}}. – Es iſt wunderſchön
               hier, wir bleiben noch bis nach 20. 8, fahren da{\geminationn} direct \textcolor{pink}{\textsc{Sils Maria}}{}\ledrightnote{\textcolor{pink}{Sils im Engadin}}. Schreiben Sie ein Wort, wies Ihnen \textcolor{blue}{allen}{}\ledrightnote{→\textcolor{blue}{Naëmah Beer-Hofmann}{\newline}→\textcolor{blue}{Paula Beer-Hofmann}{\newline}→\textcolor{blue}{Gabriel Beer-Hofmann}{\newline}→\textcolor{blue}{Mirjam Beer-Hofmann}} geht.\pend
           \pstart
           Ihr{\\[\baselineskip]}\spacefill\mbox{Arthur}\pend
           \leftskip=0em{}\pstart
           \noindent{}{[}hs. O. Schnitzler:{]} Herzl. Grüsse!\pend
           \pstart \spacefill\mbox{Olga.}\pend{}\endnumbering\briefempfaengerindex{Beer-Hofmann, Richard@\textsc{Beer-Hofmann, Richard}!zzzSchnitzler, Olga@\emph{von Olga Schnitzler}!1912-07-301@{30. 7. 1912}|)be}\briefempfaengerindex{Beer-Hofmann, Richard@\textsc{Beer-Hofmann, Richard}!zzzSchnitzler, Arthur@\emph{von Arthur Schnitzler}!1912-07-301@{30. 7. 1912}|)be}\mylabel{h}  \normalsize

\doendnotes{C}
\bigskip
\vfill

\clearpage

\footnotesize

\lohead{\textsc{register}}

% Definiere theindex-Environment komplett neu ohne reledmac
\makeatletter
\renewenvironment{theindex}{%
  \section*{\indexname}%
  \setlength{\parindent}{0pt}%
  \setlength{\parskip}{0pt plus 0.3pt}%
  \let\item\@idxitem
}{%
  \clearpage
}
\makeatother

\IfFileExists{\jobname-pw.ind}{\input{\jobname-pw.ind}}{}

\end{document}

      