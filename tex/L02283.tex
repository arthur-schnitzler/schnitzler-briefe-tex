%% latex-korrekturansicht-vorspann.tex
%% Vorspann für die Korrekturansicht.
%% Lädt die gemeinsame Datei latex-vorspann.tex mit gesetztem Schalter.

\newif\ifkorrekturansicht
\korrekturansichttrue

\input{../tex-inputs/latex-vorspann}


               \section[Arthur Schnitzler an Richard Beer-Hofmann, 12. 4. 1918]{ Arthur Schnitzler an Richard Beer-Hofmann, 12. 4. 1918}\nopagebreak\mylabel{v}\rehead{ }\normalsize\beginnumbering\briefempfaengerindex{Beer-Hofmann, Richard@\textsc{Beer-Hofmann, Richard}!zzzSchnitzler, Arthur@\emph{von Arthur Schnitzler}!1918-04-121@{12. 4. 1918}|(be} \toendnotes[C]{\smallbreak\pagebreak[2]} \Standort{YCGL, MSS 31.}
\physDesc{Brief, 1 Blatt, 1 Seite, Umschlag
\newline{}Handschrift: Bleistift, lateinische Kurrent\newline{}Versand: Stempel: »\nobreak{}\textcolor{gray}{1}2. I\textcolor{gray}{V}. 1\textcolor{gray}{8}\nobreak{}«.  
\newline{}Beer-Hofmann: mit blauem Buntstift den Erhalt markiert: »{\pb}E« }\buchAbdrucke{\weitereDrucke{Arthur Schnitzler, Richard Beer-Hofmann: \emph{Briefwechsel 1891–1931}. Hg. Konstanze Fliedl. Wien, Zürich: \emph{Europaverlag} 1992, S. 225.} }\toendnotes[C]{\smallbreak}\pstart{}{\pb}Herrn Dr. Richard Beer-Hofma\textcolor{gray}{nn}\pend{}\pstart{}\textcolor{pink}{Wien XVIII}{}\ledrightnote{\textcolor{pink}{XVIII., Währing}}\pend{}\pstart{}\textcolor{pink}{Hasenauerstraße 59}{}\ledrightnote{\textcolor{pink}{Hasenauerstraße}}\pend{}{\bigskip}\pstart
           \centering{}{\pb}\textcolor{pink}{Wien}{}\ledrightnote{\textcolor{pink}{Wien}}, 12. 4. 18\pend
           \pstart
           mein lieber Richard, Sie sind wieder zu Hause und ich höre daß es
                  \label{KLL02283_Beer-Hofmann-1v}\edtext{viel besser}{\lemma{\textnormal{\emph{viel besser}}}\Cendnote{\textnormal{\textcolor{blue}{Gabriel Beer-Hofmann} hatte wegen einer
                  schlechten Schulnote am 20. 3. 1918 versucht, sich umzubringen. Vgl. A. S.: \emph{Tagebuch}, 24. 3. 1918}}}\label{KLL02283_Beer-Hofmann-1h} geht, jedenfalls so gut daß keinerlei Grund mehr zu irgend einer
               Beunruhigung vorliegt. Ich will Sie weder durch einen telefonischen Anruf, noch gar
               durch einen Besuch stören und bitte Sie nur mich auf irgend eine Weise wissen zu
               lassen, wa{\geminationn} Sie die Zeit für ein Wiedersehen,
               Wiedersprechen gekommen erachten. Für heute nur so viel daß wir in diesen schweren
               Tagen mit all den herzlichen Gefühlen bei Ihnen und \textcolor{blue}{Paula}{}\ledrightnote{\textcolor{blue}{Paula Beer-Hofmann}} waren, die Sie kennen und sehr froh sind den \textcolor{blue}{Buben}{}\ledrightnote{→\textcolor{blue}{Gabriel Beer-Hofmann}} auf dem Wege rascher Besserung zu
               wissen. Und so hoff ich, sind Sie auch sich selber bald gänzlich zurückgegeben! Seien
               Sie mit \textcolor{blue}{Paula}{}\ledrightnote{\textcolor{blue}{Paula Beer-Hofmann}} und den \textcolor{blue}{Kindern}{}\ledrightnote{→\textcolor{blue}{Naëmah Beer-Hofmann}{\newline}→\textcolor{blue}{Mirjam Beer-Hofmann}{\newline}→\textcolor{blue}{Gabriel Beer-Hofmann}} von \textcolor{blue}{Olga}{}\ledrightnote{\textcolor{blue}{Olga Schnitzler}} und mir viele Male und von Herzen gegrüßt\pend
           \pstart
           Ihr{\\[\baselineskip]}\spacefill\mbox{Arthur}\pend
           \leftskip=0em{}\endnumbering\briefempfaengerindex{Beer-Hofmann, Richard@\textsc{Beer-Hofmann, Richard}!zzzSchnitzler, Arthur@\emph{von Arthur Schnitzler}!1918-04-121@{12. 4. 1918}|)be}\mylabel{h}  \normalsize

\doendnotes{C}
\bigskip
\vfill

\clearpage

\footnotesize

\lohead{\textsc{register}}

% Definiere theindex-Environment komplett neu ohne reledmac
\makeatletter
\renewenvironment{theindex}{%
  \section*{\indexname}%
  \setlength{\parindent}{0pt}%
  \setlength{\parskip}{0pt plus 0.3pt}%
  \let\item\@idxitem
}{%
  \clearpage
}
\makeatother

\IfFileExists{\jobname-pw.ind}{\input{\jobname-pw.ind}}{}

\end{document}

      