%% latex-korrekturansicht-vorspann.tex
%% Vorspann für die Korrekturansicht.
%% Lädt die gemeinsame Datei latex-vorspann.tex mit gesetztem Schalter.

\newif\ifkorrekturansicht
\korrekturansichttrue

\input{../tex-inputs/latex-vorspann}


               \section[Arthur Schnitzler an Hugo von Hofmannsthal, 15. 7. 1897]{ Arthur Schnitzler an Hugo von Hofmannsthal, 15. 7. 1897}\nopagebreak\mylabel{v}\rehead{ }\normalsize\beginnumbering\briefempfaengerindex{Hofmannsthal, Hugo von@\textsc{Hofmannsthal, Hugo von}!zzzSchnitzler, Arthur@\emph{von Arthur Schnitzler}!1897-07-151@{15. 7. 1897}|(be} \toendnotes[C]{\smallbreak\pagebreak[2]} \Standort{FDH, Hs-30885,61.}
\physDesc{Brief, 1 Blatt, 4 Seiten
\newline{}Handschrift: Bleistift, deutsche Kurrent\newline{}Ordnung: mit Bleistift von Schnitzler mutmaßlich bei der
                                 Durchsicht der Korrespondenz 1929 das erste Blatt datiert:
                                    »15/7 97« }\buchAbdrucke{\weitereDrucke{Hugo von Hofmannsthal, Arthur Schnitzler: \emph{Briefwechsel}. Hg. Therese Nickl und Heinrich Schnitzler. Frankfurt am Main: \emph{S. Fischer} 1964, S. 91–92.} }\toendnotes[C]{\smallbreak}\pstart
           \noindent{}{\pb}Mein lieber Hugo, ich ka{\geminationn} keineswegs Anfang Auguſt mit Ihnen
                        zusa{\geminationm}entreffen – \label{K_L00702_1v}\edtext{Sie wiſſen ja}{\lemma{\textnormal{\emph{Sie wiſſen ja}}}\Cendnote{\textnormal{Seine Partnerin
                            \textcolor{blue}{Marie Reinhard} war schwanger. Das \textcolor{blue}{Kind} kam tot zur Welt.}}}\label{K_L00702_1h}. Dagegen
                    unterbreiten \textcolor{blue}{Richard}{}\ledrightnote{\textcolor{blue}{Richard Beer-Hofmann}} u ich Ihnen einen
                    andern Vorschlag. Wir wollen Ihnen weiter, \textsc{resp}. näher
                    entgegen. Ich möchte z. B. Freitag den 23. von hier
                    fort, nach \textcolor{pink}{Salzburg}{}\ledrightnote{\textcolor{pink}{Salzburg}}, da{\geminationn}{ }\textsc{per} Rad (we{\geminationn} ſich meines
                    bis dahin erholt hat und {\pb}\textcolor{blue}{Richard}{}\ledrightnote{\textcolor{blue}{Richard Beer-Hofmann}} nicht faul iſt) über \textcolor{pink}{Reichenhall}{}\ledrightnote{\textcolor{pink}{Bad Reichenhall}}, \textcolor{pink}{\textsc{Lofer}}{}\ledrightnote{\textcolor{pink}{Lofer}} nach \textcolor{pink}{\textsc{Zell am See}}{}\ledrightnote{\textcolor{pink}{Zell am See}}. Ich \textsc{resp}. wir würden Samſtag{ }Früh in \textcolor{pink}{Zell am See}{}\ledrightnote{\textcolor{pink}{Zell am See}}{ }{[}ſ{]}ein, dort verbringen wir
                    den Tag miteinander. Und Abend führe ich nach \textcolor{pink}{Wien}{}\ledrightnote{\textcolor{pink}{Wien}}. – Es handelt ſich alſo darum, ob Sie auf einen Tag von der \textcolor{pink}{\textsc{Fusch}}{}\ledrightnote{\textcolor{pink}{Bad Fusch}} wegkönnen. We{\geminationn}{ }\textcolor{blue}{Andrian}{}\ledrightnote{\textcolor{blue}{Leopold von Andrian-Werburg}}{ }{\pb}mit Ihnen fahren wollte, ſo käme er mit. Grüßen Sie
                    ihn herzlich von mir; es geht ihm hoffentlich wieder beſſer.\pend
           \pstart
           \textcolor{blue}{Jahn}{}\ledrightnote{\textcolor{blue}{Otto Jahn}}{ }\textcolor{green}{2. Band}{}\ledrightnote{→\textcolor{green}{W. A. Mozart}} beko{\geminationm}en? –\pend
           \pstart
           – Auf einen ſchönen So{\geminationm}ertag mit Ihnen, we{\geminationn}’s ſchon nicht mehr ſein können, möcht ich nicht
                    gern verzichten. Aber Sie ſollen ſich auch nicht die geringſte {\pb}Ungelegenheit machen.\pend
           \pstart Herzlich Ihr\spacefill\mbox{Arthur}\pend{}\pstart
           \textsc{\textcolor{pink}{Ischl}{}\ledrightnote{\textcolor{pink}{Bad Ischl}}}{ }15. 7. 97\pend
           \endnumbering\briefempfaengerindex{Hofmannsthal, Hugo von@\textsc{Hofmannsthal, Hugo von}!zzzSchnitzler, Arthur@\emph{von Arthur Schnitzler}!1897-07-151@{15. 7. 1897}|)be}\mylabel{h}  \normalsize

\doendnotes{C}
\bigskip
\vfill

\clearpage

\footnotesize

\lohead{\textsc{register}}

% Definiere theindex-Environment komplett neu ohne reledmac
\makeatletter
\renewenvironment{theindex}{%
  \section*{\indexname}%
  \setlength{\parindent}{0pt}%
  \setlength{\parskip}{0pt plus 0.3pt}%
  \let\item\@idxitem
}{%
  \clearpage
}
\makeatother

\IfFileExists{\jobname-pw.ind}{\input{\jobname-pw.ind}}{}

\end{document}

      