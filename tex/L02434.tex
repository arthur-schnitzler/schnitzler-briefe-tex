%% latex-korrekturansicht-vorspann.tex
%% Vorspann für die Korrekturansicht.
%% Lädt die gemeinsame Datei latex-vorspann.tex mit gesetztem Schalter.

\newif\ifkorrekturansicht
\korrekturansichttrue

\input{../tex-inputs/latex-vorspann}


               \section[Arthur Schnitzler an Georg Brandes, 22. 2. 1925]{ Arthur Schnitzler an Georg Brandes, 22. 2. 1925}\nopagebreak\mylabel{v}\rehead{ }\normalsize\beginnumbering\briefempfaengerindex{Brandes, Georg@\textsc{Brandes, Georg}!zzzSchnitzler, Arthur@\emph{von Arthur Schnitzler}!1925-02-221@{22. 2. 1925}|(be} \toendnotes[C]{\smallbreak\pagebreak[2]} \Standort{Kopenhagen, Det Kongelige Bibliotek, Georg Brandes Arkiv, box 125.}
\physDesc{Postkarte
\newline{}Handschrift: schwarze Tinte, lateinische Kurrent\newline{}Versand: 1) Stempel: »\nobreak{}\oindex{XVIII., Waehring@\textbf{XVIII., Währing}, \emph{Bezirk (A.BZK)}|pwk}18 Wien 110, 23. II. 2\textcolor{gray}{5}, 9\nobreak{}«.  2) Stempel: »\nobreak{}\oindex{Kopenhagen@\textbf{Kopenhagen}, \emph{Besiedelter Ort (A.BSO)}|pwk}Kjøbenhavn, 25. 2. 25, 10–11\nobreak{}«. \newline{}Ordnung: mit Bleistift von unbekannter Hand beschriftet »Schnit« und
                                    nummeriert: »51.« }\buchAbdrucke{\weitereDrucke{Georg Brandes, Arthur Schnitzler: \emph{Ein Briefwechsel}. Hg. Kurt Bergel. Bern: \emph{Francke} 1956, S. 144–145.} }\toendnotes[C]{\smallbreak}\pstart{}{\pb}\label{T_L02434-1v}\edtext{\textcolor{gray}{\textbf{A. S.}}}{\lemma{\textnormal{\emph{A. S.}}}\Cendnote{\textnormal{ovaler Absenderkleber}}}\label{T_L02434-1h}\pend{}\pstart{}\textcolor{pink}{\textcolor{gray}{\textbf{WIEN, XVIII.}}}{}\ledrightnote{\textcolor{pink}{XVIII., Währing}}\pend{}\pstart{}\textcolor{pink}{\textcolor{gray}{\textbf{STERNWARTESTR. 71}}}{}\ledrightnote{\textcolor{pink}{Sternwartestraße}}\pend{}{\bigskip}\pstart{}Herrn Prof. Georg Brandes\pend{}\pstart{}\textcolor{pink}{Kopenhagen}{}\ledrightnote{\textcolor{pink}{Kopenhagen}}.
                    \pend{}{\bigskip}\pstart
           \raggedleft{}{\pb}\textcolor{pink}{Wien}{}\ledrightnote{\textcolor{pink}{Wien}}, 22. 2. 25\pend
           \pstart
           Verehrter lieber Freund, hoffentlich ka{\geminationn} ichs so einrichten, dſs ich zur Zeit Ihrer
                    Ankunft in \textcolor{pink}{Berlin}{}\ledrightnote{\textcolor{pink}{Berlin}} noch dort bin – meine dortige
                    Anwesenheit war, aus verschiedenen Gründen für früher projectirt. Vielleicht ist
                    Frau \textcolor{blue}{Rung}{}\ledrightnote{\textcolor{blue}{Gertrud Rung}}, der ich mich bestens empfehle auch
                    noch so gütig, mir mitzutheilen, \uline{an welchem Tage}
                    Sie schon in \textcolor{pink}{Berlin}{}\ledrightnote{\textcolor{pink}{Berlin}}{ }\uline{eintreffen}. Sie sind vielleicht schon vor dem
                        25\substVorne{}\textsuperscript{.}\substDazwischen{}/3\substHinten{} dort?\pend
           \pstart
           In treuer und herzlicher Verehrung{\\[\baselineskip]}Ihr \spacefill\mbox{Arthur Schnitzler}\pend
           \leftskip=0em{}\endnumbering\briefempfaengerindex{Brandes, Georg@\textsc{Brandes, Georg}!zzzSchnitzler, Arthur@\emph{von Arthur Schnitzler}!1925-02-221@{22. 2. 1925}|)be}\mylabel{h}  \normalsize

\doendnotes{C}
\bigskip
\vfill

\clearpage

\footnotesize

\lohead{\textsc{register}}

% Definiere theindex-Environment komplett neu ohne reledmac
\makeatletter
\renewenvironment{theindex}{%
  \section*{\indexname}%
  \setlength{\parindent}{0pt}%
  \setlength{\parskip}{0pt plus 0.3pt}%
  \let\item\@idxitem
}{%
  \clearpage
}
\makeatother

\IfFileExists{\jobname-pw.ind}{\input{\jobname-pw.ind}}{}

\end{document}

      