%% latex-korrekturansicht-vorspann.tex
%% Vorspann für die Korrekturansicht.
%% Lädt die gemeinsame Datei latex-vorspann.tex mit gesetztem Schalter.

\newif\ifkorrekturansicht
\korrekturansichttrue

\input{../tex-inputs/latex-vorspann}


               \section[Albert Ehrenstein an Arthur Schnitzler, 17. 10. 1909]{ Albert Ehrenstein an Arthur Schnitzler, 17. 10. 1909}\nopagebreak\mylabel{v}\rehead{ }\normalsize\beginnumbering\briefempfaengerindex{Schnitzler, Arthur@\textsc{Schnitzler, Arthur}!zzzEhrenstein, Albert@\emph{von Albert Ehrenstein}!1909-10-171@{17. 10. 1909}|(be} \toendnotes[C]{\smallbreak\pagebreak[2]} \Standort{CUL, Schnitzler, B 30.}
\physDesc{Brief, 1 Blatt, 4 Seiten
\newline{}Handschrift: schwarze Tinte, deutsche Kurrent
\newline{}Schnitzler: mit Bleistift beschriftet: »\textsc{Ehrenst\textcolor{gray}{ein}}« }\buchAbdrucke{\weitereDrucke{Albert Ehrenstein: \emph{Briefe}. Hg. Hanni Mittelmann. München: \emph{Boer} 1989, S. 32–33 (Werke, 1).} }\toendnotes[C]{\smallbreak}\pstart
           {\pb}\textsc{Albert Ehrenstein}\hfill 17. Okt. 09. \pend
           \pstart
           \textcolor{pink}{\textsc{XVI. Ottakringerstr} 114}{}\ledrightnote{\textcolor{pink}{Ottakringerstraße}}.\pend
           \pstart{}\textsc{Sehr geehrter Herr Doktor,}\pend\pstart
           nachdem meine nach \textcolor{pink}{Venedig}{}\ledrightnote{\textcolor{pink}{Venedig}} geſandten Manuſkripte
                    einen Monat lang verſchollen und ich, da ſie auf 1000 K verſichert waren,
                    bereits geträumt hatte, in den Beſitz dieſer Unſumme zu gelangen, geſchah es
                    mir, daß ſie ſich doch noch vorfanden und einige Zeit nachher feierte ich denn
                    auch ein halb gerührtes, halb ärgerliches Wiederſehen mit meinen Arbeiten. Um
                    auch andere an meinen Gefühlen teilnehmen zu laſſen, transportierte ich einiges
                    zu Herrn \textcolor{blue}{Auernheimer}{}\ledrightnote{\textcolor{blue}{Raoul Auernheimer}}, den ich nicht antraf.
                    Weil mir die Angelegenheit {\pb}damals noch dringend
                    ſchien, machte ich mich 14 Tage darauf wieder auf den Weg in die \textcolor{pink}{Neulinggaſſe}{}\ledrightnote{\textcolor{pink}{Neulinggasse}}. Da nun ergab es ſich, daß \textcolor{blue}{A.}{}\ledrightnote{\textcolor{blue}{Raoul Auernheimer}} bis dahin jede Berührung mit meinen Operaten ängſtlich
                    vermieden hatte und auch bis Mittwoch, als ich beſtelltermaßen zu
                    ihm kam, hatte er noch nicht jenen Heroismus aufgebracht, der zur reſtloſen
                    Bewältigung mir entſtammender ſchriftſtelleriſcher Gebilde leider unbedingt
                    nötig ſein dürfte. Nichtsdeſtoweniger und obwohl er nur in kleineren und
                    keineswegs in den für die \textcolor{brown}{Preſſe}{}\ledrightnote{\textcolor{brown}{Neue Freie Presse}} beſtimmten
                    Erzählungen geblättert hatte, kam er ſpielend zu einem erſchöpfenden Urteil über
                    mich. Er nannte mich ein unreifes Talent, phantaſtiſch nach \textcolor{blue}{Meyrink}{}\ledrightnote{\textcolor{blue}{Gustav Meyrink}}s Art, meine Sachen ungeeignet zur Publikation,
                    möglich höchſtens für den »\textcolor{brown}{Hyperion}{}\ledrightnote{\textcolor{brown}{Hyperion}}« oder »\textcolor{brown}{Spiegel}{}\ledrightnote{\textcolor{brown}{Der Spiegel. Münchner Halbmonatsschrift}}« – Zeitſchriften {\pb}übrigens, die mein profanes Auge niemals ſchaute und
                    von denen ich bloß weiß, daß ſie im Lande \textcolor{blue}{Blei}{}\ledrightnote{\textcolor{blue}{Franz Blei}} liegen. Seine Rede krönte er mit einem anſcheinend unſchuldigen
                    Satz, dem vortrefflich gewählten \textsc{ceterum censeo}: »Was
                    wollen ſie eigentlich? Falls bei ihnen einmal mehr als Anſätze, nämlich
                    Erfüllungen vorhanden ſein ſollten, wird ſie Schnitzler an die \textcolor{brown}{Neue Rundſchau}{}\ledrightnote{\textcolor{brown}{Neue Rundschau, Neue Deutsche Rundschau, Freie Bühne}} empfehlen und das wird viel mehr ſein als
                    wenn ſie in ſo einem Literatenblättchen gedruckt würden.« Schließlich verſtand
                    er ſich dazu, mir die Zuſendung von Recenſionsexemplaren zu verſprechen, womit
                    die ganze Affaire für mich abgetan ſein wird. Mehr brauche ich nämlich
                    glücklicherweiſe von der \textcolor{brown}{Preſſe}{}\ledrightnote{\textcolor{brown}{Neue Freie Presse}} nicht und wenn
                    ich früher erfahren hätte, was ich leider erſt Donnerstag erfuhr,
                    daß nämlich an der Verzögerung der Approbation {\pb}meiner
                        \textcolor{green}{Diſſertation}{}\ledrightnote{→\textcolor{green}{Die Lage in Ungarn (Siebenbürgen und Serbien ausgenommen) im Jahre 1790}} nicht ſo
                    ſehr Übelwollen als Schlamperei die Schuld trug, dann hätte ich Ihnen, ſehr
                    geehrter Herr Doktor, und mir allerhand erſparen können{\dotsfour} Allerdings ſehne ich mich noch immer danach, nicht etwa einer Zelle
                    in jener papierenen Welt, ſondern eines Platzes an der Sonne teilhaftig zu
                    werden, um endlich zu einigem Genuße meines Lebens zu gelangen. Meine
                    Perſonalkenntniſſe der \textcolor{pink}{Wien}{}\ledrightnote{\textcolor{pink}{Wien}}er Journaliſtik
                    wünſche ich dennoch nicht zu bereichern, ich möchte vielmehr äußerſt gern aus
                    Ihrem Munde vernehmen, ob der in »\textcolor{green}{Baber}{}\ledrightnote{\textcolor{green}{Tod des Zehir eddin Muhammed Baber}}« und
                        »\textcolor{green}{Apaturien}{}\ledrightnote{\textcolor{green}{Apaturien}}« gezeigte Stil für mich und
                    andere von Wert iſt und ob eine Veröffentlichung oder Edition der beſſeren
                    meiner Skizzen und Erzählungen einen materiellen Effekt haben könnte? Soll ich
                    ſchon jetzt daran gehen, meine Sammlung redaktioneller Kundgebungen durch
                    Angliederung ähnlich negativer Beſcheide von Verlegern gebührend auszubauen?
                    Vielleicht können Sie, ſehr geehrter Herr Doktor, raten\pend
           \pstart
           Ihrem ergebenſten{\\[\baselineskip]}\spacefill\mbox{Albert Ehrenstein.}\pend
           \leftskip=0em{}\endnumbering\briefempfaengerindex{Schnitzler, Arthur@\textsc{Schnitzler, Arthur}!zzzEhrenstein, Albert@\emph{von Albert Ehrenstein}!1909-10-171@{17. 10. 1909}|)be}\mylabel{h}  \normalsize

\doendnotes{C}
\bigskip
\vfill

\clearpage

\footnotesize

\lohead{\textsc{register}}

% Definiere theindex-Environment komplett neu ohne reledmac
\makeatletter
\renewenvironment{theindex}{%
  \section*{\indexname}%
  \setlength{\parindent}{0pt}%
  \setlength{\parskip}{0pt plus 0.3pt}%
  \let\item\@idxitem
}{%
  \clearpage
}
\makeatother

\IfFileExists{\jobname-pw.ind}{\input{\jobname-pw.ind}}{}

\end{document}

      