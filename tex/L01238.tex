%% latex-korrekturansicht-vorspann.tex
%% Vorspann für die Korrekturansicht.
%% Lädt die gemeinsame Datei latex-vorspann.tex mit gesetztem Schalter.

\newif\ifkorrekturansicht
\korrekturansichttrue

\input{../tex-inputs/latex-vorspann}


               \section[Arthur Schnitzler an Hugo von Hofmannsthal, 7. 10. 1902]{ Arthur Schnitzler an Hugo von Hofmannsthal, 7. 10. 1902}\nopagebreak\mylabel{v}\rehead{ }\normalsize\beginnumbering\briefempfaengerindex{Hofmannsthal, Hugo von@\textsc{Hofmannsthal, Hugo von}!zzzSchnitzler, Arthur@\emph{von Arthur Schnitzler}!1902-10-071@{7. 10. 1902}|(be} \toendnotes[C]{\smallbreak\pagebreak[2]} \Standort{FDH, Hs-30885,98.}
\physDesc{Brief, 2 Blätter, 7 Seiten
\newline{}Handschrift: schwarze Tinte, deutsche Kurrent\newline{}Ordnung: mit Bleistift von Schnitzler (?) mutmaßlich bei der
                                            Durchsicht der Korrespondenz 1929 das zweite Blatt datiert: »7/10 902« }\buchAbdrucke{\weitereDrucke{1) Hugo von Hofmannsthal, Arthur Schnitzler: \emph{Briefwechsel}. Hg. Therese Nickl und Heinrich Schnitzler. Frankfurt am Main: \emph{S. Fischer} 1964, S. 160–161.} \weitereDrucke{2) Hermann Bahr, Arthur Schnitzler: \emph{Briefwechsel, Aufzeichnungen, Dokumente
                                (1891–1931)}. Hg. Kurt Ifkovits und Martin Anton Müller. Göttingen: \emph{Wallstein} 2018, S. 244.} }\toendnotes[C]{\smallbreak}\pstart
           \raggedleft{}{\pb}\textcolor{pink}{Wien}{}\ledrightnote{\textcolor{pink}{Wien}}, 7. X. 902\pend
           \pstart
           mein lieber Hugo,  Ihren Brief hab ich mit meiner Antwort
                    zugleich an \textcolor{blue}{Bahr}{}\ledrightnote{\textcolor{blue}{Hermann Bahr}} geſchickt; habe mich
                    gleichfalls gegen monatliche Verpflichtung verwehrt, mich aber zu gelegentlichen
                    die Monatsrate überſteigenden Beiträgen bereit erklärt. Ich fand den Brief der
                    Frau \textcolor{blue}{D.}{}\ledrightnote{\textcolor{blue}{Paula Dehmel}} von einer bemerkenswerten
                    Taktloſigkeit.\pend
           \pstart
           Leider bin ich nicht mehr dazu geko{\geminationm}en, Sie vor
                    Ihrer Abreiſe zu ſehn; die Um{\pb}zugspräparationen
                    hatten begonnen; nun ſind die \textcolor{blue}{Meinen}{}\ledrightnote{→\textcolor{blue}{Olga Schnitzler}{\newline}→\textcolor{blue}{Heinrich Schnitzler}} natürlich ſchon geraume Zeit herin; nur fehlen
                    leider vorläufig die meiſten Möbel, wie das im \textcolor{pink}{Wien}{}\ledrightnote{\textcolor{pink}{Wien}}er Lieferantenweſen nun einmal nicht anders ſein kann. Aber es
                    genirt nicht beſonders, u ich bin recht froh, daſs wir ſo nah von einander
                    sind.\pend
           \pstart
           Mit dem \textcolor{green}{Stück}{}\ledrightnote{→\textcolor{green}{Der einsame Weg. Schauspiel in fünf Akten}} bin ich etliche
                    Male ſtecken geblieben; heut iſt die Arbeit ſeit längerer Zeit das erſte Mal
                    wieder beſſer gegangen, und ich werde wohl zu {\pb}Ende
                    kommen – wenn auch nicht in dieſem Moment. Ich ſchreibe das Stück nun bis zum
                    Schluſs und halte es ſelbſt \introOben{}nur\introOben{} für eine ſehr
                    ausführliche Skizze. Wenn dann einige Auftritte fertiger ſind als ich geahnt,
                    ſoll es mich angenehm überraſchen. Keinesfalls ſetz ich mir einen Termin. – \textcolor{blue}{Hans}{}\ledrightnote{\textcolor{blue}{Hans Bernhard Schlesinger}} hab ich anläßlich des
                    Leichenbegängniſſes von \textcolor{blue}{Richard}{}\ledrightnote{\textcolor{blue}{Richard Beer-Hofmann}}’s \textcolor{blue}{Vater}{}\ledrightnote{→\textcolor{blue}{Hermann Beer}} geſehen, und habe
                    viel Sympathie für ihn. –\pend
           \pstart
           Anfang nächſter Woche denke ich nach \textcolor{pink}{Berlin}{}\ledrightnote{\textcolor{pink}{Berlin}} zu
                    fahren; für acht Tage etwa. {\pb}\textcolor{blue}{Brahm}{}\ledrightnote{\textcolor{blue}{Otto Brahm}}{ }ſcheint plötzlich von Stücken ſo überſchwemmt
                    zu werden, daſs die liebe \textcolor{green}{\textsc{Beatrice}}{}\ledrightnote{\textcolor{green}{Der Schleier der Beatrice. Schauspiel in fünf Akten}} wieder unter den Tiſch fallen wird. Aber ich denke, unterm Tiſch wird der
                        \textcolor{blue}{\textsc{Loewenfeld}}{}\ledrightnote{\textcolor{blue}{Raphael Löwenfeld}}{ }ſitzen. –\pend
           \pstart
           – Die \textcolor{green}{Leb. St.}{}\ledrightnote{\textcolor{green}{Lebendige Stunden. Vier Einakter}} kommen im März mit
                    der \textcolor{blue}{Sandrock}{}\ledrightnote{\textcolor{blue}{Adele Sandrock}} am \textcolor{pink}{Volksth.}{}\ledrightnote{\textcolor{pink}{Volkstheater}} zur Aufführung. –\pend
           \pstart
           Ich bin ſchon ſehr geſpannt von Ihnen zu hören. Ich verſpreche mir für Sie von
                    dem \textcolor{pink}{römiſchen}{}\ledrightnote{\textcolor{pink}{Rom}} Aufenthalt unendlich viel. Laſſen
                    Sie ſich nur nicht verſtimmen, wenn {\pb}Arbeitsluſt u
                    kraft nicht gleich wieder da ſind. Denken Sie nur was »Production« für ein
                    unfaßbares, unmeßbares und unbegreifliches Ding iſt – wie wir zuweilen ſchaffen,
                    ohne es zu bemerken u ein andres Mal (mir geht es öfters ſo!) in aller
                    Geſchäftigkeit ſo gut wie nichts geleiſtet haben. – Daſs das »Aufgeſchriebene«
                    das einzige ist, was von den Fernerſtehenden controlirt werden ka{\geminationn}, ſollte uns nie verwirren. Für die {\pb}andern werd ich gewiſs nie ein Dichter ſein wie ich
                    es vor 3 Jahren einmal auf einem einſamen Spaziergang von \textsc{\textcolor{pink}{Wiesbaden}{}\ledrightnote{\textcolor{pink}{Wiesbaden}}} nach \textcolor{pink}{\textsc{Biberich}}{}\ledrightnote{\textcolor{pink}{Biebrich}} und heuer im Sommer zehn oder gar zwanzig Minuten auf dem Lichtenstein war – Und das »übrig bleiben« ka{\geminationn} doch wohl kein \textsc{Criterium}{ }ſein. In hundert – oder zehntauſend oder
                    ſiebzigtauſend Jahren iſt gar nichts {\pb}übrig.\pend
           \pstart
           Aber das führt ins allgemeine, und da weht einem die Luft zu kalt um die
                    Ohren.\pend
           \pstart
           Schreiben Sie mir bald. Ich grüße Sie herzlich\hspace*{1.5em}Ihr{\\[\baselineskip]}\spacefill\mbox{A.}\pend
           \leftskip=0em{}\endnumbering\briefempfaengerindex{Hofmannsthal, Hugo von@\textsc{Hofmannsthal, Hugo von}!zzzSchnitzler, Arthur@\emph{von Arthur Schnitzler}!1902-10-071@{7. 10. 1902}|)be}\mylabel{h}  \normalsize

\doendnotes{C}
\bigskip
\vfill

\clearpage

\footnotesize

\lohead{\textsc{register}}

% Definiere theindex-Environment komplett neu ohne reledmac
\makeatletter
\renewenvironment{theindex}{%
  \section*{\indexname}%
  \setlength{\parindent}{0pt}%
  \setlength{\parskip}{0pt plus 0.3pt}%
  \let\item\@idxitem
}{%
  \clearpage
}
\makeatother

\IfFileExists{\jobname-pw.ind}{\input{\jobname-pw.ind}}{}

\end{document}

      