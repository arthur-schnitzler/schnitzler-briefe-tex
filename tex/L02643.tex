%% latex-korrekturansicht-vorspann.tex
%% Vorspann für die Korrekturansicht.
%% Lädt die gemeinsame Datei latex-vorspann.tex mit gesetztem Schalter.

\newif\ifkorrekturansicht
\korrekturansichttrue

\input{../tex-inputs/latex-vorspann}


               \section[Paul Goldmann an Arthur Schnitzler, 6. 8. 1889]{ Paul Goldmann an Arthur Schnitzler, 6. 8. 1889}\nopagebreak\mylabel{v}\rehead{ }\normalsize\beginnumbering\briefempfaengerindex{Schnitzler, Arthur@\textsc{Schnitzler, Arthur}!zzzGoldmann, Paul@\emph{von Paul Goldmann}!1889-08-061@{6. 8. 1889}|(be} \toendnotes[C]{\smallbreak\pagebreak[2]} \Standort{DLA, A:Schnitzler, HS.NZ85.1.3162.}
\physDesc{Brief, 1 Blatt, 2 Seiten
\newline{}Handschrift: schwarze Tinte, deutsche Kurrent}\toendnotes[C]{\smallbreak}\pstart
           \noindent{}\centering{}{\pb}\textcolor{gray}{\textbf{\textbf{Adminiſtration: \textcolor{pink}{VII.
                           Seidengaſſe 7}{}\ledrightnote{\textcolor{pink}{Seidengasse}}} (\textcolor{brown}{Jos. Eberle {\kaufmannsund} Co.}{}\ledrightnote{\textcolor{brown}{Josef Eberle  Stein-, Buch und Musikaliendruckerei}})}}\pend
           \pstart
           \noindent{}\centering{}\textcolor{gray}{\textbf{\textcolor{brown}{An der Schönen Blauen Donau}{}\ledrightnote{\textcolor{brown}{An der schönen blauen Donau}}}}\pend
           \pstart
           \noindent{}\centering{}\textcolor{gray}{\textbf{Chef-Redacteur: Dr. \textcolor{blue}{F.
                        Mamroth}{}\ledrightnote{\textcolor{blue}{Fedor Mamroth}}. – Redaction: \textcolor{pink}{IX.,
                        Berggaſſe 31}{}\ledrightnote{\textcolor{pink}{Berggasse}}.}}\pend
           \pstart
           \raggedleft{}\textcolor{gray}{\textbf{\textcolor{pink}{Wien}{}\ledrightnote{\textcolor{pink}{Wien}}, den}}{ }6. Auguſt \textcolor{gray}{\textbf{18}}89.\pend
           \pstart\center{}Verehrter Herr Doctor!\pend\pstart
           Herzlichſten Dank ſür Ihre ausführlichen Mittheilungen. Ich hoffe, Freitag{ }früh in \label{K_L02643-1v}\edtext{\textsc{\textcolor{pink}{Ischl}{}\ledrightnote{\textcolor{pink}{Bad Ischl}}}}{\lemma{\textnormal{\emph{Ischl}}}\Cendnote{\textnormal{Am 9. 8. 1889 reisten \emph{\textcolor{green}{Goldmann}}, \textcolor{blue}{Schnitzler}
                  und dessen Bruder \textcolor{blue}{Julius Schnitzler} nach
                     \textcolor{pink}{Traunkirchen}. Auf dem Weg dahin,
                  möglicherweise bereits in \textcolor{pink}{Ischl}, trafen sie
                  aufeinander.}}}\label{K_L02643-1h} ſein zu können. Freilich kann mir leicht etwas dazwiſchen
               kommen. Jedenfalls erhalten Sie Donnerſtag ein
               telegraphiſches \label{K_L02643-2v}\edtext{Aviſo}{\lemma{\textnormal{\emph{Aviſo}}}\Cendnote{\textnormal{nicht überliefert}}}\label{K_L02643-2h}.\pend
           \pstart
           Die \label{K_L02643-3v}\edtext{Ausrüſtung}{\lemma{\textnormal{\emph{Ausrüſtung}}}\Cendnote{\textnormal{für die bevorstehende Wanderung}}}\label{K_L02643-3h} beſorge ich mir,
               ſoweit es in der kurzen {\pb}Zeit noch
               möglich iſt. Ein Punkt dürfte auf Schwierigkeiten ſtoßen: \label{K_L02643-4v}\edtext{Sacktücher}{\lemma{\textnormal{\emph{Sacktücher}}}\Cendnote{\textnormal{Taschentücher}}}\label{K_L02643-4h}! Wo ſoll man die in \textcolor{pink}{Wien}{}\ledrightnote{\textcolor{pink}{Wien}}
                  herbekommen!{\dots}\pend
           \pstart
           Herzlichen Gruß dem Dr. \textsc{\textcolor{blue}{Spitzer}{}\ledrightnote{\textcolor{blue}{Alfred Spitzer}}}, \label{K_L02643-11v}\edtext{dafern}{\lemma{\textnormal{\emph{dafern}}}\Cendnote{\textnormal{veraltet: sofern}}}\label{K_L02643-11h} er noch in \textsc{\textcolor{pink}{Ischl}{}\ledrightnote{\textcolor{pink}{Bad Ischl}}} iſt.\pend
           \pstart
           Herzlichen Gruß auch Ihnen! {\\[\baselineskip]}Ihr ergebener {\\[\baselineskip]}\spacefill\mbox{Dr. Paul Goldmann.}\pend
           \leftskip=0em{}\endnumbering\briefempfaengerindex{Schnitzler, Arthur@\textsc{Schnitzler, Arthur}!zzzGoldmann, Paul@\emph{von Paul Goldmann}!1889-08-061@{6. 8. 1889}|)be}\mylabel{h}  \normalsize

\doendnotes{C}
\bigskip
\vfill

\clearpage

\footnotesize

\lohead{\textsc{register}}

% Definiere theindex-Environment komplett neu ohne reledmac
\makeatletter
\renewenvironment{theindex}{%
  \section*{\indexname}%
  \setlength{\parindent}{0pt}%
  \setlength{\parskip}{0pt plus 0.3pt}%
  \let\item\@idxitem
}{%
  \clearpage
}
\makeatother

\IfFileExists{\jobname-pw.ind}{\input{\jobname-pw.ind}}{}

\end{document}

      