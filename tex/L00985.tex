%% latex-korrekturansicht-vorspann.tex
%% Vorspann für die Korrekturansicht.
%% Lädt die gemeinsame Datei latex-vorspann.tex mit gesetztem Schalter.

\newif\ifkorrekturansicht
\korrekturansichttrue

\input{../tex-inputs/latex-vorspann}


               \section[Richard Beer-Hofmann an Arthur Schnitzler, 1. 10. 1899]{ Richard Beer-Hofmann an Arthur Schnitzler, 1. 10. 1899}\nopagebreak\mylabel{v}\rehead{ }\normalsize\beginnumbering\briefempfaengerindex{Schnitzler, Arthur@\textsc{Schnitzler, Arthur}!zzzBeer-Hofmann, Richard@\emph{von Richard Beer-Hofmann}!1899-10-011@{1. 10. 1899}|(be} \toendnotes[C]{\smallbreak\pagebreak[2]} \Standort{CUL, Schnitzler, B 8.}
\physDesc{Bildpostkarte
\newline{}Handschrift: schwarze Tinte, lateinische Kurrent\newline{}Versand: 1) Stempel: »\nobreak{}\oindex{Sankt Michael@\textbf{Sankt Michael}, \emph{Bezirk (A.BZK)}|pwk}St. Michael in Eppan, 2 10 99\nobreak{}«.  2) Stempel: »\nobreak{}\oindex{Wiesbaden@\textbf{Wiesbaden}, \emph{Besiedelter Ort (A.BSO)}|pwk}Wiesbaden, 3. 10. 99, 9–10V\nobreak{}«. \newline{}Ordnung: mit Bleistift von unbekannter Hand nummeriert: »142« }\buchAbdrucke{\weitereDrucke{Arthur Schnitzler, Richard Beer-Hofmann: \emph{Briefwechsel 1891–1931}. Hg. Konstanze Fliedl. Wien, Zürich: \emph{Europaverlag} 1992, S. 138.} }\toendnotes[C]{\smallbreak}\pstart{}{\pb}D\textsuperscript{r} Arthur Schnitzler\pend{}\pstart{}\textcolor{pink}{Wiesbaden}{}\ledrightnote{\textcolor{pink}{Wiesbaden}}\pend{}\pstart{}\textcolor{pink}{Park-Hôtel}{}\ledrightnote{\textcolor{pink}{Hôtel du Parc & Bristol}}\pend{}{\bigskip}\pstart
           \noindent{}\centering{}\textcolor{gray}{\textbf{{\pb}Hotel und
                  Pension \textcolor{pink}{Eppaner Hof}{}\ledrightnote{\textcolor{pink}{Eppaner Hof}} in
                  \textcolor{pink}{Eppan}{}\ledrightnote{\textcolor{pink}{Eppan an der Weinstraße}}.}}\pend
           \pstart
           \raggedleft{}1/\strikeout{I}X 1899\pend
           \pstart
           Die × Fenster bewohnen \textcolor{blue}{wir}{}\ledrightnote{→\textcolor{blue}{Paula Beer-Hofmann}{\newline}→\textcolor{blue}{Mirjam Beer-Hofmann}{\newline}→\textcolor{blue}{Naëmah Beer-Hofmann}}. Die
               zwei rechts, ich. (Historisch).\pend
           \pstart
           Ich bin leider schon beim \textcolor{green}{420\textsuperscript{ten} Vers}{}\ledrightnote{→\textcolor{green}{Der Graf von Charolais. Ein Trauerspiel}} angelangt und noch i{\geminationm}er in der
               ersten Verwandl. des \textcolor{green}{I Aktes}{}\ledrightnote{→\textcolor{green}{Der Graf von Charolais. Ein Trauerspiel}}. Das wird ein
               den Abend überfüllendes \textcolor{green}{Stück}{}\ledrightnote{→\textcolor{green}{Der Graf von Charolais. Ein Trauerspiel}}! Ihr\spacefill\mbox{
                  R.}\pend
           \endnumbering\briefempfaengerindex{Schnitzler, Arthur@\textsc{Schnitzler, Arthur}!zzzBeer-Hofmann, Richard@\emph{von Richard Beer-Hofmann}!1899-10-011@{1. 10. 1899}|)be}\mylabel{h}  \normalsize

\doendnotes{C}
\bigskip
\vfill

\clearpage

\footnotesize

\lohead{\textsc{register}}

% Definiere theindex-Environment komplett neu ohne reledmac
\makeatletter
\renewenvironment{theindex}{%
  \section*{\indexname}%
  \setlength{\parindent}{0pt}%
  \setlength{\parskip}{0pt plus 0.3pt}%
  \let\item\@idxitem
}{%
  \clearpage
}
\makeatother

\IfFileExists{\jobname-pw.ind}{\input{\jobname-pw.ind}}{}

\end{document}

      