%% latex-korrekturansicht-vorspann.tex
%% Vorspann für die Korrekturansicht.
%% Lädt die gemeinsame Datei latex-vorspann.tex mit gesetztem Schalter.

\newif\ifkorrekturansicht
\korrekturansichttrue

\input{../tex-inputs/latex-vorspann}


               \section[Arthur Schnitzler an Richard Beer-Hofmann, 22. 5. 1894]{ Arthur Schnitzler an Richard Beer-Hofmann, 22. 5. 1894}\nopagebreak\mylabel{v}\rehead{ }\normalsize\beginnumbering\briefempfaengerindex{Beer-Hofmann, Richard@\textsc{Beer-Hofmann, Richard}!zzzSchnitzler, Arthur@\emph{von Arthur Schnitzler}!1894-05-221@{22. 5. 1894}|(be} \toendnotes[C]{\smallbreak\pagebreak[2]} \Standort{YCGL, MSS 31.}
\physDesc{Brief, 1 Blatt, 2 Seiten, Umschlag
\newline{}Handschrift: Bleistift, deutsche Kurrent\newline{}Versand: Stempel: »\nobreak{}\oindex{I., Innere Stadt@\textbf{I., Innere Stadt}, \emph{Bezirk (A.BZK)}|pwk}Wien 1/1, 22. 5. 94, 5–6N\nobreak{}«.  }\pstart{}{\pb}Herrn \textsc{Dr. Rich.
                     Beer-Hofmann}\pend{}\pstart{}\textcolor{pink}{Wien}{}\ledrightnote{\textcolor{pink}{Wien}}\pend{}\pstart{}\textcolor{pink}{\textsc{I. Wollzeile 15}}{}\ledrightnote{\textcolor{pink}{Wollzeile}}.\pend{}{\bigskip}\pstart{}{\pb}Lieber Richard,\pend\pstart
           ich erſuche Sie nochmals, ſchicken Sie an \textcolor{blue}{\textsc{Fels}}{}\ledrightnote{\textcolor{blue}{Friedrich Michael Fels}}. Nehmen Sie mir die wiederholte Mahnung
               nicht {\pb}übel, aber ich bin ſchon geradezu krank.\pend
           \pstart
           Herzlich grüßt{\\[\baselineskip]}Ihr{\\[\baselineskip]}\spacefill\mbox{Arthur}\pend
           \leftskip=0em{}\endnumbering\briefempfaengerindex{Beer-Hofmann, Richard@\textsc{Beer-Hofmann, Richard}!zzzSchnitzler, Arthur@\emph{von Arthur Schnitzler}!1894-05-221@{22. 5. 1894}|)be}\mylabel{h}  \normalsize

\doendnotes{C}
\bigskip
\vfill

\clearpage

\footnotesize

\lohead{\textsc{register}}

% Definiere theindex-Environment komplett neu ohne reledmac
\makeatletter
\renewenvironment{theindex}{%
  \section*{\indexname}%
  \setlength{\parindent}{0pt}%
  \setlength{\parskip}{0pt plus 0.3pt}%
  \let\item\@idxitem
}{%
  \clearpage
}
\makeatother

\IfFileExists{\jobname-pw.ind}{\input{\jobname-pw.ind}}{}

\end{document}

      