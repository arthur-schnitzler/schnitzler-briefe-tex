%% latex-korrekturansicht-vorspann.tex
%% Vorspann für die Korrekturansicht.
%% Lädt die gemeinsame Datei latex-vorspann.tex mit gesetztem Schalter.

\newif\ifkorrekturansicht
\korrekturansichttrue

\input{../tex-inputs/latex-vorspann}


               \section[Hugo von Hofmannsthal an Arthur Schnitzler, 27. 7. {[}1892{]}]{ Hugo von Hofmannsthal an Arthur Schnitzler, 27. 7. {[}1892{]}}\nopagebreak\mylabel{v}\rehead{ }\normalsize\beginnumbering\briefempfaengerindex{Schnitzler, Arthur@\textsc{Schnitzler, Arthur}!zzzHofmannsthal, Hugo von@\emph{von Hugo von Hofmannsthal}!1892-07-271@{27. 7. {[}1892{]}}|(be} \toendnotes[C]{\smallbreak\pagebreak[2]} \Standort{CUL, Schnitzler, B 43.}
\physDesc{Briefkarte (aufgeprägtes Wappen)
\newline{}Handschrift: schwarze Tinte, deutsche Kurrent
\newline{}Schnitzler: mit Bleistift die Jahreszahl ergänzt: »92« \newline{}Ordnung: mit Bleistift von unbekannter Hand nummeriert:
                                                »28« }\buchAbdrucke{\weitereDrucke{Hugo von Hofmannsthal, Arthur Schnitzler: \emph{Briefwechsel}. Hg. Therese Nickl und Heinrich Schnitzler. Frankfurt am Main: \emph{S. Fischer} 1964, S. 24–25.} }\toendnotes[C]{\smallbreak}\pstart
           \raggedleft{}{\pb}27 VII\pend
           \pstart{}Lieber Arthur.\pend\pstart
           Beſten Dank für die übergroße Rückſicht. Natürlich \uline{keine} Erwähnung im \textcolor{green}{Inhaltsverzeichnis}{}\ledrightnote{→\textcolor{green}{Anatol}}. Als Titel ginge nur: »\textcolor{green}{Einleitung}{}\ledrightnote{\textcolor{green}{Einleitung}}« »\textcolor{green}{als Einleitung}{}\ledrightnote{\textcolor{green}{Einleitung}}« oder
                    dergleichen, wie \strikeout{ſ}Sie wollen. Unter dem Gedicht,
                    glaub ich, ſollte ſtehen etwa: \textsc{Loris}, Herbſt
                        1892 oder {\pb}ein
                    noch genaueres Datum. Ich freue mich daſs \textcolor{green}{es}{}\ledrightnote{→\textcolor{green}{Anatol}} endlich zu Stande gekommen iſt und erwarte recht
                    bald einen Brief.\pend
           \pstart
           Herzlichſt{\\[\baselineskip]}\spacefill\mbox{Loris.}\pend
           \leftskip=0em{}\endnumbering\briefempfaengerindex{Schnitzler, Arthur@\textsc{Schnitzler, Arthur}!zzzHofmannsthal, Hugo von@\emph{von Hugo von Hofmannsthal}!1892-07-271@{27. 7. {[}1892{]}}|)be}\mylabel{h}  \normalsize

\doendnotes{C}
\bigskip
\vfill

\clearpage

\footnotesize

\lohead{\textsc{register}}

% Definiere theindex-Environment komplett neu ohne reledmac
\makeatletter
\renewenvironment{theindex}{%
  \section*{\indexname}%
  \setlength{\parindent}{0pt}%
  \setlength{\parskip}{0pt plus 0.3pt}%
  \let\item\@idxitem
}{%
  \clearpage
}
\makeatother

\IfFileExists{\jobname-pw.ind}{\input{\jobname-pw.ind}}{}

\end{document}

      