%% latex-korrekturansicht-vorspann.tex
%% Vorspann für die Korrekturansicht.
%% Lädt die gemeinsame Datei latex-vorspann.tex mit gesetztem Schalter.

\newif\ifkorrekturansicht
\korrekturansichttrue

\input{../tex-inputs/latex-vorspann}


               \section[Richard Beer-Hofmann an Arthur Schnitzler, 20. 6. 1900]{ Richard Beer-Hofmann an Arthur Schnitzler, 20. 6. 1900}\nopagebreak\mylabel{v}\rehead{ }\normalsize\beginnumbering\briefempfaengerindex{Schnitzler, Arthur@\textsc{Schnitzler, Arthur}!zzzBeer-Hofmann, Richard@\emph{von Richard Beer-Hofmann}!1900-06-201@{20. 6. 1900}|(be} \toendnotes[C]{\smallbreak\pagebreak[2]} \Standort{CUL, Schnitzler, B 8.}
\physDesc{Brief, 1 Blatt, 2 Seiten
\newline{}Handschrift: schwarze Tinte, lateinische Kurrent\newline{}Ordnung: mit Bleistift von unbekannter Hand nummeriert: »154« }\buchAbdrucke{\weitereDrucke{Arthur Schnitzler, Richard Beer-Hofmann: \emph{Briefwechsel 1891–1931}. Hg. Konstanze Fliedl. Wien, Zürich: \emph{Europaverlag} 1992, S. 145–146.} }\toendnotes[C]{\smallbreak}\pstart
           \raggedleft{}{\pb}\textcolor{pink}{Alt-Aussee}{}\ledrightnote{\textcolor{pink}{Altaussee}}{ }20/VI 1900\pend
           \pstart
           Lieber Arthur! Natürlich sollen Sie herko{\geminationm}en. Schreiben Sie mir für wann, und ob ich Zimmer
               (eins) für Sie bestellen soll. \textcolor{pink}{\uline{Seewirth}}{}\ledrightnote{\textcolor{pink}{Seewirt}} oder \textcolor{pink}{Brunnthaler}{}\ledrightnote{\textcolor{pink}{Gasthaus Brunnthaler}} (wo \textcolor{blue}{Hugo}{}\ledrightnote{\textcolor{blue}{Hugo von Hofmannsthal}} wohnte). Übrigens ist es überflüssig da keine Überfülle
               von Fremden hier ist. Jedenfalls telegraphiren Sie. Ich arbeite erst seit 5 Tagen;
               mehr, wäre mehr. \textcolor{blue}{S.}{}\ledrightnote{\textcolor{blue}{Paul Schlenther}} richtet sich danach, daß \textcolor{blue}{B.}{}\ledrightnote{\textcolor{blue}{Otto Brahm}} es nicht geno{\geminationm}en
               hat (S = \textcolor{blue}{Schlenther}{}\ledrightnote{\textcolor{blue}{Paul Schlenther}}, B = \textcolor{blue}{Brahm}{}\ledrightnote{\textcolor{blue}{Otto Brahm}}. Bemerk. des Herausgebers). Ich habe aber wirklich keinen
               Grund »Witze« zu machen. Ich halte meine Laune mit knapper Mühe auf arbeits{\pb}fähigem Niveau. Ende
                  Juli könnte ich nicht mit. Je später im August, desto wahrscheinlicher;
               jedenfalls etwas Süden ins Programm nehmen. \label{K_L01046_1v}\edtext{Im Juli}{\lemma{\textnormal{\emph{Im Juli}}}\Cendnote{\textnormal{Er ist am
                     11. 7. 1866 geboren.}}}\label{K_L01046_1h} werde ich vierunddreißig, –
               um Ihnen zum Schluß noch etwas Angenehmes zu sagen.\pend
           \pstart Von Herzen Ihr \spacefill\mbox{Richard}\pend{}\endnumbering\briefempfaengerindex{Schnitzler, Arthur@\textsc{Schnitzler, Arthur}!zzzBeer-Hofmann, Richard@\emph{von Richard Beer-Hofmann}!1900-06-201@{20. 6. 1900}|)be}\mylabel{h}  \normalsize

\doendnotes{C}
\bigskip
\vfill

\clearpage

\footnotesize

\lohead{\textsc{register}}

% Definiere theindex-Environment komplett neu ohne reledmac
\makeatletter
\renewenvironment{theindex}{%
  \section*{\indexname}%
  \setlength{\parindent}{0pt}%
  \setlength{\parskip}{0pt plus 0.3pt}%
  \let\item\@idxitem
}{%
  \clearpage
}
\makeatother

\IfFileExists{\jobname-pw.ind}{\input{\jobname-pw.ind}}{}

\end{document}

      