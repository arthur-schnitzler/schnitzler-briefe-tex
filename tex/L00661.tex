%% latex-korrekturansicht-vorspann.tex
%% Vorspann für die Korrekturansicht.
%% Lädt die gemeinsame Datei latex-vorspann.tex mit gesetztem Schalter.

\newif\ifkorrekturansicht
\korrekturansichttrue

\input{../tex-inputs/latex-vorspann}


               \section[Peter Altenberg an Arthur Schnitzler, {[}24. 3. 1897{]}]{ Peter Altenberg an Arthur Schnitzler, {[}24. 3. 1897{]}}\nopagebreak\mylabel{v}\rehead{ }\normalsize\beginnumbering\briefempfaengerindex{Schnitzler, Arthur@\textsc{Schnitzler, Arthur}!zzzAltenberg, Peter@\emph{von Peter Altenberg}!1897-03-241@{{[}24. 3. 1897{]}}|(be} \toendnotes[C]{\smallbreak\pagebreak[2]} \Standort{CUL, Schnitzler, B 2.}
\physDesc{Brief, 1 Blatt, 1 Seite
\newline{}Handschrift: schwarze Tinte, deutsche Kurrent
\newline{}Schnitzler: mit Bleistift datiert: »24/3 97« \newline{}Ordnung: mit Bleistift von unbekannter Hand nummeriert: »5« }\toendnotes[C]{\smallbreak}\pstart{}{\pb}Lieber \textsc{D\textsuperscript{r}} Arthur Schnitzler:\pend\pstart
           danke ſehr für Brief. Konnte nicht kommen, da \textsc{Souper-Rendezvous} hatte. \label{K_L00661_1v}\edtext{Leſe natürlich nicht}{\lemma{\textnormal{\emph{Leſe natürlich nicht}}}\Cendnote{\textnormal{Am
                            24. 3. 1897 fand die Probe zur öffentlichen Lesung von \textcolor{blue}{Hermann Bahr}, \textcolor{blue}{Georg Hirschfeld} und \textcolor{blue}{Hugo von Hofmannsthal} statt. Auch \textcolor{blue}{Altenberg} war eingeladen, sagte aber mit diesem Brief einen
                        etwaigen Auftritt an der Veranstaltung ab. Die Lesung selbst fand am
                            28. 3. 1897 statt.}}}\label{K_L00661_1h}. Aber könnte man eine
                    Umſonſt-Karte erhalten?! Bitte, laſſen Sie es mir in das \textcolor{pink}{\textsc{Pucher-Café}}{}\ledrightnote{\textcolor{pink}{Café Pucher}}{ }ſagen.\pend
           \pstart
           Ihr{\\[\baselineskip]}\spacefill\mbox{Peter Altenberg }\pend
           \leftskip=0em{}\endnumbering\briefempfaengerindex{Schnitzler, Arthur@\textsc{Schnitzler, Arthur}!zzzAltenberg, Peter@\emph{von Peter Altenberg}!1897-03-241@{{[}24. 3. 1897{]}}|)be}\mylabel{h}  \normalsize

\doendnotes{C}
\bigskip
\vfill

\clearpage

\footnotesize

\lohead{\textsc{register}}

% Definiere theindex-Environment komplett neu ohne reledmac
\makeatletter
\renewenvironment{theindex}{%
  \section*{\indexname}%
  \setlength{\parindent}{0pt}%
  \setlength{\parskip}{0pt plus 0.3pt}%
  \let\item\@idxitem
}{%
  \clearpage
}
\makeatother

\IfFileExists{\jobname-pw.ind}{\input{\jobname-pw.ind}}{}

\end{document}

      