%% latex-korrekturansicht-vorspann.tex
%% Vorspann für die Korrekturansicht.
%% Lädt die gemeinsame Datei latex-vorspann.tex mit gesetztem Schalter.

\newif\ifkorrekturansicht
\korrekturansichttrue

\input{../tex-inputs/latex-vorspann}


               \section[Arthur Schnitzler an Richard Beer-Hofmann, 27. 10. 1909]{ Arthur Schnitzler an Richard Beer-Hofmann, 27. 10. 1909}\nopagebreak\mylabel{v}\rehead{ }\normalsize\beginnumbering\briefempfaengerindex{Beer-Hofmann, Richard@\textsc{Beer-Hofmann, Richard}!zzzSchnitzler, Arthur@\emph{von Arthur Schnitzler}!1909-10-271@{27. 10. 1909}|(be} \toendnotes[C]{\smallbreak\pagebreak[2]} \Standort{YCGL, MSS 31.}
\physDesc{Postkarte
\newline{}Handschrift: Bleistift, deutsche Kurrent\newline{}Versand: 1) Stempel: »\nobreak{}\oindex{I., Innere Stadt@\textbf{I., Innere Stadt}, \emph{Bezirk (A.BZK)}|pwk}1/3 Wien 100, 27. X. \textcolor{gray}{09}\nobreak{}«.  2) Stempel: »\nobreak{}\oindex{XVIII., Waehring@\textbf{XVIII., Währing}, \emph{Bezirk (A.BZK)}|pwk}18/\textcolor{gray}{1} Wien
                              \textcolor{gray}{110}, 27. X. \textcolor{gray}{09}, 2\textsuperscript{10}\nobreak{}«. }\buchAbdrucke{\weitereDrucke{Arthur Schnitzler, Richard Beer-Hofmann: \emph{Briefwechsel 1891–1931}. Hg. Konstanze Fliedl. Wien, Zürich: \emph{Europaverlag} 1992, S. 195.} }\toendnotes[C]{\smallbreak}\pstart{}{\pb}\textsc{Dr. Richard Beer-Hofmann}\pend{}\pstart{}\textsc{\textcolor{pink}{Wien XVIII}{}\ledrightnote{\textcolor{pink}{XVIII., Währing}}}\pend{}\pstart{}\textsc{\textcolor{pink}{\textsc{Hasenauerstr 59}}{}\ledrightnote{\textcolor{pink}{Hasenauerstraße}}.}\pend{}{\bigskip}\pstart
           \noindent{}{\pb}\label{K_L01881_1v}\edtext{Montag
                     \textsc{ersten}}{\lemma{\textnormal{\emph{Montag
                     ersten}}}\Cendnote{\textnormal{siehe A. S.: \emph{Tagebuch}, 1. 11. 1909}}}\label{K_L01881_1h},
                  Nm. 5 ½ möcht ich \textcolor{green}{Medardus}{}\ledrightnote{\textcolor{green}{Der junge Medardus. Dramatische Historie in einem Vorspiel und fünf Aufzügen}}
               vorleſen. Bitte Mittheilg, recht bald, ob\damage{\textcolor{gray}{’s}} Ihnen paſſt.\pend
           \pstart
           Herzlichſt Ihr{\\[\baselineskip]}\spacefill\mbox{Arthur}\pend
           \leftskip=0em{}\endnumbering\briefempfaengerindex{Beer-Hofmann, Richard@\textsc{Beer-Hofmann, Richard}!zzzSchnitzler, Arthur@\emph{von Arthur Schnitzler}!1909-10-271@{27. 10. 1909}|)be}\mylabel{h}  \normalsize

\doendnotes{C}
\bigskip
\vfill

\clearpage

\footnotesize

\lohead{\textsc{register}}

% Definiere theindex-Environment komplett neu ohne reledmac
\makeatletter
\renewenvironment{theindex}{%
  \section*{\indexname}%
  \setlength{\parindent}{0pt}%
  \setlength{\parskip}{0pt plus 0.3pt}%
  \let\item\@idxitem
}{%
  \clearpage
}
\makeatother

\IfFileExists{\jobname-pw.ind}{\input{\jobname-pw.ind}}{}

\end{document}

      