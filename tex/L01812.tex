%% latex-korrekturansicht-vorspann.tex
%% Vorspann für die Korrekturansicht.
%% Lädt die gemeinsame Datei latex-vorspann.tex mit gesetztem Schalter.

\newif\ifkorrekturansicht
\korrekturansichttrue

\input{../tex-inputs/latex-vorspann}


               \section[Olga Schnitzler an Richard Beer-Hofmann, {[}28. 11. 1908?{]}]{ Olga Schnitzler an Richard Beer-Hofmann, {[}28. 11. 1908?{]}}\nopagebreak\mylabel{v}\rehead{ }\normalsize\beginnumbering\briefempfaengerindex{Beer-Hofmann, Richard@\textsc{Beer-Hofmann, Richard}!zzzSchnitzler, Olga@\emph{von Olga Schnitzler}!1908-11-282@{{[}28. 11. 1908?{]}}|(be} \toendnotes[C]{\smallbreak\pagebreak[2]} \Standort{YCGL, MSS 31.}
\physDesc{Brief, 1 Blatt, 1 Seite, Umschlag
\newline{}Handschrift: schwarze Tinte, lateinische Kurrent\newline{}Versand: ohne postalischen Übermittlungsvermerk }\toendnotes[C]{\smallbreak}\pstart{}{\pb}\textcolor{gray}{\textbf{O. S.}}\pend{}{\bigskip}\pstart{}{\pb}Herrn D\textsuperscript{r} Richard
                  Beer-Hofmann\pend{}{\bigskip}\pstart
           \noindent{}{\pb}\textcolor{gray}{\textbf{O. S.}}\pend
           \pstart{}Lieber Herr Doctor,\pend\pstart
           \textcolor{blue}{Arthur}{}\ledrightnote{} lässt sagen: wir wollten zuhause
               nachtmalen, möchten Sie und \textcolor{blue}{Kerr}{}\ledrightnote{\textcolor{blue}{Alfred Kerr}} aber sehr gerne
               treffen, daher sind wir nach dem Concert bei \textcolor{pink}{Meissl {\kaufmannsund} Schadn}{}\ledrightnote{\textcolor{pink}{Meissl & Schadn}}, so gegen ½ oder
                  ¾ 10.\pend
           \pstart
           Herzliche Grüsse an Sie \textcolor{blue}{Beide}{}\ledrightnote{→\textcolor{blue}{Paula Beer-Hofmann}}{\\[\baselineskip]}\spacefill\mbox{OlgaS.}\pend
           \leftskip=0em{}\endnumbering\briefempfaengerindex{Beer-Hofmann, Richard@\textsc{Beer-Hofmann, Richard}!zzzSchnitzler, Olga@\emph{von Olga Schnitzler}!1908-11-282@{{[}28. 11. 1908?{]}}|)be}\mylabel{h}  \normalsize

\doendnotes{C}
\bigskip
\vfill

\clearpage

\footnotesize

\lohead{\textsc{register}}

% Definiere theindex-Environment komplett neu ohne reledmac
\makeatletter
\renewenvironment{theindex}{%
  \section*{\indexname}%
  \setlength{\parindent}{0pt}%
  \setlength{\parskip}{0pt plus 0.3pt}%
  \let\item\@idxitem
}{%
  \clearpage
}
\makeatother

\IfFileExists{\jobname-pw.ind}{\input{\jobname-pw.ind}}{}

\end{document}

      