%% latex-korrekturansicht-vorspann.tex
%% Vorspann für die Korrekturansicht.
%% Lädt die gemeinsame Datei latex-vorspann.tex mit gesetztem Schalter.

\newif\ifkorrekturansicht
\korrekturansichttrue

\input{../tex-inputs/latex-vorspann}


               \section[Arthur Schnitzler an Hugo Hofmannsthal, 15. 1. 1923]{ Arthur Schnitzler an Hugo Hofmannsthal, 15. 1. 1923}\nopagebreak\mylabel{v}\rehead{ }\normalsize\beginnumbering\briefempfaengerindex{Hofmannsthal, Hugo von@\textsc{Hofmannsthal, Hugo von}!zzzSchnitzler, Arthur@\emph{von Arthur Schnitzler}!1923-01-151@{15. 1. 1923}|(be} \toendnotes[C]{\smallbreak\pagebreak[2]} \buchAlsQuelle{Hugo von Hofmannsthal, Arthur Schnitzler: \emph{Briefwechsel}. Hg. Therese Nickl und Heinrich Schnitzler. Frankfurt am Main: \emph{S. Fischer} 1964, S. 296–297.}\buchAbdrucke{\weitereDrucke{Arthur Schnitzler: \emph{Briefe 1913–1931}. Hg. Peter Michael Braunwarth, Richard Miklin, Susanne Pertlik und Heinrich Schnitzler. Frankfurt am Main: \emph{S. Fischer} 1984, S. 301–302.} }\toendnotes[C]{\smallbreak}\pstart
           \noindent{}{\pb}{[}\label{K_L02395_1v}\edtext{Maschinenschrift}{\lemma{\textnormal{\emph{Maschinenschrift}}}\Cendnote{\textnormal{Das originale Typoskript ist nicht
                        auffindbar.}}}\label{K_L02395_1h}{]}\pend
           \pstart
           \raggedleft{}15. 1. 1923\pend
           \pstart{}Mein lieber Hugo. \pend\pstart
           Sie wissen vielleicht, daß die »\textcolor{green}{Beatrice}{}\ledrightnote{\textcolor{green}{Der Schleier der Beatrice. Schauspiel in fünf Akten}}« von \textcolor{blue}{Heinrich Noren}{}\ledrightnote{\textcolor{blue}{Heinrich Noren}} komponiert worden ist. Auf mein
               Ersuchen die Partitur anzusehen, resp. sich Teile aus der \textcolor{green}{Oper}{}\ledrightnote{→\textcolor{green}{Der Schleier der Beatrice. Schauspiel in fünf Akten}} von \textcolor{blue}{Noren}{}\ledrightnote{\textcolor{blue}{Heinrich Noren}}{ }selbst (der einen höchst geachteten musikalischen
               Namen besitzt) vorspielen zu lassen, erwiderte mir \textcolor{blue}{Richard Strauss}{}\ledrightnote{\textcolor{blue}{Richard Strauss}}, daß die \textcolor{brown}{Oper}{}\ledrightnote{\textcolor{brown}{Staatsoper}} überhaupt
               nicht daran denken könne Uraufführungen zu bringen – aus hauptsächlich materiellen,
               aber gewiß plausiblen Gründen. Es gibt vielleicht Fälle, in denen man von diesem
               Prinzip abgehen könnte, es scheint ja auch, daß es manchmal geschieht. Ich selbst
               konnte natürlich in meinem Falle nicht insistieren, obwohl gerade er am ehesten Anlaß
               gäbe von jenem Prinzip wenigstens insoweit abzuweichen, als die Direktion der \textcolor{brown}{Oper}{}\ledrightnote{\textcolor{brown}{Staatsoper}} immerhin den Versuch riskieren könnte, das Werk
               kennen zu lernen. Warum ich das Ihnen erzähle, lieber Hugo? Weil mir neulich \textcolor{blue}{Noren}{}\ledrightnote{\textcolor{blue}{Heinrich Noren}}{ }schreibt, und weil \textcolor{blue}{Bruno Walter}{}\ledrightnote{\textcolor{blue}{Bruno Walter}} gleichfalls behauptet, daß Sie der einzige Mensch wären, der
               auf \textcolor{blue}{Strauss}{}\ledrightnote{\textcolor{blue}{Richard Strauss}} oder \textcolor{blue}{Schalk}{}\ledrightnote{\textcolor{blue}{Franz Schalk}} oder auf sie Beide in dem Sinne einwirken könnte, daß diese zum
               mindesten von der Existenz des in Frage stehenden Werkes Notiz nähmen, der vielleicht
               sogar (dies sind \textcolor{blue}{Bruno Walter}{}\ledrightnote{\textcolor{blue}{Bruno Walter}}s Worte) auf die
               Absurdität hinweisen dürfte, die nicht nur dem Komponisten darin zu liegen scheint,
               daß die \textcolor{brown}{Wiener Oper}{}\ledrightnote{\textcolor{brown}{Staatsoper}} ein sozusagen von zwei \textcolor{pink}{Österreichern}{}\ledrightnote{\textcolor{pink}{Österreich}} verfaßtes Werk, und von nicht ganz
               unbekannten überdies, nicht nur nicht zu eventueller Uraufführung in Erwägung ziehen,
               sondern vorläufig sogar eine Prüfung lieber vermeiden möchte. Auch ich fühle etwas
               von der Absurdität, die in \textcolor{blue}{Strauss}{}\ledrightnote{\textcolor{blue}{Richard Strauss}}ens Vorgehen
               steckt (mit \textcolor{blue}{Schalk}{}\ledrightnote{\textcolor{blue}{Franz Schalk}} habe ich nicht gesprochen, er
               weiß vielleicht von der Existenz der \textcolor{green}{Oper}{}\ledrightnote{→\textcolor{green}{Der Schleier der Beatrice. Schauspiel in fünf Akten}} bis heute gar nichts); trotzdem hätte ich Sie in der Sache nicht
               bemüht, wenn ich es nicht allzu schwer fände \textcolor{blue}{Heinrich
                  Noren}{}\ledrightnote{\textcolor{blue}{Heinrich Noren}} die Erfüllung eines Wunsches zu verweigern, die ihm die Erfüllung
               seines wesent{\pb}lichern – die Aufführung seiner
                  \textcolor{green}{Oper}{}\ledrightnote{→\textcolor{green}{Der Schleier der Beatrice. Schauspiel in fünf Akten}} in \textcolor{pink}{Wien}{}\ledrightnote{\textcolor{pink}{Wien}} – in die Nähe zu rücken scheint. Ich weiß weder, ob Sie,
               lieber Hugo, Gelegenheit, noch ob Sie Lust haben sich mit dieser Sache in irgend
               einer Form zu befassen. Vielleicht sprechen wir bald einmal darüber, wenn Sie wieder
               nach \textcolor{pink}{Wien}{}\ledrightnote{\textcolor{pink}{Wien}} hereinkommen. Es wäre ja überhaupt schon
               Zeit, daß man sich wieder einmal sieht und spricht. Ich habe Ihnen noch nicht einmal
               zum \label{K_L02395_2v}\edtext{Erfolg des »\textcolor{green}{Großen Welttheaters}{}\ledrightnote{\textcolor{green}{Das Salzburger große Welttheater}}«}{\lemma{\textnormal{\emph{Erfolg … Welttheaters«}}}\Cendnote{\textnormal{Die Uraufführung fand am 12. 8. 1922 in der \textcolor{pink}{Kollegienkirche in Salzburg} statt. Regie führte \textcolor{blue}{Max Reinhardt}.}}}\label{K_L02395_2h} gratuliert und nicht
               gesagt, wie schön Ihre beiden \textcolor{green}{Artikel}{}\ledrightnote{→\textcolor{green}{Vienna Letter}{\newline}→\textcolor{green}{Vienna Letter}} im »\textcolor{green}{Dial}{}\ledrightnote{\textcolor{green}{The Dial}}« (nicht nur \textcolor{green}{der über mich}{}\ledrightnote{→\textcolor{green}{Vienna Letter}}) waren.\pend
           \pstart Seien Sie herzlichst gegrüßt \spacefill\mbox{\textcolor{gray}{A. S.}}\pend{}\endnumbering\briefempfaengerindex{Hofmannsthal, Hugo von@\textsc{Hofmannsthal, Hugo von}!zzzSchnitzler, Arthur@\emph{von Arthur Schnitzler}!1923-01-151@{15. 1. 1923}|)be}\mylabel{h}  \normalsize

\doendnotes{C}
\bigskip
\vfill

\clearpage

\footnotesize

\lohead{\textsc{register}}

% Definiere theindex-Environment komplett neu ohne reledmac
\makeatletter
\renewenvironment{theindex}{%
  \section*{\indexname}%
  \setlength{\parindent}{0pt}%
  \setlength{\parskip}{0pt plus 0.3pt}%
  \let\item\@idxitem
}{%
  \clearpage
}
\makeatother

\IfFileExists{\jobname-pw.ind}{\input{\jobname-pw.ind}}{}

\end{document}

      