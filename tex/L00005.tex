%% latex-korrekturansicht-vorspann.tex
%% Vorspann für die Korrekturansicht.
%% Lädt die gemeinsame Datei latex-vorspann.tex mit gesetztem Schalter.

\newif\ifkorrekturansicht
\korrekturansichttrue

\input{../tex-inputs/latex-vorspann}


               \section[Arthur Schnitzler an Wilhelm Bölsche, {[}Anfang September{]} 1890]{ Arthur Schnitzler an Wilhelm Bölsche, {[}Anfang September{]} 1890}\nopagebreak\mylabel{v}\rehead{ }\normalsize\beginnumbering\briefempfaengerindex{Boelsche, Wilhelm@\textsc{Bölsche, Wilhelm}!zzzSchnitzler, Arthur@\emph{von Arthur Schnitzler}!1890-09-011@{{[}Anfang September{]} 1890}|(be} \toendnotes[C]{\smallbreak\pagebreak[2]} \Standort{Wrocław, Biblioteka Uniwersytecka, Böl.Pis 1773.}
\physDesc{Brief, 1 Blatt, 2 Seiten
\newline{}Handschrift: schwarze Tinte, deutsche Kurrent
\newline{}Bölsche: als »Erledigt« gezeichnet }\buchAbdrucke{\weitereDrucke{1) Alois Woldan: \emph{Arthur Schnitzler – Briefe an Wilhelm Bölsche.} In: \emph{Germanica Wratislaviensia} (1987) Nr. 77, S. 465–466.} \weitereDrucke{2) Wilhelm Bölsche: \emph{Briefwechsel. Mit Autoren der Freien Bühne}. Hg. Gerd-Hermann Susen. Berlin: \emph{Weidler} 2010, S. 667 (Werke und Briefe. Wissenschaftliche Ausgabe, Briefe I).} }\toendnotes[C]{\smallbreak}\pstart\center{}{\pb}Sehr geehrter Herr Redakteur!\pend\pstart
           Erlauben Sie mir, Ihnen beifolgende \label{K_L00005_1v}\edtext{\textcolor{green}{Skizze}{}\ledrightnote{→\textcolor{green}{Aus der Kaffeehausecke}}}{\lemma{\textnormal{\emph{Skizze}}}\Cendnote{\textnormal{\emph{\textcolor{green}{Aus der Kaffeehausecke}}; Schnitzler hat sie am
                     3. 2. 1890 und unmittelbar vor diesem Brief, am
                  29. 8. 1890, abgefasst und dann wohl gleich an \textcolor{blue}{Bölsche} geschickt. Die Skizze blieb zu Lebzeiten
                  unpubliziert.}}}\label{K_L00005_1h} vorzulegen. Sie iſt raſch geleſen; ich fürchte kaum, Sie
               allzuſehr in Anſpruch zu nehmen. Vielleicht finden Sie, daß ſie ſich dem Rahmen Ihrer
                  \textcolor{green}{\textsc{Freien Bühne für modernes Leben}}{}\ledrightnote{\textcolor{green}{Freie Bühne für modernes Leben}} ohne allzu ſchli{\geminationm}en Zwang einfügen ließe – in
               dieſem Falle würde ich Sie höflichſt um Veröffentlichung derſelben erſuchen. Misfällt
               ſie Ihnen, ſehr geehrter Herr, {\pb}\damage{ha}ben Sie wohl die Güte, das kleine Heft an meine Adreſſe zurückzuſenden.\pend
           \pstart
           Ich bin mit ausgezeichneter Hochachtung{\\[\baselineskip]}Ihr ergebner{\\[\baselineskip]}\spacefill\mbox{Dr.  med. Arthur Schnitzler}\pend
           \leftskip=0em{}\pstart
           \noindent{}\textsc{\textcolor{pink}{Wien, I. Giselastraße 11}{}\ledrightnote{\textcolor{pink}{Bösendorferstraße}}.}\pend
           \endnumbering\briefempfaengerindex{Boelsche, Wilhelm@\textsc{Bölsche, Wilhelm}!zzzSchnitzler, Arthur@\emph{von Arthur Schnitzler}!1890-09-011@{{[}Anfang September{]} 1890}|)be}\mylabel{h}  \normalsize

\doendnotes{C}
\bigskip
\vfill

\clearpage

\footnotesize

\lohead{\textsc{register}}

% Definiere theindex-Environment komplett neu ohne reledmac
\makeatletter
\renewenvironment{theindex}{%
  \section*{\indexname}%
  \setlength{\parindent}{0pt}%
  \setlength{\parskip}{0pt plus 0.3pt}%
  \let\item\@idxitem
}{%
  \clearpage
}
\makeatother

\IfFileExists{\jobname-pw.ind}{\input{\jobname-pw.ind}}{}

\end{document}

      