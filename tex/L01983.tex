%% latex-korrekturansicht-vorspann.tex
%% Vorspann für die Korrekturansicht.
%% Lädt die gemeinsame Datei latex-vorspann.tex mit gesetztem Schalter.

\newif\ifkorrekturansicht
\korrekturansichttrue

\input{../tex-inputs/latex-vorspann}


               \section[Hugo von Hofmannsthal an Arthur Schnitzler, {[}21. 11. 1910{]}]{ Hugo von Hofmannsthal an Arthur Schnitzler, {[}21. 11. 1910{]}}\nopagebreak\mylabel{v}\rehead{ }\normalsize\beginnumbering\briefempfaengerindex{Schnitzler, Arthur@\textsc{Schnitzler, Arthur}!zzzHofmannsthal, Hugo von@\emph{von Hugo von Hofmannsthal}!1910-11-211@{{[}21. 11. 1910{]}}|(be} \toendnotes[C]{\smallbreak\pagebreak[2]} \Standort{CUL, Schnitzler, B 43.}
\physDesc{Brief, 1 Blatt, 2 Seiten
\newline{}Handschrift: schwarze Tinte, deutsche Kurrent
\newline{}Schnitzler: mit Bleistift falsch auf einen Sonntag datiert: »20/11 910« und beschriftet: »Hugo« \newline{}Ordnung: 1) mit Bleistift von unbekannter Hand nummeriert: »\strikeout{309}« 2) mit Bleistift von unbekannter Hand nummeriert: »326«}\buchAbdrucke{\weitereDrucke{Hugo von Hofmannsthal, Arthur Schnitzler: \emph{Briefwechsel}. Hg. Therese Nickl und Heinrich Schnitzler. Frankfurt am Main: \emph{S. Fischer} 1964, S. 260.} }\toendnotes[C]{\smallbreak}\pstart
           \raggedleft{}{\pb}Montg.\pend
           \pstart{}mein lieber Arthur, \pend\pstart
           ich glaube es iſt beſſer, ich verzichte auf die \label{K_L01983_1v}\edtext{Generalprobe}{\lemma{\textnormal{\emph{Generalprobe}}}\Cendnote{\textnormal{siehe A. S.: \emph{Tagebuch}, 23. 11. 1910}}}\label{K_L01983_1h} und gehe nur in die \label{K_L01983_2v}\edtext{\textcolor{green}{Vorſtellung}{}\ledrightnote{→\textcolor{green}{Der junge Medardus. Dramatische Historie in einem Vorspiel und fünf Aufzügen}}}{\lemma{\textnormal{\emph{Vorſtellung}}}\Cendnote{\textnormal{siehe A. S.: \emph{Tagebuch}, 24. 11. 1910}}}\label{K_L01983_2h}. Die Generalprobe, dann Eſſen in der
               Stadt, dann Herausfahren koſtet mich einen ganzen Tag, den Do{\geminationn}erstag bin ich ohnedies {\pb}in \textcolor{pink}{Wien}{}\ledrightnote{\textcolor{pink}{Wien}}, wenn dies nun ſchon der 2\textsuperscript{te} Tag iſt den
               ich ohne Ruhe, ohne Arbeit oder Concentration zerſtreut hinbringe, bin ich ſicher
                  \strikeout{zerſtreut} ein abgeſpannter ſchlechter Zuhörer.\pend
           \pstart Alſo beſſer ſo. Von Herzen Ihr\spacefill\mbox{Hugo.}\pend{}\endnumbering\briefempfaengerindex{Schnitzler, Arthur@\textsc{Schnitzler, Arthur}!zzzHofmannsthal, Hugo von@\emph{von Hugo von Hofmannsthal}!1910-11-211@{{[}21. 11. 1910{]}}|)be}\mylabel{h}  \normalsize

\doendnotes{C}
\bigskip
\vfill

\clearpage

\footnotesize

\lohead{\textsc{register}}

% Definiere theindex-Environment komplett neu ohne reledmac
\makeatletter
\renewenvironment{theindex}{%
  \section*{\indexname}%
  \setlength{\parindent}{0pt}%
  \setlength{\parskip}{0pt plus 0.3pt}%
  \let\item\@idxitem
}{%
  \clearpage
}
\makeatother

\IfFileExists{\jobname-pw.ind}{\input{\jobname-pw.ind}}{}

\end{document}

      