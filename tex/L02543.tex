%% latex-korrekturansicht-vorspann.tex
%% Vorspann für die Korrekturansicht.
%% Lädt die gemeinsame Datei latex-vorspann.tex mit gesetztem Schalter.

\newif\ifkorrekturansicht
\korrekturansichttrue

\input{../tex-inputs/latex-vorspann}


               \section[Arthur Schnitzler an Gerty von Hofmannsthal, 17. 2. 1931]{ Arthur Schnitzler an Gerty von Hofmannsthal, 17. 2. 1931}\nopagebreak\mylabel{v}\rehead{ }\normalsize\beginnumbering\briefempfaengerindex{Hofmannsthal, Gertrude von@\textsc{Hofmannsthal, Gertrude von}!zzzSchnitzler, Arthur@\emph{von Arthur Schnitzler}!1931-02-171@{17. 2. 1931}|(be} \toendnotes[C]{\smallbreak\pagebreak[2]} \Standort{FDH, Hs-31346,4.}
\physDesc{Postkarte
\newline{}Handschrift: schwarze Tinte, lateinische Kurrent\newline{}Versand: Stempel: »\nobreak{}Wien 68, \textcolor{gray}{1}7. II. {[}1931{]}\nobreak{}«.  }\toendnotes[C]{\smallbreak}\pstart{}{\pb}\label{T_L02543-1v}\edtext{\textcolor{gray}{\textbf{A. S.}}}{\lemma{\textnormal{\emph{A. S.}}}\Cendnote{\textnormal{ovaler Absenderkleber}}}\label{T_L02543-1h}\pend{}\pstart{}\textcolor{pink}{\textcolor{gray}{\textbf{WIEN, XVIII.}}}{}\ledrightnote{\textcolor{pink}{XVIII., Währing}}\pend{}\pstart{}\textcolor{pink}{\textcolor{gray}{\textbf{STERNWARTESTR. 71}}}{}\ledrightnote{\textcolor{pink}{Sternwartestraße}}\pend{}{\bigskip}\pstart{}Frau Gerty von Hofmannsthal\pend{}\pstart{}\textcolor{pink}{Wien IV}{}\ledrightnote{\textcolor{pink}{IV., Wieden}}\pend{}\pstart{}\textcolor{pink}{Mozartgasse 4}{}\ledrightnote{\textcolor{pink}{Mozartgasse}}\pend{}{\bigskip}\pstart
           \raggedleft{}{\pb}\textcolor{pink}{Wien}{}\ledrightnote{\textcolor{pink}{Wien}}{ }\label{K_L02543_1v}\edtext{18/2 931}{\lemma{\textnormal{\emph{18/2 931}}}\Cendnote{\textnormal{Beide Rollstempel weisen
                        unzweifelhaft eine »7« aus, so dass \textcolor{blue}{Schnitzler} falsch datiert haben dürfte.}}}\label{K_L02543_1h}\pend
           \pstart
           liebe Gerty, ich danke Ihnen sehr un\textcolor{gray}{d} hoffe Sie baldigst zu sehen.
               Sie haben mir Ihre Tel. Nummer nicht gesagt, hier, zur Revanche die meine: \textsc{\label{K_L02543_2v}\edtext{A 10.0.81}{\lemma{\textnormal{\emph{A 10.0.81}}}\Cendnote{\textnormal{Es handelt sich um eine Geheimnummer. Im offiziellen Adressbuch steht \textcolor{blue}{Schnitzler} bis zum Tod mit der Nummer
                     »A-14.432«.}}}\label{K_L02543_2h}}. Bitte rufen Sie mich an, damit wir was ausmachen können.\pend
           \pstart
           Alles herzliche{\\[\baselineskip]}Ihr \spacefill\mbox{Arthur.}\pend
           \leftskip=0em{}\endnumbering\briefempfaengerindex{Hofmannsthal, Gertrude von@\textsc{Hofmannsthal, Gertrude von}!zzzSchnitzler, Arthur@\emph{von Arthur Schnitzler}!1931-02-171@{17. 2. 1931}|)be}\mylabel{h}  \normalsize

\doendnotes{C}
\bigskip
\vfill

\clearpage

\footnotesize

\lohead{\textsc{register}}

% Definiere theindex-Environment komplett neu ohne reledmac
\makeatletter
\renewenvironment{theindex}{%
  \section*{\indexname}%
  \setlength{\parindent}{0pt}%
  \setlength{\parskip}{0pt plus 0.3pt}%
  \let\item\@idxitem
}{%
  \clearpage
}
\makeatother

\IfFileExists{\jobname-pw.ind}{\input{\jobname-pw.ind}}{}

\end{document}

      