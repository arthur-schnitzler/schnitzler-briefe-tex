%% latex-korrekturansicht-vorspann.tex
%% Vorspann für die Korrekturansicht.
%% Lädt die gemeinsame Datei latex-vorspann.tex mit gesetztem Schalter.

\newif\ifkorrekturansicht
\korrekturansichttrue

\input{../tex-inputs/latex-vorspann}


               \section[Hugo von Hofmannsthal an Arthur Schnitzler, 25. 1. 1898]{ Hugo von Hofmannsthal an Arthur Schnitzler, 25. 1. 1898}\nopagebreak\mylabel{v}\rehead{ }\normalsize\beginnumbering\briefempfaengerindex{Schnitzler, Arthur@\textsc{Schnitzler, Arthur}!zzzHofmannsthal, Hugo von@\emph{von Hugo von Hofmannsthal}!1898-01-254@{25. 1. 1898}|(be} \toendnotes[C]{\smallbreak\pagebreak[2]} \Standort{CUL, Schnitzler, B 43b/1.}
\physDesc{Postkarte
\newline{}Handschrift: Bleistift, deutsche Kurrent\newline{}Versand: 1) Rohrpost 2) Stempel: »\nobreak{}Wien 3/3, 25 1 98, 10 20V\nobreak{}«. 3) Stempel: »\nobreak{}Wien 9/2, 25 1 98, 11 10V\nobreak{}«. 
\newline{}Schnitzler: mit Bleistift datiert: »25/1 98« \newline{}Ordnung: 1) mit Bleistift von unbekannter Hand nummeriert: »\strikeout{109}« 2) mit Bleistift von unbekannter Hand nummeriert: »106«}\buchAbdrucke{\weitereDrucke{Hugo von Hofmannsthal, Arthur Schnitzler: \emph{Briefwechsel}. Hg. Therese Nickl und Heinrich Schnitzler. Frankfurt am Main: \emph{S. Fischer} 1964, S. 98–99.} }\toendnotes[C]{\smallbreak}\pstart{}{\pb}\textsc{Herrn D\textsuperscript{r} Arthur Schnitzler}\pend{}\pstart{}\textcolor{pink}{\textsc{Franckgasse 1}\hspace*{2.5em}IX}{}\ledrightnote{\textcolor{pink}{Frankgasse}}\pend{}{\bigskip}\pstart
           \raggedleft{}{\pb}10\textsuperscript{h} früh\pend
           \pstart
           \textsc{\textcolor{blue}{Poldy}{}\ledrightnote{\textcolor{blue}{Leopold von Andrian-Werburg}}} iſt wegen »\label{K_L00769_1v}\edtext{mangelnder \textsc{Patellarreflexe}}{\lemma{\textnormal{\emph{mangelnder Patellarreflexe}}}\Cendnote{\textnormal{Durch leichten Schlag auf die
                        unterhalb der Kniescheibe befindliche Sehne wird ein Reflex ausgelöst. Das
                        Unterbleiben einer Reaktion kann auf eine Erkrankung des Nervensystems
                        verweisen.}}}\label{K_L00769_1h}« außer ſich und will durchaus ich ſoll Ihnen um die
                    »Wahrheit« telefonieren, ihm dann ſchreiben. Ich halte das für Zeitverluſt,
                    ſchreibe ihm beruhigend pneumatiſch, \uline{als ob ich sie
                        gefragt hätte}. Sollte er zu Ihnen kommen, ſo thuen Sie als ob ich
                    gefragt hätte. Sollte etwas zu ſagen ſein, was ich nicht glaube, bitte ſchreiben
                    Sie \uline{mir ſogleich}.\pend
           \pstart Ihr \spacefill\mbox{Hugo}\pend{}\endnumbering\briefempfaengerindex{Schnitzler, Arthur@\textsc{Schnitzler, Arthur}!zzzHofmannsthal, Hugo von@\emph{von Hugo von Hofmannsthal}!1898-01-254@{25. 1. 1898}|)be}\mylabel{h}  \normalsize

\doendnotes{C}
\bigskip
\vfill

\clearpage

\footnotesize

\lohead{\textsc{register}}

% Definiere theindex-Environment komplett neu ohne reledmac
\makeatletter
\renewenvironment{theindex}{%
  \section*{\indexname}%
  \setlength{\parindent}{0pt}%
  \setlength{\parskip}{0pt plus 0.3pt}%
  \let\item\@idxitem
}{%
  \clearpage
}
\makeatother

\IfFileExists{\jobname-pw.ind}{\input{\jobname-pw.ind}}{}

\end{document}

      