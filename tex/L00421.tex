%% latex-korrekturansicht-vorspann.tex
%% Vorspann für die Korrekturansicht.
%% Lädt die gemeinsame Datei latex-vorspann.tex mit gesetztem Schalter.

\newif\ifkorrekturansicht
\korrekturansichttrue

\input{../tex-inputs/latex-vorspann}


               \section[Richard Beer-Hofmann an Arthur Schnitzler, {[}17. 2. 1895?{]}]{ Richard Beer-Hofmann an Arthur Schnitzler, {[}17. 2. 1895?{]}}\nopagebreak\mylabel{v}\rehead{ }\normalsize\beginnumbering\briefempfaengerindex{Schnitzler, Arthur@\textsc{Schnitzler, Arthur}!zzzBeer-Hofmann, Richard@\emph{von Richard Beer-Hofmann}!1895-02-171@{{[}17. 2. 1895?{]}}|(be} \toendnotes[C]{\smallbreak\pagebreak[2]} \Standort{CUL, Schnitzler, B 8.}
\physDesc{Visitenkarte
\newline{}Handschrift: Bleistift, lateinische Kurrent
\newline{}Schnitzler: mit Bleistift datiert: »17/2 95.« und nummeriert: »556« }\buchAbdrucke{\weitereDrucke{Arthur Schnitzler, Richard Beer-Hofmann: \emph{Briefwechsel 1891–1931}. Hg. Konstanze Fliedl. Wien, Zürich: \emph{Europaverlag} 1992, S. 71.} }\toendnotes[C]{\smallbreak}\pstart
           \noindent{}{\pb}Lieber Arthur! Ich bin \label{K_L00421_1v}\edtext{heute}{\lemma{\textnormal{\emph{heute}}}\Cendnote{\textnormal{Obzwar von \textcolor{blue}{Schnitzler} datiert, sind Zweifel
                  anzumelden, da \textcolor{blue}{Beer-Hofmann} den Abend erst
                  recht in der Gesellschaft \textcolor{blue}{Schnitzler}s
                  verbrachte, eine Teilnahme \textcolor{blue}{Hofmannsthals}
                  wiederum nicht nachgewiesen werden kann.}}}\label{K_L00421_1h} Nachmittag zu Hause und, arbeite.
               Wegen des Herrn Hund’s werde ich kaum \strikeout{Nachmittag}
               Abends ins Gasthaus gehen können, weil das \textcolor{blue}{Stubenmädchen}{}\ledrightnote{→\textcolor{blue}{?? [Stubenfrau bei Richard Beer-Hofmann]}} weggeht. Wenn Sie und \textcolor{blue}{Hugo}{}\ledrightnote{\textcolor{blue}{Hugo von Hofmannsthal}}
               am Abend {\pb}vielleicht vorüber kommen
               schauen oder läuten Sie vielleicht zu mir herauf\pend
           \pstart
           herzlichst{\\[\baselineskip]}\spacefill\mbox{Richard}\pend
           \leftskip=0em{}\pstart
           \centering{}\label{T_L00421_1v}\edtext{\textcolor{gray}{\textbf{D\textsuperscript{r} Richard Beer-Hofmann}}}{\lemma{\textnormal{\emph{Dr Richard Beer-Hofmann}}}\Cendnote{\textnormal{Die Visitenkarte wurde so beschrieben, dass der
                  Aufdruck auf dem Kopf steht.}}}\label{T_L00421_1h}\pend
           \endnumbering\briefempfaengerindex{Schnitzler, Arthur@\textsc{Schnitzler, Arthur}!zzzBeer-Hofmann, Richard@\emph{von Richard Beer-Hofmann}!1895-02-171@{{[}17. 2. 1895?{]}}|)be}\mylabel{h}  \normalsize

\doendnotes{C}
\bigskip
\vfill

\clearpage

\footnotesize

\lohead{\textsc{register}}

% Definiere theindex-Environment komplett neu ohne reledmac
\makeatletter
\renewenvironment{theindex}{%
  \section*{\indexname}%
  \setlength{\parindent}{0pt}%
  \setlength{\parskip}{0pt plus 0.3pt}%
  \let\item\@idxitem
}{%
  \clearpage
}
\makeatother

\IfFileExists{\jobname-pw.ind}{\input{\jobname-pw.ind}}{}

\end{document}

      