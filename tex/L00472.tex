%% latex-korrekturansicht-vorspann.tex
%% Vorspann für die Korrekturansicht.
%% Lädt die gemeinsame Datei latex-vorspann.tex mit gesetztem Schalter.

\newif\ifkorrekturansicht
\korrekturansichttrue

\input{../tex-inputs/latex-vorspann}


               \section[Arthur Schnitzler an Lou Andreas-Salomé, 11. 8. 1895]{ Arthur Schnitzler an Lou Andreas-Salomé,
                    11. 8. 1895}\nopagebreak\mylabel{v}\rehead{ }\normalsize\beginnumbering\briefempfaengerindex{Andreas-Salome, Lou@\textsc{Andreas-Salomé, Lou}!zzzSchnitzler, Arthur@\emph{von Arthur Schnitzler}!1895-08-111@{11. 8. 1895}|(be} \toendnotes[C]{\smallbreak\pagebreak[2]} \Standort{Göttingen, Lou Andreas-Salomé Archiv, Schnitzler.}
\physDesc{Brief, 1 Blatt, 3 Seiten
\newline{}Handschrift: schwarze Tinte, deutsche Kurrent}\pstart{}{\pb}Verehrte Frau Lou,\pend\pstart
           es trifft ſich alles aufs beſte. Heute früh ko{\geminationm}’ ich
                    in \textcolor{pink}{Wien}{}\ledrightnote{\textcolor{pink}{Wien}} an, und \strikeout{treffe} finde Ihre lieben Zeilen, für die ich herzlich danke.\pend
           \pstart
           Ich fahre in 2 oder 3 Tagen nach \textcolor{pink}{Iſchl}{}\ledrightnote{\textcolor{pink}{Bad Ischl}}
                    und ko{\geminationm}e etwa 20. oder
                        21. nach \textcolor{pink}{Salzburg}{}\ledrightnote{\textcolor{pink}{Salzburg}}. Dort
                    einige Tage zugleich mit Ihnen und in Ihrer Geſellſchaft zu verbringen, freut
                        {\pb}mich ganz beſonders. Von \textcolor{pink}{S.}{}\ledrightnote{\textcolor{pink}{Salzburg}} aus fahre ich, wahrſcheinlich per Rad u auf einem
                    Umweg nach \textcolor{pink}{München}{}\ledrightnote{\textcolor{pink}{München}}. Es geht aus Ihrer
                    Karte nicht deutlich hervor, ob Sie \textcolor{pink}{München}{}\ledrightnote{\textcolor{pink}{München}} vor oder nach \textcolor{pink}{Salzburg}{}\ledrightnote{\textcolor{pink}{Salzburg}} zu beſuchen denken. Sollte das letztere der Fall ſein,
                    ſo wärs aber ganz beſonders ſchön.\pend
           \pstart
           In \textcolor{pink}{Iſchl}{}\ledrightnote{\textcolor{pink}{Bad Ischl}} wohne ich \textcolor{pink}{\textsc{Rudolfshöhe}}{}\ledrightnote{\textcolor{pink}{Hotel und Pension Rudolfshöhe (Leopold Petter)}}, {\pb}wo ich Nachricht von Ihnen vorzufinden hoffe. In
                        \textcolor{pink}{Salzb.}{}\ledrightnote{\textcolor{pink}{Salzburg}} werde ich wahrſcheinlich im
                        \textcolor{pink}{oesterr. Hof}{}\ledrightnote{\textcolor{pink}{Österreichischer Hof}} abſteigen. \textcolor{blue}{Richard}{}\ledrightnote{\textcolor{blue}{Richard Beer-Hofmann}} iſt wohl von den genauen \textcolor{pink}{Salzb.}{}\ledrightnote{\textcolor{pink}{Salzburg}} Daten gleichfalls in Ke{\geminationn}tnis geſetzt? – \pend
           \pstart
           Viele Grüße und auf angenehmes Wiederſehen!{\\[\baselineskip]}Ihr Sie
                        hochſch\textcolor{gray}{ätz}ender{\\[\baselineskip]}\spacefill\mbox{ArthSch}\pend
           \leftskip=0em{}\pstart
           11. 8. 95.{\\}\textcolor{pink}{Wien}{}\ledrightnote{\textcolor{pink}{Wien}}\pend
           \endnumbering\briefempfaengerindex{Andreas-Salome, Lou@\textsc{Andreas-Salomé, Lou}!zzzSchnitzler, Arthur@\emph{von Arthur Schnitzler}!1895-08-111@{11. 8. 1895}|)be}\mylabel{h}  \normalsize

\doendnotes{C}
\bigskip
\vfill

\clearpage

\footnotesize

\lohead{\textsc{register}}

% Definiere theindex-Environment komplett neu ohne reledmac
\makeatletter
\renewenvironment{theindex}{%
  \section*{\indexname}%
  \setlength{\parindent}{0pt}%
  \setlength{\parskip}{0pt plus 0.3pt}%
  \let\item\@idxitem
}{%
  \clearpage
}
\makeatother

\IfFileExists{\jobname-pw.ind}{\input{\jobname-pw.ind}}{}

\end{document}

      