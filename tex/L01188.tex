%% latex-korrekturansicht-vorspann.tex
%% Vorspann für die Korrekturansicht.
%% Lädt die gemeinsame Datei latex-vorspann.tex mit gesetztem Schalter.

\newif\ifkorrekturansicht
\korrekturansichttrue

\input{../tex-inputs/latex-vorspann}


               \section[Arthur Schnitzler an Richard Beer-Hofmann, 25. 11. 1901]{ Arthur Schnitzler an Richard Beer-Hofmann, 25. 11. 1901}\nopagebreak\mylabel{v}\rehead{ }\normalsize\beginnumbering\briefempfaengerindex{Beer-Hofmann, Richard@\textsc{Beer-Hofmann, Richard}!zzzSchnitzler, Arthur@\emph{von Arthur Schnitzler}!1901-11-251@{25. 11. 1901}|(be} \toendnotes[C]{\smallbreak\pagebreak[2]} \Standort{YCGL, MSS 31.}
\physDesc{Brief, 1 Blatt, 3 Seiten, Umschlag
\newline{}Handschrift: Bleistift, deutsche Kurrent\newline{}Versand: 1) Rohrpost 2) Stempel: »\nobreak{}\oindex{IX., Alsergrund@\textbf{IX., Alsergrund}, \emph{Bezirk (A.BZK)}|pwk}Wien 9/1, 25 XI 01, 3 50N\nobreak{}«. 3) Stempel: »\nobreak{}\oindex{I., Innere Stadt@\textbf{I., Innere Stadt}, \emph{Bezirk (A.BZK)}|pwk}{\pb}Wien 1/1, 25 XI 01, 4 10\textcolor{gray}{N}\nobreak{}«. }\buchAbdrucke{\weitereDrucke{Arthur Schnitzler, Richard Beer-Hofmann: \emph{Briefwechsel 1891–1931}. Hg. Konstanze Fliedl. Wien, Zürich: \emph{Europaverlag} 1992, S. 156–157.} }\toendnotes[C]{\smallbreak}\pstart{}{\pb}Herrn \textsc{Dr. Richard
                     Beer-Hofmann}\pend{}\pstart{}\textcolor{pink}{Wien}{}\ledrightnote{\textcolor{pink}{Wien}}\pend{}\pstart{}\textsc{\textcolor{pink}{I. Wollzeile 15}{}\ledrightnote{\textcolor{pink}{Wollzeile}}}\pend{}{\bigskip}\pstart
           \raggedleft{}{\pb}25. 11. 901\pend
           \pstart{}lieber Richard.\pend\pstart
           Ich war heute Vormittag bei \textcolor{blue}{Hugo}{}\ledrightnote{\textcolor{blue}{Hugo von Hofmannsthal}}.\pend
           \pstart
           Wollen Sie, daſs ich Ihnen beiden Mittwoch oder Donnerſtag{ }Nachmittag gegen 6 meine \label{K_L01188_1v}\edtext{\textcolor{green}{4 Stücke}{}\ledrightnote{→\textcolor{green}{Lebendige Stunden}{\newline}→\textcolor{green}{Die letzten Masken}{\newline}→\textcolor{green}{Literatur}{\newline}→\textcolor{green}{Die Frau mit dem Dolche}}
                  vorleſe}{\lemma{\textnormal{\emph{4 Stücke
                  vorleſe}}}\Cendnote{\textnormal{vgl. A. S.: \emph{Tagebuch}, 14. 12. 1901}}}\label{K_L01188_1h}? Wir (Sie u ich{[}){]}{ }{\pb}könnten dann Abends zuſammen
               hereinfahren. (Eventuell auch zuſa{\geminationm}en hinaus, wenn Sie
               nicht aus Wohnungsgründen früher draußen ſein müſſen.)\pend
           \pstart
           Alſo Mittwoch oder Donnerſtag oder in dieſer Woche gar
               nicht.\pend
           \pstart
           {\pb}Schreiben Sie mir, ich benachrichtige dann \textcolor{blue}{Hugo}{}\ledrightnote{\textcolor{blue}{Hugo von Hofmannsthal}}.\pend
           \pstart
           Herzlichst{\\[\baselineskip]}Ihr{\\[\baselineskip]}\spacefill\mbox{Arthur}\pend
           \leftskip=0em{}\endnumbering\briefempfaengerindex{Beer-Hofmann, Richard@\textsc{Beer-Hofmann, Richard}!zzzSchnitzler, Arthur@\emph{von Arthur Schnitzler}!1901-11-251@{25. 11. 1901}|)be}\mylabel{h}  \normalsize

\doendnotes{C}
\bigskip
\vfill

\clearpage

\footnotesize

\lohead{\textsc{register}}

% Definiere theindex-Environment komplett neu ohne reledmac
\makeatletter
\renewenvironment{theindex}{%
  \section*{\indexname}%
  \setlength{\parindent}{0pt}%
  \setlength{\parskip}{0pt plus 0.3pt}%
  \let\item\@idxitem
}{%
  \clearpage
}
\makeatother

\IfFileExists{\jobname-pw.ind}{\input{\jobname-pw.ind}}{}

\end{document}

      