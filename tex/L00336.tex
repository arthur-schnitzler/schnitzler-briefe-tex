%% latex-korrekturansicht-vorspann.tex
%% Vorspann für die Korrekturansicht.
%% Lädt die gemeinsame Datei latex-vorspann.tex mit gesetztem Schalter.

\newif\ifkorrekturansicht
\korrekturansichttrue

\input{../tex-inputs/latex-vorspann}


               \section[Arthur Schnitzler an Georg Brandes, 12. 6. 1894]{ Arthur Schnitzler an Georg Brandes, 12. 6. 1894}\nopagebreak\mylabel{v}\rehead{ }\normalsize\beginnumbering\briefempfaengerindex{Brandes, Georg@\textsc{Brandes, Georg}!zzzSchnitzler, Arthur@\emph{von Arthur Schnitzler}!1894-06-121@{12. 6. 1894}|(be} \toendnotes[C]{\smallbreak\pagebreak[2]} \Standort{Kopenhagen, Det Kongelige Bibliotek, Georg Brandes Arkiv, box 125.}
\physDesc{Brief, 2 Blätter, 7 Seiten
\newline{}Handschrift: schwarze Tinte, deutsche Kurrent\newline{}Ordnung: auf der ersten Seite mit
                                    Bleistift »Schnitzler« und Briefnummerierung:
                                        »1«, das zweite Blatt mit »12/6 94« gekennzeichnet }\buchAbdrucke{\weitereDrucke{1) Georg Brandes, Arthur Schnitzler: \emph{Ein Briefwechsel}. Hg. Kurt Bergel. Bern: \emph{Francke} 1956, S. 55–56.} \weitereDrucke{2) Arthur Schnitzler: \emph{Briefe 1875–1912}. Hg. Therese Nickl und Heinrich Schnitzler. Frankfurt am Main: \emph{S. Fischer} 1981, S. 225–227.} }\toendnotes[C]{\smallbreak}\pstart
           \raggedleft{}{\pb}\textcolor{pink}{\textsc{IX.
                            Frankgasse 1.}}{}\ledrightnote{\textcolor{pink}{Frankgasse}}{\\}\textsc{\textcolor{pink}{Wien}{}\ledrightnote{\textcolor{pink}{Wien}}, 12. Juni
                                94.}\pend
           \pstart\center{}Hochverehrter Herr,\pend\pstart
           es iſt nicht ſchwer ſich vorzuſtellen, wie viel Bücher Sie zugeſandt beko{\geminationm}en, und als ich mir erlaubte, Ihnen die \textcolor{green}{meinen}{}\ledrightnote{→\textcolor{green}{Das Märchen. Schauspiel in drei Aufzügen}{\newline}→\textcolor{green}{Anatol}} zu ſchicken, hab ich
                    natürlich gehofft – habe aber gewiſs nicht darauf gerechnet, daſs Sie Zeit und
                    Luſt haben würden, die Bücher eines ziemlich Unbeka{\geminationn}ten zu leſen. Und nun habe ich Ihren Brief beko{\geminationm}en, mit all dem liebens{\pb}würdigen und
                    ehrenvollen, das er enthält; und ich ka{\geminationn} Ihnen gar
                    nicht ſagen, eine wie tiefe Freude er mir bedeutet hat. Auf eine kurze Reiſe,
                    von der ich eben zurückgekehrt bin, hatte ich Ihr letztes mir unbeka{\geminationn}tes Buch »\textcolor{green}{Menſchen u
                        Werke}{}\ledrightnote{\textcolor{green}{Menschen und Werke}}« mitgeno{\geminationm}en. Ich bin es
                    gewohnt, Ihre Bücher mit der ſtillen Bewunderung zu leſen, die man großen und
                    fernen Geiſtern entgegen{\pb}bringt; diesmal
                    habe ich aber auch andres empfunden. Ich glaube, es war eine Art von Stolz. Mit
                    einem Male iſt meine Exiſtenz in das Bereich Ihres Schauens gerückt, und we{\geminationn} ich Ihnen ſage, daſs ich Sie verehre, ſo geht
                    meine Stimme nicht unter den tauſenden verloren, deren Namen Sie nicht kennen.
                    Dieſe vielleicht etwas hochmütige Empfindung blieb mir {\pb}von der erſten bis zur letzten Zeile, –
                    und, ich will es Ihnen nur geſtehn, ſie hat mir ſo wohl gethan, daſs ich mir
                    ſehr feſt vorgenommen habe, von Ihnen nicht wieder vergeſſen zu werden. Ihre
                    Worte, hochverehrter Herr, ſind mehr als Anerke{\geminationn}ung, Lob, Ermuthigung – ich betrachte ſie als Würde, die mir verliehen iſt; –
                    laſſen Sie mich Ihnen aufs innigſte dafür {\pb}danken.\pend
           \pstart
           Es iſt Ihnen, hochverehrter Herr, kaum beka{\geminationn}t
                    geworden, daſs »\textcolor{green}{Das Märchen}{}\ledrightnote{\textcolor{green}{Das Märchen. Schauspiel in drei Aufzügen}}« bereits
                    aufgeführt worden iſt. Man hat es in \textcolor{pink}{Wien}{}\ledrightnote{\textcolor{pink}{Wien}}, im
                        \textcolor{pink}{Deutſchen Volkstheater}{}\ledrightnote{\textcolor{pink}{Volkstheater}} gegeben. Die zwei
                    erſten Akte gefielen; der dritte misfiel ſo gründlich, daſs er das ganze Stück
                    mitriſs. Insbeſondere ſcheint man über die moraliſchen Qualitäten des Stückes
                    wenig erbaut geweſen zu ſein; – ein \textcolor{blue}{Kritiker}{}\ledrightnote{→\textcolor{blue}{Emil Granichstaedten}} rief mir zu: »\label{K_L00336_1v}\edtext{\textcolor{green}{Um {\pb}Reinlichkeit wird gebeten}{}\ledrightnote{→\textcolor{green}{Deutsches Volkstheater}}}{\lemma{\textnormal{\emph{Um … gebeten}}}\Cendnote{\textnormal{\textcolor{blue}{Emil Granichstaedten}: \emph{\textcolor{green}{Deutsches Volkstheater}}. In: \emph{\textcolor{green}{Die Presse}}, Jg. 46, Nr. 334,
                                3. 12. 1893, S. 1–2, hier S. 2.}}}\label{K_L00336_1h}«; ein \textcolor{blue}{anderer}{}\ledrightnote{→\textcolor{blue}{–r–}}{ }ſprach geradezu von der »\label{K_L00336_2v}\edtext{\textcolor{green}{wahrhaft erſchreckenden ſittlichen
                        Verwahrloſung}{}\ledrightnote{→\textcolor{green}{Deutsches Volkstheater}}}{\lemma{\textnormal{\emph{wahrhaft … Verwahrloſung}}}\Cendnote{\textnormal{\textcolor{blue}{–r–}: \emph{\textcolor{green}{(Deutsches Volkstheater.)}} In: \emph{\textcolor{green}{Das Vaterland}}, Jg. 34, Nr. 333,
                            2. 12. 1893, S. 7.}}}\label{K_L00336_2h}«, von der das Schauſpiel
                    Zeugnis gebe. Eine \label{K_L00336_3v}\edtext{\textcolor{pink}{Berliner Bühne}{}\ledrightnote{→\textcolor{pink}{Lessing-Theater}}}{\lemma{\textnormal{\emph{Berliner Bühne}}}\Cendnote{\textnormal{Das \emph{\textcolor{brown}{Lessing-Theater}} hatte es bereits im Dezember
                            1891 angenommen.}}}\label{K_L00336_3h}, die das \textcolor{green}{Märchen}{}\ledrightnote{\textcolor{green}{Das Märchen. Schauspiel in drei Aufzügen}}{ }ſchon angeno{\geminationm}en hatte, trat
                    auf den \textcolor{pink}{Wien}{}\ledrightnote{\textcolor{pink}{Wien}}er Miserfolg hin von \substVorne{}\textsuperscript{ſeiner}{\allowbreak}\substDazwischen{}ihrer\substHinten{} Verpflichtung zurück, und ſomit ka{\geminationn} ich
                    wohl die Bühnenlaufbahn dieſes Stückes als abgeſchlossen anſehn. – Ich {\pb}habe mich beinahe verpflichtet gefühlt,
                    Ihnen dieſe äußern Umſtände mitzutheilen, die mich anfangs wohl verſtimmt haben,
                    die ich aber bald als das betrachten konnte, was ſie ſind – als \uline{äußere} Umſtände. –\pend
           \pstart
           Nochmals, hochverehrter Herr, bitte ich Sie meiner tiefſten Dankbarkeit und
                    meiner unveränderlichen Bewunderung verſichert zu ſein,{\\[\baselineskip]}\spacefill\mbox{Arthur Schnitzler}\pend
           \leftskip=0em{}\endnumbering\briefempfaengerindex{Brandes, Georg@\textsc{Brandes, Georg}!zzzSchnitzler, Arthur@\emph{von Arthur Schnitzler}!1894-06-121@{12. 6. 1894}|)be}\mylabel{h}  \normalsize

\doendnotes{C}
\bigskip
\vfill

\clearpage

\footnotesize

\lohead{\textsc{register}}

% Definiere theindex-Environment komplett neu ohne reledmac
\makeatletter
\renewenvironment{theindex}{%
  \section*{\indexname}%
  \setlength{\parindent}{0pt}%
  \setlength{\parskip}{0pt plus 0.3pt}%
  \let\item\@idxitem
}{%
  \clearpage
}
\makeatother

\IfFileExists{\jobname-pw.ind}{\input{\jobname-pw.ind}}{}

\end{document}

      