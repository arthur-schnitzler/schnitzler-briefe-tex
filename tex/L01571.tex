%% latex-korrekturansicht-vorspann.tex
%% Vorspann für die Korrekturansicht.
%% Lädt die gemeinsame Datei latex-vorspann.tex mit gesetztem Schalter.

\newif\ifkorrekturansicht
\korrekturansichttrue

\input{../tex-inputs/latex-vorspann}


               \section[Hugo von Hofmannsthal an Arthur Schnitzler, 2. 1. 1906]{ Hugo von Hofmannsthal an Arthur Schnitzler, 2. 1. 1906}\nopagebreak\mylabel{v}\rehead{ }\normalsize\beginnumbering\briefempfaengerindex{Schnitzler, Arthur@\textsc{Schnitzler, Arthur}!zzzHofmannsthal, Hugo von@\emph{von Hugo von Hofmannsthal}!1906-01-021@{2. 1. 1906}|(be} \toendnotes[C]{\smallbreak\pagebreak[2]} \Standort{CUL, Schnitzler, B 43.}
\physDesc{Postkarte
\newline{}Handschrift: schwarze Tinte, deutsche Kurrent\newline{}Versand: 1) Stempel: »\nobreak{}\oindex{Kaltenleutgeben@\textbf{Kaltenleutgeben}, \emph{https://www.geonames.org/ontologyP.PPLA3}|pwk}{[}Kal{]}tenleutgeben, 2. 1. {[}1906{]}\nobreak{}«.  2) Stempel: »\nobreak{}\oindex{XVIII., Waehring@\textbf{XVIII., Währing}, \emph{Bezirk (A.BZK)}|pwk}18/1 {[}Wie{]}n, 3. 1. 06, 8.V\nobreak{}«. 
\newline{}Schnitzler: mit Bleistift datiert: »3/1 906« \newline{}Ordnung: 1) mit Bleistift von unbekannter Hand nummeriert: »\strikeout{220}« 2) mit Bleistift von unbekannter Hand nummeriert: »\strikeout{216}«3) mit Bleistift von unbekannter Hand nummeriert: »259«}\buchAbdrucke{\weitereDrucke{Hugo von Hofmannsthal, Arthur Schnitzler: \emph{Briefwechsel}. Hg. Therese Nickl und Heinrich Schnitzler. Frankfurt am Main: \emph{S. Fischer} 1964, S. 225.} }\toendnotes[C]{\smallbreak}\pstart{}{\pb}\textcolor{gray}{\textbf{Absender:}}\pend{}\pstart{}\textsc{\textcolor{blue}{Sophokles}{}\ledrightnote{→\textcolor{blue}{Sophokles}}}.\pend{}{\bigskip}\pstart{}\textsc{Herrn D\textsuperscript{r} Arthur Schnitzler}\pend{}\pstart{}\textcolor{pink}{\textsc{Wien}}{}\ledrightnote{\textcolor{pink}{Wien}}\pend{}\pstart{}\textcolor{pink}{\textsc{XVIII. Spöttelgasse 7}.}{}\ledrightnote{\textcolor{pink}{Edmund-Weiß-Gasse}}\pend{}{\bigskip}\pstart
           \noindent{}{\pb}lieber, bitte
               ſchreiben Sie mir doch 2 Worte über das \label{K_L01571_1v}\edtext{\textcolor{green}{Stück}{}\ledrightnote{→\textcolor{green}{Der Jäger}}}{\lemma{\textnormal{\emph{Stück}}}\Cendnote{\textnormal{\emph{\textcolor{green}{Der Jäger}} blieb in dieser Gestalt unveröffentlicht und wurde,
                  zur Novelle umgearbeitet, 1912 publiziert.}}}\label{K_L01571_1h} von \textcolor{blue}{Michel}{}\ledrightnote{\textcolor{blue}{Robert Michel}}, ſchicken es aber bitte nicht an mich
               zurück ſondern gleich an ihn: \pend
           \pstart
           \textsc{Oberleutnant \textcolor{blue}{Robert Michel}{}\ledrightnote{\textcolor{blue}{Robert Michel}}}\pend
           \pstart
           \textcolor{pink}{\textsc{Innsbruck}}{}\ledrightnote{\textcolor{pink}{Innsbruck}}\pend
           \pstart
           \textsc{Infanterie Cadettenschule.}\pend
           \pstart \spacefill\mbox{Hugo.}\pend{}\pstart
           \noindent{}2 I.\pend
           \endnumbering\briefempfaengerindex{Schnitzler, Arthur@\textsc{Schnitzler, Arthur}!zzzHofmannsthal, Hugo von@\emph{von Hugo von Hofmannsthal}!1906-01-021@{2. 1. 1906}|)be}\mylabel{h}  \normalsize

\doendnotes{C}
\bigskip
\vfill

\clearpage

\footnotesize

\lohead{\textsc{register}}

% Definiere theindex-Environment komplett neu ohne reledmac
\makeatletter
\renewenvironment{theindex}{%
  \section*{\indexname}%
  \setlength{\parindent}{0pt}%
  \setlength{\parskip}{0pt plus 0.3pt}%
  \let\item\@idxitem
}{%
  \clearpage
}
\makeatother

\IfFileExists{\jobname-pw.ind}{\input{\jobname-pw.ind}}{}

\end{document}

      