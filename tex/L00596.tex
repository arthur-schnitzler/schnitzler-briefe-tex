%% latex-korrekturansicht-vorspann.tex
%% Vorspann für die Korrekturansicht.
%% Lädt die gemeinsame Datei latex-vorspann.tex mit gesetztem Schalter.

\newif\ifkorrekturansicht
\korrekturansichttrue

\input{../tex-inputs/latex-vorspann}


               \section[Arthur Schnitzler an Richard Beer-Hofmann, 21. 9. 1896]{ Arthur Schnitzler an Richard Beer-Hofmann, 21. 9. 1896}\nopagebreak\mylabel{v}\rehead{ }\normalsize\beginnumbering\briefempfaengerindex{Beer-Hofmann, Richard@\textsc{Beer-Hofmann, Richard}!zzzSchnitzler, Arthur@\emph{von Arthur Schnitzler}!1896-09-211@{21. 9. 1896}|(be} \toendnotes[C]{\smallbreak\pagebreak[2]} \Standort{YCGL, MSS 31.}
\physDesc{Brief, 1 Blatt, 4 Seiten, Umschlag
\newline{}Handschrift: Bleistift, deutsche Kurrent\newline{}Versand: 1) Stempel: »\nobreak{}\oindex{IX., Alsergrund@\textbf{IX., Alsergrund}, \emph{Bezirk (A.BZK)}|pwk}Wien 9/3, 21. 9. 96, 3–4N\nobreak{}«.  2) Stempel: »\nobreak{}\oindex{Baden bei Wien@\textbf{Baden bei Wien}, \emph{Besiedelter Ort (A.BSO)}|pwk}Baden, 22. 9. 96, 7–10V, Bestellt\nobreak{}«. 3) Stempel: »\nobreak{}\oindex{I., Innere Stadt@\textbf{I., Innere Stadt}, \emph{Bezirk (A.BZK)}|pwk}{[}Wie{]}n 1/1, 22. 9. 96, 3–4½N, {[}Be{]}stellt\nobreak{}«. 4) von
                           unbekannter Hand nachgesandt nach \textcolor{pink}{Wien}, \textcolor{pink}{I Wollzeile 15}}\buchAbdrucke{\weitereDrucke{Arthur Schnitzler, Richard Beer-Hofmann: \emph{Briefwechsel 1891–1931}. Hg. Konstanze Fliedl. Wien, Zürich: \emph{Europaverlag} 1992, S. 98–99.} }\toendnotes[C]{\smallbreak}\pstart{}{\pb}Herrn Doctor \textsc{Rich.
                     Beer-Hofmann}\pend{}\pstart{}\textsc{\textcolor{pink}{Baden bei Wien}{}\ledrightnote{\textcolor{pink}{Baden bei Wien}}.}\pend{}\pstart{}\textcolor{pink}{Franzensgaſſe 54}{}\ledrightnote{\textcolor{pink}{Kaiser-Franz-Ring}}, Th. 8.\pend{}{\bigskip}\pstart
           \noindent{}{\pb}Lieber Richard, gerade wie ich die Sitze nehmen wollte,
                  treff\textcolor{gray}{e} ich \textcolor{blue}{Dörma{\geminationn}}{}\ledrightnote{\textcolor{blue}{Felix Dörmann}} der eben einen Brief erhalten (ich las den Brief) daſs \textcolor{green}{Sein Sohn}{}\ledrightnote{\textcolor{green}{Sein Sohn}} auf \label{K_L00596_1v}\edtext{unbesti{\geminationm}te Zeit}{\lemma{\textnormal{\emph{unbestite Zeit}}}\Cendnote{\textnormal{\textcolor{blue}{Hugo Ranzenberg} starb am
                     21. 9. 1896, die Uraufführung fand dann am
                     16. 10. 1896 statt.}}}\label{K_L00596_1h} verſchoben wegen {\pb}Erkrankung \textcolor{blue}{Ranzenberg}{}\ledrightnote{\textcolor{blue}{Hugo Ranzenberg}}s. –\pend
           \pstart
           Am Mittwoch{ }Abend hole ich Sie gegen acht ab; ich werde unten
               läuten. –\pend
           \pstart
           Im übrigen könnte man auch ein Stück in 9 Akten ſchreiben, \textcolor{green}{Märchen}{}\ledrightnote{\textcolor{green}{Das Märchen. Schauspiel in drei Aufzügen}}, \textcolor{green}{Liebelei}{}\ledrightnote{\textcolor{green}{Liebelei. Schauspiel in drei Akten}}, u \textcolor{green}{Freiwild}{}\ledrightnote{\textcolor{green}{Freiwild. Schauspiel in 3 Akten}} zuſa{\geminationm}en. Nur
               kleine Aenderungen {\pb}wären nothwendig, der alte \textcolor{green}{Geiger}{}\ledrightnote{→\textcolor{green}{Liebelei. Schauspiel in drei Akten}} wär eine alte Geigerin (bei einer
               Damenkapelle) als Mutter der \label{T_L00596_1v}\edtext{\textcolor{green}{Fanny}{}\ledrightnote{→\textcolor{green}{Das Märchen. Schauspiel in drei Aufzügen}}–\textcolor{green}{Chriſtine}{}\ledrightnote{→\textcolor{green}{Liebelei. Schauspiel in drei Akten}}–\textcolor{green}{Anna}{}\ledrightnote{→\textcolor{green}{Freiwild. Schauspiel in 3 Akten}}}{\lemma{\textnormal{\emph{Fanny–Chriſtine–Anna}}}\Cendnote{\textnormal{Eine geschwungene Klammer oberhalb
                  verbindet die Namen und scheint sie der Damenkapelle zuzuordnen.}}}\label{T_L00596_1h}, der
               Doctor \textcolor{green}{Witte}{}\ledrightnote{→\textcolor{green}{Das Märchen. Schauspiel in drei Aufzügen}} wär \substVorne{}\textsuperscript{d}\substDazwischen{}n\substHinten{}ahe daran, ſeine Praxis niederzulegen weil ſich der \textcolor{green}{Fedor Denner}{}\ledrightnote{→\textcolor{green}{Das Märchen. Schauspiel in drei Aufzügen}} nicht mit ihm ſchlagen will, und
                  {\pb}der \textcolor{green}{Moritzki}{}\ledrightnote{→\textcolor{green}{Freiwild. Schauspiel in 3 Akten}} wäre vom Direktor \textcolor{green}{Schneider}{}\ledrightnote{→\textcolor{green}{Freiwild. Schauspiel in 3 Akten}} ins Haus der alten Geigerin geſandt. –\pend
           \pstart
           \textcolor{green}{Die Athenerin}{}\ledrightnote{\textcolor{green}{Die Athenerin}} hat großen Erfolg gehabt, und \textcolor{blue}{Bauer}{}\ledrightnote{\textcolor{blue}{Julius Bauer}} war bei der Première aufgeregter als der \textcolor{blue}{Autor}{}\ledrightnote{→\textcolor{blue}{Leo Ebermann}}, (wie er \introOben{}(\textcolor{blue}{B.}{}\ledrightnote{\textcolor{blue}{Julius Bauer}})\introOben{} ſelbſt im Parquet
               erzählte). –\pend
           \pstart
           Herzlich Ihr{\\[\baselineskip]}\spacefill\mbox{Arthur}\pend
           \leftskip=0em{}\endnumbering\briefempfaengerindex{Beer-Hofmann, Richard@\textsc{Beer-Hofmann, Richard}!zzzSchnitzler, Arthur@\emph{von Arthur Schnitzler}!1896-09-211@{21. 9. 1896}|)be}\mylabel{h}  \normalsize

\doendnotes{C}
\bigskip
\vfill

\clearpage

\footnotesize

\lohead{\textsc{register}}

% Definiere theindex-Environment komplett neu ohne reledmac
\makeatletter
\renewenvironment{theindex}{%
  \section*{\indexname}%
  \setlength{\parindent}{0pt}%
  \setlength{\parskip}{0pt plus 0.3pt}%
  \let\item\@idxitem
}{%
  \clearpage
}
\makeatother

\IfFileExists{\jobname-pw.ind}{\input{\jobname-pw.ind}}{}

\end{document}

      