%% latex-korrekturansicht-vorspann.tex
%% Vorspann für die Korrekturansicht.
%% Lädt die gemeinsame Datei latex-vorspann.tex mit gesetztem Schalter.

\newif\ifkorrekturansicht
\korrekturansichttrue

\input{../tex-inputs/latex-vorspann}


               \section[Peter Altenberg an Arthur Schnitzler, {[}17. 4. 1909{]}]{ Peter Altenberg an Arthur Schnitzler, {[}17. 4. 1909{]}}\nopagebreak\mylabel{v}\rehead{ }\normalsize\beginnumbering\briefempfaengerindex{Schnitzler, Arthur@\textsc{Schnitzler, Arthur}!zzzAltenberg, Peter@\emph{von Peter Altenberg}!1909-04-171@{{[}17. 4. 1909{]}}|(be} \toendnotes[C]{\smallbreak\pagebreak[2]} \Standort{CUL, Schnitzler, B 2.}
\physDesc{Brief, 1 Blatt, 1 Seite
\newline{}Handschrift: schwarze Tinte, deutsche Kurrent
\newline{}Schnitzler: mit Bleistift beschriftet: »\textsc{Altenbg}« und datiert: »17/4 09« \newline{}Ordnung: mit Bleistift von unbekannter Hand nummeriert:
                                 »8« }\buchAbdrucke{\weitereDrucke{Kurt Bergel: \emph{Arthur Schnitzlers unveröffentlichte Tragikomödie Das Wort.} In: \emph{Studies in Arthur Schnitzler. Centennial Commemorative
                        Volume}. Hg. Herbert W. Reichert und Herman Salinger. Chapel Hill: \emph{University of North Carolina Press} 1963, S. 21 (UNC Studies in the Germanic Languages and Literatures, 42).} }\toendnotes[C]{\smallbreak}\pstart{}{\pb}Lieber \textsc{D}\textsuperscript{r} Arthur Schnitzler,\pend\pstart
           wenn Sie mein zerfahrenes unruhiges verkommenes Leben auch nur annähernd kennen
               könnten, würden Sie ſich nicht wundern, daſs ich Ihnen erſt heute für Ihr wunderbares
                  \label{K_L01839_1v}\edtext{Schreiben}{\lemma{\textnormal{\emph{Schreiben}}}\Cendnote{\textnormal{vgl. A. S.: \emph{Tagebuch}, 24. 1. 1909. Der Geburtstag war
                  am 9. 3. 1909.}}}\label{K_L01839_1h} danke.\pend
           \pstart
           Ich kann es ruhig ſagen, ich bin, bei meinem eng umgrenzten Talentchen, voll und ganz
               gewürdigt worden, alſo eigentlich ein beſonderes Gnadengeſchenk des in anderen
               Angelegenheiten heimtückiſchen Schickſals!\pend
           \pstart
           Mit herzlichſtem Gruße an Ihre edle \textcolor{blue}{Frau}{}\ledrightnote{→\textcolor{blue}{Olga Schnitzler}}\pend
           \pstart Ihr\hspace*{1.5em}\spacefill\mbox{Peter Altenberg}\pend{}\endnumbering\briefempfaengerindex{Schnitzler, Arthur@\textsc{Schnitzler, Arthur}!zzzAltenberg, Peter@\emph{von Peter Altenberg}!1909-04-171@{{[}17. 4. 1909{]}}|)be}\mylabel{h}  \normalsize

\doendnotes{C}
\bigskip
\vfill

\clearpage

\footnotesize

\lohead{\textsc{register}}

% Definiere theindex-Environment komplett neu ohne reledmac
\makeatletter
\renewenvironment{theindex}{%
  \section*{\indexname}%
  \setlength{\parindent}{0pt}%
  \setlength{\parskip}{0pt plus 0.3pt}%
  \let\item\@idxitem
}{%
  \clearpage
}
\makeatother

\IfFileExists{\jobname-pw.ind}{\input{\jobname-pw.ind}}{}

\end{document}

      