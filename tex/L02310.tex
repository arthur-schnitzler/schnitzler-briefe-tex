%% latex-korrekturansicht-vorspann.tex
%% Vorspann für die Korrekturansicht.
%% Lädt die gemeinsame Datei latex-vorspann.tex mit gesetztem Schalter.

\newif\ifkorrekturansicht
\korrekturansichttrue

\input{../tex-inputs/latex-vorspann}


               \section[Arthur Schnitzler an Robert Adam, 13. 11. 1918]{ Arthur Schnitzler an Robert Adam, 13. 11. 1918}\nopagebreak\mylabel{v}\rehead{ }\normalsize\beginnumbering\briefempfaengerindex{Adam, Robert@\textsc{Adam, Robert}!zzzSchnitzler, Arthur@\emph{von Arthur Schnitzler}!1918-11-131@{13. 11. 1918}|(be} \toendnotes[C]{\smallbreak\pagebreak[2]} \Standort{DLA, 96.34.2/15.}
\physDesc{Briefkarte, Umschlag
\newline{}Schreibmaschine
\newline{}Handschrift: schwarze Tinte, lateinische Kurrent (\noindent{}Korrekturen,
                                        Grußformel und Unterschrift)\newline{}Versand: Stempel: »\nobreak{}\textcolor{gray}{13.} XI. 18, 3\nobreak{}«.  }\toendnotes[C]{\smallbreak}\pstart{}{\pb}\textcolor{gray}{\textbf{D\textsuperscript{R} ARTHUR
                                SCHNITZLER}}\pend{}\pstart{}\textcolor{gray}{\textbf{\textcolor{pink}{WIEN, XVIII. STERNWARTESTRASSE 71}{}\ledrightnote{\textcolor{pink}{Sternwartestraße}}.}}\pend{}{\bigskip}\pstart{}{\pb}Herrn\pend{}\pstart{}Landesgerichtsrat Dr. Robert Adam-Pollak\pend{}\pstart{}\textcolor{pink}{Wien XII}{}\ledrightnote{\textcolor{pink}{XII., Meidling}}.\pend{}\pstart{}\textcolor{pink}{Meidlinger Hauptstrasse 56}{}\ledrightnote{\textcolor{pink}{Meidlinger Hauptstraße}}.\pend{}{\bigskip}\pstart
           {\pb}\textcolor{gray}{\textbf{D\textsuperscript{R} ARTHUR
                                    SCHNITZLER}}\hfill 13. 11. 1918\pend
           \pstart
           \textcolor{gray}{\textbf{\textcolor{pink}{WIEN, XVIII. STERNWARTESTRASSE 71}{}\ledrightnote{\textcolor{pink}{Sternwartestraße}}.}}\pend
           \pstart{}Lieber und verehrter Herr Doktor.\pend\pstart
           Man ist im \textcolor{pink}{Deutschen Volkstheater}{}\ledrightnote{\textcolor{pink}{Volkstheater}} auf die
                    Einsendung Ihrer \textcolor{green}{Stücke}{}\ledrightnote{→\textcolor{green}{Yppl. Idylle in fünf Akten}{\newline}→\textcolor{green}{Der Fremde}} vorbereitet \substVorne{}\textsuperscript{. Man}\substDazwischen{}und\substHinten{} hat mir zugesagt sie sofort und mit aller Aufmerksamkeit zu lesen.
                    Vielleicht senden Sie sowohl den »\textcolor{green}{Fremden}{}\ledrightnote{\textcolor{green}{Der Fremde}}« als
                    auch »\textcolor{green}{\substVorne{}\textsuperscript{Ue}\substDazwischen{}Y\substHinten{}ppel}{}\ledrightnote{\textcolor{green}{Yppl. Idylle in fünf Akten}}« ein und beziehen sich mit ein paar Worten auf meine
                    Rücksprache in der Direktion. – Auf baldiges Wiedersehen und herzliche
                    Grüsse.\pend
           \pstart
           {[}hs.:{]} Ihr{\\[\baselineskip]}\spacefill\mbox{Arthur Schnitzler}\pend
           \leftskip=0em{}\endnumbering\briefempfaengerindex{Adam, Robert@\textsc{Adam, Robert}!zzzSchnitzler, Arthur@\emph{von Arthur Schnitzler}!1918-11-131@{13. 11. 1918}|)be}\mylabel{h}  \normalsize

\doendnotes{C}
\bigskip
\vfill

\clearpage

\footnotesize

\lohead{\textsc{register}}

% Definiere theindex-Environment komplett neu ohne reledmac
\makeatletter
\renewenvironment{theindex}{%
  \section*{\indexname}%
  \setlength{\parindent}{0pt}%
  \setlength{\parskip}{0pt plus 0.3pt}%
  \let\item\@idxitem
}{%
  \clearpage
}
\makeatother

\IfFileExists{\jobname-pw.ind}{\input{\jobname-pw.ind}}{}

\end{document}

      