%% latex-korrekturansicht-vorspann.tex
%% Vorspann für die Korrekturansicht.
%% Lädt die gemeinsame Datei latex-vorspann.tex mit gesetztem Schalter.

\newif\ifkorrekturansicht
\korrekturansichttrue

\input{../tex-inputs/latex-vorspann}


               \section[Hugo von Hofmannsthal an Arthur Schnitzler, {[}17. 2. 1892{]}]{ Hugo von Hofmannsthal an Arthur Schnitzler, {[}17. 2. 1892{]}}\nopagebreak\mylabel{v}\rehead{ }\normalsize\beginnumbering\briefempfaengerindex{Schnitzler, Arthur@\textsc{Schnitzler, Arthur}!zzzHofmannsthal, Hugo von@\emph{von Hugo von Hofmannsthal}!1892-02-171@{{[}17. 2. 1892{]}}|(be} \toendnotes[C]{\smallbreak\pagebreak[2]} \Standort{CUL, Schnitzler, B 43.}
\physDesc{Brief, 1 Blatt, 1 Seite
\newline{}Handschrift: schwarze Tinte, deutsche Kurrent
\newline{}Schnitzler: mit Bleistift datiert: »17/2 92« \newline{}Ordnung: mit Bleistift von unbekannter Hand nummeriert:
                                        »17« }\buchAbdrucke{\weitereDrucke{1) Hugo von Hofmannsthal, Arthur Schnitzler: \emph{Briefwechsel}. Hg. Therese Nickl und Heinrich Schnitzler. Frankfurt am Main: \emph{S. Fischer} 1964, S. 16.} \weitereDrucke{2) Hermann Bahr, Arthur Schnitzler: \emph{Briefwechsel, Aufzeichnungen, Dokumente
                                (1891–1931)}. Hg. Kurt Ifkovits und Martin Anton Müller. Göttingen: \emph{Wallstein} 2018, S. 21.} }\toendnotes[C]{\smallbreak}\pstart
           \noindent{}\centering{}{\pb}Thatsachen: \pend
           \pstart
           \noindent{}1.) Bitte adreſſieren \textcolor{blue}{Sie}{}\ledrightnote{→} den beiliegenden
                    Wiſch an Herrn \textcolor{blue}{Lothar}{}\ledrightnote{\textcolor{blue}{Rudolf Lothar}} und ſchicken Sie ihn
                    weg.\pend
           \pstart
           2.) \textcolor{blue}{Maeterlinck}{}\ledrightnote{\textcolor{blue}{Maurice Maeterlinck}} hat mich zur Überſetzung
                    freundlichſt autoriſiert.\pend
           \pstart
           3.) Die Empfehlung an die \textcolor{blue}{Palmay}{}\ledrightnote{\textcolor{blue}{Ilka Pálmay}} habe ich
                    verlangt und werde ſie \textcolor{blue}{Bahr}{}\ledrightnote{\textcolor{blue}{Hermann Bahr}} nächſtens
                    ſchicken.\pend
           \pstart
           4.) Vielleicht könnte \textcolor{blue}{Kafka}{}\ledrightnote{\textcolor{blue}{Eduard Michael Kafka}} die erſten
                    Vierteljahrsbeiträge raſch einkaſſieren und uns gegen Garantie durch perſönliche
                    Unterſchrift leihen. Das wären doch vielleicht 200 fl.\pend
           \pstart
           5.) Suchen Sie \textcolor{blue}{Bauer}{}\ledrightnote{\textcolor{blue}{Arnold Bauer}} gegenüber uns wichtig
                    und ernſt zu machen und trachten Sie, \introOben{}daß\introOben{} das erſte \textcolor{brown}{Heft}{}\ledrightnote{→\textcolor{brown}{Wiener Literatur-Zeitung}} möglichſt bald erſcheint. An
                    die Premièren: \textcolor{blue}{Fulda}{}\ledrightnote{\textcolor{blue}{Ludwig Fulda}} »\textcolor{green}{Sclavin}{}\ledrightnote{\textcolor{green}{Die Sklavin. Schauspiel in vier Aufzügen}}«, \textcolor{green}{\textsc{Griselidis}}{}\ledrightnote{\textcolor{green}{Grisélidis. Oper in drei Akten und einem Prolog}} und \textcolor{blue}{Schleſinger}{}\ledrightnote{\textcolor{blue}{Sigmund Schlesinger}} »\textcolor{green}{\textsc{Derby}}{}\ledrightnote{\textcolor{green}{Derby}}« läſst ſich künſtleriſch und ſocial unendlich viel anhängen.\pend
           \pstart \spacefill\mbox{Loris.}\pend{}\endnumbering\briefempfaengerindex{Schnitzler, Arthur@\textsc{Schnitzler, Arthur}!zzzHofmannsthal, Hugo von@\emph{von Hugo von Hofmannsthal}!1892-02-171@{{[}17. 2. 1892{]}}|)be}\mylabel{h}  \normalsize

\doendnotes{C}
\bigskip
\vfill

\clearpage

\footnotesize

\lohead{\textsc{register}}

% Definiere theindex-Environment komplett neu ohne reledmac
\makeatletter
\renewenvironment{theindex}{%
  \section*{\indexname}%
  \setlength{\parindent}{0pt}%
  \setlength{\parskip}{0pt plus 0.3pt}%
  \let\item\@idxitem
}{%
  \clearpage
}
\makeatother

\IfFileExists{\jobname-pw.ind}{\input{\jobname-pw.ind}}{}

\end{document}

      