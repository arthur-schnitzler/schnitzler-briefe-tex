%% latex-korrekturansicht-vorspann.tex
%% Vorspann für die Korrekturansicht.
%% Lädt die gemeinsame Datei latex-vorspann.tex mit gesetztem Schalter.

\newif\ifkorrekturansicht
\korrekturansichttrue

\input{../tex-inputs/latex-vorspann}


               \section[Arthur Schnitzler an Hugo von Hofmannsthal, 20. 5. 1897]{ Arthur Schnitzler an Hugo von Hofmannsthal, 20. 5. 1897}\nopagebreak\mylabel{v}\rehead{ }\normalsize\beginnumbering\briefempfaengerindex{Hofmannsthal, Hugo von@\textsc{Hofmannsthal, Hugo von}!zzzSchnitzler, Arthur@\emph{von Arthur Schnitzler}!1897-05-202@{20. 5. 1897}|(be} \toendnotes[C]{\smallbreak\pagebreak[2]} \Standort{FDH, Hs-30885,12.}
\physDesc{Brief, 1 Blatt, 4 Seiten
\newline{}Handschrift: schwarze Tinte, deutsche Kurrent}\buchAbdrucke{\weitereDrucke{Hugo von Hofmannsthal, Arthur Schnitzler: \emph{Briefwechsel}. Hg. Therese Nickl und Heinrich Schnitzler. Frankfurt am Main: \emph{S. Fischer} 1964, S. 86–87.} }\toendnotes[C]{\smallbreak}\pstart
           \raggedleft{}{\pb}\textcolor{pink}{\textsc{Paris}}{}\ledrightnote{\textcolor{pink}{Paris}}{ }20. 5. 97\pend
           \pstart
           Mein lieber Hugo, Sagen Sie, haben Sie alle meine Briefe
                    bekommen? Dieſer iſt der \uline{vierte}.\pend
           \pstart
           Ich reiſe Montag von hier nach \textcolor{pink}{London}{}\ledrightnote{\textcolor{pink}{London}}; meine
                    Adreſſe dort: bei \textsc{\textcolor{blue}{Felix Markbreiter}{}\ledrightnote{\textcolor{blue}{Felix Markbreiter}}, \textcolor{pink}{London S. E. Woodville Hall, Honor Oak}{}\ledrightnote{\textcolor{pink}{Honor Oak}}.}\pend
           \pstart
           Um den erſten herum bin ich in \textcolor{pink}{Wien}{}\ledrightnote{\textcolor{pink}{Wien}}.
                    Es war ſehr geſcheit, daſs ich fortgefahren bin; für {\pb}das gegenwärtige ſicher; aber es wird ſicher auch für die Zukunft was zu
                    bedeuten ha\substVorne{}\textsuperscript{tt}\substDazwischen{}b\substHinten{}en, wenn nicht alles Erleben Unſinn iſt. Man weiſs ja nie, was man von
                    irgendwoher mitni{\geminationm}t; wenn man den Koffer auspackt,
                    ſo wundert man ſich über die ſchönen Dinge, die man ſich gar nicht mehr erinnern
                        {\pb}kann hineingeſtopft zu haben.\pend
           \pstart
           – Ich freue mich ſehr, dſs ich Sie noch in \textcolor{pink}{Wien}{}\ledrightnote{\textcolor{pink}{Wien}}
                    finde. Werden wir miteinander Radfahren? – – Rieſengebirge? Und wie wär es im Auguſt mit ein
                    paar \textcolor{pink}{Bayreuth}{}\ledrightnote{\textcolor{pink}{Bayreuth}}er Tagen? \textcolor{blue}{Goldmann}{}\ledrightnote{\textcolor{blue}{Paul Goldmann}} wird wohl nach \textcolor{pink}{Iſchl}{}\ledrightnote{\textcolor{pink}{Bad Ischl}} kommen, möchte auch gern nach \textcolor{pink}{Bay{\pb}reuth}{}\ledrightnote{\textcolor{pink}{Bayreuth}}. Bitte ſagen Sie das dem \textcolor{blue}{Richard}{}\ledrightnote{\textcolor{blue}{Richard Beer-Hofmann}}, ich hab vergeſſen ihm das zu
                    ſchreiben. –\pend
           \pstart
           – Nach dem Arbeiten glaub ich hab ich mich in meinem ganzen Leben nicht ſo
                    geſehnt wie jetzt! –\pend
           \pstart
           Bitte grüßen Sie Ihre \textcolor{blue}{Eltern}{}\ledrightnote{→\textcolor{blue}{Hugo August von Hofmannsthal}{\newline}→\textcolor{blue}{Anna von Hofmannsthal}} von mir.\pend
           \pstart
           Herzlich der Ihre{\\[\baselineskip]}\spacefill\mbox{Arthur.}\pend
           \leftskip=0em{}\endnumbering\briefempfaengerindex{Hofmannsthal, Hugo von@\textsc{Hofmannsthal, Hugo von}!zzzSchnitzler, Arthur@\emph{von Arthur Schnitzler}!1897-05-202@{20. 5. 1897}|)be}\mylabel{h}  \normalsize

\doendnotes{C}
\bigskip
\vfill

\clearpage

\footnotesize

\lohead{\textsc{register}}

% Definiere theindex-Environment komplett neu ohne reledmac
\makeatletter
\renewenvironment{theindex}{%
  \section*{\indexname}%
  \setlength{\parindent}{0pt}%
  \setlength{\parskip}{0pt plus 0.3pt}%
  \let\item\@idxitem
}{%
  \clearpage
}
\makeatother

\IfFileExists{\jobname-pw.ind}{\input{\jobname-pw.ind}}{}

\end{document}

      