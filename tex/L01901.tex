%% latex-korrekturansicht-vorspann.tex
%% Vorspann für die Korrekturansicht.
%% Lädt die gemeinsame Datei latex-vorspann.tex mit gesetztem Schalter.

\newif\ifkorrekturansicht
\korrekturansichttrue

\input{../tex-inputs/latex-vorspann}


               \section[Arthur Schnitzler an Hermann Bahr, 14. 12. 1909]{ Arthur Schnitzler an Hermann Bahr, 14. 12. 1909}\nopagebreak\mylabel{v}\rehead{ }\normalsize\beginnumbering\briefempfaengerindex{Bahr, Hermann@\textsc{Bahr, Hermann}!zzzSchnitzler, Arthur@\emph{von Arthur Schnitzler}!1909-12-141@{14. 12. 1909}|(be} \toendnotes[C]{\smallbreak\pagebreak[2]} \Standort{TMW, HS AM 60150 Ba.}
\physDesc{Briefkarte, 2 Karten (die zweite Karte von Schnitzler mit »II.« beschriftet), 4 Seiten
\newline{}Handschrift: schwarze Tinte, deutsche Kurrent}\buchAbdrucke{\weitereDrucke{1) \emph{14. 12. 1909, Abschrift.} In: Arthur Schnitzler: \emph{The Letters of Arthur Schnitzler to Hermann Bahr}. Edited, annotated, and with an introduction, by Donald G.
                        Daviau. Chapel Hill: \emph{The University of North Carolina Press} 1978, S. 104–105 (University of North Carolina studies in the Germanic languages
                        and literatures, 89).} \weitereDrucke{2) Hermann Bahr, Arthur Schnitzler: \emph{Briefwechsel, Aufzeichnungen, Dokumente (1891–1931)}. Hg. Kurt Ifkovits und Martin Anton Müller. Göttingen: \emph{Wallstein} 2018, S. 428–429.} }\toendnotes[C]{\smallbreak}\pstart
           \noindent{}{\pb}\textcolor{gray}{\textbf{Dr. Arthur Schnitzler}}\hfill 14/12 09\pend
           \pstart
           \textcolor{gray}{\textbf{\textcolor{pink}{Wien XVIII. Spoettelgasse 7}{}\ledrightnote{\textcolor{pink}{Edmund-Weiß-Gasse}}.}}\pend
           \pstart
           mein lieber Hermann, bei \textcolor{pink}{Berlin}{}\ledrightnote{\textcolor{pink}{Berlin}}er
               Gelegenheit einmal \textcolor{pink}{Halle}{}\ledrightnote{\textcolor{pink}{Halle an der Saale}} mitzunehmen hab ich mir
               längſt vorgeno{\geminationm}en – nur fügt es ſich immer ſo ſchwer,
               weil man ja viel früher einen besti{\geminationm}ten Vorleſe-Tag
               fixiren muſs als man den \textcolor{pink}{Berliner}{}\ledrightnote{\textcolor{pink}{Berlin}} Premièrentag
               weiſs. Und mir perſönlich macht weder das Zweck-Reiſen noch das Vorleſen (in großen
               Räumen) ſonderlich {\pb}viel Spaſs. Aber wir wollen ſehen. Deine Gicht aber laß dir lieber von einem
               Dichter als von einem \textcolor{blue}{Oberingenieur}{}\ledrightnote{→\textcolor{blue}{Oskar Bacher}} behandeln – (nur nicht von einem Arzt natürlich) Ich ſtehe dir
               ſtets zur Verfügung – und hoffe mediziniſch ſchon genug vergeſſen zu haben, um dir
               nicht empfindlich zu ſchaden.\pend
           \pstart
           Ja, wenn ich eine luſtige Novelle hätte! Und \textcolor{gray}{nun} gar eine kurze!
               Mit dem Gegentheil ka{\geminationn} ich dienen: \textcolor{green}{Tragoedie in 5 Akten und einem Vorſpiel}{}\ledrightnote{→\textcolor{green}{Der junge Medardus. Dramatische Historie in einem Vorspiel und fünf Aufzügen}} aber
               die eignet ſich eher zum Aufgeführtwerden {\pb}(Wie du ſchon daraus
               erſehen kannſt, daſs es mir nicht möglich iſt, von \textcolor{blue}{\textsc{Schlenther}}{}\ledrightnote{\textcolor{blue}{Paul Schlenther}}{ }ſowohl als von \textcolor{blue}{\textsc{Reinhardt}}{}\ledrightnote{\textcolor{blue}{Max Reinhardt}} eine endgiltige Entſcheidung zu kriegen.) – Die \textcolor{green}{\textsc{Comtesse Mizzi}}{}\ledrightnote{\textcolor{green}{Komtesse Mizzi oder Der Familientag}} wird nun doch nicht zu deinem »\textcolor{green}{Concert}{}\ledrightnote{\textcolor{green}{Das Konzert. Lustspiel in drei Akten}}«
               gegeben, der Abend würde zu lang, ſchreibt \textcolor{blue}{Brahm}{}\ledrightnote{\textcolor{blue}{Otto Brahm}}.
               Dabei hatt ich ſchon an den \textcolor{pink}{Münchner}{}\ledrightnote{\textcolor{pink}{München}}{ }\textcolor{blue}{Speidel}{}\ledrightnote{\textcolor{blue}{Albert von Speidel}}{ }{\pb}ſchreiben laſſen, er
               möchte auch womöglich \textcolor{green}{die zwei
                  Stücke}{}\ledrightnote{→\textcolor{green}{Komtesse Mizzi oder Der Familientag}{\newline}→\textcolor{green}{Das Konzert. Lustspiel in drei Akten}} zusa{\geminationm}enſpielen. Nun hat \textcolor{blue}{\textsc{Speidel}}{}\ledrightnote{\textcolor{blue}{Albert von Speidel}} aber die \textcolor{green}{\textsc{Comtesse}}{}\ledrightnote{\textcolor{green}{Komtesse Mizzi oder Der Familientag}} wegen Frivolität, Kinderkriegen und Liebhaber-haben refuſirt.\pend
           \pstart
           Die Hoffnung dich wieder einmal zu ſprechen, geb ich noch immer nicht auf. Vielleicht
               auf dem \textcolor{pink}{Se{\geminationm}ering}{}\ledrightnote{\textcolor{pink}{Semmering}}. Und daſs du den Leuten allerorten
               ſo viel von mir erzählſt, dank ich dir von Herzen. \textcolor{blue}{Wir}{}\ledrightnote{→\textcolor{blue}{Olga Schnitzler}} grüßen alle aufs beſte und wollen auch Deiner verehrten
                  \textcolor{blue}{Frau}{}\ledrightnote{→\textcolor{blue}{Anna Bahr-Mildenburg}} empfohlen ſein.\pend
           \pstart Dein getreuer \spacefill\mbox{Arthur.}\pend{}\endnumbering\briefempfaengerindex{Bahr, Hermann@\textsc{Bahr, Hermann}!zzzSchnitzler, Arthur@\emph{von Arthur Schnitzler}!1909-12-141@{14. 12. 1909}|)be}\mylabel{h}  \normalsize

\doendnotes{C}
\bigskip
\vfill

\clearpage

\footnotesize

\lohead{\textsc{register}}

% Definiere theindex-Environment komplett neu ohne reledmac
\makeatletter
\renewenvironment{theindex}{%
  \section*{\indexname}%
  \setlength{\parindent}{0pt}%
  \setlength{\parskip}{0pt plus 0.3pt}%
  \let\item\@idxitem
}{%
  \clearpage
}
\makeatother

\IfFileExists{\jobname-pw.ind}{\input{\jobname-pw.ind}}{}

\end{document}

      