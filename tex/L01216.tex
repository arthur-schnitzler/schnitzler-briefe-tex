%% latex-korrekturansicht-vorspann.tex
%% Vorspann für die Korrekturansicht.
%% Lädt die gemeinsame Datei latex-vorspann.tex mit gesetztem Schalter.

\newif\ifkorrekturansicht
\korrekturansichttrue

\input{../tex-inputs/latex-vorspann}


               \section[Hermann Bahr an Arthur Schnitzler, 1. 4. 1902]{ Hermann Bahr an Arthur Schnitzler, 1. 4. 1902}\nopagebreak\mylabel{v}\rehead{ }\normalsize\beginnumbering\briefempfaengerindex{Schnitzler, Arthur@\textsc{Schnitzler, Arthur}!zzzBahr, Hermann@\emph{von Hermann Bahr}!1902-04-012@{1. 4. 1902}|(be} \toendnotes[C]{\smallbreak\pagebreak[2]} \Standort{CUL, Schnitzler, B 5b.}
\physDesc{Brief, 1 Blatt, 2 Seiten
\newline{}Handschrift: schwarze Tinte, deutsche Kurrent
\newline{}Schnitzler: mit Bleistift die Jahreszahl »902« ergänzt \newline{}Ordnung: mit Bleistift von unbekannter Hand nummeriert: »87« }\buchAbdrucke{\weitereDrucke{Hermann Bahr, Arthur Schnitzler: \emph{Briefwechsel, Aufzeichnungen, Dokumente (1891–1931)}. Hg. Kurt Ifkovits und Martin Anton Müller. Göttingen: \emph{Wallstein} 2018, S. 228.} }\toendnotes[C]{\smallbreak}\pstart
           \raggedleft{}{\pb}1. 4.\pend
           \pstart\center{}Lieber Arthur!\pend\pstart
           Die mir zugeſchickten Proben ſind von jener heute ſo weit verbreiteten
               Talentloſigkeit, die glaubt, es genüge einige Wendungen von »modernen« Autoren
               aufzuſchnappen, und gar nicht zu bemerken ſcheint, daß ſie gar nichts zu ſagen hat.
               Dies ſchließt nicht aus, daß der \textcolor{blue}{Ver{\pb}faſſer}{}\ledrightnote{→\textcolor{blue}{Gustav Modry}} vielleicht ſich zum
               Journaliſten eignen könnte. Eine »Schmuck-Notiz« über Allerheiligen oder die
               Eröffnung oder Schließung eines Cafés oder eine ſchöne Leich’ iſt ja ganz was
               anderes. Doch müßte man davon Proben ſehen und wiſſen, was er ſich unter »Journaliſt«
               (der er, wie Du ſchreibſt, werden will) eigentlich denkt.\pend
           \pstart
           Herzlichſt{\\[\baselineskip]}in Eile{\\[\baselineskip]}Dein alter{\\[\baselineskip]}\spacefill\mbox{Hermann}\pend
           \leftskip=0em{}\endnumbering\briefempfaengerindex{Schnitzler, Arthur@\textsc{Schnitzler, Arthur}!zzzBahr, Hermann@\emph{von Hermann Bahr}!1902-04-012@{1. 4. 1902}|)be}\mylabel{h}  \normalsize

\doendnotes{C}
\bigskip
\vfill

\clearpage

\footnotesize

\lohead{\textsc{register}}

% Definiere theindex-Environment komplett neu ohne reledmac
\makeatletter
\renewenvironment{theindex}{%
  \section*{\indexname}%
  \setlength{\parindent}{0pt}%
  \setlength{\parskip}{0pt plus 0.3pt}%
  \let\item\@idxitem
}{%
  \clearpage
}
\makeatother

\IfFileExists{\jobname-pw.ind}{\input{\jobname-pw.ind}}{}

\end{document}

      