%% latex-korrekturansicht-vorspann.tex
%% Vorspann für die Korrekturansicht.
%% Lädt die gemeinsame Datei latex-vorspann.tex mit gesetztem Schalter.

\newif\ifkorrekturansicht
\korrekturansichttrue

\input{../tex-inputs/latex-vorspann}


               \section[Georg Brandes an Arthur Schnitzler, 11. 3. 1906]{ Georg Brandes an Arthur Schnitzler, 11. 3. 1906}\nopagebreak\mylabel{v}\rehead{ }\normalsize\beginnumbering\briefempfaengerindex{Schnitzler, Arthur@\textsc{Schnitzler, Arthur}!zzzBrandes, Georg@\emph{von Georg Brandes}!1906-03-111@{11. 3. 1906}|(be} \toendnotes[C]{\smallbreak\pagebreak[2]} \Standort{CUL, Schnitzler, B 17.}
\physDesc{Brief, 1 Blatt, 3 Seiten
\newline{}Handschrift: blaue Tinte, lateinische Kurrent\newline{}Ordnung: mit Bleistift von unbekannter Hand nummeriert:
                                    »30« }\buchAbdrucke{\weitereDrucke{Georg Brandes, Arthur Schnitzler: \emph{Ein Briefwechsel}. Hg. Kurt Bergel. Bern: \emph{Francke} 1956, S. 91.} }\toendnotes[C]{\smallbreak}\pstart
           \raggedleft{}{\pb}\textcolor{pink}{Kopenhagen}{}\ledrightnote{\textcolor{pink}{Kopenhagen}}{ }11 März 1906\pend
           \pstart{}Verehrter und lieber Freund\pend\pstart
           Haben Sie herzlichen Dank für die gute Gabe, die Sie mir schickten, Ihr letztes \textcolor{green}{Schauspiel}{}\ledrightnote{→\textcolor{green}{Der Ruf des Lebens. Schauspiel in drei Akten}}. Ich habe meine Freude
               daran gehabt. Die Welt Ihrer Phantasie zieht mich immer an und erregt meine
               Bewunderung, da ich selbst wenig Phantasie besitze und erstaune, dass ein anderer all
               das erfinden kann.\pend
           \pstart
           Seit lange beschäftigt es Sie, wie der Gedanke an den nahen Tod die Gefühle
               beeinflusst, \textcolor{green}{Schleier der Beatrice}{}\ledrightnote{\textcolor{green}{Der Schleier der Beatrice. Schauspiel in fünf Akten}}, \textcolor{green}{Lieutenant Gustl}{}\ledrightnote{\textcolor{green}{Lieutenant Gustl. Novelle}}, usw. Hier variiren Sie das Thema;
               der Gedanke an den Tod des Liebsten wirkt ebenso. Sie sind ein Grübler über den Tod,
               wie schon Ihr »\textcolor{green}{Sterben}{}\ledrightnote{\textcolor{green}{Sterben. Novelle}}« zeigte. Die Hälfte {\pb}Ihrer Produktion ist Thanatos, die
               Hälfte Eros gewidmet. Aber dadurch haben Ihre Arbeiten eine so grosse Spannweite
               (wenn das Wort deutsch ist).\pend
           \pstart
           Ich las eine sehr unverständige \label{K_L01589_1v}\edtext{\textcolor{green}{Kritik}{}\ledrightnote{→\textcolor{green}{»Der Ruf des Lebens.« Schauspiel von Artur Schnitzler. Erste Aufführung im Lessingtheater}}}{\lemma{\textnormal{\emph{Kritik}}}\Cendnote{\textnormal{\textcolor{blue}{L. Schönhoff}: \emph{\textcolor{green}{»Der Ruf des Lebens.« Schauspiel von Artur Schnitzler. Erste Aufführung im
                        Lessing-Theater}}. In: \emph{\textcolor{green}{Der Tag}},
                     Nr. 105, Ausgabe A, 27. 2. 1906, Erster Teil,
                  S. 1–2.}}}\label{K_L01589_1h} über Ihr \textcolor{green}{Werk}{}\ledrightnote{→\textcolor{green}{Der Ruf des Lebens. Schauspiel in drei Akten}} in dem \textcolor{brown}{\uline{Tag}}{}\ledrightnote{\textcolor{brown}{Der Tag}}; es scheint mir, dass die meiste deutsche Kritik allzu viel fertige Begriffe
               und Ansprüche mitbringt; sie ist weniger geschmeidig als die unsrige.\pend
           \pstart
           Es war mir sehr lieb, Sie jene \label{K_L01589_2v}\edtext{Stunde}{\lemma{\textnormal{\emph{Stunde}}}\Cendnote{\textnormal{am 19. 11. 1905
                  in \textcolor{pink}{Berlin}}}}\label{K_L01589_2h} bei \textcolor{blue}{Fulda}{}\ledrightnote{\textcolor{blue}{Ludwig Fulda}} zu treffen. Ich möchte, dass Sie wieder einmal nach \textcolor{pink}{Dänemark}{}\ledrightnote{\textcolor{pink}{Dänemark}} kämen.\pend
           \pstart
           Ihr dankbar verbundener{\\[\baselineskip]}\spacefill\mbox{Georg Brandes}\pend
           \leftskip=0em{}\endnumbering\briefempfaengerindex{Schnitzler, Arthur@\textsc{Schnitzler, Arthur}!zzzBrandes, Georg@\emph{von Georg Brandes}!1906-03-111@{11. 3. 1906}|)be}\mylabel{h}  \normalsize

\doendnotes{C}
\bigskip
\vfill

\clearpage

\footnotesize

\lohead{\textsc{register}}

% Definiere theindex-Environment komplett neu ohne reledmac
\makeatletter
\renewenvironment{theindex}{%
  \section*{\indexname}%
  \setlength{\parindent}{0pt}%
  \setlength{\parskip}{0pt plus 0.3pt}%
  \let\item\@idxitem
}{%
  \clearpage
}
\makeatother

\IfFileExists{\jobname-pw.ind}{\input{\jobname-pw.ind}}{}

\end{document}

      