%% latex-korrekturansicht-vorspann.tex
%% Vorspann für die Korrekturansicht.
%% Lädt die gemeinsame Datei latex-vorspann.tex mit gesetztem Schalter.

\newif\ifkorrekturansicht
\korrekturansichttrue

\input{../tex-inputs/latex-vorspann}


               \section[Arthur Schnitzler an Richard Beer-Hofmann, 20. 2. 1907]{ Arthur Schnitzler an Richard Beer-Hofmann, 20. 2. 1907}\nopagebreak\mylabel{v}\rehead{ }\normalsize\beginnumbering\briefempfaengerindex{Beer-Hofmann, Richard@\textsc{Beer-Hofmann, Richard}!zzzSchnitzler, Arthur@\emph{von Arthur Schnitzler}!1907-02-201@{20. 2. 1907}|(be} \toendnotes[C]{\smallbreak\pagebreak[2]} \Standort{YCGL, MSS 31.}
\physDesc{Visitenkarte, Umschlag
\newline{}Handschrift: schwarze Tinte, deutsche Kurrent\newline{}Versand: Eine persönliche Übergabe durch \textcolor{blue}{Feuchtwang} ist anzunehmen. }\buchAbdrucke{\weitereDrucke{Arthur Schnitzler, Richard Beer-Hofmann: \emph{Briefwechsel 1891–1931}. Hg. Konstanze Fliedl. Wien, Zürich: \emph{Europaverlag} 1992, S. 180.} }\pstart{}{\pb}\textsc{Dr Richard Beer-Hofmann}\pend{}\pstart{}\textcolor{pink}{Wien XVIII}{}\ledrightnote{\textcolor{pink}{XVIII., Währing}}\pend{}\pstart{}\textcolor{pink}{\textsc{Hasenauerstraße}}{}\ledrightnote{\textcolor{pink}{Hasenauerstraße}}\pend{}{\bigskip}\pstart
           \noindent{}\centering{}{\pb}\textcolor{gray}{\textbf{D\textsuperscript{r} Arthur Schnitzler}}\pend
           \pstart
           \noindent{}\raggedleft{}\textcolor{pink}{\textcolor{gray}{\textbf{Wien}}}{}\ledrightnote{\textcolor{pink}{Wien}}\pend
           \pstart
           \noindent{}{\pb}lieber Richard, erlauben Sie mir Ihnen Herrn Dr. \textcolor{blue}{\textsc{Feuchtwang}}{}\ledrightnote{\textcolor{blue}{David Feuchtwang}} vorzuſtellen, \textsc{resp}. ihn Ihnen in Eri{\geminationn}erung zu bringen.\pend
           \pstart
           Herzlich grüß\textcolor{gray}{end}{\\[\baselineskip]}Ihr \spacefill\mbox{Arthur}\pend
           \leftskip=0em{}\pstart
           \noindent{}20/2 907\pend
           \endnumbering\briefempfaengerindex{Beer-Hofmann, Richard@\textsc{Beer-Hofmann, Richard}!zzzSchnitzler, Arthur@\emph{von Arthur Schnitzler}!1907-02-201@{20. 2. 1907}|)be}\mylabel{h}  \normalsize

\doendnotes{C}
\bigskip
\vfill

\clearpage

\footnotesize

\lohead{\textsc{register}}

% Definiere theindex-Environment komplett neu ohne reledmac
\makeatletter
\renewenvironment{theindex}{%
  \section*{\indexname}%
  \setlength{\parindent}{0pt}%
  \setlength{\parskip}{0pt plus 0.3pt}%
  \let\item\@idxitem
}{%
  \clearpage
}
\makeatother

\IfFileExists{\jobname-pw.ind}{\input{\jobname-pw.ind}}{}

\end{document}

      