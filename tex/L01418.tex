%% latex-korrekturansicht-vorspann.tex
%% Vorspann für die Korrekturansicht.
%% Lädt die gemeinsame Datei latex-vorspann.tex mit gesetztem Schalter.

\newif\ifkorrekturansicht
\korrekturansichttrue

\input{../tex-inputs/latex-vorspann}


               \section[Hugo von Hofmannsthal an Arthur Schnitzler, {[}24./25.?{]} 7. 1904]{ Hugo von Hofmannsthal an Arthur Schnitzler, {[}24./25.?{]} 7. 1904}\nopagebreak\mylabel{v}\rehead{ }\normalsize\beginnumbering\briefempfaengerindex{Schnitzler, Arthur@\textsc{Schnitzler, Arthur}!zzzHofmannsthal, Hugo von@\emph{von Hugo von Hofmannsthal}!1904-07-241@{{[}24./25.?{]} 7. 1904}|(be} \toendnotes[C]{\smallbreak\pagebreak[2]} \Standort{CUL, Schnitzler, B 43.}
\physDesc{Brief, 1 Blatt, 4 Seiten
\newline{}Handschrift: schwarze Tinte, deutsche Kurrent
\newline{}Schnitzler: mit Bleistift Monat und Jahreszahl ergänzt: »7. 904.« \newline{}Ordnung: 1) mit Bleistift von unbekannter Hand nummeriert: »\strikeout{77}« 2) mit Bleistift von unbekannter Hand nummeriert:
                                    »230«}\buchAbdrucke{\weitereDrucke{Hugo von Hofmannsthal, Arthur Schnitzler: \emph{Briefwechsel}. Hg. Therese Nickl und Heinrich Schnitzler. Frankfurt am Main: \emph{S. Fischer} 1964, S. 191.} }\toendnotes[C]{\smallbreak}\pstart
           \raggedleft{}{\pb}\textcolor{pink}{Bad Fuſch}{}\ledrightnote{\textcolor{pink}{Bad Fusch}}{ }2\textcolor{gray}{×}\textsc{ten}\pend
           \pstart{}lieber, \pend\pstart
           hier bin ich wirklich wie unter dem erſten Anhauch der Luft geſund geworden, und von
               einem innern Reichthum, daſs ich manchmal, gegen Abend, auf eine ſteile Berglehne hin
               aufklettern muſs, nur um das Blut vom Kopf abzuleiten und den unaufhörlichen {\pb}Zudrang von Gedanken, Bildern,
               Situationen, abzuleiten.\pend
           \pstart
           Es iſt mir ſchwerer, in ſolchen Zeiten ein Buch zu leſen. Ich möchte alles, was mir
               in die Hände fällt, dramatiſieren, ſelbſt den \textcolor{green}{\textcolor{blue}{Goethe}{}\ledrightnote{\textcolor{blue}{Johann Wolfgang von Goethe}}–\textcolor{blue}{Schiller}{}\ledrightnote{\textcolor{blue}{Friedrich von Schiller}}’ſchen Briefwechſel}{}\ledrightnote{\textcolor{green}{Briefwechsel zwischen Schiller und Goethe}}, oder die \textcolor{green}{Linzer Tages-poſt}{}\ledrightnote{\textcolor{green}{Linzer Tages-Post}}.\pend
           \pstart
           Das »\label{T_L01418_1v}\edtext{\textcolor{green}{gerettete Venedig}{}\ledrightnote{\textcolor{green}{Das gerettete Venedig. Trauerspiel in fünf Aufzügen}}}{\lemma{\textnormal{\emph{gerettete Venedig}}}\Cendnote{\textnormal{wohl von Schnitzler mit Bleistift
                  unterstrichen}}}\label{T_L01418_1h}« hab ich \label{K_L01418_1v}\edtext{heute abgeschloſſen}{\lemma{\textnormal{\emph{heute abgeschloſſen}}}\Cendnote{\textnormal{Das erlaubt die
                  annähernde Datierung: Am 24. 7. 1904 schreibt \textcolor{blue}{Hofmannsthal} dem \textcolor{blue}{Vater}, das Stück beendet zu haben. (\textcolor{blue}{Hugo von Hofmannsthal}: \emph{Aufzeichnungen}. Hg. Rudolf Hirsch † und Ellen Ritter † in
                     Zusammenarbeit mit Konrad Heumann und Peter Michael Braunwarth. Frankfurt am
                     Main: \emph{\textcolor{brown}{S. Fischer}}{ }2013, Erläuterungen, S. 789 (\emph{Sämtliche
                        Werke}, XXXIX)) Am Folgetag, dem 25. 7. 1904, hält er zudem den
                  Abschluss in einer persönlichen Aufzeichnung fest.
                  (S. 482.)}}}\label{K_L01418_1h}. Was noch {\pb}daran zu thun iſt, das wenige
               läſst ſich unter dem Abſchreiben thun.\hspace*{1.5em}Indeſſen ſind
               aber, wie leuchtende Wolkeninſeln hinter den Bergen hervor andere Stoffe geſtiegen,
               zum Theil aus dem geheimnisvollen Abgrund des niemals ſchlafenden, umbildenden
               Gedächtniſſes: das »\textcolor{green}{Leben ein Traum}{}\ledrightnote{\textcolor{green}{Der Turm. Ein Trauerspiel}}« dieſer faſt zu
               große Stoff, hat ſeinen tiefen {\pb}dem \textcolor{blue}{Calderon}{}\ledrightnote{\textcolor{blue}{Pedro Calderón de la Barca}} faſt entgegen geſetzten Schluſs
               gefunden, »\textcolor{green}{\textsc{Pentheus}}{}\ledrightnote{\textcolor{green}{Pentheus. Trauerspiel in zwei Aufzügen}}« im Stoff den \textcolor{green}{\textsc{Bacchen}}{}\ledrightnote{\textcolor{green}{Die Bakchen}} des \textcolor{blue}{\textsc{Euripides}}{}\ledrightnote{\textcolor{blue}{Euripides}} nahe, aber viel reicher und ſchöner, hat ſich zum Scenarium gegliedert,
               zweiactig; »\textcolor{green}{\textsc{Orest in Delphi}}{}\ledrightnote{\textcolor{green}{Orest in Delphi}}« der \textcolor{green}{\textsc{Elektra}}{}\ledrightnote{\textcolor{green}{Elektra. Tragödie in einem Aufzug}} 2\textsuperscript{ter} Theil zeigt ſeine Geſtalten unheimlich
               deutlich – mit dieſer Fracht gehe ich den 31\textsuperscript{ten} nach \textcolor{pink}{\textsc{Markt-Aussee}, Rammgut}{}\ledrightnote{\textcolor{pink}{Ramgut}}.\pend
           \pstart Laſſen Sie mich hier oder dort nicht ohne Nachricht. Ihr\spacefill\mbox{Hugo.}\pend{}\endnumbering\briefempfaengerindex{Schnitzler, Arthur@\textsc{Schnitzler, Arthur}!zzzHofmannsthal, Hugo von@\emph{von Hugo von Hofmannsthal}!1904-07-241@{{[}24./25.?{]} 7. 1904}|)be}\mylabel{h}  \normalsize

\doendnotes{C}
\bigskip
\vfill

\clearpage

\footnotesize

\lohead{\textsc{register}}

% Definiere theindex-Environment komplett neu ohne reledmac
\makeatletter
\renewenvironment{theindex}{%
  \section*{\indexname}%
  \setlength{\parindent}{0pt}%
  \setlength{\parskip}{0pt plus 0.3pt}%
  \let\item\@idxitem
}{%
  \clearpage
}
\makeatother

\IfFileExists{\jobname-pw.ind}{\input{\jobname-pw.ind}}{}

\end{document}

      