%% latex-korrekturansicht-vorspann.tex
%% Vorspann für die Korrekturansicht.
%% Lädt die gemeinsame Datei latex-vorspann.tex mit gesetztem Schalter.

\newif\ifkorrekturansicht
\korrekturansichttrue

\input{../tex-inputs/latex-vorspann}


               \section[Arthur Schnitzler an Robert Adam, 12. 6. 1920]{ Arthur Schnitzler an Robert Adam, 12. 6. 1920}\nopagebreak\mylabel{v}\rehead{ }\normalsize\beginnumbering\briefempfaengerindex{Adam, Robert@\textsc{Adam, Robert}!zzzSchnitzler, Arthur@\emph{von Arthur Schnitzler}!1920-06-121@{12. 6. 1920}|(be} \toendnotes[C]{\smallbreak\pagebreak[2]} \Standort{DLA, 96.34.2/20.}
\physDesc{Postkarte
\newline{}Handschrift: schwarze Tinte, lateinische Kurrent\newline{}Versand: Stempel: »\nobreak{}Wien 66, 15. VI. 20, X\nobreak{}«.  }\toendnotes[C]{\smallbreak}\pstart{}{\pb}\textcolor{gray}{\textbf{D\textsuperscript{R}
                                ARTHUR SCHNITZLER}}\pend{}\pstart{}\textcolor{gray}{\textbf{\textcolor{pink}{WIEN, XVIII.
                                    STERNWARTESTRASSE 71}{}\ledrightnote{\textcolor{pink}{Sternwartestraße}}.}}\pend{}{\bigskip}\pstart{}Hrn Dr. Rob. Ad. Pollak\pend{}\pstart{}Ob-Landesger-Rath\pend{}\pstart{}\textcolor{pink}{Wien XII}{}\ledrightnote{\textcolor{pink}{XII., Meidling}}.\pend{}\pstart{}\textcolor{pink}{Meidlinger Hptstr. 58}{}\ledrightnote{\textcolor{pink}{Meidlinger Hauptstraße}}.\pend{}{\bigskip}\pstart
           \raggedleft{}{\pb}12. 6. 1920\pend
           \pstart\center{}Verehrter Herr Doctor\pend\pstart
           Vielen Dank für die liebenswürdige Übersendg Ihres \label{K_L02341_1v}\edtext{\textcolor{green}{Artikels über
                        Rechtsprincipien}{}\ledrightnote{→\textcolor{green}{Über Rechtsprinzipien. Eine analytische Untersuchung}}}{\lemma{\textnormal{\emph{Artikels über Rechtsprincipien}}}\Cendnote{\textnormal{\textcolor{blue}{Robert Adam Pollak}: \emph{\textcolor{green}{Ueber Rechtsprinzipien. Eine Analytische
                                Untersuchung}}. In: \emph{\textcolor{green}{Archiv für
                                Rechts- und Wirtschaftsphilosophie}}, Jg. 13, H. 2,
                                1919, S. 110–135. }}}\label{K_L02341_1h}, den ich mit starkem Interesse lese.\pend
           \pstart
           W\textcolor{gray}{o}\textcolor{gray}{l}eben Sie für den Sommer? Hab ich vielleicht wieder einmal
                    das Vergnügen Sie zu sehen?\pend
           \pstart
           Herzlichst grüsst Sie{\\[\baselineskip]}Ihr sehr ergeben\textcolor{gray}{er}{\\[\baselineskip]}\spacefill\mbox{ArthurSchnitzler}\pend
           \leftskip=0em{}\endnumbering\briefempfaengerindex{Adam, Robert@\textsc{Adam, Robert}!zzzSchnitzler, Arthur@\emph{von Arthur Schnitzler}!1920-06-121@{12. 6. 1920}|)be}\mylabel{h}  \normalsize

\doendnotes{C}
\bigskip
\vfill

\clearpage

\footnotesize

\lohead{\textsc{register}}

% Definiere theindex-Environment komplett neu ohne reledmac
\makeatletter
\renewenvironment{theindex}{%
  \section*{\indexname}%
  \setlength{\parindent}{0pt}%
  \setlength{\parskip}{0pt plus 0.3pt}%
  \let\item\@idxitem
}{%
  \clearpage
}
\makeatother

\IfFileExists{\jobname-pw.ind}{\input{\jobname-pw.ind}}{}

\end{document}

      