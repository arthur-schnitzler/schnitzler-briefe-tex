%% latex-korrekturansicht-vorspann.tex
%% Vorspann für die Korrekturansicht.
%% Lädt die gemeinsame Datei latex-vorspann.tex mit gesetztem Schalter.

\newif\ifkorrekturansicht
\korrekturansichttrue

\input{../tex-inputs/latex-vorspann}


               \section[Richard Beer-Hofmann an Arthur Schnitzler, 23. 8. 1907]{ Richard Beer-Hofmann an Arthur Schnitzler, 23. 8. 1907}\nopagebreak\mylabel{v}\rehead{ }\normalsize\beginnumbering\briefempfaengerindex{Schnitzler, Arthur@\textsc{Schnitzler, Arthur}!zzzBeer-Hofmann, Richard@\emph{von Richard Beer-Hofmann}!1907-08-231@{23. 8. 1907}|(be} \toendnotes[C]{\smallbreak\pagebreak[2]} \Standort{CUL, Schnitzler, B 8.}
\physDesc{Brief, 2 Blätter (Briefpapier mit Trauerrand), 4 Seiten
\newline{}Handschrift: Bleistift, lateinische Kurrent\newline{}Ordnung: mit Bleistift von unbekannter Hand nummeriert: »211« }\buchAbdrucke{\weitereDrucke{Arthur Schnitzler, Richard Beer-Hofmann: \emph{Briefwechsel 1891–1931}. Hg. Konstanze Fliedl. Wien, Zürich: \emph{Europaverlag} 1992, S. 183.} }\toendnotes[C]{\smallbreak}\pstart
           \raggedleft{}{\pb}\textcolor{pink}{Velden}{}\ledrightnote{\textcolor{pink}{Velden}}{ }23/VIII 07\pend
           \pstart
           Lieber Arthur! Ihre Karte vom 19. erhalte ich heute
               nachgeschickt – nach \textcolor{pink}{Villach}{}\ledrightnote{\textcolor{pink}{Villach}}, wo ich bis heute
               Früh war. Wir sind am 20. von \textcolor{pink}{Wien}{}\ledrightnote{\textcolor{pink}{Wien}} weg,
               haben \textcolor{pink}{Veld\uline{es}}{}\ledrightnote{\textcolor{pink}{Veldes}} angesehen, dann doch aber hier – bei \textcolor{pink}{Wahliss}{}\ledrightnote{\textcolor{pink}{Etablissement Ernst Wahliss}}
               – Wohnung geno{\geminationm}en. Wollen hier acht Tage ungefähr
               bleiben, wenn uns kühles Wetter nicht vorher südlich treibt. Dann – bei schönem
               Wetter über ein Stück der \textcolor{pink}{Dolomitenstrasse}{}\ledrightnote{\textcolor{pink}{Große Dolomitenstraße}} nach
                  {\pb}\textcolor{pink}{Bozen}{}\ledrightnote{\textcolor{pink}{Bozen}} – schliesslich \textcolor{pink}{Lido}{}\ledrightnote{\textcolor{pink}{Lido}}, bei kühlem Wetter direkt an den \textcolor{pink}{Lido}{}\ledrightnote{\textcolor{pink}{Lido}}!
                  I{\geminationm}erhin ist ziemliche Wahrscheinlichkeit vorhanden
               dass wir zwischen 2 – und 5 September in \textcolor{pink}{Bozen}{}\ledrightnote{\textcolor{pink}{Bozen}} oder \textcolor{pink}{Bozen}{}\ledrightnote{\textcolor{pink}{Bozen}}s Nähe
               sind.\pend
           \pstart
           {\pb}Wenn Sie getreulich Ihren
               Aufenthalt mir melden, eventuell auch mir sagen wohin ich restante Briefe oder
                  Telegra{\geminationm}e richten soll\strikeout{en}, können wir uns vielleicht doch treffen – was sehr schön wäre.\pend
           \pstart
           {\pb}Wenn Sie \textcolor{blue}{Goldmann}{}\ledrightnote{\textcolor{blue}{Paul Goldmann}} sehen, sagen Sie ihm, bitte, dass ich ihm sehr für
               seine lieben Zeilen danke, dass ich ihm \label{K_L01701-1v}\edtext{als \textcolor{blue}{Mamroth}{}\ledrightnote{\textcolor{blue}{Fedor Mamroth}} starb}{\lemma{\textnormal{\emph{als Mamroth starb}}}\Cendnote{\textnormal{am 25. 6. 1907}}}\label{K_L01701-1h}, nicht schrieb,
               weil ich um diese Zeit den Kopf mit der beabsichtigten Operation an \textcolor{blue}{Papa}{}\ledrightnote{→\textcolor{blue}{Hermann Beer}} – die dann unterblieb – voll hatte, und
               ruhigere Tage abwarten wollte um ihm zu schreiben. Ich will mich aber nie mehr selbst
               auf »ruhige« oder »ruhigere« Tage vertrösten, ich {\pb}entdecke – vielleicht ein bischen
               zu spät – daß es keine giebt, – nie giebt, für Leute wie ich bin, zumindest
               nicht.\pend
           \pstart
           Geht \textcolor{blue}{Goldmann}{}\ledrightnote{\textcolor{blue}{Paul Goldmann}} mit Ihnen dann sagen Sie mir wie
               lange er \substVorne{}\textsuperscript{bei}\substDazwischen{}mit\substHinten{} Ihnen bleibt, vielleicht kann ich es (wenn {\pb}es sich nur um 1–2 Tage handelt) so
               einrichten, daß ich ihn noch treffe. Geht er aber \textcolor{pink}{Wien}{}\ledrightnote{\textcolor{pink}{Wien}}wärts, so liegen wir an seiner Route und erwarten ein Telegra{\geminationm} »\uline{\textcolor{pink}{Wahliss}{}\ledrightnote{\textcolor{pink}{Etablissement Ernst Wahliss}} – \textcolor{pink}{Velden}{}\ledrightnote{\textcolor{pink}{Velden}}}« wann
               er hieher ko{\geminationm}mt. Alles Herzliche Ihnen und Frau \textcolor{blue}{Olga}{}\ledrightnote{\textcolor{blue}{Olga Schnitzler}}.\pend
           \pstart Ihr \spacefill\mbox{Richard}\pend{}\endnumbering\briefempfaengerindex{Schnitzler, Arthur@\textsc{Schnitzler, Arthur}!zzzBeer-Hofmann, Richard@\emph{von Richard Beer-Hofmann}!1907-08-231@{23. 8. 1907}|)be}\mylabel{h}  \normalsize

\doendnotes{C}
\bigskip
\vfill

\clearpage

\footnotesize

\lohead{\textsc{register}}

% Definiere theindex-Environment komplett neu ohne reledmac
\makeatletter
\renewenvironment{theindex}{%
  \section*{\indexname}%
  \setlength{\parindent}{0pt}%
  \setlength{\parskip}{0pt plus 0.3pt}%
  \let\item\@idxitem
}{%
  \clearpage
}
\makeatother

\IfFileExists{\jobname-pw.ind}{\input{\jobname-pw.ind}}{}

\end{document}

      