%% latex-korrekturansicht-vorspann.tex
%% Vorspann für die Korrekturansicht.
%% Lädt die gemeinsame Datei latex-vorspann.tex mit gesetztem Schalter.

\newif\ifkorrekturansicht
\korrekturansichttrue

\input{../tex-inputs/latex-vorspann}


               \section[Arthur Schnitzler an Wilhelm Bölsche, 10. 4. 1893]{ Arthur Schnitzler an Wilhelm Bölsche, 10. 4. 1893}\nopagebreak\mylabel{v}\rehead{ }\normalsize\beginnumbering\briefempfaengerindex{Boelsche, Wilhelm@\textsc{Bölsche, Wilhelm}!zzzSchnitzler, Arthur@\emph{von Arthur Schnitzler}!1893-04-101@{10. 4. 1893}|(be} \toendnotes[C]{\smallbreak\pagebreak[2]} \Standort{Wrocław, Biblioteka Uniwersytecka, Böl.Pis 1766.}
\physDesc{Brief, 1 Blatt, 2 Seiten
\newline{}Handschrift: schwarze Tinte, deutsche Kurrent
\newline{}Bölsche: als »Erl{[}edigt{]}« gezeichnet }\buchAbdrucke{\weitereDrucke{1) Alois Woldan: \emph{Arthur Schnitzler – Briefe an Wilhelm Bölsche.} In: \emph{Germanica Wratislaviensia} (1987) Nr. 77, S. 461.} \weitereDrucke{2) Wilhelm Bölsche: \emph{Briefwechsel. Mit Autoren der Freien Bühne}. Hg. Gerd-Hermann Susen. Berlin: \emph{Weidler} 2010, S. 683 (Werke und Briefe. Wissenschaftliche Ausgabe, Briefe I).} }\toendnotes[C]{\smallbreak}\pstart\center{}{\pb}Sehr geehrter Herr,\pend\pstart
           anbei eine \textcolor{green}{Studie}{}\ledrightnote{→\textcolor{green}{Die Braut}} für Ihr
                    erg. \textcolor{green}{Blatt}{}\ledrightnote{→\textcolor{green}{Freie Bühne für den Entwickelungskampf der Zeit}}. Falls Sie
                    dieſelbe drucken wollen, ſo erſuche ich \uline{beſti{\geminationm}t} um Correcturbogen. – Jedenfalls würden
                    Sie mich durch \uuline{baldige} Verſtändigung ſehr
                    verbinden. –\pend
           \pstart
           Ich habe mir erlaubt, der \textcolor{brown}{Fr. B.}{}\ledrightnote{\textcolor{brown}{Neue Rundschau, Neue Deutsche Rundschau, Freie Bühne}} mein Buch »\textcolor{green}{Anatol}{}\ledrightnote{\textcolor{green}{Anatol}}« zu ſenden. Vielleicht wäre es möglich,
                    in Ihrer Zeitung ein paar Zeilen {\pb}darüber zu
                    bringen? –\pend
           \pstart
           Ich bin in beſonderer Hochachtung{\\[\baselineskip]}Ihr ergebner{\\[\baselineskip]}\spacefill\mbox{ Dr Arthur Schnitzler}\pend
           \leftskip=0em{}\pstart
           \noindent{}\textcolor{pink}{Wien I. \textsc{Grillparzerstraße 7}.}{}\ledrightnote{\textcolor{pink}{Grillparzerstraße}}\pend
           \pstart
           \textsc{Am 10. April 93}. –\pend
           \endnumbering\briefempfaengerindex{Boelsche, Wilhelm@\textsc{Bölsche, Wilhelm}!zzzSchnitzler, Arthur@\emph{von Arthur Schnitzler}!1893-04-101@{10. 4. 1893}|)be}\mylabel{h}  \normalsize

\doendnotes{C}
\bigskip
\vfill

\clearpage

\footnotesize

\lohead{\textsc{register}}

% Definiere theindex-Environment komplett neu ohne reledmac
\makeatletter
\renewenvironment{theindex}{%
  \section*{\indexname}%
  \setlength{\parindent}{0pt}%
  \setlength{\parskip}{0pt plus 0.3pt}%
  \let\item\@idxitem
}{%
  \clearpage
}
\makeatother

\IfFileExists{\jobname-pw.ind}{\input{\jobname-pw.ind}}{}

\end{document}

      