%% latex-korrekturansicht-vorspann.tex
%% Vorspann für die Korrekturansicht.
%% Lädt die gemeinsame Datei latex-vorspann.tex mit gesetztem Schalter.

\newif\ifkorrekturansicht
\korrekturansichttrue

\input{../tex-inputs/latex-vorspann}


               \section[Hermann Bahr an Arthur Schnitzler, 24. 1. 1898]{ Hermann Bahr an Arthur Schnitzler, 24. 1. 1898}\nopagebreak\mylabel{v}\rehead{ }\normalsize\beginnumbering\briefempfaengerindex{Schnitzler, Arthur@\textsc{Schnitzler, Arthur}!zzzBahr, Hermann@\emph{von Hermann Bahr}!1898-01-241@{24. 1. 1898}|(be} \toendnotes[C]{\smallbreak\pagebreak[2]} \Standort{CUL, Schnitzler, B 5b.}
\physDesc{Brief, 1 Blatt, 2 Seiten
\newline{}Handschrift: schwarze Tinte, deutsche Kurrent\newline{}Ordnung: mit Bleistift von unbekannter Hand nummeriert: »58« }\buchAbdrucke{\weitereDrucke{Hermann Bahr, Arthur Schnitzler: \emph{Briefwechsel, Aufzeichnungen, Dokumente (1891–1931)}. Hg. Kurt Ifkovits und Martin Anton Müller. Göttingen: \emph{Wallstein} 2018, S. 160.} }\toendnotes[C]{\smallbreak}\pstart
           \noindent{}{\pb}\textcolor{gray}{\textbf{»\textcolor{brown}{Die Zeit}{}\ledrightnote{\textcolor{brown}{Die Zeit. Wiener Wochenschrift}}«}}\hfill \textcolor{gray}{\textbf{\textbf{\textcolor{pink}{Wien}{}\ledrightnote{\textcolor{pink}{Wien}}}, den }}24 Januar \textcolor{gray}{\textbf{189}}8\pend
           \pstart
           \textcolor{gray}{\textbf{Wiener Wochenſchrift}}\hfill \textcolor{gray}{\textbf{\textcolor{pink}{IX/3, Günthergaſſe 1}{}\ledrightnote{\textcolor{pink}{Günthergasse}}.}}\pend
           \pstart
           \textcolor{gray}{\textbf{\textbf{Herausgeber}:}}{\\}\textcolor{gray}{\textbf{Profeſſor Dr. \textcolor{blue}{I. Singer}{}\ledrightnote{\textcolor{blue}{Isidor Singer}},
                        \textcolor{blue}{Hermann Bahr}{}\ledrightnote{\textcolor{blue}{Hermann Bahr}}, Dr. \textcolor{blue}{Heinrich Kanner}{}\ledrightnote{\textcolor{blue}{Heinrich Kanner}}.}}\pend
           \pstart
           \textcolor{gray}{\textbf{Telephon Nr. 6415.}}\pend
           \pstart\center{}Lieber Arthur!\pend\pstart
           Ich bitte Dich, an einem Abschiedsabend für \textcolor{blue}{Burckhard}{}\ledrightnote{\textcolor{blue}{Max Eugen Burckhard}} am 2. Februar theilzunehmen – ganz intim, jeder zahlt sein
               Couvert, wahrſcheinlich bei \textcolor{pink}{Sacher}{}\ledrightnote{\textcolor{pink}{Hotel Sacher}}, etwa 40 Personen,
                  \textcolor{blue}{Saar}{}\ledrightnote{\textcolor{blue}{Ferdinand von Saar}}, \textcolor{blue}{Speidel}{}\ledrightnote{\textcolor{blue}{Ludwig Speidel}}, \textcolor{blue}{Julius \textsc{Bauer}}{}\ledrightnote{\textcolor{blue}{Julius Bauer}}, \textcolor{blue}{Groß}{}\ledrightnote{\textcolor{blue}{Ferdinand Gross}}, \textcolor{blue}{Karlweis}{}\ledrightnote{\textcolor{blue}{Carl Karlweis}}, \textcolor{blue}{Chiavacci}{}\ledrightnote{\textcolor{blue}{Vincenz Chiavacci}}, \textcolor{blue}{\textsc{Ebermann}}{}\ledrightnote{\textcolor{blue}{Leo Ebermann}}, einige Maler, \textcolor{blue}{Bukovics}{}\ledrightnote{\textcolor{blue}{Emerich von Bukovics}}, \textcolor{blue}{Gettke}{}\ledrightnote{\textcolor{blue}{Ernst Gettke}}, Baron \textcolor{blue}{\textsc{Berger}}{}\ledrightnote{\textcolor{blue}{Alfred von Berger}} uſw uſw. Hoffentlich biſt Du {\pb}dabei und
               ſchreibst baldigſt ein Ja\pend
           \pstart
           Deinem alten{\\[\baselineskip]}\spacefill\mbox{Hermann Bahr}\pend
           \leftskip=0em{}\pstart
           \textcolor{gray}{\textbf{\label{T_L00765_1v}\edtext{Alle für »\textcolor{brown}{Die Zeit}{}\ledrightnote{\textcolor{brown}{Die Zeit. Wiener Wochenschrift}}« beſtimmten Zuſchriften und Sendungen
                  ſind an die Redaction der »\textcolor{brown}{Zeit}{}\ledrightnote{\textcolor{brown}{Die Zeit. Wiener Wochenschrift}}« und \textbf{nicht} an die Perſon eines der Herausgeber oder Mitarbeiter zu
                     richten.}{\lemma{\textnormal{\emph{Alle … richten.}}}\Cendnote{\textnormal{am unteren
                     Rand der ersten Seite}}}\label{T_L00765_1h}}}\pend
           \endnumbering\briefempfaengerindex{Schnitzler, Arthur@\textsc{Schnitzler, Arthur}!zzzBahr, Hermann@\emph{von Hermann Bahr}!1898-01-241@{24. 1. 1898}|)be}\mylabel{h}  \normalsize

\doendnotes{C}
\bigskip
\vfill

\clearpage

\footnotesize

\lohead{\textsc{register}}

% Definiere theindex-Environment komplett neu ohne reledmac
\makeatletter
\renewenvironment{theindex}{%
  \section*{\indexname}%
  \setlength{\parindent}{0pt}%
  \setlength{\parskip}{0pt plus 0.3pt}%
  \let\item\@idxitem
}{%
  \clearpage
}
\makeatother

\IfFileExists{\jobname-pw.ind}{\input{\jobname-pw.ind}}{}

\end{document}

      