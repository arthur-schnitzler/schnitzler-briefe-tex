%% latex-korrekturansicht-vorspann.tex
%% Vorspann für die Korrekturansicht.
%% Lädt die gemeinsame Datei latex-vorspann.tex mit gesetztem Schalter.

\newif\ifkorrekturansicht
\korrekturansichttrue

\input{../tex-inputs/latex-vorspann}


               \section[Hugo von Hofmannsthal an Arthur Schnitzler, 19. 3. 1892]{ Hugo von Hofmannsthal an Arthur Schnitzler, 19. 3. 1892}\nopagebreak\mylabel{v}\rehead{ }\normalsize\beginnumbering\briefempfaengerindex{Schnitzler, Arthur@\textsc{Schnitzler, Arthur}!zzzHofmannsthal, Hugo von@\emph{von Hugo von Hofmannsthal}!1892-03-191@{19. 3. 1892}|(be} \toendnotes[C]{\smallbreak\pagebreak[2]} \Standort{CUL, Schnitzler, B 43.}
\physDesc{Postkarte
\newline{}Handschrift: schwarze Tinte, deutsche Kurrent\newline{}Versand: 1) Rohrpost 2) Stempel: »\nobreak{}\oindex{III., Landstrasse@\textbf{III., Landstraße}, \emph{Bezirk (A.BZK)}|pwk}Wien 3/1 40, 19. 3. 92, 1–2N\nobreak{}«. 3) Stempel: »\nobreak{}\oindex{Kaerntnerring@\textbf{Kärntnerring}, \emph{Straße (K.STR)}|pwk}Wien Kärntnerring, 19. 3. 92, 1–2N\nobreak{}«. 
\newline{}Schnitzler: mit Bleistift datiert: »19/3 92« und nummeriert: »20« }\buchAbdrucke{\weitereDrucke{Hugo von Hofmannsthal, Arthur Schnitzler: \emph{Briefwechsel}. Hg. Therese Nickl und Heinrich Schnitzler. Frankfurt am Main: \emph{S. Fischer} 1964, S. 18.} }\toendnotes[C]{\smallbreak}\pstart{}{\pb}Herrn \textsc{D\textsuperscript{r} Arthur Schnitzler}\pend{}\pstart{}\textsc{\textcolor{pink}{Wien}{}\ledrightnote{\textcolor{pink}{Wien}}}\pend{}\pstart{}\textsc{\textcolor{pink}{I. Kärnthnerring 12}{}\ledrightnote{\textcolor{pink}{Kärntnerring}}}\pend{}\pstart{}\textsc{II Stiege 3 Stock}\pend{}{\bigskip}\pstart{}{\pb}Lieber Freund.\pend\pstart
           Das erſtemal ſchreibe ich einen Brief an Sie ängſtlich. Ich muſs nämlich ſehr
                    unartig ſein. Verzeihen Sie, bitte. \textcolor{blue}{Kainz}{}\ledrightnote{\textcolor{blue}{Josef Kainz}},
                    dem ich irgend einen Sonntag nach \textcolor{pink}{Purkersdorf}{}\ledrightnote{\textcolor{pink}{Purkersdorf}}
                    zu kommen verſprochen hatte, reiſt Montag nach \textcolor{pink}{Graz}{}\ledrightnote{\textcolor{pink}{Graz}}, \textcolor{pink}{Prag}{}\ledrightnote{\textcolor{pink}{Prag}}, \textcolor{pink}{Moskau}{}\ledrightnote{\textcolor{pink}{Moskau}}{ }\textsc{etc}. und will mich abſolut morgen draußen haben. Bitte
                    bedenken Sie alſo, daſs \textcolor{blue}{Kainz}{}\ledrightnote{\textcolor{blue}{Josef Kainz}} für mich
                    dasſelbe vorſtellt, wie \textcolor{blue}{Reicher}{}\ledrightnote{\textcolor{blue}{Emanuel Reicher}} für Sie und
                    entſchuldigen Sie dieſen Eingriff der Außendinge in das Unſere. Ich komme
                    vielleicht \label{K_L00085_1v}\edtext{Montag}{\lemma{\textnormal{\emph{Montag}}}\Cendnote{\textnormal{Tatsächlich kam \textcolor{blue}{Hofmannsthal} am Montag, dem 21. 3. 1892
                        vorbei.}}}\label{K_L00085_1h} zu Ihnen und wir verabreden gleich irgend eine Stunde.\pend
           \pstart
           Herzlichſt{\\[\baselineskip]}\spacefill\mbox{Loris.}\pend
           \leftskip=0em{}\pstart
           \noindent{}Bitte auch \textcolor{blue}{Salten}{}\ledrightnote{\textcolor{blue}{Felix Salten}} grüßen und
                        entſchuldigen.\pend
           \endnumbering\briefempfaengerindex{Schnitzler, Arthur@\textsc{Schnitzler, Arthur}!zzzHofmannsthal, Hugo von@\emph{von Hugo von Hofmannsthal}!1892-03-191@{19. 3. 1892}|)be}\mylabel{h}  \normalsize

\doendnotes{C}
\bigskip
\vfill

\clearpage

\footnotesize

\lohead{\textsc{register}}

% Definiere theindex-Environment komplett neu ohne reledmac
\makeatletter
\renewenvironment{theindex}{%
  \section*{\indexname}%
  \setlength{\parindent}{0pt}%
  \setlength{\parskip}{0pt plus 0.3pt}%
  \let\item\@idxitem
}{%
  \clearpage
}
\makeatother

\IfFileExists{\jobname-pw.ind}{\input{\jobname-pw.ind}}{}

\end{document}

      