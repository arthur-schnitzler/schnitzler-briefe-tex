%% latex-korrekturansicht-vorspann.tex
%% Vorspann für die Korrekturansicht.
%% Lädt die gemeinsame Datei latex-vorspann.tex mit gesetztem Schalter.

\newif\ifkorrekturansicht
\korrekturansichttrue

\input{../tex-inputs/latex-vorspann}


               \section[Hugo von Hofmannsthal an Arthur Schnitzler, {[}4. 3. 1906{]}]{ Hugo von Hofmannsthal an Arthur Schnitzler, {[}4. 3. 1906{]}}\nopagebreak\mylabel{v}\rehead{ }\normalsize\beginnumbering\briefempfaengerindex{Schnitzler, Arthur@\textsc{Schnitzler, Arthur}!zzzHofmannsthal, Hugo von@\emph{von Hugo von Hofmannsthal}!1906-03-041@{{[}4. 3. 1906{]}}|(be} \toendnotes[C]{\smallbreak\pagebreak[2]} \Standort{CUL, Schnitzler, B 43.}
\physDesc{Brief, 1 Blatt, 4 Seiten
\newline{}Handschrift: schwarze Tinte, deutsche Kurrent
\newline{}Schnitzler: mit Bleistift datiert: »4/3 906« \newline{}Ordnung: 1) mit Bleistift von unbekannter Hand nummeriert: »\strikeout{264}« 2) mit Bleistift von unbekannter Hand nummeriert:
                                    »261«}\buchAbdrucke{\weitereDrucke{Hugo von Hofmannsthal, Arthur Schnitzler: \emph{Briefwechsel}. Hg. Therese Nickl und Heinrich Schnitzler. Frankfurt am Main: \emph{S. Fischer} 1964, S. 217.} }\toendnotes[C]{\smallbreak}\pstart
           \raggedleft{}{\pb}Sonntag.\pend
           \pstart{}mein lieber Arthur \pend\pstart
           ich wünſche mir ſo ſehr, ein paar Stunden mit Ihnen ruhig zu verbringen, von Ihrem
                  \textcolor{green}{Stück}{}\ledrightnote{→\textcolor{green}{Der Ruf des Lebens. Schauspiel in drei Akten}} zu reden, das ich ſo
               ſehr ſchön finde (habs wieder geleſen) und von anderen Dingen.\pend
           \pstart
           Bitte ſchlagen Sie uns einen Abend der Woche vor, uns iſt jeder recht. Soll man denn
               alt werden und einander ſo wenig gehabt haben? – Völlig {\pb}bestürzt, direct getroffen wie von
               etwas ganz Schlechtem, die Nerven aufregenden bin ich von dieſem unſinnigen brutalen
                  \label{K_L01586_1v}\edtext{\textcolor{green}{Aufſatz}{}\ledrightnote{→\textcolor{green}{Theater}}}{\lemma{\textnormal{\emph{Aufſatz}}}\Cendnote{\textnormal{\textcolor{blue}{Harden} hatte einer längeren, ausführlichen
                  Besprechung von \emph{\textcolor{green}{Ödipus und die Sphinx}} einen
                  einseitigen Verriss von \emph{\textcolor{green}{Der Ruf des Lebens}}
                  angehängt (\textcolor{blue}{M. H.}: \emph{\textcolor{green}{Theater}}. In: \emph{\textcolor{green}{Die Zukunft}}, Bd. 54,
                     H. 9, 3. 3. 1906, S. 346–356).}}}\label{K_L01586_1h} von \textcolor{blue}{\textsc{Harden}}{}\ledrightnote{\textcolor{blue}{Maximilian Harden}}. So muſs man ſich denn entſchließen, dieſen bedeutenden Menſchen zu den
               pathologiſchen Existenzen, deren Gefährlichkeit mit ihrer {\pb}Unberechenbarkeit wächſt, zu
               werfen! Wie traurig. Ich mühe mich, es zu begreifen, die Wurzel dieſer wilden, um
               ſich freſſenden Parteilichkeit, dieſer fieberhaften Zerrüttung zu faſſen –\hspace*{1.5em}Ich habe an ihn \label{K_L01586_2v}\edtext{geſchrieben}{\lemma{\textnormal{\emph{geſchrieben}}}\Cendnote{\textnormal{der
                  Brief vom 4. 3. 1906 (Hans Georg Schede, Hg.: \emph{Hugo von Hofmannsthal – Maximilian Harden}. In: \emph{Hofmannsthal-Jahrbuch}, Jg. 6, 1998, S. 93–97).
                  Die noch harschere Antwort \textcolor{blue}{Harden}s ist nicht
                  überliefert, \textcolor{blue}{Hofmannsthal} zog dann aber – wohl
                  in Abstimmung mit \textcolor{blue}{Schnitzler} – seinen Vorschlag
                  einer Replik zurück.}}}\label{K_L01586_2h}, mit den bitterſten Vorwürfen und ihn gefragt, ob er
               mir {\pb}erlauben will, in der \textcolor{green}{Zukunft}{}\ledrightnote{\textcolor{green}{Die Zukunft}} ein »Geſpräch über einige neue
               Theaterſtücke« (ich denke an \textcolor{green}{Ruf des Lebens}{}\ledrightnote{\textcolor{green}{Der Ruf des Lebens. Schauspiel in drei Akten}} – \textcolor{green}{Pippa}{}\ledrightnote{\textcolor{green}{Und Pippa tanzt!}} – \textcolor{green}{Leidenſchaft}{}\ledrightnote{\textcolor{green}{Leidenschaft. Trauerspiel in fünf Aufzügen}}) zu bringen. Bin neugierig, was er antwortet.\pend
           \pstart
           Ihr{\\[\baselineskip]}\spacefill\mbox{Hugo.}\pend
           \leftskip=0em{}\endnumbering\briefempfaengerindex{Schnitzler, Arthur@\textsc{Schnitzler, Arthur}!zzzHofmannsthal, Hugo von@\emph{von Hugo von Hofmannsthal}!1906-03-041@{{[}4. 3. 1906{]}}|)be}\mylabel{h}  \normalsize

\doendnotes{C}
\bigskip
\vfill

\clearpage

\footnotesize

\lohead{\textsc{register}}

% Definiere theindex-Environment komplett neu ohne reledmac
\makeatletter
\renewenvironment{theindex}{%
  \section*{\indexname}%
  \setlength{\parindent}{0pt}%
  \setlength{\parskip}{0pt plus 0.3pt}%
  \let\item\@idxitem
}{%
  \clearpage
}
\makeatother

\IfFileExists{\jobname-pw.ind}{\input{\jobname-pw.ind}}{}

\end{document}

      