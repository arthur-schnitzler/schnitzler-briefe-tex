%% latex-korrekturansicht-vorspann.tex
%% Vorspann für die Korrekturansicht.
%% Lädt die gemeinsame Datei latex-vorspann.tex mit gesetztem Schalter.

\newif\ifkorrekturansicht
\korrekturansichttrue

\input{../tex-inputs/latex-vorspann}


               \section[Arthur Schnitzler an Richard Beer-Hofmann, 16. 7. 1920]{ Arthur Schnitzler an Richard Beer-Hofmann, 16. 7. 1920}\nopagebreak\mylabel{v}\rehead{ }\normalsize\beginnumbering\briefempfaengerindex{Beer-Hofmann, Richard@\textsc{Beer-Hofmann, Richard}!zzzSchnitzler, Arthur@\emph{von Arthur Schnitzler}!1920-07-161@{16. 7. 1920}|(be} \toendnotes[C]{\smallbreak\pagebreak[2]} \Standort{YCGL, MSS 31.}
\physDesc{Brief, 1 Blatt, 2 Seiten, Umschlag
\newline{}Handschrift: Bleistift, lateinische Kurrent\newline{}Versand: Stempel: »\nobreak{}Wien 1\textcolor{gray}{1}0, 16. VII. 20, \textcolor{gray}{6}\nobreak{}«.  }\buchAbdrucke{\weitereDrucke{Arthur Schnitzler, Richard Beer-Hofmann: \emph{Briefwechsel 1891–1931}. Hg. Konstanze Fliedl. Wien, Zürich: \emph{Europaverlag} 1992, S. 227–228.} }\toendnotes[C]{\smallbreak}\pstart{}{\pb}A. S. \textcolor{pink}{Wien XVIII
                     Sternwartestr 71}{}\ledrightnote{\textcolor{pink}{Sternwartestraße}}.\pend{}{\bigskip}\pstart{}{\pb}Hrn Dr. Richard Beer Hofmann\pend{}\pstart{}\textcolor{pink}{Markt Aussee}{}\ledrightnote{\textcolor{pink}{Bad Aussee}}\pend{}\pstart{}\textcolor{pink}{Gartengasse}{}\ledrightnote{\textcolor{pink}{Gartengasse}}\pend{}\pstart{}\textcolor{pink}{Steiermark}{}\ledrightnote{\textcolor{pink}{Steiermark}}\pend{}{\bigskip}\pstart
           \raggedleft{}{\pb}\textcolor{pink}{Wien}{}\ledrightnote{\textcolor{pink}{Wien}}{ }16. 7. 1920\pend
           \pstart{}lieber Richard,\pend\pstart
           über den Vorschlag \textcolor{blue}{Fischer}{}\ledrightnote{\textcolor{blue}{Samuel Fischer}} denk ich wie Sie, daß
               uns unter den augenblicklichen Verhältnissen kaum was übrig bleiben wird als
               anzunehmen, ist klar. Gegen all das wird man sich erst wehren können, wenn eine
               völlige in jeder Hinsicht gewährleistete und gesetzlich geschützte Solidarität der
               Schriftsteller bestehen wird – und ob nicht sogar da{\geminationn}
               die Unternehmersolidarität den Sieg davontragen wird, bleibt fraglich. \textcolor{blue}{Hugo}{}\ledrightnote{\textcolor{blue}{Hugo von Hofmannsthal}} war \label{KLL02350_Beer-Hofmann-1v}\edtext{gestern}{\lemma{\textnormal{\emph{gestern}}}\Cendnote{\textnormal{siehe A. S.: \emph{Tagebuch}, 15. 7. 1920}}}\label{KLL02350_Beer-Hofmann-1h} bei mir; er ist ungefähr der gleichen Ansicht. Ich bin eben wieder in einer
               »scharfen« Correspondenz mit \textcolor{blue}{Fischer}{}\ledrightnote{\textcolor{blue}{Samuel Fischer}} begriffen,
               wegen meiner »\textcolor{green}{Gesa{\geminationm}elten}{}\ledrightnote{\textcolor{green}{Gesammelte Werke}}«, ich »reagiere ab« aber sonst ko{\geminationm}t
               nicht viel {\pb}dabei heraus. –\pend
           \pstart
           Unsre Sommerpläne sind noch immer so vag als möglich. Frau \textcolor{blue}{Lucy von Jacoby}{}\ledrightnote{\textcolor{blue}{Lucy von Jacobi}} wohnt jetzt bei uns; wahrscheinlich wird \textcolor{blue}{Olga}{}\ledrightnote{\textcolor{blue}{Olga Schnitzler}} mit ihr nach \textcolor{pink}{Salzburg}{}\ledrightnote{\textcolor{pink}{Salzburg}} oder \textcolor{pink}{Bayern}{}\ledrightnote{\textcolor{pink}{Bayern}} fahren, und es ist
               möglich, dſs man sich etwa am 15. August irgendwo trifft. \textcolor{pink}{Abtenau}{}\ledrightnote{\textcolor{pink}{Abtenau}} (\textcolor{pink}{Curh\substVorne{}\textsuperscript{ot}\substDazwischen{}aus\substHinten{}}{}\ledrightnote{\textcolor{pink}{Kurhaus Abtenau-Bad}}) wird in Erwägung gezogen.\pend
           \pstart
           Lassen Sie sichs wohl ergehen mein lieber Richard grüßen Sie die Ihren\pend
           \pstart
           Von Herzen Ihr{\\[\baselineskip]}\spacefill\mbox{Arthur}\pend
           \leftskip=0em{}\endnumbering\briefempfaengerindex{Beer-Hofmann, Richard@\textsc{Beer-Hofmann, Richard}!zzzSchnitzler, Arthur@\emph{von Arthur Schnitzler}!1920-07-161@{16. 7. 1920}|)be}\mylabel{h}  \normalsize

\doendnotes{C}
\bigskip
\vfill

\clearpage

\footnotesize

\lohead{\textsc{register}}

% Definiere theindex-Environment komplett neu ohne reledmac
\makeatletter
\renewenvironment{theindex}{%
  \section*{\indexname}%
  \setlength{\parindent}{0pt}%
  \setlength{\parskip}{0pt plus 0.3pt}%
  \let\item\@idxitem
}{%
  \clearpage
}
\makeatother

\IfFileExists{\jobname-pw.ind}{\input{\jobname-pw.ind}}{}

\end{document}

      