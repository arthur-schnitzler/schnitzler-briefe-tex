%% latex-korrekturansicht-vorspann.tex
%% Vorspann für die Korrekturansicht.
%% Lädt die gemeinsame Datei latex-vorspann.tex mit gesetztem Schalter.

\newif\ifkorrekturansicht
\korrekturansichttrue

\input{../tex-inputs/latex-vorspann}


               \section[Max Burckhard an Arthur Schnitzler, 15. 9. 1895]{ Max Burckhard an Arthur Schnitzler, 15. 9. 1895}\nopagebreak\mylabel{v}\rehead{ }\normalsize\beginnumbering\briefempfaengerindex{Schnitzler, Arthur@\textsc{Schnitzler, Arthur}!zzzBurckhard, Max Eugen@\emph{von Max Eugen Burckhard}!1895-09-152@{15. 9. 1895}|(be} \toendnotes[C]{\smallbreak\pagebreak[2]} \Standort{CUL, Schnitzler, B 20.}
\physDesc{Brief, 1 Blatt, 2 Seiten
\newline{}Handschrift: schwarze Tinte, deutsche Kurrent\newline{}Ordnung: mit rotem Buntstift von unbekannter Hand nummeriert:
                                        »6.«, mutmaßlich von anderer Hand mit
                                    Bleistift durchgestrichen und nummeriert:
                                    »7« }\toendnotes[C]{\smallbreak}\pstart
           \noindent{}{\pb}\textcolor{gray}{\textbf{\textcolor{brown}{\label{T_L00484-1v}\edtext{k. k. Hofburgtheater Direction}{\lemma{\textnormal{\emph{k. k. … Direction}}}\Cendnote{\textnormal{Wappen in Prägedruck}}}\label{T_L00484-1h}}{}\ledrightnote{\textcolor{brown}{Burgtheater}}}}\hfill \textcolor{pink}{Wien}{}\ledrightnote{\textcolor{pink}{Wien}}{ }15. 9. 95\pend
           \pstart{}Sehr verehrter Herr Doctor!\pend\pstart
           Ich bin ſo frei Sie herzlichſt zur \textcolor{green}{Leſeprobe}{}\ledrightnote{→\textcolor{green}{Liebelei. Schauspiel in drei Akten}} für Mittwoch 18 d. M. einzuladen. \uline{Es iſt Alles in Ordnung}. Ich bin leider an dem
                    Tage in \textcolor{pink}{Sprottau}{}\ledrightnote{\textcolor{pink}{Sprottau}}, Hr \textcolor{blue}{So{\geminationn}enthal}{}\ledrightnote{\textcolor{blue}{Adolf von Sonnenthal}} wird die Leſeprobe
                    leiten. Wenn etwas mit dem Dialect nicht zuſa{\geminationm}engeht, machen Sie ſich nichts draus, bei den Proben werde ich das ſchon
                    ausgleichen. Eine Rolle habe ich doch anders beſetzt – die \textcolor{green}{Katharina}{}\ledrightnote{→\textcolor{green}{Liebelei. Schauspiel in drei Akten}} mit der \textcolor{blue}{Walbeck}{}\ledrightnote{\textcolor{blue}{Fanny Walbeck}}: die \textcolor{blue}{Bauer}{}\ledrightnote{\textcolor{blue}{Anna Bauer}}
                    iſt zu fein; ich werde die \textcolor{blue}{Walbeck}{}\ledrightnote{\textcolor{blue}{Fanny Walbeck}}{ }ſchon »zurückhalten«.\pend
           \pstart
           {\pb}Ich habe jetzt auch einen Einakter
                    dazu, der würdig iſt und doch nicht im Styl widerſtreitet: \textcolor{blue}{\textsc{Giacosa}}{}\ledrightnote{\textcolor{blue}{Giuseppe Giacosa}}’s \textcolor{green}{Rechte der Seele}{}\ledrightnote{\textcolor{green}{Rechte der Seele}}.\pend
           \pstart
           Anfangs Oktober hoffe ich ſind wir heraußen.\pend
           \pstart
           Herzlichſt Ihr ergebener{\\[\baselineskip]}\spacefill\mbox{D\textsuperscript{r}Burckhard}\pend
           \leftskip=0em{}\endnumbering\briefempfaengerindex{Schnitzler, Arthur@\textsc{Schnitzler, Arthur}!zzzBurckhard, Max Eugen@\emph{von Max Eugen Burckhard}!1895-09-152@{15. 9. 1895}|)be}\mylabel{h}  \normalsize

\doendnotes{C}
\bigskip
\vfill

\clearpage

\footnotesize

\lohead{\textsc{register}}

% Definiere theindex-Environment komplett neu ohne reledmac
\makeatletter
\renewenvironment{theindex}{%
  \section*{\indexname}%
  \setlength{\parindent}{0pt}%
  \setlength{\parskip}{0pt plus 0.3pt}%
  \let\item\@idxitem
}{%
  \clearpage
}
\makeatother

\IfFileExists{\jobname-pw.ind}{\input{\jobname-pw.ind}}{}

\end{document}

      