%% latex-korrekturansicht-vorspann.tex
%% Vorspann für die Korrekturansicht.
%% Lädt die gemeinsame Datei latex-vorspann.tex mit gesetztem Schalter.

\newif\ifkorrekturansicht
\korrekturansichttrue

\input{../tex-inputs/latex-vorspann}


               \section[Arthur Schnitzler an Hugo von Hofmannsthal, {[}15. 1. 1909?{]}]{ Arthur Schnitzler an Hugo von Hofmannsthal, {[}15. 1. 1909?{]}}\nopagebreak\mylabel{v}\rehead{ }\normalsize\beginnumbering\briefempfaengerindex{Hofmannsthal, Hugo von@\textsc{Hofmannsthal, Hugo von}!zzzSchnitzler, Arthur@\emph{von Arthur Schnitzler}!1909-01-151@{{[}15. 1. 1909?{]}}|(be} \toendnotes[C]{\smallbreak\pagebreak[2]} \Standort{FDH, Hs-30885,142.}
\physDesc{Brief, 1 Blatt, 1 Seite
\newline{}Handschrift: schwarze Tinte, deutsche Kurrent\newline{}Ordnung: von Schnitzler – mutmaßlich bei
                           der Durchsicht der Briefe 1929 – mit Bleistift datiert: »1910?« }\buchAbdrucke{\weitereDrucke{Hugo von Hofmannsthal, Arthur Schnitzler: \emph{Briefwechsel}. Hg. Therese Nickl und Heinrich Schnitzler. Frankfurt am Main: \emph{S. Fischer} 1964, S. 259.} }\toendnotes[C]{\smallbreak}\pstart
           \noindent{}{\pb}Ja richtig, eine Frage – we{\geminationn} Sie glauben ſie beantworten zu dürfen: wieviel haben Sie von der \textcolor{brown}{Oest. Rundſchau}{}\ledrightnote{\textcolor{brown}{Österreichische Rundschau}} für den \label{K_L01821_1v}\edtext{\textcolor{green}{\textsc{Cristina-Act}}{}\ledrightnote{\textcolor{green}{Cristinas Heimreise. Komödie}}}{\lemma{\textnormal{\emph{Cristina-Act}}}\Cendnote{\textnormal{\textcolor{blue}{Hugo von Hofmannsthal}: \emph{\textcolor{green}{Komödie in Prosa}}. In: \emph{\textcolor{green}{Österreichische Rundschau}}, Bd. 18, H. 1, 1. 1. 1909,
                     S. 11–23.}}}\label{K_L01821_1h} Honorar gekriegt? (Weil ich ihnen nemlich auch einen
               erſten \label{K_L01821_2v}\edtext{Act geben}{\lemma{\textnormal{\emph{Act geben}}}\Cendnote{\textnormal{\textcolor{blue}{Schnitzler}s Kontaktpersonen zur \emph{\textcolor{brown}{Österreichischen Rundschau}} waren die beiden Herausgeber \textcolor{blue}{Karl Glossy} und \textcolor{blue}{Felix Oppenheimer}. Die nachweisbaren Kontakte 1910 sind zu
                  Zeiten, an denen \textcolor{blue}{Hofmannsthal} sich gerade auf
                  Reisen befindet. Eine solche formlose Anfrage scheint damit eher unwahrscheinlich.
                  Zwei Wochen nach Erscheinen des teilweisen Vorabdrucks von \emph{\textcolor{green}{Cristinas Heimreise}} (\emph{\textcolor{green}{Komödie in
                     Prosa}}) – am 15. 1. 1909 – vermerkt sich \textcolor{blue}{Schnitzler} den Besuch \textcolor{blue}{Oppenheimer}s,
                  was mutmaßlich auch der Ausgangspunkt für diese Überlegung darstellt. In der \emph{\textcolor{green}{Österreichischen Rundschau}} erschien in Folge
                  nichts von Schnitzler.}}}\label{K_L01821_2h} will.)\pend
           \endnumbering\briefempfaengerindex{Hofmannsthal, Hugo von@\textsc{Hofmannsthal, Hugo von}!zzzSchnitzler, Arthur@\emph{von Arthur Schnitzler}!1909-01-151@{{[}15. 1. 1909?{]}}|)be}\mylabel{h}  \normalsize

\doendnotes{C}
\bigskip
\vfill

\clearpage

\footnotesize

\lohead{\textsc{register}}

% Definiere theindex-Environment komplett neu ohne reledmac
\makeatletter
\renewenvironment{theindex}{%
  \section*{\indexname}%
  \setlength{\parindent}{0pt}%
  \setlength{\parskip}{0pt plus 0.3pt}%
  \let\item\@idxitem
}{%
  \clearpage
}
\makeatother

\IfFileExists{\jobname-pw.ind}{\input{\jobname-pw.ind}}{}

\end{document}

      