%% latex-korrekturansicht-vorspann.tex
%% Vorspann für die Korrekturansicht.
%% Lädt die gemeinsame Datei latex-vorspann.tex mit gesetztem Schalter.

\newif\ifkorrekturansicht
\korrekturansichttrue

\input{../tex-inputs/latex-vorspann}


               \section[Arthur Schnitzler an Richard Beer-Hofmann, 19. 10. 1894]{ Arthur Schnitzler an Richard Beer-Hofmann, 19. 10. 1894}\nopagebreak\mylabel{v}\rehead{ }\normalsize\beginnumbering\briefempfaengerindex{Beer-Hofmann, Richard@\textsc{Beer-Hofmann, Richard}!zzzSchnitzler, Arthur@\emph{von Arthur Schnitzler}!1894-10-191@{19. 10. 1894}|(be} \toendnotes[C]{\smallbreak\pagebreak[2]} \Standort{YCGL, MSS 31.}
\physDesc{Postkarte
\newline{}Handschrift: Bleistift, deutsche Kurrent\newline{}Versand: 1) nachgesandt nach \textcolor{pink}{\textsc{Hotel Hassler}} 2) Stempel: »\nobreak{}\oindex{I., Innere Stadt@\textbf{I., Innere Stadt}, \emph{Bezirk (A.BZK)}|pwk}Wien 1/1, 19. 10. 94, 9–10 N\nobreak{}«. 3) Stempel: »\nobreak{}\oindex{Neapel@\textbf{Neapel}, \emph{Besiedelter Ort (A.BSO)}|pwk}Napoli, 21 10–94, 8 S\nobreak{}«. }\buchAbdrucke{\weitereDrucke{Arthur Schnitzler, Richard Beer-Hofmann: \emph{Briefwechsel 1891–1931}. Hg. Konstanze Fliedl. Wien, Zürich: \emph{Europaverlag} 1992, S. 65.} }\toendnotes[C]{\smallbreak}\pstart{}{\pb}Herrn \textsc{Dr. Richard Beer
                     Hofmann}\pend{}\pstart{}\textsc{\textcolor{pink}{Neapel}{}\ledrightnote{\textcolor{pink}{Neapel}}}\pend{}\pstart{}\textsc{a posta ferma}\pend{}\pstart{}\textsc{\label{T_L00386_1v}\edtext{\textcolor{pink}{Italien}{}\ledrightnote{\textcolor{pink}{Italien}}}{\lemma{\textnormal{\emph{Italien}}}\Cendnote{\textnormal{in jede Ecke der Karte
                        geschrieben.}}}\label{T_L00386_1h}}\pend{}{\bigskip}\pstart
           \noindent{}{\pb}Lieber Richard, ich habe Ihren Brief aus
                  \textcolor{pink}{\textsc{Frascati}}{}\ledrightnote{\textcolor{pink}{Frascati}} beko{\geminationm}en und danke beſtens. Sie meinen
               erſten nach \textcolor{pink}{Neapel}{}\ledrightnote{\textcolor{pink}{Neapel}} und die \textcolor{green}{\uline{Zeit}}{}\ledrightnote{\textcolor{green}{Die Zeit. Wiener Wochenschrift}} doch wohl auch? Ihre gute und hohe Sti{\geminationm}ung
               iſt ſehr erfreulich – man kann gewiſs beſſeres von Reiſen heimbringen als Novellen –
               ob aber auch beſſeres – als \uline{Ihre} Novellen??? – Mein
                  \textcolor{green}{Stück}{}\ledrightnote{→\textcolor{green}{Liebelei. Schauspiel in drei Akten}} beim \textcolor{blue}{Abſchreiber}{}\ledrightnote{→\textcolor{blue}{?? [Schreibkraft für Arthur Schnitzler]}}; vielleicht ka{\geminationn} ich bei Ihrer
               Heimkehr ſchon mit Reſultaten aufwarten. Mache die Correcturen am Buch (\textcolor{green}{Sterben}{}\ledrightnote{\textcolor{green}{Sterben. Novelle}}.) – Heute arges Kopfweh. – Viele herzliche
               Grüße, bitte ſchreiben Sie mir.\pend
           \pstart Ihr \spacefill\mbox{Arth.}\pend{}\endnumbering\briefempfaengerindex{Beer-Hofmann, Richard@\textsc{Beer-Hofmann, Richard}!zzzSchnitzler, Arthur@\emph{von Arthur Schnitzler}!1894-10-191@{19. 10. 1894}|)be}\mylabel{h}  \normalsize

\doendnotes{C}
\bigskip
\vfill

\clearpage

\footnotesize

\lohead{\textsc{register}}

% Definiere theindex-Environment komplett neu ohne reledmac
\makeatletter
\renewenvironment{theindex}{%
  \section*{\indexname}%
  \setlength{\parindent}{0pt}%
  \setlength{\parskip}{0pt plus 0.3pt}%
  \let\item\@idxitem
}{%
  \clearpage
}
\makeatother

\IfFileExists{\jobname-pw.ind}{\input{\jobname-pw.ind}}{}

\end{document}

      