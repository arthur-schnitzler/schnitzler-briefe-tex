%% latex-korrekturansicht-vorspann.tex
%% Vorspann für die Korrekturansicht.
%% Lädt die gemeinsame Datei latex-vorspann.tex mit gesetztem Schalter.

\newif\ifkorrekturansicht
\korrekturansichttrue

\input{../tex-inputs/latex-vorspann}


               \section[Arthur Schnitzler an Richard Beer-Hofmann, 3. 8. 1900]{ Arthur Schnitzler an Richard Beer-Hofmann, 3. 8. 1900}\nopagebreak\mylabel{v}\rehead{ }\normalsize\beginnumbering\briefempfaengerindex{Beer-Hofmann, Richard@\textsc{Beer-Hofmann, Richard}!zzzSchnitzler, Arthur@\emph{von Arthur Schnitzler}!1900-08-031@{3. 8. 1900}|(be} \toendnotes[C]{\smallbreak\pagebreak[2]} \Standort{YCGL, MSS 31.}
\physDesc{Brief, 2 Blätter, 7 Seiten, Umschlag
\newline{}Handschrift: Bleistift, deutsche Kurrent\newline{}Versand: 1) Stempel: »\nobreak{}\oindex{Bad Ischl@\textbf{Bad Ischl}, \emph{Besiedelter Ort (A.BSO)}|pwk}Ischl, 3. 3. {[}1900{]}, 2–3N\nobreak{}«.  2) Stempel: »\nobreak{}\oindex{Altaussee@\textbf{Altaussee}, \emph{http://www.geonames.org/ontologyA.ADM3}|pwk}Alt-Aussee, 4/8 \textcolor{gray}{00}\nobreak{}«. 
\newline{}Beer-Hofmann: mit Bleistift am Umschlag eine Notiz in Lateinschrift: »\noindent{}{\pb}\uline{Tuch 20}{ / }\uline{Karten 40}{ / }Rahmen 18{ / }\uline{40}« }\buchAbdrucke{\weitereDrucke{Arthur Schnitzler, Richard Beer-Hofmann: \emph{Briefwechsel 1891–1931}. Hg. Konstanze Fliedl. Wien, Zürich: \emph{Europaverlag} 1992, S. 149–151.} }\toendnotes[C]{\smallbreak}\pstart{}{\pb}Herrn \textsc{Dr. Rich.
                     Beer-Hofmann}\pend{}\pstart{}\textsc{\textcolor{pink}{Altaussee}{}\ledrightnote{\textcolor{pink}{Altaussee}}}\pend{}{\bigskip}\pstart
           \raggedleft{}3. 8. 900.\pend
           \pstart
           {\pb}lieber Richard, ich ka{\geminationn} den Vortheil Ihres neuen
               Vorſchlag\textcolor{gray}{e}s nicht einſehn. Das miſſliche daran iſt: \uline{doch}{ }\textsc{per} Bahn nach \textcolor{pink}{Jenbach}{}\ledrightnote{\textcolor{pink}{Jenbach}}
               fahren müſſsen, dann wieder von \textcolor{pink}{Sterzing}{}\ledrightnote{\textcolor{pink}{Sterzing}} nach \textcolor{pink}{Innsbruck}{}\ledrightnote{\textcolor{pink}{Innsbruck}} zurück müſſen. Vergeſſen Sie nicht, unſre
               Abſicht iſt: von \textcolor{pink}{Zell a/See}{}\ledrightnote{\textcolor{pink}{Zell am See}} nach \textcolor{pink}{Innsbruck}{}\ledrightnote{\textcolor{pink}{Innsbruck}}, auf einem neuen Weg, zu kommen. {\pb}Überdies \substVorne{}\textsuperscript{\textcolor{gray}{×}}\substDazwischen{}k\substHinten{}oſtet Ihre Tour 1 Tag mehr, \textcolor{gray}{u}. \textcolor{blue}{Kerr}{}\ledrightnote{\textcolor{blue}{Alfred Kerr}} möchte uns in \textcolor{pink}{Innsbruck}{}\ledrightnote{\textcolor{pink}{Innsbruck}}
               treffen.\pend
           \pstart
           Nach \uline{meinem} Reiſebuch bietet das Pfitſcher Joch kaum mehr als \textcolor{pink}{\textsc{Krimml}}{}\ledrightnote{\textcolor{pink}{Krimml}} und \textcolor{pink}{\textsc{Gerlos}}{}\ledrightnote{\textcolor{pink}{Gerlos}}, und die Sache iſt weit bequemer.\pend
           \pstart
           Ich ſchlage alſo vor:\pend
           \pstart
           \textcolor{pink}{Salzburg}{}\ledrightnote{\textcolor{pink}{Salzburg}} ab Montag (ſpäteſtens
                  Dinſtag) Nachmittag 3.12.\pend
           \leftskip=3em{}\pstart
           \noindent{}{\pb}Ankunft Zell am See{ }5.43.\pend
           \leftskip=0em{}\leftskip=3em{}\pstart
           \textcolor{pink}{Poſt Keſſelfall}{}\ledrightnote{\textcolor{pink}{Alpenhaus Kesselfall}}\pend
           \leftskip=0em{}\leftskip=3em{}\pstart
           Übernachten.\pend
           \leftskip=0em{}\pstart
           \noindent{}\uline{Dinſtag.} (\textsc{resp}. Mittwoch)\pend
           \leftskip=3em{}\pstart
           \noindent{}Spazierg \textcolor{pink}{Moſerboden}{}\ledrightnote{\textcolor{pink}{Mooserboden}}, zurück \textcolor{pink}{Keſſelfall}{}\ledrightnote{\textcolor{pink}{Alpenhaus Kesselfall}}, bis \textcolor{pink}{Zell
                  am See}{}\ledrightnote{\textcolor{pink}{Zell am See}}\pend
           \leftskip=0em{}\leftskip=3em{}\pstart
           Bahn (4.50 nach \textcolor{pink}{\textsc{Kri{\geminationm}l}}{}\ledrightnote{\textcolor{pink}{Krimml}})\pend
           \leftskip=0em{}\leftskip=3em{}\pstart
           Übernachten.\pend
           \leftskip=0em{}\pstart
           \noindent{}\uline{Mittwoch}{ }\introOben{}(\textsc{resp}{ }Do{\geminationn})\introOben{}{ }\textcolor{pink}{\textsc{Kri{\geminationm}l}}{}\ledrightnote{\textcolor{pink}{Krimml}}{ }\textcolor{pink}{\textsc{Gerlos}}{}\ledrightnote{\textcolor{pink}{Gerlos}} (Fußpartie – 4 Stunden)\pend
           \pstart
           \textsc{\textcolor{pink}{Gerlos}{}\ledrightnote{\textcolor{pink}{Gerlos}}} – \textcolor{pink}{\textsc{Zell} (Zillerthal)}{}\ledrightnote{\textcolor{pink}{Zell am Ziller}} 4 Stunden\pend
           \pstart
           \textsc{\textcolor{pink}{Zell}{}\ledrightnote{\textcolor{pink}{Zell am Ziller}} – \textcolor{pink}{Jenbach}{}\ledrightnote{\textcolor{pink}{Jenbach}}} (Wagen\textcolor{gray}{)}\pend
           \leftskip=3em{}\pstart
           \noindent{}abds{ }\textcolor{pink}{Innsbruck}{}\ledrightnote{\textcolor{pink}{Innsbruck}}, 4 Stunden.\pend
           \leftskip=0em{}\pstart
           \noindent{}{\pb}Das Pfitſcher
                  Joch iſt einfach »lohnend«, hat nicht einmal einen Stern! – und iſt
               viel ſchwerer als \textcolor{pink}{\textsc{Gerlos}}{}\ledrightnote{\textcolor{pink}{Gerlos}}. –\pend
           \pstart
           Was nun die \textcolor{pink}{Schweiz}{}\ledrightnote{\textcolor{pink}{Schweiz}} anbelangt: Übergang direct
               nach \textcolor{pink}{\textsc{Klosters}}{}\ledrightnote{\textcolor{pink}{Klosters Dorf}} dem Überg nach \textcolor{pink}{\textsc{Küblis}}{}\ledrightnote{\textcolor{pink}{Küblis}} vorzuziehn, da wir jedenfalls nach \textcolor{pink}{\textsc{Klosters}}{}\ledrightnote{\textcolor{pink}{Klosters Dorf}}{ }{\pb}und von da nach \textcolor{pink}{\textsc{Davos}}{}\ledrightnote{\textcolor{pink}{Davos}} müſſen; von da \textcolor{pink}{\textsc{\uline{Flüelapass}}}{}\ledrightnote{\textcolor{pink}{Flüelapass}} nach \textcolor{pink}{\textsc{Samaden}}{}\ledrightnote{\textcolor{pink}{Samedan}} u \textcolor{pink}{\textsc{Pontresina}}{}\ledrightnote{\textcolor{pink}{Pontresina}}. (Fahrſtraſſe)\pend
           \pstart
           – Im übrigen werden wir keinen Richter brauchen, dagegen Träger. –\pend
           \pstart
           \textcolor{blue}{Georg H.}{}\ledrightnote{\textcolor{blue}{Georg Hirschfeld}} wird faſt ſicher \uline{nicht} mitko{\geminationm}en, obwohl ich ihn auf den
               Knieen beſchworen habe. Menſch{\pb}licher Vorausſicht nach
               (faſſen Sie dieſes »Menſch-« nicht falſch auf) werd’ ich Sonntag \introOben{}den\introOben{}{ }12. in \textcolor{pink}{Salzburg}{}\ledrightnote{\textcolor{pink}{Salzburg}}{ }ſein. Ich bin ſehr dafür, ſchon Montag
               abzufahren.\pend
           \pstart
           Von \textcolor{blue}{Schwarzk.}{}\ledrightnote{\textcolor{blue}{Gustav Schwarzkopf}} u \textcolor{blue}{Salten}{}\ledrightnote{\textcolor{blue}{Felix Salten}} noch keine Nachricht. Auch von \textcolor{blue}{Paul
                  G.}{}\ledrightnote{\textcolor{blue}{Paul Goldmann}} nichts neues. –\pend
           \pstart
           {\pb}Leben Sie wohl. –\pend
           \pstart
           Herzlichſt Ihr{\\[\baselineskip]}\spacefill\mbox{Arthur}\pend
           \leftskip=0em{}\pstart
           \textcolor{blue}{Hugo}{}\ledrightnote{\textcolor{blue}{Hugo von Hofmannsthal}} hat mir \label{K_L01063_1v}\edtext{geſchrieben}{\lemma{\textnormal{\emph{geſchrieben}}}\Cendnote{\textnormal{Hugo von Hofmannsthal an Arthur Schnitzler, 27. 7. 1900}}}\label{K_L01063_1h} iſt wohl ſchon in \textcolor{pink}{Salzburg}{}\ledrightnote{\textcolor{pink}{Salzburg}} bleibt bis
                  15. Er ſchrieb mir auch von ſeiner Verlobung.\pend
           \endnumbering\briefempfaengerindex{Beer-Hofmann, Richard@\textsc{Beer-Hofmann, Richard}!zzzSchnitzler, Arthur@\emph{von Arthur Schnitzler}!1900-08-031@{3. 8. 1900}|)be}\mylabel{h}  \normalsize

\doendnotes{C}
\bigskip
\vfill

\clearpage

\footnotesize

\lohead{\textsc{register}}

% Definiere theindex-Environment komplett neu ohne reledmac
\makeatletter
\renewenvironment{theindex}{%
  \section*{\indexname}%
  \setlength{\parindent}{0pt}%
  \setlength{\parskip}{0pt plus 0.3pt}%
  \let\item\@idxitem
}{%
  \clearpage
}
\makeatother

\IfFileExists{\jobname-pw.ind}{\input{\jobname-pw.ind}}{}

\end{document}

      