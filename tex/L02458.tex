%% latex-korrekturansicht-vorspann.tex
%% Vorspann für die Korrekturansicht.
%% Lädt die gemeinsame Datei latex-vorspann.tex mit gesetztem Schalter.

\newif\ifkorrekturansicht
\korrekturansichttrue

\input{../tex-inputs/latex-vorspann}


               \section[Felix Braun an Arthur Schnitzler, 14. 12. 1925]{ Felix Braun an Arthur Schnitzler, 14. 12. 1925}\nopagebreak\mylabel{v}\rehead{ }\normalsize\beginnumbering\briefempfaengerindex{Schnitzler, Arthur@\textsc{Schnitzler, Arthur}!zzzBraun, Felix@\emph{von Felix Braun}!1925-12-141@{14. 12. 1925}|(be} \toendnotes[C]{\smallbreak\pagebreak[2]} \Standort{DLA, A:Schnitzler, HS.NZ85.1.2604,6.}
\physDesc{Briefkarte
\newline{}Handschrift: schwarze Tinte, deutsche Kurrent
\newline{}Schnitzler: 1) mit Bleistift beschriftet: »\textsc{Braun}« 2) mit rotem Buntstift zwei Unterstreichungen}\toendnotes[C]{\smallbreak}\pstart
           \centering{}{\pb}\textcolor{pink}{Wien}{}\ledrightnote{\textcolor{pink}{Wien}}, den 14. XII. 25\pend
           \pstart{}Verehrter Herr Doktor!\pend\pstart
           Haben Sie den herzlichſten Dank für die Überſendung Ihres Buchs »\textcolor{green}{Die Frau des Richters}{}\ledrightnote{\textcolor{green}{Die Frau des Richters. Novelle}}« durch den \textcolor{brown}{Propyläen-Verlag}{}\ledrightnote{\textcolor{brown}{Propyläen Verlag}}. War ſchon der Empfang durch das
                    Bewußtſein, daß Sie ſelbſt, verehrter Herr Doktor, der Auftraggeber geweſen
                    ſind, eine große Freude, ſo auch die Lektüre. Denn ein meiſterliches Werk iſt
                    Ihnen da wieder und makellos geglückt. Sowohl die herrliche Proſa als auch die
                    Geſtaltung der Charaktere kann nur mit dem Prädikat der Meiſter{\pb}ſchaft gerühmt werden. Solange ſolche Bewältigungen möglich ſind, kann von
                    einem Abſtieg unſerer Zeit und Kunſt die Rede nicht ſein.\pend
           \pstart
           Immer war das Menſchliche – in einem weiteren als nur dem ethiſchen Sinn genommen
                    – Ihnen zu dichten gegeben: auch hier, am ſchönſten in der Geſtalt der \textcolor{green}{Frau}{}\ledrightnote{→\textcolor{green}{Die Frau des Richters. Novelle}}, und frei und leicht in
                    der des \textcolor{green}{Rebellen}{}\ledrightnote{→\textcolor{green}{Die Frau des Richters. Novelle}}, iſt es
                    Ihnen geglückt.\hspace*{1.5em}In Verehrung grüße ich Sie,
                    werter Herr Doktor, und ſage nochmals wärmſten Dank.\pend
           \pstart
           Ihr ergebener{\\[\baselineskip]}\spacefill\mbox{Felix Braun.}\pend
           \leftskip=0em{}\endnumbering\briefempfaengerindex{Schnitzler, Arthur@\textsc{Schnitzler, Arthur}!zzzBraun, Felix@\emph{von Felix Braun}!1925-12-141@{14. 12. 1925}|)be}\mylabel{h}  \normalsize

\doendnotes{C}
\bigskip
\vfill

\clearpage

\footnotesize

\lohead{\textsc{register}}

% Definiere theindex-Environment komplett neu ohne reledmac
\makeatletter
\renewenvironment{theindex}{%
  \section*{\indexname}%
  \setlength{\parindent}{0pt}%
  \setlength{\parskip}{0pt plus 0.3pt}%
  \let\item\@idxitem
}{%
  \clearpage
}
\makeatother

\IfFileExists{\jobname-pw.ind}{\input{\jobname-pw.ind}}{}

\end{document}

      