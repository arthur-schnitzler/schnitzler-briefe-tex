%% latex-korrekturansicht-vorspann.tex
%% Vorspann für die Korrekturansicht.
%% Lädt die gemeinsame Datei latex-vorspann.tex mit gesetztem Schalter.

\newif\ifkorrekturansicht
\korrekturansichttrue

\input{../tex-inputs/latex-vorspann}


               \section[Hugo von Hofmannsthal an Arthur Schnitzler, 22. 4. 1893]{ Hugo von Hofmannsthal an Arthur Schnitzler, 22. 4. 1893}\nopagebreak\mylabel{v}\rehead{ }\normalsize\beginnumbering\briefempfaengerindex{Schnitzler, Arthur@\textsc{Schnitzler, Arthur}!zzzHofmannsthal, Hugo von@\emph{von Hugo von Hofmannsthal}!1893-04-221@{22. 4. 1893}|(be} \toendnotes[C]{\smallbreak\pagebreak[2]} \Standort{CUL, Schnitzler, B 43.}
\physDesc{Postkarte
\newline{}Handschrift: Bleistift, deutsche Kurrent\newline{}Versand: 1) Rohrpost 2) Stempel: »\nobreak{}\oindex{III., Landstrasse@\textbf{III., Landstraße}, \emph{Bezirk (A.BZK)}|pwk}Wien 3/1, 22. IV. 93, 1030V\nobreak{}«. 
\newline{}Schnitzler: mit Bleistift nummeriert: »44« }\buchAbdrucke{\weitereDrucke{1) Hugo von Hofmannsthal, Arthur Schnitzler: \emph{Briefwechsel}. Hg. Therese Nickl und Heinrich Schnitzler. Frankfurt am Main: \emph{S. Fischer} 1964, S. 37.} \weitereDrucke{2) Hermann Bahr, Arthur Schnitzler: \emph{Briefwechsel, Aufzeichnungen, Dokumente
                                (1891–1931)}. Hg. Kurt Ifkovits und Martin Anton Müller. Göttingen: \emph{Wallstein} 2018, S. 35.} }\pstart{}{\pb}Herrn \textsc{D\textsuperscript{r}{ } Arthur Schnitzler}\pend{}\pstart{}\textsc{\textcolor{pink}{I Grillparzerstrasse 7}{}\ledrightnote{\textcolor{pink}{Grillparzerstraße}}}\pend{}\pstart{}\textsc{\textcolor{pink}{Wien}{}\ledrightnote{\textcolor{pink}{Wien}}}\pend{}{\bigskip}\pstart
           \noindent{}{\pb}\textcolor{blue}{Bahr}{}\ledrightnote{\textcolor{blue}{Hermann Bahr}}{ }ſagt: bei Zeitungen abſolut nichts zu
                    erreichen, als monatliche Annahme einiger Feuilletons. Wir haben 2 greifbare
                    Projecte ausgearbeitet. Details morgen. Verlangen Sie vor allem die genaue
                    Schuldenſumme zu erfahren: dann wird man einen Theil zahlen, der andere wird
                    wohl nachgelaſſen oder in nachträgl. Raten verwandelt werden können. Ich gehe
                    heute auf \textsc{\textcolor{blue}{Davids}{}\ledrightnote{\textcolor{blue}{Jakob Julius David}}} Aufforderung in die \textsc{\textcolor{brown}{Concordia}{}\ledrightnote{\textcolor{brown}{Concordia}}}.\pend
           \pstart \spacefill\mbox{Hugo}\pend{}\endnumbering\briefempfaengerindex{Schnitzler, Arthur@\textsc{Schnitzler, Arthur}!zzzHofmannsthal, Hugo von@\emph{von Hugo von Hofmannsthal}!1893-04-221@{22. 4. 1893}|)be}\mylabel{h}  \normalsize

\doendnotes{C}
\bigskip
\vfill

\clearpage

\footnotesize

\lohead{\textsc{register}}

% Definiere theindex-Environment komplett neu ohne reledmac
\makeatletter
\renewenvironment{theindex}{%
  \section*{\indexname}%
  \setlength{\parindent}{0pt}%
  \setlength{\parskip}{0pt plus 0.3pt}%
  \let\item\@idxitem
}{%
  \clearpage
}
\makeatother

\IfFileExists{\jobname-pw.ind}{\input{\jobname-pw.ind}}{}

\end{document}

      