%% latex-korrekturansicht-vorspann.tex
%% Vorspann für die Korrekturansicht.
%% Lädt die gemeinsame Datei latex-vorspann.tex mit gesetztem Schalter.

\newif\ifkorrekturansicht
\korrekturansichttrue

\input{../tex-inputs/latex-vorspann}


               \section[Hermann Bahr an Arthur Schnitzler, 10. 2. 1915]{ Hermann Bahr an Arthur Schnitzler, 10. 2. 1915}\nopagebreak\mylabel{v}\rehead{ }\normalsize\beginnumbering\briefempfaengerindex{Schnitzler, Arthur@\textsc{Schnitzler, Arthur}!zzzBahr, Hermann@\emph{von Hermann Bahr}!1915-02-101@{10. 2. 1915}|(be} \toendnotes[C]{\smallbreak\pagebreak[2]} \Standort{CUL, Schnitzler, B 5b.}
\physDesc{Brief, 1 Blatt, 1 Seite
\newline{}Handschrift: schwarze Tinte, deutsche Kurrent
\newline{}Schnitzler: 1) mit Bleistift ergänzt »Bahr« 2) mit rotem Buntstift eine Unterstreichung\newline{}Ordnung: mit Bleistift von unbekannter Hand
                           nummeriert: »181« }\buchAbdrucke{\weitereDrucke{Hermann Bahr, Arthur Schnitzler: \emph{Briefwechsel, Aufzeichnungen, Dokumente (1891–1931)}. Hg. Kurt Ifkovits und Martin Anton Müller. Göttingen: \emph{Wallstein} 2018, S. 497–498.} }\toendnotes[C]{\smallbreak}\pstart
           \raggedleft{}{\pb}10. 2. 15\pend
           \pstart\center{}Lieber Arthur!\pend\pstart
           Herzlichen Dank für den lieben Brief, der \textcolor{blue}{uns Beiden}{}\ledrightnote{→\textcolor{blue}{Anna Bahr-Mildenburg}} eine große Freude gemacht hat! Meine \textcolor{blue}{Frau}{}\ledrightnote{→\textcolor{blue}{Anna Bahr-Mildenburg}} möchte ſehr gern einmal in
                  \textcolor{pink}{Wien}{}\ledrightnote{\textcolor{pink}{Wien}} Lieder ſingen, \textcolor{blue}{Schubert}{}\ledrightnote{\textcolor{blue}{Franz Peter Schubert}}, \textcolor{blue}{Hugo Wolf}{}\ledrightnote{\textcolor{blue}{Hugo Wolf}} und die \textcolor{green}{Weſendoncklieder}{}\ledrightnote{\textcolor{green}{Fünf Gedichte von Mathilde Wesendonk für eine Frauenstimme und Klavier}} am liebſten. Jetzt aber geht das
               nicht, sie kann hier nicht abkommen von ihrem \label{K_L02204_1v}\edtext{\textcolor{pink}{Spital}{}\ledrightnote{→\textcolor{pink}{Krankenhaus der Barmherzigen Brüder}}}{\lemma{\textnormal{\emph{Spital}}}\Cendnote{\textnormal{Sie arbeitete als freiwillige
                  Pflegehelferin im \textcolor{pink}{Salzburger Truppenspital
                     Nonntal}.}}}\label{K_L02204_1h} (ich ſchrieb das \textcolor{blue}{Heller}{}\ledrightnote{\textcolor{blue}{Hugo Heller}} geſtern ſchon). Auch bin ich der Meinung, daß es beſſer iſt, dazu
               eine ſtillere, für Kunſt empfänglichere Zeit abzuwarten. Willſt Du aber nicht ſo
               lange warten, ſo komm doch her, Du kannſt es bei uns viel ſchöner haben als je in
               einem Konzert, was doch von vorneherein die ſcheußlichſte Kunstwidrigkeit iſt! Wir
               würden uns herzlich freuen und ich hätte ja ſo viel mit Dir zu reden, Tage lang!\pend
           \pstart
           Grüße Frau \textcolor{blue}{Olga}{}\ledrightnote{\textcolor{blue}{Olga Schnitzler}} in alter herzlicher
                  \textcolor{gray}{Ve}rehrung ſchönſtens von mir und kommt wirklich bald einmal!
               (Aber mit Nachricht ein paar Tage früher, damit ich nicht gerade weg bin, in \textcolor{pink}{München}{}\ledrightnote{\textcolor{pink}{München}} oder in den Bergen!)\pend
           \pstart
           Herzlichſt{\\[\baselineskip]}Dein alter{\\[\baselineskip]}\spacefill\mbox{H}\pend
           \leftskip=0em{}\endnumbering\briefempfaengerindex{Schnitzler, Arthur@\textsc{Schnitzler, Arthur}!zzzBahr, Hermann@\emph{von Hermann Bahr}!1915-02-101@{10. 2. 1915}|)be}\mylabel{h}  \normalsize

\doendnotes{C}
\bigskip
\vfill

\clearpage

\footnotesize

\lohead{\textsc{register}}

% Definiere theindex-Environment komplett neu ohne reledmac
\makeatletter
\renewenvironment{theindex}{%
  \section*{\indexname}%
  \setlength{\parindent}{0pt}%
  \setlength{\parskip}{0pt plus 0.3pt}%
  \let\item\@idxitem
}{%
  \clearpage
}
\makeatother

\IfFileExists{\jobname-pw.ind}{\input{\jobname-pw.ind}}{}

\end{document}

      