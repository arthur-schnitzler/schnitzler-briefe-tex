%% latex-korrekturansicht-vorspann.tex
%% Vorspann für die Korrekturansicht.
%% Lädt die gemeinsame Datei latex-vorspann.tex mit gesetztem Schalter.

\newif\ifkorrekturansicht
\korrekturansichttrue

\input{../tex-inputs/latex-vorspann}


               \section[Arthur Schnitzler an Robert Adam, 7. 2. 1911]{ Arthur Schnitzler an Robert Adam, 7. 2. 1911}\nopagebreak\mylabel{v}\rehead{ }\normalsize\beginnumbering\briefempfaengerindex{Adam, Robert@\textsc{Adam, Robert}!zzzSchnitzler, Arthur@\emph{von Arthur Schnitzler}!1911-02-071@{7. 2. 1911}|(be} \toendnotes[C]{\smallbreak\pagebreak[2]} \Standort{DLA, 96.34.1/4.}
\physDesc{Bildpostkarte
\newline{}Handschrift: schwarze Tinte, deutsche Kurrent\newline{}Versand: Stempel: »\nobreak{}\oindex{IX., Alsergrund@\textbf{IX., Alsergrund}, \emph{Bezirk (A.BZK)}|pwk}9/1 Wien, 7. {[}2. 1911{]}, 5\nobreak{}«.  }\pstart{}{\pb}Herrn \textsc{Robert
                        Adam},\pend{}\pstart{}\textcolor{pink}{Wien XII}{}\ledrightnote{\textcolor{pink}{XII., Meidling}}.\pend{}\pstart{}\textcolor{pink}{\textsc{Meidlinger Hptstr} 56}{}\ledrightnote{\textcolor{pink}{Meidlinger Hauptstraße}}.\pend{}{\bigskip}\pstart
           \noindent{}\centering{}{\pb}\textcolor{gray}{\textbf{\textcolor{pink}{Türkenschanz-Park}{}\ledrightnote{\textcolor{pink}{Türkenschanzpark}}{ }Aussichtsturm. \textcolor{pink}{Wien XVIII/1}{}\ledrightnote{\textcolor{pink}{XVIII., Währing}}}}\pend
           \pstart
           {\pb}Beſten Dank für die freundlichen Mittheilungen. Aller
                    Anfang {\dotsfour} Aber ich will nicht gar zu originell
                    ſein.\pend
           \pstart
           Verbindlichſt grüßt{\\[\baselineskip]}Ihr ſehr ergebner{\\[\baselineskip]}\spacefill\mbox{Arthur Schnitzler}\pend
           \leftskip=0em{}\pstart
           \raggedleft{}7/2. 911.\pend
           \endnumbering\briefempfaengerindex{Adam, Robert@\textsc{Adam, Robert}!zzzSchnitzler, Arthur@\emph{von Arthur Schnitzler}!1911-02-071@{7. 2. 1911}|)be}\mylabel{h}  \normalsize

\doendnotes{C}
\bigskip
\vfill

\clearpage

\footnotesize

\lohead{\textsc{register}}

% Definiere theindex-Environment komplett neu ohne reledmac
\makeatletter
\renewenvironment{theindex}{%
  \section*{\indexname}%
  \setlength{\parindent}{0pt}%
  \setlength{\parskip}{0pt plus 0.3pt}%
  \let\item\@idxitem
}{%
  \clearpage
}
\makeatother

\IfFileExists{\jobname-pw.ind}{\input{\jobname-pw.ind}}{}

\end{document}

      