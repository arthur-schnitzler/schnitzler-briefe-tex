%% latex-korrekturansicht-vorspann.tex
%% Vorspann für die Korrekturansicht.
%% Lädt die gemeinsame Datei latex-vorspann.tex mit gesetztem Schalter.

\newif\ifkorrekturansicht
\korrekturansichttrue

\input{../tex-inputs/latex-vorspann}


               \section[Arthur Schnitzler an Georg Brandes, 19. 5. 1899]{ Arthur Schnitzler an Georg Brandes, 19. 5. 1899}\nopagebreak\mylabel{v}\rehead{ }\normalsize\beginnumbering\briefempfaengerindex{Brandes, Georg@\textsc{Brandes, Georg}!zzzSchnitzler, Arthur@\emph{von Arthur Schnitzler}!1899-05-191@{19. 5. 1899}|(be} \toendnotes[C]{\smallbreak\pagebreak[2]} \Standort{Kopenhagen, Det Kongelige Bibliotek, Georg Brandes Arkiv, box 125.}
\physDesc{Brief, 2 Blätter, 6 Seiten
\newline{}Handschrift: schwarze Tinte, deutsche Kurrent\newline{}Ordnung: mit Bleistift von unbekannter Hand nummeriert und datiert:
                                    »16. Schnitzler 19/5 99« und auf der sechsten Seite:
                                 »Schnitzler« }\buchAbdrucke{\weitereDrucke{Georg Brandes, Arthur Schnitzler: \emph{Ein Briefwechsel}. Hg. Kurt Bergel. Bern: \emph{Francke} 1956, S. 77.} }\toendnotes[C]{\smallbreak}\pstart{}{\pb}Lieber und verehrter Herr Brandes,\pend\pstart
           innigen Dank für Ihre herzlichen Worte. Es iſt etwas erquickendes in der Art, wie Sie
               einem Worte ſagen, die von einem andern ausgeſprochen, eben nichts als Worte wären.
               Ich bin jung, ſagen Sie? Nun, wenn es ſelbſt ſo wäre – unter gewiſſen Umſtänden ſind
               Jugend, Frühling, Sonne ſo traurige Dinge, daſs man in ihrem Bewußtſein zuſa{\geminationm}enſchauert ſtatt ſich zu {\pb}freun. Dieſe Abende, die ich jetzt manchmal auf
               dem Land draußen verbringe, die Orte wo ich hinkomme, alles das dampft von
               Erinnerungen; – ahnt man denn, wie tief manche Gräber ſind! –\pend
           \pstart
           Verzeihen Sie daſs ich ſchon wieder davon rede; während Sie ſelbſt ohnedies nicht in
               der glücklichſten Sti{\geminationm}ung ſind. Ich wußte abſolut nicht,
               dſs Sie noch immer bettläge{\pb}rig \substVorne{}\textsuperscript{ſind}\substDazwischen{}waren\substHinten{}; wie gern möcht ich endlich hören, dſs Sie ganz geneſen ſind. Dabei iſt doch
               ſehr erfreulich, dſs die Sache völlig unbedenklich iſt und daſs Sie dabei arbeiten
               und ſich über den Zuſa{\geminationm}enfluſs von Büchern und Briefen
               auf Ihre\substVorne{}\textsuperscript{m}\substDazwischen{}r\substHinten{} Bettdecke freuen. Der Erfolg Ihrer \textcolor{green}{Geſa{\geminationm}tausgabe}{}\ledrightnote{→\textcolor{green}{Samlede Skrifter [Gesammelte Werke]}} iſt ja
               ſelbſtverſtändlich. \textcolor{blue}{Ludwig Fulda}{}\ledrightnote{\textcolor{blue}{Ludwig Fulda}}, auf deſſen
               Schreibtiſch ich vor ein paar Wochen {\pb}Ihre \textcolor{green}{Gedichte}{}\ledrightnote{→\textcolor{green}{Ungdomsvers [Jugendgedichte]}} liegen ſah, hab ich ein
               wenig um ſein \textcolor{pink}{däniſch}{}\ledrightnote{\textcolor{pink}{Dänemark}} können beneidet. Die \textcolor{brown}{Zukunft}{}\ledrightnote{\textcolor{brown}{Die Zukunft}}snu{\geminationm}er vom
                  7. April hab ich noch nicht geſehen, laſſe ſie mir durch meine
               Buchhandlung kommen.\pend
           \pstart
           Ich will in dieſem Frühjahr noch einige kleine Touren (mit dem Rade zumeiſt) in der
               Umgegend von \textcolor{pink}{Wien}{}\ledrightnote{\textcolor{pink}{Wien}} machen; immer neues entdeckt man in
               dieſem wunderſchönen aber vertrottelten \textcolor{pink}{Niederoeſterreich}{}\ledrightnote{\textcolor{pink}{Niederösterreich}}.\pend
           \pstart
           {\pb}Leben Sie wohl, mein verehrter Herr Brandes und
               ſeien vielmals gegrüßt.\pend
           \pstart Ihr \spacefill\mbox{ArthurSchnitzler}\pend{}\pstart
           19. 5. 99.\pend
           \endnumbering\briefempfaengerindex{Brandes, Georg@\textsc{Brandes, Georg}!zzzSchnitzler, Arthur@\emph{von Arthur Schnitzler}!1899-05-191@{19. 5. 1899}|)be}\mylabel{h}  \normalsize

\doendnotes{C}
\bigskip
\vfill

\clearpage

\footnotesize

\lohead{\textsc{register}}

% Definiere theindex-Environment komplett neu ohne reledmac
\makeatletter
\renewenvironment{theindex}{%
  \section*{\indexname}%
  \setlength{\parindent}{0pt}%
  \setlength{\parskip}{0pt plus 0.3pt}%
  \let\item\@idxitem
}{%
  \clearpage
}
\makeatother

\IfFileExists{\jobname-pw.ind}{\input{\jobname-pw.ind}}{}

\end{document}

      