%% latex-korrekturansicht-vorspann.tex
%% Vorspann für die Korrekturansicht.
%% Lädt die gemeinsame Datei latex-vorspann.tex mit gesetztem Schalter.

\newif\ifkorrekturansicht
\korrekturansichttrue

\input{../tex-inputs/latex-vorspann}


               \section[Arthur Schnitzler an Richard Beer-Hofmann, 30. 1. 1906]{ Arthur Schnitzler an Richard Beer-Hofmann, 30. 1. 1906}\nopagebreak\mylabel{v}\rehead{ }\normalsize\beginnumbering\briefempfaengerindex{Beer-Hofmann, Richard@\textsc{Beer-Hofmann, Richard}!zzzSchnitzler, Arthur@\emph{von Arthur Schnitzler}!1906-01-301@{30. 1. 1906}|(be} \toendnotes[C]{\smallbreak\pagebreak[2]} \Standort{YCGL, MSS 31.}
\physDesc{Brief, 1 Blatt, 3 Seiten
\newline{}Handschrift: Bleistift, deutsche Kurrent}\buchAbdrucke{\weitereDrucke{Arthur Schnitzler, Richard Beer-Hofmann: \emph{Briefwechsel 1891–1931}. Hg. Konstanze Fliedl. Wien, Zürich: \emph{Europaverlag} 1992, S. 176.} }\toendnotes[C]{\smallbreak}\pstart
           \noindent{}{\pb}\textcolor{gray}{\textbf{Dr. Arthur Schnitzler}}\hfill 30. 1. 906\pend
           \pstart
           \textcolor{gray}{\textbf{\textcolor{pink}{Wien, XVIII. Spoettelgasse 7}{}\ledrightnote{\textcolor{pink}{Edmund-Weiß-Gasse}}.}}\pend
           \pstart{}lieber Richard,\pend\pstart
           dieſer Tage hab ich die Bühnenexemplare des »\textcolor{green}{Ruf}{}\ledrightnote{\textcolor{green}{Der Ruf des Lebens. Schauspiel in drei Akten}}{[}«{]} beko{\geminationm}en, hier iſt
               eines, bitte ſagen Sie niemandem, dſs ich Ihnen eins geſchickt habe, es wollen {\pb}zu viele Leute eins haben.\pend
           \pstart
           Es wär denkbar, dſs ich Samſtag auf ein paar Tage (Arrangirproben, \textcolor{blue}{Brahm}{}\ledrightnote{\textcolor{blue}{Otto Brahm}}’s 50. Geburtstg) nach \textcolor{pink}{Berlin}{}\ledrightnote{\textcolor{pink}{Berlin}} fahre; da{\geminationn} ko{\geminationm} ich wieder zurück (hoffentlich), und am
                  17.{ }{\pb}will ich mit \textcolor{blue}{Olga}{}\ledrightnote{\textcolor{blue}{Olga Schnitzler}}
               hin zur \textcolor{green}{\textsc{Première}}{}\ledrightnote{→\textcolor{green}{Der Ruf des Lebens. Schauspiel in drei Akten}} am 24. –\pend
           \pstart
           Wie gehts Ihnen? Und \textcolor{blue}{Paula}{}\ledrightnote{\textcolor{blue}{Paula Beer-Hofmann}}? Und den \textcolor{blue}{Kindern}{}\ledrightnote{\textcolor{blue}{Naëmah Beer-Hofmann}{\newline}\textcolor{blue}{Gabriel Beer-Hofmann}{\newline}\textcolor{blue}{Mirjam Beer-Hofmann}}?\pend
           \pstart
           Herzlichſt, mit Grüßen von uns \textcolor{blue}{beiden}{}\ledrightnote{→\textcolor{blue}{Olga Schnitzler}}{\\[\baselineskip]}Ihr{\\[\baselineskip]}\spacefill\mbox{A.}\pend
           \leftskip=0em{}\endnumbering\briefempfaengerindex{Beer-Hofmann, Richard@\textsc{Beer-Hofmann, Richard}!zzzSchnitzler, Arthur@\emph{von Arthur Schnitzler}!1906-01-301@{30. 1. 1906}|)be}\mylabel{h}  \normalsize

\doendnotes{C}
\bigskip
\vfill

\clearpage

\footnotesize

\lohead{\textsc{register}}

% Definiere theindex-Environment komplett neu ohne reledmac
\makeatletter
\renewenvironment{theindex}{%
  \section*{\indexname}%
  \setlength{\parindent}{0pt}%
  \setlength{\parskip}{0pt plus 0.3pt}%
  \let\item\@idxitem
}{%
  \clearpage
}
\makeatother

\IfFileExists{\jobname-pw.ind}{\input{\jobname-pw.ind}}{}

\end{document}

      