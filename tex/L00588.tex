%% latex-korrekturansicht-vorspann.tex
%% Vorspann für die Korrekturansicht.
%% Lädt die gemeinsame Datei latex-vorspann.tex mit gesetztem Schalter.

\newif\ifkorrekturansicht
\korrekturansichttrue

\input{../tex-inputs/latex-vorspann}


               \section[Arthur Schnitzler an Richard Beer-Hofmann, {[}13. 9. 1896{]}]{ Arthur Schnitzler an Richard Beer-Hofmann, {[}13. 9. 1896{]}}\nopagebreak\mylabel{v}\rehead{ }\normalsize\beginnumbering\briefempfaengerindex{Beer-Hofmann, Richard@\textsc{Beer-Hofmann, Richard}!zzzSchnitzler, Arthur@\emph{von Arthur Schnitzler}!1896-09-131@{{[}13. 9. 1896{]}}|(be} \toendnotes[C]{\smallbreak\pagebreak[2]} \Standort{YCGL, MSS 31.}
\physDesc{Briefkarte, Umschlag
\newline{}Handschrift: Bleistift, deutsche Kurrent\newline{}Versand: ohne postalischen Übermittlungsvermerk }\buchAbdrucke{\weitereDrucke{Arthur Schnitzler, Richard Beer-Hofmann: \emph{Briefwechsel 1891–1931}. Hg. Konstanze Fliedl. Wien, Zürich: \emph{Europaverlag} 1992, S. 96.} }\toendnotes[C]{\smallbreak}\pstart{}{\pb}\textsc{An Dr. Rich. Beer Hofmann}\pend{}{\bigskip}\pstart
           \raggedleft{}{\pb}\label{T_L00588-1v}\edtext{12/IX 96}{\lemma{\textnormal{\emph{12/IX 96}}}\Cendnote{\textnormal{auf der Rückseite des Umschlags}}}\label{T_L00588-1h}\pend
           \pstart
           \raggedleft{}{\pb}\label{K_L00588_1v}\edtext{So{\geminationn}tag}{\lemma{\textnormal{\emph{Sotag}}}\Cendnote{\textnormal{Der 13. 9. 1896 war
                     ein Sonntag, \textcolor{blue}{Schnitzler} irrt sich mit der
                     Beschriftung »12/IX 96«.}}}\label{K_L00588_1h}. – ½ 6. \textsc{N. M.}\pend
           \pstart
           Lieber Richard, wie ka{\geminationn} man nicht
               einmal eine Poſt zu Haus laſſen wo man zu finden wäre! Ich ko{\geminationm}e per Rad von \textcolor{pink}{Mödling}{}\ledrightnote{\textcolor{pink}{Mödling}}
               – {\pb}\textcolor{blue}{Tini}{}\ledrightnote{\textcolor{blue}{Christine Schönberger}} – \textcolor{pink}{Alland}{}\ledrightnote{\textcolor{pink}{Alland}}
               – \textcolor{pink}{Neuhaus}{}\ledrightnote{\textcolor{pink}{Neuhaus}} – \textcolor{pink}{Pottenſtein}{}\ledrightnote{\textcolor{pink}{Pottenstein}} – \textcolor{pink}{Antonsgaſſe 4}{}\ledrightnote{\textcolor{pink}{Antonsgasse}} – \textcolor{pink}{Franzensgaſſe 54}{}\ledrightnote{\textcolor{pink}{Kaiser-Franz-Ring}} –\pend
           \pstart
           Der Doctor \textcolor{blue}{Schwarzkopf}{}\ledrightnote{\textcolor{blue}{Gustav Schwarzkopf}} iſt auch da, der grüßt
               Sie, aber nicht ſo herzlich wie ich.\pend
           \pstart Ihr \spacefill\mbox{ArthSch}\pend{}\endnumbering\briefempfaengerindex{Beer-Hofmann, Richard@\textsc{Beer-Hofmann, Richard}!zzzSchnitzler, Arthur@\emph{von Arthur Schnitzler}!1896-09-131@{{[}13. 9. 1896{]}}|)be}\mylabel{h}  \normalsize

\doendnotes{C}
\bigskip
\vfill

\clearpage

\footnotesize

\lohead{\textsc{register}}

% Definiere theindex-Environment komplett neu ohne reledmac
\makeatletter
\renewenvironment{theindex}{%
  \section*{\indexname}%
  \setlength{\parindent}{0pt}%
  \setlength{\parskip}{0pt plus 0.3pt}%
  \let\item\@idxitem
}{%
  \clearpage
}
\makeatother

\IfFileExists{\jobname-pw.ind}{\input{\jobname-pw.ind}}{}

\end{document}

      