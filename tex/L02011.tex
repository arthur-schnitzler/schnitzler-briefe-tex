%% latex-korrekturansicht-vorspann.tex
%% Vorspann für die Korrekturansicht.
%% Lädt die gemeinsame Datei latex-vorspann.tex mit gesetztem Schalter.

\newif\ifkorrekturansicht
\korrekturansichttrue

\input{../tex-inputs/latex-vorspann}


               \section[Hugo von Hofmannsthal an Arthur Schnitzler, {[}2. 3. 1911{]}]{ Hugo von Hofmannsthal an Arthur Schnitzler, {[}2. 3. 1911{]}}\nopagebreak\mylabel{v}\rehead{ }\normalsize\beginnumbering\briefempfaengerindex{Schnitzler, Arthur@\textsc{Schnitzler, Arthur}!zzzHofmannsthal, Hugo von@\emph{von Hugo von Hofmannsthal}!1911-03-021@{{[}2. 3. 1911{]}}|(be} \toendnotes[C]{\smallbreak\pagebreak[2]} \Standort{CUL, Schnitzler, B 43.}
\physDesc{Brief, 1 Blatt, 1 Seite
\newline{}Handschrift: schwarze Tinte, deutsche Kurrent
\newline{}Schnitzler: mit Bleistift datiert: »2/3 911« und beschriftet: »Hugo« \newline{}Ordnung: 1) mit Bleistift von unbekannter Hand nummeriert: »\strikeout{318}« 2) mit Bleistift von unbekannter Hand nummeriert:
                                    »329«}\buchAbdrucke{\weitereDrucke{Hugo von Hofmannsthal, Arthur Schnitzler: \emph{Briefwechsel}. Hg. Therese Nickl und Heinrich Schnitzler. Frankfurt am Main: \emph{S. Fischer} 1964, S. 260.} }\toendnotes[C]{\smallbreak}\pstart
           \raggedleft{}{\pb}Donnerstag abends\pend
           \pstart{}mein lieber Arthur, \pend\pstart
           ich höre eben von \textcolor{blue}{Richard}{}\ledrightnote{\textcolor{blue}{Richard Beer-Hofmann}} daſs Ihr ſchon \label{K_L02011_1v}\edtext{hier ſeid}{\lemma{\textnormal{\emph{hier ſeid}}}\Cendnote{\textnormal{\textcolor{blue}{Olga} und \textcolor{blue}{Arthur
                     Schnitzler} waren von 22. 2. 1911 bis zum 28. 2. 1911 in \textcolor{pink}{Berlin}.}}}\label{K_L02011_1h}.
               Man hat ſich, weiß Gott, lange genug nicht geſehen.\hspace*{1.5em}Würde Euch paſſen wenn wir \uline{Sonntag} zu mittag zu
               Euch kämen? Uns würde es gut paſſen. Bitte um sofortige Depeſche nach \textcolor{pink}{Rodaun}{}\ledrightnote{\textcolor{pink}{Rodaun}}.\pend
           \pstart Ihr \spacefill\mbox{Hugo}\pend{}\pstart
           \noindent{}Werde melden warum nichts über \label{K_L02011_2v}\edtext{\textcolor{blue}{Reinhardt}{}\ledrightnote{\textcolor{blue}{Max Reinhardt}}{ }\textcolor{green}{\textsc{Medardus}}{}\ledrightnote{\textcolor{green}{Der junge Medardus. Dramatische Historie in einem Vorspiel und fünf Aufzügen}} referierte}{\lemma{\textnormal{\emph{Reinhardt … referierte}}}\Cendnote{\textnormal{Unklar, \textcolor{blue}{Reinhardt} hatte das \textcolor{green}{Stück} nur unter für \textcolor{blue}{Schnitzler} nicht akzeptablen Bedingungen inszenieren
                     wollen.}}}\label{K_L02011_2h}.\pend
           \endnumbering\briefempfaengerindex{Schnitzler, Arthur@\textsc{Schnitzler, Arthur}!zzzHofmannsthal, Hugo von@\emph{von Hugo von Hofmannsthal}!1911-03-021@{{[}2. 3. 1911{]}}|)be}\mylabel{h}  \normalsize

\doendnotes{C}
\bigskip
\vfill

\clearpage

\footnotesize

\lohead{\textsc{register}}

% Definiere theindex-Environment komplett neu ohne reledmac
\makeatletter
\renewenvironment{theindex}{%
  \section*{\indexname}%
  \setlength{\parindent}{0pt}%
  \setlength{\parskip}{0pt plus 0.3pt}%
  \let\item\@idxitem
}{%
  \clearpage
}
\makeatother

\IfFileExists{\jobname-pw.ind}{\input{\jobname-pw.ind}}{}

\end{document}

      