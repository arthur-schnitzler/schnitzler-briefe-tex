%% latex-korrekturansicht-vorspann.tex
%% Vorspann für die Korrekturansicht.
%% Lädt die gemeinsame Datei latex-vorspann.tex mit gesetztem Schalter.

\newif\ifkorrekturansicht
\korrekturansichttrue

\input{../tex-inputs/latex-vorspann}


               \section[Hermann Bahr an Arthur Schnitzler, 9. 2. 1921]{ Hermann Bahr an Arthur Schnitzler, 9. 2. 1921}\nopagebreak\mylabel{v}\rehead{ }\normalsize\beginnumbering\briefempfaengerindex{Schnitzler, Arthur@\textsc{Schnitzler, Arthur}!zzzBahr, Hermann@\emph{von Hermann Bahr}!1921-02-091@{9. 2. 1921}|(be} \toendnotes[C]{\smallbreak\pagebreak[2]} \Standort{CUL, Schnitzler, B 5b.}
\physDesc{Postkarte
\newline{}Handschrift: schwarze Tinte, deutsche Kurrent\newline{}Versand: Stempel: »\nobreak{}\oindex{Salzburg@\textbf{Salzburg}, \emph{Besiedelter Ort (A.BSO)}|pwk}Salzburg, 10. II. {[}1921{]}\nobreak{}«.  
\newline{}Schnitzler: mit Bleistift Vermerk: »\textsc{A}«, vermutlich für »Abzuschreiben«/»Abschrift« \newline{}Ordnung: mit Bleistift von unbekannter Hand nummeriert: »184« }\buchAbdrucke{\weitereDrucke{Hermann Bahr, Arthur Schnitzler: \emph{Briefwechsel, Aufzeichnungen, Dokumente (1891–1931)}. Hg. Kurt Ifkovits und Martin Anton Müller. Göttingen: \emph{Wallstein} 2018, S. 540.} }\toendnotes[C]{\smallbreak}\pstart{}{\pb}Herrn D\textsuperscript{r} Arthur
                  Schnitzler\pend{}\pstart{}\textsc{\textcolor{pink}{Wien XVIII}{}\ledrightnote{\textcolor{pink}{XVIII., Währing}}}\pend{}\pstart{}\textcolor{pink}{Sternwarteſtraße 71}{}\ledrightnote{\textcolor{pink}{Wien}}\pend{}{\bigskip}\pstart
           \raggedleft{}{\pb}9. 2. 21\pend
           \pstart{}Lieber Arthur!\pend\pstart
           Herzlichſten Dank für Deinen lieben Brief! Aber als er kam, war mein für das \textcolor{green}{Journal}{}\ledrightnote{\textcolor{green}{Neues Wiener Journal}}{ }\label{K_L02361_1v}\edtext{vom 20.}{\lemma{\textnormal{\emph{vom 20.}}}\Cendnote{\textnormal{\textcolor{blue}{Hermann Bahr}: \emph{\textcolor{green}{Tagebuch. 30. Januar, 1. Februar und 3. Februar}}. In: \emph{\textcolor{green}{Neues Wiener Journal}}, Jg. 29, Nr. 9803,
                        20. 2. 1921, S. 6.}}}\label{K_L02361_1h} beſtimmtes \textcolor{green}{Tagebuch}{}\ledrightnote{\textcolor{green}{Tagebuch [Kolumne im Neuen Wiener Journal]}}{ }ſchon abgegangen. Wenns irgend geht, hoff ich aber
               dennoch des verehrten \textcolor{blue}{Mannes}{}\ledrightnote{→\textcolor{blue}{Josef Popper-Lynkeus}} u.
               ſeines Geburtstags zu gedenken, wenn auch \label{K_L02361_2v}\edtext{\textsc{post festum}}{\lemma{\textnormal{\emph{post festum}}}\Cendnote{\textnormal{Im \emph{\textcolor{green}{Tagebuch. 20. Februar}} (damit den falschen Tag aus \textcolor{blue}{Schnitzlers} Brief übernehmend), erschienen am 13. 3. 1921 (\emph{\textcolor{green}{Neues Wiener Journal}}, Jg. 29, Nr. 9824,
                     S. 7).}}}\label{K_L02361_2h}. – Ich leſe jetzt Deinen Namen ſo oft – erinnerſt Du Dich
               denn, daß ich der erſte war, der »\textcolor{green}{Reigen}{}\ledrightnote{\textcolor{green}{Reigen. Zehn Dialoge}}«
               öffentlich vorleſen wollte, ja ſogar bis zu \label{K_L02361_3v}\edtext{\textcolor{blue}{Körber}{}\ledrightnote{\textcolor{blue}{Ernest von Koerber}}{ }ſelber ging}{\lemma{\textnormal{\emph{Körber ſelber ging}}}\Cendnote{\textnormal{Vgl. \emph{Briefwechsel} Bahr/Schnitzler 276.}}}\label{K_L02361_3h}, um es
               durchzuſetzen, leider vergebens? – Ich wäre ſehr froh, Dich bald einmal endlich
               wiederzuſehen!\pend
           \pstart
           Dich u. die \textcolor{blue}{Deinen}{}\ledrightnote{→\textcolor{blue}{Olga Schnitzler}{\newline}→\textcolor{blue}{Heinrich Schnitzler}{\newline}→\textcolor{blue}{Lili Schnitzler}} herzlichſt grüßend{\\[\baselineskip]}Dein alter\spacefill\mbox{Hermann}\pend
           \leftskip=0em{}\endnumbering\briefempfaengerindex{Schnitzler, Arthur@\textsc{Schnitzler, Arthur}!zzzBahr, Hermann@\emph{von Hermann Bahr}!1921-02-091@{9. 2. 1921}|)be}\mylabel{h}  \normalsize

\doendnotes{C}
\bigskip
\vfill

\clearpage

\footnotesize

\lohead{\textsc{register}}

% Definiere theindex-Environment komplett neu ohne reledmac
\makeatletter
\renewenvironment{theindex}{%
  \section*{\indexname}%
  \setlength{\parindent}{0pt}%
  \setlength{\parskip}{0pt plus 0.3pt}%
  \let\item\@idxitem
}{%
  \clearpage
}
\makeatother

\IfFileExists{\jobname-pw.ind}{\input{\jobname-pw.ind}}{}

\end{document}

      