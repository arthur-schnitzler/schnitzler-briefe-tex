%% latex-korrekturansicht-vorspann.tex
%% Vorspann für die Korrekturansicht.
%% Lädt die gemeinsame Datei latex-vorspann.tex mit gesetztem Schalter.

\newif\ifkorrekturansicht
\korrekturansichttrue

\input{../tex-inputs/latex-vorspann}


               \section[Hugo von Hofmannsthal an Arthur Schnitzler, 9. 7. 1907]{ Hugo von Hofmannsthal an Arthur Schnitzler,
               9. 7. 1907}\nopagebreak\mylabel{v}\rehead{ }\normalsize\beginnumbering\briefempfaengerindex{Schnitzler, Arthur@\textsc{Schnitzler, Arthur}!zzzHofmannsthal, Hugo von@\emph{von Hugo von Hofmannsthal}!1907-07-091@{9. 7. 1907}|(be} \toendnotes[C]{\smallbreak\pagebreak[2]} \Standort{CUL, Schnitzler, B 43.}
\physDesc{Postkarte
\newline{}Handschrift: schwarze Tinte, deutsche Kurrent\newline{}Versand: 1) Stempel: »\nobreak{}\oindex{Cortina d'Ampezzo@\textbf{Cortina d'Ampezzo}, \emph{Besiedelter Ort (A.BSO)}|pwk}Cortina, 9. VII. 07\nobreak{}«.  2) Stempel: »\nobreak{}\oindex{Welsberg-Taisten@\textbf{Welsberg-Taisten}, \emph{Besiedelter Ort (A.BSO)}|pwk}Welsbe\textcolor{gray}{r}g, \textcolor{gray}{9. 7. 07}\nobreak{}«. 
\newline{}Schnitzler: mit Bleistift datiert: »9/7 907« \newline{}Ordnung: 1) mit Bleistift von unbekannter Hand nummeriert: »\strikeout{280}« 2) mit Bleistift von unbekannter Hand nummeriert: »282«}\buchAbdrucke{\weitereDrucke{Hugo von Hofmannsthal, Arthur Schnitzler: \emph{Briefwechsel}. Hg. Therese Nickl und Heinrich Schnitzler. Frankfurt am Main: \emph{S. Fischer} 1964, S. 230.} }\toendnotes[C]{\smallbreak}\pstart{}{\pb}\textsc{Herrn D\textsuperscript{r} Arthur Schnitzler}\pend{}\pstart{}\textsc{\textcolor{pink}{Pension Waldbrunn}{}\ledrightnote{\textcolor{pink}{Wildbad Waldbrunn}}}\pend{}\pstart{}\textsc{\textcolor{pink}{Welsberg}{}\ledrightnote{\textcolor{pink}{Welsberg-Taisten}}}\pend{}\pstart{}\textsc{bei \textcolor{pink}{Toblach.}{}\ledrightnote{\textcolor{pink}{Toblach}}}\pend{}{\bigskip}\pstart
           \noindent{}{\pb}Sind Sie wirklich dort? Das wäre
               nett.\pend
           \pstart
           Würde es Ihnen dann paſſen, daſs wir Anfang oder Mitte nächſter Woche auf 2 Tage hin
               kämen? (natürlich ohne Störung Ihrer Arbeitsſtunden, ich \textcolor{green}{arbeite auch}{}\ledrightnote{→\textcolor{green}{Silvia im »Stern«}}.) Würden wir
               dort für 2 Nächte Unterkunft finden?\hspace*{1.5em}Bitte um
               Antwort.\pend
           \pstart \spacefill\mbox{Hugo}\pend{}\pstart
           \noindent{}\textcolor{pink}{\textsc{Cortina Hôtel Bellevue}}{}\ledrightnote{\textcolor{pink}{Hotel Bellevue}}\pend
           \endnumbering\briefempfaengerindex{Schnitzler, Arthur@\textsc{Schnitzler, Arthur}!zzzHofmannsthal, Hugo von@\emph{von Hugo von Hofmannsthal}!1907-07-091@{9. 7. 1907}|)be}\mylabel{h}  \normalsize

\doendnotes{C}
\bigskip
\vfill

\clearpage

\footnotesize

\lohead{\textsc{register}}

% Definiere theindex-Environment komplett neu ohne reledmac
\makeatletter
\renewenvironment{theindex}{%
  \section*{\indexname}%
  \setlength{\parindent}{0pt}%
  \setlength{\parskip}{0pt plus 0.3pt}%
  \let\item\@idxitem
}{%
  \clearpage
}
\makeatother

\IfFileExists{\jobname-pw.ind}{\input{\jobname-pw.ind}}{}

\end{document}

      