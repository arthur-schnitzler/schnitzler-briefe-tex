%% latex-korrekturansicht-vorspann.tex
%% Vorspann für die Korrekturansicht.
%% Lädt die gemeinsame Datei latex-vorspann.tex mit gesetztem Schalter.

\newif\ifkorrekturansicht
\korrekturansichttrue

\input{../tex-inputs/latex-vorspann}


               \section[Arthur Schnitzler an Albert Ehrenstein, 26. 1. 1909]{ Arthur Schnitzler an Albert Ehrenstein,
                    26. 1. 1909}\nopagebreak\mylabel{v}\rehead{ }\normalsize\beginnumbering\briefempfaengerindex{Ehrenstein, Albert@\textsc{Ehrenstein, Albert}!zzzSchnitzler, Arthur@\emph{von Arthur Schnitzler}!1909-01-262@{26. 1. 1909}|(be} \toendnotes[C]{\smallbreak\pagebreak[2]} \Standort{Jerusalem, The National Library of Israel, ARC. Ms. Var. 306 1 118.}
\physDesc{Bildpostkarte
\newline{}Handschrift: Bleistift, deutsche Kurrent\newline{}Versand: Stempel: »\nobreak{}\oindex{Semmering@\textbf{Semmering}, \emph{Besiedelter Ort (A.BSO)}|pwk}Semmering, 2{[}6. 1. 190{]}9\nobreak{}«.  \newline{}Ordnung: mit Bleistift von unbekannter Hand
                                    nummeriert: »9« }\pstart{}{\pb}Hrn \textsc{Albert Ehrenstein}\pend{}\pstart{}\textsc{\textcolor{pink}{Wien XIV}{}\ledrightnote{\textcolor{pink}{XIV., Penzing}}}\pend{}\pstart{}\textsc{\textcolor{pink}{Ottakringerstr. 114}{}\ledrightnote{\textcolor{pink}{Ottakringerstraße}}}\pend{}{\bigskip}\pstart
           \noindent{}\centering{}\textcolor{gray}{\textbf{{\pb}\textcolor{pink}{Semmering.\hspace*{1.5em}Südbahnhotel.}{}\ledrightnote{\textcolor{pink}{Südbahnhotel}}}}\pend
           \pstart
           {\pb}Zu mündlicher Erkärung in \textcolor{pink}{Wien}{}\ledrightnote{\textcolor{pink}{Wien}}, etwa in 10–14 Tagen, gern bereit.\pend
           \pstart
           Beſtens grüßend{\\[\baselineskip]}bis dahin Ihr{\\[\baselineskip]}\spacefill\mbox{Arth. Sch.}\pend
           \leftskip=0em{}\pstart
           26. 1. 09.\pend
           \endnumbering\briefempfaengerindex{Ehrenstein, Albert@\textsc{Ehrenstein, Albert}!zzzSchnitzler, Arthur@\emph{von Arthur Schnitzler}!1909-01-262@{26. 1. 1909}|)be}\mylabel{h}  \normalsize

\doendnotes{C}
\bigskip
\vfill

\clearpage

\footnotesize

\lohead{\textsc{register}}

% Definiere theindex-Environment komplett neu ohne reledmac
\makeatletter
\renewenvironment{theindex}{%
  \section*{\indexname}%
  \setlength{\parindent}{0pt}%
  \setlength{\parskip}{0pt plus 0.3pt}%
  \let\item\@idxitem
}{%
  \clearpage
}
\makeatother

\IfFileExists{\jobname-pw.ind}{\input{\jobname-pw.ind}}{}

\end{document}

      