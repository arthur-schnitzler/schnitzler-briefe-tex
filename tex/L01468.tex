%% latex-korrekturansicht-vorspann.tex
%% Vorspann für die Korrekturansicht.
%% Lädt die gemeinsame Datei latex-vorspann.tex mit gesetztem Schalter.

\newif\ifkorrekturansicht
\korrekturansichttrue

\input{../tex-inputs/latex-vorspann}


               \section[Richard Beer-Hofmann an Arthur Schnitzler, 9. 11. 1904]{ Richard Beer-Hofmann an Arthur Schnitzler,
               9. 11. 1904}\nopagebreak\mylabel{v}\rehead{ }\normalsize\beginnumbering\briefempfaengerindex{Schnitzler, Arthur@\textsc{Schnitzler, Arthur}!zzzBeer-Hofmann, Richard@\emph{von Richard Beer-Hofmann}!1904-11-091@{9. 11. 1904}|(be} \toendnotes[C]{\smallbreak\pagebreak[2]} \Standort{CUL, Schnitzler, B 8.}
\physDesc{Brief, 1 Blatt, 1 Seite
\newline{}Handschrift: schwarze Tinte, lateinische Kurrent\newline{}Ordnung: mit Bleistift von unbekannter Hand nummeriert: »195« }\buchAbdrucke{\weitereDrucke{Arthur Schnitzler, Richard Beer-Hofmann: \emph{Briefwechsel 1891–1931}. Hg. Konstanze Fliedl. Wien, Zürich: \emph{Europaverlag} 1992, S. 169.} }\toendnotes[C]{\smallbreak}\pstart
           \centering{}\uline{Noch} – \textcolor{pink}{Rodaun}{}\ledrightnote{\textcolor{pink}{Rodaun}}{ }9./XI. 04\pend
           \pstart
           Lieber Arthur! Ich bin selbstverständlich ohne jede Nachricht von
                  \textcolor{pink}{Berlin}{}\ledrightnote{\textcolor{pink}{Berlin}}. Werde morgen telegraphiren. Wenn
               erfolglos, werde ich Alles auf Ihre Schultern laden. Jedenfalls:\pend
           \pstart
           1) Wann fahren Sie – \label{K_L01468_1v}\edtext{Samstag}{\lemma{\textnormal{\emph{Samstag}}}\Cendnote{\textnormal{vgl. A. S.: \emph{Tagebuch}, 12. 11. 1904}}}\label{K_L01468_1h}? \introOben{}(Stunde Bahnhof)\introOben{}\pend
           \pstart
           2.) Wo wohnen Sie in \textcolor{pink}{Berlin}{}\ledrightnote{\textcolor{pink}{Berlin}}?\pend
           \pstart
           Mein \textcolor{blue}{Hausherr}{}\ledrightnote{→\textcolor{blue}{Rudolf Berger}}? »Arisch«
               »Bodenständig« »Deutsche Biederkeit« »Ehrliches Bürgerthum« »Gerader deutscher Sinn«
               »Abhold jeder Tücke« »Germanische Treue«. Sie – die \textcolor{blue}{Selcherin}{}\ledrightnote{→\textcolor{blue}{Berger}} – hat einen Hausaltar – und die \textcolor{blue}{Kinder}{}\ledrightnote{→\textcolor{blue}{Berger}} ko{\geminationm}en nach
                  \textcolor{brown}{Kalksburg}{}\ledrightnote{\textcolor{brown}{Kollegium Kalksburg}}.\pend
           \pstart
           Herzlichst Ihr{\\[\baselineskip]}\spacefill\mbox{Richard}\pend
           \leftskip=0em{}\endnumbering\briefempfaengerindex{Schnitzler, Arthur@\textsc{Schnitzler, Arthur}!zzzBeer-Hofmann, Richard@\emph{von Richard Beer-Hofmann}!1904-11-091@{9. 11. 1904}|)be}\mylabel{h}  \normalsize

\doendnotes{C}
\bigskip
\vfill

\clearpage

\footnotesize

\lohead{\textsc{register}}

% Definiere theindex-Environment komplett neu ohne reledmac
\makeatletter
\renewenvironment{theindex}{%
  \section*{\indexname}%
  \setlength{\parindent}{0pt}%
  \setlength{\parskip}{0pt plus 0.3pt}%
  \let\item\@idxitem
}{%
  \clearpage
}
\makeatother

\IfFileExists{\jobname-pw.ind}{\input{\jobname-pw.ind}}{}

\end{document}

      