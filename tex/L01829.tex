%% latex-korrekturansicht-vorspann.tex
%% Vorspann für die Korrekturansicht.
%% Lädt die gemeinsame Datei latex-vorspann.tex mit gesetztem Schalter.

\newif\ifkorrekturansicht
\korrekturansichttrue

\input{../tex-inputs/latex-vorspann}


               \section[Arthur Schnitzler an Adalbert Seligmann, 12. 2. 1909]{ Arthur Schnitzler an Adalbert Seligmann,
                    12. 2. 1909}\nopagebreak\mylabel{v}\rehead{ }\normalsize\beginnumbering\briefempfaengerindex{Seligmann, Adalbert Franz@\textsc{Seligmann, Adalbert Franz}!zzzSchnitzler, Arthur@\emph{von Arthur Schnitzler}!1909-02-121@{12. 2. 1909}|(be} \toendnotes[C]{\smallbreak\pagebreak[2]} \Standort{Wienbibliothek im Rathaus, H.I.N.-95600.}
\physDesc{Briefkarte, Umschlag
\newline{}Handschrift: schwarze Tinte, lateinische Kurrent\newline{}Versand: 1) Stempel: »\nobreak{}{[}Wien{]} 110, 1{[}2. 02. 190{]}9, 4\nobreak{}«.  2) Stempel: »\nobreak{}\oindex{IX., Alsergrund@\textbf{IX., Alsergrund}, \emph{Bezirk (A.BZK)}|pwk}9/3 Wien, 12. II. 09, 6, Bestellt\nobreak{}«. }\pstart{}{\pb}\textcolor{gray}{\textbf{Dr. Arthur Schnitzler}}\pend{}\pstart{}\textcolor{gray}{\textbf{\textcolor{pink}{Wien XVIII.
                                Spoettelgasse 7}{}\ledrightnote{\textcolor{pink}{Edmund-Weiß-Gasse}}.}}\pend{}{\bigskip}\pstart{}{\pb}Hrn A. F.
                        Seligmann,\pend{}\pstart{}\textcolor{pink}{Wien IX}{}\ledrightnote{\textcolor{pink}{IX., Alsergrund}}\pend{}\pstart{}\textcolor{pink}{Schwarzspanierstr 15}{}\ledrightnote{\textcolor{pink}{Schwarzspanierstraße}}.\pend{}{\bigskip}\pstart
           \noindent{}{\pb}\textcolor{gray}{\textbf{Dr. Arthur Schnitzler}}\hfill 12. 2. 09\pend
           \pstart
           \textcolor{gray}{\textbf{\textcolor{pink}{Wien XVIII.
                                Spoettelgasse 7}{}\ledrightnote{\textcolor{pink}{Edmund-Weiß-Gasse}}.}}\pend
           \pstart
           verehrter Freund, Ihren Brief habe ich zu gef. directen
                    Beantwortung an Hrn \textcolor{blue}{Paul Brann}{}\ledrightnote{\textcolor{blue}{Paul Brann}}, \textcolor{pink}{München-Schwabing, Gedonstraße 6}{}\ledrightnote{\textcolor{pink}{Gedonstraße}},
                    gesandt\pend
           \pstart
           Herzlichst grüßt Ihr ergebner{\\[\baselineskip]}\spacefill\mbox{A. S.}\pend
           \leftskip=0em{}\endnumbering\briefempfaengerindex{Seligmann, Adalbert Franz@\textsc{Seligmann, Adalbert Franz}!zzzSchnitzler, Arthur@\emph{von Arthur Schnitzler}!1909-02-121@{12. 2. 1909}|)be}\mylabel{h}  \normalsize

\doendnotes{C}
\bigskip
\vfill

\clearpage

\footnotesize

\lohead{\textsc{register}}

% Definiere theindex-Environment komplett neu ohne reledmac
\makeatletter
\renewenvironment{theindex}{%
  \section*{\indexname}%
  \setlength{\parindent}{0pt}%
  \setlength{\parskip}{0pt plus 0.3pt}%
  \let\item\@idxitem
}{%
  \clearpage
}
\makeatother

\IfFileExists{\jobname-pw.ind}{\input{\jobname-pw.ind}}{}

\end{document}

      