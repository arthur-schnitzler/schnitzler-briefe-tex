%% latex-korrekturansicht-vorspann.tex
%% Vorspann für die Korrekturansicht.
%% Lädt die gemeinsame Datei latex-vorspann.tex mit gesetztem Schalter.

\newif\ifkorrekturansicht
\korrekturansichttrue

\input{../tex-inputs/latex-vorspann}


               \section[Hermann Bahr an Arthur Schnitzler, 2. 12. 1909]{ Hermann Bahr an Arthur Schnitzler, 2. 12. 1909}\nopagebreak\mylabel{v}\rehead{ }\normalsize\beginnumbering\briefempfaengerindex{Schnitzler, Arthur@\textsc{Schnitzler, Arthur}!zzzBahr, Hermann@\emph{von Hermann Bahr}!1909-12-021@{2. 12. 1909}|(be} \toendnotes[C]{\smallbreak\pagebreak[2]} \Standort{CUL, Schnitzler, B 5b.}
\physDesc{Bildpostkarte
\newline{}Handschrift: schwarze Tinte, deutsche Kurrent\newline{}Versand: Stempel: »\nobreak{}\oindex{Dirschau@\textbf{Dirschau}, \emph{Besiedelter Ort (A.BSO)}|pwk}\oindex{Eydtkuhnen@\textbf{Eydtkuhnen}, \emph{Besiedelter Ort (A.BSO)}|pwk}Dirschau Eydtkuhnen Bahnpost, 2. 12 09\nobreak{}«.  
\newline{}Schnitzler: mit Bleistift ergänzt »Bahr« \newline{}Ordnung: mit Bleistift von unbekannter Hand nummeriert: »162« }\buchAbdrucke{\weitereDrucke{Hermann Bahr, Arthur Schnitzler: \emph{Briefwechsel, Aufzeichnungen, Dokumente (1891–1931)}. Hg. Kurt Ifkovits und Martin Anton Müller. Göttingen: \emph{Wallstein} 2018, S. 426.} }\toendnotes[C]{\smallbreak}\pstart{}{\pb}\textsc{D\textsuperscript{r} Arthur Schnitzler}\pend{}\pstart{}\textsc{\textcolor{pink}{Spöttelgasse 7}{}\ledrightnote{\textcolor{pink}{Edmund-Weiß-Gasse}}}\pend{}\pstart{}\textsc{\textcolor{pink}{Wien XVIII}{}\ledrightnote{\textcolor{pink}{XVIII., Währing}}}\pend{}{\bigskip}\pstart
           \noindent{}\centering{}\textcolor{gray}{\textbf{{\pb}\textcolor{pink}{Königsberg}{}\ledrightnote{\textcolor{pink}{Kaliningrad}}}}\pend
           \pstart
           \noindent{}\centering{}\textcolor{gray}{\textbf{
                  Partie am Pregel und Blick
                  auf den \textcolor{pink}{Dom}{}\ledrightnote{\textcolor{pink}{Dom}}}}\pend
           \pstart
           \noindent{}{\pb}2. Dez. 09\pend
           \pstart{}Lieber Arthur!\pend\pstart
           Heute hier im \textcolor{brown}{Goethebund}{}\ledrightnote{\textcolor{brown}{Goethebund}}:{\\}Schnitzlerabend von
               HermannBahr.\pend
           \pstart
           \label{LL323-1v}So bin ich unermüdlich um Deinen Ruhm in Nord
                  u. Süd beſorgt.\label{LL323-1h}\pend
           \pstart
           Herzlichſt, mit ſchönen Grüßen an \textcolor{blue}{Frau}{}\ledrightnote{→\textcolor{blue}{Olga Schnitzler}}, \textcolor{blue}{Sohn}{}\ledrightnote{→\textcolor{blue}{Heinrich Schnitzler}} und \textcolor{blue}{Tochter}{}\ledrightnote{→\textcolor{blue}{Lili Schnitzler}},{\\[\baselineskip]}Dein{\\[\baselineskip]}\spacefill\mbox{H\textcolor{gray}{m}B.}\pend
           \leftskip=0em{}\endnumbering\briefempfaengerindex{Schnitzler, Arthur@\textsc{Schnitzler, Arthur}!zzzBahr, Hermann@\emph{von Hermann Bahr}!1909-12-021@{2. 12. 1909}|)be}\mylabel{h}  \normalsize

\doendnotes{C}
\bigskip
\vfill

\clearpage

\footnotesize

\lohead{\textsc{register}}

% Definiere theindex-Environment komplett neu ohne reledmac
\makeatletter
\renewenvironment{theindex}{%
  \section*{\indexname}%
  \setlength{\parindent}{0pt}%
  \setlength{\parskip}{0pt plus 0.3pt}%
  \let\item\@idxitem
}{%
  \clearpage
}
\makeatother

\IfFileExists{\jobname-pw.ind}{\input{\jobname-pw.ind}}{}

\end{document}

      