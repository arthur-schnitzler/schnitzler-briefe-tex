%% latex-korrekturansicht-vorspann.tex
%% Vorspann für die Korrekturansicht.
%% Lädt die gemeinsame Datei latex-vorspann.tex mit gesetztem Schalter.

\newif\ifkorrekturansicht
\korrekturansichttrue

\input{../tex-inputs/latex-vorspann}


               \section[Mirjam Beer-Hofmann an Olga Schnitzler, 12. 6. 1906]{ Mirjam Beer-Hofmann an Olga Schnitzler,
                    12. 6. 1906}\nopagebreak\mylabel{v}\rehead{ }\normalsize\beginnumbering\briefempfaengerindex{Schnitzler, Olga@\textsc{Schnitzler, Olga}!zzzBeer-Hofmann, Mirjam@\emph{von Mirjam Beer-Hofmann}!1906-06-121@{12. 6. 1906}|(be} \toendnotes[C]{\smallbreak\pagebreak[2]} \Standort{DLA, A:Schnitzler, HS.NZ.85.1.5204.}
\physDesc{Bildpostkarte
\newline{}Handschrift: schwarze Tinte, lateinische Kurrent\newline{}Versand: Stempel: »\nobreak{}\oindex{Dahmsdorf@\textbf{Dahmsdorf}, \emph{https://www.geonames.org/ontologyP.PPL}|pwk}Dahmsdorf Müncheberg, 12 6 06, 8–12N\nobreak{}«.  }\toendnotes[C]{\smallbreak}\pstart{}{\pb}Frau\pend{}\pstart{}Olga Schnitzler\pend{}\pstart{}\textcolor{pink}{Wien XVIII}{}\ledrightnote{\textcolor{pink}{XVIII., Währing}}.\pend{}\pstart{}\textcolor{pink}{Spöttelgasse 7}{}\ledrightnote{\textcolor{pink}{Edmund-Weiß-Gasse}}\pend{}{\bigskip}\pstart
           \noindent{}\centering{}{\pb}\textcolor{gray}{\textbf{\textcolor{pink}{Buckow Märk.-Schweiz}{}\ledrightnote{\textcolor{pink}{Buckow}} – \textcolor{pink}{Neue Promenade}{}\ledrightnote{\textcolor{pink}{Neue Promenade}}}}\pend
           \pstart
           \label{T_L01600_1v}\edtext{Das ist}{\lemma{\textnormal{\emph{Das ist}}}\Cendnote{\textnormal{durch einen
                        Strich auf der Abbildung kenntlich gemacht}}}\label{T_L01600_1h} unsere Wohnung. Im 1\textsuperscript{ten} Stock.\pend
           \pstart
           {\pb}Schreib, Elende!\pend
           \pstart
           Elende ſchreib!\pend
           \pstart
           Deine {\\[\baselineskip]}\spacefill\mbox{Mirjam.}\pend
           \leftskip=0em{}\endnumbering\briefempfaengerindex{Schnitzler, Olga@\textsc{Schnitzler, Olga}!zzzBeer-Hofmann, Mirjam@\emph{von Mirjam Beer-Hofmann}!1906-06-121@{12. 6. 1906}|)be}\mylabel{h}  \normalsize

\doendnotes{C}
\bigskip
\vfill

\clearpage

\footnotesize

\lohead{\textsc{register}}

% Definiere theindex-Environment komplett neu ohne reledmac
\makeatletter
\renewenvironment{theindex}{%
  \section*{\indexname}%
  \setlength{\parindent}{0pt}%
  \setlength{\parskip}{0pt plus 0.3pt}%
  \let\item\@idxitem
}{%
  \clearpage
}
\makeatother

\IfFileExists{\jobname-pw.ind}{\input{\jobname-pw.ind}}{}

\end{document}

      