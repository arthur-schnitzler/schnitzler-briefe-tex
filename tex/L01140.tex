%% latex-korrekturansicht-vorspann.tex
%% Vorspann für die Korrekturansicht.
%% Lädt die gemeinsame Datei latex-vorspann.tex mit gesetztem Schalter.

\newif\ifkorrekturansicht
\korrekturansichttrue

\input{../tex-inputs/latex-vorspann}


               \section[Arthur Schnitzler an Richard Beer-Hofmann, 4. 7. 1901]{ Arthur Schnitzler an Richard Beer-Hofmann, 4. 7. 1901}\nopagebreak\mylabel{v}\rehead{ }\normalsize\beginnumbering\briefempfaengerindex{Beer-Hofmann, Richard@\textsc{Beer-Hofmann, Richard}!zzzSchnitzler, Arthur@\emph{von Arthur Schnitzler}!1901-07-041@{4. 7. 1901}|(be} \toendnotes[C]{\smallbreak\pagebreak[2]} \Standort{YCGL, MSS 31.}
\physDesc{Brief, 2 Blätter, 7 Seiten, Umschlag
\newline{}Handschrift: 1) Bleistift, deutsche Kurrent\hspace{1em}2) schwarze Tinte, deutsche Kurrent (\noindent{}Umschlag)\hspace{1em}\newline{}Versand: 1) Stempel: »\nobreak{}\oindex{St. Anton am Arlberg@\textbf{St. Anton am Arlberg}, \emph{Besiedelter Ort (A.BSO)}|pwk}St. Anton am Arlberg, 4 7 01\nobreak{}«.  2) Stempel: »\nobreak{}\oindex{Poertschach@\textbf{Pörtschach}, \emph{https://www.geonames.org/ontologyP.PPL}|pwk}{\pb}Pörtschach am See, 5 7 01\nobreak{}«. }\buchAbdrucke{\weitereDrucke{Arthur Schnitzler, Richard Beer-Hofmann: \emph{Briefwechsel 1891–1931}. Hg. Konstanze Fliedl. Wien, Zürich: \emph{Europaverlag} 1992, S. 152–153.} }\toendnotes[C]{\smallbreak}\pstart{}{\pb}Herrn \textsc{Dr. Richard
                     Beer-Hofmann}\pend{}\pstart{}\textsc{\textcolor{pink}{Pörtschach}{}\ledrightnote{\textcolor{pink}{Pörtschach}}}\pend{}\pstart{}\textsc{am \textcolor{pink}{Wörthersee}{}\ledrightnote{\textcolor{pink}{Wörthersee}}}\pend{}\pstart{}\textsc{\textcolor{pink}{Villa Arnstein}{}\ledrightnote{\textcolor{pink}{Villa Arnstein}}}\pend{}{\bigskip}\pstart
           \raggedleft{}{\pb}\textcolor{pink}{\textsc{St. Anton a (Arlberg)}}{}\ledrightnote{\textcolor{pink}{St. Anton am Arlberg}}{\\}4. 7. 901.\pend
           \pstart
           mein lieber Richard, ich war zuerſt 14 Tage in \textcolor{pink}{Salzburg}{}\ledrightnote{\textcolor{pink}{Salzburg}}, \textcolor{pink}{oeſterr Hof}{}\ledrightnote{\textcolor{pink}{Österreichischer Hof}}, mit \textcolor{blue}{ihr}{}\ledrightnote{→\textcolor{blue}{Olga Schnitzler}}, es war ſehr ſchön. Dann
               2 Tage \textcolor{pink}{Innsbruck}{}\ledrightnote{\textcolor{pink}{Innsbruck}} (daſs ich \textcolor{pink}{Schönberg}{}\ledrightnote{\textcolor{pink}{Schönberg im Stubaital}} aufgeſucht habe, wiſſen Sie), da{\geminationn} fuhren wir nach \textcolor{pink}{\textsc{Landeck}}{}\ledrightnote{\textcolor{pink}{Landeck}}, wo ihre \textcolor{blue}{Schweſter}{}\ledrightnote{→\textcolor{blue}{Elisabeth Steinrück}} kam,
               und nun ſind wir in \textcolor{pink}{\textsc{St. Anton}}{}\ledrightnote{\textcolor{pink}{St. Anton am Arlberg}} – ich habe ein \introOben{}ſehr behagliches\introOben{}
               Zimmer zu 60 Kreuzer in einem Privat{\pb}haus, und es wäre
               ſehr nett, we{\geminationn} nicht das Wetter elend wäre. Wie lang ich
               hier bleibe, ka{\geminationn} ich natürlich \introOben{}nicht\introOben{} ſagen (daher bitte ich um Nachricht nach \textcolor{pink}{\uline{Wien}}{}\ledrightnote{\textcolor{pink}{Wien}}) wahrſcheinlich fahre ich von hier aus in die \textcolor{pink}{Schweiz}{}\ledrightnote{\textcolor{pink}{Schweiz}}. Anfang August ſoll ich dort \textcolor{blue}{Mama}{}\ledrightnote{→\textcolor{blue}{Louise Schnitzler}} treffen (\textcolor{pink}{\textsc{Flims}}{}\ledrightnote{\textcolor{pink}{Flims}} von \textcolor{pink}{\textsc{Reichenau}}{}\ledrightnote{\textcolor{pink}{Reichenau}} – (\textcolor{pink}{\textsc{Chur}}{}\ledrightnote{\textcolor{pink}{Chur}} – \textcolor{pink}{\textsc{Tham}}{}\ledrightnote{\textcolor{pink}{Tamins}}) aus 3 Stunden) auf etwa {\pb}8 Tage. Der \textcolor{pink}{\textsc{Wörther}ſee}{}\ledrightnote{\textcolor{pink}{Wörthersee}} fiel ins Waſſer, weil Scharlach Gerüchte
               umgingen, und überdies wollte \textcolor{blue}{Mama}{}\ledrightnote{→\textcolor{blue}{Louise Schnitzler}} nicht zu \textcolor{pink}{\textsc{Pundschu}}{}\ledrightnote{\textcolor{pink}{Pension Pundschu}}, weil ich nicht
               wußte, auf wie lang ich hingehn würde. Nun bin ich ſo weit von dort, dſs ich Sie
               heuer im Sommer kaum ſehn werde, we{\geminationn} Sie nicht mir, \textsc{resp}. mir und {\pb}\textcolor{blue}{Paul Goldmann}{}\ledrightnote{\textcolor{blue}{Paul Goldmann}} (von dem ich übrigens noch keine
                  beſti{\geminationm}te Nachricht habe) irgendwie entgegenko{\geminationm}en.\pend
           \pstart
           Haben Sie ſchon irgendwelche Auguſtpläne? Sie ſchreiben mir wenig, faſt gar nichts
               über ſich; was thun Sie? Arbeiten Sie? Wie gehts Ihrer \textcolor{blue}{Frau}{}\ledrightnote{→\textcolor{blue}{Paula Beer-Hofmann}} und den \textcolor{blue}{Kindern}{}\ledrightnote{→\textcolor{blue}{Naëmah Beer-Hofmann}{\newline}→\textcolor{blue}{Mirjam Beer-Hofmann}}?\pend
           \pstart
           \textcolor{blue}{Salten}{}\ledrightnote{\textcolor{blue}{Felix Salten}} iſt auf Reiſen, {\pb}wie mir eine Karte von ihm flüchtig mittheilt, aus \textcolor{pink}{Brettlgründen}{}\ledrightnote{→\textcolor{pink}{Jung-Wiener Theater zum Lieben Augustin}}. Ich ſchreibe ein
               3aktiges \textcolor{green}{Stück}{}\ledrightnote{→\textcolor{green}{Der einsame Weg. Schauspiel in fünf Akten}{\newline}→\textcolor{green}{Professor Bernhardi. Komödie in fünf Akten}} und glaube
               im Sommer damit und auch mit \textcolor{green}{2 Einaktern}{}\ledrightnote{→\textcolor{green}{Lebendige Stunden}{\newline}→\textcolor{green}{Die Frau mit dem Dolche}} fertig zu werden. – An \textcolor{blue}{Hugo}{}\ledrightnote{\textcolor{blue}{Hugo von Hofmannsthal}} und \textcolor{blue}{Gerty}{}\ledrightnote{\textcolor{blue}{Gertrude von Hofmannsthal}}{ }ſauſte ich (\textsc{resp}. \textcolor{blue}{wir}{}\ledrightnote{→\textcolor{blue}{Olga Schnitzler}}) in
                  \textcolor{pink}{Innsbruck}{}\ledrightnote{\textcolor{pink}{Innsbruck}} in einem Einſpänner vorüber. –
               \textcolor{pink}{Innsbruck}{}\ledrightnote{\textcolor{pink}{Innsbruck}}
               verſucht ich diesmal \textcolor{pink}{Tiroler {\pb}Hof}{}\ledrightnote{\textcolor{pink}{Tiroler Hof}}. Ich warne Sie. Es iſt ſchmierig und
                  ver\textsc{snobt}. Das ſchönſte bisher war natürlich \textcolor{pink}{\textsc{Hel\introOben{}l\introOben{}brunn}}{}\ledrightnote{\textcolor{pink}{Hellbrunn}}. Heuer zum erſten Mal hab ich auch das \textcolor{pink}{Schloſs}{}\ledrightnote{→\textcolor{pink}{Hellbrunn}} geſehn, innen (nicht das »\textcolor{pink}{Monatsſchlößel}{}\ledrightnote{\textcolor{pink}{Monatsschlössl}}«, ſondern das ununterbrochene.) –\pend
           \pstart
           Leben Sie wohl und ſchreiben Sie bald.\pend
           \pstart
           {\pb}Von Herzen Ihr{\\[\baselineskip]}\spacefill\mbox{Arthur}\pend
           \leftskip=0em{}\endnumbering\briefempfaengerindex{Beer-Hofmann, Richard@\textsc{Beer-Hofmann, Richard}!zzzSchnitzler, Arthur@\emph{von Arthur Schnitzler}!1901-07-041@{4. 7. 1901}|)be}\mylabel{h}  \normalsize

\doendnotes{C}
\bigskip
\vfill

\clearpage

\footnotesize

\lohead{\textsc{register}}

% Definiere theindex-Environment komplett neu ohne reledmac
\makeatletter
\renewenvironment{theindex}{%
  \section*{\indexname}%
  \setlength{\parindent}{0pt}%
  \setlength{\parskip}{0pt plus 0.3pt}%
  \let\item\@idxitem
}{%
  \clearpage
}
\makeatother

\IfFileExists{\jobname-pw.ind}{\input{\jobname-pw.ind}}{}

\end{document}

      