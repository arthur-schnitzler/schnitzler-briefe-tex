%% latex-korrekturansicht-vorspann.tex
%% Vorspann für die Korrekturansicht.
%% Lädt die gemeinsame Datei latex-vorspann.tex mit gesetztem Schalter.

\newif\ifkorrekturansicht
\korrekturansichttrue

\input{../tex-inputs/latex-vorspann}


               \section[Wilhelm Bölsche an Arthur Schnitzler, 15. 12. 1891]{ Wilhelm Bölsche an Arthur Schnitzler, 15. 12. 1891}\nopagebreak\mylabel{v}\rehead{ }\normalsize\beginnumbering\briefempfaengerindex{Schnitzler, Arthur@\textsc{Schnitzler, Arthur}!zzzBoelsche, Wilhelm@\emph{von Wilhelm Bölsche}!1891-12-152@{15. 12. 1891}|(be} \toendnotes[C]{\smallbreak\pagebreak[2]} \Standort{DLA, A:Schnitzler, HS.NZ85.1.2577,3.}
\physDesc{Brief, 1 Blatt, 1 Seite
\newline{}Handschrift: schwarze Tinte, deutsche Kurrent
\newline{}Schnitzler: mit rotem Buntstift nummeriert: »4« }\buchAbdrucke{\weitereDrucke{Wilhelm Bölsche: \emph{Briefwechsel. Mit Autoren der Freien Bühne}. Hg. Gerd-Hermann Susen. Berlin: \emph{Weidler} 2010, S. 673 (Werke und Briefe. Wissenschaftliche Ausgabe, Briefe I).} }\toendnotes[C]{\smallbreak}\pstart
           \noindent{}\centering{}{\pb}\textcolor{gray}{\textbf{\textsc{\textcolor{green}{Freie Bühne}{}\ledrightnote{\textcolor{green}{Freie Bühne für modernes Leben}}}}}\pend
           \pstart
           \noindent{}\centering{}\textcolor{gray}{\textbf{\textsc{für modernes Leben.}}}\pend
           \pstart
           \noindent{}\centering{}\textcolor{gray}{\textbf{\textsc{Herausgegeben von \textcolor{blue}{\textbf{Otto Brahm}}{}\ledrightnote{\textcolor{blue}{Otto Brahm}}.}}}\pend
           \pstart
           \noindent{}\textcolor{gray}{\textbf{Verlag und Expedition: \textcolor{brown}{S. Fischer}{}\ledrightnote{\textcolor{brown}{S. Fischer Verlag}}.}}\pend
           \pstart
           \textcolor{gray}{\textbf{Sprechstunden: Mittwoch und Freitag 12–2 Uhr.}}\pend
           \pstart
           \textcolor{gray}{\textbf{Alle für die Redaction bestimmten Sendungen (Beiträge,
                            Recensions-Exempl.) bitten wir \textbf{ohne Angabe eines
                                Personennamens} an die Redaction der Wochenschrift »\textcolor{green}{\so{Freie Bühne}}{}\ledrightnote{\textcolor{green}{Freie Bühne für modernes Leben}}« \textcolor{pink}{Berlin W. Link-Strasse 25}{}\ledrightnote{\textcolor{pink}{Linkstraße}} zu
                            addressiren.}}\pend
           \pstart
           \textcolor{gray}{\textbf{Wir ersuchen unsere geehrten Mitarbeiter, jedes
                            Manuscript auf der ersten Seite mit ihrer genauen Adresse zu
                            versehen.}}\pend
           \pstart
           \raggedleft{}\textcolor{pink}{Friedrichshagen}{}\ledrightnote{\textcolor{pink}{Friedrichshagen}}\pend
           \pstart
           \noindent{}\raggedleft{}bei \textcolor{gray}{\textbf{\textsc{\textcolor{pink}{Berlin}{}\ledrightnote{\textcolor{pink}{Berlin}}}, den}}{ }15. XII. \textcolor{gray}{\textbf{189}}1.\pend
           \pstart
           \noindent{}\raggedleft{}\textcolor{gray}{\textbf{\textcolor{pink}{\strikeout{W. Link-Straße 25}}{}\ledrightnote{\textcolor{pink}{Linkstraße}}.}}\pend
           \pstart
           \noindent{}\raggedleft{}\textcolor{pink}{Wilhelmſtr. 72}{}\ledrightnote{\textcolor{pink}{Peter-Hille-Straße}}.\pend
           \pstart{}Hochgeehrter Herr Doktor!\pend\pstart
           Vom 1. Jan. ab wird die \textcolor{green}{Freie
                        Bühne}{}\ledrightnote{\textcolor{green}{Freie Bühne für den Entwickelungskampf der Zeit}}{ }\label{K_L00053_1v}\edtext{Monatsſchrift}{\lemma{\textnormal{\emph{Monatsſchrift}}}\Cendnote{\textnormal{In den Jahren 1890 und 1891 erschien die \emph{\textcolor{green}{Freie Bühne}} wöchentlich.}}}\label{K_L00053_1h} unter \uline{meiner ausſchließlichen} Leitung. Ich freue mich,
                    daß Ihre \textcolor{green}{Novelle}{}\ledrightnote{→\textcolor{green}{Der Sohn. Aus den Papieren eines Arztes}}, ſo lange
                    zum Warten verurteilt, nun an gewichtiger Stelle grade das neue Quartal im
                    erſten Monatsheft eröffnen kann. Und ich füge die Bitte bei um freundliche
                    weitere Teilnahme.\pend
           \pstart
           Mit vorzüglicher Hochachtung{\\[\baselineskip]}\spacefill\mbox{Wilhelm Bölsche.}\pend
           \leftskip=0em{}\endnumbering\briefempfaengerindex{Schnitzler, Arthur@\textsc{Schnitzler, Arthur}!zzzBoelsche, Wilhelm@\emph{von Wilhelm Bölsche}!1891-12-152@{15. 12. 1891}|)be}\mylabel{h}  \normalsize

\doendnotes{C}
\bigskip
\vfill

\clearpage

\footnotesize

\lohead{\textsc{register}}

% Definiere theindex-Environment komplett neu ohne reledmac
\makeatletter
\renewenvironment{theindex}{%
  \section*{\indexname}%
  \setlength{\parindent}{0pt}%
  \setlength{\parskip}{0pt plus 0.3pt}%
  \let\item\@idxitem
}{%
  \clearpage
}
\makeatother

\IfFileExists{\jobname-pw.ind}{\input{\jobname-pw.ind}}{}

\end{document}

      