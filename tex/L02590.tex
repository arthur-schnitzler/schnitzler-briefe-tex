%% latex-korrekturansicht-vorspann.tex
%% Vorspann für die Korrekturansicht.
%% Lädt die gemeinsame Datei latex-vorspann.tex mit gesetztem Schalter.

\newif\ifkorrekturansicht
\korrekturansichttrue

\input{../tex-inputs/latex-vorspann}


               \section[Marie Herzfeld an Arthur Schnitzler, 7. 8. 1896]{ Marie Herzfeld an Arthur Schnitzler, 7. 8. 1896}\nopagebreak\mylabel{v}\rehead{ }\normalsize\beginnumbering\briefempfaengerindex{Schnitzler, Arthur@\textsc{Schnitzler, Arthur}!zzzHerzfeld, Marie@\emph{von Marie Herzfeld}!1896-08-073@{7. 8. 1896}|(be} \toendnotes[C]{\smallbreak\pagebreak[2]} \Standort{DLA, A:Schnitzler, HS.1985.1.03436,1.}
\physDesc{Brief, 1 Blatt, 4 Seiten
\newline{}Handschrift: schwarze Tinte, lateinische Kurrent
\newline{}Schnitzler: 1) mit Bleistift Vermerk »\textsc{Herzfeld}« 2) mit rotem Buntstift »\textsc{(\textcolor{blue}{Brand}}{[}es{]}«}\toendnotes[C]{\smallbreak}\pstart
           \raggedleft{}{\pb}\textcolor{pink}{Grundlsee}{}\ledrightnote{\textcolor{pink}{Grundlsee}}, 7. Aug. 96\pend
           \pstart\center{}Sehr geehrter Herr Doktor!\pend\pstart
           Im \textcolor{pink}{dänischen}{}\ledrightnote{\textcolor{pink}{Dänemark}} Blatt »\textcolor{green}{Politiken}{}\ledrightnote{\textcolor{green}{Politiken}}« v. 5. Aug. steht ein \label{K_L02590-1v}\edtext{\textcolor{green}{Artikel}{}\ledrightnote{→\textcolor{green}{To Forestillinger af Henrik IV}} von \textcolor{blue}{Georg Brandes}{}\ledrightnote{\textcolor{blue}{Georg Brandes}} »\textcolor{green}{Zwei Vorstellungen
                  Heinrich IV}{}\ledrightnote{\textcolor{green}{To Forestillinger af Henrik IV}}«}{\lemma{\textnormal{\emph{Artikel … Heinrich IV«}}}\Cendnote{\textnormal{\textcolor{blue}{G. B.
                        [=Georg Brandes]}: \emph{\textcolor{green}{To Forestillinger af
                        Henrik IV}}. In: \emph{\textcolor{green}{Politiken}},
                        5. 8. 1896, S. 1–2.}}}\label{K_L02590-1h}, in welchem
               folgende Stelle sich findet: »\label{K_L02590-2v}\edtext{Unter
               den Stücken, die ich da (›\textcolor{pink}{Deutsches Theater}{}\ledrightnote{\textcolor{pink}{Deutsches Theater Berlin}}‹ in \textcolor{pink}{Berlin}{}\ledrightnote{\textcolor{pink}{Berlin}}) mit vollendeter Kunst dargestellt sah, nenne
               ich das bewunderungswürdige \textcolor{pink}{östreichische}{}\ledrightnote{\textcolor{pink}{Österreich}}
               Trauerspiel ›\textcolor{green}{Liebelei}{}\ledrightnote{\textcolor{green}{Liebelei. Schauspiel in drei Akten}}‹ von \textcolor{blue}{Arthur Schnitzler}{}\ledrightnote{}, \strikeout{unter}
               demjenigen \strikeout{und}{ }{\pb}unter \substVorne{}\textsuperscript{den}\substDazwischen{}allen\substHinten{}{ }\textcolor{pink}{östr.}{}\ledrightnote{\textcolor{pink}{Österreich}}
               Dichtern, dessen Talent am eigentümlichsten und sichersten ist.«}{\lemma{\textnormal{\emph{Unter … ist.«}}}\Cendnote{\textnormal{siehe A. S.: \emph{Tagebuch}, 18. 8. 1896}}}\label{K_L02590-2h} Ich weiß, dass dieser Ausspruch,
               den ich lieber genau als elegant zu übersetzen bemüht war, Ihnen Freude machen wird;
               denn man mag von \textcolor{blue}{Brandes}{}\ledrightnote{\textcolor{blue}{Georg Brandes}} denken, wie man will –
               ich gehöre nur \uline{sehr} bedingt zu seinen Bewunderern, –
               er ist ein geistvoller Mensch mit sehr sicherem Instinkt für das, was durchdringen
               wird, u. er hat eine so umfassende Kenntnis der modernen Erscheinungen, dass von ihm
                  be{\pb}merkt und »bewundert« zu werden etwas Auszeichnendes
               hat. Nach diesem kann es Ihnen wol höchstens als anmaßend scheinen, wenn ich Ihnen
               meine Eindrücke von Ihrem \textcolor{green}{Stück}{}\ledrightnote{→\textcolor{green}{Liebelei. Schauspiel in drei Akten}},
               das ich – durch ein \label{K_L02590-3v}\edtext{Trauerjahr}{\lemma{\textnormal{\emph{Trauerjahr}}}\Cendnote{\textnormal{Am 2. 11. 1894 starb
                  ihre Mutter \textcolor{blue}{Betty Herzfeld}, die wie \textcolor{blue}{Schnitzler}s \textcolor{blue}{Mutter} in \textcolor{pink}{Kőszeg} geboren
                  war.}}}\label{K_L02590-3h} und eine vielmonatliche Krankenpflege auch noch diesen Winter
               verhindert – erst im Mai{ }\introOben{}od Juni\introOben{} vor unserer Abreise sah, eingehend schildere.\pend
           \pstart
           Ich will nicht behaupten, dass es im Ganzen über Ihren \textcolor{green}{Anatol}{}\ledrightnote{\textcolor{green}{Anatol}} Scenen steht; damit bewundere ich aber nur \textcolor{green}{Anatol}{}\ledrightnote{\textcolor{green}{Anatol}}. Gewiss sind Sie mit dieser Arbeit in {\pb}die erste Linie deutscher Bühnenschriftsteller gerückt –
               obwol Ihr Talent darin noch novellistisch \strikeout{arbeitet}
               gestaltet, bei allem Gefühl für das Theatralische in besserem Sinn. Ich habe mir Ihre
                  \label{K_L02590-4v}\edtext{Erzälungen}{\lemma{\textnormal{\emph{Erzälungen}}}\Cendnote{\textnormal{keine klare Bezugnahme, die erste Zusammenstellung von
                  Prosatexten in Buchform erschien erst 1898}}}\label{K_L02590-4h}{ }\textcolor{pink}{hieher}{}\ledrightnote{→\textcolor{pink}{Grundlsee}}
               mitgenommen und hoffe sie \textcolor{pink}{hier}{}\ledrightnote{→\textcolor{pink}{Grundlsee}}
               in ein paar ruhigen Stunden zu lesen.\pend
           \pstart
           Mit besten Wünschen für Ihre Arbeiten, {\\[\baselineskip]}\spacefill\mbox{Marie
               Herzfeld}\pend
           \leftskip=0em{}\endnumbering\briefempfaengerindex{Schnitzler, Arthur@\textsc{Schnitzler, Arthur}!zzzHerzfeld, Marie@\emph{von Marie Herzfeld}!1896-08-073@{7. 8. 1896}|)be}\mylabel{h}  \normalsize

\doendnotes{C}
\bigskip
\vfill

\clearpage

\footnotesize

\lohead{\textsc{register}}

% Definiere theindex-Environment komplett neu ohne reledmac
\makeatletter
\renewenvironment{theindex}{%
  \section*{\indexname}%
  \setlength{\parindent}{0pt}%
  \setlength{\parskip}{0pt plus 0.3pt}%
  \let\item\@idxitem
}{%
  \clearpage
}
\makeatother

\IfFileExists{\jobname-pw.ind}{\input{\jobname-pw.ind}}{}

\end{document}

      