%% latex-korrekturansicht-vorspann.tex
%% Vorspann für die Korrekturansicht.
%% Lädt die gemeinsame Datei latex-vorspann.tex mit gesetztem Schalter.

\newif\ifkorrekturansicht
\korrekturansichttrue

\input{../tex-inputs/latex-vorspann}


               \section[Arthur Schnitzler an Richard Beer-Hofmann, 7. 12. 1909]{ Arthur Schnitzler an Richard Beer-Hofmann, 7. 12. 1909}\nopagebreak\mylabel{v}\rehead{ }\normalsize\beginnumbering\briefempfaengerindex{Beer-Hofmann, Richard@\textsc{Beer-Hofmann, Richard}!zzzSchnitzler, Arthur@\emph{von Arthur Schnitzler}!1909-12-071@{7. 12. 1909}|(be} \toendnotes[C]{\smallbreak\pagebreak[2]} \Standort{YCGL, MSS 31.}
\physDesc{Visitenkarte
\newline{}Handschrift: Bleistift, deutsche Kurrent}\buchAbdrucke{\weitereDrucke{Arthur Schnitzler, Richard Beer-Hofmann: \emph{Briefwechsel 1891–1931}. Hg. Konstanze Fliedl. Wien, Zürich: \emph{Europaverlag} 1992, S. 196.} }\toendnotes[C]{\smallbreak}\pstart
           \raggedleft{}{\pb}\substVorne{}\textsuperscript{6}\substDazwischen{}7\substHinten{}/12 09.\pend
           \pstart
           \centering{}\textcolor{gray}{\textbf{D\textsuperscript{r} Arthur Schnitzler}}\pend
           \pstart
           {\pb}lieber Richard, ich höre von
               verſchiedenen Seiten dſs im \textcolor{pink}{Apollo-Theater}{}\ledrightnote{\textcolor{pink}{Apollo-Theater}} ein
                  \label{K_L01893_1v}\edtext{\textcolor{green}{Plagiat}{}\ledrightnote{→\textcolor{green}{Die schwarze Mali. Sketch}}}{\lemma{\textnormal{\emph{Plagiat}}}\Cendnote{\textnormal{In \emph{\textcolor{green}{Ma
                     gosse}} (\emph{\textcolor{green}{Die schwarze Mali}}) heuert ein
                  Wirt Schauspieler an, die den Gästen ein Kriminalstück vorspielen.}}}\label{K_L01893_1h} des \textcolor{green}{Kakadu}{}\ledrightnote{\textcolor{green}{Der grüne Kakadu. Groteske in einem Akt}} »nach dem franzöſiſchen« geſpielt wird; es
               liegt mir daran die Sache ſo bald als möglich zu ſehen – wollen Sie \introOben{}\textcolor{blue}{Beide}{}\ledrightnote{→\textcolor{blue}{Paula Beer-Hofmann}}\introOben{}
               heute mit uns eine \textcolor{green}{Loge}{}\ledrightnote{→\textcolor{green}{Die schwarze Mali. Sketch}} nehmen?
               Herzlichst Ihr\pend
           \pstart \spacefill\mbox{A.}\pend{}\endnumbering\briefempfaengerindex{Beer-Hofmann, Richard@\textsc{Beer-Hofmann, Richard}!zzzSchnitzler, Arthur@\emph{von Arthur Schnitzler}!1909-12-071@{7. 12. 1909}|)be}\mylabel{h}  \normalsize

\doendnotes{C}
\bigskip
\vfill

\clearpage

\footnotesize

\lohead{\textsc{register}}

% Definiere theindex-Environment komplett neu ohne reledmac
\makeatletter
\renewenvironment{theindex}{%
  \section*{\indexname}%
  \setlength{\parindent}{0pt}%
  \setlength{\parskip}{0pt plus 0.3pt}%
  \let\item\@idxitem
}{%
  \clearpage
}
\makeatother

\IfFileExists{\jobname-pw.ind}{\input{\jobname-pw.ind}}{}

\end{document}

      