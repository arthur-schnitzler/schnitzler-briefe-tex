%% latex-korrekturansicht-vorspann.tex
%% Vorspann für die Korrekturansicht.
%% Lädt die gemeinsame Datei latex-vorspann.tex mit gesetztem Schalter.

\newif\ifkorrekturansicht
\korrekturansichttrue

\input{../tex-inputs/latex-vorspann}


               \section[Paul Goldmann an Arthur Schnitzler, 3. 4. {[}1894{]}]{ Paul Goldmann an Arthur Schnitzler, 3. 4. {[}1894{]}}\nopagebreak\mylabel{v}\rehead{ }\normalsize\beginnumbering\briefempfaengerindex{Schnitzler, Arthur@\textsc{Schnitzler, Arthur}!zzzGoldmann, Paul@\emph{von Paul Goldmann}!1894-04-031@{3. 4. {[}1894{]}}|(be} \toendnotes[C]{\smallbreak\pagebreak[2]} \Standort{DLA, A:Schnitzler, HS.NZ85.1.3164.}
\physDesc{Brief, 1 Blatt, 4 Seiten
\newline{}Handschrift: schwarze Tinte, deutsche Kurrent
\newline{}Schnitzler: 1) mit Bleistift auf dem ersten Blatt die Jahreszahl »94« vermerkt 2) mit rotem Buntstift zwei Unterstreichungen}\toendnotes[C]{\smallbreak}\pstart
           \raggedleft{}{\pb}\textsc{\textcolor{pink}{Paris}{}\ledrightnote{\textcolor{pink}{Paris}}}, 3. April.\pend
           \pstart\center{}Mein lieber Freund,\pend\pstart
           Ich habe Dir für zwei liebe Briefe zu danken, und ich muß Dir immer und immer
               wiederholen, wie wohl mir Deine treue Freundſchaft thut und Deine Antheilnahme an
               Allem, was ich leiſte. Es gibt mir beim Arbeiten eine gewiſſe Anregung, wenn ich
               daran denke, daß ich Dein Lob verdienen muß. Haſt Du mein \textcolor{green}{Feuilleton}{}\ledrightnote{→\textcolor{green}{[?? Feuilleton über Charles Meunier]}}{ }{\pb}über den armen \textsc{\textcolor{blue}{Charles Meunier}{}\ledrightnote{\textcolor{blue}{Charles Meunier}}} geleſen? Da habe ich auch \label{K_L02615-3v}\edtext{viel
               für Dich geſchrieben}{\lemma{\textnormal{\emph{viel … geſchrieben}}}\Cendnote{\textnormal{XXXX}}}\label{K_L02615-3h}. Wenn es Dir entgangen iſt,
               ſo will ichs Dir ſchicken.\pend
           \pstart
           Du biſt aber auch der Einzige, der Antheil an meinem Schaffen nimmt. Sonſt verhallts
               in der Wüſte. Ich ſehe immer mehr, daß nichts aus mir wird.\pend
           \pstart
           Gern hätte ich mich mit Dir getroffen. Seit unſerm \label{K_L02615-4v}\edtext{letzten Beiſammenſein}{\lemma{\textnormal{\emph{letzten Beiſammenſein}}}\Cendnote{\textnormal{am 14. 11. 1894}}}\label{K_L02615-4h} denke ich fortwährend daran und mache allerhand Pläne. {\pb}Aber es iſt ein furchtbarer Strich durch die
               Rechnung gekommen. Ich werde immer kränker. Der aufreibende Beruf vergrößert das
               Übel, das ſtetig um ſich greift. Ich fürchte, ich werde nicht mehr lange die Feder
               führen können. Jedenfalls verlangt mein \textcolor{blue}{Schwager}{}\ledrightnote{→\textcolor{blue}{Josef Rosengart}}, daß ich meinen Urlaub in \textcolor{pink}{Frankfurt}{}\ledrightnote{\textcolor{pink}{Frankfurt am Main}} verbringe, damit er mich \label{K_L02615-2v}\edtext{behandeln}{\lemma{\textnormal{\emph{behandeln}}}\Cendnote{\textnormal{\textcolor{blue}{Josef Rosengart}, Ehemann von Goldmanns
                  Schwester \textcolor{blue}{Vally}, war Arzt.}}}\label{K_L02615-2h}
               könne.\pend
           \pstart
           \textsc{\textcolor{blue}{Albert}{}\ledrightnote{\textcolor{blue}{Henri Albert}}} will natürlich keinen \label{K_L02615-1v}\edtext{Preis
                  beſtimmen}{\lemma{\textnormal{\emph{Preis
                  beſtimmen}}}\Cendnote{\textnormal{siehe Paul Goldmann an Arthur Schnitzler, 21. 3. [1894]}}}\label{K_L02615-1h}. Das mittlere Überſetzungs-{\pb}Honorar für
               einen Deiner kleinen \textcolor{green}{Dialoge}{}\ledrightnote{→\textcolor{green}{Weihnachts-Einkäufe}} wären 25 bis 30 \textsc{Francs}. Wäre Dir das zu
               viel? Schreib’ ganz offen, ich richte die Sache ſchon ein, wie es für Dich am Beſten
               iſt.\pend
           \pstart
           \textsc{\textcolor{blue}{Herzl}{}\ledrightnote{\textcolor{blue}{Theodor Herzl}}} hat ſich ſehr mit deiner \label{K_L02615-5v}\edtext{Anerkennung}{\lemma{\textnormal{\emph{Anerkennung}}}\Cendnote{\textnormal{nicht ermittelt;
                  eventuell bezieht sich das auf die \textcolor{green}{Rezension} des \emph{\textcolor{green}{Modernen
                     Musen-Almanachs für das Jahr 1894}} durch \textcolor{blue}{Henri Albert}, doch die erschien bereits im März. In den
                  Korrespondenzstücken zwischen \textcolor{blue}{Schnitzler} und
                     \textcolor{blue}{Herzl} findet sich in dieser Zeit nichts,
                  was näheren Aufschluss gibt.}}}\label{K_L02615-5h} gefreut. Ich glaube, Du wirſt nächſtens etwas
               wahrhaft \label{K_L02615-6v}\edtext{Schönes}{\lemma{\textnormal{\emph{Schönes}}}\Cendnote{\textnormal{eventuell der Einakter \emph{\textcolor{green}{Die Glosse}}, vgl. A. S.: \emph{Tagebuch}, 31. 8. 1894}}}\label{K_L02615-6h} von ihm zu genießen bekommen, darf aber nicht reden.\pend
           \pstart
           Herzlichſt und in Treue{\\[\baselineskip]}Dein {\\[\baselineskip]}\spacefill\mbox{Paul Goldmann}\pend
           \leftskip=0em{}\pstart
           \noindent{}Was haſt Du Oſtern gemacht?\pend
           \endnumbering\briefempfaengerindex{Schnitzler, Arthur@\textsc{Schnitzler, Arthur}!zzzGoldmann, Paul@\emph{von Paul Goldmann}!1894-04-031@{3. 4. {[}1894{]}}|)be}\mylabel{h}  \normalsize

\doendnotes{C}
\bigskip
\vfill

\clearpage

\footnotesize

\lohead{\textsc{register}}

% Definiere theindex-Environment komplett neu ohne reledmac
\makeatletter
\renewenvironment{theindex}{%
  \section*{\indexname}%
  \setlength{\parindent}{0pt}%
  \setlength{\parskip}{0pt plus 0.3pt}%
  \let\item\@idxitem
}{%
  \clearpage
}
\makeatother

\IfFileExists{\jobname-pw.ind}{\input{\jobname-pw.ind}}{}

\end{document}

      