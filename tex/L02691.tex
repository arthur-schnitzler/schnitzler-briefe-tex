%% latex-korrekturansicht-vorspann.tex
%% Vorspann für die Korrekturansicht.
%% Lädt die gemeinsame Datei latex-vorspann.tex mit gesetztem Schalter.

\newif\ifkorrekturansicht
\korrekturansichttrue

\input{../tex-inputs/latex-vorspann}


               \section[Paul Goldmann an Arthur Schnitzler, {[}24. 8. 1896?{]}]{ Paul Goldmann an Arthur Schnitzler, {[}24. 8. 1896?{]}}\nopagebreak\mylabel{v}\rehead{ }\normalsize\beginnumbering\briefempfaengerindex{Schnitzler, Arthur@\textsc{Schnitzler, Arthur}!zzzGoldmann, Paul@\emph{von Paul Goldmann}!1896-08-241@{{[}24. 8. 1896?{]}}|(be} \toendnotes[C]{\smallbreak\pagebreak[2]} \Standort{DLA, A:Schnitzler, HS.NZ85.1.3166.}
\physDesc{Telegramm1 Blatt, 1 Seite
\newline{}maschinell
\newline{}Schnitzler: mit Bleistift datiert: »Jann 96« \newline{}Ordnung: beschnitten }\toendnotes[C]{\smallbreak}\pstart
           \noindent{}{\pb}tausend dank fuer frohe nachricht und von ganzen herzen glueckwunsch
               jetzt ist dir das \label{K_L02691-1v}\edtext{\textcolor{green}{stueck}{}\ledrightnote{→\textcolor{green}{Freiwild. Schauspiel in 3 Akten}}}{\lemma{\textnormal{\emph{stueck}}}\Cendnote{\textnormal{Das Telegramm weist keine
               Datierung auf, wird aber in \textcolor{blue}{Schnitzler}s Nachlass mit den Korrespondenzstücken des
                  Jahres 1896 aufbewahrt. Inhaltlich passt es in diesem Jahr am besten zu \emph{\textcolor{green}{Freiwild}},
                  dem \textcolor{blue}{Schnitzler} selbst skeptisch gegenüber stand, das aber bei einem Treffen
                  am 23. 8. 1896 von \textcolor{blue}{Otto Brahm} für das \emph{\textcolor{brown}{Deutsche Theater}}
                  angenommen wurde.}}}\label{K_L02691-1h} hoffentlich sympathischer \spacefill\mbox{goldmann =}\pend
           \endnumbering\briefempfaengerindex{Schnitzler, Arthur@\textsc{Schnitzler, Arthur}!zzzGoldmann, Paul@\emph{von Paul Goldmann}!1896-08-241@{{[}24. 8. 1896?{]}}|)be}\mylabel{h}\begin{anhang}\end{anhang}\normalsize

\doendnotes{C}
\bigskip
\vfill

\clearpage

\footnotesize

\lohead{\textsc{register}}

% Definiere theindex-Environment komplett neu ohne reledmac
\makeatletter
\renewenvironment{theindex}{%
  \section*{\indexname}%
  \setlength{\parindent}{0pt}%
  \setlength{\parskip}{0pt plus 0.3pt}%
  \let\item\@idxitem
}{%
  \clearpage
}
\makeatother

\IfFileExists{\jobname-pw.ind}{\input{\jobname-pw.ind}}{}

\end{document}

      