%% latex-korrekturansicht-vorspann.tex
%% Vorspann für die Korrekturansicht.
%% Lädt die gemeinsame Datei latex-vorspann.tex mit gesetztem Schalter.

\newif\ifkorrekturansicht
\korrekturansichttrue

\input{../tex-inputs/latex-vorspann}


               \section[Arthur Schnitzler an Hermann Bahr, 13. 10. 1905]{ Arthur Schnitzler an Hermann Bahr, 13. 10. 1905}\nopagebreak\mylabel{v}\rehead{ }\normalsize\beginnumbering\briefempfaengerindex{Bahr, Hermann@\textsc{Bahr, Hermann}!zzzSchnitzler, Arthur@\emph{von Arthur Schnitzler}!1905-10-132@{13. 10. 1905}|(be} \toendnotes[C]{\smallbreak\pagebreak[2]} \Standort{TMW, HS AM 60177 Ba.}
\physDesc{Briefkarte
\newline{}Handschrift: schwarze Tinte, deutsche Kurrent\newline{}Ordnung: Lochung }\buchAbdrucke{\weitereDrucke{1) \emph{13. 10. 1905, Abschrift.} In: Arthur Schnitzler: \emph{The Letters of Arthur Schnitzler to Hermann Bahr}. Edited, annotated, and with an introduction, by Donald G.
                        Daviau. Chapel Hill: \emph{The University of North Carolina Press} 1978, S. 93 (University of North Carolina studies in the Germanic languages
                        and literatures, 89).} \weitereDrucke{2) Hermann Bahr, Arthur Schnitzler: \emph{Briefwechsel, Aufzeichnungen, Dokumente (1891–1931)}. Hg. Kurt Ifkovits und Martin Anton Müller. Göttingen: \emph{Wallstein} 2018, S. 361.} }\toendnotes[C]{\smallbreak}\pstart
           \noindent{}{\pb}\textcolor{gray}{\textbf{Dr. Arthur Schnitzler}}\hfill 13. X. 905\pend
           \pstart
           \textcolor{gray}{\textbf{\textcolor{pink}{Wien, XVIII. Spoettelgasse 7}{}\ledrightnote{\textcolor{pink}{Edmund-Weiß-Gasse}}.}}\pend
           \pstart
           eben, lieber Hermann, ko{\geminationm}t der \textcolor{green}{\textsc{Klub} der Erlöſer}{}\ledrightnote{\textcolor{green}{Der Klub der Erlöser}}, und dazu, zum 2. Mal, der \textsc{\textcolor{green}{arme Narr}{}\ledrightnote{\textcolor{green}{Der arme Narr}}}, den ich alſo ſchon geleſen, der mir eines deiner \label{LL087-1v}merkwürdigſten Produkte\label{LL087-1h} zu ſein ſcheint, und den ich am
               liebſten als eine Art von Vorſpiel zu \damage{e}inem ganz voll tönenden Drama auf dem Theater {\pb}ſehen möchte, das aber
               natürlich auch von \damage{dir}{ }ſein müßte, und zu dem mir alle Elemente in geheimnisvoller Weiſe ſchon in
               dieſem ſeltſamen Akt zu liegen ſcheinen.\pend
           \pstart
           Darf ich dir bei dieſer Gelegenheit gleich für \textcolor{green}{deine lieben Worte}{}\ledrightnote{→\textcolor{green}{Zwischenspiel. (Komödie in drei Akten von Arthur Schnitzler. Zum erstenmal aufgeführt im Burgtheater am 12. Oktober 1905)}} in der \textcolor{brown}{Volkszeitg}{}\ledrightnote{\textcolor{brown}{Österreichische Volks-Zeitung}} die Hand drücken?\pend
           \pstart
           \label{K_L01562_1v}\edtext{So{\geminationn}tag oder Montag}{\lemma{\textnormal{\emph{Sotag oder Montag}}}\Cendnote{\textnormal{Am Montag, dem 16. 10. 1905, fuhr
                  Schnitzler mit \textcolor{blue}{Brahm} auf den \textcolor{pink}{Semmering}.}}}\label{K_L01562_1h} fahr ich fort, auf einige Tage nur, dann
               auf Wiederſehen.\pend
           \pstart Von Herzen dein \spacefill\mbox{A.}\pend{}\endnumbering\briefempfaengerindex{Bahr, Hermann@\textsc{Bahr, Hermann}!zzzSchnitzler, Arthur@\emph{von Arthur Schnitzler}!1905-10-132@{13. 10. 1905}|)be}\mylabel{h}  \normalsize

\doendnotes{C}
\bigskip
\vfill

\clearpage

\footnotesize

\lohead{\textsc{register}}

% Definiere theindex-Environment komplett neu ohne reledmac
\makeatletter
\renewenvironment{theindex}{%
  \section*{\indexname}%
  \setlength{\parindent}{0pt}%
  \setlength{\parskip}{0pt plus 0.3pt}%
  \let\item\@idxitem
}{%
  \clearpage
}
\makeatother

\IfFileExists{\jobname-pw.ind}{\input{\jobname-pw.ind}}{}

\end{document}

      