%% latex-korrekturansicht-vorspann.tex
%% Vorspann für die Korrekturansicht.
%% Lädt die gemeinsame Datei latex-vorspann.tex mit gesetztem Schalter.

\newif\ifkorrekturansicht
\korrekturansichttrue

\input{../tex-inputs/latex-vorspann}


               \section[Hugo von Hofmannsthal an Arthur Schnitzler, 27. 7. 1897]{ Hugo von Hofmannsthal an Arthur Schnitzler, 27. 7. 1897}\nopagebreak\mylabel{v}\rehead{ }\normalsize\beginnumbering\briefempfaengerindex{Schnitzler, Arthur@\textsc{Schnitzler, Arthur}!zzzHofmannsthal, Hugo von@\emph{von Hugo von Hofmannsthal}!1897-07-271@{27. 7. 1897}|(be} \toendnotes[C]{\smallbreak\pagebreak[2]} \Standort{CUL, Schnitzler, B 43.}
\physDesc{Postkarte
\newline{}Handschrift: schwarze Tinte, deutsche Kurrent\newline{}Versand: 1) Stempel: »\nobreak{}\oindex{Bad Fusch@\textbf{Bad Fusch}, \emph{Besiedelter Ort (A.BSO)}|pwk}Bad Fusch, 28/7 \textcolor{gray}{9}7\nobreak{}«.  2) Stempel: »\nobreak{}Wien, 29. 7. 97, 8.V, Bestellt\nobreak{}«. \newline{}Ordnung: mit Bleistift von unbekannter Hand nummeriert:
                                        »94a« }\buchAbdrucke{\weitereDrucke{Hugo von Hofmannsthal, Arthur Schnitzler: \emph{Briefwechsel}. Hg. Therese Nickl und Heinrich Schnitzler. Frankfurt am Main: \emph{S. Fischer} 1964, S. 95.} }\pstart{}{\pb}\textsc{Herrn D\textsuperscript{r} Arthur Schnitzler}\pend{}\pstart{}\textcolor{pink}{\textsc{Wien}}{}\ledrightnote{\textcolor{pink}{Wien}}\pend{}\pstart{}\textcolor{pink}{\textsc{IX Franckgasse 1}}{}\ledrightnote{\textcolor{pink}{Frankgasse}}\pend{}{\bigskip}\pstart
           \raggedleft{}{\pb}27. VII.\pend
           \pstart{}lieber Arthur\pend\pstart
           bitte ſeien Sie ſo lieb und ſchicken an \textcolor{blue}{Poldy}{}\ledrightnote{\textcolor{blue}{Leopold von Andrian-Werburg}}
                        (\textcolor{pink}{\textsc{Villa Andrian}}{}\ledrightnote{\textcolor{pink}{Villa Andrian}}{ }\textcolor{pink}{\textsc{Alt-Aussee}}{}\ledrightnote{\textcolor{pink}{Altaussee}}) möglichſt bald ein \uline{Buch} wo
                    Symptome gewöhnlicher Neuraſthenie drinſtehen, damit er nervöſe Beklemmungen u.
                    dgl. als ſolche erkennen kann.\pend
           \pstart
           Bin wohl und hätte gern Nachricht von Ihnen, bis 5\textsuperscript{ten} Auguſt hierher.\pend
           \pstart
           Herzlich Ihr{\\[\baselineskip]}\spacefill\mbox{Hugo.}\pend
           \leftskip=0em{}\endnumbering\briefempfaengerindex{Schnitzler, Arthur@\textsc{Schnitzler, Arthur}!zzzHofmannsthal, Hugo von@\emph{von Hugo von Hofmannsthal}!1897-07-271@{27. 7. 1897}|)be}\mylabel{h}  \normalsize

\doendnotes{C}
\bigskip
\vfill

\clearpage

\footnotesize

\lohead{\textsc{register}}

% Definiere theindex-Environment komplett neu ohne reledmac
\makeatletter
\renewenvironment{theindex}{%
  \section*{\indexname}%
  \setlength{\parindent}{0pt}%
  \setlength{\parskip}{0pt plus 0.3pt}%
  \let\item\@idxitem
}{%
  \clearpage
}
\makeatother

\IfFileExists{\jobname-pw.ind}{\input{\jobname-pw.ind}}{}

\end{document}

      