%% latex-korrekturansicht-vorspann.tex
%% Vorspann für die Korrekturansicht.
%% Lädt die gemeinsame Datei latex-vorspann.tex mit gesetztem Schalter.

\newif\ifkorrekturansicht
\korrekturansichttrue

\input{../tex-inputs/latex-vorspann}


               \section[Richard Beer-Hofmann an Arthur Schnitzler, 28. 7. 1896]{ Richard Beer-Hofmann an Arthur Schnitzler, 28. 7. 1896}\nopagebreak\mylabel{v}\rehead{ }\normalsize\beginnumbering\briefempfaengerindex{Schnitzler, Arthur@\textsc{Schnitzler, Arthur}!zzzBeer-Hofmann, Richard@\emph{von Richard Beer-Hofmann}!1896-07-281@{28. 7. 1896}|(be} \toendnotes[C]{\smallbreak\pagebreak[2]} \Standort{CUL, Schnitzler, B 8.}
\physDesc{Brief, 1 Blatt, 4 Seiten
\newline{}Handschrift: Bleistift, lateinische Kurrent
\newline{}Schnitzler: mit Bleistift am Beginn des Briefes datiert: »28/7 96« \newline{}Ordnung: mit Bleistift von unbekannter Hand nummeriert:
                              »78« }\buchAbdrucke{\weitereDrucke{Arthur Schnitzler, Richard Beer-Hofmann: \emph{Briefwechsel 1891–1931}. Hg. Konstanze Fliedl. Wien, Zürich: \emph{Europaverlag} 1992, S. 93–94.} }\pstart
           \noindent{}{\pb}Lieber Arthur! Es
               ist infam.\pend
           \pstart
           \textcolor{pink}{\uline{Klampenborg}}{}\ledrightnote{\textcolor{pink}{Klampenborg}} wegen Eleganz ausgeschlossen\pend
           \pstart
           \textcolor{pink}{\uline{Skodsborg}}{}\ledrightnote{\textcolor{pink}{Skodsborg}} sehr voll und vermutlich geräuschvoll\pend
           \pstart
           Also \textcolor{pink}{\uline{Vedbaek}}{}\ledrightnote{\textcolor{pink}{Vedbæk}} (10 Minuten weiter als \textcolor{pink}{Klampenborg}{}\ledrightnote{\textcolor{pink}{Klampenborg}}.)\pend
           \pstart
           das ist bescheiden billig – für \substVorne{}\textsuperscript{g}\substDazwischen{}e\substHinten{}in Zi{\geminationm}er mit 2 Betten und Pension für 2 Personen
               10 Kronen, aber das Zi{\geminationm}er wird erst {\pb}Samstag oder Sonntag frei, und ich bin also noch
               unentschlossen was tun. Ko{\geminationm}en Sie daher lieber \uline{direkt}{ }\textcolor{pink}{Kopenhagen}{}\ledrightnote{\textcolor{pink}{Kopenhagen}} und entweder bin ich noch dort und wir
               berathen gemeinsam, oder ich bin schon wo und ko{\geminationm}e Sie
               abholen nach \textcolor{pink}{Kopenhagen}{}\ledrightnote{\textcolor{pink}{Kopenhagen}}. –\pend
           \pstart
           {\pb}\textcolor{pink}{Vedbaek}{}\ledrightnote{\textcolor{pink}{Vedbæk}}, das weiteste, ist von \textcolor{pink}{Kopenhagen}{}\ledrightnote{\textcolor{pink}{Kopenhagen}} 1 Stunde 10 Minuten mit dem Schiff. Wo treffen Sie
               mit \textcolor{blue}{Paul}{}\ledrightnote{\textcolor{blue}{Paul Goldmann}} zusa{\geminationm}en\pend
           \pstart
           Wann ko{\geminationm}en Sie (\uline{genau})\pend
           \pstart
           \textcolor{blue}{Brandes}{}\ledrightnote{\textcolor{blue}{Georg Brandes}} ko{\geminationm}t morgen
               vom Land und fährt übermorgen weg, ich hoffe ihn zu sprechen. Vielleicht ist schon
               Brief von Ihnen da. {\pb}Ich war
               nämlich gestern nicht bei der Post, und gehe erst jetzt hin. Herrlich sind nur die
               Bäder hier. \textcolor{pink}{\uline{König von Dänemark}}{}\ledrightnote{\textcolor{pink}{Hotel König von Dänemark}} wohne ich.\pend
           \pstart
           Herzlichst{\\[\baselineskip]}Ihr{\\[\baselineskip]}\spacefill\mbox{Richard}\pend
           \leftskip=0em{}\pstart
           28/VII 96{ }\textcolor{pink}{Kopenhagen}{}\ledrightnote{\textcolor{pink}{Kopenhagen}}\pend
           \endnumbering\briefempfaengerindex{Schnitzler, Arthur@\textsc{Schnitzler, Arthur}!zzzBeer-Hofmann, Richard@\emph{von Richard Beer-Hofmann}!1896-07-281@{28. 7. 1896}|)be}\mylabel{h}  \normalsize

\doendnotes{C}
\bigskip
\vfill

\clearpage

\footnotesize

\lohead{\textsc{register}}

% Definiere theindex-Environment komplett neu ohne reledmac
\makeatletter
\renewenvironment{theindex}{%
  \section*{\indexname}%
  \setlength{\parindent}{0pt}%
  \setlength{\parskip}{0pt plus 0.3pt}%
  \let\item\@idxitem
}{%
  \clearpage
}
\makeatother

\IfFileExists{\jobname-pw.ind}{\input{\jobname-pw.ind}}{}

\end{document}

      