%% latex-korrekturansicht-vorspann.tex
%% Vorspann für die Korrekturansicht.
%% Lädt die gemeinsame Datei latex-vorspann.tex mit gesetztem Schalter.

\newif\ifkorrekturansicht
\korrekturansichttrue

\input{../tex-inputs/latex-vorspann}


               \section[Hugo von Hofmannsthal an Arthur Schnitzler, {[}1. 12. 1892{]}]{ Hugo von Hofmannsthal an Arthur Schnitzler, {[}1. 12. 1892{]}}\nopagebreak\mylabel{v}\rehead{ }\normalsize\beginnumbering\briefempfaengerindex{Schnitzler, Arthur@\textsc{Schnitzler, Arthur}!zzzHofmannsthal, Hugo von@\emph{von Hugo von Hofmannsthal}!1892-12-012@{{[}1. 12. 1892{]}}|(be} \toendnotes[C]{\smallbreak\pagebreak[2]} \Standort{CUL, Schnitzler, B 43.}
\physDesc{Briefkarte
\newline{}Handschrift: schwarze Tinte, deutsche Kurrent
\newline{}Schnitzler: mit Bleistift nummeriert: »34« und mit Jahreszahl
            versehen: »92« }\buchAbdrucke{\weitereDrucke{Hugo von Hofmannsthal, Arthur Schnitzler: \emph{Briefwechsel}. Hg. Therese Nickl und Heinrich Schnitzler. Frankfurt am Main: \emph{S. Fischer} 1964, S. 32.} }\toendnotes[C]{\smallbreak}\pstart
           \noindent{}\textcolor{gray}{\textbf{\label{T_L00141-1v}\edtext{AvH}{\lemma{\textnormal{\emph{AvH}}}\Cendnote{\textnormal{Monogramm der Mutter \textcolor{blue}{Anna von Hofmannsthal} mit Krone in Golddruck}}}\label{T_L00141-1h}}}\pend
           \pstart
           \raggedleft{}{\pb}Donnerstag.\pend
           \pstart{}Lieber Arthur.\pend\pstart
           Bitte alſo ſchicken Sie die Photographie dem \textcolor{blue}{Devrient}{}\ledrightnote{\textcolor{blue}{Max Devrient}} mit der Bitte um nicht zu langſame Rückſendung in
                    unterſchriebenem Zustand für irgend eine Verehrerin. Auf Wiederſehen
                        Sonntag! Besten dankend\pend
           \pstart \spacefill\mbox{Loris}\pend{}\endnumbering\briefempfaengerindex{Schnitzler, Arthur@\textsc{Schnitzler, Arthur}!zzzHofmannsthal, Hugo von@\emph{von Hugo von Hofmannsthal}!1892-12-012@{{[}1. 12. 1892{]}}|)be}\mylabel{h}  \normalsize

\doendnotes{C}
\bigskip
\vfill

\clearpage

\footnotesize

\lohead{\textsc{register}}

% Definiere theindex-Environment komplett neu ohne reledmac
\makeatletter
\renewenvironment{theindex}{%
  \section*{\indexname}%
  \setlength{\parindent}{0pt}%
  \setlength{\parskip}{0pt plus 0.3pt}%
  \let\item\@idxitem
}{%
  \clearpage
}
\makeatother

\IfFileExists{\jobname-pw.ind}{\input{\jobname-pw.ind}}{}

\end{document}

      