%% latex-korrekturansicht-vorspann.tex
%% Vorspann für die Korrekturansicht.
%% Lädt die gemeinsame Datei latex-vorspann.tex mit gesetztem Schalter.

\newif\ifkorrekturansicht
\korrekturansichttrue

\input{../tex-inputs/latex-vorspann}


               \section[Peter Altenberg an Arthur Schnitzler, {[}10.? 5. 1912{]}]{ Peter Altenberg an Arthur Schnitzler, {[}10.? 5. 1912{]}}\nopagebreak\mylabel{v}\rehead{ }\normalsize\beginnumbering\briefempfaengerindex{Schnitzler, Arthur@\textsc{Schnitzler, Arthur}!zzzAltenberg, Peter@\emph{von Peter Altenberg}!1912-05-101@{{[}10.? 5. 1912{]}}|(be} \toendnotes[C]{\smallbreak\pagebreak[2]} \Standort{CUL, Schnitzler, B 2.}
\physDesc{Brief, 1 Blatt, 2 Seiten
\newline{}Handschrift: blaue Tinte, deutsche Kurrent
\newline{}Schnitzler: mit rotem Buntstift beschrieben: »(an Tisch Mai
                  1912« \newline{}Ordnung: mit Bleistift von unbekannter Hand neben den Wunsch nach einem
                                            Exemplar Vermerk: »erledigt« }\toendnotes[C]{\smallbreak}\pstart
           \noindent{}\raggedleft{}{\pb}\textcolor{gray}{\textbf{Motto: \emph{Schneeglöcklein, läutest den
                            Frühling ein,}}}\pend
           \pstart
           \noindent{}\raggedleft{}\textcolor{gray}{\textbf{\emph{Für mich begräbst du den herrlichen Winter.}}}\pend
           \pstart
           \noindent{}\centering{}\textcolor{gray}{\textbf{\textcolor{pink}{HOTEL PANHANS AM SEMMERING}{}\ledrightnote{\textcolor{pink}{Hotel Panhans}}}}\pend
           \pstart
           \noindent{}\centering{}\textcolor{gray}{\textbf{mit dazugehörigem \textcolor{pink}{Hotel
                            Erzherzog Johann}{}\ledrightnote{\textcolor{pink}{Hotel Erzherzog Johann}}.}}\pend
           \pstart
           \noindent{}\textcolor{gray}{\textbf{1025 m Seehöhe.}}\hfill \textcolor{gray}{\textbf{1025 m Seehöhe.}}\pend
           \pstart
           \centering{}\textcolor{gray}{\textbf{400 Zimmer und Salons, meist mit Balkons,
                        Gesellschaftsloggien und gemeinsame Terrassen für Freiluft- und Liegekuren
                        in jedem Stockwerke.}}\pend
           \pstart
           \noindent{}\centering{}\textcolor{gray}{\textbf{Komplette Appartements mit Bad, Dusche und Toilette.
                        Überall elektrisches Licht und Warmwasserheizung, welche in jedem Zimmer
                        genau regulierbar (auch Wohnungen mit Öfen). Hausarzt, Apotheke, Lift.
                        Photographische Dunkelkammer, Automobil-Remise.}}\pend
           \pstart
           \noindent{}\centering{}\textcolor{gray}{\textbf{Großes Kaffeehaus, luxuriöse Halle, Konversations-, Spiel-,
                        Lese-, Musik- und Damensalons. Feinstes Orchester vom 20. Juni bis
                        20. September und vom 20. Dezember bis 20. März.}}\pend
           \pstart
           \noindent{}\centering{}\textcolor{gray}{\textbf{Neben dem Hotel befindet sich das schmucke \textcolor{pink}{Semmering-Kirchlein}{}\ledrightnote{\textcolor{pink}{Kirche zur heiligen Familie}} (jeden Tag heilige Messe).}}\pend
           \pstart
           \noindent{}\centering{}\textcolor{gray}{\textbf{Wintersportplatz und Höhenkurort allerersten Ranges.}}\pend
           \pstart
           \noindent{}\centering{}\textcolor{gray}{\textbf{Mittelpunkt des hiesigen Wintersports.}}\pend
           \pstart
           \noindent{}\centering{}\textcolor{gray}{\textbf{Sitz des \textcolor{brown}{Österreichischen
                            Wintersport-Klubs}{}\ledrightnote{\textcolor{brown}{Österreichischer Wintersport-Klub}} im \textcolor{pink}{Hotel Erzherzog
                            Johann}{}\ledrightnote{\textcolor{pink}{Hotel Erzherzog Johann}}.}}\pend
           \pstart
           \noindent{}\centering{}\textcolor{gray}{\textbf{Eigene Hochwildjagd, Forellenfischerei, Reitpferde.
                        Fahrräder und Wintersportrequisiten.}}\pend
           \pstart
           \noindent{}\centering{}\textcolor{gray}{\textbf{Tennis-, Croquet-, Eislauf-, Ski- und Rodelplätze.}}\pend
           \pstart
           \noindent{}\centering{}\textcolor{gray}{\textbf{Elektrischer Aufzug für Personen und Sportgeräte bei der
                        4 km langen Rodel- und Bobbahn.}}\pend
           \pstart
           \noindent{}\centering{}\textcolor{gray}{\textbf{Bade- und Wasserkur unter Leitung bewährter Ärzte.
                        Kohlensäure-, elektrische Dampfbäder, Inhalationen System Dr. \textcolor{blue}{Bulling}{}\ledrightnote{\textcolor{blue}{Anton Bulling}}. Hochquellenleitung.}}\pend
           \pstart
           \noindent{}\centering{}\textcolor{gray}{\textbf{Bester Nachkurort nach \textcolor{pink}{Karlsbad}{}\ledrightnote{\textcolor{pink}{Karlsbad}}, \textcolor{pink}{Marienbad}{}\ledrightnote{\textcolor{pink}{Marienbad}}, \textcolor{pink}{Franzensbad}{}\ledrightnote{\textcolor{pink}{Franzensbad}}, \textcolor{pink}{Teplitz}{}\ledrightnote{\textcolor{pink}{Teplice}}, \textcolor{pink}{Abbazia}{}\ledrightnote{\textcolor{pink}{Opatija}}, \textcolor{pink}{Meran}{}\ledrightnote{\textcolor{pink}{Opatija}}, \textcolor{pink}{Grado}{}\ledrightnote{\textcolor{pink}{Grado}}, \textcolor{pink}{Gastein}{}\ledrightnote{\textcolor{pink}{Bad Gastein}}, \textcolor{pink}{Pestyan}{}\ledrightnote{\textcolor{pink}{Piešťany}}, \textcolor{pink}{Davos}{}\ledrightnote{\textcolor{pink}{Davos}} usw.
                        Winterkuren.}}\pend
           \pstart
           \noindent{}\centering{}\textcolor{gray}{\textbf{Kammerlieferant der Kaiserl. Hoheiten Erzh. \textcolor{blue}{Franz Ferdinand}{}\ledrightnote{\textcolor{blue}{Franz Ferdinand von Österreich-Este}}, Erzh. \textcolor{blue}{Karl}{}\ledrightnote{\textcolor{blue}{Karl I. von Österreich-Ungarn}} und Erzh. \textcolor{blue}{Stephan}{}\ledrightnote{\textcolor{blue}{Karl Stephan von Österreich}}.}}\pend
           \pstart
           \noindent{}\centering{}\textcolor{gray}{\textbf{Sieben zum Hotel gehörige Villen mit Küchen und
                        Herrschaftsstallungen.}}\pend
           \pstart
           \noindent{}\centering{}\textcolor{gray}{\textbf{Vom Allerhöchsten Hofe und der hohen Aristokratie seit
                        vielen Jahren sehr bevorzugt.}}\pend
           \pstart
           \noindent{}\centering{}\textcolor{gray}{\textbf{Acht Jahre Sommeraufenthalt des Reichskanzlers Fürsten \textcolor{blue}{Bülow}{}\ledrightnote{\textcolor{blue}{Bernhard von Bülow}}.}}\pend
           \pstart
           \noindent{}\raggedleft{}\textcolor{gray}{\textbf{\textbf{\textcolor{blue}{Franz Panhans}{}\ledrightnote{\textcolor{blue}{Franz Panhans}}}, Besitzer und persönlicher Leiter.}}\pend
           \pstart
           \noindent{}\raggedleft{}\textcolor{gray}{\textbf{Semmering, am ..........}}\pend
           \pstart
           \noindent{}Ich bitte ſehr, es dem Herrn \uline{\textsc{D\textsuperscript{r}} Arthur Schnitzler} mitzuteilen, daſs ich noch nie eine ſo feine
                    Novelle geleſen habe wie: »\textcolor{green}{\textsc{Der Tod {\pb}des
                            Junggesellen}}{}\ledrightnote{\textcolor{green}{Der Tod des Junggesellen}}« in ſeinem \label{K_L02062_1v}\edtext{neuen Buche}{\lemma{\textnormal{\emph{neuen Buche}}}\Cendnote{\textnormal{\textcolor{blue}{Schnitzler} hatte am
                            6. 5. 1912 sein erstes Exemplar in der Hand. Nachdem er im
                            Mai keinen Aufenthalt am \textcolor{pink}{Semmering} im \emph{\textcolor{green}{Tagebuch}} erwähnt,
                        bietet sich nur die Reise 10.–11. 5. 1912 nach \textcolor{pink}{Triest} an, auf der er den \textcolor{pink}{Semmering} passiert und möglicherweise Zwischenstation
                        einlegte.}}}\label{K_L02062_1h}: »\textcolor{green}{\uline{\textsc{Masken und Wunder}}}{}\ledrightnote{\textcolor{green}{Masken und Wunder. Novellen}}«!\pend
           \pstart
           Auch bitte ich um ein Exemplar dieſes \textcolor{green}{Buches}{}\ledrightnote{→\textcolor{green}{Masken und Wunder. Novellen}} gratis.\pend
           \pstart
           Ihr{\\[\baselineskip]}\spacefill\mbox{Peter Altenberg}\pend
           \leftskip=0em{}\pstart
           \noindent{}\textcolor{pink}{Semmering, Hotel Panhans}{}\ledrightnote{\textcolor{pink}{Hotel Panhans}}\pend
           \endnumbering\briefempfaengerindex{Schnitzler, Arthur@\textsc{Schnitzler, Arthur}!zzzAltenberg, Peter@\emph{von Peter Altenberg}!1912-05-101@{{[}10.? 5. 1912{]}}|)be}\mylabel{h}  \normalsize

\doendnotes{C}
\bigskip
\vfill

\clearpage

\footnotesize

\lohead{\textsc{register}}

% Definiere theindex-Environment komplett neu ohne reledmac
\makeatletter
\renewenvironment{theindex}{%
  \section*{\indexname}%
  \setlength{\parindent}{0pt}%
  \setlength{\parskip}{0pt plus 0.3pt}%
  \let\item\@idxitem
}{%
  \clearpage
}
\makeatother

\IfFileExists{\jobname-pw.ind}{\input{\jobname-pw.ind}}{}

\end{document}

      