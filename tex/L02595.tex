%% latex-korrekturansicht-vorspann.tex
%% Vorspann für die Korrekturansicht.
%% Lädt die gemeinsame Datei latex-vorspann.tex mit gesetztem Schalter.

\newif\ifkorrekturansicht
\korrekturansichttrue

\input{../tex-inputs/latex-vorspann}


               \section[Marie Herzfeld an Arthur Schnitzler, 10. 3. 1931]{ Marie Herzfeld an Arthur Schnitzler, 10. 3. 1931}\nopagebreak\mylabel{v}\rehead{ }\normalsize\beginnumbering\briefempfaengerindex{Schnitzler, Arthur@\textsc{Schnitzler, Arthur}!zzzHerzfeld, Marie@\emph{von Marie Herzfeld}!1931-03-101@{10. 3. 1931}|(be} \toendnotes[C]{\smallbreak\pagebreak[2]} \Standort{DLA, A:Schnitzler, HS.1985.1.03436,6.}
\physDesc{Brief, 1 Blatt (Briefpapier mit Trauerrand), 2 Seiten
\newline{}Handschrift: schwarze Tinte, lateinische Kurrent
\newline{}Schnitzler: mit rotem Buntstift Vermerk »\textsc{Herzfeld.}«
                                 und »\textsc{(\textsc{\textcolor{blue}{Hofmsthl}}}« sowie drei Unterstreichungen }\toendnotes[C]{\smallbreak}\pstart
           \raggedleft{}{\pb}\textcolor{pink}{Wien III/\textsubscript{3}, Oetzeltg. 1 \textsuperscript{III}/\textsubscript{ii}}{}\ledrightnote{\textcolor{pink}{Ölzeltgasse}}{\\}den 10. März 1931\pend
           \pstart\center{}Sehr geehrter Herr Doktor!\pend\pstart
           Trotz des negativen Inhaltes Ihrer \label{K_L02595-1v}\edtext{Zeilen }{\lemma{\textnormal{\emph{Zeilen }}}\Cendnote{\textnormal{siehe Arthur Schnitzler an Marie Herzfeld, 7. 3. 1931}}}\label{K_L02595-1h}haben sie mich doch sehr erfreut. Mir war
               es, trotz der Maschinschrift, als hörte ich plötzlich Ihre Stimme, nur war sie
               tiefer und ernster geworden, im Lauf der Jahre, in denen man {\pb}allerlei durch- und mitgemacht hat.\pend
           \pstart
           Ich gehe leider gar nicht mehr ins Theater, – ich bin fast taub, – doch ich folge
               Ihrer Produktion für die Bühne, indem ich Ihre Stücke lese: sie verlieren dabei
               nichts. Mit Dank und den wärmsten Grüßen,\pend
           \pstart \spacefill\mbox{Marie Herzfeld}\pend{}\pstart
           \noindent{}\label{K_L02595-2v}\edtext{NB.}{\lemma{\textnormal{\emph{NB.}}}\Cendnote{\textnormal{Notabene, lateinisch: merke wohl}}}\label{K_L02595-2h} Ich schreibe an
                     \textcolor{blue}{Prof. Zimmer}{}\ledrightnote{\textcolor{blue}{Heinrich Zimmer}}, wegen des \label{K_L02595-3v}\edtext{\textcolor{green}{Ren.-Dramas}{}\ledrightnote{\textcolor{green}{Ascanio und Gioconda}}}{\lemma{\textnormal{\emph{Ren.-Dramas}}}\Cendnote{\textnormal{siehe Marie Herzfeld an Arthur Schnitzler, 5. 3. 1931, Arthur Schnitzler an Marie Herzfeld, 7. 3. 1931}}}\label{K_L02595-3h}; der wird mehr wissen!\pend
           \endnumbering\briefempfaengerindex{Schnitzler, Arthur@\textsc{Schnitzler, Arthur}!zzzHerzfeld, Marie@\emph{von Marie Herzfeld}!1931-03-101@{10. 3. 1931}|)be}\mylabel{h}  \normalsize

\doendnotes{C}
\bigskip
\vfill

\clearpage

\footnotesize

\lohead{\textsc{register}}

% Definiere theindex-Environment komplett neu ohne reledmac
\makeatletter
\renewenvironment{theindex}{%
  \section*{\indexname}%
  \setlength{\parindent}{0pt}%
  \setlength{\parskip}{0pt plus 0.3pt}%
  \let\item\@idxitem
}{%
  \clearpage
}
\makeatother

\IfFileExists{\jobname-pw.ind}{\input{\jobname-pw.ind}}{}

\end{document}

      