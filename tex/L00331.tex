%% latex-korrekturansicht-vorspann.tex
%% Vorspann für die Korrekturansicht.
%% Lädt die gemeinsame Datei latex-vorspann.tex mit gesetztem Schalter.

\newif\ifkorrekturansicht
\korrekturansichttrue

\input{../tex-inputs/latex-vorspann}


               \section[Georg Brandes an Arthur Schnitzler, 26. 5. 1894]{ Georg Brandes an Arthur Schnitzler, 26. 5. 1894}\nopagebreak\mylabel{v}\rehead{ }\normalsize\beginnumbering\briefempfaengerindex{Schnitzler, Arthur@\textsc{Schnitzler, Arthur}!zzzBrandes, Georg@\emph{von Georg Brandes}!1894-05-262@{26. 5. 1894}|(be} \toendnotes[C]{\smallbreak\pagebreak[2]} \Standort{CUL, Schnitzler, B 17.}
\physDesc{Brief, 1 Blatt, 1 Seite
\newline{}Handschrift: blaue Tinte, lateinische Kurrent
\newline{}Schnitzler: mit Bleistift beschriftet »\textsc{Brandes}« \newline{}Ordnung: von unbekannter Hand nummeriert:
                                    »=1?« }\buchAbdrucke{\weitereDrucke{Georg Brandes, Arthur Schnitzler: \emph{Ein Briefwechsel}. Hg. Kurt Bergel. Bern: \emph{Francke} 1956, S. 55.} }\toendnotes[C]{\smallbreak}\pstart
           \raggedleft{}{\pb}\textcolor{pink}{Kopenhagen}{}\ledrightnote{\textcolor{pink}{Kopenhagen}}{ }26. Mai 94\pend
           \pstart{}Hochgeehrter Herr\pend\pstart
           Zwei Mal schon haben Sie mich verpflichtet, das erste Mal durch Zusendung Ihres
                        \textcolor{green}{\uline{Anatol}}{}\ledrightnote{\textcolor{green}{Anatol}} und jetzt durch Ihr \textcolor{green}{\uline{Märchen}}{}\ledrightnote{\textcolor{green}{Das Märchen. Schauspiel in drei Aufzügen}}. Wenn ich nicht gedankt habe, so liegt es nur daran dass ich täglich
                    allzu viel Bücher erhalte um mich bedanken zu können.\pend
           \pstart
           Aber für das \textcolor{green}{Märchen}{}\ledrightnote{\textcolor{green}{Das Märchen. Schauspiel in drei Aufzügen}}{ }\uline{muss} ich Ihnen danken. Es ist eine so gute und
                    gediegene Arbeit, wie ein Kritiker sie selten empfängt. Sie haben hier eine viel
                    höhere Stufe erreicht als in Ihrem früheren \textcolor{green}{Buch}{}\ledrightnote{→\textcolor{green}{Anatol}}. Die Frauengestalten sind alle sehr fein und richtig gezeichnet
                    und die Handlung des Stücks ist gut und logisch geführt.\pend
           \pstart Hochachtungsvoll\hspace*{2em}Ihr\spacefill\mbox{Georg
                        Brandes.}\pend{}\endnumbering\briefempfaengerindex{Schnitzler, Arthur@\textsc{Schnitzler, Arthur}!zzzBrandes, Georg@\emph{von Georg Brandes}!1894-05-262@{26. 5. 1894}|)be}\mylabel{h}  \normalsize

\doendnotes{C}
\bigskip
\vfill

\clearpage

\footnotesize

\lohead{\textsc{register}}

% Definiere theindex-Environment komplett neu ohne reledmac
\makeatletter
\renewenvironment{theindex}{%
  \section*{\indexname}%
  \setlength{\parindent}{0pt}%
  \setlength{\parskip}{0pt plus 0.3pt}%
  \let\item\@idxitem
}{%
  \clearpage
}
\makeatother

\IfFileExists{\jobname-pw.ind}{\input{\jobname-pw.ind}}{}

\end{document}

      