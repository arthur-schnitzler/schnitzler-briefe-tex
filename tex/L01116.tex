%% latex-korrekturansicht-vorspann.tex
%% Vorspann für die Korrekturansicht.
%% Lädt die gemeinsame Datei latex-vorspann.tex mit gesetztem Schalter.

\newif\ifkorrekturansicht
\korrekturansichttrue

\input{../tex-inputs/latex-vorspann}


               \section[Georg Brandes an Arthur Schnitzler, 10. 5. {[}1901{]}]{ Georg Brandes an Arthur Schnitzler, 10. 5. {[}1901{]}}\nopagebreak\mylabel{v}\rehead{ }\normalsize\beginnumbering\briefempfaengerindex{Schnitzler, Arthur@\textsc{Schnitzler, Arthur}!zzzBrandes, Georg@\emph{von Georg Brandes}!1901-05-101@{10. 5. {[}1901{]}}|(be} \toendnotes[C]{\smallbreak\pagebreak[2]} \Standort{CUL, Schnitzler, B 17.}
\physDesc{Brief, 1 Blatt, 2 Seiten
\newline{}Handschrift: schwarze Tinte, lateinische Kurrent
\newline{}Schnitzler: mit Bleistift die Jahreszahl ergänzt: »901« \newline{}Ordnung: mit Bleistift von unbekannter Hand nummeriert:
                                        »21« }\buchAbdrucke{\weitereDrucke{Georg Brandes, Arthur Schnitzler: \emph{Ein Briefwechsel}. Hg. Kurt Bergel. Bern: \emph{Francke} 1956, S. 85.} }\toendnotes[C]{\smallbreak}\pstart
           \raggedleft{}{\pb}\textcolor{pink}{Schloss Strzebowitz}{}\ledrightnote{\textcolor{pink}{Schloss Strzebowitz}}{\\}\textcolor{pink}{Schlesien. Oesterreich}{}\ledrightnote{\textcolor{pink}{Schlesien}}{\\}10 Mai\pend
           \pstart
           Liebster! Ich habe Ihren Brief und ich habe den \textcolor{green}{Roman}{}\ledrightnote{→\textcolor{green}{Frau Bertha Garlan. Roman}} mit der grössten
                    Freude gelesen. Er ist so wahr und tief. Ein ganz klein wenig zu roh haben Sie
                    doch vielleicht den Virtuosen gemacht. Man hat den Eindruck, er habe eine
                    sinnliche Enttäuschung erfahren, die Dame hat ja freilich nicht vor der Umarmung
                    Toilette machen können. Wie es bei der \textcolor{blue}{Marni}{}\ledrightnote{\textcolor{blue}{Jeanne Marni}} heisst \label{K_L01116_1v}\edtext{\textcolor{green}{tub be or not tub be, that is
                        the question}{}\ledrightnote{→\textcolor{green}{Hamlet}}}{\lemma{\textnormal{\emph{tub … question}}}\Cendnote{\textnormal{nicht nachgewiesen}}}\label{K_L01116_1h}. Oder er
                    hat vielleicht, wie es geht, so viele Frauen an den Hals, dass er nicht mehr
                    verträgt. Jedenfalls {\pb}das \textcolor{green}{Buch}{}\ledrightnote{→\textcolor{green}{Frau Bertha Garlan. Roman}} ist gut. Die
                    Nebenhandlung, die Geschichte der schönen Frau, sehr fein geführt.\pend
           \pstart
           Ich glaube dass ich am 16\textsuperscript{sten} von hier über \textcolor{pink}{Wien}{}\ledrightnote{\textcolor{pink}{Wien}} nach \textcolor{pink}{Abbazia}{}\ledrightnote{\textcolor{pink}{Opatija}} reise.\hspace*{1.5em}Wenn Sie in \textcolor{pink}{Wien}{}\ledrightnote{\textcolor{pink}{Wien}} dann sind
                    und ein Paar Stunden für mich übrig haben, möchte ich schon Mittags um
                        3,48 nach \textcolor{pink}{Wien}{}\ledrightnote{\textcolor{pink}{Wien}} kommen und bis
                        8 Uhr Abends bleiben.\hspace*{1.5em}Sonst
                    reise ich durch.\pend
           \pstart
           Bitte, liebster Freund und Poet, um eine Zeile Antwort.\pend
           \pstart
           Ihr{\\[\baselineskip]}\spacefill\mbox{Georg Brandes}\pend
           \leftskip=0em{}\endnumbering\briefempfaengerindex{Schnitzler, Arthur@\textsc{Schnitzler, Arthur}!zzzBrandes, Georg@\emph{von Georg Brandes}!1901-05-101@{10. 5. {[}1901{]}}|)be}\mylabel{h}  \normalsize

\doendnotes{C}
\bigskip
\vfill

\clearpage

\footnotesize

\lohead{\textsc{register}}

% Definiere theindex-Environment komplett neu ohne reledmac
\makeatletter
\renewenvironment{theindex}{%
  \section*{\indexname}%
  \setlength{\parindent}{0pt}%
  \setlength{\parskip}{0pt plus 0.3pt}%
  \let\item\@idxitem
}{%
  \clearpage
}
\makeatother

\IfFileExists{\jobname-pw.ind}{\input{\jobname-pw.ind}}{}

\end{document}

      