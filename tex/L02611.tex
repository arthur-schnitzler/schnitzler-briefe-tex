%% latex-korrekturansicht-vorspann.tex
%% Vorspann für die Korrekturansicht.
%% Lädt die gemeinsame Datei latex-vorspann.tex mit gesetztem Schalter.

\newif\ifkorrekturansicht
\korrekturansichttrue

\input{../tex-inputs/latex-vorspann}


               \section[Paul Goldmann an Arthur Schnitzler, 28. 2. {[}1894{]}]{ Paul Goldmann an Arthur Schnitzler, 28. 2. {[}1894{]}}\nopagebreak\mylabel{v}\rehead{ }\normalsize\beginnumbering\briefempfaengerindex{Schnitzler, Arthur@\textsc{Schnitzler, Arthur}!zzzGoldmann, Paul@\emph{von Paul Goldmann}!1894-02-281@{28. 2. {[}1894{]}}|(be} \toendnotes[C]{\smallbreak\pagebreak[2]} \Standort{DLA, A:Schnitzler, HS.NZ85.1.3164.}
\physDesc{Brief, 1 Blatt, 3 Seiten
\newline{}Handschrift: schwarze Tinte, deutsche Kurrent
\newline{}Schnitzler: 1) mit Bleistift auf dem ersten Blatt die Jahreszahl »94« vermerkt 2) mit rotem Buntstift zwei Unterstreichungen}\toendnotes[C]{\smallbreak}\pstart
           \raggedleft{}{\pb}\textsc{\textcolor{pink}{Paris}{}\ledrightnote{\textcolor{pink}{Paris}}}, 28. Februar.\pend
           \pstart\center{}Mein lieber Arthur,\pend\pstart
           Anbei erhälſt Du den »\textsc{\textcolor{green}{Mercure de France}{}\ledrightnote{\textcolor{green}{Mercure de France}}}«, die bedeutendſte unter den Pariſer jungen \textsc{Revuen}.
                  \textsc{\textcolor{blue}{Henri Albert}{}\ledrightnote{\textcolor{blue}{Henri Albert}}}, von dem ich Dir \label{K_L02611-1v}\edtext{neulich}{\lemma{\textnormal{\emph{neulich}}}\Cendnote{\textnormal{siehe Paul Goldmann an Arthur Schnitzler, 17. 2. [1894]}}}\label{K_L02611-1h} ſchrieb, hat \label{K_L02611-2v}\edtext{Dir und \textsc{\textcolor{blue}{Loris}{}\ledrightnote{\textcolor{blue}{Hugo von Hofmannsthal}}} darin ein paar \textcolor{green}{Worte}{}\ledrightnote{→\textcolor{green}{Le nouvel almanach de M. Bierbaum}}
                  gewidmet}{\lemma{\textnormal{\emph{Dir … gewidmet}}}\Cendnote{\textnormal{\textcolor{blue}{Henri Albert}: \emph{\textcolor{green}{Le nouvel almanach de M. Bierbaum}}. In: \emph{\textcolor{green}{Mercure de France}}, Jg. 10, Nr. 51,
                        März 1894, S. 243–246, hier: S. 244–245.}}}\label{K_L02611-2h}
               (S. 244). Noch ſteht mein \strikeout{Urt} Urtheil nicht ganz
               feſt, aber ich glaube, der Mann gehört zu uns.\pend
           \pstart
           Wenn Du willſt, ſo {\pb}\label{K_L02611-33v}\edtext{ſchreib’ ihm direct}{\lemma{\textnormal{\emph{ſchreib’ ihm direct}}}\Cendnote{\textnormal{\textcolor{blue}{Schnitzler} paraphrasiert diese Stelle in
                  seinem Brief an \textcolor{blue}{Hofmannsthal}, [9. 3. 1894]}}}\label{K_L02611-33h} ein paar \label{K_L02611-3v}\edtext{Worte}{\lemma{\textnormal{\emph{Worte}}}\Cendnote{\textnormal{\textcolor{blue}{Schnitzler} dürfte \textcolor{blue}{Albert} geschrieben haben, denn diese Stelle in dessen
                  Antwortschreiben vom 9. 4. 1894 scheint hierauf Bezug zu nehmen:
                     »Meine kleine \textcolor{green}{Besprechung} wurde abgefasst, als ich unseren lieben Freund \textcolor{blue}{Paul Goldmann} erst sehr oberflächlich
                     kannte. Sie blieb zwei Monate lang auf der Redaction liegen – Diese
                     Freundschaft hat aber in keiner Weise mein Urtheil beeinflusst.« (\emph{DLA}, HS.1985.1.2331,1.)}}}\label{K_L02611-3h}. Das wird ihn
               freuen (\textsc{M. \textcolor{blue}{Henri Albert}{}\ledrightnote{\textcolor{blue}{Henri Albert}}}, \textsc{\textcolor{pink}{25. Rue Jacob, Paris}{}\ledrightnote{\textcolor{pink}{rue Jacob}}}.). Natürlich deutſch. Auch \label{K_L02611-4v}\edtext{»\textsc{\begin{otherlanguage}{french}le génial\end{otherlanguage}{ }\textcolor{blue}{Loris}{}\ledrightnote{\textcolor{blue}{Hugo von Hofmannsthal}}}«}{\lemma{\textnormal{\emph{»le génial Loris«}}}\Cendnote{\textnormal{Zitat aus der angeführten \emph{\textcolor{green}{Besprechung}}{ }\textcolor{blue}{Albert}s, S. 245.}}}\label{K_L02611-4h} ſoll ihm
               ſchreiben und vielleicht für mich einen Gruß zufügen, damit ich wieder einmal
               wenigſtens etwas Indirectes von ihm höre. Willſt Du glauben, daß ich nichts weiß, was
               er ſchreibt? Daß er mir nicht einmal »\label{K_L02611-55v}\edtext{\textcolor{green}{Der Thor und der Tod}{}\ledrightnote{\textcolor{green}{Der Thor und der Tod}}}{\lemma{\textnormal{\emph{Der Thor und der Tod}}}\Cendnote{\textnormal{\emph{\textcolor{green}{Der Thor und der Tod}} ist im \emph{\textcolor{green}{Modernen Musen-Almanach auf das Jahr 1894}} enthalten, den
                     \textcolor{blue}{Henri Albert} bespricht.}}}\label{K_L02611-55h}« geſchickt
               hat? Ich kenne alles das nur aus Deinen Briefen. Und was das {\pb}heißt, eine Sache aus Deinen Briefen kennen, darüber
               machſt Du Dir wohl ſelbſt keine Illuſionen.\pend
           \pstart
           Schreibſt Du mir bald wieder einmal?\pend
           \pstart
           In Treue {\\[\baselineskip]}Dein{\\[\baselineskip]}\spacefill\mbox{Paul Goldmann}\pend
           \leftskip=0em{}\endnumbering\briefempfaengerindex{Schnitzler, Arthur@\textsc{Schnitzler, Arthur}!zzzGoldmann, Paul@\emph{von Paul Goldmann}!1894-02-281@{28. 2. {[}1894{]}}|)be}\mylabel{h}  \normalsize

\doendnotes{C}
\bigskip
\vfill

\clearpage

\footnotesize

\lohead{\textsc{register}}

% Definiere theindex-Environment komplett neu ohne reledmac
\makeatletter
\renewenvironment{theindex}{%
  \section*{\indexname}%
  \setlength{\parindent}{0pt}%
  \setlength{\parskip}{0pt plus 0.3pt}%
  \let\item\@idxitem
}{%
  \clearpage
}
\makeatother

\IfFileExists{\jobname-pw.ind}{\input{\jobname-pw.ind}}{}

\end{document}

      