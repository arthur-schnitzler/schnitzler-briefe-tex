%% latex-korrekturansicht-vorspann.tex
%% Vorspann für die Korrekturansicht.
%% Lädt die gemeinsame Datei latex-vorspann.tex mit gesetztem Schalter.

\newif\ifkorrekturansicht
\korrekturansichttrue

\input{../tex-inputs/latex-vorspann}


               \section[Hugo von Hofmannsthal an Arthur Schnitzler, 2. 5. {[}1897{]}]{ Hugo von Hofmannsthal an Arthur Schnitzler, 2. 5. {[}1897{]}}\nopagebreak\mylabel{v}\rehead{ }\normalsize\beginnumbering\briefempfaengerindex{Schnitzler, Arthur@\textsc{Schnitzler, Arthur}!zzzHofmannsthal, Hugo von@\emph{von Hugo von Hofmannsthal}!1897-05-021@{2. 5. {[}1897{]}}|(be} \toendnotes[C]{\smallbreak\pagebreak[2]} \Standort{CUL, Schnitzler, B 43.}
\physDesc{Brief, 1 Blatt, 4 Seiten
\newline{}Handschrift: schwarze Tinte, deutsche Kurrent
\newline{}Schnitzler: mit Bleistift die Jahreszahl ergänzt: »97« \newline{}Ordnung: mit Bleistift von unbekannter Hand nummeriert:
                                        »89« }\buchAbdrucke{\weitereDrucke{Hugo von Hofmannsthal, Arthur Schnitzler: \emph{Briefwechsel}. Hg. Therese Nickl und Heinrich Schnitzler. Frankfurt am Main: \emph{S. Fischer} 1964, S. 83–84.} }\toendnotes[C]{\smallbreak}\pstart
           \noindent{}{\pb}\textcolor{gray}{\textbf{\label{T_L00673-1v}\edtext{hvH}{\lemma{\textnormal{\emph{hvH}}}\Cendnote{\textnormal{gedrucktes Monogramm mit Krone in roter
                            Farbe}}}\label{T_L00673-1h}}}\pend
           \pstart
           \raggedleft{}Sonntag 2\textsuperscript{ten} Mai\pend
           \pstart{}lieber Arthur,\pend\pstart
           wie komiſch man eigentlich iſt: es hat mich einen Moment ganz ſtark geärgert zu
                    hören, daſs Sie wieder gemiſchtes Hausbrot eſſen. Ich hätte ſo gern gehört, daſs
                    Sie auf einmal etwas ganz anderes eſſen! Aber das iſt natürlich eine
                    Kinderei.\pend
           \pstart
           Hier iſt es jetzt ſehr ſchön. (Nur gerade heute regnet es zufällig.) Der Frühling
                    war {\pb}durch eine lange kühle
                    Zeit zurückgehalten und dann war er auf einmal da und ſo warm und ſo farbig,
                    daſs die Farben der Blumenbeete, der Baumwipfel und des Himmels mit ihren
                    Contouren auszutreten und die Luft zu überſchwemmen ſchienen. Das Radfahren
                    macht mir eine große Freude: es iſt wunderſchön, ein biſſel ermüdet und erhitzt
                    ſich irgendwo ſtill hinzuſetzen {\pb}und über die Sträuche, die
                    Wieſen und die Hügel hinzuſchauen, und abends iſt es ſogar wunderſchön, in den
                    Straßen der Vorſtädte zu fahren.\pend
           \pstart
           Schreiben Sie mir doch ein paar ſchöne kleine Ausflüge, an die \substVorne{}\textsuperscript{s}\substDazwischen{}S\substHinten{}ie ſich erinnern. Ich war erſt in \textcolor{pink}{Weidling
                        am Bach}{}\ledrightnote{\textcolor{pink}{Weidlingbach}}, und in \textcolor{pink}{Heiligenkreuz}{}\ledrightnote{\textcolor{pink}{Heiligenkreuz}}.\pend
           \pstart
           Ihre Bemerkungen über das \textcolor{pink}{franzöſiſche}{}\ledrightnote{\textcolor{pink}{Frankreich}} Theater
                    verſtehe ich ſehr gut, weil jetzt gerade {\pb}eine \textcolor{pink}{franzöſiſche}{}\ledrightnote{\textcolor{pink}{Frankreich}} Truppe im \textcolor{pink}{Carltheater}{}\ledrightnote{\textcolor{pink}{Carl-Theater}} war und lauter ſolche \textsc{Vie-\textcolor{pink}{paris}{}\ledrightnote{\textcolor{pink}{Paris}}ienne}{ }Stücke geſpielt hat. Vergeſſen Sie doch nicht,
                    die \textcolor{blue}{Delna}{}\ledrightnote{\textcolor{blue}{Marie Delna}} als \textcolor{green}{Orpheus}{}\ledrightnote{→\textcolor{green}{Orpheus und Eurydike}} zu hören.\pend
           \pstart
           Ich arbeite noch immer nichts, lerne nur fleißig an meinen romaniſchen Texten.
                    Aber ich fühle mich doch nun recht viel freier und weniger verworren und bin
                    viel zufriedener.\pend
           \pstart
           Ich freue mich recht auf Ihre Rückkehr. »\textcolor{green}{Götterliebling}{}\ledrightnote{\textcolor{green}{Der Tod Georgs}}« dürfte bald fertig ſein, auch das \textcolor{green}{Stück}{}\ledrightnote{→\textcolor{green}{Agnes Jordan. Schauspiel in fünf Akten}} vom \textcolor{blue}{Hirſchfeld}{}\ledrightnote{\textcolor{blue}{Georg Hirschfeld}}.\pend
           \pstart Ihr\spacefill\mbox{Hugo.}\pend{}\endnumbering\briefempfaengerindex{Schnitzler, Arthur@\textsc{Schnitzler, Arthur}!zzzHofmannsthal, Hugo von@\emph{von Hugo von Hofmannsthal}!1897-05-021@{2. 5. {[}1897{]}}|)be}\mylabel{h}  \normalsize

\doendnotes{C}
\bigskip
\vfill

\clearpage

\footnotesize

\lohead{\textsc{register}}

% Definiere theindex-Environment komplett neu ohne reledmac
\makeatletter
\renewenvironment{theindex}{%
  \section*{\indexname}%
  \setlength{\parindent}{0pt}%
  \setlength{\parskip}{0pt plus 0.3pt}%
  \let\item\@idxitem
}{%
  \clearpage
}
\makeatother

\IfFileExists{\jobname-pw.ind}{\input{\jobname-pw.ind}}{}

\end{document}

      