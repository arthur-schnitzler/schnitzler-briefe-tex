%% latex-korrekturansicht-vorspann.tex
%% Vorspann für die Korrekturansicht.
%% Lädt die gemeinsame Datei latex-vorspann.tex mit gesetztem Schalter.

\newif\ifkorrekturansicht
\korrekturansichttrue

\input{../tex-inputs/latex-vorspann}


               \section[Hugo von Hofmannsthal an Arthur Schnitzler, 3. 12. 1898]{ Hugo von Hofmannsthal an Arthur Schnitzler, 3. 12. 1898}\nopagebreak\mylabel{v}\rehead{ }\normalsize\beginnumbering\briefempfaengerindex{Schnitzler, Arthur@\textsc{Schnitzler, Arthur}!zzzHofmannsthal, Hugo von@\emph{von Hugo von Hofmannsthal}!1898-12-031@{3. 12. 1898}|(be} \toendnotes[C]{\smallbreak\pagebreak[2]} \Standort{CUL, Schnitzler, B 43.}
\physDesc{Brief, 1 Blatt, 4 Seiten
\newline{}Handschrift: schwarze Tinte, deutsche Kurrent\newline{}Ordnung: mit Bleistift von unbekannter Hand nummeriert: »\strikeout{131}
                                    128« }\buchAbdrucke{\weitereDrucke{Hugo von Hofmannsthal, Arthur Schnitzler: \emph{Briefwechsel}. Hg. Therese Nickl und Heinrich Schnitzler. Frankfurt am Main: \emph{S. Fischer} 1964, S. 115.} }\pstart
           \raggedleft{}{\pb}3. XII. 98.\pend
           \pstart{}mein lieber Arthur\pend\pstart
           ich bitte Sie vielmals um eine Gefälligkeit, nämlich daſs Sie Herrn \textcolor{blue}{Otto Eiſenſchitz}{}\ledrightnote{\textcolor{blue}{Otto Eisenschitz}}, den Sie ja perſönlich kennen,
                    einen Brief ſchreiben, oder daſs Sie ihm dieſen Brief hier ſchicken.\pend
           \pstart
           Herr \textcolor{blue}{\textsc{Lauria}}{}\ledrightnote{\textcolor{blue}{Amilcare Lauria}} in \textcolor{pink}{\textsc{Rom}}{}\ledrightnote{\textcolor{pink}{Rom}}, Redacteur der \textcolor{brown}{\textsc{Fanfulla}}{}\ledrightnote{\textcolor{brown}{Fanfulla della domenica}}, hat ſich an mich um \textsc{Intervention}{ }{\pb}gewandt, weil Herr \textcolor{blue}{Eiſenſchitz}{}\ledrightnote{\textcolor{blue}{Otto Eisenschitz}} ein einactiges Manuſcript von ihm
                        »\textcolor{green}{\textsc{ein Epilog}}{}\ledrightnote{\textcolor{green}{Ein Epilog}}« zum Überſetzen und zum Vertrieb bei den Bühnen übernommen hat und
                    Herr \textcolor{blue}{\textsc{Lauria}}{}\ledrightnote{\textcolor{blue}{Amilcare Lauria}} nun trotz mehrfacher Briefe keine Auskunft über den Verlauf dieſer
                    Sache bekommen kann, ja nicht einmal {\pb}weiß, ob das Stück bis jetzt
                        \introOben{}von Herrn \textcolor{blue}{Eiſenſchitz}{}\ledrightnote{\textcolor{blue}{Otto Eisenschitz}}\introOben{} ins Deutſche überſetzt wurde.\pend
           \pstart
           Wahrſcheinlich liegt hier ein Miſsverſtändnis vor und Herr \textcolor{blue}{Eiſenſchitz}{}\ledrightnote{\textcolor{blue}{Otto Eisenschitz}} wird wohl ſo freundlich ſein, an Sie
                    eine aufklärende Zeile zu richten. Übrigens iſt Herr \textcolor{blue}{\textsc{Lauria}}{}\ledrightnote{\textcolor{blue}{Amilcare Lauria}} ein {\pb}Autor, von
                    dem ich viel Gutes gehört habe.\pend
           \pstart
           Herzlich Ihr{\\[\baselineskip]}\spacefill\mbox{Hofmannsthal}\pend
           \leftskip=0em{}\endnumbering\briefempfaengerindex{Schnitzler, Arthur@\textsc{Schnitzler, Arthur}!zzzHofmannsthal, Hugo von@\emph{von Hugo von Hofmannsthal}!1898-12-031@{3. 12. 1898}|)be}\mylabel{h}  \normalsize

\doendnotes{C}
\bigskip
\vfill

\clearpage

\footnotesize

\lohead{\textsc{register}}

% Definiere theindex-Environment komplett neu ohne reledmac
\makeatletter
\renewenvironment{theindex}{%
  \section*{\indexname}%
  \setlength{\parindent}{0pt}%
  \setlength{\parskip}{0pt plus 0.3pt}%
  \let\item\@idxitem
}{%
  \clearpage
}
\makeatother

\IfFileExists{\jobname-pw.ind}{\input{\jobname-pw.ind}}{}

\end{document}

      