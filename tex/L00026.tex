%% latex-korrekturansicht-vorspann.tex
%% Vorspann für die Korrekturansicht.
%% Lädt die gemeinsame Datei latex-vorspann.tex mit gesetztem Schalter.

\newif\ifkorrekturansicht
\korrekturansichttrue

\input{../tex-inputs/latex-vorspann}


               \section[Hugo von Hofmannsthal an Arthur Schnitzler, {[}Anfang August{]} 1891]{ Hugo von Hofmannsthal an Arthur Schnitzler, {[}Anfang August{]}
               1891}\nopagebreak\mylabel{v}\rehead{ }\normalsize\beginnumbering\briefempfaengerindex{Schnitzler, Arthur@\textsc{Schnitzler, Arthur}!zzzHofmannsthal, Hugo von@\emph{von Hugo von Hofmannsthal}!1891-08-011@{{[}Anfang August{]} 1891}|(be} \toendnotes[C]{\smallbreak\pagebreak[2]} \Standort{CUL, Schnitzler, B 43.}
\physDesc{Brief, 1 Blatt, 4 Seiten
\newline{}Handschrift: schwarze Tinte, deutsche Kurrent
\newline{}Schnitzler: mit Bleistift datiert: »Anf Jul 91« \newline{}Ordnung: mit Bleistift von unbekannter Hand nummeriert:
                                 »3« }\buchAbdrucke{\weitereDrucke{1) Hugo von Hofmannsthal: \emph{Briefe. 1890–1901}. Berlin: \emph{S. Fischer} 1935, S. 23–24.} \weitereDrucke{2) Hugo von Hofmannsthal, Arthur Schnitzler: \emph{Briefwechsel}. Hg. Therese Nickl und Heinrich Schnitzler. Frankfurt am Main: \emph{S. Fischer} 1964, S. 10–11.} }\toendnotes[C]{\smallbreak}\pstart
           \noindent{}{\pb}Ich danke Ihnen wirklich für
               Ihren Brief. Sie müſſen ihn ſehr ſchwer geſchrieben haben. Ich habe das damals
               empfunden und empfinde es jetzt wieder.\pend
           \pstart
           \label{K_L00026_1v}\edtext{Damals}{\lemma{\textnormal{\emph{Damals}}}\Cendnote{\textnormal{zwischen dem 22. und 31. 7. 1891, vgl. \textcolor{blue}{Hugo von Hofmannsthal}: \emph{Aufzeichnungen}. Hg. Rudolf Hirsch † und Ellen Ritter † in
                     Zusammenarbeit mit Konrad Heumann und Peter Michael Braunwarth. Frankfurt am
                     Main: \emph{\textcolor{brown}{S. Fischer}}{ }2013, S. 128 (\emph{Sämtliche Werke},
                     XXXIX).}}}\label{K_L00026_1h} – um mich, als ich ihn las, ſtanden \textcolor{blue}{\textsc{Robert}}{}\ledrightnote{\textcolor{blue}{Robert Hirschfeld}} und \textcolor{blue}{\textsc{Olga} Hirſchfeld}{}\ledrightnote{\textcolor{blue}{Olga Hirschfeld}}, \textcolor{blue}{Schwarzkopf}{}\ledrightnote{\textcolor{blue}{Gustav Schwarzkopf}} und \textcolor{blue}{\textsc{Boris Fan-Junk}}{}\ledrightnote{\textcolor{blue}{Boris Van-Jung}} – berührte er mich wie eine Erinnerung an Längſtvergeſſenes,
               Unerreichbar-fernes. Sie fragten nach meinen Arbeiten. Sie gedachten gemeinſamer
               Pläne. Um mich und in mir waren neue Dinge, Gleiten, Plätſchern, Rieſeln, Auflöſung,
               vages Verſchwimmen. Ich kann nicht arbeiten. Heute ſo wenig als damals. Noch weniger
                  {\pb}vielleicht. Ich gleite, ich
               treibe. Kein Gedanke cryſtalliſiert ſich und es wird kein Vers. Ich kann nicht weiter
               denken als Stunden.\pend
           \pstart
           Aber mir iſt wohl. Anders wohl, neu wohl, wechſelnd wohl. Ich fühle mich wachſen.
               Wollt ich mich zwingen, müſst ich verzweifeln\strikeout{d},
               abwartend ſehe ich mir fluthen zu und empfinde ein glückliches Michbeſcheiden, das
               gute Schweſtergefühl zur Reſignation. Wäre nur mehr Sonne. So aber bin ich
               verſchnupft und krank möcht ich nicht werden, denn ich kann jetzt das Alleinſein
               nicht brauchen. Wenn Sie vielleicht in der \textcolor{green}{Kunſtchronik}{}\ledrightnote{\textcolor{green}{Allgemeine Kunst-Chronik}} meinem \label{K_L00026_2v}\edtext{\textcolor{green}{\textcolor{pink}{Salzburg}{}\ledrightnote{\textcolor{pink}{Salzburg}}erbericht}{}\ledrightnote{→\textcolor{green}{Die Mozart-Centenarfeier in Salzburg}}}{\lemma{\textnormal{\emph{Salzburgerbericht}}}\Cendnote{\textnormal{\textcolor{blue}{Loris}: \emph{\textcolor{green}{Die
                        Mozart-Centenarfeier in Salzburg}}. In: \emph{\textcolor{green}{Allgemeine Kunst-Chronik}}, Bd. 15, Nr. 16, 1. August-Heft,
                        1. 8. 1891, S. 423–433.}}}\label{K_L00026_2h}{ }\label{K_L00026_3v}\edtext{begegnen}{\lemma{\textnormal{\emph{begegnen}}}\Cendnote{\textnormal{Nachdem die \emph{\textcolor{brown}{\textcolor{blue}{Mozart}-Zentenarfeier}} vom
                     14.–17. 7. 1891 in \textcolor{pink}{Salzburg}{ }stattfand, ist die Datierung von Schnitzler mit »Anf Jul 91« auszuschließen. Wahrscheinlicher antwortet der Brief auf Schnitzlers Schreiben
                  vom 27. 7. 1891. Das Erscheinen des \textcolor{green}{Artikels} begrenzt
                  die Datierung nach hinten auf Anfang August.}}}\label{K_L00026_3h}, ſo laſſen Sie ſich von mir {\pb}ein paar Vorworte ſagen. Ich habe
               dort in 4 Tagen und 2 Nächten die concentrierteſte Menge von Eindrücken
               zuſammengetrunken, die mein Nervenſyſtem überhaupt vorläufig erträgt. Den Bericht
               habe ich im vollſtändigen Halbſchlaf geſchrieben in dem ſeltſamen Zuſtand, wo das
               Gehirn loſe Bilder, Geſprächstheile der letzten Nacht mit ſchmerzender Deutlichkeit
               bis zum Ekel reproduciert. Wenn der Bericht überhaupt deutſch iſt (ich habe ihn noch
               nicht bekommen) dann ſchläft in mir ein unbewuſster Reporter, \label{K_L00026_4v}\edtext{\textsc{qui parfois se réveille}}{\lemma{\textnormal{\emph{qui parfois se réveille}}}\Cendnote{\textnormal{französisch: der gelegentlich erwacht; Zitat in
                  der Gestalt nicht nachweisbar}}}\label{K_L00026_4h} wie \textcolor{blue}{\textsc{Ste. Beuve}}{}\ledrightnote{\textcolor{blue}{Charles Augustin de Sainte-Beuve}}{ }ſagt. D\textsuperscript{r} \textcolor{blue}{\textsc{Hoffmann}}{}\ledrightnote{\textcolor{blue}{Richard Beer-Hofmann}} hat mir auf einen 4 Seiten langen Brief nach \textcolor{pink}{Wien}{}\ledrightnote{\textcolor{pink}{Wien}} nicht geantwortet; ich habe ihm nach {\pb}\textcolor{pink}{\textsc{Markt-Aussee}}{}\ledrightnote{\textcolor{pink}{Bad Aussee}} (??) geſchrieben er ſoll doch zum Teufel hieher kommen. Warum kommt er denn
               nicht?!!! Ich arbeite \uline{garnichts} und hoffe daß die
               Comités der \textcolor{brown}{Freien Bühne}{}\ledrightnote{\textcolor{brown}{»Freie Bühne« Verein für moderne Literatur}} das Gegentheil thuen.\pend
           \pstart
           Während der Eiſenbahnfahrt nach \textcolor{pink}{Wien}{}\ledrightnote{\textcolor{pink}{Wien}}
                  (15 September) ſchreibe ich\pend
           \pstart
           1.) die letzte Scene von »\textcolor{green}{Geſtern}{}\ledrightnote{\textcolor{green}{Gestern. Dramatische Studie in einem Akt in Versen}}«\pend
           \pstart
           2.) \label{K_L00026_5v}\edtext{\textcolor{green}{\textcolor{blue}{\textsc{Maurice Barrès}}{}\ledrightnote{\textcolor{blue}{Maurice Barrès}}}{}\ledrightnote{→\textcolor{green}{Maurice Barrès}}}{\lemma{\textnormal{\emph{Maurice Barrès}}}\Cendnote{\textnormal{\textcolor{blue}{Loris}: \emph{\textcolor{green}{Maurice
                        Barrès}}. In: \emph{\textcolor{green}{Moderne Rundschau}}, Bd. 4,
                     H. 1, 1. 10. 1891, S. 15–18.}}}\label{K_L00026_5h}\textsc{, eine Studie}\pend
           \pstart
           3.) \textsc{eine psychologische \textcolor{green}{Novelle}{}\ledrightnote{→\textcolor{green}{Age of Innocence}} aus einem 12jährigen Kinderkopf}\pend
           \pstart
           4.) \textsc{\textcolor{blue}{Conway}{}\ledrightnote{\textcolor{blue}{Hugh Conway}}, der Novellist der Telepathie}\pend
           \pstart
           5.) \textsc{das grosse \textcolor{green}{Buch}{}\ledrightnote{→\textcolor{green}{Englisches Leben}} von }\label{K_L00026_6v}\edtext{\textsc{1891 in \textcolor{pink}{England}{}\ledrightnote{\textcolor{pink}{England}}}}{\lemma{\textnormal{\emph{1891 in England}}}\Cendnote{\textnormal{\textcolor{blue}{Loris}: \emph{\textcolor{green}{Englisches Leben}}. In: \emph{\textcolor{green}{Moderne
                        Rundschau}}, Bd. 4, H. 5, 1. 12. 1891,
                  S. 174–177.}}}\label{K_L00026_6h}.\pend
           \pstart
           \centering{}\textsc{Telle est la vie!}\pend
           \pstart \spacefill\mbox{Loris.}\pend{}\endnumbering\briefempfaengerindex{Schnitzler, Arthur@\textsc{Schnitzler, Arthur}!zzzHofmannsthal, Hugo von@\emph{von Hugo von Hofmannsthal}!1891-08-011@{{[}Anfang August{]} 1891}|)be}\mylabel{h}  \normalsize

\doendnotes{C}
\bigskip
\vfill

\clearpage

\footnotesize

\lohead{\textsc{register}}

% Definiere theindex-Environment komplett neu ohne reledmac
\makeatletter
\renewenvironment{theindex}{%
  \section*{\indexname}%
  \setlength{\parindent}{0pt}%
  \setlength{\parskip}{0pt plus 0.3pt}%
  \let\item\@idxitem
}{%
  \clearpage
}
\makeatother

\IfFileExists{\jobname-pw.ind}{\input{\jobname-pw.ind}}{}

\end{document}

      