%% latex-korrekturansicht-vorspann.tex
%% Vorspann für die Korrekturansicht.
%% Lädt die gemeinsame Datei latex-vorspann.tex mit gesetztem Schalter.

\newif\ifkorrekturansicht
\korrekturansichttrue

\input{../tex-inputs/latex-vorspann}


               \section[Arthur Schnitzler an Hugo von Hofmannsthal, 26. 6. 1903]{ Arthur Schnitzler an Hugo von Hofmannsthal, 26. 6. 1903}\nopagebreak\mylabel{v}\rehead{ }\normalsize\beginnumbering\briefempfaengerindex{Hofmannsthal, Hugo von@\textsc{Hofmannsthal, Hugo von}!zzzSchnitzler, Arthur@\emph{von Arthur Schnitzler}!1903-06-261@{26. 6. 1903}|(be} \toendnotes[C]{\smallbreak\pagebreak[2]} \Standort{FDH, Hs-30885,103.}
\physDesc{Brief, 2 Blätter, 8 Seiten
\newline{}Handschrift: schwarze Tinte, deutsche Kurrent\newline{}Ordnung: mit Bleistift von unbekannter Hand das zweite Blatt datiert: »26/6 903« }\buchAbdrucke{\weitereDrucke{1) Hugo von Hofmannsthal, Arthur Schnitzler: \emph{Briefwechsel}. Hg. Therese Nickl und Heinrich Schnitzler. Frankfurt am Main: \emph{S. Fischer} 1964, S. 170–172.} \weitereDrucke{2) Arthur Schnitzler: \emph{Briefe 1875–1912}. Hg. Therese Nickl und Heinrich Schnitzler. Frankfurt am Main: \emph{S. Fischer} 1981, S. 463–464.} \weitereDrucke{3) Hermann Bahr, Arthur Schnitzler: \emph{Briefwechsel, Aufzeichnungen, Dokumente
                                (1891–1931)}. Hg. Kurt Ifkovits und Martin Anton Müller. Göttingen: \emph{Wallstein} 2018, S. 267.} }\toendnotes[C]{\smallbreak}\pstart
           \raggedleft{}{\pb}\textcolor{pink}{Wien}{}\ledrightnote{\textcolor{pink}{Wien}}, 26. 6. 903\pend
           \pstart
           mein lieber Hugo, aus Ihrem Brief muſs ich entnehmen, daſs
                    unſre Karten von der Reiſe gar nicht zu Ihnen gelangt sind. Ich habe Ihnen aus
                        \textcolor{pink}{Venedig}{}\ledrightnote{\textcolor{pink}{Venedig}} (auch \textcolor{blue}{Hans}{}\ledrightnote{\textcolor{blue}{Hans Bernhard Schlesinger}} war auf dieſer Karte unterſchrieben) und aus \textcolor{pink}{Lugano}{}\ledrightnote{\textcolor{pink}{Lugano}} eine (ſogar \textsc{versificirte}) Nachricht geſandt. In \textcolor{pink}{Lugano}{}\ledrightnote{\textcolor{pink}{Lugano}} haben wir im \textsc{\textcolor{pink}{H. d. parc}{}\ledrightnote{\textcolor{pink}{Hôtel du Parc}}} gewohnt, und die liebenswürdige verheiratete \textcolor{blue}{Tochter}{}\ledrightnote{→\textcolor{blue}{Bertha Ober}} der Madame \textsc{\textcolor{blue}{Bèha}{}\ledrightnote{\textcolor{blue}{Elisa Béha}}} zeigte uns die »Stätte«, wo {\pb}Sie zu ſchreiben
                    pflegten. Was war es nur, das Sie damals arbeiteten? Vom Wetter waren wir nicht
                    ſehr begünſtigt; auf dem \textsc{\textcolor{pink}{Generoso}{}\ledrightnote{\textcolor{pink}{Monte Generoso}}} Nebel, Gewitter; in \textsc{\textcolor{pink}{Varese}{}\ledrightnote{\textcolor{pink}{Varese}}} ein Platzregen, daſs wir nicht \introOben{}bis\introOben{} zum \textsc{\textcolor{pink}{Grd Hotel}{}\ledrightnote{\textcolor{pink}{Grand Hotel Varese}}} gelangten u lieber gleich zurück fuhren. Die andern Seen fielen ſozuſagen
                    ins Waſſer, was ſie doch gar nicht mehr notwendig haben. Vor \textcolor{pink}{Lugano}{}\ledrightnote{\textcolor{pink}{Lugano}}: \textcolor{pink}{Venedig}{}\ledrightnote{\textcolor{pink}{Venedig}} (\textcolor{blue}{Hans}{}\ledrightnote{\textcolor{blue}{Hans Bernhard Schlesinger}} zeigte uns einige {\pb}palazzi, die wir ſonſt gewiſs nicht geſehen
                    hätten), Segelfahrt nach \textsc{\textcolor{pink}{Torcello}{}\ledrightnote{\textcolor{pink}{Torcello}}} (wenn Sie es nicht kennen, verſäumen Sie’s nicht bei nächſter \textcolor{pink}{Venezianer}{}\ledrightnote{\textcolor{pink}{Ponte di Rialto}} Gelegenheit) – \textsc{\textcolor{pink}{Padua}{}\ledrightnote{\textcolor{pink}{Padua}}}, \textsc{\textcolor{pink}{Vicenza}{}\ledrightnote{\textcolor{pink}{Vicenza}}}, \textsc{\textcolor{pink}{Verona}{}\ledrightnote{\textcolor{pink}{Verona}}}, \textsc{\textcolor{pink}{Mailand}{}\ledrightnote{\textcolor{pink}{Mailand}}}. \textcolor{blue}{Luini}{}\ledrightnote{\textcolor{blue}{Bernardino Luini}}, an dem ich (rein körperlich
                    gemeint) vor Jahren vorbeigegangen war, ging mir wundervoll auf. –\pend
           \pstart
           Von »geordneter« Arbeit wäre nichts mitzutheilen. Zumeiſt beſchäftigte mich das
                    ſonderbare, {\pb}oft begonnene, einige Mal beendete,
                    jedes Mal hingeworfene \textcolor{green}{Junggeſellen\textcolor{gray}{-}Egoiſtenſtück}{}\ledrightnote{→\textcolor{green}{Der einsame Weg. Schauspiel in fünf Akten}{\newline}→\textcolor{green}{Professor Bernhardi. Komödie in fünf Akten}}; Sie wiſſen, daſs es
                    zuletzt als Misgeburt zur Welt kam, ſiameſiſch gezwillingt. Nun ſcheint der
                    operative Eingriff, der mit Vorſicht unternommen werden mußte, gelungen – d. h.
                    beide Geſchöpfe leben, das eine ſchwächlich, das andre mit höherer Vitalkraft
                    begnadet, {\pb}aber ob ſie endgiltig gedeihen werden,
                    iſt noch nicht zu ſagen. Das eine Kind wird eben aufgepäppelt.\pend
           \pstart
           – Am \textcolor{green}{Roman}{}\ledrightnote{→\textcolor{green}{Der Weg ins Freie. Roman}} geſchah nichts
                    weiteres; über eine luſtſpielartige, moderne Komödie wurde meditirt. Im ganzen
                    mehr Kunſt- und Gedankenſpiel als Schaffensintenſität. –\pend
           \pstart
           Mit großem Vergnügen las ich die \textcolor{green}{\textsc{mousquetaires}}{}\ledrightnote{\textcolor{green}{Die drei Musketiere}} v. \textcolor{blue}{\textsc{Dumas}}{}\ledrightnote{\textcolor{blue}{Alexandre père Dumas}} auf der Reiſe. Welche Leichtigkeit,
                    welcher Reichtum! Einiger Leichtſinn verzeiht ſich von ſelbſt; {\pb}und die paar falſchen Münzen wirken, als machte
                    sich ein Kind \strikeout{damit} einen Spaſs ſie ſtatt
                    echten, die doch da ſind, auszuſtreuen. –\pend
           \pstart
           – \textsc{\textcolor{blue}{Bahr}{}\ledrightnote{\textcolor{blue}{Hermann Bahr}}} hat mir von Ihren letzten \label{K_L01300_1v}\edtext{Plänen}{\lemma{\textnormal{\emph{Plänen}}}\Cendnote{\textnormal{\emph{\textcolor{green}{Elektra}} und \emph{\textcolor{green}{Das gerettete Venedig}}.}}}\label{K_L01300_1h} erzählt, \textcolor{blue}{Richard}{}\ledrightnote{\textcolor{blue}{Richard Beer-Hofmann}}, der geſtern mit \textcolor{blue}{Paula}{}\ledrightnote{\textcolor{blue}{Paula Beer-Hofmann}} u \textcolor{blue}{Mirjam}{}\ledrightnote{\textcolor{blue}{Mirjam Beer-Hofmann}} bei mir war,
                    desgleichen. Ich wünſchte bald zu hören wie weit Sie gediehen ſind.\pend
           \pstart
           Die \textcolor{pink}{deutſchen}{}\ledrightnote{\textcolor{pink}{Deutschland}}{ }\textcolor{pink}{Schall u Raucher}{}\ledrightnote{\textcolor{pink}{Schall und Rauch}}{ }ſah ich \introOben{}vor\introOben{}geſtern, \textcolor{green}{Erdgeiſt}{}\ledrightnote{\textcolor{green}{Erdgeist. Tragödie in vier Aufzügen}}, das Talent, das große \textcolor{blue}{Wedekind}{}\ledrightnote{\textcolor{blue}{Frank Wedekind}}eſche {\pb}blitzt meines
                    Erachtens nur ſelten auf. Vielleicht ernſthaft nur in der Figur des Dr \textcolor{green}{Schön}{}\ledrightnote{→\textcolor{green}{Erdgeist. Tragödie in vier Aufzügen}} (der einzigen, die
                    wirklich vollendet geſpielt wurde \introOben{}(\textsc{\textcolor{blue}{Reicher}{}\ledrightnote{\textcolor{blue}{Emanuel Reicher}}})\introOben{}.) Das unerträgliche aber an dem Stück iſt mir, daſs der Humor
                    darin der ſich ſo sataniſch geberdet, nicht viel teufliſcher iſt als ein
                    weitgereiſter Commis \introOben{}als \textsc{\textcolor{green}{Mephisto}{}\ledrightnote{→\textcolor{green}{Faust}}}\introOben{} auf einem Maskenball, – der mit dämoniſchen Weibern Champagner zu trinken
                    vermeint – während es {\pb}ſich um Köchinnen und \textsc{Kleinoscheg} handelt. – Im ganzen lieb ich Dichter
                    nicht, die ihren Nachlaſs bei Lebzeiten herausgeben. –\pend
           \pstart
           Wie steht es mit Ihren ferneren Sommerplänen? Ich denke etwa um den 10.
                        Auguſt nach \textcolor{pink}{Südtirol}{}\ledrightnote{\textcolor{pink}{Südtirol}} zu gehen. \textsc{Mendel}, \textcolor{pink}{Campiglio}{}\ledrightnote{\textcolor{pink}{Madonna di Campiglio}}{[}.{]}{ }\textcolor{blue}{Richard}{}\ledrightnote{\textcolor{blue}{Richard Beer-Hofmann}} will mit – radeln.\pend
           \pstart
           Laſſen Sie baldigſt von ſich hören. Wir grüßen Sie und \textcolor{blue}{Gerty}{}\ledrightnote{\textcolor{blue}{Gertrude von Hofmannsthal}} herzlichſt.\pend
           \pstart
           Ihr{\\[\baselineskip]}\spacefill\mbox{A.}\pend
           \leftskip=0em{}\endnumbering\briefempfaengerindex{Hofmannsthal, Hugo von@\textsc{Hofmannsthal, Hugo von}!zzzSchnitzler, Arthur@\emph{von Arthur Schnitzler}!1903-06-261@{26. 6. 1903}|)be}\mylabel{h}  \normalsize

\doendnotes{C}
\bigskip
\vfill

\clearpage

\footnotesize

\lohead{\textsc{register}}

% Definiere theindex-Environment komplett neu ohne reledmac
\makeatletter
\renewenvironment{theindex}{%
  \section*{\indexname}%
  \setlength{\parindent}{0pt}%
  \setlength{\parskip}{0pt plus 0.3pt}%
  \let\item\@idxitem
}{%
  \clearpage
}
\makeatother

\IfFileExists{\jobname-pw.ind}{\input{\jobname-pw.ind}}{}

\end{document}

      