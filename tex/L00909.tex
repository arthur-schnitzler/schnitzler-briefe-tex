%% latex-korrekturansicht-vorspann.tex
%% Vorspann für die Korrekturansicht.
%% Lädt die gemeinsame Datei latex-vorspann.tex mit gesetztem Schalter.

\newif\ifkorrekturansicht
\korrekturansichttrue

\input{../tex-inputs/latex-vorspann}


               \section[Hugo von Hofmannsthal an Arthur Schnitzler, {[}23. 3. 1899{]}]{ Hugo von Hofmannsthal an Arthur Schnitzler, {[}23. 3. 1899{]}}\nopagebreak\mylabel{v}\rehead{ }\normalsize\beginnumbering\briefempfaengerindex{Schnitzler, Arthur@\textsc{Schnitzler, Arthur}!zzzHofmannsthal, Hugo von@\emph{von Hugo von Hofmannsthal}!1899-03-231@{{[}23. 3. 1899{]}}|(be} \toendnotes[C]{\smallbreak\pagebreak[2]} \Standort{CUL, Schnitzler, B 43.}
\physDesc{Brief, 1 Blatt, 3 Seiten
\newline{}Handschrift: schwarze Tinte, deutsche Kurrent
\newline{}Schnitzler: mit Bleistift datiert: »23/3? 99« \newline{}Ordnung: 1) mit Bleistift von unbekannter Hand nummeriert: »\strikeout{144}« 2) mit Bleistift von unbekannter Hand nummeriert:
                                    »141«}\buchAbdrucke{\weitereDrucke{Hugo von Hofmannsthal, Arthur Schnitzler: \emph{Briefwechsel}. Hg. Therese Nickl und Heinrich Schnitzler. Frankfurt am Main: \emph{S. Fischer} 1964, S. 120.} }\toendnotes[C]{\smallbreak}\pstart
           \raggedleft{}{\pb}Berlin, \textcolor{pink}{Windsor Behrenſtraße}{}\ledrightnote{\textcolor{pink}{Hotel Windsor}}\pend
           \pstart{}Mein guter lieber Arthur\pend\pstart
           Könnten Sie nicht \textcolor{pink}{hierher}{}\ledrightnote{→\textcolor{pink}{Berlin}} ko{\geminationm}en? wir könnten ſehr viel beiſammen ſein und auch ſonſt
               ſieht man viele ernſte und liebenswürdige Menſchen und es wäre Ihnen doch leichter,
               ſich ein biſſl in die Höh zu bringen, als in \textcolor{pink}{Wien}{}\ledrightnote{\textcolor{pink}{Wien}}, wo
               die \textcolor{blue}{Erinnerung}{}\ledrightnote{→\textcolor{blue}{Marie Reinhard}} Ihnen bei jedem
               Schritt {\pb}friſch weh thut. Ich
               ſehne mich ſehr, mit Ihnen zu ſprechen, zu ſchreiben bin ich nicht im Stand.\pend
           \pstart
           Daſs dieſe Erinnerung immer mit meinen erſten \textcolor{green}{Stücken}{}\ledrightnote{→\textcolor{green}{Die Hochzeit der Sobeide}{\newline}→\textcolor{green}{Der Abenteurer und die Sängerin oder Die Geschenke des Lebens}} verknüpft bleiben muſs!\pend
           \pstart
           Von Herzen Ihr{\\[\baselineskip]}\spacefill\mbox{Hugo.}\pend
           \leftskip=0em{}\pstart
           \noindent{}P. S. \label{K_L00909_1v}\edtext{Hier}{\lemma{\textnormal{\emph{Hier}}}\Cendnote{\textnormal{Die Uraufführung im \emph{\textcolor{brown}{Deutschen Theater}} war am
                  18. 3. 1899 und damit zugleich mit der \textcolor{pink}{Wien}er Uraufführung angesetzt.}}}\label{K_L00909_1h}{ }ſind meine armen \textcolor{green}{Stücke}{}\ledrightnote{→\textcolor{green}{Die Hochzeit der Sobeide}{\newline}→\textcolor{green}{Der Abenteurer und die Sängerin oder Die Geschenke des Lebens}} von einer beiſpiellos böſen
                     {\pb}Preſſe erſchlagen worden
                  und muſsten nach dem dritten Mal abgeſetzt werden.\pend
           \endnumbering\briefempfaengerindex{Schnitzler, Arthur@\textsc{Schnitzler, Arthur}!zzzHofmannsthal, Hugo von@\emph{von Hugo von Hofmannsthal}!1899-03-231@{{[}23. 3. 1899{]}}|)be}\mylabel{h}  \normalsize

\doendnotes{C}
\bigskip
\vfill

\clearpage

\footnotesize

\lohead{\textsc{register}}

% Definiere theindex-Environment komplett neu ohne reledmac
\makeatletter
\renewenvironment{theindex}{%
  \section*{\indexname}%
  \setlength{\parindent}{0pt}%
  \setlength{\parskip}{0pt plus 0.3pt}%
  \let\item\@idxitem
}{%
  \clearpage
}
\makeatother

\IfFileExists{\jobname-pw.ind}{\input{\jobname-pw.ind}}{}

\end{document}

      