%% latex-korrekturansicht-vorspann.tex
%% Vorspann für die Korrekturansicht.
%% Lädt die gemeinsame Datei latex-vorspann.tex mit gesetztem Schalter.

\newif\ifkorrekturansicht
\korrekturansichttrue

\input{../tex-inputs/latex-vorspann}


               \section[Arthur Schnitzler an Richard Beer-Hofmann, {[}Erste Hälfte Juli? 1914{]}]{ Arthur Schnitzler an Richard Beer-Hofmann, {[}Erste Hälfte
               Juli? 1914{]}}\nopagebreak\mylabel{v}\rehead{ }\normalsize\beginnumbering\briefempfaengerindex{Beer-Hofmann, Richard@\textsc{Beer-Hofmann, Richard}!zzzSchnitzler, Arthur@\emph{von Arthur Schnitzler}!1914-07-011@{{[}Erste Hälfte
                  Juli? 1914{]}}|(be} \toendnotes[C]{\smallbreak\pagebreak[2]} \Standort{CUL, Schnitzler, B 8.1, S. 148.}
\physDesc{maschinelle Abschrift
\newline{}Schreibmaschine\newline{}Ordnung: von unbekannter Hand als Briefnummer »334«
                                 gekennzeichnet }\buchAbdrucke{\weitereDrucke{Arthur Schnitzler, Richard Beer-Hofmann: \emph{Briefwechsel 1891–1931}. Hg. Konstanze Fliedl. Wien, Zürich: \emph{Europaverlag} 1992, S. 220.} }\toendnotes[C]{\smallbreak}\pstart
           \raggedleft{}{\pb}\textcolor{pink}{Wien}{}\ledrightnote{\textcolor{pink}{Wien}}, ? 1914.\pend
           \pstart
           Lieber Richard – bleiben Sie nur in den Bergen, so lang Sie wollen
               und können. Ich wüsste absolut nicht, was Sie (vorläufig) hier machen sollten.
               Nachrichten gibt es hier kaum früher als bei Ihnen – Gerüchte vielleicht – aber die
               glaubt man sowieso nicht. Die \label{K_L02185_1v}\edtext{Spannung}{\lemma{\textnormal{\emph{Spannung}}}\Cendnote{\textnormal{Am 25. 6. 1914
                  hatte \textcolor{blue}{Beer-Hofmann} eine Unterkunft in \textcolor{pink}{Weißenbach am Attersee} bezogen. Die hier
                  augenscheinliche politische Anspannung dürfte sich auf die Zeit vor der
                  Kriegserklärung am 28. 7. 1914 beziehen. Da aber \textcolor{blue}{Schnitzler} am 17. 7. 1914 selbst aus \textcolor{pink}{Wien}
                  abreiste und erst am 1. 9. 1914 zurückkehrte, ist das Korrespondenzstück 
                     zeitlich davor anzusiedeln.}}}\label{K_L02185_1h} in den letzten Tagen war ungeheuer – heute ist man etwas
               ruhiger. Lassen Sie sichs wohl ergehen, grüssen Sie \textcolor{blue}{Paula}{}\ledrightnote{\textcolor{blue}{Paula Beer-Hofmann}} und die \textcolor{blue}{Kinder}{}\ledrightnote{→\textcolor{blue}{Gabriel Beer-Hofmann}{\newline}→\textcolor{blue}{Mirjam Beer-Hofmann}{\newline}→\textcolor{blue}{Naëmah Beer-Hofmann}} von uns Allen.\pend
           \pstart Herzlichst Ihr \spacefill\mbox{Arthur.}\pend{}\pstart
           \noindent{}(nach \textcolor{pink}{Weissenbach}{}\ledrightnote{\textcolor{pink}{Weißenbach am Attersee}})\pend
           \endnumbering\briefempfaengerindex{Beer-Hofmann, Richard@\textsc{Beer-Hofmann, Richard}!zzzSchnitzler, Arthur@\emph{von Arthur Schnitzler}!1914-07-011@{{[}Erste Hälfte
                  Juli? 1914{]}}|)be}\mylabel{h}  \normalsize

\doendnotes{C}
\bigskip
\vfill

\clearpage

\footnotesize

\lohead{\textsc{register}}

% Definiere theindex-Environment komplett neu ohne reledmac
\makeatletter
\renewenvironment{theindex}{%
  \section*{\indexname}%
  \setlength{\parindent}{0pt}%
  \setlength{\parskip}{0pt plus 0.3pt}%
  \let\item\@idxitem
}{%
  \clearpage
}
\makeatother

\IfFileExists{\jobname-pw.ind}{\input{\jobname-pw.ind}}{}

\end{document}

      