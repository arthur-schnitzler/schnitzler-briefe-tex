%% latex-korrekturansicht-vorspann.tex
%% Vorspann für die Korrekturansicht.
%% Lädt die gemeinsame Datei latex-vorspann.tex mit gesetztem Schalter.

\newif\ifkorrekturansicht
\korrekturansichttrue

\input{../tex-inputs/latex-vorspann}


               \section[Hugo von Hofmannsthal an Arthur Schnitzler, {[}25. 8. 1916{]}]{ Hugo von Hofmannsthal an Arthur Schnitzler, {[}25. 8. 1916{]}}\nopagebreak\mylabel{v}\rehead{ }\normalsize\beginnumbering\briefempfaengerindex{Schnitzler, Arthur@\textsc{Schnitzler, Arthur}!zzzHofmannsthal, Hugo von@\emph{von Hugo von Hofmannsthal}!1916-08-251@{{[}25. 8. 1916{]}}|(be} \toendnotes[C]{\smallbreak\pagebreak[2]} \Standort{CUL, Schnitzler, B 43.}
\physDesc{Briefkarte
\newline{}Handschrift: schwarze Tinte, deutsche Kurrent
\newline{}Schnitzler: 1) mit Bleistift datiert: »25/8 16« und beschriftet: »\textsc{\textcolor{pink}{Aussee}}« und »Hugo« 2) mit rotem Buntstift eine Unterstreichung\newline{}Ordnung: 1) mit Bleistift von \textcolor{blue}{Frieda Pollak} (?) mit dem Buchstaben »A« (Abgeschrieben/Abschrift) gekennzeichnet 2) mit Bleistift von unbekannter Hand nummeriert: »\strikeout{344}«3) mit Bleistift von unbekannter Hand nummeriert:
                                    »354«}\buchAbdrucke{\weitereDrucke{Hugo von Hofmannsthal, Arthur Schnitzler: \emph{Briefwechsel}. Hg. Therese Nickl und Heinrich Schnitzler. Frankfurt am Main: \emph{S. Fischer} 1964, S. 280.} }\toendnotes[C]{\smallbreak}\pstart
           \raggedleft{}{\pb}Freitag.\pend
           \pstart{}mein guter Arthur\pend\pstart
           ich will Sie nicht bedrängen u. beläſtigen aber ich fühle wie woltätig mir – ſo oder
               ſo – die Möglichkeit Ihnen dieſe problematiſchen \textcolor{green}{Fragmente}{}\ledrightnote{→\textcolor{green}{Der Sohn des Geisterkönigs}} vorzuleſen ſein wird. Ich werde dieſe vielleicht
               allzu gewagte Arbeit nachher entweder {\pb}weglegen oder mit größerer
               Zuverſicht wieder anpacken.\pend
           \pstart
           Wäre es zu denken daſs Sie dieſe 1½ Stunden in den allernächſten Tagen mir ſchenken
               könnten – in der Früh – am ſpäten Vormittag{[},{]} am Abend oder wann
                  i{\geminationm}er? \pend
           \pstart Herzlich Ihr\spacefill\mbox{Hugo.}\pend{}\endnumbering\briefempfaengerindex{Schnitzler, Arthur@\textsc{Schnitzler, Arthur}!zzzHofmannsthal, Hugo von@\emph{von Hugo von Hofmannsthal}!1916-08-251@{{[}25. 8. 1916{]}}|)be}\mylabel{h}  \normalsize

\doendnotes{C}
\bigskip
\vfill

\clearpage

\footnotesize

\lohead{\textsc{register}}

% Definiere theindex-Environment komplett neu ohne reledmac
\makeatletter
\renewenvironment{theindex}{%
  \section*{\indexname}%
  \setlength{\parindent}{0pt}%
  \setlength{\parskip}{0pt plus 0.3pt}%
  \let\item\@idxitem
}{%
  \clearpage
}
\makeatother

\IfFileExists{\jobname-pw.ind}{\input{\jobname-pw.ind}}{}

\end{document}

      