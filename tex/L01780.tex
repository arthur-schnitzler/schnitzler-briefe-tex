%% latex-korrekturansicht-vorspann.tex
%% Vorspann für die Korrekturansicht.
%% Lädt die gemeinsame Datei latex-vorspann.tex mit gesetztem Schalter.

\newif\ifkorrekturansicht
\korrekturansichttrue

\input{../tex-inputs/latex-vorspann}


               \section[Olga und Arthur Schnitzler an Hermann Bahr, 6. 7. 1908]{ Olga und Arthur Schnitzler an Hermann Bahr, 6. 7. 1908}\nopagebreak\mylabel{v}\rehead{ }\normalsize\beginnumbering\briefempfaengerindex{Bahr, Hermann@\textsc{Bahr, Hermann}!zzzSchnitzler, Arthur@\emph{von Arthur Schnitzler}!1908-07-061@{6. 7. 1908}|(be}\briefempfaengerindex{Bahr, Hermann@\textsc{Bahr, Hermann}!zzzSchnitzler, Olga@\emph{von Olga Schnitzler}!1908-07-061@{6. 7. 1908}|(be} \toendnotes[C]{\smallbreak\pagebreak[2]} \Standort{TMW, HS AM 60163 Ba.}
\physDesc{Bildpostkarte
\newline{}Handschrift Olga Schnitzler: schwarze Tinte, lateinische Kurrent\newline{}Handschrift Arthur Schnitzler: schwarze Tinte, deutsche Kurrent\newline{}Versand: Stempel: »\nobreak{}6. 7. 8\nobreak{}«.  \newline{}Ordnung: Lochung }\buchAbdrucke{\weitereDrucke{1) \emph{6. 7. 1908, Abschrift.} In: Arthur Schnitzler: \emph{The Letters of Arthur Schnitzler to Hermann Bahr}. Edited, annotated, and with an introduction, by Donald G.
                        Daviau. Chapel Hill: \emph{The University of North Carolina Press} 1978, S. 102 (University of North Carolina studies in the Germanic languages
                        and literatures, 89).} \weitereDrucke{2) Hermann Bahr, Arthur Schnitzler: \emph{Briefwechsel, Aufzeichnungen, Dokumente (1891–1931)}. Hg. Kurt Ifkovits und Martin Anton Müller. Göttingen: \emph{Wallstein} 2018, S. 405.} }\toendnotes[C]{\smallbreak}\pstart{}{\pb}Herrn\pend{}\pstart{}Hermann Bahr\pend{}\pstart{}\textcolor{pink}{Ober St. Veit bei Wien}{}\ledrightnote{\textcolor{pink}{Ober Sankt Veit}}\pend{}\pstart{}\textcolor{pink}{Veitlissengasse.}{}\ledrightnote{\textcolor{pink}{Veitlissengasse}}\pend{}{\bigskip}\pstart
           \noindent{}\centering{}\textcolor{gray}{\textbf{{\pb}\textcolor{pink}{Tirol: \label{T_L01780_1v}\edtext{\uline{Villa Heufler, Seis am Schlern}}{\lemma{\textnormal{\emph{Villa … Schlern}}}\Cendnote{\textnormal{Unterstreichung mit schwarzer Tinte}}}\label{T_L01780_1h}}{}\ledrightnote{\textcolor{pink}{Villa Heufler}}, 1000m. Nach dem \textcolor{green}{Aquarell}{}\ledrightnote{→\textcolor{green}{Partie in Seis am Schlern}} von \textcolor{blue}{F. A. C. M. Reisch}{}\ledrightnote{\textcolor{blue}{Franz August Carl Maria Reisch}}, \textcolor{pink}{Meran}{}\ledrightnote{\textcolor{pink}{Meran}}.}}\pend
           \pstart
           \raggedleft{}{\pb}6. Juli{\\}08.\pend
           \pstart{}Lieber Herr Bahr,\pend\pstart
           \textcolor{blue}{wir}{}\ledrightnote{→\textcolor{blue}{Olga Schnitzler}} haben Ihr wunderschönes
                  \label{K_L01780_1v}\edtext{\textcolor{green}{Feuilleton über \textcolor{blue}{Moppchen}{}\ledrightnote{\textcolor{blue}{Katharina Selma Hartleben}}}{}\ledrightnote{→\textcolor{green}{Moppchen}}}{\lemma{\textnormal{\emph{Feuilleton über Moppchen}}}\Cendnote{\textnormal{\textcolor{blue}{Hermann Bahr}: \emph{\textcolor{green}{Moppchen}}. In: \emph{\textcolor{green}{Neue Freie Presse}},
                     Nr. 15757, 4. 7. 1908. Morgenblatt, S. 1–5 (»Moppchen« war
                  der Spitzname von \textcolor{blue}{Otto Erich Hartleben}s
                  Ehefrau \textcolor{blue}{Selma}).}}}\label{K_L01780_1h} mit Ergriffenheit
               gelesen, schicken Ihnen die herzlichsten Grüsse und viel gute Wünsche für den
               Sommer.\pend
           \pstart \spacefill\mbox{Olga Schnitzler.}\pend{}\pstart
           \noindent{}{[}hs. Schnitzler:{]} He\damage{rzl}ichſt dein{\\}\spacefill\mbox{Arthur}. \pend
           \pstart
           \noindent{}{\pb}{[}hs. O. Schnitzler:{]} \label{T_L01780_2v}\edtext{Unser Balcon.}{\lemma{\textnormal{\emph{Unser Balcon.}}}\Cendnote{\textnormal{auf dem Motiv mit einem Pfeil markiert}}}\label{T_L01780_2h}\pend
           \endnumbering\briefempfaengerindex{Bahr, Hermann@\textsc{Bahr, Hermann}!zzzSchnitzler, Arthur@\emph{von Arthur Schnitzler}!1908-07-061@{6. 7. 1908}|)be}\briefempfaengerindex{Bahr, Hermann@\textsc{Bahr, Hermann}!zzzSchnitzler, Olga@\emph{von Olga Schnitzler}!1908-07-061@{6. 7. 1908}|)be}\mylabel{h}  \normalsize

\doendnotes{C}
\bigskip
\vfill

\clearpage

\footnotesize

\lohead{\textsc{register}}

% Definiere theindex-Environment komplett neu ohne reledmac
\makeatletter
\renewenvironment{theindex}{%
  \section*{\indexname}%
  \setlength{\parindent}{0pt}%
  \setlength{\parskip}{0pt plus 0.3pt}%
  \let\item\@idxitem
}{%
  \clearpage
}
\makeatother

\IfFileExists{\jobname-pw.ind}{\input{\jobname-pw.ind}}{}

\end{document}

      