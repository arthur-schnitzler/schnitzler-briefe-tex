%% latex-korrekturansicht-vorspann.tex
%% Vorspann für die Korrekturansicht.
%% Lädt die gemeinsame Datei latex-vorspann.tex mit gesetztem Schalter.

\newif\ifkorrekturansicht
\korrekturansichttrue

\input{../tex-inputs/latex-vorspann}


               \section[Paul Goldmann an Arthur Schnitzler, {[}20. 1. 1895?{]}]{ Paul Goldmann an Arthur Schnitzler, {[}20. 1. 1895?{]}}\nopagebreak\mylabel{v}\rehead{ }\normalsize\beginnumbering\briefempfaengerindex{Schnitzler, Arthur@\textsc{Schnitzler, Arthur}!zzzGoldmann, Paul@\emph{von Paul Goldmann}!1895-01-201@{{[}20. 1. 1895?{]}}|(be} \toendnotes[C]{\smallbreak\pagebreak[2]} \Standort{DLA, A:Schnitzler, HS.NZ85.1.3165.}
\physDesc{Telegramm1 Blatt, 1 Seite
\newline{}maschinell
\newline{}Schnitzler: mit Bleistift datiert: »Januar 9\strikeout{4}« \newline{}Ordnung: 1) von unbekannter Hand mit Tinte einen Strich bei der ergänzten
                                 Jahreszahl hinzugefügt, eventuell in der Absicht, aus der
                                 gestrichenen »4« eine »5« zu machen 2) beschnitten}\toendnotes[C]{\smallbreak}\pstart
           \centering{}{\pb}\textcolor{pink}{w}{}\ledrightnote{\textcolor{pink}{Wien}} fr \textcolor{pink}{paris}{}\ledrightnote{\textcolor{pink}{Paris}}
                  30298 20{ }1/38=\pend
           \pstart
           hab meine \label{K_L02692-1v}\edtext{innige freude}{\lemma{\textnormal{\emph{innige freude}}}\Cendnote{\textnormal{Das Datum lässt sich durch das
                  Zusammenlesen dreier Indizien ermitteln: Die Datierung \textcolor{blue}{Schnitzler}s auf »Januar«, die Angabe des
                  Tages »20« in der Adressierungszeile des Telegramms und dem
                  Umstand, dass am 19. 1. 1895 bekannt wurde, dass \emph{\textcolor{green}{Liebelei}} vom \emph{\textcolor{brown}{Burgtheater}} angenommen
                  worden war.}}}\label{K_L02692-1h} dran nun wirds rasch aufwaerts gehen\pend
           \pstart haendedruck und glueckwuensche = \spacefill\mbox{goldmann. +}\pend{}\endnumbering\briefempfaengerindex{Schnitzler, Arthur@\textsc{Schnitzler, Arthur}!zzzGoldmann, Paul@\emph{von Paul Goldmann}!1895-01-201@{{[}20. 1. 1895?{]}}|)be}\mylabel{h}\begin{anhang}\end{anhang}\normalsize

\doendnotes{C}
\bigskip
\vfill

\clearpage

\footnotesize

\lohead{\textsc{register}}

% Definiere theindex-Environment komplett neu ohne reledmac
\makeatletter
\renewenvironment{theindex}{%
  \section*{\indexname}%
  \setlength{\parindent}{0pt}%
  \setlength{\parskip}{0pt plus 0.3pt}%
  \let\item\@idxitem
}{%
  \clearpage
}
\makeatother

\IfFileExists{\jobname-pw.ind}{\input{\jobname-pw.ind}}{}

\end{document}

      