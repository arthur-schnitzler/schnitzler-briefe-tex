%% latex-korrekturansicht-vorspann.tex
%% Vorspann für die Korrekturansicht.
%% Lädt die gemeinsame Datei latex-vorspann.tex mit gesetztem Schalter.

\newif\ifkorrekturansicht
\korrekturansichttrue

\input{../tex-inputs/latex-vorspann}


               \section[Hugo von Hofmannsthal an Arthur Schnitzler, 9. 3. 1894]{ Hugo von Hofmannsthal an Arthur Schnitzler, 9. 3. 1894}\nopagebreak\mylabel{v}\rehead{ }\normalsize\beginnumbering\briefempfaengerindex{Schnitzler, Arthur@\textsc{Schnitzler, Arthur}!zzzHofmannsthal, Hugo von@\emph{von Hugo von Hofmannsthal}!1894-03-091@{9. 3. 1894}|(be} \toendnotes[C]{\smallbreak\pagebreak[2]} \Standort{CUL, Schnitzler, B 43.}
\physDesc{Briefkarte mit aufgeprägtem Wappen
\newline{}Handschrift: Bleistift, deutsche Kurrent
\newline{}Schnitzler: mit Bleistift nummeriert: »63« }\buchAbdrucke{\weitereDrucke{Hugo von Hofmannsthal, Arthur Schnitzler: \emph{Briefwechsel}. Hg. Therese Nickl und Heinrich Schnitzler. Frankfurt am Main: \emph{S. Fischer} 1964, S. 50.} }\pstart
           \raggedleft{}{\pb}9. III. 94.\pend
           \pstart{}lieber Arthur!\pend\pstart
           Ich möchte mit Ihnen 1.) ins \textcolor{pink}{Arſenal}{}\ledrightnote{\textcolor{pink}{Arsenal}} 2.) auf
                    den \textcolor{pink}{Stephansthurm}{}\ledrightnote{\textcolor{pink}{Stephansdom}} gehen.\pend
           \pstart
           Bitte erkundigen Sie ſich um die möglichen Stunden, wählen Sie dann ein paar
                    Stunden und Tage, die Ihnen paſſen und ſchreiben Sie mirs ſogleich. Ich werde
                    ſofort antworten {\pb}und ſo
                    wirds hoffentlich zuſammengehen.\pend
           \pstart
           Sonntag gehe ich wahrſcheinlich zu den »\textcolor{green}{Nibelungen}{}\ledrightnote{\textcolor{green}{Die Nibelungen}}« (Loge) dann gewiſs zu Ihnen.\pend
           \pstart
           Oder Nicht?\pend
           \pstart
           von Herzen{\\[\baselineskip]}Ihr{\\[\baselineskip]}\spacefill\mbox{Hugo.}\pend
           \leftskip=0em{}\endnumbering\briefempfaengerindex{Schnitzler, Arthur@\textsc{Schnitzler, Arthur}!zzzHofmannsthal, Hugo von@\emph{von Hugo von Hofmannsthal}!1894-03-091@{9. 3. 1894}|)be}\mylabel{h}  \normalsize

\doendnotes{C}
\bigskip
\vfill

\clearpage

\footnotesize

\lohead{\textsc{register}}

% Definiere theindex-Environment komplett neu ohne reledmac
\makeatletter
\renewenvironment{theindex}{%
  \section*{\indexname}%
  \setlength{\parindent}{0pt}%
  \setlength{\parskip}{0pt plus 0.3pt}%
  \let\item\@idxitem
}{%
  \clearpage
}
\makeatother

\IfFileExists{\jobname-pw.ind}{\input{\jobname-pw.ind}}{}

\end{document}

      