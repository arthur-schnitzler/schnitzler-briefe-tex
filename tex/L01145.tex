%% latex-korrekturansicht-vorspann.tex
%% Vorspann für die Korrekturansicht.
%% Lädt die gemeinsame Datei latex-vorspann.tex mit gesetztem Schalter.

\newif\ifkorrekturansicht
\korrekturansichttrue

\input{../tex-inputs/latex-vorspann}


               \section[Arthur Schnitzler an Richard Beer-Hofmann, 11. 7. 1901]{ Arthur Schnitzler an Richard Beer-Hofmann, 11. 7. 1901}\nopagebreak\mylabel{v}\rehead{ }\normalsize\beginnumbering\briefempfaengerindex{Beer-Hofmann, Richard@\textsc{Beer-Hofmann, Richard}!zzzSchnitzler, Arthur@\emph{von Arthur Schnitzler}!1901-07-111@{11. 7. 1901}|(be} \toendnotes[C]{\smallbreak\pagebreak[2]} \Standort{YCGL, MSS 31.}
\physDesc{Telegramm
\newline{}Handschrift einer Schreibkraft: schwarze Tinte, deutsche Kurrent\newline{}Versand: »\noindent{}\textcolor{gray}{\textbf{Von}}{ }\textcolor{pink}{Station Arlsberg}{ / }\textcolor{gray}{\textbf{Aufgabe-Nr}} 38 \textcolor{gray}{\textbf{mit {\dots}
                                             Taxworten ({\dots} Worten}}
                                         21 \textcolor{gray}{\textbf{Chiffern)}}{ / }\textcolor{gray}{\textbf{Eingelangt von {\dots} auf Leitung Nr. {\dots} am}}{ }11/7 \textcolor{gray}{\textbf{190}}{\dots}{ }\textcolor{gray}{\textbf{um}}{ }4 \textcolor{gray}{\textbf{Uhr}} 45 \textcolor{gray}{\textbf{Min.}}{\dots}\textcolor{gray}{\textbf{Mittag}}{ / }\textcolor{gray}{\textbf{Aufgegeben am}}{ }11/7 \textcolor{gray}{\textbf{190}}{\dots}{ }\textcolor{gray}{\textbf{um}}{ }3 \textcolor{gray}{\textbf{Uhr}} 50 \textcolor{gray}{\textbf{Min. {\dots}
                                             Mittag}}« }\pstart{}{\pb}Hr Richard Beerhofman\pend{}\pstart{}\textcolor{pink}{Villa Arnstein}{}\ledrightnote{\textcolor{pink}{Villa Arnstein}}\pend{}\pstart{}\textcolor{gray}{\textbf{\textit{\textcolor{pink}{PÖRTSCHACH AM SEE}{}\ledrightnote{\textcolor{pink}{Pörtschach}}}}}\pend{}{\bigskip}\pstart
           \noindent{}{\pb}Bitte wo wohnt man in \textcolor{pink}{\textsc{Vahrn}}{}\ledrightnote{\textcolor{pink}{Vahrn}} am beſten Antwort nach \textcolor{pink}{Innsbruck Hotel
                        Kreid}{}\ledrightnote{\textcolor{pink}{Hotel Kreid}}\pend
           \pstart
           Herzlichſt{\\[\baselineskip]}\spacefill\mbox{Arthur}\pend
           \leftskip=0em{}\endnumbering\briefempfaengerindex{Beer-Hofmann, Richard@\textsc{Beer-Hofmann, Richard}!zzzSchnitzler, Arthur@\emph{von Arthur Schnitzler}!1901-07-111@{11. 7. 1901}|)be}\mylabel{h}  \normalsize

\doendnotes{C}
\bigskip
\vfill

\clearpage

\footnotesize

\lohead{\textsc{register}}

% Definiere theindex-Environment komplett neu ohne reledmac
\makeatletter
\renewenvironment{theindex}{%
  \section*{\indexname}%
  \setlength{\parindent}{0pt}%
  \setlength{\parskip}{0pt plus 0.3pt}%
  \let\item\@idxitem
}{%
  \clearpage
}
\makeatother

\IfFileExists{\jobname-pw.ind}{\input{\jobname-pw.ind}}{}

\end{document}

      