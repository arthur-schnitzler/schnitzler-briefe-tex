%% latex-korrekturansicht-vorspann.tex
%% Vorspann für die Korrekturansicht.
%% Lädt die gemeinsame Datei latex-vorspann.tex mit gesetztem Schalter.

\newif\ifkorrekturansicht
\korrekturansichttrue

\input{../tex-inputs/latex-vorspann}


               \section[Georg Brandes an Arthur Schnitzler, 19. 12. 1917]{ Georg Brandes an Arthur Schnitzler, 19. 12. 1917}\nopagebreak\mylabel{v}\rehead{ }\normalsize\beginnumbering\briefempfaengerindex{Schnitzler, Arthur@\textsc{Schnitzler, Arthur}!zzzBrandes, Georg@\emph{von Georg Brandes}!1917-12-191@{19. 12. 1917}|(be} \toendnotes[C]{\smallbreak\pagebreak[2]} \Standort{CUL, Schnitzler, B 17.}
\physDesc{Postkarte
\newline{}Handschrift: schwarze Tinte, lateinische Kurrent\newline{}Versand: 1) Stempel: »\nobreak{}\oindex{Kopenhagen@\textbf{Kopenhagen}, \emph{Besiedelter Ort (A.BSO)}|pwk}Kjøbenhavn, 20. 12. 17, 5–6F\nobreak{}«.  2) Stempel: »\nobreak{}Zensuriert \textcolor{brown}{{[}k.{]} u. k. Zensurstelle Wien}\nobreak{}«. \newline{}Ordnung: mit Bleistift von unbekannter Hand nummeriert: »48« }\buchAbdrucke{\weitereDrucke{Georg Brandes, Arthur Schnitzler: \emph{Ein Briefwechsel}. Hg. Kurt Bergel. Bern: \emph{Francke} 1956, S. 122.} }\toendnotes[C]{\smallbreak}\pstart{}{\pb}Herrn Dr. Arthur
                        Schnitzler\pend{}\pstart{}\textcolor{pink}{Sternwartestrasse 71}{}\ledrightnote{\textcolor{pink}{Sternwartestraße}}\pend{}\pstart{}\textcolor{pink}{Wien \textsubscript{XVIII}}{}\ledrightnote{\textcolor{pink}{XVIII., Währing}}\pend{}{\bigskip}\pstart
           \raggedleft{}{\pb}\textcolor{pink}{Kopenhagen}{}\ledrightnote{\textcolor{pink}{Kopenhagen}}{ }19 Dec. 17\pend
           \pstart
           Verehrter, lieber Freund \hspace*{3.5em}Niemand ist treu und liebenswürdig wie Sie.
                    Obwohl ich nie in der Lage bin, Vergelt zu üben, senden Sie mir fortwährend Ihre
                    Erzählungen und Schauspiele, die mir so viel Freude bereiten. Nun das letzte Mal
                        \textcolor{green}{\uline{Fink und Fliederbusch}}{}\ledrightnote{\textcolor{green}{Fink und Fliederbusch. Komödie in drei Akten}}, ein heiteres Stück in trauriger Zeit, nicht ohne satirischen Stachel,
                    aber dennoch human. Ein \textcolor{blue}{Franzose}{}\ledrightnote{→\textcolor{blue}{Nicolas Eugène Géruzez}}
              sagte: \label{K_L02282_1v}\edtext{L’âge mûr
               méprise avec tolérance}{\lemma{\textnormal{\emph{L’âge … tolérance}}}\Cendnote{\textnormal{französisch: Das reife
                        Alter verachtet durch Toleranz}}}\label{K_L02282_1h}.\pend
           \pstart
           Wäre ich so glücklich all das was ich geschrieben habe, seit wir uns sahen, würde
                    es eine stattliche Reihe Bücher ausmachen, nicht weniger als 7 schwere Bände.
                    Mein \textcolor{green}{Buch über den Weltkrieg}{}\ledrightnote{→\textcolor{green}{Verdenskrigen [The World at War]}}
                    erreicht in diesen Tagen hier die vierte Ausgabe, hat in \textcolor{pink}{Nordamerika}{}\ledrightnote{\textcolor{pink}{Amerika}} zwei. Die Bücher über \textcolor{green}{\textcolor{blue}{Goethe}{}\ledrightnote{\textcolor{blue}{Johann Wolfgang von Goethe}}}{}\ledrightnote{→\textcolor{green}{Wolfgang Goethe}}, über \textcolor{green}{\textcolor{blue}{Voltaire}{}\ledrightnote{\textcolor{blue}{Voltaire}}}{}\ledrightnote{→\textcolor{green}{Voltaire und sein Jahrhundert}} usw. sind gut gegangen. Ein \textcolor{green}{Buch}{}\ledrightnote{→\textcolor{green}{Udvalgte Stykker}}, worin ich meine letzten Essays und Reden gesammelt habe, wurde
                        {\pb}in nur 14 Tagen
                    ausverkauft. Seit April bin ich damit beschäftigt eine grosse \textcolor{green}{Maschine}{}\ledrightnote{→\textcolor{green}{Gaius Julius Cæsar}} über meinen
                    vergötterten \textcolor{blue}{Cajus Julius Cäsar}{}\ledrightnote{\textcolor{blue}{Gaius Iulius Caesar}} zu
                    fabricieren, wird wol auch ein paar Bände werden. Der Stoff ist sehr
                    umfangreich, \textcolor{pink}{röm}{}\ledrightnote{\textcolor{pink}{Rom}}isches Leben von
                        c. 120 bis c. 40, aber er fesselt mich sehr. Bin
                    ich doch kein Erfinder, nur ein enthusiastischer Forscher. Ich hoffe, dass es
                    Ihnen und den Ihrigen, auch unseren wenigen gemeinsamen Freunden wohl geht. Ihr
                        \spacefill\mbox{Georg Brandes}\pend
           \endnumbering\briefempfaengerindex{Schnitzler, Arthur@\textsc{Schnitzler, Arthur}!zzzBrandes, Georg@\emph{von Georg Brandes}!1917-12-191@{19. 12. 1917}|)be}\mylabel{h}  \normalsize

\doendnotes{C}
\bigskip
\vfill

\clearpage

\footnotesize

\lohead{\textsc{register}}

% Definiere theindex-Environment komplett neu ohne reledmac
\makeatletter
\renewenvironment{theindex}{%
  \section*{\indexname}%
  \setlength{\parindent}{0pt}%
  \setlength{\parskip}{0pt plus 0.3pt}%
  \let\item\@idxitem
}{%
  \clearpage
}
\makeatother

\IfFileExists{\jobname-pw.ind}{\input{\jobname-pw.ind}}{}

\end{document}

      