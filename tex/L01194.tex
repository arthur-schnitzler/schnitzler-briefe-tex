%% latex-korrekturansicht-vorspann.tex
%% Vorspann für die Korrekturansicht.
%% Lädt die gemeinsame Datei latex-vorspann.tex mit gesetztem Schalter.

\newif\ifkorrekturansicht
\korrekturansichttrue

\input{../tex-inputs/latex-vorspann}


               \section[Richard Dehmel an Arthur Schnitzler, 1. 1. 1902]{ Richard Dehmel an Arthur Schnitzler, 1. 1. 1902}\nopagebreak\mylabel{v}\rehead{ }\normalsize\beginnumbering\briefempfaengerindex{Schnitzler, Arthur@\textsc{Schnitzler, Arthur}!zzzDehmel, Richard@\emph{von Richard Dehmel}!1902-01-013@{1. 1. 1902}|(be} \toendnotes[C]{\smallbreak\pagebreak[2]} \Standort{Hamburg, Staats- und Universitätsbibliothek, DA:Br:D:4173.}
\physDesc{Brief, 2 Blätter, 3 Seiten
\newline{}Handschrift: schwarze Tinte, lateinische Kurrent
\newline{}Schnitzler: mit rotem Buntstift eine Unterstreichung \newline{}Zusatz: Da dieses
                                    Korrespondenzstück im Nachlass Dehmels überliefert ist, dürfte
                                    es sich um eine Abschrift des tatsächlich versandten Briefes
                                    handeln }\toendnotes[C]{\smallbreak}\pstart
           \noindent{}{\pb}\textcolor{gray}{\textbf{RD}}\hfill \textcolor{pink}{Blankenese \textsuperscript{b}/Hamburg}{}\ledrightnote{\textcolor{pink}{Blankenese}}, 1. 1. 2.\pend
           \pstart
           Verehrter Herr Schnitzler!\pend
           \pstart
           Ich danke Ihnen herzlich für Ihr \textcolor{green}{Buch}{}\ledrightnote{→\textcolor{green}{Der Schleier der Beatrice. Schauspiel in fünf Akten}}. In Ermangelung einer Gegengabe – (aber »aufgeschoben ist nicht
                    aufgehoben«) – überfalle ich Sie gleich noch mit einer Bitte. Ich will in etwa
                    2 Jahren ein Kinderbuch herausgeben:\pend
           \pstart
           \centering{}\textcolor{green}{Der Buntscheck}{}\ledrightnote{\textcolor{green}{Der Buntscheck. Ein Sammelbuch herzhafter Kunst für Ohr und Auge deutscher Kinder}},{\\}ein Sammelbuch herzhafter
                    Kunst für Ohr und Auge unsrer Kinder –\pend
           \pstart
           \noindent{}{\pb}würden Sie mir dazu eine einfache kurze Geschichte
                    beisteuern können? Sie brauchen durchaus nicht \uline{vom} Kinde zu handeln, jeder andre »Stoff« ist mir sogar lieber; nur
                    soll eben Alles ganz vom Kinde \uline{aus} dargestellt,
                    also ohne sentimental\substVorne{}\textsuperscript{e}\substDazwischen{}ische\substHinten{} oder ironische Sehnsucht nach dem »verlorenen Paradiese«. Auf das
                    Mscrpt – (es darf aber noch nicht gedruckt sein und darf bis 1. Oktober
                            190\uline{5} auch nirgendwo anders veröffentlicht werden) – kann ich bis in den
                        September dies. Js. warten; länger {\pb}nicht aus illustrativen Gründen. Im übrigen hat der Verleger (\textcolor{brown}{Schafstein {\kaufmannsund} Co.}{}\ledrightnote{\textcolor{brown}{Schafstein {\kaufmannsund} Co.}} in \textcolor{pink}{Köln}{}\ledrightnote{\textcolor{pink}{Köln}}) mir völlig freie Hand bewilligt, sodaß
                    ich für die Urheberansprüche meiner Mitarbeiter in künstlerischer wie
                    geschäftlicher Hinsicht nach Gebühr eintreten kann.\pend
           \pstart
           Mit der Bitte um baldigen Bescheid und mit meinen besten Neujahrswünschen\pend
           \pstart
           Ihr hochachtungsvoll ergebener{\\[\baselineskip]}\spacefill\mbox{R. Dehmel.}\pend
           \leftskip=0em{}\endnumbering\briefempfaengerindex{Schnitzler, Arthur@\textsc{Schnitzler, Arthur}!zzzDehmel, Richard@\emph{von Richard Dehmel}!1902-01-013@{1. 1. 1902}|)be}\mylabel{h}  \normalsize

\doendnotes{C}
\bigskip
\vfill

\clearpage

\footnotesize

\lohead{\textsc{register}}

% Definiere theindex-Environment komplett neu ohne reledmac
\makeatletter
\renewenvironment{theindex}{%
  \section*{\indexname}%
  \setlength{\parindent}{0pt}%
  \setlength{\parskip}{0pt plus 0.3pt}%
  \let\item\@idxitem
}{%
  \clearpage
}
\makeatother

\IfFileExists{\jobname-pw.ind}{\input{\jobname-pw.ind}}{}

\end{document}

      