%% latex-korrekturansicht-vorspann.tex
%% Vorspann für die Korrekturansicht.
%% Lädt die gemeinsame Datei latex-vorspann.tex mit gesetztem Schalter.

\newif\ifkorrekturansicht
\korrekturansichttrue

\input{../tex-inputs/latex-vorspann}


               \section[Arthur Schnitzler an Richard Beer-Hofmann, 8. 3. 1893]{ Arthur Schnitzler an Richard Beer-Hofmann, 8. 3. 1893}\nopagebreak\mylabel{v}\rehead{ }\normalsize\beginnumbering\briefempfaengerindex{Beer-Hofmann, Richard@\textsc{Beer-Hofmann, Richard}!zzzSchnitzler, Arthur@\emph{von Arthur Schnitzler}!1893-03-081@{8. 3. 1893}|(be} \toendnotes[C]{\smallbreak\pagebreak[2]} \Standort{YCGL, MSS 31.}
\physDesc{Brief, 1 Blatt, 4 Seiten, Umschlag
\newline{}Handschrift: blaue Tinte, deutsche Kurrent\newline{}Versand: 1) Stempel: »\nobreak{}\oindex{Opatija@\textbf{Opatija}, \emph{http://www.geonames.org/ontologyP.PPLA2}|pwk}Abbazia, 9 3 93\nobreak{}«.  2) Stempel: »\nobreak{}10/3. 93, 11½V–1N\nobreak{}«. }\buchAbdrucke{\weitereDrucke{Arthur Schnitzler, Richard Beer-Hofmann: \emph{Briefwechsel 1891–1931}. Hg. Konstanze Fliedl. Wien, Zürich: \emph{Europaverlag} 1992, S. 43.} }\toendnotes[C]{\smallbreak}\pstart{}{\pb}\textsc{Herrn Doctor Richard Beer Hofmann}\pend{}\pstart{}\textsc{\textcolor{pink}{Wien}{}\ledrightnote{\textcolor{pink}{Wien}}.}\pend{}\pstart{}\textsc{\textcolor{pink}{I Wollzeile 15.}{}\ledrightnote{\textcolor{pink}{Wollzeile}}}.\pend{}{\bigskip}\pstart{}{\pb}Lieber Richard,\pend\pstart
           ich habe eine Bitte an Sie. Wollen Sie die Liebenswürdigkeit haben, mir für \uuline{So{\geminationn}tag} Abend einen \textcolor{green}{Sitz}{}\ledrightnote{→\textcolor{green}{Aus der Vorstadt. Volksstück mit Gesang in 3 Acten}} ins \textcolor{pink}{Volkstheater}{}\ledrightnote{\textcolor{pink}{Volkstheater}} zu beſorgen? Gern ginge ich mit Ihnen, Sie
               werden aber wohl \label{K_L00188_1v}\edtext{Samſtag}{\lemma{\textnormal{\emph{Samſtag}}}\Cendnote{\textnormal{\emph{\textcolor{green}{Aus der Vorstadt}} hatte am
                     11. 3. 1893 Uraufführung.}}}\label{K_L00188_1h} gehn? – Vielleicht
               ſitzt \textcolor{blue}{\textsc{Loris}}{}\ledrightnote{\textcolor{blue}{Hugo von Hofmannsthal}} oder {\pb}\textcolor{blue}{\textsc{Salten}}{}\ledrightnote{\textcolor{blue}{Felix Salten}}{ }\introOben{}oder \textcolor{blue}{\textsc{Schwarzkopf}}{}\ledrightnote{\textcolor{blue}{Gustav Schwarzkopf}}\introOben{} an meiner Seite? –\pend
           \pstart
           Daſs ich den Sitz am liebsten Mittelgang Ecke, 1, 2, 3, oder 4. Reihe hätte, brauch
               ich Ihnen nicht zu verſichern. – Finde ich ihn nicht bei mir, ſo ſchmeichle ich mir
               mit der Hoffnung, daſs Sie ihn mir am So{\geminationn}tag{ }Nachmittag um 5 Uhr perſönlich überbringen wollen; jedenfalls würde ich
                  {\pb}mich ſehr freuen, Sie und die oben genannten,
               wenn Ihr nichts beſſres vorhabt, auf eine Stunde bei mir zu ſehn. So{\geminationn}tag früh komm ich nämlich an.\pend
           \pstart
           Herzliche Grüße und entſchuldigen Sie die Mühe gütigſt! – Grüßen Sie mir auch die
               andern! Ich befinde mich ſehr wohl – {\pb}es iſt kein
               leerer Wahn, – was kein leerer Wahn, folgt mündlich.\pend
           \pstart
           Der Ihrige herzlichſt{\\[\baselineskip]}\spacefill\mbox{Arthur}\pend
           \leftskip=0em{}\pstart
           \textcolor{pink}{\textsc{Abbazia}}{}\ledrightnote{\textcolor{pink}{Opatija}}, 8. 3. 93.\pend
           \endnumbering\briefempfaengerindex{Beer-Hofmann, Richard@\textsc{Beer-Hofmann, Richard}!zzzSchnitzler, Arthur@\emph{von Arthur Schnitzler}!1893-03-081@{8. 3. 1893}|)be}\mylabel{h}  \normalsize

\doendnotes{C}
\bigskip
\vfill

\clearpage

\footnotesize

\lohead{\textsc{register}}

% Definiere theindex-Environment komplett neu ohne reledmac
\makeatletter
\renewenvironment{theindex}{%
  \section*{\indexname}%
  \setlength{\parindent}{0pt}%
  \setlength{\parskip}{0pt plus 0.3pt}%
  \let\item\@idxitem
}{%
  \clearpage
}
\makeatother

\IfFileExists{\jobname-pw.ind}{\input{\jobname-pw.ind}}{}

\end{document}

      