%% latex-korrekturansicht-vorspann.tex
%% Vorspann für die Korrekturansicht.
%% Lädt die gemeinsame Datei latex-vorspann.tex mit gesetztem Schalter.

\newif\ifkorrekturansicht
\korrekturansichttrue

\input{../tex-inputs/latex-vorspann}


               \section[Arthur Schnitzler an Hugo Hofmannsthal, 15. 7. 1929]{ Arthur Schnitzler an Hugo Hofmannsthal, 15. 7. 1929}\nopagebreak\mylabel{v}\rehead{ }\normalsize\beginnumbering\briefempfaengerindex{Hofmannsthal, Hugo von@\textsc{Hofmannsthal, Hugo von}!zzzSchnitzler, Arthur@\emph{von Arthur Schnitzler}!1929-07-151@{15. 7. 1929}|(be} \toendnotes[C]{\smallbreak\pagebreak[2]} \Standort{FDH, Hs-30885,5.}
\physDesc{Brief, 1 Blatt, 1 Seite
\newline{}Handschrift: schwarze Tinte, lateinische Kurrent}\buchAbdrucke{\weitereDrucke{1) Arthur Schnitzler: \emph{Letzter Brief an Hugo von Hofmannsthal.} In: \emph{Fischer Almanach. Das achtzigste Jahr} (1966).} \weitereDrucke{2) Hans-Ulrich Lindken: \emph{Arthur Schnitzler. Aspekte und Akzente. Materialien zu
                                Leben und Werk}. Frankfurt am Main, Bern, Göttingen: \emph{Peter Lang} 1984, S. 174 (Europäische Hochschulschriften, Reihe 1, Deutsche
                                Sprache und Literatur, 754).} }\toendnotes[C]{\smallbreak}\pstart
           \raggedleft{}{\pb}\textcolor{pink}{Wien}{}\ledrightnote{\textcolor{pink}{Wien}}{ }15. 7. 929.\pend
           \pstart
           mein lieber Hugo, in tiefster \textcolor{blue}{Antheilnahme}{}\ledrightnote{→\textcolor{blue}{Franz von Hofmannsthal}}, im Gefühl alter und immer neuer
                    Freundschaft umarme ich Sie. Ich bin für Sie da wann Sie wollen. Und auch \textcolor{blue}{Gerty}{}\ledrightnote{\textcolor{blue}{Gertrude von Hofmannsthal}} wie die \textcolor{blue}{Kinder}{}\ledrightnote{\textcolor{blue}{Christiane von Hofmannsthal}{\newline}\textcolor{blue}{Raimund von Hofmannsthal}} werden wissen wie ich ihr Leid mitempfinde.\pend
           \pstart
           mein lieber lieber Hugo auf Wiedersehen!{\\[\baselineskip]}Immer der Ihre{\\[\baselineskip]}\spacefill\mbox{Arthur}\pend
           \leftskip=0em{}\endnumbering\briefempfaengerindex{Hofmannsthal, Hugo von@\textsc{Hofmannsthal, Hugo von}!zzzSchnitzler, Arthur@\emph{von Arthur Schnitzler}!1929-07-151@{15. 7. 1929}|)be}\mylabel{h}  \normalsize

\doendnotes{C}
\bigskip
\vfill

\clearpage

\footnotesize

\lohead{\textsc{register}}

% Definiere theindex-Environment komplett neu ohne reledmac
\makeatletter
\renewenvironment{theindex}{%
  \section*{\indexname}%
  \setlength{\parindent}{0pt}%
  \setlength{\parskip}{0pt plus 0.3pt}%
  \let\item\@idxitem
}{%
  \clearpage
}
\makeatother

\IfFileExists{\jobname-pw.ind}{\input{\jobname-pw.ind}}{}

\end{document}

      