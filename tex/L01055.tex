%% latex-korrekturansicht-vorspann.tex
%% Vorspann für die Korrekturansicht.
%% Lädt die gemeinsame Datei latex-vorspann.tex mit gesetztem Schalter.

\newif\ifkorrekturansicht
\korrekturansichttrue

\input{../tex-inputs/latex-vorspann}


               \section[Arthur Schnitzler an Richard Beer-Hofmann, 15. 7. 1900]{ Arthur Schnitzler an Richard Beer-Hofmann, 15. 7. 1900}\nopagebreak\mylabel{v}\rehead{ }\normalsize\beginnumbering\briefempfaengerindex{Beer-Hofmann, Richard@\textsc{Beer-Hofmann, Richard}!zzzSchnitzler, Arthur@\emph{von Arthur Schnitzler}!1900-07-151@{15. 7. 1900}|(be} \toendnotes[C]{\smallbreak\pagebreak[2]} \Standort{YCGL, MSS 31.}
\physDesc{Postkarte
\newline{}Handschrift: Bleistift, deutsche Kurrent\newline{}Versand: 1) Stempel: »\nobreak{}\oindex{Payerbach@\textbf{Payerbach}, \emph{Besiedelter Ort (A.BSO)}|pwk}Payerbach, 15. 7. \textcolor{gray}{00}, 7–12V\nobreak{}«.  2) Stempel: »\nobreak{}\oindex{Altaussee@\textbf{Altaussee}, \emph{http://www.geonames.org/ontologyA.ADM3}|pwk}Alt-Aussee, 16/7 00\nobreak{}«. \newline{}Ordnung: mit Bleistift von unbekannter Hand datiert:
                                 »15. 7.« }\buchAbdrucke{\weitereDrucke{Arthur Schnitzler, Richard Beer-Hofmann: \emph{Briefwechsel 1891–1931}. Hg. Konstanze Fliedl. Wien, Zürich: \emph{Europaverlag} 1992, S. 148–149.} }\pstart{}{\pb}Hrn \textsc{Dr. Richard
                     Beer-Hofmann}\pend{}\pstart{}\textsc{\textcolor{pink}{Altaussee}{}\ledrightnote{\textcolor{pink}{Altaussee}}}\pend{}{\bigskip}\pstart
           \noindent{}{\pb}lieber Richard, eben ko{\geminationm}t Ihr Brief, alſo nehm ich meine geſtrige Karte
               zurück. – \textcolor{blue}{M.}{}\ledrightnote{\textcolor{blue}{Oskar Mayer}} hat mir aus \textcolor{pink}{\textsc{Lev}.}{}\ledrightnote{\textcolor{pink}{Levico Terme}} bereits vor 8 Tagen eine längere begeiſterte
               Sauce über \textcolor{blue}{B.}{}\ledrightnote{\textcolor{blue}{Hermine von Schaffgotsch}} geſchrieben. Es iſt wirklich
               ziemlich egal. Denken Sie doch nach\strikeout{,} wie wir von \textcolor{pink}{Klöſterle}{}\ledrightnote{\textcolor{pink}{Klösterle}} oder wie das heißt weiter ko{\geminationm}en
               ſollen. – Vielleicht fahren wir zusa{\geminationm}en von \textcolor{pink}{Auſſee}{}\ledrightnote{\textcolor{pink}{Bad Aussee}}{ }\textsc{resp}. \textcolor{pink}{Iſchl}{}\ledrightnote{\textcolor{pink}{Bad Ischl}} nach \textcolor{pink}{Insbruck}{}\ledrightnote{\textcolor{pink}{Innsbruck}} (über \textcolor{brown}{\textsc{Svatek}}{}\ledrightnote{\textcolor{brown}{Wenzel Swatek}})\pend
           \pstart
           – Von der nächſten Zeit weiſs ich noch i{\geminationm}er nichts. (Es geht den meisten Menschen so.) –\pend
           \pstart
           Herzlichst{\\[\baselineskip]}Ihr\spacefill\mbox{A.}\pend
           \leftskip=0em{}\endnumbering\briefempfaengerindex{Beer-Hofmann, Richard@\textsc{Beer-Hofmann, Richard}!zzzSchnitzler, Arthur@\emph{von Arthur Schnitzler}!1900-07-151@{15. 7. 1900}|)be}\mylabel{h}  \normalsize

\doendnotes{C}
\bigskip
\vfill

\clearpage

\footnotesize

\lohead{\textsc{register}}

% Definiere theindex-Environment komplett neu ohne reledmac
\makeatletter
\renewenvironment{theindex}{%
  \section*{\indexname}%
  \setlength{\parindent}{0pt}%
  \setlength{\parskip}{0pt plus 0.3pt}%
  \let\item\@idxitem
}{%
  \clearpage
}
\makeatother

\IfFileExists{\jobname-pw.ind}{\input{\jobname-pw.ind}}{}

\end{document}

      