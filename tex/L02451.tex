%% latex-korrekturansicht-vorspann.tex
%% Vorspann für die Korrekturansicht.
%% Lädt die gemeinsame Datei latex-vorspann.tex mit gesetztem Schalter.

\newif\ifkorrekturansicht
\korrekturansichttrue

\input{../tex-inputs/latex-vorspann}


               \section[Stefan Großmann an Arthur Schnitzler, {[}nach dem 25. 9. 1925{]}]{ Stefan Großmann an Arthur Schnitzler, {[}nach dem 25. 9. 1925{]}}\nopagebreak\mylabel{v}\rehead{ }\normalsize\beginnumbering\briefempfaengerindex{Schnitzler, Arthur@\textsc{Schnitzler, Arthur}!zzzGrossmann, Stefan@\emph{von Stefan Großmann}!1925-09-251@{{[}nach dem
                  25. 9. 1925{]}}|(be} \toendnotes[C]{\smallbreak\pagebreak[2]} \Standort{DLA, A:Schnitzler, HS.NZ85.1.3232.}
\physDesc{Brief, 1 Blatt, 1 Seite
\newline{}Schreibmaschine
\newline{}Handschrift Arthur Schnitzler: roter Buntstift, deutsche Kurrent (\noindent{}Nummerierung:
                                    »25«; eine Unterstreichung)}\toendnotes[C]{\smallbreak}\pstart
           \noindent{}\centering{}{\pb}\textcolor{gray}{\textbf{\textcolor{brown}{Das Tage-Buch}{}\ledrightnote{\textcolor{brown}{Das Tage-Buch}}}}\pend
           \pstart
           \noindent{}\centering{}\textcolor{gray}{\textbf{\emph{Herausgeber: Stefan Großmann und \textcolor{blue}{Leopold Schwarzschild}{}\ledrightnote{\textcolor{blue}{Leopold Schwarzschild}}}}}\pend
           \pstart
           \noindent{}\centering{}\textcolor{gray}{\textbf{Tagebuchverlag m. b. H., \textcolor{pink}{Berlin
                        SW 19}{}\ledrightnote{\textcolor{pink}{Berlin}}}}\pend
           \pstart
           \noindent{}\centering{}\textcolor{gray}{\textbf{\textcolor{pink}{BEUTHSTRASSE 19}{}\ledrightnote{\textcolor{pink}{Beuthstrasse}}}}\pend
           \pstart
           \noindent{}\centering{}\textcolor{gray}{\textbf{\emph{Telegramm-Adresse: Tagebuch \textcolor{pink}{Berlin}{}\ledrightnote{\textcolor{pink}{Berlin}} ⋅ Fernsprecher: Merkur 8790–8792}}}\pend
           \pstart
           \noindent{}\centering{}\textcolor{gray}{\textbf{\emph{\so{Sprechstunde der Redaktion: 12–1 Uhr}}}}\pend
           \pstart
           \noindent{}\centering{}\textcolor{gray}{\textbf{*}}\pend
           \pstart
           \noindent{}\raggedleft{}Herrn\pend
           \pstart
           \noindent{}\raggedleft{}Dr. Arnold \so{Schnitzler}\pend
           \pstart
           \noindent{}\raggedleft{}\textcolor{pink}{\so{Wien } XVIII}{}\ledrightnote{\textcolor{pink}{XVIII., Währing}}\pend
           \pstart
           \noindent{}\raggedleft{}\textcolor{pink}{Sternwartestr. 71}{}\ledrightnote{\textcolor{pink}{Sternwartestraße}}. \pend
           \pstart\center{}Sehr verehrter Herr Doktor!\pend\pstart
           Herzlichen Dank für Ihre prinzipielle Zusage. Mein Dank wäre noch grösser, wenn Sie
               sich entschliessen würden, recht bald die nun versprochenen \label{K_L02451_1v}\edtext{Beiträge}{\lemma{\textnormal{\emph{Beiträge}}}\Cendnote{\textnormal{\textcolor{blue}{Schnitzler} hielt seine Zusage nicht. Ein halbes
                  Jahr später erschien ein \textcolor{green}{Nachdruck
                     von Aphorismen} im \emph{\textcolor{green}{Tage-Buch}}.}}}\label{K_L02451_1h} zu
               senden. Ich wäre Ihnen für die Uebersendung von Beiträgen gerade jetzt, im Herbst,
               ganz besonders dankbar.\pend
           \pstart
           Mit ergebensten Grüssen{\\[\baselineskip]}Ihr{\\[\baselineskip]}\spacefill\mbox{{[}hs.:{]} Stefan Großmann}\pend
           \leftskip=0em{}\endnumbering\briefempfaengerindex{Schnitzler, Arthur@\textsc{Schnitzler, Arthur}!zzzGrossmann, Stefan@\emph{von Stefan Großmann}!1925-09-251@{{[}nach dem
                  25. 9. 1925{]}}|)be}\mylabel{h}  \normalsize

\doendnotes{C}
\bigskip
\vfill

\clearpage

\footnotesize

\lohead{\textsc{register}}

% Definiere theindex-Environment komplett neu ohne reledmac
\makeatletter
\renewenvironment{theindex}{%
  \section*{\indexname}%
  \setlength{\parindent}{0pt}%
  \setlength{\parskip}{0pt plus 0.3pt}%
  \let\item\@idxitem
}{%
  \clearpage
}
\makeatother

\IfFileExists{\jobname-pw.ind}{\input{\jobname-pw.ind}}{}

\end{document}

      