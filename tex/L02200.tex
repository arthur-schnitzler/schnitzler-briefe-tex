%% latex-korrekturansicht-vorspann.tex
%% Vorspann für die Korrekturansicht.
%% Lädt die gemeinsame Datei latex-vorspann.tex mit gesetztem Schalter.

\newif\ifkorrekturansicht
\korrekturansichttrue

\input{../tex-inputs/latex-vorspann}


               \section[Arthur Schnitzler an Georg Brandes, 18. 12. 1914]{ Arthur Schnitzler an Georg Brandes, 18. 12. 1914}\nopagebreak\mylabel{v}\rehead{ }\normalsize\beginnumbering\briefempfaengerindex{Brandes, Georg@\textsc{Brandes, Georg}!zzzSchnitzler, Arthur@\emph{von Arthur Schnitzler}!1914-12-181@{18. 12. 1914}|(be} \toendnotes[C]{\smallbreak\pagebreak[2]} \Standort{Kopenhagen, Det Kongelige Bibliotek, Georg Brandes Arkiv, box 125.}
\physDesc{Bildpostkarte
\newline{}Handschrift: schwarze Tinte, deutsche Kurrent\newline{}Versand: 1) Stempel: »\nobreak{}18. XII. 14\nobreak{}«.  2) Stempel: »\nobreak{}Wien 1, Überprüft\nobreak{}«. \newline{}Ordnung: mit Bleistift von unbekannter Hand nummeriert:
                                    »37« \newline{}Zusatz: Postkartenmotiv mit \textcolor{blue}{Olga} und
                                    \textcolor{blue}{Heinrich} links vor dem Haus
                                 und Schnitzler und \textcolor{blue}{Lili} auf dem
                                 Söller }\buchAbdrucke{\weitereDrucke{Georg Brandes, Arthur Schnitzler: \emph{Ein Briefwechsel}. Hg. Kurt Bergel. Bern: \emph{Francke} 1956, S. 114.} }\pstart{}{\pb}\textsc{Hrn Georg Brandes}\pend{}\pstart{}\textcolor{pink}{\textsc{Kopenhagen}}{}\ledrightnote{\textcolor{pink}{Kopenhagen}}\pend{}{\bigskip}\pstart
           \noindent{}\centering{}\textcolor{gray}{\textbf{{\pb}\textcolor{pink}{Wien, XVIII, Sternwartestr. 71}{}\ledrightnote{\textcolor{pink}{Sternwartestraße}}.}}\pend
           \pstart
           \raggedleft{}{\pb}18. 12. 914\pend
           \pstart
           mein lieber u verehrter Freund, ſeien Sie zu den Feiertagen und dem
                  ko{\geminationm}enden Jahr wieder einmal herzlichſst gegrüßt.
               Heute eben ko{\geminationm}en beſonders gute Nachrichten aus dem
               Nordoſten – vielleicht iſt es mit all dem Graun doch früher zu Ende als wie
               befürchtet. Hier iſt alles in ſchönſster Ordnung, – und man iſt voll Zuverſicht. Ein
               Wort von Ihnen thäte mir wohl! Wir alle {\pb}denken
               Ihrer in Freundſchaft!\pend
           \pstart
           Von Herzen Ihr{\\[\baselineskip]}\spacefill\mbox{Arthur Schnitzler.}\pend
           \leftskip=0em{}\endnumbering\briefempfaengerindex{Brandes, Georg@\textsc{Brandes, Georg}!zzzSchnitzler, Arthur@\emph{von Arthur Schnitzler}!1914-12-181@{18. 12. 1914}|)be}\mylabel{h}  \normalsize

\doendnotes{C}
\bigskip
\vfill

\clearpage

\footnotesize

\lohead{\textsc{register}}

% Definiere theindex-Environment komplett neu ohne reledmac
\makeatletter
\renewenvironment{theindex}{%
  \section*{\indexname}%
  \setlength{\parindent}{0pt}%
  \setlength{\parskip}{0pt plus 0.3pt}%
  \let\item\@idxitem
}{%
  \clearpage
}
\makeatother

\IfFileExists{\jobname-pw.ind}{\input{\jobname-pw.ind}}{}

\end{document}

      