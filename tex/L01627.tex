%% latex-korrekturansicht-vorspann.tex
%% Vorspann für die Korrekturansicht.
%% Lädt die gemeinsame Datei latex-vorspann.tex mit gesetztem Schalter.

\newif\ifkorrekturansicht
\korrekturansichttrue

\input{../tex-inputs/latex-vorspann}


               \section[Hugo von Hofmannsthal an Arthur Schnitzler, {[}12. 9. 1906{]}]{ Hugo von Hofmannsthal an Arthur Schnitzler, {[}12. 9. 1906{]}}\nopagebreak\mylabel{v}\rehead{ }\normalsize\beginnumbering\briefempfaengerindex{Schnitzler, Arthur@\textsc{Schnitzler, Arthur}!zzzHofmannsthal, Hugo von@\emph{von Hugo von Hofmannsthal}!1906-09-121@{12. 9. 1906}|(be} \toendnotes[C]{\smallbreak\pagebreak[2]} \Standort{CUL, Schnitzler, B 43.}
\physDesc{Telegramm
\newline{}maschinell\newline{}Versand: Stempel des Telegrafenbeamten: »\textcolor{blue}{Steiner}« 
\newline{}Schnitzler: mit Bleistift datiert »12/9 906« \newline{}Ordnung: 1) beschnitten 2) mit Bleistift von unbekannter Hand nummeriert:
                              »265a«}\buchAbdrucke{\weitereDrucke{Hugo von Hofmannsthal, Arthur Schnitzler: \emph{Briefwechsel}. Hg. Therese Nickl und Heinrich Schnitzler. Frankfurt am Main: \emph{S. Fischer} 1964, S. 223.} }\pstart
           {\pb}fr \textcolor{pink}{st. gilgen}{}\ledrightnote{\textcolor{pink}{St. Gilgen}} 380 24 11/25 m\pend
           \pstart
           schlagen vor begegnung \textcolor{pink}{graz}{}\ledrightnote{\textcolor{pink}{Graz}} das wier
               nicht kennen haetten grosse freude haben raeder mit erwarten postwendent praecise
               vorschlaege freuen uns\pend
           \pstart \spacefill\mbox{= hugo =}\pend{}\endnumbering\briefempfaengerindex{Schnitzler, Arthur@\textsc{Schnitzler, Arthur}!zzzHofmannsthal, Hugo von@\emph{von Hugo von Hofmannsthal}!1906-09-121@{12. 9. 1906}|)be}\mylabel{h}  \normalsize

\doendnotes{C}
\bigskip
\vfill

\clearpage

\footnotesize

\lohead{\textsc{register}}

% Definiere theindex-Environment komplett neu ohne reledmac
\makeatletter
\renewenvironment{theindex}{%
  \section*{\indexname}%
  \setlength{\parindent}{0pt}%
  \setlength{\parskip}{0pt plus 0.3pt}%
  \let\item\@idxitem
}{%
  \clearpage
}
\makeatother

\IfFileExists{\jobname-pw.ind}{\input{\jobname-pw.ind}}{}

\end{document}

      