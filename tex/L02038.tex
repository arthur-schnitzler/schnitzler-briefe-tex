%% latex-korrekturansicht-vorspann.tex
%% Vorspann für die Korrekturansicht.
%% Lädt die gemeinsame Datei latex-vorspann.tex mit gesetztem Schalter.

\newif\ifkorrekturansicht
\korrekturansichttrue

\input{../tex-inputs/latex-vorspann}


               \section[Hugo von Hofmannsthal an Arthur Schnitzler, 16. 10. {[}1911?{]}]{ Hugo von Hofmannsthal an Arthur Schnitzler, 16. 10. {[}1911?{]}}\nopagebreak\mylabel{v}\rehead{ }\normalsize\beginnumbering\briefempfaengerindex{Schnitzler, Arthur@\textsc{Schnitzler, Arthur}!zzzHofmannsthal, Hugo von@\emph{von Hugo von Hofmannsthal}!1911-10-161@{16. 10. {[}1911?{]}}|(be} \toendnotes[C]{\smallbreak\pagebreak[2]} \Standort{CUL, Schnitzler, B 43.}
\physDesc{Telegramm
\newline{}maschinell\newline{}Ordnung: beschnitten }\buchAbdrucke{\weitereDrucke{Hugo von Hofmannsthal, Arthur Schnitzler: \emph{Briefwechsel}. Hg. Therese Nickl und Heinrich Schnitzler. Frankfurt am Main: \emph{S. Fischer} 1964, S. 263.} }\toendnotes[C]{\smallbreak}\pstart
           \noindent{}{\pb}\textcolor{pink}{neubeuern}{}\ledrightnote{\textcolor{pink}{Neubeuern}} 25 18/17 16/10{ }4 15 n\pend
           \pstart
           freue mich unendlich \label{K_L02038_1v}\edtext{doppelten
                        Erfolg}{\lemma{\textnormal{\emph{doppelten
                        Erfolg}}}\Cendnote{\textnormal{Am
                            14. 10. 1911 fanden die die Uraufführungen von \emph{\textcolor{green}{Das weite Land}} am \textcolor{pink}{Burgtheater} und am \emph{\textcolor{brown}{Berliner Lessingtheater}}, aber auch in sieben weiteren Städten
                        statt.}}}\label{K_L02038_1h} so schoenen lieben \textcolor{green}{werkes}{}\ledrightnote{→\textcolor{green}{Das weite Land. Tragikomödie in fünf Akten}} auf wiedersehen baldigst\pend
           \pstart
           \raggedleft{}= hugo +\pend
           \endnumbering\briefempfaengerindex{Schnitzler, Arthur@\textsc{Schnitzler, Arthur}!zzzHofmannsthal, Hugo von@\emph{von Hugo von Hofmannsthal}!1911-10-161@{16. 10. {[}1911?{]}}|)be}\mylabel{h}  \normalsize

\doendnotes{C}
\bigskip
\vfill

\clearpage

\footnotesize

\lohead{\textsc{register}}

% Definiere theindex-Environment komplett neu ohne reledmac
\makeatletter
\renewenvironment{theindex}{%
  \section*{\indexname}%
  \setlength{\parindent}{0pt}%
  \setlength{\parskip}{0pt plus 0.3pt}%
  \let\item\@idxitem
}{%
  \clearpage
}
\makeatother

\IfFileExists{\jobname-pw.ind}{\input{\jobname-pw.ind}}{}

\end{document}

      