%% latex-korrekturansicht-vorspann.tex
%% Vorspann für die Korrekturansicht.
%% Lädt die gemeinsame Datei latex-vorspann.tex mit gesetztem Schalter.

\newif\ifkorrekturansicht
\korrekturansichttrue

\input{../tex-inputs/latex-vorspann}


               \section[Arthur Schnitzler an Hermann Bahr, 28. 9. 1905]{ Arthur Schnitzler an Hermann Bahr, 28. 9. 1905}\nopagebreak\mylabel{v}\rehead{ }\normalsize\beginnumbering\briefempfaengerindex{Bahr, Hermann@\textsc{Bahr, Hermann}!zzzSchnitzler, Arthur@\emph{von Arthur Schnitzler}!1905-09-281@{28. 9. 1905}|(be} \toendnotes[C]{\smallbreak\pagebreak[2]} \Standort{TMW, HS AM 23373 Ba.}
\physDesc{Brief, 1 Blatt, 2 Seiten
\newline{}Handschrift: schwarze Tinte, deutsche Kurrent\newline{}Ordnung: Lochung }\buchAbdrucke{\weitereDrucke{1) \emph{28. 9. 1905.} In: Arthur Schnitzler: \emph{The Letters of Arthur Schnitzler to Hermann Bahr}. Edited, annotated, and with an introduction, by Donald G.
                        Daviau. Chapel Hill: \emph{The University of North Carolina Press} 1978, S. 92 (University of North Carolina studies in the Germanic languages
                        and literatures, 89).} \weitereDrucke{2) Hermann Bahr, Arthur Schnitzler: \emph{Briefwechsel, Aufzeichnungen, Dokumente (1891–1931)}. Hg. Kurt Ifkovits und Martin Anton Müller. Göttingen: \emph{Wallstein} 2018, S. 355.} }\toendnotes[C]{\smallbreak}\pstart
           \noindent{}\centering{}{\pb}\textsc{\textcolor{pink}{Wien XVIII}{}\ledrightnote{\textcolor{pink}{XVIII., Währing}}}\pend
           \pstart
           \raggedleft{}28. 9. 905\pend
           \pstart
           lieber Hermann, nun fangen meine \label{K_L01552_1v}\edtext{\textcolor{green}{Proben}{}\ledrightnote{→\textcolor{green}{Zwischenspiel. Komödie in drei Akten}}}{\lemma{\textnormal{\emph{Proben}}}\Cendnote{\textnormal{zu \emph{\textcolor{green}{Zwischenspiel}}}}}\label{K_L01552_1h} an, da ich eben vom \textcolor{pink}{Se{\geminationm}ering}{}\ledrightnote{\textcolor{pink}{Semmering}} zurück bin, und mit den Vormittagen ist
               es wieder nichts. Könnte man ſich nicht doch vielleicht an einem Abend, in \textcolor{pink}{Hietzing}{}\ledrightnote{\textcolor{pink}{XIII., Hietzing}} etwa, zum Nachtmahl, wenn du einmal kein
               Theater haſt, treffen? Anfang nächſter Woche? – Sonſt müßten wir {\pb}unſer Wiederſehen auf
               die zweite Oktoberhälfte verſchieben. Was mir ſehr leid wäre.\pend
           \pstart
           Bitte dich ſchick mir nur gütigſt den »\textcolor{green}{Ruf des
                  Lebens}{}\ledrightnote{\textcolor{green}{Der Ruf des Lebens. Schauspiel in drei Akten}}« zurück. –\pend
           \pstart
           Herzlichſt, mit vielen Grüßen auch von \textcolor{blue}{Olga}{}\ledrightnote{\textcolor{blue}{Olga Schnitzler}}{\\[\baselineskip]}dein{\\[\baselineskip]}\spacefill\mbox{Arthur}\pend
           \leftskip=0em{}\endnumbering\briefempfaengerindex{Bahr, Hermann@\textsc{Bahr, Hermann}!zzzSchnitzler, Arthur@\emph{von Arthur Schnitzler}!1905-09-281@{28. 9. 1905}|)be}\mylabel{h}  \normalsize

\doendnotes{C}
\bigskip
\vfill

\clearpage

\footnotesize

\lohead{\textsc{register}}

% Definiere theindex-Environment komplett neu ohne reledmac
\makeatletter
\renewenvironment{theindex}{%
  \section*{\indexname}%
  \setlength{\parindent}{0pt}%
  \setlength{\parskip}{0pt plus 0.3pt}%
  \let\item\@idxitem
}{%
  \clearpage
}
\makeatother

\IfFileExists{\jobname-pw.ind}{\input{\jobname-pw.ind}}{}

\end{document}

      