%% latex-korrekturansicht-vorspann.tex
%% Vorspann für die Korrekturansicht.
%% Lädt die gemeinsame Datei latex-vorspann.tex mit gesetztem Schalter.

\newif\ifkorrekturansicht
\korrekturansichttrue

\input{../tex-inputs/latex-vorspann}


               \section[Arthur Schnitzler an Richard Beer-Hofmann, 1. 12. 1892]{ Arthur Schnitzler an Richard Beer-Hofmann, 1. 12. 1892}\nopagebreak\mylabel{v}\rehead{ }\normalsize\beginnumbering\briefempfaengerindex{Beer-Hofmann, Richard@\textsc{Beer-Hofmann, Richard}!zzzSchnitzler, Arthur@\emph{von Arthur Schnitzler}!1892-12-011@{1. 12. 1892}|(be} \toendnotes[C]{\smallbreak\pagebreak[2]} \Standort{YCGL, MSS 31.}
\physDesc{Brief, 1 Blatt, 2 Seiten, Umschlag
\newline{}Handschrift: Bleistift, deutsche Kurrent\newline{}Versand: ohne postalischen Übermittlungsvermerk }\buchAbdrucke{\weitereDrucke{Arthur Schnitzler, Richard Beer-Hofmann: \emph{Briefwechsel 1891–1931}. Hg. Konstanze Fliedl. Wien, Zürich: \emph{Europaverlag} 1992, S. 40.} }\pstart{}{\pb}\textsc{Herrn Dr Rich Beer Hofmann}\pend{}\pstart{}\textsc{\textcolor{pink}{Wien}{}\ledrightnote{\textcolor{pink}{Wien}}.}\pend{}\pstart{}\textsc{\textcolor{pink}{I Wollzeile 15}{}\ledrightnote{\textcolor{pink}{Wollzeile}}}.\pend{}{\bigskip}\pstart{}{\pb}Lieber Richard, \pend\pstart
           hier ſchickt mir \textcolor{blue}{\textsc{Beraton}}{}\ledrightnote{\textcolor{blue}{Ferry Bératon}} den Sitz für Sie. Ich
               denke, wir treffen uns zwiſchen 6 und ½ 7 im \textcolor{pink}{\textsc{Grstdl}}{}\ledrightnote{\textcolor{pink}{Café Griensteidl}} und fahren zuſa{\geminationm}en hinaus. Ich zweifle
               nicht, daſs uns da ein ſehr billiger \introOben{}u praktischer\introOben{} Modus
               einfallen wird; z. B. mit dem Fiaker bis zur \textcolor{pink}{\textsc{Elisabeth}brücke}{}\ledrightnote{\textcolor{pink}{Elisabethbrücke}} und da{\geminationn}{ }{\pb}mit der Tram. –\pend
           \pstart
           Herzlich Ihr{\\[\baselineskip]}\spacefill\mbox{Arthur.}\pend
           \leftskip=0em{}\pstart
           1/12 92.
               \pend
           \endnumbering\briefempfaengerindex{Beer-Hofmann, Richard@\textsc{Beer-Hofmann, Richard}!zzzSchnitzler, Arthur@\emph{von Arthur Schnitzler}!1892-12-011@{1. 12. 1892}|)be}\mylabel{h}  \normalsize

\doendnotes{C}
\bigskip
\vfill

\clearpage

\footnotesize

\lohead{\textsc{register}}

% Definiere theindex-Environment komplett neu ohne reledmac
\makeatletter
\renewenvironment{theindex}{%
  \section*{\indexname}%
  \setlength{\parindent}{0pt}%
  \setlength{\parskip}{0pt plus 0.3pt}%
  \let\item\@idxitem
}{%
  \clearpage
}
\makeatother

\IfFileExists{\jobname-pw.ind}{\input{\jobname-pw.ind}}{}

\end{document}

      