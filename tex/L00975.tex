%% latex-korrekturansicht-vorspann.tex
%% Vorspann für die Korrekturansicht.
%% Lädt die gemeinsame Datei latex-vorspann.tex mit gesetztem Schalter.

\newif\ifkorrekturansicht
\korrekturansichttrue

\input{../tex-inputs/latex-vorspann}


               \section[Arthur Schnitzler an Richard Beer-Hofmann, 17. 9. 1899]{ Arthur Schnitzler an Richard Beer-Hofmann, 17. 9. 1899}\nopagebreak\mylabel{v}\rehead{ }\normalsize\beginnumbering\briefempfaengerindex{Beer-Hofmann, Richard@\textsc{Beer-Hofmann, Richard}!zzzSchnitzler, Arthur@\emph{von Arthur Schnitzler}!1899-09-171@{17. 9. 1899}|(be} \toendnotes[C]{\smallbreak\pagebreak[2]} \Standort{CUL, Schnitzler, B 8.}
\physDesc{Bildpostkarte
\newline{}Handschrift: Bleistift, deutsche Kurrent\newline{}Versand: 1) Stempel: »\nobreak{}\oindex{Zirndorf@\textbf{Zirndorf}, \emph{http://www.geonames.org/ontologyP.PPL}|pwk}Zirndorf, 17. {[}Sep{]} 99, 6–7\textcolor{gray}{NM}\nobreak{}«.  2) Stempel: »\nobreak{}\oindex{Vahrn@\textbf{Vahrn}, \emph{Besiedelter Ort (A.BSO)}|pwk}{[}Vahr{]}n, 18. 9. 99\nobreak{}«. \newline{}Ordnung: mit Bleistift von unbekannter Hand datiert: »17. 9.« }\buchAbdrucke{\weitereDrucke{Arthur Schnitzler, Richard Beer-Hofmann: \emph{Briefwechsel 1891–1931}. Hg. Konstanze Fliedl. Wien, Zürich: \emph{Europaverlag} 1992, S. 137.} }\pstart{}{\pb}Hrn \textsc{Dr. Rich.
                     Beer-Hofmann}\pend{}\pstart{}\textcolor{pink}{\textsc{Vahrn}}{}\ledrightnote{\textcolor{pink}{Vahrn}}\pend{}\pstart{}bei \textsc{\textcolor{pink}{Brixen}{}\ledrightnote{\textcolor{pink}{Brixen}}}\pend{}\pstart{}\textcolor{pink}{\textsc{Tirol}}{}\ledrightnote{\textcolor{pink}{Tirol}}\pend{}{\bigskip}\pstart
           \noindent{}\centering{}\textcolor{gray}{\textbf{{\pb}Gruss aus \textcolor{pink}{Zirndorf}{}\ledrightnote{\textcolor{pink}{Zirndorf}}. \textcolor{pink}{Alte Veste}{}\ledrightnote{\textcolor{pink}{Alte Veste}}. \textcolor{pink}{Scheidlers Haus}{}\ledrightnote{\textcolor{pink}{Scheidlers Haus}}}}\pend
           \pstart
           Daſs ich einmal hieher käme, hab ich nicht geahnt.\pend
           \pstart
           \substVorne{}\textsuperscript{\textcolor{gray}{Send}}\substDazwischen{}Schrei\substHinten{}ben Sie mir nach \textcolor{pink}{Frankfurt}{}\ledrightnote{\textcolor{pink}{Frankfurt am Main}}{ }\textsc{post. rest.}\pend
           \endnumbering\briefempfaengerindex{Beer-Hofmann, Richard@\textsc{Beer-Hofmann, Richard}!zzzSchnitzler, Arthur@\emph{von Arthur Schnitzler}!1899-09-171@{17. 9. 1899}|)be}\mylabel{h}  \normalsize

\doendnotes{C}
\bigskip
\vfill

\clearpage

\footnotesize

\lohead{\textsc{register}}

% Definiere theindex-Environment komplett neu ohne reledmac
\makeatletter
\renewenvironment{theindex}{%
  \section*{\indexname}%
  \setlength{\parindent}{0pt}%
  \setlength{\parskip}{0pt plus 0.3pt}%
  \let\item\@idxitem
}{%
  \clearpage
}
\makeatother

\IfFileExists{\jobname-pw.ind}{\input{\jobname-pw.ind}}{}

\end{document}

      