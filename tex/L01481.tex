%% latex-korrekturansicht-vorspann.tex
%% Vorspann für die Korrekturansicht.
%% Lädt die gemeinsame Datei latex-vorspann.tex mit gesetztem Schalter.

\newif\ifkorrekturansicht
\korrekturansichttrue

\input{../tex-inputs/latex-vorspann}


               \section[Arthur Schnitzler an Richard Beer-Hofmann, 23. 12. 1904]{ Arthur Schnitzler an Richard Beer-Hofmann, 23. 12. 1904}\nopagebreak\mylabel{v}\rehead{ }\normalsize\beginnumbering\briefempfaengerindex{Beer-Hofmann, Richard@\textsc{Beer-Hofmann, Richard}!zzzSchnitzler, Arthur@\emph{von Arthur Schnitzler}!1904-12-231@{23. 12. 1904}|(be} \toendnotes[C]{\smallbreak\pagebreak[2]} \Standort{YCGL, MSS 31.}
\physDesc{Telegramm
\newline{}maschinell\newline{}Versand: 1) Stempel: »\nobreak{}\oindex{Berlin@\textbf{Berlin}, \emph{https://www.geonames.org/ontologyP.PPLC}|pwk}Berlin N.W. 6, 23. 12. 04, 11–V\nobreak{}«.  2) »\textcolor{gray}{\textbf{\textbf{Aufgenommen} von}}{ }W{ }\textcolor{gray}{\textbf{den}}{ }23\textcolor{gray}{\textbf{/}}12{ }\textcolor{gray}{\textbf{um}}{ }10 \textcolor{gray}{\textbf{Uhr}} 30 \textcolor{gray}{\textbf{M.}}n{ }\textcolor{gray}{\textbf{durch}}{ }\textcolor{gray}{Hw}«}\buchAbdrucke{\weitereDrucke{Arthur Schnitzler, Richard Beer-Hofmann: \emph{Briefwechsel 1891–1931}. Hg. Konstanze Fliedl. Wien, Zürich: \emph{Europaverlag} 1992, S. 171.} }\toendnotes[C]{\smallbreak}\pstart{}{\pb}richard beerhofmann berlin\pend{}\pstart{}\textcolor{pink}{neues theater.=}{}\ledrightnote{\textcolor{pink}{Neues Theater}}\pend{}{\bigskip}\pstart
           {\pb}\textcolor{gray}{\textbf{Telegramm aus}} de \textcolor{pink}{wien}{}\ledrightnote{\textcolor{pink}{Wien}}
                  lll.-580 3l 239 40–m= \pend
           \pstart
           dieser wunsch sei meinem freund geweiht dass in seinem sehr geliebten \textcolor{green}{werke}{}\ledrightnote{→\textcolor{green}{Der Graf von Charolais. Ein Trauerspiel}} jeder alle weichheit alle staerke einer
               ungebrochenen menschlichkeit keiner den beruehmten \label{K_L01481_1v}\edtext{bruch}{\lemma{\textnormal{\emph{bruch}}}\Cendnote{\textnormal{Zwischen 3.
                  und 4. Akt ist die Psychologie und Motivierung der Figuren nicht völlig stringent,
                  was auch von der Kritik wahrgenommen wurde.}}}\label{K_L01481_1h} bemerke =
                  \spacefill\mbox{= arthur +}\pend
           \endnumbering\briefempfaengerindex{Beer-Hofmann, Richard@\textsc{Beer-Hofmann, Richard}!zzzSchnitzler, Arthur@\emph{von Arthur Schnitzler}!1904-12-231@{23. 12. 1904}|)be}\mylabel{h}  \normalsize

\doendnotes{C}
\bigskip
\vfill

\clearpage

\footnotesize

\lohead{\textsc{register}}

% Definiere theindex-Environment komplett neu ohne reledmac
\makeatletter
\renewenvironment{theindex}{%
  \section*{\indexname}%
  \setlength{\parindent}{0pt}%
  \setlength{\parskip}{0pt plus 0.3pt}%
  \let\item\@idxitem
}{%
  \clearpage
}
\makeatother

\IfFileExists{\jobname-pw.ind}{\input{\jobname-pw.ind}}{}

\end{document}

      