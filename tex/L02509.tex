%% latex-korrekturansicht-vorspann.tex
%% Vorspann für die Korrekturansicht.
%% Lädt die gemeinsame Datei latex-vorspann.tex mit gesetztem Schalter.

\newif\ifkorrekturansicht
\korrekturansichttrue

\input{../tex-inputs/latex-vorspann}


               \section[Hugo Hofmannsthal an Arthur Schnitzler, 3. 6. 1929]{ Hugo Hofmannsthal an Arthur Schnitzler, 3. 6. 1929}\nopagebreak\mylabel{v}\rehead{ }\normalsize\beginnumbering\briefempfaengerindex{Schnitzler, Arthur@\textsc{Schnitzler, Arthur}!zzzHofmannsthal, Hugo von@\emph{von Hugo von Hofmannsthal}!1929-06-031@{3. 6. 1929}|(be} \toendnotes[C]{\smallbreak\pagebreak[2]} \Standort{CUL, Schnitzler, B 43.}
\physDesc{Brief, 1 Blatt, 2 Seiten
\newline{}Handschrift: schwarze Tinte, lateinische Kurrent
\newline{}Schnitzler: mit rotem Buntstift mehrere Unterstreichungen \newline{}Ordnung: 1) mit Bleistift von unbekannter Hand nummeriert: »\strikeout{372}« 2) mit Bleistift von unbekannter Hand nummeriert: »381«}\buchAbdrucke{\weitereDrucke{Hugo von Hofmannsthal, Arthur Schnitzler: \emph{Briefwechsel}. Hg. Therese Nickl und Heinrich Schnitzler. Frankfurt am Main: \emph{S. Fischer} 1964, S. 312.} }\toendnotes[C]{\smallbreak}\pstart
           {\pb}\textcolor{pink}{Rodaun}{}\ledrightnote{\textcolor{pink}{Rodaun}}{ }3. Juni 29\pend
           \pstart{}mein lieber Arthur, \pend\pstart
           so waren Sie also in der Zwischenzeit nicht verreist. Sie haben den Besuch Ihres \textcolor{blue}{Schwiegersohnes}{}\ledrightnote{→\textcolor{blue}{Arnoldo Cappellini}} hier empfangen,
               statt mit ihm zu reisen, \strikeout{und} Sie waren eine Woche
               lang recht unwohl, sind aber gottlob wieder völlig davon hergestellt – dies alles,
               wenn ich den Bericht der guten Freundin \textcolor{blue}{B. Z.}{}\ledrightnote{\textcolor{blue}{Berta Zuckerkandl}}
               recht verstehe.\pend
           \pstart
           Ich war 14 Tage, genau 13 Tage, in \textcolor{pink}{Italien}{}\ledrightnote{\textcolor{pink}{Italien}}, bis
               gegen \textcolor{pink}{Rom}{}\ledrightnote{\textcolor{pink}{Rom}} hin, ohne das eigentlich \textcolor{pink}{röm}{}\ledrightnote{\textcolor{pink}{Rom}}ische Gebiet zu berühren. Es waren sehr schöne Tage.\pend
           \pstart
           Vor dem Wegfahren las ich sehr viel in Ihren Sachen, erzählendes u. dramatisches \substVorne{}\textsuperscript{D}\substDazwischen{}d\substHinten{}urcheinander, alles mit dem größten Vergnügen. Ja, so gut \textcolor{green}{Leutnant Gustl}{}\ledrightnote{\textcolor{green}{Lieutenant Gustl. Novelle}} erzählt ist, »\textcolor{green}{Fräulein Else}{}\ledrightnote{\textcolor{green}{Fräulein Else}}« schlägt ihn freilich noch; das ist i{\geminationn}erhalb der deutschen Literatur wirklich ein genre für
               sich, das Sie geschaffen haben. Sehr großen Eindruck machte mir auch der »\textcolor{green}{Einsame Weg}{}\ledrightnote{\textcolor{green}{Der einsame Weg. Schauspiel in fünf Akten}}«; so wenige Figuren eigentlich, und ein
               so großer Reichtum erreicht. Den \textcolor{green}{Roman}{}\ledrightnote{→\textcolor{green}{Der Weg ins Freie. Roman}} habe ich {\pb}auch
               wieder gelesen, so wie Sie es vorschlugen, von Capitel V bis zum Ende. Aber ich habe
               diese Arbeit nun einmal weniger gern, und ich kö{\geminationn}te es
               auch begründen. Die Einwände begi{\geminationn}en bei der Hauptfigur,
               die mir nicht ganz consistent erscheint (ihr Äußeres und Inneres nicht ganz
                  übereinsti{\geminationm}end) – aber der Haupteinwand geht tiefer.
               Aber darüber müsste man sich, wenn überhaupt, mündlich \substVorne{}\textsuperscript{sprechen}{\allowbreak}\substDazwischen{}unterhalten\substHinten{}. – Vor ein paar Tagen, gegen Abend, kam ich zurück, wollte \introOben{}mir\introOben{} irgend ein Buch suchen, und griff wieder nach einem von
               Ihnen: nach den \textcolor{green}{Dämmerseelen}{}\ledrightnote{\textcolor{green}{Dämmerseelen. Novellen}}, und las dann alle 5
               oder 6 Geschichten mit der größten Bewunderung. Dieser schwebende Ton und diese
               bezaubernde Leichtigkeit (\damage{\textcolor{gray}{nicht}} ohne Unheimlichkeit dabei) gehört wirklich nur Ihnen. Vielleicht ist dies,
               alles in allem, Ihr meisterhaftestes Buch; aber man soll keine Censuren austeilen. –
               Ich möchte Sie so gerne bald \label{K_L02509_1v}\edtext{wiedersehen}{\lemma{\textnormal{\emph{wiedersehen}}}\Cendnote{\textnormal{In Folge fand das letzte Treffen der
                  beiden statt, vgl. A. S.: \emph{Tagebuch}, 11. 6. 1929}}}\label{K_L02509_1h}. \textcolor{blue}{B. Z.}{}\ledrightnote{\textcolor{blue}{Berta Zuckerkandl}}{ }sagt mir, Sie fahren gerne Auto. Kann ich Sie nicht
               abholen, für einen halben Tag, – vor- oder nachmittag oder wie es Ihnen passt? Ich
               brauche nicht zu sagen, dass es mir die größte Freude machen würde. Rufen Sie
               vielleicht einmal zwischen 9\textsuperscript{h} und 10\textsuperscript{h}{ }\textcolor{pink}{Rodaun}{}\ledrightnote{\textcolor{pink}{Rodaun}} N. 3 an? \pend
           \pstart Von Herzen Ihr\spacefill\mbox{Hugo.}\pend{}\endnumbering\briefempfaengerindex{Schnitzler, Arthur@\textsc{Schnitzler, Arthur}!zzzHofmannsthal, Hugo von@\emph{von Hugo von Hofmannsthal}!1929-06-031@{3. 6. 1929}|)be}\mylabel{h}  \normalsize

\doendnotes{C}
\bigskip
\vfill

\clearpage

\footnotesize

\lohead{\textsc{register}}

% Definiere theindex-Environment komplett neu ohne reledmac
\makeatletter
\renewenvironment{theindex}{%
  \section*{\indexname}%
  \setlength{\parindent}{0pt}%
  \setlength{\parskip}{0pt plus 0.3pt}%
  \let\item\@idxitem
}{%
  \clearpage
}
\makeatother

\IfFileExists{\jobname-pw.ind}{\input{\jobname-pw.ind}}{}

\end{document}

      