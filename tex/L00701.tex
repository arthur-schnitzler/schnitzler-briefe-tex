%% latex-korrekturansicht-vorspann.tex
%% Vorspann für die Korrekturansicht.
%% Lädt die gemeinsame Datei latex-vorspann.tex mit gesetztem Schalter.

\newif\ifkorrekturansicht
\korrekturansichttrue

\input{../tex-inputs/latex-vorspann}


               \section[Georg Brandes an Arthur Schnitzler, {[}13. 7. 1897{]}]{ Georg Brandes an Arthur Schnitzler, {[}13. 7. 1897{]}}\nopagebreak\mylabel{v}\rehead{ }\normalsize\beginnumbering\briefempfaengerindex{Schnitzler, Arthur@\textsc{Schnitzler, Arthur}!zzzBrandes, Georg@\emph{von Georg Brandes}!1897-07-131@{{[}13. 7. 1897{]}}|(be} \toendnotes[C]{\smallbreak\pagebreak[2]} \Standort{CUL, Schnitzler, B 17.}
\physDesc{Brief, 1 Blatt, 2 Seiten
\newline{}Handschrift  : blaue Tinte, lateinische Kurrent\newline{}Handschrift Georg Brandes: blaue Tinte, lateinische Kurrent (\noindent{}Unterschrift)
\newline{}Schnitzler: mit schwarzer Tinte datiert: »etwa 13. Juli 97« und mit Bleistift nummeriert: »6« }\buchAbdrucke{\weitereDrucke{Georg Brandes, Arthur Schnitzler: \emph{Ein Briefwechsel}. Hg. Kurt Bergel. Bern: \emph{Francke} 1956, S. 63–64.} }\toendnotes[C]{\smallbreak}\pstart{}{\pb}Lieber und verehrter
                        Herr Schnitzler!\pend\pstart
           Ich kann leider nicht mit eigener Hand Ihren liebenswürdigen Brief beantworten.
                    Seit Ende April bin ich krank, habe eine heftige Aderentzündung,
                    die mich zwingt ganz still zu liegen, und habe im Juni eine schwere
                    Lungenentzündung durchgemacht, die mich dem Tode nah brachte. Jetzt ist die
                    Lunge einigermassen heil, doch in der eigentlichen Krankheit ist noch keine
                    Konvalescenz eingetreten. Ich werde voraussichtlich noch mehr als einen Monat im
                    Bette bleiben müssen. Mein ganzer Sommer ist dahin. Ich habe grosse Schmerzen
                    ausgestanden und bin noch sehr leidend.\pend
           \pstart
           Es freut mich sehr, dass Sie etwas in {\pb}meinem \textcolor{green}{Buch}{}\ledrightnote{→\textcolor{green}{William Shakespeare}} über \textcolor{blue}{Shakespeare}{}\ledrightnote{\textcolor{blue}{William Shakespeare}} gefunden haben. Ich lese in dieser Zeit die Korrekturbogen
                    der zweiten deutschen Ausgabe und bin über die fürchterliche Sprache ganz
                    erschreckt. Es wimmelt von den plumpsten Misverständnischen meines \textcolor{pink}{dänischen}{}\ledrightnote{\textcolor{pink}{Dänemark}} Textes; ich schreibe um und
                    verbessere ins unendliche.\pend
           \pstart
           Ich bitte Sie Ihre Freunde sehr herzlich von mir zu grüssen. Hr \textcolor{blue}{Goldmann}{}\ledrightnote{\textcolor{blue}{Paul Goldmann}} verstummte mir gegenüber plötzlich. Sie sind mir
                    aber alle drei gleich lieb.\pend
           \pstart
           Ihr ganz ergebener{\\[\baselineskip]}\spacefill\mbox{{[}hs. Brandes:{]} Georg Brandes}\pend
           \leftskip=0em{}\endnumbering\briefempfaengerindex{Schnitzler, Arthur@\textsc{Schnitzler, Arthur}!zzzBrandes, Georg@\emph{von Georg Brandes}!1897-07-131@{{[}13. 7. 1897{]}}|)be}\mylabel{h}  \normalsize

\doendnotes{C}
\bigskip
\vfill

\clearpage

\footnotesize

\lohead{\textsc{register}}

% Definiere theindex-Environment komplett neu ohne reledmac
\makeatletter
\renewenvironment{theindex}{%
  \section*{\indexname}%
  \setlength{\parindent}{0pt}%
  \setlength{\parskip}{0pt plus 0.3pt}%
  \let\item\@idxitem
}{%
  \clearpage
}
\makeatother

\IfFileExists{\jobname-pw.ind}{\input{\jobname-pw.ind}}{}

\end{document}

      