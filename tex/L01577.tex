%% latex-korrekturansicht-vorspann.tex
%% Vorspann für die Korrekturansicht.
%% Lädt die gemeinsame Datei latex-vorspann.tex mit gesetztem Schalter.

\newif\ifkorrekturansicht
\korrekturansichttrue

\input{../tex-inputs/latex-vorspann}


               \section[Hermann Bahr an Arthur Schnitzler, 29. 1. 1906]{ Hermann Bahr an Arthur Schnitzler, 29. 1. 1906}\nopagebreak\mylabel{v}\rehead{ }\normalsize\beginnumbering\briefempfaengerindex{Schnitzler, Arthur@\textsc{Schnitzler, Arthur}!zzzBahr, Hermann@\emph{von Hermann Bahr}!1906-01-291@{29. 1. 1906}|(be} \toendnotes[C]{\smallbreak\pagebreak[2]} \Standort{TMW, HS Schn 1/29/1.}
\physDesc{Brief, 1 Blatt, 1 Seite
\newline{}Handschrift: schwarze Tinte, deutsche Kurrent}\buchAbdrucke{\weitereDrucke{Hermann Bahr, Arthur Schnitzler: \emph{Briefwechsel, Aufzeichnungen, Dokumente (1891–1931)}. Hg. Kurt Ifkovits und Martin Anton Müller. Göttingen: \emph{Wallstein} 2018, S. 372.} }\toendnotes[C]{\smallbreak}\pstart
           \raggedleft{}{\pb}29. 1. 06{\\}\textcolor{pink}{Wien XIII\textsubscript{/7}}{}\ledrightnote{\textcolor{pink}{Ober Sankt Veit}}\pend
           \pstart\center{}Lieber Arthur!\pend\pstart
           Ich hatte den »\textcolor{green}{Ruf des Lebens}{}\ledrightnote{\textcolor{green}{Der Ruf des Lebens. Schauspiel in drei Akten}}« sogleich mit der
               Bezeichnung »von mir angenommen« nach \textcolor{pink}{München}{}\ledrightnote{\textcolor{pink}{München}}
               geschickt und mir die Genehmigung des \textcolor{blue}{Intendanten}{}\ledrightnote{→\textcolor{blue}{Albert von Speidel}} als mir besonders wichtig dringend erbeten.
               Eben kommt sein Brief, der sie verweigert, angeblich aus Bedenken gegen den \textcolor{green}{zweiten Akt}{}\ledrightnote{→\textcolor{green}{Der Ruf des Lebens. Schauspiel in drei Akten}}. Es ist das nur ein
               Glied in der Kette von kleinen Gemeinheiten, durch welche man mich jetzt aus meinem
               Contract herausekeln will, was vermutlich gelingen wird.\pend
           \pstart
           Mit vielen Grüßen an Frau \textcolor{blue}{Olga}{}\ledrightnote{\textcolor{blue}{Olga Schnitzler}}{\\[\baselineskip]}herzlichst{\\[\baselineskip]}Dein{\\[\baselineskip]}\spacefill\mbox{Hermann}\pend
           \leftskip=0em{}\endnumbering\briefempfaengerindex{Schnitzler, Arthur@\textsc{Schnitzler, Arthur}!zzzBahr, Hermann@\emph{von Hermann Bahr}!1906-01-291@{29. 1. 1906}|)be}\mylabel{h}  \normalsize

\doendnotes{C}
\bigskip
\vfill

\clearpage

\footnotesize

\lohead{\textsc{register}}

% Definiere theindex-Environment komplett neu ohne reledmac
\makeatletter
\renewenvironment{theindex}{%
  \section*{\indexname}%
  \setlength{\parindent}{0pt}%
  \setlength{\parskip}{0pt plus 0.3pt}%
  \let\item\@idxitem
}{%
  \clearpage
}
\makeatother

\IfFileExists{\jobname-pw.ind}{\input{\jobname-pw.ind}}{}

\end{document}

      