%% latex-korrekturansicht-vorspann.tex
%% Vorspann für die Korrekturansicht.
%% Lädt die gemeinsame Datei latex-vorspann.tex mit gesetztem Schalter.

\newif\ifkorrekturansicht
\korrekturansichttrue

\input{../tex-inputs/latex-vorspann}


               \section[Lou Andreas-Salomé an Arthur Schnitzler, 9. 8. 1895]{ Lou Andreas-Salomé an Arthur Schnitzler, 9. 8. 1895}\nopagebreak\mylabel{v}\rehead{ }\normalsize\beginnumbering\briefempfaengerindex{Schnitzler, Arthur@\textsc{Schnitzler, Arthur}!zzzAndreas-Salome, Lou@\emph{von Lou Andreas-Salomé}!1895-08-091@{9. 8. 1895}|(be} \toendnotes[C]{\smallbreak\pagebreak[2]} \Standort{CUL, Schnitzler, B 3.}
\physDesc{Postkarte
\newline{}Handschrift: schwarze Tinte, deutsche Kurrent\newline{}Versand: 1) Stempel: »\nobreak{}\oindex{Schmargendorf@\textbf{Schmargendorf}, \emph{https://www.geonames.org/ontologyP.PPLX}|pwk}Schmargendorf, 9/8 95, 3–4 N\nobreak{}«.  2) Stempel: »\nobreak{}\oindex{IX., Alsergrund@\textbf{IX., Alsergrund}, \emph{Bezirk (A.BZK)}|pwk}Wien 9/3, 10.8 95, 7.N, Bestellt\nobreak{}«. \newline{}Ordnung: mit rotem Buntstift von unbekannter Hand nummeriert: »5« }\pstart{}{\pb}Herrn \textsc{D\textsuperscript{r}}\pend{}\pstart{}\textsc{Arthur Schnitzler}\pend{}\pstart{}\textsc{\textcolor{pink}{Wien}{}\ledrightnote{\textcolor{pink}{Wien}}}\pend{}\pstart{}\textcolor{pink}{Frankgasse \textsc{N\textsuperscript{o}} 1}{}\ledrightnote{\textcolor{pink}{Frankgasse}}. \pend{}{\bigskip}\pstart
           \noindent{}{\pb}Lieber Herr \textsc{D\textsuperscript{r}}! ich ſchreibe Ihnen auf gut Glück in Ihre \textcolor{pink}{Wien}{}\ledrightnote{\textcolor{pink}{Wien}}er Wohnung, um Ihnen zu erzählen, daß ich alſo
                    wirklich nach \textcolor{pink}{München}{}\ledrightnote{\textcolor{pink}{München}} reiſe und daß ich am \uline{Montag den 19\textsuperscript{ten}
                            Auguſt}{ }\uline{nach \textcolor{pink}{Salzburg}{}\ledrightnote{\textcolor{pink}{Salzburg}}} komme um dort bis gegen den 25\textsuperscript{ten} zu bleiben. Nachrichten erreichen mich \uline{hier} bis zum Mittwoch Morgen, 14. Auguſt.\pend
           \pstart
           Auf Wiederſehen!\pend
           \pstart Mit herzlichem Gruß \spacefill\mbox{Lou Andreas-Salomé}\pend{}\endnumbering\briefempfaengerindex{Schnitzler, Arthur@\textsc{Schnitzler, Arthur}!zzzAndreas-Salome, Lou@\emph{von Lou Andreas-Salomé}!1895-08-091@{9. 8. 1895}|)be}\mylabel{h}  \normalsize

\doendnotes{C}
\bigskip
\vfill

\clearpage

\footnotesize

\lohead{\textsc{register}}

% Definiere theindex-Environment komplett neu ohne reledmac
\makeatletter
\renewenvironment{theindex}{%
  \section*{\indexname}%
  \setlength{\parindent}{0pt}%
  \setlength{\parskip}{0pt plus 0.3pt}%
  \let\item\@idxitem
}{%
  \clearpage
}
\makeatother

\IfFileExists{\jobname-pw.ind}{\input{\jobname-pw.ind}}{}

\end{document}

      