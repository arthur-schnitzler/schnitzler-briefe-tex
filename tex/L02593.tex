%% latex-korrekturansicht-vorspann.tex
%% Vorspann für die Korrekturansicht.
%% Lädt die gemeinsame Datei latex-vorspann.tex mit gesetztem Schalter.

\newif\ifkorrekturansicht
\korrekturansichttrue

\input{../tex-inputs/latex-vorspann}


               \section[Marie Herzfeld an Arthur Schnitzler, 19. 4. 1909]{ Marie Herzfeld an Arthur Schnitzler, 19. 4. 1909}\nopagebreak\mylabel{v}\rehead{ }\normalsize\beginnumbering\briefempfaengerindex{Schnitzler, Arthur@\textsc{Schnitzler, Arthur}!zzzHerzfeld, Marie@\emph{von Marie Herzfeld}!1909-04-191@{19. 4. 1909}|(be} \toendnotes[C]{\smallbreak\pagebreak[2]} \Standort{DLA, A:Schnitzler, HS.1985.1.03436,4.}
\physDesc{Brief, 1 Blatt, 4 Seiten
\newline{}Handschrift: schwarze Tinte, lateinische Kurrent
\newline{}Schnitzler: 1) mit Bleistift Vermerk »\textsc{Herzfeld}« 2) mit rotem Buntstift Vermerk »\textsc{\textcolor{blue}{Tesi}, \textcolor{blue}{Mi\textcolor{gray}{c}h\textcolor{gray}{ae}{[}lis{]}}}«
                                 und drei Unterstreichungen}\toendnotes[C]{\smallbreak}\pstart
           \raggedleft{}{\pb}\textcolor{pink}{Wien II/\textsubscript{2}, Lichtenauerg. 5}{}\ledrightnote{\textcolor{pink}{Lichtenauergasse}}{\\}den
                     19. April 1909\pend
           \pstart{}Sehr geehrter Herr Doktor!\pend\pstart
           Es besuchte mich heute Frau \textcolor{blue}{Anna Tesi}{}\ledrightnote{\textcolor{blue}{Anna Rotenstern-Tesi}}, die mir sagt, Sie hätten von \textcolor{blue}{Soph. Michaëlis}{}\ledrightnote{\textcolor{blue}{Sophus Michaëlis}}{ }\label{K_L02593-1v}\edtext{»\textcolor{green}{Revolutionshochzeit}{}\ledrightnote{\textcolor{green}{Revolutionsbryllup. Skuespil i tre Akter}}«}{\lemma{\textnormal{\emph{»Revolutionshochzeit«}}}\Cendnote{\textnormal{\textcolor{blue}{Sophus Michaelis}: \emph{\textcolor{green}{Revolutionshochzeit. Schauspiel in drei Aufzügen}}. Aus dem
                     Dänischen von \textcolor{blue}{Marie Herzfeld}.
                     \textcolor{pink}{Berlin}: \emph{\textcolor{brown}{Erich
                        Reiss-Verlag}}{ }1909.}}}\label{K_L02593-1h} gesprochen und ihr Interesse für das \textcolor{green}{Stück}{}\ledrightnote{→\textcolor{green}{Revolutionsbryllup. Skuespil i tre Akter}} so lebhaft erweckt, dass sie
               gern das Uebersetzungs- und Vertretungsrecht für \textcolor{pink}{Frankreich}{}\ledrightnote{\textcolor{pink}{Frankreich}} und \textcolor{pink}{Russland}{}\ledrightnote{\textcolor{pink}{Russland}} erwerben möchte.
                  {\pb}Sie sagt, sie sei \label{K_L02593-2v}\edtext{Ihre russ. Uebersetzerin}{\lemma{\textnormal{\emph{Ihre russ. Uebersetzerin}}}\Cendnote{\textnormal{Autorisierter Übersetzer \textcolor{blue}{Schnitzlers} war der Ehemann von \textcolor{blue}{Anna Rotenstern-Tesi},
                   \textcolor{blue}{Peter Rotenstern}. Zu den von \textcolor{blue}{Tesi} übersetzten Texten siehe Arthur Schnitzler an Marie Herzfeld, 20. 4. 1909.}}}\label{K_L02593-2h}, sei Mitglied der \textcolor{brown}{Société des Auteurs dramatiques in \textcolor{pink}{Paris}{}\ledrightnote{\textcolor{pink}{Paris}}}{}\ledrightnote{\textcolor{brown}{Société des Auteurs et Compositeurs Dramatiques}} u. in \textcolor{brown}{\textcolor{pink}{Moskau}{}\ledrightnote{\textcolor{pink}{Moskau}}}{}\ledrightnote{→\textcolor{brown}{Verein dramatischer und musikalischer Autoren}} (\textcolor{pink}{Petersburg}{}\ledrightnote{\textcolor{pink}{Sankt Petersburg}}?). Sie will Abmachungen, die dahin gehen, dass sie die Hälfte
               aller Tantièmen u. Honorare mir abliefert. Die Controlle, sagt sie, sei in Händen
               jener \textcolor{brown}{Sociétés}{}\ledrightnote{→\textcolor{brown}{Société des Auteurs et Compositeurs Dramatiques}{\newline}→\textcolor{brown}{Verein dramatischer und musikalischer Autoren}}. Ich glaube
               ihr alles; sie macht mir persön{\pb}lich einen
               vertrauenerweckenden Eindruck; aber sie hat einen Gesellschafter, ihren \textcolor{blue}{Mann}{}\ledrightnote{→\textcolor{blue}{Peter Rotenstern}}, den sie mir nicht zeigte {\dots} kurz – so sehr ich in der Regel meinem Gefühl folge, so
               muss ich, als Vertreterin der Interessen des \textcolor{blue}{dänischen Dichters}{}\ledrightnote{→\textcolor{blue}{Sophus Michaëlis}} dennoch etwas vorsichtig sein und wäre Ihnen
               daher sehr dankbar, wenn Sie mir sagten, {\pb}ob \introOben{}a)\introOben{} Sie mit dem Ehepaar \textcolor{blue}{Tesi}{}\ledrightnote{\textcolor{blue}{Anna Rotenstern-Tesi}{\newline}\textcolor{blue}{Peter Rotenstern}} gute Erfahrungen machten und ob b) die Bedingungen, die man mir bietet,
               billige sind. Ich habe mich um dergleichen nie bekümmert und bin naïv wie ein
               neugeborenes Kind.\pend
           \pstart
           Pardon, dass ich Ihnen Mühe mache! Ihnen stets zu Gegendiensten, in großer
               Schätzung\pend
           \pstart \spacefill\mbox{Marie Herzfeld}\pend{}\endnumbering\briefempfaengerindex{Schnitzler, Arthur@\textsc{Schnitzler, Arthur}!zzzHerzfeld, Marie@\emph{von Marie Herzfeld}!1909-04-191@{19. 4. 1909}|)be}\mylabel{h}  \normalsize

\doendnotes{C}
\bigskip
\vfill

\clearpage

\footnotesize

\lohead{\textsc{register}}

% Definiere theindex-Environment komplett neu ohne reledmac
\makeatletter
\renewenvironment{theindex}{%
  \section*{\indexname}%
  \setlength{\parindent}{0pt}%
  \setlength{\parskip}{0pt plus 0.3pt}%
  \let\item\@idxitem
}{%
  \clearpage
}
\makeatother

\IfFileExists{\jobname-pw.ind}{\input{\jobname-pw.ind}}{}

\end{document}

      