%% latex-korrekturansicht-vorspann.tex
%% Vorspann für die Korrekturansicht.
%% Lädt die gemeinsame Datei latex-vorspann.tex mit gesetztem Schalter.

\newif\ifkorrekturansicht
\korrekturansichttrue

\input{../tex-inputs/latex-vorspann}


               \section[Richard Beer-Hofmann an Arthur Schnitzler, {[}10.? 9. 1911{]}]{ Richard Beer-Hofmann an Arthur Schnitzler,
               {[}10.? 9. 1911{]}}\nopagebreak\mylabel{v}\rehead{ }\normalsize\beginnumbering\briefempfaengerindex{Schnitzler, Arthur@\textsc{Schnitzler, Arthur}!zzzBeer-Hofmann, Richard@\emph{von Richard Beer-Hofmann}!1911-09-101@{{[}10.? 9. 1911{]}}|(be} \toendnotes[C]{\smallbreak\pagebreak[2]} \Standort{CUL, Schnitzler, B 8.}
\physDesc{Telegramm
\newline{}maschinell
\newline{}Schnitzler: mit Bleistift beschriftet: »BH« \newline{}Ordnung: mit Bleistift von unbekannter Hand nummeriert:
                              »263« }\buchAbdrucke{\weitereDrucke{Arthur Schnitzler, Richard Beer-Hofmann: \emph{Briefwechsel 1891–1931}. Hg. Konstanze Fliedl. Wien, Zürich: \emph{Europaverlag} 1992, S. 215.} }\toendnotes[C]{\smallbreak}\pstart
           {\pb}\textcolor{pink}{badaussee}{}\ledrightnote{\textcolor{pink}{Bad Aussee}} 496 44 11/40\pend
           \pstart
           bitte uebermitteln sye ihrem \textcolor{blue}{bruder}{}\ledrightnote{→\textcolor{blue}{Julius Schnitzler}} u ihrer \textcolor{blue}{schwester}{}\ledrightnote{→\textcolor{blue}{Gisela Hajek}} mein aufrichtigstes \label{K_L02026_1v}\edtext{bejlejd}{\lemma{\textnormal{\emph{bejlejd}}}\Cendnote{\textnormal{Die Mutter \textcolor{blue}{Louise Schnitzler} war am 9. 9. 1911
                  gestorben.}}}\label{K_L02026_1h} sye selbst lieber arthur wiszen denke ich dasz ich heute wie
               immer an allem was sye an gutem und boesem trifft von herzen antejl nehme
                  \spacefill\mbox{richard .+}\pend
           \endnumbering\briefempfaengerindex{Schnitzler, Arthur@\textsc{Schnitzler, Arthur}!zzzBeer-Hofmann, Richard@\emph{von Richard Beer-Hofmann}!1911-09-101@{{[}10.? 9. 1911{]}}|)be}\mylabel{h}  \normalsize

\doendnotes{C}
\bigskip
\vfill

\clearpage

\footnotesize

\lohead{\textsc{register}}

% Definiere theindex-Environment komplett neu ohne reledmac
\makeatletter
\renewenvironment{theindex}{%
  \section*{\indexname}%
  \setlength{\parindent}{0pt}%
  \setlength{\parskip}{0pt plus 0.3pt}%
  \let\item\@idxitem
}{%
  \clearpage
}
\makeatother

\IfFileExists{\jobname-pw.ind}{\input{\jobname-pw.ind}}{}

\end{document}

      