%% latex-korrekturansicht-vorspann.tex
%% Vorspann für die Korrekturansicht.
%% Lädt die gemeinsame Datei latex-vorspann.tex mit gesetztem Schalter.

\newif\ifkorrekturansicht
\korrekturansichttrue

\input{../tex-inputs/latex-vorspann}


               \section[Hugo von Hofmannsthal an Arthur Schnitzler, 1{[}1?{]}. 6. 1902]{ Hugo von Hofmannsthal an Arthur Schnitzler, 1{[}1?{]}. 6. 1902}\nopagebreak\mylabel{v}\rehead{ }\normalsize\beginnumbering\briefempfaengerindex{Schnitzler, Arthur@\textsc{Schnitzler, Arthur}!zzzHofmannsthal, Hugo von@\emph{von Hugo von Hofmannsthal}!1902-06-111@{1{[}1?{]}. 6. 1902}|(be} \toendnotes[C]{\smallbreak\pagebreak[2]} \Standort{CUL, Schnitzler, B 43.}
\physDesc{Brief, 1 Blatt, 4 Seiten
\newline{}Handschrift: schwarze Tinte, deutsche Kurrent
\newline{}Schnitzler: mit Bleistift datiert: »10/6 902« \newline{}Ordnung: 1) mit Bleistift von unbekannter Hand nummeriert: »\strikeout{196}« 2) mit Bleistift von unbekannter Hand nummeriert: »189«}\buchAbdrucke{\weitereDrucke{Hugo von Hofmannsthal, Arthur Schnitzler: \emph{Briefwechsel}. Hg. Therese Nickl und Heinrich Schnitzler. Frankfurt am Main: \emph{S. Fischer} 1964, S. 158–159.} }\toendnotes[C]{\smallbreak}\pstart
           \raggedleft{}{\pb}\label{K_L01222_1v}\edtext{Mittwoch}{\lemma{\textnormal{\emph{Mittwoch}}}\Cendnote{\textnormal{Schnitzlers Datierung verweist
                        auf einen Dienstag. Unter der Annahme, dass er – und nicht Hofmannsthal –
                        sich geirrt hat, wurde auf den Folgetag datiert.}}}\label{K_L01222_1h}\pend
           \pstart{}lieber Arthur\pend\pstart
           wenn nächſten Sonntag (15.\textsuperscript{ten}) ſchönes Wetter iſt, möchten \textcolor{blue}{Richard}{}\ledrightnote{\textcolor{blue}{Richard Beer-Hofmann}} und
               ich ſehr gern um 11\textsuperscript{h} vormittag auf dem \textcolor{pink}{Friedhof in \textsc{Gutenſtein}}{}\ledrightnote{\textcolor{pink}{Bergfriedhof}}
               bei der Beſtattung von \label{K_L01222_2v}\edtext{\textcolor{blue}{Raimund}{}\ledrightnote{\textcolor{blue}{Ferdinand Raimund}} im neuen Grab}{\lemma{\textnormal{\emph{Raimund im neuen Grab}}}\Cendnote{\textnormal{Die Wiederbestattung in der
                  renovierten Gruft fand am 15. 6. 1902 um
                     11 Uhr vormittags statt. Einige kulturelle Prominenz aus \textcolor{pink}{Wien} war dafür angereist, \textcolor{blue}{Schnitzler} aber nicht.}}}\label{K_L01222_2h} dabei ſein. Mir ſagt ein für
               gewöhnlich bei {\pb}mir nicht ſo
               lebhaftes Gefühl, daſs ich es thuen ſoll.\pend
           \pstart
           Wir würden in \textcolor{pink}{\textsc{Mödling}}{}\ledrightnote{\textcolor{pink}{Mödling}} in den Schnellzug einſteigen
               der in \textcolor{pink}{\textsc{Mödling}}{}\ledrightnote{\textcolor{pink}{Mödling}}{ }7\textsuperscript{h}15 durchfährt, in \textcolor{pink}{Wien}{}\ledrightnote{\textcolor{pink}{Wien}} geht er 6\textsuperscript{h}50 ab.
               Ich möchte dann in \textcolor{pink}{Guthenſtein}{}\ledrightnote{\textcolor{pink}{Gutenstein}} mittageſſen {\pb}und den ſchönen Weg über \textcolor{pink}{\textsc{Vöslau}}{}\ledrightnote{\textcolor{pink}{Bad Vöslau}} etc. nachmittag mit dem Rad zurück-machen. Ich hoffe, mit Ihnen.\pend
           \pstart
           Wenn Sie nichts ſagen laſſen und \uline{es kein Regentag}
               iſt, ſo hoffen wir, Sie ſind im Zug oder ſteigen in \textcolor{pink}{\textsc{Mödling}}{}\ledrightnote{\textcolor{pink}{Mödling}} in ihn ein.\pend
           \pstart
           {\pb}Iſt das Wetter zweifelhaft ſo
               kann man ſich noch Samstag bis 9\textsuperscript{h} abends im Telephon ſprechen.\pend
           \pstart
           Von Herzen Ihr{\\[\baselineskip]}\spacefill\mbox{Hugo.}\pend
           \leftskip=0em{}\pstart
           \noindent{}\textsc{Circa} 20\textsuperscript{ten} hoffe ich wir
                  fahren \textsc{\textcolor{pink}{Salzburg}{}\ledrightnote{\textcolor{pink}{Salzburg}} – \textcolor{pink}{Lofer}{}\ledrightnote{\textcolor{pink}{Lofer}} – \textcolor{pink}{Innsbruck}{}\ledrightnote{\textcolor{pink}{Innsbruck}} – (Seitenausflug
                        \textcolor{pink}{Stubaithal}{}\ledrightnote{\textcolor{pink}{Stubaital}}) – \textcolor{pink}{Brenner}{}\ledrightnote{\textcolor{pink}{Brenner}} – \textcolor{pink}{Toblach}{}\ledrightnote{\textcolor{pink}{Toblach}} (Seitenausflug \textcolor{pink}{Ampezzothal}{}\ledrightnote{\textcolor{pink}{Ampezzo}}) – \textcolor{pink}{Spital a. Drau}{}\ledrightnote{\textcolor{pink}{Spittal an der Drau}} – \textcolor{pink}{Radstadt}{}\ledrightnote{\textcolor{pink}{Radstadt}} – \textcolor{pink}{Bischofshofen}{}\ledrightnote{\textcolor{pink}{Bischofshofen}} – \textcolor{pink}{Salzburg}{}\ledrightnote{\textcolor{pink}{Salzburg}}, circa 12 Tage}.\pend
           \endnumbering\briefempfaengerindex{Schnitzler, Arthur@\textsc{Schnitzler, Arthur}!zzzHofmannsthal, Hugo von@\emph{von Hugo von Hofmannsthal}!1902-06-111@{1{[}1?{]}. 6. 1902}|)be}\mylabel{h}  \normalsize

\doendnotes{C}
\bigskip
\vfill

\clearpage

\footnotesize

\lohead{\textsc{register}}

% Definiere theindex-Environment komplett neu ohne reledmac
\makeatletter
\renewenvironment{theindex}{%
  \section*{\indexname}%
  \setlength{\parindent}{0pt}%
  \setlength{\parskip}{0pt plus 0.3pt}%
  \let\item\@idxitem
}{%
  \clearpage
}
\makeatother

\IfFileExists{\jobname-pw.ind}{\input{\jobname-pw.ind}}{}

\end{document}

      