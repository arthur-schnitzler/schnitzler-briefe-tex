%% latex-korrekturansicht-vorspann.tex
%% Vorspann für die Korrekturansicht.
%% Lädt die gemeinsame Datei latex-vorspann.tex mit gesetztem Schalter.

\newif\ifkorrekturansicht
\korrekturansichttrue

\input{../tex-inputs/latex-vorspann}


               \section[Richard Beer-Hofmann an Arthur Schnitzler, 26. 7. 1907]{ Richard Beer-Hofmann an Arthur Schnitzler,
               26. 7. 1907}\nopagebreak\mylabel{v}\rehead{ }\normalsize\beginnumbering\briefempfaengerindex{Schnitzler, Arthur@\textsc{Schnitzler, Arthur}!zzzBeer-Hofmann, Richard@\emph{von Richard Beer-Hofmann}!1907-07-261@{26. 7. 1907}|(be} \toendnotes[C]{\smallbreak\pagebreak[2]} \Standort{CUL, Schnitzler, B 8.}
\physDesc{Brief, 1 Blatt, 3 Seiten
\newline{}Handschrift: Bleistift, lateinische Kurrent\newline{}Ordnung: mit Bleistift von unbekannter Hand nummeriert: »209« }\buchAbdrucke{\weitereDrucke{Arthur Schnitzler, Richard Beer-Hofmann: \emph{Briefwechsel 1891–1931}. Hg. Konstanze Fliedl. Wien, Zürich: \emph{Europaverlag} 1992, S. 181.} }\pstart
           \raggedleft{}{\pb}\textcolor{pink}{Maria-Schutz}{}\ledrightnote{\textcolor{pink}{Maria Schutz}}{ }26./VII.  07.\pend
           \pstart
           Lieber Arthur! Ihren lieben Brief vom 14. habe ich
               erhalten. Am 4. Abends sind wir hier angekommen, am 6. bin
               ich nach \textcolor{pink}{Wien}{}\ledrightnote{\textcolor{pink}{Wien}} zurück, am 7. wieder
               hieher um \textcolor{blue}{Paula}{}\ledrightnote{\textcolor{blue}{Paula Beer-Hofmann}} zu holen, und von 8.
               an bis zum 11. waren {[}wir{]} wieder in \textcolor{pink}{Wien}{}\ledrightnote{\textcolor{pink}{Wien}}, die letzten zwei Tage davon in \textcolor{pink}{Purkersdorf}{}\ledrightnote{\textcolor{pink}{Purkersdorf}}. Ich habe für lange, ich glaube für sehr lange,
               einen recht bittern Geschmack im Munde.\pend
           \pstart
           Wir bleiben bis 3./VIII. hier, gehen dann nach \textcolor{pink}{Wien}{}\ledrightnote{\textcolor{pink}{Wien}}. Zwischen 14. und 19. August
               wollen wir {\pb}nach \textcolor{pink}{Villach}{}\ledrightnote{\textcolor{pink}{Villach}}, um an irgendeinem \textcolor{pink}{Kärntner}{}\ledrightnote{\textcolor{pink}{Kärnten}}see für 8 Tage unterzuko{\geminationm}en. Dann \textcolor{pink}{Südtirol}{}\ledrightnote{\textcolor{pink}{Südtirol}}, womöglich \textcolor{pink}{Gardasee}{}\ledrightnote{\textcolor{pink}{Lago di Garda}}. Waren Sie im \textcolor{pink}{Lido-Hôtel}{}\ledrightnote{\textcolor{pink}{Palast Hotel Lido}} in \textcolor{pink}{Riva}{}\ledrightnote{\textcolor{pink}{Riva del Garda}} zufrieden? Und hat das Hôtel eine wirkliche
                  Bade\uline{anstalt}? Mit Schwimmmeister? Ist vielleicht
                  \textcolor{pink}{Hôtel du Lac (Witzmann)}{}\ledrightnote{\textcolor{pink}{Hotel du Lac}} zu empfehlen? oder \textcolor{pink}{Torbole}{}\ledrightnote{\textcolor{pink}{Torbole sul Garda}}? Da Sie Ende August in \textcolor{pink}{Bozen}{}\ledrightnote{\textcolor{pink}{Bozen}}s Umgebung sein wollen, so rechne ich damit
               Sie um diese Zeit irgendwo sehen zu können. Ich würde {\pb}mich sehr freuen. Ich glaube, es
               wird ganz leicht gehen, wenn Sie mich rechtzeitig verständigen. Von \textcolor{pink}{Kärnten}{}\ledrightnote{\textcolor{pink}{Kärnten}} nach \textcolor{pink}{Bozen}{}\ledrightnote{\textcolor{pink}{Bozen}} möchte ich
               über die neue \textcolor{pink}{Dolomitenstrasse}{}\ledrightnote{\textcolor{pink}{Große Dolomitenstraße}}.\pend
           \pstart
           Ich grüsse Sie, \textcolor{blue}{Olga}{}\ledrightnote{\textcolor{blue}{Olga Schnitzler}} und \textcolor{blue}{Heini}{}\ledrightnote{\textcolor{blue}{Heinrich Schnitzler}} herzlichst. \textcolor{blue}{Paula}{}\ledrightnote{\textcolor{blue}{Paula Beer-Hofmann}} tut
               dasselbe.\pend
           \pstart
           Ihr{\\[\baselineskip]}\spacefill\mbox{Richard}\pend
           \leftskip=0em{}\endnumbering\briefempfaengerindex{Schnitzler, Arthur@\textsc{Schnitzler, Arthur}!zzzBeer-Hofmann, Richard@\emph{von Richard Beer-Hofmann}!1907-07-261@{26. 7. 1907}|)be}\mylabel{h}  \normalsize

\doendnotes{C}
\bigskip
\vfill

\clearpage

\footnotesize

\lohead{\textsc{register}}

% Definiere theindex-Environment komplett neu ohne reledmac
\makeatletter
\renewenvironment{theindex}{%
  \section*{\indexname}%
  \setlength{\parindent}{0pt}%
  \setlength{\parskip}{0pt plus 0.3pt}%
  \let\item\@idxitem
}{%
  \clearpage
}
\makeatother

\IfFileExists{\jobname-pw.ind}{\input{\jobname-pw.ind}}{}

\end{document}

      