%% latex-korrekturansicht-vorspann.tex
%% Vorspann für die Korrekturansicht.
%% Lädt die gemeinsame Datei latex-vorspann.tex mit gesetztem Schalter.

\newif\ifkorrekturansicht
\korrekturansichttrue

\input{../tex-inputs/latex-vorspann}


               \section[Hugo von Hofmannsthal an Arthur Schnitzler, 28. 6. 1904]{ Hugo von Hofmannsthal an Arthur Schnitzler, 28. 6. 1904}\nopagebreak\mylabel{v}\rehead{ }\normalsize\beginnumbering\briefempfaengerindex{Schnitzler, Arthur@\textsc{Schnitzler, Arthur}!zzzHofmannsthal, Hugo von@\emph{von Hugo von Hofmannsthal}!1904-06-281@{28. 6. 1904}|(be} \toendnotes[C]{\smallbreak\pagebreak[2]} \Standort{CUL, Schnitzler, B 43.}
\physDesc{Brief, 1 Blatt, 4 Seiten
\newline{}Handschrift: schwarze Tinte, deutsche Kurrent\newline{}Ordnung: 1) mit Bleistift von unbekannter Hand nummeriert: »\strikeout{240}« 2) mit Bleistift von unbekannter Hand nummeriert:
                                    »226«}\buchAbdrucke{\weitereDrucke{Hugo von Hofmannsthal, Arthur Schnitzler: \emph{Briefwechsel}. Hg. Therese Nickl und Heinrich Schnitzler. Frankfurt am Main: \emph{S. Fischer} 1964, S. 189.} }\toendnotes[C]{\smallbreak}\pstart
           \raggedleft{}{\pb}28 VI 1904.\pend
           \pstart{}mein lieber Arthur\pend\pstart
           im Grund bin ich froh, daſs ſich Ihr ſchleichendes Übelbefinden, das mich beſorgt
               gemacht hatte, in dieſer verhältnismäßig harmloſen Form erklärt hat.\pend
           \pstart
           Aber daſs ſich i{\geminationm}er wieder etwas dazwiſchenſtellt und
               dieſe kleinen Zuſa{\geminationm}enkünfte nicht will ſchneller
               aufeinander folgen laſſen. Und doch {\pb}weiß ich unter allem, was das
               Leben mit ſich bringt, faſt nichts ſo ſchönes als ein Nachmittag wie der \label{K_L01411_1v}\edtext{neulich}{\lemma{\textnormal{\emph{neulich}}}\Cendnote{\textnormal{vgl. A. S.: \emph{Tagebuch}, 22. 6. 1904}}}\label{K_L01411_1h}, ein Geſpräch, das manchmal in die tiefſten Tiefen untertaucht und ſich dann
               wieder mit harmloſer Freude an der Oberfläche hält, ein paar Lieder dazwiſchen, {\pb}der Spaziergang, alles das, faſt
               unglaublich viel und ſchön und harmoniſch.\pend
           \pstart
           Ich wollte folgendes vorſchlagen: ſind Sie Anfang nächſter Woche vielleicht wohl
               genug, um an unſerer Geſellſchaft Vergnügen zu finden, noch aber zu ſchwach, um etwas
               zu unternehmen, ſo würden {\pb}wir
               ſehr gern wieder zu Tiſch hinüberko{\geminationm}en, und uns dann für
               den gleichen Tag gegen 6\textsuperscript{h} zu \textcolor{blue}{Saltens}{}\ledrightnote{\textcolor{blue}{Ottilie Salten}{\newline}\textcolor{blue}{Felix Salten}} anſagen, dieſe ſpaziergangsweiſe \label{K_L01411_2v}\edtext{aufſuchen}{\lemma{\textnormal{\emph{aufſuchen}}}\Cendnote{\textnormal{Diese lebten in der \textcolor{pink}{Porzellangasse 45}.}}}\label{K_L01411_2h}.\pend
           \pstart
           Vielleicht, wenn Ihr Befinden es erlaubt, ſchlagen Sie uns dazu telegraphiſch einen
               Tag vor. Wenn nicht, ſo nicht.\pend
           \pstart
           Von Herzen Ihr{\\[\baselineskip]}\spacefill\mbox{Hugo}\pend
           \leftskip=0em{}\endnumbering\briefempfaengerindex{Schnitzler, Arthur@\textsc{Schnitzler, Arthur}!zzzHofmannsthal, Hugo von@\emph{von Hugo von Hofmannsthal}!1904-06-281@{28. 6. 1904}|)be}\mylabel{h}  \normalsize

\doendnotes{C}
\bigskip
\vfill

\clearpage

\footnotesize

\lohead{\textsc{register}}

% Definiere theindex-Environment komplett neu ohne reledmac
\makeatletter
\renewenvironment{theindex}{%
  \section*{\indexname}%
  \setlength{\parindent}{0pt}%
  \setlength{\parskip}{0pt plus 0.3pt}%
  \let\item\@idxitem
}{%
  \clearpage
}
\makeatother

\IfFileExists{\jobname-pw.ind}{\input{\jobname-pw.ind}}{}

\end{document}

      