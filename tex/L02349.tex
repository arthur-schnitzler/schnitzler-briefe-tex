%% latex-korrekturansicht-vorspann.tex
%% Vorspann für die Korrekturansicht.
%% Lädt die gemeinsame Datei latex-vorspann.tex mit gesetztem Schalter.

\newif\ifkorrekturansicht
\korrekturansichttrue

\input{../tex-inputs/latex-vorspann}


               \section[Arthur Schnitzler an Robert Adam, 14. 7. 1920]{ Arthur Schnitzler an Robert Adam, 14. 7. 1920}\nopagebreak\mylabel{v}\rehead{ }\normalsize\beginnumbering\briefempfaengerindex{Adam, Robert@\textsc{Adam, Robert}!zzzSchnitzler, Arthur@\emph{von Arthur Schnitzler}!1920-07-141@{14. 7. 1920}|(be} \toendnotes[C]{\smallbreak\pagebreak[2]} \Standort{DLA, 96.34.2/22.}
\physDesc{Brief, 1 Blatt, 1 Seite
\newline{}Schreibmaschine
\newline{}Handschrift: schwarze Tinte (\noindent{}Unterschrift)}\toendnotes[C]{\smallbreak}\pstart
           \noindent{}{\pb}\textcolor{gray}{\textbf{Dr. Arthur Schnitzler}}\hfill 14. 7. 1920.\pend
           \pstart
           \textcolor{gray}{\textbf{\textcolor{pink}{Wien XVIII. Sternwartestrasse 71}{}\ledrightnote{\textcolor{pink}{Sternwartestraße}}}}\pend
           \pstart{}Sehr verehrter Herr Doktor.\pend\pstart
           So viel ich weiss sind Sie zu allem andern auch ein Kenner der arabischen
                    Sprache. Ich lege Ihnen hier \label{K_L02349_1v}\edtext{zwei
                        Zettel}{\lemma{\textnormal{\emph{zwei
                        Zettel}}}\Cendnote{\textnormal{Beilage nicht
                        erhalten.}}}\label{K_L02349_1h} bei, die einen arabischen Spruch enthalten sollen, und
                    würde Sie recht sehr um Uebersetzun{[}g{]} resp. Aufklärung
                    bitten. Ich sende dies für alle Fälle an Ihre \textcolor{pink}{Wien}{}\ledrightnote{\textcolor{pink}{Wien}}er Adresse, nehme aber an, dass Sie sich schon in \textcolor{pink}{Gutenstein}{}\ledrightnote{\textcolor{pink}{Gutenstein}} befinden, wo Sie sich völlig erholen
                    werden.\pend
           \pstart
           Entschuldigen Sie die Bemühung und seien Sie herzlichst gegrüsst
                    von{\\[\baselineskip]}Ihrem sehr ergebenen{\\[\baselineskip]}\spacefill\mbox{{[}hs.:{]} Arthur Schnitzler}\pend
           \leftskip=0em{}\pstart
           \noindent{}Herrn Oberlandesgerichtsrat Dr. Adam Pollak.\pend
           \endnumbering\briefempfaengerindex{Adam, Robert@\textsc{Adam, Robert}!zzzSchnitzler, Arthur@\emph{von Arthur Schnitzler}!1920-07-141@{14. 7. 1920}|)be}\mylabel{h}  \normalsize

\doendnotes{C}
\bigskip
\vfill

\clearpage

\footnotesize

\lohead{\textsc{register}}

% Definiere theindex-Environment komplett neu ohne reledmac
\makeatletter
\renewenvironment{theindex}{%
  \section*{\indexname}%
  \setlength{\parindent}{0pt}%
  \setlength{\parskip}{0pt plus 0.3pt}%
  \let\item\@idxitem
}{%
  \clearpage
}
\makeatother

\IfFileExists{\jobname-pw.ind}{\input{\jobname-pw.ind}}{}

\end{document}

      