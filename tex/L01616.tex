%% latex-korrekturansicht-vorspann.tex
%% Vorspann für die Korrekturansicht.
%% Lädt die gemeinsame Datei latex-vorspann.tex mit gesetztem Schalter.

\newif\ifkorrekturansicht
\korrekturansichttrue

\input{../tex-inputs/latex-vorspann}


               \section[Arthur Schnitzler an Georg Brandes, 25. 7. 1906]{ Arthur Schnitzler an Georg Brandes, 25. 7. 1906}\nopagebreak\mylabel{v}\rehead{ }\normalsize\beginnumbering\briefempfaengerindex{Brandes, Georg@\textsc{Brandes, Georg}!zzzSchnitzler, Arthur@\emph{von Arthur Schnitzler}!1906-07-251@{25. 7. 1906}|(be} \toendnotes[C]{\smallbreak\pagebreak[2]} \Standort{Kopenhagen, Det Kongelige Bibliotek, Georg Brandes Arkiv, box 125.}
\physDesc{Kartenbrief
\newline{}Handschrift: schwarze Tinte, deutsche Kurrent\newline{}Versand: Stempel: »\nobreak{}\oindex{Kopenhagen@\textbf{Kopenhagen}, \emph{Besiedelter Ort (A.BSO)}|pwk}Kjøbenhavn, 25 7 06, 70MB\nobreak{}«.  \newline{}Ordnung: mit Bleistift von unbekannter Hand nummeriert:
                                                »27.« }\buchAbdrucke{\weitereDrucke{Georg Brandes, Arthur Schnitzler: \emph{Ein Briefwechsel}. Hg. Kurt Bergel. Bern: \emph{Francke} 1956, S. 94.} }\pstart{}\textsc{{\pb}S}\pend{}\pstart{}Herrn \textsc{Georg Brandes}\pend{}\pstart{}\textcolor{pink}{\textsc{Kopenhagen}}{}\ledrightnote{\textcolor{pink}{Kopenhagen}}\pend{}\pstart{}\textcolor{pink}{\textsc{Havnegade} 55}{}\ledrightnote{\textcolor{pink}{Havnegade}}\pend{}{\bigskip}\pstart
           \raggedleft{}{\pb}\textcolor{pink}{\textsc{Marienlyst}}{}\ledrightnote{\textcolor{pink}{Marienlyst}}, 25. Juli 906\pend
           \pstart{}Verehrter Herr Brandes,\pend\pstart
           ja ich bin noch hier und werde mich ſehr freuen, Sie am Samſtag zu
                    ſehen. Um 6.15 ko{\geminationm}t Ihr Zug.\pend
           \pstart
           Herzlichſt{\\[\baselineskip]}Ihr{\\[\baselineskip]}\spacefill\mbox{Arthur Schnitzler}\pend
           \leftskip=0em{}\endnumbering\briefempfaengerindex{Brandes, Georg@\textsc{Brandes, Georg}!zzzSchnitzler, Arthur@\emph{von Arthur Schnitzler}!1906-07-251@{25. 7. 1906}|)be}\mylabel{h}  \normalsize

\doendnotes{C}
\bigskip
\vfill

\clearpage

\footnotesize

\lohead{\textsc{register}}

% Definiere theindex-Environment komplett neu ohne reledmac
\makeatletter
\renewenvironment{theindex}{%
  \section*{\indexname}%
  \setlength{\parindent}{0pt}%
  \setlength{\parskip}{0pt plus 0.3pt}%
  \let\item\@idxitem
}{%
  \clearpage
}
\makeatother

\IfFileExists{\jobname-pw.ind}{\input{\jobname-pw.ind}}{}

\end{document}

      