%% latex-korrekturansicht-vorspann.tex
%% Vorspann für die Korrekturansicht.
%% Lädt die gemeinsame Datei latex-vorspann.tex mit gesetztem Schalter.

\newif\ifkorrekturansicht
\korrekturansichttrue

\input{../tex-inputs/latex-vorspann}


               \section[Hugo von Hofmannsthal an Arthur Schnitzler, 26. 3. 1892]{ Hugo von Hofmannsthal an Arthur Schnitzler, 26. 3. 1892}\nopagebreak\mylabel{v}\rehead{ }\normalsize\beginnumbering\briefempfaengerindex{Schnitzler, Arthur@\textsc{Schnitzler, Arthur}!zzzHofmannsthal, Hugo von@\emph{von Hugo von Hofmannsthal}!1892-03-261@{26. 3. 1892}|(be} \toendnotes[C]{\smallbreak\pagebreak[2]} \Standort{CUL, Schnitzler, B 43.}
\physDesc{Postkarte
\newline{}Handschrift: schwarze Tinte, lateinische Kurrent\newline{}Versand: Stempel: »\nobreak{}Wien 1/1, 26. 3. 92, 10–1\textcolor{gray}{2}N\nobreak{}«.  
\newline{}Schnitzler: mit Bleistift datiert: »26/3 92« \newline{}Ordnung: von unbekannter Hand nummeriert:
                                    »21« }\buchAbdrucke{\weitereDrucke{Hugo von Hofmannsthal, Arthur Schnitzler: \emph{Briefwechsel}. Hg. Therese Nickl und Heinrich Schnitzler. Frankfurt am Main: \emph{S. Fischer} 1964, S. 18.} }\pstart{}{\pb}\textsc{Herrn D\textsuperscript{r} Arthur
                            Schnitzler}\pend{}\pstart{}\textsc{\textcolor{pink}{I. Wien}{}\ledrightnote{\textcolor{pink}{I., Innere Stadt}}}\pend{}\pstart{}\textsc{\textcolor{pink}{Kärntnerring 12}{}\ledrightnote{\textcolor{pink}{Kärntnerring}}}\pend{}{\bigskip}\pstart{}{\pb}Lieber Freund,\pend\pstart
           Ich bin für morgen zu Tisch geladen. Es ist also wieder nichts. Herr \textcolor{blue}{Bölsche}{}\ledrightnote{\textcolor{blue}{Wilhelm Bölsche}} hat mir das »\textcolor{green}{Kind}{}\ledrightnote{\textcolor{green}{Age of Innocence}}« zurückgeschickt; natürlich mit einem sehr artigen
                    Brief.\pend
           \pstart
           Auf Wiedersehen!{\\[\baselineskip]}\spacefill\mbox{Loris.}\pend
           \leftskip=0em{}\pstart
           Samstag.\pend
           \endnumbering\briefempfaengerindex{Schnitzler, Arthur@\textsc{Schnitzler, Arthur}!zzzHofmannsthal, Hugo von@\emph{von Hugo von Hofmannsthal}!1892-03-261@{26. 3. 1892}|)be}\mylabel{h}  \normalsize

\doendnotes{C}
\bigskip
\vfill

\clearpage

\footnotesize

\lohead{\textsc{register}}

% Definiere theindex-Environment komplett neu ohne reledmac
\makeatletter
\renewenvironment{theindex}{%
  \section*{\indexname}%
  \setlength{\parindent}{0pt}%
  \setlength{\parskip}{0pt plus 0.3pt}%
  \let\item\@idxitem
}{%
  \clearpage
}
\makeatother

\IfFileExists{\jobname-pw.ind}{\input{\jobname-pw.ind}}{}

\end{document}

      