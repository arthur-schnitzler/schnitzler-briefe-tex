%% latex-korrekturansicht-vorspann.tex
%% Vorspann für die Korrekturansicht.
%% Lädt die gemeinsame Datei latex-vorspann.tex mit gesetztem Schalter.

\newif\ifkorrekturansicht
\korrekturansichttrue

\input{../tex-inputs/latex-vorspann}


               \section[Hugo von Hofmannsthal und Richard Beer-Hofmann an Arthur Schnitzler, 27. {[}9. 1899{]}]{ Hugo von Hofmannsthal und Richard Beer-Hofmann an Arthur Schnitzler,
               27. {[}9. 1899{]}}\nopagebreak\mylabel{v}\rehead{ }\normalsize\beginnumbering\briefempfaengerindex{Schnitzler, Arthur@\textsc{Schnitzler, Arthur}!zzzBeer-Hofmann, Richard@\emph{von Richard Beer-Hofmann}!1899-09-271@{27. 9. 1899}|(be}\briefempfaengerindex{Schnitzler, Arthur@\textsc{Schnitzler, Arthur}!zzzHofmannsthal, Hugo von@\emph{von Hugo von Hofmannsthal}!1899-09-271@{27. 9. 1899}|(be} \toendnotes[C]{\smallbreak\pagebreak[2]} \Standort{CUL, Schnitzler, B 43.}
\physDesc{Brief, 1 Blatt, 2 Seiten
\newline{}Handschrift Richard Beer-Hofmann: schwarze Tinte, lateinische Kurrent\newline{}Handschrift Hugo von Hofmannsthal: schwarze Tinte, deutsche Kurrent
\newline{}Schnitzler: mit Bleistift Monat und Jahreszahl ergänzt: »9. 99.« \newline{}Ordnung: 1) mit Bleistift von unbekannter Hand nummeriert: »\strikeout{162}« 2) mit Bleistift von unbekannter Hand nummeriert: »159«}\buchAbdrucke{\weitereDrucke{Hugo von Hofmannsthal, Arthur Schnitzler: \emph{Briefwechsel}. Hg. Therese Nickl und Heinrich Schnitzler. Frankfurt am Main: \emph{S. Fischer} 1964, S. 130–131.} }\toendnotes[C]{\smallbreak}\pstart
           \raggedleft{}{\pb}\textcolor{pink}{Vahrn}{}\ledrightnote{\textcolor{pink}{Vahrn}}, 27.\pend
           \pstart{}mein lieber Arthur\pend\pstart
           wir ſind beide recht fleißig, ſo ähnlich wie wir 2 in \textcolor{pink}{Iſchl}{}\ledrightnote{\textcolor{pink}{Bad Ischl}}. Mein \textcolor{green}{Stück}{}\ledrightnote{→\textcolor{green}{Das Bergwerk zu Falun}} aber
               wird immer ſchwerer oder ich immer dümmer. Morgen geht der
               Richard nach \textcolor{pink}{\textsc{St. Michael im Eppan}}{}\ledrightnote{\textcolor{pink}{Sankt Michael}}, und ich nach \textcolor{pink}{Venedig}{}\ledrightnote{\textcolor{pink}{Venedig}}, \textcolor{pink}{Hotel Britannia}{}\ledrightnote{\textcolor{pink}{Grand Hotel Britannia}}. Vielleicht werde ich dort geſcheidter. Dieſes
               wünſcht Ihnen ſehr\pend
           \pstart
           Ihr{\\[\baselineskip]}\spacefill\mbox{Hugo}\pend
           \leftskip=0em{}\pstart
           \noindent{}{[}hs. Beer-Hofmann:{]} Hugos Wünschen schließe ich mich an. \textcolor{blue}{Paul}{}\ledrightnote{\textcolor{blue}{Paul Goldmann}} scheint nach \textcolor{pink}{Florenz}{}\ledrightnote{\textcolor{pink}{Florenz}}
               gereist zu sein – ohne mich aufzusuchen. Was für Folgerungen hätte \textcolor{blue}{Paul}{}\ledrightnote{\textcolor{blue}{Paul Goldmann}} gezogen wenn ich das gethan hätte! Ich bin sehr froh daß
               ich nicht nach \textcolor{pink}{Florenz}{}\ledrightnote{\textcolor{pink}{Florenz}} gereist bin u. \textcolor{blue}{Paul}{}\ledrightnote{\textcolor{blue}{Paul Goldmann}} in \textcolor{pink}{Vahrn}{}\ledrightnote{\textcolor{pink}{Vahrn}}
               ist. Meine Adresse ist \textcolor{pink}{St. Michael im Eppan}{}\ledrightnote{\textcolor{pink}{Sankt Michael}} – und
                  »\uline{fartig}«.\pend
           \pstart
           \label{T_L00981_1v}\edtext{Das}{\lemma{\textnormal{\emph{Das}}}\Cendnote{\textnormal{Ein Pfeil weist auf »fartig«.}}}\label{T_L00981_1h} wünscht Ihnen Ihr{\\[\baselineskip]}\spacefill\mbox{Richard}\pend
           \leftskip=0em{}\endnumbering\briefempfaengerindex{Schnitzler, Arthur@\textsc{Schnitzler, Arthur}!zzzBeer-Hofmann, Richard@\emph{von Richard Beer-Hofmann}!1899-09-271@{27. 9. 1899}|)be}\briefempfaengerindex{Schnitzler, Arthur@\textsc{Schnitzler, Arthur}!zzzHofmannsthal, Hugo von@\emph{von Hugo von Hofmannsthal}!1899-09-271@{27. 9. 1899}|)be}\mylabel{h}  \normalsize

\doendnotes{C}
\bigskip
\vfill

\clearpage

\footnotesize

\lohead{\textsc{register}}

% Definiere theindex-Environment komplett neu ohne reledmac
\makeatletter
\renewenvironment{theindex}{%
  \section*{\indexname}%
  \setlength{\parindent}{0pt}%
  \setlength{\parskip}{0pt plus 0.3pt}%
  \let\item\@idxitem
}{%
  \clearpage
}
\makeatother

\IfFileExists{\jobname-pw.ind}{\input{\jobname-pw.ind}}{}

\end{document}

      