%% latex-korrekturansicht-vorspann.tex
%% Vorspann für die Korrekturansicht.
%% Lädt die gemeinsame Datei latex-vorspann.tex mit gesetztem Schalter.

\newif\ifkorrekturansicht
\korrekturansichttrue

\input{../tex-inputs/latex-vorspann}


               \section[Arthur Schnitzler an Robert Adam, 29. 6. 1915]{ Arthur Schnitzler an Robert Adam, 29. 6. 1915}\nopagebreak\mylabel{v}\rehead{ }\normalsize\beginnumbering\briefempfaengerindex{Adam, Robert@\textsc{Adam, Robert}!zzzSchnitzler, Arthur@\emph{von Arthur Schnitzler}!1915-06-291@{29. 6. 1915}|(be} \toendnotes[C]{\smallbreak\pagebreak[2]} \Standort{DLA, 96.34.1/13.}
\physDesc{Briefkarte, Umschlag
\newline{}Handschrift: schwarze Tinte, lateinische Kurrent\newline{}Versand: Stempel: »\nobreak{}Wien\nobreak{}«.  }\toendnotes[C]{\smallbreak}\pstart{}{\pb}\textcolor{gray}{\textbf{Dr. Arthur Schnitzler}}\pend{}\pstart{}\textcolor{gray}{\textbf{\textcolor{pink}{Wien XVIII. Sternwartestrasse 71}{}\ledrightnote{\textcolor{pink}{Sternwartestraße}}}}\pend{}{\bigskip}\pstart{}{\pb}Hrn Dr. Robert Adam Pollak,\pend{}\pstart{}\textcolor{pink}{Wien XII}{}\ledrightnote{\textcolor{pink}{XII., Meidling}}\pend{}\pstart{}\textcolor{pink}{Meidlinger Hauptstr 56}{}\ledrightnote{\textcolor{pink}{Meidlinger Hauptstraße}}\pend{}{\bigskip}\pstart
           \noindent{}{\pb}\textcolor{gray}{\textbf{Dr. Arthur Schnitzler}}\hfill 29. 6. 191\textcolor{gray}{5}\pend
           \pstart
           \textcolor{gray}{\textbf{\textcolor{pink}{Wien XVIII. Sternwartestrasse 71}{}\ledrightnote{\textcolor{pink}{Sternwartestraße}}}}\pend
           \pstart
           verehrter Herr Doctor, es hat sich in all diesen Tagen nicht
                    gefügt, daß ich den \textcolor{blue}{Leiter}{}\ledrightnote{→\textcolor{blue}{Hugo Thimig}}
                    des \textcolor{pink}{Burgtheaters}{}\ledrightnote{\textcolor{pink}{Burgtheater}} sprach; – doch hab ich mir
                    erlaubt, den Regisseur und Schauspieler des \textcolor{brown}{Münchner
                        Hoftheaters}{}\ledrightnote{\textcolor{brown}{Königliche Hof- und Nationaltheater München}}, meinen Schwager \textcolor{blue}{Albert
                        Steinrück}{}\ledrightnote{\textcolor{blue}{Albert Steinrück}}, der über den Mangel an neuen Stücken klagte, auf Sie und
                    Ihre drei Dramen (\textcolor{green}{Abû Bekkr}{}\ledrightnote{\textcolor{green}{Die Geschichte des Alî ibn Bekkâr mit Schams an-Nahâr}}, \textcolor{green}{Fremdling}{}\ledrightnote{\textcolor{green}{Der Fremde}} und das \textcolor{green}{dritte}{}\ledrightnote{→\textcolor{green}{Fatme}}, dessen Name mir eben entfiel –) in
                    gebührender Weise aufmerksam zu machen, {\pb}und ich würde
                    Ihnen rathen, all das, unter Berufung auf mich an \textcolor{blue}{St.}{}\ledrightnote{\textcolor{blue}{Albert Steinrück}}, d. h. \textcolor{pink}{Partenkirchen, Villa
                        Zufriedenheit}{}\ledrightnote{\textcolor{pink}{Villa Zufriedenheit}} abzusenden. – Die anderen Chancen verlier ich damit nicht
                    aus dem Auge; aber wie schon gesagt, ich warte ein persönliches Zusammentreffen
                    mit den betreffenden Partnern ab.\pend
           \pstart
           Übermorgen fahr ich nach \textcolor{pink}{Altaussee (Villa
                        Annerl}{}\ledrightnote{\textcolor{pink}{Villa Annerl}}), denke im September daheim zu sein und hoffe Sie
                    bald wiederzusehn.\pend
           \pstart
           herzlichlich grüßend Ihr ergebner{\\[\baselineskip]}\spacefill\mbox{A. S.}\pend
           \leftskip=0em{}\endnumbering\briefempfaengerindex{Adam, Robert@\textsc{Adam, Robert}!zzzSchnitzler, Arthur@\emph{von Arthur Schnitzler}!1915-06-291@{29. 6. 1915}|)be}\mylabel{h}  \normalsize

\doendnotes{C}
\bigskip
\vfill

\clearpage

\footnotesize

\lohead{\textsc{register}}

% Definiere theindex-Environment komplett neu ohne reledmac
\makeatletter
\renewenvironment{theindex}{%
  \section*{\indexname}%
  \setlength{\parindent}{0pt}%
  \setlength{\parskip}{0pt plus 0.3pt}%
  \let\item\@idxitem
}{%
  \clearpage
}
\makeatother

\IfFileExists{\jobname-pw.ind}{\input{\jobname-pw.ind}}{}

\end{document}

      