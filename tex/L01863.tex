%% latex-korrekturansicht-vorspann.tex
%% Vorspann für die Korrekturansicht.
%% Lädt die gemeinsame Datei latex-vorspann.tex mit gesetztem Schalter.

\newif\ifkorrekturansicht
\korrekturansichttrue

\input{../tex-inputs/latex-vorspann}


               \section[Richard Beer-Hofmann an Arthur Schnitzler, 4. 8. 1909]{ Richard Beer-Hofmann an Arthur Schnitzler, 4. 8. 1909}\nopagebreak\mylabel{v}\rehead{ }\normalsize\beginnumbering\briefempfaengerindex{Schnitzler, Arthur@\textsc{Schnitzler, Arthur}!zzzBeer-Hofmann, Richard@\emph{von Richard Beer-Hofmann}!1909-08-041@{4. 8. 1909}|(be} \toendnotes[C]{\smallbreak\pagebreak[2]} \Standort{CUL, Schnitzler, B 8.}
\physDesc{Brief, 1 Blatt, 2 Seiten
\newline{}Handschrift: schwarze Tinte, lateinische Kurrent
\newline{}Schnitzler: mit Bleistift beschriftet: »\textsc{Beer Hofm.}« \newline{}Ordnung: mit Bleistift von unbekannter Hand nummeriert:
                              »221« }\buchAbdrucke{\weitereDrucke{Arthur Schnitzler, Richard Beer-Hofmann: \emph{Briefwechsel 1891–1931}. Hg. Konstanze Fliedl. Wien, Zürich: \emph{Europaverlag} 1992, S. 194–195.} }\toendnotes[C]{\smallbreak}\pstart
           \raggedleft{}{\pb}\textcolor{pink}{Wien}{}\ledrightnote{\textcolor{pink}{Wien}}{ }4./VIII. 09.\pend
           \pstart
           Lieber Arthur! Dank für Ihren Brief. Es war nicht schön; man prügelt
               uns zu oft.\pend
           \pstart
           Wir wollen am 9 hier weg, in \textcolor{pink}{Villach}{}\ledrightnote{\textcolor{pink}{Villach}}
               übernachten, am 10, am \textcolor{pink}{Lido}{}\ledrightnote{\textcolor{pink}{Lido}}. \textcolor{blue}{Paula}{}\ledrightnote{\textcolor{blue}{Paula Beer-Hofmann}} braucht Wärme, und Sonne, und die haben wir
               – hoffe ich – doch sicherer da unten – am \textcolor{pink}{Lido}{}\ledrightnote{\textcolor{pink}{Lido}} meine
               ich. Sonst wären wir sehr gerne mit Ihnen beisa{\geminationm}en
               gewesen.\pend
           \pstart
           Von \textcolor{blue}{Leo}{}\ledrightnote{\textcolor{blue}{Leo Van-Jung}} hörte ich, dass es {\pb}Ihnen Allen gut geht.\pend
           \pstart
           Ich dachte daran auf einen Tag zu Ihnen zu ko{\geminationm}en, aber
               es ist zu viel Hetze und wir sind so müde.\pend
           \pstart
           Des \textcolor{green}{Medardus}{}\ledrightnote{\textcolor{green}{Der junge Medardus. Dramatische Historie in einem Vorspiel und fünf Aufzügen}} Schicksal hat mich sehr gefreut. Wann
               werde ich ihn kennen lernen. Herzliche Grüsse Ihnen, Ihrer \textcolor{blue}{Frau}{}\ledrightnote{→\textcolor{blue}{Olga Schnitzler}} und dem \textcolor{blue}{Buben}{}\ledrightnote{→\textcolor{blue}{Heinrich Schnitzler}}.\pend
           \pstart
           Ihr{\\[\baselineskip]}\spacefill\mbox{Richard}\pend
           \leftskip=0em{}\endnumbering\briefempfaengerindex{Schnitzler, Arthur@\textsc{Schnitzler, Arthur}!zzzBeer-Hofmann, Richard@\emph{von Richard Beer-Hofmann}!1909-08-041@{4. 8. 1909}|)be}\mylabel{h}  \normalsize

\doendnotes{C}
\bigskip
\vfill

\clearpage

\footnotesize

\lohead{\textsc{register}}

% Definiere theindex-Environment komplett neu ohne reledmac
\makeatletter
\renewenvironment{theindex}{%
  \section*{\indexname}%
  \setlength{\parindent}{0pt}%
  \setlength{\parskip}{0pt plus 0.3pt}%
  \let\item\@idxitem
}{%
  \clearpage
}
\makeatother

\IfFileExists{\jobname-pw.ind}{\input{\jobname-pw.ind}}{}

\end{document}

      