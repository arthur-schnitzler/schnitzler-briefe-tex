%% latex-korrekturansicht-vorspann.tex
%% Vorspann für die Korrekturansicht.
%% Lädt die gemeinsame Datei latex-vorspann.tex mit gesetztem Schalter.

\newif\ifkorrekturansicht
\korrekturansichttrue

\input{../tex-inputs/latex-vorspann}


               \section[Julius Rodenberg an Arthur Schnitzler, 9. 3. 1899]{ Julius Rodenberg an Arthur Schnitzler, 9. 3. 1899}\nopagebreak\mylabel{v}\rehead{ }\normalsize\beginnumbering\briefempfaengerindex{Schnitzler, Arthur@\textsc{Schnitzler, Arthur}!zzzRodenberg, Julius@\emph{von Julius Rodenberg}!1899-03-092@{9. 3. 1899}|(be} \toendnotes[C]{\smallbreak\pagebreak[2]} \Standort{CUL, Schnitzler, B 85.}
\physDesc{Brief, 1 Blatt, 1 Seite
\newline{}Handschrift: schwarze Tinte, deutsche Kurrent
\newline{}Schnitzler: mit rotem Buntstift eine Unterstreichung }\toendnotes[C]{\smallbreak}\pstart
           \noindent{}\centering{}{\pb}\textcolor{gray}{\textbf{\textcolor{brown}{DEUTSCHE RUNDSCHAU}{}\ledrightnote{\textcolor{brown}{Deutsche Rundschau}}}}\pend
           \pstart
           \noindent{}\textcolor{gray}{\textbf{Expedition u. Redaction:}}\hfill \textcolor{gray}{\textbf{Herausgeber:}}\pend
           \pstart
           \textcolor{gray}{\textbf{\textcolor{brown}{Gebrüder Paetel}{}\ledrightnote{\textcolor{brown}{Gebrüder Paetel Verlag}} in \textcolor{pink}{Berlin}{}\ledrightnote{\textcolor{pink}{Berlin}}}}\hfill \textcolor{gray}{\textbf{Julius Rodenberg in \textcolor{pink}{Berlin}{}\ledrightnote{\textcolor{pink}{Berlin}}}}\pend
           \pstart
           \textcolor{gray}{\textbf{(\textcolor{blue}{Elwin
                                Paetel}{}\ledrightnote{\textcolor{blue}{Elwin Paetel}})}}\hfill \textcolor{gray}{\textbf{\textcolor{pink}{W., Margarethenstr. 1}{}\ledrightnote{\textcolor{pink}{Margaretenstraße}}.}}\pend
           \pstart
           \textcolor{gray}{\textbf{\textcolor{pink}{W., Lützowstr. 7}{}\ledrightnote{\textcolor{pink}{Lützowstraße}}.}}\pend
           \pstart
           \raggedleft{}\textbf{\textcolor{gray}{\textbf{\textcolor{pink}{Berlin W.}{}\ledrightnote{\textcolor{pink}{Berlin}},}} den}{ }9. März \textcolor{gray}{\textbf{189}}9.\pend
           \pstart{}Hochgeehrter Herr Doctor!\pend\pstart
           Für Ihr freundliches Anerbieten bin ich Ihnen aufrichtig dankbar, doch vermuthen
                    Sie mit Recht, daß die »\textcolor{brown}{\textsc{Rundschau}}{}\ledrightnote{\textcolor{brown}{Deutsche Rundschau}}« dramatiſche Dichtungen grundſätzlich nicht bringt. Wir haben wohl, in
                    weiten Abſtänden, einmal eine Ausnahme gemacht, aber i{\geminationm}er nur, um wieder zu der Regel zurückzukehren; u.
                    ſo gern ich Ihren geiſtvollen \textcolor{green}{Einakter}{}\ledrightnote{→\textcolor{green}{Die Gefährtin. Schauspiel in einem Akt}} in unſerer Zeitſchrift ſähe, ſo kann ich es doch nicht, ohne
                    inconſequent gegen Andere zu erſcheinen – um ſo weniger, als ich vor Jahr und
                    Tag ſchon eine ſzeniſche Kleinigkeit von einem unſerer berühmten Mitarbeiter
                    angenommen habe, die doch zuerſt publiciert werden müßte. Sie werden es unter
                    dieſen Umſtänden entſchuldbar finden, wenn ich mit wiederholtem Dank ablehne,
                    dagegen hoffe, recht bald durch eine Novelle ſchadlos gehalten zu werden, die
                    des Willkomms ſicher ſein darf.\pend
           \pstart
           Hochachtungsvoll ergeben{\\[\baselineskip]}Ihr{\\[\baselineskip]}\spacefill\mbox{Dr Julius Rodenberg.}\pend
           \leftskip=0em{}\endnumbering\briefempfaengerindex{Schnitzler, Arthur@\textsc{Schnitzler, Arthur}!zzzRodenberg, Julius@\emph{von Julius Rodenberg}!1899-03-092@{9. 3. 1899}|)be}\mylabel{h}  \normalsize

\doendnotes{C}
\bigskip
\vfill

\clearpage

\footnotesize

\lohead{\textsc{register}}

% Definiere theindex-Environment komplett neu ohne reledmac
\makeatletter
\renewenvironment{theindex}{%
  \section*{\indexname}%
  \setlength{\parindent}{0pt}%
  \setlength{\parskip}{0pt plus 0.3pt}%
  \let\item\@idxitem
}{%
  \clearpage
}
\makeatother

\IfFileExists{\jobname-pw.ind}{\input{\jobname-pw.ind}}{}

\end{document}

      