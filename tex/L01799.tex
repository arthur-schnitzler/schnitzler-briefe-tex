%% latex-korrekturansicht-vorspann.tex
%% Vorspann für die Korrekturansicht.
%% Lädt die gemeinsame Datei latex-vorspann.tex mit gesetztem Schalter.

\newif\ifkorrekturansicht
\korrekturansichttrue

\input{../tex-inputs/latex-vorspann}


               \section[Hermann Bodmer und andere an Arthur Schnitzler, 9. 11. 1908]{ Hermann Bodmer und andere an Arthur Schnitzler, 9. 11. 1908}\nopagebreak\mylabel{v}\rehead{ }\normalsize\beginnumbering\briefempfaengerindex{Schnitzler, Arthur@\textsc{Schnitzler, Arthur}!zzzBahr, Hermann@\emph{von Hermann Bahr}!1908-11-092@{9. 11. 1908}|(be}\briefempfaengerindex{Schnitzler, Arthur@\textsc{Schnitzler, Arthur}!zzzBodmer, Hans@\emph{von Hans Bodmer}!1908-11-092@{9. 11. 1908}|(be}\briefempfaengerindex{Schnitzler, Arthur@\textsc{Schnitzler, Arthur}!zzzSartoris, M.@\emph{von M. Sartoris}!1908-11-092@{9. 11. 1908}|(be}\briefempfaengerindex{Schnitzler, Arthur@\textsc{Schnitzler, Arthur}!zzzSartoris, Spyridon Demetrius@\emph{von Spyridon Demetrius Sartoris}!1908-11-092@{9. 11. 1908}|(be}\briefempfaengerindex{Schnitzler, Arthur@\textsc{Schnitzler, Arthur}!zzzBodman, Emanuel von@\emph{von Emanuel von Bodman}!1908-11-092@{9. 11. 1908}|(be}\briefempfaengerindex{Schnitzler, Arthur@\textsc{Schnitzler, Arthur}!zzzBodmer, Hermann@\emph{von Hermann Bodmer}!1908-11-092@{9. 11. 1908}|(be}\briefempfaengerindex{Schnitzler, Arthur@\textsc{Schnitzler, Arthur}!zzzBodmer, Mathilde@\emph{von Mathilde Bodmer}!1908-11-092@{9. 11. 1908}|(be}\briefempfaengerindex{Schnitzler, Arthur@\textsc{Schnitzler, Arthur}!zzzKesser, Hermann@\emph{von Hermann Kesser}!1908-11-092@{9. 11. 1908}|(be}\briefempfaengerindex{Schnitzler, Arthur@\textsc{Schnitzler, Arthur}!zzzStaehelin-Baechtold, Gertrud@\emph{von Gertrud Staehelin-Baechtold}!1908-11-092@{9. 11. 1908}|(be}\briefempfaengerindex{Schnitzler, Arthur@\textsc{Schnitzler, Arthur}!zzzHuber, R. W.@\emph{von R. W. Huber}!1908-11-092@{9. 11. 1908}|(be}\briefempfaengerindex{Schnitzler, Arthur@\textsc{Schnitzler, Arthur}!zzzStaehelin-Baechtold, Sepp@\emph{von Sepp Staehelin-Baechtold}!1908-11-092@{9. 11. 1908}|(be}\briefempfaengerindex{Schnitzler, Arthur@\textsc{Schnitzler, Arthur}!zzzProbst, R.@\emph{von R. Probst}!1908-11-092@{9. 11. 1908}|(be}\briefempfaengerindex{Schnitzler, Arthur@\textsc{Schnitzler, Arthur}!zzzProbst, E.@\emph{von E. Probst}!1908-11-092@{9. 11. 1908}|(be} \toendnotes[C]{\smallbreak\pagebreak[2]} \Standort{DLA, A:Schnitzler, HS.NZ85.1.2568.}
\physDesc{Bildpostkarte
\newline{}Handschrift E. Probst: Bleistift\newline{}Handschrift R. Probst: Bleistift\newline{}Handschrift Sepp Staehelin-Baechtold: Bleistift\newline{}Handschrift R. W. Huber: Bleistift\newline{}Handschrift Gertrud Staehelin-Baechtold: Bleistift\newline{}Handschrift Hermann Kesser: Bleistift\newline{}Handschrift Mathilde Bodmer: Bleistift\newline{}Handschrift Hermann Bodmer: Bleistift, lateinische Kurrent\newline{}Handschrift Emanuel von Bodman: Bleistift\newline{}Handschrift Spyridon Demetrius Sartoris: Bleistift\newline{}Handschrift M. Sartoris: Bleistift\newline{}Handschrift Hans Bodmer: Bleistift, lateinische Kurrent\newline{}Handschrift Hermann Bahr: Bleistift, deutsche Kurrent\newline{}Versand: Stempel: »\nobreak{}\oindex{Neumuenster@\textbf{Neumünster}, \emph{Bezirk (A.BZK)}|pwk}Zürich 12 Neumünster, 10. XI. 08, 2\nobreak{}«.  }\toendnotes[C]{\smallbreak}\pstart{}{\pb}D\textsuperscript{r} Artur
                  Schnitzler\pend{}\pstart{}\textcolor{pink}{Wien XVIII}{}\ledrightnote{\textcolor{pink}{XVIII., Währing}}\pend{}\pstart{}\textcolor{pink}{Spöttelgasse 7}{}\ledrightnote{\textcolor{pink}{Edmund-Weiß-Gasse}}\pend{}{\bigskip}\pstart
           \noindent{}\centering{}{\pb}\textcolor{gray}{\textbf{\textcolor{pink}{Zürich – Rotes Schloss}{}\ledrightnote{\textcolor{pink}{Rotes Schloss}} und \textcolor{pink}{Tonhalle}{}\ledrightnote{\textcolor{pink}{Kongresshaus Zürich}}}}\pend
           \stanza{}{\pb}{[}hs. Bodmer:{]} Von Bahr vernahmen das Dilemma\newverse{}Wir eben zwischen \textcolor{green}{Franz}{}\ledrightnote{→\textcolor{green}{Die Toten schweigen}} u \textcolor{green}{Emma}{}\ledrightnote{→\textcolor{green}{Die Toten schweigen}}\newverse{}Ja es ist bitter, wenn die »\textcolor{green}{Toten schweigen}{}\ledrightnote{\textcolor{green}{Die Toten schweigen}}«\newverse{}Doch jetzt, in frohem, lust’gem »\textcolor{green}{Reigen}{}\ledrightnote{→\textcolor{green}{Reigen. Zehn Dialoge}}« \newverse{}Von guter Speis’ u Trank ganz voll\newverse{}Gedenken wir des »\textcolor{green}{Anatol}{}\ledrightnote{\textcolor{green}{Anatol}}«\newverse{}Und wünschen sehnlich ihn herbei\newverse{}Das gäb’ ne nette »\textcolor{green}{Liebelei}{}\ledrightnote{\textcolor{green}{Liebelei. Schauspiel in drei Akten}}«\newverse{}Und mit dem »\textcolor{green}{grünen Kakadu}{}\ledrightnote{\textcolor{green}{Der grüne Kakadu. Groteske in einem Akt}}«\newverse{}Wär’n wir gar balde Du und Du!!\stanzaend{}\pstart
           \noindent{}{[}hs. Bahr:{]} \label{T_L01799_1v}\edtext{Herzlich \spacefill\mbox{Hermann}}{\lemma{\textnormal{\emph{Herzlich Hermann}}}\Cendnote{\textnormal{in der oberen linken Ecke, verkehrt zum
                  Text}}}\label{T_L01799_1h}\pend
           \pstart
           \noindent{}{\pb}{[}hs. Bodmer:{]} Vom Schnitzler-Abend, den uns Herma{\geminationn} Bahr bot, senden wir Ihnen herzliche Grüsse:\pend
           \pstart
           \spacefill\mbox{Hans Bodmer}\pend
           \pstart
           \spacefill\mbox{{[}hs. Probst:{]} EProbst}\pend
           \pstart
           \spacefill\mbox{{[}hs. Probst:{]} R. Probst}\pend
           \pstart
           \spacefill\mbox{{[}hs. Staehelin-Baechtold:{]} Sepp Staehelin}\pend
           \pstart
           \spacefill\mbox{{[}hs. Huber:{]} R. W. Huber.}\pend
           \pstart
           \spacefill\mbox{{[}hs. Staehelin-Baechtold:{]} Gertrud Staehelin-Baechtold.}\pend
           \pstart
           \spacefill\mbox{{[}hs. Kesser:{]} Hermann Kesser}\pend
           \pstart
           \spacefill\mbox{{[}hs. Bodmer:{]} Mathilde Bodmer}\pend
           \pstart
           \spacefill\mbox{{[}hs. Bodmer:{]} Hermann Bodmer}\pend
           \pstart
           \spacefill\mbox{{[}hs. Bodman:{]} Emanuel von Bodman}\pend
           \pstart
           \spacefill\mbox{{[}hs. Sartoris:{]} Spyridon Sartoris}\pend
           \pstart
           \spacefill\mbox{{[}hs. Sartoris:{]} M. Sartoris.}\pend
           \endnumbering\briefempfaengerindex{Schnitzler, Arthur@\textsc{Schnitzler, Arthur}!zzzBahr, Hermann@\emph{von Hermann Bahr}!1908-11-092@{9. 11. 1908}|)be}\briefempfaengerindex{Schnitzler, Arthur@\textsc{Schnitzler, Arthur}!zzzBodmer, Hans@\emph{von Hans Bodmer}!1908-11-092@{9. 11. 1908}|)be}\briefempfaengerindex{Schnitzler, Arthur@\textsc{Schnitzler, Arthur}!zzzSartoris, M.@\emph{von M. Sartoris}!1908-11-092@{9. 11. 1908}|)be}\briefempfaengerindex{Schnitzler, Arthur@\textsc{Schnitzler, Arthur}!zzzSartoris, Spyridon Demetrius@\emph{von Spyridon Demetrius Sartoris}!1908-11-092@{9. 11. 1908}|)be}\briefempfaengerindex{Schnitzler, Arthur@\textsc{Schnitzler, Arthur}!zzzBodman, Emanuel von@\emph{von Emanuel von Bodman}!1908-11-092@{9. 11. 1908}|)be}\briefempfaengerindex{Schnitzler, Arthur@\textsc{Schnitzler, Arthur}!zzzBodmer, Hermann@\emph{von Hermann Bodmer}!1908-11-092@{9. 11. 1908}|)be}\briefempfaengerindex{Schnitzler, Arthur@\textsc{Schnitzler, Arthur}!zzzBodmer, Mathilde@\emph{von Mathilde Bodmer}!1908-11-092@{9. 11. 1908}|)be}\briefempfaengerindex{Schnitzler, Arthur@\textsc{Schnitzler, Arthur}!zzzKesser, Hermann@\emph{von Hermann Kesser}!1908-11-092@{9. 11. 1908}|)be}\briefempfaengerindex{Schnitzler, Arthur@\textsc{Schnitzler, Arthur}!zzzStaehelin-Baechtold, Gertrud@\emph{von Gertrud Staehelin-Baechtold}!1908-11-092@{9. 11. 1908}|)be}\briefempfaengerindex{Schnitzler, Arthur@\textsc{Schnitzler, Arthur}!zzzHuber, R. W.@\emph{von R. W. Huber}!1908-11-092@{9. 11. 1908}|)be}\briefempfaengerindex{Schnitzler, Arthur@\textsc{Schnitzler, Arthur}!zzzStaehelin-Baechtold, Sepp@\emph{von Sepp Staehelin-Baechtold}!1908-11-092@{9. 11. 1908}|)be}\briefempfaengerindex{Schnitzler, Arthur@\textsc{Schnitzler, Arthur}!zzzProbst, R.@\emph{von R. Probst}!1908-11-092@{9. 11. 1908}|)be}\briefempfaengerindex{Schnitzler, Arthur@\textsc{Schnitzler, Arthur}!zzzProbst, E.@\emph{von E. Probst}!1908-11-092@{9. 11. 1908}|)be}\mylabel{h}  \normalsize

\doendnotes{C}
\bigskip
\vfill

\clearpage

\footnotesize

\lohead{\textsc{register}}

% Definiere theindex-Environment komplett neu ohne reledmac
\makeatletter
\renewenvironment{theindex}{%
  \section*{\indexname}%
  \setlength{\parindent}{0pt}%
  \setlength{\parskip}{0pt plus 0.3pt}%
  \let\item\@idxitem
}{%
  \clearpage
}
\makeatother

\IfFileExists{\jobname-pw.ind}{\input{\jobname-pw.ind}}{}

\end{document}

      