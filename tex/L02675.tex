%% latex-korrekturansicht-vorspann.tex
%% Vorspann für die Korrekturansicht.
%% Lädt die gemeinsame Datei latex-vorspann.tex mit gesetztem Schalter.

\newif\ifkorrekturansicht
\korrekturansichttrue

\input{../tex-inputs/latex-vorspann}


               \section[Paul Goldmann an Arthur Schnitzler, 18. 12. {[}1891{]}]{ Paul Goldmann an Arthur Schnitzler, 18. 12. {[}1891{]}}\nopagebreak\mylabel{v}\rehead{ }\normalsize\beginnumbering\briefempfaengerindex{Schnitzler, Arthur@\textsc{Schnitzler, Arthur}!zzzGoldmann, Paul@\emph{von Paul Goldmann}!1891-12-181@{18. 12. {[}1891{]}}|(be} \toendnotes[C]{\smallbreak\pagebreak[2]} \Standort{DLA, A:Schnitzler, HS.NZ85.1.3162.}
\physDesc{Brief, 1 Blatt, 4 Seiten
\newline{}Handschrift: schwarze Tinte, deutsche Kurrent
\newline{}Schnitzler: mit Bleistift das Jahr »91« vermerkt }\toendnotes[C]{\smallbreak}\pstart
           \centering{}{\pb}\textcolor{pink}{Paris}{}\ledrightnote{\textcolor{pink}{Paris}}, 18. December.\pend
           \pstart\center{}Mein lieber Arthur!\pend\pstart
           Ich habe gerade deinen Brief erhalten u. laufe raſch in das \label{K_L02675-445v}\edtext{nächſtliegende \textsc{\textcolor{pink}{Café de la Paix}{}\ledrightnote{\textcolor{pink}{Café de la Paix}}}}{\lemma{\textnormal{\emph{nächſtliegende … Paix}}}\Cendnote{\textnormal{nächstliegend hier im Sinne von: in der Nähe liegend; es gab nur \textcolor{pink}{Café de la Paix}}}}\label{K_L02675-445h} hinein, um mir meine Freude von der Seele zu ſchreiben. Wie froh ich bin,
               Unrecht gehabt zu haben! Ich \label{K_L02675-1v}\edtext{beglückwünſche}{\lemma{\textnormal{\emph{beglückwünſche}}}\Cendnote{\textnormal{\textcolor{blue}{Goldmann} gratuliert \textcolor{blue}{Schnitzler} zur Annahme des \emph{\textcolor{green}{Märchen}}s
                  am \textcolor{pink}{Berlin}er \emph{\textcolor{brown}{Lessing-Theater}} (siehe Oscar Blumenthal an Arthur Schnitzler, 15. 12. 1891).
                  Zu dieser Inszenierung kam es nicht.}}}\label{K_L02675-1h} Dich innig und von ganzem Herzen, und
               ich rufe aller guten Engel Beiſtand auf Dich herab, auf daß das große \textcolor{green}{Werk}{}\ledrightnote{→\textcolor{green}{Das Märchen. Schauspiel in drei Aufzügen}} gelinge. Iſt der Wind Dir nur ein
               wenig günſtig, ſo biſt Du von heut auf morgen ein in ganz \textcolor{pink}{Deutſchland}{}\ledrightnote{\textcolor{pink}{Deutschland}} bekannter Mann. Wie eitel ich darauf bin, daß ich
               ſo feſt an Dich geglaubt. Nun aber folge mir ein wenig, mein lieber Junge
               (entſchuldige, es iſt nicht wegen der Jugend, ſondern {\pb}wegen der Herzlichkeit) und ſei nicht bockbeinig und
               mache die Änderungen, die erfahrene \textcolor{blue}{Theaterpraktike\textcolor{gray}{r}}{}\ledrightnote{→\textcolor{blue}{Oskar Blumenthal}} von Dir verlangen, ſo roh ſie Dir auch erſcheinen mögen. Das Geheimniß des
               Erfolges liegt nicht am Wenigſten in der Kunſt, Conceſſionen zu machen. Vor allem muß
               der dritte \textcolor{green}{Akt}{}\ledrightnote{→\textcolor{green}{Das Märchen. Schauspiel in drei Aufzügen}} umgearbeitet
               werden – muß, glaube mir! Wenn Du die lauten Exploſionen verabſcheuſt – gut! Aber conciſer\substVorne{}\textsuperscript{{ }und}\substDazwischen{},\substHinten{} compacter, kräftiger anſteigend und einheitlicher muß die Sache werden. Eine
               Kleinigkeit: mach’ \textsc{\textcolor{green}{Moritzki}{}\ledrightnote{→\textcolor{green}{Das Märchen. Schauspiel in drei Aufzügen}}} etwas komiſcher! {\pb}So iſt er zu trocken und ledern. Der polniſche
               Accent allein genügt nicht; es muß auch in den Worten etwas ſein. Ich bitte Dich,
               mich über die Änderungen \label{K_L02675-2v}\edtext{\textsc{\begin{otherlanguage}{french}au courant\end{otherlanguage}}}{\lemma{\textnormal{\emph{au courant}}}\Cendnote{\textnormal{französisch: auf dem Laufenden}}}\label{K_L02675-2h} zu
               erhalten. Vielleicht daß ich doch etwas noch dazu bemerken kann! Und nochmals: von
               ganzem Herzen Glückauf! Das Leben iſt doch manchmal auch gut, und das war eine
               freudige Überraſchung heut{ }Abend{\dotsfour}\pend
           \pstart
           Vielen Dank für die lieben Empfehlungen!\pend
           \pstart
           Grüß’ Dich Gott! {\\[\baselineskip]}Dein {\\[\baselineskip]}\spacefill\mbox{Paul Goldmann}\pend
           \leftskip=0em{}\pstart
           \noindent{}\label{K_L02675-3v}\edtext{\uline{verte}}{\lemma{\textnormal{\emph{verte}}}\Cendnote{\textnormal{lateinisch: umblättern, wenden}}}\label{K_L02675-3h}!\pend
           \pstart
           {\pb}Darf ich \textsc{\textcolor{blue}{Herzl}{}\ledrightnote{\textcolor{blue}{Theodor Herzl}}} dein \textcolor{green}{Stück}{}\ledrightnote{\textcolor{green}{Das Märchen. Schauspiel in drei Aufzügen}} geben?\pend
           \pstart
           Dabei fällt mir ein, daß dieſer Erfolg in nächſter Saiſon mich einen Freund koſten
                  wird. \strikeout{T} Du wirſt wohlwollend gegen mich werden.
                     \label{K_L02675-4v}\edtext{\textsc{\begin{otherlanguage}{french}Enfin, c’est la vie ça\end{otherlanguage}}}{\lemma{\textnormal{\emph{Enfin, c’est la vie ça}}}\Cendnote{\textnormal{französisch: nun, so ist das
                     Leben}}}\label{K_L02675-4h}!\pend
           \endnumbering\briefempfaengerindex{Schnitzler, Arthur@\textsc{Schnitzler, Arthur}!zzzGoldmann, Paul@\emph{von Paul Goldmann}!1891-12-181@{18. 12. {[}1891{]}}|)be}\mylabel{h}  \normalsize

\doendnotes{C}
\bigskip
\vfill

\clearpage

\footnotesize

\lohead{\textsc{register}}

% Definiere theindex-Environment komplett neu ohne reledmac
\makeatletter
\renewenvironment{theindex}{%
  \section*{\indexname}%
  \setlength{\parindent}{0pt}%
  \setlength{\parskip}{0pt plus 0.3pt}%
  \let\item\@idxitem
}{%
  \clearpage
}
\makeatother

\IfFileExists{\jobname-pw.ind}{\input{\jobname-pw.ind}}{}

\end{document}

      