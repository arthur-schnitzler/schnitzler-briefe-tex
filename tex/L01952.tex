%% latex-korrekturansicht-vorspann.tex
%% Vorspann für die Korrekturansicht.
%% Lädt die gemeinsame Datei latex-vorspann.tex mit gesetztem Schalter.

\newif\ifkorrekturansicht
\korrekturansichttrue

\input{../tex-inputs/latex-vorspann}


               \section[Arthur Schnitzler an Hugo von Hofmannsthal, 30. 7. 1910]{ Arthur Schnitzler an Hugo von Hofmannsthal, 30. 7. 1910}\nopagebreak\mylabel{v}\rehead{ }\normalsize\beginnumbering\briefempfaengerindex{Hofmannsthal, Hugo von@\textsc{Hofmannsthal, Hugo von}!zzzSchnitzler, Arthur@\emph{von Arthur Schnitzler}!1910-07-301@{30. 7. 1910}|(be} \toendnotes[C]{\smallbreak\pagebreak[2]} \Standort{FDH, Hs-30885,138.}
\physDesc{Brief, 1 Blatt, 4 Seiten
\newline{}Handschrift: schwarze Tinte, deutsche Kurrent}\buchAbdrucke{\weitereDrucke{Hugo von Hofmannsthal, Arthur Schnitzler: \emph{Briefwechsel}. Hg. Therese Nickl und Heinrich Schnitzler. Frankfurt am Main: \emph{S. Fischer} 1964, S. 252.} }\toendnotes[C]{\smallbreak}\pstart
           \noindent{}{\pb}\textcolor{gray}{\textbf{Dr. Arthur Schnitzler}}\hfill \textcolor{pink}{XVIII. \textsc{Sternwartestr}. 71.}{}\ledrightnote{\textcolor{pink}{Sternwartestraße}}\pend
           \pstart
           \textcolor{gray}{\textbf{\strikeout{\textcolor{pink}{Wien XVIII. Spoettelgasse 7.}{}\ledrightnote{\textcolor{pink}{Edmund-Weiß-Gasse}}}}}\hfill 30. 7. 1910!\pend
           \pstart
           mein lieber Hugo, Sie ſehen: wir ſind ſchon \label{K_L01952_1v}\edtext{überſiedelt}{\lemma{\textnormal{\emph{überſiedelt}}}\Cendnote{\textnormal{siehe A. S.: \emph{Tagebuch}, 14. 7. 1910}}}\label{K_L01952_1h} – und das ſind auch ſchon wieder faſt drei Wochen her, natürlich gings recht
               allmälig, und auch jetzt ſind wir noch nicht in völliger Ordnung. Aber mein
               Arbeitszimmer iſt längſt ſo wohnlich, daſs es kaum einen rechten Grund gibt das
               Stückeſchreiben länger hinauszuſchieben. Übrigens war ich \label{K_L01952_2v}\edtext{zweimal fort}{\lemma{\textnormal{\emph{zweimal fort}}}\Cendnote{\textnormal{zuerst
                     vom 6. 7. 1910 bis zum 10. 7. 1910, dann vom 26. 7. 1910
                     bis zum 28. 7. 1910}}}\label{K_L01952_2h}, auf dem \textcolor{pink}{Se{\geminationm}ering}{}\ledrightnote{\textcolor{pink}{Semmering}}, mit \textcolor{blue}{Olga}{}\ledrightnote{\textcolor{blue}{Olga Schnitzler}} u \textcolor{blue}{Heini}{}\ledrightnote{\textcolor{blue}{Heinrich Schnitzler}}, knapp vor dem Umzug; und
               jetzt wieder ein paar Tage allein auf {\pb}dem \textcolor{pink}{Se{\geminationm}ering}{}\ledrightnote{\textcolor{pink}{Semmering}}, viel mit \textcolor{blue}{Brahm}{}\ledrightnote{\textcolor{blue}{Otto Brahm}} zuſammen; mit Frau \textcolor{blue}{\textsc{Jonas}}{}\ledrightnote{\textcolor{blue}{Clara Jonas}}, mit \textcolor{blue}{Kainz}{}\ledrightnote{\textcolor{blue}{Josef Kainz}} (der, we{\geminationn} alles gut geht, bald wieder eine neue Rolle von mir
               ſpielen dürfte.) Von \textcolor{pink}{Se{\geminationm}ering}{}\ledrightnote{\textcolor{pink}{Semmering}} aus hab ich eine \label{K_L01952_3v}\edtext{Fußpartie}{\lemma{\textnormal{\emph{Fußpartie}}}\Cendnote{\textnormal{siehe A. S.: \emph{Tagebuch}, 28. 7. 1910}}}\label{K_L01952_3h} gemacht (denken Sie, mein Rad hab ich – verſchenkt{\dotstwo}), über den \textcolor{pink}{So{\geminationn}wendſtein}{}\ledrightnote{\textcolor{pink}{Sonnwendstein}}, ins \textcolor{pink}{Otterthal}{}\ledrightnote{\textcolor{pink}{Otterthal}}, über \textcolor{pink}{Kirchberg}{}\ledrightnote{\textcolor{pink}{Kirchberg am Wechsel}}, \textcolor{pink}{Aspang}{}\ledrightnote{\textcolor{pink}{Aspang-Markt}} nach \textcolor{pink}{Mönichkirchen}{}\ledrightnote{\textcolor{pink}{Mönichkirchen}} – etwas ganz
               beſonders ſchönes, von \textcolor{pink}{oeſterreichiſcher}{}\ledrightnote{\textcolor{pink}{Österreich}}
               Unberühmtheit; ich hatte mich jahrelange geſehnt, es kennen zu lernen, ſo daſs es ein
               Witzwort unſres Hauſes, beſonders \textcolor{blue}{Heini}{}\ledrightnote{\textcolor{blue}{Heinrich Schnitzler}}s zu
               werden anfing; – und als {\pb}ich es endlich, nach etwa
               zehnſtündiger Wanderung erreichte, – gab es kein Bett im ganzen Ort, ſo daſs ich
               gleich wieder hinunter fahren mußte – (was in jüngern Jahren gewiſs ſymboliſch
               empfunden worden wäre.)\pend
           \pstart
           Ich hoffe wir reiſen heuer doch noch einmal weg, gegen Ende Auguſt, –
                  \textcolor{pink}{\textsc{St. Gilgen}}{}\ledrightnote{\textcolor{pink}{St. Gilgen}} vielleicht, oder \textcolor{pink}{Iſchl}{}\ledrightnote{\textcolor{pink}{Bad Ischl}}, aber kaum auf lang, da
               die \textcolor{green}{\textsc{Medardus}}{}\ledrightnote{\textcolor{green}{Der junge Medardus. Dramatische Historie in einem Vorspiel und fünf Aufzügen}} Proben ſehr früh beginnen dürften. \strikeout{\textcolor{gray}{Also}} Es wäre wirklich ſchön, wieder einmal ein paar So{\geminationm}ertage miteinander zu verleben; aber daſs man ſich in \textcolor{pink}{Wien}{}\ledrightnote{\textcolor{pink}{Wien}}{ }ſo ſelten, ja nahezu ſchon gar nicht ſieht, iſt
               wahrhaftig nicht {\pb}meine Schuld allein. Erſtens reiſen Sie
               viel zu viel – und we{\geminationn} Sie von \textcolor{pink}{Rodaun}{}\ledrightnote{\textcolor{pink}{Rodaun}} nach \textcolor{pink}{Wien}{}\ledrightnote{\textcolor{pink}{Wien}} ko{\geminationm}en, erfährt man es doch meiſtens nur ganz zufällig oder
               gar nicht. Entſchließen Sie ſich doch wieder öfter telegrafiſch oder ſonſtwie ſich
               anzuſagen oder anzufragen – da{\geminationn} ſollen Sie mich ke{\geminationn}en lernen! Eine hiſtoriſche Berichtigung: \textcolor{pink}{\textsc{Welsberg}}{}\ledrightnote{\textcolor{pink}{Welsberg-Taisten}} ist nicht \substVorne{}\textsuperscript{3}\substDazwischen{}4\substHinten{}, sondern 3 Jahre her – auch lang genug! Haben Sie meine \label{K_L01952_4v}\edtext{Karte aus \textcolor{pink}{Glion}{}\ledrightnote{\textcolor{pink}{Glion}}}{\lemma{\textnormal{\emph{Karte aus Glion}}}\Cendnote{\textnormal{vgl. A. S.: \emph{Tagebuch}, 28. 5. 1910}}}\label{K_L01952_4h} beko{\geminationm}en – was \label{K_L01952_5v}\edtext{12 Jahre her}{\lemma{\textnormal{\emph{12 Jahre her}}}\Cendnote{\textnormal{siehe A. S.: \emph{Tagebuch}, 14. 8. 1898}}}\label{K_L01952_5h} iſt! – Man ka{\geminationn} den Feuilletoniſten nicht Unrecht
               geben: die Zeit verrinnt{\dots}\pend
           \pstart
           Schönen Dank für die gemeinſame Karte mit \textcolor{blue}{Friedmanns}{}\ledrightnote{\textcolor{blue}{Rose Friedmann}{\newline}\textcolor{blue}{Louis Philipp Friedmann}}, u Grüße auch an dieſe ſowie \label{T_L01952_1v}\edtext{an Sie u \textcolor{blue}{Gerty}{}\ledrightnote{\textcolor{blue}{Gertrude von Hofmannsthal}}}{\lemma{\textnormal{\emph{an Sie u Gerty}}}\Cendnote{\textnormal{weiter quer am rechten Rand}}}\label{T_L01952_1h} von
               uns \textcolor{blue}{Beiden}{}\ledrightnote{→\textcolor{blue}{Olga Schnitzler}}. Herzlichſt Ihr
                  \spacefill\mbox{A.}\pend
           \endnumbering\briefempfaengerindex{Hofmannsthal, Hugo von@\textsc{Hofmannsthal, Hugo von}!zzzSchnitzler, Arthur@\emph{von Arthur Schnitzler}!1910-07-301@{30. 7. 1910}|)be}\mylabel{h}  \normalsize

\doendnotes{C}
\bigskip
\vfill

\clearpage

\footnotesize

\lohead{\textsc{register}}

% Definiere theindex-Environment komplett neu ohne reledmac
\makeatletter
\renewenvironment{theindex}{%
  \section*{\indexname}%
  \setlength{\parindent}{0pt}%
  \setlength{\parskip}{0pt plus 0.3pt}%
  \let\item\@idxitem
}{%
  \clearpage
}
\makeatother

\IfFileExists{\jobname-pw.ind}{\input{\jobname-pw.ind}}{}

\end{document}

      