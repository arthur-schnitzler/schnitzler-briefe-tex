%% latex-korrekturansicht-vorspann.tex
%% Vorspann für die Korrekturansicht.
%% Lädt die gemeinsame Datei latex-vorspann.tex mit gesetztem Schalter.

\newif\ifkorrekturansicht
\korrekturansichttrue

\input{../tex-inputs/latex-vorspann}


               \section[Hermann Bahr an Arthur Schnitzler, {[}20. 4. 1894{]}]{ Hermann Bahr an Arthur Schnitzler, {[}20. 4. 1894{]}}\nopagebreak\mylabel{v}\rehead{ }\normalsize\beginnumbering\briefempfaengerindex{Schnitzler, Arthur@\textsc{Schnitzler, Arthur}!zzzBahr, Hermann@\emph{von Hermann Bahr}!1894-04-201@{{[}20. 4. 1894{]}}|(be} \toendnotes[C]{\smallbreak\pagebreak[2]} \Standort{CUL, Schnitzler, B 5b.}
\physDesc{Brief, 1 Blatt, 2 Seiten
\newline{}Handschrift: schwarze Tinte, deutsche Kurrent
\newline{}Schnitzler: mit Bleistift datiert: »20/4 94« \newline{}Ordnung: 1) mit rotem Buntstift von unbekannter Hand nummeriert: »19« 2) mit Bleistift von unbekannter Hand nummeriert: »19«}\buchAbdrucke{\weitereDrucke{Hermann Bahr, Arthur Schnitzler: \emph{Briefwechsel, Aufzeichnungen, Dokumente (1891–1931)}. Hg. Kurt Ifkovits und Martin Anton Müller. Göttingen: \emph{Wallstein} 2018, S. 70.} }\pstart\center{}{\pb}Lieber Arthur!\pend\pstart
           \textcolor{blue}{Adele Sandrock}{}\ledrightnote{\textcolor{blue}{Adele Sandrock}} erzählte mir geſtern von einer
               für Sonntag geplanten Partie mit \textsc{Rendezvous} in \textcolor{pink}{Rodaun}{}\ledrightnote{\textcolor{pink}{Rodaun}}. Ich möchte ſehr gern mit und könnte vielleicht ſchon in der Früh mit
               Dir hinaus. Allerdings unter der Vorausſetzung, daß wir ganz unter uns ſind, dh. Du,
                  \textcolor{blue}{\textsc{Loris}}{}\ledrightnote{\textcolor{blue}{Hugo von Hofmannsthal}} und \textcolor{blue}{\textsc{Richard}}{}\ledrightnote{\textcolor{blue}{Richard Beer-Hofmann}}, wozu dann Nachmittags ſich noch {\pb}\textcolor{blue}{\textsc{Dilly}}{}\ledrightnote{\textcolor{blue}{Adele Sandrock}} und \substVorne{}\textsuperscript{der}\substDazwischen{}etwa\substHinten{} der \textcolor{blue}{\textsc{Baumgartl}}{}\ledrightnote{\textcolor{blue}{Theodor Baumgarten}} geſellen. Größere Horden ſind mir unſympathiſch; am liebſten wäre es mir zu
               viert; kommt außer den Genannten noch wer, ſo bitte, ſchreib mir das – dann gehe ich
               lieber ganz allein.\pend
           \pstart
           Herzlichst{\\[\baselineskip]}Dein{\\[\baselineskip]}\spacefill\mbox{HermannBahr}\pend
           \leftskip=0em{}\endnumbering\briefempfaengerindex{Schnitzler, Arthur@\textsc{Schnitzler, Arthur}!zzzBahr, Hermann@\emph{von Hermann Bahr}!1894-04-201@{{[}20. 4. 1894{]}}|)be}\mylabel{h}  \normalsize

\doendnotes{C}
\bigskip
\vfill

\clearpage

\footnotesize

\lohead{\textsc{register}}

% Definiere theindex-Environment komplett neu ohne reledmac
\makeatletter
\renewenvironment{theindex}{%
  \section*{\indexname}%
  \setlength{\parindent}{0pt}%
  \setlength{\parskip}{0pt plus 0.3pt}%
  \let\item\@idxitem
}{%
  \clearpage
}
\makeatother

\IfFileExists{\jobname-pw.ind}{\input{\jobname-pw.ind}}{}

\end{document}

      