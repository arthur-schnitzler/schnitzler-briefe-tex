%% latex-korrekturansicht-vorspann.tex
%% Vorspann für die Korrekturansicht.
%% Lädt die gemeinsame Datei latex-vorspann.tex mit gesetztem Schalter.

\newif\ifkorrekturansicht
\korrekturansichttrue

\input{../tex-inputs/latex-vorspann}


               \section[Arthur Schnitzler an Richard Beer-Hofmann, 14. 7. 1907]{ Arthur Schnitzler an Richard Beer-Hofmann, 14. 7. 1907}\nopagebreak\mylabel{v}\rehead{ }\normalsize\beginnumbering\briefempfaengerindex{Beer-Hofmann, Richard@\textsc{Beer-Hofmann, Richard}!zzzSchnitzler, Arthur@\emph{von Arthur Schnitzler}!1907-07-141@{14. 7. 1907}|(be} \toendnotes[C]{\smallbreak\pagebreak[2]} \Standort{YCGL, MSS 31.}
\physDesc{Brief, 1 Blatt (Briefpapier mit Trauerrand), 2 Seiten, Umschlag
\newline{}Handschrift: schwarze Tinte, deutsche Kurrent\newline{}Versand: Stempel: »\nobreak{}\oindex{Welsberg-Taisten@\textbf{Welsberg-Taisten}, \emph{Besiedelter Ort (A.BSO)}|pwk}{[}Wels{]}berg, \textcolor{gray}{15}. 7. 07\nobreak{}«.  
\newline{}Beer-Hofmann: mit blauem Buntstift das Datum der Beantwortung festgehalten:
                  »B 26/VII 07« }\buchAbdrucke{\weitereDrucke{Arthur Schnitzler, Richard Beer-Hofmann: \emph{Briefwechsel 1891–1931}. Hg. Konstanze Fliedl. Wien, Zürich: \emph{Europaverlag} 1992, S. 180.} }\toendnotes[C]{\smallbreak}\pstart{}{\pb}\textcolor{gray}{\textbf{Dr. Arthur Schnitzler}}\pend{}\pstart{}\textcolor{gray}{\textbf{\textcolor{pink}{Wien XVIII. Spoettelgasse 7}{}\ledrightnote{\textcolor{pink}{Edmund-Weiß-Gasse}}.}}\pend{}{\bigskip}\pstart{}{\pb}\textsc{Herrn Dr. Richard Beer-Hofmann}\pend{}\pstart{}\textcolor{pink}{Wien XVIII}{}\ledrightnote{\textcolor{pink}{XVIII., Währing}}\pend{}\pstart{}\textcolor{pink}{\textsc{Hasenauerstr.} 59}{}\ledrightnote{\textcolor{pink}{Hasenauerstraße}}.\pend{}{\bigskip}\pstart
           \raggedleft{}{\pb}\textcolor{pink}{\textsc{Welsberg-Waldbrunn}}{}\ledrightnote{\textcolor{pink}{Wildbad Waldbrunn}}, 14. 7. 907\pend
           \pstart{}mein lieber Richard,\pend\pstart
           eben leſe ich in der \textcolor{green}{Zeit}{}\ledrightnote{\textcolor{green}{Die Zeit}} die \label{K_L01691_1v}\edtext{\textcolor{green}{Anzeige}{}\ledrightnote{→\textcolor{green}{[Todesanzeige von Alois Hofmann]}}}{\lemma{\textnormal{\emph{Anzeige}}}\Cendnote{\textnormal{Die Anzeige erschien am
                     13. 7. 1907 (Jg. 6, Nr. 1723, Morgenblatt) auf
                  S. 12.}}}\label{K_L01691_1h} vom Tod Ihres \textcolor{blue}{Vaters}{}\ledrightnote{→\textcolor{blue}{Alois Hofmann}}. Gerade um die Stunde, da ich Ihnen dieſe Zeilen ſchreibe, wird er zu
               Grabe getragen. Im Herzen bin ich bei Ihnen und drücke Ihnen die Hand, ſo wie Sie
               wiſſen.\pend
           \pstart
           Sie haben meine Karten wohl erhalten. Hier in \textcolor{pink}{\textsc{Welsberg Waldbrunn}}{}\ledrightnote{\textcolor{pink}{Wildbad Waldbrunn}} denken wir möglichſt lange zu bleiben, bis Mitte, vielleicht {\pb}Ende Auguſt. \textcolor{blue}{Heini}{}\ledrightnote{\textcolor{blue}{Heinrich Schnitzler}}
               iſt mit uns. Später wollen wir, \textcolor{blue}{Olga}{}\ledrightnote{\textcolor{blue}{Olga Schnitzler}} u ich,
               ſüdlicher, \textcolor{pink}{Meran}{}\ledrightnote{\textcolor{pink}{Meran}} vielleicht. Ich hoffe ſehr, daſs
               der Sommer nicht zu Ende geht, ohne daſs wir einander in ſchöner Landſchaft begegnen.
               Laſſen Sie bald, ſehr bald von ſich hören, wär es auch nur ein paar Zeilen. Von \textcolor{blue}{Olga}{}\ledrightnote{\textcolor{blue}{Olga Schnitzler}} an Sie, \textcolor{blue}{Paula}{}\ledrightnote{\textcolor{blue}{Paula Beer-Hofmann}}, die \textcolor{blue}{Kinder}{}\ledrightnote{→\textcolor{blue}{Naëmah Beer-Hofmann}{\newline}→\textcolor{blue}{Gabriel Beer-Hofmann}{\newline}→\textcolor{blue}{Mirjam Beer-Hofmann}}, eben ſo wie von mir, alles herzliche, theilnehmende, gute.\pend
           \pstart
           Ihr{\\[\baselineskip]}\spacefill\mbox{Arthur.}\pend
           \leftskip=0em{}\endnumbering\briefempfaengerindex{Beer-Hofmann, Richard@\textsc{Beer-Hofmann, Richard}!zzzSchnitzler, Arthur@\emph{von Arthur Schnitzler}!1907-07-141@{14. 7. 1907}|)be}\mylabel{h}  \normalsize

\doendnotes{C}
\bigskip
\vfill

\clearpage

\footnotesize

\lohead{\textsc{register}}

% Definiere theindex-Environment komplett neu ohne reledmac
\makeatletter
\renewenvironment{theindex}{%
  \section*{\indexname}%
  \setlength{\parindent}{0pt}%
  \setlength{\parskip}{0pt plus 0.3pt}%
  \let\item\@idxitem
}{%
  \clearpage
}
\makeatother

\IfFileExists{\jobname-pw.ind}{\input{\jobname-pw.ind}}{}

\end{document}

      