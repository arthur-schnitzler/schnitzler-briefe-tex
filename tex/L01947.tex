%% latex-korrekturansicht-vorspann.tex
%% Vorspann für die Korrekturansicht.
%% Lädt die gemeinsame Datei latex-vorspann.tex mit gesetztem Schalter.

\newif\ifkorrekturansicht
\korrekturansichttrue

\input{../tex-inputs/latex-vorspann}


               \section[Hugo von Hofmannsthal an Arthur Schnitzler, 13. 7. {[}1910{]}]{ Hugo von Hofmannsthal an Arthur Schnitzler, 13. 7. {[}1910{]}}\nopagebreak\mylabel{v}\rehead{ }\normalsize\beginnumbering\briefempfaengerindex{Schnitzler, Arthur@\textsc{Schnitzler, Arthur}!zzzHofmannsthal, Hugo von@\emph{von Hugo von Hofmannsthal}!1910-07-131@{13. 7. {[}1910{]}}|(be} \toendnotes[C]{\smallbreak\pagebreak[2]} \Standort{CUL, Schnitzler, B 43.}
\physDesc{Brief, 2 Blätter (Das zweite Blatt mit »2« gekennzeichnet), 8 Seiten
\newline{}Handschrift: schwarze Tinte, deutsche Kurrent
\newline{}Schnitzler: mit Bleistift die Jahreszahl ergänzt: »910« und beschriftet: »\textsc{Hugo}« \newline{}Ordnung: 1) mit Bleistift von unbekannter Hand nummeriert: »\strikeout{313}« 2) mit Bleistift von unbekannter Hand nummeriert:
                                    »320«}\buchAbdrucke{\weitereDrucke{Hugo von Hofmannsthal, Arthur Schnitzler: \emph{Briefwechsel}. Hg. Therese Nickl und Heinrich Schnitzler. Frankfurt am Main: \emph{S. Fischer} 1964, S. 250.} }\toendnotes[C]{\smallbreak}\pstart
           \raggedleft{}{\pb}\textcolor{pink}{Rodaun}{}\ledrightnote{\textcolor{pink}{Rodaun}}{ }13 Juli.\pend
           \pstart{}Mein lieber Arthur, \pend\pstart
           neulich hatte ich einmal den Gedanken: man wohnt doch in der ſelben Stadt – ſo kann
               man doch \uline{ein Mal}, wenn man ſich wünſcht, den andern
               zu ſehen, auch Glück haben, ohne erſt einen Brief zu ſchreiben oder ein Telegramm zu
               ſchicken – und als ich dann bei Euch die Treppe heruntergehen {\pb}muſste, war ich unverhältnismäßig
               traurig. Freilich das einzelne iſt ja immer ein Zufall oder ein unbeträchtliches
               Detail, aber das Ganze macht mich wachſend traurig, ich kann mir nicht helfen. Man
               iſt ſeit 20 Jahren gut miteinander, man iſt ſich weder fremder, noch unintereſſanter,
               noch weniger lieb geworden, {\pb}ſondern im Gegentheil vielleicht, man gehört demſelben Berufe an, man wohnt in \uline{einer} Stadt – und man verbringt keine 20 Stunden im
               Jahr miteinander! Mir geht es furchtbar ab – Euch, Ihnen und \textcolor{blue}{Richard}{}\ledrightnote{\textcolor{blue}{Richard Beer-Hofmann}} offenbar viel weniger, das iſt ja Temperamentsſache. Am
                  \label{K_L01947_1v}\edtext{\textcolor{pink}{Lido}{}\ledrightnote{\textcolor{pink}{Lido}}}{\lemma{\textnormal{\emph{Lido}}}\Cendnote{\textnormal{Sie hielten sich von
                     12. 6. 1910 bis zum Ende des Monats im \textcolor{pink}{Grand Hotel Excelsior} auf.}}}\label{K_L01947_1h}{ }{\pb}hatte ich oft daran gedacht,
               hatte ſo ſicher gehofft, in dieſen drei Wochen Juli würde man ſich mehr
               als einmal ſehen, – es ſind Jahre her, daſs Sie nicht in meinem \textcolor{pink}{Haus}{}\ledrightnote{→\textcolor{pink}{Hofmannsthal-Schlössl}} waren! – und nun ko{\geminationm}t es \uline{ſo}. In dieſer
               Woche, wo wir noch hier ſind, \strikeout{trafen} überſiedelt Ihr,
               zu \label{K_L01947_2v}\edtext{Anfang der nächſten Woche}{\lemma{\textnormal{\emph{Anfang … Woche}}}\Cendnote{\textnormal{Am 21. 7. 1910 reisten sie
                  ab.}}}\label{K_L01947_2h} fahren wir mit {\pb}den
                  \textcolor{blue}{Friedmanns}{}\ledrightnote{\textcolor{blue}{Rose Friedmann}{\newline}\textcolor{blue}{Louis Philipp Friedmann}} fort, über \textcolor{pink}{München}{}\ledrightnote{\textcolor{pink}{München}} an den \textcolor{pink}{Bodenſee}{}\ledrightnote{\textcolor{pink}{Bodensee}} (eine
               Landschaft die ich nicht kenne und mir lange wünſche) dann über den \textcolor{pink}{Arlberg}{}\ledrightnote{\textcolor{pink}{Arlberg}} nach \textcolor{pink}{Tirol}{}\ledrightnote{\textcolor{pink}{Tirol}} hinein und
               ſind ungefähr die erſten 10 Tage des Auguſt in \textcolor{pink}{\textsc{Canazei}}{}\ledrightnote{\textcolor{pink}{Canazei}}. Dann ſind wir für viele Wochen {\pb}in \textcolor{pink}{Auſſee}{}\ledrightnote{\textcolor{pink}{Bad Aussee}}. Ko{\geminationm}t doch im September ein
               biſſl dorthin, da iſt gewöhnlich eine ſo ſchöne Zeit.\pend
           \pstart
           Wenn Ihr jemals wieder nach \textcolor{pink}{Tirol}{}\ledrightnote{\textcolor{pink}{Tirol}} geht, will ich
               alles tun, um für eine Zeit an den gleichen Ort zu ko{\geminationm}en; ich habe {\pb}eine ſo ſchöne
               liebe Erinnerung an die Tage in \textcolor{pink}{Welsberg}{}\ledrightnote{\textcolor{pink}{Welsberg-Taisten}} – das iſt
               aber auch schon wieder \label{K_L01947_3v}\edtext{4 Jahre}{\lemma{\textnormal{\emph{4 Jahre}}}\Cendnote{\textnormal{Wie \textcolor{blue}{Schnitzler} in seiner Antwort (siehe Arthur Schnitzler an Hugo von Hofmannsthal, 30. 7. 1910)
                  bemerkt, nur drei, im Juli 1907.}}}\label{K_L01947_3h} her.\pend
           \pstart
           Vielen Dank für Ihre ſo lieben Zeilen nach der \textcolor{green}{Cristina}{}\ledrightnote{\textcolor{green}{Cristinas Heimreise. Komödie}}.\pend
           \pstart
           Ein Wort über eine Arbeit von Ihnen (auch die Einſchränkungen, die ich mir ganz zu
               eigen machen kann) das iſt ſo {\pb}ganz dasſelbe was es vor 18 Jahren war, und ganz etwas anderes, als was von
               fremderem Mund ko{\geminationm}t.\pend
           \pstart
           Wie gerne hätte ich wieder ein neues Buch von Ihnen in der Hand. Wie gerne möchte ich
               Ihnen meine \textcolor{green}{Spieloper}{}\ledrightnote{→\textcolor{green}{Der Rosenkavalier}}
                  vorleſen.\hspace*{1.5em}Schicken Sie mir ein paar Zeilen nach
                  \textcolor{pink}{\textsc{Canazei, Südtirol}}{}\ledrightnote{\textcolor{pink}{Canazei}}, dann ſpäter.\pend
           \pstart
           Von Herzen Ihr{\\[\baselineskip]}\spacefill\mbox{Hugo.}\pend
           \leftskip=0em{}\pstart
           \noindent{}\label{T_L01947_1v}\edtext{Alles Gute \textcolor{blue}{Olga}{}\ledrightnote{\textcolor{blue}{Olga Schnitzler}} und den \textcolor{blue}{Kleinen}{}\ledrightnote{→\textcolor{blue}{Heinrich Schnitzler}{\newline}→\textcolor{blue}{Lilly Schnitzler}} von uns beiden.}{\lemma{\textnormal{\emph{Alles … beiden.}}}\Cendnote{\textnormal{quer am linken Rand der letzten Seite}}}\label{T_L01947_1h}\pend
           \endnumbering\briefempfaengerindex{Schnitzler, Arthur@\textsc{Schnitzler, Arthur}!zzzHofmannsthal, Hugo von@\emph{von Hugo von Hofmannsthal}!1910-07-131@{13. 7. {[}1910{]}}|)be}\mylabel{h}  \normalsize

\doendnotes{C}
\bigskip
\vfill

\clearpage

\footnotesize

\lohead{\textsc{register}}

% Definiere theindex-Environment komplett neu ohne reledmac
\makeatletter
\renewenvironment{theindex}{%
  \section*{\indexname}%
  \setlength{\parindent}{0pt}%
  \setlength{\parskip}{0pt plus 0.3pt}%
  \let\item\@idxitem
}{%
  \clearpage
}
\makeatother

\IfFileExists{\jobname-pw.ind}{\input{\jobname-pw.ind}}{}

\end{document}

      