%% latex-korrekturansicht-vorspann.tex
%% Vorspann für die Korrekturansicht.
%% Lädt die gemeinsame Datei latex-vorspann.tex mit gesetztem Schalter.

\newif\ifkorrekturansicht
\korrekturansichttrue

\input{../tex-inputs/latex-vorspann}


               \section[Arthur Schnitzler an Hugo Hofmannsthal, 21. 7. 1928]{ Arthur Schnitzler an Hugo Hofmannsthal, 21. 7. 1928}\nopagebreak\mylabel{v}\rehead{ }\normalsize\beginnumbering\briefempfaengerindex{Hofmannsthal, Hugo von@\textsc{Hofmannsthal, Hugo von}!zzzSchnitzler, Arthur@\emph{von Arthur Schnitzler}!1928-07-211@{21. 7. 1928}|(be} \toendnotes[C]{\smallbreak\pagebreak[2]} \Standort{FDH, Hs-30885,158.}
\physDesc{Brief, 2 Blätter, 3 Seiten
\newline{}Handschrift: schwarze Tinte, lateinische Kurrent\newline{}Ordnung: 1) mit Bleistift von unbekannter Hand beschriftet: »5 Tage
                                    vor \textcolor{blue}{Lili}’s
                                 Tod« 2) mit Bleistift von unbekannter Hand das zweite Blatt nummeriert mit »2.«}\buchAbdrucke{\weitereDrucke{1) Hugo von Hofmannsthal, Arthur Schnitzler: \emph{Briefwechsel}. Hg. Therese Nickl und Heinrich Schnitzler. Frankfurt am Main: \emph{S. Fischer} 1964, S. 310–311.} \weitereDrucke{2) Arthur Schnitzler: \emph{Briefe 1913–1931}. Hg. Peter Michael Braunwarth, Richard Miklin, Susanne Pertlik und Heinrich Schnitzler. Frankfurt am Main: \emph{S. Fischer} 1984, S. 557–559.} }\toendnotes[C]{\smallbreak}\pstart
           \raggedleft{}{\pb}\textcolor{pink}{Wien}{}\ledrightnote{\textcolor{pink}{Wien}}, 21. 7. 928\pend
           \pstart
           mein lieber Hugo, Sie werden schon von unserer \textcolor{blue}{Freundin-Hofrätin}{}\ledrightnote{→\textcolor{blue}{Berta Zuckerkandl}} gehört haben, wie sehr Ihr
               Brief über die \textcolor{green}{Therese}{}\ledrightnote{\textcolor{green}{Therese. Chronik eines Frauenlebens}} mich gefreut hat; – das Buch
               hat, sowohl beim Publikum, als bei den paar Menschen, auf die es \introOben{}mir\introOben{} besonders anko{\geminationm}t, mehr Erfolg als ich je
               hätte vermuthen dürfen. Die Entstehungsgeschichte ist einigermaßen merkwürdig, ich
               erzähle Ihnen einmal mehr davon.\pend
           \pstart
           – \textcolor{blue}{Christiane}{}\ledrightnote{\textcolor{blue}{Christiane von Hofmannsthal}} war mir immer außerordentlich
               sympathisch – ich glaube das klare, gerade, kluge wahrhaft verläßliche ihres Wesens
               seit jeher gespürt zu haben u bin froh, daſs der rechte \textcolor{blue}{Mann}{}\ledrightnote{→\textcolor{blue}{Heinrich Zimmer}} die rechte Wahl getroffen hat. Mögen Sie
               ihr bald das \textcolor{pink}{Heidelberg}{}\ledrightnote{\textcolor{pink}{Heidelberg}}er Häuschen bauen können.
               Meine \textcolor{blue}{Kinder}{}\ledrightnote{→\textcolor{blue}{Lili Schnitzler}{\newline}→\textcolor{blue}{Arnoldo Cappellini}} in \textcolor{pink}{Venedig}{}\ledrightnote{\textcolor{pink}{Venedig}} haben jetzt etliche Wohnungsschwierigkeiten
               durch einen kläglichen wahrhaft \textcolor{blue}{Goldoni}{}\ledrightnote{\textcolor{blue}{Carlo Goldoni}}schen \textcolor{blue}{Hausherrn}{}\ledrightnote{→\textcolor{blue}{Levi}} – (»nur halt daſs er
               leider lebt«.) – Im übrigen sind sie glücklich, und ich hab \textcolor{blue}{ihn}{}\ledrightnote{→\textcolor{blue}{Arnoldo Cappellini}} (von \textcolor{blue}{Lili}{}\ledrightnote{\textcolor{blue}{Lili Schnitzler}} gar nicht zu reden) sehr gern. Sie wissen, daſs wir drei im Frühjahr
               eine schöne Reise gemacht haben. \textcolor{pink}{Corfu}{}\ledrightnote{\textcolor{pink}{Korfu}}, \textcolor{pink}{Athen}{}\ledrightnote{\textcolor{pink}{Athen}}, \textcolor{pink}{Kon{\pb}stantinopel}{}\ledrightnote{\textcolor{pink}{Istanbul}}, \textcolor{pink}{Rhodus}{}\ledrightnote{\textcolor{pink}{Rhodos}}. Jetzt war \textcolor{blue}{Heini}{}\ledrightnote{\textcolor{blue}{Heinrich Schnitzler}} 10 Tage bei mir,
               und ich habe viel Freude von ihm gehabt.\pend
           \pstart
           Die So{\geminationm}ermonate wer\textcolor{gray}{d} ich wohl hier
               verbringen; ich sehe recht viel Menschen, insbesondere \textcolor{pink}{Amerika}{}\ledrightnote{\textcolor{pink}{Amerika}} findet sich in zahlreichen, oft verständnisvollen Exemplaren ein.
               Mit dem Arbeiten geht es ganz leidlich, aber Dilettant, der ich bin und bleibe, spiel
               ich mich mit Figuren und Stoffen mehr herum, – und eigentlich lieber, als daſs ich
               die Dictatur meines sogenannten Talentes oder wie wir es nennen wollen über sie
               ausübe. Immerhin wird gelegentlich schon wieder was herausko{\geminationm}en, und ans Geldverdienen muſs man ja leider immer
               ernstlicher und continuirlicher denken.\pend
           \pstart
           Die \textcolor{green}{aegyptische}{}\ledrightnote{→\textcolor{green}{Die ägyptische Helena}} hab ich natürlich
               schon gekannt; in der \label{K_L02504_1v}\edtext{\textcolor{pink}{Oper}{}\ledrightnote{\textcolor{pink}{Oper}}}{\lemma{\textnormal{\emph{Oper}}}\Cendnote{\textnormal{Die \textcolor{pink}{Wien}er Erstaufführung von \emph{\textcolor{green}{Die ägyptische
                     Helena}} fand am 11. 6. 1928 statt, \textcolor{blue}{Schnitzler} war aber erst am 23. 6. 1928 in der Vorstellung.}}}\label{K_L02504_1h} hab ich
               einen schönen Eindruck gehabt, und es war mir über alle Maßen interessant, Ihre
               Dichtung so für mich hin zu lesen – und daſs Musik mir immer mitklang, spricht für
               Dichter wie für Musiker. Es ist unglaublich, wie Ihre Sprache Möglichkeiten u
               Einfälle des Componisten oft vorauszuahnen scheint; es ist wahrhaftig Dichtung für
               Musik und aus Musik zugleich. Die beiden Akte sind mir {\pb}jeder für sich, einleuchtender, als in ihrem innern Zusa{\geminationm}enhang; das ganze Problem hat mich sehr bewegt, und ich
               denke, Sie hätten es noch tiefer erschöpft, we{\geminationn} Sie sich
               – ohne jeden Gedanken \substVorne{}\textsuperscript{an die}{\allowbreak}\substDazwischen{}an\substHinten{} un\textcolor{gray}{d} ohne jede Rücksicht auf Melodisirung \substVorne{}\textsuperscript{,}\substDazwischen{}un\textcolor{gray}{d}\substHinten{} auf Operisierung Ihrem dramatischen Ingenium hätten hingeben dürfen (wie ich
               derartiges in Ihren einleitenden \label{K_L02504_2v}\edtext{\textcolor{green}{Worten}{}\ledrightnote{→\textcolor{green}{»Die ägyptische Helena«}}}{\lemma{\textnormal{\emph{Worten}}}\Cendnote{\textnormal{\textcolor{blue}{Hugo von Hofmannsthal}: \emph{\textcolor{green}{»Die ägyptische Helena«}}. In: \emph{\textcolor{green}{Neue Freie Presse}}, Nr. 22832, 8. 4. 1928,
                  S. 31–33.}}}\label{K_L02504_2h}, schon in d \textcolor{brown}{N. Fr Pr.}{}\ledrightnote{\textcolor{brown}{Neue Freie Presse}}
               wunderbar angedeutet fand.). Nur mit den Liebestränken, besonders den
               Dosirungsmöglichkeiten konnt ich mich nicht befreunden; irgendwo in mir steckt doch
               ein Pedant und Rationalist und der Teufel soll mich holen, am Ende gar ein
               Recensent.\pend
           \pstart
           Nun mein lieber Hugo lassen Sie sich nochmals danken – und nach allen Richtungen
               bestes und gutes wünschen. Und wer weiſs vielleicht sieht man sich sogar wieder
               einmal.{\\[\baselineskip]}Ihr getreuer{\\[\baselineskip]}\spacefill\mbox{Arth}\pend
           \leftskip=0em{}\endnumbering\briefempfaengerindex{Hofmannsthal, Hugo von@\textsc{Hofmannsthal, Hugo von}!zzzSchnitzler, Arthur@\emph{von Arthur Schnitzler}!1928-07-211@{21. 7. 1928}|)be}\mylabel{h}  \normalsize

\doendnotes{C}
\bigskip
\vfill

\clearpage

\footnotesize

\lohead{\textsc{register}}

% Definiere theindex-Environment komplett neu ohne reledmac
\makeatletter
\renewenvironment{theindex}{%
  \section*{\indexname}%
  \setlength{\parindent}{0pt}%
  \setlength{\parskip}{0pt plus 0.3pt}%
  \let\item\@idxitem
}{%
  \clearpage
}
\makeatother

\IfFileExists{\jobname-pw.ind}{\input{\jobname-pw.ind}}{}

\end{document}

      