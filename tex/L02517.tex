%% latex-korrekturansicht-vorspann.tex
%% Vorspann für die Korrekturansicht.
%% Lädt die gemeinsame Datei latex-vorspann.tex mit gesetztem Schalter.

\newif\ifkorrekturansicht
\korrekturansichttrue

\input{../tex-inputs/latex-vorspann}


               \section[Christiane von Hofmannsthal an Arthur Schnitzler, 5. 8. 1929]{ Christiane von Hofmannsthal an Arthur Schnitzler,
                    5. 8. 1929}\nopagebreak\mylabel{v}\rehead{ }\normalsize\beginnumbering\briefempfaengerindex{Schnitzler, Arthur@\textsc{Schnitzler, Arthur}!zzzHofmannsthal, Christiane von@\emph{von Christiane von Hofmannsthal}!1929-08-051@{13. 8. 1929}|(be} \toendnotes[C]{\smallbreak\pagebreak[2]} \Standort{CUL, Schnitzler, B 43.}
\physDesc{Brief, 1 Blatt (Briefpapier mit Trauerrand), 1 Seite
\newline{}Schreibmaschine
\newline{}Handschrift: schwarze Tinte, lateinische Kurrent (\noindent{}Unterschrift, Adresse)
\newline{}Schnitzler: mit rotem Buntstift vier Unterstreichungen sowie die
                                            Beschriftung: »\textsc{Hofm}« und »\textsc{\uuline{Christiane}}« }\pstart
           \raggedleft{}{\pb}\textcolor{pink}{Bad Aussee}{}\ledrightnote{\textcolor{pink}{Bad Aussee}}, am 5. August 1929\pend
           \pstart
           \raggedleft{}{[}hs.:{]} \textcolor{pink}{Obertressen 6}{}\ledrightnote{\textcolor{pink}{Obertressen}}\pend
           \pstart{}{[}ms.:{]} Lieber Arthur,\pend\pstart
           \textcolor{blue}{Mama}{}\ledrightnote{\textcolor{blue}{Gertrude von Hofmannsthal}} veranlasst mich, Ihnen für Ihren lieben
                    Brief zu danken, da ihr schreiben noch schwer fällt.\pend
           \pstart
           Wir wären Ihnen für baldigste Uebersendung der Briefe sowohl an Sie als an \textcolor{blue}{Gustav Schwarzkopf}{}\ledrightnote{\textcolor{blue}{Gustav Schwarzkopf}} und wenn Sie können auch
                    der unveröffentlichten Gedichte an diesen, sehr dankbar, wir würden möglichst
                    schnell 2 Abschriften davon herstellen und Sie bekommen die Originale und eine
                    Copie wieder zurück. Es ist uns doch sehr
                    wichtig,
                    das ganze vorhandene Material überschauen zu können, bezüglich einer
                    Veröffentlichung würde natürlich nichts geschehen ohne Ihre ausdrückliche
                    Einwilligung.\pend
           \pstart
           Wir gehen sehr achtsam damit um.\pend
           \pstart
           Mit herzlichstem Dank und vielen Grüssen{\\[\baselineskip]}\spacefill\mbox{{[}hs.:{]} Christiane}\pend
           \leftskip=0em{}\endnumbering\briefempfaengerindex{Schnitzler, Arthur@\textsc{Schnitzler, Arthur}!zzzHofmannsthal, Christiane von@\emph{von Christiane von Hofmannsthal}!1929-08-051@{13. 8. 1929}|)be}\mylabel{h}  \normalsize

\doendnotes{C}
\bigskip
\vfill

\clearpage

\footnotesize

\lohead{\textsc{register}}

% Definiere theindex-Environment komplett neu ohne reledmac
\makeatletter
\renewenvironment{theindex}{%
  \section*{\indexname}%
  \setlength{\parindent}{0pt}%
  \setlength{\parskip}{0pt plus 0.3pt}%
  \let\item\@idxitem
}{%
  \clearpage
}
\makeatother

\IfFileExists{\jobname-pw.ind}{\input{\jobname-pw.ind}}{}

\end{document}

      