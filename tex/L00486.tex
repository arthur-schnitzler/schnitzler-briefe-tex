%% latex-korrekturansicht-vorspann.tex
%% Vorspann für die Korrekturansicht.
%% Lädt die gemeinsame Datei latex-vorspann.tex mit gesetztem Schalter.

\newif\ifkorrekturansicht
\korrekturansichttrue

\input{../tex-inputs/latex-vorspann}


               \section[Richard Beer-Hofmann an Arthur Schnitzler, 17. 9. 1895]{ Richard Beer-Hofmann an Arthur Schnitzler, 17. 9. 1895}\nopagebreak\mylabel{v}\rehead{ }\normalsize\beginnumbering\briefempfaengerindex{Schnitzler, Arthur@\textsc{Schnitzler, Arthur}!zzzBeer-Hofmann, Richard@\emph{von Richard Beer-Hofmann}!1895-09-171@{17. 9. 1895}|(be} \toendnotes[C]{\smallbreak\pagebreak[2]} \Standort{CUL, Schnitzler, B 8.}
\physDesc{Brief, 1 Blatt, 4 Seiten
\newline{}Handschrift: Bleistift, lateinische Kurrent
\newline{}Schnitzler: mit Bleistift nummeriert: »66« }\buchAbdrucke{\weitereDrucke{Arthur Schnitzler, Richard Beer-Hofmann: \emph{Briefwechsel 1891–1931}. Hg. Konstanze Fliedl. Wien, Zürich: \emph{Europaverlag} 1992, S. 82.} }\toendnotes[C]{\smallbreak}\pstart
           \raggedleft{}{\pb}\textcolor{pink}{Schönberg}{}\ledrightnote{\textcolor{pink}{Schönberg im Stubaital}}{ }17/IX 95{ }Abends\pend
           \pstart
           Lieber Arthur! \uline{Soeben} erhalte ich Ihren Brief. Ich bin wirklich in
               guter Sti{\geminationm}ung; hoffentlich merken Sie es an Manchem wenn
               ich nach \textcolor{pink}{Wien}{}\ledrightnote{\textcolor{pink}{Wien}} zurückko{\geminationm}e{[}.{]} Daß ich seit Sonntag{ }Früh allein bin wissen Sie wol. Wie das Alleinreisen von \textcolor{blue}{L.}{}\ledrightnote{\textcolor{blue}{Lou Andreas-Salomé}} aufgeno{\geminationm}en wurde?
               Zu schwierig in Worte zu kleiden. Nur vorläufig: Sie geht nicht nach \textcolor{pink}{Kopenhagen}{}\ledrightnote{\textcolor{pink}{Kopenhagen}} – sagt sie. Aber das ist nicht offiziell. \strikeout{Hier will ich bis Freitag}{ }Samstag{ }\introOben{}Früh\introOben{} will ich von {\pb}hier fort nach
                  \textcolor{pink}{Riva}{}\ledrightnote{\textcolor{pink}{Riva del Garda}}, – einen Tag dort bleiben und dann nach \textcolor{pink}{Salò}{}\ledrightnote{\textcolor{pink}{Salò}}, Südwestende des \textcolor{pink}{Gardasees}{}\ledrightnote{\textcolor{pink}{Lago di Garda}}. Vielleicht gefällt es mir aber dort nicht, dann vielleicht \textcolor{pink}{Verona}{}\ledrightnote{\textcolor{pink}{Verona}}, das ich nicht kenne. Jedenfalls erwarte ich
               noch einen Brief hieher, einen nach \textcolor{pink}{\uline{Riva}}{}\ledrightnote{\textcolor{pink}{Riva del Garda}}{ }\uline{Poste restante}.\pend
           \pstart
           \textcolor{blue}{Paul Horn}{}\ledrightnote{\textcolor{blue}{Paul Horn}} ist mir in der Erinnerung widerlich,
               Mann mit »lustigen Streichen« in der Jugend, kein Mensch.\pend
           \pstart
           {\pb}Wozu Brosamen wie »Alles erkundigt
               sich«? Wer verübelt uns übrigens daß wir nicht fort Litteratur reden?\pend
           \pstart
           Wie kommt \textcolor{blue}{Speidel}{}\ledrightnote{\textcolor{blue}{Ludwig Speidel}} zu \textcolor{blue}{Ebermann}{}\ledrightnote{\textcolor{blue}{Leo Ebermann}}? Momentan bin ich \uline{der},
               der einzige Gast im \textcolor{pink}{Wirtshaus}{}\ledrightnote{→\textcolor{pink}{Gasthaus Jagerhof}}.
               Ich »lebe u genieße«. Nochmals: Wann \uline{frühestens} kann
                  »\textcolor{green}{Liebelei}{}\ledrightnote{\textcolor{green}{Liebelei. Schauspiel in drei Akten}}« ko{\geminationm}en,
               denn vielleicht verzögert sich ja meine Ankunft, in den October
               hinein.\pend
           \pstart
           {\pb}Adieu, ich will noch vor der
               Dunkelheit ein wenig spazieren. Die Zirbelkiefer die an der Strasse steht, ko{\geminationm}t in \textcolor{blue}{Goethes}{}\ledrightnote{\textcolor{blue}{Johann Wolfgang von Goethe}}{ }\textcolor{green}{italienischer Reise}{}\ledrightnote{\textcolor{green}{Italienische Reise}} vor. (Reise über den \textcolor{pink}{Brenner}{}\ledrightnote{\textcolor{pink}{Brenner}}) »Bei \textcolor{pink}{Schemberg}{}\ledrightnote{\textcolor{pink}{Schönberg im Stubaital}}« etc. das weiß ich aus dem \label{K_L00486_1v}\edtext{\textcolor{green}{Meyer}{}\ledrightnote{→\textcolor{green}{Deutsche Alpen}}}{\lemma{\textnormal{\emph{Meyer}}}\Cendnote{\textnormal{»Dagegen gelangt man {[}\ldots{]} auf dem \emph{alten}, r. abgehenden (schlechten) Fahrweg, {[}\ldots{]} den sogen. \emph{Alten
                        Schönberg} (dessen Zirben schon \textcolor{blue}{Goethe} in seiner ›\textcolor{green}{Italienischer
                        Reise}‹ erwähnt; bei einer ›\textcolor{blue}{Goethe}bank‹ schöne Aussicht) hinan«. (\emph{\textcolor{green}{Meyers Reisebücher}}. \emph{\textcolor{green}{Deutsche Alpen. Erster Teil: Bayerisches Hochland, Allgäu,
                        Vorarlberg, Tirol, Brennerbahn, Ötztaler-, Stubaier-, und Ortlergruppe,
                        Bozen, Schlern und Rosengarten, Meran, Brenta- und Adamellogruppe;
                        Bergamasker Alpen, Gardasee.}} Fünfte Auflage. Mit 23 Karten, 4 Plänen
                     und 12 Panoramen. Leipzig, Wien: \emph{\textcolor{brown}{Bibliographisches
                        Institut}}{ }1896, S. 217.)}}}\label{K_L00486_1h}. Werden uns je Bäume irgendwo wachsen
               – bei \textcolor{green}{Meyer}{}\ledrightnote{\textcolor{green}{Meyers Reisebücher}}?\pend
           \pstart
           »Laßt uns lächeln.«\pend
           \pstart
           Herzlichst Ihr{\\[\baselineskip]}\spacefill\mbox{Richard}\pend
           \leftskip=0em{}\pstart
           \noindent{}Ich freu mich so sehr mit Ihren Briefen\pend
           \pstart
           »\label{K_L00486_2v}\edtext{schreiben Sie
                     augenscharf}{\lemma{\textnormal{\emph{schreiben Sie augenscharf}}}\Cendnote{\textnormal{offenbar ein
                     stehender Ausdruck der Gruppe, der sich auch im Briefwechsel zwischen \textcolor{blue}{Salten} und \textcolor{blue}{Hofmannsthal} nachweisen lässt.}}}\label{K_L00486_2h}«\pend
           \endnumbering\briefempfaengerindex{Schnitzler, Arthur@\textsc{Schnitzler, Arthur}!zzzBeer-Hofmann, Richard@\emph{von Richard Beer-Hofmann}!1895-09-171@{17. 9. 1895}|)be}\mylabel{h}  \normalsize

\doendnotes{C}
\bigskip
\vfill

\clearpage

\footnotesize

\lohead{\textsc{register}}

% Definiere theindex-Environment komplett neu ohne reledmac
\makeatletter
\renewenvironment{theindex}{%
  \section*{\indexname}%
  \setlength{\parindent}{0pt}%
  \setlength{\parskip}{0pt plus 0.3pt}%
  \let\item\@idxitem
}{%
  \clearpage
}
\makeatother

\IfFileExists{\jobname-pw.ind}{\input{\jobname-pw.ind}}{}

\end{document}

      