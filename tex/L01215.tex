%% latex-korrekturansicht-vorspann.tex
%% Vorspann für die Korrekturansicht.
%% Lädt die gemeinsame Datei latex-vorspann.tex mit gesetztem Schalter.

\newif\ifkorrekturansicht
\korrekturansichttrue

\input{../tex-inputs/latex-vorspann}


               \section[Arthur Schnitzler an Richard Beer-Hofmann, 1. 4. 1902]{ Arthur Schnitzler an Richard Beer-Hofmann, 1. 4. 1902}\nopagebreak\mylabel{v}\rehead{ }\normalsize\beginnumbering\briefempfaengerindex{Beer-Hofmann, Richard@\textsc{Beer-Hofmann, Richard}!zzzSchnitzler, Arthur@\emph{von Arthur Schnitzler}!1902-04-011@{1. 4. 1902}|(be} \toendnotes[C]{\smallbreak\pagebreak[2]} \Standort{YCGL, MSS 31.}
\physDesc{Briefkarte, Umschlag
\newline{}Handschrift: Bleistift, deutsche Kurrent\newline{}Versand: 1) Stempel: »\nobreak{}\oindex{I., Innere Stadt@\textbf{I., Innere Stadt}, \emph{Bezirk (A.BZK)}|pwk}Wien 1/1 1, 1. 4. 0\textcolor{gray}{2}, 11–12N\nobreak{}«.  2) Stempel: »\nobreak{}\oindex{Rodaun@\textbf{Rodaun}, \emph{Teil eines besiedelten Ortes (A.BSOX)}|pwk}{\pb}Rodaun, 1. 4. {[}02{]}, 7–9V\nobreak{}«. \newline{}Ordnung: mit Bleistift von unbekannter Hand datiert: »1. 4.« }\toendnotes[C]{\smallbreak}\pstart{}{\pb}Herrn \textsc{Dr. Rich.
                            Beer-Hofmann}\pend{}\pstart{}\textcolor{pink}{\textsc{Rodaun} bei Lieſing}{}\ledrightnote{\textcolor{pink}{Rodaun}}\pend{}\pstart{}\textcolor{pink}{Lieſinger Straße 2}{}\ledrightnote{\textcolor{pink}{Liesingerstraße}}.\pend{}{\bigskip}\pstart
           \noindent{}{\pb}lieber Richard, ich habe mich bei \textcolor{blue}{Schlichter}{}\ledrightnote{\textcolor{blue}{Felix Schlichter}} für \uline{\label{K_L01215_1v}\edtext{Samſtag{ }4}{\lemma{\textnormal{\emph{Samſtag 4}}}\Cendnote{\textnormal{vgl. A. S.: \emph{Tagebuch}, 12. 4. 1902}}}\label{K_L01215_1h}} angeſagt u Ihr wahrſcheinl. Kommen in Ausſicht geſtellt, neuen Impfstoff
                    beſtellt.\pend
           \pstart
           Leider ko{\geminationn}t ich heut nicht zu Ihnen, wir müſſen doch
                    endlich wieder {\pb}einige ungehetzte Stunden
                    miteinander verbringen find ich.\pend
           \pstart
           Ihr{\\[\baselineskip]}\spacefill\mbox{A.}\pend
           \leftskip=0em{}\endnumbering\briefempfaengerindex{Beer-Hofmann, Richard@\textsc{Beer-Hofmann, Richard}!zzzSchnitzler, Arthur@\emph{von Arthur Schnitzler}!1902-04-011@{1. 4. 1902}|)be}\mylabel{h}  \normalsize

\doendnotes{C}
\bigskip
\vfill

\clearpage

\footnotesize

\lohead{\textsc{register}}

% Definiere theindex-Environment komplett neu ohne reledmac
\makeatletter
\renewenvironment{theindex}{%
  \section*{\indexname}%
  \setlength{\parindent}{0pt}%
  \setlength{\parskip}{0pt plus 0.3pt}%
  \let\item\@idxitem
}{%
  \clearpage
}
\makeatother

\IfFileExists{\jobname-pw.ind}{\input{\jobname-pw.ind}}{}

\end{document}

      