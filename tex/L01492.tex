%% latex-korrekturansicht-vorspann.tex
%% Vorspann für die Korrekturansicht.
%% Lädt die gemeinsame Datei latex-vorspann.tex mit gesetztem Schalter.

\newif\ifkorrekturansicht
\korrekturansichttrue

\input{../tex-inputs/latex-vorspann}


               \section[Hermann Bahr an Arthur Schnitzler, 21. 1. 1905]{ Hermann Bahr an Arthur Schnitzler, 21. 1. 1905}\nopagebreak\mylabel{v}\rehead{ }\normalsize\beginnumbering\briefempfaengerindex{Schnitzler, Arthur@\textsc{Schnitzler, Arthur}!zzzBahr, Hermann@\emph{von Hermann Bahr}!1905-01-211@{21. 1. 1905}|(be} \toendnotes[C]{\smallbreak\pagebreak[2]} \Standort{CUL, Schnitzler, B 5b.}
\physDesc{Brief, 1 Blatt, 1 Seite
\newline{}Handschrift: schwarze Tinte, deutsche Kurrent\newline{}Ordnung: mit Bleistift von unbekannter Hand nummeriert: »126« }\buchAbdrucke{\weitereDrucke{Hermann Bahr, Arthur Schnitzler: \emph{Briefwechsel, Aufzeichnungen, Dokumente (1891–1931)}. Hg. Kurt Ifkovits und Martin Anton Müller. Göttingen: \emph{Wallstein} 2018, S. 339.} }\toendnotes[C]{\smallbreak}\pstart
           \raggedleft{}{\pb}21. 1. 05\pend
           \pstart\center{}Lieber Arthur!\pend\pstart
           Haſt Du irgend \label{K_L01492_1v}\edtext{etwas Kurzes, womöglich
                  unediert}{\lemma{\textnormal{\emph{etwas … unediert}}}\Cendnote{\textnormal{Nach \textcolor{blue}{Schnitzler}s Absage im Antwortschreiben las \textcolor{blue}{Bahr}{ }\emph{\textcolor{green}{Exzentric}} vor.}}}\label{K_L01492_1h} oder doch in \textcolor{pink}{Wien}{}\ledrightnote{\textcolor{pink}{Wien}} noch nicht geleſen, und womöglich luſtig, am liebſten in
               der Art von »\textcolor{green}{Exzentrik}{}\ledrightnote{\textcolor{green}{Excentric}}«, was Du mir zum Vorleſen in
               der \textcolor{blue}{Hervayvorleſung}{}\ledrightnote{\textcolor{blue}{Elvira Leontine Hervay von Kirchberg}}, für die ich eingefangen
               worden bin, geben könnteſt? Mir geſchähe damit ein großer Dienſt.\pend
           \pstart
           Ich höre, daß bei Euch die Influenza herumzieht, und will ſchon längſt immer kommen,
               hab aber einen rechten Wirrwarr in mir. Doch jetzt müſſen wir uns einmal wieder
               ſehen.\pend
           \pstart
           Mit vielen herzlichen Grüßen, auch{\\[\baselineskip]}an Deine \textcolor{blue}{Frau}{}\ledrightnote{→\textcolor{blue}{Olga Schnitzler}},{\\[\baselineskip]}Dein{\\[\baselineskip]}\spacefill\mbox{Hermann}\pend
           \leftskip=0em{}\endnumbering\briefempfaengerindex{Schnitzler, Arthur@\textsc{Schnitzler, Arthur}!zzzBahr, Hermann@\emph{von Hermann Bahr}!1905-01-211@{21. 1. 1905}|)be}\mylabel{h}  \normalsize

\doendnotes{C}
\bigskip
\vfill

\clearpage

\footnotesize

\lohead{\textsc{register}}

% Definiere theindex-Environment komplett neu ohne reledmac
\makeatletter
\renewenvironment{theindex}{%
  \section*{\indexname}%
  \setlength{\parindent}{0pt}%
  \setlength{\parskip}{0pt plus 0.3pt}%
  \let\item\@idxitem
}{%
  \clearpage
}
\makeatother

\IfFileExists{\jobname-pw.ind}{\input{\jobname-pw.ind}}{}

\end{document}

      