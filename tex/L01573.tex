%% latex-korrekturansicht-vorspann.tex
%% Vorspann für die Korrekturansicht.
%% Lädt die gemeinsame Datei latex-vorspann.tex mit gesetztem Schalter.

\newif\ifkorrekturansicht
\korrekturansichttrue

\input{../tex-inputs/latex-vorspann}


               \section[Arthur Schnitzler an Gerhart Hauptmann, {[}17. 1. 1906{]}]{ Arthur Schnitzler an Gerhart Hauptmann,
                    {[}17. 1. 1906{]}}\nopagebreak\mylabel{v}\rehead{ }\normalsize\beginnumbering\briefempfaengerindex{Hauptmann, Gerhart@\textsc{Hauptmann, Gerhart}!zzzSchnitzler, Arthur@\emph{von Arthur Schnitzler}!1906-01-171@{{[}17. 1. 1906{]}}|(be} \toendnotes[C]{\smallbreak\pagebreak[2]} \Standort{Staatsbibliothek Berlin – Preußischer Kulturbesitz, GHBrBl A:Schnitzler (1).}
\physDesc{Briefkarte
\newline{}Handschrift: schwarze Tinte, deutsche Kurrent\newline{}Ordnung: mit Bleistift von unbekannter Hand nummeriert: »1« }\toendnotes[C]{\smallbreak}\pstart
           \noindent{}{\pb}\textcolor{gray}{\textbf{Dr. Arthur
                            Schnitzler}}{\\}\textcolor{gray}{\textbf{\textcolor{pink}{Wien XVIII. Spoettelgasse 7}{}\ledrightnote{\textcolor{pink}{Edmund-Weiß-Gasse}}.}}\pend
           \pstart
           lieber Herr Hauptmann, der \textcolor{blue}{Überbringer}{}\ledrightnote{\textcolor{blue}{Siegfried Knapitsch}} dieſer \label{K_L01573_1v}\edtext{Karte}{\lemma{\textnormal{\emph{Karte}}}\Cendnote{\textnormal{Vgl. A. S.: \emph{Tagebuch}, 17. 1. 1906: \textcolor{blue}{Knapitsch} »behufs Unterzeichnung eines
                        Aufrufs zur Gründung einer deutschen ›société‹. – Ich gab ihm eine Karte an
                            \textcolor{blue}{Hauptmann}.«}}}\label{K_L01573_1h}, Herr \textsc{cand. jur. \textcolor{blue}{Siegfried
                            Knapitsch}{}\ledrightnote{\textcolor{blue}{Siegfried Knapitsch}}}, beabſichtigt, in Verbindung mit vielen andren
                    Bühnenautoren eine deutſche \strikeout{Bühnen} Geſellschaft
                    in der Art der »\textsc{\textcolor{brown}{societé des
                        auteurs dram}{}\ledrightnote{\textcolor{brown}{Société des Auteurs et Compositeurs Dramatiques}}. etc}{[}«{]} ins Leben zu rufen, eine {\pb}Idee, die gewiſs alle Förderung verdient.\pend
           \pstart
           Wollen Sie die große Güte haben, Herrn \textsc{\textcolor{blue}{Knapitsch}{}\ledrightnote{\textcolor{blue}{Siegfried Knapitsch}}} anzuhören?\pend
           \pstart
           Herzlich grüßend{\\[\baselineskip]}Ihr{\\[\baselineskip]}\spacefill\mbox{Arth Schnitzler}\pend
           \leftskip=0em{}\endnumbering\briefempfaengerindex{Hauptmann, Gerhart@\textsc{Hauptmann, Gerhart}!zzzSchnitzler, Arthur@\emph{von Arthur Schnitzler}!1906-01-171@{{[}17. 1. 1906{]}}|)be}\mylabel{h}  \normalsize

\doendnotes{C}
\bigskip
\vfill

\clearpage

\footnotesize

\lohead{\textsc{register}}

% Definiere theindex-Environment komplett neu ohne reledmac
\makeatletter
\renewenvironment{theindex}{%
  \section*{\indexname}%
  \setlength{\parindent}{0pt}%
  \setlength{\parskip}{0pt plus 0.3pt}%
  \let\item\@idxitem
}{%
  \clearpage
}
\makeatother

\IfFileExists{\jobname-pw.ind}{\input{\jobname-pw.ind}}{}

\end{document}

      