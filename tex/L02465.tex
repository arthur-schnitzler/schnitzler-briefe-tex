%% latex-korrekturansicht-vorspann.tex
%% Vorspann für die Korrekturansicht.
%% Lädt die gemeinsame Datei latex-vorspann.tex mit gesetztem Schalter.

\newif\ifkorrekturansicht
\korrekturansichttrue

\input{../tex-inputs/latex-vorspann}


               \section[Arthur Schnitzler an Georg Brandes, 3. 1. 1926]{ Arthur Schnitzler an Georg Brandes, 3. 1. 1926}\nopagebreak\mylabel{v}\rehead{ }\normalsize\beginnumbering\briefempfaengerindex{Brandes, Georg@\textsc{Brandes, Georg}!zzzSchnitzler, Arthur@\emph{von Arthur Schnitzler}!1926-01-031@{3. 1. 1926}|(be} \toendnotes[C]{\smallbreak\pagebreak[2]} \Standort{Kopenhagen, Det Kongelige Bibliotek, Georg Brandes Arkiv, box 125.}
\physDesc{Brief, 2 Blätter, 4 Seiten
\newline{}Handschrift: schwarze Tinte, lateinische Kurrent\newline{}Ordnung: mit Bleistift von unbekannter Hand nummeriert:
                                                »54.«, auf der ersten Seite
                                                »Schn«, auf dem zweiten Blatt das
                                            Datum ergänzt: »3/1 26« }\buchAbdrucke{\weitereDrucke{Georg Brandes, Arthur Schnitzler: \emph{Ein Briefwechsel}. Hg. Kurt Bergel. Bern: \emph{Francke} 1956, S. 151–152.} }\pstart
           \raggedleft{}{\pb}\textcolor{pink}{Wien}{}\ledrightnote{\textcolor{pink}{Wien}}, 3. 1. 926\pend
           \pstart{}Mein verehrter und lieber Freund,\pend\pstart
           Ihr Neujahrsgruſs hat mich beschämt. Was ko{\geminationn}te ich
                    in \textcolor{pink}{Wien}{}\ledrightnote{\textcolor{pink}{Wien}} besondres für Sie thun – die höchst
                    bescheidene Gastfreundschaft die ich Ihnen und Frau \textcolor{blue}{Rung}{}\ledrightnote{\textcolor{blue}{Gertrud Rung}} erweisen durfte, \substVorne{}\textsuperscript{von}\substDazwischen{}bedeutete\substHinten{} mir mindestens so viel Freude als Ihnen – und wie wenig war es in jedem
                    Fall im Verhältnis zu der tiefen Dankbarkeit und Liebe, die ich für Sie
                    empfinde. Und ich halte Sie immer zumindest geistig in meiner Nähe: kaum ein
                    Abend ist im Lauf des letzten oder der letzten Jahre vergangen, ohne daſs ich
                    ein paar, und öfters recht viele Seiten von Ihnen gelesen. Und in Ihnen ist eine
                    so wunder{\pb}bare Identität des Menschen und des
                    Schriftstellers, daſs man immer \uline{mit} Ihnen ist,
                        we{\geminationn} man Sie liest.\pend
           \pstart
           Die »\textcolor{green}{Frau des Richters}{}\ledrightnote{\textcolor{green}{Die Frau des Richters. Novelle}}« war zuerst in der \textcolor{brown}{Vossischen}{}\ledrightnote{\textcolor{brown}{Vossische Zeitung}} gedruckt – und eigentlich als
                    Einakter intendirt. Das Stück wollte mir nicht gelingen, so hab ich die Handlung
                    zu erzählen versucht. Mein Herz hängt \uline{nicht} an
                    dieser kleinen Geschichte. Viele Jahre aber hat mich ein fünfactiges Versdrama
                        »\textcolor{green}{Der Gang zum Weiher}{}\ledrightnote{\textcolor{green}{Der Gang zum Weiher. Dramatische Dichtung}}« begleitet, das schon
                    im Druck ist und das ich Ihnen hoffentlich bald schicken kann, ebenso wie eine
                    größere »\textcolor{green}{Traumnovelle}{}\ledrightnote{\textcolor{green}{Traumnovelle}}« (die eben in
                    Fortsetzungen in der \textcolor{brown}{Dame}{}\ledrightnote{\textcolor{brown}{Die Dame}}
                        erscheint –){[}.{]}{ }{\pb}Und ganz besonders viel beschäftigt mich –
                    auch seit Jahren schon – allerlei aphoristisch–fragmentistisches, – worunter
                    vielleicht zwei Diagramme »\textcolor{green}{Der Geist im Wort« und
                        »der Geist in der That}{}\ledrightnote{\textcolor{green}{Der Geist im Wort und der Geist in der Tat}}«, philosophische Spielereien nicht ohne tieferen
                    Sinn, Sie unterhalten werden.\pend
           \pstart
           Ich hoffe Sie sind so wohl und gesund als Ihre jungen und festen Schriftzüge
                    vermuthen lassen. Darf ich Sie bitten, der liebenswürdigen Frau \textcolor{blue}{Gertrud Rung}{}\ledrightnote{\textcolor{blue}{Gertrud Rung}} meine herzlichsten Gegengrüße zu bestellen?
                    Und an Sie, mein lieber und wahrhaft verehrter Freund, gehen meine innigsten
                    Wünsche {\pb}Tag für Tag.\pend
           \pstart
           Auf Wiedersehen.{\\[\baselineskip]}Ihr getreuer{\\[\baselineskip]}\spacefill\mbox{Arthur Schnitzler}\pend
           \leftskip=0em{}\endnumbering\briefempfaengerindex{Brandes, Georg@\textsc{Brandes, Georg}!zzzSchnitzler, Arthur@\emph{von Arthur Schnitzler}!1926-01-031@{3. 1. 1926}|)be}\mylabel{h}  \normalsize

\doendnotes{C}
\bigskip
\vfill

\clearpage

\footnotesize

\lohead{\textsc{register}}

% Definiere theindex-Environment komplett neu ohne reledmac
\makeatletter
\renewenvironment{theindex}{%
  \section*{\indexname}%
  \setlength{\parindent}{0pt}%
  \setlength{\parskip}{0pt plus 0.3pt}%
  \let\item\@idxitem
}{%
  \clearpage
}
\makeatother

\IfFileExists{\jobname-pw.ind}{\input{\jobname-pw.ind}}{}

\end{document}

      