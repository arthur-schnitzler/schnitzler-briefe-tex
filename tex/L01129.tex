%% latex-korrekturansicht-vorspann.tex
%% Vorspann für die Korrekturansicht.
%% Lädt die gemeinsame Datei latex-vorspann.tex mit gesetztem Schalter.

\newif\ifkorrekturansicht
\korrekturansichttrue

\input{../tex-inputs/latex-vorspann}


               \section[Georg Brandes an Arthur Schnitzler, 16. 6. 1901]{ Georg Brandes an Arthur Schnitzler, 16. 6. 1901}\nopagebreak\mylabel{v}\rehead{ }\normalsize\beginnumbering\briefempfaengerindex{Schnitzler, Arthur@\textsc{Schnitzler, Arthur}!zzzBrandes, Georg@\emph{von Georg Brandes}!1901-06-161@{16. 6. 1901}|(be} \toendnotes[C]{\smallbreak\pagebreak[2]} \Standort{CUL, Schnitzler, B 17.}
\physDesc{Brief, 1 Blatt, 3 Seiten
\newline{}Handschrift: blaue Tinte, lateinische Kurrent\newline{}Ordnung: mit Bleistift von unbekannter Hand nummeriert:
                                    »25« }\buchAbdrucke{\weitereDrucke{Georg Brandes, Arthur Schnitzler: \emph{Ein Briefwechsel}. Hg. Kurt Bergel. Bern: \emph{Francke} 1956, S. 88–89.} }\toendnotes[C]{\smallbreak}\pstart
           \raggedleft{}{\pb}\textcolor{pink}{Kopenhagen}{}\ledrightnote{\textcolor{pink}{Kopenhagen}}\hspace*{1.5em}16 Juni 1901\pend
           \pstart{}Verehrter Freund\pend\pstart
           Zwar ist \textcolor{green}{Krotkaja}{}\ledrightnote{\textcolor{green}{Die Sanfte}} ein Monolog – es gibt so viele
               Monologe, \textcolor{blue}{Flaubert}{}\ledrightnote{\textcolor{blue}{Gustave Flaubert}}s \textcolor{green}{St. Antoine}{}\ledrightnote{\textcolor{green}{Die Versuchung des heiligen Antonius}} ist auch ein Monolog – aber das kleine Buch hat gar keine
               Form-Aehnlichkeit mit der Ihrigen. \textcolor{green}{Les lauriers sont
                  coupés}{}\ledrightnote{\textcolor{green}{Les lauriers sont coupés}} las ich vor – 16 Jahren glaub ich, als die \label{K_L01129_1v}\edtext{Erzählung in \textcolor{brown}{la Révue
                  Indépendante}{}\ledrightnote{\textcolor{brown}{La revue indépendante}}}{\lemma{\textnormal{\emph{Erzählung … Indépendante}}}\Cendnote{\textnormal{von Mai bis August 1887 in
                  vier Teilen, Bd. 3, H. 7, Mai, H. 8, S. 289–316; H. 9,
                        Juni, S. 472–494; H. 10, Juli, S. 122–137; H. 11,
                        August, S. 221–244.}}}\label{K_L01129_1h} stand, und es machte mir
               einen starken und originellen Eindruck, aber das Einzelne hab ich vergessen.\pend
           \pstart
           Ich kam zwar durch \textcolor{pink}{Wien}{}\ledrightnote{\textcolor{pink}{Wien}}, blieb aber {\pb}dort nur zwei Stunden. Ich hatte
               eine Scheu, Sie wieder aufzusuchen. Ich finde mich selbst sehr oft für Fremde
               ermüdend, fuhr deshalb nur durch; ich war bewegt, unaufgelegt zum Sprechen.\pend
           \pstart
           Durch Ihre Güte erhielt ich \textcolor{green}{Renate Fuchs}{}\ledrightnote{\textcolor{green}{Die Geschichte der jungen Renate Fuchs}}; es ist
               ein starkes Buch, aber die Grundidee so willkürlich, das Nachtwandern der Heldin. Das
               Beste sind die Details, scheint mir, die vielen tiefen Reflexionen. Im Ganzen jedoch
                  \label{K_L01129_2v}\edtext{Kunst = Kunst, nicht Kunst =
                  Natur}{\lemma{\textnormal{\emph{Kunst = … Natur}}}\Cendnote{\textnormal{Anspielung auf \textcolor{blue}{Arno Holz}’ Formel: »Kunst = Natur – x« aus \emph{\textcolor{green}{Die Kunst. Ihr Wesen und ihre Gesetze}}. Berlin: \emph{\textcolor{brown}{Issleib}}{ }1891.}}}\label{K_L01129_2h}. Ist es nicht wahr? Aber der \textcolor{blue}{Mann}{}\ledrightnote{→\textcolor{blue}{Jakob Wassermann}} hat sehr viel Talent.\pend
           \pstart
           {\pb}Hier haben wir scheussliches
               Wetter, fast Winter. Mitte Juli gehe ich nach \textcolor{pink}{Karlsbad}{}\ledrightnote{\textcolor{pink}{Karlsbad}}, ich habe mit \textcolor{blue}{Georges
                  Clemenceau}{}\ledrightnote{\textcolor{blue}{Georges Clemenceau}} verabredet, ihn dort zu treffen.\pend
           \pstart
           Von ganzem Herzen\pend
           \pstart
           Ihr{\\[\baselineskip]}\spacefill\mbox{Georg Brandes}\pend
           \leftskip=0em{}\endnumbering\briefempfaengerindex{Schnitzler, Arthur@\textsc{Schnitzler, Arthur}!zzzBrandes, Georg@\emph{von Georg Brandes}!1901-06-161@{16. 6. 1901}|)be}\mylabel{h}  \normalsize

\doendnotes{C}
\bigskip
\vfill

\clearpage

\footnotesize

\lohead{\textsc{register}}

% Definiere theindex-Environment komplett neu ohne reledmac
\makeatletter
\renewenvironment{theindex}{%
  \section*{\indexname}%
  \setlength{\parindent}{0pt}%
  \setlength{\parskip}{0pt plus 0.3pt}%
  \let\item\@idxitem
}{%
  \clearpage
}
\makeatother

\IfFileExists{\jobname-pw.ind}{\input{\jobname-pw.ind}}{}

\end{document}

      