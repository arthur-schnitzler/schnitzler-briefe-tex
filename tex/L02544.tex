%% latex-korrekturansicht-vorspann.tex
%% Vorspann für die Korrekturansicht.
%% Lädt die gemeinsame Datei latex-vorspann.tex mit gesetztem Schalter.

\newif\ifkorrekturansicht
\korrekturansichttrue

\input{../tex-inputs/latex-vorspann}


               \section[Gerty von Hofmannsthal an Arthur Schnitzler, {[}5. 3. 1931{]}]{ Gerty von Hofmannsthal an Arthur Schnitzler, {[}5. 3. 1931{]}}\nopagebreak\mylabel{v}\rehead{ }\normalsize\beginnumbering\briefempfaengerindex{Schnitzler, Arthur@\textsc{Schnitzler, Arthur}!zzzHofmannsthal, Gertrude von@\emph{von Gertrude von Hofmannsthal}!1931-03-051@{{[}5. 3. 1931{]}}|(be} \toendnotes[C]{\smallbreak\pagebreak[2]} \Standort{CUL, Schnitzler, B 43.}
\physDesc{Brief, 1 Blatt, 1 Seite
\newline{}Schreibmaschine
\newline{}Handschrift: Bleistift (\noindent{}Unterschrift)
\newline{}Schnitzler: mit rotem Buntstift beschriftet »\textsc{Hugo}«, das Datum ergänzt: »\substVorne{}\textsuperscript{5}\substDazwischen{}4\substHinten{}/3 931« und eine Unterstreichung vorgenommen \newline{}Ordnung: von unbekannter Hand nummeriert:
                                        »650« }\toendnotes[C]{\smallbreak}\pstart
           \raggedleft{}{\pb}\label{K_L02544_1v}\edtext{Donnerstag}{\lemma{\textnormal{\emph{Donnerstag}}}\Cendnote{\textnormal{Der 5. 3. 1931 war
                                ein Donnerstag, \textcolor{blue}{Schnitzler} macht also eine falsche Korrektur
                                an der Datumsangabe.}}}\label{K_L02544_1h}\pend
           \pstart{}Lieber Arthur,\pend\pstart
           Hier zwei Verträge die insofern differieren, als
                    die Summe von \textcolor{blue}{Strauss}{}\ledrightnote{\textcolor{blue}{Richard Strauss}} an \textcolor{blue}{Hugo}{}\ledrightnote{\textcolor{blue}{Hugo von Hofmannsthal}} bei Abschluss des Vertrages (die ich übrigens vergass zu erwähnen) in dem früheren
                    Vertrag in Mark ausgedrückt war und dann in Dollar, ferner jetzt nur mehr 20{\%} vom Ladenpreis des Buches statt wie im Anfang 25{\%} gezahlt werden. Sonst sehe ich nichts wesentlich
                    verschiedenes. Diese Vorauszahlung d. h. einmalige Zahlung von 3500 Dollar
                    vermindert etwas die Ungerechtigkeit dass nur \textcolor{blue}{Strauss}{}\ledrightnote{\textcolor{blue}{Richard Strauss}} von \textcolor{brown}{Fürstner}{}\ledrightnote{\textcolor{brown}{Musikverlag Adolph Fürstner}}{ }so hohe Bezahlung bei Ablieferung der Oper
                    kriegt. Aber dass die Oper blos 7{\%} abrechnet ist schon
                    irgend eine komische Sache, denn tatsächlich rechnen sie schon mehr ab nur nimmt
                    sich \textcolor{brown}{Fürstner}{}\ledrightnote{\textcolor{brown}{Musikverlag Adolph Fürstner}} wegen der hohen Zahlung an \textcolor{blue}{Strauss}{}\ledrightnote{\textcolor{blue}{Richard Strauss}} die Differenz, was bei gut gehenden
                    Opern doch zu hohen Gewinn für ihn ist. Verstehen tu ich die Sache nicht recht,
                    weiss aber dass \textcolor{blue}{Hugo}{}\ledrightnote{\textcolor{blue}{Hugo von Hofmannsthal}} mit \textcolor{blue}{Schalk}{}\ledrightnote{\textcolor{blue}{Franz Schalk}} darüber sprach, auch mit \textcolor{blue}{Strauss}{}\ledrightnote{\textcolor{blue}{Richard Strauss}} darüber Aussprachen hatte, die aber zu nichts
                    führten.\pend
           \pstart
           Bitte telephonieren Sie mich einmal an, womöglich doch lieber einen Tag früher
                    und kommen einen Sprung heraus, auch vormittag wies Ihnen passt.\pend
           \pstart
           Wenn es vor dem 11ten{ }sein könnte wärs mir sehr lieb weil ich dann
                    immer unsicher bin ob nicht der \textcolor{blue}{Raimund}{}\ledrightnote{\textcolor{blue}{Raimund von Hofmannsthal}}
                    gerade ankommt mit dem ich dann hier in aller Eile vieles Geschäftliche zu tun
                    habe.\pend
           \pstart
           Von Herzen Ihre{\\[\baselineskip]}\spacefill\mbox{{[}hs.:{]} Gerty}\pend
           \leftskip=0em{}\endnumbering\briefempfaengerindex{Schnitzler, Arthur@\textsc{Schnitzler, Arthur}!zzzHofmannsthal, Gertrude von@\emph{von Gertrude von Hofmannsthal}!1931-03-051@{{[}5. 3. 1931{]}}|)be}\mylabel{h}  \normalsize

\doendnotes{C}
\bigskip
\vfill

\clearpage

\footnotesize

\lohead{\textsc{register}}

% Definiere theindex-Environment komplett neu ohne reledmac
\makeatletter
\renewenvironment{theindex}{%
  \section*{\indexname}%
  \setlength{\parindent}{0pt}%
  \setlength{\parskip}{0pt plus 0.3pt}%
  \let\item\@idxitem
}{%
  \clearpage
}
\makeatother

\IfFileExists{\jobname-pw.ind}{\input{\jobname-pw.ind}}{}

\end{document}

      