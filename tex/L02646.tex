%% latex-korrekturansicht-vorspann.tex
%% Vorspann für die Korrekturansicht.
%% Lädt die gemeinsame Datei latex-vorspann.tex mit gesetztem Schalter.

\newif\ifkorrekturansicht
\korrekturansichttrue

\input{../tex-inputs/latex-vorspann}


               \section[Paul Goldmann an Arthur Schnitzler, 6. 12. 1889]{ Paul Goldmann an Arthur Schnitzler, 6. 12. 1889}\nopagebreak\mylabel{v}\rehead{ }\normalsize\beginnumbering\briefempfaengerindex{Schnitzler, Arthur@\textsc{Schnitzler, Arthur}!zzzGoldmann, Paul@\emph{von Paul Goldmann}!1889-12-061@{6. 12. 1889}|(be} \toendnotes[C]{\smallbreak\pagebreak[2]} \Standort{DLA, A:Schnitzler, HS.NZ85.1.3162.}
\physDesc{Brief, 1 Blatt, 4 Seiten
\newline{}Handschrift: blaue Tinte, deutsche Kurrent}\toendnotes[C]{\smallbreak}\pstart
           \noindent{}\centering{}{\pb}\textcolor{gray}{\textbf{\textbf{Adminiſtration: \textcolor{pink}{VII.
                           Seidengaſſe 7}{}\ledrightnote{\textcolor{pink}{Seidengasse}}} (\textcolor{brown}{Jos. Eberle {\kaufmannsund} Co.}{}\ledrightnote{\textcolor{brown}{Josef Eberle  Stein-, Buch und Musikaliendruckerei}})}}\pend
           \pstart
           \noindent{}\centering{}\textcolor{gray}{\textbf{\textcolor{brown}{An der Schönen Blauen Donau}{}\ledrightnote{\textcolor{brown}{An der schönen blauen Donau}}}}\pend
           \pstart
           \noindent{}\centering{}\textcolor{gray}{\textbf{Chef-Redacteur: Dr. \textcolor{blue}{F.
                        Mamroth}{}\ledrightnote{\textcolor{blue}{Fedor Mamroth}}. – Redaction: \textcolor{pink}{IX.,
                        Berggaſſe 31}{}\ledrightnote{\textcolor{pink}{Berggasse}}.}}\pend
           \pstart
           \raggedleft{}\textcolor{gray}{\textbf{\textcolor{pink}{Wien}{}\ledrightnote{\textcolor{pink}{Wien}}, den}}{ }6. December \textcolor{gray}{\textbf{18}}89.\pend
           \pstart\center{}Lieber Freund!\pend\pstart
           Sie haben Recht, es iſt ein fatales Zuſammentreffen geweſen. Aber – ich habe mir die
               Sache reiflich überlegt – es trifft mich nicht ſoviel Schuld, als Sie meinen.
               Zunächſt habe ich ja des Geſpräch nicht geſucht; zweitens iſt dasſelbe nicht, wie Ihr
                  \textcolor{blue}{Gewährsmann}{}\ledrightnote{→\textcolor{blue}{?? [Mann, der Gespräch über Schnitzler in der Straßenbahn belauscht, Ende November 1889]}} angibt, »laut
               und lebhaft« geführt worden; überdies hatte ich von der Anweſenheit eines Dritten
               natürlich keine Ahnung; Sachen, die Sie irgendwie kompromittiren könnten, ſind
               ſelbſtverſtändlich nicht geſprochen worden; es iſt eben nur Ihr Name genannt worden,
               da es ja unmöglich iſt, die Nennung des Namens von demjenigen zu umgehen, über den
               man ſpricht. Soweit kann man in ſeiner Vorſicht unmöglich gehen, daß man von
               Perſonen, von denen man ganz {\pb}allgemein und unverfänglich ſpricht, nur die Anfangs-Buchſtaben nennt; überdies
               bitte ich Sie, ſich zu überlegen, wie beleidigend ein ſolches Verfahren der
               betreffenden \textcolor{blue}{Dame}{}\ledrightnote{→\textcolor{blue}{?? [Frau, die mit Goldmann in der Straßenbahn spricht, Ende November 1889]}} gegenüber
               iſt, mit der man ſpricht, und wie lächerlich man ſich ſelbſt dadurch macht. Schuld
               trägt nur der Zufall, der es gefügt hat, daß ein Geſpräch zwiſchen der \textcolor{blue}{Betreffenden}{}\ledrightnote{→\textcolor{blue}{?? [Frau, die mit Goldmann in der Straßenbahn spricht, Ende November 1889]}} und mir
               überhaupt auf der Tramway geführt wurde. Und Schuld trägt ferner der \textcolor{blue}{Dritte}{}\ledrightnote{→\textcolor{blue}{?? [Mann, der Gespräch über Schnitzler in der Straßenbahn belauscht, Ende November 1889]}}, der indiskret genug
               war, auf ein nicht für ihn beſtimmtes Geſpräch zu hören, darüber einem Andren zu
               berichten und offenbar in einer Weise zu berichten, welche dasjenige, was an \strikeout{f} und für ſich nicht \introOben{}für Sie\introOben{}
               kompromittirend war, erſt dazu machte. An \uline{deſſen}
               Adreſſe alſo hätten Sie ſich, wie ich meine, mit Ihren Vorwürfen wenden müſſen, und
               nicht an die meinige.\pend
           \pstart
           Sie werden begreifen, daß Ihr Brief mich, der ich mich ſchuldlos fühle, ſehr
               verſtimmt hat. Ich begreife vollkommen, wie peinlich Ihnen jene Unterredung geweſen
               iſt; ich bedaure auch von ganzem Herzen, daß ich der unſchuldige Anlaß war, daß Ihnen
               ein Ärgerniß bereitet wurde. Aber ich finde es – ganz offen geſtanden – {\pb}nicht recht ſreundſchaftlich von
               Ihnen gehandelt, daß Sie mich ohneweiters für Alles verantwortlich machen und mich in
               einer etwas odioſen Form zur Rechenſchaft ziehen, odios vor allem deshalb, weil, wie
               Sie jedenfalls wiſſen, für einen Herrn mit etwas ausgebildeter Empfindlichkeit, es
               nichts Verletzenderes gibt, als eine Rüge und eine Belehrung, die mir beide in Ihrem
               Briefe ertheilt werden. Wäre ich an Ihrer Stelle geweſen, ſo glaube ich, daß ich
               nicht ſo vorgegangen wäre. Ich hätte entweder ganz darüber geſchwiegen, oder aber ich
               hätte die Sache in jenem gewiſſen Tone ſcherzhaften Vorwurfs zur Sprache gebracht und
               es dem Tacte des anderen Theiles überlaſſen, ſich das, was darin Rüge und Belehrung
               iſt, ſelbſt herauszufinden.\pend
           \pstart
           Daß Sie \strikeout{keines} keinen von dieſen beiden Wegen
               eingeſchlagen haben, verletzt mich ſehr. Es reſultirt daraus, wie geſagt, eine
               gewiſſe Verſtimmung gegen Sie. Und da es mir ſchwer fallen würde, dieſelbe zu
               verbergen, ſo bitte ich Sie, \strikeout{\textcolor{gray}{d}} mir zu geſtatten, daß ich für die nächſten Wochen von einem {\pb}\label{K_L02646-3v}\edtext{Zuſammenſein}{\lemma{\textnormal{\emph{Zuſammenſein}}}\Cendnote{\textnormal{Der Kontaktabbruch hielt nur bis zum 8. 12. 1889.}}}\label{K_L02646-3h} mit Ihnen abſehe. Es fällt
               mir freilich ſchwer, Ihre ſo lieb gewordene Geſellſchaft mir zu verſagen; aber Sie
               haben mich da in eine Zwangslage verſetzt, aus der ich keinen andren Ausweg ſehe, als
               dieſen.\pend
           \pstart
           Ich grüße Sie herzlichſt! {\\[\baselineskip]}Ihr {\\[\baselineskip]}\spacefill\mbox{Dr. Paul Goldmann.}\pend
           \leftskip=0em{}\endnumbering\briefempfaengerindex{Schnitzler, Arthur@\textsc{Schnitzler, Arthur}!zzzGoldmann, Paul@\emph{von Paul Goldmann}!1889-12-061@{6. 12. 1889}|)be}\mylabel{h}  \normalsize

\doendnotes{C}
\bigskip
\vfill

\clearpage

\footnotesize

\lohead{\textsc{register}}

% Definiere theindex-Environment komplett neu ohne reledmac
\makeatletter
\renewenvironment{theindex}{%
  \section*{\indexname}%
  \setlength{\parindent}{0pt}%
  \setlength{\parskip}{0pt plus 0.3pt}%
  \let\item\@idxitem
}{%
  \clearpage
}
\makeatother

\IfFileExists{\jobname-pw.ind}{\input{\jobname-pw.ind}}{}

\end{document}

      