%% latex-korrekturansicht-vorspann.tex
%% Vorspann für die Korrekturansicht.
%% Lädt die gemeinsame Datei latex-vorspann.tex mit gesetztem Schalter.

\newif\ifkorrekturansicht
\korrekturansichttrue

\input{../tex-inputs/latex-vorspann}


               \section[Hermann Bahr an Arthur Schnitzler, 22. 2. 1903]{ Hermann Bahr an Arthur Schnitzler, 22. 2. 1903}\nopagebreak\mylabel{v}\rehead{ }\normalsize\beginnumbering\briefempfaengerindex{Schnitzler, Arthur@\textsc{Schnitzler, Arthur}!zzzBahr, Hermann@\emph{von Hermann Bahr}!1903-02-221@{22. 2. 1903}|(be} \toendnotes[C]{\smallbreak\pagebreak[2]} \Standort{CUL, Schnitzler, B 5b.}
\physDesc{Kartenbrief
\newline{}Handschrift: schwarze Tinte, deutsche Kurrent\newline{}Versand: 1) Stempel: »\nobreak{}\oindex{XIII., Hietzing@\textbf{XIII., Hietzing}, \emph{Bezirk (A.BZK)}|pwk}Wien 13/7, 23. 2. 02, 10–11 V\nobreak{}«.  2) Stempel: »\nobreak{}24{[}.{]} 2. 03, 12 ¼ – 1½ N, Bestellt vom Postamte 9\nobreak{}«. 
\newline{}Schnitzler: mit Bleistift die Jahreszahl ergänzt: »903« \newline{}Ordnung: mit Bleistift von unbekannter Hand nummeriert:
                                    »93« }\buchAbdrucke{\weitereDrucke{Hermann Bahr, Arthur Schnitzler: \emph{Briefwechsel, Aufzeichnungen, Dokumente (1891–1931)}. Hg. Kurt Ifkovits und Martin Anton Müller. Göttingen: \emph{Wallstein} 2018, S. 248.} }\toendnotes[C]{\smallbreak}\pstart{}{\pb}Herrn \textsc{D\textsuperscript{r} Arthur Schnitzler}\pend{}\pstart{}\textcolor{pink}{\textsc{Berlin}}{}\ledrightnote{\textcolor{pink}{Berlin}}\pend{}\pstart{}\textcolor{pink}{\textsc{Palasthotel}}{}\ledrightnote{\textcolor{pink}{Palasthotel Berlin}}\pend{}{\bigskip}\pstart
           \raggedleft{}{\pb}22/2\pend
           \pstart{}Lieber Arthur!\pend\pstart
           Ich hätte Dir ſo viel zu ſagen, ſo viel zu danken – da ich wirklich das Gefühl habe,
               wenn Du mich nicht zu Deinem \textcolor{blue}{Bruder}{}\ledrightnote{→\textcolor{blue}{Julius Schnitzler}} geſchickt hätteſt, verloren geweſen zu ſein, und da mich auch Deine
               Theilnahme an meiner Krankheit ſehr gerührt hat – aber ich kanns nicht, da ich noch
               immer ſo hin und ſo grenzenlos müd bin, daß ich, wenn \introOben{}ich\introOben{}
               ein paar Zeilen kritzle, gleich ganz in Schweiß gebadet bin. Sonſt geht es mir, bis
               auf die leichte Bauchdeckeneiterung, die immer noch andauert, ganz gut. Aber ich
               erwarte immer noch die berühmte \label{K_L01272_1v}\edtext{\textcolor{green}{Stimmung der Geneſung}{}\ledrightnote{→\textcolor{green}{Genesung. Roman}}}{\lemma{\textnormal{\emph{Stimmung der Geneſung}}}\Cendnote{\textnormal{Anspielung auf \textcolor{blue}{Siegfried Trebitsch}s Roman \emph{\textcolor{green}{Genesung}}
                     (Berlin: \emph{\textcolor{brown}{S. Fischer}}{ }1902). Hermann Bahr an Arthur Schnitzler, 14. 12. 1904.}}}\label{K_L01272_1h}, die der Dichter \textcolor{blue}{Trebitſch}{}\ledrightnote{\textcolor{blue}{Siegfried Trebitsch}}{ }ſo ſchön geſchildert hat.\pend
           \pstart
           Mit Grüßen an \textcolor{blue}{Brahm}{}\ledrightnote{\textcolor{blue}{Otto Brahm}} u. alle
               Bekannten{\\[\baselineskip]}herzlichst Dein dankbarer \spacefill\mbox{Herma\damage{nn}}\pend
           \leftskip=0em{}\endnumbering\briefempfaengerindex{Schnitzler, Arthur@\textsc{Schnitzler, Arthur}!zzzBahr, Hermann@\emph{von Hermann Bahr}!1903-02-221@{22. 2. 1903}|)be}\mylabel{h}  \normalsize

\doendnotes{C}
\bigskip
\vfill

\clearpage

\footnotesize

\lohead{\textsc{register}}

% Definiere theindex-Environment komplett neu ohne reledmac
\makeatletter
\renewenvironment{theindex}{%
  \section*{\indexname}%
  \setlength{\parindent}{0pt}%
  \setlength{\parskip}{0pt plus 0.3pt}%
  \let\item\@idxitem
}{%
  \clearpage
}
\makeatother

\IfFileExists{\jobname-pw.ind}{\input{\jobname-pw.ind}}{}

\end{document}

      