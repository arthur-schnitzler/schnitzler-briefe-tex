%% latex-korrekturansicht-vorspann.tex
%% Vorspann für die Korrekturansicht.
%% Lädt die gemeinsame Datei latex-vorspann.tex mit gesetztem Schalter.

\newif\ifkorrekturansicht
\korrekturansichttrue

\input{../tex-inputs/latex-vorspann}


               \section[Max Burckhard: Widmungsexemplar Quer durch das Leben für Arthur Schnitzler, 14. 11. 1907]{ Max Burckhard: Widmungsexemplar Quer durch das Leben für Arthur Schnitzler,
                    14. 11. 1907}\nopagebreak\mylabel{v}\rehead{ }\normalsize\beginnumbering\briefempfaengerindex{Schnitzler, Arthur@\textsc{Schnitzler, Arthur}!zzzBurckhard, Max Eugen@\emph{von Max Eugen Burckhard}!1907-11-141@{14. 11. 1907}|(be} \toendnotes[C]{\smallbreak\pagebreak[2]} \Standort{DLA, G:Schnitzler, Arthur (Sammlung Heinrich Schnitzler).}
\physDesc{Widmung am Titelblatt
\newline{}Handschrift: schwarze Tinte, deutsche Kurrent\newline{}Ordnung: bei der Enteignung des Exemplars 1938 von unbekannter Hand mit Bleistift
                                    als Dublette markiert: »= 455782-B« }\pstart
           \noindent{}{\pb}Arthur Schnitzler herzlich in treuer
                    Verehrung\pend
           \pstart \spacefill\mbox{D\textsuperscript{r} Burckhard}\pend{}\pstart
           \textcolor{pink}{St. Gilgen}{}\ledrightnote{\textcolor{pink}{St. Gilgen}}{ }14/11 07\pend
           {\bigskip}\pstart
           \noindent{}\centering{}\textcolor{gray}{\textbf{\textcolor{green}{QUER DURCH DAS LEBEN}{}\ledrightnote{\textcolor{green}{Quer durch das Leben. Fünfzig Aufsätze}}.}}\pend
           \pstart
           \noindent{}\centering{}\textcolor{gray}{\textbf{FÜNFZIG
                    AUFSÄTZE}}\pend
           \pstart
           \noindent{}\centering{}\textcolor{gray}{\textbf{VON}}{\\}\textcolor{gray}{\textbf{MAX BURCKHARD}}.\pend
           {\bigskip}\pstart
           \noindent{}\textcolor{gray}{\textbf{\textcolor{pink}{WIEN}{}\ledrightnote{\textcolor{pink}{Wien}}.}}\hfill \textcolor{gray}{\textbf{\textcolor{pink}{LEIPZIG}{}\ledrightnote{\textcolor{pink}{Leipzig}}.}}\pend
           \pstart
           \textcolor{gray}{\textbf{\textcolor{brown}{\so{F. Tempsky}}{}\ledrightnote{\textcolor{brown}{F. Tempsky}}.}}\hfill \textcolor{gray}{\textbf{\textcolor{brown}{\so{G. Freytag G. m. b. H.}}{}\ledrightnote{\textcolor{brown}{G. Freytag}}}}\pend
           \pstart
           \centering{}\textcolor{gray}{\textbf{1908.}}\pend
           \endnumbering\briefempfaengerindex{Schnitzler, Arthur@\textsc{Schnitzler, Arthur}!zzzBurckhard, Max Eugen@\emph{von Max Eugen Burckhard}!1907-11-141@{14. 11. 1907}|)be}\mylabel{h}  \normalsize

\doendnotes{C}
\bigskip
\vfill

\clearpage

\footnotesize

\lohead{\textsc{register}}

% Definiere theindex-Environment komplett neu ohne reledmac
\makeatletter
\renewenvironment{theindex}{%
  \section*{\indexname}%
  \setlength{\parindent}{0pt}%
  \setlength{\parskip}{0pt plus 0.3pt}%
  \let\item\@idxitem
}{%
  \clearpage
}
\makeatother

\IfFileExists{\jobname-pw.ind}{\input{\jobname-pw.ind}}{}

\end{document}

      