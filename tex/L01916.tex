%% latex-korrekturansicht-vorspann.tex
%% Vorspann für die Korrekturansicht.
%% Lädt die gemeinsame Datei latex-vorspann.tex mit gesetztem Schalter.

\newif\ifkorrekturansicht
\korrekturansichttrue

\input{../tex-inputs/latex-vorspann}


               \section[Richard Beer-Hofmann an Arthur Schnitzler, 6. 3. 1910]{ Richard Beer-Hofmann an Arthur Schnitzler, 6. 3. 1910}\nopagebreak\mylabel{v}\rehead{ }\normalsize\beginnumbering\briefempfaengerindex{Schnitzler, Arthur@\textsc{Schnitzler, Arthur}!zzzBeer-Hofmann, Richard@\emph{von Richard Beer-Hofmann}!1910-03-061@{6. 3. 1910}|(be} \toendnotes[C]{\smallbreak\pagebreak[2]} \Standort{CUL, Schnitzler, B 8.}
\physDesc{Kartenbrief, 1 Blatt, 3 Seiten
\newline{}Handschrift: Bleistift, lateinische Kurrent\newline{}Versand: ohne postalischen Übermittlungsvermerk \newline{}Ordnung: mit Bleistift von unbekannter Hand nummeriert: »229« }\buchAbdrucke{\weitereDrucke{Arthur Schnitzler, Richard Beer-Hofmann: \emph{Briefwechsel 1891–1931}. Hg. Konstanze Fliedl. Wien, Zürich: \emph{Europaverlag} 1992, S. 207.} }\pstart{}{\pb}Herrn\pend{}\pstart{}Arthur Schnitzler\pend{}{\bigskip}\pstart
           \raggedleft{}{\pb}6/3. 10\pend
           \pstart{}Lieber Arthur!\pend\pstart
           \textcolor{blue}{Hans Sch.}{}\ledrightnote{\textcolor{blue}{Hans Bernhard Schlesinger}} bittet mich \strikeout{bei} teleph. bei Ihnen anzufragen, ob er heute um 7 ½ mit dem
               Cellisten \textcolor{blue}{Ba\introOben{}r\introOben{}czanski}{}\ledrightnote{\textcolor{blue}{Alexander Barjanski}} zu
               Ihnen ko{\geminationm}en kann.\pend
           \pstart
           {\pb}Herzlichen Glückwunsch{\\[\baselineskip]}Ihr{\\[\baselineskip]}\spacefill\mbox{Richard}\pend
           \leftskip=0em{}\pstart
           \noindent{}Sie sind doch Abends eingeladen bei \textcolor{blue}{T.}{}\ledrightnote{\textcolor{blue}{Siegfried Trebitsch}}?\pend
           \endnumbering\briefempfaengerindex{Schnitzler, Arthur@\textsc{Schnitzler, Arthur}!zzzBeer-Hofmann, Richard@\emph{von Richard Beer-Hofmann}!1910-03-061@{6. 3. 1910}|)be}\mylabel{h}  \normalsize

\doendnotes{C}
\bigskip
\vfill

\clearpage

\footnotesize

\lohead{\textsc{register}}

% Definiere theindex-Environment komplett neu ohne reledmac
\makeatletter
\renewenvironment{theindex}{%
  \section*{\indexname}%
  \setlength{\parindent}{0pt}%
  \setlength{\parskip}{0pt plus 0.3pt}%
  \let\item\@idxitem
}{%
  \clearpage
}
\makeatother

\IfFileExists{\jobname-pw.ind}{\input{\jobname-pw.ind}}{}

\end{document}

      