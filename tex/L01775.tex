%% latex-korrekturansicht-vorspann.tex
%% Vorspann für die Korrekturansicht.
%% Lädt die gemeinsame Datei latex-vorspann.tex mit gesetztem Schalter.

\newif\ifkorrekturansicht
\korrekturansichttrue

\input{../tex-inputs/latex-vorspann}


               \section[Richard Dehmel an Arthur Schnitzler, 8. 6. 1908]{ Richard Dehmel an Arthur Schnitzler, 8. 6. 1908}\nopagebreak\mylabel{v}\rehead{ }\normalsize\beginnumbering\briefempfaengerindex{Schnitzler, Arthur@\textsc{Schnitzler, Arthur}!zzzDehmel, Richard@\emph{von Richard Dehmel}!1908-06-081@{8. 6. 1908}|(be} \toendnotes[C]{\smallbreak\pagebreak[2]} \Standort{DLA, A:Schnitzler, HS.NZ85.1.2730.}
\physDesc{maschinelle Abschrift\newline{}Zusatz: Original nicht nachweisbar }\toendnotes[C]{\smallbreak}\pstart
           \raggedleft{}{\pb}\textcolor{pink}{Braunwald}{}\ledrightnote{\textcolor{pink}{Braunwald}},
                        8. 6. 1908.\pend
           \pstart{}Verehrter Herr Schnitzler!\pend\pstart
           Möge der Titel Ihres \textcolor{green}{Romans}{}\ledrightnote{→\textcolor{green}{Der Weg ins Freie. Roman}}
                    mir ein Omen sein. Ich sitze nämlich auf einem \textcolor{pink}{Schweiz}{}\ledrightnote{\textcolor{pink}{Schweiz}}er Berg in dickem Nebel, und es wird wohl noch eine Woche
                    dauern, bis der Regen herunter ist. Da kann ich also Ihrem »\textcolor{green}{Weg ins Freie}{}\ledrightnote{\textcolor{green}{Der Weg ins Freie. Roman}}« – (zum Glück konnte ich mich nicht
                    entschliessen, ihn in der \textcolor{green}{Neuen Rdschau}{}\ledrightnote{\textcolor{green}{Die neue Rundschau}} zu
                    lesen) – die verständnisvollste Andacht widmen.\pend
           \pstart
           Mit schönstem Dank{\\[\baselineskip]}Ihr{\\[\baselineskip]}\spacefill\mbox{Dehmel.}\pend
           \leftskip=0em{}\endnumbering\briefempfaengerindex{Schnitzler, Arthur@\textsc{Schnitzler, Arthur}!zzzDehmel, Richard@\emph{von Richard Dehmel}!1908-06-081@{8. 6. 1908}|)be}\mylabel{h}  \normalsize

\doendnotes{C}
\bigskip
\vfill

\clearpage

\footnotesize

\lohead{\textsc{register}}

% Definiere theindex-Environment komplett neu ohne reledmac
\makeatletter
\renewenvironment{theindex}{%
  \section*{\indexname}%
  \setlength{\parindent}{0pt}%
  \setlength{\parskip}{0pt plus 0.3pt}%
  \let\item\@idxitem
}{%
  \clearpage
}
\makeatother

\IfFileExists{\jobname-pw.ind}{\input{\jobname-pw.ind}}{}

\end{document}

      