%% latex-korrekturansicht-vorspann.tex
%% Vorspann für die Korrekturansicht.
%% Lädt die gemeinsame Datei latex-vorspann.tex mit gesetztem Schalter.

\newif\ifkorrekturansicht
\korrekturansichttrue

\input{../tex-inputs/latex-vorspann}


               \section[Franz Blei an Arthur Schnitzler, {[}Anfang? Januar 1911{]}]{ Franz Blei an Arthur Schnitzler, {[}Anfang? Januar 1911{]}}\nopagebreak\mylabel{v}\rehead{ }\normalsize\beginnumbering\briefempfaengerindex{Schnitzler, Arthur@\textsc{Schnitzler, Arthur}!zzzBlei, Franz@\emph{von Franz Blei}!1911-01-011@{{[}Anfang? Januar 1911{]}}|(be} \toendnotes[C]{\smallbreak\pagebreak[2]} \Standort{CUL, Schnitzler, B 14.}
\physDesc{Brief, 1 Blatt, 2 Seiten
\newline{}Handschrift: grüne Tinte, lateinische Kurrent
\newline{}Schnitzler: mit Bleistift datiert: »Jänner 911« \newline{}Ordnung: mit Bleistift von unbekannter Hand nummeriert:
                                 »7« }\toendnotes[C]{\smallbreak}\pstart
           \noindent{}{\pb}\textcolor{gray}{\textbf{DR. FRANZ BLEI}}\hfill \textcolor{pink}{\textcolor{gray}{\textbf{MÜNCHEN}}}{}\ledrightnote{\textcolor{pink}{München}}\pend
           \pstart
           \raggedleft{}\textcolor{pink}{\textcolor{gray}{\textbf{Tengstraße 41}}}{}\ledrightnote{\textcolor{pink}{Tengstraße}}\pend
           \pstart{}Verehrter Herr Doktor,\pend\pstart
           den \label{K_L01998_1v}\edtext{\textcolor{green}{Aufsatz}{}\ledrightnote{→\textcolor{green}{Über Sehnsucht und Form}}}{\lemma{\textnormal{\emph{Aufsatz}}}\Cendnote{\textnormal{Ein Teil davon erschien unmittelbar nach
                  dem Brief gedruckt: \textcolor{blue}{Georg von Lukacs}: \emph{\textcolor{green}{Über Sehnsucht und Form}}. In: \emph{\textcolor{green}{Die neue Rundschau}}, Jg. 22, H. 2, Februar 1911,
                     S. 192–198.}}}\label{K_L01998_1h} über \textcolor{blue}{Ch. L. Philippe}{}\ledrightnote{\textcolor{blue}{Charles-Louis Philippe}}, von D\textsuperscript{r} \textcolor{blue}{Georg von \label{T_L01998_1v}\edtext{Lukacs}{\lemma{\textnormal{\emph{Lukacs}}}\Cendnote{\textnormal{korrigiert aus: »Lukasc«}}}\label{T_L01998_1h}}{}\ledrightnote{\textcolor{blue}{Georg von Lukács}} schicke ich Ihnen, sowie ich ihn \label{K_L01998_2v}\edtext{zurück{[}be{]}komme}{\lemma{\textnormal{\emph{zurückbekomme}}}\Cendnote{\textnormal{Zwei Briefe von \textcolor{blue}{Blei} an \textcolor{blue}{Lukács} lassen diesen Brief näher eingrenzen. Am
                     26. 12. 1909 schreibt \textcolor{blue}{Blei},
                     \textcolor{blue}{Schnitzler} habe ihn bei einem Treffen in \textcolor{pink}{München} um den Text gebeten. (\textcolor{blue}{Georg Lukács}: \emph{Briefwechsel
                        1902–1917}. Hgg. Éva Karádi und Éva Fekete. Stuttgart:
                        \emph{Metzler}{ }1982, S. 189.) Am 6. 1. 1910 schreibt \textcolor{blue}{Lukács}, er nehme die Vermittlung zur \emph{\textcolor{brown}{Neuen Freien Presse}} an. (Ebd.,
                  S. 196.) Dadurch, dass eine solche Vermittlung gerade nicht stattfindet,
                  der Text aber schon ab Mitte des Monats für \emph{\textcolor{green}{Die neue
                     Rundschau}} blockiert sein musste, ist der Brief davor anzusiedeln.}}}\label{K_L01998_2h} –
               ich fürchte zwar, er wird für die \textcolor{brown}{N. F. P.}{}\ledrightnote{\textcolor{brown}{Neue Freie Presse}} zu lang
               sein, so etwa 10 Spalten. Aber, er wird doch Sie interessiren.\pend
           \pstart
           An \textcolor{pink}{Forte dei Marmi}{}\ledrightnote{\textcolor{pink}{Forte dei Marmi}} will ich Sie noch erinnern.
               Weg: \textcolor{pink}{Florenz}{}\ledrightnote{\textcolor{pink}{Florenz}}–\textcolor{pink}{Pisa}{}\ledrightnote{\textcolor{pink}{Pisa}}–\textcolor{pink}{Pietrasanta}{}\ledrightnote{\textcolor{pink}{Pietrasanta}}. Von da im Wägelchen.
               Ein Haus (5 Zimmer) mit Garten kostet für die Saison (ein Wort zu grossartig für das
               ganz unluxuriöse \textcolor{pink}{Forte}{}\ledrightnote{\textcolor{pink}{Forte dei Marmi}}), d. h.
                  1. Juni bis Ende September 400–500 francs. Die \label{K_L01998_3v}\edtext{Capana}{\lemma{\textnormal{\emph{Capana}}}\Cendnote{\textnormal{italienisch capanna: Hütte}}}\label{K_L01998_3h} für diese Zeit etwa 80 frs.
               Die Person, die kommt, um einzukaufen, zu kochen, aufzuräumen, bekommt 1 Lira pro Tag
               – wenn sie im Haus schläft 20 frs im Monat. {\pb}Der sehr schöne Strand ist 4–5 Stunden
               lang, reicht von \textcolor{pink}{Viareggio}{}\ledrightnote{\textcolor{pink}{Viareggio}} bis \textcolor{pink}{Massa Carrara}{}\ledrightnote{\textcolor{pink}{Massa}}. Es giebt Wälder und die sehr schönen Carraraberge. Es regnet so gut wie nie und die Wärme
               ist immer erträglich. –\pend
           \pstart
           Pensionen nehmen 7 frs pro Tag den erwachsenen Menschen, Kinder 3 frs.\pend
           \pstart
           Es ist sehr schön, sehr still da und sehr viel Raum. An den \textcolor{pink}{Lido}{}\ledrightnote{\textcolor{pink}{Lido}} dürfen Sie nicht denken.\pend
           \pstart
           Das ist alles was über \textcolor{pink}{Forte}{}\ledrightnote{\textcolor{pink}{Forte dei Marmi}} zu sagen ist.\pend
           \pstart
           Herzlich \textcolor{gray}{Ihr} ergebener{\\[\baselineskip]}\spacefill\mbox{Frz Blei}\pend
           \leftskip=0em{}\endnumbering\briefempfaengerindex{Schnitzler, Arthur@\textsc{Schnitzler, Arthur}!zzzBlei, Franz@\emph{von Franz Blei}!1911-01-011@{{[}Anfang? Januar 1911{]}}|)be}\mylabel{h}  \normalsize

\doendnotes{C}
\bigskip
\vfill

\clearpage

\footnotesize

\lohead{\textsc{register}}

% Definiere theindex-Environment komplett neu ohne reledmac
\makeatletter
\renewenvironment{theindex}{%
  \section*{\indexname}%
  \setlength{\parindent}{0pt}%
  \setlength{\parskip}{0pt plus 0.3pt}%
  \let\item\@idxitem
}{%
  \clearpage
}
\makeatother

\IfFileExists{\jobname-pw.ind}{\input{\jobname-pw.ind}}{}

\end{document}

      