%% latex-korrekturansicht-vorspann.tex
%% Vorspann für die Korrekturansicht.
%% Lädt die gemeinsame Datei latex-vorspann.tex mit gesetztem Schalter.

\newif\ifkorrekturansicht
\korrekturansichttrue

\input{../tex-inputs/latex-vorspann}


               \section[Hugo von Hofmannsthal an Arthur Schnitzler, 17. {[}7. 1895{]}]{ Hugo von Hofmannsthal an Arthur Schnitzler, 17. {[}7. 1895{]}}\nopagebreak\mylabel{v}\rehead{ }\normalsize\beginnumbering\briefempfaengerindex{Schnitzler, Arthur@\textsc{Schnitzler, Arthur}!zzzHofmannsthal, Hugo von@\emph{von Hugo von Hofmannsthal}!1895-07-172@{17. {[}7. 1895{]}}|(be} \toendnotes[C]{\smallbreak\pagebreak[2]} \Standort{CUL, Schnitzler, B 43.}
\physDesc{Brief, 1 Blatt, 3 Seiten
\newline{}Handschrift: schwarze Tinte, deutsche Kurrent
\newline{}Schnitzler: mit Bleistift Datum der Beantwortung vermerkt: »7 95« und nummeriert: »73« }\buchAbdrucke{\weitereDrucke{1) Hugo von Hofmannsthal: \emph{Briefe. 1890–1901}. Berlin: \emph{S. Fischer} 1935, S. 152–153.} \weitereDrucke{2) Hugo von Hofmannsthal, Arthur Schnitzler: \emph{Briefwechsel}. Hg. Therese Nickl und Heinrich Schnitzler. Frankfurt am Main: \emph{S. Fischer} 1964, S. 56.} }\toendnotes[C]{\smallbreak}\pstart
           \raggedleft{}{\pb}\textcolor{pink}{Göding}{}\ledrightnote{\textcolor{pink}{Hodonín}}, 17\textsuperscript{ten}{ }11 Uhr. \pend
           \pstart
           \raggedleft{}\textcolor{gray}{\textbf{\strikeout{\textcolor{pink}{Salesianergasse 12}{}\ledrightnote{\textcolor{pink}{Salesianergasse}}}}}\pend
           \pstart
           es macht mir eine merkwürdige Freude, dieſem Brief in Gedanken nachzugehen. Ich
                    habe voriges Jahr ſehr glücklich vor mich hingelebt, von den Tagen in \textcolor{pink}{Salzburg}{}\ledrightnote{\textcolor{pink}{Salzburg}} bis in den September fühle
                    ich im Zurückdenken das complexe Glück von Bewegung, Blick und Gedanken,
                    ſich-Hergeben und ſich-Behalten, Mitleid, Verliebtheit und Einſamkeit, dunklen
                    Gewittern am Abend und blaßgelben lautloſen Blitzen in der Nacht; am Anfang mehr
                    die Melancholie der kleinen Eiſenbahn mit dem Roth vom Sonnenuntergang auf den
                    Kupfernägeln der Bänke, mit den geſchminkten und lautredenden {\pb}Frauen in allen Stationen,
                    mit dem plötzlichen Dunkel- und Kaltwerden in dem kleinen Tunnel und gleich
                    darauf den harmloſen von nichts wiſſenden Bauernhäuſern und kleinen Gärten; am
                    Ende mehr die ſtundenlangen Geſpräche in der Nacht im Regen, im Wald und auf der
                    weißen naſſen Landſtraße mit \textcolor{blue}{Edgar}{}\ledrightnote{\textcolor{blue}{Edgar von Karg-Bebenburg}} und das ſo
                    ſtarke aufgeregte Fühlen von ſein und meinem Leben wie in einem.\pend
           \pstart
           Als ein beſonders merkwürdiger \label{K_L00464_1v}\edtext{Tag}{\lemma{\textnormal{\emph{Tag}}}\Cendnote{\textnormal{der
                        3. 9. 1894}}}\label{K_L00464_1h} erſcheint mir der, wo wir mit \textcolor{blue}{Goldmann}{}\ledrightnote{\textcolor{blue}{Paul Goldmann}} vor ſeiner Abreiſe zuerſt beim \textcolor{pink}{Leopold}{}\ledrightnote{\textcolor{pink}{Hotel und Pension Rudolfshöhe (Leopold Petter)}} waren und dann ein großes Gewitter
                    gekommen iſt. Ich kann aber nicht finden, warum.\pend
           \pstart
           {\pb}Heute nachmittag gehe ich
                    auf Patrouille und bleib über Nacht aus. Morgen wenn ich zurückkomm und gebadet
                    hab, wird der \textcolor{green}{Pan}{}\ledrightnote{\textcolor{green}{Pan}} daliegen, den mir der \textcolor{blue}{Salten}{}\ledrightnote{\textcolor{blue}{Felix Salten}} geſchickt hat. An ſolchen kleinen
                    Freuden bringe ich mich wie an Springſtöcken von Stein zu Stein über dieſe Öde
                    hinüber.\pend
           \pstart
           Adieu, ſchreiben Sie und \textcolor{blue}{Richard}{}\ledrightnote{\textcolor{blue}{Richard Beer-Hofmann}} mir
                    doch bald.{\\[\baselineskip]} Ihr{\\[\baselineskip]}\spacefill\mbox{Hugo.}\pend
           \leftskip=0em{}\endnumbering\briefempfaengerindex{Schnitzler, Arthur@\textsc{Schnitzler, Arthur}!zzzHofmannsthal, Hugo von@\emph{von Hugo von Hofmannsthal}!1895-07-172@{17. {[}7. 1895{]}}|)be}\mylabel{h}  \normalsize

\doendnotes{C}
\bigskip
\vfill

\clearpage

\footnotesize

\lohead{\textsc{register}}

% Definiere theindex-Environment komplett neu ohne reledmac
\makeatletter
\renewenvironment{theindex}{%
  \section*{\indexname}%
  \setlength{\parindent}{0pt}%
  \setlength{\parskip}{0pt plus 0.3pt}%
  \let\item\@idxitem
}{%
  \clearpage
}
\makeatother

\IfFileExists{\jobname-pw.ind}{\input{\jobname-pw.ind}}{}

\end{document}

      