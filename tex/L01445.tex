%% latex-korrekturansicht-vorspann.tex
%% Vorspann für die Korrekturansicht.
%% Lädt die gemeinsame Datei latex-vorspann.tex mit gesetztem Schalter.

\newif\ifkorrekturansicht
\korrekturansichttrue

\input{../tex-inputs/latex-vorspann}


               \section[Richard Beer-Hofmann an Arthur Schnitzler, 15. 9. 1904]{ Richard Beer-Hofmann an Arthur Schnitzler,
               15. 9. 1904}\nopagebreak\mylabel{v}\rehead{ }\normalsize\beginnumbering\briefempfaengerindex{Schnitzler, Arthur@\textsc{Schnitzler, Arthur}!zzzBeer-Hofmann, Richard@\emph{von Richard Beer-Hofmann}!1904-09-151@{15. 9. 1904}|(be} \toendnotes[C]{\smallbreak\pagebreak[2]} \Standort{CUL, Schnitzler, B 8.}
\physDesc{Brief, 1 Blatt, 1 Seite
\newline{}Handschrift: blaue Tinte, lateinische Kurrent\newline{}Ordnung: mit Bleistift von unbekannter Hand nummeriert: »190« }\toendnotes[C]{\smallbreak}\pstart
           \centering{}{\pb}\textcolor{pink}{Aussee}{}\ledrightnote{\textcolor{pink}{Bad Aussee}}{ }15/IX. 04\pend
           \pstart
           Lieber Arthur! \textcolor{blue}{Paula}{}\ledrightnote{\textcolor{blue}{Paula Beer-Hofmann}} und \textcolor{blue}{Kinder}{}\ledrightnote{→\textcolor{blue}{Gabriel Beer-Hofmann}{\newline}→\textcolor{blue}{Naëmah Beer-Hofmann}{\newline}→\textcolor{blue}{Mirjam Beer-Hofmann}} fahren
                  morgen Mittag zu \textcolor{blue}{Papa}{}\ledrightnote{→\textcolor{blue}{Hermann Beer}} nach \textcolor{pink}{Baden}{}\ledrightnote{\textcolor{pink}{Baden bei Wien}}. In \textcolor{pink}{Lueg}{}\ledrightnote{\textcolor{pink}{Lueg am Wolfgangsee}} übernachten hat nicht viel Sinn; außerdem suche ich mit
               meinem Husten möglichst bald aus allzu feuchter Luft zu entweichen. Wir treffen uns
               also in \textcolor{pink}{Salzburg}{}\ledrightnote{\textcolor{pink}{Salzburg}}. Bin noch unentschlossen, wo ich
               wohnen soll.\pend
           \pstart
           Ich hinterlege Brief für Sie im Caffée \textcolor{pink}{Tomaselli}{}\ledrightnote{\textcolor{pink}{Café Tomaselli}}
               mit meiner Adresse. \uline{Vielleicht} wohne ich »\textcolor{pink}{Schiff}{}\ledrightnote{\textcolor{pink}{Hotel Schiff}}«. Geben Sie mir – ebenfalls \strikeout{bis} bei \textcolor{pink}{Tomaselli}{}\ledrightnote{\textcolor{pink}{Café Tomaselli}} –
               Ihre Adresse an.\pend
           \pstart
           Herzlichst Ihr{\\[\baselineskip]}\spacefill\mbox{Richard}\pend
           \leftskip=0em{}\endnumbering\briefempfaengerindex{Schnitzler, Arthur@\textsc{Schnitzler, Arthur}!zzzBeer-Hofmann, Richard@\emph{von Richard Beer-Hofmann}!1904-09-151@{15. 9. 1904}|)be}\mylabel{h}  \normalsize

\doendnotes{C}
\bigskip
\vfill

\clearpage

\footnotesize

\lohead{\textsc{register}}

% Definiere theindex-Environment komplett neu ohne reledmac
\makeatletter
\renewenvironment{theindex}{%
  \section*{\indexname}%
  \setlength{\parindent}{0pt}%
  \setlength{\parskip}{0pt plus 0.3pt}%
  \let\item\@idxitem
}{%
  \clearpage
}
\makeatother

\IfFileExists{\jobname-pw.ind}{\input{\jobname-pw.ind}}{}

\end{document}

      