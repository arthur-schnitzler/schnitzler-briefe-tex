%% latex-korrekturansicht-vorspann.tex
%% Vorspann für die Korrekturansicht.
%% Lädt die gemeinsame Datei latex-vorspann.tex mit gesetztem Schalter.

\newif\ifkorrekturansicht
\korrekturansichttrue

\input{../tex-inputs/latex-vorspann}


               \section[Arthur Schnitzler an Hugo von Hofmannsthal, 10. 12. 1903]{ Arthur Schnitzler an Hugo von Hofmannsthal, 10. 12. 1903}\nopagebreak\mylabel{v}\rehead{ }\normalsize\beginnumbering\briefempfaengerindex{Hofmannsthal, Hugo von@\textsc{Hofmannsthal, Hugo von}!zzzSchnitzler, Arthur@\emph{von Arthur Schnitzler}!1903-12-101@{10. 12. 1903}|(be} \toendnotes[C]{\smallbreak\pagebreak[2]} \Standort{FDH, Hs-30885,106.}
\physDesc{Brief, 2 Blätter, 6 Seiten
\newline{}Handschrift: schwarze Tinte, deutsche Kurrent\newline{}Ordnung: 1) von Schnitzler mutmaßlich bei der Durchsicht der Korrespondenz
                                    1929 mit Bleistift datiert: »910« 2) mit Bleistift von \textcolor{blue}{Olga
                                    Schnitzler} neben der Adressangabe vermerkt: »\textsc{Irrtum: damals wohnten wir schon in der \textcolor{pink}{Sternwartestrasse}. O.}«, was
                                 sich auf die (falsche) nachträgliche Einordnung auf das Jahr 1910
                                 bezieht3) das zweite Blatt von unbekannter Hand mit Bleistift beschriftet:
                                    »II 10/12 910«4) mit Bleistift von unbekannter Hand nummeriert
                                    »106a«}\buchAbdrucke{\weitereDrucke{Hugo von Hofmannsthal, Arthur Schnitzler: \emph{Briefwechsel}. Hg. Therese Nickl und Heinrich Schnitzler. Frankfurt am Main: \emph{S. Fischer} 1964, S. 179–180.} }\toendnotes[C]{\smallbreak}\pstart
           \raggedleft{}{\pb}\textcolor{pink}{XVIII Spöttelg. 7}{}\ledrightnote{\textcolor{pink}{Edmund-Weiß-Gasse}}. {\\}\textcolor{pink}{Wien}{}\ledrightnote{\textcolor{pink}{Wien}}{ }10. 12. 9\textcolor{gray}{03}\pend
           \pstart{}mein lieber Hugo, \pend\pstart
           Sie haben offenbar einen Brief von mir nicht beko{\geminationm}en,
               den ich an Sie vor etwa 14 Tagen, ich glaube an dem Tag wo Ihre \textcolor{green}{Elektra}{}\ledrightnote{\textcolor{green}{Elektra. Tragödie in einem Aufzug}} bei mir erſchien, an Sie geſchrieben habe. Das
               weſentlichſte, was dieſer Brief enthielt war die Bitte Ihre \textcolor{green}{Elektra}{}\ledrightnote{\textcolor{green}{Elektra. Tragödie in einem Aufzug}} an \textcolor{blue}{\textsc{Antoine}}{}\ledrightnote{\textcolor{blue}{André Antoine}}, \textsc{resp}. an Dr \textsc{\textcolor{blue}{Stephan Epstein}{}\ledrightnote{\textcolor{blue}{Stephan Epstein}}{ }\textcolor{pink}{Paris 78 rue de l’Assomption}{}\ledrightnote{\textcolor{pink}{rue de l’Assomption}}, \textcolor{blue}{Antoine}{}\ledrightnote{\textcolor{blue}{André Antoine}}s} Dramaturgen fürs Ausland zu ſenden, dem ich
               neulich \strikeout{darüber} über das \textcolor{green}{Stück}{}\ledrightnote{→\textcolor{green}{Elektra. Tragödie in einem Aufzug}} kurz berichtet habe.\pend
           \pstart
           {\pb}Daſs \textcolor{green}{\introOben{}B.\introOben{} Garlan}{}\ledrightnote{\textcolor{green}{Frau Bertha Garlan. Roman}} beim zweiten Leſen ſo angenehm auf Sie
               wirkte, freut mich ſehr – ich hab es ſeit dem Erſcheinen nicht wieder geleſen wie ich
               es (we{\geminationn} mich nicht äußerliche Gründe zu einer
               wiederholten Lectüre nöthigen) mit allen meinen gedruckten Sachen halte. Daher weiſs
               ich auch ſeit etwa 8 Jahren nichts mehr von »\textcolor{green}{Sterben}{}\ledrightnote{\textcolor{green}{Sterben. Novelle}}«. Es sta{\geminationm}t aus der Zeit, wo mich der
               »Fall« mehr intereſſirt hat als die Menſchen, und ich denke das meiſte aus dieſer
               Epoche muſs wie luftlos wirken. Dieſe Sachen – ich hab es neulich wieder am »\textcolor{green}{\textsc{Jour de {\pb}gloire}}{}\ledrightnote{\textcolor{green}{Der Ehrentag}}« \substVorne{}\textsuperscript{g}\substDazwischen{}e\substHinten{}rfahren, wirken in anſtändiger franzöſiſcher Übertragung beſſer als in meinem
               Deutſch. Die reine Tendenz des Erzählens iſt dem romaniſchen Sprachgeiſt eingeboren,
               während es im deutſchen gleichſam wie gegen die Natur wirkt, wenn die Mittheilung von
               Thatſachen der Seele und Menſchlichkeit entbehrt. Die umgekehrte Probe kann man
               machen, we{\geminationn} man irgend eine kurze \textcolor{blue}{\textsc{Maupassant}}{}\ledrightnote{\textcolor{blue}{Guy de Maupassant}} Geſchichte die franzöſiſch noch lange nicht ſchwach wirkt, in deutſcher
               Ueberſetzung lieſt.\pend
           \pstart
           – Immerhin hab ich die Empfindg daſs {\pb}meine Technik der
               inneren Entwicklung meiner Production noch nicht nachgekommen iſt – was mir übrigens
               nicht bange macht. Es iſt jetzt in mir wieder ſo eine Neigung Sachen nur anzufangen
               und zu ſkizziren wie in der Zeit, die der \textcolor{green}{Anatol}{}\ledrightnote{\textcolor{green}{Anatol}}-Epoche vorherging. Am meiſten beſchäftige ich mich jetzt mit einer Art
               von \textcolor{green}{Komödie}{}\ledrightnote{→\textcolor{green}{Fink und Fliederbusch. Komödie in drei Akten}} und bin innerlich
                  \strikeout{von dem \textcolor{green}{Roman}{}\ledrightnote{→\textcolor{green}{Der Weg ins Freie. Roman}}} am meiſten von dem \textcolor{green}{Roman}{}\ledrightnote{→\textcolor{green}{Der Weg ins Freie. Roman}}
               erfüllt, den ich im Frühjahr begonnen, den aber fortzuſetzen ich nicht in genügend
               reiner Sti{\geminationm}ung mich befinde.\pend
           \pstart
           In Concerte gehen wir nicht ſelten, ins Theater beinahe nie, aus perſönlichen {\pb}Gründen waren wir bei der \label{K_L01348_1v}\edtext{\textcolor{green}{\textsc{Novella d’Andrea}}{}\ledrightnote{\textcolor{green}{Novella d’Andrea}}}{\lemma{\textnormal{\emph{Novella d’Andrea}}}\Cendnote{\textnormal{siehe A. S.: \emph{Tagebuch}, 21. 11. 1903}}}\label{K_L01348_1h} – und ich hab es nicht ohne Bitterkeit empfunden, daſs ich den \textcolor{blue}{Kainz}{}\ledrightnote{\textcolor{blue}{Josef Kainz}} nie werde den \textcolor{green}{Sala}{}\ledrightnote{→\textcolor{green}{Der einsame Weg. Schauspiel in fünf Akten}}{ }ſpielen \strikeout{k}{ }ſehen. Denn das \textcolor{brown}{Burgtheater}{}\ledrightnote{\textcolor{brown}{Burgtheater}}, wie Herr \textcolor{blue}{Schlenther}{}\ledrightnote{\textcolor{blue}{Paul Schlenther}} an \textcolor{blue}{Fiſcher}{}\ledrightnote{\textcolor{blue}{Samuel Fischer}} geſchrieben, »reflectirt nicht« auf dieſes
                  \textcolor{green}{Stück}{}\ledrightnote{→\textcolor{green}{Der einsame Weg. Schauspiel in fünf Akten}}. \textcolor{blue}{Brahm}{}\ledrightnote{\textcolor{blue}{Otto Brahm}} gegenüber (was Sie ja wohl wiſſen dürften) hat sich \textcolor{blue}{Schl.}{}\ledrightnote{\textcolor{blue}{Paul Schlenther}} über das \textcolor{green}{Stück}{}\ledrightnote{→\textcolor{green}{Der einsame Weg. Schauspiel in fünf Akten}}{ }ſehr misfällig geäußert; ſcheint es aber, wie \textcolor{blue}{Brahm}{}\ledrightnote{\textcolor{blue}{Otto Brahm}}{ }ſagt, ganz oberflächlich – und wie ich überzeugt
               bin – mit böſem Willen geleſen zu haben.\pend
           \pstart
           Und nun, wann ſieht man ſich wieder? Wie wär es, Montag oder
                  Mittwoch{ }Abend in dem \textcolor{pink}{Hietzinger
                  Restaurant}{}\ledrightnote{→\textcolor{pink}{Ottakringer Bräu}}? Schrei{\pb}ben Sie mir, wann es Ihnen
               beſſer paſſt und ob auch Ihre \textcolor{blue}{Frau}{}\ledrightnote{→\textcolor{blue}{Gertrude von Hofmannsthal}} mitkommt.\pend
           \pstart
           Und \textcolor{blue}{Richard}{}\ledrightnote{\textcolor{blue}{Richard Beer-Hofmann}}? Ich höre u ſehe nichts von ihm. –
               Sobald das Wetter ein bischen angenehmer wird, kommen wir gern nach \textcolor{pink}{Rodaun}{}\ledrightnote{\textcolor{pink}{Rodaun}}.\pend
           \pstart
           \label{K_L01348_2v}\edtext{Das andere}{\lemma{\textnormal{\emph{Das andere}}}\Cendnote{\textnormal{vgl. Hugo von Hofmannsthal an Arthur Schnitzler, 8. 12. [1903]}}}\label{K_L01348_2h}, das ich bald bekomme, iſt wohl das \textcolor{green}{gerettete
                     \textsc{Venedig}}{}\ledrightnote{\textcolor{green}{Das gerettete Venedig. Trauerspiel in fünf Aufzügen}}? –\pend
           \pstart
           Leben Sie wohl. Herzlichſt\hspace*{1.5em}Ihr{\\[\baselineskip]}\spacefill\mbox{A.}\pend
           \leftskip=0em{}\endnumbering\briefempfaengerindex{Hofmannsthal, Hugo von@\textsc{Hofmannsthal, Hugo von}!zzzSchnitzler, Arthur@\emph{von Arthur Schnitzler}!1903-12-101@{10. 12. 1903}|)be}\mylabel{h}  \normalsize

\doendnotes{C}
\bigskip
\vfill

\clearpage

\footnotesize

\lohead{\textsc{register}}

% Definiere theindex-Environment komplett neu ohne reledmac
\makeatletter
\renewenvironment{theindex}{%
  \section*{\indexname}%
  \setlength{\parindent}{0pt}%
  \setlength{\parskip}{0pt plus 0.3pt}%
  \let\item\@idxitem
}{%
  \clearpage
}
\makeatother

\IfFileExists{\jobname-pw.ind}{\input{\jobname-pw.ind}}{}

\end{document}

      