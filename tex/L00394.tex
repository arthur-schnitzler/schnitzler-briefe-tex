%% latex-korrekturansicht-vorspann.tex
%% Vorspann für die Korrekturansicht.
%% Lädt die gemeinsame Datei latex-vorspann.tex mit gesetztem Schalter.

\newif\ifkorrekturansicht
\korrekturansichttrue

\input{../tex-inputs/latex-vorspann}


               \section[Richard Beer-Hofmann an Arthur Schnitzler, 28. 10. 1894]{ Richard Beer-Hofmann an Arthur Schnitzler, 28. 10. 1894}\nopagebreak\mylabel{v}\rehead{ }\normalsize\beginnumbering\briefempfaengerindex{Schnitzler, Arthur@\textsc{Schnitzler, Arthur}!zzzBeer-Hofmann, Richard@\emph{von Richard Beer-Hofmann}!1894-10-281@{28. 10. 1894}|(be} \toendnotes[C]{\smallbreak\pagebreak[2]} \Standort{CUL, Schnitzler, B 8.}
\physDesc{Postkarte
\newline{}Handschrift: Bleistift, deutsche Kurrent\newline{}Versand: 1) Stempel: »\nobreak{}\oindex{Grand Hotel Bauer-Gruenwald@\textbf{Grand Hotel Bauer-Grünwald}, \emph{Hotel (K.HTL)}|pwk}Hotel d’Italie {\kaufmannsund}
                              Bauer Bauer Grünwald Venise, 28 Oct. 94\nobreak{}«.  2) Stempel: »\nobreak{}\oindex{Bahnhof@\textbf{Bahnhof}, \emph{Bahnhofsgebäude (K.BHF)}|pwk}Venezia Ferrovia, 28 10–94, 9 S\nobreak{}«. 3) Stempel: »\nobreak{}\oindex{IX., Alsergrund@\textbf{IX., Alsergrund}, \emph{Bezirk (A.BZK)}|pwk}Wien 9/3, 30. 10. 94, 8.V, Bestellt\nobreak{}«. 
\newline{}Schnitzler: mit Bleistift datiert: »28/10 94« und nummeriert: »31« }\buchAbdrucke{\weitereDrucke{Arthur Schnitzler, Richard Beer-Hofmann: \emph{Briefwechsel 1891–1931}. Hg. Konstanze Fliedl. Wien, Zürich: \emph{Europaverlag} 1992, S. 69–70.} }\toendnotes[C]{\smallbreak}\pstart{}{\pb}\textcolor{gray}{\textbf{A}}n\pend{}\pstart{}Herrn D\textsuperscript{r} Arthur Schnitzler\pend{}\pstart{}\textcolor{pink}{Wien}{}\ledrightnote{\textcolor{pink}{Wien}}\pend{}\pstart{}\textcolor{pink}{IX Frankgasse 1}{}\ledrightnote{\textcolor{pink}{Frankgasse}}\pend{}\pstart{}\textcolor{pink}{Austria}{}\ledrightnote{\textcolor{pink}{Österreich}}\pend{}{\bigskip}\pstart
           \raggedleft{}{\pb}\textcolor{pink}{Venedig}{}\ledrightnote{\textcolor{pink}{Venedig}}. Sonntag
                     Abends\pend
           \pstart
           Lieber Arthur! Ihren Brief hab ich erhalten. Es ist wahrscheinlich
               daß ich schon Donnerstag in \textcolor{pink}{Wien}{}\ledrightnote{\textcolor{pink}{Wien}} bin
                  (\uline{Das ist aber njcht officiell}). Jedenfalls
               verständigen Sie mich in meine Wohnung was Donnerstag ist. Den kleinen
                  \textcolor{blue}{Andrian}{}\ledrightnote{\textcolor{blue}{Leopold von Andrian-Werburg}} hab ich hier getroffen. Herr \textcolor{blue}{Moritz Mayer}{}\ledrightnote{\textcolor{blue}{Moritz Mayer}} der Ihr »\textcolor{green}{Märchen}{}\ledrightnote{\textcolor{green}{Das Märchen. Schauspiel in drei Aufzügen}}« so hasst daß er hier wieder davon zu reden anfieng hebt die
                  »\textcolor{green}{Schmetterlingsschlacht}{}\ledrightnote{\textcolor{green}{Die Schmetterlingsschlacht}}« in den Hi{\geminationm}el. Das hat \label{T_L00394_1v}\edtext{ihr
                  noch}{\lemma{\textnormal{\emph{ihr
                  noch}}}\Cendnote{\textnormal{weiter am rechten Rand}}}\label{T_L00394_1h} gefehlt!
                  \spacefill\mbox{Richard}\pend
           \endnumbering\briefempfaengerindex{Schnitzler, Arthur@\textsc{Schnitzler, Arthur}!zzzBeer-Hofmann, Richard@\emph{von Richard Beer-Hofmann}!1894-10-281@{28. 10. 1894}|)be}\mylabel{h}  \normalsize

\doendnotes{C}
\bigskip
\vfill

\clearpage

\footnotesize

\lohead{\textsc{register}}

% Definiere theindex-Environment komplett neu ohne reledmac
\makeatletter
\renewenvironment{theindex}{%
  \section*{\indexname}%
  \setlength{\parindent}{0pt}%
  \setlength{\parskip}{0pt plus 0.3pt}%
  \let\item\@idxitem
}{%
  \clearpage
}
\makeatother

\IfFileExists{\jobname-pw.ind}{\input{\jobname-pw.ind}}{}

\end{document}

      