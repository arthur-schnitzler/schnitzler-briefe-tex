%% latex-korrekturansicht-vorspann.tex
%% Vorspann für die Korrekturansicht.
%% Lädt die gemeinsame Datei latex-vorspann.tex mit gesetztem Schalter.

\newif\ifkorrekturansicht
\korrekturansichttrue

\input{../tex-inputs/latex-vorspann}


               \section[Arthur Schnitzler an Georg Brandes, 3. 5. 1900]{ Arthur Schnitzler an Georg Brandes, 3. 5. 1900}\nopagebreak\mylabel{v}\rehead{ }\normalsize\beginnumbering\briefempfaengerindex{Brandes, Georg@\textsc{Brandes, Georg}!zzzSchnitzler, Arthur@\emph{von Arthur Schnitzler}!1900-05-031@{3. 5. 1900}|(be} \toendnotes[C]{\smallbreak\pagebreak[2]} \Standort{Kopenhagen, Det Kongelige Bibliotek, Georg Brandes Arkiv, box 125.}
\physDesc{Brief, 3 Blätter, 10 Seiten
\newline{}Handschrift: schwarze Tinte, deutsche Kurrent\newline{}Ordnung: auf der ersten Seite von unbekannter Hand mit Bleistift
                                 nummeriert: »20. \textsc{Schnitzler}« und datiert: »3/5 00«, die Datierung jeweils auf den ersten Seiten der weiteren
                                 Blätter mit Bleistift wiederholt, diesmal in Verbindung mit einem
                                 vorangestellten »?« }\buchAbdrucke{\weitereDrucke{1) Georg Brandes, Arthur Schnitzler: \emph{Ein Briefwechsel}. Hg. Kurt Bergel. Bern: \emph{Francke} 1956, S. 81–83.} \weitereDrucke{2) Arthur Schnitzler: \emph{Briefe 1875–1912}. Hg. Therese Nickl und Heinrich Schnitzler. Frankfurt am Main: \emph{S. Fischer} 1981, S. 382–384.} }\toendnotes[C]{\smallbreak}\pstart{}{\pb}Mein lieber und verehrter Herr
                  Brandes,\pend\pstart
           ſchon vor einigen Tagen las ich in einer Zeitung, daſs Sie ſich wieder \label{K_L01034_1v}\edtext{leidend befinden und in ein \textcolor{pink}{Sanatorium}{}\ledrightnote{→\textcolor{pink}{Kommunehospitalet}}}{\lemma{\textnormal{\emph{leidend … Sanatorium}}}\Cendnote{\textnormal{Vermutlich bezieht er sich auf diese
                  Meldung: [O. V.:] \emph{\textcolor{green}{Personal-Nachrichten.
                        [Dr. Georg Brandes]}}. In: \emph{\textcolor{green}{Neue Freie
                        Presse}}, Nr. 12811, 24. 4. 1900, S. 6:
                        »Dr. \textcolor{blue}{Georg \so{Brandes}}, dessen rheumatisches Leiden wieder heftiger aufgetreten ist, hat sich,
                     um eine so sachverständige und sorgfältige Behandlung als möglich zu finden, in
                     das \textcolor{pink}{Commune-Hospital in
                        Kopenhagen} begeben. Sein Zustand gibt nicht zu Besorgnissen
                     Anlaß.«}}}\label{K_L01034_1h} gegangen wären; aber nach dem ganzen Tun u auch nach der
               Schrift Ihres Briefes ſcheint mir, daſs die Krankheit diesmal leichter auftritt als
               die erſten Male, und hoffentlich ſtehn Sie bald wieder auf und ſind endlich ganz
               geſund. Es iſt gewiſs ein gutes Zeichen, wenn \label{K_L01034_2v}\edtext{Recidive}{\lemma{\textnormal{\emph{Recidive}}}\Cendnote{\textnormal{Rückfall}}}\label{K_L01034_2h} in abgeſchwächter Form auftreten; {\pb}ich wünſche von Herzen, daſs es das letzte iſt. –
               Sehr bedauert hab ich dſs ich in \textcolor{pink}{Abbazia}{}\ledrightnote{\textcolor{pink}{Opatija}} Ihren
               Abſagebrief fand nicht Sie ſelbſt. Ich habe auf der \textcolor{pink}{dalmatiniſchen}{}\ledrightnote{\textcolor{pink}{Dalmatien}} Reiſe meiſt ſchlechtes Wetter gehabt; nur in \textcolor{pink}{Raguſa}{}\ledrightnote{\textcolor{pink}{Dubrovnik}} zwei ſonnige Tage; überdies gerieth ich anfangs in einen
                  \label{K_L01034_3v}\edtext{Balneologen}{\lemma{\textnormal{\emph{Balneologen}}}\Cendnote{\textnormal{Balneologie: die Lehre von den
                  Heilbädern.}}}\label{K_L01034_3h}congreſs, deſſen Mitglieder Schiffe und Hotels füllten, von
               denen ich auch manche perſönlich kannte, es war ziemlich unangenehm. Unter ſolchen
                  Halb{\pb}bekannten ſein iſt die ſchli{\geminationm}ſte Form – der Einſamkeit, nicht der Geſelligkeit. Von
                  \textcolor{pink}{Abbazia}{}\ledrightnote{\textcolor{pink}{Opatija}} aus, wo es ununterbrochen regnete,
               flüchtete ich bald nach Hauſe. Das ſchönſte was ich mitbrachte, iſt die Eri{\geminationn}erung an die Trümmer von \textcolor{pink}{Salona}{}\ledrightnote{\textcolor{pink}{Solin}}, ich ka{\geminationn} gar nicht verſtehen, warum man
               da nicht immer und immer weitergräbt; die Erde wegkratzen und die Vergangenheit
               finden – wie ko{\geminationm}t es, dſs darüber noch keiner wahnſi{\geminationn}ig {\pb}geworden
               iſt? –\pend
           \pstart
           Auch die albernen Angriffe gegen Sie wegen Ihrer \label{K_L01034_4v}\edtext{\textcolor{pink}{Budapeſt}{}\ledrightnote{\textcolor{pink}{Budapest}}er Einleitung}{\lemma{\textnormal{\emph{Budapeſter Einleitung}}}\Cendnote{\textnormal{Möglicherweise bezieht sich Schnitzler auf diese Meldung:
                     [O. V.:] \emph{\textcolor{green}{Ein recht ungezogener Mensch}}.
                     In: \emph{\textcolor{green}{Arbeiter-Zeitung}}, Nr. 103,
                        15. 4. 1900, S. 6–7, hier S. 6: »Ein recht
                     ungezogener Mensch scheint Herr \textcolor{blue}{Georg \so{Brandes}}, der \textcolor{pink}{dänische} Literaturkritiker, zu
                     sein. Er hielt am letzten des vorigen Monats in einem \textcolor{pink}{Budapester Klub} einen Vortrag über \textcolor{blue}{Ibsen}. Da Herr \textcolor{blue}{Brandes} nicht \textcolor{pink}{ungarisch}{ }spricht, die \textcolor{pink}{Budapest}er aber wenig \textcolor{pink}{dänisch}
                     verstehen, so sprach Herr \textcolor{blue}{Brandes} –
                     natürlich \so{deutsch}. Er begann nun seine Rede mit
                     folgenden Worten: ›Meine Damen und Herren! Die Sprache, in der ich zu ihnen
                     rede, ist nicht die ihrige, und sie ist auch nicht die meine. Ich gestehe, \so{daß ich die deutsche Sprache nicht sehr liebe}; wie
                     ich weiß, ist sie \so{auch bei ihnen nicht sehr beliebt}.
                     Allein dieses einemal muß ich mich ihrer dennoch bedienen, denn schließlich ist
                     es doch die Hauptsache, daß wir einander verstehen. Ich habe das Deutsche erst
                     in meinem 30. Lebensjahr gelernt, und obwohl ich es vollkommen beherrsche, so
                     ist doch meine Aussprache mangelhaft. Deshalb ist es keine Phrase, wenn ich um
                     Nachsicht bitte.‹ Man braucht nicht viel Worte zu machen, um zu sagen, was das
                     ist, dessen sich Herr \textcolor{blue}{Brandes} hier schuldig
                     gemacht hat: eine \so{Unanständigkeit}. Niemand hat
                     weniger Anlaß, über das deutsche Volk Klage zu führen, wie Herr \textcolor{blue}{Brandes}, der in deutschen
                     Schriftstellerkreisen stets mit der größten Unbefangenheit und mit warmem
                     Wohlwollen aufgenommen worden ist. Es ist also eine Unziemlichkeit sehr arger
                     Art, wenn Herr \textcolor{blue}{Brandes}, der kurz vorher in
                        \textcolor{pink}{Wien} der deutschen Sprache so große
                     Komplimente gemacht hat, den deutschfresserischen Instinkten der \textcolor{pink}{Budapest}er Clique so niedrige Konzessionen
                     bereitet.«}}}\label{K_L01034_4h} habe ich geleſen. Es iſt ja wirklich gar nicht
               ernſthaft darüber zu reden. Und doch ſcheint es, ka{\geminationn} man
               die Empfindlichkeit gegenüber dem dü{\geminationm}ſten, we{\geminationn} es nur einmal gedruckt iſt, nicht ganz verlieren. Ich
               erinnere mich, wie ich ſeinerzeit mit einigem Staunen im Briefwechſel von \textcolor{blue}{Goethe}{}\ledrightnote{\textcolor{blue}{Johann Wolfgang von Goethe}} und \textcolor{blue}{Schiller}{}\ledrightnote{\textcolor{blue}{Friedrich von Schiller}} Denkmäler ihres Aergers über die nichtigſten Scribenten antraf.
               Seither ſtaune ich {\pb}aber nicht mehr, we{\geminationn} ich ſehe, wie ſich zuweilen die Klügſten über die
               Thörichteſten ärgern. Die Philoſophie hilft wohl gegen die Todesangſt, aber nicht
               gegen Flohſtiche.\pend
           \pstart
           Daſs Sie auch mir für \textcolor{pink}{Wien}{}\ledrightnote{\textcolor{pink}{Wien}} danken, iſt zu
               liebenswürdig; ich fühle, daſs ich Ihnen, beſonders diesmal, nicht viel ſein konnte.
               Im Anfang waren dieſe langweiligen Zahngeſchichten; und dann liegen die Schatten von
               jenem traurigen Ereignis oft, und nun gar in dieſen Frühlingstagen ſchwer auf meiner
               Seele. Dazu kommen noch mancherlei zum {\pb}Theil
               nervöſe Dinge (aber nur zum Theil), über die ich nicht gern rede, hauptſächlich ein
               quälendes Ohrenſauſen, an dem ich nun ſeit drei einhalb Jahren ununterbrochen leide,
               mit beginnender Verſchlechterung des Gehörs – das macht mich natürlich auch nicht
               viel froher. Immerhin arbeite ich ſeit einiger Zeit mehr als je und mit einer
               Empfindung – wenigſtens zuweilen – von innerer Fülle wie niemals früher. Ich bin
               jetzt daran eine \textcolor{green}{Novelle}{}\ledrightnote{→\textcolor{green}{Frau Bertha Garlan. Roman}} zu
               dictiren, die vor ein paar Wochen beendet wurde, ſchreibe jetzt einige {\pb}kleinere und möchte im Sommer eine Komödie
               ſchreiben. Der \textcolor{green}{Schleier der \textsc{Beatrice}}{}\ledrightnote{\textcolor{green}{Der Schleier der Beatrice. Schauspiel in fünf Akten}} wird wahrſcheinlich im \substVorne{}\textsuperscript{Sommer}{\allowbreak}\substDazwischen{}Herbſt\substHinten{} an der \textcolor{pink}{Burg}{}\ledrightnote{\textcolor{pink}{Burgtheater}} aufgeführt; wo ich aber mit den
               neuen Sachen hin ſoll die ich im Kopf habe weiſs ich nicht recht. Es wird nemlich
               kaum möglich ſein in der nächſten Zeit etwas \textcolor{pink}{wien}{}\ledrightnote{\textcolor{pink}{Wien}}eriſches zu ſchreiben, in das nicht die antiſemitiſche Frage hineinſpielt –
               und meine Art darüber zu denken wird weder den Chriſten noch den Juden recht ſein. –
               Das neue \textcolor{green}{Buch}{}\ledrightnote{→\textcolor{green}{Familiendramen}} von \textcolor{blue}{\textsc{Bour{\pb}get}}{}\ledrightnote{\textcolor{blue}{Paul Bourget}} ke{\geminationn} ich nicht, habe ſchon lange nicht von ihm
               geleſen; auch das \textcolor{green}{Reiſewerk}{}\ledrightnote{→\textcolor{green}{Rund um die Erde 1888–89}} von
                  \textcolor{blue}{\textsc{Lanckoronsky}}{}\ledrightnote{\textcolor{blue}{Karl Lanckoroński}} iſt mir noch unbekannt. Ich leſe jetzt – denken Sie! zum erſten Mal – we{\geminationn} ich von einer Jugendbearbeitung abſehe – den \textcolor{green}{\textsc{Don Quixote}}{}\ledrightnote{\textcolor{green}{Don Quijote}}; da{\geminationn} ein vorzügliches \textcolor{green}{Buch}{}\ledrightnote{→\textcolor{green}{Dante}} über \textcolor{blue}{\textsc{\uline{Dante}}}{}\ledrightnote{\textcolor{blue}{Dante Alighieri}} von \textcolor{blue}{\textsc{Federn}}{}\ledrightnote{\textcolor{blue}{Karl Federn}}, demſelben, der den \textcolor{blue}{\textsc{Emerson}}{}\ledrightnote{\textcolor{blue}{Ralph Waldo Emerson}} trefflich \textcolor{green}{überſetzt}{}\ledrightnote{→\textcolor{green}{Essays}} hat.
                  \textcolor{green}{\textcolor{blue}{\textsc{Gibbon}}{}\ledrightnote{\textcolor{blue}{Edward Gibbon}}}{}\ledrightnote{→\textcolor{green}{Verfall und Untergang des Römischen Reiches}} begleitet mich bereits längere Zeit.\pend
           \pstart
           Seit das Wetter ſchön iſt, radl ich auch manchmal aufs Land, und für den Sommer hab
               ich {\pb}größere Touren auf dem Rad vor. Vielleicht
               entſchließen Sie ſich einmal, in der heißen Zeit ins Gebirge zu gehen; ich habe mich
               ſchon darauf  gefreut,
               einmal mit Ihnen im Freien zu ſein, außerhalb von Stadt und Mauern herumzuſpaziren.
               Vielleicht läßt es ſich gar machen, dſs Sie, \textcolor{blue}{Goldmann}{}\ledrightnote{\textcolor{blue}{Paul Goldmann}} und \textcolor{blue}{Beer Hofma{\geminationn}}{}\ledrightnote{\textcolor{blue}{Richard Beer-Hofmann}} u ich irgendwo zuſammentreffen, fern von allen Zeitungen – und am Ende auch von
               aller »Lit\damage{\textcolor{gray}{era}}tur«. –\pend
           \pstart
           Jedenfalls hoff ich Sie ſagen mir bald wieder ein Wort, wies Ihnen {\pb}geht. Es iſt eine meiner wirklichen Freuden, daſs
               Sie meiner mit Sympathie gedenken. Ich grüße Sie herzlich.\pend
           \pstart Ihr \spacefill\mbox{Arthur Schnitzler}\pend{}\pstart
           \textcolor{pink}{Wien}{}\ledrightnote{\textcolor{pink}{Wien}}, 3. 5. 900.\pend
           \endnumbering\briefempfaengerindex{Brandes, Georg@\textsc{Brandes, Georg}!zzzSchnitzler, Arthur@\emph{von Arthur Schnitzler}!1900-05-031@{3. 5. 1900}|)be}\mylabel{h}  \normalsize

\doendnotes{C}
\bigskip
\vfill

\clearpage

\footnotesize

\lohead{\textsc{register}}

% Definiere theindex-Environment komplett neu ohne reledmac
\makeatletter
\renewenvironment{theindex}{%
  \section*{\indexname}%
  \setlength{\parindent}{0pt}%
  \setlength{\parskip}{0pt plus 0.3pt}%
  \let\item\@idxitem
}{%
  \clearpage
}
\makeatother

\IfFileExists{\jobname-pw.ind}{\input{\jobname-pw.ind}}{}

\end{document}

      