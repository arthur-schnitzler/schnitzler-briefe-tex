%% latex-korrekturansicht-vorspann.tex
%% Vorspann für die Korrekturansicht.
%% Lädt die gemeinsame Datei latex-vorspann.tex mit gesetztem Schalter.

\newif\ifkorrekturansicht
\korrekturansichttrue

\input{../tex-inputs/latex-vorspann}


               \section[Thomas Mann an Arthur Schnitzler, 25. 9. 1911]{ Thomas Mann an Arthur Schnitzler, 25. 9. 1911}\nopagebreak\mylabel{v}\rehead{ }\normalsize\beginnumbering\briefempfaengerindex{Schnitzler, Arthur@\textsc{Schnitzler, Arthur}!zzzMann, Thomas@\emph{von Thomas Mann}!1911-09-251@{25. 9.  1911}|(be} \toendnotes[C]{\smallbreak\pagebreak[2]} \Standort{CUL, Schnitzler, B 67.}
\physDesc{Brief, 1 Blatt, 4 Seiten
\newline{}Handschrift: schwarze Tinte, deutsche Kurrent
\newline{}Schnitzler: 1) mit Bleistift beschriftet: »Ma{\geminationn}« 2) mit rotem Buntstift zwei Unterstreichungen}\buchAbdrucke{\weitereDrucke{1) Hertha Krotkoff: \emph{Arthur Schnitzler – Thomas Mann: Briefe.} In: \emph{Modern Austrian Literature}, Jg. 7 (1974) Nr. 1/2, S. 14–15.} \weitereDrucke{2) Hans-Ulrich Lindken: \emph{Arthur Schnitzler. Aspekte und Akzente. Materialien zu Leben
                        und Werk}. Frankfurt am Main, Bern, Göttingen: \emph{Peter Lang} 1984, S. 196–197 (Europäische Hochschulschriften, Reihe 1, Deutsche Sprache und
                        Literatur, 754).} }\toendnotes[C]{\smallbreak}\pstart
           \noindent{}\raggedleft{}{\pb}\textcolor{gray}{\textbf{\textcolor{pink}{\textsc{Bad Tölz}}{}\ledrightnote{\textcolor{pink}{Bad Tölz}}\textsc{, den}}}{ }25. IX. 1911.\pend
           \pstart
           \noindent{}\raggedleft{}\textcolor{gray}{\textbf{\textcolor{pink}{LANDHAUS THOMAS MANN.}{}\ledrightnote{\textcolor{pink}{Thomas Mann Villa}}}}\pend
           \pstart{}Sehr verehrter Herr Doctor:\pend\pstart
           Durch meinen \textcolor{blue}{Bruder}{}\ledrightnote{→\textcolor{blue}{Heinrich Mann}}, der
                    zur Zeit bei uns wohnt, erfahre ich von dem Hinſcheiden Ihrer \textcolor{blue}{Mutter}{}\ledrightnote{→\textcolor{blue}{Louise Schnitzler}} und möchte Sie bitten, den
                    Ausdruck auch meiner herzlichen Teilnahme freundlichſt entgegenzunehmen.\pend
           \pstart
           Ich las mit großer Bewunderung Ihre ſo wunderbar gehobene \textcolor{green}{Dichtung}{}\ledrightnote{→\textcolor{green}{Die Hirtenflöte. Novelle}} in der »\textcolor{green}{Rundſchau}{}\ledrightnote{\textcolor{green}{Die neue Rundschau}}« und erwarte mit freudiger Ungeduld die \textcolor{pink}{Münchner}{}\ledrightnote{\textcolor{pink}{München}}{ }{\pb}\label{K_L02032_1v}\edtext{Erſtaufführung}{\lemma{\textnormal{\emph{Erſtaufführung}}}\Cendnote{\textnormal{Am 14. 10. 1911
                        fand die Uraufführung in mehreren Städten gleichzeitig statt.}}}\label{K_L02032_1h} Ihres
                    neuen \textcolor{green}{Stückes}{}\ledrightnote{→\textcolor{green}{Das weite Land. Tragikomödie in fünf Akten}}. Meinen \textcolor{blue}{Bruder}{}\ledrightnote{→\textcolor{blue}{Heinrich Mann}}{ }ſehe ich ſchwer verſtimmt – und bin es mit ihm
                    – über das Fehlſchlagen der Hoffnungen, die er auf sein \textcolor{green}{Drama}{}\ledrightnote{→\textcolor{green}{Schauspielerin}} geſetzt hatte. Ich habe es erſt
                    jetzt hier in der Korrektur geleſen und muß zum Mindeſten die Energie bewundern,
                    mit der ein an weit ausladender Breite gewöhnter \textcolor{blue}{Romancier}{}\ledrightnote{→\textcolor{blue}{Heinrich Mann}}{ }ſo viel Leidenſchaft und Schickſal in ein paar
                    knappe Dialoge zuſammenzupreſſen vermochte. Gewiß, die Theaterdirektoren thun
                        {\pb}höchſt Unrecht, das \textcolor{green}{Stück}{}\ledrightnote{→\textcolor{green}{Schauspielerin}} zurückzuweiſen! Es mag
                    ſein, daß die beiden ſpäteren Akte gegen den erſten an Bühnenwirkſamkeit
                    zurückſtehen, aber dichteriſch genommen bringen ſie die eindringlichſten Dinge,
                    und die ſchönſten Repliken ſind in ihnen enthalten. Und iſt es nicht ſchließlich
                    ſo, daß eine dramatiſche Arbeit dieſes \textcolor{blue}{Autors}{}\ledrightnote{→\textcolor{blue}{Heinrich Mann}} ohne Weiteres aufgeführt werden müßte? Wäre
                    das nicht eine ſelbſtverſtändliche Aufmerkſamkeit des Theaters gegen den \textcolor{blue}{Dichter}{}\ledrightnote{→\textcolor{blue}{Heinrich Mann}} der »\textcolor{green}{Kleinen Stadt}{}\ledrightnote{\textcolor{green}{Die kleine Stadt}}«? Entfällt da{\pb}bei für die Direktoren nicht jede
                    künſtleriſche Verantwortung? Hoffentlich erkennt nun wenigſtens Frau \textcolor{blue}{Durieux}{}\ledrightnote{\textcolor{blue}{Tilla Durieux}} in \textcolor{pink}{Berlin}{}\ledrightnote{\textcolor{pink}{Berlin}} in der \textcolor{green}{Leonie}{}\ledrightnote{→\textcolor{green}{Schauspielerin}} eine gute Rolle.\pend
           \pstart
           Mit den beſten Empfehlungen an Sie und Ihre \textcolor{blue}{Gattin}{}\ledrightnote{→\textcolor{blue}{Olga Schnitzler}}, {\\}ſehr verehrter Herr Doctor,\pend
           \pstart
           Ihr ergebenſter{\\[\baselineskip]}\spacefill\mbox{Thomas Mann.}\pend
           \leftskip=0em{}\endnumbering\briefempfaengerindex{Schnitzler, Arthur@\textsc{Schnitzler, Arthur}!zzzMann, Thomas@\emph{von Thomas Mann}!1911-09-251@{25. 9.  1911}|)be}\mylabel{h}  \normalsize

\doendnotes{C}
\bigskip
\vfill

\clearpage

\footnotesize

\lohead{\textsc{register}}

% Definiere theindex-Environment komplett neu ohne reledmac
\makeatletter
\renewenvironment{theindex}{%
  \section*{\indexname}%
  \setlength{\parindent}{0pt}%
  \setlength{\parskip}{0pt plus 0.3pt}%
  \let\item\@idxitem
}{%
  \clearpage
}
\makeatother

\IfFileExists{\jobname-pw.ind}{\input{\jobname-pw.ind}}{}

\end{document}

      