%% latex-korrekturansicht-vorspann.tex
%% Vorspann für die Korrekturansicht.
%% Lädt die gemeinsame Datei latex-vorspann.tex mit gesetztem Schalter.

\newif\ifkorrekturansicht
\korrekturansichttrue

\input{../tex-inputs/latex-vorspann}


               \section[Arthur Schnitzler an Richard Beer-Hofmann, 1. 2. 1892]{ Arthur Schnitzler an Richard Beer-Hofmann, 1. 2. 1892}\nopagebreak\mylabel{v}\rehead{ }\normalsize\beginnumbering\briefempfaengerindex{Beer-Hofmann, Richard@\textsc{Beer-Hofmann, Richard}!zzzSchnitzler, Arthur@\emph{von Arthur Schnitzler}!1892-02-014@{1. 2. 1892}|(be} \toendnotes[C]{\smallbreak\pagebreak[2]} \Standort{YCGL, MSS 31.}
\physDesc{Briefkarte, Umschlag
\newline{}Handschrift: 1) schwarze Tinte, deutsche Kurrent\hspace{1em}2) Bleistift, deutsche Kurrent (\noindent{}Umschlag)\hspace{1em}\newline{}Versand: 1) Stempel: »\nobreak{}Wien, 1. 2. 92, 5–6 N\nobreak{}«.  2) Stempel: »\nobreak{}Wien \textcolor{gray}{3}/2 45, 2-2 \textcolor{gray}{92}, 8, Bestellt\nobreak{}«. }\pstart{}{\pb}Herrn \textsc{Dr Richard
                     Beer-Hofmann}\pend{}\pstart{}\textsc{\textcolor{pink}{Wien}{}\ledrightnote{\textcolor{pink}{Wien}}}\pend{}\pstart{}\textsc{\textcolor{pink}{III. Seidlgasse 30}{}\ledrightnote{\textcolor{pink}{Seidlgasse}}}.\pend{}{\bigskip}\pstart
           \noindent{}{\pb}Lieber Richard. Morgen nichts. Eventuell ko{\geminationm} ich um 7 ins \textcolor{pink}{\textsc{Griensteidl}}{}\ledrightnote{\textcolor{pink}{Café Griensteidl}}. Gewiſs um 10. – Auch So{\geminationn}tag wird ſich nichts machen laſſen, weil da \textcolor{blue}{\textsc{Goldschmidt}}{}\ledrightnote{\textcolor{blue}{Adalbert von Goldschmidt}} iſt {\pb}u ich Abend eingeladen bin. Ich ſchrieb
               eben an \textcolor{blue}{\textsc{Loris}}{}\ledrightnote{\textcolor{blue}{Hugo von Hofmannsthal}} wegen eines Wochenabends, eventuell Samſtag\pend
           \pstart
           Herzlichſt Ihr{\\[\baselineskip]}\spacefill\mbox{Arthur}\pend
           \leftskip=0em{}\endnumbering\briefempfaengerindex{Beer-Hofmann, Richard@\textsc{Beer-Hofmann, Richard}!zzzSchnitzler, Arthur@\emph{von Arthur Schnitzler}!1892-02-014@{1. 2. 1892}|)be}\mylabel{h}  \normalsize

\doendnotes{C}
\bigskip
\vfill

\clearpage

\footnotesize

\lohead{\textsc{register}}

% Definiere theindex-Environment komplett neu ohne reledmac
\makeatletter
\renewenvironment{theindex}{%
  \section*{\indexname}%
  \setlength{\parindent}{0pt}%
  \setlength{\parskip}{0pt plus 0.3pt}%
  \let\item\@idxitem
}{%
  \clearpage
}
\makeatother

\IfFileExists{\jobname-pw.ind}{\input{\jobname-pw.ind}}{}

\end{document}

      