%% latex-korrekturansicht-vorspann.tex
%% Vorspann für die Korrekturansicht.
%% Lädt die gemeinsame Datei latex-vorspann.tex mit gesetztem Schalter.

\newif\ifkorrekturansicht
\korrekturansichttrue

\input{../tex-inputs/latex-vorspann}


               \section[Arthur Schnitzler an Felix Braun, 19. 4. 1918]{ Arthur Schnitzler an Felix Braun, 19. 4. 1918}\nopagebreak\mylabel{v}\rehead{ }\normalsize\beginnumbering\briefempfaengerindex{Braun, Felix@\textsc{Braun, Felix}!zzzSchnitzler, Arthur@\emph{von Arthur Schnitzler}!1918-04-191@{19. 4. 1918}|(be} \toendnotes[C]{\smallbreak\pagebreak[2]} \Standort{Wienbibliothek im Rathaus, H.I.N.-198045.}
\physDesc{Brief, 1 Blatt, 2 Seiten
\newline{}Schreibmaschine
\newline{}Handschrift: schwarze Tinte, deutsche Kurrent (\noindent{}Ergänzungen, Unterstreichungen
                                        und Unterschrift)}\Standort{DLA, A:Schnitzler, HS.1985.1.447.}
\physDesc{Brief, 2 Blätter, 2 Seiten, maschineller Durchschlag
\newline{}Schreibmaschine
\newline{}Handschrift: roter Buntstift, lateinische Kurrent (\noindent{}Beschriftung »Fel Braun«)}\toendnotes[C]{\smallbreak}\pstart
           {\pb}\textcolor{gray}{\textbf{Dr. Arthur Schnitzler}}\hfill 19. 4. 1918.\pend
           \pstart
           \textcolor{gray}{\textbf{\textcolor{pink}{Wien XVIII. Sternwartestrasse 71}{}\ledrightnote{\textcolor{pink}{Sternwartestraße}}}}\pend
           \pstart\center{}Verehrtester Herr Felix Braun.\pend\pstart
           Aus meinem Telegramm entnehmen Sie, dass meine Angelegenheit mit \textcolor{brown}{Fischer}{}\ledrightnote{\textcolor{brown}{S. Fischer Verlag}} noch immer in Schwebe ist. Es wäre immerhin doch
                    sehr möglich, dass er sich das nötige Papier \uline{sowohl} für meine alten \uline{als} für meine
                    neuen Sachen verschafft; und bei meinen persönlichen und geschäftlichen
                    Beziehungen zu ihm schiene es mir in keinem Sinne richtig, anderswo anzuknüpfen,
                    ehe ganz zwingende Gründe hiezu vorliegen. Darum ist es mir auch nicht möglich
                    Ihnen \introOben{}irgend\introOben{}welche \uline{Vorschläge} zu machen, sondern ich will mich vorläufig damit begnügen,
                        \strikeout{un} einige Anfragen an Sie zu stellen, durch
                    deren rasche Beantwortung Sie mich sehr verpflichten würden.\pend
           \pstart
           Innerhalb welcher Zeit und in wie viel Auflagen (zu tausend Exemplaren) könnte
                    der Verlag \textcolor{brown}{Müller}{}\ledrightnote{\textcolor{brown}{Georg Müller}} eine neue \textcolor{green}{Novelle}{}\ledrightnote{→\textcolor{green}{Casanovas Heimfahrt}} (Ausdehnung {\pb}etwa wie »\textcolor{green}{Badearzt Gräsler}{}\ledrightnote{\textcolor{green}{Doktor Gräsler, Badearzt}}« drucken und erscheinen lassen und zwar
                    unter der Bedingung vorheriger Bezahlung, \introOben{}von\introOben{} 25 {\%} des Ladenpreises\substVorne{}\textsuperscript{,}\substDazwischen{}?\substHinten{}\strikeout{und 2.} Ferner müsste ich mir das Recht
                    vorbehalten, diese \textcolor{green}{Novelle}{}\ledrightnote{→\textcolor{green}{Casanovas Heimfahrt}}
                    in einer Neuauflage meiner bei \textcolor{brown}{S. Fischer}{}\ledrightnote{\textcolor{brown}{S. Fischer Verlag}}
                    erscheinenden \textcolor{green}{gesammelten Werke}{}\ledrightnote{\textcolor{green}{Gesammelte Werke}}{ }\introOben{}(\introOben{}frühestens 1922\introOben{})\introOben{} auf\strikeout{zu}nehmen zu
                    dürfen.\pend
           \pstart
           Gleiches gälte für mein neues \textcolor{green}{Stück}{}\ledrightnote{→\textcolor{green}{Die Schwestern oder Casanova in Spa. Lustspiel in Versen}}, das jedenfalls erst im Spätherbst oder
                        Winter erscheinen sollte.\pend
           \pstart
           Es wird mir angenehm sein, recht bald Ihre Meinung zu vernehmen.\pend
           \pstart
           Mit verbindlichen Grüssen{\\[\baselineskip]}Ihr sehr ergebener{\\[\baselineskip]}\spacefill\mbox{{[}hs.:{]} Arthur Schnitzler}\pend
           \leftskip=0em{}\endnumbering\briefempfaengerindex{Braun, Felix@\textsc{Braun, Felix}!zzzSchnitzler, Arthur@\emph{von Arthur Schnitzler}!1918-04-191@{19. 4. 1918}|)be}\mylabel{h}  \normalsize

\doendnotes{C}
\bigskip
\vfill

\clearpage

\footnotesize

\lohead{\textsc{register}}

% Definiere theindex-Environment komplett neu ohne reledmac
\makeatletter
\renewenvironment{theindex}{%
  \section*{\indexname}%
  \setlength{\parindent}{0pt}%
  \setlength{\parskip}{0pt plus 0.3pt}%
  \let\item\@idxitem
}{%
  \clearpage
}
\makeatother

\IfFileExists{\jobname-pw.ind}{\input{\jobname-pw.ind}}{}

\end{document}

      