%% latex-korrekturansicht-vorspann.tex
%% Vorspann für die Korrekturansicht.
%% Lädt die gemeinsame Datei latex-vorspann.tex mit gesetztem Schalter.

\newif\ifkorrekturansicht
\korrekturansichttrue

\input{../tex-inputs/latex-vorspann}


               \section[Hermann Bahr an Arthur Schnitzler, 12. 2. 1907]{ Hermann Bahr an Arthur Schnitzler, 12. 2. 1907}\nopagebreak\mylabel{v}\rehead{ }\normalsize\beginnumbering\briefempfaengerindex{Schnitzler, Arthur@\textsc{Schnitzler, Arthur}!zzzBahr, Hermann@\emph{von Hermann Bahr}!1907-02-121@{12. 2. 1907}|(be} \toendnotes[C]{\smallbreak\pagebreak[2]} \Standort{CUL, Schnitzler, B 5b.}
\physDesc{Brief, 1 Blatt, 1 Seite
\newline{}Handschrift: schwarze Tinte, deutsche Kurrent\newline{}Ordnung: mit Bleistift von unbekannter Hand nummeriert: »144« }\buchAbdrucke{\weitereDrucke{Hermann Bahr, Arthur Schnitzler: \emph{Briefwechsel, Aufzeichnungen, Dokumente (1891–1931)}. Hg. Kurt Ifkovits und Martin Anton Müller. Göttingen: \emph{Wallstein} 2018, S. 389.} }\toendnotes[C]{\smallbreak}\pstart
           \raggedleft{}{\pb}\textcolor{pink}{Berlin NW 6 Marienstr 18}{}\ledrightnote{\textcolor{pink}{Marienstraße}}{\\}12. 2. 07\pend
           \pstart\center{}Lieber Artur!\pend\pstart
           Es iſt möglich, daß es mir gelingt, bei \textcolor{blue}{Reinhardt}{}\ledrightnote{\textcolor{blue}{Max Reinhardt}}
                  »\textcolor{green}{Liebelei}{}\ledrightnote{\textcolor{green}{Liebelei. Schauspiel in drei Akten}}« durchzuſetzen (\textcolor{blue}{Höflich}{}\ledrightnote{\textcolor{blue}{Lucie Höflich}}! \textcolor{blue}{Pagay}{}\ledrightnote{\textcolor{blue}{Hans Pagay}}!). Ich
               arbeite ſehr stark daran und dränge, es gleich nach \textcolor{green}{Hedda Gabler}{}\ledrightnote{\textcolor{green}{Hedda Gabler}} zu machen. Sicher iſt es noch gar nicht, Du darfſt auch noch zu
               keinem Menſchen was ſagen, ich möchte aber für alle Fälle raſcheſtens ein Buch haben,
               um mir meine Inſcenierung ruhiger zu überlegen, als es ſpäter geſchehen kann.\pend
           \pstart
           In größter Eile{\\[\baselineskip]}mir vielen Grüßen an Deine \textcolor{blue}{Frau}{}\ledrightnote{→\textcolor{blue}{Olga Schnitzler}}{\\[\baselineskip]}herzlichſt{\\[\baselineskip]}\spacefill\mbox{Hermann}\pend
           \leftskip=0em{}\endnumbering\briefempfaengerindex{Schnitzler, Arthur@\textsc{Schnitzler, Arthur}!zzzBahr, Hermann@\emph{von Hermann Bahr}!1907-02-121@{12. 2. 1907}|)be}\mylabel{h}  \normalsize

\doendnotes{C}
\bigskip
\vfill

\clearpage

\footnotesize

\lohead{\textsc{register}}

% Definiere theindex-Environment komplett neu ohne reledmac
\makeatletter
\renewenvironment{theindex}{%
  \section*{\indexname}%
  \setlength{\parindent}{0pt}%
  \setlength{\parskip}{0pt plus 0.3pt}%
  \let\item\@idxitem
}{%
  \clearpage
}
\makeatother

\IfFileExists{\jobname-pw.ind}{\input{\jobname-pw.ind}}{}

\end{document}

      