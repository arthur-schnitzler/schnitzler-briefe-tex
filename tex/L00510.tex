%% latex-korrekturansicht-vorspann.tex
%% Vorspann für die Korrekturansicht.
%% Lädt die gemeinsame Datei latex-vorspann.tex mit gesetztem Schalter.

\newif\ifkorrekturansicht
\korrekturansichttrue

\input{../tex-inputs/latex-vorspann}


               \section[Arthur Schnitzler an Richard Beer-Hofmann, {[}19. 10. 1895?{]}]{ Arthur Schnitzler an Richard Beer-Hofmann, {[}19. 10. 1895?{]}}\nopagebreak\mylabel{v}\rehead{ }\normalsize\beginnumbering\briefempfaengerindex{Beer-Hofmann, Richard@\textsc{Beer-Hofmann, Richard}!zzzSchnitzler, Arthur@\emph{von Arthur Schnitzler}!1895-10-192@{{[}19. 10. 1895?{]}}|(be} \toendnotes[C]{\smallbreak\pagebreak[2]} \Standort{YCGL, MSS 31.}
\physDesc{Briefkarte, Umschlag
\newline{}Handschrift: Bleistift, deutsche Kurrent\newline{}Versand: ohne postalischen Übermittlungsvermerk }\pstart{}{\pb}\textsc{Herrn Dr. Richard
                     Beer-Hofmann}\pend{}\pstart{}\textcolor{pink}{Wien}{}\ledrightnote{\textcolor{pink}{Wien}}.\pend{}{\bigskip}\pstart
           \noindent{}{\pb}lieber Richard, zwiſchen
                  6 u 7 werd ich heute kaum ins \textsc{\textcolor{pink}{Gr.}{}\ledrightnote{\textcolor{pink}{Café Griensteidl}}} ko{\geminationm}en
               können. Erkältet bin ich mehr als Sie, werde aber – halt, da fällt mir eben ein, daß
               ich nach 8 Zeit habe. Ich werde zwiſchen 8 u
                  ½ 9 bei {\pb}Ihnen anläuten,
               vielleicht ſoupiren wir zuſa{\geminationm}en, oder laſſen Sie mir
               eine Poſt.\pend
           \pstart
           Herzlich der Ihre{\\[\baselineskip]}\spacefill\mbox{Arth}\pend
           \leftskip=0em{}\endnumbering\briefempfaengerindex{Beer-Hofmann, Richard@\textsc{Beer-Hofmann, Richard}!zzzSchnitzler, Arthur@\emph{von Arthur Schnitzler}!1895-10-192@{{[}19. 10. 1895?{]}}|)be}\mylabel{h}  \normalsize

\doendnotes{C}
\bigskip
\vfill

\clearpage

\footnotesize

\lohead{\textsc{register}}

% Definiere theindex-Environment komplett neu ohne reledmac
\makeatletter
\renewenvironment{theindex}{%
  \section*{\indexname}%
  \setlength{\parindent}{0pt}%
  \setlength{\parskip}{0pt plus 0.3pt}%
  \let\item\@idxitem
}{%
  \clearpage
}
\makeatother

\IfFileExists{\jobname-pw.ind}{\input{\jobname-pw.ind}}{}

\end{document}

      