%% latex-korrekturansicht-vorspann.tex
%% Vorspann für die Korrekturansicht.
%% Lädt die gemeinsame Datei latex-vorspann.tex mit gesetztem Schalter.

\newif\ifkorrekturansicht
\korrekturansichttrue

\input{../tex-inputs/latex-vorspann}


               \section[Max Burckhard an Arthur Schnitzler, {[}4. 11. 1894{]}]{ Max Burckhard an Arthur Schnitzler, {[}4. 11. 1894{]}}\nopagebreak\mylabel{v}\rehead{ }\normalsize\beginnumbering\briefempfaengerindex{Schnitzler, Arthur@\textsc{Schnitzler, Arthur}!zzzBurckhard, Max Eugen@\emph{von Max Eugen Burckhard}!1894-11-041@{{[}4. 11. 1894{]}}|(be} \toendnotes[C]{\smallbreak\pagebreak[2]} \Standort{CUL, Schnitzler, B 20.}
\physDesc{Briefkarte
\newline{}Handschrift: schwarze Tinte, deutsche Kurrent
\newline{}Schnitzler: mit Bleistift datiert: »4/11 94« \newline{}Ordnung: mit Bleistift von unbekannter Hand nummeriert:
                                 »3« }\toendnotes[C]{\smallbreak}\pstart{}{\pb}Sehr geehrter Herr Doctor!\pend\pstart
           Könnten Sie mir heute 1 Uhr im Bureau oder morgen ſo circa
                  3 Uhr in der \label{K_L00396_1v}\edtext{Wohnung}{\lemma{\textnormal{\emph{Wohnung}}}\Cendnote{\textnormal{\textcolor{blue}{Burckhard} ist im Adressverzeichnis \emph{\textcolor{green}{Lehmann}} von 1890 bis
                     1905 in der \textcolor{pink}{Frankgasse 1}
                     gelistet. \textcolor{blue}{Schnitzler} wohnt vom 15. 11. 1893 bis zum 11. 9. 1903 im selben Haus,
                  einen Stock tiefer.}}}\label{K_L00396_1h} das Vergnügen Ihres Beſuches machen?\pend
           \pstart
           Mit beſten Empfehlungen{\\[\baselineskip]}\spacefill\mbox{D\textsuperscript{r}Burckhard}\pend
           \leftskip=0em{}\endnumbering\briefempfaengerindex{Schnitzler, Arthur@\textsc{Schnitzler, Arthur}!zzzBurckhard, Max Eugen@\emph{von Max Eugen Burckhard}!1894-11-041@{{[}4. 11. 1894{]}}|)be}\mylabel{h}  \normalsize

\doendnotes{C}
\bigskip
\vfill

\clearpage

\footnotesize

\lohead{\textsc{register}}

% Definiere theindex-Environment komplett neu ohne reledmac
\makeatletter
\renewenvironment{theindex}{%
  \section*{\indexname}%
  \setlength{\parindent}{0pt}%
  \setlength{\parskip}{0pt plus 0.3pt}%
  \let\item\@idxitem
}{%
  \clearpage
}
\makeatother

\IfFileExists{\jobname-pw.ind}{\input{\jobname-pw.ind}}{}

\end{document}

      