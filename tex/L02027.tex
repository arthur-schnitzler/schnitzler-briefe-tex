%% latex-korrekturansicht-vorspann.tex
%% Vorspann für die Korrekturansicht.
%% Lädt die gemeinsame Datei latex-vorspann.tex mit gesetztem Schalter.

\newif\ifkorrekturansicht
\korrekturansichttrue

\input{../tex-inputs/latex-vorspann}


               \section[Hugo von Hofmannsthal an Arthur Schnitzler, 11. 9. 1911]{ Hugo von Hofmannsthal an Arthur Schnitzler, 11. 9. 1911}\nopagebreak\mylabel{v}\rehead{ }\normalsize\beginnumbering\briefempfaengerindex{Schnitzler, Arthur@\textsc{Schnitzler, Arthur}!zzzHofmannsthal, Hugo von@\emph{von Hugo von Hofmannsthal}!1911-09-111@{11. 9. 1911}|(be} \toendnotes[C]{\smallbreak\pagebreak[2]} \Standort{CUL, Schnitzler, B 43.}
\physDesc{Brief, 1 Blatt, 4 Seiten
\newline{}Handschrift: schwarze Tinte, deutsche Kurrent
\newline{}Schnitzler: mit Bleistift die Jahreszahl ergänzt: »911« und beschriftet: »\textsc{Hugo}« \newline{}Ordnung: 1) mit Bleistift von unbekannter Hand nummeriert: »\strikeout{323}« 2) mit Bleistift von unbekannter Hand nummeriert: »332«}\buchAbdrucke{\weitereDrucke{Hugo von Hofmannsthal, Arthur Schnitzler: \emph{Briefwechsel}. Hg. Therese Nickl und Heinrich Schnitzler. Frankfurt am Main: \emph{S. Fischer} 1964, S. 262.} }\toendnotes[C]{\smallbreak}\pstart
           \raggedleft{}{\pb}\textcolor{pink}{\textsc{Aussee}}{}\ledrightnote{\textcolor{pink}{Bad Aussee}}, 11. IX.\pend
           \pstart{}mein lieber Arthur \pend\pstart
           die traurige Nachricht fand ich, nach einigen trüben Andeutungen durch Freunde, heute
               morgens in der Zeitung – ſo war es unmöglich, zurechtzukommen, um dem \label{K_L02027_1v}\edtext{Begräbnis}{\lemma{\textnormal{\emph{Begräbnis}}}\Cendnote{\textnormal{Dieses fand an eben
                  diesem Tag, dem 11. 9. 1911 statt.}}}\label{K_L02027_1h} Ihrer guten \textcolor{blue}{Mutter}{}\ledrightnote{→\textcolor{blue}{Louise Schnitzler}} beizuwohnen.\hspace*{1.5em}Daſs jemand nicht mehr iſt, iſt auch für den Fernerſtehenden
               unfaſsbar, ja es iſt, als antwortete das menſchliche Innere {\pb}auf die Zumutung, dies
               hinzunehmen, mit einer verdoppelten Lebhaftigkeit der Vorſtellung. So lebt Ihre \textcolor{blue}{Mutter}{}\ledrightnote{→\textcolor{blue}{Louise Schnitzler}} für mich in dieſen
               Stunden – und immer wieder, nach 10 nach 15, nach 20 Jahren kommt für mich ein
               einſamer Spaziergang, eine ſtockende Arbeitsſtunde, in der ein Todter ſo völlig
               auflebt, dies iſt eines der Geheimniſſe unseres Innern.\pend
           \pstart
           Es iſt mir ein lieber Gedanke, daſs Sie nach der Qual dieſer Tage daran {\pb}gehen, ein \textcolor{green}{dichteriſches Gebilde}{}\ledrightnote{→\textcolor{green}{Das weite Land. Tragikomödie in fünf Akten}}, in dem ſo viel Ihres ſtärkſten
               wahrſten inneren Lebens zuſammengedrängt iſt, auf die Bühne {[}zu{]}
               bringen. Daſs man auf dieſe Weiſe, ebenſo wie in den \textcolor{blue}{Kindern}{}\ledrightnote{→\textcolor{blue}{Christiane von Hofmannsthal}{\newline}→\textcolor{blue}{Raimund von Hofmannsthal}{\newline}→\textcolor{blue}{Franz von Hofmannsthal}}, irgend etwas von ſich
               weitergibt, gleichſam ans Unendliche weitergibt, iſt für mich eine von den
               Compenſationen. Es gibt noch geheimnisvollere, wenn man in das Myſterium des Lebens
               eindringt, wie es manchmal geſtattet, aber {\pb}nicht mitteilbar iſt. In den
               Tiefen der Arbeit liegen ſie und auch in den Tiefen des \substVorne{}\textsuperscript{A}\substDazwischen{}a\substHinten{}ufnehmenden Lebens, und ſind Ihnen bekannt wie mir. – Es ſcheint mir in
               manchen Momenten als das einzig Natürliche, jetzt zu Ihnen zu fahren und Tage bei
               Ihnen zu ſein. Ich thäte es augenblicklich, wären Sie auf dem Lande, wo ich wirklich
               andauernd bei Ihnen wäre.\pend
           \pstart
           Auch hält mich noch etwas zurück. Mein \textcolor{blue}{Vater}{}\ledrightnote{→\textcolor{blue}{Hugo August von Hofmannsthal}} war dieſen ganzen ſchweren So{\geminationm}er in \textcolor{pink}{Wien}{}\ledrightnote{\textcolor{pink}{Wien}},
               iſt jetzt bei uns und freut ſich auf eine kleine aufheiternde Reiſe nach \textcolor{pink}{Hamburg}{}\ledrightnote{\textcolor{pink}{Hamburg}} u. \textcolor{pink}{Kopenhagen}{}\ledrightnote{\textcolor{pink}{Kopenhagen}}, \label{T_L02027_1v}\edtext{der ich auch meine Herbſtarbeitswochen
               zunächſt opfere. Wir treten ſie am 16\textsuperscript{ten}}{\lemma{\textnormal{\emph{der … 16ten}}}\Cendnote{\textnormal{quer am linken
                  Rand der letzten Seite}}}\label{T_L02027_1h}{ }\label{T_L02027_2v}\edtext{von \textcolor{pink}{München}{}\ledrightnote{\textcolor{pink}{München}} aus an}{\lemma{\textnormal{\emph{von München aus an}}}\Cendnote{\textnormal{weiter quer am rechten Rand der
                  letzten Seite}}}\label{T_L02027_2h}.\pend
           \pstart Von Herzen Ihr\spacefill\mbox{Hugo.}\pend{}\endnumbering\briefempfaengerindex{Schnitzler, Arthur@\textsc{Schnitzler, Arthur}!zzzHofmannsthal, Hugo von@\emph{von Hugo von Hofmannsthal}!1911-09-111@{11. 9. 1911}|)be}\mylabel{h}  \normalsize

\doendnotes{C}
\bigskip
\vfill

\clearpage

\footnotesize

\lohead{\textsc{register}}

% Definiere theindex-Environment komplett neu ohne reledmac
\makeatletter
\renewenvironment{theindex}{%
  \section*{\indexname}%
  \setlength{\parindent}{0pt}%
  \setlength{\parskip}{0pt plus 0.3pt}%
  \let\item\@idxitem
}{%
  \clearpage
}
\makeatother

\IfFileExists{\jobname-pw.ind}{\input{\jobname-pw.ind}}{}

\end{document}

      