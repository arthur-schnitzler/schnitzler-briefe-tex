%% latex-korrekturansicht-vorspann.tex
%% Vorspann für die Korrekturansicht.
%% Lädt die gemeinsame Datei latex-vorspann.tex mit gesetztem Schalter.

\newif\ifkorrekturansicht
\korrekturansichttrue

\input{../tex-inputs/latex-vorspann}


               \section[Arthur Schnitzler an Hugo von Hofmannsthal, 27. 3. 1911]{ Arthur Schnitzler an Hugo von Hofmannsthal, 27. 3. 1911}\nopagebreak\mylabel{v}\rehead{ }\normalsize\beginnumbering\briefempfaengerindex{Hofmannsthal, Hugo von@\textsc{Hofmannsthal, Hugo von}!zzzSchnitzler, Arthur@\emph{von Arthur Schnitzler}!1911-03-271@{27. 3. 1911}|(be} \toendnotes[C]{\smallbreak\pagebreak[2]} \Standort{FDH, Hs-30885,3.}
\physDesc{Briefkarte
\newline{}Handschrift: schwarze Tinte, deutsche Kurrent}\Standort{FDH, Hs-30885,143.}
\physDesc{Briefkarte, Fotokopie}\buchAbdrucke{\weitereDrucke{Hugo von Hofmannsthal, Arthur Schnitzler: \emph{Briefwechsel}. Hg. Therese Nickl und Heinrich Schnitzler. Frankfurt am Main: \emph{S. Fischer} 1964, S. 261.} }\toendnotes[C]{\smallbreak}\pstart
           \noindent{}{\pb}\textcolor{gray}{\textbf{Dr. Arthur Schnitzler}}\hfill 27. 3. 911\pend
           \pstart
           \textcolor{gray}{\textbf{\textcolor{pink}{Wien XVIII. Sternwartestrasse 71}{}\ledrightnote{\textcolor{pink}{Sternwartestraße}}}}\pend
           \pstart
           lieber Hugo, auch wir ſind am 8. April nicht mehr
                    in \textcolor{pink}{Wien}{}\ledrightnote{\textcolor{pink}{Wien}}; fahren am 5. od.
                        6. zuerſt nach \textcolor{pink}{München}{}\ledrightnote{\textcolor{pink}{München}} (\textcolor{pink}{\textsc{Partenkirchen}}{}\ledrightnote{\textcolor{pink}{Partenkirchen}}, zu \textcolor{blue}{\textsc{Lisl}}{}\ledrightnote{\textcolor{blue}{Elisabeth Steinrück}},) dann vorausſichtlich weiter nach \textcolor{pink}{\textsc{Genua}}{}\ledrightnote{\textcolor{pink}{Genua}}, \textcolor{pink}{\textsc{Mentone}}{}\ledrightnote{\textcolor{pink}{Menton}}, auf 2–3 Wochen.\pend
           \pstart
           Ich hatte die \textcolor{green}{Première}{}\ledrightnote{→\textcolor{green}{Der Rosenkavalier}} für
                    früher angeno{\geminationm}en. {\pb}Es
                    thut uns leid, ſie nicht \label{K_L02015_1v}\edtext{mitmachen}{\lemma{\textnormal{\emph{mitmachen}}}\Cendnote{\textnormal{Die Abreise
                        verzögert sich bis zum 10. 4. 1911, er nimmt aber trotzdem
                        nicht teil.}}}\label{K_L02015_1h} zu kö{\geminationn}en.\pend
           \pstart
           Schönen Dank und Gruß.{\\[\baselineskip]}Ihr{\\[\baselineskip]}\spacefill\mbox{Arthur}\pend
           \leftskip=0em{}\endnumbering\briefempfaengerindex{Hofmannsthal, Hugo von@\textsc{Hofmannsthal, Hugo von}!zzzSchnitzler, Arthur@\emph{von Arthur Schnitzler}!1911-03-271@{27. 3. 1911}|)be}\mylabel{h}  \normalsize

\doendnotes{C}
\bigskip
\vfill

\clearpage

\footnotesize

\lohead{\textsc{register}}

% Definiere theindex-Environment komplett neu ohne reledmac
\makeatletter
\renewenvironment{theindex}{%
  \section*{\indexname}%
  \setlength{\parindent}{0pt}%
  \setlength{\parskip}{0pt plus 0.3pt}%
  \let\item\@idxitem
}{%
  \clearpage
}
\makeatother

\IfFileExists{\jobname-pw.ind}{\input{\jobname-pw.ind}}{}

\end{document}

      