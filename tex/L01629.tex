%% latex-korrekturansicht-vorspann.tex
%% Vorspann für die Korrekturansicht.
%% Lädt die gemeinsame Datei latex-vorspann.tex mit gesetztem Schalter.

\newif\ifkorrekturansicht
\korrekturansichttrue

\input{../tex-inputs/latex-vorspann}


               \section[Arthur Schnitzler an Hermann Bahr, 17. 9. 1906]{ Arthur Schnitzler an Hermann Bahr, 17. 9. 1906}\nopagebreak\mylabel{v}\rehead{ }\normalsize\beginnumbering\briefempfaengerindex{Bahr, Hermann@\textsc{Bahr, Hermann}!zzzSchnitzler, Arthur@\emph{von Arthur Schnitzler}!1906-09-171@{17. 9. 1906}|(be} \toendnotes[C]{\smallbreak\pagebreak[2]} \Standort{TMW, HS AM 60119 Ba.}
\physDesc{Bildpostkarte
\newline{}Handschrift: Bleistift, deutsche Kurrent\newline{}Versand: Stempel: »\nobreak{}\oindex{Semmering@\textbf{Semmering}, \emph{Besiedelter Ort (A.BSO)}|pwk}Semmering, 17. IX{[}. 06{]}\nobreak{}«.  \newline{}Ordnung: Lochung }\buchAbdrucke{\weitereDrucke{Hermann Bahr, Arthur Schnitzler: \emph{Briefwechsel, Aufzeichnungen, Dokumente (1891–1931)}. Hg. Kurt Ifkovits und Martin Anton Müller. Göttingen: \emph{Wallstein} 2018, S. 381.} }\toendnotes[C]{\smallbreak}\pstart{}{\pb}Hr \textsc{Hermann Bahr}\pend{}\pstart{}\textcolor{pink}{Wien XIII}{}\ledrightnote{\textcolor{pink}{XIII., Hietzing}}\pend{}\pstart{}\textcolor{pink}{\textsc{Ober St Veit}}{}\ledrightnote{\textcolor{pink}{Ober Sankt Veit}}\pend{}\pstart{}\textsc{\textcolor{pink}{Veitlissengasse}{}\ledrightnote{\textcolor{pink}{Veitlissengasse}}}\pend{}{\bigskip}\pstart
           \noindent{}\centering{}\textcolor{gray}{\textbf{{\pb}\textcolor{pink}{Südbahnhotel Semmering}{}\ledrightnote{\textcolor{pink}{Südbahnhotel}}}}\pend
           \pstart
           {\pb}lieber Hermann,
               das \label{K_L01629_1v}\edtext{\textsc{\textcolor{green}{Mscrpt}{}\ledrightnote{→\textcolor{green}{Ringelspiel}}}}{\lemma{\textnormal{\emph{Mscrpt}}}\Cendnote{\textnormal{zu \emph{\textcolor{green}{Das
                     Ringelspiel}}}}}\label{K_L01629_1h} haſt du hoffentlich rechtzeitig erhalten. Herzlichen Dank – beſonders für die
               erſten 2 Akte. Gegen den 3. hab ich viel auf dem Herzen.\pend
           \pstart
           Sind Ende der Woche daheim – wir ſehn uns doch beſti{\geminationm}t
               vor deiner \label{K_L01629_2v}\edtext{Abreiſe}{\lemma{\textnormal{\emph{Abreiſe}}}\Cendnote{\textnormal{Ursprünglich meinte \textcolor{blue}{Bahr}
                  sein Engagement als Regisseur bei \textcolor{blue}{Max
                     Reinhardt} mit Anfang Oktober anzutreten, es sollte aber erst
                  einen Monat später beginnen.}}}\label{K_L01629_2h}?\pend
           \pstart
           Dein{\\[\baselineskip]}\spacefill\mbox{A. S.}\pend
           \leftskip=0em{}\pstart
           \noindent{}\label{T_L01629_1v}\edtext{17/9 906}{\lemma{\textnormal{\emph{17/9 906}}}\Cendnote{\textnormal{seitlich am Textrand}}}\label{T_L01629_1h}\pend
           \endnumbering\briefempfaengerindex{Bahr, Hermann@\textsc{Bahr, Hermann}!zzzSchnitzler, Arthur@\emph{von Arthur Schnitzler}!1906-09-171@{17. 9. 1906}|)be}\mylabel{h}  \normalsize

\doendnotes{C}
\bigskip
\vfill

\clearpage

\footnotesize

\lohead{\textsc{register}}

% Definiere theindex-Environment komplett neu ohne reledmac
\makeatletter
\renewenvironment{theindex}{%
  \section*{\indexname}%
  \setlength{\parindent}{0pt}%
  \setlength{\parskip}{0pt plus 0.3pt}%
  \let\item\@idxitem
}{%
  \clearpage
}
\makeatother

\IfFileExists{\jobname-pw.ind}{\input{\jobname-pw.ind}}{}

\end{document}

      