%% latex-korrekturansicht-vorspann.tex
%% Vorspann für die Korrekturansicht.
%% Lädt die gemeinsame Datei latex-vorspann.tex mit gesetztem Schalter.

\newif\ifkorrekturansicht
\korrekturansichttrue

\input{../tex-inputs/latex-vorspann}


               \section[Arthur Schnitzler an Richard Beer-Hofmann, {[}30. 7.?{]} 1893]{ Arthur Schnitzler an Richard Beer-Hofmann, {[}30. 7.?{]} 1893}\nopagebreak\mylabel{v}\rehead{ }\normalsize\beginnumbering\briefempfaengerindex{Beer-Hofmann, Richard@\textsc{Beer-Hofmann, Richard}!zzzSchnitzler, Arthur@\emph{von Arthur Schnitzler}!1893-07-301@{{[}30. 7.?{]} 1893}|(be} \toendnotes[C]{\smallbreak\pagebreak[2]} \Standort{YCGL, MSS 31.}
\physDesc{Telegramm
\newline{}maschinell\newline{}Versand: mit Bleistift Eintragung am Vordruck: »\noindent{}\textcolor{gray}{\textbf{Aufgenommen von}} Wi06{ / }\textcolor{gray}{\textbf{auf Leitung Nr. ..........}}{ / }\textcolor{gray}{\textbf{am}}{ }\textcolor{gray}{30/7} 93{ }\textcolor{gray}{\textbf{um}}{ }21\textcolor{gray}{\textbf{Uhr}}{ }{\dots}\textcolor{gray}{\textbf{Min.}}{ }Vor\textcolor{gray}{\textbf{Mittag}}« }\toendnotes[C]{\smallbreak}\pstart{}{\pb}richard beer hofmann \textcolor{pink}{ischl}{}\ledrightnote{\textcolor{pink}{Bad Ischl}}\pend{}\pstart{}\textcolor{pink}{schulgasze 8}{}\ledrightnote{\textcolor{pink}{Schulgasse}}\pend{}{\bigskip}\pstart
           \noindent{}{\pb}\textcolor{pink}{ischl}{}\ledrightnote{\textcolor{pink}{Bad Ischl}} fr \textcolor{pink}{wien}{}\ledrightnote{\textcolor{pink}{Wien}}{ }10+1166{ }20{ }1+\pend
           \pstart
           \textcolor{blue}{abschreiber}{}\ledrightnote{→\textcolor{blue}{?? [Schreibkraft für Arthur Schnitzler]}} brachte trotz
               wiederholten draengens die \textcolor{green}{novelle}{}\ledrightnote{→\textcolor{green}{Das Kind}} heute nicht, morgen sicher\pend
           \pstart herzlichen grusz \spacefill\mbox{= arthur =}\pend{}\endnumbering\briefempfaengerindex{Beer-Hofmann, Richard@\textsc{Beer-Hofmann, Richard}!zzzSchnitzler, Arthur@\emph{von Arthur Schnitzler}!1893-07-301@{{[}30. 7.?{]} 1893}|)be}\mylabel{h}  \normalsize

\doendnotes{C}
\bigskip
\vfill

\clearpage

\footnotesize

\lohead{\textsc{register}}

% Definiere theindex-Environment komplett neu ohne reledmac
\makeatletter
\renewenvironment{theindex}{%
  \section*{\indexname}%
  \setlength{\parindent}{0pt}%
  \setlength{\parskip}{0pt plus 0.3pt}%
  \let\item\@idxitem
}{%
  \clearpage
}
\makeatother

\IfFileExists{\jobname-pw.ind}{\input{\jobname-pw.ind}}{}

\end{document}

      