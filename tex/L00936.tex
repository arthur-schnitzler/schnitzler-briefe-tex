%% latex-korrekturansicht-vorspann.tex
%% Vorspann für die Korrekturansicht.
%% Lädt die gemeinsame Datei latex-vorspann.tex mit gesetztem Schalter.

\newif\ifkorrekturansicht
\korrekturansichttrue

\input{../tex-inputs/latex-vorspann}


               \section[Hugo von Hofmannsthal an Arthur Schnitzler, 7. 7. {[}1899{]}]{ Hugo von Hofmannsthal an Arthur Schnitzler, 7. 7. {[}1899{]}}\nopagebreak\mylabel{v}\rehead{ }\normalsize\beginnumbering\briefempfaengerindex{Schnitzler, Arthur@\textsc{Schnitzler, Arthur}!zzzHofmannsthal, Hugo von@\emph{von Hugo von Hofmannsthal}!1899-07-071@{7. 7. {[}1899{]}}|(be} \toendnotes[C]{\smallbreak\pagebreak[2]} \Standort{CUL, Schnitzler, B 43.}
\physDesc{Brief, 1 Blatt, 4 Seiten
\newline{}Handschrift: schwarze Tinte, deutsche Kurrent
\newline{}Schnitzler: mit Bleistift die Jahreszahl ergänzt: »99« \newline{}Ordnung: 1) mit Bleistift von unbekannter Hand nummeriert:
                                        »150« 2) mit Bleistift von unbekannter Hand nummeriert: »\strikeout{153}«}\buchAbdrucke{\weitereDrucke{1) Hugo von Hofmannsthal: \emph{Briefe. 1890–1901}. Berlin: \emph{S. Fischer} 1935, S. 286–287.} \weitereDrucke{2) Hugo von Hofmannsthal, Arthur Schnitzler: \emph{Briefwechsel}. Hg. Therese Nickl und Heinrich Schnitzler. Frankfurt am Main: \emph{S. Fischer} 1964, S. 124–125.} }\toendnotes[C]{\smallbreak}\pstart
           \raggedleft{}{\pb}7 VII.\pend
           \pstart
           Bin ſehr froh endlich zu wiſſen, wo Sie ſind, denn ſelbſt darüber in Ungewiſsheit
                    zu ſein, iſt peinlich. Von \textcolor{blue}{Richard}{}\ledrightnote{\textcolor{blue}{Richard Beer-Hofmann}} hab ich
                    nach wie vor keine Zeile.\pend
           \pstart
           Der »\textcolor{brown}{Zeit}{}\ledrightnote{\textcolor{brown}{Die Zeit. Wiener Wochenschrift}}« ſtelle ich meinen Namen in
                    unverbindlicher Weiſe natürlich gern zur Verfügung. Habe an einem \textcolor{green}{Stück}{}\ledrightnote{→\textcolor{green}{Das Bergwerk zu Falun}} (5 Acte, in Verſen) {\pb}zu arbeiten begonnen, bin
                    aber gleich in den Anfängen durch ganz unglaubliches deprimierendes Wetter
                    gehemmt worden.\pend
           \pstart
           Bleibe wohl bis gegen Ende July hier und werde dann, hoffentlich
                    mitten in der Arbeit, wohl nach \textcolor{pink}{Salzburg}{}\ledrightnote{\textcolor{pink}{Salzburg}}
                    überſiedeln. Gegen Ende Auguſt hoffe ich die innere {\pb}und äußere Möglichkeit zu
                    einer kleinen \textcolor{pink}{deutſchen}{}\ledrightnote{\textcolor{pink}{Deutschland}} Tour zu finden.\pend
           \pstart
           \textcolor{blue}{Minnie}{}\ledrightnote{\textcolor{blue}{Hermine von Schaffgotsch}}{ }ſehe ich ungefähr täglich ¼ –
                    ½ Stunde. Das Geſpräch entfernt ſich nie vom peinlich-banalen. Sie thut mir
                    recht leid. Es kommt etwas tief Freudloſes und Bitteres in ihr Weſen.\hspace*{1.5em}Sind Sie wenigſtens {\pb}einigermaßen im Stand ſich
                    mit \textcolor{green}{Stück}{}\ledrightnote{→\textcolor{green}{Der Schleier der Beatrice. Schauspiel in fünf Akten}} oder \textcolor{green}{Novelle}{}\ledrightnote{→\textcolor{green}{Die Nächste}} zu beſchäftigen?\pend
           \pstart
           Herzlich Ihr{\\[\baselineskip]}\spacefill\mbox{Hugo.}\pend
           \leftskip=0em{}\pstart
           \noindent{}\textsc{P. S.}{ }\uline{Giebt} es ein Leben zweiter oder dritter
                        Ordnung? Auf die Dauer doch wohl kaum.\pend
           \endnumbering\briefempfaengerindex{Schnitzler, Arthur@\textsc{Schnitzler, Arthur}!zzzHofmannsthal, Hugo von@\emph{von Hugo von Hofmannsthal}!1899-07-071@{7. 7. {[}1899{]}}|)be}\mylabel{h}  \normalsize

\doendnotes{C}
\bigskip
\vfill

\clearpage

\footnotesize

\lohead{\textsc{register}}

% Definiere theindex-Environment komplett neu ohne reledmac
\makeatletter
\renewenvironment{theindex}{%
  \section*{\indexname}%
  \setlength{\parindent}{0pt}%
  \setlength{\parskip}{0pt plus 0.3pt}%
  \let\item\@idxitem
}{%
  \clearpage
}
\makeatother

\IfFileExists{\jobname-pw.ind}{\input{\jobname-pw.ind}}{}

\end{document}

      