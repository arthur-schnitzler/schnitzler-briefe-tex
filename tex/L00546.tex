%% latex-korrekturansicht-vorspann.tex
%% Vorspann für die Korrekturansicht.
%% Lädt die gemeinsame Datei latex-vorspann.tex mit gesetztem Schalter.

\newif\ifkorrekturansicht
\korrekturansichttrue

\input{../tex-inputs/latex-vorspann}


               \section[Arthur Schnitzler an Hugo von Hofmannsthal, 23. 5. 1896]{ Arthur Schnitzler an Hugo von Hofmannsthal, 23. 5. 1896}\nopagebreak\mylabel{v}\rehead{ }\normalsize\beginnumbering\briefempfaengerindex{Hofmannsthal, Hugo von@\textsc{Hofmannsthal, Hugo von}!zzzSchnitzler, Arthur@\emph{von Arthur Schnitzler}!1896-05-231@{23. 5. 1896}|(be} \toendnotes[C]{\smallbreak\pagebreak[2]} \Standort{FDH, Hs-30885,49.}
\physDesc{Brief, 1 Blatt, 4 Seiten
\newline{}Handschrift: schwarze Tinte, deutsche Kurrent}\buchAbdrucke{\weitereDrucke{Hugo von Hofmannsthal, Arthur Schnitzler: \emph{Briefwechsel}. Hg. Therese Nickl und Heinrich Schnitzler. Frankfurt am Main: \emph{S. Fischer} 1964, S. 66–67.} }\toendnotes[C]{\smallbreak}\pstart
           \raggedleft{}{\pb}\textcolor{pink}{Wien}{}\ledrightnote{\textcolor{pink}{Wien}}, 23. 5. 96.\pend
           \pstart
           Mein lieber Hugo, ich freue mich ſehr daſs Sie ſich meiner
                    erinnert haben u noch mehr, daſs Sie bald zurückko{\geminationm}en. Im Juni wollen wir dann doch noch ein paar Mal zuſa{\geminationm}en ſein. Und das eine Mal von den paar werde ich
                    wohl das \textcolor{green}{Stück}{}\ledrightnote{→\textcolor{green}{Freiwild. Schauspiel in 3 Akten}} vorleſen
                    können. Ich habe jetzt mehr Zuverſicht. Aber mit meinem ganzen Herzen bin ich
                    doch nicht dabei. Vielleicht iſt das ſogar gut: vielleicht {\pb}iſt es ein Fehler von vielen meiner Sachen, daſs ich
                    mit ihnen im Schreiben zu zärtlich geworden bin.\pend
           \pstart
           Ihren \textcolor{green}{Artikel über Poeſie und
                        Leben}{}\ledrightnote{→\textcolor{green}{Poesie und Leben. Aus einem Vortrage}} hab\textcolor{gray}{e} ich als ein ſchönes Gedicht empfunden; aber es kam mir vor,
                    als we{\geminationn} Sie die Grenzen der Poeſie zu eng gezogen
                    hätten, während es doch Ihre Abſicht war, ſie zu erweitern. Woher eigentlich
                    dieſes ſonderbare Bedürfnis kommt, über Kunſt zu reden. Ich ſelbſt fühl es
                    manchmal, und {\pb}habe nachher i{\geminationm}er oder oft das Gefühl etwas überflüſſiges oder
                    gar unrechtes gethan \introOben{}zu\introOben{} haben. Es ko{\geminationm}t besti{\geminationm}t \uline{nicht allein} daher, daſs das Theoretiſiren einfach
                    meinem Weſen nicht entſpricht. Und meine Sehnſucht, ins Klare zu kommen, iſt
                    gewiſs auch nicht gering. Und was \textcolor{blue}{Goethe}{}\ledrightnote{\textcolor{blue}{Johann Wolfgang von Goethe}},
                        \textcolor{blue}{Leſſing}{}\ledrightnote{\textcolor{blue}{Gotthold Ephraim Lessing}}, \textcolor{blue}{Hebbel}{}\ledrightnote{\textcolor{blue}{Friedrich Hebbel}}, was Sie und andre über Kunſt ſagen, leſe ich gern; manches
                    beruhigt mich, indem es abſchließt, andres bewegt {\pb}mich, indem es Thore aufſchließt. Wir ſprechen einmal darüber.\pend
           \pstart
           \textcolor{blue}{\textsc{Brahm}}{}\ledrightnote{\textcolor{blue}{Otto Brahm}} iſt jetzt da, den ich perſönlich gern
                    habe. Geſtern Abend waren er, \textcolor{blue}{Richard}{}\ledrightnote{\textcolor{blue}{Richard Beer-Hofmann}}, \textcolor{blue}{Salten}{}\ledrightnote{\textcolor{blue}{Felix Salten}} u. \textcolor{blue}{Schwarzkopf}{}\ledrightnote{\textcolor{blue}{Gustav Schwarzkopf}} bei mir. – Geleſen hab ich die \textcolor{green}{Frzſ. Revol.}{}\ledrightnote{\textcolor{green}{Die Revolution}} von \textcolor{blue}{\textsc{Taine}}{}\ledrightnote{\textcolor{blue}{Hippolyte Taine}}, die \textcolor{green}{Olla potrida des durchtriebenen
                        Fuchsmundi}{}\ledrightnote{\textcolor{green}{Ollapatrida des durchgetriebenen Fuchsmundi}}, die \textcolor{green}{Noten zum Divan}{}\ledrightnote{\textcolor{green}{West-östlicher Divan}} und
                    einen engliſchen \label{K_L00546_1v}\edtext{\textcolor{green}{Kriminalroman}{}\ledrightnote{→\textcolor{green}{?? [Englischer Kriminalroman]}}}{\lemma{\textnormal{\emph{Kriminalroman}}}\Cendnote{\textnormal{nicht identifiziert}}}\label{K_L00546_1h}. – Mein
               So{\geminationm}erplan iſt jetzt \textcolor{pink}{Norwegen}{}\ledrightnote{\textcolor{pink}{Norwegen}}, \textcolor{pink}{Schweden}{}\ledrightnote{\textcolor{pink}{Schweden}}, \textcolor{pink}{Dänemark}{}\ledrightnote{\textcolor{pink}{Dänemark}}; und eine \textcolor{green}{Novelle}{}\ledrightnote{→\textcolor{green}{Die Frau des Weisen. Erzählung}}. – Jetzt iſt ein Gewitter mit
                    Blitz und Donner und Abend geh ich zum »\textcolor{green}{Zerriſſenen}{}\ledrightnote{\textcolor{green}{Der Zerrissene}}«.\pend
           \pstart Herzlich der Ihre, \spacefill\mbox{AS.}\pend{}\endnumbering\briefempfaengerindex{Hofmannsthal, Hugo von@\textsc{Hofmannsthal, Hugo von}!zzzSchnitzler, Arthur@\emph{von Arthur Schnitzler}!1896-05-231@{23. 5. 1896}|)be}\mylabel{h}  \normalsize

\doendnotes{C}
\bigskip
\vfill

\clearpage

\footnotesize

\lohead{\textsc{register}}

% Definiere theindex-Environment komplett neu ohne reledmac
\makeatletter
\renewenvironment{theindex}{%
  \section*{\indexname}%
  \setlength{\parindent}{0pt}%
  \setlength{\parskip}{0pt plus 0.3pt}%
  \let\item\@idxitem
}{%
  \clearpage
}
\makeatother

\IfFileExists{\jobname-pw.ind}{\input{\jobname-pw.ind}}{}

\end{document}

      