%% latex-korrekturansicht-vorspann.tex
%% Vorspann für die Korrekturansicht.
%% Lädt die gemeinsame Datei latex-vorspann.tex mit gesetztem Schalter.

\newif\ifkorrekturansicht
\korrekturansichttrue

\input{../tex-inputs/latex-vorspann}


               \section[Arthur Schnitzler an Richard Beer-Hofmann, 15. 10. 1894]{ Arthur Schnitzler an Richard Beer-Hofmann, 15. 10. 1894}\nopagebreak\mylabel{v}\rehead{ }\normalsize\beginnumbering\briefempfaengerindex{Beer-Hofmann, Richard@\textsc{Beer-Hofmann, Richard}!zzzSchnitzler, Arthur@\emph{von Arthur Schnitzler}!1894-10-151@{15. 10. 1894}|(be} \toendnotes[C]{\smallbreak\pagebreak[2]} \Standort{YCGL, MSS 31.}
\physDesc{Brief, 2 Blätter (Briefpapier mit Trauerrand), 7 Seiten, Umschlag
\newline{}Handschrift: schwarze Tinte, deutsche Kurrent\newline{}Versand: 1) nachgesandt nach \textcolor{pink}{\textsc{Hotel Hassler}} 2) Stempel: »\nobreak{}\oindex{I., Innere Stadt@\textbf{I., Innere Stadt}, \emph{Bezirk (A.BZK)}|pwk}Wien 1/1, 15. 10. 94, 11\textcolor{gray}{–12}N\nobreak{}«. 3) Stempel: »\nobreak{}\oindex{Neapel@\textbf{Neapel}, \emph{Besiedelter Ort (A.BSO)}|pwk}Napoli, \textcolor{gray}{7} 10–94, 8 S\nobreak{}«. }\buchAbdrucke{\weitereDrucke{1) Arthur Schnitzler: \emph{Briefe 1875–1912}. Hg. Therese Nickl und Heinrich Schnitzler. Frankfurt am Main: \emph{S. Fischer} 1981, S. 231.} \weitereDrucke{2) Arthur Schnitzler, Richard Beer-Hofmann: \emph{Briefwechsel 1891–1931}. Hg. Konstanze Fliedl. Wien, Zürich: \emph{Europaverlag} 1992, S. 63–64.} }\toendnotes[C]{\smallbreak}\pstart{}{\pb}\textsc{Dr. Arthur Schnitzler}, \textcolor{pink}{Wien,
                     IX. Frankgaſſe 1.}{}\ledrightnote{\textcolor{pink}{Frankgasse}}\pend{}{\bigskip}\pstart{}{\pb}Herrn \textsc{Dr. Richard
                     Beer-Hofmann}\pend{}\pstart{}\textsc{\textcolor{pink}{Neapel}{}\ledrightnote{\textcolor{pink}{Neapel}}}\pend{}\pstart{}\textsc{(\textcolor{pink}{Napoli}{}\ledrightnote{\textcolor{pink}{Neapel}})}\pend{}\pstart{}\textsc{a poste ferma}\pend{}\pstart{}\textsc{\textcolor{pink}{Italien}{}\ledrightnote{\textcolor{pink}{Italien}}}\pend{}{\bigskip}\pstart
           \raggedleft{}{\pb}\textcolor{pink}{Wien}{}\ledrightnote{\textcolor{pink}{Wien}}, \uline{15. Oct. 94}.\pend
           \pstart
           Lieber Richard – Sie würden es nicht verdienen, daſs man Ihnen
               ſchreibt – aber ich nehme an, Sie empfinden den Empfang eines Briefs von mir nicht
               als Glück – alſo – Sie verstehen ja dieses \label{K_L00382_1v}\edtext{linke Ohr}{\lemma{\textnormal{\emph{linke Ohr}}}\Cendnote{\textnormal{»Pollack,
                  wo hast Du Dein linkes Ohr?« – Stehende Redewendung für den Griff mit der rechten
                  Hand über den Kopf zum linken Ohr. Ein (jüdischer) Junge, vom Lehrer gefragt, wo
                  er sein linkes Ohr habe, soll diese umständliche Geste gemacht haben. Vgl. Richard Beer-Hofmann an Arthur Schnitzler, 22. 2. 1900}}}\label{K_L00382_1h}? –\pend
           \pstart
           {\pb}Gestern hab ich dem \textcolor{blue}{Hugo}{}\ledrightnote{\textcolor{blue}{Hugo von Hofmannsthal}} und \textcolor{blue}{Salten}{}\ledrightnote{\textcolor{blue}{Felix Salten}} mein \textcolor{green}{Stück}{}\ledrightnote{→\textcolor{green}{Liebelei. Schauspiel in drei Akten}} vorgeleſen, – mit einem von mir nicht
               geahnten Erfolg. Es ſollen nur ein paar Wendungen drin zu ändern und ſonſt ſoll es
               ganz fertig ſein – das übrige Lob ſchäm ich mich beizufügen. Ich bin aber ſehr
               froh. – Momentan ſchreib ich {\pb}einen \textcolor{green}{Einakter}{}\ledrightnote{→\textcolor{green}{Die Frau mit dem Dolche}}. (15. Jahrhundert – aber es iſt
               eigentlich eine Fälſchung.) –\pend
           \pstart
           Es iſt läppiſch, daſs Sie mir ſo gut wie gar nichts ſchreiben. Ich ſage läppiſch, in
               der Ueberzeugung dſs das Sie viel mehr beleidigt als infam oder schurkiſch, was man
               auch ſagen könnte. – \textcolor{blue}{Hugo}{}\ledrightnote{\textcolor{blue}{Hugo von Hofmannsthal}}{ }ſieht als Dragoner {\pb}ausgezeichnet aus. Ein \textsc{Oberlieutn}. zum andern: »Du, ich
               hör, du haſt in deiner Abthlg einen, der Trauerſpiel dicht’ –?« –\pend
           \pstart
           \textcolor{blue}{\textsc{Salten}}{}\ledrightnote{\textcolor{blue}{Felix Salten}}, hab ich Ihnen das ſchon geſchrieben?, – ist in der Redaction der \textcolor{brown}{allgem. Zeitung}{}\ledrightnote{\textcolor{brown}{Wiener Allgemeine Zeitung}}. – Neulich hat er den \textcolor{blue}{\textsc{Suderma{\geminationn}}}{}\ledrightnote{\textcolor{blue}{Hermann Sudermann}}{ }\label{K_L00382_2v}\edtext{\textcolor{green}{\textsc{interviewt}}{}\ledrightnote{→\textcolor{green}{Bei Hermann Sudermann}}}{\lemma{\textnormal{\emph{interviewt}}}\Cendnote{\textnormal{\textcolor{blue}{–x.–n.}: \emph{\textcolor{green}{Bei
                        Hermann Sudermann}}. In: \emph{\textcolor{green}{Wiener Allgemeine
                        Zeitung}}, Nr. 4977, 13. 10. 1894, S. 2–3.}}}\label{K_L00382_2h},
               und der kleine \textcolor{blue}{Kraus}{}\ledrightnote{\textcolor{blue}{Karl Kraus}} erklärt das für unerhört
               charakterlos.\pend
           \pstart
           {\pb}Wünſchen Sie auch von \textcolor{blue}{\textsc{Fels}}{}\ledrightnote{\textcolor{blue}{Friedrich Michael Fels}} was zu wiſſen? Ich zweifle nicht daran. Alſo: alles beim alten; – was Sie ſchon
               merken werden, wenn Sie zurückko{\geminationm}en. – Wünſchen Sie was
               von \textcolor{blue}{\textsc{Korff}}{}\ledrightnote{\textcolor{blue}{Heinrich von Korff}} zu wiſſen? Er hat eine \textcolor{blue}{Hebamme}{}\ledrightnote{→\textcolor{blue}{Irma von Korff}} geheiratet, welche aber kaum 15 Jahre älter iſt als er. – Und \textcolor{blue}{\textsc{Specht}}{}\ledrightnote{\textcolor{blue}{Richard Specht}}? – Er fährt nächſtens {\pb}auf ein Jahr nach \textcolor{pink}{\textsc{Liverpool}}{}\ledrightnote{\textcolor{pink}{Liverpool}}. Und \textcolor{blue}{\textsc{Paul von Schönthan}}{}\ledrightnote{\textcolor{blue}{Richard Specht}}? Er wünſcht ſehnlichſt, Sie zum \textcolor{brown}{Saubermann}{}\ledrightnote{\textcolor{brown}{Saubermänner}} zu
               geſtalten. – Neulich hab ich den \textcolor{blue}{\textsc{Julian Sternberg}}{}\ledrightnote{\textcolor{blue}{Julian Sternberg}} (den bei dem Sie ſich ſo einzuſchmeicheln »gewußt« haben) ke{\geminationn}en gelernt; da hat er mir ſehr gut gefallen. –\pend
           \pstart
           {\pb}Außerdem regnets, iſt kalt, und der Winter iſt
               da. –\pend
           \pstart
           Leben Sie wohl und ſchreiben Sie einem doch wenigſtens endlich einmal, wann man sie
               »wieder haben« wird.\pend
           \pstart
           Herzlich der Ihre{\\[\baselineskip]}\spacefill\mbox{Arthur}\pend
           \leftskip=0em{}\pstart
           \noindent{}»\textcolor{brown}{Zeit}{}\ledrightnote{\textcolor{brown}{Die Zeit. Wiener Wochenschrift}}« wird beſorgt. Sie iſt \uline{ſehr} gut\pend
           \endnumbering\briefempfaengerindex{Beer-Hofmann, Richard@\textsc{Beer-Hofmann, Richard}!zzzSchnitzler, Arthur@\emph{von Arthur Schnitzler}!1894-10-151@{15. 10. 1894}|)be}\mylabel{h}  \normalsize

\doendnotes{C}
\bigskip
\vfill

\clearpage

\footnotesize

\lohead{\textsc{register}}

% Definiere theindex-Environment komplett neu ohne reledmac
\makeatletter
\renewenvironment{theindex}{%
  \section*{\indexname}%
  \setlength{\parindent}{0pt}%
  \setlength{\parskip}{0pt plus 0.3pt}%
  \let\item\@idxitem
}{%
  \clearpage
}
\makeatother

\IfFileExists{\jobname-pw.ind}{\input{\jobname-pw.ind}}{}

\end{document}

      