%% latex-korrekturansicht-vorspann.tex
%% Vorspann für die Korrekturansicht.
%% Lädt die gemeinsame Datei latex-vorspann.tex mit gesetztem Schalter.

\newif\ifkorrekturansicht
\korrekturansichttrue

\input{../tex-inputs/latex-vorspann}


               \section[Arthur Schnitzler an Richard Beer-Hofmann, 31. 10. 1904]{ Arthur Schnitzler an Richard Beer-Hofmann, 31. 10. 1904}\nopagebreak\mylabel{v}\rehead{ }\normalsize\beginnumbering\briefempfaengerindex{Beer-Hofmann, Richard@\textsc{Beer-Hofmann, Richard}!zzzSchnitzler, Arthur@\emph{von Arthur Schnitzler}!1904-10-311@{31. 10. 1904}|(be} \toendnotes[C]{\smallbreak\pagebreak[2]} \Standort{YCGL, MSS 31.}
\physDesc{Telegramm
\newline{}Handschrift einer Schreibkraft: Bleistift, deutsche Kurrent\newline{}Versand: »\noindent{}\textcolor{gray}{\textbf{Gattung des Telegrammes.}} p{ / }\textcolor{gray}{\textbf{Aufg\damage{egeben am}}}{ }31/X \textcolor{gray}{\textbf{190}}{\dots}{ }\textcolor{gray}{\textbf{um}}{ }\textcolor{gray}{X} \textcolor{gray}{\textbf{Uhr {\dots} Min {\dots} Mittag}}{ / }\textcolor{gray}{\textbf{Eingelangt von}} W.40 \textcolor{pink}{Wien}{ }\textcolor{gray}{\textbf{auf Leitung Nr. {\dots} am}}{ }31/X \textcolor{gray}{\textbf{190{\dots}}}{ }\textcolor{gray}{\textbf{um}}{ }XII \textcolor{gray}{\textbf{Uhr}} – \textcolor{gray}{\textbf{Min.}}{ }\textcolor{gray}{v} \textcolor{gray}{\textbf{Mittag}}{ / }\textcolor{gray}{\textbf{Aufgenommen durch}}{ }\textcolor{gray}{RM}{ / }\textcolor{gray}{\textbf{Von}}{ }\textcolor{pink}{Wien 111}{ }\textcolor{gray}{\textbf{Aufgabe-Nr.}} 902{ }\textcolor{gray}{\textbf{mit}} 21 \textcolor{gray}{\textbf{Taxworten ({\dots} Worten {\dots} Chiffern)}}« \newline{}Ordnung: mit Bleistift von unbekannter Hand datiert: »31. 10. 1904« }\pstart{}{\pb}Richard Beer Hofma{\geminationn}\pend{}\pstart{}\textcolor{pink}{Lieſingſtraſſe}{}\ledrightnote{\textcolor{pink}{Liesingerstraße}}\pend{}\pstart{}\textcolor{gray}{\textbf{\textit{\textcolor{pink}{Rodaun}{}\ledrightnote{\textcolor{pink}{Rodaun}}}}}\pend{}{\bigskip}\pstart
           \noindent{}{\pb}Freue mich ſehr Sie heute Abends zu ſehn \textcolor{blue}{Olga}{}\ledrightnote{\textcolor{blue}{Olga Schnitzler}} ko{\geminationm}t mit \textcolor{blue}{Paula}{}\ledrightnote{\textcolor{blue}{Paula Beer-Hofmann}} hoffentlich auch\pend
           \pstart
           Herzlichſt{\\[\baselineskip]}\spacefill\mbox{Arthur}\pend
           \leftskip=0em{}\endnumbering\briefempfaengerindex{Beer-Hofmann, Richard@\textsc{Beer-Hofmann, Richard}!zzzSchnitzler, Arthur@\emph{von Arthur Schnitzler}!1904-10-311@{31. 10. 1904}|)be}\mylabel{h}  \normalsize

\doendnotes{C}
\bigskip
\vfill

\clearpage

\footnotesize

\lohead{\textsc{register}}

% Definiere theindex-Environment komplett neu ohne reledmac
\makeatletter
\renewenvironment{theindex}{%
  \section*{\indexname}%
  \setlength{\parindent}{0pt}%
  \setlength{\parskip}{0pt plus 0.3pt}%
  \let\item\@idxitem
}{%
  \clearpage
}
\makeatother

\IfFileExists{\jobname-pw.ind}{\input{\jobname-pw.ind}}{}

\end{document}

      