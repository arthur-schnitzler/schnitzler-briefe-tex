%% latex-korrekturansicht-vorspann.tex
%% Vorspann für die Korrekturansicht.
%% Lädt die gemeinsame Datei latex-vorspann.tex mit gesetztem Schalter.

\newif\ifkorrekturansicht
\korrekturansichttrue

\input{../tex-inputs/latex-vorspann}


               \section[Arthur Schnitzler an Richard Beer-Hofmann, 17. 8. 1901]{ Arthur Schnitzler an Richard Beer-Hofmann, 17. 8. 1901}\nopagebreak\mylabel{v}\rehead{ }\normalsize\beginnumbering\briefempfaengerindex{Beer-Hofmann, Richard@\textsc{Beer-Hofmann, Richard}!zzzSchnitzler, Arthur@\emph{von Arthur Schnitzler}!1901-08-171@{17. 8. 1901}|(be} \toendnotes[C]{\smallbreak\pagebreak[2]} \Standort{YCGL, MSS 31.}
\physDesc{Brief, 1 Blatt, 3 Seiten, Umschlag
\newline{}Handschrift: Bleistift, deutsche Kurrent\newline{}Versand: 1) Stempel: »\nobreak{}\oindex{Welsberg-Taisten@\textbf{Welsberg-Taisten}, \emph{Besiedelter Ort (A.BSO)}|pwk}Welsberg, 17. 8. 01\nobreak{}«.  2) Stempel: »\nobreak{}\oindex{Wildbad Waldbrunn@\textbf{Wildbad Waldbrunn}, \emph{Hotel (K.HTL)}|pwk}{\pb}Grand Hôtel Wildbad
                              Waldbrunn Pusterthal, 17 AUG 19\textcolor{gray}{01}\nobreak{}«. 3) Stempel: »\nobreak{}\oindex{Poertschach@\textbf{Pörtschach}, \emph{https://www.geonames.org/ontologyP.PPL}|pwk}Pörtschach {[}am See{]}, 18 {[}8 01{]}\nobreak{}«. }\buchAbdrucke{\weitereDrucke{Arthur Schnitzler, Richard Beer-Hofmann: \emph{Briefwechsel 1891–1931}. Hg. Konstanze Fliedl. Wien, Zürich: \emph{Europaverlag} 1992, S. 154–155.} }\toendnotes[C]{\smallbreak}\pstart{}{\pb}\textsc{Dr. Richard Beer-Hofma{\geminationn}}\pend{}\pstart{}\textsc{\textcolor{pink}{Pörtschach}{}\ledrightnote{\textcolor{pink}{Pörtschach}}}\pend{}\pstart{}\textsc{\textcolor{pink}{Villa Arnstein}{}\ledrightnote{\textcolor{pink}{Villa Arnstein}}.}\pend{}{\bigskip}\pstart
           \raggedleft{}{\pb}\textcolor{pink}{\textsc{Welsberg, Waldbrunn}}{}\ledrightnote{\textcolor{pink}{Wildbad Waldbrunn}}{\\}17. 8. 901\pend
           \pstart
           mein lieber Richard, ſeit vorgeſtern bin ich hier u finde es
               unverſtändlich, dſs dieſer Ort nicht populärer iſt: \textcolor{pink}{\textsc{Waldbrunn}}{}\ledrightnote{\textcolor{pink}{Wildbad Waldbrunn}} liegt eine ¼ Std höher als \textcolor{pink}{\textsc{Welsberg}}{}\ledrightnote{\textcolor{pink}{Welsberg-Taisten}}, hat einen ſchönen Ausblick und gleich hinter dem Hotel (Penſion 3.50 alles
               wirklich gut) herrlichen Wald. \textcolor{blue}{Paul}{}\ledrightnote{\textcolor{blue}{Paul Goldmann}} iſt noch am
                  \textcolor{pink}{Gardaſee}{}\ledrightnote{\textcolor{pink}{Lago di Garda}} und ko{\geminationm}t
               morgen. Es hätte keinen Sinn, wenn Sie nur auf ein paar Stunden {\pb}kämen; würden Sie ſich aber zu einem längern
               Aufenthalt (6–8 Tage) entſchließen, ſo würde ich auch meinen Aufenthalt verlängern.
               Unter andern Umſtänden führe ich in etwa 10 Tagen von hier ab; ich würde Sie dann in
                  \textcolor{pink}{Pörtſchach}{}\ledrightnote{\textcolor{pink}{Pörtschach}} beſuchen (mit \textcolor{blue}{Paul}{}\ledrightnote{\textcolor{blue}{Paul Goldmann}} denk ich) oder wir treffen uns in \textcolor{pink}{Villach}{}\ledrightnote{\textcolor{pink}{Villach}}? {\pb}Aber das weitaus
               ſympathiſcheſte wäre doch, we{\geminationn} Sie hieherkämen, die
               beiden jungen \textcolor{blue}{Damen}{}\ledrightnote{→\textcolor{blue}{Olga Schnitzler}{\newline}→\textcolor{blue}{Elisabeth Steinrück}}, die
               mit mir zugleich hier ſind, würden Sie gewiſs nicht ſtören.\pend
           \pstart
           Jedenfalls ſchreiben Sie mir gleich ein Wort hieher.\pend
           \pstart
           Von \textcolor{blue}{\textsc{Kerr}}{}\ledrightnote{\textcolor{blue}{Alfred Kerr}} hab ich keine Nachricht.\pend
           \pstart
           Von Herzen{\\[\baselineskip]}Ihr\spacefill\mbox{Arthur}\pend
           \leftskip=0em{}\endnumbering\briefempfaengerindex{Beer-Hofmann, Richard@\textsc{Beer-Hofmann, Richard}!zzzSchnitzler, Arthur@\emph{von Arthur Schnitzler}!1901-08-171@{17. 8. 1901}|)be}\mylabel{h}  \normalsize

\doendnotes{C}
\bigskip
\vfill

\clearpage

\footnotesize

\lohead{\textsc{register}}

% Definiere theindex-Environment komplett neu ohne reledmac
\makeatletter
\renewenvironment{theindex}{%
  \section*{\indexname}%
  \setlength{\parindent}{0pt}%
  \setlength{\parskip}{0pt plus 0.3pt}%
  \let\item\@idxitem
}{%
  \clearpage
}
\makeatother

\IfFileExists{\jobname-pw.ind}{\input{\jobname-pw.ind}}{}

\end{document}

      