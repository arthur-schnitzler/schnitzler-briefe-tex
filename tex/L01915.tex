%% latex-korrekturansicht-vorspann.tex
%% Vorspann für die Korrekturansicht.
%% Lädt die gemeinsame Datei latex-vorspann.tex mit gesetztem Schalter.

\newif\ifkorrekturansicht
\korrekturansichttrue

\input{../tex-inputs/latex-vorspann}


               \section[Hugo von Hofmannsthal an Arthur Schnitzler, 1. 3. 1910]{ Hugo von Hofmannsthal an Arthur Schnitzler, 1. 3. 1910}\nopagebreak\mylabel{v}\rehead{ }\normalsize\beginnumbering\briefempfaengerindex{Schnitzler, Arthur@\textsc{Schnitzler, Arthur}!zzzHofmannsthal, Hugo von@\emph{von Hugo von Hofmannsthal}!1910-03-011@{1. 3. 1910}|(be} \toendnotes[C]{\smallbreak\pagebreak[2]} \Standort{CUL, Schnitzler, B 43.}
\physDesc{Bildpostkarte
\newline{}Handschrift: schwarze Tinte, deutsche Kurrent\newline{}Versand: Stempel: »\nobreak{}\oindex{Weimar@\textbf{Weimar}, \emph{Besiedelter Ort (A.BSO)}|pwk}Weimar\nobreak{}«.  
\newline{}Schnitzler: mit Bleistift die Jahreszahl ergänzt: »910« \newline{}Ordnung: 1) mit Bleistift von unbekannter Hand nummeriert: »\strikeout{317}« 2) mit Bleistift von unbekannter Hand nummeriert: »314«}\buchAbdrucke{\weitereDrucke{Hugo von Hofmannsthal, Arthur Schnitzler: \emph{Briefwechsel}. Hg. Therese Nickl und Heinrich Schnitzler. Frankfurt am Main: \emph{S. Fischer} 1964, S. 248.} }\toendnotes[C]{\smallbreak}\pstart{}{\pb}\textsc{Herrn D\textsuperscript{r} Arthur Schnitzler}\pend{}\pstart{}\textcolor{pink}{\textsc{Wien}}{}\ledrightnote{\textcolor{pink}{Wien}}\pend{}\pstart{}\textsc{\textcolor{pink}{XVIII Spöttelgasse 7}{}\ledrightnote{\textcolor{pink}{Edmund-Weiß-Gasse}}}\pend{}{\bigskip}\pstart
           \noindent{}\centering{}\textcolor{gray}{\textbf{{\pb}\textcolor{blue}{Anna Amalia}{}\ledrightnote{\textcolor{blue}{Anna Amalia von Sachsen-Weimar und Eisenach}} und ihre Begleiter in der \textcolor{pink}{Villa d’Este}{}\ledrightnote{\textcolor{pink}{Villa d’Este}}. \textcolor{blue}{Herder}{}\ledrightnote{\textcolor{blue}{Johann Gottfried von Herder}}, \textcolor{blue}{Göchhausen}{}\ledrightnote{\textcolor{blue}{Luise Ernestine Christiane Juliane von Göchhausen}}, \textcolor{blue}{Angelika Kauffmann}{}\ledrightnote{\textcolor{blue}{Angelika Kauffmann}}. \textcolor{green}{Aquarell}{}\ledrightnote{→\textcolor{green}{Anna Amalia und ihre Begleiter im Garten der Villa d’Este zu Tivoli}} v. \textcolor{blue}{Schütz}{}\ledrightnote{\textcolor{blue}{Johann Georg Schütz}}. (Schloss \textcolor{pink}{Tiefurt}{}\ledrightnote{\textcolor{pink}{Schloss Tiefurt}} bei \textcolor{pink}{Weimar}{}\ledrightnote{\textcolor{pink}{Weimar}}.)}}\pend
           \pstart
           \raggedleft{}\textcolor{pink}{Weimar}{}\ledrightnote{\textcolor{pink}{Weimar}}{ }1. III.\pend
           \pstart
           Danke Ihnen, lieber Arthur, ſehr für die lieben Zeilen (und wünſche
               mir ſehr mündlich weiteres). Wir ſind Montag zurück und freuen uns auf
               Euch.\pend
           \pstart \spacefill\mbox{Hugo.}\pend{}\endnumbering\briefempfaengerindex{Schnitzler, Arthur@\textsc{Schnitzler, Arthur}!zzzHofmannsthal, Hugo von@\emph{von Hugo von Hofmannsthal}!1910-03-011@{1. 3. 1910}|)be}\mylabel{h}  \normalsize

\doendnotes{C}
\bigskip
\vfill

\clearpage

\footnotesize

\lohead{\textsc{register}}

% Definiere theindex-Environment komplett neu ohne reledmac
\makeatletter
\renewenvironment{theindex}{%
  \section*{\indexname}%
  \setlength{\parindent}{0pt}%
  \setlength{\parskip}{0pt plus 0.3pt}%
  \let\item\@idxitem
}{%
  \clearpage
}
\makeatother

\IfFileExists{\jobname-pw.ind}{\input{\jobname-pw.ind}}{}

\end{document}

      