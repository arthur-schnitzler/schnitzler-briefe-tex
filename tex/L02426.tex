%% latex-korrekturansicht-vorspann.tex
%% Vorspann für die Korrekturansicht.
%% Lädt die gemeinsame Datei latex-vorspann.tex mit gesetztem Schalter.

\newif\ifkorrekturansicht
\korrekturansichttrue

\input{../tex-inputs/latex-vorspann}


               \section[Arthur Schnitzler an Felix Braun, 28. 12. 1924]{ Arthur Schnitzler an Felix Braun, 28. 12. 1924}\nopagebreak\mylabel{v}\rehead{ }\normalsize\beginnumbering\briefempfaengerindex{Braun, Felix@\textsc{Braun, Felix}!zzzSchnitzler, Arthur@\emph{von Arthur Schnitzler}!1924-12-281@{28. 12. 1924}|(be} \toendnotes[C]{\smallbreak\pagebreak[2]} \Standort{Wienbibliothek im Rathaus, H.I.N.-198.047.}
\physDesc{Postkarte
\newline{}Handschrift: schwarze Tinte, lateinische Kurrent\newline{}Versand: 1) Stempel: »\nobreak{}\oindex{XVIII., Waehring@\textbf{XVIII., Währing}, \emph{Bezirk (A.BZK)}|pwk}18/1 Wien 110, 29. XII. 24, 17\nobreak{}«.  2) mit Bleistift von unbekannter Hand die falsche Hausnummer
                                    durchgestrichen und mit »191«
                                    ersetzt}\toendnotes[C]{\smallbreak}\pstart{}{\pb}\label{T_L02426-1v}\edtext{\textcolor{gray}{\textbf{A. S.}}}{\lemma{\textnormal{\emph{A. S.}}}\Cendnote{\textnormal{ovaler Absenderkleber}}}\label{T_L02426-1h}\pend{}\pstart{}\textcolor{pink}{\textcolor{gray}{\textbf{WIEN, XVIII.}}}{}\ledrightnote{\textcolor{pink}{XVIII., Währing}}\pend{}\pstart{}\textcolor{pink}{\textcolor{gray}{\textbf{STERNWARTESTR. 71}}}{}\ledrightnote{\textcolor{pink}{Sternwartestraße}}\pend{}{\bigskip}\pstart{}Hrn\pend{}\pstart{}Felix Braun\pend{}\pstart{}\textcolor{pink}{Wien XIX}{}\ledrightnote{\textcolor{pink}{XIX., Döbling}}\pend{}\pstart{}\textcolor{pink}{Sieveringer Straße 99}{}\ledrightnote{\textcolor{pink}{Sieveringer Straße}}\pend{}{\bigskip}\pstart
           \raggedleft{}{\pb}28. 12. 24\pend
           \pstart{}lieber und verehrter Herr Braun, \pend\pstart
           schönsten Dank für das neue Buch – die »\textcolor{green}{Else}{}\ledrightnote{\textcolor{green}{Fräulein Else}}«
                    (Sie hätten das zweite Exemplar auch von mir direct haben können) ist nun
                    hoffentlich richtig in Ihren{\pb} Besitz
                    gelangt.\pend
           \pstart
           Herzliche Neujahrsgrüße!{\\[\baselineskip]}Ihr{\\[\baselineskip]}\spacefill\mbox{Arthur Schnitzler}\pend
           \leftskip=0em{}\endnumbering\briefempfaengerindex{Braun, Felix@\textsc{Braun, Felix}!zzzSchnitzler, Arthur@\emph{von Arthur Schnitzler}!1924-12-281@{28. 12. 1924}|)be}\mylabel{h}  \normalsize

\doendnotes{C}
\bigskip
\vfill

\clearpage

\footnotesize

\lohead{\textsc{register}}

% Definiere theindex-Environment komplett neu ohne reledmac
\makeatletter
\renewenvironment{theindex}{%
  \section*{\indexname}%
  \setlength{\parindent}{0pt}%
  \setlength{\parskip}{0pt plus 0.3pt}%
  \let\item\@idxitem
}{%
  \clearpage
}
\makeatother

\IfFileExists{\jobname-pw.ind}{\input{\jobname-pw.ind}}{}

\end{document}

      