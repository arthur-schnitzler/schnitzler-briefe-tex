%% latex-korrekturansicht-vorspann.tex
%% Vorspann für die Korrekturansicht.
%% Lädt die gemeinsame Datei latex-vorspann.tex mit gesetztem Schalter.

\newif\ifkorrekturansicht
\korrekturansichttrue

\input{../tex-inputs/latex-vorspann}


               \section[Hermann Bahr an Arthur Schnitzler, 7. 12. 1912]{ Hermann Bahr an Arthur Schnitzler, 7. 12. 1912}\nopagebreak\mylabel{v}\rehead{ }\normalsize\beginnumbering\briefempfaengerindex{Schnitzler, Arthur@\textsc{Schnitzler, Arthur}!zzzBahr, Hermann@\emph{von Hermann Bahr}!1912-12-071@{7. 12. 1912}|(be} \toendnotes[C]{\smallbreak\pagebreak[2]} \Standort{CUL, Schnitzler, B 5b.}
\physDesc{Brief, 1 Blatt, 1 Seite
\newline{}Handschrift: schwarze Tinte, deutsche Kurrent\newline{}Ordnung: mit Bleistift von unbekannter Hand
                                 nummeriert: »175« und ergänzt: »\textsc{Bahr}« }\buchAbdrucke{\weitereDrucke{Hermann Bahr, Arthur Schnitzler: \emph{Briefwechsel, Aufzeichnungen, Dokumente (1891–1931)}. Hg. Kurt Ifkovits und Martin Anton Müller. Göttingen: \emph{Wallstein} 2018, S. 479.} }\toendnotes[C]{\smallbreak}\pstart
           \noindent{}{\pb}\textcolor{gray}{\textbf{\textcolor{pink}{GRAND HOTEL DE L’EUROPE}{}\ledrightnote{\textcolor{pink}{Grand Hotel de L’Europe, G. Jung}}}}\pend
           \pstart
           \textcolor{gray}{\textbf{\textcolor{blue}{G. JUNG}{}\ledrightnote{\textcolor{blue}{Georg Jung}}}}\pend
           \pstart
           \raggedleft{}\textcolor{gray}{\textbf{\textcolor{pink}{Salzburg}{}\ledrightnote{\textcolor{pink}{Salzburg}}, }}{ }7. 12. 12\pend
           \pstart\center{}Lieber Arthur!\pend\pstart
           Ich war ſechs Wochen unterwegs, jeden Abend in einer anderen Stadt auf dem »Brettl«,
               ſo komm ich nun hier erſt dazu, Deinen lieben Brief zu beantworten. An \textcolor{blue}{Altenberg}{}\ledrightnote{\textcolor{blue}{Peter Altenberg}} kann ich mich nicht beteiligen. Ich tu
               nach meinem Gefühl genug für andere, für anonyme Armut, die mich braucht und ohne
               mich ſich keinen Rat wüßte, während der Betrag, den ich dem guten \textcolor{blue}{Peter}{}\ledrightnote{\textcolor{blue}{Peter Altenberg}} geben könnte, für ihn nichts bedeuten würde und er
               tauſendfach Gelegenheit hat, ſich ihn zu beſchaffen. Misverſteh mich \substVorne{}\textsuperscript{\textcolor{gray}{ſ}\textcolor{gray}{×}\-\textcolor{gray}{×}}\substDazwischen{}ni\substHinten{}cht: ich ſchätze \textcolor{blue}{Altenberg}{}\ledrightnote{\textcolor{blue}{Peter Altenberg}} als Dichter
               ſehr, aber als »Armen« gar nicht, auf dieſem Gebiet leiſten andere viel mehr.\pend
           \pstart
           Ich freue mich ſehr über alle Deine Erfolge und habe das gute Gefühl, daß Du nun »in
               Fülle« haſt, was Du Dir je gewünſcht. Möge es Dir ſo bleiben! Und auch Deiner lieben
                  \textcolor{blue}{Frau}{}\ledrightnote{→\textcolor{blue}{Olga Schnitzler}} und den \textcolor{blue}{Kindern}{}\ledrightnote{\textcolor{blue}{Heinrich Schnitzler}{\newline}\textcolor{blue}{Lili Schnitzler}} wünſch ich immer alles Beſte!\pend
           \pstart
           Mit den ſchönſten Grüßen von uns \textcolor{blue}{Beiden}{}\ledrightnote{→\textcolor{blue}{Anna Bahr-Mildenburg}}{\\[\baselineskip]}Dein alter{\\[\baselineskip]}\spacefill\mbox{Hermann}\pend
           \leftskip=0em{}\endnumbering\briefempfaengerindex{Schnitzler, Arthur@\textsc{Schnitzler, Arthur}!zzzBahr, Hermann@\emph{von Hermann Bahr}!1912-12-071@{7. 12. 1912}|)be}\mylabel{h}  \normalsize

\doendnotes{C}
\bigskip
\vfill

\clearpage

\footnotesize

\lohead{\textsc{register}}

% Definiere theindex-Environment komplett neu ohne reledmac
\makeatletter
\renewenvironment{theindex}{%
  \section*{\indexname}%
  \setlength{\parindent}{0pt}%
  \setlength{\parskip}{0pt plus 0.3pt}%
  \let\item\@idxitem
}{%
  \clearpage
}
\makeatother

\IfFileExists{\jobname-pw.ind}{\input{\jobname-pw.ind}}{}

\end{document}

      