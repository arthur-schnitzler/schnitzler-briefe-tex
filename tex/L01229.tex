%% latex-korrekturansicht-vorspann.tex
%% Vorspann für die Korrekturansicht.
%% Lädt die gemeinsame Datei latex-vorspann.tex mit gesetztem Schalter.

\newif\ifkorrekturansicht
\korrekturansichttrue

\input{../tex-inputs/latex-vorspann}


               \section[Arthur Schnitzler an Hermann Bahr, {[}9. 7. 1902{]}]{ Arthur Schnitzler an Hermann Bahr, {[}9. 7. 1902{]}}\nopagebreak\mylabel{v}\rehead{ }\normalsize\beginnumbering\briefempfaengerindex{Bahr, Hermann@\textsc{Bahr, Hermann}!zzzSchnitzler, Arthur@\emph{von Arthur Schnitzler}!1902-07-091@{9. 7. 1902}|(be} \toendnotes[C]{\smallbreak\pagebreak[2]} \Standort{TMW, HS AM 23386 Ba.}
\physDesc{Brief, 1 Blatt, 2 Seiten
\newline{}Handschrift: schwarze Tinte, deutsche Kurrent\newline{}Ordnung: Lochung }\buchAbdrucke{\weitereDrucke{1) \emph{9. 7. 1907.} In: Arthur Schnitzler: \emph{The Letters of Arthur Schnitzler to Hermann Bahr}. Edited, annotated, and with an introduction, by Donald G.
                        Daviau. Chapel Hill: \emph{The University of North Carolina Press} 1978, S. 98 (University of North Carolina studies in the Germanic languages
                        and literatures, 89).} \weitereDrucke{2) Hermann Bahr, Arthur Schnitzler: \emph{Briefwechsel, Aufzeichnungen, Dokumente (1891–1931)}. Hg. Kurt Ifkovits und Martin Anton Müller. Göttingen: \emph{Wallstein} 2018, S. 240.} }\toendnotes[C]{\smallbreak}\pstart
           \raggedleft{}{\pb}9/7 \label{T_L01229_1v}\edtext{902}{\lemma{\textnormal{\emph{902}}}\Cendnote{\textnormal{Die nachgezogene Ziffer »2« von
                        unbekannter Hand fälschlich durch »7« überschrieben.}}}\label{T_L01229_1h}\pend
           \pstart
           lieber Hermann, \label{K_L01229_1v}\edtext{beifolgenden Wiſch}{\lemma{\textnormal{\emph{beifolgenden Wiſch}}}\Cendnote{\textnormal{Ein Schreiben von \textcolor{blue}{Leopold Hipp} mit Aufforderung zur Angabe von Informationen
                  über \textcolor{blue}{Bahrs} finanzielle Situation, sich heute in der \emph{Cambridge University Library} befindet, \textcolor{blue}{Bahr} retournierte es wohl mit seinem
                  Antwortschreiben. (Abgedruckt in Bahr/Schnitzler, S. 239).}}}\label{K_L01229_1h}
               erhielt ich nachgeſandt. Ich beabſichtigte nicht zu antworten, aber man ſagt mir,
               daſs unerhörter Weiſe eine \uline{Verpflichtung} dazu
               beſteht. Ich würde ſagen, dſs ich keine Ahnung habe. Aber vielleicht wünſchest du
               ſelbst irgend eine andre\substVorne{}\textsuperscript{.}\substDazwischen{} Antwort.\substHinten{} Bitte theile mir mit, was {\pb}du für recht \strikeout{h\textcolor{gray}{ie}lt\textcolor{gray}{e}ſt} hältſt,
               und ſchicke mir das Formular zurück.\pend
           \pstart
           Ich wollte dich ſelbſt beſuchen, komme aber in den allernächſten Tagen nicht dazu;
               daher iſt leider briefliche Erledigung nothwendig.\pend
           \pstart
           Die Tour war ſehr ſchön; \textsc{\textcolor{blue}{Hugo}{}\ledrightnote{\textcolor{blue}{Hugo von Hofmannsthal}}} iſt noch ein paar Tage in \textsc{\textcolor{pink}{Welsberg}{}\ledrightnote{\textcolor{pink}{Welsberg-Taisten}}} geblieben,\pend
           \pstart
           Von Herzen{\\[\baselineskip]}dein \spacefill\mbox{Arthur}\pend
           \leftskip=0em{}\endnumbering\briefempfaengerindex{Bahr, Hermann@\textsc{Bahr, Hermann}!zzzSchnitzler, Arthur@\emph{von Arthur Schnitzler}!1902-07-091@{9. 7. 1902}|)be}\mylabel{h}  \normalsize

\doendnotes{C}
\bigskip
\vfill

\clearpage

\footnotesize

\lohead{\textsc{register}}

% Definiere theindex-Environment komplett neu ohne reledmac
\makeatletter
\renewenvironment{theindex}{%
  \section*{\indexname}%
  \setlength{\parindent}{0pt}%
  \setlength{\parskip}{0pt plus 0.3pt}%
  \let\item\@idxitem
}{%
  \clearpage
}
\makeatother

\IfFileExists{\jobname-pw.ind}{\input{\jobname-pw.ind}}{}

\end{document}

      