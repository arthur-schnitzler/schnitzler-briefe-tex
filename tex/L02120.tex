%% latex-korrekturansicht-vorspann.tex
%% Vorspann für die Korrekturansicht.
%% Lädt die gemeinsame Datei latex-vorspann.tex mit gesetztem Schalter.

\newif\ifkorrekturansicht
\korrekturansichttrue

\input{../tex-inputs/latex-vorspann}


               \section[Peter Altenberg und Georg Engländer an Arthur Schnitzler, {[}Mitte April{]} 1913]{ Peter Altenberg und Georg Engländer an Arthur Schnitzler, {[}Mitte April{]}
               1913}\nopagebreak\mylabel{v}\rehead{ }\normalsize\beginnumbering\briefempfaengerindex{Schnitzler, Arthur@\textsc{Schnitzler, Arthur}!zzzEnglaender, Georg@\emph{von Georg Engländer}!1913-04-152@{{[}Mitte April{]} 1913}|(be}\briefempfaengerindex{Schnitzler, Arthur@\textsc{Schnitzler, Arthur}!zzzAltenberg, Peter@\emph{von Peter Altenberg}!1913-04-152@{{[}Mitte April{]} 1913}|(be} \toendnotes[C]{\smallbreak\pagebreak[2]} \Standort{CUL, Schnitzler, B 2.}
\physDesc{Brief, 1 Blatt, 4 Seiten
\newline{}Handschrift: schwarze Tinte, deutsche Kurrent
\newline{}Schnitzler: 1) mit Bleistift erstes Blatt beschriftet: »\textsc{Altenberg}« und datiert: »April 1913« 2) mit rotem Buntstift eine Unterstreichung\newline{}Ordnung: mit Bleistift von unbekannter Hand nummeriert:
                                    »14« }\Standort{CUL, Schnitzler, B 2.}
\physDesc{Brief, 1 Blatt, 3 Seiten
\newline{}Handschrift: schwarze Tinte, deutsche Kurrent
\newline{}Schnitzler: mit Bleistift beschriftet: »\textsc{Engländer}« und datiert: »1914/1915« \newline{}Editorischer Hinweis: Die Hinzufügung dieses Blattes zum Korrespondenzstück erfolgt in
                                 Abgleich mit einem Brief Altenbergs und Engländers an Bahr (\emph{Briefwechsel} Bahr/Schnitzler,
                                    480–481), der offensichtlich zeitnah entstand. Zudem ist
                                 aus dem Inhalt erkenntlich, dass es sich nicht um ein
                                 eigenständiges Schreiben handelt. }\buchAbdrucke{\weitereDrucke{Kurt Bergel: \emph{Arthur Schnitzlers unveröffentlichte Tragikomödie Das Wort.} In: \emph{Studies in Arthur Schnitzler. Centennial Commemorative
                        Volume}. Hg. Herbert W. Reichert und Herman Salinger. Chapel Hill: \emph{University of North Carolina Press} 1963, S. 22 (UNC Studies in the Germanic Languages and Literatures, 42).} }\toendnotes[C]{\smallbreak}\pstart{}{\pb}Lieber lieber Herr \textsc{D\textsuperscript{r}} Arthur Schnitzler,\pend\pstart
           ein Verlorener, Zuſammengeſtürzter, unmittelbar nach einem paradieſiſchen \textcolor{pink}{Semmering}{}\ledrightnote{\textcolor{pink}{Semmering}}-Jahr 1912, ein \uuline{\edtext{tiefſt}{\Cendnote{dreifach unterstrichen}}} Verzweifelter, wendet ſich an Sie als
               Menſchenfreundlichen und Dichter vor allem, dann als Kollegen und langjährigen
               litterariſchen Genoſſen – – – Hilfe, Rettung, Erbarmen, in einer ſo \uline{ſchauerlichen} Situation, die noch nie, noch nie, noch nie,
               ein Dichter, ein Künſtler-Menſch erlitten hat! {\pb}Der ſüßen unentbehrlichen Freiheit
               beraubt, verbringe ich meine Tage u. Nächte in unermeſslichen Qualen, eingefangen,
               kontrollirt wie ein \uline{böſes gefährliches giftiges
                  Reptil}!\pend
           \pstart
           \label{K_L02120_1v}\edtext{Hilfe, Errettung, \textcolor{green}{Weg ins Freie}{}\ledrightnote{→\textcolor{green}{Der Weg ins Freie. Roman}}}{\lemma{\textnormal{\emph{Hilfe, … Freie}}}\Cendnote{\textnormal{Vermutlich Mitte April 1913{ }schrieb \textcolor{blue}{Altenberg} an \textcolor{blue}{Hermann Bahr} und,
                  separat, an dessen Gattin \textcolor{blue}{Anna Bahr-Mildenburg} (\emph{Korrespondenz von Peter Altenberg an Hermann Bahr
                        (1895–1913)}. Hgg. Heinz Lunzer, Victoria Lunzer-Talos. In: Jeanne
                     Bennay, Alfred Pfabigan, Hgg.: \emph{Hermann Bahr – Für eine andere
                        Moderne}. Bern: \emph{Peter Lang}{ }2004, S. 249-262, hier S. 259–262.) In Folge dessen schrieb
                     \textcolor{blue}{Bahr} am 16. 4. 1913 an \textcolor{blue}{Schnitzler} über den »verworrenen
                     Brief«. Dieser antwortete zwei Tage später, er habe gleichfalls einen
                  Brief \textcolor{blue}{Altenberg}s erhalten. Die sprachliche
                  Entsprechung von Formulierungen, wie »Hilfe, Errettung,
                  Erbarmen!!!« an \textcolor{blue}{Bahr} legen die
                  zeitliche Unmittelbarkeit der beiden Korrespondenzstücke an \textcolor{blue}{Bahr} und \textcolor{blue}{Schnitzler}
                  nahe.}}}\label{K_L02120_1h}!!!\pend
           \pstart
           Auch geht es mir ökonomiſch ſchlecht, und bitte ich Sie und \textcolor{blue}{Hofmannsthal}{}\ledrightnote{\textcolor{blue}{Hugo von Hofmannsthal}} um die mir {\pb}zugeſagten \uline{20} Kr. monatlich ſeit \uline{November 1912}, da ich
               gerade damals zuſammenbrach und nicht mehr denken konnte!\pend
           \pstart
           \uuline{\edtext{Hilfe}{\Cendnote{dreifach unterstrichen}}}, um Gotteswillen, ehe ich ganz zerſtört
               bin!\pend
           \pstart
           Ich möchte auf dem \textcolor{pink}{Semmering}{}\ledrightnote{\textcolor{pink}{Semmering}} ruhig vegetiren, in
               Freiheit und Frieden! Hilfe von \uuline{\edtext{Bruder}{\Cendnote{dreifach unterstrichen}}}-Seelen!
               Dichter, Künſtler, Menſchen, helft mir!!!\pend
           \pstart \spacefill\mbox{Peter Altenberg}\pend{}\pstart
           \noindent{}{\pb}\label{T_L02120_1v}\edtext{Adreſſe}{\lemma{\textnormal{\emph{Adreſſe}}}\Cendnote{\textnormal{Hier wechselt die Schreibrichtung und das Blatt ist entlang
                     des Mittelfalzes beschrieben.}}}\label{T_L02120_1h}: \textcolor{pink}{XIII/\textsubscript{12}{ }\label{K_L02120_2v}\edtext{\textsc{Villa Austria}}{\lemma{\textnormal{\emph{Villa Austria}}}\Cendnote{\textnormal{Pavillon der
                        Landesnervenheilanstalt \textcolor{pink}{Am Steinhof}.}}}\label{K_L02120_2h}}{}\ledrightnote{\textcolor{pink}{Otto-Wagner-Spital}}\pend
           \pstart
           Leſen Sie mein letztes Buch:\pend
           \pstart
           \centering{}»\textcolor{green}{Semmering 1912}{}\ledrightnote{\textcolor{green}{»Semmering 1912«}}«\pend
           \pstart
           \noindent{}und denken Sie, wie dem Autor zumute iſt, der nun wie ein wildes Tier eingeſperrt
                  ſchmachtet, ſeit \uline{5} Monaten!!!\pend
           \pstart
           Ihr{\\}\spacefill\mbox{PA}\pend
           \pstart
           \noindent{}{\pb}{[}hs. Engländer:{]} \uline{Zur Aufklärung}. \textsc{(Diskret!)}\pend
           \pstart{}Sehr geehrter Herr.\pend\pstart
           Am 10 Dec. v. J. mußte ich meinen Bruder in einem erbarmungswürdigen \textsc{Nerven-Zustand} auf den \textcolor{pink}{\textsc{Steinhof}}{}\ledrightnote{\textcolor{pink}{Otto-Wagner-Spital}} überführen.\pend
           \pstart
           Nun erſt ſeit 3 {\pb}Wochen ko{\geminationm}t er allmählich zum \textsc{Bewusstsein}{ }{\kaufmannsund} iſt empört über den Zwang den Ärzte{ }{\kaufmannsund} Pfleger auf ihn ausüben {\kaufmannsund}
               will durchaus entfliehen. Ärztliche {\pb}Freunde finden aber auch jetzt noch ſeinen Kopf {\kaufmannsund}{ }\textsc{Nervenzustand} ſo labil daſs ſie auch nur einige Tage
               Freiheit ſchon für ſeine Gesundheit als \textsc{katastrophal}
               befürchten.\pend
           \pstart
           Hochachtend{\\[\baselineskip]}\spacefill\mbox{G. Engländer}\pend
           \leftskip=0em{}\pstart
           \noindent{}\textcolor{pink}{III \textsc{Seidlgasse} 23}{}\ledrightnote{\textcolor{pink}{Seidlgasse}}.\pend
           \pstart
           P.S. Seine \textsc{Correſp}. wird mir von der \textcolor{pink}{\textsc{Anstalt}}{}\ledrightnote{\textcolor{pink}{Otto-Wagner-Spital}} offen zugeſandt!!\pend
           \endnumbering\briefempfaengerindex{Schnitzler, Arthur@\textsc{Schnitzler, Arthur}!zzzEnglaender, Georg@\emph{von Georg Engländer}!1913-04-152@{{[}Mitte April{]} 1913}|)be}\briefempfaengerindex{Schnitzler, Arthur@\textsc{Schnitzler, Arthur}!zzzAltenberg, Peter@\emph{von Peter Altenberg}!1913-04-152@{{[}Mitte April{]} 1913}|)be}\mylabel{h}  \normalsize

\doendnotes{C}
\bigskip
\vfill

\clearpage

\footnotesize

\lohead{\textsc{register}}

% Definiere theindex-Environment komplett neu ohne reledmac
\makeatletter
\renewenvironment{theindex}{%
  \section*{\indexname}%
  \setlength{\parindent}{0pt}%
  \setlength{\parskip}{0pt plus 0.3pt}%
  \let\item\@idxitem
}{%
  \clearpage
}
\makeatother

\IfFileExists{\jobname-pw.ind}{\input{\jobname-pw.ind}}{}

\end{document}

      