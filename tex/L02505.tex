%% latex-korrekturansicht-vorspann.tex
%% Vorspann für die Korrekturansicht.
%% Lädt die gemeinsame Datei latex-vorspann.tex mit gesetztem Schalter.

\newif\ifkorrekturansicht
\korrekturansichttrue

\input{../tex-inputs/latex-vorspann}


               \section[Hugo Hofmannsthal an Arthur Schnitzler, 31. 7. 1928]{ Hugo Hofmannsthal an Arthur Schnitzler, 31. 7. 1928}\nopagebreak\mylabel{v}\rehead{ }\normalsize\beginnumbering\briefempfaengerindex{Schnitzler, Arthur@\textsc{Schnitzler, Arthur}!zzzHofmannsthal, Hugo von@\emph{von Hugo von Hofmannsthal}!1928-07-311@{31. 7. 1928}|(be} \toendnotes[C]{\smallbreak\pagebreak[2]} \buchAlsQuelle{Hugo von Hofmannsthal, Arthur Schnitzler: \emph{Briefwechsel}. Hg. Therese Nickl und Heinrich Schnitzler. Frankfurt am Main: \emph{S. Fischer} 1964, S. 311.}\toendnotes[C]{\smallbreak}\pstart
           \raggedleft{}{\pb}\textcolor{pink}{Rodaun}{}\ledrightnote{\textcolor{pink}{Rodaun}}, 31. Juli 1928\pend
           \pstart{}mein lieber guter Arthur, \pend\pstart
           was kann, und was darf ich Ihnen sagen! Wir sind auch Eltern, und wir \textcolor{blue}{weinen}{}\ledrightnote{→\textcolor{blue}{Lili Schnitzler}} mit Ihnen! \pend
           \pstart
           Diese ganzen Jahrzehnte, die wir als Freunde verlebt haben, stehen, mit Gewalt
               heraufgerufen, wie eine einzige Landschaft vor meiner Seele, aber es ringt sich in
               mir nicht zu klaren Gedanken durch, was mich dabei furchtbar bewegt. In solchen
               Stunden steht alles als ein Ganzes da, das geht über die Kräfte – und alles drängt in
               eine letzte Ahnung hinein: ich nenne sie Gott – und Sie vielleicht nennen sie
               anders. – Ich möchte Sie sehen, mein lieber Arthur – aber wenn Sie alles abweisen,
               was an Sie heranwill, und auch mich – so verstehe ich es ja so gut.\pend
           \pstart
           In alter liebevoller Freundschaft{\\[\baselineskip]}Ihr\spacefill\mbox{Hugo.}\pend
           \leftskip=0em{}\endnumbering\briefempfaengerindex{Schnitzler, Arthur@\textsc{Schnitzler, Arthur}!zzzHofmannsthal, Hugo von@\emph{von Hugo von Hofmannsthal}!1928-07-311@{31. 7. 1928}|)be}\mylabel{h}  \normalsize

\doendnotes{C}
\bigskip
\vfill

\clearpage

\footnotesize

\lohead{\textsc{register}}

% Definiere theindex-Environment komplett neu ohne reledmac
\makeatletter
\renewenvironment{theindex}{%
  \section*{\indexname}%
  \setlength{\parindent}{0pt}%
  \setlength{\parskip}{0pt plus 0.3pt}%
  \let\item\@idxitem
}{%
  \clearpage
}
\makeatother

\IfFileExists{\jobname-pw.ind}{\input{\jobname-pw.ind}}{}

\end{document}

      