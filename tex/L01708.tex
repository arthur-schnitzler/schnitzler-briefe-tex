%% latex-korrekturansicht-vorspann.tex
%% Vorspann für die Korrekturansicht.
%% Lädt die gemeinsame Datei latex-vorspann.tex mit gesetztem Schalter.

\newif\ifkorrekturansicht
\korrekturansichttrue

\input{../tex-inputs/latex-vorspann}


               \section[Hugo von Hofmannsthal an Arthur Schnitzler, 18. 9. {[}1907{]}]{ Hugo von Hofmannsthal an Arthur Schnitzler, 18. 9. {[}1907{]}}\nopagebreak\mylabel{v}\rehead{ }\normalsize\beginnumbering\briefempfaengerindex{Schnitzler, Arthur@\textsc{Schnitzler, Arthur}!zzzHofmannsthal, Hugo von@\emph{von Hugo von Hofmannsthal}!1907-09-181@{{[}18. 9. 1907{]}}|(be} \toendnotes[C]{\smallbreak\pagebreak[2]} \Standort{CUL, Schnitzler, B 43.}
\physDesc{Brief, 1 Blatt, 3 Seiten
\newline{}Handschrift: schwarze Tinte, deutsche Kurrent
\newline{}Schnitzler: mit Bleistift die Jahreszahl ergänzt: »907« \newline{}Ordnung: mit Bleistift von unbekannter Hand eine vorausgehende Nummerierung geändert zu: »286« }\buchAbdrucke{\weitereDrucke{Hugo von Hofmannsthal, Arthur Schnitzler: \emph{Briefwechsel}. Hg. Therese Nickl und Heinrich Schnitzler. Frankfurt am Main: \emph{S. Fischer} 1964, S. 231.} }\toendnotes[C]{\smallbreak}\pstart
           \raggedleft{}18 IX.\pend
           \pstart{}{\pb}lieber\pend\pstart
           Diplomatenprüfung im Alter 28/29 natürlich ſehr ungewöhnlich, nur erklärlich – wie
               Sie ſelbſt annehmen – durch Umſatteln aus dem \uline{inneren}
               Dienſt (Statthalterei.) allenfalls aus der Officierslaufbahn. Diplomatenprüfung ſetzt
               volles jus (ohne Doctorat) voraus, hat aber mit \textcolor{brown}{orient.
                  Akademie}{}\ledrightnote{\textcolor{brown}{Orientalische Akademie}} gar nichts zu thuen; dieſe bereitet zur Conſularcarrière {\pb}vor, welche dienſtlich und ſocial
               von Diplomatie \uline{geſchieden}.\pend
           \pstart
           Mein \textcolor{green}{Stück}{}\ledrightnote{\textcolor{green}{Silvia im »Stern«}}{ }ſchreitet, in ungleichem tempo, vor.\hspace*{1.5em}Wir ſind jedenfalls 1\textsuperscript{ten} October in \textcolor{pink}{Wien}{}\ledrightnote{\textcolor{pink}{Wien}}.\pend
           \pstart
           Herzlich Ihr{\\[\baselineskip]}\spacefill\mbox{Hugo.}\pend
           \leftskip=0em{}\pstart
           \textsc{P. S.} Rathe dringend »\textcolor{brown}{Morgen}{}\ledrightnote{\textcolor{brown}{Morgen. Wochenschrift für deutsche Kultur}}« und allen andern Reflectanten gegenüber den Preis \uuline{halten}, nicht ſich eilen, nicht {\pb}Geduld verlieren, nicht ſich ein
               paar Briefe mehr verdrießen laſſen. \textcolor{blue}{Waſſermann}{}\ledrightnote{\textcolor{blue}{Jakob Wassermann}}
                  beko{\geminationm}t von \textcolor{brown}{Über Land u
                  Meer}{}\ledrightnote{\textcolor{brown}{Über Land und Meer}}\pend
           \settowidth{\longeste}{8 Auflagen im vorhinein}\settowidth{\longestz}{= 24000 Kronen.}\settowidth{\longestd}{}\settowidth{\longestv}{}\settowidth{\longestf}{}\addtolength\longeste{1em}
        \addtolength\longestz{1em}
      \pstart\noindent\makebox[\the\longeste][l]{für den \textcolor{green}{Romanabdruck}{}\ledrightnote{→\textcolor{green}{Caspar Hauser oder Die Trägheit des Herzens}}}\makebox[\the\longestz][l]{\hspace*{1.5em}12000{ }}
                  \pend\pstart\noindent\makebox[\the\longeste][l]{8 Auflagen im vorhinein}\makebox[\the\longestz][l]{\hspace*{1.5em}\uline{{ }8000{ }}}
                  \pend\pstart\noindent\makebox[\the\longeste][l]{}\makebox[\the\longestz][l]{\hspace*{1.5em}20000M}
                  \pend\pstart\noindent\makebox[\the\longeste][l]{}\makebox[\the\longestz][l]{\hspace*{1.5em}= 24000 Kronen.}
                  \pend\pstart
           Und Sie haben einen viel ſtärkern Namen!\pend
           \endnumbering\briefempfaengerindex{Schnitzler, Arthur@\textsc{Schnitzler, Arthur}!zzzHofmannsthal, Hugo von@\emph{von Hugo von Hofmannsthal}!1907-09-181@{{[}18. 9. 1907{]}}|)be}\mylabel{h}  \normalsize

\doendnotes{C}
\bigskip
\vfill

\clearpage

\footnotesize

\lohead{\textsc{register}}

% Definiere theindex-Environment komplett neu ohne reledmac
\makeatletter
\renewenvironment{theindex}{%
  \section*{\indexname}%
  \setlength{\parindent}{0pt}%
  \setlength{\parskip}{0pt plus 0.3pt}%
  \let\item\@idxitem
}{%
  \clearpage
}
\makeatother

\IfFileExists{\jobname-pw.ind}{\input{\jobname-pw.ind}}{}

\end{document}

      