%% latex-korrekturansicht-vorspann.tex
%% Vorspann für die Korrekturansicht.
%% Lädt die gemeinsame Datei latex-vorspann.tex mit gesetztem Schalter.

\newif\ifkorrekturansicht
\korrekturansichttrue

\input{../tex-inputs/latex-vorspann}


               \section[Richard Beer-Hofmann an Arthur Schnitzler, {[}18. 8. 1894{]}]{ Richard Beer-Hofmann an Arthur Schnitzler, {[}18. 8. 1894{]}}\nopagebreak\mylabel{v}\rehead{ }\normalsize\beginnumbering\briefempfaengerindex{Schnitzler, Arthur@\textsc{Schnitzler, Arthur}!zzzBeer-Hofmann, Richard@\emph{von Richard Beer-Hofmann}!1894-08-181@{{[}18. 8. 1894{]}}|(be} \toendnotes[C]{\smallbreak\pagebreak[2]} \Standort{CUL, Schnitzler, B 8.}
\physDesc{Briefkarte
\newline{}Handschrift: blauer Buntstift, lateinische Kurrent
\newline{}Schnitzler: mit Bleistift datiert: »18/8 94« und nummeriert: »35« }\toendnotes[C]{\smallbreak}\pstart
           \noindent{}{\pb}Lieber Arthur! Also
                  \textcolor{blue}{Goldmann}{}\ledrightnote{\textcolor{blue}{Paul Goldmann}}{ }ko{\geminationm}t. \label{K_L00362_1v}\edtext{Prosceniumsloge}{\lemma{\textnormal{\emph{Prosceniumsloge}}}\Cendnote{\textnormal{seitlich der Vorderbühne befindliche Logen, die sich gut für
                  Repräsentationszwecke eignen.}}}\label{K_L00362_1h} links sowie die daran anstossenden Logen sind
               Saison über in festen Händen. Zu haben ist {\pb}nur die rechte Prosceniumsloge die
               bei erhöhten Preisen 18 fl. kostet und die \strikeout{daran} mit
               2. (\label{T_L00362_1v}\edtext{rechts}{\lemma{\textnormal{\emph{rechts}}}\Cendnote{\textnormal{in deutscher Kurrentschrift}}}\label{T_L00362_1h}) \label{K_L00362_2v}\edtext{bezeichnete Loge}{\lemma{\textnormal{\emph{bezeichnete Loge}}}\Cendnote{\textnormal{Eine Skizze illustriert die Lage der Loge, es ist die dritte seitlich von der
                  Bühne aus gesehen.}}}\label{K_L00362_2h} die 12 fl kostet; welche soll ich nehmen? Ko{\geminationm}en Sie bald?\pend
           \pstart Herzlichst Ihr \spacefill\mbox{Rich}\pend{}\endnumbering\briefempfaengerindex{Schnitzler, Arthur@\textsc{Schnitzler, Arthur}!zzzBeer-Hofmann, Richard@\emph{von Richard Beer-Hofmann}!1894-08-181@{{[}18. 8. 1894{]}}|)be}\mylabel{h}  \normalsize

\doendnotes{C}
\bigskip
\vfill

\clearpage

\footnotesize

\lohead{\textsc{register}}

% Definiere theindex-Environment komplett neu ohne reledmac
\makeatletter
\renewenvironment{theindex}{%
  \section*{\indexname}%
  \setlength{\parindent}{0pt}%
  \setlength{\parskip}{0pt plus 0.3pt}%
  \let\item\@idxitem
}{%
  \clearpage
}
\makeatother

\IfFileExists{\jobname-pw.ind}{\input{\jobname-pw.ind}}{}

\end{document}

      