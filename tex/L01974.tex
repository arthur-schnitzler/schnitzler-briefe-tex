%% latex-korrekturansicht-vorspann.tex
%% Vorspann für die Korrekturansicht.
%% Lädt die gemeinsame Datei latex-vorspann.tex mit gesetztem Schalter.

\newif\ifkorrekturansicht
\korrekturansichttrue

\input{../tex-inputs/latex-vorspann}


               \section[Arthur Schnitzler an Robert Adam, 31. 10. 1910]{ Arthur Schnitzler an Robert Adam, 31. 10. 1910}\nopagebreak\mylabel{v}\rehead{ }\normalsize\beginnumbering\briefempfaengerindex{Adam, Robert@\textsc{Adam, Robert}!zzzSchnitzler, Arthur@\emph{von Arthur Schnitzler}!1910-10-311@{31. 10. 1910}|(be} \toendnotes[C]{\smallbreak\pagebreak[2]} \Standort{DLA, 96.34.1/2.}
\physDesc{Brief, 1 Blatt, 1 Seite, Umschlag
\newline{}Schreibmaschine
\newline{}Handschrift: schwarze Tinte (\noindent{}Unterschrift)\newline{}Versand: Stempel: »\nobreak{}\oindex{IX., Alsergrund@\textbf{IX., Alsergrund}, \emph{Bezirk (A.BZK)}|pwk}9/1 Wien 66, 31. X. 10, 3\nobreak{}«.  }\Standort{DLA, A:Schnitzler, 85.1.1621.}
\physDesc{Brief, 1 Blatt, 1 Seite, Umschlag, maschineller Durchschlag
\newline{}Schreibmaschine
\newline{}Handschrift: Bleistift, lateinische Kurrent (\noindent{}»Adam«)}\toendnotes[C]{\smallbreak}\pstart{}{\pb}\textcolor{gray}{\textbf{Dr. Arthur Schnitzler}}\pend{}\pstart{}\textcolor{pink}{\textcolor{gray}{\textbf{Wien, XVIII. Sternwartestrasse 71}}}{}\ledrightnote{\textcolor{pink}{Sternwartestraße}}\pend{}{\bigskip}\pstart{}{\pb}Herrn Robert Adam\pend{}\pstart{}\textcolor{pink}{\so{Wien XII}}{}\ledrightnote{\textcolor{pink}{XII., Meidling}}.\pend{}\pstart{}\textcolor{pink}{Meidlinger Hauptstraße 56}{}\ledrightnote{\textcolor{pink}{Meidlinger Hauptstraße}}.\pend{}{\bigskip}\pstart
           {\pb}\textcolor{gray}{\textbf{Dr. Arthur Schnitzler}}\hfill 31. 10. 1910.\pend
           \pstart
           \textcolor{gray}{\textbf{\textcolor{pink}{Wien XVIII. Sternwartestrasse 71}{}\ledrightnote{\textcolor{pink}{Sternwartestraße}}}}\pend
           \pstart\center{}Sehr geehrter Herr Adam.\pend\pstart
           Ihrer \introOben{}»\introOben{}\textcolor{green}{Ali Bekkar\introOben{}«\introOben{}-Komödie}{}\ledrightnote{\textcolor{green}{Die Geschichte des Alî ibn Bekkâr mit Schams an-Nahâr}}
               erinnere ich gern und so interessiert es mich natürlich auch Ihr neues \textcolor{green}{Stück}{}\ledrightnote{→\textcolor{green}{Neidhard}} kennen zu lernen, nur möchte
               bitten 1. es mir in Schreibmaschinenschrift zu\introOben{}{ }\introOben{}senden zu wollen und 2. sich fre{[}u{]}ndlichst bis nach meiner
                  \label{K_L01974_1v}\edtext{\textcolor{pink}{Burgtheater}{}\ledrightnote{\textcolor{pink}{Burgtheater}}première}{\lemma{\textnormal{\emph{Burgtheaterpremière}}}\Cendnote{\textnormal{am 24. 11. 1910 Uraufführung von \emph{\textcolor{green}{Der junge Medardus}}}}}\label{K_L01974_1h} zu gedulden.\pend
           \pstart
           Mit verbindlichem Gruss{\\[\baselineskip]}Ihr sehr ergebener\spacefill\mbox{{[}hs.:{]} Arthur Schnitzler}\pend
           \leftskip=0em{}\pstart
           \noindent{}{[}ms.:{]} Herrn Robert Adam, \textcolor{pink}{Wien}{}\ledrightnote{\textcolor{pink}{Wien}}.\pend
           \endnumbering\briefempfaengerindex{Adam, Robert@\textsc{Adam, Robert}!zzzSchnitzler, Arthur@\emph{von Arthur Schnitzler}!1910-10-311@{31. 10. 1910}|)be}\mylabel{h}  \normalsize

\doendnotes{C}
\bigskip
\vfill

\clearpage

\footnotesize

\lohead{\textsc{register}}

% Definiere theindex-Environment komplett neu ohne reledmac
\makeatletter
\renewenvironment{theindex}{%
  \section*{\indexname}%
  \setlength{\parindent}{0pt}%
  \setlength{\parskip}{0pt plus 0.3pt}%
  \let\item\@idxitem
}{%
  \clearpage
}
\makeatother

\IfFileExists{\jobname-pw.ind}{\input{\jobname-pw.ind}}{}

\end{document}

      