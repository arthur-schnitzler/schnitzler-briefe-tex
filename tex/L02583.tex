%% latex-korrekturansicht-vorspann.tex
%% Vorspann für die Korrekturansicht.
%% Lädt die gemeinsame Datei latex-vorspann.tex mit gesetztem Schalter.

\newif\ifkorrekturansicht
\korrekturansichttrue

\input{../tex-inputs/latex-vorspann}


               \section[Arthur Schnitzler an Auguste Hauschner, 23. 1. 1909]{ Arthur Schnitzler an Auguste Hauschner, 23. 1. 1909}\nopagebreak\mylabel{v}\rehead{ }\normalsize\beginnumbering\briefempfaengerindex{Hauschner, Auguste@\textsc{Hauschner, Auguste}!zzzSchnitzler, Arthur@\emph{von Arthur Schnitzler}!1909-01-231@{23. 1. 1909}|(be} \toendnotes[C]{\smallbreak\pagebreak[2]} \Standort{Staatsbibliothek Berlin – Preußischer Kulturbesitz, Handschriftenabteilung, Nachlass Auguste Hauschner.}
\physDesc{Brief, 1 Blatt, 4 Seiten
\newline{}Handschrift: schwarze Tinte, lateinische Kurrent
\newline{}Hauschner: mit rotem Buntstift eine Unterstreichung unter
                                    »tautologisch«, eventuell, weil die Entzifferung
                                 Probleme bereitete }\buchAbdrucke{\weitereDrucke{1) \pwindex{Brief an Auguste Hauschner, 23.1.1909]@\emph{[Brief an Auguste Hauschner, 23.1.1909]}|pwk}\pwindex{Briefe an Auguste Hauschner@\emph{Briefe an Auguste Hauschner}|pwk}Arthur Schnitzler: \emph{[Brief an Auguste Hauschner zum Weg ins Freie].} In: \emph{Briefe an Auguste Hauschner}. Hg. Martin Beradt und Lotte Bloch-Zavřel. Berlin: \emph{Ernst Rowohlt Verlag} [Ende Oktober 1928, vordatiert auf:]
                           1929, S. 106.} \weitereDrucke{2) Arthur Schnitzler: \emph{Briefe 1875–1912}. Hg. Therese Nickl und Heinrich Schnitzler. Frankfurt am Main: \emph{S. Fischer} 1981, S. 588.} }\toendnotes[C]{\smallbreak}\pstart
           {\pb}\textcolor{gray}{\textbf{Dr. Arthur Schnitzler}}\hfill 23. 1. 09\pend
           \pstart{}verehrte Frau, \pend\pstart
           ich danke Ihnen sehr, dass Sie mir Ihren schönen \label{K_M178-1v}\edtext{\textcolor{green}{Artikel}{}\ledrightnote{→\textcolor{green}{Der Weg ins Freie}}}{\lemma{\textnormal{\emph{Artikel}}}\Cendnote{\textnormal{\textcolor{blue}{Auguste Hauschner}: \emph{\textcolor{green}{Der Weg ins
                        Freie}}. In: \emph{\textcolor{green}{Die Hilfe}}, Jg. 15,
                     Nr. 3, 17. 1. 1909, S. 39–40. \textcolor{blue}{Schnitzler} urteilte im \emph{\textcolor{green}{Tagebuch}} am 15. 1. 1909: »Neue Kritikensammlung, von \textcolor{blue}{Fischer} gesandt, über den Weg. Die \textcolor{blue}{Hauschner}, fand endlich in der ›\textcolor{green}{Hilfe}‹ eine Stätte für ihren mir nun erst bekannt
                     werdenden sehr freundlichen Aufsatz.«}}}\label{K_M178-1h} geschickt haben. Gar viel
               wäre darüber zu sagen, wenn es mir nicht so fatal wäre, über meine eignen Sachen was
               niederzuschreiben. Reden kö{\geminationn}t ich schon eher drüber, nun
               vielleicht fügt es mein gutes Glück, dass {\pb}ich Ihnen irgend einmal in der Welt
               begegne. Übrigens, einfacher: we{\geminationn} Sie nach \textcolor{pink}{Wien}{}\ledrightnote{\textcolor{pink}{Wien}} kommen, lassen Sie michs wissen, gnädige Frau,
               und we{\geminationn} ich nach \textcolor{pink}{Berlin}{}\ledrightnote{\textcolor{pink}{Berlin}}
               komme, darf ich mich wohl auch melden –? Vorher aber noch möcht ich Ihnen sagen,
               daß Sie Unrecht haben Ihren \label{K_M178-11v}\edtext{Schluss »mislungen«}{\lemma{\textnormal{\emph{Schluss »mislungen«}}}\Cendnote{\textnormal{siehe Auguste Hauschner an Arthur Schnitzler, 16. 1. 1909}}}\label{K_M178-11h} zu finden – auch ohne Ihren Brief {\pb}wüßt ich sehr gut, was Sie eigentlich
               sagen wollten. Und so viel tief und liebevoll (oder ist das tautologisch?)
               eindringendes in den vorherigen Absätzen. Wie
               viele Leseri{\geminationn}en Ihrer Art denken Sie gibt es wohl? Und
               gar eine, die zugleich Künstlerin ist { }{\dotsfive} jetzt aber ko{\geminationm}t es immer
               näher, – noch drei Zeilen, und ich fange an etwas über mein {\pb}\textcolor{green}{Buch}{}\ledrightnote{→\textcolor{green}{Der Weg ins Freie. Roman}} zu sagen – daher nicht mehr als dies: Sie haben mir durch
               gedrucktes geschriebenes und gefühltes
               herzliche Freude bereitet!\pend
           \pstart
           Ihr aufrichtig ergebner{\\[\baselineskip]}\spacefill\mbox{Arthur Schnitzler}\pend
           \leftskip=0em{}\endnumbering\briefempfaengerindex{Hauschner, Auguste@\textsc{Hauschner, Auguste}!zzzSchnitzler, Arthur@\emph{von Arthur Schnitzler}!1909-01-231@{23. 1. 1909}|)be}\mylabel{h}  \normalsize

\doendnotes{C}
\bigskip
\vfill

\clearpage

\footnotesize

\lohead{\textsc{register}}

% Definiere theindex-Environment komplett neu ohne reledmac
\makeatletter
\renewenvironment{theindex}{%
  \section*{\indexname}%
  \setlength{\parindent}{0pt}%
  \setlength{\parskip}{0pt plus 0.3pt}%
  \let\item\@idxitem
}{%
  \clearpage
}
\makeatother

\IfFileExists{\jobname-pw.ind}{\input{\jobname-pw.ind}}{}

\end{document}

      