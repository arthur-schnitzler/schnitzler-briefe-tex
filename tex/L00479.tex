%% latex-korrekturansicht-vorspann.tex
%% Vorspann für die Korrekturansicht.
%% Lädt die gemeinsame Datei latex-vorspann.tex mit gesetztem Schalter.

\newif\ifkorrekturansicht
\korrekturansichttrue

\input{../tex-inputs/latex-vorspann}


               \section[Arthur Schnitzler an Hugo von Hofmannsthal, 1. 9. 1895]{ Arthur Schnitzler an Hugo von Hofmannsthal, 1. 9. 1895}\nopagebreak\mylabel{v}\rehead{ }\normalsize\beginnumbering\briefempfaengerindex{Hofmannsthal, Hugo von@\textsc{Hofmannsthal, Hugo von}!zzzSchnitzler, Arthur@\emph{von Arthur Schnitzler}!1895-09-011@{1. 9. 1895}|(be} \toendnotes[C]{\smallbreak\pagebreak[2]} \Standort{FDH, Hs-30885,46.}
\physDesc{Brief, 2 Blätter, 6 Seiten
\newline{}Handschrift: schwarze Tinte, deutsche Kurrent\newline{}Ordnung: 1) von Schnitzler mit Bleistift mutmaßlich bei der Durchsicht der Korrespondenz 1929 auf dem ersten Blatt
                                            datiert: »1/9 95« 2) mit Bleistift von unbekannter Hand beschriftet: »\textcolor{pink}{München}«}\buchAbdrucke{\weitereDrucke{1) Hugo von Hofmannsthal, Arthur Schnitzler: \emph{Briefwechsel}. Hg. Therese Nickl und Heinrich Schnitzler. Frankfurt am Main: \emph{S. Fischer} 1964, S. 61–62.} \weitereDrucke{2) Arthur Schnitzler: \emph{Briefe 1875–1912}. Hg. Therese Nickl und Heinrich Schnitzler. Frankfurt am Main: \emph{S. Fischer} 1981, S. 275–276.} }\pstart
           \noindent{}{\pb}Lieber Hugo. Von \textcolor{pink}{Salzburg}{}\ledrightnote{\textcolor{pink}{Salzburg}} aus,
                    wo \textcolor{blue}{Richard}{}\ledrightnote{\textcolor{blue}{Richard Beer-Hofmann}}, \textcolor{blue}{\textsc{Salten}}{}\ledrightnote{\textcolor{blue}{Felix Salten}} u. die \textcolor{blue}{\textsc{Salomé}}{}\ledrightnote{\textcolor{blue}{Lou Andreas-Salomé}} zuſa{\geminationm}en waren, fuhren ich u. \textcolor{blue}{S.}{}\ledrightnote{\textcolor{blue}{Felix Salten}} per Rad davon. Das war ſehr ſchön. Man hat ſchon ganz
                    aufgehört, ſo mitten durch Dörfer und Flecken zu fahren, mitten dur\damage{ch} das Leben und die Naivität \damage{\textcolor{gray}{eines Ortes}}. Von Stationen aus, wo ſich naturgemäß künſtliches ſa{\geminationm}elt, ſieht man das alles ſchief. Auch die
                    Landſtraßen werden wieder lebendig, wachen auf, und man gehört mit zu den
                    Erweckenden. Auch Zufälle gibt es wieder, und, das beſte, man hält den Zug an,
                    wo es beliebt. {\pb}Dagegen fällt das mancherlei
                    unangenehme, dſs es regnen kann und daſs man naſs u kotig wird u ſtürzt, wenig
                    ins Gewicht. Wir hatten darunter genug zu leiden, mußten ſogar in einem Zollhaus
                    ſtundenlang ein beſſres Wetter abwarten. Amüſant war es, wie gerade an der \textcolor{pink}{bair}{}\ledrightnote{\textcolor{pink}{Bayern}}-\textcolor{pink}{oeſterr}{}\ledrightnote{\textcolor{pink}{Österreich}}
                    Grenze, zwiſchen \textcolor{pink}{Reichenhall}{}\ledrightnote{\textcolor{pink}{Bad Reichenhall}} u \textcolor{pink}{Lofer}{}\ledrightnote{\textcolor{pink}{Lofer}}, \textcolor{blue}{Burckhard}{}\ledrightnote{\textcolor{blue}{Max Eugen Burckhard}} auf einem Rad entgegenkam, der von \textcolor{pink}{Innsbruck}{}\ledrightnote{\textcolor{pink}{Innsbruck}} nach \textcolor{pink}{Iſchl}{}\ledrightnote{\textcolor{pink}{Bad Ischl}}
                    fuhr. Bei dieſem Menſchen iſt eine Miſchung von »reinem Thoren« und gefinkeltem
                    Diplomaten ſehr intereſſant, welche mir i{\geminationm}er
                    zweifelloſer {\pb}wird. Sein perſönlicher \textsc{Charme} iſt vielleicht dieſes Durchleuchtetwerden eines
                    verworrenen bunten ſelbſt trüben Äußern von innen her.\pend
           \pstart
           Worüber noch einiges zu ſagen wäre. Hier, in \textcolor{pink}{M.}{}\ledrightnote{\textcolor{pink}{München}} bin ich ſeit Donnerſtag mit \textcolor{blue}{Paul Gldm.}{}\ledrightnote{\textcolor{blue}{Paul Goldmann}} zuſa{\geminationm}en, der ſehr
                    gut ausſieht, aber mit Schickſal und Ausſichten wenig zufrieden iſt und
                    insbeſondere daran leidet, daſs er ſeine eigene Thätigkeit nicht genügend
                    ſchätzt, weil ſie nicht in der wünſchenswerthen Weiſe anerkannt wird. Iſt
                    übrigens wie i{\geminationm}er voll Verſtand, Verſtändnis,
                    Herzlichkeit, Freude am Schönen; wohlthuend in dem, was er bringt, und in {\pb}der Art wie er aufni{\geminationm}t. Seit geſtern Abend iſt auch \textcolor{blue}{Richard}{}\ledrightnote{\textcolor{blue}{Richard Beer-Hofmann}} da,
                    und die \textcolor{blue}{Salomé}{}\ledrightnote{\textcolor{blue}{Lou Andreas-Salomé}}{ }ſoll am 3. od. 4.
                        ko{\geminationm}en. – Im \textcolor{pink}{Glaspalaſt}{}\ledrightnote{\textcolor{pink}{Glaspalast}} iſt ſehr wenig gutes, viel mittelmäßiges und zu viel
                    ſchlechtes. Viel mehr iſt in der \textcolor{pink}{\textsc{Secession}}{}\ledrightnote{\textcolor{pink}{Internationalen Kunst-Ausstellung des Vereins bildender Künstler Münchens »Secession«}} zu ſehn; manches, das weit über den Schweinen und weit über den
                    Schnapsflaſchen des techniſch ausgezeichneten \textcolor{blue}{\textsc{Heyden}}{}\ledrightnote{\textcolor{blue}{Hubert Heyden}}{ }ſteht. Die \textcolor{green}{Meiſterſinger}{}\ledrightnote{\textcolor{green}{Die Meistersinger von Nürnberg}} hab ich ſchon einmal gehört, heute wieder. Neulich \textcolor{green}{Triſtan}{}\ledrightnote{\textcolor{green}{Tristan und Isolde}}, dem arger Schade zugefügt wird, indem
                    man ſich einbildet, ihn ungekürzt geben zu können oder gar zu müſſen. An den \textcolor{green}{Geſchwiſter}{}\ledrightnote{\textcolor{green}{Die Geschwister}}n u am \textcolor{green}{\textsc{Clavigo}}{}\ledrightnote{\textcolor{green}{Clavigo}} hab ich mich trotz vieler Mängel der Darſtellung {\pb}neulich tief erfreut. Zum erſten Mal (in den \textcolor{green}{Geſchwiſter}{}\ledrightnote{\textcolor{green}{Die Geschwister}}n) die \textcolor{blue}{Conrad-Ramlo}{}\ledrightnote{\textcolor{blue}{Marie Conrad-Ramlo}} geſehn, die viel zu bedeuten ſcheint. –
                    Heute wird \textcolor{pink}{Sedan}{}\ledrightnote{\textcolor{pink}{Sedan}} gefeiert; Fahnen, Wimpeln,
                    Feſtzeitungen, Feſtvorſtellungen, Menſchen auf der Straße hin u her, geſchmückte
                    Stadt – wohl auch einige von Stolz und Begeiſterung geſchwellte Herzen, die man
                    zum Glück nicht ſieht. Das andre aber iſt ein helles und freundliches Bild.\pend
           \pstart
           – Freitag den 6. werde ich wohl wieder in \textcolor{pink}{Wien}{}\ledrightnote{\textcolor{pink}{Wien}}{ }ſein; ſchreiben Sie mir von den Manövern aus,
                    wenn Sie Zeit haben, noch eine Zeile dahin. Sagen Sie, wie iſt de{\geminationn} eigentlich {\pb}Ihr
                    Rennen ausgefallen? –\pend
           \pstart
           Von \textcolor{blue}{Paul}{}\ledrightnote{\textcolor{blue}{Paul Goldmann}} u \textcolor{blue}{Richard}{}\ledrightnote{\textcolor{blue}{Richard Beer-Hofmann}}, wie von mir die herzlichſten Grüße. Jetzt wollen wir, vor der
                    Oper, nach \textcolor{pink}{\textsc{Nymphenburg}}{}\ledrightnote{\textcolor{pink}{Neuhausen-Nymphenburg}} fahren.\pend
           \pstart Ihr \spacefill\mbox{Arthur}\pend{}\pstart
           \textcolor{pink}{München}{}\ledrightnote{\textcolor{pink}{München}},
                        1. Sept. 95.\pend
           \endnumbering\briefempfaengerindex{Hofmannsthal, Hugo von@\textsc{Hofmannsthal, Hugo von}!zzzSchnitzler, Arthur@\emph{von Arthur Schnitzler}!1895-09-011@{1. 9. 1895}|)be}\mylabel{h}  \normalsize

\doendnotes{C}
\bigskip
\vfill

\clearpage

\footnotesize

\lohead{\textsc{register}}

% Definiere theindex-Environment komplett neu ohne reledmac
\makeatletter
\renewenvironment{theindex}{%
  \section*{\indexname}%
  \setlength{\parindent}{0pt}%
  \setlength{\parskip}{0pt plus 0.3pt}%
  \let\item\@idxitem
}{%
  \clearpage
}
\makeatother

\IfFileExists{\jobname-pw.ind}{\input{\jobname-pw.ind}}{}

\end{document}

      