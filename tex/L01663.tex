%% latex-korrekturansicht-vorspann.tex
%% Vorspann für die Korrekturansicht.
%% Lädt die gemeinsame Datei latex-vorspann.tex mit gesetztem Schalter.

\newif\ifkorrekturansicht
\korrekturansichttrue

\input{../tex-inputs/latex-vorspann}


               \section[Hugo von Hofmannsthal an Arthur Schnitzler, {[}13. 3. 1907{]}]{ Hugo von Hofmannsthal an Arthur Schnitzler, {[}13. 3. 1907{]}}\nopagebreak\mylabel{v}\rehead{ }\normalsize\beginnumbering\briefempfaengerindex{Schnitzler, Arthur@\textsc{Schnitzler, Arthur}!zzzHofmannsthal, Hugo von@\emph{von Hugo von Hofmannsthal}!1907-03-131@{{[}13. 3. 1907{]}}|(be} \toendnotes[C]{\smallbreak\pagebreak[2]} \Standort{CUL, Schnitzler, B 43.}
\physDesc{Brief, 1 Blatt, 2 Seiten
\newline{}Handschrift: schwarze Tinte, deutsche Kurrent
\newline{}Schnitzler: mit Bleistift datiert: »13/3 907« \newline{}Ordnung: 1) mit Bleistift von unbekannter Hand nummeriert: »\strikeout{268}« 2) mit Bleistift von unbekannter Hand nummeriert: »272«}\buchAbdrucke{\weitereDrucke{Hugo von Hofmannsthal, Arthur Schnitzler: \emph{Briefwechsel}. Hg. Therese Nickl und Heinrich Schnitzler. Frankfurt am Main: \emph{S. Fischer} 1964, S. 227.} }\toendnotes[C]{\smallbreak}\pstart{}{\pb}mein lieber
                  Arthur\pend\pstart
           bitte ſchreiben Sie mir \label{T_L01663_1v}\edtext{\uuline{gleich}}{\lemma{\textnormal{\emph{gleich}}}\Cendnote{\textnormal{nachträglich mit Bleistift eingefügt}}}\label{T_L01663_1h}{ }\uline{pneumatiſch} nach \textsc{\textcolor{pink}{Wien}{}\ledrightnote{\textcolor{pink}{Wien}}{ }\textcolor{pink}{I Hi{\geminationm}elpfortgasse 17}{}\ledrightnote{\textcolor{pink}{Himmelpfortgasse}}} ob das \textcolor{brown}{Burgtheater}{}\ledrightnote{\textcolor{brown}{Burgtheater}}
               auf das Stück »\textcolor{green}{Fuchs}{}\ledrightnote{\textcolor{green}{Fuchs. Schauspiel in einem Act}}« von mir, das es ſeit 5 Jahren
               nicht geſpielt hat, {\pb}noch Rechte
               haben kann. Dieſes Recht kann doch nicht auf ewige Zeiten gelten.\pend
           \pstart
           Ihr{\\[\baselineskip]}\spacefill\mbox{Hugo}\pend
           \leftskip=0em{}\endnumbering\briefempfaengerindex{Schnitzler, Arthur@\textsc{Schnitzler, Arthur}!zzzHofmannsthal, Hugo von@\emph{von Hugo von Hofmannsthal}!1907-03-131@{{[}13. 3. 1907{]}}|)be}\mylabel{h}  \normalsize

\doendnotes{C}
\bigskip
\vfill

\clearpage

\footnotesize

\lohead{\textsc{register}}

% Definiere theindex-Environment komplett neu ohne reledmac
\makeatletter
\renewenvironment{theindex}{%
  \section*{\indexname}%
  \setlength{\parindent}{0pt}%
  \setlength{\parskip}{0pt plus 0.3pt}%
  \let\item\@idxitem
}{%
  \clearpage
}
\makeatother

\IfFileExists{\jobname-pw.ind}{\input{\jobname-pw.ind}}{}

\end{document}

      