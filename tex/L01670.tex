%% latex-korrekturansicht-vorspann.tex
%% Vorspann für die Korrekturansicht.
%% Lädt die gemeinsame Datei latex-vorspann.tex mit gesetztem Schalter.

\newif\ifkorrekturansicht
\korrekturansichttrue

\input{../tex-inputs/latex-vorspann}


               \section[Hermann Bahr an Arthur Schnitzler, 26. 4. 1907]{ Hermann Bahr an Arthur Schnitzler, 26. 4. 1907}\nopagebreak\mylabel{v}\rehead{ }\normalsize\beginnumbering\briefempfaengerindex{Schnitzler, Arthur@\textsc{Schnitzler, Arthur}!zzzBahr, Hermann@\emph{von Hermann Bahr}!1907-04-261@{26. 4. 1907}|(be} \toendnotes[C]{\smallbreak\pagebreak[2]} \Standort{CUL, Schnitzler, B 5b.}
\physDesc{Brief, 1 Blatt, 1 Seite
\newline{}Handschrift Lisa Clarus: blaue Tinte, lateinische Kurrent\newline{}Handschrift Hermann Bahr: schwarze Tinte (\noindent{}Unterschrift)\newline{}Ordnung: mit Bleistift von unbekannter Hand nummeriert:
                                    »147« }\buchAbdrucke{\weitereDrucke{Hermann Bahr, Arthur Schnitzler: \emph{Briefwechsel, Aufzeichnungen, Dokumente (1891–1931)}. Hg. Kurt Ifkovits und Martin Anton Müller. Göttingen: \emph{Wallstein} 2018, S. 392.} }\toendnotes[C]{\smallbreak}\pstart
           \raggedleft{}{\pb}26. 4. 07\pend
           \pstart{}Lieber Arthur!\pend\pstart
           Möchtest Du so lieb sein, mir auch noch den \textcolor{green}{zweiten Band
                  Brehm}{}\ledrightnote{\textcolor{green}{Brehms Tierleben}} zu schicken? Du kriegst dann beide zusammen in ein paar Tagen zurück.
               Ich hoffe nun in der nächsten Woche, wahrscheinlich Dienstag oder Mittwoch, meine
                  \label{K_L01670_1v}\edtext{Forschungsreise}{\lemma{\textnormal{\emph{Forschungsreise}}}\Cendnote{\textnormal{Möglicherweise eine doppelte Anspielung:
                  einerseits auf \textcolor{blue}{Bahrs} Interesse für die
                  südlichen österreichischen Provinzen, andererseits auf die Suche nach einer
                  Sommervilla. \textcolor{blue}{Bahr} traf am
                     3. 5. 1907, also später als angekündigt, in \textcolor{pink}{Triest} ein und kündigte am 8. 5. 1907 die Weiterfahrt
                  nach \textcolor{pink}{Sistiana} an.}}}\label{K_L01670_1h} nach \textcolor{pink}{Fiume}{}\ledrightnote{\textcolor{pink}{Rijeka}} und \textcolor{pink}{Triest}{}\ledrightnote{\textcolor{pink}{Triest}} zu machen.
               Kommst Du mit?\pend
           \pstart
           Mit den besten Grüssen an Deine \textcolor{blue}{Frau}{}\ledrightnote{→\textcolor{blue}{Olga Schnitzler}},{\\[\baselineskip]}herzlichst{\\[\baselineskip]}\spacefill\mbox{{[}hs. Bahr:{]} HermannB}\pend
           \leftskip=0em{}\endnumbering\briefempfaengerindex{Schnitzler, Arthur@\textsc{Schnitzler, Arthur}!zzzBahr, Hermann@\emph{von Hermann Bahr}!1907-04-261@{26. 4. 1907}|)be}\mylabel{h}  \normalsize

\doendnotes{C}
\bigskip
\vfill

\clearpage

\footnotesize

\lohead{\textsc{register}}

% Definiere theindex-Environment komplett neu ohne reledmac
\makeatletter
\renewenvironment{theindex}{%
  \section*{\indexname}%
  \setlength{\parindent}{0pt}%
  \setlength{\parskip}{0pt plus 0.3pt}%
  \let\item\@idxitem
}{%
  \clearpage
}
\makeatother

\IfFileExists{\jobname-pw.ind}{\input{\jobname-pw.ind}}{}

\end{document}

      