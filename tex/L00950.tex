%% latex-korrekturansicht-vorspann.tex
%% Vorspann für die Korrekturansicht.
%% Lädt die gemeinsame Datei latex-vorspann.tex mit gesetztem Schalter.

\newif\ifkorrekturansicht
\korrekturansichttrue

\input{../tex-inputs/latex-vorspann}


               \section[Gerhart Hauptmann an Arthur Schnitzler, {[}25.?{]} 7. 1899]{ Gerhart Hauptmann an Arthur Schnitzler, {[}25.?{]} 7. 1899}\nopagebreak\mylabel{v}\rehead{ }\normalsize\beginnumbering\briefempfaengerindex{Schnitzler, Arthur@\textsc{Schnitzler, Arthur}!zzzHauptmann, Gerhart@\emph{von Gerhart Hauptmann}!1899-07-251@{{[}25.?{]} 7. 1899}|(be} \toendnotes[C]{\smallbreak\pagebreak[2]} \Standort{DLA, A:Schnitzler, 66.206.}
\physDesc{Brief, 1 Blatt, 1 Seite
\newline{}Handschrift: schwarze Tinte, lateinische Kurrent
\newline{}Schnitzler: mit Bleistift datiert: »Juli 99« \newline{}Ordnung: mit Bleistift von unbekannter Hand seitlich am Blatt: »{\char`~} e\textcolor{gray}{v.}« }\toendnotes[C]{\smallbreak}\pstart{}{\pb}Lieber Herr Schnitzler.\pend\pstart
           ich empfing erst \label{K_L00950_1v}\edtext{\textcolor{pink}{hier}{}\ledrightnote{\textcolor{pink}{Szklarska Poręba}}}{\lemma{\textnormal{\emph{hier}}}\Cendnote{\textnormal{\textcolor{blue}{Hauptmann} kam am 24. 7. 1899 nach
                     \textcolor{pink}{Schreiberhau}, wo er das Korrespondenzstück
                  vorfand. Er dürfte es an einem der darauffolgenden Tage beantwortet haben.}}}\label{K_L00950_1h}
               Ihren Brief. Sie sind so liebenswürdig und es ist mir so schwer, Ihnen etwas
               abzuschlagen. Aber das kann ich ja gar nicht thun, was Sie wünschen. Wäre ich in \textcolor{pink}{Wien}{}\ledrightnote{\textcolor{pink}{Wien}}! Allein ich bin ja meistens weit weg und fühle zu
               genau, dass es über meine Kräfte geht, in der Weise mitzuwirken, wie es sein müsste,
               wenn ich meinen Namen auf dem Blatttitel rechtfertigen sollte.\pend
           \pstart
           Seien Sie mir gegrüsst. Ich denke oft an unsern \label{K_L00950_2v}\edtext{Spaziergang}{\lemma{\textnormal{\emph{Spaziergang}}}\Cendnote{\textnormal{vgl. A. S.: \emph{Tagebuch}, 22. 1. 1899}}}\label{K_L00950_2h} auf dem \textcolor{pink}{Semmering}{}\ledrightnote{\textcolor{pink}{Semmering}} und hoffe herzlich, Sie
               bald einmal, und am liebsten ausserhalb der Stadtmauern, wiederzusehen\pend
           \pstart Viele Grüsse von Ihrem ergebenen \spacefill\mbox{Gerhart Hauptmann}\pend{}\endnumbering\briefempfaengerindex{Schnitzler, Arthur@\textsc{Schnitzler, Arthur}!zzzHauptmann, Gerhart@\emph{von Gerhart Hauptmann}!1899-07-251@{{[}25.?{]} 7. 1899}|)be}\mylabel{h}  \normalsize

\doendnotes{C}
\bigskip
\vfill

\clearpage

\footnotesize

\lohead{\textsc{register}}

% Definiere theindex-Environment komplett neu ohne reledmac
\makeatletter
\renewenvironment{theindex}{%
  \section*{\indexname}%
  \setlength{\parindent}{0pt}%
  \setlength{\parskip}{0pt plus 0.3pt}%
  \let\item\@idxitem
}{%
  \clearpage
}
\makeatother

\IfFileExists{\jobname-pw.ind}{\input{\jobname-pw.ind}}{}

\end{document}

      