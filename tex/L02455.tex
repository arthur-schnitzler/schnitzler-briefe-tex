%% latex-korrekturansicht-vorspann.tex
%% Vorspann für die Korrekturansicht.
%% Lädt die gemeinsame Datei latex-vorspann.tex mit gesetztem Schalter.

\newif\ifkorrekturansicht
\korrekturansichttrue

\input{../tex-inputs/latex-vorspann}


               \section[Arthur Schnitzler an Hugo Hofmannsthal, 16. 11. 1925]{ Arthur Schnitzler an Hugo Hofmannsthal, 16. 11. 1925}\nopagebreak\mylabel{v}\rehead{ }\normalsize\beginnumbering\briefempfaengerindex{Hofmannsthal, Hugo von@\textsc{Hofmannsthal, Hugo von}!zzzSchnitzler, Arthur@\emph{von Arthur Schnitzler}!1925-11-161@{16. 11. 1925}|(be} \toendnotes[C]{\smallbreak\pagebreak[2]} \Standort{FDH, Hs-30885,154.}
\physDesc{Brief, 1 Blatt, 2 Seiten
\newline{}Handschrift: Bleistift, lateinische Kurrent}\buchAbdrucke{\weitereDrucke{Hugo von Hofmannsthal, Arthur Schnitzler: \emph{Briefwechsel}. Hg. Therese Nickl und Heinrich Schnitzler. Frankfurt am Main: \emph{S. Fischer} 1964, S. 303.} }\toendnotes[C]{\smallbreak}\pstart
           \raggedleft{}{\pb}\textcolor{pink}{Wien}{}\ledrightnote{\textcolor{pink}{Wien}}{ }16. 11. 925\pend
           \pstart
           mein lieber Hugo, Ihr schönes \textcolor{green}{Stück}{}\ledrightnote{\textcolor{green}{Der Turm. Ein Trauerspiel}} hab ich noch in \textcolor{pink}{Berlin}{}\ledrightnote{\textcolor{pink}{Berlin}} erhalten
                    und es ist recht unhöflich, daſs ich Ihnen nicht gleich gedankt habe. Mit ein
                    Grund ist gewesen, daſs ich erst in den letzten Tagen \introOben{}dazu
                        kam\introOben{} den \textcolor{green}{\textcolor{blue}{Calderon}{}\ledrightnote{\textcolor{blue}{Pedro Calderón de la Barca}}}{}\ledrightnote{→\textcolor{green}{Das Leben ein Traum}}, der Ihnen dazu eine Anregung gab, zu lesen, und es war mir im höchsten
                    Grad interessant, wie völlig neu und selbständig {[}Sie{]} Ihr
                        \textcolor{green}{Drama}{}\ledrightnote{→\textcolor{green}{Der Turm. Ein Trauerspiel}} geschrieben haben.
                    Nur einige äußere Momente sind erhalten; – nicht nur die Gestalten sind neu
                    geschaffen; – auch das Problem, das innere Licht ist etwas ganz neues geworden,
                    und völlig Ihr Eigentum. An manchen Stellen wünscht ich mir geringere
                    Weitläufigkeit, und der Humor des Dieners ist nicht durchaus nach meinem Sinn,
                        we{\geminationn} ich auch fühle, sehr im Stil des
                    ganzen.\pend
           \pstart
           Ich freue mich, dſs Sie in der Arbeit sind; auch ich bringe allerlei weiter. Eine
                    neue Novelle (»\textcolor{green}{Traumnovelle}{}\ledrightnote{\textcolor{green}{Traumnovelle}}«) erscheint bald;
                    mein Versstück »\textcolor{green}{Der Gang zum Weiher}{}\ledrightnote{\textcolor{green}{Der Gang zum Weiher. Dramatische Dichtung}}« ist
                    fertig; nun dictir ich eine weitere {\pb}\textcolor{green}{Novelle}{}\ledrightnote{→\textcolor{green}{Spiel im Morgengrauen. Novelle}}, deren Schluſs noch
                    unsicher ist; arbeite an einem \textcolor{green}{Roman}{}\ledrightnote{→\textcolor{green}{Therese. Chronik eines Frauenlebens}} (der richtiger eine Chronik zu nennen sein wird); und bringe
                    verschiedentliches \textcolor{green}{aphoristische}{}\ledrightnote{→\textcolor{green}{Der Geist im Wort und der Geist in der Tat}{\newline}→\textcolor{green}{Buch der Sprüche und Bedenken}} und \textcolor{green}{fragmentarisches}{}\ledrightnote{→\textcolor{green}{Der Geist im Wort und der Geist in der Tat}} in Ordnung so gut es geht, ja einzelnes gewissermaßen
                    in \textcolor{green}{Systeme}{}\ledrightnote{→\textcolor{green}{Buch der Sprüche und Bedenken}}. Theatralisch
                    liegt allerlei angefangnes vor, – was ich zuerst fertig machen werde, weiß ich
                    noch nicht.\pend
           \pstart
           Um Ihre \textcolor{pink}{Aussee}{}\ledrightnote{\textcolor{pink}{Bad Aussee}}r Abgeschiedenheit beneid ich Sie
                    manchmal – weiß aber nicht, ob ich \strikeout{von} trotz
                    zeitweiliger Sehnsucht nach etwas der Art lange aushalten würde. Es ist
                    mancherlei Unruhe in meinem Leben; im ganzen fühl ich mich wohl, bei
                    gelegentlichen, am häufigsten durch das Gehörleiden verursachten und geförderten
                    Depressionen.\pend
           \pstart
           Ich hoffe Sie bald wiederzusehen.\pend
           \pstart
           Seien Sie von Herzen gegrüßt und bedankt!{\\[\baselineskip]}Ihr{\\[\baselineskip]}\spacefill\mbox{Arthur.}\pend
           \leftskip=0em{}\endnumbering\briefempfaengerindex{Hofmannsthal, Hugo von@\textsc{Hofmannsthal, Hugo von}!zzzSchnitzler, Arthur@\emph{von Arthur Schnitzler}!1925-11-161@{16. 11. 1925}|)be}\mylabel{h}  \normalsize

\doendnotes{C}
\bigskip
\vfill

\clearpage

\footnotesize

\lohead{\textsc{register}}

% Definiere theindex-Environment komplett neu ohne reledmac
\makeatletter
\renewenvironment{theindex}{%
  \section*{\indexname}%
  \setlength{\parindent}{0pt}%
  \setlength{\parskip}{0pt plus 0.3pt}%
  \let\item\@idxitem
}{%
  \clearpage
}
\makeatother

\IfFileExists{\jobname-pw.ind}{\input{\jobname-pw.ind}}{}

\end{document}

      