%% latex-korrekturansicht-vorspann.tex
%% Vorspann für die Korrekturansicht.
%% Lädt die gemeinsame Datei latex-vorspann.tex mit gesetztem Schalter.

\newif\ifkorrekturansicht
\korrekturansichttrue

\input{../tex-inputs/latex-vorspann}


               \section[Hugo von Hofmannsthal an Arthur Schnitzler, {[}nach dem 11.? 1. 1899{]}]{ Hugo von Hofmannsthal an Arthur Schnitzler, {[}nach dem
               11.? 1. 1899{]}}\nopagebreak\mylabel{v}\rehead{ }\normalsize\beginnumbering\briefempfaengerindex{Schnitzler, Arthur@\textsc{Schnitzler, Arthur}!zzzHofmannsthal, Hugo von@\emph{von Hugo von Hofmannsthal}!1899-01-111@{{[}nach dem 11.? 1. 1899{]}}|(be} \toendnotes[C]{\smallbreak\pagebreak[2]} \Standort{CUL, Schnitzler, B 43.}
\physDesc{Brief, 1 Blatt, 1 Seite
\newline{}Handschrift: Bleistift, lateinische Kurrent
\newline{}Schnitzler: mit Bleistift datiert: »Jänner 99« \newline{}Ordnung: 1) mit Bleistift von unbekannter Hand nummeriert: »\strikeout{133}« 2) mit Bleistift von unbekannter Hand nummeriert:
                                    »132«}\buchAbdrucke{\weitereDrucke{Hugo von Hofmannsthal, Arthur Schnitzler: \emph{Briefwechsel}. Hg. Therese Nickl und Heinrich Schnitzler. Frankfurt am Main: \emph{S. Fischer} 1964, S. 117.} }\toendnotes[C]{\smallbreak}\pstart
           \noindent{}\centering{}{\pb}D\textsuperscript{r}
               Arthur Schnitzler\pend
           \pstart
           \noindent{}\centering{}\textcolor{pink}{Frankgasse 1}{}\ledrightnote{\textcolor{pink}{Frankgasse}}\pend
           \pstart
           \noindent{}\centering{}--------------\pend
           \pstart
           \noindent{}\centering{}\label{K_L00879_1v}\edtext{Kürzen!}{\lemma{\textnormal{\emph{Kürzen!}}}\Cendnote{\textnormal{Worauf sich dieser Zettel bezieht, ist unklar. Da die sonstige
                  Kommunikation keinen Anhaltspunkt bietet und \textcolor{blue}{Hofmannsthal} die ersten zehn Tage des Monats nicht in \textcolor{pink}{Wien} war, könnte es sich um eine schriftlich nachgereichte
                  Antwort nach einem persönlichen Treffen handeln. Diese fanden am 11.
                  und am 17. 1. 1899 statt.}}}\label{K_L00879_1h}\pend
           \endnumbering\briefempfaengerindex{Schnitzler, Arthur@\textsc{Schnitzler, Arthur}!zzzHofmannsthal, Hugo von@\emph{von Hugo von Hofmannsthal}!1899-01-111@{{[}nach dem 11.? 1. 1899{]}}|)be}\mylabel{h}  \normalsize

\doendnotes{C}
\bigskip
\vfill

\clearpage

\footnotesize

\lohead{\textsc{register}}

% Definiere theindex-Environment komplett neu ohne reledmac
\makeatletter
\renewenvironment{theindex}{%
  \section*{\indexname}%
  \setlength{\parindent}{0pt}%
  \setlength{\parskip}{0pt plus 0.3pt}%
  \let\item\@idxitem
}{%
  \clearpage
}
\makeatother

\IfFileExists{\jobname-pw.ind}{\input{\jobname-pw.ind}}{}

\end{document}

      