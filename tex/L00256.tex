%% latex-korrekturansicht-vorspann.tex
%% Vorspann für die Korrekturansicht.
%% Lädt die gemeinsame Datei latex-vorspann.tex mit gesetztem Schalter.

\newif\ifkorrekturansicht
\korrekturansichttrue

\input{../tex-inputs/latex-vorspann}


               \section[Arthur Schnitzler an Richard Beer-Hofmann, 18. 8. 1893]{ Arthur Schnitzler an Richard Beer-Hofmann, 18. 8. 1893}\nopagebreak\mylabel{v}\rehead{ }\normalsize\beginnumbering\briefempfaengerindex{Beer-Hofmann, Richard@\textsc{Beer-Hofmann, Richard}!zzzSchnitzler, Arthur@\emph{von Arthur Schnitzler}!1893-08-181@{18. 8. 1893}|(be} \toendnotes[C]{\smallbreak\pagebreak[2]} \Standort{YCGL, MSS 31.}
\physDesc{Brief, 1 Blatt (Briefpapier mit Trauerrand), 3 Seiten, Umschlag mit Trauerrand
\newline{}Handschrift: Bleistift, deutsche Kurrent\newline{}Versand: 1) Stempel: »\nobreak{}Wien 1/1, 18. 8. 93, 7 N\nobreak{}«.  2) Stempel: »\nobreak{}\oindex{Bad Ischl@\textbf{Bad Ischl}, \emph{Besiedelter Ort (A.BSO)}|pwk}Ischl, 19 8 93, 7 F\nobreak{}«. \newline{}Ordnung: mit Rotstift von unbekannter Hand in der linken oberen Ecke mit einem »X« versehen }\buchAbdrucke{\weitereDrucke{Arthur Schnitzler, Richard Beer-Hofmann: \emph{Briefwechsel 1891–1931}. Hg. Konstanze Fliedl. Wien, Zürich: \emph{Europaverlag} 1992, S. 51.} }\pstart{}{\pb}Hrn \textsc{Dr. Rich.
                     Beer-Hofmann}\pend{}\pstart{}\textsc{\textcolor{pink}{Ischl}{}\ledrightnote{\textcolor{pink}{Bad Ischl}}}\pend{}\pstart{}\textcolor{pink}{\textsc{Schulgasse 8}}{}\ledrightnote{\textcolor{pink}{Schulgasse}}\pend{}{\bigskip}\pstart{}{\pb}Lieber Richard –\pend\pstart
           Ich verreiſe Montag oder Dinſtag. Schreiben Sie mir vorher
               2 Zeilen. Ko{\geminationm}en Sie vor der Waffenübg nach \textcolor{pink}{Wien}{}\ledrightnote{\textcolor{pink}{Wien}}? –\pend
           \pstart
           Haben Sie was über {\pb}\textcolor{blue}{\textsc{Freund}}{}\ledrightnote{\textcolor{blue}{Carl Freund}} erfahren? –\pend
           \pstart
           – Ich treffe in \textcolor{pink}{\textsc{Lienz}}{}\ledrightnote{\textcolor{pink}{Lienz}} oder ſo wo mit \textcolor{blue}{\textsc{Salten}}{}\ledrightnote{\textcolor{blue}{Felix Salten}} zusa{\geminationm}en. –\pend
           \pstart
           Es ist unglaublich, dß Sie gar nicht ſchreiben.\pend
           \pstart
           Herzlichen Gruſs{\\[\baselineskip]}\spacefill\mbox{{\pb}Arthur}\pend
           \leftskip=0em{}\pstart
           \textcolor{pink}{Wien}{}\ledrightnote{\textcolor{pink}{Wien}}{ }18/8 93\pend
           \endnumbering\briefempfaengerindex{Beer-Hofmann, Richard@\textsc{Beer-Hofmann, Richard}!zzzSchnitzler, Arthur@\emph{von Arthur Schnitzler}!1893-08-181@{18. 8. 1893}|)be}\mylabel{h}  \normalsize

\doendnotes{C}
\bigskip
\vfill

\clearpage

\footnotesize

\lohead{\textsc{register}}

% Definiere theindex-Environment komplett neu ohne reledmac
\makeatletter
\renewenvironment{theindex}{%
  \section*{\indexname}%
  \setlength{\parindent}{0pt}%
  \setlength{\parskip}{0pt plus 0.3pt}%
  \let\item\@idxitem
}{%
  \clearpage
}
\makeatother

\IfFileExists{\jobname-pw.ind}{\input{\jobname-pw.ind}}{}

\end{document}

      