%% latex-korrekturansicht-vorspann.tex
%% Vorspann für die Korrekturansicht.
%% Lädt die gemeinsame Datei latex-vorspann.tex mit gesetztem Schalter.

\newif\ifkorrekturansicht
\korrekturansichttrue

\input{../tex-inputs/latex-vorspann}


               \section[Hermann Bahr an Arthur Schnitzler, 23. 12. 1896]{ Hermann Bahr an Arthur Schnitzler, 23. 12. 1896}\nopagebreak\mylabel{v}\rehead{ }\normalsize\beginnumbering\briefempfaengerindex{Schnitzler, Arthur@\textsc{Schnitzler, Arthur}!zzzBahr, Hermann@\emph{von Hermann Bahr}!1896-12-231@{23. 12. 1896}|(be} \toendnotes[C]{\smallbreak\pagebreak[2]} \Standort{CUL, Schnitzler, B 5b.}
\physDesc{Brief, 1 Blatt, 2 Seiten
\newline{}Handschrift: schwarze Tinte, deutsche Kurrent
\newline{}Schnitzler: mit Bleistift die Jahreszahl »6« ergänzt \newline{}Ordnung: mit Bleistift von unbekannter Hand nummeriert: »50« }\buchAbdrucke{\weitereDrucke{Hermann Bahr, Arthur Schnitzler: \emph{Briefwechsel, Aufzeichnungen, Dokumente (1891–1931)}. Hg. Kurt Ifkovits und Martin Anton Müller. Göttingen: \emph{Wallstein} 2018, S. 134.} }\toendnotes[C]{\smallbreak}\pstart
           \noindent{}{\pb}\textcolor{gray}{\textbf{»\textcolor{brown}{Die
                        Zeit}{}\ledrightnote{\textcolor{brown}{Die Zeit. Wiener Wochenschrift}}«}}\hfill \textcolor{gray}{\textbf{\textbf{\textcolor{pink}{Wien}{}\ledrightnote{\textcolor{pink}{Wien}}}, den }}23. December{ }\textcolor{gray}{\textbf{189{\dotstwo}}}\pend
           \pstart
           \textcolor{gray}{\textbf{Wiener Wochenſchrift}}\hfill \textcolor{gray}{\textbf{\textcolor{pink}{IX/3, Günthergaſſe 1}{}\ledrightnote{\textcolor{pink}{Günthergasse}}.}}\pend
           \pstart
           \textcolor{gray}{\textbf{\textbf{Herausgeber}:}}{\\}\textcolor{gray}{\textbf{Profeſſor Dr. \textcolor{blue}{I. Singer}{}\ledrightnote{\textcolor{blue}{Isidor Singer}}, \textcolor{blue}{Hermann Bahr}{}\ledrightnote{\textcolor{blue}{Hermann Bahr}},
                        Dr. \textcolor{blue}{Heinrich Kanner}{}\ledrightnote{\textcolor{blue}{Heinrich Kanner}}.}}\pend
           \pstart
           \textcolor{gray}{\textbf{Telephon Nr. 6415.}}\pend
           \pstart\center{}Lieber Arthur!\pend\pstart
           In aller Eile, mitten zwiſchen Correcturen und Schreibereien, möchte ich Dir nur
               geſchwind ſagen, daß ich die »\textcolor{green}{Frau des Weiſen}{}\ledrightnote{\textcolor{green}{Die Frau des Weisen. Erzählung}}«
               geſtern abends ſofort geleſen und von ihr eine wirklich ſchöne und reine Wirkung
               gehabt habe. Nun kann ich Dir erſt recht danken, daß Du mir ein so theueres {\pb}Geſchenk für die »\textcolor{brown}{Zeit}{}\ledrightnote{\textcolor{brown}{Die Zeit. Wiener Wochenschrift}}« gegeben haſt.\pend
           \pstart
           Sie iſt ſchon in der Druckerei, Montag haſt Du die Correctur, an ihr kannſt Du noch
               ganz nach Laune ändern.\pend
           \pstart
           Herzlichſt{\\[\baselineskip]}Dein{\\[\baselineskip]}\spacefill\mbox{hr}\pend
           \leftskip=0em{}\pstart
           \textcolor{gray}{\textbf{\label{T_L00633_1v}\edtext{Alle für »\textcolor{brown}{Die Zeit}{}\ledrightnote{\textcolor{brown}{Die Zeit. Wiener Wochenschrift}}« beſtimmten Zuſchriften und Sendungen ſind an
                  die Redaction der »\textcolor{brown}{Zeit}{}\ledrightnote{\textcolor{brown}{Die Zeit. Wiener Wochenschrift}}« und \textbf{nicht} an die Perſon eines der Herausgeber zu richten.}{\lemma{\textnormal{\emph{Alle … richten.}}}\Cendnote{\textnormal{am unteren Rand der ersten Seite}}}\label{T_L00633_1h}}}\pend
           \endnumbering\briefempfaengerindex{Schnitzler, Arthur@\textsc{Schnitzler, Arthur}!zzzBahr, Hermann@\emph{von Hermann Bahr}!1896-12-231@{23. 12. 1896}|)be}\mylabel{h}  \normalsize

\doendnotes{C}
\bigskip
\vfill

\clearpage

\footnotesize

\lohead{\textsc{register}}

% Definiere theindex-Environment komplett neu ohne reledmac
\makeatletter
\renewenvironment{theindex}{%
  \section*{\indexname}%
  \setlength{\parindent}{0pt}%
  \setlength{\parskip}{0pt plus 0.3pt}%
  \let\item\@idxitem
}{%
  \clearpage
}
\makeatother

\IfFileExists{\jobname-pw.ind}{\input{\jobname-pw.ind}}{}

\end{document}

      