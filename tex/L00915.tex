%% latex-korrekturansicht-vorspann.tex
%% Vorspann für die Korrekturansicht.
%% Lädt die gemeinsame Datei latex-vorspann.tex mit gesetztem Schalter.

\newif\ifkorrekturansicht
\korrekturansichttrue

\input{../tex-inputs/latex-vorspann}


               \section[Arthur Schnitzler an Georg Brandes, 8. 5. 1899]{ Arthur Schnitzler an Georg Brandes, 8. 5. 1899}\nopagebreak\mylabel{v}\rehead{ }\normalsize\beginnumbering\briefempfaengerindex{Brandes, Georg@\textsc{Brandes, Georg}!zzzSchnitzler, Arthur@\emph{von Arthur Schnitzler}!1899-05-081@{8. 5. 1899}|(be} \toendnotes[C]{\smallbreak\pagebreak[2]} \Standort{Kopenhagen, Det Kongelige Bibliotek, Georg Brandes Arkiv, box 125.}
\physDesc{Brief, 1 Blatt, 4 Seiten
\newline{}Handschrift: schwarze Tinte, deutsche Kurrent\newline{}Ordnung: mit Bleistift von unbekannter Hand nummeriert: »15« und datiert: »8/5 99« und nummeriert: »15.« }\buchAbdrucke{\weitereDrucke{1) Georg Brandes, Arthur Schnitzler: \emph{Ein Briefwechsel}. Hg. Kurt Bergel. Bern: \emph{Francke} 1956, S. 75.} \weitereDrucke{2) Arthur Schnitzler: \emph{Briefe 1875–1912}. Hg. Therese Nickl und Heinrich Schnitzler. Frankfurt am Main: \emph{S. Fischer} 1981, S. 370–371.} }\toendnotes[C]{\smallbreak}\pstart{}{\pb}Lieber und verehrter Herr
                        Brandes,\pend\pstart
           zugleich mit dieſem Brief geht ein neues \textcolor{green}{Buch}{}\ledrightnote{→\textcolor{green}{Der grüne Kakadu – Paracelsus – Die Gefährtin. Drei Einakter}} an Sie ab, das 3 Einakter von mir enthält. Sie
                    werden ſchon ziemlich viel gegeben und insbeſondere der »\textcolor{green}{Kakadu}{}\ledrightnote{\textcolor{green}{Der grüne Kakadu. Groteske in einem Akt}}« amüſirt die Leute ſehr. –\pend
           \pstart
           – Weiter ka{\geminationn} ich Ihnen heute kaum was ſagen. Vor
                    ſieben Wochen iſt das \textcolor{blue}{Geſchöpf}{}\ledrightnote{→\textcolor{blue}{Marie Reinhard}} begraben worden, das ich von allen {\pb}Menſchen der Erde am liebſten gehabt
                    habe, meine Geliebte, Freundin und Braut – die durch mehr als vier Jahre meinem
                    Leben ſeinen ganzen Sinn und ſeine ganze Freude gegeben hat, – und ſeither
                    dämmere ich hin, aber exiſtire kaum mehr. Aus der Fülle der Geſundheit und
                    Jugend hat ſie eine blödſinnige und tückiſche Krankheit innerhalb zweier Tage
                    ins Grab geriſſen, und ich habe ſie ſterben geſehen, bei vollem Bewußt{\pb}ſein ſterben geſehn. Bitte ſagen Sie mir
                    kein Wort darüber. Ich mußte es Ihnen aber ſagen. –\pend
           \pstart
           Jener \textcolor{pink}{däniſche}{}\ledrightnote{\textcolor{pink}{Dänemark}}{ }\textcolor{blue}{Schriftſteller}{}\ledrightnote{→\textcolor{blue}{Karl Larsen}} hat ſich bei
                    mir nicht blicken laſſen. Allerdings war ich einige Male von \textcolor{pink}{Wien}{}\ledrightnote{\textcolor{pink}{Wien}} abweſend. Laſſen Sie mich recht bald hören wie es
                    Ihnen geht, ob Sie endgiltig geſund ſind und wie Sie mit Ihren Plänen für den
                    Sommer ſtehn. –\pend
           \pstart
           \textcolor{blue}{Paul Goldmann}{}\ledrightnote{\textcolor{blue}{Paul Goldmann}} iſt wieder in \textcolor{pink}{Frankfurt}{}\ledrightnote{\textcolor{pink}{Frankfurt am Main}} und reiſt viel für ſein \textcolor{brown}{Blatt}{}\ledrightnote{→\textcolor{brown}{Frankfurter Zeitung}}.\pend
           \pstart
           {\pb}\textcolor{blue}{Richard Beer Hofmann}{}\ledrightnote{\textcolor{blue}{Richard Beer-Hofmann}} hat zwei Kinder, \textcolor{blue}{Mirjam}{}\ledrightnote{\textcolor{blue}{Mirjam Beer-Hofmann}} und \textcolor{blue}{Naemi\strikeout{e}}{}\ledrightnote{\textcolor{blue}{Naëmah Beer-Hofmann}}, und befindet ſich
                    wohl.\pend
           \pstart
           Ich grüße Sie von Herzen als Ihr{\\[\baselineskip]}treuergebener
                        \spacefill\mbox{ArthSchnitzler}\pend
           \leftskip=0em{}\pstart
           \textcolor{pink}{Wien}{}\ledrightnote{\textcolor{pink}{Wien}}{ }8. 5. 99.\pend
           \endnumbering\briefempfaengerindex{Brandes, Georg@\textsc{Brandes, Georg}!zzzSchnitzler, Arthur@\emph{von Arthur Schnitzler}!1899-05-081@{8. 5. 1899}|)be}\mylabel{h}  \normalsize

\doendnotes{C}
\bigskip
\vfill

\clearpage

\footnotesize

\lohead{\textsc{register}}

% Definiere theindex-Environment komplett neu ohne reledmac
\makeatletter
\renewenvironment{theindex}{%
  \section*{\indexname}%
  \setlength{\parindent}{0pt}%
  \setlength{\parskip}{0pt plus 0.3pt}%
  \let\item\@idxitem
}{%
  \clearpage
}
\makeatother

\IfFileExists{\jobname-pw.ind}{\input{\jobname-pw.ind}}{}

\end{document}

      