%% latex-korrekturansicht-vorspann.tex
%% Vorspann für die Korrekturansicht.
%% Lädt die gemeinsame Datei latex-vorspann.tex mit gesetztem Schalter.

\newif\ifkorrekturansicht
\korrekturansichttrue

\input{../tex-inputs/latex-vorspann}


               \section[Hugo von Hofmannsthal an Richard Beer-Hofmann und Arthur Schnitzler, 8. 7. 1893]{ Hugo von Hofmannsthal an Richard Beer-Hofmann und Arthur Schnitzler,
               8. 7. 1893}\nopagebreak\mylabel{v}\rehead{ }\normalsize\beginnumbering\briefempfaengerindex{Beer-Hofmann, Richard@\textsc{Beer-Hofmann, Richard}!zzzHofmannsthal, Hugo von@\emph{von Hugo von Hofmannsthal}!1893-07-082@{8. 7. 1893}|(be}\briefempfaengerindex{Schnitzler, Arthur@\textsc{Schnitzler, Arthur}!zzzHofmannsthal, Hugo von@\emph{von Hugo von Hofmannsthal}!1893-07-082@{8. 7. 1893}|(be} \toendnotes[C]{\smallbreak\pagebreak[2]} \Standort{YCGL, MSS 32.}
\physDesc{Brief, 1 Blatt, 3 Seiten
\newline{}Handschrift: Bleistift, deutsche Kurrent\newline{}Ordnung: mit rotem Buntstift von unbekannter Hand datiert: »8. VII. 1893–13« }\buchAbdrucke{\weitereDrucke{Hugo von Hofmannsthal, Richard Beer-Hofmann: \emph{Briefwechsel}. Hg. Eugene Weber. Frankfurt am Main: \emph{S. Fischer} 1972, S. 23.} }\pstart
           \raggedleft{}{\pb}\textcolor{pink}{Fuſch}{}\ledrightnote{\textcolor{pink}{Bad Fusch}}, 8 Juli 93.\pend
           \pstart{}lieber Richard und Arthur!\pend\pstart
           Ich brauch Euch wohl nicht zu ſagen, wie ich mich freue, daſs endlich einmal ein paar
               von den graciöſen Schatten aus dem \textcolor{green}{Anatolbuch}{}\ledrightnote{\textcolor{green}{Anatol}} bei
               Sommerſonne und Lampenlicht lebendig werden ſollen. Ich käme hin, wäre ich nicht
               gerade beim zaghaften Anfang einer Erholung meines etwas in Unordnung gerathenen ſog.
               Nervenſyſtems.\pend
           \pstart
           Es thut mir merkwürdig wohl, ohne Kaffeehaus, ohne Geſelligkeit, ohne etwas das
               treibt oder bindet, ſo vor mich hin zu dämmern, {\pb}in
               lauen Bädern beinahe einzuſchlafen und \textsc{\textcolor{blue}{Shakespeare}{}\ledrightnote{\textcolor{blue}{William Shakespeare}}’sche Comödien} zu leſen,
               während kleine Katzen in der Sonne mit einem Knäuel Wolle ſpielen. Am liebſten war
               mir, Ihr möchtet am \substVorne{}\textsuperscript{m}\substDazwischen{}M\substHinten{}orgen drauf telegrafieren; jedenfalls ſchickt mir, was Ihr an \strikeout{ſonſti} localen und ſonſtigen Recenſionen bekommt,
               wenigſtens zum Anſehen hierher; ich ſchicke Euch doch auch immer alles von mir.\pend
           \pstart
           »\textcolor{green}{Geſtern}{}\ledrightnote{\textcolor{green}{Gestern. Dramatische Studie in einem Akt in Versen}}« hab ich nicht mit; wenn Richard es
               braucht, soll er an \textcolor{brown}{Manz}{}\ledrightnote{\textcolor{brown}{Manz’sche Verlags- und Universitätsbuchhandlung}} (\textcolor{pink}{\textsc{Kohlmarkt}}{}\ledrightnote{\textcolor{pink}{Kohlmarkt}}) {\pb}telegrafieren.\pend
           \pstart
           Ich tröſte mich am \textcolor{blue}{Goethe}{}\ledrightnote{\textcolor{blue}{Johann Wolfgang von Goethe}}–\textcolor{blue}{Schiller}{}\ledrightnote{\textcolor{blue}{Friedrich von Schiller}}'ſchen \textcolor{green}{Briefwechſel}{}\ledrightnote{\textcolor{green}{Briefwechsel zwischen Schiller und Goethe}}
               über unſere \strikeout{mannigfache} mangelhafte Berühmtheit (\textcolor{blue}{Goethe}{}\ledrightnote{\textcolor{blue}{Johann Wolfgang von Goethe}} mit \uline{46} Jahren in \textcolor{pink}{Karlsbad}{}\ledrightnote{\textcolor{pink}{Karlsbad}} wird mit \textcolor{blue}{\uline{\textsc{Klinger}}}{}\ledrightnote{\textcolor{blue}{Friedrich Maximilian von Klinger}} verwechſelt) und habe Euch ſehr gern.\pend
           \pstart \spacefill\mbox{Hugo.}\pend{}\endnumbering\briefempfaengerindex{Beer-Hofmann, Richard@\textsc{Beer-Hofmann, Richard}!zzzHofmannsthal, Hugo von@\emph{von Hugo von Hofmannsthal}!1893-07-082@{8. 7. 1893}|)be}\briefempfaengerindex{Schnitzler, Arthur@\textsc{Schnitzler, Arthur}!zzzHofmannsthal, Hugo von@\emph{von Hugo von Hofmannsthal}!1893-07-082@{8. 7. 1893}|)be}\mylabel{h}  \normalsize

\doendnotes{C}
\bigskip
\vfill

\clearpage

\footnotesize

\lohead{\textsc{register}}

% Definiere theindex-Environment komplett neu ohne reledmac
\makeatletter
\renewenvironment{theindex}{%
  \section*{\indexname}%
  \setlength{\parindent}{0pt}%
  \setlength{\parskip}{0pt plus 0.3pt}%
  \let\item\@idxitem
}{%
  \clearpage
}
\makeatother

\IfFileExists{\jobname-pw.ind}{\input{\jobname-pw.ind}}{}

\end{document}

      