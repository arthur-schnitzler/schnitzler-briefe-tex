%% latex-korrekturansicht-vorspann.tex
%% Vorspann für die Korrekturansicht.
%% Lädt die gemeinsame Datei latex-vorspann.tex mit gesetztem Schalter.

\newif\ifkorrekturansicht
\korrekturansichttrue

\input{../tex-inputs/latex-vorspann}


               \section[Gerty von Hofmannsthal an Arthur Schnitzler, {[}29. 5. 1907{]}]{ Gerty von Hofmannsthal an Arthur Schnitzler, {[}29. 5. 1907{]}}\nopagebreak\mylabel{v}\rehead{ }\normalsize\beginnumbering\briefempfaengerindex{Schnitzler, Arthur@\textsc{Schnitzler, Arthur}!zzzHofmannsthal, Gertrude von@\emph{von Gertrude von Hofmannsthal}!1907-05-291@{{[}29. 5. 1907{]}}|(be} \toendnotes[C]{\smallbreak\pagebreak[2]} \Standort{CUL, Schnitzler, B 43.}
\physDesc{Brief, 1 Blatt, 2 Seiten
\newline{}Handschrift: schwarze Tinte, lateinische Kurrent
\newline{}Schnitzler: mit Bleistift datiert: »29/5 907« \newline{}Ordnung: 1) mit Bleistift von unbekannter Hand nummeriert: »\strikeout{276}« 2) mit Bleistift von unbekannter Hand nummeriert: »278«}\buchAbdrucke{\weitereDrucke{Hugo von Hofmannsthal, Arthur Schnitzler: \emph{Briefwechsel}. Hg. Therese Nickl und Heinrich Schnitzler. Frankfurt am Main: \emph{S. Fischer} 1964, S. 375–376.} }\pstart
           \noindent{}{\pb}Lieber Arthur, \textcolor{blue}{Hugo}{}\ledrightnote{\textcolor{blue}{Hugo von Hofmannsthal}} schreibt mir eben, dass er bis 3ten
                  Juni in \textcolor{pink}{Perugia, Hotel Brufani}{}\ledrightnote{\textcolor{pink}{Hotel Brufani}} ist.
               Gestern war er in \textcolor{pink}{Ravenna}{}\ledrightnote{\textcolor{pink}{Ravenna}} und ist von dort mit der
               Eisenbahn die Küste entlang bis \textcolor{pink}{Rimini}{}\ledrightnote{\textcolor{pink}{Rimini}} gefahren,
               dann nach \textcolor{pink}{Ancona}{}\ledrightnote{\textcolor{pink}{Ancona}}. Heute sind sie nach \textcolor{pink}{Gubbio}{}\ledrightnote{\textcolor{pink}{Gubbio}} und von dort fahren \textcolor{blue}{sie}{}\ledrightnote{\textcolor{blue}{Hugo August von Hofmannsthal}}{ }\strikeout{wieder} nach \textcolor{pink}{Perugia}{}\ledrightnote{\textcolor{pink}{Perugia}}. Ich höre, dass es
                  {\pb}der Gräfin \textcolor{blue}{Thun}{}\ledrightnote{\textcolor{blue}{Christiane von Thun-Hohenstein-Salm-Reifferscheidt}} weiter gut geht, und ich hoffe, dass jetzt die grosse
               Gefahr schon vorüber ist glauben Sie nicht?\pend
           \pstart
           Ich komme natürlich furchtbar gern hinüber, nehme auch auf jeden Fall meine
               Tennissachen mit. Welche Stunden sind Ihnen am liebsten?\pend
           \pstart
           Auf jeden Fall frage ich mich teleph. an.\pend
           \pstart Herzliche Grüsse Ihnen und \textcolor{blue}{Olga}{}\ledrightnote{\textcolor{blue}{Olga Schnitzler}}\hspace*{1.5em}Ihre \spacefill\mbox{Gerty}\pend{}\endnumbering\briefempfaengerindex{Schnitzler, Arthur@\textsc{Schnitzler, Arthur}!zzzHofmannsthal, Gertrude von@\emph{von Gertrude von Hofmannsthal}!1907-05-291@{{[}29. 5. 1907{]}}|)be}\mylabel{h}  \normalsize

\doendnotes{C}
\bigskip
\vfill

\clearpage

\footnotesize

\lohead{\textsc{register}}

% Definiere theindex-Environment komplett neu ohne reledmac
\makeatletter
\renewenvironment{theindex}{%
  \section*{\indexname}%
  \setlength{\parindent}{0pt}%
  \setlength{\parskip}{0pt plus 0.3pt}%
  \let\item\@idxitem
}{%
  \clearpage
}
\makeatother

\IfFileExists{\jobname-pw.ind}{\input{\jobname-pw.ind}}{}

\end{document}

      