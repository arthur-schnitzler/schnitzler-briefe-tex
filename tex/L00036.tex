%% latex-korrekturansicht-vorspann.tex
%% Vorspann für die Korrekturansicht.
%% Lädt die gemeinsame Datei latex-vorspann.tex mit gesetztem Schalter.

\newif\ifkorrekturansicht
\korrekturansichttrue

\input{../tex-inputs/latex-vorspann}


               \section[Arthur Schnitzler an Richard Beer-Hofmann, {[}22.? 8. 1891{]}]{ Arthur Schnitzler an Richard Beer-Hofmann, {[}22.? 8. 1891{]}}\nopagebreak\mylabel{v}\rehead{ }\normalsize\beginnumbering\briefempfaengerindex{Beer-Hofmann, Richard@\textsc{Beer-Hofmann, Richard}!zzzSchnitzler, Arthur@\emph{von Arthur Schnitzler}!1891-08-222@{{[}22.? 8. 1891{]}}|(be} \toendnotes[C]{\smallbreak\pagebreak[2]} \Standort{YCGL, MSS 31.}
\physDesc{Briefkarte
\newline{}Handschrift: schwarze Tinte, deutsche Kurrent}\toendnotes[C]{\smallbreak}\pstart
           \noindent{}{\pb}Lieber Richard! Prof. \textcolor{blue}{\textsc{Kraft-Ebing}}{}\ledrightnote{\textcolor{blue}{Richard von Krafft-Ebing}} iſt noch nicht in \textcolor{pink}{Wien}{}\ledrightnote{\textcolor{pink}{Wien}}; er ſoll etwa am
                  20. September wieder eintreffen\pend
           \pstart
           – \textcolor{blue}{Loris}{}\ledrightnote{\textcolor{blue}{Hugo von Hofmannsthal}} hat mir noch nach \textcolor{pink}{Iſchl}{}\ledrightnote{\textcolor{pink}{Bad Ischl}} geſchrieben, die \label{K_L00036-1v}\edtext{Karte}{\lemma{\textnormal{\emph{Karte}}}\Cendnote{\textnormal{vgl. Hugo von Hofmannsthal an Arthur Schnitzler,
                    [16. 8. 1891]}}}\label{K_L00036-1h} wurde mir nachgeſchickt, er war an
               jenen Tagen nicht wohl.\pend
           \pstart
           – Ueber meine Pläne weiß ich ſelber {\pb}noch nichts,
               halten Sie mich jedenfalls am Laufenden, was Sie zu thun gedenken. – Laune
               miſerabel.\pend
           \pstart
           Herzlichen Gruſs{\\[\baselineskip]}Ihr{\\[\baselineskip]}\spacefill\mbox{Arthur.}\pend
           \leftskip=0em{}\pstart
           \noindent{}\textsc{\textcolor{blue}{Salten}{}\ledrightnote{\textcolor{blue}{Felix Salten}}} grüßt Sie herzlich. –\pend
           \endnumbering\briefempfaengerindex{Beer-Hofmann, Richard@\textsc{Beer-Hofmann, Richard}!zzzSchnitzler, Arthur@\emph{von Arthur Schnitzler}!1891-08-222@{{[}22.? 8. 1891{]}}|)be}\mylabel{h}  \normalsize

\doendnotes{C}
\bigskip
\vfill

\clearpage

\footnotesize

\lohead{\textsc{register}}

% Definiere theindex-Environment komplett neu ohne reledmac
\makeatletter
\renewenvironment{theindex}{%
  \section*{\indexname}%
  \setlength{\parindent}{0pt}%
  \setlength{\parskip}{0pt plus 0.3pt}%
  \let\item\@idxitem
}{%
  \clearpage
}
\makeatother

\IfFileExists{\jobname-pw.ind}{\input{\jobname-pw.ind}}{}

\end{document}

      