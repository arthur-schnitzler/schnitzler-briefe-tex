%% latex-korrekturansicht-vorspann.tex
%% Vorspann für die Korrekturansicht.
%% Lädt die gemeinsame Datei latex-vorspann.tex mit gesetztem Schalter.

\newif\ifkorrekturansicht
\korrekturansichttrue

\input{../tex-inputs/latex-vorspann}


               \section[Arthur Schnitzler an Hermann Bahr, 11. 11. 1903]{ Arthur Schnitzler an Hermann Bahr, 11. 11. 1903}\nopagebreak\mylabel{v}\rehead{ }\normalsize\beginnumbering\briefempfaengerindex{Bahr, Hermann@\textsc{Bahr, Hermann}!zzzSchnitzler, Arthur@\emph{von Arthur Schnitzler}!1903-11-111@{11. 11. 1903}|(be} \toendnotes[C]{\smallbreak\pagebreak[2]} \Standort{TMW, HS AM 23360 Ba.}
\physDesc{Kartenbrief
\newline{}Handschrift: schwarze Tinte, deutsche Kurrent\newline{}Versand: 1) Stempel: »\nobreak{}\oindex{XVIII., Waehring@\textbf{XVIII., Währing}, \emph{Bezirk (A.BZK)}|pwk}18/1 Wien, 11. 11. 03, 11–12 V\nobreak{}«.  2) Stempel: »\nobreak{}\oindex{XIII., Hietzing@\textbf{XIII., Hietzing}, \emph{Bezirk (A.BZK)}|pwk}Wien 13/7, 11. 11. 03\nobreak{}«. }\buchAbdrucke{\weitereDrucke{1) \emph{11. 11. 1903.} In: Arthur Schnitzler: \emph{The Letters of Arthur Schnitzler to Hermann Bahr}. Edited, annotated, and with an introduction, by Donald G.
                        Daviau. Chapel Hill: \emph{The University of North Carolina Press} 1978, S. 82 (University of North Carolina studies in the Germanic languages
                        and literatures, 89).} \weitereDrucke{2) Hermann Bahr, Arthur Schnitzler: \emph{Briefwechsel, Aufzeichnungen, Dokumente (1891–1931)}. Hg. Kurt Ifkovits und Martin Anton Müller. Göttingen: \emph{Wallstein} 2018, S. 280.} }\toendnotes[C]{\smallbreak}\pstart{}{\pb}Herrn Hermann Bahr\pend{}\pstart{}\textcolor{pink}{Wien Ob St Veit}{}\ledrightnote{\textcolor{pink}{Ober Sankt Veit}}\pend{}\pstart{}\textcolor{pink}{Veitliſſngaſſe.}{}\ledrightnote{\textcolor{pink}{Veitlissengasse}}\pend{}{\bigskip}\pstart
           {\pb}11. 11. 903.\pend
           \pstart
           lieber Hermann, ich habe mich gleich an \textcolor{blue}{Julius}{}\ledrightnote{\textcolor{blue}{Julius Schnitzler}} gewandt, da mir diese Titelſache ſelbſt nicht
               erinnerlich iſt; er wird dir wohl gleich direct antworten.\pend
           \pstart
           In einem \label{K_L01341_1v}\edtext{Brief von \textcolor{blue}{Brahm}{}\ledrightnote{\textcolor{blue}{Otto Brahm}}}{\lemma{\textnormal{\emph{Brief von Brahm}}}\Cendnote{\textnormal{Brief vom 8. 11. 1903
                  (\emph{Briefwechsel}
                     Schnitzler/Brahm
                     152–153).}}}\label{K_L01341_1h}, der vorgeſtern anlangte, ist von einem \uline{Termin} meines \textcolor{green}{Stückes}{}\ledrightnote{→\textcolor{green}{Der einsame Weg. Schauspiel in fünf Akten}} noch nicht die Rede; er ſchreibt mir nur die
               Beſetzung und will alles nähere nächſte Woche \uline{mündlich} mit mir beſprechen\introOben{}·\introOben{}) Er kommt, (was
               vielleicht noch niemand wiſſen ſoll?) zum \label{K_L01341_2v}\edtext{\textcolor{green}{Fulda}{}\ledrightnote{→\textcolor{green}{Novella d’Andrea}}}{\lemma{\textnormal{\emph{Fulda}}}\Cendnote{\textnormal{Uraufführung von \emph{\textcolor{green}{Novella d’Andrea}} am 21. 11. 1903 im \textcolor{pink}{Burgtheater}}}}\label{K_L01341_2h} her. Nach dem Telegr\damage{a{\geminationm}} an dich zu ſchließen, dürfteſt du wohl \uline{vor
                  mir}, etwa Anfang Dezember, \textcolor{green}{dranko{\geminationm}en}{}\ledrightnote{→\textcolor{green}{Der Meister}}?\pend
           \pstart Herzlichen Gruſs. Dein \spacefill\mbox{A.}\pend{}\pstart
           \noindent{}·) Auch einige (nicht beträchtliche) Aenderungen ſchlägt er vor.\pend
           \endnumbering\briefempfaengerindex{Bahr, Hermann@\textsc{Bahr, Hermann}!zzzSchnitzler, Arthur@\emph{von Arthur Schnitzler}!1903-11-111@{11. 11. 1903}|)be}\mylabel{h}  \normalsize

\doendnotes{C}
\bigskip
\vfill

\clearpage

\footnotesize

\lohead{\textsc{register}}

% Definiere theindex-Environment komplett neu ohne reledmac
\makeatletter
\renewenvironment{theindex}{%
  \section*{\indexname}%
  \setlength{\parindent}{0pt}%
  \setlength{\parskip}{0pt plus 0.3pt}%
  \let\item\@idxitem
}{%
  \clearpage
}
\makeatother

\IfFileExists{\jobname-pw.ind}{\input{\jobname-pw.ind}}{}

\end{document}

      