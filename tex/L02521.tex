%% latex-korrekturansicht-vorspann.tex
%% Vorspann für die Korrekturansicht.
%% Lädt die gemeinsame Datei latex-vorspann.tex mit gesetztem Schalter.

\newif\ifkorrekturansicht
\korrekturansichttrue

\input{../tex-inputs/latex-vorspann}


               \section[Richard Beer-Hofmann an Arthur Schnitzler, 28. 8. 1929]{ Richard Beer-Hofmann an Arthur Schnitzler, 28. 8. 1929}\nopagebreak\mylabel{v}\rehead{ }\normalsize\beginnumbering\briefempfaengerindex{Schnitzler, Arthur@\textsc{Schnitzler, Arthur}!zzzBeer-Hofmann, Richard@\emph{von Richard Beer-Hofmann}!1929-08-281@{28. 8. 1929}|(be} \toendnotes[C]{\smallbreak\pagebreak[2]} \Standort{CUL, Schnitzler, B 8.}
\physDesc{Brief, 3 Blätter, 6 Seiten (paginiert)
\newline{}Handschrift: blauer Buntstift, lateinische Kurrent\newline{}Ordnung: mit Bleistift von unbekannter Hand nummeriert:
                                    »275« }\buchAbdrucke{\weitereDrucke{Arthur Schnitzler, Richard Beer-Hofmann: \emph{Briefwechsel 1891–1931}. Hg. Konstanze Fliedl. Wien, Zürich: \emph{Europaverlag} 1992, S. 231–232.} }\toendnotes[C]{\smallbreak}\pstart
           \raggedleft{}{\pb}\textcolor{pink}{Wien}{}\ledrightnote{\textcolor{pink}{Wien}}{ }28. VIII. 29.\pend
           \pstart
           Lieber Arthur! Ich \uuline{hoffe} am \label{K_L02521_1v}\edtext{6. VIII.}{\lemma{\textnormal{\emph{6. VIII.}}}\Cendnote{\textnormal{\textcolor{blue}{Salten} hatte am 6. 9. 1929 seinen
                  60. Geburtstag.}}}\label{K_L02521_1h} schon in \textcolor{pink}{Marienbad}{}\ledrightnote{\textcolor{pink}{Marienbad}} zu
               sein. Jedenfalls werde ich \textcolor{blue}{F. S.}{}\ledrightnote{\textcolor{blue}{Felix Salten}} telegraphieren –
                  \textcolor{green}{geschrieben}{}\ledrightnote{→\textcolor{green}{[Lieber Felix Salten]}} habe ich ja für
                  \textcolor{brown}{Zsolnays}{}\ledrightnote{\textcolor{brown}{Paul Zsolnay Verlag}}{ }\textcolor{green}{Almanach}{}\ledrightnote{\textcolor{green}{Jahrbuch Paul Zsolnay Verlag}}. Blumen? – Nein! Irgend eine kleine Gabe?
               – Ich will mich nach Ihnen richten. Eigentlich: Bei einem Andern wäre all das kein
               Problem. Aber {\pb}bei \textcolor{blue}{F. S.}{}\ledrightnote{\textcolor{blue}{Felix Salten}}! Er ist mistrauisch, grundsätzlich leicht verletzt, i{\geminationm}er witternd, man schätze ihn nicht \strikeout{gar} genug, dabei – in seiner Eigenschaft als Kritiker –
               zu leicht der Ansicht zugeneigt, man tue etwas um ihn bei guter Laune zu erhalten –
               sogar \strikeout{gb}\uline{bei}{ }\uline{uns}, glaube ich, vielleicht von Argwohn befallen, und
               sich sagend: {\pb}»Ich habe weder
               Blumen noch sonst was geschickt als B-H. 60. wurde – na – wer weiss, was wäre, wenn
               ich \uline{nicht} Kritiker wäre – –« {\{}aber »beleidigt« wenn man ihm diese
               Argumentation unterschöbe (– schübe? – Gra{\geminationm}atik ist so
                  schwer!).{\}} Schwer mit ihm! Also: Telegra{\geminationm} – keine Blumen – irgendeine Aufmerksamkeit \uline{später}, wenn {\pb}\uline{\label{T_L02521_1v}\edtext{Sie}{\lemma{\textnormal{\emph{Sie}}}\Cendnote{\textnormal{im Original: »sie«}}}\label{T_L02521_1h} der Ansicht sind}.\pend
           \pstart
           \numberlinefalse{}\centering{}–\numberlinetrue{}\pend
           \pstart
           \noindent{}Was das Hôtel unter Ihrem Fenster anlangt – vor 31 Jahren \introOben{}waren
                  Sie\introOben{} mit \textcolor{blue}{Hugo}{}\ledrightnote{\textcolor{blue}{Hugo von Hofmannsthal}} dort – »in den nächsten
               31 Jahren \introOben{}wird es\introOben{} wol auch noch unter diesem Fenster \introOben{}sein\introOben{}« – Wäre ich der Hôtelbesitzer würde ich auf diese – Ihre
               – Äusserung hin, \uline{hoch} versichern. Bei Schnitzler
               pflegen solche Hôtels daraufhin {\pb}höhnisch abzubrennen. – \uline{Ich} bin in den Wehen des
                  \textcolor{green}{IV}{}\ledrightnote{→\textcolor{green}{Der junge David. Sieben Bilder}} – dh. jetzt \textcolor{green}{IV}{}\ledrightnote{→\textcolor{green}{Der junge David. Sieben Bilder}} + \textcolor{green}{V. Bild}{}\ledrightnote{→\textcolor{green}{Der junge David. Sieben Bilder}}es – ich wittere, dass \strikeout{sich} aus geheimnisvollen rythmischen Gründen die VII. Bilder auf V. \strikeout{zur} sich zurückbilden werden!\pend
           \pstart
           {\pb}Gutes Wetter! Gute Laune – soviel
               ein besserer Mensch – ohne sich etwas zu vergeben – aufbringen kann, und alles Liebe
               von \textcolor{blue}{Paula}{}\ledrightnote{\textcolor{blue}{Paula Beer-Hofmann}} und mir! Ihr\pend
           \pstart \spacefill\mbox{Richard}\pend{}\pstart
           Grüsse, und gute Wünsche für Frau \textcolor{blue}{P.}{}\ledrightnote{\textcolor{blue}{Clara Katharina Pollaczek}}\pend
           \pstart
           \label{T_L02521_2v}\edtext{Format dieses Zettels}{\lemma{\textnormal{\emph{Format dieses Zettels}}}\Cendnote{\textnormal{umlaufend zuerst quer am linken Rand, dann
                  unterhalb des Textes, dann quer am linken Rand}}}\label{T_L02521_2h} nicht Geiz – sondern weil
                  \label{K_L02521_2v}\edtext{Ducki}{\lemma{\textnormal{\emph{Ducki}}}\Cendnote{\textnormal{zahme Haustaube}}}\label{K_L02521_2h} den oberen Rand meines letzten
               Brief-Kartels, während ich schrieb – besiegelte.\pend
           \endnumbering\briefempfaengerindex{Schnitzler, Arthur@\textsc{Schnitzler, Arthur}!zzzBeer-Hofmann, Richard@\emph{von Richard Beer-Hofmann}!1929-08-281@{28. 8. 1929}|)be}\mylabel{h}  \normalsize

\doendnotes{C}
\bigskip
\vfill

\clearpage

\footnotesize

\lohead{\textsc{register}}

% Definiere theindex-Environment komplett neu ohne reledmac
\makeatletter
\renewenvironment{theindex}{%
  \section*{\indexname}%
  \setlength{\parindent}{0pt}%
  \setlength{\parskip}{0pt plus 0.3pt}%
  \let\item\@idxitem
}{%
  \clearpage
}
\makeatother

\IfFileExists{\jobname-pw.ind}{\input{\jobname-pw.ind}}{}

\end{document}

      