%% latex-korrekturansicht-vorspann.tex
%% Vorspann für die Korrekturansicht.
%% Lädt die gemeinsame Datei latex-vorspann.tex mit gesetztem Schalter.

\newif\ifkorrekturansicht
\korrekturansichttrue

\input{../tex-inputs/latex-vorspann}


               \section[Hugo von Hofmannsthal an Arthur Schnitzler, 7. 1. {[}1903{]}]{ Hugo von Hofmannsthal an Arthur Schnitzler,
               7. 1. {[}1903{]}}\nopagebreak\mylabel{v}\rehead{ }\normalsize\beginnumbering\briefempfaengerindex{Schnitzler, Arthur@\textsc{Schnitzler, Arthur}!zzzHofmannsthal, Hugo von@\emph{von Hugo von Hofmannsthal}!1903-01-072@{7. 1. {[}1903{]}}|(be} \toendnotes[C]{\smallbreak\pagebreak[2]} \Standort{CUL, Schnitzler, B 43.}
\physDesc{Brief, 1 Blatt, 2 Seiten
\newline{}Handschrift: schwarze Tinte, deutsche Kurrent\newline{}Ordnung: 1) mit Bleistift von unbekannter Hand nummeriert: »\strikeout{210}« 2) mit Bleistift von unbekannter Hand nummeriert: »192«}\buchAbdrucke{\weitereDrucke{Hugo von Hofmannsthal, Arthur Schnitzler: \emph{Briefwechsel}. Hg. Therese Nickl und Heinrich Schnitzler. Frankfurt am Main: \emph{S. Fischer} 1964, S. 166.} }\toendnotes[C]{\smallbreak}\pstart
           \raggedleft{}{\pb}7 I.\pend
           \pstart
           lieber, die geſtrige Sache hat für mich ſehr unter der gedrängten
               Zeit gelitten.\hspace*{2.5em}Hier einige Bitten.\pend
           \pstart
           1.) wie würden Sie mir rathen das Verhältnis zu dem \textcolor{blue}{Otway}{}\ledrightnote{\textcolor{blue}{Thomas Otway}}’sſchen{ }\textcolor{green}{Stück}{}\ledrightnote{→\textcolor{green}{Das gerettete Venedig}} auf dem Zettel zu
               bezeichnen? \pend
           \pstart
           2.) welche Beſetzung der 6 großen Rollen (2 Frauen 2 Männer {\pb}2 Senatoren) ſchlagen Sie
               (mit dem vorhandenen Material) für das \textcolor{brown}{Burgtheater}{}\ledrightnote{\textcolor{brown}{Burgtheater}}
               vor?\pend
           \pstart
           Wir wünſchen Ihnen und \textcolor{blue}{Olga}{}\ledrightnote{\textcolor{blue}{Olga Schnitzler}} viel Freude für den
               kleinen Ausflug.\pend
           \pstart
           Von Herzen{\\[\baselineskip]}\spacefill\mbox{Hugo.}\pend
           \leftskip=0em{}\endnumbering\briefempfaengerindex{Schnitzler, Arthur@\textsc{Schnitzler, Arthur}!zzzHofmannsthal, Hugo von@\emph{von Hugo von Hofmannsthal}!1903-01-072@{7. 1. {[}1903{]}}|)be}\mylabel{h}  \normalsize

\doendnotes{C}
\bigskip
\vfill

\clearpage

\footnotesize

\lohead{\textsc{register}}

% Definiere theindex-Environment komplett neu ohne reledmac
\makeatletter
\renewenvironment{theindex}{%
  \section*{\indexname}%
  \setlength{\parindent}{0pt}%
  \setlength{\parskip}{0pt plus 0.3pt}%
  \let\item\@idxitem
}{%
  \clearpage
}
\makeatother

\IfFileExists{\jobname-pw.ind}{\input{\jobname-pw.ind}}{}

\end{document}

      