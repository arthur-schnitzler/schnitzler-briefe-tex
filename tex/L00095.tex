%% latex-korrekturansicht-vorspann.tex
%% Vorspann für die Korrekturansicht.
%% Lädt die gemeinsame Datei latex-vorspann.tex mit gesetztem Schalter.

\newif\ifkorrekturansicht
\korrekturansichttrue

\input{../tex-inputs/latex-vorspann}


               \section[Arthur Schnitzler an Wilhelm Bölsche, 20. 4. 1892]{ Arthur Schnitzler an Wilhelm Bölsche, 20. 4. 1892}\nopagebreak\mylabel{v}\rehead{ }\normalsize\beginnumbering\briefempfaengerindex{Boelsche, Wilhelm@\textsc{Bölsche, Wilhelm}!zzzSchnitzler, Arthur@\emph{von Arthur Schnitzler}!1892-04-201@{20. 4. 1892}|(be} \toendnotes[C]{\smallbreak\pagebreak[2]} \Standort{Wrocław, Biblioteka Uniwersytecka, Böl.Pis 1764.}
\physDesc{Brief, 1 Blatt, 2 Seiten (Seite 3 quer zur üblichen Schreibrichtung)
\newline{}Handschrift: schwarze Tinte, deutsche Kurrent
\newline{}Bölsche: als »Erl{[}edigt{]}« gezeichnet }\buchAbdrucke{\weitereDrucke{1) Alois Woldan: \emph{Arthur Schnitzler – Briefe an Wilhelm Bölsche.} In: \emph{Germanica Wratislaviensia} (1987) Nr. 77, S. 460.} \weitereDrucke{2) Wilhelm Bölsche: \emph{Briefwechsel. Mit Autoren der Freien Bühne}. Hg. Gerd-Hermann Susen. Berlin: \emph{Weidler} 2010, S. 680 (Werke und Briefe. Wissenschaftliche Ausgabe, Briefe I).} }\toendnotes[C]{\smallbreak}\pstart
           \raggedleft{}{\pb}\textcolor{pink}{Wien}{}\ledrightnote{\textcolor{pink}{Wien}}, 20. April 92\pend
           \pstart{}Verehrteſter Herr,\pend\pstart
           ich ſchicke Ihnen hier die \textcolor{green}{Skizze}{}\ledrightnote{→\textcolor{green}{Das Himmelbett}} mit der beſondern Bitte, mir falls Sie ſie zu veröffentlichen
                    gedenken, gütigſt eine \uuline{Correctur}{ }ſenden laſſen zu wollen; ſie ſoll beſti{\geminationm}t in 24 Stunden erledigt ſein. Sollten Sie das
                        \textcolor{green}{Manuscript}{}\ledrightnote{→\textcolor{green}{Das Himmelbett}}{ }{\pb}nicht brauchen können,
                    was mir aufrichtig leid thäte, ſo haben Sie wohl die Liebenswürdigkeit, es mir
                    recht bald zurückzuſenden.\pend
           \pstart
           Hochachtungsvoll{\\[\baselineskip]}Ihr ſehr ergebner{\\[\baselineskip]}\spacefill\mbox{Dr Arthur Schnitzler}\pend
           \leftskip=0em{}\pstart
           \noindent{}\textcolor{pink}{\textsc{I. Giselastraße 11}.}{}\ledrightnote{\textcolor{pink}{Bösendorferstraße}}\pend
           \pstart
           {\pb}Scheint Ihnen etwa der Titel zu riskant, ſo könnte
                        die \textcolor{green}{Skizze}{}\ledrightnote{→\textcolor{green}{Das Himmelbett}} auch »\textcolor{green}{Verblaßende Farben}{}\ledrightnote{\textcolor{green}{Das Himmelbett}}« genannt werden; lieber
                        iſt mir allerdings der erſte »\textcolor{green}{Das
                            Himmelbett}{}\ledrightnote{\textcolor{green}{Das Himmelbett}}.«\pend
           \pstart
           \raggedleft{}ArthSch\pend
           \endnumbering\briefempfaengerindex{Boelsche, Wilhelm@\textsc{Bölsche, Wilhelm}!zzzSchnitzler, Arthur@\emph{von Arthur Schnitzler}!1892-04-201@{20. 4. 1892}|)be}\mylabel{h}  \normalsize

\doendnotes{C}
\bigskip
\vfill

\clearpage

\footnotesize

\lohead{\textsc{register}}

% Definiere theindex-Environment komplett neu ohne reledmac
\makeatletter
\renewenvironment{theindex}{%
  \section*{\indexname}%
  \setlength{\parindent}{0pt}%
  \setlength{\parskip}{0pt plus 0.3pt}%
  \let\item\@idxitem
}{%
  \clearpage
}
\makeatother

\IfFileExists{\jobname-pw.ind}{\input{\jobname-pw.ind}}{}

\end{document}

      