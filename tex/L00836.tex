%% latex-korrekturansicht-vorspann.tex
%% Vorspann für die Korrekturansicht.
%% Lädt die gemeinsame Datei latex-vorspann.tex mit gesetztem Schalter.

\newif\ifkorrekturansicht
\korrekturansichttrue

\input{../tex-inputs/latex-vorspann}


               \section[Arthur Schnitzler an Richard Beer-Hofmann, 22. 8. 1898]{ Arthur Schnitzler an Richard Beer-Hofmann, 22. 8. 1898}\nopagebreak\mylabel{v}\rehead{ }\normalsize\beginnumbering\briefempfaengerindex{Beer-Hofmann, Richard@\textsc{Beer-Hofmann, Richard}!zzzSchnitzler, Arthur@\emph{von Arthur Schnitzler}!1898-08-221@{22. 8. 1898}|(be} \toendnotes[C]{\smallbreak\pagebreak[2]} \Standort{YCGL, MSS 31.}
\physDesc{Postkarte
\newline{}Handschrift: Bleistift, deutsche Kurrent\newline{}Versand: 1) Stempel: »\nobreak{}\oindex{Luzern@\textbf{Luzern}, \emph{Besiedelter Ort (A.BSO)}|pwk}Luzern, 22 VIII 98, 4\nobreak{}«.  2) Stempel: »\nobreak{}\oindex{Steindorf am Ossiacher See@\textbf{Steindorf am Ossiacher See}, \emph{http://www.geonames.org/ontologyA.ADM3}|pwk}\textcolor{gray}{Steindorf} am Ossiacher See, 24 \textcolor{gray}{8 98}\nobreak{}«. }\pstart{}{\pb}Herrn \textsc{Dr. Richard
                     Beer-Hofmann}\pend{}\pstart{}\textcolor{pink}{\textsc{Steindorf am Ossiacher}ſee}{}\ledrightnote{\textcolor{pink}{Steindorf am Ossiacher See}}.\pend{}\pstart{}\textcolor{pink}{\textsc{Kärnthen}}{}\ledrightnote{\textcolor{pink}{Kärnten}}.\pend{}{\bigskip}\pstart
           \raggedleft{}{\pb}\textcolor{pink}{\textsc{Luzern}}{}\ledrightnote{\textcolor{pink}{Luzern}}, 22. 8. 98\pend
           \pstart
           Nach einer ſehr ſchönen Tour bis \textcolor{pink}{Genf}{}\ledrightnote{\textcolor{pink}{Genf}} hat ſich
                  \textcolor{blue}{Hugo}{}\ledrightnote{\textcolor{blue}{Hugo von Hofmannsthal}} nach \textcolor{pink}{Lugano}{}\ledrightnote{\textcolor{pink}{Lugano}} und ich, in prachtvollen Fahrten durchs \textcolor{pink}{\textsc{Berner} Oberland}{}\ledrightnote{\textcolor{pink}{Berner Oberland}}, hieher
               gewandt, wo ich vielleicht acht Tage bleibe, um da{\geminationn},
               möglicherweiſe streckenweiſe per Rad zurück nach \textcolor{pink}{Wien}{}\ledrightnote{\textcolor{pink}{Wien}} zurück zu reiſen. Es geht mir gut; nach Arbeit\introOben{}en\introOben{}{ }ſehne ich mich
               ein bischen; gerne hätt ich eine Nachricht von Ihnen; hieher \textsc{post
                  rest}. Von Herzen Ihr \spacefill\mbox{A.}\pend
           \endnumbering\briefempfaengerindex{Beer-Hofmann, Richard@\textsc{Beer-Hofmann, Richard}!zzzSchnitzler, Arthur@\emph{von Arthur Schnitzler}!1898-08-221@{22. 8. 1898}|)be}\mylabel{h}  \normalsize

\doendnotes{C}
\bigskip
\vfill

\clearpage

\footnotesize

\lohead{\textsc{register}}

% Definiere theindex-Environment komplett neu ohne reledmac
\makeatletter
\renewenvironment{theindex}{%
  \section*{\indexname}%
  \setlength{\parindent}{0pt}%
  \setlength{\parskip}{0pt plus 0.3pt}%
  \let\item\@idxitem
}{%
  \clearpage
}
\makeatother

\IfFileExists{\jobname-pw.ind}{\input{\jobname-pw.ind}}{}

\end{document}

      