%% latex-korrekturansicht-vorspann.tex
%% Vorspann für die Korrekturansicht.
%% Lädt die gemeinsame Datei latex-vorspann.tex mit gesetztem Schalter.

\newif\ifkorrekturansicht
\korrekturansichttrue

\input{../tex-inputs/latex-vorspann}


               \section[Arthur Schnitzler an Richard Beer-Hofmann, 23. 12. 1898]{ Arthur Schnitzler an Richard Beer-Hofmann, 23. 12. 1898}\nopagebreak\mylabel{v}\rehead{ }\normalsize\beginnumbering\briefempfaengerindex{Beer-Hofmann, Richard@\textsc{Beer-Hofmann, Richard}!zzzSchnitzler, Arthur@\emph{von Arthur Schnitzler}!1898-12-231@{23. 12. 1898}|(be} \toendnotes[C]{\smallbreak\pagebreak[2]} \Standort{CUL, Schnitzler, B 8.1, S. 76.}
\physDesc{maschinelle Abschrift
\newline{}Schreibmaschine\newline{}Ordnung: von unbekannter Hand nummeriert: »132« }\buchAbdrucke{\weitereDrucke{Arthur Schnitzler, Richard Beer-Hofmann: \emph{Briefwechsel 1891–1931}. Hg. Konstanze Fliedl. Wien, Zürich: \emph{Europaverlag} 1992, S. 125–126.} }\toendnotes[C]{\smallbreak}\pstart
           \raggedleft{}{\pb}23. 12. 98. \pend
           \pstart
           Lieber Richard, das können Sie auffassen wie Sie wollen, als
               Weihnachtsgeschenk, als \label{K_L00870_1v}\edtext{Hochzeitsgeschenk}{\lemma{\textnormal{\emph{Hochzeitsgeschenk}}}\Cendnote{\textnormal{Diese hatte am
                     14. 5. 1898 stattgefunden.}}}\label{K_L00870_1h}, als \label{K_L00870_2v}\edtext{Tauf(?)geschenk}{\lemma{\textnormal{\emph{Tauf(?)geschenk}}}\Cendnote{\textnormal{Am
                     20. 12. 1898 kam die Tochter \textcolor{blue}{Naëmah
                     Sofie Agnes} auf die Welt.}}}\label{K_L00870_2h} – oder nur als Geschmacklosigkeit – und
               auf die 2 Sesseln können sich \textcolor{blue}{Mirjam}{}\ledrightnote{\textcolor{blue}{Mirjam Beer-Hofmann}} und \label{T_L00870_1v}\edtext{\textcolor{blue}{Naëmah}{}\ledrightnote{\textcolor{blue}{Naëmah Beer-Hofmann}}}{\lemma{\textnormal{\emph{Naëmah}}}\Cendnote{\textnormal{Die Abschrift hat fälschlich
                     »Noemi«, was eher nicht auf Schnitzler zurückgehen
                  dürfte.}}}\label{T_L00870_1h} setzen und auf das Tischerl gehören Cigaretten oder ein Buch oder
               ein hölzerner Engel; oder Sie können alles zusammen in den Ofen werfen oder ich kann
               es auch umtauschen; jedenfalls leben Sie wohl und seien Sie herzlich gegrüsst wie die
                  \textcolor{blue}{Ihrigen}{}\ledrightnote{→\textcolor{blue}{Mirjam Beer-Hofmann}{\newline}→\textcolor{blue}{Naëmah Beer-Hofmann}{\newline}→\textcolor{blue}{Paula Beer-Hofmann}} alle Ihr
                  \spacefill\mbox{Arthur.}\pend
           \endnumbering\briefempfaengerindex{Beer-Hofmann, Richard@\textsc{Beer-Hofmann, Richard}!zzzSchnitzler, Arthur@\emph{von Arthur Schnitzler}!1898-12-231@{23. 12. 1898}|)be}\mylabel{h}  \normalsize

\doendnotes{C}
\bigskip
\vfill

\clearpage

\footnotesize

\lohead{\textsc{register}}

% Definiere theindex-Environment komplett neu ohne reledmac
\makeatletter
\renewenvironment{theindex}{%
  \section*{\indexname}%
  \setlength{\parindent}{0pt}%
  \setlength{\parskip}{0pt plus 0.3pt}%
  \let\item\@idxitem
}{%
  \clearpage
}
\makeatother

\IfFileExists{\jobname-pw.ind}{\input{\jobname-pw.ind}}{}

\end{document}

      