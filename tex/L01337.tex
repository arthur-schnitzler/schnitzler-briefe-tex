%% latex-korrekturansicht-vorspann.tex
%% Vorspann für die Korrekturansicht.
%% Lädt die gemeinsame Datei latex-vorspann.tex mit gesetztem Schalter.

\newif\ifkorrekturansicht
\korrekturansichttrue

\input{../tex-inputs/latex-vorspann}


               \section[Hermann Bahr an Arthur Schnitzler, 9. 11. 1903]{ Hermann Bahr an Arthur Schnitzler, 9. 11. 1903}\nopagebreak\mylabel{v}\rehead{ }\normalsize\beginnumbering\briefempfaengerindex{Schnitzler, Arthur@\textsc{Schnitzler, Arthur}!zzzBahr, Hermann@\emph{von Hermann Bahr}!1903-11-091@{9. 11. 1903}|(be} \toendnotes[C]{\smallbreak\pagebreak[2]} \Standort{CUL, Schnitzler, B 5b.}
\physDesc{Brief, 1 Blatt, 2 Seiten
\newline{}Handschrift: schwarze Tinte, deutsche Kurrent\newline{}Ordnung: mit Bleistift von unbekannter Hand nummeriert:
                                    »102« }\buchAbdrucke{\weitereDrucke{Hermann Bahr, Arthur Schnitzler: \emph{Briefwechsel, Aufzeichnungen, Dokumente (1891–1931)}. Hg. Kurt Ifkovits und Martin Anton Müller. Göttingen: \emph{Wallstein} 2018, S. 277–278.} }\toendnotes[C]{\smallbreak}\pstart
           \raggedleft{}{\pb}9. 11. 03\pend
           \pstart\center{}Lieber Arthur!\pend\pstart
           Ich habe geſtern Dein »\textcolor{green}{Excentric}{}\ledrightnote{\textcolor{green}{Excentric}}« vorgeleſen und
               die Leute haben über das liebenswürdige Fräulein de la Roſière ſo gebrüllt, daß ich
               wirklich bisweilen eine Minute lang warten mußte, bis ſie ſich ſo weit gefaßt hatten,
               mich wieder anzuhören. Die Geſchichte iſt köſtlich und zum Vorleſen ideal. Ich
               ſchicke Dir das Heft mit derſelben Poſt zurück, ich habe mir die betr. Nummer der
                  \label{K_L01337_1v}\edtext{\textcolor{green}{Jugend}{}\ledrightnote{\textcolor{green}{Jugend}}}{\lemma{\textnormal{\emph{Jugend}}}\Cendnote{\textnormal{\textcolor{blue}{Arthur Schnitzler}: \emph{\textcolor{green}{Excentric}}. In: \emph{\textcolor{green}{Jugend}},
                     Jg. 7, Nr. 30, {[}16.{]} 7. 1902, S. 492–496.}}}\label{K_L01337_1h} bereits
               verſchafft.\pend
           \pstart
           Noch etwas, ganz aufrichtig. Da Du keine Sitze von mir verlangt haſt, habe ich Dir
               keine \substVorne{}\textsuperscript{\textcolor{gray}{×}}\substDazwischen{}g\substHinten{}eſchickt, weil mir das von mir immer so furchtbar aufdringlich vorkommt,
               Jemandem ungebeten Sitze zu ſchicken, der dann am End erſt ſeine Köchin anflehen muß,
               ſie zu benützen.\pend
           \pstart
           {\pb}Anbei findeſt Du den \label{K_L01337_2v}\edtext{Rek\textcolor{gray}{o}urs, der am 5. d. der \textcolor{brown}{Statthalterei}{}\ledrightnote{\textcolor{brown}{Niederösterreichische Statthalterei}} überreicht worden iſt}{\lemma{\textnormal{\emph{Rekours, … iſt}}}\Cendnote{\textnormal{Vgl. \textcolor{blue}{Schnitzler} an \textcolor{blue}{Otto P. Schinnerer}, 6. 2. 1930, in A. S.
                        \emph{Briefe} II,660–664.}}}\label{K_L01337_2h}. Er iſt von mir
               mit \textcolor{blue}{Burckhard}{}\ledrightnote{\textcolor{blue}{Max Eugen Burckhard}} berathen und dann von dieſem
               verfaßt worden, was aber, nach ſeinem Wunſch, nicht bekannt werden ſoll. Verſuche,
               den Rekurs in irgend eine \textcolor{pink}{Wiener}{}\ledrightnote{\textcolor{pink}{Wien}} Zeitung zu bringen,
               ſind durchaus misglückt. Überlege, ob Du ihn eventuell der nächſten Auflage des \textcolor{green}{Reigens}{}\ledrightnote{\textcolor{green}{Reigen. Zehn Dialoge}} vordrucken würdeſt. Sag aber nur offen Nein,
               wenn es Dir nicht paßt.\pend
           \pstart
           \label{K_L01337_3v}\edtext{\textcolor{blue}{Salten}{}\ledrightnote{\textcolor{blue}{Felix Salten}} tuſt Du glaub ich unrecht}{\lemma{\textnormal{\emph{Salten … unrecht}}}\Cendnote{\textnormal{Das könnte auf ein verlorenes
                  Korrespondenzstück hinweisen; zum Inhalt siehe die Antwort \textcolor{blue}{Schnitzlers}.}}}\label{K_L01337_3h}. Du mußt nur doch die für ihn
               unglaublich heikle und gefährliche Situation bedenken, in der er geſchrieben hat.
               Aber darüber mündlich.\pend
           \pstart
           Mit den beſten Grüßen an Deine \textcolor{blue}{Frau}{}\ledrightnote{→\textcolor{blue}{Olga Schnitzler}}{\\[\baselineskip]}herzlichſt Dein{\\[\baselineskip]}\spacefill\mbox{Hermann}\pend
           \leftskip=0em{}\endnumbering\briefempfaengerindex{Schnitzler, Arthur@\textsc{Schnitzler, Arthur}!zzzBahr, Hermann@\emph{von Hermann Bahr}!1903-11-091@{9. 11. 1903}|)be}\mylabel{h}  \normalsize

\doendnotes{C}
\bigskip
\vfill

\clearpage

\footnotesize

\lohead{\textsc{register}}

% Definiere theindex-Environment komplett neu ohne reledmac
\makeatletter
\renewenvironment{theindex}{%
  \section*{\indexname}%
  \setlength{\parindent}{0pt}%
  \setlength{\parskip}{0pt plus 0.3pt}%
  \let\item\@idxitem
}{%
  \clearpage
}
\makeatother

\IfFileExists{\jobname-pw.ind}{\input{\jobname-pw.ind}}{}

\end{document}

      