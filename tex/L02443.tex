%% latex-korrekturansicht-vorspann.tex
%% Vorspann für die Korrekturansicht.
%% Lädt die gemeinsame Datei latex-vorspann.tex mit gesetztem Schalter.

\newif\ifkorrekturansicht
\korrekturansichttrue

\input{../tex-inputs/latex-vorspann}


               \section[Georg Brandes an Arthur Schnitzler, 21. 6. 1925]{ Georg Brandes an Arthur Schnitzler, 21. 6. 1925}\nopagebreak\mylabel{v}\rehead{ }\normalsize\beginnumbering\briefempfaengerindex{Schnitzler, Arthur@\textsc{Schnitzler, Arthur}!zzzBrandes, Georg@\emph{von Georg Brandes}!1925-06-211@{21. 6. 1925}|(be} \toendnotes[C]{\smallbreak\pagebreak[2]} \Standort{CUL, Schnitzler, B 17.}
\physDesc{Brief, 1 Blatt, 3 Seiten
\newline{}Handschrift: schwarze Tinte, lateinische Kurrent
\newline{}Schnitzler: mit rotem Buntstift vereinzelte Unterstreichungen \newline{}Ordnung: mit Bleistift von unbekannter Hand nummeriert: »59« }\buchAbdrucke{\weitereDrucke{Georg Brandes, Arthur Schnitzler: \emph{Ein Briefwechsel}. Hg. Kurt Bergel. Bern: \emph{Francke} 1956, S. 146–147.} }\toendnotes[C]{\smallbreak}\pstart
           \raggedleft{}{\pb}\textcolor{pink}{Kopenhagen}{}\ledrightnote{\textcolor{pink}{Kopenhagen}}{ }21 Juni 25\pend
           \pstart{}Mein lieber Freund\pend\pstart
           Sie waren diesmal wieder sehr gütig gegen mich in \textcolor{pink}{Wien}{}\ledrightnote{\textcolor{pink}{Wien}}. Ich ging nach \textcolor{pink}{Salzburg}{}\ledrightnote{\textcolor{pink}{Salzburg}}, verlor
                    aber dort vier Wochen mit Bronchitis, bin hier, und kann über die Gesundheit
                    nicht klagen, obwol der Sommer hier kalt und unheimlich ist.\pend
           \pstart
           Ich hätte Ihnen sehr gerne mein kleines Buch \textcolor{green}{Hellas}{}\ledrightnote{\textcolor{green}{Hellas}} geschickt, aber leider durch allerlei Verlegerschwierigkeiten
                    lässt die deutsche Uebersetzung auf sich warten.\pend
           \pstart
           Es war schön, Sie und Ihr Haus wieder zu sehn. Es that mir leid zu merken, dass
                    Ihre Stimmung nicht heiter war. Sie waren nicht deshalb weniger liebenswürdig,
                    aber ich gönnte {\pb}Ihnen mehr
                    Lebensfreude.\pend
           \pstart
           Man hat ja seitdem ein älteres \textcolor{green}{Schauspiel}{}\ledrightnote{→\textcolor{green}{Der Schleier der Beatrice. Schauspiel in fünf Akten}} von Ihnen im \textcolor{pink}{Burgtheater}{}\ledrightnote{\textcolor{pink}{Burgtheater}}\label{K_L02443_1v}\edtext{aufgeführt}{\lemma{\textnormal{\emph{aufgeführt}}}\Cendnote{\textnormal{Erste \textcolor{pink}{Wien}er Aufführung am
                            23. 5. 1925}}}\label{K_L02443_1h}; ich hoffe, dass die Poesie des \textcolor{green}{Stückes}{}\ledrightnote{→\textcolor{green}{Der Schleier der Beatrice. Schauspiel in fünf Akten}} zu ihrem Rechte kam.
                    Es muss doch ein angenehmes Gefühl sein, auf viele Menschen zugleich zu wirken.
                    Sie sind diesem Genuss gegenüber wol etwas verwöhnt und blasirt, aber nicht
                    desto weniger!\pend
           \pstart
           Ich wurde eingeladen, die Festlichkeiten wegen des 200 jährigen Bestehens der \textcolor{brown}{Academie der Wissenschaften}{}\ledrightnote{\textcolor{brown}{Akademie der Wissenschaften}} in \textcolor{pink}{Leningrad}{}\ledrightnote{\textcolor{pink}{Sankt Petersburg}} (!) mitzumachen; sie strecken sich in \textcolor{pink}{Petersburg}{}\ledrightnote{\textcolor{pink}{Sankt Petersburg}} und \textcolor{pink}{Moskau}{}\ledrightnote{\textcolor{pink}{Moskau}} von 6–16 September, aber ich wollte
                    als Gast nicht heucheln, und Entzücken über den {\pb}jetzigen Zustand in \textcolor{pink}{Russland}{}\ledrightnote{\textcolor{pink}{Russland}} wäre meinerseits Heuchelei. Reden
                    müsste ich ja, und das schreckte mich. Sonst hätte ich gerne die zwei \textcolor{pink}{Städte}{}\ledrightnote{→\textcolor{pink}{Sankt Petersburg}{\newline}→\textcolor{pink}{Moskau}} unter den
                    veränderten Umständen wiedergesehen.\pend
           \pstart
           Sie waren sehr lieb so wol gegen meine \textcolor{blue}{Begleiterin}{}\ledrightnote{→\textcolor{blue}{Gertrud Rung}} wie gegen mich.\pend
           \pstart
           Leider reist jetzt Fru \textcolor{blue}{Rung}{}\ledrightnote{\textcolor{blue}{Gertrud Rung}} mit ihrem \textcolor{blue}{Gatten}{}\ledrightnote{→\textcolor{blue}{Otto Rung}} und ihrer \textcolor{blue}{Cousine}{}\ledrightnote{→\textcolor{blue}{?? [Kusine von Gertrud Rung]}} auf 6 Wochen nach
                        \textcolor{pink}{Italien}{}\ledrightnote{\textcolor{pink}{Italien}}. Ich kann ohne sie meine
                    Correspondenz nicht bewältigen.\pend
           \pstart
           Sie wissen kaum, wie dankbar ich mich im Innersten für Ihre vieljährige
                    Freundschaft fühle.\pend
           \pstart Ihr \spacefill\mbox{Georg Brandes}\pend{}\endnumbering\briefempfaengerindex{Schnitzler, Arthur@\textsc{Schnitzler, Arthur}!zzzBrandes, Georg@\emph{von Georg Brandes}!1925-06-211@{21. 6. 1925}|)be}\mylabel{h}  \normalsize

\doendnotes{C}
\bigskip
\vfill

\clearpage

\footnotesize

\lohead{\textsc{register}}

% Definiere theindex-Environment komplett neu ohne reledmac
\makeatletter
\renewenvironment{theindex}{%
  \section*{\indexname}%
  \setlength{\parindent}{0pt}%
  \setlength{\parskip}{0pt plus 0.3pt}%
  \let\item\@idxitem
}{%
  \clearpage
}
\makeatother

\IfFileExists{\jobname-pw.ind}{\input{\jobname-pw.ind}}{}

\end{document}

      