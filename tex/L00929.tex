%% latex-korrekturansicht-vorspann.tex
%% Vorspann für die Korrekturansicht.
%% Lädt die gemeinsame Datei latex-vorspann.tex mit gesetztem Schalter.

\newif\ifkorrekturansicht
\korrekturansichttrue

\input{../tex-inputs/latex-vorspann}


               \section[Arthur Schnitzler an Richard Beer-Hofmann, 21. 6. 1899]{ Arthur Schnitzler an Richard Beer-Hofmann, 21. 6. 1899}\nopagebreak\mylabel{v}\rehead{ }\normalsize\beginnumbering\briefempfaengerindex{Beer-Hofmann, Richard@\textsc{Beer-Hofmann, Richard}!zzzSchnitzler, Arthur@\emph{von Arthur Schnitzler}!1899-06-212@{21. 6. 1899}|(be} \toendnotes[C]{\smallbreak\pagebreak[2]} \Standort{YCGL, MSS 31.}
\physDesc{Brief, 1 Blatt, 1 Seite, Umschlag
\newline{}Handschrift: Bleistift, deutsche Kurrent\newline{}Versand: 1) Stempel: »\nobreak{}\oindex{IX., Alsergrund@\textbf{IX., Alsergrund}, \emph{Bezirk (A.BZK)}|pwk}Wien 9/3, 21. 6. 99, 6–7N\nobreak{}«.  2) Stempel: »\nobreak{}\oindex{Seeboden@\textbf{Seeboden}, \emph{http://www.geonames.org/ontologyA.ADM3}|pwk}{\pb}Seeboden, 22. 6. 99\nobreak{}«. }\buchAbdrucke{\weitereDrucke{Arthur Schnitzler, Richard Beer-Hofmann: \emph{Briefwechsel 1891–1931}. Hg. Konstanze Fliedl. Wien, Zürich: \emph{Europaverlag} 1992, S. 131.} }\toendnotes[C]{\smallbreak}\pstart{}{\pb}Herrn \textsc{Dr. Rich
                     Beer-Hofmann}\pend{}\pstart{}\textcolor{pink}{\textsc{Seeboden} am Millſtätterſee}{}\ledrightnote{\textcolor{pink}{Seeboden}}\pend{}\pstart{}\textcolor{pink}{Villa Platzer}{}\ledrightnote{\textcolor{pink}{Villa Platzer}}\pend{}{\bigskip}\pstart
           \noindent{}{\pb}Nr.{ }2.\hfill 21. 6. 99\pend
           \pstart
           Noch eins:\pend
           \pstart
           wie heißen diese \label{K_L00929_1v}\edtext{Cloſche}{\lemma{\textnormal{\emph{Cloſche}}}\Cendnote{\textnormal{französisch cloche: Hütchen}}}\label{K_L00929_1h} oder vielmehr
               Selbſtſchützer, die Sie mir neulich triumphirend gezeigt, u wo beko{\geminationm}t man ſie?\pend
           \pstart I\textcolor{gray}{h}r \spacefill\mbox{A.}\pend{}\endnumbering\briefempfaengerindex{Beer-Hofmann, Richard@\textsc{Beer-Hofmann, Richard}!zzzSchnitzler, Arthur@\emph{von Arthur Schnitzler}!1899-06-212@{21. 6. 1899}|)be}\mylabel{h}  \normalsize

\doendnotes{C}
\bigskip
\vfill

\clearpage

\footnotesize

\lohead{\textsc{register}}

% Definiere theindex-Environment komplett neu ohne reledmac
\makeatletter
\renewenvironment{theindex}{%
  \section*{\indexname}%
  \setlength{\parindent}{0pt}%
  \setlength{\parskip}{0pt plus 0.3pt}%
  \let\item\@idxitem
}{%
  \clearpage
}
\makeatother

\IfFileExists{\jobname-pw.ind}{\input{\jobname-pw.ind}}{}

\end{document}

      