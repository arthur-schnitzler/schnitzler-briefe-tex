%% latex-korrekturansicht-vorspann.tex
%% Vorspann für die Korrekturansicht.
%% Lädt die gemeinsame Datei latex-vorspann.tex mit gesetztem Schalter.

\newif\ifkorrekturansicht
\korrekturansichttrue

\input{../tex-inputs/latex-vorspann}


               \section[Max Burckhard an Arthur Schnitzler, 25. 6. 1896]{ Max Burckhard an Arthur Schnitzler, 25. 6. 1896}\nopagebreak\mylabel{v}\rehead{ }\normalsize\beginnumbering\briefempfaengerindex{Schnitzler, Arthur@\textsc{Schnitzler, Arthur}!zzzBurckhard, Max Eugen@\emph{von Max Eugen Burckhard}!1896-06-252@{25. 6. 1896}|(be} \toendnotes[C]{\smallbreak\pagebreak[2]} \Standort{CUL, Schnitzler, B 20.}
\physDesc{Brief, 1 Blatt, 1 Seite
\newline{}Handschrift: schwarze Tinte, deutsche Kurrent
\newline{}Schnitzler: mit Bleistift nummeriert: »8« }\toendnotes[C]{\smallbreak}\pstart
           \noindent{}{\pb}\textcolor{gray}{\textbf{\textcolor{pink}{k. k. Hofburgtheater}{}\ledrightnote{\textcolor{pink}{Burgtheater}} Direction}}\hfill \textcolor{pink}{Wien}{}\ledrightnote{\textcolor{pink}{Wien}}{ }25/6 96\pend
           \pstart{}Sehr verehrter Herr Doctor!\pend\pstart
           Bezugnehmend auf unſere mündliche Rückſprache bin ich ſo frei, Ihnen die Komödie
                        \label{K_L00553_1v}\edtext{\textcolor{green}{Der Glückspilz}{}\ledrightnote{\textcolor{green}{Brignol et sa fille}{\newline}\textcolor{green}{Innocent}}}{\lemma{\textnormal{\emph{Der Glückspilz}}}\Cendnote{\textnormal{Unklar. Von \textcolor{blue}{Alfred Capus} kommen die zwei Stücke \emph{\textcolor{green}{Brignolle et sa fille}} (Uraufführung
                            23. 11. 1894) und \emph{\textcolor{green}{Innocent}} (Uraufführung 7. 2. 1896) in Frage.}}}\label{K_L00553_1h}
                    von \textcolor{blue}{\textsc{Capus}}{}\ledrightnote{\textcolor{blue}{Alfred Capus}} mit verbindlichem Danke für Ihre freundliche Bemühung zurückzuſenden.\pend
           \pstart
           In aufrichtiger Verehrung{\\[\baselineskip]}\spacefill\mbox{D\textsuperscript{r}Burckhard}\pend
           \leftskip=0em{}\endnumbering\briefempfaengerindex{Schnitzler, Arthur@\textsc{Schnitzler, Arthur}!zzzBurckhard, Max Eugen@\emph{von Max Eugen Burckhard}!1896-06-252@{25. 6. 1896}|)be}\mylabel{h}  \normalsize

\doendnotes{C}
\bigskip
\vfill

\clearpage

\footnotesize

\lohead{\textsc{register}}

% Definiere theindex-Environment komplett neu ohne reledmac
\makeatletter
\renewenvironment{theindex}{%
  \section*{\indexname}%
  \setlength{\parindent}{0pt}%
  \setlength{\parskip}{0pt plus 0.3pt}%
  \let\item\@idxitem
}{%
  \clearpage
}
\makeatother

\IfFileExists{\jobname-pw.ind}{\input{\jobname-pw.ind}}{}

\end{document}

      