%% latex-korrekturansicht-vorspann.tex
%% Vorspann für die Korrekturansicht.
%% Lädt die gemeinsame Datei latex-vorspann.tex mit gesetztem Schalter.

\newif\ifkorrekturansicht
\korrekturansichttrue

\input{../tex-inputs/latex-vorspann}


               \section[Hermann Bahr an Arthur Schnitzler, 29. 9. 1907]{ Hermann Bahr an Arthur Schnitzler, 29. 9. 1907}\nopagebreak\mylabel{v}\rehead{ }\normalsize\beginnumbering\briefempfaengerindex{Schnitzler, Arthur@\textsc{Schnitzler, Arthur}!zzzBahr, Hermann@\emph{von Hermann Bahr}!1907-09-291@{29. 9. 1907}|(be} \toendnotes[C]{\smallbreak\pagebreak[2]} \Standort{CUL, Schnitzler, B 5b.}
\physDesc{Brief, 1 Blatt, 2 Seiten
\newline{}Handschrift Lisa Clarus: blaue Tinte, lateinische Kurrent\newline{}Handschrift Hermann Bahr: blaue Tinte (\noindent{}Unterschrift)\newline{}Ordnung: mit Bleistift von unbekannter Hand nummeriert:
                                    »151« }\buchAbdrucke{\weitereDrucke{Hermann Bahr, Arthur Schnitzler: \emph{Briefwechsel, Aufzeichnungen, Dokumente (1891–1931)}. Hg. Kurt Ifkovits und Martin Anton Müller. Göttingen: \emph{Wallstein} 2018, S. 395.} }\toendnotes[C]{\smallbreak}\pstart
           \raggedleft{}{\pb}29. 9. 07.\pend
           \pstart\center{}Lieber Arthur!\pend\pstart
           Ich habe, seit ich \label{K_L01713_1v}\edtext{zurück}{\lemma{\textnormal{\emph{zurück}}}\Cendnote{\textnormal{Ab dem 4. 9. 1907 verbrachte \textcolor{blue}{Bahr} ein paar Tage am \textcolor{pink}{Semmering}. Möglicherweise ist das auch auf den Sommerurlaub
                  zu beziehen, von dem er spätestens am 21. 8. 1907 zurückgekehrt
                  war.}}}\label{K_L01713_1h} bin, jeden Tag zu Dir wollen, jeden Tag kam was anderes dazwischen und
               ich war so gehetzt, dass es leider wirklich nicht gieng. Nun wieder nach \textcolor{pink}{Berlin}{}\ledrightnote{\textcolor{pink}{Berlin}} abreisend, kann ich Dir und Deiner lieben \textcolor{blue}{Frau}{}\ledrightnote{→\textcolor{blue}{Olga Schnitzler}} nur noch die herzlichsten
               Grüsse und alle guten Wünsche für den Winter schicken. Ich möchte Dir noch sagen,
               dass \textcolor{blue}{uns}{}\ledrightnote{→\textcolor{blue}{Anna Bahr-Mildenburg}} im Sommer Dein neues
               Buch, »\textcolor{green}{Dämmerseelen}{}\ledrightnote{\textcolor{green}{Dämmerseelen. Novellen}}«, ein sehr lieber Gefährte war,
               und möchte Dich bitten, Dir von \textcolor{blue}{Salten}{}\ledrightnote{\textcolor{blue}{Felix Salten}}\strikeout{,} mein neues \textcolor{green}{Stück}{}\ledrightnote{→\textcolor{green}{Die gelbe Nachtigall}}{ }{\pb}geben zu lassen und es dann an \textcolor{blue}{Richard}{}\ledrightnote{\textcolor{blue}{Richard Beer-Hofmann}} weiter zu geben; ich habe leider jetzt kein anderes
               Exemplar frei und wünsche sehr, dass Du den Scherz kennen lernen mögest.\pend
           \pstart
           Herzlichst{\\[\baselineskip]}Dein alter{\\[\baselineskip]}\spacefill\mbox{{[}hs. Bahr:{]} Hermann}\pend
           \leftskip=0em{}\endnumbering\briefempfaengerindex{Schnitzler, Arthur@\textsc{Schnitzler, Arthur}!zzzBahr, Hermann@\emph{von Hermann Bahr}!1907-09-291@{29. 9. 1907}|)be}\mylabel{h}  \normalsize

\doendnotes{C}
\bigskip
\vfill

\clearpage

\footnotesize

\lohead{\textsc{register}}

% Definiere theindex-Environment komplett neu ohne reledmac
\makeatletter
\renewenvironment{theindex}{%
  \section*{\indexname}%
  \setlength{\parindent}{0pt}%
  \setlength{\parskip}{0pt plus 0.3pt}%
  \let\item\@idxitem
}{%
  \clearpage
}
\makeatother

\IfFileExists{\jobname-pw.ind}{\input{\jobname-pw.ind}}{}

\end{document}

      