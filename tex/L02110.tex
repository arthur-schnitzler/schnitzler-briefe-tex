%% latex-korrekturansicht-vorspann.tex
%% Vorspann für die Korrekturansicht.
%% Lädt die gemeinsame Datei latex-vorspann.tex mit gesetztem Schalter.

\newif\ifkorrekturansicht
\korrekturansichttrue

\input{../tex-inputs/latex-vorspann}


               \section[Arthur Schnitzler an Hugo von Hofmannsthal, 21. 12. 1912]{ Arthur Schnitzler an Hugo von Hofmannsthal, 21. 12. 1912}\nopagebreak\mylabel{v}\rehead{ }\normalsize\beginnumbering\briefempfaengerindex{Hofmannsthal, Hugo von@\textsc{Hofmannsthal, Hugo von}!zzzSchnitzler, Arthur@\emph{von Arthur Schnitzler}!1912-12-211@{21. 12. 1912}|(be} \toendnotes[C]{\smallbreak\pagebreak[2]} \Standort{FDH, Hs-30885,146.}
\physDesc{Briefkarte
\newline{}Handschrift: schwarze Tinte, deutsche Kurrent}\buchAbdrucke{\weitereDrucke{Hugo von Hofmannsthal, Arthur Schnitzler: \emph{Briefwechsel}. Hg. Therese Nickl und Heinrich Schnitzler. Frankfurt am Main: \emph{S. Fischer} 1964, S. 271.} }\toendnotes[C]{\smallbreak}\pstart
           \noindent{}{\pb}\textcolor{gray}{\textbf{Dr. Arthur
                        Schnitzler}}\hfill \textcolor{pink}{Wien}{}\ledrightnote{\textcolor{pink}{Wien}},
                     21. 12. 912\pend
           \pstart
           \textcolor{gray}{\textbf{\textcolor{pink}{Wien XVIII. Sternwartestrasse 71}{}\ledrightnote{\textcolor{pink}{Sternwartestraße}}}}\pend
           \pstart
           lieber Hugo, eben mit dem \textcolor{pink}{V.
                  Th.}{}\ledrightnote{\textcolor{pink}{Volkstheater}}{ }\textsc{telephonirt};
               ſie haben mit der \textcolor{blue}{\textsc{Roland}}{}\ledrightnote{\textcolor{blue}{Ida Roland}} noch nicht abgeschloſſen, ſchienen über die Ausſicht \textcolor{blue}{\textsc{Terwin}}{}\ledrightnote{\textcolor{blue}{Johanna Terwin}} poſitiv erfreut. Würde rathen,
               daſs ſich die \textcolor{blue}{T.}{}\ledrightnote{\textcolor{blue}{Johanna Terwin}} ganz direct mit dem \textcolor{pink}{V. Th.}{}\ledrightnote{\textcolor{pink}{Volkstheater}} in Verbindung ſetzt; u. zw. ſo geſchwind wie
               möglich. –\pend
           \pstart
           Mit \textcolor{blue}{\textsc{Thimig}}{}\ledrightnote{\textcolor{blue}{Hugo Thimig}} heute nur zwei Worte {\pb}auf der \textcolor{green}{Probe}{}\ledrightnote{→\textcolor{green}{Das Märchen vom Wolf}}; er habe mir einiges intereſſante zu
               ſagen, werde mich nächſtens beſuchen. (Er war auch \label{K_L02110_1v}\edtext{vor ein paar Wochen}{\lemma{\textnormal{\emph{vor ein paar Wochen}}}\Cendnote{\textnormal{siehe A. S.: \emph{Tagebuch}, 22. 10. 1912}}}\label{K_L02110_1h} bei mir) Bei
               dieſer Gelegenheit gedenke ich den \textcolor{green}{Jederma{\geminationn}}{}\ledrightnote{\textcolor{green}{Jedermann. Das Spiel vom Sterben des reichen Mannes}} anzuſchneiden.\pend
           \pstart
           Herzlichſt Ihr{\\[\baselineskip]}\spacefill\mbox{A.}\pend
           \leftskip=0em{}\endnumbering\briefempfaengerindex{Hofmannsthal, Hugo von@\textsc{Hofmannsthal, Hugo von}!zzzSchnitzler, Arthur@\emph{von Arthur Schnitzler}!1912-12-211@{21. 12. 1912}|)be}\mylabel{h}  \normalsize

\doendnotes{C}
\bigskip
\vfill

\clearpage

\footnotesize

\lohead{\textsc{register}}

% Definiere theindex-Environment komplett neu ohne reledmac
\makeatletter
\renewenvironment{theindex}{%
  \section*{\indexname}%
  \setlength{\parindent}{0pt}%
  \setlength{\parskip}{0pt plus 0.3pt}%
  \let\item\@idxitem
}{%
  \clearpage
}
\makeatother

\IfFileExists{\jobname-pw.ind}{\input{\jobname-pw.ind}}{}

\end{document}

      