%% latex-korrekturansicht-vorspann.tex
%% Vorspann für die Korrekturansicht.
%% Lädt die gemeinsame Datei latex-vorspann.tex mit gesetztem Schalter.

\newif\ifkorrekturansicht
\korrekturansichttrue

\input{../tex-inputs/latex-vorspann}


               \section[Arthur Schnitzler an Richard Beer-Hofmann, 15. 7. 1896]{ Arthur Schnitzler an Richard Beer-Hofmann, 15. 7. 1896}\nopagebreak\mylabel{v}\rehead{ }\normalsize\beginnumbering\briefempfaengerindex{Beer-Hofmann, Richard@\textsc{Beer-Hofmann, Richard}!zzzSchnitzler, Arthur@\emph{von Arthur Schnitzler}!1896-07-151@{15. 7. 1896}|(be} \toendnotes[C]{\smallbreak\pagebreak[2]} \Standort{YCGL, MSS 31.}
\physDesc{Postkarte
\newline{}Handschrift: Bleistift, deutsche Kurrent\newline{}Versand: 1) Stempel: »\nobreak{}\oindex{Trondheim@\textbf{Trondheim}, \emph{http://www.geonames.org/ontologyP.PPLA2}|pwk}Trondhjem, 15. VII. 96\nobreak{}«.  2) Stempel: »\nobreak{}\oindex{Kopenhagen@\textbf{Kopenhagen}, \emph{Besiedelter Ort (A.BSO)}|pwk}Kjobenhavn, 17. 7. 96, 50M6\nobreak{}«. }\buchAbdrucke{\weitereDrucke{Arthur Schnitzler, Richard Beer-Hofmann: \emph{Briefwechsel 1891–1931}. Hg. Konstanze Fliedl. Wien, Zürich: \emph{Europaverlag} 1992, S. 93.} }\pstart{}{\pb}Herrn \textsc{Dr. Richard
                     Beer-Hofmann}\pend{}\pstart{}\textcolor{pink}{\textsc{Kopenhagen}}{}\ledrightnote{\textcolor{pink}{Kopenhagen}}\pend{}\pstart{}\textsc{post rest.}\pend{}{\bigskip}\pstart
           \raggedleft{}{\pb}\textcolor{pink}{Trondhjem}{}\ledrightnote{\textcolor{pink}{Trondheim}}{ }15/7 96\pend
           \pstart
           Mein lieber Richard, ich freue mich hier Ihren Brief gefunden zu
               haben – ich antwort Ihnen gleich dieſe 2 Zeilen, weil es heut Abend wieder weiter und
                  i{\geminationm}er weiter geht. Glauben Sie mir, Sie haben viel
               verſäumt – nun eben ko{\geminationm} ich aus dem \textcolor{pink}{Dom von Drontheim}{}\ledrightnote{\textcolor{pink}{Nidarosdom}} und hab Ihnen was gekauft.\pend
           \pstart
           Sie wiſſen ja, wo mich Ihren Briefe treffen. – Arbeiten werden Sie hoffentlich mehr
               als ich –?\pend
           \pstart
           Herzlich{\\[\baselineskip]}Ihr \spacefill\mbox{Arthur.}\pend
           \leftskip=0em{}\endnumbering\briefempfaengerindex{Beer-Hofmann, Richard@\textsc{Beer-Hofmann, Richard}!zzzSchnitzler, Arthur@\emph{von Arthur Schnitzler}!1896-07-151@{15. 7. 1896}|)be}\mylabel{h}  \normalsize

\doendnotes{C}
\bigskip
\vfill

\clearpage

\footnotesize

\lohead{\textsc{register}}

% Definiere theindex-Environment komplett neu ohne reledmac
\makeatletter
\renewenvironment{theindex}{%
  \section*{\indexname}%
  \setlength{\parindent}{0pt}%
  \setlength{\parskip}{0pt plus 0.3pt}%
  \let\item\@idxitem
}{%
  \clearpage
}
\makeatother

\IfFileExists{\jobname-pw.ind}{\input{\jobname-pw.ind}}{}

\end{document}

      