%% latex-korrekturansicht-vorspann.tex
%% Vorspann für die Korrekturansicht.
%% Lädt die gemeinsame Datei latex-vorspann.tex mit gesetztem Schalter.

\newif\ifkorrekturansicht
\korrekturansichttrue

\input{../tex-inputs/latex-vorspann}


               \section[Hugo von Hofmannsthal an Arthur Schnitzler, 4. 8. {[}1892{]}]{ Hugo von Hofmannsthal an Arthur Schnitzler, 4. 8. {[}1892{]}}\nopagebreak\mylabel{v}\rehead{ }\normalsize\beginnumbering\briefempfaengerindex{Schnitzler, Arthur@\textsc{Schnitzler, Arthur}!zzzHofmannsthal, Hugo von@\emph{von Hugo von Hofmannsthal}!1892-08-041@{4. 8. {[}1892{]}}|(be} \toendnotes[C]{\smallbreak\pagebreak[2]} \Standort{CUL, Schnitzler, B 43.}
\physDesc{Brief, 1 Blatt (Briefpapier mit aufgeprägtem Wappen), 4 Seiten
\newline{}Handschrift: schwarze Tinte, deutsche Kurrent
\newline{}Schnitzler: mit Bleistift die Jahreszahl ergänzt: »92« \newline{}Ordnung: mit Bleistift von unbekannter Hand nummeriert:
                                    »29« }\buchAbdrucke{\weitereDrucke{1) Hugo von Hofmannsthal: \emph{Briefe an Freunde.} In: \emph{Die neue Rundschau}, Jg. 41, Nr. 4, April 1930, S. 513–514.} \weitereDrucke{2) Hugo von Hofmannsthal: \emph{Briefe. 1890–1901}. Berlin: \emph{S. Fischer} 1935, S. 60–61.} \weitereDrucke{3) Hugo von Hofmannsthal, Arthur Schnitzler: \emph{Briefwechsel}. Hg. Therese Nickl und Heinrich Schnitzler. Frankfurt am Main: \emph{S. Fischer} 1964, S. 26.} }\toendnotes[C]{\smallbreak}\pstart
           \raggedleft{}{\pb}\textcolor{pink}{Fuſch}{}\ledrightnote{\textcolor{pink}{Bad Fusch}}{\\}4 VIII.\pend
           \pstart\center{}Lieber Arthur.\pend\pstart
           Da haben Sie die \label{K_L00111_1v}\edtext{\textcolor{green}{Märchen}{}\ledrightnote{\textcolor{green}{Das Märchen. Schauspiel in drei Aufzügen}}kritik}{\lemma{\textnormal{\emph{Märchenkritik}}}\Cendnote{\textnormal{nicht publizierte und nicht erhaltene Kritik}}}\label{K_L00111_1h} der \textcolor{blue}{\textsc{Herzfeld}}{}\ledrightnote{\textcolor{blue}{Marie Herzfeld}}. Ich habe ihr für die ſympathiſche Ausführlichkeit gedankt und ihr von dem
               Erſcheinen des \textcolor{green}{Anatol}{}\ledrightnote{\textcolor{green}{Anatol}}-Buches geſprochen; wie heißt
               denn der \textcolor{brown}{Verlag}{}\ledrightnote{\textcolor{brown}{Bibliographisches Bureau}}? –\pend
           \pstart
           Ich habe den \textcolor{green}{erſten Act}{}\ledrightnote{→\textcolor{green}{Ascanio und Gioconda}}
               (654 Verſe) vollendet, den \textcolor{green}{zweiten}{}\ledrightnote{→\textcolor{green}{Ascanio und Gioconda}} beinahe.\pend
           \pstart
           Unſere Art zu arbeiten (im Drama) iſt nicht gar ſo verſchieden, wie Sie anzunehmen
               ſcheinen; was ich {\pb}aus ſpäteren
               Acten vorausarbeiten kann, ſind nicht geſchloſſene Scenen, ſondern reine
               Farbenſkizzen: Worte und Dialogſtellen, die oft dann gar nicht wirklich aufgenommen
               werden, mir aber als Parfümflaſchen, als Stimmungs-Accumulatoren und -Condenſatoren
               dienen, damit die Suggeſtion im Laufe der Detailarbeit nicht verloren geht; das ganze
               hängt wahrſcheinlich mit meiner Ihnen gegenüber mehr lyriſchen, mehr auf Farbe
               hinarbeitenden Technik zuſammen. Wie lange {\pb}bleiben Sie in \textcolor{pink}{Wien}{}\ledrightnote{\textcolor{pink}{Wien}}? kann man Ihnen während der Waffenübung ſchreiben?\pend
           \pstart
           Ich freue mich ſehr auf die \textcolor{green}{Novelle}{}\ledrightnote{\textcolor{green}{Sterben. Novelle}}; ich hoffe Sie
               werden nichts vor meiner Rückkehr vorleſen.\pend
           \pstart
           Ich bin vom 7\textsuperscript{ten} – 31\textsuperscript{ten} Auguſt in \textcolor{pink}{Strobl bei Iſchl}{}\ledrightnote{\textcolor{pink}{Strobl}}.\pend
           \pstart
           Herzlichſt grüßend{\\[\baselineskip]}\spacefill\mbox{Loris.}\pend
           \leftskip=0em{}\pstart
           \noindent{}\textsc{P. S.} Was die \textcolor{blue}{\textsc{Herzfeld}}{}\ledrightnote{\textcolor{blue}{Marie Herzfeld}} von nothwendiger Technik \strikeout{und} für
                  Bühnenfernwirkung und von »concentrierter« Natürlichkeit des Dialog’s ſagt,
                  ſcheint mir ſehr vernünftig; {\pb}es iſt dies thatſächlich die Erfahrung des allerletzten Theaterjahres für jeden
                  Objectiven und für künftige Arbeiten nicht unwichtig: ganz die gleichen
                  Rathſchläge, mit zahlloſen anderen höchſt wertvollen, finde ich in den kritiſchen
                  Studien von \textcolor{blue}{Otto Ludwig}{}\ledrightnote{\textcolor{blue}{Otto Ludwig}}, aus denen ich hier
                  mit Genuſs und innerer Freude eine Menge lerne. Über Technik des dramatiſchen
                  Dramas zum Unterſchied vom herrſchenden Novellendrama muſs überhaupt nächſten
                  Winter bei Ihnen ſehr viel geredet werden.\pend
           \endnumbering\briefempfaengerindex{Schnitzler, Arthur@\textsc{Schnitzler, Arthur}!zzzHofmannsthal, Hugo von@\emph{von Hugo von Hofmannsthal}!1892-08-041@{4. 8. {[}1892{]}}|)be}\mylabel{h}  \normalsize

\doendnotes{C}
\bigskip
\vfill

\clearpage

\footnotesize

\lohead{\textsc{register}}

% Definiere theindex-Environment komplett neu ohne reledmac
\makeatletter
\renewenvironment{theindex}{%
  \section*{\indexname}%
  \setlength{\parindent}{0pt}%
  \setlength{\parskip}{0pt plus 0.3pt}%
  \let\item\@idxitem
}{%
  \clearpage
}
\makeatother

\IfFileExists{\jobname-pw.ind}{\input{\jobname-pw.ind}}{}

\end{document}

      