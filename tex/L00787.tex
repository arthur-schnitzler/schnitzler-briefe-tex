%% latex-korrekturansicht-vorspann.tex
%% Vorspann für die Korrekturansicht.
%% Lädt die gemeinsame Datei latex-vorspann.tex mit gesetztem Schalter.

\newif\ifkorrekturansicht
\korrekturansichttrue

\input{../tex-inputs/latex-vorspann}


               \section[Arthur Schnitzler an Georg Brandes, 27. 3. 1898]{ Arthur Schnitzler an Georg Brandes, 27. 3. 1898}\nopagebreak\mylabel{v}\rehead{ }\normalsize\beginnumbering\briefempfaengerindex{Brandes, Georg@\textsc{Brandes, Georg}!zzzSchnitzler, Arthur@\emph{von Arthur Schnitzler}!1898-03-271@{27. 3. 1898}|(be} \toendnotes[C]{\smallbreak\pagebreak[2]} \Standort{Kopenhagen, Det Kongelige Bibliotek, Georg Brandes Arkiv, box 125.}
\physDesc{Brief, 3 Blätter, 12 Seiten
\newline{}Handschrift: schwarze Tinte, deutsche Kurrent\newline{}Ordnung: mit Bleistift von unbekannter Hand nummeriert: »11.
                                    Schnitzler« sowie das Datum unterhalb der Datierung
                                 wiederholt: »27–3–98«; auf dem zweiten und dritten Blatt ebenfalls mit
                                 Bleistift: »27/3 98« }\buchAbdrucke{\weitereDrucke{1) Georg Brandes, Arthur Schnitzler: \emph{Ein Briefwechsel}. Hg. Kurt Bergel. Bern: \emph{Francke} 1956, S. 67–69.} \weitereDrucke{2) Arthur Schnitzler: \emph{Briefe 1875–1912}. Hg. Therese Nickl und Heinrich Schnitzler. Frankfurt am Main: \emph{S. Fischer} 1981, S. 348–350.} }\toendnotes[C]{\smallbreak}\pstart
           \raggedleft{}{\pb}\textcolor{pink}{Wien}{}\ledrightnote{\textcolor{pink}{Wien}}, 27. 3. 98\pend
           \pstart{}Verehrteſter Herr Brandes,\pend\pstart
           es war wirklich nicht nothwendig uns für etwas zu danken, was uns ſelbſt ſo viel
               Freude gemacht hat wie die Möglichkeit während Ihres \textcolor{pink}{Wien}{}\ledrightnote{\textcolor{pink}{Wien}}er Aufenthalts einige Stunden mit Ihnen zu verbringen; jedenfalls aber
               freut mich Ihre liebe Nachricht aus \textcolor{pink}{Sicilien}{}\ledrightnote{\textcolor{pink}{Sizilien}}, die
               mir von Ihrem Wohlbefinden ſo ange{\pb}nehme Kunde
               gibt. Über Ihre Aufnahme in \textcolor{pink}{Rom}{}\ledrightnote{\textcolor{pink}{Rom}} hatte ich ſchon
               irgendwo geleſen; der ungeſtörte Fortgang Ihrer Reiſe ließ mich auch vermuthen, daſs
               Sie von Hauſe günſtige Mittheilungen erhielten, was mir nun durch Ihren Brief
               erfreulich beſtätigt wird. Wir haben auch aus \textcolor{pink}{Kopenhagen}{}\ledrightnote{\textcolor{pink}{Kopenhagen}} Ihre Bücher geſchickt bekommen; herzlichen Dank dafür. Den
                  \label{K_L00787_1v}\edtext{Band}{\lemma{\textnormal{\emph{Band}}}\Cendnote{\textnormal{1897 erschien von \emph{\textcolor{green}{Die Hauptströmungen
                     der Literatur des neunzehnten Jahrhunderts}} im Verlag \emph{\textcolor{brown}{Barsdorf}} eine »fünfte, gänzlich neu bearbeitete und
                     bedeutend vermehrte Auflage« in 27 Lieferungen.}}}\label{K_L00787_1h} aus den \textcolor{green}{Hauptſtrömungen}{}\ledrightnote{\textcolor{green}{Hauptströmungen der Literatur des neunzehnten Jahrhunderts}} hab ich ſchon gekannt, in der
               früheren {\pb}Ausgabe; dagegen habe ich Ihre \textcolor{green}{Rede über das Nationalgefühl}{}\ledrightnote{→\textcolor{green}{Nationalgefühl}} zum
               erſten Mal geleſen. Ich glaube dſs ſie als ein wahres Muſter ihrer Gattung gelten
               kann, da ſie ſchwungvoll und ſachlich zugleich iſt.\pend
           \pstart
           Die \label{K_L00787_2v}\edtext{Aufnahme}{\lemma{\textnormal{\emph{Aufnahme}}}\Cendnote{\textnormal{\emph{\textcolor{green}{Freiwild}} wurde vom 4. 2. 1898 bis
                  zum 26. 2. 1898 am \textcolor{pink}{Carl-Theater in
                     Wien} gegeben.}}}\label{K_L00787_2h} des »\textcolor{green}{Freiwild}{}\ledrightnote{\textcolor{green}{Freiwild. Schauspiel in 3 Akten}}«,
               nach der Sie ſich erkundigen, war hier am erſten Abend eine ſehr gute; die Kritik war
               im ganzen wenig wohlwollend. Sie wiſſen, daſs ich ſelbſt {\pb}eine geringe Meinung von dem künſtleriſchen Werth
               dieſes \textcolor{green}{Stücks}{}\ledrightnote{→\textcolor{green}{Freiwild. Schauspiel in 3 Akten}} habe; aber davon
               war wenig die Rede. Dagegen \strikeout{flo} iſt bei der
               Beſprechung der angeblichen Tendenz ſo viel Bornirtheit und Verlogenheit aufgeflogen
               – wie Staubwolken, wenn ein galoppirendes Roſs über die Landſtraße jagt. Insbeſondre
               die antiſemitiſchen Blätter leiſteten unglaubliches in Denunziationen. Es iſt
               ſchließlich ſo weit geko{\geminationm}en, daſs die Direktion {\pb}des \textcolor{pink}{Theaters}{}\ledrightnote{→\textcolor{pink}{Carl-Theater}} nach ſieben Vorſtellungen »auf einen Wink von oben«, (über den man
               mir ſelbſt nur unter 4 Augen Aufſchluß geben wollte, was ich nicht annahm) das Stück
               abſetzte. –\pend
           \pstart
           Mein neues \textcolor{green}{Schauspiel}{}\ledrightnote{→\textcolor{green}{Das Vermächtnis. Schauspiel in drei Akten}} ko{\geminationm}t im Herbſt in der \textcolor{pink}{Burg}{}\ledrightnote{\textcolor{pink}{Burgtheater}}
               dran (we{\geminationn} die Hofcensur nichts dawider hat); jetzt habe
               ich ein paar einaktige \textcolor{green}{Sachen}{}\ledrightnote{→\textcolor{green}{Der grüne Kakadu – Paracelsus – Die Gefährtin. Drei Einakter}}
               geſchrieben und möchte bald wieder an was größeres gehen. Bei dem neuen \textcolor{green}{Schauſpiel}{}\ledrightnote{→\textcolor{green}{Das Vermächtnis. Schauspiel in drei Akten}} iſt mir ſtärker als je
               ein Grundmangel {\pb}meines Schaffens zum Bewußtſein
               gekommen. Ich finde nemlich, daſs mir die \label{K_L00787_3v}\edtext{Nebenfiguren meiſtens nicht übel gelingen; hingegen iſt meine
               Hauptperson \strikeout{\textcolor{gray}{meiſtens}} i{\geminationm}er irgend wer, dem was ſehr trauriges
                  paſſirt}{\lemma{\textnormal{\emph{Nebenfiguren … paſſirt}}}\Cendnote{\textnormal{vgl. A. S.: \emph{Tagebuch}, 21. 2. 1898}}}\label{K_L00787_3h} – und nicht viel mehr.
               Sie holt ihre Bedeutung aus ihrem Schickſal, nicht aus ihrem Weſen.\pend
           \pstart
           Die »\textcolor{green}{Luſt}{}\ledrightnote{\textcolor{green}{Lust}}« von \textcolor{blue}{d’Annuncio}{}\ledrightnote{\textcolor{blue}{Gabriele D’Annunzio}}, die Sie auf der Reiſe geleſen haben, war mir auch nicht
               ſympathiſch. Vor allem ſchien mir einiger \textsc{Snobismus}{ }{\pb}drin zu ſtecken; auch Bildungs\textsc{snobismus}. Dagegen wäre möglicherweiſe nichts einzuwenden,
                  we{\geminationn} nicht gewiſſe künſtleriſche Schwächen daraus
               hervorgingen. Ein Dichter hat gewiſs das Recht zu ſagen: Sie ſah aus wie die \textcolor{green}{\textsc{Madonna}}{}\ledrightnote{\textcolor{green}{Sixtinische Madonna}} von \textcolor{blue}{\textsc{Rafael}}{}\ledrightnote{\textcolor{blue}{Raffaello Sanzio da Urbino}} in \textcolor{pink}{\textsc{Dresden}}{}\ledrightnote{\textcolor{pink}{Dresden}} oder er erinnerte mich an ein Portrait von \textcolor{blue}{Rembrandt}{}\ledrightnote{\textcolor{blue}{Rembrandt van Rijn}}; – aber er darf nicht verlangen, daſs ich mir was vorſtellen ſoll,
                  we{\geminationn} er ſchildert: Sie hat Hände wie die {\pb}Dame auf dem Bild eines unbeka{\geminationn}ten Malers das in einer unbekannten Galerie in einer
               ganz kleinen \textcolor{pink}{italieniſchen}{}\ledrightnote{\textcolor{pink}{Italien}}{ }Stadt hängt. Derartiges findet ſich in der »\textcolor{green}{Luſt}{}\ledrightnote{\textcolor{green}{Lust}}« nicht gerade ſelten. – Was ich aber ſonſt von
                  \textcolor{blue}{d’Annuncio}{}\ledrightnote{\textcolor{blue}{Gabriele D’Annunzio}} kenne, hat mich mit Bewunderung
               erfüllt. Ich meine den »\textcolor{green}{Triumph des Todes}{}\ledrightnote{\textcolor{green}{Triumph des Todes}}« und die
                  »\textcolor{green}{Unſchuldige}{}\ledrightnote{\textcolor{green}{Unschuldige}}.« –\pend
           \pstart
           Wie lange bleiben Sie noch in \textcolor{pink}{Italien}{}\ledrightnote{\textcolor{pink}{Italien}}? Werden wir
               bald wieder von {\pb}Ihnen hören? Ich brauche die
               »Wir« nicht näher zu bezeichnen. \textcolor{blue}{Paul Goldmann}{}\ledrightnote{\textcolor{blue}{Paul Goldmann}}
               geht auf etwa ein halbes Jahr nach \textcolor{pink}{China}{}\ledrightnote{\textcolor{pink}{China}} und \textcolor{pink}{Japan}{}\ledrightnote{\textcolor{pink}{Japan}}, im Auftrag ſeines \textcolor{brown}{Blattes}{}\ledrightnote{→\textcolor{brown}{Frankfurter Zeitung}}; er ſchifft ſich am 5. April in \textcolor{pink}{Genua}{}\ledrightnote{\textcolor{pink}{Genua}} ein. Ich will in der Charwoche per Rad vom \textcolor{pink}{Bre{\geminationn}er}{}\ledrightnote{\textcolor{pink}{Brenner}} aus durchs
                  Ampezzothal nach \textcolor{pink}{Venedig}{}\ledrightnote{\textcolor{pink}{Venedig}}.\pend
           \pstart
           Von meiner \textcolor{blue}{Mama}{}\ledrightnote{→\textcolor{blue}{Louise Schnitzler}} und \textcolor{blue}{Beer-Hofmann}{}\ledrightnote{\textcolor{blue}{Richard Beer-Hofmann}} habe ich Ihnen die beſten Grüße zu
               ſagen; {\pb}mögen Sie, verehrteſter Herr Brandes,
               angenehmes denken und angenehmes erleben und uns, wenn Sie ſich auf der Rückreiſe
               wieder in \textcolor{pink}{Wien}{}\ledrightnote{\textcolor{pink}{Wien}} aufhalten (was dringend gewünſcht
               wird) mancherlei davon erzählen.\pend
           \pstart
           Herzlichſt ergeben{\\[\baselineskip]}Ihr \spacefill\mbox{ArthurSchnitzler}\pend
           \leftskip=0em{}\endnumbering\briefempfaengerindex{Brandes, Georg@\textsc{Brandes, Georg}!zzzSchnitzler, Arthur@\emph{von Arthur Schnitzler}!1898-03-271@{27. 3. 1898}|)be}\mylabel{h}  \normalsize

\doendnotes{C}
\bigskip
\vfill

\clearpage

\footnotesize

\lohead{\textsc{register}}

% Definiere theindex-Environment komplett neu ohne reledmac
\makeatletter
\renewenvironment{theindex}{%
  \section*{\indexname}%
  \setlength{\parindent}{0pt}%
  \setlength{\parskip}{0pt plus 0.3pt}%
  \let\item\@idxitem
}{%
  \clearpage
}
\makeatother

\IfFileExists{\jobname-pw.ind}{\input{\jobname-pw.ind}}{}

\end{document}

      