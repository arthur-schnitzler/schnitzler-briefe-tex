%% latex-korrekturansicht-vorspann.tex
%% Vorspann für die Korrekturansicht.
%% Lädt die gemeinsame Datei latex-vorspann.tex mit gesetztem Schalter.

\newif\ifkorrekturansicht
\korrekturansichttrue

\input{../tex-inputs/latex-vorspann}


               \section[Arthur Schnitzler an Franz Blei, 14. 12. 1909]{ Arthur Schnitzler an Franz Blei, 14. 12. 1909}\nopagebreak\mylabel{v}\rehead{ }\normalsize\beginnumbering\briefempfaengerindex{Blei, Franz@\textsc{Blei, Franz}!zzzSchnitzler, Arthur@\emph{von Arthur Schnitzler}!1909-12-142@{14. 12. 1909}|(be} \toendnotes[C]{\smallbreak\pagebreak[2]} \Standort{DLA, A:Schnitzler, HS.NZ85.1.403.}
\physDesc{Brief, 1 Blatt, 1 Seite, maschineller Durchschlag
\newline{}Schreibmaschine}\toendnotes[C]{\smallbreak}\pstart
           \raggedleft{}{\pb}14. 12. 1909.\pend
           \pstart{}Verehrtester Dr. Blei!\pend\pstart
           Ich komme heute auf Ihre freundliche Aufforderung zur Mitarbeiterschaft am \textcolor{brown}{Hyperion}{}\ledrightnote{\textcolor{brown}{Hyperion}} zurück. Teilen Sie mir gütigst recht
                    bald mit, ob Sie prinzipiell geneigt wären das Vorspiel zu meinem neuen \textcolor{green}{Stück}{}\ledrightnote{→\textcolor{green}{Der junge Medardus. Dramatische Historie in einem Vorspiel und fünf Aufzügen}}, eine so ziemlich in
                    sich abgeschlossene Sache im Ganzen nahezu von der Ausdehnung des »\textcolor{green}{Grünen Kakadu}{}\ledrightnote{\textcolor{green}{Der grüne Kakadu. Groteske in einem Akt}}«, abzudrucken, und dafür im
                    Annahmefall 1000 Mark zu zahlen.\pend
           \pstart
           Mit verbindlichem Gruss{\\[\baselineskip]}Ihr ergebener\pend
           \leftskip=0em{}\pstart
           \noindent{}Herrn Dr. Franz Blei, Herausgeber des \textcolor{brown}{Hyperion}{}\ledrightnote{\textcolor{brown}{Hyperion}}, \textcolor{pink}{München}{}\ledrightnote{\textcolor{pink}{München}}.\pend
           \endnumbering\briefempfaengerindex{Blei, Franz@\textsc{Blei, Franz}!zzzSchnitzler, Arthur@\emph{von Arthur Schnitzler}!1909-12-142@{14. 12. 1909}|)be}\mylabel{h}  \normalsize

\doendnotes{C}
\bigskip
\vfill

\clearpage

\footnotesize

\lohead{\textsc{register}}

% Definiere theindex-Environment komplett neu ohne reledmac
\makeatletter
\renewenvironment{theindex}{%
  \section*{\indexname}%
  \setlength{\parindent}{0pt}%
  \setlength{\parskip}{0pt plus 0.3pt}%
  \let\item\@idxitem
}{%
  \clearpage
}
\makeatother

\IfFileExists{\jobname-pw.ind}{\input{\jobname-pw.ind}}{}

\end{document}

      