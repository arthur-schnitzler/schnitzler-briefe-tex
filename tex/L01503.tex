%% latex-korrekturansicht-vorspann.tex
%% Vorspann für die Korrekturansicht.
%% Lädt die gemeinsame Datei latex-vorspann.tex mit gesetztem Schalter.

\newif\ifkorrekturansicht
\korrekturansichttrue

\input{../tex-inputs/latex-vorspann}


               \section[Hermann Bahr: Widmungsexemplar Sanna für Arthur Schnitzler, {[}1.?{]} 3. 1905]{ Hermann Bahr: Widmungsexemplar Sanna für Arthur Schnitzler,
               {[}1.?{]} 3. 1905}\nopagebreak\mylabel{v}\rehead{ }\normalsize\beginnumbering\briefempfaengerindex{Schnitzler, Arthur@\textsc{Schnitzler, Arthur}!zzzBahr, Hermann@\emph{von Hermann Bahr}!1905-03-011@{{[}1.?{]} 3. 1905}|(be} \toendnotes[C]{\smallbreak\pagebreak[2]} \Standort{DLA, G:Schnitzler, Arthur (Sammlung Heinrich Schnitzler).}
\physDesc{Widmung am Vorsatzblatt
\newline{}Handschrift: schwarze Tinte, deutsche Kurrent\newline{}Ordnung: bei der Enteignung des Exemplars 1938 von
                                 unbekannter Hand mit Bleistift ergänzte Informationen:
                                    »Dubl. zu 439.421-B« }\buchAbdrucke{\weitereDrucke{Hermann Bahr, Arthur Schnitzler: \emph{Briefwechsel, Aufzeichnungen, Dokumente (1891–1931)}. Hg. Kurt Ifkovits und Martin Anton Müller. Göttingen: \emph{Wallstein} 2018, S. 344.} }\toendnotes[C]{\smallbreak}\pstart
           \noindent{}{\pb}Herzlichſt\pend
           \pstart
           herzlichſt{\\[\baselineskip]}\spacefill\mbox{Hermann}\pend
           \leftskip=0em{}\pstart
           \noindent{}\label{K_L01503_1v}\edtext{März 1905}{\lemma{\textnormal{\emph{März 1905}}}\Cendnote{\textnormal{am 28. 2. 1905 vom \emph{\textcolor{green}{Börsenblatt für den deutschen Buchhandel}}
                        als Neuerscheinung gemeldet}}}\label{K_L01503_1h}\pend
           {\bigskip}\pstart
           \noindent{}\centering{}{\pb}\textcolor{gray}{\textbf{\textcolor{green}{\textbf{Sanna}}{}\ledrightnote{\textcolor{green}{Sanna. Schauspiel in fünf Aufzügen}}}}\pend
           \pstart
           \noindent{}\centering{}\textcolor{gray}{\textbf{\so{Schauſpiel in fünf Aufzügen}}}\pend
           \pstart
           \noindent{}\centering{}\textcolor{gray}{\textbf{von}}\pend
           \pstart
           \noindent{}\centering{}\textcolor{gray}{\textbf{Hermann Bahr}}\pend
           {\bigskip}\pstart
           \noindent{}\raggedleft{}\textcolor{gray}{\textbf{»\label{K_L01503_2v}\edtext{Endlich gewinnt
                  doch nur unſer}{\lemma{\textnormal{\emph{Endlich … unſer}}}\Cendnote{\textnormal{in einem Brief an \textcolor{blue}{Mathilde Wesendonck},
                     15. 4. 1859}}}\label{K_L01503_2h}}}{\\}\textcolor{gray}{\textbf{Herz, wer am meiſten leidet, und}}{\\}\textcolor{gray}{\textbf{eine Stimme ſagt uns auch, daß}}{\\}\textcolor{gray}{\textbf{er am tiefſten blickt: eben weil er}}{\\}\textcolor{gray}{\textbf{in jedem Falle alle Fälle ſieht, dünkt}}{\\}\textcolor{gray}{\textbf{ihm der kleinſte so ungeheuer.«}}\pend
           \pstart
           \noindent{}\raggedleft{}\textcolor{gray}{\textbf{\textcolor{blue}{Richard Wagner}{}\ledrightnote{\textcolor{blue}{Richard Wagner}}}}\pend
           {\bigskip}\pstart
           \noindent{}\centering{}\textcolor{gray}{\textbf{\textcolor{pink}{Berlin}{}\ledrightnote{\textcolor{pink}{Berlin}}{ }1905}}\pend
           \pstart
           \noindent{}\centering{}\textcolor{gray}{\textbf{\textcolor{brown}{\so{S. Fiſcher, Verlag}}{}\ledrightnote{\textcolor{brown}{S. Fischer Verlag}}}}\pend
           \endnumbering\briefempfaengerindex{Schnitzler, Arthur@\textsc{Schnitzler, Arthur}!zzzBahr, Hermann@\emph{von Hermann Bahr}!1905-03-011@{{[}1.?{]} 3. 1905}|)be}\mylabel{h}  \normalsize

\doendnotes{C}
\bigskip
\vfill

\clearpage

\footnotesize

\lohead{\textsc{register}}

% Definiere theindex-Environment komplett neu ohne reledmac
\makeatletter
\renewenvironment{theindex}{%
  \section*{\indexname}%
  \setlength{\parindent}{0pt}%
  \setlength{\parskip}{0pt plus 0.3pt}%
  \let\item\@idxitem
}{%
  \clearpage
}
\makeatother

\IfFileExists{\jobname-pw.ind}{\input{\jobname-pw.ind}}{}

\end{document}

      