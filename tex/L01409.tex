%% latex-korrekturansicht-vorspann.tex
%% Vorspann für die Korrekturansicht.
%% Lädt die gemeinsame Datei latex-vorspann.tex mit gesetztem Schalter.

\newif\ifkorrekturansicht
\korrekturansichttrue

\input{../tex-inputs/latex-vorspann}


               \section[Hugo von Hofmannsthal an Arthur Schnitzler, {[}23. 6. 1904{]}]{ Hugo von Hofmannsthal an Arthur Schnitzler, {[}23. 6. 1904{]}}\nopagebreak\mylabel{v}\rehead{ }\normalsize\beginnumbering\briefempfaengerindex{Schnitzler, Arthur@\textsc{Schnitzler, Arthur}!zzzHofmannsthal, Hugo von@\emph{von Hugo von Hofmannsthal}!1904-06-231@{{[}23. 6. 1904{]}}|(be} \toendnotes[C]{\smallbreak\pagebreak[2]} \Standort{CUL, Schnitzler, B 43.}
\physDesc{Brief, 1 Blatt, 2 Seiten
\newline{}Handschrift: schwarze Tinte, deutsche Kurrent
\newline{}Schnitzler: mit Bleistift datiert: »23/6 904.« \newline{}Ordnung: 1) mit Bleistift von unbekannter Hand nummeriert: »\strikeout{239}« 2) mit Bleistift von unbekannter Hand nummeriert: »225«}\buchAbdrucke{\weitereDrucke{Hugo von Hofmannsthal, Arthur Schnitzler: \emph{Briefwechsel}. Hg. Therese Nickl und Heinrich Schnitzler. Frankfurt am Main: \emph{S. Fischer} 1964, S. 189.} }\toendnotes[C]{\smallbreak}\pstart{}{\pb}lieber\pend\pstart
           1.) wie gehts Ihnen\pend
           \pstart
           2.) bitte ko{\geminationm}en Sie nächſten Do{\geminationn}erstag, weil Mittwoch das \textcolor{blue}{Kinderfräulein}{}\ledrightnote{→\textcolor{blue}{?? [Kinderfrau bei Hofmannsthal]}} Ausgang hat\pend
           \pstart
           3.) wir nehmen als ſelbſtverſtändlich an, daſs Ihr \textcolor{blue}{Liſl}{}\ledrightnote{\textcolor{blue}{Elisabeth Steinrück}} mitbringt\pend
           \pstart
           4.) \textcolor{blue}{Olga}{}\ledrightnote{\textcolor{blue}{Olga Schnitzler}}{ }ſoll nur ja nicht etwa in der Abſicht, damit {\pb}einen guten Zweck zu erreichen,
               irgendwie unſere Geſpräche über die \textcolor{blue}{Bären}{}\ledrightnote{\textcolor{blue}{Paula Beer-Hofmann}{\newline}\textcolor{blue}{Richard Beer-Hofmann}}
               gegen Frl. \textcolor{blue}{Mütter}{}\ledrightnote{\textcolor{blue}{Franziska Mütter}} erwähnen. Es würde daraus ganz
               ſicher etwas unangenehmes entſtehen.\pend
           \pstart
           Von Herzen{\\[\baselineskip]}\spacefill\mbox{Hugo.}\pend
           \leftskip=0em{}\endnumbering\briefempfaengerindex{Schnitzler, Arthur@\textsc{Schnitzler, Arthur}!zzzHofmannsthal, Hugo von@\emph{von Hugo von Hofmannsthal}!1904-06-231@{{[}23. 6. 1904{]}}|)be}\mylabel{h}  \normalsize

\doendnotes{C}
\bigskip
\vfill

\clearpage

\footnotesize

\lohead{\textsc{register}}

% Definiere theindex-Environment komplett neu ohne reledmac
\makeatletter
\renewenvironment{theindex}{%
  \section*{\indexname}%
  \setlength{\parindent}{0pt}%
  \setlength{\parskip}{0pt plus 0.3pt}%
  \let\item\@idxitem
}{%
  \clearpage
}
\makeatother

\IfFileExists{\jobname-pw.ind}{\input{\jobname-pw.ind}}{}

\end{document}

      