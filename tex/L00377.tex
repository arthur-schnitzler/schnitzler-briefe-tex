%% latex-korrekturansicht-vorspann.tex
%% Vorspann für die Korrekturansicht.
%% Lädt die gemeinsame Datei latex-vorspann.tex mit gesetztem Schalter.

\newif\ifkorrekturansicht
\korrekturansichttrue

\input{../tex-inputs/latex-vorspann}


               \section[Richard Beer-Hofmann an Arthur Schnitzler, 5. 10. 1894]{ Richard Beer-Hofmann an Arthur Schnitzler, 5. 10. 1894}\nopagebreak\mylabel{v}\rehead{ }\normalsize\beginnumbering\briefempfaengerindex{Schnitzler, Arthur@\textsc{Schnitzler, Arthur}!zzzBeer-Hofmann, Richard@\emph{von Richard Beer-Hofmann}!1894-10-052@{5. 10. 1894}|(be} \toendnotes[C]{\smallbreak\pagebreak[2]} \Standort{CUL, Schnitzler, B 8.}
\physDesc{Bildpostkarte
\newline{}Handschrift: Bleistift, lateinische Kurrent\newline{}Versand: 1) Stempel: »\nobreak{}\oindex{Rom@\textbf{Rom}, \emph{Besiedelter Ort (A.BSO)}|pwk}Roma, 5 10.–94\nobreak{}«.  2) Stempel: »\nobreak{}\oindex{IX., Alsergrund@\textbf{IX., Alsergrund}, \emph{Bezirk (A.BZK)}|pwk}Wien 9/3, 8. 10. 94, 8.V, Bestellt\nobreak{}«. \newline{}Ordnung: von Schnitzler mit Bleistift nummeriert: »44«,
                           von zweiter, unbekannter Hand die Nummerierung wiederholt }\pstart{}{\pb}Herrn\pend{}\pstart{}Arthur Schnitzler\pend{}\pstart{}\textcolor{pink}{Austria}{}\ledrightnote{\textcolor{pink}{Österreich}}\pend{}\pstart{}\textcolor{pink}{Wien}{}\ledrightnote{\textcolor{pink}{Wien}}\pend{}\pstart{}\textcolor{pink}{Frankgaſſe 1}{}\ledrightnote{\textcolor{pink}{Frankgasse}}\pend{}{\bigskip}\pstart
           \noindent{}\centering{}\textcolor{gray}{\textbf{{\pb}Ricordo di \textcolor{pink}{Firenze}{}\ledrightnote{\textcolor{pink}{Florenz}} – \textcolor{pink}{Chiesa Santa Croce}{}\ledrightnote{\textcolor{pink}{Santa Croce}}}}\pend
           \pstart
           Lieber!
               Gemäldegallerie 3 ist noch \uline{\textcolor{pink}{Florenz}{}\ledrightnote{\textcolor{pink}{Florenz}}}, ich aber bin in \textcolor{pink}{Rom}{}\ledrightnote{\textcolor{pink}{Rom}}. Bitte schreiben Sie.
               Herzlichst Ihr\pend
           \pstart \spacefill\mbox{Richard}\pend{}\pstart
           \textcolor{pink}{Rom}{}\ledrightnote{\textcolor{pink}{Rom}}{ }Freitag 5/X 94{ }abends\pend
           \endnumbering\briefempfaengerindex{Schnitzler, Arthur@\textsc{Schnitzler, Arthur}!zzzBeer-Hofmann, Richard@\emph{von Richard Beer-Hofmann}!1894-10-052@{5. 10. 1894}|)be}\mylabel{h}  \normalsize

\doendnotes{C}
\bigskip
\vfill

\clearpage

\footnotesize

\lohead{\textsc{register}}

% Definiere theindex-Environment komplett neu ohne reledmac
\makeatletter
\renewenvironment{theindex}{%
  \section*{\indexname}%
  \setlength{\parindent}{0pt}%
  \setlength{\parskip}{0pt plus 0.3pt}%
  \let\item\@idxitem
}{%
  \clearpage
}
\makeatother

\IfFileExists{\jobname-pw.ind}{\input{\jobname-pw.ind}}{}

\end{document}

      