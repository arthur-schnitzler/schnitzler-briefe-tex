%% latex-korrekturansicht-vorspann.tex
%% Vorspann für die Korrekturansicht.
%% Lädt die gemeinsame Datei latex-vorspann.tex mit gesetztem Schalter.

\newif\ifkorrekturansicht
\korrekturansichttrue

\input{../tex-inputs/latex-vorspann}


               \section[Arthur Schnitzler an Richard Beer-Hofmann, 31. 7. 1899]{ Arthur Schnitzler an Richard Beer-Hofmann, 31. 7. 1899}\nopagebreak\mylabel{v}\rehead{ }\normalsize\beginnumbering\briefempfaengerindex{Beer-Hofmann, Richard@\textsc{Beer-Hofmann, Richard}!zzzSchnitzler, Arthur@\emph{von Arthur Schnitzler}!1899-07-311@{31. 7. 1899}|(be} \toendnotes[C]{\smallbreak\pagebreak[2]} \Standort{YCGL, MSS 31.}
\physDesc{Postkarte
\newline{}Handschrift: Bleistift, deutsche Kurrent\newline{}Versand: 1) Stempel: »\nobreak{}\oindex{Spittal an der Drau@\textbf{Spittal an der Drau}, \emph{http://www.geonames.org/ontologyP.PPLA3}|pwk}Spittal an der Drau, 31/7 {[}1899{]}\nobreak{}«.  2) Stempel: »\nobreak{}\oindex{Seeboden@\textbf{Seeboden}, \emph{http://www.geonames.org/ontologyA.ADM3}|pwk}{[}Seebod{]}en, 31. 7. \textcolor{gray}{9}9\nobreak{}«. \newline{}Ordnung: mit Bleistift von unbekannter Hand
                                 datiert: »31. 7.« }\pstart{}{\pb}\textsc{Dr Richard Beer Hofmann}\pend{}\pstart{}\textcolor{pink}{\textsc{Villa} Platzer}{}\ledrightnote{\textcolor{pink}{Villa Platzer}}\pend{}\pstart{}\textcolor{pink}{\textsc{Seeboden}}{}\ledrightnote{\textcolor{pink}{Seeboden}}\pend{}\pstart{}\textcolor{pink}{\textsc{am Millstätter}ſee}{}\ledrightnote{\textcolor{pink}{Millstätter See}}\pend{}{\bigskip}\pstart
           \noindent{}{\pb}lieber; es iſt abſolut unſinnnig, am 1. Tag ſich ſo raſend zu
               ſtrapaziren, und beſonders we{\geminationn} der 2. Tag die
               ſchwierigſte Partie (\textcolor{pink}{Giau}{}\ledrightnote{\textcolor{pink}{Passo di Giau}}) enthält und \strikeout{die} wir doch nur möglichſt arbeitsfriſch betreten
               wollen. Wir werden daher die Tour I in 2 Tage zerlegen, dafür am 1. Tag den \textcolor{pink}{Pragſer See}{}\ledrightnote{\textcolor{pink}{Pragser Wildsee}} mitnehmen. Da{\geminationn} bleibt es auch gewahrt dſs alle Nachmittag frei
               ſind. – Ich ſchreibe Ihnen das gleich hier, um nicht nervös zu ſein. –\pend
           \pstart Herzliche Grüße Ihr \spacefill\mbox{A. S.}\pend{}\pstart
           \textcolor{pink}{Spital}{}\ledrightnote{\textcolor{pink}{Spittal an der Drau}}, 31. 7. 99, eben ſchlägt’s
                     7 Uhr früh.\pend
           \endnumbering\briefempfaengerindex{Beer-Hofmann, Richard@\textsc{Beer-Hofmann, Richard}!zzzSchnitzler, Arthur@\emph{von Arthur Schnitzler}!1899-07-311@{31. 7. 1899}|)be}\mylabel{h}  \normalsize

\doendnotes{C}
\bigskip
\vfill

\clearpage

\footnotesize

\lohead{\textsc{register}}

% Definiere theindex-Environment komplett neu ohne reledmac
\makeatletter
\renewenvironment{theindex}{%
  \section*{\indexname}%
  \setlength{\parindent}{0pt}%
  \setlength{\parskip}{0pt plus 0.3pt}%
  \let\item\@idxitem
}{%
  \clearpage
}
\makeatother

\IfFileExists{\jobname-pw.ind}{\input{\jobname-pw.ind}}{}

\end{document}

      