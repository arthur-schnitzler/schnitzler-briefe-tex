%% latex-korrekturansicht-vorspann.tex
%% Vorspann für die Korrekturansicht.
%% Lädt die gemeinsame Datei latex-vorspann.tex mit gesetztem Schalter.

\newif\ifkorrekturansicht
\korrekturansichttrue

\input{../tex-inputs/latex-vorspann}


               \section[Oscar Blumenthal an Arthur Schnitzler, 12. 5. 1897]{ Oscar Blumenthal an Arthur Schnitzler, 12. 5. 1897}\nopagebreak\mylabel{v}\rehead{ }\normalsize\beginnumbering\briefempfaengerindex{Schnitzler, Arthur@\textsc{Schnitzler, Arthur}!zzzBlumenthal, Oskar@\emph{von Oskar Blumenthal}!1897-05-122@{12. 5. 1897}|(be} \toendnotes[C]{\smallbreak\pagebreak[2]} \Standort{CUL, Schnitzler, B 15.}
\physDesc{Brief, 1 Blatt, 2 Seiten
\newline{}Schreibmaschine
\newline{}Handschrift: 1) Bleistift (\noindent{}die ersten drei Unterstreichungen)\hspace{1em}2) schwarze Tinte (\noindent{}Unterschrift)\hspace{1em}\newline{}Ordnung: mit Bleistift von unbekannter Hand nummeriert:
                                 »9« }\toendnotes[C]{\smallbreak}\pstart
           \noindent{}\centering{}{\pb}\textcolor{gray}{\textbf{\textcolor{brown}{\textsc{Lessing-Theater}}{}\ledrightnote{\textcolor{brown}{Lessing-Theater}}}}\pend
           \pstart
           \noindent{}\centering{}\textcolor{gray}{\textbf{DIRECTOR:}}{ }\textcolor{gray}{\textbf{\textsc{Dr. Oscar Blumenthal}.}}\pend
           \pstart
           \noindent{}\raggedleft{}\textcolor{gray}{\textbf{\textcolor{pink}{BERLIN N.W. (40)}{}\ledrightnote{\textcolor{pink}{Berlin}}, den}}{ }12. Mai 1897.\pend
           \pstart
           \noindent{}\raggedleft{}z. Zeit: \textcolor{pink}{LAUFEN bei ISCHL}{}\ledrightnote{\textcolor{pink}{Lauffen}}\pend
           \pstart\center{}Sehr geehrter Herr Doctor!\pend\pstart
           Es würde mir eine grosse Freude machen, wenn Sie mir für die nächste Spielzeit des
                  »\textcolor{brown}{LESSING-THEATER}{}\ledrightnote{\textcolor{brown}{Lessing-Theater}}S« — die letzte unter meiner Direction — ein neues Bühnenwerk aus Ihrer Feder
               anvertrauen würden. Ich gestatte mir, Sie darauf aufmerksam zu machen, dass gerade in
               der nächsten Saison sich der schauspielerische Besitzstand des »\textcolor{brown}{LESSING-THEATER}{}\ledrightnote{\textcolor{brown}{Lessing-Theater}}S« durch eine Anzahl von sehr vielverheissenden Neu-Engagements beträchtlich
               vermehrt hat. Es werden in den Verband des »\textcolor{brown}{LESSING-THEATER}{}\ledrightnote{\textcolor{brown}{Lessing-Theater}}S« vom ersten September ab neu eintreten: \uline{\textcolor{blue}{ADOLF KLEIN}{}\ledrightnote{\textcolor{blue}{Adolf Klein}}} vom \textcolor{pink}{Königlichen Schauspielhaus}{}\ledrightnote{\textcolor{pink}{Schauspielhaus}}; \textcolor{blue}{WILLY ROHLAND}{}\ledrightnote{\textcolor{blue}{Willy Rohland}}, \textcolor{blue}{ALFRED HALM}{}\ledrightnote{\textcolor{blue}{Alfred Halm}} und \textcolor{blue}{HERRMANN VALENTIN}{}\ledrightnote{\textcolor{blue}{Hermann Vallentin}} vom »\textcolor{pink}{Theater des Westens}{}\ledrightnote{\textcolor{pink}{Theater des Westens}}«; \textcolor{blue}{PAULA CARLSEN}{}\ledrightnote{\textcolor{blue}{Paula Carlsen}} vom »\textcolor{pink}{Neuen Theater}{}\ledrightnote{\textcolor{pink}{Neues Theater}}«; \textcolor{blue}{META ILLING}{}\ledrightnote{\textcolor{blue}{Meta Illing}} vom »\textcolor{brown}{Deutschen Theater}{}\ledrightnote{\textcolor{brown}{Deutsches Theater München}}« in \textcolor{pink}{München}{}\ledrightnote{\textcolor{pink}{München}}; \textcolor{blue}{MATHIEU PFEIL}{}\ledrightnote{\textcolor{blue}{Mathieu Pfeil}} vom »\textcolor{brown}{Irving Place-Theatre}{}\ledrightnote{\textcolor{brown}{Irving Place Theatre}}« in \textcolor{pink}{New-York}{}\ledrightnote{\textcolor{pink}{New York City}}; \textcolor{blue}{ALBERT ULLRICH}{}\ledrightnote{\textcolor{blue}{Albert Ullrich}} vom »\textcolor{brown}{Hoftheater}{}\ledrightnote{\textcolor{brown}{Hoftheater Meiningen}}« in \textcolor{pink}{Meiningen}{}\ledrightnote{\textcolor{pink}{Meiningen}}. \uline{\textcolor{blue}{\label{T_L00676-1v}\edtext{LOUISE
                        DUMONT}{\lemma{\textnormal{\emph{Louise
                        Dumont}}}\Cendnote{\textnormal{Unterstreichung
                        mit Schreibmaschine}}}\label{T_L00676-1h}}{}\ledrightnote{\textcolor{blue}{Louise Dumont}}} wird nach einem neuen Uebereinkommen schon von Mitte October ab dem »\textcolor{brown}{LESSING-THEATER}{}\ledrightnote{\textcolor{brown}{Lessing-Theater}}« zur Verfügung stehen, und \uline{\textcolor{blue}{JENNY GROSS}{}\ledrightnote{\textcolor{blue}{Jenny Groß}}} schon in der ersten {\pb}Septemberwoche
               ihre künstlerische Thätigkeit wieder aufnehmen. Rechnet man hinzu die erprobten
               Kräfte des »\textcolor{brown}{LESSING-THEATER}{}\ledrightnote{\textcolor{brown}{Lessing-Theater}}S« — \textcolor{blue}{META JAEGER}{}\ledrightnote{\textcolor{blue}{Meta Jaeger}} und \textcolor{blue}{MARIE ELSINGER}{}\ledrightnote{\textcolor{blue}{Marie Elsinger}}, \textcolor{blue}{PAULA WIRTH}{}\ledrightnote{\textcolor{blue}{Paula Wirth}}
               und \textcolor{blue}{SOFIE PAGAY}{}\ledrightnote{\textcolor{blue}{Sofie Pagay}}, \textcolor{blue}{FRANZ GUTHERY}{}\ledrightnote{\textcolor{blue}{Franz Guthery}} und \textcolor{blue}{FRANZ SCHOENFELD}{}\ledrightnote{\textcolor{blue}{Franz Julius Schönfeld}}, \textcolor{blue}{EMANUEL STOCKHAUSEN}{}\ledrightnote{\textcolor{blue}{Emanuel Stockhausen}} und \textcolor{blue}{CARL WALDOW}{}\ledrightnote{\textcolor{blue}{Carl Waldow}}, so ergiebt sich ein künstlerisches Ensemble, wie
               es sich nicht eben häufig zusammenfindet. Bietet sich in einer Novität eine
               humoristische Characterrolle von besonderer Kraft, so hat sich mir auch \textcolor{blue}{GEORG ENGELS}{}\ledrightnote{\textcolor{blue}{Georg Engels}} wiederum für ein längeres Gastspiel
               zur Verfügung gestellt, und so bitte ich Sie freundlichst, mich durch zwei Worte
               wissen zu lassen, ob ich auf Ihre mir so werthvolle Mitarbeiterschaft für den
               Spielplan des »\textcolor{brown}{LESSING-THEATER}{}\ledrightnote{\textcolor{brown}{Lessing-Theater}}S« in der nächsten Saison hoffen darf.\pend
           \pstart
           Mit ergebenstem Gruss{\\[\baselineskip]}\spacefill\mbox{{[}hs. Blumenthal:{]} Dr. Osc. Blumenthal.}\pend
           \leftskip=0em{}\endnumbering\briefempfaengerindex{Schnitzler, Arthur@\textsc{Schnitzler, Arthur}!zzzBlumenthal, Oskar@\emph{von Oskar Blumenthal}!1897-05-122@{12. 5. 1897}|)be}\mylabel{h}  \normalsize

\doendnotes{C}
\bigskip
\vfill

\clearpage

\footnotesize

\lohead{\textsc{register}}

% Definiere theindex-Environment komplett neu ohne reledmac
\makeatletter
\renewenvironment{theindex}{%
  \section*{\indexname}%
  \setlength{\parindent}{0pt}%
  \setlength{\parskip}{0pt plus 0.3pt}%
  \let\item\@idxitem
}{%
  \clearpage
}
\makeatother

\IfFileExists{\jobname-pw.ind}{\input{\jobname-pw.ind}}{}

\end{document}

      