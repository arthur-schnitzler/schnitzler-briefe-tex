%% latex-korrekturansicht-vorspann.tex
%% Vorspann für die Korrekturansicht.
%% Lädt die gemeinsame Datei latex-vorspann.tex mit gesetztem Schalter.

\newif\ifkorrekturansicht
\korrekturansichttrue

\input{../tex-inputs/latex-vorspann}


               \section[Arthur Schnitzler an Hermann Bahr, 22. 4. 1913]{ Arthur Schnitzler an Hermann Bahr, 22. 4. 1913}\nopagebreak\mylabel{v}\rehead{ }\normalsize\beginnumbering\briefempfaengerindex{Bahr, Hermann@\textsc{Bahr, Hermann}!zzzSchnitzler, Arthur@\emph{von Arthur Schnitzler}!1913-04-222@{22. 4. 1913}|(be} \toendnotes[C]{\smallbreak\pagebreak[2]} \Standort{TMW, HS AM 23393 Ba.}
\physDesc{Brief, 2 Blätter, 3 Seiten
\newline{}Schreibmaschine
\newline{}Handschrift: schwarze Tinte, lateinische Kurrent (\noindent{}Korrekturen, Unterschrift)\newline{}Ordnung: Lochung }\Standort{DLA, A:Schnitzler, 85.1.294/4.}
\physDesc{Brief, maschineller Durchschlag
\newline{}Schreibmaschine
\newline{}Handschrift: Bleistift, deutsche Kurrent (\noindent{}Streichung »dass der
                                    Aufenth.«)}\buchAbdrucke{\weitereDrucke{1) Arthur Schnitzler: \emph{Briefe 1913–1931}. Hg. Peter Michael Braunwarth, Richard Miklin, Susanne Pertlik und Heinrich Schnitzler. Frankfurt am Main: \emph{S. Fischer} 1984, S. 20–22.} \weitereDrucke{2) \emph{22. 4. 1913.} In: Arthur Schnitzler: \emph{The Letters of Arthur Schnitzler to Hermann Bahr}. Edited, annotated, and with an introduction, by Donald G.
                        Daviau. Chapel Hill: \emph{The University of North Carolina Press} 1978, S. 110–111 (University of North Carolina studies in the Germanic languages
                        and literatures, 89).} \weitereDrucke{3) Hermann Bahr, Arthur Schnitzler: \emph{Briefwechsel, Aufzeichnungen, Dokumente (1891–1931)}. Hg. Kurt Ifkovits und Martin Anton Müller. Göttingen: \emph{Wallstein} 2018, S. 484–485.} }\toendnotes[C]{\smallbreak}\pstart
           \noindent{}{\pb}\textcolor{gray}{\textbf{Dr. Arthur Schnitzler}}\hfill 22. 4. 1913.\pend
           \pstart
           \textcolor{gray}{\textbf{\textcolor{pink}{Wien XVIII. Sternwartestrasse 71}{}\ledrightnote{\textcolor{pink}{Sternwartestraße}}}}\pend
           \pstart{}Lieber Hermann.\pend\pstart
           Ich habe \label{K_L02129_1v}\edtext{nun \textcolor{blue}{Altenberg}{}\ledrightnote{\textcolor{blue}{Peter Altenberg}}, seinen \textcolor{blue}{Bruder}{}\ledrightnote{→\textcolor{blue}{Georg Engländer}} und seinen \textcolor{blue}{Arzt}{}\ledrightnote{→\textcolor{blue}{Karl Richter}} gesprochen}{\lemma{\textnormal{\emph{nun … gesprochen}}}\Cendnote{\textnormal{am 20. 4. 1913}}}\label{K_L02129_1h} und glaube ein klares Bild von der ganzen Sache zu haben. \textcolor{blue}{Altenberg}{}\ledrightnote{\textcolor{blue}{Peter Altenberg}} ist vor zirka 4–5 Monaten wegen eines akuten
               alkoholischen Irreseins nach \textcolor{pink}{Steinhof}{}\ledrightnote{\textcolor{pink}{Otto-Wagner-Spital}} gebracht
               worden. Die schweren Erscheinungen, Verfolgungsideen etc., die, erst in der Anstalt
               selbst auftraten, dürften (was mir ärztlicherseits allerdings nicht gesagt wurde) auf
               die plötzliche vollkommene Abstinenz zurückzuführen gewesen sein (die man jetzt, ich
               weiss nicht recht warum, statt der früher geübten allmählichen Entwöhnung in vielen
               Fällen anwendet). Ich habe \textcolor{blue}{Altenberg}{}\ledrightnote{\textcolor{blue}{Peter Altenberg}} geistig
               frischer gefunden als seit langer Zeit, nur eben sehr erregt, weil er schon gerne auf
               den \textcolor{pink}{Semmering}{}\ledrightnote{\textcolor{pink}{Semmering}} möchte. Freilich besteht die Gefahr,
               besser die Sicherheit, dass er ohne ärztliche Aufsicht sofort wieder zu trinken und
               bald auch wieder alkoholisch {\pb}zu \label{K_L02129_2v}\edtext{exzedieren}{\lemma{\textnormal{\emph{exzedieren}}}\Cendnote{\textnormal{übertreiben}}}\label{K_L02129_2h} anfängt. Diese Gefahr wird aber gerade so
               wie heute in acht Tagen, in vier Wochen und in einem halben Jahr bestehen. Dazu
               kommt, dass seine steigende Erregung wegen der Internierung in \textcolor{pink}{Steinhof}{}\ledrightnote{\textcolor{pink}{Otto-Wagner-Spital}} seinem allgemeinen Zustand kaum förderlich sein dürfte.
               Dies alles habe ich auch \textcolor{blue}{Peter Altenbergs}{}\ledrightnote{\textcolor{blue}{Peter Altenberg}}{ }\textcolor{blue}{Bruder}{}\ledrightnote{→\textcolor{blue}{Georg Engländer}} gesagt, und da \strikeout{auch} der \textcolor{blue}{Chefarzt}{}\ledrightnote{→\textcolor{blue}{Karl Richter}} gegen \textcolor{blue}{P. A.’s}{}\ledrightnote{\textcolor{blue}{Peter Altenberg}}
               Entlassung nichts einzuwenden hat, wenn der \textcolor{blue}{Bruder}{}\ledrightnote{→\textcolor{blue}{Georg Engländer}} die Verantwortung übernimmt, (man muss allerdings
               fragen, wofür?), so dürfte \textcolor{blue}{P. A.}{}\ledrightnote{\textcolor{blue}{Peter Altenberg}} in wenigen Tagen
               die Reise auf den \textcolor{pink}{Semmering}{}\ledrightnote{\textcolor{pink}{Semmering}} antreten können. Der \textcolor{blue}{Bruder}{}\ledrightnote{→\textcolor{blue}{Georg Engländer}} möchte nur, was ich sehr
               vernünftig finde, dass \textcolor{blue}{P. A.}{}\ledrightnote{\textcolor{blue}{Peter Altenberg}} wenigstens anfänglich
               nicht im Hotel, sondern im \textcolor{pink}{Kurhaus}{}\ledrightnote{\textcolor{pink}{Kurhaus Semmering}}, also unter recht
               bescheidener ärztlicher Aufsicht wohne. Für den Fall, dass sich das nicht durchführen
               liesse, wäre auch die Begleitung durch einen Wärter in Erwägung zu ziehen. \textcolor{blue}{P. A.}{}\ledrightnote{\textcolor{blue}{Peter Altenberg}} möchte selbst sehr gern seinen \textcolor{blue}{Wärter}{}\ledrightnote{→\textcolor{blue}{?? [Wärter von Peter Altenberg]}} aus dem Sanatorium für
               ein paar Tage mitnehmen, wenn dem nicht, wie es den Anschein hat, von Seiten der
               Anstalt Schwierigkei{\pb}ten entgegengesetzt würden. Es hat meiner Ansicht nach wirklich keinen Sinn \textcolor{blue}{Peter Altenberg}{}\ledrightnote{\textcolor{blue}{Peter Altenberg}} länger in \textcolor{pink}{Steinhof}{}\ledrightnote{\textcolor{pink}{Otto-Wagner-Spital}} zu halten, wenn auch kaum zu bezweifeln ist, dass nach
               einiger Zeit ihm ein neues Delirium und wahrscheinlich eine neuerliche Internierung,
               die ja dann der Umgebung wegen nicht zu vermeiden ist, bevorstehen dürfte. Von den
               Degenerationserscheinungen, die man nach allerlei Gerüchten hätte befürchten können
               habe ich bei \textcolor{blue}{Altenberg}{}\ledrightnote{\textcolor{blue}{Peter Altenberg}} nicht das Geringste
               bemerkt, und ich glaube, wenn auch vielleicht die \label{T_L02129_1v}\edtext{\uline{plötzliche}}{\lemma{\textnormal{\emph{plötzliche}}}\Cendnote{\textnormal{handschriftliche Unterstreichung}}}\label{T_L02129_1h} Abstinenz zu
               Beginn der Anstaltsbehandlung nicht ausschliesslich von Vorteil \substVorne{}\textsuperscript{war, dass der Aufenthalt im Ganzen}{\allowbreak}\substDazwischen{}für ihn gewesen war\substHinten{}, – die geänderte Lebensweise im weiteren Verlauf und alles was damit
               zusammenhängt hat ihm sicher nur gut getan. Was natürlich kein Anlass ist den
               Aufenthalt ohne Notwendigkeit zu verlängern.\pend
           \pstart
           Herzlichen Gruss{\\[\baselineskip]}Dein{\\[\baselineskip]}\spacefill\mbox{{[}hs.:{]} Arthur}\pend
           \leftskip=0em{}\pstart
           \noindent{}{[}ms.:{]} Herrn Hermann Bahr, \textcolor{pink}{Salzburg}{}\ledrightnote{\textcolor{pink}{Salzburg}}.\pend
           \endnumbering\briefempfaengerindex{Bahr, Hermann@\textsc{Bahr, Hermann}!zzzSchnitzler, Arthur@\emph{von Arthur Schnitzler}!1913-04-222@{22. 4. 1913}|)be}\mylabel{h}  \normalsize

\doendnotes{C}
\bigskip
\vfill

\clearpage

\footnotesize

\lohead{\textsc{register}}

% Definiere theindex-Environment komplett neu ohne reledmac
\makeatletter
\renewenvironment{theindex}{%
  \section*{\indexname}%
  \setlength{\parindent}{0pt}%
  \setlength{\parskip}{0pt plus 0.3pt}%
  \let\item\@idxitem
}{%
  \clearpage
}
\makeatother

\IfFileExists{\jobname-pw.ind}{\input{\jobname-pw.ind}}{}

\end{document}

      