%% latex-korrekturansicht-vorspann.tex
%% Vorspann für die Korrekturansicht.
%% Lädt die gemeinsame Datei latex-vorspann.tex mit gesetztem Schalter.

\newif\ifkorrekturansicht
\korrekturansichttrue

\input{../tex-inputs/latex-vorspann}


               \section[Marie Herzfeld an Arthur Schnitzler, 23. 8. 1899]{ Marie Herzfeld an Arthur Schnitzler, 23. 8. 1899}\nopagebreak\mylabel{v}\rehead{ }\normalsize\beginnumbering\briefempfaengerindex{Schnitzler, Arthur@\textsc{Schnitzler, Arthur}!zzzHerzfeld, Marie@\emph{von Marie Herzfeld}!1899-08-232@{23. 8. 1899}|(be} \toendnotes[C]{\smallbreak\pagebreak[2]} \Standort{DLA, A:Schnitzler, HS.1985.1.03436,2.}
\physDesc{Brief, 1 Blatt, 4 Seiten
\newline{}Handschrift: schwarze Tinte, lateinische Kurrent}\toendnotes[C]{\smallbreak}\pstart
           \raggedleft{}{\pb}\textcolor{pink}{Steg 7 Hallstättersee 4}{}\ledrightnote{\textcolor{pink}{Steeg}}{\\}d.
                     23. Aug. 1899\pend
           \pstart\center{}Geehrter Herr Doktor!\pend\pstart
           Verzeihen Sie, dass ich mich telegraphisch an Sie wende – ich vermute Sie unter den
               obwaltenden Umständen in \label{K_L02591-1v}\edtext{\textcolor{pink}{Ischl}{}\ledrightnote{\textcolor{pink}{Bad Ischl}}}{\lemma{\textnormal{\emph{Ischl}}}\Cendnote{\textnormal{Im
                     August 1899 hielt sich \textcolor{blue}{Schnitzler} tatsächlich in \textcolor{pink}{Bad Ischl} auf.
                     vgl. A. S.: \emph{Tagebuch}, 15. 8. 1899, 19. 8. 1899}}}\label{K_L02591-1h} und
               habe keine Seele dort, die mir sympathisch genug wäre, um sie anzurufen. Ich bin seit
               etwas über 3 Wochen \textcolor{pink}{hier}{}\ledrightnote{→\textcolor{pink}{Steeg}}, bin
               mehreremale gelegen u. war bisher wenig {\pb}wol, dass ich
               mich zu einem Besuch in \textcolor{pink}{Ischl}{}\ledrightnote{\textcolor{pink}{Bad Ischl}} nicht aufraffen
               konnte, ja, eine Ansage bei Freunden daselbst zweimal telegraphisch absagen musste.
               Von unserer verehrten \label{K_L02591-13v}\edtext{\textcolor{blue}{Marie Schey}{}\ledrightnote{\textcolor{blue}{Marie Schey}}}{\lemma{\textnormal{\emph{Marie Schey}}}\Cendnote{\textnormal{\textcolor{blue}{Marie Schey} war eine
                  angeheiratete Großtante von \textcolor{blue}{Schnitzler}. Sie
                  starb am 22. 8. 1899.}}}\label{K_L02591-13h} wusste ich seit Monaten \uline{gar} nichts,
               hatte sie vor ihrer Abreise nicht mehr sehen können, schreibe ihr auch sonst nicht.
               Da ich aber auch etwas von ihr wissen wollte, {\pb}schrieb ich
               an sie vorgestern einen Brief voll von meinen, doch eigentlich nicht \uline{tief}gehenden Leiden u. erhalte als Antwort folgende
                  »\label{K_L02591-44v}\edtext{\begin{otherlanguage}{english}sneering words\end{otherlanguage}}{\lemma{\textnormal{\emph{sneering words}}}\Cendnote{\textnormal{englisch:
                  spöttische Worte}}}\label{K_L02591-44h}« von Herrn \label{K_L02591-5v}\edtext{\textcolor{blue}{Al. Spitzer}{}\ledrightnote{\textcolor{blue}{Alfred Spitzer}}}{\lemma{\textnormal{\emph{Al. Spitzer}}}\Cendnote{\textnormal{Die \emph{\textcolor{green}{Ischler Cur-Liste}}
                  beschreibt ihn als »Kaufmann, \textcolor{pink}{Ungarn}«. (Nr. 33, 8. 8. 1899,
                     S. 8.)}}}\label{K_L02591-5h}: »Spät erkundigen Sie sich um \textcolor{blue}{Tante Marie}{}\ledrightnote{\textcolor{blue}{Marie Schey}}; sie liegt in Agonie.« Stellen Sie sich mein
               Entsetzen vor, da ich von nichts wusste. Mein erster Gedanke war: hinüberfahren. Da
               ich {\pb}jedoch keinesfalls mich einer Beleidigung von Seite
               der Menschen aussetzen möchte, die sich als allein berechtigt ansehen, die Umgebung
               der mir theuern \textcolor{blue}{Frau}{}\ledrightnote{→\textcolor{blue}{Marie Schey}} zu bilden
               u. denen ich seit Jahren ausgewichen bin, so bleibt mir nichts übrig als dies Wort an
               Sie, das, fürchte ich, schon zu spät kommt. Mit vielem Dank für jede
               Auskunft\pend
           \pstart
           grüße Sie aufs beste {\\[\baselineskip]}\spacefill\mbox{Marie Herzfeld}\pend
           \leftskip=0em{}\endnumbering\briefempfaengerindex{Schnitzler, Arthur@\textsc{Schnitzler, Arthur}!zzzHerzfeld, Marie@\emph{von Marie Herzfeld}!1899-08-232@{23. 8. 1899}|)be}\mylabel{h}  \normalsize

\doendnotes{C}
\bigskip
\vfill

\clearpage

\footnotesize

\lohead{\textsc{register}}

% Definiere theindex-Environment komplett neu ohne reledmac
\makeatletter
\renewenvironment{theindex}{%
  \section*{\indexname}%
  \setlength{\parindent}{0pt}%
  \setlength{\parskip}{0pt plus 0.3pt}%
  \let\item\@idxitem
}{%
  \clearpage
}
\makeatother

\IfFileExists{\jobname-pw.ind}{\input{\jobname-pw.ind}}{}

\end{document}

      