%% latex-korrekturansicht-vorspann.tex
%% Vorspann für die Korrekturansicht.
%% Lädt die gemeinsame Datei latex-vorspann.tex mit gesetztem Schalter.

\newif\ifkorrekturansicht
\korrekturansichttrue

\input{../tex-inputs/latex-vorspann}


               \section[Paul Goldmann an Arthur Schnitzler, 3. 11. 1894]{ Paul Goldmann an Arthur Schnitzler, 3. 11. 1894}\nopagebreak\mylabel{v}\rehead{ }\normalsize\beginnumbering\briefempfaengerindex{Schnitzler, Arthur@\textsc{Schnitzler, Arthur}!zzzGoldmann, Paul@\emph{von Paul Goldmann}!1894-11-281@{28. 11. 1894}|(be} \toendnotes[C]{\smallbreak\pagebreak[2]} \Standort{DLA, A:Schnitzler, HS.NZ85.1.3164.}
\physDesc{Brief, 1 Blatt, 1 Seite
\newline{}Handschrift: schwarze Tinte, deutsche Kurrent
\newline{}Schnitzler: mit Bleistift die Jahreszahl »94«
                                 vermerkt }\toendnotes[C]{\smallbreak}\pstart
           \raggedleft{}4. December.\pend
           \pstart\center{}Mein lieber Freund,\pend\pstart
           die »\textcolor{brown}{Frkf. Ztg.}{}\ledrightnote{\textcolor{brown}{Frankfurter Zeitung}}« worin Dein \label{K_L02624-2v}\edtext{\textcolor{green}{Buch}{}\ledrightnote{→\textcolor{green}{Sterben. Novelle}}{ }\textcolor{green}{beſprochen}{}\ledrightnote{→\textcolor{green}{Belletristische Rundschau}}}{\lemma{\textnormal{\emph{Buch beſprochen}}}\Cendnote{\textnormal{\textcolor{blue}{J. Schwarz}: \emph{\textcolor{green}{Belletristische Rundschau}}. In: \emph{\textcolor{green}{Frankfurter Zeitung}}, Nr. 336, 4. 12. 1894,
                     S. 1–3.}}}\label{K_L02624-2h} worden, haſt Du gewiß ſchon geſehen. Der Sicherheit
               halber ſchicke ich ſie Dir zu. Schreib’, bitte, eine \label{K_L02624-1v}\edtext{Zeile an meinen \textcolor{blue}{Onkel}{}\ledrightnote{→\textcolor{blue}{Fedor Mamroth}}}{\lemma{\textnormal{\emph{Zeile an meinen Onkel}}}\Cendnote{\textnormal{siehe Arthur Schnitzler an Fedor Mamroth, 7. 12. 1894}}}\label{K_L02624-1h}, der diesmal beſonders
               brav geweſen iſt.\pend
           \pstart
           Wie gehts Dir? Und wann höre ich wieder etwas von Dir?\pend
           \pstart
           In Treue{\\[\baselineskip]}Dein{\\[\baselineskip]}\spacefill\mbox{Paul Goldmann}\pend
           \leftskip=0em{}\endnumbering\briefempfaengerindex{Schnitzler, Arthur@\textsc{Schnitzler, Arthur}!zzzGoldmann, Paul@\emph{von Paul Goldmann}!1894-11-281@{28. 11. 1894}|)be}\mylabel{h}  \normalsize

\doendnotes{C}
\bigskip
\vfill

\clearpage

\footnotesize

\lohead{\textsc{register}}

% Definiere theindex-Environment komplett neu ohne reledmac
\makeatletter
\renewenvironment{theindex}{%
  \section*{\indexname}%
  \setlength{\parindent}{0pt}%
  \setlength{\parskip}{0pt plus 0.3pt}%
  \let\item\@idxitem
}{%
  \clearpage
}
\makeatother

\IfFileExists{\jobname-pw.ind}{\input{\jobname-pw.ind}}{}

\end{document}

      