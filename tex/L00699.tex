%% latex-korrekturansicht-vorspann.tex
%% Vorspann für die Korrekturansicht.
%% Lädt die gemeinsame Datei latex-vorspann.tex mit gesetztem Schalter.

\newif\ifkorrekturansicht
\korrekturansichttrue

\input{../tex-inputs/latex-vorspann}


               \section[Arthur Schnitzler an Hermann Bahr, 11. 7. 1897]{ Arthur Schnitzler an Hermann Bahr, 11. 7. 1897}\nopagebreak\mylabel{v}\rehead{ }\normalsize\beginnumbering\briefempfaengerindex{Bahr, Hermann@\textsc{Bahr, Hermann}!zzzSchnitzler, Arthur@\emph{von Arthur Schnitzler}!1897-07-111@{11. 7. 1897}|(be} \toendnotes[C]{\smallbreak\pagebreak[2]} \Standort{TMW, HS AM 23331 Ba.}
\physDesc{Brief, 1 Blatt, 3 Seiten
\newline{}Handschrift: schwarze Tinte, deutsche Kurrent\newline{}Ordnung: 1) Lochung 2) mit Bleistift von unbekannter Hand datiert: »11. VII. 94«}\buchAbdrucke{\weitereDrucke{1) \emph{11. 7. 1897.} In: Arthur Schnitzler: \emph{The Letters of Arthur Schnitzler to Hermann Bahr}. Edited, annotated, and with an introduction, by Donald G.
                        Daviau. Chapel Hill: \emph{The University of North Carolina Press} 1978, S. 61 (University of North Carolina studies in the Germanic languages
                        and literatures, 89).} \weitereDrucke{2) Hermann Bahr, Arthur Schnitzler: \emph{Briefwechsel, Aufzeichnungen, Dokumente (1891–1931)}. Hg. Kurt Ifkovits und Martin Anton Müller. Göttingen: \emph{Wallstein} 2018, S. 149–150.} }\pstart{}{\pb}Lieber Hermann, \pend\pstart
           vielen Dank für deine freundlichen Bemühungen. Neues hab ich freilich nicht zu
               bemerken. Es freut mich ſehr, daſs \textcolor{blue}{\textsc{Neumann
                  Hofer}}{}\ledrightnote{\textcolor{blue}{Gilbert Otto Neumann-Hofer}} gern meine nächſten Stücke haben möchte. Aber, ſo wenig {\pb}ich auch Reichtümer verachte, – weder die 2 Prozente mehr
               noch die Möglichkeit ein Einreichungshonorar zu beko{\geminationm}en (was wohl auch an manchem
               andern Theater gelingen mag) können mich beſti{\geminationm}en, die angenehme Freiheit meiner
               Entschließungen durch einen Contract beſchränken zu laſſen. {\pb}Ich beg\damage{rei}fe nur eines nicht: wieſo dieſer Standpunkt nicht von allen andern Menschen
               getheilt wird.\pend
           \pstart
           Wird man dich bald hier ſehen?\pend
           \pstart
           Herzlich grüßt dich{\\[\baselineskip]}dein \spacefill\mbox{ArthSch}\pend
           \leftskip=0em{}\pstart
           \textsc{\textcolor{pink}{Ischl}{}\ledrightnote{\textcolor{pink}{Bad Ischl}}, 11. 7. 97}\pend
           \endnumbering\briefempfaengerindex{Bahr, Hermann@\textsc{Bahr, Hermann}!zzzSchnitzler, Arthur@\emph{von Arthur Schnitzler}!1897-07-111@{11. 7. 1897}|)be}\mylabel{h}  \normalsize

\doendnotes{C}
\bigskip
\vfill

\clearpage

\footnotesize

\lohead{\textsc{register}}

% Definiere theindex-Environment komplett neu ohne reledmac
\makeatletter
\renewenvironment{theindex}{%
  \section*{\indexname}%
  \setlength{\parindent}{0pt}%
  \setlength{\parskip}{0pt plus 0.3pt}%
  \let\item\@idxitem
}{%
  \clearpage
}
\makeatother

\IfFileExists{\jobname-pw.ind}{\input{\jobname-pw.ind}}{}

\end{document}

      