%% latex-korrekturansicht-vorspann.tex
%% Vorspann für die Korrekturansicht.
%% Lädt die gemeinsame Datei latex-vorspann.tex mit gesetztem Schalter.

\newif\ifkorrekturansicht
\korrekturansichttrue

\input{../tex-inputs/latex-vorspann}


               \section[Arthur Schnitzler an Hermann Bahr, 27. 9. 1910]{ Arthur Schnitzler an Hermann Bahr, 27. 9. 1910}\nopagebreak\mylabel{v}\rehead{ }\normalsize\beginnumbering\briefempfaengerindex{Bahr, Hermann@\textsc{Bahr, Hermann}!zzzSchnitzler, Arthur@\emph{von Arthur Schnitzler}!1910-09-271@{27. 9. 1910}|(be} \toendnotes[C]{\smallbreak\pagebreak[2]} \Standort{TMW, HS AM 23391 Ba.}
\physDesc{Brief, 1 Blatt, 3 Seiten
\newline{}Handschrift: 1) schwarze Tinte, deutsche Kurrent\hspace{1em}2) roter Buntstift (\noindent{}Umrahmung des gedruckten Briefkopfs mit der handschriftlichen Adresskorrektur)\hspace{1em}\newline{}Ordnung: Lochung }\buchAbdrucke{\weitereDrucke{1) \emph{27. 9. 1910.} In: Arthur Schnitzler: \emph{The Letters of Arthur Schnitzler to Hermann Bahr}. Edited, annotated, and with an introduction, by Donald G.
                        Daviau. Chapel Hill: \emph{The University of North Carolina Press} 1978, S. 106 (University of North Carolina studies in the Germanic languages
                        and literatures, 89).} \weitereDrucke{2) Hermann Bahr, Arthur Schnitzler: \emph{Briefwechsel, Aufzeichnungen, Dokumente (1891–1931)}. Hg. Kurt Ifkovits und Martin Anton Müller. Göttingen: \emph{Wallstein} 2018, S. 438.} }\toendnotes[C]{\smallbreak}\pstart
           \noindent{}{\pb}\textcolor{gray}{\textbf{Dr. Arthur Schnitzler}}\hfill 27/9. 910\pend
           \pstart
           \textcolor{gray}{\textbf{\textcolor{pink}{Wien XVIII.}{}\ledrightnote{\textcolor{pink}{XVIII., Währing}}}}{ }\substVorne{}\textsuperscript{\textcolor{gray}{\textbf{\textcolor{pink}{Spoettelgasse 7}{}\ledrightnote{\textcolor{pink}{Edmund-Weiß-Gasse}}.}}}{\allowbreak}\substDazwischen{}\textsc{Sternwartestr 71.}\substHinten{}\pend
           \pstart{}mein lieber Hermann,\pend\pstart
           wie die Dinge ſtehn, dürfte der \textsc{\textcolor{green}{\damage{M}edardus}{}\ledrightnote{\textcolor{green}{Der junge Medardus. Dramatische Historie in einem Vorspiel und fünf Aufzügen}}} gerade Anfang November, alſo zur Zeit, da du wieder für einige
               Tage oder Wochen in \textcolor{pink}{Wien}{}\ledrightnote{\textcolor{pink}{Wien}} biſt, aufgeführt werden. Mir
               wird es \damage{ſe}hr lieb ſein, wenn du das Stück auf der Bühne ſiehſt, wo es hingehört, wie
               noch ſelten was von mir hingehört hat. Aber da {\pb}ich bald \uline{fertige} Bühnenmanuſcripte kriege, ſchicke ich \strikeout{dich} dir ſehr gern ein Exemplar nach \textcolor{pink}{London}{}\ledrightnote{\textcolor{pink}{London}}, und wünſche, daſs es dich bei guter Laune u\damage{nd} Geſundheit dort antrifft (nicht um des Stückes willen.)\pend
           \pstart
           Geſtern traf dein neuer \label{K_L01959_1v}\edtext{\textcolor{green}{Roman}{}\ledrightnote{→\textcolor{green}{O Mensch!}}}{\lemma{\textnormal{\emph{Roman}}}\Cendnote{\textnormal{\textcolor{blue}{Hermann Bahr}: \emph{\textcolor{green}{O Mensch. Roman}}. Berlin: \emph{\textcolor{brown}{S. Fischer}}{ }1910.}}}\label{K_L01959_1h} von \textsc{\textcolor{brown}{S. Fischer}{}\ledrightnote{\textcolor{brown}{S. Fischer Verlag}}} bei mir ein. Ich freu
               mich ſehr darauf. Hab mich diesmal \label{K_L01959_2v}\edtext{zurückgehalten, auch nur einen \uline{Blick} in die \textsc{\textcolor{brown}{N. Fr. Pr.}{}\ledrightnote{\textcolor{brown}{Neue Freie Presse}}} zu thun}{\lemma{\textnormal{\emph{zurückgehalten, … thun}}}\Cendnote{\textnormal{Der Vorabdruck von \emph{\textcolor{green}{O Mensch}} erschien vom 31. 5. 1910
                  bis zum 4. 9. 1910 in der \emph{\textcolor{brown}{Neuen Freien
                     Presse}}.}}}\label{K_L01959_2h}.\pend
           \pstart
           {\pb}Dich und deine \textcolor{blue}{Frau}{}\ledrightnote{→\textcolor{blue}{Anna Bahr-Mildenburg}} endlich einmal bei uns zu
               begrüßen, ſoll uns eine ſchöne Winterhoffnung ſein.\pend
           \pstart
           Herzlichſt dein{\\[\baselineskip]}\spacefill\mbox{Arthur}\pend
           \leftskip=0em{}\endnumbering\briefempfaengerindex{Bahr, Hermann@\textsc{Bahr, Hermann}!zzzSchnitzler, Arthur@\emph{von Arthur Schnitzler}!1910-09-271@{27. 9. 1910}|)be}\mylabel{h}  \normalsize

\doendnotes{C}
\bigskip
\vfill

\clearpage

\footnotesize

\lohead{\textsc{register}}

% Definiere theindex-Environment komplett neu ohne reledmac
\makeatletter
\renewenvironment{theindex}{%
  \section*{\indexname}%
  \setlength{\parindent}{0pt}%
  \setlength{\parskip}{0pt plus 0.3pt}%
  \let\item\@idxitem
}{%
  \clearpage
}
\makeatother

\IfFileExists{\jobname-pw.ind}{\input{\jobname-pw.ind}}{}

\end{document}

      