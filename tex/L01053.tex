%% latex-korrekturansicht-vorspann.tex
%% Vorspann für die Korrekturansicht.
%% Lädt die gemeinsame Datei latex-vorspann.tex mit gesetztem Schalter.

\newif\ifkorrekturansicht
\korrekturansichttrue

\input{../tex-inputs/latex-vorspann}


               \section[Richard Beer-Hofmann an Arthur Schnitzler, 13. 7. 1900]{ Richard Beer-Hofmann an Arthur Schnitzler, 13. 7. 1900}\nopagebreak\mylabel{v}\rehead{ }\normalsize\beginnumbering\briefempfaengerindex{Schnitzler, Arthur@\textsc{Schnitzler, Arthur}!zzzBeer-Hofmann, Richard@\emph{von Richard Beer-Hofmann}!1900-07-131@{13. 7. 1900}|(be} \toendnotes[C]{\smallbreak\pagebreak[2]} \Standort{CUL, Schnitzler, B 8.}
\physDesc{Brief, 2 Blätter, 3 Seiten
\newline{}Handschrift: schwarze Tinte, lateinische Kurrent\newline{}Ordnung: mit Bleistift von unbekannter Hand nummeriert: »155« }\buchAbdrucke{\weitereDrucke{Arthur Schnitzler, Richard Beer-Hofmann: \emph{Briefwechsel 1891–1931}. Hg. Konstanze Fliedl. Wien, Zürich: \emph{Europaverlag} 1992, S. 147–148.} }\toendnotes[C]{\smallbreak}\pstart
           \raggedleft{}{\pb}\textcolor{pink}{Alt-Aussee}{}\ledrightnote{\textcolor{pink}{Altaussee}}{ }13/VII. 1900\pend
           \pstart
           Lieber Arthur! Meiner \textcolor{blue}{Frau}{}\ledrightnote{→\textcolor{blue}{Paula Beer-Hofmann}} geht es augenblicklich etwas besser. Seit 8 Tagen ko{\geminationm}t täglich der hiesige \textcolor{blue}{Doktor}{}\ledrightnote{→\textcolor{blue}{Ludwig Engelhardt}}. Ob ein causaler Zusa{\geminationm}enhang zwischen beiden Sätzen besteht? Von \textcolor{blue}{Hugo}{}\ledrightnote{\textcolor{blue}{Hugo von Hofmannsthal}}
               ein Brief aus \textcolor{pink}{Bad-Fusch}{}\ledrightnote{\textcolor{pink}{Bad Fusch}}; er will Ihre Adresse. Von
                  \textcolor{blue}{Goldmann}{}\ledrightnote{\textcolor{blue}{Paul Goldmann}} ein Brief wegen Fußtour. Wir fixiren
               also endgiltig (Schicksalsclauseln inbegriffen) den 15. August in \textcolor{pink}{Innsbruck}{}\ledrightnote{\textcolor{pink}{Innsbruck}}. Für den Zeitungsausschnitt Dank. Zur
               Beruhigung meines \textcolor{blue}{Papa}{}\ledrightnote{→\textcolor{blue}{Hermann Beer}}’s ganz
               gut. \textcolor{blue}{Meyer}{}\ledrightnote{\textcolor{blue}{Oskar Mayer}} war zu Besuch von \textcolor{pink}{Ischl}{}\ledrightnote{\textcolor{pink}{Bad Ischl}} hier, er will die Tour mitmachen. Er hat eine
               Unvorsichtigkeit begangen. »\textcolor{green}{Die Hochzeit der
                  Beatrice}{}\ledrightnote{\textcolor{green}{Der Schleier der Beatrice. Schauspiel in fünf Akten}}« hab ich ihm – wogegen Sie nichts hatten – geborgt. Nun setzt sich
               der Unglückliche in \textcolor{pink}{Marienbad}{}\ledrightnote{\textcolor{pink}{Marienbad}} auf {\pb}eine Bank, liest \introOben{}in\introOben{} dem Buch. Es erscheint: \textcolor{blue}{Minnie B.}{}\ledrightnote{\textcolor{blue}{Hermine von Schaffgotsch}}
               spricht \textcolor{blue}{M.}{}\ledrightnote{\textcolor{blue}{Oskar Mayer}} an erinnert ihn daß er \strikeout{S} sie eigentlich von einem Jour her kennen sollte,
               borgt sich das Buch aus; \textcolor{blue}{M.}{}\ledrightnote{\textcolor{blue}{Oskar Mayer}} wird zweimal zum
               Speisen geladen. Weiter: \textcolor{blue}{Minnie}{}\ledrightnote{\textcolor{blue}{Hermine von Schaffgotsch}} hat aber –
               (verdächtig) das Buch bei ihrer Abreise nach \textcolor{pink}{Levico}{}\ledrightnote{\textcolor{pink}{Levico Terme}} noch nicht zu Ende gelesen, und erhält von \textcolor{blue}{M.}{}\ledrightnote{\textcolor{blue}{Oskar Mayer}} den Auftrag es nach Lesung mir zu schicken was sie noch
               nicht getan hat. \textcolor{blue}{M.}{}\ledrightnote{\textcolor{blue}{Oskar Mayer}} wird nun in meinem Namen
               urgieren damit ich das Buch \strikeout{kom} beko{\geminationm}e. Hoffentlich
               haben bis dahin noch nicht die versa{\geminationm}elten irgendwie
               nennenswerthen Curgäste in \textcolor{pink}{Levico}{}\ledrightnote{\textcolor{pink}{Levico Terme}} bemerkt daß Sie
               Ihre unveröffentlichten Stücke \textcolor{blue}{Minnie}{}\ledrightnote{\textcolor{blue}{Hermine von Schaffgotsch}}
               anvertrauen. O Nachtkastelmotive. Bei alledem ärgert mich \textcolor{blue}{M.}{}\ledrightnote{\textcolor{blue}{Oskar Mayer}}’s Dummheit in dieser Sache. Er argumentirt: Da Sie mit \textcolor{blue}{Minnie}{}\ledrightnote{\textcolor{blue}{Hermine von Schaffgotsch}} gut {\pb}bekannt sind macht es nichts.
               Richtig muß es heißen: Da Sie gut bekannt sind und es ihr nicht geben, so wollen Sie
               eben nicht daß sie es hat. Außerdem ärgert mich: \textcolor{blue}{M.}{}\ledrightnote{\textcolor{blue}{Oskar Mayer}} auf dessen Verstand, Takt, und Geschicklichkeit ich einige Hoffnung
               setzte enttäuscht mich. Ob es denn mir einfiele ein als Manuscript gedrucktes Ding
               jungen \textcolor{blue}{Mädchen}{}\ledrightnote{→\textcolor{blue}{Hermine von Schaffgotsch}} in die Hand zu
               geben die – nach meiner Taxirung – gar kein wirkliches – außer persönliches –
               Interesse daran haben, und nur eine Primeurprotzerei damit anstellen wollen. Im
               Übrigen ist es wahrscheinlich nicht so wichtig.\pend
           \pstart
           Wenn Sie \textcolor{blue}{Minnie}{}\ledrightnote{\textcolor{blue}{Hermine von Schaffgotsch}} einmal – damit die Leut Recht
               behalten – doch heirathen sollten wird dieser Brief mich nicht beliebt machen.\pend
           \pstart
           Ich arbeite. Man überschätzt wie Sie sehen i{\geminationm}er noch die
               Menschen. Herzlich Ihr \spacefill\mbox{R.}\pend
           \endnumbering\briefempfaengerindex{Schnitzler, Arthur@\textsc{Schnitzler, Arthur}!zzzBeer-Hofmann, Richard@\emph{von Richard Beer-Hofmann}!1900-07-131@{13. 7. 1900}|)be}\mylabel{h}  \normalsize

\doendnotes{C}
\bigskip
\vfill

\clearpage

\footnotesize

\lohead{\textsc{register}}

% Definiere theindex-Environment komplett neu ohne reledmac
\makeatletter
\renewenvironment{theindex}{%
  \section*{\indexname}%
  \setlength{\parindent}{0pt}%
  \setlength{\parskip}{0pt plus 0.3pt}%
  \let\item\@idxitem
}{%
  \clearpage
}
\makeatother

\IfFileExists{\jobname-pw.ind}{\input{\jobname-pw.ind}}{}

\end{document}

      