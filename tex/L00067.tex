%% latex-korrekturansicht-vorspann.tex
%% Vorspann für die Korrekturansicht.
%% Lädt die gemeinsame Datei latex-vorspann.tex mit gesetztem Schalter.

\newif\ifkorrekturansicht
\korrekturansichttrue

\input{../tex-inputs/latex-vorspann}


               \section[Hugo von Hofmannsthal an Arthur Schnitzler, {[}3.? 2. 1892{]}]{ Hugo von Hofmannsthal an Arthur Schnitzler, {[}3.? 2. 1892{]}}\nopagebreak\mylabel{v}\rehead{ }\normalsize\beginnumbering\briefempfaengerindex{Schnitzler, Arthur@\textsc{Schnitzler, Arthur}!zzzHofmannsthal, Hugo von@\emph{von Hugo von Hofmannsthal}!1892-02-031@{{[}3.? 2. 1892{]}}|(be} \toendnotes[C]{\smallbreak\pagebreak[2]} \Standort{CUL, Schnitzler, B 43.}
\physDesc{Brief, 1 Blatt, 1 Seite
\newline{}Handschrift: schwarze Tinte, deutsche Kurrent
\newline{}Schnitzler: mit Bleistift nummeriert: »14« }\buchAbdrucke{\weitereDrucke{1) Hugo von Hofmannsthal, Arthur Schnitzler: \emph{Briefwechsel}. Hg. Therese Nickl und Heinrich Schnitzler. Frankfurt am Main: \emph{S. Fischer} 1964, S. 15.} \weitereDrucke{2) Hermann Bahr, Arthur Schnitzler: \emph{Briefwechsel, Aufzeichnungen, Dokumente (1891–1931)}. Hg. Kurt Ifkovits und Martin Anton Müller. Göttingen: \emph{Wallstein} 2018, S. 21.} }\toendnotes[C]{\smallbreak}\pstart{}{\pb}Lieber Freund.\pend\pstart
           Ich bitte um die \label{K_L00067_1v}\edtext{geſtern}{\lemma{\textnormal{\emph{geſtern}}}\Cendnote{\textnormal{vgl. A. S.: \emph{Tagebuch}, 31. 1. 1892. Gegen die
                  Datierung spricht, dass am 2. 2. 1892 noch ein Treffen stattfindet, das hier nicht thematisiert
                  wird.}}}\label{K_L00067_1h} vergeſſenen \label{K_L00067_2v}\edtext{\textcolor{green}{\textsc{Aveugles}}{}\ledrightnote{\textcolor{green}{Die Blinden}}}{\lemma{\textnormal{\emph{Aveugles}}}\Cendnote{\textnormal{In der Folge übersetzte \textcolor{blue}{Hofmannsthal} ausschließlich diesen Einakter von \textcolor{blue}{Maeterlinck} (vgl. Brief an \textcolor{blue}{Marie Herzfeld}, 9. 3. 1892, in:
                        \textcolor{blue}{Hugo von Hofmannsthal}: \emph{Briefe an Marie Herzfeld}. Hg. Horst Weber. Heidelberg:
                        \emph{Lothar Stiehm}{ }1967, S. 23).}}}\label{K_L00067_2h}{ }\textcolor{green}{\textsc{Bérénice}}{}\ledrightnote{\textcolor{green}{Der Garten der Bérenice}} u. \textcolor{green}{\textsc{Sept Princesses}}{}\ledrightnote{\textcolor{green}{Die sieben Prinzessinnen}}.\pend
           \pstart
           Es bleibt bei Sonntag?\pend
           \pstart \spacefill\mbox{Loris.}\pend{}\pstart
           \noindent{}Die \label{K_L00067_3v}\edtext{\textcolor{green}{Überwindung}{}\ledrightnote{\textcolor{green}{Die Überwindung des Naturalismus}}}{\lemma{\textnormal{\emph{Überwindung}}}\Cendnote{\textnormal{Wohl wegen des Artikels \emph{\textcolor{green}{Maurice Maeterlinck}}. In: \textcolor{blue}{Hermann Bahr}: \emph{\textcolor{green}{Die
                        Überwindung des Naturalismus}}. Dresden, Leipzig: \emph{\textcolor{brown}{E. Pierson}}{ }1891, S. 189–198 (Als zweite Reihe von »Zur Kritik der
                        Moderne«). Erstdruck: \emph{\textcolor{green}{Magazin für
                           Litteratur}}, Jg. 60, Nr. 2, 10. 1. 1891,
                     S. 25–27.}}}\label{K_L00067_3h} habe ich zuhauſe\pend
           \endnumbering\briefempfaengerindex{Schnitzler, Arthur@\textsc{Schnitzler, Arthur}!zzzHofmannsthal, Hugo von@\emph{von Hugo von Hofmannsthal}!1892-02-031@{{[}3.? 2. 1892{]}}|)be}\mylabel{h}  \normalsize

\doendnotes{C}
\bigskip
\vfill

\clearpage

\footnotesize

\lohead{\textsc{register}}

% Definiere theindex-Environment komplett neu ohne reledmac
\makeatletter
\renewenvironment{theindex}{%
  \section*{\indexname}%
  \setlength{\parindent}{0pt}%
  \setlength{\parskip}{0pt plus 0.3pt}%
  \let\item\@idxitem
}{%
  \clearpage
}
\makeatother

\IfFileExists{\jobname-pw.ind}{\input{\jobname-pw.ind}}{}

\end{document}

      