%% latex-korrekturansicht-vorspann.tex
%% Vorspann für die Korrekturansicht.
%% Lädt die gemeinsame Datei latex-vorspann.tex mit gesetztem Schalter.

\newif\ifkorrekturansicht
\korrekturansichttrue

\input{../tex-inputs/latex-vorspann}


               \section[Arthur Schnitzler an Georg Brandes, 15. 6. 1899]{ Arthur Schnitzler an Georg Brandes, 15. 6. 1899}\nopagebreak\mylabel{v}\rehead{ }\normalsize\beginnumbering\briefempfaengerindex{Brandes, Georg@\textsc{Brandes, Georg}!zzzSchnitzler, Arthur@\emph{von Arthur Schnitzler}!1899-06-151@{15. 6. 1899}|(be} \toendnotes[C]{\smallbreak\pagebreak[2]} \Standort{Kopenhagen, Det Kongelige Bibliotek, Georg Brandes Arkiv, box 125.}
\physDesc{Briefkarte
\newline{}Handschrift: schwarze Tinte, deutsche Kurrent\newline{}Ordnung: mit Bleistift von unbekannter Hand nummeriert:
                                    »18.« und datiert: »15/6 99« }\buchAbdrucke{\weitereDrucke{Georg Brandes, Arthur Schnitzler: \emph{Ein Briefwechsel}. Hg. Kurt Bergel. Bern: \emph{Francke} 1956, S. 78–79.} }\toendnotes[C]{\smallbreak}\pstart
           \noindent{}{\pb}Verehrter Herr Brandes, ich denke, die Adreſſe \textcolor{blue}{\textsc{Antoine}}{}\ledrightnote{\textcolor{blue}{André Antoine}}\textsc{, Direktor} des \textcolor{pink}{\textsc{theatre Antoine}}{}\ledrightnote{\textcolor{pink}{Théâtre Antoine-Simone Berriau}} in \textcolor{pink}{\textsc{Paris}}{}\ledrightnote{\textcolor{pink}{Paris}} genügt; ich weiſs wenigſtens keine andere. Noch einmal wiederhole ich, daſs ich
               Sie um nichts andres bitte, als \textcolor{blue}{\textsc{Antoine}}{}\ledrightnote{\textcolor{blue}{André Antoine}}{ }\uline{zum \introOben{}baldigen\introOben{} Leſen des \textsc{Manuscriptes}} aufzufordern; Ihr Name iſt in \textcolor{pink}{Paris}{}\ledrightnote{\textcolor{pink}{Berlin}}{ }ſo berühmt wie anderswo (muß ich Ihnen das wirklich
               ſagen?) mich ke{\geminationn}t dort kein Menſch. Ich ſelbſt habe mich
               um eine Überſetzung des »\textcolor{green}{Kakadu}{}\ledrightnote{\textcolor{green}{Der grüne Kakadu. Groteske in einem Akt}}« nicht bemüht; zwei
               Herren, einer, \textcolor{blue}{\textsc{Soutif}}{}\ledrightnote{\textcolor{blue}{Émile Soutif}} in \textcolor{pink}{Dresden}{}\ledrightnote{\textcolor{pink}{Dresden}}, ein zweiter \textcolor{blue}{\textsc{Bech}}{}\ledrightnote{\textcolor{blue}{Bech}}, in \textcolor{pink}{Paris}{}\ledrightnote{\textcolor{pink}{Paris}}{ }{\pb}haben ſich an mich um Erlaubnis gewandt; und we{\geminationn} es ſich machen ließe, wäre mir eine \textcolor{pink}{Pariſ}{}\ledrightnote{\textcolor{pink}{Paris}}er Aufführung natürlich ſehr erwünſcht. –\pend
           \pstart
           In den letzten Tagen habe ich wieder zu arbeiten begonnen; eine kleine \textcolor{green}{Novelle}{}\ledrightnote{→\textcolor{green}{Die Nächste}}, die ich gerade zu \label{K_L00925_1v}\edtext{\uline{jener} Zeit}{\lemma{\textnormal{\emph{jener Zeit}}}\Cendnote{\textnormal{Gemeint ist die postum veröffentlichte Novelle \emph{\textcolor{green}{Die Nächste}}. An der Novelle arbeitete er am 15. 3. 1899 – drei Tage
                  vor dem Tod \textcolor{blue}{Marie Reinhards}, danach hält das \emph{\textcolor{green}{Tagebuch}} am 12. 6. 1899 die Weiterarbeit fest. 
                  Er beendete sie »vorläufig« am 6. 7. 1899.}}}\label{K_L00925_1h}{ }\substVorne{}\textsuperscript{begonn}{\allowbreak}\substDazwischen{}angefa\substHinten{}ngen hatte, und in der mir heute alle möglichen Ahnungen zu zittern
               ſcheinen.\pend
           \pstart
           Ich freue mich, daſs Sie endlich außer Bette ſind; ich hoffe und wünſche \introOben{}Ihnen\introOben{} für weiterhin alles gute und ſchöne.\pend
           \pstart Ihr \spacefill\mbox{Arthur Schnitzler}\pend{}\pstart
           15. 6. 99.\pend
           \endnumbering\briefempfaengerindex{Brandes, Georg@\textsc{Brandes, Georg}!zzzSchnitzler, Arthur@\emph{von Arthur Schnitzler}!1899-06-151@{15. 6. 1899}|)be}\mylabel{h}  \normalsize

\doendnotes{C}
\bigskip
\vfill

\clearpage

\footnotesize

\lohead{\textsc{register}}

% Definiere theindex-Environment komplett neu ohne reledmac
\makeatletter
\renewenvironment{theindex}{%
  \section*{\indexname}%
  \setlength{\parindent}{0pt}%
  \setlength{\parskip}{0pt plus 0.3pt}%
  \let\item\@idxitem
}{%
  \clearpage
}
\makeatother

\IfFileExists{\jobname-pw.ind}{\input{\jobname-pw.ind}}{}

\end{document}

      