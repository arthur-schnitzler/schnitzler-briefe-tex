%% latex-korrekturansicht-vorspann.tex
%% Vorspann für die Korrekturansicht.
%% Lädt die gemeinsame Datei latex-vorspann.tex mit gesetztem Schalter.

\newif\ifkorrekturansicht
\korrekturansichttrue

\input{../tex-inputs/latex-vorspann}


               \section[Hugo von Hofmannsthal an Arthur Schnitzler, {[}9. 8. 1897{]}]{ Hugo von Hofmannsthal an Arthur Schnitzler, {[}9. 8. 1897{]}}\nopagebreak\mylabel{v}\rehead{ }\normalsize\beginnumbering\briefempfaengerindex{Schnitzler, Arthur@\textsc{Schnitzler, Arthur}!zzzHofmannsthal, Hugo von@\emph{von Hugo von Hofmannsthal}!1897-08-091@{{[}9. 8. 1897{]}}|(be} \toendnotes[C]{\smallbreak\pagebreak[2]} \Standort{CUL, Schnitzler, B 43.}
\physDesc{Telegramm
\newline{}maschinell\newline{}Ordnung: von unbekannter Hand nummeriert: »103« }\buchAbdrucke{\weitereDrucke{Hugo von Hofmannsthal, Arthur Schnitzler: \emph{Briefwechsel}. Hg. Therese Nickl und Heinrich Schnitzler. Frankfurt am Main: \emph{S. Fischer} 1964, S. 95.} }\toendnotes[C]{\smallbreak}\pstart
           \noindent{}{\pb}\textcolor{pink}{win}{}\ledrightnote{\textcolor{pink}{Wien}} fr \textcolor{pink}{salzburg}{}\ledrightnote{\textcolor{pink}{Salzburg}}
                  1†1376{ }28{ }11 30m =\pend
           \pstart
           bitte instaendig \label{T_L00715_1v}\edtext{\textcolor{blue}{andrian}{}\ledrightnote{\textcolor{blue}{Leopold von Andrian-Werburg}}}{\lemma{\textnormal{\emph{andrian}}}\Cendnote{\textnormal{korrigiert aus:
                     »andrien«}}}\label{T_L00715_1h} unbedingt heute 9 uhr abend{ }\textcolor{pink}{habsburgergasse 5}{}\ledrightnote{\textcolor{pink}{Habsburgergasse}} besuchen und ihm zu helfen suchen
               sonst mueste ich \label{K_L00715_1v}\edtext{nach \textcolor{pink}{wien}{}\ledrightnote{\textcolor{pink}{Wien}}}{\lemma{\textnormal{\emph{nach wien}}}\Cendnote{\textnormal{Ein Brief \textcolor{blue}{Hofmannsthal}s an seine \textcolor{blue}{Eltern}
                  vom selben Tag erwähnt das Telegramm und erlaubt die Datierung.}}}\label{K_L00715_1h}\spacefill\mbox{= hugo =}\pend
           \endnumbering\briefempfaengerindex{Schnitzler, Arthur@\textsc{Schnitzler, Arthur}!zzzHofmannsthal, Hugo von@\emph{von Hugo von Hofmannsthal}!1897-08-091@{{[}9. 8. 1897{]}}|)be}\mylabel{h}  \normalsize

\doendnotes{C}
\bigskip
\vfill

\clearpage

\footnotesize

\lohead{\textsc{register}}

% Definiere theindex-Environment komplett neu ohne reledmac
\makeatletter
\renewenvironment{theindex}{%
  \section*{\indexname}%
  \setlength{\parindent}{0pt}%
  \setlength{\parskip}{0pt plus 0.3pt}%
  \let\item\@idxitem
}{%
  \clearpage
}
\makeatother

\IfFileExists{\jobname-pw.ind}{\input{\jobname-pw.ind}}{}

\end{document}

      