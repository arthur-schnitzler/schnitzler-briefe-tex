%% latex-korrekturansicht-vorspann.tex
%% Vorspann für die Korrekturansicht.
%% Lädt die gemeinsame Datei latex-vorspann.tex mit gesetztem Schalter.

\newif\ifkorrekturansicht
\korrekturansichttrue

\input{../tex-inputs/latex-vorspann}


               \section[Arthur Schnitzler an Richard Beer-Hofmann, 14. 12. 1892]{ Arthur Schnitzler an Richard Beer-Hofmann, 14. 12. 1892}\nopagebreak\mylabel{v}\rehead{ }\normalsize\beginnumbering\briefempfaengerindex{Beer-Hofmann, Richard@\textsc{Beer-Hofmann, Richard}!zzzSchnitzler, Arthur@\emph{von Arthur Schnitzler}!1892-12-141@{14. 12. 1892}|(be} \toendnotes[C]{\smallbreak\pagebreak[2]} \Standort{YCGL, MSS 31.}
\physDesc{Kartenbrief
\newline{}Handschrift: Bleistift, deutsche Kurrent\newline{}Versand: 1) Stempel: »\nobreak{}Wien 9/3, 14 12 92, 2–3\nobreak{}«.  2) Stempel: »\nobreak{}Wien 1/1, 14/12. 92, 5–6½ N, Bestellt\nobreak{}«. }\buchAbdrucke{\weitereDrucke{Arthur Schnitzler, Richard Beer-Hofmann: \emph{Briefwechsel 1891–1931}. Hg. Konstanze Fliedl. Wien, Zürich: \emph{Europaverlag} 1992, S. 40.} }\toendnotes[C]{\smallbreak}\pstart{}{\pb}\textsc{Hrn Dr. Rich Beer Hofmann}\pend{}\pstart{}\textsc{\textcolor{pink}{Wien}{}\ledrightnote{\textcolor{pink}{Wien}}}\pend{}\pstart{}\textsc{\textcolor{pink}{I
                  Wollzeile 15}{}\ledrightnote{\textcolor{pink}{Wollzeile}}.}\pend{}{\bigskip}\pstart
           \noindent{}{\pb}Lieber Richard! War geſtern bei \textcolor{blue}{Singers}{}\ledrightnote{\textcolor{blue}{Marie Singer}{\newline}\textcolor{blue}{Alexander Singer}}, dort \substVorne{}\textsuperscript{\textcolor{gray}{bed}}\substDazwischen{}Frau\substHinten{}{ }\textcolor{blue}{\textsc{Flegm}.}{}\ledrightnote{\textcolor{blue}{Bertha Flegmann}} – Bitte ſehr, ko{\geminationm}en Sie Freitag mit mir zu ihr? Ja?\pend
           \pstart
           Die \textcolor{green}{Anatols}{}\ledrightnote{\textcolor{green}{Anatol}}{ }ſollen nicht in \textcolor{pink}{\textsc{Rdlfsh}}{}\ledrightnote{\textcolor{pink}{Volkstheater in Rudolphsheim}}, ſondern event. privat aufgeführt werden.\pend
           \pstart
           Wollen Sie mich Freitag um 6, ½ 7 abholen? Es
               wäre mir angenehm, wenn wir beide hingingen. –\pend
           \pstart
           Geſtern \textcolor{green}{2. Akt}{}\ledrightnote{→\textcolor{green}{Familie}} vollendet. –\pend
           \pstart Herzlich Ihr \spacefill\mbox{Arthur}\pend{}\pstart
           \noindent{}Heute will ich zur \textcolor{green}{Jüdin von Toledo}{}\ledrightnote{\textcolor{green}{Die Jüdin von Toledo}} gehn.\pend
           \endnumbering\briefempfaengerindex{Beer-Hofmann, Richard@\textsc{Beer-Hofmann, Richard}!zzzSchnitzler, Arthur@\emph{von Arthur Schnitzler}!1892-12-141@{14. 12. 1892}|)be}\mylabel{h}  \normalsize

\doendnotes{C}
\bigskip
\vfill

\clearpage

\footnotesize

\lohead{\textsc{register}}

% Definiere theindex-Environment komplett neu ohne reledmac
\makeatletter
\renewenvironment{theindex}{%
  \section*{\indexname}%
  \setlength{\parindent}{0pt}%
  \setlength{\parskip}{0pt plus 0.3pt}%
  \let\item\@idxitem
}{%
  \clearpage
}
\makeatother

\IfFileExists{\jobname-pw.ind}{\input{\jobname-pw.ind}}{}

\end{document}

      