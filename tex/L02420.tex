%% latex-korrekturansicht-vorspann.tex
%% Vorspann für die Korrekturansicht.
%% Lädt die gemeinsame Datei latex-vorspann.tex mit gesetztem Schalter.

\newif\ifkorrekturansicht
\korrekturansichttrue

\input{../tex-inputs/latex-vorspann}


               \section[Arthur Schnitzler an Thomas Mann, 6. 11. 1924]{ Arthur Schnitzler an Thomas Mann, 6. 11. 1924}\nopagebreak\mylabel{v}\rehead{ }\normalsize\beginnumbering\briefempfaengerindex{Mann, Thomas@\textsc{Mann, Thomas}!zzzSchnitzler, Arthur@\emph{von Arthur Schnitzler}!1924-11-061@{6. 11. 1924}|(be} \toendnotes[C]{\smallbreak\pagebreak[2]} \Standort{DLA, A:Schnitzler, 85.1.1371,2.}
\physDesc{Brief, 1 Blatt, 1 Seite, maschineller Durchschlag
\newline{}Schreibmaschine
\newline{}Handschrift: roter Buntstift, deutsche Kurrent (\noindent{}Beschriftung: »K{[}opie{]}«, dies durchgestrichen und durch einen Haken als erledigt
                                 markiert, Unterstreichungen)}\buchAbdrucke{\weitereDrucke{1) Hertha Krotkoff: \emph{Arthur Schnitzler – Thomas Mann: Briefe.} In: \emph{Modern Austrian Literature}, Jg. 7 (1974) Nr. 1/2, S. 22–23.} \weitereDrucke{2) Arthur Schnitzler: \emph{Briefe 1913–1931}. Hg. Peter Michael Braunwarth, Richard Miklin, Susanne Pertlik und Heinrich Schnitzler. Frankfurt am Main: \emph{S. Fischer} 1984, S. 372.} \weitereDrucke{3) Hans-Ulrich Lindken: \emph{Arthur Schnitzler. Aspekte und Akzente. Materialien zu Leben
                        und Werk}. Frankfurt am Main, Bern, Göttingen: \emph{Peter Lang} 1984, S. 197–198 (Europäische Hochschulschriften, Reihe 1, Deutsche Sprache und
                        Literatur, 754).} }\pstart
           \raggedleft{}{\pb}6. 11. 1924.\pend
           \pstart{}Lieber und verehrter Herr Thomas Mann.\pend\pstart
           Die guten Worte, die Sie mir über die »\textcolor{green}{Komödie der
                  Verführung}{}\ledrightnote{\textcolor{green}{Komödie der Verführung. In drei Akten}}« sagen, erfreuen mich herzlich. Hier bewährt sich das Stück weiter
               gut im Repertoir; Ihre Befürchtung, dass man es in \textcolor{pink}{München}{}\ledrightnote{\textcolor{pink}{München}} nicht gut genug darstellen würde, ist vorläufig
               unbegründet, da man dort kaum daran denkt es aufzuführen. Bisher hat sich in \textcolor{pink}{Deutschland}{}\ledrightnote{\textcolor{pink}{Deutschland}} nur \textcolor{pink}{Wiesbaden}{}\ledrightnote{\textcolor{pink}{Wiesbaden}} daran gewagt mit sehr gutem Erfolg, \textcolor{pink}{Köln}{}\ledrightnote{\textcolor{pink}{Köln}} folgt bald, \textcolor{pink}{Hannover}{}\ledrightnote{\textcolor{pink}{Hannover}} und \textcolor{pink}{Danzig}{}\ledrightnote{\textcolor{pink}{Danzig}} glaube ich haben es angenommen. Ueber die
               Kritik wollen wir lieber nicht reden. Das frühere Totschlage-Wort von der »grossen
               Zeit« ist nun durch das neue von der »versunkenen Welt« ersetzt worden. Es ist, als
               wäre die Weltgeschichte überhaupt nur dazu da, um den Rezensenten neue falsche
               Massstäbe an die Hand zu geben.\pend
           \pstart
           Der Zusendung des »\textcolor{green}{Zauberbergs}{}\ledrightnote{\textcolor{green}{Der Zauberberg. Roman}}« sehe ich mit
               freudiger Ungeduld entgegen. Ich hoffe Sie haben in \textcolor{pink}{Sestri-Levante}{}\ledrightnote{\textcolor{pink}{Sestri Levante}}{ }schöne Tage, – helle, arbeitsfreudige,
               lebensfrohe.\pend
           \pstart
           Seien Sie sehr herzlich gegrüsst{\\[\baselineskip]}von Ihrem aufrichtig ergebenen\pend
           \leftskip=0em{}\pstart
           \noindent{}Herrn Thomas Mann{\\}\textcolor{pink}{Sestri-Levante}{}\ledrightnote{\textcolor{pink}{Sestri Levante}},{\\}\textcolor{pink}{Italien}{}\ledrightnote{\textcolor{pink}{Italien}}.\pend
           \endnumbering\briefempfaengerindex{Mann, Thomas@\textsc{Mann, Thomas}!zzzSchnitzler, Arthur@\emph{von Arthur Schnitzler}!1924-11-061@{6. 11. 1924}|)be}\mylabel{h}  \normalsize

\doendnotes{C}
\bigskip
\vfill

\clearpage

\footnotesize

\lohead{\textsc{register}}

% Definiere theindex-Environment komplett neu ohne reledmac
\makeatletter
\renewenvironment{theindex}{%
  \section*{\indexname}%
  \setlength{\parindent}{0pt}%
  \setlength{\parskip}{0pt plus 0.3pt}%
  \let\item\@idxitem
}{%
  \clearpage
}
\makeatother

\IfFileExists{\jobname-pw.ind}{\input{\jobname-pw.ind}}{}

\end{document}

      