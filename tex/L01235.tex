%% latex-korrekturansicht-vorspann.tex
%% Vorspann für die Korrekturansicht.
%% Lädt die gemeinsame Datei latex-vorspann.tex mit gesetztem Schalter.

\newif\ifkorrekturansicht
\korrekturansichttrue

\input{../tex-inputs/latex-vorspann}


               \section[Hugo von Hofmannsthal an Arthur Schnitzler, {[}30. 9. 1902{]}]{ Hugo von Hofmannsthal an Arthur Schnitzler, {[}30. 9. 1902{]}}\nopagebreak\mylabel{v}\rehead{ }\normalsize\beginnumbering\briefempfaengerindex{Schnitzler, Arthur@\textsc{Schnitzler, Arthur}!zzzHofmannsthal, Hugo von@\emph{von Hugo von Hofmannsthal}!1902-09-301@{{[}30. 9. 1902{]}}|(be} \toendnotes[C]{\smallbreak\pagebreak[2]} \Standort{CUL, Schnitzler, B 43b/1.}
\physDesc{Brief, 1 Blatt, 1 Seite
\newline{}Handschrift: schwarze Tinte, deutsche Kurrent
\newline{}Schnitzler: mit Bleistift datiert: »Ende Sept. 902« \newline{}Ordnung: 1) mit Bleistift von unbekannter Hand nummeriert: »\strikeout{201}« 2) mit Bleistift von unbekannter Hand nummeriert: »185«}\buchAbdrucke{\weitereDrucke{1) Hugo von Hofmannsthal, Arthur Schnitzler: \emph{Briefwechsel}. Hg. Therese Nickl und Heinrich Schnitzler. Frankfurt am Main: \emph{S. Fischer} 1964, S. 160.} \weitereDrucke{2) Hermann Bahr, Arthur Schnitzler: \emph{Briefwechsel, Aufzeichnungen, Dokumente
                                (1891–1931)}. Hg. Kurt Ifkovits und Martin Anton Müller. Göttingen: \emph{Wallstein} 2018, S. 243.} }\toendnotes[C]{\smallbreak}\pstart
           \raggedleft{}{\pb}Dienstag abend\pend
           \pstart
           lieber, bitte ſchicken Sie den beiliegenden Brief mit meiner
                    gleichfalls beiliegenden Erledigung ſowie Ihrer Antwort an \textcolor{blue}{Bahr}{}\ledrightnote{\textcolor{blue}{Hermann Bahr}} zurück.\pend
           \pstart
           Ich fahre in 2 Stunden nach \textcolor{pink}{Rom}{}\ledrightnote{\textcolor{pink}{Rom}}.\pend
           \pstart
           Leben Sie und Ihrigen wohl. Hoffentlich arbeiten Sie viel und ſchön. Ihr
                        \spacefill\mbox{Hugo}.\pend
           \pstart
           \noindent{}\label{T_L01235_1v}\edtext{Adreſſe: \textcolor{pink}{Hôtel Haſſler Rom}{}\ledrightnote{\textcolor{pink}{Hôtel Hassler}}}{\lemma{\textnormal{\emph{Adreſſe: … Rom}}}\Cendnote{\textnormal{quer am rechten Rand}}}\label{T_L01235_1h}\pend
           \endnumbering\briefempfaengerindex{Schnitzler, Arthur@\textsc{Schnitzler, Arthur}!zzzHofmannsthal, Hugo von@\emph{von Hugo von Hofmannsthal}!1902-09-301@{{[}30. 9. 1902{]}}|)be}\mylabel{h}  \normalsize

\doendnotes{C}
\bigskip
\vfill

\clearpage

\footnotesize

\lohead{\textsc{register}}

% Definiere theindex-Environment komplett neu ohne reledmac
\makeatletter
\renewenvironment{theindex}{%
  \section*{\indexname}%
  \setlength{\parindent}{0pt}%
  \setlength{\parskip}{0pt plus 0.3pt}%
  \let\item\@idxitem
}{%
  \clearpage
}
\makeatother

\IfFileExists{\jobname-pw.ind}{\input{\jobname-pw.ind}}{}

\end{document}

      