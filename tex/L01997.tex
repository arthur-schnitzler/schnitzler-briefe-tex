%% latex-korrekturansicht-vorspann.tex
%% Vorspann für die Korrekturansicht.
%% Lädt die gemeinsame Datei latex-vorspann.tex mit gesetztem Schalter.

\newif\ifkorrekturansicht
\korrekturansichttrue

\input{../tex-inputs/latex-vorspann}


               \section[Albert Ehrenstein an Arthur Schnitzler, 31. 12. 1910]{ Albert Ehrenstein an Arthur Schnitzler, 31. 12. 1910}\nopagebreak\mylabel{v}\rehead{ }\normalsize\beginnumbering\briefempfaengerindex{Schnitzler, Arthur@\textsc{Schnitzler, Arthur}!zzzEhrenstein, Albert@\emph{von Albert Ehrenstein}!1910-12-311@{31. 12. 1910}|(be} \toendnotes[C]{\smallbreak\pagebreak[2]} \Standort{CUL, Schnitzler, B 30.}
\physDesc{Brief, 1 Blatt, 1 Seite
\newline{}Handschrift: schwarze Tinte, lateinische Kurrent}\buchAbdrucke{\weitereDrucke{Albert Ehrenstein: \emph{Briefe}. Hg. Hanni Mittelmann. München: \emph{Boer} 1989, S. 53 (Werke, 1).} }\pstart
           {\pb}\textcolor{pink}{XVI. Ottakringerstr 114}{}\ledrightnote{\textcolor{pink}{Ottakringerstraße}}.\hfill 31. Dez. 10.\pend
           \pstart\center{}Hochverehrter Herr Doktor!\pend\pstart
           Für Ihre liebenswürdige Gratulation danke ich herzlichst.\pend
           \pstart
           Es wäre mir sehr erfreulich, zu wissen, ob ich, nun nachdem ich auch noch zum
                    Ph. Doctor degradiert wurde, wieder einmal bei Ihnen vorsprechen darf?\pend
           \pstart
           Indem ich zum neuen Jahre wünsche: Zeit und Raum mögen sich Ihnen in der
                    willkommensten Weise gestalten,\pend
           \pstart
           bin ich\hspace*{2em}Hochachtungsvollst{\\[\baselineskip]}Euer Hochwohlgeboren{\\[\baselineskip]}ergebenster{\\[\baselineskip]}\spacefill\mbox{Albert Ehrenstein.}\pend
           \leftskip=0em{}\endnumbering\briefempfaengerindex{Schnitzler, Arthur@\textsc{Schnitzler, Arthur}!zzzEhrenstein, Albert@\emph{von Albert Ehrenstein}!1910-12-311@{31. 12. 1910}|)be}\mylabel{h}  \normalsize

\doendnotes{C}
\bigskip
\vfill

\clearpage

\footnotesize

\lohead{\textsc{register}}

% Definiere theindex-Environment komplett neu ohne reledmac
\makeatletter
\renewenvironment{theindex}{%
  \section*{\indexname}%
  \setlength{\parindent}{0pt}%
  \setlength{\parskip}{0pt plus 0.3pt}%
  \let\item\@idxitem
}{%
  \clearpage
}
\makeatother

\IfFileExists{\jobname-pw.ind}{\input{\jobname-pw.ind}}{}

\end{document}

      