%% latex-korrekturansicht-vorspann.tex
%% Vorspann für die Korrekturansicht.
%% Lädt die gemeinsame Datei latex-vorspann.tex mit gesetztem Schalter.

\newif\ifkorrekturansicht
\korrekturansichttrue

\input{../tex-inputs/latex-vorspann}


               \section[Hermann Bahr an Arthur Schnitzler, 20. 3. 1930]{ Hermann Bahr an Arthur Schnitzler, 20. 3. 1930}\nopagebreak\mylabel{v}\rehead{ }\normalsize\beginnumbering\briefempfaengerindex{Schnitzler, Arthur@\textsc{Schnitzler, Arthur}!zzzBahr, Hermann@\emph{von Hermann Bahr}!1930-03-201@{20. 3. 1930}|(be} \toendnotes[C]{\smallbreak\pagebreak[2]} \Standort{CUL, Schnitzler, B 5b.}
\physDesc{Brief, 1 Blatt, 2 Seiten
\newline{}Handschrift: schwarze Tinte, deutsche Kurrent
\newline{}Schnitzler: mit rotem Buntstift mehrere Unterstreichungen \newline{}Ordnung: mit Bleistift von unbekannter Hand nummeriert: »187« }\buchAbdrucke{\weitereDrucke{Hermann Bahr, Arthur Schnitzler: \emph{Briefwechsel, Aufzeichnungen, Dokumente (1891–1931)}. Hg. Kurt Ifkovits und Martin Anton Müller. Göttingen: \emph{Wallstein} 2018, S. 596–597.} }\toendnotes[C]{\smallbreak}\pstart
           \raggedleft{}{\pb}\textcolor{pink}{München Barerſtr. 50}{}\ledrightnote{\textcolor{pink}{Barerstraße}}{\\}20. 3. 30\pend
           \pstart{}Mein lieber Arthur!\pend\pstart
           Woltun bringt Zinſen, aber ich bin undankbar genug, Dir die Wohltat, die mir Dein
               lieber Brief erweiſt, übel zu vergelten: durch Jammern über mein \textcolor{pink}{Münchener}{}\ledrightnote{\textcolor{pink}{München}} Ungemach. Du fragſt, warum \textcolor{blue}{wir}{}\ledrightnote{→\textcolor{blue}{Anna Bahr-Mildenburg}} nach \textcolor{pink}{München}{}\ledrightnote{\textcolor{pink}{München}} überſiedelten. Wir waren Beide »ſtellungslos«, als ich zur Leitung
               des \textcolor{pink}{Burgtheaters}{}\ledrightnote{\textcolor{pink}{Burgtheater}} berufen wurde – viel zu ſpät, um
               noch etwas künſtleriſch leiſten oder doch retten zu können. Um dieſe Zeit begann auch
               die \textcolor{pink}{öſterreichiſche}{}\ledrightnote{\textcolor{pink}{Österreich}} Währung ſchon zu wanken. Das
               bischen »Vermögen«, das mir mein \textcolor{blue}{Vater}{}\ledrightnote{→\textcolor{blue}{Alois Bahr}} hinterlaſſen hatte, begann zu ſchmelzen; der Reſt ging dann bei der
                  \textcolor{pink}{deutſchen}{}\ledrightnote{\textcolor{pink}{Deutschland}} Inflation vollends auf. Ganz unverhofft
               ging da an meine \textcolor{blue}{Frau}{}\ledrightnote{→\textcolor{blue}{Anna Bahr-Mildenburg}} der Ruf,
               an der \textcolor{brown}{Münchener Akademie}{}\ledrightnote{\textcolor{brown}{Akademie der Tonkunst}} eine Profeſſur anzunehmen,
               ſie griff mit beiden Händen zu, wir waren die Sorge los, wovon wir morgen unſer
               Mittagmal {\pb}beſtreiten ſollten; nach einer Reihe von
               Jahren erhält meine \textcolor{blue}{Frau}{}\ledrightnote{→\textcolor{blue}{Anna Bahr-Mildenburg}} als
               Penſion ihren vollen Gehalt. An ſie kam übrigens auch ein \label{K_L02534_1v}\edtext{Ruf}{\lemma{\textnormal{\emph{Ruf}}}\Cendnote{\textnormal{Anfang Januar 1927 ging eine solche Übersiedlung durch die
                     Zeitungen.}}}\label{K_L02534_1h} an die \textcolor{brown}{Berliner Muſikhochſchule}{}\ledrightnote{\textcolor{brown}{Akademische Hochschule für Musik}},
               den sie natürlich ausſchlug, weil \textcolor{pink}{Berlin}{}\ledrightnote{\textcolor{pink}{Berlin}} noch weiter
               von ihrem unvergeßlichen \textcolor{pink}{Wien}{}\ledrightnote{\textcolor{pink}{Wien}} iſt als \textcolor{pink}{München}{}\ledrightnote{\textcolor{pink}{München}}. Mir perſönlich iſt es im Grunde wurſcht,
               in welcher Stadt ich lebe, ich würde ſchließlich auch auf dem Monde ganz gemütlich
               leben können. Es fällt mir nur ſchwer meine \textcolor{blue}{Frau}{}\ledrightnote{→\textcolor{blue}{Anna Bahr-Mildenburg}}{ }ſich ſo von Sehnſucht nach \textcolor{pink}{Wien}{}\ledrightnote{\textcolor{pink}{Wien}} verzehren zu ſehen. Ich \label{K_L02534_2v}\edtext{ſprach vor einigen Jahren}{\lemma{\textnormal{\emph{ſprach … Jahren}}}\Cendnote{\textnormal{Das dürfte sich
                  auf ein Gespräch beziehen, das zwischen dem 26. und
                     29. 9. 1923 in \textcolor{pink}{Wien}
                   stattgefunden
                  hat (\emph{Schicksalsjahre Österreichs. Die Erinnerungen und Tagebücher
                        Josef Redlichs 1869–1936.} Hg. Fritz Fellner und Doris A. Corradini.
                     Wien: \emph{Böhlau} 2011, II, S. 624).}}}\label{K_L02534_2h} mit dem
               Prälaten \textcolor{blue}{Seipel}{}\ledrightnote{\textcolor{blue}{Ignaz Seipel}}, den ich ſehr \substVorne{}\textsuperscript{ſ}\substDazwischen{}l\substHinten{}ange kenne, über die Möglichkeit einer Berufung meiner \textcolor{blue}{Frau}{}\ledrightnote{→\textcolor{blue}{Anna Bahr-Mildenburg}} nach \textcolor{pink}{Wien}{}\ledrightnote{\textcolor{pink}{Wien}}, ſei’s auch nur in der Form, daß sie zwei Mal im Jahre, jedes Mal drei
               Wochen, Lehrkurſe an der \textcolor{pink}{Wiener »Hochſchule und Akademie
                  für Muſik und darſtellende Kunſt«}{}\ledrightnote{\textcolor{pink}{Hochschule und Akademie für Musik und Darstellende Kunst}} halten ſollte. \textcolor{blue}{Seipel}{}\ledrightnote{\textcolor{blue}{Ignaz Seipel}} ließ mir dann ſagen, der betreffende »Akt« liege ſchon
               im \textcolor{pink}{Unterrichtsminiſterium}{}\ledrightnote{\textcolor{pink}{Ministerium für Unterricht}}. Dort liegt er offenbar
               noch heute. »\label{K_L02534_3v}\edtext{Segens ſo heiter iſt das
               Leben in \textcolor{pink}{Wien}{}\ledrightnote{\textcolor{pink}{Wien}}!}{\lemma{\textnormal{\emph{Segens … Wien!}}}\Cendnote{\textnormal{Titel eines Couplets aus \emph{\textcolor{green}{Die Wienerstadt in Wort und Bild}} von \textcolor{blue}{Julius Bauer}, \textcolor{blue}{Isidor
                     Fuchs} und \textcolor{blue}{Camillo Walzel}
                  (1887).}}}\label{K_L02534_3h}«\pend
           \pstart
           Verzeih die lange Epiſtel\hspace*{1.5em}Deinem getreuen{\\[\baselineskip]}\spacefill\mbox{Hermann}\pend
           \leftskip=0em{}\endnumbering\briefempfaengerindex{Schnitzler, Arthur@\textsc{Schnitzler, Arthur}!zzzBahr, Hermann@\emph{von Hermann Bahr}!1930-03-201@{20. 3. 1930}|)be}\mylabel{h}  \normalsize

\doendnotes{C}
\bigskip
\vfill

\clearpage

\footnotesize

\lohead{\textsc{register}}

% Definiere theindex-Environment komplett neu ohne reledmac
\makeatletter
\renewenvironment{theindex}{%
  \section*{\indexname}%
  \setlength{\parindent}{0pt}%
  \setlength{\parskip}{0pt plus 0.3pt}%
  \let\item\@idxitem
}{%
  \clearpage
}
\makeatother

\IfFileExists{\jobname-pw.ind}{\input{\jobname-pw.ind}}{}

\end{document}

      