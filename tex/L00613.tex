%% latex-korrekturansicht-vorspann.tex
%% Vorspann für die Korrekturansicht.
%% Lädt die gemeinsame Datei latex-vorspann.tex mit gesetztem Schalter.

\newif\ifkorrekturansicht
\korrekturansichttrue

\input{../tex-inputs/latex-vorspann}


               \section[Richard Beer-Hofmann an Arthur Schnitzler, 30. {[}10. 1896{]}]{ Richard Beer-Hofmann an Arthur Schnitzler, 30. {[}10. 1896{]}}\nopagebreak\mylabel{v}\rehead{ }\normalsize\beginnumbering\briefempfaengerindex{Schnitzler, Arthur@\textsc{Schnitzler, Arthur}!zzzBeer-Hofmann, Richard@\emph{von Richard Beer-Hofmann}!1896-10-302@{30. {[}10. 1896{]}}|(be} \toendnotes[C]{\smallbreak\pagebreak[2]} \Standort{CUL, Schnitzler, B 8.}
\physDesc{Telegramm
\newline{}maschinell\newline{}Ordnung: mit Bleistift von unbekannter Hand
                              nummeriert: »88« }\toendnotes[C]{\smallbreak}\pstart
           \noindent{}{\pb}\textcolor{pink}{b}{}\ledrightnote{\textcolor{pink}{Berlin}}{ }de{ }\textcolor{pink}{wien}{}\ledrightnote{\textcolor{pink}{Wien}} 111.–529 16 6{ }30–\pend
           \pstart
           den schoensten \textcolor{green}{erfolg}{}\ledrightnote{→\textcolor{green}{Freiwild. Schauspiel in 3 Akten}} und herzliche gruesse von
               dem halbwahren aus \textcolor{pink}{upsala}{}\ledrightnote{\textcolor{pink}{Uppsala}} +\pend
           \endnumbering\briefempfaengerindex{Schnitzler, Arthur@\textsc{Schnitzler, Arthur}!zzzBeer-Hofmann, Richard@\emph{von Richard Beer-Hofmann}!1896-10-302@{30. {[}10. 1896{]}}|)be}\mylabel{h}  \normalsize

\doendnotes{C}
\bigskip
\vfill

\clearpage

\footnotesize

\lohead{\textsc{register}}

% Definiere theindex-Environment komplett neu ohne reledmac
\makeatletter
\renewenvironment{theindex}{%
  \section*{\indexname}%
  \setlength{\parindent}{0pt}%
  \setlength{\parskip}{0pt plus 0.3pt}%
  \let\item\@idxitem
}{%
  \clearpage
}
\makeatother

\IfFileExists{\jobname-pw.ind}{\input{\jobname-pw.ind}}{}

\end{document}

      