%% latex-korrekturansicht-vorspann.tex
%% Vorspann für die Korrekturansicht.
%% Lädt die gemeinsame Datei latex-vorspann.tex mit gesetztem Schalter.

\newif\ifkorrekturansicht
\korrekturansichttrue

\input{../tex-inputs/latex-vorspann}


               \section[Arthur Schnitzler an Richard Beer-Hofmann, 15. 9. 1895]{ Arthur Schnitzler an Richard Beer-Hofmann, 15. 9. 1895}\nopagebreak\mylabel{v}\rehead{ }\normalsize\beginnumbering\briefempfaengerindex{Beer-Hofmann, Richard@\textsc{Beer-Hofmann, Richard}!zzzSchnitzler, Arthur@\emph{von Arthur Schnitzler}!1895-09-151@{15. 9. 1895}|(be} \toendnotes[C]{\smallbreak\pagebreak[2]} \Standort{YCGL, MSS 31.}
\physDesc{Brief, 2 Blätter, 7 Seiten, Umschlag
\newline{}Handschrift: 1) Bleistift, deutsche Kurrent\hspace{1em}2) schwarze Tinte, deutsche Kurrent (\noindent{}Adressierung)\hspace{1em}\newline{}Versand: 1) Stempel: »\nobreak{}\oindex{IX., Alsergrund@\textbf{IX., Alsergrund}, \emph{Bezirk (A.BZK)}|pwk}Wien 9/3, 16. 9. 95, 6–7 V\nobreak{}«.  2) Stempel: »\nobreak{}\oindex{Schoenberg im Stubaital@\textbf{Schönberg im Stubaital}, \emph{Besiedelter Ort (A.BSO)}|pwk}{\pb}{[}Sch{]}önb{[}e{]}rg\nobreak{}«. }\buchAbdrucke{\weitereDrucke{1) Arthur Schnitzler: \emph{Briefe 1875–1912}. Hg. Therese Nickl und Heinrich Schnitzler. Frankfurt am Main: \emph{S. Fischer} 1981, S. 277–278.} \weitereDrucke{2) Arthur Schnitzler, Richard Beer-Hofmann: \emph{Briefwechsel 1891–1931}. Hg. Konstanze Fliedl. Wien, Zürich: \emph{Europaverlag} 1992, S. 80–81.} \weitereDrucke{3) Hermann Bahr, Arthur Schnitzler: \emph{Briefwechsel, Aufzeichnungen, Dokumente (1891–1931)}. Hg. Kurt Ifkovits und Martin Anton Müller. Göttingen: \emph{Wallstein} 2018.} }\toendnotes[C]{\smallbreak}\pstart{}{\pb}Herrn Dr. \textsc{Richard
                     Beer-Hofmann}\pend{}\pstart{}\textsc{\textcolor{pink}{Schönberg im Stubaithal}{}\ledrightnote{\textcolor{pink}{Schönberg im Stubaital}}}\pend{}\pstart{}\textsc{\textcolor{pink}{Tirol}{}\ledrightnote{\textcolor{pink}{Tirol}}}\pend{}{\bigskip}\pstart
           \raggedleft{}{\pb}So{\geminationn}tg 15. 9. 95.\pend
           \pstart
           Lieber Richard. Ich freue mich, daſs Sie in guter Sti{\geminationm}ung ſind. Wahrscheinlich werden Sie bald südlicher
               gehn; kennen Sie \textsc{\textcolor{pink}{Riva}{}\ledrightnote{\textcolor{pink}{Riva del Garda}}}? Es iſt ſchön, war \introOben{}mir\introOben{} aber nicht ſympathiſch. Ich bin
               von dort nach \textcolor{pink}{Venedig}{}\ledrightnote{\textcolor{pink}{Venedig}} gegangen; es iſt so nah. Sie
               haben \uline{mich} falſch verſtanden; ich wußte, dſs Sie Ende
               Sept. in \textcolor{pink}{Wien}{}\ledrightnote{\textcolor{pink}{Wien}}{ }ſein wollten. An dieſes \textcolor{pink}{Wien}{}\ledrightnote{\textcolor{pink}{Wien}} hab ich mich noch nicht ganz gewöhnt; empfinde gleich wieder, jetzt wo
               die alten Verhältniſſe sich aufdrängen, das vielfach unzulängliche, unter dem man zu
               leiden hat. Dünne Fäden, mit denen {\pb}man an mancherlei
               gebunden iſt – dünn, aber doch Fäden. Denken Sie, ſeit ich hier bin, bin ich bereits
               2mal in der früh \introOben{}(um 6 oder ½ 7)\introOben{} geweckt worden – von
               Patienten, nicht vom \textcolor{pink}{Burgtheater}{}\ledrightnote{\textcolor{pink}{Burgtheater}}. – Am Mittwoch 18.
               ſoll \textcolor{green}{Leſeprobe}{}\ledrightnote{→\textcolor{green}{Liebelei. Schauspiel in drei Akten}}{ }ſein; wenigſtens ist sie angesetzt.\pend
           \pstart
           – Die \textcolor{blue}{S.}{}\ledrightnote{\textcolor{blue}{Adele Sandrock}} verhält ſich ſtille; ihre Feindſeligkeit
               hat ſie vorläufig nur dadurch ausgedrückt, daſs ſie ihrer \textcolor{pink}{ruſſiſchen}{}\ledrightnote{\textcolor{pink}{Russland}}{ }\textcolor{blue}{Freundin}{}\ledrightnote{→\textcolor{blue}{Olga von Golovin}} einen Brief ſchrieb,
               ſie dürfe \uline{mich} nicht mehr als Arzt nehmen, wenn ſie
               mit ihr verkehren wolle. Die \textcolor{pink}{ruſſiſche}{}\ledrightnote{\textcolor{pink}{Russland}}{ }\textcolor{blue}{Freundin}{}\ledrightnote{→\textcolor{blue}{Olga von Golovin}} kümmert ſich nicht
               drum {\pb}und läßt ſich mit Begeiſterung von mir
               behandeln. – \textcolor{blue}{\textsc{Bckhrd}}{}\ledrightnote{\textcolor{blue}{Max Eugen Burckhard}}{ }ſprach neulich das erſte Mal von der Sache: »Ich
               hab ja nur zufällig durch den \textcolor{blue}{Bahr}{}\ledrightnote{\textcolor{blue}{Hermann Bahr}} von der Sache
               erfahren {\dotstwo} aber ich werd ihr ſchon begreiflich machen,
               daſs das beim \textcolor{pink}{Burgtheater}{}\ledrightnote{\textcolor{pink}{Burgtheater}} nicht geht – beſonders \uline{ſie}{\dots} Freilich mit Ketten kann ich ſie nicht auf die Bühne
               zerren.« – Man war bei \textcolor{blue}{\textsc{Besezny}}{}\ledrightnote{\textcolor{blue}{Josef von Bezecný}}, ihm erzählen, wie du{\geminationm} und ordinär mein \textcolor{green}{Stück}{}\ledrightnote{→\textcolor{green}{Liebelei. Schauspiel in drei Akten}}{ }ſei. – Unser Freund \textcolor{blue}{J. J. David}{}\ledrightnote{\textcolor{blue}{Jakob Julius David}}: Ich werde \label{K_L00483_1v}\edtext{vielleicht \textcolor{green}{durch{\pb}fallen}{}\ledrightnote{→\textcolor{green}{Ein Regentag}}}{\lemma{\textnormal{\emph{vielleicht durchfallen}}}\Cendnote{\textnormal{\emph{\textcolor{green}{Ein Regentag}}; Uraufführung im \textcolor{pink}{Deutschen Volkstheater} am 12. 10. 1895}}}\label{K_L00483_1h},
               der \textsc{Schnitzler} aber doch ganz gewiſs. –\pend
           \pstart
           – \textcolor{blue}{\textsc{Speidel}}{}\ledrightnote{\textcolor{blue}{Ludwig Speidel}} zu \textcolor{blue}{\textsc{Eberma{\geminationn}}}{}\ledrightnote{\textcolor{blue}{Leo Ebermann}} über die \textcolor{green}{Liebelei}{}\ledrightnote{\textcolor{green}{Liebelei. Schauspiel in drei Akten}} – »Da werden die \textcolor{pink}{Wiener}{}\ledrightnote{\textcolor{pink}{Wien}}{ }ſchaun!« – Iſt vom \textcolor{green}{Anatol}{}\ledrightnote{\textcolor{green}{Anatol}} äußerst – (ich genire mich »entzückt« zu ſchreiben.) – Theater: \textcolor{green}{Alte Wiener}{}\ledrightnote{\textcolor{green}{Alte Wiener}}, ſchlechtes Stück von \textcolor{blue}{Anzengruber}{}\ledrightnote{\textcolor{blue}{Ludwig Anzengruber}}. \textcolor{green}{Böſe Zungen}{}\ledrightnote{\textcolor{green}{Böse Zungen}},
               lächerliches Stück von \textcolor{blue}{\textsc{Laube}}{}\ledrightnote{\textcolor{blue}{Heinrich Laube}}. –\pend
           \pstart
           Die \textcolor{blue}{Eltern}{}\ledrightnote{→\textcolor{blue}{Hugo August von Hofmannsthal}{\newline}→\textcolor{blue}{Anna von Hofmannsthal}}{ }\textcolor{blue}{\textsc{Hugo}}{}\ledrightnote{\textcolor{blue}{Hugo von Hofmannsthal}}s \label{K_L00483_2v}\edtext{neulich im Kaffeehaus}{\lemma{\textnormal{\emph{neulich im Kaffeehaus}}}\Cendnote{\textnormal{am 12. 9. 1895}}}\label{K_L00483_2h}. \textcolor{blue}{\textsc{Hugo}}{}\ledrightnote{\textcolor{blue}{Hugo von Hofmannsthal}} ritt durch \textcolor{pink}{Wien}{}\ledrightnote{\textcolor{pink}{Wien}}; ſie ſtanden beim \textcolor{pink}{Tegethoffmonument}{}\ledrightnote{\textcolor{pink}{Tegetthoff-Denkmal}} und ſchauten zu. Er war in \textcolor{pink}{Göding}{}\ledrightnote{\textcolor{pink}{Hodonín}}{ }ſehr unglücklich; die Manöver ſollen {\pb}ihm enorm gefallen haben. Jetzt iſt er in \textcolor{pink}{Bruck}{}\ledrightnote{\textcolor{pink}{Bruck an der Mur}}. –\hspace*{1.5em}Geſprochen:
                  \textcolor{blue}{\textsc{Salten}}{}\ledrightnote{\textcolor{blue}{Felix Salten}} oft, \textcolor{blue}{\textsc{Schwarzkopf}}{}\ledrightnote{\textcolor{blue}{Gustav Schwarzkopf}} einige Mal, \textcolor{blue}{\textsc{Gold}}{}\ledrightnote{\textcolor{blue}{Alfred Gold}}{ }ſelten, \textcolor{blue}{\textsc{Bahr}}{}\ledrightnote{\textcolor{blue}{Hermann Bahr}} (Guten Tag, wie gehts dir denn?) Seine \textcolor{blue}{Frau}{}\ledrightnote{→\textcolor{blue}{Rosa Bahr}} heute ein Stück begleitet, mich dringlich zum Beſuche
               aufgefordert. Auch \uline{er} fährt ſchon \textsc{bicycle}. –\pend
           \pstart
           – Gearbeitet noch gar nichts – ſchämen Sie ſich, daſs ich mich nicht vor Ihnen zu
               ſchämen brauche.\pend
           \pstart
           Die \textcolor{blue}{Brion}{}\ledrightnote{\textcolor{blue}{Lou Brion}}{ }ſoll über uns geäußert haben: Setzen ſich in die
               Proſceniumsloge – und {\pb}man kriegt kein \textsc{Bracelet}, nicht einmal eine Einladung zum \textsc{Souper}! – Quelle unlauter, nemlich \textcolor{blue}{Paul Horn}{}\ledrightnote{\textcolor{blue}{Paul Horn}}. Dieſer tadelt an der \textcolor{green}{kleinen Komödie}{}\ledrightnote{→\textcolor{green}{Liebelei. Schauspiel in drei Akten}} die Unmöglichkeit, daſs ſich ein Menſch
               wirklich von den Seidenſtrümpfen und den \textsc{grande marque}
               Cocotten zu einem lieben Vorſtadtmädel hingezogen fühlen ſollte. –\pend
           \pstart
           Hier regnet es i{\geminationm}er – und Sie? – Alles erkundigt ſich
               nach Ihnen; ſind Sie ſtolz? Leben Sie wohl, laſſen Sie ſchnell {\pb}wieder was von ſich hören, bringen Sie den fertigen
                  \textcolor{green}{Götterliebling}{}\ledrightnote{→\textcolor{green}{Der Tod Georgs}} und viel Luſt
               zu neuen Werken mit. Sagen Sie, wie hat denn die \textcolor{blue}{Lou}{}\ledrightnote{\textcolor{blue}{Lou Brion}} das Alleinfahrenmüſſen aufgeno{\geminationm}en? Hier ist
               es »bekannt geworden« daſs wir miteinander nicht über Literatur reden; man findet das
               höchſt anmaßend – »ſo groß ſind ſie nicht, daß ſie nicht mehr über Literatur reden
               müßten.« – Laßt uns lächeln.\pend
           \pstart Ihr \spacefill\mbox{Arthur Sch} mit vielen herzlichen Grüßen.\pend{}\endnumbering\briefempfaengerindex{Beer-Hofmann, Richard@\textsc{Beer-Hofmann, Richard}!zzzSchnitzler, Arthur@\emph{von Arthur Schnitzler}!1895-09-151@{15. 9. 1895}|)be}\mylabel{h}  \normalsize

\doendnotes{C}
\bigskip
\vfill

\clearpage

\footnotesize

\lohead{\textsc{register}}

% Definiere theindex-Environment komplett neu ohne reledmac
\makeatletter
\renewenvironment{theindex}{%
  \section*{\indexname}%
  \setlength{\parindent}{0pt}%
  \setlength{\parskip}{0pt plus 0.3pt}%
  \let\item\@idxitem
}{%
  \clearpage
}
\makeatother

\IfFileExists{\jobname-pw.ind}{\input{\jobname-pw.ind}}{}

\end{document}

      