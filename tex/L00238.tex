%% latex-korrekturansicht-vorspann.tex
%% Vorspann für die Korrekturansicht.
%% Lädt die gemeinsame Datei latex-vorspann.tex mit gesetztem Schalter.

\newif\ifkorrekturansicht
\korrekturansichttrue

\input{../tex-inputs/latex-vorspann}


               \section[Hugo von Hofmannsthal an Arthur Schnitzler, 19. 7. 1893]{ Hugo von Hofmannsthal an Arthur Schnitzler, 19. 7. 1893}\nopagebreak\mylabel{v}\rehead{ }\normalsize\beginnumbering\briefempfaengerindex{Schnitzler, Arthur@\textsc{Schnitzler, Arthur}!zzzHofmannsthal, Hugo von@\emph{von Hugo von Hofmannsthal}!1893-07-191@{19. 7. 1893}|(be} \toendnotes[C]{\smallbreak\pagebreak[2]} \Standort{CUL, Schnitzler, B 43.}
\physDesc{Brief, 1 Blatt, 4 Seiten
\newline{}Handschrift: schwarze Tinte, deutsche Kurrent\newline{}Ordnung: mit Bleistift von unbekannter Hand nummeriert: »54« }\buchAbdrucke{\weitereDrucke{1) Hugo von Hofmannsthal, Arthur Schnitzler: \emph{Briefwechsel}. Hg. Therese Nickl und Heinrich Schnitzler. Frankfurt am Main: \emph{S. Fischer} 1964, S. 40–41.} \weitereDrucke{2) Hermann Bahr, Arthur Schnitzler: \emph{Briefwechsel, Aufzeichnungen, Dokumente
                                (1891–1931)}. Hg. Kurt Ifkovits und Martin Anton Müller. Göttingen: \emph{Wallstein} 2018, S. 35.} }\toendnotes[C]{\smallbreak}\pstart
           {\pb}\textcolor{pink}{Salzburg Bad-Fuſch}{}\ledrightnote{\textcolor{pink}{Bad Fusch}},\hfill 19. VII. 93\pend
           \pstart{}lieber Arthur!\pend\pstart
           \textcolor{blue}{Richard}{}\ledrightnote{\textcolor{blue}{Richard Beer-Hofmann}}s Bericht von dem »\textcolor{green}{Abschiedsſouper}{}\ledrightnote{\textcolor{green}{Abschiedssouper}}« war recht unerfreulich; er ſcheint mit der
                    gewiſſen Hellſichtigkeit der Autoren jede Mücke als Elefanten empfunden zu
                    haben; wie es wirklich war, weiß ich natürlich nicht, jedenfalls iſt die überaus
                    freundliche, gewiſſermaßen reſpectvolle \textcolor{green}{Notiz in der »\textcolor{brown}{Neuen Freien
                            Preſſe}{}\ledrightnote{\textcolor{brown}{Neue Freie Presse}}«}{}\ledrightnote{→\textcolor{green}{Aus Ischl, 14. Juli, schreibt man uns: …}}{ }ſehr erfreulich und nützt 10mal mehr als die
                    Aufführung ſelbſt. So wird im ganzen dieſer Einbruch von äußerem Leben in Ihr
                    inneres keine ſchlechte Laune zurückgelaſſen haben.\pend
           \pstart
           {\pb}Ich freue mich ſchon recht
                    ſehr auf die \textcolor{green}{Parallel-novelle}{}\ledrightnote{→\textcolor{green}{Die kleine Komödie}}.\pend
           \pstart
           Mein Leben verſtreicht ziemlich nichtsſagend, mit \introOben{}langſam\introOben{}{ }ſteigendem inneren Wohlbefinden. Von \textcolor{pink}{Strobl}{}\ledrightnote{\textcolor{pink}{Strobl}} hoffe ich manches Schöne: Sonne und Mond
                    am Waſſer, Segeln, kindlich-lärmende Vergnügungen, \textcolor{blue}{Richard}{}\ledrightnote{\textcolor{blue}{Richard Beer-Hofmann}}, auch \textcolor{blue}{Schwarzkopf}{}\ledrightnote{\textcolor{blue}{Gustav Schwarzkopf}}; nur Sie gar nicht?\pend
           \pstart
           Ich leſe mit lebhafteſtem Intereſſe die »\textcolor{green}{Hauptſtrömungen}{}\ledrightnote{\textcolor{green}{Hauptströmungen der Literatur des neunzehnten Jahrhunderts}}« von \textcolor{blue}{Brandes}{}\ledrightnote{\textcolor{blue}{Georg Brandes}},
                    unendlich vieles aus der 1\textsuperscript{ten} Hälfte des Säculums
                    besitzt im zweiten ein Gegenbild, manches eine Carricatur; namentlich ſehe ich
                    mit halb ſchauerndem Staunen, {\pb}wie völlig ſich die \introOben{}Producte der\introOben{} jüngſten Strömungen,
                    in denen ich ja auch mit einer Fußſpitze ſtehe, der Romantik als
                    Kugelſpiegelbild, halb verſchrumpft, halb aufgedunſen, gegenüberſtellen.\pend
           \pstart
           Ich habe mir ſehr viel abzugewöhnen, aber es ſind wenigſtens lauter echte
                    Dichterkrankheiten.\pend
           \pstart
           Mir ſcheint, der Satz klingt maßlos arrogant; leſen Sie ihn nicht ſo.\pend
           \pstart
           Sie müſſen mir einen handgreiflichen Gefallen thuen: ich bin mit \textcolor{blue}{Bahr}{}\ledrightnote{\textcolor{blue}{Hermann Bahr}} verabredet, Ende Juli nach \textcolor{pink}{München}{}\ledrightnote{\textcolor{pink}{München}} zu gehen; mir paſst 24.
                    (eventuell 25.) bis 1. Auguſt; ſeit 14 Tagen
                    beantwortet \textcolor{blue}{Bahr}{}\ledrightnote{\textcolor{blue}{Hermann Bahr}} keinen Brief. Ich muſs aber
                    doch endlich wiſſen, {\pb}woran
                    ich bin. Alſo bitte, telefonieren Sie in meinem Namen an die \textcolor{brown}{Redaction der »Deutſchen Zeitung«}{}\ledrightnote{\textcolor{brown}{Deutsche Zeitung}}, man möge entweder \textcolor{blue}{Bahr}{}\ledrightnote{\textcolor{blue}{Hermann Bahr}} meine dringende Aufforderung endlich
                    zukommen laſſen, oder ſeine Adreſſe angeben, oder wenn man das nicht darf,
                    wenigſtens ſagen, wie lang er beiläufig \textsc{incognito} oder
                    verſchollen bleiben dürfte. Und bitte, ſchreiben Sie mir \uuline{ſofort} den Beſcheid.\pend
           \pstart
           Herzlichst{\\[\baselineskip]}Ihr \spacefill\mbox{Loris.}\pend
           \leftskip=0em{}\pstart
           \noindent{}Warum antwortet \textcolor{blue}{Salten}{}\ledrightnote{\textcolor{blue}{Felix Salten}} nicht?\pend
           \endnumbering\briefempfaengerindex{Schnitzler, Arthur@\textsc{Schnitzler, Arthur}!zzzHofmannsthal, Hugo von@\emph{von Hugo von Hofmannsthal}!1893-07-191@{19. 7. 1893}|)be}\mylabel{h}  \normalsize

\doendnotes{C}
\bigskip
\vfill

\clearpage

\footnotesize

\lohead{\textsc{register}}

% Definiere theindex-Environment komplett neu ohne reledmac
\makeatletter
\renewenvironment{theindex}{%
  \section*{\indexname}%
  \setlength{\parindent}{0pt}%
  \setlength{\parskip}{0pt plus 0.3pt}%
  \let\item\@idxitem
}{%
  \clearpage
}
\makeatother

\IfFileExists{\jobname-pw.ind}{\input{\jobname-pw.ind}}{}

\end{document}

      