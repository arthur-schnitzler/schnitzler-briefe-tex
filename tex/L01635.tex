%% latex-korrekturansicht-vorspann.tex
%% Vorspann für die Korrekturansicht.
%% Lädt die gemeinsame Datei latex-vorspann.tex mit gesetztem Schalter.

\newif\ifkorrekturansicht
\korrekturansichttrue

\input{../tex-inputs/latex-vorspann}


               \section[Arthur Schnitzler an Max Mell, 26. 10. 1906]{ Arthur Schnitzler an Max Mell, 26. 10. 1906}\nopagebreak\mylabel{v}\rehead{ }\normalsize\beginnumbering\briefempfaengerindex{Mell, Max@\textsc{Mell, Max}!zzzSchnitzler, Arthur@\emph{von Arthur Schnitzler}!1906-10-261@{26. 10. 1906}|(be} \toendnotes[C]{\smallbreak\pagebreak[2]} \Standort{DLA, A:Schnitzler, HS.NZ85.1.1403.}
\physDesc{Brief, 4 Blätter, 4 Seiten, maschineller Durchschlag
\newline{}Schreibmaschine
\newline{}Handschrift  : Bleistift, lateinische Kurrent (\noindent{}geringfügige Korrekturen)\newline{}Handschrift Arthur Schnitzler: 1) roter Buntstift, lateinische Kurrent (\noindent{}Beschriftung: »Mell«, »K{[}opie{]}« und mehrere Unterstreichungen)\hspace{1em}2) Bleistift, deutsche Kurrent (\noindent{}Korrekturen, Paginierung (2–4), auf dem zweiten Blatt Autorname und
                                 Datierung: 26/10 06)\hspace{1em}\newline{}Editorischer Hinweis: die handschriftlichen Korrekturen der Typistin eingearbeitet und
                                 nicht separat ausgewiesen }\buchAbdrucke{\weitereDrucke{Arthur Schnitzler: \emph{Briefe 1875–1912}. Hg. Therese Nickl und Heinrich Schnitzler. Frankfurt am Main: \emph{S. Fischer} 1981, S. 546–548.} }\toendnotes[C]{\smallbreak}\pstart
           \raggedleft{}{\pb}\textcolor{pink}{XVIII Spoettelgasse 7}{}\ledrightnote{\textcolor{pink}{Edmund-Weiß-Gasse}}.{\\}\textcolor{pink}{Wien}{}\ledrightnote{\textcolor{pink}{Wien}} am 26. Okt. 06\pend
           \pstart\center{}Sehr geehrter Herr Doktor, \pend\pstart
           In Ihrem \textcolor{green}{Stück}{}\ledrightnote{→\textcolor{green}{Die Komödianten}}, das Sie die
               Freundlichkeit hatten mich lesen zu lassen, gibt es viele wohlgelungene poetische und
               theatralische Momente und doch will am Ende kein Gefühl der wirklichen Befriedigung
               aufkommen. Woran das liegen mag? Wie ich glaube, an einer gewissen Lockerheit in der
               Behandlung der drei Hauptgestalten, denen allen nicht nur die schöne Inkonsequenz der
               Leidenschaft sondern auch jene andre zu Teil geworden ist, die durch eine gewisse
               Willkür oder Unsicherheit des Autors verschuldet wird. Ich kann nicht glauben, dass
               die Gräfin, die Sie schildern, trotz der Gefahren, die sie ahnt, die innere Kraft
               aufbringen wird, ihre Rolle zu studieren, sich zur Vorstellung bereit zu halten und
               tatsächlich aufzutreten. Und ich glaube noch weniger an die Grausamkeit ihres Grafen
               im zweiten und an seine etwas salbaderische Güte im dritten Akt. Vielleicht könnte ich
               an die Grausamkeit oder an {\pb}die Güte glauben \strikeout{b} denn es bleibt ja Grausamke{[}i{]}t,
               trotzdem oder wird sogar erst Grausamkeit weil der Graf schon im zweiten Akte weiss,
               was er im dritten tun wird. Freilich weiss ich nicht zu sagen, welchen Ausgang ich
               diesem dritten Akte wünschen würde. Gewiss nicht den tragischen, den Sie im Verlaufe
               der Begebenheiten erwarten lassen schon mit der Absicht, dass diese Erwartung
               getäuscht werde. Sie haben das innere Abrücken der Gräfin von dem Schauspieler an
               einigen Stellen angedeutet, aber ich glaube nicht, dass dieses Abrücken durch die
               paar neuen Lichter, die Sie dem Charakter des Paares aufsetzen, genügend motiviert
               erscheint. So fehlt{[}’{]}s grossenteils an der schönen
               Allmählichkeit, welche mir ein Grundgesetz aller Kunstwerke zu sein scheint, denn
               auch was als überraschend auf uns wirkt, ist im wirklichen Kunstwerk immer nur
               scheinbar überraschend, irgendwo in den Tiefen unserer Seele haben wir gewusst, dass
               es so kommen wird; sonst hätte es nicht so kommen können.\pend
           \pstart
           {\pb}Es ist schade, dass wir nicht mehr über das \textcolor{green}{Stück}{}\ledrightnote{→\textcolor{green}{Die Komödianten}} plaudern können, wie es
               neulich zwischen Ihrem Fräulein \textcolor{blue}{Schwester}{}\ledrightnote{→\textcolor{blue}{Maria Mell}}, meiner \textcolor{blue}{Frau}{}\ledrightnote{→\textcolor{blue}{Olga Schnitzler}}
               und mir geschehen ist. Es gäbe noch viel zu sagen. Natürlich auch sehr viel
               Günstiges. Doch das Günstige ist, wenn einmal, wie bei Ihnen, ein so beträchtliches
               Talentniveau angenomnen werden darf, allzu selbstverständlich. Doch möchte ich nicht
               verschweigen, dass Sie in der Behandlung des Verses nicht überall so sorgfältig
               gewesen sind, wie man es gerade von Ihnen hätte erwarten dürfen. Im Ganzen aber läuft
               die Sprache höchst gefällig. Und auch die ganze Atmosphäre der Komödie hat zuweilen
               einen ganz eigenen Reiz.\pend
           \pstart
           Und nun zur praktischen Seite der Frage. Meine Ansicht, dass dem \textcolor{green}{Stück}{}\ledrightnote{→\textcolor{green}{Die Komödianten}} bei einer event. Aufführung kein
               beträchtlicher Erfolg beschieden sein dürfte, komnt nicht in Betracht und selbst wenn
               Sie meine Meinung teilten, sollten Sie sich nicht abhalten lassen, alle die Wege zu
               beschreiten, die man eben als Verfas{[}s{]}er eines Stücks zu gehen
               hat. Alle Erfahrungen müssen zum {\pb}erstenmal gemacht
               werden und besser mit einem noch nicht ganz gelungenen, als mit Ihrem nächsten,
               wahrscheinlich bedeutenderen Stücke. Dazu kommt, dass man ja durchaus nicht
               voraussehen kann, ob wir uns nicht irren und ob Sie nicht gerade mit dieser Komödie
                  reu\strikeout{b}ssieren werden. Auch ist man ja nicht
               verpflichtet, ausschliesslich Meisterwerke zu schreiben. (Und wenn man verpflichtet
               wäre?) Also ich finde es nicht im Geringsten anstössig, selbst mit einem Stück
               hervorzutreten, an das man selbst nicht ganz glaubt. Das Wesentliche ist nur, \substVorne{}\textsuperscript{legen}\substDazwischen{}daſs\substHinten{} Sie selbst keinen allzugrossen Wert auf die innere Bedeutung Ihrer \textcolor{green}{Komödie}{}\ledrightnote{→\textcolor{green}{Die Komödianten}} legen und dass Sie dessen
               äussere Schicksale nicht allzu ernstnehmen sollten – auch wenn es sie in der
               Theaterwelt mit einem Schlage berühmt macht.\pend
           \pstart
           Ich hoffe bald wieder von Ihnen zu hören und grüsse Sie herzlich.{\\[\baselineskip]} Ihr sehr
               ergebener.,\pend
           \leftskip=0em{}\endnumbering\briefempfaengerindex{Mell, Max@\textsc{Mell, Max}!zzzSchnitzler, Arthur@\emph{von Arthur Schnitzler}!1906-10-261@{26. 10. 1906}|)be}\mylabel{h}  \normalsize

\doendnotes{C}
\bigskip
\vfill

\clearpage

\footnotesize

\lohead{\textsc{register}}

% Definiere theindex-Environment komplett neu ohne reledmac
\makeatletter
\renewenvironment{theindex}{%
  \section*{\indexname}%
  \setlength{\parindent}{0pt}%
  \setlength{\parskip}{0pt plus 0.3pt}%
  \let\item\@idxitem
}{%
  \clearpage
}
\makeatother

\IfFileExists{\jobname-pw.ind}{\input{\jobname-pw.ind}}{}

\end{document}

      