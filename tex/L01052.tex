%% latex-korrekturansicht-vorspann.tex
%% Vorspann für die Korrekturansicht.
%% Lädt die gemeinsame Datei latex-vorspann.tex mit gesetztem Schalter.

\newif\ifkorrekturansicht
\korrekturansichttrue

\input{../tex-inputs/latex-vorspann}


               \section[Arthur Schnitzler an Richard Beer-Hofmann, 7. 7. 1900]{ Arthur Schnitzler an Richard Beer-Hofmann, 7. 7. 1900}\nopagebreak\mylabel{v}\rehead{ }\normalsize\beginnumbering\briefempfaengerindex{Beer-Hofmann, Richard@\textsc{Beer-Hofmann, Richard}!zzzSchnitzler, Arthur@\emph{von Arthur Schnitzler}!1900-07-071@{7. 7. 1900}|(be} \toendnotes[C]{\smallbreak\pagebreak[2]} \Standort{YCGL, MSS 31.}
\physDesc{Brief, 1 Blatt, 4 Seiten, Umschlag
\newline{}Handschrift: Bleistift, deutsche Kurrent\newline{}Versand: 1) Stempel: »\nobreak{}\oindex{Reichenau an der Rax@\textbf{Reichenau an der Rax}, \emph{Besiedelter Ort (A.BSO)}|pwk}Reichenau N.Ö., 8 \textcolor{gray}{7} 00\nobreak{}«.  2) Stempel: »\nobreak{}\oindex{Altaussee@\textbf{Altaussee}, \emph{http://www.geonames.org/ontologyA.ADM3}|pwk}{\pb}Alt-Aussee, 8 7 00\nobreak{}«. }\buchAbdrucke{\weitereDrucke{Arthur Schnitzler, Richard Beer-Hofmann: \emph{Briefwechsel 1891–1931}. Hg. Konstanze Fliedl. Wien, Zürich: \emph{Europaverlag} 1992, S. 147.} }\toendnotes[C]{\smallbreak}\pstart{}{\pb}Herrn \textsc{Dr. Richard}\pend{}\pstart{}\textsc{Beer-Hofmann}\pend{}\pstart{}\textsc{\textcolor{pink}{Altaussee}{}\ledrightnote{\textcolor{pink}{Altaussee}}.}\pend{}\pstart{}\textsc{\textcolor{pink}{Steiermark}{}\ledrightnote{\textcolor{pink}{Steiermark}}}.\pend{}{\bigskip}\pstart{}{\pb}lieber Richard,\pend\pstart
           Danke für den nachgeſandten Brief, hier die Revanche. Wie geht es Ihrer \textcolor{blue}{Frau}{}\ledrightnote{→\textcolor{blue}{Paula Beer-Hofmann}}? Schreiben Sie mir das
               hieher, \textcolor{pink}{Reichenau, Curhaus}{}\ledrightnote{\textcolor{pink}{Kurhaus Rudolfsbad}}. \textcolor{blue}{Paul}{}\ledrightnote{\textcolor{blue}{Paul Goldmann}} iſt mit dem 15. Auguſt, \textcolor{pink}{Innsbruck}{}\ledrightnote{\textcolor{pink}{Innsbruck}} einverſtanden, \textcolor{blue}{Kerr}{}\ledrightnote{\textcolor{blue}{Alfred Kerr}} wohl auch; wir könnten nun die Sache bald {\pb}endgiltig fixiren. Ich ſehe Sie wohl noch Anfang
               Auguſt, entweder in \textcolor{pink}{Iſchl}{}\ledrightnote{\textcolor{pink}{Bad Ischl}} oder in \textcolor{pink}{Auſſee}{}\ledrightnote{\textcolor{pink}{Bad Aussee}}; oder \textcolor{pink}{Salzburg}{}\ledrightnote{\textcolor{pink}{Salzburg}}. Hier
               bleibe ich wahrſcheinlich 10–14 Tage. Dann? – Die paar Tage zwiſchen \textcolor{pink}{Altauſſee}{}\ledrightnote{\textcolor{pink}{Altaussee}} und \textcolor{pink}{Reichenau}{}\ledrightnote{\textcolor{pink}{Reichenau an der Rax}} waren
               ganz anſprechend. (\textcolor{green}{Wir lieben die
                  Frauen, die uns gleichgiltig ſind}{}\ledrightnote{→\textcolor{green}{Liebelei. Schauspiel in drei Akten}}{ }\textsc{etc}.) – Ich entwerfe {\pb}immerfort an dem \textcolor{green}{Fünfactigen}{}\ledrightnote{→\textcolor{green}{Der Weg ins Freie. Roman}}
               herum. (\textcolor{green}{Die Entrüſteten}{}\ledrightnote{→\textcolor{green}{Der Weg ins Freie. Roman}} wird es
               nicht heißen, da bisher kein Entrüſteter drin vorko{\geminationm}t;
               der beſte Titel wäre eine Geſte, mit dem Begleitton: Tz, – aber nicht ſo jüdiſch, wie
               das letzte Capitel von \textcolor{green}{Georgs Tod}{}\ledrightnote{\textcolor{green}{Der Tod Georgs}}.) ((An dieſer
               Stelle wird der Co{\geminationm}entator unſres Briefwechſels
               irrſinnig werden.))\pend
           \pstart
           {\pb}Leben Sie wohl.\pend
           \pstart
           Von Herzen Ihr{\\[\baselineskip]}\spacefill\mbox{Arthur}\pend
           \leftskip=0em{}\pstart
           7. 7. 900.\pend
           \endnumbering\briefempfaengerindex{Beer-Hofmann, Richard@\textsc{Beer-Hofmann, Richard}!zzzSchnitzler, Arthur@\emph{von Arthur Schnitzler}!1900-07-071@{7. 7. 1900}|)be}\mylabel{h}  \normalsize

\doendnotes{C}
\bigskip
\vfill

\clearpage

\footnotesize

\lohead{\textsc{register}}

% Definiere theindex-Environment komplett neu ohne reledmac
\makeatletter
\renewenvironment{theindex}{%
  \section*{\indexname}%
  \setlength{\parindent}{0pt}%
  \setlength{\parskip}{0pt plus 0.3pt}%
  \let\item\@idxitem
}{%
  \clearpage
}
\makeatother

\IfFileExists{\jobname-pw.ind}{\input{\jobname-pw.ind}}{}

\end{document}

      