%% latex-korrekturansicht-vorspann.tex
%% Vorspann für die Korrekturansicht.
%% Lädt die gemeinsame Datei latex-vorspann.tex mit gesetztem Schalter.

\newif\ifkorrekturansicht
\korrekturansichttrue

\input{../tex-inputs/latex-vorspann}


               \section[Richard Beer-Hofmann an Arthur Schnitzler, 24. 9. 1895]{ Richard Beer-Hofmann an Arthur Schnitzler, 24. 9. 1895}\nopagebreak\mylabel{v}\rehead{ }\normalsize\beginnumbering\briefempfaengerindex{Schnitzler, Arthur@\textsc{Schnitzler, Arthur}!zzzBeer-Hofmann, Richard@\emph{von Richard Beer-Hofmann}!1895-09-242@{24. 9. 1895}|(be} \toendnotes[C]{\smallbreak\pagebreak[2]} \Standort{DLA, A:Schnitzler, HS.NZ85.1.5713, S. 43–48.}
\physDesc{maschinelle Abschrift
\newline{}Handschrift: Bleistift, deutsche Kurrent (\noindent{}geringfügige Korrekturen von unbekannter Hand)\newline{}Zusatz: Original nicht nachweisbar, vgl. die handschriftliche Angabe von
                                    \textcolor{blue}{Heinrich Schnitzler} auf der
                                 Mappe B 8 mit den restlichen Originalen der Briefe: »1 Brief
                                    (vom 24. 9. 1895) für \textcolor{blue}{Mutter} entnommen. \textcolor{blue}{H. S.}15. 8. 36.« \newline{}Editorischer Hinweis: Die Korrekturen wurden eingearbeitet. }\buchAbdrucke{\weitereDrucke{1) \emph{Literatur und Kunst.} In: \emph{Neue Zürcher Zeitung}, 2. 10. 1955, S. 4.} \weitereDrucke{2) \pwindex{Spiegelbild der Freundschaft@\emph{Spiegelbild der Freundschaft}|pwk}Olga Schnitzler: \emph{Spiegelbild der Freundschaft}. Salzburg: \emph{Residenz-Verlag} 1962, S. 141–142.} \weitereDrucke{3) Arthur Schnitzler, Richard Beer-Hofmann: \emph{Briefwechsel 1891–1931}. Hg. Konstanze Fliedl. Wien, Zürich: \emph{Europaverlag} 1992, S. 86–88.} }\toendnotes[C]{\smallbreak}\pstart
           \raggedleft{}{\pb}\label{K_L00493_1v}\edtext{24. 9. 95}{\lemma{\textnormal{\emph{24. 9. 95}}}\Cendnote{\textnormal{Am 26. 9. 1895 antwortet \textcolor{blue}{Schnitzler} auf den ersten Brief vom 24. 9. 1895, nicht aber auf diesen. Da er nicht im
                     Original erhalten ist, ist die Möglichkeit gegeben, dass er zu einem anderen
                     Zeitpunkt verfasst ist.}}}\label{K_L00493_1h}.\pend
           \pstart
           Lieber Arthur! Dies schreib ich Ihnen, im Boote liegend, während man
               mich zu einer Insel rudert, auf der ein Jupitertempel stand, aus dem der heilige \textcolor{blue}{Franciscus von {\pb}Assisi}{}\ledrightnote{\textcolor{blue}{Franz von Assisi}} – mein \textcolor{blue}{Franciscus}{}\ledrightnote{\textcolor{blue}{Franz von Assisi}} – ein Kloster
               gemacht hat. Zugleich lese ich in einem \label{K_L00493_2v}\edtext{\textcolor{green}{Buch}{}\ledrightnote{→\textcolor{green}{Anthologie lyrischer und epigrammatischer Dichtungen der alten Griechen}}}{\lemma{\textnormal{\emph{Buch}}}\Cendnote{\textnormal{\emph{\textcolor{green}{Anthologie lyrischer und epigrammatischer
                        Dichtungen der alten Griechen}}. Hg. \textcolor{blue}{Edmund Boesel}. Stuttgart: \emph{\textcolor{brown}{Philipp Reclam
                        jun.}}{ }{[}1884{]}.}}}\label{K_L00493_2h} wunderschöne Sachen – wie das \textcolor{green}{Buch}{}\ledrightnote{→\textcolor{green}{Anthologie lyrischer und epigrammatischer Dichtungen der alten Griechen}} aber heisst schreibe ich hier nicht, denn der Name
               könnte Ihnen entgleiten, und der \textcolor{blue}{B}{}\ledrightnote{\textcolor{blue}{Hermann Bahr}}{\dots} mittelst 3–4 Ausschrotartikeln es einem ruinieren und
               verekeln, aber es ist sehr schön. Im dritten Jahrhundert vor Christi Geburt schreibt
               ein Herr \textcolor{blue}{Posidippus}{}\ledrightnote{\textcolor{blue}{Poseidippos}} – ohne »\textcolor{green}{Märchen}{}\ledrightnote{\textcolor{green}{Das Märchen. Schauspiel in drei Aufzügen}}« und »\textcolor{green}{Elixire}{}\ledrightnote{\textcolor{green}{Die drei Elixire}}«-Schmerzen – heiter \label{K_L00493_3v}\edtext{konstatirend}{\lemma{\textnormal{\emph{konstatirend}}}\Cendnote{\textnormal{Das Gedicht findet sich
                  in \textcolor{blue}{Boesel}s \emph{\textcolor{green}{Anthologie}} auf den S. 298–299.}}}\label{K_L00493_3h}:\pend
           \stanza{}\textcolor{green}{»Wähne, \label{T_L00493_1v}\edtext{Philänis}{\lemma{\textnormal{\emph{Philänis}}}\Cendnote{\textnormal{Die
                        Abschrift hat »Philanis«, nach der gedruckten Zitatvorlage
                        korrigiert.}}}\label{T_L00493_1h}, nicht mich durch lockende Thränen zu täuschen!}{}\ledrightnote{→\textcolor{green}{[Wähne, Philänis…]}}\newverse{}\textcolor{green}{»Freilich, ich weiss ja, du liebst
                     inniger keinen als mich,}{}\ledrightnote{→\textcolor{green}{[Wähne, Philänis…]}}\newverse{}\textcolor{green}{»Keinen, – so lange du neben mir
                     liegst. Doch hat dich ein andrer,}{}\ledrightnote{→\textcolor{green}{[Wähne, Philänis…]}}\newverse{}\textcolor{green}{»Nun, so liebest du den inniger
                     wieder als mich.}{}\ledrightnote{→\textcolor{green}{[Wähne, Philänis…]}}«\stanzaend{}\pstart
           Sollten Ihnen \textcolor{blue}{Paul Hörne}{}\ledrightnote{\textcolor{blue}{Paul Horn}} die »\textcolor{green}{kleine Comödie}{}\ledrightnote{\textcolor{green}{Die kleine Komödie}}«, verheirathete Frauen mit dem Schmerz anständig
               zu sein, das »kleine Mädel« der »\textcolor{green}{Liebelei}{}\ledrightnote{\textcolor{green}{Liebelei. Schauspiel in drei Akten}}« (um
               Gotteswillen, wie ist die \textcolor{blue}{Sandrock}{}\ledrightnote{\textcolor{blue}{Adele Sandrock}} im ersten
               Akt?) und mir das Dienstmädchen im »\textcolor{green}{Kind}{}\ledrightnote{\textcolor{green}{Das Kind}}« (mit
               Unrecht, denn die schildere ich selbst ja nicht als hervorragend begehrenswert)
               vorwerfen, dann wer{\pb}den wir mit Ihnen sagen »lasst uns
               lächeln« und folgende schöne \label{K_L00493_4v}\edtext{Verse}{\lemma{\textnormal{\emph{Verse}}}\Cendnote{\textnormal{Das Gedicht findet sich in \textcolor{blue}{Boesel}s \emph{\textcolor{green}{Anthologie}} auf den S. 299–300.}}}\label{K_L00493_4h} zitieren:\pend
           \stanza{}\textcolor{green}{Statt hoffärtiger Frauen erwählen
                     wir lieber die Magd uns,}{}\ledrightnote{→\textcolor{green}{Vorzug der Magd vor der vornehmen Frau}}\newverse{}\textcolor{green}{Welche den täuschenden Schein
                     üppigen Tandes verschmäht.}{}\ledrightnote{→\textcolor{green}{Vorzug der Magd vor der vornehmen Frau}}\newverse{}\textcolor{green}{Jene, die Haut umduftet von Salböl,
                     schreitet mit Hochmuth}{}\ledrightnote{→\textcolor{green}{Vorzug der Magd vor der vornehmen Frau}}\newverse{}\textcolor{green}{Prunkend einher; und Gefahr bringt
                     es, ihr liebend zu nahn.}{}\ledrightnote{→\textcolor{green}{Vorzug der Magd vor der vornehmen Frau}} (\textcolor{green}{Liebelei}{}\ledrightnote{\textcolor{green}{Liebelei. Schauspiel in drei Akten}}) \newverse{}\textcolor{green}{Diese, geschmückt mit natürlichem
                     Reiz und Farbe, versagt dir}{}\ledrightnote{→\textcolor{green}{Vorzug der Magd vor der vornehmen Frau}}\newverse{}\textcolor{green}{Nimmer das Lager und heischt nimmer
                     ein köstlich Geschenk.}{}\ledrightnote{→\textcolor{green}{Vorzug der Magd vor der vornehmen Frau}}\newverse{}\textcolor{green}{Pyrrhus, ich ahme dir nach, du
                     edler Sohn des Achilleus,}{}\ledrightnote{→\textcolor{green}{Vorzug der Magd vor der vornehmen Frau}}\newverse{}\textcolor{green}{Der du Andromache nahmst an der
                     Hermione Statt.}{}\ledrightnote{→\textcolor{green}{Vorzug der Magd vor der vornehmen Frau}}«\stanzaend{}\pstart
           Das ist von \textcolor{blue}{Rufinus}{}\ledrightnote{\textcolor{blue}{Rufinus}}. »\label{K_L00493_5v}\edtext{\textcolor{green}{Zur Bestimmung der Lebenszeit des \textcolor{blue}{Rufinus}{}\ledrightnote{\textcolor{blue}{Rufinus}} fehlt uns jeder Anhalt.}{}\ledrightnote{→\textcolor{green}{[Wähne, Philänis…]}}}{\lemma{\textnormal{\emph{Zur … Anhalt.}}}\Cendnote{\textnormal{Zitat von
                  S. 247}}}\label{K_L00493_5h}« –\pend
           \pstart
           Ich war auf der Insel und wir fahren im Abendwind (man hat sechs geläutet) zurück.
               Die Insel ist herrlich. Seitdem ich \textcolor{pink}{Italien}{}\ledrightnote{\textcolor{pink}{Italien}} und
               solche Inseln wie die \textcolor{pink}{Borromäischen}{}\ledrightnote{\textcolor{pink}{Borromäische Inseln}} und die kenne,
               bewundere ich \textcolor{blue}{Boeklin}{}\ledrightnote{\textcolor{blue}{Arnold Böcklin}} weniger. Wie dumm waren nur
               die Anderen, dass sie mit solchen Augen solche Schönheiten nicht sahen. Ich will
               recht oft hieher, und in den Süden, man wird ein besserer Mensch hier, alles liegt so
               weit weg, als wenn wir es von grosser Höhe klein, und uns selbst fremd unter uns
               sehen würden. {\pb}Wie widerlich ist das Gesindel, das mit
               ungezieferhafter Unruhe uns zu Hause, in \textcolor{pink}{Wien}{}\ledrightnote{\textcolor{pink}{Wien}} wieder
               umwimmeln wird. Aber dies Jahr sollen die Recht behalten, die mich »arrogant« nennen.
               Ich will ihnen eine Arroganz »hinlegen« (so sagen doch die Herren, die Ihnen die Ehre
               erweisen Ihr \textcolor{green}{Stück}{}\ledrightnote{→\textcolor{green}{Liebelei. Schauspiel in drei Akten}} zu spielen),
               dass sie starr sein werden. Und meine Arroganz wird nur die sein allein zu sein
               »höflich und allein«. Auch ein Wahlspruch für den Verkehr mit Jenen. Ich denke mit
               vieler Freude auch an unser Beisammensein im Winter, und wenn wir dabei immer den
               Daumen in der hohlen Hand verbergen, »Tütü« machen, und »unberufen« sagen, und uns
               noch ängstigen tut uns vielleicht auch der Neid der Götter nichts. Heute macht die
               Tatsache, dass wir einander haben nur unser Leben schöner und wärmer, aber ich
               glaube, wenn wir einmal alt sein werden und sehr Vieles, an das wir jetzt glauben,
               weit weg von uns sein wird, werden wir einander noch viel mehr bedeuten. Aber das
               möcht ich gar nicht, dass es so kommt, {\pb}dass wir, wenn
               wir alt sind, nichts mehr haben als uns; wir sollen Greise sein mit wunderschönen
               hellen jungen Augen und seidenweichem weissen Haar, und \uuline{sehr} berühmt. So berühmt, dass sich Frauen rühmen, wenn ihre Mütter einmal
               unsere Geliebten waren, und junge Mädchen sich mühen sollen, um reizend zu erscheinen
               – und ich meine »reizend« wörtlich. Und weil wir Blumen lieb haben, und bis dahin
               auch den Wein lieben gelernt haben, kommen aus dem Süden täglich Körbe mit Obst und
               Wein und Blumen. Denn wer hinunterreist in den Süden wird an uns denken müssen, die
               wir, in einer Zeit, wo hässlich geschäftige Menschen lebten, die Reichtum und
               Anerkennung wollten und widerliche Literatur machten, die einzigen waren, die
               wussten, dass es Schönheit und Sonne und Liebe gibt, die nur genossen, und erkannt
               sein will, – nicht mehr. – Jetzt wird es aber ganz dunkel; gegen \textcolor{pink}{Riva}{}\ledrightnote{\textcolor{pink}{Riva del Garda}} zu liegt der See im Nebel, gegen \textcolor{pink}{Salò}{}\ledrightnote{\textcolor{pink}{Salò}} ist der Himmel noch rötlich, und gegen \textcolor{pink}{Cap Manerba}{}\ledrightnote{\textcolor{pink}{Manerba del Garda}} steht im grünlichen Abendhimmel eine zarte silberne
                  {\pb}Sichel. Der Ruderer setzt stark ein, weil die
               Nacht kommt und mit jedem Ruderschlag sprüht mirs feucht ins Gesicht. Unendlich schön
               ists, und es wäre mir sehr leid, wenn ich jetzt ertrinken müsste. – Adieu lieber
               Arthur und grüssen Sie mir auch die, die Sie lieb haben, und die ich nicht kenne. Und
               sie hat Sie wohl jetzt noch mehr lieb als sonst, wo Sie vielleicht am Thor des
               Berühmtseins stehen, und sie wird sehr viel Herzklopfen haben, wenn das Orchester die
               Schlusstakte spielen wird. Nicht wahr! – Herzlichst Ihr\pend
           \pstart \spacefill\mbox{R.}\pend{}\pstart
           \noindent{}Es ist finster.\pend
           \endnumbering\briefempfaengerindex{Schnitzler, Arthur@\textsc{Schnitzler, Arthur}!zzzBeer-Hofmann, Richard@\emph{von Richard Beer-Hofmann}!1895-09-242@{24. 9. 1895}|)be}\mylabel{h}  \normalsize

\doendnotes{C}
\bigskip
\vfill

\clearpage

\footnotesize

\lohead{\textsc{register}}

% Definiere theindex-Environment komplett neu ohne reledmac
\makeatletter
\renewenvironment{theindex}{%
  \section*{\indexname}%
  \setlength{\parindent}{0pt}%
  \setlength{\parskip}{0pt plus 0.3pt}%
  \let\item\@idxitem
}{%
  \clearpage
}
\makeatother

\IfFileExists{\jobname-pw.ind}{\input{\jobname-pw.ind}}{}

\end{document}

      