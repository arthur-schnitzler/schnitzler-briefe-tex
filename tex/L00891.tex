%% latex-korrekturansicht-vorspann.tex
%% Vorspann für die Korrekturansicht.
%% Lädt die gemeinsame Datei latex-vorspann.tex mit gesetztem Schalter.

\newif\ifkorrekturansicht
\korrekturansichttrue

\input{../tex-inputs/latex-vorspann}


               \section[Hugo von Hofmannsthal an Arthur Schnitzler, {[}17. 2. 1899{]}]{ Hugo von Hofmannsthal an Arthur Schnitzler, {[}17. 2. 1899{]}}\nopagebreak\mylabel{v}\rehead{ }\normalsize\beginnumbering\briefempfaengerindex{Schnitzler, Arthur@\textsc{Schnitzler, Arthur}!zzzHofmannsthal, Hugo von@\emph{von Hugo von Hofmannsthal}!1899-02-171@{{[}17. 2. 1899{]}}|(be} \toendnotes[C]{\smallbreak\pagebreak[2]} \Standort{CUL, Schnitzler, B 43.}
\physDesc{Brief, 1 Blatt, 4 Seiten
\newline{}Handschrift: schwarze Tinte, deutsche Kurrent
\newline{}Schnitzler: mit Bleistift datiert: »Feber 99« \newline{}Ordnung: mit Bleistift von unbekannter Hand nummeriert:
                                        »138« }\buchAbdrucke{\weitereDrucke{Hugo von Hofmannsthal, Arthur Schnitzler: \emph{Briefwechsel}. Hg. Therese Nickl und Heinrich Schnitzler. Frankfurt am Main: \emph{S. Fischer} 1964, S. 118.} }\toendnotes[C]{\smallbreak}\pstart
           \raggedleft{}{\pb}Freitag{ }Früh\pend
           \pstart
           lieber, ich höre von \textcolor{blue}{Roſenbaum}{}\ledrightnote{\textcolor{blue}{Richard Rosenbaum}} daſs \textcolor{blue}{Sonnenthal}{}\ledrightnote{\textcolor{blue}{Adolf von Sonnenthal}} auch den
                        \textcolor{green}{Henry}{}\ledrightnote{→\textcolor{green}{Der grüne Kakadu. Groteske in einem Akt}}{ }ſpielt, was ich
                    ſehr geſcheidt und richtig finde. Nur möchte ich doch nicht, daſs die
                    nachträgliche Folge davon wäre, daſs er auch nicht einmal die eine Rolle des \textcolor{green}{Kaufmanns}{}\ledrightnote{→\textcolor{green}{Die Hochzeit der Sobeide}} in meinen \textcolor{green}{Stücken}{}\ledrightnote{→\textcolor{green}{Die Hochzeit der Sobeide}{\newline}→\textcolor{green}{Der Abenteurer und die Sängerin oder Die Geschenke des Lebens}}{ }{\pb}lernen kann oder will, weil
                    ja auf dieſe Art der Abend immer mehr gefährdet würde. Ich meine alſo, daſs Sie
                    – wenn einmal Ihre \textcolor{green}{Proben}{}\ledrightnote{→\textcolor{green}{Der grüne Kakadu – Paracelsus – Die Gefährtin. Drei Einakter}} in
                    Gang ſind, nicht früher – bei \textcolor{blue}{ihm}{}\ledrightnote{→\textcolor{blue}{Adolf von Sonnenthal}} und \textcolor{blue}{Schlenther}{}\ledrightnote{\textcolor{blue}{Paul Schlenther}} dahin wirken
                    könnten, daſs {\pb}\textcolor{blue}{er}{}\ledrightnote{→\textcolor{blue}{Adolf von Sonnenthal}}{ }ſich bereit erklärt,
                    nach Ihrer \textcolor{green}{Premiere}{}\ledrightnote{→\textcolor{green}{Der grüne Kakadu – Paracelsus – Die Gefährtin. Drei Einakter}} nicht
                    plötzlich ermüdet zu ſein und ſicher die gar nicht anſtrengende Rolle, in der er
                    mir unentbehrlich ſcheint, zu übernehmen.\pend
           \pstart
           Herzlich Ihr{\\[\baselineskip]}\spacefill\mbox{Hugo}\pend
           \leftskip=0em{}\pstart
           \label{K_L00891_1v}\edtext{Samstag{ }\textcolor{pink}{Rebhuhn}{}\ledrightnote{\textcolor{pink}{Café Rebhuhn}}}{\lemma{\textnormal{\emph{Samstag Rebhuhn}}}\Cendnote{\textnormal{vgl. A. S.: \emph{Tagebuch}, 18. 2. 1899}}}\label{K_L00891_1h}!\pend
           \pstart
           {\pb}Ich möchte, ſolang ſich
                        kein greifbares Hindernis ſondern nur die allgemeine Indolenz
                        entgegenſtellt, natürlich an dem Datum des \label{K_L00891_2v}\edtext{11\textsuperscript{ten} März}{\lemma{\textnormal{\emph{11ten März}}}\Cendnote{\textnormal{Tatsächlich fand sie am
                                18. 3. 1899
                      statt.}}}\label{K_L00891_2h} feſthalten und dazu iſt
                        natürlich ſehr nöthig, daſs Ihre \textcolor{green}{Aufführung}{}\ledrightnote{→\textcolor{green}{Der grüne Kakadu – Paracelsus – Die Gefährtin. Drei Einakter}} nicht über den \label{K_L00891_3v}\edtext{25\textsuperscript{ten} dieſes}{\lemma{\textnormal{\emph{25ten dieſes}}}\Cendnote{\textnormal{Diese verzögerte
                            sich auf den 1. 3. 1899.}}}\label{K_L00891_3h} verzögert wird.\pend
           \endnumbering\briefempfaengerindex{Schnitzler, Arthur@\textsc{Schnitzler, Arthur}!zzzHofmannsthal, Hugo von@\emph{von Hugo von Hofmannsthal}!1899-02-171@{{[}17. 2. 1899{]}}|)be}\mylabel{h}  \normalsize

\doendnotes{C}
\bigskip
\vfill

\clearpage

\footnotesize

\lohead{\textsc{register}}

% Definiere theindex-Environment komplett neu ohne reledmac
\makeatletter
\renewenvironment{theindex}{%
  \section*{\indexname}%
  \setlength{\parindent}{0pt}%
  \setlength{\parskip}{0pt plus 0.3pt}%
  \let\item\@idxitem
}{%
  \clearpage
}
\makeatother

\IfFileExists{\jobname-pw.ind}{\input{\jobname-pw.ind}}{}

\end{document}

      