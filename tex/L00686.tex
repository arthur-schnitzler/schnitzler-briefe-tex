%% latex-korrekturansicht-vorspann.tex
%% Vorspann für die Korrekturansicht.
%% Lädt die gemeinsame Datei latex-vorspann.tex mit gesetztem Schalter.

\newif\ifkorrekturansicht
\korrekturansichttrue

\input{../tex-inputs/latex-vorspann}


               \section[Richard Beer-Hofmann an Arthur Schnitzler, 13. 6. 1897]{ Richard Beer-Hofmann an Arthur Schnitzler,
               13. 6. 1897}\nopagebreak\mylabel{v}\rehead{ }\normalsize\beginnumbering\briefempfaengerindex{Schnitzler, Arthur@\textsc{Schnitzler, Arthur}!zzzBeer-Hofmann, Richard@\emph{von Richard Beer-Hofmann}!1897-06-131@{13. 6. 1897}|(be} \toendnotes[C]{\smallbreak\pagebreak[2]} \Standort{CUL, Schnitzler, B 8.}
\physDesc{Brief, 3 Blätter, 9 Seiten
\newline{}Handschrift: blauer Buntstift, lateinische Kurrent\newline{}Ordnung: mit Bleistift von unbekannter Hand nummeriert: »99« }\buchAbdrucke{\weitereDrucke{Arthur Schnitzler, Richard Beer-Hofmann: \emph{Briefwechsel 1891–1931}. Hg. Konstanze Fliedl. Wien, Zürich: \emph{Europaverlag} 1992, S. 109–110.} }\toendnotes[C]{\smallbreak}\pstart
           \centering{}{\pb}\textcolor{pink}{Ischl}{}\ledrightnote{\textcolor{pink}{Bad Ischl}}\hspace*{1.5em}13/VI 97\pend
           \pstart
           Lieber Arthur, ich weiß noch gar nichts wegen \textcolor{pink}{Bayreuth}{}\ledrightnote{\textcolor{pink}{Bayreuth}}, und will mich nicht entschließen.\pend
           \pstart
           Ihr Brief ist wieder so unleserlich! An \uline{was} arbeiten
               Sie? An einem Stück – da Sie von Scenen sprechen aber soll das \strikeout{»}Unleser{\pb}liche »\textcolor{green}{Revolutionsstück}{}\ledrightnote{→\textcolor{green}{Der grüne Kakadu. Groteske in einem Akt}}« heißen?\pend
           \pstart
           Ob mich’s mit »\textcolor{green}{ahnungsvoller Gegenwart
                  ängstigt}{}\ledrightnote{→\textcolor{green}{Faust}}«? fragen Sie? In mir wird so Vieles jetzt Anders als es bis her war
               daß ich nicht weiß wie viel auf Rech{\pb}nung »\uline{davon}« zu setzen ist. Manchmal hab ich die
               Empfindung als würde ich im Herbst nicht »Vater« sondern »Großvater« wenn ich sehe
               wie kindisch und jung noch \textcolor{blue}{Paula}{}\ledrightnote{\textcolor{blue}{Paula Beer-Hofmann}} ist, und dann
               muß ich wieder {\pb}über mich lachen
               mit meiner Neigung die Dinge zu leicht oder zu schwer zu nehmen. Augenblicklich
               sitzen wir – das ist \textcolor{blue}{Paula}{}\ledrightnote{\textcolor{blue}{Paula Beer-Hofmann}}, und ich, und die \textcolor{blue}{ko{\geminationm}ende
                  Generation}{}\ledrightnote{→\textcolor{blue}{Mirjam Beer-Hofmann}} und Flirt der bald sechs Jahre {\pb}alt wird – es gibt Hunde die
               achtzehn werden – in einem kleinen Lusthaus das man eigens für uns zurechtgezi{\geminationm}ert hat. Unter uns sehen wir die Strasse, und dann die
               Bahn, und dann die \textcolor{pink}{Traun}{}\ledrightnote{\textcolor{pink}{Traun}} und drüben wieder die
               Straße.\pend
           \pstart
           Ich scheine recht nervös {\pb}zu sein,
               oder sonst was, so sehr impressioniren mich jetzt gleichgiltige Dinge. Ich glaube
               manchmal daß ganz alte gute Leute, die bald sterben müßen diese leichte Rührung und
               Zärtlichkeit bei todten Dingen – wie Bäumen und {\pb}Straßen, und Flüßen haben; wie ich
               dazu ko{\geminationm}e weiß ich nicht. Oder ist am Ende doch daran
               schuld daß ich weiß, daß jetzt das im Werden ist was uns – oder mich – überleben und
               begraben soll. {\pb}Am Ende fängt mit
               jedem Kinderhaben doch ein unbewußtes Abdanken und Resigniren an; oder spüren wir daß
               wir nun überflüssig sind nachdem etwas von uns in {\pb}Anderem weiter lebt.\pend
           \pstart
           Wann müßen Sie eigentlich wieder nach \textcolor{pink}{Wien}{}\ledrightnote{\textcolor{pink}{Wien}} zurück?
               Ich muß wol zwischen 15{ }{\kaufmannsund}{ }20 Aug. auf einige Tage nach \textcolor{pink}{Wien}{}\ledrightnote{\textcolor{pink}{Wien}} »deswegen«. Wo werden Sie um diese Zeit sein? Wann ko{\geminationm}t voraussichtlich \textcolor{blue}{Paul}{}\ledrightnote{\textcolor{blue}{Paul Goldmann}} hieher? Grüßen {\pb}Sie \textcolor{blue}{Schwarzkopf}{}\ledrightnote{\textcolor{blue}{Gustav Schwarzkopf}} und \textcolor{blue}{Hugo}{}\ledrightnote{\textcolor{blue}{Hugo von Hofmannsthal}} von mir und
               schreiben Sie mir bald.\pend
           \pstart
           Ihr{\\[\baselineskip]}\spacefill\mbox{Richard}\pend
           \leftskip=0em{}\endnumbering\briefempfaengerindex{Schnitzler, Arthur@\textsc{Schnitzler, Arthur}!zzzBeer-Hofmann, Richard@\emph{von Richard Beer-Hofmann}!1897-06-131@{13. 6. 1897}|)be}\mylabel{h}  \normalsize

\doendnotes{C}
\bigskip
\vfill

\clearpage

\footnotesize

\lohead{\textsc{register}}

% Definiere theindex-Environment komplett neu ohne reledmac
\makeatletter
\renewenvironment{theindex}{%
  \section*{\indexname}%
  \setlength{\parindent}{0pt}%
  \setlength{\parskip}{0pt plus 0.3pt}%
  \let\item\@idxitem
}{%
  \clearpage
}
\makeatother

\IfFileExists{\jobname-pw.ind}{\input{\jobname-pw.ind}}{}

\end{document}

      