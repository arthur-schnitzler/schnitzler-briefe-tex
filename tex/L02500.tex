%% latex-korrekturansicht-vorspann.tex
%% Vorspann für die Korrekturansicht.
%% Lädt die gemeinsame Datei latex-vorspann.tex mit gesetztem Schalter.

\newif\ifkorrekturansicht
\korrekturansichttrue

\input{../tex-inputs/latex-vorspann}


               \section[Robert Adam an Arthur Schnitzler, 11. 5. 1928]{ Robert Adam an Arthur Schnitzler, 11. 5. 1928}\nopagebreak\mylabel{v}\rehead{ }\normalsize\beginnumbering\briefempfaengerindex{Schnitzler, Arthur@\textsc{Schnitzler, Arthur}!zzzAdam, Robert@\emph{von Robert Adam}!1928-05-111@{11. 5. 1928}|(be} \toendnotes[C]{\smallbreak\pagebreak[2]} \Standort{CUL, Schnitzler, B 1.}
\physDesc{Brief, 1 Blatt (Briefpapier mit Trauerrand), 4 Seiten
\newline{}Handschrift: schwarze Tinte, deutsche Kurrent
\newline{}Schnitzler: 1) mit Bleistift beschriftet: »\textsc{Adam}« 2) mit rotem Buntstift Vermerk: »\textcolor{green}{\textsc{Therese}}« und vereinzelte Unterstreichungen\newline{}Ordnung: mit Bleistift von unbekannter Hand nummeriert: »20« }\Standort{Wien, Österreichische Nationalbibliothek, Cod.ser. 52.268, 355 verso, 356.}
\physDesc{handschriftliche Abschrift
\newline{}Handschrift: schwarze Tinte, Gabelsberger Kurzschrift}\Standort{Wien, Österreichische Nationalbibliothek, Cod.ser. 52.268, 355 verso, 356.}
\physDesc{maschinelle Abschrift
\newline{}Schreibmaschine}\toendnotes[C]{\smallbreak}\pstart
           \raggedleft{}{\pb}\textcolor{pink}{Wien}{}\ledrightnote{\textcolor{pink}{Wien}}, am 11. Mai 1928\pend
           \pstart{}Hochverehrter Herr Doktor!\pend\pstart
           Ich vermute, daß Sie nunmehr von Ihrer Reiſe in Gegenden, zu denen auch mich ſeit
                    Jahren eine in meine ſtändigen Lektüre wurzelnde, noch unerfüllbare Sehnſucht
                    oder Neugier lockt, von den Erdbeben unbetroffen zurückgekehrt ſind, und will
                    Ihnen für zwei Dinge danken.\pend
           \pstart
           Vorerſt für Ihren \textcolor{green}{Roman}{}\ledrightnote{→\textcolor{green}{Therese. Chronik eines Frauenlebens}}, den
                    ich in der freien Zeit, die mir meine jetzt grauſam-anstrengende Amtstätigkeit
                    ließ, mit herzhafter Freude und bewunderndem Schauer geleſen habe. {\pb}Ich habe natürlich Ihre Thereſe
                    gekannt, wenn auch nicht unter dieſem Namen; ich kannte ſie unter mancherlei
                    Geſtalten, von Kindheit auf, als ſie um mich bemüht war – damals hieß ſie vor
                    allem Fräulein Joſefine –, und ſpäterhin, als ich, ein junger Menſch, um ſie
                    bemüht war, im \textcolor{pink}{Volksgarten}{}\ledrightnote{\textcolor{pink}{Volksgarten}}, im \textcolor{pink}{Prater}{}\ledrightnote{\textcolor{pink}{Prater}}, in \textcolor{pink}{Schönbrunn}{}\ledrightnote{\textcolor{pink}{Schloß Schönbrunn}}
                    und auch im \textcolor{pink}{Luxembourg}{}\ledrightnote{\textcolor{pink}{Jardin du Luxembourg}}, und ſchließlich iſt
                    ſie mir oft bei Gericht entgegengetreten. Aber in welch wunderbar-exakte
                    einfache Chronik haben Sie den furchtbar-troſtloſen Lebenslauf dieser
                    ſympathiſchen Alltagskreatur zuſammengefaßt! Ich kenne nur noch ein Buch, das,
                    wie Ihr \textcolor{blue}{Schopenhauer}{}\ledrightnote{\textcolor{blue}{Arthur Schopenhauer}}iſches,die unendliche
                    Troſt- und Fruchtloſigkeit des Menſchendaſeins (\textsc{\label{K_L02500_1v}\edtext{tat twam asi}{\lemma{\textnormal{\emph{tat twam asi}}}\Cendnote{\textnormal{»Das bist Du!«, wie \textcolor{blue}{Schopenhauer} den Satz aus den \emph{\textcolor{green}{Upanishaden}} übersetzte.}}}\label{K_L02500_1h}}) im Aufrollen der Qual eines endloſen Einzelſchickſals aufzeigt: \textsc{\textcolor{green}{Une Vie}{}\ledrightnote{\textcolor{green}{Ein Leben}}}.\pend
           \pstart
           {\pb}Nur der Juriſt in mir, dem alles
                    Menſchliche nur Tatbestand iſt, fühlt ſich nicht gleich befriedigt: denn er
                    ſchüttelt darüber den Kopf, daß \textcolor{green}{Thereſens}{}\ledrightnote{→\textcolor{green}{Therese. Chronik eines Frauenlebens}} böſer Bub ganz ohne Vormund auskommen muß – trotz der gut
                    funktionierenden \textcolor{pink}{Wien}{}\ledrightnote{\textcolor{pink}{Wien}}er
                    Vormundſchaftsgerichte –, und auch die Altersgrenze von ſechzehn Jahren (auf
                    S. 277) will ihm nicht gefallen. Aber dieſe kleinlichen Bedenken der Juriſten
                    haben einem großen Kunstwerk gegenüber, wie Ihr \textcolor{green}{Roman}{}\ledrightnote{→\textcolor{green}{Therese. Chronik eines Frauenlebens}} es iſt, wirklich nichts zu beſagen.\pend
           \pstart
           Und dann danke ich Ihnen herzlich für die Mühe, die Sie ſich mit der Lektüre
                    meiner korpulenten \textcolor{green}{Komödie}{}\ledrightnote{→\textcolor{green}{Märchenkomödie}}
                    gemacht haben, und für Ihren liebenswürdigen kritiſierenden Brief. Ich bin für
                    die Mängel meiner Arbeit keineswegs blind. Als einen ihrer Hauptfehler ſehe ich
                    es an, daß der gedankliche Aufbau in einer theaterwidrigen und abſtruſen Szene –
                    der Wanderung durch das Gehirn {\pb}und
                    Unterbewußtſein in’s Tranſzendente – gipfelt, während der Höhepunkt des äußeren
                    Geſchehens, der Sieg der Revolu\damage{tio}n, ganz gegen den Schluß verſchoben iſt, ſodaß Inkongruenz und
                    Unſymmetrie beſtehen. Auch die unwillkürliche Annäherung an den von mir zwar
                    geehrten, aber tief perhorreſzierten \textcolor{blue}{Ibſen}{}\ledrightnote{\textcolor{blue}{Henrik Ibsen}}
                    iſt mir ſehr unangenehm und für die Erſchaffung dieſer unverzeihlichen Liga
                    möchte ich mich am liebſten, wenn’s nicht ohnedies zu ſpät wäre, ſelbſt
                    prügeln.\pend
           \pstart
           Hoffentlich flicht ſich meine nächſte Arbeit um einen weniger abſurden Stoff. Es
                    iſt ſchrecklich, daß man Stoffe nicht wählen kann.\pend
           \pstart
           Mit den beſten Grüßen und Empfehlungen\hspace*{1.5em}Ihr\pend
           \pstart
           tief ergebener{\\[\baselineskip]}\spacefill\mbox{D\textsuperscript{r}RAdam}\pend
           \leftskip=0em{}\endnumbering\briefempfaengerindex{Schnitzler, Arthur@\textsc{Schnitzler, Arthur}!zzzAdam, Robert@\emph{von Robert Adam}!1928-05-111@{11. 5. 1928}|)be}\mylabel{h}  \normalsize

\doendnotes{C}
\bigskip
\vfill

\clearpage

\footnotesize

\lohead{\textsc{register}}

% Definiere theindex-Environment komplett neu ohne reledmac
\makeatletter
\renewenvironment{theindex}{%
  \section*{\indexname}%
  \setlength{\parindent}{0pt}%
  \setlength{\parskip}{0pt plus 0.3pt}%
  \let\item\@idxitem
}{%
  \clearpage
}
\makeatother

\IfFileExists{\jobname-pw.ind}{\input{\jobname-pw.ind}}{}

\end{document}

      