%% latex-korrekturansicht-vorspann.tex
%% Vorspann für die Korrekturansicht.
%% Lädt die gemeinsame Datei latex-vorspann.tex mit gesetztem Schalter.

\newif\ifkorrekturansicht
\korrekturansichttrue

\input{../tex-inputs/latex-vorspann}


               \section[Hugo von Hofmannsthal an Arthur Schnitzler, 23. 10. {[}1902{]}]{ Hugo von Hofmannsthal an Arthur Schnitzler, 23. 10. {[}1902{]}}\nopagebreak\mylabel{v}\rehead{ }\normalsize\beginnumbering\briefempfaengerindex{Schnitzler, Arthur@\textsc{Schnitzler, Arthur}!zzzHofmannsthal, Hugo von@\emph{von Hugo von Hofmannsthal}!1902-10-231@{23. 10. {[}1902{]}}|(be} \toendnotes[C]{\smallbreak\pagebreak[2]} \Standort{CUL, Schnitzler, B 43.}
\physDesc{Brief, 1 Blatt, 4 Seiten
\newline{}Handschrift: schwarze Tinte, deutsche Kurrent
\newline{}Schnitzler: mit Bleistift die Jahreszahl ergänzt: »902« \newline{}Ordnung: 1) mit Bleistift von unbekannter Hand nummeriert: »\strikeout{204}« 2) mit Bleistift von unbekannter Hand nummeriert: »187«}\buchAbdrucke{\weitereDrucke{Hugo von Hofmannsthal, Arthur Schnitzler: \emph{Briefwechsel}. Hg. Therese Nickl und Heinrich Schnitzler. Frankfurt am Main: \emph{S. Fischer} 1964, S. 162–163.} }\toendnotes[C]{\smallbreak}\pstart
           \raggedleft{}{\pb}23 X\hspace*{1.5em}\textcolor{pink}{Rom}{}\ledrightnote{\textcolor{pink}{Rom}}.\pend
           \pstart
           lieber, ich danke Ihnen herzlich für Ihre Karte und noch mehr für
               den frühern lieben und guten Brief, der mir damals in einem Moment, wo mich ſelbſt
                  \textcolor{blue}{Goethe}{}\ledrightnote{\textcolor{blue}{Johann Wolfgang von Goethe}} im Stich gelaſſen hatte, ungemein wohl
               gethan hat. Ich bin die erſten 14 Tage hier in einer ſinnloſen Depreſſion und
               Hilfloſigkeit herum{\pb}gelaufen.
               Plötzlich am morgen des 15\textsuperscript{ten}, hab ich gefühlt daſs etwas in mir da iſt. Und zwar nicht das »\textcolor{green}{Leben ein Traum}{}\ledrightnote{\textcolor{green}{Der Turm. Ein Trauerspiel}}«, nicht die \textcolor{green}{Elektra}{}\ledrightnote{\textcolor{green}{Elektra. Tragödie in einem Aufzug}}, ſondern ein anderer \textcolor{green}{Stoff}{}\ledrightnote{→\textcolor{green}{Das gerettete Venedig}} den ich mir einmal flüchtig zurechtgelegt hatte,
               gleichfalls \strikeout{h} nach einem ältern Vorbild. Seither hab
               ich meinen Arbeitstiſch, {\pb}der je
               nach dem Wetter entweder auf dem flachen Dach oder in meinem Zimmer ſteht, kaum mehr
               viel verlaſſen und heute den erſten \textcolor{green}{Act}{}\ledrightnote{→\textcolor{green}{Das gerettete Venedig. Trauerspiel in fünf Aufzügen}}, den weitaus längſten, mit 695 Verſen abgeſchloſſen.\pend
           \pstart
           Kommt von außen nichts Schlimmes, ſo glaub ich faſt ſicher gegen Ende
                  November mit dem \textcolor{green}{Stück}{}\ledrightnote{→\textcolor{green}{Das gerettete Venedig. Trauerspiel in fünf Aufzügen}}
               fertig zu {\pb}ſein. Laſſen Sie mich
               nicht ohne einige Nachricht, auch über Ihre Arbeit. In ſolchen glücklicheren Tagen
               empfinde ich das freundliche ſolcher lieber Briefe doppelt ſtark. Von Herzen Ihr\pend
           \pstart \spacefill\mbox{Hugo}\pend{}\pstart
           \noindent{}P. S. Wir müſſen wieder eine Radtour zuſammen machen!\pend
           \pstart
           \numberlinefalse{}\centering{}–\numberlinetrue{}\pend
           \pstart
           \noindent{}\textcolor{brown}{Eiſenſtein}{}\ledrightnote{\textcolor{brown}{J. Eisenstein {\kaufmannsund} Co.}} wird das Exemplar »\textcolor{green}{Tod d. T.}{}\ledrightnote{\textcolor{green}{Der Tod des Tizian}}« an Sie ſchicken!!\pend
           \endnumbering\briefempfaengerindex{Schnitzler, Arthur@\textsc{Schnitzler, Arthur}!zzzHofmannsthal, Hugo von@\emph{von Hugo von Hofmannsthal}!1902-10-231@{23. 10. {[}1902{]}}|)be}\mylabel{h}  \normalsize

\doendnotes{C}
\bigskip
\vfill

\clearpage

\footnotesize

\lohead{\textsc{register}}

% Definiere theindex-Environment komplett neu ohne reledmac
\makeatletter
\renewenvironment{theindex}{%
  \section*{\indexname}%
  \setlength{\parindent}{0pt}%
  \setlength{\parskip}{0pt plus 0.3pt}%
  \let\item\@idxitem
}{%
  \clearpage
}
\makeatother

\IfFileExists{\jobname-pw.ind}{\input{\jobname-pw.ind}}{}

\end{document}

      