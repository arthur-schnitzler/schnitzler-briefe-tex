%% latex-korrekturansicht-vorspann.tex
%% Vorspann für die Korrekturansicht.
%% Lädt die gemeinsame Datei latex-vorspann.tex mit gesetztem Schalter.

\newif\ifkorrekturansicht
\korrekturansichttrue

\input{../tex-inputs/latex-vorspann}


               \section[Arthur Schnitzler an Richard Beer-Hofmann, 26. 4. 1904]{ Arthur Schnitzler an Richard Beer-Hofmann, 26. 4. 1904}\nopagebreak\mylabel{v}\rehead{ }\normalsize\beginnumbering\briefempfaengerindex{Beer-Hofmann, Richard@\textsc{Beer-Hofmann, Richard}!zzzSchnitzler, Arthur@\emph{von Arthur Schnitzler}!1904-04-262@{26. 4. 1904}|(be} \toendnotes[C]{\smallbreak\pagebreak[2]} \Standort{YCGL, MSS 31.}
\physDesc{Brief, 1 Blatt, 3 Seiten, Umschlag
\newline{}Handschrift: Bleistift, deutsche Kurrent\newline{}Versand: 1) Stempel: »\nobreak{}\oindex{XVIII., Waehring@\textbf{XVIII., Währing}, \emph{Bezirk (A.BZK)}|pwk}18 Wien 110, 26. 4. 04, 11–12V\nobreak{}«.  2) Stempel: »\nobreak{}\oindex{Rodaun@\textbf{Rodaun}, \emph{Teil eines besiedelten Ortes (A.BSOX)}|pwk}{\pb}Rodaun, {[}2{]}\textcolor{gray}{7}{[}. 4. 04{]}, V\nobreak{}«. }\toendnotes[C]{\smallbreak}\pstart{}{\pb}Herrn \textsc{Dr Richard Beer-Hofmann
                  }\pend{}\pstart{}\textcolor{pink}{\textsc{Rodaun}}{}\ledrightnote{\textcolor{pink}{Rodaun}}\pend{}\pstart{}\textsc{\textcolor{pink}{Liesinger Straße 2}{}\ledrightnote{\textcolor{pink}{Liesingerstraße}}.}\pend{}{\bigskip}\pstart
           \raggedleft{}{\pb}Dinſtg, 26. 4. 904\pend
           \pstart
           lieber Richard, aus einer Karte \textcolor{blue}{Hugo}{}\ledrightnote{\textcolor{blue}{Hugo von Hofmannsthal}}s vom \textcolor{pink}{Semmering}{}\ledrightnote{\textcolor{pink}{Semmering}} entnehme ich daſs er
               die meine nicht erhalten hat. Dieſe meine Karte ſchlug ein Rendezvous für \label{K_L01395_1v}\edtext{Mittwoch Abend}{\lemma{\textnormal{\emph{Mittwoch Abend}}}\Cendnote{\textnormal{siehe A. S.: \emph{Tagebuch}, 27. 4. 1904}}}\label{K_L01395_1h}{ }\textcolor{pink}{Hietzing \textsc{Kuffner}}{}\ledrightnote{\textcolor{pink}{Ottakringer Bräu}} vor und bat ihn, das {\pb}auch Ihnen mitzutheilen.
               Es wär mir, \textsc{resp} uns beiden \textcolor{blue}{Olga}{}\ledrightnote{\textcolor{blue}{Olga Schnitzler}} u mir ſehr lieb, Sie Beide noch vor unſerer Abreiſe zu ſehen. \substVorne{}\textsuperscript{\textcolor{gray}{Jed}}\substDazwischen{}W\substHinten{}ir werden alſo jedenfalls in \textcolor{pink}{Hietzing}{}\ledrightnote{\textcolor{pink}{XIII., Hietzing}} ſein.
               (Auch \textcolor{blue}{Bahr}{}\ledrightnote{\textcolor{blue}{Hermann Bahr}} hatt’ ich geſchrieben.)\pend
           \pstart
           {\pb}Herzlich{\\[\baselineskip]}Ihr{\\[\baselineskip]}\spacefill\mbox{Arthur}\pend
           \leftskip=0em{}\endnumbering\briefempfaengerindex{Beer-Hofmann, Richard@\textsc{Beer-Hofmann, Richard}!zzzSchnitzler, Arthur@\emph{von Arthur Schnitzler}!1904-04-262@{26. 4. 1904}|)be}\mylabel{h}  \normalsize

\doendnotes{C}
\bigskip
\vfill

\clearpage

\footnotesize

\lohead{\textsc{register}}

% Definiere theindex-Environment komplett neu ohne reledmac
\makeatletter
\renewenvironment{theindex}{%
  \section*{\indexname}%
  \setlength{\parindent}{0pt}%
  \setlength{\parskip}{0pt plus 0.3pt}%
  \let\item\@idxitem
}{%
  \clearpage
}
\makeatother

\IfFileExists{\jobname-pw.ind}{\input{\jobname-pw.ind}}{}

\end{document}

      