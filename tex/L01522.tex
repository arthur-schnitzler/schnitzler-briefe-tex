%% latex-korrekturansicht-vorspann.tex
%% Vorspann für die Korrekturansicht.
%% Lädt die gemeinsame Datei latex-vorspann.tex mit gesetztem Schalter.

\newif\ifkorrekturansicht
\korrekturansichttrue

\input{../tex-inputs/latex-vorspann}


               \section[Hugo von Hofmannsthal an Arthur Schnitzler, {[}1. 6. 1905{]}]{ Hugo von Hofmannsthal an Arthur Schnitzler, {[}1. 6. 1905{]}}\nopagebreak\mylabel{v}\rehead{ }\normalsize\beginnumbering\briefempfaengerindex{Schnitzler, Arthur@\textsc{Schnitzler, Arthur}!zzzHofmannsthal, Hugo von@\emph{von Hugo von Hofmannsthal}!1905-06-011@{{[}1. 6. 1905{]}}|(be} \toendnotes[C]{\smallbreak\pagebreak[2]} \Standort{CUL, Schnitzler, B 43.}
\physDesc{Brief, 1 Blatt, 1 Seite
\newline{}Handschrift: schwarze Tinte, deutsche Kurrent
\newline{}Schnitzler: mit Bleistift datiert: »Juni 905« \newline{}Ordnung: 1) mit Bleistift von unbekannter Hand nummeriert:
                                    »252« 2) mit Bleistift von unbekannter Hand nummeriert: »254a«}\buchAbdrucke{\weitereDrucke{Hugo von Hofmannsthal, Arthur Schnitzler: \emph{Briefwechsel}. Hg. Therese Nickl und Heinrich Schnitzler. Frankfurt am Main: \emph{S. Fischer} 1964, S. 211.} }\toendnotes[C]{\smallbreak}\pstart
           \raggedleft{}{\pb}Do{\geminationn}erstg\pend
           \pstart
           Müſſen ausgerechnet Samstag{ }\label{K_L01522_1v}\edtext{\textcolor{green}{Sommernachtstraum}{}\ledrightnote{\textcolor{green}{Ein Sommernachtstraum}}}{\lemma{\textnormal{\emph{Sommernachtstraum}}}\Cendnote{\textnormal{Sie besuchten ein Gastspiel des \emph{\textcolor{brown}{Kleinen}} und des \emph{\textcolor{brown}{Neuen
                     Theaters}} im \textcolor{pink}{Theater an der Wien}.}}}\label{K_L01522_1h}
               gehen. Erklärung mündlich. Erbitten morgen Freitag Depeſche ob \textsc{rendezvous}{ }7\textsuperscript{h} morgen Freitag möglich. Andernfalls Montag??\pend
           \pstart \spacefill\mbox{Hugo.}\pend{}\endnumbering\briefempfaengerindex{Schnitzler, Arthur@\textsc{Schnitzler, Arthur}!zzzHofmannsthal, Hugo von@\emph{von Hugo von Hofmannsthal}!1905-06-011@{{[}1. 6. 1905{]}}|)be}\mylabel{h}  \normalsize

\doendnotes{C}
\bigskip
\vfill

\clearpage

\footnotesize

\lohead{\textsc{register}}

% Definiere theindex-Environment komplett neu ohne reledmac
\makeatletter
\renewenvironment{theindex}{%
  \section*{\indexname}%
  \setlength{\parindent}{0pt}%
  \setlength{\parskip}{0pt plus 0.3pt}%
  \let\item\@idxitem
}{%
  \clearpage
}
\makeatother

\IfFileExists{\jobname-pw.ind}{\input{\jobname-pw.ind}}{}

\end{document}

      