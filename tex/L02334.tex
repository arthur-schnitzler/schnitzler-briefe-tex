%% latex-korrekturansicht-vorspann.tex
%% Vorspann für die Korrekturansicht.
%% Lädt die gemeinsame Datei latex-vorspann.tex mit gesetztem Schalter.

\newif\ifkorrekturansicht
\korrekturansichttrue

\input{../tex-inputs/latex-vorspann}


               \section[Hugo Hofmannsthal an Arthur Schnitzler, 23. 1. 1920]{ Hugo Hofmannsthal an Arthur Schnitzler, 23. 1. 1920}\nopagebreak\mylabel{v}\rehead{ }\normalsize\beginnumbering\briefempfaengerindex{Schnitzler, Arthur@\textsc{Schnitzler, Arthur}!zzzHofmannsthal, Hugo von@\emph{von Hugo von Hofmannsthal}!1920-01-231@{23. 1. 1920}|(be} \toendnotes[C]{\smallbreak\pagebreak[2]} \Standort{CUL, Schnitzler, B 43.}
\physDesc{Brief, 1 Blatt, 3 Seiten
\newline{}Handschrift: schwarze Tinte, deutsche Kurrent\newline{}Ordnung: 1) mit Bleistift von \textcolor{blue}{Frieda Pollak} (?) mit dem Buchstaben »A« (Abgeschrieben/Abschrift) gekennzeichnet 2) mit Bleistift von unbekannter Hand nummeriert: »\strikeout{264}«3) mit Bleistift von unbekannter Hand nummeriert: »362«}\buchAbdrucke{\weitereDrucke{Hugo von Hofmannsthal, Arthur Schnitzler: \emph{Briefwechsel}. Hg. Therese Nickl und Heinrich Schnitzler. Frankfurt am Main: \emph{S. Fischer} 1964, S. 290.} }\toendnotes[C]{\smallbreak}\pstart
           \raggedleft{}{\pb}Freitag 23 I 20.\pend
           \pstart{}mein lieber Arthur\pend\pstart
           \label{K_L02334_1v}\edtext{neulich}{\lemma{\textnormal{\emph{neulich}}}\Cendnote{\textnormal{siehe A. S.: \emph{Tagebuch}, 14. 1. 1920}}}\label{K_L02334_1h}, in einer
               ängſtlichen Stunde, war mir ſo ſehr woltuend, Ihre Sti{\geminationm}e zu hören und Ihren Rat zu empfangen.\hspace*{1.5em}Die vieljährige Zuſa{\geminationm}engehörigkeit iſt doch ein ſo großes Wirkliches.\hspace*{1.5em}– Wie nahe war mir in dieſem Augenblick der Tag vor
               20 Jahren, das \label{K_L02334_2v}\edtext{Unglück}{\lemma{\textnormal{\emph{Unglück}}}\Cendnote{\textnormal{Am
                     18. 3. 1899 starb \textcolor{blue}{Marie
                     Reinhard}; am gleichen Tag hatte \emph{\textcolor{green}{Die Hochzeit
                     der Sobeide}} Uraufführung.}}}\label{K_L02334_2h}, wodurch die erſte Aufführung meiner
               Stücke {\pb}mir für immer beſchattet
               wurde – auch das \textcolor{blue}{Weſen}{}\ledrightnote{→\textcolor{blue}{Marie Reinhard}}, das ich
               nie geſehen u. von dem ich doch ein unverlöſchliches Phantaſiebild in mir trage.\pend
           \pstart
           Lieber Arthur, ich ko{\geminationm}e demnächst vormittags zu Ihnen,
               melde mich vorher.\pend
           \pstart
           Bitte blättern Sie die Stelle im \textcolor{green}{Märchen}{}\ledrightnote{→\textcolor{green}{Die Frau ohne Schatten. Erzählung}} auf und ſchreiben Sie mir, wodurch Ihr Eindruck von \textcolor{green}{\textsc{Baraks}}{}\ledrightnote{→\textcolor{green}{Die Frau ohne Schatten. Erzählung}} phyſiſcher {\pb}Erſcheinung als
               einer widerwärtigen ſich ſo fixiert hat.\hspace*{1.5em}Ich überlas
               die Stelle, die mir vorſchwebte, fand ſie relativ harmlos, in groben epiſch
               primitiven Zügen: ein Maul wie ein Spalt – das heißt aber doch nicht: eine geſpaltene
               Lippe.\pend
           \pstart
           Ich würde es gerne retouchieren.\pend
           \pstart
           Von Herzen Ihr{\\[\baselineskip]}\spacefill\mbox{Hugo.}\pend
           \leftskip=0em{}\endnumbering\briefempfaengerindex{Schnitzler, Arthur@\textsc{Schnitzler, Arthur}!zzzHofmannsthal, Hugo von@\emph{von Hugo von Hofmannsthal}!1920-01-231@{23. 1. 1920}|)be}\mylabel{h}  \normalsize

\doendnotes{C}
\bigskip
\vfill

\clearpage

\footnotesize

\lohead{\textsc{register}}

% Definiere theindex-Environment komplett neu ohne reledmac
\makeatletter
\renewenvironment{theindex}{%
  \section*{\indexname}%
  \setlength{\parindent}{0pt}%
  \setlength{\parskip}{0pt plus 0.3pt}%
  \let\item\@idxitem
}{%
  \clearpage
}
\makeatother

\IfFileExists{\jobname-pw.ind}{\input{\jobname-pw.ind}}{}

\end{document}

      