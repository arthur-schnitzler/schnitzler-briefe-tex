%% latex-korrekturansicht-vorspann.tex
%% Vorspann für die Korrekturansicht.
%% Lädt die gemeinsame Datei latex-vorspann.tex mit gesetztem Schalter.

\newif\ifkorrekturansicht
\korrekturansichttrue

\input{../tex-inputs/latex-vorspann}


               \section[Arthur Schnitzler an Hugo von Hofmannsthal, 10. 11. 1910]{ Arthur Schnitzler an Hugo von Hofmannsthal, 10. 11. 1910}\nopagebreak\mylabel{v}\rehead{ }\normalsize\beginnumbering\briefempfaengerindex{Hofmannsthal, Hugo von@\textsc{Hofmannsthal, Hugo von}!zzzSchnitzler, Arthur@\emph{von Arthur Schnitzler}!1910-11-101@{10. 11. 1910}|(be} \toendnotes[C]{\smallbreak\pagebreak[2]} \Standort{FDH, Hs-30885,141.}
\physDesc{Brief, 1 Blatt, 2 Seiten
\newline{}Handschrift: schwarze Tinte, deutsche Kurrent}\buchAbdrucke{\weitereDrucke{Hugo von Hofmannsthal, Arthur Schnitzler: \emph{Briefwechsel}. Hg. Therese Nickl und Heinrich Schnitzler. Frankfurt am Main: \emph{S. Fischer} 1964, S. 259.} }\toendnotes[C]{\smallbreak}\pstart
           \noindent{}{\pb}\textcolor{gray}{\textbf{Dr. Arthur
                        Schnitzler}}\hfill 10. 11. 1910\pend
           \pstart
           \textcolor{gray}{\textbf{\textcolor{pink}{Wien XVIII.
                        Sternwartestrasse 71}{}\ledrightnote{\textcolor{pink}{Sternwartestraße}}}}\pend
           \pstart{}mein lieber Hugo,\pend\pstart
           Ihre guten Worte hätten auch ſchlimmeres wieder gut machen können! Nun aber hätt ich
               ein rechtes Bedürfnis Ihnen wieder die Hand zu drücken und mit Ihnen zu reden.
               Wolltet Ihr nicht einmal ganz gemütlich – vorläufig {\pb}ohne
                  \textcolor{green}{Vorleſung}{}\ledrightnote{→\textcolor{green}{Der Rosenkavalier}} – nur wir \textcolor{blue}{vier}{}\ledrightnote{→\textcolor{blue}{Olga Schnitzler}{\newline}→\textcolor{blue}{Gertrude von Hofmannsthal}} – noch vor dem \textcolor{green}{\textsc{Medardus}}{}\ledrightnote{\textcolor{green}{Der junge Medardus. Dramatische Historie in einem Vorspiel und fünf Aufzügen}} bei uns
               nachtmahlen? Wählen Sie einen \label{K_L01979_1v}\edtext{Abend}{\lemma{\textnormal{\emph{Abend}}}\Cendnote{\textnormal{siehe A. S.: \emph{Tagebuch}, 16. 11. 1910}}}\label{K_L01979_1h} (der um ½ 7 anfangen ka{\geminationn}) – anfangs nächſter Woche von Dinſtag ab;
               aber ſchreiben Sie rechtzeitig. (Die \textsc{\textcolor{green}{Med.}{}\ledrightnote{\textcolor{green}{Der junge Medardus. Dramatische Historie in einem Vorspiel und fünf Aufzügen}}-Première} iſt gewiſs noch nicht am 19.
               Vielleicht 23. od 24. November.)\pend
           \pstart
           Herzlichst{\\[\baselineskip]}Ihr{\\[\baselineskip]}\spacefill\mbox{Arthur.}\pend
           \leftskip=0em{}\endnumbering\briefempfaengerindex{Hofmannsthal, Hugo von@\textsc{Hofmannsthal, Hugo von}!zzzSchnitzler, Arthur@\emph{von Arthur Schnitzler}!1910-11-101@{10. 11. 1910}|)be}\mylabel{h}  \normalsize

\doendnotes{C}
\bigskip
\vfill

\clearpage

\footnotesize

\lohead{\textsc{register}}

% Definiere theindex-Environment komplett neu ohne reledmac
\makeatletter
\renewenvironment{theindex}{%
  \section*{\indexname}%
  \setlength{\parindent}{0pt}%
  \setlength{\parskip}{0pt plus 0.3pt}%
  \let\item\@idxitem
}{%
  \clearpage
}
\makeatother

\IfFileExists{\jobname-pw.ind}{\input{\jobname-pw.ind}}{}

\end{document}

      