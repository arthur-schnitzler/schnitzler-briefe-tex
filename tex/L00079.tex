%% latex-korrekturansicht-vorspann.tex
%% Vorspann für die Korrekturansicht.
%% Lädt die gemeinsame Datei latex-vorspann.tex mit gesetztem Schalter.

\newif\ifkorrekturansicht
\korrekturansichttrue

\input{../tex-inputs/latex-vorspann}


               \section[Arthur Schnitzler an Richard Beer-Hofmann, 11. 3. 1892]{ Arthur Schnitzler an Richard Beer-Hofmann, 11. 3. 1892}\nopagebreak\mylabel{v}\rehead{ }\normalsize\beginnumbering\briefempfaengerindex{Beer-Hofmann, Richard@\textsc{Beer-Hofmann, Richard}!zzzSchnitzler, Arthur@\emph{von Arthur Schnitzler}!1892-03-111@{11. 3. 1892}|(be} \toendnotes[C]{\smallbreak\pagebreak[2]} \Standort{YCGL, MSS 31.}
\physDesc{Brief, 2 Blätter, 8 Seiten, Umschlag
\newline{}Handschrift: schwarze Tinte, deutsche Kurrent\newline{}Versand: 1) Stempel: »\nobreak{}\oindex{I., Innere Stadt@\textbf{I., Innere Stadt}, \emph{Bezirk (A.BZK)}|pwk}Wien 1/1, 11 3 92, 7–8 N\nobreak{}«.  2) Stempel: »\nobreak{}\oindex{Opatija@\textbf{Opatija}, \emph{http://www.geonames.org/ontologyP.PPLA2}|pwk}Abbazia, 13{[}. 3.{]} 92\nobreak{}«. }\buchAbdrucke{\weitereDrucke{1) Arthur Schnitzler: \emph{Briefe 1875–1912}. Hg. Therese Nickl und Heinrich Schnitzler. Frankfurt am Main: \emph{S. Fischer} 1981, S. 121–122.} \weitereDrucke{2) Arthur Schnitzler: \emph{Briefe 1875–1912}. Hg. Therese Nickl und Heinrich Schnitzler. Frankfurt am Main: \emph{S. Fischer} 1981, S. 120–121.} \weitereDrucke{3) Arthur Schnitzler, Richard Beer-Hofmann: \emph{Briefwechsel 1891–1931}. Hg. Konstanze Fliedl. Wien, Zürich: \emph{Europaverlag} 1992, S. 34–35.} \weitereDrucke{4) Hermann Bahr, Arthur Schnitzler: \emph{Briefwechsel, Aufzeichnungen, Dokumente
                                (1891–1931)}. Hg. Kurt Ifkovits und Martin Anton Müller. Göttingen: \emph{Wallstein} 2018, S. 22–23.} }\toendnotes[C]{\smallbreak}\pstart{}{\pb}\textcolor{gray}{\textbf{\textit{\label{T_L00079-1v}\edtext{AS}{\lemma{\textnormal{\emph{AS}}}\Cendnote{\textnormal{rotes Wachssiegel}}}\label{T_L00079-1h}}}}\pend{}{\bigskip}\pstart{}{\pb}Herrn \textsc{Dr. Rich.
                            Beer-Hofm\damage{a}nn}\pend{}\pstart{}\textsc{\textcolor{pink}{Abbazia}{}\ledrightnote{\textcolor{pink}{Opatija}}}\pend{}\pstart{}\textsc{\textcolor{pink}{Pension Quisisana}{}\ledrightnote{\textcolor{pink}{Pension Quisisana}}}\pend{}{\bigskip}\pstart
           \raggedleft{}{\pb}\textcolor{pink}{Wien}{}\ledrightnote{\textcolor{pink}{Wien}}, 11. März 92.\pend
           \pstart{}Lieber Richard,\pend\pstart
           \textcolor{blue}{Kafka}{}\ledrightnote{\textcolor{blue}{Eduard Michael Kafka}} habe ich die letzten Tage nicht geſehn.
                    Das letzte Mal an unſerem \textcolor{brown}{Vereinsabend}{}\ledrightnote{→\textcolor{brown}{»Freie Bühne« Verein für moderne Literatur}}, der nur einen Lichtpunkt hatte: \textcolor{blue}{Bahr’s}{}\ledrightnote{\textcolor{blue}{Hermann Bahr}} »\label{K_L00079_1v}\edtext{\textcolor{green}{treue Adele}{}\ledrightnote{\textcolor{green}{Die treue Adele. Eine vergeßliche Geschichte}}}{\lemma{\textnormal{\emph{treue Adele}}}\Cendnote{\textnormal{\textcolor{blue}{Hermann Bahr}: \emph{\textcolor{green}{Die treue Adele. Eine vergeßliche
                                Geschichte}}. In: \emph{\textcolor{green}{Die
                                Gesellschaft}}, Jg. 5, Nr. 11, November 1889,
                            S. 1556–1564 (Erstausgabe in \emph{\textcolor{green}{Fin de
                            Siècle}}, S. 71–88).}}}\label{K_L00079_1h}« von \textcolor{blue}{Bahr}{}\ledrightnote{\textcolor{blue}{Hermann Bahr}} vorgeleſen. Er las entzückend. \textcolor{blue}{\textsc{Meixner}}{}\ledrightnote{\textcolor{blue}{Julius Meixner}} las Parabeln von \textcolor{blue}{Kafka}{}\ledrightnote{\textcolor{blue}{Eduard Michael Kafka}} und ein
                    Gedicht \textcolor{blue}{\textsc{Liliencron}}{}\ledrightnote{\textcolor{blue}{Detlev von Liliencron}}{ }ſehr ſchlecht vor. \textcolor{blue}{\textsc{Polland}}{}\ledrightnote{\textcolor{blue}{Max Pollandt}} das \textcolor{green}{Kaffehaus}{}\ledrightnote{\textcolor{green}{[Das Kaffeehaus]}} von \textcolor{blue}{\textsc{Salten}}{}\ledrightnote{\textcolor{blue}{Felix Salten}}, Gedichte von \textcolor{blue}{\textsc{Loris}}{}\ledrightnote{\textcolor{blue}{Hugo von Hofmannsthal}}, \textcolor{blue}{Korff}{}\ledrightnote{\textcolor{blue}{Heinrich von Korff}} u mir unbeſchreiblich
                    entſetzlich. Es iſt unmöglich, ſich von dieſer talentloſen Brüllerei einen
                    Begriff zu machen, we{\geminationn} man nicht dabei {\pb}war. –
                    Zum Schluſs wurde getanzt. Von mir nicht, bitte. – \pend
           \pstart
           \textcolor{blue}{\textsc{Blumenthal}}{}\ledrightnote{\textcolor{blue}{Oskar Blumenthal}} war hier, ich ſprach ihn. Er will Kürzungen und einige Aenderungen
                    am \textcolor{green}{Mährchen}{}\ledrightnote{\textcolor{green}{Das Märchen. Schauspiel in drei Aufzügen}}. Einiges wird ſich wohl thun
                    laſſen; ich habe mich ſchon daran gemacht, und die ſchöne Fremdheit, die mich
                    vom \textcolor{green}{Märchen}{}\ledrightnote{\textcolor{green}{Das Märchen. Schauspiel in drei Aufzügen}} bereits tre{\geminationn}t, läßt mich die
                    Dinge leichter vollbringen. Daß \textcolor{blue}{\textsc{Blumenthal}}{}\ledrightnote{\textcolor{blue}{Oskar Blumenthal}} auch den Titel des Stückes geändert haben möchte, iſt
                    Caeſarenwahnſinn. Es iſt ihm auch ſchon ſelbſt ein neuer eingefallen – er{\pb}ſchrecken Sie nicht – »Die Vergangenheit.« Erke{\geminationn}en Sie
                    ihn!? Und noch i{\geminationm}er läßt man die erſt- und
                    zweitgradigen frei herum laufen, die doch nur dazu da ſind, um den dritt und
                    viertgradigen das Leben zu vermießen. –\pend
           \pstart
           Geſtern hab ich mein neues \textcolor{green}{Stück}{}\ledrightnote{→\textcolor{green}{Familie}} begonnen.
                    Außerdem schreibe ich \textsc{slowly}, langſam an meiner \textcolor{green}{Novelle}{}\ledrightnote{→\textcolor{green}{Sterben. Novelle}}. –\pend
           \pstart
           \textcolor{brown}{\textsc{Fontane} (Verlag)}{}\ledrightnote{\textcolor{brown}{F. Fontane}} hat mich freundlichſt
                    erſucht, den \textcolor{green}{\textsc{Anatol-Cyclus}}{}\ledrightnote{\textcolor{green}{Anatol}} – \uline{nicht} einzuſenden, {\pb}»da ſie kaum die Zeit finden dürften, meiner Sa{\geminationm}lung einen ſorgfältigen u energiſchen Vertrieb
                    angedeihen zu laſſen \textsc{etc etc}«\pend
           \pstart
           – Aus den »\textsc{\textcolor{green}{Aveugles}{}\ledrightnote{\textcolor{green}{Die Blinden}}}« ſcheint wirklich was zu werden. Doch ſoll dazu weder Pantomime noch
                        \textcolor{green}{Abschiedsſouper}{}\ledrightnote{\textcolor{green}{Abschiedssouper}} gegeben werden, ſondern
                        »\textsc{\textcolor{green}{l’Intrus}{}\ledrightnote{\textcolor{green}{L’Intruse}}}«. – Zu den beiden ein Vortrag von \textcolor{blue}{\textsc{Bahr}}{}\ledrightnote{\textcolor{blue}{Hermann Bahr}}. Später ſoll ein Pantomimen u Luſtſpielabend arrangirt werden. Man
                    kam mit dem \label{K_L00079_2v}\edtext{\textsc{fait accompli}}{\lemma{\textnormal{\emph{fait accompli}}}\Cendnote{\textnormal{französisch: beschlossene Sache}}}\label{K_L00079_2h} zu uns, das {\pb}freilich meinen Beifall nicht hat. –\pend
           \pstart
           \textcolor{blue}{\textsc{Loris}}{}\ledrightnote{\textcolor{blue}{Hugo von Hofmannsthal}}{ }ſchreibt viel, \textcolor{blue}{\textsc{Salten}}{}\ledrightnote{\textcolor{blue}{Felix Salten}}{ }ſchreibt wenig. Die andern ſeh ich gar nicht; das \textcolor{pink}{\textsc{Café Griensteidl}}{}\ledrightnote{\textcolor{pink}{Café Griensteidl}} exiſtirt für mich nicht mehr. – \pend
           \pstart
           Ich leſe \textcolor{blue}{\textsc{Taine}}{}\ledrightnote{\textcolor{blue}{Hippolyte Taine}}, \textcolor{green}{\textsc{ancien régime}}{}\ledrightnote{\textcolor{green}{L’Ancien régime}}, \textcolor{blue}{\textsc{Du Prel}}{}\ledrightnote{\textcolor{blue}{Carl Du Prel}}, \textcolor{green}{Philoſophie der Myſtik}{}\ledrightnote{\textcolor{green}{Die Philosophie der Mystik}},
                        \textcolor{blue}{\textsc{Restif de la Bretonne}}{}\ledrightnote{\textcolor{blue}{Nicolas Rétif de la Bretonne}}, \textcolor{green}{\textsc{l’amour à 45 ans}}{}\ledrightnote{\textcolor{green}{Sara, ou L’amour à quarante-cinq ans}}, \textcolor{blue}{\textsc{Kretzer}}{}\ledrightnote{\textcolor{blue}{Max Kretzer}}, \textcolor{green}{die Betrogenen}{}\ledrightnote{\textcolor{green}{Die Betrogenen}}
                    u. a. – \pend
           \pstart
           Die Menſchen \textsc{enerviren} mich. Manche miſchen ſich in
                    meine Privatangelegenheiten, und nie{\pb}manden gehen ſie
                    an. Das Geſindel hat tauſend Augen für Vorfälle, dafür taube Ohren für Einfälle.
                    Aber mit der Zeit wird ſich die Menſchheit wohl »ausſchalten« laſſen, wie? Einen
                    Harfeniſten ka{\geminationn} man aus dem Hofe weiſen laſſen,
                        we{\geminationn} er einen mit ſeinem Geklimper quält; wer
                    aber befreit mich von den – andern? \pend
           \pstart
           Ich will verſuchen, ein Virtuoſe der Einſamkeit zu werden. Eines ſchönen Tages
                    werden alle Leute, die mich geniren, {\pb}nicht mehr daſein
                    – und werden es nicht einmal bemerken. So wollen wir die Unbequemen zu relativem
                    Tod verurtheilen: wir vom »großen Orden«! – Oder hätte Sie \textcolor{blue}{\textsc{Salten}}{}\ledrightnote{\textcolor{blue}{Felix Salten}} abreiſen laſſen, ohne Ihnen den großen Orden zu erläutern? –\pend
           \pstart
           Schreiben Sie mir bald, und möglichſt viel, es muſs doch ganz ſchön ſein, we{\geminationn}
                    man einmal wo anders iſt. Und dann, ſchreiben Sie – wir erwarten es, wir – vom
                    großen Orden. –\pend
           \pstart
           {\pb}Herzlichſt Ihr{\\[\baselineskip]}\spacefill\mbox{Arthur Sch}\pend
           \leftskip=0em{}\endnumbering\briefempfaengerindex{Beer-Hofmann, Richard@\textsc{Beer-Hofmann, Richard}!zzzSchnitzler, Arthur@\emph{von Arthur Schnitzler}!1892-03-111@{11. 3. 1892}|)be}\mylabel{h}  \normalsize

\doendnotes{C}
\bigskip
\vfill

\clearpage

\footnotesize

\lohead{\textsc{register}}

% Definiere theindex-Environment komplett neu ohne reledmac
\makeatletter
\renewenvironment{theindex}{%
  \section*{\indexname}%
  \setlength{\parindent}{0pt}%
  \setlength{\parskip}{0pt plus 0.3pt}%
  \let\item\@idxitem
}{%
  \clearpage
}
\makeatother

\IfFileExists{\jobname-pw.ind}{\input{\jobname-pw.ind}}{}

\end{document}

      