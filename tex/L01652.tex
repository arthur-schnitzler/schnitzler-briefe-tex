%% latex-korrekturansicht-vorspann.tex
%% Vorspann für die Korrekturansicht.
%% Lädt die gemeinsame Datei latex-vorspann.tex mit gesetztem Schalter.

\newif\ifkorrekturansicht
\korrekturansichttrue

\input{../tex-inputs/latex-vorspann}


               \section[Arthur Schnitzler an Richard Beer-Hofmann, {[}Mitte Januar 1907?{]}]{ Arthur Schnitzler an Richard Beer-Hofmann, {[}Mitte Januar 1907?{]}}\nopagebreak\mylabel{v}\rehead{ }\normalsize\beginnumbering\briefempfaengerindex{Beer-Hofmann, Richard@\textsc{Beer-Hofmann, Richard}!zzzSchnitzler, Arthur@\emph{von Arthur Schnitzler}!1907-01-151@{{[}Mitte Januar 1907?{]}}|(be} \toendnotes[C]{\smallbreak\pagebreak[2]} \Standort{YCGL, MSS 31.}
\physDesc{Brief, 1 Blatt, 2 Seiten, Umschlag
\newline{}Handschrift: Bleistift, deutsche Kurrent\newline{}Versand: ohne postalischen Übermittlungsvermerk }\toendnotes[C]{\smallbreak}\pstart{}{\pb}\textcolor{gray}{\textbf{Dr. Arthur Schnitzler}}\pend{}\pstart{}\textcolor{gray}{\textbf{\textcolor{pink}{Wien, XVIII. Spoettelgasse 7}{}\ledrightnote{\textcolor{pink}{Edmund-Weiß-Gasse}}.}}\pend{}{\bigskip}\pstart{}{\pb}Hrn \textsc{Dr. Richard
                     Beer-Hofmann}\pend{}\pstart{}\textcolor{pink}{Wien XVIII}{}\ledrightnote{\textcolor{pink}{XVIII., Währing}}\pend{}\pstart{}\textcolor{pink}{\textsc{Hasenauerstraße}}{}\ledrightnote{\textcolor{pink}{Hasenauerstraße}}\pend{}{\bigskip}\pstart
           \noindent{}{\pb}\textcolor{gray}{\textbf{Dr. Arthur Schnitzler}}{\\}\textcolor{gray}{\textbf{\textcolor{pink}{Wien, XVIII. Spoettelgasse 7}{}\ledrightnote{\textcolor{pink}{Edmund-Weiß-Gasse}}.}}\pend
           \pstart{}lieber Richard,\pend\pstart
           \label{K_L01652_1v}\edtext{Ende Feber}{\lemma{\textnormal{\emph{Ende Feber}}}\Cendnote{\textnormal{Die Veranstaltung fand bereits am 10. 2. 1907 statt.}}}\label{K_L01652_1h} ſoll ein
               Vortragsabend für die \uline{jüdiſchen Waiſen} ſtattfinden;
               es iſt der einzige Fall, in dem ich heuer zugeſagt habe; außer mir ſollen \textsc{\textcolor{blue}{Wasserma{\geminationn}}{}\ledrightnote{\textcolor{blue}{Jakob Wassermann}}} u \textcolor{blue}{\textsc{Salten}}{}\ledrightnote{\textcolor{blue}{Felix Salten}} leſen – vielleicht beſtimmt {\pb}Sie der gute Zweck
                  \label{K_L01652_2v}\edtext{mitzuthun}{\lemma{\textnormal{\emph{mitzuthun}}}\Cendnote{\textnormal{Das Korrespondenzstück ist undatiert. Da die bis zum [14. 1. 1907] mögliche Teilnahme \textcolor{blue}{Hofmannsthal}s keine Erwähnung findet, wird das
                  Korrespondenzstück danach angesiedelt. \textcolor{blue}{Beer-Hofmann} nahm teil.}}}\label{K_L01652_2h}.\pend
           \pstart
           Herzlichſt{\\[\baselineskip]}Ihr{\\[\baselineskip]}\spacefill\mbox{A.}\pend
           \leftskip=0em{}\endnumbering\briefempfaengerindex{Beer-Hofmann, Richard@\textsc{Beer-Hofmann, Richard}!zzzSchnitzler, Arthur@\emph{von Arthur Schnitzler}!1907-01-151@{{[}Mitte Januar 1907?{]}}|)be}\mylabel{h}  \normalsize

\doendnotes{C}
\bigskip
\vfill

\clearpage

\footnotesize

\lohead{\textsc{register}}

% Definiere theindex-Environment komplett neu ohne reledmac
\makeatletter
\renewenvironment{theindex}{%
  \section*{\indexname}%
  \setlength{\parindent}{0pt}%
  \setlength{\parskip}{0pt plus 0.3pt}%
  \let\item\@idxitem
}{%
  \clearpage
}
\makeatother

\IfFileExists{\jobname-pw.ind}{\input{\jobname-pw.ind}}{}

\end{document}

      