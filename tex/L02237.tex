%% latex-korrekturansicht-vorspann.tex
%% Vorspann für die Korrekturansicht.
%% Lädt die gemeinsame Datei latex-vorspann.tex mit gesetztem Schalter.

\newif\ifkorrekturansicht
\korrekturansichttrue

\input{../tex-inputs/latex-vorspann}


               \section[Arthur Schnitzler an Richard Beer-Hofmann, 20. 8. 1916]{ Arthur Schnitzler an Richard Beer-Hofmann, 20. 8. 1916}\nopagebreak\mylabel{v}\rehead{ }\normalsize\beginnumbering\briefempfaengerindex{Beer-Hofmann, Richard@\textsc{Beer-Hofmann, Richard}!zzzSchnitzler, Arthur@\emph{von Arthur Schnitzler}!1916-08-201@{20. 8. 1916}|(be} \toendnotes[C]{\smallbreak\pagebreak[2]} \Standort{YCGL, MSS 31.}
\physDesc{Kartenbrief
\newline{}Handschrift: Bleistift, deutsche Kurrent\newline{}Versand: Stempel: »\nobreak{}\oindex{Altaussee@\textbf{Altaussee}, \emph{http://www.geonames.org/ontologyA.ADM3}|pwk}Alt Aussee, 21. VIII. 16\nobreak{}«.  
\newline{}Beer-Hofmann: mit blauem Buntstift den Empfang vermerkt:
                                 »E« }\buchAbdrucke{\weitereDrucke{Arthur Schnitzler, Richard Beer-Hofmann: \emph{Briefwechsel 1891–1931}. Hg. Konstanze Fliedl. Wien, Zürich: \emph{Europaverlag} 1992, S. 222.} }\pstart{}{\pb}Abſ. \textsc{Schnitzler}\pend{}\pstart{}\textsc{\textcolor{pink}{Altaussee}{}\ledrightnote{\textcolor{pink}{Altaussee}}, \textcolor{pink}{Fischerndorf 79}{}\ledrightnote{\textcolor{pink}{Fischerndorf}}}\pend{}{\bigskip}\pstart{}\textsc{Herrn Doctor Richard Beer-Hofmann}\pend{}\pstart{}\textcolor{pink}{\textsc{Bad Ischl}}{}\ledrightnote{\textcolor{pink}{Bad Ischl}}\pend{}\pstart{}\textcolor{pink}{\textsc{Grazerstr 52}}{}\ledrightnote{\textcolor{pink}{Grazer Straße}}\pend{}{\bigskip}\pstart
           \raggedleft{}{\pb}\textsc{\textcolor{pink}{Altaussee}{}\ledrightnote{\textcolor{pink}{Altaussee}}, \textcolor{pink}{Fischerndorf 79}{}\ledrightnote{\textcolor{pink}{Fischerndorf}}}{\\}20. 8. 1916\pend
           \pstart
           lieber Richard, darf ich Sie um die große Gefälligkeit bitten, uns
               für Mittwoch oder Donnerſtag{ }Mittag zwei einbettige Zimmer (womöglich nebeneinander) in der \textcolor{pink}{Kaiſerkrone}{}\ledrightnote{\textcolor{pink}{Hotel Kaiserkrone}} – event. \textcolor{pink}{Poſt}{}\ledrightnote{\textcolor{pink}{Hotel Post}} reſerviren zu laſſen? Lieber Mittwoch als Do{\geminationn}erſtag und lieber \textcolor{pink}{Kaiſerkrone}{}\ledrightnote{\textcolor{pink}{Hotel Kaiserkrone}} als \textcolor{pink}{Poſt}{}\ledrightnote{\textcolor{pink}{Hotel Post}}\footnote{\noindent{}aber im Grunde gleichgiltig, insbeſondre ob Mittwoch oder
                        Donnerſtag.} Ich ko{\geminationm}e bei ſchönem Wetter zu Fuſs hinüber,
               ev. über \textcolor{pink}{Koppen}{}\ledrightnote{\textcolor{pink}{Hoher Koppen}} oder \textcolor{pink}{Pötſchen}{}\ledrightnote{\textcolor{pink}{Pötschenpass}}, wohl erſt Nachmittag\pend
           \pstart
           Am Abend des Ankunfttages
               hoffen wir mit Ihnen beim \textcolor{pink}{S.–ſchein}{}\ledrightnote{\textcolor{pink}{Restaurant Sonnenschein}} zu
               nachtmahlen, vorher kommen wir natürlich (we{\geminationn}’s Ihnen
               paſſt) zu Ihnen. Am nächſten Tag \textcolor{pink}{Aſchau}{}\ledrightnote{\textcolor{pink}{Aschau}} und
               Retourfahrt nach \textcolor{pink}{Auſſee}{}\ledrightnote{\textcolor{pink}{Bad Aussee}}. Für telegraf. Verſtändigung
               – welcher Tag welches Hotel wären wir ſehr dankbar!\pend
           \pstart
           Von Herzen Ihr{\\[\baselineskip]}\spacefill\mbox{Arthur}\pend
           \leftskip=0em{}\endnumbering\briefempfaengerindex{Beer-Hofmann, Richard@\textsc{Beer-Hofmann, Richard}!zzzSchnitzler, Arthur@\emph{von Arthur Schnitzler}!1916-08-201@{20. 8. 1916}|)be}\mylabel{h}  \normalsize

\doendnotes{C}
\bigskip
\vfill

\clearpage

\footnotesize

\lohead{\textsc{register}}

% Definiere theindex-Environment komplett neu ohne reledmac
\makeatletter
\renewenvironment{theindex}{%
  \section*{\indexname}%
  \setlength{\parindent}{0pt}%
  \setlength{\parskip}{0pt plus 0.3pt}%
  \let\item\@idxitem
}{%
  \clearpage
}
\makeatother

\IfFileExists{\jobname-pw.ind}{\input{\jobname-pw.ind}}{}

\end{document}

      