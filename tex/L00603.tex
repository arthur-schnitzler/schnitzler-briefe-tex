%% latex-korrekturansicht-vorspann.tex
%% Vorspann für die Korrekturansicht.
%% Lädt die gemeinsame Datei latex-vorspann.tex mit gesetztem Schalter.

\newif\ifkorrekturansicht
\korrekturansichttrue

\input{../tex-inputs/latex-vorspann}


               \section[Hermann Bahr an Arthur Schnitzler, 8. 10. 1896]{ Hermann Bahr an Arthur Schnitzler, 8. 10. 1896}\nopagebreak\mylabel{v}\rehead{ }\normalsize\beginnumbering\briefempfaengerindex{Schnitzler, Arthur@\textsc{Schnitzler, Arthur}!zzzBahr, Hermann@\emph{von Hermann Bahr}!1896-10-082@{8. 10. 1896}|(be} \toendnotes[C]{\smallbreak\pagebreak[2]} \Standort{CUL, Schnitzler, B 5b.}
\physDesc{Brief, 1 Blatt, 3 Seiten
\newline{}Handschrift: schwarze Tinte, deutsche Kurrent\newline{}Ordnung: mit Bleistift von unbekannter Hand nummeriert: »43« }\buchAbdrucke{\weitereDrucke{Hermann Bahr, Arthur Schnitzler: \emph{Briefwechsel, Aufzeichnungen, Dokumente (1891–1931)}. Hg. Kurt Ifkovits und Martin Anton Müller. Göttingen: \emph{Wallstein} 2018, S. 127–128.} }\toendnotes[C]{\smallbreak}\pstart
           \noindent{}{\pb}\textcolor{gray}{\textbf{»\textcolor{brown}{Die
                        Zeit}{}\ledrightnote{\textcolor{brown}{Die Zeit. Wiener Wochenschrift}}«}}\hfill \textcolor{gray}{\textbf{\textbf{\textcolor{pink}{Wien}{}\ledrightnote{\textcolor{pink}{Wien}}}, den }}8/10 \textcolor{gray}{\textbf{189}}\pend
           \pstart
           \textcolor{gray}{\textbf{Wiener Wochenſchrift}}\hfill \textcolor{gray}{\textbf{\textcolor{pink}{IX/3, Günthergaſſe 1}{}\ledrightnote{\textcolor{pink}{Günthergasse}}.}}\pend
           \pstart
           \textcolor{gray}{\textbf{\textbf{Herausgeber}:}}{\\}\textcolor{gray}{\textbf{Profeſſor Dr. \textcolor{blue}{I. Singer}{}\ledrightnote{\textcolor{blue}{Isidor Singer}}, \textcolor{blue}{Hermann Bahr}{}\ledrightnote{\textcolor{blue}{Hermann Bahr}},
                        Dr. \textcolor{blue}{Heinrich Kanner}{}\ledrightnote{\textcolor{blue}{Heinrich Kanner}}.}}\pend
           \pstart
           \textcolor{gray}{\textbf{Telephon Nr. 6415.}}\pend
           \pstart\center{}Lieber Arthur!\pend\pstart
           Ich habe \textcolor{blue}{Brandes}{}\ledrightnote{\textcolor{blue}{Georg Brandes}}{ }ſofort ausführlich
               geſchrieben. Ich kann ihm belegen, daß ich den \textcolor{green}{Artikel}{}\ledrightnote{→\textcolor{green}{Censur in Polen}} von einer ihm u. mir bekannten, ſehr angeſehenen
                  \textcolor{pink}{Berlin}{}\ledrightnote{\textcolor{pink}{Berlin}}er \textcolor{blue}{Dame}{}\ledrightnote{→\textcolor{blue}{Adele Neustädter}} erhielt, als aus einem \textcolor{green}{Buche}{}\ledrightnote{→\textcolor{green}{Polen}}{ }ſta{\geminationm}end, das den
               nächſten Winter erst deutſch erſcheinen ſoll, von ihm autoriſiert, ja mit der
               Ermächtigung, {\pb}für ein beſonderes Honorar das
               Fragment als Originalartikel zu bringen. Ich bin alſo unſchuldig.\pend
           \pstart
           Dir danke ich jedenfalls ſehr, daß Du ſo lieb geweſen biſt, mich gleich zu
               verſtändigen. Intereſſiert Dich die Sache, ſo kannſt Du die ganze Correspondenz mit
               der \textcolor{blue}{Berlinerin}{}\ledrightnote{→\textcolor{blue}{Adele Neustädter}} in unſerem
               Copierbuche ſehen.\pend
           \pstart
           Was macht Deine \label{K_L00603_1v}\edtext{\textcolor{green}{Novelle}{}\ledrightnote{→\textcolor{green}{Die Frau des Weisen. Erzählung}}}{\lemma{\textnormal{\emph{Novelle}}}\Cendnote{\textnormal{Daraus wird: \textcolor{blue}{Arthur Schnitzler}: \emph{\textcolor{green}{Die Frau des Weisen}}. In: \emph{\textcolor{green}{Die Zeit}}, Bd. 10, H. 118, 2. 1. 1897,
                     S. 15–16; H. 119, 9. 1. 1897, S. 31–32;
                     H. 129, 16. 1. 1897, S. 47–48.}}}\label{K_L00603_1h}? Ich
               rechne beſtimmt auf ſie! Auch bin ich ſehr {\pb}neugierig, was aus dem »\textcolor{green}{Freiwild}{}\ledrightnote{\textcolor{green}{Freiwild. Schauspiel in 3 Akten}}« wird.\pend
           \pstart
           Nochmals dankt herzlich{\\[\baselineskip]}mit beſten Grüßen{\\[\baselineskip]}Dein{\\[\baselineskip]}\spacefill\mbox{Hermann}\pend
           \leftskip=0em{}\pstart
           \noindent{}Herrn \textsc{D\textsuperscript{r} Arthur Schnitzler}{\\}\textsc{\textcolor{pink}{Wien}{}\ledrightnote{\textcolor{pink}{Wien}}{ }\textcolor{pink}{IX Frankgasse 1}{}\ledrightnote{\textcolor{pink}{Frankgasse}}.}\pend
           \pstart
           \textcolor{gray}{\textbf{\label{T_L00603_1v}\edtext{Alle für »\textcolor{brown}{Die Zeit}{}\ledrightnote{\textcolor{brown}{Die Zeit. Wiener Wochenschrift}}« beſtimmten Zuſchriften und Sendungen ſind an
                  die Redaction der »\textcolor{brown}{Zeit}{}\ledrightnote{\textcolor{brown}{Die Zeit. Wiener Wochenschrift}}« und \textbf{nicht} an die Perſon eines der Herausgeber zu richten.}{\lemma{\textnormal{\emph{Alle … richten.}}}\Cendnote{\textnormal{am unteren Rand der ersten Seite}}}\label{T_L00603_1h}}}\pend
           \endnumbering\briefempfaengerindex{Schnitzler, Arthur@\textsc{Schnitzler, Arthur}!zzzBahr, Hermann@\emph{von Hermann Bahr}!1896-10-082@{8. 10. 1896}|)be}\mylabel{h}  \normalsize

\doendnotes{C}
\bigskip
\vfill

\clearpage

\footnotesize

\lohead{\textsc{register}}

% Definiere theindex-Environment komplett neu ohne reledmac
\makeatletter
\renewenvironment{theindex}{%
  \section*{\indexname}%
  \setlength{\parindent}{0pt}%
  \setlength{\parskip}{0pt plus 0.3pt}%
  \let\item\@idxitem
}{%
  \clearpage
}
\makeatother

\IfFileExists{\jobname-pw.ind}{\input{\jobname-pw.ind}}{}

\end{document}

      