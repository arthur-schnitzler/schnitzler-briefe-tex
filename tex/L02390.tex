%% latex-korrekturansicht-vorspann.tex
%% Vorspann für die Korrekturansicht.
%% Lädt die gemeinsame Datei latex-vorspann.tex mit gesetztem Schalter.

\newif\ifkorrekturansicht
\korrekturansichttrue

\input{../tex-inputs/latex-vorspann}


               \section[Arthur Schnitzler an Thomas Mann, 26. 6. 1922]{ Arthur Schnitzler an Thomas Mann, 26. 6. 1922}\nopagebreak\mylabel{v}\rehead{ }\normalsize\beginnumbering\briefempfaengerindex{Mann, Thomas@\textsc{Mann, Thomas}!zzzSchnitzler, Arthur@\emph{von Arthur Schnitzler}!1922-06-262@{26. 6. 1922}|(be} \toendnotes[C]{\smallbreak\pagebreak[2]} \Standort{Zürich, Thomas-Mann-Archiv, B-II-SCHNM-1.}
\physDesc{Briefkarte
\newline{}Handschrift: schwarze Tinte, lateinische Kurrent}\buchAbdrucke{\weitereDrucke{Hertha Krotkoff: \emph{Arthur Schnitzler – Thomas Mann: Briefe.} In: \emph{Modern Austrian Literature}, Jg. 7 (1974) Nr. 1/2, S. 17–18.} }\toendnotes[C]{\smallbreak}\pstart
           \raggedleft{}{\pb}\textcolor{pink}{Wien}{}\ledrightnote{\textcolor{pink}{Wien}}{ }26. 6. 22\pend
           \pstart
           Verehrter und lieber Herr Thomas Mann, erlauben Sie, dſs ich
                    Ihnen Mr. \textcolor{blue}{Scofield Thayer}{}\ledrightnote{\textcolor{blue}{Scofield Thayer}} vorstelle, den
                    Herausgeber der »\textcolor{brown}{Dial}{}\ledrightnote{\textcolor{brown}{The Dial}},« der Ihre Werke liebt
                    und bewundert. Mr. \textcolor{blue}{Thayer}{}\ledrightnote{\textcolor{blue}{Scofield Thayer}} hat sich fast ein
                    Jahr lang in \textcolor{pink}{Wien}{}\ledrightnote{\textcolor{pink}{Wien}} aufgehalten, ich habe höchst
                    anregende Stunden mit ihm verbracht; und so kostbar Ihre Zeit ist – ich glaube,
                    daß auch Ihnen die Beka{\geminationn}tschaft mit diesem auf
                    vielen Gebieten interessanten, um die {\pb}Verbreitung deutscher Literatur in \textcolor{pink}{Amerika}{}\ledrightnote{\textcolor{pink}{Amerika}}
                    höchst verdienten und wahrhaft liebenswürdigen jungen \textcolor{blue}{Mannes}{}\ledrightnote{→\textcolor{blue}{Scofield Thayer}} nicht unangenehm sein wird.\pend
           \pstart
           Darf ich hier meinen herzlichen Dank für die schönen \textcolor{green}{Worte}{}\ledrightnote{→\textcolor{green}{Arthur Schnitzler zu seinem sechzigsten Geburtstag}} anschließen, die Sie mir zu meinem immerhin
                    sechzigsten Geburtstag in der \textcolor{green}{N. R.}{}\ledrightnote{\textcolor{green}{Die neue Rundschau}} gewidmet
                    haben?\pend
           \pstart
           Ich sehe Sie hoffentlich bald wieder; und grüße Sie in freundschaftlicher
                    Bewunderung als Ihr ergebener \spacefill\mbox{Arthur Schnitzler}\pend
           \endnumbering\briefempfaengerindex{Mann, Thomas@\textsc{Mann, Thomas}!zzzSchnitzler, Arthur@\emph{von Arthur Schnitzler}!1922-06-262@{26. 6. 1922}|)be}\mylabel{h}  \normalsize

\doendnotes{C}
\bigskip
\vfill

\clearpage

\footnotesize

\lohead{\textsc{register}}

% Definiere theindex-Environment komplett neu ohne reledmac
\makeatletter
\renewenvironment{theindex}{%
  \section*{\indexname}%
  \setlength{\parindent}{0pt}%
  \setlength{\parskip}{0pt plus 0.3pt}%
  \let\item\@idxitem
}{%
  \clearpage
}
\makeatother

\IfFileExists{\jobname-pw.ind}{\input{\jobname-pw.ind}}{}

\end{document}

      