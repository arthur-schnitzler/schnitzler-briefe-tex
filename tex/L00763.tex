%% latex-korrekturansicht-vorspann.tex
%% Vorspann für die Korrekturansicht.
%% Lädt die gemeinsame Datei latex-vorspann.tex mit gesetztem Schalter.

\newif\ifkorrekturansicht
\korrekturansichttrue

\input{../tex-inputs/latex-vorspann}


               \section[Max Burckhard an Arthur Schnitzler, {[}16. 1. 1898{]}]{ Max Burckhard an Arthur Schnitzler, {[}16. 1. 1898{]}}\nopagebreak\mylabel{v}\rehead{ }\normalsize\beginnumbering\briefempfaengerindex{Schnitzler, Arthur@\textsc{Schnitzler, Arthur}!zzzBurckhard, Max Eugen@\emph{von Max Eugen Burckhard}!1898-01-161@{{[}16. 1. 1898{]}}|(be} \toendnotes[C]{\smallbreak\pagebreak[2]} \Standort{CUL, Schnitzler, B 20.}
\physDesc{Brief, 1 Blatt, 1 Seite
\newline{}Handschrift: schwarze Tinte, deutsche Kurrent
\newline{}Schnitzler: mit Bleistift datiert: »16/1 98« \newline{}Ordnung: mit Bleistift von unbekannter Hand nummeriert:
                                    »11« }\toendnotes[C]{\smallbreak}\pstart
           \raggedleft{}{\pb}\pend
           \pstart{}Sehr verehrter Herr Doctor!\pend\pstart
           Leider iſt mir ein Hindernis für heute unterlaufen, da der Beſitzer der Jagdhütte, wo
               ich den So{\geminationm}er bin, heute Abend anko{\geminationm}t u ich ihn erwarten muß. Ich retourniere alſo mit
               herzlichem Dank die \label{K_L00763_1v}\edtext{Karten}{\lemma{\textnormal{\emph{Karten}}}\Cendnote{\textnormal{Am 16. 1. 1898 fand in den \textcolor{pink}{Sofiensälen} in \textcolor{pink}{Wien} eine Wohltätigkeitsveranstaltung zugunsten des Vereines \emph{\textcolor{brown}{Ferienheim}} statt, der Landaufenthalte von Kindern
                  förderte und organisierte. Von \textcolor{blue}{Schnitzler} wurden
                     \emph{\textcolor{green}{Weihnachts-Einkäufe}} und \emph{\textcolor{green}{Abschiedssouper}} gegeben.}}}\label{K_L00763_1h}.\pend
           \pstart
           Mit herzlicher Empfehlung{\\[\baselineskip]}\spacefill\mbox{D\textsuperscript{r}Burckhard}\pend
           \leftskip=0em{}\endnumbering\briefempfaengerindex{Schnitzler, Arthur@\textsc{Schnitzler, Arthur}!zzzBurckhard, Max Eugen@\emph{von Max Eugen Burckhard}!1898-01-161@{{[}16. 1. 1898{]}}|)be}\mylabel{h}  \normalsize

\doendnotes{C}
\bigskip
\vfill

\clearpage

\footnotesize

\lohead{\textsc{register}}

% Definiere theindex-Environment komplett neu ohne reledmac
\makeatletter
\renewenvironment{theindex}{%
  \section*{\indexname}%
  \setlength{\parindent}{0pt}%
  \setlength{\parskip}{0pt plus 0.3pt}%
  \let\item\@idxitem
}{%
  \clearpage
}
\makeatother

\IfFileExists{\jobname-pw.ind}{\input{\jobname-pw.ind}}{}

\end{document}

      