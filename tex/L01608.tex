%% latex-korrekturansicht-vorspann.tex
%% Vorspann für die Korrekturansicht.
%% Lädt die gemeinsame Datei latex-vorspann.tex mit gesetztem Schalter.

\newif\ifkorrekturansicht
\korrekturansichttrue

\input{../tex-inputs/latex-vorspann}


               \section[Arthur Schnitzler an Georg Brandes, 11. 7. 1906]{ Arthur Schnitzler an Georg Brandes, 11. 7. 1906}\nopagebreak\mylabel{v}\rehead{ }\normalsize\beginnumbering\briefempfaengerindex{Brandes, Georg@\textsc{Brandes, Georg}!zzzSchnitzler, Arthur@\emph{von Arthur Schnitzler}!1906-07-111@{11. 7. 1906}|(be} \toendnotes[C]{\smallbreak\pagebreak[2]} \Standort{CUL, Schnitzler, B 17-2.}
\physDesc{Bildpostkarte, maschinelle Abschrift}\buchAbdrucke{\weitereDrucke{Georg Brandes, Arthur Schnitzler: \emph{Ein Briefwechsel}. Hg. Kurt Bergel. Bern: \emph{Francke} 1956, S. 93.} }\toendnotes[C]{\smallbreak}\pstart
           \noindent{}{\pb}26) (Ansichtskarte)\pend
           \pstart
           \raggedleft{}11. 7. 906. \textcolor{pink}{Marienlyst}{}\ledrightnote{\textcolor{pink}{Marienlyst}}\pend
           \pstart
           Verehrtester Herr Brandes, heute erhalte ich eine Karte von \textcolor{blue}{Brahm}{}\ledrightnote{\textcolor{blue}{Otto Brahm}}, der mich bittet Sie herzlich zu
                    grüssen und mir von Ihrem \label{K_L01608_1v}\edtext{\textcolor{green}{\textcolor{blue}{Ibsen}{}\ledrightnote{\textcolor{blue}{Henrik Ibsen}}-Büchlein}{}\ledrightnote{→\textcolor{green}{Henrik Ibsen}}}{\lemma{\textnormal{\emph{Ibsen-Büchlein}}}\Cendnote{\textnormal{\emph{\textcolor{green}{Henrik Ibsen}} von \textcolor{blue}{Georg Brandes}. Mit zwölf Briefen \textcolor{blue}{Henrik Ibsens}. Siebzehn Vollbilder und vier
                            Faksimiles. Berlin: \emph{Bard Marquardt}{ }{[}1906{]} (Die Literatur, hg. \textcolor{blue}{Georg
                                Brandes}, Bd. 32–33).}}}\label{K_L01608_1h} erzählt. Lassen Sie mich
                    doch, wenn’s leicht geht, durch eine Zeile wissen, wie’s Ihnen geht. Mir gefällt
                    es hier ausnehmend gut. Auf Wiedersehen und herzlichen Gruss. Ihr \spacefill\mbox{Arthur
                        Schnitzler}\pend
           \endnumbering\briefempfaengerindex{Brandes, Georg@\textsc{Brandes, Georg}!zzzSchnitzler, Arthur@\emph{von Arthur Schnitzler}!1906-07-111@{11. 7. 1906}|)be}\mylabel{h}  \normalsize

\doendnotes{C}
\bigskip
\vfill

\clearpage

\footnotesize

\lohead{\textsc{register}}

% Definiere theindex-Environment komplett neu ohne reledmac
\makeatletter
\renewenvironment{theindex}{%
  \section*{\indexname}%
  \setlength{\parindent}{0pt}%
  \setlength{\parskip}{0pt plus 0.3pt}%
  \let\item\@idxitem
}{%
  \clearpage
}
\makeatother

\IfFileExists{\jobname-pw.ind}{\input{\jobname-pw.ind}}{}

\end{document}

      