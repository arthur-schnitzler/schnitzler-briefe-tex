%% latex-korrekturansicht-vorspann.tex
%% Vorspann für die Korrekturansicht.
%% Lädt die gemeinsame Datei latex-vorspann.tex mit gesetztem Schalter.

\newif\ifkorrekturansicht
\korrekturansichttrue

\input{../tex-inputs/latex-vorspann}


               \section[Robert Adam an Arthur Schnitzler, 22. 6. 1915]{ Robert Adam an Arthur Schnitzler, 22. 6. 1915}\nopagebreak\mylabel{v}\rehead{ }\normalsize\beginnumbering\briefempfaengerindex{Schnitzler, Arthur@\textsc{Schnitzler, Arthur}!zzzAdam, Robert@\emph{von Robert Adam}!1915-06-221@{22. 6. 1915}|(be} \toendnotes[C]{\smallbreak\pagebreak[2]} \Standort{DLA, A:Schnitzler, HS.NZ85.1.4230,9.}
\physDesc{Brief, 1 Blatt, 2 Seiten
\newline{}Handschrift: schwarze Tinte, deutsche Kurrent
\newline{}Schnitzler: 1) mit Bleistift beschriftet: »\textsc{Adam}« 2) mit rotem Buntstift eine Unterstreichung}\toendnotes[C]{\smallbreak}\pstart
           \raggedleft{}{\pb}\textcolor{pink}{Ziſtersdorf}{}\ledrightnote{\textcolor{pink}{Zistersdorf}}, 22. Juni 1915. \pend
           \pstart{}Hochverehrter Herr Doktor!\pend\pstart
           Ich kann Ihnen anzeigen, daß es mir nach längerer Beratung mit unſerem \textcolor{blue}{Poſtmeiſter}{}\ledrightnote{→\textcolor{blue}{?? [Postmeister in Zistersdorf]}}, der über den
                    Kriegspoſtverkehr mit den Verbündeten nicht viel beſſer informiert zu ſein
                    ſcheint als ich, gelungen iſt, das Manuſkript des »\textcolor{green}{\textsc{Fremden}}{}\ledrightnote{\textcolor{green}{Der Fremde}}« mit einem Briefe an den \textcolor{brown}{Fiſcherſchen
                        Verlag}{}\ledrightnote{\textcolor{brown}{S. Fischer Verlag}} zu ſenden, und ich gebe mich der Hoffnung hin, daß beides den
                    Beſtimmungsort erreicht.\pend
           \pstart
           Zugleich erlaube ich mir, Ihnen das Manuſkript der Komödie: »\textcolor{green}{Geſellſchaft}{}\ledrightnote{\textcolor{green}{Gesellschaft [Eine Gaunerkomödie]}}« zu ſchicken, die, wie ich Ihnen erzählte, vom
                        »\textcolor{pink}{Deutſchen Volkstheater}{}\ledrightnote{\textcolor{pink}{Volkstheater}}« abgelehnt wurde.
                    Ein Meiſterwerk iſt ſie ja gewiß nicht, obwohl ich meinen möchte, daß ſie, vom
                    techniſchen Geſichtspunkt aus betrachtet, einem gelernten »Dramaturgen«
                    Freudentränen entlocken könnte. Aber \strikeout{iſ}{ }ſie iſt {\pb}wohl
                    vergnüglich; allerdings kann ich dieſe ihre Eigenſchaft ſelbſt nicht objektiv
                    einſchätzen, aber ich ſchließe es daraus, daß ich ſie mit derſelben
                    Behaglichkeit niederſchrieb, die den alten \textcolor{blue}{Dumas}{}\ledrightnote{\textcolor{blue}{Alexandre père Dumas}} beim Verfaſſen ſeiner heitern Romane hell auflachen ließ. Wenn
                    die Erlebniſſe meiner Helden, die ich zum größten Teil perſönlich kennen lernen
                    durfte – den Daniel Rubinſtein ſchilderten mir nur Perſonen, die er mit ſeiner
                    intereſſanten Bekanntſchaft beehrt hatte –, Sie auch nur ein wenig erheitern,
                    wird es mich außerordentlich freuen. Eigentlich habe ich doch die Hoffnung noch
                    nicht ganz aufgegeben, dieſe Komödie bei einer Bühne anzubringen (allenfalls
                    nach einigen Verbeſſerungen); denn ich glaube, daß ſie eine ganze Anzahl »guter
                    Rollen« enthält.\pend
           \pstart
           Indem ich Ihnen, hochverehrter Herr Doktor, für Ihre große Liebenswürdigkeit
                    nochmals herzlich danke, verbleibe ich Ihr ſehr ergebener\pend
           \pstart \spacefill\mbox{Robert Adam}\pend{}\endnumbering\briefempfaengerindex{Schnitzler, Arthur@\textsc{Schnitzler, Arthur}!zzzAdam, Robert@\emph{von Robert Adam}!1915-06-221@{22. 6. 1915}|)be}\mylabel{h}  \normalsize

\doendnotes{C}
\bigskip
\vfill

\clearpage

\footnotesize

\lohead{\textsc{register}}

% Definiere theindex-Environment komplett neu ohne reledmac
\makeatletter
\renewenvironment{theindex}{%
  \section*{\indexname}%
  \setlength{\parindent}{0pt}%
  \setlength{\parskip}{0pt plus 0.3pt}%
  \let\item\@idxitem
}{%
  \clearpage
}
\makeatother

\IfFileExists{\jobname-pw.ind}{\input{\jobname-pw.ind}}{}

\end{document}

      