%% latex-korrekturansicht-vorspann.tex
%% Vorspann für die Korrekturansicht.
%% Lädt die gemeinsame Datei latex-vorspann.tex mit gesetztem Schalter.

\newif\ifkorrekturansicht
\korrekturansichttrue

\input{../tex-inputs/latex-vorspann}


               \section[Paul Goldmann an Arthur Schnitzler, 3. 12. {[}1893?{]}]{ Paul Goldmann an Arthur Schnitzler, 3. 12. {[}1893?{]}}\nopagebreak\mylabel{v}\rehead{ }\normalsize\beginnumbering\briefempfaengerindex{Schnitzler, Arthur@\textsc{Schnitzler, Arthur}!zzzGoldmann, Paul@\emph{von Paul Goldmann}!1893-12-031@{3. 12. {[}1893?{]}}|(be} \toendnotes[C]{\smallbreak\pagebreak[2]} \Standort{DLA, A:Schnitzler, HS.NZ85.1.3163.}
\physDesc{Telegramm1 Blatt, 1 Seite
\newline{}maschinell\newline{}Versand: von unbekannter Hand mit Bleistift Vermerk: »\textcolor{gray}{\textbf{mit}} 27 \textcolor{gray}{\textbf{Taxworten}}« \newline{}Ordnung: beschnitten }\toendnotes[C]{\smallbreak}\pstart
           \centering{}{\pb}\textcolor{pink}{w}{}\ledrightnote{\textcolor{pink}{Wien}}{ }\textcolor{pink}{paris}{}\ledrightnote{\textcolor{pink}{Paris}} 5298 4\label{K_L02695-1v}\edtext{3 12}{\lemma{\textnormal{\emph{3 12}}}\Cendnote{\textnormal{Trotz des verschobenen Leerzeichens
                     in der Vorlage findet sich hier die Datumsangabe des Telegramms.}}}\label{K_L02695-1h}{ }17\pend
           \pstart
           tausend herzliche glueckwuensche fuer zwei ersten \label{K_L02695-2v}\edtext{\textcolor{green}{acte}{}\ledrightnote{→\textcolor{green}{Das Märchen. Schauspiel in drei Aufzügen}}}{\lemma{\textnormal{\emph{acte}}}\Cendnote{\textnormal{\emph{\textcolor{green}{Das Märchen}} hatte am 1. 12. 1893 Uraufführung am \emph{\textcolor{brown}{Deutschen Volkstheater}} in \textcolor{pink}{Wien}. Die Kritik an dem abfallenden dritten Akt notierte sich \textcolor{blue}{Schnitzler} im \emph{\textcolor{green}{Tagebuch}} (2. 12. 1893) und kürzte ihn für die zweite Vorstellung am selben
                  Tag. Trotzdem wurde das Stück nach dieser Vorstellung abgesetzt. In Druck erschien
                  dann im folgenden Mai ein geänderter Schluss (\emph{\textcolor{brown}{E. Pierson}}{ }1894), der 1902 für die 2. Auflage neuerlich abgeändert wurde (\emph{\textcolor{brown}{S. Fischer Verlag}}). }}}\label{K_L02695-2h} lass die dummen buben schrejben wohl dem welchem zum vollendeten
                  \label{T_L02695-1v}\edtext{dramatiker}{\lemma{\textnormal{\emph{dramatiker}}}\Cendnote{\textnormal{in der Vorlage: »dramatiken«}}}\label{T_L02695-1h} nur noch
               ein \label{T_L02695-2v}\edtext{dritter}{\lemma{\textnormal{\emph{dritter}}}\Cendnote{\textnormal{in der Vorlage: »dritten«}}}\label{T_L02695-2h} act fehlt
               jetzt geht es unaufhaltsam hinauf bitte schicke mir alle kritiken\pend
           \pstart dank gruesze + \spacefill\mbox{goldmann}\pend{}\endnumbering\briefempfaengerindex{Schnitzler, Arthur@\textsc{Schnitzler, Arthur}!zzzGoldmann, Paul@\emph{von Paul Goldmann}!1893-12-031@{3. 12. {[}1893?{]}}|)be}\mylabel{h}\begin{anhang}\end{anhang}\normalsize

\doendnotes{C}
\bigskip
\vfill

\clearpage

\footnotesize

\lohead{\textsc{register}}

% Definiere theindex-Environment komplett neu ohne reledmac
\makeatletter
\renewenvironment{theindex}{%
  \section*{\indexname}%
  \setlength{\parindent}{0pt}%
  \setlength{\parskip}{0pt plus 0.3pt}%
  \let\item\@idxitem
}{%
  \clearpage
}
\makeatother

\IfFileExists{\jobname-pw.ind}{\input{\jobname-pw.ind}}{}

\end{document}

      