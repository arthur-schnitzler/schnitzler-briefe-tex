%% latex-korrekturansicht-vorspann.tex
%% Vorspann für die Korrekturansicht.
%% Lädt die gemeinsame Datei latex-vorspann.tex mit gesetztem Schalter.

\newif\ifkorrekturansicht
\korrekturansichttrue

\input{../tex-inputs/latex-vorspann}


               \section[Hugo von Hofmannsthal an Arthur Schnitzler, 9. 11. {[}1896{]}]{ Hugo von Hofmannsthal an Arthur Schnitzler, 9. 11. {[}1896{]}}\nopagebreak\mylabel{v}\rehead{ }\normalsize\beginnumbering\briefempfaengerindex{Schnitzler, Arthur@\textsc{Schnitzler, Arthur}!zzzHofmannsthal, Hugo von@\emph{von Hugo von Hofmannsthal}!1896-11-092@{9. 11. {[}1896{]}}|(be} \toendnotes[C]{\smallbreak\pagebreak[2]} \Standort{CUL, Schnitzler, B 43.}
\physDesc{Brief, 1 Blatt (Briefpapier mit aufgeprägtem Wappen), 2 Seiten
\newline{}Handschrift: Bleistift, deutsche Kurrent
\newline{}Schnitzler: mit Bleistift die Jahreszahl ergänzt: »96« \newline{}Ordnung: mit Bleistift von unbekannter Hand nummeriert:
                                        »82« }\buchAbdrucke{\weitereDrucke{Hugo von Hofmannsthal, Arthur Schnitzler: \emph{Briefwechsel}. Hg. Therese Nickl und Heinrich Schnitzler. Frankfurt am Main: \emph{S. Fischer} 1964, S. 76.} }\toendnotes[C]{\smallbreak}\pstart
           \raggedleft{}{\pb}\textcolor{pink}{Wien}{}\ledrightnote{\textcolor{pink}{Wien}}{ }9\textsuperscript{ten} 11.\pend
           \pstart{}mein lieber Arthur,\pend\pstart
           ich bin durch die Zeitungen und \textcolor{blue}{Salten}{}\ledrightnote{\textcolor{blue}{Felix Salten}} über
                    den Erfolg Ihres \textcolor{green}{Stückes}{}\ledrightnote{→\textcolor{green}{Freiwild. Schauspiel in 3 Akten}}{ }ſo völlig beruhigt,
                    daſs ich faſt vergeſſen hatte, Ihnen ein Wort darüber zu ſagen.\pend
           \pstart
           \substVorne{}\textsuperscript{E}\substDazwischen{}I\substHinten{}ch denke, es muſs Ihnen eher hübſch vorkommen, daſs es einige Menſchen
                    gibt, die des abſoluten {\pb}Werthes Ihrer Arbeiten innerlich ſo verſichert ſind, daſs ihnen der äußere
                    Erfolg dann ziemlich gleichgiltig iſt.\pend
           \pstart
           Daſs das Telegramm nicht von mir war, werden Sie ſich wohl ſpäter ſelbſt gedacht
                    haben.\pend
           \pstart
           Ich freue mich ſehr darauf Sie zu ſehen.\pend
           \pstart
           Von Herzen Ihr{\\[\baselineskip]}\spacefill\mbox{Hugo}\pend
           \leftskip=0em{}\endnumbering\briefempfaengerindex{Schnitzler, Arthur@\textsc{Schnitzler, Arthur}!zzzHofmannsthal, Hugo von@\emph{von Hugo von Hofmannsthal}!1896-11-092@{9. 11. {[}1896{]}}|)be}\mylabel{h}  \normalsize

\doendnotes{C}
\bigskip
\vfill

\clearpage

\footnotesize

\lohead{\textsc{register}}

% Definiere theindex-Environment komplett neu ohne reledmac
\makeatletter
\renewenvironment{theindex}{%
  \section*{\indexname}%
  \setlength{\parindent}{0pt}%
  \setlength{\parskip}{0pt plus 0.3pt}%
  \let\item\@idxitem
}{%
  \clearpage
}
\makeatother

\IfFileExists{\jobname-pw.ind}{\input{\jobname-pw.ind}}{}

\end{document}

      