%% latex-korrekturansicht-vorspann.tex
%% Vorspann für die Korrekturansicht.
%% Lädt die gemeinsame Datei latex-vorspann.tex mit gesetztem Schalter.

\newif\ifkorrekturansicht
\korrekturansichttrue

\input{../tex-inputs/latex-vorspann}


               \section[Richard Beer-Hofmann an Arthur Schnitzler, 14. 9. 1931]{ Richard Beer-Hofmann an Arthur Schnitzler,
               14. 9. 1931}\nopagebreak\mylabel{v}\rehead{ }\normalsize\beginnumbering\briefempfaengerindex{Schnitzler, Arthur@\textsc{Schnitzler, Arthur}!zzzBeer-Hofmann, Richard@\emph{von Richard Beer-Hofmann}!1931-09-141@{14. 9. 1931}|(be} \toendnotes[C]{\smallbreak\pagebreak[2]} \Standort{CUL, Schnitzler, B 8.}
\physDesc{Bildpostkarte
\newline{}Handschrift: Bleistift, lateinische Kurrent\newline{}Versand: 1) Stempel: »\nobreak{}Gebrauchet die heilkräftigen Solbäder Bad Ischls\nobreak{}«.  2) Stempel: »\nobreak{}\oindex{Bad Ischl@\textbf{Bad Ischl}, \emph{Besiedelter Ort (A.BSO)}|pwk}Bad Isch\textcolor{gray}{l}, 15. IX. 31, 13\nobreak{}«. 
\newline{}Schnitzler: mit rotem Buntstift datiert: »13/9 31« \newline{}Ordnung: mit Bleistift von unbekannter Hand nummeriert:
                              »278« }\buchAbdrucke{\weitereDrucke{Arthur Schnitzler, Richard Beer-Hofmann: \emph{Briefwechsel 1891–1931}. Hg. Konstanze Fliedl. Wien, Zürich: \emph{Europaverlag} 1992, S. 232.} }\toendnotes[C]{\smallbreak}\pstart{}{\pb}Herrn\pend{}\pstart{}D\textsuperscript{r} Arthur Schnitzler\pend{}\pstart{}\textcolor{pink}{Wien XVIII.}{}\ledrightnote{\textcolor{pink}{XVIII., Währing}}\pend{}\pstart{}\textcolor{pink}{Sternwartstrasse 71}{}\ledrightnote{\textcolor{pink}{Sternwartestraße}}\pend{}{\bigskip}\pstart
           \noindent{}\centering{}\textcolor{gray}{\textbf{{\pb}\textcolor{pink}{Bad Ischl}{}\ledrightnote{\textcolor{pink}{Bad Ischl}}}}\pend
           \pstart
           \centering{}{\pb}14. IX. 31\pend
           \pstart
           Wir sind seit 1 von \textcolor{pink}{Wien}{}\ledrightnote{\textcolor{pink}{Wien}} weg, seit
                  2. – hier und werden in 2–3 Tagen wieder in \textcolor{pink}{Wien}{}\ledrightnote{\textcolor{pink}{Wien}} sein. Wir haben 10 Tage schönes Wetter gehabt – müssen also
               zufrieden sein. Auf allen Wegen sind hier Erinnerungen, wehmütige, aber auch \label{T_L02548_1v}\edtext{schöne!}{\lemma{\textnormal{\emph{schöne!}}}\Cendnote{\textnormal{weiter quer am
                  rechten Rand}}}\label{T_L02548_1h}\pend
           \pstart Herzlichst Ihr\spacefill\mbox{Richard.}\pend{}\endnumbering\briefempfaengerindex{Schnitzler, Arthur@\textsc{Schnitzler, Arthur}!zzzBeer-Hofmann, Richard@\emph{von Richard Beer-Hofmann}!1931-09-141@{14. 9. 1931}|)be}\mylabel{h}  \normalsize

\doendnotes{C}
\bigskip
\vfill

\clearpage

\footnotesize

\lohead{\textsc{register}}

% Definiere theindex-Environment komplett neu ohne reledmac
\makeatletter
\renewenvironment{theindex}{%
  \section*{\indexname}%
  \setlength{\parindent}{0pt}%
  \setlength{\parskip}{0pt plus 0.3pt}%
  \let\item\@idxitem
}{%
  \clearpage
}
\makeatother

\IfFileExists{\jobname-pw.ind}{\input{\jobname-pw.ind}}{}

\end{document}

      