%% latex-korrekturansicht-vorspann.tex
%% Vorspann für die Korrekturansicht.
%% Lädt die gemeinsame Datei latex-vorspann.tex mit gesetztem Schalter.

\newif\ifkorrekturansicht
\korrekturansichttrue

\input{../tex-inputs/latex-vorspann}


               \section[Arthur und Olga Schnitzler an Richard Beer-Hofmann, 20. 7. 1908]{ Arthur und Olga Schnitzler an Richard Beer-Hofmann, 20. 7. 1908}\nopagebreak\mylabel{v}\rehead{ }\normalsize\beginnumbering\briefempfaengerindex{Beer-Hofmann, Richard@\textsc{Beer-Hofmann, Richard}!zzzSchnitzler, Olga@\emph{von Olga Schnitzler}!1908-07-201@{20. 7. 1908}|(be}\briefempfaengerindex{Beer-Hofmann, Richard@\textsc{Beer-Hofmann, Richard}!zzzSchnitzler, Arthur@\emph{von Arthur Schnitzler}!1908-07-201@{20. 7. 1908}|(be} \toendnotes[C]{\smallbreak\pagebreak[2]} \Standort{YCGL, MSS 31.}
\physDesc{Bildpostkarte
\newline{}Handschrift Arthur Schnitzler: Bleistift, deutsche Kurrent\newline{}Handschrift Olga Schnitzler: Bleistift, lateinische Kurrent\newline{}Versand: Stempel: »\nobreak{}\oindex{Seis am Schlern@\textbf{Seis am Schlern}, \emph{Besiedelter Ort (A.BSO)}|pwk}{[}S{]}eis, 20–7–{[}1908{]}\nobreak{}«.  \newline{}Ordnung: mit Bleistift von unbekannter Hand datiert: »20. 7.« }\toendnotes[C]{\smallbreak}\pstart{}{\pb}\textsc{Dr. Richard Beer-Hofmann}\pend{}\pstart{}\textcolor{pink}{\textsc{Strobl}}{}\ledrightnote{\textcolor{pink}{Strobl}}\pend{}\pstart{}\textcolor{pink}{\textsc{Hotel am See}}{}\ledrightnote{\textcolor{pink}{Hotel am See}}\pend{}\pstart{}\textsc{\textcolor{pink}{Salzka{\geminationm}ergut}{}\ledrightnote{\textcolor{pink}{Salzkammergut}}.}\pend{}{\bigskip}\pstart
           \noindent{}\centering{}{\pb}\textcolor{gray}{\textbf{\textcolor{pink}{Tirol}{}\ledrightnote{\textcolor{pink}{Südtirol}}: Partie in \textcolor{pink}{Seis a. Schlern}{}\ledrightnote{\textcolor{pink}{Seis am Schlern}}, 1000m. N. d. \textcolor{green}{Aquarell}{}\ledrightnote{→\textcolor{green}{Partie in Seis am Schlern}} v. \textcolor{blue}{F. A. C. M\textcolor{gray}{.} Reisch}{}\ledrightnote{\textcolor{blue}{Franz August Carl Maria Reisch}}, \textcolor{pink}{Meran}{}\ledrightnote{\textcolor{pink}{Meran}}.}}\pend
           \pstart
           {\pb}Herzliche Grüße. Uns geht es hier und gefällt’s hier
               weiter ſehr gut. Seit \label{K_L01784_1v}\edtext{10.}{\lemma{\textnormal{\emph{10.}}}\Cendnote{\textnormal{siehe A. S.: \emph{Tagebuch}, 10. 7. 1908}}}\label{K_L01784_1h} iſt \textcolor{blue}{Brahm}{}\ledrightnote{\textcolor{blue}{Otto Brahm}} da.\footnote{\noindent{}und grüsst herzlich.} Wir denken bis zweite Hälfte Auguſt zu bleiben. Da{\geminationn} Reiſe. Wohin unbeſti{\geminationm}t.
                  \textcolor{pink}{\textsc{Martino}}{}\ledrightnote{\textcolor{pink}{San Martino di Castrozza}}? \textcolor{pink}{\textsc{Campiglio}}{}\ledrightnote{\textcolor{pink}{Madonna di Campiglio}}? \textcolor{pink}{\textsc{Engadin}}{}\ledrightnote{\textcolor{pink}{Engadin}}? – Schreiben Sie recht bald, wie’s Ihnen geht und was Sie {\pb}treiben. \pend
           \pstart Von Herzen Ihr \spacefill\mbox{A.}\pend{}\pstart
           \noindent{}{[}hs. O. Schnitzler:{]} Herzliche Grüsse!\pend
           \pstart \spacefill\mbox{OlgaS.}\pend{}\endnumbering\briefempfaengerindex{Beer-Hofmann, Richard@\textsc{Beer-Hofmann, Richard}!zzzSchnitzler, Olga@\emph{von Olga Schnitzler}!1908-07-201@{20. 7. 1908}|)be}\briefempfaengerindex{Beer-Hofmann, Richard@\textsc{Beer-Hofmann, Richard}!zzzSchnitzler, Arthur@\emph{von Arthur Schnitzler}!1908-07-201@{20. 7. 1908}|)be}\mylabel{h}  \normalsize

\doendnotes{C}
\bigskip
\vfill

\clearpage

\footnotesize

\lohead{\textsc{register}}

% Definiere theindex-Environment komplett neu ohne reledmac
\makeatletter
\renewenvironment{theindex}{%
  \section*{\indexname}%
  \setlength{\parindent}{0pt}%
  \setlength{\parskip}{0pt plus 0.3pt}%
  \let\item\@idxitem
}{%
  \clearpage
}
\makeatother

\IfFileExists{\jobname-pw.ind}{\input{\jobname-pw.ind}}{}

\end{document}

      