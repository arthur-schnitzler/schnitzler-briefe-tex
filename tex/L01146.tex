%% latex-korrekturansicht-vorspann.tex
%% Vorspann für die Korrekturansicht.
%% Lädt die gemeinsame Datei latex-vorspann.tex mit gesetztem Schalter.

\newif\ifkorrekturansicht
\korrekturansichttrue

\input{../tex-inputs/latex-vorspann}


               \section[Arthur Schnitzler an Richard Beer-Hofmann, 1{[}2?{]}. 7. 1901]{ Arthur Schnitzler an Richard Beer-Hofmann, 1{[}2?{]}. 7. 1901}\nopagebreak\mylabel{v}\rehead{ }\normalsize\beginnumbering\briefempfaengerindex{Beer-Hofmann, Richard@\textsc{Beer-Hofmann, Richard}!zzzSchnitzler, Arthur@\emph{von Arthur Schnitzler}!1901-07-121@{1{[}2?{]}. 7. 1901}|(be} \toendnotes[C]{\smallbreak\pagebreak[2]} \Standort{YCGL, MSS 31.}
\physDesc{Bildpostkarte
\newline{}Handschrift: Bleistift, deutsche Kurrent\newline{}Versand: 1) Stempel: »\nobreak{}\oindex{Innsbruck@\textbf{Innsbruck}, \emph{Besiedelter Ort (A.BSO)}|pwk}I{[}nnbru{]}\textcolor{gray}{ck}, \textcolor{gray}{13}. {[}7. 1901{]}\nobreak{}«.  2) Stempel: »\nobreak{}\oindex{Poertschach@\textbf{Pörtschach}, \emph{https://www.geonames.org/ontologyP.PPL}|pwk}Pörtschach am {[}See{]}, 14/7 01\nobreak{}«. \newline{}Ordnung: mit Bleistift von unbekannter Hand datiert: »14. 7.« }\toendnotes[C]{\smallbreak}\pstart{}{\pb}\strikeout{\textcolor{pink}{N.Oe.}{}\ledrightnote{\textcolor{pink}{Niederösterreich}}}\pend{}\pstart{}Hrn Dr. \textsc{Rich Beer-Hofmann}\pend{}\pstart{}\textcolor{pink}{\textsc{Pörtschach}}{}\ledrightnote{\textcolor{pink}{Pörtschach}}\pend{}\pstart{}\textsc{\textcolor{pink}{Villa Arnstein}{}\ledrightnote{\textcolor{pink}{Villa Arnstein}}}\pend{}{\bigskip}\pstart
           \noindent{}{\pb}\textcolor{gray}{\textbf{\textcolor{pink}{INNSBRUCK}{}\ledrightnote{\textcolor{pink}{Innsbruck}} VON NORD.}}\pend
           \pstart
           Herzlichen Dank für das \label{K_L01146-2v}\edtext{Telegramm}{\lemma{\textnormal{\emph{Telegramm}}}\Cendnote{\textnormal{nicht überliefert}}}\label{K_L01146-2h}. \label{K_L01146-1v}\edtext{Morgen}{\lemma{\textnormal{\emph{Morgen}}}\Cendnote{\textnormal{Das erlaubt die Datierung noch vor dem Poststempel, der,
                  unsicher gelesen, vom 13. 7. 1901 stammen dürfte, da \textcolor{blue}{Schnitzler} bereits an diesem Tag in \textcolor{pink}{Vahrn} ankommt.}}}\label{K_L01146-1h} fahr ich hin.\pend
           \pstart Ihr \spacefill\mbox{A.}\pend{}\endnumbering\briefempfaengerindex{Beer-Hofmann, Richard@\textsc{Beer-Hofmann, Richard}!zzzSchnitzler, Arthur@\emph{von Arthur Schnitzler}!1901-07-121@{1{[}2?{]}. 7. 1901}|)be}\mylabel{h}  \normalsize

\doendnotes{C}
\bigskip
\vfill

\clearpage

\footnotesize

\lohead{\textsc{register}}

% Definiere theindex-Environment komplett neu ohne reledmac
\makeatletter
\renewenvironment{theindex}{%
  \section*{\indexname}%
  \setlength{\parindent}{0pt}%
  \setlength{\parskip}{0pt plus 0.3pt}%
  \let\item\@idxitem
}{%
  \clearpage
}
\makeatother

\IfFileExists{\jobname-pw.ind}{\input{\jobname-pw.ind}}{}

\end{document}

      