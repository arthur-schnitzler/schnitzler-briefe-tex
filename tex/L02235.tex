%% latex-korrekturansicht-vorspann.tex
%% Vorspann für die Korrekturansicht.
%% Lädt die gemeinsame Datei latex-vorspann.tex mit gesetztem Schalter.

\newif\ifkorrekturansicht
\korrekturansichttrue

\input{../tex-inputs/latex-vorspann}


               \section[Arthur Schnitzler an Richard Beer-Hofmann, 30. 7. 1916]{ Arthur Schnitzler an Richard Beer-Hofmann, 30. 7. 1916}\nopagebreak\mylabel{v}\rehead{ }\normalsize\beginnumbering\briefempfaengerindex{Beer-Hofmann, Richard@\textsc{Beer-Hofmann, Richard}!zzzSchnitzler, Arthur@\emph{von Arthur Schnitzler}!1916-07-301@{30. 7. 1916}|(be} \toendnotes[C]{\smallbreak\pagebreak[2]} \Standort{YCGL, MSS 31.}
\physDesc{Bildpostkarte
\newline{}Handschrift: Bleistift, lateinische Kurrent\newline{}Versand: Stempel: »\nobreak{}30. VII. 16\nobreak{}«.  
\newline{}Beer-Hofmann: mit blauem Buntstift Erhalt
                                 festgehalten: »E.« }\toendnotes[C]{\smallbreak}\pstart{}{\pb}\textcolor{pink}{Fischerndorf 79}{}\ledrightnote{\textcolor{pink}{Fischerndorf}}\pend{}\pstart{}Schnitzler\pend{}{\bigskip}\pstart{}Hrn Dr. Richard Beer-Hofmann\pend{}\pstart{}\textcolor{pink}{Bad Ischl}{}\ledrightnote{\textcolor{pink}{Bad Ischl}}.\pend{}\pstart{}\textcolor{pink}{Grazerstr 52}{}\ledrightnote{\textcolor{pink}{Grazer Straße}}\pend{}{\bigskip}\pstart
           \noindent{}\centering{}{\pb}\textcolor{gray}{\textbf{\textcolor{pink}{ALT-AUSSEE}{}\ledrightnote{\textcolor{pink}{Altaussee}} 717 m Seehöhe mit dem \textcolor{pink}{Dachstein}{}\ledrightnote{\textcolor{pink}{Dachstein}}
                     2996 m. \textcolor{pink}{Steier. Salzkammergut}{}\ledrightnote{\textcolor{pink}{Salzkammergut}}.}}\pend
           \pstart
           \raggedleft{}{\pb}30. 7.\pend
           \pstart
           lieber Richard, \label{KLL02235_Beer-Hofmann-1v}\edtext{Mittwoch}{\lemma{\textnormal{\emph{Mittwoch}}}\Cendnote{\textnormal{Es wurde Donnerstag. Vgl. A. S.: \emph{Tagebuch}, 3. 8. 1916}}}\label{KLL02235_Beer-Hofmann-1h} dürften wir bei schönem Wetter in \textcolor{pink}{Ischl}{}\ledrightnote{\textcolor{pink}{Bad Ischl}} sein
               u dort (wahrscheinlich \textcolor{pink}{Kaiserkrone}{}\ledrightnote{\textcolor{pink}{Hotel Kaiserkrone}}) übernachten. Ich
               will zu Fuß hinüber und um 1, ½ 2 in der \textcolor{pink}{Kaiserkr.}{}\ledrightnote{\textcolor{pink}{Hotel Kaiserkrone}} speisen. \textcolor{blue}{Olga}{}\ledrightnote{\textcolor{blue}{Olga Schnitzler}} ko{\geminationm}t erst Nachm.
                  (\textcolor{gray}{Kopfwaschen}, \textcolor{pink}{Aschau}{}\ledrightnote{\textcolor{pink}{Aschau}} etc.)
               Am Abend hoffen wir bei \textcolor{pink}{Sonnenschein}{}\ledrightnote{\textcolor{pink}{Restaurant Sonnenschein}} mit Ihnen \textcolor{blue}{Beiden}{}\ledrightnote{\textcolor{blue}{Paula Beer-Hofmann}} zu nachtmahlen – aber ich suche Sie (binden
               Sie sich nicht!) {\pb}we{\geminationn}s
               irgend geht schon vorher auf. Herzlichst und viele Grüße von uns zu Ihnen.\pend
           \pstart
           Ihr{\\[\baselineskip]}\spacefill\mbox{Arthur}\pend
           \leftskip=0em{}\endnumbering\briefempfaengerindex{Beer-Hofmann, Richard@\textsc{Beer-Hofmann, Richard}!zzzSchnitzler, Arthur@\emph{von Arthur Schnitzler}!1916-07-301@{30. 7. 1916}|)be}\mylabel{h}  \normalsize

\doendnotes{C}
\bigskip
\vfill

\clearpage

\footnotesize

\lohead{\textsc{register}}

% Definiere theindex-Environment komplett neu ohne reledmac
\makeatletter
\renewenvironment{theindex}{%
  \section*{\indexname}%
  \setlength{\parindent}{0pt}%
  \setlength{\parskip}{0pt plus 0.3pt}%
  \let\item\@idxitem
}{%
  \clearpage
}
\makeatother

\IfFileExists{\jobname-pw.ind}{\input{\jobname-pw.ind}}{}

\end{document}

      