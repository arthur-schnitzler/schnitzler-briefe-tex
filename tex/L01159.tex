%% latex-korrekturansicht-vorspann.tex
%% Vorspann für die Korrekturansicht.
%% Lädt die gemeinsame Datei latex-vorspann.tex mit gesetztem Schalter.

\newif\ifkorrekturansicht
\korrekturansichttrue

\input{../tex-inputs/latex-vorspann}


               \section[Arthur Schnitzler an Hugo von Hofmannsthal, 10. 8. 1901]{ Arthur Schnitzler an Hugo von Hofmannsthal,
                    10. 8. 1901}\nopagebreak\mylabel{v}\rehead{ }\normalsize\beginnumbering\briefempfaengerindex{Hofmannsthal, Hugo von@\textsc{Hofmannsthal, Hugo von}!zzzSchnitzler, Arthur@\emph{von Arthur Schnitzler}!1901-08-102@{10. 8. 1901}|(be} \toendnotes[C]{\smallbreak\pagebreak[2]} \Standort{FDH, Hs-30885,96.}
\physDesc{Brief, 1 Blatt, 4 Seiten
\newline{}Handschrift: schwarze Tinte, deutsche Kurrent}\buchAbdrucke{\weitereDrucke{1) Hugo von Hofmannsthal, Arthur Schnitzler: \emph{Briefwechsel}. Hg. Therese Nickl und Heinrich Schnitzler. Frankfurt am Main: \emph{S. Fischer} 1964, S. 150–151.} \weitereDrucke{2) Hermann Bahr, Arthur Schnitzler: \emph{Briefwechsel, Aufzeichnungen, Dokumente
                                (1891–1931)}. Hg. Kurt Ifkovits und Martin Anton Müller. Göttingen: \emph{Wallstein} 2018, S. 215.} }\toendnotes[C]{\smallbreak}\pstart
           \raggedleft{}{\pb}\textcolor{pink}{\textsc{Vahrn}}{}\ledrightnote{\textcolor{pink}{Vahrn}}, 10. 8. 901\pend
           \pstart
           mein lieber Hugo, ſeit vier Wochen bin ich hier, und habe mich,
                    in angenehmer Gesellſchaft, mit Neigung zu Arbeit u\textcolor{gray}{.} einigem
                    Fleiſs und gelegentlichem Talent, in einer wunderbaren Luft, mit Sonne und Wald,
                    recht behaglich gefühlt. Montag reiſ\damage{en} wir nach \textcolor{pink}{Bozen}{}\ledrightnote{\textcolor{pink}{Bozen}}, wo man \textcolor{blue}{Goldma{\geminationn}}{}\ledrightnote{\textcolor{blue}{Paul Goldmann}}
                    trifft, dann nach \textcolor{pink}{Trient}{}\ledrightnote{\textcolor{pink}{Trient}}, und endlich etwa
                        16. 8. gehts nach \textcolor{pink}{\uline{\textsc{Welsberg}} im Puſthertal}{}\ledrightnote{\textcolor{pink}{Welsberg-Taisten}}, \uline{\textcolor{pink}{\textsc{Bad Waldbrunn}}{}\ledrightnote{\textcolor{pink}{Wildbad Waldbrunn}}}, das ich neulich entdeckt habe u von dem ich mich nur wundre {\pb}daſs es kaum bekannt iſt. Ende
                        Auguſt möchte ich in \textcolor{pink}{Wien}{}\ledrightnote{\textcolor{pink}{Wien}}{ }ſein,
                    vor allem \textcolor{green}{2 neue
                        Einakter}{}\ledrightnote{→\textcolor{green}{Lebendige Stunden}{\newline}→\textcolor{green}{Die Frau mit dem Dolche}} dictiren, die der »\textcolor{green}{Literatur}{}\ledrightnote{\textcolor{green}{Literatur}}« vorangehen ſollen. Die drei Stückchen ſind nur durch einen
                    Grundgedanken verbunden, und eines mag immer das andre beleuchten. Auch das
                    dreiaktige \textcolor{green}{Stück}{}\ledrightnote{→\textcolor{green}{Der einsame Weg. Schauspiel in fünf Akten}} kann
                    bald beendet sein.\pend
           \pstart
           Ich freue mich auf einen ſchönen Septemberabend, wo wir einander allerlei
                    erzählen und vorleſen{\pb} können. Um den verlornen
                        \textcolor{pink}{Innsbruck}{}\ledrightnote{\textcolor{pink}{Innsbruck}}er Abend thut es mir ſehr leid.
                    Anonymität wäre übrigens gar nicht vonnöthen geweſen, jeder Grund fehlt,
                    beſonders Ihnen und Ihrer \textcolor{blue}{Frau}{}\ledrightnote{→\textcolor{blue}{Gertrude von Hofmannsthal}} gegenüber. \textcolor{blue}{Wir}{}\ledrightnote{→\textcolor{blue}{Olga Schnitzler}} waren damals an der Bahn, – der andre einzige Ort, wo man \strikeout{\textcolor{gray}{nie}} im Freien speiſen kann, nachdem mir der dritte einzige Ort, in der Nähe
                    der \textcolor{pink}{\textsc{Weierburg}}{}\ledrightnote{\textcolor{pink}{Schloss Weiherburg}}, nicht zuſagte. –\pend
           \pstart
           Viel Freude habe ich heuer wieder vom Radfahren gehabt und mich mehr{\pb} als einmal an unsre Fahrt am Genfer See erinnert, die nun drei Jahre hinter uns
                    liegt.\pend
           \pstart
           Ich höre hoffentlich noch von Ihnen, ehe wir uns wiederſehn\pend
           \pstart
           Herzliche Grüße{\\[\baselineskip]}Ihr{\\[\baselineskip]}\spacefill\mbox{Arthur.}\pend
           \leftskip=0em{}\pstart
           \noindent{}Wenn \textcolor{blue}{Poldi}{}\ledrightnote{\textcolor{blue}{Leopold von Andrian-Werburg}} bei Ihnen iſt, grüßen Sie
                        ihn vielmals. \textcolor{blue}{Michel}{}\ledrightnote{\textcolor{blue}{Robert Michel}} hat mir einen ſo
                        netten Brief geſchrieben. Auch \textcolor{blue}{Bahr}{}\ledrightnote{\textcolor{blue}{Hermann Bahr}},
                        den Sie ja öfters ſehn, grüßen Sie herzlich. Und empfehlen mich Ihrer \textcolor{blue}{Frau}{}\ledrightnote{→\textcolor{blue}{Gertrude von Hofmannsthal}}.{\\}Ihr
                            \spacefill\mbox{A.}\pend
           \endnumbering\briefempfaengerindex{Hofmannsthal, Hugo von@\textsc{Hofmannsthal, Hugo von}!zzzSchnitzler, Arthur@\emph{von Arthur Schnitzler}!1901-08-102@{10. 8. 1901}|)be}\mylabel{h}  \normalsize

\doendnotes{C}
\bigskip
\vfill

\clearpage

\footnotesize

\lohead{\textsc{register}}

% Definiere theindex-Environment komplett neu ohne reledmac
\makeatletter
\renewenvironment{theindex}{%
  \section*{\indexname}%
  \setlength{\parindent}{0pt}%
  \setlength{\parskip}{0pt plus 0.3pt}%
  \let\item\@idxitem
}{%
  \clearpage
}
\makeatother

\IfFileExists{\jobname-pw.ind}{\input{\jobname-pw.ind}}{}

\end{document}

      