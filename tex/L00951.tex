%% latex-korrekturansicht-vorspann.tex
%% Vorspann für die Korrekturansicht.
%% Lädt die gemeinsame Datei latex-vorspann.tex mit gesetztem Schalter.

\newif\ifkorrekturansicht
\korrekturansichttrue

\input{../tex-inputs/latex-vorspann}


               \section[Arthur Schnitzler an Hugo von Hofmannsthal, 27. 7. 1899]{ Arthur Schnitzler an Hugo von Hofmannsthal, 27. 7. 1899}\nopagebreak\mylabel{v}\rehead{ }\normalsize\beginnumbering\briefempfaengerindex{Hofmannsthal, Hugo von@\textsc{Hofmannsthal, Hugo von}!zzzSchnitzler, Arthur@\emph{von Arthur Schnitzler}!1899-07-271@{27. 7. 1899}|(be} \toendnotes[C]{\smallbreak\pagebreak[2]} \Standort{FDH, Hs-30885,85.}
\physDesc{Brief, 1 Blatt, 4 Seiten
\newline{}Handschrift: Bleistift, deutsche Kurrent}\buchAbdrucke{\weitereDrucke{Hugo von Hofmannsthal, Arthur Schnitzler: \emph{Briefwechsel}. Hg. Therese Nickl und Heinrich Schnitzler. Frankfurt am Main: \emph{S. Fischer} 1964, S. 127–128.} }\toendnotes[C]{\smallbreak}\pstart
           \raggedleft{}{\pb}\textcolor{pink}{\textsc{Velden, Pension Pundschu}}{}\ledrightnote{\textcolor{pink}{Pension Pundschu}}{\\}27. 7. 99.\pend
           \pstart
           mein lieber Hugo; etwa am 5. Auguſt{ }ſoll von \textcolor{pink}{\textsc{Toblach}}{}\ledrightnote{\textcolor{pink}{Toblach}} aus die Fußtour angetreten werden, \textcolor{blue}{Richard}{}\ledrightnote{\textcolor{blue}{Richard Beer-Hofmann}}, (der bis dahin mit der \textcolor{green}{Novelle}{}\ledrightnote{→\textcolor{green}{Der Tod Georgs}} fertig iſt und der neulich, in viel beſſrer
                        Sti{\geminationm}g als ich vermuthet, hier war, und den
                    ich So{\geminationn}tag am \textsc{\textcolor{pink}{Millstätter}{}\ledrightnote{\textcolor{pink}{Millstätter See}}}ſee sehe), \textcolor{blue}{\textsc{Wassermann}}{}\ledrightnote{\textcolor{blue}{Jakob Wassermann}}, ich, (am End auch \textcolor{blue}{Rob. Hirſchfeld}{}\ledrightnote{\textcolor{blue}{Robert Hirschfeld}} und
                        we{\geminationn} er ſich dazu entſchließt \textcolor{blue}{Gustav Schwk.}{}\ledrightnote{\textcolor{blue}{Gustav Schwarzkopf}}); \textcolor{pink}{südtiroliſ}{}\ledrightnote{\textcolor{pink}{Südtirol}}che Päſſe, Ende etwa 15. Auguſt in \textcolor{pink}{Trient}{}\ledrightnote{\textcolor{pink}{Trient}}, \textsc{resp}. \textcolor{pink}{Bozen}{}\ledrightnote{\textcolor{pink}{Bozen}}. Zweite {\pb}Hälfte Auguſt verbring ich in \textcolor{pink}{Iſchl}{}\ledrightnote{\textcolor{pink}{Bad Ischl}}. \substVorne{}\textsuperscript{I}\substDazwischen{}S\substHinten{}o käme dann, wie es ja auch Ihnen lieb wäre, unſere \textcolor{pink}{thüringiſche}{}\ledrightnote{\textcolor{pink}{Thüringen}} Radpartie Anfang
                        September. Bleiben wir aber dabei, wenns möglich.\pend
           \pstart
           – Ich habe zu arbeiten begonnen; das \textcolor{green}{Stück}{}\ledrightnote{→\textcolor{green}{Der Schleier der Beatrice. Schauspiel in fünf Akten}}; es war doch weiter als ich gedacht, und wenn
                    ich auch auf der Reiſe arbeiten kann, bin ich im Herbſt am Ende {\pb}fertig. Manchmal ſcheints mir dſs es was werden
                    könnte – oft aber bin ich wie vor den Kopf geſchlagen. Das Gefühl hab ich halt
                    noch immer, dſs ich nicht weiß – für wen eigentlich –?\pend
           \pstart
           – Schreiben Sie mir gleich ein Wort nach \uline{\textcolor{pink}{\textsc{Toblach}, Südbahnhotel}{}\ledrightnote{\textcolor{pink}{Südbahnhotel}}}. Wo werden Sie in der 2. Hälfte Auguſt{ }ſein? Und was Ihr \textcolor{green}{Stück}{}\ledrightnote{→\textcolor{green}{Das Bergwerk zu Falun}} anlangt, ſo darf {\pb}man ja da wirklich ſagen: »Glück auf –«?\pend
           \pstart
           Das Bad hier war prächtig; nun freu ich mich aber, dſs ich wieder woanders
                    hinkomme. \textcolor{blue}{Waſſerm.}{}\ledrightnote{\textcolor{blue}{Jakob Wassermann}}{ }ſchreibt ſeinen \textcolor{green}{Roman}{}\ledrightnote{→\textcolor{green}{Die Geschichte der jungen Renate Fuchs}} ab. –\pend
           \pstart
           – In \textcolor{pink}{\textsc{Tobl}.}{}\ledrightnote{\textcolor{pink}{Toblach}} bin ich noch mit \textcolor{blue}{Mama}{}\ledrightnote{→\textcolor{blue}{Louise Schnitzler}} u \textcolor{blue}{Schweſter}{}\ledrightnote{→\textcolor{blue}{Gisela Hajek}}.\pend
           \pstart Herzlichſt Ihr \spacefill\mbox{Arth}\pend{}\endnumbering\briefempfaengerindex{Hofmannsthal, Hugo von@\textsc{Hofmannsthal, Hugo von}!zzzSchnitzler, Arthur@\emph{von Arthur Schnitzler}!1899-07-271@{27. 7. 1899}|)be}\mylabel{h}  \normalsize

\doendnotes{C}
\bigskip
\vfill

\clearpage

\footnotesize

\lohead{\textsc{register}}

% Definiere theindex-Environment komplett neu ohne reledmac
\makeatletter
\renewenvironment{theindex}{%
  \section*{\indexname}%
  \setlength{\parindent}{0pt}%
  \setlength{\parskip}{0pt plus 0.3pt}%
  \let\item\@idxitem
}{%
  \clearpage
}
\makeatother

\IfFileExists{\jobname-pw.ind}{\input{\jobname-pw.ind}}{}

\end{document}

      