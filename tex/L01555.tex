%% latex-korrekturansicht-vorspann.tex
%% Vorspann für die Korrekturansicht.
%% Lädt die gemeinsame Datei latex-vorspann.tex mit gesetztem Schalter.

\newif\ifkorrekturansicht
\korrekturansichttrue

\input{../tex-inputs/latex-vorspann}


               \section[Arthur Schnitzler an Hugo von Hofmannsthal, 3. 10. 1905]{ Arthur Schnitzler an Hugo von Hofmannsthal, 3. 10. 1905}\nopagebreak\mylabel{v}\rehead{ }\normalsize\beginnumbering\briefempfaengerindex{Hofmannsthal, Hugo von@\textsc{Hofmannsthal, Hugo von}!zzzSchnitzler, Arthur@\emph{von Arthur Schnitzler}!1905-10-032@{3. 10. 1905}|(be} \toendnotes[C]{\smallbreak\pagebreak[2]} \Standort{FDH, Hs-30885,123.}
\physDesc{Brief, 1 Blatt, 4 Seiten
\newline{}Handschrift: schwarze Tinte, deutsche Kurrent}\buchAbdrucke{\weitereDrucke{Hugo von Hofmannsthal, Arthur Schnitzler: \emph{Briefwechsel}. Hg. Therese Nickl und Heinrich Schnitzler. Frankfurt am Main: \emph{S. Fischer} 1964, S. 216.} }\toendnotes[C]{\smallbreak}\pstart
           \raggedleft{}{\pb}\textcolor{pink}{Wien}{}\ledrightnote{\textcolor{pink}{Wien}}{ }3/X 905\pend
           \pstart
           lieber Hugo, den \textcolor{green}{Ruf d. Lebens}{}\ledrightnote{\textcolor{green}{Der Ruf des Lebens. Schauspiel in drei Akten}}
               will ich jetzt gleich drucken laſſen und möchte Ihnen, zu erhöhter Bequemlichkeit der
               Lecture, die Correcturbogen zuſenden. Ich habe mich mit dem \textcolor{green}{3. Akt}{}\ledrightnote{→\textcolor{green}{Der Ruf des Lebens. Schauspiel in drei Akten}} nicht wenig geplagt, und bin eines Tags
               an den Punkt geko{\geminationm}en, wo ich nicht höher konnte. Mir
               iſt, als lägen gewiſſe Schwächen, die es wohl {\pb}auch jetzt
               noch darbietet, mehr im einakts-cycliſchen des Stoffs (worauf Sie ſelbſt ſchon
               hingewieſen haben) als in höchſt meiner Unfähigkeit begründet
               lägen. –\pend
           \pstart
           Hätte ich bezüglich des \textcolor{green}{Zwiſchenſpiels}{}\ledrightnote{\textcolor{green}{Zwischenspiel. Komödie in drei Akten}} auf andrer
               Beſetzung beſtanden, ſo wäre ein Aufſchub, wer weiſs auf wie lang, unvermeidlich
               geweſen. Freuen Sie ſich i{\geminationm}erhin auf \textcolor{blue}{Kainz}{}\ledrightnote{\textcolor{blue}{Josef Kainz}}. \textcolor{blue}{Brahm}{}\ledrightnote{\textcolor{blue}{Otto Brahm}}{ }{\pb}ko{\geminationm}t wahrſcheinlich zur \label{K_L01555_1v}\edtext{\textcolor{green}{\textsc{Première}}{}\ledrightnote{→\textcolor{green}{Zwischenspiel. Komödie in drei Akten}}}{\lemma{\textnormal{\emph{Première}}}\Cendnote{\textnormal{am
                  12. 10. 1905}}}\label{K_L01555_1h} her. –\pend
           \pstart
           Ihre Karte deutet an, dſs man Sie vorläufig nicht ſehen ka{\geminationn}. Hoffentlich aber leſen Sie uns bälder vor. »\textcolor{green}{Jederma{\geminationn}}{}\ledrightnote{\textcolor{green}{Jedermann. Das Spiel vom Sterben des reichen Mannes}}«?\strikeout{«}\pend
           \pstart
           – Donnerſtag nächſter Woche iſt »\textcolor{green}{Zwiſchenſpiel}{}\ledrightnote{\textcolor{green}{Zwischenspiel. Komödie in drei Akten}}«, Samſtag »\textcolor{green}{Kakadu}{}\ledrightnote{\textcolor{green}{Der grüne Kakadu. Groteske in einem Akt}}«. – \pend
           \pstart
           Herzlichst Ihr{\\[\baselineskip]}\spacefill\mbox{A.}\pend
           \leftskip=0em{}\pstart
           \noindent{}Grüßen Sie \textcolor{blue}{Gerty}{}\ledrightnote{\textcolor{blue}{Gertrude von Hofmannsthal}}, und \textcolor{blue}{Richard}{}\ledrightnote{\textcolor{blue}{Richard Beer-Hofmann}}s, die wohl ſchon daheim ſind. Schreiben Sie
                  gelegentlich ein {\pb}Wort, we{\geminationn} man ſchon nicht zuſa{\geminationm}enko{\geminationm}en kann. Ich hab natürlich jetzt täglich Proben.\pend
           \endnumbering\briefempfaengerindex{Hofmannsthal, Hugo von@\textsc{Hofmannsthal, Hugo von}!zzzSchnitzler, Arthur@\emph{von Arthur Schnitzler}!1905-10-032@{3. 10. 1905}|)be}\mylabel{h}  \normalsize

\doendnotes{C}
\bigskip
\vfill

\clearpage

\footnotesize

\lohead{\textsc{register}}

% Definiere theindex-Environment komplett neu ohne reledmac
\makeatletter
\renewenvironment{theindex}{%
  \section*{\indexname}%
  \setlength{\parindent}{0pt}%
  \setlength{\parskip}{0pt plus 0.3pt}%
  \let\item\@idxitem
}{%
  \clearpage
}
\makeatother

\IfFileExists{\jobname-pw.ind}{\input{\jobname-pw.ind}}{}

\end{document}

      