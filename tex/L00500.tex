%% latex-korrekturansicht-vorspann.tex
%% Vorspann für die Korrekturansicht.
%% Lädt die gemeinsame Datei latex-vorspann.tex mit gesetztem Schalter.

\newif\ifkorrekturansicht
\korrekturansichttrue

\input{../tex-inputs/latex-vorspann}


               \section[Jakob Julius David an Arthur Schnitzler, {[}5. 10. 1895{]}]{ Jakob Julius David an Arthur Schnitzler,
                    {[}5. 10. 1895{]}}\nopagebreak\mylabel{v}\rehead{ }\normalsize\beginnumbering\briefempfaengerindex{Schnitzler, Arthur@\textsc{Schnitzler, Arthur}!zzzDavid, Jakob Julius@\emph{von Jakob Julius David}!1895-10-051@{{[}5. 10. 1895{]}}|(be} \toendnotes[C]{\smallbreak\pagebreak[2]} \Standort{CUL, Schnitzler, B 25.}
\physDesc{Briefkarte
\newline{}Handschrift: schwarze Tinte, lateinische Kurrent
\newline{}Schnitzler: mit Bleistift datiert: »5/10 95« \newline{}Ordnung: mit Bleistift von unbekannter Hand nummeriert:
                                        »2.«, von anderer Hand:
                                    »3« }\toendnotes[C]{\smallbreak}\pstart
           \noindent{}{\pb}\textcolor{gray}{\textbf{\textcolor{brown}{Neues Wiener
                                    Journal}{}\ledrightnote{\textcolor{brown}{Neues Wiener Journal}}}}\hfill \textcolor{gray}{\textbf{\textbf{\textcolor{pink}{Wien, IX.}{}\ledrightnote{\textcolor{pink}{IX., Alsergrund}},}
                                den ..........}}\pend
           \pstart
           \textcolor{gray}{\textbf{Herausgeber und
                            Chefradacteur:}}\hfill \textcolor{gray}{\textbf{\textcolor{pink}{Nußdorferſtraße 3}{}\ledrightnote{\textcolor{pink}{Nussdorfer Straße}}.}}\pend
           \pstart
           \textcolor{gray}{\textbf{\textcolor{blue}{\textbf{J. Lippowitz}}{}\ledrightnote{\textcolor{blue}{Jakob Lippowitz}}}}\hfill \textcolor{gray}{\textbf{Telegramm-Adreſſe: Neujournal,
                                \textcolor{pink}{Wien}{}\ledrightnote{\textcolor{pink}{Wien}}.}}\pend
           \pstart
           \textcolor{gray}{\textbf{Telephon Nr. \textbf{7920}.}}\pend
           \pstart\center{}Werther und verehrter Freund!\pend\pstart
           An Ihrem \label{K_L00500_1v}\edtext{\textcolor{green}{Premièren}{}\ledrightnote{→\textcolor{green}{Liebelei. Schauspiel in drei Akten}}tage}{\lemma{\textnormal{\emph{Premièrentage}}}\Cendnote{\textnormal{am 9. 10. 1895}}}\label{K_L00500_1h}
                    veröffentliche ich selbst eine \textcolor{green}{Studie}{}\ledrightnote{→\textcolor{green}{Arthur Schnitzler}} über Sie bei uns. Ist es \label{K_L00500_2v}\edtext{ganz unmöglich}{\lemma{\textnormal{\emph{ganz unmöglich}}}\Cendnote{\textnormal{Offensichtlich, jedenfalls erschien
                        nichts von \textcolor{blue}{Schnitzler} im Vorspann des
                        Texts.}}}\label{K_L00500_2h}, daß Sie mir, sagen wir 100 Zeilen geben,
                    autobiographisch. Sti{\geminationm}ung oder was Sie wollen, die
                    ich voranstellen könnte? Ich werde {\pb}es Ihnen immer danken und es als einen \uline{mir
                        persönlich erwiesenen Dienst} betrachten.\pend
           \pstart
           Waidmannsheil!{\\[\baselineskip]}Herzlichst Ihr{\\[\baselineskip]}\spacefill\mbox{David}\pend
           \leftskip=0em{}\endnumbering\briefempfaengerindex{Schnitzler, Arthur@\textsc{Schnitzler, Arthur}!zzzDavid, Jakob Julius@\emph{von Jakob Julius David}!1895-10-051@{{[}5. 10. 1895{]}}|)be}\mylabel{h}  \normalsize

\doendnotes{C}
\bigskip
\vfill

\clearpage

\footnotesize

\lohead{\textsc{register}}

% Definiere theindex-Environment komplett neu ohne reledmac
\makeatletter
\renewenvironment{theindex}{%
  \section*{\indexname}%
  \setlength{\parindent}{0pt}%
  \setlength{\parskip}{0pt plus 0.3pt}%
  \let\item\@idxitem
}{%
  \clearpage
}
\makeatother

\IfFileExists{\jobname-pw.ind}{\input{\jobname-pw.ind}}{}

\end{document}

      