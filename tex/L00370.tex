%% latex-korrekturansicht-vorspann.tex
%% Vorspann für die Korrekturansicht.
%% Lädt die gemeinsame Datei latex-vorspann.tex mit gesetztem Schalter.

\newif\ifkorrekturansicht
\korrekturansichttrue

\input{../tex-inputs/latex-vorspann}


               \section[Richard Beer-Hofmann an Arthur Schnitzler, 16. 9. 1894]{ Richard Beer-Hofmann an Arthur Schnitzler, 16. 9. 1894}\nopagebreak\mylabel{v}\rehead{ }\normalsize\beginnumbering\briefempfaengerindex{Schnitzler, Arthur@\textsc{Schnitzler, Arthur}!zzzBeer-Hofmann, Richard@\emph{von Richard Beer-Hofmann}!1894-09-161@{16. 9. 1894}|(be} \toendnotes[C]{\smallbreak\pagebreak[2]} \Standort{CUL, Schnitzler, B 8.}
\physDesc{Postkarte
\newline{}Handschrift: blaue Tinte, lateinische Kurrent\newline{}Versand: 1) Stempel: »\nobreak{}\oindex{Bellagio@\textbf{Bellagio}, \emph{Besiedelter Ort (A.BSO)}|pwk}Bellagio (Como), 16 9 94\nobreak{}«.  2) Stempel: »\nobreak{}\oindex{IX., Alsergrund@\textbf{IX., Alsergrund}, \emph{Bezirk (A.BZK)}|pwk}Wien 9/3, 18. 9. 94, 8.V, Bestellt\nobreak{}«. 
\newline{}Schnitzler: mit Bleistift nummeriert: »41« }\buchAbdrucke{\weitereDrucke{Arthur Schnitzler, Richard Beer-Hofmann: \emph{Briefwechsel 1891–1931}. Hg. Konstanze Fliedl. Wien, Zürich: \emph{Europaverlag} 1992, S. 59–60.} }\toendnotes[C]{\smallbreak}\pstart{}{\pb}\textcolor{gray}{\textbf{A}}n Herrn\pend{}\pstart{}D\textsuperscript{r} Arthur Schnitzler\pend{}\pstart{}\textcolor{pink}{Wien}{}\ledrightnote{\textcolor{pink}{Wien}}\pend{}\pstart{}\textcolor{pink}{Austria}{}\ledrightnote{\textcolor{pink}{Österreich}}\pend{}\pstart{}\textcolor{pink}{IX Frankgasse 1}{}\ledrightnote{\textcolor{pink}{Frankgasse}}\pend{}{\bigskip}\pstart
           \noindent{}{\pb}Lieber Arthur! Bin in \textcolor{pink}{Bellagio}{}\ledrightnote{\textcolor{pink}{Bellagio}}, – bis Ende der \label{K_L00370_1v}\edtext{Wochen}{\lemma{\textnormal{\emph{Wochen}}}\Cendnote{\textnormal{öst. dialektal für
                  »Woche«.}}}\label{K_L00370_1h} an den Seen. Wenn Sie was zum Schreiben haben und lieb sind,
               schreiben Sie mir bis Ende der Woche. »\textcolor{pink}{Pallanza}{}\ledrightnote{\textcolor{pink}{Pallanza}}.« »Posta fermata«. Adresse mit lat. Lettern. Anfangsbuchstaben \uline{B} unterstrichen.\pend
           \pstart Herzlichst Ihr\spacefill\mbox{Richard}\pend{}\pstart
           16/IX. 94. \pend
           \pstart
           Grüße an \textcolor{blue}{Hugo}{}\ledrightnote{\textcolor{blue}{Hugo von Hofmannsthal}}{ }\textcolor{blue}{Bahr}{}\ledrightnote{\textcolor{blue}{Hermann Bahr}} etc.\pend
           \pstart
           Entschuldigung an \textcolor{blue}{A. S.}{}\ledrightnote{\textcolor{blue}{Adele Sandrock}} wegen verspäteten
                  Dankes.\pend
           \endnumbering\briefempfaengerindex{Schnitzler, Arthur@\textsc{Schnitzler, Arthur}!zzzBeer-Hofmann, Richard@\emph{von Richard Beer-Hofmann}!1894-09-161@{16. 9. 1894}|)be}\mylabel{h}  \normalsize

\doendnotes{C}
\bigskip
\vfill

\clearpage

\footnotesize

\lohead{\textsc{register}}

% Definiere theindex-Environment komplett neu ohne reledmac
\makeatletter
\renewenvironment{theindex}{%
  \section*{\indexname}%
  \setlength{\parindent}{0pt}%
  \setlength{\parskip}{0pt plus 0.3pt}%
  \let\item\@idxitem
}{%
  \clearpage
}
\makeatother

\IfFileExists{\jobname-pw.ind}{\input{\jobname-pw.ind}}{}

\end{document}

      