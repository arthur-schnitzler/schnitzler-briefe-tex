%% latex-korrekturansicht-vorspann.tex
%% Vorspann für die Korrekturansicht.
%% Lädt die gemeinsame Datei latex-vorspann.tex mit gesetztem Schalter.

\newif\ifkorrekturansicht
\korrekturansichttrue

\input{../tex-inputs/latex-vorspann}


               \section[Ferdinand von Saar an Arthur Schnitzler, 11. 10. 1902]{ Ferdinand von Saar an Arthur Schnitzler, 11. 10. 1902}\nopagebreak\mylabel{v}\rehead{ }\normalsize\beginnumbering\briefempfaengerindex{Schnitzler, Arthur@\textsc{Schnitzler, Arthur}!zzzSaar, Ferdinand von@\emph{von Ferdinand von Saar}!1902-10-111@{11. 10. 1902}|(be} \toendnotes[C]{\smallbreak\pagebreak[2]} \Standort{CUL, Schnitzler, B 88.}
\physDesc{Brief, 1 Blatt, 1 Seite
\newline{}Handschrift: schwarze Tinte, deutsche Kurrent
\newline{}Schnitzler: mit Bleistift nummeriert: »7« }\toendnotes[C]{\smallbreak}\pstart
           \raggedleft{}{\pb}\textcolor{pink}{\textsc{Wien-Döbling}}{}\ledrightnote{\textcolor{pink}{XIX., Döbling}}.{ }11/10. 1902.\pend
           \pstart
           Herzlichen Dank, verehrteſter Poet, für Ihre ſo freundliche \label{K_L01240_1v}\edtext{Kundgebung}{\lemma{\textnormal{\emph{Kundgebung}}}\Cendnote{\textnormal{Zu seinem Geburtstag am 30. 9. 1902 wurde \textcolor{blue}{Saar} eine Adresse verschiedener Schriftsteller
                  und ein Widmungsband überreicht, der von \textcolor{blue}{Schnitzler}{ }\emph{\textcolor{green}{Liebelei. Erstes Bild}} enthielt. (\emph{\textcolor{green}{Widmungen zur Feier des siebzigsten Geburtstages
                        Ferdinand von Saar’s}}. Hg. v. \textcolor{blue}{Richard
                        Specht}. Buchschmuck v. \textcolor{blue}{A. F.
                        Seligmann}. Wien: \emph{\textcolor{brown}{Wiener Verlag}}{ }1903, S. 175–196, vordatiert vom 14. 11. 1902). 
                  Ob hier Adresse oder Widmungsband gemeint ist, ließ sich nicht klären.}}}\label{K_L01240_1h} an
               meinem »70\textsuperscript{ten}«!\pend
           \pstart
           Mit allen guten Wünſchen und{\\[\baselineskip]}in treuer Erinnerung{\\[\baselineskip]}Ihr{\\[\baselineskip]}\spacefill\mbox{Ferdinand von Saar.}\pend
           \leftskip=0em{}\endnumbering\briefempfaengerindex{Schnitzler, Arthur@\textsc{Schnitzler, Arthur}!zzzSaar, Ferdinand von@\emph{von Ferdinand von Saar}!1902-10-111@{11. 10. 1902}|)be}\mylabel{h}  \normalsize

\doendnotes{C}
\bigskip
\vfill

\clearpage

\footnotesize

\lohead{\textsc{register}}

% Definiere theindex-Environment komplett neu ohne reledmac
\makeatletter
\renewenvironment{theindex}{%
  \section*{\indexname}%
  \setlength{\parindent}{0pt}%
  \setlength{\parskip}{0pt plus 0.3pt}%
  \let\item\@idxitem
}{%
  \clearpage
}
\makeatother

\IfFileExists{\jobname-pw.ind}{\input{\jobname-pw.ind}}{}

\end{document}

      