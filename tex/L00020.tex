%% latex-korrekturansicht-vorspann.tex
%% Vorspann für die Korrekturansicht.
%% Lädt die gemeinsame Datei latex-vorspann.tex mit gesetztem Schalter.

\newif\ifkorrekturansicht
\korrekturansichttrue

\input{../tex-inputs/latex-vorspann}


               \section[Fedor Mamroth an Arthur Schnitzler, 21. 6. 1891]{ Fedor Mamroth an Arthur Schnitzler, 21. 6. 1891}\nopagebreak\mylabel{v}\rehead{ }\normalsize\beginnumbering\briefempfaengerindex{Schnitzler, Arthur@\textsc{Schnitzler, Arthur}!zzzMamroth, Fedor@\emph{von Fedor Mamroth}!1891-06-211@{21. 6. 1891}|(be} \toendnotes[C]{\smallbreak\pagebreak[2]} \Standort{CUL, Schnitzler, B 68.}
\physDesc{Brief, 1 Blatt, 2 Seiten
\newline{}Handschrift: blaue Tinte, deutsche Kurrent
\newline{}Schnitzler: 1) mit Bleistift nummeriert: »2.« 2) mit rotem Buntstift eine Unterstreichung}\toendnotes[C]{\smallbreak}\pstart
           \noindent{}{\pb}\textcolor{brown}{\textcolor{gray}{\textbf{\textsc{Frankfurter Zeitung}}}{\\}\textcolor{gray}{\textbf{\textsc{und}}}{\\}\textcolor{gray}{\textbf{\textsc{Handelsblatt.}}}}{}\ledrightnote{\textcolor{brown}{Frankfurter Zeitung}}\pend
           \pstart
           \textcolor{gray}{\textbf{\textsc{Redaction.}}}\hfill \textcolor{gray}{\textbf{\textsc{\textcolor{pink}{Frankfurt a. M.}{}\ledrightnote{\textcolor{pink}{Frankfurt am Main}},}}}{ }21. Juni. \textcolor{gray}{\textbf{\textsc{189}}}1\pend
           \pstart
           \textcolor{gray}{\textbf{\textsc{Telegramm-Adresse:}}}\pend
           \pstart
           \textcolor{gray}{\textbf{\textsc{Zeitung Frankfurt Main.}}}\pend
           \pstart{}Hochgeehrter Herr Doctor!\pend\pstart
           Mit aufrichtigem Vergnügen las ich Ihre »\textcolor{green}{Drei
                  Elixire}{}\ledrightnote{\textcolor{green}{Die drei Elixire}}« und ich verſage es mir ungern, Ihnen eine Menge ſchöner Dinge
               darüber zu ſagen, weil ich in der Hauptſache weder Ihren noch meinen Wünſchen zu
               entſprechen vermag. Vermutlich wird die \textcolor{green}{Frankf.
                  Ztg.}{}\ledrightnote{\textcolor{green}{Frankfurter Zeitung}} im Jahre 1920 eine Arbeit dieſer Art veröffentlichen
               dürfen, ohne Straßenkämpfe hervorzurufen. Namens unſeres Publikums danke ich Ihnen
               für die Überſchätzung, die Sie ſeinem Niveau zu teil werden laſſen. Außer \textcolor{blue}{Brahm}{}\ledrightnote{\textcolor{blue}{Otto Brahm}}’s »\textcolor{green}{Freier
                  Bühne}{}\ledrightnote{\textcolor{green}{Freie Bühne für modernes Leben}}« wüßte ich auch kein deutſches Blatt, das dieſe reizende Dichtung
               veröffentlichen könnte. Es ſei denn, Sie überſetzten ſie ins Franzöſiſche u ſchickten
               ſie dem »\textcolor{brown}{\textsc{Echo de Paris}}{}\ledrightnote{\textcolor{brown}{L’Écho de Paris}}« oder dem »\textcolor{brown}{\textsc{Gil Blas}}{}\ledrightnote{\textcolor{brown}{Gil Blas}}«, – dann könnte ſie vielleicht \label{K_L00020_1v}\edtext{von dort aus den Weg}{\lemma{\textnormal{\emph{von dort aus den Weg}}}\Cendnote{\textnormal{Anspielung auf den
                  in \textcolor{pink}{Deutschland} kaum rezipierten Roman von \textcolor{blue}{Karl Bleibtreu}: \emph{\textcolor{green}{Dies Irae. Erinnerungen eines französischen Offiziers an die
                        Tage von Sedan}}. Stuttgart: \emph{\textcolor{brown}{Krabbe}}{ }1882, dessen vielbeachtete französische Übersetzung für das Original gehalten
                  und ins Deutsche rückübersetzt wurde.}}}\label{K_L00020_1h}{ }{\pb}nach \textcolor{pink}{Deutſchland}{}\ledrightnote{\textcolor{pink}{Deutschland}} finden. – – –  \textcolor{blue}{Paul}{}\ledrightnote{\textcolor{blue}{Paul Goldmann}}{ }ſcheint es gut zu gehen; ſeine Privatberichte ſind
               zumeiſt ſo mißgeſti{\geminationm}t, daß ich überzeugt bin, es gefalle
               ihm in \textcolor{pink}{Brüſſel}{}\ledrightnote{\textcolor{pink}{Brüssel}} ganz ausgezeichnet. Laſſen Sie mich
               hoffen, daß es Ihnen mindeſtens ebenſo gut gehe u empfangen Sie meine herzlichſten
               Grüße.\pend
           \pstart
           Ihr ergebener{\\[\baselineskip]}\spacefill\mbox{FMamroth}\pend
           \leftskip=0em{}\endnumbering\briefempfaengerindex{Schnitzler, Arthur@\textsc{Schnitzler, Arthur}!zzzMamroth, Fedor@\emph{von Fedor Mamroth}!1891-06-211@{21. 6. 1891}|)be}\mylabel{h}  \normalsize

\doendnotes{C}
\bigskip
\vfill

\clearpage

\footnotesize

\lohead{\textsc{register}}

% Definiere theindex-Environment komplett neu ohne reledmac
\makeatletter
\renewenvironment{theindex}{%
  \section*{\indexname}%
  \setlength{\parindent}{0pt}%
  \setlength{\parskip}{0pt plus 0.3pt}%
  \let\item\@idxitem
}{%
  \clearpage
}
\makeatother

\IfFileExists{\jobname-pw.ind}{\input{\jobname-pw.ind}}{}

\end{document}

      