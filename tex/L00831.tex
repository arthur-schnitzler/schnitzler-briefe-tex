%% latex-korrekturansicht-vorspann.tex
%% Vorspann für die Korrekturansicht.
%% Lädt die gemeinsame Datei latex-vorspann.tex mit gesetztem Schalter.

\newif\ifkorrekturansicht
\korrekturansichttrue

\input{../tex-inputs/latex-vorspann}


               \section[Hugo von Hofmannsthal an Arthur Schnitzler, 6. 8. {[}1898{]}]{ Hugo von Hofmannsthal an Arthur Schnitzler, 6. 8. {[}1898{]}}\nopagebreak\mylabel{v}\rehead{ }\normalsize\beginnumbering\briefempfaengerindex{Schnitzler, Arthur@\textsc{Schnitzler, Arthur}!zzzHofmannsthal, Hugo von@\emph{von Hugo von Hofmannsthal}!1898-08-061@{6. 8. {[}1898{]}}|(be} \toendnotes[C]{\smallbreak\pagebreak[2]} \Standort{CUL, Schnitzler, B 43.}
\physDesc{Brief, 1 Blatt, 4 Seiten
\newline{}Handschrift: schwarze Tinte, deutsche Kurrent
\newline{}Schnitzler: mit Bleistift die Jahreszahl ergänzt: »98« \newline{}Ordnung: 1) mit Bleistift von unbekannter Hand nummeriert: »\strikeout{133}« 2) mit Bleistift von unbekannter Hand nummeriert: »119a«}\buchAbdrucke{\weitereDrucke{Hugo von Hofmannsthal, Arthur Schnitzler: \emph{Briefwechsel}. Hg. Therese Nickl und Heinrich Schnitzler. Frankfurt am Main: \emph{S. Fischer} 1964, S. 109.} }\toendnotes[C]{\smallbreak}\pstart
           \raggedleft{}{\pb}\textcolor{pink}{Brühl}{}\ledrightnote{\textcolor{pink}{Hinterbrühl}}{ }6\textsc{\textsuperscript{ten}} VIII.\pend
           \pstart{}mein lieber Arthur\pend\pstart
           auf meinen letzten Brief \introOben{}nach \textcolor{pink}{Tegernſee}{}\ledrightnote{\textcolor{pink}{Tegernsee}}\introOben{} bin ich noch ohne Antwort, aber gar nicht beunruhigend, da ja Ihr letzter
                    die Verſicherung enthielt, daſs Ihnen unſer Rendezvous
                        10–15 recht iſt. Nun fange ich an mich ſchon ſehr
                    nach dem Arbeiten zu ſehnen und mit den Tagen geizig zu ſein.\pend
           \pstart
           {\pb}Ich möchte daher ſchon
                        Mittwoch d. 10\textsc{\textsuperscript{ten}} vormittag (circa 10\textsc{\textsuperscript{h}} glaub ich) von \textcolor{pink}{Zell am See}{}\ledrightnote{\textcolor{pink}{Zell am See}} her in \textcolor{pink}{Innsbruck}{}\ledrightnote{\textcolor{pink}{Innsbruck}} anko{\geminationm}en.
                    Werden Sie da ſchon dort ſein? und am \textcolor{pink}{Bahnhof}{}\ledrightnote{→\textcolor{pink}{Hauptbahnhof}} oder wo treffen wir uns? Ich nehme an daſs
                    wir am ſelben Tag weiterfahren gegen \textcolor{pink}{Bregenz}{}\ledrightnote{\textcolor{pink}{Bregenz}}.
                    Sollte es practiſch ſein mit demſelben {\pb}Zug weiterzufahren, in dem
                    ich anko{\geminationm}e, ſo müſsten Sie mich natürlich auch das
                    wiſſen laſſen. Ich reiſe Montag 8\textsc{\textsuperscript{ten}} von \textcolor{pink}{Wien}{}\ledrightnote{\textcolor{pink}{Wien}} abends ab, bin 9\textsc{\textsuperscript{ten}}{ }früh bis 9\textsc{\textsuperscript{ten}}{ }abends{ }\textcolor{pink}{Bad Fuſch}{}\ledrightnote{\textcolor{pink}{Bad Fusch}}. Entweder ſchreiben Sie alſo umgehend
                    in die \textcolor{pink}{Fuſch}{}\ledrightnote{\textcolor{pink}{Bad Fusch}} oder was mir noch lieber wäre {\pb}telegrafieren in die \textcolor{pink}{Saleſianergaſſe}{}\ledrightnote{\textcolor{pink}{Salesianergasse}} (am Montag) das
                    Dringendſte, ob Sie Mittwoch{ }\textcolor{pink}{Innsbruck}{}\ledrightnote{\textcolor{pink}{Innsbruck}} und wo.\pend
           \pstart
           Von Herzen Ihr{\\[\baselineskip]}\spacefill\mbox{Hugo.}\pend
           \leftskip=0em{}\endnumbering\briefempfaengerindex{Schnitzler, Arthur@\textsc{Schnitzler, Arthur}!zzzHofmannsthal, Hugo von@\emph{von Hugo von Hofmannsthal}!1898-08-061@{6. 8. {[}1898{]}}|)be}\mylabel{h}  \normalsize

\doendnotes{C}
\bigskip
\vfill

\clearpage

\footnotesize

\lohead{\textsc{register}}

% Definiere theindex-Environment komplett neu ohne reledmac
\makeatletter
\renewenvironment{theindex}{%
  \section*{\indexname}%
  \setlength{\parindent}{0pt}%
  \setlength{\parskip}{0pt plus 0.3pt}%
  \let\item\@idxitem
}{%
  \clearpage
}
\makeatother

\IfFileExists{\jobname-pw.ind}{\input{\jobname-pw.ind}}{}

\end{document}

      