%% latex-korrekturansicht-vorspann.tex
%% Vorspann für die Korrekturansicht.
%% Lädt die gemeinsame Datei latex-vorspann.tex mit gesetztem Schalter.

\newif\ifkorrekturansicht
\korrekturansichttrue

\input{../tex-inputs/latex-vorspann}


               \section[Richard Beer-Hofmann an Arthur Schnitzler, 13. 9. 1895]{ Richard Beer-Hofmann an Arthur Schnitzler,
               13. 9. 1895}\nopagebreak\mylabel{v}\rehead{ }\normalsize\beginnumbering\briefempfaengerindex{Schnitzler, Arthur@\textsc{Schnitzler, Arthur}!zzzBeer-Hofmann, Richard@\emph{von Richard Beer-Hofmann}!1895-09-131@{13. 9. 1895}|(be} \toendnotes[C]{\smallbreak\pagebreak[2]} \Standort{CUL, Schnitzler, B 8.}
\physDesc{Briefkarte
\newline{}Handschrift: Bleistift, lateinische Kurrent
\newline{}Schnitzler: mit Bleistift nummeriert: »69« }\buchAbdrucke{\weitereDrucke{Arthur Schnitzler, Richard Beer-Hofmann: \emph{Briefwechsel 1891–1931}. Hg. Konstanze Fliedl. Wien, Zürich: \emph{Europaverlag} 1992, S. 80.} }\toendnotes[C]{\smallbreak}\pstart
           \raggedleft{}{\pb}\textcolor{pink}{Schönberg}{}\ledrightnote{\textcolor{pink}{Schönberg im Stubaital}}{ }13 Sept 95\pend
           \pstart
           Lieber Arthur! Bitte um den ausführlichen Brief. Frau \textcolor{blue}{Lou}{}\ledrightnote{\textcolor{blue}{Lou Andreas-Salomé}} erwidert Grüße etc. Von morgen früh an bin ich
               allein!!! Ich bleibe hier solange es schön ist – ich arbeite hier sehr gut – dann
               gehe ich etwas südlicher. \textcolor{pink}{Bozen}{}\ledrightnote{\textcolor{pink}{Bozen}} oder \textcolor{pink}{Riva}{}\ledrightnote{\textcolor{pink}{Riva del Garda}}. Sie haben mich falsch verstanden; nicht
                  Ende Oktober, Ende Sept. will ich in \textcolor{pink}{Wien}{}\ledrightnote{\textcolor{pink}{Wien}}
                sein\pend
           \pstart
           {\pb}Was macht \textcolor{blue}{Hugo}{}\ledrightnote{\textcolor{blue}{Hugo von Hofmannsthal}}? Grüßen Sie \textcolor{blue}{Salten}{}\ledrightnote{\textcolor{blue}{Felix Salten}}{ }\textcolor{blue}{Schwarzkopf}{}\ledrightnote{\textcolor{blue}{Gustav Schwarzkopf}}, \textcolor{blue}{Sokal}{}\ledrightnote{\textcolor{blue}{Clemens Sokal}} – genug. Momentan ist es kalt aber schön. Im übrigen teile ich Ihnen mit daß es am schönsten
               ist \uline{allein} zu reisen. Uns Zwei \introOben{}(Mich und Sie!)\introOben{} und \textcolor{blue}{Hugo}{}\ledrightnote{\textcolor{blue}{Hugo von Hofmannsthal}} ausgeno{\geminationm}en. \textcolor{blue}{Paul}{}\ledrightnote{\textcolor{blue}{Paul Goldmann}} leidet
               zuviel an Familie. Mein \textcolor{blue}{Papa}{}\ledrightnote{→\textcolor{blue}{Hermann Beer}}
               hat einen herrlichen Brief geschrieben. Ich zeig ihn Ihnen in \textcolor{pink}{Wien}{}\ledrightnote{\textcolor{pink}{Wien}}. Herzlichst Ihr\pend
           \pstart \spacefill\mbox{R.}\pend{}\endnumbering\briefempfaengerindex{Schnitzler, Arthur@\textsc{Schnitzler, Arthur}!zzzBeer-Hofmann, Richard@\emph{von Richard Beer-Hofmann}!1895-09-131@{13. 9. 1895}|)be}\mylabel{h}  \normalsize

\doendnotes{C}
\bigskip
\vfill

\clearpage

\footnotesize

\lohead{\textsc{register}}

% Definiere theindex-Environment komplett neu ohne reledmac
\makeatletter
\renewenvironment{theindex}{%
  \section*{\indexname}%
  \setlength{\parindent}{0pt}%
  \setlength{\parskip}{0pt plus 0.3pt}%
  \let\item\@idxitem
}{%
  \clearpage
}
\makeatother

\IfFileExists{\jobname-pw.ind}{\input{\jobname-pw.ind}}{}

\end{document}

      