%% latex-korrekturansicht-vorspann.tex
%% Vorspann für die Korrekturansicht.
%% Lädt die gemeinsame Datei latex-vorspann.tex mit gesetztem Schalter.

\newif\ifkorrekturansicht
\korrekturansichttrue

\input{../tex-inputs/latex-vorspann}


               \section[Arthur Schnitzler an Hugo von Hofmannsthal, 18. 7. 1898]{ Arthur Schnitzler an Hugo von Hofmannsthal, 18. 7. 1898}\nopagebreak\mylabel{v}\rehead{ }\normalsize\beginnumbering\briefempfaengerindex{Hofmannsthal, Hugo von@\textsc{Hofmannsthal, Hugo von}!zzzSchnitzler, Arthur@\emph{von Arthur Schnitzler}!1898-07-182@{18. 7. 1898}|(be} \toendnotes[C]{\smallbreak\pagebreak[2]} \Standort{FDH, Hs-30885,71.}
\physDesc{Bildpostkarte
\newline{}Handschrift: Bleistift, deutsche Kurrent\newline{}Versand: Stempel: »\nobreak{}\oindex{Tschortkiw@\textbf{Tschortkiw}, \emph{https://www.geonames.org/ontologyP.PPLA2}|pwk}Czortków, 20 7. 98, X\nobreak{}«.  \newline{}Ordnung: von Schnitzler mit Bleistift mutmaßlich bei der
                                            Durchsicht der Briefe 1929 zweimal mit dem
                                            Datum des Stempels datiert: »20/7 98« }\buchAbdrucke{\weitereDrucke{Hugo von Hofmannsthal, Arthur Schnitzler: \emph{Briefwechsel}. Hg. Therese Nickl und Heinrich Schnitzler. Frankfurt am Main: \emph{S. Fischer} 1964, S. 107.} }\toendnotes[C]{\smallbreak}\pstart{}{\pb}Herrn Hugo von Hofmannsthal\pend{}\pstart{}\textsc{kuk Ltnd i. d. R. des kuk VIII. Ulan-Rgmts}\pend{}\pstart{}\textsc{\textcolor{pink}{Czortków}{}\ledrightnote{\textcolor{pink}{Tschortkiw}}}\pend{}\pstart{}\textsc{\textcolor{pink}{Galizien}{}\ledrightnote{\textcolor{pink}{Galizien}}}\pend{}{\bigskip}\pstart
           \noindent{}\centering{}\textcolor{gray}{\textbf{{\pb}\textcolor{pink}{Liechtensteinklamm}{}\ledrightnote{\textcolor{pink}{Liechtensteinklamm}}}}\pend
           \pstart
           \centering{}\textcolor{gray}{\textbf{Gruss aus der \textcolor{pink}{Liechtensteinklamm}{}\ledrightnote{\textcolor{pink}{Liechtensteinklamm}}}}\pend
           \pstart
           \noindent{}Schöne Radtour: geſtern \introOben{}Nachm\introOben{}{ }\textcolor{pink}{Steinach}{}\ledrightnote{\textcolor{pink}{Steinach}} bis \textcolor{pink}{Schladming}{}\ledrightnote{\textcolor{pink}{Schladming}}; \label{K_L00823_1v}\edtext{heute}{\lemma{\textnormal{\emph{heute}}}\Cendnote{\textnormal{Die beschriebenen Ausflüge und das
                        Treffen mit den \textcolor{blue}{Eltern}{ }\textcolor{blue}{Hofmannsthal}s erlauben die Datierung auf
                        den 18. 7. 1898.}}}\label{K_L00823_1h}{ }Vormitt{ }\textcolor{pink}{Schladming}{}\ledrightnote{\textcolor{pink}{Schladming}}, bis zur \textcolor{pink}{Liechtenſteinkla{\geminationm}}{}\ledrightnote{\textcolor{pink}{Liechtensteinklamm}}; heut abends werd ich wohl in der \textcolor{pink}{Fuſch}{}\ledrightnote{\textcolor{pink}{Bad Fusch}}
                    Ihre \textcolor{blue}{Eltern}{}\ledrightnote{→\textcolor{blue}{Hugo August von Hofmannsthal}{\newline}→\textcolor{blue}{Anna von Hofmannsthal}}{ }ſehn. Seien Sie herzlich gegrüßt. Ihr
                        \spacefill\mbox{Arth.}\pend
           \endnumbering\briefempfaengerindex{Hofmannsthal, Hugo von@\textsc{Hofmannsthal, Hugo von}!zzzSchnitzler, Arthur@\emph{von Arthur Schnitzler}!1898-07-182@{18. 7. 1898}|)be}\mylabel{h}  \normalsize

\doendnotes{C}
\bigskip
\vfill

\clearpage

\footnotesize

\lohead{\textsc{register}}

% Definiere theindex-Environment komplett neu ohne reledmac
\makeatletter
\renewenvironment{theindex}{%
  \section*{\indexname}%
  \setlength{\parindent}{0pt}%
  \setlength{\parskip}{0pt plus 0.3pt}%
  \let\item\@idxitem
}{%
  \clearpage
}
\makeatother

\IfFileExists{\jobname-pw.ind}{\input{\jobname-pw.ind}}{}

\end{document}

      