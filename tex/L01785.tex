%% latex-korrekturansicht-vorspann.tex
%% Vorspann für die Korrekturansicht.
%% Lädt die gemeinsame Datei latex-vorspann.tex mit gesetztem Schalter.

\newif\ifkorrekturansicht
\korrekturansichttrue

\input{../tex-inputs/latex-vorspann}


               \section[Hugo von Hofmannsthal an Arthur Schnitzler, 24. 7. {[}1908{]}]{ Hugo von Hofmannsthal an Arthur Schnitzler, 24. 7. {[}1908{]}}\nopagebreak\mylabel{v}\rehead{ }\normalsize\beginnumbering\briefempfaengerindex{Schnitzler, Arthur@\textsc{Schnitzler, Arthur}!zzzHofmannsthal, Hugo von@\emph{von Hugo von Hofmannsthal}!1908-07-241@{24. 7. {[}1908{]}}|(be} \toendnotes[C]{\smallbreak\pagebreak[2]} \Standort{CUL, Schnitzler, B 43.}
\physDesc{Brief, 2 Blätter (Das zweite Blatt mit »2« beschriftet), 8 Seiten
\newline{}Handschrift: schwarze Tinte, deutsche Kurrent
\newline{}Schnitzler: mit Bleistift die Jahreszahl ergänzt: »08« und beschriftet: »Hugo v H.« \newline{}Ordnung: 1) mit Bleistift von unbekannter Hand nummeriert: »\strikeout{294}« 2) mit Bleistift von unbekannter Hand nummeriert: »299«}\buchAbdrucke{\weitereDrucke{Hugo von Hofmannsthal, Arthur Schnitzler: \emph{Briefwechsel}. Hg. Therese Nickl und Heinrich Schnitzler. Frankfurt am Main: \emph{S. Fischer} 1964, S. 237.} }\toendnotes[C]{\smallbreak}\pstart
           \raggedleft{}{\pb}\textcolor{pink}{Bad Fuſch}{}\ledrightnote{\textcolor{pink}{Bad Fusch}}{ }24\textsuperscript{ten} VII.\pend
           \pstart{}mein lieber Arthur\pend\pstart
           ich habe dieſe 14 Tage hier ſo viel gearbeitet, gedacht, notiert daſs ich wirklich
               außer kleinen Karten an \textcolor{blue}{Gerty}{}\ledrightnote{\textcolor{blue}{Gertrude von Hofmannsthal}} und meinen \textcolor{blue}{Vater}{}\ledrightnote{→\textcolor{blue}{Hugo August von Hofmannsthal}} nichts Briefartiges habe
               ſchreiben können und wollen, ſchon aus Angſt vor einem Überſpannen und
                  Nicht-ſchlafen.\hspace*{1.5em}Übermorgen kommt \textcolor{blue}{Gerty}{}\ledrightnote{\textcolor{blue}{Gertrude von Hofmannsthal}} mir nach und {\pb}wir fahren nach \textcolor{pink}{\textsc{Sils}}{}\ledrightnote{\textcolor{pink}{Sils im Engadin}}. Dort hoffe ich nicht nur mit dieſer \textcolor{green}{Comödie}{}\ledrightnote{→\textcolor{green}{Der Rosenkavalier}} fertig zu werden, ſondern auch ein anderes, kurzes \textcolor{green}{Stück}{}\ledrightnote{→\textcolor{green}{Der Mann von fünfzig Jahren}}, das mir mit
               zudringlicher Lebhaftigkeit vorſchwebt, zum mindeſten anzufangen. \hspace*{1.5em}Statt nach \textcolor{pink}{\textsc{Sils}}{}\ledrightnote{\textcolor{pink}{Sils im Engadin}} könnten wir doch ganz wohl auch dorthin kommen wo {\pb}Ihr ſeid – ich meine: »hätten wir
               können.« Es iſt eine Schrulle von mir daß wenn jemand wie Sie nach dem ich mich gerne
               richte, einen Plan ausſpricht, wie Sie im Winter den, in die \textcolor{pink}{Schweiz}{}\ledrightnote{\textcolor{pink}{Schweiz}} zu gehen – ich mich ſo daran halte als ob es etwas ganz
               Feſtes wäre. Auf dieſe Weiſe habe ich in \textcolor{pink}{\textsc{Sils}}{}\ledrightnote{\textcolor{pink}{Sils im Engadin}} gemiethet – um eine Begegnung mit Euch {\pb}bequem zu haben. Dann im
                  Mai wäre dieſe Sache wohl noch rückgängig zu machen geweſen, da hat
               aber meinen Willen und meine Luſt etwas anderes gelähmt: ich meine mein gar nicht
               glückliches Verhältnis zu Ihrem \textcolor{green}{Roman}{}\ledrightnote{→\textcolor{green}{Der Weg ins Freie. Roman}}. Da ich Sie eben ſehr gerne habe, und zwiſchen Ihnen und Ihren
               Arbeiten natürlich keine Grenze ziehen kann, ſo hat mich dies {\pb}durch einige Wochen ſehr verſtört.
               Es wäre mir ebenſo qualvoll geweſen, darüber reden zu müſſen, als es mir peinlich
               war, \strikeout{darüber} zu ſchweigen.\pend
           \pstart
           Jetzt bin ich darüber ruhiger geworden, und ich erwähne es jetzt abſichtlich, weil
               Ihnen ja doch mein Schweigen aufgefallen ſein muſs.\pend
           \pstart
           Jetzt macht es mir gar nichts, {\pb}entweder niemals darüber zu reden oder doch zu reden, wenn es sich einmal
               ergibt.\pend
           \pstart
           \numberlinefalse{}–\numberlinetrue{}\pend
           \pstart
           Ich bin ſo begierig was Sie machen.\hspace*{1.5em}Bitte ſchreiben
               Sie mir ein paar Zeilen, oder es ſchreibt vielleicht \textcolor{blue}{Olga}{}\ledrightnote{\textcolor{blue}{Olga Schnitzler}} an \textcolor{blue}{Gerty}{}\ledrightnote{\textcolor{blue}{Gertrude von Hofmannsthal}}.\pend
           \pstart
           Von Herzen Ihr{\\[\baselineskip]}\spacefill\mbox{Hugo}\pend
           \leftskip=0em{}\pstart
           \noindent{}PS. Habe, um unter vielen Büchern auch etwas von Ihnen mitzuhaben, den »\textcolor{green}{einſamen Weg}{}\ledrightnote{\textcolor{green}{Der einsame Weg. Schauspiel in fünf Akten}}« mitgenommen und ihn auf einem
                  Spaziergang mit großer Freude vom Anfang zum Ende \substVorne{}\textsuperscript{geſehen.}{\allowbreak}\substDazwischen{}geleſen.\substHinten{}\hspace*{1.5em}Es iſt doch für eine zweite Periode ihres
                  Schaffens ebenſo ſchön und bedeutend, als {\pb}»\textcolor{green}{Liebelei}{}\ledrightnote{\textcolor{green}{Liebelei. Schauspiel in drei Akten}}« für eine erſte.\pend
           \pstart
           Unſere Adreſſe:\pend
           \leftskip=3em{}\pstart
           \noindent{}\textcolor{pink}{\textsc{Sils Maria im Engadin}}{}\ledrightnote{\textcolor{pink}{Sils im Engadin}}\pend
           \leftskip=0em{}\leftskip=3em{}\pstart
           \textcolor{pink}{\textsc{\uline{Hôtel Alpenrose}}}{}\ledrightnote{\textcolor{pink}{Hotel Alpenrose}}.\pend
           \leftskip=0em{}\endnumbering\briefempfaengerindex{Schnitzler, Arthur@\textsc{Schnitzler, Arthur}!zzzHofmannsthal, Hugo von@\emph{von Hugo von Hofmannsthal}!1908-07-241@{24. 7. {[}1908{]}}|)be}\mylabel{h}  \normalsize

\doendnotes{C}
\bigskip
\vfill

\clearpage

\footnotesize

\lohead{\textsc{register}}

% Definiere theindex-Environment komplett neu ohne reledmac
\makeatletter
\renewenvironment{theindex}{%
  \section*{\indexname}%
  \setlength{\parindent}{0pt}%
  \setlength{\parskip}{0pt plus 0.3pt}%
  \let\item\@idxitem
}{%
  \clearpage
}
\makeatother

\IfFileExists{\jobname-pw.ind}{\input{\jobname-pw.ind}}{}

\end{document}

      