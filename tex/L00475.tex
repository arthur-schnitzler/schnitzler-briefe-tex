%% latex-korrekturansicht-vorspann.tex
%% Vorspann für die Korrekturansicht.
%% Lädt die gemeinsame Datei latex-vorspann.tex mit gesetztem Schalter.

\newif\ifkorrekturansicht
\korrekturansichttrue

\input{../tex-inputs/latex-vorspann}


               \section[Arthur Schnitzler an Richard Beer-Hofmann, 19. 8. 1895]{ Arthur Schnitzler an Richard Beer-Hofmann, 19. 8. 1895}\nopagebreak\mylabel{v}\rehead{ }\normalsize\beginnumbering\briefempfaengerindex{Beer-Hofmann, Richard@\textsc{Beer-Hofmann, Richard}!zzzSchnitzler, Arthur@\emph{von Arthur Schnitzler}!1895-08-191@{19. 8. 1895}|(be} \toendnotes[C]{\smallbreak\pagebreak[2]} \Standort{YCGL, MSS 31.}
\physDesc{Telegramm
\newline{}Handschrift einer Schreibkraft: blaue Tinte, deutsche Kurrent\newline{}Versand: »\noindent{}\textcolor{gray}{\textbf{Aufgegeben am}}{ }19/8 \textcolor{gray}{\textbf{18}}95{ }\textcolor{gray}{\textbf{um}}{ }3 \textcolor{gray}{\textbf{Uhr}} 20 \textcolor{gray}{\textbf{Min.}} N \textcolor{gray}{\textbf{Mittag}}{ / }\textcolor{gray}{\textbf{Eingelangt von}} S \textcolor{gray}{\textbf{auf Leitung Nr.}} 587 \textcolor{gray}{\textbf{am}}{ }19/8\textcolor{gray}{\textbf{189}}5{ }\textcolor{gray}{\textbf{um}}{ }4 \textcolor{gray}{\textbf{Uhr}} 20 \textcolor{gray}{\textbf{Min. {\dots} Mittag}}{ / }\textcolor{gray}{\textbf{Aufgenommen durch}}{ }\textcolor{gray}{K}{ / }\textcolor{gray}{\textbf{Von}}{ }\textcolor{pink}{Salzburg}{ }\textcolor{gray}{\textbf{mit}} 1099\textcolor{gray}{P}{ }\textcolor{gray}{\textbf{Taxworten (}}21 \textcolor{gray}{\textbf{Worten {\dots} Chiffern)}}« }\pstart{}{\pb}Richard Beer-Hoffmann\pend{}\pstart{}\textcolor{pink}{Eglmosgasse 22}{}\ledrightnote{\textcolor{pink}{Eglmoosgasse}}\pend{}\pstart{}\textcolor{pink}{\textcolor{gray}{\textbf{\textit{Ischl}}}}{}\ledrightnote{\textcolor{pink}{Bad Ischl}}\pend{}{\bigskip}\pstart
           \noindent{}{\pb}Kommen Sie doch morgen abend damit wir wenigſtens
               ganzen Mittwoch Alle zuſa{\geminationm}en ſind.\pend
           \pstart
           Herzlichſt{\\[\baselineskip]}\spacefill\mbox{Arthur}\pend
           \leftskip=0em{}\endnumbering\briefempfaengerindex{Beer-Hofmann, Richard@\textsc{Beer-Hofmann, Richard}!zzzSchnitzler, Arthur@\emph{von Arthur Schnitzler}!1895-08-191@{19. 8. 1895}|)be}\mylabel{h}  \normalsize

\doendnotes{C}
\bigskip
\vfill

\clearpage

\footnotesize

\lohead{\textsc{register}}

% Definiere theindex-Environment komplett neu ohne reledmac
\makeatletter
\renewenvironment{theindex}{%
  \section*{\indexname}%
  \setlength{\parindent}{0pt}%
  \setlength{\parskip}{0pt plus 0.3pt}%
  \let\item\@idxitem
}{%
  \clearpage
}
\makeatother

\IfFileExists{\jobname-pw.ind}{\input{\jobname-pw.ind}}{}

\end{document}

      