%% latex-korrekturansicht-vorspann.tex
%% Vorspann für die Korrekturansicht.
%% Lädt die gemeinsame Datei latex-vorspann.tex mit gesetztem Schalter.

\newif\ifkorrekturansicht
\korrekturansichttrue

\input{../tex-inputs/latex-vorspann}


               \section[Hugo von Hofmannsthal an Arthur Schnitzler, 4. 1. 1904]{ Hugo von Hofmannsthal an Arthur Schnitzler, 4. 1. 1904}\nopagebreak\mylabel{v}\rehead{ }\normalsize\beginnumbering\briefempfaengerindex{Schnitzler, Arthur@\textsc{Schnitzler, Arthur}!zzzHofmannsthal, Hugo von@\emph{von Hugo von Hofmannsthal}!1904-01-042@{4. 1. 1904}|(be} \toendnotes[C]{\smallbreak\pagebreak[2]} \Standort{CUL, Schnitzler, B 43.}
\physDesc{Postkarte
\newline{}Handschrift: schwarze Tinte, deutsche Kurrent\newline{}Versand: 1) Rohrpost 2) Stempel: »\nobreak{}\oindex{I., Innere Stadt@\textbf{I., Innere Stadt}, \emph{Bezirk (A.BZK)}|pwk}Wien 1/1, 4 1 04, 8 40N\nobreak{}«. 3) Stempel: »\nobreak{}\oindex{XVIII., Waehring@\textbf{XVIII., Währing}, \emph{Bezirk (A.BZK)}|pwk}18/1 Wien, 4 1 04, 9 30N\nobreak{}«. 
\newline{}Schnitzler: mit Bleistift beschrift: »Hugo« \newline{}Ordnung: 1) mit Bleistift von unbekannter Hand nummeriert: »\strikeout{\textcolor{gray}{249}}« 2) mit Bleistift von unbekannter Hand nummeriert:
                                    »211«}\toendnotes[C]{\smallbreak}\pstart{}{\pb}\textsc{Herrn D\textsuperscript{r} Arthur Schnitzler}\pend{}\pstart{}\textcolor{pink}{\textsc{Wien}}{}\ledrightnote{\textcolor{pink}{Wien}}\pend{}\pstart{}\textcolor{pink}{\textsc{XVIII Spöttelgasse 7}}{}\ledrightnote{\textcolor{pink}{Edmund-Weiß-Gasse}}\pend{}{\bigskip}\pstart
           \noindent{}{\pb}\textsc{Miss \textcolor{blue}{Isadora Duncan}{}\ledrightnote{\textcolor{blue}{Isadora Duncan}}}\pend
           \pstart
           bittet \textsc{Herrn u. \textcolor{blue}{Frau D\textsuperscript{r} Schnitzler}{}\ledrightnote{\textcolor{blue}{Olga Schnitzler}}} zu Ihrer \label{K_L01357_1v}\edtext{Generalprobe}{\lemma{\textnormal{\emph{Generalprobe}}}\Cendnote{\textnormal{\textcolor{blue}{Hofmannsthal} sollte einleitende Worte
                  sprechen; siehe A. S.: \emph{Tagebuch}, 5. 1. 1904}}}\label{K_L01357_1h}\pend
           \pstart
           morgen Dienstag{ }\uline{3½} Uhr\pend
           \pstart
           \textcolor{pink}{Carltheater}{}\ledrightnote{\textcolor{pink}{Carl-Theater}}.\pend
           \endnumbering\briefempfaengerindex{Schnitzler, Arthur@\textsc{Schnitzler, Arthur}!zzzHofmannsthal, Hugo von@\emph{von Hugo von Hofmannsthal}!1904-01-042@{4. 1. 1904}|)be}\mylabel{h}  \normalsize

\doendnotes{C}
\bigskip
\vfill

\clearpage

\footnotesize

\lohead{\textsc{register}}

% Definiere theindex-Environment komplett neu ohne reledmac
\makeatletter
\renewenvironment{theindex}{%
  \section*{\indexname}%
  \setlength{\parindent}{0pt}%
  \setlength{\parskip}{0pt plus 0.3pt}%
  \let\item\@idxitem
}{%
  \clearpage
}
\makeatother

\IfFileExists{\jobname-pw.ind}{\input{\jobname-pw.ind}}{}

\end{document}

      