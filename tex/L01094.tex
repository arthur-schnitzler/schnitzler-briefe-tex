%% latex-korrekturansicht-vorspann.tex
%% Vorspann für die Korrekturansicht.
%% Lädt die gemeinsame Datei latex-vorspann.tex mit gesetztem Schalter.

\newif\ifkorrekturansicht
\korrekturansichttrue

\input{../tex-inputs/latex-vorspann}


               \section[Arthur Schnitzler an Hermann Bahr, 25. 1. 1901]{ Arthur Schnitzler an Hermann Bahr, 25. 1. 1901}\nopagebreak\mylabel{v}\rehead{ }\normalsize\beginnumbering\briefempfaengerindex{Bahr, Hermann@\textsc{Bahr, Hermann}!zzzSchnitzler, Arthur@\emph{von Arthur Schnitzler}!1901-01-251@{25. 1. 1901}|(be} \toendnotes[C]{\smallbreak\pagebreak[2]} \Standort{TMW, HS AM 23341 Ba.}
\physDesc{Brief, 1 Blatt (Briefpapier mit Trauerrand), 2 Seiten
\newline{}Handschrift: schwarze Tinte, deutsche Kurrent\newline{}Ordnung: Lochung }\buchAbdrucke{\weitereDrucke{1) \emph{25. 1. 1901.} In: Arthur Schnitzler: \emph{The Letters of Arthur Schnitzler to Hermann Bahr}. Edited, annotated, and with an introduction, by Donald G.
                        Daviau. Chapel Hill: \emph{The University of North Carolina Press} 1978, S. 68 (University of North Carolina studies in the Germanic languages
                        and literatures, 89).} \weitereDrucke{2) Hermann Bahr, Arthur Schnitzler: \emph{Briefwechsel, Aufzeichnungen, Dokumente (1891–1931)}. Hg. Kurt Ifkovits und Martin Anton Müller. Göttingen: \emph{Wallstein} 2018, S. 192.} }\pstart
           \noindent{}{\pb}lieber Hermann, es freut mich ſehr, dſs dir die \textcolor{green}{Marionetten}{}\ledrightnote{\textcolor{green}{Zum großen Wurstel}} einigen Spaſs gemacht haben. Wenn ſie auf der Bühne
               wirken ſollten, wird ja die Wirkung wahrſcheinlich aus den derberen Momenten kommen,
               weniger aus denen, die uns behagen. Ob das Couplet des Herzogs mit {\pb}Ringkämpfer, todtem
               Mädchen u. ſ. w. nicht gefährlich ſein könnte, wird ſich wohl erſt auf den Proben
               entſcheiden laſſen.\pend
           \pstart
           Ich danke dir ſehr und bin mit herzlichen Grüßen dein{\\[\baselineskip]}\spacefill\mbox{Arth Sch}\pend
           \leftskip=0em{}\pstart
           25. 1. 1901.\pend
           \endnumbering\briefempfaengerindex{Bahr, Hermann@\textsc{Bahr, Hermann}!zzzSchnitzler, Arthur@\emph{von Arthur Schnitzler}!1901-01-251@{25. 1. 1901}|)be}\mylabel{h}  \normalsize

\doendnotes{C}
\bigskip
\vfill

\clearpage

\footnotesize

\lohead{\textsc{register}}

% Definiere theindex-Environment komplett neu ohne reledmac
\makeatletter
\renewenvironment{theindex}{%
  \section*{\indexname}%
  \setlength{\parindent}{0pt}%
  \setlength{\parskip}{0pt plus 0.3pt}%
  \let\item\@idxitem
}{%
  \clearpage
}
\makeatother

\IfFileExists{\jobname-pw.ind}{\input{\jobname-pw.ind}}{}

\end{document}

      