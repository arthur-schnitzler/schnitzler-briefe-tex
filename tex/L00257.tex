%% latex-korrekturansicht-vorspann.tex
%% Vorspann für die Korrekturansicht.
%% Lädt die gemeinsame Datei latex-vorspann.tex mit gesetztem Schalter.

\newif\ifkorrekturansicht
\korrekturansichttrue

\input{../tex-inputs/latex-vorspann}


               \section[Richard Beer-Hofmann an Arthur Schnitzler, {[}19. 8. 1893?{]}]{ Richard Beer-Hofmann an Arthur Schnitzler, {[}19. 8. 1893?{]}}\nopagebreak\mylabel{v}\rehead{ }\normalsize\beginnumbering\briefempfaengerindex{Schnitzler, Arthur@\textsc{Schnitzler, Arthur}!zzzBeer-Hofmann, Richard@\emph{von Richard Beer-Hofmann}!1893-08-191@{{[}19. 8. 1893?{]}}|(be} \toendnotes[C]{\smallbreak\pagebreak[2]} \Standort{CUL, Schnitzler, B 8.}
\physDesc{Brief, 1 Blatt, 2 Seiten
\newline{}Handschrift: Bleistift, deutsche Kurrent
\newline{}Schnitzler: mit Bleistift nummeriert: »23« }\buchAbdrucke{\weitereDrucke{Arthur Schnitzler, Richard Beer-Hofmann: \emph{Briefwechsel 1891–1931}. Hg. Konstanze Fliedl. Wien, Zürich: \emph{Europaverlag} 1992, S. 51.} }\pstart
           \noindent{}{\pb}Lieber Arthur! Verzeihen Sie meine Nachlässigkeit; war in den
               letzten Tagen stark beschäftigt. Ich ko{\geminationm}{ }Montag{ }Abends{ }\introOben{}gegen\introOben{}{ }8 Uhr in \textcolor{pink}{Wien}{}\ledrightnote{\textcolor{pink}{Wien}} an. \damage{\textcolor{gray}{H}}abe mit Ihnen zu sprechen; und \damage{\textcolor{gray}{w}}erde Ihnen dann mündlich Alles beantworten. Schreiben Sie zwei Zeilen wo Sie
                  Montag{ }8 Uhr Abends
               sind, oder besser noch erwarten Sie mich zwischen 8 u ½ 9{ }\textcolor{pink}{Caffée Europe}{}\ledrightnote{\textcolor{pink}{Café de l’Europe}}{ }\textcolor{pink}{Stefansplatz}{}\ledrightnote{\textcolor{pink}{Stephansplatz}}. Ich war in \textcolor{pink}{Marienbad}{}\ledrightnote{\textcolor{pink}{Marienbad}} bei \textcolor{blue}{Freund}{}\ledrightnote{\textcolor{blue}{Carl Freund}} – Nichts
               Positives erreicht. Näheres mündlich. Vielleicht kann ich auch \textcolor{blue}{Schwarzkopf}{}\ledrightnote{\textcolor{blue}{Gustav Schwarzkopf}}{ }{\pb}sehen. Ich reise
                  Mittwoch{ }Früh nach \textcolor{pink}{Znaim}{}\ledrightnote{\textcolor{pink}{Znaim}}.\pend
           \pstart
           Herzlichst{\\[\baselineskip]}\spacefill\mbox{Richard}\pend
           \leftskip=0em{}\pstart
           \uline{Samstag Mittag}\pend
           \endnumbering\briefempfaengerindex{Schnitzler, Arthur@\textsc{Schnitzler, Arthur}!zzzBeer-Hofmann, Richard@\emph{von Richard Beer-Hofmann}!1893-08-191@{{[}19. 8. 1893?{]}}|)be}\mylabel{h}  \normalsize

\doendnotes{C}
\bigskip
\vfill

\clearpage

\footnotesize

\lohead{\textsc{register}}

% Definiere theindex-Environment komplett neu ohne reledmac
\makeatletter
\renewenvironment{theindex}{%
  \section*{\indexname}%
  \setlength{\parindent}{0pt}%
  \setlength{\parskip}{0pt plus 0.3pt}%
  \let\item\@idxitem
}{%
  \clearpage
}
\makeatother

\IfFileExists{\jobname-pw.ind}{\input{\jobname-pw.ind}}{}

\end{document}

      