%% latex-korrekturansicht-vorspann.tex
%% Vorspann für die Korrekturansicht.
%% Lädt die gemeinsame Datei latex-vorspann.tex mit gesetztem Schalter.

\newif\ifkorrekturansicht
\korrekturansichttrue

\input{../tex-inputs/latex-vorspann}


               \section[Arthur Schnitzler an Hugo von Hofmannsthal, 29. 7. 1892]{ Arthur Schnitzler an Hugo von Hofmannsthal, 29. 7. 1892}\nopagebreak\mylabel{v}\rehead{ }\normalsize\beginnumbering\briefempfaengerindex{Hofmannsthal, Hugo von@\textsc{Hofmannsthal, Hugo von}!zzzSchnitzler, Arthur@\emph{von Arthur Schnitzler}!1892-07-291@{29. 7. 1892}|(be} \toendnotes[C]{\smallbreak\pagebreak[2]} \Standort{FDH, Hs-30885,22.}
\physDesc{Brief, 1 Blatt, 4 Seiten
\newline{}Handschrift: schwarze Tinte, deutsche Kurrent}\buchAbdrucke{\weitereDrucke{Hugo von Hofmannsthal, Arthur Schnitzler: \emph{Briefwechsel}. Hg. Therese Nickl und Heinrich Schnitzler. Frankfurt am Main: \emph{S. Fischer} 1964, S. 25.} }\toendnotes[C]{\smallbreak}\pstart
           {\pb}\textcolor{pink}{Wien}{}\ledrightnote{\textcolor{pink}{Wien}}\hfill 29/7 92\pend
           \pstart{}Lieber Freund,\pend\pstart
           nachdem Sie Ihr \textcolor{green}{Gedicht}{}\ledrightnote{→\textcolor{green}{Einleitung}} nicht im Inhalt haben
                    wollen, möchte ich auch jeden Titel weglaſſen, und es nur im ſelben Druck wie
                    alles übrige \introOben{}haben\introOben{}, jedoch mit oben weit freigelaſſenen
                    Rändern \strikeout{haben}. – Einverſtanden? –\pend
           \pstart
           Vorgeſtern habe ich meine \textcolor{green}{Novelle}{}\ledrightnote{→\textcolor{green}{Sterben. Novelle}} beendet. –
                    Ich hoffe, {\pb}ſie wird, we{\geminationn}{ }ſie erſt durchgefeilt iſt, als ehrenwerte Studie
                    gelten können. Ich habe ſie plötzlich zu Ende ſchreiben müſſen, Nachts im Cafè,
                    während ſchläfrige Kellner bereits die Seſſel aufeinander thürmten. Ich habe ſie
                    ſehr lieb gehabt – ich fühle mich ordentlich einſam, ſeit ich nicht mehr drüber
                    denken muſs. {\pb}(Siehe \textcolor{green}{Freund \textsc{Y}}{}\ledrightnote{\textcolor{green}{Mein Freund Ypsilon}}). – Nun will ich wieder
                    ans \textcolor{green}{Stück}{}\ledrightnote{→\textcolor{green}{Liebelei. Schauspiel in drei Akten}}. – Eben hab ich \textcolor{blue}{Blumenthal}{}\ledrightnote{\textcolor{blue}{Oskar Blumenthal}} u \textcolor{blue}{Reicher}{}\ledrightnote{\textcolor{blue}{Emanuel Reicher}} geſchrieben! – wie verdreht eigentlich die Welt
                    iſt! –\pend
           \pstart
           Was macht Ihr \textcolor{green}{Stück}{}\ledrightnote{→\textcolor{green}{Ascanio und Gioconda}}? – Ich wundre mich, daſs
                    Sie zugleich zweiten und fünften Akt ſchreiben können. So ſicher bin ich meiner
                    Geſtalten nie! Es kann ihnen doch im dritten Akt {\pb}was einfallen oder gar paſſiren, wovon ich im zweiten noch nichts rechtes
                    weiſs. Selbſt we{\geminationn} eine genaue Skizze vorliegt, wage
                    ich es nicht und habe gewiſs keine Luſt dazu! Ich will mit ihnen weiter leben,
                    und erleben, Gedanke für Gedanke und That für That, wie ſie ſelber. Ich darf
                    manches vorausahnen, aber wiſſen darf ichs nicht.\pend
           \pstart Herzlichſt Ihr \spacefill\mbox{Arthur}\pend{}\endnumbering\briefempfaengerindex{Hofmannsthal, Hugo von@\textsc{Hofmannsthal, Hugo von}!zzzSchnitzler, Arthur@\emph{von Arthur Schnitzler}!1892-07-291@{29. 7. 1892}|)be}\mylabel{h}  \normalsize

\doendnotes{C}
\bigskip
\vfill

\clearpage

\footnotesize

\lohead{\textsc{register}}

% Definiere theindex-Environment komplett neu ohne reledmac
\makeatletter
\renewenvironment{theindex}{%
  \section*{\indexname}%
  \setlength{\parindent}{0pt}%
  \setlength{\parskip}{0pt plus 0.3pt}%
  \let\item\@idxitem
}{%
  \clearpage
}
\makeatother

\IfFileExists{\jobname-pw.ind}{\input{\jobname-pw.ind}}{}

\end{document}

      