%% latex-korrekturansicht-vorspann.tex
%% Vorspann für die Korrekturansicht.
%% Lädt die gemeinsame Datei latex-vorspann.tex mit gesetztem Schalter.

\newif\ifkorrekturansicht
\korrekturansichttrue

\input{../tex-inputs/latex-vorspann}


               \section[Thomas Mann an Arthur Schnitzler, 7. 8. 1908]{ Thomas Mann an Arthur Schnitzler, 7. 8. 1908}\nopagebreak\mylabel{v}\rehead{ }\normalsize\beginnumbering\briefempfaengerindex{Schnitzler, Arthur@\textsc{Schnitzler, Arthur}!zzzMann, Thomas@\emph{von Thomas Mann}!1908-08-071@{7. 8. 1908}|(be} \toendnotes[C]{\smallbreak\pagebreak[2]} \Standort{CUL, Schnitzler, B 67.}
\physDesc{Briefkarte
\newline{}Handschrift: schwarze Tinte, deutsche Kurrent
\newline{}Schnitzler: mit Bleistift beschriftet: »\textsc{Mann}« }\buchAbdrucke{\weitereDrucke{Hertha Krotkoff: \emph{Arthur Schnitzler – Thomas Mann: Briefe.} In: \emph{Modern Austrian Literature}, Jg. 7 (1974) Nr. 1/2, S. 13–14.} }\toendnotes[C]{\smallbreak}\pstart
           {\pb}\textcolor{pink}{Tölz}{}\ledrightnote{\textcolor{pink}{Bad Tölz}} den 7. August 1908\pend
           \pstart{}Verehrter Herr Doctor:\pend\pstart
           Ich ſchreibe Ihnen nochmals unter Ihrer \textcolor{pink}{Wien}{}\ledrightnote{\textcolor{pink}{Wien}}er
                    Adreſſe, weil es mir vollkommen unmöglich iſt, die ländliche zu entziffern, –
                    woran wohl noch mehr als Ihre Handſchrift meine mangelhaften geographiſchen
                    Kenntnisse ſchuld ſind.\pend
           \pstart
           Ich habe nichts dagegen, daß Sie {\pb}»\textcolor{green}{Wälſungenblut}{}\ledrightnote{\textcolor{green}{Wälsungenblut}}« \textcolor{blue}{Waſſermann}{}\ledrightnote{\textcolor{blue}{Jakob Wassermann}} zu leſen geben, geſetzt, daß er noch bei Ihnen iſt. Sagen
                    Sie ihm aber, bitte, daß ich ſie Ihnen der Sache wegen und im Hinblick auf den
                        »\textcolor{green}{Weg ins Freie}{}\ledrightnote{\textcolor{green}{Der Weg ins Freie. Roman}}« geſchickt habe. Er könnte
                    ſich ſonſt gekränkt fühlen. Daß die \textcolor{green}{Novelle}{}\ledrightnote{→\textcolor{green}{Wälsungenblut}} weiter kurſiert, möchte ich Sie bitten zu
                    verhindern.\pend
           \pstart
           Mit den verbindlichſten Grüßen bin ich, verehrter Herr Doctor, Ihr ergebener\pend
           \pstart \spacefill\mbox{Thomas Mann.}\pend{}\endnumbering\briefempfaengerindex{Schnitzler, Arthur@\textsc{Schnitzler, Arthur}!zzzMann, Thomas@\emph{von Thomas Mann}!1908-08-071@{7. 8. 1908}|)be}\mylabel{h}  \normalsize

\doendnotes{C}
\bigskip
\vfill

\clearpage

\footnotesize

\lohead{\textsc{register}}

% Definiere theindex-Environment komplett neu ohne reledmac
\makeatletter
\renewenvironment{theindex}{%
  \section*{\indexname}%
  \setlength{\parindent}{0pt}%
  \setlength{\parskip}{0pt plus 0.3pt}%
  \let\item\@idxitem
}{%
  \clearpage
}
\makeatother

\IfFileExists{\jobname-pw.ind}{\input{\jobname-pw.ind}}{}

\end{document}

      