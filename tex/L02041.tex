%% latex-korrekturansicht-vorspann.tex
%% Vorspann für die Korrekturansicht.
%% Lädt die gemeinsame Datei latex-vorspann.tex mit gesetztem Schalter.

\newif\ifkorrekturansicht
\korrekturansichttrue

\input{../tex-inputs/latex-vorspann}


               \section[Arthur Schnitzler an Hugo von Hofmannsthal, 22. 10. 1911]{ Arthur Schnitzler an Hugo von Hofmannsthal, 22. 10. 1911}\nopagebreak\mylabel{v}\rehead{ }\normalsize\beginnumbering\briefempfaengerindex{Hofmannsthal, Hugo von@\textsc{Hofmannsthal, Hugo von}!zzzSchnitzler, Arthur@\emph{von Arthur Schnitzler}!1911-10-221@{22. 10. 1911}|(be} \toendnotes[C]{\smallbreak\pagebreak[2]} \Standort{FDH, Hs-30885,144.}
\physDesc{Briefkarte
\newline{}Handschrift: schwarze Tinte, deutsche Kurrent}\buchAbdrucke{\weitereDrucke{Hugo von Hofmannsthal, Arthur Schnitzler: \emph{Briefwechsel}. Hg. Therese Nickl und Heinrich Schnitzler. Frankfurt am Main: \emph{S. Fischer} 1964, S. 264.} }\pstart
           \raggedleft{}{\pb}22/X 911\pend
           \pstart
           \textcolor{gray}{\textbf{A. S.}}\pend
           \pstart
           mein lieber Hugo, ich danke für Ihr liebes Telegra{\geminationm} aus \textcolor{pink}{Neubeuern}{}\ledrightnote{\textcolor{pink}{Neubeuern}}, das
               ich für alle Fälle ſchon nach \textcolor{pink}{Rodaun}{}\ledrightnote{\textcolor{pink}{Rodaun}} beantworte. Ich
               reiſe Ende der Woche ab, \textcolor{pink}{Prag}{}\ledrightnote{\textcolor{pink}{Prag}}, \textcolor{pink}{Dresden}{}\ledrightnote{\textcolor{pink}{Dresden}} (Vorleſungen), – da{\geminationn}{ }\textcolor{pink}{Berlin}{}\ledrightnote{\textcolor{pink}{Berlin}} – \textcolor{pink}{Hamburg}{}\ledrightnote{\textcolor{pink}{Hamburg}}
                  (\textcolor{green}{Beatrice}{}\ledrightnote{\textcolor{green}{Der Schleier der Beatrice. Schauspiel in fünf Akten}}, \textcolor{green}{Weites
                  Land}{}\ledrightnote{\textcolor{green}{Das weite Land. Tragikomödie in fünf Akten}}, \textcolor{green}{Anatol}{}\ledrightnote{\textcolor{green}{Anatol}}) bin gegen {\pb}Mitte November zurück. Vorher werden wir einander wohl kaum ſehen. Für
                  Herbſt und Winter aber hoff ich ein häufigeres Zuſa{\geminationm}enſein als es mir die letzten Jahre beſchieden war. Was
               iſt’s mit »\textcolor{green}{Jedermann}{}\ledrightnote{\textcolor{green}{Jedermann. Das Spiel vom Sterben des reichen Mannes}}« und Allerlei? \pend
           \pstart
           Wir grüßen Euch herzlichſt!{\\[\baselineskip]}Ihr{\\[\baselineskip]}\spacefill\mbox{Arthur.}\pend
           \leftskip=0em{}\endnumbering\briefempfaengerindex{Hofmannsthal, Hugo von@\textsc{Hofmannsthal, Hugo von}!zzzSchnitzler, Arthur@\emph{von Arthur Schnitzler}!1911-10-221@{22. 10. 1911}|)be}\mylabel{h}  \normalsize

\doendnotes{C}
\bigskip
\vfill

\clearpage

\footnotesize

\lohead{\textsc{register}}

% Definiere theindex-Environment komplett neu ohne reledmac
\makeatletter
\renewenvironment{theindex}{%
  \section*{\indexname}%
  \setlength{\parindent}{0pt}%
  \setlength{\parskip}{0pt plus 0.3pt}%
  \let\item\@idxitem
}{%
  \clearpage
}
\makeatother

\IfFileExists{\jobname-pw.ind}{\input{\jobname-pw.ind}}{}

\end{document}

      