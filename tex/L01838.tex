%% latex-korrekturansicht-vorspann.tex
%% Vorspann für die Korrekturansicht.
%% Lädt die gemeinsame Datei latex-vorspann.tex mit gesetztem Schalter.

\newif\ifkorrekturansicht
\korrekturansichttrue

\input{../tex-inputs/latex-vorspann}


               \section[Max Burckhard: Widmungsexemplar Der Richter für Arthur Schnitzler, 16. 4. 1909]{ Max Burckhard: Widmungsexemplar Der Richter für Arthur Schnitzler,
                    16. 4. 1909}\nopagebreak\mylabel{v}\rehead{ }\normalsize\beginnumbering\briefempfaengerindex{Schnitzler, Arthur@\textsc{Schnitzler, Arthur}!zzzBurckhard, Max Eugen@\emph{von Max Eugen Burckhard}!1909-04-161@{16. 4. 1909}|(be} \toendnotes[C]{\smallbreak\pagebreak[2]} \Standort{DLA, G:Schnitzler, Arthur (Sammlung Heinrich Schnitzler).}
\physDesc{Widmung am Vortitel
\newline{}Handschrift: schwarze Tinte, deutsche Kurrent\newline{}Ordnung: bei der Enteignung
                                    des Exemplars 1938 mit Bleistift von
                                    unbekannter Hand als Dublette markiert: »\noindent{}= 467.288–B{ / }vol. IV.« }\pstart
           \noindent{}{\pb}Arthur Schnitzler\pend
           \pstart
           mit herzlicher Verehrung{\\[\baselineskip]}\spacefill\mbox{D\textsuperscript{r}Burckhard}\pend
           \leftskip=0em{}\pstart
           \textcolor{pink}{München}{}\ledrightnote{\textcolor{pink}{München}}{ }16/4 09\pend
           {\bigskip}\pstart
           \noindent{}\centering{}\textcolor{gray}{\textbf{Übersetzungsrecht, sowie alle anderen Rechte
                        vorbehalten.}}\pend
           {\bigskip}\pstart
           \noindent{}\centering{}{\pb}\textcolor{gray}{\textbf{\textcolor{green}{DER
                    RICHTER}{}\ledrightnote{\textcolor{green}{Der Richter}}}}\pend
           \pstart
           \noindent{}\centering{}\textcolor{gray}{\textbf{VON}}\pend
           \pstart
           \noindent{}\centering{}\textcolor{gray}{\textbf{\textsc{Dr.}{ }MAX
                    BURCKHARD}}\pend
           {\bigskip}\pstart
           \noindent{}\centering{}\textcolor{gray}{\textbf{\textcolor{pink}{BERLIN}{}\ledrightnote{\textcolor{pink}{Berlin}}{ }1909}}\pend
           \pstart
           \noindent{}\centering{}\textcolor{gray}{\textbf{\textcolor{brown}{PUTTKAMMER {\kaufmannsund}
                                MÜHLBRECHT}{}\ledrightnote{\textcolor{brown}{Puttkammer {\kaufmannsund} Mühlbrecht, Buchhandlung für Staats- und Rechtswissenschaft}}}}\pend
           \pstart
           \noindent{}\centering{}\textcolor{gray}{\textbf{Buchhandlung für Staats- u.
                        Rechtswissenschaft.}}\pend
           \endnumbering\briefempfaengerindex{Schnitzler, Arthur@\textsc{Schnitzler, Arthur}!zzzBurckhard, Max Eugen@\emph{von Max Eugen Burckhard}!1909-04-161@{16. 4. 1909}|)be}\mylabel{h}  \normalsize

\doendnotes{C}
\bigskip
\vfill

\clearpage

\footnotesize

\lohead{\textsc{register}}

% Definiere theindex-Environment komplett neu ohne reledmac
\makeatletter
\renewenvironment{theindex}{%
  \section*{\indexname}%
  \setlength{\parindent}{0pt}%
  \setlength{\parskip}{0pt plus 0.3pt}%
  \let\item\@idxitem
}{%
  \clearpage
}
\makeatother

\IfFileExists{\jobname-pw.ind}{\input{\jobname-pw.ind}}{}

\end{document}

      