%% latex-korrekturansicht-vorspann.tex
%% Vorspann für die Korrekturansicht.
%% Lädt die gemeinsame Datei latex-vorspann.tex mit gesetztem Schalter.

\newif\ifkorrekturansicht
\korrekturansichttrue

\input{../tex-inputs/latex-vorspann}


               \section[Arthur Schnitzler an Richard Beer-Hofmann, 23. 9. 1895]{ Arthur Schnitzler an Richard Beer-Hofmann, 23. 9. 1895}\nopagebreak\mylabel{v}\rehead{ }\normalsize\beginnumbering\briefempfaengerindex{Beer-Hofmann, Richard@\textsc{Beer-Hofmann, Richard}!zzzSchnitzler, Arthur@\emph{von Arthur Schnitzler}!1895-09-231@{23. 9. 1895}|(be} \toendnotes[C]{\smallbreak\pagebreak[2]} \Standort{YCGL, MSS 31.}
\physDesc{Postkarte
\newline{}Handschrift: schwarze Tinte, deutsche Kurrent\newline{}Versand: Stempel: »\nobreak{}\oindex{IX., Alsergrund@\textbf{IX., Alsergrund}, \emph{Bezirk (A.BZK)}|pwk}Wien 9/3 72, 23. 9. 95, 3–4N\nobreak{}«.  }\buchAbdrucke{\weitereDrucke{Arthur Schnitzler, Richard Beer-Hofmann: \emph{Briefwechsel 1891–1931}. Hg. Konstanze Fliedl. Wien, Zürich: \emph{Europaverlag} 1992, S. 84.} }\toendnotes[C]{\smallbreak}\pstart{}{\pb}\textsc{Dr. Richard Beer-Hofmann}\pend{}\pstart{}\textcolor{pink}{\textsc{Gardone}}{}\ledrightnote{\textcolor{pink}{Gardone Riviera}}\pend{}\pstart{}\textsc{am \textcolor{pink}{Gardasee}{}\ledrightnote{\textcolor{pink}{Lago di Garda}}}\pend{}\pstart{}\textcolor{pink}{\textsc{Italia.}}{}\ledrightnote{\textcolor{pink}{Italien}}\pend{}{\bigskip}\pstart
           \noindent{}{\pb}Lieber Richard, nach \textcolor{pink}{\textsc{Riva}}{}\ledrightnote{\textcolor{pink}{Riva del Garda}} hab ich Ihnen nicht nur eine Karte, ſondern einen längern Brief geſchrieben,
               den Sie gef. reclamiren wollen. Schreiben Sie mir endlich auch einmal wieder.\pend
           \pstart
           Vom \textcolor{brown}{Burgth.}{}\ledrightnote{\textcolor{brown}{Burgtheater}} nichts Neues. –\pend
           \pstart
           »\label{K_L00490_1v}\edtext{\textcolor{green}{\textsc{Mourir}}{}\ledrightnote{\textcolor{green}{Mourir}}}{\lemma{\textnormal{\emph{Mourir}}}\Cendnote{\textnormal{Zuvor war \emph{\textcolor{green}{Sterben}} in der Übersetzung von \textcolor{blue}{Gaspard
                     Vallette} in sechs Teilen zwischen 27. 4. 1895 und
                     1. 6.1895 in der \emph{\textcolor{brown}{Semaine
                     littéraire}} erschienen. Die gebundene Ausgabe hatte \textcolor{blue}{Schnitzler} am 12. 4. 1896 in der Hand.}}}\label{K_L00490_1h}« erſcheint bei \textcolor{brown}{\textsc{Perrin}}{}\ledrightnote{\textcolor{brown}{Éditions Perrin}} in \textcolor{pink}{\textsc{Paris}}{}\ledrightnote{\textcolor{pink}{Paris}} (durch Vermittlung der Red. der \textcolor{brown}{\textsc{Sem. litt.}}{}\ledrightnote{\textcolor{brown}{La Semaine Littéraire}})\pend
           \pstart
           – Sie müſſen es jetzt da unten herrlich haben. Ich denke an den \textcolor{pink}{Gardaſee}{}\ledrightnote{\textcolor{pink}{Lago di Garda}} bei \textcolor{pink}{Gardone}{}\ledrightnote{\textcolor{pink}{Gardone Riviera}} zurück wie
               an ein Meer.\pend
           \pstart Seien Sie herzlich gegrüßt! Ihr \spacefill\mbox{Arthur}\pend{}\endnumbering\briefempfaengerindex{Beer-Hofmann, Richard@\textsc{Beer-Hofmann, Richard}!zzzSchnitzler, Arthur@\emph{von Arthur Schnitzler}!1895-09-231@{23. 9. 1895}|)be}\mylabel{h}  \normalsize

\doendnotes{C}
\bigskip
\vfill

\clearpage

\footnotesize

\lohead{\textsc{register}}

% Definiere theindex-Environment komplett neu ohne reledmac
\makeatletter
\renewenvironment{theindex}{%
  \section*{\indexname}%
  \setlength{\parindent}{0pt}%
  \setlength{\parskip}{0pt plus 0.3pt}%
  \let\item\@idxitem
}{%
  \clearpage
}
\makeatother

\IfFileExists{\jobname-pw.ind}{\input{\jobname-pw.ind}}{}

\end{document}

      