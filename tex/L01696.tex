%% latex-korrekturansicht-vorspann.tex
%% Vorspann für die Korrekturansicht.
%% Lädt die gemeinsame Datei latex-vorspann.tex mit gesetztem Schalter.

\newif\ifkorrekturansicht
\korrekturansichttrue

\input{../tex-inputs/latex-vorspann}


               \section[Max Mell an Arthur Schnitzler, 29. 7. 1907]{ Max Mell an Arthur Schnitzler, 29. 7. 1907}\nopagebreak\mylabel{v}\rehead{ }\normalsize\beginnumbering\briefempfaengerindex{Schnitzler, Arthur@\textsc{Schnitzler, Arthur}!zzzMell, Max@\emph{von Max Mell}!1907-07-292@{29. 7. 1907}|(be} \toendnotes[C]{\smallbreak\pagebreak[2]} \Standort{CUL, Schnitzler, B 70.}
\physDesc{Briefkarte
\newline{}Handschrift: schwarze Tinte, deutsche Kurrent}\toendnotes[C]{\smallbreak}\pstart
           \raggedleft{}{\pb}\textcolor{pink}{Wien}{}\ledrightnote{\textcolor{pink}{Wien}}, 29. Juli 1907\pend
           \pstart{}Sehr geehrter Herr Doktor,\pend\pstart
           es wird mir ſehr ſchmerzlich ſein, in meinem \textcolor{green}{Almanach}{}\ledrightnote{→\textcolor{green}{Almanach der Wiener Werkstätte}} nichts von Ihnen zu haben. Wäre es nicht
                    möglich, daß Sie mir ein Fragment aus der größeren \textcolor{green}{Arbeit}{}\ledrightnote{→\textcolor{green}{Der Weg ins Freie. Roman}} die Sie jetzt ſchreiben, gäben? Im ſchlimmſten
                    Fall möchte ich wenigſtens etwas ſchon gedrucktes, (etwa Gedichte?) bringen, und
                    bäte Sie dafür um Rat.\pend
           \pstart
           Mit Ihrer Anſichtskarte haben Sie mir eine große Freude gemacht, \textsc{Dr. \textcolor{blue}{Servaes}{}\ledrightnote{\textcolor{blue}{Franz Servaes}}}, der am ſelben Tag zu mir kam, hat mich ordentlich {\pb}beneidet darum. Bitte empfehlen Sie
                    mich Ihrer verehrten \textcolor{blue}{Frau}{}\ledrightnote{→\textcolor{blue}{Olga Schnitzler}}!\pend
           \pstart
           Mit den herzlichſten Grüßen{\\[\baselineskip]}Ihr{\\[\baselineskip]}\spacefill\mbox{Max Mell.}\pend
           \leftskip=0em{}\endnumbering\briefempfaengerindex{Schnitzler, Arthur@\textsc{Schnitzler, Arthur}!zzzMell, Max@\emph{von Max Mell}!1907-07-292@{29. 7. 1907}|)be}\mylabel{h}  \normalsize

\doendnotes{C}
\bigskip
\vfill

\clearpage

\footnotesize

\lohead{\textsc{register}}

% Definiere theindex-Environment komplett neu ohne reledmac
\makeatletter
\renewenvironment{theindex}{%
  \section*{\indexname}%
  \setlength{\parindent}{0pt}%
  \setlength{\parskip}{0pt plus 0.3pt}%
  \let\item\@idxitem
}{%
  \clearpage
}
\makeatother

\IfFileExists{\jobname-pw.ind}{\input{\jobname-pw.ind}}{}

\end{document}

      