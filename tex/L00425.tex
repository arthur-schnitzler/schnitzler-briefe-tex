%% latex-korrekturansicht-vorspann.tex
%% Vorspann für die Korrekturansicht.
%% Lädt die gemeinsame Datei latex-vorspann.tex mit gesetztem Schalter.

\newif\ifkorrekturansicht
\korrekturansichttrue

\input{../tex-inputs/latex-vorspann}


               \section[Arthur Schnitzler an Richard Beer-Hofmann, {[}26. 3. 1895{]}]{ Arthur Schnitzler an Richard Beer-Hofmann, {[}26. 3. 1895{]}}\nopagebreak\mylabel{v}\rehead{ }\normalsize\beginnumbering\briefempfaengerindex{Beer-Hofmann, Richard@\textsc{Beer-Hofmann, Richard}!zzzSchnitzler, Arthur@\emph{von Arthur Schnitzler}!1895-03-261@{{[}26. 3. 1895{]}}|(be} \toendnotes[C]{\smallbreak\pagebreak[2]} \Standort{YCGL, MSS 31.}
\physDesc{Brief, 1 Blatt, 4 Seiten, Umschlag
\newline{}Handschrift: Bleistift, deutsche Kurrent\newline{}Versand: ohne postalischen Übermittlungsvermerk }\buchAbdrucke{\weitereDrucke{Arthur Schnitzler, Richard Beer-Hofmann: \emph{Briefwechsel 1891–1931}. Hg. Konstanze Fliedl. Wien, Zürich: \emph{Europaverlag} 1992, S. 71–72.} }\toendnotes[C]{\smallbreak}\pstart{}{\pb}Herrn Dr. \textsc{Richard
                     Beer-Hofmann}\pend{}\pstart{}\textcolor{pink}{Wien}{}\ledrightnote{\textcolor{pink}{Wien}}\pend{}\pstart{}\textsc{\textcolor{pink}{I. Wollzeile 15}{}\ledrightnote{\textcolor{pink}{Wollzeile}}}, 4. Stock.\pend{}{\bigskip}\pstart\center{}{\pb}Lieber Richard.\pend\pstart
           1) Ich habe noch nichts zu \textcolor{green}{\textsc{Faust}}{}\ledrightnote{\textcolor{green}{Faust}}, da ich den beſtechlichen nicht fand; ich zweifle aber nicht, daſs ich morgen
               Vormittag welche beko{\geminationm}en werde, reflectiren Sie denn
               drauf? Und,\pend
           \pstart
           2.) we{\geminationn} ich keine bekomm, wollen Sie mit mir morgen in
               ein andres Theater (»\textcolor{green}{Karlsſchülerin}{}\ledrightnote{\textcolor{green}{Die Karlsschülerin}}« oder »\textcolor{green}{Touriſten}{}\ledrightnote{\textcolor{green}{Wiener Touristen}}«) gehn?\pend
           \pstart
           3.) \textcolor{blue}{\textsc{Herzl}}{}\ledrightnote{\textcolor{blue}{Theodor Herzl}} iſt da, möchte mit uns, {\pb}dh. Ihnen, \textcolor{blue}{\textsc{Hugo}}{}\ledrightnote{\textcolor{blue}{Hugo von Hofmannsthal}}, mir, eventuell \textcolor{blue}{Bahr}{}\ledrightnote{\textcolor{blue}{Hermann Bahr}}{ }ſoupiren. Ich ſagte ihm, Freitag nach dem \textcolor{blue}{\textsc{Hubermann}}{}\ledrightnote{\textcolor{blue}{Bronisław Huberman}}\textsc{concert} – Sie ſind doch einverſtanden? Zu \textcolor{blue}{\textsc{Bahr}}{}\ledrightnote{\textcolor{blue}{Hermann Bahr}}{ }ſagen Sie vorläufig nichts, weil ich noch ein
               definitives Wort von \textcolor{blue}{\textsc{Herzl}}{}\ledrightnote{\textcolor{blue}{Theodor Herzl}} erwarte. \textcolor{blue}{\textsc{Hugo}}{}\ledrightnote{\textcolor{blue}{Hugo von Hofmannsthal}} theilen Sie’s vielleicht mit?\pend
           \pstart
           4.) bitte kaufen Sie \textsc{vis à vis}{ }{\pb}bei \textcolor{brown}{\textsc{Goldschmidt}}{}\ledrightnote{\textcolor{brown}{Hermann Goldschmiedt {\kaufmannsund} Co.}} die \textcolor{green}{Münchner Allgemeine}{}\ledrightnote{\textcolor{green}{Allgemeine Zeitung}} von Samstag den
                  23. d. mit \label{K_L00425_1v}\edtext{Beilage}{\lemma{\textnormal{\emph{Beilage}}}\Cendnote{\textnormal{wohl wegen: \textcolor{blue}{b. m.}: \emph{\textcolor{green}{Arthur
                        Schnitzler: Sterben}}. In: \emph{\textcolor{green}{Beilage zur
                        Allgemeinen Zeitung}}, Beilage-Nr. 69, 23. 3. 1895,
                     S. 5}}}\label{K_L00425_1h} für mich.\pend
           \pstart
           5.) hier iſt \label{K_L00425_2v}\edtext{\textsc{\textcolor{green}{Carlos}{}\ledrightnote{\textcolor{green}{Don Karlos, Infant von Spanien}}{ }\textcolor{blue}{Schnabl}{}\ledrightnote{\textcolor{blue}{C. Schnabel}}}}{\lemma{\textnormal{\emph{Carlos Schnabl}}}\Cendnote{\textnormal{vermutlich die Edition: \emph{\textcolor{green}{Don Carlos, Infant von Spanien. Ein
                     dramatisches Gedicht. Zum Uebersetzen aus dem Deutschen in das Französische für
                     bereits vorgerückte Schüler, die in den Geist der beiden Idiome tiefer
                     eindringen und die Conversationssprache sich aneignen wollen. Mit Anmerkungen
                     der nöthigen Phraseologie und einem Wörterbuche. Zum Schul- und Privatgebrauch}}.
                     Herausgegeben von \textcolor{blue}{C. Schnabel}, öffentlicher
                     Lehrer. Leipzig: \emph{\textcolor{brown}{Baumgärtner’sche Buchhandlung}}{ }1846.}}}\label{K_L00425_2h}.\pend
           \pstart
           6.) vielleicht – ſo jetzt haben Sie mir telephonirt, alſo es bleibt dabei, {\pb}wir treffen uns im \textcolor{pink}{\textsc{Griensteidl}}{}\ledrightnote{\textcolor{pink}{Café Griensteidl}} gegen 8. Herzlich\pend
           \pstart Ihr \spacefill\mbox{Arth}\pend{}\endnumbering\briefempfaengerindex{Beer-Hofmann, Richard@\textsc{Beer-Hofmann, Richard}!zzzSchnitzler, Arthur@\emph{von Arthur Schnitzler}!1895-03-261@{{[}26. 3. 1895{]}}|)be}\mylabel{h}  \normalsize

\doendnotes{C}
\bigskip
\vfill

\clearpage

\footnotesize

\lohead{\textsc{register}}

% Definiere theindex-Environment komplett neu ohne reledmac
\makeatletter
\renewenvironment{theindex}{%
  \section*{\indexname}%
  \setlength{\parindent}{0pt}%
  \setlength{\parskip}{0pt plus 0.3pt}%
  \let\item\@idxitem
}{%
  \clearpage
}
\makeatother

\IfFileExists{\jobname-pw.ind}{\input{\jobname-pw.ind}}{}

\end{document}

      