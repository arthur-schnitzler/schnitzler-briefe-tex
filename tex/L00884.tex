%% latex-korrekturansicht-vorspann.tex
%% Vorspann für die Korrekturansicht.
%% Lädt die gemeinsame Datei latex-vorspann.tex mit gesetztem Schalter.

\newif\ifkorrekturansicht
\korrekturansichttrue

\input{../tex-inputs/latex-vorspann}


               \section[Arthur Schnitzler an Ludwig Ganghofer, 4. 2. 1899]{ Arthur Schnitzler an Ludwig Ganghofer, 4. 2. 1899}\nopagebreak\mylabel{v}\rehead{ }\normalsize\beginnumbering\briefempfaengerindex{Ganghofer, Ludwig@\textsc{Ganghofer, Ludwig}!zzzSchnitzler, Arthur@\emph{von Arthur Schnitzler}!1899-02-041@{4. 2. 1899}|(be} \toendnotes[C]{\smallbreak\pagebreak[2]} \Standort{München, Monacensia, Nachl. Ludwig Ganghofer, B 170.}
\physDesc{Brief, 1 Blatt, 3 Seiten
\newline{}Handschrift: schwarze Tinte, deutsche Kurrent}\toendnotes[C]{\smallbreak}\pstart
           \noindent{}{\pb}Sehr geehrter Herr, mein Telegramm hat Ihnen bereits
                    mitgetheilt, dſs der »\textcolor{green}{grüne Kakadu}{}\ledrightnote{\textcolor{green}{Der grüne Kakadu. Groteske in einem Akt}}« (mit
                    einigen Strichen natürlich) am \textcolor{brown}{Burgtheater}{}\ledrightnote{\textcolor{brown}{Burgtheater}} zur
                    Aufführg kommt. Das ſoll zu \label{K_L00884_1v}\edtext{Anfang März}{\lemma{\textnormal{\emph{Anfang März}}}\Cendnote{\textnormal{Die Uraufführung fand am 1. 3. 1899
                        statt.}}}\label{K_L00884_1h} geſchehen. Nun habe ich auch mit \textcolor{blue}{\textsc{Fulda}}{}\ledrightnote{\textcolor{blue}{Ludwig Fulda}}, der eben in \textcolor{pink}{Wien}{}\ledrightnote{\textcolor{pink}{Wien}} iſt, wegen der \textcolor{pink}{Berlin}{}\ledrightnote{\textcolor{pink}{Berlin}}er Prem. früher geſprochen, und die
                    Zuſage erhalten, daſs der »\textcolor{green}{Kakadu}{}\ledrightnote{\textcolor{green}{Der grüne Kakadu. Groteske in einem Akt}}« {\pb}\label{K_L00884_2v}\edtext{Anfang April}{\lemma{\textnormal{\emph{Anfang April}}}\Cendnote{\textnormal{Die Premiere am \emph{\textcolor{brown}{Deutschen Theater}} fand am 29. 4. 1899
                        statt.}}}\label{K_L00884_2h}, ſpäteſtens 10. in \textcolor{pink}{Berlin}{}\ledrightnote{\textcolor{pink}{Berlin}} geſpielt werden wird. Ich möchte Sie alſo bitten, das Stück
                    nicht früher zu geben; mir wäre es am liebſten, we{\geminationn}
                    Sie es etwa um den 15. April herum herausbringen könnten, ſo daſs
                    ich von \textcolor{pink}{Berlin}{}\ledrightnote{\textcolor{pink}{Berlin}} aus zu Ihren Proben reiſen
                    könnte. Eine Aufführg in \textcolor{pink}{München}{}\ledrightnote{\textcolor{pink}{München}} vor \textcolor{pink}{Berlin}{}\ledrightnote{\textcolor{pink}{Berlin}} wäre mir in Hinblick auf frühere
                    Verabredungen {\pb}mit \textcolor{blue}{Brahm}{}\ledrightnote{\textcolor{blue}{Otto Brahm}} und \textcolor{blue}{Fulda}{}\ledrightnote{\textcolor{blue}{Ludwig Fulda}}, \uline{nicht} erwünſcht und ich hoffe, es hat keine
                    Schwierigkeiten für Sie, die \label{K_L00884_3v}\edtext{Aufführg bis Mitte April}{\lemma{\textnormal{\emph{Aufführg bis Mitte April}}}\Cendnote{\textnormal{Die Aufführung durch die \emph{\textcolor{brown}{Münchener Litterarische Gesellschaft}} fand
                        am Tag der \textcolor{pink}{Berlin}er Premiere, am
                        29. 4. 1899, im \emph{\textcolor{brown}{Residenztheater}} statt.}}}\label{K_L00884_3h} hinauszuſchieben.\pend
           \pstart
           Iſt ſchon eine Wahl in Hinſicht auf das \label{K_L00884-4v}\edtext{\textcolor{green}{Stück}{}\ledrightnote{\textcolor{green}{Traum eines Frühlingsmorgens}{\newline}\textcolor{green}{Mein Fürst}}}{\lemma{\textnormal{\emph{Stück}}}\Cendnote{\textnormal{Gegeben wurde es mit \emph{\textcolor{green}{Traum eines Frühlingsmorgens}} von \textcolor{blue}{Gabriele D’Annunzio} und \emph{\textcolor{green}{Mein Fürst}} von \textcolor{blue}{Wilhelm von
                            Scholz}.}}}\label{K_L00884-4h} getroffen, das zum \textcolor{green}{Kakadu}{}\ledrightnote{\textcolor{green}{Der grüne Kakadu. Groteske in einem Akt}} gegeben werden ſoll?\pend
           \pstart
           In beſondrer Hochſchätzg ergebenſt{\\[\baselineskip]}\spacefill\mbox{DrArthur Schnitzler}\pend
           \leftskip=0em{}\pstart
           \textcolor{pink}{Wien}{}\ledrightnote{\textcolor{pink}{Wien}}, 4. Feber 99.\pend
           \endnumbering\briefempfaengerindex{Ganghofer, Ludwig@\textsc{Ganghofer, Ludwig}!zzzSchnitzler, Arthur@\emph{von Arthur Schnitzler}!1899-02-041@{4. 2. 1899}|)be}\mylabel{h}  \normalsize

\doendnotes{C}
\bigskip
\vfill

\clearpage

\footnotesize

\lohead{\textsc{register}}

% Definiere theindex-Environment komplett neu ohne reledmac
\makeatletter
\renewenvironment{theindex}{%
  \section*{\indexname}%
  \setlength{\parindent}{0pt}%
  \setlength{\parskip}{0pt plus 0.3pt}%
  \let\item\@idxitem
}{%
  \clearpage
}
\makeatother

\IfFileExists{\jobname-pw.ind}{\input{\jobname-pw.ind}}{}

\end{document}

      