%% latex-korrekturansicht-vorspann.tex
%% Vorspann für die Korrekturansicht.
%% Lädt die gemeinsame Datei latex-vorspann.tex mit gesetztem Schalter.

\newif\ifkorrekturansicht
\korrekturansichttrue

\input{../tex-inputs/latex-vorspann}


               \section[Hugo von Hofmannsthal an Arthur Schnitzler, 27. 7. 1900]{ Hugo von Hofmannsthal an Arthur Schnitzler, 27. 7. 1900}\nopagebreak\mylabel{v}\rehead{ }\normalsize\beginnumbering\briefempfaengerindex{Schnitzler, Arthur@\textsc{Schnitzler, Arthur}!zzzHofmannsthal, Hugo von@\emph{von Hugo von Hofmannsthal}!1900-07-272@{27. 7. 1900}|(be} \toendnotes[C]{\smallbreak\pagebreak[2]} \Standort{CUL, Schnitzler, B 43.}
\physDesc{Brief, 2 Blätter, 7 Seiten
\newline{}Handschrift: schwarze Tinte, deutsche Kurrent\newline{}Ordnung: mit Bleistift von unbekannter Hand nummeriert: »164.1« beziehungsweise »164.2« }\buchAbdrucke{\weitereDrucke{Hugo von Hofmannsthal, Arthur Schnitzler: \emph{Briefwechsel}. Hg. Therese Nickl und Heinrich Schnitzler. Frankfurt am Main: \emph{S. Fischer} 1964, S. 143.} }\toendnotes[C]{\smallbreak}\pstart
           \raggedleft{}{\pb}\textcolor{pink}{Fuſch}{}\ledrightnote{\textcolor{pink}{Bad Fusch}}{ }27 VII.\pend
           \pstart{}mein lieber Arthur \pend\pstart
           es iſt ſehr angenehm, durch die kleine \textcolor{blue}{Dora}{}\ledrightnote{\textcolor{blue}{Dora Michaelis}},
               welche wirklich ein überaus nettes und angenehmes Geſchöpf iſt, von Zeit zu Zeit ein
               Wort über Sie zu hören.\pend
           \pstart
           Die Tage in \textcolor{pink}{Salzburg}{}\ledrightnote{\textcolor{pink}{Salzburg}} mit \textcolor{blue}{Richard}{}\ledrightnote{\textcolor{blue}{Richard Beer-Hofmann}} waren mir doppelt wohlthuend, da ich gerade im Verkehr
               mit ihm immer das Gefühl zu ſeltenen Zusammenseins, ungeſtillten {\pb}Hungers habe.\hspace*{1.5em}Gerade an dem Tag, wo Ihr Eure Fußreiſe antretet, dürfte ich zur
               Waffenübung einrücken. Nachher werd ich, um die Mitte September,
               wahrſcheinlich an den \textcolor{pink}{\textsc{Gardasee}}{}\ledrightnote{\textcolor{pink}{Lago di Garda}} gehen.\pend
           \pstart
           Nun aber, die nächſten Tage, etwa vom letzten July an, bin ich in \textcolor{pink}{Salzburg}{}\ledrightnote{\textcolor{pink}{Salzburg}}, im \textcolor{pink}{oeſterreich.
                  Hof}{}\ledrightnote{\textcolor{pink}{Österreichischer Hof}}. Auch meine \textcolor{blue}{Eltern}{}\ledrightnote{→\textcolor{blue}{Anna von Hofmannsthal}{\newline}→\textcolor{blue}{Hugo August von Hofmannsthal}} werden zur ſelben Zeit dort ſein, und einen Theil der Zeit auch die
                  {\pb}\textcolor{blue}{Gerty}{}\ledrightnote{\textcolor{blue}{Gertrude von Hofmannsthal}} mit ihrer \textcolor{blue}{Mutter}{}\ledrightnote{→\textcolor{blue}{Franziska Schlesinger}}.\pend
           \pstart
           Hier ſcheint mir, indem ich ſchreibe, in dem Nicht-erwähnen einer beſtehenden
               Situation zum erſten Mal eine wirkliche Unwahrheit zu liegen, und ſo will ich denn,
               wie vor einigen Tagen dem \textcolor{blue}{Richard}{}\ledrightnote{\textcolor{blue}{Richard Beer-Hofmann}}, auch Ihnen
               gern ſagen, daß ich die \textcolor{blue}{Gerty}{}\ledrightnote{\textcolor{blue}{Gertrude von Hofmannsthal}} im Lauf des nächſten
               Frühjahrs heirathen werde. Ich bitte Sie, davon zu niemandem als etwa {\pb}zu \textcolor{blue}{Richard}{}\ledrightnote{\textcolor{blue}{Richard Beer-Hofmann}} zu ſprechen. Freilich weiß ich daſs ein ſolches Gerücht und die
               Überzeugung maſſenhafter Menſchen von dieſer Sache ſeit langem, ja mir ſcheint ſchon
               ſeit mehreren Jahren beſteht. Aber das war, bevor in den beiden, um die es ſich
               handelt, irgend ein Gedanke, ja ſogar bevor der Wunsch nach einer ſolchen
               Verwirklichung beſtanden hatte. Und ſo hatte das Gerede damals, und hat auch jetzt
                  {\pb}mit der Sache ſelbſt
               eigentlich nichts zu thun, und ſoll auch davon getrennt bleiben. Denn wenn man auch
               dazu geführt wird, etwas zu thun, was die Leute vorausgeſagt haben, ſo iſt es doch,
               indem man’s thut durch ganze Abgründe von dem, was die andern in ihren Köpfen haben
               getrennt. – Ich bin alſo bis {\pb}halben Auguſt in \textcolor{pink}{Salzburg}{}\ledrightnote{\textcolor{pink}{Salzburg}}. Ich hoffe
               beſtimmt, daſs wir uns da ſehen. Sie können mich natürlich allein haben, ſoviel wir
               uns das verlangen. Was ſollte ſich darin ändern oder künftig ändern müſſen? Und
               übrigens ergibt ja das Rad eine nette Form des Zuſa{\geminationm}enſeins.\pend
           \pstart
           Gearbeitet hab ich recht {\pb}wenig,
               will ſolche Zeiten aber von je\introOben{}t\introOben{}zt an ohne dieſe übermäßige
               Ungeduld ertragen. Auf Ihre phantaſtiſche \textcolor{green}{Novelle}{}\ledrightnote{→\textcolor{green}{Lieutenant Gustl. Novelle}} freu ich mich ſehr. Wenn ich ſie bald hören könnte, oder
               die \textcolor{green}{lange}{}\ledrightnote{→\textcolor{green}{Frau Bertha Garlan. Roman}}? Das iſt immer eine
               Freude, der nachher das Leſen nicht mehr gleichkommt.\pend
           \pstart Alſo hoffentlich ſehn wir uns bald. Von Herzen Ihr \spacefill\mbox{Hugo.}\pend{}\endnumbering\briefempfaengerindex{Schnitzler, Arthur@\textsc{Schnitzler, Arthur}!zzzHofmannsthal, Hugo von@\emph{von Hugo von Hofmannsthal}!1900-07-272@{27. 7. 1900}|)be}\mylabel{h}  \normalsize

\doendnotes{C}
\bigskip
\vfill

\clearpage

\footnotesize

\lohead{\textsc{register}}

% Definiere theindex-Environment komplett neu ohne reledmac
\makeatletter
\renewenvironment{theindex}{%
  \section*{\indexname}%
  \setlength{\parindent}{0pt}%
  \setlength{\parskip}{0pt plus 0.3pt}%
  \let\item\@idxitem
}{%
  \clearpage
}
\makeatother

\IfFileExists{\jobname-pw.ind}{\input{\jobname-pw.ind}}{}

\end{document}

      