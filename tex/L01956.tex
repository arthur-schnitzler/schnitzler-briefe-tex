%% latex-korrekturansicht-vorspann.tex
%% Vorspann für die Korrekturansicht.
%% Lädt die gemeinsame Datei latex-vorspann.tex mit gesetztem Schalter.

\newif\ifkorrekturansicht
\korrekturansichttrue

\input{../tex-inputs/latex-vorspann}


               \section[Arthur und Olga Schnitzler an Hermann Bahr, 18. 9. 1910]{ Arthur und Olga Schnitzler an Hermann Bahr, 18. 9. 1910}\nopagebreak\mylabel{v}\rehead{ }\normalsize\beginnumbering\briefempfaengerindex{Bahr, Hermann@\textsc{Bahr, Hermann}!zzzSchnitzler, Olga@\emph{von Olga Schnitzler}!1910-09-181@{18. 9. 1910}|(be}\briefempfaengerindex{Bahr, Hermann@\textsc{Bahr, Hermann}!zzzSchnitzler, Arthur@\emph{von Arthur Schnitzler}!1910-09-181@{18. 9. 1910}|(be} \toendnotes[C]{\smallbreak\pagebreak[2]} \Standort{TMW, HS AM 60148 Ba.}
\physDesc{Bildpostkarte
\newline{}Handschrift Arthur Schnitzler: Bleistift, deutsche Kurrent\newline{}Handschrift Olga Schnitzler: Bleistift\newline{}Versand: Stempel: »\nobreak{}\oindex{Darmstadt@\textbf{Darmstadt}, \emph{Besiedelter Ort (A.BSO)}|pwk}Darmstadt, 18 9 10, 12\nobreak{}«.  \newline{}Ordnung: Lochung }\buchAbdrucke{\weitereDrucke{1) \emph{18. 9. 1910, Abschrift.} In: Arthur Schnitzler: \emph{The Letters of Arthur Schnitzler to Hermann Bahr}. Edited, annotated, and with an introduction, by Donald G.
                        Daviau. Chapel Hill: \emph{The University of North Carolina Press} 1978, S. 104 (University of North Carolina studies in the Germanic languages
                        and literatures, 89).} \weitereDrucke{2) Hermann Bahr, Arthur Schnitzler: \emph{Briefwechsel, Aufzeichnungen, Dokumente (1891–1931)}. Hg. Kurt Ifkovits und Martin Anton Müller. Göttingen: \emph{Wallstein} 2018, S. 437.} }\toendnotes[C]{\smallbreak}\pstart{}{\pb}\textcolor{pink}{Wien}{}\ledrightnote{\textcolor{pink}{Wien}}.\pend{}\pstart{}\textsc{Herrn Hermann Bahr}\pend{}\pstart{}\textcolor{pink}{Wien – \textsc{Ob St Veit}}{}\ledrightnote{\textcolor{pink}{Ober Sankt Veit}}\pend{}\pstart{}\textsc{\textcolor{pink}{Veitlissengasse}{}\ledrightnote{\textcolor{pink}{Veitlissengasse}}}\pend{}{\bigskip}\pstart
           \noindent{}\centering{}\textcolor{gray}{\textbf{{\pb}\textcolor{pink}{Darmstadt}{}\ledrightnote{\textcolor{pink}{Darmstadt}}. Glockenspiel}}\pend
           \pstart
           \raggedleft{}{\pb}18/9 1910.\pend
           \pstart
           Herzliche Grüße dir und deiner verehrten \textcolor{blue}{Gattin}{}\ledrightnote{→\textcolor{blue}{Anna Bahr-Mildenburg}}\pend
           \pstart Dein\spacefill\mbox{Arthur}\pend{}\pstart \spacefill\mbox{{[}hs. O. Schnitzler:{]} OlgaSchnitzler}\pend{}\endnumbering\briefempfaengerindex{Bahr, Hermann@\textsc{Bahr, Hermann}!zzzSchnitzler, Olga@\emph{von Olga Schnitzler}!1910-09-181@{18. 9. 1910}|)be}\briefempfaengerindex{Bahr, Hermann@\textsc{Bahr, Hermann}!zzzSchnitzler, Arthur@\emph{von Arthur Schnitzler}!1910-09-181@{18. 9. 1910}|)be}\mylabel{h}  \normalsize

\doendnotes{C}
\bigskip
\vfill

\clearpage

\footnotesize

\lohead{\textsc{register}}

% Definiere theindex-Environment komplett neu ohne reledmac
\makeatletter
\renewenvironment{theindex}{%
  \section*{\indexname}%
  \setlength{\parindent}{0pt}%
  \setlength{\parskip}{0pt plus 0.3pt}%
  \let\item\@idxitem
}{%
  \clearpage
}
\makeatother

\IfFileExists{\jobname-pw.ind}{\input{\jobname-pw.ind}}{}

\end{document}

      