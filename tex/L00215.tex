%% latex-korrekturansicht-vorspann.tex
%% Vorspann für die Korrekturansicht.
%% Lädt die gemeinsame Datei latex-vorspann.tex mit gesetztem Schalter.

\newif\ifkorrekturansicht
\korrekturansichttrue

\input{../tex-inputs/latex-vorspann}


               \section[Arthur Schnitzler an Wilhelm Bölsche, 1. 6. 1893]{ Arthur Schnitzler an Wilhelm Bölsche, 1. 6. 1893}\nopagebreak\mylabel{v}\rehead{ }\normalsize\beginnumbering\briefempfaengerindex{Boelsche, Wilhelm@\textsc{Bölsche, Wilhelm}!zzzSchnitzler, Arthur@\emph{von Arthur Schnitzler}!1893-06-011@{1. 6. 1893}|(be} \toendnotes[C]{\smallbreak\pagebreak[2]} \Standort{Wrocław, Biblioteka Uniwersytecka, Böl.Pis 1767.}
\physDesc{Brief, 1 Blatt (Briefpapier mit Trauerrand), 4 Seiten
\newline{}Handschrift: schwarze Tinte, deutsche Kurrent}\buchAbdrucke{\weitereDrucke{1) Alois Woldan: \emph{Arthur Schnitzler – Briefe an Wilhelm Bölsche.} In: \emph{Germanica Wratislaviensia} (1987) Nr. 77, S. 461–462.} \weitereDrucke{2) Wilhelm Bölsche: \emph{Briefwechsel. Mit Autoren der Freien Bühne}. Hg. Gerd-Hermann Susen. Berlin: \emph{Weidler} 2010, S. 685 (Werke und Briefe. Wissenschaftliche Ausgabe, Briefe I).} }\toendnotes[C]{\smallbreak}\pstart
           \raggedleft{}{\pb}1. Juni 93\pend
           \pstart{}Sehr geehrter Herr\label{K_L00215_1v}\edtext{Doktor}{\lemma{\textnormal{\emph{Doktor}}}\Cendnote{\textnormal{\textcolor{blue}{Bölsche} hatte zwar studiert, aber
                            keinen Universitätsabschluss.}}}\label{K_L00215_1h},\pend\pstart
           eine Frage: Wollen Sie mein dreiaktiges Schauſpiel \textcolor{green}{\uline{Das Märchen}}{}\ledrightnote{\textcolor{green}{Das Märchen. Schauspiel in drei Aufzügen}}, welches nächſte Saiſon am \textcolor{brown}{Leſſingtheater}{}\ledrightnote{\textcolor{brown}{Lessing-Theater}}
                    zur Aufführung kommt, in der \textcolor{green}{\uline{Freien Bühne}}{}\ledrightnote{\textcolor{green}{Freie Bühne für den Entwickelungskampf der Zeit}} bringen? Falls Sie im Princip einverſtanden ſind, ſo erlaube ich mir die
                    weitere Frage, \uline{unter welchen Bedingungen}{ }{\pb}und wann Sie mit der Veröffentlichung begi{\geminationn}en kö{\geminationn}ten. Mir läge
                    daran, daſs der erſte Akt ſchon im Juliheft erſchiene – das Stück ſelbſt hab ich
                        \strikeout{vor} Ihnen vor etwa 1 Jahre als Manuscript
                    gedruckt, eingeſchickt; ich ſende Ihnen natürlich ein andres Exemplar, ſobald
                    Sie das Drama veröffentlichen wollen. –\pend
           \pstart
           Vor etwa 6 oder 7 Wochen hab {\pb}ich Ihnen eine kleine
                    Skizze geſandt »\textcolor{green}{Die Braut}{}\ledrightnote{\textcolor{green}{Die Braut}}« – was iſt’s mit
                    der? –\pend
           \pstart
           – Jedenfalls will ich noch das höfliche Erſuchen hinzuſetzen, mich nicht zu lang
                    auf Antwort warten zu laſſen; es kommt mir auf eine raſche Erledigung meiner
                    Frage an, und ich appellire an Ihre Liebenswürdigkeit, mir Ihre Entſcheidung in
                    möglichſt kurzer Zeit zu{\pb}ko{\geminationm}en zu laſſen.\pend
           \pstart
           Mit beſondrer Hochachtung{\\[\baselineskip]}\spacefill\mbox{ Dr Arthur Schnitzler}\pend
           \leftskip=0em{}\pstart
           \noindent{}\textcolor{pink}{\textsc{Wien I. Grillparzerstraße 7}}{}\ledrightnote{\textcolor{pink}{Grillparzerstraße}}.\pend
           \endnumbering\briefempfaengerindex{Boelsche, Wilhelm@\textsc{Bölsche, Wilhelm}!zzzSchnitzler, Arthur@\emph{von Arthur Schnitzler}!1893-06-011@{1. 6. 1893}|)be}\mylabel{h}  \normalsize

\doendnotes{C}
\bigskip
\vfill

\clearpage

\footnotesize

\lohead{\textsc{register}}

% Definiere theindex-Environment komplett neu ohne reledmac
\makeatletter
\renewenvironment{theindex}{%
  \section*{\indexname}%
  \setlength{\parindent}{0pt}%
  \setlength{\parskip}{0pt plus 0.3pt}%
  \let\item\@idxitem
}{%
  \clearpage
}
\makeatother

\IfFileExists{\jobname-pw.ind}{\input{\jobname-pw.ind}}{}

\end{document}

      