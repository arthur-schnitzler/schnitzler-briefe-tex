%% latex-korrekturansicht-vorspann.tex
%% Vorspann für die Korrekturansicht.
%% Lädt die gemeinsame Datei latex-vorspann.tex mit gesetztem Schalter.

\newif\ifkorrekturansicht
\korrekturansichttrue

\input{../tex-inputs/latex-vorspann}


               \section[Arthur und Olga Schnitzler an Hermann Bahr, 14. 7. 1906]{ Arthur und Olga Schnitzler an Hermann Bahr, 14. 7. 1906}\nopagebreak\mylabel{v}\rehead{ }\normalsize\beginnumbering\briefempfaengerindex{Bahr, Hermann@\textsc{Bahr, Hermann}!zzzSchnitzler, Olga@\emph{von Olga Schnitzler}!1906-07-141@{14. 7. 1906}|(be}\briefempfaengerindex{Bahr, Hermann@\textsc{Bahr, Hermann}!zzzSchnitzler, Arthur@\emph{von Arthur Schnitzler}!1906-07-141@{14. 7. 1906}|(be} \toendnotes[C]{\smallbreak\pagebreak[2]} \Standort{TMW, HS AM 60174 Ba.}
\physDesc{Bildpostkarte
\newline{}Handschrift Arthur Schnitzler: Bleistift, deutsche Kurrent\newline{}Handschrift Olga Schnitzler: Bleistift, lateinische Kurrent\newline{}Versand: 1) Stempel: »\nobreak{}1{[}4{]}. 7. 06, 2–5 E\nobreak{}«.  2) Stempel: »\nobreak{}\oindex{Venedig@\textbf{Venedig}, \emph{Besiedelter Ort (A.BSO)}|pwk}Venezia, 16. 7{[}. 06{]}\nobreak{}«. \newline{}Ordnung: Lochung }\buchAbdrucke{\weitereDrucke{1) \emph{14. 7. 1906, Abschrift.} In: Arthur Schnitzler: \emph{The Letters of Arthur Schnitzler to Hermann Bahr}. Edited, annotated, and with an introduction, by Donald G.
                        Daviau. Chapel Hill: \emph{The University of North Carolina Press} 1978, S. 95 (University of North Carolina studies in the Germanic languages
                        and literatures, 89).} \weitereDrucke{2) Hermann Bahr, Arthur Schnitzler: \emph{Briefwechsel, Aufzeichnungen, Dokumente (1891–1931)}. Hg. Kurt Ifkovits und Martin Anton Müller. Göttingen: \emph{Wallstein} 2018, S. 380.} }\toendnotes[C]{\smallbreak}\pstart{}{\pb}\textsc{Herrn}\pend{}\pstart{}\textsc{Hermann Bahr}\pend{}\pstart{}\textsc{\textcolor{pink}{Venezia}{}\ledrightnote{\textcolor{pink}{Italien}}}\pend{}\pstart{}\textsc{\textcolor{pink}{Casa \textcolor{blue}{Petrarca}{}\ledrightnote{→\textcolor{blue}{Francesco Petrarca}}}{}\ledrightnote{\textcolor{pink}{Casa Petrarca}}}\pend{}\pstart{}\textsc{\textcolor{pink}{Italie}{}\ledrightnote{\textcolor{pink}{Venedig}}}\pend{}{\bigskip}\pstart
           \noindent{}\centering{}\textcolor{gray}{\textbf{{\pb}Hilsen fra \textcolor{pink}{Marienlyst Parken}{}\ledrightnote{\textcolor{pink}{Marienlyst}}}}\pend
           \pstart
           \textsc{{\pb}\textcolor{pink}{Marienlyst}{}\ledrightnote{\textcolor{pink}{Marienlyst}}}, 14. 7. 906\pend
           \pstart
           hier, lieber Hermann, wohnen wir ſeit 14 Tagen. Es ist wunderſchön,
               und we{\geminationn} du herkämſt, kö{\geminationn}teſt du ein angenehmes Leben, ohne Strand zwar, aber auch ohne Brandwunden führen.
               Wir bleiben bis auf weiteres.\pend
           \pstart
           Herzlichſt{\\[\baselineskip]}dein \spacefill\mbox{A.}\pend
           \leftskip=0em{}\pstart
           \noindent{}{\pb}{[}hs. O. Schnitzler:{]} (\label{T_L01612_1v}\edtext{Dies}{\lemma{\textnormal{\emph{Dies}}}\Cendnote{\textnormal{handschriftlicher Pfeil auf eine Statue}}}\label{T_L01612_1h} ſoll \textsc{\textcolor{green}{Hamlet}{}\ledrightnote{→\textcolor{green}{Hamlet}} sein.})\pend
           \pstart Herzliche Erwiderung Ihrer lieben Grüsse!
               { }\spacefill\mbox{Olga Schnitzler.}\pend{}\endnumbering\briefempfaengerindex{Bahr, Hermann@\textsc{Bahr, Hermann}!zzzSchnitzler, Olga@\emph{von Olga Schnitzler}!1906-07-141@{14. 7. 1906}|)be}\briefempfaengerindex{Bahr, Hermann@\textsc{Bahr, Hermann}!zzzSchnitzler, Arthur@\emph{von Arthur Schnitzler}!1906-07-141@{14. 7. 1906}|)be}\mylabel{h}  \normalsize

\doendnotes{C}
\bigskip
\vfill

\clearpage

\footnotesize

\lohead{\textsc{register}}

% Definiere theindex-Environment komplett neu ohne reledmac
\makeatletter
\renewenvironment{theindex}{%
  \section*{\indexname}%
  \setlength{\parindent}{0pt}%
  \setlength{\parskip}{0pt plus 0.3pt}%
  \let\item\@idxitem
}{%
  \clearpage
}
\makeatother

\IfFileExists{\jobname-pw.ind}{\input{\jobname-pw.ind}}{}

\end{document}

      