%% latex-korrekturansicht-vorspann.tex
%% Vorspann für die Korrekturansicht.
%% Lädt die gemeinsame Datei latex-vorspann.tex mit gesetztem Schalter.

\newif\ifkorrekturansicht
\korrekturansichttrue

\input{../tex-inputs/latex-vorspann}


               \section[Karl Kraus an Arthur Schnitzler, 11. 1. 1893]{ Karl Kraus an Arthur Schnitzler, 11. 1. 1893}\nopagebreak\mylabel{v}\rehead{ }\normalsize\beginnumbering\briefempfaengerindex{Schnitzler, Arthur@\textsc{Schnitzler, Arthur}!zzzKraus, Karl@\emph{von Karl Kraus}!1893-01-111@{11. 1. 1893}|(be} \toendnotes[C]{\smallbreak\pagebreak[2]} \Standort{DLA, A:Schnitzler, 69.61.}
\physDesc{Brief, 1 Blatt, 2 Seiten
\newline{}Handschrift: schwarze Tinte, deutsche Kurrent}\buchAbdrucke{\weitereDrucke{\emph{Karl Kraus und Arthur Schnitzler. Eine Dokumentation.} Hg. Reinhard Urbach. In: \emph{Literatur und Kritik}, Bd. 49, Oktober 1970, S. 514.} }\toendnotes[C]{\smallbreak}\pstart
           \noindent{}{\pb}\textcolor{gray}{\textbf{Karl Kraus}}\hfill \textcolor{gray}{\textbf{\textcolor{pink}{Wien}{}\ledrightnote{\textcolor{pink}{Wien}},}}{ }11/I \textcolor{gray}{\textbf{189}}3\pend
           \pstart
           \raggedleft{}\textcolor{gray}{\textbf{\textcolor{pink}{I., Maximilianstr. 13}{}\ledrightnote{\textcolor{pink}{Mahlerstraße}}.}}\pend
           \pstart{}Mein guter Herr Docter!\pend\pstart
           Anbei mit beſtem Danke für Ihre frdl. Bemühungen 1 Sitz neben Ihren \textcolor{blue}{Freunden}{}\ledrightnote{→\textcolor{blue}{Richard Beer-Hofmann}{\newline}→\textcolor{blue}{Hugo von Hofmannsthal}}; nur Herr
                        \textcolor{blue}{Schick}{}\ledrightnote{\textcolor{blue}{Friedrich Schik}}{ }ſitzt ein paar Sitze vor Ihnen. Ich hatte
                    nichts anderes, Doctor! Alſo \textcolor{blue}{Salten}{}\ledrightnote{\textcolor{blue}{Felix Salten}} kommt
                    auch? Na, das iſt ja ſehr schön! Das wird eine Hetz’ werden!! Bitte, lachen Sie
                    mir nur nicht zu viel und machen Sie in der erſten Reihe ein recht freundliches
                    Geſicht!\pend
           \pstart
           Erſuche höflichſt, da ich 24 Stunden vor d. \textcolor{green}{Vorstellung}{}\ledrightnote{→\textcolor{green}{Die Räuber}} dem \textcolor{blue}{Director}{}\ledrightnote{→\textcolor{blue}{Moriz von Barth}} abliefern muſs, bis Freitag
                    mittag den Betrag 1 fl. 20 zu ſchicken. {\pb}Ein kleines Deficit dürfte
                    ich haben; \uline{alle} Karten bring’ ich \uline{nicht} an!\pend
           \pstart
           Ich bin ſehr gerne bereit, eine kleine \label{K_L00156_1v}\edtext{Notiz}{\lemma{\textnormal{\emph{Notiz}}}\Cendnote{\textnormal{Diese
                        schrieb nicht \textcolor{blue}{Kraus}, sondern \textcolor{blue}{Josef Schmid-Braunfels} (\emph{\textcolor{green}{Arthur Schnitzler: Anatol}}. In: \emph{\textcolor{green}{Neue litterarische Blätter}}, Jg. 1,
                            Nr. 7, 1. 4. 1893, S. 87–88).}}}\label{K_L00156_1h} über Ihren »\textcolor{green}{Anatol}{}\ledrightnote{\textcolor{green}{Anatol}}« in den »\textcolor{green}{\uline{Neuen litterariſchen Blättern}}{}\ledrightnote{\textcolor{green}{Neue litterarische Blätter}}« (\textcolor{pink}{Bremen}{}\ledrightnote{\textcolor{pink}{Bremen}}, Herausgeber \textcolor{blue}{Franziskus Haehnel}{}\ledrightnote{\textcolor{blue}{Franziskus Haehnel}}, Verlag \textcolor{brown}{Kühtmann}{}\ledrightnote{\textcolor{brown}{Kühtmann}}) zu bringen. Nur müſsten Sie einen \textcolor{green}{Recensionsexemplarabgang}{}\ledrightnote{→\textcolor{green}{Eingesandte Neuerscheinungen [Arthur Schnitzler: Anatol]}} an dieſe
                    Monatsblätter von \strikeout{d} Ihrem \textcolor{blue}{Verleger}{}\ledrightnote{→\textcolor{blue}{Samuel Fischer}} erwirken.\pend
           \pstart
           \textcolor{blue}{Alexander Engel}{}\ledrightnote{\textcolor{blue}{Alexander Engel}} dürfte in den \textcolor{brown}{Breslauer Monatsblättern}{}\ledrightnote{\textcolor{brown}{Monatsblätter}} (\textcolor{blue}{Paul Barsch}{}\ledrightnote{\textcolor{blue}{Paul Barsch}}) bringen.\pend
           \pstart
           Und nun herzlichen Gruß{\\[\baselineskip]} von Ihrem ſehr ergebenen \spacefill\mbox{Karl
                        Kraus}\pend
           \leftskip=0em{}\pstart
           \noindent{}\textcolor{pink}{Wien}{}\ledrightnote{\textcolor{pink}{Wien}}\pend
           \endnumbering\briefempfaengerindex{Schnitzler, Arthur@\textsc{Schnitzler, Arthur}!zzzKraus, Karl@\emph{von Karl Kraus}!1893-01-111@{11. 1. 1893}|)be}\mylabel{h}  \normalsize

\doendnotes{C}
\bigskip
\vfill

\clearpage

\footnotesize

\lohead{\textsc{register}}

% Definiere theindex-Environment komplett neu ohne reledmac
\makeatletter
\renewenvironment{theindex}{%
  \section*{\indexname}%
  \setlength{\parindent}{0pt}%
  \setlength{\parskip}{0pt plus 0.3pt}%
  \let\item\@idxitem
}{%
  \clearpage
}
\makeatother

\IfFileExists{\jobname-pw.ind}{\input{\jobname-pw.ind}}{}

\end{document}

      