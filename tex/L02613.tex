%% latex-korrekturansicht-vorspann.tex
%% Vorspann für die Korrekturansicht.
%% Lädt die gemeinsame Datei latex-vorspann.tex mit gesetztem Schalter.

\newif\ifkorrekturansicht
\korrekturansichttrue

\input{../tex-inputs/latex-vorspann}


               \section[Paul Goldmann an Arthur Schnitzler, 21. 3. {[}1894{]}]{ Paul Goldmann an Arthur Schnitzler, 21. 3. {[}1894{]}}\nopagebreak\mylabel{v}\rehead{ }\normalsize\beginnumbering\briefempfaengerindex{Schnitzler, Arthur@\textsc{Schnitzler, Arthur}!zzzGoldmann, Paul@\emph{von Paul Goldmann}!1894-03-213@{21. 3. {[}1894{]}}|(be} \toendnotes[C]{\smallbreak\pagebreak[2]} \Standort{DLA, A:Schnitzler, HS.NZ85.1.3164.}
\physDesc{Brief, 1 Blatt, 4 Seiten
\newline{}Handschrift: schwarze Tinte, deutsche Kurrent
\newline{}Schnitzler: 1) mit Bleistift auf dem ersten Blatt die Jahreszahl »94« vermerkt 2) mit rotem Buntstift eine Unterstreichung}\toendnotes[C]{\smallbreak}\pstart
           \raggedleft{}{\pb}\textsc{\textcolor{pink}{Paris}{}\ledrightnote{\textcolor{pink}{Paris}}}, 21. März.\pend
           \pstart\center{}Mein lieber Freund,\pend\pstart
           Es iſt wirklich wahr: Seit dem Empfang Deines lieben Briefes iſt kein Tag vergangen,
               wo ich Dir nicht ſchreiben wollte. Heut habe ich endlich einmal ein wenig Zeit.\pend
           \pstart
           Die Überſetzung Deiner \label{K_L02613-3v}\edtext{Artikel ins
                  Franzöſiſche}{\lemma{\textnormal{\emph{Artikel ins
                  Franzöſiſche}}}\Cendnote{\textnormal{Es kam in Folge nur zur
                  Übersetzung des Einakters \emph{\textcolor{green}{Weihnachts-Einkäufe}}. Im Brief \textcolor{blue}{Albert}s
                  an \textcolor{blue}{Schnitzler} vom 9. 4. 1894
                  schreibt er, dass er bereits an der Übersetzung sitze (\emph{DLA}, HS.1985.1.2331,1): \textcolor{blue}{Arthur Schnitzler}: \emph{\textcolor{green}{Les Emplettes de Noël}}. Traduit de l’allemand par \textcolor{blue}{Henri Albert}. In: \emph{\textcolor{green}{L’Idée libre. Revue mensuelle de Littérature et d’Art}},
                     Jg. 3, Nr. 5–6, Mai–Juni 1894, S. 215–225. \textcolor{blue}{Schnitzler} beurteilte die Qualität der
                  Übersetzung negativ, vgl. A. S.: \emph{Tagebuch}, 21. 7. 1894.}}}\label{K_L02613-3h} habe ich ſofort nach meiner Bekanntwerdung mit \textsc{\textcolor{blue}{Albert}{}\ledrightnote{\textcolor{blue}{Henri Albert}}} beſprochen. Er iſt gleich bereit, wird gewiß auch etwas in einer der jungen
               Revüen anbringen können. Aber ein {\pb}Haken iſt da: die
               Revüen zahlen nicht, \textsc{\textcolor{blue}{Albert}{}\ledrightnote{\textcolor{blue}{Henri Albert}}} muß von ſeiner Feder leben. Du kannſt \strikeout{ihm} daher
               die Frage am Beſten löſen, indem Du ihm ein Honorar anbieteſt. Natürlich macht er
               ſehr geringe Anſprüche. \label{K_L02613-4v}\edtext{Schicke ihm
               alſo Deine Schriften, mache ihm unumwunden den Honorar-Vorſchlag}{\lemma{\textnormal{\emph{Schicke … Honorar-Vorſchlag}}}\Cendnote{\textnormal{Aus dem Brief, den \textcolor{blue}{Albert} am 23. 5. 1894 an \textcolor{blue}{Schnitzler} schrieb, geht hervor, dass ein nicht näher
                  bezeichnete Summe bezahlt wurde (\emph{DLA}, HS.1985.1.2331,2). Er bedankt sich zudem
                  für die Zusendung des \emph{\textcolor{green}{Märchen}}s.}}}\label{K_L02613-4h}, indem
               Du Dich auf meinen Brief beziehſt, und überlaß mir das übrige. Die Fixirung der Summe
               mache ich dann ſchon aus, um zwiſchen {\pb}Euch Beiden
               keine \label{K_L02613-2v}\edtext{\textsc{\begin{otherlanguage}{french}Gêne\end{otherlanguage}}}{\lemma{\textnormal{\emph{Gêne}}}\Cendnote{\textnormal{französisch: Befangenheit, Verlegenheit.
                  »Être dans la gêne« bedeutet »in Geldverlegenheit sein«.}}}\label{K_L02613-2h} aufkommen zu
               laſſen. Schreibe ihm ſofort. Denn er hat gerade jetzt etwas Zeit, die er mit einer
               Überſetzung ausfüllen könnte.\pend
           \pstart
           Sonſt erfahre ich aus Deinem Briefe mit Freuden, daß du rüſtig weiter ſchaffſt. Mehr
               brauche ich nicht zu wiſſen. Über den Erfolg bin ich beruhigt. Aber ich habe ſchon
               gar ſo lange nichts von Dir geleſen. Könnteſt Du mir nicht einmal eine Kleinigkeit
               ſchicken? Ich gebe ſie eventuell wieder zurück.\pend
           \pstart
           {\pb}Vielen Dank für die intereſſanten poſitiven
               Mittheilungen. \textsc{\textcolor{blue}{Hermann Bahr}{}\ledrightnote{\textcolor{blue}{Hermann Bahr}}}{ }\label{K_L02613-1v}\edtext{gründet ein \textcolor{brown}{Blatt}{}\ledrightnote{→\textcolor{brown}{Die Zeit. Wiener Wochenschrift}}}{\lemma{\textnormal{\emph{gründet ein Blatt}}}\Cendnote{\textnormal{Es handelt sich um die seit
                     Frühjahr 1894 laufende Entwicklung der »Wiener Wochenschrift« \emph{\textcolor{brown}{Die Zeit}}, die ab Oktober des
                     Jahres erschien. Als Herausgeber fungierte \textcolor{blue}{Hermann Bahr} gemeinsam mit \textcolor{blue}{Heinrich Kanner} und \textcolor{blue}{Isidor Singer}. \textcolor{blue}{Bahr} verantwortete
                  den Kulturteil.}}}\label{K_L02613-1h}? Der \textcolor{blue}{Burſch}{}\ledrightnote{→\textcolor{blue}{Hermann Bahr}} weiß wirklich aus Steinen Brot zu machen. Iſt das aber auch
               ſeriös?\pend
           \pstart
           Von mir? Hoffnungslosigkeit und Verzweiflung.\pend
           \pstart
           Grüß die Freunde vielmals und vergiß nicht, daß wir Zwei uns im Sommer treffen
               wollen. Sei von Herzen gegrüßt und bedankt für Deine Treue (Du biſt der \uline{Einzige}, der meine Artikel lobt!). Schreibe recht
               bald.\pend
           \pstart
           In Treue {\\[\baselineskip]}Dein \spacefill\mbox{Paul Goldm}\pend
           \leftskip=0em{}\endnumbering\briefempfaengerindex{Schnitzler, Arthur@\textsc{Schnitzler, Arthur}!zzzGoldmann, Paul@\emph{von Paul Goldmann}!1894-03-213@{21. 3. {[}1894{]}}|)be}\mylabel{h}\begin{anhang}\end{anhang}\normalsize

\doendnotes{C}
\bigskip
\vfill

\clearpage

\footnotesize

\lohead{\textsc{register}}

% Definiere theindex-Environment komplett neu ohne reledmac
\makeatletter
\renewenvironment{theindex}{%
  \section*{\indexname}%
  \setlength{\parindent}{0pt}%
  \setlength{\parskip}{0pt plus 0.3pt}%
  \let\item\@idxitem
}{%
  \clearpage
}
\makeatother

\IfFileExists{\jobname-pw.ind}{\input{\jobname-pw.ind}}{}

\end{document}

      