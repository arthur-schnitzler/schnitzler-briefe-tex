%% latex-korrekturansicht-vorspann.tex
%% Vorspann für die Korrekturansicht.
%% Lädt die gemeinsame Datei latex-vorspann.tex mit gesetztem Schalter.

\newif\ifkorrekturansicht
\korrekturansichttrue

\input{../tex-inputs/latex-vorspann}


               \section[Arthur Schnitzler an Richard Beer-Hofmann, {[}17.? 11. 1908{]}]{ Arthur Schnitzler an Richard Beer-Hofmann, {[}17.? 11. 1908{]}}\nopagebreak\mylabel{v}\rehead{ }\normalsize\beginnumbering\briefempfaengerindex{Beer-Hofmann, Richard@\textsc{Beer-Hofmann, Richard}!zzzSchnitzler, Arthur@\emph{von Arthur Schnitzler}!1908-11-172@{{[}17.? 11. 1908{]}}|(be} \toendnotes[C]{\smallbreak\pagebreak[2]} \Standort{YCGL, MSS 31.}
\physDesc{Briefkarte, Umschlag
\newline{}Handschrift: Bleistift, deutsche Kurrent\newline{}Versand: ohne postalischen Übermittlungsvermerk 
\newline{}Beer-Hofmann: auf der Rückseite des Umschlags mit blauem Buntstift datiert: »19/XI 08«, wobei es sich um den Empfang oder eine (nicht überlieferte) Beantwortung
            handeln könnte }\buchAbdrucke{\weitereDrucke{Arthur Schnitzler, Richard Beer-Hofmann: \emph{Briefwechsel 1891–1931}. Hg. Konstanze Fliedl. Wien, Zürich: \emph{Europaverlag} 1992, S. 191.} }\toendnotes[C]{\smallbreak}\pstart{}{\pb}\textcolor{gray}{\textbf{Dr. Arthur Schnitzler}}\pend{}\pstart{}\textcolor{gray}{\textbf{\textcolor{pink}{Wien XVIII. Spoettelgasse 7}{}\ledrightnote{\textcolor{pink}{Edmund-Weiß-Gasse}}.}}\pend{}{\bigskip}\pstart{}{\pb}\textsc{Dr. Richard Beer Hofmann}\pend{}\pstart{}\textcolor{pink}{\textsc{Wien XVIII}}{}\ledrightnote{\textcolor{pink}{XVIII., Währing}}\pend{}\pstart{}\textcolor{pink}{\textsc{Hasenauerstr} 59}{}\ledrightnote{\textcolor{pink}{Hasenauerstraße}}\pend{}{\bigskip}\pstart
           \noindent{}{\pb}\textcolor{gray}{\textbf{Dr. Arthur Schnitzler}}{\\}\textcolor{gray}{\textbf{\textcolor{pink}{Wien XVIII. Spoettelgasse 7}{}\ledrightnote{\textcolor{pink}{Edmund-Weiß-Gasse}}.}}\pend
           \pstart
           lieber Richard, hier der \textcolor{green}{\textsc{Tantris}}{}\ledrightnote{\textcolor{green}{Tantris der Narr. Drama in fünf Aufzügen}}. Bringen Sie ihn bitte \label{K_L01806-1v}\edtext{morgen}{\lemma{\textnormal{\emph{morgen}}}\Cendnote{\textnormal{Das deutet darauf, dass das
                  Korrespondenzstück zwei Tage vor dem Datumsvermerk von \textcolor{blue}{Beer-Hofmann} anzusiedeln ist, da am 18. 11. 1908 die Generalprobe von \emph{\textcolor{green}{Tantris}} stattfand. Als weiteres Indiz antwortet
                  die Korrespondenzkarte auf ein mündliches Gespräch vom selben Tag.}}}\label{K_L01806-1h} gleich
               mit, auf dſs er eventuell {\pb}zur Hand wäre.\pend
           \pstart
           Mir fiel noch als \label{K_L01806_1v}\edtext{Ma{\geminationn} der Wiſſenſchaft}{\lemma{\textnormal{\emph{Ma der Wiſſenſchaft}}}\Cendnote{\textnormal{\textcolor{blue}{Beer-Hofmann} sammelte Unterstützer für einen
                  Aufruf für ein jüdisches Studentenheim.}}}\label{K_L01806_1h} Hofrat Prof \textcolor{blue}{\textsc{Oser}}{}\ledrightnote{\textcolor{blue}{Leopold Oser}} ein; als Großinduſtrieller \textcolor{blue}{\textsc{Gutma{\geminationn} v Gelse}}{}\ledrightnote{\textcolor{blue}{Edmund von Gutmann-Gelse}}!\pend
           \pstart Herzlichſt Ihr \spacefill\mbox{A.}\pend{}\endnumbering\briefempfaengerindex{Beer-Hofmann, Richard@\textsc{Beer-Hofmann, Richard}!zzzSchnitzler, Arthur@\emph{von Arthur Schnitzler}!1908-11-172@{{[}17.? 11. 1908{]}}|)be}\mylabel{h}  \normalsize

\doendnotes{C}
\bigskip
\vfill

\clearpage

\footnotesize

\lohead{\textsc{register}}

% Definiere theindex-Environment komplett neu ohne reledmac
\makeatletter
\renewenvironment{theindex}{%
  \section*{\indexname}%
  \setlength{\parindent}{0pt}%
  \setlength{\parskip}{0pt plus 0.3pt}%
  \let\item\@idxitem
}{%
  \clearpage
}
\makeatother

\IfFileExists{\jobname-pw.ind}{\input{\jobname-pw.ind}}{}

\end{document}

      