%% latex-korrekturansicht-vorspann.tex
%% Vorspann für die Korrekturansicht.
%% Lädt die gemeinsame Datei latex-vorspann.tex mit gesetztem Schalter.

\newif\ifkorrekturansicht
\korrekturansichttrue

\input{../tex-inputs/latex-vorspann}


               \section[Hugo von Hofmannsthal an Arthur Schnitzler, 3. 11. {[}1903{]}]{ Hugo von Hofmannsthal an Arthur Schnitzler, 3. 11. {[}1903{]}}\nopagebreak\mylabel{v}\rehead{ }\normalsize\beginnumbering\briefempfaengerindex{Schnitzler, Arthur@\textsc{Schnitzler, Arthur}!zzzHofmannsthal, Hugo von@\emph{von Hugo von Hofmannsthal}!1903-11-031@{3. 11. {[}1903{]}}|(be} \toendnotes[C]{\smallbreak\pagebreak[2]} \Standort{CUL, Schnitzler, B 43.}
\physDesc{Brief, 1 Blatt, 3 Seiten
\newline{}Handschrift: schwarze Tinte, deutsche Kurrent
\newline{}Schnitzler: mit Bleistift die Jahreszahl ergänzt: »903« \newline{}Ordnung: 1) mit Bleistift von unbekannter Hand nummeriert: »\strikeout{211}« 2) mit Bleistift von unbekannter Hand nummeriert: »204«}\buchAbdrucke{\weitereDrucke{Hugo von Hofmannsthal, Arthur Schnitzler: \emph{Briefwechsel}. Hg. Therese Nickl und Heinrich Schnitzler. Frankfurt am Main: \emph{S. Fischer} 1964, S. 175–176.} }\toendnotes[C]{\smallbreak}\pstart
           \raggedleft{}{\pb}3 XI.\pend
           \pstart{}lieber, \pend\pstart
           \textcolor{blue}{Hauptmann}{}\ledrightnote{\textcolor{blue}{Gerhart Hauptmann}}, \textcolor{blue}{Brahm}{}\ledrightnote{\textcolor{blue}{Otto Brahm}}, \textcolor{blue}{Harden}{}\ledrightnote{\textcolor{blue}{Maximilian Harden}} laſſen Sie herzlich
                  grüßen. Mittlerer bittet dringend, ihn \uuline{unverweilt} zu verſtändigen, wie bald er Ihr \textcolor{green}{Stück}{}\ledrightnote{→\textcolor{green}{Der einsame Weg. Schauspiel in fünf Akten}} erwarten darf. Er hat große
                  \textsc{chancen}, es \uline{baldigſt} zu
               ſpielen.\pend
           \pstart
           Aber Vorleſen! Bitten leſen Sie \textcolor{green}{es}{}\ledrightnote{→\textcolor{green}{Der einsame Weg. Schauspiel in fünf Akten}} vor. Das ſind ſo gemüthliche Abende. Bei {\pb}Ihnen, bei \textcolor{blue}{Richard}{}\ledrightnote{\textcolor{blue}{Richard Beer-Hofmann}}, wo immer. Hoffentlich bald.\pend
           \pstart
           Von Herzen{\\[\baselineskip]}\spacefill\mbox{Hugo}\pend
           \leftskip=0em{}\pstart
           \noindent{}P. S. \textcolor{blue}{Gerty}{}\ledrightnote{\textcolor{blue}{Gertrude von Hofmannsthal}} und das neue \textcolor{blue}{baby}{}\ledrightnote{→\textcolor{blue}{Franz von Hofmannsthal}} ſind wohl, \textcolor{green}{Elektra}{}\ledrightnote{\textcolor{green}{Elektra. Tragödie in einem Aufzug}} in \textcolor{pink}{Berlin}{}\ledrightnote{\textcolor{pink}{Berlin}} desgleichen. Die
                  Bekannten des Bearbeiters haben dort vorläufig für 7 oder 8 Vorſtellungen alle
                  Plätze vorgemerkt. Es iſt doch ein Glück, \substVorne{}\textsuperscript{wenn}\substDazwischen{}daſs\substHinten{} man ſo viele {\pb}Bekannte
                  hat und daſs Dr. \textcolor{blue}{Goldmann}{}\ledrightnote{\textcolor{blue}{Paul Goldmann}} nicht zu ihnen
                  gehört.\pend
           \endnumbering\briefempfaengerindex{Schnitzler, Arthur@\textsc{Schnitzler, Arthur}!zzzHofmannsthal, Hugo von@\emph{von Hugo von Hofmannsthal}!1903-11-031@{3. 11. {[}1903{]}}|)be}\mylabel{h}  \normalsize

\doendnotes{C}
\bigskip
\vfill

\clearpage

\footnotesize

\lohead{\textsc{register}}

% Definiere theindex-Environment komplett neu ohne reledmac
\makeatletter
\renewenvironment{theindex}{%
  \section*{\indexname}%
  \setlength{\parindent}{0pt}%
  \setlength{\parskip}{0pt plus 0.3pt}%
  \let\item\@idxitem
}{%
  \clearpage
}
\makeatother

\IfFileExists{\jobname-pw.ind}{\input{\jobname-pw.ind}}{}

\end{document}

      