%% latex-korrekturansicht-vorspann.tex
%% Vorspann für die Korrekturansicht.
%% Lädt die gemeinsame Datei latex-vorspann.tex mit gesetztem Schalter.

\newif\ifkorrekturansicht
\korrekturansichttrue

\input{../tex-inputs/latex-vorspann}


               \section[Gabriel Beer-Hofmann an Arthur Schnitzler, 12. 10. 1926]{ Gabriel Beer-Hofmann an Arthur Schnitzler,
                    12. 10. 1926}\nopagebreak\mylabel{v}\rehead{ }\normalsize\beginnumbering\briefempfaengerindex{Schnitzler, Arthur@\textsc{Schnitzler, Arthur}!zzzBeer-Hofmann, Gabriel@\emph{von Gabriel Beer-Hofmann}!1926-10-121@{12. 10. 1926}|(be} \toendnotes[C]{\smallbreak\pagebreak[2]} \Standort{CUL, Schnitzler, B 8.}
\physDesc{Brief, 1 Blatt, 2 Seiten
\newline{}Handschrift: blaue Tinte, lateinische Kurrent
\newline{}Schnitzler: mit Bleistift beschriftet: »Bab BH« \newline{}Ordnung: mit Bleistift von unbekannter Hand nummeriert:
                                        »272« }\buchAbdrucke{\weitereDrucke{Arthur Schnitzler, Richard Beer-Hofmann: \emph{Briefwechsel 1891–1931}. Hg. Konstanze Fliedl. Wien, Zürich: \emph{Europaverlag} 1992, S. 229.} }\pstart
           \noindent{}\textcolor{gray}{\textbf{{\pb}Am Ausgang des
                                    \textcolor{pink}{Hauptbahnhof}{}\ledrightnote{\textcolor{pink}{Hauptbahnhof}}es}}\hfill \textcolor{gray}{\textbf{\textcolor{pink}{Kirchenallee Nr. 35–36}{}\ledrightnote{\textcolor{pink}{Kirchenallee}},
                                gegenüber}}\pend
           \pstart
           {\dotssix}Ankunftsseite{\dotssix}\hfill {\dots}Ausgang Hauptbahnhof{\dots}\pend
           \pstart
           \centering{}\textcolor{gray}{\textbf{\textcolor{pink}{Hotel Reichshof}{}\ledrightnote{\textcolor{pink}{Hotel Reichshof}} Hamburg}}\pend
           \pstart
           \noindent{}\centering{}\textcolor{gray}{\textbf{Direktion: \textcolor{blue}{Emil
                                Langer}{}\ledrightnote{\textcolor{blue}{Anton-Emil Langer}}}}\pend
           \pstart
           \noindent{}\centering{}\textcolor{gray}{\textbf{Mehr als 300 Zimmer und Salons}}\pend
           \pstart
           \noindent{}\centering{}\textcolor{gray}{\textbf{50 Badezimmer}}\pend
           \pstart
           \noindent{}\textcolor{gray}{\textbf{Telegramm-Adresse:}}\hfill \textcolor{gray}{\textbf{Fernsprecher:}}\pend
           \pstart
           \textcolor{gray}{\textbf{Reichshof Hamburg}}\hfill \textcolor{gray}{\textbf{Alster 870, 2836, 2837}}\pend
           \pstart
           \centering{}\textcolor{gray}{\textbf{Im Frühstücks-Saal: Grosses und Abendessen nach der
                            Karte}}\pend
           \pstart
           \noindent{}\centering{}\textcolor{gray}{\textbf{Kachel-Waschtische mit fliessendem kalten und warmen
                            Wasser in allen Zimmern}}\pend
           \pstart
           \noindent{}\textcolor{gray}{\textbf{Fernsprecher in allen Zimmern}}\pend
           \pstart
           \textcolor{gray}{\textbf{Auto-Unterstand für 20 Automobile}}\pend
           \pstart
           \textcolor{gray}{\textbf{Rasier- und Frisier-Salon im Hause}}\pend
           \pstart
           \raggedleft{}\textcolor{gray}{\textbf{\textcolor{pink}{Hamburg}{}\ledrightnote{\textcolor{pink}{Hamburg}}, den}}{ }12. Oktober \textcolor{gray}{\textbf{192}}6\pend
           \pstart
           \raggedleft{}\textcolor{gray}{\textbf{\textcolor{pink}{Kirchenallee Nr. 35–36}{}\ledrightnote{\textcolor{pink}{Kirchenallee}}}}\pend
           \pstart{}Verehrter, lieber Doktor Schnitzler!\pend\pstart
           Wie sehr es mir Wunsch und Bedürfniss gewesen wäre, mich von Ihnen zu
                    verabschieden, so war es mir doch schliesslich zeitlich unmöglich. Trotz aller
                    Vorbereitungen war meine Abreise doch überstürzt. –\pend
           \pstart
           Ich hätte Sie, lieber Herr Doktor, wie auch ganz besonders gerne \textcolor{blue}{Lily}{}\ledrightnote{\textcolor{blue}{Lili Schnitzler}} noch einmal gesehen. –\pend
           \pstart
           Nach ein paar Tagen \textcolor{pink}{Berlin}{}\ledrightnote{\textcolor{pink}{Berlin}} und drei kalten und
                    verregneten Tagen in \textcolor{pink}{Hamburg}{}\ledrightnote{\textcolor{pink}{Hamburg}}, fahre ich morgen
                    mit der »Thuringia« nach \textcolor{pink}{New-York}{}\ledrightnote{\textcolor{pink}{New York City}}.\pend
           \pstart
           Zwölf Tage Seefahrt – wie sehr habe ich mir dies – seit Jahren – gewünscht und
                    jetzt wird es Erfüllung – wie ein Traum zauberhaft und unglaublich –\pend
           \pstart
           Ich habe leider nicht die Adresse (\textcolor{pink}{Venedig}{}\ledrightnote{\textcolor{pink}{Venedig}}) von
                        \textcolor{blue}{Lily}{}\ledrightnote{\textcolor{blue}{Lili Schnitzler}}.\pend
           \pstart
           Es ist doch nicht unbescheiden, wenn ich Sie, lieber Herr Doktor {\pb}bitte, \textcolor{blue}{Lily}{}\ledrightnote{\textcolor{blue}{Lili Schnitzler}} sehr schön und herzlich von mir zu grüssen. Ich will
                    ihr gleich von drüben schreiben.\pend
           \pstart
           Inzwischen, Ihnen, lieber Doktor Schnitzler und der lieben \textcolor{blue}{Lily}{}\ledrightnote{\textcolor{blue}{Lili Schnitzler}}, alle guten Wünsche für die nächste Zeit\pend
           \pstart
           von ganzem Herzen{\\[\baselineskip]}Ihr{\\[\baselineskip]}\spacefill\mbox{Gabriel Beer-Hofmann}\pend
           \leftskip=0em{}\endnumbering\briefempfaengerindex{Schnitzler, Arthur@\textsc{Schnitzler, Arthur}!zzzBeer-Hofmann, Gabriel@\emph{von Gabriel Beer-Hofmann}!1926-10-121@{12. 10. 1926}|)be}\mylabel{h}  \normalsize

\doendnotes{C}
\bigskip
\vfill

\clearpage

\footnotesize

\lohead{\textsc{register}}

% Definiere theindex-Environment komplett neu ohne reledmac
\makeatletter
\renewenvironment{theindex}{%
  \section*{\indexname}%
  \setlength{\parindent}{0pt}%
  \setlength{\parskip}{0pt plus 0.3pt}%
  \let\item\@idxitem
}{%
  \clearpage
}
\makeatother

\IfFileExists{\jobname-pw.ind}{\input{\jobname-pw.ind}}{}

\end{document}

      