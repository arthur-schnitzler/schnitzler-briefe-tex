%% latex-korrekturansicht-vorspann.tex
%% Vorspann für die Korrekturansicht.
%% Lädt die gemeinsame Datei latex-vorspann.tex mit gesetztem Schalter.

\newif\ifkorrekturansicht
\korrekturansichttrue

\input{../tex-inputs/latex-vorspann}


               \section[Hugo von Hofmannsthal an Arthur Schnitzler, 11. 9. {[}1914{]}]{ Hugo von Hofmannsthal an Arthur Schnitzler, 11. 9. {[}1914{]}}\nopagebreak\mylabel{v}\rehead{ }\normalsize\beginnumbering\briefempfaengerindex{Schnitzler, Arthur@\textsc{Schnitzler, Arthur}!zzzHofmannsthal, Hugo von@\emph{von Hugo von Hofmannsthal}!1914-09-111@{11. 9. {[}1914{]}}|(be} \toendnotes[C]{\smallbreak\pagebreak[2]} \Standort{CUL, Schnitzler, B 43.}
\physDesc{Brief, 1 Blatt, 2 Seiten
\newline{}Handschrift: schwarze Tinte, deutsche Kurrent
\newline{}Schnitzler: 1) mit Bleistift beschriftet: »Hugo« 2) mit rotem Buntstift eine Unterstreichung\newline{}Ordnung: 1) mit Bleistift von unbekannter Hand nummeriert: »\strikeout{336}« 2) mit Bleistift von unbekannter Hand nummeriert:
                                    »351«}\buchAbdrucke{\weitereDrucke{Hugo von Hofmannsthal, Arthur Schnitzler: \emph{Briefwechsel}. Hg. Therese Nickl und Heinrich Schnitzler. Frankfurt am Main: \emph{S. Fischer} 1964, S. 276.} }\toendnotes[C]{\smallbreak}\pstart
           \raggedleft{}{\pb}\textcolor{pink}{Auſſee}{}\ledrightnote{\textcolor{pink}{Bad Aussee}}{ }11 IX.\pend
           \pstart{}lieber Arthur\pend\pstart
           ich bin für 2–3 Tage hier, dann wieder \textcolor{pink}{Eliſabethſtraße}{}\ledrightnote{\textcolor{pink}{Elisabethstraße}}.\hspace*{1.5em}Ich weiß daſs Sie ſchon
               größere Beträge fürs \textcolor{brown}{rote Kreuz}{}\ledrightnote{\textcolor{brown}{Internationales Komitee vom Roten Kreuz}} gegeben haben, aber
                  \uline{bitte} geben Sie nun noch etwas und das ſogleich
               für die \textcolor{brown}{Rettungsgeſellſchaft}{}\ledrightnote{\textcolor{brown}{Wiener freiwillige Rettungsgesellschaft}}, die vorzügliches
               leiſtet und dringend Hilfe braucht und bitte geben Sie es \label{K_L02196_1v}\edtext{durch die \textcolor{brown}{\textsc{Neue Freie Presse}}{}\ledrightnote{\textcolor{brown}{Neue Freie Presse}}}{\lemma{\textnormal{\emph{durch … Presse}}}\Cendnote{\textnormal{Am 10. 9. 1914 erschien ein
                     »Erster Spendenausweis« der Sammlung, die 819 Kronen nachwies,
                  wobei jeweils 200 von \textcolor{blue}{Hofmannsthal} und seinem
                     \textcolor{blue}{Vater} stammten (\emph{\textcolor{brown}{Neue Freie Presse}}, Nr. 17976, S. 7). In
                  den Folgetagen wurden weitere Spenden ausgwiesen, aber keine von \textcolor{blue}{Schnitzler}.}}}\label{K_L02196_1h}, das zieht wieder andere {\pb}Leute mit, deshalb gab ich auch
               dort, gab nur einen kleinem Beitrag \introOben{}(200)\introOben{}, um mehrmals
               wieder geben zu können, es wird noch allſeits viel zu wenig gegeben, es iſt ein Meer
               von Not und Schwierigkeiten.\pend
           \pstart
           Ich bitte Sie und \textcolor{blue}{Olga}{}\ledrightnote{\textcolor{blue}{Olga Schnitzler}}, dies unter Euren Bekannten
                  \label{K_L02196_2v}\edtext{weiterzuſagen}{\lemma{\textnormal{\emph{weiterzuſagen}}}\Cendnote{\textnormal{Am 19. 9. 1914 wird eine
                  Spende von 300 Kronen durch \textcolor{blue}{Paula Beer-Hofmann}
                  ausgewiesen (\emph{\textcolor{brown}{Neue Freie Presse}}, Nr. 17985,
                  S. 5).}}}\label{K_L02196_2h}, es iſt eine der dringendſten Notwendigkeiten.\pend
           \pstart
           Von Herzen{\\[\baselineskip]}\spacefill\mbox{Hugo.}\pend
           \leftskip=0em{}\endnumbering\briefempfaengerindex{Schnitzler, Arthur@\textsc{Schnitzler, Arthur}!zzzHofmannsthal, Hugo von@\emph{von Hugo von Hofmannsthal}!1914-09-111@{11. 9. {[}1914{]}}|)be}\mylabel{h}  \normalsize

\doendnotes{C}
\bigskip
\vfill

\clearpage

\footnotesize

\lohead{\textsc{register}}

% Definiere theindex-Environment komplett neu ohne reledmac
\makeatletter
\renewenvironment{theindex}{%
  \section*{\indexname}%
  \setlength{\parindent}{0pt}%
  \setlength{\parskip}{0pt plus 0.3pt}%
  \let\item\@idxitem
}{%
  \clearpage
}
\makeatother

\IfFileExists{\jobname-pw.ind}{\input{\jobname-pw.ind}}{}

\end{document}

      