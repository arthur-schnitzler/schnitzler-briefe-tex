%% latex-korrekturansicht-vorspann.tex
%% Vorspann für die Korrekturansicht.
%% Lädt die gemeinsame Datei latex-vorspann.tex mit gesetztem Schalter.

\newif\ifkorrekturansicht
\korrekturansichttrue

\input{../tex-inputs/latex-vorspann}


               \section[Richard Beer-Hofmann an Olga Schnitzler, {[}nach dem 25. 8. 1903?{]}]{ Richard Beer-Hofmann an Olga Schnitzler, {[}nach dem
               25. 8. 1903?{]}}\nopagebreak\mylabel{v}\rehead{ }\normalsize\beginnumbering\briefempfaengerindex{Schnitzler, Olga@\textsc{Schnitzler, Olga}!zzzBeer-Hofmann, Richard@\emph{von Richard Beer-Hofmann}!1903-08-261@{{[}nach dem
                  25. 8. 1903?{]}}|(be} \toendnotes[C]{\smallbreak\pagebreak[2]} \Standort{CUL, Schnitzler, B 8.}
\physDesc{Kartenbrief, 1 Blatt, 2 Seiten
\newline{}Handschrift: schwarze Tinte, lateinische Kurrent\newline{}Versand: ohne postalischen Übermittlungsvermerk \newline{}Ordnung: 1) von Schnitzler mit Bleistift beschriftet: »\textsc{Beerhofma{\geminationn}}« 2) mit Bleistift von unbekannter Hand nummeriert:
                              »278b«}\toendnotes[C]{\smallbreak}\pstart{}{\pb}S. H.\pend{}\pstart{}Frau\pend{}\pstart{}\textcolor{blue}{Olga Schnitzler}{}\ledrightnote{\textcolor{blue}{Olga Schnitzler}}\pend{}{\bigskip}\pstart
           \noindent{}{\pb}{[}Zeichnung, wie eine
                  Braut (Olga Schnitzler) vom Bräutigam (\textcolor{blue}{Arthur
                     Schnitzler}{}\ledrightnote{}) in die{]}{ }\label{KLL01316_OS-1v}\edtext{\textcolor{pink}{Spöttelgasse 7}{}\ledrightnote{\textcolor{pink}{Edmund-Weiß-Gasse}}}{\lemma{\textnormal{\emph{Spöttelgasse 7}}}\Cendnote{\textnormal{Die Zeichnung ist undatiert, dürfte
                  aber in zeitlicher Nähe zur Hochzeit und dem darauffolgenden Einzug in die neue
                  Wohnung entstanden sein.}}}\label{KLL01316_OS-1h}{ }{[}geführt
                  wird{]}\pend
           \endnumbering\briefempfaengerindex{Schnitzler, Olga@\textsc{Schnitzler, Olga}!zzzBeer-Hofmann, Richard@\emph{von Richard Beer-Hofmann}!1903-08-261@{{[}nach dem
                  25. 8. 1903?{]}}|)be}\mylabel{h}  \normalsize

\doendnotes{C}
\bigskip
\vfill

\clearpage

\footnotesize

\lohead{\textsc{register}}

% Definiere theindex-Environment komplett neu ohne reledmac
\makeatletter
\renewenvironment{theindex}{%
  \section*{\indexname}%
  \setlength{\parindent}{0pt}%
  \setlength{\parskip}{0pt plus 0.3pt}%
  \let\item\@idxitem
}{%
  \clearpage
}
\makeatother

\IfFileExists{\jobname-pw.ind}{\input{\jobname-pw.ind}}{}

\end{document}

      