%% latex-korrekturansicht-vorspann.tex
%% Vorspann für die Korrekturansicht.
%% Lädt die gemeinsame Datei latex-vorspann.tex mit gesetztem Schalter.

\newif\ifkorrekturansicht
\korrekturansichttrue

\input{../tex-inputs/latex-vorspann}


               \section[Thomas Mann an Arthur Schnitzler, {[}5. 12. 1930{]}]{ Thomas Mann an Arthur Schnitzler, {[}5. 12. 1930{]}}\nopagebreak\mylabel{v}\rehead{ }\normalsize\beginnumbering\briefempfaengerindex{Schnitzler, Arthur@\textsc{Schnitzler, Arthur}!zzzMann, Thomas@\emph{von Thomas Mann}!1930-12-051@{{[}5. 12. 1930{]}}|(be} \toendnotes[C]{\smallbreak\pagebreak[2]} \Standort{CUL, Schnitzler, B 67.}
\physDesc{Visitenkarte
\newline{}Handschrift: schwarze Tinte, deutsche Kurrent
\newline{}Schnitzler: mit rotem Buntstift datiert: »5/12 930« }\buchAbdrucke{\weitereDrucke{Hertha Krotkoff: \emph{Arthur Schnitzler – Thomas Mann: Briefe.} In: \emph{Modern Austrian Literature}, Jg. 7 (1974) Nr. 1/2, S. 27.} }\pstart
           \noindent{}\centering{}{\pb}\textcolor{gray}{\textbf{\emph{Dr. Thomas Mann}}}\pend
           \pstart
           \noindent{}mit herzlichen Grüßen! Ich möchte den kindiſchen Gang der Umſtändlichkeiten ein
                    wenig abkürzen.\pend
           \pstart
           \textcolor{gray}{\textbf{\textcolor{pink}{\emph{München}}{}\ledrightnote{\textcolor{pink}{München}}}}\hfill \textcolor{gray}{\textbf{\textcolor{pink}{\emph{Poschingerstr. 1}}{}\ledrightnote{\textcolor{pink}{Poschingerstraße}}}}\pend
           \endnumbering\briefempfaengerindex{Schnitzler, Arthur@\textsc{Schnitzler, Arthur}!zzzMann, Thomas@\emph{von Thomas Mann}!1930-12-051@{{[}5. 12. 1930{]}}|)be}\mylabel{h}  \normalsize

\doendnotes{C}
\bigskip
\vfill

\clearpage

\footnotesize

\lohead{\textsc{register}}

% Definiere theindex-Environment komplett neu ohne reledmac
\makeatletter
\renewenvironment{theindex}{%
  \section*{\indexname}%
  \setlength{\parindent}{0pt}%
  \setlength{\parskip}{0pt plus 0.3pt}%
  \let\item\@idxitem
}{%
  \clearpage
}
\makeatother

\IfFileExists{\jobname-pw.ind}{\input{\jobname-pw.ind}}{}

\end{document}

      