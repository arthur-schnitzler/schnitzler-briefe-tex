%% latex-korrekturansicht-vorspann.tex
%% Vorspann für die Korrekturansicht.
%% Lädt die gemeinsame Datei latex-vorspann.tex mit gesetztem Schalter.

\newif\ifkorrekturansicht
\korrekturansichttrue

\input{../tex-inputs/latex-vorspann}


               \section[Thomas Mann an Arthur Schnitzler, 22. 11. 1923]{ Thomas Mann an Arthur Schnitzler, 22. 11. 1923}\nopagebreak\mylabel{v}\rehead{ }\normalsize\beginnumbering\briefempfaengerindex{Schnitzler, Arthur@\textsc{Schnitzler, Arthur}!zzzMann, Thomas@\emph{von Thomas Mann}!1923-11-221@{22. 11. 1923}|(be} \toendnotes[C]{\smallbreak\pagebreak[2]} \Standort{Düsseldorf, Heinrich-Heine-Institut, HHI.94.5036.397.}
\physDesc{Briefkarte
\newline{}Handschrift: schwarze Tinte, deutsche Kurrent
\newline{}Schnitzler: mit rotem Buntstift eine Unterstreichung }\toendnotes[C]{\smallbreak}\pstart
           \noindent{}{\pb}\textcolor{gray}{\textbf{\textsc{Thomas Mann}}}\hfill \textcolor{gray}{\textbf{\textcolor{pink}{MÜNCHEN}{}\ledrightnote{\textcolor{pink}{München}}, den}}{ }22. XI. 23.\pend
           \pstart
           \raggedleft{}\textcolor{gray}{\textbf{\textcolor{pink}{POSCHINGERSTR. 1}{}\ledrightnote{\textcolor{pink}{Poschingerstraße}}}}\pend
           \pstart{}Lieber, verehrter Herr Dr. Schnitzler,\pend\pstart
           ich bin wahrhaft gerührt durch Ihr gütiges Eingehen auf den »\textcolor{green}{Krull}{}\ledrightnote{\textcolor{green}{Bekenntnisse des Hochstaplers Felix Krull}}« und danke Ihnen herzlich. Ich weiß nicht, warum ich
                    damals ſtecken blieb, – vielleicht, weil der extrem individualiſtiſche und
                    unſoziale Charakter des Buches mir nicht zeitgemäß ſchien, vielleicht auch, weil
                    ich das Gefühl hatte, in dieſem erſten Teil alles We{\pb}ſentliche
                    eigentlich ſchon gegeben zu haben. Immerhin habe ich den Plan nie ganz aus den
                    Augen verloren, und wenn ich abgewälzt habe, woran ich jetzt ſchleppe, findet
                    ſich wohl einmal die Laune, das abſonderliche Ding zu beenden.\pend
           \pstart
           Ich freue mich auf \textcolor{pink}{Wien}{}\ledrightnote{\textcolor{pink}{Wien}}, wohin ich – diesmal wohl
                    mit meiner \textcolor{blue}{Frau}{}\ledrightnote{→\textcolor{blue}{Katia Mann}}, die Ihnen
                    herzlich verehrungsvolle Grüße ſendet – Ende des Winters, im März
                    etwa, zu kommen hoffe, freue mich auf die Freunde dort und vor Allem auf
                    Sie.\pend
           \pstart
           Ihr ergebener{\\[\baselineskip]}\spacefill\mbox{Thomas Mann.}\pend
           \leftskip=0em{}\endnumbering\briefempfaengerindex{Schnitzler, Arthur@\textsc{Schnitzler, Arthur}!zzzMann, Thomas@\emph{von Thomas Mann}!1923-11-221@{22. 11. 1923}|)be}\mylabel{h}  \normalsize

\doendnotes{C}
\bigskip
\vfill

\clearpage

\footnotesize

\lohead{\textsc{register}}

% Definiere theindex-Environment komplett neu ohne reledmac
\makeatletter
\renewenvironment{theindex}{%
  \section*{\indexname}%
  \setlength{\parindent}{0pt}%
  \setlength{\parskip}{0pt plus 0.3pt}%
  \let\item\@idxitem
}{%
  \clearpage
}
\makeatother

\IfFileExists{\jobname-pw.ind}{\input{\jobname-pw.ind}}{}

\end{document}

      