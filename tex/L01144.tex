%% latex-korrekturansicht-vorspann.tex
%% Vorspann für die Korrekturansicht.
%% Lädt die gemeinsame Datei latex-vorspann.tex mit gesetztem Schalter.

\newif\ifkorrekturansicht
\korrekturansichttrue

\input{../tex-inputs/latex-vorspann}


               \section[Richard Beer-Hofmann an Arthur Schnitzler, 10. 7. 1901]{ Richard Beer-Hofmann an Arthur Schnitzler,
               10. 7. 1901}\nopagebreak\mylabel{v}\rehead{ }\normalsize\beginnumbering\briefempfaengerindex{Schnitzler, Arthur@\textsc{Schnitzler, Arthur}!zzzBeer-Hofmann, Richard@\emph{von Richard Beer-Hofmann}!1901-07-101@{10. 7. 1901}|(be} \toendnotes[C]{\smallbreak\pagebreak[2]} \Standort{CUL, Schnitzler, B 8.}
\physDesc{Brief, 1 Blatt, 2 Seiten
\newline{}Handschrift: blauer Buntstift, lateinische Kurrent\newline{}Ordnung: mit Bleistift von unbekannter Hand nummeriert: »164« }\buchAbdrucke{\weitereDrucke{Arthur Schnitzler, Richard Beer-Hofmann: \emph{Briefwechsel 1891–1931}. Hg. Konstanze Fliedl. Wien, Zürich: \emph{Europaverlag} 1992, S. 153.} }\toendnotes[C]{\smallbreak}\pstart
           \raggedleft{}{\pb}\textcolor{pink}{Pörtschach}{}\ledrightnote{\textcolor{pink}{Pörtschach}}{ }10/VII. 1901\pend
           \pstart
           Lieber Arthur! Wir waren am 1, 2,
                  3, in \textcolor{pink}{Wien}{}\ledrightnote{\textcolor{pink}{Wien}}; seit 4. sind
               wir wieder hier mit \textcolor{blue}{Papa Hermann}{}\ledrightnote{\textcolor{blue}{Hermann Beer}}, dessen \textcolor{blue}{Frau}{}\ledrightnote{→\textcolor{blue}{Rosa Beer}} am 1.
               gestorben ist. Da der \textcolor{blue}{Papa}{}\ledrightnote{→\textcolor{blue}{Hermann Beer}} auch
               physisch sehr hergeno{\geminationm}en ist, haben wir vorläufig mit
               ihm zu tun. Obgleich er nicht lange hierbleiben will, weiß ich doch nicht ob ich
               gegen Mitte oder Ende August Sie irgendwo werde treffen
               können.\pend
           \pstart
           {\pb}Auch nicht ob ich Lust haben werde
               irgendwohin zu reisen, da ich endlich arbeiten möchte. Ich freue mich sehr daß Sie
               sich wol fühlen. Hoffentlich nimmt »man« Ihnen Ihre \textcolor{blue}{Grillparzer}{}\ledrightnote{\textcolor{blue}{Franz Grillparzer}}grämlichkeit der letzten Zeit. Schreiben Sie mir bald und
               viel.\pend
           \pstart
           Von Herzen Ihr{\\[\baselineskip]}\spacefill\mbox{Richard}\pend
           \leftskip=0em{}\endnumbering\briefempfaengerindex{Schnitzler, Arthur@\textsc{Schnitzler, Arthur}!zzzBeer-Hofmann, Richard@\emph{von Richard Beer-Hofmann}!1901-07-101@{10. 7. 1901}|)be}\mylabel{h}  \normalsize

\doendnotes{C}
\bigskip
\vfill

\clearpage

\footnotesize

\lohead{\textsc{register}}

% Definiere theindex-Environment komplett neu ohne reledmac
\makeatletter
\renewenvironment{theindex}{%
  \section*{\indexname}%
  \setlength{\parindent}{0pt}%
  \setlength{\parskip}{0pt plus 0.3pt}%
  \let\item\@idxitem
}{%
  \clearpage
}
\makeatother

\IfFileExists{\jobname-pw.ind}{\input{\jobname-pw.ind}}{}

\end{document}

      