%% latex-korrekturansicht-vorspann.tex
%% Vorspann für die Korrekturansicht.
%% Lädt die gemeinsame Datei latex-vorspann.tex mit gesetztem Schalter.

\newif\ifkorrekturansicht
\korrekturansichttrue

\input{../tex-inputs/latex-vorspann}


               \section[Arthur Schnitzler an Georg Brandes, 25. 4. 1901]{ Arthur Schnitzler an Georg Brandes, 25. 4. 1901}\nopagebreak\mylabel{v}\rehead{ }\normalsize\beginnumbering\briefempfaengerindex{Brandes, Georg@\textsc{Brandes, Georg}!zzzSchnitzler, Arthur@\emph{von Arthur Schnitzler}!1901-04-251@{25. 4. 1901}|(be} \toendnotes[C]{\smallbreak\pagebreak[2]} \Standort{Kopenhagen, Det Kongelige Bibliotek, Georg Brandes Arkiv, box 125.}
\physDesc{Brief, 2 Blätter, 8 Seiten
\newline{}Handschrift: schwarze Tinte, deutsche Kurrent\newline{}Ordnung: mit Bleistift von unbekannter Hand nummeriert: »21. \textsc{Schnitzler}«, die Datierung auf der ersten Seite des zweiten
                                    Blattes mit Bleistift wiederholt }\buchAbdrucke{\weitereDrucke{Georg Brandes, Arthur Schnitzler: \emph{Ein Briefwechsel}. Hg. Kurt Bergel. Bern: \emph{Francke} 1956, S. 83–84.} }\toendnotes[C]{\smallbreak}\pstart
           \raggedleft{}{\pb}\textcolor{pink}{Wien}{}\ledrightnote{\textcolor{pink}{Wien}}, 25. 4. 901.\pend
           \pstart{}Lieber Herr Brandes,\pend\pstart
           \textcolor{blue}{\textsc{Paul Goldmann}}{}\ledrightnote{\textcolor{blue}{Paul Goldmann}} hat mir \textcolor{green}{\textsc{Politiken}}{}\ledrightnote{\textcolor{green}{Politiken}} mit Ihrem \label{K_L01114_1v}\edtext{\textcolor{green}{Artikel}{}\ledrightnote{→\textcolor{green}{Arthur Schnitzler [dänisch]}}}{\lemma{\textnormal{\emph{Artikel}}}\Cendnote{\textnormal{Es dürfte sich um einen Fehler \textcolor{blue}{Schnitzler}s handeln. Zumindest findet
                        sich der Text in seinen Zeitungsausschnitten (Exeter, box 37/2)
                        mit dem Titelzusatz »För Handelstidning« als Ausschnitt aus
                        \emph{\textcolor{brown}{Göteborgs Handels- och Sjöfartstidning}}
                        vom 9. 4. 1901.}}}\label{K_L01114_1h} über mich geſandt und ich verſuchte \textcolor{pink}{däniſch}{}\ledrightnote{\textcolor{pink}{Dänemark}} zu verſtehen, was mir nur zum Theil
                    gelang; die \textcolor{brown}{Neue Freie Preſſe}{}\ledrightnote{\textcolor{brown}{Neue Freie Presse}} kam mir zu \textcolor{green}{Hilfe}{}\ledrightnote{→\textcolor{green}{Arthur Schnitzler}} – und Sie können ſich
                    denken, wie ſehr ich mich gefreut habe, als ich nun alles, was Sie über mich
                    ſchrieben, we{\geminationn} auch nur in der Überſetzung leſen
                    konnte. Laſſen Sie mich Ihnen die Hand drücken – und {\pb}weiter nichts ſagen – wie es Ihnen ja gewiſs
                    am liebſten iſt.\pend
           \pstart
           Sie haben hoffentlich meine Karte aus \textcolor{pink}{Rom}{}\ledrightnote{\textcolor{pink}{Rom}}
                    bekommen und wiſſen, dſs ich \textcolor{blue}{\textsc{Ellen Key}}{}\ledrightnote{\textcolor{blue}{Ellen Key}} ke{\geminationn}engelernt habe, die mir zu meiner Freude
                    erzählte, dſs Sie den letzten Winter in vollkommener Geſundheit verbracht haben.
                    Wenige Tage nachdem ich \textcolor{blue}{\textsc{Ellen Key}}{}\ledrightnote{\textcolor{blue}{Ellen Key}}, deren Weſen mir wahrhaft wohl that, bei \textcolor{blue}{\textsc{Wasserma{\geminationn}s}}{}\ledrightnote{\textcolor{blue}{Jakob Wassermann}{\newline}\textcolor{blue}{Julie Wassermann}} kennen gelernt, traf ich ſie ein zweites Mal und {\pb}\textcolor{blue}{\textsc{Helge Rhode}}{}\ledrightnote{\textcolor{blue}{Helge Rode}}, den ſie mitbrachte. Ich war kaum zwei Wochen in \textcolor{pink}{Rom}{}\ledrightnote{\textcolor{pink}{Rom}}, eben genug, um zu wiſſen, wie man es ein nächſtes Mal
                    anzufangen hat, um ſeine Zeit gut auszunützen. Von \textcolor{pink}{Rom}{}\ledrightnote{\textcolor{pink}{Rom}} ging ich nach \textcolor{pink}{Florenz}{}\ledrightnote{\textcolor{pink}{Florenz}}, wo ich mit
                    meiner \textcolor{blue}{Mama}{}\ledrightnote{→\textcolor{blue}{Louise Schnitzler}} Rendezvous
                    hatte – aber den Frühling fand ich nirgends. Man fror beinah immer.\pend
           \pstart
           Sie waren – oder ſind noch? – in \textcolor{pink}{Berlin}{}\ledrightnote{\textcolor{pink}{Berlin}}, wie mir
                        \textcolor{blue}{Georg Hirſchfeld}{}\ledrightnote{\textcolor{blue}{Georg Hirschfeld}}{ }{\pb}ſchrieb; wann ko{\geminationm}en Sie wieder zu uns? Sie würden nicht viel verändert finden – \textcolor{blue}{\textsc{Beer Hofmann}}{}\ledrightnote{\textcolor{blue}{Richard Beer-Hofmann}} hat nun auch zu ſeinen \textcolor{blue}{Töchtern}{}\ledrightnote{→\textcolor{blue}{Mirjam Beer-Hofmann}{\newline}→\textcolor{blue}{Naëmah Beer-Hofmann}} einen \textcolor{blue}{Sohn}{}\ledrightnote{→\textcolor{blue}{Gabriel Beer-Hofmann}} beko{\geminationm}en, aber von
                    dem iſt begreiflicherweiſe noch nicht viel zu erzählen. Ich werde diesmal
                    wahrſcheinlich ſehr bald ins Gebirge reiſen; und nach mancherlei Kleinigkeiten,
                    die ich in der letzten Zeit gemacht, mich wohl endlich wieder \introOben{}an\introOben{} was größeres {\pb}wagen. Einen
                    kleinen \textcolor{green}{Roman}{}\ledrightnote{→\textcolor{green}{Frau Bertha Garlan. Roman}}, den ich
                        vorigen Winter{ }ſchrieb, haben Sie wohl ſchon erhalten. Die
                        \textcolor{green}{\textsc{Beatrice}}{}\ledrightnote{\textcolor{green}{Der Schleier der Beatrice. Schauspiel in fünf Akten}} iſt im Dezember einige Male in \textcolor{pink}{Breſlau}{}\ledrightnote{\textcolor{pink}{Breslau}} geſpielt worden, ohne beſonderes Glück. Auch war die
                    Darſtellung recht ſchwach. Eine gute Aufführung müßte dem Stück wohl Erfolg
                    bringen. Aber das \textcolor{pink}{Burgtheater}{}\ledrightnote{\textcolor{pink}{Burgtheater}} hat wichtigeres zu
                    thun. –\pend
           \pstart Leben Sie wohl und ſeien Sie herzlich gegrüßt von Ihrem treuen
                        \spacefill\mbox{ArthurSchnitzler}\pend{}\pstart
           \noindent{}{\pb}Dieſer Tage erſcheint eine Novelle von
                        mir, die ich Ihnen natürlich ſchicken werde, \textcolor{green}{Lieutenant Guſtl}{}\ledrightnote{\textcolor{green}{Lieutenant Gustl. Novelle}}, – Sie haben ſie vielleicht in der \textcolor{brown}{N. Fr. Pr.}{}\ledrightnote{\textcolor{brown}{Neue Freie Presse}} geleſen. Wegen dieſer Novelle
                        ſtehe ich – (da ich noch \textsc{Militärarzt} »in der
                        Evidenz« bin) in »ehrengerichtlicher« Unterſuchung und werde wahrſcheinlich
                        meine \textsc{Charge} verlieren. Wenn Sie die Novelle {\pb}noch nicht kennen und ſie leſen werden –
                        und ſich dieſer Mittheilung erinnern – wird Ihnen wieder manches »\textcolor{pink}{oeſterreichiſche}{}\ledrightnote{\textcolor{pink}{Österreich}}« klar werden.\hspace*{1.5em}Die Sache iſt für mich natürlich gleichgiltig
                        – da ich ja mit den Leuten nichts mehr zu thun habe und meine Charge nur im
                        Kriegsfall von Bedeutung wäre – aber ſie iſt charakteriſtiſch für {\pb}die man könnte ſagen naïve Heuchelei in
                        Kreiſen, von denen man in gewiſſem Sinne i{\geminationm}er
                        abhängig iſt; we{\geminationn}{ }ſie auch keine unmittelbare Macht über
                        einen beſitzen.\pend
           \pstart
           Ihr \spacefill\mbox{A. S.}\pend
           \endnumbering\briefempfaengerindex{Brandes, Georg@\textsc{Brandes, Georg}!zzzSchnitzler, Arthur@\emph{von Arthur Schnitzler}!1901-04-251@{25. 4. 1901}|)be}\mylabel{h}  \normalsize

\doendnotes{C}
\bigskip
\vfill

\clearpage

\footnotesize

\lohead{\textsc{register}}

% Definiere theindex-Environment komplett neu ohne reledmac
\makeatletter
\renewenvironment{theindex}{%
  \section*{\indexname}%
  \setlength{\parindent}{0pt}%
  \setlength{\parskip}{0pt plus 0.3pt}%
  \let\item\@idxitem
}{%
  \clearpage
}
\makeatother

\IfFileExists{\jobname-pw.ind}{\input{\jobname-pw.ind}}{}

\end{document}

      