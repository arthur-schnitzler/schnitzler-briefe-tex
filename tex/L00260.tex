%% latex-korrekturansicht-vorspann.tex
%% Vorspann für die Korrekturansicht.
%% Lädt die gemeinsame Datei latex-vorspann.tex mit gesetztem Schalter.

\newif\ifkorrekturansicht
\korrekturansichttrue

\input{../tex-inputs/latex-vorspann}


               \section[Arthur Schnitzler und Felix Salten an Richard Beer-Hofmann, 27. 8. 1893]{ Arthur Schnitzler und Felix Salten an Richard Beer-Hofmann,
                    27. 8. 1893}\nopagebreak\mylabel{v}\rehead{ }\normalsize\beginnumbering\briefempfaengerindex{Beer-Hofmann, Richard@\textsc{Beer-Hofmann, Richard}!zzzSalten, Felix@\emph{von Felix Salten}!1893-08-271@{27. 8. 1893}|(be}\briefempfaengerindex{Beer-Hofmann, Richard@\textsc{Beer-Hofmann, Richard}!zzzSchnitzler, Arthur@\emph{von Arthur Schnitzler}!1893-08-271@{27. 8. 1893}|(be} \toendnotes[C]{\smallbreak\pagebreak[2]} \Standort{YCGL, MSS 31.}
\physDesc{Kartenbrief
\newline{}Handschrift Arthur Schnitzler: Bleistift, deutsche Kurrent\newline{}Handschrift Felix Salten: Bleistift, lateinische Kurrent\newline{}Versand: 1) Stempel: »\nobreak{}\oindex{Poertschach@\textbf{Pörtschach}, \emph{https://www.geonames.org/ontologyP.PPL}|pwk}{[}Pört{]}schach am See, 27 8 93\nobreak{}«.  2) Stempel: »\nobreak{}\oindex{Znaim@\textbf{Znaim}, \emph{Besiedelter Ort (A.BSO)}|pwk}Znaim
                                        Znoj\textcolor{gray}{m}o, 28 8 93, 1\textcolor{gray}{2}–4 N\nobreak{}«. }\buchAbdrucke{\weitereDrucke{Arthur Schnitzler, Richard Beer-Hofmann: \emph{Briefwechsel 1891–1931}. Hg. Konstanze Fliedl. Wien, Zürich: \emph{Europaverlag} 1992, S. 51.} }\toendnotes[C]{\smallbreak}\pstart{}{\pb}Herrn \textsc{Dr. Richard
                            Beer-Hofmann}\pend{}\pstart{}k. k. Lieutenant im Infanterie-Regimente Nr. 99\pend{}\pstart{}\textsc{\textcolor{pink}{Znaim}{}\ledrightnote{\textcolor{pink}{Znaim}}}\pend{}\pstart{}\textcolor{pink}{Mähren}{}\ledrightnote{\textcolor{pink}{Mähren}}
                            (?)\pend{}{\bigskip}\pstart
           \noindent{}{\pb}Lieber Richard, aus \textcolor{pink}{\textsc{Pieve di Cadore}}{}\ledrightnote{\textcolor{pink}{Pieve di Cadore}}{ }ſchrieben wir dem \textcolor{blue}{Verfaſſer}{}\ledrightnote{→\textcolor{blue}{Hugo von Hofmannsthal}}
                    von \textcolor{green}{Tizians Tod}{}\ledrightnote{\textcolor{green}{Der Tod des Tizian}}; – aus \textcolor{pink}{\textsc{Pörtschach}}{}\ledrightnote{\textcolor{pink}{Pörtschach}} dem Verfaſſer des \textcolor{green}{Kindes}{}\ledrightnote{\textcolor{green}{Das Kind}} – denn
                    ebenſowahr es iſt dß \textcolor{blue}{\textsc{Tizian}}{}\ledrightnote{\textcolor{blue}{Tizian}} in \textcolor{pink}{\textsc{Pieve di Cadore}}{}\ledrightnote{\textcolor{pink}{Pieve di Cadore}} geboren word\textcolor{gray}{en}, ebenſo wahr iſt es, dß hier ſchon manches Kind
                    geboren ward.\pend
           \pstart
           – Wir haben eine ſchöne Tour gemacht; näheres mündlich. Ihnen gehts hoffentlich
                    gut, und wir grüßen Sie herzlich!\pend
           \pstart \spacefill\mbox{Arthur}\pend{}\pstart
           \noindent{}{[}hs. Salten:{]} Ich habe Sie hier ohne Backenbart gesehen, sorgen
                    dafür, dass er rasch wieder wächst. Frl. \textcolor{blue}{Anna
                        Hiller}{}\ledrightnote{\textcolor{blue}{Anna Kupelwieser}}, die mir das Bild zeigte grüßt Sie. Ich auch\pend
           \pstart
           Ihr{\\[\baselineskip]}\spacefill\mbox{Salten}\pend
           \leftskip=0em{}\endnumbering\briefempfaengerindex{Beer-Hofmann, Richard@\textsc{Beer-Hofmann, Richard}!zzzSalten, Felix@\emph{von Felix Salten}!1893-08-271@{27. 8. 1893}|)be}\briefempfaengerindex{Beer-Hofmann, Richard@\textsc{Beer-Hofmann, Richard}!zzzSchnitzler, Arthur@\emph{von Arthur Schnitzler}!1893-08-271@{27. 8. 1893}|)be}\mylabel{h}  \normalsize

\doendnotes{C}
\bigskip
\vfill

\clearpage

\footnotesize

\lohead{\textsc{register}}

% Definiere theindex-Environment komplett neu ohne reledmac
\makeatletter
\renewenvironment{theindex}{%
  \section*{\indexname}%
  \setlength{\parindent}{0pt}%
  \setlength{\parskip}{0pt plus 0.3pt}%
  \let\item\@idxitem
}{%
  \clearpage
}
\makeatother

\IfFileExists{\jobname-pw.ind}{\input{\jobname-pw.ind}}{}

\end{document}

      