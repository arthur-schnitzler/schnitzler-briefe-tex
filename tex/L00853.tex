%% latex-korrekturansicht-vorspann.tex
%% Vorspann für die Korrekturansicht.
%% Lädt die gemeinsame Datei latex-vorspann.tex mit gesetztem Schalter.

\newif\ifkorrekturansicht
\korrekturansichttrue

\input{../tex-inputs/latex-vorspann}


               \section[Arthur Schnitzler an Hugo von Hofmannsthal, 14. 10. 1898]{ Arthur Schnitzler an Hugo von Hofmannsthal,
                    14. 10. 1898}\nopagebreak\mylabel{v}\rehead{ }\normalsize\beginnumbering\briefempfaengerindex{Hofmannsthal, Hugo von@\textsc{Hofmannsthal, Hugo von}!zzzSchnitzler, Arthur@\emph{von Arthur Schnitzler}!1898-10-142@{14. 10. 1898}|(be} \toendnotes[C]{\smallbreak\pagebreak[2]} \Standort{FDH, Hs-30885,78.}
\physDesc{Brief, 1 Blatt, 3 Seiten
\newline{}Handschrift: schwarze Tinte, deutsche Kurrent\newline{}Ordnung: von Schnitzler mutmaßlich bei der Durchsicht der Korrespondenz 1929
                                    mit Bleistift datiert: »14/10 98« }\buchAbdrucke{\weitereDrucke{Hugo von Hofmannsthal, Arthur Schnitzler: \emph{Briefwechsel}. Hg. Therese Nickl und Heinrich Schnitzler. Frankfurt am Main: \emph{S. Fischer} 1964, S. 114.} }\toendnotes[C]{\smallbreak}\pstart
           \noindent{}{\pb}mein lieber Hugo, es iſt jetzt ſo grau und kühl und feucht, und
                    ich bin ſo verſchnupft und habe eine ganz geſchwollene Naſe, dſs wohl an eine
                        \textcolor{pink}{Hinterbrühl}{}\ledrightnote{\textcolor{pink}{Hinterbrühl}}erreiſe kaum zu denken iſt,
                    vielmehr vermute ich Sie ko{\geminationm}en früher nach \textcolor{pink}{Wien}{}\ledrightnote{\textcolor{pink}{Wien}}. Viele Grüße hab ich Ihnen von \textcolor{blue}{Brahm}{}\ledrightnote{\textcolor{blue}{Otto Brahm}}, \textcolor{blue}{Harden}{}\ledrightnote{\textcolor{blue}{Maximilian Harden}} und der \textcolor{blue}{Dumont}{}\ledrightnote{\textcolor{blue}{Louise Dumont}} zu
                    bringen. Die Leute ſpüren doch ungefähr, wer Sie ſind. Man freut ſich auf Ihr
                        Wiederko{\geminationm}en, auf Ihr neues \textcolor{green}{Stück}{}\ledrightnote{→\textcolor{green}{Der Abenteurer und die Sängerin oder Die Geschenke des Lebens}}, {\pb}– mir ſcheint, im Jänner{ }ſind einige Abende für Sie frei; (von
                    den künftigen Monaten ganz zu geſchweigen.)\pend
           \pstart
           Über meinen \textcolor{pink}{Berl}{}\ledrightnote{\textcolor{pink}{Berlin}}. Aufenthalt mündlich. Der
                    Erfolg nach dem \textcolor{green}{3. Akt}{}\ledrightnote{→\textcolor{green}{Das Vermächtnis. Schauspiel in drei Akten}}
                    war überraſchend ſtark. Während des \textcolor{green}{Akts}{}\ledrightnote{→\textcolor{green}{Das Vermächtnis. Schauspiel in drei Akten}} hatte ich die Empfindung, das \textcolor{green}{Stück}{}\ledrightnote{→\textcolor{green}{Das Vermächtnis. Schauspiel in drei Akten}} iſt hin. Da kamen die letzten
                    paar Scenen, die wirkten unmittelbar und ſind ja wirklich aller Ehren wert. Aber
                    aus welchen {\pb}Tiefen ſteigen ſie empor! –\pend
           \pstart
           – Im übrigen wird ſich das \textcolor{green}{Stück}{}\ledrightnote{→\textcolor{green}{Das Vermächtnis. Schauspiel in drei Akten}} nicht lang halten; ſchon die 3. Vorſtellung war ſchwach
                    beſucht.\pend
           \pstart
           – Von meinen 3 \textcolor{green}{Einaktern}{}\ledrightnote{→\textcolor{green}{Der grüne Kakadu – Paracelsus – Die Gefährtin. Drei Einakter}}
                    hat dem \textcolor{blue}{Br.}{}\ledrightnote{\textcolor{blue}{Otto Brahm}} der \textcolor{green}{gefärbte Vogel}{}\ledrightnote{→\textcolor{green}{Der grüne Kakadu. Groteske in einem Akt}} (wie es ſcheint
                    weitaus) am beſten gefallen. \introOben{}Aufführung wahrſcheinlich
                            Februar mit \textcolor{blue}{Kainz}{}\ledrightnote{\textcolor{blue}{Josef Kainz}}.\introOben{}\pend
           \pstart
           Seien Sie herzlich gegrüßt und laſſen Sie uns bald zuſa{\geminationm}en ſein.\pend
           \pstart Ihr \spacefill\mbox{Arthur}\pend{}\pstart
           \textcolor{pink}{Wien}{}\ledrightnote{\textcolor{pink}{Wien}},
                        14. X. 98.\pend
           \endnumbering\briefempfaengerindex{Hofmannsthal, Hugo von@\textsc{Hofmannsthal, Hugo von}!zzzSchnitzler, Arthur@\emph{von Arthur Schnitzler}!1898-10-142@{14. 10. 1898}|)be}\mylabel{h}  \normalsize

\doendnotes{C}
\bigskip
\vfill

\clearpage

\footnotesize

\lohead{\textsc{register}}

% Definiere theindex-Environment komplett neu ohne reledmac
\makeatletter
\renewenvironment{theindex}{%
  \section*{\indexname}%
  \setlength{\parindent}{0pt}%
  \setlength{\parskip}{0pt plus 0.3pt}%
  \let\item\@idxitem
}{%
  \clearpage
}
\makeatother

\IfFileExists{\jobname-pw.ind}{\input{\jobname-pw.ind}}{}

\end{document}

      