%% latex-korrekturansicht-vorspann.tex
%% Vorspann für die Korrekturansicht.
%% Lädt die gemeinsame Datei latex-vorspann.tex mit gesetztem Schalter.

\newif\ifkorrekturansicht
\korrekturansichttrue

\input{../tex-inputs/latex-vorspann}


               \section[Hugo von Hofmannsthal an Arthur Schnitzler, 12. 6. 1908]{ Hugo von Hofmannsthal an Arthur Schnitzler, 12. 6. 1908}\nopagebreak\mylabel{v}\rehead{ }\normalsize\beginnumbering\briefempfaengerindex{Schnitzler, Arthur@\textsc{Schnitzler, Arthur}!zzzHofmannsthal, Hugo von@\emph{von Hugo von Hofmannsthal}!1908-06-121@{12. 6. 1908}|(be} \toendnotes[C]{\smallbreak\pagebreak[2]} \Standort{CUL, Schnitzler, B 43.}
\physDesc{Postkarte
\newline{}Handschrift: schwarze Tinte, deutsche Kurrent\newline{}Versand: Stempel: »\nobreak{}\oindex{Rodaun@\textbf{Rodaun}, \emph{Teil eines besiedelten Ortes (A.BSOX)}|pwk}Rodaun, 12 6 08, 9 V\nobreak{}«.  
\newline{}Schnitzler: mit Bleistift datiert: »12/6 908« \newline{}Ordnung: 1) mit Bleistift von unbekannter Hand nummeriert:
                                 »\strikeout{291}« 2) mit Bleistift von unbekannter Hand nummeriert:
                                 »297«}\buchAbdrucke{\weitereDrucke{Hugo von Hofmannsthal, Arthur Schnitzler: \emph{Briefwechsel}. Hg. Therese Nickl und Heinrich Schnitzler. Frankfurt am Main: \emph{S. Fischer} 1964, S. 237.} }\toendnotes[C]{\smallbreak}\pstart{}{\pb}\textsc{Herrn Dr Arthur Schnitzler}\pend{}\pstart{}\textcolor{pink}{\textsc{Wien}}{}\ledrightnote{\textcolor{pink}{Wien}}\pend{}\pstart{}\textsc{\textcolor{pink}{XVIII Spöttelgasse 7}{}\ledrightnote{\textcolor{pink}{Edmund-Weiß-Gasse}}}\pend{}{\bigskip}\pstart
           \noindent{}{\pb}Lieber, \textcolor{blue}{Clemens Franckenſtein}{}\ledrightnote{\textcolor{blue}{Clemens von Franckenstein}} bittet mich, ſeinen Dank zu \substVorne{}\textsuperscript{Ü}\substDazwischen{}ü\substHinten{}bermitteln für gütige Zuſendung Ihres \textcolor{green}{Buches}{}\ledrightnote{→\textcolor{green}{Der Weg ins Freie. Roman}}, da er Ihre Adreſſe nicht weiß.\pend
           \pstart
           Ich hoffe, baldigſt von Ihnen zu hören, daſs Sie in der Arbeit u. zufrieden ſind. Ich
               arbeite.\pend
           \pstart
           Von Herzen.{\\[\baselineskip]}\spacefill\mbox{Hugo.}\pend
           \leftskip=0em{}\endnumbering\briefempfaengerindex{Schnitzler, Arthur@\textsc{Schnitzler, Arthur}!zzzHofmannsthal, Hugo von@\emph{von Hugo von Hofmannsthal}!1908-06-121@{12. 6. 1908}|)be}\mylabel{h}  \normalsize

\doendnotes{C}
\bigskip
\vfill

\clearpage

\footnotesize

\lohead{\textsc{register}}

% Definiere theindex-Environment komplett neu ohne reledmac
\makeatletter
\renewenvironment{theindex}{%
  \section*{\indexname}%
  \setlength{\parindent}{0pt}%
  \setlength{\parskip}{0pt plus 0.3pt}%
  \let\item\@idxitem
}{%
  \clearpage
}
\makeatother

\IfFileExists{\jobname-pw.ind}{\input{\jobname-pw.ind}}{}

\end{document}

      