%% latex-korrekturansicht-vorspann.tex
%% Vorspann für die Korrekturansicht.
%% Lädt die gemeinsame Datei latex-vorspann.tex mit gesetztem Schalter.

\newif\ifkorrekturansicht
\korrekturansichttrue

\input{../tex-inputs/latex-vorspann}


               \section[Gerty von Hofmannsthal an Olga Schnitzler, 15. 9. 1909]{ Gerty von Hofmannsthal an Olga Schnitzler, 15. 9. 1909}\nopagebreak\mylabel{v}\rehead{ }\normalsize\beginnumbering\briefempfaengerindex{Schnitzler, Olga@\textsc{Schnitzler, Olga}!zzzHofmannsthal, Gertrude von@\emph{von Gertrude von Hofmannsthal}!1909-09-153@{15. 9. 1909}|(be} \toendnotes[C]{\smallbreak\pagebreak[2]} \Standort{CUL, Schnitzler, B 43.}
\physDesc{Brief, 1 Blatt, 3 Seiten
\newline{}Handschrift: schwarze Tinte, lateinische Kurrent
\newline{}Schnitzler: mit Bleistift beschriftet: »\textsc{Hofma{\geminationn}sthal}« und datiert: »Sept 909« \newline{}Ordnung: mit Bleistift von unbekannter Hand nummeriert:
                                                »308« }\toendnotes[C]{\smallbreak}\pstart
           \raggedleft{}{\pb}\label{K_L01876_1v}\edtext{Mittwoch}{\lemma{\textnormal{\emph{Mittwoch}}}\Cendnote{\textnormal{Die Datierung erfolgt auf den
                            ersten Mittwoch nach der Geburt von \textcolor{blue}{Lili
                                Schnitzler}.}}}\label{K_L01876_1h}\pend
           \pstart
           Liebe Olga, wir freuen uns ja riesig! Wie schön, dass \textcolor{blue}{es}{}\ledrightnote{→\textcolor{blue}{Lili Schnitzler}} auch ein Mäderl ist;
                    ist’s ein schwarzes oder blondes? Ich darf gar nicht anfangen zu fragen sonst
                    werd ich gar nicht fertig. Ich lebe in Gedanken alle Stunden und Tag mit und
                    kann mir vorstellen wie {\pb}glücklich
                    und zufrieden Sie sein werden, wenn alles gut vorübergegangen ist. Da kommen
                    dann so ruhige Tage, in denen man sich nur dafür interessiert ob das Kind
                    trinkt, ob es schläft etc. nicht wahr?\pend
           \pstart
           Und dem \textcolor{blue}{Arthur}{}\ledrightnote{} sagen Sie auch alle Liebe von
                    uns {\pb}und wie sehr wir uns über
                    seine kleine \textcolor{blue}{Tochter}{}\ledrightnote{→\textcolor{blue}{Lili Schnitzler}}
                    freuen!\pend
           \pstart
           Wie leid thut’s mir, dass ich nicht so nah von Ihnen bin um Sie zu sehen und
                    Ihnen hie und da ein bissl Gesellschaft leisten kann!\pend
           \pstart
           Also Adieu liebe Olga{\\[\baselineskip]}Von Herzen Ihre \spacefill\mbox{Gerty}\pend
           \leftskip=0em{}\endnumbering\briefempfaengerindex{Schnitzler, Olga@\textsc{Schnitzler, Olga}!zzzHofmannsthal, Gertrude von@\emph{von Gertrude von Hofmannsthal}!1909-09-153@{15. 9. 1909}|)be}\mylabel{h}  \normalsize

\doendnotes{C}
\bigskip
\vfill

\clearpage

\footnotesize

\lohead{\textsc{register}}

% Definiere theindex-Environment komplett neu ohne reledmac
\makeatletter
\renewenvironment{theindex}{%
  \section*{\indexname}%
  \setlength{\parindent}{0pt}%
  \setlength{\parskip}{0pt plus 0.3pt}%
  \let\item\@idxitem
}{%
  \clearpage
}
\makeatother

\IfFileExists{\jobname-pw.ind}{\input{\jobname-pw.ind}}{}

\end{document}

      