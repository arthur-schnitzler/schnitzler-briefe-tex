%% latex-korrekturansicht-vorspann.tex
%% Vorspann für die Korrekturansicht.
%% Lädt die gemeinsame Datei latex-vorspann.tex mit gesetztem Schalter.

\newif\ifkorrekturansicht
\korrekturansichttrue

\input{../tex-inputs/latex-vorspann}


               \section[Richard Beer-Hofmann an Arthur Schnitzler, 12. 7. 1920]{ Richard Beer-Hofmann an Arthur Schnitzler,
               12. 7. 1920}\nopagebreak\mylabel{v}\rehead{ }\normalsize\beginnumbering\briefempfaengerindex{Schnitzler, Arthur@\textsc{Schnitzler, Arthur}!zzzBeer-Hofmann, Richard@\emph{von Richard Beer-Hofmann}!1920-07-121@{12. 7. 1920}|(be} \toendnotes[C]{\smallbreak\pagebreak[2]} \Standort{CUL, Schnitzler, B 8.}
\physDesc{Brief, 1 Blatt, 2 Seiten
\newline{}Handschrift: Bleistift, lateinische Kurrent\newline{}Ordnung: mit Bleistift von unbekannter Hand nummeriert: »270« }\buchAbdrucke{\weitereDrucke{Arthur Schnitzler, Richard Beer-Hofmann: \emph{Briefwechsel 1891–1931}. Hg. Konstanze Fliedl. Wien, Zürich: \emph{Europaverlag} 1992, S. 227.} }\pstart
           {\pb}\textcolor{pink}{Bad Aussee}{}\ledrightnote{\textcolor{pink}{Bad Aussee}}{ }12. VII.  20\pend
           \pstart
           Lieber Arthur! Eben erhalte ich von \textcolor{blue}{S. Fischer}{}\ledrightnote{\textcolor{blue}{Samuel Fischer}} die Mitteilung von einem 25 {\%}
               Teuerungszuschlag – der »\uline{tantièmenfrei}« sein soll.
               Wie stellen Sie sich dazu? Wie \textcolor{blue}{Hugo}{}\ledrightnote{\textcolor{blue}{Hugo von Hofmannsthal}}, der ja noch
               in \textcolor{pink}{Wien}{}\ledrightnote{\textcolor{pink}{Wien}} ist. Bitte schreiben Sie mir zwei Zeilen was
               Sie tun. Ich finde es unerhört! Tatsächlich \strikeout{tra} beko{\geminationm}t der Autor 15
               od. 16 {\%} des Ladenpreises der Sortimenter mindestens 50 wozu
               noch sein privater {\pb}25 {\%} Teuerungszuschlag ko{\geminationm}t. Muss
               man sich das gefallen lassen?\pend
           \pstart
           Herzlichst Ihr{\\[\baselineskip]}\spacefill\mbox{Richard}\pend
           \leftskip=0em{}\endnumbering\briefempfaengerindex{Schnitzler, Arthur@\textsc{Schnitzler, Arthur}!zzzBeer-Hofmann, Richard@\emph{von Richard Beer-Hofmann}!1920-07-121@{12. 7. 1920}|)be}\mylabel{h}  \normalsize

\doendnotes{C}
\bigskip
\vfill

\clearpage

\footnotesize

\lohead{\textsc{register}}

% Definiere theindex-Environment komplett neu ohne reledmac
\makeatletter
\renewenvironment{theindex}{%
  \section*{\indexname}%
  \setlength{\parindent}{0pt}%
  \setlength{\parskip}{0pt plus 0.3pt}%
  \let\item\@idxitem
}{%
  \clearpage
}
\makeatother

\IfFileExists{\jobname-pw.ind}{\input{\jobname-pw.ind}}{}

\end{document}

      