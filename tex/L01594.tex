%% latex-korrekturansicht-vorspann.tex
%% Vorspann für die Korrekturansicht.
%% Lädt die gemeinsame Datei latex-vorspann.tex mit gesetztem Schalter.

\newif\ifkorrekturansicht
\korrekturansichttrue

\input{../tex-inputs/latex-vorspann}


               \section[Hermann Bahr an Arthur Schnitzler, {[}23. 3. 1906{]}]{ Hermann Bahr an Arthur Schnitzler, {[}23. 3. 1906{]}}\nopagebreak\mylabel{v}\rehead{ }\normalsize\beginnumbering\briefempfaengerindex{Schnitzler, Arthur@\textsc{Schnitzler, Arthur}!zzzBahr, Hermann@\emph{von Hermann Bahr}!1906-03-233@{{[}23. 3. 1906{]}}|(be} \toendnotes[C]{\smallbreak\pagebreak[2]} \Standort{CUL, Schnitzler, B 5b.}
\physDesc{Telegramm
\newline{}Handschrift einer Schreibkraft: blaue Tinte, lateinische Kurrent\newline{}Versand: 1) »\textcolor{pink}{Wien} 93 \textcolor{gray}{\textbf{Nr.}} 132 \textcolor{gray}{\textbf{Taxw.}} 11 \textcolor{gray}{\textbf{(W.{\dots} Ch.{\dots}) aufgegeben am {\dots}/{\dots}190{\dots} um}} 6 \textcolor{gray}{\textbf{Uhr}} n\textcolor{gray}{\textbf{Mittag.}}« 2) beschnitten
\newline{}Schnitzler: mit Bleistift datiert: »23/3/906« \newline{}Ordnung: mit Bleistift von unbekannter Hand nummeriert:
                                    »138« }\buchAbdrucke{\weitereDrucke{Hermann Bahr, Arthur Schnitzler: \emph{Briefwechsel, Aufzeichnungen, Dokumente (1891–1931)}. Hg. Kurt Ifkovits und Martin Anton Müller. Göttingen: \emph{Wallstein} 2018, S. 376.} }\toendnotes[C]{\smallbreak}\pstart
           \noindent{}{\pb}\textcolor{blue}{Mildenburg}{}\ledrightnote{\textcolor{blue}{Anna Bahr-Mildenburg}}{ }\label{K_L01594_1v}\edtext{singt morgen}{\lemma{\textnormal{\emph{singt morgen}}}\Cendnote{\textnormal{Am 24. 3. 1906 brachte die \textcolor{pink}{Hofoper}{ }\emph{\textcolor{green}{Don Giovanni}}; \textcolor{blue}{Mildenburg} sang \textcolor{green}{Donna
                     Anna}.}}}\label{K_L01594_1h} herzlichſt \spacefill\mbox{herrmann}\pend
           \endnumbering\briefempfaengerindex{Schnitzler, Arthur@\textsc{Schnitzler, Arthur}!zzzBahr, Hermann@\emph{von Hermann Bahr}!1906-03-233@{{[}23. 3. 1906{]}}|)be}\mylabel{h}  \normalsize

\doendnotes{C}
\bigskip
\vfill

\clearpage

\footnotesize

\lohead{\textsc{register}}

% Definiere theindex-Environment komplett neu ohne reledmac
\makeatletter
\renewenvironment{theindex}{%
  \section*{\indexname}%
  \setlength{\parindent}{0pt}%
  \setlength{\parskip}{0pt plus 0.3pt}%
  \let\item\@idxitem
}{%
  \clearpage
}
\makeatother

\IfFileExists{\jobname-pw.ind}{\input{\jobname-pw.ind}}{}

\end{document}

      