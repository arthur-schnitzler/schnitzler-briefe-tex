%% latex-korrekturansicht-vorspann.tex
%% Vorspann für die Korrekturansicht.
%% Lädt die gemeinsame Datei latex-vorspann.tex mit gesetztem Schalter.

\newif\ifkorrekturansicht
\korrekturansichttrue

\input{../tex-inputs/latex-vorspann}


               \section[Arthur Schnitzler an Hugo von Hofmannsthal, 11. 9. 1905]{ Arthur Schnitzler an Hugo von Hofmannsthal, 11. 9. 1905}\nopagebreak\mylabel{v}\rehead{ }\normalsize\beginnumbering\briefempfaengerindex{Hofmannsthal, Hugo von@\textsc{Hofmannsthal, Hugo von}!zzzSchnitzler, Arthur@\emph{von Arthur Schnitzler}!1905-09-111@{11. 9. 1905}|(be} \toendnotes[C]{\smallbreak\pagebreak[2]} \Standort{FDH, Hs-30885,122.}
\physDesc{Brief, 1 Blatt, 4 Seiten
\newline{}Handschrift: schwarze Tinte, deutsche Kurrent}\buchAbdrucke{\weitereDrucke{Hugo von Hofmannsthal, Arthur Schnitzler: \emph{Briefwechsel}. Hg. Therese Nickl und Heinrich Schnitzler. Frankfurt am Main: \emph{S. Fischer} 1964, S. 214.} }\toendnotes[C]{\smallbreak}\pstart
           \raggedleft{}{\pb}\textcolor{pink}{Wien}{}\ledrightnote{\textcolor{pink}{Wien}}{ }11. 9. 905\pend
           \pstart{}lieber Hugo,\pend\pstart
           die Sache mit dem \textcolor{brown}{Burgtheater}{}\ledrightnote{\textcolor{brown}{Burgtheater}} war ungeheuer einfach.
                  \textcolor{blue}{Brahm}{}\ledrightnote{\textcolor{blue}{Otto Brahm}}{ }ſchrieb mir \label{K_L01545_1v}\edtext{Ende Auguſt}{\lemma{\textnormal{\emph{Ende Auguſt}}}\Cendnote{\textnormal{am 27. 8. 1905 (\emph{Briefwechsel Schnitzer/Brahm},
                  S. 187–189.)}}}\label{K_L01545_1h}, \textcolor{blue}{Schlenther}{}\ledrightnote{\textcolor{blue}{Paul Schlenther}} habe
               ihn mit der Miſſion betraut, mich zur Einſendg meines neueſten \label{T_L01545_1v}\edtext{aufzufordern}{\lemma{\textnormal{\emph{aufzufordern}}}\Cendnote{\textnormal{Er schreibt: »einzuſenden«.}}}\label{T_L01545_1h}.
               Ich hierauf, nicht faul, ſchreibe \textcolor{blue}{Schl.}{}\ledrightnote{\textcolor{blue}{Paul Schlenther}}, daſs ich
               eine fertige \textcolor{green}{Komoedie}{}\ledrightnote{→\textcolor{green}{Zwischenspiel. Komödie in drei Akten}}, u 2 \textcolor{green}{Dramenakte}{}\ledrightnote{→\textcolor{green}{Der Ruf des Lebens. Schauspiel in drei Akten}} auf Lager ha\substVorne{}\textsuperscript{tte}\substDazwischen{}be\substHinten{}, er telegrafirt, noch fleißiger, ſoll ihm alles schicken; \introOben{}ich thu es,\introOben{} er antwortet 5 Tage drauf, die Entſcheidg über \textcolor{green}{Dra{\pb}ma}{}\ledrightnote{→\textcolor{green}{Der Ruf des Lebens. Schauspiel in drei Akten}}{ }\substVorne{}\textsuperscript{laſſe}\substDazwischen{}bitte\substHinten{} er bis nach Vollendg aufſchieben zu dürfen, \textcolor{green}{Komoedie}{}\ledrightnote{→\textcolor{green}{Zwischenspiel. Komödie in drei Akten}} nehme er an Mitte
                  October (ich hatte frühen Termin zur Beding gemacht), wolle meine
               Beſetzsvorſchläge, er ni{\geminationm}t ſie ſelben Tags ebenſo
               telegrafiſch an, und am nächſten Morgen ſteht die \label{K_L01545_2v}\edtext{Notiz}{\lemma{\textnormal{\emph{Notiz}}}\Cendnote{\textnormal{»Ende
                     Oktober geht \textcolor{blue}{Schnitzler}s neue Komödie ›\textcolor{green}{\so{Zwischenspiel}}‹ zum erstenmal in Szene.« ([O. V.:] \emph{\textcolor{green}{Aus den Theatern. Wien, 9. September}}. In: \emph{\textcolor{green}{Neue Freie Presse}}, Nr. 14744, 9. 9. 1905,
                     Abendblatt, S. 4.)}}}\label{K_L01545_2h} in der Zeitung. Es ko{\geminationm}t hier vor \textcolor{pink}{Berlin}{}\ledrightnote{\textcolor{pink}{Berlin}}; mit
                  \textcolor{blue}{Brahm}{}\ledrightnote{\textcolor{blue}{Otto Brahm}} bin ich erſt heute (vor 5 Minuten kam
               das endgiltige Telegra{\geminationm}) einig geworden; Verzögerung,
               weil er durchaus beide \textcolor{green}{Stücke}{}\ledrightnote{→\textcolor{green}{Zwischenspiel. Komödie in drei Akten}{\newline}→\textcolor{green}{Der Ruf des Lebens. Schauspiel in drei Akten}} wollte – Mit dem \textcolor{brown}{\textsc{Reinhardt}theater}{}\ledrightnote{→\textcolor{brown}{Deutsches Theater Berlin}} wird ſich wahr{\pb}ſcheinlich nichts machen laſſen; was ſie mir im Lauf der
               letzten 10 Tage an (mildeſter Ausdruck) Schlampereien angethan, iſt unglaublich. Der
               letzte Scherz war, daſs ich \label{K_L01545_3v}\edtext{Mittwoch{ }ein Telegramm bekam}{\lemma{\textnormal{\emph{Mittwoch … bekam}}}\Cendnote{\textnormal{abgedruckt in: \emph{Der Briefwechsel Arthur Schnitzlers mit Max Reinhardt und
                        dessen Mitarbeitern}. Hg. Renate Wagner. Salzburg: \emph{Otto
                        Müller Verlag}{ }1971, S. 50. Den versprochenen Brief (und einen weiteren,
                  der am 12. 9. 1905 angekündigt wurde) dürfte er nicht erhalten
                  haben.}}}\label{K_L01545_3h} dſs ein ausführlicher Brief auf d. Wege – und der bisher nicht da
               iſt. Es ſtand beinah ſchon feſt für mich, dſs die \textcolor{blue}{\textsc{Sorma}}{}\ledrightnote{\textcolor{blue}{Agnes Sorma}} die \textcolor{green}{Komoedie}{}\ledrightnote{→\textcolor{green}{Zwischenspiel. Komödie in drei Akten}}{ }ſpielen müſſte. Über all dies mündlich
               näheres. –\pend
           \pstart
           Wir bleiben bis nach 15. hier, wohl 20., denken da{\geminationn} auf 10 Tage fortzugehn, – \textcolor{pink}{Salzka{\geminationm}ergut}{}\ledrightnote{\textcolor{pink}{Salzkammergut}} kaum; vielleicht nur {\pb}\label{K_L01545_4v}\edtext{\textcolor{pink}{Semmering}{}\ledrightnote{\textcolor{pink}{Semmering}}}{\lemma{\textnormal{\emph{Semmering}}}\Cendnote{\textnormal{Dahin fuhren sie vom 22.
                  bis zum 26. 9. 1905.}}}\label{K_L01545_4h}. – Mit dem \textcolor{green}{3. Akt}{}\ledrightnote{→\textcolor{green}{Der Ruf des Lebens. Schauspiel in drei Akten}} glaub ich zu einer Art Reſultat zu ko{\geminationm}en – das 3 mal einaktige des Stoffes iſt natürlich
               nicht ganz zu beſiegen, es ko{\geminationm}t im weſentlichen, was man
               auch thut, dramatiſch auf einen Schwindel heraus. Nun, das iſt unſer Metier.\pend
           \pstart
           Ich freue mich, dſs Sie viel arbeiten, und ſehe dem nächſten \label{K_L01545_5v}\edtext{Vorleſungsabend}{\lemma{\textnormal{\emph{Vorleſungsabend}}}\Cendnote{\textnormal{Gemeint ist eine Vorlesung von Werken in privatem Kreis.}}}\label{K_L01545_5h}
               mit ſchönſter Erwartung entgegen. Was hat Sie ſo raſch aus \textcolor{pink}{\textsc{Misurina}}{}\ledrightnote{\textcolor{pink}{Misurina}} vertrieben?\pend
           \pstart
           Wir grüßen Sie \textcolor{blue}{Beide}{}\ledrightnote{→\textcolor{blue}{Olga Schnitzler}}{ }\textcolor{blue}{Beide}{}\ledrightnote{→\textcolor{blue}{Gertrude von Hofmannsthal}}.\pend
           \pstart Herzlichst Ihr \spacefill\mbox{A.}\pend{}\pstart
           \noindent{}\label{T_L01545_2v}\edtext{Sehen Sie \textcolor{blue}{Burckhard}{}\ledrightnote{\textcolor{blue}{Max Eugen Burckhard}}, grüßen Sie ihn ſehr.}{\lemma{\textnormal{\emph{Sehen … ſehr.}}}\Cendnote{\textnormal{neben der Anrede auf dem Kopf}}}\label{T_L01545_2h}\pend
           \endnumbering\briefempfaengerindex{Hofmannsthal, Hugo von@\textsc{Hofmannsthal, Hugo von}!zzzSchnitzler, Arthur@\emph{von Arthur Schnitzler}!1905-09-111@{11. 9. 1905}|)be}\mylabel{h}  \normalsize

\doendnotes{C}
\bigskip
\vfill

\clearpage

\footnotesize

\lohead{\textsc{register}}

% Definiere theindex-Environment komplett neu ohne reledmac
\makeatletter
\renewenvironment{theindex}{%
  \section*{\indexname}%
  \setlength{\parindent}{0pt}%
  \setlength{\parskip}{0pt plus 0.3pt}%
  \let\item\@idxitem
}{%
  \clearpage
}
\makeatother

\IfFileExists{\jobname-pw.ind}{\input{\jobname-pw.ind}}{}

\end{document}

      