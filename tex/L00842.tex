%% latex-korrekturansicht-vorspann.tex
%% Vorspann für die Korrekturansicht.
%% Lädt die gemeinsame Datei latex-vorspann.tex mit gesetztem Schalter.

\newif\ifkorrekturansicht
\korrekturansichttrue

\input{../tex-inputs/latex-vorspann}


               \section[Arthur Schnitzler an Hugo von Hofmannsthal, 31. 8. 1898]{ Arthur Schnitzler an Hugo von Hofmannsthal, 31. 8. 1898}\nopagebreak\mylabel{v}\rehead{ }\normalsize\beginnumbering\briefempfaengerindex{Hofmannsthal, Hugo von@\textsc{Hofmannsthal, Hugo von}!zzzSchnitzler, Arthur@\emph{von Arthur Schnitzler}!1898-08-311@{31. 8. 1898}|(be} \toendnotes[C]{\smallbreak\pagebreak[2]} \Standort{FDH, Hs-30885,76.}
\physDesc{Bildpostkarte
\newline{}Handschrift: Bleistift, deutsche Kurrent\newline{}Versand: Stempel: »\nobreak{}\oindex{Lugano@\textbf{Lugano}, \emph{Besiedelter Ort (A.BSO)}|pwk}Lugano Lettere, 1. IX. 98\nobreak{}«.  }\buchAbdrucke{\weitereDrucke{Hugo von Hofmannsthal, Arthur Schnitzler: \emph{Briefwechsel}. Hg. Therese Nickl und Heinrich Schnitzler. Frankfurt am Main: \emph{S. Fischer} 1964, S. 111–112.} }\toendnotes[C]{\smallbreak}\pstart{}{\pb}Hrn \textsc{Hugo v
                            Hofmannsthal}\pend{}\pstart{}\textsc{\textcolor{pink}{Lugano}{}\ledrightnote{\textcolor{pink}{Lugano}}}\pend{}\pstart{}\textsc{\textcolor{pink}{Hotel du parc}{}\ledrightnote{\textcolor{pink}{Hôtel du Parc}}}.\pend{}\pstart{}\textcolor{pink}{\textsc{Svizzera}}{}\ledrightnote{\textcolor{pink}{Schweiz}}\pend{}{\bigskip}\pstart
           \noindent{}\centering{}\textcolor{gray}{\textbf{{\pb}\textcolor{pink}{Bologna}{}\ledrightnote{\textcolor{pink}{Bologna}}. \textcolor{pink}{Le Torri Carisenda e Asinelli}{}\ledrightnote{\textcolor{pink}{Le due Torri: Garisenda e degli Asinelli}}.}}\pend
           \pstart
           \raggedleft{}31. 8. 98.\pend
           \pstart{}Mein Lieber Hugo, \pend\pstart
           Ich freue mich ſehr, meinem Einfall nachgegeben zu haben und ein paar \textcolor{pink}{ital.}{}\ledrightnote{\textcolor{pink}{Italien}}{ }Städte zu ſehen. Wär’s mir doch bald möglich,
                    weiter und auf längre Zeit, und, ich glaub das zu wünſchen, nicht allein. – Hier
                    ſende ich Ihnen die \textcolor{pink}{zwei ſchiefen Türme}{}\ledrightnote{\textcolor{pink}{Le due Torri: Garisenda e degli Asinelli}}; der
                    eine gehört dem \textcolor{blue}{Richard}{}\ledrightnote{\textcolor{blue}{Richard Beer-Hofmann}}\footnote{\noindent{}er kann wählen}, ebenſo wie Ihnen beiden meine herzlichſten Grüße. Schreiben Sie mir
                    nach \textcolor{pink}{Wien}{}\ledrightnote{\textcolor{pink}{Wien}}; ich bin wahrſcheinlich So{\geminationn}tag zu Hauſe.\pend
           \pstart Ihr \spacefill\mbox{Arthur}\pend{}\pstart
           \noindent{}\label{T_L00842_1v}\edtext{Was für Proſa ſchreiben Sie?}{\lemma{\textnormal{\emph{Was … Sie?}}}\Cendnote{\textnormal{in der linken oberen Ecke auf dem
                            Kopf}}}\label{T_L00842_1h}\pend
           \endnumbering\briefempfaengerindex{Hofmannsthal, Hugo von@\textsc{Hofmannsthal, Hugo von}!zzzSchnitzler, Arthur@\emph{von Arthur Schnitzler}!1898-08-311@{31. 8. 1898}|)be}\mylabel{h}  \normalsize

\doendnotes{C}
\bigskip
\vfill

\clearpage

\footnotesize

\lohead{\textsc{register}}

% Definiere theindex-Environment komplett neu ohne reledmac
\makeatletter
\renewenvironment{theindex}{%
  \section*{\indexname}%
  \setlength{\parindent}{0pt}%
  \setlength{\parskip}{0pt plus 0.3pt}%
  \let\item\@idxitem
}{%
  \clearpage
}
\makeatother

\IfFileExists{\jobname-pw.ind}{\input{\jobname-pw.ind}}{}

\end{document}

      