%% latex-korrekturansicht-vorspann.tex
%% Vorspann für die Korrekturansicht.
%% Lädt die gemeinsame Datei latex-vorspann.tex mit gesetztem Schalter.

\newif\ifkorrekturansicht
\korrekturansichttrue

\input{../tex-inputs/latex-vorspann}


               \section[Arthur Schnitzler an Stefan Großmann, 9. 10. 1907]{ Arthur Schnitzler an Stefan Großmann, 9. 10. 1907}\nopagebreak\mylabel{v}\rehead{ }\normalsize\beginnumbering\briefempfaengerindex{Grossmann, Stefan@\textsc{Großmann, Stefan}!zzzSchnitzler, Arthur@\emph{von Arthur Schnitzler}!1907-10-091@{9. 10. 1907}|(be} \toendnotes[C]{\smallbreak\pagebreak[2]} \Standort{DLA, A:Schnitzler, HS.NZ85.1.896.}
\physDesc{Brief, 1 Blatt, 1 Seite, maschineller Durchschlag
\newline{}Schreibmaschine
\newline{}Handschrift: 1) Bleistift, deutsche Kurrent (\noindent{}eine Ergänzung)\hspace{1em}2) roter Buntstift, deutsche Kurrent (\noindent{}vier Unterstreichungen)\hspace{1em}}\toendnotes[C]{\smallbreak}\pstart
           \raggedleft{}{\pb}9. Okt. 07. \pend
           \pstart{}Sehr geehrter Herr,\pend\pstart
           Die beiden Titel, die Sie in meinem vorigen Brief nicht lesen konnten waren »\textcolor{green}{das neue Lied}{}\ledrightnote{\textcolor{green}{Das neue Lied}}« und die »\textcolor{green}{letzten Masken}{}\ledrightnote{\textcolor{green}{Die letzten Masken}}«. Das erste, eine Novelle aus der Sammlung
                        »\textcolor{green}{Dämmerseelen}{}\ledrightnote{\textcolor{green}{Die Fremde}}«, das zweite ein Einakter
                    aus dem Zyklus »\textcolor{green}{lebendige Stunden}{}\ledrightnote{\textcolor{green}{Lebendige Stunden. Vier Einakter}}«, beide
                    nicht besonders heiter und wohl auch zu lang.\pend
           \pstart
           Kennen Sie vielleicht die kleine Novellette »\textcolor{green}{Exzentrik}{}\ledrightnote{\textcolor{green}{Excentric}}« aus der Sammlung »\textcolor{green}{die
                        griechische Tänzerin}{}\ledrightnote{\textcolor{green}{Die griechische Tänzerin. Novellette}}«? Sie wird von den Leuten im Allgemeinen für
                    lustig gehalten, hat sich schon einigemale als Vorlesestück bewährt. Wollen Sie
                    vielleicht die Güte haben sie sich anzusehen und mir zu sagen, ob Sie sie für
                    den Abschluss des Abends für geeignet halten.\pend
           \pstart
           Ich bitte Sie auch mir womöglich die Hausnummer mitzuteilen, wo ich in der \textcolor{pink}{Königseggasse}{}\ledrightnote{\textcolor{pink}{Königseggasse}} lesen soll.\pend
           \pstart
           Ihrer freundlichen Antwort entgegensehend{\\[\baselineskip]}Ihr sehr ergebener\pend
           \leftskip=0em{}{\bigskip}\pstart
           \noindent{}\label{T_L01718_1v}\edtext{Herrn}{\lemma{\textnormal{\emph{Herrn}}}\Cendnote{\textnormal{nachträglich handschriftlich ergänzt}}}\label{T_L01718_1h}
                        Stefan Grossmann, \textcolor{pink}{Wien}{}\ledrightnote{\textcolor{pink}{Wien}}\pend
           \endnumbering\briefempfaengerindex{Grossmann, Stefan@\textsc{Großmann, Stefan}!zzzSchnitzler, Arthur@\emph{von Arthur Schnitzler}!1907-10-091@{9. 10. 1907}|)be}\mylabel{h}  \normalsize

\doendnotes{C}
\bigskip
\vfill

\clearpage

\footnotesize

\lohead{\textsc{register}}

% Definiere theindex-Environment komplett neu ohne reledmac
\makeatletter
\renewenvironment{theindex}{%
  \section*{\indexname}%
  \setlength{\parindent}{0pt}%
  \setlength{\parskip}{0pt plus 0.3pt}%
  \let\item\@idxitem
}{%
  \clearpage
}
\makeatother

\IfFileExists{\jobname-pw.ind}{\input{\jobname-pw.ind}}{}

\end{document}

      