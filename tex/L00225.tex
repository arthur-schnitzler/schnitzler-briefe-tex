%% latex-korrekturansicht-vorspann.tex
%% Vorspann für die Korrekturansicht.
%% Lädt die gemeinsame Datei latex-vorspann.tex mit gesetztem Schalter.

\newif\ifkorrekturansicht
\korrekturansichttrue

\input{../tex-inputs/latex-vorspann}


               \section[Hugo von Hofmannsthal an Arthur Schnitzler, {[}21.? 6. 1893{]}]{ Hugo von Hofmannsthal an Arthur Schnitzler, {[}21.? 6. 1893{]}}\nopagebreak\mylabel{v}\rehead{ }\normalsize\beginnumbering\briefempfaengerindex{Schnitzler, Arthur@\textsc{Schnitzler, Arthur}!zzzHofmannsthal, Hugo von@\emph{von Hugo von Hofmannsthal}!1893-06-212@{{[}21.? 6. 1893{]}}|(be} \toendnotes[C]{\smallbreak\pagebreak[2]} \Standort{CUL, Schnitzler, B 43.}
\physDesc{Brief, 1 Blatt (mit aufgeprägtem Wappen), 1 Seite
\newline{}Handschrift: schwarze Tinte, deutsche Kurrent (\noindent{}Tinte stark verwischt)
\newline{}Schnitzler: mit Bleistift datiert: »Juni 93« und nummeriert »49« }\buchAbdrucke{\weitereDrucke{Hugo von Hofmannsthal, Arthur Schnitzler: \emph{Briefwechsel}. Hg. Therese Nickl und Heinrich Schnitzler. Frankfurt am Main: \emph{S. Fischer} 1964, S. 39.} }\toendnotes[C]{\smallbreak}\pstart{}{\pb}Lieber Arthur.\pend\pstart
           Heute geht nicht. Möchte morgen auf ganzen Tag, außer Regen. Schreiben Sie
               pneumatiſch, ob recht iſt. Wenn Sie nicht auf \uline{viele}
               Zeit nach \label{K_L00225_1v}\edtext{\textcolor{pink}{Baden}{}\ledrightnote{\textcolor{pink}{Baden bei Wien}}}{\lemma{\textnormal{\emph{Baden}}}\Cendnote{\textnormal{Traut man der die Datierung \textcolor{blue}{Schnitzler}s mit »Juni 93«, so dürfte dieses Korrespondenzstück am Vortag des einzigen Besuchs \textcolor{blue}{Schnitzler}s in \textcolor{pink}{Baden} verfasst sein. Zu einem Treffen kam es laut \textcolor{blue}{Schnitzler}s \textcolor{green}{Tagebuch} dann aber nicht.}}}\label{K_L00225_1h} müſſen, ſtehts ja doch dafür. Vielleicht
                  \textcolor{blue}{\uline{Salten}}{}\ledrightnote{\textcolor{blue}{Felix Salten}} auch.\pend
           \pstart \spacefill\mbox{Hugo.}\pend{}\endnumbering\briefempfaengerindex{Schnitzler, Arthur@\textsc{Schnitzler, Arthur}!zzzHofmannsthal, Hugo von@\emph{von Hugo von Hofmannsthal}!1893-06-212@{{[}21.? 6. 1893{]}}|)be}\mylabel{h}  \normalsize

\doendnotes{C}
\bigskip
\vfill

\clearpage

\footnotesize

\lohead{\textsc{register}}

% Definiere theindex-Environment komplett neu ohne reledmac
\makeatletter
\renewenvironment{theindex}{%
  \section*{\indexname}%
  \setlength{\parindent}{0pt}%
  \setlength{\parskip}{0pt plus 0.3pt}%
  \let\item\@idxitem
}{%
  \clearpage
}
\makeatother

\IfFileExists{\jobname-pw.ind}{\input{\jobname-pw.ind}}{}

\end{document}

      