%% latex-korrekturansicht-vorspann.tex
%% Vorspann für die Korrekturansicht.
%% Lädt die gemeinsame Datei latex-vorspann.tex mit gesetztem Schalter.

\newif\ifkorrekturansicht
\korrekturansichttrue

\input{../tex-inputs/latex-vorspann}


               \section[Arthur Schnitzler an Richard Beer-Hofmann, 24. 6. 1895]{ Arthur Schnitzler an Richard Beer-Hofmann,
               24. 6. 1895}\nopagebreak\mylabel{v}\rehead{ }\normalsize\beginnumbering\briefempfaengerindex{Beer-Hofmann, Richard@\textsc{Beer-Hofmann, Richard}!zzzSchnitzler, Arthur@\emph{von Arthur Schnitzler}!1895-06-241@{24. 6. 1895}|(be} \toendnotes[C]{\smallbreak\pagebreak[2]} \Standort{YCGL, MSS 31.}
\physDesc{Brief, 2 Blätter, 8 Seiten, Umschlag
\newline{}Handschrift: 1) Bleistift, deutsche Kurrent\hspace{1em}2) schwarze Tinte, deutsche Kurrent (\noindent{}Umschlag)\hspace{1em}\newline{}Versand: 1) Stempel: »\nobreak{}\oindex{I., Innere Stadt@\textbf{I., Innere Stadt}, \emph{Bezirk (A.BZK)}|pwk}Wien 1/1, 24. 6. 95, 9–10 N\nobreak{}«.  2) Stempel: »\nobreak{}\oindex{Caslau@\textbf{Caslau}, \emph{Besiedelter Ort (A.BSO)}|pwk}Časlau, 25 6 95\nobreak{}«. }\buchAbdrucke{\weitereDrucke{Arthur Schnitzler, Richard Beer-Hofmann: \emph{Briefwechsel 1891–1931}. Hg. Konstanze Fliedl. Wien, Zürich: \emph{Europaverlag} 1992, S. 76–77.} }\toendnotes[C]{\smallbreak}\pstart{}{\pb}Herrn n. a. Lieutenant\pend{}\pstart{}\textsc{Dr. Richard Beer Hofmann}\pend{}\pstart{}im k.k. Landw Inf Regimt.\pend{}\pstart{}\textsc{\textcolor{pink}{Caslau}{}\ledrightnote{\textcolor{pink}{Caslau}} Nr 12}\pend{}{\bigskip}\pstart
           \noindent{}{\pb}Lieber Richard. Ich freue mich ſehr, daſs ich Sie noch in \textcolor{pink}{Wien}{}\ledrightnote{\textcolor{pink}{Wien}}{ }ſehen werde. – \textcolor{blue}{\textsc{Nobl}}{}\ledrightnote{\textcolor{blue}{Gabor Nobl}}{ }ſprach ich vorgeſtern, er hat, »angeregt« durch Ihr\introOben{}e\introOben{}
                  perſönlich\textcolor{gray}{e}{ }\substVorne{}\textsuperscript{\textcolor{gray}{Epiſödchen}}{\allowbreak}\substDazwischen{}Beka{\geminationn}tſchaft\substHinten{}, das \textcolor{green}{Kind}{}\ledrightnote{\textcolor{green}{Das Kind}} geleſen. Sie werden erſucht,
               ſich nächſtens auf {\pb}gefahrloſere Weiſe Leſer zu
               verſchaffen. – Habe heute Kopfweh, nach einer »\so{un}gemeinen« Landpartie die ich geſtern gemacht und die – entſchuldigen – in
               zwei miſerabeln Betten einer \textcolor{pink}{niederoeſterreichiſchen Stadt}{}\ledrightnote{→\textcolor{pink}{Klosterneuburg}} endete.\pend
           \pstart
           – Von der \textcolor{blue}{\textsc{Lou Salomé}}{}\ledrightnote{\textcolor{blue}{Lou Andreas-Salomé}} ha\textcolor{gray}{b} ich {\pb}noch i{\geminationm}er gar nichts gehört. Sie? – Wie wird es mit \textcolor{pink}{Kopenhagen}{}\ledrightnote{\textcolor{pink}{Kopenhagen}}{ }ſein? – Auch von \textcolor{blue}{\textsc{Paul}}{}\ledrightnote{\textcolor{blue}{Paul Goldmann}} iſt noch nichts Definitives
                  herauszubeko{\geminationm}en. – Ke{\geminationn}en Sie den \textcolor{green}{Briefwechſel \textsc{\textcolor{blue}{Lessing}{}\ledrightnote{\textcolor{blue}{Gotthold Ephraim Lessing}} – \textcolor{blue}{Eva König}{}\ledrightnote{\textcolor{blue}{Eva König}}}}{}\ledrightnote{→\textcolor{green}{Lessings Briefwechsel mit seiner Frau}}. Er iſt nicht ſehr intereſſant. Merkwürdig nur, wie ſie ſich i{\geminationm}er über Lotterienu{\geminationm}ern {\pb}berathen. – Leſen Sie den \textcolor{green}{\textsc{Candide}}{}\ledrightnote{\textcolor{green}{Candide oder der Optimismus}}. – Hingegen weniger
               nothwendig das »\textcolor{green}{Gelächter}{}\ledrightnote{\textcolor{green}{Gelächter}}« von \textcolor{blue}{Dörmann}{}\ledrightnote{\textcolor{blue}{Felix Dörmann}}. – Ich übe mich in erzählender Proſa: Schreibe
               »Hiſtorietten« – we{\geminationn}
               Sie wollen. Ja, den \textcolor{green}{alten Dichter}{}\ledrightnote{\textcolor{green}{Später Ruhm}} hab ich erheblich geſtrichen; ich find
               ihn aber noch i{\geminationm}er {\pb}etwas langweilig. Die ſtiliſtiſchen Schlampereien (»ich bin erschrocken«) ſind wohl
               alle draußen. –\pend
           \pstart
           – Für \textcolor{pink}{Iſchl}{}\ledrightnote{\textcolor{pink}{Bad Ischl}} hab ich literariſch gute Hoffnungen –
               möchte mein \textcolor{green}{Stück}{}\ledrightnote{→\textcolor{green}{Liebelei. Schauspiel in drei Akten}} gern
               beenden. – Von \textcolor{blue}{Dörmann}{}\ledrightnote{\textcolor{blue}{Felix Dörmann}}{ }ſoll dort ein Einakter
               gegeben werden, den er mir auch zum leſen gegeben hat u über den ich {\pb}eigentlich nicht ſprechen darf. (»Auch von Frl.
                  \textcolor{blue}{Albrecht}{}\ledrightnote{\textcolor{blue}{Albrecht}} müſſen wir einige freundliche Worte
               sagen.«) – Er heißt »\textcolor{green}{Der Eisbrecher}{}\ledrightnote{\textcolor{green}{Der Eisbrecher}}«. –
               Jo. –\pend
           \pstart
           – \textcolor{blue}{Hugo}{}\ledrightnote{\textcolor{blue}{Hugo von Hofmannsthal}} war geſtern in \textcolor{pink}{Wien}{}\ledrightnote{\textcolor{pink}{Wien}}, ich hab ihn verſäumt. – Heut bin ich braver Sohn und
               hole \textcolor{blue}{Mama}{}\ledrightnote{→\textcolor{blue}{Louise Schnitzler}} von der Bahn
               ab. –\pend
           \pstart
           – In dieſem Augenblick {\pb}ſitzt der \textcolor{blue}{Schreiber}{}\ledrightnote{→\textcolor{blue}{?? [Schreibkraft für Arthur Schnitzler]}} im Nebenzi{\geminationm}er u paginirt den \textcolor{green}{alten
                  Dichter}{}\ledrightnote{\textcolor{green}{Später Ruhm}}.\pend
           \pstart
           Leben Sie wohl und nehmen Sie von Ihrer schönen Arbeitsſehnſucht recht viel ins Civil
               herüber. So kö{\geminationn}ten Sie z. B. den \textcolor{green}{Götterliebling}{}\ledrightnote{\textcolor{green}{Der Tod Georgs}} zu Ende ſchreiben. Finden Sie nicht? – Viele {\pb}herzliche Grüße\pend
           \pstart Ihr \spacefill\mbox{Arthur}\pend{}\pstart
           24/6 95.\pend
           \endnumbering\briefempfaengerindex{Beer-Hofmann, Richard@\textsc{Beer-Hofmann, Richard}!zzzSchnitzler, Arthur@\emph{von Arthur Schnitzler}!1895-06-241@{24. 6. 1895}|)be}\mylabel{h}  \normalsize

\doendnotes{C}
\bigskip
\vfill

\clearpage

\footnotesize

\lohead{\textsc{register}}

% Definiere theindex-Environment komplett neu ohne reledmac
\makeatletter
\renewenvironment{theindex}{%
  \section*{\indexname}%
  \setlength{\parindent}{0pt}%
  \setlength{\parskip}{0pt plus 0.3pt}%
  \let\item\@idxitem
}{%
  \clearpage
}
\makeatother

\IfFileExists{\jobname-pw.ind}{\input{\jobname-pw.ind}}{}

\end{document}

      