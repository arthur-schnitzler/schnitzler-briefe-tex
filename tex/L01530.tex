%% latex-korrekturansicht-vorspann.tex
%% Vorspann für die Korrekturansicht.
%% Lädt die gemeinsame Datei latex-vorspann.tex mit gesetztem Schalter.

\newif\ifkorrekturansicht
\korrekturansichttrue

\input{../tex-inputs/latex-vorspann}


               \section[Hugo und Gerty von Hofmannsthal an Arthur Schnitzler, {[}16. 7. 1905?{]}]{ Hugo und Gerty von Hofmannsthal an Arthur Schnitzler,
               {[}16. 7. 1905?{]}}\nopagebreak\mylabel{v}\rehead{ }\normalsize\beginnumbering\briefempfaengerindex{Schnitzler, Arthur@\textsc{Schnitzler, Arthur}!zzzHofmannsthal, Gertrude von@\emph{von Gertrude von Hofmannsthal}!1905-07-162@{{[}16. 7. 1905?{]}}|(be}\briefempfaengerindex{Schnitzler, Arthur@\textsc{Schnitzler, Arthur}!zzzHofmannsthal, Hugo von@\emph{von Hugo von Hofmannsthal}!1905-07-162@{{[}16. 7. 1905?{]}}|(be} \toendnotes[C]{\smallbreak\pagebreak[2]} \buchAlsQuelle{Hugo von Hofmannsthal, Arthur Schnitzler: \emph{Briefwechsel}. Hg. Therese Nickl und Heinrich Schnitzler. Frankfurt am Main: \emph{S. Fischer} 1964, S. 214.}\toendnotes[C]{\smallbreak}\pstart
           \noindent{}{\pb}{[}Ansichtskarte{]}\hfill {[}\textcolor{pink}{\label{K_L01530_1v}\edtext{Kremsmünster}{\lemma{\textnormal{\emph{Kremsmünster}}}\Cendnote{\textnormal{Dieser Besuch fand, als
                              Tagesausflug von \textcolor{pink}{Wels}, am
                                 16. 7. 1905 statt.}}}\label{K_L01530_1h}}{}\ledrightnote{\textcolor{pink}{Kremsmünster}}, 1905{]}\pend
           \pstart
           Einen solchen \label{K_L01530_2v}\edtext{Gang}{\lemma{\textnormal{\emph{Gang}}}\Cendnote{\textnormal{Gemeint dürfte der Arkadengang des \textcolor{pink}{Stiftes Kremsmünster} sein.}}}\label{K_L01530_2h} wünschen Ihnen
                  \spacefill\mbox{Hugo}{ }\spacefill\mbox{{[}hs. G. Hofmannsthal:{]} – Gerty.}\pend
           \endnumbering\briefempfaengerindex{Schnitzler, Arthur@\textsc{Schnitzler, Arthur}!zzzHofmannsthal, Gertrude von@\emph{von Gertrude von Hofmannsthal}!1905-07-162@{{[}16. 7. 1905?{]}}|)be}\briefempfaengerindex{Schnitzler, Arthur@\textsc{Schnitzler, Arthur}!zzzHofmannsthal, Hugo von@\emph{von Hugo von Hofmannsthal}!1905-07-162@{{[}16. 7. 1905?{]}}|)be}\mylabel{h}  \normalsize

\doendnotes{C}
\bigskip
\vfill

\clearpage

\footnotesize

\lohead{\textsc{register}}

% Definiere theindex-Environment komplett neu ohne reledmac
\makeatletter
\renewenvironment{theindex}{%
  \section*{\indexname}%
  \setlength{\parindent}{0pt}%
  \setlength{\parskip}{0pt plus 0.3pt}%
  \let\item\@idxitem
}{%
  \clearpage
}
\makeatother

\IfFileExists{\jobname-pw.ind}{\input{\jobname-pw.ind}}{}

\end{document}

      