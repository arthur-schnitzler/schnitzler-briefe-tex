%% latex-korrekturansicht-vorspann.tex
%% Vorspann für die Korrekturansicht.
%% Lädt die gemeinsame Datei latex-vorspann.tex mit gesetztem Schalter.

\newif\ifkorrekturansicht
\korrekturansichttrue

\input{../tex-inputs/latex-vorspann}


               \section[Arthur Schnitzler an Hermann Bahr, 10. 11. {[}1903{]}]{ Arthur Schnitzler an Hermann Bahr, 10. 11. {[}1903{]}}\nopagebreak\mylabel{v}\rehead{ }\normalsize\beginnumbering\briefempfaengerindex{Bahr, Hermann@\textsc{Bahr, Hermann}!zzzSchnitzler, Arthur@\emph{von Arthur Schnitzler}!1903-11-102@{10. 11. {[}1903{]}}|(be} \toendnotes[C]{\smallbreak\pagebreak[2]} \Standort{TMW, HS AM 60156 Ba.}
\physDesc{Postkarte
\newline{}Handschrift: Bleistift, deutsche Kurrent\newline{}Versand: 1) Stempel: »\nobreak{}Wien, 10–11 V\nobreak{}«.  2) Stempel: »\nobreak{}\oindex{XIII., Hietzing@\textbf{XIII., Hietzing}, \emph{Bezirk (A.BZK)}|pwk}Wien 13/7, 10. 11. 0{[}3{]}\nobreak{}«. \newline{}Ordnung: Lochung }\buchAbdrucke{\weitereDrucke{1) \emph{[1902?], Abschrift.} In: Arthur Schnitzler: \emph{The Letters of Arthur Schnitzler to Hermann Bahr}. Edited, annotated, and with an introduction, by Donald G.
                        Daviau. Chapel Hill: \emph{The University of North Carolina Press} 1978, S. 77 (University of North Carolina studies in the Germanic languages
                        and literatures, 89).} \weitereDrucke{2) Hermann Bahr, Arthur Schnitzler: \emph{Briefwechsel, Aufzeichnungen, Dokumente (1891–1931)}. Hg. Kurt Ifkovits und Martin Anton Müller. Göttingen: \emph{Wallstein} 2018, S. 279.} }\toendnotes[C]{\smallbreak}\pstart{}{\pb}\textsc{Herrn Hermann Bahr}\pend{}\pstart{}\textcolor{pink}{Wien-Ob. St. Veit}{}\ledrightnote{\textcolor{pink}{Ober Sankt Veit}}\pend{}\pstart{}\textcolor{pink}{Veitliſſengaſſe 15}{}\ledrightnote{\textcolor{pink}{Veitlissengasse}}\pend{}{\bigskip}\pstart
           \noindent{}{\pb}lieber Hermann,
               du haſt dich für den \label{K_L01339_1v}\edtext{\textcolor{green}{\uline{Goethe Zelter} Briefwechſel}{}\ledrightnote{\textcolor{green}{Briefwechsel zwischen Goethe und Zelter}}}{\lemma{\textnormal{\emph{Goethe … Briefwechſel}}}\Cendnote{\textnormal{nicht unter den aus \textcolor{blue}{Bahr}s Besitz überlieferten Büchern}}}\label{K_L01339_1h} intereſſirt. Im
                  \textsc{Catalog XXXIV}{ }\textcolor{brown}{ſüddeutſches Antiquariat}{}\ledrightnote{\textcolor{brown}{Süddeutsches Antiquariat}}, \textcolor{pink}{\uline{München} Gallerieſtraße 20}{}\ledrightnote{\textcolor{pink}{Galeriestraße}}, unter Nr 699 iſt
               ein Exemplar, 6 Bände um 40 Mark angekündigt. Herzlichſt dein{\\}\spacefill\mbox{Arth Sch}\pend
           \endnumbering\briefempfaengerindex{Bahr, Hermann@\textsc{Bahr, Hermann}!zzzSchnitzler, Arthur@\emph{von Arthur Schnitzler}!1903-11-102@{10. 11. {[}1903{]}}|)be}\mylabel{h}  \normalsize

\doendnotes{C}
\bigskip
\vfill

\clearpage

\footnotesize

\lohead{\textsc{register}}

% Definiere theindex-Environment komplett neu ohne reledmac
\makeatletter
\renewenvironment{theindex}{%
  \section*{\indexname}%
  \setlength{\parindent}{0pt}%
  \setlength{\parskip}{0pt plus 0.3pt}%
  \let\item\@idxitem
}{%
  \clearpage
}
\makeatother

\IfFileExists{\jobname-pw.ind}{\input{\jobname-pw.ind}}{}

\end{document}

      