%% latex-korrekturansicht-vorspann.tex
%% Vorspann für die Korrekturansicht.
%% Lädt die gemeinsame Datei latex-vorspann.tex mit gesetztem Schalter.

\newif\ifkorrekturansicht
\korrekturansichttrue

\input{../tex-inputs/latex-vorspann}


               \section[Albert Ehrenstein an Arthur Schnitzler, 18. 7. 1910]{ Albert Ehrenstein an Arthur Schnitzler, 18. 7. 1910}\nopagebreak\mylabel{v}\rehead{ }\normalsize\beginnumbering\briefempfaengerindex{Schnitzler, Arthur@\textsc{Schnitzler, Arthur}!zzzEhrenstein, Albert@\emph{von Albert Ehrenstein}!1910-07-181@{18. 7. 1910}|(be} \toendnotes[C]{\smallbreak\pagebreak[2]} \Standort{TMW, HS Schn 2/7/1.}
\physDesc{Brief, 1 Blatt, 2 Seiten
\newline{}Handschrift: schwarze Tinte, deutsche Kurrent
\newline{}Schnitzler: mit Bleistift beschrieben: »\textsc{Ehrenst\textcolor{gray}{ein}}« }\toendnotes[C]{\smallbreak}\pstart
           {\pb}\textcolor{pink}{\textsc{Vradist bei Holics}}{}\ledrightnote{\textcolor{pink}{Vrádište}}.\hfill \textsc{18. Juli 1910}.\pend
           \pstart{}\textsc{Hochverehrter Herr Doktor,}\pend\pstart
           in der Meinung, meiner Unluſt zu jeden Studium lägen äußere Umſtände zugrunde, bin
               ich in die \textcolor{pink}{Slowakei}{}\ledrightnote{\textcolor{pink}{Slowakei}} gefahren, in eine wald- und
               reizloſe Gegend, in der auch die Menſchen nur Land ſind, bewegliche Erde, vermodernde
               Pflanzen. Aber mit dem Lernen geht es auch hier nicht beſonders, und ſo dürfte ich
                  Anfang September wieder in \textcolor{pink}{Wien}{}\ledrightnote{\textcolor{pink}{Wien}}{ }ſein. – Wenn das nicht gerade eine Zeit ſein
               ſollte, wo Sie durch Proben zu sehr in Anſpruch genommen ſind, möchte ich Ihnen gerne
               meine Aufwartung machen. Sehr angenehm wäre es mir aber, falls Sie, hochverehrter
               Herr Doktor, {\pb}mir
               früher, wenn einmal Ihre Möbelwanderungen – Völkerwanderungen ſind übrigens
               mindeſtens ebenſo unangenehm – zu einem Abschluſſe gekommen ſein werden, etwas über
               meine Sachen zu ſagen die Güte hätten.\pend
           \pstart
           Ich glaube nämlich nicht, daß hierbei auch bei mir ein inneres Manco vorliegt, was
                  \label{K_L01948_1v}\edtext{\textcolor{blue}{Gumppenberg}{}\ledrightnote{\textcolor{blue}{Hanns von Gumppenberg}} andeutete}{\lemma{\textnormal{\emph{Gumppenberg andeutete}}}\Cendnote{\textnormal{eine im unmittelbaren Verkehr getätigte Aussage}}}\label{K_L01948_1h}, indem
               er dem »\textcolor{green}{Grafen Cilli}{}\ledrightnote{\textcolor{green}{Graf Cilli}}« eine kunſtloſe, rohe,
               gefliſſentlich derbe Sprache vorwarf, der »\textcolor{brown}{März}{}\ledrightnote{\textcolor{brown}{März}}«,
               indem er rein artiſtiſche Gebarung meinerſeits als Hindernis einer Annahme meiner
               Arbeiten \label{K_L01948_2v}\edtext{deklarierte}{\lemma{\textnormal{\emph{deklarierte}}}\Cendnote{\textnormal{die Ablehnung gleichfalls kein publiziertes
                  Urteil}}}\label{K_L01948_2h}. –\pend
           \pstart
           Hochachtungsvoll{\\[\baselineskip]}Ihr ergebenſter{\\[\baselineskip]}\spacefill\mbox{Albert Ehrenstein.}\pend
           \leftskip=0em{}\endnumbering\briefempfaengerindex{Schnitzler, Arthur@\textsc{Schnitzler, Arthur}!zzzEhrenstein, Albert@\emph{von Albert Ehrenstein}!1910-07-181@{18. 7. 1910}|)be}\mylabel{h}  \normalsize

\doendnotes{C}
\bigskip
\vfill

\clearpage

\footnotesize

\lohead{\textsc{register}}

% Definiere theindex-Environment komplett neu ohne reledmac
\makeatletter
\renewenvironment{theindex}{%
  \section*{\indexname}%
  \setlength{\parindent}{0pt}%
  \setlength{\parskip}{0pt plus 0.3pt}%
  \let\item\@idxitem
}{%
  \clearpage
}
\makeatother

\IfFileExists{\jobname-pw.ind}{\input{\jobname-pw.ind}}{}

\end{document}

      