%% latex-korrekturansicht-vorspann.tex
%% Vorspann für die Korrekturansicht.
%% Lädt die gemeinsame Datei latex-vorspann.tex mit gesetztem Schalter.

\newif\ifkorrekturansicht
\korrekturansichttrue

\input{../tex-inputs/latex-vorspann}


               \section[Hermann Bahr an Arthur Schnitzler, 5. 7. 1901]{ Hermann Bahr an Arthur Schnitzler, 5. 7. 1901}\nopagebreak\mylabel{v}\rehead{ }\normalsize\beginnumbering\briefempfaengerindex{Schnitzler, Arthur@\textsc{Schnitzler, Arthur}!zzzBahr, Hermann@\emph{von Hermann Bahr}!1901-07-051@{5. 7. 1901}|(be} \toendnotes[C]{\smallbreak\pagebreak[2]} \Standort{CUL, Schnitzler, B 5b.}
\physDesc{Brief, 1 Blatt, 4 Seiten
\newline{}Handschrift: schwarze Tinte, deutsche Kurrent
\newline{}Schnitzler: mit Bleistift die Jahreszahl »901« ergänzt \newline{}Ordnung: mit Bleistift von unbekannter Hand nummeriert:
                                    »78« }\buchAbdrucke{\weitereDrucke{Hermann Bahr, Arthur Schnitzler: \emph{Briefwechsel, Aufzeichnungen, Dokumente (1891–1931)}. Hg. Kurt Ifkovits und Martin Anton Müller. Göttingen: \emph{Wallstein} 2018, S. 212–213.} }\toendnotes[C]{\smallbreak}\pstart
           \raggedleft{}{\pb}\substVorne{}\textsuperscript{7}\substDazwischen{}5\substHinten{}/7\pend
           \pstart\center{}Lieber Arthur!\pend\pstart
           Ich danke Dir herzlich für Deinen lieben Brief. Ich habe neulich mit \textcolor{blue}{Hugo}{}\ledrightnote{\textcolor{blue}{Hugo von Hofmannsthal}} davon geſprochen, wie es mich freut, zu Dir endlich ein
               aufrichtiges und gutes Verhältnis gefunden zu haben und zu empfinden, daß \introOben{}es\introOben{} wohl nicht mehr geſtört werden kann, mögen unſere Meinungen
               immerhin auch künftig noch manchmal auseinandergehen.\pend
           \pstart
           {\pb}\textcolor{blue}{Hugo}{}\ledrightnote{\textcolor{blue}{Hugo von Hofmannsthal}} iſt ſehr ſtolz, weil er das Gefühl hat, in
               dieſer Sache von jeher geſcheiter geweſen zu ſein, als wir es Jahre lang waren.\pend
           \pstart
           Für \textcolor{blue}{Pötzl}{}\ledrightnote{\textcolor{blue}{Eduard Pötzl}} kann ich, ſo unerfreulich er ſich gegen
               mich, mit anonymen Briefen und auf Hintertreppen operierend, fortgeſetzt benimmt,
                  ein{[}e{]}{ }ſtille Bewunderung nicht los werden, weil er doch
               das vollendet{\pb}ſte Exemplar des biederen \textcolor{pink}{Wieners}{}\ledrightnote{\textcolor{pink}{Wien}} iſt, und mir immer nur leid thut, daß ihn \textcolor{blue}{Flaubert}{}\ledrightnote{\textcolor{blue}{Gustave Flaubert}} nicht gekannt hat, der ein wahres
               Freudengeheul über ihn ausgestoßen hätte. »\label{K_L01143_1v}\edtext{Den Arier}{\lemma{\textnormal{\emph{Den Arier}}}\Cendnote{\textnormal{\textcolor{blue}{Pötzl} behandelte in seinen Texten häufig \textcolor{pink}{Wien}er Typen.}}}\label{K_L01143_1h}« müßte einmal Jemand ſchildern
               und müßte einmal die andere Seite der »armen Spielleute« zeigen, den gemütlichen
                  \label{K_L01143_2v}\edtext{Naderer}{\lemma{\textnormal{\emph{Naderer}}}\Cendnote{\textnormal{österreichisch: Verräter, Petze}}}\label{K_L01143_2h}, der eigentlich der
               Grundtypus des \textcolor{pink}{Öſtreichers}{}\ledrightnote{\textcolor{pink}{Österreich}} zu ſein ſcheint, was
               irgendwie {\pb}ſehr tief mit dem Katholicismus
                  zuſammen\introOben{}zu\introOben{}hängen ſcheint – worüber \textcolor{blue}{Poldi}{}\ledrightnote{\textcolor{blue}{Leopold von Andrian-Werburg}} und \textcolor{blue}{Hugo}{}\ledrightnote{\textcolor{blue}{Hugo von Hofmannsthal}} freilich
               Zeter und Mordio ſchreien würden. \textcolor{blue}{Pötzl}{}\ledrightnote{\textcolor{blue}{Eduard Pötzl}} oder der
               Herr \label{LL238-1v}\textcolor{blue}{Davis}{}\ledrightnote{\textcolor{blue}{Gustav Davis}} von der »\textcolor{brown}{Reichswehr}{}\ledrightnote{\textcolor{brown}{Reichswehr}}«\label{LL238-1h} oder der Ton des \label{K_L01143_3v}\edtext{\textcolor{brown}{Kikeriki}{}\ledrightnote{\textcolor{brown}{Kikeriki}}}{\lemma{\textnormal{\emph{Kikeriki}}}\Cendnote{\textnormal{antisemitische
                  Satirezeitschrift}}}\label{K_L01143_3h} – das ſind lauter Sachen, die an den Hof \textcolor{blue}{Philipps}{}\ledrightnote{\textcolor{blue}{Phillipp II. von Spanien}} gehören und die ich mir großartig von \textcolor{blue}{\textsc{Velasquez}}{}\ledrightnote{\textcolor{blue}{Diego Rodríguez de Silva y Velázquez}} gemalt denken könnte.\pend
           \pstart
           Einen guten Sommer wünſcht Dir{\\[\baselineskip]}herzlichſt{\\[\baselineskip]}Dein{\\[\baselineskip]}\spacefill\mbox{Hermann}\pend
           \leftskip=0em{}\endnumbering\briefempfaengerindex{Schnitzler, Arthur@\textsc{Schnitzler, Arthur}!zzzBahr, Hermann@\emph{von Hermann Bahr}!1901-07-051@{5. 7. 1901}|)be}\mylabel{h}  \normalsize

\doendnotes{C}
\bigskip
\vfill

\clearpage

\footnotesize

\lohead{\textsc{register}}

% Definiere theindex-Environment komplett neu ohne reledmac
\makeatletter
\renewenvironment{theindex}{%
  \section*{\indexname}%
  \setlength{\parindent}{0pt}%
  \setlength{\parskip}{0pt plus 0.3pt}%
  \let\item\@idxitem
}{%
  \clearpage
}
\makeatother

\IfFileExists{\jobname-pw.ind}{\input{\jobname-pw.ind}}{}

\end{document}

      