%% latex-korrekturansicht-vorspann.tex
%% Vorspann für die Korrekturansicht.
%% Lädt die gemeinsame Datei latex-vorspann.tex mit gesetztem Schalter.

\newif\ifkorrekturansicht
\korrekturansichttrue

\input{../tex-inputs/latex-vorspann}


               \section[Hugo von Hofmannsthal an Arthur Schnitzler, 10. 7. 1904]{ Hugo von Hofmannsthal an Arthur Schnitzler, 10. 7. 1904}\nopagebreak\mylabel{v}\rehead{ }\normalsize\beginnumbering\briefempfaengerindex{Schnitzler, Arthur@\textsc{Schnitzler, Arthur}!zzzHofmannsthal, Hugo von@\emph{von Hugo von Hofmannsthal}!1904-07-101@{10. 7. 1904}|(be} \toendnotes[C]{\smallbreak\pagebreak[2]} \Standort{CUL, Schnitzler, B 43.}
\physDesc{Postkarte
\newline{}Handschrift: schwarze Tinte, deutsche Kurrent\newline{}Versand: 1) Stempel: »\nobreak{}\oindex{Rodaun@\textbf{Rodaun}, \emph{Teil eines besiedelten Ortes (A.BSOX)}|pwk}Rodaun, 10. 7. 04\nobreak{}«.  2) Stempel: »\nobreak{}\oindex{XVIII., Waehring@\textbf{XVIII., Währing}, \emph{Bezirk (A.BZK)}|pwk}18/1 Wien, 11. 7. 04, 8.V, Bestellt\nobreak{}«. 
\newline{}Schnitzler: mit Bleistift datiert: »11. 7 904« \newline{}Ordnung: 1) mit Bleistift von unbekannter Hand nummeriert: »\strikeout{237}« 2) mit Bleistift von unbekannter Hand nummeriert:
                                    »228«}\buchAbdrucke{\weitereDrucke{Hugo von Hofmannsthal, Arthur Schnitzler: \emph{Briefwechsel}. Hg. Therese Nickl und Heinrich Schnitzler. Frankfurt am Main: \emph{S. Fischer} 1964, S. 191.} }\toendnotes[C]{\smallbreak}\pstart{}{\pb}\textsc{Herrn D\textsuperscript{r} Arthur Schnitzler}\pend{}\pstart{}\textcolor{pink}{\textsc{Wien}}{}\ledrightnote{\textcolor{pink}{Wien}}\pend{}\pstart{}\textcolor{pink}{\textsc{XVIII. Spöttelgasse 7}.}{}\ledrightnote{\textcolor{pink}{Edmund-Weiß-Gasse}}\pend{}{\bigskip}\pstart
           \noindent{}{\pb}Vielleicht »\textsc{\label{K_L01416_1v}\edtext{chasse libre}{\lemma{\textnormal{\emph{chasse libre}}}\Cendnote{\textnormal{französisch wörtlich: freie Jagd. \textcolor{blue}{Schnitzler} arbeitete für eine französische
                     Aufführung an \emph{\textcolor{green}{Freiwild}}, die aber nicht
                     realisiert worden sein dürfte.}}}\label{K_L01416_1h}}«, das giebt den Begriff treu wieder und klingt nicht ſchlecht.\hspace*{1.5em}Ich denke Dienstag oder
                  Mittwoch{ }abends zu \label{K_L01416_2v}\edtext{fahren}{\lemma{\textnormal{\emph{fahren}}}\Cendnote{\textnormal{Der genaue Abreisezeitpunkt konnte nicht
                  ermittelt werden. Von 15. bis 29. 7. 1904 ist er als
                  erste Station seines Sommerurlaubs in \textcolor{pink}{Bad Fusch}.
                  Er und \textcolor{blue}{Schnitzler} sehen sich erst am
                     3. 9. 1904 wieder.}}}\label{K_L01416_2h}.\pend
           \pstart
           So ſehen wir uns wohl nicht wieder? Aber im Herbſt! Ich hoffe ſehr.\pend
           \pstart
           Von Herzen{\\[\baselineskip]}\spacefill\mbox{Hugo.}\pend
           \leftskip=0em{}\endnumbering\briefempfaengerindex{Schnitzler, Arthur@\textsc{Schnitzler, Arthur}!zzzHofmannsthal, Hugo von@\emph{von Hugo von Hofmannsthal}!1904-07-101@{10. 7. 1904}|)be}\mylabel{h}  \normalsize

\doendnotes{C}
\bigskip
\vfill

\clearpage

\footnotesize

\lohead{\textsc{register}}

% Definiere theindex-Environment komplett neu ohne reledmac
\makeatletter
\renewenvironment{theindex}{%
  \section*{\indexname}%
  \setlength{\parindent}{0pt}%
  \setlength{\parskip}{0pt plus 0.3pt}%
  \let\item\@idxitem
}{%
  \clearpage
}
\makeatother

\IfFileExists{\jobname-pw.ind}{\input{\jobname-pw.ind}}{}

\end{document}

      