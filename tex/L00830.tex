%% latex-korrekturansicht-vorspann.tex
%% Vorspann für die Korrekturansicht.
%% Lädt die gemeinsame Datei latex-vorspann.tex mit gesetztem Schalter.

\newif\ifkorrekturansicht
\korrekturansichttrue

\input{../tex-inputs/latex-vorspann}


               \section[Arthur Schnitzler an Hugo von Hofmannsthal, 5. 8. 1898]{ Arthur Schnitzler an Hugo von Hofmannsthal, 5. 8. 1898}\nopagebreak\mylabel{v}\rehead{ }\normalsize\beginnumbering\briefempfaengerindex{Hofmannsthal, Hugo von@\textsc{Hofmannsthal, Hugo von}!zzzSchnitzler, Arthur@\emph{von Arthur Schnitzler}!1898-08-051@{5. 8. 1898}|(be} \toendnotes[C]{\smallbreak\pagebreak[2]} \Standort{FDH, Hs-30885,73.}
\physDesc{Brief, 1 Blatt, 4 Seiten
\newline{}Handschrift: Bleistift, deutsche Kurrent}\buchAbdrucke{\weitereDrucke{Hugo von Hofmannsthal, Arthur Schnitzler: \emph{Briefwechsel}. Hg. Therese Nickl und Heinrich Schnitzler. Frankfurt am Main: \emph{S. Fischer} 1964, S. 108–109.} }\toendnotes[C]{\smallbreak}\pstart
           \raggedleft{}{\pb}\textcolor{pink}{Tegernſee}{}\ledrightnote{\textcolor{pink}{Tegernsee}}{ }5. 8. 98\pend
           \pstart
           Mein lieber Hugo, die Radtour, die wir vorhaben, iſt \introOben{}(\introOben{}ungefähr\introOben{})\introOben{}{ }\textcolor{pink}{\textsc{Basel}}{}\ledrightnote{\textcolor{pink}{Basel}}–\textcolor{pink}{\textsc{Biel}}{}\ledrightnote{\textcolor{pink}{Biel}} bis hinunter zum \textcolor{pink}{Genferſee}{}\ledrightnote{\textcolor{pink}{Genfer See}}. Ob wir nur am
                        \textcolor{pink}{Genferſee}{}\ledrightnote{\textcolor{pink}{Genfer See}} bleiben oder da{\geminationn} ins \textcolor{pink}{italieniſche}{}\ledrightnote{\textcolor{pink}{Italien}} hinüber fahren, können wir uns an Ort u Stelle überlegen,
                    jedenfalls ſteht die Sache heute ſo, dſs ich nicht nur bis zum 20.
                    Zeit habe, ſondern bis Ende Auguſt mit Ihnen bleiben kann und auch
                    Luſt habe {\pb}mich an irgd einen See zu ſetzen. Dazu iſt
                    ja auch \textcolor{blue}{Richard}{}\ledrightnote{\textcolor{blue}{Richard Beer-Hofmann}} vielleicht zu haben, es
                    könnte ſehr ſchön ſein.\pend
           \pstart
           Nun zu den Modalitäten unſrer Begegnung. \uline{Ich} bin
                    am 12.{ }\substVorne{}\textsuperscript{a}\substDazwischen{}i\substHinten{}n \textcolor{pink}{München}{}\ledrightnote{\textcolor{pink}{München}} (aus verſchiedenen Gründen
                        \uline{muſs} ich nach \textcolor{pink}{München}{}\ledrightnote{\textcolor{pink}{München}}, u \uline{ka{\geminationn}
                        nicht} nach \textcolor{pink}{Innsbruck}{}\ledrightnote{\textcolor{pink}{Innsbruck}}) und ſchlage
                    Ihnen daher vor: treffen wir uns entweder am 12.{ }{\pb}in \textcolor{pink}{München}{}\ledrightnote{\textcolor{pink}{München}} oder,
                    was Ihnen wahrſcheinlich bequemer ſein wird, \uuline{am
                            13. in \textcolor{pink}{Baſel}{}\ledrightnote{\textcolor{pink}{Basel}}}. (Sie führen da{\geminationn} direct \textcolor{pink}{Wien}{}\ledrightnote{\textcolor{pink}{Wien}}–\introOben{}\textcolor{pink}{I{\geminationn}sbruck}{}\ledrightnote{\textcolor{pink}{Innsbruck}}–\introOben{}\textcolor{pink}{Baſel}{}\ledrightnote{\textcolor{pink}{Basel}}, \label{T_L00830_1v}\edtext{(}{\lemma{\textnormal{\emph{(}}}\Cendnote{\textnormal{In der
                        Handschrift setzt Schnitzler eine eckige Klammer für die öffnende und
                        schließende Klammer innerhalb der Klammer. Auf die Wiedergabe wurde, wegen
                        der möglichen Verwechslungen mit editorischen Zeichen, verzichtet.}}}\label{T_L00830_1h}\textcolor{pink}{München}{}\ledrightnote{\textcolor{pink}{München}} iſt ein kleiner Umweg für Sie)). Ich
                    denke, ſo iſt die Sache am einfachſten. Hier bin ich noch bis
                        Dinſtag; jedenfalls bitte \uline{antworten
                        Sie mir gleich}. Ob wir uns ſchon in \textcolor{pink}{Innsbruck}{}\ledrightnote{\textcolor{pink}{Innsbruck}} oder erſt {\pb}in \textcolor{pink}{Baſel}{}\ledrightnote{\textcolor{pink}{Basel}} treffen, iſt bei dem Weſen unſrer Tour egal.\pend
           \pstart
           Hoffentlich hat dieſe Correſpondenz ſchon endgiltige Bedeutung; ich freu mich
                    rieſig auf die Reiſe, u. beſonders, dſs auch meine Zeit verhältnismäßg
                    unbeſchränkt iſt. Alſo nochmals bitte \uline{gleich}
                    Antwort. Von Herzen Ihr\hspace*{1.5em}\spacefill\mbox{Arthur}\pend
           \pstart
           \noindent{}\label{T_L00830_2v}\edtext{\textcolor{blue}{Richard}{}\ledrightnote{\textcolor{blue}{Richard Beer-Hofmann}} hat \textcolor{blue}{Schwarzk.}{}\ledrightnote{\textcolor{blue}{Gustav Schwarzkopf}} u mir in \textcolor{pink}{Salzburg}{}\ledrightnote{\textcolor{pink}{Salzburg}}{ }ſein \textcolor{green}{3. Capitel}{}\ledrightnote{→\textcolor{green}{Der Tod Georgs}}{ }\label{K_L00830_1v}\edtext{vorgeleſen}{\lemma{\textnormal{\emph{vorgeleſen}}}\Cendnote{\textnormal{siehe A. S.: \emph{Tagebuch}, 28. 7. 1898}}}\label{K_L00830_1h}. Es iſt außerordentlich.}{\lemma{\textnormal{\emph{Richard … außerordentlich.}}}\Cendnote{\textnormal{am
                            unteren Blattrand auf dem Kopf}}}\label{T_L00830_2h}\pend
           \endnumbering\briefempfaengerindex{Hofmannsthal, Hugo von@\textsc{Hofmannsthal, Hugo von}!zzzSchnitzler, Arthur@\emph{von Arthur Schnitzler}!1898-08-051@{5. 8. 1898}|)be}\mylabel{h}  \normalsize

\doendnotes{C}
\bigskip
\vfill

\clearpage

\footnotesize

\lohead{\textsc{register}}

% Definiere theindex-Environment komplett neu ohne reledmac
\makeatletter
\renewenvironment{theindex}{%
  \section*{\indexname}%
  \setlength{\parindent}{0pt}%
  \setlength{\parskip}{0pt plus 0.3pt}%
  \let\item\@idxitem
}{%
  \clearpage
}
\makeatother

\IfFileExists{\jobname-pw.ind}{\input{\jobname-pw.ind}}{}

\end{document}

      