%% latex-korrekturansicht-vorspann.tex
%% Vorspann für die Korrekturansicht.
%% Lädt die gemeinsame Datei latex-vorspann.tex mit gesetztem Schalter.

\newif\ifkorrekturansicht
\korrekturansichttrue

\input{../tex-inputs/latex-vorspann}


               \section[Hugo von Hofmannsthal an Arthur Schnitzler, 4. 2. 1908]{ Hugo von Hofmannsthal an Arthur Schnitzler, 4. 2. 1908}\nopagebreak\mylabel{v}\rehead{ }\normalsize\beginnumbering\briefempfaengerindex{Schnitzler, Arthur@\textsc{Schnitzler, Arthur}!zzzHofmannsthal, Hugo von@\emph{von Hugo von Hofmannsthal}!1908-02-041@{4. 2. 1908}|(be} \toendnotes[C]{\smallbreak\pagebreak[2]} \Standort{CUL, Schnitzler, B 43.}
\physDesc{Postkarte
\newline{}Handschrift: Bleistift, lateinische Kurrent\newline{}Versand: Stempel: »\nobreak{}\oindex{Rodaun@\textbf{Rodaun}, \emph{Teil eines besiedelten Ortes (A.BSOX)}|pwk}Rodaun, 4 2 08, 5 N\nobreak{}«.  
\newline{}Schnitzler: mit Bleistift datiert: »4/2 908« \newline{}Ordnung: 1) mit Bleistift von unbekannter Hand nummeriert: »\strikeout{296}« 2) mit Bleistift von unbekannter Hand nummeriert: »294«}\buchAbdrucke{\weitereDrucke{Hugo von Hofmannsthal, Arthur Schnitzler: \emph{Briefwechsel}. Hg. Therese Nickl und Heinrich Schnitzler. Frankfurt am Main: \emph{S. Fischer} 1964, S. 236.} }\pstart{}{\pb}Herrn\pend{}\pstart{}D\textsuperscript{r} A Schnitzler\pend{}\pstart{}\textcolor{pink}{Wien}{}\ledrightnote{\textcolor{pink}{Wien}}\pend{}\pstart{}\textcolor{pink}{XVIII Spöttelgasse 7}{}\ledrightnote{\textcolor{pink}{Edmund-Weiß-Gasse}}\pend{}\pstart{}neben \textcolor{pink}{Türkenschanzstrasse}{}\ledrightnote{\textcolor{pink}{Türkenschanzstrasse}}\pend{}{\bigskip}\pstart
           \noindent{}{\pb}Depeschieren Sie mir \uline{gleich} falls Sie nicht am \textcolor{pink}{Se{\geminationm}ering}{}\ledrightnote{\textcolor{pink}{Semmering}} sind denn ich fahre vor \uline{allem} Ihretwegen Freitag hinauf.\pend
           \pstart \spacefill\mbox{Hugo.}\pend{}\endnumbering\briefempfaengerindex{Schnitzler, Arthur@\textsc{Schnitzler, Arthur}!zzzHofmannsthal, Hugo von@\emph{von Hugo von Hofmannsthal}!1908-02-041@{4. 2. 1908}|)be}\mylabel{h}  \normalsize

\doendnotes{C}
\bigskip
\vfill

\clearpage

\footnotesize

\lohead{\textsc{register}}

% Definiere theindex-Environment komplett neu ohne reledmac
\makeatletter
\renewenvironment{theindex}{%
  \section*{\indexname}%
  \setlength{\parindent}{0pt}%
  \setlength{\parskip}{0pt plus 0.3pt}%
  \let\item\@idxitem
}{%
  \clearpage
}
\makeatother

\IfFileExists{\jobname-pw.ind}{\input{\jobname-pw.ind}}{}

\end{document}

      