%% latex-korrekturansicht-vorspann.tex
%% Vorspann für die Korrekturansicht.
%% Lädt die gemeinsame Datei latex-vorspann.tex mit gesetztem Schalter.

\newif\ifkorrekturansicht
\korrekturansichttrue

\input{../tex-inputs/latex-vorspann}


\renewcommand{\erwaehntePersonen}{Personen: Louis Bolle-Ritz, Theodore Rottenberg, Olga Schnitzler, ?? [Liebhaber von Theodore Rottenberg, Ende 1902 und Anfang 1903]}
\renewcommand{\erwaehnteInstitutionen}{Institutionen: Hotel Fürstenhof}
\renewcommand{\erwaehnteOrte}{Orte: Frankfurt am Main, Fürstenhof, Kaiserstraße, Münchener Straße, Wien}
\renewcommand{\erwaehnteWerke}{Werke: Neue Freie Presse}
\section[ Paul Goldmann an Arthur Schnitzler, 28. 12. {[}1902{]}]{Paul Goldmann an Arthur Schnitzler, 28. 12. {[}1902{]}}
\nopagebreak\mylabel{v}
\rehead{ }\normalsize\beginnumbering\briefempfaengerindex{Schnitzler, Arthur@\textsc{Schnitzler, Arthur}!zzzGoldmann, Paul@\emph{von Paul Goldmann}!1902-12-281@{28. 12. {[}1902{]}}|(be}
\toendnotes[C]{\smallbreak\pagebreak[2]}\Standort{DLA, A:Schnitzler, HS.NZ85.1.3172.}
\physDesc{Brief, 1 Blatt, 3 Seiten
\newline{}Handschrift: schwarze Tinte, deutsche Kurrent
\newline{}Schnitzler: mit Bleistift das Jahr »{[}1{]}902« vermerkt }\toendnotes[C]{\smallbreak}
\pstart
           \noindent{}{\pb}\textcolor{gray}{\textbf{\textsc{Telephon}{ }4167.
                     }}\hfill \textcolor{gray}{\textbf{\textsc{Telegramm-adresse}:}}\pend
           
\pstart
           \textcolor{gray}{\textbf{UND{ }3940.
                     }}\hfill \textcolor{gray}{\textbf{\textsc{\textcolor{pink}{Palast Fürstenhof}{}\ledrightnote{\textcolor{pink}{Fürstenhof}}{ }\textcolor{pink}{Frankfurtmain}{}\ledrightnote{\textcolor{pink}{Frankfurt am Main}}.
                        }}}\pend
           
\pstart
           \centering{}\textcolor{gray}{\textbf{\textsc{\textbf{\textcolor{brown}{Palast-Hotel}{}\ledrightnote{\textcolor{brown}{Hotel Fürstenhof}}}}}}\pend
           
\pstart
           \noindent{}\centering{}\textcolor{gray}{\textbf{\textsc{\textcolor{brown}{Fürstenhof}{}\ledrightnote{\textcolor{brown}{Hotel Fürstenhof}}}}}\pend
           
\pstart
           \noindent{}\centering{}\textcolor{gray}{\textbf{\textcolor{blue}{LOUIS BOLLE-RITZ}{}\ledrightnote{\textcolor{blue}{Louis Bolle-Ritz}}.}}\pend
           
\pstart
           \noindent{}\centering{}\textcolor{gray}{\textbf{(\textsc{\textcolor{pink}{Kaiserstrasse}{}\ledrightnote{\textcolor{pink}{Kaiserstraße}} – \textcolor{pink}{Kronprinzenstrasse}{}\ledrightnote{\textcolor{pink}{Münchener Straße}}})}}\pend
           
\pstart
           \raggedleft{}\textcolor{pink}{\textcolor{gray}{\textbf{Frankfurt \textsuperscript{a/}M.}}}{}\ledrightnote{\textcolor{pink}{Frankfurt am Main}}{ }28. Dezember.\pend
           
\pstart\center{}Mein lieber Freund,\pend
\pstart
           Ich habe Wochen verſtreichen laſſen müſſen, ehe ich für Deinen lieben Brief, der mich
               ganz beſonders erfreut hat, weil er ſo viel Schönes über Dich ſelbſt enthielt, auch
               nur danken konnte. Eine das gewöhnliche Maß noch weit überſteigende Häufung von
               Arbeit (Du wirſt ſie ja ſelbſt in der \textcolor{green}{N. Fr. Pr.}{}\ledrightnote{\textcolor{green}{Neue Freie Presse}}
               beobachtet haben) war die Urſache. Hier in \textcolor{pink}{Frankfurt}{}\ledrightnote{\textcolor{pink}{Frankfurt am Main}}, wo ich, meiner Gewohnheit gemäß, die Zeit von Weihnachten bis Neujahr
               verbringe, finde ich endlich die {\pb}Muße, Dir zu
               ſchreiben. Freilich, der ausführliche Brief, den ich plante, kommt wieder nicht zu
               Stande. Und das geſchieht deshalb nicht, weil ich ſo Fürchterliches hier erlebe, daß
               ich nicht fähig bin, zu ſchreiben. Meine Beziehungen zu der \label{K_L03231-2v}\edtext{\textcolor{blue}{Frau}{}\ledrightnote{{$\rightarrow$}\textcolor{blue}{Theodore Rottenberg}}, die Du
                  kennſt}{\lemma{\textnormal{\emph{Frau, die Du
                  kennſt}}}\Cendnote{\textnormal{womöglich \textcolor{blue}{Theodore Rottenberg}, siehe Paul Goldmann an Arthur Schnitzler, 8. 10. [1899]}}}\label{K_L03231-2h}, haben in dieſen Tagen ihr Ende gefunden. Durch meine Schuld: Denn als ich
               vor drei Monaten allerlei Klatſch über ſie erfuhr, ſtieß ich ſie fort. Sonst iſt ſie
               immer wiedergekommen. Diesmal aber habe ich ihr offenbar Unrecht gethan. Und das
               Schlimmſte: es war ein \label{K_L03231-3v}\edtext{\textcolor{blue}{Tröſter}{}\ledrightnote{{$\rightarrow$}\textcolor{blue}{?? [Liebhaber von Theodore Rottenberg, Ende 1902 und Anfang 1903]}}}{\lemma{\textnormal{\emph{Tröſter}}}\Cendnote{\textnormal{nicht ermittelt}}}\label{K_L03231-3h} bei der Hand.
                  Geſtern erhielt ich den Abſchiedsbrief: »Lebe wohl!
               Du haſt ſchlecht an mir gehandelt! Ich kann Dir nicht verzeihen. Ich habe einen \textcolor{blue}{Beſſeren}{}\ledrightnote{{$\rightarrow$}\textcolor{blue}{?? [Liebhaber von Theodore Rottenberg, Ende 1902 und Anfang 1903]}} gefunden!«\pend
           
\pstart
           Und das Entſetzliche iſt, daß ich ſie jetzt liebe, – liebe, wie ich ſie nie geliebt
               habe. Und daß in meinem armen Leben nirgends ein Erſatz iſt und nie mehr ſich finden
               wird. Ich erinnere mich nicht, jemals ſo gelitten zu haben. Am Tage die Erinnerungen
               auf Schritt und Tritt – Nachts die Marter {\pb}der
               Gewiſſensvorwürfe!\pend
           
\pstart
           Liebſter Freund! Verzeih’ mir, daß ich Dir nicht mehr, – daß ich Dir nicht über Dich
               ſchreibe. Entſchuldige mich auch bei \textsc{\textcolor{blue}{Olga}{}\ledrightnote{\textcolor{blue}{Olga Schnitzler}}}, der ich von \textcolor{pink}{hier}{}\ledrightnote{{$\rightarrow$}\textcolor{pink}{Frankfurt am Main}} auf für
               ihren lieben Brief danken wollte. Ich wünſche Euch \textcolor{blue}{Beiden}{}\ledrightnote{{$\rightarrow$}\textcolor{blue}{Olga Schnitzler}} ein glückliches neues Jahr!\pend
           
\pstart
           Viele treue Grüße! {\\[\baselineskip]}Dein {\\[\baselineskip]}\spacefill\mbox{Paul Goldmann.}\pend
           \leftskip=0em{}\endnumbering\briefempfaengerindex{Schnitzler, Arthur@\textsc{Schnitzler, Arthur}!zzzGoldmann, Paul@\emph{von Paul Goldmann}!1902-12-281@{28. 12. {[}1902{]}}|)be}\mylabel{h}
\begin{anhang}
\end{anhang}\normalsize

\doendnotes{C}
\bigskip
\vfill

\clearpage

\footnotesize

\lohead{\textsc{register}}

% Definiere theindex-Environment komplett neu ohne reledmac
\makeatletter
\renewenvironment{theindex}{%
  \section*{\indexname}%
  \setlength{\parindent}{0pt}%
  \setlength{\parskip}{0pt plus 0.3pt}%
  \let\item\@idxitem
}{%
  \clearpage
}
\makeatother

\IfFileExists{\jobname-pw.ind}{\input{\jobname-pw.ind}}{}

\end{document}

      