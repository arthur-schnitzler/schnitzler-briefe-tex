%% latex-korrekturansicht-vorspann.tex
%% Vorspann für die Korrekturansicht.
%% Lädt die gemeinsame Datei latex-vorspann.tex mit gesetztem Schalter.

\newif\ifkorrekturansicht
\korrekturansichttrue

\input{../tex-inputs/latex-vorspann}


               \section[Paul Goldmann an Arthur Schnitzler, Paul Goldmann an Arthur Schnitzler, 29. 6. {[}1896{]}]{ Paul Goldmann an Arthur Schnitzler, 29. 6. {[}1896{]}}\nopagebreak\mylabel{v}\rehead{ }\normalsize\beginnumbering\briefempfaengerindex{Schnitzler, Arthur@\textsc{Schnitzler, Arthur}!zzzGoldmann, Paul@\emph{von Paul Goldmann}!1896-06-292@{29. 6. {[}1896{]}}|(be} \toendnotes[C]{\smallbreak\pagebreak[2]} \Standort{DLA, A:Schnitzler, HS.NZ85.1.3166.}
\physDesc{Brief, 1 Blatt, 4 Seiten
\newline{}Handschrift: blaue Tinte, deutsche Kurrent
\newline{}Schnitzler: 1) mit Bleistift das Jahr »96« vermerkt 2) mit rotem Buntstift eine Unterstreichung}\toendnotes[C]{\smallbreak}\pstart
           \noindent{}{\pb}\textcolor{gray}{\textbf{\textbf{\textcolor{brown}{Frankfurter Zeitung}{}\ledrightnote{\textcolor{brown}{Frankfurter Zeitung}}}}}\pend
           \pstart
           \textcolor{gray}{\textbf{(\textcolor{brown}{\begin{otherlanguage}{french}Gazette de Francfort\end{otherlanguage}}{}\ledrightnote{\textcolor{brown}{Frankfurter Zeitung}}).}}\pend
           \pstart
           \textcolor{gray}{\textbf{\textbf{\begin{otherlanguage}{french}Fondateur M.\end{otherlanguage}{ }\textcolor{blue}{L. Sonnemann}{}\ledrightnote{\textcolor{blue}{Leopold Sonnemann}}.}}}\pend
           \pstart
           \begin{otherlanguage}{french}\textcolor{gray}{\textbf{\textcolor{green}{Journal}{}\ledrightnote{→\textcolor{green}{Frankfurter Zeitung}} politique,
                        financier,}}\end{otherlanguage}\pend
           \pstart
           \begin{otherlanguage}{french}\textcolor{gray}{\textbf{commercial et littéraire.}}\end{otherlanguage}\pend
           \pstart
           \begin{otherlanguage}{french}\textcolor{gray}{\textbf{\textbf{Paraissant trois fois par jour.}}}\end{otherlanguage}\hfill \textsc{\textcolor{pink}{Paris}{}\ledrightnote{\textcolor{pink}{Paris}}}, 29. Juni.\pend
           \pstart
           \begin{otherlanguage}{french}\textcolor{gray}{\textbf{\textbf{Bureau à \textcolor{pink}{Paris}{}\ledrightnote{\textcolor{pink}{Paris}}}}}\end{otherlanguage}\pend
           \pstart
           \begin{otherlanguage}{french}\textcolor{gray}{\textbf{\textbf{\textcolor{pink}{24. Rue Feydeau}{}\ledrightnote{\textcolor{pink}{rue Feydeau}}.}}}\end{otherlanguage}\pend
           \pstart{}Mein lieber Freund,\pend\pstart
           Was ſoll man gegen ein viermal unterſtrichenes »durchaus« machen? Gar ſo »durchaus« \substVorne{}\textsuperscript{\textcolor{gray}{×}\-\textcolor{gray}{×}\-\textcolor{gray}{×}\-\textcolor{gray}{×}}\substDazwischen{}bin\substHinten{} ich ja nicht gegen \textcolor{pink}{Dänemark}{}\ledrightnote{\textcolor{pink}{Dänemark}}
               eingenommen. Ich habe nur nicht die Mittel, um hinzufahren, und nicht die mindeſte
               Luſt, dortzubleiben. Da Du aber meinſt, daß dies ſchwächliche Gründe ſind, ſo haſt Du
               jedenfalls Recht und ich werde \label{K_L02779-1v}\edtext{hinkommen}{\lemma{\textnormal{\emph{hinkommen}}}\Cendnote{\textnormal{siehe Paul Goldmann an Arthur Schnitzler, 29. 4. [1896]}}}\label{K_L02779-1h}. Alſo, wenn ich bis Anfang {\pb}Auguſt nicht ganz bankrott bin (was möglich iſt) und wenn
               nichts Anderes Wichtiges dazwiſchen kommt, ſo treffen wir uns zwiſchen dem 5. u. 10. Auguſt in \textsc{\textcolor{pink}{Scottsborg}{}\ledrightnote{\textcolor{pink}{Skodsborg}}}, welcher \textcolor{pink}{Ort}{}\ledrightnote{→\textcolor{pink}{Skodsborg}} nach Deinen
               Schilderungen ſo billig iſt, daß man ihn ſchon wegen ſeiner Billigkeit aufſuchen
               müßte. Ich kehre ſicher mit großen Erſparniſſen heim. Andere Leute gehen auf die
               Goldfelder von \textsc{\textcolor{pink}{Transvaal}{}\ledrightnote{\textcolor{pink}{Transvaal}}}, ich werde nach \textsc{\textcolor{pink}{Scottsborg}{}\ledrightnote{\textcolor{pink}{Skodsborg}}} gehen. Gott allein weiß, wer Euch dieſe \textcolor{pink}{dän}{}\ledrightnote{→\textcolor{pink}{Dänemark}}iſche Idee in den Kopf geſetzt hat! {\pb}\textcolor{pink}{Europa}{}\ledrightnote{\textcolor{pink}{Europa}} iſt ſo ſchön und es gibt ſoviel Herrliches
               zu ſehen. Muß man alſo gerade in ein \textcolor{pink}{Land}{}\ledrightnote{→\textcolor{pink}{Dänemark}} gehen, in dem es \strikeout{a\textcolor{gray}{b}ſ\textcolor{gray}{o}} abſolut nichts gibt: weder Gebirge, noch Kunſt, noch Vergangenheit, – höchſtens Meer, aber auch das wird vielleicht ein Schwindel ſein und ich
               werde es erſt glauben, wenn ich es geſehen habe.\pend
           \pstart
           \label{K_L02779-2v}\edtext{\begin{otherlanguage}{french}\textsc{Enfin}\end{otherlanguage}}{\lemma{\textnormal{\emph{Enfin}}}\Cendnote{\textnormal{französisch: kurzum}}}\label{K_L02779-2h},
               ich komme nach \textcolor{pink}{Dänemark}{}\ledrightnote{\textcolor{pink}{Dänemark}}. Ihr werdet mich
               hoffentlich über Eure Unterwegs-Adreſſen auf dem Laufenden halten. \textsc{\textcolor{blue}{Richard}{}\ledrightnote{\textcolor{blue}{Richard Beer-Hofmann}}} wird ſich auch zu einer Correſpondenzkarte einmal entſchließen {\pb}müſſen; aber ich glaube, die \textcolor{pink}{dän}{}\ledrightnote{→\textcolor{pink}{Dänemark}}iſchen Poſtkarten ſind kleiner als die
                  \textcolor{pink}{öſterreich}{}\ledrightnote{→\textcolor{pink}{Österreich}}iſchen, was
               wieder ein Vortheil dieſes ſchönen \textcolor{pink}{Land}{}\ledrightnote{→\textcolor{pink}{Dänemark}}es iſt.\pend
           \pstart
           Du aber, mein lieber Freund, reiſe glücklich. Ich wünſche Dir von Herzen alles Gute
               auf den Weg.\pend
           \pstart
           Die Zeitungen, die Du auf dem Zettel angegeben, kann ich Dir erſt morgen ſchicken, da weite Wege zu ihrer Beſorgung zu
               machen ſind. Gib alſo \label{K_L02779-3v}\edtext{\begin{otherlanguage}{french}Ordre\end{otherlanguage}}{\lemma{\textnormal{\emph{Ordre}}}\Cendnote{\textnormal{französisch: Anordnung}}}\label{K_L02779-3h}, daß ſie Dir nachgeſandt werden.\pend
           \pstart
           Von Herzen und in Treue{\\[\baselineskip]}Dein{\\[\baselineskip]}\spacefill\mbox{Paul Goldmann.}\pend
           \leftskip=0em{}\pstart
           \noindent{}Schick’ mir, bitte, das \textcolor{green}{Buch}{}\ledrightnote{→\textcolor{green}{Wie ich es sehe}}
                  von \textsc{\textcolor{blue}{Altenberg}{}\ledrightnote{\textcolor{blue}{Peter Altenberg}}}.\pend
           \endnumbering\briefempfaengerindex{Schnitzler, Arthur@\textsc{Schnitzler, Arthur}!zzzGoldmann, Paul@\emph{von Paul Goldmann}!1896-06-292@{29. 6. {[}1896{]}}|)be}\mylabel{h}  \normalsize

\doendnotes{C}
\bigskip
\vfill

\clearpage

\footnotesize

\lohead{\textsc{register}}

% Definiere theindex-Environment komplett neu ohne reledmac
\makeatletter
\renewenvironment{theindex}{%
  \section*{\indexname}%
  \setlength{\parindent}{0pt}%
  \setlength{\parskip}{0pt plus 0.3pt}%
  \let\item\@idxitem
}{%
  \clearpage
}
\makeatother

\IfFileExists{\jobname-pw.ind}{\input{\jobname-pw.ind}}{}

\end{document}

      