%% latex-korrekturansicht-vorspann.tex
%% Vorspann für die Korrekturansicht.
%% Lädt die gemeinsame Datei latex-vorspann.tex mit gesetztem Schalter.

\newif\ifkorrekturansicht
\korrekturansichttrue

\input{../tex-inputs/latex-vorspann}


\renewcommand{\erwaehnteOrte}{Orte: Berlin, Frankgasse, Wien}
\renewcommand{\erwaehnteWerke}{Werke: Reigen. Zehn Dialoge}
\section[ Paul Goldmann an Arthur Schnitzler, 29. 6. 1903]{Paul Goldmann an Arthur Schnitzler, 29. 6. 1903}
\nopagebreak\mylabel{v}
\rehead{ }\normalsize\beginnumbering\briefempfaengerindex{Schnitzler, Arthur@\textsc{Schnitzler, Arthur}!zzzGoldmann, Paul@\emph{von Paul Goldmann}!1903-06-291@{29. 6. 1903}|(be}
\toendnotes[C]{\smallbreak\pagebreak[2]}\Standort{DLA, A:Schnitzler, HS.NZ85.1.3173.}
\physDesc{Postkarte
\newline{}Handschrift: 1) blaue Tinte, deutsche Kurrent\hspace{1em}2) blaue Tinte, lateinische Kurrent (\noindent{}Adresse)\hspace{1em}
\newline{}Versand: Stempel: »\nobreak{}\oindex{Berlin@\textbf{Berlin}, \emph{https://www.geonames.org/ontologyP.PPLC}|pwk}Berlin S.W. 11, 29. 6. 03., 4–5 N.\nobreak{}«. Stempel: »\nobreak{}9/3 Wien 72, 30. 6. 03, 9. V, Bestellt\nobreak{}«.  
\newline{}Schnitzler: mit Bleistift das Jahr »{[}1{]}903« vermerkt }\toendnotes[C]{\smallbreak}\pstart{}{\pb}Herrn\pend{}\pstart{}Dr. Arthur Schnitzler\pend{}\pstart{}\textcolor{pink}{Wien}{}\ledrightnote{\textcolor{pink}{Wien}}\pend{}\pstart{}\textcolor{pink}{IX. Frankgaſse 1}{}\ledrightnote{\textcolor{pink}{Frankgasse}}.\pend{}
{\bigskip}
\pstart
           {\pb}\textcolor{pink}{Berlin}{}\ledrightnote{\textcolor{pink}{Berlin}}, 29. Juni.\pend
           
\pstart{}Mein lieber Freund,\pend
\pstart
           Ich leſe eben, daß der »\textcolor{green}{Reigen}{}\ledrightnote{\textcolor{green}{Reigen. Zehn Dialoge}}« bereits in
                  \label{K_L03376-1v}\edtext{achter Auflage}{\lemma{\textnormal{\emph{achter Auflage}}}\Cendnote{\textnormal{vgl. A. S.: \emph{Tagebuch}, 28. 6. 1903}}}\label{K_L03376-1h} erſchienen iſt. Ich beglückwünſche Dich zu dem Rieſenerfolg dieſes \textcolor{green}{Buch}{}\ledrightnote{{$\rightarrow$}\textcolor{green}{Reigen. Zehn Dialoge}}es, welches auch für mich
               zum Feinſten und Reizendſten gehört, das Du geſchrieben haſt.\pend
           
\pstart
           Herzlichſt Dein {\\[\baselineskip]}\spacefill\mbox{Paul Goldmnn}\pend
           \leftskip=0em{}\endnumbering\briefempfaengerindex{Schnitzler, Arthur@\textsc{Schnitzler, Arthur}!zzzGoldmann, Paul@\emph{von Paul Goldmann}!1903-06-291@{29. 6. 1903}|)be}\mylabel{h}
\begin{anhang}
\end{anhang}\normalsize

\doendnotes{C}
\bigskip
\vfill

\clearpage

\footnotesize

\lohead{\textsc{register}}

% Definiere theindex-Environment komplett neu ohne reledmac
\makeatletter
\renewenvironment{theindex}{%
  \section*{\indexname}%
  \setlength{\parindent}{0pt}%
  \setlength{\parskip}{0pt plus 0.3pt}%
  \let\item\@idxitem
}{%
  \clearpage
}
\makeatother

\IfFileExists{\jobname-pw.ind}{\input{\jobname-pw.ind}}{}

\end{document}

      