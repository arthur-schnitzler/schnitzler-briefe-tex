%% latex-korrekturansicht-vorspann.tex
%% Vorspann für die Korrekturansicht.
%% Lädt die gemeinsame Datei latex-vorspann.tex mit gesetztem Schalter.

\newif\ifkorrekturansicht
\korrekturansichttrue

\input{../tex-inputs/latex-vorspann}


               \section[Hermann Bahr an Arthur Schnitzler, {[}24. 7. 1893{]}]{ Hermann Bahr an Arthur Schnitzler, {[}24. 7. 1893{]}}\nopagebreak\mylabel{v}\rehead{ }\normalsize\beginnumbering\briefempfaengerindex{Schnitzler, Arthur@\textsc{Schnitzler, Arthur}!zzzBahr, Hermann@\emph{von Hermann Bahr}!1893-07-241@{{[}24. 7. 1893{]}}|(be} \toendnotes[C]{\smallbreak\pagebreak[2]} \Standort{CUL, Schnitzler, B 5b.}
\physDesc{Brief, 1 Blatt, 1 Seite
\newline{}Handschrift: Bleistift, deutsche Kurrent
\newline{}Schnitzler: mit Bleistift datiert: »24. 7. 93« \newline{}Ordnung: 1) mit Bleistift von unbekannter Hand nummeriert:
                                    »11« 2) mit rotem Buntstift von unbekannter Hand nummeriert:
                                    »11«}\buchAbdrucke{\weitereDrucke{Hermann Bahr, Arthur Schnitzler: \emph{Briefwechsel, Aufzeichnungen, Dokumente (1891–1931)}. Hg. Kurt Ifkovits und Martin Anton Müller. Göttingen: \emph{Wallstein} 2018, S. 36.} }\toendnotes[C]{\smallbreak}\pstart
           \noindent{}{\pb}\textcolor{gray}{\textbf{\textcolor{brown}{Deutſche Zeitung}{}\ledrightnote{\textcolor{brown}{Deutsche Zeitung}}}}\pend
           \pstart
           \textcolor{gray}{\textbf{\textcolor{pink}{Wien}{}\ledrightnote{\textcolor{pink}{Wien}}}}\pend
           \pstart
           \textcolor{gray}{\textbf{\textcolor{pink}{IX., Pelikangaſſe 4}{}\ledrightnote{\textcolor{pink}{Pelikangasse}}.}}\pend
           \pstart\center{}Lieber Freund!\pend\pstart
           Von Ihrer Anfrage über \textcolor{blue}{Loris}{}\ledrightnote{\textcolor{blue}{Hugo von Hofmannsthal}} hat man mir nichts
               mitgeteilt. Ich ko{\geminationm}e morgen entweder zwiſchen
                  3 u. 4{ }\textcolor{pink}{Burgring}{}\ledrightnote{\textcolor{pink}{Burgring}} oder um ½ 5{ }\textcolor{pink}{Grillparzerſtr}{}\ledrightnote{\textcolor{pink}{Grillparzerstraße}}. Daß Sie \uline{uns} u. nur uns keine Notiz über \label{K_L00241_1v}\edtext{\textcolor{pink}{\textsc{Ischler}}{}\ledrightnote{\textcolor{pink}{Bad Ischl}} Aufführung}{\lemma{\textnormal{\emph{Ischler Aufführung}}}\Cendnote{\textnormal{Uraufführung von \emph{\textcolor{green}{Abschiedssouper}}, 14. 7. 1893}}}\label{K_L00241_1h} geſchickt, iſt nicht
               ſchön.\pend
           \pstart
           Herzlichſt{\\[\baselineskip]}Ihr{\\[\baselineskip]}\spacefill\mbox{HermannBahr}\pend
           \leftskip=0em{}\endnumbering\briefempfaengerindex{Schnitzler, Arthur@\textsc{Schnitzler, Arthur}!zzzBahr, Hermann@\emph{von Hermann Bahr}!1893-07-241@{{[}24. 7. 1893{]}}|)be}\mylabel{h}  \normalsize

\doendnotes{C}
\bigskip
\vfill

\clearpage

\footnotesize

\lohead{\textsc{register}}

% Definiere theindex-Environment komplett neu ohne reledmac
\makeatletter
\renewenvironment{theindex}{%
  \section*{\indexname}%
  \setlength{\parindent}{0pt}%
  \setlength{\parskip}{0pt plus 0.3pt}%
  \let\item\@idxitem
}{%
  \clearpage
}
\makeatother

\IfFileExists{\jobname-pw.ind}{\input{\jobname-pw.ind}}{}

\end{document}

      