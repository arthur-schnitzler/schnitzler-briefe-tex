%% latex-korrekturansicht-vorspann.tex
%% Vorspann für die Korrekturansicht.
%% Lädt die gemeinsame Datei latex-vorspann.tex mit gesetztem Schalter.

\newif\ifkorrekturansicht
\korrekturansichttrue

\input{../tex-inputs/latex-vorspann}


               \section[Paul Goldmann an Arthur Schnitzler, 6. 2. {[}1895{]}]{ Paul Goldmann an Arthur Schnitzler, 6. 2. {[}1895{]}}\nopagebreak\mylabel{v}\rehead{ }\normalsize\beginnumbering\briefempfaengerindex{Schnitzler, Arthur@\textsc{Schnitzler, Arthur}!zzzGoldmann, Paul@\emph{von Paul Goldmann}!1895-02-061@{6. 2. {[}1895{]}}|(be} \toendnotes[C]{\smallbreak\pagebreak[2]} \Standort{DLA, A:Schnitzler, HS.NZ85.1.3165.}
\physDesc{Brief, 2 Blätter, 7 Seiten
\newline{}Handschrift: schwarze Tinte, deutsche Kurrent
\newline{}Schnitzler: 1) mit Bleistift das Jahr »95« vermerkt 2) mit rotem Buntstift eine Unterstreichung}\toendnotes[C]{\smallbreak}\pstart
           \noindent{}{\pb}\textcolor{gray}{\textbf{\textbf{\textcolor{brown}{Frankfurter Zeitung}{}\ledrightnote{\textcolor{brown}{Frankfurter Zeitung}}}}}\pend
           \pstart
           \textcolor{gray}{\textbf{(\textcolor{brown}{\begin{otherlanguage}{french}Gazette de Francfort\end{otherlanguage}}{}\ledrightnote{\textcolor{brown}{Frankfurter Zeitung}}.) }}\hfill \textcolor{pink}{\textsc{Paris},}{}\ledrightnote{\textcolor{pink}{Paris}}6. Februar.\pend
           \pstart
           \textcolor{gray}{\textbf{\textbf{\begin{otherlanguage}{french}Fondateur M. \textcolor{blue}{L. Sonnemann}{}\ledrightnote{\textcolor{blue}{Leopold Sonnemann}}\end{otherlanguage}.}}}\pend
           \pstart
           \begin{otherlanguage}{french}\textcolor{gray}{\textbf{\textcolor{green}{Journal}{}\ledrightnote{\textcolor{green}{Frankfurter Zeitung}} politique, financier,}}\end{otherlanguage}\pend
           \pstart
           \begin{otherlanguage}{french}\textcolor{gray}{\textbf{commercial et littéraire.}}\end{otherlanguage}\pend
           \pstart
           \begin{otherlanguage}{french}\textcolor{gray}{\textbf{\textbf{Paraissant trois fois par jour.}}}\end{otherlanguage}\pend
           \pstart
           \begin{otherlanguage}{french}\textcolor{gray}{\textbf{\textbf{Bureau à \textcolor{pink}{Paris}{}\ledrightnote{\textcolor{pink}{Paris}}:}}}\end{otherlanguage}\pend
           \pstart
           \begin{otherlanguage}{french}\textcolor{gray}{\textbf{\textbf{\textcolor{pink}{24. Rue Feydeau}{}\ledrightnote{\textcolor{pink}{rue Feydeau}}.}}}\end{otherlanguage}\pend
           \pstart\center{}Mein lieber Freund,\pend\pstart
           Ich hätte Dir Deinen Brief gern umgehend beantwortet, hatte aber gerade ausnahmsweis
               viel zu thun und komme nun erſt heut zur Antwort.\pend
           \pstart
           Was Du mir da ſchreibſt aus einer Aufregung und Verſtimmung heraus, die noch an jedem
               Worte haften geblieben iſt, hat mich recht ſehr geſchmerzt. Freilich nur in dem
               Sinne, daß es mir unendlich leid thut, Dich inmitten all dieſer \label{K_L02728-1v}\edtext{Widerwärtigkeiten}{\lemma{\textnormal{\emph{Widerwärtigkeiten}}}\Cendnote{\textnormal{Wie \textcolor{blue}{Schnitzler} in
                  seinem \emph{\textcolor{green}{Tagebuch}} ausführlich dokumentierte,
                  machte ihm in dieser Zeit vor allem \textcolor{blue}{Adele
                     Sandrock} zu schaffen. Die \textcolor{blue}{Schauspielerin}, mit der er – neben anderen – ein
                  Verhältnis führte, kam \textcolor{blue}{Schnitzler}s \textcolor{blue}{Freund}{ }\textcolor{blue}{Felix Salten} näher. \textcolor{blue}{Schnitzler} fühlte sich betrogen, obgleich es zu keiner
                  körperlichen Intimität gekommen sei, und brach mit ihr. \textcolor{blue}{Sandrock} drohte \textcolor{blue}{Schnitzler} daraufhin nicht nur damit, sich das Leben zu nehmen. Er
                  fürchtete auch, sie würde versuchen, die \emph{\textcolor{green}{Liebelei}} vom \emph{\textcolor{brown}{Burgtheater}} wieder
                  abzusetzen. Laut \textcolor{blue}{Hermann Bahr} soll \textcolor{blue}{Sandrock} sogar das \textcolor{green}{Stück} und ihre Rolle, jene der \textcolor{green}{Christine}, auch vor \textcolor{blue}{Max Eugen Burckhard}, dem \textcolor{blue}{Leiter} des \emph{\textcolor{brown}{Burgtheater}}s, schlechtgeredet und versucht haben, die
                  Aufführung des \textcolor{green}{Stück}s
                  hinauszuschieben, um \textcolor{blue}{Schnitzler}s
                  Aufmerksamkeit und Zuneigung zu erhalten. Bei der Uraufführung am 9. 10. 1895 am \textcolor{pink}{Burgtheater} spielte \textcolor{blue}{Sandrock} in der Hauptrolle.}}}\label{K_L02728-1h} zu wiſſen. {\pb}Um das \textcolor{green}{Endreſultat}{}\ledrightnote{→\textcolor{green}{Liebelei. Schauspiel in drei Akten}} machen ſie mich nicht im Mindeſten bekümmert. Ich
               ſehe die Dinge von fern an, wie aus den Wolken. Da ſehe ich denn ein Schiff, das
               unaufhaltſam dem Ziele zufährt. Die einzelnen Zickzacklinien des Kurſes ſehe ich
               nicht. Ich ſehe nur, daß es vorwärts geht, nicht zurück – daß es nicht zurückgehen
               kann. Ein paar intriguante \textcolor{blue}{Weibsbild}{}\ledrightnote{→\textcolor{blue}{Adele Sandrock}}er ſollen Dein \textcolor{green}{Werk}{}\ledrightnote{→\textcolor{green}{Liebelei. Schauspiel in drei Akten}}\strikeout{an} aufhalten, das mit der Kraft Deines Talentes dem
               Ziele zuſtrebt? Der Gedanke macht mich heiter, ſo unſinnig iſt er. {\pb}Und ich verlier meine Heiterkeit nur, wenn ich
               Deinen Brief wieder vornehme und Deine Verſtimmung herausleſe, die ich Dir gern
               erſpart wüßte. Aber ſchön! Du kämpfſt. Wer kämpft nicht? Und vergleiche Dein
               glückliches Loos, für ein hohes Ziel kämpfen zu dürfen, mit dem Anderer, mit dem
               meinen zum Beiſpiel, der ich mit Widerwärtigkeiten und tauſend Verhängiſſen ringen
               muß, nicht um hinaufzugelangen, wie Du, ſondern um nicht tiefer zu fallen, als ich
               ſchon ſtehe. {\pb}Hab’ Geduld, mein lieber Freund! Sei
               ruhig und laß’ die Dinge gehen, wie ſie gehen. Das Entſcheidende iſt bereits
               geſchehen: Du haſt ein ſchönes \textcolor{green}{Stück}{}\ledrightnote{→\textcolor{green}{Liebelei. Schauspiel in drei Akten}} geſchrieben. Alles Übrige iſt volſtändig gleichgiltig. \strikeout{Laß’} Laß’ Dich alſo nicht erregen. Blick’ weit hinaus
               in die Zukunft, laß’ Dich vom Tage nicht unterkriegen und vertrau’ auf Dich, wie ich
               auf Dich vertraue.\pend
           \pstart
           Das iſt freilich Alles recht vag und allgemein. Ich wünſchte, ich wüßte \strikeout{Nah} Näheres oder könnte gar bei Dir ſein, um {\pb}die Dinge im Einzelnen mit durchzuleben. Du ſollſt
               aber jedenfalls nicht glauben, daß Du mir ſchreiben mußt. Ich verſtehe es, daß Du
               wenig Stimmung zu Briefen findeſt, und warte ſchon meine Zeit ab. Nur möchte ich
               wiſſen, wann ungefähr die \textcolor{green}{Aufführung}{}\ledrightnote{→\textcolor{green}{Liebelei. Schauspiel in drei Akten}} ſein wird; und wenn ſie dann iſt, möchte ich mir am nächſten Morgen eine Depeſche über das Reſultat
               erbitten.\pend
           \pstart
           Iſt \textsc{\textcolor{blue}{Bahr}{}\ledrightnote{\textcolor{blue}{Hermann Bahr}}} nicht mit {\pb}unter Denen, gegen die Du zu
               kämpfen haſt? Die \label{K_L02728-5v}\edtext{\textcolor{green}{Kritik}{}\ledrightnote{→\textcolor{green}{Arthur Schnitzler: Sterben}}}{\lemma{\textnormal{\emph{Kritik}}}\Cendnote{\textnormal{\textcolor{blue}{A. G.} [=\textcolor{blue}{Alfred Gold}]: \textcolor{green}{\textcolor{blue}{Arthur Schnitzler}: Sterben}. In:
                     \emph{\textcolor{green}{Die Zeit}}, Bd. 2, Nr. 14, 5. 1. 1895, S. 14.}}}\label{K_L02728-5h} über »\textcolor{green}{Sterben}{}\ledrightnote{\textcolor{green}{Sterben. Novelle}}« in der »\textcolor{green}{Zeit}{}\ledrightnote{\textcolor{green}{Die Zeit. Wiener Wochenschrift}}« war ebenſo dumm als \label{K_L02728-2v}\edtext{beſchmockt}{\lemma{\textnormal{\emph{beſchmockt}}}\Cendnote{\textnormal{pejorativ: auf Wirkung /
                  Effekt bedacht}}}\label{K_L02728-2h}.\pend
           \pstart
           Ich ſandte Dir dieſer Tage ein paar \textcolor{pink}{franzöſiſch}{}\ledrightnote{→\textcolor{pink}{Frankreich}}e \label{K_L02728-4v}\edtext{Zeitungsartikel}{\lemma{\textnormal{\emph{Zeitungsartikel}}}\Cendnote{\textnormal{nicht
                  überliefert}}}\label{K_L02728-4h}. Du findeſt darunter vielleicht Manches, das Dich zerſtreut.
               Kann ich Dir ſonſt was aus \textsc{\textcolor{pink}{Paris}{}\ledrightnote{\textcolor{pink}{Paris}}} ſchicken? Das Geſcheiteſte wäre, Du ließeſt den ganzen Kram in \textcolor{pink}{Wien}{}\ledrightnote{\textcolor{pink}{Wien}} im Stich und kämeſt auf vierzehn Tage hierher. Das würde
               Dir gut thun!\pend
           \pstart
           {\pb}In Sommer werden wir uns \label{K_L02728-3v}\edtext{kaum ſehen}{\lemma{\textnormal{\emph{kaum ſehen}}}\Cendnote{\textnormal{Trotz
                     \textcolor{blue}{Goldmann}s Kuraufenthalt in \textcolor{pink}{Bad Tölz} sahen sich die beiden zwischen 28. 8. 1895 und 6. 9. 1895 in \textcolor{pink}{Bayern}.}}}\label{K_L02728-3h} können. Ich werde krank und
               kränker und mein \textcolor{blue}{Schwager}{}\ledrightnote{→\textcolor{blue}{Josef Rosengart}}
               beſteht darauf, daß ich während meines Urlaubs eine Kur gebrauche, vielleicht in \textsc{\textcolor{pink}{Toelz}{}\ledrightnote{\textcolor{pink}{Bad Tölz}}}, im \textcolor{pink}{bairiſch}{}\ledrightnote{→\textcolor{pink}{Bayern}}en
               Hochgebirge.\pend
           \pstart
           Grüß’ Dich Gott, mein lieber Freund, und ſei guten Muths!\pend
           \pstart
           Dein {\\[\baselineskip]}treuer {\\[\baselineskip]}\spacefill\mbox{Paul Goldmann}\pend
           \leftskip=0em{}\endnumbering\briefempfaengerindex{Schnitzler, Arthur@\textsc{Schnitzler, Arthur}!zzzGoldmann, Paul@\emph{von Paul Goldmann}!1895-02-061@{6. 2. {[}1895{]}}|)be}\mylabel{h}\begin{anhang}\end{anhang}\normalsize

\doendnotes{C}
\bigskip
\vfill

\clearpage

\footnotesize

\lohead{\textsc{register}}

% Definiere theindex-Environment komplett neu ohne reledmac
\makeatletter
\renewenvironment{theindex}{%
  \section*{\indexname}%
  \setlength{\parindent}{0pt}%
  \setlength{\parskip}{0pt plus 0.3pt}%
  \let\item\@idxitem
}{%
  \clearpage
}
\makeatother

\IfFileExists{\jobname-pw.ind}{\input{\jobname-pw.ind}}{}

\end{document}

      