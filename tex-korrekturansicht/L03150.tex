%% latex-korrekturansicht-vorspann.tex
%% Vorspann für die Korrekturansicht.
%% Lädt die gemeinsame Datei latex-vorspann.tex mit gesetztem Schalter.

\newif\ifkorrekturansicht
\korrekturansichttrue

\input{../tex-inputs/latex-vorspann}


\renewcommand{\erwaehntePersonen}{Personen: Hermann Bahr}
\renewcommand{\erwaehnteOrte}{Orte: Café Arkaden, Wien}
\renewcommand{\erwaehnteWerke}{}
\section[ Felix Salten an Arthur Schnitzler, {[}7. 2. 1895{]}]{Felix Salten an Arthur Schnitzler, {[}7. 2. 1895{]}}
\nopagebreak\mylabel{v}
\rehead{ }\normalsize\beginnumbering\briefempfaengerindex{Schnitzler, Arthur@\textsc{Schnitzler, Arthur}!zzzSalten, Felix@\emph{von Felix Salten}!1895-02-071@{{[}7. 2. 1895{]}}|(be}
\toendnotes[C]{\smallbreak\pagebreak[2]}\Standort{CUL, Schnitzler, B 89, A 1.}
\physDesc{Brief, 1 Blatt, 1 Seite, 440 Zeichen
\newline{}Handschrift: Bleistift, lateinische Kurrent
\newline{}Schnitzler: mit Bleistift datiert: »7/2 95« 
\newline{}Ordnung: mit Bleistift von unbekannter Hand nummeriert: »51« }
\buchAbdrucke{\weitereDrucke{Hermann Bahr, Arthur Schnitzler: \emph{Briefwechsel, Aufzeichnungen, Dokumente (1891–1931)}. Hg. Kurt Ifkovits und Martin Anton Müller. Göttingen: \emph{Wallstein} 2018, S. 96.} }\toendnotes[C]{\smallbreak}
\pstart
           \noindent{}{\pb}\label{K_L03150-1v}\edtext{L. F.}{\lemma{\textnormal{\emph{L. F.}}}\Cendnote{\textnormal{Lieber Freund}}}\label{K_L03150-1h} Von \textcolor{blue}{Bahr}{}\ledrightnote{\textcolor{blue}{Hermann Bahr}} noch lange aufgehalten, kam ich leider zu
               spät ins Caféhaus, Ich bedaure das am meisten, weil ich gewünscht hätte, mich gleich
               mit Ihnen \label{K_L03150-2v}\edtext{auseinanderzusetzen}{\lemma{\textnormal{\emph{auseinanderzusetzen}}}\Cendnote{\textnormal{Ein senkrechter Strich nach
                     »ausein« könnte darauf hindeuten, dass \textcolor{blue}{Salten} hier nachträglich eine Trennung des Wortes andeuten
                  wollte.}}}\label{K_L03150-2h}. Es wäre mir sehr werthvoll, wenn ich Sie \uline{jetzt gleich} sprechen könnte, oder zu Mittag. Wollen Sie
               nicht \introOben{}jetzt\introOben{} auf einem Sprung \label{K_L03150-3v}\edtext{ins \textcolor{pink}{Arcadencafé}{}\ledrightnote{\textcolor{pink}{Café Arkaden}}
                  kommen}{\lemma{\textnormal{\emph{ins Arcadencafé
                  kommen}}}\Cendnote{\textnormal{nicht nachweisbar}}}\label{K_L03150-3h}? Ich
               würde die Sache nur höchst ungern auf \substVorne{}\textsuperscript{n}\substDazwischen{}N\substHinten{}achmittag verschoben sehen, da mir für N. M. noch
               vieles zu thun \substVorne{}\textsuperscript{\textcolor{gray}{u}}\substDazwischen{}b\substHinten{}leibt.\pend
           
\pstart
           Ihr treuer {\\[\baselineskip]}\spacefill\mbox{Salten}\pend
           \leftskip=0em{}\endnumbering\briefempfaengerindex{Schnitzler, Arthur@\textsc{Schnitzler, Arthur}!zzzSalten, Felix@\emph{von Felix Salten}!1895-02-071@{{[}7. 2. 1895{]}}|)be}\mylabel{h}  \normalsize

\doendnotes{C}
\bigskip
\vfill

\clearpage

\footnotesize

\lohead{\textsc{register}}

% Definiere theindex-Environment komplett neu ohne reledmac
\makeatletter
\renewenvironment{theindex}{%
  \section*{\indexname}%
  \setlength{\parindent}{0pt}%
  \setlength{\parskip}{0pt plus 0.3pt}%
  \let\item\@idxitem
}{%
  \clearpage
}
\makeatother

\IfFileExists{\jobname-pw.ind}{\input{\jobname-pw.ind}}{}

\end{document}

      