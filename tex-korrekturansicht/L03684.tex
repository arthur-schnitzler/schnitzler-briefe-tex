%% latex-korrekturansicht-vorspann.tex
%% Vorspann für die Korrekturansicht.
%% Lädt die gemeinsame Datei latex-vorspann.tex mit gesetztem Schalter.

\newif\ifkorrekturansicht
\korrekturansichttrue

\input{../tex-inputs/latex-vorspann}


\section[Stefan Zweig an Arthur Schnitzler, 14. 8. 1920]{L03684 Stefan Zweig an Arthur Schnitzler, 14. 8. 1920}
\nopagebreak\mylabel{L03684v}
\rehead{ }\normalsize\beginnumbering\briefempfaengerindex{, @\textsc{, }!zzz, @\emph{von  }!1920-08-141@{14. 8. 1920}|(be}
\toendnotes[C]{\smallbreak\pagebreak[2]}\Standort{CUL, Schnitzler, B 118.}
\physDesc{Brief, 1 Blatt, 1 Seite, 643 Zeichen
\newline{}Schreibmaschine
\newline{}Handschrift: blaue Tinte, lateinische Kurrent (\noindent{}Unterschrift)
\newline{}Schnitzler: mit Bleistift beschriftet: »\textsc{Zweig}« }
\buchAbdrucke{\weitereDrucke{Stefan Zweig: \emph{Briefwechsel mit Hermann Bahr, Sigmund Freud, Rainer Maria
                        Rilke und Arthur Schnitzler}. Herausgegeben von Jeffrey B. Berlin, Hans-Ulrich Lindken und Donald A. Prater. Frankfurt am Main: \emph{S. Fischer} 1987, S. 410–411.} }\toendnotes[C]{\smallbreak}
\pstart
           \raggedleft{}{\pb}\textcolor{pink}{Salzburg}\oindex{Salzburg@\textbf{Salzburg}, \emph{Verwaltungsgebiet}|pw}{}\ledrightnote{\textcolor{pink}{Salzburg}}, am 14. August 1920\pend
           
\pstart\center{}Lieber verehrter Herr Doktor!\pend\vspace{0.5em}
\pstart
           Ich erhielt heute \label{K_L03684-1v}\edtext{beifolgendes Telegramm}{\lemma{\textnormal{\emph{beifolgendes Telegramm}}}\Cendnote{\textnormal{Beilage nicht erhalten. Es handelte sich aber um die Berechtigung
                  für eine amerikanische Ausgabe von \emph{\textcolor{green}{Casanova’s Heimfahrt}\pwindex{Schnitzler, Arthur 15. 5. 1862 Wien – 21. 10. 1931 ebd.@\textsc{Schnitzler, Arthur} (15. 5. 1862 Wien – 21. 10. 1931 ebd.), \emph{Schriftsteller, Mediziner}!Casanovas Heimfahrt@\strich\emph{Casanovas Heimfahrt}|pwk}}, siehe Arthur Schnitzler an Stefan Zweig, 20. 8. 1920.}}}\label{K_L03684-1} von dem \textcolor{pink}{New-Yorker}\oindex{New York City@\textbf{New York City}|pw}{}\ledrightnote{\textcolor{pink}{New York City}} Verleger \textcolor{blue}{Thomas Seltzer}\pwindex{Seltzer, Thomas 22.\,2.\,1875 Poltava – 11.\,9.\,1943 New York City@\textsc{Seltzer, Thomas} (22.\,2.\,1875 Poltava – 11.\,9.\,1943 New York City), \emph{Übersetzer, Verleger}|pw}{}\ledrightnote{\textcolor{blue}{Thomas Seltzer}},
                  \textcolor{pink}{5, West Fifth Street}\oindex{5 west 50th Street@\textbf{5 west 50th Street}, \emph{Bürogebäude}|pw}{}\ledrightnote{\textcolor{pink}{5 west 50th Street}}, \textcolor{pink}{New-York}\oindex{New York City@\textbf{New York City}|pw}{}\ledrightnote{\textcolor{pink}{New York City}}, der auch von
               mir \label{K_L03684-2v}\edtext{einige \textcolor{green}{Bücher}\pwindex{Zweig, Stefan 28.\,11.\,1881 Wien – 23.\,2.\,1942 Petrópolis@\textsc{Zweig, Stefan} (28.\,11.\,1881 Wien – 23.\,2.\,1942 Petrópolis), \emph{Schriftsteller}!Romain Rolland. Der Mann und das Werk.@\strich\emph{Romain Rolland. Der Mann und das Werk.}|pwv}\pwindex{Zweig, Stefan 28.\,11.\,1881 Wien – 23.\,2.\,1942 Petrópolis@\textsc{Zweig, Stefan} (28.\,11.\,1881 Wien – 23.\,2.\,1942 Petrópolis), \emph{Schriftsteller}!Jeremias. Ein dramatische Dichtung in neun Bildern@\strich\emph{Jeremias. Ein dramatische Dichtung in neun Bildern}|pwv}{}\ledrightnote{{$\rightarrow$}\emph{\textcolor{green}{Romain Rolland. Der Mann und das Werk.}}{\newline}{$\rightarrow$}\emph{\textcolor{green}{Jeremias. Ein dramatische Dichtung in neun Bildern}}}}{\lemma{\textnormal{\emph{einige Bücher}}}\Cendnote{\textnormal{Bis auf den Raubdruck \emph{\textcolor{green}{The Burning Secret}\pwindex{Zweig, Stefan 28.\,11.\,1881 Wien – 23.\,2.\,1942 Petrópolis@\textsc{Zweig, Stefan} (28.\,11.\,1881 Wien – 23.\,2.\,1942 Petrópolis), \emph{Schriftsteller}!Romain Rolland. Der Mann und das Werk.@\strich\emph{Romain Rolland. Der Mann und das Werk.}|pwk}} (siehe unten) hatte
                  das Verlagshaus \emph{\textcolor{brown}{Thomas Seltzer}\orgindex{Thomas Seltzer, Inc.@Thomas Seltzer, Inc.|pwk}} zu diesem Zeitpunkt noch nichts von \textcolor{blue}{Zweig}\pwindex{Zweig, Stefan 28.\,11.\,1881 Wien – 23.\,2.\,1942 Petrópolis@\textsc{Zweig, Stefan} (28.\,11.\,1881 Wien – 23.\,2.\,1942 Petrópolis), \emph{Schriftsteller}|pwk}
                  publiziert. 1921 erschien \textcolor{blue}{Zweigs}\pwindex{Zweig, Stefan 28.\,11.\,1881 Wien – 23.\,2.\,1942 Petrópolis@\textsc{Zweig, Stefan} (28.\,11.\,1881 Wien – 23.\,2.\,1942 Petrópolis), \emph{Schriftsteller}|pwk}{ }\textcolor{green}{Rolland-Biografie}\pwindex{Zweig, Stefan 28.\,11.\,1881 Wien – 23.\,2.\,1942 Petrópolis@\textsc{Zweig, Stefan} (28.\,11.\,1881 Wien – 23.\,2.\,1942 Petrópolis), \emph{Schriftsteller}!Romain Rolland. Der Mann und das Werk.@\strich\emph{Romain Rolland. Der Mann und das Werk.}|pwkv} (\textcolor{blue}{Stefan Zweig}\pwindex{Zweig, Stefan 28.\,11.\,1881 Wien – 23.\,2.\,1942 Petrópolis@\textsc{Zweig, Stefan} (28.\,11.\,1881 Wien – 23.\,2.\,1942 Petrópolis), \emph{Schriftsteller}|pwk}: \emph{\textcolor{green}{Romain Rolland. The man and his work}\pwindex{Zweig, Stefan 28.\,11.\,1881 Wien – 23.\,2.\,1942 Petrópolis@\textsc{Zweig, Stefan} (28.\,11.\,1881 Wien – 23.\,2.\,1942 Petrópolis), \emph{Schriftsteller}!Romain Rolland. The man and his work@\strich\emph{Romain Rolland. The man and his work}|pwk}}. Translated
                     from the original manuscript by \textcolor{blue}{Eden}\pwindex{Paul, Eden 1.\,11.\,1865 Sturminster Marshall – 1.\,12.\,1944@\textsc{Paul, Eden} (1.\,11.\,1865 Sturminster Marshall – 1.\,12.\,1944), \emph{Schriftsteller, Mediziner}|pwk} and \textcolor{blue}{Cedar Paul}\pwindex{Paul, Cedar 1880 – 18.\,3.\,1972@\textsc{Paul, Cedar} (1880 – 18.\,3.\,1972), \emph{Schriftstellerin}|pwk}.
                     New York: \emph{\textcolor{brown}{Seltzer}\orgindex{Thomas Seltzer, Inc.@Thomas Seltzer, Inc.|pwk}}{ }1921.), 1922 seine dramatische Dichtung \emph{\textcolor{green}{Jeremias}\pwindex{Zweig, Stefan 28.\,11.\,1881 Wien – 23.\,2.\,1942 Petrópolis@\textsc{Zweig, Stefan} (28.\,11.\,1881 Wien – 23.\,2.\,1942 Petrópolis), \emph{Schriftsteller}!Jeremias. Ein dramatische Dichtung in neun Bildern@\strich\emph{Jeremias. Ein dramatische Dichtung in neun Bildern}|pwk}} (\textcolor{blue}{Stefan Zweig}\pwindex{Zweig, Stefan 28.\,11.\,1881 Wien – 23.\,2.\,1942 Petrópolis@\textsc{Zweig, Stefan} (28.\,11.\,1881 Wien – 23.\,2.\,1942 Petrópolis), \emph{Schriftsteller}|pwk}: \emph{\textcolor{green}{Jeremiah. A Drama in Nine Scenes}\pwindex{Zweig, Stefan 28.\,11.\,1881 Wien – 23.\,2.\,1942 Petrópolis@\textsc{Zweig, Stefan} (28.\,11.\,1881 Wien – 23.\,2.\,1942 Petrópolis), \emph{Schriftsteller}!Jeremiah. A Drama in Nine Scenes@\strich\emph{Jeremiah. A Drama in Nine Scenes}|pwk}}. Translated
                        from the author’s revised German text by \textcolor{blue}{Eden}\pwindex{Paul, Eden 1.\,11.\,1865 Sturminster Marshall – 1.\,12.\,1944@\textsc{Paul, Eden} (1.\,11.\,1865 Sturminster Marshall – 1.\,12.\,1944), \emph{Schriftsteller, Mediziner}|pwk} and \textcolor{blue}{Cedar Paul}\pwindex{Paul, Cedar 1880 – 18.\,3.\,1972@\textsc{Paul, Cedar} (1880 – 18.\,3.\,1972), \emph{Schriftstellerin}|pwk}.
                     New York: \emph{\textcolor{brown}{T. Seltzer}\orgindex{Thomas Seltzer, Inc.@Thomas Seltzer, Inc.|pwk}}{ }1922.)}}}\label{K_L03684-2} bringt. Er hatte ursprünglich ein \textcolor{green}{Buch}\pwindex{Zweig, Stefan 28.\,11.\,1881 Wien – 23.\,2.\,1942 Petrópolis@\textsc{Zweig, Stefan} (28.\,11.\,1881 Wien – 23.\,2.\,1942 Petrópolis), \emph{Schriftsteller}!Brennendes Geheimnis@\strich\emph{Brennendes Geheimnis}|pwv}{}\ledrightnote{{$\rightarrow$}\emph{\textcolor{green}{Brennendes Geheimnis}}} widerrechtlich von mir
               \textcolor{green}{gebracht}\pwindex{Zweig, Stefan 28.\,11.\,1881 Wien – 23.\,2.\,1942 Petrópolis@\textsc{Zweig, Stefan} (28.\,11.\,1881 Wien – 23.\,2.\,1942 Petrópolis), \emph{Schriftsteller}!Romain Rolland. Der Mann und das Werk.@\strich\emph{Romain Rolland. Der Mann und das Werk.}|pwv}{}\ledrightnote{{$\rightarrow$}\emph{\textcolor{green}{Romain Rolland. Der Mann und das Werk.}}}, sogar \label{K_L03684-3v}\edtext{unter falschem
                  Namen}{\lemma{\textnormal{\emph{unter falschem
                  Namen}}}\Cendnote{\textnormal{1919 erschien die Novelle \emph{\textcolor{green}{Brennendes
                     Geheimnis}\pwindex{Zweig, Stefan 28.\,11.\,1881 Wien – 23.\,2.\,1942 Petrópolis@\textsc{Zweig, Stefan} (28.\,11.\,1881 Wien – 23.\,2.\,1942 Petrópolis), \emph{Schriftsteller}!Brennendes Geheimnis@\strich\emph{Brennendes Geheimnis}|pwk}} von \textcolor{blue}{Stefan Zweig}\pwindex{Zweig, Stefan 28.\,11.\,1881 Wien – 23.\,2.\,1942 Petrópolis@\textsc{Zweig, Stefan} (28.\,11.\,1881 Wien – 23.\,2.\,1942 Petrópolis), \emph{Schriftsteller}|pwk} nicht
                     autorisiert unter ebenfalls nicht autorisiertem, ins \textcolor{pink}{Englische}\oindex{England@\textbf{England}, \emph{Land}|pwk} übersetzten
                  Autornamen \textcolor{blue}{Stephen Branch}\pwindex{Zweig, Stefan 28.\,11.\,1881 Wien – 23.\,2.\,1942 Petrópolis@\textsc{Zweig, Stefan} (28.\,11.\,1881 Wien – 23.\,2.\,1942 Petrópolis), \emph{Schriftsteller}|pwk} [ = \textcolor{blue}{Stefan Zweig}\pwindex{Zweig, Stefan 28.\,11.\,1881 Wien – 23.\,2.\,1942 Petrópolis@\textsc{Zweig, Stefan} (28.\,11.\,1881 Wien – 23.\,2.\,1942 Petrópolis), \emph{Schriftsteller}|pwk}]: \emph{\textcolor{green}{The Burning Secret}\pwindex{Burning Secret@\emph{The Burning Secret}|pwk}}. New York: \emph{\textcolor{brown}{Seltzer and Scott}\orgindex{Thomas Seltzer, Inc.@Thomas Seltzer, Inc.|pwk}}{ }1919.}}}\label{K_L03684-3}, hat aber dann die Sache anständig beigelegt und gilt als einer der
               tatkräftigsten Unternehmer. Ich würde Ihnen immerhin raten ihm ein Angebot zu machen,
               das jedenfalls durch den Unterschied der Valuta schon erfreulich wird.\pend
           
\pstart
            Ich nutze den guten Anlass um mich Ihnen in Erinnerung zu bringen und bleibe
               mit vielen Grüssen Ihr{\\[\baselineskip]}aufrichtig ergebener{\\[\baselineskip]}\spacefill\mbox{{[}hs.:{]} Stefan Zweig}\pend
           \leftskip=0em{}\selectlanguage{ngerman}\endnumbering\briefempfaengerindex{, @\textsc{, }!zzz, @\emph{von  }!1920-08-141@{14. 8. 1920}|)be}\mylabel{L03684h}  \normalsize

\doendnotes{C}
\bigskip
\vfill

\clearpage

\footnotesize

\lohead{\textsc{register}}

% Definiere theindex-Environment komplett neu ohne reledmac
\makeatletter
\renewenvironment{theindex}{%
  \section*{\indexname}%
  \setlength{\parindent}{0pt}%
  \setlength{\parskip}{0pt plus 0.3pt}%
  \let\item\@idxitem
}{%
  \clearpage
}
\makeatother

\IfFileExists{\jobname-pw.ind}{\input{\jobname-pw.ind}}{}

\end{document}

      