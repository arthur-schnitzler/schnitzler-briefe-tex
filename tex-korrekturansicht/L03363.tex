%% latex-korrekturansicht-vorspann.tex
%% Vorspann für die Korrekturansicht.
%% Lädt die gemeinsame Datei latex-vorspann.tex mit gesetztem Schalter.

\newif\ifkorrekturansicht
\korrekturansichttrue

\input{../tex-inputs/latex-vorspann}


\renewcommand{\erwaehntePersonen}{Personen:  ?? [Partner von Theodore Rottenberg, Ende 1902/Anfang 1903], Paul Goldmann, Theodore Rottenberg, Olga Schnitzler}
\renewcommand{\erwaehnteOrte}{Orte: Berlin, Dessauer Straße, Palasthotel Berlin, Wien}
\renewcommand{\erwaehnteWerke}{}
\section[ Paul Goldmann an Arthur Schnitzler, 17. 2. {[}1903{]}]{Paul Goldmann an Arthur Schnitzler, 17. 2. {[}1903{]}}
\nopagebreak\mylabel{v}
\rehead{ }\normalsize\beginnumbering\briefempfaengerindex{Schnitzler, Arthur@\textsc{Schnitzler, Arthur}!zzzGoldmann, Paul@\emph{von Paul Goldmann}!1903-02-171@{17. 2. {[}1903{]}}|(be}
\toendnotes[C]{\smallbreak\pagebreak[2]}\Standort{DLA, A:Schnitzler, HS.NZ85.1.3173.}
\physDesc{Brief, 1 Blatt, 4 Seiten, 1597 Zeichen
\newline{}Handschrift: blaue Tinte, deutsche Kurrent
\newline{}Schnitzler: 1) mit Bleistift das Jahr »903« vermerkt  2) mit rotem Buntstift eine Unterstreichung}\toendnotes[C]{\smallbreak}
\pstart
           \noindent{}\raggedleft{}{\pb}\textcolor{gray}{\textbf{\textcolor{pink}{DESSAUERSTRASSE 19}{}\ledrightnote{\textcolor{pink}{Dessauer Straße}}}}\pend
           
\pstart
           \textcolor{pink}{Berlin}{}\ledrightnote{\textcolor{pink}{Berlin}}, 17. Februar.\pend
           
\pstart{}Mein lieber Freund,\pend
\pstart
           Ich freue mich unendlich, Dich \label{K_L03363-1v}\edtext{bald
                  \textcolor{pink}{hier}{}\ledrightnote{{$\rightarrow$}\textcolor{pink}{Berlin}} zu ſehen}{\lemma{\textnormal{\emph{bald
                  hier zu ſehen}}}\Cendnote{\textnormal{\textcolor{blue}{Schnitzler} war von 22. 2. 1903 bis 9. 3. 1903 in \textcolor{pink}{Berlin}. \textcolor{blue}{Goldmann} traf er mehrfach, zumindest am 22. 2. 1903, 24. 2. 1903, 25. 2. 1903, 3. 3. 1903, 4. 3. 1903, 7. 3. 1903 und am 9. 3. 1903.}}}\label{K_L03363-1h},
               und werde Dich, wenn ich nichts Gegentheiliges höre, am Sonntag{ }Vormittag gegen 12 Uhr im \label{K_L03363-2v}\edtext{\textcolor{pink}{Palaſthotel}{}\ledrightnote{\textcolor{pink}{Palasthotel Berlin}}}{\lemma{\textnormal{\emph{Palaſthotel}}}\Cendnote{\textnormal{\textcolor{blue}{Schnitzler}s Unterkunft}}}\label{K_L03363-2h} aufſuchen. Du
               kannſt Dir gar nicht denken, wie ſehr ich mich danach ſehne, mit Dir zu beſprechen,
                  \label{K_L03363-3v}\edtext{was mein Herz bedrückt}{\lemma{\textnormal{\emph{was mein Herz bedrückt}}}\Cendnote{\textnormal{Bezug auf \textcolor{blue}{Theodore Rottenberg}s Trennung von ihm, über die sie dann auch sprachen
                     (vgl. A. S.: \emph{Tagebuch}, 22. 2. 1903).}}}\label{K_L03363-3h}.
               Freilich, viel wirſt auch Du mir nicht helfen können. {\pb}Denn Du kannſt mir ja auch nicht das Verlorene
               wiederbringen; und das allein wäre die Heilung. Aber jede Hoffnung iſt vergeblich.
               Ich bin aus dem Leben dieſer \textcolor{blue}{Frau}{}\ledrightnote{{$\rightarrow$}\textcolor{blue}{Theodore Rottenberg}}, \strikeout{die noch} für die ich vor wenig Monaten
               noch Alles bedeutet habe, vollkommen ausgeſtrichen. Sie hat ihr Leben ganz auf den
                  \textcolor{blue}{Andern}{}\ledrightnote{{$\rightarrow$}\textcolor{blue}{?? [Partner von Theodore Rottenberg, Ende 1902/Anfang 1903]}} übertragen, und
               ich höre nur, wie glücklich ſie mit ihm iſt. Ich ſelbſt aber bekomme nicht einmal
               mehr ein Lebenszeichen. Alle meine Briefe, – flehende, reuige, verzweifelte Briefe –
                  {\pb}bleiben ohne Antwort und ſelbſt die Möglichkeit,
               indirekt Nachrichten\strikeout{\textcolor{gray}{×}} von ihr zu erhalten, ſchneidet \textcolor{blue}{ſie}{}\ledrightnote{{$\rightarrow$}\textcolor{blue}{Theodore Rottenberg}} mir ab. Ich verzehre mich in Sehnſucht. Ich warte – und
               ich warte vergebens. Jeder Tag bringt sie dem \textcolor{blue}{Andern}{}\ledrightnote{{$\rightarrow$}\textcolor{blue}{?? [Partner von Theodore Rottenberg, Ende 1902/Anfang 1903]}}{ }\strikeout{nä} näher und treibt ſie weiter von mir fort. Und ich
               muß mir ſagen, daß ich ſelbſt an Allem ſchuld \introOben{}bin\introOben{}, daß ich
               die zärtlichſte und hingebendſte Geliebte in einer finſteren Laune fortgeſtoßen habe,
               nicht ahnend, {\pb}welch’ koſtbaren Schatz ich beſaß,
               was ich jetzt erſt, zu ſpät, eingeſehen habe. Ein Wahnſinniger war ich, – ein
               verblendeter Thor – ein unerfahrener dummer Junge trotz meiner 38 Jahre! {\dotsfour}\pend
           
\pstart
           Reiſe glücklich nach \textcolor{pink}{Berlin}{}\ledrightnote{\textcolor{pink}{Berlin}}, grüße \textsc{\textcolor{blue}{Olga}{}\ledrightnote{\textcolor{blue}{Olga Schnitzler}}} vielmals (auf
               deren \label{K_L03363-4v}\edtext{Ankunft}{\lemma{\textnormal{\emph{Ankunft}}}\Cendnote{\textnormal{\textcolor{blue}{Olga Gussmann} kam am 4. 3. 1903 in \textcolor{pink}{Berlin} an und reiste am 9. 3. 1903 gemeinsam
                  mit \textcolor{blue}{Schnitzler} zurück nach \textcolor{pink}{Wien}.}}}\label{K_L03363-4h} ich mich auch ſchon ſehr freue) und ſei ſelbſt
               von Herzen gegrüßt von {\\[\baselineskip]}Deinem getreuen {\\[\baselineskip]}\spacefill\mbox{Paul Goldmnn}\pend
           \leftskip=0em{}\endnumbering\briefempfaengerindex{Schnitzler, Arthur@\textsc{Schnitzler, Arthur}!zzzGoldmann, Paul@\emph{von Paul Goldmann}!1903-02-171@{17. 2. {[}1903{]}}|)be}\mylabel{h}  \normalsize

\doendnotes{C}
\bigskip
\vfill

\clearpage

\footnotesize

\lohead{\textsc{register}}

% Definiere theindex-Environment komplett neu ohne reledmac
\makeatletter
\renewenvironment{theindex}{%
  \section*{\indexname}%
  \setlength{\parindent}{0pt}%
  \setlength{\parskip}{0pt plus 0.3pt}%
  \let\item\@idxitem
}{%
  \clearpage
}
\makeatother

\IfFileExists{\jobname-pw.ind}{\input{\jobname-pw.ind}}{}

\end{document}

      