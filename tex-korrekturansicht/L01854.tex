%% latex-korrekturansicht-vorspann.tex
%% Vorspann für die Korrekturansicht.
%% Lädt die gemeinsame Datei latex-vorspann.tex mit gesetztem Schalter.

\newif\ifkorrekturansicht
\korrekturansichttrue

\input{../tex-inputs/latex-vorspann}


               \section[Arthur Schnitzler an Albert Ehrenstein, 7. 7. 1909]{ Arthur Schnitzler an Albert Ehrenstein, 7. 7. 1909}\nopagebreak\mylabel{v}\rehead{ }\normalsize\beginnumbering\briefempfaengerindex{Ehrenstein, Albert@\textsc{Ehrenstein, Albert}!zzzSchnitzler, Arthur@\emph{von Arthur Schnitzler}!1909-07-071@{7. 7. 1909}|(be} \toendnotes[C]{\smallbreak\pagebreak[2]} \Standort{Jerusalem, The National Library of Israel, ARC. Ms. Var. 306 1 118.}
\physDesc{Brief, 1 Blatt, 4 Seiten
\newline{}Handschrift: Bleistift, lateinische Kurrent}\pstart
           \noindent{}{\pb}\textcolor{gray}{\textbf{Dr. Arthur Schnitzler}}\hfill \textcolor{pink}{Edlach}{}\ledrightnote{\textcolor{pink}{Edlach}}{ }7/7 09\pend
           \pstart
           \textcolor{gray}{\textbf{\textcolor{pink}{Wien XVIII.
                            Spoettelgasse 7}{}\ledrightnote{\textcolor{pink}{Edmund-Weiß-Gasse}}.}}\hfill \textcolor{pink}{Edlacher Hof}{}\ledrightnote{\textcolor{pink}{Hotel Edlacherhof}}\pend
           \pstart{}Lieber Herr Ehrenstein,\pend\pstart
           die Manuscripte liegen in meiner Wohnung zum Abholen für Sie (unter Ihrem Namen)
                    bereit.\pend
           \pstart
           Im Herbst sprechen wir über die Sachen, we{\geminationn}s Ihnen
                    recht ist. Für heute nur so viel, {\pb}dass ich einen äußern
                    Erfolg gerade dieser letzten Sachen, d. h. insbesondere eine Annahme bei
                        \textcolor{green}{Zeit}{}\ledrightnote{\textcolor{green}{Die Zeit}} oder \textcolor{brown}{Presse}{}\ledrightnote{\textcolor{brown}{Neue Freie Presse}} für nicht wahrscheinlich halte. Mit \textcolor{blue}{Auernh.}{}\ledrightnote{\textcolor{blue}{Raoul Auernheimer}}, der jetzt hier ist, will ich übrigens im
                    allgemeinen über Sie reden, we{\geminationn}
               sie nichts dagegen
                    haben. Auf dieser Bahn scheint mir ja nun {\pb}allerdings
                    Ihre Zukunft nicht zu liegen (ich meine die \textcolor{green}{Zeit}{}\ledrightnote{\textcolor{green}{Die Zeit}} und \textcolor{brown}{Presse}{}\ledrightnote{\textcolor{brown}{Neue Freie Presse}}-Bahn) Ihre
                    Auffassung, dass \introOben{}selbst\introOben{} die Veröffentlichung einer oder
                    der andern Arbeit in einer dieser Blätter Ihre Position bei den Professoren zu
                    Gunsten der Prüfung beeinflussen könnte, theil ich nicht. Sie werden Ihre {\pb}Examen sicher bestehen, auch so.\pend
           \pstart
           – Auf Wiedersehen und beste Grüße. Ihr ergebener{\\[\baselineskip]}\spacefill\mbox{A. S.}\pend
           \leftskip=0em{}\endnumbering\briefempfaengerindex{Ehrenstein, Albert@\textsc{Ehrenstein, Albert}!zzzSchnitzler, Arthur@\emph{von Arthur Schnitzler}!1909-07-071@{7. 7. 1909}|)be}\mylabel{h}  \normalsize

\doendnotes{C}
\bigskip
\vfill

\clearpage

\footnotesize

\lohead{\textsc{register}}

% Definiere theindex-Environment komplett neu ohne reledmac
\makeatletter
\renewenvironment{theindex}{%
  \section*{\indexname}%
  \setlength{\parindent}{0pt}%
  \setlength{\parskip}{0pt plus 0.3pt}%
  \let\item\@idxitem
}{%
  \clearpage
}
\makeatother

\IfFileExists{\jobname-pw.ind}{\input{\jobname-pw.ind}}{}

\end{document}

      