%% latex-korrekturansicht-vorspann.tex
%% Vorspann für die Korrekturansicht.
%% Lädt die gemeinsame Datei latex-vorspann.tex mit gesetztem Schalter.

\newif\ifkorrekturansicht
\korrekturansichttrue

\input{../tex-inputs/latex-vorspann}


\renewcommand{\erwaehntePersonen}{Personen:  ?? [Portier des Hotel Continental Berlin], Olga Schnitzler}
\renewcommand{\erwaehnteOrte}{Orte: Berlin, Edmund-Weiß-Gasse, Hotel Continental (Berlin), Wien}
\renewcommand{\erwaehnteWerke}{}
\section[ Paul Goldmann an Arthur Schnitzler, 7. 2. 1906]{Paul Goldmann an Arthur Schnitzler, 7. 2. 1906}
\nopagebreak\mylabel{v}
\rehead{ }\normalsize\beginnumbering\briefempfaengerindex{Schnitzler, Arthur@\textsc{Schnitzler, Arthur}!zzzGoldmann, Paul@\emph{von Paul Goldmann}!1906-02-071@{7. 2. 1906}|(be}
\toendnotes[C]{\smallbreak\pagebreak[2]}\Standort{DLA, A:Schnitzler, HS.NZ85.1.3175.}
\physDesc{Postkarte
\newline{}Handschrift: 1) blaue Tinte, deutsche Kurrent\hspace{1em}2) blaue Tinte, lateinische Kurrent (\noindent{}Adresse)\hspace{1em}
\newline{}Versand: 1) Stempel: »\nobreak{}\oindex{Berlin@\textbf{Berlin}, \emph{https://www.geonames.org/ontologyP.PPLC}|pwk}Berlin, S. W. 11, 7. 2. 06, 10–11 N.\nobreak{}«.   2) Stempel: »\nobreak{}¹⁸⁄₁ Wi{[}en{]}, 8. II. 06, 5, Bestellt\nobreak{}«. }\toendnotes[C]{\smallbreak}\pstart{}{\pb}Herrn\pend{}\pstart{}Dr. Arthur Schnitzler\pend{}\pstart{}\textcolor{pink}{Wien}{}\ledrightnote{\textcolor{pink}{Wien}}\pend{}\pstart{}\textcolor{pink}{XVIII. Spöttelgaſse 7}{}\ledrightnote{\textcolor{pink}{Edmund-Weiß-Gasse}}.\pend{}
{\bigskip}
\pstart
           \noindent{}{\pb}\textcolor{pink}{Berlin}{}\ledrightnote{\textcolor{pink}{Berlin}}, 7. Februar. Lieber Freund,
               Als ich heut um 5 Uhr im \textsc{\textcolor{pink}{Hotel Continental}{}\ledrightnote{\textcolor{pink}{Hotel Continental (Berlin)}}} vorſprach, mußte ich leider vom \label{K_L03239-11v}\edtext{\textcolor{blue}{Portier}{}\ledrightnote{\textcolor{blue}{?? [Portier des Hotel Continental Berlin]}}}{\lemma{\textnormal{\emph{Portier}}}\Cendnote{\textnormal{nicht ermittelt}}}\label{K_L03239-11h} erfahren, daß Du bereits \label{K-L03239-1v}\edtext{abgereiſt}{\lemma{\textnormal{\emph{abgereiſt}}}\Cendnote{\textnormal{\textcolor{blue}{Schnitzler} war seit 4. 2. 1906 in \textcolor{pink}{Berlin}. Er reiste am 7. 2. 1906 zurück
                  nach \textcolor{pink}{Wien}, wo er am 8. 2. 1906
                  ankam.}}}\label{K-L03239-1h} ſeieſt. Es thut mir unendlich leid, Dich heut und geſtern{ }\label{K-L03239-2v}\edtext{verfehlt}{\lemma{\textnormal{\emph{verfehlt}}}\Cendnote{\textnormal{vgl. A. S.: \emph{Tagebuch}, 6. 2. 1906}}}\label{K-L03239-2h} zu
               haben. Ich danke Dir für Deinen lieben Beſuch, hoffe, Dich bald \label{K-L03239-3v}\edtext{wieder \textcolor{pink}{hier}{}\ledrightnote{{$\rightarrow$}\textcolor{pink}{Berlin}}}{\lemma{\textnormal{\emph{wieder hier}}}\Cendnote{\textnormal{\textcolor{blue}{Schnitzler} traf \textcolor{blue}{Goldmann} am 21. 2. 1906 in
                     \textcolor{pink}{Berlin} wieder.}}}\label{K-L03239-3h} zu ſehen, und bin mit
               herzlichen Grüßen an Dich und Deine \textcolor{blue}{Frau}{}\ledrightnote{{$\rightarrow$}\textcolor{blue}{Olga Schnitzler}}{\\}Dein {\\}\spacefill\mbox{Paul Goldmann.}\pend
           \endnumbering\briefempfaengerindex{Schnitzler, Arthur@\textsc{Schnitzler, Arthur}!zzzGoldmann, Paul@\emph{von Paul Goldmann}!1906-02-071@{7. 2. 1906}|)be}\mylabel{h}  \normalsize

\doendnotes{C}
\bigskip
\vfill

\clearpage

\footnotesize

\lohead{\textsc{register}}

% Definiere theindex-Environment komplett neu ohne reledmac
\makeatletter
\renewenvironment{theindex}{%
  \section*{\indexname}%
  \setlength{\parindent}{0pt}%
  \setlength{\parskip}{0pt plus 0.3pt}%
  \let\item\@idxitem
}{%
  \clearpage
}
\makeatother

\IfFileExists{\jobname-pw.ind}{\input{\jobname-pw.ind}}{}

\end{document}

      