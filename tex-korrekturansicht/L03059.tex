%% latex-korrekturansicht-vorspann.tex
%% Vorspann für die Korrekturansicht.
%% Lädt die gemeinsame Datei latex-vorspann.tex mit gesetztem Schalter.

\newif\ifkorrekturansicht
\korrekturansichttrue

\input{../tex-inputs/latex-vorspann}


\renewcommand{\erwaehntePersonen}{Personen: Richard Beer-Hofmann, Jean-Gaspard Deburau, Marie Glümer, Yvette Guilbert, Paul Lindau, Rudolf Renvers, Camille Saint-Saëns, Max Schiller, Olga Schnitzler, Elisabeth Steinrück, Ernst von Wolzogen}
\renewcommand{\erwaehnteInstitutionen}{Institutionen: Berliner Theater, Überbrettl}
\renewcommand{\erwaehnteOrte}{Orte: Berlin, Dessauer Straße, Frankreich, Wien, [Sanatorium]}
\renewcommand{\erwaehnteWerke}{Werke: Der blinde Geronimo und sein Bruder, Die Zeit. Wiener Wochenschrift, Frau Bertha Garlan. Roman, Jugend, Neue Deutsche Rundschau, Sylvesternacht. Ein Dialog}
\section[ Paul Goldmann an Arthur Schnitzler, 18. 2. {[}1901{]}]{Paul Goldmann an Arthur Schnitzler, 18. 2. {[}1901{]}}
\nopagebreak\mylabel{v}
\rehead{ }\normalsize\beginnumbering\briefempfaengerindex{Schnitzler, Arthur@\textsc{Schnitzler, Arthur}!zzzGoldmann, Paul@\emph{von Paul Goldmann}!1901-02-181@{18. 2. {[}1901{]}}|(be}
\toendnotes[C]{\smallbreak\pagebreak[2]}\Standort{DLA, A:Schnitzler, HS.NZ85.1.3171.}
\physDesc{Brief, 1 Blatt, 3 Seiten
\newline{}Handschrift: blaue Tinte, deutsche Kurrent
\newline{}Schnitzler: 1) mit Bleistift das Jahr »{[}1{]}901« vermerkt  2) mit rotem Buntstift sechs Unterstreichungen}\toendnotes[C]{\smallbreak}
\pstart
           \noindent{}\raggedleft{}{\pb}\textsc{\textcolor{pink}{DESSAUERSTRASSE 19}{}\ledrightnote{\textcolor{pink}{Dessauer Straße}}}\pend
           
\pstart
           \textcolor{pink}{Berlin}{}\ledrightnote{\textcolor{pink}{Berlin}}, 18. Februar.\pend
           
\pstart\center{}Mein lieber Freund,\pend
\pstart
           Ich war Freitag bei \textsc{\textcolor{blue}{Mizzi Gl.}{}\ledrightnote{\textcolor{blue}{Marie Glümer}}}, ehe ſie ins \label{K_L03059-7v}\edtext{\textcolor{pink}{Sanatorium}{}\ledrightnote{{$\rightarrow$}\textcolor{pink}{[Sanatorium]}}}{\lemma{\textnormal{\emph{Sanatorium}}}\Cendnote{\textnormal{nicht ermittelt}}}\label{K_L03059-7h} ging. Seither
               keine Nachricht. Auch ich verſtehe abſolut nicht, was ſie hat, bin aber feſt
               überzeugt, daß es nicht \label{K_L03059-1v}\edtext{\textsc{Neuralgie}}{\lemma{\textnormal{\emph{Neuralgie}}}\Cendnote{\textnormal{siehe Paul Goldmann an Arthur Schnitzler, 12. 2. [1901]}}}\label{K_L03059-1h} ſein kann. Das arme \textcolor{blue}{Mädel}{}\ledrightnote{{$\rightarrow$}\textcolor{blue}{Marie Glümer}} iſt ſehr heruntergekommen. Ich habe immer eine Blutkrankheit
               vermuthet, und aus den vagen Andeutungen, die \label{K_L03059-5v}\edtext{\textsc{\textcolor{blue}{Renvers}{}\ledrightnote{\textcolor{blue}{Rudolf Renvers}}}}{\lemma{\textnormal{\emph{Renvers}}}\Cendnote{\textnormal{siehe Paul Goldmann an Arthur Schnitzler, 14. 2. [1901]}}}\label{K_L03059-5h} gemacht zu haben ſcheint, höre ich etwas wie eine Beſtätigung heraus
               (Blutzerſetzung?). Ich kann zu \textsc{\textcolor{blue}{Renvers}{}\ledrightnote{\textcolor{blue}{Rudolf Renvers}}} nicht gehen. \label{K_L03059-4v}\edtext{\begin{otherlanguage}{french}\textsc{À quel titre}\end{otherlanguage}}{\lemma{\textnormal{\emph{À quel titre}}}\Cendnote{\textnormal{französisch: auf welcher Grundlage, mit
                  welchem Recht}}}\label{K_L03059-4h}? Aber ich hoffe doch noch einen Weg zu finden, um mich an
               mediziniſcher Quelle zu infomiren.\pend
           
\pstart
           Daß Du den \label{K_L03059-3v}\edtext{Plan haſt herzukommen}{\lemma{\textnormal{\emph{Plan haſt herzukommen}}}\Cendnote{\textnormal{\textcolor{blue}{Schnitzler} war zwischen 3. 3. 1901 und 10. 3. 1901 in \textcolor{pink}{Berlin}.}}}\label{K_L03059-3h}, iſt ſehr ſchön. Ich hoffe, Du
               führſt ihn aus.\pend
           
\pstart
           {\pb}Es iſt nicht unmöglich, daß ich für \textsc{\textcolor{blue}{Olga}{}\ledrightnote{\textcolor{blue}{Olga Schnitzler}}} etwas bei \label{K_L03059-12v}\edtext{\textsc{\textcolor{blue}{Lindau}{}\ledrightnote{\textcolor{blue}{Paul Lindau}}}}{\lemma{\textnormal{\emph{Lindau}}}\Cendnote{\textnormal{\textcolor{blue}{Paul Lindau}, Leiter des \emph{\textcolor{brown}{Berliner Theater}}s, das wohl von \textcolor{blue}{Olga}, Schauspielerin und Sängerin, in Betracht gezogen
                  wurde}}}\label{K_L03059-12h} thun könnte. Aber Du müßteſt auch eingreifen, Dein Wort würde mehr
               ins Gewicht fallen als meines. \label{K_L03059-13v}\edtext{\textsc{\textcolor{blue}{Wolzogen}{}\ledrightnote{\textcolor{blue}{Ernst von Wolzogen}}}}{\lemma{\textnormal{\emph{Wolzogen}}}\Cendnote{\textnormal{\textcolor{blue}{Ernst von Wolzogen}, der 1901 das literarische Kabarett \emph{\textcolor{brown}{Überbrettl}} (auch bekannt als \textcolor{brown}{Wolzogen-Theater} und \textcolor{brown}{Buntes Theater}) in \textcolor{pink}{Berlin} gegründet hatte}}}\label{K_L03059-13h} kenne ich perſönlich. Auch bei ihm könnteſt
               Du viel ausrichten, ich könnte nur mithelfen. Aber wäre das \textcolor{brown}{Überbrettl}{}\ledrightnote{\textcolor{brown}{Überbrettl}} denn eine Exiſtenz? Und \strikeout{\textcolor{gray}{biſt}{ }\textcolor{gray}{×}} iſt die \label{K_L03059-11v}\edtext{\textcolor{blue}{Kleine}{}\ledrightnote{{$\rightarrow$}\textcolor{blue}{Elisabeth Steinrück}}}{\lemma{\textnormal{\emph{Kleine}}}\Cendnote{\textnormal{womöglich Bezug auf \textcolor{blue}{Elisabeth alias Liesl Gussmann}, später Steinrück, \textcolor{blue}{Olga}s jüngere Schwester, ebenfalls Schauspielerin, siehe Paul Goldmann an Arthur Schnitzler, 6. 4. [1901]}}}\label{K_L03059-11h} mit ihren Studien ſchon fertig?\pend
           
\pstart
           \textsc{\textcolor{blue}{Yvette Guilbert}{}\ledrightnote{\textcolor{blue}{Yvette Guilbert}}}, deren \textcolor{blue}{Mann}{}\ledrightnote{{$\rightarrow$}\textcolor{blue}{Max Schiller}} Dich kennt
               und liebt (Deine Werke nämlich), läßt Dich fragen, ob Du ihr nicht einen \label{K_L03059-17v}\edtext{Einakter ſchreiben}{\lemma{\textnormal{\emph{Einakter ſchreiben}}}\Cendnote{\textnormal{nicht geschehen}}}\label{K_L03059-17h} möchteſt? Eine \label{K_L03059-19v}\edtext{\textsc{Pierrot}}{\lemma{\textnormal{\emph{Pierrot}}}\Cendnote{\textnormal{männlicher Komödienfigurentyp, der
                  insbesondere durch den \textcolor{pink}{fran}zösischen Pantomimen \textcolor{blue}{Jean-Gaspard
                     Deburau} berühmt wurde}}}\label{K_L03059-19h}-Komödie, und zwar einen revolutionären \textsc{Pierrot}. Keine \textsc{Pantomine}. Die
               Komödie ſoll von einem großen \textcolor{pink}{fran}{}\ledrightnote{{$\rightarrow$}\textcolor{pink}{Frankreich}}zöſiſchen Componiſten (vielleicht \textsc{\textcolor{blue}{Saint-Saëns}{}\ledrightnote{\textcolor{blue}{Camille Saint-Saëns}}}) {\pb}in Muſik geſetzt werden. Bitte, antworte mir
               ſofort, da ich der \textsc{Mad. \textcolor{blue}{Yvette}{}\ledrightnote{\textcolor{blue}{Yvette Guilbert}}} noch Beſcheid geben möchte, ſolange ſie \textcolor{pink}{hier}{}\ledrightnote{{$\rightarrow$}\textcolor{pink}{Berlin}} iſt.\pend
           
\pstart
           Den \label{K_L03059-21v}\edtext{\textcolor{green}{Roman}{}\ledrightnote{{$\rightarrow$}\textcolor{green}{Frau Bertha Garlan. Roman}}}{\lemma{\textnormal{\emph{Roman}}}\Cendnote{\textnormal{\textcolor{blue}{Arthur Schnitzler}: \emph{\textcolor{green}{Frau Bertha Garlan. Roman}}. In: \emph{\textcolor{green}{Neue Deutsche Rundschau}}, Jg. 12, H. 1, Januar, S. 41–64; H. 2, Februar, S. 181–206; H. 3, März,
                     S. 237–272, 1901.}}}\label{K_L03059-21h} in der \textcolor{green}{N. D. Rundſchau}{}\ledrightnote{\textcolor{green}{Neue Deutsche Rundschau}} leſe
               ich nicht, weil ich mir das \textcolor{green}{Werk}{}\ledrightnote{{$\rightarrow$}\textcolor{green}{Frau Bertha Garlan. Roman}} nicht will in Fortſetzungen zerhacken laſſen. Sehr reizend war der
                  \label{K_L03059-24v}\edtext{\textcolor{green}{Dialog}{}\ledrightnote{{$\rightarrow$}\textcolor{green}{Sylvesternacht. Ein Dialog}}}{\lemma{\textnormal{\emph{Dialog}}}\Cendnote{\textnormal{\textcolor{blue}{Arthur Schnitzler}: \emph{\textcolor{green}{Sylvesternacht. Ein Dialog}}. In: \emph{\textcolor{green}{Jugend}}, Jg. 6, Nr. 8, 18. 2. 1901, S. 118–119, 121–122.}}}\label{K_L03059-24h} in der »\textcolor{green}{Jugend}{}\ledrightnote{\textcolor{green}{Jugend}}«. Weniger gefallen hat mir der \label{K_L03059-22v}\edtext{»\textcolor{green}{Blinde
                     \textsc{Hieronymo}}{}\ledrightnote{\textcolor{green}{Der blinde Geronimo und sein Bruder}}}{\lemma{\textnormal{\emph{»Blinde
                     Hieronymo}}}\Cendnote{\textnormal{\textcolor{blue}{Arthur Schnitzler}: \emph{\textcolor{green}{Der blinde Hieronymo und sein Bruder}}. In: \emph{\textcolor{green}{Die Zeit}}, Jg. 25–16, Nr. 325, 22. 12. 1900, S. 190–191; Nr. 326, 29. 12. 1900, S. 207–208; Nr. 327, 5. 1. 1901, S. 15–16; Nr. 328, 12. 1. 1901, S. 31–32.}}}\label{K_L03059-22h}«! Die
               Geſchichte iſt geiſtvoll ausgedacht, bleibt aber weit zurück hinter der wilden Tragik
               des \label{K_L03059-15v}\edtext{Originals}{\lemma{\textnormal{\emph{Originals}}}\Cendnote{\textnormal{siehe Paul Goldmann an Arthur Schnitzler, 28. 8. [1900]}}}\label{K_L03059-15h}.\pend
           
\pstart
           \textsc{\textcolor{blue}{Richard}{}\ledrightnote{\textcolor{blue}{Richard Beer-Hofmann}}} hat mir nicht geſchrieben. Sag’ ihm auch nichts mehr. Der Teufel ſoll ihn
               holen!\pend
           
\pstart
           Viele treue Grüße! {\\[\baselineskip]}Dein \spacefill\mbox{Paul Goldmann.}\pend
           \leftskip=0em{}\endnumbering\briefempfaengerindex{Schnitzler, Arthur@\textsc{Schnitzler, Arthur}!zzzGoldmann, Paul@\emph{von Paul Goldmann}!1901-02-181@{18. 2. {[}1901{]}}|)be}\mylabel{h}
\begin{anhang}
\end{anhang}\normalsize

\doendnotes{C}
\bigskip
\vfill

\clearpage

\footnotesize

\lohead{\textsc{register}}

% Definiere theindex-Environment komplett neu ohne reledmac
\makeatletter
\renewenvironment{theindex}{%
  \section*{\indexname}%
  \setlength{\parindent}{0pt}%
  \setlength{\parskip}{0pt plus 0.3pt}%
  \let\item\@idxitem
}{%
  \clearpage
}
\makeatother

\IfFileExists{\jobname-pw.ind}{\input{\jobname-pw.ind}}{}

\end{document}

      