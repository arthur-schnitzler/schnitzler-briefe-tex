%% latex-korrekturansicht-vorspann.tex
%% Vorspann für die Korrekturansicht.
%% Lädt die gemeinsame Datei latex-vorspann.tex mit gesetztem Schalter.

\newif\ifkorrekturansicht
\korrekturansichttrue

\input{../tex-inputs/latex-vorspann}


\renewcommand{\erwaehntePersonen}{Personen: Richard Beer-Hofmann, Rudolf Lothar, Marie Reinhard}
\renewcommand{\erwaehnteOrte}{Orte: Astor House Hotel [Tianjin], Berlin, China, Deutsches Theater Berlin, Salzburg, Tianjin, Wien}
\renewcommand{\erwaehnteWerke}{Werke: Beim Tao-tai Tsai von Shanghai, Briefe an eine Dame, Chinesische Zeitungen, Das Vermächtnis. Schauspiel in drei Akten, Der General Tscheng-Ki-tong, Der Tod Georgs, Der Tod Georgs. Fragment, Der französisch-chinesische Zwischenfall in Shanghai, Der französisch-chinesische Zwischenfall in Shanghai, Die Pest in Hongkong, Die Wage. Eine Wiener Wochenschrift, Die Zeit. Wiener Wochenschrift, Die deutschen Militär-Instruktoren in China, Ein Kapitel über chinesische Eisenbahnen. Chinesische Eisenbahnen und deutsche Versäumnisse, Ein Kapitel über chinesische Eisenbahnen. Die Bahn in Shanghai nach Wu-sung, Ein Sommer in China. Reisebilder, Eine Unterredung mit dem Tao-tai Wang, dem Sekretär des Vicekönigs von Canton, Eine Unterredung mit dem Tao-tai Wang, dem Sekretär des Vicekönigs von Canton [zweiter Teil], Frankfurter Zeitung, Heimkehr, In Kiautschou. I, In Kiautschou. II, In Kiautschou. III, In Kiautschou. IV [letzter Teil], In Ostasien. Reiseskizzen, In Ostasien. Reiseskizzen. Anlage der Stadt Peking, In Ostasien. Reiseskizzen. Anlage der Stadt Peking (Schluß), In Ostasien. Reiseskizzen. Auf dem Perlfluß nach Canton-Shameen, In Ostasien. Reiseskizzen. Auf dem Perlfluß nach Canton-Shameen [zweiter Teil], In Ostasien. Reiseskizzen. Auf dem Yang-tse-Kiang, In Ostasien. Reiseskizzen. Auf dem Yang-tse-Kiang [zweiter Teil], In Ostasien. Reiseskizzen. Canton{\rufezeichen}, In Ostasien. Reiseskizzen. Canton{\rufezeichen} [zweiter Teil], In Ostasien. Reiseskizzen. Chinesisches Nachtleben, In Ostasien. Reiseskizzen. Chinesisches Nachtleben [zweiter Teil], In Ostasien. Reiseskizzen. Hankow, In Ostasien. Reiseskizzen. Hongkong, In Ostasien. Reiseskizzen. Hongkong [zweiter Teil], In Ostasien. Reiseskizzen. Im Golf von Pe-tschi-li, In Ostasien. Reiseskizzen. Im Golf von Pe-tschi-li [zweiter Teil], In Ostasien. Reiseskizzen. In Tsientsin, In Ostasien. Reiseskizzen. Shanghai, In Ostasien. Reiseskizzen. Shanghai [zweiter Teil], In Ostasien. Reiseskizzen. Singapore, In Ostasien. Reiseskizzen. Straßenleben in Peking, In Ostasien. Reiseskizzen. Straßenleben in Peking (Schluß), In Ostasien. Reiseskizzen. Tsientsin (Fortsetzung), In Ostasien. Reiseskizzen. Tsientsin (Schluß), In Ostasien. Reiseskizzen. Von Hongkong nach Shanghai, In Ostasien. Reiseskizzen. Von Hongkong nach Shanghai [zweiter Teil], In Ostasien. Reiseskizzen. Von Tschifu nach Tientsin, In Ostasien. Reiseskizzen. Von Tschifu nach Tientsin [zweiter Teil], In Ostasien. Reiseskizzen. Wu-tschang, In Ostasien. Reiseskizzen. Wu-tschang [zweiter Teil], Kiautschou-Eindrücke. I. Wie man ankommt, Kiautschou-Eindrücke. I. Wie man ankommt [zweiter Teil], Kiautschou-Eindrücke. II. Tsintau, Kiautschou-Eindrücke. II. Tsintau [zweiter Teil], Nach Ostasien. Reiseskizzen, Nach Ostasien. Reiseskizzen, Nach Ostasien. Reiseskizzen, Nach Ostasien. Reiseskizzen. Eine Nacht und ein Morgen in Colombo}
\section[ Paul Goldmann an Arthur Schnitzler, 25. 9. 1898]{Paul Goldmann an Arthur Schnitzler, 25. 9. 1898}
\nopagebreak\mylabel{v}
\rehead{ }\normalsize\beginnumbering\briefempfaengerindex{Schnitzler, Arthur@\textsc{Schnitzler, Arthur}!zzzGoldmann, Paul@\emph{von Paul Goldmann}!1898-09-251@{25. 9. 1898}|(be}
\toendnotes[C]{\smallbreak\pagebreak[2]}\Standort{DLA, A:Schnitzler, HS.NZ85.1.3168.}
\physDesc{Brief, 2 Blätter, 7 Seiten
\newline{}Handschrift: blaue Tinte, deutsche Kurrent
\newline{}Schnitzler: 1) mit Bleistift das Jahr »98« vermerkt  2) mit rotem Buntstift zwei Unterstreichungen}\toendnotes[C]{\smallbreak}
\pstart
           \noindent{}\centering{}{\pb}\textcolor{pink}{\textsc{\textcolor{gray}{\textbf{Astor House Hôtel, L\textsuperscript{td}.}}}}{}\ledrightnote{\textcolor{pink}{Astor House Hotel [Tianjin]}}\pend
           
\pstart
           \noindent{}\raggedleft{}\textcolor{gray}{\textbf{\textcolor{pink}{Tientsin}{}\ledrightnote{\textcolor{pink}{Tianjin}},}}{ }25. September \textcolor{gray}{\textbf{189}}8\pend
           
\pstart\center{}Mein lieber Freund,\pend
\pstart
           Ich bin jetzt ſehr außerhalb der Poſt-Verbindungen u. habe daher erſt dieſer Tage
               Deinen lieben Brief aus \label{K_L02858-1v}\edtext{\textsc{\textcolor{pink}{Salzburg}{}\ledrightnote{\textcolor{pink}{Salzburg}}}}{\lemma{\textnormal{\emph{Salzburg}}}\Cendnote{\textnormal{siehe A. S.: \emph{Tagebuch}, 28. 7. 1898}}}\label{K_L02858-1h} vom 28. Juli erhalten. Inzwiſchen biſt Du ja
               längſt glücklich heimgekehrt; und wenn Du meinen Brief erhältſt, iſt wohl auch ſchon
               die \label{K_L02858-2v}\edtext{\begin{otherlanguage}{french}\textsc{Première}\end{otherlanguage}{ }Deines neuen \textcolor{green}{Stück}{}\ledrightnote{{$\rightarrow$}\textcolor{green}{Das Vermächtnis. Schauspiel in drei Akten}}es}{\lemma{\textnormal{\emph{Première … Stückes}}}\Cendnote{\textnormal{\emph{\textcolor{green}{Das Vermächtnis}} wurde am 8. 10. 1898 am \textcolor{pink}{Deutschen Theater} in \textcolor{pink}{Berlin} uraufgeführt.}}}\label{K_L02858-2h} vorüber und Du biſt um einen
               neuen Erfolg reicher.\pend
           
\pstart
           {\pb}Es iſt heut wieder
               ein Tag, wo ich unſägliches Heimweh habe. Manchmal erwache ich wie aus einen Traume
               und frage\strikeout{,} mich, was ich denn eigentlich hier in
               dieſem \textcolor{pink}{Lande}{}\ledrightnote{{$\rightarrow$}\textcolor{pink}{China}} mache? Noch dazu
               bin ich ſeit einigen Wochen recht elend. Die \label{K_L02858-3v}\edtext{\textsc{Dysenterie}}{\lemma{\textnormal{\emph{Dysenterie}}}\Cendnote{\textnormal{Darmentzündung}}}\label{K_L02858-3h} iſt mir in den
               Leib gefahren\substVorne{}\textsuperscript{\textcolor{gray}{,}}\substDazwischen{}und\substHinten{} geht natürlich nicht wieder weg. Das iſt eine ſchlimme Geſchichte. Allein im
               fremden \textcolor{pink}{Lande}{}\ledrightnote{{$\rightarrow$}\textcolor{pink}{China}} und auch noch
               krank dazu und die Heimath ſo weit! {\dotsfive}\pend
           
\pstart
           {\pb}Ich danke Dir von Herzen für die Aufmerkſamkeit,
               mit der Du meine \label{K_L02858-4v}\edtext{\textcolor{green}{Arbeiten}{}\ledrightnote{{$\rightarrow$}\textcolor{green}{In Ostasien. Reiseskizzen}}}{\lemma{\textnormal{\emph{Arbeiten}}}\Cendnote{\textnormal{\textcolor{blue}{Schnitzler} dürfte regelmäßig die \emph{\textcolor{green}{Frankfurter Zeitung}} gelesen haben, in der \textcolor{blue}{Goldmann}s \textcolor{green}{Feuilletons} (unter Angabe des vollen Namens) mit dem 
                  Titel \emph{\textcolor{green}{In Ostasien. Reiseskizzen}} erschienen.
                  Diese erschienen
                  am \textcolor{green}{24. 4. 1898}, \textcolor{green}{1. 5. 1898}, \textcolor{green}{19. 5. 1898}, \textcolor{green}{22. 5. 1898}, \textcolor{green}{12. 6. 1898}, \textcolor{green}{16. 6. 1898}, \textcolor{green}{17. 6. 1898}, \textcolor{green}{23. 6. 1898}, \textcolor{green}{24. 6. 1898}, \textcolor{green}{29. 6. 1898}, \textcolor{green}{30. 6. 1898}, \textcolor{green}{14. 7. 1898}, \textcolor{green}{15. 7. 1898}, \textcolor{green}{24. 7. 1898}, \textcolor{green}{26. 7. 1898}, \textcolor{green}{7. 8. 1898}, \textcolor{green}{9. 8. 1898}, \textcolor{green}{21. 8. 1898}, \textcolor{green}{22. 8. 1898}, \textcolor{green}{28. 8. 1898}, \textcolor{green}{30. 8. 1898}, \textcolor{green}{31. 8. 1898}, \textcolor{green}{5. 10. 1898}, \textcolor{green}{6. 10. 1898}, \textcolor{green}{8. 10. 1898}, \textcolor{green}{9. 10. 1898}, \textcolor{green}{16. 10. 1898}, \textcolor{green}{18. 10. 1898}, \textcolor{green}{30. 10. 1898} und \textcolor{green}{31. 10. 1898}, \textcolor{green}{13. 11. 1898}, \textcolor{green}{14. 11. 1898}, \textcolor{green}{15. 11. 1898}, \textcolor{green}{18. 12. 1898}, \textcolor{green}{20. 12. 1898}, \textcolor{green}{25. 12. 1898} und am \textcolor{green}{28. 12. 1898}. Am \textcolor{green}{30. 4. 1899} erschien mit \emph{\textcolor{green}{Heimkehr}} noch ein
                  Schlussartikel, der womöglich bereits für die Buchausgabe der Feuilletons – \emph{\textcolor{green}{Ein Sommer in China}} verfasst war. 
                  Zusätzlich erschienen tagesaktuelle Berichterstattungen,  \textcolor{green}{Berichterstattungen}, die unter Angabe des Kürzels »\textcolor{blue}{G}« publiziert wurden. Sie erschienen
                   am \textcolor{green}{8. 6. 1898}, \textcolor{green}{23. 6. 1898}, \textcolor{green}{21. 7. 1898}, \textcolor{green}{23. 7. 1898}, \textcolor{green}{3. 8. 1898}, \textcolor{green}{4. 8. 1898}, \textcolor{green}{17. 8. 1898}, \textcolor{green}{25. 8. 1898}, \textcolor{green}{9. 9. 1898}, \textcolor{green}{23. 9. 1898}, \textcolor{green}{24. 9. 1898}, \textcolor{green}{25. 9. 1898}, \textcolor{green}{26. 9. 1898} und \textcolor{green}{25. 10. 1898} und darüber hinaus.}}}\label{K_L02858-4h} verfolgſt. Du nennſt ſie »intereſſant« und ahnſt
               gewiß nicht, daß das ihre Verurtheilung iſt. Intereſſant iſt die Rubrik »Vermiſchtes«
               in den Zeitungen, die von einem wunderbaren Walfiſch-Fang berichtet oder vom
               tätowirten Indianer. Die unbeſchreibliche künſtleriſche Anſtrengung, die ich auf
               meine Arbeiten verwende, das Beſtreben, einfach, klar und doch maleriſch
               darzuſtellen, {\pb}kommt alſo nicht zum Ausdruck. Wenn
               ſelbſt Du es nicht ſiehſt, ſo beweiſt das, daß meine Arbeiten verfehlt ſind, was ich
               von Anfang an \strikeout{\textcolor{gray}{×}\-\textcolor{gray}{×}\-\textcolor{gray}{×}\-\textcolor{gray}{×}\-\textcolor{gray}{×}\-\textcolor{gray}{×}\-\textcolor{gray}{×}} geahnt habe. Es iſt ſehr bitter, liebſter Freund, intereſſant zu
               ſchreiben.\pend
           
\pstart
           Mein Brief findet Dich hoffentlich in guter, froher Arbeit und in heller Stimmung.
               Denke Dir nur, welch’ ein \label{K_L02858-6v}\edtext{\textsc{Schemen}}{\lemma{\textnormal{\emph{Schemen}}}\Cendnote{\textnormal{Trugbild}}}\label{K_L02858-6h}{ }\strikeout{alle} alle Deine Leiden ſein müſſen, {\pb}wenn eine einzige \label{K_L02858-8v}\edtext{Reiſe}{\lemma{\textnormal{\emph{Reiſe}}}\Cendnote{\textnormal{siehe Paul Goldmann an Arthur Schnitzler, 16. 5. 1898}}}\label{K_L02858-8h} von \textcolor{pink}{Wien}{}\ledrightnote{\textcolor{pink}{Wien}} nach \textcolor{pink}{Salzburg}{}\ledrightnote{\textcolor{pink}{Salzburg}} ſie verblaſſen macht. Quäle Dich nicht und mache Dir
               einen frohen Winter!\pend
           
\pstart
           Grüß’ mir den \textsc{\textcolor{blue}{Richard}{}\ledrightnote{\textcolor{blue}{Richard Beer-Hofmann}}}! Ich \strikeout{h\textcolor{gray}{×}\-\textcolor{gray}{×}\-\textcolor{gray}{×}} freue mich, daß er das \label{K_L02858-11v}\edtext{dritte
               Capitel des »\textcolor{green}{Götterliebling}{}\ledrightnote{{$\rightarrow$}\textcolor{green}{Der Tod Georgs}}«}{\lemma{\textnormal{\emph{dritte … »Götterliebling«}}}\Cendnote{\textnormal{Als \textcolor{blue}{Schnitzler} am 28. 7. 1898 in \textcolor{pink}{Salzburg} war, las ihm \textcolor{blue}{Beer-Hofmann}
                  das dritte Kapitel des \textcolor{green}{Götterliebling}s vor. Die \textcolor{green}{Erzählung} erschien zuerst zwischen 4. 11. 1899 und 25. 11. 1899 als
                  Fragment unter dem Titel \emph{\textcolor{green}{Der Tod Georgs}} in
                  der \emph{\textcolor{green}{Zeit}}.}}}\label{K_L02858-11h} beendet hat. Nur fürchte
               ich, im vierten \textcolor{green}{Capitel}{}\ledrightnote{{$\rightarrow$}\textcolor{green}{Der Tod Georgs}} wird
               der Held wieder einſchlafen {\pb}und einige Jahrhundert
               Weltgeſchichte \strikeout{t\textcolor{gray}{r}} träumen, und das wird \substVorne{}\textsuperscript{\textcolor{gray}{wie}d\textcolor{gray}{er}}{\allowbreak}\substDazwischen{}noch\substHinten{} recht lang werden.\pend
           
\pstart
           Man ſandte mir hierher einen \label{K_L02858-17v}\edtext{\textcolor{green}{Artikel}{}\ledrightnote{{$\rightarrow$}\textcolor{green}{Briefe an eine Dame}}}{\lemma{\textnormal{\emph{Artikel}}}\Cendnote{\textnormal{\textcolor{blue}{Rudolf Lothar}: \emph{\textcolor{green}{Briefe an eine Dame}}. In: \emph{\textcolor{green}{Die Wage. Eine Wiener Wochenschrift}}, Jg. 1, Nr. 26,
                        25. 6. 1898, S. 439–440.}}}\label{K_L02858-17h} von
                  \textsc{\textcolor{blue}{Rudolf Lothar}{}\ledrightnote{\textcolor{blue}{Rudolf Lothar}}} über Dich in der »\textcolor{green}{Wage}{}\ledrightnote{\textcolor{green}{Die Wage. Eine Wiener Wochenschrift}}«. Wenn Du den \textcolor{blue}{Autor}{}\ledrightnote{{$\rightarrow$}\textcolor{blue}{Rudolf Lothar}} ſiehſt, ſo grüße ihn von
               mir und ſage ihm, meines Wiſſens ſei noch nie über Dich ein ähnlicher Blödſinn
               geſchrieben worden. Auch erfahre ich daraus, daß \strikeout{D} Du
                  {\pb}durch \textsc{\textcolor{blue}{Rudolf Lothar}{}\ledrightnote{\textcolor{blue}{Rudolf Lothar}}} zum Schreiben ermuntert worden biſt. Jetzt weiß ich, warum Du ein Dichter
               biſt!\pend
           
\pstart
           Grüß’ Dich Gott, liebſter Freund!\pend
           
\pstart
           Dein treuer {\\[\baselineskip]}\spacefill\mbox{Paul Goldmann}\pend
           \leftskip=0em{}
\pstart
           \noindent{}Viele Grüße an Deine \textcolor{blue}{Freundin}{}\ledrightnote{{$\rightarrow$}\textcolor{blue}{Marie Reinhard}}!\pend
           \endnumbering\briefempfaengerindex{Schnitzler, Arthur@\textsc{Schnitzler, Arthur}!zzzGoldmann, Paul@\emph{von Paul Goldmann}!1898-09-251@{25. 9. 1898}|)be}\mylabel{h}  \normalsize

\doendnotes{C}
\bigskip
\vfill

\clearpage

\footnotesize

\lohead{\textsc{register}}

% Definiere theindex-Environment komplett neu ohne reledmac
\makeatletter
\renewenvironment{theindex}{%
  \section*{\indexname}%
  \setlength{\parindent}{0pt}%
  \setlength{\parskip}{0pt plus 0.3pt}%
  \let\item\@idxitem
}{%
  \clearpage
}
\makeatother

\IfFileExists{\jobname-pw.ind}{\input{\jobname-pw.ind}}{}

\end{document}

      