%% latex-korrekturansicht-vorspann.tex
%% Vorspann für die Korrekturansicht.
%% Lädt die gemeinsame Datei latex-vorspann.tex mit gesetztem Schalter.

\newif\ifkorrekturansicht
\korrekturansichttrue

\input{../tex-inputs/latex-vorspann}


\renewcommand{\erwaehntePersonen}{Personen: Otto Brahm, Leo Ebermann, Gerhart Hauptmann, Siegfried Loewy, Wilhelm Meyer-Förster, Moriz Neuda, Felix Salten, Olga Schnitzler, Elisabeth Steinrück}
\renewcommand{\erwaehnteInstitutionen}{Institutionen: Jung-Wiener Theater zum Lieben Augustin, Neue Freie Presse, Reichstag}
\renewcommand{\erwaehnteOrte}{Orte: Berlin, Berliner Theater, Dessauer Straße, Frankfurt am Main, Reichenau an der Rax, Wien}
\renewcommand{\erwaehnteWerke}{Werke: Alt-Heidelberg. Schauspiel in 5 Aufzügen, Berliner Theater. »Der Rothe Hahn.«, Berliner Theater. »Einsame Menschen« im Deutschen Theater, Der einsame Weg. Schauspiel in fünf Akten, Der rothe Hahn. Tragikomödie in vier Akten, Lebendige Stunden. Vier Einakter, Neue Freie Presse, Theater- und Kunstnachrichten. Jung-Wiener-Theater »Zum lieben Augustin«}
\section[ Paul Goldmann an Arthur Schnitzler, 23. 11. {[}1901{]}]{Paul Goldmann an Arthur Schnitzler, 23. 11. {[}1901{]}}
\nopagebreak\mylabel{v}
\rehead{ }\normalsize\beginnumbering\briefempfaengerindex{Schnitzler, Arthur@\textsc{Schnitzler, Arthur}!zzzGoldmann, Paul@\emph{von Paul Goldmann}!1901-11-231@{23. 11. {[}1901{]}}|(be}
\toendnotes[C]{\smallbreak\pagebreak[2]}\Standort{DLA, A:Schnitzler, HS.NZ85.1.3171.}
\physDesc{Brief, 2 Blätter, 8 Seiten
\newline{}Handschrift: blaue Tinte, deutsche Kurrent
\newline{}Schnitzler: 1) mit Bleistift das Jahr »1901« vermerkt  2) mit rotem Buntstift sieben Unterstreichungen}\toendnotes[C]{\smallbreak}
\pstart
           \noindent{}\raggedleft{}{\pb}\textcolor{pink}{\textcolor{gray}{\textbf{DESSAUERSTRASSE 19}}}{}\ledrightnote{\textcolor{pink}{Dessauer Straße}}\pend
           
\pstart
           \textcolor{pink}{Berlin}{}\ledrightnote{\textcolor{pink}{Berlin}}, 23. November.\pend
           
\pstart\center{}Mein lieber Freund,\pend
\pstart
           Tauend Dank für Deine lieben Worte! Es war wirklich nicht nöthig, mir deshalb einen
               großen Brief zu ſchreiben, und ich bitte Dich, auch \textsc{\textcolor{blue}{Olga}{}\ledrightnote{\textcolor{blue}{Olga Schnitzler}}} zu veranlaſſen, daß ſie mir über die \label{K_L03091-1v}\edtext{Affaire}{\lemma{\textnormal{\emph{Affaire}}}\Cendnote{\textnormal{Bezug auf
                  den Konflikt rund um \textcolor{blue}{Goldmann}s Kritik an
                     \textcolor{blue}{Gerhart Hauptmann}, siehe Paul Goldmann an Arthur Schnitzler, 9. 11. [1901]. }}}\label{K_L03091-1h} nicht mehr
               ſchreibt. Die Sache iſt abgethan; und ich bedaure lebhaft, daß ich dem Unwillen, den
               ich über den zurechtweiſenden Ton von \textsc{\textcolor{blue}{Olga}{}\ledrightnote{\textcolor{blue}{Olga Schnitzler}}s} Brief empfunden, überhaupt
               Ausdruck gegeben habe. Im Übrigen nimmſt Du nach wie vor in der Frage einen
               erkundlich \label{K_L03091-3v}\edtext{einſeitigen Standpunkt}{\lemma{\textnormal{\emph{einſeitigen Standpunkt}}}\Cendnote{\textnormal{Auch \textcolor{blue}{Schnitzler} schätzte \textcolor{blue}{Goldmann}s
                  Standpunkt als einseitig ein, vgl. A. S.: \emph{Tagebuch}, 27. 11. 1901.}}}\label{K_L03091-3h} ein. Ich kann Dir verſichern, daß {\pb}nicht nur \label{K_L03091-2v}\edtext{widerliche Kerle}{\lemma{\textnormal{\emph{widerliche Kerle}}}\Cendnote{\textnormal{womöglich Anspielung auf \textcolor{blue}{Leo Ebermann}, {XXXX ref}}}}\label{K_L03091-2h} ſich über meine \textcolor{green}{Kritik}{}\ledrightnote{{$\rightarrow$}\textcolor{green}{Berliner Theater. »Einsame Menschen« im Deutschen Theater}}en freuen, ſondern auch ſehr anſtändige Leute. Und was habe ich mich um
               die Wirkungen zu bekümmern, die meine \textcolor{green}{Kritik}{}\ledrightnote{{$\rightarrow$}\textcolor{green}{Berliner Theater. »Einsame Menschen« im Deutschen Theater}}en { } auf widerliche
               Kerle \substVorne{}\textsuperscript{\textcolor{gray}{×}\-\textcolor{gray}{×}\-\textcolor{gray}{×}\-\textcolor{gray}{×}\-\textcolor{gray}{×}}\substDazwischen{}auſüben\substHinten{}? Was habe ich mich überhaupt um die Wirkungen meiner Arbeiten zu bekümmern?
               Das iſt \strikeout{\textcolor{gray}{doch}} ein ganz unkünſtleriſches Verlangen, das Du da an mich ſtellſt. Die einzige
               Frage kann doch nur die ſein, ob meine \textcolor{green}{Kritik}{}\ledrightnote{{$\rightarrow$}\textcolor{green}{Berliner Theater. »Einsame Menschen« im Deutschen Theater}}en meine Überzeugung und meine Stimmung ausdrücken.
               Und da meine Überzeugung die iſt, daß \textsc{\textcolor{blue}{Gerhart Hauptmann}{}\ledrightnote{\textcolor{blue}{Gerhart Hauptmann}}} ein minderwerthiger {\pb}und verworrener Geiſt
               iſt, und da ich Erbitterung darüber empfinde, dieſen minderwerthigen \textcolor{blue}{Geiſt}{}\ledrightnote{{$\rightarrow$}\textcolor{blue}{Gerhart Hauptmann}} als großen Dichter
               geprieſen zu ſehen, ſo \strikeout{\textcolor{gray}{ſ}} können meine \textcolor{green}{Kritik}{}\ledrightnote{{$\rightarrow$}\textcolor{green}{Berliner Theater. »Einsame Menschen« im Deutschen Theater}}en
               absſolut nicht anders lauten und können auch in keinem anderen Tone geſchrieben
               ſein.\pend
           
\pstart
           Du irrſt Dich auch, wenn Du glaubſt, daß Du mir immer ſchreibſt, wenn Du über eine
               meiner Arbeiten »entzückt« biſt. Ich bin überzeugt, daß Du in \textcolor{pink}{Wien}{}\ledrightnote{\textcolor{pink}{Wien}} dieſem »Entzücken« Worte verleihſt, Du vergißt es nur in
               der {\pb}Regel, mir mitzutheilen. Ich habe oft genug, wenn
               ich das Bewußtſein hatte, eine Arbeit von Werth vollendet zu haben, mich nach einem
               Wort der Zuſtimmung von Deiner Seite geſehnt, und oft genug iſt dieſes Wort der
               Zuſtimmung ausgeblieben. Pünktlich und ausführlich ſchreibſt Du mir nur, wenn Du an
               meinen Arbeiten etwas zu tadeln haſt.\pend
           
\pstart
           So, und nun genug!\pend
           
\pstart
           Ich habe mich von Herzen gefreut, endlich wieder einmal etwas von Dir zu hören, und
               habe mich insbeſondere gefreut, {\pb}daß Du und \textsc{\textcolor{blue}{Olga}{}\ledrightnote{\textcolor{blue}{Olga Schnitzler}}} (wie ich aus \textsc{\textcolor{blue}{Olga}{}\ledrightnote{\textcolor{blue}{Olga Schnitzler}}s} Brief erſehen) in \label{K_L03091-4v}\edtext{\textsc{\textcolor{pink}{Reichenau}{}\ledrightnote{\textcolor{pink}{Reichenau an der Rax}}}}{\lemma{\textnormal{\emph{Reichenau}}}\Cendnote{\textnormal{\textcolor{blue}{Schnitzler} und \textcolor{blue}{Olga Gussmann} waren zwischen 11. 11. 1901 und 13. 11. 1901 in \textcolor{pink}{Reichenau}.}}}\label{K_L03091-4h} ſo ſchöne Tage verlebt
               habt.\pend
           
\pstart
           Die Aufführung Deiner \textcolor{green}{Einakter}{}\ledrightnote{{$\rightarrow$}\textcolor{green}{Lebendige Stunden. Vier Einakter}}
               am 4. Jänner ſollteſt Du zu \label{K_L03091-7v}\edtext{verhindern}{\lemma{\textnormal{\emph{verhindern}}}\Cendnote{\textnormal{nicht
                  geschehen}}}\label{K_L03091-7h} ſuchen. So wenige Tage nach Neujahr iſt eine recht ungünſtige Theaterzeit. Hat \textcolor{blue}{Brahm}{}\ledrightnote{\textcolor{blue}{Otto Brahm}} ſolange gewartet, ſo kann er auch noch eine Woche
               länger warten. Ich ſelbſt werde \label{K_L03091-9v}\edtext{am
                  4. Jänner kaum in \textcolor{pink}{Berlin}{}\ledrightnote{\textcolor{pink}{Berlin}} ſein}{\lemma{\textnormal{\emph{am … ſein}}}\Cendnote{\textnormal{\textcolor{blue}{Goldmann} war zur Uraufführung von \emph{\textcolor{green}{Lebendige Stunden}} wieder in \textcolor{pink}{Berlin}.}}}\label{K_L03091-9h}, da ich, wie alljährlich, {\pb}die Weihnachts- und Neujahrstage bei meiner \strikeout{Sch\textcolor{gray}{w}} Familie in \textcolor{pink}{Frankfurt}{}\ledrightnote{\textcolor{pink}{Frankfurt am Main}} zu verbringen
               hoffe.\pend
           
\pstart
           Geſtern ſahen wir hier ein ſtellenweiſe ſehr hübſches
                  \label{K_L03091-111v}\edtext{\textcolor{green}{Stück}{}\ledrightnote{{$\rightarrow$}\textcolor{green}{Alt-Heidelberg. Schauspiel in 5 Aufzügen}} von \textsc{\textcolor{blue}{Meyer-Förster}{}\ledrightnote{\textcolor{blue}{Wilhelm Meyer-Förster}}}}{\lemma{\textnormal{\emph{Stück von Meyer-Förster}}}\Cendnote{\textnormal{Am 22. 11. 1901 hatte \textcolor{blue}{Wilhelm
                     Meyer-Förster}s \emph{\textcolor{green}{Alt-Heidelberg. Schauspiel
                     in 5 Aufzügen}} die Uraufführung am \textcolor{pink}{Berliner
                     Theater}.}}}\label{K_L03091-111h}. Ich werde leider kaum Zeit finden, darüber zu ſchreiben,
               da nächſte Woche der \textcolor{brown}{Reichſtag}{}\ledrightnote{\textcolor{brown}{Reichstag}} zuſammentritt.
               Auch muß ich in meinem nächſtem \label{K_L03091-11v}\edtext{\textcolor{green}{Feuilleton}{}\ledrightnote{{$\rightarrow$}\textcolor{green}{Berliner Theater. »Der Rothe Hahn.«}} den »\textcolor{green}{Rothen Hahn}{}\ledrightnote{\textcolor{green}{Der rothe Hahn. Tragikomödie in vier Akten}}«}{\lemma{\textnormal{\emph{Feuilleton … Hahn«}}}\Cendnote{\textnormal{siehe Paul Goldmann an Arthur Schnitzler, 29. 11. [1901] und Paul Goldmann an Arthur Schnitzler, 6. 12. [1901]}}}\label{K_L03091-11h} behandeln\pend
           
\pstart
           {\pb}Was Du über \label{K_L03091-12v}\edtext{die \textcolor{green}{Haltung}{}\ledrightnote{{$\rightarrow$}\textcolor{green}{Theater- und Kunstnachrichten. Jung-Wiener-Theater »Zum lieben Augustin«}} der \textcolor{brown}{N. Fr. Pr.}{}\ledrightnote{\textcolor{brown}{Neue Freie Presse}} gegenüber dem
                  »\textcolor{brown}{Jung Wiener Theater}{}\ledrightnote{\textcolor{brown}{Jung-Wiener Theater zum Lieben Augustin}}«}{\lemma{\textnormal{\emph{die … Theater«}}}\Cendnote{\textnormal{\textcolor{blue}{–da} [=\textcolor{blue}{Moriz Neuda}]: \emph{\textcolor{green}{Theater- und Kunstnachrichten. Jung-Wiener-Theater »Zum
                        lieben Augustin«}}. In: \emph{\textcolor{green}{Neue Freie
                        Presse}}, Nr. 13.374, 17. 11. 1901,
                     Morgenblatt, S. 8–9.}}}\label{K_L03091-12h} ſchreibſt, iſt durchaus berichtigt. Aber \textsc{\textcolor{blue}{Salten}{}\ledrightnote{\textcolor{blue}{Felix Salten}}} trägt doch wohl die Hauptſchuld. Er machte \strikeout{h\textcolor{gray}{i}} mir hier in \textcolor{pink}{Berlin}{}\ledrightnote{\textcolor{pink}{Berlin}} den Eindruck eines
               Mannes, der abſolut keine Ahnung hat, was er will. Und wie kann man ſich zu einem
                  \label{K_L03091-21v}\edtext{künſtleriſchen Unternehmen mit \textsc{\textcolor{blue}{Si\textcolor{gray}{e}gfried Löwy}{}\ledrightnote{\textcolor{blue}{Siegfried Loewy}}}}{\lemma{\textnormal{\emph{künſtleriſchen … Löwy}}}\Cendnote{\textnormal{\textcolor{blue}{Felix Salten} hatte das \emph{\textcolor{brown}{Jung-Wiener Theater zum lieben Augustin}} gemeinsam mit \textcolor{blue}{Siegfried Loewy} gegründet und am 16. 11. 1901 eröffnet.
                  Die Resonanz war schlecht. Bereits nach sechs Aufführungen wurde das \textcolor{brown}{Theater} wieder
                  eingestellt.}}}\label{K_L03091-21h} aſſociiren?\pend
           
\pstart
           Mit Deinem neuen \textcolor{green}{Stück}{}\ledrightnote{{$\rightarrow$}\textcolor{green}{Der einsame Weg. Schauspiel in fünf Akten}} wirſt
               Du Dich ſchon wieder {\pb}\label{K_L03091-16v}\edtext{zurechtfinden}{\lemma{\textnormal{\emph{zurechtfinden}}}\Cendnote{\textnormal{siehe A. S.: \emph{Tagebuch}, 20. 11. 1901}}}\label{K_L03091-16h}. Je mehr Du daran arbeiteſt, umſo tiefer wird es werden. Quäle Dich alſo nur
               ein wenig. Es ſchadet gar nichts.\pend
           
\pstart
           Grüße mir die \textcolor{blue}{Mädeln}{}\ledrightnote{{$\rightarrow$}\textcolor{blue}{Olga Schnitzler}{\newline}{$\rightarrow$}\textcolor{blue}{Elisabeth Steinrück}} und ſei Du ſelbſt vielmals und von Herzen gegrüßt! {\\[\baselineskip]}Dein {\\[\baselineskip]}\spacefill\mbox{Paul Goldmann}\pend
           \leftskip=0em{}\endnumbering\briefempfaengerindex{Schnitzler, Arthur@\textsc{Schnitzler, Arthur}!zzzGoldmann, Paul@\emph{von Paul Goldmann}!1901-11-231@{23. 11. {[}1901{]}}|)be}\mylabel{h}
\begin{anhang}
\end{anhang}\normalsize

\doendnotes{C}
\bigskip
\vfill

\clearpage

\footnotesize

\lohead{\textsc{register}}

% Definiere theindex-Environment komplett neu ohne reledmac
\makeatletter
\renewenvironment{theindex}{%
  \section*{\indexname}%
  \setlength{\parindent}{0pt}%
  \setlength{\parskip}{0pt plus 0.3pt}%
  \let\item\@idxitem
}{%
  \clearpage
}
\makeatother

\IfFileExists{\jobname-pw.ind}{\input{\jobname-pw.ind}}{}

\end{document}

      