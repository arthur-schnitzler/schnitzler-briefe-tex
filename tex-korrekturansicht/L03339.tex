%% latex-korrekturansicht-vorspann.tex
%% Vorspann für die Korrekturansicht.
%% Lädt die gemeinsame Datei latex-vorspann.tex mit gesetztem Schalter.

\newif\ifkorrekturansicht
\korrekturansichttrue

\input{../tex-inputs/latex-vorspann}


\renewcommand{\erwaehntePersonen}{Personen:  ?? [Englischlehrerin von Felix Salten], Kurt Aram, Mirjam Horwitz,  Horwitz, Siegfried Jacobsohn, George Bernard Shaw, Elisabeth Steinrück, Carl Wiene}
\renewcommand{\erwaehnteInstitutionen}{Institutionen: Die Zeit. Wiener Wochenschrift}
\renewcommand{\erwaehnteOrte}{Orte: Berlin, Bosnien und Herzegowina, Dalmatien, Deutsches Theater Berlin, London, Raimund-Theater, Wien}
\renewcommand{\erwaehnteWerke}{Werke: Der Hund von Florenz, Der Schleier der Beatrice. Schauspiel in fünf Akten, Die Gespräche des göttlichen Pietro Aretino, Die Zeit, Ein Teufelskerl. Schauspiel in drei Akten, The Devil’s Disciple, Vom göttlichen Aretino, »Schleier der Beatrice.« Man telegraphirt uns aus Berlin, 7. d.}
\section[ Felix Salten an Arthur Schnitzler, 3. 3. 1903]{Felix Salten an Arthur Schnitzler, 3. 3. 1903}
\nopagebreak\mylabel{v}
\rehead{ }\normalsize\beginnumbering\briefempfaengerindex{Schnitzler, Arthur@\textsc{Schnitzler, Arthur}!zzzSalten, Felix@\emph{von Felix Salten}!1903-03-031@{3. 3. 1903}|(be}
\toendnotes[C]{\smallbreak\pagebreak[2]}\Standort{CUL, Schnitzler, B 89, A 2.}
\physDesc{Brief, 1 Blatt, 4 Seiten, 2821 Zeichen
\newline{}Handschrift: Bleistift, lateinische Kurrent
\newline{}Schnitzler: mit Bleistift »\textsc{Salten}« vermerkt 
\newline{}Ordnung: mit Bleistift von unbekannter Hand nummeriert: »164« }\toendnotes[C]{\smallbreak}
\pstart
           \raggedleft{}{\pb}\textcolor{pink}{Wien}{}\ledrightnote{\textcolor{pink}{Wien}}, 3. III. 03\pend
           
\pstart
           Lieber, zur \label{K_L03339-1v}\edtext{\textcolor{green}{Premiere}{}\ledrightnote{{$\rightarrow$}\textcolor{green}{Der Schleier der Beatrice. Schauspiel in fünf Akten}}}{\lemma{\textnormal{\emph{Premiere}}}\Cendnote{\textnormal{\textcolor{blue}{Schnitzler} weilte zur Vorbereitung der
                  Premiere von \emph{\textcolor{green}{Der Schleier der Beatrice}} in \textcolor{pink}{Berlin}. Diese fand am 7. 3. 1903 am \textcolor{pink}{Deutschen Theater} in seiner Anwesenheit
                  statt.}}}\label{K_L03339-1h} kann ich nun leider doch nicht nach \textcolor{pink}{Berlin}{}\ledrightnote{\textcolor{pink}{Berlin}}; schade. Ich werde erst so gegen 14.\textsuperscript{ten} März reisen; und habe vorher noch enorm viel
               zu thun. Was sagen Sie zum \label{K_L03339-2v}\edtext{\textcolor{green}{Teufelskerl}{}\ledrightnote{\textcolor{green}{Ein Teufelskerl. Schauspiel in drei Akten}}? Das Stück hat Herr \textcolor{blue}{Wiene}{}\ledrightnote{\textcolor{blue}{Carl Wiene}}}{\lemma{\textnormal{\emph{Teufelskerl? … Wiene}}}\Cendnote{\textnormal{\textcolor{blue}{Carl Wiene} trat als Gastschauspieler am
                     25. 2. 1903 im \textcolor{pink}{Raimund-Theater} in der Hauptrolle in \emph{\textcolor{green}{Ein
                     Teufelskerl}} (\emph{\textcolor{green}{The Devil’s Disciple}})
                  von \textcolor{blue}{George Bernard Shaw} auf. \textcolor{blue}{Schnitzler} war zu diesem Zeitpunkt bereits in
                     \textcolor{pink}{Berlin} und sah die Vorstellung
                  nicht.}}}\label{K_L03339-2h} ruinirt, wie vorauszusehen war. Sehr fühlbar wurde mir die tiefe
                  Unmoral\textcolor{gray}{,} die darin steckt, wenn das Alter sich als Jugend
               verkleidet und geberdet. Der Widerwille, den man bei solchem Schauspiel empfindet
               geht bis an ein sexuelles Missbehagen, wenigstens begreift man die Nervenzerrüttung
               einer Frau, an der ein impotenter Mann heuchlerische Versuche vornimmt, denn mit
               ähnlicher Bereitwilligkeit zur Empfängnis sitzt so ein Publikum im Theater. Mir wäre
               es sehr lieb, wenn Sie mir statt einer Ansichtskarte einmal näheres über die Proben
               ec. \textcolor{pink}{Berlin}{}\ledrightnote{\textcolor{pink}{Berlin}} ec. schrieben, falls es Ihre Zeit
               gestattet.\pend
           
\pstart
           In Angelegenheit der \label{K_L03339-3v}\edtext{\textcolor{blue}{Mirjam H.}{}\ledrightnote{\textcolor{blue}{Mirjam Horwitz}}}{\lemma{\textnormal{\emph{Mirjam H.}}}\Cendnote{\textnormal{\textcolor{blue}{Salten} und die Schauspielerin \textcolor{blue}{Mirjam Horwitz}
                  hatten eine Affäre, die, wenn man die Hinweise zusammenliest, von ihrem \textcolor{blue}{Vater} beendet wurde, indem er eine Entscheidung von \textcolor{blue}{Salten} 
                  forderte. \textcolor{blue}{Salten} sah darin die Möglichkeit, die Sache zu beenden
                  und bat \textcolor{blue}{Schnitzler} um Vermittlung, was er während seines \textcolor{pink}{Berlin}-Aufenthalts tat. \textcolor{blue}{Horwitz} war
                  auch eine Freundin von \textcolor{blue}{Schnitzler}s Schwägerin \textcolor{blue}{Elisabeth Gussmann}.}}}\label{K_L03339-3h} muß ich Sie
               nochmals bemühen: bald und möglichst schonend. Sie schreibt mir heute einen confusen Brief; ob sie »nach hier« kommen
               soll, oder wann ich »nach dort« komme, ferner, dass ich nicht durch mein Wort an
               ihren \textcolor{blue}{Vater}{}\ledrightnote{{$\rightarrow$}\textcolor{blue}{Horwitz}} gebunden bin,
               falls \uline{sie} mit \uline{mir}
               verkehrt, endlich, dass ich an einen Vertrauten von ihr schreiben soll, das sei auch
               nicht gegen mein Versprechen ec. Dann noch recht enervirende Dinge
               von »sich angehören vor aller {\pb}Welt –« »den Leuten zum Trotz« ec. und in diesem Stil, der die Liebe recht
               unangenehm macht.\pend
           
\pstart
           Das Wesentliche an der Sache: dass ich ihrem \textcolor{blue}{Vater}{}\ledrightnote{{$\rightarrow$}\textcolor{blue}{Horwitz}} wahrscheinlich kein Versprechen gegeben hätte, wenn
               ich \textcolor{blue}{Mirjam}{}\ledrightnote{\textcolor{blue}{Mirjam Horwitz}} sehr lieb hätte. Ferner: dass ich
               aber, nun ich das Versprechen gab, keine Lust habe Geschichten zu machen. Bringen Sie
               ihr das bitte schonend bei. D\textcolor{gray}{a}s mit dem Versprechen nämlich, und
               vor allem, dass sie nichts gewinnt, wenn sie gewaltsame Streiche macht, da mir solche
               von jeher zuwider waren. Aber bitte, seien Sie sehr schonend, weil sie mir mit
               Selbstmord droht, was auch eine hübsche Gewohnheit von ihr ist.\pend
           
\pstart
           Am 14. fahre ich auf 8 Tage nach \textcolor{pink}{Berlin}{}\ledrightnote{\textcolor{pink}{Berlin}}. Im April voraussichtlich
               nach \textcolor{pink}{Bosnien}{}\ledrightnote{\textcolor{pink}{Bosnien und Herzegowina}} und \textcolor{pink}{Dalmatien}{}\ledrightnote{\textcolor{pink}{Dalmatien}}. Im Mai nach \textcolor{pink}{London}{}\ledrightnote{\textcolor{pink}{London}} auf 14 Tage.\pend
           
\pstart
           {\pb}Ich lese jetzt die »\label{K_L03339-4v}\edtext{\textcolor{green}{Gespräche des göttlichen Aretino}{}\ledrightnote{\textcolor{green}{Die Gespräche des göttlichen Pietro Aretino}}}{\lemma{\textnormal{\emph{Gespräche … Aretino}}}\Cendnote{\textnormal{\textcolor{blue}{Salten} schrieb auch ein \textcolor{green}{Feuilleton} darüber: \textcolor{blue}{Felix Salten}: \emph{\textcolor{green}{Vom göttlichen Aretino}}. In: \emph{\textcolor{green}{Die Zeit}}, Jg. 2, Nr. 165, 15. 3. 1903, Morgenblatt, S. 1–2.}}}\label{K_L03339-4h},« und finde darin zu
               meinem Erstaunen die römische Buhlerin, die Bekenntnisse ablegt. Sie wissen, dass ich
               ein solches Buch schreiben wollte. Arbeiten kann ich nur wenig, da mir die \textcolor{brown}{Zeit}{}\ledrightnote{\textcolor{brown}{Die Zeit. Wiener Wochenschrift}} fast
               alles weg nimmt. Nun soll \textcolor{blue}{Aram}{}\ledrightnote{\textcolor{blue}{Kurt Aram}} fort, und ich
                  für 8400fl\textcolor{gray}{.} jährlich auch das Feuilleton übernehmen; außerdem
               heißt es, – mit mir wurde \textcolor{gray}{noch} nicht davon gesprochen – dass ich
               Chef-Stellvertreter werden soll. Ich wünschte mir, dass der Tag dann – 36 Stunden haben möge, eine
               Erhöhung, mit der ich noch mehr einverstanden wäre. Für \textcolor{pink}{London}{}\ledrightnote{\textcolor{pink}{London}} habe ich mir jetzt eine 
               \textcolor{blue}{Engländerin}{}\ledrightnote{{$\rightarrow$}\textcolor{blue}{?? [Englischlehrerin von Felix Salten]}}
                angeschafft,
               die 3mal die Woche kommt. Ich \label{K_L03339-5v}\edtext{beginne
               den {[}»{]}\textcolor{green}{Hund {\pb}von Florenz}{}\ledrightnote{\textcolor{green}{Der Hund von Florenz}}«}{\lemma{\textnormal{\emph{beginne … Florenz«}}}\Cendnote{\textnormal{\textcolor{blue}{Salten} arbeitete noch Jahre an der \textcolor{green}{Novelle}, die erst 1923 erschien. Siehe Felix Salten an Arthur Schnitzler, 15. 8. 1907 und 18. 3. 1921.}}}\label{K_L03339-5h} den ich vielleicht dann in \textcolor{pink}{Bosnien}{}\ledrightnote{\textcolor{pink}{Bosnien und Herzegowina}} fertig mache.\pend
           
\pstart
           Schreiben Sie mir bitte recht bald. Bin neugierig, wie sich Herr \label{K_L03339-6v}\edtext{\textcolor{blue}{Jacobsohn}{}\ledrightnote{\textcolor{blue}{Siegfried Jacobsohn}} benehmen}{\lemma{\textnormal{\emph{Jacobsohn benehmen}}}\Cendnote{\textnormal{\textcolor{blue}{Jacobsohn} war der \textcolor{pink}{Berlin}er Theaterkorrespondent der
                  \emph{\textcolor{brown}{Zeit}}. \textcolor{blue}{Salten} dürfte hier seiner Neugier Ausdruck
                  verleihen, wie dieser die Premiere von \emph{\textcolor{green}{Der Schleier der Beatrice}} besprechen würde. Die Depesche lautetete:
                  »Der Dichter wurde häufig gerufen, und ſtarker Beifall behauptete ſich ſiegreich gegen einzelne energiſche Ziſcher«. ([\textcolor{blue}{Siegfried Jacobsohn}]: \emph{\textcolor{green}{»Schleier der Beatrice.« Man telegraphirt uns aus Berlin, 7. d.}}. 
                     In: \emph{\textcolor{green}{Die Zeit}}, Jg. 2, Nr. 158, S. 5.)
               }}}\label{K_L03339-6h} wird.\pend
           
\pstart
           herzlichst Ihr {\\[\baselineskip]}\spacefill\mbox{Salten}\pend
           \leftskip=0em{}\endnumbering\briefempfaengerindex{Schnitzler, Arthur@\textsc{Schnitzler, Arthur}!zzzSalten, Felix@\emph{von Felix Salten}!1903-03-031@{3. 3. 1903}|)be}\mylabel{h}  \normalsize

\doendnotes{C}
\bigskip
\vfill

\clearpage

\footnotesize

\lohead{\textsc{register}}

% Definiere theindex-Environment komplett neu ohne reledmac
\makeatletter
\renewenvironment{theindex}{%
  \section*{\indexname}%
  \setlength{\parindent}{0pt}%
  \setlength{\parskip}{0pt plus 0.3pt}%
  \let\item\@idxitem
}{%
  \clearpage
}
\makeatother

\IfFileExists{\jobname-pw.ind}{\input{\jobname-pw.ind}}{}

\end{document}

      