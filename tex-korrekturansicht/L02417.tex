%% latex-korrekturansicht-vorspann.tex
%% Vorspann für die Korrekturansicht.
%% Lädt die gemeinsame Datei latex-vorspann.tex mit gesetztem Schalter.

\newif\ifkorrekturansicht
\korrekturansichttrue

\input{../tex-inputs/latex-vorspann}


               \section[Thomas Mann an Arthur Schnitzler, 22. 10. 1924]{ Thomas Mann an Arthur Schnitzler, 22. 10. 1924}\nopagebreak\mylabel{v}\rehead{ }\normalsize\beginnumbering\briefempfaengerindex{Schnitzler, Arthur@\textsc{Schnitzler, Arthur}!zzzMann, Thomas@\emph{von Thomas Mann}!1924-10-221@{22. 10. 1924}|(be} \toendnotes[C]{\smallbreak\pagebreak[2]} \Standort{CUL, Schnitzler, B 67.}
\physDesc{Brief, 1 Blatt, 2 Seiten
\newline{}Handschrift: schwarze Tinte, deutsche Kurrent
\newline{}Schnitzler: 1) mit Bleistift beschriftet: »\textsc{Thomas Ma{\geminationn}}« 2) mit Bleistift unterhalb des Brieftextes Antwortskizze: »Der Zumuthg
               den \textcolor{green}{Zauberberg} zu leſen{\dotstwo}{ }ſeh«3) mit rotem Buntstift mehrere Unterstreichungen}\buchAbdrucke{\weitereDrucke{1) Hertha Krotkoff: \emph{Arthur Schnitzler – Thomas Mann: Briefe.} In: \emph{Modern Austrian Literature}, Jg. 7 (1974) Nr. 1/2, S. 22.} \weitereDrucke{2) Hans-Ulrich Lindken: \emph{Arthur Schnitzler. Aspekte und Akzente. Materialien zu Leben
                        und Werk}. Frankfurt am Main, Bern, Göttingen: \emph{Peter Lang} 1984, S. 197 (Europäische Hochschulschriften, Reihe 1, Deutsche Sprache und
                        Literatur, 754).} }\toendnotes[C]{\smallbreak}\pstart
           \raggedleft{}{\pb}\textsc{\textcolor{pink}{Sestri-Lev.}{}\ledrightnote{\textcolor{pink}{Sestri Levante}}} den
                            22. X. 24.\pend
           \pstart{}Verehrter Herr Dr. Schnitzler,\pend\pstart
           es iſt mir ein Bedürfnis, Ihnen für die ſchönen Stunden zu danken, die ich hier
                    mit der Lektüre Ihrer neuen \textcolor{green}{Komödie}{}\ledrightnote{→\textcolor{green}{Komödie der Verführung. In drei Akten}} verbrachte, dieſes glänzenden, leidenſchaftlichen
                    Geſellchaftsſtückes, das die Maße und Grenzen dieſer Gattung auf ſo feſtliche
                    Weiſe weitert oder ſoll man ſagen: zerbricht. Ich kann es kaum erwarten, das
                    Werk auf dem Theater zu ſehen, und doch bangt mir auch wieder davor. Werden
                    unſere Schauſpieler eine »Konverſation« beherrſchen, die ſich jeden Augenblick
                    zur Sprache des großen Dramas erhebt? Jedenfalls hoffe ich, daß das \textcolor{pink}{Münchener Reſidenztheater}{}\ledrightnote{\textcolor{pink}{Residenztheater München}} recht bald die
                    Gelegenheit ergreift, zu {\pb}zeigen, was
                    es kann.\pend
           \pstart
           Nächſten Monat verſendet \textcolor{brown}{Fiſcher}{}\ledrightnote{\textcolor{brown}{S. Fischer Verlag}} meinen Roman
                        »\textcolor{green}{Der Zauberberg}{}\ledrightnote{\textcolor{green}{Der Zauberberg. Roman}}«. Natürlich werde ich ihn
                    bitten, Ihnen ein Exemplar zu ſchicken, aber Sie bitte ich, erblicken Sie
                    keinerlei Zumutung darin! Ich denke ſehr zögernd über die Menſchenmöglichkeit
                    des unförmigen \textcolor{green}{Opus}{}\ledrightnote{→\textcolor{green}{Der Zauberberg. Roman}} und
                    entbinde jeden, dem ich es zugehen laſſe, feierlich von jeder Aeußerung
                    darüber.\pend
           \pstart
           Ihr ergebenſter{\\[\baselineskip]}\spacefill\mbox{Thomas Mann.}\pend
           \leftskip=0em{}\endnumbering\briefempfaengerindex{Schnitzler, Arthur@\textsc{Schnitzler, Arthur}!zzzMann, Thomas@\emph{von Thomas Mann}!1924-10-221@{22. 10. 1924}|)be}\mylabel{h}  \normalsize

\doendnotes{C}
\bigskip
\vfill

\clearpage

\footnotesize

\lohead{\textsc{register}}

% Definiere theindex-Environment komplett neu ohne reledmac
\makeatletter
\renewenvironment{theindex}{%
  \section*{\indexname}%
  \setlength{\parindent}{0pt}%
  \setlength{\parskip}{0pt plus 0.3pt}%
  \let\item\@idxitem
}{%
  \clearpage
}
\makeatother

\IfFileExists{\jobname-pw.ind}{\input{\jobname-pw.ind}}{}

\end{document}

      