%% latex-korrekturansicht-vorspann.tex
%% Vorspann für die Korrekturansicht.
%% Lädt die gemeinsame Datei latex-vorspann.tex mit gesetztem Schalter.

\newif\ifkorrekturansicht
\korrekturansichttrue

\input{../tex-inputs/latex-vorspann}


\renewcommand{\erwaehnteOrte}{Orte: Wien}
\renewcommand{\erwaehnteWerke}{Werke: Die Zeit, Studie}
\section[ Felix Salten an Arthur Schnitzler, 28. 9. 1903]{Felix Salten an Arthur Schnitzler, 28. 9. 1903}
\nopagebreak\mylabel{v}
\rehead{ }\normalsize\beginnumbering\briefempfaengerindex{Schnitzler, Arthur@\textsc{Schnitzler, Arthur}!zzzSalten, Felix@\emph{von Felix Salten}!1903-09-282@{28. 9. 1903}|(be}
\toendnotes[C]{\smallbreak\pagebreak[2]}\Standort{CUL, Schnitzler, B 89, A 2.}
\physDesc{Karte, 280 Zeichen
\newline{}Handschrift: Bleistift, lateinische Kurrent
\newline{}Ordnung: mit Bleistift von unbekannter Hand nummeriert: »{\pb}170« }\toendnotes[C]{\smallbreak}
\pstart
           \raggedleft{}{\pb}28/IX. 03\pend
           
\pstart
           Lieber, es war mir ganz entfallen, dass der 27.\textsuperscript{te} unsere \textcolor{green}{Jahres N\textsuperscript{o}}{}\ledrightnote{{$\rightarrow$}\textcolor{green}{Die Zeit}} bringt. Da konnte ich die »\textcolor{green}{Studie}{}\ledrightnote{\textcolor{green}{Studie}}« nicht
               hineinsetzen, weil sie doch zu schwach gewesen wäre, und ich sowol diese \textcolor{green}{Arbeit}{}\ledrightnote{{$\rightarrow$}\textcolor{green}{Studie}}\strikeout{e} als mich überflüßigen Kritiken empfohlen hätte. Sie
               erscheint ganz sicher am 4. X.\pend
           \pstart herzl. \spacefill\mbox{S.}\pend{}\endnumbering\briefempfaengerindex{Schnitzler, Arthur@\textsc{Schnitzler, Arthur}!zzzSalten, Felix@\emph{von Felix Salten}!1903-09-282@{28. 9. 1903}|)be}\mylabel{h}  \normalsize

\doendnotes{C}
\bigskip
\vfill

\clearpage

\footnotesize

\lohead{\textsc{register}}

% Definiere theindex-Environment komplett neu ohne reledmac
\makeatletter
\renewenvironment{theindex}{%
  \section*{\indexname}%
  \setlength{\parindent}{0pt}%
  \setlength{\parskip}{0pt plus 0.3pt}%
  \let\item\@idxitem
}{%
  \clearpage
}
\makeatother

\IfFileExists{\jobname-pw.ind}{\input{\jobname-pw.ind}}{}

\end{document}

      