%% latex-korrekturansicht-vorspann.tex
%% Vorspann für die Korrekturansicht.
%% Lädt die gemeinsame Datei latex-vorspann.tex mit gesetztem Schalter.

\newif\ifkorrekturansicht
\korrekturansichttrue

\input{../tex-inputs/latex-vorspann}


\renewcommand{\erwaehntePersonen}{Personen: Olga Schnitzler}
\renewcommand{\erwaehnteInstitutionen}{Institutionen: Franz-Grillparzer-Preis}
\renewcommand{\erwaehnteOrte}{Orte: Berlin, Edmund-Weiß-Gasse, Wien}
\renewcommand{\erwaehnteWerke}{Werke: Zwischenspiel. Komödie in drei Akten}
\section[ Paul Goldmann an Arthur Schnitzler, 16. 1. 1908]{Paul Goldmann an Arthur Schnitzler, 16. 1. 1908}
\nopagebreak\mylabel{v}
\rehead{ }\normalsize\beginnumbering\briefempfaengerindex{Schnitzler, Arthur@\textsc{Schnitzler, Arthur}!zzzGoldmann, Paul@\emph{von Paul Goldmann}!1908-01-162@{16. 1. 1908}|(be}
\toendnotes[C]{\smallbreak\pagebreak[2]}\Standort{DLA, A:Schnitzler, HS.NZ85.1.3175.}
\physDesc{Bildpostkarte, 258 Zeichen
\newline{}Handschrift: 1) blaue Tinte, deutsche Kurrent\hspace{1em}2) blaue Tinte, lateinische Kurrent (\noindent{}Adresse)\hspace{1em}
\newline{}Versand: Stempel: »\nobreak{}\oindex{Berlin@\textbf{Berlin}, \emph{kein passender Code gefunden}|pwk}Berlin SW 11 c, 16. 1. 08, 5—6 N.\nobreak{}«.  
\newline{}Schnitzler: mit Bleistift »\textcolor{blue}{Goldmann}« unterstrichen }\toendnotes[C]{\smallbreak}\pstart{}{\pb}Herrn\pend{}\pstart{}Dr. Arthur Schnitzler\pend{}\pstart{}\textcolor{pink}{Wien}{}\ledrightnote{\textcolor{pink}{Wien}}\pend{}\pstart{}\textcolor{pink}{XVIII. Spöttelgaſse 7}{}\ledrightnote{\textcolor{pink}{Edmund-Weiß-Gasse}}.\pend{}
{\bigskip}
\pstart
           \noindent{}{\pb}\textcolor{gray}{\textbf{Glückliches Neujahr!}}\pend
           
\pstart
           {\pb}16. 1. 08.\pend
           
\pstart{}Lieber Freund,\pend
\pstart
           Daß Dir der \label{K_L03460-1v}\edtext{\textcolor{brown}{Grillparzerpreis}{}\ledrightnote{\textcolor{brown}{Franz-Grillparzer-Preis}}}{\lemma{\textnormal{\emph{Grillparzerpreis}}}\Cendnote{\textnormal{\textcolor{blue}{Schnitzler} hatte den \emph{\textcolor{brown}{Franz-Grillparzer-Preis}} für seine Komödie \emph{\textcolor{green}{Zwischenspiel}} erhalten.}}}\label{K_L03460-1h} verliehen worden iſt, hat
               mich aufrichtig gefreut, u. ich beglückwünſche Dich auf das Herzlichſte.\pend
           
\pstart
           Mit vielen Grüßen an Dich u. Deine \textcolor{blue}{Frau}{}\ledrightnote{{$\rightarrow$}\textcolor{blue}{Olga Schnitzler}}{ }{\\[\baselineskip]}Dein {\\[\baselineskip]}\spacefill\mbox{Paul Goldmann.}\pend
           \leftskip=0em{}\endnumbering\briefempfaengerindex{Schnitzler, Arthur@\textsc{Schnitzler, Arthur}!zzzGoldmann, Paul@\emph{von Paul Goldmann}!1908-01-162@{16. 1. 1908}|)be}\mylabel{h}
\begin{anhang}
\end{anhang}\normalsize

\doendnotes{C}
\bigskip
\vfill

\clearpage

\footnotesize

\lohead{\textsc{register}}

% Definiere theindex-Environment komplett neu ohne reledmac
\makeatletter
\renewenvironment{theindex}{%
  \section*{\indexname}%
  \setlength{\parindent}{0pt}%
  \setlength{\parskip}{0pt plus 0.3pt}%
  \let\item\@idxitem
}{%
  \clearpage
}
\makeatother

\IfFileExists{\jobname-pw.ind}{\input{\jobname-pw.ind}}{}

\end{document}

      