%% latex-korrekturansicht-vorspann.tex
%% Vorspann für die Korrekturansicht.
%% Lädt die gemeinsame Datei latex-vorspann.tex mit gesetztem Schalter.

\newif\ifkorrekturansicht
\korrekturansichttrue

\input{../tex-inputs/latex-vorspann}


\renewcommand{\erwaehntePersonen}{Personen: Hermann Bahr, Paul Goldmann, Ludwig Grillich, Olga Schnitzler}
\renewcommand{\erwaehnteOrte}{Orte: Berlin, Dessauer Straße, Wien, Wörthersee}
\renewcommand{\erwaehnteWerke}{Werke: Maria Magdalena. Ein bürgerliches Trauerspiel in drei Akten, Neue Freie Presse, Neues Wiener Tagblatt, Theater, Kunst und Literatur [Vorstellung des Konservatoriums], [Portraitfoto von Olga Gussmann]}
\section[ Paul Goldmann an Olga Gussmann, 10. 5. {[}1901{]}]{Paul Goldmann an Olga Gussmann, 10. 5. {[}1901{]}}
\nopagebreak\mylabel{v}
\rehead{ }\normalsize\beginnumbering\briefempfaengerindex{Schnitzler, Olga@\textsc{Schnitzler, Olga}!zzzGoldmann, Paul@\emph{von Paul Goldmann}!1901-05-102@{10. 5. {[}1901{]}}|(be}
\toendnotes[C]{\smallbreak\pagebreak[2]}\Standort{DLA, A:Schnitzler, HS.NZ85.1.5247.}
\physDesc{Brief, 1 Blatt, 4 Seiten, 1470 Zeichen
\newline{}Handschrift: blaue Tinte, deutsche Kurrent
\newline{}Ordnung: mit Bleistift von \textcolor{blue}{Arthur Schnitzler} das
                                 Jahr »1901.« vermerkt }\toendnotes[C]{\smallbreak}
\pstart
           \noindent{}\raggedleft{}{\pb}\textcolor{gray}{\textbf{\textcolor{pink}{DESSAUERSTRASSE 19}{}\ledrightnote{\textcolor{pink}{Dessauer Straße}}}}\pend
           
\pstart
           \textcolor{pink}{Berlin}{}\ledrightnote{\textcolor{pink}{Berlin}}, 10. Mai.\pend
           
\pstart\center{}Liebes Fräulein \textsc{Olga},\pend
\pstart
           Haben Sie vielen herzlichen Dank für das ſchöne \label{K_L03527-1v}\edtext{\textcolor{green}{Bild}{}\ledrightnote{{$\rightarrow$}\textcolor{green}{[Portraitfoto von Olga Gussmann]}}}{\lemma{\textnormal{\emph{Bild}}}\Cendnote{\textnormal{höchstwahrscheinlich das von \textcolor{blue}{Ludwig Grillich} angefertige \textcolor{green}{Portraitfoto} (\emph{DLA}, B 1989.Q 0249)}}}\label{K_L03527-1h}! Es ſoll mir ein
               lieber Beſitz ſein. Diese \textcolor{pink}{Wien}{}\ledrightnote{\textcolor{pink}{Wien}}er Photographen ſind
               doch mehr Künſtler. Man bekommt nach dieſem \textcolor{green}{Bilde}{}\ledrightnote{{$\rightarrow$}\textcolor{green}{[Portraitfoto von Olga Gussmann]}} wirklich eine lebendige Vorſtellung von Ihnen, und
               Ihre Perſönlichkeit iſt ſehr reizvoll darin ausgedrückt.\pend
           
\pstart
           Mit Dank ſende ich Ihnen die \label{K_L03527-2v}\edtext{\textcolor{green}{Zeitungausſchnitt}{}\ledrightnote{{$\rightarrow$}\textcolor{green}{Theater, Kunst und Literatur [Vorstellung des Konservatoriums]}}e}{\lemma{\textnormal{\emph{Zeitungausſchnitte}}}\Cendnote{\textnormal{Beilagen nicht erhalten. \textcolor{blue}{Bahr} hatte folgende lobende \textcolor{green}{Notiz} über die Aufführung von \emph{\textcolor{green}{Maria Magdalena}} mit \textcolor{blue}{Olga Gussmann} verfasst: \textcolor{blue}{H. B.} [ = \textcolor{blue}{Hermann Bahr}]: \emph{\textcolor{green}{Theater, Kunst und Literatur}}. In: \emph{\textcolor{green}{Neues Wiener Tagblatt}}, Jg. 35, Nr. 118, 1. 5. 1901, S. 7. Siehe auch Arthur Schnitzler an Hermann Bahr, 19. 4. 1901}}}\label{K_L03527-2h} zurück. \textsc{\textcolor{blue}{Bahr}{}\ledrightnote{\textcolor{blue}{Hermann Bahr}}} hat, {\pb}wie gewöhnlich, \substVorne{}\textsuperscript{Blech}\substDazwischen{}Blech\substHinten{}{ }\textcolor{green}{geſchrieben}{}\ledrightnote{{$\rightarrow$}\textcolor{green}{Theater, Kunst und Literatur [Vorstellung des Konservatoriums]}}. Das ſpürt man
               heraus, wenn man auch die \textcolor{green}{Vorſtellung}{}\ledrightnote{{$\rightarrow$}\textcolor{green}{Maria Magdalena. Ein bürgerliches Trauerspiel in drei Akten}} ſelbſt nicht geſehen hat. Ich freue mich, daß Alles gut gegangen
               iſt. Auf die \textcolor{green}{N. Fr. Pr.}{}\ledrightnote{\textcolor{green}{Neue Freie Presse}} bin ich neugierig. Oder
               iſt das \label{K_L03527-3v}\edtext{Referat}{\lemma{\textnormal{\emph{Referat}}}\Cendnote{\textnormal{nicht ermittelt}}}\label{K_L03527-3h} vielleicht ſchon erſchienen und habe
               ich es überſehen?\pend
           
\pstart
           Ob ich Sie \label{K_L03527-4v}\edtext{im Sommer wiederſehen}{\lemma{\textnormal{\emph{im Sommer wiederſehen}}}\Cendnote{\textnormal{siehe Paul Goldmann an Arthur Schnitzler, 26. 4. [1901]}}}\label{K_L03527-4h} werde, weiß ich noch nicht. Jedenfalls kann ich nur im Auguſt auf Urlaub gehen, {\pb}und auch dann
               will ich nicht herumreiſen, ſondern irgendwo feſtſitzen, etwa am \textcolor{pink}{Wörtherſee}{}\ledrightnote{\textcolor{pink}{Wörthersee}}. Ich bat \textsc{\textcolor{blue}{Arthur}{}\ledrightnote{}}{ }\strikeout{darum} deshalb, daß er mit Ihnen im Auguſt an den \textcolor{pink}{Wörtherſee}{}\ledrightnote{\textcolor{pink}{Wörthersee}}
               kommen möge. Wenn das nicht geht, ſehen wir uns hoffentlich auf meiner Rückreiſe in
                  \textcolor{pink}{Wien}{}\ledrightnote{\textcolor{pink}{Wien}}.\pend
           
\pstart
           Sie ſelbſt werden mit \textsc{\textcolor{blue}{Arthur}{}\ledrightnote{}} gewiß einige ſchöne {\pb}Sommermonate verleben.
               Laſſen Sie alle trüben Gedanken zu Hauſe und genießen Sie die ſchöne Welt, die ja
               überhaupt nur dann wirklich ſchön iſt, wenn man Jemanden neben ſich hat, den man \strikeout{\textcolor{gray}{l}} liebt. Auch der Naturgenuß kann nur aus dem Herzen kommen; und das Herz bleibt
               ungerührt, wenn nicht eine Liebe es bewegt. Es gibt keine ſchönen Landſchaften (ohne
               Liebe nämlich).\pend
           
\pstart
           Seien Sie herzlichſt gegrüßt von Ihrem ergebenen {\\[\baselineskip]}\spacefill\mbox{Dr. Paul Goldmann.}\pend
           \leftskip=0em{}\endnumbering\briefempfaengerindex{Schnitzler, Olga@\textsc{Schnitzler, Olga}!zzzGoldmann, Paul@\emph{von Paul Goldmann}!1901-05-102@{10. 5. {[}1901{]}}|)be}\mylabel{h}  \normalsize

\doendnotes{C}
\bigskip
\vfill

\clearpage

\footnotesize

\lohead{\textsc{register}}

% Definiere theindex-Environment komplett neu ohne reledmac
\makeatletter
\renewenvironment{theindex}{%
  \section*{\indexname}%
  \setlength{\parindent}{0pt}%
  \setlength{\parskip}{0pt plus 0.3pt}%
  \let\item\@idxitem
}{%
  \clearpage
}
\makeatother

\IfFileExists{\jobname-pw.ind}{\input{\jobname-pw.ind}}{}

\end{document}

      