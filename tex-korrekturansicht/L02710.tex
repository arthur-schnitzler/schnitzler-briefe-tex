%% latex-korrekturansicht-vorspann.tex
%% Vorspann für die Korrekturansicht.
%% Lädt die gemeinsame Datei latex-vorspann.tex mit gesetztem Schalter.

\newif\ifkorrekturansicht
\korrekturansichttrue

\input{../tex-inputs/latex-vorspann}


               \section[Paul Goldmann und Fedor Mamroth an Arthur Schnitzler, 4. 6. 1893]{ Paul Goldmann und Fedor Mamroth an Arthur Schnitzler,
               4. 6. 1893}\nopagebreak\mylabel{v}\rehead{ }\normalsize\beginnumbering\briefempfaengerindex{Schnitzler, Arthur@\textsc{Schnitzler, Arthur}!zzzMamroth, Fedor@\emph{von Fedor Mamroth}!1893-06-042@{4. 6. 1893}|(be}\briefempfaengerindex{Schnitzler, Arthur@\textsc{Schnitzler, Arthur}!zzzGoldmann, Paul@\emph{von Paul Goldmann}!1893-06-042@{4. 6. 1893}|(be} \toendnotes[C]{\smallbreak\pagebreak[2]} \Standort{DLA, A:Schnitzler, HS.NZ85.1.3163.}
\physDesc{Brief, 1 Blatt, 2 Seiten
\newline{}Handschrift Paul Goldmann: blaue Tinte, deutsche Kurrent}\toendnotes[C]{\smallbreak}\pstart
           \noindent{}{\pb}\textcolor{brown}{\textcolor{gray}{\textbf{\textbf{Frankfurter Zeitung}}}}{}\ledrightnote{\textcolor{brown}{Frankfurter Zeitung}}\pend
           \pstart
           \textcolor{gray}{\textbf{und}}\pend
           \pstart
           \textcolor{gray}{\textbf{\textcolor{brown}{\textbf{Handelsblatt}}{}\ledrightnote{→\textcolor{brown}{Frankfurter Zeitung}}.}}\hfill \textcolor{pink}{\textcolor{gray}{\textbf{Frankfurt a. M.}}}{}\ledrightnote{\textcolor{pink}{Frankfurt am Main}}, 4. Juni \textcolor{gray}{\textbf{189}}3.
                  \pend
           \pstart
           \textcolor{gray}{\textbf{\textcolor{brown}{\textbf{Redaktion.}}{}\ledrightnote{→\textcolor{brown}{Frankfurter Zeitung}}\footnote{\noindent{}\textcolor{gray}{\textbf{Für die \textcolor{brown}{Redaktion} bestimmte Briefe und Sendungen wolle
                              man \so{nicht} an die Person eines Redakteurs,
                              sondern stets \textbf{an die \textcolor{brown}{Redaktion} der \textcolor{green}{Frankfurter Zeitung}} adressiren.}}}}}\pend
           \pstart
           \textcolor{gray}{\textbf{\textbf{Telegramm-Adresse:}}}\pend
           \pstart
           \textcolor{gray}{\textbf{\textbf{\textcolor{brown}{Zeitung}{}\ledrightnote{→\textcolor{brown}{Frankfurter Zeitung}}{ }\textcolor{pink}{Frankfurt Main}{}\ledrightnote{\textcolor{pink}{Frankfurt am Main}}.}}}\pend
           \pstart
           Adreſſen von Verlegern, an die \label{K_L02710-1v}\edtext{wir}{\lemma{\textnormal{\emph{wir}}}\Cendnote{\textnormal{Das macht, in Fortführung der
                  Überlegungeng, die im Brief vom Vortag (Paul Goldmann an Arthur Schnitzler, 3. 6. 1893)
                  dargelegt sind, auch \textcolor{blue}{Fedor Mamroth} zum
                  Absender des Briefes.}}}\label{K_L02710-1h} Dir rathen, Dich zu wenden (zuerſt an \textsc{\textcolor{brown}{\textcolor{blue}{Fischer}{}\ledrightnote{\textcolor{blue}{Samuel Fischer}}}{}\ledrightnote{→\textcolor{brown}{S. Fischer Verlag}}}.)\pend
           \pstart
           \textsc{\textcolor{brown}{\textcolor{blue}{Wilhelm Friedrich}{}\ledrightnote{\textcolor{blue}{Wilhelm Friedrich}}}{}\ledrightnote{→\textcolor{brown}{Verlag Wilhelm Friedrich}}}{ }\textsc{\textcolor{pink}{Leipzig}{}\ledrightnote{\textcolor{pink}{Leipzig}}}.\pend
           \pstart
           \textsc{\textcolor{brown}{Schlesische Buchdruckerei Kunst- und
                     Verlags-Anstalt vorm. \textcolor{blue}{S. Schottlaender}{}\ledrightnote{\textcolor{blue}{Salo Schottlaender}}}{}\ledrightnote{\textcolor{brown}{S. Schottländer}}, \textcolor{pink}{Breslau}{}\ledrightnote{\textcolor{pink}{Breslau}}}.\pend
           \pstart
           \textsc{\textcolor{brown}{\textcolor{blue}{E. Piersons}{}\ledrightnote{\textcolor{blue}{Edgar Pierson}} Verlag}{}\ledrightnote{\textcolor{brown}{E. Pierson’s Verlag}}, \textcolor{pink}{Dresden, Altstadt}{}\ledrightnote{\textcolor{pink}{Dresden}}}.\pend
           \pstart
           \textsc{\textcolor{brown}{\textcolor{blue}{S. Fischer}{}\ledrightnote{\textcolor{blue}{Samuel Fischer}}}{}\ledrightnote{\textcolor{brown}{S. Fischer Verlag}}, \textcolor{pink}{Berlin}{}\ledrightnote{\textcolor{pink}{Berlin}}}{ }\textcolor{pink}{\textsc{Koethenerstraße} 44}{}\ledrightnote{\textcolor{pink}{Köthenerstraße}}.\pend
           \pstart
           \textsc{\textcolor{brown}{\textcolor{blue}{Freund}{}\ledrightnote{\textcolor{blue}{Carl Freund}} und \textcolor{blue}{Jeckel}{}\ledrightnote{\textcolor{blue}{Max Jeckel}}}{}\ledrightnote{\textcolor{brown}{Freund {\kaufmannsund} Jeckel}}}, {\pb}\textsc{\textcolor{pink}{Berlin N. W. 23, Altonaerstraße 37a}{}\ledrightnote{\textcolor{pink}{Altonaer Straße}}}.\pend
           \endnumbering\briefempfaengerindex{Schnitzler, Arthur@\textsc{Schnitzler, Arthur}!zzzMamroth, Fedor@\emph{von Fedor Mamroth}!1893-06-042@{4. 6. 1893}|)be}\briefempfaengerindex{Schnitzler, Arthur@\textsc{Schnitzler, Arthur}!zzzGoldmann, Paul@\emph{von Paul Goldmann}!1893-06-042@{4. 6. 1893}|)be}\mylabel{h}\begin{anhang}\end{anhang}\normalsize

\doendnotes{C}
\bigskip
\vfill

\clearpage

\footnotesize

\lohead{\textsc{register}}

% Definiere theindex-Environment komplett neu ohne reledmac
\makeatletter
\renewenvironment{theindex}{%
  \section*{\indexname}%
  \setlength{\parindent}{0pt}%
  \setlength{\parskip}{0pt plus 0.3pt}%
  \let\item\@idxitem
}{%
  \clearpage
}
\makeatother

\IfFileExists{\jobname-pw.ind}{\input{\jobname-pw.ind}}{}

\end{document}

      