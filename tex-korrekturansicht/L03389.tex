%% latex-korrekturansicht-vorspann.tex
%% Vorspann für die Korrekturansicht.
%% Lädt die gemeinsame Datei latex-vorspann.tex mit gesetztem Schalter.

\newif\ifkorrekturansicht
\korrekturansichttrue

\input{../tex-inputs/latex-vorspann}


\renewcommand{\erwaehntePersonen}{Personen:  ?? [Berliner Musikkorrespondent der National-Zeitung], Hermann Bahr, Ludwig Bauer, Richard Beer-Hofmann, Emilie Dorothea Popper, Theodore Rottenberg, Olga Schnitzler, Heinrich Schnitzler, Louise Schnitzler, Karl Eduard Vehse}
\renewcommand{\erwaehnteInstitutionen}{Institutionen: National-Zeitung, Reichstag, Verein zur Förderung der Künste}
\renewcommand{\erwaehnteOrte}{Orte: Berlin, Dessauer Straße, Frankfurt am Main, Semmering, Wien, Ägypten}
\renewcommand{\erwaehnteWerke}{Werke: Briefe, die ihn nicht erreichten, Der einsame Weg. Schauspiel in fünf Akten, Die Gouvernante, Neue Freie Presse, Theater- und Kunstnachrichten. [Konzerte.]. [Man schreibt uns aus Berlin], [Aus Berlin wird uns gemeldet: »Der einsame Weg«]}
\section[ Paul Goldmann an Arthur Schnitzler, 13. 12. {[}1903{]}]{Paul Goldmann an Arthur Schnitzler, 13. 12. {[}1903{]}}
\nopagebreak\mylabel{v}
\rehead{ }\normalsize\beginnumbering\briefempfaengerindex{Schnitzler, Arthur@\textsc{Schnitzler, Arthur}!zzzGoldmann, Paul@\emph{von Paul Goldmann}!1903-12-132@{13. 12. {[}1903{]}}|(be}
\toendnotes[C]{\smallbreak\pagebreak[2]}\Standort{DLA, A:Schnitzler, HS.NZ85.1.3173.}
\physDesc{Brief, 2 Blätter, 7 Seiten
\newline{}Handschrift: blaue Tinte, deutsche Kurrent
\newline{}Schnitzler: 1) mit Bleistift das Jahr »{[}1{]}903« vermerkt  2) mit rotem Buntstift neun Unterstreichungen}\toendnotes[C]{\smallbreak}
\pstart
           \noindent{}\raggedleft{}{\pb}\textcolor{gray}{\textbf{\textcolor{pink}{DESSAUERSTRASSE 19}{}\ledrightnote{\textcolor{pink}{Dessauer Straße}}}}\pend
           
\pstart
           \textcolor{pink}{Berlin}{}\ledrightnote{\textcolor{pink}{Berlin}}, 13. Dezember.\pend
           
\pstart\center{}Mein lieber Freund,\pend
\pstart
           Ich habe mich ſehr gefreut, wieder einmal einen Brief von Dir zu erhalten. Auch die
               guten Nachrichten über Deine »\textcolor{blue}{engſte Familie}{}\ledrightnote{{$\rightarrow$}\textcolor{blue}{Olga Schnitzler}{\newline}{$\rightarrow$}\textcolor{blue}{Heinrich Schnitzler}}« haben mir viel Freude bereitet.\pend
           
\pstart
           Daß ich \introOben{}für\introOben{}{ }\label{K_L03389-1v}\edtext{Fräulein \textsc{\textcolor{blue}{Popper}{}\ledrightnote{\textcolor{blue}{Emilie Dorothea Popper}}}}{\lemma{\textnormal{\emph{Fräulein Popper}}}\Cendnote{\textnormal{siehe Paul Goldmann an Arthur Schnitzler, 14. 11. [1903]}}}\label{K_L03389-1h}, nachdem ſie mir von Dir und Deiner \textcolor{blue}{Mutter}{}\ledrightnote{{$\rightarrow$}\textcolor{blue}{Louise Schnitzler}} empfohlen worden, Alles that, was in meiner Macht
               ſtand, iſt ſelbſtverſtändlich. Wenn Du ſie ſiehſt, ſo ſage ihr, daß der \textcolor{blue}{Referent}{}\ledrightnote{{$\rightarrow$}\textcolor{blue}{?? [Berliner Musikkorrespondent der National-Zeitung]}} der »\textcolor{brown}{Nationalzeitung}{}\ledrightnote{\textcolor{brown}{National-Zeitung}}«, an den ich ſie empfohlen, ſehr freundlich
               über ſie \label{K_L03389-2v}\edtext{\textcolor{green}{geſchrieben}{}\ledrightnote{{$\rightarrow$}\textcolor{green}{Theater- und Kunstnachrichten. [Konzerte.]. [Man schreibt uns aus Berlin]}}}{\lemma{\textnormal{\emph{geſchrieben}}}\Cendnote{\textnormal{Höchstwahrscheinlich Bezug auf folgende
                     \textcolor{green}{Meldung} über ein
                  Konzert von \textcolor{blue}{Dora Popper}: [\textcolor{blue}{Berliner Musikkorrespondent der
                        National-Zeitung}:] \emph{\textcolor{green}{Theater- und
                        Kunstnachrichten. [Konzerte.]. [Man schreibt uns aus Berlin]}}. In: \emph{\textcolor{green}{Neue Freie Presse}}, Nr. 14093, 20. 11. 1903, S. 9.}}}\label{K_L03389-2h} hat.\pend
           
\pstart
           {\pb}Am \label{K_L03389-3v}\edtext{\textcolor{pink}{Semmering}{}\ledrightnote{\textcolor{pink}{Semmering}}}{\lemma{\textnormal{\emph{Semmering}}}\Cendnote{\textnormal{\textcolor{blue}{Arthur} und \textcolor{blue}{Olga Schnitzler} waren zwischen 6. 11. 1903 und 9. 11. 1903 am \textcolor{pink}{Semmering} gewesen.}}}\label{K_L03389-3h} muß es im Spätherbſt ſehr ſchön geweſen ſein.
               Haſt Du weitere Winter-Reiſepläne? Über die Vorleſung Deines \textcolor{green}{Stück}{}\ledrightnote{{$\rightarrow$}\textcolor{green}{Die Gouvernante}}es durch \textsc{\textcolor{blue}{Ludwig Bauer}{}\ledrightnote{\textcolor{blue}{Ludwig Bauer}}} habe ich ſelbſtverſtändlich ein \label{K_L03389-4v}\edtext{Telegramm}{\lemma{\textnormal{\emph{Telegramm}}}\Cendnote{\textnormal{\textcolor{blue}{Ludwig Bauer}s Vorlesung von \emph{\textcolor{green}{Die Gouvernante}} fand am 2. 12. 1903 in \textcolor{pink}{Berlin} statt und
                  wurde vom \emph{\textcolor{brown}{Verein zur Förderung der Künste}}
                  veranstaltet. Siehe auch A. S.: \emph{Tagebuch}, 4. 12. 1903.
                     \textcolor{blue}{Goldmann}s Telegramm dürfte tatsächlich
                  nicht veröffentlicht worden sein.}}}\label{K_L03389-4h} geſandt. Es iſt nicht erſchienen (oder
               ſollte es mir entgangen ſein?) Dieſes Nichterſcheinen richtet ſich aber ſicherlich
               gegen \textsc{\textcolor{blue}{Bauer}{}\ledrightnote{\textcolor{blue}{Ludwig Bauer}}} und nicht gegen Dich. Mein \label{K_L03389-5v}\edtext{\textcolor{green}{Telegramm}{}\ledrightnote{{$\rightarrow$}\textcolor{green}{[Aus Berlin wird uns gemeldet: »Der einsame Weg«]}}}{\lemma{\textnormal{\emph{Telegramm}}}\Cendnote{\textnormal{[\textcolor{blue}{Paul Goldmann}:] \emph{\textcolor{green}{[Aus Berlin wird uns gemeldet: »Der einsame Weg«]}}. In:
                        \emph{\textcolor{green}{Neue Freie Presse}}, Nr. 14115, 12. 12. 1903, Morgenblatt, S. 10.}}}\label{K_L03389-5h}
               über das Bevorſtehen Deiner \textsc{\begin{otherlanguage}{french}\textcolor{green}{Première}{}\ledrightnote{{$\rightarrow$}\textcolor{green}{Der einsame Weg. Schauspiel in fünf Akten}}\end{otherlanguage}} iſt ja erſchienen.\pend
           
\pstart
           Zum Leſen komme ich gar nicht mehr, ſeit die furchtbare \textcolor{brown}{Reichstag}{}\ledrightnote{\textcolor{brown}{Reichstag}}sarbeit begonnen hat. \label{K_L03389-6v}\edtext{\textsc{\textcolor{blue}{Vehse}{}\ledrightnote{\textcolor{blue}{Karl Eduard Vehse}}}}{\lemma{\textnormal{\emph{Vehse}}}\Cendnote{\textnormal{Werk nicht ermittelt}}}\label{K_L03389-6h} habe ich
               habe ich mir gekauft (für 67 \textsc{MK}; was haſt Du gezahlt?).
               Haſt Du das gegenwärtige {\pb}deutſche Modebuch \label{K_L03389-7v}\edtext{»\textcolor{green}{Briefe\textcolor{gray}{,} die ihn nicht erreichten}{}\ledrightnote{\textcolor{green}{Briefe, die ihn nicht erreichten}}«}{\lemma{\textnormal{\emph{»Briefe, … erreichten«}}}\Cendnote{\textnormal{\textcolor{blue}{Schnitzler} hatte den \textcolor{green}{Briefroman} nicht gelesen, siehe Paul Goldmann an Arthur Schnitzler, 27. 6. [1903].}}}\label{K_L03389-7h} ſchon geleſen?
               Es iſt zu empfehlen.\pend
           
\pstart
           Meine \textcolor{blue}{Freundin}{}\ledrightnote{{$\rightarrow$}\textcolor{blue}{Theodore Rottenberg}} in \textcolor{pink}{Frankfurt}{}\ledrightnote{\textcolor{pink}{Frankfurt am Main}} war krank. Lungenentzündung oder ſo
               etwas. Ich bin ſehr beſorgt. Aus ihren Briefen werde ich nicht recht klug inbezug auf
               ihre Krankheit. Die Ärzte ſagen ihr auch offenbar nicht die Wahrheit; aber aus dem
               Umſtande, daß die Ärzte eine ſofortige Reiſe nach dem Süden, womöglich \textcolor{pink}{Egypten}{}\ledrightnote{\textcolor{pink}{Ägypten}}, empfehlen, folgere ich allerlei
               Schlimmes.\pend
           
\pstart
           Als ich \label{K_L03389-8v}\edtext{das letzte Mal in \textcolor{pink}{Wien}{}\ledrightnote{\textcolor{pink}{Wien}}}{\lemma{\textnormal{\emph{das letzte Mal in Wien}}}\Cendnote{\textnormal{vermutlich Ende September/Anfang Oktober 1903, siehe Paul Goldmann an Arthur Schnitzler, 7. 9. 1903}}}\label{K_L03389-8h} mit Dir und Deiner \textcolor{blue}{Frau}{}\ledrightnote{{$\rightarrow$}\textcolor{blue}{Olga Schnitzler}} über dieſe Angelegenheit ſprach, ſagteſt Du, daß ich eigentlich nunmehr
               gegen \strikeout{die} meine \textcolor{blue}{Freundin}{}\ledrightnote{{$\rightarrow$}\textcolor{blue}{Theodore Rottenberg}} ſei, indem ich ſie in {\pb}der \label{K_L03389-9v}\edtext{Illuſion}{\lemma{\textnormal{\emph{Illuſion}}}\Cendnote{\textnormal{siehe Paul Goldmann an Arthur Schnitzler, 14. 11. [1903]}}}\label{K_L03389-9h} ließe, ich würde \textcolor{blue}{ſie}{}\ledrightnote{{$\rightarrow$}\textcolor{blue}{Theodore Rottenberg}}
               heirathen. Ich habe über dieſe Deine Worte oft nachgedacht. \strikeout{\textcolor{gray}{D}} Du haſt im Weſentlichen Recht; und da mich der Vorwurf der Unwahrheit ſehr
               bedrückt, bin ich ſeit Wochen bemüht, in meinen Briefen allmälig zur Wahrheit
               einzulenken. Sie weiß heut, daß ich ſie, fürs Erſte wenigſtens, nicht heirathen kann;
               aber ſie klammert ſich trotzdem an mich als \strikeout{ih\textcolor{gray}{ren}} denjenigen, der ſie, wie ſie ſchreibt, »vom Abgrund zurückgeriſſen hat« und
               als ihren einzigen Halt.\pend
           
\pstart
           Was aus Alledem werden ſoll, weiß der liebe Gott allein.\pend
           
\pstart
           Das Unglück wollte es, daß {\pb}\strikeout{daß} ich \textsc{\textcolor{blue}{Bahr}{}\ledrightnote{\textcolor{blue}{Hermann Bahr}}}, nachdem ich das Glück gehabt hatte, \strikeout{wahrſ\textcolor{gray}{e}} während ſeines \textcolor{pink}{Berlin}{}\ledrightnote{\textcolor{pink}{Berlin}}er Aufenthalts
               nigends mit ihm zuſammen zukommen, \introOben{}geſtern\introOben{} auf der Straße traf. Ich blieb ſtehen, und wir geriethen in ein längeres
               Geſpräch. Dieſer alberne, dünkelhafte und verlogene Menſch hat \strikeout{mich} mich immer heftig gereizt. Diesmal war dies ganz
               beſonders der Fall, und er ſchien es auch darauf angelegt zu haben, mich zu
               provoziren. So theilte er mir Äußerungen mit, die Du und \textsc{\textcolor{blue}{Beer-Hofmann}{}\ledrightnote{\textcolor{blue}{Richard Beer-Hofmann}}} gethan haben ſollen. Ich gerieth in Hitze und antwortete {\pb}demgemäß. Hinterher wurde es mir klar, daß Deine und
                  \textsc{\textcolor{blue}{Richard}{}\ledrightnote{\textcolor{blue}{Richard Beer-Hofmann}}s} Äußerungen offenbar entſtellt
               wiedergegeben waren. Ich vermuthe, daß \textcolor{blue}{er}{}\ledrightnote{{$\rightarrow$}\textcolor{blue}{Hermann Bahr}} Dir jetzt auch meine Äußerungen entſtellt \label{K_L03389-11v}\edtext{berichten}{\lemma{\textnormal{\emph{berichten}}}\Cendnote{\textnormal{\textcolor{blue}{Schnitzler} und \textcolor{blue}{Bahr} sprachen jedenfalls kurz darauf über \textcolor{blue}{Goldmann}, vgl. A. S.: \emph{Tagebuch}, 18. 12. 1903 und Bahr/Schnitzler, D041448.}}}\label{K_L03389-11h} wird, und bitte Dich, falls dies geſchehen ſollte,
               nicht darauf zu achten.\pend
           
\pstart
           Wenn Du nächſtens einmal wieder Zeit findeſt, mir zu ſchreiben, wirſt Du mir eine
               große Freude machen. Weihnachten gehe ich
               wahrſcheinlich nach {\pb}\textcolor{pink}{Frankfurt}{}\ledrightnote{\textcolor{pink}{Frankfurt am Main}}.\pend
           
\pstart
           Viele herzliche Grüße an Dich und Deine \textcolor{blue}{Frau}{}\ledrightnote{{$\rightarrow$}\textcolor{blue}{Olga Schnitzler}} von Deinem getreuen {\\[\baselineskip]}\spacefill\mbox{Paul Goldmann.}\pend
           \leftskip=0em{}\endnumbering\briefempfaengerindex{Schnitzler, Arthur@\textsc{Schnitzler, Arthur}!zzzGoldmann, Paul@\emph{von Paul Goldmann}!1903-12-132@{13. 12. {[}1903{]}}|)be}\mylabel{h}
\begin{anhang}
\end{anhang}\normalsize

\doendnotes{C}
\bigskip
\vfill

\clearpage

\footnotesize

\lohead{\textsc{register}}

% Definiere theindex-Environment komplett neu ohne reledmac
\makeatletter
\renewenvironment{theindex}{%
  \section*{\indexname}%
  \setlength{\parindent}{0pt}%
  \setlength{\parskip}{0pt plus 0.3pt}%
  \let\item\@idxitem
}{%
  \clearpage
}
\makeatother

\IfFileExists{\jobname-pw.ind}{\input{\jobname-pw.ind}}{}

\end{document}

      