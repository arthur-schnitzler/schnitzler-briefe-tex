%% latex-korrekturansicht-vorspann.tex
%% Vorspann für die Korrekturansicht.
%% Lädt die gemeinsame Datei latex-vorspann.tex mit gesetztem Schalter.

\newif\ifkorrekturansicht
\korrekturansichttrue

\input{../tex-inputs/latex-vorspann}


               \section[Richard Beer-Hofmann an Arthur Schnitzler, 28. 7. 1893]{ Richard Beer-Hofmann an Arthur Schnitzler, 28. 7. 1893}\nopagebreak\mylabel{v}\rehead{ }\normalsize\beginnumbering\briefempfaengerindex{Schnitzler, Arthur@\textsc{Schnitzler, Arthur}!zzzBeer-Hofmann, Richard@\emph{von Richard Beer-Hofmann}!1893-07-281@{28. 7. 1893}|(be} \toendnotes[C]{\smallbreak\pagebreak[2]} \Standort{CUL, Schnitzler, B 8.}
\physDesc{Brief, 1 Blatt, 4 Seiten
\newline{}Handschrift: Bleistift, deutsche Kurrent
\newline{}Schnitzler: mit Bleistift nummeriert: »22« }\buchAbdrucke{\weitereDrucke{Arthur Schnitzler, Richard Beer-Hofmann: \emph{Briefwechsel 1891–1931}. Hg. Konstanze Fliedl. Wien, Zürich: \emph{Europaverlag} 1992, S. 48–49.} }\toendnotes[C]{\smallbreak}\pstart
           \raggedleft{}{\pb}Freitag{ }Mittag. \pend
           \pstart
           Lieber Arthur! Bin wieder seit vorgestern nachts hier. Las Ihren
               Brief an Frau \textcolor{blue}{F.}{}\ledrightnote{\textcolor{blue}{Bertha Flegmann}}; das \textcolor{green}{Telegramm}{}\ledrightnote{→\textcolor{green}{Aus Ischl, 14. Juli, schreibt man uns: …}} ist nicht von ihr; von \textcolor{blue}{Ben.}{}\ledrightnote{\textcolor{blue}{Markus Benedict}}?\pend
           \pstart
           Im \textcolor{brown}{Börsencourir}{}\ledrightnote{\textcolor{brown}{Berliner Börsen-Courier}} von \strikeout{ge} – ? – ich höre in dem, der vorgestern hier war, – ich hoffe ihn zu
               erhalten {[}–{]} soll eine lange günstige \textcolor{green}{Notiz}{}\ledrightnote{→\textcolor{green}{[Man schreibt uns aus Ischl]}} stehen.\pend
           \pstart
           {\pb}Ich habe \textcolor{blue}{Paul Horn}{}\ledrightnote{\textcolor{blue}{Paul Horn}} als er hier war sämtliche Daten gegeben; auch bez.
               Lektüre durch \textcolor{blue}{Reicher}{}\ledrightnote{\textcolor{blue}{Emanuel Reicher}} u. \textcolor{blue}{Jarno}{}\ledrightnote{\textcolor{blue}{Josef Jarno}} in \textcolor{pink}{Berlin}{}\ledrightnote{\textcolor{pink}{Berlin}}; dürfte also
               darin stehen. Heute wieder \label{K_L00245_1v}\edtext{\textcolor{blue}{Mamroth}{}\ledrightnote{\textcolor{blue}{Fedor Mamroth}} zitirt}{\lemma{\textnormal{\emph{Mamroth zitirt}}}\Cendnote{\textnormal{Fedor Mamroth an Arthur Schnitzler, 5. 3. 1893.}}}\label{K_L00245_1h} (\textcolor{blue}{Tolstoi}{}\ledrightnote{\textcolor{blue}{Leo N. von Tolstoi}}) vor Frau
                  \textcolor{blue}{Kalbek}{}\ledrightnote{\textcolor{blue}{Julie Kalbeck}}.\pend
           \pstart
           Ich glaube es wird gehen. Verhalten Sie sich nur gut mit \textcolor{blue}{F.}{}\ledrightnote{\textcolor{blue}{Bertha Flegmann}}; sie setzt sich {\pb}wirklich für ihre Freunde ein.
               Bitte \uline{urgiren} Sie den \textcolor{blue}{Abschreiber}{}\ledrightnote{→\textcolor{blue}{?? [Schreibkraft für Arthur Schnitzler]}}; mir ist sehr darum zu thun die Sache hier
               vorlesen zu können solange \textcolor{blue}{Kalbeks}{}\ledrightnote{\textcolor{blue}{Max Kalbeck}{\newline}\textcolor{blue}{Julie Kalbeck}} u. \substVorne{}\textsuperscript{I}\substDazwischen{}i\substHinten{}hre Schwester eine Frau \textcolor{blue}{Lion}{}\ledrightnote{\textcolor{blue}{Lion}} da ist.
               Bitte!\pend
           \pstart
           Heute, Freitag Mittag, – ist noch nichts eingetroffen,
               hoffentlich kreuzt {\pb}es sich mit
               meinem Brief; der Schluss des \textcolor{green}{Kindes}{}\ledrightnote{\textcolor{green}{Das Kind}} ist endgiltig
               geändert, hoffentlich gefällt er jetzt besser.\pend
           \pstart
           Grüßen Sie \textcolor{blue}{Schwarzkopf}{}\ledrightnote{\textcolor{blue}{Gustav Schwarzkopf}}{ }\textcolor{blue}{Salten}{}\ledrightnote{\textcolor{blue}{Felix Salten}}. Herzlichst Ihr\pend
           \pstart \spacefill\mbox{Richard}\pend{}\pstart
           \textcolor{pink}{Ischl}{}\ledrightnote{\textcolor{pink}{Bad Ischl}}. 28 Juli 93.\pend
           \pstart
           \noindent{}Was sagen Sie zu \strikeout{Schr}{ }\textcolor{blue}{Wengraf}{}\ledrightnote{\textcolor{blue}{Edmund Wengraf}}{ }\textcolor{green}{\textcolor{blue}{Hirschfeld}{}\ledrightnote{\textcolor{blue}{Robert Hirschfeld}}}{}\ledrightnote{→\textcolor{green}{Zwei Freunde Burckhards}}?\pend
           \pstart
           Schreiben Sie \textcolor{blue}{Löbl}{}\ledrightnote{\textcolor{blue}{Emil Löbl}} ein paar Zeilen. Vide: \textcolor{green}{Ischler Brief}{}\ledrightnote{→\textcolor{green}{Ischler Brief}}.\pend
           \endnumbering\briefempfaengerindex{Schnitzler, Arthur@\textsc{Schnitzler, Arthur}!zzzBeer-Hofmann, Richard@\emph{von Richard Beer-Hofmann}!1893-07-281@{28. 7. 1893}|)be}\mylabel{h}  \normalsize

\doendnotes{C}
\bigskip
\vfill

\clearpage

\footnotesize

\lohead{\textsc{register}}

% Definiere theindex-Environment komplett neu ohne reledmac
\makeatletter
\renewenvironment{theindex}{%
  \section*{\indexname}%
  \setlength{\parindent}{0pt}%
  \setlength{\parskip}{0pt plus 0.3pt}%
  \let\item\@idxitem
}{%
  \clearpage
}
\makeatother

\IfFileExists{\jobname-pw.ind}{\input{\jobname-pw.ind}}{}

\end{document}

      