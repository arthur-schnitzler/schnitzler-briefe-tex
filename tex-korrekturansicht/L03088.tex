%% latex-korrekturansicht-vorspann.tex
%% Vorspann für die Korrekturansicht.
%% Lädt die gemeinsame Datei latex-vorspann.tex mit gesetztem Schalter.

\newif\ifkorrekturansicht
\korrekturansichttrue

\input{../tex-inputs/latex-vorspann}


\renewcommand{\erwaehntePersonen}{Personen: Richard Beer-Hofmann, Moriz Benedikt, Berta Doepler, Peter Dorner, Marie Glümer, Auguste Glümer, Maximilian Harden, Gerhart Hauptmann, Herman Heijermans, Herman (Sr.) Heijermans, Hugo von Hofmannsthal, Alfred Kerr, Else Lasker-Schüler, Felix Salten, Olga Schnitzler, Irene Triesch}
\renewcommand{\erwaehnteInstitutionen}{Institutionen: Deutsches Theater Berlin, Frankfurter Stadttheater, Jung-Wiener Theater zum Lieben Augustin, Verlag der Zukunft, Überbrettl}
\renewcommand{\erwaehnteOrte}{Orte: Amsterdam, Berlin, Dessauer Straße, Wien}
\renewcommand{\erwaehnteWerke}{Werke: Berliner Theater. »Einsame Menschen« im Deutschen Theater, Die Hoffnung auf Segen. Eine Fischertragödie in vier Acten, Die Schmiedekunst seit dem Ende der Renaissance, Die Zukunft, Einsame Menschen. Drama, Lebendige Stunden. Vier Einakter, Neue Freie Presse, Op hoop van zegen. Spel van de zee in vier bedrijven, Physiologie des Kunstempfindens. Der Grundsatz, Theater- und Kunstnachrichten [zur Eröffnung des Jung-Wiener Theaters zum lieben Augustin], [Kritik zu Die Hoffnung]}
\section[ Paul Goldmann an Arthur Schnitzler, 7. 10. {[}1901{]}]{Paul Goldmann an Arthur Schnitzler, 7. 10. {[}1901{]}}
\nopagebreak\mylabel{v}
\rehead{ }\normalsize\beginnumbering\briefempfaengerindex{Schnitzler, Arthur@\textsc{Schnitzler, Arthur}!zzzGoldmann, Paul@\emph{von Paul Goldmann}!1901-10-072@{7. 10. {[}1901{]}}|(be}
\toendnotes[C]{\smallbreak\pagebreak[2]}\Standort{DLA, A:Schnitzler, HS.NZ85.1.3171.}
\physDesc{Brief, 1 Blatt, 4 Seiten
\newline{}Handschrift: blaue Tinte, deutsche Kurrent
\newline{}Schnitzler: 1) mit Bleistift das Jahr »{[}1{]}901« vermerkt  2) mit rotem Buntstift acht Unterstreichungen}\toendnotes[C]{\smallbreak}
\pstart
           \noindent{}\raggedleft{}{\pb}\textcolor{pink}{\textcolor{gray}{\textbf{DESSAUERSTRASSE 19}}}{}\ledrightnote{\textcolor{pink}{Dessauer Straße}}\pend
           
\pstart
           \textcolor{pink}{Berlin}{}\ledrightnote{\textcolor{pink}{Berlin}}, 7. Oktober.\pend
           
\pstart\center{}Mein lieber Freund,\pend
\pstart
           Dein Brief iſt im Ganzen recht erfreulich, – mit Ausnahme von Kopfſchmerzen und
                  \label{K_L03088-1v}\edtext{Ohrenklingen}{\lemma{\textnormal{\emph{Ohrenklingen}}}\Cendnote{\textnormal{\textcolor{blue}{Schnitzler} litt seit
                     Herbst 1896 an Otosklerose – einer Verknöcherung des Innenohrs mit
                  zunehmender Schwerhörigkeit.}}}\label{K_L03088-1h}, gegen die ich Dir leider nicht helfen kann.
               Das ſpielt in Deinem Leben offenbar dieſelbe Rolle, wie \label{K_L03088-2v}\edtext{\textsc{\textcolor{blue}{Benedikt}{}\ledrightnote{\textcolor{blue}{Moriz Benedikt}}}}{\lemma{\textnormal{\emph{Benedikt}}}\Cendnote{\textnormal{\textcolor{blue}{Moriz Benedikt} war als Herausgeber der \emph{\textcolor{green}{Neuen Freien Presse}}{ }\textcolor{blue}{Goldmann}s Vorgesetzter.}}}\label{K_L03088-2h} in dem
               meinen. Es ſcheint, daß zu jedem Leben ein wenig \textsc{\textcolor{blue}{Benedikt}{}\ledrightnote{\textcolor{blue}{Moriz Benedikt}}} gehört.\pend
           
\pstart
           Gegen ein \label{K_L03088-3v}\edtext{Auftreten \textsc{\textcolor{blue}{Olga}{}\ledrightnote{\textcolor{blue}{Olga Schnitzler}}s} bei \textsc{\textcolor{brown}{\textcolor{blue}{Salten}{}\ledrightnote{\textcolor{blue}{Felix Salten}}}{}\ledrightnote{{$\rightarrow$}\textcolor{brown}{Jung-Wiener Theater zum Lieben Augustin}}}}{\lemma{\textnormal{\emph{Auftreten … Salten}}}\Cendnote{\textnormal{siehe Paul Goldmann an Arthur Schnitzler, 16. 5. [1901]}}}\label{K_L03088-3h} wäre ich entſchieden. Soll ihr für alle Zeiten die \strikeout{\textcolor{gray}{×}}{ }\label{K_L03088-4v}\edtext{Überbrettl}{\lemma{\textnormal{\emph{Überbrettl}}}\Cendnote{\textnormal{Bezug auf die Orientierung des \emph{\textcolor{brown}{Jung-Wiener Theaters zum Lieben Augustin}} am \emph{\textcolor{brown}{Überbrettl}}, siehe Paul Goldmann an Arthur Schnitzler, 18. 2. [1901]}}}\label{K_L03088-4h}-Marke aufgeprägt werden? Das Programm der \textsc{\textcolor{blue}{Salten}{}\ledrightnote{\textcolor{blue}{Felix Salten}}schen}{ }\textcolor{brown}{Unternehmung}{}\ledrightnote{{$\rightarrow$}\textcolor{brown}{Jung-Wiener Theater zum Lieben Augustin}}, das ich heut in der \textcolor{green}{N. Fr.
                  Pr.}{}\ledrightnote{\textcolor{green}{Neue Freie Presse}}{ }\label{K_L03088-8v}\edtext{\textcolor{green}{leſe}{}\ledrightnote{{$\rightarrow$}\textcolor{green}{Theater- und Kunstnachrichten [zur Eröffnung des Jung-Wiener Theaters zum lieben Augustin]}}}{\lemma{\textnormal{\emph{leſe}}}\Cendnote{\textnormal{[O. V.]: \emph{\textcolor{green}{Theater- und Kunstnachrichten.
                        [Zur Eröffnung des Jung-Wiener Theaters zum lieben Augustin]}}. In: \emph{\textcolor{green}{Neue Freie Presse}}, Nr. 13332, 6. 10. 1901, S. 8.}}}\label{K_L03088-8h}, iſt ein großer
               Kuddelmuddel. Der Mann ſcheint \strikeout{ab\textcolor{gray}{zo}} abſolut nicht zu wiſſen, was er will.\pend
           
\pstart
           »\textcolor{green}{Lebendige Stunden}{}\ledrightnote{\textcolor{green}{Lebendige Stunden. Vier Einakter}}« iſt ein hübſcher Titel. {\pb}Aber er ſagt mir nichts. \label{K_L03088-9v}\edtext{Warum »lebendig«? Warum »Stunden«?}{\lemma{\textnormal{\emph{Warum … »Stunden«?}}}\Cendnote{\textnormal{\textcolor{blue}{Schnitzler} rekurrierte mit dem Titel \emph{\textcolor{green}{Lebendige Stunden}} wohl auf das verbindende
                  thematische Element seines \textcolor{green}{Einakterzyklus}: das Verhältnis von Kunst und Leben, das immer wieder vom
                  Tod durchkreuzt wird.}}}\label{K_L03088-9h} Und Worte ohne \strikeout{Sin\textcolor{gray}{n}} Sinn zu gebrauchen, blos weil ſie ſchön klingen, iſt doch gar zu \label{K_L03088-19v}\edtext{\textsc{\textcolor{blue}{Hoffmannsthal}{}\ledrightnote{\textcolor{blue}{Hugo von Hofmannsthal}}isch}}{\lemma{\textnormal{\emph{Hoffmannsthalisch}}}\Cendnote{\textnormal{siehe Paul Goldmann an Arthur Schnitzler, 19. 6. [1894]}}}\label{K_L03088-19h}.\pend
           
\pstart
           Ich ſah neulich »\textcolor{green}{Einſame Menſchen}{}\ledrightnote{\textcolor{green}{Einsame Menschen. Drama}}« und war ſtarr
               über die Talentloſigkeit. Ich begreife Euch nicht, daß Ihr dieſen \label{K_L03088-90v}\edtext{\textcolor{blue}{Menſchen}{}\ledrightnote{{$\rightarrow$}\textcolor{blue}{Gerhart Hauptmann}}}{\lemma{\textnormal{\emph{Menſchen}}}\Cendnote{\textnormal{Zu \textcolor{blue}{Goldmann}s Kritik an \textcolor{blue}{Gerhart
                     Hauptmann} siehe etwa Paul Goldmann an Arthur Schnitzler, 31. 12. [1900].
                  Siehe auch \textcolor{blue}{Paul Goldmann}: \emph{\textcolor{green}{Berliner Theater. »Einsame Menschen« im Deutschen Theater}}.
                     In: \emph{\textcolor{green}{Neue Freie Presse}}, Nr. 13.345, 19. 10. 1901, Morgenblatt, S. 1–3.}}}\label{K_L03088-90h}
               auch nur einen Augenblick ernſt nehmen könnt.\pend
           
\pstart
           Ein ſehr \strikeout{ſchön} ſchönes Stück iſt \label{K_L03088-24v}\edtext{»\textcolor{green}{Die
                  Hoffnung}{}\ledrightnote{\textcolor{green}{Die Hoffnung auf Segen. Eine Fischertragödie in vier Acten}}«}{\lemma{\textnormal{\emph{»Die
                  Hoffnung«}}}\Cendnote{\textnormal{niederl. \begin{otherlanguage}{dutch}\emph{\textcolor{green}{Op hoop van zegen. Spel van de zee in vier
                        bedrijven}}\end{otherlanguage}, UA am 24. 12. 1900 in \textcolor{pink}{Amsterdam}}}}\label{K_L03088-24h} von \textsc{\textcolor{blue}{Heyermans}{}\ledrightnote{\textcolor{blue}{Herman Heijermans}}}, der \textcolor{blue}{Verfaſſer}{}\ledrightnote{{$\rightarrow$}\textcolor{blue}{Herman Heijermans}} ein
               Jude, – \label{K_L03088-11v}\edtext{reichen \textcolor{blue}{Rheders}{}\ledrightnote{{$\rightarrow$}\textcolor{blue}{Herman (Sr.) Heijermans}}{ }\textcolor{blue}{Sohn}{}\ledrightnote{{$\rightarrow$}\textcolor{blue}{Herman Heijermans}}}{\lemma{\textnormal{\emph{reichen Rheders Sohn}}}\Cendnote{\textnormal{Das dürfte auf einer Verwechslung
                  beruhen, der Vater \textcolor{blue}{Herman Heijermans
                     (senior)} war Redakteur.}}}\label{K_L03088-11h}{ }\textcolor{gray}{–} die \textcolor{pink}{Berlin}{}\ledrightnote{\textcolor{pink}{Berlin}}er Kritik hat das
                  \textcolor{green}{Stück}{}\ledrightnote{\textcolor{green}{Die Hoffnung auf Segen. Eine Fischertragödie in vier Acten}} verriſſen, – Allen voran \label{K_L03088-21v}\edtext{\textsc{\textcolor{green}{\textcolor{blue}{Kerr}{}\ledrightnote{\textcolor{blue}{Alfred Kerr}}}{}\ledrightnote{{$\rightarrow$}\textcolor{green}{[Kritik zu Die Hoffnung]}}}}{\lemma{\textnormal{\emph{Kerr}}}\Cendnote{\textnormal{XXXX}}}\label{K_L03088-21h}, der doch zu Zeiten enervirend verſtändniſlos iſt.\pend
           
\pstart
           Was \label{K_L03088-26v}\edtext{\textsc{\textcolor{blue}{Glümers}{}\ledrightnote{\textcolor{blue}{Marie Glümer}{\newline}\textcolor{blue}{Auguste Glümer}}}}{\lemma{\textnormal{\emph{Glümers}}}\Cendnote{\textnormal{siehe Paul Goldmann an Arthur Schnitzler, 28. 9. [1901]}}}\label{K_L03088-26h} anlangt, ſo bin ich nicht beleidigt, ſondern erbittert. \strikeout{I\textcolor{gray}{hre}}{ }{\pb}Ich verzeihe Alles, nur keine Ungezogenheiten.
               Gratulirt habe ich nicht, und ich werde auch nicht gratuliren.\pend
           
\pstart
           Die \textsc{\textcolor{blue}{Triesch}{}\ledrightnote{\textcolor{blue}{Irene Triesch}}} iſt unglücklich, wird \label{K_L03088-12v}\edtext{falſch
                  beſchäftigt}{\lemma{\textnormal{\emph{falſch
                  beſchäftigt}}}\Cendnote{\textnormal{\textcolor{blue}{Irene Triesch} hatte ihren letzten Auftritt
                  am \emph{\textcolor{brown}{Frankfurter Stadttheater}} am 24. 8. 1901. Danach ging sie an das \emph{\textcolor{brown}{Deutsche Theater Berlin}}. Dort trat sie Anfang Oktober 1901 in \textcolor{blue}{Gerhart
                     Hauptmann}s \emph{\textcolor{green}{Einsame Menschen}} als \textcolor{green}{Anna Mahr} auf.}}}\label{K_L03088-12h} und
               ſehnt ſich nach Deinen \textcolor{green}{Stücken}{}\ledrightnote{{$\rightarrow$}\textcolor{green}{Lebendige Stunden. Vier Einakter}}. Iſt mir im Übrigen ſehr zuwider, weil ſie gerade die \label{K_L03088-13v}\edtext{zwei Typen}{\lemma{\textnormal{\emph{zwei Typen}}}\Cendnote{\textnormal{siehe Paul Goldmann an Arthur Schnitzler, 23. 9. [1901]}}}\label{K_L03088-13h} repräſentirt, die ich nicht vertragen kann: den der Jüdin und den der
               Komödiantin.\pend
           
\pstart
           Sage dem \textsc{\textcolor{blue}{Richard}{}\ledrightnote{\textcolor{blue}{Richard Beer-Hofmann}}}, daß die Frau Profeſſor \textsc{\textcolor{blue}{Döpler}{}\ledrightnote{\textcolor{blue}{Berta Doepler}}} ſich mit \strikeout{Moph} Morphium \label{K_L03088-14v}\edtext{vergiftet}{\lemma{\textnormal{\emph{vergiftet}}}\Cendnote{\textnormal{\textcolor{blue}{Berta Doepler}, eine Cousine von \textcolor{blue}{Else
                  Lasker-Schüler}, verstarb wenig später, am 10. 2. 1902. Zu \textcolor{blue}{Beer-Hofmann}s
                  Bekanntschaft mit ihr vgl. Richard Beer-Hofmann an Arthur Schnitzler, 22. 2. 1900.}}}\label{K_L03088-14h} hat, um den unerträglichen Schmerzen zu entgehen, die ihre unheilbare
               Krankheit ihr bereitet hat.\pend
           
\pstart
           Wollen wir dem \label{K_L03088-16v}\edtext{\textsc{\textcolor{blue}{Peter Dorner}{}\ledrightnote{\textcolor{blue}{Peter Dorner}}}}{\lemma{\textnormal{\emph{Peter Dorner}}}\Cendnote{\textnormal{siehe Paul Goldmann an Arthur Schnitzler, 23. 9. [1901]}}}\label{K_L03088-16h} nicht zuſammen das \textcolor{green}{Werk}{}\ledrightnote{{$\rightarrow$}\textcolor{green}{Die Schmiedekunst seit dem Ende der Renaissance}} über die »\textcolor{green}{Deutſche Schmiedekunſt}{}\ledrightnote{{$\rightarrow$}\textcolor{green}{Die Schmiedekunst seit dem Ende der Renaissance}}« ſchenken? Du 22 \textsc{MK} und ich 22 \textsc{MK}.\pend
           
\pstart
           {\pb}Lies’ in der letzten »\textcolor{green}{Zukunft}{}\ledrightnote{\textcolor{green}{Die Zukunft}}« den geiſtvollen Aufſatz \label{K_L03088-17v}\edtext{»\textcolor{green}{Phyſiologie des
                  Kunſtempfindens}{}\ledrightnote{\textcolor{green}{Physiologie des Kunstempfindens. Der Grundsatz}}«}{\lemma{\textnormal{\emph{»Phyſiologie des Kunſtempfindens«}}}\Cendnote{\textnormal{[O. V.]: \emph{\textcolor{green}{Physiologie des Kunstempfindens.
                        Der Grundsatz}}. In: \emph{\textcolor{green}{Die Zukunft}},
                     hg. v. \textcolor{blue}{Maximilian Harden}, Bd. 37. \textcolor{pink}{Berlin}: \emph{\textcolor{brown}{Verlag der Zukunft}}{ }1901, S. 34–48.}}}\label{K_L03088-17h}.\pend
           
\pstart
           Viele herzliche Grüße an die {\\[\baselineskip]}Mädels und an Dich. {\\[\baselineskip]}Dein {\\[\baselineskip]}\spacefill\mbox{Paul Goldmann.}\pend
           \leftskip=0em{}\endnumbering\briefempfaengerindex{Schnitzler, Arthur@\textsc{Schnitzler, Arthur}!zzzGoldmann, Paul@\emph{von Paul Goldmann}!1901-10-072@{7. 10. {[}1901{]}}|)be}\mylabel{h}
\begin{anhang}
\end{anhang}\normalsize

\doendnotes{C}
\bigskip
\vfill

\clearpage

\footnotesize

\lohead{\textsc{register}}

% Definiere theindex-Environment komplett neu ohne reledmac
\makeatletter
\renewenvironment{theindex}{%
  \section*{\indexname}%
  \setlength{\parindent}{0pt}%
  \setlength{\parskip}{0pt plus 0.3pt}%
  \let\item\@idxitem
}{%
  \clearpage
}
\makeatother

\IfFileExists{\jobname-pw.ind}{\input{\jobname-pw.ind}}{}

\end{document}

      