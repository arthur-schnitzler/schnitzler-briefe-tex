%% latex-korrekturansicht-vorspann.tex
%% Vorspann für die Korrekturansicht.
%% Lädt die gemeinsame Datei latex-vorspann.tex mit gesetztem Schalter.

\newif\ifkorrekturansicht
\korrekturansichttrue

\input{../tex-inputs/latex-vorspann}


\section[Franz Goldstein und Thomas Mann an Arthur Schnitzler, 19. 8. 1931]{L03881 Franz Goldstein und Thomas Mann an Arthur Schnitzler, 19. 8. 1931}
\nopagebreak\mylabel{L03881v}
\rehead{ }\normalsize\beginnumbering\briefempfaengerindex{, @\textsc{, }!zzz, @\emph{von  }!1931-08-191@{19. 8. 1931}|(be}\briefempfaengerindex{, @\textsc{, }!zzz, @\emph{von  }!1931-08-191@{19. 8. 1931}|(be}
\toendnotes[C]{\smallbreak\pagebreak[2]}\Standort{DLA, A:Schnitzler, 1985.1.3183,7.}
\physDesc{Bildpostkarte, 364 Zeichen
\newline{}Handschrift  : blaue Tinte, deutsche Kurrent
\newline{}Handschrift  : blaue Tinte, deutsche Kurrent
\newline{}Versand: 1) Stempel: »\nobreak{}\oindex{Nida@\textbf{Nida}|pwk}Nidden, 2\textcolor{gray}{0. V}{[}III. 31{]}\nobreak{}«.   2) Stempel: »\nobreak{}\oindex{XVIII., Währing@\textbf{XVIII., Währing}, \emph{Verwaltungsgebiet}|pwk}18 Wien 110, 22. VIII. 31, 11\nobreak{}«.  3) mit schwarzer Tinte Streichung der beiden Adresszeilen mit der \textcolor{pink}{Wiener}\oindex{Wien@\textbf{Wien}, \emph{Verwaltungsgebiet}|pw} Adresse und Ersatz durch:
                                          »\noindent{}\textsc{\textcolor{pink}{Gmunden}\oindex{Gmunden@\textbf{Gmunden}|pw}}{ / }\textsc{\textcolor{pink}{Hotel
                                                \uline{Austria}}\oindex{Hotel Austria [Gmunden]@\textbf{Hotel Austria [Gmunden]}, \emph{Hotel}|pw}}« Schnitzler dürfte einen
                                 Nachsendeauftrag durch die Post verfügt haben. }\toendnotes[C]{\smallbreak}\pstart{}{\pb}\textsc{Herrn Dr. Arthur Schnitzler}\pend{}\pstart{}\textsc{\textcolor{pink}{Sternwartestr. 71}\oindex{Wien@\textbf{Wien}!XVIII., Währing@\textbf{XVIII., Währing}!Sternwartestraße 71@\textbf{Sternwartestraße 71}, \emph{Wohngebäude}|pw}{}\ledrightnote{\textcolor{pink}{Sternwartestraße 71}}}\pend{}\pstart{}\textsc{\textcolor{pink}{Österreich}\oindex{Österreich@\textbf{Österreich}|pw}{}\ledrightnote{\textcolor{pink}{Österreich}}}\pend{}{\bigskip}
\pstart
           {\pb}\textcolor{gray}{\textbf{\textcolor{pink}{Kuhrische Nerhrung}\oindex{Kurische Nehrung@\textbf{Kurische Nehrung}, \emph{Naturdenkmal}|pw}{}\ledrightnote{\textcolor{pink}{Kurische Nehrung}}}}\hfill \textcolor{gray}{\textbf{\textcolor{pink}{Hohe Düne bei Nidden}\oindex{Parnidis Düne@\textbf{Parnidis Düne}, \emph{Naturdenkmal}|pw}{}\ledrightnote{\textcolor{pink}{Parnidis Düne}}}}\pend
           \vspace{1em}
\pstart
           {\pb}\textsc{\textcolor{pink}{Nidden, Hôtel Herm. Blode}\oindex{Hotel Hermann Blode@\textbf{Hotel Hermann Blode}, \emph{Hotel}|pw}{}\ledrightnote{\textcolor{pink}{Hotel Hermann Blode}} (\textcolor{pink}{Litauen}\oindex{Litauen@\textbf{Litauen}|pw}{}\ledrightnote{\textcolor{pink}{Litauen}}) 19. VIII.}\pend
           
\pstart{}Sehr verehrter Herr Dr.!\pend\vspace{0.5em}
\pstart
           Aus dem zauberhaften \textcolor{pink}{\textsc{Nidden}}\oindex{Nida@\textbf{Nida}|pw}{}\ledrightnote{\textcolor{pink}{Nida}}, wohin ich von
               \textcolor{blue}{Thomas Manns}\pwindex{Mann, Katia 24.\,7.\,1883 Feldafing – 25.\,4.\,1980 Kilchberg@\textsc{Mann, Katia} (24.\,7.\,1883 Feldafing – 25.\,4.\,1980 Kilchberg)|pwv}{}\ledrightnote{{$\rightarrow$}\emph{\textcolor{blue}{Katia Mann}}} eingeladen bin – wir gedenken Ihrer wiederholt
            herzlichſt – geſtatte ich mir Ihnen ſchönſte Grüße zu ſenden. Stets Ihr ganz ergebner\pend
           \pstart \spacefill\mbox{Franz Goldstein}\pend{}\selectlanguage{ngerman}\vspace{1em}
\pstart
           \noindent{}{[}hs. :{]} Herzlich verehrungsvollen Gruß!\pend
           \pstart \spacefill\mbox{Thomas Mann.}\pend{}\selectlanguage{ngerman}\endnumbering\briefempfaengerindex{, @\textsc{, }!zzz, @\emph{von  }!1931-08-191@{19. 8. 1931}|)be}\briefempfaengerindex{, @\textsc{, }!zzz, @\emph{von  }!1931-08-191@{19. 8. 1931}|)be}\mylabel{L03881h}
\begin{anhang}
\end{anhang}\normalsize

\doendnotes{C}
\bigskip
\vfill

\clearpage

\footnotesize

\lohead{\textsc{register}}

% Definiere theindex-Environment komplett neu ohne reledmac
\makeatletter
\renewenvironment{theindex}{%
  \section*{\indexname}%
  \setlength{\parindent}{0pt}%
  \setlength{\parskip}{0pt plus 0.3pt}%
  \let\item\@idxitem
}{%
  \clearpage
}
\makeatother

\IfFileExists{\jobname-pw.ind}{\input{\jobname-pw.ind}}{}

\end{document}

      