%% latex-korrekturansicht-vorspann.tex
%% Vorspann für die Korrekturansicht.
%% Lädt die gemeinsame Datei latex-vorspann.tex mit gesetztem Schalter.

\newif\ifkorrekturansicht
\korrekturansichttrue

\input{../tex-inputs/latex-vorspann}


\renewcommand{\erwaehntePersonen}{Personen: Wilhelmine Adamović, Samuel Fischer, André Giron, Hugo von Hofmannsthal,  Luise von Sachsen,  Luise von Sachsen, Leopold Ferdinand Salvator Wölfling}
\renewcommand{\erwaehnteOrte}{Orte: Bern, Europa, Genf, Hotel du Cygne Montreux, Montreux, Schweiz, Wien}
\renewcommand{\erwaehnteWerke}{Werke: Das gerettete Venedig. Trauerspiel in fünf Aufzügen, Erinnerungen, Tagebuch}
\section[ Felix Salten an Arthur Schnitzler, 28. 12. 1902]{Felix Salten an Arthur Schnitzler, 28. 12. 1902}
\nopagebreak\mylabel{v}
\rehead{ }\normalsize\beginnumbering\briefempfaengerindex{Schnitzler, Arthur@\textsc{Schnitzler, Arthur}!zzzSalten, Felix@\emph{von Felix Salten}!1902-12-282@{28. 12. 1902}|(be}
\toendnotes[C]{\smallbreak\pagebreak[2]}\Standort{CUL, Schnitzler, B 89, A 2.}
\physDesc{Brief, 1 Blatt, 4 Seiten, 902 Zeichen (Schwan im Prägedruck)
\newline{}Handschrift: Bleistift, lateinische Kurrent
\newline{}Ordnung: mit Bleistift von unbekannter Hand nummeriert: »163« }\toendnotes[C]{\smallbreak}
\pstart
           \noindent{}\raggedleft{}{\pb}\textcolor{gray}{\textbf{\textcolor{pink}{HOTEL DU CYGNE}{}\ledrightnote{\textcolor{pink}{Hotel du Cygne Montreux}}}}\pend
           
\pstart
           \noindent{}\raggedleft{}\textcolor{gray}{\textbf{\textcolor{pink}{MONTREUX}{}\ledrightnote{\textcolor{pink}{Montreux}}}}\pend
           
\pstart
           \raggedleft{}28. XII. 02\pend
           
\pstart
           Liebster leider konnte ich Sie vor meiner Abreise nicht mehr
               sprechen. Nun habe ich Ihnen inzwischen noch mehr zu sagen als früher.\pend
           
\pstart
           Bin hier bei \label{K_L03338-1v}\edtext{\textcolor{blue}{Erzh. Leopold}{}\ledrightnote{\textcolor{blue}{Leopold Ferdinand Salvator Wölfling}}}{\lemma{\textnormal{\emph{Erzh. Leopold}}}\Cendnote{\textnormal{\textcolor{blue}{Leopold Ferdinand von Österreich-Toskana} war mit seiner
                  Haushälterin und zukünftigen Ehefrau, der ehemaligen Sexarbeiterin \textcolor{blue}{Wilhelmine Adamović}, in die \textcolor{pink}{Schweiz} geflohen und aus dem Kaiserhaus
                  ausgetreten, um fortan als einfacher Bürger unter dem Namen \textcolor{blue}{Leopold Wölfling} leben zu können. 
                  Zugleich hatte seine Schwester, die verheiratete \textcolor{blue}{Luise von Sachsen} ihre
                  Ehe verlassen, um mit dem Sprachlehrer \textcolor{blue}{André Giron} nach
                  \textcolor{pink}{Genf} zu fliehen.
                  In seinen \emph{\textcolor{green}{Erinnerungen}} schrieb \textcolor{blue}{Salten}:
                  »Zu Weihnachten 1902 ging durch all \textcolor{pink}{europäischen} Zeitungen die
                     Sensationsnachricht, Erzherzog \textcolor{blue}{Leopold Ferdinand} und seine Schwester
                     Kronprinzessin \textcolor{blue}{Luisa von Sachsen} seien nach \textcolor{pink}{Genf} durchgebrannt.
                     \textcolor{blue}{Leopold} telegraphierte mir, wenn ich Abschied nehmen wolle, wäre
                     ich ihm willkommen. Ich fuhr sofort nach \textcolor{pink}{Genf}. In \textcolor{pink}{Bern} während der
                        Zug hielt, wurde mir ein Telegramm gereicht, \textcolor{blue}{Leopold} bat mich nach
                        \textcolor{pink}{Montreux} zu kommen. Mit meiner Redaktion vereinbarte ich einen
                     Code und in \textcolor{pink}{Montreux} erwartete mich \textcolor{blue}{Leopold} auf dem Bahnhof.
                     »Was sagen sie zu dem Wirbel, den wir gemacht haben?« Die Zeitung-
                     en hatten gemeldet, \textcolor{blue}{Leopold} sei mit seiner Geliebten \textcolor{blue}{Adamowitsch}
                     durchgebrannt, die \textcolor{blue}{Kronprinzessin}, \textcolor{blue}{Leopold}s Schwester mit ihrem
                        Schatz \textcolor{blue}{Giron}.« (\emph{Wienbibliothek im Rathaus}, Nachlass
                           \textcolor{blue}{Salten}, ZPH 1681/1 1.1.1.9.1, [S. 11]
                        ) 
                  Vgl. Paul Goldmann an Arthur Schnitzler, 22. 3. [1900].
               }}}\label{K_L03338-1h} und fahre jetzt nach \textcolor{pink}{Genf}{}\ledrightnote{\textcolor{pink}{Genf}} um den
                  \label{K_L03338-2v}\edtext{Nachmittag mit seiner \textcolor{blue}{Schwester}{}\ledrightnote{{$\rightarrow$}\textcolor{blue}{Luise von Sachsen}}}{\lemma{\textnormal{\emph{Nachmittag … Schwester}}}\Cendnote{\textnormal{Diesen Besuch schildert \textcolor{blue}{Salten}
                         in seinen \emph{\textcolor{green}{Erinnerungen}}.}}}\label{K_L03338-2h} zu verbringen\pend
           
\pstart
           Reise \textcolor{gray}{morgen} nach \textcolor{pink}{Wien}{}\ledrightnote{\textcolor{pink}{Wien}} zurück, wo ich \substVorne{}\textsuperscript{Montag}{\allowbreak}\substDazwischen{}Dienstag\substHinten{}{ }{\pb}früh eintreffe. Vielleicht rufen Sie mich \textcolor{gray}{V.}
                  Mittg an, oder ich \label{K_L03338-3v}\edtext{\textcolor{gray}{komme} so zwischen 4 {\kaufmannsund} 5
               zu Ihnen}{\lemma{\textnormal{\emph{komme … Ihnen}}}\Cendnote{\textnormal{Zu einem Treffen kam es erst am
                     2. 1. 1903.}}}\label{K_L03338-3h}, da es ja aus dem Cafébesuch von mir nichts wird. »Das Leben ist eine
               Rutschbahn« könnte der \textcolor{blue}{Leop.}{}\ledrightnote{\textcolor{blue}{Leopold Ferdinand Salvator Wölfling}} jetzt auch sagen.
               Er \label{K_L03338-4v}\edtext{thut mir furchtbar leid}{\lemma{\textnormal{\emph{thut mir furchtbar leid}}}\Cendnote{\textnormal{}}}\label{K_L03338-4h}. Hier ist’s übrigens bald
                  Frühling\textcolor{gray}{.}\pend
           
\pstart
           Herzlichst Ihr {\\[\baselineskip]}\spacefill\mbox{Salten}\pend
           \leftskip=0em{}
\pstart
           \noindent{}\label{T_L03338-1v}\edtext{Wenn \label{K_L03338-5v}\edtext{\textcolor{blue}{Hofmannsthal}{}\ledrightnote{\textcolor{blue}{Hugo von Hofmannsthal}} noch nicht \textcolor{green}{gelesen}{}\ledrightnote{{$\rightarrow$}\textcolor{green}{Das gerettete Venedig. Trauerspiel in fünf Aufzügen}}}{\lemma{\textnormal{\emph{Hofmannsthal … gelesen}}}\Cendnote{\textnormal{siehe A. S.: \emph{Tagebuch}, 6. 1. 1903}}}\label{K_L03338-5h}{ }hat, bitte ich ihn auf mich zu warten.
                     \textcolor{gray}{S}chreibe ihm das aber.}{\lemma{\textnormal{\emph{Wenn … aber.}}}\Cendnote{\textnormal{am oberen Seitenrand, quer zum Text über die ersten beiden
                     Seiten}}}\label{T_L03338-1h}\pend
           
\pstart
           {\pb}Sollte \label{K_L03338-6v}\edtext{\textcolor{blue}{S.
                     Fischer}{}\ledrightnote{\textcolor{blue}{Samuel Fischer}} in \textcolor{pink}{Wien}{}\ledrightnote{\textcolor{pink}{Wien}}}{\lemma{\textnormal{\emph{S.
                     Fischer in Wien}}}\Cendnote{\textnormal{Zumindest im \emph{\textcolor{green}{Tagebuch}}{ }\textcolor{blue}{Schnitzler}s ist in diesen Tagen keine
                     Anwesenheit \textcolor{blue}{Fischer}s in \textcolor{pink}{Wien} erwähnt.}}}\label{K_L03338-6h} sein, bitte ihm meine Abwesenheit
                  entschuldigen.\pend
           
\pstart
           habe ihn eingeladen und mußte abreisen. Mittheilen konnte ich ihm nichts davon,
                  weil ich ihn auf dem Weg nach \textcolor{pink}{Wien}{}\ledrightnote{\textcolor{pink}{Wien}} glaubte und
                  eine {\pb}\textcolor{pink}{Wien}{}\ledrightnote{\textcolor{pink}{Wien}}er Adreße von ihm nicht hatte. {\\}\spacefill\mbox{F. S}\pend
           \endnumbering\briefempfaengerindex{Schnitzler, Arthur@\textsc{Schnitzler, Arthur}!zzzSalten, Felix@\emph{von Felix Salten}!1902-12-282@{28. 12. 1902}|)be}\mylabel{h}  \normalsize

\doendnotes{C}
\bigskip
\vfill

\clearpage

\footnotesize

\lohead{\textsc{register}}

% Definiere theindex-Environment komplett neu ohne reledmac
\makeatletter
\renewenvironment{theindex}{%
  \section*{\indexname}%
  \setlength{\parindent}{0pt}%
  \setlength{\parskip}{0pt plus 0.3pt}%
  \let\item\@idxitem
}{%
  \clearpage
}
\makeatother

\IfFileExists{\jobname-pw.ind}{\input{\jobname-pw.ind}}{}

\end{document}

      