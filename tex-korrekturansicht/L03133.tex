%% latex-korrekturansicht-vorspann.tex
%% Vorspann für die Korrekturansicht.
%% Lädt die gemeinsame Datei latex-vorspann.tex mit gesetztem Schalter.

\newif\ifkorrekturansicht
\korrekturansichttrue

\input{../tex-inputs/latex-vorspann}


\renewcommand{\erwaehntePersonen}{Personen: Richard Specht}
\renewcommand{\erwaehnteInstitutionen}{Institutionen: Kaiser Franz Josephs-Bahn}
\renewcommand{\erwaehnteOrte}{Orte: Dornbach, Tulln an der Donau, Wien}
\renewcommand{\erwaehnteWerke}{}
\section[Felix Salten an Arthur Schnitzler, {[}7.? 5. 1894{]}]{Felix Salten an Arthur Schnitzler, {[}7.? 5. 1894{]}}
\nopagebreak\mylabel{v}
\rehead{ }\normalsize\beginnumbering\briefempfaengerindex{Schnitzler, Arthur@\textsc{Schnitzler, Arthur}!zzzSalten, Felix@\emph{von Felix Salten}!1894-05-072@{{[}7.? 5. 1894{]}}|(be}
\toendnotes[C]{\smallbreak\pagebreak[2]}\Standort{CUL, Schnitzler, B 89, A 1.}
\physDesc{Brief, 1 Blatt, 1 Seite, 367 Zeichen
\newline{}Handschrift: Bleistift, lateinische Kurrent
\newline{}Schnitzler: mit Bleistift datiert: »Mai 94« 
\newline{}Ordnung: mit Bleistift von unbekannter Hand nummeriert: »35« }\toendnotes[C]{\smallbreak}
\pstart
           \noindent{}{\pb}Lieber Frd, ich bekomme \label{K_L03133-1v}\edtext{\uline{keine} N\textsuperscript{o}}{\lemma{\textnormal{\emph{keine N\textsuperscript{o}}}}\Cendnote{\textnormal{In \textcolor{pink}{Wien} war das Fahrradfahren auf der Straße nur nach Absolvierung einer
                  Fahrprüfung erlaubt, die durch eine Nummer bestätigt wurde, welche wiederum sichtbar am Rad
                  montiert sein musste. Da \textcolor{blue}{Salten} diese nicht
                  hatte, musste er, wie er weiter unten projektiert, sein Rad an die \textcolor{pink}{Stadt}grenze transportieren
                  lassen und Ausflüge außerhalb machen.}}}\label{K_L03133-1h}, \textcolor{blue}{Specht}{}\ledrightnote{\textcolor{blue}{Richard Specht}} will nicht, u. zureden kann ich auch nicht, ich \strikeout{werde} denke, es ist vielleicht das beste, wenn wir die
               Tour abändern, u. mit der \label{K_L03133-2v}\edtext{\textcolor{brown}{Franzjosefsbahn}{}\ledrightnote{\textcolor{brown}{Kaiser Franz Josephs-Bahn}} fahren}{\lemma{\textnormal{\emph{Franzjosefsbahn fahren}}}\Cendnote{\textnormal{Von den gemeinsamen Ausflügen, die \textcolor{blue}{Salten} und \textcolor{blue}{Schnitzler}
                  im Mai 1894 unternahmen, deuten die Angabe des \textcolor{pink}{Startort}es und der benutzten
                     \textcolor{brown}{Bahnlinie} auf den
                  Ausflug nach \textcolor{pink}{Tulln} am 7. 5. 1894 hin. Da
                  das Korrespondenzstück keine zeitliche Verortung zum Ausflug enthält, könnte es
                  auch in den Tagen vor der Tour verfasst worden sein.}}}\label{K_L03133-2h}, oder, sonst irgend
               wie. Ich frage jedenfalls auch einen Einspänner, was es kostet, wenn er mich bis \textcolor{pink}{Dornbach}{}\ledrightnote{\textcolor{pink}{Dornbach}} führt.\pend
           
\pstart
           Bitte, theilen Sie mir jetzt gleich mit, was geschehen soll.\pend
           
\pstart
           Ihr {\\[\baselineskip]}\spacefill\mbox{Salten}\pend
           \leftskip=0em{}\endnumbering\briefempfaengerindex{Schnitzler, Arthur@\textsc{Schnitzler, Arthur}!zzzSalten, Felix@\emph{von Felix Salten}!1894-05-072@{{[}7.? 5. 1894{]}}|)be}\mylabel{h}  \normalsize

\doendnotes{C}
\bigskip
\vfill

\clearpage

\footnotesize

\lohead{\textsc{register}}

% Definiere theindex-Environment komplett neu ohne reledmac
\makeatletter
\renewenvironment{theindex}{%
  \section*{\indexname}%
  \setlength{\parindent}{0pt}%
  \setlength{\parskip}{0pt plus 0.3pt}%
  \let\item\@idxitem
}{%
  \clearpage
}
\makeatother

\IfFileExists{\jobname-pw.ind}{\input{\jobname-pw.ind}}{}

\end{document}

      