%% latex-korrekturansicht-vorspann.tex
%% Vorspann für die Korrekturansicht.
%% Lädt die gemeinsame Datei latex-vorspann.tex mit gesetztem Schalter.

\newif\ifkorrekturansicht
\korrekturansichttrue

\input{../tex-inputs/latex-vorspann}


         
         \renewcommand{\erwaehntePersonen}{Personen: Richard Beer-Hofmann, Erich Freund, Eduard Grisebach, Ernst Theodor Amadeus Hoffmann, Alfred Kerr, Olga Schnitzler, Elisabeth Steinrück, Jakob Wassermann, Julie Wassermann}
         \renewcommand{\erwaehnteInstitutionen}{Institutionen: Max Hesses Verlag, Neue Freie Presse}
         \renewcommand{\erwaehnteOrte}{Orte: Berlin, Breslau, Dessauer Straße, Leipzig, Reichstag, Rotensterngasse, Wien}
         \renewcommand{\erwaehnteWerke}{Werke: Der Schleier der Beatrice. Schauspiel in fünf Akten, E. T. A. Hoffmanns sämtliche Werke in fünfzehn Bänden, Lieutenant Gustl. Novelle, Neue Freie Presse}
               \section[ Paul Goldmann an Arthur Schnitzler, 30. 10. {[}1900{]}]{Paul Goldmann an Arthur Schnitzler, 30. 10. {[}1900{]}}\nopagebreak\mylabel{v}\rehead{ }\normalsize\beginnumbering\briefempfaengerindex{Schnitzler, Arthur@\textsc{Schnitzler, Arthur}!zzzGoldmann, Paul@\emph{von Paul Goldmann}!1900-10-301@{30. 10. {[}1900{]}}|(be} \toendnotes[C]{\smallbreak\pagebreak[2]} \Standort{DLA, A:Schnitzler, HS.NZ85.1.3170.}
\physDesc{Brief, 1 Blatt, 4 Seiten
\newline{}Handschrift: blaue Tinte, deutsche Kurrent
\newline{}Schnitzler: 1) mit Bleistift das Jahr »{[}1{]}900« vermerkt  2) mit rotem Buntstift vier Unterstreichungen}\toendnotes[C]{\smallbreak}\pstart
           \noindent{}{\pb}\textcolor{pink}{Berlin}{}\ledrightnote{\textcolor{pink}{Berlin}}, 30. Oktober.\hfill \textcolor{pink}{\textcolor{gray}{\textbf{DESSAUERSTRASSE 19}}}{}\ledrightnote{\textcolor{pink}{Dessauer Straße}}\pend
           \pstart\center{}Mein lieber Freund,\pend\pstart
           Als »Menſch« werde ich leider auch nicht nach \textcolor{pink}{Breslau}{}\ledrightnote{\textcolor{pink}{Breslau}} kommen. Die \label{K_L02937-1v}\edtext{\textcolor{green}{Aufführung}{}\ledrightnote{{$\rightarrow$}\textcolor{green}{Der Schleier der Beatrice. Schauspiel in fünf Akten}}}{\lemma{\textnormal{\emph{Aufführung}}}\Cendnote{\textnormal{siehe Paul Goldmann an Arthur Schnitzler, 30. 10. [1900]}}}\label{K_L02937-1h} iſt am 17., und am 14. wird \textcolor{pink}{hier}{}\ledrightnote{{$\rightarrow$}\textcolor{pink}{Berlin}} der
                  \textcolor{pink}{Reichstag}{}\ledrightnote{\textcolor{pink}{Reichstag}} eröffnet. Da darf ich mich nicht
               wegrühren. Aber ich rechne beſtimmt darauf, daß Du von \textcolor{pink}{Breslau}{}\ledrightnote{\textcolor{pink}{Breslau}} nach \textcolor{pink}{Berlin}{}\ledrightnote{\textcolor{pink}{Berlin}} kommſt, damit ich
               wenigſtens die Freude habe, Dich zu ſehen. Auch habe ich die Abſicht, der \textcolor{brown}{N. Fr. Pr.}{}\ledrightnote{\textcolor{brown}{Neue Freie Presse}} den \textsc{Dr. \textcolor{blue}{Erich Freund}{}\ledrightnote{\textcolor{blue}{Erich Freund}}} in \textcolor{pink}{Breslau}{}\ledrightnote{\textcolor{pink}{Breslau}}, den Du ja auch kennſt, {\pb}als Referenten vorzuſchlagen, damit wenigſtens ein
               anſtändiger und ehrlicher \textcolor{blue}{Kritiker}{}\ledrightnote{{$\rightarrow$}\textcolor{blue}{Erich Freund}} über Dich \label{K_L02937-2v}\edtext{berichtet}{\lemma{\textnormal{\emph{berichtet}}}\Cendnote{\textnormal{siehe Paul Goldmann an Arthur Schnitzler, 28. 2. [1898] und 3. 12. [1900]}}}\label{K_L02937-2h}.\pend
           \pstart
           Wann gedenkſt Du \label{K_L02937-3v}\edtext{nach \textcolor{pink}{Breslau}{}\ledrightnote{\textcolor{pink}{Breslau}}}{\lemma{\textnormal{\emph{nach Breslau}}}\Cendnote{\textnormal{\textcolor{blue}{Schnitzler} hielt sich von 22. 11. 1900 bis 24. 11. 1900 und von
                     29. 11. 1900 bis
                     2. 12. 1900 in
                     \textcolor{pink}{Breslau} auf. Dazwischen war er in \textcolor{pink}{Berlin}.}}}\label{K_L02937-3h} zu reiſen?\pend
           \pstart
           Iſt es \strikeout{\textcolor{gray}{×}} wahr, daß \textsc{\textcolor{blue}{Wassermann}{}\ledrightnote{\textcolor{blue}{Jakob Wassermann}}} ſich mit einem Frl. \textcolor{blue}{\textsc{Speier}}{}\ledrightnote{\textcolor{blue}{Julie Wassermann}}{ }\label{K_L02937-4v}\edtext{verlobt}{\lemma{\textnormal{\emph{verlobt}}}\Cendnote{\textnormal{siehe A. S.: \emph{Tagebuch}, 11. 10. 1900}}}\label{K_L02937-4h} hat? Schön und reich?\pend
           \pstart
           Welches iſt die Adreſſe der \label{K_L02937-5v}\edtext{\textcolor{blue}{Fräulein}{}\ledrightnote{{$\rightarrow$}\textcolor{blue}{Olga Schnitzler}{\newline}{$\rightarrow$}\textcolor{blue}{Elisabeth Steinrück}} aus der \textcolor{pink}{Rothen-Stern-Gaſſe}{}\ledrightnote{\textcolor{pink}{Rotensterngasse}}}{\lemma{\textnormal{\emph{Fräulein … Rothen-Stern-Gaſſe}}}\Cendnote{\textnormal{siehe Paul Goldmann an Arthur Schnitzler, 19. 9. [1900]}}}\label{K_L02937-5h}?\pend
           \pstart
           {\pb}Wann erſcheint der \label{K_L02937-6v}\edtext{»\textcolor{green}{Lieutnant Guſtl}{}\ledrightnote{\textcolor{green}{Lieutenant Gustl. Novelle}}«}{\lemma{\textnormal{\emph{»Lieutnant Guſtl«}}}\Cendnote{\textnormal{\textcolor{blue}{Arthur Schnitzler}: \emph{\textcolor{green}{Lieutnant Gustl}}. In: \emph{\textcolor{green}{Neue Freie Presse}}, Nr. 13053, 25. 12. 1900, S. 34–41. Siehe auch A. S.: \emph{Tagebuch}, 25. 12. 1900.}}}\label{K_L02937-6h}?\pend
           \pstart
           Wie gehts Dir ſonſt? Frauen, Stimmung, Arbeit?\pend
           \pstart
           Mein Leben iſt troſtlos öde, ohne auch nur einen Schimmer von Freude. Aber ich leſe
                  \textsc{\textcolor{blue}{\textcolor{green}{E. T. A. Hoffmann}{}\ledrightnote{{$\rightarrow$}\textcolor{green}{E. T. A. Hoffmanns sämtliche Werke in fünfzehn Bänden}}}{}\ledrightnote{\textcolor{blue}{Ernst Theodor Amadeus Hoffmann}}}. Bitte, \label{K_L02937-7v}\edtext{thue das auch}{\lemma{\textnormal{\emph{thue das auch}}}\Cendnote{\textnormal{\emph{\textcolor{green}{E. T. A. Hoffmanns sämtliche Werke in fünfzehn
                        Bänden}}. Hg. v. \textcolor{blue}{Eduard
                     Grisebach}. \textcolor{pink}{Leipzig}: \emph{\textcolor{brown}{Max Hesse}}{ }1900. Eine Lektüre \textcolor{blue}{Schnitzler}s ist nicht
                  bekannt.}}}\label{K_L02937-7h}! (\textcolor{green}{Ausgabe}{}\ledrightnote{{$\rightarrow$}\textcolor{green}{E. T. A. Hoffmanns sämtliche Werke in fünfzehn Bänden}}
               von \textsc{\textcolor{blue}{Grisebach}{}\ledrightnote{\textcolor{blue}{Eduard Grisebach}}}).\pend
           \pstart
           \textsc{\textcolor{blue}{Richard}{}\ledrightnote{\textcolor{blue}{Richard Beer-Hofmann}}} benimmt ſich wieder einmal abſcheulich. {\pb}Antwortet mir nicht, \label{K_L02937-8v}\edtext{ſchickt mir
               nicht, worum ich ihn gebeten}{\lemma{\textnormal{\emph{ſchickt … gebeten}}}\Cendnote{\textnormal{siehe Paul Goldmann an Arthur Schnitzler, 19. 9. [1900]}}}\label{K_L02937-8h}! Rüttle ihn doch in meinem Namen etwas auf!\pend
           \pstart
           \textsc{\textcolor{blue}{Kerr}{}\ledrightnote{\textcolor{blue}{Alfred Kerr}}} ſehe ich einmal im Monat auf fünf Minuten, die er jedesmal dazu benutzt, um mir
               zu erzählen, wie herrlich das Leben iſt.\pend
           \pstart
           Grüß’ Dich Gott, liebſter {\\[\baselineskip]}Freund! In Treue {\\[\baselineskip]}Dein {\\[\baselineskip]}\spacefill\mbox{Paul Goldmann.}\pend
           \leftskip=0em{}\endnumbering\briefempfaengerindex{Schnitzler, Arthur@\textsc{Schnitzler, Arthur}!zzzGoldmann, Paul@\emph{von Paul Goldmann}!1900-10-301@{30. 10. {[}1900{]}}|)be}\mylabel{h}  \normalsize

\doendnotes{C}
\bigskip
\vfill

\clearpage

\footnotesize

\lohead{\textsc{register}}

% Definiere theindex-Environment komplett neu ohne reledmac
\makeatletter
\renewenvironment{theindex}{%
  \section*{\indexname}%
  \setlength{\parindent}{0pt}%
  \setlength{\parskip}{0pt plus 0.3pt}%
  \let\item\@idxitem
}{%
  \clearpage
}
\makeatother

\IfFileExists{\jobname-pw.ind}{\input{\jobname-pw.ind}}{}

\end{document}

      