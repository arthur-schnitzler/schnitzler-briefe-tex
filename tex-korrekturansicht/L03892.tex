%% latex-korrekturansicht-vorspann.tex
%% Vorspann für die Korrekturansicht.
%% Lädt die gemeinsame Datei latex-vorspann.tex mit gesetztem Schalter.

\newif\ifkorrekturansicht
\korrekturansichttrue

\input{../tex-inputs/latex-vorspann}


\section[Sigmund Freud: Widmungsexemplar von Die Zukunft einer Illusion, {[}zwischen 1. und 28. 11. 1927{]}]{L03892 Sigmund Freud: Widmungsexemplar von Die Zukunft einer Illusion, {[}zwischen 1. und 28. 11. 1927{]}}
\nopagebreak\mylabel{L03892v}
\rehead{ }\normalsize\beginnumbering\briefempfaengerindex{, @\textsc{, }!zzz, @\emph{von  }!1927-11-281@{{[}zwischen 1. und 28. 11. 1927{]}}|(be}
\toendnotes[C]{\smallbreak\pagebreak[2]}\buchAlsQuelle{Sigmund Freud: \emph{Briefe an Arthur Schnitzler.} Herausgegeben von Henry Schnitzler. In: \emph{Neue deutsche Rundschau}, Jg. 66 (Januar 1955) Nr. 1, S. 106.}\toendnotes[C]{\smallbreak}
\pstart
           \noindent{}\label{K_L03891-1v}\edtext{\textcolor{green}{Arthur Schnitzler als geringe Gegengabe, d. Verf. Nov. 1927.}\pwindex{Freud, Sigmund 6.\,5.\,1856 Pribor – 23.\,9.\,1939 London@\textsc{Freud, Sigmund} (6.\,5.\,1856 Pribor – 23.\,9.\,1939 London), \emph{Psychoanalytiker}!Zukunft einer Illusion@\strich\emph{Die Zukunft einer Illusion}|pwv}{}\ledrightnote{{$\rightarrow$}\emph{\textcolor{green}{Die Zukunft einer Illusion}}}}{\lemma{\textnormal{\emph{Arthur … 1927.}}}\Cendnote{\textnormal{Das Widmungsexemplar
                  von \emph{\textcolor{green}{Die Zukunft einer Illusion}\pwindex{Freud, Sigmund 6.\,5.\,1856 Pribor – 23.\,9.\,1939 London@\textsc{Freud, Sigmund} (6.\,5.\,1856 Pribor – 23.\,9.\,1939 London), \emph{Psychoanalytiker}!Zukunft einer Illusion@\strich\emph{Die Zukunft einer Illusion}|pwk}} (Leipzig, Wien, Zürich: \emph{\textcolor{brown}{Internationaler Psychoanalytischer Verlag}\orgindex{Internationaler Psychoanalytischer Verlag@Internationaler Psychoanalytischer Verlag|pwk}}{ }1927) ist nur durch eine Anmerkung 
                  in der Edition (1955) von \textcolor{blue}{Heinrich Schnitzler}\pwindex{Schnitzler, Heinrich 9.\,8.\,1902 Hinterbrühl – 12.\,7.\,1982 Wien@\textsc{Schnitzler, Heinrich} (9.\,8.\,1902 Hinterbrühl – 12.\,7.\,1982 Wien), \emph{Regisseur, Schauspieler}|pwk} belegt. Der
                  Verbleib ist ungeklärt. \textcolor{blue}{Schnitzler} vermerkte die Lektüre
                  für den Vorabend des 29. 11. 1927, 
                  womit eine zeitliche Eingrenzung nach hinten möglich ist.}}}\label{K_L03891-1}\pend
           \selectlanguage{ngerman}\endnumbering\briefempfaengerindex{, @\textsc{, }!zzz, @\emph{von  }!1927-11-011@{{[}zwischen 1. und 28. 11. 1927{]}}|)be}\mylabel{L03892h}
\begin{anhang}
\end{anhang}\normalsize

\doendnotes{C}
\bigskip
\vfill

\clearpage

\footnotesize

\lohead{\textsc{register}}

% Definiere theindex-Environment komplett neu ohne reledmac
\makeatletter
\renewenvironment{theindex}{%
  \section*{\indexname}%
  \setlength{\parindent}{0pt}%
  \setlength{\parskip}{0pt plus 0.3pt}%
  \let\item\@idxitem
}{%
  \clearpage
}
\makeatother

\IfFileExists{\jobname-pw.ind}{\input{\jobname-pw.ind}}{}

\end{document}

      