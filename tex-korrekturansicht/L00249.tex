%% latex-korrekturansicht-vorspann.tex
%% Vorspann für die Korrekturansicht.
%% Lädt die gemeinsame Datei latex-vorspann.tex mit gesetztem Schalter.

\newif\ifkorrekturansicht
\korrekturansichttrue

\input{../tex-inputs/latex-vorspann}


               \section[Arthur Schnitzler an Richard Beer-Hofmann, 3. 8. 1893]{ Arthur Schnitzler an Richard Beer-Hofmann, 3. 8. 1893}\nopagebreak\mylabel{v}\rehead{ }\normalsize\beginnumbering\briefempfaengerindex{Beer-Hofmann, Richard@\textsc{Beer-Hofmann, Richard}!zzzSchnitzler, Arthur@\emph{von Arthur Schnitzler}!1893-08-031@{3. 8. 1893}|(be} \toendnotes[C]{\smallbreak\pagebreak[2]} \Standort{CUL, Schnitzler, B 8.1, S. 16–17.}
\physDesc{maschinelle Abschrift
\newline{}Handschrift: Bleistift, deutsche Kurrent (\noindent{}eine Korrektur)\newline{}Ordnung: von unbekannter Hand nummeriert mit: »30« }\buchAbdrucke{\weitereDrucke{Arthur Schnitzler, Richard Beer-Hofmann: \emph{Briefwechsel 1891–1931}. Hg. Konstanze Fliedl. Wien, Zürich: \emph{Europaverlag} 1992, S. 49–50.} }\toendnotes[C]{\smallbreak}\pstart
           \raggedleft{}{\pb}\textcolor{pink}{Wien}{}\ledrightnote{\textcolor{pink}{Wien}}, 3. 8. 93.\pend
           \pstart
           Lieber Richard, eben habe ich die \textcolor{green}{Camelia}{}\ledrightnote{\textcolor{green}{Camelias}}’s wiedergelesen und kann Sie versichern, dass sie die gefährliche
               Probe des Wiedererlebens aufs glücklichste bestanden haben. Die Skizze ist eine
               Stiefschwester Ihres »\textcolor{green}{Kind}{}\ledrightnote{\textcolor{green}{Das Kind}}’s«; das Blut des Vaters
               pulsirt drin und dass Sie nun eine neue Muse haben, darf Sie gegen die frühere, mit
               der Sie die \textcolor{green}{Camelias}{}\ledrightnote{\textcolor{green}{Camelias}} gezeugt haben, nicht ungerecht
               machen. Dagegen muss ich aber bemerken, dass mir die Miederstelle noch unangenehmer
               auffiel, als das erste Mal; sie ist absolut überflüssig und ausschliesslich
               widerlich. Mit demselben Recht dürften Sie darauf bestehen, den abendlichen Stuhlgang
               Ihres Helden zu schildern; ja beinahe mit mehr Recht; denn er ist natürlicher und
               berechtigter als das Mieder. Zur Charakteristik \textcolor{green}{Freddys}{}\ledrightnote{→\textcolor{green}{Camelias}} gehört es auch absolut nicht. Sie sollten \textcolor{green}{Freddy}{}\ledrightnote{→\textcolor{green}{Camelias}} auch etwas älter machen;
               denn es ist mir unangenehm, dass man sich mit 38 Jahren schon so fürchterlich {\pb}in der Decadence fühlen soll; – oder, was
               einfacher ist, gehen Sie bei dem Gefühl des Altseins von \textcolor{green}{Freddy}{}\ledrightnote{→\textcolor{green}{Camelias}} mehr auf das psychologische \label{T_L00249_1v}\edtext{als}{\lemma{\textnormal{\emph{als}}}\Cendnote{\textnormal{korrigiert aus: »aus«}}}\label{T_L00249_1h} auf die ganz groben
               körperlichen Dinge. Kurzum, ich will mir nicht von Ihrer \textcolor{green}{Novellette}{}\ledrightnote{→\textcolor{green}{Camelias}} die Möglichkeit nehmen lassen, in sieben Jahren
               ein junges Mädel zu heiraten! Verstehen Sie? – Aber das wesentliche: die \textcolor{green}{Camelia}{}\ledrightnote{\textcolor{green}{Camelias}}’s gehören in Ihr \textcolor{green}{Buch}{}\ledrightnote{→\textcolor{green}{Novellen}}. –\pend
           \pstart
           – Haben Sie das \textcolor{green}{Kind}{}\ledrightnote{\textcolor{green}{Das Kind}} vorgelesen? – Schreiben Sie
               mir darüber! – Ich habe keine Einberufung. Werde vielleicht mit \textcolor{blue}{Salten}{}\ledrightnote{\textcolor{blue}{Felix Salten}} eine Bicycletour machen. –\pend
           \pstart
           Gibts was neues in \textcolor{pink}{Ischl}{}\ledrightnote{\textcolor{pink}{Bad Ischl}}? –\pend
           \pstart
           Las »\textcolor{green}{Die Erziehung zur Ehe}{}\ledrightnote{\textcolor{green}{Die Erziehung zur Ehe}}« von \textcolor{blue}{Hartleben}{}\ledrightnote{\textcolor{blue}{Otto Erich Hartleben}}; gefiel mir bis zum letzten Akt ganz
               ausnehmend. –\pend
           \pstart
           Meine \textcolor{green}{Briefnovellette}{}\ledrightnote{→\textcolor{green}{Die kleine Komödie}} ist bis auf
               ein paar Zeilen fertig. Hoffentlich bring ich doch wieder einmal ein Stück
               zusammen. –\pend
           \pstart
           »Wieder einmal« – Grössenwahn? –\pend
           \pstart Herzlich Ihr \spacefill\mbox{Arthur.}\pend{}\pstart
           \noindent{}Grüssen Sie das nothwendige. \textcolor{green}{Götterliebling}{}\ledrightnote{\textcolor{green}{Der Tod Georgs}}? –\pend
           \pstart
           (nach \textcolor{pink}{Ischl, Schulg.}{}\ledrightnote{\textcolor{pink}{Schulgasse}})\pend
           \endnumbering\briefempfaengerindex{Beer-Hofmann, Richard@\textsc{Beer-Hofmann, Richard}!zzzSchnitzler, Arthur@\emph{von Arthur Schnitzler}!1893-08-031@{3. 8. 1893}|)be}\mylabel{h}  \normalsize

\doendnotes{C}
\bigskip
\vfill

\clearpage

\footnotesize

\lohead{\textsc{register}}

% Definiere theindex-Environment komplett neu ohne reledmac
\makeatletter
\renewenvironment{theindex}{%
  \section*{\indexname}%
  \setlength{\parindent}{0pt}%
  \setlength{\parskip}{0pt plus 0.3pt}%
  \let\item\@idxitem
}{%
  \clearpage
}
\makeatother

\IfFileExists{\jobname-pw.ind}{\input{\jobname-pw.ind}}{}

\end{document}

      