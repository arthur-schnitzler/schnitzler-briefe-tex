%% latex-korrekturansicht-vorspann.tex
%% Vorspann für die Korrekturansicht.
%% Lädt die gemeinsame Datei latex-vorspann.tex mit gesetztem Schalter.

\newif\ifkorrekturansicht
\korrekturansichttrue

\input{../tex-inputs/latex-vorspann}


\renewcommand{\erwaehntePersonen}{Personen: Richard Beer-Hofmann, Olga Schnitzler, Heinrich Schnitzler}
\renewcommand{\erwaehnteOrte}{Orte: Frankfurt am Main, Hinterbrühl, Mürren, Schweiz, Wien}
\renewcommand{\erwaehnteWerke}{Werke: Andreas Thameyers letzter Brief, Der Schleier der Beatrice. Schauspiel in fünf Akten, Die Zeit. Wiener Wochenschrift, Excentric, Jugend}
\section[ Paul Goldmann an Arthur Schnitzler, 30. 7. {[}1902{]}]{Paul Goldmann an Arthur Schnitzler, 30. 7. {[}1902{]}}
\nopagebreak\mylabel{v}
\rehead{ }\normalsize\beginnumbering\briefempfaengerindex{Schnitzler, Arthur@\textsc{Schnitzler, Arthur}!zzzGoldmann, Paul@\emph{von Paul Goldmann}!1902-07-301@{30. 7. {[}1902{]}}|(be}
\toendnotes[C]{\smallbreak\pagebreak[2]}\Standort{DLA, A:Schnitzler, HS.NZ85.1.3172.}
\physDesc{Brief, 1 Blatt, 4 Seiten
\newline{}Handschrift: blaue Tinte, deutsche Kurrent
\newline{}Schnitzler: mit Bleistift das Jahr »{[}1{]}90\textcolor{gray}{2}« vermerkt }\toendnotes[C]{\smallbreak}
\pstart
           \centering{}{\pb}\textcolor{pink}{Frankfurt}{}\ledrightnote{\textcolor{pink}{Frankfurt am Main}}{ }30. Juli.\pend
           
\pstart\center{}Mein lieber Freund,\pend
\pstart
           Ich bin hier auf der Durchreiſe nach der \textcolor{pink}{Schweiz}{}\ledrightnote{\textcolor{pink}{Schweiz}}. Bitte, ſchreib’ mir ein Wort über Dein und \textsc{\textcolor{blue}{Olga}{}\ledrightnote{\textcolor{blue}{Olga Schnitzler}}s} Ergehen \textsc{\begin{otherlanguage}{french}Poste restante\end{otherlanguage}} nach \textsc{\textcolor{pink}{Mürren}{}\ledrightnote{\textcolor{pink}{Mürren}} 
                  (\textcolor{pink}{Schweiz}{}\ledrightnote{\textcolor{pink}{Schweiz}})}, wo ich etwa \strikeout{den} am 5. Auguſt eintreffe. Läßt ſich ſchon der Tag
               des großen {\pb}\strikeout{Ereigniſſte}{ }\label{K_L03215-1v}\edtext{Ereigniſſes}{\lemma{\textnormal{\emph{Ereigniſſes}}}\Cendnote{\textnormal{\textcolor{blue}{Heinrich Schnitzler}s Geburt am 9. 8. 1902}}}\label{K_L03215-1h} ungefähr präciſiren? Ich wäre für eine Depeſche \strikeout{übe} über das Ereigniß ſelbſt ſehr dankbar und möchte namentlich wiſſen, ob
               Du den \textcolor{blue}{Sohn}{}\ledrightnote{{$\rightarrow$}\textcolor{blue}{Heinrich Schnitzler}} haſt, \strikeout{den} den ich Dir wünſche.\pend
           
\pstart
           Hier habe ich Deine \label{K_L03215-2v}\edtext{\textcolor{green}{Geſchichten}{}\ledrightnote{{$\rightarrow$}\textcolor{green}{Andreas Thameyers letzter Brief}{\newline}{$\rightarrow$}\textcolor{green}{Excentric}}}{\lemma{\textnormal{\emph{Geſchichten}}}\Cendnote{\textnormal{\textcolor{blue}{Arthur Schnitzler}: \emph{\textcolor{green}{Andreas Thameyers letzter Brief}}. In: \emph{\textcolor{green}{Die Zeit. Wiener Wochenschrift}}, Jg. 32, Nr. 408, 26. 7. 1902, S. 63–64; \textcolor{blue}{Arthur Schnitzler}: \emph{\textcolor{green}{Excentric}}. In: \emph{\textcolor{green}{Jugend}}, Jg. 7, Nr. 30, {[}16.{]} 7. 1902,
                     S. 492–496.}}}\label{K_L03215-2h} in der »\textcolor{green}{Zeit}{}\ledrightnote{\textcolor{green}{Die Zeit. Wiener Wochenschrift}}«
               und in der »\textcolor{green}{Jugend}{}\ledrightnote{\textcolor{green}{Jugend}}« geleſen. Die \textcolor{green}{erſte}{}\ledrightnote{{$\rightarrow$}\textcolor{green}{Andreas Thameyers letzter Brief}} hat mir gar nicht gefallen, die \textcolor{green}{zweite}{}\ledrightnote{{$\rightarrow$}\textcolor{green}{Excentric}} finde ich {\pb}köſtlich. Oh Gott, wenn Du doch der Humoriſt, der
               glänzende Humoriſt \strikeout{i\textcolor{gray}{mm}} immer ſein wollteſt, \strikeout{d\textcolor{gray}{e}} der Du biſt! Einen Stoff humoriſtiſch behandeln heißt ſich über ihn erheben. Ich
               glaube, das ſollte in den Jahren der Reife das höchſte Ziel ſein.\pend
           
\pstart
           Bitte, grüße mir \textsc{\textcolor{blue}{Richard}{}\ledrightnote{\textcolor{blue}{Richard Beer-Hofmann}}}. Es thut mir unendlich leid, daß ich {\pb}Dich
               und \textcolor{blue}{ihn}{}\ledrightnote{{$\rightarrow$}\textcolor{blue}{Richard Beer-Hofmann}} jetzt nicht ſehen
               werde.\pend
           
\pstart
           Viele treue Grüße! {\\[\baselineskip]}Dein {\\[\baselineskip]}\spacefill\mbox{Paul Goldmann}\pend
           \leftskip=0em{}
\pstart
           \noindent{}Grüße an die \label{K_L03215-3v}\edtext{\textcolor{pink}{Hinterbrühl}{}\ledrightnote{\textcolor{pink}{Hinterbrühl}}}{\lemma{\textnormal{\emph{Hinterbrühl}}}\Cendnote{\textnormal{siehe Paul Goldmann an Arthur Schnitzler, 14. 1. [1902]}}}\label{K_L03215-3h}. Was iſt mit der \label{K_L03215-6v}\edtext{»\textsc{\textcolor{green}{Beatrice}{}\ledrightnote{\textcolor{green}{Der Schleier der Beatrice. Schauspiel in fünf Akten}}}«}{\lemma{\textnormal{\emph{»Beatrice«}}}\Cendnote{\textnormal{siehe Paul Goldmann an Arthur Schnitzler, 14. 7. [1902]}}}\label{K_L03215-6h}?\pend
           \endnumbering\briefempfaengerindex{Schnitzler, Arthur@\textsc{Schnitzler, Arthur}!zzzGoldmann, Paul@\emph{von Paul Goldmann}!1902-07-301@{30. 7. {[}1902{]}}|)be}\mylabel{h}  \normalsize

\doendnotes{C}
\bigskip
\vfill

\clearpage

\footnotesize

\lohead{\textsc{register}}

% Definiere theindex-Environment komplett neu ohne reledmac
\makeatletter
\renewenvironment{theindex}{%
  \section*{\indexname}%
  \setlength{\parindent}{0pt}%
  \setlength{\parskip}{0pt plus 0.3pt}%
  \let\item\@idxitem
}{%
  \clearpage
}
\makeatother

\IfFileExists{\jobname-pw.ind}{\input{\jobname-pw.ind}}{}

\end{document}

      