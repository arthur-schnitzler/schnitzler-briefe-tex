%% latex-korrekturansicht-vorspann.tex
%% Vorspann für die Korrekturansicht.
%% Lädt die gemeinsame Datei latex-vorspann.tex mit gesetztem Schalter.

\newif\ifkorrekturansicht
\korrekturansichttrue

\input{../tex-inputs/latex-vorspann}


               \section[ Paul Goldmann an Arthur Schnitzler, 10. 4. 1898]{Paul Goldmann an Arthur Schnitzler, 10. 4. 1898}\nopagebreak\mylabel{v}\rehead{ }\normalsize\beginnumbering\briefempfaengerindex{Schnitzler, Arthur@\textsc{Schnitzler, Arthur}!zzzGoldmann, Paul@\emph{von Paul Goldmann}!1898-04-101@{10. 4. 1898}|(be} \toendnotes[C]{\smallbreak\pagebreak[2]} \Standort{DLA, A:Schnitzler, HS.NZ85.1.3168.}
\physDesc{Postkarte
\newline{}Handschrift: Bleistift, lateinische Kurrent\newline{}Versand: 1) Stempel: »\nobreak{}\oindex{Port Said@\textbf{Port Said}, \emph{Besiedelter Ort (A.BSO)}|pwk}Port-Said, 10 IV {[}9{]}8, 1\textcolor{gray}{1}\nobreak{}«.  2) Stempel: »\nobreak{}Wien
                                       \textcolor{gray}{9}/3 72, 17. 4. {[}9{]}8, \textcolor{gray}{9}. V, Bestellt\nobreak{}«. 
\newline{}Schnitzler: mit Bleistift das Datum »Apr 98« vermerkt }\toendnotes[C]{\smallbreak}\pstart{}{\pb}\textcolor{gray}{\textbf{\begin{otherlanguage}{french}A\end{otherlanguage}}}\pend{}\pstart{}\textsc{\begin{otherlanguage}{french}M. le Dr.\end{otherlanguage}}\pend{}\pstart{}\textsc{Arthur Schnitzler}\pend{}\pstart{}\textsc{\textcolor{pink}{IX. Frankgaße 1}{}\ledrightnote{\textcolor{pink}{Frankgasse}}}\pend{}\pstart{}\textsc{\textcolor{pink}{Wien}{}\ledrightnote{\textcolor{pink}{Wien}}}\pend{}\pstart{}\textsc{\begin{otherlanguage}{french}\textcolor{pink}{Autriche}{}\ledrightnote{\textcolor{pink}{Österreich}}\end{otherlanguage}}\pend{}{\bigskip}\pstart
           {\pb}Oſtermorgen in \textsc{\textcolor{pink}{Port-Said}{}\ledrightnote{\textcolor{pink}{Port Said}}}\pend
           \pstart
           Eine \textcolor{pink}{engl}{}\ledrightnote{→\textcolor{pink}{England}}iſche Muſikkapelle
               ſpielt auf dem \textcolor{pink}{\textsc{Lesseps}-Platz}{}\ledrightnote{\textcolor{pink}{Ferdinand-de-Lesseps-Platz}}, und in der \textcolor{pink}{Araberſtadt}{}\ledrightnote{\textcolor{pink}{Araberstadt}} wird Hochzeit gefeiert, und die Muſiker ſitzen
               auf dem Pflaſter mit Pauken u. Flöten. Ich bin ſehr ſeekrank. Viele treue Grüße Dir
               u. \textsc{\textcolor{blue}{Richard}{}\ledrightnote{\textcolor{blue}{Richard Beer-Hofmann}}}.\pend
           \pstart Dein
               \spacefill\mbox{P. G.}\pend{}\endnumbering\briefempfaengerindex{Schnitzler, Arthur@\textsc{Schnitzler, Arthur}!zzzGoldmann, Paul@\emph{von Paul Goldmann}!1898-04-101@{10. 4. 1898}|)be}\mylabel{h}\begin{anhang}\end{anhang}\normalsize

\doendnotes{C}
\bigskip
\vfill

\clearpage

\footnotesize

\lohead{\textsc{register}}

% Definiere theindex-Environment komplett neu ohne reledmac
\makeatletter
\renewenvironment{theindex}{%
  \section*{\indexname}%
  \setlength{\parindent}{0pt}%
  \setlength{\parskip}{0pt plus 0.3pt}%
  \let\item\@idxitem
}{%
  \clearpage
}
\makeatother

\IfFileExists{\jobname-pw.ind}{\input{\jobname-pw.ind}}{}

\end{document}

      