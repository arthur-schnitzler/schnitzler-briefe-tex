%% latex-korrekturansicht-vorspann.tex
%% Vorspann für die Korrekturansicht.
%% Lädt die gemeinsame Datei latex-vorspann.tex mit gesetztem Schalter.

\newif\ifkorrekturansicht
\korrekturansichttrue

\input{../tex-inputs/latex-vorspann}


\renewcommand{\erwaehntePersonen}{Personen: G. G.}
\renewcommand{\erwaehnteOrte}{Orte: Brühl, Wien}
\renewcommand{\erwaehnteWerke}{}
\section[ Felix Salten an Arthur Schnitzler, {[}10?. 3. 1902{]}]{Felix Salten an Arthur Schnitzler, {[}10?. 3. 1902{]}}
\nopagebreak\mylabel{v}
\rehead{ }\normalsize\beginnumbering\briefempfaengerindex{Schnitzler, Arthur@\textsc{Schnitzler, Arthur}!zzzSalten, Felix@\emph{von Felix Salten}!1902-03-101@{{[}10?. 3. 1902{]}}|(be}
\toendnotes[C]{\smallbreak\pagebreak[2]}\Standort{CUL, Schnitzler, B 89, A 2.}
\physDesc{Brief, 1 Blatt, 1 Seite, 347 Zeichen
\newline{}Handschrift: schwarze Tinte, lateinische Kurrent
\newline{}Schnitzler: mit Bleistift datiert: »1\substVorne{}\textsuperscript{\textcolor{gray}{5}}\substDazwischen{}\textcolor{gray}{0}\substHinten{}. 3. 902« 
\newline{}Ordnung: mit Bleistift von unbekannter Hand nummeriert: »148« }\toendnotes[C]{\smallbreak}
\pstart
           \raggedleft{}{\pb}\label{K_L03324-1v}\edtext{Montag}{\lemma{\textnormal{\emph{Montag}}}\Cendnote{\textnormal{Die zweite Ziffer des Kalendertags von \textcolor{blue}{Schnitzler}s Datierung ist nicht mit
                        Sicherheit zu entziffern. \textcolor{blue}{Salten}s Angabe »Montag« 
                        erlaubt nur den 10. und den 17. 3. 1902 als mögliche Daten. Eine ›7‹ ist nicht zu erkennen.
                        Dazu kommt, dass \textcolor{blue}{Schnitzler} wohl in
                        Folge dieses Briefs am Folgetag, dem 11. 3. 1902,
                        \textcolor{blue}{Salten} einen Krankenbesuch abstattete,
                           vgl. Felix Salten an Arthur Schnitzler, [11. 3. 1902].}}}\label{K_L03324-1h}.\pend
           
\pstart
           Lieber, bin seit acht Tagen recht krank und zu Bett. \label{K_L03324-2v}\edtext{Geschichte mit \textcolor{blue}{G. G.}{}\ledrightnote{\textcolor{blue}{G. G.}}}{\lemma{\textnormal{\emph{Geschichte mit G. G.}}}\Cendnote{\textnormal{Bezug unklar}}}\label{K_L03324-2h} hat sich nur auf
                  N\textsuperscript{r} 10 bezogen, die »Conservatoristin« wurde dazu
               erfunden. So wird man manchmal beunruhigt. Warum sind Sie noch auf der \label{K_L03324-3v}\edtext{Suche}{\lemma{\textnormal{\emph{Suche}}}\Cendnote{\textnormal{siehe Paul Goldmann an Arthur Schnitzler, 14. 1. [1902]}}}\label{K_L03324-3h}? Sagten Sie mir nicht, Sie hätten in der \textcolor{pink}{Brühl}{}\ledrightnote{\textcolor{pink}{Brühl}} schon fix gemiethet?\pend
           
\pstart
           Hoffentlich bin ich in 8 Tagen wieder wol. Herzlichst Ihr {\\[\baselineskip]}\spacefill\mbox{Salten}\pend
           \leftskip=0em{}\endnumbering\briefempfaengerindex{Schnitzler, Arthur@\textsc{Schnitzler, Arthur}!zzzSalten, Felix@\emph{von Felix Salten}!1902-03-101@{{[}10?. 3. 1902{]}}|)be}\mylabel{h}  \normalsize

\doendnotes{C}
\bigskip
\vfill

\clearpage

\footnotesize

\lohead{\textsc{register}}

% Definiere theindex-Environment komplett neu ohne reledmac
\makeatletter
\renewenvironment{theindex}{%
  \section*{\indexname}%
  \setlength{\parindent}{0pt}%
  \setlength{\parskip}{0pt plus 0.3pt}%
  \let\item\@idxitem
}{%
  \clearpage
}
\makeatother

\IfFileExists{\jobname-pw.ind}{\input{\jobname-pw.ind}}{}

\end{document}

      