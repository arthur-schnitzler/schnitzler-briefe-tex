%% latex-korrekturansicht-vorspann.tex
%% Vorspann für die Korrekturansicht.
%% Lädt die gemeinsame Datei latex-vorspann.tex mit gesetztem Schalter.

\newif\ifkorrekturansicht
\korrekturansichttrue

\input{../tex-inputs/latex-vorspann}


               \section[Arthur Schnitzler an Robert Adam, 27. 10. 1917]{ Arthur Schnitzler an Robert Adam, 27. 10. 1917}\nopagebreak\mylabel{v}\rehead{ }\normalsize\beginnumbering\briefempfaengerindex{Adam, Robert@\textsc{Adam, Robert}!zzzSchnitzler, Arthur@\emph{von Arthur Schnitzler}!1917-10-271@{27. 10. 1917}|(be} \toendnotes[C]{\smallbreak\pagebreak[2]} \Standort{DLA, 96.34.2/7.}
\physDesc{Briefkarte, Umschlag
\newline{}Schreibmaschine
\newline{}Handschrift: schwarze Tinte (\noindent{}Unterschrift)\newline{}Versand: Stempel: »\nobreak{}26. X. 17\nobreak{}«.  }\toendnotes[C]{\smallbreak}\pstart{}{\pb}\textcolor{gray}{\textbf{Dr. Arthur Schnitzler}}\pend{}\pstart{}\textcolor{gray}{\textbf{\textcolor{pink}{Wien XVIII. Sternwartestrasse 71}{}\ledrightnote{\textcolor{pink}{Sternwartestraße}}}}\pend{}{\bigskip}\pstart{}{\pb}Herrn Dr. Robert Adam\pend{}\pstart{}\textcolor{pink}{Wien XII}{}\ledrightnote{\textcolor{pink}{XII., Meidling}}.\pend{}\pstart{}\textcolor{pink}{Meidlinger Hauptstr. 56}{}\ledrightnote{\textcolor{pink}{Meidlinger Hauptstraße}}.\pend{}{\bigskip}\pstart
           \raggedleft{}{\pb}27. 10. 1917.\pend
           \pstart
           \textcolor{gray}{\textbf{Dr. Arthur Schnitzler}}{\\}\textcolor{gray}{\textbf{\textcolor{pink}{Wien XVIII. Sternwartestrasse 71}{}\ledrightnote{\textcolor{pink}{Sternwartestraße}}}}\pend
           \pstart\center{}Verehrter Herr Doktor.\pend\pstart
           Das \textcolor{green}{Manuscript}{}\ledrightnote{→\textcolor{green}{Das Ende des Judas}} geht soeben an
                    Sie zurück. Eines vergass ich gestern noch zu erwähnen. Gewisse Schnoddrigkeiten
                    der Sprache (ich weiss wohl, dass sie künstlerische Absicht waren), Worte, wie
                    arrogant und dergleichen, wären vielleicht doch besser zu eliminieren; wie Sie
                    überhaupt bei neuer Durchsicht gewiss noch Gelegenheit haben werden allerlei
                    sprachliche Flüchtigkeiten zu verbessern. Wir sprechen vielleicht noch einmal im
                    Einzelnen auch darüber.\pend
           \pstart
           Mit herzlichem Gruss{\\[\baselineskip]}Ihr sehr ergebener{\\[\baselineskip]}\spacefill\mbox{{[}hs.:{]} Arthur Schnitzler}\pend
           \leftskip=0em{}\endnumbering\briefempfaengerindex{Adam, Robert@\textsc{Adam, Robert}!zzzSchnitzler, Arthur@\emph{von Arthur Schnitzler}!1917-10-271@{27. 10. 1917}|)be}\mylabel{h}  \normalsize

\doendnotes{C}
\bigskip
\vfill

\clearpage

\footnotesize

\lohead{\textsc{register}}

% Definiere theindex-Environment komplett neu ohne reledmac
\makeatletter
\renewenvironment{theindex}{%
  \section*{\indexname}%
  \setlength{\parindent}{0pt}%
  \setlength{\parskip}{0pt plus 0.3pt}%
  \let\item\@idxitem
}{%
  \clearpage
}
\makeatother

\IfFileExists{\jobname-pw.ind}{\input{\jobname-pw.ind}}{}

\end{document}

      