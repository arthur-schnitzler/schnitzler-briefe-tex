%% latex-korrekturansicht-vorspann.tex
%% Vorspann für die Korrekturansicht.
%% Lädt die gemeinsame Datei latex-vorspann.tex mit gesetztem Schalter.

\newif\ifkorrekturansicht
\korrekturansichttrue

\input{../tex-inputs/latex-vorspann}


               \section[Paul Goldmann an Arthur Schnitzler, Paul Goldmann an Arthur Schnitzler, 11. 1. {[}1896{]}]{ Paul Goldmann an Arthur Schnitzler, 11. 1. {[}1896{]}}\nopagebreak\mylabel{v}\rehead{ }\normalsize\beginnumbering\briefempfaengerindex{Schnitzler, Arthur@\textsc{Schnitzler, Arthur}!zzzGoldmann, Paul@\emph{von Paul Goldmann}!1896-01-111@{11. 1. {[}1896{]}}|(be} \toendnotes[C]{\smallbreak\pagebreak[2]} \Standort{DLA, A:Schnitzler, HS.NZ85.1.3166.}
\physDesc{Brief, 4 Blätter, 15 Seiten
\newline{}Handschrift: blaue Tinte, deutsche Kurrent\newline{}Beilagen: 1) Vordruck mit handschriftlicher
                                 Nachricht: 1 Blatt, 1 Seite 2) handschriftlicher Brief: 1 Blatt, 1 Seite
\newline{}Schnitzler: 1) mit Bleistift das Jahr »96« vermerkt 2) mit rotem Buntstift zwei Unterstreichungen}\toendnotes[C]{\smallbreak}\pstart
           \noindent{}{\pb}\textcolor{gray}{\textbf{\textbf{\textcolor{brown}{Frankfurter Zeitung}{}\ledrightnote{\textcolor{brown}{Frankfurter Zeitung}}}}}\pend
           \pstart
           \textcolor{gray}{\textbf{(\textcolor{brown}{\begin{otherlanguage}{french}Gazette de Francfort\end{otherlanguage}}{}\ledrightnote{\textcolor{brown}{Frankfurter Zeitung}}).}}\pend
           \pstart
           \textcolor{gray}{\textbf{\textbf{\begin{otherlanguage}{french}Fondateur M.\end{otherlanguage}{ }\textcolor{blue}{L. Sonnemann}{}\ledrightnote{\textcolor{blue}{Leopold Sonnemann}}.}}}\pend
           \pstart
           \begin{otherlanguage}{french}\textcolor{gray}{\textbf{\textcolor{green}{Journal}{}\ledrightnote{→\textcolor{green}{Frankfurter Zeitung}} politique,
                        financier,}}\end{otherlanguage}\pend
           \pstart
           \begin{otherlanguage}{french}\textcolor{gray}{\textbf{commercial et littéraire.}}\end{otherlanguage}\pend
           \pstart
           \begin{otherlanguage}{french}\textcolor{gray}{\textbf{\textbf{Paraissant trois fois par jour.}}}\end{otherlanguage}\pend
           \pstart
           \begin{otherlanguage}{french}\textcolor{gray}{\textbf{\textbf{Bureau à \textcolor{pink}{Paris}{}\ledrightnote{\textcolor{pink}{Paris}}}}}\end{otherlanguage}\hfill \textsc{\textcolor{pink}{Paris}{}\ledrightnote{\textcolor{pink}{Paris}}}, 11. Januar.\pend
           \pstart
           \begin{otherlanguage}{french}\textcolor{gray}{\textbf{\textbf{\textcolor{pink}{24. Rue Feydeau}{}\ledrightnote{\textcolor{pink}{rue Feydeau}}.}}}\end{otherlanguage}\pend
           \pstart\center{}Mein lieber Freund,\pend\pstart
           Heut geht das Opernglas an Dich ab. Ich habe Dich
               lange warten laſſen müſſen. Erſtens hatte ich viel zu thun, zweitens war es keine
               leichte Geſchichte. Ich bin bei allen möglichen Optikern herumgelaufen. Die große
               Schwierigkeit war der Ausſchluß von Perlmutter. Alles, was hier hübſch und \textcolor{pink}{Pariſ}{}\ledrightnote{\textcolor{pink}{Paris}}eriſch ausſieht, wird in Perlmutter aller
               Arten und Farben gemacht. Dann hat man noch ganz ſchwarze {\pb}Operngläſer, endlich \label{K_L02762-1v}\edtext{Schildpatt}{\lemma{\textnormal{\emph{Schildpatt}}}\Cendnote{\textnormal{Material aus Schuppen von Meeresschildkröten, auch als Musterbezeichnung
                  gebrauchbar}}}\label{K_L02762-1h}. Ich habe mich zu letzterem entſchloſſen, damit wenigſtens
               etwas Farbe daran iſt. Weitere Schwierigkeit: Die wirklich guten Gläſer finden ſich
               nur bei den großen Inſtrumenten. Je kleiner die Gläſer, umſo weniger gut ſieht man.
               Je kleiner die Gläſer, umſo zierlicher freilich und umſo reicher ornamentirt iſt die
               Form des Ganzen. Mich ſtrict an Deine Weiſungen haltend, habe ich das den Gläſern {\pb}nach beſte Opernglas genommen, das ich in \strikeout{de} der betreffenden Preislage finden konnte. Es enthält
               zwölf Gläſer und ſtammt von einem in \textsc{\textcolor{pink}{Paris}{}\ledrightnote{\textcolor{pink}{Paris}}} beſtbekannten Optiker. \strikeout{Um eine gewiſſſe} Man
               ſieht gut dadurch, freilich mußte ich deshalb ein etwas größeres Format wählen. Es
               iſt zur Herſtellung \textsc{Aluminium} verwendet, was jetzt hier
               ſehr in der Mode iſt. Ich kann das zwar abſolut nicht leiden, aber das Opernglas hat
               dadurch den Vortheil {\pb}größter Leichtigkeit. Auch
               ſonſt gefällt mir meine Wahl äußerlich gar nicht; \strikeout{abe}
               aber Du haſt mir zu enge Grenzen geſteckt, und mein Geſchmack konnte ſich darin nicht
               frei bewegen. Jedenfalls habe ich mit dem Optiker den \uline{Umtauſch} ausgemacht. Gefällts Dir alſo nicht, ſo ſchickſt Du mirs zurück und
               gibſt mir nähere Weiſungen. Koſten ſollte es 60 \textsc{Frcs}, ich
               habe aber einige Tage {\pb}manörvrirt und
                  ſchließlich 50 \textsc{Frcs} herausgehandelt. Freilich dürfte
               ſich der Ehrenmann wohl noch 5 \textsc{Frcs} für Verpackung Porto
                  \textsc{etc.} herausſchwindeln. Soll ich Dir den Reſt ſchicken
               oder ſoll ich noch etwas dafür hier kaufen?\pend
           \pstart
           Über die verſchwundenen Goldſtücke hat die hieſige \textcolor{brown}{Poſt}{}\ledrightnote{\textcolor{brown}{Französische Post}} auf meine Beſchwerde eine Unterſuchung eingeleitet, wie beifolgendes
                  {\pb}Papier beſtätigt, das Dir vielleicht als Ausweis
               gegenüber der \textcolor{brown}{öſterreichiſchen Poſt}{}\ledrightnote{\textcolor{brown}{Österreichische Post}} dienen
               kann.\pend
           \pstart
           Auch ſende ich Dir einen Brief von \textsc{\textcolor{blue}{Thorel}{}\ledrightnote{\textcolor{blue}{Jean Thorel}}}, der ein \label{K_L02762-3v}\edtext{\textcolor{green}{Stück}{}\ledrightnote{→\textcolor{green}{Deux sœurs, pièce en 3 actes}}}{\lemma{\textnormal{\emph{Stück}}}\Cendnote{\textnormal{\emph{\textcolor{green}{Deux sœurs, pièce en 3 actes}} wurde am 23. 4. 1896 im \textcolor{pink}{Paris}er \emph{\textcolor{brown}{Odéon}} uraufgeführt.}}}\label{K_L02762-3h} im
                  »\textsc{\textcolor{brown}{Odéon}{}\ledrightnote{\textcolor{brown}{Odéon}}}« aufgeführt bekommen ſoll. Man zieht ihn furchtbar damit herum, und das macht
               ihm den Kopf verrückt. Laſſen wir ihm noch etwas \label{K_L02762-2v}\edtext{Zeit}{\lemma{\textnormal{\emph{Zeit}}}\Cendnote{\textnormal{mit der \textcolor{green}{Übersetzung} der \emph{\textcolor{green}{Liebelei}}}}}\label{K_L02762-2h}.\pend
           \pstart
           Den guten \label{K_L02762-4v}\edtext{\textcolor{blue}{Mann}{}\ledrightnote{→\textcolor{blue}{Henri de Riaz}}{ }{\pb}aus \textsc{\textcolor{pink}{Lyon}{}\ledrightnote{\textcolor{pink}{Lyon}}}}{\lemma{\textnormal{\emph{Mann aus Lyon}}}\Cendnote{\textnormal{siehe Paul Goldmann an Arthur Schnitzler, 5. 12. [1895]}}}\label{K_L02762-4h} beſcheide aufſchiebend. Viel Vertrauen flößt er mir nicht ein. Die
               Zeitſchriften, die er \strikeout{vor} nennt, ſind unbedeutend,
               die \strikeout{Refer} Beziehungen, die er angibt, noch mehr. Für
               das »\textsc{\textcolor{brown}{Œuvre}{}\ledrightnote{\textcolor{brown}{Théâtre de l’Œuvre}}}« oder das »\textsc{\textcolor{brown}{Théâtre Libre}{}\ledrightnote{\textcolor{brown}{Théâtre Libre}}}« brauchen wir ihn nicht. Mit denen ſtehe ich allein in Verbindung. Auch ſpielt
               man dort ſo erbärmlich, daß ich Dich nicht gern dort aufgeführt ſehen {\pb}möchte. Endlich ſoll Dein \textcolor{green}{Stück}{}\ledrightnote{→\textcolor{green}{Liebelei. Schauspiel in drei Akten}} in \textsc{\uline{\textcolor{pink}{Paris}{}\ledrightnote{\textcolor{pink}{Paris}}}} überſetzt werden. Was aus der Provinz, aus \textsc{\textcolor{pink}{Lyon}{}\ledrightnote{\textcolor{pink}{Lyon}}} kommt, darüber rümpfen ſie in \textsc{\textcolor{pink}{Paris}{}\ledrightnote{\textcolor{pink}{Paris}}} bereits die Naſe. Nach einem großen Erfolge in \textcolor{pink}{Berlin}{}\ledrightnote{\textcolor{pink}{Berlin}} – den ich \strikeout{\textcolor{gray}{×}} vorausſehe – werden ſich Dir ganz andere Leute anbieten; vorher darfſt Du wohl
               kein Engagement eingehen.\pend
           \pstart
           {\pb}Vielen Dank noch für Deine Einladung zum
               Zuſammentreffen in \label{K_L02762-5v}\edtext{\textsc{\textcolor{pink}{Frankfurt}{}\ledrightnote{\textcolor{pink}{Frankfurt am Main}}}}{\lemma{\textnormal{\emph{Frankfurt}}}\Cendnote{\textnormal{\textcolor{blue}{Schnitzler} hielt sich zwischen 10. 1. 1896 und 13. 1. 1896 in \textcolor{pink}{Frankfurt am Main} auf.}}}\label{K_L02762-5h}! Das wäre ſchön
               geweſen. Aber die Idee war phantaſtiſch. Im Januar von
               hier fort! Ich glaube, ich wäre entlaſſen worden. Und kein Geld zur Reiſe! Nur
               Schulden! Nie im Leben bin ich dem Bankerott ſo nahe geweſen. Aber es war lieb, daß
               Du an mich gedacht haſt. Wann {\pb}werden wir uns
                  \label{K_L02762-7v}\edtext{wiederſehen}{\lemma{\textnormal{\emph{wiederſehen}}}\Cendnote{\textnormal{Sie sahen sich am 5. 8. 1896
                  in \textcolor{pink}{Kopenhagen} wieder.}}}\label{K_L02762-7h}? Gott weiß! Ich
               glaube, ich gehe nicht mehr aus \textsc{\textcolor{pink}{Paris}{}\ledrightnote{\textcolor{pink}{Paris}}} heraus. Hier bin ich vergraben, die Welt draußen aber thut mir \strikeout{wehe} weh. Neugierig bin ich auf das Ergebniß der
                  \label{K_L02762-8v}\edtext{erſten \textcolor{green}{Aufführung}{}\ledrightnote{→\textcolor{green}{Liebelei. Schauspiel in drei Akten}} in \textcolor{pink}{Deutſchland}{}\ledrightnote{\textcolor{pink}{Deutschland}}}{\lemma{\textnormal{\emph{erſten … Deutſchland}}}\Cendnote{\textnormal{\emph{\textcolor{green}{Liebelei}}-Premiere am 4. 2. 1896 im \textcolor{pink}{Deutschen Theater Berlin}}}}\label{K_L02762-8h} und – auf meinen \textcolor{blue}{Onkel}{}\ledrightnote{→\textcolor{blue}{Fedor Mamroth}}. Ich habe ihm dieſer Tage geſchrieben, weil ich \strikeout{furch} fürchte, daß er Dir wehthut aus Haß gegen \textsc{\strikeout{Speid}}{ }\label{K_L02762-9v}\edtext{\textsc{\textcolor{blue}{Speidel}{}\ledrightnote{\textcolor{blue}{Ludwig Speidel}}}}{\lemma{\textnormal{\emph{Speidel}}}\Cendnote{\textnormal{\textcolor{blue}{Ludwig Speidel} hatte sich zuvor äußerst
                  positiv zur \emph{\textcolor{green}{Liebelei}} geäußert. Siehe
                        [\textcolor{blue}{Ludwig Speidel}]: \emph{\textcolor{green}{Theater- und Kunstnachrichten.
                        [Burgtheater]}}. In: \emph{\textcolor{green}{Neue Freie
                        Presse}}, Nr. 11.181, 10. 10. 1895,
                     S. 7 und \textcolor{blue}{L. Sp.} [=\textcolor{blue}{Ludwig Speidel}]: \emph{\textcolor{green}{Burgtheater. (»Liebelei«, Schauspiel in drei Aufzügen von
                        Arthur Schnitzler. – »Rechte der Seele«, Schauspiel in einem Act von
                        Giuseppe Giacosa, deutsch von Otto Eisenschitz.)}}. In: \emph{\textcolor{green}{Neue Freie Presse}}, Nr. 11.184, 13. 10. 1895, Morgenblatt, S. 1–3.}}}\label{K_L02762-9h}. – Im Grunde {\pb}aber iſt er doch ein hochanſtändiger und
               kunſtliebender Mann – und darauf hoffe ich.\pend
           \pstart
           Ich habe Dir für ſo viele liebe Briefe zu danken. Dein letzter war melancholiſch.
               Dein Talent ſoll nur Deine Jugend geweſen ſein. Oh Du Kind! Wenn irgend ein Talent zu
               reifen beſtimmt iſt, ſo iſt es Deines. Es iſt kein Schwindel und kein Dunſt darin! Es
               beruht auf klarer {\pb}und \strikeout{ve} ernſter Anſchauung des Lebens. \strikeout{Das} Das
               kann nicht altern. Im Gegentheil. Da ſich Einem das Leben immer größer und
               vielgeſtaltiger aufthut, je älter man wird – was wird Dein Talent erſt daraus ziehen,
                  \strikeout{wenn} nachdem es aus dem Bischen Jugend und Liebe
               ſchon ſo viel gezogen hat! Oder wirſt Du vielleicht morgen plötzlich {\pb}aufhören, ein Poet zu ſein? Glaubſt Du, das verliert
               ſich mit den Jahren\substVorne{}\textsuperscript{?}\substDazwischen{}.\substHinten{} Oh Du Kind!{\dotsfive}\pend
           \pstart
           Von meinem Leben will ich Dir nicht ſprechen. Ich ſchäme mich. Es iſt zu ſehr
               dieſelbe Geſchichte. Das Leben, unermündlich mir \strikeout{ne}
               neue Glücks-Möglichkeiten in die Hand zu ſpielen, und {\pb}ich unermüdlich, ſie mir ſtets auf dieſelbe Weiſe zu
               verderben: durch Schwäche, durch mangelnde Mannhaftigkeit \textsc{etc}. Wenn man 31 Jahre geworden iſt, ſo ändert man ſein Leben nicht mehr.
               Und wenn es einmal in eine falſche Richtung eingelenkt iſt, ſo geht es unaufhaltſam
               in dieſer Richtung weiter. Verfahren! Unglücklich ſein, das kann man {\pb}ertragen. Aber wenn man ſtets durch eigene Schuld
               unglücklich iſt, – das erträgt man kaum.\pend
           \pstart
           Grüß’ Dich Gott, mein lieber Freund! Schreib’ mir bald! Wie ſtehts mit dem neuen \textcolor{green}{Stücke}{}\ledrightnote{→\textcolor{green}{Freiwild. Schauspiel in 3 Akten}}? Rückt die \label{K_L02762-11v}\edtext{zweite Niederſchrift}{\lemma{\textnormal{\emph{zweite Niederſchrift}}}\Cendnote{\textnormal{\textcolor{blue}{Schnitzler}, der mit \emph{\textcolor{green}{Freiwild}} äußerst unzufrieden war, begann das \textcolor{green}{Stück} am 31. 12. 1895
                  neu.}}}\label{K_L02762-11h} vorwärts?\pend
           \pstart
           Viele Grüße an \textsc{\textcolor{blue}{Richard}{}\ledrightnote{\textcolor{blue}{Richard Beer-Hofmann}}}!\pend
           \pstart
           In Treue {\\[\baselineskip]}Dein {\\[\baselineskip]}\spacefill\mbox{Paul Goldmann.}\pend
           \leftskip=0em{}{\bigskip}\pstart
           \noindent{}{\pb}\textcolor{gray}{\textbf{\textsc{\textbf{N\textsuperscript{o} 1293 G⋅A⋅C.}}}}\pend
           \pstart
           \textcolor{gray}{\textbf{\textbf{\begin{otherlanguage}{french}Décembre 91\end{otherlanguage} – Coq 55.}}}\pend
           \pstart
           \begin{otherlanguage}{french}\textcolor{brown}{\emph{Ministère} du Commerce, de l’Industrie et des
                        Colonies}{}\ledrightnote{→\textcolor{brown}{Ministère du Commerce, de l’Industrie et des Colonies}}\end{otherlanguage}\pend
           \pstart
           \begin{otherlanguage}{french}\textcolor{gray}{\textbf{\textcolor{brown}{\emph{Direction Générale} des Postes et des
                           Télégraphes.}{}\ledrightnote{→\textcolor{brown}{Ministère du Commerce, de l’Industrie et des Colonies}}}}\end{otherlanguage}\pend
           \pstart
           \begin{otherlanguage}{french}\textcolor{gray}{\textbf{\emph{Exploitation Postale.}}}\end{otherlanguage}\pend
           \pstart
           \begin{otherlanguage}{french}\textcolor{gray}{\textbf{\textbf{Bureau des Réclamations.}}}\end{otherlanguage}\pend
           \pstart
           \begin{otherlanguage}{french}\textcolor{gray}{\textbf{\emph{2\textsuperscript{e} Section.}}}\end{otherlanguage}\pend
           \pstart
           \textcolor{gray}{\textbf{–6–}}\pend
           \pstart
           \textcolor{gray}{\textbf{\emph{N\textsuperscript{o}}}}{[}hs. [Bl]an[qui?]:{]} sp. 344\textcolor{gray}{\textbf{\emph{.}}}\pend
           \pstart
           \begin{otherlanguage}{french}\textcolor{gray}{\textbf{\textbf{Avis d’enquête.}}}\end{otherlanguage}\pend
           \pstart
           \centering{}\textcolor{gray}{\textbf{\textbf{\begin{otherlanguage}{french}\textcolor{pink}{République Française}{}\ledrightnote{\textcolor{pink}{Frankreich}}. \end{otherlanguage}}}}\pend
           \pstart
           \noindent{}\raggedleft{}\begin{otherlanguage}{french}\textcolor{gray}{\textbf{\emph{\textcolor{pink}{Paris}{}\ledrightnote{\textcolor{pink}{Paris}}, le}}}{ }23 décembre \textcolor{gray}{\textbf{\emph{189}}}5\textcolor{gray}{\textbf{\emph{.}}}\end{otherlanguage}\pend
           \pstart{}\begin{otherlanguage}{french}\textcolor{gray}{\textbf{\emph{M}}}onsieur,\end{otherlanguage}\pend\pstart
           \label{K_L02762-14v}\edtext{\begin{otherlanguage}{french}\textcolor{gray}{\textbf{\emph{J’ai reçu la réclamation que vous m’avez adressé le}}}{ }21 décembre courant, à l’occasion d’une lettre
                  recommandée qui vous a été expédiée de \textcolor{pink}{Vienne}{}\ledrightnote{\textcolor{pink}{Wien}}
                     (\textcolor{pink}{Autriche}{}\ledrightnote{\textcolor{pink}{Österreich}}), le 19 décembre, sous le N\textsuperscript{o} 745, par M.
                  Schnitzler, {\kaufmannsund} dans laquelle vous déclarez n’avoir
                  plus trouvé trois pièces de 20 \textsuperscript{f}. qui y auraient été
                  insérées.\end{otherlanguage}}{\lemma{\textnormal{\emph{J’ai … insérées.}}}\Cendnote{\textnormal{französisch: Ich habe Ihre Beschwerde
                  erhalten, die Sie am 21. Dezember an mich gerichtet
                  haben, betreffs eines eingeschriebenen Briefes, der Ihnen am 19. Dezember von Herrn \textcolor{blue}{Schnitzler} aus \textcolor{pink}{Wien} (\textcolor{pink}{Österreich}) zugesandt wurde und in dem Sie
                  angeben, drei Münzen zu 20f. nicht mehr gefunden haben, die darin
                  enthalten gewesen sein sollen.}}}\label{K_L02762-14h}\pend
           \pstart
           \label{K_L02762-15v}\edtext{\begin{otherlanguage}{french}\textcolor{gray}{\textbf{\emph{Des ordres ont été immédiatement donnés pour que les faits
                        que vous m’avez signalés soient l’objet d’une d’une enquête dont je vous
                        ferai connaître le résultat dès qu’elle sera terminée.}}}\end{otherlanguage}}{\lemma{\textnormal{\emph{Des … terminée.}}}\Cendnote{\textnormal{französisch: Anweisungen wurden unmittelbar getroffen,
                  dass die Tatsachen, auf die Sie mich hingewiesen haben, die Grundlage einer Untersuchung bilden, 
                  deren Ergebnis ich nach Abschluss Ihnen mitteilen werde.}}}\label{K_L02762-15h}\pend
           \pstart
           \label{K_L02762-16v}\edtext{\begin{otherlanguage}{french}\textcolor{gray}{\textbf{\emph{Agréez, M}}}onsieur, \textcolor{gray}{\textbf{\emph{l’assurance de ma considération distinguée}}}\end{otherlanguage}}{\lemma{\textnormal{\emph{Agréez, … distinguée}}}\Cendnote{\textnormal{französisch: Gestatten Sie mir, mein
                  Herr, die Versicherung meiner vorzüglichen Hochachtung}}}\label{K_L02762-16h}\pend
           \pstart
           \begin{otherlanguage}{french}\textcolor{gray}{\textbf{\textbf{Pour le \textcolor{blue}{Directeur}{}\ledrightnote{→\textcolor{blue}{Justin de Selves}} Général des Postes et des Télégraphes:}}}{ }{\\}\textcolor{gray}{×}{ }\textcolor{gray}{\textbf{\emph{L’\textcolor{blue}{Administrateur}{}\ledrightnote{→\textcolor{blue}{[Bl]an[qui?]}},}}}\end{otherlanguage}{ }{\\[\baselineskip]}\spacefill\mbox{\textcolor{blue}{\textcolor{gray}{Bl}an\textcolor{gray}{qui}}{}\ledrightnote{\textcolor{blue}{[Bl]an[qui?]}}\textcolor{gray}{×}\-\textcolor{gray}{×}}\pend
           \leftskip=0em{}\pstart
           \noindent{}\begin{otherlanguage}{french}\textcolor{gray}{\textbf{\emph{M}}}onsieur\end{otherlanguage} Paul Goldmann\pend
           {\bigskip}\pstart
           \raggedleft{}{\pb}\textcolor{pink}{12 rue de Milan}{}\ledrightnote{\textcolor{pink}{Rue de Milan}}\pend
           \pstart\center{}\begin{otherlanguage}{french}Cher Monsieur Goldmann.\end{otherlanguage}\pend\pstart
           \label{K_L02762-12v}\edtext{\begin{otherlanguage}{french}Trés touché de votre aimable attention le jour del’an. Je vous
                  envoie auſsi tous mes meilheurs souhaits.\end{otherlanguage}}{\lemma{\textnormal{\emph{Trés … souhaits.}}}\Cendnote{\textnormal{französisch: Sehr berührt von Ihrer
               freundlichen Aufmerksamkeit zum Neujahrstag.}}}\label{K_L02762-12h}\pend
           \pstart
           \label{K_L02762-44v}\edtext{\begin{otherlanguage}{french}Pourriez-vous me dire l’adreſse de Schnitzler? Elle était bien
                  sur sa lettre, mais illisible. J’ai été trés pris ce mois-ci par une affaire que
                  je voudrais entreprendre\strikeout{,} et je n’ai pas encore eu
                  le temps de lire »\textcolor{green}{Liebelei}{}\ledrightnote{\textcolor{green}{Liebelei. Schauspiel in drei Akten}}«, mais je pense
                  bien pouvoir le lire en ces-jours-ci.\end{otherlanguage}}{\lemma{\textnormal{\emph{Pourriez-vous … ces-jours-ci.}}}\Cendnote{\textnormal{französisch: Können Sie
                  mir die Adresse von Schnitzler mitteilen? Sie stand wohl auf seinem Brief, aber unleserlich. Ich war diesen Monat von
                  einer Sache mit Beschlag belegt, die ich unternehmen möchte, und hatte noch keine Zeit,
                  »\emph{\textcolor{green}{Liebelei}}« zu lesen, aber ich bin zuversichtlich,
                      in den kommenden Tagen dazu zu kommen.}}}\label{K_L02762-44h}\pend
           \pstart
           \label{K_L02762-444v}\edtext{\begin{otherlanguage}{french}Votre tous dévoué\end{otherlanguage}}{\lemma{\textnormal{\emph{Votre tous dévoué}}}\Cendnote{\textnormal{französisch: Ihr sehr ergebener}}}\label{K_L02762-444h}{\\[\baselineskip]}\spacefill\mbox{\textcolor{blue}{Jean Thorel}{}\ledrightnote{\textcolor{blue}{Jean Thorel}}}\pend
           \leftskip=0em{}\endnumbering\briefempfaengerindex{Schnitzler, Arthur@\textsc{Schnitzler, Arthur}!zzzGoldmann, Paul@\emph{von Paul Goldmann}!1896-01-111@{11. 1. {[}1896{]}}|)be}\mylabel{h}\begin{anhang}\end{anhang}\normalsize

\doendnotes{C}
\bigskip
\vfill

\clearpage

\footnotesize

\lohead{\textsc{register}}

% Definiere theindex-Environment komplett neu ohne reledmac
\makeatletter
\renewenvironment{theindex}{%
  \section*{\indexname}%
  \setlength{\parindent}{0pt}%
  \setlength{\parskip}{0pt plus 0.3pt}%
  \let\item\@idxitem
}{%
  \clearpage
}
\makeatother

\IfFileExists{\jobname-pw.ind}{\input{\jobname-pw.ind}}{}

\end{document}

      