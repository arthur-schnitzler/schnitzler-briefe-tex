%% latex-korrekturansicht-vorspann.tex
%% Vorspann für die Korrekturansicht.
%% Lädt die gemeinsame Datei latex-vorspann.tex mit gesetztem Schalter.

\newif\ifkorrekturansicht
\korrekturansichttrue

\input{../tex-inputs/latex-vorspann}


\renewcommand{\erwaehntePersonen}{Personen: Johann Benedickter, Anna Katharina Rehmann, Ottilie Salten}
\renewcommand{\erwaehnteOrte}{Orte: Edmund-Weiß-Gasse 7, Riedhof, Schlösselgasse, Wickenburggasse, Wien, XVIII., Währing}
\renewcommand{\erwaehnteWerke}{Werke: Symphonie Nr. 3 D-Moll}
\section[ Felix Salten an Arthur Schnitzler, {[}22. 12. 1904?{]}]{Felix Salten an Arthur Schnitzler, {[}22. 12. 1904?{]}}
\nopagebreak\mylabel{v}
\rehead{ }\normalsize\beginnumbering\briefempfaengerindex{Schnitzler, Arthur@\textsc{Schnitzler, Arthur}!zzzSalten, Felix@\emph{von Felix Salten}!1904-12-221@{{[}22. 12. 1904?{]}}|(be}
\toendnotes[C]{\smallbreak\pagebreak[2]}\Standort{CUL, Schnitzler, B 89, B 1.}
\physDesc{Bildpostkarte, 215 Zeichen
\newline{}Handschrift: schwarze Tinte, lateinische Kurrent
\newline{}Ordnung: mit Bleistift von unbekannter Hand nummeriert: »196« }\toendnotes[C]{\smallbreak}\pstart{}{\pb}Herrn D\textsuperscript{r} Arthur Schnitzler\pend{}\pstart{}\textcolor{pink}{Wien XVIII.}{}\ledrightnote{\textcolor{pink}{XVIII., Währing}}\pend{}\pstart{}\textcolor{pink}{Spöttelgaße 7}{}\ledrightnote{\textcolor{pink}{Edmund-Weiß-Gasse 7}}\pend{}
{\bigskip}
\pstart
           \noindent{}\centering{}{\pb}\textcolor{gray}{\textbf{\textcolor{blue}{Johann Benedickter}{}\ledrightnote{\textcolor{blue}{Johann Benedickter}}’s}}\pend
           
\pstart
           \noindent{}\centering{}\textcolor{pink}{\textcolor{gray}{\textbf{Restaurant u. Weinhandlung »zum Riedhof«}}}{}\ledrightnote{\textcolor{pink}{Riedhof}}\pend
           
\pstart
           \noindent{}\centering{}\textcolor{gray}{\textbf{\textcolor{pink}{WIEN}{}\ledrightnote{\textcolor{pink}{Wien}}}}\pend
           
\pstart
           \noindent{}\centering{}\textcolor{gray}{\textbf{\textcolor{pink}{VIII, Schlösselgasse 14}{}\ledrightnote{\textcolor{pink}{Schlösselgasse}}}}\pend
           
\pstart
           \noindent{}\centering{}\textcolor{pink}{\textcolor{gray}{\textbf{Wickenburggasse 15}}}{}\ledrightnote{\textcolor{pink}{Wickenburggasse}}\pend
           
\pstart
           \noindent{}\centering{}\textcolor{gray}{\textbf{Garten mit Spiegelveranda}}\pend
           
\pstart
           \noindent{}\centering{}\textcolor{gray}{\textbf{Marmorsaal}}\pend
           
\pstart
           zwischen ¾ 11–11\pend
           
\pstart
           Sind etwas verspätet \label{K_L03403-1v}\edtext{gekommen}{\lemma{\textnormal{\emph{gekommen}}}\Cendnote{\textnormal{in den \textcolor{pink}{Riefhof}, vgl. Felix Salten an Arthur Schnitzler, [20. 12. 1904]}}}\label{K_L03403-1h}, weil \textcolor{blue}{Otti}{}\ledrightnote{\textcolor{blue}{Ottilie Salten}} nach dem \textcolor{green}{Conzert}{}\ledrightnote{{$\rightarrow$}\textcolor{green}{Symphonie Nr. 3 D-Moll}}{ }\label{K_L03403-2v}\edtext{des \textcolor{blue}{Kind}{}\ledrightnote{{$\rightarrow$}\textcolor{blue}{Anna Katharina Rehmann}}es wegen}{\lemma{\textnormal{\emph{des Kindes wegen}}}\Cendnote{\textnormal{siehe Felix Salten an Arthur Schnitzler, [15. 12. 1904]}}}\label{K_L03403-2h} nochmals nach hause mußte, und sind sehr erstaunt, dass Sie es so eilig
               hatten.\pend
           \pstart \spacefill\mbox{S.}\pend{}\endnumbering\briefempfaengerindex{Schnitzler, Arthur@\textsc{Schnitzler, Arthur}!zzzSalten, Felix@\emph{von Felix Salten}!1904-12-221@{{[}22. 12. 1904?{]}}|)be}\mylabel{h}  \normalsize

\doendnotes{C}
\bigskip
\vfill

\clearpage

\footnotesize

\lohead{\textsc{register}}

% Definiere theindex-Environment komplett neu ohne reledmac
\makeatletter
\renewenvironment{theindex}{%
  \section*{\indexname}%
  \setlength{\parindent}{0pt}%
  \setlength{\parskip}{0pt plus 0.3pt}%
  \let\item\@idxitem
}{%
  \clearpage
}
\makeatother

\IfFileExists{\jobname-pw.ind}{\input{\jobname-pw.ind}}{}

\end{document}

      