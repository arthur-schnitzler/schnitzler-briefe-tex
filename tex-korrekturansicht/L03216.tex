%% latex-korrekturansicht-vorspann.tex
%% Vorspann für die Korrekturansicht.
%% Lädt die gemeinsame Datei latex-vorspann.tex mit gesetztem Schalter.

\newif\ifkorrekturansicht
\korrekturansichttrue

\input{../tex-inputs/latex-vorspann}


\renewcommand{\erwaehntePersonen}{Personen: Maurice Maeterlinck}
\renewcommand{\erwaehnteInstitutionen}{Institutionen: Deutsches Theater Berlin, Schiller-Theater}
\renewcommand{\erwaehnteOrte}{Orte: Basel, Berlin, Frankgasse, Schweiz, Wien}
\renewcommand{\erwaehnteWerke}{Werke: Der Schleier der Beatrice. Schauspiel in fünf Akten, Monna Vanna. Schauspiel in drei Akten}
\section[ Paul Goldmann an Arthur Schnitzler, 31. 7. 1902]{Paul Goldmann an Arthur Schnitzler, 31. 7. 1902}
\nopagebreak\mylabel{v}
\rehead{ }\normalsize\beginnumbering\briefempfaengerindex{Schnitzler, Arthur@\textsc{Schnitzler, Arthur}!zzzGoldmann, Paul@\emph{von Paul Goldmann}!1902-07-311@{31. 7. 1902}|(be}
\toendnotes[C]{\smallbreak\pagebreak[2]}\Standort{DLA, A:Schnitzler, HS.NZ85.1.3172.}
\physDesc{Postkarte
\newline{}Handschrift: 1) Bleistift, deutsche Kurrent\hspace{1em}2) Bleistift, lateinische Kurrent (\noindent{}Adresse)\hspace{1em}
\newline{}Versand: 1) Stempel: »\nobreak{}\oindex{Basel@\textbf{Basel}, \emph{Besiedelter Ort (A.BSO)}|pwk}Basel 1 Fil. S. B., 31. VII. 02, 9\nobreak{}«.   2) Stempel: »\nobreak{}9/3 {[}Wien{]} 72, 2. 8. 02, 8.V, Bes{[}tellt{]}\nobreak{}«. 
\newline{}Schnitzler: mit Bleistift das Jahr »{[}1{]}902« vermerkt }\toendnotes[C]{\smallbreak}\pstart{}{\pb}Herrn\pend{}\pstart{}Dr. Arthur Schnitzler\pend{}\pstart{}\textcolor{pink}{Wien}{}\ledrightnote{\textcolor{pink}{Wien}}\pend{}\pstart{}\textcolor{pink}{IX. Frankgaſse 1}{}\ledrightnote{\textcolor{pink}{Frankgasse}}.\pend{}
{\bigskip}
\pstart
           \centering{}{\pb}\textcolor{pink}{Baſel}{}\ledrightnote{\textcolor{pink}{Basel}}{ }31. Juli\pend
           
\pstart
           Mein lieber Freund, Kurz vor der Abreiſe nach der \textcolor{pink}{Schweiz}{}\ledrightnote{\textcolor{pink}{Schweiz}} erhielt ich Deine \label{K_L03216-1v}\edtext{l.}{\lemma{\textnormal{\emph{l.}}}\Cendnote{\textnormal{liebe}}}\label{K_L03216-1h} Karte. Da iſt ſchwer zu rathen. Aber ich meine
               doch, das \label{K_L03216-2v}\edtext{\textcolor{brown}{D.th}{}\ledrightnote{\textcolor{brown}{Deutsches Theater Berlin}}}{\lemma{\textnormal{\emph{D.th}}}\Cendnote{\textnormal{\emph{\textcolor{brown}{Deutsches Theater}}; Bezug auf die \textcolor{pink}{Berlin}er Premiere von \emph{\textcolor{green}{Der Schleier der Beatrice}}, siehe Paul Goldmann an Arthur Schnitzler, 14. 7. [1902]}}}\label{K_L03216-2h}, ſelbſt \label{K_L03216-22v}\edtext{\uline{nach}{ }\textsc{\textcolor{green}{Monna Vanna}{}\ledrightnote{\textcolor{green}{Monna Vanna. Schauspiel in drei Akten}}}}{\lemma{\textnormal{\emph{nach Monna Vanna}}}\Cendnote{\textnormal{\emph{\textcolor{green}{Der Schleier der Beatrice}} und \emph{\textcolor{green}{Monna Vanna}} haben offensichtliche
                     Parallelen, vor allem im Ort der Handlung und der zentralen Figur einer Frau
                     zwischen zwei Männern. Obzwar \textcolor{blue}{Schnitzler}s
                     Stück früher erschienen ist, war es offensichtlich eine schwierige
                     Entscheidung, ob es auch am \emph{\textcolor{brown}{Deutschen
                        Theater}} gegeben werden sollte, nachdem dort \textcolor{blue}{Maeterlinck}s{ }\textcolor{green}{Stück} am Spielplan
                     gestanden hatte.}}}\label{K_L03216-22h}
               , iſt beſſer als das \textcolor{brown}{Schillertheater}{}\ledrightnote{\textcolor{brown}{Schiller-Theater}}.\pend
           
\pstart
           Viele Grüße {\\[\baselineskip]}Dein {\\[\baselineskip]}\spacefill\mbox{P. Goldm}\pend
           \leftskip=0em{}\endnumbering\briefempfaengerindex{Schnitzler, Arthur@\textsc{Schnitzler, Arthur}!zzzGoldmann, Paul@\emph{von Paul Goldmann}!1902-07-311@{31. 7. 1902}|)be}\mylabel{h}  \normalsize

\doendnotes{C}
\bigskip
\vfill

\clearpage

\footnotesize

\lohead{\textsc{register}}

% Definiere theindex-Environment komplett neu ohne reledmac
\makeatletter
\renewenvironment{theindex}{%
  \section*{\indexname}%
  \setlength{\parindent}{0pt}%
  \setlength{\parskip}{0pt plus 0.3pt}%
  \let\item\@idxitem
}{%
  \clearpage
}
\makeatother

\IfFileExists{\jobname-pw.ind}{\input{\jobname-pw.ind}}{}

\end{document}

      