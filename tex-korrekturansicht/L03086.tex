%% latex-korrekturansicht-vorspann.tex
%% Vorspann für die Korrekturansicht.
%% Lädt die gemeinsame Datei latex-vorspann.tex mit gesetztem Schalter.

\newif\ifkorrekturansicht
\korrekturansichttrue

\input{../tex-inputs/latex-vorspann}


\renewcommand{\erwaehntePersonen}{Personen: Peter Altenberg, Richard Beer-Hofmann, Hugo von Hofmannsthal, Lotte Medelsky, Olga Schnitzler, Elisabeth Steinrück}
\renewcommand{\erwaehnteInstitutionen}{Institutionen: Deutsches Theater Berlin, Volkstheater}
\renewcommand{\erwaehnteOrte}{Orte: Berlin, Dessauer Straße, Frankreich, Pörtschach, Welsberg-Taisten, Wien}
\renewcommand{\erwaehnteWerke}{}
\section[Paul Goldmann an Arthur Schnitzler, Olga und Elisabeth Gussmann, 27. 9. {[}1901{]}]{Paul Goldmann an Arthur Schnitzler, Olga und Elisabeth
               Gussmann, 27. 9. {[}1901{]}}
\nopagebreak\mylabel{v}
\rehead{ }\normalsize\beginnumbering\briefempfaengerindex{Steinrueck, Elisabeth@\textsc{Steinrück, Elisabeth}!zzzGoldmann, Paul@\emph{von Paul Goldmann}!1901-09-272@{27. 9. {[}1901{]}}|(be}\briefempfaengerindex{Schnitzler, Olga@\textsc{Schnitzler, Olga}!zzzGoldmann, Paul@\emph{von Paul Goldmann}!1901-09-272@{27. 9. {[}1901{]}}|(be}\briefempfaengerindex{Schnitzler, Arthur@\textsc{Schnitzler, Arthur}!zzzGoldmann, Paul@\emph{von Paul Goldmann}!1901-09-272@{27. 9. {[}1901{]}}|(be}
\toendnotes[C]{\smallbreak\pagebreak[2]}\Standort{DLA, A:Schnitzler, HS.NZ85.1.3171.}
\physDesc{Brief, 2 Blätter, 5 Seiten
\newline{}Handschrift: blaue Tinte, deutsche Kurrent
\newline{}Schnitzler: mit Bleistift das Jahr »{[}1{]}901« vermerkt }\toendnotes[C]{\smallbreak}
\pstart
           \centering{}{\pb}\textcolor{pink}{Berlin}{}\ledrightnote{\textcolor{pink}{Berlin}}, 27. September.\pend
           
\pstart{}Mein lieber Freund,\pend
\pstart
           Bitte übermittle dieſen Brief an ſeine Adreſſe, da Du mir auf meine \label{K_L03086-3v}\edtext{Frage}{\lemma{\textnormal{\emph{Frage}}}\Cendnote{\textnormal{siehe Paul Goldmann an Arthur Schnitzler, 23. 9. [1901]}}}\label{K_L03086-3h}, wo die Mädeln jetzt wohnen, noch nicht geantwortet haſt.\pend
           
\pstart
           Herzlichſt Dein {\\[\baselineskip]}\spacefill\mbox{Paul Goldmn}\pend
           \leftskip=0em{}
\pstart
           \noindent{}Iſt \label{K_L03086-2v}\edtext{\textsc{\textcolor{blue}{Richard}{}\ledrightnote{\textcolor{blue}{Richard Beer-Hofmann}}}}{\lemma{\textnormal{\emph{Richard}}}\Cendnote{\textnormal{vermutlich gemeint: aus dessen
                     Sommerquartier in \textcolor{pink}{Pörtschach}, von wo er
                        Mitte September nach \textcolor{pink}{Wien}
                     zurückgekehrt war.}}}\label{K_L03086-2h} ſchon in \textcolor{pink}{Wien}{}\ledrightnote{\textcolor{pink}{Wien}}?
               \pend
           
\pstart
           \noindent{}\raggedleft{}{\pb}\textcolor{pink}{\textcolor{gray}{\textbf{DESSAUERSTRASSE 19}}}{}\ledrightnote{\textcolor{pink}{Dessauer Straße}}\pend
           
\pstart
           \centering{}\textcolor{pink}{Berlin}{}\ledrightnote{\textcolor{pink}{Berlin}}, 27. September.\pend
           
\pstart\center{}Liebes Fräulein Olga,\pend
\pstart
           Endlich eine freie halbe Stunde! Gleich hole ich mir Ihren Brief heraus aus den
               Paket, das auf meinen Pult ſich aufgehäuft hat. Sie haben mir ſo lieb geſchrieben und
               haben mir damit eine ſo große Freude gemacht! (das »Sie« iſt immer Mehrzahl und
               bedeutet, hier \textsc{Olga} und \textsc{Liesl}).
               Ich habe \textsc{Arthur} bereits gebeten, Ihnen zu danken. Da er
               dies, wie ich vorausſetze, vergeſſen hat, ſo danke ich Ihnen hier noch einmal.\pend
           
\pstart
           Liebes Fräulein \textsc{Olga} (jetzt in der Einzahl): Daß Sie keine
               Berichte über Kaiſerzuſammenkünfte leſen, thut mir leid. Erſtens war mein Bericht
               hübſch. Zweitens iſt die Nichtachtung {\pb}der Politik,
               wie ſie unter unſeren Wiener Freunden beſteht, ein Fehler. Alles Menſchliche iſt
               intereſſant; und \strikeout{ſo} eine Kaiſerzuſammenkunft bietet
               nicht weniger menſchliches Intereſſe als das erſte Auftreten von Fräulein \textcolor{blue}{\textsc{Medelsky}}{}\ledrightnote{\textcolor{blue}{Lotte Medelsky}} im \textcolor{brown}{Volkstheater}{}\ledrightnote{\textcolor{brown}{Volkstheater}}. Geſchichte betreiben unſere Freunde. Aber was iſt
               Politik Anderes, als Geſchichte, die wir miterleben! Die großen Frauen den
               Renaiſſance und in \textcolor{pink}{Frankreich}{}\ledrightnote{\textcolor{pink}{Frankreich}} haben ſich mit Politik
               immer viel beſchäftigt und haben viel davon \strikeout{verſtanden} verſtanden.\pend
           
\pstart
           Das Feuilleton\textcolor{red}{\textsuperscript{\textbf{KEY}}} von \textsc{Lesser}\textcolor{red}{\textsuperscript{\textbf{KEY}}} habe ich nicht geleſen. Er iſt
               perſönlich ein braver Menſch. Meinetwegen alſo ſoll er für \textcolor{blue}{\textsc{Altenberg}}{}\ledrightnote{\textcolor{blue}{Peter Altenberg}} ſchwärmen und ſogar für \textcolor{blue}{\textsc{Hoffmannsthal}}{}\ledrightnote{\textcolor{blue}{Hugo von Hofmannsthal}}. Von Letzterem
               werden wir im »\textcolor{brown}{Deutſchen Theater}{}\ledrightnote{\textcolor{brown}{Deutsches Theater Berlin}}« ein Versdrama\textcolor{red}{\textsuperscript{\textbf{KEY}}} zu ſehen bekommen. Ich freue mich ſchon rieſig.\pend
           
\pstart
           {\pb}Daß Ihr Vater\textcolor{red}{\textsuperscript{\textbf{KEY}}} ſich ſo abſcheulich benimmt, thut mir unendlich leid. Kann man da gar
               nichts machen? \textsc{Arthur} ſoll den Prozeß nur jedenfalls
               einleiten. Ich bedaure namentlich, daß \uline{ich} in der
               Angelegenheit ſo gar nicht zu Hilfe kommen kann. Zum Beiſpiel, wenn ich eine Million
               hätte, wäre das ſehr einfach. Bitte, warum hab’ ich keine Million?\pend
           
\pstart
           Diese neuen Kleider müſſen herrlich ſein. Beſonders, wenn ich den himmel¬ blauen
               Gürtel ſehen könnte, es thäte meinem Herzen wohl!\pend
           
\pstart
           Ich denke oft und herzlich an Sie (wieder Mehrzahl). \textcolor{pink}{\textsc{Welsberg}}{}\ledrightnote{\textcolor{pink}{Welsberg-Taisten}} liegt fern. Ich lebe wieder mein elendes
               Leben und bin unbeſchreiblich einſam in dieſer kalten \textcolor{pink}{Stadt}{}\ledrightnote{{$\rightarrow$}\textcolor{pink}{Berlin}}, in der Niemand mich mag, Keiner und Keine.\pend
           
\pstart
           {\pb}Liebes Fräulein \textsc{Olga},
               kommen Sie bald mit dem \textsc{Arthur} nach \textcolor{pink}{Berlin}{}\ledrightnote{\textcolor{pink}{Berlin}}, ſchreiben Sie mir bis dahin noch manchen lieben Brief
               und ſeien Sie herzlich gegrüßt von\pend
           
\pstart
            Ihrem ergebenen{\\[\baselineskip]}\spacefill\mbox{Dr. Paul Goldmann.}\pend
           \leftskip=0em{}\endnumbering\briefempfaengerindex{Steinrueck, Elisabeth@\textsc{Steinrück, Elisabeth}!zzzGoldmann, Paul@\emph{von Paul Goldmann}!1901-09-272@{27. 9. {[}1901{]}}|)be}\briefempfaengerindex{Schnitzler, Olga@\textsc{Schnitzler, Olga}!zzzGoldmann, Paul@\emph{von Paul Goldmann}!1901-09-272@{27. 9. {[}1901{]}}|)be}\briefempfaengerindex{Schnitzler, Arthur@\textsc{Schnitzler, Arthur}!zzzGoldmann, Paul@\emph{von Paul Goldmann}!1901-09-272@{27. 9. {[}1901{]}}|)be}\mylabel{h}
\begin{anhang}
\end{anhang}\normalsize

\doendnotes{C}
\bigskip
\vfill

\clearpage

\footnotesize

\lohead{\textsc{register}}

% Definiere theindex-Environment komplett neu ohne reledmac
\makeatletter
\renewenvironment{theindex}{%
  \section*{\indexname}%
  \setlength{\parindent}{0pt}%
  \setlength{\parskip}{0pt plus 0.3pt}%
  \let\item\@idxitem
}{%
  \clearpage
}
\makeatother

\IfFileExists{\jobname-pw.ind}{\input{\jobname-pw.ind}}{}

\end{document}

      