%% latex-korrekturansicht-vorspann.tex
%% Vorspann für die Korrekturansicht.
%% Lädt die gemeinsame Datei latex-vorspann.tex mit gesetztem Schalter.

\newif\ifkorrekturansicht
\korrekturansichttrue

\input{../tex-inputs/latex-vorspann}


\renewcommand{\erwaehntePersonen}{Personen: Karl Emil Franzos}
\renewcommand{\erwaehnteOrte}{Orte: Berlin, Bösendorferstraße 7, Wien}
\renewcommand{\erwaehnteWerke}{Werke: Anfang vom Ende, Deutsche Dichtung}
\section[Arthur Schnitzler an Karl Emil Franzos, 25. 6. 1892]{Arthur Schnitzler an Karl Emil Franzos, 25. 6. 1892}
\nopagebreak\mylabel{v}
\rehead{ }\normalsize\beginnumbering\briefempfaengerindex{Franzos, Karl Emil@\textsc{Franzos, Karl Emil}!zzzSchnitzler, Arthur@\emph{von Arthur Schnitzler}!1892-06-251@{25. 6. 1892}|(be}
\toendnotes[C]{\smallbreak\pagebreak[2]}\Standort{Wienbibliothek im Rathaus, H.I.N.-60192.}
\physDesc{Brief, 1 Blatt, 1 Seite, 208 Zeichen
\newline{}Handschrift: schwarze Tinte, deutsche Kurrent}\toendnotes[C]{\smallbreak}
\pstart
           \raggedleft{}{\pb}25/6 92\pend
           
\pstart{}Hochgeschätzter Herr,\pend
\pstart
           es wäre mir eine beſondere Ehre,  wenn Sie das beifolgende
               \label{K_L03615-1v}\edtext{\textcolor{green}{Gedicht}{}\ledrightnote{{$\rightarrow$}\textcolor{green}{Anfang vom Ende}}}{\lemma{\textnormal{\emph{Gedicht}}}\Cendnote{\textnormal{\textcolor{blue}{Arthur Schnitzler}: \emph{\textcolor{green}{Anfang vom Ende}}. In:
                     \emph{\textcolor{green}{Deutsche Dichtung}}, Bd. 12, Nr. 8,
                     15. 7. 1892, S. 192.
               }}}\label{K_L03615-1h} der Aufnahme in der »\textsc{\textcolor{green}{Deutschen Dichtung}{}\ledrightnote{\textcolor{green}{Deutsche Dichtung}}}« werth hielten. \pend
           
\pstart
           Hochachtungsvoll{\\[\baselineskip]}\spacefill\mbox{Dr. Arthur Schnitzler}\pend
           \leftskip=0em{}
\pstart
           \noindent{}\textcolor{pink}{Wien, I. \textsc{Giselastraße 11}}{}\ledrightnote{\textcolor{pink}{Bösendorferstraße 7}}.\pend
           \endnumbering\briefempfaengerindex{Franzos, Karl Emil@\textsc{Franzos, Karl Emil}!zzzSchnitzler, Arthur@\emph{von Arthur Schnitzler}!1892-06-251@{25. 6. 1892}|)be}\mylabel{h}  \normalsize

\doendnotes{C}
\bigskip
\vfill

\clearpage

\footnotesize

\lohead{\textsc{register}}

% Definiere theindex-Environment komplett neu ohne reledmac
\makeatletter
\renewenvironment{theindex}{%
  \section*{\indexname}%
  \setlength{\parindent}{0pt}%
  \setlength{\parskip}{0pt plus 0.3pt}%
  \let\item\@idxitem
}{%
  \clearpage
}
\makeatother

\IfFileExists{\jobname-pw.ind}{\input{\jobname-pw.ind}}{}

\end{document}

      