%% latex-korrekturansicht-vorspann.tex
%% Vorspann für die Korrekturansicht.
%% Lädt die gemeinsame Datei latex-vorspann.tex mit gesetztem Schalter.

\newif\ifkorrekturansicht
\korrekturansichttrue

\input{../tex-inputs/latex-vorspann}


\renewcommand{\erwaehntePersonen}{Personen: Frieda Pollak, Felix Salten, Julius Ferdinand Wollf}
\renewcommand{\erwaehnteOrte}{Orte: Hannover, Klostergang, Sternwartestraße 71, Wien}
\renewcommand{\erwaehnteWerke}{}
\section[ Felix Salten und Julius Wollf an Arthur Schnitzler, 31. 3. 1921]{Felix Salten und Julius Wollf an Arthur Schnitzler, 31. 3. 1921}
\nopagebreak\mylabel{v}
\rehead{ }\normalsize\beginnumbering\briefempfaengerindex{Schnitzler, Arthur@\textsc{Schnitzler, Arthur}!zzzWollf, Julius Ferdinand@\emph{von Julius Ferdinand Wollf}!1921-03-211@{31. 3. 1921}|(be}\briefempfaengerindex{Schnitzler, Arthur@\textsc{Schnitzler, Arthur}!zzzSalten, Felix@\emph{von Felix Salten}!1921-03-211@{31. 3. 1921}|(be}
\toendnotes[C]{\smallbreak\pagebreak[2]}\Standort{CUL, Schnitzler, B 89, B 2.}
\physDesc{Bildpostkarte, 123 Zeichen
\newline{}Handschrift Felix Salten: schwarze Tinte, lateinische Kurrent
\newline{}Handschrift Julius Ferdinand Wollf: schwarze Tinte, deutsche Kurrent
\newline{}Versand: Stempel: »\nobreak{}\oindex{Hannover@\textbf{Hannover}, \emph{P.PPLA}|pwk}Hannover, 31 3 21, 2–3 N\nobreak{}«.  
\newline{}Ordnung: 1) mit Bleistift von \textcolor{blue}{Frieda Pollak} (?) mit
                                 dem Buchstaben »A« (Abgeschrieben/Abschrift)
                                 gekennzeichnet  2) mit Bleistift von unbekannter Hand nummeriert: »284«}\pstart{}{\pb}Herrn\pend{}\pstart{}D\textsuperscript{r} Arthur Schnitzler\pend{}\pstart{}\textcolor{pink}{Wien}{}\ledrightnote{\textcolor{pink}{Wien}}\pend{}\pstart{}\textcolor{pink}{XVIII. Sternwartestraße 71}{}\ledrightnote{\textcolor{pink}{Sternwartestraße 71}}\pend{}
{\bigskip}
\pstart
           \noindent{}{\pb}\textcolor{pink}{\textcolor{gray}{\textbf{Hannover}}}{}\ledrightnote{\textcolor{pink}{Hannover}}\hfill \textcolor{pink}{\textcolor{gray}{\textbf{Klostergang}}}{}\ledrightnote{\textcolor{pink}{Klostergang}}\pend
           
\pstart
           {\pb}Alles Herzliche {\\}Ihr {\\}\spacefill\mbox{Felix Salten}\pend
           
\pstart
           \noindent{}{[}hs. Wollf:{]} Herzliche Grüße von Ihrem {\\}\spacefill\mbox{Julius F\textcolor{gray}{.} Wollf}\pend
           \endnumbering\briefempfaengerindex{Schnitzler, Arthur@\textsc{Schnitzler, Arthur}!zzzWollf, Julius Ferdinand@\emph{von Julius Ferdinand Wollf}!1921-03-211@{31. 3. 1921}|)be}\briefempfaengerindex{Schnitzler, Arthur@\textsc{Schnitzler, Arthur}!zzzSalten, Felix@\emph{von Felix Salten}!1921-03-211@{31. 3. 1921}|)be}\mylabel{h}  \normalsize

\doendnotes{C}
\bigskip
\vfill

\clearpage

\footnotesize

\lohead{\textsc{register}}

% Definiere theindex-Environment komplett neu ohne reledmac
\makeatletter
\renewenvironment{theindex}{%
  \section*{\indexname}%
  \setlength{\parindent}{0pt}%
  \setlength{\parskip}{0pt plus 0.3pt}%
  \let\item\@idxitem
}{%
  \clearpage
}
\makeatother

\IfFileExists{\jobname-pw.ind}{\input{\jobname-pw.ind}}{}

\end{document}

      