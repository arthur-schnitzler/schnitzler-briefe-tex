%% latex-korrekturansicht-vorspann.tex
%% Vorspann für die Korrekturansicht.
%% Lädt die gemeinsame Datei latex-vorspann.tex mit gesetztem Schalter.

\newif\ifkorrekturansicht
\korrekturansichttrue

\input{../tex-inputs/latex-vorspann}


               \section[Paul Goldmann an Arthur Schnitzler, 21. 3. {[}1895{]}]{ Paul Goldmann an Arthur Schnitzler, 21. 3. {[}1895{]}}\nopagebreak\mylabel{v}\rehead{ }\normalsize\beginnumbering\briefempfaengerindex{Schnitzler, Arthur@\textsc{Schnitzler, Arthur}!zzzGoldmann, Paul@\emph{von Paul Goldmann}!1895-03-211@{21. 3. {[}1895{]}}|(be} \toendnotes[C]{\smallbreak\pagebreak[2]} \Standort{DLA, A:Schnitzler, HS.NZ85.1.3165.}
\physDesc{Brief, 1 Blatt, 2 Seiten
\newline{}Handschrift: schwarze Tinte, deutsche Kurrent
\newline{}Schnitzler: 1) mit Bleistift das Jahr »95« vermerkt 2) mit rotem Buntstift eine Unterstreichung}\toendnotes[C]{\smallbreak}\pstart
           \noindent{}{\pb}\textcolor{gray}{\textbf{\textbf{\textcolor{brown}{Frankfurter Zeitung}{}\ledrightnote{\textcolor{brown}{Frankfurter Zeitung}}}}}\pend
           \pstart
           \textcolor{gray}{\textbf{(\textcolor{brown}{\begin{otherlanguage}{french}Gazette de Francfort\end{otherlanguage}}{}\ledrightnote{\textcolor{brown}{Frankfurter Zeitung}}). }}\pend
           \pstart
           \textcolor{gray}{\textbf{\textbf{\begin{otherlanguage}{french}Fondateur M. \textcolor{blue}{L.
                              Sonnemann}{}\ledrightnote{\textcolor{blue}{Leopold Sonnemann}}\end{otherlanguage}.}}}\pend
           \pstart
           \begin{otherlanguage}{french}\textcolor{gray}{\textbf{\textcolor{green}{Journal}{}\ledrightnote{\textcolor{green}{Frankfurter Zeitung}} politique, financier,}}\end{otherlanguage}\pend
           \pstart
           \begin{otherlanguage}{french}\textcolor{gray}{\textbf{commercial et littéraire.}}\end{otherlanguage}\pend
           \pstart
           \begin{otherlanguage}{french}\textcolor{gray}{\textbf{\textbf{Paraissant trois fois par jour.}}}\end{otherlanguage}\pend
           \pstart
           \begin{otherlanguage}{french}\textcolor{gray}{\textbf{\textbf{Bureau à \textcolor{pink}{Paris}{}\ledrightnote{\textcolor{pink}{Paris}}:}}}\end{otherlanguage}\pend
           \pstart
           \begin{otherlanguage}{french}\textcolor{gray}{\textbf{\textbf{\textcolor{pink}{24. Rue Feydeau}{}\ledrightnote{\textcolor{pink}{rue Feydeau}}.}}}\end{otherlanguage}\pend
           \pstart
           \centering{}\textcolor{gray}{\textbf{\label{K_L02731-11v}\edtext{\begin{otherlanguage}{french}AU JOUR LE JOUR\end{otherlanguage}}{\lemma{\textnormal{\emph{Au Jour le Jour}}}\Cendnote{\textnormal{Bei dem hier abgedruckten
                           Zeitungsausschnitt handelt es sich um eine Rekonstruktion. Auf dem
                           originalen Brief ist nur eine Klebespur mit der Rückseite des Texts von
                              \textcolor{blue}{Pierre Lalo} überliefert, die
                           belegt, dass hier ursprünglich die Besprechung angebracht war.
                              (P. L.: \emph{\textcolor{green}{Au jour le jour. M. \textcolor{blue}{Arthur Schnitzler}}}. In: \emph{\textcolor{green}{Journal des débats}},
                              Jg. 107, 21. 3. 1895, S. 1.)
                           Deutsche Übersetzung: »VON TAG ZU
                                 TAG{ / }ARTHUR
                                 SCHNITZLER{ / }Arthur Schnitzler ist einer der jüngsten
                                 Zugänge zu den Schriftstellern des jungen Deutschlands. Bisher
                                 kannte man von ihm eine Sammlung von Erzählungen und ein \textcolor{green}{dreiaktiges
                                    Stück}, in denen sich seine besonderen Qualitäten zeigten,
                                 hoben ihn aber noch nicht hervor – bis zur kürzlich erfolgten
                                 Veröffentlichtung eines Romans mit dem Titel \textcolor{green}{\emph{Sterben}} in der \textcolor{green}{\emph{Neue Deutsche Rundschau}}. Der Erfolg war sehr groß, und es scheint, dass er in jeder
                                 Hinsicht verdient ist. \textcolor{green}{\emph{Sterben}} ist ein sehr kurzer Roman oder, wenn man so will, eine lange
                                 Novelle: kaum 150 Seiten. Es gibt nur drei Personen: einen jungen
                                 Mann und eine junge Frau, Felix und Marie, die zärtlich miteinander
                                 verbunden sind, und einen Arzt. In der ersten Szene, die durch die
                                 sichere Darstellung und die Wahl der Details beeindruckt, erfährt
                                 Felix, dass er unheilbar krank ist und nur noch ein Jahr zu leben
                                 hat. Er teilt dies Marie mit, und sie ruft verzweifelt, dass sie
                                 mit ihrem Freund sterben werde. Er versucht sie zu besänftigen und
                                 ihr klarzumachen, dass sie leben muss und noch glücklich sein kann,
                                 aber sie will nicht hören{\dots} Auf den
                                 letzten Seiten des Romans, in den letzten Tagen von Felix’
                                 Krankheit, ist er es, der sie leidenschaftlich mit in den Tod
                                 nehmen möchte, und sie ist es, die leben möchte. Dieser langsame
                                 Zerfall von Gefühlen und Zuneigung ist das Thema von \textcolor{green}{\emph{Sterben}}. Stellen Sie sich vor, dieses Thema würde von einem unserer
                                 Romanautoren behandelt: Er würde zweifellos dazu neigen, die
                                 moralische Hässlichkeit seiner Figuren zu übertreiben. Bei
                                 Schnitzler gibt es nichts dergleichen: keine Exzesse, keine Gewalt,
                                 keine Brutalität; die Darstellung, so stark sie auch sein mag,
                                 behält perfekt Maß und Genauigkeit. Was in Marie vorgeht, was an
                                 unbewusster Ungeduld und Überdruss unter ihrer Zärtlichkeit und
                                 ihrem Mitleid erwacht und sich einschleicht, all das wird
                                 tiefgehend beobachtet und akzentuiert mit seltener Präzision{\dots} Wenn ich noch hinzufüge, dass die
                                 Entwicklung der Erzählung kurz und nüchtern ist, dass die
                                 Komposition nahezu klassisch ist in Befolgung von Logik, Reihung
                                 und Klarheit aufweist, habe ich genug gesagt, um den Erfolg von \textcolor{green}{\emph{Sterben}} zu erklären und zu zeigen, dass die deutsche Literatur von
                                 nun an viel von Herrn Schnitzler erhoffen darf. -
                              P. L.«}}}\label{K_L02731-11h}}}\pend
           \pstart
           \noindent{}\centering{}\textcolor{gray}{\textbf{M. ARTHUR SCHNITZLER}}\pend
           \pstart
           \noindent{}\textcolor{gray}{\textbf{\begin{otherlanguage}{french}M. Arthur Schnitzler est un des derniers venus parmi les
                     écrivains de la Jeune Allemagne. On connaissait jusqu’ici de lui un recueil de
                     nouvelles et une \textcolor{green}{pièce en
                        trois actes}{}\ledrightnote{→\textcolor{green}{Das Märchen. Schauspiel in drei Aufzügen}}, où se révélaient des qualités éminentes, mais qui ne
                     l’avaient point encore fait sortir du rang, lorsque, récemment, il publia dans
                     la \textcolor{green}{\emph{Neue Deutsche Rundschau}}{}\ledrightnote{\textcolor{green}{Neue Deutsche Rundschau}} un roman intitulé: \textcolor{green}{\emph{Sterben}}{}\ledrightnote{\textcolor{green}{Sterben. Novelle}} – \textcolor{green}{\emph{Mourir}}{}\ledrightnote{\textcolor{green}{Sterben. Novelle}}. Le succès en fut très vif; il semble bien qu’il soit de tout point
                     mérité. \textcolor{green}{\emph{Sterben}}{}\ledrightnote{\textcolor{green}{Sterben. Novelle}} est un très court roman ou, si l’on veut, une longue nouvelle: cent
                     cinquante pages à peine. Trois personnages seulement: un jeune homme et une
                     jeune femme tendrement unis, Félix et Marie, et un médecin. En la première
                     scène, singulièrement saisissante par la sûreté des traits et lé choix des
                     détails, Félix vient d’apprendre qu’il est atteint d’une maladie incurable et
                     qu’il n’a pas plus d’une année de vie: il l’annonce à Marie, et celle-ci,
                     désespérée, s'écrie qu’elle mourra avec son ami. Il s’efforce de l’apaiser, de
                     lui faire comprendre qu’elle doit vivre et qu’elle pourra encore être heureuse:
                     elle ne veut rien entendre{\dots} Aux dernières pages du
                     roman, aux derniers jours de la maladie de Félix, c’est lui quidésirera
                     passionnément l’emmener aveclui dans la mort, c’est elle qui voudra vivre.
                     Cette lente décomposition des sentiments et des affections, tel est le sujet de
                        \textcolor{green}{\emph{Sterben}}{}\ledrightnote{\textcolor{green}{Sterben. Novelle}}. Imaginez ce thème traité par un de nos romanciers: sans doute il sera
                     porté à exagérer la laideur morale de ses personnages. Rien de pareil chez
                     M.Schnitzler: aucun excès, aucune violence, aucune brutalité; la peinture, si
                     forte qu’elle soit, garde une mesure et une justesse parfaites. Ce qui se passe
                     chez Marie, ce qui s’éveille et se glisse d’inconsciente impatience et de
                     lassitude sous sa tendresse et sa pitié, tout cela est profondément observé,
                     nuancé avec une rare précision{\dots} Si j’ajoute que les
                     développements du récit sont brefs et sobres, que la composition a une logique,
                     une suite et une clarté presque classiques, j’en aurai assez dit pour expliquer
                     le succès de \textcolor{green}{\emph{Sterben}}{}\ledrightnote{\textcolor{green}{Sterben. Novelle}} et pour montrer que les lettres allemandes ont désormais le droit
                     d’attendre beaucoup de M. Schnitzler. – P. L.\end{otherlanguage}}}\pend
           \pstart
           \raggedleft{}{\pb}\textsc{\textcolor{pink}{Paris}{}\ledrightnote{\textcolor{pink}{Paris}}}, 21. März.\pend
           \pstart\center{}Mein lieber Freund,\pend\pstart
           \textsc{\textcolor{blue}{Pierre Lalo}{}\ledrightnote{\textcolor{blue}{Pierre Lalo}}} hat alſo endlich ſein \label{K_L02731-1v}\edtext{Verſprechen}{\lemma{\textnormal{\emph{Verſprechen}}}\Cendnote{\textnormal{Siehe Paul Goldmann an Arthur Schnitzler, 12. 1. [1895]}}}\label{K_L02731-1h} gehalten und hat einen ſchönen \textcolor{green}{Artikel}{}\ledrightnote{→\textcolor{green}{Au jour le jour. M. Arthur Schnitzler}} geſchrieben. Das heißt, die Schönheit des \textcolor{green}{Artikel}{}\ledrightnote{→\textcolor{green}{Au jour le jour. M. Arthur Schnitzler}}s hat natürlich nichts
               mit dem Verſprechen zu thun, ſondern mit der Schönheit Deines \textcolor{green}{Buch}{}\ledrightnote{→\textcolor{green}{Sterben. Novelle}}es, die den \textcolor{pink}{franzöſiſch}{}\ledrightnote{→\textcolor{pink}{Frankreich}}en \textcolor{blue}{Kritiker}{}\ledrightnote{→\textcolor{blue}{Pierre Lalo}} hocherfreut hat. Ich beglückwünſche
               Dich zu dem neuen Erfolge und bin recht ſtolz darauf, Dich in dem ernſteſten und
               vornehmſten \textcolor{green}{Blatte}{}\ledrightnote{→\textcolor{green}{Journal des débats. Politiques et littéraires}} der großen
                  \textcolor{pink}{Pariſ}{}\ledrightnote{\textcolor{pink}{Paris}}er Tagespreſſe an erſter Stelle in
               ſolcher Weiſe beſprochen zu ſehen.\pend
           \pstart
           {\pb}Anbei erhältſt Du einige \textcolor{green}{Exemplare}{}\ledrightnote{→\textcolor{green}{Journal des débats. Politiques et littéraires}}. Bitte ſchreibe \uline{umgehend} und recht herzlich an \textsc{\textcolor{blue}{Lalo}{}\ledrightnote{\textcolor{blue}{Pierre Lalo}}} (\textcolor{pink}{\textsc{19. Boulevard de Courcelles}}{}\ledrightnote{\textcolor{pink}{Boulevard de Courcelles}}). \pend
           \pstart
           In Treue {\\[\baselineskip]}Dein {\\[\baselineskip]}\spacefill\mbox{Paul Goldmann.}\pend
           \leftskip=0em{}\pstart
           \noindent{}Bitte, ſchick’ mir bei Gelegenheit ein Exemplar von »\textcolor{green}{\textsc{Alkandis} Lied}{}\ledrightnote{\textcolor{green}{Alkandi’s Lied}}«. Zu Progaganda-Zwecken!\pend
           \endnumbering\briefempfaengerindex{Schnitzler, Arthur@\textsc{Schnitzler, Arthur}!zzzGoldmann, Paul@\emph{von Paul Goldmann}!1895-03-211@{21. 3. {[}1895{]}}|)be}\mylabel{h}  \normalsize

\doendnotes{C}
\bigskip
\vfill

\clearpage

\footnotesize

\lohead{\textsc{register}}

% Definiere theindex-Environment komplett neu ohne reledmac
\makeatletter
\renewenvironment{theindex}{%
  \section*{\indexname}%
  \setlength{\parindent}{0pt}%
  \setlength{\parskip}{0pt plus 0.3pt}%
  \let\item\@idxitem
}{%
  \clearpage
}
\makeatother

\IfFileExists{\jobname-pw.ind}{\input{\jobname-pw.ind}}{}

\end{document}

      