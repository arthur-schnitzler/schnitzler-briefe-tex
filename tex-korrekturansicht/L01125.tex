%% latex-korrekturansicht-vorspann.tex
%% Vorspann für die Korrekturansicht.
%% Lädt die gemeinsame Datei latex-vorspann.tex mit gesetztem Schalter.

\newif\ifkorrekturansicht
\korrekturansichttrue

\input{../tex-inputs/latex-vorspann}


               \section[Arthur Schnitzler an Hugo von Hofmannsthal, 7. 6. 1901]{ Arthur Schnitzler an Hugo von Hofmannsthal, 7. 6. 1901}\nopagebreak\mylabel{v}\rehead{ }\normalsize\beginnumbering\briefempfaengerindex{Hofmannsthal, Hugo von@\textsc{Hofmannsthal, Hugo von}!zzzSchnitzler, Arthur@\emph{von Arthur Schnitzler}!1901-06-071@{7. 6. 1901}|(be} \toendnotes[C]{\smallbreak\pagebreak[2]} \Standort{FDH, Hs-30885,94.}
\physDesc{Brief, 1 Blatt, 2 Seiten
\newline{}Handschrift: schwarze Tinte, deutsche Kurrent\newline{}Ordnung: von Schnitzler mutmaßlich bei der Durchsicht der Korrespondenz
                                    1929 mit Bleistift datiert: »7/6 901« }\buchAbdrucke{\weitereDrucke{Hugo von Hofmannsthal, Arthur Schnitzler: \emph{Briefwechsel}. Hg. Therese Nickl und Heinrich Schnitzler. Frankfurt am Main: \emph{S. Fischer} 1964, S. 146.} }\toendnotes[C]{\smallbreak}\pstart{}{\pb}Mein lieber Hugo, \pend\pstart
           Sie erinnern ſich vielleicht dieſer kleinen Kaſſette oder wie Sies nennen wollen, aus
                  \textcolor{pink}{Salzburg}{}\ledrightnote{\textcolor{pink}{Salzburg}}. Ich möchte gern, daſs Sie irgendwo in
               der \textcolor{pink}{Rodauner Villa}{}\ledrightnote{\textcolor{pink}{Hofmannsthal-Schlössl}} einen Platz fänden ſie
               hinzuſtellen und ſich dabei manchmal jenes \textcolor{pink}{Salzburg}{}\ledrightnote{\textcolor{pink}{Salzburg}}er
                  \label{K_L01125_1v}\edtext{Tags beim \textcolor{brown}{\textsc{Svatek}}{}\ledrightnote{\textcolor{brown}{Wenzel Swatek}}}{\lemma{\textnormal{\emph{Tags beim Svatek}}}\Cendnote{\textnormal{siehe A. S.: \emph{Tagebuch}, 12. 8. 1900}}}\label{K_L01125_1h} erinnern; und {\pb}andrer Tage auch. Adieu alſo und auf ein ſchönes
               Wiederſehn, ſpäteſtens zu Anfang des Herbſtes.\pend
           \pstart
           Grüßen Sie \textcolor{blue}{Gerty}{}\ledrightnote{\textcolor{blue}{Gertrude von Hofmannsthal}}, ich brauche Ihnen beiden nicht
               erſt zu ſagen, wie viel Glück ich Ihnen wünſche.\pend
           \pstart
           Immer Ihr{\\[\baselineskip]}\spacefill\mbox{Arthur}\pend
           \leftskip=0em{}\pstart
           \textcolor{pink}{Wien}{}\ledrightnote{\textcolor{pink}{Wien}}{ }7. Juni 901.\pend
           \endnumbering\briefempfaengerindex{Hofmannsthal, Hugo von@\textsc{Hofmannsthal, Hugo von}!zzzSchnitzler, Arthur@\emph{von Arthur Schnitzler}!1901-06-071@{7. 6. 1901}|)be}\mylabel{h}  \normalsize

\doendnotes{C}
\bigskip
\vfill

\clearpage

\footnotesize

\lohead{\textsc{register}}

% Definiere theindex-Environment komplett neu ohne reledmac
\makeatletter
\renewenvironment{theindex}{%
  \section*{\indexname}%
  \setlength{\parindent}{0pt}%
  \setlength{\parskip}{0pt plus 0.3pt}%
  \let\item\@idxitem
}{%
  \clearpage
}
\makeatother

\IfFileExists{\jobname-pw.ind}{\input{\jobname-pw.ind}}{}

\end{document}

      