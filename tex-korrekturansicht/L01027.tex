%% latex-korrekturansicht-vorspann.tex
%% Vorspann für die Korrekturansicht.
%% Lädt die gemeinsame Datei latex-vorspann.tex mit gesetztem Schalter.

\newif\ifkorrekturansicht
\korrekturansichttrue

\input{../tex-inputs/latex-vorspann}


               \section[Arthur Schnitzler an Hugo von Hofmannsthal, 3. 4. 1900]{ Arthur Schnitzler an Hugo von Hofmannsthal, 3. 4. 1900}\nopagebreak\mylabel{v}\rehead{ }\normalsize\beginnumbering\briefempfaengerindex{Hofmannsthal, Hugo von@\textsc{Hofmannsthal, Hugo von}!zzzSchnitzler, Arthur@\emph{von Arthur Schnitzler}!1900-04-032@{3. 4. 1900}|(be} \toendnotes[C]{\smallbreak\pagebreak[2]} \Standort{FDH, Hs-30885,91.}
\physDesc{Bildpostkarte
\newline{}Handschrift: Bleistift, deutsche Kurrent\newline{}Versand: 1) Stempel: »\nobreak{}\oindex{Split@\textbf{Split}, \emph{Besiedelter Ort (A.BSO)}|pwk}Spluet – Spalato, 3 4 00\nobreak{}«.  2) Stempel: »\nobreak{}\oindex{Paris@\textbf{Paris}, \emph{Besiedelter Ort (A.BSO)}|pwk}Paris Etranger, 5\textsuperscript{E} Avril 00\nobreak{}«. \newline{}Ordnung: von unbekannter Hand datiert: »3/4 00« }\buchAbdrucke{\weitereDrucke{Hugo von Hofmannsthal, Arthur Schnitzler: \emph{Briefwechsel}. Hg. Therese Nickl und Heinrich Schnitzler. Frankfurt am Main: \emph{S. Fischer} 1964, S. 137.} }\pstart{}{\pb}\textsc{Hugo von Hofmannsthal}\pend{}\pstart{}\textcolor{pink}{\textsc{Paris}}{}\ledrightnote{\textcolor{pink}{Paris}}\pend{}\pstart{}\textcolor{pink}{\textsc{192 Boulevard Haussman}}{}\ledrightnote{\textcolor{pink}{Boulevard Haussmann}}\pend{}{\bigskip}\pstart
           \noindent{}\centering{}\textcolor{gray}{\textbf{{\pb}\textcolor{pink}{Spalato – Split. Porta aurea}{}\ledrightnote{\textcolor{pink}{Goldenes Tor}}.}}\pend
           \pstart
           lieber Hugo, ich bin in \textcolor{pink}{Trieſt}{}\ledrightnote{\textcolor{pink}{Triest}}
                    u \textcolor{pink}{Raguſa}{}\ledrightnote{\textcolor{pink}{Dubrovnik}} geweſen, habe manches merkwürdige
                    geſehen, bin heute in \textcolor{pink}{Spalato}{}\ledrightnote{\textcolor{pink}{Split}}, fahre Nachts
                    nach \textcolor{pink}{Abazia}{}\ledrightnote{\textcolor{pink}{Opatija}} und fühle, dſs ich ſehr allein
                    bin.\pend
           \pstart Ihr \spacefill\mbox{Arthur}\pend{}\endnumbering\briefempfaengerindex{Hofmannsthal, Hugo von@\textsc{Hofmannsthal, Hugo von}!zzzSchnitzler, Arthur@\emph{von Arthur Schnitzler}!1900-04-032@{3. 4. 1900}|)be}\mylabel{h}  \normalsize

\doendnotes{C}
\bigskip
\vfill

\clearpage

\footnotesize

\lohead{\textsc{register}}

% Definiere theindex-Environment komplett neu ohne reledmac
\makeatletter
\renewenvironment{theindex}{%
  \section*{\indexname}%
  \setlength{\parindent}{0pt}%
  \setlength{\parskip}{0pt plus 0.3pt}%
  \let\item\@idxitem
}{%
  \clearpage
}
\makeatother

\IfFileExists{\jobname-pw.ind}{\input{\jobname-pw.ind}}{}

\end{document}

      