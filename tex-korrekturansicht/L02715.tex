%% latex-korrekturansicht-vorspann.tex
%% Vorspann für die Korrekturansicht.
%% Lädt die gemeinsame Datei latex-vorspann.tex mit gesetztem Schalter.

\newif\ifkorrekturansicht
\korrekturansichttrue

\input{../tex-inputs/latex-vorspann}


               \section[Paul Goldmann an Arthur Schnitzler, 12. 9. 1893]{ Paul Goldmann an Arthur Schnitzler, 12. 9. 1893}\nopagebreak\mylabel{v}\rehead{ }\normalsize\beginnumbering\briefempfaengerindex{Schnitzler, Arthur@\textsc{Schnitzler, Arthur}!zzzGoldmann, Paul@\emph{von Paul Goldmann}!1893-09-121@{12. 9. 1893}|(be} \toendnotes[C]{\smallbreak\pagebreak[2]} \Standort{DLA, A:Schnitzler, HS.NZ85.1.3163.}
\physDesc{Kartenbrief
\newline{}Handschrift: schwarze Tinte, deutsche Kurrent\newline{}Versand: 1) Stempel: »\nobreak{}\oindex{Salzburg@\textbf{Salzburg}, \emph{Besiedelter Ort (A.BSO)}|pwk}Salzburg Stadt, 12/9 93, 6 A\nobreak{}«.  2) Stempel: »\nobreak{}Wien 1/1 1, {[}1{]}3/9.  93, 8–9½V\textcolor{gray}{.}, Bestellt\nobreak{}«. 
\newline{}Schnitzler: mit schwarzer Tinte das Jahr »93« vermerkt }\toendnotes[C]{\smallbreak}\pstart{}{\pb}\textcolor{gray}{\textbf{An}}\pend{}\pstart{}\textsc{Herrn}\pend{}\pstart{}\textsc{Dr. Arthur Schnitzler}\label{T_L02715-1v}\edtext{}{\lemma{\textnormal{\emph{XXXX Lemmafehler}}}\Cendnote{\textnormal{zur
                     Verdeutlichung aufgrund von Wasserflecken neuerlich der Nachname »Schnitzler« darüber geschrieben}}}\label{T_L02715-1h}\pend{}\pstart{}\textsc{\textcolor{pink}{Wien}{}\ledrightnote{\textcolor{pink}{Wien}}}\pend{}\pstart{}\textcolor{pink}{I. \textsc{Grillparzerstraße 7}}{}\ledrightnote{\textcolor{pink}{Grillparzerstraße}}.\pend{}{\bigskip}\pstart
           \noindent{}\raggedleft{}{\pb}\textsc{\textcolor{pink}{Salzburg}{}\ledrightnote{\textcolor{pink}{Salzburg}}}, 12. September\pend
           \pstart\center{}Mein lieber Freund!\pend\pstart
           Ich bin in \textsc{\textcolor{pink}{Salzburg}{}\ledrightnote{\textcolor{pink}{Salzburg}}}, \textcolor{pink}{Hotel Goldenes Horn}{}\ledrightnote{\textcolor{pink}{Hotel Goldenes Horn}}, \textcolor{pink}{Getreidemarkt}{}\ledrightnote{→\textcolor{pink}{Getreidegasse}}, und erwarte Dich mit
               Ungeduld. Bin geſtern{ }Abend angekommen und werde etwa acht Tage bleiben. Die Freude, Dich zu
               ſehen, wirſt Du mir nicht vorenthalten, nicht wahr? Nur bitte ich um vorherige
               telegraphiſche Nachricht.\pend
           \pstart
           In Treue{\\[\baselineskip]}Dein {\\[\baselineskip]}\spacefill\mbox{Paul Goldmann.}\pend
           \leftskip=0em{}\endnumbering\briefempfaengerindex{Schnitzler, Arthur@\textsc{Schnitzler, Arthur}!zzzGoldmann, Paul@\emph{von Paul Goldmann}!1893-09-121@{12. 9. 1893}|)be}\mylabel{h}  \normalsize

\doendnotes{C}
\bigskip
\vfill

\clearpage

\footnotesize

\lohead{\textsc{register}}

% Definiere theindex-Environment komplett neu ohne reledmac
\makeatletter
\renewenvironment{theindex}{%
  \section*{\indexname}%
  \setlength{\parindent}{0pt}%
  \setlength{\parskip}{0pt plus 0.3pt}%
  \let\item\@idxitem
}{%
  \clearpage
}
\makeatother

\IfFileExists{\jobname-pw.ind}{\input{\jobname-pw.ind}}{}

\end{document}

      