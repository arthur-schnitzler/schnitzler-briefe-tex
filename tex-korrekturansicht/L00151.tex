%% latex-korrekturansicht-vorspann.tex
%% Vorspann für die Korrekturansicht.
%% Lädt die gemeinsame Datei latex-vorspann.tex mit gesetztem Schalter.

\newif\ifkorrekturansicht
\korrekturansichttrue

\input{../tex-inputs/latex-vorspann}


               \section[Friedrich M. Fels an Arthur Schnitzler, {[}Ende 1892?{]}]{ Friedrich M. Fels an Arthur Schnitzler, {[}Ende 1892?{]}}\nopagebreak\mylabel{v}\rehead{ }\normalsize\beginnumbering\briefempfaengerindex{Schnitzler, Arthur@\textsc{Schnitzler, Arthur}!zzzFels, Friedrich Michael@\emph{von Friedrich Michael Fels}!1892-12-312@{{[}Ende 1892?{]}}|(be} \toendnotes[C]{\smallbreak\pagebreak[2]} \Standort{DLA, A:Schnitzler, HS.NZ85.1.2956.}
\physDesc{Briefkarte
\newline{}Handschrift: schwarze Tinte, lateinische Kurrent}\toendnotes[C]{\smallbreak}\pstart
           \noindent{}{\pb}Lieber Dr Schnitzler! Warum sind Sie heute nicht geko{\geminationm}en? Ich bin \label{K_L00151_1v}\edtext{schwach}{\lemma{\textnormal{\emph{schwach}}}\Cendnote{\textnormal{Am 20. 12. 1892 notiert
                     \textcolor{blue}{Schnitzler} erstmals nach einem Besuch von \textcolor{blue}{Fels} dessen desolaten Zustand: »der
                     beinahe hungert. – Schrecklich ist das. –«. In den folgenden Wochen
                  involvierte sich \textcolor{blue}{Schnitzler} stärker, mehrere
                  undatierte Korrespondenzstücke dürften in der Zeit, bis der Kranke Mitte
                     Februar 1893 nach \textcolor{pink}{Meran} abreiste, zu
                  verorten sein. Nur teilweise lassen sich implizite Reihungen vornehmen.}}}\label{K_L00151_1h},
               weil ich gestern den ganzen Nachmittag vom Durchfall geplagt war. Deshalb ka{\geminationn} ich nicht zu Ihnen ko{\geminationm}en.
               Bitte dem \textcolor{blue}{Boten}{}\ledrightnote{→\textcolor{blue}{?? [Bote von Friedrich M. Fels]}} etwas Geld
               mitzugeben; ich brauche zum Leben, für Schneider, Schuster, Hutmacher; der \textcolor{blue}{Bote}{}\ledrightnote{→\textcolor{blue}{?? [Bote von Friedrich M. Fels]}} ist \uline{ganz sicher}, der Sohn meines \textcolor{blue}{Hauswirts}{}\ledrightnote{→\textcolor{blue}{?? [Vermieter von F. M. Fels]}} – können ihm also die \uline{gröſte}{ }Su{\geminationm}e mitgeben. Ich
               sitze \label{K_L00151_2v}\edtext{NB}{\lemma{\textnormal{\emph{NB}}}\Cendnote{\textnormal{\textcolor{blue}{Fels} nutzt die Abkürzung »NB«, »notabene« in
                  der Bedeutung von »übrigens«.}}}\label{K_L00151_2h} ohne alles hier; nicht einmal die Cigarette
                  {\pb}die ich rauche ist bezahlt. NB. Bitte um Adreſse
               (genaue) von \textcolor{blue}{Beer-Hofma{\geminationn}}{}\ledrightnote{\textcolor{blue}{Richard Beer-Hofmann}} u. \textcolor{blue}{Loris}{}\ledrightnote{\textcolor{blue}{Hugo von Hofmannsthal}}.\pend
           \pstart
           H. {\\[\baselineskip]}\spacefill\mbox{Fels}\pend
           \leftskip=0em{}\endnumbering\briefempfaengerindex{Schnitzler, Arthur@\textsc{Schnitzler, Arthur}!zzzFels, Friedrich Michael@\emph{von Friedrich Michael Fels}!1892-12-312@{{[}Ende 1892?{]}}|)be}\mylabel{h}  \normalsize

\doendnotes{C}
\bigskip
\vfill

\clearpage

\footnotesize

\lohead{\textsc{register}}

% Definiere theindex-Environment komplett neu ohne reledmac
\makeatletter
\renewenvironment{theindex}{%
  \section*{\indexname}%
  \setlength{\parindent}{0pt}%
  \setlength{\parskip}{0pt plus 0.3pt}%
  \let\item\@idxitem
}{%
  \clearpage
}
\makeatother

\IfFileExists{\jobname-pw.ind}{\input{\jobname-pw.ind}}{}

\end{document}

      