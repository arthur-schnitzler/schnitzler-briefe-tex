%% latex-korrekturansicht-vorspann.tex
%% Vorspann für die Korrekturansicht.
%% Lädt die gemeinsame Datei latex-vorspann.tex mit gesetztem Schalter.

\newif\ifkorrekturansicht
\korrekturansichttrue

\input{../tex-inputs/latex-vorspann}


               \section[Paul Goldmann, Eva Goldmann und Georg Brandes an Arthur Schnitzler, 26. 7. 1909]{ Paul Goldmann, Eva Goldmann und Georg Brandes an Arthur Schnitzler,
               26. 7. 1909}\nopagebreak\mylabel{v}\rehead{ }\normalsize\beginnumbering\briefempfaengerindex{Schnitzler, Arthur@\textsc{Schnitzler, Arthur}!zzzBrandes, Georg@\emph{von Georg Brandes}!1909-07-261@{26. 7. 1909}|(be}\briefempfaengerindex{Schnitzler, Arthur@\textsc{Schnitzler, Arthur}!zzzGoldmann, Eva Marie@\emph{von Eva Marie Goldmann}!1909-07-261@{26. 7. 1909}|(be}\briefempfaengerindex{Schnitzler, Arthur@\textsc{Schnitzler, Arthur}!zzzGoldmann, Paul@\emph{von Paul Goldmann}!1909-07-261@{26. 7. 1909}|(be} \toendnotes[C]{\smallbreak\pagebreak[2]} \Standort{DLA, A:Schnitzler, HS.NZ85.1.3175.}
\physDesc{Bildpostkarte
\newline{}Handschrift Paul Goldmann: 1) blaue Tinte, deutsche Kurrent\hspace{1em}2) blaue Tinte, lateinische Kurrent (\noindent{}Adresse)\hspace{1em}\newline{}Handschrift Eva Marie Goldmann: blaue Tinte\newline{}Handschrift Georg Brandes: blaue Tinte\newline{}Versand: 1) Stempel: »\nobreak{}\oindex{Karlsbad@\textbf{Karlsbad}, \emph{Besiedelter Ort (A.BSO)}|pwk}Karlsbad 1, 26. VII. 0\textcolor{gray}{9}, 4\nobreak{}«.  2) von unbekannter Hand mit blauer Tinte die Straßenangabe der
                                 Adresszeile gestrichen und darüber ergänzt: »\noindent{}\textcolor{pink}{\textsc{Edlach NÖest}}{ / }\textcolor{pink}{\textsc{Edlacherhof}}« 3) Stempel: »\nobreak{}\oindex{Payerbach@\textbf{Payerbach}, \emph{Besiedelter Ort (A.BSO)}|pwk}{[}P{]}ayerbach, 27. VII. \textcolor{gray}{0}9, 4\nobreak{}«. 
\newline{}Schnitzler: mit Bleistift Vermerk: »\textsc{Goldma{\geminationn}}« }\buchAbdrucke{\weitereDrucke{Georg Brandes, Arthur Schnitzler: \emph{Ein Briefwechsel}. Hg. Kurt Bergel. Bern: \emph{Francke} 1956, S. 198.} }\pstart{}{\pb}Herrn\pend{}\pstart{}Dr. Arthur Schnitzler\pend{}\pstart{}\textcolor{pink}{Wien}{}\ledrightnote{\textcolor{pink}{Wien}}\pend{}\pstart{}\textcolor{pink}{XVIII. Spöttelgaſse 7}{}\ledrightnote{\textcolor{pink}{Edmund-Weiß-Gasse}}.\pend{}{\bigskip}\pstart
           \noindent{}\centering{}{\pb}\textcolor{gray}{\textbf{\textcolor{pink}{Karlsbad}{}\ledrightnote{\textcolor{pink}{Karlsbad}} Sprudel}}\pend
           \pstart
           {\pb}26. 7. 09{ }\textcolor{pink}{Karlsbad}{}\ledrightnote{\textcolor{pink}{Karlsbad}}\pend
           \pstart{}Lieber Freund,\pend\pstart
           Wir ſitzen hier beiſammen, gedenken Deiner in Freundſchaft u ſenden Dir herzliche
               Grüße.\pend
           \pstart
           \spacefill\mbox{Paul Goldmann}{\\[\baselineskip]}\spacefill\mbox{{[}hs. Goldmann:{]} EvaGoldmann}{\\[\baselineskip]}\spacefill\mbox{{[}hs. Brandes:{]} Georg Brandes}\pend
           \leftskip=0em{}\endnumbering\briefempfaengerindex{Schnitzler, Arthur@\textsc{Schnitzler, Arthur}!zzzBrandes, Georg@\emph{von Georg Brandes}!1909-07-261@{26. 7. 1909}|)be}\briefempfaengerindex{Schnitzler, Arthur@\textsc{Schnitzler, Arthur}!zzzGoldmann, Eva Marie@\emph{von Eva Marie Goldmann}!1909-07-261@{26. 7. 1909}|)be}\briefempfaengerindex{Schnitzler, Arthur@\textsc{Schnitzler, Arthur}!zzzGoldmann, Paul@\emph{von Paul Goldmann}!1909-07-261@{26. 7. 1909}|)be}\mylabel{h}\begin{anhang}\end{anhang}\normalsize

\doendnotes{C}
\bigskip
\vfill

\clearpage

\footnotesize

\lohead{\textsc{register}}

% Definiere theindex-Environment komplett neu ohne reledmac
\makeatletter
\renewenvironment{theindex}{%
  \section*{\indexname}%
  \setlength{\parindent}{0pt}%
  \setlength{\parskip}{0pt plus 0.3pt}%
  \let\item\@idxitem
}{%
  \clearpage
}
\makeatother

\IfFileExists{\jobname-pw.ind}{\input{\jobname-pw.ind}}{}

\end{document}

      