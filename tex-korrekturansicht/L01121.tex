%% latex-korrekturansicht-vorspann.tex
%% Vorspann für die Korrekturansicht.
%% Lädt die gemeinsame Datei latex-vorspann.tex mit gesetztem Schalter.

\newif\ifkorrekturansicht
\korrekturansichttrue

\input{../tex-inputs/latex-vorspann}


               \section[Georg Brandes an Arthur Schnitzler, 17. 5. 1901]{ Georg Brandes an Arthur Schnitzler, 17. 5. 1901}\nopagebreak\mylabel{v}\rehead{ }\normalsize\beginnumbering\briefempfaengerindex{Schnitzler, Arthur@\textsc{Schnitzler, Arthur}!zzzBrandes, Georg@\emph{von Georg Brandes}!1901-05-171@{17. 5. 1901}|(be} \toendnotes[C]{\smallbreak\pagebreak[2]} \Standort{CUL, Schnitzler, B 17.}
\physDesc{Brief, 1 Blatt, 2 Seiten
\newline{}Handschrift: schwarze Tinte, lateinische Kurrent\newline{}Ordnung: 1) mit Bleistift von unbekannter Hand nummeriert: »\strikeout{22}« 2) mit Bleistift von unbekannter Hand nummeriert: »23«}\buchAbdrucke{\weitereDrucke{Georg Brandes, Arthur Schnitzler: \emph{Ein Briefwechsel}. Hg. Kurt Bergel. Bern: \emph{Francke} 1956, S. 87.} }\pstart
           \noindent{}\centering{}{\pb}\textcolor{pink}{Abbazia}{}\ledrightnote{\textcolor{pink}{Opatija}}{\\}\textcolor{pink}{Hotel Quitta}{}\ledrightnote{\textcolor{pink}{Pension Quitta}}\pend
           \pstart
           \raggedleft{}17 May 1901\pend
           \pstart{}Verehrter Freund\pend\pstart
           Anbei die 30 Gulden. Hier ist überzogen, durchaus nicht sehr warm und ich durch
                    die Unpünktlichkeit meiner Mitmenschen völlig allein, mindestens \uline{zwei} Tage, worüber ich wüthe, da ich diese zwei Tage
                    vorzüglich in \textcolor{pink}{Wien}{}\ledrightnote{\textcolor{pink}{Wien}} zugebracht haben könnte,
                    während ich mich hier über die verlorene Zeit ärgere.\pend
           \pstart
           Der Weg von \textcolor{pink}{Mattuglie}{}\ledrightnote{\textcolor{pink}{Matulji}} nach \textcolor{pink}{Abbazia}{}\ledrightnote{\textcolor{pink}{Opatija}} erinnert ein wenig an den von \textcolor{pink}{Taormina}{}\ledrightnote{\textcolor{pink}{Taormina}} nach \textcolor{pink}{Giardini}{}\ledrightnote{\textcolor{pink}{Giardini Naxos}}.
                    Hier blühen die Rosen, nur nicht die meinen.\pend
           \pstart
           Haben Sie aufrichtigen und herzlichen Dank für alle mir erwiesenen Dienste. Ich,
                    der ich selbst überlaufen werde, weiss {\pb}was es heisst, dass Jemand
                    plötzlich kommt und uns die Zeit raubt.\hspace*{1.5em}Nur
                    unsere alte Freundschaft macht die Sache etwas leidlicher.\pend
           \pstart
           Nun erfuhr ich gar nicht, was \textcolor{blue}{Beer-Hofmann}{}\ledrightnote{\textcolor{blue}{Richard Beer-Hofmann}}
                    vorhat, und das interessirt mich doch lebhaft. Das ist die Folge
                    jugendlich-seniler Schwatzhaftigkeit, dass die Anderen nicht zu Worte
                    kommen.\pend
           \pstart
           Auf Wiedersehen in 14 Tagen.\pend
           \pstart
           Ihr{\\[\baselineskip]}\spacefill\mbox{Georg Brandes}\pend
           \leftskip=0em{}\endnumbering\briefempfaengerindex{Schnitzler, Arthur@\textsc{Schnitzler, Arthur}!zzzBrandes, Georg@\emph{von Georg Brandes}!1901-05-171@{17. 5. 1901}|)be}\mylabel{h}  \normalsize

\doendnotes{C}
\bigskip
\vfill

\clearpage

\footnotesize

\lohead{\textsc{register}}

% Definiere theindex-Environment komplett neu ohne reledmac
\makeatletter
\renewenvironment{theindex}{%
  \section*{\indexname}%
  \setlength{\parindent}{0pt}%
  \setlength{\parskip}{0pt plus 0.3pt}%
  \let\item\@idxitem
}{%
  \clearpage
}
\makeatother

\IfFileExists{\jobname-pw.ind}{\input{\jobname-pw.ind}}{}

\end{document}

      