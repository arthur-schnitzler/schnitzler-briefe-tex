%% latex-korrekturansicht-vorspann.tex
%% Vorspann für die Korrekturansicht.
%% Lädt die gemeinsame Datei latex-vorspann.tex mit gesetztem Schalter.

\newif\ifkorrekturansicht
\korrekturansichttrue

\input{../tex-inputs/latex-vorspann}


               \section[Wilhelm Bölsche an Arthur Schnitzler, 17. 9. 1890]{ Wilhelm Bölsche an Arthur Schnitzler, 17. 9. 1890}\nopagebreak\mylabel{v}\rehead{ }\normalsize\beginnumbering\briefempfaengerindex{Schnitzler, Arthur@\textsc{Schnitzler, Arthur}!zzzBoelsche, Wilhelm@\emph{von Wilhelm Bölsche}!1890-09-171@{17. 9. 1890}|(be} \toendnotes[C]{\smallbreak\pagebreak[2]} \Standort{TMW, HS Schn 1/63/1.}
\physDesc{Brief, 1 Blatt, 2 Seiten
\newline{}Handschrift: schwarze Tinte, deutsche Kurrent
\newline{}Schnitzler: mit rotem Buntstift nummeriert: »1« }\toendnotes[C]{\smallbreak}\pstart
           \noindent{}\centering{}{\pb}\textcolor{gray}{\textbf{\textsc{\textcolor{green}{Freie Bühne}{}\ledrightnote{\textcolor{green}{Freie Bühne für modernes Leben}}}}}\pend
           \pstart
           \noindent{}\centering{}\textcolor{gray}{\textbf{\textsc{für modernes Leben.}}}\pend
           \pstart
           \noindent{}\centering{}\textcolor{gray}{\textbf{\textsc{Herausgegeben von \textcolor{blue}{\textbf{Otto Brahm}}{}\ledrightnote{\textcolor{blue}{Otto Brahm}}.}}}\pend
           \pstart
           \noindent{}\textcolor{gray}{\textbf{Verlag und Expedition: \textcolor{brown}{S. Fischer}{}\ledrightnote{\textcolor{brown}{S. Fischer Verlag}}.}}\pend
           \pstart
           \textcolor{gray}{\textbf{Sprechstunden: Mittwoch und Freitag 12–2 Uhr.}}\pend
           \pstart
           \textcolor{gray}{\textbf{Alle für die Redaction bestimmten Sendungen (Beiträge,
                            Recensions-Exempl.) bitten wir \textbf{ohne Angabe eines
                                Personennamens} an die Redaction der Wochenschrift »\textcolor{green}{\so{Freie Bühn}}{}\ledrightnote{\textcolor{green}{Freie Bühne für modernes Leben}}e« \textcolor{pink}{Berlin W. Link-Strasse 25}{}\ledrightnote{\textcolor{pink}{Linkstraße}} zu
                            addressiren.}}\pend
           \pstart
           \textcolor{gray}{\textbf{Wir ersuchen unsere geehrten Mitarbeiter, jedes
                            Manuscript auf der ersten Seite mit ihrer genauen Adresse zu
                            versehen.}}\pend
           \pstart
           \raggedleft{}\textcolor{gray}{\textbf{\textsc{\textcolor{pink}{Berlin}{}\ledrightnote{\textcolor{pink}{Berlin}}}, den}}{ }17. IX. \textcolor{gray}{\textbf{189}}0.\pend
           \pstart
           \noindent{}\raggedleft{}\textcolor{gray}{\textbf{\textcolor{pink}{W. Link-Straße 25}{}\ledrightnote{\textcolor{pink}{Linkstraße}}.}}\pend
           \pstart\center{}Hochgeehrter Herr Doktor!\pend\pstart
           Ihre dramatiſche \textcolor{green}{Skizze}{}\ledrightnote{→\textcolor{green}{Aus der Kaffeehausecke}} habe
                    ich mit Intereſſe geleſen, kann mich aber doch nicht recht mit ihr befreunden.
                    Der Grundgedanke iſt originell, aber der Dialog ſagt mir nicht zu. Bei breiterer
                    Ausmalung würde man an den Fall glauben, – ſo grell nicht! Es iſt eben eine
                    verzweifelt ſchwere Sache um ſolche Skizzen. Doch bitte ich recht ſehr,
                    gelegentlich etwas anderes einzuſenden.\pend
           \pstart
           Mit vorzüglicher Hochachtung{\\[\baselineskip]}\spacefill\mbox{Wilhelm Bölsche.}\pend
           \leftskip=0em{}\endnumbering\briefempfaengerindex{Schnitzler, Arthur@\textsc{Schnitzler, Arthur}!zzzBoelsche, Wilhelm@\emph{von Wilhelm Bölsche}!1890-09-171@{17. 9. 1890}|)be}\mylabel{h}  \normalsize

\doendnotes{C}
\bigskip
\vfill

\clearpage

\footnotesize

\lohead{\textsc{register}}

% Definiere theindex-Environment komplett neu ohne reledmac
\makeatletter
\renewenvironment{theindex}{%
  \section*{\indexname}%
  \setlength{\parindent}{0pt}%
  \setlength{\parskip}{0pt plus 0.3pt}%
  \let\item\@idxitem
}{%
  \clearpage
}
\makeatother

\IfFileExists{\jobname-pw.ind}{\input{\jobname-pw.ind}}{}

\end{document}

      