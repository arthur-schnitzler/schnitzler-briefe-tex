%% latex-korrekturansicht-vorspann.tex
%% Vorspann für die Korrekturansicht.
%% Lädt die gemeinsame Datei latex-vorspann.tex mit gesetztem Schalter.

\newif\ifkorrekturansicht
\korrekturansichttrue

\input{../tex-inputs/latex-vorspann}


               \section[Arthur Schnitzler an Richard Beer-Hofmann, 29. 7. 1896]{ Arthur Schnitzler an Richard Beer-Hofmann,
               29. 7. 1896}\nopagebreak\mylabel{v}\rehead{ }\normalsize\beginnumbering\briefempfaengerindex{Beer-Hofmann, Richard@\textsc{Beer-Hofmann, Richard}!zzzSchnitzler, Arthur@\emph{von Arthur Schnitzler}!1896-07-291@{29. 7. 1896}|(be} \toendnotes[C]{\smallbreak\pagebreak[2]} \Standort{YCGL, MSS 31.}
\physDesc{Brief, 1 Blatt, 2 Seiten, Umschlag
\newline{}Handschrift: Bleistift, deutsche Kurrent\newline{}Versand: 1) Stempel: »\nobreak{}\oindex{Stockholm@\textbf{Stockholm}, \emph{Besiedelter Ort (A.BSO)}|pwk}Stockholm, 29 7 96\nobreak{}«.  2) Stempel: »\nobreak{}\oindex{Kopenhagen@\textbf{Kopenhagen}, \emph{Besiedelter Ort (A.BSO)}|pwk}Kjøbenhavn, 30. 7. 96, 20 MB\nobreak{}«. }\buchAbdrucke{\weitereDrucke{Arthur Schnitzler, Richard Beer-Hofmann: \emph{Briefwechsel 1891–1931}. Hg. Konstanze Fliedl. Wien, Zürich: \emph{Europaverlag} 1992, S. 94.} }\pstart{}{\pb}Herrn \textsc{Dr. Richard
                     Beer-Hofmann}\pend{}\pstart{}\textcolor{pink}{\textsc{Kopenhagen}}{}\ledrightnote{\textcolor{pink}{Kopenhagen}}\pend{}\pstart{}\textcolor{pink}{\textsc{Hotel König von
                     Dänemark}}{}\ledrightnote{\textcolor{pink}{Hotel König von Dänemark}}\pend{}{\bigskip}\pstart
           \raggedleft{}{\pb}\textcolor{pink}{Stockholm}{}\ledrightnote{\textcolor{pink}{Stockholm}}{ }29/7 96. 6 Uhr Nm\pend
           \pstart
           Lieber Richard, finde eben Ihren Brief. Ich bleibe hier bis
                  Freitag{ }Abend,
                  31., fahre am Abend nach \textcolor{pink}{Gothenburg}{}\ledrightnote{\textcolor{pink}{Göteborg}}, bin dort Samſtag{ }\strikeout{(\introOben{}am\introOben{} nächſt} fahre
                     So{\geminationn}tag früh nach \textcolor{pink}{\textsc{Kopenhagen}}{}\ledrightnote{\textcolor{pink}{Kopenhagen}}, bin Abends in \textcolor{pink}{\textsc{Kopenhagen}}{}\ledrightnote{\textcolor{pink}{Kopenhagen}}. Gibts was neues, ſo kann ich Nachricht von Ihnen, wohl Telegramm ſpäteſtens
                  Freitag{ }\introOben{}Nach-\introOben{}Mittag hieher ins \textcolor{pink}{\textsc{Grand Hotel}}{}\ledrightnote{\textcolor{pink}{Grand Hotel Stockholm}} empfangen. Erfahre ich nichts weitres, ſo nehme ich an, dſs Sie mich in Ihrem
               Hotel in \textcolor{pink}{K.}{}\ledrightnote{\textcolor{pink}{Kopenhagen}}{ }So{\geminationn}tag Abend wiſſen laſſen, wo Sie zu
               finden (Wahrſcheinlich ſteig ich {\pb}auch dort
               ab.) Vielleicht geht doch \textcolor{pink}{\textsc{Skotsborg}}{}\ledrightnote{\textcolor{pink}{Skodsborg}}, wäre mir ſympathiſcher – im übrigen wie Sie wollen. Muſs jedenfalls noch
               8 Tage ſehr fleißig arbeiten. Dem \textcolor{blue}{Paul}{}\ledrightnote{\textcolor{blue}{Paul Goldmann}} hab
               ich auch nur ſchreiben können, \textcolor{pink}{\textsc{Kopenhagen}}{}\ledrightnote{\textcolor{pink}{Kopenhagen}} u dann wahrſcheinlich \textcolor{pink}{\textsc{Skottsborg}}{}\ledrightnote{\textcolor{pink}{Skodsborg}} – wir werden einander wohl nicht verfehlen. Vergeſſen Sie Vornamen auf Telegr.
               nicht – es läuft hier noch ein \textcolor{blue}{Schnitzler}{}\ledrightnote{\textcolor{blue}{Schnitzler}} mit
               einer Frau \textcolor{blue}{A. Schnitzler}{}\ledrightnote{\textcolor{blue}{A. Schnitzler}} herum, der
               wahrſcheinlich die meiſten meiner Briefe bekommt. Freue mich ſehr auf Wiederſehen\pend
           \pstart Herzlich Ihr \spacefill\mbox{Arthur}\pend{}\endnumbering\briefempfaengerindex{Beer-Hofmann, Richard@\textsc{Beer-Hofmann, Richard}!zzzSchnitzler, Arthur@\emph{von Arthur Schnitzler}!1896-07-291@{29. 7. 1896}|)be}\mylabel{h}  \normalsize

\doendnotes{C}
\bigskip
\vfill

\clearpage

\footnotesize

\lohead{\textsc{register}}

% Definiere theindex-Environment komplett neu ohne reledmac
\makeatletter
\renewenvironment{theindex}{%
  \section*{\indexname}%
  \setlength{\parindent}{0pt}%
  \setlength{\parskip}{0pt plus 0.3pt}%
  \let\item\@idxitem
}{%
  \clearpage
}
\makeatother

\IfFileExists{\jobname-pw.ind}{\input{\jobname-pw.ind}}{}

\end{document}

      