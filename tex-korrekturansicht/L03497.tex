%% latex-korrekturansicht-vorspann.tex
%% Vorspann für die Korrekturansicht.
%% Lädt die gemeinsame Datei latex-vorspann.tex mit gesetztem Schalter.

\newif\ifkorrekturansicht
\korrekturansichttrue

\input{../tex-inputs/latex-vorspann}


\renewcommand{\erwaehntePersonen}{Personen: Felix Salten, Ottilie Salten, Michael Emil Salzmann, Marie Salzmann, Olga Schnitzler}
\renewcommand{\erwaehnteOrte}{Orte: Bayreuth, Franzensbad, Noordwijk, Salzkammergut, Wien}
\renewcommand{\erwaehnteWerke}{}
\section[ Felix Salten an Arthur Schnitzler, 5. 7. 1908]{Felix Salten an Arthur Schnitzler, 5. 7. 1908}
\nopagebreak\mylabel{v}
\rehead{ }\normalsize\beginnumbering\briefempfaengerindex{Schnitzler, Arthur@\textsc{Schnitzler, Arthur}!zzzSalten, Felix@\emph{von Felix Salten}!1908-07-051@{5. 7. 1908}|(be}
\toendnotes[C]{\smallbreak\pagebreak[2]}\Standort{CUL, Schnitzler, B 89, B 1.}
\physDesc{Brief, 1 Blatt, 1 Seite, 1268 Zeichen (Briefpapier mit Trauerrand)
\newline{}Handschrift: schwarze Tinte, lateinische Kurrent
\newline{}Schnitzler: mit Bleistift Vermerk: »\textsc{Salten}« 
\newline{}Ordnung: mit Bleistift von unbekannter Hand nummeriert: »246« }\toendnotes[C]{\smallbreak}
\pstart
           \raggedleft{}{\pb}\textcolor{pink}{Wien}{}\ledrightnote{\textcolor{pink}{Wien}}, 5. Juli 08\pend
           
\pstart
           Lieber, vielen Dank für Ihren \label{K_L03497-1v}\edtext{teilnehmenden Brief}{\lemma{\textnormal{\emph{teilnehmenden Brief}}}\Cendnote{\textnormal{Arthur Schnitzler an Felix Salten, 29. 6. 1908}}}\label{K_L03497-1h}, und danken Sie, bitte, auch Ihrer \label{K_L03497-2v}\edtext{l.}{\lemma{\textnormal{\emph{l.}}}\Cendnote{\textnormal{lieben}}}\label{K_L03497-2h}{ }\textcolor{blue}{Frau}{}\ledrightnote{{$\rightarrow$}\textcolor{blue}{Olga Schnitzler}} für ihre Teilnahme. Mein
               armer \textcolor{blue}{Bruder}{}\ledrightnote{{$\rightarrow$}\textcolor{blue}{Michael Emil Salzmann}} hat uns bis vor
               etwa vier Wochen noch immer Hoffnung gelaßen. Sein Befinden war schwankend, aber
               nicht verzweifelt. Gelitten hat er in den ganzen fünfzehn Monaten beständig, mehr als
               sich sagen läßt. Dann aber begann ganz plötzlich das sterben und dauerte mit allen
               Qualen, die sich nur denken laßen, vier Wochen lang. Ich war viel, namentlich aber in
               den allerletzten Stunden bei ihm, und habe versucht, ihm durch fortwährendes Morphium
               wenigstens einen Teil seiner ungemein frischen Besinnung zu nehmen. Was für eine
               Krankheit ihn weggenommen hat von uns, das wissen wir noch immer nicht. Aber \uline{jetzt} braucht man’s auch nicht mehr wißen. Ich bin
               jetzt an den Nerven wieder total herunter und von meinen Darmzuständen in peinigender
               Weise, mehr als je, heimgesucht. Trotz alledem muß ich sehr viel arbeiten, und muß
               den ganzen Sommer an die Arbeit wenden. Wir reisen Mittwoch{ }früh nach \textcolor{pink}{Noordwijk}{}\ledrightnote{\textcolor{pink}{Noordwijk}}, wo wir Sonntag eintreffen und bis 15. August bleiben. \textcolor{blue}{Otti}{}\ledrightnote{\textcolor{blue}{Ottilie Salten}} geht von
               dort nach \textcolor{pink}{Franzensbad}{}\ledrightnote{\textcolor{pink}{Franzensbad}}; ich über \textcolor{pink}{Baireuth}{}\ledrightnote{\textcolor{pink}{Bayreuth}} und \textcolor{pink}{Salzkammergut}{}\ledrightnote{\textcolor{pink}{Salzkammergut}}
               nach \textcolor{pink}{Wien}{}\ledrightnote{\textcolor{pink}{Wien}}.\pend
           
\pstart
           Nochmals vielen Dank Ihnen \textcolor{blue}{Beiden}{}\ledrightnote{{$\rightarrow$}\textcolor{blue}{Olga Schnitzler}}, auch meine \textcolor{blue}{Mama}{}\ledrightnote{{$\rightarrow$}\textcolor{blue}{Marie Salzmann}} dankt Ihnen vielmals. Schönste Grüße von uns zu Ihnen. {\\[\baselineskip]}Ihr
                  \spacefill\mbox{Salten}\pend
           \leftskip=0em{}\endnumbering\briefempfaengerindex{Schnitzler, Arthur@\textsc{Schnitzler, Arthur}!zzzSalten, Felix@\emph{von Felix Salten}!1908-07-051@{5. 7. 1908}|)be}\mylabel{h}  \normalsize

\doendnotes{C}
\bigskip
\vfill

\clearpage

\footnotesize

\lohead{\textsc{register}}

% Definiere theindex-Environment komplett neu ohne reledmac
\makeatletter
\renewenvironment{theindex}{%
  \section*{\indexname}%
  \setlength{\parindent}{0pt}%
  \setlength{\parskip}{0pt plus 0.3pt}%
  \let\item\@idxitem
}{%
  \clearpage
}
\makeatother

\IfFileExists{\jobname-pw.ind}{\input{\jobname-pw.ind}}{}

\end{document}

      