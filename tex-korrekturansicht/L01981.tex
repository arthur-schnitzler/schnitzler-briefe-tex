%% latex-korrekturansicht-vorspann.tex
%% Vorspann für die Korrekturansicht.
%% Lädt die gemeinsame Datei latex-vorspann.tex mit gesetztem Schalter.

\newif\ifkorrekturansicht
\korrekturansichttrue

\input{../tex-inputs/latex-vorspann}


               \section[Arthur Schnitzler an Hermann Bahr, 17. 11. 1910]{ Arthur Schnitzler an Hermann Bahr, 17. 11. 1910}\nopagebreak\mylabel{v}\rehead{ }\normalsize\beginnumbering\briefempfaengerindex{Bahr, Hermann@\textsc{Bahr, Hermann}!zzzSchnitzler, Arthur@\emph{von Arthur Schnitzler}!1910-11-171@{17. 11. 1910}|(be} \toendnotes[C]{\smallbreak\pagebreak[2]} \Standort{TMW, HS AM 23392 Ba.}
\physDesc{Brief, 2 Blätter (Nummerierung des zweiten Blattes mit Schreibmaschine: »3«), 4 Seiten
\newline{}Schreibmaschine
\newline{}Handschrift: schwarze Tinte, deutsche Kurrent (\noindent{}Korrekturen, Schlussformel, Unterschschrift)}\Standort{DLA, A:Schnitzler, 85.1.294/3.}
\physDesc{Brief, maschineller Durchschlag
\newline{}Schreibmaschine
\newline{}Handschrift: Bleistift, deutsche Kurrent (\noindent{}Korrekturen im letzten Absatz und Schlussformel: »Mit v. fr.
                                    Grüßen / Dein / A«)}\buchAbdrucke{\weitereDrucke{1) Arthur Schnitzler: \emph{Briefe 1875–1912}. Hg. Therese Nickl und Heinrich Schnitzler. Frankfurt am Main: \emph{S. Fischer} 1981, S. 633–635.} \weitereDrucke{2) \emph{17. 11. 1910.} In: Arthur Schnitzler: \emph{The Letters of Arthur Schnitzler to Hermann Bahr}. Edited, annotated, and with an introduction, by Donald G.
                        Daviau. Chapel Hill: \emph{The University of North Carolina Press} 1978, S. 106–108 (University of North Carolina studies in the Germanic languages
                        and literatures, 89).} \weitereDrucke{3) Hermann Bahr, Arthur Schnitzler: \emph{Briefwechsel, Aufzeichnungen, Dokumente (1891–1931)}. Hg. Kurt Ifkovits und Martin Anton Müller. Göttingen: \emph{Wallstein} 2018, S. 443–445.} }\toendnotes[C]{\smallbreak}\pstart
           \noindent{}{\pb}\textcolor{gray}{\textbf{Dr. Arthur Schnitzler}}\hfill 17. 11. 1910.\pend
           \pstart
           \textcolor{gray}{\textbf{\textcolor{pink}{Wien XVIII. Sternwartestrasse 71}{}\ledrightnote{\textcolor{pink}{Sternwartestraße}}}}\pend
           \pstart{}Lieber Hermann.\pend\pstart
           Schönsten Dank für Deinen lieben Brief. Jedenfalls tut es mir leid, dass Du nicht
               über mein \textcolor{green}{Stück}{}\ledrightnote{→\textcolor{green}{Der junge Medardus. Dramatische Historie in einem Vorspiel und fünf Aufzügen}} schreiben wirst,
               denn was immer Du unter den Unannehmlichkeiten verstehst, die daraus für Dich, für
               mich, für alle Beteiligten folgen könnten, für mich wären sie jedenfalls durch das
               Vergnügen reichlich aufgewogen eine ausführliche Darlegung Deiner mir immer
               wertvollen Meinungen zu lesen. Ueberdies erscheint das \textcolor{green}{Stück}{}\ledrightnote{→\textcolor{green}{Der junge Medardus. Dramatische Historie in einem Vorspiel und fünf Aufzügen}} etwa \label{LL120-1v}acht Tage
                  vor der Premiere im Buchhandel\label{LL120-1h}, so dass eine Aeusserung über das Werk als
               solches ohne Rücksicht auf die Darstellung nicht als unstatthaft aufgefasst werden
               könnte.\pend
           \pstart
           Das Missverständnis, das Du befürchtest, ich hätte in dem \textcolor{green}{Medardus}{}\ledrightnote{→\textcolor{green}{Der junge Medardus. Dramatische Historie in einem Vorspiel und fünf Aufzügen}} einen tragischen Helden zeichnen
               wollen, kann meines Erachtens als solches überhaupt nicht auftreten. Dass Viele sich
               so stellen werden, als glaubten sie, ich selber hielte den \textcolor{green}{Medardus}{}\ledrightnote{→\textcolor{green}{Der junge Medardus. Dramatische Historie in einem Vorspiel und fünf Aufzügen}} für einen tragischen Helden, ist
               hingegen selbstverständlich. In {\pb}dieser Voraussicht war
               ich nahe daran der Buchausgabe ein kurzes Geleitwort mitzugeben ungefähr des
               folgenden Inhalts: \introOben{}»\introOben{}Es ist mir bekannt, dass dieses Stück
               sehr lang und dass der Medardus ein ausnehmend inkonsequentes Subjekt ist.\introOben{}« (\introOben{}Darum \label{T_L01981_1v}\edtext{passieren}{\lemma{\textnormal{\emph{passieren}}}\Cendnote{\textnormal{korrigiert aus:
                  »passierem«}}}\label{T_L01981_1h} ihm ja so sonderbare Dinge.\introOben{})\introOben{} Aber am
               Ende sind in dem Drama selbst so klare Ansichten über das Wesen des \textcolor{green}{Medardus}{}\ledrightnote{→\textcolor{green}{Der junge Medardus. Dramatische Historie in einem Vorspiel und fünf Aufzügen}} ausgesprochen, hauptsächlich durch
                  \textcolor{green}{Eschenbacher}{}\ledrightnote{→\textcolor{green}{Der junge Medardus. Dramatische Historie in einem Vorspiel und fünf Aufzügen}}, durch \textcolor{green}{Etzelt}{}\ledrightnote{→\textcolor{green}{Der junge Medardus. Dramatische Historie in einem Vorspiel und fünf Aufzügen}} und auch durch die \textcolor{green}{Frau Klähr}{}\ledrightnote{→\textcolor{green}{Der junge Medardus. Dramatische Historie in einem Vorspiel und fünf Aufzügen}}, dass der Unverstand,
               der sich durch die dramatische Historie selbst nicht belehren liesse, auch mit einem
               solchen Vorwort nichts anzufangen wüsste. Auch glaube ich mich mit Dir eines Sinnes,
               wenn ich behaupte, dass kein dramatischer Autor verpflichtet ist in den Mittelpunkt
               seiner Stücke gerade einen sogenannt\introOben{}en\introOben{} tragischen Helden
               hineinzustellen. Der \textcolor{green}{Hamlet}{}\ledrightnote{→\textcolor{green}{Hamlet}} ist
               es im dogmatischen Sinne so wenig als der \label{K_L01981_1v}\edtext{\textcolor{green}{Oswald}{}\ledrightnote{→\textcolor{green}{Gespenster}}}{\lemma{\textnormal{\emph{Oswald}}}\Cendnote{\textnormal{Figur aus \emph{\textcolor{green}{Gespenster}} von \textcolor{blue}{Ibsen}}}}\label{K_L01981_1h}, der
                  \label{K_L01981_2v}\edtext{\textcolor{green}{Prinz von Homburg}{}\ledrightnote{→\textcolor{green}{Prinz Friedrich von Homburg}}}{\lemma{\textnormal{\emph{Prinz von Homburg}}}\Cendnote{\textnormal{die Titelrolle in \emph{\textcolor{green}{Prinz Friedrich von Homburg}} von \textcolor{blue}{Kleist}}}}\label{K_L01981_2h} so wenig als der \label{K_L01981_3v}\edtext{\textcolor{green}{Tasso}{}\ledrightnote{→\textcolor{green}{Torquato Tasso}}}{\lemma{\textnormal{\emph{Tasso}}}\Cendnote{\textnormal{die Titelrolle in \emph{\textcolor{green}{Torquato Tasso}} von \textcolor{blue}{Goethe}}}}\label{K_L01981_3h}. Dies sind natürlich Beispiele nicht etwa Vergleiche. Kein
               Zweifel übrigens, dass sich der Autor nach dieser Richtung umso mehr erlauben darf je
               verstorbener er ist. – Was Deine weitere Befürchtung anbe{\pb}langt, dass das
               Publikum ein anderes Stück zu sehen bekommen wird als ich geschrieben habe, so ist
               sie zum Teil vielleicht gerechtfertigt, aber nicht durchaus als Befürchtung. Ich habe
               für die Zwecke der Bühne nicht nur sehr viel gestrichen, sondern auch gewisse
               Umstellungen vorgenommen; Kompromisse ohne die auch manche andere\introOben{},\introOben{} und grössere\introOben{},\introOben{} Werke sich auf der Bühne nicht
               hätten halten, ja nicht einmal auf sie hätten gelangen können. Leider muss ich auch
               zugestehen, dass der \textcolor{green}{Medardus}{}\ledrightnote{→\textcolor{green}{Der junge Medardus. Dramatische Historie in einem Vorspiel und fünf Aufzügen}}
               selbst heute in dem \textcolor{pink}{Burgtheater}{}\ledrightnote{\textcolor{pink}{Burgtheater}} nicht zu besetzen ist
                  (\uline{\label{T_L01981_2v}\edtext{dies ganz unter uns}{\lemma{\textnormal{\emph{dies ganz unter uns}}}\Cendnote{\textnormal{Unterstreichung mit Tinte von der
                     Schreiberin, vgl. Karte vom 19. 11. 1910.}}}\label{T_L01981_2h}}). Der Einzige, der ihn heute überhaupt spielen könnte, ist \textcolor{blue}{Moissi}{}\ledrightnote{\textcolor{blue}{Alexander Moissi}}. \label{K_L01981_4v}\edtext{\textcolor{blue}{Reinhardt}{}\ledrightnote{\textcolor{blue}{Max Reinhardt}}, als ich ihm das Stück vorlas}{\lemma{\textnormal{\emph{Reinhardt, … vorlas}}}\Cendnote{\textnormal{am 26. 8. 1909 in \textcolor{pink}{München}}}}\label{K_L01981_4h}, war auch ganz entschlossen ihm
               diese Rolle zuzuteilen, erst später erfuhr ich, dass er das Stück nur dann geben
               wollte, wenn ich ihm noch ein zweites überliesse, worauf ich aus prinzipiellen
               Gründen nicht einging. Bei \textcolor{blue}{Reinhardt}{}\ledrightnote{\textcolor{blue}{Max Reinhardt}} wären
               zweifellos auch die Massenszenen besser herausgekommen als es bei uns der Fall sein
               wird. Aber die übrige Besetzung hier ist zum grösseren und wichtigeren Teile von der
               Art, dass keine deutsche Bühne sie heute besser bieten könnte. Die \textcolor{blue}{Bleibtreu}{}\ledrightnote{\textcolor{blue}{Hedwig Bleibtreu}} als Frau Klähr, \textcolor{blue}{Balaithy}{}\ledrightnote{\textcolor{blue}{Robert von Balajthy}}{ }{\pb}als Eschenbacher, \textcolor{blue}{Tressler}{}\ledrightnote{\textcolor{blue}{Otto Tressler}} als Etzelt, \textcolor{blue}{Korff}{}\ledrightnote{\textcolor{blue}{Arnold Korff}} als Wachshuber, \textcolor{blue}{Hartmann}{}\ledrightnote{\textcolor{blue}{Ernst Hartmann}} als
               Herzog, \textcolor{blue}{Heine}{}\ledrightnote{\textcolor{blue}{Albert Heine}} als Assalagny, von der \textcolor{blue}{Medelsky}{}\ledrightnote{\textcolor{blue}{Lotte Medelsky}}, der \textcolor{blue}{Wolgemut}{}\ledrightnote{\textcolor{blue}{Else Wohlgemuth}}, von \textcolor{blue}{Reimers}{}\ledrightnote{\textcolor{blue}{Georg Reimers}} und \textcolor{blue}{Strassny}{}\ledrightnote{\textcolor{blue}{Fritz Strassni}} und \textcolor{blue}{Heller}{}\ledrightnote{\textcolor{blue}{Eduard Heller}} und Andern ganz zu geschweigen, das sind Leistungen im Einzelnen,
               meist auch im Zusammenspiel, dass Du, lieber Hermann, wenn Du die Vorstellung zu
               sehen bekämest gewiss nicht von herumdilettierenden Herrschaften sprächest, sondern
               das denen überliessest (es wird ja nicht an ihnen fehlen) denen vorgefasste Meinungen
               den teuersten und ach so bequemen Besitz bedeuten.\pend
           \pstart
           Nun will ich Dir noch von Herzen glückliche Vortragsreise wünschen und \strikeout{die} diesmal \substVorne{}\textsuperscript{hoffentlich}{\allowbreak}\substDazwischen{}die Hoffnung\substHinten{} nicht vergeblich\strikeout{e Hoffnung} aussprechen Dich
               und Deine verehrte Frau \textcolor{blue}{Gemahlin}{}\ledrightnote{→\textcolor{blue}{Anna Bahr-Mildenburg}} recht bald nach Deiner Rückkehr bei uns zu sehen. Ich selbst fahre
               etwa am 7. Dezember nach \textcolor{pink}{München}{}\ledrightnote{\textcolor{pink}{München}} (\label{K_L01981_5v}\edtext{Vorlesung}{\lemma{\textnormal{\emph{Vorlesung}}}\Cendnote{\textnormal{am 9. 12. 1909}}}\label{K_L01981_5h}) und
               auch nach \textcolor{pink}{Partenkirchen}{}\ledrightnote{\textcolor{pink}{Garmisch-Partenkirchen}} zu meiner \textcolor{blue}{Schwägerin}{}\ledrightnote{→\textcolor{blue}{Elisabeth Steinrück}}. Um den 15. herum denke ich wieder daheim zu sein.\pend
           \pstart
           {[}hs.:{]} Mit vielen treuen Grüßen{\\[\baselineskip]}Dein{\\[\baselineskip]}\spacefill\mbox{Arthur.}\pend
           \leftskip=0em{}\endnumbering\briefempfaengerindex{Bahr, Hermann@\textsc{Bahr, Hermann}!zzzSchnitzler, Arthur@\emph{von Arthur Schnitzler}!1910-11-171@{17. 11. 1910}|)be}\mylabel{h}  \normalsize

\doendnotes{C}
\bigskip
\vfill

\clearpage

\footnotesize

\lohead{\textsc{register}}

% Definiere theindex-Environment komplett neu ohne reledmac
\makeatletter
\renewenvironment{theindex}{%
  \section*{\indexname}%
  \setlength{\parindent}{0pt}%
  \setlength{\parskip}{0pt plus 0.3pt}%
  \let\item\@idxitem
}{%
  \clearpage
}
\makeatother

\IfFileExists{\jobname-pw.ind}{\input{\jobname-pw.ind}}{}

\end{document}

      