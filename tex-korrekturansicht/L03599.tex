%% latex-korrekturansicht-vorspann.tex
%% Vorspann für die Korrekturansicht.
%% Lädt die gemeinsame Datei latex-vorspann.tex mit gesetztem Schalter.

\newif\ifkorrekturansicht
\korrekturansichttrue

\input{../tex-inputs/latex-vorspann}


\renewcommand{\erwaehntePersonen}{Personen: Felix Salten}
\renewcommand{\erwaehnteInstitutionen}{Institutionen: S. Fischer Verlag}
\renewcommand{\erwaehnteOrte}{Orte: Berlin, Wien}
\renewcommand{\erwaehnteWerke}{Werke: Börsenblatt für den Deutschen Buchhandel, Der Schleier der Beatrice. Schauspiel in fünf Akten}
\section[ Arthur Schnitzler: Widmungsexemplar Der Schleier der Beatrice für Felix Salten, 27. 2. 1901]{Arthur Schnitzler: Widmungsexemplar Der Schleier der Beatrice für Felix
               Salten, 27. 2. 1901}
\nopagebreak\mylabel{v}
\rehead{ }\normalsize\beginnumbering\briefempfaengerindex{Salten, Felix@\textsc{Salten, Felix}!zzzSchnitzler, Arthur@\emph{von Arthur Schnitzler}!1901-02-271@{27. 2. 1901}|(be}
\toendnotes[C]{\smallbreak\pagebreak[2]}\Standort{Wienbibliothek im Rathaus, A-35809/2.Ex., DS-2019-33.}
\physDesc{Widmung am Vorsatzblatt, 51 Zeichen
\newline{}Handschrift: schwarze Tinte, deutsche Kurrent}\toendnotes[C]{\smallbreak}
\pstart
           \noindent{}{\pb}Meinem lieben Felix Salten\pend
           \pstart \spacefill\mbox{\uline{ArthSch}}\pend{}
\pstart
           \uline{\textcolor{pink}{Wien}{}\ledrightnote{\textcolor{pink}{Wien}}, 27. 2. 901.}\pend
           {\bigskip}
\pstart
           \noindent{}\centering{}{\pb}\textcolor{gray}{\textbf{\textcolor{green}{Der Schleier der Beatrice}{}\ledrightnote{\textcolor{green}{Der Schleier der Beatrice. Schauspiel in fünf Akten}}}}\pend
           
\pstart
           \noindent{}\centering{}\textcolor{gray}{\textbf{Schauſpiel in fünf Akten}}{\\}\textcolor{gray}{\textbf{von}}{\\}\textcolor{gray}{\textbf{\textbf{Arthur Schnitzler}}}\pend
           {\bigskip}
\pstart
           \noindent{}\centering{}\textcolor{gray}{\textbf{\textcolor{pink}{\so{Berlin}}{}\ledrightnote{\textcolor{pink}{Berlin}}}}\pend
           
\pstart
           \noindent{}\centering{}\textcolor{gray}{\textbf{\textcolor{brown}{S. Fiſcher, Verlag}{}\ledrightnote{\textcolor{brown}{S. Fischer Verlag}}}}\pend
           
\pstart
           \noindent{}\centering{}\textcolor{gray}{\textbf{\label{K_L03599-1v}\edtext{1901}{\lemma{\textnormal{\emph{1901}}}\Cendnote{\textnormal{am 30. 1. 1901 vom \emph{\textcolor{green}{Börsenblatt für den deutschen Buchhandel}}
                        als Neuerscheinung gemeldet}}}\label{K_L03599-1h}}}\pend
           \endnumbering\briefempfaengerindex{Salten, Felix@\textsc{Salten, Felix}!zzzSchnitzler, Arthur@\emph{von Arthur Schnitzler}!1901-02-271@{27. 2. 1901}|)be}\mylabel{h}  \normalsize

\doendnotes{C}
\bigskip
\vfill

\clearpage

\footnotesize

\lohead{\textsc{register}}

% Definiere theindex-Environment komplett neu ohne reledmac
\makeatletter
\renewenvironment{theindex}{%
  \section*{\indexname}%
  \setlength{\parindent}{0pt}%
  \setlength{\parskip}{0pt plus 0.3pt}%
  \let\item\@idxitem
}{%
  \clearpage
}
\makeatother

\IfFileExists{\jobname-pw.ind}{\input{\jobname-pw.ind}}{}

\end{document}

      