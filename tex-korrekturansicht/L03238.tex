%% latex-korrekturansicht-vorspann.tex
%% Vorspann für die Korrekturansicht.
%% Lädt die gemeinsame Datei latex-vorspann.tex mit gesetztem Schalter.

\newif\ifkorrekturansicht
\korrekturansichttrue

\input{../tex-inputs/latex-vorspann}


\renewcommand{\erwaehntePersonen}{Personen: Olga Schnitzler}
\renewcommand{\erwaehnteOrte}{Orte: Berlin, Dessauer Straße, Deutsches Theater Berlin, Wien}
\renewcommand{\erwaehnteWerke}{Werke: Zwischenspiel. Komödie in drei Akten}
\section[ Paul Goldmann an Arthur Schnitzler, 27. 11. {[}1905{]}]{Paul Goldmann an Arthur Schnitzler, 27. 11. {[}1905{]}}
\nopagebreak\mylabel{v}
\rehead{ }\normalsize\beginnumbering\briefempfaengerindex{Schnitzler, Arthur@\textsc{Schnitzler, Arthur}!zzzGoldmann, Paul@\emph{von Paul Goldmann}!1905-11-271@{27. 11. {[}1905{]}}|(be}
\toendnotes[C]{\smallbreak\pagebreak[2]}\Standort{DLA, A:Schnitzler, HS.NZ85.1.3175.}
\physDesc{Brief, 1 Blatt, 2 Seiten
\newline{}Handschrift: blaue Tinte, deutsche Kurrent
\newline{}Schnitzler: mit Bleistift das Jahr »{[}1{]}905« vermerkt }\toendnotes[C]{\smallbreak}
\pstart
           \noindent{}\raggedleft{}{\pb}\textcolor{pink}{\textcolor{gray}{\textbf{DESSAUERSTRASSE 19}}}{}\ledrightnote{\textcolor{pink}{Dessauer Straße}}\pend
           
\pstart
           \textcolor{pink}{Berlin}{}\ledrightnote{\textcolor{pink}{Berlin}}, 27. Nov.\pend
           
\pstart\center{}Lieber Freund,\pend
\pstart
           Ich danke Dir herzlichſt für die \label{K-L03238-1v}\edtext{Überſendung des \textcolor{green}{Buch}{}\ledrightnote{{$\rightarrow$}\textcolor{green}{Zwischenspiel. Komödie in drei Akten}}es}{\lemma{\textnormal{\emph{Überſendung des Buches}}}\Cendnote{\textnormal{\emph{\textcolor{green}{Zwischenspiel}}. Die Widmungsexemplare wurden
                  am 24. 11. 1905
                  versandt. vgl. Arthur Schnitzler: Widmungsexemplar Zwischenspiel für Hugo von
               Hofmannsthal, 24. 11. 1905 und Max Burckhard an Arthur Schnitzler, 30. 11. 1905.}}}\label{K-L03238-1h} und freue mich ſchon ſehr
               darauf, es in der erſten freien Stunde zu leſen. \pend
           
\pstart
           Soweit ich nach den Zeitungen urteilen kann, darf man Dich zum Erfolge der \label{K-L03238-2v}\edtext{\textcolor{green}{\textsc{Première}}{}\ledrightnote{{$\rightarrow$}\textcolor{green}{Zwischenspiel. Komödie in drei Akten}}}{\lemma{\textnormal{\emph{Première}}}\Cendnote{\textnormal{Am 25. 11. 1905 hatte die Premiere von \textcolor{blue}{Schnitzler}s \emph{\textcolor{green}{Zwischenspiel}} am \textcolor{pink}{Deutschen Theater
                     Berlin} in Anwesenheit des Autors
                  stattgefunden.}}}\label{K-L03238-2h} beglückwünſchen, was ich denn auch mit aller Herzlichkeit
               thue.\pend
           
\pstart
           {\pb}Hoffentlich biſt Du wohlbehalten \label{K-L03238-3v}\edtext{heimgekehrt}{\lemma{\textnormal{\emph{heimgekehrt}}}\Cendnote{\textnormal{\textcolor{blue}{Schnitzler} kam am 27. 11. 1905 wieder in
                     \textcolor{pink}{Wien} an.}}}\label{K-L03238-3h}. Grüße mir, bitte, Deine \textcolor{blue}{Frau}{}\ledrightnote{{$\rightarrow$}\textcolor{blue}{Olga Schnitzler}} und ſei ſelbſt \strikeout{von} vielmals gegrüßt von {\\}Deinem getreuen {\\}\spacefill\mbox{Paul Goldmann}\pend
           \endnumbering\briefempfaengerindex{Schnitzler, Arthur@\textsc{Schnitzler, Arthur}!zzzGoldmann, Paul@\emph{von Paul Goldmann}!1905-11-271@{27. 11. {[}1905{]}}|)be}\mylabel{h}  \normalsize

\doendnotes{C}
\bigskip
\vfill

\clearpage

\footnotesize

\lohead{\textsc{register}}

% Definiere theindex-Environment komplett neu ohne reledmac
\makeatletter
\renewenvironment{theindex}{%
  \section*{\indexname}%
  \setlength{\parindent}{0pt}%
  \setlength{\parskip}{0pt plus 0.3pt}%
  \let\item\@idxitem
}{%
  \clearpage
}
\makeatother

\IfFileExists{\jobname-pw.ind}{\input{\jobname-pw.ind}}{}

\end{document}

      