%% latex-korrekturansicht-vorspann.tex
%% Vorspann für die Korrekturansicht.
%% Lädt die gemeinsame Datei latex-vorspann.tex mit gesetztem Schalter.

\newif\ifkorrekturansicht
\korrekturansichttrue

\input{../tex-inputs/latex-vorspann}


               \section[Paul Goldmann an Arthur Schnitzler, 22. 8. {[}1895{]}]{ Paul Goldmann an Arthur Schnitzler, 22. 8. {[}1895{]}}\nopagebreak\mylabel{v}\rehead{ }\normalsize\beginnumbering\briefempfaengerindex{Schnitzler, Arthur@\textsc{Schnitzler, Arthur}!zzzGoldmann, Paul@\emph{von Paul Goldmann}!1895-08-221@{22. 8. {[}1895{]}}|(be} \toendnotes[C]{\smallbreak\pagebreak[2]} \Standort{DLA, A:Schnitzler, HS.NZ85.1.3165.}
\physDesc{Brief, 1 Blatt, 4 Seiten
\newline{}Handschrift: schwarze Tinte, deutsche Kurrent
\newline{}Schnitzler: 1) mit Bleistift das Jahr »95« vermerkt 2) mit rotem Buntstift zwei Unterstreichungen}\toendnotes[C]{\smallbreak}\pstart
           \noindent{}{\pb}\textcolor{gray}{\textbf{\textbf{\textcolor{brown}{Frankfurter Zeitung}{}\ledrightnote{\textcolor{brown}{Frankfurter Zeitung}}}}}\pend
           \pstart
           \textcolor{gray}{\textbf{(\textcolor{brown}{\begin{otherlanguage}{french}Gazette de Francfort\end{otherlanguage}}{}\ledrightnote{\textcolor{brown}{Frankfurter Zeitung}}.) }}\hfill \textsc{\textcolor{pink}{Toelz}{}\ledrightnote{\textcolor{pink}{Bad Tölz}}}, 22. Auguſt.\pend
           \pstart
           \textcolor{gray}{\textbf{\textbf{\begin{otherlanguage}{french}Fondateur M. \textcolor{blue}{L.
                              Sonnemann}{}\ledrightnote{\textcolor{blue}{Leopold Sonnemann}}\end{otherlanguage}.}}}\pend
           \pstart
           \begin{otherlanguage}{french}\textcolor{gray}{\textbf{\textcolor{green}{Journal}{}\ledrightnote{→\textcolor{green}{Frankfurter Zeitung}} politique,
                        financier,}}\end{otherlanguage}\pend
           \pstart
           \begin{otherlanguage}{french}\textcolor{gray}{\textbf{commercial et littéraire.}}\end{otherlanguage}\pend
           \pstart
           \begin{otherlanguage}{french}\textcolor{gray}{\textbf{\textbf{Paraissant trois fois par jour.}}}\end{otherlanguage}\pend
           \pstart
           \begin{otherlanguage}{french}\textcolor{gray}{\textbf{\textbf{Bureau à \textcolor{pink}{Paris}{}\ledrightnote{\textcolor{pink}{Paris}}:}}}\end{otherlanguage}\pend
           \pstart
           \begin{otherlanguage}{french}\textcolor{gray}{\textbf{\textbf{\textcolor{pink}{24. Rue Feydeau}{}\ledrightnote{\textcolor{pink}{rue Feydeau}}.}}}\end{otherlanguage}\pend
           \pstart\center{}Mein lieber Freund,\pend\pstart
           Telegraphire mir jedenſalls, \strikeout{\textcolor{gray}{×}} wann Du in \textcolor{pink}{Tegernſee}{}\ledrightnote{\textcolor{pink}{Tegernsee}} eintriffſt u. ob
               ich Dir hier Nachtquartier beſtellen ſoll? Ich möchte Dir ſchon gern entgegenkommen
               u. es lag auch ohne Deine Anregung in meiner Abſicht. Nun habe ich aber ſeit einigen
               Tagen als Folge der Kur einen ſo ſchrecklichen \label{K_L02746-1v}\edtext{Magen-Katarrh}{\lemma{\textnormal{\emph{Magen-Katarrh}}}\Cendnote{\textnormal{Entzündung der Magenschleimhaut}}}\label{K_L02746-1h}, daß ich kaum kriechen kann. Außerdem habe
               ich in \textcolor{pink}{Tegernſee}{}\ledrightnote{\textcolor{pink}{Tegernsee}} Verwandte, ſo daß mir ein
               anderer Rendezvous-Ort lieber wäre. Wie wäre es denn mit \textsc{\textcolor{pink}{Schliersee}{}\ledrightnote{\textcolor{pink}{Schliersee}}}? Dort {\pb}ſpielt am Sonntag{ }Abend das \label{K_L02746-55v}\edtext{\textcolor{brown}{Bauern-Theater}{}\ledrightnote{\textcolor{brown}{Schlierseer Bauerntheater}}}{\lemma{\textnormal{\emph{Bauern-Theater}}}\Cendnote{\textnormal{Das 1892 gegründete Theater war ein von ehemaligen Handwerkern
                  betriebenes Unternehmen, das durch Tourneen weithin berühmt war.}}}\label{K_L02746-55h}, was ſehr
               intereſſant ſein ſoll. Liegt das nicht auch auf \label{K_L02746-88v}\edtext{Eurer}{\lemma{\textnormal{\emph{Eurer}}}\Cendnote{\textnormal{\textcolor{blue}{Schnitzler} wurde von \textcolor{blue}{Felix Salten} begleitet}}}\label{K_L02746-88h} Route?\pend
           \pstart
           Übrigens, wie Du willſt. Du beſtimmſt, und wenn ich irgend mich bewegen kann, komme
               ich hin. Wenn nicht, erwarte ich Dich in \textsc{\textcolor{pink}{Toelz}{}\ledrightnote{\textcolor{pink}{Bad Tölz}}}.\pend
           \pstart
           Auch anderes Ärgerniß gibt es inzwiſchen. Ich fürchte, ich werde nur wenige Tage mit
               Euch zuſammenſein können. Familien-Pflichten! Meinem \textcolor{blue}{Onkel}{}\ledrightnote{→\textcolor{blue}{Fedor Mamroth}} fällt es jetzt plötzlich ein, ich
               müßte \uline{mich} mit ihm in der \textcolor{pink}{Schweiz}{}\ledrightnote{\textcolor{pink}{Schweiz}} treffen. Mein \textcolor{blue}{Schwager}{}\ledrightnote{→\textcolor{blue}{Josef Rosengart}} will nach \textsc{\textcolor{pink}{Muenchen}{}\ledrightnote{\textcolor{pink}{München}}} kommen und mich mit ſich fort nach der {\pb}\textcolor{pink}{Schweiz}{}\ledrightnote{\textcolor{pink}{Schweiz}} nehmen. Es iſt allerlei Wichtiges in
               Familien-Dingen zu erörtern. Ich erkläre Dir das Nähere mündlich. Würdeſt Du
               eventuell auf ein paar Tage mit nach der \label{K_L02746-2v}\edtext{\textcolor{pink}{Schweiz}{}\ledrightnote{\textcolor{pink}{Schweiz}}}{\lemma{\textnormal{\emph{Schweiz}}}\Cendnote{\textnormal{nicht geschehen}}}\label{K_L02746-2h} kommen?\pend
           \pstart
           Wirklich, diesmal geht Alles ſchief. Es iſt ekelhaft.\pend
           \pstart
           Ich erhalte ſoeben die \label{K_L02746-3v}\edtext{»\textcolor{green}{Freie Bühne}{}\ledrightnote{\textcolor{green}{Freie Bühne für den Entwickelungskampf der Zeit}}« mit der \strikeout{»E\textcolor{gray}{r}} »\textcolor{green}{kleinen Komödie}{}\ledrightnote{\textcolor{green}{Die kleine Komödie}}«}{\lemma{\textnormal{\emph{»Freie … Komödie«}}}\Cendnote{\textnormal{\textcolor{blue}{Arthur Schnitzler}:
                     \emph{\textcolor{green}{Die kleine Komödie}}. In: \emph{\textcolor{green}{Neue Deutsche Rundschau}}, Jg. 6, H. 8, 1. 8. 1895, S. 779-798.
                  (Die \emph{\textcolor{green}{Neue Deutsche Rundschau}} war als \emph{\textcolor{green}{Freie Bühne}} gegründet worden, aber nach vier Jahrgängen umbenannt worden.)}}}\label{K_L02746-3h}. Es ſind
               glänzende Sachen darin, und beſonders gelungen ſind die \textcolor{green}{Anfangsbriefe}{}\ledrightnote{→\textcolor{green}{Die kleine Komödie}}, welche die beiderſeitigen
                  \label{K_L02746-4v}\edtext{\begin{otherlanguage}{french}\textsc{états d’âme}\end{otherlanguage}}{\lemma{\textnormal{\emph{états d’âme}}}\Cendnote{\textnormal{französisch: Seelenstände (die deutsche Begriffsprägung
                  stammt von \textcolor{blue}{Hermann Bahr})}}}\label{K_L02746-4h}
               auseinanderſetzen. Aber im Ganzen {\pb}\strikeout{mag ich es} mag ich es nicht ſehr. Es iſt gar zu
               erzwungen und zu gekünſtelt in ſeinen thatſächlichen Vorausſetzungen. Auch fehlt mir
               das einfach und tief Menſchliche, das ich an Deinen ſonſtigen Arbeiten ſo liebe. Aber
               auch bei dieſer weniger gelungenen \textcolor{green}{Arbeit}{}\ledrightnote{→\textcolor{green}{Die kleine Komödie}} iſt Eines zu bemerken: die ungemeine Sicherheit der Schreibweiſe, –
               ſo, was beim Maler die feſte Hand iſt, welche die künſtleriſche Reife mit ſich
               bringt{\dotstwo}\textcolor{gray}{{\dotstwo}}\pend
           \pstart
           Viele treue Grüße an Euch Alle! {\\[\baselineskip]}Dein {\\[\baselineskip]}\spacefill\mbox{Paul Goldmann}\pend
           \leftskip=0em{}\endnumbering\briefempfaengerindex{Schnitzler, Arthur@\textsc{Schnitzler, Arthur}!zzzGoldmann, Paul@\emph{von Paul Goldmann}!1895-08-221@{22. 8. {[}1895{]}}|)be}\mylabel{h}  \normalsize

\doendnotes{C}
\bigskip
\vfill

\clearpage

\footnotesize

\lohead{\textsc{register}}

% Definiere theindex-Environment komplett neu ohne reledmac
\makeatletter
\renewenvironment{theindex}{%
  \section*{\indexname}%
  \setlength{\parindent}{0pt}%
  \setlength{\parskip}{0pt plus 0.3pt}%
  \let\item\@idxitem
}{%
  \clearpage
}
\makeatother

\IfFileExists{\jobname-pw.ind}{\input{\jobname-pw.ind}}{}

\end{document}

      