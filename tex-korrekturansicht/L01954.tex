%% latex-korrekturansicht-vorspann.tex
%% Vorspann für die Korrekturansicht.
%% Lädt die gemeinsame Datei latex-vorspann.tex mit gesetztem Schalter.

\newif\ifkorrekturansicht
\korrekturansichttrue

\input{../tex-inputs/latex-vorspann}


               \section[Arthur und Olga Schnitzler an Richard Beer-Hofmann, 23. 8. 1910]{ Arthur und Olga Schnitzler an Richard Beer-Hofmann, 23. 8. 1910}\nopagebreak\mylabel{v}\rehead{ }\normalsize\beginnumbering\briefempfaengerindex{Beer-Hofmann, Richard@\textsc{Beer-Hofmann, Richard}!zzzSchnitzler, Olga@\emph{von Olga Schnitzler}!1910-08-231@{23. 8. 1910}|(be}\briefempfaengerindex{Beer-Hofmann, Richard@\textsc{Beer-Hofmann, Richard}!zzzSchnitzler, Arthur@\emph{von Arthur Schnitzler}!1910-08-231@{23. 8. 1910}|(be} \toendnotes[C]{\smallbreak\pagebreak[2]} \Standort{YCGL, MSS 31.}
\physDesc{Bildpostkarte
\newline{}Handschrift Arthur Schnitzler: Bleistift, deutsche Kurrent\newline{}Handschrift Olga Schnitzler: Bleistift, lateinische Kurrent\newline{}Versand: Stempel: »\nobreak{}\oindex{Partenkirchen@\textbf{Partenkirchen}, \emph{Teil eines besiedelten Ortes (A.BSOX)}|pwk}Parten\textcolor{gray}{kirchen}, {[}23. 8.{]} 10, 6–7\nobreak{}«.  }\buchAbdrucke{\weitereDrucke{Arthur Schnitzler, Richard Beer-Hofmann: \emph{Briefwechsel 1891–1931}. Hg. Konstanze Fliedl. Wien, Zürich: \emph{Europaverlag} 1992, S. 212 .} }\toendnotes[C]{\smallbreak}\pstart{}{\pb}Hrn Dr. \textsc{Richard Beer
                     Hofmann}\pend{}\pstart{}\textsc{\textcolor{pink}{Ischl}{}\ledrightnote{\textcolor{pink}{Bad Ischl}}}\pend{}\pstart{}\textsc{\textcolor{pink}{Steinfeld\strikeout{\textcolor{gray}{gass}} 6}{}\ledrightnote{\textcolor{pink}{Steinfeld}}}.\pend{}{\bigskip}\pstart
           \noindent{}\centering{}{\pb}\textcolor{gray}{\textbf{\textcolor{pink}{Partenkirchen}{}\ledrightnote{\textcolor{pink}{Partenkirchen}}. \textcolor{pink}{St. Anton}{}\ledrightnote{\textcolor{pink}{St. Anton}} mit Blick auf \textcolor{pink}{Dreithorspitze}{}\ledrightnote{\textcolor{pink}{Dreitorspitze}}.}}\pend
           \pstart
           \centering{}{\pb}23. 8. 1910\pend
           \pstart
           Herzliche Grüße!{\\[\baselineskip]}\spacefill\mbox{A.}\pend
           \leftskip=0em{}\pstart
           \noindent{}{[}hs. O. Schnitzler:{]} Heute wurde am Krankenbett meiner \textcolor{blue}{Schwester}{}\ledrightnote{→\textcolor{blue}{Elisabeth Steinrück}} viel von Ihnen gesprochen. Sie sagt
               immer wieder: »B.-H. ist von Euch allen der merscht Begabte!«\pend
           \pstart
           Von hier fahren wir \uline{nicht} nach \textcolor{pink}{Ischl}{}\ledrightnote{\textcolor{pink}{Bad Ischl}}, sondern \textcolor{pink}{Frankfurt}{}\ledrightnote{\textcolor{pink}{Frankfurt am Main}}{ }\label{K_L01954_1v}\edtext{\textcolor{green}{Liebelei Opern-Première}{}\ledrightnote{\textcolor{green}{Liebelei. Oper in drei Akten}}}{\lemma{\textnormal{\emph{Liebelei Opern-Première}}}\Cendnote{\textnormal{siehe A. S.: \emph{Tagebuch}, 18. 9. 1910}}}\label{K_L01954_1h}, vorher \textcolor{pink}{Heidelberg}{}\ledrightnote{\textcolor{pink}{Heidelberg}}.\pend
           \pstart Herzlichste Grüsse Ihnen Allen! \spacefill\mbox{O. S.}\pend{}\endnumbering\briefempfaengerindex{Beer-Hofmann, Richard@\textsc{Beer-Hofmann, Richard}!zzzSchnitzler, Olga@\emph{von Olga Schnitzler}!1910-08-231@{23. 8. 1910}|)be}\briefempfaengerindex{Beer-Hofmann, Richard@\textsc{Beer-Hofmann, Richard}!zzzSchnitzler, Arthur@\emph{von Arthur Schnitzler}!1910-08-231@{23. 8. 1910}|)be}\mylabel{h}  \normalsize

\doendnotes{C}
\bigskip
\vfill

\clearpage

\footnotesize

\lohead{\textsc{register}}

% Definiere theindex-Environment komplett neu ohne reledmac
\makeatletter
\renewenvironment{theindex}{%
  \section*{\indexname}%
  \setlength{\parindent}{0pt}%
  \setlength{\parskip}{0pt plus 0.3pt}%
  \let\item\@idxitem
}{%
  \clearpage
}
\makeatother

\IfFileExists{\jobname-pw.ind}{\input{\jobname-pw.ind}}{}

\end{document}

      