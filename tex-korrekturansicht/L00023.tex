%% latex-korrekturansicht-vorspann.tex
%% Vorspann für die Korrekturansicht.
%% Lädt die gemeinsame Datei latex-vorspann.tex mit gesetztem Schalter.

\newif\ifkorrekturansicht
\korrekturansichttrue

\input{../tex-inputs/latex-vorspann}


               \section[Hugo von Hofmannsthal an Arthur Schnitzler, 13. 7. {[}1891{]}]{ Hugo von Hofmannsthal an Arthur Schnitzler, 13. 7. {[}1891{]}}\nopagebreak\mylabel{v}\rehead{ }\normalsize\beginnumbering\briefempfaengerindex{Schnitzler, Arthur@\textsc{Schnitzler, Arthur}!zzzHofmannsthal, Hugo von@\emph{von Hugo von Hofmannsthal}!1891-07-131@{13. 7. {[}1891{]}}|(be} \toendnotes[C]{\smallbreak\pagebreak[2]} \Standort{CUL, Schnitzler, B 43.}
\physDesc{Brief, 1 Blatt (Briefpapier mit aufgeprägtem Wappen), 3 Seiten
\newline{}Handschrift: schwarze Tinte, deutsche Kurrent
\newline{}Schnitzler: mit Bleistift die Jahreszahl hinzugefügt:
                                 »1891« \newline{}Ordnung: mit Bleistift von unbekannter Hand nummeriert:
                                 »4« }\Standort{FDH, Hs-29002.}
\physDesc{Brief, 1 Blatt (Briefpapier mit aufgeprägtem Wappen), 1 Seite, Entwurf
\newline{}Handschrift: schwarze Tinte, deutsche Kurrent}\buchAbdrucke{\weitereDrucke{1) Hugo von Hofmannsthal: \emph{Briefe. 1890–1901}. Berlin: \emph{S. Fischer} 1935, S. 21–23.} \weitereDrucke{2) Hugo von Hofmannsthal, Arthur Schnitzler: \emph{Briefwechsel}. Hg. Therese Nickl und Heinrich Schnitzler. Frankfurt am Main: \emph{S. Fischer} 1964, S. 7–8.} }\toendnotes[C]{\smallbreak}\pstart
           \raggedleft{}{\pb}\textcolor{pink}{Bad Fuſch}{}\ledrightnote{\textcolor{pink}{Bad Fusch}}, 13 Juli.\pend
           \pstart
           Mir fehlt hier irgend etwas; was, weiß ich ſelbſt nicht. Vielleicht Sonne. Vielleicht
               Lärm. Dann wird wohl \textcolor{pink}{Salzburg}{}\ledrightnote{\textcolor{pink}{Salzburg}} helfen. Ich habe einen
               dicken \label{K_L00023_1v}\edtext{Paletot}{\lemma{\textnormal{\emph{Paletot}}}\Cendnote{\textnormal{Herrenmantel}}}\label{K_L00023_1h} an, auf dem Papier
               tanzen grelle kalte Lichter, der Brunnen plätſchert und es riecht nach reinlichen
               kleinen Kindern. Wenn das eine Stimmung iſt, ſo iſts zumindeſten nicht die, die ich
               brauchen kann. \label{K_L00023_2v}\edtext{\textsc{En attendant}}{\lemma{\textnormal{\emph{En attendant}}}\Cendnote{\textnormal{französisch: in Erwartung}}}\label{K_L00023_2h}{ }\label{K_L00023_3v}\edtext{les’ ich \textcolor{blue}{Nietzſche}{}\ledrightnote{\textcolor{blue}{Friedrich Nietzsche}}}{\lemma{\textnormal{\emph{les’ ich Nietzſche}}}\Cendnote{\textnormal{Friedrich Nietzsche: \emph{\textcolor{green}{Menschliches,
                        Allzumenschliches. Ein Buch für freie Geister}}. Chemnitz: \emph{\textcolor{brown}{Schmeitzner}}{ }1878.}}}\label{K_L00023_3h} und freue mich wie in ſeiner kalten Klarheit, der »\label{K_L00023_4v}\edtext{hellen Luft der Cordilleren}{\lemma{\textnormal{\emph{hellen … Cordilleren}}}\Cendnote{\textnormal{\textcolor{blue}{Hofmannsthal} markiert die Stelle eindeutig als
                  Zitat. Dabei variiert er zumindest seine eigenen Aufzeichnungen vom
                     21. 5. 1891: »In \textcolor{blue}{Nietzsche} ist die freudige Klarheit der Zerstörung wie in einem einem
                     hellen Sturm der Cordilleren oder in dem reinen Lodern grosser
                  Flammen«. (\textcolor{blue}{Hugo von Hofmannsthal}: \emph{Aufzeichnungen}. Hg. Rudolf Hirsch † und Ellen Ritter † in
                     Zusammenarbeit mit Konrad Heumann und Peter Michael Braunwarth. Frankfurt am
                     Main: \emph{\textcolor{brown}{S. Fischer}}{ }2013, S. 108 (\emph{Sämtliche Werke},
                     XXXIX).) Vgl. auch den Briefentwurf in der gedruckten Ausgabe,
                  S. 323.}}}\label{K_L00023_4h}«, meine eigenen Gedanken ſchön cryſtalliſieren. Ich denke ſehr
               viel, wie immer wenn mir nichts einfällt, und ſchlecke künftige Geburtstagstorten ab:
               das heißt, ich genieße in zahlloſen Plänen das Beſte von künftigen Arbeiten: das
               Grauen vor der tragiſchen Situation und die Freude am Combinieren. Wozu verdirbt man
               ſich das eigentlich alles, indem man die ſchlechteſte Momentphotographie davon
               feſthält und aufhebt? Dumme Frage {\pb}übrigens, Kunſt kommt von Können und Können heißt ſchreibenkönnen. (\label{K_L00023_5v}\edtext{\textcolor{green}{\textsc{Mod. Rundschau}}{}\ledrightnote{\textcolor{green}{Moderne Rundschau}} 5 u. 6 Heft, Seite 17{\dots}ff.}{\lemma{\textnormal{\emph{Mod. … Seite 17ff.}}}\Cendnote{\textnormal{\textcolor{blue}{Hermann Bahr}: \emph{\textcolor{green}{Vorsatz.}} In: \emph{\textcolor{green}{Moderne Rundschau}},
                     Bd. 3, H. 5/6, 15. 6. 1891, S. 178–180. Es handelt sich um
                  die »Einleitung zu \textcolor{blue}{Bahr}’s demnächst
                     (bei \textcolor{brown}{E. Pierson} in \textcolor{pink}{Dresden}) erscheinendem neuesten Buche: ›\textcolor{green}{Russische Reise, ein lyrischer Zwischenakt}‹.«}}}\label{K_L00023_5h})\pend
           \pstart
           So dumme Fragen frage ich nur wenn ich Gedanken denke ſtatt mein Leben zu leben. Ich
               möchte mich alſo verlieben, oder täglich \textsc{lawn-tennis}{ }ſpielen, oder meinetwegen \label{K_L00023_6v}\edtext{\textsc{Macao}}{\lemma{\textnormal{\emph{Macao}}}\Cendnote{\textnormal{Glücksspiel mit Karten}}}\label{K_L00023_6h}, oder
               ſonſt eine Beſchäftigung erleben.\pend
           \pstart
           Sonſt werd ich noch ein »ganzer Politiker«, wie der \textcolor{blue}{Sauhirt}{}\ledrightnote{→\textcolor{blue}{?? [Schweinehirt in Bad Fusch]}} von ſeinem alten Vorſtehhund neulich ſagte, der aus
               Altersſchwäche dumm geworden iſt. Der \textcolor{blue}{Sauhirt}{}\ledrightnote{→\textcolor{blue}{?? [Schweinehirt in Bad Fusch]}} iſt keine \textsc{Fiction},
               ſondern mein liebſter Umgang, ſeine \textcolor{blue}{Tochter}{}\ledrightnote{→\textcolor{blue}{Veronica no name provided}} aber, das liebliche Saumenſch, heißt \textcolor{green}{\textsc{Berenike}}{}\ledrightnote{→\textcolor{green}{Der Garten der Bérenice}} (abgek. \textcolor{blue}{\textsc{Vroni}}{}\ledrightnote{\textcolor{blue}{Veronica no name provided}}) und war zu ihrer Blütezeit Kellnerin. Außerdem laſſe ich mir von einer alten
                  \textcolor{blue}{Engländerin}{}\ledrightnote{→\textcolor{blue}{Jane Emily Łaszowska}} auf naſskalten
               Spaziergängen viel erzählen: von der \textsc{Mozambiquebai}, wo die Leute meiſtens Würmer unter der Haut haben (ſie war dort als
               junge Frau) oder von dem häſslichen \textsc{boycott} in \textcolor{pink}{\textsc{Irland}}{}\ledrightnote{\textcolor{pink}{Irland}} und den ſchönen rothhaarigen \textsc{Cocotten} von \textcolor{pink}{Dublin}{}\ledrightnote{\textcolor{pink}{Dublin}} (von denen ſpricht ſie ſo giftig gut, wie aus
               einem \textsc{ressentiment} heraus, ſie muſs dort etwas unangenehmes
               erlebt haben) oder von \textcolor{blue}{\textsc{Henry Irving}}{}\ledrightnote{\textcolor{blue}{Henry Irving}} oder von \textcolor{blue}{\textsc{Sir Laurence Oliphant}}{}\ledrightnote{\textcolor{blue}{Laurence Oliphant}}, dem großen Medium.\pend
           \pstart
           {\pb}Ihre \label{K_L00023_7v}\edtext{Tochter}{\lemma{\textnormal{\emph{Tochter}}}\Cendnote{\textnormal{keine
                  weiblichen Nachkommen nachweisbar}}}\label{K_L00023_7h} wäre mir natürlich lieber, aber die iſt
               in \textcolor{pink}{Ceylon}{}\ledrightnote{\textcolor{pink}{Sri Lanka}}. Ich leſe \textcolor{blue}{\textsc{Homer}}{}\ledrightnote{\textcolor{blue}{Homer}}, \textcolor{blue}{\textsc{Maupassant}}{}\ledrightnote{\textcolor{blue}{Guy de Maupassant}}, das \textcolor{brown}{Linzer Volksblatt}{}\ledrightnote{\textcolor{brown}{Linzer Volksblatt}}, \textcolor{blue}{Eichendorff}{}\ledrightnote{\textcolor{blue}{Joseph von Eichendorff}} und \textcolor{green}{\label{K_L00023_8v}\edtext{\textsc{cette touchante histoire de petite Secousse}}{\lemma{\textnormal{\emph{cette … Secousse}}}\Cendnote{\textnormal{französisch: die rührende Geschichte von der
                     kleinen Schüttlerin (\textcolor{blue}{Barrès} bezeichnet so
                     die Hauptfigur \textcolor{green}{Berénice}.)}}}\label{K_L00023_8h}}{}\ledrightnote{→\textcolor{green}{Der Garten der Bérenice}}, die manchmal ſo ſchön iſt, \label{K_L00023_9v}\edtext{\textsc{qu’elle donne presque envie de pleurer}}{\lemma{\textnormal{\emph{qu’elle … pleurer}}}\Cendnote{\textnormal{französisch: dass sie nahezu Lust zu weinen
                  macht}}}\label{K_L00023_9h}, trotz \textcolor{blue}{\textsc{Boulange}}{}\ledrightnote{→\textcolor{blue}{Georges Boulanger}}, Mysti-, \strikeout{Ch\textcolor{gray}{×}\-\textcolor{gray}{×}\-\textcolor{gray}{×}- }, Stoi- und Katholi-cismus. Ich habe gar keine eigenen
               Empfindungen, citiere fortwährend in Gedanken mich ſelbſt oder andere, habe auch die
               dumme letzte Scene von »\textcolor{green}{Geſtern}{}\ledrightnote{\textcolor{green}{Gestern. Dramatische Studie in einem Akt in Versen}}« noch immer nicht
               fertig gebracht, dafür aber von \textcolor{blue}{Goldmann}{}\ledrightnote{\textcolor{blue}{Paul Goldmann}}, der
               immer auf der Eiſenbahn zu ſein ſcheint eine, ſoweit man ſie leſen kann, ſehr
               herzliche Karte bekommen. Jetzt muſs ich packen (ganz origineller Abgang!) ſchreiben
               Sie mir, mein verehrter Freund, bitte, bald und geben Sie Ihr Project mich irgendwo
               zu beſuchen, nicht auf.\pend
           \pstart
           Herzlichſt{\\[\baselineskip]}\spacefill\mbox{Loris}\pend
           \leftskip=0em{}\endnumbering\briefempfaengerindex{Schnitzler, Arthur@\textsc{Schnitzler, Arthur}!zzzHofmannsthal, Hugo von@\emph{von Hugo von Hofmannsthal}!1891-07-131@{13. 7. {[}1891{]}}|)be}\mylabel{h}  \normalsize

\doendnotes{C}
\bigskip
\vfill

\clearpage

\footnotesize

\lohead{\textsc{register}}

% Definiere theindex-Environment komplett neu ohne reledmac
\makeatletter
\renewenvironment{theindex}{%
  \section*{\indexname}%
  \setlength{\parindent}{0pt}%
  \setlength{\parskip}{0pt plus 0.3pt}%
  \let\item\@idxitem
}{%
  \clearpage
}
\makeatother

\IfFileExists{\jobname-pw.ind}{\input{\jobname-pw.ind}}{}

\end{document}

      