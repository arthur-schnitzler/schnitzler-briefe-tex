%% latex-korrekturansicht-vorspann.tex
%% Vorspann für die Korrekturansicht.
%% Lädt die gemeinsame Datei latex-vorspann.tex mit gesetztem Schalter.

\newif\ifkorrekturansicht
\korrekturansichttrue

\input{../tex-inputs/latex-vorspann}


               \section[Arthur Schnitzler an Stefan Großmann, 2. 12. 1913]{ Arthur Schnitzler an Stefan Großmann, 2. 12. 1913}\nopagebreak\mylabel{v}\rehead{ }\normalsize\beginnumbering\briefempfaengerindex{Grossmann, Stefan@\textsc{Großmann, Stefan}!zzzSchnitzler, Arthur@\emph{von Arthur Schnitzler}!1913-12-021@{2. 12. 1913}|(be} \toendnotes[C]{\smallbreak\pagebreak[2]} \Standort{DLA, A:Schnitzler, HS.NZ85.1.896.}
\physDesc{Brief, maschineller Durchschlag
\newline{}Schreibmaschine
\newline{}Handschrift: roter Buntstift, deutsche Kurrent (\noindent{}drei
                                        Unterstreichungen)}\toendnotes[C]{\smallbreak}\pstart
           \raggedleft{}{\pb}2. 12. 1913.\pend
           \pstart\center{}Verehrter Herr Grossmann.\pend\pstart
           Besten Dank für die freundliche Einladung des Verlags \textcolor{brown}{Ullstein}{}\ledrightnote{\textcolor{brown}{Ullstein Verlag}}. Ich bin jetzt so sehr mit einer grösseren \textcolor{green}{Arbeit}{}\ledrightnote{→\textcolor{green}{Komödie der Worte. Drei Einakter}} beschäftigt, dass
                    ich in der nächsten Zeit kaum dazu kommen werde, etwas für die Jubiläumsnummer
                    der \textcolor{brown}{Morgenpost}{}\ledrightnote{\textcolor{brown}{Morgenpost}} zu verfassen. Uebrigens merke
                    ich eben, dass Sie das Datum des Jubiläums nicht angegeben haben. Vielleicht
                    sagen Sie mir darüber noch ein Wort.\pend
           \pstart
           Mit vorzüglicher Hochachtung{\\[\baselineskip]}Ihr sehr ergebener\pend
           \leftskip=0em{}{\bigskip}\pstart
           \noindent{}Herrn Stefan Grossmann, \textcolor{pink}{Wien}{}\ledrightnote{\textcolor{pink}{Wien}}.\pend
           \endnumbering\briefempfaengerindex{Grossmann, Stefan@\textsc{Großmann, Stefan}!zzzSchnitzler, Arthur@\emph{von Arthur Schnitzler}!1913-12-021@{2. 12. 1913}|)be}\mylabel{h}  \normalsize

\doendnotes{C}
\bigskip
\vfill

\clearpage

\footnotesize

\lohead{\textsc{register}}

% Definiere theindex-Environment komplett neu ohne reledmac
\makeatletter
\renewenvironment{theindex}{%
  \section*{\indexname}%
  \setlength{\parindent}{0pt}%
  \setlength{\parskip}{0pt plus 0.3pt}%
  \let\item\@idxitem
}{%
  \clearpage
}
\makeatother

\IfFileExists{\jobname-pw.ind}{\input{\jobname-pw.ind}}{}

\end{document}

      