%% latex-korrekturansicht-vorspann.tex
%% Vorspann für die Korrekturansicht.
%% Lädt die gemeinsame Datei latex-vorspann.tex mit gesetztem Schalter.

\newif\ifkorrekturansicht
\korrekturansichttrue

\input{../tex-inputs/latex-vorspann}


\renewcommand{\erwaehnteInstitutionen}{Institutionen: Deutsches Theater Berlin}
\renewcommand{\erwaehnteOrte}{Orte: Berlin, Brühl, Carl-Theater, Dessauer Straße, Wien}
\renewcommand{\erwaehnteWerke}{Werke: Lebendige Stunden. Vier Einakter}
\section[ Paul Goldmann an Arthur Schnitzler, 12. 5. {[}1902{]}]{Paul Goldmann an Arthur Schnitzler, 12. 5. {[}1902{]}}
\nopagebreak\mylabel{v}
\rehead{ }\normalsize\beginnumbering\briefempfaengerindex{Schnitzler, Arthur@\textsc{Schnitzler, Arthur}!zzzGoldmann, Paul@\emph{von Paul Goldmann}!1902-05-121@{12. 5. {[}1902{]}}|(be}
\toendnotes[C]{\smallbreak\pagebreak[2]}\Standort{DLA, A:Schnitzler, HS.NZ85.1.3172.}
\physDesc{Brief, 1 Blatt, 1 Seite
\newline{}Handschrift: blaue Tinte, deutsche Kurrent
\newline{}Schnitzler: mit Bleistift das Jahr »1902« vermerkt }\toendnotes[C]{\smallbreak}
\pstart
           \noindent{}\raggedleft{}{\pb}\textcolor{pink}{\textcolor{gray}{\textbf{DESSAUERSTRASSE 19}}}{}\ledrightnote{\textcolor{pink}{Dessauer Straße}}\pend
           
\pstart
           \textcolor{pink}{Berlin}{}\ledrightnote{\textcolor{pink}{Berlin}}, 12. Mai.\pend
           
\pstart\center{}Mein lieber Freund,\pend
\pstart
           Ich warte vergeblich auf Deine Antwort: Biſt Du \label{K_L03208-1v}\edtext{Pfingſten in \textcolor{pink}{Wien}{}\ledrightnote{\textcolor{pink}{Wien}}}{\lemma{\textnormal{\emph{Pfingſten in Wien}}}\Cendnote{\textnormal{siehe Paul Goldmann an Arthur Schnitzler, 5. 5. [1902]}}}\label{K_L03208-1h}? Oder wohnſt Du in der \textcolor{pink}{Brühl}{}\ledrightnote{\textcolor{pink}{Brühl}}? Ich weiß
               noch nicht, ob ich fahren werde. Wenn ja, ſo dürfte ich \label{K_L03208-2v}\edtext{Samſtag{ }Abend}{\lemma{\textnormal{\emph{Samſtag Abend}}}\Cendnote{\textnormal{\textcolor{blue}{Goldmann} kam am 18. 5. 1902 in \textcolor{pink}{Wien} an.}}}\label{K_L03208-2h} in \textcolor{pink}{Wien}{}\ledrightnote{\textcolor{pink}{Wien}}
               eintreffen. Biſt Du dann in der \textcolor{pink}{Stadt}{}\ledrightnote{{$\rightarrow$}\textcolor{pink}{Wien}}? Selbſtverſtändlich darfſt Du Dich in Deinen Dispoſitionen durch mich
               in keiner Weiſe ſtören laſſen. \strikeout{\textcolor{gray}{Ic}} Ich beglückwünſche Dich herzlichſt zu Deinem \label{K_L03208-3v}\edtext{\textcolor{pink}{Wie}{}\ledrightnote{\textcolor{pink}{Wien}}ner \textcolor{green}{Erfolge}{}\ledrightnote{{$\rightarrow$}\textcolor{green}{Lebendige Stunden. Vier Einakter}}}{\lemma{\textnormal{\emph{Wiener Erfolge}}}\Cendnote{\textnormal{Am 6. 5. 1902 hatte die erfolgreiche Premiere des
                  Gastspiels von \emph{\textcolor{green}{Lebendige Stunden}} des \emph{\textcolor{brown}{Deutschen Theaters Berlin}} am \textcolor{pink}{Wien}er \textcolor{pink}{Carl-Theater}
                  stattgefunden. Auch die Kritiken fielen gut aus (vgl. A. S.: \emph{Tagebuch}, 7. 5. 1902).}}}\label{K_L03208-3h}. Viele treue Grüße!\pend
           
\pstart
           Dein {\\[\baselineskip]}\spacefill\mbox{Paul Goldmann}\pend
           \leftskip=0em{}\endnumbering\briefempfaengerindex{Schnitzler, Arthur@\textsc{Schnitzler, Arthur}!zzzGoldmann, Paul@\emph{von Paul Goldmann}!1902-05-121@{12. 5. {[}1902{]}}|)be}\mylabel{h}
\begin{anhang}
\end{anhang}\normalsize

\doendnotes{C}
\bigskip
\vfill

\clearpage

\footnotesize

\lohead{\textsc{register}}

% Definiere theindex-Environment komplett neu ohne reledmac
\makeatletter
\renewenvironment{theindex}{%
  \section*{\indexname}%
  \setlength{\parindent}{0pt}%
  \setlength{\parskip}{0pt plus 0.3pt}%
  \let\item\@idxitem
}{%
  \clearpage
}
\makeatother

\IfFileExists{\jobname-pw.ind}{\input{\jobname-pw.ind}}{}

\end{document}

      