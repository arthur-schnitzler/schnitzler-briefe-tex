%% latex-korrekturansicht-vorspann.tex
%% Vorspann für die Korrekturansicht.
%% Lädt die gemeinsame Datei latex-vorspann.tex mit gesetztem Schalter.

\newif\ifkorrekturansicht
\korrekturansichttrue

\input{../tex-inputs/latex-vorspann}


\renewcommand{\erwaehntePersonen}{Personen: Richard Beer-Hofmann, Paula Beer-Hofmann, Mirjam Beer-Hofmann, Maximilian Harden, Theodore Rottenberg, Olga Schnitzler, Heinrich Schnitzler}
\renewcommand{\erwaehnteOrte}{Orte: Berlin, Dessauer Straße, Schneeberg, Südtirol, Tirol, Welsberg-Taisten, Wien}
\renewcommand{\erwaehnteWerke}{}
\section[ Paul Goldmann an Arthur Schnitzler, 31. 7. {[}1903{]}]{Paul Goldmann an Arthur Schnitzler, 31. 7. {[}1903{]}}
\nopagebreak\mylabel{v}
\rehead{ }\normalsize\beginnumbering\briefempfaengerindex{Schnitzler, Arthur@\textsc{Schnitzler, Arthur}!zzzGoldmann, Paul@\emph{von Paul Goldmann}!1903-07-311@{31. 7. {[}1903{]}}|(be}
\toendnotes[C]{\smallbreak\pagebreak[2]}\Standort{DLA, A:Schnitzler, HS.NZ85.1.3173.}
\physDesc{Brief, 1 Blatt, 3 Seiten
\newline{}Handschrift: blaue Tinte, deutsche Kurrent
\newline{}Schnitzler: 1) mit Bleistift das Jahr »{[}1{]}903« vermerkt  2) mit rotem Buntstift zwei Unterstreichungen}\toendnotes[C]{\smallbreak}
\pstart
           \noindent{}\raggedleft{}{\pb}\textcolor{gray}{\textbf{\textcolor{pink}{DESSAUERSTRASSE 19}{}\ledrightnote{\textcolor{pink}{Dessauer Straße}}}}\pend
           
\pstart
           \textcolor{pink}{Berlin}{}\ledrightnote{\textcolor{pink}{Berlin}}, 31. Juli.\pend
           
\pstart\center{}Mein lieber Freund,\pend
\pstart
           Dank für Deine liebe Karte vom \label{K_L03379-2v}\edtext{\textcolor{pink}{Schneeberg}{}\ledrightnote{\textcolor{pink}{Schneeberg}}}{\lemma{\textnormal{\emph{Schneeberg}}}\Cendnote{\textnormal{\textcolor{blue}{Schnitzler} war am 28. 7. 1903 und 29. 7. 1903 auf dem
                     \textcolor{pink}{Schneeberg}, wo auch \textcolor{blue}{Richard} und \textcolor{blue}{Paula
                     Beer-Hofmann} sowie deren Tochter \textcolor{blue}{Mirjam} hinkamen.}}}\label{K_L03379-2h} und Deine Briefe.\pend
           
\pstart
           Noch iſts ungewiß, wann ich weggehe. Nächſte Woche wird ſichs entſcheiden, ob »\textcolor{blue}{sie}{}\ledrightnote{{$\rightarrow$}\textcolor{blue}{Theodore Rottenberg}}« \label{K_L03379-1v}\edtext{mitkommt}{\lemma{\textnormal{\emph{mitkommt}}}\Cendnote{\textnormal{siehe Paul Goldmann an Arthur Schnitzler, 27. 6. [1903]}}}\label{K_L03379-1h}. Wenn ja, ſo \strikeout{\textcolor{gray}{k}} reiſe ich über \textcolor{pink}{Wien}{}\ledrightnote{\textcolor{pink}{Wien}} nach \textcolor{pink}{Tirol}{}\ledrightnote{\textcolor{pink}{Tirol}{\newline}\textcolor{pink}{Südtirol}}; wenn nicht, ſo weiß ich noch gar nicht, was ich
               mache. Da das Alles ſo ungewiß iſt, bitte ich Dich dringend, nicht auf mich zu
               warten, mich aber immer in Kenntniß Deines Aufenthalts {\pb}zu laſſen.\pend
           
\pstart
           \label{K_L03379-4v}\edtext{\textsc{\textcolor{blue}{Harden}{}\ledrightnote{\textcolor{blue}{Maximilian Harden}}}}{\lemma{\textnormal{\emph{Harden}}}\Cendnote{\textnormal{nicht geschehen}}}\label{K_L03379-4h} hätte nicht übel
               Luſt, mit Dir und mir ein wenig nach \textcolor{pink}{Tirol}{}\ledrightnote{\textcolor{pink}{Tirol}{\newline}\textcolor{pink}{Südtirol}} zu kommen, – auch mit Dir allein, wenn ich nicht mitthäte. Ich habe
               ihm geſtern geſagt, daß Du Dich gewiß freuen wirſt,
               ihn zum Begleiter zu haben, und ich bitte Dich, ihm gleich zu ſchreiben\substVorne{}\textsuperscript{,}\substDazwischen{} und\substHinten{} ihn zum Mitkommen zu animiren. Er wäre gewiß ein charmanter und
               unterhaltender Gefährte.\pend
           
\pstart
           Laß’ mich alſo wiſſen, {\pb}welche Reiſe-Entſchlüſſe Du gefaßt haſt, ebenſo wie ich Dir sofort Mitteilung
               machen werde, ſobald ich Genaues weiß. (Möglich, daß ich, wenn ich \textcolor{blue}{Begleitung}{}\ledrightnote{{$\rightarrow$}\textcolor{blue}{Theodore Rottenberg}} habe, doch nach \textsc{\textcolor{pink}{Welsberg}{}\ledrightnote{\textcolor{pink}{Welsberg-Taisten}}} gehe.)\pend
           
\pstart
           Viele herzliche Grüße an Dich, \textsc{\textcolor{blue}{Olga}{}\ledrightnote{\textcolor{blue}{Olga Schnitzler}}} und \textsc{\textcolor{blue}{Heinrich}{}\ledrightnote{\textcolor{blue}{Heinrich Schnitzler}}}! {\\[\baselineskip]}Dein getreuer {\\[\baselineskip]}\spacefill\mbox{Paul Goldm}\pend
           \leftskip=0em{}\endnumbering\briefempfaengerindex{Schnitzler, Arthur@\textsc{Schnitzler, Arthur}!zzzGoldmann, Paul@\emph{von Paul Goldmann}!1903-07-311@{31. 7. {[}1903{]}}|)be}\mylabel{h}
\begin{anhang}
\end{anhang}\normalsize

\doendnotes{C}
\bigskip
\vfill

\clearpage

\footnotesize

\lohead{\textsc{register}}

% Definiere theindex-Environment komplett neu ohne reledmac
\makeatletter
\renewenvironment{theindex}{%
  \section*{\indexname}%
  \setlength{\parindent}{0pt}%
  \setlength{\parskip}{0pt plus 0.3pt}%
  \let\item\@idxitem
}{%
  \clearpage
}
\makeatother

\IfFileExists{\jobname-pw.ind}{\input{\jobname-pw.ind}}{}

\end{document}

      