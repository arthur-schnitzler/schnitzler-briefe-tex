%% latex-korrekturansicht-vorspann.tex
%% Vorspann für die Korrekturansicht.
%% Lädt die gemeinsame Datei latex-vorspann.tex mit gesetztem Schalter.

\newif\ifkorrekturansicht
\korrekturansichttrue

\input{../tex-inputs/latex-vorspann}


               \section[Arthur Schnitzler an Robert Adam, 22. 10. 1917]{ Arthur Schnitzler an Robert Adam, 22. 10. 1917}\nopagebreak\mylabel{v}\rehead{ }\normalsize\beginnumbering\briefempfaengerindex{Adam, Robert@\textsc{Adam, Robert}!zzzSchnitzler, Arthur@\emph{von Arthur Schnitzler}!1917-10-221@{22. 10. 1917}|(be} \toendnotes[C]{\smallbreak\pagebreak[2]} \Standort{DLA, 96.34.2/6.}
\physDesc{Postkarte
\newline{}Handschrift: Bleistift, deutsche Kurrent\newline{}Versand: Stempel: »\nobreak{}\textcolor{gray}{Wien}, 22. X. 1\textcolor{gray}{7}, 8\nobreak{}«.  }\toendnotes[C]{\smallbreak}\pstart{}{\pb}\textsc{Arthur Schnitzler}\pend{}\pstart{}\textcolor{pink}{Wien XVIII}{}\ledrightnote{\textcolor{pink}{VIII., Josefstadt}}\pend{}\pstart{}\textcolor{pink}{\textsc{Sternwartestr} 71}{}\ledrightnote{\textcolor{pink}{Sternwartestraße}}.\pend{}{\bigskip}\pstart{}\textsc{Herrn}\pend{}\pstart{}\textsc{Dr. Robert Adam Pollak}\pend{}\pstart{}\textcolor{pink}{\textsc{Wien XII}}{}\ledrightnote{\textcolor{pink}{XII., Meidling}}\pend{}\pstart{}\textcolor{pink}{\textsc{Meidlinger Hauptstr} 56}{}\ledrightnote{\textcolor{pink}{Meidlinger Hauptstraße}}.\pend{}{\bigskip}\pstart
           \raggedleft{}{\pb}22. X. 917\pend
           \pstart{}Verehrter Herr Doktor, \pend\pstart
           wollen Sie mir am Donnerſtag gegen 7 Uhr das Vergnügen
                    ſchenken ſo möcht ich Ihnen gern mancher\textcolor{gray}{lei} über Ihr ſehr
                    intereſſantes \textcolor{green}{\textsc{Manuscript}}{}\ledrightnote{→\textcolor{green}{Das Ende des Judas}} ſagen.\pend
           \pstart
           mit herzl Gruß Ihr erg{\\[\baselineskip]}\spacefill\mbox{Arth Schn}\pend
           \leftskip=0em{}\endnumbering\briefempfaengerindex{Adam, Robert@\textsc{Adam, Robert}!zzzSchnitzler, Arthur@\emph{von Arthur Schnitzler}!1917-10-221@{22. 10. 1917}|)be}\mylabel{h}  \normalsize

\doendnotes{C}
\bigskip
\vfill

\clearpage

\footnotesize

\lohead{\textsc{register}}

% Definiere theindex-Environment komplett neu ohne reledmac
\makeatletter
\renewenvironment{theindex}{%
  \section*{\indexname}%
  \setlength{\parindent}{0pt}%
  \setlength{\parskip}{0pt plus 0.3pt}%
  \let\item\@idxitem
}{%
  \clearpage
}
\makeatother

\IfFileExists{\jobname-pw.ind}{\input{\jobname-pw.ind}}{}

\end{document}

      