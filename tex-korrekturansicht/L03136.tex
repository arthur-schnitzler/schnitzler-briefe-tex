%% latex-korrekturansicht-vorspann.tex
%% Vorspann für die Korrekturansicht.
%% Lädt die gemeinsame Datei latex-vorspann.tex mit gesetztem Schalter.

\newif\ifkorrekturansicht
\korrekturansichttrue

\input{../tex-inputs/latex-vorspann}


\renewcommand{\erwaehntePersonen}{Personen: Hermann Bahr, Richard Beer-Hofmann}
\renewcommand{\erwaehnteOrte}{Orte: Café Arkaden, Frankgasse, I., Innere Stadt, IX., Alsergrund, München, Wien}
\renewcommand{\erwaehnteWerke}{}
\section[ Felix Salten an Arthur Schnitzler, 23. 5. 1894]{Felix Salten an Arthur Schnitzler, 23. 5. 1894}
\nopagebreak\mylabel{v}
\rehead{ }\normalsize\beginnumbering\briefempfaengerindex{Schnitzler, Arthur@\textsc{Schnitzler, Arthur}!zzzSalten, Felix@\emph{von Felix Salten}!1894-05-231@{23. 5. 1894}|(be}
\toendnotes[C]{\smallbreak\pagebreak[2]}\Standort{CUL, Schnitzler, B 89, A 1.}
\physDesc{Postkarte, 245 Zeichen
\newline{}Handschrift: schwarze Tinte, lateinische Kurrent
\newline{}Versand: 1) Stempel: »\nobreak{}\oindex{I., Innere Stadt@\textbf{I., Innere Stadt}, \emph{A.ADM3}|pwk}Wien 1/1 C.R., 23 V 94, 4–N\nobreak{}«.   2) Stempel: »\nobreak{}\oindex{IX., Alsergrund@\textbf{IX., Alsergrund}, \emph{A.ADM3}|pwk}Wien 9/2 71 r, 23 V 94, 4 10N\nobreak{}«. 
\newline{}Schnitzler: mit Bleistift datiert: »23/5 94« 
\newline{}Ordnung: mit Bleistift von unbekannter Hand nummeriert: »37« }\toendnotes[C]{\smallbreak}\pstart{}{\pb}Herrn D\textsuperscript{r} Arthur Schnitzler\pend{}\pstart{}\textcolor{pink}{IX. Frankgasse N\textsuperscript{o} 1}{}\ledrightnote{\textcolor{pink}{Frankgasse}}\pend{}
{\bigskip}
\pstart
           \noindent{}{\pb}Lieber Frd. Leider kann ich \label{K_L03136-1v}\edtext{morgen nicht mitthun}{\lemma{\textnormal{\emph{morgen nicht mitthun}}}\Cendnote{\textnormal{siehe A. S.: \emph{Tagebuch}, 24. 5. 1894}}}\label{K_L03136-1h}. Abends, (morgen) bin ich
               möglicherweise im \textcolor{pink}{Arkadencafé}{}\ledrightnote{\textcolor{pink}{Café Arkaden}}. Es wäre gut, wenn
               wir Alle wieder einmal beisammen wären, bevor \label{K_L03136-2v}\edtext{\textcolor{blue}{Beerhfm.}{}\ledrightnote{\textcolor{blue}{Richard Beer-Hofmann}} wegreist}{\lemma{\textnormal{\emph{Beerhfm. wegreist}}}\Cendnote{\textnormal{\textcolor{blue}{Beer-Hofmann} verreiste Anfang Juni gemeinsam mit \textcolor{blue}{Schnitzler} und \textcolor{blue}{Bahr} nach \textcolor{pink}{München}.}}}\label{K_L03136-2h}.\pend
           
\pstart
           Herzlichst Ihr {\\[\baselineskip]}\spacefill\mbox{Salten}\pend
           \leftskip=0em{}\endnumbering\briefempfaengerindex{Schnitzler, Arthur@\textsc{Schnitzler, Arthur}!zzzSalten, Felix@\emph{von Felix Salten}!1894-05-231@{23. 5. 1894}|)be}\mylabel{h}  \normalsize

\doendnotes{C}
\bigskip
\vfill

\clearpage

\footnotesize

\lohead{\textsc{register}}

% Definiere theindex-Environment komplett neu ohne reledmac
\makeatletter
\renewenvironment{theindex}{%
  \section*{\indexname}%
  \setlength{\parindent}{0pt}%
  \setlength{\parskip}{0pt plus 0.3pt}%
  \let\item\@idxitem
}{%
  \clearpage
}
\makeatother

\IfFileExists{\jobname-pw.ind}{\input{\jobname-pw.ind}}{}

\end{document}

      