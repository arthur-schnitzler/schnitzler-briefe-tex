%% latex-korrekturansicht-vorspann.tex
%% Vorspann für die Korrekturansicht.
%% Lädt die gemeinsame Datei latex-vorspann.tex mit gesetztem Schalter.

\newif\ifkorrekturansicht
\korrekturansichttrue

\input{../tex-inputs/latex-vorspann}


\renewcommand{\erwaehntePersonen}{Personen: Felix Salten, Ottilie Salten, Olga Schnitzler}
\renewcommand{\erwaehnteInstitutionen}{Institutionen: Hamburg-Amerika-Linie, Schnellpostdampfer Vaterland}
\renewcommand{\erwaehnteOrte}{Orte: Cuxhaven, Genua, IJmuiden, London, Nordsee, Paris, Southampton, Sternwartestraße 71, Wien}
\renewcommand{\erwaehnteWerke}{}
\section[ Felix und Ottilie Salten an Arthur und Olga Schnitzler, 14. 5. 1914]{Felix und Ottilie Salten an Arthur und Olga
               Schnitzler, 14. 5. 1914}
\nopagebreak\mylabel{v}
\rehead{ }\normalsize\beginnumbering\briefempfaengerindex{Schnitzler, Olga@\textsc{Schnitzler, Olga}!zzzSalten, Ottilie@\emph{von Ottilie Salten}!1914-05-142@{14. 5. 1914}|(be}\briefempfaengerindex{Schnitzler, Olga@\textsc{Schnitzler, Olga}!zzzSalten, Felix@\emph{von Felix Salten}!1914-05-142@{14. 5. 1914}|(be}\briefempfaengerindex{Schnitzler, Arthur@\textsc{Schnitzler, Arthur}!zzzSalten, Ottilie@\emph{von Ottilie Salten}!1914-05-142@{14. 5. 1914}|(be}\briefempfaengerindex{Schnitzler, Arthur@\textsc{Schnitzler, Arthur}!zzzSalten, Felix@\emph{von Felix Salten}!1914-05-142@{14. 5. 1914}|(be}
\toendnotes[C]{\smallbreak\pagebreak[2]}\Standort{CUL, Schnitzler, B 89, B 2.}
\physDesc{Bildpostkarte, 300 Zeichen
\newline{}Handschrift Felix Salten: schwarze Tinte, lateinische Kurrent
\newline{}Handschrift Ottilie Salten: schwarze Tinte, deutsche Kurrent
\newline{}Versand: Stempel: »\nobreak{}\oindex{Cuxhaven@\textbf{Cuxhaven}, \emph{P.PPLA3}|pwk}C{[}uxh{]}aven 1, 14. 5. 14, 3–4 N.\nobreak{}«.  
\newline{}Ordnung: mit Bleistift von unbekannter Hand nummeriert: »276« }\toendnotes[C]{\smallbreak}\pstart{}{\pb}Herrn u. Frau D\textsuperscript{r} Arthur Schnitzler\pend{}\pstart{}\textcolor{pink}{Wien}{}\ledrightnote{\textcolor{pink}{Wien}}\pend{}\pstart{}\textcolor{pink}{XVIII. Sternwartestraße 71}{}\ledrightnote{\textcolor{pink}{Sternwartestraße 71}}\pend{}
{\bigskip}
\pstart
           \noindent{}\centering{}{\pb}\textcolor{gray}{\textbf{\textcolor{brown}{HAMBURG-AMERIKA-LINIE}{}\ledrightnote{\textcolor{brown}{Hamburg-Amerika-Linie}}}}\pend
           
\pstart
           \noindent{}\centering{}\textcolor{gray}{\textbf{An Bord des Vierschrauben-Turbinen-Schnellpostdampfers}}\pend
           
\pstart
           \noindent{}\centering{}\textcolor{gray}{\textbf{»\textcolor{brown}{VATERLAND}{}\ledrightnote{\textcolor{brown}{Schnellpostdampfer Vaterland}}«}}\pend
           
\pstart
           \noindent{}\textcolor{gray}{\textbf{Speisesaal I. Klasse}}\pend
           
\pstart
           {\pb}\textcolor{gray}{\textbf{den}}{ }14. V. 14.\pend
           
\pstart
           Nun sind wir doch \label{K_L03563-1v}\edtext{auch zur \textcolor{pink}{See}{}\ledrightnote{{$\rightarrow$}\textcolor{pink}{Nordsee}}}{\lemma{\textnormal{\emph{auch zur See}}}\Cendnote{\textnormal{\textcolor{blue}{Arthur} und \textcolor{blue}{Olga Schnitzler} befanden sich auf einer
                  Schiffsreise, die am 13. 5. 1914 in \textcolor{pink}{Genua} begann und
                  am 22. 5. 1914 in
                     \textcolor{pink}{IJmuiden} endete. Am 20. 5. 1914 machten
                  sie – wenige Tage nach \textcolor{blue}{Saltens} –
                  in \textcolor{pink}{Southampton} Station. }}}\label{K_L03563-1h}. Wir fahren
               in einer Stunde. Steigen in \textcolor{pink}{\begin{otherlanguage}{english}Southampton\end{otherlanguage}}{}\ledrightnote{\textcolor{pink}{Southampton}} aus, fahren von dort nach \textcolor{pink}{London}{}\ledrightnote{\textcolor{pink}{London}} u. \textcolor{pink}{Paris}{}\ledrightnote{\textcolor{pink}{Paris}}. Viele herzliche Grüße und Reisewünsche gehen von uns zu Ihnen. Ihr {\\}\spacefill\mbox{Felix Salten}\pend
           
\pstart
           \noindent{}{[}hs. Ottilie Salten:{]} Herzliche Grüße\pend
           \pstart \spacefill\mbox{Ottilie Salten}\pend{}\endnumbering\briefempfaengerindex{Schnitzler, Olga@\textsc{Schnitzler, Olga}!zzzSalten, Ottilie@\emph{von Ottilie Salten}!1914-05-142@{14. 5. 1914}|)be}\briefempfaengerindex{Schnitzler, Olga@\textsc{Schnitzler, Olga}!zzzSalten, Felix@\emph{von Felix Salten}!1914-05-142@{14. 5. 1914}|)be}\briefempfaengerindex{Schnitzler, Arthur@\textsc{Schnitzler, Arthur}!zzzSalten, Ottilie@\emph{von Ottilie Salten}!1914-05-142@{14. 5. 1914}|)be}\briefempfaengerindex{Schnitzler, Arthur@\textsc{Schnitzler, Arthur}!zzzSalten, Felix@\emph{von Felix Salten}!1914-05-142@{14. 5. 1914}|)be}\mylabel{h}  \normalsize

\doendnotes{C}
\bigskip
\vfill

\clearpage

\footnotesize

\lohead{\textsc{register}}

% Definiere theindex-Environment komplett neu ohne reledmac
\makeatletter
\renewenvironment{theindex}{%
  \section*{\indexname}%
  \setlength{\parindent}{0pt}%
  \setlength{\parskip}{0pt plus 0.3pt}%
  \let\item\@idxitem
}{%
  \clearpage
}
\makeatother

\IfFileExists{\jobname-pw.ind}{\input{\jobname-pw.ind}}{}

\end{document}

      