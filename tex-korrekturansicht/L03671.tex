%% latex-korrekturansicht-vorspann.tex
%% Vorspann für die Korrekturansicht.
%% Lädt die gemeinsame Datei latex-vorspann.tex mit gesetztem Schalter.

\newif\ifkorrekturansicht
\korrekturansichttrue

\input{../tex-inputs/latex-vorspann}


\section[Stefan Zweig an Arthur Schnitzler, 21. 8. 1926]{L03671 Stefan Zweig an Arthur Schnitzler, 21. 8. 1926}
\nopagebreak\mylabel{L03671v}
\rehead{ }\normalsize\beginnumbering\briefempfaengerindex{Schnitzler, Arthur@\textsc{Schnitzler, Arthur}!zzzZweig, Stefan@\emph{von Stefan Zweig}!1926-08-211@{21. 8. 1926}|(be}
\toendnotes[C]{\smallbreak\pagebreak[2]}
\correspDesc{Versand  durch Stefan Zweig am 21. 8. 1926 \textbf{Ort fehlend} 
\newline{}Erhalt  durch Arthur Schnitzler im Zeitraum [22. 8. 1926 – 26. 8. 1926?] in Zermatt}\toendnotes[C]{\smallbreak}
\Standort{CUL, Schnitzler, B 118.}
\physDesc{Bildpostkarte, 511 Zeichen
\newline{}Handschrift: schwarze Tinte, lateinische Kurrent
\newline{}Versand: Stempel: »\nobreak{}\oindex{Montreux@\textbf{Montreux}|pwk}Montreux – Bon Port, 21. VIII. 26, 17\nobreak{}«.  }
\buchAbdrucke{\weitereDrucke{Stefan Zweig: \emph{Briefwechsel mit Hermann Bahr, Sigmund Freud, Rainer Maria
                        Rilke und Arthur Schnitzler}. Herausgegeben von Jeffrey B. Berlin, Hans-Ulrich Lindken und Donald A. Prater. Frankfurt am Main: \emph{S. Fischer} 1987, S. 421–422.} }\toendnotes[C]{\smallbreak}\pstart{}{\pb}D\textsuperscript{r}
                  Arthur Schnitzler\pend{}\pstart{}\textcolor{pink}{Zermatt}\oindex{Zermatt@\textbf{Zermatt}, \emph{Verwaltungsgebiet}|pw}{}\ledrightnote{\textcolor{pink}{Zermatt}}\pend{}\pstart{}\textcolor{pink}{Hotel Beau Site}\oindex{Parkhotel Beau Site@\textbf{Parkhotel Beau Site}, \emph{Hotel}|pw}{}\ledrightnote{\textcolor{pink}{Parkhotel Beau Site}}\pend{}{\bigskip}
\pstart
           \noindent{}\centering{}{\pb}\textcolor{gray}{\textbf{\textcolor{pink}{Château de Chillon}\oindex{Schloss Chillon@\textbf{Schloss Chillon}, \emph{Schloss}|pw}{}\ledrightnote{\textcolor{pink}{Schloss Chillon}}}}\pend
           \vspace{1em}
\pstart
           \noindent{}{\pb}Lieber verehrter Herr
                  Doktor, ich habe nachgefragt: in \textcolor{pink}{Montreux}\oindex{Montreux@\textbf{Montreux}|pw}{}\ledrightnote{\textcolor{pink}{Montreux}} kann man nicht Seebaden, nur in \textcolor{pink}{Clarens}\oindex{Clarens@\textbf{Clarens}|pw}{}\ledrightnote{\textcolor{pink}{Clarens}} und \textcolor{pink}{Ouchy}\oindex{Ouchy@\textbf{Ouchy}|pw}{}\ledrightnote{\textcolor{pink}{Ouchy}}. Ich denke hier,
               herrlich still in glühendster Sonne im \textcolor{pink}{Hotel
                  Byron}\oindex{Hôtel Byron@\textbf{Hôtel Byron}, \emph{Hotel}|pw}{}\ledrightnote{\textcolor{pink}{Hôtel Byron}} in \textcolor{pink}{Villeneuve}\oindex{Villeneuve@\textbf{Villeneuve}|pw}{}\ledrightnote{\textcolor{pink}{Villeneuve}} rastend, mit viel
               Dankbarkeit unserer \label{K_L03671-1v}\edtext{Begegnung}{\lemma{\textnormal{\emph{Begegnung}}}\Cendnote{\textnormal{siehe A. S.: \emph{Tagebuch}, 20. 8. 1926.}}}\label{K_L03671-1} im Bergland!\pend
           
\pstart
           Grüssen Sie, bitte, Frau \textcolor{blue}{Pollaczek}\pwindex{Pollaczek, Clara Katharina 15.\,1.\,1875 Wien – 22.\,7.\,1951 ebd.@\textsc{Pollaczek, Clara Katharina} (15.\,1.\,1875 Wien – 22.\,7.\,1951 ebd.), \emph{Schriftstellerin}|pw}{}\ledrightnote{\textcolor{blue}{Clara Katharina Pollaczek}}
               ergebenst von mir und denken Sie freundlichst Ihres immer getreuen{\\[\baselineskip]}\spacefill\mbox{Stefan Zweig}\pend
           \leftskip=0em{}
\pstart
           \noindent{}{\pb}Der Blick von meinem Fenster! Ein
                  menschenleeres wunderbares \textcolor{pink}{Hotel}\oindex{Hôtel Byron@\textbf{Hôtel Byron}, \emph{Hotel}|pwv}{}\ledrightnote{{$\rightarrow$}\emph{\textcolor{pink}{Hôtel Byron}}}, herrlich abseits in dem man Monate leben möchte!\pend
           \selectlanguage{ngerman}\endnumbering\briefempfaengerindex{Schnitzler, Arthur@\textsc{Schnitzler, Arthur}!zzzZweig, Stefan@\emph{von Stefan Zweig}!1926-08-211@{21. 8. 1926}|)be}\mylabel{L03671h}
\begin{anhang}
\end{anhang}\normalsize

\doendnotes{C}
\bigskip
\vfill

\clearpage

\footnotesize

\lohead{\textsc{register}}

% Definiere theindex-Environment komplett neu ohne reledmac
\makeatletter
\renewenvironment{theindex}{%
  \section*{\indexname}%
  \setlength{\parindent}{0pt}%
  \setlength{\parskip}{0pt plus 0.3pt}%
  \let\item\@idxitem
}{%
  \clearpage
}
\makeatother

\IfFileExists{\jobname-pw.ind}{\input{\jobname-pw.ind}}{}

\end{document}

      