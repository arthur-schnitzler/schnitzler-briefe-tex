%% latex-korrekturansicht-vorspann.tex
%% Vorspann für die Korrekturansicht.
%% Lädt die gemeinsame Datei latex-vorspann.tex mit gesetztem Schalter.

\newif\ifkorrekturansicht
\korrekturansichttrue

\input{../tex-inputs/latex-vorspann}


\section[Stefan und Friderike Zweig an Arthur Schnitzler, {[}18. 2. 1931?{]}]{L03677 Stefan und Friderike Zweig an Arthur Schnitzler, {[}18. 2. 1931?{]}}
\nopagebreak\mylabel{L03677v}
\rehead{ }\normalsize\beginnumbering\briefempfaengerindex{Schnitzler, Arthur@\textsc{Schnitzler, Arthur}!zzzZweig, Friderike Maria@\emph{von Friderike Maria Zweig}!1931-02-181@{{[}18. 2. 1931?{]}}|(be}\briefempfaengerindex{Schnitzler, Arthur@\textsc{Schnitzler, Arthur}!zzzZweig, Stefan@\emph{von Stefan Zweig}!1931-02-181@{{[}18. 2. 1931?{]}}|(be}
\toendnotes[C]{\smallbreak\pagebreak[2]}
\correspDesc{Versand  durch Stefan Zweig, Friderike Zweig am [18. 2. 1931?] in Antibes
\newline{}Erhalt  durch Arthur Schnitzler im Zeitraum [19. 2. 1931
                  – 25. 2. 1931?] in Wien}\toendnotes[C]{\smallbreak}
\Standort{CUL, Schnitzler, B 118.}
\physDesc{Bildpostkarte, 293 Zeichen
\newline{}Handschrift Stefan Zweig: lila Tinte, lateinische Kurrent
\newline{}Handschrift Friderike Maria Zweig: schwarze Tinte, lateinische Kurrent
\newline{}Versand: 1) Stempel: »\nobreak{}\oindex{Antibes@\textbf{Antibes}|pwk}Antibes, Son cap, sa plage de Juan de Pins été hiver\nobreak{}«.   2) Stempel: »\nobreak{}\oindex{Antibes@\textbf{Antibes}|pwk}Antibes Alpes Maritimes, 18{[}. 2. 1931{]}, 2\nobreak{}«. }
\buchAbdrucke{\weitereDrucke{Stefan Zweig: \emph{Briefwechsel mit Hermann Bahr, Sigmund Freud, Rainer Maria
                        Rilke und Arthur Schnitzler}. Herausgegeben von Jeffrey B. Berlin, Hans-Ulrich Lindken und Donald A. Prater. Frankfurt am Main: \emph{S. Fischer} 1987, S. 450–451.} }\toendnotes[C]{\smallbreak}\pstart{}{\pb}D\textsuperscript{r} Arthur
                  Schnitzler\pend{}\pstart{}\textcolor{pink}{Wien (Autriche)}\oindex{Wien@\textbf{Wien}, \emph{Verwaltungsgebiet}|pw}{}\ledrightnote{\textcolor{pink}{Wien}}\pend{}\pstart{}\textcolor{pink}{Sternwartestrasse 71}\oindex{Wien@\textbf{Wien}!XVIII., Währing@\textbf{XVIII., Währing}!Sternwartestraße 71@\textbf{Sternwartestraße 71}, \emph{Wohngebäude}|pw}{}\ledrightnote{\textcolor{pink}{Sternwartestraße 71}}\pend{}{\bigskip}
\pstart
           \noindent{}\centering{}{\pb}\textcolor{gray}{\textbf{LA DOUCE \textcolor{pink}{FRANCE}\oindex{Frankreich@\textbf{Frankreich}|pw}{}\ledrightnote{\textcolor{pink}{Frankreich}} – \textcolor{pink}{CÔTE D’AZUR}\oindex{Côte d’Azur@\textbf{Côte d’Azur}|pw}{}\ledrightnote{\textcolor{pink}{Côte d’Azur}}}}\pend
           
\pstart
           \centering{}\textcolor{gray}{\textbf{\textcolor{pink}{ANTIBES}\oindex{Antibes@\textbf{Antibes}|pw}{}\ledrightnote{\textcolor{pink}{Antibes}}}}\pend
           
\pstart
           \centering{}\textcolor{gray}{\textbf{Maisons contruites dans le Roc}}\pend
           \vspace{1em}
\pstart
           {\pb}\textcolor{pink}{Antibes, Hotel du Cap}\oindex{Hotel du Cap@\textbf{Hotel du Cap}, \emph{Hotel}|pw}{}\ledrightnote{\textcolor{pink}{Hotel du Cap}}\pend
           
\pstart{}Lieber verehrter Herr Doktor,\pend\vspace{0.5em}
\pstart
           wenn auch räumlich fern, war ich doch mit ganzem Herzen bei Ihrem grossen \label{K_L03677-1v}\edtext{Erfolge}{\lemma{\textnormal{\emph{Erfolge}}}\Cendnote{\textnormal{\textcolor{blue}{Zweig}\pwindex{Zweig, Stefan 28.\,11.\,1881 Wien – 23.\,2.\,1942 Petrópolis@\textsc{Zweig, Stefan} (28.\,11.\,1881 Wien – 23.\,2.\,1942 Petrópolis), \emph{Schriftsteller}|pwk} war von Februar bis
                     Mitte März 1931 in \textcolor{pink}{Antibes}\oindex{Antibes@\textbf{Antibes}|pwk}.
                  In diese Zeit fällt in \textcolor{blue}{Schnitzlers}
                  öffentliches Leben die Uraufführung von \emph{\textcolor{green}{Der Gang
                     zum Weiher}\pwindex{Schnitzler, Arthur 15. 5. 1862 Wien – 21. 10. 1931 ebd.@\textsc{Schnitzler, Arthur} (15. 5. 1862 Wien – 21. 10. 1931 ebd.), \emph{Schriftsteller, Mediziner}!Gang zum Weiher. Dramatische Dichtung@\strich\emph{Der Gang zum Weiher. Dramatische Dichtung}|pwk}} am 14. 2. 1931 am \emph{\textcolor{brown}{Burgtheater}\orgindex{Burgtheater@Burgtheater|pwk}}.}}}\label{K_L03677-1} und beglückwünsche Sie auf das herzlichste.\pend
           
\pstart
           Ihr aufrichtiger{\\[\baselineskip]}\spacefill\mbox{Stefan Zweig}\pend
           \leftskip=0em{}\selectlanguage{ngerman}\vspace{1em}
\pstart
           \noindent{}{[}hs. Zweig:{]} Viele ergebene Grüße v.\pend
           \pstart \spacefill\mbox{Friderike Zweig}\pend{}\selectlanguage{ngerman}\endnumbering\briefempfaengerindex{Schnitzler, Arthur@\textsc{Schnitzler, Arthur}!zzzZweig, Friderike Maria@\emph{von Friderike Maria Zweig}!1931-02-181@{{[}18. 2. 1931?{]}}|)be}\briefempfaengerindex{Schnitzler, Arthur@\textsc{Schnitzler, Arthur}!zzzZweig, Stefan@\emph{von Stefan Zweig}!1931-02-181@{{[}18. 2. 1931?{]}}|)be}\mylabel{L03677h}  \normalsize

\doendnotes{C}
\bigskip
\vfill

\clearpage

\footnotesize

\lohead{\textsc{register}}

% Definiere theindex-Environment komplett neu ohne reledmac
\makeatletter
\renewenvironment{theindex}{%
  \section*{\indexname}%
  \setlength{\parindent}{0pt}%
  \setlength{\parskip}{0pt plus 0.3pt}%
  \let\item\@idxitem
}{%
  \clearpage
}
\makeatother

\IfFileExists{\jobname-pw.ind}{\input{\jobname-pw.ind}}{}

\end{document}

      