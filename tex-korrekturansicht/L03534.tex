%% latex-korrekturansicht-vorspann.tex
%% Vorspann für die Korrekturansicht.
%% Lädt die gemeinsame Datei latex-vorspann.tex mit gesetztem Schalter.

\newif\ifkorrekturansicht
\korrekturansichttrue

\input{../tex-inputs/latex-vorspann}


\renewcommand{\erwaehntePersonen}{Personen: Paul Goldmann, Rudolf Gussmann, Olga Schnitzler, Heinrich Schnitzler, Elisabeth Steinrück}
\renewcommand{\erwaehnteInstitutionen}{Institutionen: Preußische Polzei, Schiller-Theater}
\renewcommand{\erwaehnteOrte}{Orte: Berlin, Dessauer Straße, Gentzgasse, Hinterbrühl, Preußen, Wien}
\renewcommand{\erwaehnteWerke}{}
\section[ Paul Goldmann an Olga Gussmann, 29. 9. {[}1902?{]}]{Paul Goldmann an Olga Gussmann, 29. 9. {[}1902?{]}}
\nopagebreak\mylabel{v}
\rehead{ }\normalsize\beginnumbering\briefempfaengerindex{Schnitzler, Olga@\textsc{Schnitzler, Olga}!zzzGoldmann, Paul@\emph{von Paul Goldmann}!1902-09-291@{29. 9. {[}1902?{]}}|(be}
\toendnotes[C]{\smallbreak\pagebreak[2]}\Standort{DLA, A:Schnitzler, HS.NZ85.1.5247.}
\physDesc{Brief, 1 Blatt, 3 Seiten, 856 Zeichen
\newline{}Handschrift: blaue Tinte, deutsche Kurrent}\toendnotes[C]{\smallbreak}
\pstart
           \noindent{}\raggedleft{}{\pb}\textcolor{gray}{\textbf{\textcolor{pink}{DESSAUERSTRASSE 19}{}\ledrightnote{\textcolor{pink}{Dessauer Straße}}}}\pend
           
\pstart
           \textcolor{pink}{Berlin}{}\ledrightnote{\textcolor{pink}{Berlin}}, 29. September.\pend
           
\pstart\center{}Liebe Freundin,\pend
\pstart
           Ich habe mich ſehr gefreut, einen Brief von Ihnen zu erhalten, weil dies das beſte
               Zeichen iſt, daß es Ihnen wohl ergeht.\pend
           
\pstart
           Das \label{K_L03534-1v}\edtext{Gewitter, das über \textsc{\textcolor{blue}{Liesl}{}\ledrightnote{\textcolor{blue}{Elisabeth Steinrück}}s} Haupt ſchwebte}{\lemma{\textnormal{\emph{Gewitter, … ſchwebte}}}\Cendnote{\textnormal{\textcolor{blue}{Elisabeth Gussmann} war ohne entsprechende
                  Dokumente für ihre Anstellung am \emph{\textcolor{brown}{Schiller-Theater}} nach \textcolor{pink}{Berlin} gezogen,
                     siehe A. S.: \emph{Tagebuch}, 25. 9. 1902.}}}\label{K_L03534-1h},
               iſt einſtweilen beſchworen. Wir haben eine Friſt von einem Monat durch Intervention
               der \textcolor{pink}{Ort}{}\ledrightnote{{$\rightarrow$}\textcolor{pink}{Berlin}}ſchaft erreicht. In
               dieſem Monat muß aber das fehlende Dokument {\pb}unbedingt beſchafft werden. Mit der \textcolor{pink}{preuß}{}\ledrightnote{{$\rightarrow$}\textcolor{pink}{Preußen}}iſchen \textcolor{brown}{Polizei}{}\ledrightnote{\textcolor{brown}{Preußische Polzei}} iſt nicht zu ſchaffen. Es genügt, daß Ihr \textcolor{blue}{Vater}{}\ledrightnote{{$\rightarrow$}\textcolor{blue}{Rudolf Gussmann}} das Verfahren wegen Erlangung ſeiner
               Zuſtändigkeit \uline{einleitet}, um die Ausſtellung eines \uline{Interims}paſſes zu ermöglichen. Dazu wird man ihn doch
               wohl zwingen können?\pend
           
\pstart
           Auf die Frage: ob es mich »noch immer« intereſſirt, wenn Sie mir von ſich und Ihrem
                  \textcolor{blue}{Buben}{}\ledrightnote{{$\rightarrow$}\textcolor{blue}{Heinrich Schnitzler}} erzählen, finde ich
                  {\pb}keine Antwort.\pend
           
\pstart
           Ich wünſche Ihnen einen glücklichen \label{K_L03534-2v}\edtext{Einzug in \textcolor{pink}{Wien}{}\ledrightnote{\textcolor{pink}{Wien}}}{\lemma{\textnormal{\emph{Einzug in Wien}}}\Cendnote{\textnormal{\textcolor{blue}{Olga Gussmann} hatte für die meiste Zeit der 
                  Schwangerschaft und die Geburt des gemeinsamen Sohnes 
                  \textcolor{blue}{Heinrich} in \textcolor{pink}{Hinterbrühl}
                  gelebt. Am 29. 9. 1902 übersiedelten sie und das Kind in die \textcolor{pink}{Gentzgasse 110},
                  wo sie bis zur Verheiratung am 26. 8. 1903
                  wohnten.}}}\label{K_L03534-2h} und bin mit
               herzlichen Grüßen an Sie und \textsc{\textcolor{blue}{Arthur}{}\ledrightnote{}}{ }{\\[\baselineskip]}Ihr ergebener {\\[\baselineskip]}\spacefill\mbox{Dr. Paul Goldmann.}\pend
           \leftskip=0em{}\endnumbering\briefempfaengerindex{Schnitzler, Olga@\textsc{Schnitzler, Olga}!zzzGoldmann, Paul@\emph{von Paul Goldmann}!1902-09-291@{29. 9. {[}1902?{]}}|)be}\mylabel{h}  \normalsize

\doendnotes{C}
\bigskip
\vfill

\clearpage

\footnotesize

\lohead{\textsc{register}}

% Definiere theindex-Environment komplett neu ohne reledmac
\makeatletter
\renewenvironment{theindex}{%
  \section*{\indexname}%
  \setlength{\parindent}{0pt}%
  \setlength{\parskip}{0pt plus 0.3pt}%
  \let\item\@idxitem
}{%
  \clearpage
}
\makeatother

\IfFileExists{\jobname-pw.ind}{\input{\jobname-pw.ind}}{}

\end{document}

      