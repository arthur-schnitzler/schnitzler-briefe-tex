%% latex-korrekturansicht-vorspann.tex
%% Vorspann für die Korrekturansicht.
%% Lädt die gemeinsame Datei latex-vorspann.tex mit gesetztem Schalter.

\newif\ifkorrekturansicht
\korrekturansichttrue

\input{../tex-inputs/latex-vorspann}


\renewcommand{\erwaehntePersonen}{Personen: Richard Specht}
\renewcommand{\erwaehnteOrte}{Orte: Wien}
\renewcommand{\erwaehnteWerke}{Werke: Der Schleier der Pierrette, Sterben. Novelle}
\section[Felix Salten an Arthur Schnitzler, {[}zwischen 16. 11. 1892 und 3. 12. 1892{]}]{Felix Salten an Arthur Schnitzler, {[}zwischen 16. 11. 1892 und
               3. 12. 1892{]}}
\nopagebreak\mylabel{v}
\rehead{ }\normalsize\beginnumbering\briefempfaengerindex{Schnitzler, Arthur@\textsc{Schnitzler, Arthur}!zzzSalten, Felix@\emph{von Felix Salten}!1892-11-161@{{[}zwischen 16. 11. 1892 und
                  3. 12. 1892{]}}|(be}
\toendnotes[C]{\smallbreak\pagebreak[2]}\Standort{CUL, Schnitzler, B 89, A 1.}
\physDesc{Brief, 1 Blatt, 2 Seiten, 201 Zeichen
\newline{}Handschrift: Bleistift, lateinische Kurrent
\newline{}Schnitzler: mit Bleistift datiert: »Ende 92« 
\newline{}Ordnung: mit Bleistift von unbekannter Hand nummeriert: »21« }\toendnotes[C]{\smallbreak}
\pstart
           \noindent{}{\pb}Lieber Freund! Ich sende Ihnen die \label{K_L03118-1v}\edtext{\textcolor{green}{Pantomime}{}\ledrightnote{\textcolor{green}{Der Schleier der Pierrette}}}{\lemma{\textnormal{\emph{Pantomime}}}\Cendnote{\textnormal{Am 15. 11. 1892 hatte \textcolor{blue}{Schnitzler} in Anwesenheit \textcolor{blue}{Salten}s seine \emph{\textcolor{green}{Pantomime}}, die Jahre
                  später als \emph{\textcolor{green}{Der Schleier der Pierrette}}
                  publiziert werden sollte, vorgelesen. Sofern dieses \textcolor{green}{Werk} gemeint war, würde das den Tag nach
                  der Lesung als frühesten möglichen Termin für das undatierte Korrespondenzstück
                  festlegen. Da \emph{\textcolor{green}{Sterben}} bereits vorlag, ist
                  anzunehmen, dass \textcolor{blue}{Salten} das \textcolor{green}{Manuskript} in Folge der Lesung der \emph{\textcolor{green}{Pantomime}} bekommen hatte. Bei dem in Folge
                  angedachten Treffen bei \textcolor{blue}{Specht} dürfte es
                  sich – sofern es stattfand – um den 4. 12. 1892 handeln, was das zeitliche Ende einer
                  möglichen Datierung bildet.}}}\label{K_L03118-1h}, da ich momentan zu müd und unwol bin, um
               selbst zu Ihnen zu kommen. Ich liege hier, und lese Ihre \label{K_L03118-2v}\edtext{\textcolor{green}{Novelle}{}\ledrightnote{{$\rightarrow$}\textcolor{green}{Sterben. Novelle}}}{\lemma{\textnormal{\emph{Novelle}}}\Cendnote{\textnormal{Am 30. 10. 1892 hatte \textcolor{blue}{Schnitzler} in Anwesenheit \textcolor{blue}{Salten}s seine Novelle \emph{\textcolor{green}{Sterben}}
                  vorgelesen.}}}\label{K_L03118-2h}.\pend
           
\pstart
           Auf Wiedersehen {\pb}eventuell
               bei \textcolor{blue}{Specht}{}\ledrightnote{\textcolor{blue}{Richard Specht}}. {\\[\baselineskip]}Herzlich {\\[\baselineskip]}Ihr {\\[\baselineskip]}\spacefill\mbox{Salten}\pend
           \leftskip=0em{}\endnumbering\briefempfaengerindex{Schnitzler, Arthur@\textsc{Schnitzler, Arthur}!zzzSalten, Felix@\emph{von Felix Salten}!1892-11-161@{{[}zwischen 16. 11. 1892 und
                  3. 12. 1892{]}}|)be}\mylabel{h}  \normalsize

\doendnotes{C}
\bigskip
\vfill

\clearpage

\footnotesize

\lohead{\textsc{register}}

% Definiere theindex-Environment komplett neu ohne reledmac
\makeatletter
\renewenvironment{theindex}{%
  \section*{\indexname}%
  \setlength{\parindent}{0pt}%
  \setlength{\parskip}{0pt plus 0.3pt}%
  \let\item\@idxitem
}{%
  \clearpage
}
\makeatother

\IfFileExists{\jobname-pw.ind}{\input{\jobname-pw.ind}}{}

\end{document}

      