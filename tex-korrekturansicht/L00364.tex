%% latex-korrekturansicht-vorspann.tex
%% Vorspann für die Korrekturansicht.
%% Lädt die gemeinsame Datei latex-vorspann.tex mit gesetztem Schalter.

\newif\ifkorrekturansicht
\korrekturansichttrue

\input{../tex-inputs/latex-vorspann}


               \section[Adele Sandrock und Arthur Schnitzler an Richard Beer-Hofmann, 29. 8. 1894]{ Adele Sandrock und Arthur Schnitzler an Richard Beer-Hofmann,
               29. 8. 1894}\nopagebreak\mylabel{v}\rehead{ }\normalsize\beginnumbering\briefempfaengerindex{Beer-Hofmann, Richard@\textsc{Beer-Hofmann, Richard}!zzzSchnitzler, Arthur@\emph{von Arthur Schnitzler}!1894-08-291@{29. 8. 1894}|(be}\briefempfaengerindex{Beer-Hofmann, Richard@\textsc{Beer-Hofmann, Richard}!zzzSandrock, Adele@\emph{von Adele Sandrock}!1894-08-291@{29. 8. 1894}|(be} \toendnotes[C]{\smallbreak\pagebreak[2]} \Standort{YCGL, MSS 31.}
\physDesc{Brief, 1 Blatt, 2 Seiten, Umschlag
\newline{}Handschrift Arthur Schnitzler: schwarze Tinte\newline{}Handschrift Adele Sandrock: schwarze Tinte, deutsche Kurrent\newline{}Versand: ohne postalischen Übermittlungsvermerk }\buchAbdrucke{\weitereDrucke{Arthur Schnitzler, Richard Beer-Hofmann: \emph{Briefwechsel 1891–1931}. Hg. Konstanze Fliedl. Wien, Zürich: \emph{Europaverlag} 1992, S. 58.} }\pstart{}{\pb}\textsc{Herrn Dr. Richard Beer-Hofmann}\pend{}\pstart{}\textsc{in}\pend{}\pstart{}\textsc{\textcolor{pink}{Ischl}{}\ledrightnote{\textcolor{pink}{Bad Ischl}}}\pend{}\pstart{}\textsc{\textcolor{pink}{Egelmoos 22}{}\ledrightnote{\textcolor{pink}{Eglmoosgasse}}.}\pend{}{\bigskip}\pstart
           \raggedleft{}{\pb}29. Aug 94{\\}\textcolor{pink}{Ischl}{}\ledrightnote{\textcolor{pink}{Bad Ischl}}\pend
           \pstart{}Meine Herren!\pend\pstart
           \uline{Wir gehen um }\uline{6}\uline{, }\substVorne{}\textsuperscript{\uline{6}}\substDazwischen{}\uline{7}\substHinten{}\uline{ Uhr}\uline{ jedenfalls \textcolor{pink}{\textsc{Eglmoos 22}}{}\ledrightnote{\textcolor{pink}{Eglmoosgasse}} vorbei} und werden
               pfeifen oder auch nicht pfeifen. Sie werden zu Hauſe ſein oder auch nicht zu Hauſe
               ſein. Im Falle wir uns nicht {\pb}treffen, bin ich
               (die Tragödin Adele Sandrock) vor zehn Uhr im \textcolor{pink}{Hotel
                  Bauer}{}\ledrightnote{\textcolor{pink}{Hotel Bauer}}{ }ſoupirend anzutreffen. Ich (der Dramatiker Arthur Schnitzler)
               ſpeise \substVorne{}\textsuperscript{L}\substDazwischen{}½ 9\substHinten{} beim \textcolor{pink}{Leopold}{}\ledrightnote{\textcolor{pink}{Hotel und Pension Rudolfshöhe (Leopold Petter)}}, wo ich Sie, meine Herren,
               jedenfalls zu ſehen hoffe.\pend
           \pstart
           Herzliche Grüße{\\[\baselineskip]}\spacefill\mbox{Sandrock A.}{\\[\baselineskip]}\spacefill\mbox{{[}hs. Schnitzler:{]} Schnitzler}\pend
           \leftskip=0em{}\endnumbering\briefempfaengerindex{Beer-Hofmann, Richard@\textsc{Beer-Hofmann, Richard}!zzzSchnitzler, Arthur@\emph{von Arthur Schnitzler}!1894-08-291@{29. 8. 1894}|)be}\briefempfaengerindex{Beer-Hofmann, Richard@\textsc{Beer-Hofmann, Richard}!zzzSandrock, Adele@\emph{von Adele Sandrock}!1894-08-291@{29. 8. 1894}|)be}\mylabel{h}  \normalsize

\doendnotes{C}
\bigskip
\vfill

\clearpage

\footnotesize

\lohead{\textsc{register}}

% Definiere theindex-Environment komplett neu ohne reledmac
\makeatletter
\renewenvironment{theindex}{%
  \section*{\indexname}%
  \setlength{\parindent}{0pt}%
  \setlength{\parskip}{0pt plus 0.3pt}%
  \let\item\@idxitem
}{%
  \clearpage
}
\makeatother

\IfFileExists{\jobname-pw.ind}{\input{\jobname-pw.ind}}{}

\end{document}

      