%% latex-korrekturansicht-vorspann.tex
%% Vorspann für die Korrekturansicht.
%% Lädt die gemeinsame Datei latex-vorspann.tex mit gesetztem Schalter.

\newif\ifkorrekturansicht
\korrekturansichttrue

\input{../tex-inputs/latex-vorspann}


\renewcommand{\erwaehntePersonen}{Personen: Felix Salten}
\renewcommand{\erwaehnteOrte}{Orte: Berlin, Sternwartestraße 71, Wien}
\renewcommand{\erwaehnteWerke}{}
\section[ Felix Salten an Arthur Schnitzler, 24. 12. {[}1929?{]}]{Felix Salten an Arthur Schnitzler, 24. 12. {[}1929?{]}}
\nopagebreak\mylabel{v}
\rehead{ }\normalsize\beginnumbering\briefempfaengerindex{Schnitzler, Arthur@\textsc{Schnitzler, Arthur}!zzzSalten, Felix@\emph{von Felix Salten}!1929-12-241@{24. 12. {[}1929?{]}}|(be}
\toendnotes[C]{\smallbreak\pagebreak[2]}\Standort{CUL, Schnitzler, B 89, B 2.}
\physDesc{Telegramm, 1 Blatt, 1 Seite, 213 Zeichen
\newline{}maschinell
\newline{}Versand: 1) gestempelt am Vordruck: »\noindent{}\textcolor{gray}{\textbf{Aufgenommen von}}{ }\textcolor{gray}{\textbf{\textit{B 25}}}{ / }\textcolor{gray}{\textbf{auf Ltg. Nr.}}{ }\textcolor{gray}{\textbf{\textit{24/12}}}{ / }\textcolor{gray}{\textbf{\textit{K\textcolor{gray}{L}G}}}«  2) Stempel: »\nobreak{}24 Dec\textcolor{gray}{.}, 13, Ausgefertigt\nobreak{}«. 
\newline{}Schnitzler: mit rotem Buntstift datiert: »24/12 \textcolor{gray}{9}« und zwei Unterstreichungen }\pstart{}{\pb}schnitzler\pend{}\pstart{}\textcolor{pink}{sternwartestrasse 71}{}\ledrightnote{\textcolor{pink}{Sternwartestraße 71}}\pend{}\pstart{}\textcolor{pink}{wien}{}\ledrightnote{\textcolor{pink}{Wien}} =\pend{}
{\bigskip}
\pstart
           54 \textcolor{pink}{berlin}{}\ledrightnote{\textcolor{pink}{Berlin}} /50 757 27/26 24 1215 /\pend
           
\pstart
           wir alle denken in diesen tagen vol freundschaft liebe und verehrung an sie und
               senden tausend herzliche wuensche aufrichtig = ihr \spacefill\mbox{felix salten +}\pend
           \endnumbering\briefempfaengerindex{Schnitzler, Arthur@\textsc{Schnitzler, Arthur}!zzzSalten, Felix@\emph{von Felix Salten}!1929-12-241@{24. 12. {[}1929?{]}}|)be}\mylabel{h}  \normalsize

\doendnotes{C}
\bigskip
\vfill

\clearpage

\footnotesize

\lohead{\textsc{register}}

% Definiere theindex-Environment komplett neu ohne reledmac
\makeatletter
\renewenvironment{theindex}{%
  \section*{\indexname}%
  \setlength{\parindent}{0pt}%
  \setlength{\parskip}{0pt plus 0.3pt}%
  \let\item\@idxitem
}{%
  \clearpage
}
\makeatother

\IfFileExists{\jobname-pw.ind}{\input{\jobname-pw.ind}}{}

\end{document}

      