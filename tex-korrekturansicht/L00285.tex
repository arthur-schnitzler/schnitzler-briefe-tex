%% latex-korrekturansicht-vorspann.tex
%% Vorspann für die Korrekturansicht.
%% Lädt die gemeinsame Datei latex-vorspann.tex mit gesetztem Schalter.

\newif\ifkorrekturansicht
\korrekturansichttrue

\input{../tex-inputs/latex-vorspann}


               \section[Arthur Schnitzler an Hermann Bahr, 2. 12. 1893]{ Arthur Schnitzler an Hermann Bahr, 2. 12. 1893}\nopagebreak\mylabel{v}\rehead{ }\pwindex{XXXX Abgedrucktes Werk, Nummer nicht vorhanden|pwt}\normalsize\beginnumbering\briefempfaengerindex{Bahr, Hermann@\textsc{Bahr, Hermann}!zzzSchnitzler, Arthur@\emph{von Arthur Schnitzler}!1893-12-021@{2. 12. 1893}|(be} \toendnotes[C]{\smallbreak\pagebreak[2]} \Standort{TMW, FS PK266826.}
\physDesc{Fotografie von Carl Pietzner
\newline{}Handschrift: schwarze Tinte, deutsche Kurrent\newline{}Zusatz: von unbekannter Hand Plattennummer auf der Rückseite vermerkt:
                                    »13955.« }\buchAbdrucke{\weitereDrucke{Hermann Bahr, Arthur Schnitzler: \emph{Briefwechsel, Aufzeichnungen, Dokumente (1891–1931)}. Hg. Kurt Ifkovits und Martin Anton Müller. Göttingen: \emph{Wallstein} 2018, S. 54.} }\pstart{[}Abbildung{]}\pend\pstart
           \noindent{}{\pb}Meinem lieben Freund
                  \textsc{Hermann Bahr} in herzlicher Verehrung\pend
           \pstart \spacefill\mbox{ArthSch}\pend{}\pstart
           \noindent{}\uline{\textcolor{pink}{Wien}{}\ledrightnote{\textcolor{pink}{Wien}}, 2. 12
                     93.}\pend
           \pstart
           \centering{}\textcolor{gray}{\textbf{\textcolor{blue}{\textsc{K. u. k. Hof-Photograph C. Pietzner}}{}\ledrightnote{\textcolor{blue}{Carl Pietzner}}}}\pend
           \endnumbering\briefempfaengerindex{Bahr, Hermann@\textsc{Bahr, Hermann}!zzzSchnitzler, Arthur@\emph{von Arthur Schnitzler}!1893-12-021@{2. 12. 1893}|)be}\mylabel{h}  \normalsize

\doendnotes{C}
\bigskip
\vfill

\clearpage

\footnotesize

\lohead{\textsc{register}}

% Definiere theindex-Environment komplett neu ohne reledmac
\makeatletter
\renewenvironment{theindex}{%
  \section*{\indexname}%
  \setlength{\parindent}{0pt}%
  \setlength{\parskip}{0pt plus 0.3pt}%
  \let\item\@idxitem
}{%
  \clearpage
}
\makeatother

\IfFileExists{\jobname-pw.ind}{\input{\jobname-pw.ind}}{}

\end{document}

      