%% latex-korrekturansicht-vorspann.tex
%% Vorspann für die Korrekturansicht.
%% Lädt die gemeinsame Datei latex-vorspann.tex mit gesetztem Schalter.

\newif\ifkorrekturansicht
\korrekturansichttrue

\input{../tex-inputs/latex-vorspann}


               \section[Paul Goldmann an Arthur Schnitzler, 12. 1. {[}1895{]}]{ Paul Goldmann an Arthur Schnitzler, 12. 1. {[}1895{]}}\nopagebreak\mylabel{v}\rehead{ }\normalsize\beginnumbering\briefempfaengerindex{Schnitzler, Arthur@\textsc{Schnitzler, Arthur}!zzzGoldmann, Paul@\emph{von Paul Goldmann}!1895-01-121@{12. 1. {[}1895{]}}|(be} \toendnotes[C]{\smallbreak\pagebreak[2]} \Standort{DLA, A:Schnitzler, HS.NZ85.1.3165.}
\physDesc{Brief, 1 Blatt, 3 Seiten
\newline{}Handschrift: schwarze Tinte, deutsche Kurrent
\newline{}Schnitzler: 1) mit Bleistift das Jahr »95« vermerkt 2) mit rotem Buntstift eine Unterstreichung}\toendnotes[C]{\smallbreak}\pstart
           \noindent{}{\pb}\textcolor{gray}{\textbf{\textbf{\textcolor{brown}{Frankfurter Zeitung}{}\ledrightnote{\textcolor{brown}{Frankfurter Zeitung}}}}}\pend
           \pstart
           \textcolor{gray}{\textbf{(\textcolor{brown}{\begin{otherlanguage}{french}Gazette de Francfort\end{otherlanguage}}{}\ledrightnote{\textcolor{brown}{Frankfurter Zeitung}}). }}\pend
           \pstart
           \textcolor{gray}{\textbf{\textbf{\begin{otherlanguage}{french}Fondateur M. \textcolor{blue}{L.
                              Sonnemann}{}\ledrightnote{\textcolor{blue}{Leopold Sonnemann}}\end{otherlanguage}.}}}\pend
           \pstart
           \begin{otherlanguage}{french}\textcolor{gray}{\textbf{\textcolor{green}{Journal}{}\ledrightnote{\textcolor{green}{Frankfurter Zeitung}} politique, financier,}}\end{otherlanguage}\hfill \textsc{\textcolor{pink}{Paris}{}\ledrightnote{\textcolor{pink}{Paris}}}, 12. Januar.\pend
           \pstart
           \begin{otherlanguage}{french}\textcolor{gray}{\textbf{commercial et littéraire.}}\end{otherlanguage}\pend
           \pstart
           \begin{otherlanguage}{french}\textcolor{gray}{\textbf{\textbf{Paraissant trois fois par jour.}}}\end{otherlanguage}\pend
           \pstart
           \begin{otherlanguage}{french}\textcolor{gray}{\textbf{\textbf{Bureau à \textcolor{pink}{Paris}{}\ledrightnote{\textcolor{pink}{Paris}}:}}}\end{otherlanguage}\pend
           \pstart
           \begin{otherlanguage}{french}\textcolor{gray}{\textbf{\textbf{\textcolor{pink}{24. Rue Feydeau}{}\ledrightnote{\textcolor{pink}{rue Feydeau}}.}}}\end{otherlanguage}\pend
           \pstart\center{}Mein lieber Freund,\pend\pstart
           \textsc{\textcolor{blue}{Lalo}{}\ledrightnote{\textcolor{blue}{Pierre Lalo}}}, vom »\textsc{\textcolor{brown}{Journal des Débats}{}\ledrightnote{\textcolor{brown}{Journal des débats}}}«, war geſtern bei mir. »\textcolor{green}{Sterben}{}\ledrightnote{\textcolor{green}{Sterben. Novelle}}« hat ihm ungemein gefallen, \textsc{\textcolor{blue}{Richard}{}\ledrightnote{\textcolor{blue}{Richard Beer-Hofmann}}s}{ }\textcolor{green}{Buch}{}\ledrightnote{→\textcolor{green}{Novellen}} weniger
               (ſags ihm aber nicht). Er hat beſtimmt verſprochen, über Euch zu \label{K_L02727-1v}\edtext{ſchreiben}{\lemma{\textnormal{\emph{ſchreiben}}}\Cendnote{\textnormal{Nicht über \textcolor{blue}{Richard Beer-Hofmann},
                     jedoch über \textcolor{blue}{Schnitzler} und seine \textcolor{green}{Novelle}{ }\emph{\textcolor{green}{Sterben}} schrieb \textcolor{blue}{Pierre Lalo} am 21. 3. 1895: [\textcolor{blue}{Pierre Lalo}]:
                        \emph{\textcolor{green}{Au jour le jour. M. \textcolor{blue}{Arthur Schnitzler}}}. In: \emph{\textcolor{green}{Journal des débats. Politiques et
                        littéraires}}, Jg. 107, 21. 3. 1895,
                     S. 1.}}}\label{K_L02727-1h}. Ob ers halten wird???\pend
           \pstart
           Bitte, ſchick’ mir \textsc{\textcolor{blue}{Torresani}{}\ledrightnote{\textcolor{blue}{Carl von Torresani-Lanzenfeld}}s} Adreſſe.\pend
           \pstart
           Hat Frl. \textsc{\textcolor{blue}{Sandrock}{}\ledrightnote{\textcolor{blue}{Adele Sandrock}}} meine Briefe erhalten?\pend
           \pstart
           \textcolor{blue}{\textcolor{pink}{Franzoſ}{}\ledrightnote{→\textcolor{pink}{Frankreich}}en}{}\ledrightnote{→\textcolor{blue}{Maurice Donnay}{\newline}→\textcolor{blue}{Paul Ernest Hervieu}{\newline}→\textcolor{blue}{Georges d' Esparbès}{\newline}→\textcolor{blue}{Abel Hermant}{\newline}→\textcolor{blue}{Henri Léon Lavedan}{\newline}→\textcolor{blue}{Fernand Vandérem}{\newline}→\textcolor{blue}{Alfred Capus}{\newline}→\textcolor{blue}{François de Nion}{\newline}→\textcolor{blue}{Henry de Fleurigny}{\newline}→\textcolor{blue}{Georges Courteline}{\newline}→\textcolor{blue}{Jean Ajalbert}{\newline}→\textcolor{blue}{Léon Xanrof}{\newline}→\textcolor{blue}{Jules Renard}{\newline}→\textcolor{blue}{Henri Antoine Jules-Bois}{\newline}→\textcolor{blue}{Jules Case}{\newline}→\textcolor{blue}{Paul Adam}}, die kleine
               Geſchichten ſchreiben, ſind: \textsc{\textcolor{blue}{Maurice Donnay}{}\ledrightnote{\textcolor{blue}{Maurice Donnay}}}, \textsc{\textcolor{blue}{Paul Hervieu}{}\ledrightnote{\textcolor{blue}{Paul Ernest Hervieu}}}, {\pb}\textsc{\textcolor{blue}{Georges d'Esparbès}{}\ledrightnote{\textcolor{blue}{Georges d' Esparbès}}}, \textsc{\textcolor{blue}{Abel Hermant}{}\ledrightnote{\textcolor{blue}{Abel Hermant}}}, \textsc{\strikeout{\textcolor{blue}{Hen}{}\ledrightnote{→\textcolor{blue}{Henri Léon Lavedan}}}}{ }\textsc{\textcolor{blue}{Henri \strikeout{La}
                     Lavedan}{}\ledrightnote{\textcolor{blue}{Henri Léon Lavedan}}}, \textsc{\textcolor{blue}{Ferdinand Vanderem}{}\ledrightnote{\textcolor{blue}{Fernand Vandérem}}}, \textsc{\textcolor{blue}{Alfred Capus}{}\ledrightnote{\textcolor{blue}{Alfred Capus}}}, \textsc{\textcolor{blue}{François de Nion}{}\ledrightnote{\textcolor{blue}{François de Nion}}, \textcolor{blue}{Henry de Fleurigny}{}\ledrightnote{\textcolor{blue}{Henry de Fleurigny}}}, \textsc{\textcolor{blue}{Georges Courteline}{}\ledrightnote{\textcolor{blue}{Georges Courteline}}}, \textsc{\textcolor{blue}{Jean Ajalbert}{}\ledrightnote{\textcolor{blue}{Jean Ajalbert}}, \textcolor{blue}{L. Xanrof}{}\ledrightnote{\textcolor{blue}{Léon Xanrof}}}, \textsc{\textcolor{blue}{Jules Renard}{}\ledrightnote{\textcolor{blue}{Jules Renard}}}, \textsc{\textcolor{blue}{Jules Bois}{}\ledrightnote{\textcolor{blue}{Henri Antoine Jules-Bois}}}, \textsc{\textcolor{blue}{Jules Case}{}\ledrightnote{\textcolor{blue}{Jules Case}}}, \textsc{\textcolor{blue}{Paul Adam}{}\ledrightnote{\textcolor{blue}{Paul Adam}}}{ }\textsc{etc}.\pend
           \pstart
           Wenn Du damit nicht genug haſt, kannſt Du mehr bekommen. Meiſtens ſind ſie recht
               mäßig. Die gegenwärtig aufgehende Saat iſt nicht gut gerathen. Außer den verwöhnten
                  \textcolor{blue}{Mode-Pinſeln}{}\ledrightnote{→\textcolor{blue}{Marcel Prévost}{\newline}→\textcolor{blue}{Abel Hermant}{\newline}→\textcolor{blue}{Fernand Vandérem}} (\textsc{\textcolor{blue}{Prevost}{}\ledrightnote{\textcolor{blue}{Marcel Prévost}}}, \textsc{\textcolor{blue}{Hermant}{}\ledrightnote{\textcolor{blue}{Abel Hermant}}, \textcolor{blue}{Vanderem}{}\ledrightnote{\textcolor{blue}{Fernand Vandérem}}}), kann man ſie zum Überſetzen zweifellos {\pb}billig, meiſt umſonſt bekommen. Man ſchreibt Ihnen: \label{K_L02727-2v}\edtext{\textsc{\begin{otherlanguage}{french}Nous serions très-heureux d’obtenir l’autorisation de
                     traduire {\dotssix} Céla servirait comme échantillon de vos
                     œuvres pour vous introduire auprès du public \textcolor{pink}{autrich}{}\ledrightnote{\textcolor{pink}{Österreich}}ièn.\end{otherlanguage}}}{\lemma{\textnormal{\emph{Nous … autrichièn.}}}\Cendnote{\textnormal{französisch: Wir würden uns sehr freuen,
                  wenn wir die Erlaubnis bekämen, {\dotssix} zu übersetzen. Dies
                  würde als Kostprobe Ihrer Werke dienen, um Sie dem \textcolor{pink}{österreich}ischen Publikum bekannt zu machen.}}}\label{K_L02727-2h} So
               natürlich nur den Unbekannten. Die Bekannten ſetzen voraus, daß man in \textcolor{pink}{Wien}{}\ledrightnote{\textcolor{pink}{Wien}} nichts mehr lieſt, als ſie. Oder aber man
               ſchreibt gar nicht. Wer kümmert ſich in \textsc{\textcolor{pink}{Paris}{}\ledrightnote{\textcolor{pink}{Paris}}} um die \label{K_L02727-3v}\edtext{\textcolor{brown}{Allgemeine Zeitung}{}\ledrightnote{\textcolor{brown}{Wiener Allgemeine Zeitung}}}{\lemma{\textnormal{\emph{Allgemeine Zeitung}}}\Cendnote{\textnormal{Seit Oktober 1894 war \textcolor{blue}{Felix Salten}
                  bei der \emph{\textcolor{brown}{Wiener Allgemeinen Zeitung}} engagiert, was ein möglicher 
                  Hintergrund für die Anfrage darstellt. Ob \textcolor{blue}{Schnitzler} überlegte, selbst
                  durch Übersetzungen sich einen Verdienst zu verschaffen, ist ungewiss.}}}\label{K_L02727-3h}?\pend
           \pstart
           Herzlichſt {\\[\baselineskip]}Dein {\\[\baselineskip]}\spacefill\mbox{Paul Goldmann}\pend
           \leftskip=0em{}\endnumbering\briefempfaengerindex{Schnitzler, Arthur@\textsc{Schnitzler, Arthur}!zzzGoldmann, Paul@\emph{von Paul Goldmann}!1895-01-121@{12. 1. {[}1895{]}}|)be}\mylabel{h}  \normalsize

\doendnotes{C}
\bigskip
\vfill

\clearpage

\footnotesize

\lohead{\textsc{register}}

% Definiere theindex-Environment komplett neu ohne reledmac
\makeatletter
\renewenvironment{theindex}{%
  \section*{\indexname}%
  \setlength{\parindent}{0pt}%
  \setlength{\parskip}{0pt plus 0.3pt}%
  \let\item\@idxitem
}{%
  \clearpage
}
\makeatother

\IfFileExists{\jobname-pw.ind}{\input{\jobname-pw.ind}}{}

\end{document}

      