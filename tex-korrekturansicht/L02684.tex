%% latex-korrekturansicht-vorspann.tex
%% Vorspann für die Korrekturansicht.
%% Lädt die gemeinsame Datei latex-vorspann.tex mit gesetztem Schalter.

\newif\ifkorrekturansicht
\korrekturansichttrue

\input{../tex-inputs/latex-vorspann}


               \section[Arthur Schnitzler an Paul Goldmann, 21. 11. 1896]{ Arthur Schnitzler an Paul Goldmann, 21. 11. 1896}\nopagebreak\mylabel{v}\rehead{ }\normalsize\beginnumbering\briefempfaengerindex{Goldmann, Paul@\textsc{Goldmann, Paul}!zzzSchnitzler, Arthur@\emph{von Arthur Schnitzler}!1896-11-211@{21. 11. 1896}|(be} \toendnotes[C]{\smallbreak\pagebreak[2]} \Standort{DLA, A:Schnitzler, HS85.1.5681.}
\physDesc{Telegramm, Fotokopie
\newline{}maschinell\newline{}Ordnung: mit blauem Kugelschreiber von unbekannter Hand teilweise den
                                 schwer leserlichen Text nachgezogen \newline{}Zusatz: Von den Korrespondenzstücken \textcolor{blue}{Schnitzler}s an \textcolor{blue}{Goldmann} fehlt weitgehend jede Spur. In der Edition von
                                    \textcolor{green}{Ritterlichkeit}
                                    (1975) schreibt die Herausgeberin \textcolor{blue}{Rena R. Schlein}: »Zwei Telegramme
                                    und ein Brief \textcolor{blue}{Schnitzler}s
                                    an \textcolor{blue}{Goldmann} wurden mir
                                    von Dr. \textcolor{blue}{Leo P. Reckford},
                                    der diese Dokumente von der Familie \textcolor{blue}{Goldmann}s zum Geschenk bekam, für meine
                                    Arbeit zur Verfügung gestellt« (S. 1). \textcolor{blue}{Reckford} starb 1988, seine
                                 Nachkommen haben keine Kenntnis von diesen (und etwaigen weiteren)
                                 Korrespondenzstücken und sie sind auch nicht auffindbar. \textcolor{blue}{Rena R. Schlein} kam
                                    1919 zur Welt. Ein Kontakt konnte nicht hergestellt
                                 werden. Die Kopie des vorliegenden Telegramms dürfte durch \textcolor{blue}{Reckford} oder \textcolor{blue}{Schlein} in den Besitz \textcolor{blue}{Heinrich Schnitzlers} gelangt
                                 sein. }\buchAbdrucke{\weitereDrucke{\pwindex{Ritterlichkeit@\emph{Ritterlichkeit}|pwk}Arthur Schnitzler: \emph{Ritterlichkeit. Fragment aus dem Nachlaß}. Bonn: \emph{Bouvier Verlag Herbert Grundmann} 1975, S. 5 (Abhandlungen zur Kunst-, Musik- und
                        Literaturwissenschaft, 176).} }\toendnotes[C]{\smallbreak}\pstart{}{\pb}PAUL GOLDMANN{ }\textcolor{pink}{PARIS}{}\ledrightnote{\textcolor{pink}{Paris}}\pend{}\pstart{}\textcolor{pink}{24 RUE FEYDEAU}{}\ledrightnote{\textcolor{pink}{rue Feydeau}}\pend{}{\bigskip}\pstart
           \noindent{}\centering{}{\pb}FR \textcolor{pink}{WIEN}{}\ledrightnote{\textcolor{pink}{Wien}} 72\textcolor{gray}{×}\-\textcolor{gray}{×}685\pend
           \pstart
           = SENDE MIR SOFORT \label{K_L02684-1v}\edtext{NACHRICHT}{\lemma{\textnormal{\emph{Nachricht}}}\Cendnote{\textnormal{Entrüstet über \textcolor{blue}{Goldmann}s Berichterstattung über die \textcolor{blue}{Dreyfus}-Affäre für die \emph{\textcolor{brown}{Frankfurter Zeitung}} (\textcolor{blue}{G.} [=\textcolor{blue}{Paul Goldmann}]: \emph{\textcolor{green}{Die
                           Enthüllungen über die Affaire Dreyfus}}, Jg. 41, Nr. 258,
                           16. 9. 1896, Erstes Morgenblatt, S. 1. \textcolor{blue}{G.} [=\textcolor{blue}{Paul Goldmann}]: \emph{\textcolor{green}{Die
                           Affaire Dreyfus}}, Jg. 41, Nr. 314, 11. 11. 1896, Zweites
                        Morgenblatt, S. 1. \textcolor{blue}{G.} [=\textcolor{blue}{Paul Goldmann}]: \emph{\textcolor{green}{Dreyfus, die öffentliche Meinung und die deutsche Regierung}},
                        Jg. 41, Nr. 315, 12. 11. 1896, Erstes Morgenblatt,
                     S. 1.), in der für die Wiederaufnahme des Prozesses gegen \textcolor{blue}{Dreyfus} Partei ergriffen wurde, hatte der
                     antisemitische Chefredakteur \textcolor{blue}{Lucien
                        Millevoye} über ihn geschrieben (\emph{\textcolor{green}{Justice!}} In: \emph{\textcolor{green}{La Patrie}}, Jg. 56, 15. 11. 1896,
                        S. 1.): »\begin{otherlanguage}{french}Le lâche coquin se croit à l’arbi.\end{otherlanguage}« – Der ungezogene Feigling glaubt sich in Sicherheit. Daraufhin wurde
                     er von \textcolor{blue}{Goldmann} zum Pistolenduell
                     gefordert. \textcolor{blue}{Goldmann}s \textcolor{blue}{Sekundanten} waren die
                     Journalisten \textcolor{blue}{Félix Fénéon} und \textcolor{blue}{Rowland Strong}. Nach zwei Kugelwechseln
                     mit 25 Schritt Abstand war niemand verletzt. vgl. A. S.: \emph{Tagebuch}, 23. 11. 1896, ungezeichnete \textcolor{green}{Notiz} in: \emph{\textcolor{green}{Le Petit Parisien}}, Jg. 21, Nr. 7.331,
                           22. 11. 1896, S. 2 und \emph{\textcolor{green}{Wiener Zeitung}}, Nr. 272,
                           22. 11. 1896, S. 11.}}}\label{K_L02684-1h} DEIN\pend
           \pstart \spacefill\mbox{ARTHUR +}\pend{}\endnumbering\briefempfaengerindex{Goldmann, Paul@\textsc{Goldmann, Paul}!zzzSchnitzler, Arthur@\emph{von Arthur Schnitzler}!1896-11-211@{21. 11. 1896}|)be}\mylabel{h}\begin{anhang}\end{anhang}\normalsize

\doendnotes{C}
\bigskip
\vfill

\clearpage

\footnotesize

\lohead{\textsc{register}}

% Definiere theindex-Environment komplett neu ohne reledmac
\makeatletter
\renewenvironment{theindex}{%
  \section*{\indexname}%
  \setlength{\parindent}{0pt}%
  \setlength{\parskip}{0pt plus 0.3pt}%
  \let\item\@idxitem
}{%
  \clearpage
}
\makeatother

\IfFileExists{\jobname-pw.ind}{\input{\jobname-pw.ind}}{}

\end{document}

      