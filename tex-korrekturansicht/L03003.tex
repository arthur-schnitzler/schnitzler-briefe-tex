%% latex-korrekturansicht-vorspann.tex
%% Vorspann für die Korrekturansicht.
%% Lädt die gemeinsame Datei latex-vorspann.tex mit gesetztem Schalter.

\newif\ifkorrekturansicht
\korrekturansichttrue

\input{../tex-inputs/latex-vorspann}


\renewcommand{\erwaehntePersonen}{Personen: Julius Bauer,  Engel, Grete Hofmann, Siegfried Jacobsohn, Arthur Kaufmann, Alfred Kerr, Felix Salten, Ottilie Salten, Olga Schnitzler, Elisabeth Steinrück, Siegfried Trebitsch, Jakob Wassermann, Julie Wassermann}
\renewcommand{\erwaehnteOrte}{Orte: Bayern, Berlin, Deutschland, Dänemark, Edmund-Weiß-Gasse 7, Genf, Harz, Kopenhagen, Lugano, Marienlyst, Nordpol, Norwegen, Ostsee, Russland, Rügen, Salzkammergut, Schweden, Südtirol, Tirol, Volkstheater in Rudolfsheim, Wien, Öresund}
\renewcommand{\erwaehnteWerke}{Werke: B.Z. am Mittag, Bund der Bühnendichter. II, Bühnenvertrieb, Der Ruf des Lebens. Schauspiel in drei Akten, Die Schaubühne, Die letzten Masken, Die neue Rundschau, Hamlet, Russisches Theater. II, Zum großen Wurstel. Burleske in einem Akt, »Kater Lampe«, Ödipus und der Ruf des Lebens}
\section[ Arthur Schnitzler an Felix Salten, 2. 4. 1906]{Arthur Schnitzler an Felix Salten, 2. 4. 1906}
\nopagebreak\mylabel{v}
\rehead{ }\normalsize\beginnumbering\briefempfaengerindex{Salten, Felix@\textsc{Salten, Felix}!zzzSchnitzler, Arthur@\emph{von Arthur Schnitzler}!1906-04-021@{2. 4. 1906}|(be}
\toendnotes[C]{\smallbreak\pagebreak[2]}\Standort{Wienbibliothek im Rathaus, ZPH 1681, 2.1.516.}
\physDesc{Brief, 2 Blätter, 8 Seiten, 3559 Zeichen
\newline{}Handschrift: schwarze Tinte, deutsche Kurrent
\newline{}Ordnung: mit Bleistift von unbekannter Hand Nummerierung der Doppelseiten des
                                 Konvoluts: »20«–»23«  }\toendnotes[C]{\smallbreak}
\pstart
           \noindent{}{\pb}\textcolor{gray}{\textbf{Dr. Arthur Schnitzler}}\hfill 2. April 906\pend
           
\pstart
           \textcolor{gray}{\textbf{\textcolor{pink}{Wien, XVIII. Spoettelgasse 7}{}\ledrightnote{\textcolor{pink}{Edmund-Weiß-Gasse 7}}.}}\pend
           
\pstart
           lieber, vor einigen Wochen ſchrieb mir \textcolor{blue}{Liesl}{}\ledrightnote{\textcolor{blue}{Elisabeth Steinrück}}, daſs ihr ein Bekannter, namens \textcolor{blue}{Engel}{}\ledrightnote{\textcolor{blue}{Engel}}, eine ermäßigte \label{K_L03003-1v}\edtext{Seereiſe}{\lemma{\textnormal{\emph{Seereiſe}}}\Cendnote{\textnormal{vermutlich Bezug auf ihre Reise nach \textcolor{pink}{Dänemark}, siehe A. S.: \emph{Tagebuch}, 9. 7. 1906}}}\label{K_L03003-1h} verſchaffen werde; daſs ſie ſich nun in dieſer Sache an Sie zu wenden ſcheint
               (wie mir Ihr letzter Brief andeutet) iſt mir\textcolor{gray}{,} wie Sie ſich
               denken können, ſo wenig recht als möglich. – Meinen \label{K_L03003-2v}\edtext{begeiſterten Brief}{\lemma{\textnormal{\emph{begeiſterten Brief}}}\Cendnote{\textnormal{siehe Felix Salten an Arthur Schnitzler, 28. 3. 1906}}}\label{K_L03003-2h} an \textcolor{blue}{Trebitſch}{}\ledrightnote{\textcolor{blue}{Siegfried Trebitsch}} kö{\geminationn}en Sie ſich ja ungefähr vorſtellen. Er ſchrieb mir
               gleich nach Erſcheinen jenes \textcolor{green}{Artikel}{}\ledrightnote{{$\rightarrow$}\textcolor{green}{Bühnenvertrieb}}s in der \textcolor{green}{Schb.}{}\ledrightnote{\textcolor{green}{Die Schaubühne}} ich ſolle ihn
               »beruhigen«. Ich hab {\pb}ihn beruhigt. Im
               übrigen hat die Bühnenvertriebsſache ſchon \substVorne{}\textsuperscript{I}\substDazwischen{}i\substHinten{}hre Bedeutung. Nur muſs ſie in Zuſa{\geminationm}enhang mit
               den andern Fragen behandelt werden, die ſich auf das Verhältnis des Autors zu ſeiner
               geſchäftl. Umwelt beziehen. Einige dieſer Fragen hab ich \label{K_L03003-3v}\edtext{in einem \textcolor{green}{Brief}{}\ledrightnote{{$\rightarrow$}\textcolor{green}{Bund der Bühnendichter. II}}}{\lemma{\textnormal{\emph{in einem Brief}}}\Cendnote{\textnormal{\emph{\textcolor{green}{Bund der Bühnendichter. II}} In: \emph{\textcolor{green}{Die Schaubühne}}, Jg. 33, Nr. 11.176, 12. 4. 1906, S. 10. Siehe A. S.: \emph{»Das Zeitlose ist von kürzester Dauer«}, Bund der Bühnendichter, 12. 4. 1906.}}}\label{K_L03003-3h} an \textcolor{blue}{Jacobsohn}{}\ledrightnote{\textcolor{blue}{Siegfried Jacobsohn}} kurz formulirt. –\pend
           
\pstart
           Nun unſre Radreiſe »oder ſo«. Wenn Sie irgendwas \textcolor{pink}{deutſch}{}\ledrightnote{{$\rightarrow$}\textcolor{pink}{Deutschland}}es, \textcolor{pink}{Thüringer Harz}{}\ledrightnote{\textcolor{pink}{Harz}}{ }\textsc{etc} vorziehen, ſo möchte ich dieſe Reiſe mehr gegen den
                  So{\geminationm}er verſchieben, etwa gegen Mitte Juli, um dann gleich das \textcolor{pink}{Seebad}{}\ledrightnote{{$\rightarrow$}\textcolor{pink}{Marienlyst}} an{\pb}ſchließen zu
               können. Ziehen Sie \textcolor{pink}{Tirol}{}\ledrightnote{\textcolor{pink}{Tirol}{\newline}\textcolor{pink}{Südtirol}} ev. \textcolor{pink}{Salzka{\geminationm}ergut}{}\ledrightnote{\textcolor{pink}{Salzkammergut}}, (\textcolor{pink}{bayriſches Hochgebirge}{}\ledrightnote{\textcolor{pink}{Bayern}}?) vor, ſo ſchlage ich erſte
               Hälfte Juni vor. Geht Ihre \textcolor{blue}{Frau}{}\ledrightnote{{$\rightarrow$}\textcolor{blue}{Ottilie Salten}} mit, ſo käme die \textcolor{blue}{meine}{}\ledrightnote{{$\rightarrow$}\textcolor{blue}{Olga Schnitzler}} auch, und wir würden da{\geminationn} mehr eine Radialradpartie machen, d. h. allerlei
               Fahrten, mit feſtem Stützpunkt. – Ko{\geminationm}t \textcolor{blue}{Otti}{}\ledrightnote{\textcolor{blue}{Ottilie Salten}} nicht, ſo ſoll es eine Längspartie werden, »wie einſt im
               Mai«, (we{\geminationn} Sie uns jetzt als Julier, \textsc{resp.} Auguſtiner (Sie \introOben{}Anfang\introOben{} Julier
               und ich Endauguſtiner anſprechen.). Gar zu weite Bahnreiſe (\textcolor{pink}{Genf}{}\ledrightnote{\textcolor{pink}{Genf}}, \textcolor{pink}{Lugano}{}\ledrightnote{\textcolor{pink}{Lugano}}) möcht ich
               gern vermeiden, {\pb}aus 17 Gründen. – Von meiner
                  \textcolor{pink}{daen}{}\ledrightnote{{$\rightarrow$}\textcolor{pink}{Dänemark}}iſchen Idee, lieber,
               werd ich ſchwer abzubringen ſein. Hingegen habe {[}ich{]} folgendes zu
               bemerken. Wenn Sie auf einige Wochen an die \textcolor{pink}{See}{}\ledrightnote{{$\rightarrow$}\textcolor{pink}{Ostsee}} gehn, kann Ihnen doch auch die um ein paar Stunden
               verlängerte Reiſe nicht ankommen. Ko{\geminationm}en Sie aber immer
               nur auf 24 Stunden ans Ufer, ſo hab ich ohnedies ſehr wenig, \textsc{resp\textcolor{gray}{.}} zu wenig von Ihnen. Alles, was ich von \textcolor{pink}{deutſch}{}\ledrightnote{{$\rightarrow$}\textcolor{pink}{Deutschland}}en Seebädern höre, ni{\geminationm}t
               mich dagegen ein; die bekannten {\pb}ſind in
               Hinſicht auf Publikum \textsc{etc} berüchtigt, die unbekannten
               ſollen was Comfor\strikeout{\textcolor{gray}{×}}t \textsc{etc} anbelangt übel ausſehen. Wälder gibts nur auf
                  \textcolor{pink}{Rügen}{}\ledrightnote{\textcolor{pink}{Rügen}}. \textcolor{pink}{Daenemark}{}\ledrightnote{\textcolor{pink}{Dänemark}} ke{\geminationn} ich. Seit ich \label{K_L03003-4v}\edtext{dort geweſen}{\lemma{\textnormal{\emph{dort geweſen}}}\Cendnote{\textnormal{im Sommer 1896}}}\label{K_L03003-4h} bin, ſehn ich mich zurück. Die Menſchen dort (die man ja nicht kennt), der
               Himmel, die Wälder, allerlei undefinirbares iſt in der Erinnerung für mich von einem
               wahren Zauber umgeben. Auch denk ich lebhaft an einen \label{K_L03003-5v}\edtext{Abſtecher nach \textcolor{pink}{Schweden}{}\ledrightnote{\textcolor{pink}{Schweden}}, ev \textcolor{pink}{Norwegen}{}\ledrightnote{\textcolor{pink}{Norwegen}}}{\lemma{\textnormal{\emph{Abſtecher … Norwegen}}}\Cendnote{\textnormal{nicht geschehen}}}\label{K_L03003-5h}. Wir wollen
                  \label{K_L03003-6v}\edtext{auf 2, 3 {\pb}Tage nach \textcolor{pink}{Kopenhagen}{}\ledrightnote{\textcolor{pink}{Kopenhagen}}}{\lemma{\textnormal{\emph{auf 2, 3 Tage nach Kopenhagen}}}\Cendnote{\textnormal{\textcolor{blue}{Schnitzler} war vor seinem Aufenthalt in \textcolor{pink}{Marienlyst} nur am 28. 6. 1906 in \textcolor{pink}{Kopenhagen}.}}}\label{K_L03003-6h}, von dort aus inſpizire
               ich die \textcolor{pink}{Seeſeite}{}\ledrightnote{\textcolor{pink}{Öresund}} nach geeignetem
               Aufenthalt. –\pend
           
\pstart
           Schönen Dank für die noch ſchönern \label{K_L03003-7v}\edtext{\textcolor{green}{Feu{[}i{]}lletons}{}\ledrightnote{{$\rightarrow$}\textcolor{green}{»Kater Lampe«}{\newline}{$\rightarrow$}\textcolor{green}{Russisches Theater. II}}}{\lemma{\textnormal{\emph{Feuilletons}}}\Cendnote{\textnormal{\textcolor{blue}{Felix Salten}: \emph{\textcolor{green}{Russisches Theater. II}}. In: \emph{\textcolor{green}{B. Z. am Mittag}}, Jg. 30, Nr. 70, 23. 3. 1906, S. 2–3; \textcolor{blue}{ders.}: \emph{\textcolor{green}{»Kater Lampe«}}. In: \emph{\textcolor{green}{ebd.}}, Jg. 30,
                     Nr. 72, 26. 3. 1906, S. 2.}}}\label{K_L03003-7h}, \textcolor{pink}{\textcolor{green}{Rußland}{}\ledrightnote{{$\rightarrow$}\textcolor{green}{Russisches Theater. II}}}{}\ledrightnote{\textcolor{pink}{Russland}} und \textcolor{green}{Lampe}{}\ledrightnote{{$\rightarrow$}\textcolor{green}{»Kater Lampe«}} betreffend.
               Sie \uline{haben} ſich halt immer. Wenn Sie mit ſich ſelber
               raufen, bleiben Sie doch auf immer der Gewinner. Ich ko{\geminationm}
               zu oft gegen mich nicht auf. – Immerhin, ich arbeite jetzt. Sie ſind ſchon alle wieder
               da, die Geſtältchen und Geſtalten, – aber mit meiner Macht über ſie ſiehts noch
               ziemlich {\pb}flau aus. – Komiſch, ja ſogar ein
               wenig traurig waren manche Kritiken über den \textcolor{green}{Wurſtelſpaſs}{}\ledrightnote{{$\rightarrow$}\textcolor{green}{Zum großen Wurstel. Burleske in einem Akt}}. Es wurde mir \label{K_L03003-8v}\edtext{ſo anerkennend vermerkt, daſs mir endgiltig mies zu mir
               geworden zu ſein ſcheint}{\lemma{\textnormal{\emph{ſo … ſcheint}}}\Cendnote{\textnormal{siehe A. S.: \emph{Tagebuch}, 27. 3. 1906}}}\label{K_L03003-8h}. Ja, »\label{K_L03003-9v}\edtext{\textcolor{green}{\textcolor{pink}{Nordpol}{}\ledrightnote{\textcolor{pink}{Nordpol}}fahrer müſte man ſein}{}\ledrightnote{{$\rightarrow$}\textcolor{green}{Die letzten Masken}}}{\lemma{\textnormal{\emph{Nordpolfahrer … ſein}}}\Cendnote{\textnormal{\textcolor{blue}{Schnitzler} paraphrasierte \emph{\textcolor{green}{Die letzten Masken}}. Dort heißt es wörtlich: »\textcolor{green}{Ein Bauer auf dem Land möcht
                        ich sein, ein Schafhirt, ein \textcolor{pink}{Nordpol}fahrer – ah, was du willst! –}«}}}\label{K_L03003-9h}« ſagt \textcolor{green}{Weihgaſt}{}\ledrightnote{{$\rightarrow$}\textcolor{green}{Die letzten Masken}}, mit dem mich ſonſt nur geringe Sympathie \strikeout{\textcolor{gray}{bef}} verbindet. – \label{K_L03003-10v}\edtext{\textcolor{blue}{\textcolor{green}{Kerr}{}\ledrightnote{{$\rightarrow$}\textcolor{green}{Ödipus und der Ruf des Lebens}}}{}\ledrightnote{\textcolor{blue}{Alfred Kerr}}}{\lemma{\textnormal{\emph{Kerr}}}\Cendnote{\textnormal{\textcolor{blue}{Alfred Kerr}: \emph{\textcolor{green}{Ödipus und der Ruf des Lebens}}. In: \emph{\textcolor{green}{Die neue Rundschau}}, Jg. 17, H. 5, Mai 1906, S. 492–498. Siehe A. S.: \emph{Tagebuch}, 30. 3. 1906.}}}\label{K_L03003-10h} hab ich eigentlich, innerlich, (das
               innerlich bezieht ſich auf ihn), charmant gefunden{\dots} Wiſſen
               Sie um wen es mir eigentlich am leideſten thut? Um die gute {\pb}\textcolor{green}{Katharina}{}\ledrightnote{{$\rightarrow$}\textcolor{green}{Der Ruf des Lebens. Schauspiel in drei Akten}}, die als \textcolor{green}{Ophelia}{}\ledrightnote{{$\rightarrow$}\textcolor{green}{Hamlet}}{ }\substVorne{}\textsuperscript{,}\substDazwischen{}(\substHinten{}ja wär ich \textcolor{blue}{Julius Bauer}{}\ledrightnote{\textcolor{blue}{Julius Bauer}} ſo ſagt ich als
               Pophelia) behandelt wird, – weil Frl \textcolor{blue}{Hofmann}{}\ledrightnote{\textcolor{blue}{Grete Hofmann}}
               im letzten \textcolor{green}{Akt}{}\ledrightnote{{$\rightarrow$}\textcolor{green}{Der Ruf des Lebens. Schauspiel in drei Akten}} Blumen im Haar
               hatte. Als abſichtlich von mir aus \textcolor{green}{Hamlet}{}\ledrightnote{\textcolor{green}{Hamlet}}
               herausgeſtohlene \textcolor{green}{Ophelia}{}\ledrightnote{{$\rightarrow$}\textcolor{green}{Hamlet}}.
               Einer wie der andre. –\pend
           
\pstart
           \label{K_L03003-11v}\edtext{Neulich im \textcolor{pink}{Coloſſeum}{}\ledrightnote{\textcolor{pink}{Volkstheater in Rudolfsheim}}}{\lemma{\textnormal{\emph{Neulich im Coloſſeum}}}\Cendnote{\textnormal{siehe A. S.: \emph{Tagebuch}, 28. 3. 1906}}}\label{K_L03003-11h}; mit \textcolor{blue}{Wasserma{\geminationn}s}{}\ledrightnote{\textcolor{blue}{Jakob Wassermann}{\newline}\textcolor{blue}{Julie Wassermann}} u. \textcolor{blue}{Kaufmann}{}\ledrightnote{\textcolor{blue}{Arthur Kaufmann}}. Zwei Clowns als Nachtigallen den Unvergeßlichkeiten anzureihn.\pend
           
\pstart
           Grüß Sie Gott. Herzlichſt Ihr {\\[\baselineskip]}\spacefill\mbox{A.}\pend
           \leftskip=0em{}\endnumbering\briefempfaengerindex{Salten, Felix@\textsc{Salten, Felix}!zzzSchnitzler, Arthur@\emph{von Arthur Schnitzler}!1906-04-021@{2. 4. 1906}|)be}\mylabel{h}  \normalsize

\doendnotes{C}
\bigskip
\vfill

\clearpage

\footnotesize

\lohead{\textsc{register}}

% Definiere theindex-Environment komplett neu ohne reledmac
\makeatletter
\renewenvironment{theindex}{%
  \section*{\indexname}%
  \setlength{\parindent}{0pt}%
  \setlength{\parskip}{0pt plus 0.3pt}%
  \let\item\@idxitem
}{%
  \clearpage
}
\makeatother

\IfFileExists{\jobname-pw.ind}{\input{\jobname-pw.ind}}{}

\end{document}

      