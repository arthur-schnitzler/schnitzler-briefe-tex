%% latex-korrekturansicht-vorspann.tex
%% Vorspann für die Korrekturansicht.
%% Lädt die gemeinsame Datei latex-vorspann.tex mit gesetztem Schalter.

\newif\ifkorrekturansicht
\korrekturansichttrue

\input{../tex-inputs/latex-vorspann}


\renewcommand{\erwaehntePersonen}{Personen: Ferdinand Raimund, Julian Sternberg}
\renewcommand{\erwaehnteInstitutionen}{Institutionen: Südbahnstrecke, Wiener Allgemeine Zeitung}
\renewcommand{\erwaehnteOrte}{Orte: Raimund-Theater, Schulerstraße, Universitätsstraße, Wien}
\renewcommand{\erwaehnteWerke}{Werke: Das Mädchen aus der Feenwelt oder Der Bauer als Millionär}
\section[ Felix Salten an Arthur Schnitzler, {[}5. 9. 1896{]}]{Felix Salten an Arthur Schnitzler, {[}5. 9. 1896{]}}
\nopagebreak\mylabel{v}
\rehead{ }\normalsize\beginnumbering\briefempfaengerindex{Schnitzler, Arthur@\textsc{Schnitzler, Arthur}!zzzSalten, Felix@\emph{von Felix Salten}!1896-09-052@{{[}5. 9. 1896{]}}|(be}
\toendnotes[C]{\smallbreak\pagebreak[2]}\Standort{CUL, Schnitzler, B 89, A 1.}
\physDesc{Brief, 1 Blatt, 1 Seite, 224 Zeichen
\newline{}Handschrift: schwarze Tinte, lateinische Kurrent
\newline{}Schnitzler: mit Bleistift datiert: »5/9 96« 
\newline{}Ordnung: mit Bleistift von unbekannter Hand nummeriert: »79« }\toendnotes[C]{\smallbreak}
\pstart
           \noindent{}{\pb}\textcolor{gray}{\textbf{\textbf{»\textcolor{brown}{Wiener Allgemeine
                        Zeitung}{}\ledrightnote{\textcolor{brown}{Wiener Allgemeine Zeitung}}«}}}\pend
           
\pstart
           \textcolor{gray}{\textbf{Redaction und Adminiſtration:}}\pend
           
\pstart
           \textcolor{gray}{\textbf{\textcolor{pink}{Wien}{}\ledrightnote{\textcolor{pink}{Wien}}}}\pend
           
\pstart
           \textcolor{gray}{\textbf{\textcolor{pink}{\textbf{IX}/3,{ }\textbf{\so{Univerſitätsſtraße Nr. 6}}}{}\ledrightnote{\textcolor{pink}{Universitätsstraße}}\textbf{.}}}\pend
           
\pstart
           \textcolor{gray}{\textbf{Ankündigungs-Bureau:}}\pend
           
\pstart
           \textcolor{gray}{\textbf{\textbf{\textcolor{pink}{\so{I. Schulerſtraße Nr. 14}}{}\ledrightnote{\textcolor{pink}{Schulerstraße}}.
                     }}}\pend
           
\pstart
           \textcolor{gray}{\textbf{Telegramm-Adreſſe: »Allgemeine, \textcolor{pink}{Wien}{}\ledrightnote{\textcolor{pink}{Wien}}«.}}\pend
           
\pstart
           \textcolor{gray}{\textbf{Telephon der Redaction: Nr. 2180.}}\pend
           
\pstart
           \textcolor{gray}{\textbf{\hspace*{1.5em}„\hspace*{1.5em}„\hspace*{1.5em} Adminiſtration: Nr. 805.}}\pend
           
\pstart
           Lieber Arthur leider gibt’s keinen \label{K_L03180-1v}\edtext{Sitz heute}{\lemma{\textnormal{\emph{Sitz heute}}}\Cendnote{\textnormal{vermutlich für die Vorstellung von \textcolor{blue}{Ferdinand Raimund}s \emph{\textcolor{green}{Das Mädchen aus der Feenwelt oder Der Bauer als Millionär}}
                  im \textcolor{pink}{Raimund-Theater}, vgl. A. S.: \emph{Tagebuch}, 5. 9. 1896}}}\label{K_L03180-1h}. \textcolor{blue}{St-g.}{}\ledrightnote{\textcolor{blue}{Julian Sternberg}} hat mir ihn nicht gegeben {\kaufmannsund} mich hoch {\kaufmannsund} theuer
               gebeten, ich möge ihm denselben laßen.\pend
           
\pstart
           Also wenns nicht regnet \label{K_L03180-2v}\edtext{morgen{ }8'40{ }\textcolor{brown}{Südbahn}{}\ledrightnote{\textcolor{brown}{Südbahnstrecke}}}{\lemma{\textnormal{\emph{morgen 8'40 Südbahn}}}\Cendnote{\textnormal{für einen gemeinsamen Radausflug, vgl. A. S.: \emph{Tagebuch}, 6. 8. 1896}}}\label{K_L03180-2h}.\pend
           
\pstart
           Jedenfalls heute{ }Abend noch im \label{K_L03180-3v}\edtext{Caféhaus}{\lemma{\textnormal{\emph{Caféhaus}}}\Cendnote{\textnormal{Welches gemeint ist, konnte nicht bestimmt werden.}}}\label{K_L03180-3h}\pend
           
\pstart
           herzlichst {\\[\baselineskip]}\spacefill\mbox{Salten}\pend
           \leftskip=0em{}\endnumbering\briefempfaengerindex{Schnitzler, Arthur@\textsc{Schnitzler, Arthur}!zzzSalten, Felix@\emph{von Felix Salten}!1896-09-052@{{[}5. 9. 1896{]}}|)be}\mylabel{h}  \normalsize

\doendnotes{C}
\bigskip
\vfill

\clearpage

\footnotesize

\lohead{\textsc{register}}

% Definiere theindex-Environment komplett neu ohne reledmac
\makeatletter
\renewenvironment{theindex}{%
  \section*{\indexname}%
  \setlength{\parindent}{0pt}%
  \setlength{\parskip}{0pt plus 0.3pt}%
  \let\item\@idxitem
}{%
  \clearpage
}
\makeatother

\IfFileExists{\jobname-pw.ind}{\input{\jobname-pw.ind}}{}

\end{document}

      