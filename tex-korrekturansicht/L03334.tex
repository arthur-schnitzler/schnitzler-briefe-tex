%% latex-korrekturansicht-vorspann.tex
%% Vorspann für die Korrekturansicht.
%% Lädt die gemeinsame Datei latex-vorspann.tex mit gesetztem Schalter.

\newif\ifkorrekturansicht
\korrekturansichttrue

\input{../tex-inputs/latex-vorspann}


\renewcommand{\erwaehntePersonen}{Personen: Heinrich Kanner, Isidor Singer}
\renewcommand{\erwaehnteInstitutionen}{Institutionen: Die Zeit}
\renewcommand{\erwaehnteOrte}{Orte: Wien, Wipplingerstraße}
\renewcommand{\erwaehnteWerke}{Werke: Die griechische Tänzerin. Novellette, Die kleine Veronika}
\section[ Felix Salten an Arthur Schnitzler, 4. 9. 1902]{Felix Salten an Arthur Schnitzler, 4. 9. 1902}
\nopagebreak\mylabel{v}
\rehead{ }\normalsize\beginnumbering\briefempfaengerindex{Schnitzler, Arthur@\textsc{Schnitzler, Arthur}!zzzSalten, Felix@\emph{von Felix Salten}!1902-09-041@{4. 9. 1902}|(be}
\toendnotes[C]{\smallbreak\pagebreak[2]}\Standort{CUL, Schnitzler, B 89, A 2.}
\physDesc{Brief, 1 Blatt, 1 Seite, 1342 Zeichen
\newline{}Handschrift: schwarze Tinte, lateinische Kurrent
\newline{}Ordnung: mit Bleistift von unbekannter Hand nummeriert: »159« }\toendnotes[C]{\smallbreak}
\pstart
           \noindent{}{\pb}\textcolor{gray}{\textbf{DIE}}\pend
           
\pstart
           \textcolor{gray}{\textbf{\textcolor{brown}{ZEIT}{}\ledrightnote{\textcolor{brown}{Die Zeit}}}}\hfill \textcolor{gray}{\textbf{\emph{\textcolor{pink}{WIEN}{}\ledrightnote{\textcolor{pink}{Wien}}},}}{ }4. Septemb. \textcolor{gray}{\textbf{\emph{190}}}2.\pend
           
\pstart
           \textcolor{gray}{\textbf{\textsc{\textbf{\so{Wiener Tagblatt}}}}}\pend
           
\pstart
           \textcolor{gray}{\textbf{HERAUSGEBER:}}\pend
           
\pstart
           \textcolor{gray}{\textbf{\textbf{PROF. DR. \textcolor{blue}{I. SINGER}{}\ledrightnote{\textcolor{blue}{Isidor Singer}}}}}\pend
           
\pstart
           \textcolor{gray}{\textbf{\textbf{DR. \textcolor{blue}{HEINRICH KANNER}{}\ledrightnote{\textcolor{blue}{Heinrich Kanner}}}}}\pend
           
\pstart
           \textcolor{gray}{\textbf{\textbf{\so{REDACTION:}}}}\pend
           
\pstart
           \textcolor{gray}{\textbf{\textcolor{pink}{I/\textsubscript{21},
                           WIPPLINGERSTRASSE 38}{}\ledrightnote{\textcolor{pink}{Wipplingerstraße}}}}\pend
           
\pstart
           Lieber, gleich als Ihr Brief kam, schrieb ich Ihnen
               mit dem Vermerk auf dem Couvert, der Brief solle Ihnen nachgesendet werden.
               Telefonisch konnte ich Sie nicht mehr erreichen, – Sie waren schon abgereist. Jetzt
               weiß ich nicht, ob mein \label{K_L03334-1v}\edtext{erstes
                  Schreiben}{\lemma{\textnormal{\emph{erstes
                  Schreiben}}}\Cendnote{\textnormal{Felix Salten an Arthur Schnitzler, 2. 9. 1902}}}\label{K_L03334-1h} Sie erreicht hat, und so sage ich Ihnen hier das wesentliche noch
               einmal: 1) Die 80 Kr. waren ein Versehen. D\textsuperscript{r}{ }\textcolor{blue}{Kanner}{}\ledrightnote{\textcolor{blue}{Heinrich Kanner}} hat einfach vergessen dem Prof. \textcolor{blue}{Singer}{}\ledrightnote{\textcolor{blue}{Isidor Singer}} von Ihrem Honorar Mittheilung zu
               machen. 2.) In dem jetzt herrschenden Arbeitstrubel ist ein solcher Irrthum
               begreiflich und kann nichts verletzendes für Sie haben. 3.) Die restlichen 120 Kr.
               wurden sofort an Sie abgesendet. 4.) Ich hoffe, Sie haben die \textcolor{green}{Novelle}{}\ledrightnote{{$\rightarrow$}\textcolor{green}{Die griechische Tänzerin. Novellette}} doch, wie verabredet, mitgenommen,
               und diesen Vorfall nicht zum Anlaß ergriffen, die Sache beiseite zu legen. 5.) Es
               thut mir leid, dass Sie mich nicht einfach telef. angerufen haben, wodurch die Sache
               sofort aufgeklärt worden wäre. 6.) Ich wäre in großer Verlegenheit, wenn Sie mich mit
               dieser Arbeit jetzt sitzen ließen.\pend
           
\pstart
           Ohnehin habe ich in einer anderen, ähnlichen Angelegenheit eine sehr deprimirende
               Erfahrung gemacht, und es wäre mir unangenehm, wenn man hier die Sachen, wie es ja
               doch einmal geschieht, anders auffaßen würde. Schreiben Sie mir, bitte, eine
               Zeile.\pend
           
\pstart
           herzlich Ihr {\\[\baselineskip]}\spacefill\mbox{Salten}\pend
           \leftskip=0em{}
\pstart
           \noindent{}\label{K_L03334-2v}\edtext{NB.}{\lemma{\textnormal{\emph{NB.}}}\Cendnote{\textnormal{lateinisch: nota bene (merke wohl, übrigens)}}}\label{K_L03334-2h} Die \label{K_L03334-3v}\edtext{\textcolor{green}{Veronika}{}\ledrightnote{\textcolor{green}{Die kleine Veronika}} ist jetzt fertig, ich warte mit dem
                  Lesen bis Sie zurück}{\lemma{\textnormal{\emph{Veronika … zurück}}}\Cendnote{\textnormal{siehe A. S.: \emph{Tagebuch}, 14. 9. 1902}}}\label{K_L03334-3h} sind.\pend
           \endnumbering\briefempfaengerindex{Schnitzler, Arthur@\textsc{Schnitzler, Arthur}!zzzSalten, Felix@\emph{von Felix Salten}!1902-09-041@{4. 9. 1902}|)be}\mylabel{h}  \normalsize

\doendnotes{C}
\bigskip
\vfill

\clearpage

\footnotesize

\lohead{\textsc{register}}

% Definiere theindex-Environment komplett neu ohne reledmac
\makeatletter
\renewenvironment{theindex}{%
  \section*{\indexname}%
  \setlength{\parindent}{0pt}%
  \setlength{\parskip}{0pt plus 0.3pt}%
  \let\item\@idxitem
}{%
  \clearpage
}
\makeatother

\IfFileExists{\jobname-pw.ind}{\input{\jobname-pw.ind}}{}

\end{document}

      