%% latex-korrekturansicht-vorspann.tex
%% Vorspann für die Korrekturansicht.
%% Lädt die gemeinsame Datei latex-vorspann.tex mit gesetztem Schalter.

\newif\ifkorrekturansicht
\korrekturansichttrue

\input{../tex-inputs/latex-vorspann}


\renewcommand{\erwaehntePersonen}{Personen: Alfred von Berger, Otto Brahm,  Franz Ferdinand von Österreich-Este, Richard Kralik, Max Reinhardt, Rudolf Rittner, Felix Salten, Ottilie Salten, Hugo Thimig, Karl Gustav Vollmoeller}
\renewcommand{\erwaehnteInstitutionen}{Institutionen: Burgtheater, Österreichische Leo-Gesellschaft}
\renewcommand{\erwaehnteOrte}{Orte: Berghof, Prater, Rotunde, Tutzing, Unterach am Attersee, Wien}
\renewcommand{\erwaehnteWerke}{Werke: Das Mirakel}
\section[ Felix Salten an Arthur Schnitzler, 2. 9. 1912]{Felix Salten an Arthur Schnitzler, 2. 9. 1912}
\nopagebreak\mylabel{v}
\rehead{ }\normalsize\beginnumbering\briefempfaengerindex{Schnitzler, Arthur@\textsc{Schnitzler, Arthur}!zzzSalten, Felix@\emph{von Felix Salten}!1912-09-021@{2. 9. 1912}|(be}
\toendnotes[C]{\smallbreak\pagebreak[2]}\Standort{CUL, Schnitzler, B 89, B 2.}
\physDesc{Karte, 1321 Zeichen
\newline{}Handschrift: schwarze Tinte, lateinische Kurrent
\newline{}Ordnung: mit Bleistift von unbekannter Hand nummeriert: »274a« }\toendnotes[C]{\smallbreak}
\pstart
           \raggedleft{}{\pb}\textcolor{pink}{Berghof}{}\ledrightnote{\textcolor{pink}{Berghof}}, 2. IX. 12\pend
           
\pstart{}Lieber,\pend
\pstart
           ich hoffe sehr, dass \label{K_L03559-1v}\edtext{\textcolor{blue}{Reinhardt}{}\ledrightnote{\textcolor{blue}{Max Reinhardt}}s \textcolor{green}{Mirakel}{}\ledrightnote{\textcolor{green}{Das Mirakel}}}{\lemma{\textnormal{\emph{Reinhardts Mirakel}}}\Cendnote{\textnormal{\emph{\textcolor{green}{Das Mirakel}} von \textcolor{blue}{Karl Gustav Vollmoeller} wurde am 18. 9. 1912 in der \textcolor{pink}{Rotunde} im
                  \textcolor{pink}{Wien}er \textcolor{pink}{Prater} erstmals auf Deutsch gegeben, wo Platz
                  für 8.000 Zuschauerinnen und Zuschauer war. Die Inszenierung stammte von \textcolor{blue}{Max Reinhardt}. \textcolor{blue}{Schnitzler} besuchte
               die Aufführung am 5. 10. 1912.}}}\label{K_L03559-1h} verspätet
               aufgeführt wird, und dass mich also nichts dazu zwingt, die Eucharistische Luft in
                  \textcolor{pink}{Wien}{}\ledrightnote{\textcolor{pink}{Wien}} zu atmen. Wenn \textcolor{blue}{Otti}{}\ledrightnote{\textcolor{blue}{Ottilie Salten}} wieder da und der \textcolor{pink}{Berghof}{}\ledrightnote{\textcolor{pink}{Berghof}} ruhiger geworden ist, möchte ich wol gerne noch ein paar Wochen
               still hier arbeiten. Was sagen Sie zum \label{K_L03559-2v}\edtext{\textcolor{brown}{Burgtheater}{}\ledrightnote{\textcolor{brown}{Burgtheater}}}{\lemma{\textnormal{\emph{Burgtheater}}}\Cendnote{\textnormal{\textcolor{blue}{Alfred von Berger}, der Direktor des \emph{\textcolor{brown}{Burgtheater}}s, war am 24. 8. 1912
                  verstorben. Am 1. 9. 1912 folgte ihm \textcolor{blue}{Hugo Thimig} nach.}}}\label{K_L03559-2h}? Der arme \textcolor{blue}{Berger}{}\ledrightnote{\textcolor{blue}{Alfred von Berger}} tut mir leid, aber ich kann mir nicht
               helfen – wenn auch ein Fi\textcolor{gray}{asco} oftmals besser ist als das Sterben,
               hier hat der Tod doch einen an sich schon nicht übermäßig glücklichen Menschen vor
               sehr unglücklichen Enttäuschungen bewahrt. Könnten wir \textcolor{blue}{Brahm}{}\ledrightnote{\textcolor{blue}{Otto Brahm}} oder vielleicht sogar \textcolor{blue}{Rudolf Rittner}{}\ledrightnote{\textcolor{blue}{Rudolf Rittner}} bekommen, dann wäre doch vielleicht für die Zukunft ein gutes
               menschliches und künstlerisches Verhältnis zum \textcolor{brown}{Burgtheater}{}\ledrightnote{\textcolor{brown}{Burgtheater}} möglich. Aber das{[}s{]} Herr \textcolor{blue}{von Kralik}{}\ledrightnote{\textcolor{blue}{Richard Kralik}} als Director auch nur genannt werden {\pb}kann, dass die \textcolor{brown}{Leo-Gesellschaft}{}\ledrightnote{\textcolor{brown}{Österreichische Leo-Gesellschaft}} ihre Zeit schon so sehr für gekommen hält,
               das ist ein böses Zeichen. \textcolor{blue}{Franz Ferdinand}{}\ledrightnote{\textcolor{blue}{Franz Ferdinand von Österreich-Este}}
               wirft eben auch hier schon seine schwarzen Schatten voraus! Wie ich die Gesellschaft im \textcolor{brown}{Burgtheater}{}\ledrightnote{\textcolor{brown}{Burgtheater}} zu kennen glaube, werden sie mit Wonne und Schadenfreude und mit
               allen Übertreibungen der Strebsamkeit an der \label{K_L03559-3v}\edtext{Katholisisirung}{\lemma{\textnormal{\emph{Katholisisirung}}}\Cendnote{\textnormal{Die \emph{\textcolor{brown}{Österreichische Leo-Gesellschaft}}
                  förderte explizit katholische Kunst und Wissenschaft.}}}\label{K_L03559-3h} des Repertoires mithelfen. Ich habe sehr das
               Gefühl, dass in dieser Beziehung ungeahnte Dinge bevorstehen. \label{K_L03559-4v}\edtext{Wer ljäben wird, wird sehen}{\lemma{\textnormal{\emph{Wer … sehen}}}\Cendnote{\textnormal{vermutlich eine jiddelnde Eindeutschung der
                  französischen Phrase »\begin{otherlanguage}{french}qui vivra, verra\end{otherlanguage}« (wer
                  leben wird, wird sehen)}}}\label{K_L03559-4h}!\pend
           
\pstart
           Auf gutes Wiedersehen und viele herzliche Grüße {\\[\baselineskip]}Ihr \spacefill\mbox{Salten}\pend
           \leftskip=0em{}\endnumbering\briefempfaengerindex{Schnitzler, Arthur@\textsc{Schnitzler, Arthur}!zzzSalten, Felix@\emph{von Felix Salten}!1912-09-021@{2. 9. 1912}|)be}\mylabel{h}  \normalsize

\doendnotes{C}
\bigskip
\vfill

\clearpage

\footnotesize

\lohead{\textsc{register}}

% Definiere theindex-Environment komplett neu ohne reledmac
\makeatletter
\renewenvironment{theindex}{%
  \section*{\indexname}%
  \setlength{\parindent}{0pt}%
  \setlength{\parskip}{0pt plus 0.3pt}%
  \let\item\@idxitem
}{%
  \clearpage
}
\makeatother

\IfFileExists{\jobname-pw.ind}{\input{\jobname-pw.ind}}{}

\end{document}

      