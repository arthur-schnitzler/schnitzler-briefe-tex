%% latex-korrekturansicht-vorspann.tex
%% Vorspann für die Korrekturansicht.
%% Lädt die gemeinsame Datei latex-vorspann.tex mit gesetztem Schalter.

\newif\ifkorrekturansicht
\korrekturansichttrue

\input{../tex-inputs/latex-vorspann}


               \section[Paul Goldmann an Arthur Schnitzler, 2. 3. {[}1895{]}]{ Paul Goldmann an Arthur Schnitzler, 2. 3. {[}1895{]}}\nopagebreak\mylabel{v}\rehead{ }\normalsize\beginnumbering\briefempfaengerindex{Schnitzler, Arthur@\textsc{Schnitzler, Arthur}!zzzGoldmann, Paul@\emph{von Paul Goldmann}!1895-03-021@{2. 3. {[}1895{]}}|(be} \toendnotes[C]{\smallbreak\pagebreak[2]} \Standort{DLA, A:Schnitzler, HS.NZ85.1.3165.}
\physDesc{Brief, 3 Blätter, 11 Seiten
\newline{}Handschrift: schwarze Tinte, deutsche Kurrent
\newline{}Schnitzler: 1) mit Bleistift das Jahr »95« vermerkt 2) mit rotem Buntstift fünf Unterstreichungen}\toendnotes[C]{\smallbreak}\pstart
           \noindent{}{\pb}\textcolor{gray}{\textbf{\textbf{\textcolor{brown}{Frankfurter Zeitung}{}\ledrightnote{\textcolor{brown}{Frankfurter Zeitung}}.}}}\pend
           \pstart
           \textcolor{gray}{\textbf{(\textcolor{brown}{\begin{otherlanguage}{french}Gazette de Francfort\end{otherlanguage}}{}\ledrightnote{\textcolor{brown}{Frankfurter Zeitung}}). }}\pend
           \pstart
           \textcolor{gray}{\textbf{\textbf{\begin{otherlanguage}{french}Fondateur M. \textcolor{blue}{L.
                              Sonnemann}{}\ledrightnote{\textcolor{blue}{Leopold Sonnemann}}\end{otherlanguage}.}}}\pend
           \pstart
           \begin{otherlanguage}{french}\textcolor{gray}{\textbf{\textcolor{green}{Journal}{}\ledrightnote{\textcolor{green}{Frankfurter Zeitung}} politique, financier,}}\end{otherlanguage}\hfill \textsc{\textcolor{pink}{Paris}{}\ledrightnote{\textcolor{pink}{Paris}}}, 2. März.\pend
           \pstart
           \begin{otherlanguage}{french}\textcolor{gray}{\textbf{commercial et littéraire.}}\end{otherlanguage}\pend
           \pstart
           \begin{otherlanguage}{french}\textcolor{gray}{\textbf{\textbf{Paraissant trois fois par jour.}}}\end{otherlanguage}\pend
           \pstart
           \begin{otherlanguage}{french}\textcolor{gray}{\textbf{\textbf{Bureaux à \textcolor{pink}{Paris}{}\ledrightnote{\textcolor{pink}{Paris}}:}}}\end{otherlanguage}\pend
           \pstart
           \begin{otherlanguage}{french}\textcolor{gray}{\textbf{\textbf{\textcolor{pink}{24. Rue Feydeau}{}\ledrightnote{\textcolor{pink}{rue Feydeau}}.}}}\end{otherlanguage}\pend
           \pstart{}Mein lieber Freund,\pend\pstart
           Nun geht es mir langſam wieder beſſer, und ich kann Dir ſchreiben. Als Folge der
               allgemeinen Krankheit hat ſich ein hartnäckiges \label{K_L02729-44v}\edtext{Augenübel}{\lemma{\textnormal{\emph{Augenübel}}}\Cendnote{\textnormal{Syphilis hatte als mögliche sekundäre Folge eine Entzündung des Auges}}}\label{K_L02729-44h}
               ergeben. Es kam zum zweiten Male bereits und hält diesmal lange Wochen vor. Da ich
               meinen Beruf nicht ausſetzen kann, ſollte ich alles Schreiben und Leſen auf das
               unerläßlich Berufliche beſchränken. Da blieb alſo für Briefe nichts übrig. Auch war
               es nicht gut möglich, meinen armen dummen Kopf zu einem andern Gedanken zu bringen
               als zu dem an die Krankheit. Was der Beruf eiſern {\pb}erzwang, \strikeout{ging} ging noch. Sonſt aber ſaß ich da,
               Tage und Nächte, und hörte alle Geſpenſter meines unglückſeeligen Lebens um mich
               ſtreichen. Das wird ſchlimm enden, liebſter Freund.\pend
           \pstart
           Nun laß’ Dich von Herzen beglückwünſchen zur \label{K_L02729-1v}\edtext{Annahme}{\lemma{\textnormal{\emph{Annahme}}}\Cendnote{\textnormal{Am 15. 2. 1895 erhielt \textcolor{blue}{Schnitzler} die Nachricht, dass die \emph{\textcolor{green}{Liebelei}} am \emph{\textcolor{brown}{Deutschen Theater}} in \textcolor{pink}{Berlin}
                  angenommen wurde. Premiere feierte das \textcolor{green}{Stück} dort am 4. 2. 1896.}}}\label{K_L02729-1h} im »\textcolor{brown}{Deutſchen Theater}{}\ledrightnote{\textcolor{brown}{Deutsches Theater Berlin}}«. \strikeout{\textcolor{gray}{Ve}} Das iſt, in Bezug auf den Vertrieb am \textcolor{pink}{deutſch}{}\ledrightnote{→\textcolor{pink}{Deutschland}}en Markt, womöglich noch beſſer, als das \textcolor{brown}{Burgtheater}{}\ledrightnote{\textcolor{brown}{Burgtheater}}. Von \textcolor{pink}{Berlin}{}\ledrightnote{\textcolor{pink}{Berlin}}
               aus kommt man direkt in die \textcolor{pink}{deutſch}{}\ledrightnote{→\textcolor{pink}{Deutschland}}e Literatur. Das Alles ſind ſo ſchöne Erfolge; und wenn ich ſehe, wie
               man ſonſt Erfolge davonträgt, und wie Du dazu kommſt: ohne Conceſſion, ohne die \strikeout{leiſ} leiſeſte Nacken-Beugung, {\pb}ruhig und ehrlich und Dir ſelbſt getreu – ſo gibt
               mir das ein recht ſtolzes Bild, und es iſt beinahe noch ſchöner als Dein \textcolor{green}{Stück}{}\ledrightnote{→\textcolor{green}{Liebelei. Schauspiel in drei Akten}}. \strikeout{Ob} Daß die geniale \textcolor{blue}{Dame}{}\ledrightnote{→\textcolor{blue}{Adele Sandrock}}{ }\label{K_L02729-2v}\edtext{keine Schwierigkeiten mehr}{\lemma{\textnormal{\emph{keine … mehr}}}\Cendnote{\textnormal{\textcolor{blue}{Adele Sandrock} schien zwar keine Drohungen
                  im Hinblick auf die Aufführung der \emph{\textcolor{green}{Liebelei}}
                  am \emph{\textcolor{brown}{Burgtheater}} mehr gemacht zu haben, bemühte
                  sich jedoch immer noch täglich um \textcolor{blue}{Schnitzler}s Zuneigung.}}}\label{K_L02729-2h} macht, iſt gut. Sie wird wohl wieder anfangen;
               aber ſie kann nichts mehr verderben, und wenn \strikeout{ich} ihr
               auch alle Teufel der Hölle im Leibe ſäßen. Ob das \textcolor{brown}{Burgtheater}{}\ledrightnote{\textcolor{brown}{Burgtheater}} das \textcolor{green}{Stück}{}\ledrightnote{→\textcolor{green}{Liebelei. Schauspiel in drei Akten}}
               jetzt oder in der nächſten Saiſon ſpielt, iſt völlig gleichgiltig. Dir zuliebe möchte
               ich wünſchen, daß es bald wäre. Mir wäre es lieber, ich hätte Dich noch ein halbes
               Jahr unaufgeführt. Der \textsc{Schnitzler} der \label{K_L02729-3v}\edtext{»\textcolor{green}{zum klangvollſten Namenskreis moderner {\pb}Schriftſteller gehört}{}\ledrightnote{→\textcolor{green}{Feuilleton. Literatur [Sterben]}}«}{\lemma{\textnormal{\emph{»zum … gehört«}}}\Cendnote{\textnormal{Das Zitat stammt aus einer Kritik zu \emph{\textcolor{green}{Sterben}}: \textcolor{blue}{Bruno Walden} [= \textcolor{blue}{Florentine Galliny}]: \emph{\textcolor{green}{Feuilleton. Literatur}}. In: \emph{\textcolor{green}{Wiener Abendpost}}, Jg. 192, Nr. 33, 9. 2. 1895, S. 5–6, hier: S. 5.}}}\label{K_L02729-3h},
               kommt mir recht kalt und fremd vor. Aber welch’ eine ſchöne \textcolor{green}{Kritik}{}\ledrightnote{→\textcolor{green}{Feuilleton. Literatur [Sterben]}}, dieſer \textsc{\textcolor{blue}{Bruno Walden}{}\ledrightnote{→\textcolor{blue}{Florentine Galliny}}}. Da iſt einmal \textcolor{blue}{Einer}{}\ledrightnote{→\textcolor{blue}{Florentine Galliny}},
               der Dich nach Verdienſt würdigt. Der Erfolg iſt umſo größer, als der \textcolor{blue}{Ochs}{}\ledrightnote{→\textcolor{blue}{Florentine Galliny}} – oder \strikeout{die \textcolor{blue}{Gans}{}\ledrightnote{→\textcolor{blue}{Florentine Galliny}} –} die \textcolor{blue}{Gans}{}\ledrightnote{→\textcolor{blue}{Florentine Galliny}} – ſich ſo im \label{K_L02729-66v}\edtext{\textcolor{green}{Urtheil}{}\ledrightnote{→\textcolor{green}{Feuilleton. Literatur [Anatol]}} über \textsc{\textcolor{green}{Anatol}{}\ledrightnote{\textcolor{green}{Anatol}}} vergriffen}{\lemma{\textnormal{\emph{Urtheil … vergriffen}}}\Cendnote{\textnormal{vgl. Paul Goldmann an Arthur Schnitzler, 8. 8. 1893}}}\label{K_L02729-66h} hat. Auch dazu laß’ Dich von Herzen
               beglückwünſchen! Und Dank für die Überſendung. Es hat mir große Freude gemacht, den
                  \textcolor{green}{Artikel}{}\ledrightnote{→\textcolor{green}{Feuilleton. Literatur [Sterben]}} – er iſt überdies
               ſchön geſchrieben – zu leſen.\pend
           \pstart
           Jedesmal noch ärgere ich mich über den \label{K_L02729-99v}\edtext{Titel}{\lemma{\textnormal{\emph{Titel}}}\Cendnote{\textnormal{vgl. Paul Goldmann an Arthur Schnitzler, 31. 12. [1894]}}}\label{K_L02729-99h}, »\textcolor{green}{Liebelei}{}\ledrightnote{\textcolor{green}{Liebelei. Schauspiel in drei Akten}}«. Wenn Du wüßteſt, wie garſtig er \strikeout{kli} klingt und wie er das \textcolor{green}{Werk}{}\ledrightnote{→\textcolor{green}{Liebelei. Schauspiel in drei Akten}} verkleinert! {\pb}Daß Du Dir ſo gar nichts
               ſagen laſſen willſt! Warum nicht »\textcolor{green}{Eine Liebſchaft}{}\ledrightnote{→\textcolor{green}{Liebelei. Schauspiel in drei Akten}}«?\pend
           \pstart
           Möchte wiſſen, was Du ſchreibſt und lieſt. Ich leſe gar nicht mehr. Ich habe es
               aufgegeben, – ſtrebe nicht mehr mit – laſſe mich ſinken.\pend
           \pstart
           Und wie lebſt Du? Still oder innerlich bewegt? Gehen neue Dinge vor? Bitte, ſchreib’
               mir ein wenig wie Du lebſt.\pend
           \pstart
           Und, was macht \textsc{\textcolor{blue}{Richard}{}\ledrightnote{\textcolor{blue}{Richard Beer-Hofmann}}}? Schreibt natürlich keine Zeile? Aber gedenkt {\pb}er wenigſtens ſeines Verſprechens nach \textsc{\textcolor{pink}{Paris}{}\ledrightnote{\textcolor{pink}{Paris}}} zu kommen?\pend
           \pstart
           \textsc{\textcolor{blue}{Bahr}{}\ledrightnote{\textcolor{blue}{Hermann Bahr}}} haſſe ich mehr und mehr. Welch’ ein \textcolor{blue}{Schwindler}{}\ledrightnote{→\textcolor{blue}{Hermann Bahr}}! Welch’ ein \textsc{\textcolor{blue}{Charlatan}{}\ledrightnote{→\textcolor{blue}{Hermann Bahr}}}! Ein \textcolor{blue}{Mann}{}\ledrightnote{→\textcolor{blue}{Hermann Bahr}}, der nach
               Geſetzen und Strömungen geht in der Literatur, – der dem Publikum einreden will, man
               könne ſo eine Art exakte Literatur-Forſchung treiben, während es doch da nur
               Individualitäten gibt, alſo Zufälliges, Unberechenbares, Geheimnißvolles. Und gerade
               die ſieht er und verſteht er nicht, der \textcolor{blue}{Urtheilsloſe}{}\ledrightnote{→\textcolor{blue}{Hermann Bahr}}. Nicht einen Neuen hat er in der »\textcolor{green}{Zeit}{}\ledrightnote{\textcolor{green}{Die Zeit. Wiener Wochenschrift}}« heraufgebracht, {\pb}und ich bin überzeugt, es gäbe Manchen in \textcolor{pink}{Wien}{}\ledrightnote{\textcolor{pink}{Wien}} zu finden. Aber immer nur \textsc{\textcolor{blue}{Bahr}{}\ledrightnote{\textcolor{blue}{Hermann Bahr}}} – \textsc{\textcolor{blue}{Bahr}{}\ledrightnote{\textcolor{blue}{Hermann Bahr}}} über Theater und \textsc{\textcolor{blue}{Bahr}{}\ledrightnote{\textcolor{blue}{Hermann Bahr}}} über Kunſt – \textsc{\textcolor{blue}{Bahr}{}\ledrightnote{\textcolor{blue}{Hermann Bahr}}} über \label{K_L02729-77v}\edtext{\textcolor{green}{\textcolor{blue}{\textsc{Emerson}}{}\ledrightnote{\textcolor{blue}{Ralph Waldo Emerson}}}{}\ledrightnote{\textcolor{green}{Emerson}}}{\lemma{\textnormal{\emph{Emerson}}}\Cendnote{\textnormal{\textcolor{blue}{Hermann Bahr}: \emph{\textcolor{green}{Emerson}}. In: \emph{\textcolor{green}{Die
                        Zeit}}, Bd. 1, H. 13, 29. 12. 1894,
                     S. 199.}}}\label{K_L02729-77h} und \textsc{\textcolor{blue}{Bahr}{}\ledrightnote{\textcolor{blue}{Hermann Bahr}}}{ }\label{K_L02729-678v}\edtext{über \textsc{\textcolor{blue}{Goethe}{}\ledrightnote{\textcolor{blue}{Johann Wolfgang von Goethe}}}}{\lemma{\textnormal{\emph{über Goethe}}}\Cendnote{\textnormal{Das bezieht sich nicht auf einen
                        spezifischen Text, sondern die regelmäßige Erwähnung \textcolor{blue}{Goethe}s in \textcolor{blue}{Bahr}s Texten.}}}\label{K_L02729-678h}. Und immer
               »modern«! Jetzt hat er heraus, daß das Alte modern iſt. Darum muß man alſo jetzt ſich
               mit dem Alten beſchäftigen. Alles nach Außen und nichts von Innen. Der \textcolor{blue}{Pinſel}{}\ledrightnote{→\textcolor{blue}{Hermann Bahr}}!\pend
           \pstart
           \textsc{\label{K_L02729-456v}\edtext{\textcolor{blue}{Kanner}{}\ledrightnote{\textcolor{blue}{Heinrich Kanner}}}{\lemma{\textnormal{\emph{Kanner}}}\Cendnote{\textnormal{Im \emph{\textcolor{green}{Tagebuch}}
                     von \textcolor{blue}{Schnitzler} wird er in dieser Zeit nicht erwähnt und auch 
                     sonst ist nur eine Begegnung festgehalten.}}}\label{K_L02729-456h}} aber iſt herrlich in der »\textcolor{green}{Zeit}{}\ledrightnote{\textcolor{green}{Die Zeit. Wiener Wochenschrift}}«. Feſt,
               klar und ſcharf. Ein männlicher \textcolor{blue}{Geiſt}{}\ledrightnote{→\textcolor{blue}{Heinrich Kanner}}! Siehſt Du ihn manchmal? Wie ſtehſt Du mit ihm?\pend
           \pstart
           {\pb}Daß Du mich im Sommer doch treffen willſt, iſt lieb
               von Dir. Vielleicht daß ich alſo doch nach der Kur auf ein paar Tage nach \textsc{\textcolor{pink}{Muenchen}{}\ledrightnote{\textcolor{pink}{München}}} kann. Ich möchte Dich ja ſo gern ſehen und ſprechen. Nach \textsc{\textcolor{pink}{Paris}{}\ledrightnote{\textcolor{pink}{Paris}}} könnteſt Du nicht auf 14 Tage kommen?\pend
           \pstart
           Zeitungsartikel ſende ich Dir heut nicht. \strikeout{Ich habe} Es hat keine intereſſanten gegeben; habe auch
               wenig leſen dürfen. Intereſſiren ſie Dich überhaupt? Dann macht es mir eine Freude,
               weiterzuſammeln.\pend
           \pstart
           {\pb}Was Du über \textsc{\textcolor{blue}{Drumont}{}\ledrightnote{\textcolor{blue}{Édouard Drumont}}} ſchreibſt, iſt im Weſentlichen richtig. Aber ſo ganz blos literariſch iſt ſein
               dämoniſcher Juden-Typus doch nicht. In \label{K_L02729-6v}\edtext{\textsc{\textcolor{blue}{Cornelius Herz}{}\ledrightnote{\textcolor{blue}{Cornelius Herz}}}}{\lemma{\textnormal{\emph{Cornelius Herz}}}\Cendnote{\textnormal{\textcolor{blue}{Édouard Drumont} war ein \textcolor{pink}{französisch}er \textcolor{blue}{Antisemit}, der die Idee einer
                  entarteten, degenerierten jüdischen »Rasse« propagierte. Er übte unter anderem im
                  Rahmen des \textcolor{pink}{Panama}-Skandals, in den auch \textcolor{blue}{Cornelius Herz} verwickelt war,
                  antisemitische Korruptionskritik.}}}\label{K_L02729-6h} iſt er zum Theil wahr geworden. Gewiß \textsc{\textcolor{blue}{Drumont}{}\ledrightnote{\textcolor{blue}{Édouard Drumont}}} iſt ſtark \label{K_L02729-7v}\edtext{monoman}{\lemma{\textnormal{\emph{monoman}}}\Cendnote{\textnormal{eine Zwangsvorstellung oder fixe Idee
                  haben}}}\label{K_L02729-7h}. Aber er iſt der beſte \textcolor{blue}{Kenner}{}\ledrightnote{→\textcolor{blue}{Édouard Drumont}} der heutigen \textcolor{pink}{Pariſ}{}\ledrightnote{\textcolor{pink}{Paris}}er Corruption. Was dem Draußenſtehenden darin \strikeout{\textcolor{gray}{d}} wahnſinnig ſcheint, iſt oft blos wahr. Und in allen \textcolor{pink}{Pariſ}{}\ledrightnote{\textcolor{pink}{Paris}}er Corruptionen ſteckt der Jude. Es iſt ein infames
               Geſindel. In dieſem \textsc{\textcolor{pink}{Babylon}{}\ledrightnote{→\textcolor{pink}{Paris}}}{ }{\pb}iſt \textsc{\textcolor{blue}{Drumont}{}\ledrightnote{\textcolor{blue}{Édouard Drumont}}} der \textcolor{blue}{Mann}{}\ledrightnote{→\textcolor{blue}{Édouard Drumont}}, der das
               flammende \label{K_L02729-8v}\edtext{\textsc{Mene Tekel}}{\lemma{\textnormal{\emph{Mene Tekel}}}\Cendnote{\textnormal{Warnung}}}\label{K_L02729-8h} ſchreibt. Als
                  \strikeout{Cor}{ }\textcolor{blue}{Corruptions-Epiker}{}\ledrightnote{→\textcolor{blue}{Édouard Drumont}} muß man
               ihn ernſt nehmen; ſonſt iſt er eitel und verrückt.\pend
           \pstart
           Ich ſende Dir »\textsc{\textcolor{green}{Les Phonographies de\strikeout{’} l’Amour}{}\ledrightnote{\textcolor{green}{Phonographie de l’amour}}}«. Eine amüſante kleine Unanſtändigkeit.\pend
           \pstart
           Bekommſt Du noch das »\textcolor{green}{Journal}{}\ledrightnote{→\textcolor{green}{Frankfurter Zeitung}}«? Möchteſt Du ein anderes Blatt? Bekommt Ihr den »\textsc{\textcolor{green}{Courrier \strikeout{de}
                     Français}{}\ledrightnote{\textcolor{green}{Le Courrier français}}}«? Kann ich Dir ſonſt etwas in \textsc{\textcolor{pink}{Paris}{}\ledrightnote{\textcolor{pink}{Paris}}} beſorgen?\pend
           \pstart
           {\pb}Denk’ Dir: Deinem \textcolor{blue}{Bruder}{}\ledrightnote{→\textcolor{blue}{Julius Schnitzler}} und \textcolor{blue}{Schwägerin}{}\ledrightnote{→\textcolor{blue}{Helene Schnitzler}} habe ich noch nicht für das
               entzückende \label{K_L02729-9v}\edtext{Bild}{\lemma{\textnormal{\emph{Bild}}}\Cendnote{\textnormal{Siehe Paul Goldmann an Arthur Schnitzler, 5. 1. [1895]}}}\label{K_L02729-9h} gedankt, an dem ich täglich meine Freude habe. Sag’ ihnen, daß ich augenkrank
               war, – bitte – und daß ich ihnen nächſtens ſchreibe. Grüße ſie \textcolor{blue}{Beide}{}\ledrightnote{→\textcolor{blue}{Julius Schnitzler}{\newline}→\textcolor{blue}{Helene Schnitzler}} recht herzlich.\pend
           \pstart
           Bitte, empfiehl’ mich Deiner Frau \textcolor{blue}{Mama}{}\ledrightnote{→\textcolor{blue}{Louise Schnitzler}}.\pend
           \pstart
           Sei herzlichſt und in Treue begrüßt! Nun höre ich hoffentlich bald von Dir. Aber
               antworte einmal auf alle Fragen (ausnahmsweiſe!) Dein\pend
           \pstart \spacefill\mbox{Paul Goldmann}\pend{}\endnumbering\briefempfaengerindex{Schnitzler, Arthur@\textsc{Schnitzler, Arthur}!zzzGoldmann, Paul@\emph{von Paul Goldmann}!1895-03-021@{2. 3. {[}1895{]}}|)be}\mylabel{h}\begin{anhang}\end{anhang}\normalsize

\doendnotes{C}
\bigskip
\vfill

\clearpage

\footnotesize

\lohead{\textsc{register}}

% Definiere theindex-Environment komplett neu ohne reledmac
\makeatletter
\renewenvironment{theindex}{%
  \section*{\indexname}%
  \setlength{\parindent}{0pt}%
  \setlength{\parskip}{0pt plus 0.3pt}%
  \let\item\@idxitem
}{%
  \clearpage
}
\makeatother

\IfFileExists{\jobname-pw.ind}{\input{\jobname-pw.ind}}{}

\end{document}

      