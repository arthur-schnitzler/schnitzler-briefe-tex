%% latex-korrekturansicht-vorspann.tex
%% Vorspann für die Korrekturansicht.
%% Lädt die gemeinsame Datei latex-vorspann.tex mit gesetztem Schalter.

\newif\ifkorrekturansicht
\korrekturansichttrue

\input{../tex-inputs/latex-vorspann}


               \section[Paul Goldmann an Arthur Schnitzler, Paul Goldmann an Arthur Schnitzler, 23. 1. {[}1896{]}]{ Paul Goldmann an Arthur Schnitzler, 23. 1. {[}1896{]}}\nopagebreak\mylabel{v}\rehead{ }\normalsize\beginnumbering\briefempfaengerindex{Schnitzler, Arthur@\textsc{Schnitzler, Arthur}!zzzGoldmann, Paul@\emph{von Paul Goldmann}!1896-01-231@{23. 1. {[}1896{]}}|(be} \toendnotes[C]{\smallbreak\pagebreak[2]} \Standort{DLA, A:Schnitzler, HS.NZ85.1.3166.}
\physDesc{Brief, 3 Blätter, 11 Seiten
\newline{}Handschrift: blaue Tinte, deutsche Kurrent
\newline{}Schnitzler: 1) mit Bleistift das Jahr »96« vermerkt 2) mit rotem Buntstift vier Unterstreichungen}\toendnotes[C]{\smallbreak}\pstart
           \noindent{}{\pb}\textcolor{gray}{\textbf{\textbf{\textcolor{brown}{Frankfurter Zeitung}{}\ledrightnote{\textcolor{brown}{Frankfurter Zeitung}}}}}\pend
           \pstart
           \textcolor{gray}{\textbf{(\textcolor{brown}{\begin{otherlanguage}{french}Gazette de Francfort\end{otherlanguage}}{}\ledrightnote{\textcolor{brown}{Frankfurter Zeitung}}).}}\pend
           \pstart
           \textcolor{gray}{\textbf{\textbf{\begin{otherlanguage}{french}Fondateur M.\end{otherlanguage}{ }\textcolor{blue}{L. Sonnemann}{}\ledrightnote{\textcolor{blue}{Leopold Sonnemann}}.}}}\pend
           \pstart
           \begin{otherlanguage}{french}\textcolor{gray}{\textbf{\textcolor{green}{Journal}{}\ledrightnote{→\textcolor{green}{Frankfurter Zeitung}} politique,
                        financier,}}\end{otherlanguage}\pend
           \pstart
           \begin{otherlanguage}{french}\textcolor{gray}{\textbf{commercial et littéraire.}}\end{otherlanguage}\pend
           \pstart
           \begin{otherlanguage}{french}\textcolor{gray}{\textbf{\textbf{Paraissant trois fois par jour.}}}\end{otherlanguage}\pend
           \pstart
           \begin{otherlanguage}{french}\textcolor{gray}{\textbf{\textbf{Bureau à \textcolor{pink}{Paris}{}\ledrightnote{\textcolor{pink}{Paris}}}}}\end{otherlanguage}\pend
           \pstart
           \begin{otherlanguage}{french}\textcolor{gray}{\textbf{\textbf{\textcolor{pink}{24. Rue Feydeau}{}\ledrightnote{\textcolor{pink}{rue Feydeau}}.}}}\end{otherlanguage}\hfill \textsc{\textcolor{pink}{Paris}{}\ledrightnote{\textcolor{pink}{Paris}}}, 23. Januar.\pend
           \pstart\center{}Mein lieber Freund,\pend\pstart
           Wann iſt alſo die \label{K_L02765-1v}\edtext{\textcolor{pink}{Berlin}{}\ledrightnote{\textcolor{pink}{Berlin}}er \textcolor{green}{Aufführung}{}\ledrightnote{→\textcolor{green}{Liebelei. Schauspiel in drei Akten}}}{\lemma{\textnormal{\emph{Berliner Aufführung}}}\Cendnote{\textnormal{Die Premiere der \emph{\textcolor{green}{Liebelei}} am \textcolor{pink}{Deutschen Theater
                     Berlin} fand am 4. 2. 1896 unter Anwesenheit \textcolor{blue}{Schnitzler}s statt.}}}\label{K_L02765-1h}? Ich ſehe mit Vergnügen, wie ein Stück nach dem
               andern dort durchfällt: \textsc{\label{K_L02765-88v}\edtext{\textcolor{blue}{\textcolor{green}{Hauptmann}{}\ledrightnote{→\textcolor{green}{Florian Geyer. Die Tragödie des Bauernkrieges}}}{}\ledrightnote{\textcolor{blue}{Gerhart Hauptmann}}}{\lemma{\textnormal{\emph{Hauptmann}}}\Cendnote{\textnormal{\textcolor{blue}{Gerhart Hauptmann}: \emph{\textcolor{green}{Florian Geyer. Die Tragödie des Bauernkrieges}} hatte am
                        4. 1. 1896 am \emph{\textcolor{brown}{Deutschen
                        Theater}} in \textcolor{pink}{Berlin} die
                     Uraufführung.}}}\label{K_L02765-88h}}, \textsc{\label{K_L02765-77v}\edtext{\textcolor{blue}{\textcolor{green}{Halbe}{}\ledrightnote{→\textcolor{green}{Lebenswende. Tragikomödie in 5 Akten}}}{}\ledrightnote{\textcolor{blue}{Max Halbe}}}{\lemma{\textnormal{\emph{Halbe}}}\Cendnote{\textnormal{\textcolor{blue}{Max Halbe}: \emph{\textcolor{green}{Lebenswende. Tragikomödie in 5 Akten}} hatte am
                        21. 1. 1896 am \emph{\textcolor{brown}{Deutschen
                        Theater}} in \textcolor{pink}{Berlin} die Uraufführung.
                  }}}\label{K_L02765-77h}}{ }\textsc{etc}., das iſt vom Schickſal glänzend arrangirt, um Deinen
               Erfolg \strikeout{ins rech} das nöthige Relief zu geben. Mein College
               { }\textsc{\textcolor{blue}{Wolff}{}\ledrightnote{\textcolor{blue}{Theodor Wolff}}} vom »\textcolor{brown}{Berl. Tageblatt}{}\ledrightnote{\textcolor{brown}{Berliner Tageblatt}}«, der Dir zu Deinem
                  \textcolor{pink}{Frankfurt}{}\ledrightnote{\textcolor{pink}{Frankfurt am Main}}er Erfolge gratuliren läßt, läßt Dich
               auch fragen, ob er Dir in \textcolor{pink}{Berlin}{}\ledrightnote{\textcolor{pink}{Berlin}} irgendwie mit
               Einführungen dienen kann? Er kennt dort natürlich {\pb}die ganze Welt. Ich glaube, die beſte Einführung iſt Dein \textcolor{green}{Stück}{}\ledrightnote{→\textcolor{green}{Liebelei. Schauspiel in drei Akten}} und Deine Perſon. Immerhin wollte ich
               Dir doch das Anerbieten übermitteln.\pend
           \pstart
           \textsc{\textcolor{blue}{Thorel}{}\ledrightnote{\textcolor{blue}{Jean Thorel}}} habe ich lange nicht geſehen; aber ſobald ich Zeit habe, ſuche ich ihn auf.\pend
           \pstart
           Daß Dir das \label{K_L02765-2v}\edtext{Opernglas}{\lemma{\textnormal{\emph{Opernglas}}}\Cendnote{\textnormal{siehe Paul Goldmann an Arthur Schnitzler, 11. 1. [1896]}}}\label{K_L02765-2h} gefällt, erſtaunt mich. Mir gefällt es nicht. Aber im Theater hat es ſich
               wohl bewährt? Ja? Was ſoll ich mit den 5 \textsc{Frcs} 40 machen,
               die mir von der Kaufſumme übrig bleiben?\pend
           \pstart
           \textsc{\textcolor{blue}{Bahr}{}\ledrightnote{\textcolor{blue}{Hermann Bahr}}s} kleine \label{K_L02765-5v}\edtext{Erbärmlichkeiten}{\lemma{\textnormal{\emph{Erbärmlichkeiten}}}\Cendnote{\textnormal{Am 21. 1. 1896 kam es zu einer Aussprache zwischen \textcolor{blue}{Schnitzler} und \textcolor{blue}{Bahr}, die sowohl den Freundeskreis betraf als auch die
                     Reaktion \textcolor{blue}{Bahr}s auf den Erfolg der \emph{\textcolor{green}{Liebelei}}.}}}\label{K_L02765-5h} ſind recht heiter; {\pb}es werden ſchon größere nach folgen, ſei beruhigt!
               Die »\textcolor{green}{Zeit}{}\ledrightnote{\textcolor{green}{Die Zeit. Wiener Wochenschrift}}« leſe ich kaum mehr; ſie iſt gar zu
               ſchlecht geworden. Höchſtens hier und da ein Artikel von \textsc{\textcolor{blue}{Loris}{}\ledrightnote{\textcolor{blue}{Hugo von Hofmannsthal}}}, und auch an dem habe ich wenig Freude. Ich wende mich immer mehr von ihm ab,
               und vor Allem werde ich ihm nie verzeihen, daß er nicht in entſchiedener Weiſe
               zwiſchen Dir und \textsc{\textcolor{blue}{Bahr}{}\ledrightnote{\textcolor{blue}{Hermann Bahr}}} gewählt hat. Lieſt Du \label{K_L02765-3v}\edtext{\textsc{\textcolor{blue}{Kanner}{}\ledrightnote{\textcolor{blue}{Heinrich Kanner}}s}{ }\textcolor{green}{Feuilletons}{}\ledrightnote{→\textcolor{green}{[?? Feuilletons aus China]}} aus \textcolor{pink}{China}{}\ledrightnote{\textcolor{pink}{China}}}{\lemma{\textnormal{\emph{Kanners … China}}}\Cendnote{\textnormal{\textcolor{blue}{Heinrich Kanner} war im Auftrag der \emph{\textcolor{brown}{Frankfurter Zeitung}} nach \textcolor{pink}{China} gereist und publizierte seine Reiseeindrücke in
                  dieser Zeitung. Teilweise wurden sie aber auch in der Wochenschrift \emph{\textcolor{green}{Die Zeit}} nachgedruckt.}}}\label{K_L02765-3h}? Sie ſind
               erbärmlich. Der \textcolor{blue}{Mann}{}\ledrightnote{→\textcolor{blue}{Heinrich Kanner}} hat
               keine Augen und ſieht nichts.\pend
           \pstart
           {\pb}Natürlich waren meine Leute in \textsc{\textcolor{pink}{Frankfurt}{}\ledrightnote{\textcolor{pink}{Frankfurt am Main}}} von Dir entzückt, beſonders meine \textcolor{blue}{Mutter}{}\ledrightnote{→\textcolor{blue}{Clementine Goldmann}}. Mein \textcolor{blue}{Schwager}{}\ledrightnote{→\textcolor{blue}{Josef Rosengart}} findet, Du hätteſt Ähnlichkeit mit mir. Bedank’ Dich
               bei ihm für das Compliment.\pend
           \pstart
           Deine \label{K_L02765-4v}\edtext{Zweifel, Melancholien und
                  Hypochondrien}{\lemma{\textnormal{\emph{Zweifel, … Hypochondrien}}}\Cendnote{\textnormal{siehe A. S.: \emph{Tagebuch}, 27. 1. 1896, siehe A. S.: \emph{Tagebuch}, 29. 1. 1896}}}\label{K_L02765-4h} nehme ich recht gleichmüthig auf. Das heißt, es thut mir innig leid, daß Du
               von alledem gequält wirſt. Aber da man auf \strikeout{Erden
                  ſchon} Erden ſchon einmal gequält werden muß, ſo iſt es beſſer, daß das Leid
               in dieſer Form an Dich \strikeout{heran}{ }{\pb}herantritt, als in einer andern! In dem, was Du
               ſchreibſt, iſt nichts, was nicht normal wäre. Du biſt ein großes Talent, und Du mußt
               infolgedeſſen naturnothwendig zu Zeiten glauben, daß Du es \uline{nicht} biſt. All’ das, was Du von Deinen Verſtimmungen ſchilderſt, – das iſt
               der \strikeout{M} Nebel, der im Grunde jeder Künſtlerſeele braut,
                  \strikeout{und} – der Schöpfungsnebel, aus dem die Kunſtwerke
               erſtehen. Und ſo iſt des Künſtlers Erdenwallen: durch Verſtimmungen zur Stimmung! {\dots} Daß Dir {\pb}die
               Vergänglichkeit des Lebens wehthut, iſt traurig. Aber ich kann Dir darauf nur immer
               antworten: Wenn Du, wie jemand Anderer, den ich kenne, bereits immer am 15. jedes
               Monats mit Deinem Gehalt fertig wäreſt und nicht wüßteſt, woher Du Geld nehmen
               ſollſt, um weiter zu leben und Schulden zu zahlen – ſo hätteſt Du keine Zeit, Dich um
               die Vergänglichkeit des Lebens zu ſorgen. Und – ganz im Ernſt geſprochen – es iſt
               beſſer, vor dem Tode zu zittern, als vor {\pb}dem
               Schneider, der die unbezahlte Rechnung präſentiren kommt. Du haſt die edleren
               Schmerzen, mein lieber Freund – und ſelbſt hier biſt Du ein »Sonntagskind«. Und wenn
               ich Deinen Kummer leſe, ſo ruft das in mir nur ein Gefühl des – Neides wach. Oh wenn
               ich auch ſo \strikeout{leid} leiden könnte, wie dieſer glückliche
               junge Mann! Und dann: Du erlebſt nichts zu Ende. Aber wenigſtens erlebſt Du etwas.
               Aber ich kenne {\pb}Leute, bei \strikeout{denn} denen es im ganzen Leben nie auch nur zum Anfang kommt. Das iſt das
               Entſetzliche, wenn man ſieht, wie das Leben vorüberraſt – wenn man mitleben möchte
               und nicht die Kraft dazu hat – wenn man eines ſchönen Tages \strikeout{en} entdeckt, daß die Jugend vorbei iſt, ohne daß man jemals jung war – und
               wenn man genau weiß, daß das immer ſo ſein wird und daß man eines \strikeout{Ta} anderen ſchönen Tages auf das {\pb}ganze Leben zurückblicken wird mit dem Bewußtſein,
               mit der zehrenden Reue, daß man nie gelebt hat! Du hingegen – Du lebſt! Kein
               glühendes Gefühl des Daſeins – meinetwegen! Aber wo iſt es, dieſes glühende Gefühl,
               als bei den ganz Animaliſchen? Und auch bei denen, glaube ich, iſt es nicht ſo
               glühend. Ich meine, auch das iſt ein Ideal, das nicht exiſtirt. Alles Menſchliche iſt
                  \strikeout{unv} unvollkommen, und ich glaube, nicht einmal {\pb}\uline{leben} können wir ordentlich. Nicht Du allein –
               keiner! Es gibt keine ganzen, keine glühenden Gefühle. Oder doch\strikeout{,} ein einziges: die Sehnſucht. Was wir nicht haben – oh
               ja, in dem iſt Gluth, Schönheit und Vollendung. Aber in dem, was wir haben, – in dem,
               was wir leben, – da iſt Alles halb, jämmerlich und ungefähr.\pend
           \pstart
           {\pb}Schreib weiter an Deinem \textcolor{green}{Stücke}{}\ledrightnote{→\textcolor{green}{Freiwild. Schauspiel in 3 Akten}}, mein theurer Freund,
               und ſei guter Dinge!\pend
           \pstart
           In Treue {\\[\baselineskip]}Dein {\\[\baselineskip]}\spacefill\mbox{Paul Goldmann}\pend
           \leftskip=0em{}\pstart
           \noindent{}Und grüß’ mir meinen lieben \textsc{\textcolor{blue}{Richard}{}\ledrightnote{\textcolor{blue}{Richard Beer-Hofmann}}}.\pend
           \endnumbering\briefempfaengerindex{Schnitzler, Arthur@\textsc{Schnitzler, Arthur}!zzzGoldmann, Paul@\emph{von Paul Goldmann}!1896-01-231@{23. 1. {[}1896{]}}|)be}\mylabel{h}  \normalsize

\doendnotes{C}
\bigskip
\vfill

\clearpage

\footnotesize

\lohead{\textsc{register}}

% Definiere theindex-Environment komplett neu ohne reledmac
\makeatletter
\renewenvironment{theindex}{%
  \section*{\indexname}%
  \setlength{\parindent}{0pt}%
  \setlength{\parskip}{0pt plus 0.3pt}%
  \let\item\@idxitem
}{%
  \clearpage
}
\makeatother

\IfFileExists{\jobname-pw.ind}{\input{\jobname-pw.ind}}{}

\end{document}

      