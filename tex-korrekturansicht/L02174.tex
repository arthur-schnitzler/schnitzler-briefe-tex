%% latex-korrekturansicht-vorspann.tex
%% Vorspann für die Korrekturansicht.
%% Lädt die gemeinsame Datei latex-vorspann.tex mit gesetztem Schalter.

\newif\ifkorrekturansicht
\korrekturansichttrue

\input{../tex-inputs/latex-vorspann}


               \section[Arthur und Olga Schnitzler an Richard Beer-Hofmann, 9. 4. 1914]{ Arthur und Olga Schnitzler an Richard Beer-Hofmann,
                    9. 4. 1914}\nopagebreak\mylabel{v}\rehead{ }\normalsize\beginnumbering\briefempfaengerindex{Beer-Hofmann, Richard@\textsc{Beer-Hofmann, Richard}!zzzSchnitzler, Olga@\emph{von Olga Schnitzler}!1914-04-091@{9. 4. 1914}|(be}\briefempfaengerindex{Beer-Hofmann, Richard@\textsc{Beer-Hofmann, Richard}!zzzSchnitzler, Arthur@\emph{von Arthur Schnitzler}!1914-04-091@{9. 4. 1914}|(be} \toendnotes[C]{\smallbreak\pagebreak[2]} \Standort{YCGL, MSS 31.}
\physDesc{Bildpostkarte
\newline{}Handschrift Arthur Schnitzler: schwarze Tinte, deutsche Kurrent\newline{}Handschrift Olga Schnitzler: schwarze Tinte, deutsche Kurrent\newline{}Versand: 1) Stempel: »\nobreak{}Wien 111, 10. IV. 14, \textcolor{gray}{4}\nobreak{}«.  2) Stempel: »\nobreak{}\oindex{Menton@\textbf{Menton}, \emph{http://www.geonames.org/ontologyP.PPL}|pwk}Menton Alpes
                                                  Maritimes, 14–4 14\nobreak{}«. 3) mit blauem Buntstift von unbekannter Hand Adresse gestrichen und ersetzt durch: »\noindent{}\textcolor{pink}{\textsc{Hôtel Regina Palace}}{ / }\textcolor{pink}{\textsc{Menton}}«\newline{}Zusatz: Postkartenmotiv mit Olga und \textcolor{blue}{Heinrich} links vor dem Haus und Schnitzler und
                                                \textcolor{blue}{Lili} auf dem
                                            Söller }\buchAbdrucke{\weitereDrucke{Arthur Schnitzler, Richard Beer-Hofmann: \emph{Briefwechsel 1891–1931}. Hg. Konstanze Fliedl. Wien, Zürich: \emph{Europaverlag} 1992, S. 219.} }\pstart{}{\pb}Hrn \textsc{Dr. Richard}\pend{}\pstart{}\textsc{Beerhofmann}\pend{}\pstart{}aus \textcolor{pink}{Wien}{}\ledrightnote{\textcolor{pink}{Wien}}\pend{}\pstart{}\textcolor{pink}{\textsc{Kap d’Ail}}{}\ledrightnote{\textcolor{pink}{Cap-d'Ail}}\pend{}\pstart{}\textcolor{pink}{\textsc{Riviera}}{}\ledrightnote{\textcolor{pink}{Riviera}}\pend{}\pstart{}\textsc{\textcolor{pink}{Frankreich}{}\ledrightnote{\textcolor{pink}{Frankreich}}}\pend{}{\bigskip}\pstart
           \noindent{}\textcolor{gray}{\textbf{{\pb}\textcolor{pink}{Wien, XVIII, Sternwartestr. 71}{}\ledrightnote{\textcolor{pink}{Sternwartestraße}}.}}\pend
           \pstart
           {\pb}9. 4. 914.\pend
           \pstart
           Herzliche Grüße. Wir freuen uns daſs es Ihnen wohl gefällt u gut geht; und wagen
                    es ohne Fragezeichen, die auch nichts nützen, ein gutes Wiederſehen zu
                    erhoffen.\pend
           \pstart
           Ihr \spacefill\mbox{Arthur}{\\[\baselineskip]}\spacefill\mbox{{[}hs. O. Schnitzler:{]} Olga}\pend
           \leftskip=0em{}\endnumbering\briefempfaengerindex{Beer-Hofmann, Richard@\textsc{Beer-Hofmann, Richard}!zzzSchnitzler, Olga@\emph{von Olga Schnitzler}!1914-04-091@{9. 4. 1914}|)be}\briefempfaengerindex{Beer-Hofmann, Richard@\textsc{Beer-Hofmann, Richard}!zzzSchnitzler, Arthur@\emph{von Arthur Schnitzler}!1914-04-091@{9. 4. 1914}|)be}\mylabel{h}  \normalsize

\doendnotes{C}
\bigskip
\vfill

\clearpage

\footnotesize

\lohead{\textsc{register}}

% Definiere theindex-Environment komplett neu ohne reledmac
\makeatletter
\renewenvironment{theindex}{%
  \section*{\indexname}%
  \setlength{\parindent}{0pt}%
  \setlength{\parskip}{0pt plus 0.3pt}%
  \let\item\@idxitem
}{%
  \clearpage
}
\makeatother

\IfFileExists{\jobname-pw.ind}{\input{\jobname-pw.ind}}{}

\end{document}

      