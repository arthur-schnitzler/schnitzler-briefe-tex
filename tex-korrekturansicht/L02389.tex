%% latex-korrekturansicht-vorspann.tex
%% Vorspann für die Korrekturansicht.
%% Lädt die gemeinsame Datei latex-vorspann.tex mit gesetztem Schalter.

\newif\ifkorrekturansicht
\korrekturansichttrue

\input{../tex-inputs/latex-vorspann}


               \section[Arthur Schnitzler an Hermann Bahr, 26. 6. 1922]{ Arthur Schnitzler an Hermann Bahr, 26. 6. 1922}\nopagebreak\mylabel{v}\rehead{ }\normalsize\beginnumbering\briefempfaengerindex{Bahr, Hermann@\textsc{Bahr, Hermann}!zzzSchnitzler, Arthur@\emph{von Arthur Schnitzler}!1922-06-261@{26. 6. 192{[}2{]}}|(be} \toendnotes[C]{\smallbreak\pagebreak[2]} \Standort{TMW, HS AM 60133 Ba.}
\physDesc{Briefkarte
\newline{}Handschrift: schwarze Tinte, lateinische Kurrent}\buchAbdrucke{\weitereDrucke{1) \emph{26. 6. 1920, Abschrift.} In: Arthur Schnitzler: \emph{The Letters of Arthur Schnitzler to Hermann Bahr}. Edited, annotated, and with an introduction, by Donald G.
                        Daviau. Chapel Hill: \emph{The University of North Carolina Press} 1978, S. 115 (University of North Carolina studies in the Germanic languages
                        and literatures, 89).} \weitereDrucke{2) Hermann Bahr, Arthur Schnitzler: \emph{Briefwechsel, Aufzeichnungen, Dokumente (1891–1931)}. Hg. Kurt Ifkovits und Martin Anton Müller. Göttingen: \emph{Wallstein} 2018, S. 562.} }\pstart
           \raggedleft{}{\pb}\textcolor{pink}{Wien}{}\ledrightnote{\textcolor{pink}{Wien}}{ }26. 6. 22
                  \pend
           \pstart
           lieber Hermann, darf ich dir Mr \textcolor{blue}{Scofield Thayer}{}\ledrightnote{\textcolor{blue}{Scofield Thayer}} vorstellen, Herausgeber der \textcolor{brown}{Dial}{}\ledrightnote{\textcolor{brown}{The Dial}}, einer der charmantesten und anregendsten jungen \textcolor{pink}{Amerikaner}{}\ledrightnote{\textcolor{pink}{Amerika}}, die mir begegnet sind! Ich sage nicht mehr, denn
               ich hoffe, du wirst dir die Zeit nehmen Mr \textcolor{blue}{Thayer}{}\ledrightnote{\textcolor{blue}{Scofield Thayer}}{ }{\pb}einmal zu empfangen
               und ebenso persönlich kennen lernen.\pend
           \pstart
           Er bringt dir meine herzlichsten Grüße.\pend
           \pstart
           Auf Wiedersehen!{\\[\baselineskip]}Dein getreuer{\\[\baselineskip]}\spacefill\mbox{Arthur}\pend
           \leftskip=0em{}\endnumbering\briefempfaengerindex{Bahr, Hermann@\textsc{Bahr, Hermann}!zzzSchnitzler, Arthur@\emph{von Arthur Schnitzler}!1922-06-261@{26. 6. 192{[}2{]}}|)be}\mylabel{h}  \normalsize

\doendnotes{C}
\bigskip
\vfill

\clearpage

\footnotesize

\lohead{\textsc{register}}

% Definiere theindex-Environment komplett neu ohne reledmac
\makeatletter
\renewenvironment{theindex}{%
  \section*{\indexname}%
  \setlength{\parindent}{0pt}%
  \setlength{\parskip}{0pt plus 0.3pt}%
  \let\item\@idxitem
}{%
  \clearpage
}
\makeatother

\IfFileExists{\jobname-pw.ind}{\input{\jobname-pw.ind}}{}

\end{document}

      