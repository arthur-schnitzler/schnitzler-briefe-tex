%% latex-korrekturansicht-vorspann.tex
%% Vorspann für die Korrekturansicht.
%% Lädt die gemeinsame Datei latex-vorspann.tex mit gesetztem Schalter.

\newif\ifkorrekturansicht
\korrekturansichttrue

\input{../tex-inputs/latex-vorspann}


               \section[Hugo von Hofmannsthal an Arthur Schnitzler, 27. 7. 1906]{ Hugo von Hofmannsthal an Arthur Schnitzler, 27. 7. 1906}\nopagebreak\mylabel{v}\rehead{ }\normalsize\beginnumbering\briefempfaengerindex{Schnitzler, Arthur@\textsc{Schnitzler, Arthur}!zzzHofmannsthal, Hugo von@\emph{von Hugo von Hofmannsthal}!1906-07-271@{27. 7. 1906}|(be} \toendnotes[C]{\smallbreak\pagebreak[2]} \Standort{CUL, Schnitzler, B 43.}
\physDesc{Bildpostkarte
\newline{}Handschrift: Bleistift, deutsche Kurrent\newline{}Versand: 1) Stempel: »\nobreak{}\oindex{St. Gilgen@\textbf{St. Gilgen}, \emph{Besiedelter Ort (A.BSO)}|pwk}St. Gilgen\nobreak{}«.  2) Stempel: »\nobreak{}\oindex{Helsingør@\textbf{Helsingør}, \emph{Besiedelter Ort (A.BSO)}|pwk}Helsingør, 29. 7. 06, 11–12F\nobreak{}«. \newline{}Ordnung: 1) mit Bleistift von unbekannter Hand nummeriert: »\strikeout{262}« 2) mit Bleistift von unbekannter Hand nummeriert: »263«}\buchAbdrucke{\weitereDrucke{Hugo von Hofmannsthal, Arthur Schnitzler: \emph{Briefwechsel}. Hg. Therese Nickl und Heinrich Schnitzler. Frankfurt am Main: \emph{S. Fischer} 1964, S. 220.} }\toendnotes[C]{\smallbreak}\pstart{}{\pb}\textsc{Herrn Arthur Schnitzler}\pend{}\pstart{}\textsc{\textcolor{pink}{Marienlyst}{}\ledrightnote{\textcolor{pink}{Marienlyst}}}\pend{}\pstart{}\textsc{\textcolor{pink}{Kurhaus}{}\ledrightnote{\textcolor{pink}{Kurhotellet}}}\pend{}\pstart{}\textsc{per \textcolor{pink}{Kopenhagen}{}\ledrightnote{\textcolor{pink}{Kopenhagen}}}\pend{}\pstart{}\textsc{\textcolor{pink}{Dänemark}{}\ledrightnote{\textcolor{pink}{Dänemark}}}\pend{}{\bigskip}\pstart
           \noindent{}\centering{}\textcolor{gray}{\textbf{{\pb}\textcolor{pink}{Gasthof Lueg}{}\ledrightnote{\textcolor{pink}{Hotel und Pension Lueg}}. \textcolor{pink}{Lueg}{}\ledrightnote{\textcolor{pink}{Lueg am Wolfgangsee}} bei \textcolor{pink}{St. Gilgen}{}\ledrightnote{\textcolor{pink}{St. Gilgen}} am \textcolor{pink}{Wolfgang-}{}\ledrightnote{\textcolor{pink}{Wolfgangsee}}(Aber-) See.}}\pend
           \pstart
           {\pb}27.\pend
           \pstart
           Freuen uns herzlich von Ihnen Gutes zu hören.\pend
           \pstart
           Gehen nächſte Woche 2 Tage \textcolor{pink}{Baireuth}{}\ledrightnote{\textcolor{pink}{Bayreuth}}, ſind hier bis
                  Ende Auguſt mit allen \textcolor{blue}{Kindern}{}\ledrightnote{→\textcolor{blue}{Christiane von Hofmannsthal}{\newline}→\textcolor{blue}{Raimund von Hofmannsthal}{\newline}→\textcolor{blue}{Franz von Hofmannsthal}}. Bitte bald wieder paar Zeilen.\pend
           \pstart
           Ihr{\\[\baselineskip]}\spacefill\mbox{Hugo.}\pend
           \leftskip=0em{}\endnumbering\briefempfaengerindex{Schnitzler, Arthur@\textsc{Schnitzler, Arthur}!zzzHofmannsthal, Hugo von@\emph{von Hugo von Hofmannsthal}!1906-07-271@{27. 7. 1906}|)be}\mylabel{h}  \normalsize

\doendnotes{C}
\bigskip
\vfill

\clearpage

\footnotesize

\lohead{\textsc{register}}

% Definiere theindex-Environment komplett neu ohne reledmac
\makeatletter
\renewenvironment{theindex}{%
  \section*{\indexname}%
  \setlength{\parindent}{0pt}%
  \setlength{\parskip}{0pt plus 0.3pt}%
  \let\item\@idxitem
}{%
  \clearpage
}
\makeatother

\IfFileExists{\jobname-pw.ind}{\input{\jobname-pw.ind}}{}

\end{document}

      