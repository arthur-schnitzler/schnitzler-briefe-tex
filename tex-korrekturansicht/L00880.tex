%% latex-korrekturansicht-vorspann.tex
%% Vorspann für die Korrekturansicht.
%% Lädt die gemeinsame Datei latex-vorspann.tex mit gesetztem Schalter.

\newif\ifkorrekturansicht
\korrekturansichttrue

\input{../tex-inputs/latex-vorspann}


               \section[Arthur Schnitzler an Georg Brandes, 12. 1. 1899]{ Arthur Schnitzler an Georg Brandes, 12. 1. 1899}\nopagebreak\mylabel{v}\rehead{ }\normalsize\beginnumbering\briefempfaengerindex{Brandes, Georg@\textsc{Brandes, Georg}!zzzSchnitzler, Arthur@\emph{von Arthur Schnitzler}!1899-01-121@{12. 1. 1899}|(be} \toendnotes[C]{\smallbreak\pagebreak[2]} \Standort{Kopenhagen, Det Kongelige Bibliotek, Georg Brandes Arkiv, box 125.}
\physDesc{Brief, 3 Blätter, 11 Seiten
\newline{}Handschrift: schwarze Tinte, deutsche Kurrent\newline{}Ordnung: mit Bleistift von unbekannter Hand beschriftet: »Schnitzler
                                            12. 1.99.« und mit Bleistift
                                    nummeriert »13.«, das zweite Blatt mit
                                        »2« versehen und auf dieses und das dritte
                                    erneut das Datum vermerkt: »12/1 99« }\buchAbdrucke{\weitereDrucke{1) Georg Brandes, Arthur Schnitzler: \emph{Ein Briefwechsel}. Hg. Kurt Bergel. Bern: \emph{Francke} 1956, S. 70–72.} \weitereDrucke{2) Arthur Schnitzler: \emph{Briefe 1875–1912}. Hg. Therese Nickl und Heinrich Schnitzler. Frankfurt am Main: \emph{S. Fischer} 1981, S. 366–368.} }\toendnotes[C]{\smallbreak}\pstart{}{\pb}Verehrter Herr Brandes,\pend\pstart
           geſtern hab ich Ihren Brief bekommen und aus dem erfahren, dſs Sie wieder zu
                    Bette liegen. Abends ſtand es \label{K_L00880_1v}\edtext{in
                    einer \textcolor{green}{Berliner Zeitung}{}\ledrightnote{→\textcolor{green}{Berliner Tageblatt}} zu
                        leſen}{\lemma{\textnormal{\emph{in … leſen}}}\Cendnote{\textnormal{Vgl. \textcolor{blue}{V. A.}: \emph{\textcolor{green}{Bei Georg Brandes}}. In: \emph{\textcolor{green}{Berliner Tageblatt}}, Jg. 28, Nr. 16,
                                9. 1. 1899, Abend-Ausgabe, S. 3: »Aus \textcolor{pink}{\so{Kopenhagen}}{ }schreibt uns unser dortiger
                                Korrespondent: Dr. \textcolor{blue}{Georg
                                    Brandes} muß leider wieder das Bett hüten und zwar wegen
                                seines alten Leidens: Venenentzündung. Ich besuchte gestern den
                                berühmten Autor. {[}{\dots}{]} ›Und jetzt liege ich hier seit drei Wochen auf meinem
                                Schmerzenslager,‹ sagte \textcolor{blue}{Brandes}
                                mit einem matten Lächeln; ›wann und wie die Aerzte mir wieder auf
                                die Beine helfen können, wissen sie ja selber nicht.‹{ / }{[}{\dots}{]} Eine Besserung ist jedoch augenscheinlich eingetreten,
                                welche hoffentlich fortschreiten wird.«.}}}\label{K_L00880_1h}, mit dem Beiſatz, dſs Sie ſich ſchon auf dem Weg der
                    Beſſerung befinden. Ich hoffe, daſs es ſich ſo verhält und daſs Sie bald ganz
                    geſund \substVorne{}\textsuperscript{iſt}\substDazwischen{}ſind\substHinten{}. Meine innigſten Wünſche ſind bei Ihnen, {\pb}das wiſſen Sie. Auch von Ihrem Streit mit
                    den \label{K_L00880_2v}\edtext{Deutſchen hab ich durch die
                        Zeitung}{\lemma{\textnormal{\emph{Deutſchen … Zeitung}}}\Cendnote{\textnormal{Vgl. [O. V.:] \emph{\textcolor{green}{Köllers Erfolge}}. In: \emph{\textcolor{green}{Berliner Tageblatt}}, Jg. 28, Nr. 9,
                                5. 1. 1899, Abend-Ausgabe, S. 2: »\textcolor{blue}{\so{Georg Brandes}}, der vom ›\textcolor{brown}{\so{Verein Berliner Presse}}‹ aufgefordert worden war, nach \textcolor{pink}{Berlin} zu kommen, um einen Vortrag zum Besten der Hilfskasse
                            des genannten Vereins zu halten, hat geantwortet, daß ein \textcolor{pink}{\so{dänischer}}\so{ Autor} während der gegenwärtigen Verhältnisse
                            in \textcolor{pink}{Nordschleswig} unmöglich Vorträge in
                                \textcolor{pink}{Berlin} halten
                        könne.«}}}\label{K_L00880_2h} erfahren; Sie ſollen irgend einen Vortrag abgeſagt
                    haben, im Verein »\textcolor{brown}{Berliner Preſſe}{}\ledrightnote{\textcolor{brown}{Verein Berliner Presse}}«, aus »polit.
                    Gründen«. Fügen Sie Ihren Antipathien gegen \strikeout{De}\textcolor{pink}{Preußen}{}\ledrightnote{\textcolor{pink}{Preußen}} und \textcolor{pink}{Frankreich}{}\ledrightnote{\textcolor{pink}{Frankreich}} nur getroſt \introOben{}die\introOben{} gegen \textcolor{pink}{Oeſterreich}{}\ledrightnote{\textcolor{pink}{Österreich}} bei. Leſen Sie manchmal \textcolor{pink}{Wien}{}\ledrightnote{\textcolor{pink}{Wien}}er Zeitungen, Parlaments- und
                    Gemeinderathsberichte? Es iſt ſtaunenswerth, unter was für Schweinen wir hier
                    leben; – und {\pb}ich denke i{\geminationm}er, ſelbſt Antiſemiten müßte es doch auffallen,
                    daſs der Antiſemitismus – von allem andern abgeſehen – jedenfalls die ſonderbare
                    Kraft hat, die verlogenſten Gemeinheiten der menſchlichen Natur zu Tage zu
                    fördern und ſie aufs höchſte auszubilden. Wie merkwürdig, daſs ſogar die
                    offenbaren Mängel, Fehler, meinetwegen Verbrechen der Judenpreſſe, die man als
                    ſo ſpezifiſch jüdiſch hinſtellen wollte, von der Antiſemiten{\pb}preſſe ins ungeheuerliche ausgebildet worden
                    ſind. Aber wir wollen über dieſe widerlichen Dinge lieber gar nicht reden.\pend
           \pstart
           Ich freue mich, dſs das »\textcolor{green}{Vermächtnis}{}\ledrightnote{\textcolor{green}{Das Vermächtnis. Schauspiel in drei Akten}}« einigen
                    Beifall bei Ihnen gefunden hat. Mir ſelbſt iſt nur der erſte Akt lieb; dann
                    gewiſſe Partien des letzten. Solange die Hauptperſon auf der Scene iſt, hab ich
                    das Stück nicht gern. Die iſt ganz unperſönlich geblieben find ich. Während der
                    Proben fiel mir mancherlei ein, wodurch ich das Stück hätte höher bringen
                    können; vor allem hätt ich das Kind {\pb}müſſen
                    am Leben laſſen; – aber es ſcheint ich bin nicht anſtändig genug, um ein Stück
                    noch auf der Probe zurückzuziehn, ſelbſt we{\geminationn} ich
                    weiſs, wie es beſſer zu machen wäre. Es hat in \textcolor{pink}{Berlin}{}\ledrightnote{\textcolor{pink}{Berlin}} un\textcolor{gray}{d}{ }\textcolor{pink}{Wien}{}\ledrightnote{\textcolor{pink}{Wien}} bei der Erſtaufführung viel Erfolg gehabt;
                    in \textcolor{pink}{Berlin}{}\ledrightnote{\textcolor{pink}{Berlin}} verſchwand es bald; hier ſcheint es
                    ſich zu halten. Irgend eine Zukunft hat es gewiſs nicht – und wahrhaftig nicht
                    nur wegen ſeiner Traurigkeit –! – Nun hab ich was geſchrieben, das mir lieber
                    iſt; drei kleine \textcolor{green}{Stücke}{}\ledrightnote{→\textcolor{green}{Die Gefährtin. Schauspiel in einem Akt}{\newline}→\textcolor{green}{Paracelsus. Versspiel in einem Akt}{\newline}→\textcolor{green}{Der grüne Kakadu. Groteske in einem Akt}}, von denen das {\pb}eine »\textcolor{green}{Der grüne Kakadu}{}\ledrightnote{\textcolor{green}{Der grüne Kakadu. Groteske in einem Akt}}«, das beſte, großen
                    Schwierigkeiten begegnet. In \textcolor{pink}{Berlin}{}\ledrightnote{\textcolor{pink}{Berlin}} haben ſie
                    es verboten; – hier will die Hofcenſur die unmöglichſten Aenderungen. Es ſpielt
                    am Abend der \textcolor{pink}{Baſtille}{}\ledrightnote{\textcolor{pink}{Bastille}}nerſtürmung zu \textcolor{pink}{Paris}{}\ledrightnote{\textcolor{pink}{Paris}} – aber ich ſoll den »Blutgeruch«
                    herausſtreichen. Auch daſs ein Herzog umgebracht wird, will den Leuten nicht
                    gefallen. Ich freu mich Ihnen das Ding bald zu ſchicken; es wird Sie
                    wahrſcheinlich amuſiren.\pend
           \pstart
           Und jetzt bin ich mit einer ganz phantaſtiſchen {\pb}fünfactigen \textcolor{green}{Sache}{}\ledrightnote{→\textcolor{green}{Der Schleier der Beatrice. Schauspiel in fünf Akten}} beſchäftigt; mir ſcheint überhaupt als käme ich
                    jetzt in andere Gegenden. Wer weiſs, ob alles bisherige nicht doch nur Tagebuch
                    war; wenigſtens von einer gewiſſen Zeit an. (Denn früher einmal, von meinem 9.
                    bis zu meinem 20. Jahr hab ich geſchrieben, »wie der Vogel ſingt« – ich muſs
                    damals ſehr glücklich geweſen ſein; de{\geminationn} ich eri{\geminationn}ere mich gar nicht, wie ichs eigentlich gemacht
                    habe. Ich habe noch manches; Trauerſpiele und Faſtnachtsſpiele und {\pb}komiſche Romane; nahezu durchaus blödſinnig;
                    aber ich habe ſelbſt zu der Zeit, da ich dieſe Dinge ſchrieb, nie das Bedürfnis
                    gehabt, es irgend wem zu zeigen. So wird man zudringlicher, niedriger und
                    unfröhlicher von Jahr zu Jahr. –)\pend
           \pstart
           Hoffentlich ſchwingt ſich \textcolor{blue}{Beer-Hofma{\geminationn}}{}\ledrightnote{\textcolor{blue}{Richard Beer-Hofmann}} auf, Ihnen ſelbſt zu ſchreiben; faul iſt er allerdings enorm. Sie wiſſen
                    wahrſcheinlich nicht einmal, dſs er \label{K_L00880_3v}\edtext{geheiratet}{\lemma{\textnormal{\emph{geheiratet}}}\Cendnote{\textnormal{Die Hochzeit hatte am
                            14. 5. 1898 in einer Synagoge in \textcolor{pink}{Wien} stattgefunden.}}}\label{K_L00880_3h} hat, \textcolor{blue}{Paula}{}\ledrightnote{\textcolor{blue}{Paula Beer-Hofmann}}, die Sie kennen {\pb}auch hat er
                    ſchon zwei Töchter, die \textcolor{blue}{Mirjam}{}\ledrightnote{\textcolor{blue}{Mirjam Beer-Hofmann}} und \textcolor{blue}{Naëmie}{}\ledrightnote{\textcolor{blue}{Naëmah Beer-Hofmann}} heißen. Aber ſeine neue \textcolor{green}{Novelle}{}\ledrightnote{→\textcolor{green}{Der Tod Georgs}} (was ich davon kenne
                    iſt wunderſchön) iſt noch nicht fertig.\pend
           \pstart
           Iſt Ihnen ein Roman bekannt, \textcolor{green}{die Juden von
                        Zirndorf}{}\ledrightnote{\textcolor{green}{Die Juden von Zirndorf}}, von \textcolor{blue}{Waſſermann}{}\ledrightnote{\textcolor{blue}{Jakob Wassermann}}? Ich
                    glaube, das iſt derjenige Menſch, der den \introOben{}deutſchen\introOben{}
                    Roman vom Anfang des nächſten Jahrhunderts ſchreiben wird. Sind Ihnen die
                    Novelletten zugeko{\geminationm}en, die ich Ihnen im Frühjahr
                    ſchickte? {\pb}(»\textcolor{green}{Frau des Weiſen}{}\ledrightnote{\textcolor{green}{Die Frau des Weisen. Novelletten}}«. –)\pend
           \pstart
           Von Ihrem Ausflug nach \textcolor{pink}{Polen}{}\ledrightnote{\textcolor{pink}{Polen}} und Ihrem Empfang
                    haben wir hier \label{K_L00880_4v}\edtext{geleſen}{\lemma{\textnormal{\emph{geleſen}}}\Cendnote{\textnormal{Die \textcolor{pink}{Wien}er Zeitungen hatten mehrfach über den Besuch \textcolor{blue}{Brandes} in \textcolor{pink}{Lemberg}
                        berichtet, so etwa die \emph{\textcolor{brown}{Neue Freie Presse}} in der ungezeichneten Meldung
                        \emph{\textcolor{green}{Georg Brandes in Lemberg}}
                            ([O. V.], Nr. 12300,
                            19. 11. 1898, Morgenausgabe, S. 4)
                        »\textcolor{blue}{Georg Brandes}, der einer
                            Einladung nach \textcolor{pink}{Lemberg} zu der am
                                20. November stattfindenden Enthüllung des \textcolor{blue}{Sobiesky}-Denkmals Folge gegeben hat,
                            wurde bei seiner Ankunft dasselbst von einer Deputation feierlich
                            empfangen. Die Spitzen der Gesellschaft wetteifern in dem Bestreben,
                            sich dem großen \textcolor{blue}{dänischen Schriftsteller} für die in seinem ebenaso geistvollen
                            als anregenden Werke ›\textcolor{green}{Polen}‹ zum
                            Ausdrucke gebrachten Sympathien erkenntlich zu zeigen.« }}}\label{K_L00880_4h};
                    dagegen hab ich von Ihren \textcolor{green}{Gedichten}{}\ledrightnote{→\textcolor{green}{Ungdomsvers [Jugendgedichte]}} abſolut nichts gewußt\substVorne{}\textsuperscript{?}\substDazwischen{}.\substHinten{} Werden Sie ſie \label{K_L00880_5v}\edtext{überſetzen}{\lemma{\textnormal{\emph{überſetzen}}}\Cendnote{\textnormal{Eine deutsche
                        Übersetzung der Jugendgedichte erschien nicht.}}}\label{K_L00880_5h} laſſen? Sind ſie
                    ſchön? Haben Sie ſie gern? Wie viele Stunden hat Ihr Tag! Zu allem haben Sie
                    Zeit. Und alles bewahren Sie auf, das iſt das Bewunderungswürdige, und darum {\pb}ſind Sie ſo reich.\pend
           \pstart
           Ich wünſchte, Sie würden gleich geſund, reiſten wieder nach \textcolor{pink}{Italien}{}\ledrightnote{\textcolor{pink}{Italien}}, und blieben wieder ein paar Tage in \textcolor{pink}{Wien}{}\ledrightnote{\textcolor{pink}{Wien}}. Ein Wort von Ihnen, wie’s Ihnen geht,
                    brächte mir jedenfalls viel Freude.\pend
           \pstart
           Herzlich grüßt Sie Ihr Ihnen {\\[\baselineskip]}treuergebener{\\[\baselineskip]}\spacefill\mbox{ArthurSchnitzler}\pend
           \leftskip=0em{}\pstart
           \textcolor{pink}{Wien}{}\ledrightnote{\textcolor{pink}{Wien}}{ }12. 1. 99.\pend
           \endnumbering\briefempfaengerindex{Brandes, Georg@\textsc{Brandes, Georg}!zzzSchnitzler, Arthur@\emph{von Arthur Schnitzler}!1899-01-121@{12. 1. 1899}|)be}\mylabel{h}  \normalsize

\doendnotes{C}
\bigskip
\vfill

\clearpage

\footnotesize

\lohead{\textsc{register}}

% Definiere theindex-Environment komplett neu ohne reledmac
\makeatletter
\renewenvironment{theindex}{%
  \section*{\indexname}%
  \setlength{\parindent}{0pt}%
  \setlength{\parskip}{0pt plus 0.3pt}%
  \let\item\@idxitem
}{%
  \clearpage
}
\makeatother

\IfFileExists{\jobname-pw.ind}{\input{\jobname-pw.ind}}{}

\end{document}

      