%% latex-korrekturansicht-vorspann.tex
%% Vorspann für die Korrekturansicht.
%% Lädt die gemeinsame Datei latex-vorspann.tex mit gesetztem Schalter.

\newif\ifkorrekturansicht
\korrekturansichttrue

\input{../tex-inputs/latex-vorspann}


               \section[Edith Brandes an Arthur Schnitzler, 26. 6. 1901]{ Edith Brandes an Arthur Schnitzler, 26. 6. 1901}\nopagebreak\mylabel{v}\rehead{ }\normalsize\beginnumbering\briefempfaengerindex{Schnitzler, Arthur@\textsc{Schnitzler, Arthur}!zzzPhilipp, Edith@\emph{von Edith Philipp}!1901-06-262@{26. 6. 1901}|(be} \toendnotes[C]{\smallbreak\pagebreak[2]} \Standort{CUL, Schnitzler, B 17.}
\physDesc{Brief, 1 Blatt, 1 Seite
\newline{}Handschrift: blaue Tinte, lateinische Kurrent\newline{}Ordnung: mit Bleistift von unbekannter Hand nummeriert:
                                    »26« }\Standort{DLA, A:Schnitzler, HS.NZ85.1.2595.}
\physDesc{1 Blatt, 1 Seite, maschinelle Abschrift}\buchAbdrucke{\weitereDrucke{Georg Brandes, Arthur Schnitzler: \emph{Ein Briefwechsel}. Hg. Kurt Bergel. Bern: \emph{Francke} 1956, S. 89.} }\toendnotes[C]{\smallbreak}\pstart
           \raggedleft{}{\pb}Mittwoch, 26-6-1901\pend
           \pstart{}Verehrter Herr Schnitzler!\pend\pstart
           Ich kenne Sie ein wenig durch die Freundschaft die mein \textcolor{blue}{Vater}{}\ledrightnote{→\textcolor{blue}{Georg Brandes}} für Sie hegt; ich habe
                    ausserdem alle Ihre Schriften gelesen. Recht sehr würden Sie mich verpflichten,
                    wollten Sie mir für mein Album, worin eine Menge grosser Männer geschrieben
                    haben ein Paar Zeilen senden.\pend
           \pstart
           Ihre grosse Bewunderin{\\[\baselineskip]}\spacefill\mbox{Edith Brandes}\pend
           \leftskip=0em{}\pstart
           \noindent{}\textcolor{pink}{Havnegade 55. Kopenhagen.}{}\ledrightnote{\textcolor{pink}{Havnegade}}\pend
           \endnumbering\briefempfaengerindex{Schnitzler, Arthur@\textsc{Schnitzler, Arthur}!zzzPhilipp, Edith@\emph{von Edith Philipp}!1901-06-262@{26. 6. 1901}|)be}\mylabel{h}  \normalsize

\doendnotes{C}
\bigskip
\vfill

\clearpage

\footnotesize

\lohead{\textsc{register}}

% Definiere theindex-Environment komplett neu ohne reledmac
\makeatletter
\renewenvironment{theindex}{%
  \section*{\indexname}%
  \setlength{\parindent}{0pt}%
  \setlength{\parskip}{0pt plus 0.3pt}%
  \let\item\@idxitem
}{%
  \clearpage
}
\makeatother

\IfFileExists{\jobname-pw.ind}{\input{\jobname-pw.ind}}{}

\end{document}

      