%% latex-korrekturansicht-vorspann.tex
%% Vorspann für die Korrekturansicht.
%% Lädt die gemeinsame Datei latex-vorspann.tex mit gesetztem Schalter.

\newif\ifkorrekturansicht
\korrekturansichttrue

\input{../tex-inputs/latex-vorspann}


               \section[Georg Brandes an Arthur Schnitzler, 13. 5. {[}1901{]}]{ Georg Brandes an Arthur Schnitzler, 13. 5. {[}1901{]}}\nopagebreak\mylabel{v}\rehead{ }\normalsize\beginnumbering\briefempfaengerindex{Schnitzler, Arthur@\textsc{Schnitzler, Arthur}!zzzBrandes, Georg@\emph{von Georg Brandes}!1901-05-131@{13. 5. 1901}|(be} \toendnotes[C]{\smallbreak\pagebreak[2]} \Standort{CUL, Schnitzler, B 17.}
\physDesc{Brief, 1 Blatt, 2 Seiten
\newline{}Handschrift: schwarze Tinte, lateinische Kurrent
\newline{}Schnitzler: mit Bleistift die Jahreszahl ergänzt: »901« \newline{}Ordnung: 1) mit Bleistift von unbekannter Hand nummeriert: »\strikeout{21}« 2) mit Bleistift von unbekannter Hand nummeriert: »22«}\buchAbdrucke{\weitereDrucke{Georg Brandes, Arthur Schnitzler: \emph{Ein Briefwechsel}. Hg. Kurt Bergel. Bern: \emph{Francke} 1956, S. 86.} }\pstart
           \raggedleft{}{\pb}\textcolor{pink}{Schloss Strzebowitz}{}\ledrightnote{\textcolor{pink}{Schloss Strzebowitz}}{\\}\textcolor{pink}{Oesterr. Schlesien}{}\ledrightnote{\textcolor{pink}{Schlesien}}{\\}13 May\pend
           \pstart{}Verehrter Freund\pend\pstart
           Es ist meine Absicht, am 16\textsuperscript{sten} um 3\textsubscript{45} in \textcolor{pink}{Wien}{}\ledrightnote{\textcolor{pink}{Wien}} anzukommen und um 8\textsubscript{25} Abends nach \textcolor{pink}{Abbazia}{}\ledrightnote{\textcolor{pink}{Opatija}} abzureisen.\pend
           \pstart
           Ich will sehr gern von der \textcolor{pink}{Nordbahn}{}\ledrightnote{\textcolor{pink}{Nordbahnhof}} zu Ihnen
                    fahren, weiss nur nicht, da ich die Lage der Bahnhöfe nicht kenne, ob es nicht
                    besser wäre, erst meinen Koffer nach der \textcolor{pink}{Südbahnstation}{}\ledrightnote{\textcolor{pink}{Südbahnhof}} zu fahren.\pend
           \pstart
           Es versteht sich von selbst, dass es mir nur lieb sein kann, Herrn \textcolor{blue}{Beer-Hofmann}{}\ledrightnote{\textcolor{blue}{Richard Beer-Hofmann}} zu treffen. Ich weiss nicht,
                    wann Sie Mittag essen, ich werde wohl im Zuge etwas frühstücken, also sagen wir
                    um 5 Uhr{ }{\pb}(oder wann es Sie passt, wer
                    weiss im voraus, wann man an einem bestimmten Tag Hunger hat?) Recht bedacht
                    überlasse ich Ihnen die Esszeit.\pend
           \pstart
           Haben Sie vielen Dank für Ihre freundliche Antwort.\pend
           \pstart
           Von Herzen{\\[\baselineskip]}Ihr{\\[\baselineskip]}\spacefill\mbox{Georg Brandes}\pend
           \leftskip=0em{}\endnumbering\briefempfaengerindex{Schnitzler, Arthur@\textsc{Schnitzler, Arthur}!zzzBrandes, Georg@\emph{von Georg Brandes}!1901-05-131@{13. 5. 1901}|)be}\mylabel{h}  \normalsize

\doendnotes{C}
\bigskip
\vfill

\clearpage

\footnotesize

\lohead{\textsc{register}}

% Definiere theindex-Environment komplett neu ohne reledmac
\makeatletter
\renewenvironment{theindex}{%
  \section*{\indexname}%
  \setlength{\parindent}{0pt}%
  \setlength{\parskip}{0pt plus 0.3pt}%
  \let\item\@idxitem
}{%
  \clearpage
}
\makeatother

\IfFileExists{\jobname-pw.ind}{\input{\jobname-pw.ind}}{}

\end{document}

      