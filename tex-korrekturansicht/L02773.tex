%% latex-korrekturansicht-vorspann.tex
%% Vorspann für die Korrekturansicht.
%% Lädt die gemeinsame Datei latex-vorspann.tex mit gesetztem Schalter.

\newif\ifkorrekturansicht
\korrekturansichttrue

\input{../tex-inputs/latex-vorspann}


               \section[Paul Goldmann an Arthur Schnitzler, Paul Goldmann an Arthur Schnitzler, 4. 5. {[}1896{]}]{ Paul Goldmann an Arthur Schnitzler, 4. 5. {[}1896{]}}\nopagebreak\mylabel{v}\rehead{ }\normalsize\beginnumbering\briefempfaengerindex{Schnitzler, Arthur@\textsc{Schnitzler, Arthur}!zzzGoldmann, Paul@\emph{von Paul Goldmann}!1896-05-041@{4. 5. {[}1896{]}}|(be} \toendnotes[C]{\smallbreak\pagebreak[2]} \Standort{DLA, A:Schnitzler, HS.NZ85.1.3166.}
\physDesc{Brief, 1 Blatt, 2 Seiten
\newline{}Handschrift: blaue Tinte, deutsche Kurrent\newline{}Beilage: maschinschriftlicher Brief mit handschriftlicher
                                 Unterschrift, 1 Blatt, 1 Seite 
\newline{}Schnitzler: mit Bleistift das Jahr »96« vermerkt }\toendnotes[C]{\smallbreak}\pstart
           \noindent{}{\pb}\textcolor{gray}{\textbf{\textbf{\textcolor{brown}{Frankfurter Zeitung}{}\ledrightnote{\textcolor{brown}{Frankfurter Zeitung}}}}}\pend
           \pstart
           \textcolor{gray}{\textbf{(\textcolor{brown}{\begin{otherlanguage}{french}Gazette de Francfort\end{otherlanguage}}{}\ledrightnote{\textcolor{brown}{Frankfurter Zeitung}}).}}\pend
           \pstart
           \textcolor{gray}{\textbf{\textbf{\begin{otherlanguage}{french}Fondateur M.\end{otherlanguage}{ }\textcolor{blue}{L. Sonnemann}{}\ledrightnote{\textcolor{blue}{Leopold Sonnemann}}.}}}\pend
           \pstart
           \begin{otherlanguage}{french}\textcolor{gray}{\textbf{\textcolor{green}{Journal}{}\ledrightnote{→\textcolor{green}{Frankfurter Zeitung}} politique,
                        financier,}}\end{otherlanguage}\pend
           \pstart
           \begin{otherlanguage}{french}\textcolor{gray}{\textbf{commercial et littéraire.}}\end{otherlanguage}\pend
           \pstart
           \begin{otherlanguage}{french}\textcolor{gray}{\textbf{\textbf{Paraissant trois fois par jour.}}}\end{otherlanguage}\pend
           \pstart
           \begin{otherlanguage}{french}\textcolor{gray}{\textbf{\textbf{Bureau à \textcolor{pink}{Paris}{}\ledrightnote{\textcolor{pink}{Paris}}}}}\end{otherlanguage}\pend
           \pstart
           \begin{otherlanguage}{french}\textcolor{gray}{\textbf{\textbf{\textcolor{pink}{24. Rue Feydeau}{}\ledrightnote{\textcolor{pink}{rue Feydeau}}.}}}\end{otherlanguage}\hfill \textsc{\textcolor{pink}{Paris}{}\ledrightnote{\textcolor{pink}{Paris}}}, 4. Mai.\pend
           \pstart
           {\pb}Entſchuldige nur, mein lieber
                  Freund. Ich habe einfach vergeſſen, den Brief mit den anderen ins Couvert
               zu legen, und den Irrthum \label{K_L02773-1v}\edtext{ſofort}{\lemma{\textnormal{\emph{ſofort}}}\Cendnote{\textnormal{nachdem der vorige Brief bereits am 4. 5. [1896] verfasst ist,
                  dürfte sich das »sofort« auf eine zu diesem Zeitpunkt bereits
                  erfolgte Beschwerde \textcolor{blue}{Schnitzler}s
                  beziehen.}}}\label{K_L02773-1h} nach der Abſendung bemerkt.\pend
           \pstart
           Herzlichſt {\\[\baselineskip]}Dein {\\[\baselineskip]}\spacefill\mbox{P. Goldm}\pend
           \leftskip=0em{}{\bigskip}\pstart
           {\pb}{[}ms.:{]} \begin{otherlanguage}{french}\textcolor{pink}{MELUN, 12 rue Doré}{}\ledrightnote{\textcolor{pink}{Rue Doré}}, ce jeudi 9 avril.\end{otherlanguage}\pend
           \pstart{}\begin{otherlanguage}{french}Cher Monsieur\end{otherlanguage},\pend\pstart
           \label{K_L02773-111v}\edtext{\begin{otherlanguage}{french}Je mets à la poste, en même temps que la présente lettre, le
                     \textcolor{green}{volume}{}\ledrightnote{→\textcolor{green}{Liebelei. Schauspiel in drei Akten}} que vous avez
                  bien voulu me prèter et que je n’ai pu vous renvoyer plus tôt, n’étant pas certain
                  de votre adresse. Je vous suis très reconnaissant de m’avoir ainsi fait connaitre
                     »\textcolor{green}{Liebelei}{}\ledrightnote{\textcolor{green}{Liebelei. Schauspiel in drei Akten}}«, que j’ai lu avec beaucoup
                  d’intérêt, et puisque vous m’avez dit que je recevrais à la \textcolor{brown}{Nouvelle Revue}{}\ledrightnote{\textcolor{brown}{Nouvelle Revue}}, les autres \textcolor{green}{écrits}{}\ledrightnote{→\textcolor{green}{Sterben. Novelle}{\newline}→\textcolor{green}{Anatol}} de M. Schnitzler, je
                  lui consacrerai certainement une \label{K_L02773-987v}\edtext{\textcolor{green}{chronique}{}\ledrightnote{→\textcolor{green}{Un jeune écrivain viennois: M. Arthur Schnitzler}}}{\lemma{\textnormal{\emph{chronique}}}\Cendnote{\textnormal{\textcolor{blue}{Christian Schefer}: \emph{\textcolor{green}{Un jeune écrivain viennois: M. Arthur Schnitzler}}.
                        In: \emph{\textcolor{green}{La Nouvelle Revue}}, Jg. 18, Nr. 100,
                           Mai–Juni 1896,
                        S. 855–859. Siehe Paul Goldmann an Arthur Schnitzler, 2. 4. [1896]}}}\label{K_L02773-987h}.\end{otherlanguage}}{\lemma{\textnormal{\emph{Je … chronique.}}}\Cendnote{\textnormal{französisch: Lieber Herr, ich
                     retourniere mit dem vorliegenden Brief das \textcolor{green}{Buch}, das Sie mir liehen und das ich nicht früher
                     zurückschicken konnte, weil ich mir Ihrer Adresse nicht sicher war. Ich bin
                     Ihnen sehr dankbar, dass Sie mich mit »\textcolor{green}{Liebelei}« bekannt gemacht haben, das ich mit großem Interesse gelesen
                     habe; und da Sie mir gesagt haben, dass ich an die \textcolor{brown}{Nouvelle Revue} auch die anderen \textcolor{green}{Schriften} von
                     Herrn \textcolor{blue}{Schnitzler} gesandt bekomme, werde
                     ich ihm sicherliche eine \textcolor{green}{Besprechung} widmen. Sehr geehrter Herr, in Verbindung mit erneutem
                     Dank verbleibe ich mit freundlichen Grüßen.}}}\label{K_L02773-111h}\pend
           \pstart
           \label{K_L02773-88v}\edtext{\begin{otherlanguage}{french}Agréez, Cher Monsieur, en même temps que mes nouveaux
                  remerciements, l’assurance de mes sentiments très distingués.\end{otherlanguage}}{\lemma{\textnormal{\emph{Agréez, … distingués.}}}\Cendnote{\textnormal{französisch: »Nehmen
                     Sie, verehrter Herr, zusammen mit meinem neuerlichen Dank die Versicherung
                     meiner vorzüglichsten Gefühle entgegen.«}}}\label{K_L02773-88h}\pend
           \pstart \spacefill\mbox{{[}hs. Schefer:{]} \textcolor{blue}{Christian Schefer}{}\ledrightnote{\textcolor{blue}{Christian Schefer}}.}\pend{}\endnumbering\briefempfaengerindex{Schnitzler, Arthur@\textsc{Schnitzler, Arthur}!zzzGoldmann, Paul@\emph{von Paul Goldmann}!1896-05-041@{4. 5. {[}1896{]}}|)be}\mylabel{h}  \normalsize

\doendnotes{C}
\bigskip
\vfill

\clearpage

\footnotesize

\lohead{\textsc{register}}

% Definiere theindex-Environment komplett neu ohne reledmac
\makeatletter
\renewenvironment{theindex}{%
  \section*{\indexname}%
  \setlength{\parindent}{0pt}%
  \setlength{\parskip}{0pt plus 0.3pt}%
  \let\item\@idxitem
}{%
  \clearpage
}
\makeatother

\IfFileExists{\jobname-pw.ind}{\input{\jobname-pw.ind}}{}

\end{document}

      