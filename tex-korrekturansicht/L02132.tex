%% latex-korrekturansicht-vorspann.tex
%% Vorspann für die Korrekturansicht.
%% Lädt die gemeinsame Datei latex-vorspann.tex mit gesetztem Schalter.

\newif\ifkorrekturansicht
\korrekturansichttrue

\input{../tex-inputs/latex-vorspann}


               \section[Arthur Schnitzler an Hermann Bahr, 2{[}5{]}. 4. 1913]{ Arthur Schnitzler an Hermann Bahr, 2{[}5{]}. 4. 1913}\nopagebreak\mylabel{v}\rehead{ }\normalsize\beginnumbering\briefempfaengerindex{Bahr, Hermann@\textsc{Bahr, Hermann}!zzzSchnitzler, Arthur@\emph{von Arthur Schnitzler}!1913-04-252@{2{[}5{]}. 4. 1913}|(be} \toendnotes[C]{\smallbreak\pagebreak[2]} \Standort{TMW, HS AM 60139 Ba.}
\physDesc{Briefkarte
\newline{}Handschrift: schwarze Tinte, deutsche Kurrent\newline{}Ordnung: Lochung }\buchAbdrucke{\weitereDrucke{1) \emph{25. 4. 1913, Abschrift.} In: Arthur Schnitzler: \emph{The Letters of Arthur Schnitzler to Hermann Bahr}. Edited, annotated, and with an introduction, by Donald G.
                        Daviau. Chapel Hill: \emph{The University of North Carolina Press} 1978, S. 112 (University of North Carolina studies in the Germanic languages
                        and literatures, 89).} \weitereDrucke{2) Hermann Bahr, Arthur Schnitzler: \emph{Briefwechsel, Aufzeichnungen, Dokumente (1891–1931)}. Hg. Kurt Ifkovits und Martin Anton Müller. Göttingen: \emph{Wallstein} 2018, S. 485.} }\toendnotes[C]{\smallbreak}\pstart
           \noindent{}{\pb}\textcolor{gray}{\textbf{Dr. Arthur Schnitzler}}\hfill 2\substVorne{}\textsuperscript{\textcolor{gray}{4}}\substDazwischen{}\textcolor{gray}{5}\substHinten{}/4 913.\pend
           \pstart
           \textcolor{gray}{\textbf{\textcolor{pink}{Wien XVIII. Sternwartestrasse 71}{}\ledrightnote{\textcolor{pink}{Sternwartestraße}}}}\pend
           \pstart{}lieber Hermann, \pend\pstart
           für heute nur die Mittheilg, daſs \label{K_L02132_1v}\edtext{\textcolor{blue}{\textsc{P. A.}}{}\ledrightnote{\textcolor{blue}{Peter Altenberg}} Montag mit ſeinem \textcolor{blue}{Bruder}{}\ledrightnote{→\textcolor{blue}{Georg Engländer}}
               auf den \textcolor{pink}{Se{\geminationm}ering}{}\ledrightnote{\textcolor{pink}{Semmering}},
               zuerſt zu \textsc{\textcolor{blue}{Hansy}{}\ledrightnote{\textcolor{blue}{Franz Hansy}}}, hinauffährt}{\lemma{\textnormal{\emph{P. A. … hinauffährt}}}\Cendnote{\textnormal{Schnitzler hatte das
                     \textcolor{pink}{Kurhaus} von Dr. \textcolor{blue}{Franz Hansy} vorgeschlagen (vgl. Arthur Schnitzler an Peter Altenberg, 22. 4. 1913); \textcolor{blue}{Georg
                     Engländer} schrieb \textcolor{blue}{Schnitzler} am 25. 4. 1913, dass das umgesetzt
                  werde.}}}\label{K_L02132_1h}.\pend
           \pstart
           Für deinen Brief herzlichen Dank. Wa{\geminationn}{\pb}wir nach \textcolor{pink}{Salzburg}{}\ledrightnote{\textcolor{pink}{Salzburg}} kommen, weiſs ich noch nicht, aber hoffentlich noch in
               dieſem Jahr. Zu welcher Zeit ſeid Ihr \textcolor{gray}{dort}?\pend
           \pstart
           Auf Wiederſehen, u alles gute von Haus zu Haus.{\\[\baselineskip]}Dein{\\[\baselineskip]}\spacefill\mbox{Arthur}\pend
           \leftskip=0em{}\endnumbering\briefempfaengerindex{Bahr, Hermann@\textsc{Bahr, Hermann}!zzzSchnitzler, Arthur@\emph{von Arthur Schnitzler}!1913-04-252@{2{[}5{]}. 4. 1913}|)be}\mylabel{h}  \normalsize

\doendnotes{C}
\bigskip
\vfill

\clearpage

\footnotesize

\lohead{\textsc{register}}

% Definiere theindex-Environment komplett neu ohne reledmac
\makeatletter
\renewenvironment{theindex}{%
  \section*{\indexname}%
  \setlength{\parindent}{0pt}%
  \setlength{\parskip}{0pt plus 0.3pt}%
  \let\item\@idxitem
}{%
  \clearpage
}
\makeatother

\IfFileExists{\jobname-pw.ind}{\input{\jobname-pw.ind}}{}

\end{document}

      