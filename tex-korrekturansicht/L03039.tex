%% latex-korrekturansicht-vorspann.tex
%% Vorspann für die Korrekturansicht.
%% Lädt die gemeinsame Datei latex-vorspann.tex mit gesetztem Schalter.

\newif\ifkorrekturansicht
\korrekturansichttrue

\input{../tex-inputs/latex-vorspann}


\renewcommand{\erwaehntePersonen}{Personen: Felix Salten}
\renewcommand{\erwaehnteInstitutionen}{Institutionen: Concordia}
\renewcommand{\erwaehnteOrte}{Orte: Burgtheater, Hörlgasse, Kaufmännischer Verein, Wien}
\renewcommand{\erwaehnteWerke}{Werke: Liebelei. Schauspiel in drei Akten}
\section[ Arthur Schnitzler an Felix Salten, {[}9. 6. 1896?{]}]{Arthur Schnitzler an Felix Salten, {[}9. 6. 1896?{]}}
\nopagebreak\mylabel{v}
\rehead{ }\normalsize\beginnumbering\briefempfaengerindex{Salten, Felix@\textsc{Salten, Felix}!zzzSchnitzler, Arthur@\emph{von Arthur Schnitzler}!1896-06-092@{{[}9. 6. 1896?{]}}|(be}
\toendnotes[C]{\smallbreak\pagebreak[2]}\Standort{Wienbibliothek im Rathaus, ZPH 1681, 2.1.516.}
\physDesc{Brief, 1 Blatt, 3 Seiten, 288 Zeichen
\newline{}Handschrift: Bleistift, deutsche Kurrent
\newline{}Ordnung: mit Bleistift von unbekannter Hand Nummerierung der Blätter des Konvoluts:
                                    »22«–»23« }\toendnotes[C]{\smallbreak}
\pstart
           \raggedleft{}{\pb}Dinſtag\pend
           
\pstart
           lieber, wollen Sie heut{ }Abend mit mir in eine verborgne \textcolor{pink}{Loge}{}\ledrightnote{{$\rightarrow$}\textcolor{pink}{Burgtheater}} jener \label{K_L03039-1v}\edtext{\textcolor{green}{Liebelei}{}\ledrightnote{\textcolor{green}{Liebelei. Schauspiel in drei Akten}}-Auffühg}{\lemma{\textnormal{\emph{Liebelei-Auffühg}}}\Cendnote{\textnormal{Zwei Dienstage, an denen \textcolor{blue}{Schnitzler} in \emph{\textcolor{green}{Liebelei}}-Aufführungen
                  war, bieten sich zur Datierung dieses Korrespondenzstücks an. Bei der am 15. 1. 1901 handelte
                  es sich um eine Inszenierung von Schauspielschülerinnen im \textcolor{pink}{Kaufmännischen Verein}, wobei die Existenz einer
                     »geheimen Loge« eher abwegig scheint. In einem Brief, den \textcolor{blue}{Salten} mutmaßlich am selben Tag \textcolor{blue}{Schnitzler} sendete, deutete er an, am
                     Abend möglicherweise verhindert zu sein, womit sein Fernbleiben
                  erklärt ist (vgl. Felix Salten an Arthur Schnitzler, [9. 6. 1896?]).}}}\label{K_L03039-1h}
               gehen \introOben{}(½ 8)\introOben{}, ſo laſſen Sie michs gütigſt am
                  frühen Nachmittg wiſſen. Ich hole \substVorne{}\textsuperscript{\textcolor{gray}{ſ}}\substDazwischen{}S\substHinten{}ie da{\geminationn}, we{\geminationn}s Ihnen
                  {\pb}recht iſt, um \label{K_L03039-2v}\edtext{¼ 8}{\lemma{\textnormal{\emph{¼ 8}}}\Cendnote{\textnormal{19 Uhr 15}}}\label{K_L03039-2h} oder ½ in
               Ihrer \textcolor{pink}{Wohnung}{}\ledrightnote{{$\rightarrow$}\textcolor{pink}{Hörlgasse}} ab?\pend
           
\pstart
           Herzlichſt {\\[\baselineskip]}Ihr {\\[\baselineskip]}\spacefill\mbox{Arth}\pend
           \leftskip=0em{}
\pstart
           \noindent{}{\pb}Und noch eins: ich habe geſtern mit Ihnen im \label{K_L03039-3v}\edtext{\textcolor{brown}{Club}{}\ledrightnote{{$\rightarrow$}\textcolor{brown}{Concordia}}}{\lemma{\textnormal{\emph{Club}}}\Cendnote{\textnormal{Welcher Klub gemeint war, lässt sich
                     nicht mit Sicherheit bestimmen. Da \textcolor{blue}{Schnitzler} seit zumindest 13. 10. 1889 Veranstaltungen des Clubs der \emph{\textcolor{brown}{Concordia}} besuchte, war dieser vermutlich
                     der gemeinte.}}}\label{K_L03039-3h} soupirt.\pend
           \endnumbering\briefempfaengerindex{Salten, Felix@\textsc{Salten, Felix}!zzzSchnitzler, Arthur@\emph{von Arthur Schnitzler}!1896-06-092@{{[}9. 6. 1896?{]}}|)be}\mylabel{h}  \normalsize

\doendnotes{C}
\bigskip
\vfill

\clearpage

\footnotesize

\lohead{\textsc{register}}

% Definiere theindex-Environment komplett neu ohne reledmac
\makeatletter
\renewenvironment{theindex}{%
  \section*{\indexname}%
  \setlength{\parindent}{0pt}%
  \setlength{\parskip}{0pt plus 0.3pt}%
  \let\item\@idxitem
}{%
  \clearpage
}
\makeatother

\IfFileExists{\jobname-pw.ind}{\input{\jobname-pw.ind}}{}

\end{document}

      