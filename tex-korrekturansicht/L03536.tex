%% latex-korrekturansicht-vorspann.tex
%% Vorspann für die Korrekturansicht.
%% Lädt die gemeinsame Datei latex-vorspann.tex mit gesetztem Schalter.

\newif\ifkorrekturansicht
\korrekturansichttrue

\input{../tex-inputs/latex-vorspann}


\renewcommand{\erwaehntePersonen}{Personen: Alfred Gold, Paul Goldmann, Gerhart Hauptmann,  Kohrl, Johanna Mamroth, Fedor Mamroth, Paul Marx, Olga Schnitzler, Elisabeth Steinrück}
\renewcommand{\erwaehnteInstitutionen}{Institutionen: Frankfurter Zeitung}
\renewcommand{\erwaehnteOrte}{Orte: ?? [Wohnung von Olga und Elisabeth Gussmann, 1901/1902], Berlin, Dessauer Straße, Grünentorgasse, Hauptstraße 56, Südtirol, Tirol, Wien}
\renewcommand{\erwaehnteWerke}{Werke: Barcarole, Der rothe Hahn. Tragikomödie in vier Akten, Hoffmanns Erzählungen}
\section[ Paul Goldmann an Olga und Elisabeth Gussmann, 10. 12. {[}1901{]}]{Paul Goldmann an Olga und Elisabeth Gussmann, 10. 12. {[}1901{]}}
\nopagebreak\mylabel{v}
\rehead{ }\normalsize\beginnumbering\briefempfaengerindex{Steinrueck, Elisabeth@\textsc{Steinrück, Elisabeth}!zzzGoldmann, Paul@\emph{von Paul Goldmann}!1901-12-101@{10. 12. {[}1901{]}}|(be}\briefempfaengerindex{Schnitzler, Olga@\textsc{Schnitzler, Olga}!zzzGoldmann, Paul@\emph{von Paul Goldmann}!1901-12-101@{10. 12. {[}1901{]}}|(be}
\toendnotes[C]{\smallbreak\pagebreak[2]}\Standort{DLA, A:Schnitzler, HS.NZ85.1.5247.}
\physDesc{Brief, 1 Blatt, 4 Seiten, 1608 Zeichen
\newline{}Handschrift: blaue Tinte, deutsche Kurrent}\toendnotes[C]{\smallbreak}
\pstart
           \noindent{}\raggedleft{}{\pb}\textcolor{gray}{\textbf{\textcolor{pink}{DESSAUERSTRASSE 19}{}\ledrightnote{\textcolor{pink}{Dessauer Straße}}}}\pend
           
\pstart
           \textcolor{pink}{Berlin}{}\ledrightnote{\textcolor{pink}{Berlin}}, 10. Dezember.\pend
           
\pstart\center{}Liebes Fräulein \textsc{Olga},\pend
\pstart
           Haben Sie vielen Dank für Ihren lieben Brief! Antworten kann ich Ihnen noch nicht. Es
               iſt nicht mit Worten zu beſchreiben, was ich zu thun habe! Ich will Ihnen nur ſagen,
               wie ſehr mich Ihre Zeilen gefreut \strikeout{h\textcolor{gray}{a}} haben, in denen Sie als das liebe \textcolor{pink}{Wien}{}\ledrightnote{\textcolor{pink}{Wien}}er
               Mädel erſcheinen, als das ich Sie kenne. Warum man weinen muß, wenn \textsc{\textcolor{blue}{Hauptmann}{}\ledrightnote{\textcolor{blue}{Gerhart Hauptmann}}} ein \label{K_L03536-1v}\edtext{ſchlechtes \textcolor{green}{Stück}{}\ledrightnote{{$\rightarrow$}\textcolor{green}{Der rothe Hahn. Tragikomödie in vier Akten}}}{\lemma{\textnormal{\emph{ſchlechtes Stück}}}\Cendnote{\textnormal{\emph{\textcolor{green}{Der rothe Hahn}}, siehe Paul Goldmann an Arthur Schnitzler, 29. 11. [1901]}}}\label{K_L03536-1h} ſchreibt, iſt mir {\pb}zwar unklar, aber \label{K_L03536-2v}\edtext{über \textsc{\textcolor{blue}{Hauptmann}{}\ledrightnote{\textcolor{blue}{Gerhart Hauptmann}}} wollen wir nicht mehr miteinander reden}{\lemma{\textnormal{\emph{über … reden}}}\Cendnote{\textnormal{vgl. Paul Goldmann an Olga Gussmann, 15. 11. [1901]}}}\label{K_L03536-2h}. Bezüglich des dritten \textcolor{green}{Akt}{}\ledrightnote{{$\rightarrow$}\textcolor{green}{Hoffmanns Erzählungen}}es von \label{K_L03536-3v}\edtext{\textcolor{green}{\textsc{Hoffmanns} Erzählungen}{}\ledrightnote{\textcolor{green}{Hoffmanns Erzählungen}}}{\lemma{\textnormal{\emph{Hoffmanns Erzählungen}}}\Cendnote{\textnormal{Das deutet darauf hin, dass 
                  \textcolor{blue}{Olga Gussmann} die \textcolor{green}{Oper} am 29. 11. 1901 gemeinsam
                  mit \textcolor{blue}{Schnitzler} besuchte.}}}\label{K_L03536-3h} bin ich ganz
               Ihrer Anſicht. Ich habe ihn immer für das ſchönſte gehalten, wenn auch die \textsc{\textcolor{green}{Barcarole}{}\ledrightnote{\textcolor{green}{Barcarole}}} mein Lieblingsſtück bleibt. Nur \textsc{\textcolor{blue}{Arthur}{}\ledrightnote{}} hat, wie Sie ſich erinnern werden, die ganze \textcolor{green}{Oper}{}\ledrightnote{{$\rightarrow$}\textcolor{green}{Hoffmanns Erzählungen}} als talentloſes Machwerk \label{K_L03536-4v}\edtext{bezeichnet}{\lemma{\textnormal{\emph{bezeichnet}}}\Cendnote{\textnormal{Am 28. 11. 1900 hatten 
                  \textcolor{blue}{Schnitzler} und \textcolor{blue}{Goldmann} die \textcolor{green}{Oper} gemeinsam besucht und danach
                  noch gemeinsam gegessen. Vermutlich war auch \textcolor{blue}{Olga Gussmann} dabei.}}}\label{K_L03536-4h}
               und hat dadurch wieder bewieſen, daß er vom Theater nichts verſteht.\pend
           
\pstart
           \textsc{\textcolor{blue}{Alfred Gold}{}\ledrightnote{\textcolor{blue}{Alfred Gold}}}, der verworrene {\pb}und alberne \textcolor{blue}{Literatur-Lausbub}{}\ledrightnote{{$\rightarrow$}\textcolor{blue}{Alfred Gold}}, ein \textsc{\begin{otherlanguage}{french}Protégé\end{otherlanguage}} der \textcolor{blue}{Frau}{}\ledrightnote{{$\rightarrow$}\textcolor{blue}{Johanna Mamroth}} meines \textcolor{blue}{Onkel}{}\ledrightnote{{$\rightarrow$}\textcolor{blue}{Fedor Mamroth}}s, iſt von meinem \textcolor{blue}{Onkel}{}\ledrightnote{{$\rightarrow$}\textcolor{blue}{Fedor Mamroth}} als \textcolor{pink}{Berlin}{}\ledrightnote{\textcolor{pink}{Berlin}}er Feuilleton-Correſpondent der \textcolor{brown}{Frankfurter Zeitung}{}\ledrightnote{\textcolor{brown}{Frankfurter Zeitung}} engagirt worden!!!\pend
           
\pstart
           Laſſen Sie es ſich gut gehen in Ihrer \label{K_L03536-5v}\edtext{neuen \textcolor{pink}{Penſion}{}\ledrightnote{{$\rightarrow$}\textcolor{pink}{?? [Wohnung von Olga und Elisabeth Gussmann, 1901/1902]}}}{\lemma{\textnormal{\emph{neuen Penſion}}}\Cendnote{\textnormal{Im 
               Frühling 1901 waren \textcolor{blue}{Olga} und \textcolor{blue}{Elisabeth Gussmann}
               in die \textcolor{pink}{Grünentorgasse} gezogen. Vermutlich seit November 1901 war  \textcolor{blue}{Olga}
               schwanger (vgl. A. S.: \emph{Tagebuch}, 10. 11. 1901). Die vorliegende Stelle deutet auf eine neue Unterkunft, die sie bis zur Übersiedelung in die \textcolor{pink}{Hauptstraße 56 in Hinterbrühl} am 21. 3. 1902
               bewohnte.}}}\label{K_L03536-5h} mit den \label{K_L03536-6v}\edtext{\textsc{\begin{otherlanguage}{english}new style\end{otherlanguage}}-Möbeln}{\lemma{\textnormal{\emph{new style-Möbeln}}}\Cendnote{\textnormal{›\begin{otherlanguage}{english}New Style\end{otherlanguage}‹ ist synonym mit
                  \begin{otherlanguage}{french}l’art nouveau\end{otherlanguage}/Jugendstil.}}}\label{K_L03536-6h} und ſeien Sie (bis ich Ihnen ausführlich ſchreibe) einſtweilen
                  herzlich\uuline{ſt} (nicht herzl\uuline{ich}, wie Sie ſchreiben) gegrüßt von Ihrem getreuen{\\[\baselineskip]}\spacefill\mbox{Paul Goldmann.}\pend
           \leftskip=0em{}{\bigskip}
\pstart
           \noindent{}{\pb}Liebes Fräulein \textsc{Liesl}, der unglaublich
               blöde Brief, den Sie mir geſchrieben haben, hat mich ſehr gefreut. Seien Sie brav und
               lernen Sie was! Zur Belohnung dürfen Sie dann auch \strikeout{wieder}{ }\label{K_L03536-7v}\edtext{nach \textcolor{pink}{Berlin}{}\ledrightnote{\textcolor{pink}{Berlin}} kommen}{\lemma{\textnormal{\emph{nach Berlin kommen}}}\Cendnote{\textnormal{\textcolor{blue}{Elisabeth Gussmann} war jedenfalls Ende Januar 1902 in \textcolor{pink}{Berlin}, vgl. die Korrespondenz zwischen \textcolor{blue}{Goldmann} und \textcolor{blue}{Elisabeth
                        Gussmann}: \emph{DLA}, HS.1985.1.5246.}}}\label{K_L03536-7h} und
               wieder einmal in meinem Umgang ſich fortbilden. \label{K_L03536-8v}\edtext{\textsc{\textcolor{blue}{Kohrl}{}\ledrightnote{\textcolor{blue}{Kohrl}}}}{\lemma{\textnormal{\emph{Kohrl}}}\Cendnote{\textnormal{nicht ermittelt}}}\label{K_L03536-8h} verlebt in \textcolor{pink}{Tirol}{}\ledrightnote{\textcolor{pink}{Tirol}{\newline}\textcolor{pink}{Südtirol}} gewiß glückliche Tage, \label{K_L03536-9v}\edtext{ſeit er Sie los iſt}{\lemma{\textnormal{\emph{ſeit er Sie los iſt}}}\Cendnote{\textnormal{Bezug unklar}}}\label{K_L03536-9h}. Grüßen Sie Herrn \textsc{\textcolor{blue}{Paul}{}\ledrightnote{\textcolor{blue}{Paul Marx}}} und ſeien Sie ſelbſt herzlichſt gegrüßt von Ihrem getreuen {\\}\spacefill\mbox{Paul Goldmann}\pend
           \endnumbering\briefempfaengerindex{Steinrueck, Elisabeth@\textsc{Steinrück, Elisabeth}!zzzGoldmann, Paul@\emph{von Paul Goldmann}!1901-12-101@{10. 12. {[}1901{]}}|)be}\briefempfaengerindex{Schnitzler, Olga@\textsc{Schnitzler, Olga}!zzzGoldmann, Paul@\emph{von Paul Goldmann}!1901-12-101@{10. 12. {[}1901{]}}|)be}\mylabel{h}  \normalsize

\doendnotes{C}
\bigskip
\vfill

\clearpage

\footnotesize

\lohead{\textsc{register}}

% Definiere theindex-Environment komplett neu ohne reledmac
\makeatletter
\renewenvironment{theindex}{%
  \section*{\indexname}%
  \setlength{\parindent}{0pt}%
  \setlength{\parskip}{0pt plus 0.3pt}%
  \let\item\@idxitem
}{%
  \clearpage
}
\makeatother

\IfFileExists{\jobname-pw.ind}{\input{\jobname-pw.ind}}{}

\end{document}

      