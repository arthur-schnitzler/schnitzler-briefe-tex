%% latex-korrekturansicht-vorspann.tex
%% Vorspann für die Korrekturansicht.
%% Lädt die gemeinsame Datei latex-vorspann.tex mit gesetztem Schalter.

\newif\ifkorrekturansicht
\korrekturansichttrue

\input{../tex-inputs/latex-vorspann}


\renewcommand{\erwaehntePersonen}{Personen: Julius Bauer, Paul Horn}
\renewcommand{\erwaehnteOrte}{Orte: Café Pfob, Café Union, Riedhof, Volkstheater, Wien}
\renewcommand{\erwaehnteWerke}{Werke: ?? [Feuilleton], Musotte, Tagebuch}
\section[Felix Salten an Arthur Schnitzler, {[}12. 11. 1892{]}]{Felix Salten an Arthur Schnitzler, {[}12. 11. 1892{]}}
\nopagebreak\mylabel{v}
\rehead{ }\normalsize\beginnumbering\briefempfaengerindex{Schnitzler, Arthur@\textsc{Schnitzler, Arthur}!zzzSalten, Felix@\emph{von Felix Salten}!1892-11-121@{{[}12. 11. 1892{]}}|(be}
\toendnotes[C]{\smallbreak\pagebreak[2]}\Standort{CUL, Schnitzler, B 89, A 1.}
\physDesc{Brief, 1 Blatt, 2 Seiten, 779 Zeichen
\newline{}Handschrift: Bleistift, lateinische Kurrent
\newline{}Schnitzler: mit Bleistift datiert: »12/XI 92« 
\newline{}Ordnung: mit Bleistift von unbekannter Hand nummeriert: »20« }\toendnotes[C]{\smallbreak}
\pstart
           \noindent{}{\pb}Verehrtester Freund! Dass es mir sehr, sehr unangenehm
               ist, mich an Sie zu wenden, nach allem, was Sie \label{K_L03117-1v}\edtext{bereits für mich gethan}{\lemma{\textnormal{\emph{bereits für mich gethan}}}\Cendnote{\textnormal{siehe Felix Salten an Arthur Schnitzler, 10. 8. 1892}}}\label{K_L03117-1h}, können Sie sich denken, doppelt, da ich weiss, dass Sie ja selbst nicht viel
               übrig haben. Allein Sie können sich auch hoffentlich denken, wie elend es mir geht, \substVorne{}\textsuperscript{\textcolor{gray}{d}}\substDazwischen{}w\substHinten{}enn ich es trotz alledem thun muss, muss, weil ich mir keinen anderen Ausweg
               weiss\substVorne{}\textsuperscript{, wenn}{\allowbreak}\substDazwischen{}. Wenn\substHinten{} es halbwegs in Ihrer Macht steht so
               bitte ich Sie \uline{sehr}\textcolor{gray}{,} mir freundlichst 5 f zu leihen, welche ich {\pb}Ihnen, – da \label{K_L03117-2v}\edtext{\textcolor{blue}{Bauer}{}\ledrightnote{\textcolor{blue}{Julius Bauer}} mein \textcolor{green}{Feuilleton}{}\ledrightnote{{$\rightarrow$}\textcolor{green}{?? [Feuilleton]}}}{\lemma{\textnormal{\emph{Bauer mein Feuilleton}}}\Cendnote{\textnormal{Nicht nachgewiesen. Es ist unklar, ob
                     \textcolor{blue}{Salten}s \textcolor{green}{Text} ohne Namensnennung, überhaupt nicht oder zu einem
                  viel späteren Zeitpunkt erschien.}}}\label{K_L03117-2h} diese\textcolor{gray}{r} Tage zu bringen
               versprach – wohl Ende der nächsten Woche \introOben{}gewiss\introOben{} retour geben
               kann.\pend
           
\pstart
           Kommen Sie heute{ }Abend – wenn auch spät – zu \textcolor{pink}{Pfob}{}\ledrightnote{\textcolor{pink}{Café Pfob}}? Ich
               gehe nicht zu \label{K_L03117-3v}\edtext{\textcolor{green}{Musotte}{}\ledrightnote{\textcolor{green}{Musotte}}}{\lemma{\textnormal{\emph{Musotte}}}\Cendnote{\textnormal{\textcolor{blue}{Schnitzler}s Besuch der Aufführung von \emph{\textcolor{green}{Musotte}} lässt sich nur indirekt, durch die
                  Erwähnung des \textcolor{pink}{Volkstheater}s, im \emph{\textcolor{green}{Tagebuch}}-Eintrag zum 12. 11. 1892 ableiten. Ein Besuch in einem der
                  genannten Lokale ist nicht belegt.}}}\label{K_L03117-3h}! Oder, da Sie mit \textcolor{blue}{Paul}{}\ledrightnote{\textcolor{blue}{Paul Horn}} soupiren u. wie ich höre \textcolor{pink}{Riedhof}{}\ledrightnote{\textcolor{pink}{Riedhof}}, \textcolor{pink}{Union}{}\ledrightnote{\textcolor{pink}{Café Union}}? Besser wäre \textcolor{pink}{Pfob}{}\ledrightnote{\textcolor{pink}{Café Pfob}} weil alles heute da sein wird.\pend
           
\pstart
           Ihr {\\[\baselineskip]}\spacefill\mbox{Salten}\pend
           \leftskip=0em{}\endnumbering\briefempfaengerindex{Schnitzler, Arthur@\textsc{Schnitzler, Arthur}!zzzSalten, Felix@\emph{von Felix Salten}!1892-11-121@{{[}12. 11. 1892{]}}|)be}\mylabel{h}  \normalsize

\doendnotes{C}
\bigskip
\vfill

\clearpage

\footnotesize

\lohead{\textsc{register}}

% Definiere theindex-Environment komplett neu ohne reledmac
\makeatletter
\renewenvironment{theindex}{%
  \section*{\indexname}%
  \setlength{\parindent}{0pt}%
  \setlength{\parskip}{0pt plus 0.3pt}%
  \let\item\@idxitem
}{%
  \clearpage
}
\makeatother

\IfFileExists{\jobname-pw.ind}{\input{\jobname-pw.ind}}{}

\end{document}

      