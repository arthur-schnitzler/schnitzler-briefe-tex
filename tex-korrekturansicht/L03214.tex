%% latex-korrekturansicht-vorspann.tex
%% Vorspann für die Korrekturansicht.
%% Lädt die gemeinsame Datei latex-vorspann.tex mit gesetztem Schalter.

\newif\ifkorrekturansicht
\korrekturansichttrue

\input{../tex-inputs/latex-vorspann}


\renewcommand{\erwaehntePersonen}{Personen: Otto Brahm, Raphael Löwenfeld, Walther Rathenau, Olga Schnitzler, Elisabeth Steinrück}
\renewcommand{\erwaehnteInstitutionen}{Institutionen: S. Hirzel Verlag (Leipzig), Schiller-Theater}
\renewcommand{\erwaehnteOrte}{Orte: Berlin, Dessauer Straße, Kaltenleutgeben, Leipzig, Schweiz, Südtirol, Tirol, Wien}
\renewcommand{\erwaehnteWerke}{Werke: Der Schleier der Beatrice. Schauspiel in fünf Akten, Impressionen}
\section[ Paul Goldmann an Arthur Schnitzler, 25. 7. {[}1902{]}]{Paul Goldmann an Arthur Schnitzler, 25. 7. {[}1902{]}}
\nopagebreak\mylabel{v}
\rehead{ }\normalsize\beginnumbering\briefempfaengerindex{Schnitzler, Arthur@\textsc{Schnitzler, Arthur}!zzzGoldmann, Paul@\emph{von Paul Goldmann}!1902-07-251@{25. 7. {[}1902{]}}|(be}
\toendnotes[C]{\smallbreak\pagebreak[2]}\Standort{DLA, A:Schnitzler, HS.NZ85.1.3172.}
\physDesc{Brief, 1 Blatt, 4 Seiten
\newline{}Handschrift: blaue Tinte, deutsche Kurrent
\newline{}Schnitzler: mit Bleistift das Jahr »{[}1{]}902« vermerkt }\toendnotes[C]{\smallbreak}
\pstart
           \noindent{}\raggedleft{}{\pb}\textcolor{pink}{\textcolor{gray}{\textbf{DESSAUERSTRASSE 19}}}{}\ledrightnote{\textcolor{pink}{Dessauer Straße}}\pend
           
\pstart
           \textcolor{pink}{Berlin}{}\ledrightnote{\textcolor{pink}{Berlin}}, 25. Juli.\pend
           
\pstart\center{}Mein lieber Freund,\pend
\pstart
           Nach langem Schwanken habe ich mich entſchloſſen, in die \textcolor{pink}{Schweiz}{}\ledrightnote{\textcolor{pink}{Schweiz}} zu gehen. Ich komme alſo nicht über \textcolor{pink}{Wien}{}\ledrightnote{\textcolor{pink}{Wien}}. Der \textcolor{pink}{Wien}{}\ledrightnote{\textcolor{pink}{Wien}}er Aufenthalt
               hat mir zu \label{K_L03214-1v}\edtext{Pfingſten}{\lemma{\textnormal{\emph{Pfingſten}}}\Cendnote{\textnormal{siehe Paul Goldmann an Arthur Schnitzler, 5. 5. [1902]}}}\label{K_L03214-1h} gar nicht gut gethan; ich \strikeout{kam} bin ſehr
               angegriffen zurückgekehrt. Nach \textcolor{pink}{Tirol}{}\ledrightnote{\textcolor{pink}{Tirol}{\newline}\textcolor{pink}{Südtirol}} gehe ich
               nicht, weil ich fürchte, dort zu viel Bekannte zu treffen und in ein ermüdendes
               geſellſchaftliches {\pb}\substVorne{}\textsuperscript{Treiben}{\allowbreak}\substDazwischen{}Treiben\substHinten{} hineinzugerathen. Ich will einmal ein paar Wochen lang ganz der Ruhe leben
               und es ſogar mit der Einſamkeit verſuchen. Vielleicht thut dieſe meinen gequälten
               Nerven gut.\pend
           
\pstart
           Es thut mir unendlich leid, daß ich durch dieſe Änderung meiner Reiſepläne auch der
               Freude verluſtig gehe, Dich wiederzuſehen. Ich rechne aber ſehr darauf, daß die
                  \label{K_L03214-2v}\edtext{»\textsc{\textcolor{green}{Beatrice}{}\ledrightnote{\textcolor{green}{Der Schleier der Beatrice. Schauspiel in fünf Akten}}}«-Angelegenheit}{\lemma{\textnormal{\emph{»Beatrice«-Angelegenheit}}}\Cendnote{\textnormal{siehe Paul Goldmann an Arthur Schnitzler, 14. 7. [1902]}}}\label{K_L03214-2h}{ }{\pb}Dich ſchon im Anfang des Winters
               nach \textcolor{pink}{Berlin}{}\ledrightnote{\textcolor{pink}{Berlin}} führen wird. Hat \label{K_L03214-3v}\edtext{\textsc{\textcolor{blue}{Brahm}{}\ledrightnote{\textcolor{blue}{Otto Brahm}}}}{\lemma{\textnormal{\emph{Brahm}}}\Cendnote{\textnormal{Vgl. \emph{Der Briefwechsel Arthur Schnitzler — Otto
                        Brahm}. Vollständige Ausgabe. Herausgegeben, eingeleitet und
                     erläutert von Oskar Seidlin. Tübingen:
                        \emph{Niemeyer}{ }1975, S. 126–127.}}}\label{K_L03214-3h} geantwortet? Und in
               welchem Sinne? \textsc{Dr. \textcolor{blue}{Löwenfeld}{}\ledrightnote{\textcolor{blue}{Raphael Löwenfeld}}}, vom »\textcolor{brown}{Schillertheater}{}\ledrightnote{\textcolor{brown}{Schiller-Theater}}«, iſt in \textcolor{pink}{Kaltenleutgeben}{}\ledrightnote{\textcolor{pink}{Kaltenleutgeben}}; und wenn Du mit \textsc{\textcolor{blue}{Brahm}{}\ledrightnote{\textcolor{blue}{Otto Brahm}}} nicht einig wirſt (was ich aber hoffe) kannſt Du gleich mit ihm verhandeln.\pend
           
\pstart
           Ich bleibe noch etwa acht Tage \textcolor{pink}{hier}{}\ledrightnote{{$\rightarrow$}\textcolor{pink}{Berlin}} und hoffe, von Dir bald zu hören. {\pb}Grüß\textcolor{gray}{e} mir \textsc{\textcolor{blue}{Olga}{}\ledrightnote{\textcolor{blue}{Olga Schnitzler}}} und \textsc{\textcolor{blue}{Liesl}{}\ledrightnote{\textcolor{blue}{Elisabeth Steinrück}}} und ſei Du ſelbſt vielmals und von Herzen gegrüßt von {\\[\baselineskip]}Deinem getreuen {\\[\baselineskip]}\spacefill\mbox{Paul Goldm}\pend
           \leftskip=0em{}
\pstart
           \noindent{}Lies das Buch \label{K_L03214-7v}\edtext{»\textcolor{green}{Impreſſionen}{}\ledrightnote{\textcolor{green}{Impressionen}}« von \textsc{\textcolor{blue}{Walther Rathenau}{}\ledrightnote{\textcolor{blue}{Walther Rathenau}}}}{\lemma{\textnormal{\emph{»Impreſſionen« … Rathenau}}}\Cendnote{\textnormal{\textcolor{blue}{Walter Rathenau}: \emph{\textcolor{green}{Impressionen}}. \textcolor{pink}{Leipzig}: \emph{\textcolor{brown}{S. Hirzel}}{ }1902. Eine Lektüre durch \textcolor{blue}{Schnitzler}
                     ist nicht bekannt.}}}\label{K_L03214-7h}.\pend
           \endnumbering\briefempfaengerindex{Schnitzler, Arthur@\textsc{Schnitzler, Arthur}!zzzGoldmann, Paul@\emph{von Paul Goldmann}!1902-07-251@{25. 7. {[}1902{]}}|)be}\mylabel{h}  \normalsize

\doendnotes{C}
\bigskip
\vfill

\clearpage

\footnotesize

\lohead{\textsc{register}}

% Definiere theindex-Environment komplett neu ohne reledmac
\makeatletter
\renewenvironment{theindex}{%
  \section*{\indexname}%
  \setlength{\parindent}{0pt}%
  \setlength{\parskip}{0pt plus 0.3pt}%
  \let\item\@idxitem
}{%
  \clearpage
}
\makeatother

\IfFileExists{\jobname-pw.ind}{\input{\jobname-pw.ind}}{}

\end{document}

      