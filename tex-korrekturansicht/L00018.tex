%% latex-korrekturansicht-vorspann.tex
%% Vorspann für die Korrekturansicht.
%% Lädt die gemeinsame Datei latex-vorspann.tex mit gesetztem Schalter.

\newif\ifkorrekturansicht
\korrekturansichttrue

\input{../tex-inputs/latex-vorspann}


               \section[Arthur Schnitzler an Richard Beer-Hofmann, 6. 6. 1891]{ Arthur Schnitzler an Richard Beer-Hofmann, 6. 6. 1891}\nopagebreak\mylabel{v}\rehead{ }\normalsize\beginnumbering\briefempfaengerindex{Beer-Hofmann, Richard@\textsc{Beer-Hofmann, Richard}!zzzSchnitzler, Arthur@\emph{von Arthur Schnitzler}!1891-06-061@{6. 6. 1891}|(be} \toendnotes[C]{\smallbreak\pagebreak[2]} \Standort{YCGL, MSS 31.}
\physDesc{Brief, 1 Blatt, 4 Seiten, Umschlag
\newline{}Handschrift: schwarze Tinte, deutsche Kurrent\newline{}Versand: Stempel: »\nobreak{}Wien, 6 6 91, 4–5 N\nobreak{}«.  }\buchAbdrucke{\weitereDrucke{1) Arthur Schnitzler: \emph{Briefe 1875–1912}. Hg. Therese Nickl und Heinrich Schnitzler. Frankfurt am Main: \emph{S. Fischer} 1981, S. 117.} \weitereDrucke{2) Arthur Schnitzler, Richard Beer-Hofmann: \emph{Briefwechsel 1891–1931}. Hg. Konstanze Fliedl. Wien, Zürich: \emph{Europaverlag} 1992, S. 30–31.} }\toendnotes[C]{\smallbreak}\pstart{}{\pb}\textcolor{gray}{\textbf{\textit{\label{T_L00018-1v}\edtext{AS}{\lemma{\textnormal{\emph{AS}}}\Cendnote{\textnormal{rotes Wachssiegel}}}\label{T_L00018-1h}}}}\pend{}{\bigskip}\pstart{}{\pb}\textsc{Herrn Dr. Rich. Beer-Hofmann}\pend{}\pstart{}\textcolor{pink}{\textsc{Brünn}}{}\ledrightnote{\textcolor{pink}{Brünn}}\pend{}\pstart{}\textcolor{pink}{\textsc{Hotel Neuhauser}}{}\ledrightnote{\textcolor{pink}{Hotel Neuhauser}}\pend{}\pstart{}\textcolor{pink}{\textsc{Mähren}}{}\ledrightnote{\textcolor{pink}{Mähren}}\pend{}{\bigskip}\pstart
           \raggedleft{}{\pb}\textcolor{pink}{Wien}{}\ledrightnote{\textcolor{pink}{Wien}}{ }6. 6. 91.\pend
           \pstart
           Lieber Richard, ich grüße Sie vielmals und danke Ihnen für Ihre
               liebenswürdigen Zeilen. Nächſtens werden Sie etwas ſchreiben müſſen; das ſteht feſt.
               Ich habe die Idee angeregt, zuſa{\geminationm}en ein Buch zu ediren
               (was \label{K_L00018_1v}\edtext{nicht von \textcolor{blue}{Edi = Kafka}{}\ledrightnote{\textcolor{blue}{Eduard Michael Kafka}}}{\lemma{\textnormal{\emph{nicht von Edi = Kafka}}}\Cendnote{\textnormal{\textcolor{blue}{Kafka} forderte \textcolor{blue}{Schnitzler} erst Ende August 1891 auf, an einem
                  »Oesterreichischen Jahrbuch für moderne Literatur« mitzuarbeiten; vgl. Eduard Michael Kafka an Arthur Schnitzler, 30. 8. 1891.}}}\label{K_L00018_1h} ko{\geminationm}t) Titel: \label{K_L00018_2v}\edtext{\textcolor{green}{Aus der Kaffehausecke}{}\ledrightnote{\textcolor{green}{Aus der Kaffeehausecke}}}{\lemma{\textnormal{\emph{Aus der Kaffehausecke}}}\Cendnote{\textnormal{Diesen Titel trug die von \textcolor{blue}{Bölsche} vor Jahresfrist abgelehnte \textcolor{green}{Novelle}, die bislang
                  unveröffentlicht geblieben war; vgl. Wilhelm Bölsche an Arthur Schnitzler, 17. 9. 1890.}}}\label{K_L00018_2h}. Sa{\geminationm}lung von Skizzen, Noveletten, Impreſſionen, Aphorismen
                  – {\pb}jeder hat möglichſt individuell zu ſein – außerdem
               würde ich einen erhöhten \textcolor{pink}{Wien}{}\ledrightnote{\textcolor{pink}{Wien}}er Ton (jenen Ton, der
               nicht im Dialekt beſteht) bevorzugen\strikeout{)}.\pend
           \pstart
           Ich ſpreche noch näher mit Ihnen drüber; Sie haben meiner Idee nach ſehr viel damit
               zu ſchaffen. Intereſſant iſt, wie einige, als Ihr Name gena{\geminationn}t wurde, mit einer gewiſſen Wehmut ſagten: »Ja, we{\geminationn}{ }{\pb}man von dem was kriegen könnte« –\pend
           \pstart
           – In Ihnen muſs ja ſchließlich die Poeſie \uline{herangeglaubt} werden. Ich mache Sie auf dieſes Wort ganz beſonders
               aufmerkſam. – Die Zwiſchengeſpräche und Zwiſchengeſchichten der Kaffehausecke,
               bedürfen beſondrer Ueberlegung – ich freue mich ſehr, mit Ihnen drüber plaudern zu
               können. Darüber u über andres, {\pb}bitte recht ſehr,
               deſertiren Sie ehebaldigſt. Wie lang wird man Sie denn da{\geminationn} in \textcolor{pink}{Wien}{}\ledrightnote{\textcolor{pink}{Wien}} genießen können? Man ſehnt ſich nach
               Ihnen, und die meiſten grüßen Sie herzlichſt. Haben Sie wirklich gar ſo viel zu
               thun?\pend
           \pstart
           Schreiben Sie mir, ſobald Sie wieder hier ſind, d. h. lieber früher, we{\geminationn}{ }Sie Laune haben u ſobald Sie da, ko{\geminationm}en Sie zu\pend
           \pstart Ihrem \spacefill\mbox{Arthur S}\pend{}\endnumbering\briefempfaengerindex{Beer-Hofmann, Richard@\textsc{Beer-Hofmann, Richard}!zzzSchnitzler, Arthur@\emph{von Arthur Schnitzler}!1891-06-061@{6. 6. 1891}|)be}\mylabel{h}  \normalsize

\doendnotes{C}
\bigskip
\vfill

\clearpage

\footnotesize

\lohead{\textsc{register}}

% Definiere theindex-Environment komplett neu ohne reledmac
\makeatletter
\renewenvironment{theindex}{%
  \section*{\indexname}%
  \setlength{\parindent}{0pt}%
  \setlength{\parskip}{0pt plus 0.3pt}%
  \let\item\@idxitem
}{%
  \clearpage
}
\makeatother

\IfFileExists{\jobname-pw.ind}{\input{\jobname-pw.ind}}{}

\end{document}

      