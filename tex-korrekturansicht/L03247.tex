%% latex-korrekturansicht-vorspann.tex
%% Vorspann für die Korrekturansicht.
%% Lädt die gemeinsame Datei latex-vorspann.tex mit gesetztem Schalter.

\newif\ifkorrekturansicht
\korrekturansichttrue

\input{../tex-inputs/latex-vorspann}


\renewcommand{\erwaehntePersonen}{Personen: Olga Schnitzler}
\renewcommand{\erwaehnteOrte}{Orte: Berlin, Dänemark, Helsingør, Kopenhagen, Marienlyst, Wien}
\renewcommand{\erwaehnteWerke}{}
\section[ Paul Goldmann an Arthur Schnitzler, 10. 7. 1906]{Paul Goldmann an Arthur Schnitzler, 10. 7. 1906}
\nopagebreak\mylabel{v}
\rehead{ }\normalsize\beginnumbering\briefempfaengerindex{Schnitzler, Arthur@\textsc{Schnitzler, Arthur}!zzzGoldmann, Paul@\emph{von Paul Goldmann}!1906-07-101@{10. 7. 1906}|(be}
\toendnotes[C]{\smallbreak\pagebreak[2]}\Standort{DLA, A:Schnitzler, HS.NZ85.1.3175.}
\physDesc{Postkarte
\newline{}Handschrift: 1) blaue Tinte, deutsche Kurrent\hspace{1em}2) blaue Tinte, lateinische Kurrent (\noindent{}Adresse)\hspace{1em}
\newline{}Versand: 1) Stempel: »\nobreak{}\oindex{Berlin@\textbf{Berlin}, \emph{https://www.geonames.org/ontologyP.PPLC}|pwk}Berlin, S.W. 11\textsuperscript{e}, 10. 7. 06, 12–1 N.\nobreak{}«.   2) Stempel: »\nobreak{}\oindex{Helsingør@\textbf{Helsingør}, \emph{Besiedelter Ort (A.BSO)}|pwk}Helsingør, 11. 7. 06, 9–10 F\nobreak{}«. 
\newline{}Schnitzler: mit Bleistift das Jahr »{[}1{]}906« vermerkt }\toendnotes[C]{\smallbreak}\pstart{}{\pb}Welt-\textcolor{gray}{\textbf{Poſtkarte}}\pend{}\pstart{}Herrn\pend{}\pstart{}Dr. Arthur Schnitzler (aus \textcolor{pink}{Wien}{}\ledrightnote{\textcolor{pink}{Wien}})\pend{}\pstart{}\textcolor{pink}{Marienlyst}{}\ledrightnote{\textcolor{pink}{Marienlyst}}\pend{}\pstart{}\textcolor{pink}{Dänemark}{}\ledrightnote{\textcolor{pink}{Dänemark}}\pend{}
{\bigskip}
\pstart
           \noindent{}{\pb}\textcolor{pink}{Berlin}{}\ledrightnote{\textcolor{pink}{Berlin}}, 10. Juli.
               Herzlichen Dank, mein lieber Freund, für Deine Karte!
               Ich freue mich, daß es \textcolor{blue}{Euch}{}\ledrightnote{{$\rightarrow$}\textcolor{blue}{Olga Schnitzler}}
               gut geht u. daß es Euch in \label{K-L03247-1v}\edtext{\textcolor{pink}{Dänemark}{}\ledrightnote{\textcolor{pink}{Dänemark}}}{\lemma{\textnormal{\emph{Dänemark}}}\Cendnote{\textnormal{\textcolor{blue}{Schnitzler} war von 28. 6. 1906 bis 11. 8. 1906 in \textcolor{pink}{Dänemark}, hauptsächlich in \textcolor{pink}{Marienlyst} und ein paar Tage in \textcolor{pink}{Kopenhagen}.}}}\label{K-L03247-1h} wieder ſo gut gefällt. Gern hätte ich
               Dich auf der \label{K-L03247-2v}\edtext{Durchreiſe}{\lemma{\textnormal{\emph{Durchreiſe}}}\Cendnote{\textnormal{\textcolor{blue}{Schnitzler} hatte sich zwischen 26. 6. 1906 und 27. 6. 1906 in \textcolor{pink}{Berlin} aufgehalten.}}}\label{K-L03247-2h} in \textcolor{pink}{Berlin}{}\ledrightnote{\textcolor{pink}{Berlin}} geſehen; aber ich begreife, daß die Zeit dazu zu kurz
               war, u. hoffe auf ein baldiges \label{K-L03247-3v}\edtext{Wiederſehen}{\lemma{\textnormal{\emph{Wiederſehen}}}\Cendnote{\textnormal{\textcolor{blue}{Schnitzler} und \textcolor{blue}{Goldmann} trafen sich erst am 24. 5. 1907 in \textcolor{pink}{Wien} wieder.}}}\label{K-L03247-3h} bei günſtigerer Gelegenheit.
               Ich wünſche Dir u. Deiner \textcolor{blue}{Frau}{}\ledrightnote{{$\rightarrow$}\textcolor{blue}{Olga Schnitzler}} auch weiterhin einen frohen u. behaglichen Verlauf der Sommerreiſe u.
               bin mit vielen herzlichen Grüßen Dein {\\}\spacefill\mbox{Paul Goldmnn}\pend
           \endnumbering\briefempfaengerindex{Schnitzler, Arthur@\textsc{Schnitzler, Arthur}!zzzGoldmann, Paul@\emph{von Paul Goldmann}!1906-07-101@{10. 7. 1906}|)be}\mylabel{h}  \normalsize

\doendnotes{C}
\bigskip
\vfill

\clearpage

\footnotesize

\lohead{\textsc{register}}

% Definiere theindex-Environment komplett neu ohne reledmac
\makeatletter
\renewenvironment{theindex}{%
  \section*{\indexname}%
  \setlength{\parindent}{0pt}%
  \setlength{\parskip}{0pt plus 0.3pt}%
  \let\item\@idxitem
}{%
  \clearpage
}
\makeatother

\IfFileExists{\jobname-pw.ind}{\input{\jobname-pw.ind}}{}

\end{document}

      