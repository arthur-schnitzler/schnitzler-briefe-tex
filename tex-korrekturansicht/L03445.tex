%% latex-korrekturansicht-vorspann.tex
%% Vorspann für die Korrekturansicht.
%% Lädt die gemeinsame Datei latex-vorspann.tex mit gesetztem Schalter.

\newif\ifkorrekturansicht
\korrekturansichttrue

\input{../tex-inputs/latex-vorspann}


\renewcommand{\erwaehntePersonen}{Personen:  Eduard VII., Hugo von Hofmannsthal, Detlev von Liliencron, Max Reinhardt, Theodore Rottenberg, Ludwig Rottenberg, Olga Schnitzler, Heinrich Schnitzler,  Wilhelm II. von Preußen}
\renewcommand{\erwaehnteInstitutionen}{Institutionen: Deutsches Theater Berlin}
\renewcommand{\erwaehnteOrte}{Orte: Berlin, Burgtheater, Dessauer Straße, Deutschland, Edmund-Weiß-Gasse, England, Frankfurt am Main, Kiel, Marienbad, Pompei, Taormina, Wien}
\renewcommand{\erwaehnteWerke}{Werke: Zwischenspiel. Komödie in drei Akten}
\section[ Paul Goldmann an Arthur Schnitzler, 23. 6. {[}1904{]}]{Paul Goldmann an Arthur Schnitzler, 23. 6. {[}1904{]}}
\nopagebreak\mylabel{v}
\rehead{ }\normalsize\beginnumbering\briefempfaengerindex{Schnitzler, Arthur@\textsc{Schnitzler, Arthur}!zzzGoldmann, Paul@\emph{von Paul Goldmann}!1904-06-232@{23. 6. {[}1904{]}}|(be}
\toendnotes[C]{\smallbreak\pagebreak[2]}\Standort{DLA, A:Schnitzler, HS.NZ85.1.3174.}
\physDesc{Brief, 1 Blatt, 4 Seiten
\newline{}Handschrift: blaue Tinte, deutsche Kurrent
\newline{}Schnitzler: 1) mit Bleistift das Jahr »{[}1{]}904« vermerkt  2) mit rotem Buntstift zwei Unterstreichungen}\toendnotes[C]{\smallbreak}
\pstart
           \noindent{}\raggedleft{}{\pb}\textcolor{gray}{\textbf{\textcolor{pink}{DESSAUERSTRASSE 19}{}\ledrightnote{\textcolor{pink}{Dessauer Straße}}}}\pend
           
\pstart
           \textcolor{pink}{Berlin}{}\ledrightnote{\textcolor{pink}{Berlin}}, 23. Juni.\pend
           
\pstart\center{}Mein lieber Freund,\pend
\pstart
           Ich habe mich ſehr gefreut, zu erſehen, daß Ihr, Du und Deine \textcolor{blue}{Frau}{}\ledrightnote{{$\rightarrow$}\textcolor{blue}{Olga Schnitzler}}, wohlbehalten zurückgekommen ſeid und
               daß Eure \label{K_L03445-1v}\edtext{Reiſe}{\lemma{\textnormal{\emph{Reiſe}}}\Cendnote{\textnormal{siehe Paul Goldmann an Arthur Schnitzler, 14. 3. [1904]}}}\label{K_L03445-1h} ſo ſchön verlaufen iſt. Und bei der Rückkehr aus \textsc{\textcolor{pink}{Taormina}{}\ledrightnote{\textcolor{pink}{Taormina}}} und \textsc{\textcolor{pink}{Pompeji}{}\ledrightnote{\textcolor{pink}{Pompei}}} zu Hauſe einen blondlockigen \textcolor{blue}{Sohn}{}\ledrightnote{{$\rightarrow$}\textcolor{blue}{Heinrich Schnitzler}} vorzufinden, iſt auch nicht übel.\pend
           
\pstart
           Ob mich mein Weg dieſes Jahr nach \label{K_L03445-2v}\edtext{\textcolor{pink}{Wien}{}\ledrightnote{\textcolor{pink}{Wien}}}{\lemma{\textnormal{\emph{Wien}}}\Cendnote{\textnormal{\textcolor{blue}{Goldmann} war jedenfalls am 10. 8. 1904 und am 11. 8. 1904 in \textcolor{pink}{Wien}. Am 11. 8. 1904 besuchte er \textcolor{blue}{Arthur und Olga Schnitzler}. Im September war
                  er noch einmal in Wien, vgl. A. S.: \emph{Tagebuch}, 21. 9. 1904.}}}\label{K_L03445-2h} führen wird, iſt fraglich. Sollte es der Fall ſein, ſo
               wird es mir natürlich eine große Freude ſein, Dich dort wiederzuſehen. {\pb}Bei \textcolor{pink}{Marienbad}{}\ledrightnote{\textcolor{pink}{Marienbad}}
               bleibt es wahrſcheinlich. Was hinterher noch geſchehen wird, iſt ganz ungewiß. Sobald
               ich Genaueres weiß, theile ich es Dir mit; und es wäre ſehr ſchön, wenn ſich eine
               Möglichkeit finden ließe, Dich unterwegs zu treffen.\pend
           
\pstart
           Jetzt im Sommer werden ſich wohl wieder alle Vorzüge Eurer prachtvoll gelegenen
                  \label{K_L03445-3v}\edtext{\textcolor{pink}{Wohnung}{}\ledrightnote{{$\rightarrow$}\textcolor{pink}{Edmund-Weiß-Gasse}}}{\lemma{\textnormal{\emph{Wohnung}}}\Cendnote{\textnormal{siehe Felix Salten an Arthur Schnitzler, 17. 9. 1903}}}\label{K_L03445-3h} entfalten, und ich wünſche Dir eine Reihe guter Arbeitsſtunden auf Deiner
               Veranda mit dem Blick ins {\pb}Grüne. Schreibſt Du ein
                  \label{K_L03445-4v}\edtext{neues Stück}{\lemma{\textnormal{\emph{neues Stück}}}\Cendnote{\textnormal{Das nächste große dramatische Werk, an dem \textcolor{blue}{Schnitzler} arbeitete, war die Komödie \emph{\textcolor{green}{Zwischenspiel}}. Sie wurde am 12. 10. 1905 am \textcolor{pink}{Burgtheater} uraufgeführt.}}}\label{K_L03445-4h}? Und gedenkſt
               Du Dich \substVorne{}\textsuperscript{damit}\substDazwischen{}damit\substHinten{} an dem \label{K_L03445-5v}\edtext{Wettkampf der
                  Theater}{\lemma{\textnormal{\emph{Wettkampf der
                  Theater}}}\Cendnote{\textnormal{womöglich Bezug auf die
                  Übernahme des \emph{\textcolor{brown}{Deutschen Theater}}s durch \textcolor{blue}{Max Reinhardt} im Oktober 1905}}}\label{K_L03445-5h} zu betheiligen, der im kommenden Winter in \textcolor{pink}{Berlin}{}\ledrightnote{\textcolor{pink}{Berlin}} mit noch nicht dageweſener Heftigkeit entbrennen wird?\pend
           
\pstart
           Meine \textcolor{blue}{Freundin}{}\ledrightnote{{$\rightarrow$}\textcolor{blue}{Theodore Rottenberg}} erwidert
               herzlich Deinen Gruß. Es geht \strikeout{ih} ihr, wie es ihr
               ging. Sie leidet ſchwer unter den unerträglichen Verhältniſſen ihrer \textcolor{blue}{Ehe}{}\ledrightnote{{$\rightarrow$}\textcolor{blue}{Ludwig Rottenberg}} und der Enge und gemeinen
               Klatſchſucht der \textcolor{pink}{Kleinſtadt}{}\ledrightnote{{$\rightarrow$}\textcolor{pink}{Frankfurt am Main}}.
               Sie ſehnt ſich danach, \strikeout{\textcolor{gray}{n}} ſich mit
               mir zu vereinigen; ich ſehne mich nach ihr. Aber die {\pb}materiellen Verhältniſſe erlauben es nicht, dieſe beiderfeitige Sehnſucht endgiltig
               zu befriedigen. Und die Löſung iſt nach wie vor: Fortwurſteln{\dotsfour}\pend
           
\pstart
           Daß Ihr \textsc{\textcolor{blue}{Hoffmannsthal}{}\ledrightnote{\textcolor{blue}{Hugo von Hofmannsthal}}} in der \label{K_L03445-6v}\edtext{\textsc{\textcolor{blue}{Liliencron}{}\ledrightnote{\textcolor{blue}{Detlev von Liliencron}}}-Affaire}{\lemma{\textnormal{\emph{Liliencron-Affaire}}}\Cendnote{\textnormal{siehe Hugo von Hofmannsthal an Arthur Schnitzler, 1[9?]. 6. [1904] und A. S.: \emph{Tagebuch}, 2. 6. 1904}}}\label{K_L03445-6h} Unrecht gebt, erfreut mich ebenſoſehr, wie es mich überraſcht.\pend
           
\pstart
           Ich fahre heut{ }Mittag nach \textsc{\textcolor{pink}{Kiel}{}\ledrightnote{\textcolor{pink}{Kiel}}}, um über die \label{K_L03445-7v}\edtext{\textcolor{blue}{Monarchen}{}\ledrightnote{{$\rightarrow$}\textcolor{blue}{Eduard VII.}{\newline}{$\rightarrow$}\textcolor{blue}{Wilhelm II. von Preußen}}-Zuſammenkunft}{\lemma{\textnormal{\emph{Monarchen-Zuſammenkunft}}}\Cendnote{\textnormal{Bezug
                  auf den Besuch des \textcolor{pink}{engl}ischen Königs \textcolor{blue}{Eduard VII.} im
                  Rahmen der \textcolor{pink}{Kiel}er Woche und dem
                  Zusammentreffen mit dessen Neffen, dem \textcolor{pink}{deutsch}en Kaiser \textcolor{blue}{Wilhelm
                     II.}}}}\label{K_L03445-7h} zu berichten.\pend
           
\pstart
           Herzliche Grüße an Dich und Deine \textcolor{blue}{Frau}{}\ledrightnote{{$\rightarrow$}\textcolor{blue}{Olga Schnitzler}} von Deinem getreuen {\\[\baselineskip]}\spacefill\mbox{Paul Goldmann}\pend
           \leftskip=0em{}\endnumbering\briefempfaengerindex{Schnitzler, Arthur@\textsc{Schnitzler, Arthur}!zzzGoldmann, Paul@\emph{von Paul Goldmann}!1904-06-232@{23. 6. {[}1904{]}}|)be}\mylabel{h}
\begin{anhang}
\end{anhang}\normalsize

\doendnotes{C}
\bigskip
\vfill

\clearpage

\footnotesize

\lohead{\textsc{register}}

% Definiere theindex-Environment komplett neu ohne reledmac
\makeatletter
\renewenvironment{theindex}{%
  \section*{\indexname}%
  \setlength{\parindent}{0pt}%
  \setlength{\parskip}{0pt plus 0.3pt}%
  \let\item\@idxitem
}{%
  \clearpage
}
\makeatother

\IfFileExists{\jobname-pw.ind}{\input{\jobname-pw.ind}}{}

\end{document}

      