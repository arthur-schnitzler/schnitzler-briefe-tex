%% latex-korrekturansicht-vorspann.tex
%% Vorspann für die Korrekturansicht.
%% Lädt die gemeinsame Datei latex-vorspann.tex mit gesetztem Schalter.

\newif\ifkorrekturansicht
\korrekturansichttrue

\input{../tex-inputs/latex-vorspann}


\section[Olga Schnitzler an Stefan Zweig, 20. 11. 1916]{L03766 Olga Schnitzler an Stefan Zweig, 20. 11. 1916}
\nopagebreak\mylabel{L03766v}
\rehead{ }\normalsize\beginnumbering\briefempfaengerindex{, @\textsc{, }!zzz, @\emph{von  }!1916-11-202@{20. 11. 1916}|(be}
\toendnotes[C]{\smallbreak\pagebreak[2]}\Standort{Jerusalem, National Library of Israel, ARC. Ms. Var. 305 1 58 Stefan Zweig Collection.}
\physDesc{Briefkarte, 427 Zeichen
\newline{}Handschrift: schwarze Tinte, lateinische Kurrent}\toendnotes[C]{\smallbreak}
\pstart
           \noindent{}{\pb}Herzlichsten Dank, lieber Herr Doctor für die \label{K_L03766-1v}\edtext{prachtvolle Blumen}{\lemma{\textnormal{\emph{prachtvolle Blumen}}}\Cendnote{\textnormal{Am 18. 11. 1916 war \textcolor{blue}{Olga Schnitzler}\pwindex{Schnitzler, Olga 17.\,1.\,1882 Wien – 13.\,1.\,1970 Lugano@\textsc{Schnitzler, Olga} (17.\,1.\,1882 Wien – 13.\,1.\,1970 Lugano), \emph{Schauspielerin, Sängerin}|pwk}
                  an einem \emph{\textcolor{violet}{Liederkonzert}\eventindex{Wiener Konzerthaus@\textbf{Wiener Konzerthaus}!Gesangskonzert von Olga Schnitzler, 18.11.1916@Gesangskonzert von Olga Schnitzler, 18.11.1916|pwk}} im \textcolor{pink}{Wiener Konzerthaus}\oindex{Wien@\textbf{Wien}!III., Landstraße@\textbf{III., Landstraße}!Wiener Konzerthaus@\textbf{Wiener Konzerthaus}, \emph{Konzertsaal}|pwk} beteiligt gewesen.}}}\label{K_L03766-1}! ich habe mich
               sehr gefreut! Und Dank vor Allem für Ihr immer reges und warmes Interesse, das mir
               sehr wol tut. Gestern war’s mir leider nicht möglich, sie telefonisch zu erreichen, –
               ich wollte Sie für den \label{K_L03766-2v}\edtext{Abend mit der
                  \textcolor{blue}{Hofräthin}\pwindex{Zuckerkandl, Berta 13.\,4.\,1864 Wien – 16.\,10.\,1945 Paris@\textsc{Zuckerkandl, Berta} (13.\,4.\,1864 Wien – 16.\,10.\,1945 Paris), \emph{Journalistin, Übersetzerin}|pwv}{}\ledrightnote{{$\rightarrow$}\emph{\textcolor{blue}{Berta Zuckerkandl}}}}{\lemma{\textnormal{\emph{Abend mit der
                  Hofräthin}}}\Cendnote{\textnormal{Vgl. A. S.: \emph{Tagebuch}, 19. 11. 1916.}}}\label{K_L03766-2}
               herbitten. So hoff ich auf ein anderes Mal, – und {\pb}auf
               bald.\pend
           
\pstart
           Mit den herzlichsten Grüssen, auch von \textcolor{blue}{Arthur}{}\ledrightnote{},{\\[\baselineskip]}Ihre{\\[\baselineskip]}\spacefill\mbox{Olga Schnitzler}\pend
           \leftskip=0em{}
\pstart
           20. Nov. 1916.\pend
           \selectlanguage{ngerman}\endnumbering\briefempfaengerindex{, @\textsc{, }!zzz, @\emph{von  }!1916-11-202@{20. 11. 1916}|)be}\mylabel{L03766h}  \normalsize

\doendnotes{C}
\bigskip
\vfill

\clearpage

\footnotesize

\lohead{\textsc{register}}

% Definiere theindex-Environment komplett neu ohne reledmac
\makeatletter
\renewenvironment{theindex}{%
  \section*{\indexname}%
  \setlength{\parindent}{0pt}%
  \setlength{\parskip}{0pt plus 0.3pt}%
  \let\item\@idxitem
}{%
  \clearpage
}
\makeatother

\IfFileExists{\jobname-pw.ind}{\input{\jobname-pw.ind}}{}

\end{document}

      