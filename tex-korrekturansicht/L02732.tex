%% latex-korrekturansicht-vorspann.tex
%% Vorspann für die Korrekturansicht.
%% Lädt die gemeinsame Datei latex-vorspann.tex mit gesetztem Schalter.

\newif\ifkorrekturansicht
\korrekturansichttrue

\input{../tex-inputs/latex-vorspann}


               \section[Paul Goldmann an Arthur Schnitzler, 28. 3. {[}1895{]}]{ Paul Goldmann an Arthur Schnitzler, 28. 3. {[}1895{]}}\nopagebreak\mylabel{v}\rehead{ }\normalsize\beginnumbering\briefempfaengerindex{Schnitzler, Arthur@\textsc{Schnitzler, Arthur}!zzzGoldmann, Paul@\emph{von Paul Goldmann}!1895-03-281@{28. 3. {[}1895{]}}|(be} \toendnotes[C]{\smallbreak\pagebreak[2]} \Standort{DLA, A:Schnitzler, HS.NZ85.1.3165.}
\physDesc{Brief, 1 Blatt, 4 Seiten
\newline{}Handschrift: schwarze Tinte, deutsche Kurrent
\newline{}Schnitzler: 1) mit schwarzer Tinte das Jahr »95« vermerkt 2) mit rotem Buntstift drei Unterstreichungen}\toendnotes[C]{\smallbreak}\pstart
           \noindent{}{\pb}\textcolor{gray}{\textbf{\textbf{\textcolor{brown}{Frankfurter Zeitung}{}\ledrightnote{\textcolor{brown}{Frankfurter Zeitung}}}}}\pend
           \pstart
           \textcolor{gray}{\textbf{(\textcolor{brown}{\begin{otherlanguage}{french}Gazette de Francfort\end{otherlanguage}}{}\ledrightnote{\textcolor{brown}{Frankfurter Zeitung}}). }}\pend
           \pstart
           \textcolor{gray}{\textbf{\textbf{\begin{otherlanguage}{french}Fondateur M. \textcolor{blue}{L.
                                 Sonnemann}{}\ledrightnote{\textcolor{blue}{Leopold Sonnemann}}\end{otherlanguage}.}}}\hfill \textsc{\textcolor{pink}{Paris}{}\ledrightnote{\textcolor{pink}{Paris}}}, 28. März.\pend
           \pstart
           \begin{otherlanguage}{french}\textcolor{gray}{\textbf{\textcolor{green}{Journal}{}\ledrightnote{\textcolor{green}{Frankfurter Zeitung}} politique, financier,}}\end{otherlanguage}\pend
           \pstart
           \begin{otherlanguage}{french}\textcolor{gray}{\textbf{commercial et littéraire.}}\end{otherlanguage}\pend
           \pstart
           \begin{otherlanguage}{french}\textcolor{gray}{\textbf{\textbf{Paraissant trois fois par jour.}}}\end{otherlanguage}\pend
           \pstart
           \begin{otherlanguage}{french}\textcolor{gray}{\textbf{\textbf{Bureau à \textcolor{pink}{Paris}{}\ledrightnote{\textcolor{pink}{Paris}}:}}}\end{otherlanguage}\pend
           \pstart
           \begin{otherlanguage}{french}\textcolor{gray}{\textbf{\textbf{\textcolor{pink}{24. Rue Feydeau}{}\ledrightnote{\textcolor{pink}{rue Feydeau}}.}}}\end{otherlanguage}\pend
           \pstart\center{}Mein lieber Freund,\pend\pstart
           \textsc{\textcolor{blue}{Henri Albert}{}\ledrightnote{\textcolor{blue}{Henri Albert}}s }{ }\label{K_L02732-1v}\edtext{\textcolor{green}{Artikel}{}\ledrightnote{→\textcolor{green}{Les Jeunes Viennois}}}{\lemma{\textnormal{\emph{Artikel}}}\Cendnote{\textnormal{\textcolor{blue}{Henri Albert}: \emph{\textcolor{green}{Les Jeunes Viennois}}. In: \emph{\textcolor{green}{Revue des Revues}}, Bd. 13, 1. 4. 1895, S. 8–13.}}}\label{K_L02732-1h} erscheint morgen oder übermorgen in der »\textsc{\textcolor{green}{Revue des Revues}{}\ledrightnote{\textcolor{green}{Revue des Revues}}}«. Ich ſende Dir zwei \textcolor{green}{Bürſtenabzüge}{}\ledrightnote{→\textcolor{green}{Revue des Revues}}, einen für Dich, einen für \textsc{\textcolor{blue}{Richard}{}\ledrightnote{\textcolor{blue}{Richard Beer-Hofmann}}}. Der \textcolor{green}{Artikel}{}\ledrightnote{→\textcolor{green}{Les Jeunes Viennois}} hat manche
               Fehler in Auffaſſung und Ausdruck. \textsc{\textcolor{blue}{Bahr}{}\ledrightnote{\textcolor{blue}{Hermann Bahr}}} iſt zu ſehr herausgeſtrichen, Du zu wenig. Aber im Ganzen gefällt mir die
               kleine \textcolor{green}{Abhandlung}{}\ledrightnote{→\textcolor{green}{Les Jeunes Viennois}} und wird Dir
               wohl auch gefallen.\pend
           \pstart
           Über Deinen lieben ausführlichen Brief habe ich {\pb}mich ſehr gefreut. Ich danke Dir einſtweilen dafür und ſchreibe Dir nächſtens.\pend
           \pstart
           Schreib’ bitte an \textsc{\textcolor{blue}{Henri Albert}{}\ledrightnote{\textcolor{blue}{Henri Albert}}} (\textcolor{pink}{21. \textsc{Rue Jacob}}{}\ledrightnote{\textcolor{pink}{rue Jacob}}) ein ſaar Zeilen des Dankes. Auch \textsc{\textcolor{blue}{Richard}{}\ledrightnote{\textcolor{blue}{Richard Beer-Hofmann}}} ſoll das thun.\pend
           \pstart
           Schreib’ mir, ob Dir der \textcolor{green}{Artikel}{}\ledrightnote{→\textcolor{green}{Les Jeunes Viennois}} gefallen hat, ob ich Dir weiter \textcolor{pink}{Pariſ}{}\ledrightnote{\textcolor{pink}{Paris}}er Zeitungsartikel ſchicken ſoll, ob Ihr den \textsc{\textcolor{green}{Courrier Français}{}\ledrightnote{\textcolor{green}{Le Courrier français}}} bekommt? Die letzten beiden Fragen muß ich nun ſchon zum dritten Mal ſtellen.
               Oh! Oh! Oh!\pend
           \pstart
           {\pb}Bitte, bitte komm’ nach \textsc{\textcolor{pink}{Paris}{}\ledrightnote{\textcolor{pink}{Paris}}}!\pend
           \pstart
           Auch \textsc{\textcolor{blue}{Richard}{}\ledrightnote{\textcolor{blue}{Richard Beer-Hofmann}}} ſoll kommen: es iſt Frühling hier und große Schönheit.\pend
           \pstart
           Über das \textcolor{green}{Buch}{}\ledrightnote{→\textcolor{green}{Der Garten der Erkenntnis}} von \textsc{\textcolor{blue}{Andrian}{}\ledrightnote{\textcolor{blue}{Leopold von Andrian-Werburg}}} bin ich Zeile für Zeile und Wort für Wort Deiner Anſicht. Eine unreife \textcolor{green}{\textcolor{blue}{Dilettant}{}\ledrightnote{→\textcolor{blue}{Leopold von Andrian-Werburg}}en-Arbeit}{}\ledrightnote{→\textcolor{green}{Der Garten der Erkenntnis}},
               mit viel Selbſtgefälligkeit, viel Unklarheit, viel Anempfindung. \introOben{}und einigen ſchönen Wendungen.\introOben{} Solche \textcolor{green}{Sachen}{}\ledrightnote{→\textcolor{green}{Der Garten der Erkenntnis}} läßt man in ſeinem Pult liegen und gibt ſie nicht als
                  \textcolor{green}{Buch}{}\ledrightnote{→\textcolor{green}{Der Garten der Erkenntnis}} heraus. Es gehört die
               ganze Urtheilslosigkeit und {\pb}Gewiſſensloſigkeit
               eines \textsc{\textcolor{blue}{Bahr}{}\ledrightnote{\textcolor{blue}{Hermann Bahr}}} dazu, um das \label{K_L02732-2v}\edtext{als ein\strikeout{e}{ }\textcolor{green}{Literatur-Ereigniß}{}\ledrightnote{→\textcolor{green}{Der Garten der Erkenntnis}} zu
               proklamiren}{\lemma{\textnormal{\emph{als … proklamiren}}}\Cendnote{\textnormal{\textcolor{blue}{Leopold von Andrian-Werburg}s Erzählung \emph{\textcolor{green}{Der Garten der Erkenntnis}} hatte durch \textcolor{blue}{Bahr} einen Verleger gefunden. Dieser hatte \textcolor{blue}{Samuel Fischer} in einem Brief vom 25. 1. 1895{ }\textcolor{blue}{Andrian}s Text als »das beste \textcolor{green}{Werk} nach meinem Urteile, was bisher
                        die \textcolor{pink}{europ}äische Moderne
                        hervorgebracht hat« zur Veröffentlichung anempfohlen. (\textcolor{blue}{Samuel Fischer}. \textcolor{blue}{Hedwig
                              Fischer}. Briefwechsel mit Autoren. Hg. v. Dierk Rodewald und Corinna
                  Fiedler. Mit einer Einführung von Bernhard Zeller. \textcolor{pink}{Frankfurt a. M.}: \emph{\textcolor{brown}{S. Fischer}}{ }1989, S. 171–172.) Anlässlich des Erscheinens
                  veröffentlichte \textcolor{blue}{Bahr} eine Rezension in der \emph{\textcolor{green}{Die Zeit}}: \textcolor{blue}{Hermann Bahr}: \emph{\textcolor{green}{Der Garten der Erkenntnis}}. In: \emph{\textcolor{green}{Die Zeit. Wiener Wochenschrift}}, Bd. 2, H. 24, 16. 3. 1895, S. 171–172. \textcolor{blue}{Schnitzler} las 
                   am Tag nach Erscheinen von \textcolor{blue}{Bahr}s Besprechung,
                   am 17. 3. 1895 und notierte
                  sich dazu: »Spuren eines
                        \textcolor{blue}{Künstler}s, schöne
                     Vergleiche.– Keine Gestaltung, Affectation, Unklarheiten, – unreifer \textcolor{blue}{Loris} – nicht reifer \textcolor{blue}{Goethe}, wie \textcolor{blue}{Bahr}
                     sagte.– Es mit »\textcolor{green}{Kind}« oder »\textcolor{green}{Sterben}« vergleichen ist dumm und
                     frech.«.}}}\label{K_L02732-2h}! Welch\strikeout{e}’ ein \textcolor{blue}{Verderber}{}\ledrightnote{→\textcolor{blue}{Hermann Bahr}}
               von Geſchmack und Talent!\pend
           \pstart
           Aber nein, ich habe \strikeout{keine} ja keine Zeit, Dir heut zu ſchreiben. Auf nächſtens alſo!\pend
           \pstart
           Grüß’ Dich Gott! {\\[\baselineskip]}Dein {\\[\baselineskip]}treuer {\\[\baselineskip]}\spacefill\mbox{Paul Goldmann.}\pend
           \leftskip=0em{}\endnumbering\briefempfaengerindex{Schnitzler, Arthur@\textsc{Schnitzler, Arthur}!zzzGoldmann, Paul@\emph{von Paul Goldmann}!1895-03-281@{28. 3. {[}1895{]}}|)be}\mylabel{h}\begin{anhang}\end{anhang}\normalsize

\doendnotes{C}
\bigskip
\vfill

\clearpage

\footnotesize

\lohead{\textsc{register}}

% Definiere theindex-Environment komplett neu ohne reledmac
\makeatletter
\renewenvironment{theindex}{%
  \section*{\indexname}%
  \setlength{\parindent}{0pt}%
  \setlength{\parskip}{0pt plus 0.3pt}%
  \let\item\@idxitem
}{%
  \clearpage
}
\makeatother

\IfFileExists{\jobname-pw.ind}{\input{\jobname-pw.ind}}{}

\end{document}

      