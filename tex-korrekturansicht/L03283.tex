%% latex-korrekturansicht-vorspann.tex
%% Vorspann für die Korrekturansicht.
%% Lädt die gemeinsame Datei latex-vorspann.tex mit gesetztem Schalter.

\newif\ifkorrekturansicht
\korrekturansichttrue

\input{../tex-inputs/latex-vorspann}


\renewcommand{\erwaehnteInstitutionen}{Institutionen: Wiener Allgemeine Zeitung}
\renewcommand{\erwaehnteOrte}{Orte: Schlosspark Schönbrunn, Wien}
\renewcommand{\erwaehnteWerke}{}
\section[ Felix Salten an Arthur Schnitzler, 13. 9. 1898]{Felix Salten an Arthur Schnitzler, 13. 9. 1898}
\nopagebreak\mylabel{v}
\rehead{ }\normalsize\beginnumbering\briefempfaengerindex{Schnitzler, Arthur@\textsc{Schnitzler, Arthur}!zzzSalten, Felix@\emph{von Felix Salten}!1898-09-131@{13. 9. 1898}|(be}
\toendnotes[C]{\smallbreak\pagebreak[2]}\Standort{CUL, Schnitzler, B 89, A 2.}
\physDesc{Brief, 1 Blatt, 1 Seite, 357 Zeichen
\newline{}Handschrift: schwarze Tinte, lateinische Kurrent
\newline{}Ordnung: mit Bleistift von unbekannter Hand nummeriert: »107« }\toendnotes[C]{\smallbreak}
\pstart
           \raggedleft{}{\pb}\textcolor{pink}{Wien}{}\ledrightnote{\textcolor{pink}{Wien}}, 13. September 98\pend
           
\pstart
           Lieber Arthur, haben Sie meinen \label{K_L03283-1v}\edtext{Brief}{\lemma{\textnormal{\emph{Brief}}}\Cendnote{\textnormal{Felix Salten an Arthur Schnitzler, [8.? 9. 1898]}}}\label{K_L03283-1h} bekommen? Ich konnte damals nicht ins Theater und schlug Ihnen vor, an einem
               Nachmittag in \textcolor{pink}{Schönbrunn}{}\ledrightnote{\textcolor{pink}{Schlosspark Schönbrunn}} spazieren zu gehen.
               Möchten Sie nicht? Ich würde mich schon sehr freuen, Sie wieder zu sehen. Schreiben
               Sie mir doch eine Zeile, – am besten, Sie verständigen mich Vormittag im
                  \textcolor{brown}{Bureau}{}\ledrightnote{{$\rightarrow$}\textcolor{brown}{Wiener Allgemeine Zeitung}}.\pend
           
\pstart
           Herzlichst Ihr {\\[\baselineskip]}\spacefill\mbox{Salten}\pend
           \leftskip=0em{}\endnumbering\briefempfaengerindex{Schnitzler, Arthur@\textsc{Schnitzler, Arthur}!zzzSalten, Felix@\emph{von Felix Salten}!1898-09-131@{13. 9. 1898}|)be}\mylabel{h}  \normalsize

\doendnotes{C}
\bigskip
\vfill

\clearpage

\footnotesize

\lohead{\textsc{register}}

% Definiere theindex-Environment komplett neu ohne reledmac
\makeatletter
\renewenvironment{theindex}{%
  \section*{\indexname}%
  \setlength{\parindent}{0pt}%
  \setlength{\parskip}{0pt plus 0.3pt}%
  \let\item\@idxitem
}{%
  \clearpage
}
\makeatother

\IfFileExists{\jobname-pw.ind}{\input{\jobname-pw.ind}}{}

\end{document}

      