%% latex-korrekturansicht-vorspann.tex
%% Vorspann für die Korrekturansicht.
%% Lädt die gemeinsame Datei latex-vorspann.tex mit gesetztem Schalter.

\newif\ifkorrekturansicht
\korrekturansichttrue

\input{../tex-inputs/latex-vorspann}


\renewcommand{\erwaehntePersonen}{Personen: Felix Salten}
\renewcommand{\erwaehnteOrte}{Orte: Burgtheater, Dresden, Loschwitz, Sanatorium am Königspark, Sternwartestraße 71, Wien, Währinger Cottage}
\renewcommand{\erwaehnteWerke}{}
\section[ Felix Salten an Arthur Schnitzler, 8. 2. 1927]{Felix Salten an Arthur Schnitzler, 8. 2. 1927}
\nopagebreak\mylabel{v}
\rehead{ }\normalsize\beginnumbering\briefempfaengerindex{Schnitzler, Arthur@\textsc{Schnitzler, Arthur}!zzzSalten, Felix@\emph{von Felix Salten}!1927-02-081@{8. 2. 1927}|(be}
\toendnotes[C]{\smallbreak\pagebreak[2]}\Standort{CUL, Schnitzler, B 89, B 2.}
\physDesc{Bildpostkarte, 402 Zeichen
\newline{}Handschrift: schwarze Tinte, lateinische Kurrent
\newline{}Versand: Stempel: »\nobreak{}\oindex{Dresden@\textbf{Dresden}, \emph{P.PPLA}|pwk}Dresden
                                       \textcolor{gray}{L}oschwitz, 10. 2. 27, 11–12 V\nobreak{}«.  
\newline{}Ordnung: mit Bleistift von unbekannter Hand nummeriert: »298« }\toendnotes[C]{\smallbreak}\pstart{}{\pb}Herrn D\textsuperscript{r} Arthur Schnitzler\pend{}\pstart{}\textcolor{pink}{Wien}{}\ledrightnote{\textcolor{pink}{Wien}}\pend{}\pstart{}\textcolor{pink}{XVIII. Sternwartestrasse 71}{}\ledrightnote{\textcolor{pink}{Sternwartestraße 71}}\pend{}
{\bigskip}
\pstart
           \noindent{}\centering{}{\pb}\textcolor{pink}{\textcolor{gray}{\textbf{Sanatorium am Königspark}}}{}\ledrightnote{\textcolor{pink}{Sanatorium am Königspark}}\pend
           
\pstart
           \noindent{}\centering{}\textcolor{pink}{\textcolor{gray}{\textbf{Dresden-Loschwitz}}}{}\ledrightnote{\textcolor{pink}{Loschwitz}}\pend
           
\pstart
           \noindent{}\centering{}\textcolor{gray}{\textbf{Bibliothek}}\pend
           
\pstart
           \raggedleft{}{\pb}8-II-27\pend
           
\pstart
           Lieber,{ }\label{K_L03596-1v}\edtext{wo sind Sie}{\lemma{\textnormal{\emph{wo sind Sie}}}\Cendnote{\textnormal{\textcolor{blue}{Schnitzler} war in \textcolor{pink}{Wien}.}}}\label{K_L03596-1h}? Wie geht es Ihnen? Im \textcolor{pink}{Cottage}{}\ledrightnote{\textcolor{pink}{Währinger Cottage}} bleiben wir einander so fern, als sei der Weg zu weit.
               Wie es mir geht – falls Sie das noch kümmert – sehen Sie nach dem Ort, von dem ich
               Ihnen schreibe. Ich denke viel an Sie – nicht blos \textcolor{pink}{hier}{}\ledrightnote{{$\rightarrow$}\textcolor{pink}{Loschwitz}}! Wenn ich wieder in \textcolor{pink}{Wien}{}\ledrightnote{\textcolor{pink}{Wien}} bin, \label{K_L03596-2v}\edtext{klopfe ich bei
               Ihnen an}{\lemma{\textnormal{\emph{klopfe ich bei
               Ihnen an}}}\Cendnote{\textnormal{Nachweislich trafen sie sich das
                  nächste Mal am 25. 2. 1927 im \textcolor{pink}{Burgtheater}.}}}\label{K_L03596-2h}. Die Zeit ist so kurz!\pend
           
\pstart
           Herzlich Ihr {\\[\baselineskip]}\spacefill\mbox{Felix Salten}\pend
           \leftskip=0em{}\endnumbering\briefempfaengerindex{Schnitzler, Arthur@\textsc{Schnitzler, Arthur}!zzzSalten, Felix@\emph{von Felix Salten}!1927-02-081@{8. 2. 1927}|)be}\mylabel{h}  \normalsize

\doendnotes{C}
\bigskip
\vfill

\clearpage

\footnotesize

\lohead{\textsc{register}}

% Definiere theindex-Environment komplett neu ohne reledmac
\makeatletter
\renewenvironment{theindex}{%
  \section*{\indexname}%
  \setlength{\parindent}{0pt}%
  \setlength{\parskip}{0pt plus 0.3pt}%
  \let\item\@idxitem
}{%
  \clearpage
}
\makeatother

\IfFileExists{\jobname-pw.ind}{\input{\jobname-pw.ind}}{}

\end{document}

      