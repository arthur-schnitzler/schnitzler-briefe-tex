%% latex-korrekturansicht-vorspann.tex
%% Vorspann für die Korrekturansicht.
%% Lädt die gemeinsame Datei latex-vorspann.tex mit gesetztem Schalter.

\newif\ifkorrekturansicht
\korrekturansichttrue

\input{../tex-inputs/latex-vorspann}


\renewcommand{\erwaehnteInstitutionen}{Institutionen: Stadttheater Teplitz}
\renewcommand{\erwaehnteOrte}{Orte: Dresden, Teplice, Wien}
\renewcommand{\erwaehnteWerke}{}
\section[ Felix Salten an Arthur Schnitzler, 6. 5. 1899]{Felix Salten an Arthur Schnitzler, 6. 5. 1899}
\nopagebreak\mylabel{v}
\rehead{ }\normalsize\beginnumbering\briefempfaengerindex{Schnitzler, Arthur@\textsc{Schnitzler, Arthur}!zzzSalten, Felix@\emph{von Felix Salten}!1899-05-061@{6. 5. 1899}|(be}
\toendnotes[C]{\smallbreak\pagebreak[2]}\Standort{CUL, Schnitzler, B 89, A 2.}
\physDesc{Brief, 1 Blatt, 3 Seiten, 947 Zeichen
\newline{}Handschrift: schwarze Tinte, lateinische Kurrent
\newline{}Ordnung: mit Bleistift von unbekannter Hand nummeriert: »115« }\toendnotes[C]{\smallbreak}
\pstart
           \raggedleft{}{\pb}\textcolor{pink}{Teplitz}{}\ledrightnote{\textcolor{pink}{Teplice}}, 6. Mai 99.\pend
           
\pstart
           Lieber Freund, ich nehme an, dass die Telegramme von heute, wie die \label{K_L03291-1v}\edtext{Sendung an den Magistrat}{\lemma{\textnormal{\emph{Sendung an den Magistrat}}}\Cendnote{\textnormal{Am 11. 4. 1899 verlautbarte
                  die Stadtregierung von \textcolor{pink}{Teplitz}, dass das \emph{\textcolor{brown}{Stadttheater}} ab dem 1. 10. 1899 auf vier Jahre zur Pacht frei würde. \textcolor{blue}{Salten} hatte sich bereits zwei Jahre zuvor darum bemüht
                     (vgl. Felix Salten an Arthur Schnitzler, [10. 1. 1897]) und erneuerte
                  sein Interesse. Die für die Bewerbung benötigten  1.000 Gulden dürfte er von \textcolor{blue}{Schnitzler} ausgeliehen oder vermittelt
                  bekommen haben. Woran \textcolor{blue}{Salten}s Bewerbung
                  scheiterte, ist nicht bekannt. Seine fehlende Erfahrung als Theaterleiter dürfte
                  jedenfalls nicht geholfen haben.}}}\label{K_L03291-1h} von Ihnen herrühren\textcolor{gray}{,}
               und danke Ihnen sehr herzlich dafür. Ich wußte wirklich nicht, dass der Termin so
               kurz gestellt ist, sonst hätte ich mir die Sache vorher geordnet. Überhaupt habe ich
               mich erst vor ein paar Tagen zu \textcolor{pink}{Teplitz}{}\ledrightnote{\textcolor{pink}{Teplice}}
               entschloßen, und schrieb Ihnen deshalb vor meiner Abreise kurz »\textcolor{pink}{Dresden}{}\ledrightnote{\textcolor{pink}{Dresden}}«, wie ich es allen gesagt hatte. Ich hatte {\pb}weder Zeit noch Ruhe, Ihnen
               diese neue \textcolor{pink}{Teplitz}{}\ledrightnote{\textcolor{pink}{Teplice}}er Affaire brieflich zu
               erklären. Entschuldigen Sie, bitte, dass ich Sie so plötzlich und so dringend in
               Anspruch nahm. Ich brauche Ihnen wol nicht erst zu sagen, dass die Tausend Gulden
               ganz sicher sind, und dass Sie sie in der kürzesten Zeit (1 Monat längstens) wieder
               erhalten.\pend
           
\pstart
           Dienstag früh bin ich wieder in \textcolor{pink}{Wien}{}\ledrightnote{\textcolor{pink}{Wien}}. Wenn ich zu Hause eine Zeile von Ihnen fände, wo ich Sie
                  Abends{ }{\pb}\label{K_L03291-2v}\edtext{treffen}{\lemma{\textnormal{\emph{treffen}}}\Cendnote{\textnormal{nicht nachweisbar}}}\label{K_L03291-2h} kann, wär es mir sehr lieb.\pend
           
\pstart
           Nochmals wärmsten Dank. {\\[\baselineskip]}Herzlichst Ihr {\\[\baselineskip]}\spacefill\mbox{Salten}\pend
           \leftskip=0em{}\endnumbering\briefempfaengerindex{Schnitzler, Arthur@\textsc{Schnitzler, Arthur}!zzzSalten, Felix@\emph{von Felix Salten}!1899-05-061@{6. 5. 1899}|)be}\mylabel{h}  \normalsize

\doendnotes{C}
\bigskip
\vfill

\clearpage

\footnotesize

\lohead{\textsc{register}}

% Definiere theindex-Environment komplett neu ohne reledmac
\makeatletter
\renewenvironment{theindex}{%
  \section*{\indexname}%
  \setlength{\parindent}{0pt}%
  \setlength{\parskip}{0pt plus 0.3pt}%
  \let\item\@idxitem
}{%
  \clearpage
}
\makeatother

\IfFileExists{\jobname-pw.ind}{\input{\jobname-pw.ind}}{}

\end{document}

      