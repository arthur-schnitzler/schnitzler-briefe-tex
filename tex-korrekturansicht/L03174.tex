%% latex-korrekturansicht-vorspann.tex
%% Vorspann für die Korrekturansicht.
%% Lädt die gemeinsame Datei latex-vorspann.tex mit gesetztem Schalter.

\newif\ifkorrekturansicht
\korrekturansichttrue

\input{../tex-inputs/latex-vorspann}


\renewcommand{\erwaehntePersonen}{Personen: Richard Beer-Hofmann, Victor Hehn, Ottilie Salten}
\renewcommand{\erwaehnteOrte}{Orte: Hörlgasse, Kopenhagen, Skandinavien, Trondheim, Wien}
\renewcommand{\erwaehnteWerke}{Werke: Die Wahlverwandtschaften, Über Goethes Hermann und Dorothea}
\section[ Felix Salten an Arthur Schnitzler, 14. 7. {[}1896{]}]{Felix Salten an Arthur Schnitzler, 14. 7. {[}1896{]}}
\nopagebreak\mylabel{v}
\rehead{ }\normalsize\beginnumbering\briefempfaengerindex{Schnitzler, Arthur@\textsc{Schnitzler, Arthur}!zzzSalten, Felix@\emph{von Felix Salten}!1896-07-142@{14. 7. {[}1896{]}}|(be}
\toendnotes[C]{\smallbreak\pagebreak[2]}\Standort{CUL, Schnitzler, B 89, A 1.}
\physDesc{Brief, 1 Blatt, 2 Seiten, 978 Zeichen
\newline{}Handschrift: schwarze Tinte, lateinische Kurrent
\newline{}Schnitzler: mit Bleistift die Jahreszahl »96« ergänzt 
\newline{}Ordnung: mit Bleistift von unbekannter Hand nummeriert: »73« }\toendnotes[C]{\smallbreak}
\pstart
           \raggedleft{}{\pb}14. Juli\pend
           
\pstart
           lieber Arthur, ich habe eigentlich garnichts zu sagen.
               Ich bin alle Tage von ½ 2 Uhr an \textcolor{pink}{zu Hause}{}\ledrightnote{{$\rightarrow$}\textcolor{pink}{Hörlgasse}}, lese und arbeite und lege mich um ½ 11
               schlafen. Durch das schöne \label{K_L03174-1v}\edtext{\textcolor{green}{Buch}{}\ledrightnote{\textcolor{green}{Über  Goethes Hermann und Dorothea}} von \textcolor{blue}{Victor Hehn}{}\ledrightnote{\textcolor{blue}{Victor Hehn}}}{\lemma{\textnormal{\emph{Buch von Victor Hehn}}}\Cendnote{\textnormal{\emph{\textcolor{green}{Über Goethes Hermann und Dorothea}} (1893)}}}\label{K_L03174-1h} wurde ich darauf gebracht, die »\textcolor{green}{Wahlverwandtschaften}{}\ledrightnote{\textcolor{green}{Die Wahlverwandtschaften}}« zu lesen, die ich nicht kannte. (Ich
               weiss schon, aber ich hab sie vor acht Jahren nicht lesen können) Das war jetzt sehr
               viel für mich und hat mir beim Arbeiten merkwürdig geholfen. Wenn ich nicht so ganz
               allein wäre, ohne einen einzigen Menschen, mit dem ich sprechen könnte, würde es mir
               recht gut gehen. Jedenfalls erhalten Sie, bis Sie wieder da
                  sind{[},{]} Einiges zu hören, und da ich im August mit Frl. \textcolor{blue}{M.}{}\ledrightnote{\textcolor{blue}{Ottilie Salten}} manches
               Entscheidendes zu erleben hoffe, wird auch genug zu erzählen sein. Hören Sie was von
                  \textcolor{blue}{Beer-Hofmann}{}\ledrightnote{\textcolor{blue}{Richard Beer-Hofmann}}? ich möchte gerne wissen, wie
               es ihm geht. Schreiben {\pb}Sie mir
               bald, mir sind diese Postkarten sehr angenehm; und wenn Sie nach \label{K_L03174-2v}\edtext{\textcolor{pink}{Kopenhagen}{}\ledrightnote{\textcolor{pink}{Kopenhagen}}}{\lemma{\textnormal{\emph{Kopenhagen}}}\Cendnote{\textnormal{Bezug auf \textcolor{blue}{Schnitzler}s \textcolor{pink}{Skandinavien}reise im Sommer 1896}}}\label{K_L03174-2h} kommen und dort still sitzen, schwingen Sie sich wol zu einem Brief auf.\pend
           
\pstart
           Viele herzliche Grüße {\\[\baselineskip]}\spacefill\mbox{Salten}\pend
           \leftskip=0em{}\endnumbering\briefempfaengerindex{Schnitzler, Arthur@\textsc{Schnitzler, Arthur}!zzzSalten, Felix@\emph{von Felix Salten}!1896-07-142@{14. 7. {[}1896{]}}|)be}\mylabel{h}  \normalsize

\doendnotes{C}
\bigskip
\vfill

\clearpage

\footnotesize

\lohead{\textsc{register}}

% Definiere theindex-Environment komplett neu ohne reledmac
\makeatletter
\renewenvironment{theindex}{%
  \section*{\indexname}%
  \setlength{\parindent}{0pt}%
  \setlength{\parskip}{0pt plus 0.3pt}%
  \let\item\@idxitem
}{%
  \clearpage
}
\makeatother

\IfFileExists{\jobname-pw.ind}{\input{\jobname-pw.ind}}{}

\end{document}

      