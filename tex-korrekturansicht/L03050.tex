%% latex-korrekturansicht-vorspann.tex
%% Vorspann für die Korrekturansicht.
%% Lädt die gemeinsame Datei latex-vorspann.tex mit gesetztem Schalter.

\newif\ifkorrekturansicht
\korrekturansichttrue

\input{../tex-inputs/latex-vorspann}


\renewcommand{\erwaehntePersonen}{Personen: Leo Kober}
\renewcommand{\erwaehnteInstitutionen}{Institutionen: Georg Müller Verlag}
\renewcommand{\erwaehnteOrte}{Orte: Leipzig, München, Wien}
\renewcommand{\erwaehnteWerke}{Werke: Das Buch der Könige}
\section[Felix Salten: Widmungsexemplar Das Buch der Könige für Arthur Schnitzler, {[}zwischen 1. und 20. 12.{]} 1905]{Felix Salten: Widmungsexemplar Das Buch der Könige für Arthur
               Schnitzler, {[}zwischen 1. und 20. 12.{]} 1905}
\nopagebreak\mylabel{v}
\rehead{ }\normalsize\beginnumbering\briefempfaengerindex{Schnitzler, Arthur@\textsc{Schnitzler, Arthur}!zzzSalten, Felix@\emph{von Felix Salten}!1@{{[}zwischen 1. und 20. 12.{]} 1905}|(be}
\toendnotes[C]{\smallbreak\pagebreak[2]}\Standort{DLA, G:Schnitzler, Arthur (Sammlung Heinrich Schnitzler).}
\physDesc{Widmung am Titelblatt, 66 Zeichen
\newline{}Handschrift: schwarze Tinte, lateinische Kurrent}
\pstart
           \noindent{}\centering{}{\pb}Meinem lieben Arthur Schnitzler\pend
           
\pstart
           herzlichst{\\[\baselineskip]}\spacefill\mbox{Salten}\pend
           \leftskip=0em{}
\pstart
           \textcolor{pink}{Wien}{}\ledrightnote{\textcolor{pink}{Wien}}, Dezember 05\pend
           {\bigskip}
\pstart
           \noindent{}\centering{}\textcolor{gray}{\textbf{\textbf{\textcolor{green}{\so{DAS BUCH DER}{ }{\\}\so{KÖNIGE}}{}\ledrightnote{\textcolor{green}{Das Buch der Könige}}}}}\pend
           
\pstart
           \noindent{}\centering{}\textcolor{gray}{\textbf{VON}}\pend
           
\pstart
           \noindent{}\centering{}\textcolor{gray}{\textbf{\textbf{\so{FELIX SALTEN}}}}\pend
           {\bigskip}
\pstart
           \noindent{}\centering{}\textcolor{gray}{\textbf{MIT ZEICHUNGEN}}\pend
           
\pstart
           \noindent{}\centering{}\textcolor{gray}{\textbf{VON}}\pend
           
\pstart
           \noindent{}\centering{}\textcolor{gray}{\textbf{\textcolor{blue}{LEO \so{KOBER}}{}\ledrightnote{\textcolor{blue}{Leo Kober}}}}\pend
           {\bigskip}
\pstart
           \noindent{}\centering{}\textcolor{gray}{\textbf{\textcolor{pink}{MÜNCHEN}{}\ledrightnote{\textcolor{pink}{München}} UND \textcolor{pink}{LEIPZIG}{}\ledrightnote{\textcolor{pink}{Leipzig}}}}\pend
           
\pstart
           \noindent{}\centering{}\textcolor{gray}{\textbf{BEI \textcolor{brown}{GEORG MÜLLER}{}\ledrightnote{\textcolor{brown}{Georg Müller Verlag}}}}\pend
           \endnumbering\briefempfaengerindex{Schnitzler, Arthur@\textsc{Schnitzler, Arthur}!zzzSalten, Felix@\emph{von Felix Salten}!1905-12-011@{{[}zwischen 1. und 20. 12.{]} 1905}|)be}\mylabel{h}  \normalsize

\doendnotes{C}
\bigskip
\vfill

\clearpage

\footnotesize

\lohead{\textsc{register}}

% Definiere theindex-Environment komplett neu ohne reledmac
\makeatletter
\renewenvironment{theindex}{%
  \section*{\indexname}%
  \setlength{\parindent}{0pt}%
  \setlength{\parskip}{0pt plus 0.3pt}%
  \let\item\@idxitem
}{%
  \clearpage
}
\makeatother

\IfFileExists{\jobname-pw.ind}{\input{\jobname-pw.ind}}{}

\end{document}

      