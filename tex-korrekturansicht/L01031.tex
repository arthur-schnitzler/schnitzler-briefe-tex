%% latex-korrekturansicht-vorspann.tex
%% Vorspann für die Korrekturansicht.
%% Lädt die gemeinsame Datei latex-vorspann.tex mit gesetztem Schalter.

\newif\ifkorrekturansicht
\korrekturansichttrue

\input{../tex-inputs/latex-vorspann}


               \section[Arthur Schnitzler an Hugo von Hofmannsthal, 9. 4. 1900]{ Arthur Schnitzler an Hugo von Hofmannsthal, 9. 4. 1900}\nopagebreak\mylabel{v}\rehead{ }\normalsize\beginnumbering\briefempfaengerindex{Hofmannsthal, Hugo von@\textsc{Hofmannsthal, Hugo von}!zzzSchnitzler, Arthur@\emph{von Arthur Schnitzler}!1900-04-091@{9. 4. 1900}|(be} \toendnotes[C]{\smallbreak\pagebreak[2]} \Standort{FDH, Hs-30885,92.}
\physDesc{Brief, 2 Blätter, 6 Seiten
\newline{}Handschrift: schwarze Tinte, deutsche Kurrent\newline{}Ordnung: Beide Blätter von Schnitzler mutmaßlich bei der Durchsicht der Korrespondenz 1929 mit
                           Bleistift mit »9/4 900« datiert }\buchAbdrucke{\weitereDrucke{Hugo von Hofmannsthal, Arthur Schnitzler: \emph{Briefwechsel}. Hg. Therese Nickl und Heinrich Schnitzler. Frankfurt am Main: \emph{S. Fischer} 1964, S. 137–138.} }\toendnotes[C]{\smallbreak}\pstart
           \raggedleft{}{\pb}9/4 900. \pend
           \pstart
           mein lieber Hugo, heute Vormittag habe ich Ihren \textcolor{blue}{Papa}{}\ledrightnote{→\textcolor{blue}{Hugo August von Hofmannsthal}} geſprochen, und ihn zu meiner Freude ſo
               vortrefflich ausſehend und bei ſo guter Sti{\geminationm}ung
               getroffen, wie nur einer ſein kann, der morgen wieder aufſteht. Ich war geſtern
                  früh gleich nach meiner Ankunft bei Ihrer \textcolor{blue}{Mama}{}\ledrightnote{→\textcolor{blue}{Anna von Hofmannsthal}} und fand ſie ſchon vollkommen beruhigt
               und hauptſächlich froh über die viele Sympathie von allen Seiten, die bei dieſer
               Gelegenheit ſich ausſprach. {\pb}Soweit ich (ohne
               Unterſuchung) das ganze beurtheilen kann, ſcheint mir eine organiſche Erkrankung \introOben{}(des Herzens)\introOben{} vollkommen ausgeſchloſſen; ich weiſs nicht
               einmal, ob es richtig iſt, von »Anfällen von Herzſchwäche« zu ſprechen; mir ko{\geminationm}t der \textsc{vagus} als der ſchuldige
               vor, und als ich heute vor Ihrer \textcolor{blue}{Mama}{}\ledrightnote{→\textcolor{blue}{Anna von Hofmannsthal}} von \textsc{vagus} Neuroſe ſprach, ſagte ſie, Dr. \textcolor{blue}{\textsc{Schandlbauer}}{}\ledrightnote{\textcolor{blue}{Hans Schandlbauer}} habe dieſelbe Vermuthung ausgeſprochen. Jedenfalls dürfen Sie ſo vergnügt und
               unbeſorgt weiterleben als vorher. Allerdings ko{\geminationm}t’s {\pb}mir ſehr fraglich vor, daſs Ihre \textcolor{blue}{Mama}{}\ledrightnote{→\textcolor{blue}{Anna von Hofmannsthal}}{ }ſich entſchließen wird, Ihren \textcolor{blue}{Papa}{}\ledrightnote{→\textcolor{blue}{Hugo August von Hofmannsthal}} zu Ihnen nach \textcolor{pink}{Paris}{}\ledrightnote{\textcolor{pink}{Paris}} fahren zu laſſen; das iſt ganz begreiflich. Ich höre immer wieder, von
                  \textcolor{blue}{Richard}{}\ledrightnote{\textcolor{blue}{Richard Beer-Hofmann}} und von Ihrer \textcolor{blue}{Mama}{}\ledrightnote{→\textcolor{blue}{Anna von Hofmannsthal}}, dſs Sie ſich ſo wohl fühlen und mit Luſt
               arbeiten, und ſo freue ich mich nicht nur auf Sie ſondern auch auf das, was Sie
               mitbringen werden. Ich war auf meiner Reiſe eigentlich nur in den Stunden ziemlich
               gut dran, in denen ich geſchrieben habe; – {\pb}das Wetter
               war ſelten ſchön, nur in \textcolor{pink}{\textsc{Ragusa}}{}\ledrightnote{\textcolor{pink}{Dubrovnik}} 3 klare Tage, aber da wars für \textcolor{pink}{\textsc{Ragusa}}{}\ledrightnote{\textcolor{pink}{Dubrovnik}} und für Anfang April doch zu kühl. In \textcolor{pink}{Abbazia}{}\ledrightnote{\textcolor{pink}{Opatija}} hat es ununterbrochen gegoſſen; dort war ich viel mit
                  \textcolor{blue}{Georg Hirſchfeld}{}\ledrightnote{\textcolor{blue}{Georg Hirschfeld}} zuſammen, zu dem ich neue
               Sympathie gefaſſt habe. \textcolor{blue}{Elly}{}\ledrightnote{\textcolor{blue}{Elly Petersen}} liebe ich aber noch
               immer nicht. Es war mir auffallend, wie \label{K_L01031_1v}\edtext{viel ich auf meiner Reiſe}{\lemma{\textnormal{\emph{viel … Reiſe}}}\Cendnote{\textnormal{Er erwähnt mehrere davon im \emph{\textcolor{green}{Tagebuch}} (1. 4. 1900, 4. 4. 1900, 5. 4. 1900, 6. 4. 1900).}}}\label{K_L01031_1h} geträumt habe; ſo
               lebhaft und bewegt wie nie, und meine \textcolor{blue}{Todte}{}\ledrightnote{→\textcolor{blue}{Marie Reinhard}} iſt mir vier oder fünf Mal erſchienen. {\pb}Der ſonderbarſte von allen Träumen war der, dſs ich träumte, \label{K_L01031_2v}\edtext{ich hätte drei
                  Träume}{\lemma{\textnormal{\emph{ich hätte drei
                  Träume}}}\Cendnote{\textnormal{siehe A. S.: \emph{Tagebuch}, 6. 4. 1900}}}\label{K_L01031_2h} gehabt, die
               mir den Tod vorhergeſagt und erzählte jemandem dieſe 3 Träume, nach dem Aufwachen
               erinnerte ich mich nur an einen davon deutlich. – Ich bin noch immer an der langen
                  \textcolor{green}{Novelle}{}\ledrightnote{→\textcolor{green}{Frau Bertha Garlan. Roman}}, vor
                  Oſtern wird ſie doch fertig, dann dictir ich ſie; fange aber gleich
               was neues an, entweder eine kurze \textcolor{green}{Geſchichte}{}\ledrightnote{→\textcolor{green}{Ein Erfolg}} oder dieſes \textcolor{green}{Sommerſtück}{}\ledrightnote{→\textcolor{green}{Im Spiel der Sommerlüfte. In drei Aufzügen}}; – eigentlich hab ich ein Gefühl von
                  Unerſchöpf{\pb}lichkeit wie nie zuvor, aber es iſt mehr
               theoretiſch, – macht mich nicht beſonders glücklich. Ich empfinde meinen Verluſt
               ſchwerer und ſichrer als je.\pend
           \pstart
           Leben Sie wohl und ſchreiben Sie mir bald ein Wort.\pend
           \pstart Von Herzen Ihr \spacefill\mbox{Arthur.}\pend{}\pstart
           \noindent{}Ich hoffe Sie haben meinen Brief \introOben{}(aus \textcolor{pink}{Wien}{}\ledrightnote{\textcolor{pink}{Wien}})\introOben{} und auch die Karten aus \textcolor{pink}{Dalmatien}{}\ledrightnote{\textcolor{pink}{Dalmatien}} bekommen.\pend
           \pstart
           \textcolor{pink}{Wien}{}\ledrightnote{\textcolor{pink}{Wien}}, 9. 4. 900.\pend
           \endnumbering\briefempfaengerindex{Hofmannsthal, Hugo von@\textsc{Hofmannsthal, Hugo von}!zzzSchnitzler, Arthur@\emph{von Arthur Schnitzler}!1900-04-091@{9. 4. 1900}|)be}\mylabel{h}  \normalsize

\doendnotes{C}
\bigskip
\vfill

\clearpage

\footnotesize

\lohead{\textsc{register}}

% Definiere theindex-Environment komplett neu ohne reledmac
\makeatletter
\renewenvironment{theindex}{%
  \section*{\indexname}%
  \setlength{\parindent}{0pt}%
  \setlength{\parskip}{0pt plus 0.3pt}%
  \let\item\@idxitem
}{%
  \clearpage
}
\makeatother

\IfFileExists{\jobname-pw.ind}{\input{\jobname-pw.ind}}{}

\end{document}

      