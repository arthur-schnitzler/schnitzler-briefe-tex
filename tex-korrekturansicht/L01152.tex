%% latex-korrekturansicht-vorspann.tex
%% Vorspann für die Korrekturansicht.
%% Lädt die gemeinsame Datei latex-vorspann.tex mit gesetztem Schalter.

\newif\ifkorrekturansicht
\korrekturansichttrue

\input{../tex-inputs/latex-vorspann}


               \section[Arthur Schnitzler an Richard Beer-Hofmann, 26. 7. 1901]{ Arthur Schnitzler an Richard Beer-Hofmann, 26. 7. 1901}\nopagebreak\mylabel{v}\rehead{ }\normalsize\beginnumbering\briefempfaengerindex{Beer-Hofmann, Richard@\textsc{Beer-Hofmann, Richard}!zzzSchnitzler, Arthur@\emph{von Arthur Schnitzler}!1901-07-261@{26. 7. 1901}|(be} \toendnotes[C]{\smallbreak\pagebreak[2]} \Standort{YCGL, MSS 31.}
\physDesc{Brief, 1 Blatt, 2 Seiten, Umschlag
\newline{}Handschrift: 1) Bleistift, deutsche Kurrent\hspace{1em}2) schwarze Tinte, deutsche Kurrent (\noindent{}Umschlag)\hspace{1em}\newline{}Versand: 1) Stempel: »\nobreak{}\oindex{Vahrn@\textbf{Vahrn}, \emph{Besiedelter Ort (A.BSO)}|pwk}Vah\textcolor{gray}{r}n\nobreak{}«.  2) Stempel: »\nobreak{}\oindex{Poertschach@\textbf{Pörtschach}, \emph{https://www.geonames.org/ontologyP.PPL}|pwk}{\pb}Pörtschach am See, 27 7 01\nobreak{}«. }\toendnotes[C]{\smallbreak}\pstart{}{\pb} Herrn \textsc{Dr. Richard
                     Beer-Hofmann}\pend{}\pstart{}\textsc{\textcolor{pink}{Pörtschach (Wörthersee)}{}\ledrightnote{\textcolor{pink}{Pörtschach}}}\pend{}\pstart{}\textsc{\textcolor{pink}{Villa Arnstein}{}\ledrightnote{\textcolor{pink}{Villa Arnstein}}.}\pend{}{\bigskip}\pstart
           \raggedleft{}{\pb}\textcolor{pink}{\textsc{Vahrn}}{}\ledrightnote{\textcolor{pink}{Vahrn}}, 26. 7. 901\pend
           \pstart
           lieber Richard, haben Sie nicht einmal in \textcolor{pink}{Bozen}{}\ledrightnote{\textcolor{pink}{Bozen}} in der \textcolor{pink}{Stingl-Penſion}{}\ledrightnote{\textcolor{pink}{Hotel Stiegl}}
               gewohnt? Waren wir zwei nicht zuſa{\geminationm}en \label{K_L01152-1v}\edtext{dort}{\lemma{\textnormal{\emph{dort}}}\Cendnote{\textnormal{vgl. A. S.: \emph{Tagebuch}, 12. 8. 1899}}}\label{K_L01152-1h}? – Liegt etwas gegen dieſe Penſion vor? (Für \uline{ganz} kurzen Aufenthalt?) Bitte eine Zeile hieher {\pb}aber eiligſt.\pend
           \pstart
           Haben Sie \textcolor{blue}{Paul}{}\ledrightnote{\textcolor{blue}{Paul Goldmann}} ſchon geſehen?\pend
           \pstart
           Herzlichſt Ihr{\\[\baselineskip]}\spacefill\mbox{Arthur}\pend
           \leftskip=0em{}\endnumbering\briefempfaengerindex{Beer-Hofmann, Richard@\textsc{Beer-Hofmann, Richard}!zzzSchnitzler, Arthur@\emph{von Arthur Schnitzler}!1901-07-261@{26. 7. 1901}|)be}\mylabel{h}  \normalsize

\doendnotes{C}
\bigskip
\vfill

\clearpage

\footnotesize

\lohead{\textsc{register}}

% Definiere theindex-Environment komplett neu ohne reledmac
\makeatletter
\renewenvironment{theindex}{%
  \section*{\indexname}%
  \setlength{\parindent}{0pt}%
  \setlength{\parskip}{0pt plus 0.3pt}%
  \let\item\@idxitem
}{%
  \clearpage
}
\makeatother

\IfFileExists{\jobname-pw.ind}{\input{\jobname-pw.ind}}{}

\end{document}

      