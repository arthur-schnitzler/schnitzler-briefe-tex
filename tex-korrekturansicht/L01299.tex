%% latex-korrekturansicht-vorspann.tex
%% Vorspann für die Korrekturansicht.
%% Lädt die gemeinsame Datei latex-vorspann.tex mit gesetztem Schalter.

\newif\ifkorrekturansicht
\korrekturansichttrue

\input{../tex-inputs/latex-vorspann}


               \section[Arthur Schnitzler an Hermann Bahr, 24. 6. 1903]{ Arthur Schnitzler an Hermann Bahr, 24. 6. 1903}\nopagebreak\mylabel{v}\rehead{ }\normalsize\beginnumbering\briefempfaengerindex{Bahr, Hermann@\textsc{Bahr, Hermann}!zzzSchnitzler, Arthur@\emph{von Arthur Schnitzler}!1903-06-241@{24. 6. 1903}|(be} \toendnotes[C]{\smallbreak\pagebreak[2]} \Standort{TMW, HS AM 23356 Ba.}
\physDesc{Brief, 1 Blatt, 3 Seiten
\newline{}Handschrift: Bleistift, deutsche Kurrent\newline{}Ordnung: Lochung }\buchAbdrucke{\weitereDrucke{1) \emph{24. 6. 1903.} In: Arthur Schnitzler: \emph{The Letters of Arthur Schnitzler to Hermann Bahr}. Edited, annotated, and with an introduction, by Donald G.
                        Daviau. Chapel Hill: \emph{The University of North Carolina Press} 1978, S. 79 (University of North Carolina studies in the Germanic languages
                        and literatures, 89).} \weitereDrucke{2) Hermann Bahr, Arthur Schnitzler: \emph{Briefwechsel, Aufzeichnungen, Dokumente (1891–1931)}. Hg. Kurt Ifkovits und Martin Anton Müller. Göttingen: \emph{Wallstein} 2018, S. 267.} }\toendnotes[C]{\smallbreak}\pstart
           \raggedleft{}{\pb}24. \damage{6}. 903.\pend
           \pstart{}lieber Hermann,\pend\pstart
           Herr Dr \textcolor{blue}{\textsc{Stephan Epstein}}{}\ledrightnote{\textcolor{blue}{Stephan Epstein}} (der mit Hrn \textcolor{blue}{Lutz}{}\ledrightnote{\textcolor{blue}{Émile Lutz}} zuſammen \textcolor{green}{Kakadu}{}\ledrightnote{\textcolor{green}{Der grüne Kakadu. Groteske in einem Akt}} ins franzöſiſche überſetzt hat (für \textsc{\textcolor{blue}{Antoine}{}\ledrightnote{\textcolor{blue}{André Antoine}}})) \textsc{\textcolor{pink}{Paris, 78 rue de l’Assomption}{}\ledrightnote{\textcolor{pink}{rue de l’Assomption}}}, bittet mich dich zu fragen, ob du ſein Erſuchen betreffs Überſetzungsrechten
               des \textsc{\textcolor{green}{Apostel}{}\ledrightnote{\textcolor{green}{Der Apostel}}} ins franz. {\pb}\label{K_L01299_1v}\edtext{erhalten haſt}{\lemma{\textnormal{\emph{erhalten haſt}}}\Cendnote{\textnormal{nicht
                  überliefert}}}\label{K_L01299_1h}. Vielleicht biſt du ſo freundlich ihm
               direct zu antworten? –\pend
           \pstart
           – Mein \textcolor{blue}{Bruder}{}\ledrightnote{→\textcolor{blue}{Julius Schnitzler}} nennt mir als
               einen \introOben{}Arzt, der\introOben{} in \introOben{}der\introOben{} neulich von
               uns \label{K_L01299_2v}\edtext{beſprochenen Art ſeine Patienten zu
                  unterſuchen}{\lemma{\textnormal{\emph{beſprochenen … unterſuchen}}}\Cendnote{\textnormal{Vermutlich in Zusammenhang
                  mit der Abfassung von \emph{\textcolor{green}{Der Meister}} zu sehen,
                  dessen Hauptfigur ein Alternativmediziner ist.}}}\label{K_L01299_2h} pflegt: Dr \textcolor{blue}{\textsc{Kovacs}}{}\ledrightnote{\textcolor{blue}{Friedrich Kovacs}}. (Ich glaube er kennt ihn nicht persönlich.) –\pend
           \pstart
           {\pb}Herzlichen Gruſs.{\\[\baselineskip]}Dein{\\[\baselineskip]}\spacefill\mbox{A.}\pend
           \leftskip=0em{}\endnumbering\briefempfaengerindex{Bahr, Hermann@\textsc{Bahr, Hermann}!zzzSchnitzler, Arthur@\emph{von Arthur Schnitzler}!1903-06-241@{24. 6. 1903}|)be}\mylabel{h}  \normalsize

\doendnotes{C}
\bigskip
\vfill

\clearpage

\footnotesize

\lohead{\textsc{register}}

% Definiere theindex-Environment komplett neu ohne reledmac
\makeatletter
\renewenvironment{theindex}{%
  \section*{\indexname}%
  \setlength{\parindent}{0pt}%
  \setlength{\parskip}{0pt plus 0.3pt}%
  \let\item\@idxitem
}{%
  \clearpage
}
\makeatother

\IfFileExists{\jobname-pw.ind}{\input{\jobname-pw.ind}}{}

\end{document}

      