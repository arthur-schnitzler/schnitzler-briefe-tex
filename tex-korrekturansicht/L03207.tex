%% latex-korrekturansicht-vorspann.tex
%% Vorspann für die Korrekturansicht.
%% Lädt die gemeinsame Datei latex-vorspann.tex mit gesetztem Schalter.

\newif\ifkorrekturansicht
\korrekturansichttrue

\input{../tex-inputs/latex-vorspann}


\renewcommand{\erwaehntePersonen}{Personen: Eduard Bacher, Moriz Benedikt, Hugo Ganz}
\renewcommand{\erwaehnteInstitutionen}{Institutionen: Die Zeit. Wiener Wochenschrift, Neue Freie Presse}
\renewcommand{\erwaehnteOrte}{Orte: Berlin, Brühl, Dessauer Straße, Wien}
\renewcommand{\erwaehnteWerke}{Werke: Neue Freie Presse}
\section[ Paul Goldmann an Arthur Schnitzler, 5. 5. {[}1902{]}]{Paul Goldmann an Arthur Schnitzler, 5. 5. {[}1902{]}}
\nopagebreak\mylabel{v}
\rehead{ }\normalsize\beginnumbering\briefempfaengerindex{Schnitzler, Arthur@\textsc{Schnitzler, Arthur}!zzzGoldmann, Paul@\emph{von Paul Goldmann}!1902-05-051@{5. 5. {[}1902{]}}|(be}
\toendnotes[C]{\smallbreak\pagebreak[2]}\Standort{DLA, A:Schnitzler, HS.NZ85.1.3172.}
\physDesc{Brief, 1 Blatt, 1 Seite
\newline{}Handschrift: blaue Tinte, deutsche Kurrent
\newline{}Schnitzler: 1) mit Bleistift das Jahr »1902« und »\textcolor{gray}{I}« vermerkt  2) mit rotem Buntstift eine Unterstreichung}\toendnotes[C]{\smallbreak}
\pstart
           \noindent{}\raggedleft{}{\pb}\textcolor{pink}{\textcolor{gray}{\textbf{DESSAUERSTRASSE 19}}}{}\ledrightnote{\textcolor{pink}{Dessauer Straße}}\pend
           
\pstart
           \textcolor{pink}{Berlin}{}\ledrightnote{\textcolor{pink}{Berlin}}, 5. Mai.\pend
           
\pstart\center{}Mein lieber Freund,\pend
\pstart
           Ich möchte \introOben{}zu Pfingſten\introOben{} auf ein paar Tage \label{K_L03207-1v}\edtext{nach \textcolor{pink}{Wien}{}\ledrightnote{\textcolor{pink}{Wien}} kommen}{\lemma{\textnormal{\emph{nach Wien kommen}}}\Cendnote{\textnormal{siehe Paul Goldmann an Arthur Schnitzler, 2. 5. [1902]}}}\label{K_L03207-1h}, um mit den \textcolor{blue}{Herausgeber}{}\ledrightnote{{$\rightarrow$}\textcolor{blue}{Eduard Bacher}{\newline}{$\rightarrow$}\textcolor{blue}{Moriz Benedikt}}n der \textcolor{green}{N. Fr. Pr.}{}\ledrightnote{\textcolor{green}{Neue Freie Presse}} Einiges zu
               beſprechen. Schon deshalb kann ich nicht in der \textcolor{pink}{Brühl}{}\ledrightnote{\textcolor{pink}{Brühl}} wohnen. Wohnſt Du denn auch in der \label{K_L03207-2v}\edtext{\textcolor{pink}{Brühl}{}\ledrightnote{\textcolor{pink}{Brühl}}}{\lemma{\textnormal{\emph{Brühl}}}\Cendnote{\textnormal{\textcolor{blue}{Schnitzler} war zu dieser Zeit sehr häufig in
                  der \textcolor{pink}{Brühl}. Als \textcolor{blue}{Goldmann} ihn zu Pfingsten
                  in \textcolor{pink}{Wien} besuchte, waren sie auch gemeinsam
                  dort, jedenfalls am 19. 5. 1902 und am 25. 5. 1902, eventuell auch am 20. 5. 1902.}}}\label{K_L03207-2h}?\pend
           
\pstart
           \label{K_L03207-3v}\edtext{\textsc{\textcolor{blue}{Ganz}{}\ledrightnote{\textcolor{blue}{Hugo Ganz}}} geht zur »\textcolor{brown}{Zeit}{}\ledrightnote{\textcolor{brown}{Die Zeit. Wiener Wochenschrift}}«}{\lemma{\textnormal{\emph{Ganz geht zur »Zeit«}}}\Cendnote{\textnormal{\textcolor{blue}{Hugo Ganz}, der zuvor für die \emph{\textcolor{brown}{Neue Freie Presse}} arbeitete, hatte am 25. 4. 1902 einen fünfjährigen Vertrag als
                  Leitartikler, politischer Redakteur und Chefredakteurstellvertreter mit der \emph{\textcolor{brown}{Zeit}} unterzeichnet. Vgl. \emph{Das Recht. Volkstümliche Zeitschrift für österreichisches Rechtsleben.
                        Bde. 1–3}. \textcolor{pink}{Wien}{ }1902, S. 84.}}}\label{K_L03207-3h}.\pend
           
\pstart
           Viele treue Grüße! {\\[\baselineskip]}Dein {\\[\baselineskip]}\spacefill\mbox{Paul Goldm}\pend
           \leftskip=0em{}\endnumbering\briefempfaengerindex{Schnitzler, Arthur@\textsc{Schnitzler, Arthur}!zzzGoldmann, Paul@\emph{von Paul Goldmann}!1902-05-051@{5. 5. {[}1902{]}}|)be}\mylabel{h}
\begin{anhang}
\end{anhang}\normalsize

\doendnotes{C}
\bigskip
\vfill

\clearpage

\footnotesize

\lohead{\textsc{register}}

% Definiere theindex-Environment komplett neu ohne reledmac
\makeatletter
\renewenvironment{theindex}{%
  \section*{\indexname}%
  \setlength{\parindent}{0pt}%
  \setlength{\parskip}{0pt plus 0.3pt}%
  \let\item\@idxitem
}{%
  \clearpage
}
\makeatother

\IfFileExists{\jobname-pw.ind}{\input{\jobname-pw.ind}}{}

\end{document}

      