%% latex-korrekturansicht-vorspann.tex
%% Vorspann für die Korrekturansicht.
%% Lädt die gemeinsame Datei latex-vorspann.tex mit gesetztem Schalter.

\newif\ifkorrekturansicht
\korrekturansichttrue

\input{../tex-inputs/latex-vorspann}


\section[Stefan Zweig an Arthur Schnitzler, 12. 11. 1912]{L03639 Stefan Zweig an Arthur Schnitzler, 12. 11. 1912}
\nopagebreak\mylabel{L03639v}
\rehead{ }\normalsize\beginnumbering\briefempfaengerindex{Schnitzler, Arthur@\textsc{Schnitzler, Arthur}!zzzZweig, Stefan@\emph{von Stefan Zweig}!1912-11-121@{12. 11. 1912}|(be}
\toendnotes[C]{\smallbreak\pagebreak[2]}\Standort{CUL, Schnitzler, B 118.}
\physDesc{Brief, 1 Blatt, 4 Seiten, 2460 Zeichen
\newline{}Handschrift: lila Tinte, lateinische Kurrent
\newline{}Schnitzler: 1) mit Bleistift »\textsc{Zweig}«  2) mit rotem Buntstift zwei Unterstreichungen}
\buchAbdrucke{\weitereDrucke{1) Stefan Zweig: \emph{Briefwechsel mit Hermann Bahr, Sigmund Freud, Rainer Maria
                        Rilke und Arthur Schnitzler}. Frankfurt am Main: \emph{S. Fischer} 1987, S. 370–372.} \weitereDrucke{2) Stefan Zweig: \emph{Briefe. Bd. I: 1897–1914}. Frankfurt am Main: \emph{S. Fischer} 1995, S. 266–267.} }\toendnotes[C]{\smallbreak}
\pstart
           {\pb}\textcolor{gray}{\textbf{SZ}}\hfill \textcolor{gray}{\textbf{\textcolor{pink}{VIII. KOCHGASSE 8}\oindex{Kochgasse 8@\textbf{Kochgasse 8}, \emph{Wohngebäude (K.WHS)}|pw}{}\ledrightnote{\textcolor{pink}{Kochgasse 8}}}}\pend
           
\pstart
           \raggedleft{}\textcolor{gray}{\textbf{\textcolor{pink}{WIEN}\oindex{Wien@\textbf{Wien}, \emph{A.ADM2}|pw}{}\ledrightnote{\textcolor{pink}{Wien}},}}{ }12. Nov 12\pend
           {\vspace{1\baselineskip}}
\pstart{}Verehrter lieber Herr Doktor,\pend\vspace{0.5em}
\pstart
            mit ungemeiner Freude habe ich Ihren »\textcolor{green}{Professor
                  Bernhardi}\pwindex{Professor Bernhardi. Komoedie in fuenf Akten@\emph{Professor Bernhardi. Komödie in fünf Akten}|pw}{}\ledrightnote{\textcolor{green}{Professor Bernhardi. Komödie in fünf Akten}}« empfangen, mit Leidenschaft ihn sofort gelesen und eigentlich noch
               immer nicht aus der Hand gelegt, wiewohl ich schon längst bei der letzten Seite war
               und wieder mitten darin und wieder am Ende. Aber es ist ja unsere engste Welt, die
               sich hier auftut, weit freilich, unendlich weit, bis man den Himmel der grossen
               seelischen Gerechtigkeit über ihr mit allen guten Sternen sieht. Ich weiss nicht, ob
               ich Ihnen etwas Liebes damit sage, aber meine Empfindung will doch aufrichtig sein:
                  {\pb}ich spürte im ersten Lesen gar nicht
               mehr, dass dies ein Drama ist, ein Theaterstück, ein Kunstwerk, ich spürte nur
               lebendigstes Leben, das mich ergriff wie ein \label{K_L03639-1v}\edtext{\begin{otherlanguage}{french}fait divers\end{otherlanguage}}{\lemma{\textnormal{\emph{fait divers}}}\Cendnote{\textnormal{französisch: Nachricht der Rubrik
                  vermischte Meldungen}}}\label{K_L03639-1} der Zeitung, ein politischer Fall, spürte erst nur
               menschliche Empörung, Freude, Hass und Liebe. Dann später erst kam das Besinnen, dass
               dies Gestaltetes, Verwandeltes, Kunstwerk und nicht unmittelbares Leben ist. Und \strikeout{noch} immer habe ich noch keine Ruhe, um den \textcolor{green}{Bernhardi}\pwindex{Professor Bernhardi. Komoedie in fuenf Akten@\emph{Professor Bernhardi. Komödie in fünf Akten}|pw}{}\ledrightnote{\textcolor{green}{Professor Bernhardi. Komödie in fünf Akten}} als Kunstwerk oder gar auf den
               Theatererfolg hin betrachten zu können, ich bin zu passioniert davon, zu sehr mit
               Sympathie und Zorn gegen und für seine so herrlich lebendigen, so atemnahen Menschen.
                  \label{K_L03639-2v}\edtext{Nostra ipsissima res agitur}{\lemma{\textnormal{\emph{Nostra … agitur}}}\Cendnote{\textnormal{lateinisch: Es geht um unsere ureigene
                  Angelegenheit.}}}\label{K_L03639-2} – ich spür es zu sehr und {\pb}kann gar nicht recht heraus, mir's zu
               betrachten, so sehr bin ich darin. Jedesfalls: Sie haben nie eine grössere Scene
               geschrieben als die im vierten Akt zwischen dem Geistlichen und Bernhardi, es ist die
               Grossinquisitorscene Ihres dramatischen Werks, ganz weit blickend, hart und doch voll
               Güte, gross in jedem, im menschlichen, im künstlerischen Sinn. Nie waren Ihre
               Menschen lebendiger, nie Sie selbst dichterisch so weit, das spüre ich mit Beglückung
               und – verzeihen Sie! – mit Stolz, denn man darf doch niemandem versagen, auf die
               stolz zu sein, die man liebt.\pend
           
\pstart
           Dramaturgisch den \textcolor{green}{Bernhardi}\pwindex{Professor Bernhardi. Komoedie in fuenf Akten@\emph{Professor Bernhardi. Komödie in fünf Akten}|pw}{}\ledrightnote{\textcolor{green}{Professor Bernhardi. Komödie in fünf Akten}} zu betrachten,
               vermag ich noch nicht, ich sagte es ja, er ist noch zu heiß in mir. Aber ich weiß,
               solchen letzten menschlichen Entäußerungen kann nie {\pb}die Bewunderung fehlen. Ich weiß Ihr \textcolor{green}{Werk}\pwindex{Professor Bernhardi. Komoedie in fuenf Akten@\emph{Professor Bernhardi. Komödie in fünf Akten}|pwv}{}\ledrightnote{{$\rightarrow$}\emph{\textcolor{green}{Professor Bernhardi. Komödie in fünf Akten}}} wird wirken (im banalen
               bühnentechnischen Sinn und um wie viel mehr im höheren!), ich werde jedesfalls in \textcolor{pink}{Berlin}\oindex{Berlin@\textbf{Berlin}, \emph{P.PPLC}|pw}{}\ledrightnote{\textcolor{pink}{Berlin}} bei der \label{K_L03639-3v}\edtext{\textcolor{violet}{Première}\eventindex{Kleines Theater@\textbf{Kleines Theater}!Urauffuehrung von Professor Bernhardi, 28.11.1912@Uraufführung von Professor Bernhardi, 28.11.1912|pwv}{}\ledrightnote{{$\rightarrow$}\emph{\textcolor{violet}{Uraufführung von Professor Bernhardi, 28.11.1912}}}}{\lemma{\textnormal{\emph{Première}}}\Cendnote{\textnormal{Die \textcolor{violet}{Uraufführung}\eventindex{Kleines Theater@\textbf{Kleines Theater}!Urauffuehrung von Professor Bernhardi, 28.11.1912@Uraufführung von Professor Bernhardi, 28.11.1912|pwkv} von \emph{\textcolor{green}{Professor Bernhardi}\pwindex{Professor Bernhardi. Komoedie in fuenf Akten@\emph{Professor Bernhardi. Komödie in fünf Akten}|pwk}} fand am 28. 11. 1912 am \textcolor{pink}{Kleinen Theater}\oindex{Kleines Theater@\textbf{Kleines Theater}, \emph{Theater (K.THE)}|pwk} in \textcolor{pink}{Berlin}\oindex{Berlin@\textbf{Berlin}, \emph{P.PPLC}|pwk}
                  statt.}}}\label{K_L03639-3} sein, sobald ich das Datum erfahre und eine Einteilung zu treffen
               vermag. Denn ich \label{K_L03639-4v}\edtext{möchte nicht
                  fehlen}{\lemma{\textnormal{\emph{möchte nicht
                  fehlen}}}\Cendnote{\textnormal{Er dürfte den Plan, zur \textcolor{violet}{Uraufführung}\eventindex{Kleines Theater@\textbf{Kleines Theater}!Urauffuehrung von Professor Bernhardi, 28.11.1912@Uraufführung von Professor Bernhardi, 28.11.1912|pwkv} nach \textcolor{pink}{Berlin}\oindex{Berlin@\textbf{Berlin}, \emph{P.PPLC}|pwk} zu reisen, nicht umgesetzt haben.
                  Zumindest erwähnt \textcolor{blue}{Schnitzler} im \emph{\textcolor{green}{Tagebuch}\pwindex{Tagebuch@\emph{Tagebuch}|pwk}}-Eintrag zum 28. 11. 1912 seine
                  Anwesenheit nicht.}}}\label{K_L03639-4}, wenn ein solches Werk aus Buch zum Wort und vom Wort
               zur lebendigen Wirkung wird.\pend
           
\pstart
           Viele Grüsse Ihrer verehrten Frau \textcolor{blue}{Gemahlin}\pwindex{Schnitzler, Olga 17.01.1882 – 13.01.1970@\textsc{Schnitzler, Olga} (17.01.1882 – 13.01.1970), \emph{Schauspieler/Schauspielerin, Sänger/Sängerin}|pwv}{}\ledrightnote{{$\rightarrow$}\emph{\textcolor{blue}{Olga Schnitzler}}}! Innigst getreu und mit frohem Glückwunsch{\\[\baselineskip]}Ihr {\\[\baselineskip]}\spacefill\mbox{Stefan Zweig}\pend
           \leftskip=0em{}\selectlanguage{ngerman}\endnumbering\briefempfaengerindex{Schnitzler, Arthur@\textsc{Schnitzler, Arthur}!zzzZweig, Stefan@\emph{von Stefan Zweig}!1912-11-121@{12. 11. 1912}|)be}\mylabel{L03639h}  \normalsize

\doendnotes{C}
\bigskip
\vfill

\clearpage

\footnotesize

\lohead{\textsc{register}}

% Definiere theindex-Environment komplett neu ohne reledmac
\makeatletter
\renewenvironment{theindex}{%
  \section*{\indexname}%
  \setlength{\parindent}{0pt}%
  \setlength{\parskip}{0pt plus 0.3pt}%
  \let\item\@idxitem
}{%
  \clearpage
}
\makeatother

\IfFileExists{\jobname-pw.ind}{\input{\jobname-pw.ind}}{}

\end{document}

      