%% latex-korrekturansicht-vorspann.tex
%% Vorspann für die Korrekturansicht.
%% Lädt die gemeinsame Datei latex-vorspann.tex mit gesetztem Schalter.

\newif\ifkorrekturansicht
\korrekturansichttrue

\input{../tex-inputs/latex-vorspann}


               \section[Arthur Schnitzler an Richard Beer-Hofmann, 28. 11. 1904]{ Arthur Schnitzler an Richard Beer-Hofmann, 28. 11. 1904}\nopagebreak\mylabel{v}\rehead{ }\normalsize\beginnumbering\briefempfaengerindex{Beer-Hofmann, Richard@\textsc{Beer-Hofmann, Richard}!zzzSchnitzler, Arthur@\emph{von Arthur Schnitzler}!1904-11-281@{28. 11. 1904}|(be} \toendnotes[C]{\smallbreak\pagebreak[2]} \Standort{YCGL, MSS 31.}
\physDesc{Brief, 1 Blatt, 3 Seiten, Umschlag
\newline{}Handschrift: Bleistift, deutsche Kurrent\newline{}Versand: 1) Stempel: »\nobreak{}5\nobreak{}«.  2) Stempel: »\nobreak{}{\pb}Bestellt vom
                                          {[}Po{]}stamte 6\nobreak{}«. }\buchAbdrucke{\weitereDrucke{Arthur Schnitzler, Richard Beer-Hofmann: \emph{Briefwechsel 1891–1931}. Hg. Konstanze Fliedl. Wien, Zürich: \emph{Europaverlag} 1992, S. 170–171.} }\toendnotes[C]{\smallbreak}\pstart{}{\pb}\textsc{Herrn Dr. Richard}\pend{}\pstart{}\textsc{Beer-Hofmann}\pend{}\pstart{}\textsc{\textcolor{pink}{Berlin}{}\ledrightnote{\textcolor{pink}{Berlin}}}\pend{}\pstart{}\textsc{\textcolor{pink}{Hotel Bristol}{}\ledrightnote{\textcolor{pink}{Hotel Bristol}}}\pend{}{\bigskip}\pstart
           \raggedleft{}{\pb}\textcolor{pink}{Wien}{}\ledrightnote{\textcolor{pink}{Wien}}, 28. 11. 904\pend
           \pstart{}lieber Richard,\pend\pstart
           ich bitte Sie ſehr \textcolor{blue}{Reinhardt}{}\ledrightnote{\textcolor{blue}{Max Reinhardt}} nochmals in meinem
               Namen dringend zu erſuchen, er möge, ob nun \textcolor{green}{\textsc{Delorme}}{}\ledrightnote{\textcolor{green}{Das Haus Delorme. Eine Familienszene}} freigegeben oder ob es definitiv verboten wird, \uline{abſolut nichts} in die Zeitung geben und überhaupt \uline{nichts verfügen}, ohne ſich vorher mit mir in Verbin{\pb}dung zu ſetzen. –\pend
           \pstart
           Gern würde ich Ihre Meinung wiſſen, ob Sie es nicht auch für opportun hielten, ſelbſt
               im Fall eines Erlaubtwerdens, die \strikeout{\textcolor{gray}{Geſchichte}} ev. \textcolor{green}{Aufführung}{}\ledrightnote{→\textcolor{green}{Das Haus Delorme. Eine Familienszene}}
               hinauszuſchieben. An dieſer Überfracht von unfreiwilliger Reclame und geſpannten
               Erwartungen müsste meiner {\pb}Empfindung nach auch ein
               ſtärkeres Stück zu Grunde gehen.\pend
           \pstart
           Theilen Sie mir mit wie es Ihnen und Ihren \textcolor{green}{Proben}{}\ledrightnote{→\textcolor{green}{Der Graf von Charolais. Ein Trauerspiel}} geht, grüßen Sie mit mehrerem oder minderem \textsc{Empressement}.\pend
           \pstart
           Alles gute an \textcolor{blue}{\textsc{Reinhardt}}{}\ledrightnote{\textcolor{blue}{Max Reinhardt}} u noch etwas mehr an Sie.\pend
           \pstart
           Herzlichst Ihr{\\[\baselineskip]}\spacefill\mbox{A.}\pend
           \leftskip=0em{}\endnumbering\briefempfaengerindex{Beer-Hofmann, Richard@\textsc{Beer-Hofmann, Richard}!zzzSchnitzler, Arthur@\emph{von Arthur Schnitzler}!1904-11-281@{28. 11. 1904}|)be}\mylabel{h}  \normalsize

\doendnotes{C}
\bigskip
\vfill

\clearpage

\footnotesize

\lohead{\textsc{register}}

% Definiere theindex-Environment komplett neu ohne reledmac
\makeatletter
\renewenvironment{theindex}{%
  \section*{\indexname}%
  \setlength{\parindent}{0pt}%
  \setlength{\parskip}{0pt plus 0.3pt}%
  \let\item\@idxitem
}{%
  \clearpage
}
\makeatother

\IfFileExists{\jobname-pw.ind}{\input{\jobname-pw.ind}}{}

\end{document}

      