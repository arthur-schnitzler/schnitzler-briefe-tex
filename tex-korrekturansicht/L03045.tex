%% latex-korrekturansicht-vorspann.tex
%% Vorspann für die Korrekturansicht.
%% Lädt die gemeinsame Datei latex-vorspann.tex mit gesetztem Schalter.

\newif\ifkorrekturansicht
\korrekturansichttrue

\input{../tex-inputs/latex-vorspann}


\renewcommand{\erwaehntePersonen}{Personen: Felix Salten}
\renewcommand{\erwaehnteInstitutionen}{Institutionen: Paul Zsolnay Verlag}
\renewcommand{\erwaehnteOrte}{Orte: Berlin, Leipzig, Wien}
\renewcommand{\erwaehnteWerke}{Werke: Neue Menschen auf alter Erde. Eine Palästinafahrt, Tagebuch}
\section[ Felix Salten: Widmungsexemplar Neue Menschen auf alter Erde für Arthur Schnitzler, 30. 4. 1925]{Felix Salten: Widmungsexemplar Neue Menschen auf alter Erde für Arthur
               Schnitzler, 30. 4. 1925}
\nopagebreak\mylabel{v}
\rehead{ }\normalsize\beginnumbering\briefempfaengerindex{Schnitzler, Arthur@\textsc{Schnitzler, Arthur}!zzzSalten, Felix@\emph{von Felix Salten}!1925-04-301@{30. 4. 1925}|(be}
\toendnotes[C]{\smallbreak\pagebreak[2]}\Standort{DLA, G:Schnitzler, Arthur (Sammlung Heinrich Schnitzler).}
\physDesc{Widmung am Schmutztitel, 48 Zeichen
\newline{}Handschrift: schwarze Tinte, lateinische Kurrent}\toendnotes[C]{\smallbreak}
\pstart
           \noindent{}\centering{}{\pb}\textcolor{gray}{\textbf{\so{FELIX SALTEN}}}\pend
           
\pstart
           \noindent{}\centering{}\textcolor{gray}{\textbf{\textcolor{green}{NEUE MENSCHEN}{}\ledrightnote{\textcolor{green}{Neue Menschen auf alter Erde. Eine Palästinafahrt}}}}\pend
           
\pstart
           \noindent{}\centering{}\textcolor{gray}{\textbf{\textcolor{green}{AUF ALTER ERDE}{}\ledrightnote{\textcolor{green}{Neue Menschen auf alter Erde. Eine Palästinafahrt}}}}\pend
           {\bigskip}
\pstart
           \noindent{}Arthur Schnitzler {\\}herzlich{\\}\spacefill\mbox{Felix Salten}\pend
           
\pstart
           \label{K_L03045-1v}\edtext{30⋅4⋅25}{\lemma{\textnormal{\emph{30⋅4⋅25}}}\Cendnote{\textnormal{\textcolor{blue}{Schnitzler}
                     kannte das Buch in Auszügen seit ein paar Wochen, vgl. A. S.: \emph{Tagebuch}, 17. 2. 1925.
                     Trotz seiner Danksagung an \textcolor{blue}{Salten} vom 6. 5. 1925
                     las er das Buch erst am 25. 7. 1925 und beurteilte im \emph{\textcolor{green}{Tagebuch}}
                  negativ.}}}\label{K_L03045-1h}\pend
           {\bigskip}
\pstart
           \noindent{}\centering{}{\pb}\textcolor{gray}{\textbf{\so{FELIX SALTEN}}}\pend
           
\pstart
           \noindent{}\centering{}\textcolor{gray}{\textbf{\textcolor{green}{NEUE MENSCHEN}{}\ledrightnote{\textcolor{green}{Neue Menschen auf alter Erde. Eine Palästinafahrt}}}}\pend
           
\pstart
           \noindent{}\centering{}\textcolor{gray}{\textbf{\textcolor{green}{AUF ALTER ERDE}{}\ledrightnote{\textcolor{green}{Neue Menschen auf alter Erde. Eine Palästinafahrt}}}}\pend
           {\bigskip}
\pstart
           \noindent{}\centering{}\textcolor{gray}{\textbf{EINE PALÄSTINAFAHRT}}\pend
           {\bigskip}
\pstart
           \noindent{}\centering{}\textcolor{gray}{\textbf{\so{1925}}}\pend
           
\pstart
           \noindent{}\centering{}\textcolor{gray}{\textbf{\textcolor{brown}{\so{PAUL ZSOLNAY VERLAG}}{}\ledrightnote{\textcolor{brown}{Paul Zsolnay Verlag}}}}\pend
           
\pstart
           \noindent{}\centering{}\textcolor{gray}{\textbf{\textcolor{pink}{\so{BERLIN}}{}\ledrightnote{\textcolor{pink}{Berlin}}{ }–{ }\textcolor{pink}{\so{WIEN}}{}\ledrightnote{\textcolor{pink}{Wien}}{ }–{ }\textcolor{pink}{\so{LEIPZIG}}{}\ledrightnote{\textcolor{pink}{Leipzig}}}}\pend
           \endnumbering\briefempfaengerindex{Schnitzler, Arthur@\textsc{Schnitzler, Arthur}!zzzSalten, Felix@\emph{von Felix Salten}!1925-04-301@{30. 4. 1925}|)be}\mylabel{h}  \normalsize

\doendnotes{C}
\bigskip
\vfill

\clearpage

\footnotesize

\lohead{\textsc{register}}

% Definiere theindex-Environment komplett neu ohne reledmac
\makeatletter
\renewenvironment{theindex}{%
  \section*{\indexname}%
  \setlength{\parindent}{0pt}%
  \setlength{\parskip}{0pt plus 0.3pt}%
  \let\item\@idxitem
}{%
  \clearpage
}
\makeatother

\IfFileExists{\jobname-pw.ind}{\input{\jobname-pw.ind}}{}

\end{document}

      