%% latex-korrekturansicht-vorspann.tex
%% Vorspann für die Korrekturansicht.
%% Lädt die gemeinsame Datei latex-vorspann.tex mit gesetztem Schalter.

\newif\ifkorrekturansicht
\korrekturansichttrue

\input{../tex-inputs/latex-vorspann}


\renewcommand{\erwaehntePersonen}{Personen: Felix Salten}
\renewcommand{\erwaehnteOrte}{Orte: Edmund-Weiß-Gasse 7, Semmering, Wien}
\renewcommand{\erwaehnteWerke}{Werke: Zum großen Wurstel. Burleske in einem Akt}
\section[ Arthur Schnitzler an Felix Salten, 29. 4. 1905]{Arthur Schnitzler an Felix Salten, 29. 4. 1905}
\nopagebreak\mylabel{v}
\rehead{ }\normalsize\beginnumbering\briefempfaengerindex{Salten, Felix@\textsc{Salten, Felix}!zzzSchnitzler, Arthur@\emph{von Arthur Schnitzler}!1905-04-291@{29. 4. 1905}|(be}
\toendnotes[C]{\smallbreak\pagebreak[2]}\Standort{Wienbibliothek im Rathaus, ZPH 1681, 2.1.516.}
\physDesc{Karte, 246 Zeichen
\newline{}Handschrift: schwarze Tinte, deutsche Kurrent
\newline{}Ordnung: mit Bleistift von unbekannter Hand Nummerierung der Blätter des Konvoluts:
                                    »18« }\toendnotes[C]{\smallbreak}
\pstart
           \noindent{}\textcolor{gray}{\textbf{{\pb}Dr. Arthur Schnitzler}}\hfill 29. 4. 905.\pend
           
\pstart
           \textcolor{gray}{\textbf{\textcolor{pink}{Wien, XVIII. Spoettelgasse 7}{}\ledrightnote{\textcolor{pink}{Edmund-Weiß-Gasse 7}}.}}\pend
           
\pstart
           lieber, ich wiederhole meine Bitte, mir freundlichſt von der \label{K_L02999-1v}\edtext{\textcolor{green}{Oſternu{\geminationm}er}{}\ledrightnote{{$\rightarrow$}\textcolor{green}{Zum großen Wurstel. Burleske in einem Akt}} 12 Exemplare}{\lemma{\textnormal{\emph{Oſternummer 12 Exemplare}}}\Cendnote{\textnormal{siehe Felix Salten an Arthur Schnitzler, 11. 4. 1905}}}\label{K_L02999-1h} ſchicken zu laſſen. Es wäre mir ein wirklicher Gefallen.\pend
           
\pstart
           Morgen fahren wir \label{K_L02999-2v}\edtext{auf ein paar Tage}{\lemma{\textnormal{\emph{auf ein paar Tage}}}\Cendnote{\textnormal{Sie blieben dort bis zum 6. 5. 1905. Kurz danach, am 7. 5. 1905, sahen sich \textcolor{blue}{Salten} und \textcolor{blue}{Schnitzler}
                  wieder.}}}\label{K_L02999-2h}{ }{\pb}auf den \textcolor{pink}{Se{\geminationm}ering}{}\ledrightnote{\textcolor{pink}{Semmering}}. Hoffentlich auf ſehr baldigs
               Wiederſehen.\pend
           
\pstart
           Ihr {\\[\baselineskip]}\spacefill\mbox{A.}\pend
           \leftskip=0em{}\endnumbering\briefempfaengerindex{Salten, Felix@\textsc{Salten, Felix}!zzzSchnitzler, Arthur@\emph{von Arthur Schnitzler}!1905-04-291@{29. 4. 1905}|)be}\mylabel{h}  \normalsize

\doendnotes{C}
\bigskip
\vfill

\clearpage

\footnotesize

\lohead{\textsc{register}}

% Definiere theindex-Environment komplett neu ohne reledmac
\makeatletter
\renewenvironment{theindex}{%
  \section*{\indexname}%
  \setlength{\parindent}{0pt}%
  \setlength{\parskip}{0pt plus 0.3pt}%
  \let\item\@idxitem
}{%
  \clearpage
}
\makeatother

\IfFileExists{\jobname-pw.ind}{\input{\jobname-pw.ind}}{}

\end{document}

      