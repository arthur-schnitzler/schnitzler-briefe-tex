%% latex-korrekturansicht-vorspann.tex
%% Vorspann für die Korrekturansicht.
%% Lädt die gemeinsame Datei latex-vorspann.tex mit gesetztem Schalter.

\newif\ifkorrekturansicht
\korrekturansichttrue

\input{../tex-inputs/latex-vorspann}


               \section[Arthur Schnitzler an Auguste Hauschner, 29. 6. 1908]{ Arthur Schnitzler an Auguste Hauschner, 29. 6. 1908}\nopagebreak\mylabel{v}\rehead{ }\normalsize\beginnumbering\briefempfaengerindex{Hauschner, Auguste@\textsc{Hauschner, Auguste}!zzzSchnitzler, Arthur@\emph{von Arthur Schnitzler}!1908-06-291@{29. 6. 1908}|(be} \toendnotes[C]{\smallbreak\pagebreak[2]} \Standort{Staatsbibliothek Berlin – Preußischer Kulturbesitz, Handschriftenabteilung, Nachlass Auguste Hauschner.}
\physDesc{Briefkarte
\newline{}Handschrift: schwarze Tinte, lateinische Kurrent}\toendnotes[C]{\smallbreak}\pstart
           \noindent{}{\pb}\textcolor{gray}{\textbf{Dr. Arthur Schnitzler}}\hfill \textcolor{pink}{Seis am Schlern}{}\ledrightnote{\textcolor{pink}{Seis am Schlern}},\pend
           \pstart
           \textcolor{gray}{\textbf{\textcolor{pink}{Wien, XVIII. Spoettelgasse 7}{}\ledrightnote{\textcolor{pink}{Edmund-Weiß-Gasse}}.}}\hfill 29. 6. 08\pend
           \pstart
           verehrte Frau, Ihr Brief ist mir hieher nachgereist – dass er mich
               sehr gefreut hat, kö{\geminationn}en Sie sich wohl denken. Nun hab
               ich mir auch Ihr \textcolor{green}{Buch}{}\ledrightnote{→\textcolor{green}{Die Familie Lowositz. Roman}} aus \textcolor{pink}{Wien}{}\ledrightnote{\textcolor{pink}{Wien}} herschicken lassen und bin sehr begierig Ihre
                  Beka{\geminationn}tschaft zu machen. De{\geminationn} ich kenne noch gar nichts von Ihnen – zu meinen
               Vorsätzen {\pb}gehört schon lange Zeit
                  »\textcolor{green}{Kunst}{}\ledrightnote{\textcolor{green}{Kunst. Roman}}« – über das mir kluge Leute das beste zu
               sagen wußten. Seien Sie herzlichst bedankt und gegrüßt!\pend
           \pstart
           Ihr ergebener{\\[\baselineskip]}\spacefill\mbox{Arthur Schnitzler}\pend
           \leftskip=0em{}\endnumbering\briefempfaengerindex{Hauschner, Auguste@\textsc{Hauschner, Auguste}!zzzSchnitzler, Arthur@\emph{von Arthur Schnitzler}!1908-06-291@{29. 6. 1908}|)be}\mylabel{h}  \normalsize

\doendnotes{C}
\bigskip
\vfill

\clearpage

\footnotesize

\lohead{\textsc{register}}

% Definiere theindex-Environment komplett neu ohne reledmac
\makeatletter
\renewenvironment{theindex}{%
  \section*{\indexname}%
  \setlength{\parindent}{0pt}%
  \setlength{\parskip}{0pt plus 0.3pt}%
  \let\item\@idxitem
}{%
  \clearpage
}
\makeatother

\IfFileExists{\jobname-pw.ind}{\input{\jobname-pw.ind}}{}

\end{document}

      