%% latex-korrekturansicht-vorspann.tex
%% Vorspann für die Korrekturansicht.
%% Lädt die gemeinsame Datei latex-vorspann.tex mit gesetztem Schalter.

\newif\ifkorrekturansicht
\korrekturansichttrue

\input{../tex-inputs/latex-vorspann}


\renewcommand{\erwaehntePersonen}{Personen: Olga Schnitzler, Stefan Zweig}
\renewcommand{\erwaehnteOrte}{Orte: Kochgasse 8, Paschinger Schlössl, Salzburg, Wien}
\renewcommand{\erwaehnteWerke}{}
\section[Stefan Zweig an Arthur Schnitzler, 14. 4. 1919]{Stefan Zweig an Arthur Schnitzler, 14. 4. 1919}
\nopagebreak\mylabel{v}
\rehead{ }\normalsize\beginnumbering\briefempfaengerindex{Schnitzler, Arthur@\textsc{Schnitzler, Arthur}!zzzZweig, Stefan@\emph{von Stefan Zweig}!1919-04-141@{14. 4. 1919}|(be}
\toendnotes[C]{\smallbreak\pagebreak[2]}\Standort{CUL, Schnitzler, B 118.}
\physDesc{Briefkarte, 1 Blatt, 1 Seite, 408 Zeichen
\newline{}Handschrift: lila Tinte, lateinische Kurrent}\toendnotes[C]{\smallbreak}
\pstart
           {\pb}\textcolor{gray}{\textbf{SZ}}\hfill 14. April 1919\pend
           
\pstart
           \raggedleft{}\textcolor{gray}{\textbf{\textcolor{pink}{VIII. KOCHGASSE 8}{}\ledrightnote{\textcolor{pink}{Kochgasse 8}}.}}\pend
           
\pstart
           \raggedleft{}Tel. 36 404\pend
           
\pstart
           Lieber verehrter Herr Doktor, meines Bleibens in \textcolor{pink}{Wien}{}\ledrightnote{\textcolor{pink}{Wien}} wird nicht lange sein: in etwa 18 Tagen gehe ich, und
               diesmal \label{K_L03664-1v}\edtext{wohl für immer}{\lemma{\textnormal{\emph{wohl für immer}}}\Cendnote{\textnormal{Am 29. 4. 1919
                  verlegte \textcolor{blue}{Zweig} seinen Wohnsitz dauerhaft in
                  das \textcolor{pink}{Paschinger Schlössl} in \textcolor{pink}{Salzburg}.}}}\label{K_L03664-1h}, fort. Gerne hätte ich gerade Sie, den
               wandellos Verehrten, zuvor noch \label{K_L03664-2v}\edtext{gesehen}{\lemma{\textnormal{\emph{gesehen}}}\Cendnote{\textnormal{Das gewünschte Treffen fand
                  am 22. 4. 1919
                  statt.}}}\label{K_L03664-2h} und bitte Sie um Wort und Erlaubnis, wann ich zu Ihnen kommen darf.
               Mit vielen Empfehlungen Ihrer verehrten Frau \textcolor{blue}{Gemahlin}{}\ledrightnote{{$\rightarrow$}\textcolor{blue}{Olga Schnitzler}} und den herzlichsten Grüssen Ihr Treu ergebener \pend
           \pstart \spacefill\mbox{Stefan Zweig}\pend{}\endnumbering\briefempfaengerindex{Schnitzler, Arthur@\textsc{Schnitzler, Arthur}!zzzZweig, Stefan@\emph{von Stefan Zweig}!1919-04-141@{14. 4. 1919}|)be}\mylabel{h}
\begin{anhang}
\end{anhang}\normalsize

\doendnotes{C}
\bigskip
\vfill

\clearpage

\footnotesize

\lohead{\textsc{register}}

% Definiere theindex-Environment komplett neu ohne reledmac
\makeatletter
\renewenvironment{theindex}{%
  \section*{\indexname}%
  \setlength{\parindent}{0pt}%
  \setlength{\parskip}{0pt plus 0.3pt}%
  \let\item\@idxitem
}{%
  \clearpage
}
\makeatother

\IfFileExists{\jobname-pw.ind}{\input{\jobname-pw.ind}}{}

\end{document}

      