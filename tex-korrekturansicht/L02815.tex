%% latex-korrekturansicht-vorspann.tex
%% Vorspann für die Korrekturansicht.
%% Lädt die gemeinsame Datei latex-vorspann.tex mit gesetztem Schalter.

\newif\ifkorrekturansicht
\korrekturansichttrue

\input{../tex-inputs/latex-vorspann}


               \section[ Paul Goldmann an Arthur Schnitzler, 18. 6. {[}1897{]}]{Paul Goldmann an Arthur Schnitzler, 18. 6. {[}1897{]}}\nopagebreak\mylabel{v}\rehead{ }\normalsize\beginnumbering\briefempfaengerindex{Schnitzler, Arthur@\textsc{Schnitzler, Arthur}!zzzGoldmann, Paul@\emph{von Paul Goldmann}!1897-06-181@{18. 6. {[}1897{]}}|(be} \toendnotes[C]{\smallbreak\pagebreak[2]} \Standort{DLA, A:Schnitzler, HS.NZ85.1.3167.}
\physDesc{Brief, 1 Blatt, 2 Seiten
\newline{}Handschrift: blaue Tinte, deutsche Kurrent\newline{}Beilage: aufgeklebtes Brieffragment, Handschrift \textcolor{blue}{Clementine Goldmann}, blaue Tinte, deutsche
                                 Kurrentschrift. Das abschließende Anführungszeichen wurde von Paul
                                 Goldmann ergänzt. 
\newline{}Schnitzler: 1) mit Bleistift das Jahr »97« vermerkt 2) mit rotem Buntstift eine Unterstreichung}\toendnotes[C]{\smallbreak}\pstart
           \noindent{}{\pb}\textcolor{gray}{\textbf{\textbf{\textcolor{brown}{Frankfurter Zeitung}{}\ledrightnote{\textcolor{brown}{Frankfurter Zeitung}}}}}\pend
           \pstart
           \textcolor{gray}{\textbf{(\textcolor{brown}{\begin{otherlanguage}{french}Gazette de Francfort\end{otherlanguage}}{}\ledrightnote{\textcolor{brown}{Frankfurter Zeitung}}).}}\pend
           \pstart
           \textcolor{gray}{\textbf{\textbf{\begin{otherlanguage}{french}Fondateur M.\end{otherlanguage}{ }\textcolor{blue}{L. Sonnemann}{}\ledrightnote{\textcolor{blue}{Leopold Sonnemann}}.}}}\pend
           \pstart
           \begin{otherlanguage}{french}\textcolor{gray}{\textbf{Journal politique, financier,}}\end{otherlanguage}\pend
           \pstart
           \begin{otherlanguage}{french}\textcolor{gray}{\textbf{commercial et littéraire.}}\end{otherlanguage}\pend
           \pstart
           \begin{otherlanguage}{french}\textcolor{gray}{\textbf{\textbf{Paraissant trois fois par jour.}}}\end{otherlanguage}\hfill \textsc{\textcolor{pink}{Paris}{}\ledrightnote{\textcolor{pink}{Paris}}}, 18. Juni.\pend
           \pstart
           \begin{otherlanguage}{french}\textcolor{gray}{\textbf{\textbf{Bureau à \textcolor{pink}{Paris}{}\ledrightnote{\textcolor{pink}{Paris}}}}}\end{otherlanguage}\pend
           \pstart
           \begin{otherlanguage}{french}\textcolor{gray}{\textbf{\textbf{\textcolor{pink}{10 Rue de la Bourse}{}\ledrightnote{\textcolor{pink}{rue de la Bourse}}.}}}\end{otherlanguage}\pend
           \pstart\center{}Mein lieber Freund,\pend\pstart
           Das \label{K_L02815-11v}\edtext{Manuſkript des \textsc{\textcolor{blue}{Nansen}{}\ledrightnote{\textcolor{blue}{Peter Nansen}}}-\textcolor{green}{Artikel}{}\ledrightnote{→\textcolor{green}{?? [Artikel von Peter Nansen, Mai/Juni 1897]}}s}{\lemma{\textnormal{\emph{Manuſkript des Nansen-Artikels}}}\Cendnote{\textnormal{Auch wenn es letztlich nicht zu klären
                  ist, von welchem Text die Rede ist, so dürfte der Umstand, dass \textcolor{blue}{Clementine Goldmann} im Besitz des Textes war und ihn an
                  ihren Bruder \textcolor{blue}{Fedor Mamroth} weiterreichte, so
                  zu lesen sein, dass es sich nicht um einen bei der \emph{\textcolor{brown}{Frankfurter Zeitung}} eingereichten Beitrag handelte, da sie ihn sonst
                  »zurückgegeben« hätte. Weiters deutet das Wort »damals« darauf hin,
                  dass es sich schon vor einiger Zeit abgespielt hatte und also kein neuer Text \textcolor{blue}{Nansen}s gemeint ist. Eventuell dürfte
                  schlicht vom Manuskript einer deutschen Übersetzung des Aufsatzes \textcolor{blue}{–n–} [=\textcolor{blue}{Peter Nansen}]: \emph{\textcolor{green}{Arthur
                        Schnitzler. »Elskovsleg«s Forfatter}}. In: \emph{\textcolor{green}{Politiken}}, Nr. 68, 9. 3. 1897, S. 1 die Rede sein.}}}\label{K_L02815-11h} ſcheint leider futſch
               zu ſein. Meine \textcolor{blue}{Mutter}{}\ledrightnote{→\textcolor{blue}{Clementine Goldmann}}
               ſchreibt mir:\pend
           \pstart
           »An \textsc{Dr. Schnitzler} konnte ich leider das\pend
           \pstart
           {[}hs. Clementine Goldmann:{]} \textsc{\textcolor{blue}{Nansen}{}\ledrightnote{\textcolor{blue}{Peter Nansen}}} Manuskript nicht ſchicken; ich gab es damals Onkel \textcolor{blue}{Fedor}{}\ledrightnote{\textcolor{blue}{Fedor Mamroth}}, ohne es zurück zubekommen.–«\pend
           \pstart
           {[}hs. Paul Goldmann:{]} Was alſo thun?\pend
           \pstart
           Suche Dich doch ſo einzurichten, daß Du am 8. Auguſt
               nach \label{K_L02815-2v}\edtext{\textsc{\textcolor{pink}{Bayreuth}{}\ledrightnote{\textcolor{pink}{Bayreuth}}}}{\lemma{\textnormal{\emph{Bayreuth}}}\Cendnote{\textnormal{siehe Paul Goldmann an Arthur Schnitzler, 15. 6. [1897]}}}\label{K_L02815-2h} gehſt. Du, der Du nicht Berufsſklave biſt, wie ich, kannſt Dir doch eher
               Deine {\pb}Zeit eintheilen.\pend
           \pstart
           Haſt Du dieſe 
               Beſtie, den \textsc{\textcolor{blue}{Graf}{}\ledrightnote{→\textcolor{blue}{Max Graf}}},
                geſehen? Hat er irgendwelchen
               Geſtank in Bezug auf mich verurſacht?\pend
           \pstart
           Wie geht es ſonſt Dir und \textcolor{blue}{ihr}{}\ledrightnote{→\textcolor{blue}{Marie Reinhard}}?\pend
           \pstart
           Schreib’ recht bald\textcolor{gray}{!}\pend
           \pstart
           Ich begrüße Dich von Herzen {\\[\baselineskip]}Dein {\\[\baselineskip]}\spacefill\mbox{Paul Goldm}\pend
           \leftskip=0em{}\endnumbering\briefempfaengerindex{Schnitzler, Arthur@\textsc{Schnitzler, Arthur}!zzzGoldmann, Paul@\emph{von Paul Goldmann}!1897-06-181@{18. 6. {[}1897{]}}|)be}\mylabel{h}  \normalsize

\doendnotes{C}
\bigskip
\vfill

\clearpage

\footnotesize

\lohead{\textsc{register}}

% Definiere theindex-Environment komplett neu ohne reledmac
\makeatletter
\renewenvironment{theindex}{%
  \section*{\indexname}%
  \setlength{\parindent}{0pt}%
  \setlength{\parskip}{0pt plus 0.3pt}%
  \let\item\@idxitem
}{%
  \clearpage
}
\makeatother

\IfFileExists{\jobname-pw.ind}{\input{\jobname-pw.ind}}{}

\end{document}

      