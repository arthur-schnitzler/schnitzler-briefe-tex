%% latex-korrekturansicht-vorspann.tex
%% Vorspann für die Korrekturansicht.
%% Lädt die gemeinsame Datei latex-vorspann.tex mit gesetztem Schalter.

\newif\ifkorrekturansicht
\korrekturansichttrue

\input{../tex-inputs/latex-vorspann}


\renewcommand{\erwaehntePersonen}{Personen: Camill Hoffmann}
\renewcommand{\erwaehnteOrte}{Orte: Edmund-Weiß-Gasse 7, IX., Alsergrund, Wien, XVIII., Währing}
\renewcommand{\erwaehnteWerke}{}
\section[ Felix Salten an Arthur Schnitzler, 20. 1. 1905]{Felix Salten an Arthur Schnitzler, 20. 1. 1905}
\nopagebreak\mylabel{v}
\rehead{ }\normalsize\beginnumbering\briefempfaengerindex{Schnitzler, Arthur@\textsc{Schnitzler, Arthur}!zzzSalten, Felix@\emph{von Felix Salten}!1905-01-201@{20. 1. 1905}|(be}
\toendnotes[C]{\smallbreak\pagebreak[2]}\Standort{CUL, Schnitzler, B 89, B 1.}
\physDesc{Kartenbrief, 577 Zeichen
\newline{}Handschrift: schwarze Tinte, lateinische Kurrent
\newline{}Versand: 1) Stempel: »\nobreak{}\oindex{IX., Alsergrund@\textbf{IX., Alsergrund}, \emph{A.ADM3}|pwk}Wien 9/1 66 r, 20 I 05, 4 40 V\nobreak{}«.   2) Stempel: »\nobreak{}\oindex{XVIII., Waehring@\textbf{XVIII., Währing}, \emph{A.ADM3}|pwk}\textcolor{gray}{18/1 Wi}en 111 P., 5\textsuperscript{20}\nobreak{}«. 
\newline{}Schnitzler: mit Bleistift datiert: »20/1 905« 
\newline{}Ordnung: mit Bleistift von unbekannter Hand nummeriert: »198a« }\toendnotes[C]{\smallbreak}\pstart{}{\pb}Herrn D\textsuperscript{r} Arthur Schnitzler\pend{}\pstart{}\textcolor{pink}{Wien XVIII.}{}\ledrightnote{\textcolor{pink}{XVIII., Währing}}\pend{}\pstart{}\textcolor{pink}{Spöttelgaße 7}{}\ledrightnote{\textcolor{pink}{Edmund-Weiß-Gasse 7}}.\pend{}
{\bigskip}
\pstart
           \noindent{}{\pb}Lieber Freund, selbstverständlich werde ich die Publication des
                  \label{K_L03406-1v}\edtext{Interviews}{\lemma{\textnormal{\emph{Interviews}}}\Cendnote{\textnormal{siehe A. S.: \emph{Tagebuch}, 19. 1. 1905 und 21. 1. 1905 sowie A. S.: \emph{»Das Zeitlose ist von kürzester Dauer«}, [Camill Hoffmann]: Wien – Berlin. Theaterfragen, 22. 1. 1905}}}\label{K_L03406-1h} verhindern. Herr \textcolor{blue}{Hoffmann}{}\ledrightnote{\textcolor{blue}{Camill Hoffmann}} ist freilich
               sehr betrübt darüber und wird versuchen Ihnen das, was er geschrieben hat,
               vorzulegen. Wenn Sie mir aber nicht direct, oder durch H. \textcolor{blue}{Hoffmann}{}\ledrightnote{\textcolor{blue}{Camill Hoffmann}} mittheilen, dass Sie Ihren Entschluß geändert haben,
               dann bleibt’s bei Ihrem heutigen Brief.\pend
           
\pstart
           Es ist wol überflüßig, zu betonen, dass ich persönlich dabei garnicht in Frage komme,
               und dass Sie sich \uline{nicht etwa durch eine Rücksicht auf
                  mich} sollen bestimmen laßen!\pend
           
\pstart
           Herzlichst {\\[\baselineskip]}Ihr \spacefill\mbox{Salten}\pend
           \leftskip=0em{}\endnumbering\briefempfaengerindex{Schnitzler, Arthur@\textsc{Schnitzler, Arthur}!zzzSalten, Felix@\emph{von Felix Salten}!1905-01-201@{20. 1. 1905}|)be}\mylabel{h}  \normalsize

\doendnotes{C}
\bigskip
\vfill

\clearpage

\footnotesize

\lohead{\textsc{register}}

% Definiere theindex-Environment komplett neu ohne reledmac
\makeatletter
\renewenvironment{theindex}{%
  \section*{\indexname}%
  \setlength{\parindent}{0pt}%
  \setlength{\parskip}{0pt plus 0.3pt}%
  \let\item\@idxitem
}{%
  \clearpage
}
\makeatother

\IfFileExists{\jobname-pw.ind}{\input{\jobname-pw.ind}}{}

\end{document}

      