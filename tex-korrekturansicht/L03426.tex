%% latex-korrekturansicht-vorspann.tex
%% Vorspann für die Korrekturansicht.
%% Lädt die gemeinsame Datei latex-vorspann.tex mit gesetztem Schalter.

\newif\ifkorrekturansicht
\korrekturansichttrue

\input{../tex-inputs/latex-vorspann}


\renewcommand{\erwaehntePersonen}{Personen:  ?? [Frau in Königs Wusterhausen], Samuel Fischer, Hedwig Fischer,  Luise von Mecklenburg-Strelitz, Felix Salten, Ottilie Salten, Olga Schnitzler}
\renewcommand{\erwaehnteOrte}{Orte: Edmund-Weiß-Gasse 7, Königs Wusterhausen, Pfuhl’s Hotel, Wien, XVIII., Währing}
\renewcommand{\erwaehnteWerke}{}
\section[ Felix Salten u. a. an Arthur Schnitzler, 4. 6. 1906]{Felix Salten u. a. an Arthur Schnitzler, 4. 6. 1906}
\nopagebreak\mylabel{v}
\rehead{ }\normalsize\beginnumbering\briefempfaengerindex{Schnitzler, Arthur@\textsc{Schnitzler, Arthur}!zzz?? [Frau in Koenigs Wusterhausen], @\emph{von  ?? [Frau in Königs Wusterhausen]}!1906-06-041@{4. 6. 1906}|(be}\briefempfaengerindex{Schnitzler, Arthur@\textsc{Schnitzler, Arthur}!zzzFischer, Hedwig@\emph{von Hedwig Fischer}!1906-06-041@{4. 6. 1906}|(be}\briefempfaengerindex{Schnitzler, Arthur@\textsc{Schnitzler, Arthur}!zzzFischer, Samuel@\emph{von Samuel Fischer}!1906-06-041@{4. 6. 1906}|(be}\briefempfaengerindex{Schnitzler, Arthur@\textsc{Schnitzler, Arthur}!zzzSalten, Ottilie@\emph{von Ottilie Salten}!1906-06-041@{4. 6. 1906}|(be}\briefempfaengerindex{Schnitzler, Arthur@\textsc{Schnitzler, Arthur}!zzzSalten, Felix@\emph{von Felix Salten}!1906-06-041@{4. 6. 1906}|(be}
\toendnotes[C]{\smallbreak\pagebreak[2]}\Standort{CUL, Schnitzler, B 89, B 1.}
\physDesc{Bildpostkarte, 258 Zeichen
\newline{}Handschrift Felix Salten: schwarze Tinte, lateinische Kurrent
\newline{}Handschrift Ottilie Salten: schwarze Tinte
\newline{}Handschrift Samuel Fischer: schwarze Tinte, lateinische Kurrent
\newline{}Handschrift Hedwig Fischer: schwarze Tinte, deutsche Kurrent
\newline{}Handschrift  ?? [Frau in Königs Wusterhausen]: blauer Buntstift, lateinische Kurrent
\newline{}Versand: Stempel: »\nobreak{}\oindex{Koenigs Wusterhausen@\textbf{Königs Wusterhausen}, \emph{P.PPL}|pwk}Königs-Wusterhaus\textcolor{gray}{en} b, 4. 6. 06, 8–9 N.\nobreak{}«.  
\newline{}Schnitzler: mit Bleistift datiert: »4/6« 
\newline{}Ordnung: mit Bleistift von unbekannter Hand nummeriert: »217« }\toendnotes[C]{\smallbreak}\pstart{}{\pb}Herrn D\textsuperscript{r} Arthur Schnitzler\pend{}\pstart{}\textcolor{pink}{Wien XVIII.}{}\ledrightnote{\textcolor{pink}{XVIII., Währing}}\pend{}\pstart{}\textcolor{pink}{Spöttelgaſse 7}{}\ledrightnote{\textcolor{pink}{Edmund-Weiß-Gasse 7}}\pend{}
{\bigskip}
\pstart
           \noindent{}\centering{}{\pb}\textcolor{gray}{\textbf{Gruss aus \textcolor{pink}{Königs-Wusterhausen}{}\ledrightnote{\textcolor{pink}{Königs Wusterhausen}}}}\pend
           
\pstart
           \noindent{}\centering{}\textcolor{gray}{\textbf{Historisches Buffet aus der Zeit der Königin \textcolor{blue}{Louise}{}\ledrightnote{\textcolor{blue}{Luise von Mecklenburg-Strelitz}}}}\pend
           
\pstart
           \noindent{}\centering{}\textcolor{gray}{\textbf{\textcolor{pink}{Pfuhl’s Hôtel}{}\ledrightnote{\textcolor{pink}{Pfuhl’s Hotel}}}}\pend
           
\pstart
           \noindent{}\centering{}\textcolor{gray}{\textbf{Parthie aus dem Park}}\pend
           
\pstart
           Lieber, ja, \label{K_L03426-1v}\edtext{krank}{\lemma{\textnormal{\emph{krank}}}\Cendnote{\textnormal{siehe Felix Salten an Arthur Schnitzler, 6. 7. 1906}}}\label{K_L03426-1h} war ich; aber es geht wieder besser. Brief folgt.\pend
           
\pstart
           herzlichst für Sie {\kaufmannsund}{ }\textcolor{blue}{Olga}{}\ledrightnote{\textcolor{blue}{Olga Schnitzler}}{\\[\baselineskip]}Ihr \spacefill\mbox{Salten}{\\[\baselineskip]}{[}hs. Ottilie Salten:{]} \spacefill\mbox{Ottilie}\pend
           \leftskip=0em{}\pstart {[}hs. Samuel Fischer:{]} Viele Grüße von Ihrem \spacefill\mbox{SFischer}\pend{}
\pstart
           \noindent{}{[}hs. Hedwig Fischer:{]} \textsc{Hedwig Fischer} grüßt Sie u. Ihre \textcolor{blue}{Frau}{}\ledrightnote{{$\rightarrow$}\textcolor{blue}{Olga Schnitzler}}.\pend
           
\pstart
           \noindent{}{[}hs. ?? [Frau in Königs Wusterhausen]:{]} Eine \label{K_L03426-2v}\edtext{Verehrerin}{\lemma{\textnormal{\emph{Verehrerin}}}\Cendnote{\textnormal{nicht
                  identifiziert}}}\label{K_L03426-2h} grüßt auch noch herzlich.\pend
           \endnumbering\briefempfaengerindex{Schnitzler, Arthur@\textsc{Schnitzler, Arthur}!zzz?? [Frau in Koenigs Wusterhausen], @\emph{von  ?? [Frau in Königs Wusterhausen]}!1906-06-041@{4. 6. 1906}|)be}\briefempfaengerindex{Schnitzler, Arthur@\textsc{Schnitzler, Arthur}!zzzFischer, Hedwig@\emph{von Hedwig Fischer}!1906-06-041@{4. 6. 1906}|)be}\briefempfaengerindex{Schnitzler, Arthur@\textsc{Schnitzler, Arthur}!zzzFischer, Samuel@\emph{von Samuel Fischer}!1906-06-041@{4. 6. 1906}|)be}\briefempfaengerindex{Schnitzler, Arthur@\textsc{Schnitzler, Arthur}!zzzSalten, Ottilie@\emph{von Ottilie Salten}!1906-06-041@{4. 6. 1906}|)be}\briefempfaengerindex{Schnitzler, Arthur@\textsc{Schnitzler, Arthur}!zzzSalten, Felix@\emph{von Felix Salten}!1906-06-041@{4. 6. 1906}|)be}\mylabel{h}  \normalsize

\doendnotes{C}
\bigskip
\vfill

\clearpage

\footnotesize

\lohead{\textsc{register}}

% Definiere theindex-Environment komplett neu ohne reledmac
\makeatletter
\renewenvironment{theindex}{%
  \section*{\indexname}%
  \setlength{\parindent}{0pt}%
  \setlength{\parskip}{0pt plus 0.3pt}%
  \let\item\@idxitem
}{%
  \clearpage
}
\makeatother

\IfFileExists{\jobname-pw.ind}{\input{\jobname-pw.ind}}{}

\end{document}

      