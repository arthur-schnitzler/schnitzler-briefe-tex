%% latex-korrekturansicht-vorspann.tex
%% Vorspann für die Korrekturansicht.
%% Lädt die gemeinsame Datei latex-vorspann.tex mit gesetztem Schalter.

\newif\ifkorrekturansicht
\korrekturansichttrue

\input{../tex-inputs/latex-vorspann}


               \section[Arthur Schnitzler an Hugo von Hofmannsthal, 31. 12. 1904]{ Arthur Schnitzler an Hugo von Hofmannsthal, 31. 12. 1904}\nopagebreak\mylabel{v}\rehead{ }\normalsize\beginnumbering\briefempfaengerindex{Hofmannsthal, Hugo von@\textsc{Hofmannsthal, Hugo von}!zzzSchnitzler, Arthur@\emph{von Arthur Schnitzler}!1904-12-311@{31. 12. 1904}|(be} \toendnotes[C]{\smallbreak\pagebreak[2]} \Standort{FDH, Hs-30885,119.}
\physDesc{Brief, 2 Blätter, 5 Seiten
\newline{}Handschrift: schwarze Tinte, deutsche Kurrent}\buchAbdrucke{\weitereDrucke{Hugo von Hofmannsthal, Arthur Schnitzler: \emph{Briefwechsel}. Hg. Therese Nickl und Heinrich Schnitzler. Frankfurt am Main: \emph{S. Fischer} 1964, S. 209.} }\toendnotes[C]{\smallbreak}\pstart
           \raggedleft{}{\pb}\textcolor{pink}{Wien}{}\ledrightnote{\textcolor{pink}{Wien}}, 31. 12. 904.\pend
           \pstart{}lieber Hugo, \pend\pstart
           ich habe \textcolor{blue}{Grunwald}{}\ledrightnote{\textcolor{blue}{Willy Grunwald}} in \textcolor{green}{Traumulus}{}\ledrightnote{\textcolor{green}{Traumulus}} als problematiſchen Corpsſtudenten, in der \textcolor{green}{Frau vom Meer}{}\ledrightnote{\textcolor{green}{Die Frau vom Meer. Schauspiel in fünf Akten}} als \textcolor{green}{Lyngſtrand}{}\ledrightnote{→\textcolor{green}{Die Frau vom Meer. Schauspiel in fünf Akten}} und da{\geminationn} im \textcolor{green}{Geyer}{}\ledrightnote{\textcolor{green}{Florian Geyer. Die Tragödie des Bauernkrieges}} als {\dots} ich weiſs nicht mehr was
               geſehen, und \textcolor{blue}{Brahm}{}\ledrightnote{\textcolor{blue}{Otto Brahm}} weiſs, daſs ich ihn ſehr
               ſchätze und noch allerlei Möglichkeiten in ihm zu ſpüren glaube. Er iſt aber gewiſs
               keine ſehr reiche und keine ſehr ſtarke Natur und hat auch das geheimnisvolle nicht,
               das manche haben, ohne stark {\pb}und groß zu ſein; er iſt
               ſehr ſcharf umriſſen aber es iſt nicht viel Luft um ihn. Nun ſcheint es mir aber für
               den \textcolor{green}{Jaffier}{}\ledrightnote{→\textcolor{green}{Das gerettete Venedig. Trauerspiel in fünf Aufzügen}} notwendig, daſs man
               in ſeiner Perſönlichkeit den vergang\damage{ene}n Zauber ahnt und ich glaube, ſo etwas überzeugend herauszubringen, ist \strikeout{dichteriſch}{ }ſchauſpieleriſch ebenſo ſchwer, ja an der Grenze
               des Möglichen wie dichteriſch. Ihnen iſt es nur dadurch (und doch nicht ganz)
               gelungen, daſs Sie zwei in ihrer Art außerordentliche Menſchen, den \textcolor{green}{\textsc{Pierre}}{}\ledrightnote{→\textcolor{green}{Das gerettete Venedig. Trauerspiel in fünf Aufzügen}} und die \textcolor{green}{\textsc{Belvidera}}{}\ledrightnote{→\textcolor{green}{Das gerettete Venedig. Trauerspiel in fünf Aufzügen}}, {\pb}einen, deſſen Weſen Muth, die andere, deren
               Weſen Hingebung, noch zu einer Zeit unter jenem Zauber ſtehen laſſen, da wir nichts
               mehr \substVorne{}\textsuperscript{davon be}{\allowbreak}\substDazwischen{}von ihm\substHinten{} angerührt werden – aber immerhin\textcolor{gray}{,} wir denken: Muſs das
               ein Kerl geweſen ſein – daſs die zwei gar nicht merken, wie wenig er es heute iſt! –
                  \textcolor{blue}{Mitterwurzer}{}\ledrightnote{\textcolor{blue}{Friedrich Mitterwurzer}}, \textcolor{blue}{Kainz}{}\ledrightnote{\textcolor{blue}{Josef Kainz}}, \textcolor{blue}{Baſſermann}{}\ledrightnote{\textcolor{blue}{Albert Bassermann}} wieder trügen dieſes
               »geweſene« wie einen Heiligenſchein von verſtäubten Schickſalen um ihr Haupt, einen
               Schein, der eben nur in Perſönlichkeitsatmosphäre ſichtbar {\pb}wird. Davon, mein ich, wird bei \textcolor{blue}{Grunwald}{}\ledrightnote{\textcolor{blue}{Willy Grunwald}} nichts merklich ſein. Warum ich Ihnen das ſage weiſs
               ich eigentlich nicht – denn wenn \textcolor{blue}{\textsc{Bassermann}}{}\ledrightnote{\textcolor{blue}{Albert Bassermann}} abſolut nicht will, iſt \textcolor{blue}{G.}{}\ledrightnote{\textcolor{blue}{Willy Grunwald}} gewiſs der
               einzige, der in Betracht kommt. Er wird ſetze ich voraus, die Rolle von der weibiſch
                  \strikeout{ja} – verwöhnten Seite her zu nehmen ſuchen, \strikeout{und als} ja, er wird vielleicht auch das hyſteriſch
               verlogene (es iſt eine Bezeichnung, kein Schimpf) in \substVorne{}\textsuperscript{\textcolor{gray}{×}\-\textcolor{gray}{×}\-\textcolor{gray}{×}\-\textcolor{gray}{×}\-\textcolor{gray}{×}\-\textcolor{gray}{×}\-\textcolor{gray}{×}\-\textcolor{gray}{×}\-\textcolor{gray}{×}\-\textcolor{gray}{×}}\substDazwischen{}lebhafterer\substHinten{} Weiſe herausbringen, als Sie wollten. Wie immer, – es {\pb}wird durch dieſe Beſetzung \strikeout{noch} mehr als je die Tragoedie von der Enttäuſchung des \textcolor{green}{Pierre}{}\ledrightnote{→\textcolor{green}{Das gerettete Venedig. Trauerspiel in fünf Aufzügen}}, und vielleicht ko{\geminationm}t nun alles bei der Einſtudierg darauf an, mit dieſem
               Gleichgewichtsverhältnis von vornherein zu rechnen.\pend
           \pstart
           Sie haben doch nun meine Karte aus \textcolor{pink}{Lueg}{}\ledrightnote{\textcolor{pink}{Lueg am Wolfgangsee}} bekommen?
               Wir ſind alſo Montag 2.{ }Abends 8{ }\textcolor{pink}{Hietzing, \textsc{Kuffner}}{}\ledrightnote{\textcolor{pink}{Ottakringer Bräu}}. Vielleicht iſt unſer \textcolor{green}{\textcolor{blue}{\textsc{Charolais}}{}\ledrightnote{→\textcolor{blue}{Richard Beer-Hofmann}}}{}\ledrightnote{\textcolor{green}{Der Graf von Charolais. Ein Trauerspiel}} doch ſchon hier und kommt?\pend
           \pstart
           Herzlichſt Ihr{\\[\baselineskip]}\spacefill\mbox{A.}\pend
           \leftskip=0em{}\endnumbering\briefempfaengerindex{Hofmannsthal, Hugo von@\textsc{Hofmannsthal, Hugo von}!zzzSchnitzler, Arthur@\emph{von Arthur Schnitzler}!1904-12-311@{31. 12. 1904}|)be}\mylabel{h}  \normalsize

\doendnotes{C}
\bigskip
\vfill

\clearpage

\footnotesize

\lohead{\textsc{register}}

% Definiere theindex-Environment komplett neu ohne reledmac
\makeatletter
\renewenvironment{theindex}{%
  \section*{\indexname}%
  \setlength{\parindent}{0pt}%
  \setlength{\parskip}{0pt plus 0.3pt}%
  \let\item\@idxitem
}{%
  \clearpage
}
\makeatother

\IfFileExists{\jobname-pw.ind}{\input{\jobname-pw.ind}}{}

\end{document}

      