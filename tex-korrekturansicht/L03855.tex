%% latex-korrekturansicht-vorspann.tex
%% Vorspann für die Korrekturansicht.
%% Lädt die gemeinsame Datei latex-vorspann.tex mit gesetztem Schalter.

\newif\ifkorrekturansicht
\korrekturansichttrue

\input{../tex-inputs/latex-vorspann}


\section[Theodor Herzl an Arthur Schnitzler, 30. 3. 1895]{L03855 Theodor Herzl an Arthur Schnitzler, 30. 3. 1895}
\nopagebreak\mylabel{L03855v}
\rehead{ }\normalsize\beginnumbering\briefempfaengerindex{, @\textsc{, }!zzz, @\emph{von  }!1895-03-301@{30. 3. 1895}|(be}
\toendnotes[C]{\smallbreak\pagebreak[2]}\Standort{CUL, Schnitzler, B 39.}
\physDesc{Brief, 1 Blatt, 1 Seite, 251 Zeichen
\newline{}Handschrift: schwarze Tinte, lateinische Kurrent
\newline{}Ordnung: mit Bleistift von unbekannter Hand nummeriert:»34« }
\buchAbdrucke{\weitereDrucke{Theodor Herzl: \emph{Briefe und autobiographische Notizen 1866–1895}. Bearbeitet von Johannes Wachten in Zusammenarbeit mit Chaya Harel, Daisy Tycho und Manfred Winkler. Berlin, Frankfurt am Main, Wien: \emph{Propyläen} 1983, S. 580 (Briefe und Tagebücher. Herausgegeben von Alex Bein, Hermann Greive, Moshe Schaerf, Julius H. Schoeps und Johannes Wachten, 1).} }\toendnotes[C]{\smallbreak}
\pstart
           {\pb}\textcolor{gray}{\textbf{\textcolor{brown}{NEUE FREIE PRESSE}\orgindex{Neue Freie Presse@Neue Freie Presse|pw}{}\ledrightnote{\textcolor{brown}{Neue Freie Presse}}. }}\pend
           
\pstart
           \textcolor{gray}{\textbf{\textsc{Redaction}:}}\pend
           
\pstart
           \textcolor{gray}{\textbf{\textcolor{pink}{WIEN}\oindex{Wien@\textbf{Wien}, \emph{Verwaltungsgebiet}|pw}{}\ledrightnote{\textcolor{pink}{Wien}}}}\pend
           
\pstart
           \textcolor{gray}{\textbf{\textcolor{pink}{Kolowratring, Fichtegasse Nr. 11}\oindex{Wien@\textbf{Wien}!I., Innere Stadt@\textbf{I., Innere Stadt}!Fichtegasse 11@\textbf{Fichtegasse 11}, \emph{Gebäude}|pw}{}\ledrightnote{\textcolor{pink}{Fichtegasse 11}}.
                     }}\hfill 30 März 95\pend
           
\pstart{}Lieber Freund!\pend\vspace{0.5em}
\pstart
           Nachmittag{ }zwischen 4 u. 6 komme ich zu Ihnen. \pend
           
\pstart
           Kann ich das nicht{[},{]} so telephonire ich Ihnen die \label{K_L03855-1v}\edtext{Logennummer}{\lemma{\textnormal{\emph{Logennummer}}}\Cendnote{\textnormal{\textcolor{blue}{Schnitzler} vermerkt im \emph{\textcolor{green}{Tagebuch}\pwindex{Schnitzler, Arthur 15. 5. 1862 Wien – 21. 10. 1931 ebd.@\textsc{Schnitzler, Arthur} (15. 5. 1862 Wien – 21. 10. 1931 ebd.), \emph{Schriftsteller, Mediziner}!Tagebuch@\strich\emph{Tagebuch}|pwk}} den Besuch der \textcolor{violet}{Aufführung}\eventindex{Carl-Theater@\textbf{Carl-Theater}!Aufführung von Die Brillanten-Königin, 30.3.1895@Aufführung von Die Brillanten-Königin, 30.3.1895|pwkv} der Operette \emph{\textcolor{green}{die Billanten-Königin}\pwindex{Jakobowski, Edward 1856 London – 29.\,4.\,1929 ebd.@\textsc{Jakobowski, Edward} (1856 London – 29.\,4.\,1929 ebd.), \emph{Komponist}!Brillanten-Königin@\strich\emph{Die Brillanten-Königin}|pwk}} und ein anschließendes Abendessen mit \textcolor{blue}{Herzl}\pwindex{Herzl, Theodor 2.\,5.\,1860 Budapest – 3.\,7.\,1904 Edlach@\textsc{Herzl, Theodor} (2.\,5.\,1860 Budapest – 3.\,7.\,1904 Edlach), \emph{Schriftsteller, Journalist}|pwk}, vgl. A. S.: \emph{Tagebuch}, 30. 3. 1895.}}}\label{K_L03855-1}.\pend
           
\pstart
           Sollte von \textcolor{blue}{M. G.}\pwindex{Müller-Guttenbrunn, Adam 22.\,10.\,1852 Zăbrani – 5.\,1.\,1923 Wien@\textsc{Müller-Guttenbrunn, Adam} (22.\,10.\,1852 Zăbrani – 5.\,1.\,1923 Wien), \emph{Schriftsteller, Theaterleiter, Beamter}|pwv}{}\ledrightnote{{$\rightarrow$}\emph{\textcolor{blue}{Adam Müller-Guttenbrunn}}}{ }\label{K_L03855-2v}\edtext{Brief
               über \textcolor{green}{d. G.}\pwindex{Herzl, Theodor 2.\,5.\,1860 Budapest – 3.\,7.\,1904 Edlach@\textsc{Herzl, Theodor} (2.\,5.\,1860 Budapest – 3.\,7.\,1904 Edlach), \emph{Schriftsteller, Journalist}!neue Ghetto. Schauspiel in vier Acten@\strich\emph{Das neue Ghetto. Schauspiel in vier Acten}|pw}{}\ledrightnote{\textcolor{green}{Das neue Ghetto. Schauspiel in vier Acten}}}{\lemma{\textnormal{\emph{Brief
               über d. G.}}}\Cendnote{\textnormal{Der Brief findet sich unter den Korrespondenzstücken von \textcolor{blue}{Schnitzler} im Nachlass \textcolor{blue}{Herzls}\pwindex{Herzl, Theodor 2.\,5.\,1860 Budapest – 3.\,7.\,1904 Edlach@\textsc{Herzl, Theodor} (2.\,5.\,1860 Budapest – 3.\,7.\,1904 Edlach), \emph{Schriftsteller, Journalist}|pwk} (\emph{Central Zionist
                        Archives}, H1/1925 2) und dürfte, wie hier
                  vorgeschlagen, persönlich übergeben worden sein: »{\pb}\textcolor{gray}{\textbf{\textcolor{brown}{Raimund Theater}\orgindex{Raimund-Theater@Raimund-Theater|pw}.}}{ / }\textcolor{gray}{\textbf{Direction: \textcolor{blue}{A.
                              Müller-Guttenbrunn}\pwindex{Müller-Guttenbrunn, Adam 22.\,10.\,1852 Zăbrani – 5.\,1.\,1923 Wien@\textsc{Müller-Guttenbrunn, Adam} (22.\,10.\,1852 Zăbrani – 5.\,1.\,1923 Wien), \emph{Schriftsteller, Theaterleiter, Beamter}|pw}.}}{ / }\textcolor{gray}{\textbf{\textcolor{pink}{Wien}\oindex{Wien@\textbf{Wien}, \emph{Verwaltungsgebiet}|pw}, am}}26. III \textcolor{gray}{\textbf{189}}5{ / }Verehrter Herr Dr Schnitzler!{ / }Ich habe das Schauſpiel »\textcolor{green}{Ghetto}\pwindex{Herzl, Theodor 2.\,5.\,1860 Budapest – 3.\,7.\,1904 Edlach@\textsc{Herzl, Theodor} (2.\,5.\,1860 Budapest – 3.\,7.\,1904 Edlach), \emph{Schriftsteller, Journalist}!neue Ghetto. Schauspiel in vier Acten@\strich\emph{Das neue Ghetto. Schauspiel in vier Acten}|pw}« mit
                        außerordentlichem Intereſſe geleſen u. halte das \textcolor{green}{Stück}\pwindex{Herzl, Theodor 2.\,5.\,1860 Budapest – 3.\,7.\,1904 Edlach@\textsc{Herzl, Theodor} (2.\,5.\,1860 Budapest – 3.\,7.\,1904 Edlach), \emph{Schriftsteller, Journalist}!neue Ghetto. Schauspiel in vier Acten@\strich\emph{Das neue Ghetto. Schauspiel in vier Acten}|pwv}, obwohl mich die Löſung
                        nicht befriedigte u. ich dem Helden mehr Spielraum gegönnt hätte, für eine
                        der intereſſanteſten Arbeiten, die mir ſeit Langem unter\strikeout{k}geko{\geminationm}en. Das Stück hat frappante Züge von
                        Lebenswahrheit, es iſt reich an feinem Detail u. es wird getragen {\pb}von einer Idee, der man weder die Natürkichkeit,
                        noch die tiefere Bedeutung abſprechen kann.{ / }Und trotz alledem – würden \uline{Sie} es aufführen?
                        Und glauben Sie, daß ſich irgendwo in deutſchen Landen ein großes Theater
                        findet, welches »\textcolor{green}{Ghetto}\pwindex{Herzl, Theodor 2.\,5.\,1860 Budapest – 3.\,7.\,1904 Edlach@\textsc{Herzl, Theodor} (2.\,5.\,1860 Budapest – 3.\,7.\,1904 Edlach), \emph{Schriftsteller, Journalist}!neue Ghetto. Schauspiel in vier Acten@\strich\emph{Das neue Ghetto. Schauspiel in vier Acten}|pw}« aufführt? Ich
                        glaub es nicht! Sie können das \textcolor{green}{Stück}\pwindex{Herzl, Theodor 2.\,5.\,1860 Budapest – 3.\,7.\,1904 Edlach@\textsc{Herzl, Theodor} (2.\,5.\,1860 Budapest – 3.\,7.\,1904 Edlach), \emph{Schriftsteller, Journalist}!neue Ghetto. Schauspiel in vier Acten@\strich\emph{Das neue Ghetto. Schauspiel in vier Acten}|pwv} alſo ruhig noch einige Tage hier liegen laſſen, ich will es
                        noch von andern, ganz unbetheiligten Perſonen, deren Urtheil mir werthvoll
                        iſt, leſen laſſen. \uline{Wenn} das \textcolor{green}{Stück}\pwindex{Herzl, Theodor 2.\,5.\,1860 Budapest – 3.\,7.\,1904 Edlach@\textsc{Herzl, Theodor} (2.\,5.\,1860 Budapest – 3.\,7.\,1904 Edlach), \emph{Schriftsteller, Journalist}!neue Ghetto. Schauspiel in vier Acten@\strich\emph{Das neue Ghetto. Schauspiel in vier Acten}|pwv} in \textcolor{pink}{Wien}\oindex{Wien@\textbf{Wien}, \emph{Verwaltungsgebiet}|pw} jemals aufgeführt \strikeout{iſ} wird, ſo kann dies nur im \textcolor{brown}{R.
                           Th.}\orgindex{Raimund-Theater@Raimund-Theater|pw} geſchehen. \uline{Dank} werden wir kaum
                        dafür ernten, weder von den Juden, noch von den \textsc{Antisemiten}! Den Herrn \textsc{Schnabel} würde ich
                        gerne ſprechen.{ / }Ihr ergebenſter \textcolor{gray}{MGuttenbrunn}« }}}\label{K_L03855-2} da sein, so bitte ich mir ihn \uline{nicht} zu schicken, sondern ins \textcolor{violet}{\textcolor{green}{\textcolor{pink}{Theater}\oindex{Wien@\textbf{Wien}!II., Leopoldstadt@\textbf{II., Leopoldstadt}!Carl-Theater@\textbf{Carl-Theater}, \emph{Theater}|pw}{}\ledrightnote{\textcolor{pink}{Carl-Theater}}}\pwindex{Jakobowski, Edward 1856 London – 29.\,4.\,1929 ebd.@\textsc{Jakobowski, Edward} (1856 London – 29.\,4.\,1929 ebd.), \emph{Komponist}!Brillanten-Königin@\strich\emph{Die Brillanten-Königin}|pwv}{}\ledrightnote{{$\rightarrow$}\emph{\textcolor{green}{Die Brillanten-Königin}}}}\eventindex{Carl-Theater@\textbf{Carl-Theater}!Aufführung von Die Brillanten-Königin, 30.3.1895@Aufführung von Die Brillanten-Königin, 30.3.1895|pwv}{}\ledrightnote{{$\rightarrow$}\emph{\textcolor{violet}{Aufführung von Die Brillanten-Königin, 30.3.1895}}} zu bringen \pend
           
\pstart
           Herzlich Ihr{\\[\baselineskip]}\spacefill\mbox{Th H.}\pend
           \leftskip=0em{}\selectlanguage{ngerman}\endnumbering\briefempfaengerindex{, @\textsc{, }!zzz, @\emph{von  }!1895-03-301@{30. 3. 1895}|)be}\mylabel{L03855h}
\begin{anhang}
\end{anhang}\normalsize

\doendnotes{C}
\bigskip
\vfill

\clearpage

\footnotesize

\lohead{\textsc{register}}

% Definiere theindex-Environment komplett neu ohne reledmac
\makeatletter
\renewenvironment{theindex}{%
  \section*{\indexname}%
  \setlength{\parindent}{0pt}%
  \setlength{\parskip}{0pt plus 0.3pt}%
  \let\item\@idxitem
}{%
  \clearpage
}
\makeatother

\IfFileExists{\jobname-pw.ind}{\input{\jobname-pw.ind}}{}

\end{document}

      