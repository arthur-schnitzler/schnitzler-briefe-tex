%% latex-korrekturansicht-vorspann.tex
%% Vorspann für die Korrekturansicht.
%% Lädt die gemeinsame Datei latex-vorspann.tex mit gesetztem Schalter.

\newif\ifkorrekturansicht
\korrekturansichttrue

\input{../tex-inputs/latex-vorspann}


         
         \renewcommand{\erwaehntePersonen}{Personen: Wilhelm Bode, Henri de Catt, Erich Freund,  Friedrich II. von Preußen, Marie Glümer, Theodor Loewe, Paul Martin Marton, Irene Triesch, Hugo Wittmann}
         \renewcommand{\erwaehnteInstitutionen}{Institutionen: Burgtheater, Ernst Siegfried Mittler {\kaufmannsund}  Sohn, Fr. Wilh. Grunow, Lobe-Theater, Neue Freie Presse, Secessionsbühne, Volkstheater}
         \renewcommand{\erwaehnteOrte}{Orte: Berlin, Breslau, Dessauer Straße, Leipzig, Tadeusz-Kościuszko-Platz, Wien}
         \renewcommand{\erwaehnteWerke}{Werke: Berliner Börsen-Courier, Burgtheater. (Zum erstenmale: Die Orestie. Tragödie in drei Stücken. Aus dem Griechischen des Aischylos. Nach der Übersetzung des Freiherrn Ulrich v. Wilamowitz-Moellendorff für die moderne Bühne bearbeitet.), Der Schleier der Beatrice. Schauspiel in fünf Akten, Gespräche Friedrichs des Großen mit Henri de Catt, Goethes Lebenskunst, Grenzboten-Sammlung, Neue Freie Presse, Orestie, [Man telegraphirt uns aus Breslau…], [Rezension von Erich Freund über Schleier der Beatrice]}
               \section[ Paul Goldmann an Arthur Schnitzler, 9. 12. {[}1900{]}]{Paul Goldmann an Arthur Schnitzler, 9. 12. {[}1900{]}}\nopagebreak\mylabel{v}\rehead{ }\normalsize\beginnumbering\briefempfaengerindex{Schnitzler, Arthur@\textsc{Schnitzler, Arthur}!zzzGoldmann, Paul@\emph{von Paul Goldmann}!1900-12-091@{9. 12. {[}1900{]}}|(be} \toendnotes[C]{\smallbreak\pagebreak[2]} \Standort{DLA, A:Schnitzler, HS.NZ85.1.3170.}
\physDesc{Brief, 1 Blatt, 4 Seiten
\newline{}Handschrift: blaue Tinte, deutsche Kurrent\newline{}Beilage: handschriftlicher Brief, 2 Blätter, 3 Seiten, schwarze Tinte,
                                 deutsche Kurrent 
\newline{}Schnitzler: 1) mit Bleistift das Jahr »{[}1{]}900« vermerkt  2) mit rotem Buntstift drei Unterstreichungen und ein
                                    »X«}\toendnotes[C]{\smallbreak}\pstart
           \noindent{}\raggedleft{}{\pb}\textcolor{pink}{\textcolor{gray}{\textbf{DESSAUERSTRASSE 19}}}{}\ledrightnote{\textcolor{pink}{Dessauer Straße}}\pend
           \pstart
           \textcolor{pink}{Berlin}{}\ledrightnote{\textcolor{pink}{Berlin}}, 9. December.\pend
           \pstart\center{}Mein lieber Freund,\pend\pstart
           Endlich geſtern konnte ich Frl. \textsc{\label{K_L02944-11v}\edtext{\textcolor{blue}{Glümer}{}\ledrightnote{\textcolor{blue}{Marie Glümer}}}{\lemma{\textnormal{\emph{Glümer}}}\Cendnote{\textnormal{Sie war für die Uraufführung von \emph{\textcolor{green}{Der Schleier der Beatrice}} nach \textcolor{pink}{Breslau} gereist.}}}\label{K_L02944-11h}} ſprechen. Das ſcheint ja eine hübſche Schweinerei geweſen zu ſein, dieſe \textcolor{pink}{Breslau}{}\ledrightnote{\textcolor{pink}{Breslau}}er \textcolor{green}{Aufführung}{}\ledrightnote{{$\rightarrow$}\textcolor{green}{Der Schleier der Beatrice. Schauspiel in fünf Akten}}. Ja, \textcolor{pink}{Breslau}{}\ledrightnote{\textcolor{pink}{Breslau}}!
               Man muß \label{K_L02944-1v}\edtext{in dieſer \textcolor{pink}{Stadt}{}\ledrightnote{{$\rightarrow$}\textcolor{pink}{Breslau}} geboren}{\lemma{\textnormal{\emph{in dieſer Stadt geboren}}}\Cendnote{\textnormal{\textcolor{blue}{Goldmann} meint sich selbst}}}\label{K_L02944-1h} ſein, um
               ſie ganz würdigen zu können.\pend
           \pstart
           Heut ſprach ich den \label{K_L02944-19v}\edtext{Direktor {\pb}\textsc{\textcolor{blue}{Martin}{}\ledrightnote{\textcolor{blue}{Paul Martin Marton}}}}{\lemma{\textnormal{\emph{Direktor Martin}}}\Cendnote{\textnormal{\textcolor{blue}{Paul Martin Marton}, Direktor der \textcolor{pink}{Berlin}er \emph{\textcolor{brown}{Secessionsbühne}} und späterer Ehemann von \textcolor{blue}{Marie Glümer}}}}\label{K_L02944-19h} und habe ihm rieſig zugeredet, die \textsc{\textcolor{blue}{Triesch}{}\ledrightnote{\textcolor{blue}{Irene Triesch}}}, die er haben kann, zu engagiren. Dann wird er die \label{K_L02944-2v}\edtext{»\textsc{\textcolor{green}{Beatrice}{}\ledrightnote{\textcolor{green}{Der Schleier der Beatrice. Schauspiel in fünf Akten}}}« ſpielen}{\lemma{\textnormal{\emph{»Beatrice« ſpielen}}}\Cendnote{\textnormal{nicht geschehen}}}\label{K_L02944-2h},
               und es wird gut werden.\pend
           \pstart
           Dem \label{K_L02944-3v}\edtext{\textcolor{brown}{Volkstheater}{}\ledrightnote{\textcolor{brown}{Volkstheater}}}{\lemma{\textnormal{\emph{Volkstheater}}}\Cendnote{\textnormal{hier formuliert er eine Kehrtwende, vgl. Paul Goldmann an Arthur Schnitzler, 21. 6. [1900]}}}\label{K_L02944-3h} ſollteſt Du das \textcolor{green}{Stück}{}\ledrightnote{{$\rightarrow$}\textcolor{green}{Der Schleier der Beatrice. Schauspiel in fünf Akten}}
               ruhig geben. So ſchlimm wie in \textcolor{pink}{Breslau}{}\ledrightnote{\textcolor{pink}{Breslau}} kann es
               keinesfalls werden.\pend
           \pstart
           Die \textcolor{brown}{N. Fr. Pr.}{}\ledrightnote{\textcolor{brown}{Neue Freie Presse}} hat wieder einmal, wie Du {\pb}beifolgendem Briefe des \textsc{Dr.}{ }\textsc{\textcolor{blue}{Freund}{}\ledrightnote{\textcolor{blue}{Erich Freund}}} erſehen wirſt, in ihrem Glanze gezeigt.\pend
           \pstart
           Iſt die \strikeout{\textcolor{gray}{»}}{ }\textcolor{green}{Oreſtie}{}\ledrightnote{\textcolor{green}{Orestie}} im \textcolor{brown}{Burgtheater}{}\ledrightnote{\textcolor{brown}{Burgtheater}} wirklich ſo großartig, wie \textsc{\textcolor{blue}{Wittmann}{}\ledrightnote{\textcolor{blue}{Hugo Wittmann}}}{ }\label{K_L02944-4v}\edtext{\textcolor{green}{behauptet}{}\ledrightnote{{$\rightarrow$}\textcolor{green}{Burgtheater. (Zum erstenmale: Die Orestie. Tragödie in drei Stücken. Aus dem Griechischen des Aischylos. Nach der Übersetzung des Freiherrn Ulrich v. Wilamowitz-Moellendorff für die moderne Bühne bearbeitet.)}}}{\lemma{\textnormal{\emph{behauptet}}}\Cendnote{\textnormal{[\textcolor{blue}{Hugo Wittmann}]: \emph{\textcolor{green}{Burgtheater. (Zum erstenmale: Die Orestie. Tragödie in drei
                        Stücken. Aus dem Griechischen des Aischylos. Nach der Übersetzung des
                        Freiherrn Ulrich v. Wilamowitz-Moellendorff für die moderne Bühne
                        bearbeitet.)}} In: \emph{\textcolor{green}{Neue Freie
                     Presse}}, Nr. 13037, 8. 12. 1900,
                     Morgenblatt, S. 1–3.}}}\label{K_L02944-4h}? Ich habe Mißtrauen. \strikeout{Er \textcolor{gray}{w}}{ }\textsc{\textcolor{blue}{Wittmann}{}\ledrightnote{\textcolor{blue}{Hugo Wittmann}}} iſt auch kein Kritiker, ſondern ein Mann, dem es nur darum zu thun iſt, hübſch
               über eine Sache zu ſchreiben, {\pb}wobei die Sache
               ſelbſt \introOben{}ihm\introOben{} ſehr gleichgiltig iſt.\pend
           \pstart
           Viele treue Grüße! {\\[\baselineskip]}Dein {\\[\baselineskip]}\spacefill\mbox{Paul Goldmnn.}\pend
           \leftskip=0em{}\pstart
           \noindent{}Leſen: \label{K_L02944-111v}\edtext{\textcolor{green}{Geſpräche \textcolor{blue}{Friedrichs des Gr.}{}\ledrightnote{\textcolor{blue}{Friedrich II. von Preußen}} mit \textsc{\textcolor{blue}{Henri de Catt}{}\ledrightnote{\textcolor{blue}{Henri de Catt}}}}{}\ledrightnote{\textcolor{green}{Gespräche Friedrichs des Großen mit Henri de Catt}} (\textcolor{green}{Grenzboten-Sammlung}{}\ledrightnote{\textcolor{green}{Grenzboten-Sammlung}})}{\lemma{\textnormal{\emph{Geſpräche … (Grenzboten-Sammlung)}}}\Cendnote{\textnormal{\emph{\textcolor{green}{Gespräche Friedrichs des Großen mit Henri
                           de Catt}}. \textcolor{pink}{Leipzig}: \emph{\textcolor{brown}{Fr. Wilh. Grunow}}{ }1885. (\emph{\textcolor{green}{Grenzboten-Sammlung}} II, 8)}}}\label{K_L02944-111h}.\pend
           \pstart
           \label{K_L02944-12v}\edtext{\textsc{Dr. \textcolor{blue}{Wilhelm Bode}{}\ledrightnote{\textcolor{blue}{Wilhelm Bode}}}: \textcolor{green}{Goethes Lebenskunſt}{}\ledrightnote{\textcolor{green}{Goethes Lebenskunst}}}{\lemma{\textnormal{\emph{Dr. … Lebenskunſt}}}\Cendnote{\textnormal{\textcolor{blue}{Wilhelm Bode}: \emph{\textcolor{green}{Goethes Lebenskunst}}. \textcolor{pink}{Berlin}: \emph{\textcolor{brown}{Ernst Siegfried Mittler {\kaufmannsund} Sohn}}{ }1901.}}}\label{K_L02944-12h}.\pend
           {\bigskip}\pstart
           \noindent{}{\pb}{[}hs. Freund:{]} \textcolor{gray}{\textbf{\textsc{Dr. \textcolor{blue}{Erich
                        Freund}{}\ledrightnote{\textcolor{blue}{Erich Freund}}.}}}\pend
           \pstart
           \raggedleft{}\textcolor{gray}{\textbf{\textcolor{pink}{Breslau V}{}\ledrightnote{\textcolor{pink}{Breslau}},}}{ }5. 12. \textcolor{gray}{\textbf{190}}0\pend
           \pstart
           \raggedleft{}\textcolor{gray}{\textbf{\textcolor{pink}{Tauentzienplatz 1\textsuperscript{a}}{}\ledrightnote{\textcolor{pink}{Tadeusz-Kościuszko-Platz}}.}}\pend
           \pstart{}Liebes \textsc{Paulchen}!\pend\pstart
           Ganz wie ich fürchtete, iſt meiner \textcolor{green}{Telegraphirerei}{}\ledrightnote{{$\rightarrow$}\textcolor{green}{[Man telegraphirt uns aus Breslau…]}} für die \textcolor{brown}{N. fr. Pr.}{}\ledrightnote{\textcolor{brown}{Neue Freie Presse}} für
               mich nichts als Arbeit und Ärger herausgekommen. Die \textcolor{green}{Première}{}\ledrightnote{{$\rightarrow$}\textcolor{green}{Der Schleier der Beatrice. Schauspiel in fünf Akten}} dauerte bis 11, ich
               raſte per Wagen nach dem Amt, hielt in Eile die von Dir beſtellten ca 180 Mark hin,
               mußte drängeln, daß ich mit dem einzigen dienſtführenden Beamten, der \substVorne{}\textsuperscript{ſolche}{\allowbreak}\substDazwischen{}lange\substHinten{} Depeſchen nicht gewohnt iſt, zu Rande kam, war erſt nach 12 Uhr
               für die \textsc{\textcolor{green}{Morgen Ztg}{}\ledrightnote{{$\rightarrow$}\textcolor{green}{Neue Freie Presse}}} frei, ſo daß dieſe am meiſten {\pb}zu kurz, ich
               aber erſt um 1 Uhr zum Nachtmahlen kam, und das Reſultat der ganzen
               Schererei war, daß ich am nächſten Tage nur ein
                  \label{K_L02944-1111v}\edtext{Drittel meines Telegra{\geminationm}s}{\lemma{\textnormal{\emph{Drittel … Telegramms}}}\Cendnote{\textnormal{
                     »— Man telegraphirt uns aus \textcolor{pink}{\so{Breslau}}: Arthur \so{Schnitzler}’s neuestes Drama ›\textcolor{green}{\so{Der Schleier der Beatrice}}‹ kam heute nach
                     mehrfachen Verzögerungen auf die Bühne des \textcolor{brown}{\so{Lobe-Theaters}}. Das in Vers und Prosa
                     geſchriebene Werk iſt ein farbenprangendes Renaiſſance-Gemälde von bizarrer
                     Kühnheit. Seine Schönheiten breiten sich wie ein ſchimmernder Mantel über das
                     Gerüſt der Handlung. Das Publicum nahm die drei erſten Acte mit Enthuſiasmus,
                     die beiden letzten aber mit immer ſtärkerem Widerſpruch auf.«
                        [\textcolor{blue}{Erich Freund}:] \emph{\textcolor{green}{[Man telegraphirt uns aus Breslau]}}. In: \emph{\textcolor{green}{Neue Freie Presse}}, Nr. 13.031,
                        2. 12. 1900, S. 10. }}}\label{K_L02944-1111h}, vor allem
               kein Wort über die erbärmliche, ſaumäßige, empörende Aufführung in der \textcolor{green}{N. fr. Pr.}{}\ledrightnote{\textcolor{green}{Neue Freie Presse}} finde. Wahrſcheinlich iſt die \label{K_L02944-23v}\edtext{Freundſchaft für Herrn \textsc{Dr \textcolor{blue}{Löwe}{}\ledrightnote{\textcolor{blue}{Theodor Loewe}}}}{\lemma{\textnormal{\emph{Freundſchaft … Löwe}}}\Cendnote{\textnormal{siehe Paul Goldmann an Arthur Schnitzler, 14. 10. [1900]}}}\label{K_L02944-23h} dort ſo ſtark, daß ſie alle anderen Rückſichten tödtet, ſelbſt die auf
               Schnitzler, der am ſchwerſten durch dieſe lächerliche Vorſtellung geſchädigt wurde
                  \introOben{}und mich darum bat, darauf beſonders hinzuweiſen\introOben{}. Ich
               habe ſoeben an die {\pb}dortige \textcolor{brown}{Redaktion}{}\ledrightnote{{$\rightarrow$}\textcolor{brown}{Neue Freie Presse}} geſchrieben und um Erklärung
               erſucht. Auf ein Honorar verzichte ich gern. Bemerken will ich doch, daß ich nach
               Deiner Anweiſung rechtzeitig um Beihalten des Platzes in der \textcolor{green}{So{\geminationn}tags-Nu{\geminationm}er}{}\ledrightnote{{$\rightarrow$}\textcolor{green}{Neue Freie Presse}} erſucht hatte. Sollten Dich die \introOben{}hieſigen\introOben{} Kritiken über \textcolor{green}{Stück}{}\ledrightnote{{$\rightarrow$}\textcolor{green}{Der Schleier der Beatrice. Schauspiel in fünf Akten}} od Aufführung intereſſiren, ſo ſende ich ſie Dir. Am
                  Dienſtag brachte der \textsc{\textcolor{green}{B. Börsen Cour.}{}\ledrightnote{\textcolor{green}{Berliner Börsen-Courier}}} eine längere \label{K_L02944-9v}\edtext{\textcolor{green}{Beſprechung}{}\ledrightnote{{$\rightarrow$}\textcolor{green}{[Rezension von Erich Freund über Schleier der Beatrice]}}}{\lemma{\textnormal{\emph{Beſprechung}}}\Cendnote{\textnormal{XXXX}}}\label{K_L02944-9h} von mir.\pend
           \pstart
           Es grüßt Dich herzlichſt {\\[\baselineskip]}Dein getreuer {\\[\baselineskip]}\spacefill\mbox{\textcolor{blue}{Freund}{}\ledrightnote{\textcolor{blue}{Erich Freund}}}\pend
           \leftskip=0em{}\endnumbering\briefempfaengerindex{Schnitzler, Arthur@\textsc{Schnitzler, Arthur}!zzzGoldmann, Paul@\emph{von Paul Goldmann}!1900-12-091@{9. 12. {[}1900{]}}|)be}\mylabel{h}\begin{anhang}\end{anhang}\normalsize

\doendnotes{C}
\bigskip
\vfill

\clearpage

\footnotesize

\lohead{\textsc{register}}

% Definiere theindex-Environment komplett neu ohne reledmac
\makeatletter
\renewenvironment{theindex}{%
  \section*{\indexname}%
  \setlength{\parindent}{0pt}%
  \setlength{\parskip}{0pt plus 0.3pt}%
  \let\item\@idxitem
}{%
  \clearpage
}
\makeatother

\IfFileExists{\jobname-pw.ind}{\input{\jobname-pw.ind}}{}

\end{document}

      