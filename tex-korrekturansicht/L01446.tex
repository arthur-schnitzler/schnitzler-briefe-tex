%% latex-korrekturansicht-vorspann.tex
%% Vorspann für die Korrekturansicht.
%% Lädt die gemeinsame Datei latex-vorspann.tex mit gesetztem Schalter.

\newif\ifkorrekturansicht
\korrekturansichttrue

\input{../tex-inputs/latex-vorspann}


               \section[Arthur Schnitzler an Hugo von Hofmannsthal, 16. 9. 1904]{ Arthur Schnitzler an Hugo von Hofmannsthal, 16. 9. 1904}\nopagebreak\mylabel{v}\rehead{ }\normalsize\beginnumbering\briefempfaengerindex{Hofmannsthal, Hugo von@\textsc{Hofmannsthal, Hugo von}!zzzSchnitzler, Arthur@\emph{von Arthur Schnitzler}!1904-09-161@{16. 9. 1904}|(be} \toendnotes[C]{\smallbreak\pagebreak[2]} \Standort{FDH, Hs-30885,114.}
\physDesc{Brief, 1 Blatt, 3 Seiten
\newline{}Handschrift: Bleistift, deutsche Kurrent}\buchAbdrucke{\weitereDrucke{Hugo von Hofmannsthal, Arthur Schnitzler: \emph{Briefwechsel}. Hg. Therese Nickl und Heinrich Schnitzler. Frankfurt am Main: \emph{S. Fischer} 1964, S. 201.} }\toendnotes[C]{\smallbreak}\pstart
           \raggedleft{}{\pb}16. 9. 904{\\}\textcolor{pink}{\textsc{Lueg a Wolfg}ſee}{}\ledrightnote{\textcolor{pink}{Lueg am Wolfgangsee}}\pend
           \pstart
           lieber Hugo, bis heute ſind wir dageblieben, ſeit vorgeſtern arges
               Regenwetter, heute Nm fährt \textcolor{blue}{Richard}{}\ledrightnote{\textcolor{blue}{Richard Beer-Hofmann}}
               vorbei; wir ſteigen zu ihm ein u bleiben noch ein paar Tage in \textcolor{pink}{Salzburg}{}\ledrightnote{\textcolor{pink}{Salzburg}}. Da{\geminationn} wahrſcheinlich direct
                  \textcolor{pink}{Wien}{}\ledrightnote{\textcolor{pink}{Wien}}. Gearbeitet ſo gut wie nichts, aber große {\pb}Sehnſucht danach. Mit \textcolor{blue}{Burckhard}{}\ledrightnote{\textcolor{blue}{Max Eugen Burckhard}} ein paar ſehr angenehme Stunden. Das Rad ununterbrochen ſchwer
               krank – es zeigte sich daſs die Tretkurbel u noch einiges andre total hin war. Bin
                  \uline{ein} Mal von \textcolor{pink}{\textsc{St. Gilgen}}{}\ledrightnote{\textcolor{pink}{St. Gilgen}} nach \textcolor{pink}{\textsc{Lueg}}{}\ledrightnote{\textcolor{pink}{Lueg am Wolfgangsee}} gefahren. Jetzt iſt es ganz in Ordnung und wird wahrſcheinlich auf der
               Eiſenbahn zer{\pb}trümmert werden. Ihre (eine) Karte
               erhalten. Ob Sie ſchönes Wetter auf der Tour gehabt haben? Eine neulich gekommene
               Karte leg ich bei.\pend
           \pstart
           Laſſen Sie ſehr bald nach \textcolor{pink}{Wien}{}\ledrightnote{\textcolor{pink}{Wien}} einiges vernehmen.\pend
           \pstart
           Wir grüßen Sie \textcolor{blue}{Beide}{}\ledrightnote{→\textcolor{blue}{Olga Schnitzler}}{ }\textcolor{blue}{Beide}{}\ledrightnote{→\textcolor{blue}{Paula Beer-Hofmann}}.\pend
           \pstart
           Herzlichſt Ihr{\\[\baselineskip]}\spacefill\mbox{A.}\pend
           \leftskip=0em{}\endnumbering\briefempfaengerindex{Hofmannsthal, Hugo von@\textsc{Hofmannsthal, Hugo von}!zzzSchnitzler, Arthur@\emph{von Arthur Schnitzler}!1904-09-161@{16. 9. 1904}|)be}\mylabel{h}  \normalsize

\doendnotes{C}
\bigskip
\vfill

\clearpage

\footnotesize

\lohead{\textsc{register}}

% Definiere theindex-Environment komplett neu ohne reledmac
\makeatletter
\renewenvironment{theindex}{%
  \section*{\indexname}%
  \setlength{\parindent}{0pt}%
  \setlength{\parskip}{0pt plus 0.3pt}%
  \let\item\@idxitem
}{%
  \clearpage
}
\makeatother

\IfFileExists{\jobname-pw.ind}{\input{\jobname-pw.ind}}{}

\end{document}

      