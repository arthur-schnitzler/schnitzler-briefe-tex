%% latex-korrekturansicht-vorspann.tex
%% Vorspann für die Korrekturansicht.
%% Lädt die gemeinsame Datei latex-vorspann.tex mit gesetztem Schalter.

\newif\ifkorrekturansicht
\korrekturansichttrue

\input{../tex-inputs/latex-vorspann}


\renewcommand{\erwaehntePersonen}{Personen: Maximilian Harden, Hermann Sudermann}
\renewcommand{\erwaehnteOrte}{Orte: Berlin, Frankgasse, Wien}
\renewcommand{\erwaehnteWerke}{Werke: Der Schleier der Beatrice. Schauspiel in fünf Akten, Die Zukunft, Lebendige Stunden. Vier Einakter, Theater}
\section[ Paul Goldmann an Arthur Schnitzler, 21. 3. 1902]{Paul Goldmann an Arthur Schnitzler, 21. 3. 1902}
\nopagebreak\mylabel{v}
\rehead{ }\normalsize\beginnumbering\briefempfaengerindex{Schnitzler, Arthur@\textsc{Schnitzler, Arthur}!zzzGoldmann, Paul@\emph{von Paul Goldmann}!1902-03-211@{21. 3. 1902}|(be}
\toendnotes[C]{\smallbreak\pagebreak[2]}\Standort{DLA, A:Schnitzler, HS.NZ85.1.3172.}
\physDesc{Postkarte
\newline{}Handschrift: 1) blaue Tinte, deutsche Kurrent\hspace{1em}2) blaue Tinte, lateinische Kurrent (\noindent{}Adresse)\hspace{1em}
\newline{}Versand: 1) Stempel: »\nobreak{}\oindex{Berlin@\textbf{Berlin}, \emph{https://www.geonames.org/ontologyP.PPLC}|pwk}Berlin S. W. 46, 21. 3. 02, 12–1N.\nobreak{}«.   2) Stempel: »\nobreak{}9/3 Wien 7{[}2{]}, 22. 3. {[}1902{]}, 11., Beste{[}llt{]}\nobreak{}«. }\toendnotes[C]{\smallbreak}\pstart{}{\pb}Herrn\pend{}\pstart{}Dr. Arthur Schnitzler\pend{}\pstart{}\textcolor{pink}{Wien}{}\ledrightnote{\textcolor{pink}{Wien}}\pend{}\pstart{}\textcolor{pink}{IX. Frankgaſse 1}{}\ledrightnote{\textcolor{pink}{Frankgasse}}.\pend{}
{\bigskip}
\pstart
           {\pb}21. 3. 1902.\pend
           
\pstart{}Mein lieber Freund,\pend
\pstart
           Im ſoeben erſchienenen Heft der »\textcolor{green}{Zukunft}{}\ledrightnote{\textcolor{green}{Die Zukunft}}« (ich
               habe es nicht zur Hand u. kann es Dir daher nicht ſchicken) ſagt \textsc{\textcolor{blue}{Harden}{}\ledrightnote{\textcolor{blue}{Maximilian Harden}}} gegen Schluß ſeines \label{K_L03201-1v}\edtext{\textcolor{green}{Theaterartikel}{}\ledrightnote{{$\rightarrow$}\textcolor{green}{Theater}}}{\lemma{\textnormal{\emph{Theaterartikel}}}\Cendnote{\textnormal{\textcolor{blue}{M. H.} [=\textcolor{blue}{Maximilian Harden}]: \emph{\textcolor{green}{Theater}}. In: \emph{\textcolor{green}{Die
                        Zukunft}}, Jg. 38, 22. 3. 1902,
                     S. 490–498, hier: S. 497: »Herr \textcolor{blue}{Arthur Schnitzler}, den der Erfolg doch schon bekannt gemacht und
                     gesegnet hat, harrt vergebens noch immer der Stunde, die sein reifstes Werk,
                     den ›\textcolor{green}{Schleier der Beatrice}‹, auf einer
                     großen Bühne zum Leben erweckt. Und seine ›\textcolor{green}{Lebendigen Stunden}‹, drei sehr feine und ein effektvoller Einakter,
                     von denen noch zu reden sein wird, mußten nach kurzer Frist dem
                     Coulissenschmöker des Kollegen \textcolor{blue}{Sudermann}
                     weichen.«}}}\label{K_L03201-1h}s einige freundliche Worte über den »\textcolor{green}{Schleier der \textsc{Beatrice}}{}\ledrightnote{\textcolor{green}{Der Schleier der Beatrice. Schauspiel in fünf Akten}}«.\pend
           
\pstart
           Viele Grüße! Dein {\\[\baselineskip]}\spacefill\mbox{P. G.}\pend
           \leftskip=0em{}\endnumbering\briefempfaengerindex{Schnitzler, Arthur@\textsc{Schnitzler, Arthur}!zzzGoldmann, Paul@\emph{von Paul Goldmann}!1902-03-211@{21. 3. 1902}|)be}\mylabel{h}  \normalsize

\doendnotes{C}
\bigskip
\vfill

\clearpage

\footnotesize

\lohead{\textsc{register}}

% Definiere theindex-Environment komplett neu ohne reledmac
\makeatletter
\renewenvironment{theindex}{%
  \section*{\indexname}%
  \setlength{\parindent}{0pt}%
  \setlength{\parskip}{0pt plus 0.3pt}%
  \let\item\@idxitem
}{%
  \clearpage
}
\makeatother

\IfFileExists{\jobname-pw.ind}{\input{\jobname-pw.ind}}{}

\end{document}

      