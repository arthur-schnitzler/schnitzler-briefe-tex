%% latex-korrekturansicht-vorspann.tex
%% Vorspann für die Korrekturansicht.
%% Lädt die gemeinsame Datei latex-vorspann.tex mit gesetztem Schalter.

\newif\ifkorrekturansicht
\korrekturansichttrue

\input{../tex-inputs/latex-vorspann}


               \section[Richard Beer-Hofmann an Arthur Schnitzler, 11. 8. 1901]{ Richard Beer-Hofmann an Arthur Schnitzler,
                    11. 8. 1901}\nopagebreak\mylabel{v}\rehead{ }\normalsize\beginnumbering\briefempfaengerindex{Schnitzler, Arthur@\textsc{Schnitzler, Arthur}!zzzBeer-Hofmann, Richard@\emph{von Richard Beer-Hofmann}!1901-08-111@{11.  8. 1901}|(be} \toendnotes[C]{\smallbreak\pagebreak[2]} \Standort{CUL, Schnitzler, B 8.}
\physDesc{Briefkarte
\newline{}Handschrift: Bleistift, lateinische Kurrent\newline{}Ordnung: mit Bleistift von unbekannter Hand nummeriert: »167« }\pstart
           \raggedleft{}{\pb}\textcolor{pink}{Pörtschach}{}\ledrightnote{\textcolor{pink}{Pörtschach}}{ }11/VIII 01\pend
           \pstart
           Lieber Arthur! Es ist immerhin möglich daß ich auf ein paar
                    Stunden nach \textcolor{pink}{Welsberg}{}\ledrightnote{\textcolor{pink}{Welsberg-Taisten}} ko{\geminationm}e. Aber die {\pb}Nähe wäre ja auch ein Grund
                    für Sie zu ko{\geminationm}en, nicht?\pend
           \pstart
           Herzlich grüßt Sie und \textcolor{blue}{Paul}{}\ledrightnote{\textcolor{blue}{Paul Goldmann}}{ }\uline{Ihr}{\\[\baselineskip]}\spacefill\mbox{Richard}\pend
           \leftskip=0em{}\endnumbering\briefempfaengerindex{Schnitzler, Arthur@\textsc{Schnitzler, Arthur}!zzzBeer-Hofmann, Richard@\emph{von Richard Beer-Hofmann}!1901-08-111@{11.  8. 1901}|)be}\mylabel{h}  \normalsize

\doendnotes{C}
\bigskip
\vfill

\clearpage

\footnotesize

\lohead{\textsc{register}}

% Definiere theindex-Environment komplett neu ohne reledmac
\makeatletter
\renewenvironment{theindex}{%
  \section*{\indexname}%
  \setlength{\parindent}{0pt}%
  \setlength{\parskip}{0pt plus 0.3pt}%
  \let\item\@idxitem
}{%
  \clearpage
}
\makeatother

\IfFileExists{\jobname-pw.ind}{\input{\jobname-pw.ind}}{}

\end{document}

      