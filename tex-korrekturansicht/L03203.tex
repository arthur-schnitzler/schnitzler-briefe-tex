%% latex-korrekturansicht-vorspann.tex
%% Vorspann für die Korrekturansicht.
%% Lädt die gemeinsame Datei latex-vorspann.tex mit gesetztem Schalter.

\newif\ifkorrekturansicht
\korrekturansichttrue

\input{../tex-inputs/latex-vorspann}


\renewcommand{\erwaehntePersonen}{Personen: Richard Beer-Hofmann, Charlotte Bondy, Willi Handl, Hugo von Hofmannsthal, Hugo Salus, Olga Schnitzler, Carl Seltmann, Heinrich Teweles, Alice Ziegler, Arnost Ziegler}
\renewcommand{\erwaehnteOrte}{Orte: Berlin, Brühl, Böhmerwald, Hotel Blauer Stern, Prag, Wien}
\renewcommand{\erwaehnteWerke}{Werke: Die Zeit, Neue Freie Presse}
\section[ Paul Goldmann an Arthur Schnitzler, 1. 4. {[}1902{]}]{Paul Goldmann an Arthur Schnitzler, 1. 4. {[}1902{]}}
\nopagebreak\mylabel{v}
\rehead{ }\normalsize\beginnumbering\briefempfaengerindex{Schnitzler, Arthur@\textsc{Schnitzler, Arthur}!zzzGoldmann, Paul@\emph{von Paul Goldmann}!1902-04-013@{1. 4. {[}1902{]}}|(be}
\toendnotes[C]{\smallbreak\pagebreak[2]}\Standort{DLA, A:Schnitzler, HS.NZ85.1.3172.}
\physDesc{Brief, 1 Blatt, 2 Seiten
\newline{}Handschrift: schwarze Tinte, deutsche Kurrent
\newline{}Schnitzler: 1) mit Bleistift das Jahr »{[}1{]}902« und auf der letzten Bogenseite »\textcolor{gray}{Losemann}.«, »\textcolor{gray}{Haus \textcolor{blue}{Hugo}}.«, »\textcolor{gray}{Elte}{[}rn{]}.«, »\textcolor{blue}{Handl}.« und
                                       »Feu\textcolor{gray}{i}ll
                                    Tg\textcolor{gray}{bl}.« vermerkt  2) mit rotem Buntstift vier Unterstreichungen}\toendnotes[C]{\smallbreak}
\pstart
           \noindent{}{\pb}\textcolor{gray}{\textbf{\textsc{\textcolor{pink}{HÔTEL BLAUER STERN}{}\ledrightnote{\textcolor{pink}{Hotel Blauer Stern}}}}}\pend
           
\pstart
           \textcolor{gray}{\textbf{\textsc{\textbf{\textcolor{blue}{CARL SELTMANN}{}\ledrightnote{\textcolor{blue}{Carl Seltmann}}.}}}}\pend
           
\pstart
           \raggedleft{}\textcolor{gray}{\textbf{\textcolor{pink}{Prag}{}\ledrightnote{\textcolor{pink}{Prag}},}}{ }1. April.\pend
           
\pstart
           \textcolor{gray}{\textbf{TELEGRAMM-ADRESSE:}}\pend
           
\pstart
           \textcolor{gray}{\textbf{\textsc{\textcolor{pink}{STERNHÔTEL PRAG}{}\ledrightnote{\textcolor{pink}{Hotel Blauer Stern}}.}}}\pend
           
\pstart{}Mein lieber Freund,\pend
\pstart
           Ich habe einige angenehme Tage verlebt in einer ſchönen \textcolor{pink}{Stadt}{}\ledrightnote{{$\rightarrow$}\textcolor{pink}{Prag}} mit lieben Menſchen. Morgen fahre ich wieder heim.\pend
           
\pstart
           Ich habe viel von Dir geſprochen. \textsc{\textcolor{blue}{Salus}{}\ledrightnote{\textcolor{blue}{Hugo Salus}}} (ein kluger und ſympathiſcher Menſch unter einer Schicht von Affektirtheit)
               läßt Dich und \textsc{\textcolor{blue}{Richard}{}\ledrightnote{\textcolor{blue}{Richard Beer-Hofmann}}} grüßen. Ebenſo \textsc{\textcolor{blue}{Teweles}{}\ledrightnote{\textcolor{blue}{Heinrich Teweles}}} und \textsc{\textcolor{blue}{Bondy}{}\ledrightnote{{$\rightarrow$}\textcolor{blue}{Charlotte Bondy}{\newline}{$\rightarrow$}\textcolor{blue}{Alice Ziegler}}}, \textcolor{blue}{Mutter}{}\ledrightnote{{$\rightarrow$}\textcolor{blue}{Charlotte Bondy}} und \textcolor{blue}{Tochter}{}\ledrightnote{{$\rightarrow$}\textcolor{blue}{Alice Ziegler}}.\pend
           
\pstart
           \textsc{\textcolor{blue}{Alice}{}\ledrightnote{\textcolor{blue}{Alice Ziegler}}} iſt ein ſchönes Mädchen geworden und auch günſtig gereift. Ich war ein Thor
               ohnegleichen, daß ich ſie \label{K_L03203-1v}\edtext{nicht
                  geheiratet}{\lemma{\textnormal{\emph{nicht
                  geheiratet}}}\Cendnote{\textnormal{siehe Paul Goldmann an Arthur Schnitzler, 16. 1. [1902]}}}\label{K_L03203-1h} habe. Sie wäre die Frau geweſen, wie ich ſie mir immer ausgedacht habe. In
               der Kunſt, die Gelegenheiten zu verträumen, iſt mir Keiner über. Sie hat ſich als
               Bräutigam eine Art \textcolor{blue}{Kraftmenſch}{}\ledrightnote{{$\rightarrow$}\textcolor{blue}{Arnost Ziegler}} ausgeſucht, der mir {\pb}ſehr
               unſympathiſch iſt. Aber es iſt ganz natürlich. \label{K_L03203-2v}\edtext{\begin{otherlanguage}{french}\textsc{Très-femelle}\end{otherlanguage}}{\lemma{\textnormal{\emph{Très-femelle}}}\Cendnote{\textnormal{französisch: sehr weiblich}}}\label{K_L03203-2h}, wie
                  \textcolor{blue}{ſie}{}\ledrightnote{{$\rightarrow$}\textcolor{blue}{Alice Ziegler}} iſt, hat ihr
               Inſtinkt \strikeout{ſ\textcolor{gray}{e}\textcolor{gray}{×}} ſie zu dem \textcolor{blue}{Gegenpol}{}\ledrightnote{{$\rightarrow$}\textcolor{blue}{Arnost Ziegler}}{ }\label{K_L03203-3v}\edtext{\begin{otherlanguage}{french}\textsc{très-mâle}\end{otherlanguage}}{\lemma{\textnormal{\emph{très-mâle}}}\Cendnote{\textnormal{französisch: sehr männlich}}}\label{K_L03203-3h}
               geleitet.\pend
           
\pstart
           Von der »\textcolor{green}{Neuen Freien Preſſe}{}\ledrightnote{\textcolor{green}{Neue Freie Presse}}« höre ich hier ſo
               viel Schlechtes und von der »\textcolor{green}{Zeit}{}\ledrightnote{\textcolor{green}{Die Zeit}}« ſo viel
               Gutes, daß ich in ſchweren \label{K_L03203-6v}\edtext{Sorgen}{\lemma{\textnormal{\emph{Sorgen}}}\Cendnote{\textnormal{womöglich Bezug auf \textcolor{blue}{Goldmann}s Angst, die \emph{\textcolor{green}{Zeit}} könnte die \emph{\textcolor{green}{Neue Freie Presse}}
                  ablösen, siehe Paul Goldmann an Arthur Schnitzler und Olga
               Gussmann, 7. 7. [1901]}}}\label{K_L03203-6h} heimfahre!\pend
           
\pstart
           Wie geht es Dir, mein lieber Freund? Es thut mir unendlich leid, daß ich Dir nicht
               habe die Hand drücken können. Die Leute ſprechen \textcolor{pink}{hier}{}\ledrightnote{{$\rightarrow$}\textcolor{pink}{Prag}} nicht nur mit Liebe von deinem Talent, ſondern auch mit
               Reſpekt von Deinem (künſtleriſchen und moraliſchen) Charakter\strikeout{)}.\pend
           
\pstart
           Schreib’ mir nach \textcolor{pink}{Berlin}{}\ledrightnote{\textcolor{pink}{Berlin}}. Was macht \textsc{\textcolor{blue}{Olga}{}\ledrightnote{\textcolor{blue}{Olga Schnitzler}}}? Grüße ſie vielmals.\pend
           
\pstart
           In den \label{K_L03203-9v}\edtext{\textcolor{pink}{Böhmerwald}{}\ledrightnote{\textcolor{pink}{Böhmerwald}}}{\lemma{\textnormal{\emph{Böhmerwald}}}\Cendnote{\textnormal{Bezug unklar}}}\label{K_L03203-9h} werde ich mit \textcolor{blue}{Euch}{}\ledrightnote{{$\rightarrow$}\textcolor{blue}{Olga Schnitzler}} leider
               nicht gehen können. Aber ich rechne ſicher darauf, \textcolor{blue}{Euch}{}\ledrightnote{{$\rightarrow$}\textcolor{blue}{Olga Schnitzler}}\strikeout{\textcolor{gray}{an}} in \label{K_L03203-11v}\edtext{\textcolor{pink}{Berlin}{}\ledrightnote{\textcolor{pink}{Berlin}}}{\lemma{\textnormal{\emph{Berlin}}}\Cendnote{\textnormal{In \textcolor{pink}{Berlin} sahen sich \textcolor{blue}{Goldmann} und \textcolor{blue}{Schnitzler} zwischen 13. 10. 1902 und 18. 10. 1902 täglich.
                  Davor war \textcolor{blue}{Goldmann} von 18. 5. 1902 bis
                  jedenfalls 25. 5. 1902 in \textcolor{pink}{Wien} bzw. der \textcolor{pink}{Brühl}.}}}\label{K_L03203-11h} zu ſehen.\pend
           
\pstart
           Viele treue Grüße! Dein {\\[\baselineskip]}\spacefill\mbox{Paul Goldmann}\pend
           \leftskip=0em{}\endnumbering\briefempfaengerindex{Schnitzler, Arthur@\textsc{Schnitzler, Arthur}!zzzGoldmann, Paul@\emph{von Paul Goldmann}!1902-04-013@{1. 4. {[}1902{]}}|)be}\mylabel{h}
\begin{anhang}
\end{anhang}\normalsize

\doendnotes{C}
\bigskip
\vfill

\clearpage

\footnotesize

\lohead{\textsc{register}}

% Definiere theindex-Environment komplett neu ohne reledmac
\makeatletter
\renewenvironment{theindex}{%
  \section*{\indexname}%
  \setlength{\parindent}{0pt}%
  \setlength{\parskip}{0pt plus 0.3pt}%
  \let\item\@idxitem
}{%
  \clearpage
}
\makeatother

\IfFileExists{\jobname-pw.ind}{\input{\jobname-pw.ind}}{}

\end{document}

      