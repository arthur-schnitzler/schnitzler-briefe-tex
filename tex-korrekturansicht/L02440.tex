%% latex-korrekturansicht-vorspann.tex
%% Vorspann für die Korrekturansicht.
%% Lädt die gemeinsame Datei latex-vorspann.tex mit gesetztem Schalter.

\newif\ifkorrekturansicht
\korrekturansichttrue

\input{../tex-inputs/latex-vorspann}


               \section[Arthur Schnitzler an Gertrud Rung, 22. 5. 1925]{ Arthur Schnitzler an Gertrud Rung, 22. 5. 1925}\nopagebreak\mylabel{v}\rehead{ }\normalsize\beginnumbering\briefempfaengerindex{Rung, Gertrud@\textsc{Rung, Gertrud}!zzzSchnitzler, Arthur@\emph{von Arthur Schnitzler}!1925-05-221@{22. 5. 1925}|(be} \toendnotes[C]{\smallbreak\pagebreak[2]} \Standort{Kopenhagen, Det Kongelige Bibliotek, Georg Brandes Arkiv, box 125.}
\physDesc{Brief, 1 Blatt, 1 Seite
\newline{}Handschrift: schwarze Tinte, lateinische Kurrent\newline{}Ordnung: von unbekannter Hand nummeriert: »51b« und über die Monatsangabe des
                                    Datums zur Verdeutlichung »5« geschrieben }\buchAbdrucke{\weitereDrucke{Georg Brandes, Arthur Schnitzler: \emph{Ein Briefwechsel}. Hg. Kurt Bergel. Bern: \emph{Francke} 1956, S. 145.} }\toendnotes[C]{\smallbreak}\pstart
           \raggedleft{}{\pb}\textcolor{pink}{Wien}{}\ledrightnote{\textcolor{pink}{Wien}}, 22. 5. 925\pend
           \pstart{}verehrte Frau Rung,\pend\pstart
           darf ich um ein Wort bitten, wie sich \textcolor{blue}{Georg
                        Brandes}{}\ledrightnote{\textcolor{blue}{Georg Brandes}} befindet? Wie es Ihnen überhaupt in \textcolor{pink}{Salzburg}{}\ledrightnote{\textcolor{pink}{Salzburg}} behagt? Mir sind die \label{K_L02440_1v}\edtext{paar Stunden}{\lemma{\textnormal{\emph{paar Stunden}}}\Cendnote{\textnormal{\textcolor{blue}{Brandes} war den ganzen
                            April und bis zum Anfang Mai in \textcolor{pink}{Wien}. In dieser Zeit sahen sich er und \textcolor{blue}{Schnitzler} regelmäßig.}}}\label{K_L02440_1h}, die ich in \textcolor{pink}{Wien}{}\ledrightnote{\textcolor{pink}{Wien}} mit \textcolor{blue}{Brandes}{}\ledrightnote{\textcolor{blue}{Georg Brandes}} verbringen durfte, wieder eine besonders schöne Erinnerung,
                    und auch Ihnen, verehrte liebe Frau Rung hab ich für Ihre Liebenswürdigkeit sehr
                    herzlich zu danken!\pend
           \pstart
           Hoffentlich begegnet man einander bald wieder! Viele Grüße Ihnen und \textcolor{blue}{Georg Brandes}{}\ledrightnote{\textcolor{blue}{Georg Brandes}}.\pend
           \pstart
           Ihr{\\[\baselineskip]}\spacefill\mbox{Arthur Schnitzler}\pend
           \leftskip=0em{}\endnumbering\briefempfaengerindex{Rung, Gertrud@\textsc{Rung, Gertrud}!zzzSchnitzler, Arthur@\emph{von Arthur Schnitzler}!1925-05-221@{22. 5. 1925}|)be}\mylabel{h}  \normalsize

\doendnotes{C}
\bigskip
\vfill

\clearpage

\footnotesize

\lohead{\textsc{register}}

% Definiere theindex-Environment komplett neu ohne reledmac
\makeatletter
\renewenvironment{theindex}{%
  \section*{\indexname}%
  \setlength{\parindent}{0pt}%
  \setlength{\parskip}{0pt plus 0.3pt}%
  \let\item\@idxitem
}{%
  \clearpage
}
\makeatother

\IfFileExists{\jobname-pw.ind}{\input{\jobname-pw.ind}}{}

\end{document}

      