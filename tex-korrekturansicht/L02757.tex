%% latex-korrekturansicht-vorspann.tex
%% Vorspann für die Korrekturansicht.
%% Lädt die gemeinsame Datei latex-vorspann.tex mit gesetztem Schalter.

\newif\ifkorrekturansicht
\korrekturansichttrue

\input{../tex-inputs/latex-vorspann}


               \section[Paul Goldmann an Arthur Schnitzler, 29. 11. {[}1895{]}]{ Paul Goldmann an Arthur Schnitzler, 29. 11. {[}1895{]}}\nopagebreak\mylabel{v}\rehead{ }\normalsize\beginnumbering\briefempfaengerindex{Schnitzler, Arthur@\textsc{Schnitzler, Arthur}!zzzGoldmann, Paul@\emph{von Paul Goldmann}!1895-11-291@{29. 11. {[}1895{]}}|(be} \toendnotes[C]{\smallbreak\pagebreak[2]} \Standort{DLA, A:Schnitzler, HS.NZ85.1.3165.}
\physDesc{Brief, 1 Blatt, 4 Seiten
\newline{}Handschrift: blaue Tinte, deutsche Kurrent
\newline{}Schnitzler: 1) mit Bleistift das Jahr » 95« vermerkt 2) mit rotem Buntstift eine Unterstreichung sowie »\textcolor{green}{Liebelei}« auf der zweiten Seite umrahmt und dazu »\textsc{\textcolor{green}{Lbl}}« (Kürzel für \textcolor{green}{Liebelei}) vermerkt}\toendnotes[C]{\smallbreak}\pstart
           \noindent{}{\pb}\textcolor{gray}{\textbf{\textbf{\textcolor{brown}{Frankfurter Zeitung}{}\ledrightnote{\textcolor{brown}{Frankfurter Zeitung}}}}}\pend
           \pstart
           \textcolor{gray}{\textbf{(\textcolor{brown}{\begin{otherlanguage}{french}Gazette de Francfort\end{otherlanguage}}{}\ledrightnote{\textcolor{brown}{Frankfurter Zeitung}}). }}\pend
           \pstart
           \textcolor{gray}{\textbf{\textbf{\begin{otherlanguage}{french}Fondateur M. \textcolor{blue}{L.
                              Sonnemann}{}\ledrightnote{\textcolor{blue}{Leopold Sonnemann}}\end{otherlanguage}.}}}\pend
           \pstart
           \begin{otherlanguage}{french}\textcolor{gray}{\textbf{\textcolor{green}{Journal}{}\ledrightnote{→\textcolor{green}{Frankfurter Zeitung}} politique, financier,}}\end{otherlanguage}\pend
           \pstart
           \begin{otherlanguage}{french}\textcolor{gray}{\textbf{commercial et littéraire.}}\end{otherlanguage}\pend
           \pstart
           \begin{otherlanguage}{french}\textcolor{gray}{\textbf{\textbf{Paraissant trois fois par jour.}}}\end{otherlanguage}\pend
           \pstart
           \begin{otherlanguage}{french}\textcolor{gray}{\textbf{\textbf{Bureau à \textcolor{pink}{Paris}{}\ledrightnote{\textcolor{pink}{Paris}}:}}}\end{otherlanguage}\pend
           \pstart
           \begin{otherlanguage}{french}\textcolor{gray}{\textbf{\textbf{\textcolor{pink}{24. Rue Feydeau}{}\ledrightnote{\textcolor{pink}{rue Feydeau}}.}}}\end{otherlanguage}\hfill \textsc{\textcolor{pink}{Paris}{}\ledrightnote{\textcolor{pink}{Paris}}}, 29. November.\pend
           \pstart\center{}Mein lieber Freund,\pend\pstart
           Dieſen Deinen Brief habe ich mit Sorge aufgemacht. Was wirſt Du ſagen? Ich bin ſo
               ſchuldbewußt! Aber ich finde keinen Vorwurf. Gott ſei Dank’.\pend
           \pstart
           Tolle Arbeit, liebſter Freund, ſolle Arbeit und wüſtes Leben. Ich komme zu nichts
               mehr. Aber in einigen Tagen ſchreibe ich Dir doch.\pend
           \pstart
           Hier die Druckſachen. Die Bemerkungen dazu muß ich mir für ſpäter aufſparen. Denn
               gleich geht die \textcolor{brown}{Kammer}{}\ledrightnote{\textcolor{brown}{Französische Abgeordnetenkammer}} an. {\pb}Die \textcolor{green}{Überſetzung}{}\ledrightnote{→\textcolor{green}{La Petite comédie. Mœurs viennois}} der »\textcolor{green}{Liebelei}{}\ledrightnote{\textcolor{green}{Liebelei. Schauspiel in drei Akten}}« ſinde ich vorzüglich. Schreib’, bitte, an Frau \textsc{\textcolor{blue}{Aubry}{}\ledrightnote{\textcolor{blue}{[MMe. Georges] Aubry}}} – deutſch – ein artiges Wort darüber; danke auch dem \textcolor{blue}{Manne}{}\ledrightnote{→\textcolor{blue}{Georges Aubry}}, daß er es in die »\textsc{\textcolor{green}{Liberté}{}\ledrightnote{\textcolor{green}{La Liberté}}}« gebracht hat; denn das war nicht leicht \strikeout{durz}
               durchzuſetzen bei dem prüden u. etwas chauviniſtiſchen \textcolor{green}{\textsc{Bourgéois}-Blatte}{}\ledrightnote{→\textcolor{green}{La Liberté}}. \introOben{}(Adreſſe \textsc{\textcolor{pink}{10 Rue Caron}{}\ledrightnote{\textcolor{pink}{rue Caron}}}).\introOben{} Die Exemplare will ich Dir zu verſchaffen ſuchen; aber ich fürchte,
               man wird ſie zahlen müſſen.\pend
           \pstart
           {\pb}Vielen Dank für die \textsc{\textcolor{blue}{Strauss}{}\ledrightnote{\textcolor{blue}{Johann Strauss}}}-Empfehlung. Auch hat mir \textsc{\textcolor{blue}{Richard}{}\ledrightnote{\textcolor{blue}{Richard Beer-Hofmann}}} den \label{K_L02757-1v}\edtext{\textsc{\textcolor{blue}{Hogarth}{}\ledrightnote{\textcolor{blue}{William Hogarth}}}}{\lemma{\textnormal{\emph{Hogarth}}}\Cendnote{\textnormal{nicht ermittelt}}}\label{K_L02757-1h} geſchickt, wofür
               ich ihm von Herzen danke. Auch ihm ſchreibe ich einen dieſer Tage.\pend
           \pstart
           \textsc{\textcolor{blue}{Herzl}{}\ledrightnote{\textcolor{blue}{Theodor Herzl}}} war hier. Er iſt mir unſagbar widerwärtig.\pend
           \pstart
           Wüſtes Leben, mein lieber Freund. Ich will in \textsc{\textcolor{pink}{Paris}{}\ledrightnote{\textcolor{pink}{Paris}}} verſchwinden; will mich gegen draußen abſperren, von wo mir jeder Luftzug die
               Kunde meiner {\pb}verfehlten Exiſtenz bringt. Bin müde,
               zu kämpfen, und möchte leben, oh nur ein einziges Mal!\pend
           \pstart
           Grüß’ Dich Gott! {\\[\baselineskip]}Dein {\\[\baselineskip]}treuer {\\[\baselineskip]}\spacefill\mbox{Paul Goldmann}\pend
           \leftskip=0em{}\pstart
           \noindent{}Viele Grüße an die liebe \textcolor{blue}{Frau}{}\ledrightnote{→\textcolor{blue}{Marie Reinhard}}, die wieder in \textsc{\textcolor{pink}{Wien}{}\ledrightnote{\textcolor{pink}{Wien}}} iſt.\pend
           \endnumbering\briefempfaengerindex{Schnitzler, Arthur@\textsc{Schnitzler, Arthur}!zzzGoldmann, Paul@\emph{von Paul Goldmann}!1895-11-291@{29. 11. {[}1895{]}}|)be}\mylabel{h}\begin{anhang}\end{anhang}\normalsize

\doendnotes{C}
\bigskip
\vfill

\clearpage

\footnotesize

\lohead{\textsc{register}}

% Definiere theindex-Environment komplett neu ohne reledmac
\makeatletter
\renewenvironment{theindex}{%
  \section*{\indexname}%
  \setlength{\parindent}{0pt}%
  \setlength{\parskip}{0pt plus 0.3pt}%
  \let\item\@idxitem
}{%
  \clearpage
}
\makeatother

\IfFileExists{\jobname-pw.ind}{\input{\jobname-pw.ind}}{}

\end{document}

      