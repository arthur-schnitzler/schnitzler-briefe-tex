%% latex-korrekturansicht-vorspann.tex
%% Vorspann für die Korrekturansicht.
%% Lädt die gemeinsame Datei latex-vorspann.tex mit gesetztem Schalter.

\newif\ifkorrekturansicht
\korrekturansichttrue

\input{../tex-inputs/latex-vorspann}


\renewcommand{\erwaehntePersonen}{Personen: Lili Cappellini, Johann Peter Eckermann, Johann Wolfgang von Goethe, Olga Schnitzler, Heinrich Schnitzler, Stefan Zweig}
\renewcommand{\erwaehnteOrte}{Orte: Bad Pyrmont, Kochgasse 8, Lügde, Sternwartestraße 71, Weimar, Wien}
\renewcommand{\erwaehnteWerke}{Werke: Der Ruf des Lebens. Schauspiel in drei Akten, Tag- und Jahreshefte, Tagebuch}
\section[Stefan Zweig an Arthur Schnitzler, {[}5. oder 6. 10. 1910?{]}]{Stefan Zweig an Arthur Schnitzler, {[}5. oder 6. 10. 1910?{]}}
\nopagebreak\mylabel{v}
\rehead{ }\normalsize\beginnumbering\briefempfaengerindex{Schnitzler, Arthur@\textsc{Schnitzler, Arthur}!zzzZweig, Stefan@\emph{von Stefan Zweig}!1910-10-051@{{[}5. oder 6. 10. 1910?{]}}|(be}
\toendnotes[C]{\smallbreak\pagebreak[2]}\Standort{CUL, Schnitzler, B 118.}
\physDesc{Brief, 1 Blatt, 2 Seiten, 930 Zeichen
\newline{}Handschrift: lila Tinte, lateinische Kurrent
\newline{}Schnitzler: mit Bleistift »\textsc{Zweig}« }\toendnotes[C]{\smallbreak}
\pstart
           {\pb}\textcolor{gray}{\textbf{SZ}}\hfill \textcolor{gray}{\textbf{\textcolor{pink}{VIII. KOCHGASSE 8}{}\ledrightnote{\textcolor{pink}{Kochgasse 8}}}}\pend
           
\pstart
           \raggedleft{}\textcolor{gray}{\textbf{\textcolor{pink}{WIEN}{}\ledrightnote{\textcolor{pink}{Wien}},}}\pend
           
\pstart{}Sehr verehrter Herr Doktor,\pend
\pstart
           als Sie Ihr schönes \label{K_L03626-1v}\edtext{\textcolor{pink}{Haus}{}\ledrightnote{\textcolor{pink}{Sternwartestraße 71}} bezogen}{\lemma{\textnormal{\emph{Haus bezogen}}}\Cendnote{\textnormal{Am 16. 7. 1910 übersiedelte \textcolor{blue}{Arthur Schnitzler} mit seiner Frau \textcolor{blue}{Olga} und den Kindern \textcolor{blue}{Heinrich} und \textcolor{blue}{Lili} in die selbst
                  erworbene Villa in der \textcolor{pink}{Sternwartestraße
                  71}.}}}\label{K_L03626-1h} und ich es zum erstenmal \label{K_L03626-2v}\edtext{sehen durfte}{\lemma{\textnormal{\emph{sehen durfte}}}\Cendnote{\textnormal{Am
                     12. 9. 1910
                  dokumentiert \textcolor{blue}{Schnitzler} einen Besuch \textcolor{blue}{Stefan Zweigs} im \emph{\textcolor{green}{Tagebuch}}.}}}\label{K_L03626-2h}, sagte ich Ihnen von dem kleinen
               Schmuckstück, das ich mir dafür ausgedacht hatte. Es sollte der \label{K_L03626-3v}\edtext{\uline{Hausspruch} von \textcolor{blue}{\uline{Goethe}}{}\ledrightnote{\textcolor{blue}{Johann Wolfgang von Goethe}}}{\lemma{\textnormal{\emph{Hausspruch von Goethe}}}\Cendnote{\textnormal{Dieser Brief stellt das Begleitschreiben zu einem von
                     \textcolor{blue}{Johann Peter Eckermann} zertifizierten \textcolor{blue}{Goethe}autograf dar. Dieser selbst konnte 
                  nicht autopsiert werden. Auf Fotografien von \textcolor{blue}{Schnitzlers} Arbeitszimmer hängt er über den Stehpult neben anderen \textcolor{blue}{Goethe}-Memorabilien. Die Herausgeber der ersten Edition der Korrespondenz \textcolor{blue}{Schnitzler}–\textcolor{blue}{Zweig}
                  zitierten den Inhalt des Autografs, gaben aber keine Auskunft über ihre Quelle. Sie schreiben (\textcolor{blue}{Stefan Zweig}: \emph{Briefwechsel mit Hermann Bahr, Sigmund Freud, Rainer Maria Rilke und Arthur
                        Schnitzler}, S. 455–456):
                        »Vorderseite:{ / }Gott segne das Haus{ / }Zweymal rannt ich heraus,{ / }Denn zweymal ist’s abgebrannt,{ / }Komm ich zum drittenmal gerannt,{ / }Da segne Gott meinen Lauf,{ / }Ich bau’s warlich nicht wieder auf.{ / }Was mehr ist als eine Laus{ / }Trage du in’s Haus.{ / }Daß obige Zeilen von \textcolor{blue}{Goethes} eigener
                        Hand geschrieben sind, bezeuge ich hiemit. \textcolor{pink}{Weimar} d. 16: April 1851. \textcolor{blue}{J. P. Eckermann}.{ / }Rückseite (von \textcolor{blue}{Schnitzler} recherchiert
                        und aufgeklebt): \textcolor{green}{Annalen oder Tages- u-
                           Jahreshefte}{ / }1801.{ / }›In \textcolor{pink}{Pyrmont} bezog ich eine schöne
                        ruhige gegen das Ende des Orts liegende Wohnung bei dem Brunnencassierer {\dots}{[}‹{]} (folgen Bemerkungen über
                        Brunnengäste, Bekanntschaften, Wetterberichte et cet.) ›Der Flusspfad nach
                           \textcolor{pink}{LUEDGE}
                        zwischen abgeschränkten Weidenplätzen her, ward öfters zurückgelegt. In dem
                        Oertchen, das einigemal abgebrannt war, erregte eine desparate Hausinschrift
                        unsere Aufmerksamkeit, die
                  lautete:{[}‹{]}«}}}\label{K_L03626-3h} in seiner Handschrift sein, und
               wirklich glückte es mir, ihn zu erlangen.\pend
           
\pstart
           Nun ist \introOben{}er\introOben{} freilich nicht ein Edelspruch \textcolor{blue}{Goethes}{}\ledrightnote{\textcolor{blue}{Johann Wolfgang von Goethe}}, sondern eher einer wo sein Genius geschlafen hat:
               aber immerhin, nehmen Sie nur die erste Zeile davon als den Wunsch eines
               Erlauchtesten für Ihr \textcolor{pink}{Haus}{}\ledrightnote{\textcolor{pink}{Sternwartestraße 71}} und {\pb}möge er sich – aber nur die erste Zeile!
               – erfüllen. Ich hätte ihn rahmen lassen, wüsste ich, wie und wo Sie ihn placieren
               wollten: \label{K_L03626-4v}\edtext{nehmen Sie ihn \substVorne{}\textsuperscript{aber}\substDazwischen{}nun\substHinten{}}{\lemma{\textnormal{\emph{nehmen Sie ihn nun}}}\Cendnote{\textnormal{Aus dem \emph{\textcolor{green}{Tagebuch}}eintrag vom 6. 10. 1910, in dem \textcolor{blue}{Schnitzler}
                  den Erhalt des Autografen notiert, und seinem {XXXX ref} vom selben Tag, läßt sich mit Blick auf die
                  üblicherweise rasche Postzustellung innerhalb \textcolor{pink}{Wiens} dieser Brief auf den 5. oder 6. 10. 1910 datieren.}}}\label{K_L03626-4h} so als
               einen Dank für Vieles, für das Schöne, \substVorne{}\textsuperscript{w}\substDazwischen{}d\substHinten{}as ich von Ihnen mit vielen andern aus Ihren Büchern, für das Schöne, \substVorne{}\textsuperscript{w}\substDazwischen{}d\substHinten{}as ich von Ihnen allein durch \label{K_L03626-5v}\edtext{Ihr Manuscript}{\lemma{\textnormal{\emph{Ihr Manuscript}}}\Cendnote{\textnormal{\textcolor{blue}{Stefan Zweig} sammelte Handschriften. Am
                     28. 12. 1909
                  vermerkte \textcolor{blue}{Schnitzler} im \emph{\textcolor{green}{Tagebuch}}, er habe dem »Sammler Zweig Urform des
                        »\textcolor{green}{Ruf des Lebens}« geschenkt«.
                  Vermutlich bezieht sich \textcolor{blue}{Zweig} auf dieses
                  Manuskript, wenn er nicht bei seinem Besuch am 12. 9. 1910, bei dem wieder über seine
                  Manuskriptensammlung gesprochen wurde, eine weitere Handschrift \textcolor{blue}{Schnitzlers} erhalten hat.}}}\label{K_L03626-5h}, Ihr Bild, vor allem aber
               manches gute Gespräch und Ihre Güte empfangen habe. In herzlicher Liebe und Verehrung
               Ihr getreuer\pend
           \pstart \spacefill\mbox{Stefan Zweig}\pend{}\endnumbering\briefempfaengerindex{Schnitzler, Arthur@\textsc{Schnitzler, Arthur}!zzzZweig, Stefan@\emph{von Stefan Zweig}!1910-10-051@{{[}5. oder 6. 10. 1910?{]}}|)be}\mylabel{h}
\begin{anhang}
\end{anhang}\normalsize

\doendnotes{C}
\bigskip
\vfill

\clearpage

\footnotesize

\lohead{\textsc{register}}

% Definiere theindex-Environment komplett neu ohne reledmac
\makeatletter
\renewenvironment{theindex}{%
  \section*{\indexname}%
  \setlength{\parindent}{0pt}%
  \setlength{\parskip}{0pt plus 0.3pt}%
  \let\item\@idxitem
}{%
  \clearpage
}
\makeatother

\IfFileExists{\jobname-pw.ind}{\input{\jobname-pw.ind}}{}

\end{document}

      