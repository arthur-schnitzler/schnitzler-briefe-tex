%% latex-korrekturansicht-vorspann.tex
%% Vorspann für die Korrekturansicht.
%% Lädt die gemeinsame Datei latex-vorspann.tex mit gesetztem Schalter.

\newif\ifkorrekturansicht
\korrekturansichttrue

\input{../tex-inputs/latex-vorspann}


\section[Stefan Zweig an Arthur Schnitzler, 15. 5. 1928]{L03689 Stefan Zweig an Arthur Schnitzler, 15. 5. 1928}
\nopagebreak\mylabel{L03689v}
\rehead{ }\normalsize\beginnumbering\briefempfaengerindex{Schnitzler, Arthur@\textsc{Schnitzler, Arthur}!zzzZweig, Stefan@\emph{von Stefan Zweig}!1928-05-151@{15. 5. 1928}|(be}
\toendnotes[C]{\smallbreak\pagebreak[2]}
\correspDesc{Versand  durch Stefan Zweig am 15. 5. 1928 in Salzburg
\newline{}Erhalt  durch Arthur Schnitzler am [16. 5. 1928?] in Wien}\toendnotes[C]{\smallbreak}
\Standort{CUL, Schnitzler, B 118.}
\physDesc{Brief, 2 Blätter, 3 Seiten, 4973 Zeichen
\newline{}Schreibmaschine (\noindent{}einschließlich Paginierung und Datierung des zweiten Blattes)
\newline{}Handschrift: roter Buntstift, lateinische Kurrent (\noindent{}Korrekturen, Unterschrift)
\newline{}Schnitzler: 1) mit Bleistift beschriftet: »\textsc{Zweig}«  2) mit rotem Buntstift beschriftet: »\textsc{Therese}« und fünfzehn Unterstreichungen}
\buchAbdrucke{\weitereDrucke{1) Stefan Zweig: \emph{Briefwechsel mit Hermann Bahr, Sigmund Freud, Rainer Maria
                        Rilke und Arthur Schnitzler}. Herausgegeben von Jeffrey B. Berlin, Hans-Ulrich Lindken und Donald A. Prater. Frankfurt am Main: \emph{S. Fischer} 1987, S. 438–441.} \weitereDrucke{2) Stefan Zweig: \emph{Briefe. Bd. III: 1920–1931}. Herausgegeben von Knut Beck und Jeffrey B. Berlin. Frankfurt am Main: \emph{S. Fischer} 2000, S. 214–216.} }\toendnotes[C]{\smallbreak}
\pstart
           {\pb}\textcolor{gray}{\textbf{SZ}}\hfill \textcolor{gray}{\textbf{\textcolor{pink}{SALZBURG}\oindex{Salzburg@\textbf{Salzburg}, \emph{Verwaltungsgebiet}|pw}{}\ledrightnote{\textcolor{pink}{Salzburg}}}}\pend
           
\pstart
           \raggedleft{}\textcolor{gray}{\textbf{\textcolor{pink}{KAPUZINERBERG 5}\oindex{Paschinger Schlössl@\textbf{Paschinger Schlössl}, \emph{Wohngebäude}|pw}{}\ledrightnote{\textcolor{pink}{Paschinger Schlössl}}}}\pend
           
\pstart
           \raggedleft{}15. Mai 1928.\pend
           
\pstart{}Lieber und verehrter Herr Doktor!\pend\vspace{0.5em}
\pstart
           Ich habe zwiefach zu danken und beide Male sehr herzlich: sowohl für Ihren lieben
               Brief, der mich unendlich erfreute, und Ihr \textcolor{green}{Buch}\pwindex{Schnitzler, Arthur 15. 5. 1862 Wien – 21. 10. 1931 ebd.@\textsc{Schnitzler, Arthur} (15. 5. 1862 Wien – 21. 10. 1931 ebd.), \emph{Schriftsteller, Mediziner}!Therese. Chronik eines Frauenlebens@\strich\emph{Therese. Chronik eines Frauenlebens}|pwv}{}\ledrightnote{{$\rightarrow$}\emph{\textcolor{green}{Therese. Chronik eines Frauenlebens}}}, das mich überrascht hat – Ihr Fleiss gerade \label{K_L03689-1v}\edtext{in den letzten \textcolor{green}{Werken}\pwindex{Schnitzler, Arthur 15. 5. 1862 Wien – 21. 10. 1931 ebd.@\textsc{Schnitzler, Arthur} (15. 5. 1862 Wien – 21. 10. 1931 ebd.), \emph{Schriftsteller, Mediziner}!Traumnovelle@\strich\emph{Traumnovelle}|pwv}\pwindex{Schnitzler, Arthur 15. 5. 1862 Wien – 21. 10. 1931 ebd.@\textsc{Schnitzler, Arthur} (15. 5. 1862 Wien – 21. 10. 1931 ebd.), \emph{Schriftsteller, Mediziner}!Spiel im Morgengrauen. Novelle@\strich\emph{Spiel im Morgengrauen. Novelle}|pwv}\pwindex{Schnitzler, Arthur 15. 5. 1862 Wien – 21. 10. 1931 ebd.@\textsc{Schnitzler, Arthur} (15. 5. 1862 Wien – 21. 10. 1931 ebd.), \emph{Schriftsteller, Mediziner}!Buch der Sprüche und Bedenken@\strich\emph{Buch der Sprüche und Bedenken}|pwv}\pwindex{Schnitzler, Arthur 15. 5. 1862 Wien – 21. 10. 1931 ebd.@\textsc{Schnitzler, Arthur} (15. 5. 1862 Wien – 21. 10. 1931 ebd.), \emph{Schriftsteller, Mediziner}!Geist im Wort und der Geist in der Tat@\strich\emph{Der Geist im Wort und der Geist in der Tat}|pwv}{}\ledrightnote{{$\rightarrow$}\emph{\textcolor{green}{Traumnovelle}}{\newline}{$\rightarrow$}\emph{\textcolor{green}{Spiel im Morgengrauen. Novelle}}{\newline}{$\rightarrow$}\emph{\textcolor{green}{Buch der Sprüche und Bedenken}}{\newline}{$\rightarrow$}\emph{\textcolor{green}{Der Geist im Wort und der Geist in der Tat}}}}{\lemma{\textnormal{\emph{in den letzten Werken}}}\Cendnote{\textnormal{1926 publizierte \textcolor{blue}{Schnitzler} die
                     \emph{\textcolor{green}{Traumnovelle}\pwindex{Schnitzler, Arthur 15. 5. 1862 Wien – 21. 10. 1931 ebd.@\textsc{Schnitzler, Arthur} (15. 5. 1862 Wien – 21. 10. 1931 ebd.), \emph{Schriftsteller, Mediziner}!Traumnovelle@\strich\emph{Traumnovelle}|pwk}}, 1927 erschienen
                  in Buchform die Novelle \emph{\textcolor{green}{Das Spiel im
                     Morgengrauen}\pwindex{Schnitzler, Arthur 15. 5. 1862 Wien – 21. 10. 1931 ebd.@\textsc{Schnitzler, Arthur} (15. 5. 1862 Wien – 21. 10. 1931 ebd.), \emph{Schriftsteller, Mediziner}!Spiel im Morgengrauen. Novelle@\strich\emph{Spiel im Morgengrauen. Novelle}|pwk}}, die Aphorismensammlung \emph{\textcolor{green}{Buch
                     der Sprüche und Bedenken}\pwindex{Schnitzler, Arthur 15. 5. 1862 Wien – 21. 10. 1931 ebd.@\textsc{Schnitzler, Arthur} (15. 5. 1862 Wien – 21. 10. 1931 ebd.), \emph{Schriftsteller, Mediziner}!Buch der Sprüche und Bedenken@\strich\emph{Buch der Sprüche und Bedenken}|pwk}} und die typologische Studie \emph{\textcolor{green}{Der Geist im Wort und der Geist in der Tat}\pwindex{Schnitzler, Arthur 15. 5. 1862 Wien – 21. 10. 1931 ebd.@\textsc{Schnitzler, Arthur} (15. 5. 1862 Wien – 21. 10. 1931 ebd.), \emph{Schriftsteller, Mediziner}!Geist im Wort und der Geist in der Tat@\strich\emph{Der Geist im Wort und der Geist in der Tat}|pwk}}.}}}\label{K_L03689-1} wirkt
               auf uns Jüngere beschämend. Nichts selbstverständlicher, als dass ich sofort das \textcolor{green}{Buch}\pwindex{Schnitzler, Arthur 15. 5. 1862 Wien – 21. 10. 1931 ebd.@\textsc{Schnitzler, Arthur} (15. 5. 1862 Wien – 21. 10. 1931 ebd.), \emph{Schriftsteller, Mediziner}!Therese. Chronik eines Frauenlebens@\strich\emph{Therese. Chronik eines Frauenlebens}|pwv}{}\ledrightnote{{$\rightarrow$}\emph{\textcolor{green}{Therese. Chronik eines Frauenlebens}}} zur Hand nahm und Ihnen so
               heute mit dem Danke zugleich ungehemmt meinen Eindruck aussprechen darf.\pend
           
\pstart
           Sie haben sich ein ungeheuer schweres Problem gestellt, denn nichts ist in der Kunst
               schwieriger und undankbarer darzustellen als das Negative, eine gewisse Monotonie des
               Glücks und des Unglücks, die Tragik der Hoffnungslosigkeit. Ich weiss es gerade
               jetzt, weil ich eine \label{K_L03689-2v}\edtext{grössere \textcolor{green}{Arbeit}\pwindex{Zweig, Stefan 28.\,11.\,1881 Wien – 23.\,2.\,1942 Petrópolis@\textsc{Zweig, Stefan} (28.\,11.\,1881 Wien – 23.\,2.\,1942 Petrópolis), \emph{Schriftsteller}!Adam Lux@\strich\emph{Adam Lux}|pwv}{}\ledrightnote{{$\rightarrow$}\emph{\textcolor{green}{Adam Lux}}}}{\lemma{\textnormal{\emph{grössere Arbeit}}}\Cendnote{\textnormal{\textcolor{blue}{Stefan Zweig}\pwindex{Zweig, Stefan 28.\,11.\,1881 Wien – 23.\,2.\,1942 Petrópolis@\textsc{Zweig, Stefan} (28.\,11.\,1881 Wien – 23.\,2.\,1942 Petrópolis), \emph{Schriftsteller}|pwk} arbeitete bis zum Sommer 1928 am Drama \emph{\textcolor{green}{Adam Lux}\pwindex{Zweig, Stefan 28.\,11.\,1881 Wien – 23.\,2.\,1942 Petrópolis@\textsc{Zweig, Stefan} (28.\,11.\,1881 Wien – 23.\,2.\,1942 Petrópolis), \emph{Schriftsteller}!Adam Lux@\strich\emph{Adam Lux}|pwk}}, das
                  ihn schon seit 1912 beschäftigte, aber aus dem
                  Nachlass veröffentlicht wurde.}}}\label{K_L03689-2} mitten im Werke aufgegeben habe, wo
               gleichfalls ein armer Lebenslauf geschildert werden wollte\substVorne{}\textsuperscript{,}\substDazwischen{}:\substHinten{}{ }\strikeout{aber} unwillkürlich dringt die Monotonie leicht in die
               Gestaltung, und für mein Empfinden manifestierte sich \textcolor{blue}{Rembrandt}\pwindex{Rembrandt van Rijn 15.\,7.\,1606 Leiden – 4.\,10.\,1669 Amsterdam@\textsc{Rembrandt van Rijn} (15.\,7.\,1606 Leiden – 4.\,10.\,1669 Amsterdam), \emph{Maler}|pw}{}\ledrightnote{\textcolor{blue}{Rembrandt van Rijn}} nie genialer als wie er die drei riesigen Bäume\textcolor{red}{\textsuperscript{\textbf{KEY}}}
               allein in die ungeheure (sonst kaum malerisch darstellbare) Ebene stellte. Der Stoff
               also, den Sie sich gewählt haben (oder vielmehr, der Sie gewählt hat: es wählt ja für
               uns) will mir nur scheinbar \introOben{}\strikeout{un}\introOben{}bewegt vorkommen. Es ereignet sich ein fortwährendes Wellenspiel von
               Geschehnissen und Veränderungen – ich aber spüre am grossartigsten und tragischesten darin die innere Hoffnungslosigkeit dieses
               Menschen. Ich weiss nicht, wieso es kommt, aber von der 50. Seite an wusste ich schon
               bei jedem Erlebnis, es würde nicht dauern, nicht Glück produzieren, es würde wieder
               enden an ebenderselben {\pb}furchtbaren
               Verbannung von aller Freudigkeit, in welche dieser Mensch hineingeboren ist. Sie
               konnten nicht wahrer sein, indem Sie aus Millionen eine solche Gestalt herausholten,
               und für mein Empfinden stellt sich diese \textcolor{green}{Chronik}\pwindex{Schnitzler, Arthur 15. 5. 1862 Wien – 21. 10. 1931 ebd.@\textsc{Schnitzler, Arthur} (15. 5. 1862 Wien – 21. 10. 1931 ebd.), \emph{Schriftsteller, Mediziner}!Therese. Chronik eines Frauenlebens@\strich\emph{Therese. Chronik eines Frauenlebens}|pwv}{}\ledrightnote{{$\rightarrow$}\emph{\textcolor{green}{Therese. Chronik eines Frauenlebens}}} endgiltig dar. Sie romantisiert
               nicht, sondern sie bleibt grausam nüchtern und erschreckend wahr. Erschreckend – dies
               Wort gilt nicht für mich, der das Tragische und am liebsten das geheim Tragische des
               Daseins in Büchern als höchste Tugend ehrt, wohl aber vielleicht für ein breiteres
               Publikum, das, weil \label{T_L03689-1v}\edtext{diese Monotonie selbst}{\lemma{\textnormal{\emph{diese Monotonie selbst}}}\Cendnote{\textnormal{Durch ein Zeichen umgestellt aus: »selbst diese Monotonie«.}}}\label{T_L03689-1} unbewusst erlebend, im Geschriebenen wie
               auf der Leinwand immer eine Spannung sehen will, bewegte Schicksale, und das
               unbewusst Depressive dieser Gestalt als \label{K_L03689-3v}\edtext{gênant}{\lemma{\textnormal{\emph{gênant}}}\Cendnote{\textnormal{französisch: peinlich, unangenehm}}}\label{K_L03689-3} empfinden wird – gênant für ihre
               Sorglosigkeit, für ihren Amüsierwillen, ihr Darüberhinwegwollen im eigenen Dasein.
               Sie haben es gewiss von Anfang gewusst, dass Sie hier einer Publikumsneigung im
               innersten entgegenwirken – die Menschen wollen immer nur Reichtum sehen, reiche
               Milieus, tropische Charaktere, rare und kuriose Erlebnisse – aber nichts ehrt Sie
               mehr, als dass Sie auf der Höhe Ihres Schaffens das Allerschwerste auf sich genommen
               haben, das dem Künstler vorbehalten ist: die arme Existenz zu schildern, die Tragödie
               der unzähligen Anonymen. Diese Menschen lesen nicht die Bücher in den ersten vier
               Wochen, insolange sie modern sind, sie kommen erst langsam an sie heran – dann aber
               werden Sie einzelne Dankbarkeiten erhalten, die Ihnen wirklich glückhaft sein müssen.
               Als Mann des Metiers muss ich ein wenig neugierig sein auf den Wiederhall im Kreise
               der Geistigen, ob \substVorne{}\textsuperscript{sie}\substDazwischen{}diese\substHinten{} das bewusst Heroische dieser \textcolor{green}{Chronik}\pwindex{Schnitzler, Arthur 15. 5. 1862 Wien – 21. 10. 1931 ebd.@\textsc{Schnitzler, Arthur} (15. 5. 1862 Wien – 21. 10. 1931 ebd.), \emph{Schriftsteller, Mediziner}!Therese. Chronik eines Frauenlebens@\strich\emph{Therese. Chronik eines Frauenlebens}|pwv}{}\ledrightnote{{$\rightarrow$}\emph{\textcolor{green}{Therese. Chronik eines Frauenlebens}}} wahrnehmen und würdigen können, das für
               uns \textcolor{pink}{Oesterreicher}\oindex{Österreich-Ungarn@\textbf{Österreich-Ungarn}|pw}{}\ledrightnote{\textcolor{pink}{Österreich-Ungarn}} noch überdies besonderen
               dokumentarischen Wert hat. Das Gefährlich\introOben{}e\introOben{} einer solchen
                  \textcolor{green}{Chronik}\pwindex{Schnitzler, Arthur 15. 5. 1862 Wien – 21. 10. 1931 ebd.@\textsc{Schnitzler, Arthur} (15. 5. 1862 Wien – 21. 10. 1931 ebd.), \emph{Schriftsteller, Mediziner}!Therese. Chronik eines Frauenlebens@\strich\emph{Therese. Chronik eines Frauenlebens}|pwv}{}\ledrightnote{{$\rightarrow$}\emph{\textcolor{green}{Therese. Chronik eines Frauenlebens}}} im Gegensatz zu
               einem wirk{\pb}lichen Roman entgeht mir
               natürlich nicht, nämlich dass im Roman alle Gestalten auf Wiederkehr gestellt sind,
               also dramenhaft, während hier die meisten nur einmal episodisch auftreten und dadurch
               leichter dem Gedächtnisse verschatten – mir fliessen die einzelnen Familien der
               Lehrerin in der Erinnerung schon ein wenig zusammen, \introOben{}–\introOben{} aber
               dies war nicht zu vermeiden, denn sie bedeuten ja nichts als Meilensteine, um den Weg
               zu messen. Sie wissen so viel von den Geheimnissen der epischen Prosa, dass Sie da
               mit Sparsamkeit gearbeitet haben, wo ein anderer in breiten Milieuschilderungen sich
               und die Leser erschöpft hätte, und ich glaube, dass die gewisse Silhouettenhaftigkeit
               der Nebenfiguren gegenüber der Prägnanz der Hauptgestalten Ihre rechte und richtige
               Einsicht war.\pend
           
\pstart
           Lassen Sie sich nun ruhen auf solcher Leistung, die für uns Jüngere gleichzeitig eine
               Lehre bedeutet. Wie wunderbar, dass Sie aus solcher Fülle schöpfen können, und das
               selige Spiel des Erfindens \substVorne{}\textsuperscript{ist }\substDazwischen{}sich\substHinten{} Ihnen fast noch leichter als in den jugendlichen Jahren gewährt. Könnte dies
                  \textcolor{green}{Buch}\pwindex{Schnitzler, Arthur 15. 5. 1862 Wien – 21. 10. 1931 ebd.@\textsc{Schnitzler, Arthur} (15. 5. 1862 Wien – 21. 10. 1931 ebd.), \emph{Schriftsteller, Mediziner}!Therese. Chronik eines Frauenlebens@\strich\emph{Therese. Chronik eines Frauenlebens}|pwv}{}\ledrightnote{{$\rightarrow$}\emph{\textcolor{green}{Therese. Chronik eines Frauenlebens}}} meine seit den
               Knabenjahren rein bewahrte Verehrung und Liebe noch erhöhen, so hätte es dies
               gewisslich in mir getan, aber vielleicht ist schon jene kristallene Festigkeit des
               Gefühls vorhanden, die durch ein gelungenes Werk nicht mehr gesteigert und durch ein
               misslungenes nicht mehr gemindert werden könnte. Seien Sie dieser meiner lautersten
               und in ihrer Unabänderlichkeit verlässlichen Gesinnung aufrichtig gewiss! Und
               erlauben Sie mir, das, wenn ich nächstens nach \textcolor{pink}{Wien}\oindex{Wien@\textbf{Wien}, \emph{Verwaltungsgebiet}|pw}{}\ledrightnote{\textcolor{pink}{Wien}} komme, ich Ihnen noch glückwünschend die Hand reiche.\pend
           
\pstart
           Treulichst Ihr{\\[\baselineskip]}\spacefill\mbox{{[}hs.:{]} Stefan Zweig}\pend
           \leftskip=0em{}\selectlanguage{ngerman}\endnumbering\briefempfaengerindex{Schnitzler, Arthur@\textsc{Schnitzler, Arthur}!zzzZweig, Stefan@\emph{von Stefan Zweig}!1928-05-151@{15. 5. 1928}|)be}\mylabel{L03689h}  \normalsize

\doendnotes{C}
\bigskip
\vfill

\clearpage

\footnotesize

\lohead{\textsc{register}}

% Definiere theindex-Environment komplett neu ohne reledmac
\makeatletter
\renewenvironment{theindex}{%
  \section*{\indexname}%
  \setlength{\parindent}{0pt}%
  \setlength{\parskip}{0pt plus 0.3pt}%
  \let\item\@idxitem
}{%
  \clearpage
}
\makeatother

\IfFileExists{\jobname-pw.ind}{\input{\jobname-pw.ind}}{}

\end{document}

      