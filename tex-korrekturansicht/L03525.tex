%% latex-korrekturansicht-vorspann.tex
%% Vorspann für die Korrekturansicht.
%% Lädt die gemeinsame Datei latex-vorspann.tex mit gesetztem Schalter.

\newif\ifkorrekturansicht
\korrekturansichttrue

\input{../tex-inputs/latex-vorspann}


\renewcommand{\erwaehntePersonen}{Personen: Bjørnstjerne Bjørnson, Marie Glümer, Paul Goldmann, Olga Schnitzler, Elisabeth Steinrück, Ernst von Wolzogen}
\renewcommand{\erwaehnteInstitutionen}{Institutionen: Überbrettl}
\renewcommand{\erwaehnteOrte}{Orte: ?? [Sanatorium], Berlin, Dessauer Straße, Wien}
\renewcommand{\erwaehnteWerke}{Werke: Berliner Theater. [Über unsere Kraft von Bjørnstjerne Bjørnson], Neue Freie Presse, Über unsere Kraft. Zweiter Teil}
\section[ Paul Goldmann an Olga Gussmann, 3. 4. {[}1901{]}]{Paul Goldmann an Olga Gussmann, 3. 4. {[}1901{]}}
\nopagebreak\mylabel{v}
\rehead{ }\normalsize\beginnumbering\briefempfaengerindex{Schnitzler, Olga@\textsc{Schnitzler, Olga}!zzzGoldmann, Paul@\emph{von Paul Goldmann}!1901-04-032@{3. 4. {[}1901{]}}|(be}
\toendnotes[C]{\smallbreak\pagebreak[2]}\Standort{DLA, A:Schnitzler, HS.NZ85.1.5247.}
\physDesc{Brief, 2 Blätter, 6 Seiten, 2394 Zeichen
\newline{}Handschrift: blaue Tinte, deutsche Kurrent
\newline{}Ordnung: mit Bleistift von \textcolor{blue}{Arthur Schnitzler} das
                                 Jahr »1901.« vermerkt }\toendnotes[C]{\smallbreak}
\pstart
           \noindent{}\raggedleft{}{\pb}\textcolor{gray}{\textbf{\textcolor{pink}{DESSAUERSTRASSE 19}{}\ledrightnote{\textcolor{pink}{Dessauer Straße}}}}\pend
           
\pstart
           \textcolor{pink}{Berlin}{}\ledrightnote{\textcolor{pink}{Berlin}}, 3. April.\pend
           
\pstart\center{}Liebes Fräulein Olga,\pend
\pstart
           Schön, ſchön und ſchön! Und ich habe \uline{doch} Recht! Und
               wenn Sie werden ſo grob mit mir ſein, ſo werde ich bei Ihrem erſten Auftreten in \textcolor{pink}{Berlin}{}\ledrightnote{\textcolor{pink}{Berlin}} eine ſchlechte Kritik über Sie ſchreiben!
               Oder ihnen ſonſt etwas Furchtbares anthun! Und wenn \uline{alle} Menſchen \label{K_L03525-1v}\edtext{einſam}{\lemma{\textnormal{\emph{einſam}}}\Cendnote{\textnormal{siehe Paul Goldmann an Arthur Schnitzler, 7. 7. 1907}}}\label{K_L03525-1h} ſind (was übrigens nicht wahr iſt), ſo will ich es \uline{nicht} ſein, \label{K_L03525-2v}\edtext{Himmelkreuzſchockſchwerenoth}{\lemma{\textnormal{\emph{Himmelkreuzſchockſchwerenoth}}}\Cendnote{\textnormal{umgangssprachlicher Ausruf}}}\label{K_L03525-2h}! Und wenn \uline{alle}
               Frauen eine Bagage ſind, ſo will ich doch eine haben, ſchon um auf {\pb}ſie ſchimpfen zu können! Und mein \label{K_L03525-3v}\edtext{\textcolor{green}{Feuilleton}{}\ledrightnote{{$\rightarrow$}\textcolor{green}{Berliner Theater. [Über unsere Kraft von Bjørnstjerne Bjørnson]}}}{\lemma{\textnormal{\emph{Feuilleton}}}\Cendnote{\textnormal{\textcolor{blue}{Paul Goldmann}: \emph{\textcolor{green}{Berliner Theater}}. In: \emph{\textcolor{green}{Neue Freie Presse}}, Nr. 13.144,
                     29. 3. 1901, Morgenblatt, S. 1–4.}}}\label{K_L03525-3h} kam
               von Herzen und es war gut; denn es iſt \strikeout{\textcolor{gray}{Ar}} keine Kleinigkeit, den Gedankeninhalt eines ſo gewaltigen \textcolor{green}{Werkes}{}\ledrightnote{{$\rightarrow$}\textcolor{green}{Über unsere Kraft. Zweiter Teil}} zu
               entwickeln, zumal wenn man gezwungen iſt, Manches zu ſagen, was der \textcolor{blue}{Autor}{}\ledrightnote{{$\rightarrow$}\textcolor{blue}{Bjørnstjerne Bjørnson}} ſich ſelbſt nicht gedacht hat! Und
               wenn es Ihnen \strikeout{Ih\textcolor{gray}{ne}} nicht gefallen hat, ſo haben Sie mich eben nur wieder einmal unterſchätzt! Im
               Übrigen iſt es \strikeout{b\textcolor{gray}{ezeic}h} ſehr lieb
               von Ihnen, daß Sie mir geſchrieben haben, wie Sie ſchreiben. Vom Leben aber {\pb}\substVorne{}\textsuperscript{\textcolor{gray}{geschehe}}{\allowbreak}\substDazwischen{}wiſſen\substHinten{} Sie lange nicht ſo viel, als Sie ſich einbilden. Und es wäre ſehr ſchön,
               wenn ich in \textcolor{pink}{Wien}{}\ledrightnote{\textcolor{pink}{Wien}} wäre und Sie \textcolor{blue}{Beide}{}\ledrightnote{{$\rightarrow$}} öfter ſehen könnte; ich würde
               wahrſcheinlich weniger \label{K_L03525-4v}\edtext{Grillen
                  fangen}{\lemma{\textnormal{\emph{Grillen
                  fangen}}}\Cendnote{\textnormal{Anspielung auf eine Metapher im
                  vorigen Brief, vgl. Paul Goldmann an Olga Gussmann, 1. 4. [1901]}}}\label{K_L03525-4h}! Und es iſt unerhört, daß ich heut ſchon
               wieder Ihnen ſchreiben muß, ſtatt Ihrem \textcolor{blue}{Schweſterchen}{}\ledrightnote{{$\rightarrow$}\textcolor{blue}{Elisabeth Steinrück}}, wie ich eigentlich vorhatte.\pend
           
\pstart
           So, und jetzt reden wir vernünftig!\pend
           
\pstart
           Dieſes kleine Fräulein \textsc{\textcolor{blue}{Liesl}{}\ledrightnote{\textcolor{blue}{Elisabeth Steinrück}}} ſitzt ahnungslos in \textcolor{pink}{Wien}{}\ledrightnote{\textcolor{pink}{Wien}} und weiß nicht, daß
                  \label{K_L03525-5v}\edtext{\textcolor{pink}{hier}{}\ledrightnote{{$\rightarrow$}\textcolor{pink}{Berlin}} über ihr {\pb}Schickſal verhandelt}{\lemma{\textnormal{\emph{hier … verhandelt}}}\Cendnote{\textnormal{siehe auch Paul Goldmann an Arthur Schnitzler, 18. 2. [1901]}}}\label{K_L03525-5h} wird. Vorgeſtern{ }Abend war ich mit \textsc{\textcolor{blue}{Wolzogen}{}\ledrightnote{\textcolor{blue}{Ernst von Wolzogen}}} zuſammen. Es wurde über Neuengagements für das »\textcolor{brown}{Überbrettl}{}\ledrightnote{\textcolor{brown}{Überbrettl}}« geſprochen, und ich ſtellte mit großer Energie die Candidatur
               Ihrer \textcolor{blue}{Schweſter}{}\ledrightnote{{$\rightarrow$}\textcolor{blue}{Elisabeth Steinrück}} auf. \textsc{\textcolor{blue}{Wolzogen}{}\ledrightnote{\textcolor{blue}{Ernst von Wolzogen}}} hat ein Vorurtheil gegen die \textcolor{pink}{Wien}{}\ledrightnote{\textcolor{pink}{Wien}}er Art, zu
               ſpielen, und ich weiß nicht, ob es mir gelingen wird, dieſes Vorurtheil zu
               zerſtreuen. Das beſte Mittel wäre Fräulein \textsc{\textcolor{blue}{Liesl}{}\ledrightnote{\textcolor{blue}{Elisabeth Steinrück}}s} perſönliches Erſcheinen. Ich
               frage alſo: Könnte dieſe {\pb} nacherwähnte junge \textcolor{blue}{Dame}{}\ledrightnote{{$\rightarrow$}\textcolor{blue}{Elisabeth Steinrück}}, falls die Sache ernſt
               wird, auf einige Tage nach \textcolor{pink}{Berlin}{}\ledrightnote{\textcolor{pink}{Berlin}} kommen? Könnte
               ſie eventuell gleich ins \textcolor{brown}{Engagement}{}\ledrightnote{{$\rightarrow$}\textcolor{brown}{Überbrettl}} gehen? Ich betone: Dieſe Fragen ſind vorläufig rein akademiſch;
               und es iſt noch ſehr unſicher, ob die Sache ſich wird praktiſch verwirklichen
               laſſen.\pend
           
\pstart
           Weitere Frage: wiſſen Sie einen für heiteren Geſang begabten jungen Mann, {\pb}Tenor oder Baryton, ebenfalls fürs »\textcolor{brown}{Überbrettl}{}\ledrightnote{\textcolor{brown}{Überbrettl}}«?\pend
           
\pstart
           Bitte um \uline{raſche} Antwort!\pend
           
\pstart
           Die \textcolor{blue}{Glümer}{}\ledrightnote{\textcolor{blue}{Marie Glümer}} iſt auf dem Wege der \label{K_L03525-6v}\edtext{Geneſung}{\lemma{\textnormal{\emph{Geneſung}}}\Cendnote{\textnormal{\textcolor{blue}{Marie Glümer} war seit Anfang des Jahres krank, vgl. Paul Goldmann an Arthur Schnitzler, 22. 1. [1901] und folgende Briefe \textcolor{blue}{Goldmann}s an \textcolor{blue}{Schnitzler}.}}}\label{K_L03525-6h}.
               Sie hat vor einigen Tagen das \textcolor{pink}{Sanatorium}{}\ledrightnote{\textcolor{pink}{?? [Sanatorium]}}
               verlaſſen.\pend
           
\pstart
           Und nun ſchönen Dank für Alles! Und ſeien Sie ſammt dem \textcolor{blue}{Schweſterlein}{}\ledrightnote{{$\rightarrow$}\textcolor{blue}{Elisabeth Steinrück}} herzlichſt gegrüßt von
               {\\[\baselineskip]}Ihrem ergebenen {\\[\baselineskip]}\spacefill\mbox{Dr. Paul Goldmann.}\pend
           \leftskip=0em{}\endnumbering\briefempfaengerindex{Schnitzler, Olga@\textsc{Schnitzler, Olga}!zzzGoldmann, Paul@\emph{von Paul Goldmann}!1901-04-032@{3. 4. {[}1901{]}}|)be}\mylabel{h}  \normalsize

\doendnotes{C}
\bigskip
\vfill

\clearpage

\footnotesize

\lohead{\textsc{register}}

% Definiere theindex-Environment komplett neu ohne reledmac
\makeatletter
\renewenvironment{theindex}{%
  \section*{\indexname}%
  \setlength{\parindent}{0pt}%
  \setlength{\parskip}{0pt plus 0.3pt}%
  \let\item\@idxitem
}{%
  \clearpage
}
\makeatother

\IfFileExists{\jobname-pw.ind}{\input{\jobname-pw.ind}}{}

\end{document}

      