%% latex-korrekturansicht-vorspann.tex
%% Vorspann für die Korrekturansicht.
%% Lädt die gemeinsame Datei latex-vorspann.tex mit gesetztem Schalter.

\newif\ifkorrekturansicht
\korrekturansichttrue

\input{../tex-inputs/latex-vorspann}


         
         \renewcommand{\erwaehntePersonen}{Personen: Alfred Kerr, Fedor Mamroth}
         \renewcommand{\erwaehnteOrte}{Orte: Bad Aussee, Berlin, Bozen, Dessauer Straße, Innsbruck, Schweiz, Wien}
         \renewcommand{\erwaehnteWerke}{}
               \section[ Paul Goldmann an Arthur Schnitzler, 26. 7. {[}1900{]}]{Paul Goldmann an Arthur Schnitzler, 26. 7. {[}1900{]}}\nopagebreak\mylabel{v}\rehead{ }\normalsize\beginnumbering\briefempfaengerindex{Schnitzler, Arthur@\textsc{Schnitzler, Arthur}!zzzGoldmann, Paul@\emph{von Paul Goldmann}!1900-07-262@{26. 7. {[}1900{]}}|(be} \toendnotes[C]{\smallbreak\pagebreak[2]} \Standort{DLA, A:Schnitzler, HS.NZ85.1.3170.}
\physDesc{Brief, 1 Blatt, 2 Seiten
\newline{}Handschrift: blaue Tinte, deutsche Kurrent
\newline{}Schnitzler: mit Bleistift das Jahr »{[}1{]}900« vermerkt }\toendnotes[C]{\smallbreak}\pstart
           \noindent{}\raggedleft{}{\pb}\textcolor{pink}{\textcolor{gray}{\textbf{DESSAUERSTRASSE 19}}}{}\ledrightnote{\textcolor{pink}{Dessauer Straße}}\pend
           \pstart
           \textcolor{pink}{Berlin}{}\ledrightnote{\textcolor{pink}{Berlin}}, 26. Juli.\pend
           \pstart\center{}Mein lieber Freund,\pend\pstart
           Endlich den Urlaub erkämpft! Zwiſchen 10. und 15. Auguſt fahre ich von \textcolor{pink}{hier}{}\ledrightnote{{$\rightarrow$}\textcolor{pink}{Berlin}} über \textcolor{pink}{Wien}{}\ledrightnote{\textcolor{pink}{Wien}} nach \textsc{\textcolor{pink}{Innsbruck}{}\ledrightnote{\textcolor{pink}{Innsbruck}}}. Von dort \label{K_L02925-1v}\edtext{Fußwanderung}{\lemma{\textnormal{\emph{Fußwanderung}}}\Cendnote{\textnormal{siehe Paul Goldmann an Arthur Schnitzler, 16. 6. [1900]}}}\label{K_L02925-1h} ins Gebirge. Bitte, ſchreib’ mir ſofort, ob es dabei bleibt und wann Du in
                  \textsc{\textcolor{pink}{Innsbruck}{}\ledrightnote{\textcolor{pink}{Innsbruck}}} ſein kannſt. Vielleicht kannſt {\pb}Du auch
                  \label{K_L02925-2v}\edtext{\textsc{\textcolor{blue}{Kerr}{}\ledrightnote{\textcolor{blue}{Alfred Kerr}}} verſtändigen}{\lemma{\textnormal{\emph{Kerr verſtändigen}}}\Cendnote{\textnormal{nicht geschehen, siehe
                  auch siehe Paul Goldmann an Arthur Schnitzler, 2. 8. [1900]}}}\label{K_L02925-2h} nach \textsc{\textcolor{pink}{Bozen}{}\ledrightnote{\textcolor{pink}{Bozen}}}, \begin{otherlanguage}{french}\textsc{Poste restante}\end{otherlanguage}. Aber, nicht wahr, du antworteſt mir bald? Denn mein \textcolor{blue}{Onkel}{}\ledrightnote{{$\rightarrow$}\textcolor{blue}{Fedor Mamroth}} drängt mich, mit ihm in die \textcolor{pink}{Schweiz}{}\ledrightnote{\textcolor{pink}{Schweiz}} zu gehen. Und wenn Ihr zu faul wäret, zu
               laufen, ſo möchte ich mir dieſe Gelegenheit, mit meinem \textcolor{blue}{Onkel}{}\ledrightnote{{$\rightarrow$}\textcolor{blue}{Fedor Mamroth}} zu wandern, nicht entgehen laſſen.\pend
           \pstart
           Viele treue Grüße! {\\[\baselineskip]}Dein {\\[\baselineskip]}\spacefill\mbox{Paul Goldmann.}\pend
           \leftskip=0em{}\endnumbering\briefempfaengerindex{Schnitzler, Arthur@\textsc{Schnitzler, Arthur}!zzzGoldmann, Paul@\emph{von Paul Goldmann}!1900-07-262@{26. 7. {[}1900{]}}|)be}\mylabel{h}\begin{anhang}\end{anhang}\normalsize

\doendnotes{C}
\bigskip
\vfill

\clearpage

\footnotesize

\lohead{\textsc{register}}

% Definiere theindex-Environment komplett neu ohne reledmac
\makeatletter
\renewenvironment{theindex}{%
  \section*{\indexname}%
  \setlength{\parindent}{0pt}%
  \setlength{\parskip}{0pt plus 0.3pt}%
  \let\item\@idxitem
}{%
  \clearpage
}
\makeatother

\IfFileExists{\jobname-pw.ind}{\input{\jobname-pw.ind}}{}

\end{document}

      