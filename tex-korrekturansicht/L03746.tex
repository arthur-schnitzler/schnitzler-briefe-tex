%% latex-korrekturansicht-vorspann.tex
%% Vorspann für die Korrekturansicht.
%% Lädt die gemeinsame Datei latex-vorspann.tex mit gesetztem Schalter.

\newif\ifkorrekturansicht
\korrekturansichttrue

\input{../tex-inputs/latex-vorspann}


\section[Arthur Schnitzler an Stefan Zweig, 4. 5. 1925]{L03746 Arthur Schnitzler an Stefan Zweig, 4. 5. 1925}
\nopagebreak\mylabel{L03746v}
\rehead{ }\normalsize\beginnumbering\briefempfaengerindex{Zweig, Stefan@\textsc{Zweig, Stefan}!zzzSchnitzler, Arthur@\emph{von Arthur Schnitzler}!1925-05-041@{4. 5. 1925}|(be}
\toendnotes[C]{\smallbreak\pagebreak[2]}
\correspDesc{Versand  durch Arthur Schnitzler am 4. 5. 1925 in Wien
\newline{}Übermittlung  am 5. 5. 1925 in Wien
\newline{}Erhalt  durch Stefan Zweig im Zeitraum [6. 5. 1925 – 10. 5. 1925?] in Salzburg}\toendnotes[C]{\smallbreak}
\Standort{Jerusalem, National Library of Israel, ARC. Ms. Var. 305 1 58 Stefan Zweig Collection.}
\physDesc{Postkarte, 280 Zeichen
\newline{}Handschrift: schwarze Tinte, lateinische Kurrent
\newline{}Versand: Stempel: »\nobreak{}\oindex{XVIII., Währing@\textbf{XVIII., Währing}, \emph{Verwaltungsgebiet}|pwk}18/\textsubscript{1} Wien
                                       110, 5. \textcolor{gray}{5}. 25\nobreak{}«.  }\toendnotes[C]{\smallbreak}\pstart{}{\pb}\label{T_L03746-1v}\edtext{\textcolor{gray}{\textbf{A. S.}}}{\lemma{\textnormal{\emph{A. S.}}}\Cendnote{\textnormal{ovaler Absenderkleber}}}\label{T_L03746-1}\pend{}\pstart{}\textcolor{pink}{\textcolor{gray}{\textbf{WIEN, XVIII.}}}\oindex{XVIII., Währing@\textbf{XVIII., Währing}, \emph{Verwaltungsgebiet}|pw}{}\ledrightnote{\textcolor{pink}{XVIII., Währing}}\pend{}\pstart{}\textcolor{pink}{\textcolor{gray}{\textbf{STERNWARTESTR. 71}}}\oindex{Wien@\textbf{Wien}!XVIII., Währing@\textbf{XVIII., Währing}!Sternwartestraße 71@\textbf{Sternwartestraße 71}, \emph{Wohngebäude}|pw}{}\ledrightnote{\textcolor{pink}{Sternwartestraße 71}}\pend{}{\bigskip}\pstart{}Hrn Dr. Stefan Zweig\pend{}\pstart{}\textcolor{pink}{Salzburg}\oindex{Salzburg@\textbf{Salzburg}, \emph{Verwaltungsgebiet}|pw}{}\ledrightnote{\textcolor{pink}{Salzburg}}\pend{}\pstart{}\textcolor{pink}{Kapuzinerberg 5}\oindex{Paschinger Schlössl@\textbf{Paschinger Schlössl}, \emph{Wohngebäude}|pw}{}\ledrightnote{\textcolor{pink}{Paschinger Schlössl}}.\pend{}{\bigskip}\vspace{1em}
\pstart
           \raggedleft{}{\pb}\textcolor{pink}{Wien}\oindex{Wien@\textbf{Wien}, \emph{Verwaltungsgebiet}|pw}{}\ledrightnote{\textcolor{pink}{Wien}},
                        4. 5. 925\pend
           \vspace{0.5em}
\pstart
           lieber Herr Doctor, seien Sie herzlichst bedankt für Ihr schönes \textcolor{green}{Buch}\pwindex{Zweig, Stefan 28.\,11.\,1881 Wien – 23.\,2.\,1942 Petrópolis@\textsc{Zweig, Stefan} (28.\,11.\,1881 Wien – 23.\,2.\,1942 Petrópolis), \emph{Schriftsteller}!Kampf mit dem Dämon. Hölderlin – Kleist – Nietzsche@\strich\emph{Der Kampf mit dem Dämon. Hölderlin – Kleist – Nietzsche}|pwv}{}\ledrightnote{{$\rightarrow$}\emph{\textcolor{green}{Der Kampf mit dem Dämon. Hölderlin – Kleist – Nietzsche}}}, von dem ich einzelne
               Partien schon aus der \label{K_L03746-1v}\edtext{\textcolor{green}{Zeitung}\pwindex{Neue Freie Presse@\emph{Neue Freie Presse}|pwv}{}\ledrightnote{{$\rightarrow$}\emph{\textcolor{green}{Neue Freie Presse}}}}{\lemma{\textnormal{\emph{Zeitung}}}\Cendnote{\textnormal{Als Vorabdrucke erschienen zwei Texte in der \emph{\textcolor{green}{Neue Freie Presse}\pwindex{Neue Freie Presse@\emph{Neue Freie Presse}|pwk}}: \textcolor{blue}{Stefan Zweig}\pwindex{Zweig, Stefan 28.\,11.\,1881 Wien – 23.\,2.\,1942 Petrópolis@\textsc{Zweig, Stefan} (28.\,11.\,1881 Wien – 23.\,2.\,1942 Petrópolis), \emph{Schriftsteller}|pwk}: \emph{\textcolor{green}{Hölderlins Untergang}\pwindex{Zweig, Stefan 28.\,11.\,1881 Wien – 23.\,2.\,1942 Petrópolis@\textsc{Zweig, Stefan} (28.\,11.\,1881 Wien – 23.\,2.\,1942 Petrópolis), \emph{Schriftsteller}!Hölderlins Untergang@\strich\emph{Hölderlins Untergang}|pwk}}. In: \emph{\textcolor{green}{Neue Freie
                        Presse}\pwindex{Neue Freie Presse@\emph{Neue Freie Presse}|pwk}}, Nr. 21.606, 5. 11. 1924,
                     Morgenblatt, S. 1–3; Nr. 21.609, 8. 11. 1924,
                        Morgenblatt, S. 1–3; Nr. 21.612, 11. 11. 1924,
                           Morgenblatt, S. 1–3; Nr. 21.618, 18. 11. 1924,
                              Morgenblatt, S. 1–2. \textcolor{blue}{Stefan Zweig}\pwindex{Zweig, Stefan 28.\,11.\,1881 Wien – 23.\,2.\,1942 Petrópolis@\textsc{Zweig, Stefan} (28.\,11.\,1881 Wien – 23.\,2.\,1942 Petrópolis), \emph{Schriftsteller}|pwk}: \emph{\textcolor{green}{Nietzsches Untergang}\pwindex{Zweig, Stefan 28.\,11.\,1881 Wien – 23.\,2.\,1942 Petrópolis@\textsc{Zweig, Stefan} (28.\,11.\,1881 Wien – 23.\,2.\,1942 Petrópolis), \emph{Schriftsteller}!Nietzsches Untergang@\strich\emph{Nietzsches Untergang}|pwk}}.
                              In: \emph{\textcolor{green}{Neue Freie Presse}\pwindex{Neue Freie Presse@\emph{Neue Freie Presse}|pwk}}, Nr. 21.711, 22. 2. 1925, Morgenblatt, 
                              S. 1–5.}}}\label{K_L03746-1} kannte, u das
               ich jetzt mit erhöhtem Genuſs im Zusa{\geminationm}enhang lese.\pend
           
\pstart
           Mit herzlichen Grüßen{\\[\baselineskip]}Ihr{\\[\baselineskip]}\spacefill\mbox{ArthurSchnitzler}\pend
           \leftskip=0em{}\selectlanguage{ngerman}\endnumbering\briefempfaengerindex{Zweig, Stefan@\textsc{Zweig, Stefan}!zzzSchnitzler, Arthur@\emph{von Arthur Schnitzler}!1925-05-041@{4. 5. 1925}|)be}\mylabel{L03746h}  \normalsize

\doendnotes{C}
\bigskip
\vfill

\clearpage

\footnotesize

\lohead{\textsc{register}}

% Definiere theindex-Environment komplett neu ohne reledmac
\makeatletter
\renewenvironment{theindex}{%
  \section*{\indexname}%
  \setlength{\parindent}{0pt}%
  \setlength{\parskip}{0pt plus 0.3pt}%
  \let\item\@idxitem
}{%
  \clearpage
}
\makeatother

\IfFileExists{\jobname-pw.ind}{\input{\jobname-pw.ind}}{}

\end{document}

      