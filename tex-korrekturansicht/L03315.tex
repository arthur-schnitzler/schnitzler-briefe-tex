%% latex-korrekturansicht-vorspann.tex
%% Vorspann für die Korrekturansicht.
%% Lädt die gemeinsame Datei latex-vorspann.tex mit gesetztem Schalter.

\newif\ifkorrekturansicht
\korrekturansichttrue

\input{../tex-inputs/latex-vorspann}


\renewcommand{\erwaehntePersonen}{Personen: Hugo Felix, Georges Fragerolle, Henri Rivière, Julius Szeps}
\renewcommand{\erwaehnteInstitutionen}{Institutionen: Jung-Wiener Theater zum Lieben Augustin, Le Chat Noir, Wiener Allgemeine Zeitung}
\renewcommand{\erwaehnteOrte}{Orte: Arlberg, Bad Ischl, Darmstadt, Hotel Bristol Salzburg, Innsbruck, Salzburg, St. Anton am Arlberg, Wien, Zürich, Österreich}
\renewcommand{\erwaehnteWerke}{Werke: Frau Bertha Garlan. Roman, Wiener Allgemeine Zeitung, »Lieutenant Gustl.« (Ein ehrenrätliches Urtheil.)}
\section[ Felix Salten an Arthur Schnitzler, 11. 7. 1901]{Felix Salten an Arthur Schnitzler, 11. 7. 1901}
\nopagebreak\mylabel{v}
\rehead{ }\normalsize\beginnumbering\briefempfaengerindex{Schnitzler, Arthur@\textsc{Schnitzler, Arthur}!zzzSalten, Felix@\emph{von Felix Salten}!1901-07-112@{11. 7. 1901}|(be}
\toendnotes[C]{\smallbreak\pagebreak[2]}\Standort{CUL, Schnitzler, B 89, A 2.}
\physDesc{Brief, 1 Blatt, 1 Seite, 1320 Zeichen
\newline{}Handschrift: schwarze Tinte, lateinische Kurrent
\newline{}Ordnung: mit Bleistift von unbekannter Hand nummeriert: »139« }\toendnotes[C]{\smallbreak}
\pstart
           \noindent{}{\pb}\textcolor{gray}{\textbf{\textsc{Telephon interurban Nr. 124}}}\hfill \textcolor{gray}{\textbf{\textsc{Hotel}}}\pend
           
\pstart
           \textcolor{gray}{\textbf{\textsc{Telegramm-Adresse:}}}\hfill \textcolor{gray}{\textbf{\textsc{\textcolor{pink}{Bristol}{}\ledrightnote{\textcolor{pink}{Hotel Bristol Salzburg}}}}}\pend
           
\pstart
           \textcolor{gray}{\textbf{\textsc{\textcolor{pink}{\textbf{H}otel \textbf{B}ristol \textbf{S}alzburg}{}\ledrightnote{\textcolor{pink}{Hotel Bristol Salzburg}}}.}}\hfill \textcolor{gray}{\textbf{\textsc{\textcolor{pink}{Salzburg}{}\ledrightnote{\textcolor{pink}{Salzburg}}}}}\pend
           
\pstart
           \raggedleft{}\textcolor{gray}{\textbf{(\textcolor{pink}{AUSTRIA}{}\ledrightnote{\textcolor{pink}{Österreich}})}}\pend
           
\pstart
           \raggedleft{}\textcolor{pink}{Salzburg}{}\ledrightnote{\textcolor{pink}{Salzburg}}, 11. Juli 01\pend
           
\pstart
           Lieber Freund,{ }heute fand ich hier Ihre Karte aus \label{K_L03315-1v}\edtext{\textcolor{pink}{S\textsuperscript{t} Anton}{}\ledrightnote{\textcolor{pink}{St. Anton am Arlberg}}}{\lemma{\textnormal{\emph{S\textsuperscript{t} Anton}}}\Cendnote{\textnormal{\textcolor{blue}{Schnitzler} war zwischen 30. 6. 1901 und 12. 7. 1901 in \textcolor{pink}{St. Anton am Arlberg} gewesen.}}}\label{K_L03315-1h}. Ich kam
               erst gestern{ }Abend aus \textcolor{pink}{Darmstadt}{}\ledrightnote{\textcolor{pink}{Darmstadt}} hierher. Gehe
               jetzt nach \textcolor{pink}{Ischl}{}\ledrightnote{\textcolor{pink}{Bad Ischl}}, und von da erst in 14 Tagen
               nach \textcolor{pink}{Wien}{}\ledrightnote{\textcolor{pink}{Wien}}. Durch den \textcolor{pink}{Arlberg}{}\ledrightnote{\textcolor{pink}{Arlberg}} fuhr ich gestern{ }Vormittag. Meine Reise war gut, und wol auch ergiebig. Die \textcolor{brown}{Allg. Ztg.}{}\ledrightnote{\textcolor{brown}{Wiener Allgemeine Zeitung}} hatte die \textcolor{green}{Nachricht}{}\ledrightnote{{$\rightarrow$}\textcolor{green}{»Lieutenant Gustl.« (Ein ehrenrätliches Urtheil.)}} von D\textsuperscript{r}{ }\textcolor{blue}{Szeps}{}\ledrightnote{\textcolor{blue}{Julius Szeps}}, der seine Quelle nicht nennen wollte.
               Es war am Tag meiner Abreise. D\textsuperscript{r}{ }\textcolor{blue}{Szeps}{}\ledrightnote{\textcolor{blue}{Julius Szeps}} ließ mich rufen, {\kaufmannsund} fragte mich, ob ich etwas gegen die Veröffentlichung
               hätte. Mit Rücksicht auf unser Gespräch über diesen Punkt, sagte ich, es wäre mir
               recht. Sie erinnern sich wol, dass ich Ihnen einmal sagte, wenn die Sache
               durchsickert, wäre ein Verschweigen seitens der Ihnen freundlichen Presse unklug. Das
               sähe so aus, als fühlten Sie sich wirklich getroffen {\kaufmannsund}
               bestraft, und die antis. Presse würde das zweifellos auch so darstellen. Den \label{K_L03315-2v}\edtext{\textcolor{green}{Artikel}{}\ledrightnote{{$\rightarrow$}\textcolor{green}{»Lieutenant Gustl.« (Ein ehrenrätliches Urtheil.)}}}{\lemma{\textnormal{\emph{Artikel}}}\Cendnote{\textnormal{Es dürfte von dem ohne Autornennung
                  erschienenen Text \emph{\textcolor{green}{»Lieutenant Gustl.« (Ein
                     ehrenrätliches Urtheil.)}} (\emph{\textcolor{green}{Wiener Allgemeine Zeitung}}, Nr. 6.982,
                        21. 6. 1901, 6 Uhr-Blatt, S. 4) die
                  Rede gewesen sein. Darin wurde von der Aberkennung der Offizierscharge berichtet.
                  Da mehrere Zeitungen die gleiche Nachricht am selben Tag brachten, ist nicht
                  unmittelbar zu bestimmen, ob \textcolor{blue}{Schnitzler}
                  hatte wissen wollen, wie die Information in die Zeitungen gelangt war, oder ob
                  hier eine besondere Information verbreitet worden war, über die kein anderes Blatt
                  verfügte.}}}\label{K_L03315-2h} selbst hab’ ich dann erst Abends auf der Bahn lesen
               können. Was meine weiteren Pläne betrifft, ließe viel sich darüber sagen, – brieflich
               ist’s wol aber zu umständlich. Hoffentlich \label{K_L03315-3v}\edtext{sehen wir uns bald}{\lemma{\textnormal{\emph{sehen wir uns bald}}}\Cendnote{\textnormal{Nachweislich sahen sich \textcolor{blue}{Salten} und \textcolor{blue}{Schnitzler} erst am 1. 9. 1901
                  wieder.}}}\label{K_L03315-3h}. Wenn nicht, – im September? Ich habe
               die \textcolor{blue}{Fragerolles}{}\ledrightnote{\textcolor{blue}{Georges Fragerolle}}-\textcolor{blue}{Rivière}{}\ledrightnote{\textcolor{blue}{Henri Rivière}}’schen \label{K_L03315-4v}\edtext{Schattenspiele}{\lemma{\textnormal{\emph{Schattenspiele}}}\Cendnote{\textnormal{Im Kabarett \emph{\textcolor{brown}{Le chat noir}} wurden zwischen 1888 und 1897 fast 50 Stücke aufgeführt, für
                  die \textcolor{blue}{Henri Rivière} die Ausstattung und \textcolor{blue}{Georges Fragerolles} die Musik
                  verantwortete.}}}\label{K_L03315-4h} erworben (Geheimnis) und in \textcolor{pink}{Zürich}{}\ledrightnote{\textcolor{pink}{Zürich}} mit \textcolor{blue}{Felix}{}\ledrightnote{\textcolor{blue}{Hugo Felix}}{ }\label{K_L03315-5v}\edtext{Contract}{\lemma{\textnormal{\emph{Contract}}}\Cendnote{\textnormal{für das \emph{\textcolor{brown}{Jung-Wiener Theater
                     zum Lieben Augustin}}}}}\label{K_L03315-5h} gemacht. Vielleicht komme ich in \textcolor{pink}{Ischl}{}\ledrightnote{\textcolor{pink}{Bad Ischl}}
               dazu \label{K_L03315-6v}\edtext{über \textcolor{green}{Bertha Garlan}{}\ledrightnote{\textcolor{green}{Frau Bertha Garlan. Roman}} zu schreiben}{\lemma{\textnormal{\emph{über … schreiben}}}\Cendnote{\textnormal{Dazu kam es nicht, vgl. Felix Salten an Arthur Schnitzler, 22. 5. 1902.}}}\label{K_L03315-6h}, wenn nicht, dann im August in \textcolor{pink}{Wien}{}\ledrightnote{\textcolor{pink}{Wien}}. Schreiben
               Sie mir bald wieder.\pend
           \pstart Herzlichst Ihr \spacefill\mbox{Salten}\pend{}\endnumbering\briefempfaengerindex{Schnitzler, Arthur@\textsc{Schnitzler, Arthur}!zzzSalten, Felix@\emph{von Felix Salten}!1901-07-112@{11. 7. 1901}|)be}\mylabel{h}  \normalsize

\doendnotes{C}
\bigskip
\vfill

\clearpage

\footnotesize

\lohead{\textsc{register}}

% Definiere theindex-Environment komplett neu ohne reledmac
\makeatletter
\renewenvironment{theindex}{%
  \section*{\indexname}%
  \setlength{\parindent}{0pt}%
  \setlength{\parskip}{0pt plus 0.3pt}%
  \let\item\@idxitem
}{%
  \clearpage
}
\makeatother

\IfFileExists{\jobname-pw.ind}{\input{\jobname-pw.ind}}{}

\end{document}

      