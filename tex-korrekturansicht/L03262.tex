%% latex-korrekturansicht-vorspann.tex
%% Vorspann für die Korrekturansicht.
%% Lädt die gemeinsame Datei latex-vorspann.tex mit gesetztem Schalter.

\newif\ifkorrekturansicht
\korrekturansichttrue

\input{../tex-inputs/latex-vorspann}


\renewcommand{\erwaehntePersonen}{Personen: Paul Blasel}
\renewcommand{\erwaehnteInstitutionen}{Institutionen: Stadttheater (Teplitz)}
\renewcommand{\erwaehnteOrte}{Orte: Café Arkaden, Teplice, Wien}
\renewcommand{\erwaehnteWerke}{}
\section[ Felix Salten an Arthur Schnitzler, {[}10. 1. 1897{]}]{Felix Salten an Arthur Schnitzler, {[}10. 1. 1897{]}}
\nopagebreak\mylabel{v}
\rehead{ }\normalsize\beginnumbering\briefempfaengerindex{Schnitzler, Arthur@\textsc{Schnitzler, Arthur}!zzzSalten, Felix@\emph{von Felix Salten}!1897-01-101@{{[}10. 1. 1897{]}}|(be}
\toendnotes[C]{\smallbreak\pagebreak[2]}\Standort{CUL, Schnitzler, B 89, A 2.}
\physDesc{Karte, 142 Zeichen
\newline{}Handschrift: Bleistift, lateinische Kurrent
\newline{}Schnitzler: mit Bleistift datiert: »10. 1. 97« und die Tagesangabe ungenau eingekringelt 
\newline{}Ordnung: mit Bleistift von unbekannter Hand nummeriert: »84« }\toendnotes[C]{\smallbreak}
\pstart
           \noindent{}{\pb}Lieber Arthur, ich muß Sie nothwendig noch \label{K_L03262-1v}\edtext{heute{ }Abend sprechen}{\lemma{\textnormal{\emph{heute Abend sprechen}}}\Cendnote{\textnormal{\textcolor{blue}{Paul Blasel} hatte zum Jahreswechsel
                  bekanntgegeben, dass er nach zwei Spielzeiten die Leitung des \emph{\textcolor{brown}{Stadttheaters}} in \textcolor{pink}{Teplitz} mit Ablauf der Saison zurückgeben werde. \textcolor{blue}{Salten} bemühte sich um die Nachfolge. Am Abend versuchte er
                  (vergeblich), von \textcolor{blue}{Schnitzler} das Geld für
                  die Kaution zu leihen. Siehe auch Felix Salten an Arthur Schnitzler, 6. 5. 1899
                  und Felix Salten an Arthur Schnitzler, 1[3]. 5. 1899.}}}\label{K_L03262-1h}. Um 12
               etwa werde ich Sie im \textcolor{pink}{Arcaden-Café}{}\ledrightnote{\textcolor{pink}{Café Arkaden}} erwarten.
               Kommen Sie hin, ja?\pend
           
\pstart
           Herzlich {\\[\baselineskip]}\spacefill\mbox{Salten}\pend
           \leftskip=0em{}\endnumbering\briefempfaengerindex{Schnitzler, Arthur@\textsc{Schnitzler, Arthur}!zzzSalten, Felix@\emph{von Felix Salten}!1897-01-101@{{[}10. 1. 1897{]}}|)be}\mylabel{h}  \normalsize

\doendnotes{C}
\bigskip
\vfill

\clearpage

\footnotesize

\lohead{\textsc{register}}

% Definiere theindex-Environment komplett neu ohne reledmac
\makeatletter
\renewenvironment{theindex}{%
  \section*{\indexname}%
  \setlength{\parindent}{0pt}%
  \setlength{\parskip}{0pt plus 0.3pt}%
  \let\item\@idxitem
}{%
  \clearpage
}
\makeatother

\IfFileExists{\jobname-pw.ind}{\input{\jobname-pw.ind}}{}

\end{document}

      