%% latex-korrekturansicht-vorspann.tex
%% Vorspann für die Korrekturansicht.
%% Lädt die gemeinsame Datei latex-vorspann.tex mit gesetztem Schalter.

\newif\ifkorrekturansicht
\korrekturansichttrue

\input{../tex-inputs/latex-vorspann}


\renewcommand{\erwaehntePersonen}{Personen: Henrik Ibsen, Friedrich Mitterwurzer, Felix Salten}
\renewcommand{\erwaehnteOrte}{Orte: Berlin, Edmund-Weiß-Gasse 7, Wien}
\renewcommand{\erwaehnteWerke}{Werke: B.Z. am Mittag, Totenrede auf Henrik Ibsen}
\section[ Arthur Schnitzler an Felix Salten, 26. 5. 1906]{Arthur Schnitzler an Felix Salten, 26. 5. 1906}
\nopagebreak\mylabel{v}
\rehead{ }\normalsize\beginnumbering\briefempfaengerindex{Salten, Felix@\textsc{Salten, Felix}!zzzSchnitzler, Arthur@\emph{von Arthur Schnitzler}!1906-05-261@{26. 5. 1906}|(be}
\toendnotes[C]{\smallbreak\pagebreak[2]}\Standort{Wienbibliothek im Rathaus, ZPH 1681, 2.1.516.}
\physDesc{Karte, 194 Zeichen
\newline{}Handschrift: schwarze Tinte, deutsche Kurrent
\newline{}Ordnung: mit Bleistift von unbekannter Hand nummeriert: »11« }\toendnotes[C]{\smallbreak}
\pstart
           \noindent{}{\pb}\textcolor{gray}{\textbf{Dr. Arthur Schnitzler}}\hfill 26. Mai 906\pend
           
\pstart
           \textcolor{gray}{\textbf{\textcolor{pink}{Wien, XVIII. Spoettelgasse 7}{}\ledrightnote{\textcolor{pink}{Edmund-Weiß-Gasse 7}}.}}\pend
           
\pstart
           lieber, wundervoll, ergreifend iſt Ihre \label{K_L03006-1v}\edtext{\textcolor{green}{Todtenrede auf \textcolor{blue}{Ibſen}{}\ledrightnote{\textcolor{blue}{Henrik Ibsen}}}{}\ledrightnote{{$\rightarrow$}\textcolor{green}{Totenrede auf Henrik Ibsen}}}{\lemma{\textnormal{\emph{Todtenrede auf Ibſen}}}\Cendnote{\textnormal{\textcolor{blue}{Felix Salten}: \emph{\textcolor{green}{Totenrede auf Henrik Ibsen}}. In: \emph{\textcolor{green}{B. Z. am Mittag}}, Jg. 30, Nr. 121, 25. 5. 1906, Zweites Beiblatt, S. 1.}}}\label{K_L03006-1h}.
               Man möchte ſie von \textcolor{blue}{Mitterwurzer}{}\ledrightnote{\textcolor{blue}{Friedrich Mitterwurzer}} geleſen
               hören.\pend
           
\pstart
           {\pb}Iſt es wahr, dſs Sie \label{K_L03006-2v}\edtext{zu Pfingſten
               nach \textcolor{pink}{Wien}{}\ledrightnote{\textcolor{pink}{Wien}} kommen? Und am 23. wieder}{\lemma{\textnormal{\emph{zu … wieder}}}\Cendnote{\textnormal{Das dürfte nicht
                  passiert sein, siehe die Folgebriefe.}}}\label{K_L03006-2h}–?\pend
           
\pstart
           Von Herzen Ihr {\\[\baselineskip]}\spacefill\mbox{A.}\pend
           \leftskip=0em{}\endnumbering\briefempfaengerindex{Salten, Felix@\textsc{Salten, Felix}!zzzSchnitzler, Arthur@\emph{von Arthur Schnitzler}!1906-05-261@{26. 5. 1906}|)be}\mylabel{h}  \normalsize

\doendnotes{C}
\bigskip
\vfill

\clearpage

\footnotesize

\lohead{\textsc{register}}

% Definiere theindex-Environment komplett neu ohne reledmac
\makeatletter
\renewenvironment{theindex}{%
  \section*{\indexname}%
  \setlength{\parindent}{0pt}%
  \setlength{\parskip}{0pt plus 0.3pt}%
  \let\item\@idxitem
}{%
  \clearpage
}
\makeatother

\IfFileExists{\jobname-pw.ind}{\input{\jobname-pw.ind}}{}

\end{document}

      