%% latex-korrekturansicht-vorspann.tex
%% Vorspann für die Korrekturansicht.
%% Lädt die gemeinsame Datei latex-vorspann.tex mit gesetztem Schalter.

\newif\ifkorrekturansicht
\korrekturansichttrue

\input{../tex-inputs/latex-vorspann}


\renewcommand{\erwaehntePersonen}{Personen: Frieda Pollak, Felix Salten}
\renewcommand{\erwaehnteOrte}{Orte: Galleria Umberto I, Neapel, Sternwartestraße 71, Wien, via San Carlo, Österreich}
\renewcommand{\erwaehnteWerke}{}
\section[ Felix Salten an Arthur Schnitzler, {[}9?{]}. 4. 1922]{Felix Salten an Arthur Schnitzler, {[}9?{]}. 4. 1922}
\nopagebreak\mylabel{v}
\rehead{ }\normalsize\beginnumbering\briefempfaengerindex{Schnitzler, Arthur@\textsc{Schnitzler, Arthur}!zzzSalten, Felix@\emph{von Felix Salten}!1922-04-091@{{[}9?{]}. 4. 1922}|(be}
\toendnotes[C]{\smallbreak\pagebreak[2]}\Standort{CUL, Schnitzler, B 89, B 2.}
\physDesc{Bildpostkarte, 93 Zeichen
\newline{}Handschrift: schwarze Tinte, lateinische Kurrent
\newline{}Versand: Stempel: »\nobreak{}\oindex{Neapel@\textbf{Neapel}, \emph{P.PPLA}|pwk}Napo\textcolor{gray}{l}{[}i{]}, \textcolor{gray}{0}9. 4. 22\nobreak{}«.  
\newline{}Ordnung: 1) mit Bleistift von \textcolor{blue}{Frieda Pollak} (?) mit
                                 dem Buchstaben »A« (Abgeschrieben/Abschrift)
                                 gekennzeichnet  2) mit Bleistift von unbekannter Hand nummeriert: »29\substVorne{}\textsuperscript{0}\substDazwischen{}1\substHinten{}«}\toendnotes[C]{\smallbreak}\pstart{}{\pb}\textcolor{pink}{Austria}{}\ledrightnote{\textcolor{pink}{Österreich}}\pend{}\pstart{}Herrn D\textsuperscript{r} Arthur Schnitzler\pend{}\pstart{}\textcolor{pink}{XVIII. Sternwartestraße 71}{}\ledrightnote{\textcolor{pink}{Sternwartestraße 71}}\pend{}\pstart{}\textcolor{pink}{Wien}{}\ledrightnote{\textcolor{pink}{Wien}}.\pend{}
{\bigskip}
\pstart
           \noindent{}\centering{}{\pb}\textcolor{gray}{\textbf{\label{K_L03578-1v}\edtext{\textcolor{pink}{NAPOLI}{}\ledrightnote{\textcolor{pink}{Neapel}}}{\lemma{\textnormal{\emph{Napoli}}}\Cendnote{\textnormal{Die Tagesangabe ist nicht
                           eindeutig zu entziffern, vermutlich handelt es sich aber um
                              »09«. Eine verwischte ›1‹ vor der »9«
                           ist auch denkbar, da \textcolor{blue}{Salten} am 21. 4. 1922 ebenso
                           aus \textcolor{pink}{Neapel} schrieb.}}}\label{K_L03578-1h} – \textcolor{pink}{Via S. Carlo}{}\ledrightnote{\textcolor{pink}{via San Carlo}} – \textcolor{pink}{Galleria Umberto I.}{}\ledrightnote{\textcolor{pink}{Galleria Umberto I}}}}\pend
           
\pstart
           {\pb}Herzliche Grüße {\\}Ihr {\\}\spacefill\mbox{Felix Salten}\pend
           \endnumbering\briefempfaengerindex{Schnitzler, Arthur@\textsc{Schnitzler, Arthur}!zzzSalten, Felix@\emph{von Felix Salten}!1922-04-091@{{[}9?{]}. 4. 1922}|)be}\mylabel{h}  \normalsize

\doendnotes{C}
\bigskip
\vfill

\clearpage

\footnotesize

\lohead{\textsc{register}}

% Definiere theindex-Environment komplett neu ohne reledmac
\makeatletter
\renewenvironment{theindex}{%
  \section*{\indexname}%
  \setlength{\parindent}{0pt}%
  \setlength{\parskip}{0pt plus 0.3pt}%
  \let\item\@idxitem
}{%
  \clearpage
}
\makeatother

\IfFileExists{\jobname-pw.ind}{\input{\jobname-pw.ind}}{}

\end{document}

      