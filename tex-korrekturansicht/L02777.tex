%% latex-korrekturansicht-vorspann.tex
%% Vorspann für die Korrekturansicht.
%% Lädt die gemeinsame Datei latex-vorspann.tex mit gesetztem Schalter.

\newif\ifkorrekturansicht
\korrekturansichttrue

\input{../tex-inputs/latex-vorspann}


               \section[Paul Goldmann an Arthur Schnitzler, Paul Goldmann an Arthur Schnitzler, 15. 6. {[}1896{]}]{ Paul Goldmann an Arthur Schnitzler, 15. 6. {[}1896{]}}\nopagebreak\mylabel{v}\rehead{ }\normalsize\beginnumbering\briefempfaengerindex{Schnitzler, Arthur@\textsc{Schnitzler, Arthur}!zzzGoldmann, Paul@\emph{von Paul Goldmann}!1896-06-151@{15. 6. {[}1896{]}}|(be} \toendnotes[C]{\smallbreak\pagebreak[2]} \Standort{DLA, A:Schnitzler, HS.NZ85.1.3166.}
\physDesc{Brief, 1 Blatt, 3 Seiten
\newline{}Handschrift: blaue Tinte, deutsche Kurrent\newline{}Beilage: maschinenschriftlicher Brief: 1 Blatt, 2 Seiten, mit
                                 handschriftlicher Unterschrift in schwarzer Tinte 
\newline{}Schnitzler: mit Bleistift das Jahr »96« vermerkt }\toendnotes[C]{\smallbreak}\pstart
           \noindent{}{\pb}\textcolor{gray}{\textbf{\textbf{\textcolor{brown}{Frankfurter Zeitung}{}\ledrightnote{\textcolor{brown}{Frankfurter Zeitung}}}}}\pend
           \pstart
           \textcolor{gray}{\textbf{(\textcolor{brown}{\begin{otherlanguage}{french}Gazette de Francfort\end{otherlanguage}}{}\ledrightnote{\textcolor{brown}{Frankfurter Zeitung}}).}}\pend
           \pstart
           \textcolor{gray}{\textbf{\textbf{\begin{otherlanguage}{french}Fondateur M.\end{otherlanguage}{ }\textcolor{blue}{L. Sonnemann}{}\ledrightnote{\textcolor{blue}{Leopold Sonnemann}}.}}}\pend
           \pstart
           \begin{otherlanguage}{french}\textcolor{gray}{\textbf{\textcolor{green}{Journal}{}\ledrightnote{→\textcolor{green}{Frankfurter Zeitung}} politique,
                        financier,}}\end{otherlanguage}\pend
           \pstart
           \begin{otherlanguage}{french}\textcolor{gray}{\textbf{commercial et littéraire.}}\end{otherlanguage}\pend
           \pstart
           \begin{otherlanguage}{french}\textcolor{gray}{\textbf{\textbf{Paraissant trois fois par jour.}}}\end{otherlanguage}\pend
           \pstart
           \begin{otherlanguage}{french}\textcolor{gray}{\textbf{\textbf{Bureau à \textcolor{pink}{Paris}{}\ledrightnote{\textcolor{pink}{Paris}}}}}\end{otherlanguage}\hfill \textsc{\textcolor{pink}{Paris}{}\ledrightnote{\textcolor{pink}{Paris}}}, 15. Juni.\pend
           \pstart
           \begin{otherlanguage}{french}\textcolor{gray}{\textbf{\textbf{\textcolor{pink}{24. Rue Feydeau}{}\ledrightnote{\textcolor{pink}{rue Feydeau}}.}}}\end{otherlanguage}\pend
           \pstart{}Mein lieber Freund,\pend\pstart
           Anbei erhältſt Du die »\textsc{\textcolor{green}{Nouvelle Revue}{}\ledrightnote{\textcolor{green}{La Nouvelle Revue}}}« mit dem \label{K_L02777-1v}\edtext{\textcolor{green}{Artikel}{}\ledrightnote{→\textcolor{green}{Un jeune écrivain viennois: M. Arthur Schnitzler}}}{\lemma{\textnormal{\emph{Artikel}}}\Cendnote{\textnormal{\textcolor{blue}{Christian Schefer}: \emph{\textcolor{green}{Un jeune écrivain viennois: M. Arthur Schnitzler}}. In:
                        \emph{\textcolor{green}{La Nouvelle Revue}}, Jg. 18, Nr. 100,
                        Mai–Juni 1896,
                     S. 855–859.}}}\label{K_L02777-1h} über Dich. Die Eindrücke ſind nicht ſtichhaltig, aber
               ich finde den \textcolor{green}{Artikel}{}\ledrightnote{→\textcolor{green}{Un jeune écrivain viennois: M. Arthur Schnitzler}} ſehr
               liebenswürdig, beſonders mit Rückſicht auf die \textcolor{green}{Stelle}{}\ledrightnote{→\textcolor{green}{Un jeune écrivain viennois: M. Arthur Schnitzler}}, wo er \strikeout{ſich} ſteht,
               denn ſonſt iſt man dort ſehr gegen alles Deutſche. Auch den Brief von \textsc{M. \textcolor{blue}{Christian Schefer}{}\ledrightnote{\textcolor{blue}{Christian Schefer}}} lege ich bei; ſeine Adreſſe ſteht oben; nur mußt Du ſchreiben \textsc{\textcolor{pink}{Melun, \label{K_L02777-2v}\edtext{\begin{otherlanguage}{french}près\end{otherlanguage}}{\lemma{\textnormal{\emph{près}}}\Cendnote{\textnormal{französisch: nahe}}}\label{K_L02777-2h}
                     Paris}{}\ledrightnote{\textcolor{pink}{Melun}}}. Du dankſt ihm wohl mit einigen artigen Worten. {\pb}Wenn Du willſt, kannſt Du Dich auch gegen die
               Einwände rechtfertigen. Das wird ihm ſehr ſchmeicheln. Schreib ihm deutſch und
               entſchuldige Dich, daß Du nicht des Franzöſiſchen mächtig genug biſt, um ihm in
               ſeiner Sprache zu ſchreiben{\dotsfive}\pend
           \pstart
           Mit meiner Zuſage betreffs des Rendezvous in \label{K_L02777-4v}\edtext{\textcolor{pink}{Dänemark}{}\ledrightnote{\textcolor{pink}{Dänemark}}}{\lemma{\textnormal{\emph{Dänemark}}}\Cendnote{\textnormal{siehe Paul Goldmann an Arthur Schnitzler, 29. 4. [1896]}}}\label{K_L02777-4h} bin ich leichtſinnig geweſen. Ich habe nicht an die Koſten gedacht. Nach
               eingezogenen Erkundigungen ſtellt ſich die Eiſenbahn-Reiſe \textsc{\textcolor{pink}{Paris}{}\ledrightnote{\textcolor{pink}{Paris}} – \textcolor{pink}{Kopenhagen}{}\ledrightnote{\textcolor{pink}{Kopenhagen}} – \textcolor{pink}{Berlin}{}\ledrightnote{\textcolor{pink}{Berlin}} – \textcolor{pink}{Paris}{}\ledrightnote{\textcolor{pink}{Paris}}} allein {\pb}auf über 230 \textsc{Francs}, mit allen Rundreiſe-Ermäßigungen. Das geht über meine Kräfte. So
               werde ich wohl \substVorne{}\textsuperscript{\textcolor{gray}{gru}}\substDazwischen{}zu\substHinten{} meinem anfänglichen Project einer Reiſe nach der \textcolor{pink}{Schweiz}{}\ledrightnote{\textcolor{pink}{Schweiz}} zurückkehren müſſen, wo ich in einer Nacht hinkann,
               und wir werden uns in dieſem Jahre wohl kaum ſehen.\pend
           \pstart
           Wie gehts, liebſter Freund?\pend
           \pstart
           Wann trittſt Du Deine \label{K_L02777-8v}\edtext{Fahrt nach
                  Norden}{\lemma{\textnormal{\emph{Fahrt nach
                  Norden}}}\Cendnote{\textnormal{\textcolor{blue}{Schnitzler} kam am 4. 7. 1896 nach \textcolor{pink}{Hamburg}, wo er sich einschiffte. Am 9. 7. 1896 erreichte
                  er den »Norden«, er langte in \textcolor{pink}{Stavanger} (\textcolor{pink}{Norwegen}) an.}}}\label{K_L02777-8h}
               an?\pend
           \pstart
           Von Herzen Dein {\\[\baselineskip]}\spacefill\mbox{Paul Goldmann}\pend
           \leftskip=0em{}{\bigskip}\pstart
           \raggedleft{}{\pb}{[}ms.:{]} \textcolor{pink}{MELUN, 12 rue Doré}{}\ledrightnote{\textcolor{pink}{Rue Doré}}, ce
                     mercredi.\pend
           \pstart{}\begin{otherlanguage}{french}Mon cher Monsieur,\end{otherlanguage}\pend\pstart
           \begin{otherlanguage}{french}\label{K_L02777-6v}\edtext{J’ai bien des excuses à vous faire
                  pour ne vous pas avoir renvoyé plus tôt, le \textcolor{green}{numéro}{}\ledrightnote{→\textcolor{green}{Neue Deutsche Rundschau}} de la \textcolor{green}{Freie
                     Bühne}{}\ledrightnote{\textcolor{green}{Neue Deutsche Rundschau}} que je mets à la \textcolor{brown}{poste}{}\ledrightnote{→\textcolor{brown}{Französische Post}} en même temps que cette lettre. Je viens d’être assez souffrant
                  pendant plusieurs jours; sachant cela, j’espère que vous ne m’en voudrez pas de
                  mon inexactitude. – J’ai demandé à \textcolor{brown}{Nouvelle
                     Revue}{}\ledrightnote{\textcolor{brown}{Nouvelle Revue}} de vous faire parvenir, en épreuves corrigées, deux ou trois
                  exemplaires de la \textcolor{green}{chronique}{}\ledrightnote{→\textcolor{green}{Un jeune écrivain viennois: M. Arthur Schnitzler}}
                  que nous allons publier sur M. Schnitzler. Vous allez, je pense, les recevoir.
                  J’ai supposé, que si vous connaissiez quelque \textcolor{green}{journal}{}\ledrightnote{→\textcolor{green}{La Nouvelle Revue}} ami de M. Schnitzler, il vous serait agréable de
                  pouvoir lui faire parvenir ce \textcolor{green}{article}{}\ledrightnote{→\textcolor{green}{Un jeune écrivain viennois: M. Arthur Schnitzler}} avant sa publication. Ce n’est pas que l’\textcolor{green}{article}{}\ledrightnote{→\textcolor{green}{Un jeune écrivain viennois: M. Arthur Schnitzler}} soit aussi important que je
                  l’eusse souhaité, mais enfin, c’est le premier qui parait en \textcolor{pink}{France}{}\ledrightnote{\textcolor{pink}{Frankreich}}. D’autre part, si j’ai fait, çà et là, les quelques
                  réserves que me dictait mon désir d’être parfaitement sincère, je pense néanmoins
                  que vous ne serez pas mécontent de la manière dont j’ai parlé de votre ami.}{\lemma{\textnormal{\emph{J’ai … ami.}}}\Cendnote{\textnormal{französisch: Mein lieber Herr,
                        ich muss mich bei Ihnen entschuldigen, dass ich Ihnen die \textcolor{green}{Nummer} der \textcolor{green}{Freien Bühne}, die ich zusammen mit diesem Brief auf
                        die \textcolor{brown}{Post} gegeben habe,
                        nicht früher zurückgeschickt habe. Ich war gerade mehrere Tage lang ziemlich
                        krank; das wissend, hoffe ich, dass Sie mir meine Ungenauigkeit nicht übel
                        nehmen. – Ich habe die \textcolor{brown}{Nouvelle Revue}
                        gebeten, Ihnen zwei oder drei Exemplare der \textcolor{green}{Besprechung}, die wir über Herrn
                           \textcolor{blue}{Schnitzler} veröffentlichen werden,
                        in korrigierten Abzügen zukommen zu lassen. Ich denke, Sie werden sie
                        erhalten. Ich habe angenommen, dass Sie, wenn Sie eine \textcolor{green}{Zeitung} kennen, die mit Herrn \textcolor{blue}{Schnitzler} befreundet ist, diesen \textcolor{green}{Artikel} vor seiner
                        Veröffentlichung an diese übermitteln könnten. Nicht, dass der \textcolor{green}{Artikel} so wichtig
                        wäre, wie ich es mir gewünscht hätte, aber es ist der erste, der in \textcolor{pink}{Frankreich} erscheint. Ich habe zwar hier
                        und da ein paar Vorbehalte gemacht, die mir mein Wunsch nach vollkommener
                        Aufrichtigkeit diktierte, aber ich denke, dass Sie mit der Art und Weise,
                        wie ich über Ihren \textcolor{blue}{Freund} gesprochen habe, nicht unzufrieden sein werden.}}}\label{K_L02777-6h}\end{otherlanguage}\pend
           \pstart
           \label{K_L02777-77v}\edtext{\begin{otherlanguage}{french}J’ai réflechi de nouveau à tout ce que vous avez bien {\pb}voulu me dire l’autre jour, et je vais en
                  faire mon profit. Me voici, toutefois, obligé, à ma grande confusion, de vous
                  importuner encore d’une demande de renseignements. Vous m’avez signalé, les drames
                  italiens qui se jouent en Allemagne: serait abuser de votre complaisance que vous
                  prier de m’indiquer un ou deux titres?. D’autre part, vous m’avez
                  parlé des \uline{littérateurs} qui ont imité \textcolor{blue}{Wagner}{}\ledrightnote{\textcolor{blue}{Richard Wagner}} et de ceux qui, ont jugé à propos,
                  d’assassiner leurs contemporains à l’aide du Stabreim: à ce propos là, encore, un
                  ou deux noms ou titres, me rempliraient de joie.\end{otherlanguage}}{\lemma{\textnormal{\emph{J’ai … joie.}}}\Cendnote{\textnormal{französisch: Ich habe noch
                     einmal über alles nachgedacht, was Sie mir damals gesagt haben und ich werde
                     meinen Nutzen daraus ziehen. Ich bin jedoch zu meiner großen Verwirrung
                     gezwungen, Sie erneut mit der Bitte um Rat zu behelligen. Sie haben mich auf
                     italienische Dramen hingewiesen, die in Deutschland aufgeführt werden: Wäre es
                     ein Missbrauch Ihrer Gefälligkeit, wenn ich Sie bitten würde, mir einen oder
                     zwei Titel zu nennen? Andererseits haben Sie mir von den \uline{Literaten} erzählt, die \textcolor{blue}{Wagner} nachgeahmt haben, und von denen, die es für angebracht hielten,
                     ihre Zeitgenossen mit Hilfe des Stabreim zu ermorden: auch hier würden mich ein
                     oder zwei Namen oder Titel mit Freude erfüllen.}}}\label{K_L02777-77h}\pend
           \pstart
           \label{K_L02777-66v}\edtext{\begin{otherlanguage}{french}Encore toutes mes excuses pour mon indiscrétion, et en même
                  temps que pour mes nouveaux remerciements pour les précieux renseignements que
                  vous m’avez fournis déjà, veuillez, je vous prie, Mon cher Monsieur, agréer
                  l’expression de mes sentiments les plus distingués.\end{otherlanguage}}{\lemma{\textnormal{\emph{Encore … distingués.}}}\Cendnote{\textnormal{französisch: Neuerlich bitte ich
                     um Entschuldigung für meine Unaufmerksamkeit und sende gleichzeitig erneut Dank
                     für die wertvollen Informationen, die Sie mir bereits gegeben haben. Bitte
                     nehmen Sie, mein lieber Herr, den Ausdruck meiner vornehmsten Gefühle
                     entgegen.}}}\label{K_L02777-66h}\pend
           \pstart {[}hs. Schefer:{]} \spacefill\mbox{\textcolor{blue}{Christian Schefer}{}\ledrightnote{\textcolor{blue}{Christian Schefer}}}\pend{}\endnumbering\briefempfaengerindex{Schnitzler, Arthur@\textsc{Schnitzler, Arthur}!zzzGoldmann, Paul@\emph{von Paul Goldmann}!1896-06-151@{15. 6. {[}1896{]}}|)be}\mylabel{h}  \normalsize

\doendnotes{C}
\bigskip
\vfill

\clearpage

\footnotesize

\lohead{\textsc{register}}

% Definiere theindex-Environment komplett neu ohne reledmac
\makeatletter
\renewenvironment{theindex}{%
  \section*{\indexname}%
  \setlength{\parindent}{0pt}%
  \setlength{\parskip}{0pt plus 0.3pt}%
  \let\item\@idxitem
}{%
  \clearpage
}
\makeatother

\IfFileExists{\jobname-pw.ind}{\input{\jobname-pw.ind}}{}

\end{document}

      