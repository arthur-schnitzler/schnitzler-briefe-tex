%% latex-korrekturansicht-vorspann.tex
%% Vorspann für die Korrekturansicht.
%% Lädt die gemeinsame Datei latex-vorspann.tex mit gesetztem Schalter.

\newif\ifkorrekturansicht
\korrekturansichttrue

\input{../tex-inputs/latex-vorspann}


\renewcommand{\erwaehntePersonen}{Personen: Anna Katharina Rehmann, Felix Salten, Heinrich Schnitzler, Olga Schnitzler}
\renewcommand{\erwaehnteOrte}{Orte: Ampezzo, Edlach, Höhlenstein, Lago di Landro, Monte Cristallo, Niederösterreich, Südtirol}
\renewcommand{\erwaehnteWerke}{}
\section[ Felix Salten an Arthur Schnitzler, 18. 7. 1909]{Felix Salten an Arthur Schnitzler, 18. 7. 1909}
\nopagebreak\mylabel{v}
\rehead{ }\normalsize\beginnumbering\briefempfaengerindex{Schnitzler, Arthur@\textsc{Schnitzler, Arthur}!zzzSalten, Felix@\emph{von Felix Salten}!1909-07-182@{18. 7. 1909}|(be}
\toendnotes[C]{\smallbreak\pagebreak[2]}\Standort{CUL, Schnitzler, B 89, B 1.}
\physDesc{Bildpostkarte, 400 Zeichen
\newline{}Handschrift: schwarze Tinte, lateinische Kurrent
\newline{}Versand: Stempel: »\nobreak{}\oindex{Hoehlenstein@\textbf{Höhlenstein}, \emph{P.PPLQ}|pwk}{[}L{]}an\textcolor{gray}{dr}o\nobreak{}«.  
\newline{}Schnitzler: mit Bleistift Vermerk: »\textsc{Salten}« 
\newline{}Ordnung: mit Bleistift von unbekannter Hand nummeriert: »253« }\toendnotes[C]{\smallbreak}\pstart{}{\pb}Herrn\pend{}\pstart{}D\textsuperscript{r} Arthur Schnitzler\pend{}\pstart{}\textcolor{pink}{Edlach \textsuperscript{b}/Reichenau}{}\ledrightnote{\textcolor{pink}{Edlach}}\pend{}\pstart{}\textcolor{pink}{Nied. Öst.}{}\ledrightnote{\textcolor{pink}{Niederösterreich}}\pend{}
{\bigskip}
\pstart
           \noindent{}\centering{}{\pb}\textcolor{gray}{\textbf{\textcolor{pink}{Dürrensee}{}\ledrightnote{\textcolor{pink}{Lago di Landro}} (1410 m) mit \textcolor{pink}{Monte Cristallo}{}\ledrightnote{\textcolor{pink}{Monte Cristallo}} (3199 m) \textcolor{pink}{Ampezzo}{}\ledrightnote{\textcolor{pink}{Ampezzo}}{\dotstwo}{ }\textcolor{pink}{Tirol}{}\ledrightnote{\textcolor{pink}{Südtirol}}.}}\pend
           
\pstart
           {\pb}Lieber,{ }\uline{sehr} erfreut, dass es dem \textcolor{blue}{Heini}{}\ledrightnote{\textcolor{blue}{Heinrich Schnitzler}} schon besser geht. Auch \textcolor{blue}{Annerle}{}\ledrightnote{\textcolor{blue}{Anna Katharina Rehmann}} ist wieder munter, und die drohende Malaria gott sei
               dank nicht eingetroffen. Uns geht’s \textcolor{pink}{hier}{}\ledrightnote{{$\rightarrow$}\textcolor{pink}{Höhlenstein}} ganz gut, die Leute stören nicht, das \label{K_L03503-1v}\edtext{Hotel}{\lemma{\textnormal{\emph{Hotel}}}\Cendnote{\textnormal{nicht
                  ermittelt}}}\label{K_L03503-1h} ist angenehm; das Wetter allein von einer kalten Freundlichkeit.
               Alles Gute Ihnen \textcolor{blue}{Dreien}{}\ledrightnote{{$\rightarrow$}\textcolor{blue}{Olga Schnitzler}{\newline}{$\rightarrow$}\textcolor{blue}{Heinrich Schnitzler}}! \pend
           
\pstart
           Herzliche Grüße von uns zu Ihnen {\\[\baselineskip]}Ihr {\\[\baselineskip]}\spacefill\mbox{Salten}\pend
           \leftskip=0em{}
\pstart
           \textcolor{pink}{Landro}{}\ledrightnote{\textcolor{pink}{Höhlenstein}}, 18. VII. 09.\pend
           \endnumbering\briefempfaengerindex{Schnitzler, Arthur@\textsc{Schnitzler, Arthur}!zzzSalten, Felix@\emph{von Felix Salten}!1909-07-182@{18. 7. 1909}|)be}\mylabel{h}  \normalsize

\doendnotes{C}
\bigskip
\vfill

\clearpage

\footnotesize

\lohead{\textsc{register}}

% Definiere theindex-Environment komplett neu ohne reledmac
\makeatletter
\renewenvironment{theindex}{%
  \section*{\indexname}%
  \setlength{\parindent}{0pt}%
  \setlength{\parskip}{0pt plus 0.3pt}%
  \let\item\@idxitem
}{%
  \clearpage
}
\makeatother

\IfFileExists{\jobname-pw.ind}{\input{\jobname-pw.ind}}{}

\end{document}

      