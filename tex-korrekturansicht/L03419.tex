%% latex-korrekturansicht-vorspann.tex
%% Vorspann für die Korrekturansicht.
%% Lädt die gemeinsame Datei latex-vorspann.tex mit gesetztem Schalter.

\newif\ifkorrekturansicht
\korrekturansichttrue

\input{../tex-inputs/latex-vorspann}


\renewcommand{\erwaehntePersonen}{Personen: Otto Brahm, Samuel Fischer, Hedwig Fischer, Emil Heilbut, Clara Jonas, Elisabeth Maas, Emanuel Reicher, Rudolf Rittner, Felix Salten, Ottilie Salten, Olga Schnitzler}
\renewcommand{\erwaehnteInstitutionen}{Institutionen: Lessing-Theater}
\renewcommand{\erwaehnteOrte}{Orte: Berlin, Deutsches Theater Berlin, Edmund-Weiß-Gasse 7, Hotel Savoy, Marienlyst, Noordwijk, Wien, XVIII., Währing}
\renewcommand{\erwaehnteWerke}{Werke: Der einsame Weg. Schauspiel in fünf Akten}
\section[ Felix Salten u. a. an Arthur Schnitzler, 19. 4. 1906]{Felix Salten u. a. an Arthur Schnitzler, 19. 4. 1906}
\nopagebreak\mylabel{v}
\rehead{ }\normalsize\beginnumbering\briefempfaengerindex{Schnitzler, Arthur@\textsc{Schnitzler, Arthur}!zzzMaas, Elisabeth@\emph{von Elisabeth Maas}!1906-04-191@{19. 4. 1906}|(be}\briefempfaengerindex{Schnitzler, Arthur@\textsc{Schnitzler, Arthur}!zzzFischer, Hedwig@\emph{von Hedwig Fischer}!1906-04-191@{19. 4. 1906}|(be}\briefempfaengerindex{Schnitzler, Arthur@\textsc{Schnitzler, Arthur}!zzzFischer, Samuel@\emph{von Samuel Fischer}!1906-04-191@{19. 4. 1906}|(be}\briefempfaengerindex{Schnitzler, Arthur@\textsc{Schnitzler, Arthur}!zzzHeilbut, Emil@\emph{von Emil Heilbut}!1906-04-191@{19. 4. 1906}|(be}\briefempfaengerindex{Schnitzler, Arthur@\textsc{Schnitzler, Arthur}!zzzJonas, Clara@\emph{von Clara Jonas}!1906-04-191@{19. 4. 1906}|(be}\briefempfaengerindex{Schnitzler, Arthur@\textsc{Schnitzler, Arthur}!zzzBrahm, Otto@\emph{von Otto Brahm}!1906-04-191@{19. 4. 1906}|(be}\briefempfaengerindex{Schnitzler, Arthur@\textsc{Schnitzler, Arthur}!zzzSalten, Ottilie@\emph{von Ottilie Salten}!1906-04-191@{19. 4. 1906}|(be}\briefempfaengerindex{Schnitzler, Arthur@\textsc{Schnitzler, Arthur}!zzzSalten, Felix@\emph{von Felix Salten}!1906-04-191@{19. 4. 1906}|(be}
\toendnotes[C]{\smallbreak\pagebreak[2]}\Standort{CUL, Schnitzler, B 89, B 1.}
\physDesc{Postkarte, 766 Zeichen
\newline{}Handschrift Felix Salten: schwarze Tinte, lateinische Kurrent
\newline{}Handschrift Ottilie Salten: schwarze Tinte
\newline{}Handschrift Otto Brahm: schwarze Tinte, lateinische Kurrent
\newline{}Handschrift Clara Jonas: schwarze Tinte, lateinische Kurrent
\newline{}Handschrift Emil Heilbut: schwarze Tinte, lateinische Kurrent
\newline{}Handschrift Samuel Fischer: schwarze Tinte, lateinische Kurrent
\newline{}Handschrift Hedwig Fischer: schwarze Tinte, deutsche Kurrent
\newline{}Handschrift Elisabeth Maas: Bleistift, lateinische Kurrent
\newline{}Versand: Stempel: »\nobreak{}\oindex{Berlin@\textbf{Berlin}, \emph{P.PPLC}|pwk}Berlin\textcolor{gray}{,}
                                          N\textcolor{gray}{.} W\textcolor{gray}{.} 7, 20. 4. 06, 5–6 V.\nobreak{}«.  
\newline{}Ordnung: mit Bleistift von unbekannter Hand nummeriert: »210« }\toendnotes[C]{\smallbreak}\pstart{}\textcolor{gray}{\textbf{\textcolor{pink}{SAVOY-HOTEL, BERLIN N. W.}{}\ledrightnote{\textcolor{pink}{Hotel Savoy}}}}\pend{}
{\bigskip}\pstart{}{\pb}Herrn D\textsuperscript{r} Arthur Schnitzler\pend{}\pstart{}\textcolor{pink}{Wien XVIII.}{}\ledrightnote{\textcolor{pink}{XVIII., Währing}}\pend{}\pstart{}\textcolor{pink}{Spöttelgasse 7}{}\ledrightnote{\textcolor{pink}{Edmund-Weiß-Gasse 7}}\pend{}
{\bigskip}
\pstart
           {\pb}Donnerstag{ }Abds. nach dem »\textcolor{green}{Einsamen
                  Weg}{}\ledrightnote{\textcolor{green}{Der einsame Weg. Schauspiel in fünf Akten}}«.\pend
           
\pstart
           Wir sind alle ziemlich kaput – aber auf eine edle Weise. (Es gibt kaum eine
               vornehmere Manier, den Leuten die Lebensfreude abzugewöhnen, als dieses schöne \textcolor{green}{Stück}{}\ledrightnote{{$\rightarrow$}\textcolor{green}{Der einsame Weg. Schauspiel in fünf Akten}})\pend
           \pstart Viele herzliche Grüße Ihnen u. \textcolor{blue}{Olga}{}\ledrightnote{\textcolor{blue}{Olga Schnitzler}}. Ihr
                  \spacefill\mbox{Salten}\pend{}
\pstart
           \noindent{}{[}hs. Ottilie Salten:{]} \spacefill\mbox{Otti}\pend
           
\pstart
           \noindent{}{[}hs. Brahm:{]} Trotz einer miserabeln \label{K_L03419-1v}\edtext{Aufführung}{\lemma{\textnormal{\emph{Aufführung}}}\Cendnote{\textnormal{Am
                     19. 4. 1906 wurde \emph{\textcolor{green}{Der einsame Weg}} vom \emph{\textcolor{brown}{Lessing-Theater}} in \textcolor{pink}{Berlin} als
                  Neuaufnahme gegeben. Hintergrund bildete das bevorstehende Gastspiel in \textcolor{pink}{Wien}, für das das \textcolor{green}{Stück} fix gesetzt war. Die Rolle von \textcolor{green}{Julian Fichtner} wurde aber
                  nicht mehr wie bei der Uraufführung von \textcolor{blue}{Rudolf
                     Rittner}, sondern von \textcolor{blue}{Emanuel
                     Reicher} gespielt. Das führte in den folgenden Wochen zu verschiedenen
                  (erfolglosen) Versuchen, \textcolor{blue}{Rittner} zur
                  Rückkehr zu bewegen, vgl. \emph{Briefwechsel
                        Schnitzler/Brahm}, 225–228, Felix Salten an Arthur Schnitzler, 21. 4. [1906], Felix Salten an Arthur Schnitzler, 22.–23. 4. 1906.}}}\label{K_L03419-1h} hat
               mir dieses \textcolor{green}{Werk}{}\ledrightnote{{$\rightarrow$}\textcolor{green}{Der einsame Weg. Schauspiel in fünf Akten}} wieder sehr
               gefallen. Herzlich \spacefill\mbox{OBrahm}\pend
           
\pstart
           \noindent{}{[}hs. Jonas:{]} Es war doch sehr schön + alles Uebrige werde ich Ihnen
                  \label{K_L03419-2v}\edtext{den Sommer in \textcolor{pink}{Nordwijk}{}\ledrightnote{\textcolor{pink}{Noordwijk}}}{\lemma{\textnormal{\emph{den Sommer in Nordwijk}}}\Cendnote{\textnormal{\textcolor{blue}{Schnitzler} plante zu dieser Zeit noch von \textcolor{pink}{Marienlyst} an den Strand von \textcolor{pink}{Noordwijk} zu übersiedeln. Dazu kam es nicht.}}}\label{K_L03419-2h} sagen. Herzlichste
               Grüße Ihnen + Ihrer lieben \textcolor{blue}{Frau}{}\ledrightnote{{$\rightarrow$}\textcolor{blue}{Olga Schnitzler}}. \spacefill\mbox{Clara Jonas}\pend
           
\pstart
           \noindent{}{[}hs. Heilbut:{]} Von Ihrem \textcolor{green}{Werk}{}\ledrightnote{{$\rightarrow$}\textcolor{green}{Der einsame Weg. Schauspiel in fünf Akten}} tiefergriffen grüsst Sie herzlich Ihr
                  \spacefill\mbox{Heilbut}\pend
           
\pstart
           \noindent{}{[}hs. Samuel Fischer:{]} Vielen Dank und herzlichen Gruß von Ihrem \spacefill\mbox{S.
                  Fischer.}\pend
           
\pstart
           \noindent{}{[}hs. Hedwig Fischer:{]} Der »\textcolor{green}{Einſame Weg}{}\ledrightnote{\textcolor{green}{Der einsame Weg. Schauspiel in fünf Akten}}{[}«{]} hat eine herrliche \label{K_L03419-3v}\edtext{Auferſtehung}{\lemma{\textnormal{\emph{Auferſtehung}}}\Cendnote{\textnormal{Das
                     \textcolor{green}{Stück} war bereits 1904 am \textcolor{pink}{Deutschen Theater
                     Berlin} uraufgeführt worden.}}}\label{K_L03419-3h} gefeiert u wir denken Ihrer in
               Dankbarkeit. Ihre \spacefill\mbox{Hedwig Fischer}\pend
           
\pstart
           \noindent{}{[}hs. Maas:{]} Herzlichen Gruss \spacefill\mbox{Lili Jonas.}\pend
           \endnumbering\briefempfaengerindex{Schnitzler, Arthur@\textsc{Schnitzler, Arthur}!zzzMaas, Elisabeth@\emph{von Elisabeth Maas}!1906-04-191@{19. 4. 1906}|)be}\briefempfaengerindex{Schnitzler, Arthur@\textsc{Schnitzler, Arthur}!zzzFischer, Hedwig@\emph{von Hedwig Fischer}!1906-04-191@{19. 4. 1906}|)be}\briefempfaengerindex{Schnitzler, Arthur@\textsc{Schnitzler, Arthur}!zzzFischer, Samuel@\emph{von Samuel Fischer}!1906-04-191@{19. 4. 1906}|)be}\briefempfaengerindex{Schnitzler, Arthur@\textsc{Schnitzler, Arthur}!zzzHeilbut, Emil@\emph{von Emil Heilbut}!1906-04-191@{19. 4. 1906}|)be}\briefempfaengerindex{Schnitzler, Arthur@\textsc{Schnitzler, Arthur}!zzzJonas, Clara@\emph{von Clara Jonas}!1906-04-191@{19. 4. 1906}|)be}\briefempfaengerindex{Schnitzler, Arthur@\textsc{Schnitzler, Arthur}!zzzBrahm, Otto@\emph{von Otto Brahm}!1906-04-191@{19. 4. 1906}|)be}\briefempfaengerindex{Schnitzler, Arthur@\textsc{Schnitzler, Arthur}!zzzSalten, Ottilie@\emph{von Ottilie Salten}!1906-04-191@{19. 4. 1906}|)be}\briefempfaengerindex{Schnitzler, Arthur@\textsc{Schnitzler, Arthur}!zzzSalten, Felix@\emph{von Felix Salten}!1906-04-191@{19. 4. 1906}|)be}\mylabel{h}  \normalsize

\doendnotes{C}
\bigskip
\vfill

\clearpage

\footnotesize

\lohead{\textsc{register}}

% Definiere theindex-Environment komplett neu ohne reledmac
\makeatletter
\renewenvironment{theindex}{%
  \section*{\indexname}%
  \setlength{\parindent}{0pt}%
  \setlength{\parskip}{0pt plus 0.3pt}%
  \let\item\@idxitem
}{%
  \clearpage
}
\makeatother

\IfFileExists{\jobname-pw.ind}{\input{\jobname-pw.ind}}{}

\end{document}

      