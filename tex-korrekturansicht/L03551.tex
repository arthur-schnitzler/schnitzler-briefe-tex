%% latex-korrekturansicht-vorspann.tex
%% Vorspann für die Korrekturansicht.
%% Lädt die gemeinsame Datei latex-vorspann.tex mit gesetztem Schalter.

\newif\ifkorrekturansicht
\korrekturansichttrue

\input{../tex-inputs/latex-vorspann}


\renewcommand{\erwaehntePersonen}{Personen: Alfred von Berger, Samuel Fischer, Hedwig Fischer, Josef Kainz, Gustav Mahler, Felix Salten, Ottilie Salten, Olga Schnitzler, Elisabeth Steinrück}
\renewcommand{\erwaehnteInstitutionen}{Institutionen: Burgtheater}
\renewcommand{\erwaehnteOrte}{Orte: Baden bei Wien, Berghof, München, Neue Musik-Festhalle, Südtirol, Unterach am Attersee, Wien}
\renewcommand{\erwaehnteWerke}{Werke: 8. Sinfonie, Artur Schnitzler im Hofburgtheater, Das weite Land. Tragikomödie in fünf Akten, Der junge Medardus. Dramatische Historie in einem Vorspiel und fünf Aufzügen, Neue Freie Presse, Sappho. Trauerspiel in fünf Aufzügen}
\section[ Felix Salten an Arthur Schnitzler, 17. 8. 1910]{Felix Salten an Arthur Schnitzler, 17. 8. 1910}
\nopagebreak\mylabel{v}
\rehead{ }\normalsize\beginnumbering\briefempfaengerindex{Schnitzler, Arthur@\textsc{Schnitzler, Arthur}!zzzSalten, Felix@\emph{von Felix Salten}!1910-08-171@{17. 8. 1910}|(be}
\toendnotes[C]{\smallbreak\pagebreak[2]}\Standort{CUL, Schnitzler, B 89, B 2.}
\physDesc{Brief, 1 Blatt, 1 Seite, 563 Zeichen
\newline{}Handschrift: schwarze Tinte, lateinische Kurrent
\newline{}Schnitzler: mit Bleistift Vermerk: »\textsc{Salte{[}n{]}}« 
\newline{}Ordnung: mit Bleistift von unbekannter Hand nummeriert: »266« }\toendnotes[C]{\smallbreak}
\pstart
           \noindent{}\raggedleft{}{\pb}\textcolor{pink}{Unterach}{}\ledrightnote{\textcolor{pink}{Unterach am Attersee}}, \textcolor{pink}{Berghof}{}\ledrightnote{\textcolor{pink}{Berghof}}.\pend
           
\pstart
           \raggedleft{}17. VIII. 10\pend
           
\pstart{}Lieber,\pend
\pstart
           wir bleiben, denk’ ich, bis gegen den 10. September{ }\textcolor{pink}{hier}{}\ledrightnote{{$\rightarrow$}\textcolor{pink}{Unterach am Attersee}}\textcolor{gray}{,} und \textcolor{blue}{Fischers}{}\ledrightnote{\textcolor{blue}{Samuel Fischer}{\newline}\textcolor{blue}{Hedwig Fischer}},
               die zur \label{K_L03551-1v}\edtext{\textcolor{green}{Mahler-Symphonie}{}\ledrightnote{{$\rightarrow$}\textcolor{green}{8. Sinfonie}} nach \textcolor{pink}{München}{}\ledrightnote{\textcolor{pink}{München}}}{\lemma{\textnormal{\emph{Mahler-Symphonie nach München}}}\Cendnote{\textnormal{Am 12. 9. 1910 fand in der \textcolor{pink}{Neuen
                     Musik-Halle} die Uraufführung der \emph{\textcolor{green}{8.
                     Sinfonie}} unter der Leitung \textcolor{blue}{Gustav
                     Mahler}s statt.}}}\label{K_L03551-1h} wollen, werden wol auch so lange da sein. Wenn wir
               Aussicht hätten, Sie \textcolor{blue}{Beide}{}\ledrightnote{{$\rightarrow$}\textcolor{blue}{Olga Schnitzler}}
               hier \label{K_L03551-2v}\edtext{auf dem \textcolor{pink}{Berghof}{}\ledrightnote{\textcolor{pink}{Berghof}} zu begrüßen}{\lemma{\textnormal{\emph{auf … begrüßen}}}\Cendnote{\textnormal{Zu \textcolor{blue}{Schnitzler}s
                     Verhältnis zum \textcolor{pink}{Berghof}{ }siehe Felix Salten an Arthur Schnitzler, [25.? 8. 1892].}}}\label{K_L03551-2h}, würden wir uns herzlich freuen. Wann glauben Sie, dass
               Sie hierher kommen könnten? In der Zeitung lese ich, dass Sie mit dem \label{K_L03551-3v}\edtext{\textcolor{brown}{Burgtheater}{}\ledrightnote{\textcolor{brown}{Burgtheater}} einig}{\lemma{\textnormal{\emph{Burgtheater einig}}}\Cendnote{\textnormal{Am 14. 8. 1910 schrieb die \emph{\textcolor{green}{Neue Freie Presse}}: »[\textcolor{blue}{\so{Artur Schnitzler}}\so{{ }im{ }}\textcolor{brown}{\so{Hofburgtheater}}] In der kommenden Saiſon des \textcolor{brown}{Hofburgtheaters}, welches am 1.{ }September mit ›\textcolor{green}{Sappho}‹
                     eröffnet wird, werden zwei neue Werke \textcolor{blue}{Artur
                        Schnitzler} zur Aufführung gelangen. Als zweite Novität des \textcolor{brown}{Burgtheaters} geht ›\textcolor{green}{\textsc{Der junge Herr Medardus}}‹ in Szene. {[}\ldots{]} Außer
                     dieſem Werke hat Direktor \textcolor{blue}{Alfred Freiherr v.
                           \so{Berger}} auch \textcolor{blue}{Schnitzler}s Schauſpiel ›\textcolor{green}{\so{Das weite Land}}‹, das zum Teil in \textcolor{pink}{Baden bei Wien}, zum
                     Teil in \textcolor{pink}{Tirol} ſpielt, zur Aufführung
                     angenommen. Die männliche Hauptrolle wird Herr \textcolor{blue}{\so{Kainz}} ſpielen.« ([O. V.]: \emph{\textcolor{green}{Artur
                        Schnitzler im Hofburgtheater}}. In: \emph{\textcolor{green}{Neue
                        Freie Presse}}, Nr. 16.515, 14. 8. 1910, Morgenblatt,
                     S. 15.)}}}\label{K_L03551-3h} sind, was mich sehr freut. Was ist »\textcolor{green}{das weite Land}{}\ledrightnote{\textcolor{green}{Das weite Land. Tragikomödie in fünf Akten}}«{\dots}?\pend
           
\pstart
           Viele Grüße von \textcolor{blue}{uns}{}\ledrightnote{{$\rightarrow$}\textcolor{blue}{Ottilie Salten}} zu
               Ihnen, und die Bitte, uns \uline{bald} Nachricht zu geben,
                  \label{K_L03551-4v}\edtext{wie es Ihrer \textcolor{blue}{Schwägerin}{}\ledrightnote{{$\rightarrow$}\textcolor{blue}{Elisabeth Steinrück}} geht}{\lemma{\textnormal{\emph{wie … geht}}}\Cendnote{\textnormal{vgl. Arthur Schnitzler an Felix Salten, 8. 8. 191[0]}}}\label{K_L03551-4h}! Herzlichst {\\[\baselineskip]}Ihr {\\[\baselineskip]}\spacefill\mbox{Felix Salten}\pend
           \leftskip=0em{}\endnumbering\briefempfaengerindex{Schnitzler, Arthur@\textsc{Schnitzler, Arthur}!zzzSalten, Felix@\emph{von Felix Salten}!1910-08-171@{17. 8. 1910}|)be}\mylabel{h}  \normalsize

\doendnotes{C}
\bigskip
\vfill

\clearpage

\footnotesize

\lohead{\textsc{register}}

% Definiere theindex-Environment komplett neu ohne reledmac
\makeatletter
\renewenvironment{theindex}{%
  \section*{\indexname}%
  \setlength{\parindent}{0pt}%
  \setlength{\parskip}{0pt plus 0.3pt}%
  \let\item\@idxitem
}{%
  \clearpage
}
\makeatother

\IfFileExists{\jobname-pw.ind}{\input{\jobname-pw.ind}}{}

\end{document}

      