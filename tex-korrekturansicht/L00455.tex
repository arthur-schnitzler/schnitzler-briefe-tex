%% latex-korrekturansicht-vorspann.tex
%% Vorspann für die Korrekturansicht.
%% Lädt die gemeinsame Datei latex-vorspann.tex mit gesetztem Schalter.

\newif\ifkorrekturansicht
\korrekturansichttrue

\input{../tex-inputs/latex-vorspann}


               \section[Hermann Bahr an Arthur Schnitzler, {[}19. 6. 1895{]}]{ Hermann Bahr an Arthur Schnitzler, {[}19. 6. 1895{]}}\nopagebreak\mylabel{v}\rehead{ }\normalsize\beginnumbering\briefempfaengerindex{Schnitzler, Arthur@\textsc{Schnitzler, Arthur}!zzzBahr, Hermann@\emph{von Hermann Bahr}!1895-06-191@{{[}19. 6. 1895{]}}|(be} \toendnotes[C]{\smallbreak\pagebreak[2]} \Standort{CUL, Schnitzler, B 5b.}
\physDesc{Brief, 1 Blatt, 2 Seiten
\newline{}Handschrift: schwarze Tinte, deutsche Kurrent
\newline{}Schnitzler: mit Bleistift datiert: »19/6 {[}189{]}5« \newline{}Ordnung: 1) mit rotem Buntstift von unbekannter Hand nummeriert:
                                    »29« 2) mit Bleistift von unbekannter Hand nummeriert:
                                    »29«}\buchAbdrucke{\weitereDrucke{Hermann Bahr, Arthur Schnitzler: \emph{Briefwechsel, Aufzeichnungen, Dokumente (1891–1931)}. Hg. Kurt Ifkovits und Martin Anton Müller. Göttingen: \emph{Wallstein} 2018, S. 102.} }\toendnotes[C]{\smallbreak}\pstart
           \noindent{}{\pb}\textcolor{gray}{\textbf{»\textcolor{brown}{Die Zeit}{}\ledrightnote{\textcolor{brown}{Die Zeit. Wiener Wochenschrift}}«}}\hfill \textcolor{gray}{\textbf{\textbf{\textcolor{pink}{Wien}{}\ledrightnote{\textcolor{pink}{Wien}}}, den ..........189{\dotstwo}}}\pend
           \pstart
           \textcolor{gray}{\textbf{Wiener Wochenſchrift}}\hfill \textcolor{gray}{\textbf{\textcolor{pink}{IX/3, Günthergaſſe 1}{}\ledrightnote{\textcolor{pink}{Günthergasse}}.}}\pend
           \pstart
           \textcolor{gray}{\textbf{\textbf{Herausgeber}:}}{\\}\textcolor{gray}{\textbf{Profeſſor Dr. \textcolor{blue}{I. Singer}{}\ledrightnote{\textcolor{blue}{Isidor Singer}},
                        \textcolor{blue}{Hermann Bahr}{}\ledrightnote{\textcolor{blue}{Hermann Bahr}}, Dr. \textcolor{blue}{Heinrich Kanner}{}\ledrightnote{\textcolor{blue}{Heinrich Kanner}}.}}\pend
           \pstart
           \textcolor{gray}{\textbf{Telephon Nr. 6415.}}\pend
           \pstart{}Lieber Arthur!\pend\pstart
           Ich möchte ſehr, ſehr gern etwas von Dir für die »\textcolor{brown}{Zeit}{}\ledrightnote{\textcolor{brown}{Die Zeit. Wiener Wochenschrift}}« haben. Lieber wäre mir eine kurze Geſchichte, nicht über 8 Spalten des
               Blattes. \textsc{Faute de mieux}, nehme ich auch eine lange, obwohl
               ich an \label{K_L00455_1v}\edtext{\textcolor{green}{\textcolor{blue}{\textsc{d’Annunzio}}{}\ledrightnote{\textcolor{blue}{Gabriele D’Annunzio}}}{}\ledrightnote{→\textcolor{green}{Giovanni Episcopo}}}{\lemma{\textnormal{\emph{d’Annunzio}}}\Cendnote{\textnormal{\textcolor{blue}{Gabriele d’Annunzio}: \emph{\textcolor{green}{Giovanni Episcopo}}. In: \emph{\textcolor{green}{Die
                        Zeit}}, Bd. 1, Nr. 9, 1. 12. 1894 – Bd. 2, Nr. 16,
                        19. 1. 1895 (8 Teile).}}}\label{K_L00455_1h} erfahren habe, daß das
               Zerreißen in Fortſetzungen auch die stärksten Sachen umbringt.\pend
           \pstart
           Deine \textcolor{green}{Novelle}{}\ledrightnote{→\textcolor{green}{Später Ruhm}} könnte im Oktober
                  er{\pb}ſcheinen.\pend
           \pstart
           \label{K_L00455_2v}\edtext{Ich fahre heute Abend}{\lemma{\textnormal{\emph{Ich fahre heute Abend}}}\Cendnote{\textnormal{Vom 19. 6. bis zum
                     12. 7. 1895 machte \textcolor{blue}{Bahr}{ }Sommerurlaub. Er besuchte drei Tage \textcolor{pink}{München}, dann \textcolor{pink}{Schliersee} und den Starnberger See sowie
                     \textcolor{pink}{Innsbruck} und die Gegend von \textcolor{pink}{Kufstein}.}}}\label{K_L00455_2h} nach \textcolor{pink}{München}{}\ledrightnote{\textcolor{pink}{München}} und dann auf drei Wochen ins bairiſche Gebirg.\pend
           \pstart
           Herzlichst{\\[\baselineskip]}Dein{\\[\baselineskip]}\spacefill\mbox{Hermann}\pend
           \leftskip=0em{}\pstart
           \textcolor{gray}{\textbf{\label{T_L00455_1v}\edtext{Alle für »\textcolor{brown}{Die Zeit}{}\ledrightnote{\textcolor{brown}{Die Zeit. Wiener Wochenschrift}}« beſtimmten Zuſchriften und Sendungen ſind an die
                  Redaction der »\textcolor{brown}{Zeit}{}\ledrightnote{\textcolor{brown}{Die Zeit. Wiener Wochenschrift}}« und \textbf{nicht} an die Perſon eines der Herausgeber zu richten.}{\lemma{\textnormal{\emph{Alle … richten.}}}\Cendnote{\textnormal{am unteren Rand der ersten Seite}}}\label{T_L00455_1h}}}\pend
           \endnumbering\briefempfaengerindex{Schnitzler, Arthur@\textsc{Schnitzler, Arthur}!zzzBahr, Hermann@\emph{von Hermann Bahr}!1895-06-191@{{[}19. 6. 1895{]}}|)be}\mylabel{h}  \normalsize

\doendnotes{C}
\bigskip
\vfill

\clearpage

\footnotesize

\lohead{\textsc{register}}

% Definiere theindex-Environment komplett neu ohne reledmac
\makeatletter
\renewenvironment{theindex}{%
  \section*{\indexname}%
  \setlength{\parindent}{0pt}%
  \setlength{\parskip}{0pt plus 0.3pt}%
  \let\item\@idxitem
}{%
  \clearpage
}
\makeatother

\IfFileExists{\jobname-pw.ind}{\input{\jobname-pw.ind}}{}

\end{document}

      