%% latex-korrekturansicht-vorspann.tex
%% Vorspann für die Korrekturansicht.
%% Lädt die gemeinsame Datei latex-vorspann.tex mit gesetztem Schalter.

\newif\ifkorrekturansicht
\korrekturansichttrue

\input{../tex-inputs/latex-vorspann}


\renewcommand{\erwaehntePersonen}{Personen: Sigmund Lautenburg, Peter Rotenstern, Felix Salten, Felix Speidel}
\renewcommand{\erwaehnteInstitutionen}{Institutionen: Raimund-Theater}
\renewcommand{\erwaehnteOrte}{Orte: Paris, Russland, Wien}
\renewcommand{\erwaehnteWerke}{}
\section[ Arthur Schnitzler an Felix Salten, 18. 4. 1907]{Arthur Schnitzler an Felix Salten, 18. 4. 1907}
\nopagebreak\mylabel{v}
\rehead{ }\normalsize\beginnumbering\briefempfaengerindex{Salten, Felix@\textsc{Salten, Felix}!zzzSchnitzler, Arthur@\emph{von Arthur Schnitzler}!1907-04-181@{18. 4. 1907}|(be}
\toendnotes[C]{\smallbreak\pagebreak[2]}\Standort{Wienbibliothek im Rathaus, ZPH 1681, 2.1.516.}
\physDesc{Brief, 1 Blatt, 1 Seite, 350 Zeichen
\newline{}Handschrift: schwarze Tinte, deutsche Kurrent
\newline{}Ordnung: mit Bleistift von unbekannter Hand nummeriert: »12« }\toendnotes[C]{\smallbreak}
\pstart
           \raggedleft{}{\pb}18. 4. 907\pend
           
\pstart{}lieber,\pend
\pstart
           Herr \textsc{\textcolor{blue}{Rotenstern}{}\ledrightnote{\textcolor{blue}{Peter Rotenstern}}}, mein \textcolor{pink}{ruſſ.}{}\ledrightnote{\textcolor{pink}{Russland}} Überſetzer iſt jetzt in \textsc{\textcolor{pink}{Paris}{}\ledrightnote{\textcolor{pink}{Paris}}} und möchte gern »Vertreter« \textsc{\textcolor{blue}{Lautenburg}{}\ledrightnote{\textcolor{blue}{Sigmund Lautenburg}}}s, \textsc{resp.} des \textcolor{brown}{Raimundtheater}{}\ledrightnote{\textcolor{brown}{Raimund-Theater}}s dort ſein. We{\geminationn} es Ihnen bei
               Gelegenheit möglich \strikeout{iſt} und nicht aus irgd einem
               Grund unangenehm iſt, könnten Sie zu \textcolor{blue}{L.}{}\ledrightnote{\textcolor{blue}{Sigmund Lautenburg}} ein
               Wort in dieſem Sinne äußern?\pend
           \pstart Herzlichſt mit Grüßen von Haus zu Haus Ihr \spacefill\mbox{Arthur.}\pend{}
\pstart
           \noindent{}Vielleicht \label{K_L03008-1v}\edtext{morgen{ }\textsc{Tennis}}{\lemma{\textnormal{\emph{morgen Tennis}}}\Cendnote{\textnormal{Am 19. 4. 1907 spielte \textcolor{blue}{Schnitzler} letztlich mit \textcolor{blue}{Felix Speidel}.}}}\label{K_L03008-1h}?\pend
           \endnumbering\briefempfaengerindex{Salten, Felix@\textsc{Salten, Felix}!zzzSchnitzler, Arthur@\emph{von Arthur Schnitzler}!1907-04-181@{18. 4. 1907}|)be}\mylabel{h}  \normalsize

\doendnotes{C}
\bigskip
\vfill

\clearpage

\footnotesize

\lohead{\textsc{register}}

% Definiere theindex-Environment komplett neu ohne reledmac
\makeatletter
\renewenvironment{theindex}{%
  \section*{\indexname}%
  \setlength{\parindent}{0pt}%
  \setlength{\parskip}{0pt plus 0.3pt}%
  \let\item\@idxitem
}{%
  \clearpage
}
\makeatother

\IfFileExists{\jobname-pw.ind}{\input{\jobname-pw.ind}}{}

\end{document}

      