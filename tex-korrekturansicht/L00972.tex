%% latex-korrekturansicht-vorspann.tex
%% Vorspann für die Korrekturansicht.
%% Lädt die gemeinsame Datei latex-vorspann.tex mit gesetztem Schalter.

\newif\ifkorrekturansicht
\korrekturansichttrue

\input{../tex-inputs/latex-vorspann}


               \section[Arthur Schnitzler an Richard Beer-Hofmann, 10. 9. 1899]{ Arthur Schnitzler an Richard Beer-Hofmann, 10. 9. 1899}\nopagebreak\mylabel{v}\rehead{ }\normalsize\beginnumbering\briefempfaengerindex{Beer-Hofmann, Richard@\textsc{Beer-Hofmann, Richard}!zzzSchnitzler, Arthur@\emph{von Arthur Schnitzler}!1899-09-101@{10. 9. 1899}|(be} \toendnotes[C]{\smallbreak\pagebreak[2]} \Standort{YCGL, MSS 31.}
\physDesc{Postkarte
\newline{}Handschrift: Bleistift, deutsche Kurrent\newline{}Versand: 1) Stempel: »\nobreak{}\oindex{Bad Ischl@\textbf{Bad Ischl}, \emph{Besiedelter Ort (A.BSO)}|pwk}Ischl, 10. 9. 99, 7–8 V\nobreak{}«.  2) Stempel: »\nobreak{}\oindex{Brixen@\textbf{Brixen}, \emph{http://www.geonames.org/ontologyP.PPLA3}|pwk}Brixen, 11. 9. 99, 6.V\nobreak{}«. }\pstart{}{\pb}\textsc{Herrn Dr.
                  Richard Beer-Hofmann}\pend{}\pstart{}\textcolor{pink}{\textsc{Vahrn}}{}\ledrightnote{\textcolor{pink}{Vahrn}} bei \textsc{\textcolor{pink}{Brixen}{}\ledrightnote{\textcolor{pink}{Brixen}}}\pend{}\pstart{}\textcolor{pink}{\textsc{Tirol}}{}\ledrightnote{\textcolor{pink}{Tirol}}\pend{}{\bigskip}\pstart
           \noindent{}{\pb}lieber Richard, eben iſt ein
               Brief an Sie nach \textcolor{pink}{\textsc{Sachsenburg}}{}\ledrightnote{\textcolor{pink}{Sachsenburg}} abgegangen. Er enthält nichts wichtiges; nur d\substVorne{}\textsuperscript{\textcolor{gray}{en Umſtand}}{\allowbreak}\substDazwischen{}ie Bitte\substHinten{}, Sie möchten mir nach \textcolor{pink}{München}{}\ledrightnote{\textcolor{pink}{München}}{ }ſchreiben,
               wo ich Mittwoch u Donnerstag{ }ſein will.\pend
           \pstart Herzlich Ihr \spacefill\mbox{A.}\pend{}\endnumbering\briefempfaengerindex{Beer-Hofmann, Richard@\textsc{Beer-Hofmann, Richard}!zzzSchnitzler, Arthur@\emph{von Arthur Schnitzler}!1899-09-101@{10. 9. 1899}|)be}\mylabel{h}  \normalsize

\doendnotes{C}
\bigskip
\vfill

\clearpage

\footnotesize

\lohead{\textsc{register}}

% Definiere theindex-Environment komplett neu ohne reledmac
\makeatletter
\renewenvironment{theindex}{%
  \section*{\indexname}%
  \setlength{\parindent}{0pt}%
  \setlength{\parskip}{0pt plus 0.3pt}%
  \let\item\@idxitem
}{%
  \clearpage
}
\makeatother

\IfFileExists{\jobname-pw.ind}{\input{\jobname-pw.ind}}{}

\end{document}

      