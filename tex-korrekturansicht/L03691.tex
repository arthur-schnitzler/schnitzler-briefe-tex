%% latex-korrekturansicht-vorspann.tex
%% Vorspann für die Korrekturansicht.
%% Lädt die gemeinsame Datei latex-vorspann.tex mit gesetztem Schalter.

\newif\ifkorrekturansicht
\korrekturansichttrue

\input{../tex-inputs/latex-vorspann}


\section[Stefan Zweig an Arthur Schnitzler, 2. 11. 1929]{L03691 Stefan Zweig an Arthur Schnitzler, 2. 11. 1929}
\nopagebreak\mylabel{L03691v}
\rehead{ }\normalsize\beginnumbering\briefempfaengerindex{Schnitzler, Arthur@\textsc{Schnitzler, Arthur}!zzzZweig, Stefan@\emph{von Stefan Zweig}!1929-11-021@{2. 11. 1929}|(be}
\toendnotes[C]{\smallbreak\pagebreak[2]}
\correspDesc{Versand  durch Stefan Zweig am 2. 11. 1929 in Salzburg
\newline{}Erhalt  durch Arthur Schnitzler im Zeitraum [3. 11. 1929
                  – 5. 11. 1929?] in Wien}\toendnotes[C]{\smallbreak}
\Standort{CUL, Schnitzler, B 118.}
\physDesc{Brief, 1 Blatt, 1 Seite, 1101 Zeichen
\newline{}Schreibmaschine
\newline{}Handschrift: blaue Tinte, lateinische Kurrent (\noindent{}Korrekturen, Unterschrift)
\newline{}Schnitzler: 1) mit rotem Buntstift beschriftet: »\textsc{Spanien}«  2) mit rotem Buntstift fünf Unterstreichungen}
\buchAbdrucke{\weitereDrucke{Stefan Zweig: \emph{Briefwechsel mit Hermann Bahr, Sigmund Freud, Rainer Maria
                        Rilke und Arthur Schnitzler}. Herausgegeben von Jeffrey B. Berlin, Hans-Ulrich Lindken und Donald A. Prater. Frankfurt am Main: \emph{S. Fischer} 1987, S. 446.} }\toendnotes[C]{\smallbreak}
\pstart
           {\pb}\textcolor{gray}{\textbf{SZ}}\hfill \textcolor{gray}{\textbf{\textcolor{pink}{SALZBURG}\oindex{Salzburg@\textbf{Salzburg}, \emph{Verwaltungsgebiet}|pw}{}\ledrightnote{\textcolor{pink}{Salzburg}}}}\pend
           
\pstart
           \raggedleft{}\textcolor{gray}{\textbf{\textcolor{pink}{KAPUZINERBERG 5}\oindex{Paschinger Schlössl@\textbf{Paschinger Schlössl}, \emph{Wohngebäude}|pw}{}\ledrightnote{\textcolor{pink}{Paschinger Schlössl}}}}\pend
           
\pstart
           \raggedleft{}2. November 1929.\pend
           
\pstart{}Lieber, verehrter Herr Doktor!\pend\vspace{0.5em}
\pstart
           Ich nütze jede Gelegenheit gern, mich an Sie zu wenden und die vorliegende ist ein
               Brief von Herrn \label{K_L03691-1v}\edtext{\textcolor{blue}{A. del Vayo}\pwindex{Álvarez del Vayo, Julio 9.\,2.\,1891 Villaviciosa de Odón – 3.\,5.\,1975 Genf@\textsc{Álvarez del Vayo, Julio} (9.\,2.\,1891 Villaviciosa de Odón – 3.\,5.\,1975 Genf), \emph{Schriftsteller, Politiker, Journalist}|pwu}{}\ledrightnote{\textcolor{blue}{Julio Álvarez del Vayo}}, (dem Leiter des
               Verlags \textcolor{brown}{Editorial Espana}\orgindex{Espasa-Calpe@Espasa-Calpe|pwu}{}\ledrightnote{\textcolor{brown}{Espasa-Calpe}}}{\lemma{\textnormal{\emph{A. … Espana}}}\Cendnote{\textnormal{Vermutlich ist Espana ein Tippfehler und
                  es geht um eine Anfrage des Verlags \emph{\textcolor{brown}{Espasa}\orgindex{Espasa-Calpe@Espasa-Calpe|pwk}}, bei dem \textcolor{blue}{Stefan Zweig}\pwindex{Zweig, Stefan 28.\,11.\,1881 Wien – 23.\,2.\,1942 Petrópolis@\textsc{Zweig, Stefan} (28.\,11.\,1881 Wien – 23.\,2.\,1942 Petrópolis), \emph{Schriftsteller}|pwk}
                  selbst im Jahr darauf ein Buch publizierte (\textcolor{blue}{Stefan Zweig}\pwindex{Zweig, Stefan 28.\,11.\,1881 Wien – 23.\,2.\,1942 Petrópolis@\textsc{Zweig, Stefan} (28.\,11.\,1881 Wien – 23.\,2.\,1942 Petrópolis), \emph{Schriftsteller}|pwk}: \emph{\textcolor{green}{Fouché. Retrato di un Político}\pwindex{Zweig, Stefan 28.\,11.\,1881 Wien – 23.\,2.\,1942 Petrópolis@\textsc{Zweig, Stefan} (28.\,11.\,1881 Wien – 23.\,2.\,1942 Petrópolis), \emph{Schriftsteller}!Fouché. Retrato di un Político@\strich\emph{Fouché. Retrato di un Político}|pwk}}. \textcolor{pink}{Madrid}\oindex{Madrid@\textbf{Madrid}, \emph{Hauptstadt}|pwk}: \emph{\textcolor{brown}{Espasa-Calpe}\orgindex{Espasa-Calpe@Espasa-Calpe|pwk}}{ }1930). Bei dem Verleger handelt es sich wohl um den Schriftsteller \textcolor{blue}{Julio Alvares del Vayo}\pwindex{Álvarez del Vayo, Julio 9.\,2.\,1891 Villaviciosa de Odón – 3.\,5.\,1975 Genf@\textsc{Álvarez del Vayo, Julio} (9.\,2.\,1891 Villaviciosa de Odón – 3.\,5.\,1975 Genf), \emph{Schriftsteller, Politiker, Journalist}|pwk}, der schon einmal
                  nach Übersetzungsrechten für das \textcolor{pink}{Spanische}\oindex{Spanien@\textbf{Spanien}|pwk} angefragt hatte, wie aus zwei Briefen
                     \textcolor{blue}{Schnitzlers} an ihn aus dem Jahr 1923
                  hervorgeht (\emph{DLA}: HS.1985.1.02118,1-2).}}}\label{K_L03691-1}, \textcolor{pink}{Madrid}\oindex{Madrid@\textbf{Madrid}, \emph{Hauptstadt}|pw}{}\ledrightnote{\textcolor{pink}{Madrid}}, \textcolor{pink}{Palacio de la Prensa}\oindex{Palacio de la Prensa@\textbf{Palacio de la Prensa}, \emph{Bürogebäude}|pw}{}\ledrightnote{\textcolor{pink}{Palacio de la Prensa}}, \textcolor{pink}{Plaza del Callao 4}\oindex{Plaza del Callao@\textbf{Plaza del Callao}, \emph{Platz}|pw}{}\ledrightnote{\textcolor{pink}{Plaza del Callao}}), der sich bei mir beklagt, dass er an \textcolor{blue}{Fischer}\pwindex{Fischer, Samuel 24.\,12.\,1859 Liptovský Mikuláš – 15.\,10.\,1934 Berlin@\textsc{Fischer, Samuel} (24.\,12.\,1859 Liptovský Mikuláš – 15.\,10.\,1934 Berlin), \emph{Verleger}|pw}{}\ledrightnote{\textcolor{blue}{Samuel Fischer}} wegen \introOben{}des\introOben{} Uebersetzung\introOben{}srechts\introOben{} ihrer »\textcolor{green}{Therese}\pwindex{Schnitzler, Arthur 15. 5. 1862 Wien – 21. 10. 1931 ebd.@\textsc{Schnitzler, Arthur} (15. 5. 1862 Wien – 21. 10. 1931 ebd.), \emph{Schriftsteller, Mediziner}!Therese. Chronik eines Frauenlebens@\strich\emph{Therese. Chronik eines Frauenlebens}|pw}{}\ledrightnote{\textcolor{green}{Therese. Chronik eines Frauenlebens}}«
               geschrieben habe, ohne aber eine Antwort zu erhalten. Er lässt Sie nun durch mich
               bitten, erstlich, dass Sie dort nachfragen mögen, zweitens, ob Sie ihm bald etwas
               Neues von sich in Aussicht stellen könnten. Ich kenne ihn persönlich und die
               geschäftlichen Beziehungen zu dem \textcolor{brown}{Verlage}\orgindex{Espasa-Calpe@Espasa-Calpe|pwv}{}\ledrightnote{{$\rightarrow$}\emph{\textcolor{brown}{Espasa-Calpe}}} sind durchaus angenehm und korrekt.\pend
           
\pstart
           Noch in den nächsten Tagen grüsst Sie ein kleines \textcolor{green}{Buch}\pwindex{Zweig, Stefan 28.\,11.\,1881 Wien – 23.\,2.\,1942 Petrópolis@\textsc{Zweig, Stefan} (28.\,11.\,1881 Wien – 23.\,2.\,1942 Petrópolis), \emph{Schriftsteller}!Kleine Chronik@\strich\emph{Kleine Chronik}|pwv}{}\ledrightnote{{$\rightarrow$}\emph{\textcolor{green}{Kleine Chronik}}}{ }\textcolor{green}{Erzählungen}\pwindex{Zweig, Stefan 28.\,11.\,1881 Wien – 23.\,2.\,1942 Petrópolis@\textsc{Zweig, Stefan} (28.\,11.\,1881 Wien – 23.\,2.\,1942 Petrópolis), \emph{Schriftsteller}!Episode vom Genfer See@\strich\emph{Episode vom Genfer See}|pwv}\pwindex{Zweig, Stefan 28.\,11.\,1881 Wien – 23.\,2.\,1942 Petrópolis@\textsc{Zweig, Stefan} (28.\,11.\,1881 Wien – 23.\,2.\,1942 Petrópolis), \emph{Schriftsteller}!Leporella@\strich\emph{Leporella}|pwv}\pwindex{Zweig, Stefan 28.\,11.\,1881 Wien – 23.\,2.\,1942 Petrópolis@\textsc{Zweig, Stefan} (28.\,11.\,1881 Wien – 23.\,2.\,1942 Petrópolis), \emph{Schriftsteller}!unsichtbare Sammlung@\strich\emph{Die unsichtbare Sammlung}|pwv}\pwindex{Zweig, Stefan 28.\,11.\,1881 Wien – 23.\,2.\,1942 Petrópolis@\textsc{Zweig, Stefan} (28.\,11.\,1881 Wien – 23.\,2.\,1942 Petrópolis), \emph{Schriftsteller}!Buchmendel@\strich\emph{Buchmendel}|pwv}{}\ledrightnote{{$\rightarrow$}\emph{\textcolor{green}{Episode vom Genfer See}}{\newline}{$\rightarrow$}\emph{\textcolor{green}{Leporella}}{\newline}{$\rightarrow$}\emph{\textcolor{green}{Die unsichtbare Sammlung}}{\newline}{$\rightarrow$}\emph{\textcolor{green}{Buchmendel}}} von mir und hoffentlich habe ich endlich Gelegenheit, bei Ihnen
               vorzusprechen. Mein letzter Aufenthalt in \textcolor{pink}{Wien}\oindex{Wien@\textbf{Wien}, \emph{Verwaltungsgebiet}|pw}{}\ledrightnote{\textcolor{pink}{Wien}} war
               furchtbar überhetzt und als ich endlich bei \textcolor{blue}{Berta
                  Zuckerkandl}\pwindex{Zuckerkandl, Berta 13.\,4.\,1864 Wien – 16.\,10.\,1945 Paris@\textsc{Zuckerkandl, Berta} (13.\,4.\,1864 Wien – 16.\,10.\,1945 Paris), \emph{Schriftstellerin, Journalistin, Übersetzerin}|pw}{}\ledrightnote{\textcolor{blue}{Berta Zuckerkandl}} Ihre geheime Telefon-Nummer auskundschaftete und Sie anrief,
               meldete sich an jenem Sonntag Nachmittag niemand bei Ihnen\pend
           
\pstart
           In alter Herzlichkeit und Verehrung ergeben {\\[\baselineskip]}Ihr{\\[\baselineskip]}\spacefill\mbox{{[}hs.:{]} Stefan Zweig}\pend
           \leftskip=0em{}
\pstart
           \noindent{}Herrn Dr. Artur Schnitzler{\\}\textcolor{pink}{Wien}\oindex{Wien@\textbf{Wien}, \emph{Verwaltungsgebiet}|pw}{}\ledrightnote{\textcolor{pink}{Wien}}{\\}\label{K_L03691-2v}\edtext{1 Beilage}{\lemma{\textnormal{\emph{1 Beilage}}}\Cendnote{\textnormal{nicht überliefert}}}\label{K_L03691-2}\pend
           \selectlanguage{ngerman}\endnumbering\briefempfaengerindex{Schnitzler, Arthur@\textsc{Schnitzler, Arthur}!zzzZweig, Stefan@\emph{von Stefan Zweig}!1929-11-021@{2. 11. 1929}|)be}\mylabel{L03691h}
\begin{anhang}
\end{anhang}\normalsize

\doendnotes{C}
\bigskip
\vfill

\clearpage

\footnotesize

\lohead{\textsc{register}}

% Definiere theindex-Environment komplett neu ohne reledmac
\makeatletter
\renewenvironment{theindex}{%
  \section*{\indexname}%
  \setlength{\parindent}{0pt}%
  \setlength{\parskip}{0pt plus 0.3pt}%
  \let\item\@idxitem
}{%
  \clearpage
}
\makeatother

\IfFileExists{\jobname-pw.ind}{\input{\jobname-pw.ind}}{}

\end{document}

      