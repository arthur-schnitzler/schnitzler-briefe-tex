%% latex-korrekturansicht-vorspann.tex
%% Vorspann für die Korrekturansicht.
%% Lädt die gemeinsame Datei latex-vorspann.tex mit gesetztem Schalter.

\newif\ifkorrekturansicht
\korrekturansichttrue

\input{../tex-inputs/latex-vorspann}


\renewcommand{\erwaehnteOrte}{Orte: Café Griensteidl, Wien}
\renewcommand{\erwaehnteWerke}{}
\section[ Felix Salten an Arthur Schnitzler, {[}19. 9. 1896{]}]{Felix Salten an Arthur Schnitzler, {[}19. 9. 1896{]}}
\nopagebreak\mylabel{v}
\rehead{ }\normalsize\beginnumbering\briefempfaengerindex{Schnitzler, Arthur@\textsc{Schnitzler, Arthur}!zzzSalten, Felix@\emph{von Felix Salten}!1896-09-193@{{[}19. 9. 1896{]}}|(be}
\toendnotes[C]{\smallbreak\pagebreak[2]}\Standort{CUL, Schnitzler, B 89, A 1.}
\physDesc{Brief, 1 Blatt, 1 Seite, 187 Zeichen
\newline{}Handschrift: Bleistift, lateinische Kurrent
\newline{}Schnitzler: mit Bleistift datiert: »19/9 96« 
\newline{}Ordnung: mit Bleistift von unbekannter Hand nummeriert: »69« }\toendnotes[C]{\smallbreak}
\pstart
           \noindent{}{\pb}Lieber Freund, ich bin natürlich \label{K_L03170-1v}\edtext{versagt}{\lemma{\textnormal{\emph{versagt}}}\Cendnote{\textnormal{Bezug unklar}}}\label{K_L03170-1h}. Außerdem, jetzt erst, ½ 7\textsuperscript{h.} früh nach Hause geko{\geminationm}en. Es war \uline{Niemand} da.\pend
           
\pstart
           Sind Sie Abends im \label{K_L03170-2v}\edtext{\textcolor{pink}{Griensteidl}{}\ledrightnote{\textcolor{pink}{Café Griensteidl}}}{\lemma{\textnormal{\emph{Griensteidl}}}\Cendnote{\textnormal{nicht nachweisbar}}}\label{K_L03170-2h}? Ich ko{\geminationm}e jedenfalls zeitlich hin.\pend
           
\pstart
           herzlich Ihr {\\[\baselineskip]}\spacefill\mbox{Salten}\pend
           \leftskip=0em{}\endnumbering\briefempfaengerindex{Schnitzler, Arthur@\textsc{Schnitzler, Arthur}!zzzSalten, Felix@\emph{von Felix Salten}!1896-09-193@{{[}19. 9. 1896{]}}|)be}\mylabel{h}  \normalsize

\doendnotes{C}
\bigskip
\vfill

\clearpage

\footnotesize

\lohead{\textsc{register}}

% Definiere theindex-Environment komplett neu ohne reledmac
\makeatletter
\renewenvironment{theindex}{%
  \section*{\indexname}%
  \setlength{\parindent}{0pt}%
  \setlength{\parskip}{0pt plus 0.3pt}%
  \let\item\@idxitem
}{%
  \clearpage
}
\makeatother

\IfFileExists{\jobname-pw.ind}{\input{\jobname-pw.ind}}{}

\end{document}

      