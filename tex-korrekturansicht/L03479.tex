%% latex-korrekturansicht-vorspann.tex
%% Vorspann für die Korrekturansicht.
%% Lädt die gemeinsame Datei latex-vorspann.tex mit gesetztem Schalter.

\newif\ifkorrekturansicht
\korrekturansichttrue

\input{../tex-inputs/latex-vorspann}


\renewcommand{\erwaehntePersonen}{Personen: Franziska Goldmann}
\renewcommand{\erwaehnteOrte}{Orte: Berlin, Wien}
\renewcommand{\erwaehnteWerke}{Werke: Fräulein Else, Komödie der Verführung. In drei Akten}
\section[ Paul Goldmann an Arthur Schnitzler, 24. 10. 1925]{Paul Goldmann an Arthur Schnitzler, 24. 10. 1925}
\nopagebreak\mylabel{v}
\rehead{ }\normalsize\beginnumbering\briefempfaengerindex{Schnitzler, Arthur@\textsc{Schnitzler, Arthur}!zzzGoldmann, Paul@\emph{von Paul Goldmann}!1925-10-241@{24. 10. 1925}|(be}
\toendnotes[C]{\smallbreak\pagebreak[2]}\Standort{DLA, A:Schnitzler, HS.NZ85.1.3176.}
\physDesc{Brief, 1 Blatt, 2 Seiten, 831 Zeichen
\newline{}Handschrift: lila Tinte, deutsche Kurrent
\newline{}Schnitzler: mit rotem Buntstift zwei Unterstreichungen }\toendnotes[C]{\smallbreak}
\pstart
           {\pb}\textcolor{pink}{Berlin}{}\ledrightnote{\textcolor{pink}{Berlin}}, 24. 10. 25.\pend
           
\pstart{}Lieber Freund,\pend
\pstart
           Es war ſehr lieb von Dir, daß Du gleich nach Deiner \label{K_L03479-1v}\edtext{Heimkehr}{\lemma{\textnormal{\emph{Heimkehr}}}\Cendnote{\textnormal{\textcolor{blue}{Schnitzler} kam am 21. 10. 1925, aus \textcolor{pink}{Berlin} kommen, in \textcolor{pink}{Wien} an.}}}\label{K_L03479-1h}{ }\textcolor{blue}{uns}{}\ledrightnote{{$\rightarrow$}\textcolor{blue}{Franziska Goldmann}} die \textcolor{green}{Bücher}{}\ledrightnote{{$\rightarrow$}\textcolor{green}{Fräulein Else}{\newline}{$\rightarrow$}\textcolor{green}{Komödie der Verführung. In drei Akten}} geſchickt haſt. \textcolor{blue}{Tochter}{}\ledrightnote{{$\rightarrow$}\textcolor{blue}{Franziska Goldmann}} u. Vater danken Dir
               auf das Herzlichſte. \textcolor{blue}{Franzi}{}\ledrightnote{\textcolor{blue}{Franziska Goldmann}} iſt bereits in
                  »\textcolor{green}{Fräulein Elſe}{}\ledrightnote{\textcolor{green}{Fräulein Else}}« vertieft u. erklärt, es ſei
               das Schönſte, das ſie je geleſen habe, – dankt Dir auch für die eigenhändige Widmung,
               mit der ſie in ihrer Klaſſe großen Eindruck zu machen hofft. Ich freue mich darauf,
               das \textcolor{green}{Buch}{}\ledrightnote{{$\rightarrow$}\textcolor{green}{Fräulein Else}} nach meiner \textcolor{blue}{Tochter}{}\ledrightnote{{$\rightarrow$}\textcolor{blue}{Franziska Goldmann}} zu leſen. »\textcolor{green}{Komödie der Verführung}{}\ledrightnote{\textcolor{green}{Komödie der Verführung. In drei Akten}}« iſt mir bereits bekannt.
               Für die Widmung danke ich Dir noch beſonders – ebenſo wie für Deinen lieben \label{K_L03479-2v}\edtext{Beſuch}{\lemma{\textnormal{\emph{Beſuch}}}\Cendnote{\textnormal{Am 17. 10. 1925 trafen \textcolor{blue}{Goldmann} und seine Tochter \textcolor{blue}{Franziska} mit \textcolor{blue}{Schnitzler} zusammen,
                     am 20. 10. 1925 besuchte \textcolor{blue}{Schnitzler} die beiden zuhause.}}}\label{K_L03479-2h}, der für mich eine ſehr große Freude war. Wirklich – Du {\pb}biſt kaum gealtert – biſt innerlich derſelbe
               geblieben u. haſt Dich auch äußerlich nur wenig verändert.\pend
           
\pstart
           Und nun wollen wir zuſammen bleiben – in alter Freundſchaft – bis zum Schluß!
               {\\[\baselineskip]}Herzlichſt {\\[\baselineskip]}Dein {\\[\baselineskip]}\spacefill\mbox{Paul Goldmann.}\pend
           \leftskip=0em{}\endnumbering\briefempfaengerindex{Schnitzler, Arthur@\textsc{Schnitzler, Arthur}!zzzGoldmann, Paul@\emph{von Paul Goldmann}!1925-10-241@{24. 10. 1925}|)be}\mylabel{h}  \normalsize

\doendnotes{C}
\bigskip
\vfill

\clearpage

\footnotesize

\lohead{\textsc{register}}

% Definiere theindex-Environment komplett neu ohne reledmac
\makeatletter
\renewenvironment{theindex}{%
  \section*{\indexname}%
  \setlength{\parindent}{0pt}%
  \setlength{\parskip}{0pt plus 0.3pt}%
  \let\item\@idxitem
}{%
  \clearpage
}
\makeatother

\IfFileExists{\jobname-pw.ind}{\input{\jobname-pw.ind}}{}

\end{document}

      