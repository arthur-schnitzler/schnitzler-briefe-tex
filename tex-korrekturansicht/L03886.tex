%% latex-korrekturansicht-vorspann.tex
%% Vorspann für die Korrekturansicht.
%% Lädt die gemeinsame Datei latex-vorspann.tex mit gesetztem Schalter.

\newif\ifkorrekturansicht
\korrekturansichttrue

\input{../tex-inputs/latex-vorspann}


\section[Sigmund Freud an Arthur Schnitzler, 14. 5. 1912]{L03886 Sigmund Freud an Arthur Schnitzler, 14. 5. 1912}
\nopagebreak\mylabel{L03886v}
\rehead{ }\normalsize\beginnumbering\briefempfaengerindex{, @\textsc{, }!zzz, @\emph{von  }!1912-05-141@{14. 5. 1912}|(be}
\toendnotes[C]{\smallbreak\pagebreak[2]}\Standort{–, Privatbesitz, –.}
\physDesc{Brief, 1 Blatt, 2 Seiten, 1503 Zeichen
\newline{}Handschrift: schwarze Tinte, deutsche Kurrent
\newline{}Schnitzler: mit rotem Buntstift eine Streichung seitlich der Datumsangabe 
\newline{}Zusatz: Der derzeitige Aufbewahrungsort des Briefes ist nicht bekannt.
                                 Er wurde sowohl 2004 (Katalog 680, Lot 395) wie 2017 (Katalog 704,
                                 Lot 301) von Stargardt versteigert. Die Wiedergabe der ersten Seite
                                 folgt dem Katalogfaksimile von Stargardt 2017. }\Standort{Washington, DC, Library of Congress, Freud Archives, C41F8.}
\physDesc{Brief, Fotokopie, 1 Blatt, 2 Seiten, 1503 Zeichen
\newline{}Handschrift: schwarze Tinte, deutsche Kurrent}
\buchAbdrucke{\weitereDrucke{1) Sigmund Freud: \emph{Briefe an Arthur Schnitzler.} Herausgegeben von Henry Schnitzler. In: \emph{Neue deutsche Rundschau}, Jg. 66 (Januar 1955) Nr. 1, S. 95–96.} \weitereDrucke{2) Sigmund Freud: \emph{Sigmund Freud Edition. Digitale historisch-kritische Gesamtausgabe}. Herausgegeben von Christine Diercks,  Arkadi Blatow und  Elisabeth Skale. (2014–2025) \url{https://www.freudedition.net/briefe/freud-sigmund/schnitzler-arthur/1912/05/14}.} }\toendnotes[C]{\smallbreak}
\pstart
           \raggedleft{}{\pb}14. 5. 12\pend
           
\pstart
           \textcolor{gray}{\textbf{PROF. D\textsuperscript{R.} FREUD}}\hfill \textcolor{gray}{\textbf{\textcolor{pink}{WIEN, IX. BERGGASSE 19}\oindex{Wien@\textbf{Wien}!IX., Alsergrund@\textbf{IX., Alsergrund}!Berggasse 19@\textbf{Berggasse 19}, \emph{Wohngebäude}|pw}{}\ledrightnote{\textcolor{pink}{Berggasse 19}}. }}\pend
           
\pstart{}Verehrter Herr College\pend\vspace{0.5em}
\pstart
           Geſtatten Sie mir, die obige Anrede durch die Berufung auf Ihr \label{K_L03886-1v}\edtext{\textsc{rite}}{\lemma{\textnormal{\emph{rite}}}\Cendnote{\textnormal{lateinisch: rechtmäßig}}}\label{K_L03886-1} erworbenes Doktordiplom der Medizin zu
               rechtfertigen und dann mich unter die vielen Glückwünſchenden zu mengen, die Ihren
               50ſten Geburtstag feiern wollen.\pend
           
\pstart
           Es iſt mehr als ein Akt der Revanche von meiner Seite. Ich glaube mich zu erinnern,
               daß ich in der \label{K_L03886-2v}\edtext{Antwort}{\lemma{\textnormal{\emph{Antwort}}}\Cendnote{\textnormal{Sigmund Freud an Arthur Schnitzler, 8. 5. 1906. }}}\label{K_L03886-2} auf Ihre liebenswürdige
                  \label{K_L03886-3v}\edtext{Zuſchrift}{\lemma{\textnormal{\emph{Zuſchrift}}}\Cendnote{\textnormal{Sigmund Freud an Arthur Schnitzler, [nach dem 6. 5. 1931]. }}}\label{K_L03886-3} bei analogem Anlaße vor 6
               Jahren ausgeführt habe, wie ſehr ich immer Ihrer Teilnahme und Ihres Verständnißes
               bei meinen Arbeiten ſicher geweſen bin, obwol ich niemals in die Lage gekommen bin,
               ein Wort mit Ihnen zu wechſeln. Ebenſo, habe ich mich immer zu denen gerechnet, die
               Ihre ſchönen und ernſten poetiſchen Schöpfungen in ganz beſonderem Maße verſtehen und
               genießen können. Ja, ich habe mir eingebildet, daß ein Reflex der thörichten und
               frevelhaften Gering{\pb}ſchätzung, welche die Menſchen
               heute für die Erotik bereit halten, auch auf Ihr Wirken gefallen ſei, und daß Sie mir
               darum beſonders wert ſein dürften.\pend
           
\pstart
           Lachen Sie nicht darüber, daß ich ſo in die Lage komme, die feiernde Mitwelt an
               dieſem Tage bei Ihnen zu verſchwärzen – oder beſſer, lachen Sie nur darüber und
               denken Sie daß keiner von uns von ſeinen »Komplexen« frei kommt, wie meine Freunde
               ſagen.\pend
           
\pstart
           Zum Schluße aber – ich weiß nicht, ob Sie dieſes Troſtes bedürfen – laſſen Sie ſich
               ſagen, daß der Dichter ſpäter altert als gewöhnliche Menſchenkinder, und daß nach dem
               Dichter noch der Denker herauskommt.\pend
           
\pstart
           Mit herzlichen Glückwünſchen{\\[\baselineskip]} Ihr ergebener{\\[\baselineskip]}\spacefill\mbox{Freud}\pend
           \leftskip=0em{}\selectlanguage{ngerman}\endnumbering\briefempfaengerindex{, @\textsc{, }!zzz, @\emph{von  }!1912-05-141@{14. 5. 1912}|)be}\mylabel{L03886h}
\begin{anhang}
\end{anhang}\normalsize

\doendnotes{C}
\bigskip
\vfill

\clearpage

\footnotesize

\lohead{\textsc{register}}

% Definiere theindex-Environment komplett neu ohne reledmac
\makeatletter
\renewenvironment{theindex}{%
  \section*{\indexname}%
  \setlength{\parindent}{0pt}%
  \setlength{\parskip}{0pt plus 0.3pt}%
  \let\item\@idxitem
}{%
  \clearpage
}
\makeatother

\IfFileExists{\jobname-pw.ind}{\input{\jobname-pw.ind}}{}

\end{document}

      