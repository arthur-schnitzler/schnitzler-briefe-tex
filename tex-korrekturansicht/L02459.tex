%% latex-korrekturansicht-vorspann.tex
%% Vorspann für die Korrekturansicht.
%% Lädt die gemeinsame Datei latex-vorspann.tex mit gesetztem Schalter.

\newif\ifkorrekturansicht
\korrekturansichttrue

\input{../tex-inputs/latex-vorspann}


               \section[Arthur Schnitzler an Stefan Großmann, 24. 12. 1925]{ Arthur Schnitzler an Stefan Großmann, 24. 12. 1925}\nopagebreak\mylabel{v}\rehead{ }\normalsize\beginnumbering\briefempfaengerindex{Grossmann, Stefan@\textsc{Großmann, Stefan}!zzzSchnitzler, Arthur@\emph{von Arthur Schnitzler}!1925-12-241@{24. 12. 1925}|(be} \toendnotes[C]{\smallbreak\pagebreak[2]} \Standort{DLA, A:Schnitzler, HS.NZ85.1.896.}
\physDesc{Brief, 1 Blatt, 1 Seite, maschineller Durchschlag
\newline{}Schreibmaschine
\newline{}Handschrift: roter Buntstift, lateinische Kurrent (\noindent{}Beschriftung: »Großmann« und vier
                                 Unterstreichungen)}\toendnotes[C]{\smallbreak}\pstart
           \raggedleft{}{\pb}24. 12. 1925. \pend
           \pstart{}Verehrter Herr Grossmann.\pend\pstart
           Besten Dank für die freundliche Uebersendung der \label{K_L02459_1v}\edtext{\textcolor{green}{\textcolor{green}{Nummer}{}\ledrightnote{→\textcolor{green}{Das Tage-Buch}} 51}{}\ledrightnote{→\textcolor{green}{Schnitzlers Unterleibsbeschwerden}}}{\lemma{\textnormal{\emph{Nummer 51}}}\Cendnote{\textnormal{Darin ist folgende Notiz enthalten: »SCHNITZLERS UNTERLEIBSBESCHWERDEN{ / }\textcolor{blue}{Adolf Hitler} hat noch immer in \textcolor{pink}{München}, das doch längst nicht mehr die
                        dümmste Stadt der Weit sein will, seine täglich erscheinende \textcolor{brown}{Zeitung}. Dort schreibt ein
                        treudeutscher \textcolor{blue}{Mann}
                        über \textcolor{blue}{Arthur Schnitzler}s Dichtungen:{ / }›Man könnte sich mit solchen Stücken gewiß abfinden, wenn daraus das Ethos
                        eines Dichters spräche, der, indem er uns die Kehrseite solcher Liebeleien
                        wie in Geschlechtskrankheiten, Unterleibsleiden, nervösen Zerrüttungen und der Degeneration der
                        Masse der großstädtischen Bevölkerung  zeigte, warnend und
                        abschreckend wirkte.‹{ / }Kein Zweifel, in \textcolor{blue}{Schnitzler}s Dichtungen
                        fehlen die Unterleibsbeschwerden, sowohl des Darmes als der anderen Organe.
                        Eine kleine Verstopfung und \textcolor{blue}{Schnitzler}
                        wäre auch bei \textcolor{blue}{Hitler} ein gemachter
                        Mann.« (Jg. 6, H. 51, 19. 12. 1925, S. 1911.)}}}\label{K_L02459_1h}
               vom 19. Dezember. Aber warum gleich in zehn Exemplaren? Zarte Mahnung,
               weil ich meine terminlose Zusage bisher leider noch kein einziges Mal zu erfüllen
               imstande war? Ich hätte diesmal so viele Zusagen ähnlicher Art zu erfüllen gehabt,
               dass ich mich entschlossen habe keine zu erfüllen und somit will ich auch meinem
               alten Prinzip getreu nichts wirklich versprechen als was ich auch schon im selben
               Augenblick zu halten vermöchte.\pend
           \pstart
           Der \label{K_L02459_2v}\edtext{\textcolor{green}{Notiz}{}\ledrightnote{→\textcolor{green}{Residenztheater. Erstaufführung: Anatol}}}{\lemma{\textnormal{\emph{Notiz}}}\Cendnote{\textnormal{\textcolor{blue}{J. St–g. [=Josef Stolzing-Cerny]}: \emph{\textcolor{green}{Residenztheater. Erstaufführung: Anatol}}. In:
                        \emph{\textcolor{green}{Völkischer Beobachter}}, Jg. 38, Nr. 220,
                        15. 12. 1925, S. 2.}}}\label{K_L02459_2h} aus dem \textcolor{blue}{Hitler}{}\ledrightnote{\textcolor{blue}{Adolf Hitler}}-\textcolor{brown}{Blatt}{}\ledrightnote{→\textcolor{brown}{Völkischer Beobachter}}
               bleibt ein bescheidenes, aber ehrenvolles Plätzchen in meiner Sammlung gewahrt.\pend
           \pstart
           Mit verbindlichen Neujahrsgrüssen{\\[\baselineskip]}Ihr sehr ergebener\pend
           \leftskip=0em{}{\bigskip}\pstart
           \noindent{}Herrn Stefan Grossmann,{\\}Herausgeber des »\textcolor{brown}{Tagebuch}{}\ledrightnote{\textcolor{brown}{Das Tage-Buch}}\label{T_L02459_1v}\edtext{«}{\lemma{\textnormal{\emph{«}}}\Cendnote{\textnormal{an Stelle des Ausführungszeichens steht »§«}}}\label{T_L02459_1h}, \textcolor{pink}{Berlin}{}\ledrightnote{\textcolor{pink}{Berlin}}.\pend
           \endnumbering\briefempfaengerindex{Grossmann, Stefan@\textsc{Großmann, Stefan}!zzzSchnitzler, Arthur@\emph{von Arthur Schnitzler}!1925-12-241@{24. 12. 1925}|)be}\mylabel{h}  \normalsize

\doendnotes{C}
\bigskip
\vfill

\clearpage

\footnotesize

\lohead{\textsc{register}}

% Definiere theindex-Environment komplett neu ohne reledmac
\makeatletter
\renewenvironment{theindex}{%
  \section*{\indexname}%
  \setlength{\parindent}{0pt}%
  \setlength{\parskip}{0pt plus 0.3pt}%
  \let\item\@idxitem
}{%
  \clearpage
}
\makeatother

\IfFileExists{\jobname-pw.ind}{\input{\jobname-pw.ind}}{}

\end{document}

      