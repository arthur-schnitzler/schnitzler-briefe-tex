%% latex-korrekturansicht-vorspann.tex
%% Vorspann für die Korrekturansicht.
%% Lädt die gemeinsame Datei latex-vorspann.tex mit gesetztem Schalter.

\newif\ifkorrekturansicht
\korrekturansichttrue

\input{../tex-inputs/latex-vorspann}


\renewcommand{\erwaehntePersonen}{Personen: Walther Rathenau}
\renewcommand{\erwaehnteOrte}{Orte: Berlin, Frankgasse, Mürren, Station gegen das Mittaghorn, Viktoriastraße, Wien}
\renewcommand{\erwaehnteWerke}{Werke: Impressionen}
\section[ Paul Goldmann an Arthur Schnitzler, 9. 8. 1902]{Paul Goldmann an Arthur Schnitzler, 9. 8. 1902}
\nopagebreak\mylabel{v}
\rehead{ }\normalsize\beginnumbering\briefempfaengerindex{Schnitzler, Arthur@\textsc{Schnitzler, Arthur}!zzzGoldmann, Paul@\emph{von Paul Goldmann}!1902-08-091@{9. 8. 1902}|(be}
\toendnotes[C]{\smallbreak\pagebreak[2]}\Standort{DLA, A:Schnitzler, HS.NZ85.1.3172.}
\physDesc{Bildpostkarte
\newline{}Handschrift: 1) schwarze Tinte, deutsche Kurrent\hspace{1em}2) schwarze Tinte, lateinische Kurrent (\noindent{}Adresse)\hspace{1em}
\newline{}Versand: 1) Stempel: »\nobreak{}\oindex{Muerren@\textbf{Mürren}, \emph{https://www.geonames.org/ontologyP.PPL}|pwk}Mürren, 9. VIII. 02., –7\nobreak{}«.   2) Stempel: »\nobreak{}9/3 Wien 72, 12. 8. 02, 8{[}.{]} V, Bestellt\nobreak{}«. }\toendnotes[C]{\smallbreak}\pstart{}{\pb}Herrn\pend{}\pstart{}Dr. Arthur Schnitzler\pend{}\pstart{}\textcolor{pink}{Wien}{}\ledrightnote{\textcolor{pink}{Wien}}\pend{}\pstart{}\textcolor{pink}{IX. Frankgaſse 1}{}\ledrightnote{\textcolor{pink}{Frankgasse}}.\pend{}
{\bigskip}
\pstart
           \noindent{}\centering{}{\pb}\textcolor{gray}{\textbf{\textcolor{pink}{Mürren}{}\ledrightnote{\textcolor{pink}{Mürren}} – \textcolor{pink}{Station gegen das Mittagshorn}{}\ledrightnote{\textcolor{pink}{Station gegen das Mittaghorn}}}}\pend
           
\pstart
           \noindent{}\label{K_L03218-1v}\edtext{\textsc{Dr. \textcolor{blue}{Rathenau}{}\ledrightnote{\textcolor{blue}{Walther Rathenau}}}}{\lemma{\textnormal{\emph{Dr. Rathenau}}}\Cendnote{\textnormal{Bezug unklar; womöglich in Zusammenhang
                  mit \textcolor{blue}{Goldmann}s Empfehlung, \textcolor{blue}{Walther Rathenau}s \emph{\textcolor{green}{Impressionen}} zu lesen (vgl. Paul Goldmann an Arthur Schnitzler, 25. 7. [1902])}}}\label{K_L03218-1h}, \textsc{\textcolor{pink}{Berlin W.}{}\ledrightnote{\textcolor{pink}{Berlin}}}, \textcolor{pink}{\textsc{Victoriastraſse} 3}{}\ledrightnote{\textcolor{pink}{Viktoriastraße}}. Tauſend Grüße!\pend
           
\pstart
           Dein {\\[\baselineskip]}\spacefill\mbox{Paul Goldmann}\pend
           \leftskip=0em{}\endnumbering\briefempfaengerindex{Schnitzler, Arthur@\textsc{Schnitzler, Arthur}!zzzGoldmann, Paul@\emph{von Paul Goldmann}!1902-08-091@{9. 8. 1902}|)be}\mylabel{h}
\begin{anhang}
\end{anhang}\normalsize

\doendnotes{C}
\bigskip
\vfill

\clearpage

\footnotesize

\lohead{\textsc{register}}

% Definiere theindex-Environment komplett neu ohne reledmac
\makeatletter
\renewenvironment{theindex}{%
  \section*{\indexname}%
  \setlength{\parindent}{0pt}%
  \setlength{\parskip}{0pt plus 0.3pt}%
  \let\item\@idxitem
}{%
  \clearpage
}
\makeatother

\IfFileExists{\jobname-pw.ind}{\input{\jobname-pw.ind}}{}

\end{document}

      