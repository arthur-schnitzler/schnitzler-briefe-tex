%% latex-korrekturansicht-vorspann.tex
%% Vorspann für die Korrekturansicht.
%% Lädt die gemeinsame Datei latex-vorspann.tex mit gesetztem Schalter.

\newif\ifkorrekturansicht
\korrekturansichttrue

\input{../tex-inputs/latex-vorspann}


               \section[Bertha von Suttner an Arthur Schnitzler, 4. 11. 1913]{ Bertha von Suttner an Arthur Schnitzler,
                    4. 11. 1913}\nopagebreak\mylabel{v}\rehead{ }\normalsize\beginnumbering\briefempfaengerindex{Schnitzler, Arthur@\textsc{Schnitzler, Arthur}!zzzSuttner, Bertha von@\emph{von Bertha von Suttner}!1913-11-041@{4. 11. 1913}|(be} \toendnotes[C]{\smallbreak\pagebreak[2]} \Standort{CUL, Schnitzler, B 104.}
\physDesc{Postkarte
\newline{}Handschrift: schwarze Tinte, deutsche Kurrent\newline{}Versand: Stempel: »\nobreak{}\oindex{I., Innere Stadt@\textbf{I., Innere Stadt}, \emph{Bezirk (A.BZK)}|pwk}1/1 Wien 1, 5. XI. 13, VII\nobreak{}«.  
\newline{}Schnitzler: mit rotem Buntstift eine Unterstreichung }\Standort{DLA, A:Schnitzler, HS.NZ85.1.4773.}
\physDesc{1 Blatt, 1 Seite, maschinelle Abschrift}\toendnotes[C]{\smallbreak}\pstart{}{\pb}\textsc{Herrn}\pend{}\pstart{}D\textsuperscript{r}{ }\textsc{Arthur}\pend{}\pstart{}\textsc{Schnitzler}\pend{}\pstart{}\textcolor{pink}{XVIII}{}\ledrightnote{\textcolor{pink}{XVIII., Währing}}\pend{}\pstart{}\textsc{\textcolor{pink}{\label{K_L02156_1v}\edtext{Sternwartegasse}{\lemma{\textnormal{\emph{Sternwartegasse}}}\Cendnote{\textnormal{richtig: \textcolor{pink}{Sternwartestraße}}}}\label{K_L02156_1h} 71}{}\ledrightnote{\textcolor{pink}{Sternwartestraße}}}\pend{}{\bigskip}\pstart
           \centering{}{\pb}4/11 13\pend
           \pstart
           Vielen Dank! Habe jede Zeile der intereſſanten Sendung geleſen. Ueber manches
                    auch mich gründlich geärgert; beſonders über die Einſchachtlung, Etikettierg,
                    Limitierung. Damit ſoll man doch den fünf oder ſechs Vertretern der
                    Weltliteratur, die man jeweilig hat, fern bleiben!\pend
           \pstart
           Künftige Woche mache ich mich an die \label{K_L02156_2v}\edtext{Arbeit}{\lemma{\textnormal{\emph{Arbeit}}}\Cendnote{\textnormal{\textcolor{blue}{Géza
                            Baracs} gab unter seinem Pseudonym »Clément Deltour« auf
                        Subskription eine Reihe »Unsere Zeitgenossen«/»Nos contemporains« heraus.
                        Diese ist sehr selten, ein Beitrag über \textcolor{blue}{Schnitzler} konnte nicht nachgewiesen werden.}}}\label{K_L02156_2h}.\pend
           \pstart
           Meinen \label{K_L02156_3v}\edtext{Beſuch}{\lemma{\textnormal{\emph{Beſuch}}}\Cendnote{\textnormal{vgl. A. S.: \emph{Tagebuch}, 29. 10. 1913}}}\label{K_L02156_3h} in der \textcolor{pink}{Sternwartegaſſe}{}\ledrightnote{\textcolor{pink}{Sternwartestraße}} habe ich ſehr
                    genoſſen.\pend
           \pstart
           Auf bald!{\\[\baselineskip]}\spacefill\mbox{B. Suttner}\pend
           \leftskip=0em{}\endnumbering\briefempfaengerindex{Schnitzler, Arthur@\textsc{Schnitzler, Arthur}!zzzSuttner, Bertha von@\emph{von Bertha von Suttner}!1913-11-041@{4. 11. 1913}|)be}\mylabel{h}  \normalsize

\doendnotes{C}
\bigskip
\vfill

\clearpage

\footnotesize

\lohead{\textsc{register}}

% Definiere theindex-Environment komplett neu ohne reledmac
\makeatletter
\renewenvironment{theindex}{%
  \section*{\indexname}%
  \setlength{\parindent}{0pt}%
  \setlength{\parskip}{0pt plus 0.3pt}%
  \let\item\@idxitem
}{%
  \clearpage
}
\makeatother

\IfFileExists{\jobname-pw.ind}{\input{\jobname-pw.ind}}{}

\end{document}

      