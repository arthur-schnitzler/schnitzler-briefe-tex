%% latex-korrekturansicht-vorspann.tex
%% Vorspann für die Korrekturansicht.
%% Lädt die gemeinsame Datei latex-vorspann.tex mit gesetztem Schalter.

\newif\ifkorrekturansicht
\korrekturansichttrue

\input{../tex-inputs/latex-vorspann}


\renewcommand{\erwaehntePersonen}{Personen: Raoul Auernheimer, Hermann Bahr, Hugo von Hofmannsthal, Olga Schnitzler, Heinrich Schnitzler}
\renewcommand{\erwaehnteOrte}{Orte: Berlin, Dessauer Straße, Wien}
\renewcommand{\erwaehnteWerke}{Werke: Marionetten. Drei Einakter, Neue Freie Presse, Theater- und Kunstnachrichten. [Lustspieltheater, literarischer Einakterabend.], Zum großen Wurstel. Burleske in einem Akt}
\section[ Paul Goldmann an Arthur Schnitzler, 22. 3. {[}1906{]}]{Paul Goldmann an Arthur Schnitzler, 22. 3. {[}1906{]}}
\nopagebreak\mylabel{v}
\rehead{ }\normalsize\beginnumbering\briefempfaengerindex{Schnitzler, Arthur@\textsc{Schnitzler, Arthur}!zzzGoldmann, Paul@\emph{von Paul Goldmann}!1906-03-221@{22. 3. {[}1906{]}}|(be}
\toendnotes[C]{\smallbreak\pagebreak[2]}\Standort{DLA, A:Schnitzler, HS.NZ85.1.3175.}
\physDesc{Brief, 1 Blatt, 2 Seiten
\newline{}Handschrift: blaue Tinte, deutsche Kurrent
\newline{}Schnitzler: mit Bleistift das Jahr »{[}1{]}906« vermerkt }\toendnotes[C]{\smallbreak}
\pstart
           \noindent{}\raggedleft{}{\pb}\textcolor{pink}{\textcolor{gray}{\textbf{DESSAUERSTRASSE 19}}}{}\ledrightnote{\textcolor{pink}{Dessauer Straße}}\pend
           
\pstart
           \textcolor{pink}{Berlin}{}\ledrightnote{\textcolor{pink}{Berlin}}, 22. März.\pend
           
\pstart\center{}Mein lieber Freund,\pend
\pstart
           Ich habe mich ſehr über die Zuſendung Deines neuen \textcolor{green}{Werk}{}\ledrightnote{{$\rightarrow$}\textcolor{green}{Marionetten. Drei Einakter}}es gefreut und danke Dir von Herzen für das \textcolor{green}{Buch}{}\ledrightnote{{$\rightarrow$}\textcolor{green}{Marionetten. Drei Einakter}} und ganz beſonders für die
                  \label{K-L03241-1v}\edtext{Widmung}{\lemma{\textnormal{\emph{Widmung}}}\Cendnote{\textnormal{Auch \textcolor{blue}{Hermann Bahr} und
                     \textcolor{blue}{Hugo von Hofmannsthal} erhielten
                  Widmungsexemplare von \textcolor{blue}{Schnitzler}s
                  Einakterband \emph{\textcolor{green}{Marionetten}}, vgl. Arthur Schnitzler: Widmungsexemplar Marionetten für Hermann Bahr,
               23. 3. 1906 und Arthur Schnitzler: Widmungsexemplar Marionetten für Hugo von
                    Hofmannsthal, [23.?] 3. 1906.}}}\label{K-L03241-1h}.\pend
           
\pstart
           Ob ich Dir werde Oſtern in \textcolor{pink}{Wien}{}\ledrightnote{\textcolor{pink}{Wien}} die Hand drücken können, iſt \substVorne{}\textsuperscript{doch}\substDazwischen{}wieder\substHinten{} ſehr ungewiß geworden. Wahrſcheinlich komme ich {\pb}zu Oſtern überhaupt
               nicht von \textcolor{pink}{hier}{}\ledrightnote{{$\rightarrow$}\textcolor{pink}{Berlin}}{ }\label{K-L03241-2v}\edtext{fort}{\lemma{\textnormal{\emph{fort}}}\Cendnote{\textnormal{\textcolor{blue}{Goldmann} reiste zu Ostern 1906
                  nicht nach \textcolor{pink}{Wien}. Er und \textcolor{blue}{Schnitzler} sahen sich dort erst am 4. 6. 1906 und am
                     10. 6. 1906
                  wieder.}}}\label{K-L03241-2h}.\pend
           
\pstart
           Es hat mich ſehr gefreut, vom Erfolg des »\textcolor{green}{Großen
                  Wurſtl}{}\ledrightnote{\textcolor{green}{Zum großen Wurstel. Burleske in einem Akt}}« in der \textcolor{green}{N. Fr. Pr.}{}\ledrightnote{\textcolor{green}{Neue Freie Presse}} zu \label{K-L03241-3v}\edtext{leſen}{\lemma{\textnormal{\emph{leſen}}}\Cendnote{\textnormal{\textcolor{blue}{R. A.} [= \textcolor{blue}{Raoul Auernheimer}]: \emph{\textcolor{green}{Theater- und Kunstnachrichten. [Lustspieltheater,
                        literarischer Einakterabend.]}}. In: \emph{\textcolor{green}{Neue
                        Freie Presse}}, Nr. 14930, 17. 3. 1906,
                     Morgenblatt, S. 13.}}}\label{K-L03241-3h}.\pend
           
\pstart
           Alſo nochmals herzlichſten Dank und viele Grüße an Dich, \textcolor{blue}{Frau}{}\ledrightnote{{$\rightarrow$}\textcolor{blue}{Olga Schnitzler}} und \textcolor{blue}{Kind}{}\ledrightnote{{$\rightarrow$}\textcolor{blue}{Heinrich Schnitzler}} von {\\[\baselineskip]}Deinem getreuen {\\[\baselineskip]}\spacefill\mbox{Paul
                  Goldmann.}\pend
           \leftskip=0em{}\endnumbering\briefempfaengerindex{Schnitzler, Arthur@\textsc{Schnitzler, Arthur}!zzzGoldmann, Paul@\emph{von Paul Goldmann}!1906-03-221@{22. 3. {[}1906{]}}|)be}\mylabel{h}  \normalsize

\doendnotes{C}
\bigskip
\vfill

\clearpage

\footnotesize

\lohead{\textsc{register}}

% Definiere theindex-Environment komplett neu ohne reledmac
\makeatletter
\renewenvironment{theindex}{%
  \section*{\indexname}%
  \setlength{\parindent}{0pt}%
  \setlength{\parskip}{0pt plus 0.3pt}%
  \let\item\@idxitem
}{%
  \clearpage
}
\makeatother

\IfFileExists{\jobname-pw.ind}{\input{\jobname-pw.ind}}{}

\end{document}

      