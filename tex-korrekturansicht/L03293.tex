%% latex-korrekturansicht-vorspann.tex
%% Vorspann für die Korrekturansicht.
%% Lädt die gemeinsame Datei latex-vorspann.tex mit gesetztem Schalter.

\newif\ifkorrekturansicht
\korrekturansichttrue

\input{../tex-inputs/latex-vorspann}


\renewcommand{\erwaehntePersonen}{Personen: Paul Goldmann, Georg Hirschfeld, Hugo von Hofmannsthal, Gustav Schwarzkopf, Julius Szeps}
\renewcommand{\erwaehnteInstitutionen}{Institutionen: Burgtheater, Wiener Allgemeine Zeitung}
\renewcommand{\erwaehnteOrte}{Orte: Berlin, Budapest, Pelikangasse, Schulerstraße, Slawonien, Wien}
\renewcommand{\erwaehnteWerke}{Werke: ?? [Feuilleton über Paul Goldmann], Reigen. Zehn Dialoge, Scene aus der »Hochzeit der Sobeide«. (Ältere Niederschrift. Wien 1897. — Ungedruckt.), Wiener Allgemeine Montags-Zeitung, Wiener Allgemeine Rundschau}
\section[ Felix Salten an Arthur Schnitzler, 21. 6. 1899]{Felix Salten an Arthur Schnitzler, 21. 6. 1899}
\nopagebreak\mylabel{v}
\rehead{ }\normalsize\beginnumbering\briefempfaengerindex{Schnitzler, Arthur@\textsc{Schnitzler, Arthur}!zzzSalten, Felix@\emph{von Felix Salten}!1899-06-213@{21. 6. 1899}|(be}
\toendnotes[C]{\smallbreak\pagebreak[2]}\Standort{CUL, Schnitzler, B 89, A 2.}
\physDesc{Brief, 1 Blatt, 1 Seite, 1287 Zeichen
\newline{}Handschrift: schwarze Tinte, lateinische Kurrent
\newline{}Ordnung: mit Bleistift von unbekannter Hand nummeriert: »117« }\toendnotes[C]{\smallbreak}
\pstart
           \noindent{}{\pb}\textcolor{gray}{\textbf{\textbf{»\textcolor{brown}{Wiener Allgemeine
                        Zeitung}{}\ledrightnote{\textcolor{brown}{Wiener Allgemeine Zeitung}}«}}}\pend
           
\pstart
           \textcolor{gray}{\textbf{Redaction:}}\pend
           
\pstart
           \textcolor{gray}{\textbf{\textbf{\textcolor{pink}{IX/2. Pelikangaſſe Nr. 4}{}\ledrightnote{\textcolor{pink}{Pelikangasse}}.}}}\pend
           
\pstart
           \textcolor{gray}{\textbf{Adminiſtration:}}\hfill \textcolor{gray}{\textbf{\textcolor{pink}{Wien}{}\ledrightnote{\textcolor{pink}{Wien}}, am}}{ }21. Juni \textcolor{gray}{\textbf{189}}9.\pend
           
\pstart
           \textcolor{gray}{\textbf{\textbf{\textcolor{pink}{I. Schulerſtraße Nr. 20}{}\ledrightnote{\textcolor{pink}{Schulerstraße}}.}}}\pend
           
\pstart
           \textcolor{gray}{\textbf{Telegramm-Adreſſe: »Allgemeine, \textcolor{pink}{Wien}{}\ledrightnote{\textcolor{pink}{Wien}}«.}}\pend
           
\pstart
           \textcolor{gray}{\textbf{Telephon der Redaction: Nr. 805 u. 2180.}}\pend
           
\pstart
           \textcolor{gray}{\textbf{\hspace*{1.5em}„\hspace*{1.5em}„\hspace*{1.5em} Adminiſtration: Nr. 1024.}}\pend
           
\pstart{}Lieber Arthur,\pend
\pstart
           die »\textcolor{brown}{W\textsuperscript{r} Allg. Ztg}{}\ledrightnote{\textcolor{brown}{Wiener Allgemeine Zeitung}}«
               läßt vom 3. Juli an ein \label{K_L03293-1v}\edtext{\textcolor{green}{Montagfrühblatt}{}\ledrightnote{{$\rightarrow$}\textcolor{green}{Wiener Allgemeine Montags-Zeitung}}}{\lemma{\textnormal{\emph{Montagfrühblatt}}}\Cendnote{\textnormal{Die \emph{\textcolor{green}{Wiener Allgemeine Montags-Zeitung}} erschien zwischen dem 3. 7. 1899 und dem 18. 12. 1899. Chefredakteur war \textcolor{blue}{Julius Szeps}. Die Rubrik \emph{\textcolor{green}{Wiener
                     Allgemeine Rundschau}} leitete \textcolor{blue}{Salten}.}}}\label{K_L03293-1h} erscheinen, das mit einer \textcolor{green}{literarischen Revue}{}\ledrightnote{{$\rightarrow$}\textcolor{green}{Wiener Allgemeine Rundschau}} verbunden ist. Die Revue führt den Titel
                  »\textcolor{green}{W\textsuperscript{r} Allg.
                  Rundschau}{}\ledrightnote{\textcolor{green}{Wiener Allgemeine Rundschau}}.« Sie ist etwas durchaus Selbstständiges, keine Rubrik im \textcolor{green}{Blatt}{}\ledrightnote{{$\rightarrow$}\textcolor{green}{Wiener Allgemeine Montags-Zeitung}}, und soll nach dem
               Wunsch der \textcolor{brown}{Unternehmer}{}\ledrightnote{{$\rightarrow$}\textcolor{brown}{Wiener Allgemeine Zeitung}} selbst,
               »ersten Ranges« werden. Die \textcolor{green}{Zeitung}{}\ledrightnote{{$\rightarrow$}\textcolor{green}{Wiener Allgemeine Rundschau}} habe ich erhalten, und Sie können sich denken, dass ich gerne etwas
               in unserem Sinne daraus machen möchte. Da mir so wenig Zeit zur Vorbereitung bleibt,
               ist die Gefahr groß, dass ich von Anfang an, in Schwierigkeiten (in künstlerische)
               gerathe. Ich bitte Sie dringend, mir was immer zur ersten, event. zweiten N\textsuperscript{u.} zu \label{K_L03293-2v}\edtext{geben}{\lemma{\textnormal{\emph{geben}}}\Cendnote{\textnormal{Obgleich eine Veröffentlichung
                  des \emph{\textcolor{green}{Reigen}} in der \emph{\textcolor{green}{Wiener Allgemeinen Montags-Zeitung}} angedacht war (vgl. Felix Salten an Arthur Schnitzler, [27. 6. 1899]), kam es zu keiner
                  Publikation \textcolor{blue}{Schnitzler}s in dieser \textcolor{green}{Zeitung}.}}}\label{K_L03293-2h}. Großes
               oder Kleines. An \label{K_L03293-3v}\edtext{\textcolor{blue}{Hofmannsthal}{}\ledrightnote{\textcolor{blue}{Hugo von Hofmannsthal}}}{\lemma{\textnormal{\emph{Hofmannsthal}}}\Cendnote{\textnormal{In Folge erschien \textcolor{blue}{Hugo von Hofmannsthal}: \emph{\textcolor{green}{Scene aus der »Hochzeit der Sobeide«. (Ältere
                        Niederschrift. Wien 1897. — Ungedruckt.)}}. In: \emph{\textcolor{green}{Wiener Allgemeine Montags-Zeitung}}, [Jg. 1, H. 3,]
                        17. 7. 1899, S. 2–3.}}}\label{K_L03293-3h} schrieb
               ich bereits, und bitte Sie nur, nochmals auch ihn zur schleunigen Einsendung zu
               veranlaßen. Jetzt, (1\textsuperscript{h.}) besuche ich \label{K_L03293-4v}\edtext{\textcolor{blue}{Schwarzkopf}{}\ledrightnote{\textcolor{blue}{Gustav Schwarzkopf}}}{\lemma{\textnormal{\emph{Schwarzkopf}}}\Cendnote{\textnormal{Auch von \textcolor{blue}{Gustav Schwarzkopf} ist keine Publikation in der \emph{\textcolor{green}{Wiener Allgemeinen Montags-Zeitung}}
                  nachweisbar.}}}\label{K_L03293-4h}. \textcolor{blue}{Hirschfeld}{}\ledrightnote{\textcolor{blue}{Georg Hirschfeld}}, mit dem
               ich heute{ }abds. nach \textcolor{pink}{Berlin}{}\ledrightnote{\textcolor{pink}{Berlin}} fahre, hat die
               Correspondenz für \textcolor{pink}{Berlin}{}\ledrightnote{\textcolor{pink}{Berlin}} über Theater, Kunst zu
               ganz bestimmten Terminen übernommen. Montag{ }früh bin ich wieder da, Abds im \textcolor{brown}{Burgtheater}{}\ledrightnote{\textcolor{brown}{Burgtheater}} und nachher kann ich Sie hoffentl. \label{K_L03293-5v}\edtext{im Caféhaus sprechen}{\lemma{\textnormal{\emph{im Caféhaus sprechen}}}\Cendnote{\textnormal{\textcolor{blue}{Schnitzler} war zwischen 23. 6. 1899 und 28. 6. 1899 auf Reisen
                     (\textcolor{pink}{Slawonien}, \textcolor{pink}{Budapest}).}}}\label{K_L03293-5h}. Nochmals bitte, senden Sie mir was
               immer. Das Honorar ist gut.\pend
           
\pstart
           Herzlichst Ihr {\\[\baselineskip]}\spacefill\mbox{Salten}\pend
           \leftskip=0em{}
\pstart
           \noindent{}An D\textsuperscript{r}{ }\label{K_L03293-6v}\edtext{\textcolor{blue}{Goldmann}{}\ledrightnote{\textcolor{blue}{Paul Goldmann}} schreibe ich}{\lemma{\textnormal{\emph{Goldmann schreibe ich}}}\Cendnote{\textnormal{ In der Korrespondenz \textcolor{blue}{Goldmann}s mit \textcolor{blue}{Schnitzler} sind keine Hinweise darauf zu finden. In der Korrespondenz
                        \textcolor{blue}{Schnitzler}s mit \textcolor{blue}{Salten} findet sich im Brief vom 27. 7. 1899 die Erwähnung
                     eines mit \textcolor{blue}{Goldmann} in Beziehung
                     stehenden \textcolor{green}{Feuilleton}s,
                     siehe dort.}}}\label{K_L03293-6h} eben, bitte schreiben auch Sie an ihn und reden ihm zu. Es
                  ist vielleicht gut, dass er wieder auch für \textcolor{pink}{Wien}{}\ledrightnote{\textcolor{pink}{Wien}} schreibt.\pend
           \endnumbering\briefempfaengerindex{Schnitzler, Arthur@\textsc{Schnitzler, Arthur}!zzzSalten, Felix@\emph{von Felix Salten}!1899-06-213@{21. 6. 1899}|)be}\mylabel{h}  \normalsize

\doendnotes{C}
\bigskip
\vfill

\clearpage

\footnotesize

\lohead{\textsc{register}}

% Definiere theindex-Environment komplett neu ohne reledmac
\makeatletter
\renewenvironment{theindex}{%
  \section*{\indexname}%
  \setlength{\parindent}{0pt}%
  \setlength{\parskip}{0pt plus 0.3pt}%
  \let\item\@idxitem
}{%
  \clearpage
}
\makeatother

\IfFileExists{\jobname-pw.ind}{\input{\jobname-pw.ind}}{}

\end{document}

      