%% latex-korrekturansicht-vorspann.tex
%% Vorspann für die Korrekturansicht.
%% Lädt die gemeinsame Datei latex-vorspann.tex mit gesetztem Schalter.

\newif\ifkorrekturansicht
\korrekturansichttrue

\input{../tex-inputs/latex-vorspann}


               \section[Arthur Schnitzler an Hugo von Hofmannsthal, 21. 2. 1910]{ Arthur Schnitzler an Hugo von Hofmannsthal, 21. 2. 1910}\nopagebreak\mylabel{v}\rehead{ }\normalsize\beginnumbering\briefempfaengerindex{Hofmannsthal, Hugo von@\textsc{Hofmannsthal, Hugo von}!zzzSchnitzler, Arthur@\emph{von Arthur Schnitzler}!1910-02-211@{21. 2. 1910}|(be} \toendnotes[C]{\smallbreak\pagebreak[2]} \Standort{FDH, Hs-30885,135.}
\physDesc{Briefkarte
\newline{}Handschrift: schwarze Tinte, deutsche Kurrent}\buchAbdrucke{\weitereDrucke{Hugo von Hofmannsthal, Arthur Schnitzler: \emph{Briefwechsel}. Hg. Therese Nickl und Heinrich Schnitzler. Frankfurt am Main: \emph{S. Fischer} 1964, S. 248.} }\pstart
           \noindent{}{\pb}\textcolor{gray}{\textbf{Dr. Arthur
                        Schnitzler}}\hfill 21. 2. 10\pend
           \pstart
           \textcolor{gray}{\textbf{\textcolor{pink}{Wien XVIII.
                        Spoettelgasse 7}{}\ledrightnote{\textcolor{pink}{Edmund-Weiß-Gasse}}.}}\pend
           \pstart
           lieber Hugo, ich danke Ihnen herzlich für die Komoedie von \textcolor{green}{\textsc{Cristinas} Heimreiſe}{}\ledrightnote{\textcolor{green}{Cristinas Heimreise. Komödie}}; mit
               Vergnügen, bei mancherlei Bedenken mehr dramaturgiſcher Natur, hab ich ſie geleſen,
               und erwarte mir {\pb}ihre baldige \introOben{}Bühnen-\introOben{}Auferſtehung in concentrirterer Form. Worüber ich mich, auf Wunſch,
               gern und bald eingehender und mündlicher, vernehmen laſſe.\pend
           \pstart
           Morgen fahren wir auf ein paar Tage \textcolor{pink}{ſe{\geminationm}ering}{}\ledrightnote{\textcolor{pink}{Semmering}}wärts. Herzlichſt, auf bald\pend
           \pstart
           Ihr{\\[\baselineskip]}\spacefill\mbox{A.}\pend
           \leftskip=0em{}\endnumbering\briefempfaengerindex{Hofmannsthal, Hugo von@\textsc{Hofmannsthal, Hugo von}!zzzSchnitzler, Arthur@\emph{von Arthur Schnitzler}!1910-02-211@{21. 2. 1910}|)be}\mylabel{h}  \normalsize

\doendnotes{C}
\bigskip
\vfill

\clearpage

\footnotesize

\lohead{\textsc{register}}

% Definiere theindex-Environment komplett neu ohne reledmac
\makeatletter
\renewenvironment{theindex}{%
  \section*{\indexname}%
  \setlength{\parindent}{0pt}%
  \setlength{\parskip}{0pt plus 0.3pt}%
  \let\item\@idxitem
}{%
  \clearpage
}
\makeatother

\IfFileExists{\jobname-pw.ind}{\input{\jobname-pw.ind}}{}

\end{document}

      