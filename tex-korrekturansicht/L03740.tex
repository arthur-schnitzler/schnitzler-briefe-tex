%% latex-korrekturansicht-vorspann.tex
%% Vorspann für die Korrekturansicht.
%% Lädt die gemeinsame Datei latex-vorspann.tex mit gesetztem Schalter.

\newif\ifkorrekturansicht
\korrekturansichttrue

\input{../tex-inputs/latex-vorspann}


\section[Arthur Schnitzler an Stefan Zweig, 15. 10. 1927]{L03740 Arthur Schnitzler an Stefan Zweig, 15. 10. 1927}
\nopagebreak\mylabel{L03740v}
\rehead{ }\normalsize\beginnumbering\briefempfaengerindex{Zweig, Stefan@\textsc{Zweig, Stefan}!zzzSchnitzler, Arthur@\emph{von Arthur Schnitzler}!1927-10-151@{15. 10. 1927}|(be}
\toendnotes[C]{\smallbreak\pagebreak[2]}
\correspDesc{Versand  durch Arthur Schnitzler am 15. 10. 1927 in Wien
\newline{}Erhalt  durch Stefan Zweig im Zeitraum [16. 10. 1927 – 20. 10. 1927?] in Salzburg}\toendnotes[C]{\smallbreak}
\Standort{Jerusalem, National Library of Israel, ARC. Ms. Var. 305 1 58 Stefan Zweig Collection.}
\physDesc{Postkarte, 669 Zeichen
\newline{}Handschrift: schwarze Tinte, deutsche Kurrent
\newline{}Versand: Stempel: »\nobreak{}\oindex{XVIII., Währing@\textbf{XVIII., Währing}, \emph{Verwaltungsgebiet}|pwk}\textcolor{gray}{Wien} 110, 15. X. 27, \textcolor{gray}{1}9\nobreak{}«.  
\newline{}Ordnung: mit Bleistift Beschriftung: »\textsc{Schnitzler}« }\toendnotes[C]{\smallbreak}\pstart{}{\pb}\label{T_L03740-1v}\edtext{\textcolor{gray}{\textbf{A. S.}}}{\lemma{\textnormal{\emph{A. S.}}}\Cendnote{\textnormal{ovaler Absenderkleber}}}\label{T_L03740-1}\pend{}\pstart{}\textcolor{pink}{\textcolor{gray}{\textbf{WIEN, XVIII.}}}\oindex{XVIII., Währing@\textbf{XVIII., Währing}, \emph{Verwaltungsgebiet}|pw}{}\ledrightnote{\textcolor{pink}{XVIII., Währing}}\pend{}\pstart{}\textcolor{pink}{\textcolor{gray}{\textbf{STERNWARTESTR. 71}}}\oindex{Wien@\textbf{Wien}!XVIII., Währing@\textbf{XVIII., Währing}!Sternwartestraße 71@\textbf{Sternwartestraße 71}, \emph{Wohngebäude}|pw}{}\ledrightnote{\textcolor{pink}{Sternwartestraße 71}}\pend{}{\bigskip}\pstart{}Herrn Dr Stefan Zweig\pend{}\pstart{}\textcolor{pink}{Salzburg}\oindex{Salzburg@\textbf{Salzburg}, \emph{Verwaltungsgebiet}|pw}{}\ledrightnote{\textcolor{pink}{Salzburg}}\pend{}\pstart{}\textcolor{pink}{Kapuzinerberg 5}\oindex{Paschinger Schlössl@\textbf{Paschinger Schlössl}, \emph{Wohngebäude}|pw}{}\ledrightnote{\textcolor{pink}{Paschinger Schlössl}}.\pend{}{\bigskip}\vspace{1em}
\pstart
           \raggedleft{}{\pb}\textcolor{pink}{Wien}\oindex{Wien@\textbf{Wien}, \emph{Verwaltungsgebiet}|pw}{}\ledrightnote{\textcolor{pink}{Wien}}, 15. X. 27\pend
           \vspace{0.5em}
\pstart
           lieber Doctor Zweig, ich danke Ihnen sehr für die rasche
               Beantwortung meiner \textcolor{pink}{russischen}\oindex{Russland@\textbf{Russland}|pw}{}\ledrightnote{\textcolor{pink}{Russland}} Frage und
               besonders für die gleichzeitige freundliche Übersendg Ihrer zwei neuen \textcolor{green}{Bücher}\pwindex{Zweig, Stefan 28.\,11.\,1881 Wien – 23.\,2.\,1942 Petrópolis@\textsc{Zweig, Stefan} (28.\,11.\,1881 Wien – 23.\,2.\,1942 Petrópolis), \emph{Schriftsteller}!Marceline Desbordes-Valmore. Das Lebensbild einer Dichterin@\strich\emph{Marceline Desbordes-Valmore. Das Lebensbild einer Dichterin}|pwv}\pwindex{Zweig, Stefan 28.\,11.\,1881 Wien – 23.\,2.\,1942 Petrópolis@\textsc{Zweig, Stefan} (28.\,11.\,1881 Wien – 23.\,2.\,1942 Petrópolis), \emph{Schriftsteller}!Sternstunden der Menschheit@\strich\emph{Sternstunden der Menschheit}|pwv}{}\ledrightnote{{$\rightarrow$}\emph{\textcolor{green}{Marceline Desbordes-Valmore. Das Lebensbild einer Dichterin}}{\newline}{$\rightarrow$}\emph{\textcolor{green}{Sternstunden der Menschheit}}}. Vor wenigen
               Jahren hab’ ich ein \textcolor{green}{Werk}\pwindex{?? [Werk über Marceline Desbordes-Valmore]@\emph{?? [Werk über Marceline Desbordes-Valmore]}|pwv}{}\ledrightnote{{$\rightarrow$}\emph{\textcolor{green}{?? [Werk über Marceline Desbordes-Valmore]}}} über
               die \textcolor{blue}{D.-V.}\pwindex{Desbordes-Valmore, Marceline 20.\,6.\,1786 Douai – 23.\,7.\,1859 Paris@\textsc{Desbordes-Valmore, Marceline} (20.\,6.\,1786 Douai – 23.\,7.\,1859 Paris), \emph{Schauspielerin, Sängerin, Schriftstellerin}|pw}{}\ledrightnote{\textcolor{blue}{Marceline Desbordes-Valmore}} gelesen, das mir sehr dringend
               empfohlen war aber meine Erwartungen damals nicht in vollem Ausmaß erfüllt hatte. Ich
               freu mich auf Ihr \textcolor{green}{Lebensbild}\pwindex{Zweig, Stefan 28.\,11.\,1881 Wien – 23.\,2.\,1942 Petrópolis@\textsc{Zweig, Stefan} (28.\,11.\,1881 Wien – 23.\,2.\,1942 Petrópolis), \emph{Schriftsteller}!Marceline Desbordes-Valmore. Das Lebensbild einer Dichterin@\strich\emph{Marceline Desbordes-Valmore. Das Lebensbild einer Dichterin}|pwv}{}\ledrightnote{{$\rightarrow$}\emph{\textcolor{green}{Marceline Desbordes-Valmore. Das Lebensbild einer Dichterin}}}
               u auf die \textcolor{green}{Miniaturen}\pwindex{Zweig, Stefan 28.\,11.\,1881 Wien – 23.\,2.\,1942 Petrópolis@\textsc{Zweig, Stefan} (28.\,11.\,1881 Wien – 23.\,2.\,1942 Petrópolis), \emph{Schriftsteller}!Sternstunden der Menschheit@\strich\emph{Sternstunden der Menschheit}|pwv}{}\ledrightnote{{$\rightarrow$}\emph{\textcolor{green}{Sternstunden der Menschheit}}}. Ihre
               Sehnsucht, das heutige \textcolor{pink}{Rußland}\oindex{Russland@\textbf{Russland}|pw}{}\ledrightnote{\textcolor{pink}{Russland}} aus eigen
               Anschauung kennen zu ler{\pb}nen, theil ich kaum. Als erste
               Bedingung einer Hinreise würd ich stellen: Zwanzig Jahre jünger sein.\pend
           
\pstart
           Auf Wiedersehen. Herzlichst wie immer Ihr{\\[\baselineskip]}\spacefill\mbox{Arth Schnitzler}\pend
           \leftskip=0em{}\selectlanguage{ngerman}\endnumbering\briefempfaengerindex{Zweig, Stefan@\textsc{Zweig, Stefan}!zzzSchnitzler, Arthur@\emph{von Arthur Schnitzler}!1927-10-151@{15. 10. 1927}|)be}\mylabel{L03740h}
\begin{anhang}
\end{anhang}\normalsize

\doendnotes{C}
\bigskip
\vfill

\clearpage

\footnotesize

\lohead{\textsc{register}}

% Definiere theindex-Environment komplett neu ohne reledmac
\makeatletter
\renewenvironment{theindex}{%
  \section*{\indexname}%
  \setlength{\parindent}{0pt}%
  \setlength{\parskip}{0pt plus 0.3pt}%
  \let\item\@idxitem
}{%
  \clearpage
}
\makeatother

\IfFileExists{\jobname-pw.ind}{\input{\jobname-pw.ind}}{}

\end{document}

      