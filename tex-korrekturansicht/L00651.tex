%% latex-korrekturansicht-vorspann.tex
%% Vorspann für die Korrekturansicht.
%% Lädt die gemeinsame Datei latex-vorspann.tex mit gesetztem Schalter.

\newif\ifkorrekturansicht
\korrekturansichttrue

\input{../tex-inputs/latex-vorspann}


               \section[Arthur Schnitzler an Richard Beer-Hofmann, {[}14. 3. 1897?{]}]{ Arthur Schnitzler an Richard Beer-Hofmann,
                    {[}14. 3. 1897?{]}}\nopagebreak\mylabel{v}\rehead{ }\normalsize\beginnumbering\briefempfaengerindex{Beer-Hofmann, Richard@\textsc{Beer-Hofmann, Richard}!zzzSchnitzler, Arthur@\emph{von Arthur Schnitzler}!1897-03-141@{{[}14. 3. 1897?{]}}|(be} \toendnotes[C]{\smallbreak\pagebreak[2]} \Standort{YCGL, MSS 31.}
\physDesc{Briefkarte
\newline{}Handschrift: Bleistift, deutsche Kurrent\newline{}Ordnung: mit Bleistift von unbekannter Hand
                                    datiert: »ca.
                                        16. 3. 1897« }\toendnotes[C]{\smallbreak}\pstart
           \noindent{}{\pb}lieber Richard, bitte ko{\geminationm}en Sie
                        \label{K_L00651_1v}\edtext{\uline{Mittwoch}}{\lemma{\textnormal{\emph{Mittwoch}}}\Cendnote{\textnormal{Mutmaßliche Datierung unter der
                        Annahme, dass das Treffen am 14. 3. 1897 gemeint ist,
                        bei dem neben \textcolor{blue}{Beer-Hofmann} auch \textcolor{blue}{Hugo Felix}, \textcolor{blue}{Georg Hirschfeld}, 
                  \textcolor{blue}{Hugo
                            von Hofmannsthal}, \textcolor{blue}{Felix Salten}
                        und \textcolor{blue}{Leo Vanjung} anwesend waren. \textcolor{blue}{Schnitzler} las zum ersten Mal \emph{\textcolor{green}{Reigen}} im privaten Kreis vor.}}}\label{K_L00651_1h} nach
                    dem Nachtmahl zu mir; \textcolor{blue}{Hugo}{}\ledrightnote{\textcolor{blue}{Hugo von Hofmannsthal}}, \textcolor{blue}{Hirſchfeld}{}\ledrightnote{\textcolor{blue}{Georg Hirschfeld}}, \textcolor{blue}{Schwkopf}{}\ledrightnote{\textcolor{blue}{Gustav Schwarzkopf}},
                        \textcolor{blue}{Muſik}{}\ledrightnote{→\textcolor{blue}{Hugo Felix}} u. Sie\pend
           \pstart Herzlich Ihr \spacefill\mbox{Arthur}\pend{}\endnumbering\briefempfaengerindex{Beer-Hofmann, Richard@\textsc{Beer-Hofmann, Richard}!zzzSchnitzler, Arthur@\emph{von Arthur Schnitzler}!1897-03-141@{{[}14. 3. 1897?{]}}|)be}\mylabel{h}  \normalsize

\doendnotes{C}
\bigskip
\vfill

\clearpage

\footnotesize

\lohead{\textsc{register}}

% Definiere theindex-Environment komplett neu ohne reledmac
\makeatletter
\renewenvironment{theindex}{%
  \section*{\indexname}%
  \setlength{\parindent}{0pt}%
  \setlength{\parskip}{0pt plus 0.3pt}%
  \let\item\@idxitem
}{%
  \clearpage
}
\makeatother

\IfFileExists{\jobname-pw.ind}{\input{\jobname-pw.ind}}{}

\end{document}

      