%% latex-korrekturansicht-vorspann.tex
%% Vorspann für die Korrekturansicht.
%% Lädt die gemeinsame Datei latex-vorspann.tex mit gesetztem Schalter.

\newif\ifkorrekturansicht
\korrekturansichttrue

\input{../tex-inputs/latex-vorspann}


               \section[Hugo Hofmannsthal an Arthur Schnitzler, 10. 12. 1925]{ Hugo Hofmannsthal an Arthur Schnitzler, 10. 12. 1925}\nopagebreak\mylabel{v}\rehead{ }\normalsize\beginnumbering\briefempfaengerindex{Schnitzler, Arthur@\textsc{Schnitzler, Arthur}!zzzHofmannsthal, Hugo von@\emph{von Hugo von Hofmannsthal}!1925-12-101@{10. 12. 1925}|(be} \toendnotes[C]{\smallbreak\pagebreak[2]} \Standort{CUL, Schnitzler, B 43.}
\physDesc{Postkarte
\newline{}Handschrift: schwarze Tinte, lateinische Kurrent\newline{}Versand: Stempel: »\nobreak{}\oindex{Rodaun@\textbf{Rodaun}, \emph{Teil eines besiedelten Ortes (A.BSOX)}|pwk}Rodaun, 10 \textcolor{gray}{12} 25, 12V\nobreak{}«.  \newline{}Ordnung: 1) mit Bleistift von unbekannter Hand nummeriert: »\strikeout{288}\strikeout{289}\strikeout{354}\strikeout{367}\strikeout{193}« 2) mit Bleistift von unbekannter Hand nummeriert: »391«}\buchAbdrucke{\weitereDrucke{Hugo von Hofmannsthal, Arthur Schnitzler: \emph{Briefwechsel}. Hg. Therese Nickl und Heinrich Schnitzler. Frankfurt am Main: \emph{S. Fischer} 1964, S. 304.} }\toendnotes[C]{\smallbreak}\pstart{}{\pb}Herrn D\textsuperscript{r} Arthur Schnitzler\pend{}\pstart{}\textcolor{pink}{Wien}{}\ledrightnote{\textcolor{pink}{Wien}}\pend{}\pstart{}\textcolor{pink}{XVIII Sternwartestrasse 71}{}\ledrightnote{\textcolor{pink}{Sternwartestraße}}\pend{}{\bigskip}\pstart
           \raggedleft{}{\pb}\textcolor{pink}{Rodaun}{}\ledrightnote{\textcolor{pink}{Rodaun}}, Do{\geminationn}erstag\pend
           \pstart
           Mit der allergrößten Freude, lieber Arthur, an jedem beliebigen Nachmittg oder Abend
               der nächsten Woche \label{K_L02457_1v}\edtext{ab Dienstag}{\lemma{\textnormal{\emph{ab Dienstag}}}\Cendnote{\textnormal{Tatsächlich
                  entschied sich \textcolor{blue}{Schnitzler}, für Dienstag, den
                     16. 12. 1925, um \emph{\textcolor{green}{Der Gang zum Weiher}} in privatem Kreis vorzulesen.
                     Anwesend war auch \textcolor{blue}{Hofmannsthal}.}}}\label{K_L02457_1h}. Vielleicht
                  {\pb}fangen Sie ziemlich früh an
                     (7\textsuperscript{h}?) ich bin so gar kein Nachtmensch.\pend
           \pstart
           Ein Auto, um in die Stadt zu fahren, wird man ja beko{\geminationm}en
                  kö{\geminationn}en? (Ich meine natürlich ein Taxi.)\pend
           \pstart
           Also bitte telegraphiren Sie mir den Tag, den Sie wählen.\pend
           \pstart
           Herzlich Ihr{\\[\baselineskip]}\spacefill\mbox{Hugo.}\pend
           \leftskip=0em{}\endnumbering\briefempfaengerindex{Schnitzler, Arthur@\textsc{Schnitzler, Arthur}!zzzHofmannsthal, Hugo von@\emph{von Hugo von Hofmannsthal}!1925-12-101@{10. 12. 1925}|)be}\mylabel{h}  \normalsize

\doendnotes{C}
\bigskip
\vfill

\clearpage

\footnotesize

\lohead{\textsc{register}}

% Definiere theindex-Environment komplett neu ohne reledmac
\makeatletter
\renewenvironment{theindex}{%
  \section*{\indexname}%
  \setlength{\parindent}{0pt}%
  \setlength{\parskip}{0pt plus 0.3pt}%
  \let\item\@idxitem
}{%
  \clearpage
}
\makeatother

\IfFileExists{\jobname-pw.ind}{\input{\jobname-pw.ind}}{}

\end{document}

      