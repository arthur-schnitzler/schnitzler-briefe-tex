%% latex-korrekturansicht-vorspann.tex
%% Vorspann für die Korrekturansicht.
%% Lädt die gemeinsame Datei latex-vorspann.tex mit gesetztem Schalter.

\newif\ifkorrekturansicht
\korrekturansichttrue

\input{../tex-inputs/latex-vorspann}


               \section[Olga Schnitzler an Paula Beer-Hofmann, {[}30. 11. 1908?{]}]{ Olga Schnitzler an Paula Beer-Hofmann,
               {[}30. 11. 1908?{]}}\nopagebreak\mylabel{v}\rehead{ }\normalsize\beginnumbering\briefempfaengerindex{Beer-Hofmann, Paula@\textsc{Beer-Hofmann, Paula}!zzzSchnitzler, Olga@\emph{von Olga Schnitzler}!1908-11-301@{{[}30. 11. 1908?{]}}|(be} \toendnotes[C]{\smallbreak\pagebreak[2]} \Standort{YCGL, MSS 31.}
\physDesc{Brief, 1 Blatt, 2 Seiten, Umschlag
\newline{}Handschrift: schwarze Tinte, lateinische Kurrent\newline{}Versand: ohne postalischen Übermittlungsvermerk }\toendnotes[C]{\smallbreak}\pstart{}{\pb}\textcolor{gray}{\textbf{Dr. Arthur Schnitzler}}\pend{}\pstart{}\textcolor{gray}{\textbf{\textcolor{pink}{Wien XVIII. Spoettelgasse 7}{}\ledrightnote{\textcolor{pink}{Edmund-Weiß-Gasse}}.}}\pend{}{\bigskip}\pstart{}{\pb}Frau Paula Beer-Hofmann\pend{}\pstart{}\textcolor{pink}{XVIII}{}\ledrightnote{\textcolor{pink}{XVIII., Währing}}\pend{}\pstart{}\textcolor{pink}{Hasenauerstrasse 59}{}\ledrightnote{\textcolor{pink}{Hasenauerstraße}}.\pend{}{\bigskip}\pstart
           \noindent{}{\pb}\textcolor{gray}{\textbf{Dr. Arthur Schnitzler}}\pend
           \pstart
           \textcolor{gray}{\textbf{\textcolor{pink}{Wien XVIII. Spoettelgasse 7}{}\ledrightnote{\textcolor{pink}{Edmund-Weiß-Gasse}}.}}\pend
           \pstart
           Liebe Paula, ich habe eine wirtschaftliche Bitte: lassen Sie mir Ihr
               heutiges Menü sagen, damit ich den \textcolor{blue}{Herren}{}\ledrightnote{\textcolor{blue}{Richard Beer-Hofmann}{\newline}\textcolor{blue}{Alfred Kerr}}{ }\label{K_L01815_1v}\edtext{morgen nicht
               dieselben Speisen}{\lemma{\textnormal{\emph{morgen … Speisen}}}\Cendnote{\textnormal{Das erlaubt die Datierung, da seit dem Einzug in die \textcolor{pink}{Hasenauerstrasse} im November 1906 ansonsten kein Abendessen
                  unter den hier beschriebenen Bedingungen (Montag ist \textcolor{blue}{Paula Beer-Hofmann}, Dienstag \textcolor{blue}{Olga Schnitzler} Gastgeberin) belegt ist.}}}\label{K_L01815_1h} vorsetze, was sich ja
               ereignen könnte. Unsere \label{K_L01815_2v}\edtext{\textcolor{blue}{Hedwig}{}\ledrightnote{\textcolor{blue}{Hedwig Knappe}}}{\lemma{\textnormal{\emph{Hedwig}}}\Cendnote{\textnormal{Es dürfte sich
                  um \textcolor{blue}{Hedwig Knappe} handeln, ungeachtet dessen,
                  dass das \emph{\textcolor{green}{Tagebuch}} ihren Abschied für den 1. 11. 1907
                  vermerkt.}}}\label{K_L01815_2h}{ }{\pb}sehe ich heute nicht mehr wenn ich nach Hause komme,
               und sie muss zeitlich früh einkaufen. Auf Wiedersehen, Dank und einen Kuss.\pend
           \pstart
           Ihre{\\[\baselineskip]}\spacefill\mbox{Olga.}\pend
           \leftskip=0em{}\pstart
           Montag.\pend
           \endnumbering\briefempfaengerindex{Beer-Hofmann, Paula@\textsc{Beer-Hofmann, Paula}!zzzSchnitzler, Olga@\emph{von Olga Schnitzler}!1908-11-301@{{[}30. 11. 1908?{]}}|)be}\mylabel{h}  \normalsize

\doendnotes{C}
\bigskip
\vfill

\clearpage

\footnotesize

\lohead{\textsc{register}}

% Definiere theindex-Environment komplett neu ohne reledmac
\makeatletter
\renewenvironment{theindex}{%
  \section*{\indexname}%
  \setlength{\parindent}{0pt}%
  \setlength{\parskip}{0pt plus 0.3pt}%
  \let\item\@idxitem
}{%
  \clearpage
}
\makeatother

\IfFileExists{\jobname-pw.ind}{\input{\jobname-pw.ind}}{}

\end{document}

      