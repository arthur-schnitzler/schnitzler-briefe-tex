%% latex-korrekturansicht-vorspann.tex
%% Vorspann für die Korrekturansicht.
%% Lädt die gemeinsame Datei latex-vorspann.tex mit gesetztem Schalter.

\newif\ifkorrekturansicht
\korrekturansichttrue

\input{../tex-inputs/latex-vorspann}


               \section[ Paul Goldmann an Arthur Schnitzler, {[}16./17.?{]} 9. {[}1897{]}]{Paul Goldmann an Arthur Schnitzler, {[}16./17.?{]} 9. {[}1897{]}}\nopagebreak\mylabel{v}\rehead{ }\normalsize\beginnumbering\briefempfaengerindex{Schnitzler, Arthur@\textsc{Schnitzler, Arthur}!zzzGoldmann, Paul@\emph{von Paul Goldmann}!1897-09-161@{{[}16./17.?{]} 9. {[}1897{]}}|(be} \toendnotes[C]{\smallbreak\pagebreak[2]} \Standort{DLA, A:Schnitzler, HS.NZ85.1.3167.}
\physDesc{Brief, 1 Blatt, 3 Seiten
\newline{}Handschrift: schwarze Tinte, deutsche Kurrent
\newline{}Schnitzler: 1) mit Bleistift das Jahr »97« vermerkt 2) mit rotem Buntstift fünf Unterstreichungen}\toendnotes[C]{\smallbreak}\pstart
           \raggedleft{}{\pb}\textsc{\textcolor{pink}{Frankfurt}{}\ledrightnote{\textcolor{pink}{Frankfurt am Main}}}{ }\label{K_L02824-48v}\edtext{17. September}{\lemma{\textnormal{\emph{17. September}}}\Cendnote{\textnormal{Dieser und der folgende Brief (Paul Goldmann an Arthur Schnitzler, [17./18.?] 9. 1897)
                     sind auf den gleichen Tag datiert, im zweiten Brief wird aber auf den vorliegenden
                     als »geſtrigen Brief« verwiesen, wodurch entweder der vorliegende
                     auf den 16. 9. 1897 oder andernfalls der folgende auf den
                     18. 9. 1897 zu datieren wäre.}}}\label{K_L02824-48h}.\pend
           \pstart{}Mein lieber Freund,\pend\pstart
           Übermorgen gehe ich nach \textsc{\textcolor{pink}{Paris}{}\ledrightnote{\textcolor{pink}{Paris}}} zurück. Ich gehe mit ſchwerem Herzen. Der Arbeit, die mich dort erwartet, fühle
               ich mich kaum mehr gewachſen; und niederdrückend iſt das Bewußtſein, daß alle die
               harte Mühe nicht vowärts hilft und daß das einzige Reſultat meiner Thätigkeit iſt,
               mich von Jahr zu Jahr fortzufriſten. Und darüber geht das Leben \strikeout{ſo} hin. Es war hier wieder die Rede davon, mich nach
                  \textsc{\textcolor{pink}{Berlin}{}\ledrightnote{\textcolor{pink}{Berlin}}} zu ſchicken, aber Gott weiß, ob etwas daraus wird.\pend
           \pstart
           Bitte, ſchreibe mir ſofort nach \textsc{\textcolor{pink}{Paris}{}\ledrightnote{\textcolor{pink}{Paris}}}, wie es mit \textsc{\textcolor{blue}{Richard Klein}{}\ledrightnote{\textcolor{blue}{Richard Klein}}} ſteht? Was weiß man {\pb}über den Grund des
                  \label{K_L02824-1v}\edtext{Selbſtmord-Verſuches}{\lemma{\textnormal{\emph{Selbſtmord-Verſuches}}}\Cendnote{\textnormal{Am 25. 8. 1897 schoss sich der
                     seit Februar 1897 als Maler in \textcolor{pink}{Paris} lebende
                     \textcolor{blue}{Richard Klein} in seinem Atelier eine Pistolenkugel
                  in den Kopf. Obzwar die Zeitungen bereits seinen Tod meldeten, überlebte
                  er und wurde Ende September zur weiteren Genesung nach \textcolor{pink}{Wien} 
                  übersiedelt.}}}\label{K_L02824-1h}?
               Wird er mit dem Leben davon kommen?\pend
           \pstart
           Bitte, frage auch \label{K_L02824-2v}\edtext{\textsc{\textcolor{blue}{Arthur Klein}{}\ledrightnote{\textcolor{blue}{Arthur Klein}}}}{\lemma{\textnormal{\emph{Arthur Klein}}}\Cendnote{\textnormal{\textcolor{blue}{Richard Klein}s \textcolor{blue}{Bruder}}}}\label{K_L02824-2h}, ob ich nicht irgendwie in \textsc{\textcolor{pink}{Paris}{}\ledrightnote{\textcolor{pink}{Paris}}} mich des armen \textcolor{blue}{Burſchen}{}\ledrightnote{→\textcolor{blue}{Richard Klein}} annehmen kann (wenn \strikeout{\textcolor{gray}{×}} er noch dort iſt). Ich höre, daß \textsc{\textcolor{blue}{Frischauer}{}\ledrightnote{\textcolor{blue}{Berthold Frischauer}}} in \textsc{\textcolor{pink}{Paris}{}\ledrightnote{\textcolor{pink}{Paris}}} mit dem Vater \textcolor{blue}{\textsc{Klein}}{}\ledrightnote{→\textcolor{blue}{Johann Klein}} verkehrt hat. Er könnte da vielleicht gegen mich geſtänkert \strikeout{hab} und den unglücklichen \label{K_L02824-3v}\edtext{Zwiſchenfall}{\lemma{\textnormal{\emph{Zwiſchenfall}}}\Cendnote{\textnormal{siehe Paul Goldmann an Arthur Schnitzler, [25.–28.? 2. 1897]}}}\label{K_L02824-3h}, in den ich
                  ver\textcolor{gray}{wic}kelt war, lügenhaft {\pb}dargeſtellt haben. Suche doch der Sache auf den Grund zu gehen u., im Nothfalle,
               den Thatbeſtand richtigzuſtellen.\pend
           \pstart
           Ich begrüße Dich von Herzen{\\[\baselineskip]}Dein {\\[\baselineskip]}\spacefill\mbox{Paul Goldm}\pend
           \leftskip=0em{}\endnumbering\briefempfaengerindex{Schnitzler, Arthur@\textsc{Schnitzler, Arthur}!zzzGoldmann, Paul@\emph{von Paul Goldmann}!1897-09-161@{{[}16./17.?{]} 9. {[}1897{]}}|)be}\mylabel{h}  \normalsize

\doendnotes{C}
\bigskip
\vfill

\clearpage

\footnotesize

\lohead{\textsc{register}}

% Definiere theindex-Environment komplett neu ohne reledmac
\makeatletter
\renewenvironment{theindex}{%
  \section*{\indexname}%
  \setlength{\parindent}{0pt}%
  \setlength{\parskip}{0pt plus 0.3pt}%
  \let\item\@idxitem
}{%
  \clearpage
}
\makeatother

\IfFileExists{\jobname-pw.ind}{\input{\jobname-pw.ind}}{}

\end{document}

      