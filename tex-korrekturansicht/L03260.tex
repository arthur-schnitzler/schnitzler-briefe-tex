%% latex-korrekturansicht-vorspann.tex
%% Vorspann für die Korrekturansicht.
%% Lädt die gemeinsame Datei latex-vorspann.tex mit gesetztem Schalter.

\newif\ifkorrekturansicht
\korrekturansichttrue

\input{../tex-inputs/latex-vorspann}


\renewcommand{\erwaehntePersonen}{Personen:  ?? [Frau, die Salten Geld leiht], Charlotte Pohl-Glas}
\renewcommand{\erwaehnteOrte}{Orte: Wien}
\renewcommand{\erwaehnteWerke}{Werke: Tagebuch}
\section[ Felix Salten an Arthur Schnitzler, {[}27. 2. 1897{]}]{Felix Salten an Arthur Schnitzler, {[}27. 2. 1897{]}}
\nopagebreak\mylabel{v}
\rehead{ }\normalsize\beginnumbering\briefempfaengerindex{Schnitzler, Arthur@\textsc{Schnitzler, Arthur}!zzzSalten, Felix@\emph{von Felix Salten}!1897-02-271@{{[}27. 2. 1897{]}}|(be}
\toendnotes[C]{\smallbreak\pagebreak[2]}\Standort{CUL, Schnitzler, B 89, A 2.}
\physDesc{Brief, 1 Blatt, 2 Seiten, 451 Zeichen
\newline{}Handschrift: Bleistift, lateinische Kurrent
\newline{}Beilage: vermutlich von Lotte Glas, 1 Blatt, 1 Seite, schwarze Tinte, lateinische Kurrent. Mit Bleistift von unbekannter
                              Hand nummeriert: »86a« 
\newline{}Schnitzler: mit Bleistift datiert: »27/2 97« 
\newline{}Ordnung: mit Bleistift von unbekannter Hand nummeriert: »86« }\toendnotes[C]{\smallbreak}
\pstart
           \noindent{}{\pb}Lieber Arthur,{ }\label{K_L03260-1v}\edtext{\textcolor{blue}{L.}{}\ledrightnote{\textcolor{blue}{Charlotte Pohl-Glas}}}{\lemma{\textnormal{\emph{L.}}}\Cendnote{\textnormal{\textcolor{blue}{Schnitzler}s \emph{\textcolor{green}{Tagebuch}} erwähnt zum 13. 11. 1896, dass er einem Treffen von \textcolor{blue}{Charlotte Glas} und \textcolor{blue}{Salten} beigewohnt habe, das der Nachbereitung der Beziehung diente.
                  Womöglich kam es zu einem neuerlichen Kontakt?}}}\label{K_L03260-1h} schreibt mir eben wieder.
               Die \label{K_L03260-2v}\edtext{Sache}{\lemma{\textnormal{\emph{Sache}}}\Cendnote{\textnormal{unklar}}}\label{K_L03260-2h} ist noch nicht beendet und Sie drängt
               fürchterlich. Ich bitte Sie können Sie mir bis Donnerstag{ }Nachmittag{ }10f {\pb}leihen?
               Sie bekommen Sie \uline{gewiss} zurück, Donnerstag{ }Nachmittag.\pend
           
\pstart
           Herzl {\\[\baselineskip]}\spacefill\mbox{Salten}\pend
           \leftskip=0em{}{\bigskip}
\pstart
           \noindent{}{\pb}{[}hs. Pohl-Glas:{]} \label{K_L03260-3v}\edtext{Noch Eines}{\lemma{\textnormal{\emph{Noch Eines}}}\Cendnote{\textnormal{Die Zuordnung der undatierten Beilage zum Brief wird durch 
               die inhaltliche Übereinstimmung (»10fl«) und die Nummerierung mit »86a« gestützt.}}}\label{K_L03260-3h}: ich muß auch der \textcolor{blue}{Dame}{}\ledrightnote{{$\rightarrow$}\textcolor{blue}{?? [Frau, die Salten Geld leiht]}}, die mir die 10fl. gegeben hat, das
               Geld geben. Sie sagte, es ist ihr Wochengeld, sie müße es haben. Es ist doch sehr
               nett von ihr u. ich würde nicht wagen, ihr unter die Augen zu treten.\pend
           \endnumbering\briefempfaengerindex{Schnitzler, Arthur@\textsc{Schnitzler, Arthur}!zzzSalten, Felix@\emph{von Felix Salten}!1897-02-271@{{[}27. 2. 1897{]}}|)be}\mylabel{h}  \normalsize

\doendnotes{C}
\bigskip
\vfill

\clearpage

\footnotesize

\lohead{\textsc{register}}

% Definiere theindex-Environment komplett neu ohne reledmac
\makeatletter
\renewenvironment{theindex}{%
  \section*{\indexname}%
  \setlength{\parindent}{0pt}%
  \setlength{\parskip}{0pt plus 0.3pt}%
  \let\item\@idxitem
}{%
  \clearpage
}
\makeatother

\IfFileExists{\jobname-pw.ind}{\input{\jobname-pw.ind}}{}

\end{document}

      