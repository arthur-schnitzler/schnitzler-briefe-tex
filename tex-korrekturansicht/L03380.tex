%% latex-korrekturansicht-vorspann.tex
%% Vorspann für die Korrekturansicht.
%% Lädt die gemeinsame Datei latex-vorspann.tex mit gesetztem Schalter.

\newif\ifkorrekturansicht
\korrekturansichttrue

\input{../tex-inputs/latex-vorspann}


\renewcommand{\erwaehntePersonen}{Personen: Theodore Rottenberg, Olga Schnitzler}
\renewcommand{\erwaehnteOrte}{Orte: Berlin, Frankfurt am Main, Marienbad, Wien}
\renewcommand{\erwaehnteWerke}{}
\section[ Paul Goldmann an Arthur Schnitzler, 4. 8. {[}1903{]}]{Paul Goldmann an Arthur Schnitzler, 4. 8. {[}1903{]}}
\nopagebreak\mylabel{v}
\rehead{ }\normalsize\beginnumbering\briefempfaengerindex{Schnitzler, Arthur@\textsc{Schnitzler, Arthur}!zzzGoldmann, Paul@\emph{von Paul Goldmann}!1903-08-041@{4. 8. {[}1903{]}}|(be}
\toendnotes[C]{\smallbreak\pagebreak[2]}\Standort{DLA, A:Schnitzler, HS.NZ85.1.3173.}
\physDesc{Brief, 1 Blatt, 2 Seiten
\newline{}Handschrift: blaue Tinte, deutsche Kurrent
\newline{}Schnitzler: mit Bleistift das Jahr »{[}1{]}903« vermerkt }\toendnotes[C]{\smallbreak}
\pstart
           \centering{}{\pb}\textcolor{pink}{Berlin}{}\ledrightnote{\textcolor{pink}{Berlin}}, 4. Auguſt.\pend
           
\pstart{}Mein lieber Freund,\pend
\pstart
           Danke für Deinen lieben Brief!\pend
           
\pstart
           Ich habe ſchlechte Nachrichten aus \textcolor{pink}{Frankfurt}{}\ledrightnote{\textcolor{pink}{Frankfurt am Main}}.
               Vollſtändiger Stimmungsumſchlag. Von einer \label{K_L03380-1v}\edtext{gemeinſamen Reiſe}{\lemma{\textnormal{\emph{gemeinſamen Reiſe}}}\Cendnote{\textnormal{siehe Paul Goldmann an Arthur Schnitzler, 27. 6. [1903]}}}\label{K_L03380-1h} keine Rede mehr.\pend
           
\pstart
           Ich bin wieder aus allen Himmeln geſtürzt. Was ich jetzt anfange, weiß ich nicht. Mit
               Dir will ich nicht reiſen, denn ich würde zu ſehr auf Deine Stimmung drücken. Mag
               auch keine ſchönen Länder ſehen. Vielleicht gehe ich nach \textcolor{pink}{Marienbad}{}\ledrightnote{\textcolor{pink}{Marienbad}} zur Kur.\pend
           
\pstart
           An dieſer Geſchichte gehe ich wohl noch zu Grunde. Jede Schuld wird beſtraft. Ich
               hatte eine prachtvolle \textcolor{blue}{Frau}{}\ledrightnote{{$\rightarrow$}\textcolor{blue}{Theodore Rottenberg}},
               die {\pb}mich liebte. In meinem Wahn hielt ich \textcolor{blue}{ſie}{}\ledrightnote{\textcolor{blue}{Theodore Rottenberg}} für eine Dirne und trat ſie mit Füßen. Die
               Liebe iſt todt, und ich kann ſie nicht mehr erwecken. Zu ſpät bin ich zur Erkenntniß
               gekommen. Ein furchtbarer Schickſalsſpruch, dieſes: Zu ſpät.\pend
           
\pstart
           Leb’ wohl, liebſter Freund, und reiſe glücklich! {\\[\baselineskip]}Dein treuer {\\[\baselineskip]}\spacefill\mbox{Paul Goldm}\pend
           \leftskip=0em{}
\pstart
           \noindent{}Viele Grüße an \textsc{\textcolor{blue}{Olga}{}\ledrightnote{\textcolor{blue}{Olga Schnitzler}}}!\pend
           \endnumbering\briefempfaengerindex{Schnitzler, Arthur@\textsc{Schnitzler, Arthur}!zzzGoldmann, Paul@\emph{von Paul Goldmann}!1903-08-041@{4. 8. {[}1903{]}}|)be}\mylabel{h}
\begin{anhang}
\end{anhang}\normalsize

\doendnotes{C}
\bigskip
\vfill

\clearpage

\footnotesize

\lohead{\textsc{register}}

% Definiere theindex-Environment komplett neu ohne reledmac
\makeatletter
\renewenvironment{theindex}{%
  \section*{\indexname}%
  \setlength{\parindent}{0pt}%
  \setlength{\parskip}{0pt plus 0.3pt}%
  \let\item\@idxitem
}{%
  \clearpage
}
\makeatother

\IfFileExists{\jobname-pw.ind}{\input{\jobname-pw.ind}}{}

\end{document}

      