%% latex-korrekturansicht-vorspann.tex
%% Vorspann für die Korrekturansicht.
%% Lädt die gemeinsame Datei latex-vorspann.tex mit gesetztem Schalter.

\newif\ifkorrekturansicht
\korrekturansichttrue

\input{../tex-inputs/latex-vorspann}


         \renewcommand{\erwaehnteInstitutionen}{Institutionen: Neue Freie Presse}
         \renewcommand{\erwaehnteOrte}{Orte: Berlin, Hotel Saxonia, Stresemannstraße, Wien}
         \renewcommand{\erwaehnteWerke}{}
               \section[ Paul Goldmann an Arthur Schnitzler, 2. 1. {[}1900{]}]{Paul Goldmann an Arthur Schnitzler, 2. 1. {[}1900{]}}\nopagebreak\mylabel{v}\rehead{ }\normalsize\beginnumbering\briefempfaengerindex{Schnitzler, Arthur@\textsc{Schnitzler, Arthur}!zzzGoldmann, Paul@\emph{von Paul Goldmann}!1900-01-021@{2. 1. {[}1900{]}}|(be} \toendnotes[C]{\smallbreak\pagebreak[2]} \Standort{DLA, A:Schnitzler, HS.NZ85.1.3170.}
\physDesc{Brief, 1 Blatt, 1 Seite
\newline{}Handschrift: schwarze Tinte, deutsche Kurrent
\newline{}Schnitzler: mit Bleistift das Jahr »1900« vermerkt }\toendnotes[C]{\smallbreak}\pstart
           \centering{}{\pb}\textcolor{pink}{Berlin}{}\ledrightnote{\textcolor{pink}{Berlin}}, 2. Januar.\pend
           \pstart{}Mein lieber Freund,\pend\pstart
           Vielen Dank für Deine Neujahrswünſche, die ich herzlichſt erwidere!\pend
           \pstart
           Ich bin in \label{K_L02901-1v}\edtext{\textcolor{pink}{Berlin}{}\ledrightnote{\textcolor{pink}{Berlin}}}{\lemma{\textnormal{\emph{Berlin}}}\Cendnote{\textnormal{Mit dem 1. 1. 1900 begann \textcolor{blue}{Goldmann}s
                  Tätigkeit als \textcolor{pink}{Berlin}er Theaterkorrespondent
                  der \emph{\textcolor{brown}{Neuen Freien Presse}}.}}}\label{K_L02901-1h} und wohne: \textsc{\textcolor{pink}{Hotel »Saxonia«}{}\ledrightnote{\textcolor{pink}{Hotel Saxonia}}, \textcolor{pink}{Königgrätzer Straße}{}\ledrightnote{\textcolor{pink}{Stresemannstraße}}}.\pend
           \pstart
           Dein {\\[\baselineskip]}treuer {\\[\baselineskip]}\spacefill\mbox{Paul Goldmann}\pend
           \leftskip=0em{}\endnumbering\briefempfaengerindex{Schnitzler, Arthur@\textsc{Schnitzler, Arthur}!zzzGoldmann, Paul@\emph{von Paul Goldmann}!1900-01-021@{2. 1. {[}1900{]}}|)be}\mylabel{h}  \normalsize

\doendnotes{C}
\bigskip
\vfill

\clearpage

\footnotesize

\lohead{\textsc{register}}

% Definiere theindex-Environment komplett neu ohne reledmac
\makeatletter
\renewenvironment{theindex}{%
  \section*{\indexname}%
  \setlength{\parindent}{0pt}%
  \setlength{\parskip}{0pt plus 0.3pt}%
  \let\item\@idxitem
}{%
  \clearpage
}
\makeatother

\IfFileExists{\jobname-pw.ind}{\input{\jobname-pw.ind}}{}

\end{document}

      