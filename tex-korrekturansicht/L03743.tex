%% latex-korrekturansicht-vorspann.tex
%% Vorspann für die Korrekturansicht.
%% Lädt die gemeinsame Datei latex-vorspann.tex mit gesetztem Schalter.

\newif\ifkorrekturansicht
\korrekturansichttrue

\input{../tex-inputs/latex-vorspann}


\section[Arthur Schnitzler an Stefan Zweig, 4. 7. 1929]{L03743 Arthur Schnitzler an Stefan Zweig, 4. 7. 1929}
\nopagebreak\mylabel{L03743v}
\rehead{ }\normalsize\beginnumbering\briefempfaengerindex{Zweig, Stefan@\textsc{Zweig, Stefan}!zzzSchnitzler, Arthur@\emph{von Arthur Schnitzler}!1929-07-041@{4. 7. 1929}|(be}
\toendnotes[C]{\smallbreak\pagebreak[2]}
\correspDesc{Versand  durch Arthur Schnitzler am 4. 7. 1929 in Wien
\newline{}Erhalt  durch Stefan Zweig im Zeitraum [5. 7. 1929 – 9. 7. 1929?] in Salzburg}\toendnotes[C]{\smallbreak}
\Standort{Jerusalem, National Library of Israel, ARC. Ms. Var. 305 1 58 Stefan Zweig Collection.}
\physDesc{Brief, 1 Blatt, 2 Seiten, 1466 Zeichen
\newline{}Handschrift: Bleistift, lateinische Kurrent (\noindent{}zwei Unterstreichungen, eine Streichung, eine Ergänzung,
                                 Schlussformel und Unterschrift)}\toendnotes[C]{\smallbreak}
\pstart
           {\pb}\textcolor{gray}{\textbf{D\textsuperscript{R} ARTHUR SCHNITZLER}}\hfill 4. 7. 1929.\pend
           
\pstart
           \textcolor{gray}{\textbf{\textcolor{pink}{WIEN, XVIII.
                        STERNWARTESTRASSE 71}\oindex{Wien@\textbf{Wien}!XVIII., Währing@\textbf{XVIII., Währing}!Sternwartestraße 71@\textbf{Sternwartestraße 71}, \emph{Wohngebäude}|pw}{}\ledrightnote{\textcolor{pink}{Sternwartestraße 71}}.}}\pend
           
\pstart{}Lieber Doktor Zweig.\pend\vspace{0.5em}
\pstart
           Sie wissen jedesfalls von der Absicht des \textcolor{brown}{Schutzverbandes}\orgindex{Schutzverband deutscher Schriftsteller@Schutzverband deutscher Schriftsteller|pw}{}\ledrightnote{\textcolor{brown}{Schutzverband deutscher Schriftsteller}} zu wohltätigem Zwecke in einer der grossen \textcolor{pink}{Berlin}\oindex{Berlin@\textbf{Berlin}, \emph{Hauptstadt}|pw}{}\ledrightnote{\textcolor{pink}{Berlin}}er Kunsthandlungen eine Ausstellung von Manuscripten und
               im Anschluss daran eine \label{K_L03743-1v}\edtext{Versteigerung}{\lemma{\textnormal{\emph{Versteigerung}}}\Cendnote{\textnormal{Die Veranstaltung kam nicht zustande.}}}\label{K_L03743-1} vornehmen zu lassen. \textcolor{blue}{Roda-Roda}\pwindex{Roda Roda, Alexander 13.\,4.\,1872 Drnovice – 20.\,8.\,1945 New York City@\textsc{Roda Roda, Alexander} (13.\,4.\,1872 Drnovice – 20.\,8.\,1945 New York City), \emph{Schriftsteller}|pw}{}\ledrightnote{\textcolor{blue}{Alexander Roda Roda}}, der mir über die Sache geschrieben hat, ist dafür,
               dass die Sammlung entweder im Ganzen oder aber in »Loten« von etwa 10 Manuscripten
               aufgeboten werden müsse. Er schliesst sich übrigens meiner Meinung an, dass man \uline{Ausrufpreise} ansetze, \uline{unterhalb} deren ein Verkauf nicht stattfinden dürfe. Sie, lieber Doktor
               Zweig, sind ja in solchen Handschriften-Angelegenheiten besonders sachverständig. Und
               ich frage daher bei Ihnen ganz unverbindlich an, welche Ausrufpreise Sie im
               allgemeinen und im besonderen für richtig fänden, wenn ich \label{K_L03743-2v}\edtext{z. E.}{\lemma{\textnormal{\emph{z. E.}}}\Cendnote{\textnormal{zum
                  Einen}}}\label{K_L03743-2} für eine solche Versteigerung alte Manuscripte von Gedichten oder
                  \strikeout{beispielsweise} das erste bleistiftgeschriebene
               Manuscript des »\textcolor{green}{Grünen Kakadu}\pwindex{Schnitzler, Arthur 15. 5. 1862 Wien – 21. 10. 1931 ebd.@\textsc{Schnitzler, Arthur} (15. 5. 1862 Wien – 21. 10. 1931 ebd.), \emph{Schriftsteller, Mediziner}!grüne Kakadu. Groteske in einem Akt@\strich\emph{Der grüne Kakadu. Groteske in einem Akt}|pw}{}\ledrightnote{\textcolor{green}{Der grüne Kakadu. Groteske in einem Akt}}« oder eines \introOben{}andern\introOben{} Einakters zur Verfügung stellte. \pend
           
\pstart
           Dieser Brief trifft Sie wohl noch in \textcolor{pink}{Salzburg}\oindex{Salzburg@\textbf{Salzburg}, \emph{Verwaltungsgebiet}|pw}{}\ledrightnote{\textcolor{pink}{Salzburg}} an.
               Teilen Sie mir bitte mit, wohin Ihre Sommerpläne gehen. Es wäre eine angenehme
               Aussicht Ihnen wieder einmal in der \textcolor{pink}{Schweiz}\oindex{Schweiz@\textbf{Schweiz}|pw}{}\ledrightnote{\textcolor{pink}{Schweiz}}
               oder sonstwo zu begegnen. \textcolor{pink}{Salzburg}\oindex{Salzburg@\textbf{Salzburg}, \emph{Verwaltungsgebiet}|pw}{}\ledrightnote{\textcolor{pink}{Salzburg}} und \textcolor{pink}{Wien}\oindex{Wien@\textbf{Wien}, \emph{Verwaltungsgebiet}|pw}{}\ledrightnote{\textcolor{pink}{Wien}} liegen offenbar zu nah von einander. Zu
               wievielen Erfolgen habe ich Ihnen eigentlich zu gratulieren, seit wir uns zuletzt
               gesehen haben? Nehmen Sie eine Gesamtgratulation zugleich mit meinen herzlichsten
               Grüssen freundlichst entgegen. \pend
           
\pstart
           {[}hs.:{]} Ihr{\\[\baselineskip]}\spacefill\mbox{ArtSchnitzl}\pend
           \leftskip=0em{}
\pstart
           \noindent{}{[}ms.:{]} Herrn Dr. Stefan Zweig\pend
           
\pstart
           \noindent{}\textcolor{pink}{Salzburg}\oindex{Salzburg@\textbf{Salzburg}, \emph{Verwaltungsgebiet}|pw}{}\ledrightnote{\textcolor{pink}{Salzburg}}.\pend
           \selectlanguage{ngerman}\endnumbering\briefempfaengerindex{Zweig, Stefan@\textsc{Zweig, Stefan}!zzzSchnitzler, Arthur@\emph{von Arthur Schnitzler}!1929-07-041@{4. 7. 1929}|)be}\mylabel{L03743h}
\begin{anhang}
\end{anhang}\normalsize

\doendnotes{C}
\bigskip
\vfill

\clearpage

\footnotesize

\lohead{\textsc{register}}

% Definiere theindex-Environment komplett neu ohne reledmac
\makeatletter
\renewenvironment{theindex}{%
  \section*{\indexname}%
  \setlength{\parindent}{0pt}%
  \setlength{\parskip}{0pt plus 0.3pt}%
  \let\item\@idxitem
}{%
  \clearpage
}
\makeatother

\IfFileExists{\jobname-pw.ind}{\input{\jobname-pw.ind}}{}

\end{document}

      