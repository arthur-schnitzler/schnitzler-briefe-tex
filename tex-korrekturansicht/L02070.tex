%% latex-korrekturansicht-vorspann.tex
%% Vorspann für die Korrekturansicht.
%% Lädt die gemeinsame Datei latex-vorspann.tex mit gesetztem Schalter.

\newif\ifkorrekturansicht
\korrekturansichttrue

\input{../tex-inputs/latex-vorspann}


               \section[Hugo und Gerty von Hofmannsthal an Arthur Schnitzler, 23. 5. 1912]{ Hugo und Gerty von Hofmannsthal an Arthur Schnitzler,
               23. 5. 1912}\nopagebreak\mylabel{v}\rehead{ }\normalsize\beginnumbering\briefempfaengerindex{Schnitzler, Arthur@\textsc{Schnitzler, Arthur}!zzzHofmannsthal, Gertrude von@\emph{von Gertrude von Hofmannsthal}!1912-05-232@{23. 5. 1912}|(be}\briefempfaengerindex{Schnitzler, Arthur@\textsc{Schnitzler, Arthur}!zzzHofmannsthal, Hugo von@\emph{von Hugo von Hofmannsthal}!1912-05-232@{23. 5. 1912}|(be} \toendnotes[C]{\smallbreak\pagebreak[2]} \Standort{CUL, Schnitzler, B 43.}
\physDesc{Bildpostkarte
\newline{}Handschrift Hugo von Hofmannsthal: Bleistift, deutsche Kurrent\newline{}Handschrift Gertrude von Hofmannsthal: Bleistift, lateinische Kurrent\newline{}Versand: Stempel: »\nobreak{}\oindex{Sterzing@\textbf{Sterzing}, \emph{http://www.geonames.org/ontologyP.PPLA3}|pwk}\textcolor{gray}{Sterzing}\nobreak{}«.  \newline{}Ordnung: 1) mit Bleistift von unbekannter Hand nummeriert: »\strikeout{327}« 2) mit Bleistift von unbekannter Hand nummeriert:
                                    »377«}\buchAbdrucke{\weitereDrucke{Hugo von Hofmannsthal, Arthur Schnitzler: \emph{Briefwechsel}. Hg. Therese Nickl und Heinrich Schnitzler. Frankfurt am Main: \emph{S. Fischer} 1964, S. 265.} }\toendnotes[C]{\smallbreak}\pstart{}{\pb}\textsc{Herrn D\textsuperscript{r} A Schnitzler}\pend{}\pstart{}\textcolor{pink}{\textsc{Wien}}{}\ledrightnote{\textcolor{pink}{Wien}}\pend{}\pstart{}\textsc{\textcolor{pink}{XVIII Sternwartestrasse 71}{}\ledrightnote{\textcolor{pink}{Sternwartestraße}}.}\pend{}{\bigskip}\pstart
           \noindent{}\centering{}\textcolor{gray}{\textbf{{\pb}Historische Gemälde in der
                        \textcolor{pink}{Alten Post}{}\ledrightnote{\textcolor{pink}{Alte Post}} zu \textcolor{pink}{Sterzing}{}\ledrightnote{\textcolor{pink}{Sterzing}}:}}\pend
           \pstart
           \noindent{}\centering{}\textcolor{gray}{\textbf{\textcolor{green}{Zunftfahne}{}\ledrightnote{\textcolor{green}{Zunftfahne der Bäcker und Müller mit den Heiligen Elisabeth, Sebastian und Agnes}}, gemalt von \textcolor{blue}{Anton Siess}{}\ledrightnote{\textcolor{blue}{Anton Sieß}}{ }1794}}\pend
           \pstart
           \noindent{}\centering{}\textcolor{gray}{\textbf{{\pb}\textcolor{pink}{Central-Hotel Alte Post}{}\ledrightnote{\textcolor{pink}{Alte Post}}}}\pend
           \pstart
           \noindent{}\centering{}\textcolor{gray}{\textbf{(erbaut 1556)}}\pend
           \pstart
           \noindent{}\centering{}\textcolor{gray}{\textbf{in \textcolor{pink}{Sterzing a. Br.}{}\ledrightnote{\textcolor{pink}{Sterzing}}
                     (950 m)}}\pend
           \pstart
           \noindent{}\centering{}\textcolor{gray}{\textbf{Besitzer: \textcolor{blue}{F. P. KLEEWEIN}{}\ledrightnote{\textcolor{blue}{Franz Paul Kleewein}}}}\pend
           \pstart
           \raggedleft{}23 V. 912.\pend
           \pstart
           An \textcolor{pink}{Welsberg}{}\ledrightnote{\textcolor{pink}{Welsberg-Taisten}} vorüberfahrend gedachten wir lieber
               Tage, die wir gern erneuern möchten. Sind übermorgen \textcolor{pink}{\textsc{Paris}}{}\ledrightnote{\textcolor{pink}{Paris}}, längſtens 5 VI{ }\textcolor{pink}{Rodaun}{}\ledrightnote{\textcolor{pink}{Rodaun}}.\pend
           \pstart Herzlichst Ihr\spacefill\mbox{Hugo.}\pend{}\pstart
           \noindent{}{[}hs. G. Hofmannsthal:{]} \label{T_L02070_1v}\edtext{Viele herzliche Grüsse \spacefill\mbox{Gerty.}}{\lemma{\textnormal{\emph{Viele … Gerty.}}}\Cendnote{\textnormal{quer am rechten Rand}}}\label{T_L02070_1h}\pend
           \endnumbering\briefempfaengerindex{Schnitzler, Arthur@\textsc{Schnitzler, Arthur}!zzzHofmannsthal, Gertrude von@\emph{von Gertrude von Hofmannsthal}!1912-05-232@{23. 5. 1912}|)be}\briefempfaengerindex{Schnitzler, Arthur@\textsc{Schnitzler, Arthur}!zzzHofmannsthal, Hugo von@\emph{von Hugo von Hofmannsthal}!1912-05-232@{23. 5. 1912}|)be}\mylabel{h}  \normalsize

\doendnotes{C}
\bigskip
\vfill

\clearpage

\footnotesize

\lohead{\textsc{register}}

% Definiere theindex-Environment komplett neu ohne reledmac
\makeatletter
\renewenvironment{theindex}{%
  \section*{\indexname}%
  \setlength{\parindent}{0pt}%
  \setlength{\parskip}{0pt plus 0.3pt}%
  \let\item\@idxitem
}{%
  \clearpage
}
\makeatother

\IfFileExists{\jobname-pw.ind}{\input{\jobname-pw.ind}}{}

\end{document}

      