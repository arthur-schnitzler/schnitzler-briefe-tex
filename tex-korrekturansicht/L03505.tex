%% latex-korrekturansicht-vorspann.tex
%% Vorspann für die Korrekturansicht.
%% Lädt die gemeinsame Datei latex-vorspann.tex mit gesetztem Schalter.

\newif\ifkorrekturansicht
\korrekturansichttrue

\input{../tex-inputs/latex-vorspann}


\renewcommand{\erwaehntePersonen}{Personen: Anna Katharina Rehmann, Felix Salten, Paul Salten, Ottilie Salten}
\renewcommand{\erwaehnteInstitutionen}{Institutionen: Südbahnstrecke}
\renewcommand{\erwaehnteOrte}{Orte: Edlach, Höhlenstein, Kaltenleutgeben, Niederösterreich, Wien}
\renewcommand{\erwaehnteWerke}{}
\section[ Felix Salten an Arthur Schnitzler, 31. 7. 1909]{Felix Salten an Arthur Schnitzler, 31. 7. 1909}
\nopagebreak\mylabel{v}
\rehead{ }\normalsize\beginnumbering\briefempfaengerindex{Schnitzler, Arthur@\textsc{Schnitzler, Arthur}!zzzSalten, Felix@\emph{von Felix Salten}!1909-07-312@{31. 7. 1909}|(be}
\toendnotes[C]{\smallbreak\pagebreak[2]}\Standort{CUL, Schnitzler, B 89, B 1.}
\physDesc{Bildpostkarte, 394 Zeichen
\newline{}Handschrift: schwarze Tinte, lateinische Kurrent
\newline{}Versand: Stempel: »\nobreak{}\oindex{Hoehlenstein@\textbf{Höhlenstein}, \emph{P.PPLQ}|pwk}Höhlenstein\nobreak{}«.  
\newline{}Schnitzler: mit Bleistift Vermerk: »\textsc{Salten}« 
\newline{}Ordnung: mit Bleistift von unbekannter Hand nummeriert: »255« }\toendnotes[C]{\smallbreak}\pstart{}{\pb}Herrn D\textsuperscript{r} Arthur Schnitzler\pend{}\pstart{}\textcolor{pink}{Edlach \textsuperscript{b}/Reichenau}{}\ledrightnote{\textcolor{pink}{Edlach}}\pend{}\pstart{}\textcolor{pink}{Nied. Öst.}{}\ledrightnote{\textcolor{pink}{Niederösterreich}}\pend{}\pstart{}\textcolor{brown}{Südbahn}{}\ledrightnote{\textcolor{brown}{Südbahnstrecke}}\pend{}
{\bigskip}
\pstart
           \noindent{}\centering{}{\pb}{[}Fotografie von \textcolor{blue}{Paul}{}\ledrightnote{\textcolor{blue}{Paul Salten}} und \textcolor{blue}{Anna Salten}{}\ledrightnote{\textcolor{blue}{Anna Katharina Rehmann}} in
                     Tracht{]}\pend
           
\pstart
           {\pb}Lieber, wir
               sind etwa \label{K_L03505-1v}\edtext{gegen den 10. Aug.}{\lemma{\textnormal{\emph{gegen den 10. Aug.}}}\Cendnote{\textnormal{Rund um den 10. 8. 1909 ist kein Treffen zwischen \textcolor{blue}{Salten} und \textcolor{blue}{Schnitzler}
                  nachweisbar.}}}\label{K_L03505-1h} schon in \textcolor{pink}{Wien}{}\ledrightnote{\textcolor{pink}{Wien}}. Vielleicht
               komme ich – wenns Ihnen recht ist – einmal zum Tennis hinaus, ehe wir nach \textcolor{pink}{Kaltenleutgeben}{}\ledrightnote{\textcolor{pink}{Kaltenleutgeben}} gehen. – Das »umseitige« Bild
               ist mit dem neuen Apparat von mir gemacht. Viele herzliche Grüße von \textcolor{blue}{uns}{}\ledrightnote{{$\rightarrow$}\textcolor{blue}{Ottilie Salten}} zu Ihnen
                  allen\textcolor{gray}{;} Ihr \spacefill\mbox{Salten}\pend
           
\pstart
           \textcolor{pink}{Landro}{}\ledrightnote{\textcolor{pink}{Höhlenstein}}, 31. VII. 09\pend
           \endnumbering\briefempfaengerindex{Schnitzler, Arthur@\textsc{Schnitzler, Arthur}!zzzSalten, Felix@\emph{von Felix Salten}!1909-07-312@{31. 7. 1909}|)be}\mylabel{h}  \normalsize

\doendnotes{C}
\bigskip
\vfill

\clearpage

\footnotesize

\lohead{\textsc{register}}

% Definiere theindex-Environment komplett neu ohne reledmac
\makeatletter
\renewenvironment{theindex}{%
  \section*{\indexname}%
  \setlength{\parindent}{0pt}%
  \setlength{\parskip}{0pt plus 0.3pt}%
  \let\item\@idxitem
}{%
  \clearpage
}
\makeatother

\IfFileExists{\jobname-pw.ind}{\input{\jobname-pw.ind}}{}

\end{document}

      