%% latex-korrekturansicht-vorspann.tex
%% Vorspann für die Korrekturansicht.
%% Lädt die gemeinsame Datei latex-vorspann.tex mit gesetztem Schalter.

\newif\ifkorrekturansicht
\korrekturansichttrue

\input{../tex-inputs/latex-vorspann}


               \section[Hugo von Hofmannsthal an Arthur Schnitzler, {[}3.? 6. 1898{]}]{ Hugo von Hofmannsthal an Arthur Schnitzler, {[}3.? 6. 1898{]}}\nopagebreak\mylabel{v}\rehead{ }\normalsize\beginnumbering\briefempfaengerindex{Schnitzler, Arthur@\textsc{Schnitzler, Arthur}!zzzHofmannsthal, Hugo von@\emph{von Hugo von Hofmannsthal}!1898-06-031@{{[}3.? 6. 1898{]}}|(be} \toendnotes[C]{\smallbreak\pagebreak[2]} \Standort{CUL, Schnitzler, B 43b/1.}
\physDesc{Brief, 1 Blatt, 3 Seiten
\newline{}Handschrift: schwarze Tinte, deutsche Kurrent
\newline{}Schnitzler: mit Bleistift datiert: »Mai? 98« \newline{}Ordnung: mit Bleistift von unbekannter Hand nummeriert:
                                    »113« }\buchAbdrucke{\weitereDrucke{Hugo von Hofmannsthal, Arthur Schnitzler: \emph{Briefwechsel}. Hg. Therese Nickl und Heinrich Schnitzler. Frankfurt am Main: \emph{S. Fischer} 1964, S. 101.} }\toendnotes[C]{\smallbreak}\pstart
           \raggedleft{}{\pb}\textcolor{pink}{Hinterbrühl}{}\ledrightnote{\textcolor{pink}{Hinterbrühl}}, Freitag.\pend
           \pstart{}mein lieber Arthur\pend\pstart
           \label{K_L00800_1v}\edtext{Dienstag}{\lemma{\textnormal{\emph{Dienstag}}}\Cendnote{\textnormal{Durch die privaten Aufzeichungen \textcolor{blue}{Hofmannsthal}s (S. 397–398) ergibt
                  sich für die Maturalernzeit nur ein Freitag in \textcolor{pink}{Hinterbrühl}, an dem er am Dienstag und Mittwoch zuvor in \textcolor{pink}{Wien} war, nämlich der 3. 6. 1898.}}}\label{K_L00800_1h} war ich
               im Café bin aber um ½ 11{ }ſehr müd geworden und Mittwoch war ich
               überhaupt von der Lernerei ſehr müd. Auch davon iſt man ein biſſel niedergeſchlagen,
               daſs es gar {\pb}nicht So{\geminationm}er werden kann und ſo wenig Sonne iſt.\pend
           \pstart
           Bitte gehen Sie nur gleich fort nach \textcolor{pink}{Kärnten}{}\ledrightnote{\textcolor{pink}{Kärnten}}{ }ſobald es ſchön iſt, es giebt doch Möglichkeiten,
               ohne Betrug, einer ſo tiefen Verſtimmung entgegenzuarbeiten.\pend
           \pstart
           {\pb}Aber bitte laſſen Sie mich nicht
               ganz ohne Verſtändigung, es freut einen i{\geminationm}er ſo die
               Menſchen die man gern hat, in irgend einer Landſchaft zu denken.\pend
           \pstart
           Von Herzen\hspace*{2em}Ihr{\\[\baselineskip]}\spacefill\mbox{Hugo}\pend
           \leftskip=0em{}\endnumbering\briefempfaengerindex{Schnitzler, Arthur@\textsc{Schnitzler, Arthur}!zzzHofmannsthal, Hugo von@\emph{von Hugo von Hofmannsthal}!1898-06-031@{{[}3.? 6. 1898{]}}|)be}\mylabel{h}  \normalsize

\doendnotes{C}
\bigskip
\vfill

\clearpage

\footnotesize

\lohead{\textsc{register}}

% Definiere theindex-Environment komplett neu ohne reledmac
\makeatletter
\renewenvironment{theindex}{%
  \section*{\indexname}%
  \setlength{\parindent}{0pt}%
  \setlength{\parskip}{0pt plus 0.3pt}%
  \let\item\@idxitem
}{%
  \clearpage
}
\makeatother

\IfFileExists{\jobname-pw.ind}{\input{\jobname-pw.ind}}{}

\end{document}

      