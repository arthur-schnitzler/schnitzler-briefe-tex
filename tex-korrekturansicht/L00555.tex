%% latex-korrekturansicht-vorspann.tex
%% Vorspann für die Korrekturansicht.
%% Lädt die gemeinsame Datei latex-vorspann.tex mit gesetztem Schalter.

\newif\ifkorrekturansicht
\korrekturansichttrue

\input{../tex-inputs/latex-vorspann}


               \section[Arthur Schnitzler an Richard Beer-Hofmann, 27. 6. 1896]{ Arthur Schnitzler an Richard Beer-Hofmann,
               27. 6. 1896}\nopagebreak\mylabel{v}\rehead{ }\normalsize\beginnumbering\briefempfaengerindex{Beer-Hofmann, Richard@\textsc{Beer-Hofmann, Richard}!zzzSchnitzler, Arthur@\emph{von Arthur Schnitzler}!1896-06-271@{27. 6. 1896}|(be} \toendnotes[C]{\smallbreak\pagebreak[2]} \Standort{YCGL, MSS 31.}
\physDesc{Postkarte
\newline{}Handschrift: schwarze Tinte, deutsche Kurrent\newline{}Versand: 1) Stempel: »\nobreak{}\oindex{I., Innere Stadt@\textbf{I., Innere Stadt}, \emph{Bezirk (A.BZK)}|pwk}Wien 1/1, 27. 6. 96, 8–9N\nobreak{}«.  2) Stempel: »\nobreak{}\oindex{Salzburg@\textbf{Salzburg}, \emph{Besiedelter Ort (A.BSO)}|pwk}Salzburg Stadt, 28 6 96, 10F\nobreak{}«. }\buchAbdrucke{\weitereDrucke{Arthur Schnitzler, Richard Beer-Hofmann: \emph{Briefwechsel 1891–1931}. Hg. Konstanze Fliedl. Wien, Zürich: \emph{Europaverlag} 1992, S. 91.} }\toendnotes[C]{\smallbreak}\pstart{}{\pb}Herrn \textsc{Dr. Richard
                     Beer-Hofmann}\pend{}\pstart{}\textsc{\textcolor{pink}{Salzburg}{}\ledrightnote{\textcolor{pink}{Salzburg}}}\pend{}\pstart{}\textsc{post restante.}\pend{}{\bigskip}\pstart
           \raggedleft{}{\pb}27. 6. 96. \pend
           \pstart
           Lieber Richard, ich bin hier noch bis zum 2. Juli
               für Briefe anzutreffen. Ich notire Ihnen hier gleich die Daten, wann u. wohin Sie
               event. \uuline{Telegramm} abzuſenden haben:\pend
           \pstart
           am 6. Juli nach \textcolor{pink}{Hamburg}{}\ledrightnote{\textcolor{pink}{Hamburg}}\pend
           \pstart
           am 9. Juli nach \textcolor{pink}{Bergen (Norwegen)}{}\ledrightnote{\textcolor{pink}{Bergen}}\pend
           \pstart
           am 14. Juli nach \textcolor{pink}{Trondjhem}{}\ledrightnote{\textcolor{pink}{Trondheim}}\pend
           \pstart
           am 23. Juli nach \textcolor{pink}{Trondjhem}{}\ledrightnote{\textcolor{pink}{Trondheim}}\pend
           \pstart
           am 25. Juli nach \textcolor{pink}{Kriſtiania}{}\ledrightnote{\textcolor{pink}{Oslo}}.\pend
           \pstart
           Briefe, wiſſen Sie ja. –\pend
           \pstart
           Wünſch Ihnen gute Sti{\geminationm}ung und hoffe häufige Nachrichten.
               Grüßen Sie \textcolor{blue}{Paula}{}\ledrightnote{\textcolor{blue}{Paula Beer-Hofmann}}. Herzlich der Ihre
                  \spacefill\mbox{Arthur}\pend
           \pstart
           \noindent{}\label{T_L00555_1v}\edtext{\textcolor{blue}{Brahm}{}\ledrightnote{\textcolor{blue}{Otto Brahm}} läßt Sie grüßen.}{\lemma{\textnormal{\emph{Brahm läßt Sie grüßen.}}}\Cendnote{\textnormal{quer am rechten Rand}}}\label{T_L00555_1h}\pend
           \endnumbering\briefempfaengerindex{Beer-Hofmann, Richard@\textsc{Beer-Hofmann, Richard}!zzzSchnitzler, Arthur@\emph{von Arthur Schnitzler}!1896-06-271@{27. 6. 1896}|)be}\mylabel{h}  \normalsize

\doendnotes{C}
\bigskip
\vfill

\clearpage

\footnotesize

\lohead{\textsc{register}}

% Definiere theindex-Environment komplett neu ohne reledmac
\makeatletter
\renewenvironment{theindex}{%
  \section*{\indexname}%
  \setlength{\parindent}{0pt}%
  \setlength{\parskip}{0pt plus 0.3pt}%
  \let\item\@idxitem
}{%
  \clearpage
}
\makeatother

\IfFileExists{\jobname-pw.ind}{\input{\jobname-pw.ind}}{}

\end{document}

      