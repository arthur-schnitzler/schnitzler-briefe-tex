%% latex-korrekturansicht-vorspann.tex
%% Vorspann für die Korrekturansicht.
%% Lädt die gemeinsame Datei latex-vorspann.tex mit gesetztem Schalter.

\newif\ifkorrekturansicht
\korrekturansichttrue

\input{../tex-inputs/latex-vorspann}


\renewcommand{\erwaehntePersonen}{Personen:  ?? [Frankfurter Oberstaatsanwalt],  ?? [Frau eines Frankfurter Oberstaatsanwalts], Olga Schnitzler}
\renewcommand{\erwaehnteOrte}{Orte: Frankfurt am Main, Grand Hotel Lavarone, Lavarone, Palast Hotel Lido, Riva del Garda, Südtirol, Trentino-Alto Adige, Wien}
\renewcommand{\erwaehnteWerke}{}
\section[ Paul Goldmann und Theodore Rottenberg an Arthur Schnitzler, 29. 8. 1903]{Paul Goldmann und Theodore Rottenberg an Arthur
               Schnitzler, 29. 8. 1903}
\nopagebreak\mylabel{v}
\rehead{ }\normalsize\beginnumbering\briefempfaengerindex{Schnitzler, Arthur@\textsc{Schnitzler, Arthur}!zzzRottenberg, Theodore@\emph{von Theodore Rottenberg}!1903-08-291@{29. 8. 1903}|(be}\briefempfaengerindex{Schnitzler, Arthur@\textsc{Schnitzler, Arthur}!zzzGoldmann, Paul@\emph{von Paul Goldmann}!1903-08-291@{29. 8. 1903}|(be}
\toendnotes[C]{\smallbreak\pagebreak[2]}\Standort{DLA, A:Schnitzler, HS.NZ85.1.3173.}
\physDesc{Brief, 1 Blatt, 2 Seiten
\newline{}Handschrift Paul Goldmann: schwarze Tinte, deutsche Kurrent
\newline{}Handschrift Theodore Rottenberg: schwarze Tinte, deutsche Kurrent
\newline{}Schnitzler: mit Bleistift das Jahr »{[}1{]}903« vermerkt }\toendnotes[C]{\smallbreak}
\pstart
           \noindent{}{\pb}\textcolor{gray}{\textbf{\textcolor{pink}{Grand Hôtel Lavarone}{}\ledrightnote{\textcolor{pink}{Grand Hotel Lavarone}}}}\hfill \textcolor{gray}{\textbf{\begin{otherlanguage}{italian}Li\end{otherlanguage}}}{ }29. August \textcolor{gray}{\textbf{190}}3\pend
           
\pstart
           \textcolor{gray}{\textbf{\textcolor{pink}{Lavarone}{}\ledrightnote{\textcolor{pink}{Lavarone}} (\textcolor{pink}{Trentino}{}\ledrightnote{\textcolor{pink}{Trentino-Alto Adige}})}}{ }{\\}\textcolor{gray}{\textbf{m. 1200}}\pend
           
\pstart
           \textcolor{gray}{\textbf{\label{K_L03385-1v}\edtext{\begin{otherlanguage}{italian}Stessa
                        Direzione\end{otherlanguage}}{\lemma{\textnormal{\emph{Stessa
                        Direzione}}}\Cendnote{\textnormal{italienisch:
                        gleiche Direktion}}}\label{K_L03385-1h}:}}{ }{\\}\textcolor{gray}{\textbf{\textcolor{pink}{Palast Hôtel Lido}{}\ledrightnote{\textcolor{pink}{Palast Hotel Lido}}{ }\textcolor{pink}{Riva}{}\ledrightnote{\textcolor{pink}{Riva del Garda}} (\begin{otherlanguage}{italian}\textcolor{pink}{Lago di Garda}{}\ledrightnote{\textcolor{pink}{Riva del Garda}}\end{otherlanguage})}}\pend
           
\pstart
           \textcolor{gray}{\textbf{Telegrammi: \textcolor{pink}{Grandhôtel – Lavarone}{}\ledrightnote{\textcolor{pink}{Grand Hotel Lavarone}}}}\pend
           
\pstart\center{}Mein lieber Freund,\pend
\pstart
           Ich beglückwünſche Dich und Deine \textcolor{blue}{Frau}{}\ledrightnote{{$\rightarrow$}\textcolor{blue}{Olga Schnitzler}} auf das Herzlichſte zu Eurer \label{K_L03385-2v}\edtext{Vermählung}{\lemma{\textnormal{\emph{Vermählung}}}\Cendnote{\textnormal{\textcolor{blue}{Schnitzler} und \textcolor{blue}{Olga Gussmann, nun Schnitzler}, hatten am 26. 8. 1903
                  geheiratet.}}}\label{K_L03385-2h}. Ich habe mich ſehr über dieſe Nachricht gefreut und wünſche
               Euch Beiden viele glückliche Jahre.\pend
           
\pstart
           Hier in \textsc{\textcolor{pink}{Lavarone}{}\ledrightnote{\textcolor{pink}{Lavarone}}} iſt bisher Alles gut \strikeout{gelauf} verlaufen. Ein
               herrlicher Aufenthalt. Wir haben fleißig das \textcolor{pink}{Land}{}\ledrightnote{{$\rightarrow$}\textcolor{pink}{Südtirol}} durchſtreift. Der \label{K_L03385-3v}\edtext{\textcolor{pink}{Frankfurt}{}\ledrightnote{\textcolor{pink}{Frankfurt am Main}}er \textcolor{blue}{Oberſtaatsanwalt}{}\ledrightnote{{$\rightarrow$}\textcolor{blue}{?? [Frankfurter Oberstaatsanwalt]}}}{\lemma{\textnormal{\emph{Frankfurter Oberſtaatsanwalt}}}\Cendnote{\textnormal{nicht ermittelt}}}\label{K_L03385-3h} iſt ein
               freundlicher Mann. Aber ſeine \textcolor{blue}{Frau}{}\ledrightnote{{$\rightarrow$}\textcolor{blue}{?? [Frau eines Frankfurter Oberstaatsanwalts]}} ignorirt mich, offenbar aus ſittlicher Entrüſtung.\pend
           
\pstart
           Dich haben wir am Tage nach Deiner \label{K_L03385-5v}\edtext{Abreiſe}{\lemma{\textnormal{\emph{Abreiſe}}}\Cendnote{\textnormal{\textcolor{blue}{Schnitzler} war von 20. 8. 1903 bis 21. 8. 1903 in \textcolor{pink}{Lavarone} gewesen und am 22. 8. 1903 wieder in
                     \textcolor{pink}{Wien} angekommen.}}}\label{K_L03385-5h} ſehr vermißt.
               Hätteſt wirklich noch ein paar Tage bleiben ſollen.\pend
           
\pstart
           Seit zwei Tagen iſt das \textsc{Idyll} geſtört. Ich habe mir bei
               meinem Auſflug die \introOben{}rechte\introOben{} Ferſe verletzt (bin mit dem Abſatz
               auf einen Stein geſprungen und habe mir offenbar eine Concuſion des Knochens
               zugetragen.) Nun kann ich nicht mehr {\pb}gehen, muß im
                  \textsc{\textcolor{pink}{Hotel}{}\ledrightnote{{$\rightarrow$}\textcolor{pink}{Grand Hotel Lavarone}}}{ }\strikeout{ſtil\textcolor{gray}{ſ}} ſtillſitzen, – ſie des gleichen. Und das iſt recht traurig.\pend
           
\pstart
           Immerhin, wir bleiben wohl noch acht Tage, wenn auch nur ſitzend. Schreibe noch
               einmal hierher. Du machſt uns eine große Freude damit.\pend
           
\pstart
           Viele herzliche Grüße an Dich und Deine \textcolor{blue}{Frau}{}\ledrightnote{{$\rightarrow$}\textcolor{blue}{Olga Schnitzler}}! {\\[\baselineskip]}Dein {\\[\baselineskip]}\spacefill\mbox{Paul Goldmann}\pend
           \leftskip=0em{}
\pstart
           \noindent{}{[}hs. Rottenberg:{]} Ebenfalls herzliche Glückwünſche zur Vermählung {\kaufmannsund} beſte Grüße Ihnen {\kaufmannsund}
                  Ihrer Frau \textcolor{blue}{Gemahlin}{}\ledrightnote{\textcolor{blue}{Olga Schnitzler}}. –\pend
           \endnumbering\briefempfaengerindex{Schnitzler, Arthur@\textsc{Schnitzler, Arthur}!zzzRottenberg, Theodore@\emph{von Theodore Rottenberg}!1903-08-291@{29. 8. 1903}|)be}\briefempfaengerindex{Schnitzler, Arthur@\textsc{Schnitzler, Arthur}!zzzGoldmann, Paul@\emph{von Paul Goldmann}!1903-08-291@{29. 8. 1903}|)be}\mylabel{h}
\begin{anhang}
\end{anhang}\normalsize

\doendnotes{C}
\bigskip
\vfill

\clearpage

\footnotesize

\lohead{\textsc{register}}

% Definiere theindex-Environment komplett neu ohne reledmac
\makeatletter
\renewenvironment{theindex}{%
  \section*{\indexname}%
  \setlength{\parindent}{0pt}%
  \setlength{\parskip}{0pt plus 0.3pt}%
  \let\item\@idxitem
}{%
  \clearpage
}
\makeatother

\IfFileExists{\jobname-pw.ind}{\input{\jobname-pw.ind}}{}

\end{document}

      