%% latex-korrekturansicht-vorspann.tex
%% Vorspann für die Korrekturansicht.
%% Lädt die gemeinsame Datei latex-vorspann.tex mit gesetztem Schalter.

\newif\ifkorrekturansicht
\korrekturansichttrue

\input{../tex-inputs/latex-vorspann}


\renewcommand{\erwaehntePersonen}{Personen: Hermann Bahr, Marie Bardi, Arnold Barkay, Carl Michael Bellman, Otto Brahm, Emerich von Bukovics, Hugo Felix, Hugo von Hofmannsthal, Gertrude von Hofmannsthal, Irene Mandl, Ottilie Salten, Olga Schnitzler, Louise Schnitzler, Sven Scholander}
\renewcommand{\erwaehnteInstitutionen}{Institutionen: Jung-Wiener Theater zum Lieben Augustin}
\renewcommand{\erwaehnteOrte}{Orte: Bad Aussee, Frankfurt am Main, Jung-Wiener Theater zum Lieben Augustin, Karlsbad, Köln, München, Paris, Prag, Salzburg, Salzkammergut, Schweden, Stuttgart, Theater an der Wien, Volkstheater, Wien, Wiesbaden, Wörthersee, Zürich}
\renewcommand{\erwaehnteWerke}{Werke: Der Gemeine. Schauspiel in drei Aufzügen, Der einsame Weg. Schauspiel in fünf Akten, Zum großen Wurstel. Burleske in einem Akt}
\section[ Felix Salten an Arthur Schnitzler, 12. 6. 1901]{Felix Salten an Arthur Schnitzler, 12. 6. 1901}
\nopagebreak\mylabel{v}
\rehead{ }\normalsize\beginnumbering\briefempfaengerindex{Schnitzler, Arthur@\textsc{Schnitzler, Arthur}!zzzSalten, Felix@\emph{von Felix Salten}!1901-06-121@{12. 6. 1901}|(be}
\toendnotes[C]{\smallbreak\pagebreak[2]}\Standort{CUL, Schnitzler, B 89, A 2.}
\physDesc{Brief, 1 Blatt, 2 Seiten, 1702 Zeichen
\newline{}Handschrift: schwarze Tinte, lateinische Kurrent
\newline{}Ordnung: mit Bleistift von unbekannter Hand nummeriert: »137« }
\buchAbdrucke{\weitereDrucke{Hermann Bahr, Arthur Schnitzler: \emph{Briefwechsel, Aufzeichnungen, Dokumente (1891–1931)}. Hg. Kurt Ifkovits und Martin Anton Müller. Göttingen: \emph{Wallstein} 2018, S. 204–205.} }\toendnotes[C]{\smallbreak}
\pstart
           \noindent{}{\pb}\textcolor{gray}{\textbf{\textcolor{brown}{Jung-Wiener Theater}{}\ledrightnote{\textcolor{brown}{Jung-Wiener Theater zum Lieben Augustin}}}}\hfill \textcolor{gray}{\textbf{\textcolor{pink}{Wien}{}\ledrightnote{\textcolor{pink}{Wien}},}}{ }12. Juni \textcolor{gray}{\textbf{190}}1\pend
           
\pstart
           \textcolor{gray}{\textbf{\textcolor{brown}{Zum lieben Augustin}{}\ledrightnote{\textcolor{brown}{Jung-Wiener Theater zum Lieben Augustin}}.}}\hfill \textcolor{gray}{\textbf{(\textcolor{pink}{Theater a. d.
                        Wien}{}\ledrightnote{\textcolor{pink}{Theater an der Wien}})}}\pend
           
\pstart
           \textcolor{gray}{\textbf{Direction.}}\pend
           
\pstart
           Lieber Freund, es thut mir leid, dass ich Sie nicht mehr gesprochen
               habe. Bis Sonntag war ich verreist, \textcolor{pink}{Karlsbad}{}\ledrightnote{\textcolor{pink}{Karlsbad}}{ }\textcolor{pink}{Prag}{}\ledrightnote{\textcolor{pink}{Prag}}. Habe in \textcolor{pink}{Prag}{}\ledrightnote{\textcolor{pink}{Prag}} Frl. \label{K_L03313-1v}\edtext{\textcolor{blue}{Bardi}{}\ledrightnote{\textcolor{blue}{Marie Bardi}} und einen hübschen jungen \textcolor{blue}{Tenor}{}\ledrightnote{{$\rightarrow$}\textcolor{blue}{Arnold Barkay}} engagirt}{\lemma{\textnormal{\emph{Bardi … engagirt}}}\Cendnote{\textnormal{\textcolor{blue}{Marie Bardi} und \textcolor{blue}{Arnold Barkay}; das Engagement galt \textcolor{blue}{Salten}s \emph{\textcolor{brown}{Jung-Wiener Theater
                     zum lieben Augustin}}}}}\label{K_L03313-1h}, der die größte Ambition hat, ein \label{K_L03313-2v}\edtext{\textcolor{blue}{Sven Skolander}{}\ledrightnote{\textcolor{blue}{Sven Scholander}}}{\lemma{\textnormal{\emph{Sven Skolander}}}\Cendnote{\textnormal{\textcolor{blue}{Sven Scholander} war ein erfolgreicher \textcolor{pink}{schwed}ischer Sänger, der
                  bei seinen Konzerten vor allem \textcolor{blue}{Carl Michael
                     Bellman} darbot.}}}\label{K_L03313-2h} zu werden. Von D\textsuperscript{r} \textcolor{blue}{Mandl}{}\ledrightnote{\textcolor{blue}{Irene Mandl}} haben Sie gehört, dass \label{K_L03313-3v}\edtext{\textcolor{blue}{Otti}{}\ledrightnote{\textcolor{blue}{Ottilie Salten}} operirt}{\lemma{\textnormal{\emph{Otti operirt}}}\Cendnote{\textnormal{Grund nicht ermittelt}}}\label{K_L03313-3h} wurde. Das war ziemlich
               schrecklich, obwol die ganze Sache an sich ja nichts bedeutet und glücklich verlaufen
               ist. Ich bleibe nun ungefähr acht Tage in \textcolor{pink}{Wien}{}\ledrightnote{\textcolor{pink}{Wien}} und
               fahre dann nach \textcolor{pink}{München}{}\ledrightnote{\textcolor{pink}{München}}, zwei Tage, von dort
               nach \textcolor{pink}{Zürich}{}\ledrightnote{\textcolor{pink}{Zürich}}, drei Tage, (\label{K_L03313-4v}\edtext{\textcolor{blue}{Felix}{}\ledrightnote{\textcolor{blue}{Hugo Felix}}}{\lemma{\textnormal{\emph{Felix}}}\Cendnote{\textnormal{\textcolor{blue}{Hugo Felix}}}}\label{K_L03313-4h}) von da nach \textcolor{pink}{Paris}{}\ledrightnote{\textcolor{pink}{Paris}}, zwölf-14 Tage und
               d’dann nach \textcolor{pink}{Köln}{}\ledrightnote{\textcolor{pink}{Köln}}, \textcolor{pink}{Frankfurt}{}\ledrightnote{\textcolor{pink}{Frankfurt am Main}}, \textcolor{pink}{Wiesbaden}{}\ledrightnote{\textcolor{pink}{Wiesbaden}}, \textcolor{pink}{Stuttgart}{}\ledrightnote{\textcolor{pink}{Stuttgart}} – \textcolor{pink}{Wien}{}\ledrightnote{\textcolor{pink}{Wien}}. Im Juli werde ich im \textcolor{pink}{Salzkammergut}{}\ledrightnote{\textcolor{pink}{Salzkammergut}} oder am \textcolor{pink}{Wörthersee}{}\ledrightnote{\textcolor{pink}{Wörthersee}} sein.
               Auch zu einer kleinen Radtour wäre ich bereit. Den größten Theil des August bin ich in \textcolor{pink}{Wien}{}\ledrightnote{\textcolor{pink}{Wien}},
               mit Ausnahme einer kurzen Reise nach \textcolor{pink}{Prag}{}\ledrightnote{\textcolor{pink}{Prag}} und
               nach \textcolor{pink}{Aussee}{}\ledrightnote{\textcolor{pink}{Bad Aussee}}. Das ist Alles. Ich freue mich, dass
               Sie ein neues \label{K_L03313-5v}\edtext{\textcolor{green}{Stück}{}\ledrightnote{\textcolor{green}{Der einsame Weg. Schauspiel in fünf Akten}}}{\lemma{\textnormal{\emph{Stück}}}\Cendnote{\textnormal{siehe Arthur Schnitzler an Felix Salten, [10. 6. 1901?]}}}\label{K_L03313-5h} haben, und hege künstlerisch eine ganz bestimmte Erwartung davon. Vielleicht
               läßt es sich machen, das{[}s{]}{ }\label{K_L03313-6v}\edtext{\textcolor{blue}{Bukovics}{}\ledrightnote{\textcolor{blue}{Emerich von Bukovics}} mir die »\textcolor{green}{Marionetten}{}\ledrightnote{\textcolor{green}{Zum großen Wurstel. Burleske in einem Akt}}« abtritt}{\lemma{\textnormal{\emph{Bukovics … abtritt}}}\Cendnote{\textnormal{Die \textcolor{green}{Burleske} wurde weder am \textcolor{pink}{Volkstheater}
                  noch am \textcolor{pink}{Jung-Wiener Theater zum lieben Augustin}
                  gegeben.}}}\label{K_L03313-6h}, d. h. wenn Sie mir das \textcolor{green}{Stück}{}\ledrightnote{{$\rightarrow$}\textcolor{green}{Zum großen Wurstel. Burleske in einem Akt}} geben wollen. Schrei{\pb}ben Sie mir darüber. \textcolor{blue}{Brahm}{}\ledrightnote{\textcolor{blue}{Otto Brahm}} ist, wie Sie wissen, \textcolor{pink}{hier}{}\ledrightnote{{$\rightarrow$}\textcolor{pink}{Wien}}. Wir sahen uns im Theater, ohne uns zu
               grüßen. Es ist mir ja sonst ganz gleichgiltig, aber ich bereue jetzt, dass ich mich
                  \label{K_L03313-7v}\edtext{s. Z.}{\lemma{\textnormal{\emph{s. Z.}}}\Cendnote{\textnormal{seiner Zeit}}}\label{K_L03313-7h} doch habe bereden laßen, ihm mein \label{K_L03313-8v}\edtext{\textcolor{green}{Stück}{}\ledrightnote{\textcolor{green}{Der Gemeine. Schauspiel in drei Aufzügen}}}{\lemma{\textnormal{\emph{Stück}}}\Cendnote{\textnormal{\emph{\textcolor{green}{Der Gemeine}}}}}\label{K_L03313-8h} einzureichen. Nun bringt er mich durch sein Benehmen in den peinlichen
               Verdacht, als sei ich ihm \uline{deshalb} böse. Ich bin ihm
               aber garnicht böse, am wenigsten deshalb. Nur sehe ich keine Ursache, sein
               unfreundliches Verhalten einzustecken.\pend
           
\pstart
           Von \textcolor{blue}{Bahr}{}\ledrightnote{\textcolor{blue}{Hermann Bahr}} erfuhr ich, dass \label{K_L03313-9v}\edtext{\textcolor{blue}{Hofmannsthal}{}\ledrightnote{\textcolor{blue}{Hugo von Hofmannsthal}}{ }Samstag geheirathet}{\lemma{\textnormal{\emph{Hofmannsthal … geheirathet}}}\Cendnote{\textnormal{\textcolor{blue}{Hugo von Hofmannsthal} hatte am 8. 6. 1901{ }\textcolor{blue}{Gertrude (Gerty) Schlesinger}
                  geheiratet.}}}\label{K_L03313-9h} hat. Schreiben Sie mir, bitte, bald. Hauptsächlich, wohin Sie
               reisen. Ich habe das \label{K_L03313-10v}\edtext{»wir«}{\lemma{\textnormal{\emph{»wir«}}}\Cendnote{\textnormal{\textcolor{blue}{Schnitzler} war mit \textcolor{blue}{Olga Gussmann} verreist.}}}\label{K_L03313-10h} nicht verstanden. Sind Sie
               mit Ihrer \textcolor{blue}{Mama}{}\ledrightnote{{$\rightarrow$}\textcolor{blue}{Louise Schnitzler}}?\pend
           
\pstart
           herzlichst {\\[\baselineskip]}Ihr {\\[\baselineskip]}\spacefill\mbox{Salten}\pend
           \leftskip=0em{}\endnumbering\briefempfaengerindex{Schnitzler, Arthur@\textsc{Schnitzler, Arthur}!zzzSalten, Felix@\emph{von Felix Salten}!1901-06-121@{12. 6. 1901}|)be}\mylabel{h}  \normalsize

\doendnotes{C}
\bigskip
\vfill

\clearpage

\footnotesize

\lohead{\textsc{register}}

% Definiere theindex-Environment komplett neu ohne reledmac
\makeatletter
\renewenvironment{theindex}{%
  \section*{\indexname}%
  \setlength{\parindent}{0pt}%
  \setlength{\parskip}{0pt plus 0.3pt}%
  \let\item\@idxitem
}{%
  \clearpage
}
\makeatother

\IfFileExists{\jobname-pw.ind}{\input{\jobname-pw.ind}}{}

\end{document}

      