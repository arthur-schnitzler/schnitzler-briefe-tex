%% latex-korrekturansicht-vorspann.tex
%% Vorspann für die Korrekturansicht.
%% Lädt die gemeinsame Datei latex-vorspann.tex mit gesetztem Schalter.

\newif\ifkorrekturansicht
\korrekturansichttrue

\input{../tex-inputs/latex-vorspann}


\renewcommand{\erwaehntePersonen}{Personen: Eva Marie Goldmann, Olga Schnitzler}
\renewcommand{\erwaehnteOrte}{Orte: Berlin, Hotel Sacher, Wien}
\renewcommand{\erwaehnteWerke}{}
\section[ Paul Goldmann an Arthur Schnitzler, 30. 12. 1910]{Paul Goldmann an Arthur Schnitzler, 30. 12. 1910}
\nopagebreak\mylabel{v}
\rehead{ }\normalsize\beginnumbering\briefempfaengerindex{Schnitzler, Arthur@\textsc{Schnitzler, Arthur}!zzzGoldmann, Paul@\emph{von Paul Goldmann}!1910-12-301@{30. 12. 1910}|(be}
\toendnotes[C]{\smallbreak\pagebreak[2]}\Standort{DLA, A:Schnitzler, HS.NZ85.1.3175.}
\physDesc{Brief, 1 Blatt, 1 Seite, 437 Zeichen
\newline{}Handschrift: schwarze Tinte, deutsche Kurrent}\toendnotes[C]{\smallbreak}
\pstart
           \noindent{}\centering{}{\pb}\textcolor{gray}{\textbf{\textcolor{pink}{Hotel Sacher}{}\ledrightnote{\textcolor{pink}{Hotel Sacher}}}}\pend
           
\pstart
           \noindent{}\textcolor{gray}{\textbf{Telefon Nr 8008.}}\hfill \textcolor{gray}{\textbf{\textcolor{pink}{Wien I.}{}\ledrightnote{\textcolor{pink}{Wien}}}}\pend
           
\pstart
           30. 12. 10. \hfill Lieber Freund,\pend
           
\pstart
           Ich danke Dir herzlich für die Überſendung der \label{K_L03472-1v}\edtext{Kopien meiner Briefe}{\lemma{\textnormal{\emph{Kopien meiner Briefe}}}\Cendnote{\textnormal{Eine vollständige Abschrift der Korrespondenz ist nicht
                  überliefert. \textcolor{blue}{Goldmann}s Briefen aus dem
                  Jahr 1900 sind eine mit Schreibmaschine erstellte
                     Abschrift einzelner Briefstellen desselben Jahres beigelegt (\emph{DLA Marbach}, HS.1985.1.3170, 2 Durchschläge). 
                     Dass diese 9 Seiten hier gemeint sind, wäre zumindest naheliegend, da die Ausschnitte
                     sich auf Werkaussagen konzentrieren. Die Ausschnitte sind folgenden Briefen \textcolor{blue}{Goldmann}s entnommen: 11. 2. 1900, 21. 6. [1900],
                     19. 9. [1900] und 14. 10. [1900],
                     sowie ein Zitat aus der Beilage des Schreibens vom 9. 12. [1900].}}}\label{K_L03472-1h}. Nun bitte ich nur noch um die Erlaubis, ſie nach \textcolor{pink}{Berlin}{}\ledrightnote{\textcolor{pink}{Berlin}} mitzunehmen u. dort meiner \textcolor{blue}{Frau}{}\ledrightnote{{$\rightarrow$}\textcolor{blue}{Eva Marie Goldmann}} zu zeigen. Von \textcolor{pink}{Berlin}{}\ledrightnote{\textcolor{pink}{Berlin}} werde ich ſie Dir zurückſchicken u. Dir zugleich ein
               abſchließendes Wort über die \label{K_L03472-2v}\edtext{letzte
                  Unterredung}{\lemma{\textnormal{\emph{letzte
                  Unterredung}}}\Cendnote{\textnormal{siehe Paul Goldmann an Arthur Schnitzler, 26. 12. 1910}}}\label{K_L03472-2h} ſchreiben, die doch mehr in mir nachwirkt, als ich es gewünſcht hätte. – Mit
               herzlichen Grüßen an Deine \textcolor{blue}{Frau}{}\ledrightnote{{$\rightarrow$}\textcolor{blue}{Olga Schnitzler}} u. Dich bin ich Dein \spacefill\mbox{Paul Goldmann.}\pend
           \endnumbering\briefempfaengerindex{Schnitzler, Arthur@\textsc{Schnitzler, Arthur}!zzzGoldmann, Paul@\emph{von Paul Goldmann}!1910-12-301@{30. 12. 1910}|)be}\mylabel{h}
\begin{anhang}
\end{anhang}\normalsize

\doendnotes{C}
\bigskip
\vfill

\clearpage

\footnotesize

\lohead{\textsc{register}}

% Definiere theindex-Environment komplett neu ohne reledmac
\makeatletter
\renewenvironment{theindex}{%
  \section*{\indexname}%
  \setlength{\parindent}{0pt}%
  \setlength{\parskip}{0pt plus 0.3pt}%
  \let\item\@idxitem
}{%
  \clearpage
}
\makeatother

\IfFileExists{\jobname-pw.ind}{\input{\jobname-pw.ind}}{}

\end{document}

      