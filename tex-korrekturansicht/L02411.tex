%% latex-korrekturansicht-vorspann.tex
%% Vorspann für die Korrekturansicht.
%% Lädt die gemeinsame Datei latex-vorspann.tex mit gesetztem Schalter.

\newif\ifkorrekturansicht
\korrekturansichttrue

\input{../tex-inputs/latex-vorspann}


               \section[Felix Braun an Arthur Schnitzler, 27. 3. 1924]{ Felix Braun an Arthur Schnitzler, 27. 3. 1924}\nopagebreak\mylabel{v}\rehead{ }\normalsize\beginnumbering\briefempfaengerindex{Schnitzler, Arthur@\textsc{Schnitzler, Arthur}!zzzBraun, Felix@\emph{von Felix Braun}!1924-03-271@{27. 3. 1924}|(be} \toendnotes[C]{\smallbreak\pagebreak[2]} \Standort{DLA, A:Schnitzler, HS.NZ85.1.2604,2.}
\physDesc{Briefkarte
\newline{}Handschrift: schwarze Tinte, deutsche Kurrent
\newline{}Schnitzler: 1) mit Bleistift beschriftet: »\textcolor{pink}{\textsc{Siestr. 191}}« 2) mit rotem Buntstift eine Unterstreichung}\pstart
           \centering{}{\pb}\textcolor{pink}{Wien}{}\ledrightnote{\textcolor{pink}{Wien}}, den 27. III. 1924\pend
           \pstart{}Verehrter Herr Doktor!\pend\pstart
           Erlauben Sie, daß ich Ihnen ein Dankwort ſchreibe für die große
                    Liebenswürdigkeit, mit der Sie, wie mir Frau \textcolor{blue}{Heller}{}\ledrightnote{\textcolor{blue}{Hedwig Heller}} heute zu meiner Freude erzählte, als es ſich um die Zuweiſung
                    des \textcolor{blue}{\textsc{Paul Géraldy}}{}\ledrightnote{\textcolor{blue}{Paul Géraldy}} beſtimmten Honorars an einen \textcolor{pink}{Wien}{}\ledrightnote{\textcolor{pink}{Wien}}er
                    Schriftſteller handelte, für mich eingetreten ſind. Es hat mich tief gerührt,
                    daß Sie es waren, der mir dieſe Ehrung zuerkannt hat. Seien Sie, verehrter Herr
                    Doktor, dafür von Herzen bedankt!\pend
           \pstart
           Mit beſter Empfehlung, in beſonderer Verehrung, Ihr{\\[\baselineskip]}\spacefill\mbox{Felix Braun.}\pend
           \leftskip=0em{}\endnumbering\briefempfaengerindex{Schnitzler, Arthur@\textsc{Schnitzler, Arthur}!zzzBraun, Felix@\emph{von Felix Braun}!1924-03-271@{27. 3. 1924}|)be}\mylabel{h}  \normalsize

\doendnotes{C}
\bigskip
\vfill

\clearpage

\footnotesize

\lohead{\textsc{register}}

% Definiere theindex-Environment komplett neu ohne reledmac
\makeatletter
\renewenvironment{theindex}{%
  \section*{\indexname}%
  \setlength{\parindent}{0pt}%
  \setlength{\parskip}{0pt plus 0.3pt}%
  \let\item\@idxitem
}{%
  \clearpage
}
\makeatother

\IfFileExists{\jobname-pw.ind}{\input{\jobname-pw.ind}}{}

\end{document}

      