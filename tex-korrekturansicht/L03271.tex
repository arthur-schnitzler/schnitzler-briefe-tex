%% latex-korrekturansicht-vorspann.tex
%% Vorspann für die Korrekturansicht.
%% Lädt die gemeinsame Datei latex-vorspann.tex mit gesetztem Schalter.

\newif\ifkorrekturansicht
\korrekturansichttrue

\input{../tex-inputs/latex-vorspann}


\renewcommand{\erwaehntePersonen}{Personen: Hermann Bahr, Max Eugen Burckhard}
\renewcommand{\erwaehnteOrte}{Orte: Bad Ischl, Hotel und Pension Rudolfshöhe (Leopold Petter), IX., Alsergrund, Wien}
\renewcommand{\erwaehnteWerke}{Werke: Theater, Kunst und Literatur [Agnes Jordan nicht am Burgtheater]}
\section[ Felix Salten an Arthur Schnitzler, 23. 7. 1897]{Felix Salten an Arthur Schnitzler, 23. 7. 1897}
\nopagebreak\mylabel{v}
\rehead{ }\normalsize\beginnumbering\briefempfaengerindex{Schnitzler, Arthur@\textsc{Schnitzler, Arthur}!zzzSalten, Felix@\emph{von Felix Salten}!1897-07-232@{23. 7. 1897}|(be}
\toendnotes[C]{\smallbreak\pagebreak[2]}\Standort{CUL, Schnitzler, B 89, A 2.}
\physDesc{Postkarte, 169 Zeichen
\newline{}Handschrift: Bleistift, lateinische Kurrent
\newline{}Versand: Stempel: »\nobreak{}\oindex{IX., Alsergrund@\textbf{IX., Alsergrund}, \emph{A.ADM3}|pwk}Wien 9/3 72, 23 7. 97, 4–5N\nobreak{}«. Stempel: »\nobreak{}\oindex{Bad Ischl@\textbf{Bad Ischl}, \emph{P.PPL}|pwk}Isch\textcolor{gray}{l}, 24/7 97, 7–\textcolor{gray}{8}\nobreak{}«.  
\newline{}Schnitzler: mit Bleistift datiert: »23/7« 
\newline{}Ordnung: mit Bleistift von unbekannter Hand nummeriert: »94« }
\buchAbdrucke{\weitereDrucke{Hermann Bahr, Arthur Schnitzler: \emph{Briefwechsel, Aufzeichnungen, Dokumente (1891–1931)}. Hg. Kurt Ifkovits und Martin Anton Müller. Göttingen: \emph{Wallstein} 2018, S. 150.} }\toendnotes[C]{\smallbreak}\pstart{}{\pb}Herrn D\textsuperscript{r} Arthur Schnitzler\pend{}\pstart{}\textcolor{pink}{\strikeout{Wien}}{}\ledrightnote{\textcolor{pink}{Wien}}{ }\textcolor{pink}{Ischl}{}\ledrightnote{\textcolor{pink}{Bad Ischl}}\pend{}\pstart{}\textcolor{pink}{Kaltenbach, Pension Petter}{}\ledrightnote{\textcolor{pink}{Hotel und Pension Rudolfshöhe (Leopold Petter)}}.\pend{}
{\bigskip}
\pstart
           \noindent{}{\pb}Heute hab ich die Quelle jener \label{K_L03271-1v}\edtext{\textcolor{green}{Nachricht}{}\ledrightnote{{$\rightarrow$}\textcolor{green}{Theater, Kunst und Literatur [Agnes Jordan nicht am Burgtheater]}}}{\lemma{\textnormal{\emph{Nachricht}}}\Cendnote{\textnormal{siehe Felix Salten an Arthur Schnitzler, 22. 7. 1897}}}\label{K_L03271-1h} erfahren. – \label{K_L03271-2v}\edtext{\textcolor{blue}{B.}{}\ledrightnote{\textcolor{blue}{Hermann Bahr}}}{\lemma{\textnormal{\emph{B.}}}\Cendnote{\textnormal{Auch wenn sich das Initial auch 
                  auf \textcolor{blue}{Max Burckhard} beziehen könnte, wird
                  durch die Vorgeschichte (siehe Felix Salten an Arthur Schnitzler, 17. 7. 1897) deutlich,
                  dass \textcolor{blue}{Hermann Bahr} als der Fädenzieher
                  im Hintergrund betrachtet wird, von dem man sich eine solche Information an 
                  die Presse erwartete.}}}\label{K_L03271-2h}\pend
           
\pstart
           Das hätte \substVorne{}\textsuperscript{ich}\substDazwischen{}m\substHinten{}an sich eigentlich denken können.\pend
           \pstart Herzlich \spacefill\mbox{S.}\pend{}\endnumbering\briefempfaengerindex{Schnitzler, Arthur@\textsc{Schnitzler, Arthur}!zzzSalten, Felix@\emph{von Felix Salten}!1897-07-232@{23. 7. 1897}|)be}\mylabel{h}  \normalsize

\doendnotes{C}
\bigskip
\vfill

\clearpage

\footnotesize

\lohead{\textsc{register}}

% Definiere theindex-Environment komplett neu ohne reledmac
\makeatletter
\renewenvironment{theindex}{%
  \section*{\indexname}%
  \setlength{\parindent}{0pt}%
  \setlength{\parskip}{0pt plus 0.3pt}%
  \let\item\@idxitem
}{%
  \clearpage
}
\makeatother

\IfFileExists{\jobname-pw.ind}{\input{\jobname-pw.ind}}{}

\end{document}

      