%% latex-korrekturansicht-vorspann.tex
%% Vorspann für die Korrekturansicht.
%% Lädt die gemeinsame Datei latex-vorspann.tex mit gesetztem Schalter.

\newif\ifkorrekturansicht
\korrekturansichttrue

\input{../tex-inputs/latex-vorspann}


\renewcommand{\erwaehntePersonen}{Personen: Felix Salten, Olga Schnitzler, Heinrich Schnitzler}
\renewcommand{\erwaehnteOrte}{Orte: Berlin, Charlottenburg, Sternwartestraße 71, Wien, Zoologischer Garten Berlin}
\renewcommand{\erwaehnteWerke}{}
\section[ Felix Salten an Arthur und Olga Schnitzler, 7. 9. 1913]{Felix Salten an Arthur und Olga Schnitzler, 7. 9. 1913}
\nopagebreak\mylabel{v}
\rehead{ }\normalsize\beginnumbering\briefempfaengerindex{Schnitzler, Olga@\textsc{Schnitzler, Olga}!zzzSalten, Felix@\emph{von Felix Salten}!1913-09-071@{7. 9. 1913}|(be}\briefempfaengerindex{Schnitzler, Arthur@\textsc{Schnitzler, Arthur}!zzzSalten, Felix@\emph{von Felix Salten}!1913-09-071@{7. 9. 1913}|(be}
\toendnotes[C]{\smallbreak\pagebreak[2]}\Standort{CUL, Schnitzler, B 89, B 2.}
\physDesc{Bildpostkarte, 307 Zeichen
\newline{}Handschrift: schwarze Tinte, lateinische Kurrent
\newline{}Versand: Stempel: »\nobreak{}\oindex{Charlottenburg@\textbf{Charlottenburg}, \emph{P.PPLX}|pwk}Charlottenburg 2, 8. \textcolor{gray}{9}. 13, 9–10 V\nobreak{}«.  
\newline{}Ordnung: mit Bleistift von unbekannter Hand nummeriert: »275« }\toendnotes[C]{\smallbreak}\pstart{}{\pb}Herrn u. Frau D\textsuperscript{r} Arthur Schnitzler\pend{}\pstart{}\textcolor{pink}{Wien}{}\ledrightnote{\textcolor{pink}{Wien}}\pend{}\pstart{}\textcolor{pink}{XVIII. Sternwartestraße 71}{}\ledrightnote{\textcolor{pink}{Sternwartestraße 71}}\pend{}
{\bigskip}
\pstart
           \noindent{}{\pb}\textcolor{gray}{\textbf{\textcolor{pink}{ZOOLOGISCHER GARTEN BERLIN}{}\ledrightnote{\textcolor{pink}{Zoologischer Garten Berlin}}.}}\hfill \textcolor{gray}{\textbf{Löwe.}}\pend
           
\pstart
           {\pb}Viele herzliche Grüße!\pend
           
\pstart
           Ich bin nun schon eine Woche \textcolor{pink}{hier}{}\ledrightnote{\textcolor{pink}{Berlin}} und bleibe
               voraussichtlich bis gegen den Zwanzigsten. Hoffentlich
               sind Sie alle gesund und \label{K_L03562-1v}\edtext{\textcolor{blue}{Heini}{}\ledrightnote{\textcolor{blue}{Heinrich Schnitzler}} ganz erholt}{\lemma{\textnormal{\emph{Heini ganz erholt}}}\Cendnote{\textnormal{siehe Felix Salten an Arthur Schnitzler, 24. 6. 1913}}}\label{K_L03562-1h}. Auf ein frohes Wiedersehen und alles herzliche Ihnen allen\pend
           \pstart Ihr \spacefill\mbox{Salten}\pend{}
\pstart
           \textcolor{pink}{Berlin}{}\ledrightnote{\textcolor{pink}{Berlin}}, 7. 9. 913\pend
           \endnumbering\briefempfaengerindex{Schnitzler, Olga@\textsc{Schnitzler, Olga}!zzzSalten, Felix@\emph{von Felix Salten}!1913-09-071@{7. 9. 1913}|)be}\briefempfaengerindex{Schnitzler, Arthur@\textsc{Schnitzler, Arthur}!zzzSalten, Felix@\emph{von Felix Salten}!1913-09-071@{7. 9. 1913}|)be}\mylabel{h}  \normalsize

\doendnotes{C}
\bigskip
\vfill

\clearpage

\footnotesize

\lohead{\textsc{register}}

% Definiere theindex-Environment komplett neu ohne reledmac
\makeatletter
\renewenvironment{theindex}{%
  \section*{\indexname}%
  \setlength{\parindent}{0pt}%
  \setlength{\parskip}{0pt plus 0.3pt}%
  \let\item\@idxitem
}{%
  \clearpage
}
\makeatother

\IfFileExists{\jobname-pw.ind}{\input{\jobname-pw.ind}}{}

\end{document}

      