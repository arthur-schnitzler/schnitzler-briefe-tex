%% latex-korrekturansicht-vorspann.tex
%% Vorspann für die Korrekturansicht.
%% Lädt die gemeinsame Datei latex-vorspann.tex mit gesetztem Schalter.

\newif\ifkorrekturansicht
\korrekturansichttrue

\input{../tex-inputs/latex-vorspann}


               \section[Hugo von Hofmannsthal an Arthur Schnitzler, 28. 8. 1903]{ Hugo von Hofmannsthal an Arthur Schnitzler, 28. 8. 1903}\nopagebreak\mylabel{v}\rehead{ }\normalsize\beginnumbering\briefempfaengerindex{Schnitzler, Arthur@\textsc{Schnitzler, Arthur}!zzzHofmannsthal, Hugo von@\emph{von Hugo von Hofmannsthal}!1903-08-281@{28. 8. 1903}|(be} \toendnotes[C]{\smallbreak\pagebreak[2]} \Standort{CUL, Schnitzler, B 43.}
\physDesc{Bildpostkarte
\newline{}Handschrift: schwarze Tinte, deutsche Kurrent\newline{}Versand: Stempel: »\nobreak{}\oindex{Weimar@\textbf{Weimar}, \emph{Besiedelter Ort (A.BSO)}|pwk}Weimar, \textcolor{gray}{28. 08. 03}, 12–1\nobreak{}«.  \newline{}Ordnung: 1) mit Bleistift von unbekannter Hand nummeriert:
                                 »\strikeout{228}« 2) mit Bleistift von unbekannter Hand nummeriert:
                                 »199«}\buchAbdrucke{\weitereDrucke{Hugo von Hofmannsthal, Arthur Schnitzler: \emph{Briefwechsel}. Hg. Therese Nickl und Heinrich Schnitzler. Frankfurt am Main: \emph{S. Fischer} 1964, S. 173.} }\toendnotes[C]{\smallbreak}\pstart{}{\pb}\textsc{Herrn D\textsuperscript{r} Arthur Schnitzler}\pend{}\pstart{}\textcolor{pink}{\textsc{Wien}}{}\ledrightnote{\textcolor{pink}{Wien}}\pend{}\pstart{}\textcolor{pink}{\textsc{IX. Franckgasse 1}.}{}\ledrightnote{\textcolor{pink}{Frankgasse}}\pend{}{\bigskip}\pstart
           \noindent{}\centering{}\textcolor{gray}{\textbf{{\pb}\textcolor{pink}{Weimar}{}\ledrightnote{\textcolor{pink}{Weimar}}, \textcolor{blue}{Gœthe}{}\ledrightnote{\textcolor{blue}{Johann Wolfgang von Goethe}}’s \textcolor{pink}{Gartenhaus}{}\ledrightnote{\textcolor{pink}{Gartenhaus (Goethe)}}.}}\pend
           \pstart
           \raggedleft{}{\pb}28 VIII.\pend
           \pstart
           Eine \textcolor{pink}{weimar}{}\ledrightnote{\textcolor{pink}{Weimar}}iſche \textcolor{blue}{Hofdame}{}\ledrightnote{→\textcolor{blue}{?? [Hofdame]}}: »Ich intereſſiere mich ſehr für einen Landsmann
               von Ihnen, der ſo früh geſtorben iſt und deſſen Werke alle erſt nach ſeinem Tod
               erſchienen ſind: Arthur Schnitzler.« Ich: Ich glaube, Sie irren ſich. Sie: O, besti{\geminationm}t nicht. \pend
           \endnumbering\briefempfaengerindex{Schnitzler, Arthur@\textsc{Schnitzler, Arthur}!zzzHofmannsthal, Hugo von@\emph{von Hugo von Hofmannsthal}!1903-08-281@{28. 8. 1903}|)be}\mylabel{h}  \normalsize

\doendnotes{C}
\bigskip
\vfill

\clearpage

\footnotesize

\lohead{\textsc{register}}

% Definiere theindex-Environment komplett neu ohne reledmac
\makeatletter
\renewenvironment{theindex}{%
  \section*{\indexname}%
  \setlength{\parindent}{0pt}%
  \setlength{\parskip}{0pt plus 0.3pt}%
  \let\item\@idxitem
}{%
  \clearpage
}
\makeatother

\IfFileExists{\jobname-pw.ind}{\input{\jobname-pw.ind}}{}

\end{document}

      