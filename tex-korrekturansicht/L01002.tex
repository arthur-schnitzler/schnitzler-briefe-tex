%% latex-korrekturansicht-vorspann.tex
%% Vorspann für die Korrekturansicht.
%% Lädt die gemeinsame Datei latex-vorspann.tex mit gesetztem Schalter.

\newif\ifkorrekturansicht
\korrekturansichttrue

\input{../tex-inputs/latex-vorspann}


               \section[Arthur Schnitzler an Richard Beer-Hofmann, 28. 11. 1899]{ Arthur Schnitzler an Richard Beer-Hofmann,
               28. 11. 1899}\nopagebreak\mylabel{v}\rehead{ }\normalsize\beginnumbering\briefempfaengerindex{Beer-Hofmann, Richard@\textsc{Beer-Hofmann, Richard}!zzzSchnitzler, Arthur@\emph{von Arthur Schnitzler}!1899-11-281@{28. 11. 1899}|(be} \toendnotes[C]{\smallbreak\pagebreak[2]} \Standort{CUL, Schnitzler, B 8.}
\physDesc{Postkarte
\newline{}Handschrift: Bleistift, deutsche Kurrent\newline{}Versand: Stempel: »\nobreak{}\oindex{I., Innere Stadt@\textbf{I., Innere Stadt}, \emph{Bezirk (A.BZK)}|pwk}Wien 1/1, 28. 11. 99, 10–11N\nobreak{}«.  \newline{}Ordnung: mit Bleistift von unbekannter Hand datiert: »28. 11.« }\pstart{}{\pb}Herrn \textsc{Dr.
                  Richard Beer-Hofmann}\pend{}\pstart{}\textcolor{pink}{Wien}{}\ledrightnote{\textcolor{pink}{Wien}}\pend{}\pstart{}\textsc{\textcolor{pink}{I. Wollzeile 15}{}\ledrightnote{\textcolor{pink}{Wollzeile}}}.\pend{}{\bigskip}\pstart
           \noindent{}{\pb}lieber, unmöglich für den \textcolor{green}{Homburger Prinzen}{}\ledrightnote{\textcolor{green}{Prinz Friedrich von Homburg}} was zu verſchaffen. Ich mir
               ſelber durch \textsc{\textcolor{blue}{Rosenbaum}{}\ledrightnote{\textcolor{blue}{Richard Rosenbaum}}}, der aber nur \uline{einen} verſorgen
               kann. Intendanz-Erlaſs, wegen der Angriffe in den Zeitungen, dſs man an der Kaſſe nie
               was kriegt. Alſo durch Dienſtma{\geminationn} an der Kaſſe noch am eheſten möglich.\pend
           \pstart Herzlich Ihr
                  \spacefill\mbox{Arthur}\pend{}\endnumbering\briefempfaengerindex{Beer-Hofmann, Richard@\textsc{Beer-Hofmann, Richard}!zzzSchnitzler, Arthur@\emph{von Arthur Schnitzler}!1899-11-281@{28. 11. 1899}|)be}\mylabel{h}  \normalsize

\doendnotes{C}
\bigskip
\vfill

\clearpage

\footnotesize

\lohead{\textsc{register}}

% Definiere theindex-Environment komplett neu ohne reledmac
\makeatletter
\renewenvironment{theindex}{%
  \section*{\indexname}%
  \setlength{\parindent}{0pt}%
  \setlength{\parskip}{0pt plus 0.3pt}%
  \let\item\@idxitem
}{%
  \clearpage
}
\makeatother

\IfFileExists{\jobname-pw.ind}{\input{\jobname-pw.ind}}{}

\end{document}

      