%% latex-korrekturansicht-vorspann.tex
%% Vorspann für die Korrekturansicht.
%% Lädt die gemeinsame Datei latex-vorspann.tex mit gesetztem Schalter.

\newif\ifkorrekturansicht
\korrekturansichttrue

\input{../tex-inputs/latex-vorspann}


               \section[Arthur Schnitzler: Widmungsexemplar Lebendige Stunden für Hugo von Hofmannsthal, {[}7.?{]} 1. 1902]{ Arthur Schnitzler: Widmungsexemplar Lebendige Stunden für Hugo von
                    Hofmannsthal, {[}7.?{]} 1. 1902}\nopagebreak\mylabel{v}\rehead{ }\normalsize\beginnumbering\briefempfaengerindex{Hofmannsthal, Hugo von@\textsc{Hofmannsthal, Hugo von}!zzzSchnitzler, Arthur@\emph{von Arthur Schnitzler}!1902-01-071@{{[}7.?{]} 1. 1902}|(be} \toendnotes[C]{\smallbreak\pagebreak[2]} \Standort{FDH, FDH 3236.}
\physDesc{Widmung am Vorsatzblatt
\newline{}Handschrift: schwarze Tinte, deutsche Kurrent}\buchAbdrucke{\weitereDrucke{Hugo von Hofmannsthal: \emph{Bibliothek}. Hg. Ellen Ritter † in Zusammenarbeit mit Dalia Bukauskaité und
                                Konrad Heumann. Frankfurt am Main: \emph{S. Fischer} 2011, S. 604 (Sämtliche Werke. Kritische Ausgabe, XL).} }\toendnotes[C]{\smallbreak}\pstart
           \noindent{}\raggedleft{}{\pb}Meinem lieben Hugo\pend
           \pstart \spacefill\mbox{ArthSch}\pend{}\pstart
           \textcolor{pink}{Wien}{}\ledrightnote{\textcolor{pink}{Wien}}{ }\label{K_L01196_1v}\edtext{Jänner 902}{\lemma{\textnormal{\emph{Jänner 902}}}\Cendnote{\textnormal{am
                        23. 12. 1901 vom \emph{\textcolor{green}{Börsenblatt für den deutschen
                           Buchhandel}} als Neuerscheinung gemeldet. Am 7. 1. 1902 kehrt \textcolor{blue}{Schnitzler} aus \textcolor{pink}{Berlin}
                                zurück. Der Versand der Widmungsexemplare dürfte zu dieser Zeit
                                erfolgt sein.}}}\label{K_L01196_1h}\pend
           {\bigskip}\pstart
           \noindent{}\centering{}{\pb}\textcolor{gray}{\textbf{Arthur Schnitzler}}\pend
           \pstart
           \noindent{}\centering{}\textcolor{gray}{\textbf{\textcolor{green}{\textbf{Lebendige Stunden}}{}\ledrightnote{\textcolor{green}{Lebendige Stunden. Vier Einakter}}}}\pend
           \pstart
           \noindent{}\centering{}\textcolor{gray}{\textbf{Vier Einakter}}\pend
           \pstart
           \noindent{}\centering{}\textcolor{gray}{\textbf{Zweite Auflage}}\pend
           {\bigskip}\pstart
           \noindent{}\centering{}\textcolor{gray}{\textbf{\textcolor{pink}{\so{Berlin}}{}\ledrightnote{\textcolor{pink}{Berlin}}{ }1902}}\pend
           \pstart
           \noindent{}\centering{}\textcolor{gray}{\textbf{\textcolor{brown}{\so{S. Fiſcher, Verlag}}{}\ledrightnote{\textcolor{brown}{S. Fischer Verlag}}}}\pend
           \endnumbering\briefempfaengerindex{Hofmannsthal, Hugo von@\textsc{Hofmannsthal, Hugo von}!zzzSchnitzler, Arthur@\emph{von Arthur Schnitzler}!1902-01-071@{{[}7.?{]} 1. 1902}|)be}\mylabel{h}  \normalsize

\doendnotes{C}
\bigskip
\vfill

\clearpage

\footnotesize

\lohead{\textsc{register}}

% Definiere theindex-Environment komplett neu ohne reledmac
\makeatletter
\renewenvironment{theindex}{%
  \section*{\indexname}%
  \setlength{\parindent}{0pt}%
  \setlength{\parskip}{0pt plus 0.3pt}%
  \let\item\@idxitem
}{%
  \clearpage
}
\makeatother

\IfFileExists{\jobname-pw.ind}{\input{\jobname-pw.ind}}{}

\end{document}

      