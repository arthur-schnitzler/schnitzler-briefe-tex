%% latex-korrekturansicht-vorspann.tex
%% Vorspann für die Korrekturansicht.
%% Lädt die gemeinsame Datei latex-vorspann.tex mit gesetztem Schalter.

\newif\ifkorrekturansicht
\korrekturansichttrue

\input{../tex-inputs/latex-vorspann}


         
         \renewcommand{\erwaehntePersonen}{Personen: Otto Brahm, Auguste Chlum, Anna Donath, Rosa Freudenthal, Hermann Freudenthal, Marie Glümer, Franz Grillparzer, Eduard Hanslick, Josef Kainz, Paul Lindau, Julius Schnitzler, Helene Schnitzler, Adele Schreiber, D. W. Schröder, Irene Triesch}
         \renewcommand{\erwaehnteInstitutionen}{Institutionen: Berliner Theater, Deutsches Theater Berlin, Frankfurter Zeitung, Lessing-Gesellschaft für Kunst und Wissenschaft, Neue Freie Presse, Schauspielhaus Berlin}
         \renewcommand{\erwaehnteOrte}{Orte: Berlin, Breslau, Burgtheater, Deutsches Theater Berlin, Frankfurt am Main, Hotel Saxonia, Potsdamer Platz, Stresemannstraße, Tiergarten, Wien}
         \renewcommand{\erwaehnteWerke}{Werke: Der Schleier der Beatrice. Schauspiel in fünf Akten, [Vortrag über Arthur Schnitzler]}
               \section[ Paul Goldmann an Arthur Schnitzler, 20. 2. 1900]{Paul Goldmann an Arthur Schnitzler, 20. 2. 1900}\nopagebreak\mylabel{v}\rehead{ }\normalsize\beginnumbering\briefempfaengerindex{Schnitzler, Arthur@\textsc{Schnitzler, Arthur}!zzzGoldmann, Paul@\emph{von Paul Goldmann}!1900-02-201@{20. 2. 1900}|(be} \toendnotes[C]{\smallbreak\pagebreak[2]} \Standort{DLA, A:Schnitzler, HS.NZ85.1.3170.}
\physDesc{Brief, 2 Blätter, 7 Seiten
\newline{}Handschrift: schwarze Tinte, deutsche Kurrent
\newline{}Schnitzler: mit rotem Buntstift drei Unterstreichungen }\toendnotes[C]{\smallbreak}\pstart
           \noindent{}\centering{}{\pb}\textcolor{gray}{\textbf{\textbf{\textcolor{pink}{HOTEL SAXONIA}{}\ledrightnote{\textcolor{pink}{Hotel Saxonia}}}}}\pend
           \pstart
           \noindent{}\raggedleft{}\textcolor{gray}{\textbf{am \textcolor{pink}{Potsdamer Platz}{}\ledrightnote{\textcolor{pink}{Potsdamer Platz}} und
                        \textcolor{pink}{Thiergarten}{}\ledrightnote{\textcolor{pink}{Tiergarten}}}}\pend
           \pstart
           \noindent{}\centering{}\textcolor{gray}{\textbf{\textcolor{blue}{D. W. SCHRÖDER}{}\ledrightnote{\textcolor{blue}{D. W. Schröder}}.}}\pend
           \pstart
           \noindent{}\textcolor{gray}{\textbf{Fernsprecher:}}\pend
           \pstart
           \textcolor{gray}{\textbf{\textbf{Amt VI. No. 2838.}}}\pend
           \pstart
           \raggedleft{}\textcolor{gray}{\textbf{\textcolor{pink}{BERLIN W.}{}\ledrightnote{\textcolor{pink}{Berlin}}, den}}{ }20. Februar \textcolor{gray}{\textbf{1}}900. \pend
           \pstart
           \raggedleft{}\textcolor{gray}{\textbf{\textcolor{pink}{Königgrätzerstrasse 10}{}\ledrightnote{\textcolor{pink}{Stresemannstraße}}.}}\pend
           \pstart{}Mein lieber Freund,\pend\pstart
           Ich will gleich auf Deinen lieben Brief antworten, ſonſt komme ich lange nicht
               dazu.\pend
           \pstart
           Es freut mich ſehr, daß Du mit meiner \label{K_L02905-1v}\edtext{Anſicht}{\lemma{\textnormal{\emph{Anſicht}}}\Cendnote{\textnormal{siehe Paul Goldmann an Arthur Schnitzler, 11. 2. 1900}}}\label{K_L02905-1h} über dein \textcolor{green}{Stück}{}\ledrightnote{{$\rightarrow$}\textcolor{green}{Der Schleier der Beatrice. Schauspiel in fünf Akten}} zum
               Theil einverſtanden biſt. Ich habe noch einmal Dieſes und Jenes geleſen\strikeout{,} und kann Dir nur ſagen: Seit \textsc{\textcolor{blue}{Grillparzer}{}\ledrightnote{\textcolor{blue}{Franz Grillparzer}}} hat man auf dem \textcolor{pink}{Wien}{}\ledrightnote{\textcolor{pink}{Wien}}er Theater ſolche \textcolor{green}{Verſe}{}\ledrightnote{{$\rightarrow$}\textcolor{green}{Der Schleier der Beatrice. Schauspiel in fünf Akten}} nicht gehört. Das ſoll
               aber nicht bedeuten, daß es \textsc{\textcolor{blue}{Grillparzer}{}\ledrightnote{\textcolor{blue}{Franz Grillparzer}}ische} Verſe ſind. Nein, ſie
               ſind durchaus \textsc{Schnitzlerisch}, und nur der weiche \textcolor{pink}{Wien}{}\ledrightnote{\textcolor{pink}{Wien}}er Wohllaut iſt den beiden \textcolor{blue}{Dichtern}{}\ledrightnote{{$\rightarrow$}\textcolor{blue}{Franz Grillparzer}} gemeinſam. Was die Aufführung
               anlangt, ſo {\pb}möchte ich Streichungen empfehlen.
               Vielleicht auch einige \label{K_L02905-2v}\edtext{Umarbeitungen}{\lemma{\textnormal{\emph{Umarbeitungen}}}\Cendnote{\textnormal{keine entsprechenden
                  Umarbeitungen bekannt}}}\label{K_L02905-2h}. Ich bleibe dabei: die Geſtalt des \textcolor{green}{Herzog}{}\ledrightnote{{$\rightarrow$}\textcolor{green}{Der Schleier der Beatrice. Schauspiel in fünf Akten}}s erſcheint mir in zu unklaren
               Umriſſen. Wenn da auch nur ein wenig mit feſter Hand nachgezeichnet würde, könnte das
               dem \textcolor{green}{Drama}{}\ledrightnote{{$\rightarrow$}\textcolor{green}{Der Schleier der Beatrice. Schauspiel in fünf Akten}} ſehr zum Vortheil
               gereichen. Wäre es nicht doch möglich, daß die Hochzeit nur ein im Voraus
               beabſichtigter Carnevals-Scherz ſein könnte? Wenn der \textcolor{green}{Herzog}{}\ledrightnote{{$\rightarrow$}\textcolor{green}{Der Schleier der Beatrice. Schauspiel in fünf Akten}} durchaus edel ſein muß, ſo könnte
               der Edelmuth ja nachher erwachen. Mich hat übrigens in Deinem Briefe das Wort »Größe«
               ſtutzig gemacht. Warum ſoll der \textcolor{green}{Herzog}{}\ledrightnote{{$\rightarrow$}\textcolor{green}{Der Schleier der Beatrice. Schauspiel in fünf Akten}} »groß« ſein? Mir ſcheint, dieſes Streben nach Größe, dieſe abſtrakt
               hinzugedachte Eigenſchaft, iſt an der Unklarheit ſchuld. Hätteſt Du ihn nur (wie es
               ſonſt Deine Gewohnheit iſt) ruhig und \substVorne{}\textsuperscript{natürlich}{\allowbreak}\substDazwischen{}natürlich\substHinten{} leben laſſen, wie er leben mochte, ſo wäre \strikeout{\textcolor{gray}{er}} er deutlicher und wahrer geworden. Im Übrigen, vielleicht haſt Du Recht, und
                  {\pb}auf der Bühne zeigt ſich vielleicht, daß die \textcolor{green}{Figur}{}\ledrightnote{{$\rightarrow$}\textcolor{green}{Der Schleier der Beatrice. Schauspiel in fünf Akten}} richtig gedacht war.\pend
           \pstart
           Welche Rolle \label{K_L02905-4v}\edtext{\textsc{\textcolor{blue}{Kainz}{}\ledrightnote{\textcolor{blue}{Josef Kainz}}}}{\lemma{\textnormal{\emph{Kainz}}}\Cendnote{\textnormal{\textcolor{blue}{Josef Kainz} war ein von \textcolor{blue}{Schnitzler} vielgeschätzter Schauspieler und war mehrmals an
                  Inszenierungen seiner Dramen beteiligt. Für die geplante Uraufführung des \emph{\textcolor{green}{Schleiers der Beatrice}} im \textcolor{pink}{Burgtheater} wollte \textcolor{blue}{Schnitzler}{ }\textcolor{blue}{Kainz} in der Rolle des \textcolor{green}{Filippo} sehen (vgl. Arthur Schnitzler an Richard Beer-Hofmann, 17. 2. 1900). Zu dieser Aufführung
                  kam es jedoch nicht (vgl. Paul Goldmann an Arthur Schnitzler, 12. 11. [1899]).}}}\label{K_L02905-4h} ſpielen ſoll, kann ich Dir nicht ſagen. Denn ich kenne
                  \textsc{\textcolor{blue}{Kainz}{}\ledrightnote{\textcolor{blue}{Josef Kainz}}} nicht. Der \textcolor{green}{Herzog}{}\ledrightnote{{$\rightarrow$}\textcolor{green}{Der Schleier der Beatrice. Schauspiel in fünf Akten}} muß
               jedenfalls ein vollendeter \uline{Sprecher} ſein, und mir
               ſcheint, daß \textsc{\textcolor{blue}{Kainz}{}\ledrightnote{\textcolor{blue}{Josef Kainz}}} das nicht iſt. Für die \textsc{\textcolor{green}{Beatrice}{}\ledrightnote{{$\rightarrow$}\textcolor{green}{Der Schleier der Beatrice. Schauspiel in fünf Akten}}} aber gibt es meiner Anſicht nach nur \uline{eine} auf
               den deutſchen Theatern: Die \label{K_L02905-6v}\edtext{\textsc{\textcolor{blue}{Triesch}{}\ledrightnote{\textcolor{blue}{Irene Triesch}}}}{\lemma{\textnormal{\emph{Triesch}}}\Cendnote{\textnormal{\textcolor{blue}{Irene Triesch} gestaltete erst 1903 die \textcolor{green}{Beatrice} am \textcolor{pink}{Deutschen Theater Berlin}
                  aus. \textcolor{blue}{Schnitzler} missfiel sie darin jedoch
                     (vgl. A. S.: \emph{Tagebuch}, 23. 2. 1903).}}}\label{K_L02905-6h} in
                  \textcolor{pink}{Frankfurt}{}\ledrightnote{\textcolor{pink}{Frankfurt am Main}}. Sie hat geniale Kunſt-Inſtinkte,
               iſt ſelbſt ein ſo unberechenbares Luder, wie Deine \textsc{\textcolor{green}{Beatrice}{}\ledrightnote{{$\rightarrow$}\textcolor{green}{Der Schleier der Beatrice. Schauspiel in fünf Akten}}}, hat außerdem die Jugend und das ſüdliche Feuer. Damit wäre jede Frage über die
               Bühnenwirkſamkeit der Figur mit einem Schlage beſeitigt. Die \textsc{\textcolor{blue}{Triesch}{}\ledrightnote{\textcolor{blue}{Irene Triesch}}} würde etwas Unerhörtes daraus machen. Wenn Du mir folgteſt, würdeſt Du alle
               Mittel aufbieten, um die Perſon für dieſe Rolle zu gewinnen. Aber leider folgſt Du
               mir ja niemals. In \textcolor{pink}{Berlin}{}\ledrightnote{\textcolor{pink}{Berlin}} könnte meiner Anſicht
               nach nur {\pb}das \label{K_L02905-8v}\edtext{»\textcolor{brown}{Deutſche Theater}{}\ledrightnote{\textcolor{brown}{Deutsches Theater Berlin}}«}{\lemma{\textnormal{\emph{»Deutſche Theater«}}}\Cendnote{\textnormal{Zwei Jahre nach der Uraufführung am
                      in
                     \textcolor{pink}{Breslau} (1. 12. 1900) fand am 7. 3. 1903 die Premiere am \emph{\textcolor{brown}{Deutschen Theater Berlin}} statt. \textcolor{blue}{Irene Triesch} spielte die Titelrolle. \textcolor{blue}{Otto Brahm} kannte das \textcolor{green}{Stück} bereits seit 7. 10. 1899.}}}\label{K_L02905-8h} in Betracht kommen. \textsc{\textcolor{blue}{Brahms}{}\ledrightnote{\textcolor{blue}{Otto Brahm}}}{ }\strikeout{iſt} zeigt ſich ſehr urtheilslos, wenn er nach dem \textcolor{green}{Stück}{}\ledrightnote{{$\rightarrow$}\textcolor{green}{Der Schleier der Beatrice. Schauspiel in fünf Akten}} nicht mit beiden Händen
               greift. Wenn es in \textcolor{pink}{Wien}{}\ledrightnote{\textcolor{pink}{Wien}} Erfolg hat, wird er es
               übrigens ſchon thun. An das \label{K_L02905-11v}\edtext{\textcolor{brown}{Schauſpielhaus}{}\ledrightnote{\textcolor{brown}{Schauspielhaus Berlin}}}{\lemma{\textnormal{\emph{Schauſpielhaus}}}\Cendnote{\textnormal{Zu einer Inszenierung am \emph{\textcolor{brown}{Schauspielhaus
                        Berlin}} kam es nicht.}}}\label{K_L02905-11h} iſt
               bei der jetzt herrſchenden Sittlichkeits-Manie nicht zu denken. Man würde Dein \textcolor{green}{Drama}{}\ledrightnote{{$\rightarrow$}\textcolor{green}{Der Schleier der Beatrice. Schauspiel in fünf Akten}} entweder überhaupt nicht
               nehmen oder Dir zumuthen, die Hälfte wegzulaſſen. Im Nothfall könnte man es auch mit
               dem \label{K_L02905-12v}\edtext{»\textcolor{brown}{Berliner Theater}{}\ledrightnote{\textcolor{brown}{Berliner Theater}}«}{\lemma{\textnormal{\emph{»Berliner Theater«}}}\Cendnote{\textnormal{Zu einer 
                  Inszenierung von \emph{\textcolor{green}{Der Schleier der Beatrice}} am
                     \textcolor{pink}{Berliner Theater} kam es nicht.}}}\label{K_L02905-12h} (Direktion
                  \textsc{\textcolor{blue}{Paul Lindau}{}\ledrightnote{\textcolor{blue}{Paul Lindau}}}) verſuchen, wo nicht ſchlecht geſpielt wird; nur die Ausſtattung würde hier
               armſeelig ſein.\pend
           \pstart
           Deine \label{K_L02905-14v}\edtext{Aufträge}{\lemma{\textnormal{\emph{Aufträge}}}\Cendnote{\textnormal{Bezug unklar}}}\label{K_L02905-14h} an \textsc{\textcolor{blue}{Gusti}{}\ledrightnote{{$\rightarrow$}\textcolor{blue}{Auguste Chlum}}} u. die \label{K_L02905-77v}\edtext{\textcolor{blue}{Frau
                  Rechtsanwalt}{}\ledrightnote{{$\rightarrow$}\textcolor{blue}{Rosa Freudenthal}}}{\lemma{\textnormal{\emph{Frau
                  Rechtsanwalt}}}\Cendnote{\textnormal{höchstwahrscheinlich \textcolor{blue}{Schnitzler}s ehemalige Geliebte \textcolor{blue}{Rosa Freudenthal}, die mit dem Rechtsanwalt \textcolor{blue}{Hermann Freudenthal} verheiratet war; \textcolor{blue}{Goldmann} bezog sich bereits 1897 mit einer ähnlichen Formulierung auf sie (vgl. Paul Goldmann an Arthur Schnitzler, 4. 9. 1897)}}}\label{K_L02905-77h} werde ich
               beſorgen.\pend
           \pstart
           Das \textcolor{brown}{Theaterreferat}{}\ledrightnote{{$\rightarrow$}\textcolor{brown}{Neue Freie Presse}} von hier aus
               hat ſeine Schwierigkeiten. Ich muß doch alle Deine \label{K_L02905-18v}\edtext{Geliebten}{\lemma{\textnormal{\emph{Geliebten}}}\Cendnote{\textnormal{darunter etwa \textcolor{blue}{Marie Glümer}}}}\label{K_L02905-18h} loben. Um Irrthümer auszuſchließen, werde ich Dich demnächſt um einen Katalog
               bitten.\pend
           \pstart
           {\pb}Von mir willſt Du hören? Siehſt Du, ich habe wenig \substVorne{}\textsuperscript{\textcolor{gray}{h}}\substDazwischen{}Z\substHinten{}eit zum Schreiben. Ich muß alſo wählen: ſoll ich Dir von Dir ſchreiben oder
               von mir? Und Du wirſt doch nicht leugnen, daß es Dich mehr intereſſirt, wenn ich Dir
               über Dein \textcolor{green}{Stück}{}\ledrightnote{{$\rightarrow$}\textcolor{green}{Der Schleier der Beatrice. Schauspiel in fünf Akten}} ſchreibe, als
               über meine Schmerzen und \substVorne{}\textsuperscript{\textcolor{gray}{ſ}}\substDazwischen{}S\substHinten{}orgen: Oder vielmehr, Du wirſt es leugnen, aber ich werde Dir nicht
               glauben.\pend
           \pstart
           Auf Umwegen höre ich, daß Dein \label{K_L02905-45v}\edtext{\textcolor{blue}{Bruder}{}\ledrightnote{{$\rightarrow$}\textcolor{blue}{Julius Schnitzler}} ein \textcolor{blue}{Mädchen}{}\ledrightnote{{$\rightarrow$}\textcolor{blue}{Anna Donath}} bekommen}{\lemma{\textnormal{\emph{Bruder … bekommen}}}\Cendnote{\textnormal{\textcolor{blue}{Anna Schnitzler} (verh. \textcolor{blue}{Donath}), das dritte Kind von \textcolor{blue}{Julius} und \textcolor{blue}{Helene
                     Schnitzler}, war am 23. 1. 1900 geboren worden.}}}\label{K_L02905-45h} hat. Bitte, übermittle den \textcolor{blue}{Eltern}{}\ledrightnote{{$\rightarrow$}\textcolor{blue}{Julius Schnitzler}{\newline}{$\rightarrow$}\textcolor{blue}{Helene Schnitzler}} meine {\pb}Glückwünſche zugleich mit meinen herzlichen Grüßen.
               Auch Deine übrigen Angehörigen bitte ich zu grüßen.\pend
           \pstart
           Eine \textcolor{pink}{Wien}{}\ledrightnote{\textcolor{pink}{Wien}}er \textcolor{blue}{Jüdin}{}\ledrightnote{{$\rightarrow$}\textcolor{blue}{Adele Schreiber}}, ein Frl. \textsc{\textcolor{blue}{Schreiber}{}\ledrightnote{\textcolor{blue}{Adele Schreiber}}}, iſt mir mit einer Empfehlung von \textsc{\textcolor{blue}{Hanslick}{}\ledrightnote{\textcolor{blue}{Eduard Hanslick}}} ins Haus gekommen. Sie will \textcolor{pink}{hier}{}\ledrightnote{{$\rightarrow$}\textcolor{pink}{Berlin}} einen \label{K_L02905-37v}\edtext{\textcolor{green}{Vortrag}{}\ledrightnote{{$\rightarrow$}\textcolor{green}{[Vortrag über Arthur Schnitzler]}}}{\lemma{\textnormal{\emph{Vortrag}}}\Cendnote{\textnormal{Der \textcolor{green}{Vortrag} von \textcolor{blue}{Adele
                     Schreiber}, veranstaltet von der \emph{\textcolor{brown}{Gesellschaft für Kunst und Wissenschaft}} in \textcolor{pink}{Berlin}, fand am 28. 3. 1900 statt.}}}\label{K_L02905-37h}
               über Dich halten (was ich bedaure, denn der \textcolor{green}{Vortrag}{}\ledrightnote{{$\rightarrow$}\textcolor{green}{[Vortrag über Arthur Schnitzler]}} wird ſchlecht ſein) und hat mir inzwiſchen im
               Geſpräch werthvolle literariſche Aufſchlüſſe über Dich gegeben.\pend
           \pstart
           Viele treue Grüße! {\\[\baselineskip]}Dein {\\[\baselineskip]}\spacefill\mbox{Paul Goldmann.}\pend
           \leftskip=0em{}\pstart
           \noindent{}Ja, eine Bitte habe ich doch. Ich habe den Eindruck, daß ich in der \textcolor{brown}{N. Fr. Preſſe}{}\ledrightnote{\textcolor{brown}{Neue Freie Presse}}, im Gegenſatz zur \label{K_L02905-57v}\edtext{\textcolor{brown}{Frankfurter {\pb}Zeitung}{}\ledrightnote{\textcolor{brown}{Frankfurter Zeitung}}}{\lemma{\textnormal{\emph{Frankfurter Zeitung}}}\Cendnote{\textnormal{für die \textcolor{blue}{Goldmann} bis Dezember 1899
                     gearbeitet hatte}}}\label{K_L02905-57h}, vollſtändig verſchwinde. Merkt irgend Jemand, außer
                  Dir, daß ich vorhanden bin? Bitte, ſchreib’ mir ein Wort darüber!\pend
           \endnumbering\briefempfaengerindex{Schnitzler, Arthur@\textsc{Schnitzler, Arthur}!zzzGoldmann, Paul@\emph{von Paul Goldmann}!1900-02-201@{20. 2. 1900}|)be}\mylabel{h}  \normalsize

\doendnotes{C}
\bigskip
\vfill

\clearpage

\footnotesize

\lohead{\textsc{register}}

% Definiere theindex-Environment komplett neu ohne reledmac
\makeatletter
\renewenvironment{theindex}{%
  \section*{\indexname}%
  \setlength{\parindent}{0pt}%
  \setlength{\parskip}{0pt plus 0.3pt}%
  \let\item\@idxitem
}{%
  \clearpage
}
\makeatother

\IfFileExists{\jobname-pw.ind}{\input{\jobname-pw.ind}}{}

\end{document}

      