%% latex-korrekturansicht-vorspann.tex
%% Vorspann für die Korrekturansicht.
%% Lädt die gemeinsame Datei latex-vorspann.tex mit gesetztem Schalter.

\newif\ifkorrekturansicht
\korrekturansichttrue

\input{../tex-inputs/latex-vorspann}


\renewcommand{\erwaehntePersonen}{Personen: Max Eugen Burckhard, Paul Goldmann, Heinrich Kanner, Paul Schlenther, Olga Schnitzler}
\renewcommand{\erwaehnteInstitutionen}{Institutionen: Burgtheater}
\renewcommand{\erwaehnteOrte}{Orte: Wien}
\renewcommand{\erwaehnteWerke}{Werke: Berliner Theater. (»Der Schleier der Beatrice« von Arthur Schnitzler.), Der Bauernfeld-Preis. Eine Interpellation, Der Schleier der Beatrice. Schauspiel in fünf Akten, Die Zeit, Neue Freie Presse}
\section[ Felix Salten an Arthur Schnitzler, {[}19.? 3. 1903{]}]{Felix Salten an Arthur Schnitzler, {[}19.? 3. 1903{]}}
\nopagebreak\mylabel{v}
\rehead{ }\normalsize\beginnumbering\briefempfaengerindex{Schnitzler, Arthur@\textsc{Schnitzler, Arthur}!zzzSalten, Felix@\emph{von Felix Salten}!1903-03-191@{{[}19.? 3. 1903{]}}|(be}
\toendnotes[C]{\smallbreak\pagebreak[2]}\Standort{CUL, Schnitzler, B 89, A 2.}
\physDesc{Brief, 1 Blatt, 4 Seiten, 1371 Zeichen
\newline{}Handschrift: Bleistift, lateinische Kurrent
\newline{}Schnitzler: mit Bleistift datiert: »März 903.« 
\newline{}Ordnung: mit Bleistift von unbekannter Hand nummeriert: »165« }\toendnotes[C]{\smallbreak}
\pstart
           \noindent{}{\pb}Lieber – \textcolor{blue}{Goldmann}{}\ledrightnote{\textcolor{blue}{Paul Goldmann}}s \label{K_L03340-1v}\edtext{\textcolor{green}{Feuilleton}{}\ledrightnote{{$\rightarrow$}\textcolor{green}{Berliner Theater. (»Der Schleier der Beatrice« von Arthur Schnitzler.)}}}{\lemma{\textnormal{\emph{Feuilleton}}}\Cendnote{\textnormal{\textcolor{blue}{Paul Goldmann}: \emph{\textcolor{green}{Berliner Theater. (»Der Schleier der Beatrice« von Arthur
                        Schnitzler.)}}. In: \emph{\textcolor{green}{Neue Freie
                        Presse}}, Nr. 13.851, 19. 3. 1903,
                     Morgenblatt, S. 1–5.}}}\label{K_L03340-1h} ist mir – bei allen Erklärungen, die wir uns
               darüber geben und finden können – doch räthselhaft. Ich bin über die kleinliche und
               kleingeistige Form erstaunt, und wundere mich, dass einem Werk wie dem »\textcolor{green}{Schleier}{}\ledrightnote{\textcolor{green}{Der Schleier der Beatrice. Schauspiel in fünf Akten}}« gegenüber, der schärfste kritische
               Angriff in der Plattitüde gipfelt: »denn es ist besser lebendig sein ec.« So gesehen
               allerdings müßen sich alle Zusammenhänge verlieren. Dass \textcolor{green}{Filippo}{}\ledrightnote{{$\rightarrow$}\textcolor{green}{Der Schleier der Beatrice. Schauspiel in fünf Akten}} durch den Treubruch gegen die \textcolor{green}{Teresina}{}\ledrightnote{{$\rightarrow$}\textcolor{green}{Der Schleier der Beatrice. Schauspiel in fünf Akten}}{ }{\pb}aus den Angeln gehoben wird,
               und dass er im Verlust dieser edelsten Doppelbeziehung (\textcolor{green}{Teresina}{}\ledrightnote{\textcolor{green}{Der Schleier der Beatrice. Schauspiel in fünf Akten}}{ }{\kaufmannsund} ihr Bruder) schon sich selbst verloren hat, das
               übersieht \textcolor{blue}{G.}{}\ledrightnote{\textcolor{blue}{Paul Goldmann}} oder er unterschlägt es. Ich
               bedauere dieses \textcolor{green}{Feuilleton}{}\ledrightnote{{$\rightarrow$}\textcolor{green}{Der Schleier der Beatrice. Schauspiel in fünf Akten}} aus
               vielen künstlerischen und menschlichen Gründen, und vor allem deshalb, weil es der in
                  \textcolor{pink}{Wien}{}\ledrightnote{\textcolor{pink}{Wien}} spielenden \label{K_L03340-2v}\edtext{\textcolor{green}{Schleier}{}\ledrightnote{\textcolor{green}{Der Schleier der Beatrice. Schauspiel in fünf Akten}}-Affaire}{\lemma{\textnormal{\emph{Schleier-Affaire}}}\Cendnote{\textnormal{Bezug auf die teilweise in der Presse berichteten Vorgänge
                  aus dem Jahr 1901 um die halbherzige Zu- und nachmalige
                  Absage \textcolor{blue}{Paul Schlenther}s, das \textcolor{green}{Stück} am \emph{\textcolor{brown}{Burgtheater}} aufzuführen}}}\label{K_L03340-2h} vorläufig {\pb}einen unrühmlichen Abschluß
               gibt. Gerade mit Bezug \uline{darauf} bin ich von diesem
               Vorgehen doppelt impressionirt, denn \uline{G.} war in \textcolor{pink}{Wien}{}\ledrightnote{\textcolor{pink}{Wien}} als die Affaire spielte, er hat mitgeholfen
               und mitgerathen, ist mitempört gewesen, war mit mir bei \textcolor{blue}{Burckhard}{}\ledrightnote{\textcolor{blue}{Max Eugen Burckhard}}{ }{\kaufmannsund} hat sich für dieses \textcolor{green}{Werk}{}\ledrightnote{{$\rightarrow$}\textcolor{green}{Der Schleier der Beatrice. Schauspiel in fünf Akten}}, über das er damals freilich anders sprach als
                  heute{[},{]} sehr engagirt.\pend
           
\pstart
           Entschuldigen Sie diese {\pb}»Kundgebung.« Sehe ich Sie \label{K_L03340-3v}\edtext{heute{ }Abend im Café}{\lemma{\textnormal{\emph{heute Abend im Café}}}\Cendnote{\textnormal{Nachweisbar
                  war \textcolor{blue}{Schnitzler} abends bei \textcolor{blue}{Olga Gussmann}, vgl. A. S.: \emph{Tagebuch}, 19. 3. 1903.}}}\label{K_L03340-3h}? Ich bin etwa um 11
               dort.\pend
           
\pstart
           Der Titel \label{K_L03340-4v}\edtext{\uline{\textcolor{green}{Interview}{}\ledrightnote{{$\rightarrow$}\textcolor{green}{Der Bauernfeld-Preis. Eine Interpellation}}} ist durch ein Missverständnis}{\lemma{\textnormal{\emph{Interview … Missverständnis}}}\Cendnote{\textnormal{[\textcolor{blue}{Felix Salten}]: \emph{\textcolor{green}{Der Bauernfeld-Preis. Eine Interpellation}}. In: \emph{\textcolor{green}{Die Zeit}}, Jg. 2, Nr. 169, 19. 3. 1903, S. 5. Darin ist die den
                  Aussagen \textcolor{blue}{Schnitzler}s gewidmete Stelle mit
                  der Überschrift »\textcolor{green}{Ein Interview mit Arthur
                        Schnitzler}« versehen.}}}\label{K_L03340-4h}{ }\label{K_L03340-5v}\edtext{heute{ }Nachts}{\lemma{\textnormal{\emph{heute Nachts}}}\Cendnote{\textnormal{Das erlaubt die Datierung des
                  Korrespondenzstücks auf den Tag, an dem \emph{\textcolor{green}{Der
                     Bauernfeld-Preis. Eine Interpellation}} erschien.}}}\label{K_L03340-5h}{ }3\textsuperscript{h} als ich schon fort war ins \textcolor{green}{Blatt}{}\ledrightnote{{$\rightarrow$}\textcolor{green}{Die Zeit}} gekommen. D\textsuperscript{r}{ }\textcolor{blue}{Kanner}{}\ledrightnote{\textcolor{blue}{Heinrich Kanner}} läßt Sie um Entschuldigung bitten.\pend
           
\pstart
           Herzlichst {\\[\baselineskip]}Ihr \spacefill\mbox{\textcolor{gray}{FS}}\pend
           \leftskip=0em{}\endnumbering\briefempfaengerindex{Schnitzler, Arthur@\textsc{Schnitzler, Arthur}!zzzSalten, Felix@\emph{von Felix Salten}!1903-03-191@{{[}19.? 3. 1903{]}}|)be}\mylabel{h}  \normalsize

\doendnotes{C}
\bigskip
\vfill

\clearpage

\footnotesize

\lohead{\textsc{register}}

% Definiere theindex-Environment komplett neu ohne reledmac
\makeatletter
\renewenvironment{theindex}{%
  \section*{\indexname}%
  \setlength{\parindent}{0pt}%
  \setlength{\parskip}{0pt plus 0.3pt}%
  \let\item\@idxitem
}{%
  \clearpage
}
\makeatother

\IfFileExists{\jobname-pw.ind}{\input{\jobname-pw.ind}}{}

\end{document}

      