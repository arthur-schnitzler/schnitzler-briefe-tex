%% latex-korrekturansicht-vorspann.tex
%% Vorspann für die Korrekturansicht.
%% Lädt die gemeinsame Datei latex-vorspann.tex mit gesetztem Schalter.

\newif\ifkorrekturansicht
\korrekturansichttrue

\input{../tex-inputs/latex-vorspann}


               \section[Arthur und Olga Schnitzler an Richard und Paula Beer-Hofmann, 17. 9. 1910]{ Arthur und Olga Schnitzler an Richard und Paula Beer-Hofmann,
               17. 9. 1910}\nopagebreak\mylabel{v}\rehead{ }\normalsize\beginnumbering\briefempfaengerindex{Beer-Hofmann, Paula@\textsc{Beer-Hofmann, Paula}!zzzSchnitzler, Olga@\emph{von Olga Schnitzler}!1910-09-171@{17. 9. 1910}|(be}\briefempfaengerindex{Beer-Hofmann, Paula@\textsc{Beer-Hofmann, Paula}!zzzSchnitzler, Arthur@\emph{von Arthur Schnitzler}!1910-09-171@{17. 9. 1910}|(be}\briefempfaengerindex{Beer-Hofmann, Richard@\textsc{Beer-Hofmann, Richard}!zzzSchnitzler, Olga@\emph{von Olga Schnitzler}!1910-09-171@{17. 9. 1910}|(be}\briefempfaengerindex{Beer-Hofmann, Richard@\textsc{Beer-Hofmann, Richard}!zzzSchnitzler, Arthur@\emph{von Arthur Schnitzler}!1910-09-171@{17. 9. 1910}|(be} \toendnotes[C]{\smallbreak\pagebreak[2]} \Standort{YCGL, MSS 31.}
\physDesc{Bildpostkarte
\newline{}Handschrift Arthur Schnitzler: Bleistift, deutsche Kurrent\newline{}Handschrift Olga Schnitzler: Bleistift, lateinische Kurrent\newline{}Versand: Stempel: »\nobreak{}\oindex{Ruedesheim@\textbf{Rüdesheim}, \emph{Besiedelter Ort (A.BSO)}|pwk}Rüdesheim (Rhein), 17. 9. 10, 8–9N\nobreak{}«.  }\toendnotes[C]{\smallbreak}\pstart{}{\pb}Hrn Dr. \textsc{Rich. Beer
                     Hofmann}\pend{}\pstart{}u Gemahlin\pend{}\pstart{}\textcolor{pink}{Wien XVIII}{}\ledrightnote{\textcolor{pink}{XVIII., Währing}}\pend{}\pstart{}\textsc{\textcolor{pink}{Hasenauerstr. 59}{}\ledrightnote{\textcolor{pink}{Hasenauerstraße}}}\pend{}{\bigskip}\pstart
           \noindent{}\centering{}{\pb}\textcolor{gray}{\textbf{\textcolor{pink}{Rüdesheim}{}\ledrightnote{\textcolor{pink}{Rüdesheim}}.\hspace*{1.5em}Der \textcolor{pink}{Rhein}{}\ledrightnote{\textcolor{pink}{Rhein}}.}}\pend
           \pstart
           {\pb}Herzliche Grüße. Eigentlich ſind wir in \textcolor{pink}{Frankfurt \textsuperscript{a}/M
               }{}\ledrightnote{\textcolor{pink}{Frankfurt am Main}}; dies iſt ein Ausflug.\pend
           \pstart \spacefill\mbox{A.}\pend{}\pstart
           17. 9. 1910\pend
           \pstart
           \noindent{}{[}hs. O. Schnitzler:{]} Herzliche Grüsse, \label{K_L01955_1v}\edtext{Dienstag}{\lemma{\textnormal{\emph{Dienstag}}}\Cendnote{\textnormal{siehe A. S.: \emph{Tagebuch}, 20. 9. 1910}}}\label{K_L01955_1h} hoffen wir in \textcolor{pink}{Wien}{}\ledrightnote{\textcolor{pink}{Wien}} zu sein.\pend
           \pstart \spacefill\mbox{Olga.}\pend{}\endnumbering\briefempfaengerindex{Beer-Hofmann, Paula@\textsc{Beer-Hofmann, Paula}!zzzSchnitzler, Olga@\emph{von Olga Schnitzler}!1910-09-171@{17. 9. 1910}|)be}\briefempfaengerindex{Beer-Hofmann, Paula@\textsc{Beer-Hofmann, Paula}!zzzSchnitzler, Arthur@\emph{von Arthur Schnitzler}!1910-09-171@{17. 9. 1910}|)be}\briefempfaengerindex{Beer-Hofmann, Richard@\textsc{Beer-Hofmann, Richard}!zzzSchnitzler, Olga@\emph{von Olga Schnitzler}!1910-09-171@{17. 9. 1910}|)be}\briefempfaengerindex{Beer-Hofmann, Richard@\textsc{Beer-Hofmann, Richard}!zzzSchnitzler, Arthur@\emph{von Arthur Schnitzler}!1910-09-171@{17. 9. 1910}|)be}\mylabel{h}  \normalsize

\doendnotes{C}
\bigskip
\vfill

\clearpage

\footnotesize

\lohead{\textsc{register}}

% Definiere theindex-Environment komplett neu ohne reledmac
\makeatletter
\renewenvironment{theindex}{%
  \section*{\indexname}%
  \setlength{\parindent}{0pt}%
  \setlength{\parskip}{0pt plus 0.3pt}%
  \let\item\@idxitem
}{%
  \clearpage
}
\makeatother

\IfFileExists{\jobname-pw.ind}{\input{\jobname-pw.ind}}{}

\end{document}

      