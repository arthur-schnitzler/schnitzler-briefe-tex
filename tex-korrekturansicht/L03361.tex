%% latex-korrekturansicht-vorspann.tex
%% Vorspann für die Korrekturansicht.
%% Lädt die gemeinsame Datei latex-vorspann.tex mit gesetztem Schalter.

\newif\ifkorrekturansicht
\korrekturansichttrue

\input{../tex-inputs/latex-vorspann}


\renewcommand{\erwaehntePersonen}{Personen: Olga Schnitzler, ?? [Besitzer des Palasthotels Berlin, Anfang 1903]}
\renewcommand{\erwaehnteInstitutionen}{Institutionen: Palasthotel Berlin}
\renewcommand{\erwaehnteOrte}{Orte: Berlin, Dessauer Straße, Palasthotel Berlin, Wien}
\renewcommand{\erwaehnteWerke}{}
\section[ Paul Goldmann an Arthur Schnitzler, 27. 1. {[}1903{]}]{Paul Goldmann an Arthur Schnitzler, 27. 1. {[}1903{]}}
\nopagebreak\mylabel{v}
\rehead{ }\normalsize\beginnumbering\briefempfaengerindex{Schnitzler, Arthur@\textsc{Schnitzler, Arthur}!zzzGoldmann, Paul@\emph{von Paul Goldmann}!1903-01-273@{27. 1. {[}1903{]}}|(be}
\toendnotes[C]{\smallbreak\pagebreak[2]}\Standort{DLA, A:Schnitzler, HS.NZ85.1.3173.}
\physDesc{Brief, 1 Blatt, 2 Seiten
\newline{}Handschrift: blaue Tinte, deutsche Kurrent
\newline{}Schnitzler: mit Bleistift das Jahr »{[}1{]}903.« vermerkt }\toendnotes[C]{\smallbreak}
\pstart
           \noindent{}\raggedleft{}{\pb}\textcolor{gray}{\textbf{\textcolor{pink}{DESSAUERSTRASSE 19}{}\ledrightnote{\textcolor{pink}{Dessauer Straße}}}}\pend
           
\pstart
           \textcolor{pink}{Berlin}{}\ledrightnote{\textcolor{pink}{Berlin}}, 27. Januar.\pend
           
\pstart\center{}Mein lieber Freund,\pend
\pstart
           Ich habe ſo viel zu thun, daß ich Dir nur in aller Eile für Deinen lieben Brief
               Danken kann, der mich unendlich erfreut hat. Wann kommſt Du nach \label{K_L03361-1v}\edtext{\textcolor{pink}{Berlin}{}\ledrightnote{\textcolor{pink}{Berlin}}}{\lemma{\textnormal{\emph{Berlin}}}\Cendnote{\textnormal{\textcolor{blue}{Schnitzler} war von 22. 2. 1903 bis 9. 3. 1903 in \textcolor{pink}{Berlin}. In dieser Zeit wohnte er im \textcolor{pink}{Palasthotel}.}}}\label{K_L03361-1h}? Ich ſehne mich danach, mit
               Dir all’ das zu beſprechen, was mir das Herz bedrückt. Ich bin ſeit Wochen in einem
               unbeſchreiblichen Zuſtand, gequält von Vorwürfen, Reue und Sehnſucht, die niemals {\pb}wieder befriedigt werden wird. Vielleicht kannſt Du
               mir ein tröſtendes und beruhigendes Wort ſagen. Mit dem \label{K_L03361-3v}\edtext{\textcolor{blue}{Direktor}{}\ledrightnote{{$\rightarrow$}\textcolor{blue}{?? [Besitzer des Palasthotels Berlin, Anfang 1903]}}}{\lemma{\textnormal{\emph{Direktor}}}\Cendnote{\textnormal{nicht ermittelt}}}\label{K_L03361-3h} des »\textcolor{brown}{Palaſthotel}{}\ledrightnote{\textcolor{brown}{Palasthotel Berlin}}« habe ich geſprochen; er hat Dir wohl
               inzwiſchen ſelbſt geſchrieben.\pend
           
\pstart
           Herzlichſte Grüße Dir und \textcolor{blue}{Olga}{}\ledrightnote{\textcolor{blue}{Olga Schnitzler}}! {\\[\baselineskip]}Dein
               getreuer {\\[\baselineskip]}\spacefill\mbox{Paul Goldm}\pend
           \leftskip=0em{}\endnumbering\briefempfaengerindex{Schnitzler, Arthur@\textsc{Schnitzler, Arthur}!zzzGoldmann, Paul@\emph{von Paul Goldmann}!1903-01-273@{27. 1. {[}1903{]}}|)be}\mylabel{h}
\begin{anhang}
\end{anhang}\normalsize

\doendnotes{C}
\bigskip
\vfill

\clearpage

\footnotesize

\lohead{\textsc{register}}

% Definiere theindex-Environment komplett neu ohne reledmac
\makeatletter
\renewenvironment{theindex}{%
  \section*{\indexname}%
  \setlength{\parindent}{0pt}%
  \setlength{\parskip}{0pt plus 0.3pt}%
  \let\item\@idxitem
}{%
  \clearpage
}
\makeatother

\IfFileExists{\jobname-pw.ind}{\input{\jobname-pw.ind}}{}

\end{document}

      