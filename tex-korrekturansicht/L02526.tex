%% latex-korrekturansicht-vorspann.tex
%% Vorspann für die Korrekturansicht.
%% Lädt die gemeinsame Datei latex-vorspann.tex mit gesetztem Schalter.

\newif\ifkorrekturansicht
\korrekturansichttrue

\input{../tex-inputs/latex-vorspann}


               \section[Arthur Schnitzler an Gerhart Hauptmann, 27. 11. 1929]{ Arthur Schnitzler an Gerhart Hauptmann, 27. 11. 1929}\nopagebreak\mylabel{v}\rehead{ }\normalsize\beginnumbering\briefempfaengerindex{Hauptmann, Gerhart@\textsc{Hauptmann, Gerhart}!zzzSchnitzler, Arthur@\emph{von Arthur Schnitzler}!1929-11-271@{27. 11. 1929}|(be} \toendnotes[C]{\smallbreak\pagebreak[2]} \Standort{Staatsbibliothek Berlin – Preußischer Kulturbesitz, GH Br NL (ehem. AdK) B 1324.}
\physDesc{Brief, 1 Blatt, 2 Seiten
\newline{}Handschrift: schwarze Tinte, lateinische Kurrent}\Standort{DLA, A:Schnitzler, HS.NZ85.1.5684.}
\physDesc{Brief, 1 Blatt, 2 Seiten, Fotokopie
\newline{}Handschrift: schwarze Tinte, lateinische Kurrent}\buchAbdrucke{\weitereDrucke{Arthur Schnitzler: \emph{Briefe 1913–1931}. Hg. Peter Michael Braunwarth, Richard Miklin, Susanne Pertlik und Heinrich Schnitzler. Frankfurt am Main: \emph{S. Fischer} 1984, S. 637.} }\toendnotes[C]{\smallbreak}\pstart
           \raggedleft{}{\pb}\textcolor{pink}{Wien}{}\ledrightnote{\textcolor{pink}{Wien}}, 27. 11. 29.\pend
           \pstart{}Verehrter Herr Gerhart Hauptmann\pend\pstart
           Sie nehmen mir gewiſs nicht übel daſs ich an dem \label{K_L02526_1v}\edtext{Bankett}{\lemma{\textnormal{\emph{Bankett}}}\Cendnote{\textnormal{Dieses fand
                  am 28. 11. 1929 statt.}}}\label{K_L02526_1h} Ihnen zu Ehren nicht theilnehme, seit
               längerer Zeit halte ich mich (nicht aus Princip, sondern aus einer vorläufg nicht zu
               überwindenden Abneigung) von großen Gesellschaften, insbesondre aber von
               Feierlichkeiten fern, mag ich im Herzen auch so begeistert mitfeiern, wie ich es
               z. B. bei einem Hauptmann Bankett thue. Ich muß Ihnen ja nicht erst von meiner
               Bewunderung und Liebe sprechen, – Sie haben immer gewußt, was Sie {\pb}mir bedeuten. \pend
           \pstart
           In jedem Falle aber werde ich Sie während Ihres \textcolor{pink}{Wien}{}\ledrightnote{\textcolor{pink}{Wien}}er Aufenthaltes sehen, ich melde mich, sobald Sie nicht mehr allzugeplagt
               sind und bin jedenfalls bei Ihrer \label{K_L02526_2v}\edtext{\textcolor{green}{Generalprobe}{}\ledrightnote{→\textcolor{green}{Spuk}}}{\lemma{\textnormal{\emph{Generalprobe}}}\Cendnote{\textnormal{Am 2. 12. 1929, dem Tag vor
                  der Uraufführung. Vgl. A. S.: \emph{Tagebuch}, 2. 12. 1929}}}\label{K_L02526_2h}. Doch hoff ich Ihnen noch vorher persönlich zu begegnen.\pend
           \pstart
           Empfehlen Sie mich Ihrer sehr verehrten \textcolor{blue}{Gattin}{}\ledrightnote{→\textcolor{blue}{Margarete Hauptmann}} mein lieber und verehrter Gerhard
               Hauptmann und seien Sie in herzlicher Ergebenheit gegrüßt.\pend
           \pstart
           Ihr{\\[\baselineskip]}Arthur Schnitzler\pend
           \leftskip=0em{}\endnumbering\briefempfaengerindex{Hauptmann, Gerhart@\textsc{Hauptmann, Gerhart}!zzzSchnitzler, Arthur@\emph{von Arthur Schnitzler}!1929-11-271@{27. 11. 1929}|)be}\mylabel{h}  \normalsize

\doendnotes{C}
\bigskip
\vfill

\clearpage

\footnotesize

\lohead{\textsc{register}}

% Definiere theindex-Environment komplett neu ohne reledmac
\makeatletter
\renewenvironment{theindex}{%
  \section*{\indexname}%
  \setlength{\parindent}{0pt}%
  \setlength{\parskip}{0pt plus 0.3pt}%
  \let\item\@idxitem
}{%
  \clearpage
}
\makeatother

\IfFileExists{\jobname-pw.ind}{\input{\jobname-pw.ind}}{}

\end{document}

      