%% latex-korrekturansicht-vorspann.tex
%% Vorspann für die Korrekturansicht.
%% Lädt die gemeinsame Datei latex-vorspann.tex mit gesetztem Schalter.

\newif\ifkorrekturansicht
\korrekturansichttrue

\input{../tex-inputs/latex-vorspann}


\renewcommand{\erwaehnteOrte}{Orte: Edmund-Weiß-Gasse 7, Kairo, Wien, XVIII., Währing, Zitadelle von Saladin, Österreich}
\renewcommand{\erwaehnteWerke}{}
\section[ Felix Salten an Arthur Schnitzler, 8. 3. 1904]{Felix Salten an Arthur Schnitzler, 8. 3. 1904}
\nopagebreak\mylabel{v}
\rehead{ }\normalsize\beginnumbering\briefempfaengerindex{Schnitzler, Arthur@\textsc{Schnitzler, Arthur}!zzzSalten, Felix@\emph{von Felix Salten}!1904-03-081@{8. 3. 1904}|(be}
\toendnotes[C]{\smallbreak\pagebreak[2]}\Standort{CUL, Schnitzler, B 89, B 1.}
\physDesc{Bildpostkarte, 98 Zeichen
\newline{}Handschrift: schwarze Tinte, lateinische Kurrent
\newline{}Versand: 1) Stempel: »\nobreak{}\oindex{Kairo@\textbf{Kairo}, \emph{P.PPLC}|pwk}Cairo\nobreak{}«.   2) Stempel: »\nobreak{}\oindex{XVIII., Waehring@\textbf{XVIII., Währing}, \emph{A.ADM3}|pwk}18/1 Wien 110, 1\textcolor{gray}{6}. 3. \textcolor{gray}{0}4, 8, Bestellt\nobreak{}«. 
\newline{}Ordnung: mit Bleistift von unbekannter Hand nummeriert: »185« }\pstart{}{\pb}M\textsuperscript{r} D\textsuperscript{r} Arthur Schnitzler\pend{}\pstart{}\textcolor{pink}{XVIII. Spöttelgaße 7}{}\ledrightnote{\textcolor{pink}{Edmund-Weiß-Gasse 7}}\pend{}\pstart{}\textcolor{pink}{Wien}{}\ledrightnote{\textcolor{pink}{Wien}}\pend{}\pstart{}\begin{otherlanguage}{french}\textcolor{pink}{Autriche}{}\ledrightnote{\textcolor{pink}{Österreich}}\end{otherlanguage}\pend{}
{\bigskip}
\pstart
           \noindent{}{\pb}\textcolor{gray}{\textbf{\textcolor{pink}{LE CAIRE}{}\ledrightnote{\textcolor{pink}{Kairo}}}}\hfill \textcolor{gray}{\textbf{\textcolor{pink}{La Citadelle}{}\ledrightnote{\textcolor{pink}{Zitadelle von Saladin}}}}\pend
           
\pstart
           \textcolor{pink}{Cairo}{}\ledrightnote{\textcolor{pink}{Kairo}}, 8. III. 04.\pend
           
\pstart
           Wunderschön!! herzl. Ihr \spacefill\mbox{F S.}\pend
           \endnumbering\briefempfaengerindex{Schnitzler, Arthur@\textsc{Schnitzler, Arthur}!zzzSalten, Felix@\emph{von Felix Salten}!1904-03-081@{8. 3. 1904}|)be}\mylabel{h}  \normalsize

\doendnotes{C}
\bigskip
\vfill

\clearpage

\footnotesize

\lohead{\textsc{register}}

% Definiere theindex-Environment komplett neu ohne reledmac
\makeatletter
\renewenvironment{theindex}{%
  \section*{\indexname}%
  \setlength{\parindent}{0pt}%
  \setlength{\parskip}{0pt plus 0.3pt}%
  \let\item\@idxitem
}{%
  \clearpage
}
\makeatother

\IfFileExists{\jobname-pw.ind}{\input{\jobname-pw.ind}}{}

\end{document}

      