%% latex-korrekturansicht-vorspann.tex
%% Vorspann für die Korrekturansicht.
%% Lädt die gemeinsame Datei latex-vorspann.tex mit gesetztem Schalter.

\newif\ifkorrekturansicht
\korrekturansichttrue

\input{../tex-inputs/latex-vorspann}


         
         \renewcommand{\erwaehntePersonen}{Personen: Eberhard König, Agnes Sorma}
         \renewcommand{\erwaehnteOrte}{Orte: Berlin, Dessauer Straße, Lessing-Theater, Schauspielhaus, Wien}
         \renewcommand{\erwaehnteWerke}{Werke: Gevatter Tod. Ein Märchen von der Menschheit. Drama in fünf Aufzügen, Liebelei. Schauspiel in drei Akten}
               \section[ Paul Goldmann an Arthur Schnitzler, 13. 4. {[}1900{]}]{Paul Goldmann an Arthur Schnitzler, 13. 4. {[}1900{]}}\nopagebreak\mylabel{v}\rehead{ }\normalsize\beginnumbering\briefempfaengerindex{Schnitzler, Arthur@\textsc{Schnitzler, Arthur}!zzzGoldmann, Paul@\emph{von Paul Goldmann}!1900-04-131@{13. 4. {[}1900{]}}|(be} \toendnotes[C]{\smallbreak\pagebreak[2]} \Standort{DLA, A:Schnitzler, HS.NZ85.1.3170.}
\physDesc{Brief, 1 Blatt, 2 Seiten
\newline{}Handschrift: blaue Tinte, deutsche Kurrent
\newline{}Schnitzler: mit Bleistift das Jahr »{[}1{]}900« vermerkt }\toendnotes[C]{\smallbreak}\pstart
           \noindent{}{\pb}\textcolor{pink}{\textcolor{gray}{\textbf{DESSAUERSTRASSE 19}}}{}\ledrightnote{\textcolor{pink}{Dessauer Straße}}\pend
           \pstart
           \raggedleft{}\textcolor{pink}{Berlin}{}\ledrightnote{\textcolor{pink}{Berlin}}, 13. April.\pend
           \pstart{}Mein lieber Freund,\pend\pstart
           Warum höre ich denn ſo gar nichts von Dir? Die zwei Anſichtspoſtkarten habe ich wohl
               erhalten, aber ſie geben mir mehr Aufſchluß über die \label{K_L02910-1v}\edtext{Gegend}{\lemma{\textnormal{\emph{Gegend}}}\Cendnote{\textnormal{siehe Paul Goldmann an Arthur Schnitzler, 29. 3. [1900]}}}\label{K_L02910-1h}, als über Dein Ergehen. Haſt Du unterwegs nicht einmal eine Viertelſtunde, um
               mir etwas ausführlicher zu berichten, was Du erlebſt und wie Du Dich fühlſt? Ich weiß
               nicht einmal, ob Du ſchon zurück biſt. Und wann kommſt Du nach \label{K_L02910-2v}\edtext{\textcolor{pink}{Berlin}{}\ledrightnote{\textcolor{pink}{Berlin}}}{\lemma{\textnormal{\emph{Berlin}}}\Cendnote{\textnormal{\textcolor{blue}{Schnitzler} kam erst am 24. 11. 1900 wieder
                  nach \textcolor{pink}{Berlin}. Er blieb dort bis zum 28. 11. 1900.}}}\label{K_L02910-2h}?
               Hätte ich gewußt, ob Du bereits wieder {\pb}heimgekehrt
               biſt, ſo \strikeout{hätte} wäre ich vielleicht über Oſtern nach
                  \textcolor{pink}{Wien}{}\ledrightnote{\textcolor{pink}{Wien}} gekommen. Aber bei dieſer
               Nachrichtenloſigkeit habe ich mich zu einem Entſchluß nicht aufſchwingen können.
               Bitte, ſchreib’ mir bald!\pend
           \pstart
           Ich hätte gern über das \label{K_L02910-3v}\edtext{Gaſtſpiel der
                  \textsc{\textcolor{blue}{Sorma}{}\ledrightnote{\textcolor{blue}{Agnes Sorma}}} in »\textcolor{green}{Liebelei}{}\ledrightnote{\textcolor{green}{Liebelei. Schauspiel in drei Akten}}«}{\lemma{\textnormal{\emph{Gaſtſpiel … »Liebelei«}}}\Cendnote{\textnormal{\textcolor{blue}{Agnes Sorma} gastierte am 4. 6. 1900 und am 12. 4. 1900 als \textcolor{green}{Christine} in den \emph{\textcolor{green}{Liebelei}}-Aufführungen am \textcolor{pink}{Berlin}er \textcolor{pink}{Lessing-Theater}.}}}\label{K_L02910-3h} berichtet. Aber am erſten Abend war eine blödſinnige \label{K_L02910-52v}\edtext{\begin{otherlanguage}{french}\textsc{\textcolor{green}{Première}{}\ledrightnote{{$\rightarrow$}\textcolor{green}{Gevatter Tod. Ein Märchen von der Menschheit. Drama in fünf Aufzügen}}}\end{otherlanguage}}{\lemma{\textnormal{\emph{Première}}}\Cendnote{\textnormal{von \textcolor{blue}{Eberhard König}s Fünfakter \emph{\textcolor{green}{Gevatter Tod.
                     Ein Märchen von der Menschheit}}}}}\label{K_L02910-52h} im \textcolor{pink}{Schauſpielhauſe}{}\ledrightnote{\textcolor{pink}{Schauspielhaus}}; und am zweiten konnte ich auch nicht hineingehen. Es ſteht in
               den Sternen geſchrieben, daß ich nie ein Stück von Dir auf der Bühne ſehen ſoll.\pend
           \pstart
           Viele treue Grüße! {\\[\baselineskip]}Dein {\\[\baselineskip]}\spacefill\mbox{Paul Goldmann.}\pend
           \leftskip=0em{}\endnumbering\briefempfaengerindex{Schnitzler, Arthur@\textsc{Schnitzler, Arthur}!zzzGoldmann, Paul@\emph{von Paul Goldmann}!1900-04-131@{13. 4. {[}1900{]}}|)be}\mylabel{h}\begin{anhang}\end{anhang}\normalsize

\doendnotes{C}
\bigskip
\vfill

\clearpage

\footnotesize

\lohead{\textsc{register}}

% Definiere theindex-Environment komplett neu ohne reledmac
\makeatletter
\renewenvironment{theindex}{%
  \section*{\indexname}%
  \setlength{\parindent}{0pt}%
  \setlength{\parskip}{0pt plus 0.3pt}%
  \let\item\@idxitem
}{%
  \clearpage
}
\makeatother

\IfFileExists{\jobname-pw.ind}{\input{\jobname-pw.ind}}{}

\end{document}

      