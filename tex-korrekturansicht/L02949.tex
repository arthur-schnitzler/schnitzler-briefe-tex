%% latex-korrekturansicht-vorspann.tex
%% Vorspann für die Korrekturansicht.
%% Lädt die gemeinsame Datei latex-vorspann.tex mit gesetztem Schalter.

\newif\ifkorrekturansicht
\korrekturansichttrue

\input{../tex-inputs/latex-vorspann}


\renewcommand{\erwaehntePersonen}{Personen: André Rivoire, Felix Salten}
\renewcommand{\erwaehnteInstitutionen}{Institutionen: Deutsches Theater Berlin, Volkstheater}
\renewcommand{\erwaehnteOrte}{Orte: Amerika, Berlin, Dresden, Hamburg, Prag, Wien}
\renewcommand{\erwaehnteWerke}{Werke: Anatol, Burgtheater (»Das weite Land«, Tragikkomödie in fünf Akten von Arthur Schnitzler. – Zum erstenmal am 14. Oktober 1911), Burgtheater. »Das weite Land.« Tragikomödie von Arthur Schnitzler, Das weite Land. Tragikomödie in fünf Akten, Der Schleier der Beatrice. Schauspiel in fünf Akten, Der gute König Dagobert. Lustspiel in vier Aufzügen, Die Zeit, Le Bon Roi Dagobert, Pester Lloyd}
\section[ Arthur Schnitzler an Felix Salten, 20. 10. 1911]{Arthur Schnitzler an Felix Salten, 20. 10. 1911}
\nopagebreak\mylabel{v}
\rehead{ }\normalsize\beginnumbering\briefempfaengerindex{Salten, Felix@\textsc{Salten, Felix}!zzzSchnitzler, Arthur@\emph{von Arthur Schnitzler}!1911-10-201@{20. 10. 1911}|(be}
\toendnotes[C]{\smallbreak\pagebreak[2]}\Standort{DLA, A:Schnitzler, HS.NZ85.1.1751.}
\physDesc{Brief, maschinenschriftliche Abschrift, 1 Blatt, 1 Seite, 1717 Zeichen
\newline{}maschinell
\newline{}Ordnung: mit schwarzer Tinte Vermerk »\textsc{Salten}« }
\buchAbdrucke{\weitereDrucke{Arthur Schnitzler: \emph{Briefe 1875–1912}. Hg. Therese Nickl und Heinrich Schnitzler. Frankfurt am Main: \emph{S. Fischer} 1981, S. 675–676.} }\toendnotes[C]{\smallbreak}
\pstart
           \raggedleft{}{\pb}\textcolor{pink}{Wien}{}\ledrightnote{\textcolor{pink}{Wien}}, 20. 10. 1911.\pend
           
\pstart{}Lieber,\pend
\pstart
           Ihre zwei \label{K_L02949-1v}\edtext{\textcolor{green}{Feuilletons}{}\ledrightnote{{$\rightarrow$}\textcolor{green}{Burgtheater (»Das weite Land«, Tragikkomödie in fünf Akten von Arthur Schnitzler. – Zum erstenmal am 14. Oktober 1911)}{\newline}{$\rightarrow$}\textcolor{green}{Burgtheater. »Das weite Land.« Tragikomödie von Arthur Schnitzler}}}{\lemma{\textnormal{\emph{Feuilletons}}}\Cendnote{\textnormal{\textcolor{blue}{Felix Salten}: \emph{\textcolor{green}{Burgtheater (»Das weite Land«, Tragikomödie in fünf Akten
                        von Arthur Schnitzler. – Zum erstenmal am 14. Oktober 1911)}}. In: \emph{\textcolor{green}{Die Zeit}}, Jg. 10, Nr. 3254, 15. 11. 1911, S. 1–3; \textcolor{blue}{Felix Salten}: \emph{\textcolor{green}{Burgtheater. »Das weite Land.« Tragikomödie von Arthur Schnitzler}}. In:
                        \emph{\textcolor{green}{Pester Lloyd}}, Jg. 58, Nr. 246, 17. 11. 1911, Morgenblatt, S. 1–2.}}}\label{K_L02949-1h}
               sind – muss man es erst sagen – sehr schoen. In Hinsicht auf sehr Wesentliches aber
               bin ich voellig anderer Ansicht, muss es sein, nicht nur weil ich das \textcolor{green}{Stueck}{}\ledrightnote{{$\rightarrow$}\textcolor{green}{Das weite Land. Tragikomödie in fünf Akten}} geschrieben habe, sondern weil ich
               zu der ganzen Frage der ethischen Werturteile, ueber Figuren innerhalb von
               Kunstwerken offenbar anders stehe wie Sie.\pend
           
\pstart
           Darf ich Ihnen ein verwunderliches Missverstaendnis aufklaeren{[},{]}
               das Ihr \textcolor{green}{Feuilleton}{}\ledrightnote{{$\rightarrow$}\textcolor{green}{Burgtheater. »Das weite Land.« Tragikomödie von Arthur Schnitzler}} im »\textcolor{green}{Lloyd}{}\ledrightnote{\textcolor{green}{Pester Lloyd}}« enthaelt? \textcolor{green}{Hofreiter}{}\ledrightnote{{$\rightarrow$}\textcolor{green}{Das weite Land. Tragikomödie in fünf Akten}} denkt nicht daran am Schluss des
                  \textcolor{green}{Stueck}{}\ledrightnote{{$\rightarrow$}\textcolor{green}{Das weite Land. Tragikomödie in fünf Akten}}es »\textcolor{green}{ein braver Kindesvater}{}\ledrightnote{{$\rightarrow$}\textcolor{green}{Burgtheater. »Das weite Land.« Tragikomödie von Arthur Schnitzler}}« zu werden, so
               wenig ich daran gedacht habe, das irgendwen glauben zu machen. Und es liegt nicht der
               leiseste Grund vor{[},{]} mir so etwas, was wirklich eine Banalitaet waere, zuzumuten.
               (Ausser bei Ihnen habe ich diese Zumutung nur unter Dutzenden ein einziges Mal
               gefunden). Erinnern Sie sich nur: \textcolor{green}{Genia}{}\ledrightnote{{$\rightarrow$}\textcolor{green}{Das weite Land. Tragikomödie in fünf Akten}} in ihrem letzten Gespraech mit \textcolor{green}{Hofreiter}{}\ledrightnote{{$\rightarrow$}\textcolor{green}{Das weite Land. Tragikomödie in fünf Akten}} besinnt sich ploetzlich: {[}»{]}\textcolor{green}{Percy kommt}{}\ledrightnote{{$\rightarrow$}\textcolor{green}{Das weite Land. Tragikomödie in fünf Akten}}«. Darauf er: »\textcolor{green}{Den erwart ich noch – denn die
                  Andern – na! (Handbewegung)}{}\ledrightnote{{$\rightarrow$}\textcolor{green}{Das weite Land. Tragikomödie in fünf Akten}}«. Er ist also jedenfalls entschlossen ihn zu
               erwarten; und dass er dann, wenn die Stimme \textcolor{green}{Percys}{}\ledrightnote{{$\rightarrow$}\textcolor{green}{Das weite Land. Tragikomödie in fünf Akten}} im Garten toent, so weit bewegt ist (gerade in der Empfindung: nun
               ist das auch zu Ende), um leise aufzuwimmern, \label{T_L02949-1v}\edtext{das}{\lemma{\textnormal{\emph{das}}}\Cendnote{\textnormal{In der Vorlage steht »dass«.}}}\label{T_L02949-1h} ist meines Erachtens kein Anlass
               zu vermuten, dass damit eine Art innerer Umkehr eingeleitet oder angedeutet sein
               sollte. Ich war himmelweit davon entfernt ein solches
                  Missverstae{[}n{]}dnis auch nur fuer moeglich zu halten. (Sonst
               haette ich \textcolor{green}{Hofreiter}{}\ledrightnote{{$\rightarrow$}\textcolor{green}{Das weite Land. Tragikomödie in fünf Akten}} am
               Schlusse ausrufen lassen: »Nun auf nach \textcolor{pink}{Amerika}{}\ledrightnote{\textcolor{pink}{Amerika}}«). \pend
           
\pstart
           Naechsten fahre ich ueber \textcolor{pink}{Prag}{}\ledrightnote{\textcolor{pink}{Prag}}, \textcolor{pink}{Dresden}{}\ledrightnote{\textcolor{pink}{Dresden}}{ }\label{K_L02949-2v}\edtext{nach \textcolor{pink}{Berlin}{}\ledrightnote{\textcolor{pink}{Berlin}} und \textcolor{pink}{Hamburg}{}\ledrightnote{\textcolor{pink}{Hamburg}}}{\lemma{\textnormal{\emph{nach Berlin und Hamburg}}}\Cendnote{\textnormal{In \textcolor{pink}{Berlin} kam \textcolor{blue}{Schnitzler} am 2. 11. 1911 an. Am
                     5. 11. 1911
                  reiste er weiter nach \textcolor{pink}{Hamburg}, wo er bis zum
                     9. 11. 1911
                  blieb.}}}\label{K_L02949-2h}, dort »\textcolor{green}{Beatrice}{}\ledrightnote{\textcolor{green}{Der Schleier der Beatrice. Schauspiel in fünf Akten}}«, »\textcolor{green}{Weites Land}{}\ledrightnote{\textcolor{green}{Das weite Land. Tragikomödie in fünf Akten}}«, »\textcolor{green}{Anatol}{}\ledrightnote{\textcolor{green}{Anatol}}« zu sehen. Wann ist die \label{K_L02949-3v}\edtext{\textcolor{green}{Dagobert}{}\ledrightnote{\textcolor{green}{Der gute König Dagobert. Lustspiel in vier Aufzügen}}-Generalprobe}{\lemma{\textnormal{\emph{Dagobert-Generalprobe}}}\Cendnote{\textnormal{\textcolor{blue}{Salten} hatte das Stück \emph{\textcolor{green}{Le Bon Roi Dagobert}} von \textcolor{blue}{André Rivoire} auf Deutsch bearbeitet. Die Uraufführung hatte die \textcolor{green}{Übersetzung} am 19. 1. 1910 am \emph{\textcolor{brown}{Deutschen Theater}} in \textcolor{pink}{Berlin} erlebt.
                  In \textcolor{pink}{Wien} fand die Premiere am 18. 11. 1911 am \emph{\textcolor{brown}{Deutschen Volkstheater}} statt, die Generalprobe am Vortag. \textcolor{blue}{Schnitzler} besuchte erst
                  die Aufführung am 5. 12. 1911.}}}\label{K_L02949-3h}? Darf man ihr beiwohnen?\pend
           
\pstart
           Auf baldiges Wiedersehen. {\\[\baselineskip]}herzlichst Ihr {\\[\baselineskip]}\pend
           \leftskip=0em{}{\bigskip}
\pstart
           \noindent{}Felix Salten\pend
           
\pstart
           (\textcolor{green}{Weites Land}{}\ledrightnote{\textcolor{green}{Das weite Land. Tragikomödie in fünf Akten}})\pend
           \endnumbering\briefempfaengerindex{Salten, Felix@\textsc{Salten, Felix}!zzzSchnitzler, Arthur@\emph{von Arthur Schnitzler}!1911-10-201@{20. 10. 1911}|)be}\mylabel{h}  \normalsize

\doendnotes{C}
\bigskip
\vfill

\clearpage

\footnotesize

\lohead{\textsc{register}}

% Definiere theindex-Environment komplett neu ohne reledmac
\makeatletter
\renewenvironment{theindex}{%
  \section*{\indexname}%
  \setlength{\parindent}{0pt}%
  \setlength{\parskip}{0pt plus 0.3pt}%
  \let\item\@idxitem
}{%
  \clearpage
}
\makeatother

\IfFileExists{\jobname-pw.ind}{\input{\jobname-pw.ind}}{}

\end{document}

      