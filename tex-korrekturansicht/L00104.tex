%% latex-korrekturansicht-vorspann.tex
%% Vorspann für die Korrekturansicht.
%% Lädt die gemeinsame Datei latex-vorspann.tex mit gesetztem Schalter.

\newif\ifkorrekturansicht
\korrekturansichttrue

\input{../tex-inputs/latex-vorspann}


               \section[Arthur Schnitzler an Hugo von Hofmannsthal, 14. 7. 1892]{ Arthur Schnitzler an Hugo von Hofmannsthal, 14. 7. 1892}\nopagebreak\mylabel{v}\rehead{ }\normalsize\beginnumbering\briefempfaengerindex{Hofmannsthal, Hugo von@\textsc{Hofmannsthal, Hugo von}!zzzSchnitzler, Arthur@\emph{von Arthur Schnitzler}!1892-07-141@{14. 7. 1892}|(be} \toendnotes[C]{\smallbreak\pagebreak[2]} \Standort{FDH, Hs-30885,21.}
\physDesc{Brief, 1 Blatt, 4 Seiten
\newline{}Handschrift: schwarze Tinte, deutsche Kurrent\newline{}Ordnung: von Schnitzler auf der
                                  ersten Seite mutmaßlich bei der Durchsicht der Korrespondenz 1929 mit Bleistift datiert »14. 7. 92« }\buchAbdrucke{\weitereDrucke{Hugo von Hofmannsthal, Arthur Schnitzler: \emph{Briefwechsel}. Hg. Therese Nickl und Heinrich Schnitzler. Frankfurt am Main: \emph{S. Fischer} 1964, S. 22.} }\toendnotes[C]{\smallbreak}\pstart{}{\pb}Lieber Hugo,\pend\pstart
           von \textcolor{blue}{\textsc{Salten}}{}\ledrightnote{\textcolor{blue}{Felix Salten}} erfahre
                    ich, daſs Ihr \textcolor{blue}{Vater}{}\ledrightnote{→\textcolor{blue}{Hugo August von Hofmannsthal}} krank war, aber
                    bereits wiederhergeſtellt iſt. Hoffentlich erholen Sie ſich zugleich von Ihrer
                        Verſti{\geminationm}ung und Abſpa{\geminationn}ung und verbringen den ko{\geminationm}enden So{\geminationm}er und
                    Herbſt in ſo reicher Fülle des I{\geminationn}ern und Äußern,
                    wie ichs Ihnen von Herzen wünſche. –\pend
           \pstart
           Geſtern ſtarb mein \textcolor{blue}{Großvater}{}\ledrightnote{→\textcolor{blue}{Phillip Markbreiter}}; {\pb}in wenigen Tagen reiſen meine \textcolor{blue}{Eltern}{}\ledrightnote{→\textcolor{blue}{Johann Schnitzler}{\newline}→\textcolor{blue}{Louise Schnitzler}} ab, und ich übernehme die Praxis meines
                        \textcolor{blue}{Papa}{}\ledrightnote{→\textcolor{blue}{Johann Schnitzler}}.\pend
           \pstart
           Seit einiger Zeit bring ich es zuwege, auch nachts literariſch zu arbeiten, und
                    ich hoffe, meine angefangenen Sachen werden trotz anderweitiger Thätigkeit wohl
                    fortſchreiten können.\pend
           \pstart
           – \textcolor{blue}{Hebbel}{}\ledrightnote{\textcolor{blue}{Friedrich Hebbel}}s \textcolor{green}{Briefe}{}\ledrightnote{\textcolor{green}{Briefwechsel mit Freunden und berühmten Zeitgenossen}} leſe ich jetzt, \textcolor{blue}{Leſſing}{}\ledrightnote{\textcolor{blue}{Gotthold Ephraim Lessing}}’s \textcolor{green}{Leben}{}\ledrightnote{→\textcolor{green}{G. E. Lessings Leben}} von ſeinem
                        \textcolor{blue}{Bruder}{}\ledrightnote{→\textcolor{blue}{Karl Gotthelf Lessing}} geſchildert, \textcolor{green}{Annalen}{}\ledrightnote{\textcolor{green}{Tag- und Jahreshefte}} von \textcolor{blue}{Goethe}{}\ledrightnote{\textcolor{blue}{Johann Wolfgang von Goethe}}. {\pb}\textcolor{blue}{Hebbel}{}\ledrightnote{\textcolor{blue}{Friedrich Hebbel}} war wohl nach \textcolor{blue}{Goethe}{}\ledrightnote{\textcolor{blue}{Johann Wolfgang von Goethe}} der größte Geiſt, den die Deutſchen in dem Jahrhundert
                    gehabt haben; manchmal ko{\geminationm}t mir vor, daſs man ihn
                    vor \textcolor{blue}{\textsc{Nietz}ſche}{}\ledrightnote{\textcolor{blue}{Friedrich Nietzsche}} wird ne{\geminationn}en müſſen. Ich bin jetzt bei der Periode ſeines
                    Lebens, wo er auf der Verlegerſuche iſt und auf \textcolor{blue}{Gutzkow}{}\ledrightnote{\textcolor{blue}{Karl Gutzkow}}, \textcolor{blue}{Laube}{}\ledrightnote{\textcolor{blue}{Heinrich Laube}}, \textcolor{blue}{Mundt}{}\ledrightnote{\textcolor{blue}{Theodor Mundt}}, \textcolor{blue}{Körner}{}\ledrightnote{\textcolor{blue}{Christian Gottfried Körner}},
                    zuweilen wohl auch auf \textcolor{blue}{Schiller}{}\ledrightnote{\textcolor{blue}{Friedrich von Schiller}}{ }ſchimpft. Er
                    hat aber auch noch manches andre zu ſagen. – Wiſſen Sie, daſs er eine {\pb}\textcolor{blue}{Jungfrau von Orleans}{}\ledrightnote{→\textcolor{blue}{Jeanne d’ Arc}}{ }ſchreiben
                    wollte? –\pend
           \pstart
           Von \textcolor{blue}{Richard}{}\ledrightnote{→\textcolor{blue}{Richard Beer-Hofmann}} hör ich nichts. Sie? –\pend
           \pstart
           Von Ihnen hoffe ich bald ſchönes und gutes zu erfahren; empfehlen Sie mich
                    bitte den Ihren aufs wärmſte.{\\[\baselineskip]}Ihr\hspace*{3.5em}\spacefill\mbox{Arthur}\pend
           \leftskip=0em{}\pstart
           14. 7. 92.\pend
           \pstart
           \textcolor{pink}{Wien}{}\ledrightnote{\textcolor{pink}{Wien}}.\pend
           \endnumbering\briefempfaengerindex{Hofmannsthal, Hugo von@\textsc{Hofmannsthal, Hugo von}!zzzSchnitzler, Arthur@\emph{von Arthur Schnitzler}!1892-07-141@{14. 7. 1892}|)be}\mylabel{h}  \normalsize

\doendnotes{C}
\bigskip
\vfill

\clearpage

\footnotesize

\lohead{\textsc{register}}

% Definiere theindex-Environment komplett neu ohne reledmac
\makeatletter
\renewenvironment{theindex}{%
  \section*{\indexname}%
  \setlength{\parindent}{0pt}%
  \setlength{\parskip}{0pt plus 0.3pt}%
  \let\item\@idxitem
}{%
  \clearpage
}
\makeatother

\IfFileExists{\jobname-pw.ind}{\input{\jobname-pw.ind}}{}

\end{document}

      