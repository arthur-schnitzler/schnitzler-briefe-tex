%% latex-korrekturansicht-vorspann.tex
%% Vorspann für die Korrekturansicht.
%% Lädt die gemeinsame Datei latex-vorspann.tex mit gesetztem Schalter.

\newif\ifkorrekturansicht
\korrekturansichttrue

\input{../tex-inputs/latex-vorspann}


\renewcommand{\erwaehntePersonen}{Personen: Johann Wolfgang von Goethe, Olga Schnitzler, Stefan Zweig}
\renewcommand{\erwaehnteOrte}{Orte: Sternwartestraße 71, Weimar, Wien, Währinger Cottage}
\renewcommand{\erwaehnteWerke}{Werke: Die Hirtenflöte. Novelle}
\section[Stefan Zweig an Arthur Schnitzler, 29. 8. 1911]{Stefan Zweig an Arthur Schnitzler, 29. 8. 1911}
\nopagebreak\mylabel{v}
\rehead{ }\normalsize\beginnumbering\briefempfaengerindex{Schnitzler, Arthur@\textsc{Schnitzler, Arthur}!zzzZweig, Stefan@\emph{von Stefan Zweig}!1911-08-291@{29. 8. 1911}|(be}
\toendnotes[C]{\smallbreak\pagebreak[2]}\Standort{CUL, Schnitzler, B 118.}
\physDesc{Bildpostkarte, 1 Blatt, 2 Seiten, 458 Zeichen
\newline{}Handschrift: schwarze Tinte, lateinische Kurrent
\newline{}Versand: Stempel: »\nobreak{}\oindex{Weimar@\textbf{Weimar}, \emph{A.ADM3}|pwk}Weimar, 29. 8. 11, 7—8 N\nobreak{}«.  
\newline{}Schnitzler: mit Bleistift »\textsc{Zweig}« }\toendnotes[C]{\smallbreak}\pstart{}{\pb}D\textsuperscript{r}
                  Artur Schnitzler\pend{}\pstart{}\textcolor{pink}{Wien – Cottage}{}\ledrightnote{\textcolor{pink}{Währinger Cottage}}\pend{}\pstart{}\textcolor{pink}{Sternwartestrasse 71}{}\ledrightnote{\textcolor{pink}{Sternwartestraße 71}}\pend{}
{\bigskip}
\pstart
           \noindent{}{\pb}\textcolor{gray}{\textbf{Weimar, Goethes Gartenhaus\textcolor{red}{\textsuperscript{\textbf{KEY}}}.}}\pend
           \stanza{}\textcolor{gray}{\textbf{Übermüthig siehts nicht aus}}\textcolor{gray}{\textbf{Dieses stille Gartenhaus\textcolor{red}{\textsuperscript{\textbf{KEY}}}}}\textcolor{gray}{\textbf{Allen die darin verkehrt}}\textcolor{gray}{\textbf{Ward ein guter Muth bescheert}}\textcolor{gray}{\textbf{\textcolor{blue}{Goethe}{}\ledrightnote{\textcolor{blue}{Johann Wolfgang von Goethe}}{ }1828}}\stanzaend{}
\pstart
           \noindent{}{\pb}Verehrter Herr Doktor,
               ich weiss nicht, ob Sie schon einmal hier waren: man kanns auch als Sommeraufenthalt
               nehmen, statt als blosse Reverenzreise, so wundervoll still ist's jetzt in den Gängen
               an der Ilm\textcolor{red}{\textsuperscript{\textbf{KEY}}}. Ich grüsse Sie und Ihre liebe \textcolor{blue}{Frau}{}\ledrightnote{{$\rightarrow$}\textcolor{blue}{Olga Schnitzler}} herzlichst \pend
           
\pstart
           in alter Ergebenheit{\\[\baselineskip]}\spacefill\mbox{Stefan Zweig}\pend
           \leftskip=0em{}
\pstart
           \noindent{}Wie \uline{wundervoll} ist Ihre \textcolor{green}{Hirtenflöte}{}\ledrightnote{\textcolor{green}{Die Hirtenflöte. Novelle}}! Ich musste mir {\pb}es auf die Reise mitnehmen, um es beim zweiten Lesen noch inniger zu
                  geniessen.\pend
           \endnumbering\briefempfaengerindex{Schnitzler, Arthur@\textsc{Schnitzler, Arthur}!zzzZweig, Stefan@\emph{von Stefan Zweig}!1911-08-291@{29. 8. 1911}|)be}\mylabel{h}
\begin{anhang}
\end{anhang}\normalsize

\doendnotes{C}
\bigskip
\vfill

\clearpage

\footnotesize

\lohead{\textsc{register}}

% Definiere theindex-Environment komplett neu ohne reledmac
\makeatletter
\renewenvironment{theindex}{%
  \section*{\indexname}%
  \setlength{\parindent}{0pt}%
  \setlength{\parskip}{0pt plus 0.3pt}%
  \let\item\@idxitem
}{%
  \clearpage
}
\makeatother

\IfFileExists{\jobname-pw.ind}{\input{\jobname-pw.ind}}{}

\end{document}

      