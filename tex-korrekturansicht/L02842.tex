%% latex-korrekturansicht-vorspann.tex
%% Vorspann für die Korrekturansicht.
%% Lädt die gemeinsame Datei latex-vorspann.tex mit gesetztem Schalter.

\newif\ifkorrekturansicht
\korrekturansichttrue

\input{../tex-inputs/latex-vorspann}


               \section[ Paul Goldmann an Arthur Schnitzler, 10. 3. {[}1898{]}]{Paul Goldmann an Arthur Schnitzler, 10. 3. {[}1898{]}}\nopagebreak\mylabel{v}\rehead{ }\normalsize\beginnumbering\briefempfaengerindex{Schnitzler, Arthur@\textsc{Schnitzler, Arthur}!zzzGoldmann, Paul@\emph{von Paul Goldmann}!1898-03-102@{10. 3. {[}1898{]}}|(be} \toendnotes[C]{\smallbreak\pagebreak[2]} \Standort{DLA, A:Schnitzler, HS.NZ85.1.3168.}
\physDesc{Brief, 1 Blatt, 3 Seiten
\newline{}Handschrift: blaue Tinte, lateinische Kurrent
\newline{}Schnitzler: mit Bleistift das Jahr »98« vermerkt }\toendnotes[C]{\smallbreak}\pstart
           \noindent{}{\pb}\textcolor{gray}{\textbf{\textbf{\textcolor{brown}{Frankfurter Zeitung}{}\ledrightnote{\textcolor{brown}{Frankfurter Zeitung}}}}}\pend
           \pstart
           \textcolor{gray}{\textbf{(\textcolor{brown}{\begin{otherlanguage}{french}Gazette de Francfort\end{otherlanguage}}{}\ledrightnote{\textcolor{brown}{Frankfurter Zeitung}}).}}\pend
           \pstart
           \textcolor{gray}{\textbf{\textbf{\begin{otherlanguage}{french}Fondateur M.\end{otherlanguage}{ }\textcolor{blue}{L. Sonnemann}{}\ledrightnote{\textcolor{blue}{Leopold Sonnemann}}.}}}\pend
           \pstart
           \begin{otherlanguage}{french}\textcolor{gray}{\textbf{Journal politique, financier,}}\end{otherlanguage}\pend
           \pstart
           \begin{otherlanguage}{french}\textcolor{gray}{\textbf{commercial et littéraire.}}\end{otherlanguage}\pend
           \pstart
           \begin{otherlanguage}{french}\textcolor{gray}{\textbf{\textbf{Paraissant trois fois par jour.}}}\end{otherlanguage}\pend
           \pstart
           \begin{otherlanguage}{french}\textcolor{gray}{\textbf{\textbf{Bureau à \textcolor{pink}{Paris}{}\ledrightnote{\textcolor{pink}{Paris}}}}}\end{otherlanguage}\hfill \textsc{\textcolor{pink}{Paris}{}\ledrightnote{\textcolor{pink}{Paris}}}, 10. März.\pend
           \pstart
           \begin{otherlanguage}{french}\textcolor{gray}{\textbf{\textbf{\textcolor{pink}{10 Rue de la Bourse}{}\ledrightnote{\textcolor{pink}{rue de la Bourse}}.}}}\end{otherlanguage}\pend
           \pstart
           Die Geographie, mein theurer Freund, iſ niemals Deine
               ſtarke Seite geweſen. Du weißt wieder einmal nicht, wo \textsc{\textcolor{pink}{Wien}{}\ledrightnote{\textcolor{pink}{Wien}}} liegt. Es gehört eine erſtaunliche Unſchuld des Gemüthes dazu, um zu behaupten,
               daß der nächſte Weg von \textsc{\textcolor{pink}{Paris}{}\ledrightnote{\textcolor{pink}{Paris}}} nach \textsc{\textcolor{pink}{China}{}\ledrightnote{\textcolor{pink}{China}}} über 
                  \textcolor{pink}{Wien}{}\ledrightnote{\textcolor{pink}{Wien}}
                führt. Aber wenn Du nach \textsc{\textcolor{pink}{Genua}{}\ledrightnote{\textcolor{pink}{Genua}}} kämſt, ſo würdeſt Du damit zeigen, daß Du ein\pend
           \pstart
           braver Burſch biſt.
               \introOben{}(\textsc{N. B. \textcolor{pink}{Genua}{}\ledrightnote{\textcolor{pink}{Genua}}} iſt eine \textcolor{pink}{italien}{}\ledrightnote{\textcolor{pink}{Italien}}iſche \textcolor{pink}{Hafenſtadt}{}\ledrightnote{→\textcolor{pink}{Genua}}).\introOben{}\pend
           \pstart
           
               Und noch eine Bitte. Haſt Du in Deiner Umgebung Jemanden, {\pb}der mir eine
               wirkſame Empfehlung an Irgendwen in \textsc{\textcolor{pink}{China}{}\ledrightnote{\textcolor{pink}{China}}} oder \textsc{\textcolor{pink}{Japan}{}\ledrightnote{\textcolor{pink}{Japan}}} geben könnte? Ich bekomme zwar ſchon genug Empfehlungen mit, aber eine mehr
               kann nicht ſchaden, und vielleicht iſt gerade dieſe die eigentlich\strikeout{\textcolor{gray}{e}} nützliche.\pend
           \pstart
           Du glaubſt, daß Du mich beneideſt? Ich glaube, daß Du mich nicht beneiden ſollſt.
               Ruhelos und friedlos in der Welt herumirren? \strikeout{I\textcolor{gray}{n}s} Ins Weite gehen ſtatt in die Höhe, um ſich
               vorzulügen, daß man {\pb}vorwärts kommt? Ich finde darin
               nichts Beneidenswerthes. Überdies werde ich mich gräßlich \strikeout{bl} blamiren. Endlich werde ich \strikeout{\textcolor{gray}{a}} am Fieber \introOben{}oder\introOben{} an der Peſt \strikeout{\textcolor{gray}{d}\textcolor{gray}{×}\textcolor{gray}{e}} ſterben oder irgendwo an der großen Mauer \strikeout{erno}
               ermordet werden.\pend
           \pstart
           Bitte, liebſter Freund, ſchreib’ mir nach \textcolor{pink}{Frankfurt}{}\ledrightnote{\textcolor{pink}{Frankfurt am Main}} an die Adreſſe meiner \textcolor{blue}{Mutter}{}\ledrightnote{→\textcolor{blue}{Clementine Goldmann}} (Frau \textsc{\textcolor{blue}{Clementine Goldmann}{}\ledrightnote{\textcolor{blue}{Clementine Goldmann}}}, \textcolor{pink}{\textsc{Rossertstraße} 15}{}\ledrightnote{\textcolor{pink}{Rossertstraße}}). Ich gehe wahrſcheinlich ſchon
               nächſter Tage dahin ab.\pend
           \pstart
           Herzlichſt {\\[\baselineskip]}Dein {\\[\baselineskip]}\spacefill\mbox{Paul Goldmnn}\pend
           \leftskip=0em{}\endnumbering\briefempfaengerindex{Schnitzler, Arthur@\textsc{Schnitzler, Arthur}!zzzGoldmann, Paul@\emph{von Paul Goldmann}!1898-03-102@{10. 3. {[}1898{]}}|)be}\mylabel{h}  \normalsize

\doendnotes{C}
\bigskip
\vfill

\clearpage

\footnotesize

\lohead{\textsc{register}}

% Definiere theindex-Environment komplett neu ohne reledmac
\makeatletter
\renewenvironment{theindex}{%
  \section*{\indexname}%
  \setlength{\parindent}{0pt}%
  \setlength{\parskip}{0pt plus 0.3pt}%
  \let\item\@idxitem
}{%
  \clearpage
}
\makeatother

\IfFileExists{\jobname-pw.ind}{\input{\jobname-pw.ind}}{}

\end{document}

      