%% latex-korrekturansicht-vorspann.tex
%% Vorspann für die Korrekturansicht.
%% Lädt die gemeinsame Datei latex-vorspann.tex mit gesetztem Schalter.

\newif\ifkorrekturansicht
\korrekturansichttrue

\input{../tex-inputs/latex-vorspann}


               \section[Paul Goldmann an Arthur Schnitzler, 24. 4. {[}1895{]}]{ Paul Goldmann an Arthur Schnitzler, 24. 4. {[}1895{]}}\nopagebreak\mylabel{v}\rehead{ }\normalsize\beginnumbering\briefempfaengerindex{Schnitzler, Arthur@\textsc{Schnitzler, Arthur}!zzzGoldmann, Paul@\emph{von Paul Goldmann}!1895-04-242@{24. 4. {[}1895{]}}|(be} \toendnotes[C]{\smallbreak\pagebreak[2]} \Standort{DLA, A:Schnitzler, HS.NZ85.1.3165.}
\physDesc{Brief, 2 Blätter, 8 Seiten
\newline{}Handschrift: blaue Tinte, deutsche Kurrent
\newline{}Schnitzler: 1) mit Bleistift das Jahr »95« vermerkt 2) mit rotem Buntstift sechs Unterstreichungen}\toendnotes[C]{\smallbreak}\pstart
           \raggedleft{}{\pb}\textsc{\textcolor{pink}{Frankfurt}{}\ledrightnote{\textcolor{pink}{Frankfurt am Main}}}{ }24. April.\pend
           \pstart\center{}Mein lieber Freund,\pend\pstart
           Seit zehn Tagen bin ich in \textcolor{pink}{Frankfurt}{}\ledrightnote{\textcolor{pink}{Frankfurt am Main}} bei den
               Meinen. \textcolor{pink}{Deutſch}{}\ledrightnote{→\textcolor{pink}{Deutschland}}es Land,
               Frühling und Friede – das thut wohl. Aber drohend ſind die Zukunftsfragen da. Und ich
               war krank und lag einige Tage zu Bette{[}.{]}{ }{\pb}Dieſer Tage gehe ich nach \textsc{\textcolor{pink}{Paris}{}\ledrightnote{\textcolor{pink}{Paris}}} zurück. Will Dir nur von unterwegs einen Gruß ſenden. Aus \textsc{\textcolor{pink}{Paris}{}\ledrightnote{\textcolor{pink}{Paris}}} hörſt Du Näheres von mir.\pend
           \pstart
           \textsc{\textcolor{blue}{Herzl}{}\ledrightnote{\textcolor{blue}{Theodor Herzl}}} iſt ganz ſo ſchweigſam über das \label{K_L02734-1v}\edtext{Beiſammenſein mit Dir}{\lemma{\textnormal{\emph{Beiſammenſein mit Dir}}}\Cendnote{\textnormal{\textcolor{blue}{Theodor Herzl} hielt sich im März 1895 in \textcolor{pink}{Wien} auf.
                  Zwischen 26. 3. 1895 und
                     30. 3. 1895 sah er
                     \textcolor{blue}{Schnitzler} jeden Tag. Ein Konflikt
                  zwischen den beiden ist nicht bekannt.}}}\label{K_L02734-1h}. Iſt das nur ſeine eitle {\pb}\label{K_L02734-44v}\edtext{\textsc{Suffisance}}{\lemma{\textnormal{\emph{Suffisance}}}\Cendnote{\textnormal{französisch: Selbstgefälligkeit}}}\label{K_L02734-44h}?
               Oder habt Ihr was gehabt? Wie hat er Dir überhaupt gefallen?\pend
           \pstart
           Ich \substVorne{}\textsuperscript{hö\textcolor{gray}{re,}}\substDazwischen{}höre,\substHinten{} Du wirſt erſt im Herbſt aufgeführt. Beſſer im Anfang, als am Ende der
               Saison. Am Beſten wäre es freilich, die \label{K_L02734-2v}\edtext{\textcolor{pink}{Berlin}{}\ledrightnote{\textcolor{pink}{Berlin}}er \textcolor{green}{Aufführung}{}\ledrightnote{→\textcolor{green}{Liebelei. Schauspiel in drei Akten}}}{\lemma{\textnormal{\emph{Berliner Aufführung}}}\Cendnote{\textnormal{Am 4. 2. 1896 feierte die \emph{\textcolor{green}{Liebelei}} am \emph{\textcolor{brown}{Deutschen
                     Theater}} in \textcolor{pink}{Berlin} Premiere.}}}\label{K_L02734-2h}{ }{\pb}ginge der \textcolor{pink}{Wien}{}\ledrightnote{\textcolor{pink}{Wien}}er
               voran. Publikum und Kritik ſind in \textcolor{pink}{Berlin}{}\ledrightnote{\textcolor{pink}{Berlin}} doch im
               Ganzen intelligenter. Ein \textcolor{pink}{Berlin}{}\ledrightnote{\textcolor{pink}{Berlin}}er Erfolg wäre
               für \textcolor{pink}{Wien}{}\ledrightnote{\textcolor{pink}{Wien}} beſtimmend, auch für den ewig zaudernden
                  \textcolor{blue}{\textcolor{brown}{Burgtheater}{}\ledrightnote{\textcolor{brown}{Burgtheater}}-Direktor}{}\ledrightnote{→\textcolor{blue}{Max Eugen Burckhard}}. (Wie ich hier höre,
               ſtrebt \textsc{\textcolor{blue}{Paul Lindau}{}\ledrightnote{\textcolor{blue}{Paul Lindau}}} nach \label{K_L02734-3v}\edtext{\textsc{\textcolor{blue}{Burckhardt}{}\ledrightnote{\textcolor{blue}{Max Eugen Burckhard}}s}{ }\textcolor{blue}{Nachfolger}{}\ledrightnote{→\textcolor{blue}{Paul Schlenther}}ſchaft}{\lemma{\textnormal{\emph{Burckhardts Nachfolgerſchaft}}}\Cendnote{\textnormal{\textcolor{blue}{Max Burckhardt} war als Jurist eine
                  überraschende Besetzung für die Leitung des \emph{\textcolor{brown}{Burgtheater}}s gewesen. Ablösegerüchte oder -wünsche bestanden von Anfang
                  an, doch konnte er sich bis 1898 halten. Nachfolger wurde
                     \textcolor{blue}{Paul Schlenther}.}}}\label{K_L02734-3h}). {\pb}Hier ein \label{K_L02734-4v}\edtext{\textcolor{green}{Stück}{}\ledrightnote{\textcolor{green}{Frauenlob. Lustspiel in drei Aufzügen}}}{\lemma{\textnormal{\emph{Stück}}}\Cendnote{\textnormal{vermutlich \emph{\textcolor{green}{ Frauenlob. Lustspiel in drei Aufzügen }}}}}\label{K_L02734-4h} von \textsc{\textcolor{blue}{Rudolf Lothar}{}\ledrightnote{\textcolor{blue}{Rudolf Lothar}}} geſehen. Es iſt unerhört, daß man dieſen \textcolor{blue}{Buben}{}\ledrightnote{→\textcolor{blue}{Rudolf Lothar}} nicht mit Fußtritten vom Theater jagt.\pend
           \pstart
           Haſt Du frohe Oſtern gehabt? Und wie gehts Dir? Du ſchreibſt mir wohl ein kurzes
               Wort, ohne meine {\pb}längere Antwort abzuwarten.\pend
           \pstart
           \textsc{\textcolor{blue}{Bahr}{}\ledrightnote{\textcolor{blue}{Hermann Bahr}}} hat alſo wieder einen \label{K_L02734-5v}\edtext{\textcolor{green}{Vortrag}{}\ledrightnote{→\textcolor{green}{Das junge Österreich [Vortrag]}}}{\lemma{\textnormal{\emph{Vortrag}}}\Cendnote{\textnormal{Am 13. 3. 1895 fand eine Veranstaltung des \emph{\textcolor{brown}{Vereins der Literaturfreunde}} statt, bei der \textcolor{blue}{Hermann Bahr} einen Vortrag mit dem Titel \emph{\textcolor{green}{Das junge Österreich}} hielt. \textcolor{blue}{Schnitzler}, dessen Kunstschaffen als
                     »abgethan« geschildert wurde, war empört, siehe A. S.: \emph{Tagebuch}, 14. 3. 1895}}}\label{K_L02734-5h} gehalten. Der Volksſänger der Moderne! Die \label{K_L02734-6v}\edtext{Brettl-Natur}{\lemma{\textnormal{\emph{Brettl-Natur}}}\Cendnote{\textnormal{abwertend; gemeint ist
                  ein Schauspieler, der nicht auf einer gezimmerten, sondern einer aus einfachen
                  Brettern zusammengefügten Bühne auftritt}}}\label{K_L02734-6h}, das iſt der Grund in dem Weſen
               des Kerls. Wie ich den immer
               mehr haſſe! Dieſer \textcolor{blue}{Mann}{}\ledrightnote{→\textcolor{blue}{Hermann Bahr}} von
               Geiſt, aber {\pb}ohne Kunſt, ohne Urtheil, ohne
               Gewiſſen! Merkſt Du, wie er ſich langſam in die \label{K_L02734-66v}\edtext{\textsc{Clique}}{\lemma{\textnormal{\emph{Clique}}}\Cendnote{\textnormal{Hier liegt eine positive Verwendung des Worts vor, das bei
                     \textcolor{blue}{Schnitzler} hingegen meist nur in einer
                  negativen Form vorkommt, insofern er nicht als Teil einer eingeschworenen Gruppe
                  von Literaten wahrgenommen werden mochte. }}}\label{K_L02734-66h} hineinſchleicht? In wenig Jahren
               hat er irgendwo ein officiöſes k. k. Literatur-Amt. Daß dieſes Rindvieh, der \textsc{\strikeout{A}}{ }\label{K_L02734-7v}\edtext{\textsc{\textcolor{blue}{\textcolor{green}{Necker}{}\ledrightnote{→\textcolor{green}{Junge Dichter}}}{}\ledrightnote{\textcolor{blue}{Moriz Necker}}}}{\lemma{\textnormal{\emph{Necker}}}\Cendnote{\textnormal{Die Veranstaltung wurde wohlwollend
                  von \textcolor{blue}{Moriz Necker} in der \emph{\textcolor{green}{Neuen Freien Presse}} besprochen, einschließlich
                  der überraschenden Volte, dass eine neue Kunstepoche entstehe und dass frühere \textcolor{pink}{Wien}er
                  Vertreter wie »\textcolor{blue}{Hermann Bahr}, Baron \textcolor{blue}{Torresani}, 
                     \textcolor{blue}{Beer-Hoffmann}« nur eine Übergangszeit repräsentiert hätten.
                  \textcolor{blue}{Schnitzler}s Name fällt in der Rezension nicht. [\textcolor{blue}{Moriz Necker]}: \emph{\textcolor{green}{Das junge Österreich}}. In: \emph{\textcolor{green}{Neue Freie Presse}}, Nr. 10.075, 14. 3. 1895, S. 5.}}}\label{K_L02734-7h}, Dich
               angreift, iſt ſelbſt{\pb}verſtändlich. \strikeout{Wenn Du} Daran daß Du die \strikeout{Och}{ } Ochſen ſtützig machſt, kannſt Du auch ſehen, daß
               Du Jemand biſt. Aber daß dieſer \label{K_L02734-88v}\edtext{Angriff in der »\textcolor{green}{Zeit}{}\ledrightnote{\textcolor{green}{Die Zeit. Wiener Wochenschrift}}«}{\lemma{\textnormal{\emph{Angriff in der »Zeit«}}}\Cendnote{\textnormal{Gemeint dürfte
                  nicht ein spezifischer Artikel sein – auch wenn \textcolor{blue}{Bahr} Gedanken davon in seiner
                  Rezension von \textcolor{blue}{Leopold von Andrian-Werburg}s \emph{\textcolor{green}{Der Garten der Erkenntnis}} (\textcolor{blue}{Hermann Bahr}: \emph{\textcolor{green}{Der Garten der Erkenntnis}}. In: \emph{\textcolor{green}{Die Zeit. Wiener Wochenschrift}}, Bd. 2, H. 24, 16. 3. 1895, S. 171–172) steht – sondern
                  eher die allgemeine Unmut ausdrücken, dass von einem Repräsentanten der Wochenschrift, die man auf
                  der eigenen Seite vermutete, Kritik kam.}}}\label{K_L02734-88h} ſteht, macht mir
               das Blut wallen. Wenn Ihr könnt, tretet den \textsc{\textcolor{blue}{Bahr}{}\ledrightnote{\textcolor{blue}{Hermann Bahr}}} noch bei Zeiten todt. Sonſt werdet Ihr viel Schlimmeres erleben{\dotsfour}\pend
           \pstart
           Grüß’ Dich Gott, mein lieber Freund!{\\[\baselineskip]}Dein {\\[\baselineskip]}\spacefill\mbox{Paul Goldmann}\pend
           \leftskip=0em{}\endnumbering\briefempfaengerindex{Schnitzler, Arthur@\textsc{Schnitzler, Arthur}!zzzGoldmann, Paul@\emph{von Paul Goldmann}!1895-04-242@{24. 4. {[}1895{]}}|)be}\mylabel{h}  \normalsize

\doendnotes{C}
\bigskip
\vfill

\clearpage

\footnotesize

\lohead{\textsc{register}}

% Definiere theindex-Environment komplett neu ohne reledmac
\makeatletter
\renewenvironment{theindex}{%
  \section*{\indexname}%
  \setlength{\parindent}{0pt}%
  \setlength{\parskip}{0pt plus 0.3pt}%
  \let\item\@idxitem
}{%
  \clearpage
}
\makeatother

\IfFileExists{\jobname-pw.ind}{\input{\jobname-pw.ind}}{}

\end{document}

      