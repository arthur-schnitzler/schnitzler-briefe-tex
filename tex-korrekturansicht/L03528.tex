%% latex-korrekturansicht-vorspann.tex
%% Vorspann für die Korrekturansicht.
%% Lädt die gemeinsame Datei latex-vorspann.tex mit gesetztem Schalter.

\newif\ifkorrekturansicht
\korrekturansichttrue

\input{../tex-inputs/latex-vorspann}


\renewcommand{\erwaehntePersonen}{Personen: Paul Goldmann, Olga Schnitzler, Elisabeth Steinrück}
\renewcommand{\erwaehnteOrte}{Orte: Berlin, Brühl, Dessauer Straße, Wien, Wörthersee, XIX., Döbling}
\renewcommand{\erwaehnteWerke}{}
\section[ Paul Goldmann an Olga Gussmann, 28. 5. {[}1901{]}]{Paul Goldmann an Olga Gussmann, 28. 5. {[}1901{]}}
\nopagebreak\mylabel{v}
\rehead{ }\normalsize\beginnumbering\briefempfaengerindex{Schnitzler, Olga@\textsc{Schnitzler, Olga}!zzzGoldmann, Paul@\emph{von Paul Goldmann}!1901-05-281@{28. 5. {[}1901{]}}|(be}
\toendnotes[C]{\smallbreak\pagebreak[2]}\Standort{DLA, A:Schnitzler, HS.NZ85.1.5247.}
\physDesc{Brief, 1 Blatt, 4 Seiten, 1750 Zeichen
\newline{}Handschrift: blaue Tinte, deutsche Kurrent
\newline{}Ordnung: mit Bleistift von \textcolor{blue}{Arthur Schnitzler} das
                                 Jahr »1901« vermerkt }\toendnotes[C]{\smallbreak}
\pstart
           \noindent{}\raggedleft{}{\pb}\textcolor{gray}{\textbf{\textcolor{pink}{DESSAUERSTRASSE 19}{}\ledrightnote{\textcolor{pink}{Dessauer Straße}}}}\pend
           
\pstart
           \textcolor{pink}{Berlin}{}\ledrightnote{\textcolor{pink}{Berlin}}, 28. Mai.\pend
           
\pstart\center{}Liebes Fräulein \textsc{Olga},\pend
\pstart
           Ich danke Ihnen für Ihren lieben Brief und freue mich, daß Alles \label{K_L03528-1v}\edtext{glücklich vorüber iſt und daß Sie wieder
                  geneſen}{\lemma{\textnormal{\emph{glücklich … geneſen}}}\Cendnote{\textnormal{Die taktlose Formulierung
                  bezieht sich darauf, dass eine Schwangerschaft von \textcolor{blue}{Olga Gussmann} am 10. 5. 1901 operativ beendet werden
                  musste.}}}\label{K_L03528-1h} ſind. Jetzt ſollen Sie ſich einen ſchönen Sommer machen und Liebe
               und Natur und alle Herrlichkeiten der Welt genießen. Dann wird auch eines Tages das
               kleine Haus in \textsc{\textcolor{pink}{Döbling}{}\ledrightnote{\textcolor{pink}{XIX., Döbling}}} kommen, mit \textsc{\textcolor{blue}{Arthur}{}\ledrightnote{}}, mit Kindern und mit {\pb}ſonſt noch all’ dem
               Guten, das darin ſein ſoll. Die Hauptſache iſt, ſich leben zu laſſen, –
               vorausgeſetzt, daß man auf der rechten Bahn iſt. Und ich denke, Sie ſind darauf.\pend
           
\pstart
           Auch haben Sie Recht, daß Sie ſich fürs Erſte nicht viel um Ihre Kunſt kümmern. Nur
               leben, leben, leben! Es hat, weiß Gott, mehr Sinn, ſich lieb zu haben, als Theater zu
                  ſpielen{\dotsfive}\pend
           
\pstart
           Ich werde Ende Juli, Anfang Auguſt nach dem \label{K_L03528-2v}\edtext{\textcolor{pink}{Wörther See}{}\ledrightnote{\textcolor{pink}{Wörthersee}}}{\lemma{\textnormal{\emph{Wörther See}}}\Cendnote{\textnormal{siehe Paul Goldmann an Olga Gussmann, 10. 5. [1901]}}}\label{K_L03528-2h}{ }{\pb}gehen. Denn ich will ruhig ſitzen, mich von der
               Sonne beſcheinen laſſen und kalt baden. Herumreiſen kann ich nicht – vor Allem, weil
               ich kein Geld habe. Wenn wir uns alſo ſehen wollen, müſſen Sie auch nach dem \textcolor{pink}{Wörtherſee}{}\ledrightnote{\textcolor{pink}{Wörthersee}} kommen. Kommen Sie nicht, ſo ſehe ich
               Sie hoffentlich auf der Rückreiſe in \textcolor{pink}{Wien}{}\ledrightnote{\textcolor{pink}{Wien}}{\dotsfive}\pend
           
\pstart
           Liebes Fräulein und liebe Freundin, ich danke Ihnen für alle die guten Worte, mit
               denen Sie mir zuſprechen. Sie haben mir wohl gethan, denn ich bin fürchterlich
               herunter. {\pb}Phyſiſch: denn ich habe mir in dieſem
               Winter zuviel zugemuthet, habe mein Gehirn überſpannt, und meine Nerven wollen gar
               nicht mehr mit. Moraliſch: denn ich habe einen Ekel vor meinem Beruf und vor meinem
               Leben, den ich Ihnen mit Worten überhaupt nicht begreiflich machen kann. Ich hätte
               Ihnen gern mehr und auch heiterer geſchrieben. Aber es geht nicht. Grüßen Sie \textsc{\textcolor{blue}{Arthur}{}\ledrightnote{}}, das Fräulein \textsc{\textcolor{blue}{Liesl}{}\ledrightnote{\textcolor{blue}{Elisabeth Steinrück}}} (der ich demnächſt ſchreiben werde) und ſeien Sie ſelbſt vielmals und
               herzlichſt gegrüßt von Ihrem ergebenen {\\}\spacefill\mbox{Dr. Paul Goldmann.}\pend
           \endnumbering\briefempfaengerindex{Schnitzler, Olga@\textsc{Schnitzler, Olga}!zzzGoldmann, Paul@\emph{von Paul Goldmann}!1901-05-281@{28. 5. {[}1901{]}}|)be}\mylabel{h}  \normalsize

\doendnotes{C}
\bigskip
\vfill

\clearpage

\footnotesize

\lohead{\textsc{register}}

% Definiere theindex-Environment komplett neu ohne reledmac
\makeatletter
\renewenvironment{theindex}{%
  \section*{\indexname}%
  \setlength{\parindent}{0pt}%
  \setlength{\parskip}{0pt plus 0.3pt}%
  \let\item\@idxitem
}{%
  \clearpage
}
\makeatother

\IfFileExists{\jobname-pw.ind}{\input{\jobname-pw.ind}}{}

\end{document}

      