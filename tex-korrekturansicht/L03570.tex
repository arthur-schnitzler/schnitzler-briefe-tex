%% latex-korrekturansicht-vorspann.tex
%% Vorspann für die Korrekturansicht.
%% Lädt die gemeinsame Datei latex-vorspann.tex mit gesetztem Schalter.

\newif\ifkorrekturansicht
\korrekturansichttrue

\input{../tex-inputs/latex-vorspann}


\renewcommand{\erwaehntePersonen}{Personen: Frieda Pollak, Felix Salten}
\renewcommand{\erwaehnteOrte}{Orte: Berlin, Hildesheim, Sternwartestraße 71, Tempelhaus (Hildesheim), Wien}
\renewcommand{\erwaehnteWerke}{}
\section[ Felix Salten an Arthur Schnitzler, 30. 3. 1921]{Felix Salten an Arthur Schnitzler, 30. 3. 1921}
\nopagebreak\mylabel{v}
\rehead{ }\normalsize\beginnumbering\briefempfaengerindex{Schnitzler, Arthur@\textsc{Schnitzler, Arthur}!zzzSalten, Felix@\emph{von Felix Salten}!1921-03-302@{30. 3. 1921}|(be}
\toendnotes[C]{\smallbreak\pagebreak[2]}\Standort{CUL, Schnitzler, B 89, B 2.}
\physDesc{Bildpostkarte, 221 Zeichen
\newline{}Handschrift: schwarze Tinte, lateinische Kurrent
\newline{}Versand: Stempel: »\nobreak{}\oindex{Hildesheim@\textbf{Hildesheim}, \emph{P.PPLA3}|pwk}Hildesheim 2 f, 30. 3. 21, 6–7 N\nobreak{}«.  
\newline{}Ordnung: 1) mit Bleistift von \textcolor{blue}{Frieda Pollak} (?) mit
                                 dem Buchstaben »A« (Abgeschrieben/Abschrift)
                                 gekennzeichnet  2) mit Bleistift von unbekannter Hand nummeriert: »283«}\toendnotes[C]{\smallbreak}\pstart{}{\pb}Herrn\pend{}\pstart{}D\textsuperscript{r} Arthur Schnitzler\pend{}\pstart{}\textcolor{pink}{Wien}{}\ledrightnote{\textcolor{pink}{Wien}}\pend{}\pstart{}\textcolor{pink}{XVIII. Sternwartestraße 71}{}\ledrightnote{\textcolor{pink}{Sternwartestraße 71}}\pend{}
{\bigskip}
\pstart
           \noindent{}{\pb}\textcolor{gray}{\textbf{\textcolor{pink}{Hildesheim}{}\ledrightnote{\textcolor{pink}{Hildesheim}}.}}\hfill \textcolor{gray}{\textbf{\textcolor{pink}{Tempelherrenhaus}{}\ledrightnote{\textcolor{pink}{Tempelhaus (Hildesheim)}}.}}\pend
           
\pstart
           {\pb}Lieber,{ }\textcolor{pink}{hier}{}\ledrightnote{{$\rightarrow$}\textcolor{pink}{Hildesheim}} verbringe ich, ganz
               unverhofft, einen stillen Tag. Die \textcolor{pink}{Stadt}{}\ledrightnote{{$\rightarrow$}\textcolor{pink}{Hildesheim}} ist verblüffend schön. Morgen bin ich
               in \textcolor{pink}{Berlin}{}\ledrightnote{\textcolor{pink}{Berlin}}.\pend
           
\pstart
           Alles Herzliche Ihr {\\[\baselineskip]}\spacefill\mbox{Felix Salten}\pend
           \leftskip=0em{}
\pstart
           \textcolor{pink}{Hildesheim}{}\ledrightnote{\textcolor{pink}{Hildesheim}}, 30. 3. 21\pend
           \endnumbering\briefempfaengerindex{Schnitzler, Arthur@\textsc{Schnitzler, Arthur}!zzzSalten, Felix@\emph{von Felix Salten}!1921-03-302@{30. 3. 1921}|)be}\mylabel{h}  \normalsize

\doendnotes{C}
\bigskip
\vfill

\clearpage

\footnotesize

\lohead{\textsc{register}}

% Definiere theindex-Environment komplett neu ohne reledmac
\makeatletter
\renewenvironment{theindex}{%
  \section*{\indexname}%
  \setlength{\parindent}{0pt}%
  \setlength{\parskip}{0pt plus 0.3pt}%
  \let\item\@idxitem
}{%
  \clearpage
}
\makeatother

\IfFileExists{\jobname-pw.ind}{\input{\jobname-pw.ind}}{}

\end{document}

      