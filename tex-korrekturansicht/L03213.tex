%% latex-korrekturansicht-vorspann.tex
%% Vorspann für die Korrekturansicht.
%% Lädt die gemeinsame Datei latex-vorspann.tex mit gesetztem Schalter.

\newif\ifkorrekturansicht
\korrekturansichttrue

\input{../tex-inputs/latex-vorspann}


\renewcommand{\erwaehntePersonen}{Personen: Otto Brahm, Raphael Löwenfeld, Guy de Maupassant, Olga Schnitzler}
\renewcommand{\erwaehnteInstitutionen}{Institutionen: Deutsches Theater Berlin, Paul Ollendorff, Schiller-Theater}
\renewcommand{\erwaehnteOrte}{Orte: Berlin, Dessauer Straße, Deutsches Theater Berlin, Paris, Salzburg, Südtirol, Tirol, Wien}
\renewcommand{\erwaehnteWerke}{Werke: Der Schleier der Beatrice. Schauspiel in fünf Akten, Fort comme la mort}
\section[ Paul Goldmann an Arthur Schnitzler, 14. 7. {[}1902{]}]{Paul Goldmann an Arthur Schnitzler, 14. 7. {[}1902{]}}
\nopagebreak\mylabel{v}
\rehead{ }\normalsize\beginnumbering\briefempfaengerindex{Schnitzler, Arthur@\textsc{Schnitzler, Arthur}!zzzGoldmann, Paul@\emph{von Paul Goldmann}!1902-07-141@{14. 7. {[}1902{]}}|(be}
\toendnotes[C]{\smallbreak\pagebreak[2]}\Standort{DLA, A:Schnitzler, HS.NZ85.1.3172.}
\physDesc{Brief, 1 Blatt, 2 Seiten
\newline{}Handschrift: blaue Tinte, deutsche Kurrent
\newline{}Schnitzler: mit Bleistift das Jahr »{[}1{]}902« vermerkt }\toendnotes[C]{\smallbreak}
\pstart
           \noindent{}\raggedleft{}{\pb}\textcolor{pink}{\textcolor{gray}{\textbf{DESSAUERSTRASSE 19}}}{}\ledrightnote{\textcolor{pink}{Dessauer Straße}}\pend
           
\pstart
           \textcolor{pink}{Berlin}{}\ledrightnote{\textcolor{pink}{Berlin}}, 14. Juli.\pend
           
\pstart\center{}Mein lieber Freund,\pend
\pstart
           Höre ich bald von Dir? Wie war die \label{K_L03213-1v}\edtext{Reiſe}{\lemma{\textnormal{\emph{Reiſe}}}\Cendnote{\textnormal{\textcolor{blue}{Schnitzler} reiste zwischen 27. 6. 1902 und 7. 7. 1902 nach \textcolor{pink}{Salzburg}, \textcolor{pink}{Nordtirol} und \textcolor{pink}{Südtirol}.}}}\label{K_L03213-1h}? Biſt
               Du glücklich zurück? Was macht \textsc{\textcolor{blue}{Olga}{}\ledrightnote{\textcolor{blue}{Olga Schnitzler}}}?\pend
           
\pstart
           Wirſt Du die »\textsc{\textcolor{green}{Beatrice}{}\ledrightnote{\textcolor{green}{Der Schleier der Beatrice. Schauspiel in fünf Akten}}}« dem \label{K_L03213-2v}\edtext{\textsc{Dr\textcolor{gray}{.}{ }\textcolor{blue}{Löwenfeld}{}\ledrightnote{\textcolor{blue}{Raphael Löwenfeld}}}}{\lemma{\textnormal{\emph{Dr. Löwenfeld}}}\Cendnote{\textnormal{\textcolor{blue}{Schnitzler} verhandelte sowohl mit \textcolor{blue}{Raphael Löwenfeld}, dem Leiter des \emph{\textcolor{brown}{Schiller-Theater}}s, als auch mit \textcolor{blue}{Otto Brahm}, dem Leiter des \emph{\textcolor{brown}{Deutschen Theater}}s, wegen einer Aufführung von \emph{\textcolor{green}{Der Schleier der Beatrice}} (vgl. A. S.: \emph{Tagebuch}, 17. 7. 1902). Die \textcolor{pink}{Berlin}er Premiere fand am 7. 3. 1903 am \textcolor{pink}{Deutschen Theater} statt. Siehe auch Arthur Schnitzler an Hugo von Hofmannsthal, 7. 10. 1902.}}}\label{K_L03213-2h} geben?\pend
           
\pstart
           Dieſer Tage las ich \label{K_L03213-3v}\edtext{»\textsc{\begin{otherlanguage}{french}\textcolor{green}{Fort comme la mort}{}\ledrightnote{\textcolor{green}{Fort comme la mort}}\end{otherlanguage}}«}{\lemma{\textnormal{\emph{»Fort comme la mort«}}}\Cendnote{\textnormal{\textcolor{blue}{Guy de Maupassant}: \emph{\textcolor{green}{Fort comme la mort}}. \textcolor{pink}{Paris}: \emph{\textcolor{brown}{Paul Ollendorf}}{ }1889. Siehe A. S.: \emph{Lektüren}, Frankreich.}}}\label{K_L03213-3h}, das mich tief ergriffen hat. Nie iſt das Altwerden ſo geſchildert
               worden. {\pb}Es iſt übrigens Dein Stoff: der alternde
               Junggeſelle, der das junge Mädchen \strikeout{liebt} liebt. Wenn
               Du das \textcolor{green}{Buch}{}\ledrightnote{{$\rightarrow$}\textcolor{green}{Fort comme la mort}} nicht kennſt, mußt
               Du es ſchleunigſt leſen.\pend
           
\pstart
           Ich danke Dir für Deine lieben Karten \strikeout{\textcolor{gray}{aus}} von unterwegs.\pend
           
\pstart
           Viele treue Grüße!{\\[\baselineskip]}Dein{\\[\baselineskip]}\spacefill\mbox{Paul Goldm}\pend
           \leftskip=0em{}\endnumbering\briefempfaengerindex{Schnitzler, Arthur@\textsc{Schnitzler, Arthur}!zzzGoldmann, Paul@\emph{von Paul Goldmann}!1902-07-141@{14. 7. {[}1902{]}}|)be}\mylabel{h}
\begin{anhang}
\end{anhang}\normalsize

\doendnotes{C}
\bigskip
\vfill

\clearpage

\footnotesize

\lohead{\textsc{register}}

% Definiere theindex-Environment komplett neu ohne reledmac
\makeatletter
\renewenvironment{theindex}{%
  \section*{\indexname}%
  \setlength{\parindent}{0pt}%
  \setlength{\parskip}{0pt plus 0.3pt}%
  \let\item\@idxitem
}{%
  \clearpage
}
\makeatother

\IfFileExists{\jobname-pw.ind}{\input{\jobname-pw.ind}}{}

\end{document}

      