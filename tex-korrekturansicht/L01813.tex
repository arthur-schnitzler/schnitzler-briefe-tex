%% latex-korrekturansicht-vorspann.tex
%% Vorspann für die Korrekturansicht.
%% Lädt die gemeinsame Datei latex-vorspann.tex mit gesetztem Schalter.

\newif\ifkorrekturansicht
\korrekturansichttrue

\input{../tex-inputs/latex-vorspann}


               \section[Arthur Schnitzler an Richard Beer-Hofmann, 28. 11. 1908]{ Arthur Schnitzler an Richard Beer-Hofmann, 28. 11. 1908}\nopagebreak\mylabel{v}\rehead{ }\normalsize\beginnumbering\briefempfaengerindex{Beer-Hofmann, Richard@\textsc{Beer-Hofmann, Richard}!zzzSchnitzler, Arthur@\emph{von Arthur Schnitzler}!1908-11-283@{28. 11. 1908}|(be} \toendnotes[C]{\smallbreak\pagebreak[2]} \Standort{YCGL, MSS 31.}
\physDesc{Brief, 1 Blatt, 3 Seiten, Umschlag
\newline{}Handschrift: 1) Bleistift, deutsche Kurrent\hspace{1em}2) schwarze Tinte, deutsche Kurrent (\noindent{}Umschlag)\hspace{1em}\newline{}Versand: Stempel: »\nobreak{}Wien 3, 2\textcolor{gray}{8}. XI. 08, 4\nobreak{}«.  }\buchAbdrucke{\weitereDrucke{Arthur Schnitzler, Richard Beer-Hofmann: \emph{Briefwechsel 1891–1931}. Hg. Konstanze Fliedl. Wien, Zürich: \emph{Europaverlag} 1992, S. 192.} }\toendnotes[C]{\smallbreak}\pstart{}{\pb}\textcolor{gray}{\textbf{Dr. Arthur Schnitzler}}\pend{}\pstart{}\textcolor{gray}{\textbf{\textcolor{pink}{Wien XVIII. Spoettelgasse 7}{}\ledrightnote{\textcolor{pink}{Edmund-Weiß-Gasse}}.}}\pend{}{\bigskip}\pstart{}{\pb}\textsc{Dr. Richard Beer Hofmann}\pend{}\pstart{}\textcolor{pink}{Wien XVIII}{}\ledrightnote{\textcolor{pink}{XVIII., Währing}}\pend{}\pstart{}\textcolor{pink}{\textsc{Hasenauerstr. 59}}{}\ledrightnote{\textcolor{pink}{Hasenauerstraße}}.\pend{}{\bigskip}\pstart
           \noindent{}{\pb}\textcolor{gray}{\textbf{Dr. Arthur Schnitzler}}\hfill \label{K_L01813_1v}\edtext{So{\geminationn}tag 28. 11 08}{\lemma{\textnormal{\emph{Sotag 28. 11 08}}}\Cendnote{\textnormal{Der 28. 11. 1908 war ein Samstag. An diesem
                        Tag besuchte \textcolor{blue}{Schnitzler} ein \textcolor{blue}{Konzert}, was die
                        Datierung erlaubt. Trotzdem fordert das Fehlen eines Gegenbriefes (oder
                        handelte es sich um eine mündlich übermittelte Nachricht?) einige
                        Spekulation, wie die Reihenfolge des zweiten und dritten Briefes vom
                           28. 11. 1908 anzusetzen ist. Im ersten Brief erfährt \textcolor{blue}{Beer-Hofmann} vom \textcolor{blue}{Konzert} und von der geplanten Reise
                        auf den \textcolor{pink}{Semmering}. In der nicht erhaltenen
                        Reaktion \textcolor{blue}{Beer-Hofmann}s gibt dieser
                        bekannt, ebenfalls ins \textcolor{blue}{Konzert} zu wollen und lädt für den 29. 11. 1908 zu
                        zwei Treffen, eins bei ihm zu Hause für den Mittag und eins nach der Lesung
                        am Abend. Im vorliegenden Schreiben versucht \textcolor{blue}{Schnitzler} die beiden Treffen auf eines – das am Abend – zu
                        reduzieren. In seiner zweiten Mitteilung versucht er, es vom Privathaus in das Restaurant
                           \textcolor{pink}{Meißl {\kaufmannsund}
                           Schadn} zu verlegen.}}}\label{K_L01813_1h}\pend
           \pstart
           \textcolor{gray}{\textbf{\textcolor{pink}{Wien XVIII. Spoettelgasse 7}{}\ledrightnote{\textcolor{pink}{Edmund-Weiß-Gasse}}.}}\pend
           \pstart
           lieber Richard, wir fahren nicht auf den \textcolor{pink}{Se{\geminationm}ering}{}\ledrightnote{\textcolor{pink}{Semmering}}, hingegen erlaube ich mir folgenden
               Vorſchlag. Wollen Sie nicht \textcolor{blue}{Kerr}{}\ledrightnote{\textcolor{blue}{Alfred Kerr}}{ }ſtatt Mittag morgen Abend laden, wir (oder ich
               allein, (denn {\pb}\textcolor{blue}{Olga}{}\ledrightnote{\textcolor{blue}{Olga Schnitzler}} iſt nicht ſehr wohl, (daher \introOben{}auch\introOben{} möchte ich die morg. Einladg verreinen)))
               kämen nach dem Nachtmahl hinüber – \pend
           \pstart
           Sie ſagen mirs Abend im \textcolor{blue}{Concert}{}\ledrightnote{→\textcolor{blue}{Ernst von Dohnányi}}\pend
           \pstart
           Herzlichſst Ihr{\\[\baselineskip]}\spacefill\mbox{A.}\pend
           \leftskip=0em{}\pstart
           \noindent{}{\pb}Wie gehts \textcolor{blue}{Noemi}{}\ledrightnote{\textcolor{blue}{Naëmah Beer-Hofmann}}\pend
           \endnumbering\briefempfaengerindex{Beer-Hofmann, Richard@\textsc{Beer-Hofmann, Richard}!zzzSchnitzler, Arthur@\emph{von Arthur Schnitzler}!1908-11-283@{28. 11. 1908}|)be}\mylabel{h}  \normalsize

\doendnotes{C}
\bigskip
\vfill

\clearpage

\footnotesize

\lohead{\textsc{register}}

% Definiere theindex-Environment komplett neu ohne reledmac
\makeatletter
\renewenvironment{theindex}{%
  \section*{\indexname}%
  \setlength{\parindent}{0pt}%
  \setlength{\parskip}{0pt plus 0.3pt}%
  \let\item\@idxitem
}{%
  \clearpage
}
\makeatother

\IfFileExists{\jobname-pw.ind}{\input{\jobname-pw.ind}}{}

\end{document}

      