%% latex-korrekturansicht-vorspann.tex
%% Vorspann für die Korrekturansicht.
%% Lädt die gemeinsame Datei latex-vorspann.tex mit gesetztem Schalter.

\newif\ifkorrekturansicht
\korrekturansichttrue

\input{../tex-inputs/latex-vorspann}


\renewcommand{\erwaehntePersonen}{Personen: Fedor Mamroth}
\renewcommand{\erwaehnteOrte}{Orte: Berlin, Dessauer Straße, Frankfurt am Main, Wien}
\renewcommand{\erwaehnteWerke}{}
\section[ Paul Goldmann an Arthur Schnitzler, 9. 4. {[}1906{]}]{Paul Goldmann an Arthur Schnitzler, 9. 4. {[}1906{]}}
\nopagebreak\mylabel{v}
\rehead{ }\normalsize\beginnumbering\briefempfaengerindex{Schnitzler, Arthur@\textsc{Schnitzler, Arthur}!zzzGoldmann, Paul@\emph{von Paul Goldmann}!1906-04-091@{9. 4. {[}1906{]}}|(be}
\toendnotes[C]{\smallbreak\pagebreak[2]}\Standort{DLA, A:Schnitzler, HS.NZ85.1.3175.}
\physDesc{Brief, 1 Blatt, 1 Seite
\newline{}Handschrift: blaue Tinte, deutsche Kurrent
\newline{}Schnitzler: mit Bleistift das Jahr »{[}1{]}906« vermerkt }\toendnotes[C]{\smallbreak}
\pstart
           \noindent{}\raggedleft{}{\pb}\textcolor{pink}{\textcolor{gray}{\textbf{DESSAUERSTRASSE 19}}}{}\ledrightnote{\textcolor{pink}{Dessauer Straße}}\pend
           
\pstart
           \textcolor{pink}{Berlin}{}\ledrightnote{\textcolor{pink}{Berlin}}, 9. April.\pend
           
\pstart{}Lieber Freund,\pend
\pstart
           Ich werde leider die Freude nicht haben, Dir zu Oſtern
               die Hand zu drücken. Mein \textcolor{blue}{Onkel}{}\ledrightnote{{$\rightarrow$}\textcolor{blue}{Fedor Mamroth}} in \textcolor{pink}{Frankfurt}{}\ledrightnote{\textcolor{pink}{Frankfurt am Main}} iſt \label{K-L03242-1v}\edtext{ſchwer erkrankt}{\lemma{\textnormal{\emph{ſchwer erkrankt}}}\Cendnote{\textnormal{\textcolor{blue}{Fedor Mamroth} verstarb im Folgejahr, am
                     25. 6. 1907, an Darmkrebs.}}}\label{K-L03242-1h} (ein
               Werteruin: \textsc{Tumor} im \textsc{Rectum}), iſt
               dieſer Tage operirt worden, und ich fahre dieſer Tage nach \textcolor{pink}{Frankfurt}{}\ledrightnote{\textcolor{pink}{Frankfurt am Main}}, an ſein Krankenbett. Ein ſchwerer Schickſalsſchlag
               für uns Alle.\pend
           
\pstart
           Viele treue Grüße! {\\[\baselineskip]}Dein \spacefill\mbox{Paul Goldmann.}\pend
           \leftskip=0em{}\endnumbering\briefempfaengerindex{Schnitzler, Arthur@\textsc{Schnitzler, Arthur}!zzzGoldmann, Paul@\emph{von Paul Goldmann}!1906-04-091@{9. 4. {[}1906{]}}|)be}\mylabel{h}  \normalsize

\doendnotes{C}
\bigskip
\vfill

\clearpage

\footnotesize

\lohead{\textsc{register}}

% Definiere theindex-Environment komplett neu ohne reledmac
\makeatletter
\renewenvironment{theindex}{%
  \section*{\indexname}%
  \setlength{\parindent}{0pt}%
  \setlength{\parskip}{0pt plus 0.3pt}%
  \let\item\@idxitem
}{%
  \clearpage
}
\makeatother

\IfFileExists{\jobname-pw.ind}{\input{\jobname-pw.ind}}{}

\end{document}

      