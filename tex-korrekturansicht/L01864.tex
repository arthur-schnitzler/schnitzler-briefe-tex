%% latex-korrekturansicht-vorspann.tex
%% Vorspann für die Korrekturansicht.
%% Lädt die gemeinsame Datei latex-vorspann.tex mit gesetztem Schalter.

\newif\ifkorrekturansicht
\korrekturansichttrue

\input{../tex-inputs/latex-vorspann}


               \section[Arthur Schnitzler an Richard Beer-Hofmann, 12. 8. 1909]{ Arthur Schnitzler an Richard Beer-Hofmann, 12. 8. 1909}\nopagebreak\mylabel{v}\rehead{ }\normalsize\beginnumbering\briefempfaengerindex{Beer-Hofmann, Richard@\textsc{Beer-Hofmann, Richard}!zzzSchnitzler, Arthur@\emph{von Arthur Schnitzler}!1909-08-121@{12. 8. 1909}|(be} \toendnotes[C]{\smallbreak\pagebreak[2]} \Standort{YCGL, MSS 31.}
\physDesc{Bildpostkarte
\newline{}Handschrift: Bleistift, lateinische Kurrent\newline{}Versand: 1) Stempel: »\nobreak{}\oindex{Hochschneeberg@\textbf{Hochschneeberg}, \emph{Berg (N.BRG)}|pwk}Hoch{[}schneeberg{]}\nobreak{}«.  2) Stempel: »\nobreak{}\oindex{Santa Maria Elisabetta@\textbf{Santa Maria Elisabetta}, \emph{Bezirk (A.BZK)}|pwk}S. Elis{[}abetta{]}{ }\textcolor{gray}{di} Lido – (Venezia), \textcolor{gray}{13} 8 0\textcolor{gray}{9}\nobreak{}«. \newline{}Zusatz: die Karte erschien: »Im Selbstverlage des \textbf{\textcolor{brown}{St.
                                          Elisabeth-Kirchlein-Baucomité}}, \textcolor{pink}{Wien III.
                                    Gemeindehaus}. Der Erlös dieser Karte fliesst dem Baufonds
                                    zu.« }\pstart{}{\pb}Dr. Richard Beer Hofmann\pend{}\pstart{}\textcolor{pink}{Venedig – Lido}{}\ledrightnote{\textcolor{pink}{Lido}}\pend{}\pstart{}\textcolor{pink}{Grand Hotel des bains}{}\ledrightnote{\textcolor{pink}{Grand Hotel des Bains}}\pend{}{\bigskip}\pstart
           \noindent{}\centering{}{\pb}\textcolor{gray}{\textbf{\textcolor{pink}{Hotel am Hochschneeberg}{}\ledrightnote{\textcolor{pink}{Berghaus Hochschneeberg}}}}\pend
           \pstart
           12. 8. 09\pend
           \pstart
           Herzliche Grüße Ihnen Allen!\pend
           \pstart \spacefill\mbox{Arthur}\pend{}\endnumbering\briefempfaengerindex{Beer-Hofmann, Richard@\textsc{Beer-Hofmann, Richard}!zzzSchnitzler, Arthur@\emph{von Arthur Schnitzler}!1909-08-121@{12. 8. 1909}|)be}\mylabel{h}  \normalsize

\doendnotes{C}
\bigskip
\vfill

\clearpage

\footnotesize

\lohead{\textsc{register}}

% Definiere theindex-Environment komplett neu ohne reledmac
\makeatletter
\renewenvironment{theindex}{%
  \section*{\indexname}%
  \setlength{\parindent}{0pt}%
  \setlength{\parskip}{0pt plus 0.3pt}%
  \let\item\@idxitem
}{%
  \clearpage
}
\makeatother

\IfFileExists{\jobname-pw.ind}{\input{\jobname-pw.ind}}{}

\end{document}

      