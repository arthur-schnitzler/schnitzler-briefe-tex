%% latex-korrekturansicht-vorspann.tex
%% Vorspann für die Korrekturansicht.
%% Lädt die gemeinsame Datei latex-vorspann.tex mit gesetztem Schalter.

\newif\ifkorrekturansicht
\korrekturansichttrue

\input{../tex-inputs/latex-vorspann}


\renewcommand{\erwaehntePersonen}{Personen: Otto Brahm, Dora Erl, Julius von Gans-Ludassy, Siegfried Jacobsohn, Aleksandr I. Južin, Arthur Kaufmann, Leonid M. Leonidow, Linda von Lützow, Anna Katharina Rehmann, Emanuel Reicher, Rudolf Rittner, Peter Rotenstern, Anna Rotenstern-Tesi, Felix Salten, Ottilie Salten, Paul Salten, Olga Schnitzler, Julius Schnitzler, Heinrich Schnitzler, Konstantin S. Stanislavskij,  W. Fred, Alexander Leonidowitsch Wischnewski, Olga L. Čechowa}
\renewcommand{\erwaehnteOrte}{Orte: Berlin, Dänemark, Edmund-Weiß-Gasse 7, Neuwaldegg, Pötzleinsdorf, Wien}
\renewcommand{\erwaehnteWerke}{Werke: B.Z. am Mittag, Der Weg ins Freie. Roman, Der einsame Weg. Schauspiel in fünf Akten, Onkel Wanja. Szenen aus dem Landleben in vier Akten}
\section[ Arthur Schnitzler an Felix Salten, 27. 4. 1906]{Arthur Schnitzler an Felix Salten, 27. 4. 1906}
\nopagebreak\mylabel{v}
\rehead{ }\normalsize\beginnumbering\briefempfaengerindex{Salten, Felix@\textsc{Salten, Felix}!zzzSchnitzler, Arthur@\emph{von Arthur Schnitzler}!1906-04-271@{27. 4. 1906}|(be}
\toendnotes[C]{\smallbreak\pagebreak[2]}\Standort{Wienbibliothek im Rathaus, ZPH 1681, 2.1.516.}
\physDesc{Brief, 2 Blätter, 7 Seiten, 3653 Zeichen
\newline{}Handschrift: schwarze Tinte, deutsche Kurrent
\newline{}Ordnung: mit Bleistift von unbekannter Hand Nummerierung der Doppelseiten des
                                 Konvoluts: »16«–»19«  }
\buchAbdrucke{\weitereDrucke{Arthur Schnitzler: \emph{Briefe 1875–1912}. Hg. Therese Nickl und Heinrich Schnitzler. Frankfurt am Main: \emph{S. Fischer} 1981, S. 529–531.} }\toendnotes[C]{\smallbreak}
\pstart
           \noindent{}\textcolor{gray}{\textbf{Dr. Arthur Schnitzler}}\hfill {\pb}27. 4. 906\pend
           
\pstart
           \textcolor{gray}{\textbf{\textcolor{pink}{Wien, XVIII. Spoettelgasse 7}{}\ledrightnote{\textcolor{pink}{Edmund-Weiß-Gasse 7}}.}}\pend
           
\pstart
           lieber, Sie haben natürlich ganz recht. Unmöglich konnten Sie ſich
                  \textcolor{blue}{Brahm}{}\ledrightnote{\textcolor{blue}{Otto Brahm}} gegenüber als ungebetener Rathgeber
               aufſpielen, und als ich mein Telegra{\geminationm} an Sie abſandte,
               hatt ich begreiflicherweiſe nicht an irgend einen \textsc{adhoc}-Beſuch od dergl bei \textcolor{blue}{Brahm}{}\ledrightnote{\textcolor{blue}{Otto Brahm}} gedacht,
               ſondern an etwas beiläufigeres, ohne mir über das »wie« weitere Gedanken zu machen.
               (Damit dſs \textcolor{blue}{Brahm}{}\ledrightnote{\textcolor{blue}{Otto Brahm}} auf Ihr Urtheil nichts geben
               könnte, ſind Sie ſehr im Irrtum.) – Nun hab ich die Sache indeſs auf andre, directe
               Weise zu ordnen geſucht. {\pb}(\uline{Dies vollko{\geminationm}en unter
               uns.}) Nach Ihrem Brief, in dem Sie mir Ihr Geſpräch mit \textcolor{blue}{R.}{}\ledrightnote{\textcolor{blue}{Rudolf Rittner}} erzählten u einen Brief \textcolor{blue}{Jacobsohn}{}\ledrightnote{\textcolor{blue}{Siegfried Jacobsohn}}s, der auch telephoniſch eine Art Bereitwilligkeit \textcolor{blue}{R.s}{}\ledrightnote{\textcolor{blue}{Rudolf Rittner}} erfahren haben wollte, telegr ich an \textcolor{blue}{Brahm}{}\ledrightnote{\textcolor{blue}{Otto Brahm}}, ob er mir überlaſſen wolle \textcolor{blue}{\textsc{Rittner}}{}\ledrightnote{\textcolor{blue}{Rudolf Rittner}} zur Übernahme
               zu bewegen. Er konnte nichts dagegen haben, warnte mich für alle Fälle, wuſch ſeine
               Hände in Unſchuld \textsc{etc.} Ich telegr. nun an \textcolor{blue}{\textsc{Rittner}}{}\ledrightnote{\textcolor{blue}{Rudolf Rittner}}, der mir in einem ſehr
               liebenswürdigen Telegra{\geminationm} nein ſagte. Ich hatte es
               natürlich nicht anders erwartet – die Gegengründe lagen für \textcolor{blue}{Rittner}{}\ledrightnote{\textcolor{blue}{Rudolf Rittner}} zu nah, als daſs er nicht von ihnen hätte {\pb}Gebrauch machen ſollen. Aber ich wollte mir
               keine Vorwürfe zu machen haben – und da mir \textcolor{blue}{\textsc{Rittner}}{}\ledrightnote{\textcolor{blue}{Rudolf Rittner}} ſtrengſte Discretion zugeſagt hat, hoffe ich
               daſs nicht am End noch eine für die \textcolor{pink}{Wien}{}\ledrightnote{\textcolor{pink}{Wien}}er
               Aufführg (auf die ich ſchließlich doch nicht verzichten möchte) gefährliche
               Couliſſenklatſcherei heraus ko{\geminationm}t. Sonderbar iſt, daſs
               vor 2 Jahren, nach \textcolor{blue}{Rittner}{}\ledrightnote{\textcolor{blue}{Rudolf Rittner}}s Verſagen (aus
               Unluſt) an der Rolle alle, auch \textcolor{blue}{Brahm}{}\ledrightnote{\textcolor{blue}{Otto Brahm}} und ich
               dachten, \textcolor{blue}{Reicher}{}\ledrightnote{\textcolor{blue}{Emanuel Reicher}} wäre der richtige Darſteller
               für die Rolle. Nach der erſchütternden Charakteriſtik, die Sie von ſeiner Auffaſſung
               geben, ka{\geminationn} ich mir nun wohl vorſtellen, was mir {\pb}bevorſteht. Übrigens gibt es meiner Empfindg
               nach nur einen Darſteller für den \textsc{\textcolor{green}{Julian}{}\ledrightnote{{$\rightarrow$}\textcolor{green}{Der einsame Weg. Schauspiel in fünf Akten}}}: \textsc{\textcolor{blue}{Wischnevski}{}\ledrightnote{\textcolor{blue}{Alexander Leonidowitsch Wischnewski}}}. Sie haben ihn ja als \textcolor{green}{Onkel \textsc{Wanja}}{}\ledrightnote{\textcolor{green}{Onkel Wanja. Szenen aus dem Landleben in vier Akten}} geſehen. Und \textsc{\textcolor{blue}{Stanislawski}{}\ledrightnote{\textcolor{blue}{Konstantin S. Stanislavskij}}} als \textsc{\textcolor{green}{Sala}{}\ledrightnote{{$\rightarrow$}\textcolor{green}{Der einsame Weg. Schauspiel in fünf Akten}}} wär auch nicht übel. Wir haben dieſe beiden, auch \textsc{\textcolor{blue}{Ljuschin}{}\ledrightnote{\textcolor{blue}{Aleksandr I. Južin}}} (\textcolor{green}{Profeſſor}{}\ledrightnote{{$\rightarrow$}\textcolor{green}{Onkel Wanja. Szenen aus dem Landleben in vier Akten}} in \textsc{\textcolor{green}{Wanja}{}\ledrightnote{\textcolor{green}{Onkel Wanja. Szenen aus dem Landleben in vier Akten}}}), \textsc{\textcolor{blue}{Leonidow}{}\ledrightnote{\textcolor{blue}{Leonid M. Leonidow}}}, Frau \textcolor{blue}{Tſchechow}{}\ledrightnote{\textcolor{blue}{Olga L. Čechowa}} bei \textcolor{blue}{Rotenſtern’s}{}\ledrightnote{\textcolor{blue}{Peter Rotenstern}{\newline}\textcolor{blue}{Anna Rotenstern-Tesi}} kennengelernt; auch im Theater hinter
               den Couliſſen ein paar mal geſprochen. Es hat mich ſehr gefreut, daſs ihnen viel
               daran zu liegen ſchien, ein Stück von mir für ihr Theater zu beko{\geminationm}en. Jedenfalls gibt es keins, an dem ich lieber
               aufgeführt werden möchte. Sieht man ſolche {\pb}um alles dramatiſche unbekü{\geminationm}erte Geſtalten- und
               Lebensſtücke wie den \textcolor{green}{Onkel \textsc{Wanja}}{}\ledrightnote{\textcolor{green}{Onkel Wanja. Szenen aus dem Landleben in vier Akten}}, ſo ist einem, als braucht man ſich nur hinzuſetzen, um ein
               viertel Dutzend im Jahr zu ſchreiben. Und doch{\dots} Allerdings
               fiele man auch durch. –\pend
           
\pstart
           Tennis ſpielen wir ſchon ziemlich regelmäßig – d. h. meiſtens ich, Dr \textsc{\textcolor{blue}{Kaufmann}{}\ledrightnote{\textcolor{blue}{Arthur Kaufmann}}}, Frl \textsc{\textcolor{blue}{Erl}{}\ledrightnote{\textcolor{blue}{Dora Erl}}}, \textcolor{blue}{Olga}{}\ledrightnote{\textcolor{blue}{Olga Schnitzler}} ſeltener. Zuweilen geh ich im \textcolor{pink}{Pötzleinsdorferwalde}{}\ledrightnote{\textcolor{pink}{Pötzleinsdorf}} ſpaziren. Es iſt ſchon beinah
                  ſo{\geminationm}erlich, um mindeſten{[}s{]}
               vierzehn Tage weiter vor, als voriges Jahr. \label{K_L03004-1v}\edtext{Neulich war \textsc{\textcolor{blue}{Fred}{}\ledrightnote{\textcolor{blue}{W. Fred}}} bei uns}{\lemma{\textnormal{\emph{Neulich war Fred bei uns}}}\Cendnote{\textnormal{siehe A. S.: \emph{Tagebuch}, 23. 4. 1906}}}\label{K_L03004-1h}, der ſich im Lauf der Jahre höchſt vorteilhaft verändert hat. (Dieſer {\pb}Tage wird er (wahrſcheinlich von meinem \textcolor{blue}{Bruder}{}\ledrightnote{{$\rightarrow$}\textcolor{blue}{Julius Schnitzler}}) an Gallenſteinen
               operirt.) –\pend
           
\pstart
           Über Ihre So{\geminationm}erpläne möcht ich recht bald näheres
               wiſſen. Meine Karte, Frau \textsc{\textcolor{blue}{v Lützow}{}\ledrightnote{\textcolor{blue}{Linda von Lützow}}} betreffend, haben Sie wohl erhalten? Neulich war hier das Gerücht verbreitet,
               daſs Sie auf ein paar Tage nach \textcolor{pink}{Wien}{}\ledrightnote{\textcolor{pink}{Wien}} kämen. Wie
               ſteht die \label{K_L03004-2v}\edtext{Proceſsangelegenheit}{\lemma{\textnormal{\emph{Proceſsangelegenheit}}}\Cendnote{\textnormal{siehe Felix Salten an Arthur Schnitzler, 9. 3. 1906}}}\label{K_L03004-2h}? Ich ſtelle mir \textcolor{blue}{Ludaſſy}{}\ledrightnote{\textcolor{blue}{Julius von Gans-Ludassy}} verda{\geminationm}t wenig dazu gelaunt vor. –\pend
           
\pstart
           Neulich, mit dem reparirten Rad (alles mögliche, 55 Kronen!) erſter Verſuch, in \textcolor{pink}{Neuwaldegg}{}\ledrightnote{\textcolor{pink}{Neuwaldegg}} brach die Axe. Trotzdem bleibt die
               Sehnſucht nach den gemeinſchaftlichen Partien beſtehen. Haben Sie ſich nicht die
               Sache wegen \label{K_L03004-3v}\edtext{\textcolor{pink}{Daenemark}{}\ledrightnote{\textcolor{pink}{Dänemark}}}{\lemma{\textnormal{\emph{Daenemark}}}\Cendnote{\textnormal{siehe Felix Salten an Arthur Schnitzler, 28. 3. 1906}}}\label{K_L03004-3h}{ }{\pb}überlegt?\pend
           
\pstart
           Ich arbeite (am \textcolor{green}{Roman}{}\ledrightnote{{$\rightarrow$}\textcolor{green}{Der Weg ins Freie. Roman}})
               ziemlich regelmäßig aber ohne die nöthige Intenſität. Mir thut es ſo leid, daſs ich
               Sie in der \textcolor{green}{B. Z.}{}\ledrightnote{\textcolor{green}{B.Z. am Mittag}} beinah niemals finde. Was
               machen Sie ſonſt? Ich nehme an, daſs Sie mit adminiſtrativen und organiſatoriſchen
               Arbeiten überhäuft ſind. –\pend
           
\pstart
           Seien Sie herzlich gegrüßt, ebenſo \textcolor{blue}{Otti}{}\ledrightnote{\textcolor{blue}{Ottilie Salten}} u
               die \textcolor{blue}{Kinder}{}\ledrightnote{{$\rightarrow$}\textcolor{blue}{Paul Salten}{\newline}{$\rightarrow$}\textcolor{blue}{Anna Katharina Rehmann}}, von \textcolor{blue}{uns}{}\ledrightnote{{$\rightarrow$}\textcolor{blue}{Olga Schnitzler}{\newline}{$\rightarrow$}\textcolor{blue}{Heinrich Schnitzler}} allen. {\\[\baselineskip]}Ihr {\\[\baselineskip]}\spacefill\mbox{A.}\pend
           \leftskip=0em{}\endnumbering\briefempfaengerindex{Salten, Felix@\textsc{Salten, Felix}!zzzSchnitzler, Arthur@\emph{von Arthur Schnitzler}!1906-04-271@{27. 4. 1906}|)be}\mylabel{h}  \normalsize

\doendnotes{C}
\bigskip
\vfill

\clearpage

\footnotesize

\lohead{\textsc{register}}

% Definiere theindex-Environment komplett neu ohne reledmac
\makeatletter
\renewenvironment{theindex}{%
  \section*{\indexname}%
  \setlength{\parindent}{0pt}%
  \setlength{\parskip}{0pt plus 0.3pt}%
  \let\item\@idxitem
}{%
  \clearpage
}
\makeatother

\IfFileExists{\jobname-pw.ind}{\input{\jobname-pw.ind}}{}

\end{document}

      