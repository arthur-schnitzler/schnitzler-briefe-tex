%% latex-korrekturansicht-vorspann.tex
%% Vorspann für die Korrekturansicht.
%% Lädt die gemeinsame Datei latex-vorspann.tex mit gesetztem Schalter.

\newif\ifkorrekturansicht
\korrekturansichttrue

\input{../tex-inputs/latex-vorspann}


\renewcommand{\erwaehnteOrte}{Orte: Genf, Paris, Wien}
\renewcommand{\erwaehnteWerke}{Werke: Jung-Wien im Auslande, La Semaine Littéraire, Sterben. Novelle, Wiener Allgemeine Zeitung}
\section[ Felix Salten an Arthur Schnitzler, {[}11. 5. 1895{]}]{Felix Salten an Arthur Schnitzler, {[}11. 5. 1895{]}}
\nopagebreak\mylabel{v}
\rehead{ }\normalsize\beginnumbering\briefempfaengerindex{Schnitzler, Arthur@\textsc{Schnitzler, Arthur}!zzzSalten, Felix@\emph{von Felix Salten}!1895-05-111@{{[}11. 5. 1895{]}}|(be}
\toendnotes[C]{\smallbreak\pagebreak[2]}\Standort{CUL, Schnitzler, B 89, A 1.}
\physDesc{Brief, 1 Blatt, 1 Seite, 177 Zeichen
\newline{}Handschrift: Bleistift, lateinische Kurrent
\newline{}Schnitzler: mit Bleistift datiert: »11/5 95« 
\newline{}Ordnung: mit Bleistift von unbekannter Hand nummeriert: »54a?« }\toendnotes[C]{\smallbreak}
\pstart
           \noindent{}{\pb}\label{K_L03153-1v}\edtext{L. F.}{\lemma{\textnormal{\emph{L. F.}}}\Cendnote{\textnormal{Lieber Freund}}}\label{K_L03153-1h} herzlichen Dank mit
               der Bitte, zu entschuldigen, dass es nicht früher möglich war. – Die \label{K_L03153-2v}\edtext{\textcolor{green}{Notiz}{}\ledrightnote{{$\rightarrow$}\textcolor{green}{Jung-Wien im Auslande}}}{\lemma{\textnormal{\emph{Notiz}}}\Cendnote{\textnormal{[\textcolor{blue}{Felix Salten}]: \emph{\textcolor{green}{Jung-Wien im Auslande}}. In: \emph{\textcolor{green}{Wiener Allgemeine Zeitung}}, Nr. 5.156, 12. 5. 1895, S. 4: »Der erst kürzlich
                     erschienene Roman ›\textcolor{green}{\so{Sterben}}‹ des \textcolor{pink}{Wien}er Dichters \textcolor{blue}{\so{Arthur Schnitzler}} ist bereits in’s Französische übersetzt worden. Die bekannte französische
                     Wochenschrift in \textcolor{pink}{Genf} ›\textcolor{green}{\so{La Semaine Littéraire}}‹ beginnt in ihrer letzten Nummer mit der Veröffentlichung dieses \textcolor{green}{Roman}es, welcher
                     demnächst auch in \textcolor{pink}{Paris} in Buchform
                     erscheinen wird.«}}}\label{K_L03153-2h} über \textcolor{green}{Semaine
                  littéraire}{}\ledrightnote{\textcolor{green}{La Semaine Littéraire}} habe ich \label{K_L03153-3v}\edtext{heute erst, – weil \textcolor{green}{Sonntagsblatt}{}\ledrightnote{{$\rightarrow$}\textcolor{green}{La Semaine Littéraire}} – gegeben}{\lemma{\textnormal{\emph{heute … gegeben}}}\Cendnote{\textnormal{Zwei am Seitenende angebrachte Zeichen fordern zum Umblättern
                  auf und verweisen möglicherweise auf die nicht erhaltene Beilage der erwähnten \textcolor{green}{Zeitungsnotiz}.}}}\label{K_L03153-3h}.\pend
           
\pstart
           Ihr {\\[\baselineskip]}\spacefill\mbox{Salten}\pend
           \leftskip=0em{}\endnumbering\briefempfaengerindex{Schnitzler, Arthur@\textsc{Schnitzler, Arthur}!zzzSalten, Felix@\emph{von Felix Salten}!1895-05-111@{{[}11. 5. 1895{]}}|)be}\mylabel{h}  \normalsize

\doendnotes{C}
\bigskip
\vfill

\clearpage

\footnotesize

\lohead{\textsc{register}}

% Definiere theindex-Environment komplett neu ohne reledmac
\makeatletter
\renewenvironment{theindex}{%
  \section*{\indexname}%
  \setlength{\parindent}{0pt}%
  \setlength{\parskip}{0pt plus 0.3pt}%
  \let\item\@idxitem
}{%
  \clearpage
}
\makeatother

\IfFileExists{\jobname-pw.ind}{\input{\jobname-pw.ind}}{}

\end{document}

      