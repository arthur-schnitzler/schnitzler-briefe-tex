%% latex-korrekturansicht-vorspann.tex
%% Vorspann für die Korrekturansicht.
%% Lädt die gemeinsame Datei latex-vorspann.tex mit gesetztem Schalter.

\newif\ifkorrekturansicht
\korrekturansichttrue

\input{../tex-inputs/latex-vorspann}


\renewcommand{\erwaehntePersonen}{Personen: Hermann Bahr, Paul Marx, Felix Salten, Ottilie Salten, Paul Salten, Olga Schnitzler, Gustav Schwarzkopf}
\renewcommand{\erwaehnteOrte}{Orte: Eisenerz, Hochschwab, Hotel Hochschwab, Kurhaus Rudolfsbad, Mariazell, Reichenau an der Rax, Weichselboden, Wien, Wildalpen}
\renewcommand{\erwaehnteWerke}{Werke: Der Ruf des Lebens. Schauspiel in drei Akten, Die Andere, Tagebuch}
\section[ Arthur Schnitzler an Felix Salten, 20. 7. 1905]{Arthur Schnitzler an Felix Salten, 20. 7. 1905}
\nopagebreak\mylabel{v}
\rehead{ }\normalsize\beginnumbering\briefempfaengerindex{Salten, Felix@\textsc{Salten, Felix}!zzzSchnitzler, Arthur@\emph{von Arthur Schnitzler}!1905-07-201@{20. 7. 1905}|(be}
\toendnotes[C]{\smallbreak\pagebreak[2]}\Standort{Wienbibliothek im Rathaus, ZPH 1681, 2.1.516.}
\physDesc{Brief, 1 Blatt, 4 Seiten, 1240 Zeichen
\newline{}Handschrift: Bleistift, deutsche Kurrent
\newline{}Ordnung: mit Bleistift von unbekannter Hand Nummerierung der Doppelseiten des
                                 Konvoluts: »16«–»17« }\toendnotes[C]{\smallbreak}
\pstart
           \raggedleft{}{\pb}\textsc{\textcolor{pink}{Reichenau}{}\ledrightnote{\textcolor{pink}{Reichenau an der Rax}}},{\\}20/7 905\pend
           
\pstart
           lieber, unſre \label{K_L03000-1v}\edtext{Briefe
               haben ſich gekreuzt}{\lemma{\textnormal{\emph{Briefe … gekreuzt}}}\Cendnote{\textnormal{Felix Salten an Arthur Schnitzler, 18. 7. 1905. \textcolor{blue}{Schnitzler}s Brief ist nicht erhalten.}}}\label{K_L03000-1h}. Sie wiſſen alſo ſchon, daſs
               ich Sie bitten werde, unſre Tour, \textsc{resp.} Ihr Hieherkommen um
               etliche Tage zu verſchieben. Heute fahren wir ins \textcolor{pink}{Hochſchwab}{}\ledrightnote{\textcolor{pink}{Hochschwab}}gebiet, denken Samſtag wieder da zu ſein (ich und 
                  \textcolor{blue}{Paul Marx}{}\ledrightnote{\textcolor{blue}{Paul Marx}}). Ob \label{K_L03000-2v}\edtext{\textcolor{blue}{Gustav Schwarzkopf}{}\ledrightnote{\textcolor{blue}{Gustav Schwarzkopf}}}{\lemma{\textnormal{\emph{Gustav Schwarzkopf}}}\Cendnote{\textnormal{\textcolor{blue}{Gustav Schwarzkopf} kam am Montag,
                  dem 24. 7. 1905, in
                     \textcolor{pink}{Reichenau an der Rax} an. Im \emph{\textcolor{green}{Tagebuch}}
                  wird er in den darauf folgenden Tagen nicht erwähnt. An der hier verhandelten Reise nach
                  \textcolor{pink}{Mariazell} nahm er nicht teil – ebensowenig wie \textcolor{blue}{Schnitzler}.
               }}}\label{K_L03000-2h} iſt noch
               nicht ausgemacht; das wäre etwa Montag auf 2 Tage denk
               ich. \label{K_L03000-3v}\edtext{Mitte {\pb}oder Ende nächſter Woche}{\lemma{\textnormal{\emph{Mitte … Woche}}}\Cendnote{\textnormal{\textcolor{blue}{Arthur} und \textcolor{blue}{Olga Schnitzler} blieben bis zum 29. 7. 1905 in \textcolor{pink}{Reichenau an der Rax} und kehrten dann nach \textcolor{pink}{Wien} zurück. \textcolor{blue}{Salten} kam am 26. 7. 1905 in \textcolor{pink}{Reichenau an der Rax} an. Am 29. 7. 1905 trafen
                  sich die \textcolor{blue}{vier} noch.}}}\label{K_L03000-3h} ſtänden wir
               dann gern un\textcolor{gray}{d} auf möglichſt lang zur Verfügung. Vielleicht auch,
               daſs unſre Wegfahrt mit Ihnen ſchon ein Verlaſſen \textcolor{pink}{Reichenau}{}\ledrightnote{\textcolor{pink}{Reichenau an der Rax}}s zu bedeuten hätte (der \textcolor{pink}{Ort}{}\ledrightnote{{$\rightarrow$}\textcolor{pink}{Reichenau an der Rax}} bleibt wundervoll, aber das \textsc{\textcolor{pink}{Curhaus}{}\ledrightnote{\textcolor{pink}{Kurhaus Rudolfsbad}}}{ }\label{K_L03000-4v}\edtext{verbeiſelt}{\lemma{\textnormal{\emph{verbeiſelt}}}\Cendnote{\textnormal{Beisl, österreichisch: Kneipe}}}\label{K_L03000-4h} ſich i{\geminationm}er mehr) und daſs wir uns da{\geminationn} noch auf einige Tage wo anders anſiedeln. Das berühmte
                  {\pb}\label{K_L03000-5v}\edtext{\textsc{\textcolor{pink}{Fölzhotel}{}\ledrightnote{\textcolor{pink}{Hotel Hochschwab}}} hoff ich noch heute zu betreten}{\lemma{\textnormal{\emph{Fölzhotel … betreten}}}\Cendnote{\textnormal{siehe A. S.: \emph{Tagebuch}, 20. 7. 1905}}}\label{K_L03000-5h}. Eventuell gingen \textsc{resp} führen wir von \textsc{\textcolor{pink}{Mariazell}{}\ledrightnote{\textcolor{pink}{Mariazell}}}, Ihren Intentionen entſprechend, über \textsc{\textcolor{pink}{Wildalpe}{}\ledrightnote{\textcolor{pink}{Wildalpen}}}, \textsc{\textcolor{pink}{Weichselboden}{}\ledrightnote{\textcolor{pink}{Weichselboden}}} nach \textcolor{pink}{Eisenerz}{}\ledrightnote{\textcolor{pink}{Eisenerz}}. Das weſentliche bleibt, daſs
               man ein paar Sommertage wieder einmal zuſa{\geminationm}en verbringt.
               Ich hoffe bei meiner Rückkehr einige Zeilen von Ihnen zu finden. Was hat denn {\pb}Ihrem \textcolor{blue}{Paul}{}\ledrightnote{\textcolor{blue}{Paul Salten}} gefehlt? Wieder ſo eine Kehlkopfſache?\pend
           
\pstart
           Wir grüßen Sie alle herzlich {\\[\baselineskip]}Ihr {\\[\baselineskip]}\spacefill\mbox{A.}\pend
           \leftskip=0em{}
\pstart
           \noindent{}Wohin iſt das \textcolor{blue}{Bahr}{}\ledrightnote{\textcolor{blue}{Hermann Bahr}}-\textcolor{green}{Stück}{}\ledrightnote{{$\rightarrow$}\textcolor{green}{Die Andere}} zu ſenden? – Ich \label{K_L03000-6v}\edtext{leſe es erſt nach meiner Rückkehr}{\lemma{\textnormal{\emph{leſe … Rückkehr}}}\Cendnote{\textnormal{\textcolor{blue}{Schnitzler} las \emph{\textcolor{green}{Die Andere}} am 26. 7. 1905. Siehe auch Arthur Schnitzler an Hermann Bahr, 30. 7. 1905.}}}\label{K_L03000-6h}{ }\introOben{}(Samstag)\introOben{}, da ich, ſelbſt
                     \label{K_L03000-7v}\edtext{dramatiſch verſunken}{\lemma{\textnormal{\emph{dramatiſch verſunken}}}\Cendnote{\textnormal{\textcolor{blue}{Schnitzler} arbeitete an \emph{\textcolor{green}{Der Ruf des Lebens}}.}}}\label{K_L03000-7h}, in nichts andres der Art zu
                  ſteigen mich getraue.\pend
           \endnumbering\briefempfaengerindex{Salten, Felix@\textsc{Salten, Felix}!zzzSchnitzler, Arthur@\emph{von Arthur Schnitzler}!1905-07-201@{20. 7. 1905}|)be}\mylabel{h}  \normalsize

\doendnotes{C}
\bigskip
\vfill

\clearpage

\footnotesize

\lohead{\textsc{register}}

% Definiere theindex-Environment komplett neu ohne reledmac
\makeatletter
\renewenvironment{theindex}{%
  \section*{\indexname}%
  \setlength{\parindent}{0pt}%
  \setlength{\parskip}{0pt plus 0.3pt}%
  \let\item\@idxitem
}{%
  \clearpage
}
\makeatother

\IfFileExists{\jobname-pw.ind}{\input{\jobname-pw.ind}}{}

\end{document}

      