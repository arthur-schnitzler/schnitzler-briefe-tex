%% latex-korrekturansicht-vorspann.tex
%% Vorspann für die Korrekturansicht.
%% Lädt die gemeinsame Datei latex-vorspann.tex mit gesetztem Schalter.

\newif\ifkorrekturansicht
\korrekturansichttrue

\input{../tex-inputs/latex-vorspann}


\renewcommand{\erwaehntePersonen}{Personen: Richard Beer-Hofmann, Jakob Wassermann}
\renewcommand{\erwaehnteOrte}{Orte: Carbonin, Unterach am Attersee, Wien}
\renewcommand{\erwaehnteWerke}{}
\section[ Felix Salten an Arthur Schnitzler, 8. 8. 1899]{Felix Salten an Arthur Schnitzler, 8. 8. 1899}
\nopagebreak\mylabel{v}
\rehead{ }\normalsize\beginnumbering\briefempfaengerindex{Schnitzler, Arthur@\textsc{Schnitzler, Arthur}!zzzSalten, Felix@\emph{von Felix Salten}!1899-08-082@{8. 8. 1899}|(be}
\toendnotes[C]{\smallbreak\pagebreak[2]}\Standort{CUL, Schnitzler, B 89, A 2.}
\physDesc{Brief, 1 Blatt, 1 Seite, 281 Zeichen
\newline{}Handschrift: Bleistift, lateinische Kurrent
\newline{}Ordnung: mit Bleistift von unbekannter Hand nummeriert: »120« }\toendnotes[C]{\smallbreak}
\pstart
           \noindent{}{\pb}Lieber, vermuthlich habe ich Ihre Carte aus \label{K_L03296-1v}\edtext{\textcolor{pink}{Schluderbach}{}\ledrightnote{\textcolor{pink}{Carbonin}}}{\lemma{\textnormal{\emph{Schluderbach}}}\Cendnote{\textnormal{\textcolor{blue}{Schnitzler} war am 1. 8. 1899 sowie von
                     5. 8. 1899 bis
                     6. 8. 1899 in
                     \textcolor{pink}{Carbonin} (\textcolor{pink}{Schluderbach}) gewesen. Womöglich hatte er die Karte am
                  Beginn seiner Wanderung mit \textcolor{blue}{Jakob
                     Wassermann} und \textcolor{blue}{Richard Beer-Hofmann}
                  verfasst (siehe Felix Salten an Arthur Schnitzler, 27. 7. 1899). Nach \textcolor{pink}{Wien} kam er erst am 12. 10. 1899
                  wieder.}}}\label{K_L03296-1h} richtig gelesen, und Sie sind schon in \textcolor{pink}{Wien}{}\ledrightnote{\textcolor{pink}{Wien}}, oder kommen nächstens dahin. Ich reise Freitag von \textcolor{pink}{hier}{}\ledrightnote{{$\rightarrow$}\textcolor{pink}{Unterach am Attersee}} zurück.\pend
           
\pstart
           Wenn Sie da sind, senden Sie mir eine Zeile, wo wir uns treffen können.\pend
           
\pstart
           Wie geht es Ihnen?\pend
           
\pstart
           Herzlichst Ihr {\\[\baselineskip]}\spacefill\mbox{Salten}\pend
           \leftskip=0em{}
\pstart
           \textcolor{pink}{Unterach}{}\ledrightnote{\textcolor{pink}{Unterach am Attersee}}{ }8/8 99\pend
           \endnumbering\briefempfaengerindex{Schnitzler, Arthur@\textsc{Schnitzler, Arthur}!zzzSalten, Felix@\emph{von Felix Salten}!1899-08-082@{8. 8. 1899}|)be}\mylabel{h}
\begin{anhang}
\end{anhang}\normalsize

\doendnotes{C}
\bigskip
\vfill

\clearpage

\footnotesize

\lohead{\textsc{register}}

% Definiere theindex-Environment komplett neu ohne reledmac
\makeatletter
\renewenvironment{theindex}{%
  \section*{\indexname}%
  \setlength{\parindent}{0pt}%
  \setlength{\parskip}{0pt plus 0.3pt}%
  \let\item\@idxitem
}{%
  \clearpage
}
\makeatother

\IfFileExists{\jobname-pw.ind}{\input{\jobname-pw.ind}}{}

\end{document}

      