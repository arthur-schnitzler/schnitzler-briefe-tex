%% latex-korrekturansicht-vorspann.tex
%% Vorspann für die Korrekturansicht.
%% Lädt die gemeinsame Datei latex-vorspann.tex mit gesetztem Schalter.

\newif\ifkorrekturansicht
\korrekturansichttrue

\input{../tex-inputs/latex-vorspann}


\renewcommand{\erwaehntePersonen}{Personen: Heinrich Kanner, Paul Schlenther, Isidor Singer}
\renewcommand{\erwaehnteInstitutionen}{Institutionen: Die Zeit}
\renewcommand{\erwaehnteOrte}{Orte: Wien, Wipplingerstraße}
\renewcommand{\erwaehnteWerke}{Werke: Der Schleier der Beatrice. Schauspiel in fünf Akten, Die Zeit, Erklärung [Schleier der Beatrice]}
\section[ Felix Salten an Arthur Schnitzler, 9. 12. 1903]{Felix Salten an Arthur Schnitzler, 9. 12. 1903}
\nopagebreak\mylabel{v}
\rehead{ }\normalsize\beginnumbering\briefempfaengerindex{Schnitzler, Arthur@\textsc{Schnitzler, Arthur}!zzzSalten, Felix@\emph{von Felix Salten}!1903-12-091@{9. 12. 1903}|(be}
\toendnotes[C]{\smallbreak\pagebreak[2]}\Standort{CUL, Schnitzler, B 89, A 2.}
\physDesc{Brief, 1 Blatt, 1 Seite, 514 Zeichen
\newline{}Schreibmaschine
\newline{}Handschrift: schwarze Tinte, lateinische Kurrent (\noindent{}Korrektur und Unterschrift)
\newline{}Ordnung: mit Bleistift von unbekannter Hand nummeriert: »180« }\toendnotes[C]{\smallbreak}
\pstart
           \noindent{}{\pb}\textcolor{gray}{\textbf{DIE}}\pend
           
\pstart
           \textcolor{gray}{\textbf{\textcolor{brown}{ZEIT}{}\ledrightnote{\textcolor{brown}{Die Zeit}}}}\hfill \textcolor{gray}{\textbf{\emph{\textcolor{pink}{WIEN,}{}\ledrightnote{\textcolor{pink}{Wien}}}}}{ }9. Dezember 1903.\pend
           
\pstart
           \textcolor{gray}{\textbf{\textcolor{pink}{WIEN}{}\ledrightnote{\textcolor{pink}{Wien}}ER TAGESZEITUNG}}\hfill \textcolor{gray}{\textbf{\emph{\textcolor{pink}{I. Wipplingerstrasse 38}{}\ledrightnote{\textcolor{pink}{Wipplingerstraße}}.}}}\pend
           
\pstart
           \textcolor{gray}{\textbf{Herausgeber:}}\pend
           
\pstart
           \textcolor{gray}{\textbf{Prof. Dr. \textcolor{blue}{I. Singer}{}\ledrightnote{\textcolor{blue}{Isidor Singer}}}}\pend
           
\pstart
           \textcolor{gray}{\textbf{Dr. \textcolor{blue}{Heinrich Kanner}{}\ledrightnote{\textcolor{blue}{Heinrich Kanner}}}}\pend
           
\pstart
           \textcolor{gray}{\textbf{\textbf{Redaction}}}\pend
           
\pstart
           \textcolor{gray}{\textbf{Telegramm-Adresse: \textcolor{brown}{\so{Zeit}}{}\ledrightnote{\textcolor{brown}{Die Zeit}}\so{,}{ }\textcolor{pink}{\so{Wien}}{}\ledrightnote{\textcolor{pink}{Wien}}}}\pend
           
\pstart
           \textcolor{gray}{\textbf{\textbf{Telephone:}}}\pend
           
\pstart
           \textcolor{gray}{\textbf{Interurbanes Telephon Nr. 15.988}}\pend
           
\pstart
           \textcolor{gray}{\textbf{= Telephone Nr. 17.040, 17.041 =}}\pend
           
\pstart
           \textcolor{gray}{\textbf{Depeschensaal Nr. 4548.}}\pend
           
\pstart
           Sa/H\pend
           
\pstart\center{}Lieber Freund!\pend
\pstart
           Da unsere \textcolor{green}{Weihnachtsnummer}{}\ledrightnote{{$\rightarrow$}\textcolor{green}{Die Zeit}}
               jetzt fertig gestellt werden muss, frage ich Sie, ob Sie \label{K_L03354-1v}\edtext{etwas für mich haben}{\lemma{\textnormal{\emph{etwas für mich haben}}}\Cendnote{\textnormal{Von \textcolor{blue}{Schnitzler} erschien
                  nichts in der Weihnachtsbeilage der \emph{\textcolor{green}{Zeit}}.}}}\label{K_L03354-1h}. Es muss nichts Grosses sein aber aus mancherlei Gründen wäre
               es mir lieb, wenn Sie mir irgend etwas schicken können. Die \label{K_L03354-2v}\edtext{\textcolor{blue}{Schlenther}{}\ledrightnote{\textcolor{blue}{Paul Schlenther}}-Briefe}{\lemma{\textnormal{\emph{Schlenther-Briefe}}}\Cendnote{\textnormal{Handelte es sich noch um die Briefe, die \textcolor{blue}{Schlenther}{ }\textcolor{blue}{Schnitzler} zur geplanten Annahme und
                  späteren Ablehnung von \emph{\textcolor{green}{Der Schleier der
                     Beatrice}} geschickt hatte? \textcolor{blue}{Salten}
                  hatte damals den Protest organisiert, der zur \emph{\textcolor{green}{Erklärung}} von sechs Autoren in den Tageszeitungen geführt hatte. Siehe Richard Beer-Hofmann an Arthur Schnitzler, 14. 9. 1900.}}}\label{K_L03354-2h} habe ich Ihnen
               gleich am Montag rekommandiert zurückgeschickt.
               Hoffentlich bin ich in der nächsten Woche mit dem \label{K_L03354-3v}\edtext{Preisausschreiben}{\lemma{\textnormal{\emph{Preisausschreiben}}}\Cendnote{\textnormal{siehe Felix Salten an Arthur Schnitzler, 19. 9. [1903]}}}\label{K_L03354-3h} so weit fertig, um \label{K_L03354-4v}\edtext{einmal
                  nachmittags zu Ihnen kommen}{\lemma{\textnormal{\emph{einmal … kommen}}}\Cendnote{\textnormal{siehe A. S.: \emph{Tagebuch}, 16. 12. 1903}}}\label{K_L03354-4h} zu können.\pend
           
\pstart
           Herzlichst {\\[\baselineskip]}Ihr {\\[\baselineskip]}\spacefill\mbox{Salten}\pend
           \leftskip=0em{}
\pstart
           \noindent{}Herrn Dr. Arthur Schnitzler\pend
           
\pstart
           \textcolor{pink}{\so{Wien.}}{}\ledrightnote{\textcolor{pink}{Wien}}\pend
           
\pstart
           \textcolor{gray}{\textbf{\emph{Alle für »\textcolor{brown}{Die Zeit}{}\ledrightnote{\textcolor{brown}{Die Zeit}}«
                        bestimmten Zuschriften und Sendungen sind an die Redaction »\textcolor{brown}{Die Zeit}{}\ledrightnote{\textcolor{brown}{Die Zeit}}« und \textbf{nicht} an die
                        Person eines der Herausgeber oder Mitarbeiter zu richten.}}}\pend
           \endnumbering\briefempfaengerindex{Schnitzler, Arthur@\textsc{Schnitzler, Arthur}!zzzSalten, Felix@\emph{von Felix Salten}!1903-12-091@{9. 12. 1903}|)be}\mylabel{h}  \normalsize

\doendnotes{C}
\bigskip
\vfill

\clearpage

\footnotesize

\lohead{\textsc{register}}

% Definiere theindex-Environment komplett neu ohne reledmac
\makeatletter
\renewenvironment{theindex}{%
  \section*{\indexname}%
  \setlength{\parindent}{0pt}%
  \setlength{\parskip}{0pt plus 0.3pt}%
  \let\item\@idxitem
}{%
  \clearpage
}
\makeatother

\IfFileExists{\jobname-pw.ind}{\input{\jobname-pw.ind}}{}

\end{document}

      