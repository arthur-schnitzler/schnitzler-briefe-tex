%% latex-korrekturansicht-vorspann.tex
%% Vorspann für die Korrekturansicht.
%% Lädt die gemeinsame Datei latex-vorspann.tex mit gesetztem Schalter.

\newif\ifkorrekturansicht
\korrekturansichttrue

\input{../tex-inputs/latex-vorspann}


               \section[Arthur Schnitzler an Bertha von Suttner, 24. 10. 1913]{ Arthur Schnitzler an Bertha von Suttner,
                    24. 10. 1913}\nopagebreak\mylabel{v}\rehead{ }\normalsize\beginnumbering\briefempfaengerindex{Suttner, Bertha von@\textsc{Suttner, Bertha von}!zzzSchnitzler, Arthur@\emph{von Arthur Schnitzler}!1913-10-241@{24. 10. 1913}|(be} \toendnotes[C]{\smallbreak\pagebreak[2]} \Standort{Genf, United Nations Archives, BvS/27/352-1.}
\physDesc{Briefkarte
\newline{}Handschrift: schwarze Tinte, lateinische Kurrent\newline{}Ordnung: mit Bleistift von unbekannter Hand beschriftet: »X12« }\toendnotes[C]{\smallbreak}\pstart
           \noindent{}{\pb}\textcolor{gray}{\textbf{Dr. Arthur Schnitzler}}\hfill 24. X. 913\pend
           \pstart
           \textcolor{gray}{\textbf{\textcolor{pink}{Wien XVIII. Sternwartestrasse 71}{}\ledrightnote{\textcolor{pink}{Sternwartestraße}}}}\pend
           \pstart{}Verehrte Frau Baronin,\pend\pstart
           ich danke Ihnen für Ihr freundliches Schreiben. Sie in unserm Haus empfangen zu
                    dürfen \damage{wi}rd uns ein besonders Vernügen sein, \damage{u}nd we{\geminationn} Ihnen, verehrte Frau Baronin,
                        \damage{di}e Zeit passt, so hoffen wir Sie {\pb}am \introOben{}nächsten\introOben{}{ }\label{K_L02154_1v}\edtext{Mittwoch}{\lemma{\textnormal{\emph{Mittwoch}}}\Cendnote{\textnormal{siehe A. S.: \emph{Tagebuch}, 29. 10. 1913}}}\label{K_L02154_1h}{ }\introOben{}(29.)\introOben{} Nachmittag um 5 bei uns
                    zu sehen.\pend
           \pstart
           Ich freue mich sehr darauf eine alte Beka{\geminationn}tschaft zu
                    erneuern, die allzulange unterbrochen war, und bin bis dahin mit den
                    verbindlichsten Grüßen Ihr aufrichtig ergebener\pend
           \pstart \spacefill\mbox{Arthur Schnitzler}\pend{}\endnumbering\briefempfaengerindex{Suttner, Bertha von@\textsc{Suttner, Bertha von}!zzzSchnitzler, Arthur@\emph{von Arthur Schnitzler}!1913-10-241@{24. 10. 1913}|)be}\mylabel{h}  \normalsize

\doendnotes{C}
\bigskip
\vfill

\clearpage

\footnotesize

\lohead{\textsc{register}}

% Definiere theindex-Environment komplett neu ohne reledmac
\makeatletter
\renewenvironment{theindex}{%
  \section*{\indexname}%
  \setlength{\parindent}{0pt}%
  \setlength{\parskip}{0pt plus 0.3pt}%
  \let\item\@idxitem
}{%
  \clearpage
}
\makeatother

\IfFileExists{\jobname-pw.ind}{\input{\jobname-pw.ind}}{}

\end{document}

      