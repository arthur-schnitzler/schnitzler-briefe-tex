%% latex-korrekturansicht-vorspann.tex
%% Vorspann für die Korrekturansicht.
%% Lädt die gemeinsame Datei latex-vorspann.tex mit gesetztem Schalter.

\newif\ifkorrekturansicht
\korrekturansichttrue

\input{../tex-inputs/latex-vorspann}


\renewcommand{\erwaehntePersonen}{Personen: Richard Beer-Hofmann, Paula Beer-Hofmann, Max Eugen Burckhard, Jakob Julius David, Julius von Gans-Ludassy, Stefan Großmann, Maximilian Harden, Hugo von Hofmannsthal, Heinrich Kanner, Louise Metzl, Wolfgang Amadeus Mozart, Josef Redlich, Anna Katharina Rehmann, Ottilie Salten, Paul Salten, Olga Schnitzler, Wilhelm Singer, Leopold Ferdinand Salvator Wölfling}
\renewcommand{\erwaehnteInstitutionen}{Institutionen: Arbeiter-Zeitung, B.Z. am Mittag, Burgtheater, Concordia, Die Zeit, Neues Wiener Tagblatt, Wiener Allgemeine Zeitung, Wiener Beamtenverein}
\renewcommand{\erwaehnteOrte}{Orte: Berlin, Hotel Saxonia, Kantstraße, Kochstraße, Russland, Volkstheater, Wien, Zoologischer Garten Berlin}
\renewcommand{\erwaehnteWerke}{Werke: ?? [Ludassy will von Salten erpresst worden sein], B.Z. am Mittag, Der Ruf des Lebens. Schauspiel in drei Akten, Der goldene Boden. Volksstück in vier Aufzügen, Der letzte Knopf. Volksstück in drei Aufzügen, Deutsches Volkstheater. (»Der goldene Boden«, Volksstück in vier Aufzügen von Julius v. Gans-Ludassy. 26. März), Deutsches Volkstheater. (»Der letzte Knopf.« Volksstück in drei Aufzügen von J. v. Gans-Ludassy), Die Zeit, Die Zukunft, Erinnerungen, Extrapost. Unparteiische Montags-Zeitung, Frankfurter Zeitung, Gedenkrede auf Wolfgang Amade Mozart, Kritik der Kritik. (Erbauliches aus der »Concordia«.), Oedipus und die Sphinx. Tragödie in drei Aufzügen, Russisches Theater. I, Russisches Theater. II, Theater, Wiener Allgemeine Zeitung}
\section[ Felix Salten an Arthur Schnitzler, 9. 3. 1906]{Felix Salten an Arthur Schnitzler, 9. 3. 1906}
\nopagebreak\mylabel{v}
\rehead{ }\normalsize\beginnumbering\briefempfaengerindex{Schnitzler, Arthur@\textsc{Schnitzler, Arthur}!zzzSalten, Felix@\emph{von Felix Salten}!1906-03-091@{9. 3. 1906}|(be}
\toendnotes[C]{\smallbreak\pagebreak[2]}\Standort{CUL, Schnitzler, B 89, B 1.}
\physDesc{Brief, 1 Blatt, 3 Seiten, 5502 Zeichen
\newline{}Handschrift: schwarze Tinte, lateinische Kurrent
\newline{}Ordnung: mit Bleistift von unbekannter Hand nummeriert: »206« }\toendnotes[C]{\smallbreak}
\pstart
           \noindent{}{\pb}\textcolor{brown}{\textcolor{gray}{\textbf{\emph{B. Z. am Mittag}}}}{}\ledrightnote{\textcolor{brown}{B.Z. am Mittag}}\hfill \textcolor{gray}{\textbf{\emph{\textcolor{pink}{BERLIN SW}{}\ledrightnote{\textcolor{pink}{Berlin}},}}}{ }9. III. 06.\pend
           
\pstart
           \textcolor{gray}{\textbf{\emph{Chefredaktion}}}\hfill \textcolor{gray}{\textbf{\emph{\textcolor{pink}{Kochstr. 23–25}{}\ledrightnote{\textcolor{pink}{Kochstraße}}}}}\pend
           
\pstart
           Lieber, hier sende ich Ihnen das \label{K_L03415-1v}\edtext{\textcolor{green}{Feuilleton}{}\ledrightnote{{$\rightarrow$}\textcolor{green}{Russisches Theater. I}}}{\lemma{\textnormal{\emph{Feuilleton}}}\Cendnote{\textnormal{\textcolor{blue}{Felix Salten}: \emph{\textcolor{green}{Russisches Theater. I}}. In: \emph{\textcolor{green}{B. Z. am Mittag}}, Jg. 30, Nr. 55, 6. 3. 1906, S. 2.}}}\label{K_L03415-1h} – das einzige, das
               bisher kam – aus der »\textcolor{green}{B. Z.}{}\ledrightnote{\textcolor{green}{B.Z. am Mittag}}« Montag will ich nochmals \label{K_L03415-2v}\edtext{\textcolor{green}{über die \textcolor{pink}{Russ}{}\ledrightnote{{$\rightarrow$}\textcolor{pink}{Russland}}en schreiben}{}\ledrightnote{{$\rightarrow$}\textcolor{green}{Russisches Theater. I}}}{\lemma{\textnormal{\emph{über … schreiben}}}\Cendnote{\textnormal{Vermutlich: \textcolor{blue}{Felix Salten}: \emph{\textcolor{green}{Russisches Theater. II}}. In: \emph{\textcolor{green}{B. Z. am Mittag}}, Jg. 30, Nr. 70, 23. 3. 1906, S. 2–3.}}}\label{K_L03415-2h}, und schicke es
               Ihnen dann gleich zu. Dass Sie so \label{K_L03415-3v}\edtext{verstimmt von hier weggingen}{\lemma{\textnormal{\emph{verstimmt … weggingen}}}\Cendnote{\textnormal{\textcolor{blue}{Schnitzler} war anlässlich der Uraufführung
                  von \emph{\textcolor{green}{Der Ruf des Lebens}} in \textcolor{pink}{Berlin} gewesen und am 27. 2. 1906 heimgekehrt. Zu diesem Zeitpunkt
                  waren bereits einige negative Kritiken erschienen.}}}\label{K_L03415-3h}, hat auch auf mich
               deprimirend gewirkt. Dieser »\textcolor{green}{Ruf des Lebens}{}\ledrightnote{\textcolor{green}{Der Ruf des Lebens. Schauspiel in drei Akten}}«
               schien mir so unbezweifelbar, und ist es mir noch, dass seine Aufnahme für mich eine
               symptomatische Bedeutung annahm.\pend
           
\pstart
           Es ist ein Glück, dass Sie stark genug sind, um sich kommende Produktion durch
               solche, an sich keineswegs wichtige Zwischenfälle, stören zu laßen. Darauf rechne ich
               sehr, und hoffe, bald von Ihnen zu hören, dass Sie arbeiten. Schlimm wäre es ja nur,
               wenn Sie, – mehr aus künstlerischer Hypochondrie als aus Selbstkritik – anfangen
               würden, in Ihrer Abschätzung dieses \textcolor{green}{Stück}{}\ledrightnote{{$\rightarrow$}\textcolor{green}{Der Ruf des Lebens. Schauspiel in drei Akten}}es wankend zu werden. Da kann man freilich für eine Weile den Boden
               unter sich schwinden fühlen. Aber es wäre, besonders in diesem Falle, das Falscheste!
               Sie müssen unbedingt dabei bleiben, dass Ihr \textcolor{green}{Stück}{}\ledrightnote{{$\rightarrow$}\textcolor{green}{Der Ruf des Lebens. Schauspiel in drei Akten}} im Recht ist, und dass die Zufälligkeit eines Abends nichts
               beweist. Dass \label{K_L03415-4v}\edtext{\textcolor{blue}{Harden}{}\ledrightnote{\textcolor{blue}{Maximilian Harden}}{ }\uline{so}{ }\textcolor{green}{geschrieben}{}\ledrightnote{{$\rightarrow$}\textcolor{green}{Theater}}}{\lemma{\textnormal{\emph{Harden so geschrieben}}}\Cendnote{\textnormal{Bezug auf eine gemeinsame Besprechung
                  der Aufführungen von \textcolor{blue}{Hofmannsthal}s \emph{\textcolor{green}{Oedipus und die Sphinx}} und \textcolor{blue}{Schnitzler}s \emph{\textcolor{green}{Der Ruf des
                     Lebens}}: \textcolor{blue}{M. H.} [ = \textcolor{blue}{Maximilian Harden}]: \emph{\textcolor{green}{Theater}}. In: \emph{\textcolor{green}{Die Zukunft}}, Bd. 54,
                     H. 9, 3. 3. 1906, S. 346–356.}}}\label{K_L03415-4h} hat,
               ist im ersten Moment für Ihr Empfinden vielleicht sehr verletzend gewesen; tut aber
               wirklich nichts. Hätte er die \textcolor{green}{Sache}{}\ledrightnote{{$\rightarrow$}\textcolor{green}{Der Ruf des Lebens. Schauspiel in drei Akten}} ausführlich und mit der ganzen Kraft seiner Dialektik zerrupft und
               zergliedert, dann wäre es schlimmer gewesen, denn es hätte \uline{gewirkt}. So aber hat \textcolor{pink}{hier}{}\ledrightnote{{$\rightarrow$}\textcolor{pink}{Berlin}}, – und wol überall – jeder nur die Achsel gezuckt und gesagt: Das
               glaubt \textcolor{blue}{Harden}{}\ledrightnote{\textcolor{blue}{Maximilian Harden}} selber nicht. Die Politik war
               gar zu sichtbar, als dass ein kritischer Einfluß erfolgen könnte.\pend
           
\pstart
           Nach und nach kommt meine \label{K_L03415-5v}\edtext{\textcolor{pink}{Wohnung}{}\ledrightnote{{$\rightarrow$}\textcolor{pink}{Kantstraße}}}{\lemma{\textnormal{\emph{Wohnung}}}\Cendnote{\textnormal{siehe Felix Salten an Arthur Schnitzler, 29. 1. 1906}}}\label{K_L03415-5h} in Ordnung, und ich kann eine menschliche Existenz beginnen. Könnte ich jetzt
               wieder von \textcolor{pink}{hier}{}\ledrightnote{{$\rightarrow$}\textcolor{pink}{Berlin}} auswandern,
               dann wäre ich schon imstande, ein nettes Buch über \textcolor{pink}{Berlin}{}\ledrightnote{\textcolor{pink}{Berlin}} zu schreiben. Aber, ich hoffe, dass ich hier nicht sterben muß, und
               doch einmal werde reden können. Nach \textcolor{pink}{Wien}{}\ledrightnote{\textcolor{pink}{Wien}} sehne
               ich mich aber auch nicht. Dazu liegt mir die Schweinerei der letzten Affären noch zu
               sehr im Magen. Haben Sie die letzte \label{K_L03415-6v}\edtext{Schurkerei des dramatischen Dichters \textcolor{blue}{Ludassy}{}\ledrightnote{\textcolor{blue}{Julius von Gans-Ludassy}}}{\lemma{\textnormal{\emph{Schurkerei … Ludassy}}}\Cendnote{\textnormal{In seinen \emph{\textcolor{green}{Erinnerungen}} schilderte \textcolor{blue}{Salten} die Sache wie folgt: »Jetzt muss ich doch noch die Affaire
                        \textcolor{blue}{Ludassy} erzählen, unter der ich
                     kindischerweise länger als ein Jahr schmerzhaft gelitten habe. \textcolor{blue}{Ludassy} war mein erster Chefredakteur und
                     er war gewalttätig, wie man sich aus dem Aufbrechen des Schreibtisches von \textcolor{blue}{J. J. David} erinnern wird. Jetzt war er
                     nicht mehr Chefredakteur und ich bei der ›\textcolor{brown}{Zeit}‹. Nun veranstaltete die \textcolor{brown}{Concordia} eine Protestversammlung gegen den Chefredakteur der \textcolor{brown}{Zeit}, Dr. \textcolor{blue}{Kanner}, mit der Anklage, er brülle die Redakteure an. Ich erschien zu
                     seiner Verteidigung und sagte unter anderem, das Schreien bedeutet gar nichts.
                        {[}›{]}Hier neben mir sitzt mein Freund \textcolor{blue}{Ludassy}, der mein erster Chefredakteur gewesen ist und
                     der auch geschrien hat. Deswegen sind wir doch befreundet!‹ \textcolor{blue}{Ludassy} rückte von mir ab, murrte: ›Ihr Freund bin ich
                     gewesen und diese Worte werden Sie bereu{[}e{]}n!‹ Ich musste
                     diese Worte, so harmlos sie auch gemeint waren{[},{]} länger als
                     ein Jahr bitterlich bereuen. Denn Herr Dr. \textcolor{blue}{Ludassy}, der mit einem Theaterstück: \textcolor{green}{der letzte Knopf} im \textcolor{pink}{Deutschen
                        Volkstheater} zur Aufführung gelangte, schrieb ungefähr ein Jahr nach
                     der Aufführung, es sei vor seiner Premiere, der Mann mit dem Revolver vor ihm
                     gestanden.« (\emph{Wienbibliothek im Rathaus}, Nachlass \textcolor{blue}{Salten}, ZPH 1681/1 1.1.1.9.1, [S. 61].) Die
                  Stelle, die \textcolor{blue}{Salten} meint, wird von einer \textcolor{pink}{Wien}er Zeitung so zitiert: »Herr Dr.
                           \textcolor{blue}{von Ludaſſy} ließ ſich über einen
                           \textcolor{pink}{Wien}er Kritiker in einer \textcolor{pink}{Berliner} Wochenſchrift wie folgt aus:{ / }›Je weniger die handfeſten Burſchen verſtehen, deſto hochmütiger,
                        abſprechender und unflätiger ſchreiben ſie. Es gibt deren auch, die vor der
                        Aufführung den Autor um höhere Beträge anzuzumpen verſuchen (Ich habe ſelbſt
                        eine ſolche Kreatur als Zeitungsherausgeber zum Kritiker gemacht; als ich
                        auf der Bühne als Autor mein Heil verſuchte, ſtand dann dasſelbe Individuum
                        mit dem Revolver vor mir.) Antikritik hilft gegen ſolche Schelme nicht. Denn
                        niemand, dem ein unehrlicher oder übermütiger Kritiker Leides zugefügt hat,
                        will die Beſtie reizen.‹« (y.: \emph{\textcolor{green}{Kritik der Kritik. (Erbauliches aus der »Concordia«.)}}.
                     In: \emph{\textcolor{green}{Wiener Montags-Journal}}, Jg. 24,
                     Nr. 1.245, 18. 12. 1905, S. 3–4.) \textcolor{blue}{Salten}s Narration ist in der Chronologie
                  unzuverlässig. Die zeitliche Einordnung der Ereignisse lässt sich mit der
                  Uraufführung von \emph{\textcolor{green}{Der letzte Knopf}} am 8. 4. 1900 nur
                  scheinbar vornehmen. \textcolor{blue}{Salten}s \textcolor{green}{Feuilleton} erschien zwei Tage später: \textcolor{blue}{f. s.}: \emph{\textcolor{green}{Deutsches Volkstheater. (»Der letzte Knopf.« Volksstück in drei Aufzügen
                        von J. v. Gans-Ludassy)}}. In: \emph{\textcolor{green}{Wiener
                        Allgemeinen Zeitung. 6 Uhr-Blatt}}, Nr. 6.628, 10. 4. 1900, S. 2–3. Damals gab es aber die Tageszeitung \emph{\textcolor{brown}{Die Zeit}} noch nicht, sodass er das Stück
                  verwechselt haben dürfte. Zeitlich passender ist die Uraufführung von \textcolor{blue}{Ludassy}s \emph{\textcolor{green}{Der
                     goldene Boden. Volksstück in vier Aufzügen}}, die am 26. 3. 1904 (in
                  Anwesenheit \textcolor{blue}{Schnitzler}s) am \textcolor{pink}{Deutschen Volkstheater} stattfand. Eine Besprechung \textcolor{blue}{Salten}s lässt sich nicht nachweisen, doch
                  dürfte das daran liegen, dass unmittelbar davor ein längeres Feuilleton von \textcolor{blue}{Salten} abgedruckt worden war und der Eindruck
                  vermieden werden sollte, dass das \textcolor{brown}{Blatt} zu wenige Beiträgerinnen und Beiträger habe. Dementsprechend wäre
                  das Kürzel »mm« \textcolor{blue}{Salten} zuzuordnen: \textcolor{blue}{mm} [ = \textcolor{blue}{Felix Salten}?]: \emph{\textcolor{green}{Deutsches
                        Volkstheater. (»Der goldene Boden«, Volksstück in vier Aufzügen von Julius
                        v. Gans-Ludassy. 26. März)}}. In: \emph{\textcolor{green}{Die
                        Zeit}}, Jg. 3, Nr. 538, 27. 3. 1904,
                     S. 3. Den zitierten »Revolver« in \textcolor{blue}{Salten}s \emph{\textcolor{green}{Erinnerungen}}
                  klärte er an einer anderen Stelle: »Mein ehemaliger Chef \textcolor{blue}{Ludassy} verleumdete mich, ich hätte vor seiner Premiere
                     von ihm 3000 Kronen erpressen wollen. Es war mir leicht ihn zu widerlegen. Der
                     damalige Erzherzog \textcolor{blue}{Leopold Ferdinand}
                     suchte einen Kredit in dieser Höhe und ich fragte \textcolor{blue}{Ludassy} um Rat.« (ZPH 1681/1 1.1.1.9.1,
                     [S. 4]) Die \textcolor{green}{Behauptung}{ }\textcolor{blue}{Ludassy}s, es wäre vor der Premiere versucht
                  worden, ihn zu erpressen, entwickelte sich in der Darstellung \textcolor{blue}{Salten}s auf folgende Weise weiter:
                     »Verabredetermaßen fragte \textcolor{blue}{Stephan
                        Grossmann} in der \textcolor{brown}{Arbeiterzeitung}
                     nach dem Namen des Revolvermanns. Dr \textcolor{blue}{Ludassy} nannte mich. Worauf mich \textcolor{blue}{Stephan Grossmann} mit einem Kübel Unrat überschüttete. Ich kam mir in
                     meiner persönlichen und wegen meiner publizistischen Ehre schwer verletzt vor
                     und rief ein Ehrengericht gegen mich an. Bei dieser Ehrengerichtlichen
                     Verhandlung legte ich folgende Beweise vor:  1. Ich hatte \textcolor{blue}{Ludassy} nur im Namen des Erzherzog \textcolor{blue}{Leopold}s gefragt, wo man einen Kredit von 8000 Kronen
                     für den Erzherzog aufnehmen könnte{[}.{]} (Dieser Kredit wurde
                     ihm wenig später vom \textcolor{brown}{Beamtenverein}
                        erteilt{[}.{]}) 2. Ich legte mein \textcolor{green}{Feuilleton} über \textcolor{blue}{Ludassy}s \textcolor{green}{Stück} vor, das eine Lobeshymne darstellte. 3. Ich
                     legte eine Reihe von Briefen und Eilpostkarten \textcolor{blue}{Ludassy}s vor, in denen er teils für meine \textcolor{green}{Kritik} heissen Dank
                     aussprach, teils noch lange nach der Premiere und nach meiner \textcolor{green}{Kritik} mir Briefe und Eilkarten
                        schrieb{[},{]} in denen er verlangte mich zu sehen, in denen
                     er meine Freundschaft pries, und die seinige beteuerte. Der Präsident dieses
                     Ehrenrates, Chefredakteur des \textcolor{brown}{Neuen Wiener
                        Tagblatt}es{[},{]}{ }\textcolor{blue}{Wilhelm Singer}, richtete mitten in der
                     Verhandlung an \textcolor{blue}{Ludassy} die Frage: ›Sagen
                     Sie Herr Dr. schämen Sie sich denn gar nicht?!‹ \textcolor{blue}{Ludassy} wurde zu einer schweren Rüge von der \textcolor{brown}{Concordia} verurteilt und zur Unfähigkeit
                     zwei Jahre lang ein Ehrenamt in der \textcolor{brown}{Concordia} zu bekleiden.« (ebd., [S. 61–62].) Das Ehrengericht des
                     \textcolor{brown}{Journalistenverband}s
                  entschied am 12. 5. 1907 zugunsten \textcolor{blue}{Salten}s.
                  Die Rüge für \textcolor{blue}{Ludassy} lässt sich belegen,
                  doch wurde er nur für ein Jahr von jeglichen Ehrenämtern der \emph{\textcolor{brown}{Concordia}} ausgeschlossen (vgl.
                        \emph{Wienbibliothek im Rathaus}, Nachlass \textcolor{blue}{Salten}, ZPH 1681, 3.7.4). \textcolor{blue}{Salten} schrieb in seinen \emph{\textcolor{green}{Erinnerungen}} weiter: »Damit beruhigte ich mich aber nicht, rief
                     ein zweites Ehrengericht an, das aus Prof. Dr. \textcolor{blue}{Joseph Redlich}, aus dem früheren Direktor des \textcolor{brown}{Burgtheater}s Dr. \textcolor{blue}{Max Burkhard} und aus zwei anderen hohen Richtern
                     bestand, die zwar keine Strafverfügung treffen konnten, deren Urteil aber mir
                     volle Genugtuung bot. Es hatten sich einige meiner Feinde zwar gemeldet, die
                     ich nur zum Teil persönlich kannte, und deren Zeugnis glatt abgewiesen
                     wurde.« (ZPH 1681/1 1.1.1.9.1, [S. 62].) Siehe auch Felix Salten an Arthur Schnitzler, [18.? 10. 1906] sowie A. S.: \emph{Tagebuch}, 30. 12. 1905, A. S.: \emph{Tagebuch}, 14. 1. 1906 und 12. 5. 1907.}}}\label{K_L03415-6h}
               Jemandem erzählt? Wenn nicht, dann tun Sie’s doch, bitte. Es ist das Empörendste,
               dass so ein niederes {\pb}durch und
               durch verseuchtes \textcolor{blue}{Luder}{}\ledrightnote{{$\rightarrow$}\textcolor{blue}{Julius von Gans-Ludassy}}
               einen monatelang zwischen seinen Fingern halten darf; Na, Sie haben mich einmal einen
               »guten Hasser« genannt, – nicht ganz mit Recht, denn ich habe mich bisher noch nie an
               Jemandem gerächt. Aber diesmal will ich mir den Titel verdienen. So oder so. Und wenn
               nur der Prozess endlich anberaumt wird – ich hab mir’s genau überlegt – ich tue
               nichts, um ihn hinauszuschieben, dann will ich dafür sorgen, dass diesmal der
               Angeklagte wirklich Angeklagter sein soll.\pend
           
\pstart
           Übrigens, laßen wir das. Es gibt, gottseidank, bessere Menschen. Z. B. \textcolor{blue}{Beer-Hofmann}{}\ledrightnote{\textcolor{blue}{Richard Beer-Hofmann}}, nicht wahr? Wie finden Sie es,
               dass er mir bis heute noch keine Zeile schrieb, keine
               Karte, nichts! Dabei bin ich doch nicht einfach nur verreist, bin in einer
               Lebensepoche, in der es nicht ganz unwichtig ist, die Festigkeit gewisser Beziehungen
               zu spüren, bin in einer Situation, in der es \uline{vielleicht} sogar tröstlich, \uline{jedenfalls} aber
               animirend sein kann, von Freunden was zu hören. Dabei hab \uline{ich}, mitten im Übersiedlungsrummel, im Fieber der neuen \textcolor{brown}{Stellung}{}\ledrightnote{{$\rightarrow$}\textcolor{brown}{B.Z. am Mittag}}, in der Unrast des \textcolor{pink}{Hotel}{}\ledrightnote{{$\rightarrow$}\textcolor{pink}{Hotel Saxonia}}wohnens an \textcolor{blue}{B-H.}{}\ledrightnote{\textcolor{blue}{Richard Beer-Hofmann}} geschrieben, als ich sein \label{K_L03415-7v}\edtext{\textcolor{green}{\textcolor{blue}{Mozart}{}\ledrightnote{\textcolor{blue}{Wolfgang Amadeus Mozart}} Feuilleton}{}\ledrightnote{\textcolor{green}{Gedenkrede auf Wolfgang Amade Mozart}}}{\lemma{\textnormal{\emph{Mozart Feuilleton}}}\Cendnote{\textnormal{\textcolor{blue}{Richard Beer-Hofmann}: \emph{\textcolor{green}{Gedenkrede auf Wolfgang Amadé Mozart}}. In: \emph{\textcolor{green}{Frankfurter Zeitung}}, Jg. 50, Nr. 27, 28. 1. 1906, Erstes Morgenblatt, S. 1–2. \textcolor{blue}{Mozart} hätte am 27. 1. 1906 seinen 150. Geburtstag gefeiert.}}}\label{K_L03415-7h} las (auch dazu
               hatte ich Zeit gefunden){[},{]} dabei hatte ich noch ein \uline{zweitesmal} an ihn eine Karte
                  geschickt\textcolor{gray}{.} Dabei hat \textcolor{blue}{Otti}{}\ledrightnote{\textcolor{blue}{Ottilie Salten}} an \textcolor{blue}{Frau Beer-Hofmann}{}\ledrightnote{\textcolor{blue}{Paula Beer-Hofmann}}
               geschrieben. Und nichts. Nett, nicht wahr?, wenn dann die »besseren Menschen« \uline{so} aussehen. Ich hoffe, dass Sie mich so sehr arg
               nicht missverstehen, und für Empfindlichkeit oder gar für Beleidigtsein nehmen, was
               nur ein ganz klares Abrechnen ist. Bei diesem Abrechnen sind \uline{alle} mildernden Umstände, \uline{alle}
               psychologischen Möglich\substVorne{}\textsuperscript{\textcolor{gray}{g}\textcolor{gray}{×}}\substDazwischen{}k\substHinten{}eiten nachfühlenden Begreifens schon in Anschlag gebracht, mit dem Resultat:
               man kann \uline{immer} eine \uline{Karte} schreiben! \uline{eine} Zeile! Ich meine,
               dieses ist jenseits von Empfindlichkeit und Beleidigtsein. Es ist ganz, ganz was
               anderes! Das alles unter uns und im Vertrauen. Ich muß mich über diese Sache
               aussprechen, hab es gestern an \textcolor{blue}{Hofmannsthal}{}\ledrightnote{\textcolor{blue}{Hugo von Hofmannsthal}} gethan, und that es heute an Sie. \substVorne{}\textsuperscript{\textcolor{gray}{W}}\substDazwischen{}De\substHinten{}nn so ganz einfach und wortlos mochte ich diese neueste Erfahrung nicht »zu
               den übrigen legen.« Will aber keine Diskussion mit \textcolor{blue}{B\textcolor{gray}{.}-H.}{}\ledrightnote{\textcolor{blue}{Richard Beer-Hofmann}}, weil die Sache absolut nicht diskutirbar und
               für mich erledigt ist. Will auch nicht, dass dritte Personen drum wissen, weil {\dots} weil ich mich schäme!\pend
           
\pstart
           Wenn die Kur, die ich gebrauche (Kohlensäure Bäder und Vibrations-Massage) vorbei
               ist, wenn es wirklich Frühling geworden, fange ich gleich mit einer Arbeit an. Das
               ist so gut an \textcolor{pink}{Berlin}{}\ledrightnote{\textcolor{pink}{Berlin}}, dass man hier nur am
               Arbeiten Freude hat, an nichts anderem. Nicht am Spazierengehen, nicht an
               Landparthien, nicht an gemütlichem Schwatz und nicht an irgend welchen anderen
               freundlichen aber zeitraubenden Dingen. Man muß immer arbeiten, den ganzen Tag
               arbeiten, wenn man sich wol fühlen will.\pend
           
\pstart
           {\pb}Eines ist mir sehr erfreulich
                  \textcolor{pink}{hier}{}\ledrightnote{{$\rightarrow$}\textcolor{pink}{Berlin}}, wenns nur so bleibt:
               dass die \textcolor{blue}{Kinder}{}\ledrightnote{{$\rightarrow$}\textcolor{blue}{Anna Katharina Rehmann}{\newline}{$\rightarrow$}\textcolor{blue}{Paul Salten}}
               sich so wol fühlen, und so brav essen. \textcolor{blue}{Annerl}{}\ledrightnote{\textcolor{blue}{Anna Katharina Rehmann}}
               spricht jetzt schon so viel wie der \textcolor{blue}{Paul}{}\ledrightnote{\textcolor{blue}{Paul Salten}}, und
               ist so lieb, dass sich’s kaum sagen läßt. Neulich waren wir zum ersten Mal im \textcolor{pink}{Zoo}{}\ledrightnote{\textcolor{pink}{Zoologischer Garten Berlin}}\textcolor{gray}{×}. Und im Nilpferdhaus waren beide
                  \textcolor{blue}{Kinder}{}\ledrightnote{{$\rightarrow$}\textcolor{blue}{Anna Katharina Rehmann}{\newline}{$\rightarrow$}\textcolor{blue}{Paul Salten}} sprachlos
               vor Staunen. Da fing das eine Nilpferd laut zu schnauben und zu wiehern an, und \textcolor{blue}{Paul}{}\ledrightnote{\textcolor{blue}{Paul Salten}} war darüber so entsetzt, dass er in Thränen
               ausbrach, \textcolor{blue}{Annerl}{}\ledrightnote{\textcolor{blue}{Anna Katharina Rehmann}} aber rief dem Nilpferd zu:
               »Sei still, Nilpferd, sonst muß \textcolor{blue}{Pauli}{}\ledrightnote{\textcolor{blue}{Paul Salten}} weinen!«
               Und \textcolor{blue}{Pauli}{}\ledrightnote{\textcolor{blue}{Paul Salten}} erzählte zu Hause der \textcolor{blue}{Grossmama}{}\ledrightnote{{$\rightarrow$}\textcolor{blue}{Louise Metzl}}, das
               Nilpferd habe »mit dem Mund ein Gewitter gemacht!« Daran ließe sich etwa ein
               verallgemeinerndes Aphorisma knüpfen, was ich aber unterlaße.\pend
           
\pstart
           Viele herzliche Grüße von \textcolor{blue}{uns}{}\ledrightnote{{$\rightarrow$}\textcolor{blue}{Ottilie Salten}} zu \textcolor{blue}{Ihnen}{}\ledrightnote{{$\rightarrow$}\textcolor{blue}{Olga Schnitzler}}.
               {\\[\baselineskip]}Ihr {\\[\baselineskip]}\spacefill\mbox{Salten}\pend
           \leftskip=0em{}\endnumbering\briefempfaengerindex{Schnitzler, Arthur@\textsc{Schnitzler, Arthur}!zzzSalten, Felix@\emph{von Felix Salten}!1906-03-091@{9. 3. 1906}|)be}\mylabel{h}  \normalsize

\doendnotes{C}
\bigskip
\vfill

\clearpage

\footnotesize

\lohead{\textsc{register}}

% Definiere theindex-Environment komplett neu ohne reledmac
\makeatletter
\renewenvironment{theindex}{%
  \section*{\indexname}%
  \setlength{\parindent}{0pt}%
  \setlength{\parskip}{0pt plus 0.3pt}%
  \let\item\@idxitem
}{%
  \clearpage
}
\makeatother

\IfFileExists{\jobname-pw.ind}{\input{\jobname-pw.ind}}{}

\end{document}

      