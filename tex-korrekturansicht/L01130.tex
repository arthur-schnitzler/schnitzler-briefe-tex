%% latex-korrekturansicht-vorspann.tex
%% Vorspann für die Korrekturansicht.
%% Lädt die gemeinsame Datei latex-vorspann.tex mit gesetztem Schalter.

\newif\ifkorrekturansicht
\korrekturansichttrue

\input{../tex-inputs/latex-vorspann}


               \section[Ferdinand von Saar an Arthur Schnitzler, 19. 6. 1901]{ Ferdinand von Saar an Arthur Schnitzler, 19. 6. 1901}\nopagebreak\mylabel{v}\rehead{ }\normalsize\beginnumbering\briefempfaengerindex{Schnitzler, Arthur@\textsc{Schnitzler, Arthur}!zzzSaar, Ferdinand von@\emph{von Ferdinand von Saar}!1901-06-191@{19. 6. 1901}|(be} \toendnotes[C]{\smallbreak\pagebreak[2]} \Standort{CUL, Schnitzler, B 88.}
\physDesc{Brief, 1 Blatt, 2 Seiten
\newline{}Handschrift: schwarze Tinte, deutsche Kurrent
\newline{}Schnitzler: mit Bleistift nummeriert: »9« }\toendnotes[C]{\smallbreak}\pstart
           \raggedleft{}{\pb}\textcolor{pink}{\textsc{Wien-Döbling}}{}\ledrightnote{\textcolor{pink}{XIX., Döbling}}, 19/6. 1901.\pend
           \pstart{}Sehr verehrter Herr Doctor!\pend\pstart
           Ihre neueſten \textcolor{green}{Bücher}{}\ledrightnote{→\textcolor{green}{Lieutenant Gustl. Novelle}{\newline}→\textcolor{green}{Frau Bertha Garlan. Roman}} habe
               ich mit großer Aufmerkſamkeit geleſen, habe ſie in mir nachwirken laſſen – und ſo
               gelange ich erſt heute dazu, Ihnen für die ſo freundliche Überſendung zu danken. An
               beiden habe ich wieder Ihre bewährte Kraft der Seelenanalyſe und Milieuſchilderung
               bewundert. »\textcolor{green}{Lieutenant Guſtl}{}\ledrightnote{\textcolor{green}{Lieutenant Gustl. Novelle}}« iſt freilich mehr ein
               Virtuoſenſtück; hingegen erſcheint aber »\textcolor{green}{Frau Bertha
                  Garlan}{}\ledrightnote{\textcolor{green}{Frau Bertha Garlan. Roman}}« als ein umſo echteres Kunſtwerk. Man athmet die Luft der kleinen
               Landſtadt und lebt die öden, gedrückten Verhältniſſe mit, als befände man ſich dort.
               Daher kommt es auch, dſs man ſich ungefähr in der Mitte des \textcolor{green}{Buches}{}\ledrightnote{→\textcolor{green}{Frau Bertha Garlan. Roman}} fragt, ob dieſe Zuſtände ſo eingehender
               Behandlung auch wirklich werth ſeien – und man fängt an, ein bißchen ungeduldig zu
               werden. Aber die zweite Hälfte wirkt mit dem ergreifenden Schluß nach rückwärts wie
               ein mächtiger elektriſcher Lichtſtrom, der allein und vor allem der Heldin vollen
               Reiz und volle Bedeu{\pb}tung verleiht. Jeder Zug in dieſem ſtillen, ſtill
               verlangenden und eigentlich nichts erlebenden Frauenleben wird als nothwendig
               empfunden, prägt ſich tief ein, und ſo wird »\textcolor{green}{Frau Bertha
                  Garlan}{}\ledrightnote{\textcolor{green}{Frau Bertha Garlan. Roman}}« zu den Büchern gehören, die man niemals aus dem Gedächtniſſe
               verliert. Man hat ſie, wenn ich nicht irre, zu \label{K_L01130_1v}\edtext{\textcolor{green}{Madame Bovary}{}\ledrightnote{\textcolor{green}{Madame Bovary. Mœurs de province}} in Beziehung}{\lemma{\textnormal{\emph{Madame … Beziehung}}}\Cendnote{\textnormal{Auf \emph{\textcolor{green}{Madame
                     Bovary}} – \textcolor{blue}{Schnitzler} hatte den Roman mit
                  achtzehn Jahren gelesen (siehe A. S.: \emph{Tagebuch}, 14. 5. 1880) – als literarische »Vorlage« verweisen viele Rezensenten der
                  Novelle, vgl. z. B. \textcolor{blue}{Alfred Gold}: \emph{\textcolor{green}{Arthur Schnitzler: Frau Bertha Garlan}}. In: \emph{\textcolor{green}{Die Zeit}}, Nr. 344, 4. 5. 1901, S. 78
                  und [\textcolor{blue}{Joseph Victor Widmann}?]: \emph{\textcolor{green}{Kunst und Litteratur. Frau Bertha Garlan}}. In:
                        \emph{\textcolor{green}{Sonntagsblatt des Bund}}, Nr. 18,
                        5. 5. 1901, S. 141–142.}}}\label{K_L01130_1h} bringen wollen. Höchſt
               ungerechtfertigt! Denn es ist \uline{alles ganz} anders. Die
               einzige Ähnlichkeit, die man aber an den Haaren herbeiziehen müßte, beſteht darin:
               dſs beide Romane in der Provinz ſpielen. Aber ſo ſind die Menſchen: ſie können eben
               immer nur vergleichen! \pend
           \pstart
           Indem ich mich Ihnen mit wahrer Hochachtung empfehle, bin ich{\\[\baselineskip]}Ihr alt ergebener{\\[\baselineskip]}\spacefill\mbox{Ferdinand von Saar.}\pend
           \leftskip=0em{}\endnumbering\briefempfaengerindex{Schnitzler, Arthur@\textsc{Schnitzler, Arthur}!zzzSaar, Ferdinand von@\emph{von Ferdinand von Saar}!1901-06-191@{19. 6. 1901}|)be}\mylabel{h}  \normalsize

\doendnotes{C}
\bigskip
\vfill

\clearpage

\footnotesize

\lohead{\textsc{register}}

% Definiere theindex-Environment komplett neu ohne reledmac
\makeatletter
\renewenvironment{theindex}{%
  \section*{\indexname}%
  \setlength{\parindent}{0pt}%
  \setlength{\parskip}{0pt plus 0.3pt}%
  \let\item\@idxitem
}{%
  \clearpage
}
\makeatother

\IfFileExists{\jobname-pw.ind}{\input{\jobname-pw.ind}}{}

\end{document}

      