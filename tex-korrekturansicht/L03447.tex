%% latex-korrekturansicht-vorspann.tex
%% Vorspann für die Korrekturansicht.
%% Lädt die gemeinsame Datei latex-vorspann.tex mit gesetztem Schalter.

\newif\ifkorrekturansicht
\korrekturansichttrue

\input{../tex-inputs/latex-vorspann}


\renewcommand{\erwaehntePersonen}{Personen: George Willard Bonte, Ernest Nister, Louise Quarles Bonte}
\renewcommand{\erwaehnteOrte}{Orte: Bayern, Edmund-Weiß-Gasse, Hamburg, London, New York City, Ober-Klingen, Wien}
\renewcommand{\erwaehnteWerke}{Werke: ABC in Dixie. A Plantation Alphabet}
\section[ Paul Goldmann an Arthur Schnitzler, 2. 7. 1904]{Paul Goldmann an Arthur Schnitzler, 2. 7. 1904}
\nopagebreak\mylabel{v}
\rehead{ }\normalsize\beginnumbering\briefempfaengerindex{Schnitzler, Arthur@\textsc{Schnitzler, Arthur}!zzzGoldmann, Paul@\emph{von Paul Goldmann}!1904-07-021@{2. 7. 1904}|(be}
\toendnotes[C]{\smallbreak\pagebreak[2]}\Standort{DLA, A:Schnitzler, HS.NZ85.1.3174.}
\physDesc{Bildpostkarte
\newline{}Handschrift: 1) Bleistift, deutsche Kurrent\hspace{1em}2) Bleistift, lateinische Kurrent (\noindent{}Adresse)\hspace{1em}
\newline{}Versand: Stempel: »\nobreak{}\oindex{Hamburg@\textbf{Hamburg}, \emph{Besiedelter Ort (A.BSO)}|pwk}Hamburg 1, 2. 7. 04, 12–1N.\nobreak{}«.  }\toendnotes[C]{\smallbreak}\pstart{}{\pb}Herrn\pend{}\pstart{}Dr. Arthur Schnitzler\pend{}\pstart{}\textcolor{pink}{Wien}{}\ledrightnote{\textcolor{pink}{Wien}}\pend{}\pstart{}\textcolor{pink}{XVIII. Spöttelgaſse 7}{}\ledrightnote{\textcolor{pink}{Edmund-Weiß-Gasse}}\pend{}
{\bigskip}
\pstart
           \noindent{}\centering{}{\pb}{[}\label{K_L03447-1v}\edtext{Kunstdruck einer Wäscherin
                     mit schwarzer Hautfarbe von \textcolor{blue}{Bonte}{}\ledrightnote{\textcolor{blue}{Louise Quarles Bonte}{\newline}\textcolor{blue}{George Willard Bonte}}}{\lemma{\textnormal{\emph{Kunstdruck … Bonte}}}\Cendnote{\textnormal{Die Postkartenserie,
                        aus der das Motiv dieser Karte stammt, ist eine rassistische Seltsamkeit mit
                        Stereotypen schwarzer Plantagenarbeiter:innen. Die meisten (nicht alle)
                        Darstellungen fanden leicht modifiziert Verwendung in einem Kinderbuch von
                           \textcolor{blue}{Louise Quarles Bonte} und \textcolor{blue}{Georges Willard Bonte}: \emph{\textcolor{green}{ABC in Dixie. A Plantation Alphabet}},
                        das um die Zeit in \textcolor{pink}{London} und \textcolor{pink}{New York} erschien, jedoch in
                           \textcolor{pink}{Bayern} gedruckt war. Das deutet auf den aus \textcolor{pink}{Ober-Klingen} stammenden Verleger \textcolor{blue}{Ernest Nister} als den Vermittler
                  hin.}}}\label{K_L03447-1h}{]}\pend
           
\pstart
           \textcolor{pink}{\textsc{Hamburg}}{}\ledrightnote{\textcolor{pink}{Hamburg}}{ }2. Juli\pend
           
\pstart
           Herzliche Grüße!\pend
           \pstart \spacefill\mbox{P. G.}\pend{}\endnumbering\briefempfaengerindex{Schnitzler, Arthur@\textsc{Schnitzler, Arthur}!zzzGoldmann, Paul@\emph{von Paul Goldmann}!1904-07-021@{2. 7. 1904}|)be}\mylabel{h}  \normalsize

\doendnotes{C}
\bigskip
\vfill

\clearpage

\footnotesize

\lohead{\textsc{register}}

% Definiere theindex-Environment komplett neu ohne reledmac
\makeatletter
\renewenvironment{theindex}{%
  \section*{\indexname}%
  \setlength{\parindent}{0pt}%
  \setlength{\parskip}{0pt plus 0.3pt}%
  \let\item\@idxitem
}{%
  \clearpage
}
\makeatother

\IfFileExists{\jobname-pw.ind}{\input{\jobname-pw.ind}}{}

\end{document}

      