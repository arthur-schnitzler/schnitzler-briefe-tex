%% latex-korrekturansicht-vorspann.tex
%% Vorspann für die Korrekturansicht.
%% Lädt die gemeinsame Datei latex-vorspann.tex mit gesetztem Schalter.

\newif\ifkorrekturansicht
\korrekturansichttrue

\input{../tex-inputs/latex-vorspann}


\renewcommand{\erwaehntePersonen}{Personen: Hermann Bahr, Oskar Bie, Otto Brahm, Samuel Fischer, Gerhart Hauptmann, Eduard von Keyserling, Felix Salten, Émile Zola}
\renewcommand{\erwaehnteInstitutionen}{Institutionen: Burgtheater, Deutsches Theater Berlin, S. Fischer Verlag, Schiller-Theater}
\renewcommand{\erwaehnteOrte}{Orte: Berlin, Berliner Theater, Deutsches Theater Berlin, Hotel Bristol Berlin, Jagniątków, Wien}
\renewcommand{\erwaehnteWerke}{Werke: Beate und Mareile. Eine Schloßgeschichte, Der Schleier der Beatrice. Schauspiel in fünf Akten, Die kleine Veronika, Liebelei. Schauspiel in drei Akten, Monna Vanna. Schauspiel in drei Akten, Neue Deutsche Rundschau, Wienerinnen. Lustspiel in drei Akten}
\section[ Arthur Schnitzler an Felix Salten, 16. 10. 1902]{Arthur Schnitzler an Felix Salten, 16. 10. 1902}
\nopagebreak\mylabel{v}
\rehead{ }\normalsize\beginnumbering\briefempfaengerindex{Salten, Felix@\textsc{Salten, Felix}!zzzSchnitzler, Arthur@\emph{von Arthur Schnitzler}!1902-10-161@{16. 10. 1902}|(be}
\toendnotes[C]{\smallbreak\pagebreak[2]}\Standort{Wienbibliothek im Rathaus, ZPH 1681, 2.1.516.}
\physDesc{Brief, 1 Blatt, 4 Seiten, 1314 Zeichen
\newline{}Handschrift: Bleistift, deutsche Kurrent
\newline{}Ordnung: mit Bleistift von unbekannter Hand Nummerierung der Doppelseiten des
                                 Konvoluts: »67«–»68« }\toendnotes[C]{\smallbreak}
\pstart
           \raggedleft{}{\pb}\textsc{\textcolor{pink}{Berlin Bristol}{}\ledrightnote{\textcolor{pink}{Hotel Bristol Berlin}}}, \uline{16. X. 902.}\pend
           
\pstart
           lieber Freund,{ }\label{K_L02979-1v}\edtext{geſtern}{\lemma{\textnormal{\emph{geſtern}}}\Cendnote{\textnormal{siehe A. S.: \emph{Tagebuch}, 15. 10. 1902}}}\label{K_L02979-1h} ſprach ich \textsc{\textcolor{blue}{S. Fischer}{}\ledrightnote{\textcolor{blue}{Samuel Fischer}}}; nach einigen
               Einwendungen
               geſtand er der \label{K_L02979-2v}\edtext{\textcolor{green}{Novelle}{}\ledrightnote{{$\rightarrow$}\textcolor{green}{Die kleine Veronika}}}{\lemma{\textnormal{\emph{Novelle}}}\Cendnote{\textnormal{Wie später im Korrespondenzstück als Möglichkeit
                  thematisiert, erschien die \textcolor{green}{Novelle} noch im 
                     Dezember in der \emph{\textcolor{green}{Neuen
                        Deutschen Rundschau}}: \textcolor{blue}{Felix Salten}: \emph{\textcolor{green}{Die kleine Veronika}}. In: \emph{\textcolor{green}{Neue Deutsche Rundschau}}, Jg. 13, Nr. 12, Dezember 1902, S. 1285–1333.
                  Die Buchausgabe folgte 1903: \textcolor{blue}{ders.}: \emph{\textcolor{green}{Die kleine Veronika}}. \textcolor{pink}{Berlin}: \emph{\textcolor{brown}{S. Fischer}}{ }{[}Mitte Mai{]} 1903.}}}\label{K_L02979-2h}, beſonders im letzten Drittel, \textcolor{blue}{Zola}{}\ledrightnote{\textcolor{blue}{Émile Zola}}’ſche Kraft zu, und iſt \uline{jedenfalls ſofort
                  bereit} ſie als Buch zu drucken. Gegen die Veröffentlichung in der \textsc{\textcolor{green}{N. Dtsch Rds}{}\ledrightnote{\textcolor{green}{Neue Deutsche Rundschau}}} ſprechen \uline{vorläufig} noch einige Bedenken
               ausſchließlich techniſcher Natur. Sie nähme 60 Seiten ein, was für \uline{eine} Nu{\geminationm}er {\pb}zu viel ſei; und neben dem im Jänner beginnenden \label{K_L02979-3v}\edtext{\textcolor{green}{Roman}{}\ledrightnote{{$\rightarrow$}\textcolor{green}{Beate und Mareile. Eine Schloßgeschichte}}}{\lemma{\textnormal{\emph{Roman}}}\Cendnote{\textnormal{\emph{\textcolor{green}{Beate und Mareile. Eine Schloßgeschichte}} von
                     \textcolor{blue}{Eduard von Keyserling} erschien in drei
                  Teilen zwischen Januar und März 1903 in der \emph{\textcolor{green}{Neuen Deutschen
                     Rundschau}}.}}}\label{K_L02979-3h}{ }\label{T_L02979-1v}\edtext{könnten}{\lemma{\textnormal{\emph{könnten}}}\Cendnote{\textnormal{er schreibt »konnten«}}}\label{T_L02979-1h} ſie nicht ein
               Ding in 2 Fortſetzungen bringen. Inmit\textcolor{gray}{ten} der Discuſſion kam \textsc{\textcolor{blue}{Bie}{}\ledrightnote{\textcolor{blue}{Oskar Bie}}}, der die \textcolor{green}{Novelle}{}\ledrightnote{{$\rightarrow$}\textcolor{green}{Die kleine Veronika}} zur
               Lecture nach Hauſe nahm. Ich habe den Eindruck, wenn ſie ihm gefällt, wird man ſie im
                  \textcolor{green}{Dezemberheft}{}\ledrightnote{{$\rightarrow$}\textcolor{green}{Neue Deutsche Rundschau}}, trotz der 60 Seiten abdrucken. In
               Hinblick auf die Buchausgabe iſt natürlich {\pb}zuzugreifen. –\pend
           
\pstart
           In Hinſicht auf die \label{K_L02979-4v}\edtext{\textsc{\textcolor{green}{Bea}{}\ledrightnote{\textcolor{green}{Der Schleier der Beatrice. Schauspiel in fünf Akten}}}}{\lemma{\textnormal{\emph{Bea}}}\Cendnote{\textnormal{\textcolor{blue}{Schnitzler} hoffte weiterhin,
                  dass \emph{\textcolor{green}{Der Schleier der Beatrice}} durch eine qualitätvolle Aufführung 
                  in \textcolor{pink}{Berlin} Erfolg haben würde. Die \textcolor{pink}{Berlin}-Premiere wurde letztlich am 7. 3. 1903 vom \emph{\textcolor{brown}{Deutschen Theater}} veranstaltet.}}}\label{K_L02979-4h}{ }\substVorne{}\textsuperscript{iſt}\substDazwischen{}bin\substHinten{} ich ſoweit als früher. Vom \textcolor{brown}{Schillertheater}{}\ledrightnote{\textcolor{brown}{Schiller-Theater}} räth mir \uline{alles} ab; die
                  \label{K_L02979-5v}\edtext{Aufführg der \textsc{\label{K_L02979-6v}\edtext{\textcolor{green}{M. Vanna}{}\ledrightnote{\textcolor{green}{Monna Vanna. Schauspiel in drei Akten}}}{\lemma{\textnormal{\emph{M. Vanna}}}\Cendnote{\textnormal{vgl. Paul Goldmann an Arthur Schnitzler, 16. 6. [1902]}}}\label{K_L02979-6h}} im \textcolor{pink}{Dtſch Theater}{}\ledrightnote{\textcolor{pink}{Deutsches Theater Berlin}}}{\lemma{\textnormal{\emph{Aufführg … Theater}}}\Cendnote{\textnormal{siehe A. S.: \emph{Tagebuch}, 14. 10. 1902}}}\label{K_L02979-5h} iſt kläglich. \textcolor{blue}{Brahm}{}\ledrightnote{\textcolor{blue}{Otto Brahm}} will ſehr; da er
                  vorgeſtern abgereiſt iſt, reiſe ich von hier
               wahrſcheinlich \introOben{}(Samſtag)\introOben{} zu
               ihm \label{K_L02979-7v}\edtext{nach \textcolor{pink}{Agnetendorf}{}\ledrightnote{\textcolor{pink}{Jagniątków}}}{\lemma{\textnormal{\emph{nach Agnetendorf}}}\Cendnote{\textnormal{siehe A. S.: \emph{Tagebuch}, 19. 10. 1902}}}\label{K_L02979-7h}, wohin ich auch von \textcolor{blue}{Hauptm}{}\ledrightnote{\textcolor{blue}{Gerhart Hauptmann}} eine
                  \label{K_L02979-8v}\edtext{telegr. Einladg}{\lemma{\textnormal{\emph{telegr. Einladg}}}\Cendnote{\textnormal{nicht überliefert}}}\label{K_L02979-8h} erhalten {\pb}habe, – u bringe dort die Sache ins
               Reine.\pend
           
\pstart
           \textcolor{blue}{Bahr}{}\ledrightnote{\textcolor{blue}{Hermann Bahr}} hatte hier einen wirklichen \label{K_L02979-9v}\edtext{\textcolor{green}{Erfolg}{}\ledrightnote{{$\rightarrow$}\textcolor{green}{Wienerinnen. Lustspiel in drei Akten}}}{\lemma{\textnormal{\emph{Erfolg}}}\Cendnote{\textnormal{Am 14. 10. 1902 war \textcolor{blue}{Bahr}s \emph{\textcolor{green}{Wienerinnen}} – in Anwesenheit des \textcolor{blue}{Autor}s – am \textcolor{pink}{Berliner Theater} aufgeführt worden.}}}\label{K_L02979-9h}. –\pend
           
\pstart
           In Hinſicht auf die \label{K_L02979-10v}\edtext{Kündigungspflicht
               beim \textcolor{brown}{Burgtheater}{}\ledrightnote{\textcolor{brown}{Burgtheater}}}{\lemma{\textnormal{\emph{Kündigungspflicht beim Burgtheater}}}\Cendnote{\textnormal{siehe Felix Salten an Arthur Schnitzler, 15. 10. 1902}}}\label{K_L02979-10h} ſti{\geminationm}t’s. Ich muſs am 9.
                  Nov. dem \textcolor{brown}{Theater}{}\ledrightnote{{$\rightarrow$}\textcolor{brown}{Burgtheater}} das
               ausſchließliche Aufführgsrecht der \textcolor{green}{Liebelei}{}\ledrightnote{\textcolor{green}{Liebelei. Schauspiel in drei Akten}} kündigen mit 2 monatlicher Friſt. Näheres
               mündlich. –\pend
           \pstart Herzlichſt Ihr \spacefill\mbox{A. S.}\pend{}\endnumbering\briefempfaengerindex{Salten, Felix@\textsc{Salten, Felix}!zzzSchnitzler, Arthur@\emph{von Arthur Schnitzler}!1902-10-161@{16. 10. 1902}|)be}\mylabel{h}  \normalsize

\doendnotes{C}
\bigskip
\vfill

\clearpage

\footnotesize

\lohead{\textsc{register}}

% Definiere theindex-Environment komplett neu ohne reledmac
\makeatletter
\renewenvironment{theindex}{%
  \section*{\indexname}%
  \setlength{\parindent}{0pt}%
  \setlength{\parskip}{0pt plus 0.3pt}%
  \let\item\@idxitem
}{%
  \clearpage
}
\makeatother

\IfFileExists{\jobname-pw.ind}{\input{\jobname-pw.ind}}{}

\end{document}

      