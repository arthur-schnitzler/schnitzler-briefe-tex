%% latex-korrekturansicht-vorspann.tex
%% Vorspann für die Korrekturansicht.
%% Lädt die gemeinsame Datei latex-vorspann.tex mit gesetztem Schalter.

\newif\ifkorrekturansicht
\korrekturansichttrue

\input{../tex-inputs/latex-vorspann}


               \section[Hugo von Hofmannsthal an Arthur Schnitzler, {[}8. 11. 1892{]}]{ Hugo von Hofmannsthal an Arthur Schnitzler, {[}8. 11. 1892{]}}\nopagebreak\mylabel{v}\rehead{ }\normalsize\beginnumbering\briefempfaengerindex{Schnitzler, Arthur@\textsc{Schnitzler, Arthur}!zzzHofmannsthal, Hugo von@\emph{von Hugo von Hofmannsthal}!1892-11-081@{{[}8. 11. 1892{]}}|(be} \toendnotes[C]{\smallbreak\pagebreak[2]} \Standort{CUL, Schnitzler, B 43.}
\physDesc{Brief, 1 Blatt (Briefpapier mit aufgeprägtem Wappen), 2 Seiten
\newline{}Handschrift: schwarze Tinte, deutsche Kurrent
\newline{}Schnitzler: mit Bleistift nummeriert: »33« und datiert: »Nov. 92« }\buchAbdrucke{\weitereDrucke{Hugo von Hofmannsthal, Arthur Schnitzler: \emph{Briefwechsel}. Hg. Therese Nickl und Heinrich Schnitzler. Frankfurt am Main: \emph{S. Fischer} 1964, S. 30.} }\toendnotes[C]{\smallbreak}\pstart
           \raggedleft{}{\pb}Dienstag.\pend
           \pstart\center{}lieber Doctor.\pend\pstart
           Ich kann leider einer Familienverpflichtung wegen abſolut nicht zu \textcolor{pink}{\textsc{Pfob}}{}\ledrightnote{\textcolor{pink}{Café Pfob}} kommen. \label{K_L00132_1v}\edtext{Samſtag}{\lemma{\textnormal{\emph{Samſtag}}}\Cendnote{\textnormal{Erstaufführung im \textcolor{pink}{Deutschen Volkstheater} am
                        12. 11. 1892}}}\label{K_L00132_1h} gehe ich in »\textcolor{green}{\textsc{Musotte}}{}\ledrightnote{\textcolor{green}{Musotte}}«; könnten wir nicht miteinander ſoupieren? bitte gelegentlich Antwort.
                    Falls \textcolor{blue}{\textsc{Robert Ehrhart}}{}\ledrightnote{\textcolor{blue}{Robert Ehrhart von Ehrhartstein}} da iſt, ſo ſagen Sie ihm, bitte, daß ich ſeinen leider wieder verfehlten
                    Beſuch {\pb}wenn er mir nicht
                    abſchreibt, Donnerstag zwiſchen 10 u 11
                    erwidern werde, um über die \textcolor{green}{Novelle}{}\ledrightnote{→\textcolor{green}{Die kleine Lydia}} zu reden. Ich finde ſie ſehr gut gemacht und
                    wenn auch ein bißchen \label{K_L00132_2v}\edtext{\textsc{vieux jeu}}{\lemma{\textnormal{\emph{vieux jeu}}}\Cendnote{\textnormal{französisch: altes Spiel}}}\label{K_L00132_2h}, doch im
                    ganzen fertig u. verwendbar.\pend
           \pstart
           Grüße alle herzlichſt\pend
           \pstart \spacefill\mbox{Loris.}\pend{}\endnumbering\briefempfaengerindex{Schnitzler, Arthur@\textsc{Schnitzler, Arthur}!zzzHofmannsthal, Hugo von@\emph{von Hugo von Hofmannsthal}!1892-11-081@{{[}8. 11. 1892{]}}|)be}\mylabel{h}  \normalsize

\doendnotes{C}
\bigskip
\vfill

\clearpage

\footnotesize

\lohead{\textsc{register}}

% Definiere theindex-Environment komplett neu ohne reledmac
\makeatletter
\renewenvironment{theindex}{%
  \section*{\indexname}%
  \setlength{\parindent}{0pt}%
  \setlength{\parskip}{0pt plus 0.3pt}%
  \let\item\@idxitem
}{%
  \clearpage
}
\makeatother

\IfFileExists{\jobname-pw.ind}{\input{\jobname-pw.ind}}{}

\end{document}

      