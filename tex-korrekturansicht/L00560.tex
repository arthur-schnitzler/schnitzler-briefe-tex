%% latex-korrekturansicht-vorspann.tex
%% Vorspann für die Korrekturansicht.
%% Lädt die gemeinsame Datei latex-vorspann.tex mit gesetztem Schalter.

\newif\ifkorrekturansicht
\korrekturansichttrue

\input{../tex-inputs/latex-vorspann}


               \section[Arthur Schnitzler an Richard Beer-Hofmann, 7. 7. 1896]{ Arthur Schnitzler an Richard Beer-Hofmann, 7. 7. 1896}\nopagebreak\mylabel{v}\rehead{ }\normalsize\beginnumbering\briefempfaengerindex{Beer-Hofmann, Richard@\textsc{Beer-Hofmann, Richard}!zzzSchnitzler, Arthur@\emph{von Arthur Schnitzler}!1896-07-071@{7. 7. 1896}|(be} \toendnotes[C]{\smallbreak\pagebreak[2]} \Standort{YCGL, MSS 31.}
\physDesc{Postkarte
\newline{}Handschrift: Bleistift, deutsche Kurrent\newline{}Versand: 1) Stempel: »\nobreak{}\oindex{Hotel zum Kronprinzen@\textbf{Hotel zum Kronprinzen}, \emph{Hotel (K.HTL)}|pwk}Hotel zum Kronprinzen Hamburg.\nobreak{}«.  2) Stempel: »\nobreak{}\oindex{Luebeck@\textbf{Lübeck}, \emph{Besiedelter Ort (A.BSO)}|pwk}Lübeck, 7. 7. 96, 6–7N\nobreak{}«. 3) Stempel: »\nobreak{}\oindex{St. Gilgen@\textbf{St. Gilgen}, \emph{Besiedelter Ort (A.BSO)}|pwk}St. Gilgen, 9. 7. 96\nobreak{}«. }\buchAbdrucke{\weitereDrucke{Arthur Schnitzler, Richard Beer-Hofmann: \emph{Briefwechsel 1891–1931}. Hg. Konstanze Fliedl. Wien, Zürich: \emph{Europaverlag} 1992, S. 92.} }\toendnotes[C]{\smallbreak}\pstart{}{\pb}Herrn \textsc{Dr. Richard
                     Beer-Hofmann}\pend{}\pstart{}\textcolor{pink}{\textsc{Fürberg am St. Wolfgangsee}}{}\ledrightnote{\textcolor{pink}{Fürberg}}\pend{}\pstart{}\textsc{in \textcolor{pink}{Oberoesterreich}{}\ledrightnote{\textcolor{pink}{Oberösterreich}}}\pend{}{\bigskip}\pstart
           \noindent{}{\pb}Lieber Richard, leider muſs ich \textcolor{pink}{Mitteleuropa}{}\ledrightnote{\textcolor{pink}{Europa}} verlaſſen, ohne weitere Nachricht von Ihnen gefunden zu haben.
               Ich war 3 Tage in \textcolor{pink}{\textsc{Hamburg}}{}\ledrightnote{\textcolor{pink}{Hamburg}} u ſchreibe dieſe Karte in \textcolor{pink}{\textsc{Lübeck}}{}\ledrightnote{\textcolor{pink}{Lübeck}}, wohin ich mich ein Ausflug führte. Ich bin guter, aber nicht hoher Sti{\geminationm}ung. Heute Abend geh ich »an Bord« der \textsc{Sverre Sigurdson}. Iſt’s nicht ein trauriges Leben, darin man
               nicht einmal mehr »an Bord« ohne Anführungszeichen ſchreiben kann? – Ich hoffe
               in \textcolor{pink}{\textsc{Trondjhem}}{}\ledrightnote{\textcolor{pink}{Trondheim}} Briefe von Ihnen zu finden. Grüße Sie herzlich; grüßen Sie auch \textcolor{blue}{Paula}{}\ledrightnote{\textcolor{blue}{Paula Beer-Hofmann}}\pend
           \pstart \label{T_L00560_1v}\edtext{Ihr}{\lemma{\textnormal{\emph{Ihr}}}\Cendnote{\textnormal{am
                  oberen Rand auf dem Kopf}}}\label{T_L00560_1h}\spacefill\mbox{Arthur}\pend{}\endnumbering\briefempfaengerindex{Beer-Hofmann, Richard@\textsc{Beer-Hofmann, Richard}!zzzSchnitzler, Arthur@\emph{von Arthur Schnitzler}!1896-07-071@{7. 7. 1896}|)be}\mylabel{h}  \normalsize

\doendnotes{C}
\bigskip
\vfill

\clearpage

\footnotesize

\lohead{\textsc{register}}

% Definiere theindex-Environment komplett neu ohne reledmac
\makeatletter
\renewenvironment{theindex}{%
  \section*{\indexname}%
  \setlength{\parindent}{0pt}%
  \setlength{\parskip}{0pt plus 0.3pt}%
  \let\item\@idxitem
}{%
  \clearpage
}
\makeatother

\IfFileExists{\jobname-pw.ind}{\input{\jobname-pw.ind}}{}

\end{document}

      