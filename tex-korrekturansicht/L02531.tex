%% latex-korrekturansicht-vorspann.tex
%% Vorspann für die Korrekturansicht.
%% Lädt die gemeinsame Datei latex-vorspann.tex mit gesetztem Schalter.

\newif\ifkorrekturansicht
\korrekturansichttrue

\input{../tex-inputs/latex-vorspann}


               \section[Arthur Schnitzler an Hermann Bahr, 16. 2. 1930]{ Arthur Schnitzler an Hermann Bahr, 16. 2. 1930}\nopagebreak\mylabel{v}\rehead{ }\normalsize\beginnumbering\briefempfaengerindex{Bahr, Hermann@\textsc{Bahr, Hermann}!zzzSchnitzler, Arthur@\emph{von Arthur Schnitzler}!1930-02-161@{16. 2. 1930}|(be} \toendnotes[C]{\smallbreak\pagebreak[2]} \Standort{TMW, HS AM 23398 Ba.}
\physDesc{Brief, 1 Blatt, 2 Seiten
\newline{}Handschrift: schwarze Tinte, lateinische Kurrent
\newline{}Bahr: mit rotem Buntstift beschriftet: »\uline{Schnitzler}« }\buchAbdrucke{\weitereDrucke{1) \emph{16. 2. 1930.} In: Arthur Schnitzler: \emph{The Letters of Arthur Schnitzler to Hermann Bahr}. Edited, annotated, and with an introduction, by Donald G.
                        Daviau. Chapel Hill: \emph{The University of North Carolina Press} 1978, S. 116–117 (University of North Carolina studies in the Germanic languages
                        and literatures, 89).} \weitereDrucke{2) Hermann Bahr, Arthur Schnitzler: \emph{Briefwechsel, Aufzeichnungen, Dokumente (1891–1931)}. Hg. Kurt Ifkovits und Martin Anton Müller. Göttingen: \emph{Wallstein} 2018, S. 594.} }\toendnotes[C]{\smallbreak}\pstart
           \raggedleft{}{\pb}\textcolor{pink}{Wien}{}\ledrightnote{\textcolor{pink}{Wien}}{ }16. \textcolor{gray}{2}. 930\pend
           \pstart
           Mein lieber Hermann, nach so langer Zeit \textcolor{green}{hör}{}\ledrightnote{→\textcolor{green}{Tagebuch. 10. Januar [1930]}} ich wieder was von dir – und da verleihst
               du mir gleich den \textcolor{brown}{Nobelpreis}{}\ledrightnote{\textcolor{brown}{Bauernfeld-Preis}}!\pend
           \pstart
           Ich fühle ganz wie du: daſs \textcolor{blue}{Hugo}{}\ledrightnote{\textcolor{blue}{Hugo von Hofmannsthal}} derjenige
               gewesen ist, der ihn hätte beko{\geminationm}en müssen. Leider könnte
               ich diesmal nicht wieder aussprechen – wie seinerzeit als ich (noch dazu für das \textcolor{green}{Zwischenspiel}{}\ledrightnote{\textcolor{green}{Zwischenspiel. Komödie in drei Akten}}!) den \label{K_L02531_1v}\edtext{\textcolor{brown}{Grillparzerpreis}{}\ledrightnote{\textcolor{brown}{Franz-Grillparzer-Preis}} erhielt}{\lemma{\textnormal{\emph{Grillparzerpreis erhielt}}}\Cendnote{\textnormal{vgl. A. S.: \emph{Tagebuch}, 15. 1. 1908}}}\label{K_L02531_1h}, daſs der eigentlich \textcolor{blue}{Hofmannsthal}{}\ledrightnote{\textcolor{blue}{Hugo von Hofmannsthal}}
               gebühre. Auch damit hast du recht: »\uline{melden} werd ich
                  \introOben{}mich\introOben{} nicht, vielleicht weniger aus
                  »Bescheidenheit{[}«{]}, als aus Bequemlichkeit und einer immer
               wachsenden {\pb}Gleichgiltigkeit gegen alle Arten von äußeren »Ehrungen« u was man so nennt.\pend
           \pstart
           D\substVorne{}\textsuperscript{as}\substDazwischen{}ein\substHinten{} »\textcolor{green}{Tagebuch}{}\ledrightnote{\textcolor{green}{Tagebuch [Kolumne im Neuen Wiener Journal]}}{[}«{]}{ }\label{LL140-1v}les ich natürlich immer\label{LL140-1h} – so bedürfte
               es also kaum einer \label{K_L02531_2v}\edtext{freundlichen
               persönlichen Bemerkung}{\lemma{\textnormal{\emph{freundlichen … Bemerkung}}}\Cendnote{\textnormal{\textcolor{blue}{Bahr} ließ regelmäßig seine Kolumne den darin
                  behandelten Personen zukommen. Diese Textstelle deutet an, dass das \emph{\textcolor{green}{Tagebuch. 10. Januar}} in einer Fassung mit einem
                  handschriftlichen Gruß im Besitz \textcolor{blue}{Schnitzlers}
                  gewesen sein dürfte. In seinen Zeitungsausschnitten (heute in \textcolor{pink}{Exeter}) findet sich aber nur ein Abzug, auf dem sich außer
                  einem Datumsvermerk von \textcolor{blue}{Schnitzler} keine
                  Beschriftung findet (University of Exeter, \emph{Schnitzler Press
                        Cuttings Archive}, Box 42/2).}}}\label{K_L02531_2h}, – und umso mehr dank
               ich dir. Ich weiſs nicht, ob du meine kleinen Bücher »\textcolor{green}{Geist im Wort und in der That}{}\ledrightnote{\textcolor{green}{Der Geist im Wort und der Geist in der Tat}}«, u mein \textcolor{green}{Buch der
                  Sprüche u Bedenken}{}\ledrightnote{\textcolor{green}{Buch der Sprüche und Bedenken}} erhalten hast – ich würde sie dir gern schicken, auf die
               Gefahr hin, dass du mit vielem nicht einverstanden sein wirst.\pend
           \pstart
           Es wär schön wenn man einander wieder sähe {\dotstwo} »\label{K_L02531_3v}\edtext{Einer von uns wird es einmal bedauern}{\lemma{\textnormal{\emph{Einer … bedauern}}}\Cendnote{\textnormal{siehe Arthur Schnitzler an Hermann Bahr, 22. 6. 1909, Hermann Bahr an Arthur Schnitzler, 28. 6. 1909,
                     Hugo von Hofmannsthal an Olga Schnitzler, 5. 7. [1912]}}}\label{K_L02531_3h}{ }{\dotstwo}« wie \textcolor{blue}{Hugo}{}\ledrightnote{\textcolor{blue}{Hugo von Hofmannsthal}} immer
               sagte. – \pend
           \pstart
           Ich grüße dich herzlich in alter Freundschaft{\\[\baselineskip]}Dein \spacefill\mbox{Arthur}\pend
           \leftskip=0em{}\endnumbering\briefempfaengerindex{Bahr, Hermann@\textsc{Bahr, Hermann}!zzzSchnitzler, Arthur@\emph{von Arthur Schnitzler}!1930-02-161@{16. 2. 1930}|)be}\mylabel{h}  \normalsize

\doendnotes{C}
\bigskip
\vfill

\clearpage

\footnotesize

\lohead{\textsc{register}}

% Definiere theindex-Environment komplett neu ohne reledmac
\makeatletter
\renewenvironment{theindex}{%
  \section*{\indexname}%
  \setlength{\parindent}{0pt}%
  \setlength{\parskip}{0pt plus 0.3pt}%
  \let\item\@idxitem
}{%
  \clearpage
}
\makeatother

\IfFileExists{\jobname-pw.ind}{\input{\jobname-pw.ind}}{}

\end{document}

      