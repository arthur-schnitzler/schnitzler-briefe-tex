%% latex-korrekturansicht-vorspann.tex
%% Vorspann für die Korrekturansicht.
%% Lädt die gemeinsame Datei latex-vorspann.tex mit gesetztem Schalter.

\newif\ifkorrekturansicht
\korrekturansichttrue

\input{../tex-inputs/latex-vorspann}


\renewcommand{\erwaehntePersonen}{Personen: Jules Moinaux, Jacques Offenbach, Olga Schnitzler, Paul Siraudin}
\renewcommand{\erwaehnteOrte}{Orte: Edmund-Weiß-Gasse, Theater an der Wien, Wien, XVIII., Währing}
\renewcommand{\erwaehnteWerke}{Werke: Tagebuch, Venedig in Paris. Operette in drei Akten}
\section[ Paul Goldmann an Arthur Schnitzler, 17. 9. {[}1903?{]}]{Paul Goldmann an Arthur Schnitzler, 17. 9. {[}1903?{]}}
\nopagebreak\mylabel{v}
\rehead{ }\normalsize\beginnumbering\briefempfaengerindex{Schnitzler, Arthur@\textsc{Schnitzler, Arthur}!zzzGoldmann, Paul@\emph{von Paul Goldmann}!1903-09-171@{17. {[}9. 1903?{]}}|(be}
\toendnotes[C]{\smallbreak\pagebreak[2]}\Standort{DLA, A:Schnitzler, HS.NZ85.1.3173.}
\physDesc{Postkarte
\newline{}Handschrift: 1) schwarze Tinte, deutsche Kurrent\hspace{1em}2) schwarze Tinte, lateinische Kurrent (\noindent{}Adresse)\hspace{1em}
\newline{}Versand: Stempel: »\nobreak{}Wien 15, 17. {[}9. 1903{]}, 2\textsuperscript{10}N\nobreak{}«. Stempel: »\nobreak{}\oindex{XVIII., Waehring@\textbf{XVIII., Währing}, \emph{Bezirk (A.BZK)}|pwk}Wien 18 111, 17. {[}9. 1903{]}, 3\textsuperscript{10}N\nobreak{}«.  
\newline{}Schnitzler: mit Bleistift durch fehlerhafte Entzifferung des Stempels falsch
                                 datiert: »3/10 {[}1{]}903.« }\toendnotes[C]{\smallbreak}\pstart{}{\pb}Herrn\pend{}\pstart{}Dr. Arthur Schnitzler\pend{}\pstart{}\textcolor{pink}{XVIII. Spöttelgaſse 7}{}\ledrightnote{\textcolor{pink}{Edmund-Weiß-Gasse}}\pend{}
{\bigskip}
\pstart
           \raggedleft{}{\pb}Donnerſtag\pend
           
\pstart{}Mein lieber Freund,\pend
\pstart
           Ich habe für heut{ }Abend eine \strikeout{\textsc{Log\textcolor{gray}{e}}}{ }\textsc{Loge} im »\label{K_L03387-1v}\edtext{\textcolor{pink}{Theater an der Wien}{}\ledrightnote{\textcolor{pink}{Theater an der Wien}}}{\lemma{\textnormal{\emph{Theater an der Wien}}}\Cendnote{\textnormal{Am 17. 9. 1903 wurde im \textcolor{pink}{Theater an der Wien} die Operette \emph{\textcolor{green}{Venedig in Paris}} (Musik \textcolor{blue}{Jacques Offenbach}, Libretto \textcolor{blue}{Paul
                     Siraudin} und \textcolor{blue}{Jules Moinaux})
                  gegeben. \textcolor{blue}{Schnitzler}s \emph{\textcolor{green}{Tagebuch}} enthält für diesen Tag keinen Eintrag, auch die
                  Aufstellung der Theaterbesuche (\emph{CUL}, A 179a) erwähnt die Aufführung nicht, so
                  dass es eher unwahrscheinlich ist, dass \textcolor{blue}{Schnitzler} der Einladung Folge leistete.}}}\label{K_L03387-1h}« (Balkonloge I. \textsc{Gallerie} links \textsc{N\textsuperscript{o}} 2). Ich bitte Dich und Deine \textcolor{blue}{Frau}{}\ledrightnote{{$\rightarrow$}\textcolor{blue}{Olga Schnitzler}}, auch hinzukommen, – umſomehr, als dies für mich vielleicht die einzige
               Möglichkeit iſt, Dich jetzt \label{K_L03387-2v}\edtext{noch
                  einmal}{\lemma{\textnormal{\emph{noch
                  einmal}}}\Cendnote{\textnormal{siehe Paul Goldmann an Arthur Schnitzler, 7. 9. 1903}}}\label{K_L03387-2h} zu ſehen. Herzlichſt Dein \spacefill\mbox{Paul Goldmann}\pend
           \endnumbering\briefempfaengerindex{Schnitzler, Arthur@\textsc{Schnitzler, Arthur}!zzzGoldmann, Paul@\emph{von Paul Goldmann}!1903-09-171@{17. {[}9. 1903?{]}}|)be}\mylabel{h}
\begin{anhang}
\end{anhang}\normalsize

\doendnotes{C}
\bigskip
\vfill

\clearpage

\footnotesize

\lohead{\textsc{register}}

% Definiere theindex-Environment komplett neu ohne reledmac
\makeatletter
\renewenvironment{theindex}{%
  \section*{\indexname}%
  \setlength{\parindent}{0pt}%
  \setlength{\parskip}{0pt plus 0.3pt}%
  \let\item\@idxitem
}{%
  \clearpage
}
\makeatother

\IfFileExists{\jobname-pw.ind}{\input{\jobname-pw.ind}}{}

\end{document}

      