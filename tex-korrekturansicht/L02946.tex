%% latex-korrekturansicht-vorspann.tex
%% Vorspann für die Korrekturansicht.
%% Lädt die gemeinsame Datei latex-vorspann.tex mit gesetztem Schalter.

\newif\ifkorrekturansicht
\korrekturansichttrue

\input{../tex-inputs/latex-vorspann}


         
         \renewcommand{\erwaehntePersonen}{Personen: Gisela Hajek, Fedor Mamroth, Josef Rosengart, Louise Schnitzler, Julius Schnitzler, Helene Schnitzler}
         \renewcommand{\erwaehnteInstitutionen}{Institutionen: Volkstheater}
         \renewcommand{\erwaehnteOrte}{Orte: Frankfurt am Main, Reuterweg, Wien}
         \renewcommand{\erwaehnteWerke}{Werke: Der Schleier der Beatrice. Schauspiel in fünf Akten, Der blinde Geronimo und sein Bruder, Frankfurter Zeitung, Lieutenant Gustl. Novelle, Neue Freie Presse}
               \section[ Paul Goldmann an Arthur Schnitzler, 27. 12. {[}1900{]}]{Paul Goldmann an Arthur Schnitzler, 27. 12. {[}1900{]}}\nopagebreak\mylabel{v}\rehead{ }\normalsize\beginnumbering\briefempfaengerindex{Schnitzler, Arthur@\textsc{Schnitzler, Arthur}!zzzGoldmann, Paul@\emph{von Paul Goldmann}!1900-12-271@{27. 12. {[}1900{]}}|(be} \toendnotes[C]{\smallbreak\pagebreak[2]} \Standort{DLA, A:Schnitzler, HS.NZ85.1.3170.}
\physDesc{Brief, 1 Blatt, 3 Seiten
\newline{}Handschrift: blaue Tinte, deutsche Kurrent
\newline{}Schnitzler: mit Bleistift das Jahr »{[}1{]}900« vermerkt }\toendnotes[C]{\smallbreak}\pstart
           \noindent{}{\pb}\textcolor{pink}{Frankfurt}{}\ledrightnote{\textcolor{pink}{Frankfurt am Main}}{ }27. December.\hfill \textcolor{gray}{\textbf{\textcolor{pink}{Reuterweg 59}{}\ledrightnote{\textcolor{pink}{Reuterweg}}.}}\pend
           \pstart
           \centering{}Mein lieber Freund,\pend
           \pstart
           \noindent{}Ich hoffe, Du haſt frohe Weihnachten gehabt und ich wünſche Dir ein glückliches neues
                  Jahr.\pend
           \pstart
           Ich bin dieſe Woche in \textcolor{pink}{Frankfurt}{}\ledrightnote{\textcolor{pink}{Frankfurt am Main}}, ruhe mich ein
               wenig aus und laſſe es mir gut gehen.\pend
           \pstart
           Alle die Meinigen grüßen Dich. Mein \textcolor{blue}{Onkel}{}\ledrightnote{{$\rightarrow$}\textcolor{blue}{Fedor Mamroth}} hätte gern den »\textcolor{green}{blinden \textsc{Hironymo}}{}\ledrightnote{\textcolor{green}{Der blinde Geronimo und sein Bruder}}«
                für die \textcolor{green}{Frankfurter Zeitung}{}\ledrightnote{\textcolor{green}{Frankfurter Zeitung}} gehabt und
               läßt Dich bitten, wenn Du wieder einmal eine kurze Novelle fertig haſt, ſie ihm zu
               ſchicken.\pend
           \pstart
           Die Weihnachtsnummer der \textcolor{green}{N. Fr. Pr.}{}\ledrightnote{\textcolor{green}{Neue Freie Presse}} iſt mir
               nicht zu Geſicht {\pb}gekommen, und ich habe den \label{K_L02946-2v}\edtext{»\textcolor{green}{Lieutnant
                  Guſtl}{}\ledrightnote{\textcolor{green}{Lieutenant Gustl. Novelle}}«}{\lemma{\textnormal{\emph{»Lieutnant
                  Guſtl«}}}\Cendnote{\textnormal{\textcolor{blue}{Arthur Schnitzler}: \emph{\textcolor{green}{Lieutenant Gustl}}. In: \emph{\textcolor{green}{Neue Freie Presse}}, Nr. 13053, 25. 12. 1900, S. 34–41.}}}\label{K_L02946-2h} daher noch nicht geleſen.\pend
           \pstart
           Gibſt Du die »\textsc{\textcolor{green}{Beatrice}{}\ledrightnote{\textcolor{green}{Der Schleier der Beatrice. Schauspiel in fünf Akten}}}« dem \label{K_L02946-3v}\edtext{»\textcolor{brown}{Volkstheater}{}\ledrightnote{\textcolor{brown}{Volkstheater}}«}{\lemma{\textnormal{\emph{»Volkstheater«}}}\Cendnote{\textnormal{siehe Paul Goldmann an Arthur Schnitzler, 21. 6. [1900]}}}\label{K_L02946-3h}? Du ſollteſt es entſchieden thun. Auch mein \textcolor{blue}{Onkel}{}\ledrightnote{{$\rightarrow$}\textcolor{blue}{Fedor Mamroth}} iſt der Anſicht.\pend
           \pstart
           Meine Feuilletons ſammeln? Nie im Leben finde ich einen Verleger! Man weiſt mich mit
               Hohnlachen zurück, wenn ich mit ſo etwas komme.\pend
           \pstart
           Sei ſo gut und ſchreib mir ein Wort hierher an die obige Adreſſe meines Schwagers \textsc{Dr. \textcolor{blue}{Rosengart}{}\ledrightnote{\textcolor{blue}{Josef Rosengart}}}.\pend
           \pstart
           Bitte auch Deiner Frau \textcolor{blue}{Mutter}{}\ledrightnote{{$\rightarrow$}\textcolor{blue}{Louise Schnitzler}}, Deinem \textcolor{blue}{Bruder}{}\ledrightnote{{$\rightarrow$}\textcolor{blue}{Julius Schnitzler}} und Deiner \textcolor{blue}{Schwägerin}{}\ledrightnote{{$\rightarrow$}\textcolor{blue}{Helene Schnitzler}}, {\pb}Deiner \textcolor{blue}{Schweſter}{}\ledrightnote{{$\rightarrow$}\textcolor{blue}{Gisela Hajek}} und Deinem \textcolor{blue}{Schwager}{}\ledrightnote{{$\rightarrow$}\textcolor{blue}{Gisela Hajek}} meine herzlichſten
               Neujahrs-Glückwünſche zu übermitteln.\pend
           \pstart
           Viele treue Grüße! {\\[\baselineskip]}Dein {\\[\baselineskip]}\spacefill\mbox{Paul Goldmann.}\pend
           \leftskip=0em{}\endnumbering\briefempfaengerindex{Schnitzler, Arthur@\textsc{Schnitzler, Arthur}!zzzGoldmann, Paul@\emph{von Paul Goldmann}!1900-12-271@{27. 12. {[}1900{]}}|)be}\mylabel{h}  \normalsize

\doendnotes{C}
\bigskip
\vfill

\clearpage

\footnotesize

\lohead{\textsc{register}}

% Definiere theindex-Environment komplett neu ohne reledmac
\makeatletter
\renewenvironment{theindex}{%
  \section*{\indexname}%
  \setlength{\parindent}{0pt}%
  \setlength{\parskip}{0pt plus 0.3pt}%
  \let\item\@idxitem
}{%
  \clearpage
}
\makeatother

\IfFileExists{\jobname-pw.ind}{\input{\jobname-pw.ind}}{}

\end{document}

      