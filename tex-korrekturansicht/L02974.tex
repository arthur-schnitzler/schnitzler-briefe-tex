%% latex-korrekturansicht-vorspann.tex
%% Vorspann für die Korrekturansicht.
%% Lädt die gemeinsame Datei latex-vorspann.tex mit gesetztem Schalter.

\newif\ifkorrekturansicht
\korrekturansichttrue

\input{../tex-inputs/latex-vorspann}


\renewcommand{\erwaehntePersonen}{Personen: Richard Beer-Hofmann, Moriz Benedikt, Otto Brahm, Heinrich Conrad, Paul Goldmann, Hugo von Hofmannsthal, Josef Kainz, Heinrich Kanner, Henri de Rochefort, Felix Salten, Paul Schlenther, Olga Schnitzler, Gustav Schwarzkopf, Christiane Zimmer}
\renewcommand{\erwaehnteInstitutionen}{Institutionen: Burgtheater, Deutsches Theater Berlin, Die Zeit, Die Zeit. Wiener Wochenschrift, Neue Freie Presse, Robert Lutz, Théâtre Antoine, Wiener Allgemeine Zeitung}
\renewcommand{\erwaehnteOrte}{Orte: Botan. Garten, Carl-Theater, Florenz, Medici-Kapelle, San Domenico, Stuttgart, Wien}
\renewcommand{\erwaehnteWerke}{Werke: Abenteuer meines Lebens, Au Perroquet Vert, Der Hund von Florenz, Der Schleier der Beatrice. Schauspiel in fünf Akten, Der Weg zum Licht. Ein Salzburger Märchendrama in vier Akten, Der einsame Weg. Schauspiel in fünf Akten, Der grüne Kakadu. Groteske in einem Akt, Die Zeit, Die Zeit. Wiener Wochenschrift, Die kleine Veronika, Dämmerseele, Les Aventures de ma vie, Neue Freie Presse}
\section[ Arthur Schnitzler an Felix Salten, 27. 5. 1902]{Arthur Schnitzler an Felix Salten, 27. 5. 1902}
\nopagebreak\mylabel{v}
\rehead{ }\normalsize\beginnumbering\briefempfaengerindex{Salten, Felix@\textsc{Salten, Felix}!zzzSchnitzler, Arthur@\emph{von Arthur Schnitzler}!1902-05-271@{27. 5. 1902}|(be}
\toendnotes[C]{\smallbreak\pagebreak[2]}\Standort{Wienbibliothek im Rathaus, ZPH 1681, 2.1.516.}
\physDesc{Brief, 2 Blätter, 8 Seiten, 2579 Zeichen
\newline{}Handschrift: Bleistift, deutsche Kurrent
\newline{}Ordnung: mit Bleistift von unbekannter Hand Nummerierung der Doppelseiten des
                                 Konvoluts: »62«–»65« }\toendnotes[C]{\smallbreak}
\pstart
           \raggedleft{}{\pb}27. 5. 902\pend
           
\pstart
           lieber, ich freue mich ſehr über den guten Eindruck, den Sie von der
                  \label{K_L02974-1v}\edtext{\textcolor{green}{Novellette}{}\ledrightnote{{$\rightarrow$}\textcolor{green}{Dämmerseele}}}{\lemma{\textnormal{\emph{Novellette}}}\Cendnote{\textnormal{siehe Felix Salten an Arthur Schnitzler, 22. 5. 1902}}}\label{K_L02974-1h} in d. \textcolor{green}{N. Fr. Pr.}{}\ledrightnote{\textcolor{green}{Neue Freie Presse}} haben; was mir
               eigentlich ſelten paſſiert, – ich war ſelbſt ein bischen unſicher im Urtheil. Daſs
               ſie \textcolor{blue}{Schwarzk.}{}\ledrightnote{\textcolor{blue}{Gustav Schwarzkopf}} nicht mag, iſt ziemlich
               verſtändlich; – der \label{K_L02974-2v}\edtext{Einwurf \textcolor{blue}{Goldm.}{}\ledrightnote{\textcolor{blue}{Paul Goldmann}}: es handle ſich um Liebe}{\lemma{\textnormal{\emph{Einwurf … Liebe}}}\Cendnote{\textnormal{siehe A. S.: \emph{Tagebuch}, 21. 5. 1902}}}\label{K_L02974-2h}, kaum discutirbar; \textcolor{blue}{Richard}{}\ledrightnote{\textcolor{blue}{Richard Beer-Hofmann}} u \textcolor{blue}{Hugo}{}\ledrightnote{\textcolor{blue}{Hugo von Hofmannsthal}} ſcheinen ſie im ganzen gut zu finden, aber
                  {\pb}wie mir ſchien, mit einigem innern
               Widerſtand. \textcolor{blue}{Olga}{}\ledrightnote{\textcolor{blue}{Olga Schnitzler}} gefiel ſie, als ich ſie ihr
               vorlas, beſonders gut; – die gedruckte hat ſie aber enttäuscht. Meine Bedenken gehen
               nach der Seite des mä\textcolor{gray}{{\geminationn}}lichen {\dotstwo} ich f\textcolor{gray}{i}nde eben kein
               andres Wort – Helden{\dots}, wo mir was zu fehlen ſcheint. Der
               Titel ko{\geminationm}t mir, ſelbſt nach jedem Überdenken Ihrer
               Einwände, nicht un{\pb}glücklich vor. Daſs Sie
               als der erſte den Schluſs nicht als Pointe empfinden, ſondern wohl im Gegentheil
               gerade als den Ausklang ins ungewiſſe, ferne, mit Notwendgkeit weiterflutend,
                  be\textcolor{gray}{rü}hrt mich beſonders angenehm. –\pend
           
\pstart
           \textcolor{blue}{Paul G.}{}\ledrightnote{\textcolor{blue}{Paul Goldmann}} ist wieder \label{K_L02974-3v}\edtext{fort}{\lemma{\textnormal{\emph{fort}}}\Cendnote{\textnormal{\textcolor{blue}{Paul Goldmann} war über Pfingsten in \textcolor{pink}{Wien} gewesen.}}}\label{K_L02974-3h}; die \label{K_L02974-4v}\edtext{Martin Finder Sachen}{\lemma{\textnormal{\emph{Martin Finder Sachen}}}\Cendnote{\textnormal{Da \textcolor{blue}{Salten} bis zum 30. 6. 1902 bei der \emph{\textcolor{brown}{Wiener Allgemeinen Zeitung}} unter Vertrag stand, veröffentlichte er seine
                  Beiträge für die Wochenschrift \emph{\textcolor{green}{Die Zeit}} bis
                  dahin unter dem Pseudonym »Martin Finder«, in das nur wenige Personen eingeweiht
                  waren.}}}\label{K_L02974-4h} ſind ihm höchlich aufgefallen; – er hat ſich gefragt: Was ko{\geminationm}t da für ein {\pb}{[}»{]}Nachwuchs« – er iſt es, der in d \label{K_L02974-5v}\edtext{\textcolor{brown}{N. Fr. Pr.}{}\ledrightnote{\textcolor{brown}{Neue Freie Presse}} mit lebhafteſter Betonung von Ihnen
               ſprach, worauf \textcolor{blue}{\textsc{Bened.}}{}\ledrightnote{\textcolor{blue}{Moriz Benedikt}} meinte, er dächte ſchon lange Zeit an Sie {\dots} Das will
               natürlich nicht viel heißen; aber ich glaube, we{\geminationn} Sie zu
               irgendwelchen Schritten}{\lemma{\textnormal{\emph{N. Fr. Pr. … Schritten}}}\Cendnote{\textnormal{vgl. Felix Salten an Arthur Schnitzler, 2[3]. 5. 1902}}}\label{K_L02974-5h} ſich entſchlöſſen (über die natürlich noch geſprochen werden muſs), ſo wären
               hier die Chancen, mindeſtens materiell günſtiger als bei der \textcolor{brown}{Zeit}{}\ledrightnote{\textcolor{brown}{Die Zeit}}. Obwohl {\pb}\textcolor{blue}{Kanner}{}\ledrightnote{\textcolor{blue}{Heinrich Kanner}} zu \textcolor{blue}{P. G.}{}\ledrightnote{\textcolor{blue}{Paul Goldmann}}, der auch dort von Ihnen redete, geäußert hat: »\label{K_L02974-6v}\edtext{Er wird ja für uns ſchreiben.}{\lemma{\textnormal{\emph{Er … ſchreiben.}}}\Cendnote{\textnormal{\textcolor{blue}{Kanner} wahrte \textcolor{blue}{Salten}s Pseudonym und erzählte nicht, dass dieser schon
                  begonnen hatte, für die Wochenschrift \emph{\textcolor{brown}{Die Zeit}}
                  zu schreiben. Die Auskunft bezog sich nur auf die anlaufende Gründung der neuen
                     \textcolor{green}{Tageszeitung}, die ab dem
                     27. 9. 1902 erschien.}}}\label{K_L02974-6h}« –\pend
           
\pstart
           \textcolor{blue}{\textsc{Kainz}}{}\ledrightnote{\textcolor{blue}{Josef Kainz}} will durchaus im »\textcolor{green}{Weg zum Licht}{}\ledrightnote{\textcolor{green}{Der Weg zum Licht. Ein Salzburger Märchendrama in vier Akten}}« ſpielen;
               u \label{K_L02974-7v}\edtext{\textcolor{brown}{\textcolor{blue}{Schlenther}{}\ledrightnote{\textcolor{blue}{Paul Schlenther}}}{}\ledrightnote{{$\rightarrow$}\textcolor{brown}{Burgtheater}} dürfte es daher aufführen}{\lemma{\textnormal{\emph{Schlenther … aufführen}}}\Cendnote{\textnormal{nicht
                  geschehen}}}\label{K_L02974-7h} (So \textcolor{blue}{Brahm}{}\ledrightnote{\textcolor{blue}{Otto Brahm}}.) Es iſt recht
               lächerlich, daſs ein ſolcher Künſtler den \label{K_L02974-8v}\edtext{\textcolor{green}{Hahngikl}{}\ledrightnote{{$\rightarrow$}\textcolor{green}{Der Weg zum Licht. Ein Salzburger Märchendrama in vier Akten}}}{\lemma{\textnormal{\emph{Hahngikl}}}\Cendnote{\textnormal{laut Figurenliste »ein Dunkelelb
                     vom Untersberg«}}}\label{K_L02974-8h} dem \label{K_L02974-9v}\edtext{\textsc{\textcolor{green}{Bentivoglio}{}\ledrightnote{{$\rightarrow$}\textcolor{green}{Der Schleier der Beatrice. Schauspiel in fünf Akten}}}}{\lemma{\textnormal{\emph{Bentivoglio}}}\Cendnote{\textnormal{Hauptfigur von \emph{\textcolor{green}{Der Schleier der Beatrice}}. Zur Ablehnung des \textcolor{green}{Stücks} durch das \emph{\textcolor{brown}{Burgtheater}}{ }siehe Richard Beer-Hofmann an Arthur Schnitzler, 14. 9. 1900.}}}\label{K_L02974-9h} vorzieht;
               aber es liegt wohl recht tief. – Dem \label{K_L02974-10v}\edtext{\textcolor{brown}{Deutſch Theater}{}\ledrightnote{\textcolor{brown}{Deutsches Theater Berlin}} geht es hier}{\lemma{\textnormal{\emph{Deutſch … hier}}}\Cendnote{\textnormal{Das \emph{\textcolor{brown}{Deutsche Theater Berlin}} spielte von 6. 5. 1902 bis 
                     zum 5. 6. 1902 im \textcolor{pink}{Carl-Theater} in
                     \textcolor{pink}{Wien}
                     ein »Gesammt-Gastpiel«.}}}\label{K_L02974-10h} ausgezeichnet. – Der \label{K_L02974-11v}\edtext{\textcolor{green}{\textcolor{green}{Kakadu}{}\ledrightnote{\textcolor{green}{Der grüne Kakadu. Groteske in einem Akt}}}{}\ledrightnote{{$\rightarrow$}\textcolor{green}{Au Perroquet Vert}} iſt {\pb}bei \textcolor{brown}{Antoine}{}\ledrightnote{\textcolor{brown}{Théâtre Antoine}}}{\lemma{\textnormal{\emph{Kakadu iſt bei Antoine}}}\Cendnote{\textnormal{\emph{\textcolor{green}{Au
                     Perroquet Vert}}, die Übersetzung von \emph{\textcolor{green}{Der
                     grüne Kakadu}}, hatte am 7. 11. 1903 am \emph{\textcolor{brown}{Théâtre Antoine}} Premiere.}}}\label{K_L02974-11h} acceptirt. –
                  \label{K_L02974-12v}\edtext{Über die \textsc{\textcolor{green}{Bea.}{}\ledrightnote{\textcolor{green}{Der Schleier der Beatrice. Schauspiel in fünf Akten}}} ſpricht \textcolor{blue}{Brahm}{}\ledrightnote{\textcolor{blue}{Otto Brahm}} kein Wort}{\lemma{\textnormal{\emph{Über … Wort}}}\Cendnote{\textnormal{Nach der Enttäuschung der Uraufführung von
                     \emph{\textcolor{green}{Der Schleier der Beatrice}} setzte \textcolor{blue}{Schnitzler} seine Hoffnungen auf eine
                  Inszenierung am \emph{\textcolor{brown}{Deutschen Theater Berlin}}.
                  Diese fand am 7. 3. 1903 statt.}}}\label{K_L02974-12h}. – Ich überdenke und scenire mein \textcolor{green}{Stück}{}\ledrightnote{{$\rightarrow$}\textcolor{green}{Der einsame Weg. Schauspiel in fünf Akten}} u übe mich indeſs weiter
               im Erzählen!\pend
           
\pstart
           – Sagen Sie mir doch etwas über Ihre Reiſe, Ihre Arbeiten, Ihre Laune. Daſs \textcolor{blue}{Hugo}{}\ledrightnote{\textcolor{blue}{Hugo von Hofmannsthal}} ein ganz kleines \label{K_L02974-13v}\edtext{\textcolor{blue}{Kind}{}\ledrightnote{{$\rightarrow$}\textcolor{blue}{Christiane Zimmer}} beko{\geminationm}en hat, \textcolor{blue}{Chriſtiane}{}\ledrightnote{\textcolor{blue}{Christiane Zimmer}}}{\lemma{\textnormal{\emph{Kind … Chriſtiane}}}\Cendnote{\textnormal{\textcolor{blue}{Christiane von Hofmannsthal} kam am 14. 5. 1902 auf die Welt.}}}\label{K_L02974-13h} genannt, wiſſen Sie
               wohl ſchon. – Heute{ }{\pb}hatten wir beinah einen »Frühlingsabend« –
               lau, ohne Wind und Regen, man faſſt es kaum. – \textsc{\label{K_L02974-14v}\edtext{\textcolor{blue}{\textcolor{green}{Rochefort}{}\ledrightnote{{$\rightarrow$}\textcolor{green}{Abenteuer meines Lebens}}}{}\ledrightnote{\textcolor{blue}{Henri de Rochefort}}}{\lemma{\textnormal{\emph{Rochefort}}}\Cendnote{\textnormal{Es dürfte sich um die (gekürzte)
                     deutschsprachige Ausgabe der Autobiografie von \textcolor{blue}{Henri Rochefort}: \emph{\textcolor{green}{Les
                        Aventures de ma vie}} (1896) handeln: \emph{\textcolor{green}{Abenteuer meines Lebens}}. Autorisierte
                        deutsche Bearbeitung von \textcolor{blue}{Heinrich
                           Conrad}. \textcolor{pink}{Stuttgart}: \emph{\textcolor{brown}{Robert Lutz}}{ }1900.}}}\label{K_L02974-14h}} wird gegen Schluſs matter; ich beſchäftige mich ein weniges mit \label{K_L02974-15v}\edtext{Botanik}{\lemma{\textnormal{\emph{Botanik}}}\Cendnote{\textnormal{Am 23. 5. 1902 besuchte \textcolor{blue}{Schnitzler}
                  den \textcolor{pink}{Botanischen Garten}.}}}\label{K_L02974-15h} und denke
               wieder manchmal mit Wehmut, wie faul ich mein Leben lang war, und auf wie viel
                  beſſer\textcolor{gray}{m} Grund ich {\pb}ſtehen könnte, we{\geminationn} ich nicht gar ſo ſpät auf mich
               aufmerkſam geworden wäre.\pend
           
\pstart
           Leben Sie wohl. Grüßen Sie \textcolor{pink}{Florenz}{}\ledrightnote{\textcolor{pink}{Florenz}}, die \textcolor{pink}{\textsc{Mediceer} Gräber}{}\ledrightnote{\textcolor{pink}{Medici-Kapelle}}, den Garten hinter dem \textcolor{pink}{Kloſter zu \textsc{Fiesole}}{}\ledrightnote{\textcolor{pink}{San Domenico}} und \textsc{\textcolor{green}{Veronika}{}\ledrightnote{\textcolor{green}{Die kleine Veronika}}}; – und \label{K_L02974-16v}\edtext{Bern}{\lemma{\textnormal{\emph{Bern}}}\Cendnote{\textnormal{vgl. Paul Goldmann an Arthur Schnitzler, 17. 4. [1902]}}}\label{K_L02974-16h} grüßt den andern \label{K_L02974-17v}\edtext{\textcolor{green}{Hund}{}\ledrightnote{{$\rightarrow$}\textcolor{green}{Der Hund von Florenz}}}{\lemma{\textnormal{\emph{Hund}}}\Cendnote{\textnormal{vgl. Felix Salten an Arthur Schnitzler, 20. 5. 1902}}}\label{K_L02974-17h}.\pend
           
\pstart
           Herzlichst Ihr {\\[\baselineskip]}\spacefill\mbox{A.}\pend
           \leftskip=0em{}\endnumbering\briefempfaengerindex{Salten, Felix@\textsc{Salten, Felix}!zzzSchnitzler, Arthur@\emph{von Arthur Schnitzler}!1902-05-271@{27. 5. 1902}|)be}\mylabel{h}  \normalsize

\doendnotes{C}
\bigskip
\vfill

\clearpage

\footnotesize

\lohead{\textsc{register}}

% Definiere theindex-Environment komplett neu ohne reledmac
\makeatletter
\renewenvironment{theindex}{%
  \section*{\indexname}%
  \setlength{\parindent}{0pt}%
  \setlength{\parskip}{0pt plus 0.3pt}%
  \let\item\@idxitem
}{%
  \clearpage
}
\makeatother

\IfFileExists{\jobname-pw.ind}{\input{\jobname-pw.ind}}{}

\end{document}

      