%% latex-korrekturansicht-vorspann.tex
%% Vorspann für die Korrekturansicht.
%% Lädt die gemeinsame Datei latex-vorspann.tex mit gesetztem Schalter.

\newif\ifkorrekturansicht
\korrekturansichttrue

\input{../tex-inputs/latex-vorspann}


\renewcommand{\erwaehntePersonen}{Personen: Albert Bassermann, Lucie Höflich, Hans Pagay, Felix Salten, Ottilie Salten, Julius Ferdinand Wollf}
\renewcommand{\erwaehnteOrte}{Orte: Berlin, Edmund-Weiß-Gasse 7, Kammerspiele Berlin, Wien, XVIII., Währing}
\renewcommand{\erwaehnteWerke}{Werke: Liebelei. Schauspiel in drei Akten, Vom andern Ufer. Einakter}
\section[ Felix Salten an Arthur Schnitzler, 15. 10. 1907]{Felix Salten an Arthur Schnitzler, 15. 10. 1907}
\nopagebreak\mylabel{v}
\rehead{ }\normalsize\beginnumbering\briefempfaengerindex{Schnitzler, Arthur@\textsc{Schnitzler, Arthur}!zzzSalten, Felix@\emph{von Felix Salten}!1907-10-151@{15. 10. 1907}|(be}
\toendnotes[C]{\smallbreak\pagebreak[2]}\Standort{CUL, Schnitzler, B 89, B 1.}
\physDesc{Postkarte, 882 Zeichen
\newline{}Handschrift: schwarze Tinte, lateinische Kurrent
\newline{}Versand: Stempel: »\nobreak{}\oindex{Berlin@\textbf{Berlin}, \emph{P.PPLC}|pwk}Berlin\textcolor{gray}{,} W. 50, 15. 10. 07, 6–7 N.\nobreak{}«. Stempel: »\nobreak{}\oindex{XVIII., Waehring@\textbf{XVIII., Währing}, \emph{A.ADM3}|pwk}18/\textsubscript{1} Wien 110 , 17. X. 07, VIII\nobreak{}«.  
\newline{}Ordnung: mit Bleistift von unbekannter Hand nummeriert: »236« }\toendnotes[C]{\smallbreak}\pstart{}{\pb}Herrn D\textsuperscript{r} Arthur Schnitzler\pend{}\pstart{}\textcolor{pink}{Wien XVIII.}{}\ledrightnote{\textcolor{pink}{XVIII., Währing}}\pend{}\pstart{}\textcolor{pink}{Spöttelgasse 7}{}\ledrightnote{\textcolor{pink}{Edmund-Weiß-Gasse 7}}\pend{}
{\bigskip}
\pstart
           \raggedleft{}{\pb}\textcolor{pink}{Berlin}{}\ledrightnote{\textcolor{pink}{Berlin}}, 15. X. 07\pend
           
\pstart{}Lieber,\pend
\pstart
           gestern waren wir in den \label{K_L03513-1v}\edtext{\textcolor{pink}{Kammerspielen}{}\ledrightnote{\textcolor{pink}{Kammerspiele Berlin}} bei der
                  »\textcolor{green}{Liebelei}{}\ledrightnote{\textcolor{green}{Liebelei. Schauspiel in drei Akten}}}{\lemma{\textnormal{\emph{Kammerspielen … »Liebelei}}}\Cendnote{\textnormal{Seit dem 19. 9. 1907 wurde \emph{\textcolor{green}{Liebelei}} in einer Neuinszenierung an den \textcolor{pink}{Berliner Kammerspielen} gegeben. Vgl. Hermann Bahr an Arthur Schnitzler, 12. 2. 1907. }}}\label{K_L03513-1h}«. Ich möchte
               Ihnen sagen, wie sehr mich dieses \textcolor{green}{Stück}{}\ledrightnote{{$\rightarrow$}\textcolor{green}{Liebelei. Schauspiel in drei Akten}} wieder ergriffen hat. Übrigens nicht mich allein, sondern alle. \textcolor{blue}{Otti}{}\ledrightnote{\textcolor{blue}{Ottilie Salten}}, \textcolor{blue}{Wollf}{}\ledrightnote{\textcolor{blue}{Julius Ferdinand Wollf}}, und das ganze Publicum. Bei mir waren da natürlich noch andere Dinge,
               die mich im Anhören tief gerührt haben. Aber daneben und drüber hinaus hab ich doch
               gesehen, wie schön dieses \textcolor{green}{Werk}{}\ledrightnote{{$\rightarrow$}\textcolor{green}{Liebelei. Schauspiel in drei Akten}}
               ist, und habe vor allem gespürt, dass es sicherlich bleiben wird. Es ist ein Ausdruck
               unserer Epoche darin und dabei etwas so zeitlos Wahres und im Gefühl Starkes. Die \textcolor{blue}{Höflich}{}\ledrightnote{\textcolor{blue}{Lucie Höflich}} über alle Begriffe herrlich. \textcolor{blue}{Pagay}{}\ledrightnote{\textcolor{blue}{Hans Pagay}} einfach wundervoll. Die Anderen fast
               unmöglich. – Heute war \label{K_L03513-2v}\edtext{\textcolor{green}{Generalprobe}{}\ledrightnote{{$\rightarrow$}\textcolor{green}{Vom andern Ufer. Einakter}}}{\lemma{\textnormal{\emph{Generalprobe}}}\Cendnote{\textnormal{von \textcolor{blue}{Salten}s Einakterreihe \emph{\textcolor{green}{Vom andern
                     Ufer}}, die noch am selben Tag uraufgeführt wurde}}}\label{K_L03513-2h}, und ich weiß noch
               garnichts. \textcolor{blue}{Bassermann}{}\ledrightnote{\textcolor{blue}{Albert Bassermann}} beinahe schlecht. Die
               Wirkung auf mich matt. Ich bin bald in \textcolor{pink}{Wien}{}\ledrightnote{\textcolor{pink}{Wien}}.\pend
           
\pstart
           Inzwischen viele schöne Grüße von \textcolor{blue}{uns}{}\ledrightnote{{$\rightarrow$}\textcolor{blue}{Ottilie Salten}} zu Ihnen, herzlichst {\\[\baselineskip]}Ihr {\\[\baselineskip]}\spacefill\mbox{Salten}\pend
           \leftskip=0em{}\endnumbering\briefempfaengerindex{Schnitzler, Arthur@\textsc{Schnitzler, Arthur}!zzzSalten, Felix@\emph{von Felix Salten}!1907-10-151@{15. 10. 1907}|)be}\mylabel{h}  \normalsize

\doendnotes{C}
\bigskip
\vfill

\clearpage

\footnotesize

\lohead{\textsc{register}}

% Definiere theindex-Environment komplett neu ohne reledmac
\makeatletter
\renewenvironment{theindex}{%
  \section*{\indexname}%
  \setlength{\parindent}{0pt}%
  \setlength{\parskip}{0pt plus 0.3pt}%
  \let\item\@idxitem
}{%
  \clearpage
}
\makeatother

\IfFileExists{\jobname-pw.ind}{\input{\jobname-pw.ind}}{}

\end{document}

      