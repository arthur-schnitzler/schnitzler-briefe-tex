%% latex-korrekturansicht-vorspann.tex
%% Vorspann für die Korrekturansicht.
%% Lädt die gemeinsame Datei latex-vorspann.tex mit gesetztem Schalter.

\newif\ifkorrekturansicht
\korrekturansichttrue

\input{../tex-inputs/latex-vorspann}


\section[Richard Beer-Hofmann an Arthur Schnitzler, 28. 6. 1901]{L01137 Richard Beer-Hofmann an Arthur Schnitzler, 28. 6. 1901}
\nopagebreak\mylabel{L01137v}
\rehead{ }\normalsize\beginnumbering\briefempfaengerindex{Schnitzler, Arthur@\textsc{Schnitzler, Arthur}!zzzBeer-Hofmann, Richard@\emph{von Richard Beer-Hofmann}!1901-06-281@{28. 6. 1901}|(be}
\toendnotes[C]{\smallbreak\pagebreak[2]}
\correspDesc{Versand  durch Richard Beer-Hofmann am 28. 6. 1901 in Pörtschach
\newline{}Erhalt  durch Arthur Schnitzler am [2. 7. 1901] in St. Anton am Arlberg}\toendnotes[C]{\smallbreak}
\Standort{CUL, Schnitzler, B 8.}
\physDesc{Brief, 1 Blatt, 2 Seiten, 700 Zeichen
\newline{}Handschrift: blauer Buntstift, lateinische Kurrent
\newline{}Ordnung: mit Bleistift von unbekannter Hand nummeriert:
                                    »163« }
\buchAbdrucke{\weitereDrucke{Arthur Schnitzler, Richard Beer-Hofmann: \emph{Briefwechsel 1891–1931}. Herausgegeben von Konstanze Fliedl. Wien, Zürich: \emph{Europaverlag} 1992, S. 152.} }\toendnotes[C]{\smallbreak}
\pstart
           \raggedleft{}{\pb}\textcolor{pink}{Pörtschach}\oindex{Pörtschach am Wörthersee@\textbf{Pörtschach am Wörthersee}|pw}{}\ledrightnote{\textcolor{pink}{Pörtschach am Wörthersee}}{ }28/VI 1901\pend
           \vspace{0.5em}
\pstart
           Lieber Arthur! Es war Zeit daß Sie von Sich hören ließen. Ich wußte
               nur durch die \textcolor{green}{\textcolor{brown}{N. Fr Pr}\orgindex{Neue Freie Presse@Neue Freie Presse|pw}{}\ledrightnote{\textcolor{brown}{Neue Freie Presse}}}\pwindex{Kleine Chronik@\emph{Kleine Chronik}|pwv}{}\ledrightnote{{$\rightarrow$}\emph{\textcolor{green}{Kleine Chronik}}} daß Sie in \textcolor{pink}{Tirol}\oindex{Tirol@\textbf{Tirol}, \emph{Land}|pw}{}\ledrightnote{\textcolor{pink}{Tirol}} sind. Ich war – um mir
               Heiterkeit zu holen – 3 Tage in \textcolor{pink}{Venedig}\oindex{Venedig@\textbf{Venedig}|pw}{}\ledrightnote{\textcolor{pink}{Venedig}},
               gleichzeitig mit \textcolor{blue}{Hugo}\pwindex{Hofmannsthal, Hugo von 1.\,2.\,1874 Wien – 15.\,7.\,1929 Rodaun@\textsc{Hofmannsthal, Hugo von} (1.\,2.\,1874 Wien – 15.\,7.\,1929 Rodaun), \emph{Schriftsteller}|pw}{}\ledrightnote{\textcolor{blue}{Hugo von Hofmannsthal}}, doch wußten wir von
               einander nichts, und erst als ich zurückkam erfuhr ich daß er auch dort war. Ich habe
               mir aber keine Heiterkeit aus \textcolor{pink}{Venedig}\oindex{Venedig@\textbf{Venedig}|pw}{}\ledrightnote{\textcolor{pink}{Venedig}} geholt.\pend
           
\pstart
           {\pb}Ich möchte wissen wann Sie
               herkommen, und ob und wann \textcolor{blue}{Paul}\pwindex{Goldmann, Paul 31.\,1.\,1865 Breslau – 25.\,9.\,1935 Wien@\textsc{Goldmann, Paul} (31.\,1.\,1865 Breslau – 25.\,9.\,1935 Wien), \emph{Schriftsteller, Journalist}|pw}{}\ledrightnote{\textcolor{blue}{Paul Goldmann}} hieherko{\geminationm}t. \textcolor{blue}{Ludassy}\pwindex{Gans-Ludassy, Julius von 13.\,4.\,1858 Wien – 30.\,9.\,1922 ebd.@\textsc{Gans-Ludassy, Julius von} (13.\,4.\,1858 Wien – 30.\,9.\,1922 ebd.), \emph{Schriftsteller, Journalist, Herausgeber}|pw}{}\ledrightnote{\textcolor{blue}{Julius von Gans-Ludassy}} und
                  \textcolor{blue}{Alexander Engel}\pwindex{Engel, Alexander 10.\,4.\,1868 Necpaly – 17.\,11.\,1940 Wien@\textsc{Engel, Alexander} (10.\,4.\,1868 Necpaly – 17.\,11.\,1940 Wien), \emph{Schriftsteller, Journalist}|pw}{}\ledrightnote{\textcolor{blue}{Alexander Engel}} habe ich hier gesprochen. –
                  \textcolor{blue}{L.}\pwindex{Gans-Ludassy, Julius von 13.\,4.\,1858 Wien – 30.\,9.\,1922 ebd.@\textsc{Gans-Ludassy, Julius von} (13.\,4.\,1858 Wien – 30.\,9.\,1922 ebd.), \emph{Schriftsteller, Journalist, Herausgeber}|pw}{}\ledrightnote{\textcolor{blue}{Julius von Gans-Ludassy}} erklärte es unsicher daß Sie kämen. \textcolor{blue}{Hirschfeld (Robert)}\pwindex{Hirschfeld, Robert 17.\,9.\,1857 Žďár nad Sázavou – 2.\,4.\,1914 Salzburg@\textsc{Hirschfeld, Robert} (17.\,9.\,1857 Žďár nad Sázavou – 2.\,4.\,1914 Salzburg), \emph{Journalist, Musikkritiker}|pw}{}\ledrightnote{\textcolor{blue}{Robert Hirschfeld}} hat uns besucht. Was ist
               mit \textcolor{blue}{Salten}\pwindex{Salten, Felix 6.\,9.\,1869 Budapest – 8.\,10.\,1945 Zürich@\textsc{Salten, Felix} (6.\,9.\,1869 Budapest – 8.\,10.\,1945 Zürich), \emph{Schriftsteller, Journalist, Chefredakteur}|pw}{}\ledrightnote{\textcolor{blue}{Felix Salten}} und seinem bodenständigen \textcolor{brown}{Brettl}\orgindex{Jung-Wiener Theater zum Lieben Augustin@Jung-Wiener Theater zum Lieben Augustin|pwv}{}\ledrightnote{{$\rightarrow$}\emph{\textcolor{brown}{Jung-Wiener Theater zum Lieben Augustin}}}; aber wichtiger: Was ist
               mit Ihnen? Ist \textcolor{pink}{Salzburg}\oindex{Salzburg@\textbf{Salzburg}, \emph{Verwaltungsgebiet}|pw}{}\ledrightnote{\textcolor{pink}{Salzburg}} noch immer gegen Versti{\geminationm}ung gut? Von Herzen\pend
           \pstart Ihr \spacefill\mbox{Richard}\pend{}\selectlanguage{ngerman}\endnumbering\briefempfaengerindex{Schnitzler, Arthur@\textsc{Schnitzler, Arthur}!zzzBeer-Hofmann, Richard@\emph{von Richard Beer-Hofmann}!1901-06-281@{28. 6. 1901}|)be}\mylabel{L01137h}  \normalsize

\doendnotes{C}
\bigskip
\vfill

\clearpage

\footnotesize

\lohead{\textsc{register}}

% Definiere theindex-Environment komplett neu ohne reledmac
\makeatletter
\renewenvironment{theindex}{%
  \section*{\indexname}%
  \setlength{\parindent}{0pt}%
  \setlength{\parskip}{0pt plus 0.3pt}%
  \let\item\@idxitem
}{%
  \clearpage
}
\makeatother

\IfFileExists{\jobname-pw.ind}{\input{\jobname-pw.ind}}{}

\end{document}

      