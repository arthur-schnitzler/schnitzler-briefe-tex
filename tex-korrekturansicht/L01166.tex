%% latex-korrekturansicht-vorspann.tex
%% Vorspann für die Korrekturansicht.
%% Lädt die gemeinsame Datei latex-vorspann.tex mit gesetztem Schalter.

\newif\ifkorrekturansicht
\korrekturansichttrue

\input{../tex-inputs/latex-vorspann}


               \section[Richard Beer-Hofmann an Arthur Schnitzler, 21. 8. 1901]{ Richard Beer-Hofmann an Arthur Schnitzler,
               21. 8. 1901}\nopagebreak\mylabel{v}\rehead{ }\normalsize\beginnumbering\briefempfaengerindex{Schnitzler, Arthur@\textsc{Schnitzler, Arthur}!zzzBeer-Hofmann, Richard@\emph{von Richard Beer-Hofmann}!1901-08-212@{21. 8. 1901}|(be} \toendnotes[C]{\smallbreak\pagebreak[2]} \Standort{CUL, Schnitzler, B 8.}
\physDesc{Brief, 1 Blatt, 2 Seiten
\newline{}Handschrift: Bleistift, lateinische Kurrent\newline{}Ordnung: mit Bleistift von unbekannter Hand nummeriert: »169« }\buchAbdrucke{\weitereDrucke{Arthur Schnitzler, Richard Beer-Hofmann: \emph{Briefwechsel 1891–1931}. Hg. Konstanze Fliedl. Wien, Zürich: \emph{Europaverlag} 1992, S. 155–156.} }\pstart
           \raggedleft{}{\pb}\textcolor{pink}{Pörtschach}{}\ledrightnote{\textcolor{pink}{Pörtschach}}{ }21/VIII 01\pend
           \pstart
           Lieber! Von \textcolor{blue}{Paul}{}\ledrightnote{\textcolor{blue}{Paul Goldmann}} erhalte ich
               eben folgendes Telegramm:\pend
           \pstart
           »Bitte nicht dringend und morgen herzlichst \textcolor{pink}{Welsberg}{}\ledrightnote{\textcolor{pink}{Welsberg-Taisten}}
                  ko{\geminationm}en. Drathantwort erbeten\hspace*{1.5em}\textcolor{blue}{Goldmann}{}\ledrightnote{\textcolor{blue}{Paul Goldmann}}«.\pend
           \pstart
           »Zip-Zip«, oder »Was ist das?« Ich habe nach \textcolor{pink}{Toblach}{}\ledrightnote{\textcolor{pink}{Toblach}}
               – da das Telegra{\geminationm} dort aufgegeben war – telegraphirt:
                  »Telegra{\geminationm} unverständlich«.\pend
           \pstart
           Ist \textcolor{blue}{Paul}{}\ledrightnote{\textcolor{blue}{Paul Goldmann}} nicht bei Ihnen? Oder ist er Vormittag
               nach \textcolor{pink}{Toblach}{}\ledrightnote{\textcolor{pink}{Toblach}} spazieren gelaufen um von dort zu
                  tele{\pb}graphiren? Schreiben oder
               telegraphiren Sie bitte was denn los ist.\pend
           \pstart
           Von Herzen Ihr{\\[\baselineskip]}\spacefill\mbox{Richard}\pend
           \leftskip=0em{}\pstart
           Vorsichtshalber telegraphire ich an Sie auch unverständlich\pend
           \endnumbering\briefempfaengerindex{Schnitzler, Arthur@\textsc{Schnitzler, Arthur}!zzzBeer-Hofmann, Richard@\emph{von Richard Beer-Hofmann}!1901-08-212@{21. 8. 1901}|)be}\mylabel{h}  \normalsize

\doendnotes{C}
\bigskip
\vfill

\clearpage

\footnotesize

\lohead{\textsc{register}}

% Definiere theindex-Environment komplett neu ohne reledmac
\makeatletter
\renewenvironment{theindex}{%
  \section*{\indexname}%
  \setlength{\parindent}{0pt}%
  \setlength{\parskip}{0pt plus 0.3pt}%
  \let\item\@idxitem
}{%
  \clearpage
}
\makeatother

\IfFileExists{\jobname-pw.ind}{\input{\jobname-pw.ind}}{}

\end{document}

      