%% latex-korrekturansicht-vorspann.tex
%% Vorspann für die Korrekturansicht.
%% Lädt die gemeinsame Datei latex-vorspann.tex mit gesetztem Schalter.

\newif\ifkorrekturansicht
\korrekturansichttrue

\input{../tex-inputs/latex-vorspann}


               \section[Arthur und Olga Schnitzler an Richard und Paula Beer-Hofmann, 1{[}2?{]}. 5. 1914]{ Arthur und Olga Schnitzler an Richard und Paula Beer-Hofmann,
               1{[}2?{]}. 5. 1914}\nopagebreak\mylabel{v}\rehead{ }\normalsize\beginnumbering\briefempfaengerindex{Beer-Hofmann, Paula@\textsc{Beer-Hofmann, Paula}!zzzSchnitzler, Olga@\emph{von Olga Schnitzler}!1914-05-121@{1{[}2?{]}. 5. 1914}|(be}\briefempfaengerindex{Beer-Hofmann, Paula@\textsc{Beer-Hofmann, Paula}!zzzSchnitzler, Arthur@\emph{von Arthur Schnitzler}!1914-05-121@{1{[}2?{]}. 5. 1914}|(be}\briefempfaengerindex{Beer-Hofmann, Richard@\textsc{Beer-Hofmann, Richard}!zzzSchnitzler, Olga@\emph{von Olga Schnitzler}!1914-05-121@{1{[}2?{]}. 5. 1914}|(be}\briefempfaengerindex{Beer-Hofmann, Richard@\textsc{Beer-Hofmann, Richard}!zzzSchnitzler, Arthur@\emph{von Arthur Schnitzler}!1914-05-121@{1{[}2?{]}. 5. 1914}|(be} \toendnotes[C]{\smallbreak\pagebreak[2]} \Standort{YCGL, MSS 31.}
\physDesc{Bildpostkarte
\newline{}Handschrift Arthur Schnitzler: Bleistift, deutsche Kurrent\newline{}Handschrift Olga Schnitzler: Bleistift, lateinische Kurrent\newline{}Versand: 1) Stempel: »\nobreak{}\oindex{Genua@\textbf{Genua}, \emph{Besiedelter Ort (A.BSO)}|pwk}Genova 1914, Esposizione internazionale Igiene – Marina –
                                       Colonie\nobreak{}«.  2) Stempel: »\nobreak{}\oindex{Bahnhof Genua@\textbf{Bahnhof Genua}, \emph{Bahnhofsgebäude (K.BHF)}|pwk}Genova Ferrovia, 13. V. 14, 16\nobreak{}«. 3) mit blauem Buntstift von unbekannter Hand der Postrayon zur
                                 Bezirksangabe in der Adressierung ergänzt:
                                 »/2«}\buchAbdrucke{\weitereDrucke{Arthur Schnitzler, Richard Beer-Hofmann: \emph{Briefwechsel 1891–1931}. Hg. Konstanze Fliedl. Wien, Zürich: \emph{Europaverlag} 1992, S. 219.} }\toendnotes[C]{\smallbreak}\pstart{}{\pb}Hrn \textsc{Dr. Richard Beer
                     Hofmann}\pend{}\pstart{}und Frau.\pend{}\pstart{}\textsc{\textcolor{pink}{Wien XVIII}{}\ledrightnote{\textcolor{pink}{XVIII., Währing}}}\pend{}\pstart{}\textcolor{pink}{\textsc{Hasenauerstr 59}}{}\ledrightnote{\textcolor{pink}{Hasenauerstraße}}\pend{}{\bigskip}\pstart
           \noindent{}\centering{}{\pb}\textcolor{gray}{\textbf{\textcolor{pink}{Genova – Piazza De Ferrari}{}\ledrightnote{\textcolor{pink}{Piazza Raffaele de Ferrari}}}}\pend
           \pstart
           {\pb}Herzliche Grüße!\pend
           \pstart Ihr \spacefill\mbox{Arthur}\pend{}\pstart
           \noindent{}{[}hs. O. Schnitzler:{]} Haben in \textcolor{pink}{Florenz}{}\ledrightnote{\textcolor{pink}{Florenz}} in einem
                  \label{KLL02178_Beer-Hofmann-2v}\edtext{Varieté}{\lemma{\textnormal{\emph{Varieté}}}\Cendnote{\textnormal{vgl. A. S.: \emph{Tagebuch}, 9. 5. 1914}}}\label{KLL02178_Beer-Hofmann-2h} besonders heftig Ihrer gedacht. – was Sie nun neugierig machen möge!\pend
           \pstart Herzlichst \spacefill\mbox{Olga.}\pend{}\pstart
           \noindent{}{[}hs. Schnitzler:{]} \label{KLL02178_Beer-Hofmann-1v}\edtext{Morgen}{\lemma{\textnormal{\emph{Morgen}}}\Cendnote{\textnormal{Das erlaubt, die Karte am Tag
                  vor dem Poststempel zu datieren, da die Abfahrt am 13. 5. 1914 stattfand.}}}\label{KLL02178_Beer-Hofmann-1h} ab nach \textcolor{pink}{Algier}{}\ledrightnote{\textcolor{pink}{Algiers}}\pend
           \endnumbering\briefempfaengerindex{Beer-Hofmann, Paula@\textsc{Beer-Hofmann, Paula}!zzzSchnitzler, Olga@\emph{von Olga Schnitzler}!1914-05-121@{1{[}2?{]}. 5. 1914}|)be}\briefempfaengerindex{Beer-Hofmann, Paula@\textsc{Beer-Hofmann, Paula}!zzzSchnitzler, Arthur@\emph{von Arthur Schnitzler}!1914-05-121@{1{[}2?{]}. 5. 1914}|)be}\briefempfaengerindex{Beer-Hofmann, Richard@\textsc{Beer-Hofmann, Richard}!zzzSchnitzler, Olga@\emph{von Olga Schnitzler}!1914-05-121@{1{[}2?{]}. 5. 1914}|)be}\briefempfaengerindex{Beer-Hofmann, Richard@\textsc{Beer-Hofmann, Richard}!zzzSchnitzler, Arthur@\emph{von Arthur Schnitzler}!1914-05-121@{1{[}2?{]}. 5. 1914}|)be}\mylabel{h}  \normalsize

\doendnotes{C}
\bigskip
\vfill

\clearpage

\footnotesize

\lohead{\textsc{register}}

% Definiere theindex-Environment komplett neu ohne reledmac
\makeatletter
\renewenvironment{theindex}{%
  \section*{\indexname}%
  \setlength{\parindent}{0pt}%
  \setlength{\parskip}{0pt plus 0.3pt}%
  \let\item\@idxitem
}{%
  \clearpage
}
\makeatother

\IfFileExists{\jobname-pw.ind}{\input{\jobname-pw.ind}}{}

\end{document}

      