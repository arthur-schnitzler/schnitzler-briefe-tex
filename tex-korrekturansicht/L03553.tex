%% latex-korrekturansicht-vorspann.tex
%% Vorspann für die Korrekturansicht.
%% Lädt die gemeinsame Datei latex-vorspann.tex mit gesetztem Schalter.

\newif\ifkorrekturansicht
\korrekturansichttrue

\input{../tex-inputs/latex-vorspann}


\renewcommand{\erwaehntePersonen}{Personen: Raoul Auernheimer, Moriz Benedikt, Samuel Fischer, Felix Salten, Ottilie Salten}
\renewcommand{\erwaehnteInstitutionen}{Institutionen: Die Zeit, Neue Freie Presse}
\renewcommand{\erwaehnteOrte}{Orte: Bad Gastein, Berghof, Semmering, St. Gilgen, Unterach am Attersee, Wien, Österreich}
\renewcommand{\erwaehnteWerke}{Werke: Erinnerungen}
\section[ Felix Salten an Arthur Schnitzler, 16. 8. 1911]{Felix Salten an Arthur Schnitzler, 16. 8. 1911}
\nopagebreak\mylabel{v}
\rehead{ }\normalsize\beginnumbering\briefempfaengerindex{Schnitzler, Arthur@\textsc{Schnitzler, Arthur}!zzzSalten, Felix@\emph{von Felix Salten}!1911-08-161@{16. 8. 1911}|(be}
\toendnotes[C]{\smallbreak\pagebreak[2]}\Standort{CUL, Schnitzler, B 89, B 2.}
\physDesc{Briefkarte, 2 Karten, 4045 Zeichen (die zweite Karte markiert: »II« )
\newline{}Handschrift: schwarze Tinte, lateinische Kurrent
\newline{}Ordnung: mit Bleistift von unbekannter Hand nummeriert: »268« }\toendnotes[C]{\smallbreak}
\pstart
           \noindent{}\raggedleft{}{\pb}\textcolor{pink}{Unterach a. Attersee, Berghof}{}\ledrightnote{\textcolor{pink}{Berghof}}\pend
           
\pstart
           \raggedleft{}16. VIII. 11\pend
           
\pstart
           \textcolor{gray}{\textbf{\textsc{Felix Salten}}}\pend
           
\pstart{}Lieber,\pend
\pstart
           ich danke Ihnen herzlich für Ihren ausführlichen \label{K_L03553-1v}\edtext{Brief}{\lemma{\textnormal{\emph{Brief}}}\Cendnote{\textnormal{nicht
                  erhalten. Darin dürfte \textcolor{blue}{Schnitzler} von seinem Gespräch
                     mit \textcolor{blue}{Moriz Benedikt} berichtet haben, das am 9. 8. 1911
                     am \textcolor{pink}{Semmering} stattfand. \textcolor{blue}{Schnitzler} dürfte auch
                     begründet haben, warum er nicht auf \textcolor{blue}{Salten} einging. In seinen \emph{\textcolor{green}{Erinnerungen}} geht \textcolor{blue}{Salten} zwei Mal
                     darauf ein, dass ihn \textcolor{blue}{Schnitzler} an dieser
                     Stelle nicht unterstützt habe und lässt es dadurch zu einem zentralen Moment ihrer
                     Beziehung werden: \textcolor{blue}{Schnitzler} »lehnte viele Jahre später auch ab, als ich ihn in
                        einer Daseinskrisis bat, so beiläufig zu erkunden, was für eine Meinung der \textcolor{blue}{Herausgeber} der \textcolor{brown}{Neuen Freien Presse} von mir hege, und sagte,
                        das könne er aus Freundschaft für \textcolor{blue}{Auernheimer} nicht tun. Diese Freundschaft für \textcolor{blue}{Auernheimer} war ganz neu und ganz einseitig«. (\emph{Wienbibliothek im Rathaus}, Nachlass Salten, ZPH 1681/1
                                 1.1.1.9.1, S. [6], vgl. S. [52])}}}\label{K_L03553-1h}. Sie erinnern sich ja gewiß, dass Sie selbst mir \label{K_L03553-2v}\edtext{in \textcolor{pink}{St.
                  Gilgen}{}\ledrightnote{\textcolor{pink}{St. Gilgen}}}{\lemma{\textnormal{\emph{in St.
                  Gilgen}}}\Cendnote{\textnormal{\textcolor{blue}{Schnitzler} war zwischen 24. 7. 1911 und 29. 7. 1911 in \textcolor{pink}{St. Gilgen}; das Gespräch mit \textcolor{blue}{Salten} hatte am 27. 7. 1911
                  stattgefunden.}}}\label{K_L03553-2h} sagten, Sie kämen jetzt auf dem \textcolor{pink}{Semmering}{}\ledrightnote{\textcolor{pink}{Semmering}} mit Herrn \textcolor{blue}{Benedikt}{}\ledrightnote{\textcolor{blue}{Moriz Benedikt}} zusammmen, und ob es mir da recht sei, wenn Sie bei einer sich
               ergebenden Gelegenheit meiner Erwähnung tun würden. Ich wäre ja nicht auf diesen
               Einfall gerathen, denn einmal dachte ist nicht daran, dass Sie jetzt mit Herrn \textcolor{blue}{Benedikt}{}\ledrightnote{\textcolor{blue}{Moriz Benedikt}} zusammentreffen, dann auch wußte ich
               ja, dass Sie sich durch freundschaftliche Rücksichtnahme auf Herrn D\textsuperscript{r}{ }\textcolor{blue}{Auernheimer}{}\ledrightnote{\textcolor{blue}{Raoul Auernheimer}} in dieser Sache behindert fühlen. Eine Erwähnung meiner
               Person und \label{K_L03553-3v}\edtext{meines Austritts aus der »\textcolor{brown}{Zeit}{}\ledrightnote{\textcolor{brown}{Die Zeit}}}{\lemma{\textnormal{\emph{meines … »Zeit}}}\Cendnote{\textnormal{\textcolor{blue}{Salten}
                  war gekündigt worden, vgl. Arthur Schnitzler an Felix Salten, [14. 4. 1910?]. Bis zum Jahresende 1911
               erschienen Texte \textcolor{blue}{Salten}s im Blatt. XXXX}}}\label{K_L03553-3h}« Herrn
                  \textcolor{blue}{Benedikt}{}\ledrightnote{\textcolor{blue}{Moriz Benedikt}} gegenüber, hätte für mich wol auch
               nur informativen Erfolg haben sollen. Denn wie Sie wißen, waren wir übereingekommen,
               dass Sie nichts Intervenirendes sagen. Wenn Sie nun den Eindruck erhielten, dass
               selbst ein noch so beiläufiges Erwähnen meines Namens bei Herrn \textcolor{blue}{Benedikt}{}\ledrightnote{\textcolor{blue}{Moriz Benedikt}} die Vermutung des Absichtlichen und Intervenirenden
               wecken würde, dann war es natürlich sehr gut, derartiges ganz zu vermeiden, und ich
               danke Ihnen vielmals dafür. Was Ihren Rat betrifft, glaube ich nicht, dass ich ihn
               befolgen werde. Erstens weiß ich ja noch selber nicht, ob ich jemals wieder eine fixe
                  {\pb}Stellung annehmen werde.
               Dann aber würde diese Stellung wol für mich nicht acceptabel sein, wenn ich noch so
               offen und geradezu mich darum bewerbe, {\dotstwo} eben \uline{weil} ich mich bewerbe! Zuletzt aber gibt es für mich
               noch einen höheren Grund, mich \strikeout{\textcolor{gray}{×}\-\textcolor{gray}{×}\-\textcolor{gray}{×}\-\textcolor{gray}{×}\-\textcolor{gray}{×}\-\textcolor{gray}{×}\-\textcolor{gray}{×}\-\textcolor{gray}{×}\-\textcolor{gray}{×}\-\textcolor{gray}{×}\-\textcolor{gray}{×}\-\textcolor{gray}{×}\-\textcolor{gray}{×}\-\textcolor{gray}{×}\-\textcolor{gray}{×}\-\textcolor{gray}{×}\-\textcolor{gray}{×}\-\textcolor{gray}{×}\-\textcolor{gray}{×}\-\textcolor{gray}{×}} niemals Herrn \textcolor{blue}{Benedikt}{}\ledrightnote{\textcolor{blue}{Moriz Benedikt}} oder sonst
               Jemandem anzubieten. Ich habe das in meinen kleinsten und schwersten Anfängen nicht
               getan. Jetzt schreibe ich seit achtzehn Jahren; meine Leistung ist zu offenkundig und
               – wenn das Wort erlaubt ist, – mein Anspruch auf eine Stelle in einem Blatt \textcolor{pink}{Österreich}{}\ledrightnote{\textcolor{pink}{Österreich}}s zu gerecht, als dass ich selbst auf
               diese Leistung hinweisen oder diesen Anspruch geltend machen möchte.\pend
           
\pstart
           In einem einzigen Betracht bedaure ich es lebhaft, dass Sie nicht dazu gelangen, mit
               Herrn \textcolor{blue}{Benedikt}{}\ledrightnote{\textcolor{blue}{Moriz Benedikt}} zu sprechen. Und aus diesem
               Grund allein tut es mir leid, dass es nicht möglich ist, eine im Metier so viel
               beredte Angelegenheit, wie mein Austritt aus der »\textcolor{brown}{Zeit}{}\ledrightnote{\textcolor{brown}{Die Zeit}}« es ist, vor Herrn \textcolor{blue}{Benedikt}{}\ledrightnote{\textcolor{blue}{Moriz Benedikt}} zu
               erwähnen. Es ist mir nämlich dieser Tage zugetragen worden, Herr \textcolor{blue}{Benedikt}{}\ledrightnote{\textcolor{blue}{Moriz Benedikt}} sei – wahrscheinlich von einer mir schlecht
               gesinnten Seite – zu der Ansicht gebracht, ich lebe in völlig desolaten
               Geldverhältnissen, stecke bis über die Ohren in Schulden, und führe ein prassendes
               Verschwenderleben. Wenn er nun aufgeklärt hätte werden können, dass ich wol Schulden
               hatte (Familie \textcolor{gray}{usw}.) jetzt aber keine mehr habe, dass ich wol
               anständig, aber nicht verschwenderisch lebe, hoch versichert bin, und auch sonst
               keine materiellen Krisen habe, wäre mir das schon in einem ganz allgemeinen und
               prinzipiellen Sinn \uline{sehr} erwünscht gewesen, und es
               wäre nur eine einfache Richtigstellung, welche keine anderen, konkurrirenden
               Interessen verletzt. Nun wird es doch wol am besten sein, wenn ich in dieser ganzen
               Sache ruhig zuwarte. Ich weiß ja heute selbst {\pb}noch nicht, wofür ich mich
               entscheiden werde, und es liegen noch mehrere Monate vor mir, in denen ich alle
               Umstände prüfen, verschiedene größere Arbeiten fördern und alles zusa{\geminationm}en überlegen muß. Es kann ja auch sein, dass Herr \textcolor{blue}{Benedikt}{}\ledrightnote{\textcolor{blue}{Moriz Benedikt}} und ich nicht zusammenko{\geminationm}en, weil er auf eine Deklaration von mir und ich auf
               eine von ihm warte. Es kann ja auch (so leicht) sein, dass wir, \uline{wenn} wir schon einmal zusammenkommen, nicht mit einander einig werden.
               Und es kann auch sein, dass er mich überhaupt nicht mag und eine Verbindung mit mir
               garnicht in Erwägung zieht. Auch damit rechne ich.\pend
           
\pstart
           Bei \textcolor{blue}{uns}{}\ledrightnote{{$\rightarrow$}\textcolor{blue}{Ottilie Salten}} geht alles ziemlich
               wol. Arbeit, Gäste, Geburtstage, Ausflüge. Das wechselt so ab und ist bisher vom
               schönsten Wetter besonnt. Ich habe eine Kur begonnen und bin seither die Schmerzen
               los; habe die »\textcolor{brown}{Zeit}{}\ledrightnote{\textcolor{brown}{Die Zeit}}« ersucht, mich noch \textcolor{pink}{hier}{}\ledrightnote{{$\rightarrow$}\textcolor{pink}{Berghof}} zu laßen, damit ich diese
               Kur beendigen kann, und ihr dafür angeboten, von hier aus zu schreiben. Kann sein,
               dass sie mich trotzdem zwingt\textcolor{gray}{,} nach \textcolor{pink}{Wien}{}\ledrightnote{\textcolor{pink}{Wien}} zu gehen. \textcolor{blue}{Fischer}{}\ledrightnote{\textcolor{blue}{Samuel Fischer}} ist schon in
                  \textcolor{pink}{Gastein}{}\ledrightnote{\textcolor{pink}{Bad Gastein}}. Wir grüßen Sie alle in
               Herzlichkeit.\pend
           \pstart Ihr \spacefill\mbox{Salten}\pend{}\endnumbering\briefempfaengerindex{Schnitzler, Arthur@\textsc{Schnitzler, Arthur}!zzzSalten, Felix@\emph{von Felix Salten}!1911-08-161@{16. 8. 1911}|)be}\mylabel{h}  \normalsize

\doendnotes{C}
\bigskip
\vfill

\clearpage

\footnotesize

\lohead{\textsc{register}}

% Definiere theindex-Environment komplett neu ohne reledmac
\makeatletter
\renewenvironment{theindex}{%
  \section*{\indexname}%
  \setlength{\parindent}{0pt}%
  \setlength{\parskip}{0pt plus 0.3pt}%
  \let\item\@idxitem
}{%
  \clearpage
}
\makeatother

\IfFileExists{\jobname-pw.ind}{\input{\jobname-pw.ind}}{}

\end{document}

      