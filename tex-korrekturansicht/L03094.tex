%% latex-korrekturansicht-vorspann.tex
%% Vorspann für die Korrekturansicht.
%% Lädt die gemeinsame Datei latex-vorspann.tex mit gesetztem Schalter.

\newif\ifkorrekturansicht
\korrekturansichttrue

\input{../tex-inputs/latex-vorspann}


\renewcommand{\erwaehntePersonen}{Personen: Efraim Frisch,  Krügler, Olga Schnitzler, Elisabeth Steinrück, Ida d’Albert}
\renewcommand{\erwaehnteInstitutionen}{Institutionen: Fremden-Blatt, S. Fischer Verlag}
\renewcommand{\erwaehnteOrte}{Orte: Berlin, Dessauer Straße, Wien}
\renewcommand{\erwaehnteWerke}{Werke: Berliner Theater. »Der Rothe Hahn.«, Berliner Theater. »Einsame Menschen« im Deutschen Theater, Börsenblatt für den Deutschen Buchhandel, Das Verlöbnis. Geschichte eines Knaben, Der Biberpelz. Eine Diebskomödie, Der einsame Weg. Schauspiel in fünf Akten, Einsame Menschen. Drama, Neue Freie Presse}
\section[ Paul Goldmann an Arthur Schnitzler, 6. 12. {[}1901{]}]{Paul Goldmann an Arthur Schnitzler, 6. 12. {[}1901{]}}
\nopagebreak\mylabel{v}
\rehead{ }\normalsize\beginnumbering\briefempfaengerindex{Schnitzler, Arthur@\textsc{Schnitzler, Arthur}!zzzGoldmann, Paul@\emph{von Paul Goldmann}!1901-12-061@{6. 12. {[}1901{]}}|(be}
\toendnotes[C]{\smallbreak\pagebreak[2]}\Standort{DLA, A:Schnitzler, HS.NZ85.1.3171.}
\physDesc{Brief, 1 Blatt, 2 Seiten
\newline{}Handschrift: blaue Tinte, deutsche Kurrent
\newline{}Schnitzler: 1) mit Bleistift das Jahr »{[}1{]}901.« vermerkt  2) mit rotem Buntstift zwei Unterstreichungen}\toendnotes[C]{\smallbreak}
\pstart
           \noindent{}\raggedleft{}{\pb}\textcolor{pink}{\textcolor{gray}{\textbf{DESSAUERSTRASSE 19}}}{}\ledrightnote{\textcolor{pink}{Dessauer Straße}}\pend
           
\pstart
           \textcolor{pink}{Berlin}{}\ledrightnote{\textcolor{pink}{Berlin}}, 6. Dezember.\pend
           
\pstart\center{}Mein lieber Freund,\pend
\pstart
           Ich freue mich ſehr, daß Dir mein \label{K_L03094-2v}\edtext{\textcolor{green}{Feuilleton}{}\ledrightnote{{$\rightarrow$}\textcolor{green}{Berliner Theater. »Der Rothe Hahn.«}}}{\lemma{\textnormal{\emph{Feuilleton}}}\Cendnote{\textnormal{\textcolor{blue}{Paul Goldmann}: \emph{\textcolor{green}{Berliner Theater. »Der Rothe Hahn.«}}. In: \emph{\textcolor{green}{Neue Freie Presse}}, Nr. 13.391, 4. 12. 1901, Morgenblatt, S. 1–3.}}}\label{K_L03094-2h}
               gefallen hat, und danke Dir für Deine lieben Worte. Nur ſehe ich nicht ein, warum Du
               in meinem \label{K_L03094-1v}\edtext{\textcolor{green}{Feuilleton}{}\ledrightnote{{$\rightarrow$}\textcolor{green}{Berliner Theater. »Einsame Menschen« im Deutschen Theater}} über »\textcolor{green}{Einſame Menſchen}{}\ledrightnote{\textcolor{green}{Einsame Menschen. Drama}}«}{\lemma{\textnormal{\emph{Feuilleton … Menſchen«}}}\Cendnote{\textnormal{\textcolor{blue}{Paul Goldmann}: \emph{\textcolor{green}{Berliner Theater. »Einsame Menschen« im Deutschen Theater}}.
                     In: \emph{\textcolor{green}{Neue Freie Presse}}, Nr. 13.345, 19. 10. 1901, Morgenblatt, S. 1–3. Vgl. Paul Goldmann an Arthur Schnitzler, 9. 11. [1901].}}}\label{K_L03094-1h} meinen Ton
               mißbilligt haſt, da \strikeout{\textcolor{gray}{in}} in meinem letzten \textcolor{green}{Feuilleton}{}\ledrightnote{{$\rightarrow$}\textcolor{green}{Berliner Theater. »Der Rothe Hahn.«}} der Ton genau derſelbe iſt. Und daß ich im »\textcolor{green}{Biberpelz}{}\ledrightnote{\textcolor{green}{Der Biberpelz. Eine Diebskomödie}}« Einiges \textcolor{green}{anerkannt}{}\ledrightnote{{$\rightarrow$}\textcolor{green}{Berliner Theater. »Der Rothe Hahn.«}} habe, liegt daran, daß der »\textcolor{green}{Biberpelz}{}\ledrightnote{\textcolor{green}{Der Biberpelz. Eine Diebskomödie}}« Gutes enthält, das anzuerkennen iſt, »\textcolor{green}{Einſame Menſchen}{}\ledrightnote{\textcolor{green}{Einsame Menschen. Drama}}« aber nicht das Mindeſte.\pend
           
\pstart
           Wann werde ich Dir wieder ausführlich ſchreiben können? Ich weiß an Arbeit nicht ein
               noch aus.\pend
           
\pstart
           Das \label{K_L03094-55v}\edtext{\textcolor{green}{Buch}{}\ledrightnote{{$\rightarrow$}\textcolor{green}{Das Verlöbnis. Geschichte eines Knaben}} von \textsc{\textcolor{blue}{Frisch}{}\ledrightnote{\textcolor{blue}{Efraim Frisch}}}}{\lemma{\textnormal{\emph{Buch von Frisch}}}\Cendnote{\textnormal{Im \emph{\textcolor{green}{Börsenblatt für den deutschen Buchhandel}} ist das Erscheinen von \textcolor{blue}{Efraim Frisch}s \emph{\textcolor{green}{Das Verlöbnis. Geschichte eines Knaben}} am 1. 11. 1901 bei \emph{\textcolor{brown}{S.
                        Fischer}} angezeigt.}}}\label{K_L03094-55h}
                bringſt Du mir wohl nach \textcolor{pink}{Berlin}{}\ledrightnote{\textcolor{pink}{Berlin}} mit?\pend
           
\pstart
           Der gewiſſe \label{K_L03094-112v}\edtext{Herr \textsc{\textcolor{blue}{Krügler}{}\ledrightnote{\textcolor{blue}{Krügler}}}}{\lemma{\textnormal{\emph{Herr Krügler}}}\Cendnote{\textnormal{nicht ermittelt; eventuell steht
                        \textcolor{blue}{Schnitzler}s Besuch beim \emph{\textcolor{brown}{Fremdenblatt}} am 1. 12. 1901 damit in
                     Zusammenhang?}}}\label{K_L03094-112h}
                iſt ſehr {\pb}gleichgiltig. Er wird den \label{K_L03094-43v}\edtext{Stoff}{\lemma{\textnormal{\emph{Stoff}}}\Cendnote{\textnormal{vermutlich von \emph{\textcolor{green}{Der einsame
                     Weg}}}}}\label{K_L03094-43h} anderes behandelt haben, als Du, – deſſen kannſt Du ſicher ſein. Kommt es zu
               einer öffentlichen Diskuſſion, ſo bin ich Zeuge, daß Du mir den Stoff bereits vor
               zwei Jahren erzählt haſt.\pend
           
\pstart
           Mittwoch war ich bei Frau \textsc{\textcolor{blue}{Fulda}{}\ledrightnote{\textcolor{blue}{Ida d’Albert}}}. Sie war außergewöhnlich \label{K_L03094-44v}\edtext{entzückt}{\lemma{\textnormal{\emph{entzückt}}}\Cendnote{\textnormal{Bezug unklar}}}\label{K_L03094-44h} von
               Dir und ſagte, daß ſie Dich ſehr lieb hat.\pend
           
\pstart
           \label{K_L03094-32v}\edtext{Wann kommſt du?}{\lemma{\textnormal{\emph{Wann kommſt du?}}}\Cendnote{\textnormal{siehe Paul Goldmann an Arthur Schnitzler, 4. 12. [1901]}}}\label{K_L03094-32h}\pend
           
\pstart
           Grüße die \textcolor{blue}{Mädels}{}\ledrightnote{{$\rightarrow$}\textcolor{blue}{Olga Schnitzler}{\newline}{$\rightarrow$}\textcolor{blue}{Elisabeth Steinrück}} und ſei ſelbſt vielmals und herzlichſt gegrüßt von {\\[\baselineskip]}Deinem {\\[\baselineskip]}\spacefill\mbox{Paul Goldmn}\pend
           \leftskip=0em{}\endnumbering\briefempfaengerindex{Schnitzler, Arthur@\textsc{Schnitzler, Arthur}!zzzGoldmann, Paul@\emph{von Paul Goldmann}!1901-12-061@{6. 12. {[}1901{]}}|)be}\mylabel{h}
\begin{anhang}
\end{anhang}\normalsize

\doendnotes{C}
\bigskip
\vfill

\clearpage

\footnotesize

\lohead{\textsc{register}}

% Definiere theindex-Environment komplett neu ohne reledmac
\makeatletter
\renewenvironment{theindex}{%
  \section*{\indexname}%
  \setlength{\parindent}{0pt}%
  \setlength{\parskip}{0pt plus 0.3pt}%
  \let\item\@idxitem
}{%
  \clearpage
}
\makeatother

\IfFileExists{\jobname-pw.ind}{\input{\jobname-pw.ind}}{}

\end{document}

      