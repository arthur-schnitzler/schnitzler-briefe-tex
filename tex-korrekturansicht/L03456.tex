%% latex-korrekturansicht-vorspann.tex
%% Vorspann für die Korrekturansicht.
%% Lädt die gemeinsame Datei latex-vorspann.tex mit gesetztem Schalter.

\newif\ifkorrekturansicht
\korrekturansichttrue

\input{../tex-inputs/latex-vorspann}


\renewcommand{\erwaehntePersonen}{Personen: Richard Beer-Hofmann, Max Dreyer, Paul Goldmann}
\renewcommand{\erwaehnteInstitutionen}{Institutionen: Berliner Tageblatt, Kleines Theater}
\renewcommand{\erwaehnteOrte}{Orte: Berlin, Dessauer Straße, Lessing-Theater, Wien}
\renewcommand{\erwaehnteWerke}{Werke: Arthur Schnitzlers »Haus Delorme«, Berliner Tageblatt, Das Haus Delorme. Eine Familienszene, Der grüne Kakadu. Groteske in einem Akt, Der tapfere Cassian. Puppenspiel in einem Akt, Die Siebzehnjährige, Haus Delorme. (Eine Richtigstellung von Arthur Schnitzler.), Neue Freie Presse, Schnitzlers »Haus Delorme«, Tagebuch, Theater- und Kunstnachrichten [Die Siebzehnjährige]}
\section[ Paul Goldmann an Arthur Schnitzler, 18. 11. {[}1904{]}]{Paul Goldmann an Arthur Schnitzler, 18. 11. {[}1904{]}}
\nopagebreak\mylabel{v}
\rehead{ }\normalsize\beginnumbering\briefempfaengerindex{Schnitzler, Arthur@\textsc{Schnitzler, Arthur}!zzzGoldmann, Paul@\emph{von Paul Goldmann}!1904-11-181@{18. 11. {[}1904{]}}|(be}
\toendnotes[C]{\smallbreak\pagebreak[2]}\Standort{DLA, A:Schnitzler, HS.NZ85.1.3174.}
\physDesc{Brief, 1 Blatt, 2 Seiten, 670 Zeichen
\newline{}Handschrift: blaue Tinte, deutsche Kurrent
\newline{}Schnitzler: 1) mit Bleistift das Jahr »904« vermerkt  2) mit rotem Buntstift eine Unterstreichung}\toendnotes[C]{\smallbreak}
\pstart
           \noindent{}\raggedleft{}{\pb}\textcolor{gray}{\textbf{\textcolor{pink}{DESSAUERSTRASSE 19}{}\ledrightnote{\textcolor{pink}{Dessauer Straße}}}}\pend
           
\pstart
           \textcolor{pink}{Berlin}{}\ledrightnote{\textcolor{pink}{Berlin}}, \substVorne{}\textsuperscript{2}\substDazwischen{}1\substHinten{}8. November.\pend
           
\pstart\center{}Mein lieber Freund,\pend
\pstart
           Ich \strikeout{\textcolor{gray}{×}} danke Dir für Deinen Brief und werde mich ſehr freuen, Dich \label{K_L03456-1v}\edtext{bald zu ſehen}{\lemma{\textnormal{\emph{bald zu ſehen}}}\Cendnote{\textnormal{\textcolor{blue}{Schnitzler} war seit 13. 11. 1904 in \textcolor{pink}{Berlin}. Am \emph{\textcolor{brown}{Kleinen Theater}} stand die Uraufführung von \emph{\textcolor{green}{Der tapfere Cassian}} und \emph{\textcolor{green}{Das Haus Delorme}} bevor, dazu sollte \emph{\textcolor{green}{Der
                     grüne Kakadu}} neu gegeben werden. Kurzfristig wurde \emph{\textcolor{green}{Das Haus Delorme}} noch vom Programm genommen, die beiden
                  anderen Stücke wurden erstmals am 22. 11. 1904 aufgeführt. Zu einem Treffen \textcolor{blue}{Schnitzlers} und \textcolor{blue}{Goldmann}s kam es am Montag, dem 21. 11. 1904 doch –
                  anders als hier \textcolor{blue}{Goldmann} vorgeschlagen –
                  vermutlich ohne den ebenfalls in \textcolor{pink}{Berlin}
                  weilenden \textcolor{blue}{Richard Beer-Hofmann}. Am 23. 11. 1904, dem Tag
                  nach der Aufführung, sahen sich die beiden erneut. An diesem Tag dürften sie
                  gemeinsam eine Reaktion auf eine \textcolor{green}{Meldung über die Absetzung von \emph{\textcolor{green}{Das Haus Delorme}}} verfasst haben, vgl. [O. V.]: \emph{\textcolor{green}{Schnitzlers »Haus Delorme«}}. In: \emph{\textcolor{green}{Berliner Tageblatt und -Handelszeitung}}, Jg. 33, Nr. 595, 22. 11. 1904, Abend-Ausgabe, S. 2. Im \emph{\textcolor{green}{Tagebuch}} erwähnte \textcolor{blue}{Schnitzler} die Meldung als »infame Notiz«
                     (22. 11. 1904).
                  Der mit Bleistift abgefasste Text ist aus der Perspektive \textcolor{blue}{Schnitzlers} verfasst, wurde aber von \textcolor{blue}{Goldmann}s Hand niedergeschrieben. Zumindest eine Korrektur
                     (»die Meldung von Seite der Cenſur«) wurde von \textcolor{blue}{Schnitzler} vorgenommen, auch die letzten drei Worte stammen
                  von ihm. Das Blatt mit dem Text findet sich heute gemeinsam mit dem vorliegenden
                  Brief im Nachlass \textcolor{blue}{Schnitzler}s: »{\pb}\strikeout{\textcolor{gray}{E}{ }}Sehr geehrte \textcolor{brown}{Redaktion}, Geſtatten Sie mir, zur Richtigſtellung der
                           \textcolor{green}{Meldungen}, die
                        Sie geſtern bezüglich \strikeout{d} meines noch unveröffentlichten Einakters
                           ›\textcolor{green}{Das Haus \textsc{Delorme}}‹ publizirt haben, Ihnen Folgendes mitzutheilen: \strikeout{Es iſt \textcolor{gray}{manc}he} Es entſpricht
                        nicht den Thatſachen, daß die Schauſpieler ſich geweigert haben, \strikeout{daß} das \textcolor{green}{Stück} zu ſpielen. Freitag war noch Probe. \strikeout{Abends infolge die das Cenſur-} Am Freitag{ }Abend, vor der auf Sonnabend
                        angeſetzten Generalprobe, \strikeout{\textcolor{gray}{er}} erfolgte \substVorne{}\textsuperscript{das Cenſurverbot}{\allowbreak}\substDazwischen{}die Meldung von Seiten der Cenſur\substHinten{}. Nur aus dieſem Grunde wurde das \textcolor{green}{Stück} abgeſetzt. Der Inhalt des \textcolor{green}{Stück}es iſt in \strikeout{der Ihrem \textcolor{green}{Blatte}} Ihrem \textcolor{green}{Berichte} unrichtig wiedergegeben.{ / }Mit vorzgl Hoch« Abgeschickt wurde dieses Protestschreiben aller Wahrscheinlichkeit nach
                  nicht. Am 24. 11. 1904 war \textcolor{blue}{Schnitzler} wieder in \textcolor{pink}{Wien} und gab zwei \textcolor{green}{Interviews} zur Causa (A. S.: \emph{»Das Zeitlose ist von kürzester Dauer«}, [Ludwig Klinenberger]: Arthur Schnitzlers »Haus Delorme«, 25. 11. 1904 und A. S.: \emph{»Das Zeitlose ist von kürzester Dauer«}, [Marco Brociner]: Haus Delorme. (Eine Richtigstellung von Arthur Schnitzler), 25. 11. 1904). \textcolor{blue}{Schnitzlers} hier getätigten Aussagen wurden
                  am 26. 11. 1904 im \emph{\textcolor{green}{Berliner
                     Tageblatt}} aufgegriffen, zugleich wurde auf der eigenen Darstellung
                  beharrt.}}}\label{K_L03456-1h}. Samſtag{ }zwiſchen 6 und 7 bitte ich Dich nicht zu kommen. Ich muß
                  Abends ins \textcolor{pink}{Theater}{}\ledrightnote{{$\rightarrow$}\textcolor{pink}{Lessing-Theater}} (\label{K_L03456-2v}\edtext{\textsc{\textcolor{blue}{\textcolor{green}{Dreyer}{}\ledrightnote{{$\rightarrow$}\textcolor{green}{Die Siebzehnjährige}}}{}\ledrightnote{\textcolor{blue}{Max Dreyer}}}}{\lemma{\textnormal{\emph{Dreyer}}}\Cendnote{\textnormal{Die Uraufführung von \textcolor{blue}{Max Dreyer}s \emph{\textcolor{green}{Die
                  Siebzehnjährige}} fand am 20. 11. 1904 am \textcolor{pink}{Berlin}er \emph{\textcolor{brown}{Lessing-Theater}} statt. \textcolor{blue}{Goldmann}
                  nahm vermutlich an der Generalprobe teil.}}}\label{K_L03456-2h}) und muß gerade in dieſer Stunde
               meine \label{K_L03456-3v}\edtext{\textcolor{green}{Telegramme}{}\ledrightnote{{$\rightarrow$}\textcolor{green}{Theater- und Kunstnachrichten [Die Siebzehnjährige]}}}{\lemma{\textnormal{\emph{Telegramme}}}\Cendnote{\textnormal{[\textcolor{blue}{Paul Goldmann}]: \emph{\textcolor{green}{Theater- und Kunstnachrichten}}. In: \emph{\textcolor{green}{Neue Freie Presse}}, Nr. 14.455, 20. 11. 1904, Morgenblatt, S. 12. Für welche weiteren
                  Zeitungen \textcolor{blue}{Goldmann} Theatertelegramme
                  schrieb, wie die Mehrzahlform »Telegramme« hier wohl zu verstehen
                  ist, ist nicht geklärt.}}}\label{K_L03456-3h} raſch fertigſtellen. {\pb}Sonntag bin ich leider auch nicht frei, – wohl aber
                  Montag{ }Abend. Ich habe heut mit \textsc{\textcolor{blue}{Richard}{}\ledrightnote{\textcolor{blue}{Richard Beer-Hofmann}}} telephoniſch ein Beiſammenſein für Montag{ }Abend verabredet, und es wäre ſehr ſchön, wenn Du auch dabei ſein
               könnteſt. Geht das nicht, ſo triffſt Du mich jedenfalls Montag{ }zwiſchen 6 u. 7 Uhr{ }\textcolor{pink}{zu Hauſe}{}\ledrightnote{{$\rightarrow$}\textcolor{pink}{Dessauer Straße}}. Oder, wenn Du mir
               ſagen kannſt, wo ich Dich um 5 Uhr treffen kann, komme ich auch zu
               Dir.\pend
           
\pstart
           Herzlichſt {\\[\baselineskip]}Dein {\\[\baselineskip]}\spacefill\mbox{Paul Goldmann.}\pend
           \leftskip=0em{}\endnumbering\briefempfaengerindex{Schnitzler, Arthur@\textsc{Schnitzler, Arthur}!zzzGoldmann, Paul@\emph{von Paul Goldmann}!1904-11-181@{18. 11. {[}1904{]}}|)be}\mylabel{h}  \normalsize

\doendnotes{C}
\bigskip
\vfill

\clearpage

\footnotesize

\lohead{\textsc{register}}

% Definiere theindex-Environment komplett neu ohne reledmac
\makeatletter
\renewenvironment{theindex}{%
  \section*{\indexname}%
  \setlength{\parindent}{0pt}%
  \setlength{\parskip}{0pt plus 0.3pt}%
  \let\item\@idxitem
}{%
  \clearpage
}
\makeatother

\IfFileExists{\jobname-pw.ind}{\input{\jobname-pw.ind}}{}

\end{document}

      