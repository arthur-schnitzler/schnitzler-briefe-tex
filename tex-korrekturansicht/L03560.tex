%% latex-korrekturansicht-vorspann.tex
%% Vorspann für die Korrekturansicht.
%% Lädt die gemeinsame Datei latex-vorspann.tex mit gesetztem Schalter.

\newif\ifkorrekturansicht
\korrekturansichttrue

\input{../tex-inputs/latex-vorspann}


\renewcommand{\erwaehntePersonen}{Personen:  ?? [Haushaltshilfe der Familie Salten 1912], Samuel Fischer, Hedwig Fischer, Richard Kralik, Felix Salten, Ottilie Salten, Olga Schnitzler, Emil Schwarz, Elisabeth Steinrück, Julius Ferdinand Wollf, Johanna Sophie Wollf}
\renewcommand{\erwaehnteInstitutionen}{Institutionen: Burgtheater}
\renewcommand{\erwaehnteOrte}{Orte: Berghof, Cottagegasse, Dresden, München, Sanatorium Hera, Unterach am Attersee, Wien}
\renewcommand{\erwaehnteWerke}{}
\section[ Felix Salten an Olga Schnitzler, 2. 9. 1912]{Felix Salten an Olga Schnitzler, 2. 9. 1912}
\nopagebreak\mylabel{v}
\rehead{ }\normalsize\beginnumbering\briefempfaengerindex{Schnitzler, Olga@\textsc{Schnitzler, Olga}!zzzSalten, Felix@\emph{von Felix Salten}!1912-09-022@{2. 9. 1912}|(be}
\toendnotes[C]{\smallbreak\pagebreak[2]}\Standort{CUL, Schnitzler, B 89, B 2.}
\physDesc{Karte, 1862 Zeichen
\newline{}Handschrift: schwarze Tinte, lateinische Kurrent
\newline{}Ordnung: mit Bleistift von unbekannter Hand nummeriert: »274« }\toendnotes[C]{\smallbreak}
\pstart
           \raggedleft{}{\pb}\textcolor{pink}{Berghof}{}\ledrightnote{\textcolor{pink}{Berghof}}, 2. IX. 12.\pend
           
\pstart{}Verehrte, liebe Frau Olga,\pend
\pstart
           vielen Dank für den lieben Brief und für \label{K_L03560-1v}\edtext{\textcolor{blue}{Arthur}{}\ledrightnote{}s Karten}{\lemma{\textnormal{\emph{Arthurs Karten}}}\Cendnote{\textnormal{nicht erhalten}}}\label{K_L03560-1h}. Wir haben eine ziemlich unruhige Zeit
               noch nicht ganz hinter uns. \textcolor{blue}{Wollfs}{}\ledrightnote{\textcolor{blue}{Julius Ferdinand Wollf}{\newline}\textcolor{blue}{Johanna Sophie Wollf}}
               aus \textcolor{pink}{Dresden}{}\ledrightnote{\textcolor{pink}{Dresden}} sind drei Wochen lang bei uns gewesen
               und wir haben uns sehr mit Ihnen gefreut. Wir konnten nur deshalb zu keinem ganzen
               Behagen kommen, weil es fast unaufhörlich geregnet hat, und weil \textcolor{blue}{Otti}{}\ledrightnote{\textcolor{blue}{Ottilie Salten}} mit ihrer Gesundheit nicht ganz in Ordnung war. Nun ist
               sie seit Mittwoch in \textcolor{pink}{Wien}{}\ledrightnote{\textcolor{pink}{Wien}}, im Sanatorium »\textcolor{pink}{Hera}{}\ledrightnote{\textcolor{pink}{Sanatorium Hera}}«, und hat am
                  Donnerstag eine kleine Operation überstanden. Es
               ist alles sehr gut gegangen: sie befindet sich schon viel besser und es ist möglich,
               dass Sie übermorgen oder Donnerstag schon wieder \textcolor{pink}{hier}{}\ledrightnote{{$\rightarrow$}\textcolor{pink}{Unterach am Attersee}} sein wird. Bei alledem – angenehm ist sowas ja nie, weder für \textcolor{blue}{Otti}{}\ledrightnote{\textcolor{blue}{Ottilie Salten}}, die allein, nur vom \label{K_L03560-2v}\edtext{\textcolor{blue}{Stubenmädchen}{}\ledrightnote{{$\rightarrow$}\textcolor{blue}{?? [Haushaltshilfe der Familie Salten 1912]}}}{\lemma{\textnormal{\emph{Stubenmädchen}}}\Cendnote{\textnormal{nicht ermittelt}}}\label{K_L03560-2h} begleitet, in \textcolor{pink}{Wien}{}\ledrightnote{\textcolor{pink}{Wien}} sein muß, noch für mich, der hier nur warten
               und sonst nichts nützliches für sie tun kann. Vielleicht haben wir hier noch ein paar
               Wochen Zeit, dass \textcolor{blue}{Otti}{}\ledrightnote{\textcolor{blue}{Ottilie Salten}}{ }{\pb}sich erholen kann. Ohnehin
               graut uns ein bischen vor dem \label{K_L03560-3v}\edtext{Umzug}{\lemma{\textnormal{\emph{Umzug}}}\Cendnote{\textnormal{nicht eruiert; eventuell
                  handelte es sich um eine größere Wohnung im selben \textcolor{pink}{Haus}?}}}\label{K_L03560-3h} in \textcolor{pink}{Wien}{}\ledrightnote{\textcolor{pink}{Wien}}, vor allen Geschichten, die wir mit dem \textcolor{pink}{Haus}{}\ledrightnote{{$\rightarrow$}\textcolor{pink}{Cottagegasse}}, den Möbeln, den Handwerkern und
               zunächst mit dem \label{K_L03560-4v}\edtext{\textcolor{blue}{Hausherr}{}\ledrightnote{{$\rightarrow$}\textcolor{blue}{Emil Schwarz}}n}{\lemma{\textnormal{\emph{Hausherrn}}}\Cendnote{\textnormal{\textcolor{blue}{Emil Schwarz}}}}\label{K_L03560-4h} haben werden, der mich wieder und immer wieder zu schröpfen sucht.\pend
           
\pstart
           Ich freu mich \uline{sehr}, dass es Ihrer \textcolor{blue}{Schwester}{}\ledrightnote{{$\rightarrow$}\textcolor{blue}{Elisabeth Steinrück}} gut geht. Bitte, grüßen Sie sie
               vielmals von \textcolor{blue}{uns}{}\ledrightnote{{$\rightarrow$}\textcolor{blue}{Ottilie Salten}}! Haben Sie
               nun \label{K_L03560-5v}\edtext{in \textcolor{pink}{München}{}\ledrightnote{\textcolor{pink}{München}} Ihre Konzertreise zusa{\geminationm}engestellt}{\lemma{\textnormal{\emph{in … zusammengestellt}}}\Cendnote{\textnormal{Gemeint war wohl der
                  kurze Zwischenstopp in \textcolor{pink}{München} am 29. 8. 1912. Zu einer
                  Konzertreise kam es nicht.}}}\label{K_L03560-5h}? Ich bin sehr neugierig darauf, und wüßte gern,
               wann und wohin Sie gehen. Jedenfalls werde ich Sie aber doch gewiss vorher noch
               singen hören, was ich mir lebhaft wünsche, und möchte, wenn Sie’s gestatten, auch Ihr
               Progamm als Privatkonzert zu hören bekommen. Ich bin jetzt so ziemlich sicher, dass
               Sie an Ihrer Wirkung Freude haben werden, wenn Sie wieder öffentlich singen.\pend
           
\pstart
           Was haben Sie dazu gesagt, dass Herr \textcolor{blue}{v. Kralik}{}\ledrightnote{\textcolor{blue}{Richard Kralik}}
               für das \textcolor{brown}{Burgtheater}{}\ledrightnote{\textcolor{brown}{Burgtheater}} kandidirt wird?
               Symptomatisch!\pend
           
\pstart
           \textcolor{gray}{Viele} herzliche Grüße von uns allen, ebenso von \textcolor{blue}{Fischers}{}\ledrightnote{\textcolor{blue}{Samuel Fischer}{\newline}\textcolor{blue}{Hedwig Fischer}}.\pend
           \pstart Aufrichtig Ihr \spacefill\mbox{Felix Salten}\pend{}\endnumbering\briefempfaengerindex{Schnitzler, Olga@\textsc{Schnitzler, Olga}!zzzSalten, Felix@\emph{von Felix Salten}!1912-09-022@{2. 9. 1912}|)be}\mylabel{h}  \normalsize

\doendnotes{C}
\bigskip
\vfill

\clearpage

\footnotesize

\lohead{\textsc{register}}

% Definiere theindex-Environment komplett neu ohne reledmac
\makeatletter
\renewenvironment{theindex}{%
  \section*{\indexname}%
  \setlength{\parindent}{0pt}%
  \setlength{\parskip}{0pt plus 0.3pt}%
  \let\item\@idxitem
}{%
  \clearpage
}
\makeatother

\IfFileExists{\jobname-pw.ind}{\input{\jobname-pw.ind}}{}

\end{document}

      