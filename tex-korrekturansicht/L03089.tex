%% latex-korrekturansicht-vorspann.tex
%% Vorspann für die Korrekturansicht.
%% Lädt die gemeinsame Datei latex-vorspann.tex mit gesetztem Schalter.

\newif\ifkorrekturansicht
\korrekturansichttrue

\input{../tex-inputs/latex-vorspann}


\renewcommand{\erwaehntePersonen}{Personen: Otto Brahm, Olga Schnitzler, Elisabeth Steinrück}
\renewcommand{\erwaehnteInstitutionen}{Institutionen: Deutsches Theater Berlin, Lessing-Theater, Morgenpost}
\renewcommand{\erwaehnteOrte}{Orte: Berlin, Dessauer Straße, Deutsches Theater Berlin, Wien}
\renewcommand{\erwaehnteWerke}{Werke: Arthur Schnitzler [Lebendige Stunden am Deutschen Theater Berlin], Berliner Morgenpost, Kleine Chronik. [Berliner Theater.], Lebendige Stunden. Vier Einakter, Neue Freie Presse}
\section[ Paul Goldmann an Arthur Schnitzler, 18. 10. {[}1901{]}]{Paul Goldmann an Arthur Schnitzler, 18. 10. {[}1901{]}}
\nopagebreak\mylabel{v}
\rehead{ }\normalsize\beginnumbering\briefempfaengerindex{Schnitzler, Arthur@\textsc{Schnitzler, Arthur}!zzzGoldmann, Paul@\emph{von Paul Goldmann}!1901-10-182@{18. 10. {[}1901{]}}|(be}
\toendnotes[C]{\smallbreak\pagebreak[2]}\Standort{DLA, A:Schnitzler, HS.NZ85.1.3171.}
\physDesc{Brief, 1 Blatt, 3 Seiten
\newline{}Handschrift: blaue Tinte, deutsche Kurrent
\newline{}Schnitzler: 1) mit Bleistift das Jahr »{[}1{]}901« vermerkt  2) mit rotem Buntstift eine Unterstreichung}\toendnotes[C]{\smallbreak}
\pstart
           \noindent{}\raggedleft{}{\pb}\textcolor{pink}{\textcolor{gray}{\textbf{DESSAUERSTRASSE 19}}}{}\ledrightnote{\textcolor{pink}{Dessauer Straße}}\pend
           
\pstart
           \textcolor{pink}{Berlin}{}\ledrightnote{\textcolor{pink}{Berlin}}, 18. Oktober.\pend
           
\pstart\center{}Mein lieber Freund,\pend
\pstart
           Das \label{K_L03089-1v}\edtext{\textcolor{green}{Telegramm}{}\ledrightnote{{$\rightarrow$}\textcolor{green}{Kleine Chronik. [Berliner Theater.]}}}{\lemma{\textnormal{\emph{Telegramm}}}\Cendnote{\textnormal{[\textcolor{blue}{Paul Goldmann}]: \emph{\textcolor{green}{Kleine Chronik.
                        [Berliner Theater.]}}. In: \emph{\textcolor{green}{Neue Freie
                        Presse}}, Nr. 13.342, 16. 10. 1901,
                     Abendblatt, S. 1. Darin wird von der Annahme von \emph{\textcolor{green}{Lebendige Stunden}} durch das \emph{\textcolor{brown}{Deutschen Theater Berlin}} berichtet. \textcolor{blue}{Otto Brahm} hatte keine Pressemitteilung verfasst, vgl. seinen Brief an
                     \textcolor{blue}{Schnitzler},
                     19. 10. 1901. (\emph{Der Briefwechsel
                        Arthur Schnitzler — Otto Brahm}. Vollständige Ausgabe. Herausgegeben,
                     eingeleitet und erläutert von Oskar Seidlin. Tübingen:
                        \emph{Niemeyer}{ }1975, S. 100–101.)}}}\label{K_L03089-1h} kommt von mir. Die
               Nachricht iſt der \label{K_L03089-2v}\edtext{»\textcolor{green}{\textcolor{green}{Berliner Morgenpoſt}{}\ledrightnote{\textcolor{green}{Berliner Morgenpost}}}{}\ledrightnote{{$\rightarrow$}\textcolor{green}{Arthur Schnitzler [Lebendige Stunden am Deutschen Theater Berlin]}}«}{\lemma{\textnormal{\emph{»Berliner Morgenpoſt«}}}\Cendnote{\textnormal{[O. V.]: \emph{\textcolor{green}{Arthur Schnitzler}}. In:
                        \emph{\textcolor{green}{Berliner Morgenpost}}, Jg. 4,
                     Nr. 243, 16. 10. 1901, S. 3.
               }}}\label{K_L03089-2h} entnommen, einem in Theater-Angelegenheiten gut unterrichteten \textcolor{brown}{Blatte}{}\ledrightnote{{$\rightarrow$}\textcolor{brown}{Morgenpost}}.\pend
           
\pstart
           \textcolor{blue}{Brahm}{}\ledrightnote{\textcolor{blue}{Otto Brahm}} iſt \label{K_L03089-3v}\edtext{blödſinnig}{\lemma{\textnormal{\emph{blödſinnig}}}\Cendnote{\textnormal{\textcolor{blue}{Otto Brahm} äußerte mehrfach Skepsis am
                  Zyklus \emph{\textcolor{green}{Lebendige Stunden}}, vgl. \emph{Der Briefwechsel Arthur Schnitzler — Otto Brahm}.
                     Vollständige Ausgabe. Herausgegeben, eingeleitet und erläutert von Oskar
                     Seidlin. Tübingen: \emph{Niemeyer}{ }1975, S. 91–100, insb. 99–100. Zur Uraufführung
                  der \textcolor{green}{Einakter} kam es dennoch
                  am 1. 4. 1902 im
                     \textcolor{pink}{Deutschen Theater Berlin}.}}}\label{K_L03089-3h}. Ich wußte
               wohl, daß er ein unkünſtleriſcher Direktor iſt. Aber das hatte ich nicht erwartet,
                  {\pb}Wenn er bei ſeiner Weigerung bleibt, ſo ziehſt
               Du einfach ſämmtliche \textcolor{green}{Stücke}{}\ledrightnote{{$\rightarrow$}\textcolor{green}{Lebendige Stunden. Vier Einakter}}
               zurück und gibſt ſie dem \textcolor{brown}{Leſſingtheater}{}\ledrightnote{\textcolor{brown}{Lessing-Theater}}. \strikeout{So} Das iſt ja wahrhaft ſkandalös!\pend
           
\pstart
           \strikeout{\textcolor{gray}{Mir} thut \textcolor{gray}{e}} Bitte, halte mich über den weiteren Verlauf der Angelegenheit auf dem
               Laufenden!\pend
           
\pstart
           Mir thut es leid, ſo ſelten und ſo wenig von {\pb}Dir zu
               hören.\pend
           
\pstart
           Viele Grüße an die beiden \textcolor{blue}{Mädchen}{}\ledrightnote{{$\rightarrow$}\textcolor{blue}{Olga Schnitzler}{\newline}{$\rightarrow$}\textcolor{blue}{Elisabeth Steinrück}} und an Dich! {\\[\baselineskip]}Dein {\\[\baselineskip]}\spacefill\mbox{Paul Goldmann.}\pend
           \leftskip=0em{}\endnumbering\briefempfaengerindex{Schnitzler, Arthur@\textsc{Schnitzler, Arthur}!zzzGoldmann, Paul@\emph{von Paul Goldmann}!1901-10-182@{18. 10. {[}1901{]}}|)be}\mylabel{h}  \normalsize

\doendnotes{C}
\bigskip
\vfill

\clearpage

\footnotesize

\lohead{\textsc{register}}

% Definiere theindex-Environment komplett neu ohne reledmac
\makeatletter
\renewenvironment{theindex}{%
  \section*{\indexname}%
  \setlength{\parindent}{0pt}%
  \setlength{\parskip}{0pt plus 0.3pt}%
  \let\item\@idxitem
}{%
  \clearpage
}
\makeatother

\IfFileExists{\jobname-pw.ind}{\input{\jobname-pw.ind}}{}

\end{document}

      