%% latex-korrekturansicht-vorspann.tex
%% Vorspann für die Korrekturansicht.
%% Lädt die gemeinsame Datei latex-vorspann.tex mit gesetztem Schalter.

\newif\ifkorrekturansicht
\korrekturansichttrue

\input{../tex-inputs/latex-vorspann}


\renewcommand{\erwaehntePersonen}{Personen: Olga Schnitzler, Heinrich Schnitzler}
\renewcommand{\erwaehnteOrte}{Orte: Berlin, Dessauer Straße, Wien}
\renewcommand{\erwaehnteWerke}{Werke: Die Zeit, Die Zeit. Wiener Wochenschrift}
\section[ Paul Goldmann an Arthur Schnitzler, 16. 9. {[}1902{]}]{Paul Goldmann an Arthur Schnitzler, 16. 9. {[}1902{]}}
\nopagebreak\mylabel{v}
\rehead{ }\normalsize\beginnumbering\briefempfaengerindex{Schnitzler, Arthur@\textsc{Schnitzler, Arthur}!zzzGoldmann, Paul@\emph{von Paul Goldmann}!1902-09-161@{16. 9. {[}1902{]}}|(be}
\toendnotes[C]{\smallbreak\pagebreak[2]}\Standort{DLA, A:Schnitzler, HS.NZ85.1.3172.}
\physDesc{Brief, 1 Blatt, 2 Seiten
\newline{}Handschrift: blaue Tinte, deutsche Kurrent
\newline{}Schnitzler: mit Bleistift das Jahr »{[}1{]}902« vermerkt }\toendnotes[C]{\smallbreak}
\pstart
           \noindent{}\raggedleft{}{\pb}\textcolor{pink}{\textcolor{gray}{\textbf{DESSAUERSTRASSE 19}}}{}\ledrightnote{\textcolor{pink}{Dessauer Straße}}\pend
           
\pstart
           \textcolor{pink}{Berlin}{}\ledrightnote{\textcolor{pink}{Berlin}}, 16. September.\pend
           
\pstart\center{}Mein lieber Freund,\pend
\pstart
           Erſt heut komme ich dazu, Dir für Deine lieben Karten
               und Brief zu danken. Ich habe hier eine tolle Arbeit vorgefunden. Das \label{K_L03224-1v}\edtext{bevorſtehende Erſcheinen der »\textcolor{green}{Zeit}{}\ledrightnote{\textcolor{green}{Die Zeit}}«}{\lemma{\textnormal{\emph{bevorſtehende … »Zeit«}}}\Cendnote{\textnormal{zusätzlich zur
                     \textcolor{green}{Wochenzeitung} erschien ab dem
                     27. 9. 1902 eine gleichnamige 
                     \textcolor{green}{Tageszeitung}. vgl. Paul Goldmann an Arthur Schnitzler und Olga
               Gussmann, 7. 7. [1901].
               }}}\label{K_L03224-1h} wird mein Pensum \strikeout{zu} wahrſcheinlich
               verdoppeln.\pend
           
\pstart
           Ich freue mich unendlich {\pb}zu hören, daß es Dir und
                  \textsc{\textcolor{blue}{Olga}{}\ledrightnote{\textcolor{blue}{Olga Schnitzler}}} ſowie Eurem \textcolor{blue}{Sohn}{}\ledrightnote{{$\rightarrow$}\textcolor{blue}{Heinrich Schnitzler}} gut
               geht und freue mich ganz beſonders über die Ausſicht, Dir \label{K_L03224-2v}\edtext{Anfang Oktober}{\lemma{\textnormal{\emph{Anfang Oktober}}}\Cendnote{\textnormal{siehe Paul Goldmann an Arthur Schnitzler, 2. [10. 1902]}}}\label{K_L03224-2h}{ }\textcolor{pink}{hier}{}\ledrightnote{{$\rightarrow$}\textcolor{pink}{Berlin}} die Hand drücken zu
               können.\pend
           
\pstart
           Schreib’ mir bald, – und nicht ſo kurz und ſo \strikeout{eilig}
               eilig, wie ich es thun muß.\pend
           
\pstart
           Viele treue Grüße {\\[\baselineskip]}Dein {\\[\baselineskip]}\spacefill\mbox{Paul Goldmn}\pend
           \leftskip=0em{}\endnumbering\briefempfaengerindex{Schnitzler, Arthur@\textsc{Schnitzler, Arthur}!zzzGoldmann, Paul@\emph{von Paul Goldmann}!1902-09-161@{16. 9. {[}1902{]}}|)be}\mylabel{h}
\begin{anhang}
\end{anhang}\normalsize

\doendnotes{C}
\bigskip
\vfill

\clearpage

\footnotesize

\lohead{\textsc{register}}

% Definiere theindex-Environment komplett neu ohne reledmac
\makeatletter
\renewenvironment{theindex}{%
  \section*{\indexname}%
  \setlength{\parindent}{0pt}%
  \setlength{\parskip}{0pt plus 0.3pt}%
  \let\item\@idxitem
}{%
  \clearpage
}
\makeatother

\IfFileExists{\jobname-pw.ind}{\input{\jobname-pw.ind}}{}

\end{document}

      