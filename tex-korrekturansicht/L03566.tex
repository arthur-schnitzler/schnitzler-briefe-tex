%% latex-korrekturansicht-vorspann.tex
%% Vorspann für die Korrekturansicht.
%% Lädt die gemeinsame Datei latex-vorspann.tex mit gesetztem Schalter.

\newif\ifkorrekturansicht
\korrekturansichttrue

\input{../tex-inputs/latex-vorspann}


\renewcommand{\erwaehntePersonen}{Personen: Adolf Oswald von Dubsky-Třembomyslic, Frieda Pollak, Felix Salten, Oscar Wettstein}
\renewcommand{\erwaehnteInstitutionen}{Institutionen: Ministerium für Äußeres}
\renewcommand{\erwaehnteOrte}{Orte: Hotel Imperial, Schweiz, Wien}
\renewcommand{\erwaehnteWerke}{}
\section[ Felix Salten an Arthur Schnitzler, 15. 5. 1917]{Felix Salten an Arthur Schnitzler, 15. 5. 1917}
\nopagebreak\mylabel{v}
\rehead{ }\normalsize\beginnumbering\briefempfaengerindex{Schnitzler, Arthur@\textsc{Schnitzler, Arthur}!zzzSalten, Felix@\emph{von Felix Salten}!1917-05-151@{15. 5. 1917}|(be}
\toendnotes[C]{\smallbreak\pagebreak[2]}\Standort{CUL, Schnitzler, B 89, B 2.}
\physDesc{Brief, 1 Blatt, 1 Seite, 502 Zeichen
\newline{}Handschrift: schwarze Tinte, lateinische Kurrent
\newline{}Schnitzler: 1) mit Bleistift Vermerk: »\textsc{Salten}«  2) mit rotem Buntstift eine Unterstreichung
\newline{}Ordnung: 1) mit Bleistift von \textcolor{blue}{Frieda Pollak} (?) mit
                                 dem Buchstaben »A« (Abgeschrieben/Abschrift)
                                 gekennzeichnet  2) mit Bleistift von unbekannter Hand nummeriert: »279«}\toendnotes[C]{\smallbreak}
\pstart
           \raggedleft{}{\pb}\textcolor{pink}{Wien}{}\ledrightnote{\textcolor{pink}{Wien}}, 15. 5. 17\pend
           
\pstart{}Lieber,\pend
\pstart
           in Ergänzung der Einladung zu dem Vortrag des \textcolor{pink}{Schweizer}{}\ledrightnote{\textcolor{pink}{Schweiz}} Regierungsrates \textcolor{blue}{Wettstein}{}\ledrightnote{\textcolor{blue}{Oscar Wettstein}}
               am Samstag habe ich es übernommen, Sie auch zu dem
               kleinen Souper zu bitten, das Samstag{ }Abd. ½ 9 im \textcolor{pink}{Hotel Imperial}{}\ledrightnote{\textcolor{pink}{Hotel Imperial}} für Herrn \textcolor{blue}{Wettstein}{}\ledrightnote{\textcolor{blue}{Oscar Wettstein}}
               gegeben wird. Es ist wirklich nur ein kleines Souper (ohne Toaste). Ihre frdl. Zusage
               bitte ich Sie, an den Grafen \textcolor{blue}{Adolf Dubsky}{}\ledrightnote{\textcolor{blue}{Adolf Oswald von Dubsky-Třembomyslic}} im
                  \textcolor{brown}{Ministerium des Äußeren}{}\ledrightnote{\textcolor{brown}{Ministerium für Äußeres}} richten zu wollen.
               Hoffentlich \label{K_L03566-1v}\edtext{kommen Sie}{\lemma{\textnormal{\emph{kommen Sie}}}\Cendnote{\textnormal{\textcolor{blue}{Schnitzler} kam nicht, vgl. Arthur Schnitzler an Felix Salten, 17. 5. 1917.}}}\label{K_L03566-1h} sowol zu dem Vortrag, wie zum Souper.\pend
           
\pstart
           Herzliche Grüße von Haus zu Haus {\\[\baselineskip]}Ihr {\\[\baselineskip]}\spacefill\mbox{Felix Salten}\pend
           \leftskip=0em{}\endnumbering\briefempfaengerindex{Schnitzler, Arthur@\textsc{Schnitzler, Arthur}!zzzSalten, Felix@\emph{von Felix Salten}!1917-05-151@{15. 5. 1917}|)be}\mylabel{h}  \normalsize

\doendnotes{C}
\bigskip
\vfill

\clearpage

\footnotesize

\lohead{\textsc{register}}

% Definiere theindex-Environment komplett neu ohne reledmac
\makeatletter
\renewenvironment{theindex}{%
  \section*{\indexname}%
  \setlength{\parindent}{0pt}%
  \setlength{\parskip}{0pt plus 0.3pt}%
  \let\item\@idxitem
}{%
  \clearpage
}
\makeatother

\IfFileExists{\jobname-pw.ind}{\input{\jobname-pw.ind}}{}

\end{document}

      