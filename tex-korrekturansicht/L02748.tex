%% latex-korrekturansicht-vorspann.tex
%% Vorspann für die Korrekturansicht.
%% Lädt die gemeinsame Datei latex-vorspann.tex mit gesetztem Schalter.

\newif\ifkorrekturansicht
\korrekturansichttrue

\input{../tex-inputs/latex-vorspann}


               \section[Paul Goldmann an Arthur Schnitzler, 23. 9. {[}1895{]}]{ Paul Goldmann an Arthur Schnitzler, 23. 9. {[}1895{]}}\nopagebreak\mylabel{v}\rehead{ }\normalsize\beginnumbering\briefempfaengerindex{Schnitzler, Arthur@\textsc{Schnitzler, Arthur}!zzzGoldmann, Paul@\emph{von Paul Goldmann}!1895-09-233@{23. 9. {[}1895{]}}|(be} \toendnotes[C]{\smallbreak\pagebreak[2]} \Standort{DLA, A:Schnitzler, HS.NZ85.1.3165.}
\physDesc{Brief, 3 Blätter, 11 Seiten
\newline{}Handschrift: blaue Tinte, deutsche Kurrent
\newline{}Schnitzler: 1) mit Bleistift das Jahr » 95« vermerkt 2) mit rotem Buntstift neun Unterstreichungen}\toendnotes[C]{\smallbreak}\pstart
           \noindent{}{\pb}\textcolor{gray}{\textbf{\textbf{\textcolor{brown}{Frankfurter Zeitung}{}\ledrightnote{\textcolor{brown}{Frankfurter Zeitung}}}}}\pend
           \pstart
           \textcolor{gray}{\textbf{(\textcolor{brown}{\begin{otherlanguage}{french}Gazette de Francfort\end{otherlanguage}}{}\ledrightnote{\textcolor{brown}{Frankfurter Zeitung}}). }}\pend
           \pstart
           \textcolor{gray}{\textbf{\textbf{\begin{otherlanguage}{french}Fondateur M. \textcolor{blue}{L.
                                 Sonnemann}{}\ledrightnote{\textcolor{blue}{Leopold Sonnemann}}\end{otherlanguage}.}}}\hfill \textsc{\textcolor{pink}{Paris}{}\ledrightnote{\textcolor{pink}{Paris}}}, 23. September.\pend
           \pstart
           \begin{otherlanguage}{french}\textcolor{gray}{\textbf{\textcolor{green}{Journal}{}\ledrightnote{→\textcolor{green}{Frankfurter Zeitung}} politique,
                        financier,}}\end{otherlanguage}\pend
           \pstart
           \begin{otherlanguage}{french}\textcolor{gray}{\textbf{commercial et littéraire.}}\end{otherlanguage}\pend
           \pstart
           \begin{otherlanguage}{french}\textcolor{gray}{\textbf{\textbf{Paraissant trois fois par jour.}}}\end{otherlanguage}\pend
           \pstart
           \begin{otherlanguage}{french}\textcolor{gray}{\textbf{\textbf{Bureau à \textcolor{pink}{Paris}{}\ledrightnote{\textcolor{pink}{Paris}}:}}}\end{otherlanguage}\pend
           \pstart
           \begin{otherlanguage}{french}\textcolor{gray}{\textbf{\textbf{\textcolor{pink}{24. Rue Feydeau}{}\ledrightnote{\textcolor{pink}{rue Feydeau}}.}}}\end{otherlanguage}\pend
           \pstart\center{}Mein lieber Freund,\pend\pstart
           Dein Brief beginnt mit allerlei Mißſtimmungs-Äußerungen, macht ſchlimme Erwartungen
               rege, – und ſchließlich kommt \strikeout{Gutes} Gutes, nichts als
               Gutes (unberufen!){[}.{]} Über das Ergebniß der \label{K_L02748-1v}\edtext{Leſeprobe}{\lemma{\textnormal{\emph{Leſeprobe}}}\Cendnote{\textnormal{für die Uraufführung der \emph{\textcolor{green}{Liebelei}} am \emph{\textcolor{brown}{Burgtheater}}, siehe A. S.: \emph{Tagebuch}, 18. 9. 1895}}}\label{K_L02748-1h} freue ich mich von Herzen, und ich glaube, es iſt Anlaß, Dich dazu zu
               beglückwünſchen. Die Haltung der großen \textcolor{blue}{Tragödin}{}\ledrightnote{→\textcolor{blue}{Adele Sandrock}} iſt luſtig zum Sich-Schütteln. Gewiß kann noch
               allerlei Tückiſches von dieſer Seite kommen – {\pb}aber,
               glaub’ mir, ſie kann nichts mehr verderben, ſie iſt im Grunde machtlos. Das ſcheint
               ſie übrigens ſelbſt zu ſpüren, denn ſonſt hätte ſie Dir nicht \label{K_L02748-2v}\edtext{telephoniſch gratulirt}{\lemma{\textnormal{\emph{telephoniſch gratulirt}}}\Cendnote{\textnormal{siehe A. S.: \emph{Tagebuch}, 18. 9. 1895}}}\label{K_L02748-2h}. Ein \label{K_L02748-3v}\edtext{von \textsc{\textcolor{blue}{Speidel}{}\ledrightnote{\textcolor{blue}{Ludwig Speidel}}} günſtig beurtheiltes \textcolor{green}{Stück}{}\ledrightnote{→\textcolor{green}{Liebelei. Schauspiel in drei Akten}}}{\lemma{\textnormal{\emph{von … Stück}}}\Cendnote{\textnormal{siehe A. S.: \emph{Tagebuch}, 9. 9. 1895}}}\label{K_L02748-3h} iſt doch eine verdammte Geſchichte. Davor muß ſelbſt \substVorne{}\textsuperscript{\textcolor{gray}{d}\textcolor{gray}{×}}\substDazwischen{}die\substHinten{} Luderhaftigkeit ſich beugen. \textsc{\textcolor{blue}{Speidel}{}\ledrightnote{\textcolor{blue}{Ludwig Speidel}}} hält ſich übrigens wacker. Bravo! Auch \textsc{\textcolor{blue}{Burckhardt}{}\ledrightnote{\textcolor{blue}{Max Eugen Burckhard}}s}{ }\label{K_L02748-4v}\edtext{Äußerungen über die Beſetzung von \textsc{\textcolor{green}{Anatol}{}\ledrightnote{\textcolor{green}{Anatol}}}}{\lemma{\textnormal{\emph{Äußerungen … Anatol}}}\Cendnote{\textnormal{Am 8. 9. 1895 schlug \textcolor{blue}{Max Burckhard}{ }\textcolor{blue}{Schnitzler} vor, er selbst solle den \textcolor{green}{Anatol} spielen, \textcolor{blue}{Hermann Bahr} den \textcolor{green}{Max} und \textcolor{blue}{Adele Sandrock} alle weiblichen Rollen.}}}\label{K_L02748-4h} ſind ein
               artiges Stück Comödie. Es iſt erſtaunlich, wie luſtig das Leben ſein kann, wenn {\pb}es will.\pend
           \pstart
           Wie Du ſchreiben kannſt, daß Du um ſieben Jahre zurück ſeieſt, iſt mir unklar. Gibt
               es etwa in der Literatur eine Studien- und Examen-Laufbahn, wie in der Jurisprudenz
               und Medicin? Je ſpäter man zu ſchreiben anfängt, umſomehr hat man vorher gelebt. Und
               wenn in den Werken mehr durchgelebtes Leben drin iſt, ſo iſt das ein Gewinn. Hier
               könnte man das \textsc{Paradoxon} machen, daß in der Literatur die
               verlorenen Semeſter gerade die gewonnenen ſind. Hätteſt Du vor ſieben Jahren {\pb}die »\textcolor{green}{Liebelei}{}\ledrightnote{\textcolor{green}{Liebelei. Schauspiel in drei Akten}}«
               ſchreiben können oder »\textcolor{green}{Sterben}{}\ledrightnote{\textcolor{green}{Sterben. Novelle}}«? Unmöglich,
               nicht wahr? Nun alſo!\pend
           \pstart
           In der \label{K_L02748-99v}\edtext{\textcolor{green}{Correſpondenz}{}\ledrightnote{→\textcolor{green}{Wiener Brief [Die neue Saison im Burgtheater]}}, die ich
                  meinte}{\lemma{\textnormal{\emph{Correſpondenz, … meinte}}}\Cendnote{\textnormal{siehe Paul Goldmann an Arthur Schnitzler, 12. 9. [1895]}}}\label{K_L02748-99h}, ſprach \textsc{\textcolor{blue}{Uhl}{}\ledrightnote{\textcolor{blue}{Friedrich Uhl}}} nicht von Dir. Er ſagte nur: Das \textcolor{brown}{Burgtheater}{}\ledrightnote{\textcolor{brown}{Burgtheater}} verſpreche eine Reihe von Novitäten; das ſei ſchön; er wolle
               abwarten und am Ende der Saiſon Abrechnung halten, ob die \textcolor{brown}{Direction}{}\ledrightnote{→\textcolor{brown}{Burgtheater}} alle Verſprechungen erfüllt.
               Damit ſpielte er wohl auch auf die bisherige Verzögerung der »\textcolor{green}{Liebelei}{}\ledrightnote{\textcolor{green}{Liebelei. Schauspiel in drei Akten}}« an, und ich meinte, {\pb}die Abrechnungs-Drohung ſei geeignet, weitere
               Verſchiebungs-Gelüſte etwas zu dämpfen.\pend
           \pstart
           Daß \label{K_L02748-7v}\edtext{\textsc{\textcolor{blue}{Herzl}{}\ledrightnote{\textcolor{blue}{Theodor Herzl}}} liebenswürdig}{\lemma{\textnormal{\emph{Herzl liebenswürdig}}}\Cendnote{\textnormal{siehe A. S.: \emph{Tagebuch}, 18. 9. 1895}}}\label{K_L02748-7h} iſt, iſt gut u. erſtaunt mich nicht. Ich rathe Dir dringend, ſeine Einladung
               anzunehmen und für die »\textcolor{brown}{Neue Fr. P.}{}\ledrightnote{\textcolor{brown}{Neue Freie Presse}}« \label{K_L02748-6v}\edtext{Feuilletons}{\lemma{\textnormal{\emph{Feuilletons}}}\Cendnote{\textnormal{\textcolor{blue}{Schnitzler} schrieb zu keinem Zeitpunkt
                  seines Lebens Feuilletons, trotz mehrfacher Angebote von verschiedenen
                  Seiten.}}}\label{K_L02748-6h} zu ſchreiben. Sehr nützlich – beſonders um \strikeout{\textcolor{gray}{nur glen}} gelegentlich einen beſſeren Verleger zu finden.\pend
           \pstart
           {\pb}Zur \textsc{Mad. \textcolor{blue}{Candiani}{}\ledrightnote{\textcolor{blue}{Regina Candiani}}} gehe ich demnächſt. Inzwiſchen hat mich die deutſche \textcolor{blue}{Frau}{}\ledrightnote{→\textcolor{blue}{[MMe. Georges] Aubry}} eines \textcolor{pink}{franzöſiſch}{}\ledrightnote{→\textcolor{pink}{Frankreich}}en \textcolor{blue}{Collegen}{}\ledrightnote{→\textcolor{blue}{Georges Aubry}} erſucht, ich möchte ihr etwas zum
               Überſetzen empfehlen. Ich habe ihr die »\textcolor{green}{Kleine
                  Komödie}{}\ledrightnote{\textcolor{green}{Die kleine Komödie}}« gegeben. Denn der betr. \textcolor{blue}{College}{}\ledrightnote{→\textcolor{blue}{Georges Aubry}} iſt an der »\textsc{\textcolor{brown}{Liberté}{}\ledrightnote{\textcolor{brown}{La Liberté}}}«, einem ſehr angeſehenen u. anſtändigen \textcolor{brown}{Blatte}{}\ledrightnote{→\textcolor{brown}{La Liberté}}, u. könnte vielleicht die \label{K_L02748-8v}\edtext{\textcolor{green}{Überſetzung}{}\ledrightnote{→\textcolor{green}{La petite comédie. Mœurs viennois}}}{\lemma{\textnormal{\emph{Überſetzung}}}\Cendnote{\textnormal{\textcolor{blue}{Arthur Schnitzler}: \emph{\textcolor{green}{La Petite comédie. Mœurs viennois}}. Übersetzt von \textcolor{blue}{Mme. Georges Aubry}. In: \emph{\textcolor{green}{La Liberté}}, Jg. 30, Nr. 11327, 19. 11. 1895 bis Nr. 11336, 28. 11. 1895 (acht Teile).}}}\label{K_L02748-8h} dort placiren. Als
               Zeitungs-Novelle ginge die \textcolor{green}{Geſchichte}{}\ledrightnote{→\textcolor{green}{Die kleine Komödie}} recht gut. Kriegen wirſt {\pb}Du
               natürlich nichts, aber es wäre recht hübſch, wenn etwas von Dir in einem \strikeout{franz}{ }\textcolor{pink}{Pariſ}{}\ledrightnote{\textcolor{pink}{Paris}}er \textcolor{green}{Tagesblatte}{}\ledrightnote{→\textcolor{green}{La Liberté}} erſchiene. Biſt Du ein verſtanden, ſo ſchreib\substVorne{}\textsuperscript{t}\substDazwischen{}e\substHinten{} mir einen Brief, gerichtet an \textsc{Madame \textcolor{blue}{Aubry}{}\ledrightnote{\textcolor{blue}{[MMe. Georges] Aubry}}} (dies der Name). »\begin{otherlanguage}{french}\textsc{\textcolor{blue}{Madame}{}\ledrightnote{→\textcolor{blue}{Georges Aubry}}, Je vous
                     autorise bien volontiers à traduire en francais ma nouvelle } »\textcolor{green}{Kleine Komödie}{}\ledrightnote{\textcolor{green}{Die kleine Komödie}}«\end{otherlanguage}, u. ſonſt etwas
               Verbindliches. Ich würde mich freuen, wenn der kleine Plan gelänge{\dotssix}\pend
           \pstart
           Die \textsc{\textcolor{blue}{Ida Fanjung}{}\ledrightnote{\textcolor{blue}{Ida Van-Jung}}} iſt hier und läßt Euch Alle grüßen. Eine große {\pb}Freude für mich. Mit ihrem offenen Character und ihrer Geradheit iſt ſie wie ein
               männlicher Freund. Freilich ganz unkünſtleriſch und ohne Feinheiten. Sie ſpürt, daß
               ſie unkünſtleriſch iſt, und iſt darum innerlich mit ſich zerfallen. Hätte wohl nicht
               zur Bühne gehen ſollen{\dotssix}\pend
           \pstart
           Lies \textsc{\textcolor{blue}{Rubinstein}{}\ledrightnote{\textcolor{blue}{Anton Rubinstein}}}: »\textcolor{green}{Die Muſik u. ihre Meiſter}{}\ledrightnote{\textcolor{green}{Die Musik und ihre Meister. Eine Unterredung}}«. Habe ſelten
               etwas ſo Geiſtreiches über Muſik geleſen, – wenn er auch \textsc{\textcolor{blue}{Wagner}{}\ledrightnote{\textcolor{blue}{Richard Wagner}}} nicht mag. Von »\label{K_L02748-88v}\edtext{\textcolor{green}{\textsc{Juliens} Tagebuch}{}\ledrightnote{\textcolor{green}{Julies Tagebuch. Roman}}}{\lemma{\textnormal{\emph{Juliens Tagebuch}}}\Cendnote{\textnormal{\textcolor{blue}{Peter Nansen}: \emph{\textcolor{green}{Julies Tagebuch. Roman}}. Autorisierte Übersetzung aus
                     dem Dänischen von \textcolor{blue}{Mathilde Mann}. In: \emph{\textcolor{green}{Neue Deutsche Rundschau}}, Jg. 6, Nr. 1,
                        Januar 1895, S. 11–38; Nr. 2, Februar 1895,
                     S. 116–143; Nr. 3, März 1895, S. 225–254. Im selben Jahr
                  erschien die Buchausgabe bei \emph{\textcolor{brown}{S. Fischer}}.
                     (Originalausgabe: \emph{\textcolor{green}{Julies Dagbog.
                        Roman}}, 1893)}}}\label{K_L02748-88h}« bin ich nicht gar ſo entzückt. {\pb}Ich mag die Bücher nicht, die thun, als ob es nichts in der Welt gäbe, als Liebe,
               und als ob das gar ſo wichtig ſei! Freilich, ein \textcolor{blue}{Mann}{}\ledrightnote{→\textcolor{blue}{Peter Nansen}} von großem Talent. Packt Einen aber nicht in den
               Tiefen.\pend
           \pstart
           Was Dir \textsc{\textcolor{blue}{Paul Schultz}{}\ledrightnote{\textcolor{blue}{Paul Schulz}}} geſagt, iſt die \label{K_L02748-11v}\edtext{officiöſe
                  Verſion}{\lemma{\textnormal{\emph{officiöſe
                  Verſion}}}\Cendnote{\textnormal{Am 17. 9. 1895 hatte sich \textcolor{blue}{Schnitzler} mit \textcolor{blue}{Paul Schulz} unterhalten und dabei erfahren, warum \textcolor{blue}{Berthold Frischauer} zum \textcolor{pink}{Paris}er Korrespondenten der \emph{\textcolor{brown}{Neuen Freien Presse}} in Nachfolge von \textcolor{blue}{Theodor Herzl} ernannt worden war.}}}\label{K_L02748-11h} u. eine alberne Lüge. Ich habe
               hier die Wahrheit gehört. Man hat mich nicht genommen aus verſchiedenen {\pb}perſönlichen Gründen, deren hauptſächlicher die alte
                  \label{K_L02748-111v}\edtext{Todfeindſchaft}{\lemma{\textnormal{\emph{Todfeindſchaft}}}\Cendnote{\textnormal{siehe Paul Goldmann an Arthur Schnitzler, 1. 5. [1894]}}}\label{K_L02748-111h} war zwiſchen meinem \textcolor{blue}{Onkel}{}\ledrightnote{→\textcolor{blue}{Fedor Mamroth}} und dem \textcolor{brown}{Blatte}{}\ledrightnote{→\textcolor{brown}{Neue Freie Presse}}{\dotsfive}\pend
           \pstart
           Meine Stimmung? Ich wünſchte, es wäre wieder Urlaub und ich wäre wieder mit Dir
               zuſammen.\pend
           \pstart
           Grüß’ Dich Gott, mein lieber Freund, und ſchreib’ bald, – beſonders, wie die Dinge im
                  \textcolor{brown}{Burgtheater}{}\ledrightnote{\textcolor{brown}{Burgtheater}} weitergehen.\pend
           \pstart
           In Treue {\\[\baselineskip]}Dein {\\[\baselineskip]}\spacefill\mbox{Paul Goldmann}\pend
           \leftskip=0em{}\pstart
           \noindent{}Wie gefällt Dir folgender Satz: »Und alle möglichen Unzulänglichkeiten
                  menſchlicher Verhältniſſe wurden eilig wieder deutlich.«? Du meinſt, das ſei von
                     \textsc{\textcolor{blue}{Goethe}{}\ledrightnote{\textcolor{blue}{Johann Wolfgang von Goethe}}}. Aber nein, es iſt von \textsc{Arthur Schnitzler} und ſteht
                  in Deinem letzten Briefe. Wäre ich jetzt bei Dir, ſo würde ich Dir ſchleunigſt den
                     \textsc{\textcolor{blue}{Goethe}{}\ledrightnote{\textcolor{blue}{Johann Wolfgang von Goethe}}} wegnehmen. Du glaubſt, der \textcolor{blue}{Mann}{}\ledrightnote{→\textcolor{blue}{Johann Wolfgang von Goethe}} ſchreibe \strikeout{d\textcolor{gray}{a}} die auf ihre urſprüngliche Bedeutung zurückgeführte Sprache, das »Deutſche
                  an {\pb}und für ſich«. Aber nein, er ſchreibt einen
                  Styl, \uline{ſeinen} Styl, der ein ganz anderer iſt, als
                  der \textsc{Schnitzlersche}. Laß’ ihn wirklich einmal ein paar
                  Wochen liegen, den alten Herrn, wenn er ſich ſo hinterliſtig in Deine
                  Individualität einſchleicht, wie obiges Beiſpiel zeigt, das mich nicht wenig
                  vergnügt hat.\pend
           \endnumbering\briefempfaengerindex{Schnitzler, Arthur@\textsc{Schnitzler, Arthur}!zzzGoldmann, Paul@\emph{von Paul Goldmann}!1895-09-233@{23. 9. {[}1895{]}}|)be}\mylabel{h}  \normalsize

\doendnotes{C}
\bigskip
\vfill

\clearpage

\footnotesize

\lohead{\textsc{register}}

% Definiere theindex-Environment komplett neu ohne reledmac
\makeatletter
\renewenvironment{theindex}{%
  \section*{\indexname}%
  \setlength{\parindent}{0pt}%
  \setlength{\parskip}{0pt plus 0.3pt}%
  \let\item\@idxitem
}{%
  \clearpage
}
\makeatother

\IfFileExists{\jobname-pw.ind}{\input{\jobname-pw.ind}}{}

\end{document}

      