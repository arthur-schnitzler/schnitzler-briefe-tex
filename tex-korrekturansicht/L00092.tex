%% latex-korrekturansicht-vorspann.tex
%% Vorspann für die Korrekturansicht.
%% Lädt die gemeinsame Datei latex-vorspann.tex mit gesetztem Schalter.

\newif\ifkorrekturansicht
\korrekturansichttrue

\input{../tex-inputs/latex-vorspann}


               \section[Hugo von Hofmannsthal an Arthur Schnitzler, {[}4. 4. 1892?{]}]{ Hugo von Hofmannsthal an Arthur Schnitzler, {[}4. 4. 1892?{]}}\nopagebreak\mylabel{v}\rehead{ }\normalsize\beginnumbering\briefempfaengerindex{Schnitzler, Arthur@\textsc{Schnitzler, Arthur}!zzzHofmannsthal, Hugo von@\emph{von Hugo von Hofmannsthal}!1892-04-041@{{[}4. 4. 1892?{]}}|(be} \toendnotes[C]{\smallbreak\pagebreak[2]} \Standort{CUL, Schnitzler, B 43.}
\physDesc{Brief, 1 Blatt (Briefpapier mit aufgeprägtem Wappen), 4 Seiten
\newline{}Handschrift: schwarze Tinte, deutsche Kurrent
\newline{}Schnitzler: mit Bleistift datiert: »Anf April 92« und nummeriert:
               »24« \newline{}Editorischer Hinweis: eine Doppelseite fehlt; diese wird nach der Abschrift zitiert }\buchAbdrucke{\weitereDrucke{Hugo von Hofmannsthal, Arthur Schnitzler: \emph{Briefwechsel}. Hg. Therese Nickl und Heinrich Schnitzler. Frankfurt am Main: \emph{S. Fischer} 1964, S. 19–20.} }\toendnotes[C]{\smallbreak}\pstart{}{\pb}Lieber Freund.\pend\pstart
           Ich habe ausdrücklich und wiederholt gebeten, meinen Namen als Überſetzer auf den
                        \label{K_L00092_1v}\edtext{Einladungen}{\lemma{\textnormal{\emph{Einladungen}}}\Cendnote{\textnormal{Es handelt sich um die Einladung für
                        die Veranstaltung am 13. 4. 1892 (\textcolor{blue}{Maeterlinck}s \emph{\textcolor{green}{L’Intruse}}, in der Übersetzung von \textcolor{blue}{Ferry Beraton}
                   sowie eine einleitende »Conferènce« von \textcolor{blue}{Hermann Bahr}), die, da vergessen worden
                        war, eine polizeiliche Genehmigung einzuholen, kurzfristig abgesagt wurde.
                        Sie wurde dann – durch das Verbot mit gestiegenem Publikumsinteresse – am
                            2. 5. 1892 abgehalten. Die Einladungskarte an \textcolor{blue}{Marie Herzfeld} wurde am
                            4. 4. 1892 aufgegeben (\textcolor{blue}{Hugo von Hofmannsthal}: \emph{Briefe an Marie Herzfeld}. Hg. Horst Weber.
                            Heidelberg: \emph{Lothar Stiehm}{ }1967, S. 24.), am Vorabend fand eine Besprechung
                        statt – was die zeitliche Einordnung ermöglichen dürfte.}}}\label{K_L00092_1h} nicht zu
                    nennen. Man hat zwar mit Herrn \textcolor{blue}{von
                        Goldſchmid}{}\ledrightnote{\textcolor{blue}{Adalbert von Goldschmidt}} dieſe Rückſicht gehabt, mit mir aber nicht. Ich ſtreiche
                    auf meinen Einladungen, um weiter keine Geſchichten zu machen, das Loris einfach
                    durch. Ich habe {\pb}weder Lust für \textcolor{blue}{Beraton}{}\ledrightnote{\textcolor{blue}{Ferry Bératon}}s \textcolor{green}{Ueberſetzung}{}\ledrightnote{\textcolor{green}{L’Intruse}}, die ich nicht
                    kenne, einzustehen noch hätte ich eine von mir unterzeichnete Ueberſetzung jemals
                    von \textcolor{blue}{Beraton}{}\ledrightnote{\textcolor{blue}{Ferry Bératon}} korrigieren lassen. Diesen
                    groben Brief bekommen Sie, weil mir die andere{[}n{]} wurst sind, und Sie verdienen
                    ihn auch, weil Sie bei der Besprechung (½ 11) wahrscheinlich
                    ſchläfrig waren und nicht aufgelegt, Tactlosigkeiten zu verhindern.\pend
           \pstart
           Ich bitte
                    Sie, zu veranlassen, dass mein Name auf den übrigen Einladungen ausgestrichen
                    wird. Uebrigens ist der Stil der Einladungen ebenso hübsch als ihr Inhalt
                    unzureichend – »werden zur Aufführung gelangen« iſt gerade lächerlich »werden{[}«{]}
                    – wieso? von wem? wodurch?\pend
           \pstart
           Das ganze sieht aus als ob schon eine (gescheidte)
                    Erklärung vorangegangen wäre. \textcolor{green}{l’Intrus}{}\ledrightnote{\textcolor{green}{L’Intruse}} ist
                    eine directe Verfälschung, das Stück heisst \textcolor{green}{l’Intruse}{}\ledrightnote{\textcolor{green}{L’Intruse}}. {\pb}Seit
                    wann ändert man Titel?\pend
           \pstart
           Ich weiß noch nicht, ob ich mich entſchließen werde, dieſe Wiſche auszuſchicken.
                    Wozu haben Sie dann geſtern die Geſchichte vor mir feſtgeſetzt? Wozu ſind
                    überhaupt Beſprechungen, wenn hinterdrein immer alles geändert wird?\pend
           \pstart
           Ekelhaft!{\\[\baselineskip]}\spacefill\mbox{Loris.}\pend
           \leftskip=0em{}\endnumbering\briefempfaengerindex{Schnitzler, Arthur@\textsc{Schnitzler, Arthur}!zzzHofmannsthal, Hugo von@\emph{von Hugo von Hofmannsthal}!1892-04-041@{{[}4. 4. 1892?{]}}|)be}\mylabel{h}  \normalsize

\doendnotes{C}
\bigskip
\vfill

\clearpage

\footnotesize

\lohead{\textsc{register}}

% Definiere theindex-Environment komplett neu ohne reledmac
\makeatletter
\renewenvironment{theindex}{%
  \section*{\indexname}%
  \setlength{\parindent}{0pt}%
  \setlength{\parskip}{0pt plus 0.3pt}%
  \let\item\@idxitem
}{%
  \clearpage
}
\makeatother

\IfFileExists{\jobname-pw.ind}{\input{\jobname-pw.ind}}{}

\end{document}

      