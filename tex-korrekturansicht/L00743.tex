%% latex-korrekturansicht-vorspann.tex
%% Vorspann für die Korrekturansicht.
%% Lädt die gemeinsame Datei latex-vorspann.tex mit gesetztem Schalter.

\newif\ifkorrekturansicht
\korrekturansichttrue

\input{../tex-inputs/latex-vorspann}


               \section[Hermann Bahr an Arthur Schnitzler, 16. 11. 1897]{ Hermann Bahr an Arthur Schnitzler, 16. 11. 1897}\nopagebreak\mylabel{v}\rehead{ }\normalsize\beginnumbering\briefempfaengerindex{Schnitzler, Arthur@\textsc{Schnitzler, Arthur}!zzzBahr, Hermann@\emph{von Hermann Bahr}!1897-11-162@{16. 11. 1897}|(be} \toendnotes[C]{\smallbreak\pagebreak[2]} \Standort{CUL, Schnitzler, B 5b.}
\physDesc{Brief, 1 Blatt, 1 Seite
\newline{}Handschrift: schwarze Tinte, deutsche Kurrent\newline{}Ordnung: mit Bleistift von unbekannter Hand nummeriert: »57« }\buchAbdrucke{\weitereDrucke{Hermann Bahr, Arthur Schnitzler: \emph{Briefwechsel, Aufzeichnungen, Dokumente (1891–1931)}. Hg. Kurt Ifkovits und Martin Anton Müller. Göttingen: \emph{Wallstein} 2018, S. 159.} }\toendnotes[C]{\smallbreak}\pstart
           \noindent{}{\pb}\textcolor{gray}{\textbf{»\textcolor{brown}{Die
                        Zeit}{}\ledrightnote{\textcolor{brown}{Die Zeit. Wiener Wochenschrift}}«}}\hfill \textcolor{gray}{\textbf{\textbf{\textcolor{pink}{Wien}{}\ledrightnote{\textcolor{pink}{Wien}}}, den }}16. Nov. \textcolor{gray}{\textbf{189}}7\pend
           \pstart
           \textcolor{gray}{\textbf{Wiener Wochenſchrift}}\hfill \textcolor{gray}{\textbf{\textcolor{pink}{IX/3, Günthergaſſe 1}{}\ledrightnote{\textcolor{pink}{Günthergasse}}.}}\pend
           \pstart
           \textcolor{gray}{\textbf{\textbf{Herausgeber}:}}{\\}\textcolor{gray}{\textbf{Profeſſor Dr. \textcolor{blue}{I. Singer}{}\ledrightnote{\textcolor{blue}{Isidor Singer}}, \textcolor{blue}{Hermann Bahr}{}\ledrightnote{\textcolor{blue}{Hermann Bahr}},
                        Dr. \textcolor{blue}{Heinrich Kanner}{}\ledrightnote{\textcolor{blue}{Heinrich Kanner}}.}}\pend
           \pstart
           \textcolor{gray}{\textbf{Telephon Nr. 6415.}}\pend
           \pstart\center{}Lieber Arthur!\pend\pstart
           Darf ich Dich bitten, ein Exemplar des \textcolor{green}{\textsc{Cosmopolis}}{}\ledrightnote{\textcolor{green}{Cosmopolis}} mit Deiner \textcolor{green}{Novelle}{}\ledrightnote{→\textcolor{green}{Die Toten schweigen}} an Herrn \textcolor{blue}{Ignaz Kugel}{}\ledrightnote{\textcolor{blue}{Ignaz Kugel}}, \textcolor{pink}{Lindengaſſe 11}{}\ledrightnote{\textcolor{pink}{Lindengasse}}
               zu ſenden, damit er es der Polizei vorlegt? Du kriegſt es dann wieder zurück.\pend
           \pstart
           Herzlichſt{\\[\baselineskip]}Dein{\\[\baselineskip]}\spacefill\mbox{Hermann}\pend
           \leftskip=0em{}\pstart
           \textcolor{gray}{\textbf{\label{T_L00743_1v}\edtext{Alle für »\textcolor{brown}{Die Zeit}{}\ledrightnote{\textcolor{brown}{Die Zeit. Wiener Wochenschrift}}« beſtimmten Zuſchriften und Sendungen ſind an
                  die Redaction der »\textcolor{brown}{Zeit}{}\ledrightnote{\textcolor{brown}{Die Zeit. Wiener Wochenschrift}}« und \textbf{nicht} an die Perſon eines der Herausgeber zu richten.}{\lemma{\textnormal{\emph{Alle … richten.}}}\Cendnote{\textnormal{am unteren Rand der Seite}}}\label{T_L00743_1h}}}\pend
           \endnumbering\briefempfaengerindex{Schnitzler, Arthur@\textsc{Schnitzler, Arthur}!zzzBahr, Hermann@\emph{von Hermann Bahr}!1897-11-162@{16. 11. 1897}|)be}\mylabel{h}  \normalsize

\doendnotes{C}
\bigskip
\vfill

\clearpage

\footnotesize

\lohead{\textsc{register}}

% Definiere theindex-Environment komplett neu ohne reledmac
\makeatletter
\renewenvironment{theindex}{%
  \section*{\indexname}%
  \setlength{\parindent}{0pt}%
  \setlength{\parskip}{0pt plus 0.3pt}%
  \let\item\@idxitem
}{%
  \clearpage
}
\makeatother

\IfFileExists{\jobname-pw.ind}{\input{\jobname-pw.ind}}{}

\end{document}

      