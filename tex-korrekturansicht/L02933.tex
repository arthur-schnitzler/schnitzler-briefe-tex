%% latex-korrekturansicht-vorspann.tex
%% Vorspann für die Korrekturansicht.
%% Lädt die gemeinsame Datei latex-vorspann.tex mit gesetztem Schalter.

\newif\ifkorrekturansicht
\korrekturansichttrue

\input{../tex-inputs/latex-vorspann}


         \renewcommand{\erwaehnteInstitutionen}{Institutionen: Lobe-Theater}
         \renewcommand{\erwaehnteOrte}{Orte: Berlin, Breslau, Frankgasse, Wien}
         \renewcommand{\erwaehnteWerke}{Werke: Der Schleier der Beatrice. Schauspiel in fünf Akten, Neue Freie Presse, Theater- und Kunstnachrichten [Schleier der Beatrice in Breslau]}
               \section[ Paul Goldmann an Arthur Schnitzler, 22. 9. 1900]{Paul Goldmann an Arthur Schnitzler, 22. 9. 1900}\nopagebreak\mylabel{v}\rehead{ }\normalsize\beginnumbering\briefempfaengerindex{Schnitzler, Arthur@\textsc{Schnitzler, Arthur}!zzzGoldmann, Paul@\emph{von Paul Goldmann}!1900-09-221@{22. 9. 1900}|(be} \toendnotes[C]{\smallbreak\pagebreak[2]} \Standort{DLA, A:Schnitzler, HS.NZ85.1.3170.}
\physDesc{Postkarte
\newline{}Handschrift: 1) blaue Tinte, deutsche Kurrent\hspace{1em}2) blaue Tinte, lateinische Kurrent (\noindent{}Adresse)\hspace{1em}\newline{}Versand: 1) Stempel: »\nobreak{}\oindex{Berlin@\textbf{Berlin}, \emph{https://www.geonames.org/ontologyP.PPLC}|pwk}Berlin\textcolor{gray}{,}
                                       S.W., 11. 9. 00, 4–5 N., 46\nobreak{}«.   2) Stempel: »\nobreak{}Wien 9/3 72, 23. 9. 00, 9.V, Bestellt\nobreak{}«. 
\newline{}Schnitzler: mit Bleistift das Jahr »{[}1{]}900« vermerkt }\toendnotes[C]{\smallbreak}\pstart{}{\pb}Herrn\pend{}\pstart{}Dr. Arthur Schnitzler\pend{}\pstart{}\textcolor{pink}{Wien}{}\ledrightnote{\textcolor{pink}{Wien}}\pend{}\pstart{}\textcolor{pink}{IX. Frankgaſse 1}{}\ledrightnote{\textcolor{pink}{Frankgasse}}.\pend{}{\bigskip}\pstart
           {\pb}\textcolor{pink}{Berlin}{}\ledrightnote{\textcolor{pink}{Berlin}}, 22. September.\pend
           \pstart
           \label{K_L02933-1v}\edtext{M. l. F.}{\lemma{\textnormal{\emph{M. l. F.}}}\Cendnote{\textnormal{Mein lieber Freund}}}\label{K_L02933-1h}, die aus \textcolor{pink}{Berlin}{}\ledrightnote{\textcolor{pink}{Berlin}} datirte \label{K_L02933-2v}\edtext{\textcolor{green}{Mittheilung}{}\ledrightnote{{$\rightarrow$}\textcolor{green}{Theater- und Kunstnachrichten [Schleier der Beatrice in Breslau]}}}{\lemma{\textnormal{\emph{Mittheilung}}}\Cendnote{\textnormal{»– Aus \textcolor{pink}{\so{Berlin}}
                  wird uns gemeldet: ›\textcolor{green}{Der Schleier der Beatrice}‹ von \textcolor{blue}{Arthur \so{Schnitzler}}
                  wird im Einvernehmen mit dem Dichter demnächst im \textcolor{brown}{Breslauer Stadttheater}
               ſeine Erſtaufführung erleben.«
                  o. V.: \emph{\textcolor{green}{Theater- und Kunstnachrichten}}.
                     In: \emph{\textcolor{green}{Neue Freie Presse}}, Nr. 12959, 21. 9. 1900, Morgenblatt, S. 6–7, hier:
                  S. 7.}}}\label{K_L02933-2h} von der Aufführung der »\textsc{\textcolor{green}{Beatrice}{}\ledrightnote{\textcolor{green}{Der Schleier der Beatrice. Schauspiel in fünf Akten}}}« in \textcolor{pink}{Breslau}{}\ledrightnote{\textcolor{pink}{Breslau}}, welche geſtern im Theatertheil der \textcolor{green}{N. Fr.
                  Pr.}{}\ledrightnote{\textcolor{green}{Neue Freie Presse}} zu leſen war, ſtammt ſelbſtverſtändlich nicht von mir.\pend
           \pstart
           Viele Grüße! {\\[\baselineskip]}\spacefill\mbox{P. G.}\pend
           \leftskip=0em{}\endnumbering\briefempfaengerindex{Schnitzler, Arthur@\textsc{Schnitzler, Arthur}!zzzGoldmann, Paul@\emph{von Paul Goldmann}!1900-09-221@{22. 9. 1900}|)be}\mylabel{h}  \normalsize

\doendnotes{C}
\bigskip
\vfill

\clearpage

\footnotesize

\lohead{\textsc{register}}

% Definiere theindex-Environment komplett neu ohne reledmac
\makeatletter
\renewenvironment{theindex}{%
  \section*{\indexname}%
  \setlength{\parindent}{0pt}%
  \setlength{\parskip}{0pt plus 0.3pt}%
  \let\item\@idxitem
}{%
  \clearpage
}
\makeatother

\IfFileExists{\jobname-pw.ind}{\input{\jobname-pw.ind}}{}

\end{document}

      