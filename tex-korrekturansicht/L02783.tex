%% latex-korrekturansicht-vorspann.tex
%% Vorspann für die Korrekturansicht.
%% Lädt die gemeinsame Datei latex-vorspann.tex mit gesetztem Schalter.

\newif\ifkorrekturansicht
\korrekturansichttrue

\input{../tex-inputs/latex-vorspann}


               \section[Paul Goldmann an Arthur Schnitzler, Paul Goldmann an Arthur Schnitzler, 26. 7. {[}1896{]}]{ Paul Goldmann an Arthur Schnitzler, 26. 7. {[}1896{]}}\nopagebreak\mylabel{v}\rehead{ }\normalsize\beginnumbering\briefempfaengerindex{Schnitzler, Arthur@\textsc{Schnitzler, Arthur}!zzzGoldmann, Paul@\emph{von Paul Goldmann}!1896-07-263@{26. 7. {[}1896{]}}|(be} \toendnotes[C]{\smallbreak\pagebreak[2]} \Standort{DLA, A:Schnitzler, HS.NZ85.1.3166.}
\physDesc{Brief, 1 Blatt, 2 Seiten
\newline{}Handschrift: blaue Tinte, deutsche Kurrent
\newline{}Schnitzler: mit Bleistift das Jahr »96« und »nach \textcolor{pink}{Hamb{[}urg{]}}
                                       ant\textcolor{gray}{w}{[}orten{]}« vermerkt }\toendnotes[C]{\smallbreak}\pstart
           \noindent{}{\pb}\textcolor{gray}{\textbf{\textbf{\textcolor{brown}{Frankfurter Zeitung}{}\ledrightnote{\textcolor{brown}{Frankfurter Zeitung}}}}}\pend
           \pstart
           \textcolor{gray}{\textbf{(\textcolor{brown}{\begin{otherlanguage}{french}Gazette de Francfort\end{otherlanguage}}{}\ledrightnote{\textcolor{brown}{Frankfurter Zeitung}}).}}\pend
           \pstart
           \textcolor{gray}{\textbf{\textbf{\begin{otherlanguage}{french}Fondateur M.\end{otherlanguage}{ }\textcolor{blue}{L. Sonnemann}{}\ledrightnote{\textcolor{blue}{Leopold Sonnemann}}.}}}\pend
           \pstart
           \begin{otherlanguage}{french}\textcolor{gray}{\textbf{\textcolor{green}{Journal}{}\ledrightnote{→\textcolor{green}{Frankfurter Zeitung}} politique,
                        financier,}}\end{otherlanguage}\pend
           \pstart
           \begin{otherlanguage}{french}\textcolor{gray}{\textbf{commercial et littéraire.}}\end{otherlanguage}\pend
           \pstart
           \begin{otherlanguage}{french}\textcolor{gray}{\textbf{\textbf{Paraissant trois fois par jour.}}}\end{otherlanguage}\hfill \textsc{\textcolor{pink}{Paris}{}\ledrightnote{\textcolor{pink}{Paris}}}, 26. Juli.\pend
           \pstart
           \begin{otherlanguage}{french}\textcolor{gray}{\textbf{\textbf{Bureau à \textcolor{pink}{Paris}{}\ledrightnote{\textcolor{pink}{Paris}}}}}\end{otherlanguage}\pend
           \pstart
           \begin{otherlanguage}{french}\textcolor{gray}{\textbf{\textbf{\textcolor{pink}{24. Rue Feydeau}{}\ledrightnote{\textcolor{pink}{rue Feydeau}}.}}}\end{otherlanguage}\pend
           \pstart\center{}Mein lieber Freund,\pend\pstart
           Ich wollte eigentlich geſtern abreiſen, wurde aber
               durch die \label{K_L02783-1v}\edtext{Ereigniſſe von \textsc{\textcolor{pink}{Lille}{}\ledrightnote{\textcolor{pink}{Lille}}}}{\lemma{\textnormal{\emph{Ereigniſſe von Lille}}}\Cendnote{\textnormal{Rund um den Kongress der \emph{\textcolor{brown}{französischen Arbeiterpartei}} (21.–25. 7. 1896) spitzte sich Ende Juli 1896 in \textcolor{pink}{Lille}
                  die Situation zwischen sozialistisch und antisozialistisch gestimmten Bürgerinnen
                  und Bürgern zu. Dabei kam es auch zu gewaltvollen Ausschreitungen und
                  Verhaftungen.}}}\label{K_L02783-1h} zurückgehalten. Auch wünſchte meine \textcolor{brown}{Redaction}{}\ledrightnote{→\textcolor{brown}{Frankfurter Zeitung}}, ich ſolle bis Ende Monats hierbleiben. So komme ich kaum vor Freitag 31. Juli fort, vielleicht erſt Samſtag. Ich bleibe einen Tag in \textsc{\textcolor{pink}{Köln}{}\ledrightnote{\textcolor{pink}{Köln}}}, drei oder vier in \textsc{\textcolor{pink}{Hamburg}{}\ledrightnote{\textcolor{pink}{Hamburg}}}. Dann komme ich nach \textsc{\textcolor{pink}{Kopenhagen}{}\ledrightnote{\textcolor{pink}{Kopenhagen}}}. Noch habe ich keine Ahnung, wo {\pb}ich Dich
                  \label{K_L02783-2v}\edtext{treffe}{\lemma{\textnormal{\emph{treffe}}}\Cendnote{\textnormal{siehe Paul Goldmann an Arthur Schnitzler, 29. 4. [1896]}}}\label{K_L02783-2h}. Schreib’ mir Deine Adreſſe nach \textsc{\textcolor{pink}{\uline{Hamburg}}{}\ledrightnote{\textcolor{pink}{Hamburg}}}, \textsc{Poste restante}. Vielen Dank für Deine lieben
               Nachrichten von unterwegs! Ich bin in großer Sorge. Es will diesmal gar nicht gehen
               mit dem Fortkommen.\pend
           \pstart
           Viele treue Grüße!\pend
           \pstart
           Wie ſchön das iſt, daß ich Dich bald ſehen ſoll!\pend
           \pstart
           In Treue {\\[\baselineskip]}Dein {\\[\baselineskip]}\spacefill\mbox{Paul Goldmann}\pend
           \leftskip=0em{}\endnumbering\briefempfaengerindex{Schnitzler, Arthur@\textsc{Schnitzler, Arthur}!zzzGoldmann, Paul@\emph{von Paul Goldmann}!1896-07-263@{26. 7. {[}1896{]}}|)be}\mylabel{h}\begin{anhang}\end{anhang}\normalsize

\doendnotes{C}
\bigskip
\vfill

\clearpage

\footnotesize

\lohead{\textsc{register}}

% Definiere theindex-Environment komplett neu ohne reledmac
\makeatletter
\renewenvironment{theindex}{%
  \section*{\indexname}%
  \setlength{\parindent}{0pt}%
  \setlength{\parskip}{0pt plus 0.3pt}%
  \let\item\@idxitem
}{%
  \clearpage
}
\makeatother

\IfFileExists{\jobname-pw.ind}{\input{\jobname-pw.ind}}{}

\end{document}

      