%% latex-korrekturansicht-vorspann.tex
%% Vorspann für die Korrekturansicht.
%% Lädt die gemeinsame Datei latex-vorspann.tex mit gesetztem Schalter.

\newif\ifkorrekturansicht
\korrekturansichttrue

\input{../tex-inputs/latex-vorspann}


               \section[Robert Adam an Arthur Schnitzler, 10. 8. 1918]{ Robert Adam an Arthur Schnitzler, 10. 8. 1918}\nopagebreak\mylabel{v}\rehead{ }\normalsize\beginnumbering\briefempfaengerindex{Schnitzler, Arthur@\textsc{Schnitzler, Arthur}!zzzAdam, Robert@\emph{von Robert Adam}!1918-08-101@{10. 8. 1918}|(be} \toendnotes[C]{\smallbreak\pagebreak[2]} \Standort{CUL, Schnitzler, B 1.}
\physDesc{Brief, 1 Blatt, 2 Seiten
\newline{}Handschrift: schwarze Tinte, deutsche Kurrent
\newline{}Schnitzler: 1) mit Bleistift beschriftet: »\textsc{Adam}« 2) mit rotem Buntstift eine Unterstreichung\newline{}Ordnung: von unbekannter Hand nummeriert:
                                            »6« }\Standort{Wien, Österreichische Nationalbibliothek, Cod.ser. 52.263, 209 verso.}
\physDesc{Brief, maschinelle Abschrift
\newline{}Schreibmaschine}\toendnotes[C]{\smallbreak}\pstart
           \centering{}{\pb}\textcolor{pink}{Wien}{}\ledrightnote{\textcolor{pink}{Wien}}{ }10/8 1918\pend
           \pstart\center{}Hochverehrter Herr Doktor!\pend\pstart
           Ich ſende Ihnen ein kleines Verzeichnis von Büchern über jugendliche Verbrecher,
                    die ich dem Katalog der »\textcolor{brown}{Privatbibliothek der
                        Juſtizbeamten}{}\ledrightnote{\textcolor{brown}{Privatbibliothek der Wiener Justizbeamten}}« entnehme. Dieſe Bücher – wenn auch nur nach und nach –
                    könnte ich Ihnen beſchaffen. Die \textcolor{brown}{Bibliothek}{}\ledrightnote{→\textcolor{brown}{Privatbibliothek der Wiener Justizbeamten}} enthält aber gewiß – da ſie an kriminaliſtiſchen Werken
                    ſehr reichhaltig iſt – noch viele andere Bücher, die das Sie intereſſierende
                    Thema behandeln; der Katalog iſt aber äußerſt ſchlecht angelegt, die Titel ſind
                    oft unrichtig oder {\pb}unvollſtändig
                    angegeben. Wenn ich wieder einmal vormittags einige freie Zeit erübrige,
                    durchſtöbere ich die \textcolor{brown}{Bibliothek}{}\ledrightnote{→\textcolor{brown}{Privatbibliothek der Wiener Justizbeamten}}{ }ſelbſt und ſchlage insbeſondere in den
                    Inhaltsverzeichniſſen der kriminaliſtiſchen Zeitſchriften nach; es ſollte mich
                    dann ſehr wundern, wenn ſich nicht Arbeiten fänden – insbeſondere auch
                    Wiedergabe konkreter Rechtsfälle –, die Ihnen von Nutzen ſein könnten.\pend
           \pstart
           Die weniger in Betracht kommenden Bücher habe ich eingeklammert.\pend
           \pstart
           Auch die Abteilung: »Pſychiatrie und Kriminalpſychologie« unſerer \textcolor{brown}{Bibliothek}{}\ledrightnote{→\textcolor{brown}{Privatbibliothek der Wiener Justizbeamten}} iſt ziemlich reichhaltig.\pend
           \pstart
           Mit ergebenſten Grüßen\pend
           \pstart
           Ihr{\\[\baselineskip]}\spacefill\mbox{D\textsuperscript{r}RAdam}\pend
           \leftskip=0em{}\endnumbering\briefempfaengerindex{Schnitzler, Arthur@\textsc{Schnitzler, Arthur}!zzzAdam, Robert@\emph{von Robert Adam}!1918-08-101@{10. 8. 1918}|)be}\mylabel{h}  \normalsize

\doendnotes{C}
\bigskip
\vfill

\clearpage

\footnotesize

\lohead{\textsc{register}}

% Definiere theindex-Environment komplett neu ohne reledmac
\makeatletter
\renewenvironment{theindex}{%
  \section*{\indexname}%
  \setlength{\parindent}{0pt}%
  \setlength{\parskip}{0pt plus 0.3pt}%
  \let\item\@idxitem
}{%
  \clearpage
}
\makeatother

\IfFileExists{\jobname-pw.ind}{\input{\jobname-pw.ind}}{}

\end{document}

      