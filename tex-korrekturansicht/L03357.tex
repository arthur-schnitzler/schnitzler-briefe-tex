%% latex-korrekturansicht-vorspann.tex
%% Vorspann für die Korrekturansicht.
%% Lädt die gemeinsame Datei latex-vorspann.tex mit gesetztem Schalter.

\newif\ifkorrekturansicht
\korrekturansichttrue

\input{../tex-inputs/latex-vorspann}


\renewcommand{\erwaehnteOrte}{Orte: Basilika San Petronio, Bologna, Frankgasse 1, IX., Alsergrund, Wien, Österreich}
\renewcommand{\erwaehnteWerke}{Werke: Der Schleier der Beatrice. Schauspiel in fünf Akten}
\section[ Felix Salten an Arthur Schnitzler, 20. 5. 1902]{Felix Salten an Arthur Schnitzler, 20. 5. 1902}
\nopagebreak\mylabel{v}
\rehead{ }\normalsize\beginnumbering\briefempfaengerindex{Schnitzler, Arthur@\textsc{Schnitzler, Arthur}!zzzSalten, Felix@\emph{von Felix Salten}!1902-05-202@{20. 5. 1902}|(be}
\toendnotes[C]{\smallbreak\pagebreak[2]}\Standort{CUL, Schnitzler, B 89, A 2.}
\physDesc{Postkarte, 184 Zeichen
\newline{}Handschrift: Bleistift, lateinische Kurrent
\newline{}Versand: 1) Stempel: »\nobreak{}\oindex{Bologna@\textbf{Bologna}, \emph{P.PPLA}|pwk}Bologna Ferrovia 20, 20 5 - 02, 5S\nobreak{}«.   2) Stempel: »\nobreak{}\oindex{IX., Alsergrund@\textbf{IX., Alsergrund}, \emph{A.ADM3}|pwk}9/3 Wien 72, 22. 5. 02, 8. V, Bestellt\nobreak{}«. 
\newline{}Ordnung: mit Bleistift von unbekannter Hand nummeriert: »154« }\toendnotes[C]{\smallbreak}\pstart{}{\pb}Herrn D\textsuperscript{r} Arthur Schnitzler\pend{}\pstart{}\textcolor{pink}{IX. Frankgaße 1}{}\ledrightnote{\textcolor{pink}{Frankgasse 1}}\pend{}\pstart{}\textcolor{pink}{Wien}{}\ledrightnote{\textcolor{pink}{Wien}}\pend{}\pstart{}\textcolor{pink}{Austria}{}\ledrightnote{\textcolor{pink}{Österreich}}\pend{}
{\bigskip}
\pstart
           {\pb}\textcolor{pink}{Bologna}{}\ledrightnote{\textcolor{pink}{Bologna}}, 20. Mai 02.\pend
           
\pstart
           \label{K_L03357-1v}\edtext{\textcolor{green}{Bentivoglio}{}\ledrightnote{{$\rightarrow$}\textcolor{green}{Der Schleier der Beatrice. Schauspiel in fünf Akten}} – \textcolor{pink}{San Petron}{}\ledrightnote{\textcolor{pink}{Basilika San Petronio}}, – \textcolor{green}{Beatrice}{}\ledrightnote{{$\rightarrow$}\textcolor{green}{Der Schleier der Beatrice. Schauspiel in fünf Akten}} u. s. w. \textcolor{green}{Filippo Loschi}{}\ledrightnote{{$\rightarrow$}\textcolor{green}{Der Schleier der Beatrice. Schauspiel in fünf Akten}}}{\lemma{\textnormal{\emph{Bentivoglio … Loschi}}}\Cendnote{\textnormal{\textcolor{blue}{Schnitzler}s \emph{\textcolor{green}{Der Schleier der Beatrice}} ist in \textcolor{pink}{Bologna} angesiedelt. Auch die \textcolor{pink}{Basilika
                     San Petron} kommt darin vor.}}}\label{K_L03357-1h} nicht zu vergessen, und dann der
                  \label{K_L03357-2v}\edtext{durchgängige Hund}{\lemma{\textnormal{\emph{durchgängige Hund}}}\Cendnote{\textnormal{Bezug unklar; womöglich eine Arkadengalerie
                  in \textcolor{pink}{Bologna}?}}}\label{K_L03357-2h}.\pend
           
\pstart
           herzl. {\\[\baselineskip]}\spacefill\mbox{FS.}\pend
           \leftskip=0em{}\endnumbering\briefempfaengerindex{Schnitzler, Arthur@\textsc{Schnitzler, Arthur}!zzzSalten, Felix@\emph{von Felix Salten}!1902-05-202@{20. 5. 1902}|)be}\mylabel{h}  \normalsize

\doendnotes{C}
\bigskip
\vfill

\clearpage

\footnotesize

\lohead{\textsc{register}}

% Definiere theindex-Environment komplett neu ohne reledmac
\makeatletter
\renewenvironment{theindex}{%
  \section*{\indexname}%
  \setlength{\parindent}{0pt}%
  \setlength{\parskip}{0pt plus 0.3pt}%
  \let\item\@idxitem
}{%
  \clearpage
}
\makeatother

\IfFileExists{\jobname-pw.ind}{\input{\jobname-pw.ind}}{}

\end{document}

      