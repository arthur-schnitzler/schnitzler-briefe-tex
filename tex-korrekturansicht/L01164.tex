%% latex-korrekturansicht-vorspann.tex
%% Vorspann für die Korrekturansicht.
%% Lädt die gemeinsame Datei latex-vorspann.tex mit gesetztem Schalter.

\newif\ifkorrekturansicht
\korrekturansichttrue

\input{../tex-inputs/latex-vorspann}


               \section[Richard Beer-Hofmann an Arthur Schnitzler, 20. 8. 1901]{ Richard Beer-Hofmann an Arthur Schnitzler,
               20. 8. 1901}\nopagebreak\mylabel{v}\rehead{ }\normalsize\beginnumbering\briefempfaengerindex{Schnitzler, Arthur@\textsc{Schnitzler, Arthur}!zzzBeer-Hofmann, Richard@\emph{von Richard Beer-Hofmann}!1901-08-201@{20. 8. 1901}|(be} \toendnotes[C]{\smallbreak\pagebreak[2]} \Standort{CUL, Schnitzler, B 8.}
\physDesc{Brief, 1 Blatt, 2 Seiten
\newline{}Handschrift: blauer Buntstift, lateinische Kurrent\newline{}Ordnung: mit Bleistift von unbekannter Hand nummeriert: »168« }\buchAbdrucke{\weitereDrucke{Arthur Schnitzler, Richard Beer-Hofmann: \emph{Briefwechsel 1891–1931}. Hg. Konstanze Fliedl. Wien, Zürich: \emph{Europaverlag} 1992, S. 155.} }\toendnotes[C]{\smallbreak}\pstart
           \raggedleft{}{\pb}\textcolor{pink}{Pörtschach}{}\ledrightnote{\textcolor{pink}{Pörtschach}}{ }20/VIII 1901\pend
           \pstart
           Lieber Arthur! Ich möchte mir gerne \textcolor{pink}{Waldbrunn}{}\ledrightnote{\textcolor{pink}{Wildbad Waldbrunn}} für künftigen Aufenthalt (Schicksalsklausel) ansehn. Werde also,
               vor Ihrer Abreise (27 od 28?) auf ein paar Stunden hinko{\geminationm}en, was Sie und \textcolor{blue}{Paul}{}\ledrightnote{\textcolor{blue}{Paul Goldmann}} nicht abhalten darf auf der Rückreise zu mir zu ko{\geminationm}en. Ich arbeite endlich, – aber früher hätt’ ich
               anfangen sollen! –\pend
           \pstart
           {\pb}Die beiden jungen \textcolor{blue}{Damen}{}\ledrightnote{→\textcolor{blue}{Olga Schnitzler}{\newline}→\textcolor{blue}{Elisabeth Steinrück}}, von denen die \textcolor{blue}{eine}{}\ledrightnote{→\textcolor{blue}{Olga Schnitzler}} vorläufig – wie ich von
               Ihnen höre – meine »Gemeinde« bildet, und von deren Verständniß ich, daher die
               ungeheuerste Meinung habe, würden mich nicht stören aber ich brauche i{\geminationm}er ein paar Tage um mich einzugewöhnen und die 6–8 Tage
               wären verloren.\pend
           \pstart
           Herzliche Grüße an \textcolor{blue}{Paul}{}\ledrightnote{\textcolor{blue}{Paul Goldmann}}.\pend
           \pstart
           Ihr{\\[\baselineskip]}\spacefill\mbox{Richard}\pend
           \leftskip=0em{}\endnumbering\briefempfaengerindex{Schnitzler, Arthur@\textsc{Schnitzler, Arthur}!zzzBeer-Hofmann, Richard@\emph{von Richard Beer-Hofmann}!1901-08-201@{20. 8. 1901}|)be}\mylabel{h}  \normalsize

\doendnotes{C}
\bigskip
\vfill

\clearpage

\footnotesize

\lohead{\textsc{register}}

% Definiere theindex-Environment komplett neu ohne reledmac
\makeatletter
\renewenvironment{theindex}{%
  \section*{\indexname}%
  \setlength{\parindent}{0pt}%
  \setlength{\parskip}{0pt plus 0.3pt}%
  \let\item\@idxitem
}{%
  \clearpage
}
\makeatother

\IfFileExists{\jobname-pw.ind}{\input{\jobname-pw.ind}}{}

\end{document}

      