%% latex-korrekturansicht-vorspann.tex
%% Vorspann für die Korrekturansicht.
%% Lädt die gemeinsame Datei latex-vorspann.tex mit gesetztem Schalter.

\newif\ifkorrekturansicht
\korrekturansichttrue

\input{../tex-inputs/latex-vorspann}


\section[Stefan Zweig an Arthur Schnitzler, 15. 5. {[}1922{]}]{L03680 Stefan Zweig an Arthur Schnitzler, 15. 5. {[}1922{]}}
\nopagebreak\mylabel{L03680v}
\rehead{ }\normalsize\beginnumbering\briefempfaengerindex{, @\textsc{, }!zzz, @\emph{von  }!1922-05-154@{15. 5. {[}1922{]}}|(be}
\toendnotes[C]{\smallbreak\pagebreak[2]}\Standort{DLA, A:Schnitzler, HS.NZ85.1.5577.}
\physDesc{Telegramm, 1 Blatt, 1 Seite, 179 Zeichen
\newline{}maschinell
\newline{}Versand: mit Bleistift Eintragung am Vordruck: »\noindent{}\textcolor{gray}{\textbf{Aufgenommen von .......... auf Ltg. Nr. ..........
                                          am}}{ }15/5 \textcolor{gray}{\textbf{192{\dots}}}{ }\textcolor{gray}{\textbf{um {\dotsfive}
                                       Uhr{ }{\dots}M.}}{ }\textcolor{gray}{fl}\textcolor{gray}{\textbf{Mittag}}« }
\buchAbdrucke{\weitereDrucke{Stefan Zweig: \emph{Briefwechsel mit Hermann Bahr, Sigmund Freud, Rainer Maria
                        Rilke und Arthur Schnitzler}. Herausgegeben von Jeffrey B. Berlin, Hans-Ulrich Lindken und Donald A. Prater. Frankfurt am Main: \emph{S. Fischer} 1987, S. 412.} }\toendnotes[C]{\smallbreak}\pstart{}{\pb}artur schnitzler \strikeout{s}\pend{}\pstart{}\textcolor{pink}{\introOben{}s\introOben{}ternwartestrasze
                     wien}\oindex{Wien@\textbf{Wien}!XVIII., Währing@\textbf{XVIII., Währing}!Sternwartestraße 71@\textbf{Sternwartestraße 71}, \emph{Wohngebäude}|pw}{}\ledrightnote{\textcolor{pink}{Sternwartestraße 71}}\pend{}{\bigskip}\vspace{1em}
\pstart
           \centering{}{\pb}\textcolor{pink}{salzburg}\oindex{Salzburg@\textbf{Salzburg}, \emph{Verwaltungsgebiet}|pw}{}\ledrightnote{\textcolor{pink}{Salzburg}} ts 1020 21/20 15/5{ }0.10\pend
           \vspace{0.5em}
\pstart
           empfangen sie zu tausendfaeltigen \label{K_L03680-1v}\edtext{grueszen der liebe}{\lemma{\textnormal{\emph{grueszen der liebe}}}\Cendnote{\textnormal{Am 15. 5. 1922 wurde \textcolor{blue}{Schnitzler}
            60 Jahre alt. Das Telegramm ist durch die Übermittlungszeile nur auf den Tag und Monat genau datierbar, die Jahresangabe fehlt Einen gewissen Hinweis gibt der Vordruck der Drucksache: »\textcolor{gray}{\textbf{Auflage 1922}}«. Die 
                  Aufbewahrung des Telegramms im Nachlass \textcolor{blue}{Schnitzlers} zusammen mit weiteren Gratulationsschreiben
                  zu diesem Geburtstag stützt diese Einordnung.}}}\label{K_L03680-1} und verehrung guetig auch die
               ihres getreuen \spacefill\mbox{stefan zweig .+}\pend
           \selectlanguage{ngerman}\endnumbering\briefempfaengerindex{, @\textsc{, }!zzz, @\emph{von  }!1922-05-154@{15. 5. {[}1922{]}}|)be}\mylabel{L03680h}  \normalsize

\doendnotes{C}
\bigskip
\vfill

\clearpage

\footnotesize

\lohead{\textsc{register}}

% Definiere theindex-Environment komplett neu ohne reledmac
\makeatletter
\renewenvironment{theindex}{%
  \section*{\indexname}%
  \setlength{\parindent}{0pt}%
  \setlength{\parskip}{0pt plus 0.3pt}%
  \let\item\@idxitem
}{%
  \clearpage
}
\makeatother

\IfFileExists{\jobname-pw.ind}{\input{\jobname-pw.ind}}{}

\end{document}

      