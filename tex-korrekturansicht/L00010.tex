%% latex-korrekturansicht-vorspann.tex
%% Vorspann für die Korrekturansicht.
%% Lädt die gemeinsame Datei latex-vorspann.tex mit gesetztem Schalter.

\newif\ifkorrekturansicht
\korrekturansichttrue

\input{../tex-inputs/latex-vorspann}


               \section[Arthur Schnitzler an Wilhelm Bölsche, {[}5. 5. 1891{]}]{ Arthur Schnitzler an Wilhelm Bölsche, {[}5. 5. 1891{]}}\nopagebreak\mylabel{v}\rehead{ }\normalsize\beginnumbering\briefempfaengerindex{Boelsche, Wilhelm@\textsc{Bölsche, Wilhelm}!zzzSchnitzler, Arthur@\emph{von Arthur Schnitzler}!1891-05-051@{{[}5. 5. 1891{]}}|(be} \toendnotes[C]{\smallbreak\pagebreak[2]} \Standort{Wrocław, Biblioteka Uniwersytecka, Böl.Pis 1772.}
\physDesc{Brief, 1 Blatt, 2 Seiten
\newline{}Handschrift: schwarze Tinte, deutsche Kurrent}\buchAbdrucke{\weitereDrucke{1) Alois Woldan: \emph{Arthur Schnitzler – Briefe an Wilhelm Bölsche.} In: \emph{Germanica Wratislaviensia} (1987) Nr. 77, S. 465.} \weitereDrucke{2) Wilhelm Bölsche: \emph{Briefwechsel. Mit Autoren der Freien Bühne}. Hg. Gerd-Hermann Susen. Berlin: \emph{Weidler} 2010, S. 671 (Werke und Briefe. Wissenschaftliche Ausgabe, Briefe I).} }\toendnotes[C]{\smallbreak}\pstart{}{\pb}Sehr geehrter Herr Redakteur,\pend\pstart
           ich \label{K_L00010_1v}\edtext{ſende}{\lemma{\textnormal{\emph{ſende}}}\Cendnote{\textnormal{vgl. A. S.: \emph{Tagebuch}, 5. 5. 1891}}}\label{K_L00010_1h} Ihnen
                    hier eine \textcolor{green}{Skizze}{}\ledrightnote{→\textcolor{green}{Der Sohn. Aus den Papieren eines Arztes}}, vielleicht
                    finden Sie dieſelbe für Ihre \textcolor{green}{Zeitſchrift}{}\ledrightnote{→\textcolor{green}{Freie Bühne für den Entwickelungskampf der Zeit}} geeignet, was mir zur beſondern Ehre gereichte. Können Sie
                    das Ding nicht brauchen, ſo haben Sie wohl die Liebens{\pb}würdigkeit, es bald an mich zurückzuſenden.\pend
           \pstart
           Hochachtungsvoll{\\[\baselineskip]}\spacefill\mbox{Dr. Arthur Schnitzler}\pend
           \leftskip=0em{}\pstart
           \noindent{}\textcolor{pink}{\textsc{Wien, I. Giselastraße 11}}{}\ledrightnote{\textcolor{pink}{Bösendorferstraße}}.\pend
           \endnumbering\briefempfaengerindex{Boelsche, Wilhelm@\textsc{Bölsche, Wilhelm}!zzzSchnitzler, Arthur@\emph{von Arthur Schnitzler}!1891-05-051@{{[}5. 5. 1891{]}}|)be}\mylabel{h}  \normalsize

\doendnotes{C}
\bigskip
\vfill

\clearpage

\footnotesize

\lohead{\textsc{register}}

% Definiere theindex-Environment komplett neu ohne reledmac
\makeatletter
\renewenvironment{theindex}{%
  \section*{\indexname}%
  \setlength{\parindent}{0pt}%
  \setlength{\parskip}{0pt plus 0.3pt}%
  \let\item\@idxitem
}{%
  \clearpage
}
\makeatother

\IfFileExists{\jobname-pw.ind}{\input{\jobname-pw.ind}}{}

\end{document}

      