%% latex-korrekturansicht-vorspann.tex
%% Vorspann für die Korrekturansicht.
%% Lädt die gemeinsame Datei latex-vorspann.tex mit gesetztem Schalter.

\newif\ifkorrekturansicht
\korrekturansichttrue

\input{../tex-inputs/latex-vorspann}


               \section[Paul Goldmann an Arthur Schnitzler, Paul Goldmann an Arthur Schnitzler, 22. 6. {[}1896{]}]{ Paul Goldmann an Arthur Schnitzler, 22. 6. {[}1896{]}}\nopagebreak\mylabel{v}\rehead{ }\normalsize\beginnumbering\briefempfaengerindex{Schnitzler, Arthur@\textsc{Schnitzler, Arthur}!zzzGoldmann, Paul@\emph{von Paul Goldmann}!1896-06-221@{22. 6. {[}1896{]}}|(be} \toendnotes[C]{\smallbreak\pagebreak[2]} \Standort{DLA, A:Schnitzler, HS.NZ85.1.3166.}
\physDesc{Brief, 3 Blätter, 11 Seiten
\newline{}Handschrift: blaue Tinte, deutsche Kurrent
\newline{}Schnitzler: 1) mit Bleistift das Jahr »96« vermerkt 2) mit rotem Buntstift fünf Unterstreichungen}\toendnotes[C]{\smallbreak}\pstart
           \noindent{}{\pb}\textcolor{gray}{\textbf{\textbf{\textcolor{brown}{Frankfurter Zeitung}{}\ledrightnote{\textcolor{brown}{Frankfurter Zeitung}}}}}\pend
           \pstart
           \textcolor{gray}{\textbf{(\textcolor{brown}{\begin{otherlanguage}{french}Gazette de Francfort\end{otherlanguage}}{}\ledrightnote{\textcolor{brown}{Frankfurter Zeitung}}).}}\pend
           \pstart
           \textcolor{gray}{\textbf{\textbf{\begin{otherlanguage}{french}Fondateur M.\end{otherlanguage}{ }\textcolor{blue}{L. Sonnemann}{}\ledrightnote{\textcolor{blue}{Leopold Sonnemann}}.}}}\pend
           \pstart
           \begin{otherlanguage}{french}\textcolor{gray}{\textbf{\textcolor{green}{Journal}{}\ledrightnote{→\textcolor{green}{Frankfurter Zeitung}} politique,
                        financier,}}\end{otherlanguage}\pend
           \pstart
           \begin{otherlanguage}{french}\textcolor{gray}{\textbf{commercial et littéraire.}}\end{otherlanguage}\pend
           \pstart
           \begin{otherlanguage}{french}\textcolor{gray}{\textbf{\textbf{Paraissant trois fois par jour.}}}\end{otherlanguage}\hfill \textsc{\textcolor{pink}{Paris}{}\ledrightnote{\textcolor{pink}{Paris}}}, 22. Juni.\pend
           \pstart
           \begin{otherlanguage}{french}\textcolor{gray}{\textbf{\textbf{Bureau à \textcolor{pink}{Paris}{}\ledrightnote{\textcolor{pink}{Paris}}}}}\end{otherlanguage}\pend
           \pstart
           \begin{otherlanguage}{french}\textcolor{gray}{\textbf{\textbf{\textcolor{pink}{24. Rue Feydeau}{}\ledrightnote{\textcolor{pink}{rue Feydeau}}.}}}\end{otherlanguage}\pend
           \pstart{}Mein lieber Freund,\pend\pstart
           Es iſt ſehr lieb und freundſchaftlich von Dir, daß Du ſo auf dem Zuſammentreffen mit
               mir beſtehſt. Auch mir kannſt Du glauben, daß ich Dich nicht mit leichtem Herzen
               »aufgeben« würde und daß ich ſehr betrübt ſein würde, wenn ich Dich in dieſem Jahre
               nicht ſehen könnte\textcolor{gray}{!} Aber es wird ſich doch ſchwer machen laſſen.
               Da iſt zunächſt der materielle Grund. Ich habe weniger Geld als je, {\pb}und wenn ich auch mich im Princip nicht \strikeout{f} fürchten würde, mir etwas von Dir auszuleihen, ſo
               heißt doch »ausleihen« ſoviel\strikeout{,} als: Geld nehmen, um
               es wiederzugeben. Nach meinen \label{T_L02778-1v}\edtext{jetzigen}{\lemma{\textnormal{\emph{jetzigen}}}\Cendnote{\textnormal{in der Vorlage steht:
                     »jeztigen«}}}\label{T_L02778-1h} finanziellen Zuſtänden ſehe ich aber abſolut
               kein Mittel, \strikeout{\textcolor{gray}{d}} das Ausgeliehene in abſehbarer Zeit zurückzugeben. Dazu kommt noch Allerlei an
               ſonſtigen Gründen: Ich bin ſehr müde und nervös, und die weite Eiſenbahn-Reiſe
               erſchreckt mich. {\pb}Ich kann ferner weder Seeluft noch
                  \strikeout{See\textcolor{gray}{d}} Seebad vertragen, ſondern brauche zu meiner Erholung Gebirgsluft. Außerdem
               habe ich über die Preiſe in \textsc{\textcolor{pink}{Scodsborg}{}\ledrightnote{\textcolor{pink}{Skodsborg}}} von einem \textcolor{blue}{Dänen}{}\ledrightnote{→\textcolor{blue}{?? [Däne in Paris]}}, der
               jedes Jahr hingeht, ganz andere Auskünfte erhalten, als Ihr: er meint, es ſei das
               theuerſte \textcolor{pink}{dän}{}\ledrightnote{→\textcolor{pink}{Dänemark}}iſche \textcolor{pink}{Seebad}{}\ledrightnote{\textcolor{pink}{Skodsborg}}. Endlich \strikeout{iſt
                     \textcolor{gray}{mir}} intereſſirt mich der \textcolor{pink}{ſkandinaviſche}{}\ledrightnote{\textcolor{pink}{Skandinavien}} Norden
               wenig, \textcolor{pink}{Dänemark}{}\ledrightnote{\textcolor{pink}{Dänemark}} ganz beſonders wenig, {\pb}und durch das \textcolor{pink}{Dän}{}\ledrightnote{→\textcolor{pink}{Dänemark}}en-Geſindel, das ich um \textsc{Al\textcolor{gray}{b}}{ }\textsc{\textcolor{blue}{Albert Langen}{}\ledrightnote{\textcolor{blue}{Albert Langen}}} habe kriechen ſehen, habe ich ſogar einen ſtarken – vielleicht ungerechten –
               Widerwillen gegen \textcolor{pink}{Dän}{}\ledrightnote{→\textcolor{pink}{Dänemark}}enthum
               bekommen. Nun glaube ich \strikeout{f\textcolor{gray}{erner}} ſo: Du wirſt nach vier Wochen \textcolor{pink}{ſchwed}{}\ledrightnote{→\textcolor{pink}{Schweden}}iſch- \textcolor{pink}{norweg}{}\ledrightnote{→\textcolor{pink}{Norwegen}}iſcher Reiſe ausgiebig genug von \textcolor{pink}{Skandinavien}{}\ledrightnote{\textcolor{pink}{Skandinavien}} haben, desgleichen \textsc{\textcolor{blue}{Richard}{}\ledrightnote{\textcolor{blue}{Richard Beer-Hofmann}}}, wenn er bereits im {\pb}Juli hingeht. Da Ihr nun ſo wie ſo nach Mittel-Europa
               zurück müßt, wie wäre es, wenn wir uns im Auguſt{ }\label{K_L02778-1v}\edtext{in der \textcolor{pink}{Schweiz}{}\ledrightnote{\textcolor{pink}{Schweiz}} träfen}{\lemma{\textnormal{\emph{in der Schweiz träfen}}}\Cendnote{\textnormal{nicht geschehen, siehe Paul Goldmann an Arthur Schnitzler, 29. 4. [1896]}}}\label{K_L02778-1h}? Einen großen Umweg macht Ihr nicht. Auch iſt es gar nicht übel: vier Wochen
               zu reiſen und ſich dann in der \textcolor{pink}{Schweiz}{}\ledrightnote{\textcolor{pink}{Schweiz}}, im \textcolor{pink}{Engadin}{}\ledrightnote{\textcolor{pink}{Engadin}}{ }\strikeout{z\textcolor{gray}{u}} etwa, auszuruhen. Warum ſeid Ihr denn {\pb}gar ſo
               ſehr auf das verfluchte \textcolor{pink}{Dänemark}{}\ledrightnote{\textcolor{pink}{Dänemark}}{ }\strikeout{erpicht,} erpicht, wo es nicht einmal Kunſt gibt,
               außer \textsc{\textcolor{blue}{Thorwaldsen}{}\ledrightnote{\textcolor{blue}{Bertel Thorvaldsen}}}, den man doch beſſer \uline{nicht} kennt. Und \textsc{\textcolor{green}{Hamlet}{}\ledrightnote{→\textcolor{green}{Hamlet}}}, welcher der einzig intereſſante \textcolor{pink}{Dän}{}\ledrightnote{→\textcolor{pink}{Dänemark}}e war, iſt auch ſchon todt. Wenn Ihr nun darauf beſteht,
               ſo werde ich doch mein Möglichſtes thun, um zu kommen. Aber Ihr ſolltet auch Einwände
               hören.\pend
           \pstart
           {\pb}Daß man von \textsc{\textcolor{blue}{Albert Langen}{}\ledrightnote{\textcolor{blue}{Albert Langen}}} überhaupt \strikeout{Ein\textcolor{gray}{wänd}} Eindrücke empfängt, überraſcht mich. Das zählt doch gar nicht mit. Das iſt ein
               dummer \textcolor{blue}{Bube}{}\ledrightnote{→\textcolor{blue}{Albert Langen}}, \strikeout{deſſ\textcolor{gray}{te}n} deſſen geiſtige Unfähigkeit
               hart am Blödſinn grenzt\substVorne{}\textsuperscript{,}\substDazwischen{}.\substHinten{} Das iſt zugleich frech und infam. Ich bitte Dich: laß’ Dich mit dem \textcolor{blue}{Burſchen}{}\ledrightnote{→\textcolor{blue}{Albert Langen}} in keiner Weiſe ein,
               gib ihm keinen Rath und verhilf’ ihm zu \strikeout{kei} keinen
               Bekanntſchaften. {\pb}Er wird Dich ausnutzen und Dich
               mit Bübereien entlohnen. {\dotsfour}\pend
           \pstart
           Ich habe den \textsc{\textcolor{blue}{Richard Mandl}{}\ledrightnote{\textcolor{blue}{Richard Mandl}}} nun endlich kennen gelernt. Begeiſtert bin ich nicht. Ein netter und ganz
               geſcheiter Menſch, aber ſehr egoiſtiſch, ſehr berechnet, ſehr kalt, ſehr von ſich
               eingenommen, ſehr ſtolz auf ſeine \label{K_L02778-4v}\edtext{\begin{otherlanguage}{french}\textsc{relations mondaines}\end{otherlanguage}}{\lemma{\textnormal{\emph{relations mondaines}}}\Cendnote{\textnormal{französisch: (weltliche)
                  Beziehungen}}}\label{K_L02778-4h}. Talent? Einiges jedenfalls, {\pb}viel aber wahrſcheinlich nicht. Er hat ein \label{K_L02778-7v}\edtext{\textcolor{green}{Lied}{}\ledrightnote{→\textcolor{green}{Anfang vom Ende}}}{\lemma{\textnormal{\emph{Lied}}}\Cendnote{\textnormal{eine \textcolor{green}{Vertonung} von \textcolor{blue}{Schnitzler}s Gedicht \emph{\textcolor{green}{Anfang vom Ende}}.
                     \textcolor{blue}{Schnitzler} dürfte die Vertonung erst am
                     4. 1. 1898
                  kennengelernt haben.}}}\label{K_L02778-7h} von Dir componirt, wie Du weißt. Ich halte das für
               mißlungen. Die leichte Trauer des \textcolor{green}{Lied}{}\ledrightnote{→\textcolor{green}{Anfang vom Ende}}es hat er in die ſchwerſten Accente überſetzt. Das \textcolor{green}{Lied}{}\ledrightnote{→\textcolor{green}{Anfang vom Ende}} iſt melancholiſch, die Muſik tragiſch,
               Verſe und Compoſition ſehen ſich an und können ſich nicht verſtehen.\pend
           \pstart
           Bitte, danke \textsc{\textcolor{blue}{Richard}{}\ledrightnote{\textcolor{blue}{Richard Beer-Hofmann}}} für ſeine Correſpondenz-{\pb}Karte. Ich hoffe, das
               hat ihn nicht zu ſehr \label{K_L02778-88v}\edtext{ermüdet}{\lemma{\textnormal{\emph{ermüdet}}}\Cendnote{\textnormal{Spott über die Schreibfaulheit \textcolor{blue}{Beer-Hofmanns}}}}\label{K_L02778-88h}. Am Tage, wo er dieſe
               Correnſpondenz-Karte verfaßt, hat er gewiß nicht mehr am »\textcolor{green}{Götterliebling}{}\ledrightnote{→\textcolor{green}{Der Tod Georgs}}« weitergeſchrieben, –
               hoffentlich aber hat \strikeout{ſich} er ſich am nächſten Tage
               wieder dieſem \textcolor{green}{Werke}{}\ledrightnote{→\textcolor{green}{Der Tod Georgs}}
               zugewendet, deſſen \strikeout{z\textcolor{gray}{w}} zweites \textcolor{green}{Capitel}{}\ledrightnote{→\textcolor{green}{Der Tod Georgs}} jetzt
                  \strikeout{\textcolor{gray}{faſ}} ſicher bereits der {\pb}Vollendung
               entgegenreift.\pend
           \pstart
           Grüß’ Dich Gott, liebſter Freund!\pend
           \pstart
           Dein {\\[\baselineskip]}\spacefill\mbox{P. Goldm}\pend
           \leftskip=0em{}\endnumbering\briefempfaengerindex{Schnitzler, Arthur@\textsc{Schnitzler, Arthur}!zzzGoldmann, Paul@\emph{von Paul Goldmann}!1896-06-221@{22. 6. {[}1896{]}}|)be}\mylabel{h}\begin{anhang}\end{anhang}\normalsize

\doendnotes{C}
\bigskip
\vfill

\clearpage

\footnotesize

\lohead{\textsc{register}}

% Definiere theindex-Environment komplett neu ohne reledmac
\makeatletter
\renewenvironment{theindex}{%
  \section*{\indexname}%
  \setlength{\parindent}{0pt}%
  \setlength{\parskip}{0pt plus 0.3pt}%
  \let\item\@idxitem
}{%
  \clearpage
}
\makeatother

\IfFileExists{\jobname-pw.ind}{\input{\jobname-pw.ind}}{}

\end{document}

      