%% latex-korrekturansicht-vorspann.tex
%% Vorspann für die Korrekturansicht.
%% Lädt die gemeinsame Datei latex-vorspann.tex mit gesetztem Schalter.

\newif\ifkorrekturansicht
\korrekturansichttrue

\input{../tex-inputs/latex-vorspann}


               \section[Paul Goldmann an Arthur Schnitzler, {[}5. – 20.? 11. 1896{]}]{ Paul Goldmann an Arthur Schnitzler, {[}5. – 20.? 11. 1896{]}}\nopagebreak\mylabel{v}\rehead{ }\normalsize\beginnumbering\briefempfaengerindex{Schnitzler, Arthur@\textsc{Schnitzler, Arthur}!zzzGoldmann, Paul@\emph{von Paul Goldmann}!1896-11-102@{{[}10. – 30. 11.? 1896{]}}|(be} \toendnotes[C]{\smallbreak\pagebreak[2]} \Standort{DLA, A:Schnitzler, HS.NZ85.1.3166.}
\physDesc{Brief, 1 Blatt, 1 Seite
\newline{}Handschrift: blaue Tinte, deutsche Kurrent\newline{}Beilage: handschriftlicher Brief: 1 Blatt, 1 Seite, beschnitten }\toendnotes[C]{\smallbreak}\pstart
           \noindent{}{\pb}Dies iſt ein Ausſchnitt aus einem Briefe, den mein
               College \textsc{\textcolor{blue}{Th. Wolff}{}\ledrightnote{\textcolor{blue}{Theodor Wolff}}} dieſer Tage von ſeiner \textcolor{blue}{Mutter}{}\ledrightnote{→\textcolor{blue}{Recha Wolff}} erhalten hat:\pend
           \pstart
           \noindent{}{[}hs. Wolff:{]} recht zu ſagen \label{K_L02793-1v}\edtext{Gestern}{\lemma{\textnormal{\emph{Gestern}}}\Cendnote{\textnormal{\emph{\textcolor{green}{Freiwild}} wurde zwischen dem 3. 11. 1896 und dem 16. 11. 1896 am \textcolor{pink}{Deutschen Theater} in
                     \textcolor{pink}{Berlin} gespielt. Dies erlaubt eine
                  Datierung des Briefs von \textcolor{blue}{Recha Wolff} auf
                  den Zeitraum zwischen dem 4. 11. 1896 und dem 17. 11. 1896. Berücksichtigt man die Übermittlung nach
                  \textcolor{pink}{Paris}, so dürfte 
                  \textcolor{blue}{Goldmann} seinen Brief zwischen dem 10. und dem Monatsende des November abgefasst haben.}}}\label{K_L02793-1h} war ich mit \textsc{\textcolor{blue}{Martha}{}\ledrightnote{\textcolor{blue}{Marta Wolff}}} am \textcolor{pink}{Deutſchen Theater}{}\ledrightnote{\textcolor{pink}{Deutsches Theater Berlin}}, \textcolor{gray}{wo}
               wir einen wirklichen Genuß hatten. »\textcolor{green}{Freiwild}{}\ledrightnote{\textcolor{green}{Freiwild. Schauspiel in 3 Akten}}«
               von Schnitzler iſt das Schönſte, was ich ſeit lange geſehen, und geſpielt wurde
               geradezu vollendet.\pend
           \endnumbering\briefempfaengerindex{Schnitzler, Arthur@\textsc{Schnitzler, Arthur}!zzzGoldmann, Paul@\emph{von Paul Goldmann}!1896-11-102@{{[}10. – 30. 11.? 1896{]}}|)be}\mylabel{h}\begin{anhang}\end{anhang}\normalsize

\doendnotes{C}
\bigskip
\vfill

\clearpage

\footnotesize

\lohead{\textsc{register}}

% Definiere theindex-Environment komplett neu ohne reledmac
\makeatletter
\renewenvironment{theindex}{%
  \section*{\indexname}%
  \setlength{\parindent}{0pt}%
  \setlength{\parskip}{0pt plus 0.3pt}%
  \let\item\@idxitem
}{%
  \clearpage
}
\makeatother

\IfFileExists{\jobname-pw.ind}{\input{\jobname-pw.ind}}{}

\end{document}

      