%% latex-korrekturansicht-vorspann.tex
%% Vorspann für die Korrekturansicht.
%% Lädt die gemeinsame Datei latex-vorspann.tex mit gesetztem Schalter.

\newif\ifkorrekturansicht
\korrekturansichttrue

\input{../tex-inputs/latex-vorspann}


               \section[Arthur Schnitzler an Hermann Bahr, {[}1{]}3. 9. 1901]{ Arthur Schnitzler an Hermann Bahr, {[}1{]}3. 9. 1901}\nopagebreak\mylabel{v}\rehead{ }\normalsize\beginnumbering\briefempfaengerindex{Bahr, Hermann@\textsc{Bahr, Hermann}!zzzSchnitzler, Arthur@\emph{von Arthur Schnitzler}!1901-09-131@{{[}1{]}3. 9. 1901}|(be} \toendnotes[C]{\smallbreak\pagebreak[2]} \Standort{TMW, HS AM 23340 Ba.}
\physDesc{Brief, 1 Blatt, 2 Seiten
\newline{}Handschrift: schwarze Tinte, deutsche Kurrent}\buchAbdrucke{\weitereDrucke{1) \emph{13. 9. 1901.} In: Arthur Schnitzler: \emph{The Letters of Arthur Schnitzler to Hermann Bahr}. Edited, annotated, and with an introduction, by Donald G.
                        Daviau. Chapel Hill: \emph{The University of North Carolina Press} 1978, S. 71 (University of North Carolina studies in the Germanic languages
                        and literatures, 89).} \weitereDrucke{2) Hermann Bahr, Arthur Schnitzler: \emph{Briefwechsel, Aufzeichnungen, Dokumente (1891–1931)}. Hg. Kurt Ifkovits und Martin Anton Müller. Göttingen: \emph{Wallstein} 2018, S. 215.} }\toendnotes[C]{\smallbreak}\pstart
           \noindent{}{\pb}lieber Hermann, es iſt ſehr freundlich von dir, daſs du die \textcolor{green}{kleinen Sachen}{}\ledrightnote{→\textcolor{green}{Die Frau mit dem Dolche}{\newline}→\textcolor{green}{Literatur}{\newline}→\textcolor{green}{Lebendige Stunden}}{ }ſo ſchnell geleſen haſt. Die
               Verwandlungsſchwierigkeit in der \textcolor{green}{Frau mit dem Dolch}{}\ledrightnote{\textcolor{green}{Die Frau mit dem Dolche}}
               wird hoffentlich zu beheben ſein, – der Idiotismus des Publikums wohl ſchwerer. Mehr
               Sorgen aber macht mir die Beſetzung. Ich bin nun mit einem \textcolor{green}{4. Einakter}{}\ledrightnote{→\textcolor{green}{Der Puppenspieler}} beſchäftigt, für den ich mir gern
               den \textcolor{blue}{Mitterwurzer}{}\ledrightnote{\textcolor{blue}{Friedrich Mitterwurzer}} aus der Erde kratzen möchte, u
               daſs ich auch noch \strikeout{den} einen \label{K_L01173_1v}\edtext{\textcolor{green}{fünften}{}\ledrightnote{→\textcolor{green}{Die letzten Masken}}}{\lemma{\textnormal{\emph{fünften}}}\Cendnote{\textnormal{\emph{\textcolor{green}{Die letzten Masken}}}}}\label{K_L01173_1h}{ }ſchreibe, iſt ziemlich {\pb}ſicher. In dieſen
               beiden Stücken wird nun allerdings der »Literaten«typus beträchtlich erweitert,
               dadurch aber für die »Uneingeweihten« klarer ſein. Schön wärs halt, wenn einem ein
               ſehr ſcharfes Wort als \label{K_L01173_2v}\edtext{Gesamttitel}{\lemma{\textnormal{\emph{Gesamttitel}}}\Cendnote{\textnormal{Nur \emph{\textcolor{green}{Die letzten
                     Masken}} wurde letztlich zu den bestehenden drei Einaktern hinzugefügt, und
                  diese wurden unter dem Titel \emph{\textcolor{green}{Lebendige Stunden. Vier
                     Einakter}} (Berlin: \emph{\textcolor{brown}{S. Fischer}}{ }1902) zusammengefasst.}}}\label{K_L01173_2h} einfiele, das für die anderen ſo deutlich wäre,
               wie für unſereinen das Wort »Literat«; aber doch noch mehr ſagt.\pend
           \pstart
           Herzlichen Gruß. Dein{\\[\baselineskip]}\spacefill\mbox{Arthur}\pend
           \leftskip=0em{}\pstart
           \textcolor{gray}{1}3. 9. 901.\pend
           \endnumbering\briefempfaengerindex{Bahr, Hermann@\textsc{Bahr, Hermann}!zzzSchnitzler, Arthur@\emph{von Arthur Schnitzler}!1901-09-131@{{[}1{]}3. 9. 1901}|)be}\mylabel{h}  \normalsize

\doendnotes{C}
\bigskip
\vfill

\clearpage

\footnotesize

\lohead{\textsc{register}}

% Definiere theindex-Environment komplett neu ohne reledmac
\makeatletter
\renewenvironment{theindex}{%
  \section*{\indexname}%
  \setlength{\parindent}{0pt}%
  \setlength{\parskip}{0pt plus 0.3pt}%
  \let\item\@idxitem
}{%
  \clearpage
}
\makeatother

\IfFileExists{\jobname-pw.ind}{\input{\jobname-pw.ind}}{}

\end{document}

      