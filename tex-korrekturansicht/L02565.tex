%% latex-korrekturansicht-vorspann.tex
%% Vorspann für die Korrekturansicht.
%% Lädt die gemeinsame Datei latex-vorspann.tex mit gesetztem Schalter.

\newif\ifkorrekturansicht
\korrekturansichttrue

\input{../tex-inputs/latex-vorspann}


               \section[Olga Schnitzler an Richard Beer-Hofmann, {[}30. 11. 1911?{]}]{ Olga Schnitzler an Richard Beer-Hofmann, {[}30. 11. 1911?{]}}\nopagebreak\mylabel{v}\rehead{ }\normalsize\beginnumbering\briefempfaengerindex{Beer-Hofmann, Richard@\textsc{Beer-Hofmann, Richard}!zzzSchnitzler, Olga@\emph{von Olga Schnitzler}!1911-11-301@{{[}30. 11. 1911?{]}}|(be} \toendnotes[C]{\smallbreak\pagebreak[2]} \Standort{YCGL, MSS 31.}
\physDesc{Briefkarte
\newline{}Handschrift: schwarze Tinte, lateinische Kurrent
\newline{}Beer-Hofmann: mit blauem Buntstift datiert:
                                 »1911« }\toendnotes[C]{\smallbreak}\pstart
           \noindent{}{\pb}\textcolor{gray}{\textbf{O. S.}}\pend
           \pstart
           Lieber Herr D\textsuperscript{r}, ich antworte in \textcolor{blue}{Arthurs}{}\ledrightnote{} Namen, der \label{K_L02565-1v}\edtext{bei \textcolor{blue}{d’Albert}{}\ledrightnote{\textcolor{blue}{Eugen d’Albert}} zum
                  Thee}{\lemma{\textnormal{\emph{bei d’Albert zum
                  Thee}}}\Cendnote{\textnormal{Das erlaubt die Datierung der Karte. Vgl. A. S.: \emph{Tagebuch}, 30. 11. 1911}}}\label{K_L02565-1h} ist: morgen
               können wir nicht kommen, haben schon bei \textcolor{blue}{Schmidl}{}\ledrightnote{\textcolor{blue}{Hugo Schmidl}{\newline}\textcolor{blue}{Paula Schmidl}}’s abgesagt, – ich bin zu Bett, {\pb}hoffentlich gehts an einem der nächsten Tage, – wir werden mit Vergnügen
               kommen.\pend
           \pstart
           Viele Grüsse Ihnen Allen!{\\[\baselineskip]}Ihre \spacefill\mbox{OlgaS.}\pend
           \leftskip=0em{}\endnumbering\briefempfaengerindex{Beer-Hofmann, Richard@\textsc{Beer-Hofmann, Richard}!zzzSchnitzler, Olga@\emph{von Olga Schnitzler}!1911-11-301@{{[}30. 11. 1911?{]}}|)be}\mylabel{h}  \normalsize

\doendnotes{C}
\bigskip
\vfill

\clearpage

\footnotesize

\lohead{\textsc{register}}

% Definiere theindex-Environment komplett neu ohne reledmac
\makeatletter
\renewenvironment{theindex}{%
  \section*{\indexname}%
  \setlength{\parindent}{0pt}%
  \setlength{\parskip}{0pt plus 0.3pt}%
  \let\item\@idxitem
}{%
  \clearpage
}
\makeatother

\IfFileExists{\jobname-pw.ind}{\input{\jobname-pw.ind}}{}

\end{document}

      