%% latex-korrekturansicht-vorspann.tex
%% Vorspann für die Korrekturansicht.
%% Lädt die gemeinsame Datei latex-vorspann.tex mit gesetztem Schalter.

\newif\ifkorrekturansicht
\korrekturansichttrue

\input{../tex-inputs/latex-vorspann}


\section[Stefan Zweig an Arthur Schnitzler, 18. 1. 1916]{L03657 Stefan Zweig an Arthur Schnitzler, 18. 1. 1916}
\nopagebreak\mylabel{L03657v}
\rehead{ }\normalsize\beginnumbering\briefempfaengerindex{, @\textsc{, }!zzz, @\emph{von  }!1916-01-181@{18. 1. 1916}|(be}
\toendnotes[C]{\smallbreak\pagebreak[2]}\Standort{CUL, Schnitzler, B 118.}
\physDesc{Brief, 1 Blatt, 2 Seiten, 639 Zeichen
\newline{}Handschrift: schwarze Tinte, lateinische Kurrent
\newline{}Schnitzler: 1) mit Bleistift »\textsc{Zweig}«  2) mit rotem Buntstift eine Unterstreichung}
\buchAbdrucke{\weitereDrucke{1) Stefan Zweig: \emph{Briefwechsel mit Hermann Bahr, Sigmund Freud, Rainer Maria
                        Rilke und Arthur Schnitzler}. Herausgegeben von Jeffrey B. Berlin,  Hans-Ulrich Lindken und  Donald A. Prater. Frankfurt am Main: \emph{S. Fischer} 1987, S. 397–398.} \weitereDrucke{2) Stefan Zweig: \emph{Briefe. Bd. II: 1914–1919}. Herausgegeben von Knut Beck,  Jeffrey B. Berlin und  Natascha Weschenbach-Feggeler. Frankfurt am Main: \emph{S. Fischer} 1998, S. 100–101.} }\toendnotes[C]{\smallbreak}
\pstart
           \raggedleft{}{\pb}18. Januar 1916\pend
           
\pstart
           \textcolor{gray}{\textbf{SZ}}\hfill \textcolor{gray}{\textbf{\textcolor{pink}{VIII. KOCHGASSE}\oindex{Wien@\textbf{Wien}!VIII., Josefstadt@\textbf{VIII., Josefstadt}!Kochgasse 8@\textbf{Kochgasse 8}, \emph{Wohngebäude}|pw}{}\ledrightnote{\textcolor{pink}{Kochgasse 8}}}}\pend
           
\pstart
           \raggedleft{}\textcolor{gray}{\textbf{\textcolor{pink}{WIEN}\oindex{Wien@\textbf{Wien}, \emph{Verwaltungsgebiet}|pw}{}\ledrightnote{\textcolor{pink}{Wien}},}}\pend
           
\pstart{}Lieber verehrter Herr Doktor,\pend\vspace{0.5em}
\pstart
           darf ich wieder einmal \label{K_L03657-1v}\edtext{zu Ihnen
                  kommen}{\lemma{\textnormal{\emph{zu Ihnen
                  kommen}}}\Cendnote{\textnormal{Vgl. A. S.: \emph{Tagebuch}, 21. 1. 1916.}}}\label{K_L03657-1}? Oder mögen Sie Menschen jetzt nicht
               sehen. Ich würde auch dies verstehn – die Worte und Gespräche werden einem manchmal
               jetzt verhasst, man weiss, wie nutzlos wie unwissend \substVorne{}\textsuperscript{S}\substDazwischen{}s\substHinten{}ie sind.\pend
           
\pstart
           Ich möchte bei dieser Gelegenheit auch Ihren Rat \label{K_L03657-2v}\edtext{in Sachen \textcolor{blue}{Rilkes}\pwindex{Rilke, Rainer Maria 4.\,12.\,1875 Prag – 29.\,12.\,1926 Montreux@\textsc{Rilke, Rainer Maria} (4.\,12.\,1875 Prag – 29.\,12.\,1926 Montreux), \emph{Schriftsteller}|pw}{}\ledrightnote{\textcolor{blue}{Rainer Maria Rilke}}}{\lemma{\textnormal{\emph{in Sachen Rilkes}}}\Cendnote{\textnormal{Beim Treffen am 21. 1. 1916
                  unterbreitete \textcolor{blue}{Zweig}\pwindex{Zweig, Stefan 28.\,11.\,1881 Wien – 23.\,2.\,1942 Petrópolis@\textsc{Zweig, Stefan} (28.\,11.\,1881 Wien – 23.\,2.\,1942 Petrópolis), \emph{Schriftsteller}|pwk}{ }\textcolor{blue}{Schnitzler} den Vorschlag einer Eingabe beim
                  zuständigen Minister. Also Folge der Aktivitäten \textcolor{blue}{Zweigs}\pwindex{Zweig, Stefan 28.\,11.\,1881 Wien – 23.\,2.\,1942 Petrópolis@\textsc{Zweig, Stefan} (28.\,11.\,1881 Wien – 23.\,2.\,1942 Petrópolis), \emph{Schriftsteller}|pwk} wurde \textcolor{blue}{Rilke}\pwindex{Rilke, Rainer Maria 4.\,12.\,1875 Prag – 29.\,12.\,1926 Montreux@\textsc{Rilke, Rainer Maria} (4.\,12.\,1875 Prag – 29.\,12.\,1926 Montreux), \emph{Schriftsteller}|pwk} nach der
                  Grundausbildung zu \textcolor{blue}{Zweig}\pwindex{Zweig, Stefan 28.\,11.\,1881 Wien – 23.\,2.\,1942 Petrópolis@\textsc{Zweig, Stefan} (28.\,11.\,1881 Wien – 23.\,2.\,1942 Petrópolis), \emph{Schriftsteller}|pwk} ins \emph{\textcolor{brown}{Kriegsarchiv}\orgindex{Kriegsarchiv@Kriegsarchiv|pwk}} versetzt. }}}\label{K_L03657-2} erbitten, der eingerückt
               ist und der (aus vielen Gründen) sehr leidet. Vielleicht könnten Wir durch eine
               gemeinsame Initiative ihm helfen. Und {\pb}wer verdient es, wenn nicht er?\pend
           
\pstart
           Getreulichst (mit vielen Grüssen an Ihre liebe \textcolor{blue}{Frau}\pwindex{Schnitzler, Olga 17.\,1.\,1882 Wien – 13.\,1.\,1970 Lugano@\textsc{Schnitzler, Olga} (17.\,1.\,1882 Wien – 13.\,1.\,1970 Lugano), \emph{Schauspielerin, Sängerin}|pwv}{}\ledrightnote{{$\rightarrow$}\emph{\textcolor{blue}{Olga Schnitzler}}} und Sie){\\[\baselineskip]} Ihr \spacefill\mbox{Stefan Zweig}\pend
           \leftskip=0em{}
\pstart
           \noindent{}\uline{P. S.} Ich bin (ausser Mittwoch)
                  immer frei.{\\}nachmittags oder abends.\pend
           \selectlanguage{ngerman}\endnumbering\briefempfaengerindex{, @\textsc{, }!zzz, @\emph{von  }!1916-01-181@{18. 1. 1916}|)be}\mylabel{L03657h}  \normalsize

\doendnotes{C}
\bigskip
\vfill

\clearpage

\footnotesize

\lohead{\textsc{register}}

% Definiere theindex-Environment komplett neu ohne reledmac
\makeatletter
\renewenvironment{theindex}{%
  \section*{\indexname}%
  \setlength{\parindent}{0pt}%
  \setlength{\parskip}{0pt plus 0.3pt}%
  \let\item\@idxitem
}{%
  \clearpage
}
\makeatother

\IfFileExists{\jobname-pw.ind}{\input{\jobname-pw.ind}}{}

\end{document}

      