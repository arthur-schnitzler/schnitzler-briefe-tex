%% latex-korrekturansicht-vorspann.tex
%% Vorspann für die Korrekturansicht.
%% Lädt die gemeinsame Datei latex-vorspann.tex mit gesetztem Schalter.

\newif\ifkorrekturansicht
\korrekturansichttrue

\input{../tex-inputs/latex-vorspann}


\renewcommand{\erwaehnteInstitutionen}{Institutionen: Internationale Ausstellung für Musik und Theaterwesen}
\renewcommand{\erwaehnteOrte}{Orte: Café Reichsrath (Inh. Karl Auböck), Mödling, Theater an der Wien, Wien}
\renewcommand{\erwaehnteWerke}{Werke: Pagliacci, Tagebuch}
\section[Felix Salten an Arthur Schnitzler, {[}26. 4. 1894?{]}]{Felix Salten an Arthur Schnitzler, {[}26. 4. 1894?{]}}
\nopagebreak\mylabel{v}
\rehead{ }\normalsize\beginnumbering\briefempfaengerindex{Schnitzler, Arthur@\textsc{Schnitzler, Arthur}!zzzSalten, Felix@\emph{von Felix Salten}!1894-04-262@{{[}26. 4. 1894?{]}}|(be}
\toendnotes[C]{\smallbreak\pagebreak[2]}\Standort{CUL, Schnitzler, B 89, A 1.}
\physDesc{Brief, 1 Blatt, 2 Seiten, 405 Zeichen
\newline{}Handschrift: Bleistift, lateinische Kurrent
\newline{}Schnitzler: mit Bleistift datiert: »92« 
\newline{}Ordnung: mit Bleistift von unbekannter Hand nummeriert: »23« }\toendnotes[C]{\smallbreak}
\pstart
           \noindent{}{\pb}Lieber Freund! Ihr Brief von gestern\textcolor{gray}{,} hat mich leider nicht zu Hause getroffen, ich kam den
                  Abend überhaupt nicht nach Hause, weil ich bei \label{K_L03120-1v}\edtext{\textcolor{green}{Pagliacci}{}\ledrightnote{\textcolor{green}{Pagliacci}}}{\lemma{\textnormal{\emph{Pagliacci}}}\Cendnote{\textnormal{Das Korrespondenzstück wurde von \textcolor{blue}{Salten} nicht datiert, die Datierung \textcolor{blue}{Schnitzler}s auf »92« ist falsch, da er erst am 13. 6. 1893 Fahrradfahren lernte und erst
                  ab dem 
                  19. 7. 1893 gemeinsame Ausfahrten mit 
                   \textcolor{blue}{Salten} unternahm. Obzwar 
                  \emph{\textcolor{green}{I Pagliacci}} erstmals am 17. 9. 1892 bei der \emph{\textcolor{brown}{Wiener Musik- und
                        Theaterausstellung}} in \textcolor{pink}{Wien}
                  gegeben wurde und danach einige Aufführungen folgten (\textcolor{blue}{Schnitzler} selbst sah das \textcolor{green}{Stück} am 25. 9. 1892), war die Oper erst wieder ab 19. 11. 1893
                  am Spielplan, diesmal in der \textcolor{pink}{Wien}er 
                  \emph{\textcolor{brown}{Hofoper}}. Das deutet auf den Frühling 1894 für dieses Korrespondenzstück,
                  weil dies zugleich den einzigen Zeitraum im \emph{\textcolor{green}{Tagebuch}} darstellt, 
                  an dem mehrere Radausflüge nach \textcolor{pink}{Mödling} belegt sind.
                  Unter der Annahme, dass \textcolor{blue}{Schnitzler} auch ohne 
                  \textcolor{blue}{Salten} einen Radausflug unternommen hat, kommt 
                  nur die Aufführung vom 26. 4. 1894 zur Datierung des
                  Korrespondenzstücks in Betracht.}}}\label{K_L03120-1h} war, und dann in der \textcolor{pink}{Stadt}{}\ledrightnote{{$\rightarrow$}\textcolor{pink}{Wien}} soupirte. Schade, dass ich nicht wusste, Sie sind im Café.
               Nach \textcolor{pink}{Mödling}{}\ledrightnote{\textcolor{pink}{Mödling}} kann ich heute auch nicht {\pb}fahren, weil das Bicycle gebrochen ist. Zeigen Sie mir an, wann Sie wieder ins \textcolor{pink}{Aubö\textcolor{gray}{ck}}{}\ledrightnote{\textcolor{pink}{Café Reichsrath (Inh. Karl Auböck)}} kommen, ich sehne mich schon wirklich danach\pend
           
\pstart
           Herzlich {\\[\baselineskip]}Ihr {\\[\baselineskip]}\spacefill\mbox{Salten}\pend
           \leftskip=0em{}\endnumbering\briefempfaengerindex{Schnitzler, Arthur@\textsc{Schnitzler, Arthur}!zzzSalten, Felix@\emph{von Felix Salten}!1894-04-262@{{[}26. 4. 1894?{]}}|)be}\mylabel{h}  \normalsize

\doendnotes{C}
\bigskip
\vfill

\clearpage

\footnotesize

\lohead{\textsc{register}}

% Definiere theindex-Environment komplett neu ohne reledmac
\makeatletter
\renewenvironment{theindex}{%
  \section*{\indexname}%
  \setlength{\parindent}{0pt}%
  \setlength{\parskip}{0pt plus 0.3pt}%
  \let\item\@idxitem
}{%
  \clearpage
}
\makeatother

\IfFileExists{\jobname-pw.ind}{\input{\jobname-pw.ind}}{}

\end{document}

      