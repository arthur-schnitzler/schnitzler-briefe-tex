%% latex-korrekturansicht-vorspann.tex
%% Vorspann für die Korrekturansicht.
%% Lädt die gemeinsame Datei latex-vorspann.tex mit gesetztem Schalter.

\newif\ifkorrekturansicht
\korrekturansichttrue

\input{../tex-inputs/latex-vorspann}


\renewcommand{\erwaehntePersonen}{Personen: Peter Cornelius, Paul Marx, Olga Schnitzler, Elisabeth Steinrück}
\renewcommand{\erwaehnteOrte}{Orte: Berlin, Dessauer Straße, Wien}
\renewcommand{\erwaehnteWerke}{Werke: Brautlieder}
\section[ Paul Goldmann an Olga Gussmann, 1. 4. {[}1901{]}]{Paul Goldmann an Olga Gussmann, 1. 4. {[}1901{]}}
\nopagebreak\mylabel{v}
\rehead{ }\normalsize\beginnumbering\briefempfaengerindex{Schnitzler, Olga@\textsc{Schnitzler, Olga}!zzzGoldmann, Paul@\emph{von Paul Goldmann}!1901-04-012@{1. 4. {[}1901{]}}|(be}
\toendnotes[C]{\smallbreak\pagebreak[2]}\Standort{DLA, A:Schnitzler, HS.NZ85.1.5247.}
\physDesc{Brief, 1 Blatt, 4 Seiten, 2559 Zeichen
\newline{}Handschrift: blaue Tinte, deutsche Kurrent
\newline{}Ordnung: mit Bleistift von \textcolor{blue}{Arthur
                                    Schnitzler} das Jahr »1901« vermerkt }\toendnotes[C]{\smallbreak}
\pstart
           \noindent{}\raggedleft{}{\pb}\textcolor{gray}{\textbf{\textcolor{pink}{DESSAUERSTRASSE 19}{}\ledrightnote{\textcolor{pink}{Dessauer Straße}}}}\pend
           
\pstart
           \textcolor{pink}{Berlin}{}\ledrightnote{\textcolor{pink}{Berlin}}, 1. April.\pend
           
\pstart\center{}Liebes Fräulein \textsc{Olga},\pend
\pstart
           Endlich einmal eine freie Stunde, nach arbeitsſchweren Tagen. Heut will ich erſt Ihren lieben Brief beantworten. Das \textcolor{blue}{Schweſterchen}{}\ledrightnote{{$\rightarrow$}\textcolor{blue}{Elisabeth Steinrück}} kommt nächſtens an die
               Reihe.\pend
           
\pstart
           Dieſer Ihr Brief war alſo ſehr ſchön. Ich ſage ihnen, es thut wohl, ein wenig
               geſtreichelt zu werden, – namentlich wenn man in dieſer Beziehung gar nicht, aber
               auch ſchon gar nicht verwöhnt iſt. Und doch, er kam ein wenig zu ſpät, dieſer Brief.
               Ich merke gar zu deutlich, daß mein lieber Freund \textsc{\textcolor{blue}{Arthur}{}\ledrightnote{}} hinter den Couliſſen die \label{K_L03524-1v}\edtext{Regie}{\lemma{\textnormal{\emph{Regie}}}\Cendnote{\textnormal{\textcolor{blue}{Goldmann} hatte sich jedenfalls von 6. 9. 1900 bis 16. 9. [1900] in \textcolor{pink}{Wien} aufgehalten. Am 6. 9. 1900, 8. 9. 1900, 9. 9. 1900 und 13. 9. 1900 hatten \textcolor{blue}{Goldmann}, \textcolor{blue}{Schnitzler} und \textcolor{blue}{Olga Gussmann}
                  gemeinsam Zeit verbracht.}}}\label{K_L03524-1h} führt. Ich habe ſchon aus \label{T_L03524-1v}\edtext{\textcolor{pink}{Wien}{}\ledrightnote{\textcolor{pink}{Wien}}}{\lemma{\textnormal{\emph{Wien}}}\Cendnote{\textnormal{korrigiert aus »den \textcolor{pink}{Wien}«}}}\label{T_L03524-1h}{ }{\pb}den Eindruck mitgebracht, \strikeout{\textcolor{gray}{×}\-\textcolor{gray}{×}\-\textcolor{gray}{×}} daß Sie auf mich nur aufmerkſam geworden ſind, weil ich Ihnen an der Seite
               dieſes meines lieben \textcolor{blue}{Freund}{}\ledrightnote{{$\rightarrow$}}es
               erſchienen bin. Sonſt wären Sie wahrſcheinlich an mir vorübergegangen, ohne mich zu
               ſehen. Ihre Briefe haben mir die Wahrnehmung beſtätigt. Natürlich werden Sie jetzt
               proteſtiren. Aber, glauben Sie mir, ich kenne ſo gut den Ton, den Diejenigen
               annehmen, die Einen verkennen. Ich höre ihn mit ſcharfem Ohr ſelbſt aus der
               Freundſchaft heraus. Ich bin ein Fachmann im Verkanntwerden.\pend
           
\pstart
           Und da ich müde bin, immer wieder das ſelbe zu erleben, ſelbſt bei den ganz Klugen
               (es gibt kluge Frauen, die doch {\pb}nur Denjenigen
               richtig beurtheilen, den ſie lieben), ſo habe ich Ihnen vielleicht nicht ſo oft
               geſchrieben, als ich es hätte thun ſollen. Das iſt aber keine Behandlung »\label{K_L03524-2v}\edtext{\textsc{\begin{otherlanguage}{french}en canaille\end{otherlanguage}}}{\lemma{\textnormal{\emph{en canaille}}}\Cendnote{\textnormal{französisch: verachtend}}}\label{K_L03524-2h}«,
               wahrhaftig nicht. Mit der Freundſchaſt hat das gar nichts zu thun. Ich will mit der
               Freundſchaft keine Geſchäfte machen, und es iſt mir ein ſehr feines und ein wenig
               weiſches Vergnügen, mehr geben zu können, als ich bekomme.\pend
           
\pstart
           Vielleicht hätte ich das, was ich Ihnen, liebes Fräulein \textsc{Olga}, da erzählt habe, gar nicht geſpürt, wenn ich \strikeout{\textcolor{gray}{×}} nur einen einzigen Menſchen hätte (ſtatt Menſch iſt natürlich »Frau« zu
               leſen), der ſich für mich intereſſirt und der mich lieb hat. Aber ich habe {\pb}Niemand. So ſitze ich in der Einſamkeit und fange
               Grillen. Dieſer Brief iſt nichts als eine große gefangene Grille. Sie werden ihn als
                  ſolche{[}n{]} behandeln und darüber lachen. Aber jetzt, wo mir die
                  Oſterſonne zum Fenſter hereinſcheint, wird es \uline{gar} ſchlimm. Die dumme Frage regt ſich wieder, warum
               es für die ganze Welt Frühling wird und warum ich allein davon ausgenommen ſein ſoll?
               Es iſt ſchwer, ſeinen Gleichmuth zu bewahren, wenn einem ſo eine Frage im Kopfe
               rumort.\pend
           \pstart Liebes Fräulein ich möchte wiſſen, wie es Ihnen und dem lieben \textcolor{blue}{Schweſterchen}{}\ledrightnote{{$\rightarrow$}\textcolor{blue}{Elisabeth Steinrück}} geht. Und die \label{K_L03524-3v}\edtext{\textcolor{green}{Brautlieder}{}\ledrightnote{{$\rightarrow$}\textcolor{green}{Brautlieder}} von \textsc{\textcolor{blue}{Cornelius}{}\ledrightnote{\textcolor{blue}{Peter Cornelius}}}}{\lemma{\textnormal{\emph{Brautlieder von Cornelius}}}\Cendnote{\textnormal{eine \textcolor{green}{Komposition} aus dem Jahr 1856,
                  die \textcolor{blue}{Olga Gussmann} vermutlich gesanglich
                  einübte}}}\label{K_L03524-3h} möchte ich auch wohl einmal hören. Schreiben Sie mir bald wieder!
               Und fröhliche Oſtern! Ihr \spacefill\mbox{Dr. Paul
                  Goldmann.}\pend{}
\pstart
           \noindent{}{\pb}\label{T_L03524-3v}\edtext{Herzliche Grüße an Sie \textcolor{blue}{Beide}{}\ledrightnote{} und an Herrn \textsc{\textcolor{blue}{Paul}{}\ledrightnote{\textcolor{blue}{Paul Marx}}}!}{\lemma{\textnormal{\emph{Herzliche … Paul!}}}\Cendnote{\textnormal{am Kopf der ersten Seite, verkehrt zum Text}}}\label{T_L03524-3h}\pend
           \endnumbering\briefempfaengerindex{Schnitzler, Olga@\textsc{Schnitzler, Olga}!zzzGoldmann, Paul@\emph{von Paul Goldmann}!1901-04-012@{1. 4. {[}1901{]}}|)be}\mylabel{h}
\begin{anhang}
\end{anhang}\normalsize

\doendnotes{C}
\bigskip
\vfill

\clearpage

\footnotesize

\lohead{\textsc{register}}

% Definiere theindex-Environment komplett neu ohne reledmac
\makeatletter
\renewenvironment{theindex}{%
  \section*{\indexname}%
  \setlength{\parindent}{0pt}%
  \setlength{\parskip}{0pt plus 0.3pt}%
  \let\item\@idxitem
}{%
  \clearpage
}
\makeatother

\IfFileExists{\jobname-pw.ind}{\input{\jobname-pw.ind}}{}

\end{document}

      