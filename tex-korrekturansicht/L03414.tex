%% latex-korrekturansicht-vorspann.tex
%% Vorspann für die Korrekturansicht.
%% Lädt die gemeinsame Datei latex-vorspann.tex mit gesetztem Schalter.

\newif\ifkorrekturansicht
\korrekturansichttrue

\input{../tex-inputs/latex-vorspann}


\renewcommand{\erwaehntePersonen}{Personen: Hermann Bahr, Hans Oberländer, Max Reinhardt, Ottilie Salten, Olga Schnitzler}
\renewcommand{\erwaehnteInstitutionen}{Institutionen: Kammerspiele Berlin, Kleines Theater}
\renewcommand{\erwaehnteOrte}{Orte: Berlin, Wien}
\renewcommand{\erwaehnteWerke}{Werke: Zum großen Wurstel. Burleske in einem Akt}
\section[ Felix Salten an Arthur Schnitzler, 24. 3. 1906]{Felix Salten an Arthur Schnitzler, 24. 3. 1906}
\nopagebreak\mylabel{v}
\rehead{ }\normalsize\beginnumbering\briefempfaengerindex{Schnitzler, Arthur@\textsc{Schnitzler, Arthur}!zzzSalten, Felix@\emph{von Felix Salten}!1906-03-241@{24. 3. 1906}|(be}
\toendnotes[C]{\smallbreak\pagebreak[2]}\Standort{CUL, Schnitzler, B 89, B 1.}
\physDesc{Brief, 1 Blatt, 1 Seite, 873 Zeichen (Briefpapier mit Trauerrand)
\newline{}Handschrift: schwarze Tinte, lateinische Kurrent
\newline{}Ordnung: mit Bleistift von unbekannter Hand nummeriert: »205« }\toendnotes[C]{\smallbreak}
\pstart
           \raggedleft{}{\pb}\textcolor{pink}{Berlin}{}\ledrightnote{\textcolor{pink}{Berlin}}, 24. III. 06.\pend
           
\pstart
           Lieber, in Eile und Arbeit nur ganz kurz: gegen das \label{K_L03414-1v}\edtext{»\textcolor{brown}{Kleine
                  Theater}{}\ledrightnote{\textcolor{brown}{Kleines Theater}}«}{\lemma{\textnormal{\emph{»Kleine
                  Theater«}}}\Cendnote{\textnormal{Er beantwortet die
                  Frage, ob es für eine Inszenierung von \emph{\textcolor{green}{Zum großen
                     Wurstel}} in Frage käme, vgl. A. S.: \emph{Tagebuch}, 25. 3. 1906}}}\label{K_L03414-1h} bin ich unbedingt. Es ist mit seinem jetzigen Bestand an Schauspielern, und
               der retorischen Unfähigkeit des Herrn D\textsuperscript{r}{ }\textcolor{blue}{Oberländer}{}\ledrightnote{\textcolor{blue}{Hans Oberländer}} garnicht imstande ein so
               stilisirtes und in seinen Reizen vom Dutzend-Regisseur so schwer auffindbares \textcolor{green}{Stück}{}\ledrightnote{{$\rightarrow$}\textcolor{green}{Zum großen Wurstel. Burleske in einem Akt}} zu
               reproduziren. Ich hielte es für aussichtslos. Auch wäre, bei der jetzigen Conjunctur
               von so einem Experiment nur abzurathen. Besser, Sie warten auf \textcolor{blue}{Reinhardt}{}\ledrightnote{\textcolor{blue}{Max Reinhardt}}s \label{K_L03414-2v}\edtext{»\textcolor{brown}{intimes Theater}{}\ledrightnote{\textcolor{brown}{Kammerspiele Berlin}}«, das im nächsten Jahr bestehen und von 
               \textcolor{blue}{Bahr}{}\ledrightnote{\textcolor{blue}{Hermann Bahr}} geleitet}{\lemma{\textnormal{\emph{»intimes … geleitet}}}\Cendnote{\textnormal{\textcolor{blue}{Max Reinhardt}s \emph{\textcolor{brown}{Kammerspiele}}, die 
                  als kleinere Bühne für experimentellere und anspruchsvollere Stücke Verwendung finden sollten. \textcolor{blue}{Bahr} arbeitete zwar in Folge für mehrere Inszenierungen als Regisseur
                  bei \textcolor{blue}{Reinhardt}, doch tatsächliche
                  Verantwortung als Theaterleiter bekam er nicht übertragen. Nach vier Aufenthalten zwischen November 1906
               und März 1908 endete die Zusammenarbeit.}}}\label{K_L03414-2h} wird. Folgen Sie
               mir!\pend
           
\pstart
           Ich schreibe bald und mehr. Dass wir einander wieder herzlich nah sind, empfinde ich
               auch, und es hat mir meinen Abgang von \textcolor{pink}{Wien}{}\ledrightnote{\textcolor{pink}{Wien}}
               erschwert. Dass etwas Unverlierbares, an das jederzeit ohneweiters angeknüpft werden
               kann, uns verbindet, hab ich immer geglaubt. Viele Grüße von \textcolor{blue}{Otti}{}\ledrightnote{\textcolor{blue}{Ottilie Salten}} u. mir an Sie \textcolor{blue}{Beide}{}\ledrightnote{{$\rightarrow$}\textcolor{blue}{Olga Schnitzler}}.\pend
           \pstart Ihr \spacefill\mbox{Salten}\pend{}\endnumbering\briefempfaengerindex{Schnitzler, Arthur@\textsc{Schnitzler, Arthur}!zzzSalten, Felix@\emph{von Felix Salten}!1906-03-241@{24. 3. 1906}|)be}\mylabel{h}  \normalsize

\doendnotes{C}
\bigskip
\vfill

\clearpage

\footnotesize

\lohead{\textsc{register}}

% Definiere theindex-Environment komplett neu ohne reledmac
\makeatletter
\renewenvironment{theindex}{%
  \section*{\indexname}%
  \setlength{\parindent}{0pt}%
  \setlength{\parskip}{0pt plus 0.3pt}%
  \let\item\@idxitem
}{%
  \clearpage
}
\makeatother

\IfFileExists{\jobname-pw.ind}{\input{\jobname-pw.ind}}{}

\end{document}

      