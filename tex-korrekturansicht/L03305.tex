%% latex-korrekturansicht-vorspann.tex
%% Vorspann für die Korrekturansicht.
%% Lädt die gemeinsame Datei latex-vorspann.tex mit gesetztem Schalter.

\newif\ifkorrekturansicht
\korrekturansichttrue

\input{../tex-inputs/latex-vorspann}


\renewcommand{\erwaehntePersonen}{Personen: Hermann Bahr, Emerich von Bukovics, Paul Schlenther}
\renewcommand{\erwaehnteInstitutionen}{Institutionen: Burgtheater, Volkstheater}
\renewcommand{\erwaehnteOrte}{Orte: Wien}
\renewcommand{\erwaehnteWerke}{Werke: Der Schleier der Beatrice. Schauspiel in fünf Akten}
\section[ Felix Salten an Arthur Schnitzler, {[}20. 6. 1900{]}]{Felix Salten an Arthur Schnitzler, {[}20. 6. 1900{]}}
\nopagebreak\mylabel{v}
\rehead{ }\normalsize\beginnumbering\briefempfaengerindex{Schnitzler, Arthur@\textsc{Schnitzler, Arthur}!zzzSalten, Felix@\emph{von Felix Salten}!1900-06-202@{{[}20. 6. 1900{]}}|(be}
\toendnotes[C]{\smallbreak\pagebreak[2]}\Standort{CUL, Schnitzler, B 89, A 2.}
\physDesc{Brief, 1 Blatt, 1 Seite, 747 Zeichen ({\pb}die Rückseite
                                 weist das Blatt als Abriss eines mit schwarzer Tinte beschriebenen
                                 Blattes aus)
\newline{}Handschrift: Bleistift, lateinische Kurrent
\newline{}Schnitzler: mit Bleistift datiert: »20/6 900.« 
\newline{}Ordnung: mit Bleistift von unbekannter Hand nummeriert: »129« }
\buchAbdrucke{\weitereDrucke{Hermann Bahr, Arthur Schnitzler: \emph{Briefwechsel, Aufzeichnungen, Dokumente (1891–1931)}. Hg. Kurt Ifkovits und Martin Anton Müller. Göttingen: \emph{Wallstein} 2018, S. 176.} }\toendnotes[C]{\smallbreak}
\pstart
           \noindent{}{\pb}Lieber, ich war eben bei Ihnen, um Ihnen folgendes zu sagen:
               Überlegen Sie, ob Sie nicht lieber gleich zum \label{K_L03305-1v}\edtext{\textcolor{brown}{Volksth.}{}\ledrightnote{\textcolor{brown}{Volkstheater}}}{\lemma{\textnormal{\emph{Volksth.}}}\Cendnote{\textnormal{Seit Februar des Jahres glaubte \textcolor{blue}{Schnitzler},
                  \emph{\textcolor{green}{Der Schleier der Beatrice}} wäre vom \emph{\textcolor{brown}{Burgtheater}} zur Uraufführung angenommen. Direktor \textcolor{blue}{Schlenther}
                  teilte \textcolor{blue}{Schnitzler} aber am 18. 6. 1900
                  mit, dass er die Annahme noch überlege. \textcolor{blue}{Schnitzler} besprach bereits
                  am Folgetag die Sachlage mit \textcolor{blue}{Salten}, vgl. A. S.: \emph{Tagebuch}, 18. 6. 1900.
                   Zu einer
                  Aufführung durch das \emph{\textcolor{brown}{Volkstheater}} kam
                  nicht.}}}\label{K_L03305-1h} gehen wollen. In diesem Fall wäre die Nachricht von der Annahme
               Ihres \textcolor{green}{Stück}{}\ledrightnote{{$\rightarrow$}\textcolor{green}{Der Schleier der Beatrice. Schauspiel in fünf Akten}}es am \textcolor{brown}{Volksth.}{}\ledrightnote{\textcolor{brown}{Volkstheater}} die vorläufig beste Antwort für \textcolor{blue}{Schlenther}{}\ledrightnote{\textcolor{blue}{Paul Schlenther}}. Und dem \textcolor{brown}{Volksth.}{}\ledrightnote{\textcolor{brown}{Volkstheater}} gegenüber wären Sie jetzt in der Lage zu sagen, dass
               Ihnen \uline{der Termin}{ }\uline{des \textcolor{brown}{Burgtheater}{}\ledrightnote{\textcolor{brown}{Burgtheater}}s}
               nicht passt, während Sie, falls Sie ein Refus von \textcolor{blue}{Schlenth.}{}\ledrightnote{\textcolor{blue}{Paul Schlenther}} provoziren, mit einem abgelehnten \textcolor{green}{Stück}{}\ledrightnote{{$\rightarrow$}\textcolor{green}{Der Schleier der Beatrice. Schauspiel in fünf Akten}} zu \textcolor{blue}{Bukovics}{}\ledrightnote{\textcolor{blue}{Emerich von Bukovics}}
               kommen, der vielleicht daraus wieder Capital schlägt, und Ihnen sagt, (von \textcolor{blue}{Bahr}{}\ledrightnote{\textcolor{blue}{Hermann Bahr}} gehetzt) dass Sie nur das für ihn haben,
               was \textcolor{blue}{Schlenther}{}\ledrightnote{\textcolor{blue}{Paul Schlenther}} übrig läßt. Ganz abgesehen
               davon, dass \textcolor{blue}{Sch.}{}\ledrightnote{\textcolor{blue}{Paul Schlenther}} – wenn er von Ihnen keine
               Antwort kriegt, und nur hört, Ihr \textcolor{green}{Stück}{}\ledrightnote{{$\rightarrow$}\textcolor{green}{Der Schleier der Beatrice. Schauspiel in fünf Akten}} sei am \textcolor{brown}{Volksth.}{}\ledrightnote{\textcolor{brown}{Volkstheater}} – gewiß gelaufen
               kommt. ec. ec. ec.\pend
           \pstart Herzl. \spacefill\mbox{Salten}\pend{}\endnumbering\briefempfaengerindex{Schnitzler, Arthur@\textsc{Schnitzler, Arthur}!zzzSalten, Felix@\emph{von Felix Salten}!1900-06-202@{{[}20. 6. 1900{]}}|)be}\mylabel{h}  \normalsize

\doendnotes{C}
\bigskip
\vfill

\clearpage

\footnotesize

\lohead{\textsc{register}}

% Definiere theindex-Environment komplett neu ohne reledmac
\makeatletter
\renewenvironment{theindex}{%
  \section*{\indexname}%
  \setlength{\parindent}{0pt}%
  \setlength{\parskip}{0pt plus 0.3pt}%
  \let\item\@idxitem
}{%
  \clearpage
}
\makeatother

\IfFileExists{\jobname-pw.ind}{\input{\jobname-pw.ind}}{}

\end{document}

      