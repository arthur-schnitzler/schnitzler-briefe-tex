%% latex-korrekturansicht-vorspann.tex
%% Vorspann für die Korrekturansicht.
%% Lädt die gemeinsame Datei latex-vorspann.tex mit gesetztem Schalter.

\newif\ifkorrekturansicht
\korrekturansichttrue

\input{../tex-inputs/latex-vorspann}


\renewcommand{\erwaehntePersonen}{Personen: Hermann Bahr, Charlotte Pohl-Glas}
\renewcommand{\erwaehnteInstitutionen}{Institutionen: Berliner Neueste Nachrichten, Münchener General-Anzeiger}
\renewcommand{\erwaehnteOrte}{Orte: Hörlgasse, Wien}
\renewcommand{\erwaehnteWerke}{Werke: Studien zur Kritik der Moderne}
\section[ Felix Salten an Arthur Schnitzler, {[}11. 9. 1894{]}]{Felix Salten an Arthur Schnitzler, {[}11. 9. 1894{]}}
\nopagebreak\mylabel{v}
\rehead{ }\normalsize\beginnumbering\briefempfaengerindex{Schnitzler, Arthur@\textsc{Schnitzler, Arthur}!zzzSalten, Felix@\emph{von Felix Salten}!1894-09-111@{{[}11. 9. 1894{]}}|(be}
\toendnotes[C]{\smallbreak\pagebreak[2]}\Standort{CUL, Schnitzler, B 89, A 1.}
\physDesc{Visitenkarte, 168 Zeichen
\newline{}Handschrift: Bleistift, lateinische Kurrent
\newline{}Schnitzler: mit Bleistift datiert: »11/9 94« 
\newline{}Ordnung: mit Bleistift von unbekannter Hand nummeriert: »46« }
\buchAbdrucke{\weitereDrucke{Hermann Bahr, Arthur Schnitzler: \emph{Briefwechsel, Aufzeichnungen, Dokumente (1891–1931)}. Hg. Kurt Ifkovits und Martin Anton Müller. Göttingen: \emph{Wallstein} 2018, S. 80.} }\toendnotes[C]{\smallbreak}
\pstart
           \noindent{}\centering{}{\pb}\textcolor{gray}{\textbf{FELIX SALTEN}}\pend
           
\pstart
           \noindent{}\textcolor{gray}{\textbf{\textcolor{pink}{WIEN}{}\ledrightnote{\textcolor{pink}{Wien}},}}\hfill \textcolor{gray}{\textbf{»\textcolor{brown}{Berliner Neueste
                           Nachrichten}{}\ledrightnote{\textcolor{brown}{Berliner Neueste Nachrichten}}.«}}\pend
           
\pstart
           \textcolor{gray}{\textbf{\textcolor{pink}{IX., Hörlgasse 16}{}\ledrightnote{\textcolor{pink}{Hörlgasse}}.}}\hfill \textcolor{gray}{\textbf{»\textcolor{brown}{Münchener
                           General-Anzeiger}{}\ledrightnote{\textcolor{brown}{Münchener General-Anzeiger}}.«}}\pend
           
\pstart
           {\pb}Dear Sir,\pend
           
\pstart
           To-day, I cannot \label{K_L03145-1v}\edtext{glide with you}{\lemma{\textnormal{\emph{glide with you}}}\Cendnote{\textnormal{Er
                  dürfte das deutsche Wort »gleiten« übersetzen, hier eventuell als Ausdruck für
                  ›Radfahren‹ verwendet.}}}\label{K_L03145-1h} because I must visit the p\substVorne{}\textsuperscript{\textcolor{gray}{oore}}\substDazwischen{}oor\substHinten{} little \textcolor{blue}{girl}{}\ledrightnote{{$\rightarrow$}\textcolor{blue}{Charlotte Pohl-Glas}} in the
                  \label{K_L03145-2v}\edtext{prison}{\lemma{\textnormal{\emph{prison}}}\Cendnote{\textnormal{siehe Felix Salten an Arthur Schnitzler, 7. 8. 1894}}}\label{K_L03145-2h}, You must excuse me.\pend
           
\pstart
           Perhaps you can sent the \label{K_L03145-3v}\edtext{\textcolor{green}{\substVorne{}\textsuperscript{Bo}\substDazwischen{}bo\substHinten{}ok}{}\ledrightnote{{$\rightarrow$}\textcolor{green}{Studien zur Kritik der Moderne}}}{\lemma{\textnormal{\emph{book}}}\Cendnote{\textnormal{\emph{\textcolor{green}{Studien zur Kritik der Moderne}}?}}}\label{K_L03145-3h} from
                  \textcolor{blue}{H. Bahr}{}\ledrightnote{\textcolor{blue}{Hermann Bahr}}?\pend
           
\pstart
           Yours {\\[\baselineskip]}\spacefill\mbox{Salten}\pend
           \leftskip=0em{}\endnumbering\briefempfaengerindex{Schnitzler, Arthur@\textsc{Schnitzler, Arthur}!zzzSalten, Felix@\emph{von Felix Salten}!1894-09-111@{{[}11. 9. 1894{]}}|)be}\mylabel{h}  \normalsize

\doendnotes{C}
\bigskip
\vfill

\clearpage

\footnotesize

\lohead{\textsc{register}}

% Definiere theindex-Environment komplett neu ohne reledmac
\makeatletter
\renewenvironment{theindex}{%
  \section*{\indexname}%
  \setlength{\parindent}{0pt}%
  \setlength{\parskip}{0pt plus 0.3pt}%
  \let\item\@idxitem
}{%
  \clearpage
}
\makeatother

\IfFileExists{\jobname-pw.ind}{\input{\jobname-pw.ind}}{}

\end{document}

      