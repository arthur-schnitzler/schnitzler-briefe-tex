%% latex-korrekturansicht-vorspann.tex
%% Vorspann für die Korrekturansicht.
%% Lädt die gemeinsame Datei latex-vorspann.tex mit gesetztem Schalter.

\newif\ifkorrekturansicht
\korrekturansichttrue

\input{../tex-inputs/latex-vorspann}


               \section[Hermann Bahr an Arthur Schnitzler, 16. 12. {[}1903{]}]{ Hermann Bahr an Arthur Schnitzler, 16. 12. {[}1903{]}}\nopagebreak\mylabel{v}\rehead{ }\normalsize\beginnumbering\briefempfaengerindex{Schnitzler, Arthur@\textsc{Schnitzler, Arthur}!zzzBahr, Hermann@\emph{von Hermann Bahr}!1903-12-161@{16. 12. {[}1903{]}}|(be} \toendnotes[C]{\smallbreak\pagebreak[2]} \Standort{CUL, Schnitzler, B 5b.}
\physDesc{Brief, 1 Blatt, 2 Seiten
\newline{}Handschrift: schwarze Tinte, deutsche Kurrent
\newline{}Schnitzler: mit Bleistift Jahreszahl ergänzt: »903.« \newline{}Ordnung: mit Bleistift von unbekannter Hand nummeriert: »104« }\buchAbdrucke{\weitereDrucke{Hermann Bahr, Arthur Schnitzler: \emph{Briefwechsel, Aufzeichnungen, Dokumente (1891–1931)}. Hg. Kurt Ifkovits und Martin Anton Müller. Göttingen: \emph{Wallstein} 2018, S. 286.} }\toendnotes[C]{\smallbreak}\pstart
           \raggedleft{}{\pb}16. 12.\pend
           \pstart\center{}Lieber Arthur!\pend\pstart
           Herzlichſten Dank für Dein liebes Telegramm. Und die beſten Grüße von \textcolor{blue}{Brahm}{}\ledrightnote{\textcolor{blue}{Otto Brahm}}, \textcolor{blue}{Fiſcher}{}\ledrightnote{\textcolor{blue}{Samuel Fischer}}
               und allen möglichen Leuten.\pend
           \pstart
           Im \textcolor{brown}{Tageblatt}{}\ledrightnote{\textcolor{brown}{Berliner Tageblatt}} hatte man mir ſchon beinahe
               verſprochen, den \textcolor{green}{Rek\substVorne{}\textsuperscript{ou}\substDazwischen{}urs\substHinten{}}{}\ledrightnote{→\textcolor{green}{Reigen. Zehn Dialoge}} an die \textcolor{brown}{Statthalterei}{}\ledrightnote{\textcolor{brown}{Niederösterreichische Statthalterei}} abzudrucken,
               dann haben ſie aber vorgeſtern blos eine einzige Stelle abgedruckt und dies auch noch
               mit ſehr dummen Bemerkungen. Viel geſcheiter ſind ſie ja in \textcolor{pink}{Berlin}{}\ledrightnote{\textcolor{pink}{Berlin}} auch nicht als {\pb}bei
               uns, ſondern nur etwas anſtändiger.\pend
           \pstart
           Ich hoffe Dich bald zu ſehen. Mit den beſten Grüßen an Deine \textcolor{blue}{Frau}{}\ledrightnote{→\textcolor{blue}{Olga Schnitzler}}\pend
           \pstart
           herzlichſt{\\[\baselineskip]}\spacefill\mbox{H.}\pend
           \leftskip=0em{}\endnumbering\briefempfaengerindex{Schnitzler, Arthur@\textsc{Schnitzler, Arthur}!zzzBahr, Hermann@\emph{von Hermann Bahr}!1903-12-161@{16. 12. {[}1903{]}}|)be}\mylabel{h}  \normalsize

\doendnotes{C}
\bigskip
\vfill

\clearpage

\footnotesize

\lohead{\textsc{register}}

% Definiere theindex-Environment komplett neu ohne reledmac
\makeatletter
\renewenvironment{theindex}{%
  \section*{\indexname}%
  \setlength{\parindent}{0pt}%
  \setlength{\parskip}{0pt plus 0.3pt}%
  \let\item\@idxitem
}{%
  \clearpage
}
\makeatother

\IfFileExists{\jobname-pw.ind}{\input{\jobname-pw.ind}}{}

\end{document}

      