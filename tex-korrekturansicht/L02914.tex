%% latex-korrekturansicht-vorspann.tex
%% Vorspann für die Korrekturansicht.
%% Lädt die gemeinsame Datei latex-vorspann.tex mit gesetztem Schalter.

\newif\ifkorrekturansicht
\korrekturansichttrue

\input{../tex-inputs/latex-vorspann}


         
         \renewcommand{\erwaehntePersonen}{Personen: A. Beit, Georg von Bleichröder, Rosa Freudenthal, Lydia von Fulmen, Carl von Lang-Puchhof, Sándor Petőfi, Karl August von Schmieder, W. Warne}
         \renewcommand{\erwaehnteInstitutionen}{Institutionen: Rennstall Lang-Puchhof und Schmieder}
         \renewcommand{\erwaehnteOrte}{Orte: Berlin, Cadore, Dessauer Straße, Deutschland, Hamburg, Hamburg-Groß Borstel, Hoppegarten, Köln, Rheinland, Wien}
         \renewcommand{\erwaehnteWerke}{Werke: Die Quellen des Nil, Liebelei. Schauspiel in drei Akten, Nemzeti dal, Reigen. Zehn Dialoge, Ritterlichkeit, [Rennbericht, Pferd Liebelei]}
               \section[ Paul Goldmann an Arthur Schnitzler, 2. 5. {[}1900{]}]{Paul Goldmann an Arthur Schnitzler, 2. 5. {[}1900{]}}\nopagebreak\mylabel{v}\rehead{ }\normalsize\beginnumbering\briefempfaengerindex{Schnitzler, Arthur@\textsc{Schnitzler, Arthur}!zzzGoldmann, Paul@\emph{von Paul Goldmann}!1900-05-021@{2. 5. {[}1900{]}}|(be} \toendnotes[C]{\smallbreak\pagebreak[2]} \Standort{DLA, A:Schnitzler, HS.NZ85.1.3170.}
\physDesc{Brief, 1 Blatt, 3 Seiten
\newline{}Handschrift: blaue Tinte, deutsche Kurrent\newline{}Beilage: ein Zeitungsausschnitt, beschnitten }\toendnotes[C]{\smallbreak}\pstart
           \noindent{}{\pb}\textcolor{pink}{\textcolor{gray}{\textbf{DESSAUERSTRASSE 19}}}{}\ledrightnote{\textcolor{pink}{Dessauer Straße}}\pend
           \pstart
           \raggedleft{}\textcolor{pink}{Berlin}{}\ledrightnote{\textcolor{pink}{Berlin}}, 2. Mai.\pend
           \pstart{}Mein lieber Freund,\pend\pstart
           In aller Eile Dank für Deinen lieben Brief!\pend
           \pstart
           Mich hat die \label{K_L02914-1v}\edtext{\textcolor{blue}{Frau
                  Rechtsanwalt}{}\ledrightnote{{$\rightarrow$}\textcolor{blue}{Rosa Freudenthal}}}{\lemma{\textnormal{\emph{Frau
                  Rechtsanwalt}}}\Cendnote{\textnormal{siehe Paul Goldmann an Arthur Schnitzler, 20. 2. 1900}}}\label{K_L02914-1h} um den »\textcolor{green}{Reigen}{}\ledrightnote{\textcolor{green}{Reigen. Zehn Dialoge}}« erſucht. Ich hielt mich
               aber nicht für berechtigt, der \textcolor{blue}{Frau}{}\ledrightnote{{$\rightarrow$}\textcolor{blue}{Rosa Freudenthal}} das \textcolor{green}{Buch}{}\ledrightnote{\textcolor{green}{Reigen. Zehn Dialoge}} zu geben, und habe mich
               damit ausgeredet, ich hätte es verborgt.\pend
           \pstart
           Wie Du aus beifolgendem \label{K_L02914-3v}\edtext{\textcolor{green}{Rennbericht}{}\ledrightnote{{$\rightarrow$}\textcolor{green}{[Rennbericht, Pferd Liebelei]}}}{\lemma{\textnormal{\emph{Rennbericht}}}\Cendnote{\textnormal{es ist unklar, aus welcher Zeitung der
                     \textcolor{green}{Ausschnitt}
                  stammt}}}\label{K_L02914-3h} ſiehſt, iſt hier beim letzten Rennen ein Pferd »\textcolor{green}{Liebelei}{}\ledrightnote{\textcolor{green}{Liebelei. Schauspiel in drei Akten}}« gelaufen. Es {\pb}gehört \label{K_L02914-11v}\edtext{einem ſüd\textcolor{pink}{deutſch}{}\ledrightnote{{$\rightarrow$}\textcolor{pink}{Deutschland}}en \textcolor{blue}{Beſitzer}{}\ledrightnote{{$\rightarrow$}\textcolor{blue}{Carl von Lang-Puchhof}{\newline}{$\rightarrow$}\textcolor{blue}{Karl August von Schmieder}}}{\lemma{\textnormal{\emph{einem … Beſitzer}}}\Cendnote{\textnormal{Das Pferd »\emph{\textcolor{green}{Liebelei}}« gehörte \textcolor{blue}{Carl
                     von Lang-Puchhof} und \textcolor{blue}{Karl August von
                     Schmieder}, die von 1898 bis 1907 einen \textcolor{brown}{Pferderennstall} in \textcolor{pink}{Hoppegarten}
                  betrieben. \textcolor{blue}{Goldmann} bezog sich vermutlich
                  auf den \textcolor{pink}{Rheinländ}er \textcolor{blue}{Lang-Puchhof}.}}}\label{K_L02914-11h} und heißt offenbar
               nach Deinem \textcolor{green}{Stück}{}\ledrightnote{{$\rightarrow$}\textcolor{green}{Liebelei. Schauspiel in drei Akten}}. Dies iſt
               der Ruhm, mein lieber Freund!\pend
           \pstart
           Es freut mich ſehr, zu hören, daß Du eine \label{K_L02914-5v}\edtext{\textcolor{green}{Poſſe}{}\ledrightnote{{$\rightarrow$}\textcolor{green}{Ritterlichkeit}}}{\lemma{\textnormal{\emph{Poſſe}}}\Cendnote{\textnormal{womöglich Bezug auf das Fragment
                  gebliebene und erst posthum veröffentlichte Drama \emph{\textcolor{green}{Ritterlichkeit}}, das \textcolor{blue}{Schnitzler} am
                     23. 4. 1900
                  vorläufig unter dem Titel »\textcolor{green}{Drama}« beendet hatte}}}\label{K_L02914-5h} geſchrieben haſt. So biſt Du \strikeout{\textcolor{gray}{×}} auf halbem Wege zu dem \label{K_L02914-7v}\edtext{Luſtſpiel}{\lemma{\textnormal{\emph{Luſtſpiel}}}\Cendnote{\textnormal{In der Korrespondenz mit
                     \textcolor{blue}{Goldmann} ist davon mehrfach die Rede:
                     vgl. Paul Goldmann an Arthur Schnitzler, 8. 12. [1893], Paul Goldmann an Arthur Schnitzler, 23. 12. [1893] und Paul Goldmann an Arthur Schnitzler, 2. [1.? 1897]. Im Sommer dieses Jahres arbeitete \textcolor{blue}{Schnitzler} an \emph{\textcolor{green}{Die Quellen
                     des Nil}} weiter, vgl. Arthur Schnitzler an Hugo von Hofmannsthal, 17. 7. 1900. }}}\label{K_L02914-7h}, das ich nicht ablaſſen werde, von Dir zu verlangen.\pend
           \pstart
           Nächſtens mehr! Heut habe ich nur zwei Minuten.\pend
           \pstart
           {\pb}Viele treue Grüße! {\\[\baselineskip]}Dein {\\[\baselineskip]}\spacefill\mbox{Paul Goldmann}\pend
           \leftskip=0em{}{\bigskip}\pstart
           \noindent{}\textcolor{gray}{\textbf{{\pb}Unter den Pferden, die bereits »was gezeigt haben«
                  fallen ganz beſonders }}\textcolor{green}{\textcolor{gray}{\textbf{\so{Liebelei}}}}{}\ledrightnote{{$\rightarrow$}\textcolor{green}{Liebelei. Schauspiel in drei Akten}}\textcolor{gray}{\textbf{, die Dritte zu Over Norton und Seraphine im Großen \textcolor{pink}{Köln}{}\ledrightnote{\textcolor{pink}{Köln}}iſchen Handicap und \textcolor{pink}{\so{Cadore}}{}\ledrightnote{\textcolor{pink}{Cadore}}, der mit friſchem Lorbeer gekrönte Sieger des \textcolor{pink}{Hamburg}{}\ledrightnote{\textcolor{pink}{Hamburg}}er Godeffroy-Rennens, auf. Für die \textcolor{pink}{Hamburg}{}\ledrightnote{\textcolor{pink}{Hamburg}}er Ueberraſchung muß der \textcolor{blue}{Bleichröder}{}\ledrightnote{\textcolor{blue}{Georg von Bleichröder}}’ſche Wallach volle zehn Pfund mehr aufnehmen und wir glauben
                  offen geſtanden nicht, daß es dem Dreijährigen mit dem hohen Gewicht von 55½\textsc{kg} gelingen wird, die Situation zu beherrſchen.}}{ }\textcolor{green}{\textcolor{gray}{\textbf{\so{Liebelei}}}}{}\ledrightnote{{$\rightarrow$}\textcolor{green}{Liebelei. Schauspiel in drei Akten}}\textcolor{gray}{\textbf{{ }iſt viel besſſer daran. Zwar drücken 64½\textsc{kg} auch, aber die \label{K_L02914-88v}\edtext{\textcolor{green}{Talpra-Magyar}{}\ledrightnote{{$\rightarrow$}\textcolor{green}{Nemzeti dal}}-Tochter}{\lemma{\textnormal{\emph{Talpra-Magyar-Tochter}}}\Cendnote{\textnormal{»Talpra-Magyar« war eines der begehrtesten Zuchtpferde der Zeit, benannt nach
                        den ersten beiden Worten des revolutionären Gedichts \emph{\textcolor{green}{Nemzeti dal}} (1848) von \textcolor{blue}{Sándor Petőfi}.}}}\label{K_L02914-88h} iſt ein Pferd mit reellen
                  Fähigkeiten – ein »Frühjahrspferd«, – das auch in \textcolor{pink}{Köln}{}\ledrightnote{\textcolor{pink}{Köln}} eine gute Leiſtung vollbrachte. Seitdem ſoll ſie ſich ganz
                  weſentlich verbeſſert haben. Wir würden ihr auch ohne Bedenken unſere Sympathien
                  zuwenden, wenn der }}\textcolor{pink}{\textcolor{gray}{\textbf{\so{Borſtel}}}}{}\ledrightnote{{$\rightarrow$}\textcolor{pink}{Hamburg-Groß Borstel}}\textcolor{gray}{\textbf{\so{er Stall}, der augenblicklich auf der Höhe ſteht, nicht
                  \so{Heroine}, die im Gewicht außerordentlich begünſtigt
                  iſt, im Rennen hätte. Wie aus guter Quelle verlautet, iſt Heroine in
                  ausgezeichneter Verfaſſung und ſll ihren \textcolor{blue}{Trainer}{}\ledrightnote{{$\rightarrow$}\textcolor{blue}{A. Beit}} in der Arbeit ſehr befriedigt
                  haben. Man wird gut thun, der \textcolor{blue}{Fulmen-Tochter}{}\ledrightnote{{$\rightarrow$}\textcolor{blue}{Lydia von Fulmen}} für das große Rennen die gebührende
                  Beachtung zu ſchenken. \so{Nicolo} iſt ebenfalls nicht
                  ſchlecht im Handicap, jedoch nicht in Form. Sein Laufen in \textcolor{pink}{Köln}{}\ledrightnote{\textcolor{pink}{Köln}} war durchaus nicht berühmt und wir glauben kaum, daß
                  von ihm eine Ueberraſchung zu erwarten iſt. Eher von \textsc{X},
                  der von \textcolor{blue}{Warne}{}\ledrightnote{\textcolor{blue}{W. Warne}} geſteuert, bei der günſtigen
                  Diſtanz durchaus nicht ohne Chancen iſt. \so{Connex} und \so{Radler} Schluß dürfte aber doch}}\pend
           \pstart
           \centering{}\textcolor{gray}{\textbf{\textbf{Heroine}}}\pend
           \pstart
           \noindent{}\textcolor{gray}{\textbf{das beſſere Ende vor}}{ }\textcolor{green}{\textcolor{gray}{\textbf{\so{Liebelei}}}}{}\ledrightnote{\textcolor{green}{Liebelei. Schauspiel in drei Akten}}\textcolor{gray}{\textbf{{ }und \textsc{X} behalten.}}\pend
           \endnumbering\briefempfaengerindex{Schnitzler, Arthur@\textsc{Schnitzler, Arthur}!zzzGoldmann, Paul@\emph{von Paul Goldmann}!1900-05-021@{2. 5. {[}1900{]}}|)be}\mylabel{h}\begin{anhang}\end{anhang}\normalsize

\doendnotes{C}
\bigskip
\vfill

\clearpage

\footnotesize

\lohead{\textsc{register}}

% Definiere theindex-Environment komplett neu ohne reledmac
\makeatletter
\renewenvironment{theindex}{%
  \section*{\indexname}%
  \setlength{\parindent}{0pt}%
  \setlength{\parskip}{0pt plus 0.3pt}%
  \let\item\@idxitem
}{%
  \clearpage
}
\makeatother

\IfFileExists{\jobname-pw.ind}{\input{\jobname-pw.ind}}{}

\end{document}

      