%% latex-korrekturansicht-vorspann.tex
%% Vorspann für die Korrekturansicht.
%% Lädt die gemeinsame Datei latex-vorspann.tex mit gesetztem Schalter.

\newif\ifkorrekturansicht
\korrekturansichttrue

\input{../tex-inputs/latex-vorspann}


\renewcommand{\erwaehntePersonen}{Personen: Joseph Arthur de Gobineau, Baruch de Spinoza, Stefan Zweig}
\renewcommand{\erwaehnteOrte}{Orte: Paschinger Schlössl, Salzburg, Wien}
\renewcommand{\erwaehnteWerke}{Werke: Der Geist im Wort und der Geist in der Tat, Der Kampf mit dem Dämon. Hölderlin – Kleist – Nietzsche, Die Baumeister der Welt. Versuch einer Typologie des Geistes, Drei Dichter ihres Lebens. Casanova – Stendhal – Tolstoi, Drei Meister. Balzac – Dickens – Dostojewski}
\section[Stefan Zweig an Arthur Schnitzler, 4. 2. 1927]{Stefan Zweig an Arthur Schnitzler, 4. 2. 1927}
\nopagebreak\mylabel{v}
\rehead{ }\normalsize\beginnumbering\briefempfaengerindex{Schnitzler, Arthur@\textsc{Schnitzler, Arthur}!zzzZweig, Stefan@\emph{von Stefan Zweig}!1927-02-041@{4. 2. 1927}|(be}
\toendnotes[C]{\smallbreak\pagebreak[2]}\Standort{CUL, Schnitzler, B 118.}
\physDesc{Brief, 1 Blatt, 2 Seiten, 2462 Zeichen
\newline{}Handschrift: blaue Tinte, lateinische Kurrent
\newline{}Schnitzler: mit rotem Buntstift 13 Unterstreichungen und beschriftet: »\noindent{}\textsc{Zweig}{ / }(\textcolor{green}{Diagram}« }\toendnotes[C]{\smallbreak}
\pstart
           {\pb}\textcolor{gray}{\textbf{SZ}}\hfill \textcolor{gray}{\textbf{\textcolor{pink}{SALZBURG}{}\ledrightnote{\textcolor{pink}{Salzburg}}}}{ }4. II 1927\pend
           
\pstart
           \raggedleft{}\textcolor{gray}{\textbf{\textcolor{pink}{KAPUZINERBERG 5}{}\ledrightnote{\textcolor{pink}{Paschinger Schlössl}}}}\pend
           
\pstart
           Lieber verehrter Herr Doktor, ich war, wie wohl alle, erst
               überrascht, von Ihnen ein characterologisches \textcolor{green}{Buch}{}\ledrightnote{\textcolor{green}{Der Geist im Wort und der Geist in der Tat}} zu empfangen, doch gleichzeitig sehr neugierig gereizt, wie ein so
               verantwortliches Problem bei Ihnen Lösung finde. Sie wissen ja, dass mein
               essayistisches \textcolor{green}{Hauptwerk}{}\ledrightnote{{$\rightarrow$}\textcolor{green}{Die Baumeister der Welt. Versuch einer Typologie des Geistes}} von
               dem nur zwei \textcolor{green}{Bände}{}\ledrightnote{{$\rightarrow$}\textcolor{green}{Drei Meister. Balzac – Dickens – Dostojewski}{\newline}{$\rightarrow$}\textcolor{green}{Der Kampf mit dem Dämon. Hölderlin – Kleist – Nietzsche}}
               bisher erschienen sind, eine »\label{K_L03673-1v}\edtext{\textcolor{green}{Typologie des Geistes}{}\ledrightnote{{$\rightarrow$}\textcolor{green}{Die Baumeister der Welt. Versuch einer Typologie des Geistes}}}{\lemma{\textnormal{\emph{Typologie des Geistes}}}\Cendnote{\textnormal{Der Reihentitel lautet vollständig:
                     »Die Baumeister der Welt. Versuch einer Typologie des Geistes«
                  und wurde 1928 noch um den Band \emph{\textcolor{green}{Drei
                     Dichter ihres Lebens. Casanova – Stendhal – Tolstoi}} erweitert.}}}\label{K_L03673-1h}«
               sein will, also die ganzen Identitäten in Varianten aufzeigen; so war Ihre
               Formulierung mir eine Art Bestätigung und insbesondere jener tragende Gedanke, dass
               jede Erscheinung ihren Schatten wirft wie ein organisches Gebilde, dass jeder Sinn
               tätig seinen Widersinn, seine Verzerrung in der irdischen Erscheinung erschafft, will
               mir ausserordentlich fruchtbar erscheinen. Dazu formt sich die Abwandlung durchaus
               klar: more geometrico im Sinne unseres \textcolor{blue}{Spinoza}{}\ledrightnote{\textcolor{blue}{Baruch de Spinoza}}
               und auch das Widerspiel fehlt nicht, amor intellectualis, die rein geistige Liebe zur
               beinahe metaphysischen Problematik. Ich bin für Sie dieses kleinen \textcolor{green}{Büchleins}{}\ledrightnote{\textcolor{green}{Der Geist im Wort und der Geist in der Tat}} sehr froh, denn die Menschen nehmen den Künstler am
               liebsten dort, wo er leicht und locker wird, in ihre Wertung auf. Hier werden manche
               über den sachlichen Ernst erstaunen, der in Ihnen die Urmacht ist – ich freilich
               erstaune nicht, ich weiss ja auch von Ihren verstreuten und leider noch nicht
               gesammelten Reflexionen über die Kunst, wie sehr Sie die innerliche Mechanik dessen
               beschäftigt, was nach aussen hin als Selbstverständlich-Wirkendes erscheint. Es wäre
               mir innige Freude, einmal ausführlich mit Ihnen über diese Probleme sprechen zu
               dürfen: im tiefsten Grunde sind Sie damit dem Sinn der Zeit nahegekommen, die endlich
               – endlich! – müde wird der collectiven Typenlehre von den »Rassen« und »Nationen« wie
               sie \textcolor{blue}{Gobineau}{}\ledrightnote{\textcolor{blue}{Joseph Arthur de Gobineau}} in die Welt setzte und die \strikeout{individueller} Einordnung in den \introOben{}Individual-\introOben{}Typus begehrt. Das haben Sie mit {\pb}dieser kleinen \textcolor{green}{Studie}{}\ledrightnote{\textcolor{green}{Der Geist im Wort und der Geist in der Tat}}, die nur den Rand zu berühren scheint, in Wahrheit
               aber auf das Wesentliche zielt, sehr gefördert.\pend
           
\pstart
           Ich fahre jetzt ein wenig nach Süden, hoffentlich in neue Arbeit hinein. Das letzte
               Jahr war äusserlich so gut zu mir, dass ich nun doppelt anpruchsvoll wider mich sein
               muss, um den unerwarteten Erfolg nicht zu dementieren. Aber je schwerer sie wird,
               desto lieber hat man die Arbeit: ich weiss, es geht Ihnen ebenso und nie war Ihr
               geistiger Ertrag fülliger und bedeutsamer als in den letzten Jahren.\pend
           
\pstart
           Muss ich noch besonders sagen, wie sehr und innig ich Ihnen anhänge? Ich hoffe,
               Sie wissen’s und gedenken freundlich Ihres getreuen{\\[\baselineskip]}\spacefill\mbox{Stefan Zweig}\pend
           \leftskip=0em{}\endnumbering\briefempfaengerindex{Schnitzler, Arthur@\textsc{Schnitzler, Arthur}!zzzZweig, Stefan@\emph{von Stefan Zweig}!1927-02-041@{4. 2. 1927}|)be}\mylabel{h}
\begin{anhang}
\end{anhang}\normalsize

\doendnotes{C}
\bigskip
\vfill

\clearpage

\footnotesize

\lohead{\textsc{register}}

% Definiere theindex-Environment komplett neu ohne reledmac
\makeatletter
\renewenvironment{theindex}{%
  \section*{\indexname}%
  \setlength{\parindent}{0pt}%
  \setlength{\parskip}{0pt plus 0.3pt}%
  \let\item\@idxitem
}{%
  \clearpage
}
\makeatother

\IfFileExists{\jobname-pw.ind}{\input{\jobname-pw.ind}}{}

\end{document}

      