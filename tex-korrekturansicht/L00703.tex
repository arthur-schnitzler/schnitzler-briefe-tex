%% latex-korrekturansicht-vorspann.tex
%% Vorspann für die Korrekturansicht.
%% Lädt die gemeinsame Datei latex-vorspann.tex mit gesetztem Schalter.

\newif\ifkorrekturansicht
\korrekturansichttrue

\input{../tex-inputs/latex-vorspann}


               \section[Hugo von Hofmannsthal an Arthur Schnitzler, 16. {[}7. 1897{]}]{ Hugo von Hofmannsthal an Arthur Schnitzler,
                    16. {[}7. 1897{]}}\nopagebreak\mylabel{v}\rehead{ }\normalsize\beginnumbering\briefempfaengerindex{Schnitzler, Arthur@\textsc{Schnitzler, Arthur}!zzzHofmannsthal, Hugo von@\emph{von Hugo von Hofmannsthal}!1897-07-161@{16. {[}7. 1897{]}}|(be} \toendnotes[C]{\smallbreak\pagebreak[2]} \Standort{CUL, Schnitzler, B 43.}
\physDesc{Brief, 1 Blatt, 4 Seiten
\newline{}Handschrift: schwarze Tinte, deutsche Kurrent
\newline{}Schnitzler: mit Bleistift Monat und Jahreszahl ergänzt: »7 97« \newline{}Ordnung: 1) mit Bleistift von unbekannter Hand nummeriert:
                                    »97« 2) mit Bleistift von unbekannter Hand nummeriert:
                                    »94«}\buchAbdrucke{\weitereDrucke{Hugo von Hofmannsthal, Arthur Schnitzler: \emph{Briefwechsel}. Hg. Therese Nickl und Heinrich Schnitzler. Frankfurt am Main: \emph{S. Fischer} 1964, S. 92.} }\toendnotes[C]{\smallbreak}\pstart
           \raggedleft{}{\pb}\textcolor{pink}{Fuſch}{}\ledrightnote{\textcolor{pink}{Bad Fusch}}{ }16\textsuperscript{ten}.\pend
           \pstart{}mein lieber Arthur\pend\pstart
           ich danke herzlich für Brief und Vorſchlag. Auch den \textcolor{green}{\textcolor{blue}{Mozart}{}\ledrightnote{\textcolor{blue}{Wolfgang Amadeus Mozart}}band}{}\ledrightnote{→\textcolor{green}{W. A. Mozart}} hab ich bekommen. Es
                    thut mir ſehr ſehr leid, daſs es mit \textcolor{pink}{Salzburg}{}\ledrightnote{\textcolor{pink}{Salzburg}}
                    nicht zuſammengeht und wenn es ein geringerer Grund wäre als der völlig
                    zuſammengebrochene Zuſtand \textcolor{blue}{Poldys}{}\ledrightnote{\textcolor{blue}{Leopold von Andrian-Werburg}} der mich
                    ſehr nötig braucht und den ich in dieſen nächſten 14 Tagen nicht mehr Stunden
                    allein laſſen will, {\pb}als
                    täglich meine Arbeit nöthig macht, ſo würde ich noch jetzt trachten, es möglich
                    zu machen. Auch hab ich eine kleine \textcolor{green}{Arbeit}{}\ledrightnote{→\textcolor{green}{Geschichte eines österreichischen Officiers}} in Verſen angefangen, deren \label{K_L00703_1v}\edtext{Hintergrund}{\lemma{\textnormal{\emph{Hintergrund}}}\Cendnote{\textnormal{In seinen Aufzeichnungen (\textcolor{blue}{Hugo von Hofmannsthal}: \emph{Aufzeichnungen}. Hg. Rudolf Hirsch † und Ellen
                            Ritter † in Zusammenarbeit mit Konrad Heumann und Peter Michael
                            Braunwarth. Frankfurt am Main: \emph{\textcolor{brown}{S. Fischer}}{ }2013, S. 381 (\emph{Sämtliche Werke},
                            XXXIX)) erwähnt \textcolor{blue}{Hofmannsthal}
                        eine Stiftsdame aus \textcolor{pink}{Salzburg} für die Arbeit
                        an der zu Lebzeiten unveröffentlicht gebliebenen \emph{\textcolor{green}{Geschichte eines österreichischen Officiers}}.}}}\label{K_L00703_1h}
                    etwas mit \textcolor{pink}{Salzburg}{}\ledrightnote{\textcolor{pink}{Salzburg}} zu thun hat und habe mich in
                    übertriebener Weiſe darauf gefreut, es Euch dort, wo wir immer ſo glücklich
                    zuſammen waren, vorzuleſen. Dieſe kleine \textcolor{green}{Arbeit}{}\ledrightnote{→\textcolor{green}{Geschichte eines österreichischen Officiers}} wird freilich jetzt {\pb}durch das finſtere
                    regneriſche Wetter etwas verzögert und wäre wohl erſt Ende Juli
                    fertig geworden.\pend
           \pstart
           Auf Euren Vorſchlag möchte ich am liebſten folgendes antworten: wenn das Wetter
                    gut wird und Ihr nur etwas Luſt habt die ſchöne Radtour zu machen (\textcolor{pink}{\uline{Salzburg}}{}\ledrightnote{\textcolor{pink}{Salzburg}} – \textcolor{pink}{Berchtesgaden}{}\ledrightnote{\textcolor{pink}{Berchtesgaden}} – \textcolor{pink}{Ramſau}{}\ledrightnote{\textcolor{pink}{Ramsau bei Berchtesgaden}} – \textcolor{pink}{Hirſchbichel}{}\ledrightnote{\textcolor{pink}{Hirschbichl}} –
                        \textcolor{pink}{Saalfelden}{}\ledrightnote{\textcolor{pink}{Saalfelden am Steinernen Meer}} – \textcolor{pink}{\uline{Zell a See}}{}\ledrightnote{\textcolor{pink}{Zell am See}}; wozu \textcolor{pink}{Lofer}{}\ledrightnote{\textcolor{pink}{Lofer}}?) ſo macht ſie und
                    verſtändigt {\pb}mich unmittelbar
                    vorher \introOben{}recht genau\introOben{}, damit ich rechtzeitig
                    hinunterkommen eventuell ein Stück (\textcolor{pink}{Saalfelden}{}\ledrightnote{\textcolor{pink}{Saalfelden am Steinernen Meer}}!) entgegenfahren kann. Geht es dann wegen \textcolor{blue}{Poldy}{}\ledrightnote{\textcolor{blue}{Leopold von Andrian-Werburg}} oder anderm nicht, ſo habt Ihr doch nichts
                    ſchlechtes gemacht.\pend
           \pstart
           Herzlich Ihr{\\[\baselineskip]}\spacefill\mbox{Hugo.}\pend
           \leftskip=0em{}\endnumbering\briefempfaengerindex{Schnitzler, Arthur@\textsc{Schnitzler, Arthur}!zzzHofmannsthal, Hugo von@\emph{von Hugo von Hofmannsthal}!1897-07-161@{16. {[}7. 1897{]}}|)be}\mylabel{h}  \normalsize

\doendnotes{C}
\bigskip
\vfill

\clearpage

\footnotesize

\lohead{\textsc{register}}

% Definiere theindex-Environment komplett neu ohne reledmac
\makeatletter
\renewenvironment{theindex}{%
  \section*{\indexname}%
  \setlength{\parindent}{0pt}%
  \setlength{\parskip}{0pt plus 0.3pt}%
  \let\item\@idxitem
}{%
  \clearpage
}
\makeatother

\IfFileExists{\jobname-pw.ind}{\input{\jobname-pw.ind}}{}

\end{document}

      