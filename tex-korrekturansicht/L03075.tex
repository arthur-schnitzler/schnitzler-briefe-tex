%% latex-korrekturansicht-vorspann.tex
%% Vorspann für die Korrekturansicht.
%% Lädt die gemeinsame Datei latex-vorspann.tex mit gesetztem Schalter.

\newif\ifkorrekturansicht
\korrekturansichttrue

\input{../tex-inputs/latex-vorspann}


\renewcommand{\erwaehntePersonen}{Personen: Richard Beer-Hofmann, Olga Schnitzler, Elisabeth Steinrück}
\renewcommand{\erwaehnteOrte}{Orte: Dolomiten, Etablissement Werzer, Madonna di Campiglio, Pörtschach, Vahrn}
\renewcommand{\erwaehnteWerke}{}
\section[ Paul Goldmann an Arthur Schnitzler, 29. 7. {[}1901{]}]{Paul Goldmann an Arthur Schnitzler, 29. 7. {[}1901{]}}
\nopagebreak\mylabel{v}
\rehead{ }\normalsize\beginnumbering\briefempfaengerindex{Schnitzler, Arthur@\textsc{Schnitzler, Arthur}!zzzGoldmann, Paul@\emph{von Paul Goldmann}!1901-07-291@{29. 7. {[}1901{]}}|(be}
\toendnotes[C]{\smallbreak\pagebreak[2]}\Standort{DLA, A:Schnitzler, HS.NZ85.1.3171.}
\physDesc{Brief, 1 Blatt, 2 Seiten
\newline{}Handschrift: blaue Tinte, deutsche Kurrent
\newline{}Schnitzler: 1) mit schwarzer Tinte das Jahr »{[}1{]}901« vermerkt  2) mit rotem Buntstift eine Unterstreichung}\toendnotes[C]{\smallbreak}
\pstart
           {\pb}\textsc{\textcolor{pink}{Pörtschach}{}\ledrightnote{\textcolor{pink}{Pörtschach}}}, 29. Juli.\pend
           
\pstart\center{}Mein lieber Freund,\pend
\pstart
           Ich danke Dir für Deinen lieben Brief und Deine Forſchungsreiſen. Finde nur etwas
               Hohes und Kühles. Hier iſt es mir zu lau und die Luft iſt mir zu matt. Trozdem bleibe
               ich wohl eine Woche hier, weil ich ein wenig das Beiſammenſein mit \textsc{\textcolor{blue}{Richard}{}\ledrightnote{\textcolor{blue}{Richard Beer-Hofmann}}} genießen will. Könnteſt Du nicht igend etwas in den \textcolor{pink}{Dolomiten}{}\ledrightnote{\textcolor{pink}{Dolomiten}}, ſo um \textsc{\textcolor{pink}{Madonna \strikeout{di\textcolor{gray}{e}} di Campiglio}{}\ledrightnote{\textcolor{pink}{Madonna di Campiglio}}} herum, \label{K_L03075-3v}\edtext{finden}{\lemma{\textnormal{\emph{finden}}}\Cendnote{\textnormal{siehe Paul Goldmann an Arthur Schnitzler, 26. 4. [1901]}}}\label{K_L03075-3h}? Was geht uns die Geſellſchaft an, wenn \strikeout{\textcolor{gray}{×}} wir {\pb}mit einander ſind? Nach
               einem warmen Ort komme ich nicht. Ich ſchlafe keine Nacht und brauche ſtarke Luft, um
               Schlaf zu finden.\pend
           
\pstart
           Wenn Du Dich zu einer Niederlaſſung etſchloſſen haft, ſo ſende mir Nachricht hieher,
                  \textcolor{pink}{Etabliſſement \textsc{Werzer}}{}\ledrightnote{\textcolor{pink}{Etablissement Werzer}}, \substVorne{}\textsuperscript{Zimmer}{\allowbreak}\substDazwischen{}\textsc{Villa}\substHinten{} 8, Zimmer 31.\pend
           
\pstart
           Viele Grüße Dir und den lieblichen \textcolor{blue}{Schweſtern}{}\ledrightnote{{$\rightarrow$}\textcolor{blue}{Olga Schnitzler}{\newline}{$\rightarrow$}\textcolor{blue}{Elisabeth Steinrück}}! {\\[\baselineskip]}Dein {\\[\baselineskip]}\spacefill\mbox{Paul Goldmnn}\pend
           \leftskip=0em{}\endnumbering\briefempfaengerindex{Schnitzler, Arthur@\textsc{Schnitzler, Arthur}!zzzGoldmann, Paul@\emph{von Paul Goldmann}!1901-07-291@{29. 7. {[}1901{]}}|)be}\mylabel{h}
\begin{anhang}
\end{anhang}\normalsize

\doendnotes{C}
\bigskip
\vfill

\clearpage

\footnotesize

\lohead{\textsc{register}}

% Definiere theindex-Environment komplett neu ohne reledmac
\makeatletter
\renewenvironment{theindex}{%
  \section*{\indexname}%
  \setlength{\parindent}{0pt}%
  \setlength{\parskip}{0pt plus 0.3pt}%
  \let\item\@idxitem
}{%
  \clearpage
}
\makeatother

\IfFileExists{\jobname-pw.ind}{\input{\jobname-pw.ind}}{}

\end{document}

      