%% latex-korrekturansicht-vorspann.tex
%% Vorspann für die Korrekturansicht.
%% Lädt die gemeinsame Datei latex-vorspann.tex mit gesetztem Schalter.

\newif\ifkorrekturansicht
\korrekturansichttrue

\input{../tex-inputs/latex-vorspann}


               \section[Max Burckhard an Arthur Schnitzler, {[}Juni 1907?{]}]{ Max Burckhard an Arthur Schnitzler, {[}Juni 1907?{]}}\nopagebreak\mylabel{v}\rehead{ }\normalsize\beginnumbering\briefempfaengerindex{Schnitzler, Arthur@\textsc{Schnitzler, Arthur}!zzzBurckhard, Max Eugen@\emph{von Max Eugen Burckhard}!1907-06-011@{{[}Juni 1907?{]}}|(be} \toendnotes[C]{\smallbreak\pagebreak[2]} \Standort{CUL, Schnitzler, B 20.}
\physDesc{Brief, 1 Blatt, 1 Seite
\newline{}Handschrift: schwarze Tinte, deutsche Kurrent\newline{}Ordnung: von Schnitzler mit
                                    Bleistift datiert: »So{\geminationm}er 907«, von unbekannter Hand mit Bleistift nummeriert:
                                        »18« }\toendnotes[C]{\smallbreak}\pstart
           \noindent{}{\pb}\textcolor{gray}{\textbf{D\textsuperscript{r.} Max Burckhard}}\hfill \textcolor{gray}{\textbf{\textcolor{pink}{Wien, IX. Porzellangasse 48}{}\ledrightnote{\textcolor{pink}{Porzellangasse}}{ }..........}}\pend
           \pstart
           \raggedleft{}\textcolor{gray}{\textbf{\textcolor{pink}{St. Gilgen}{}\ledrightnote{\textcolor{pink}{St. Gilgen}}}}\hspace*{3.5em}\pend
           \pstart{}Sehr verehrter lieber Herr Doctor!\pend\pstart
           Das \label{K_L01681_1v}\edtext{Wirtshaus}{\lemma{\textnormal{\emph{Wirtshaus}}}\Cendnote{\textnormal{\textcolor{blue}{Schnitzler} ist am
                            28. 6. 1907 in der Unterkunft. Entsprechend dürfte die
                        Empfehlung vorher übermittelt worden sein. Die Angabe \textcolor{blue}{Schnitzler}s »So{\geminationm}er 907«, sofern sie sich nicht einzig am Zeitpunkt der Reise
                        orientieren sollte, erlaubt eine Einschränkung auf Juni.}}}\label{K_L01681_1h}
                    heißt »\textcolor{pink}{Die Wochein}{}\ledrightnote{\textcolor{pink}{Die Wochein}}«, hat einen See \introOben{}(\textcolor{pink}{Wocheinersee}{}\ledrightnote{\textcolor{pink}{Wocheiner See}})\introOben{}{ }\uline{u.} gute Küche, liegt 2 Stunden ober \textcolor{pink}{Veldes}{}\ledrightnote{\textcolor{pink}{Veldes}} (leider geht jetzt eine Bahn hin), es
                    wird von der \textcolor{blue}{Frau}{}\ledrightnote{→\textcolor{blue}{Friederike Stöhr}} des
                    Malers \textcolor{blue}{Stöhr}{}\ledrightnote{\textcolor{blue}{Ernst Stöhr}} bewirtſchaftet. Es ſoll \uline{nicht} heiß ſein im So{\geminationm}er. Schöne Gemsjagden, alſo auch Gemſen
                    vorhanden!\pend
           \pstart
           Herzlichſt{\\[\baselineskip]}\spacefill\mbox{DrBurckhard}\pend
           \leftskip=0em{}\endnumbering\briefempfaengerindex{Schnitzler, Arthur@\textsc{Schnitzler, Arthur}!zzzBurckhard, Max Eugen@\emph{von Max Eugen Burckhard}!1907-06-011@{{[}Juni 1907?{]}}|)be}\mylabel{h}  \normalsize

\doendnotes{C}
\bigskip
\vfill

\clearpage

\footnotesize

\lohead{\textsc{register}}

% Definiere theindex-Environment komplett neu ohne reledmac
\makeatletter
\renewenvironment{theindex}{%
  \section*{\indexname}%
  \setlength{\parindent}{0pt}%
  \setlength{\parskip}{0pt plus 0.3pt}%
  \let\item\@idxitem
}{%
  \clearpage
}
\makeatother

\IfFileExists{\jobname-pw.ind}{\input{\jobname-pw.ind}}{}

\end{document}

      