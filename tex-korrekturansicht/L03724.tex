%% latex-korrekturansicht-vorspann.tex
%% Vorspann für die Korrekturansicht.
%% Lädt die gemeinsame Datei latex-vorspann.tex mit gesetztem Schalter.

\newif\ifkorrekturansicht
\korrekturansichttrue

\input{../tex-inputs/latex-vorspann}


\section[Elsa Plessner an Arthur Schnitzler, 10. 1. 1900]{L03724 Elsa Plessner an Arthur Schnitzler, 10. 1. 1900}
\nopagebreak\mylabel{L03724v}
\rehead{ }\normalsize\beginnumbering\briefempfaengerindex{Schnitzler, Arthur@\textsc{Schnitzler, Arthur}!zzzPlessner, Elsa@\emph{von Elsa Plessner}!1900-01-101@{10. 1. 1900}|(be}
\toendnotes[C]{\smallbreak\pagebreak[2]}
\correspDesc{Versand  durch Elsa Plessner am 10. 1. 1900 in Wien
\newline{}Erhalt  durch Arthur Schnitzler im Zeitraum [10. 1. 1900
                  – 13. 1. 1900?] in Wien}\toendnotes[C]{\smallbreak}
\Standort{DLA, A:Schnitzler, HS.1985.1.419.}
\physDesc{Brief, 1 Blatt, 4 Seiten, 2016 Zeichen
\newline{}Handschrift: schwarze Tinte, lateinische Kurrent}\toendnotes[C]{\smallbreak}
\pstart
           {\pb}\textcolor{pink}{Wien I. Kärnthnerstraße 10}\oindex{Wien@\textbf{Wien}!I., Innere Stadt@\textbf{I., Innere Stadt}!Kärntner Straße 10@\textbf{Kärntner Straße 10}, \emph{Wohngebäude}|pw}{}\ledrightnote{\textcolor{pink}{Kärntner Straße 10}}\pend
           
\pstart
           \raggedleft{}den 10. Januar 1900\pend
           
\pstart{}Verehrter Herr Doctor!\pend\vspace{0.5em}
\pstart
           So schnell!! Dafür danke ich Ihnen doppelt!\pend
           
\pstart
           Ihr heutiger{ }\label{K_L03724-1v}\edtext{Brief}{\lemma{\textnormal{\emph{Brief}}}\Cendnote{\textnormal{nicht überliefert}}}\label{K_L03724-1} hat mir viel Freude gemacht. Sie
               haben nicht über »Schlamperei« und »Leichtsinn« geschimpft, wie sonst immer – das ist
               für mich der größte Erfolg! – – Sehr überrascht war ich, dass Sie die
               Theaterwirksamkeit \textcolor{gray}{»}des »\textcolor{green}{ersten
                  C.}\pwindex{Plessner, Elsa 22.\,8.\,1875 Wien – 7.\,5.\,1932 Alicante@\textsc{Plessner, Elsa} (22.\,8.\,1875 Wien – 7.\,5.\,1932 Alicante), \emph{Schriftstellerin}!erste Kapitel. Schauspiel in drei Akten@\strich\emph{Das erste Kapitel. Schauspiel in drei Akten}|pw}{}\ledrightnote{\textcolor{green}{Das erste Kapitel. Schauspiel in drei Akten}}« in Abrede stellen. Zugegeben dass der Stoff eigentlich für eine Novelle
               gepasst hätte – ich selbst habe ihn darauf{\pb}hin
               ernstlich studiert –, bot er mir andrerseits durch die zahlreichen auch in der
               Novelle nothwendigen Scènen – d. h. Dialoge, durch die Steigerung der Handlung und
               deren geringe Zeitdauer (1 ½ Tage) unleugbare dramatische, ja sogar
               Bühnenmöglichkeiten. Sie haben ja ganz recht – der Stoff ist sehr dünn und ich habe
               das nicht übersehen – aber er hat mich trotzdem gereizt – und ich will doch die Probe
               auf die Bühnentragfähigkeit machen. – Als Erstlingsstück ist es rettungslos dem
               Durchfallen geweiht – das weiß ich. – Aber als zweites – auf einen {\pb}gewissen literarischen Credit hin, will ich den Versuch
               einer Aufführung wagen. –\pend
           
\pstart
           D. h. ein auswärtiges großes Theater wird gegen Ende März ein
               anderes \textcolor{green}{Stück}\pwindex{Plessner, Elsa 22.\,8.\,1875 Wien – 7.\,5.\,1932 Alicante@\textsc{Plessner, Elsa} (22.\,8.\,1875 Wien – 7.\,5.\,1932 Alicante), \emph{Schriftstellerin}!Ehrlosen. Schauspiel in drei Acten@\strich\emph{Die Ehrlosen. Schauspiel in drei Acten}|pwuv}{}\ledrightnote{{$\rightarrow$}\emph{\textcolor{green}{Die Ehrlosen. Schauspiel in drei Acten}}} von
               mir \label{K_L03724-2v}\edtext{aufführen}{\lemma{\textnormal{\emph{aufführen}}}\Cendnote{\textnormal{Eine Inszenierung eines ihrer Stücke außerhalb von \textcolor{pink}{Wien}\oindex{Wien@\textbf{Wien}, \emph{Verwaltungsgebiet}|pwk} ist nicht nachgewiesen. Es dürfte
                  jedenfalls von \emph{\textcolor{green}{Die Ehrlosen}\pwindex{Plessner, Elsa 22.\,8.\,1875 Wien – 7.\,5.\,1932 Alicante@\textsc{Plessner, Elsa} (22.\,8.\,1875 Wien – 7.\,5.\,1932 Alicante), \emph{Schriftstellerin}!Ehrlosen. Schauspiel in drei Acten@\strich\emph{Die Ehrlosen. Schauspiel in drei Acten}|pwk}} die Rede sein.
                  Ein paar Wochen später, am 28. 2. 1900 meldeten mehrere
                  Tageszeitungen, das \emph{\textcolor{brown}{Deutsche Volkstheater}\orgindex{Volkstheater@Volkstheater|pwk}}
                  hätte das Stück angenommen. Die \textcolor{violet}{Uraufführung}\eventindex{Volkstheater@\textbf{Volkstheater}!Uraufführung von Die Ehrlosen, 16.3.1901@Uraufführung von Die Ehrlosen, 16.3.1901|pwkv} fand am 16. 3. 1901
                  statt.}}}\label{K_L03724-2} – und das weitere wird sich finden. Doch das ist Zukunftsmusik
               – .\pend
           
\pstart
           Für heute will ich Ihnen nur nochmals herzlich danken und schließlich
               noch \strikeout{B} bemerken, daß Sie ganz recht hatten bezüglich
               der »Widmung«! Ich hatte sie \introOben{}mit Bleistift\introOben{} auf das
               Titelblatt meines \textcolor{green}{Conceptes}\pwindex{Plessner, Elsa 22.\,8.\,1875 Wien – 7.\,5.\,1932 Alicante@\textsc{Plessner, Elsa} (22.\,8.\,1875 Wien – 7.\,5.\,1932 Alicante), \emph{Schriftstellerin}!erste Kapitel. Schauspiel in drei Akten@\strich\emph{Das erste Kapitel. Schauspiel in drei Akten}|pwv}{}\ledrightnote{{$\rightarrow$}\emph{\textcolor{green}{Das erste Kapitel. Schauspiel in drei Akten}}} ge\substVorne{}\textsuperscript{schrieben}\substDazwischen{}setzt\substHinten{} und mich, so oft ich mich zur {\pb}Arbeit setzte –
               daran »gestimmt«. Als es die Abschreiberin erhielt, vergaß ich ganz auf diese \introOben{}nur\introOben{} zu meinem persönlichen Gebrauch dienenden Zeilen. So sind
               sie auf die zwei Abschriften übergegangen – die natürlich nicht für die
               Öffentlichkeit bestimmt sind – geschweige erst für Herrn »\label{K_L03724-3v}\edtext{\textcolor{green}{Fery
                  Derffler}\pwindex{Plessner, Elsa 22.\,8.\,1875 Wien – 7.\,5.\,1932 Alicante@\textsc{Plessner, Elsa} (22.\,8.\,1875 Wien – 7.\,5.\,1932 Alicante), \emph{Schriftstellerin}!erste Kapitel. Schauspiel in drei Akten@\strich\emph{Das erste Kapitel. Schauspiel in drei Akten}|pwv}{}\ledrightnote{{$\rightarrow$}\emph{\textcolor{green}{Das erste Kapitel. Schauspiel in drei Akten}}}}{\lemma{\textnormal{\emph{Fery
                  Derffler}}}\Cendnote{\textnormal{In \emph{\textcolor{green}{Das erste Kapitel}\pwindex{Plessner, Elsa 22.\,8.\,1875 Wien – 7.\,5.\,1932 Alicante@\textsc{Plessner, Elsa} (22.\,8.\,1875 Wien – 7.\,5.\,1932 Alicante), \emph{Schriftstellerin}!erste Kapitel. Schauspiel in drei Akten@\strich\emph{Das erste Kapitel. Schauspiel in drei Akten}|pwk}} heißt einer der Beamten Ferry Derffler
                  und wird beschrieben: »(ſchlank, dunkelblond) ſehr weich und liebenswürdig,
                     impulſiv und innig – manchmal ein wenig Poſe.« Die Widmung scheint eine
                  Entschlüsselung mit einer realen Person angeboten zu haben. Im Druck
                     (1910) erschien das Stück ohne Widmung. }}}\label{K_L03724-3}«. – Auch ich liebe
               keine Intimitäten mit dem Publikum. Ich bitte Sie also, mich einer \uline{solchen} Geschmacklosigkeit doch nicht für fähig zu
               halten – so viele andere ich auch auf dem Gewissen haben möge.\pend
           
\pstart
           Mit alter Verehrung{\\[\baselineskip]}\spacefill\mbox{Elsa Plessner}.\pend
           \leftskip=0em{}\selectlanguage{ngerman}\endnumbering\briefempfaengerindex{Schnitzler, Arthur@\textsc{Schnitzler, Arthur}!zzzPlessner, Elsa@\emph{von Elsa Plessner}!1900-01-101@{10. 1. 1900}|)be}\mylabel{L03724h}  \normalsize

\doendnotes{C}
\bigskip
\vfill

\clearpage

\footnotesize

\lohead{\textsc{register}}

% Definiere theindex-Environment komplett neu ohne reledmac
\makeatletter
\renewenvironment{theindex}{%
  \section*{\indexname}%
  \setlength{\parindent}{0pt}%
  \setlength{\parskip}{0pt plus 0.3pt}%
  \let\item\@idxitem
}{%
  \clearpage
}
\makeatother

\IfFileExists{\jobname-pw.ind}{\input{\jobname-pw.ind}}{}

\end{document}

      