%% latex-korrekturansicht-vorspann.tex
%% Vorspann für die Korrekturansicht.
%% Lädt die gemeinsame Datei latex-vorspann.tex mit gesetztem Schalter.

\newif\ifkorrekturansicht
\korrekturansichttrue

\input{../tex-inputs/latex-vorspann}


               \section[Hugo von Hofmannsthal an Arthur Schnitzler, 8. 10. 1904]{ Hugo von Hofmannsthal an Arthur Schnitzler, 8. 10. 1904}\nopagebreak\mylabel{v}\rehead{ }\normalsize\beginnumbering\briefempfaengerindex{Schnitzler, Arthur@\textsc{Schnitzler, Arthur}!zzzHofmannsthal, Hugo von@\emph{von Hugo von Hofmannsthal}!1904-10-081@{8. 10. 1904}|(be} \toendnotes[C]{\smallbreak\pagebreak[2]} \Standort{CUL, Schnitzler, B 43.}
\physDesc{Postkarte
\newline{}Handschrift Gertrude von Hofmannsthal: schwarze Tinte, lateinische Kurrent\newline{}Versand: 1) Stempel: »\nobreak{}\oindex{Moedling@\textbf{Mödling}, \emph{Besiedelter Ort (A.BSO)}|pwk}Mödling 1, 8. 10. 04, 4–5N\nobreak{}«.  2) Stempel: »\nobreak{}\oindex{XVIII., Waehring@\textbf{XVIII., Währing}, \emph{Bezirk (A.BZK)}|pwk}18/1 W{[}ien 1{]}\textcolor{gray}{10}, 9. 10. 04, 9.V, Bestellt\nobreak{}«. \newline{}Ordnung: mit Bleistift von unbekannter Hand nummeriert:
                                    »237« }\buchAbdrucke{\weitereDrucke{Hugo von Hofmannsthal, Arthur Schnitzler: \emph{Briefwechsel}. Hg. Therese Nickl und Heinrich Schnitzler. Frankfurt am Main: \emph{S. Fischer} 1964, S. 203.} }\toendnotes[C]{\smallbreak}\pstart{}{\pb}Herrn Dr Arthur Schnitzler \pend{}\pstart{}\textcolor{pink}{Wien}{}\ledrightnote{\textcolor{pink}{Wien}}\pend{}\pstart{}\textcolor{pink}{XVIII Spöttlgasse 7}{}\ledrightnote{\textcolor{pink}{Edmund-Weiß-Gasse}}. \pend{}{\bigskip}\pstart
           \noindent{}{\pb}Schon gut. \label{K_L01453_1v}\edtext{Montag}{\lemma{\textnormal{\emph{Montag}}}\Cendnote{\textnormal{vgl. A. S.: \emph{Tagebuch}, 10. 10. 1904}}}\label{K_L01453_1h}{ }8 Uhr. Wünsche besseres Wetter. (mir)\pend
           \pstart \spacefill\mbox{Hugo.}\pend{}\pstart
           8. 10. 1904.\pend
           \endnumbering\briefempfaengerindex{Schnitzler, Arthur@\textsc{Schnitzler, Arthur}!zzzHofmannsthal, Hugo von@\emph{von Hugo von Hofmannsthal}!1904-10-081@{8. 10. 1904}|)be}\mylabel{h}  \normalsize

\doendnotes{C}
\bigskip
\vfill

\clearpage

\footnotesize

\lohead{\textsc{register}}

% Definiere theindex-Environment komplett neu ohne reledmac
\makeatletter
\renewenvironment{theindex}{%
  \section*{\indexname}%
  \setlength{\parindent}{0pt}%
  \setlength{\parskip}{0pt plus 0.3pt}%
  \let\item\@idxitem
}{%
  \clearpage
}
\makeatother

\IfFileExists{\jobname-pw.ind}{\input{\jobname-pw.ind}}{}

\end{document}

      