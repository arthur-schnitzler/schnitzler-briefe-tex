%% latex-korrekturansicht-vorspann.tex
%% Vorspann für die Korrekturansicht.
%% Lädt die gemeinsame Datei latex-vorspann.tex mit gesetztem Schalter.

\newif\ifkorrekturansicht
\korrekturansichttrue

\input{../tex-inputs/latex-vorspann}


               \section[Paul Goldmann an Arthur Schnitzler, Paul Goldmann an Arthur Schnitzler, 27. 1. {[}1897{]}]{ Paul Goldmann an Arthur Schnitzler, 27. 1. {[}1897{]}}\nopagebreak\mylabel{v}\rehead{ }\normalsize\beginnumbering\briefempfaengerindex{Schnitzler, Arthur@\textsc{Schnitzler, Arthur}!zzzGoldmann, Paul@\emph{von Paul Goldmann}!1897-01-271@{27. 1. {[}1897{]}}|(be} \toendnotes[C]{\smallbreak\pagebreak[2]} \Standort{DLA, A:Schnitzler, HS.NZ85.1.3167.}
\physDesc{Brief, 1 Blatt, 4 Seiten
\newline{}Handschrift: blaue Tinte, deutsche Kurrent
\newline{}Schnitzler: 1) mit Bleistift das Jahr »97« vermerkt 2) mit rotem Buntstift eine Unterstreichung}\toendnotes[C]{\smallbreak}\pstart
           \noindent{}{\pb}\textcolor{gray}{\textbf{\textbf{\textcolor{brown}{Frankfurter Zeitung}{}\ledrightnote{\textcolor{brown}{Frankfurter Zeitung}}}}}\pend
           \pstart
           \textcolor{gray}{\textbf{(\textcolor{brown}{\begin{otherlanguage}{french}Gazette de Francfort\end{otherlanguage}}{}\ledrightnote{\textcolor{brown}{Frankfurter Zeitung}}).}}\pend
           \pstart
           \textcolor{gray}{\textbf{\textbf{\begin{otherlanguage}{french}Fondateur M.\end{otherlanguage}{ }\textcolor{blue}{L. Sonnemann}{}\ledrightnote{\textcolor{blue}{Leopold Sonnemann}}.}}}\pend
           \pstart
           \begin{otherlanguage}{french}\textcolor{gray}{\textbf{Journal politique, financier,}}\end{otherlanguage}\pend
           \pstart
           \begin{otherlanguage}{french}\textcolor{gray}{\textbf{commercial et littéraire.}}\end{otherlanguage}\pend
           \pstart
           \begin{otherlanguage}{french}\textcolor{gray}{\textbf{\textbf{Paraissant trois fois par jour.}}}\end{otherlanguage}\hfill \textsc{\textcolor{pink}{Paris}{}\ledrightnote{\textcolor{pink}{Paris}}}, 27. Januar.\pend
           \pstart
           \begin{otherlanguage}{french}\textcolor{gray}{\textbf{\textbf{Bureau à \textcolor{pink}{Paris}{}\ledrightnote{\textcolor{pink}{Paris}}}}}\end{otherlanguage}\pend
           \pstart
           \begin{otherlanguage}{french}\textcolor{gray}{\textbf{\textbf{\textcolor{pink}{24. Rue Feydeau}{}\ledrightnote{\textcolor{pink}{rue Feydeau}}.}}}\end{otherlanguage}\pend
           \pstart\center{}Mein lieber Freund,\pend\pstart
           Nur wenige Worte heut!\pend
           \pstart
           Dein lieber Brief hat mich beunruhigt. Was für \label{K_L02801-1v}\edtext{Aufregungen}{\lemma{\textnormal{\emph{Aufregungen}}}\Cendnote{\textnormal{\textcolor{blue}{Marie Reinhard} war im Dezember 1896 von \textcolor{blue}{Schnitzler}
                  schwanger geworden, was sie im Jänner 1897
                  feststellten.}}}\label{K_L02801-1h} ſind das\substVorne{}\textsuperscript{?}\substDazwischen{},\substHinten{} welche Du durchzumachen haſt?\pend
           \pstart
           Ich will keine Einzelheiten wiſſen. Du wirſt mir ſchreiben, wenn Du ruhig biſt und
               Zeit haſt. Aber nur in einer Zeile ſollteſt Du mir ſagen: Hängt die Sache mit Frauen,
               mit der gewiſſen \textcolor{blue}{Dame}{}\ledrightnote{→\textcolor{blue}{Marie Reinhard}}
               zuſammen? Oder ſind es Vorgänge nicht weiblicher Art? Im erſteren Falle würde ich
               bedeutend ruhiger \strikeout{\textcolor{gray}{ſa}} ſein. Das mag Dir frivol erſcheinen – Dir, der Du mitten darin ſtehſt. Aber
               ich {\pb}huldige doch der hier zu Lande üblichen
                  Auffaſſung\strikeout{:}\introOben{},\introOben{} daß Erlebniſſe mit Frauen ſelten ſchwere und weſentliche
               Schädigungen im Leben zurücklaſſen{\dotsfour}\pend
           \pstart
           Innigen Dank für die Wärme, mit welcher Du Dich der \label{K_L02801-2v}\edtext{\textsc{\textcolor{green}{Lorenzaccio}{}\ledrightnote{\textcolor{green}{Lorenzaccio. Drame romantique en cinq actes}}}-Angelegenheit}{\lemma{\textnormal{\emph{Lorenzaccio-Angelegenheit}}}\Cendnote{\textnormal{siehe Paul Goldmann an Arthur Schnitzler, 2. [1.? 1897]}}}\label{K_L02801-2h} angenommen haſt! Ich weiß nicht, ob ich mich an die Arbeit machen werde. Es
               liegt eine complicirte Rechts-Situation vor. Nach \textcolor{pink}{fran}{}\ledrightnote{→\textcolor{pink}{Frankreich}}zöſiſchem Rechte iſt \textsc{\textcolor{blue}{Musset}{}\ledrightnote{\textcolor{blue}{Alfred de Musset}}} noch nicht frei (er wird es erſt in zehn Jahren), und die \label{K_L02801-5v}\edtext{\textcolor{blue}{Erben}{}\ledrightnote{→\textcolor{blue}{Paul Lardin de Musset}}}{\lemma{\textnormal{\emph{Erben}}}\Cendnote{\textnormal{Es ist unklar, mit wem \textcolor{blue}{Goldmann} in Kontakt stand. Die Rechte an den Werken \textcolor{blue}{Alfred de Musset}s verwaltete jedenfalls
                  dessen Neffe \textcolor{blue}{Paul Lardin de Musset}.}}}\label{K_L02801-5h}
               ſtellen unverſchämte Forderungen. Ich erwarte die Antwort eines \label{K_L02801-4v}\edtext{\textcolor{pink}{deutſch}{}\ledrightnote{→\textcolor{pink}{Deutschland}}en Advocaten}{\lemma{\textnormal{\emph{deutſchen Advocaten}}}\Cendnote{\textnormal{nicht ermittelt}}}\label{K_L02801-4h} über den Fall. {\pb}Bin auch wenig zur Arbeit geſtimmt. Bin krank und
               werde täglich von der gräßlichen Angſt geplagt, \label{K_L02801-3v}\edtext{blind zu werden}{\lemma{\textnormal{\emph{blind zu werden}}}\Cendnote{\textnormal{aufgrund seiner Augenprobleme}}}\label{K_L02801-3h}{\dots}\pend
           \pstart
           Geſtern ſandte ich Dir den »\textsc{\textcolor{green}{Temps}{}\ledrightnote{\textcolor{green}{Le Temps}}}« mit der ſchönen \label{K_L02801-8v}\edtext{\textcolor{green}{Beſprechung}{}\ledrightnote{→\textcolor{green}{Un vaudevilliste viennois}}}{\lemma{\textnormal{\emph{Beſprechung}}}\Cendnote{\textnormal{\textcolor{blue}{Théodore de Wyzewa}: \emph{\textcolor{green}{Un vaudevilliste viennois}}. In: \emph{\textcolor{green}{Le Temps}}, Jg. 37, Nr. 13023, 27. 1. 1897, S. 2.}}}\label{K_L02801-8h} über \textcolor{green}{Dich}{}\ledrightnote{→\textcolor{green}{Mourir. Roman}}. Der »\textsc{\textcolor{green}{Temps}{}\ledrightnote{\textcolor{green}{Le Temps}}}« iſt das angeſehenſte und geleſenſte \textcolor{pink}{fran}{}\ledrightnote{→\textcolor{pink}{Frankreich}}zöſiſche \textcolor{green}{Blatt}{}\ledrightnote{→\textcolor{green}{Le Temps}}, die »\textcolor{green}{Neue Freie Preſſe}{}\ledrightnote{\textcolor{green}{Neue Freie Presse}}« von \textsc{\textcolor{pink}{Paris}{}\ledrightnote{\textcolor{pink}{Paris}}}. Schreib’ dem \textsc{\textcolor{blue}{Wyzewa}{}\ledrightnote{\textcolor{blue}{Théodore de Wyzewa}}} (der ein \textcolor{blue}{Freund}{}\ledrightnote{→\textcolor{blue}{Théodore de Wyzewa}}{ }\textsc{\textcolor{blue}{Thorel}{}\ledrightnote{\textcolor{blue}{Jean Thorel}}s} iſt) ein Wort des Dankes. Das
               kann gut thun, denn der \textcolor{blue}{Mann}{}\ledrightnote{→\textcolor{blue}{Théodore de Wyzewa}}
               hat großen Einfluß. Von \textsc{\textcolor{blue}{Thorel}{}\ledrightnote{\textcolor{blue}{Jean Thorel}}} höre ich nichts. Ich gehe dieſer Tage zu ihm{\dotsfour}\pend
           \pstart
           Den Schluß des \label{K_L02801-9v}\edtext{\textcolor{green}{Feuilleton}{}\ledrightnote{→\textcolor{green}{?? [Feuilleton über Lorenzaccio von Musset]}}s über \textsc{\textcolor{green}{Lorenzaccio}{}\ledrightnote{\textcolor{green}{Lorenzaccio. Drame romantique en cinq actes}}}}{\lemma{\textnormal{\emph{Feuilletons über Lorenzaccio}}}\Cendnote{\textnormal{XXXX. Die Uraufführung von \emph{\textcolor{green}{Lorenzaccio}} fand am 3. 12. 1896 im \emph{\textcolor{brown}{Théâtre de la Renaissance}} in \textcolor{pink}{Paris} statt.}}}\label{K_L02801-9h} ſende ich Dir deshalb {\pb}nicht, weil er nur mit wenigen Worten die \textcolor{pink}{Pariſ}{}\ledrightnote{\textcolor{pink}{Paris}}er \textcolor{green}{Aufführung}{}\ledrightnote{→\textcolor{green}{Lorenzaccio. Drame romantique en cinq actes}} beſpricht.\pend
           \pstart
           Bald höre ich hoffentlich von Dir. \label{K_L02801-6v}\edtext{Arbeiteſt Du gar nichts?}{\lemma{\textnormal{\emph{Arbeiteſt Du gar nichts?}}}\Cendnote{\textnormal{\textcolor{blue}{Schnitzler} war aufgrund der Aufregungen rund
                  um \textcolor{blue}{Marie Reinhard}s Schwangerschaft
                  tatsächlich arbeitsunfähig, wie er mehrmals im \emph{\textcolor{green}{Tagebuch}} notierte (vgl. A. S.: \emph{Tagebuch}, 17. 1. 1897, A. S.: \emph{Tagebuch}, 21. 1. 1897).}}}\label{K_L02801-6h}\pend
           \pstart
           Sei von Herzen gegrüßt!\pend
           \pstart
           Dein treuer {\\[\baselineskip]}\spacefill\mbox{Paul Goldmann}\pend
           \leftskip=0em{}\endnumbering\briefempfaengerindex{Schnitzler, Arthur@\textsc{Schnitzler, Arthur}!zzzGoldmann, Paul@\emph{von Paul Goldmann}!1897-01-271@{27. 1. {[}1897{]}}|)be}\mylabel{h}\begin{anhang}\end{anhang}\normalsize

\doendnotes{C}
\bigskip
\vfill

\clearpage

\footnotesize

\lohead{\textsc{register}}

% Definiere theindex-Environment komplett neu ohne reledmac
\makeatletter
\renewenvironment{theindex}{%
  \section*{\indexname}%
  \setlength{\parindent}{0pt}%
  \setlength{\parskip}{0pt plus 0.3pt}%
  \let\item\@idxitem
}{%
  \clearpage
}
\makeatother

\IfFileExists{\jobname-pw.ind}{\input{\jobname-pw.ind}}{}

\end{document}

      