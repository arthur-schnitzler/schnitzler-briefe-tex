%% latex-korrekturansicht-vorspann.tex
%% Vorspann für die Korrekturansicht.
%% Lädt die gemeinsame Datei latex-vorspann.tex mit gesetztem Schalter.

\newif\ifkorrekturansicht
\korrekturansichttrue

\input{../tex-inputs/latex-vorspann}


\renewcommand{\erwaehntePersonen}{Personen: Ernest von Gréger-Jurco, Gerhart Hauptmann, Olga Schnitzler, Karl Schönherr, Karl Strecker}
\renewcommand{\erwaehnteInstitutionen}{Institutionen: Tägliche Rundschau}
\renewcommand{\erwaehnteOrte}{Orte: Berlin, Brühl, Dessauer Straße, Deutschland, Wien}
\renewcommand{\erwaehnteWerke}{Werke: Das angebliche Telegramm Arthur Schnitzlers, Der Sonnwendtag. Drama in fünf Akten, Die Weber, Ein litterarisch-dramatisches Hochstapler-Stücklein, Tägliche Rundschau}
\section[ Paul Goldmann an Arthur Schnitzler, 2. 5. {[}1902{]}]{Paul Goldmann an Arthur Schnitzler, 2. 5. {[}1902{]}}
\nopagebreak\mylabel{v}
\rehead{ }\normalsize\beginnumbering\briefempfaengerindex{Schnitzler, Arthur@\textsc{Schnitzler, Arthur}!zzzGoldmann, Paul@\emph{von Paul Goldmann}!1902-05-022@{2. 5. {[}1902{]}}|(be}
\toendnotes[C]{\smallbreak\pagebreak[2]}\Standort{DLA, A:Schnitzler, HS.NZ85.1.3172.}
\physDesc{Brief, 1 Blatt, 2 Seiten
\newline{}Handschrift: blaue Tinte, deutsche Kurrent
\newline{}Schnitzler: mit Bleistift das Jahr »{[}1{]}902« vermerkt }\toendnotes[C]{\smallbreak}
\pstart
           \noindent{}\raggedleft{}{\pb}\textcolor{pink}{\textcolor{gray}{\textbf{DESSAUERSTRASSE 19}}}{}\ledrightnote{\textcolor{pink}{Dessauer Straße}}\pend
           
\pstart
           \textcolor{pink}{Berlin}{}\ledrightnote{\textcolor{pink}{Berlin}}, 2. Mai.\pend
           
\pstart\center{}Mein lieber Freund,\pend
\pstart
           Daß Du den \label{K_L03206-1v}\edtext{\textcolor{blue}{Schwindler}{}\ledrightnote{{$\rightarrow$}\textcolor{blue}{Ernest von Gréger-Jurco}}}{\lemma{\textnormal{\emph{Schwindler}}}\Cendnote{\textnormal{siehe Paul Goldmann an Arthur Schnitzler, 29. 4. [1902]}}}\label{K_L03206-1h}, den \textsc{\textcolor{blue}{Jurco}{}\ledrightnote{\textcolor{blue}{Ernest von Gréger-Jurco}}} ſelbſt, laufen läßt, verſtehe ich. Der Kerl hat ſein Theil. Aber ganz and gar
               nicht einverſtanden bin ich damit, daß Du Herrn \textsc{\textcolor{blue}{Strecker}{}\ledrightnote{\textcolor{blue}{Karl Strecker}}}, dem deutſchen \textcolor{blue}{Mann}{}\ledrightnote{{$\rightarrow$}\textcolor{blue}{Karl Strecker}} und
               literariſchen \textcolor{blue}{Kritiker}{}\ledrightnote{{$\rightarrow$}\textcolor{blue}{Karl Strecker}}, ſo
               vollſtändig nachgibſt. Das Benehmen dieſes \textcolor{blue}{Menſchen}{}\ledrightnote{{$\rightarrow$}\textcolor{blue}{Karl Strecker}} iſt von einer ſo unerhörten Unanſtändigkeit, daß Du
               gerade darum energiſch auf Deinem Recht beſtehen müßteſt. Die Leſer der »\textcolor{green}{täglichen Rundſchau}{}\ledrightnote{\textcolor{green}{Tägliche Rundschau}}« (und das \textcolor{green}{Blatt}{}\ledrightnote{{$\rightarrow$}\textcolor{green}{Tägliche Rundschau}} iſt in \textcolor{pink}{Deutſchland}{}\ledrightnote{\textcolor{pink}{Deutschland}} mehr geleſen, als irgendeine \textcolor{pink}{Wien}{}\ledrightnote{\textcolor{pink}{Wien}}er Zeitung) müſſen glauben, daß Du, da Du auf die \label{K_L03206-2v}\edtext{»\textcolor{green}{offene Frage}{}\ledrightnote{{$\rightarrow$}\textcolor{green}{Ein litterarisch-dramatisches Hochstapler-Stücklein}}«}{\lemma{\textnormal{\emph{»offene Frage«}}}\Cendnote{\textnormal{siehe Paul Goldmann an Arthur Schnitzler, 26. 4. 1902}}}\label{K_L03206-2h} nicht geantwortet haſt, an dem Schwindel des Herrn \textsc{\textcolor{blue}{Jurco}{}\ledrightnote{\textcolor{blue}{Ernest von Gréger-Jurco}}} mitbetheiligt biſt. Ich würde es nicht begreifen, wenn {\pb}Du \strikeout{es} darauf
               verzichteteſt, in dieſer Angelegenheit entſchieden Dein Recht zu verlangen. Du mußt
               es um Deintwegen thun, und dann beſteht auch ein gewiſſes allgemeines Intereſſe, daß
               die Unanſtändigkeit eines ehrenfeſten \textcolor{pink}{deutſch}{}\ledrightnote{{$\rightarrow$}\textcolor{pink}{Deutschland}}en \textcolor{blue}{Mann}{}\ledrightnote{{$\rightarrow$}\textcolor{blue}{Karl Strecker}}es, des Kritikers eines all\textcolor{pink}{deutſch}{}\ledrightnote{{$\rightarrow$}\textcolor{pink}{Deutschland}}en und antiſemitiſchen \textcolor{brown}{Blatt}{}\ledrightnote{{$\rightarrow$}\textcolor{brown}{Tägliche Rundschau}}es, an die Öffentlichkeit gebracht wird. Du \strikeout{\textcolor{gray}{m}} mußt ihm ſofort ſchreiben und auf der Veröffentlichung Deiner \textcolor{green}{Antwort}{}\ledrightnote{{$\rightarrow$}\textcolor{green}{Das angebliche Telegramm Arthur Schnitzlers}} beſtehen. Das wird dem \textcolor{blue}{Herr}{}\ledrightnote{{$\rightarrow$}\textcolor{blue}{Karl Strecker}}n lehren, im nächſten »\textcolor{green}{Fall \textsc{Schnitzler}}{}\ledrightnote{{$\rightarrow$}\textcolor{green}{Ein litterarisch-dramatisches Hochstapler-Stücklein}}« vorſichtiger zu ſein.\pend
           
\pstart
           Ich habe eben den »\textcolor{green}{Sonnwendtag}{}\ledrightnote{\textcolor{green}{Der Sonnwendtag. Drama in fünf Akten}}« geleſen. Das \textcolor{green}{Stück}{}\ledrightnote{{$\rightarrow$}\textcolor{green}{Der Sonnwendtag. Drama in fünf Akten}} hat mich ſehr ergriffen.
               Wieviel höher ſteht dieſes \textcolor{green}{Werk}{}\ledrightnote{{$\rightarrow$}\textcolor{green}{Der Sonnwendtag. Drama in fünf Akten}} eines \textcolor{blue}{Dichter}{}\ledrightnote{{$\rightarrow$}\textcolor{blue}{Karl Schönherr}}s
               als ſämmtliche \textsc{\textcolor{blue}{Hauptmann}{}\ledrightnote{\textcolor{blue}{Gerhart Hauptmann}}sche} Dramen (mit Ausnahme der
                  »\label{K_L03206-4v}\edtext{\textcolor{green}{Weber}{}\ledrightnote{\textcolor{green}{Die Weber}}}{\lemma{\textnormal{\emph{Weber}}}\Cendnote{\textnormal{siehe Paul Goldmann an Arthur Schnitzler, 31. 12. [1900]}}}\label{K_L03206-4h}«)!\pend
           
\pstart
           Grüße \textsc{\textcolor{blue}{Olga}{}\ledrightnote{\textcolor{blue}{Olga Schnitzler}}} und ſei vielmals und von Herzen gegrüßt von Deinem {\\[\baselineskip]}\spacefill\mbox{Paul Goldmnn}\pend
           \leftskip=0em{}
\pstart
           \noindent{}Biſt Du Pfingſten in \textcolor{pink}{Wien}{}\ledrightnote{\textcolor{pink}{Wien}}? Vielleicht \label{K_L03206-3v}\edtext{komme
                     ich}{\lemma{\textnormal{\emph{komme
                     ich}}}\Cendnote{\textnormal{\textcolor{blue}{Goldmann} war von 18. 5. 1902 bis
                     jedenfalls 25. 5. 1902 in \textcolor{pink}{Wien} bzw. der
                        \textcolor{pink}{Brühl}.}}}\label{K_L03206-3h} hin.\pend
           \endnumbering\briefempfaengerindex{Schnitzler, Arthur@\textsc{Schnitzler, Arthur}!zzzGoldmann, Paul@\emph{von Paul Goldmann}!1902-05-022@{2. 5. {[}1902{]}}|)be}\mylabel{h}
\begin{anhang}
\end{anhang}\normalsize

\doendnotes{C}
\bigskip
\vfill

\clearpage

\footnotesize

\lohead{\textsc{register}}

% Definiere theindex-Environment komplett neu ohne reledmac
\makeatletter
\renewenvironment{theindex}{%
  \section*{\indexname}%
  \setlength{\parindent}{0pt}%
  \setlength{\parskip}{0pt plus 0.3pt}%
  \let\item\@idxitem
}{%
  \clearpage
}
\makeatother

\IfFileExists{\jobname-pw.ind}{\input{\jobname-pw.ind}}{}

\end{document}

      