%% latex-korrekturansicht-vorspann.tex
%% Vorspann für die Korrekturansicht.
%% Lädt die gemeinsame Datei latex-vorspann.tex mit gesetztem Schalter.

\newif\ifkorrekturansicht
\korrekturansichttrue

\input{../tex-inputs/latex-vorspann}


\renewcommand{\erwaehntePersonen}{Personen: Eva Marie Goldmann, Franziska Goldmann}
\renewcommand{\erwaehnteInstitutionen}{Institutionen: Neue Freie Presse, S. Fischer Verlag}
\renewcommand{\erwaehnteOrte}{Orte: Bendlerstraße, Berlin, Wien}
\renewcommand{\erwaehnteWerke}{Werke: Therese. Chronik eines Frauenlebens}
\section[ Paul Goldmann an Arthur Schnitzler, 3. 5. 1928]{Paul Goldmann an Arthur Schnitzler, 3. 5. 1928}
\nopagebreak\mylabel{v}
\rehead{ }\normalsize\beginnumbering\briefempfaengerindex{Schnitzler, Arthur@\textsc{Schnitzler, Arthur}!zzzGoldmann, Paul@\emph{von Paul Goldmann}!1928-05-031@{3. 5. 1928}|(be}
\toendnotes[C]{\smallbreak\pagebreak[2]}\Standort{DLA, A:Schnitzler, HS.NZ85.1.3176.}
\physDesc{Brief, 1 Blatt, 1 Seite, 489 Zeichen
\newline{}Schreibmaschine
\newline{}Handschrift: lila Tinte, lateinische Kurrent (\noindent{}eine Korrektur und Unterschrift)
\newline{}Schnitzler: mit rotem Buntstift »\textcolor{green}{Theres{[}e{]}}« vermerkt und eine Unterstreichung }\toendnotes[C]{\smallbreak}
\pstart
           \noindent{}{\pb}\textcolor{gray}{\textbf{Dr. Paul Goldmann}}\hfill \textcolor{gray}{\textbf{\textcolor{pink}{Berlin W. 10}{}\ledrightnote{\textcolor{pink}{Berlin}}}}\pend
           
\pstart
           \textcolor{gray}{\textbf{Vertreter der »\textcolor{brown}{Neuen Freien
                           Presse}{}\ledrightnote{\textcolor{brown}{Neue Freie Presse}}«}}\hfill \textcolor{gray}{\textbf{\textcolor{pink}{Bendlerſtraße 36}{}\ledrightnote{\textcolor{pink}{Bendlerstraße}}.}}\pend
           
\pstart
           \raggedleft{}\textcolor{gray}{\textbf{Tel. Lützow 9142}}\pend
           
\pstart
           \raggedleft{}3. 5. 28.\pend
           
\pstart\center{}Lieber Freund,\pend
\pstart
           Für die Übersendung Deines neuen \label{K_L03516-1v}\edtext{\textcolor{green}{Roman}{}\ledrightnote{{$\rightarrow$}\textcolor{green}{Therese. Chronik eines Frauenlebens}}}{\lemma{\textnormal{\emph{Roman}}}\Cendnote{\textnormal{\textcolor{blue}{Schnitzler} Roman \emph{\textcolor{green}{Therese. Chronik eines Frauenlebens}} war am 27. 3. 1928 im \textcolor{pink}{Berlin}er \emph{\textcolor{brown}{S. Fischer-Verlag}}
                  erschienen.}}}\label{K_L03516-1h}s sagen wir alle Dir unseren herzlichsten Dank. Er geht
               gegenwärtig in meinem Haushalt von Hand zu Hand und findet den Beifall von Jung und
               Alt. Wenn \textcolor{blue}{Frau}{}\ledrightnote{{$\rightarrow$}\textcolor{blue}{Eva Marie Goldmann}} und \textcolor{blue}{Tochter}{}\ledrightnote{{$\rightarrow$}\textcolor{blue}{Franziska Goldmann}} fertig sind, darf ich
               dann das \textcolor{green}{Buch}{}\ledrightnote{{$\rightarrow$}\textcolor{green}{Therese. Chronik eines Frauenlebens}} auch lesen.
               Darum kann ich einstweilen nur für die Übersendung danken.\pend
           
\pstart
           Ich ho\substVorne{}\textsuperscript{gf}\substDazwischen{}ff\substHinten{}e, dass es Dir gut geht, und dass wir bald wieder einmal die Freude haben
               werden, Dich in \label{K_L03516-2v}\edtext{\textcolor{pink}{Berlin}{}\ledrightnote{\textcolor{pink}{Berlin}}}{\lemma{\textnormal{\emph{Berlin}}}\Cendnote{\textnormal{In \textcolor{pink}{Berlin} sahen sich \textcolor{blue}{Goldmann} und \textcolor{blue}{Schnitzler} erst am 11. 11. 1930 und 16. 11. 1930 wieder. Am
                     16. 5. 1930 hatte
                     \textcolor{blue}{Goldmann}{ }\textcolor{blue}{Schnitzler} noch vorgeworfen, ihn nicht in
                     \textcolor{pink}{Berlin} zu besuchen.}}}\label{K_L03516-2h} zu sehen.\pend
           
\pstart
           Alles Herzliche von uns Allen! {\\[\baselineskip]}{[}hs. Goldmann:{]} Dein {\\[\baselineskip]}\spacefill\mbox{Paul Goldmann.}\pend
           \leftskip=0em{}\endnumbering\briefempfaengerindex{Schnitzler, Arthur@\textsc{Schnitzler, Arthur}!zzzGoldmann, Paul@\emph{von Paul Goldmann}!1928-05-031@{3. 5. 1928}|)be}\mylabel{h}
\begin{anhang}
\end{anhang}\normalsize

\doendnotes{C}
\bigskip
\vfill

\clearpage

\footnotesize

\lohead{\textsc{register}}

% Definiere theindex-Environment komplett neu ohne reledmac
\makeatletter
\renewenvironment{theindex}{%
  \section*{\indexname}%
  \setlength{\parindent}{0pt}%
  \setlength{\parskip}{0pt plus 0.3pt}%
  \let\item\@idxitem
}{%
  \clearpage
}
\makeatother

\IfFileExists{\jobname-pw.ind}{\input{\jobname-pw.ind}}{}

\end{document}

      