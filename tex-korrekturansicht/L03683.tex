%% latex-korrekturansicht-vorspann.tex
%% Vorspann für die Korrekturansicht.
%% Lädt die gemeinsame Datei latex-vorspann.tex mit gesetztem Schalter.

\newif\ifkorrekturansicht
\korrekturansichttrue

\input{../tex-inputs/latex-vorspann}


\renewcommand{\erwaehntePersonen}{Personen: Anatole France, Detlev von Liliencron, Maurice Maeterlinck, Romain Rolland, Olga Schnitzler, Franz Peter Schubert, Robert Schumann, William Shakespeare, Leo N. von Tolstoi, Jakob von Winternitz, Stefan Zweig}
\renewcommand{\erwaehnteOrte}{Orte: Frankreich, Kochgasse 29, Kochgasse 8, Romandy, Russland, Schweiz, Volkshochschule Ottakring, Wien, Währinger Cottage}
\renewcommand{\erwaehnteWerke}{Werke: Die Schaubühne, Journal de Genève, Liliencron, Tagebuch, Une protestation d’Arthur Schnitzler, Winterreise [op. 89 D 911], Winterreise. Der Wegweiser}
\section[Stefan Zweig an Arthur Schnitzler, 3. 12. 1914]{Stefan Zweig an Arthur Schnitzler, 3. 12. 1914}
\nopagebreak\mylabel{v}
\rehead{ }\normalsize\beginnumbering\briefempfaengerindex{Schnitzler, Arthur@\textsc{Schnitzler, Arthur}!zzzZweig, Stefan@\emph{von Stefan Zweig}!1914-12-031@{3. 12. 1914}|(be}
\toendnotes[C]{\smallbreak\pagebreak[2]}\Standort{CUL, Schnitzler, B 118.}
\physDesc{Brief, 1 Blatt, 2 Seiten, 2236 Zeichen
\newline{}Schreibmaschine
\newline{}Handschrift: blaue Tinte, lateinische Kurrent (\noindent{}Korrekturen, Unterschrift und Postskriptum)
\newline{}Schnitzler: 1) mit rotem Buntstift drei Unterstreichungen  2) mit Bleistift beschriftet: »\textsc{Zweig}«}\toendnotes[C]{\smallbreak}
\pstart
           {\pb}\textcolor{gray}{\textbf{SZ}}\hfill \textcolor{gray}{\textbf{\textcolor{pink}{VIII. KOCHGASSE 8}{}\ledrightnote{\textcolor{pink}{Kochgasse 8}}}}\pend
           
\pstart
           \raggedleft{}\textcolor{gray}{\textbf{\textcolor{pink}{WIEN}{}\ledrightnote{\textcolor{pink}{Wien}},}}\pend
           
\pstart
           \raggedleft{}\textcolor{pink}{Wien}{}\ledrightnote{\textcolor{pink}{Wien}}, 3. Dezember 14\pend
           
\pstart{}Sehr verehrter lieber Herr Doktor\pend\vspace{0.5em}
\pstart
           Ich danke Ihnen viele Male für Ihren lieben Brief und das schöne \label{K_L03683-1v}\edtext{\textcolor{green}{Dokument}{}\ledrightnote{{$\rightarrow$}\textcolor{green}{Une protestation d’Arthur Schnitzler}} Ihrer gerechten
                  Gesinnung}{\lemma{\textnormal{\emph{Dokument … Gesinnung}}}\Cendnote{\textnormal{Nachdem \textcolor{blue}{Schnitzler} zugetragen worden war, dass unter seinem Namen
                  in einer \textcolor{pink}{russischen} Zeitung \textcolor{blue}{Leo Tolstoi}, \textcolor{blue}{Maurice Maeterlinck}, \textcolor{blue}{Anatole France} und \textcolor{blue}{William
                     Shakespeare} verunglimpft worden waren (vgl. A. S.: \emph{Tagebuch}, 23. 11. 1914), verfasste er ein Dementi, das mit einem
                  Vorwort \textcolor{blue}{Romain Rollands} und in dessen
                  Übersetzung ins \textcolor{pink}{Französische} zunächst in der
                     \textcolor{pink}{französischen Schweiz} publiziert wurde
                     (\textcolor{blue}{Romain Rolland}, \textcolor{blue}{Schnitzler}: \emph{\textcolor{green}{Une
                     protestation d’Arthur Schnitzler}}. In: \emph{\textcolor{green}{Journal de Genève}}, Jg. 85, 3. Ausgabe, 21. 12. 1914, S. [1])
                  und später ohne die Übersetzung in weiteren Zeitungen Abdruck fand.}}}\label{}. Ich
               glaube, dass auch ein so geleg{[}e{]}ntliches Wort nur durch den Geist
               und die Güte, die es bezeugt, in diesen Tagen zum Manifest wird und zweifle nicht,
               dass es überall (ausser bei jenen Menschen, mit denen eine innere Verständigung über
               alles für uns unmöglich ist) die vorteilhafteste Wirkung im Gefolge haben \substVorne{}\textsuperscript{wird}\substDazwischen{}muss\substHinten{}. Ich habe es \textcolor{blue}{Romain Rolland}{}\ledrightnote{\textcolor{blue}{Romain Rolland}} gesandt
               und ihn gebeten, die Uebersetzung ins \textcolor{pink}{Französische}{}\ledrightnote{\textcolor{pink}{Frankreich}} womöglich selbst vorzunehmen, damit auch nicht ein Wort in
               seiner Bedeutung oder bloss \introOben{}in\introOben{} seinem Tonfall durch
               schlechte Nachbildung verändert werde. Ich bin sicher, dass er sich eine Freude
               daraus machen wird\introOben{},\introOben{} Ihnen und vor allem der uns gemeinsamen
               Sache der gegenseitigen Aufklärung dienlich zu sein. In wenigen Tagen werde ich mehr
               darüber wissen.\pend
           
\pstart
           Eine Veröffentlichung in \textcolor{pink}{Wien}{}\ledrightnote{\textcolor{pink}{Wien}} wäre vielleicht
               vorteilhafter, sobald der \textcolor{green}{Abdruck}{}\ledrightnote{{$\rightarrow$}\textcolor{green}{Une protestation d’Arthur Schnitzler}} in der \textcolor{pink}{Schweiz}{}\ledrightnote{\textcolor{pink}{Schweiz}} erfolgt ist und
               der Regierungsrat \textcolor{blue}{v. Winternitz}{}\ledrightnote{\textcolor{blue}{Jakob von Winternitz}} würde
               sicherlich gerne die offizielle Verlautbarung übernehmen. Seine Privatadresse ist \textcolor{pink}{VIII. Kochgasse 29}{}\ledrightnote{\textcolor{pink}{Kochgasse 29}}. Ich hoffe aber, ihn schon
               in diesen Tagen sprechen und mich seiner zweifellosen Zustimmung versichern zu
               können. \pend
           
\pstart
           Ich wäre sehr glücklich, wenn ich Sie, verehrter Herr \label{K_L03683-2v}\edtext{Doktor}{\lemma{\textnormal{\emph{Doktor}}}\Cendnote{\textnormal{Im
                  Manuskript steht Doktir.}}}\label{} bald sehen oder wenigstens Ihre Stimme durch das
               Telephon hören dürfte. Ich bin jetzt {\pb}immer zwischen 4 und 5 Uhr zuhause, vorher hält mich der kriegerische Dienst,
               nachher verlockt mich jetzt oft und öfter die Musik. Aber ich will gern jede Stunde
               des Nachmittags von 4 Uhr, die Sie mir erlauben wollen, dazu wahrnehmen, um in das
                  \textcolor{pink}{Cottage}{}\ledrightnote{\textcolor{pink}{Währinger Cottage}} hinauszukommen oder wohin immer es
               Ihnen gutdünkt und Sie dann nicht\strikeout{s} nur Nachts im
               Traum, ohne Ihre Erlaubnis, sondern am lichten Tag, mit Ihrer freundlichen
               Verstattung heim \introOben{}zu\introOben{}suchen{[}.{]}\pend
           
\pstart
           Ich beschäftige mich auch damit, für Ihre Frau \textcolor{blue}{Gemahlin}{}\ledrightnote{{$\rightarrow$}\textcolor{blue}{Olga Schnitzler}} ein paar schöne \label{K_L03683-3v}\edtext{Lieder}{\lemma{\textnormal{\emph{Lieder}}}\Cendnote{\textnormal{Aus \textcolor{blue}{Schnitzlers}{ }\emph{\textcolor{green}{Tagebuch}}eintrag geht hervor, dass \textcolor{blue}{Olga Schnitzler} Lieder von \textcolor{blue}{Schumann} und \textcolor{blue}{Schubert} sang, darunter das Lied \emph{\textcolor{green}{Wegweiser}} aus der \emph{\textcolor{green}{Winterreise}}, vgl. A. S.: \emph{Tagebuch}, 3. 1. 1915.}}}\label{} für jenen
                  \label{K_L03683-4v}\edtext{\textcolor{blue}{Liliencron}{}\ledrightnote{\textcolor{blue}{Detlev von Liliencron}}-Abend}{\lemma{\textnormal{\emph{Liliencron-Abend}}}\Cendnote{\textnormal{Die Veranstaltung fand am 3. 1. 1915 im \textcolor{pink}{Volksheim} statt zum Andenken an den Dichter \textcolor{blue}{Detlev von Liliencron}, der im Vorjahr
                  siebzig Jahre alt geworden wäre. Der Vortrag wurde publiziert als \textcolor{blue}{Stefan Zweig}: \emph{\textcolor{green}{Liliencron}}. In: \emph{\textcolor{green}{Die
                        Schaubühne}}, Jg. 11/I, Nr. 8, 25. 2. 1915,
                     S. 176–181.}}}\label{} zusammenzustellen, dessen Gelingen mich schon um des
               denkbaren Arbeiterpublikums willen so sehr freuen würde. Bishin vielen Dank und die
               herzlichsten Grüsse von\pend
           \pstart Ihrem immer getreuen \spacefill\mbox{Stefan Zweig}\pend{}
\pstart
           \noindent{}Verzeihen Sie die Schreibmaschine! Ich schreibe den ganzen Vormittag im Amt und
                  gebe dann meinern Fingern Rast!\pend
           \endnumbering\briefempfaengerindex{Schnitzler, Arthur@\textsc{Schnitzler, Arthur}!zzzZweig, Stefan@\emph{von Stefan Zweig}!1914-12-031@{3. 12. 1914}|)be}\mylabel{h}
\begin{anhang}
\end{anhang}\normalsize

\doendnotes{C}
\bigskip
\vfill

\clearpage

\footnotesize

\lohead{\textsc{register}}

% Definiere theindex-Environment komplett neu ohne reledmac
\makeatletter
\renewenvironment{theindex}{%
  \section*{\indexname}%
  \setlength{\parindent}{0pt}%
  \setlength{\parskip}{0pt plus 0.3pt}%
  \let\item\@idxitem
}{%
  \clearpage
}
\makeatother

\IfFileExists{\jobname-pw.ind}{\input{\jobname-pw.ind}}{}

\end{document}

      