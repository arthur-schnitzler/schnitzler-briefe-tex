%% latex-korrekturansicht-vorspann.tex
%% Vorspann für die Korrekturansicht.
%% Lädt die gemeinsame Datei latex-vorspann.tex mit gesetztem Schalter.

\newif\ifkorrekturansicht
\korrekturansichttrue

\input{../tex-inputs/latex-vorspann}


\renewcommand{\erwaehntePersonen}{Personen: Josef Rosengart, Olga Schnitzler, Heinrich Schnitzler}
\renewcommand{\erwaehnteOrte}{Orte: Berlin, Dessauer Straße, Deutsches Theater Berlin, Frankfurt am Main, Reuterweg, Wien}
\renewcommand{\erwaehnteWerke}{Werke: Die Frau mit dem Dolche, Die letzten Masken, Lebendige Stunden, Lebendige Stunden. Vier Einakter}
\section[ Paul Goldmann an Arthur Schnitzler und Olga Gussmann, 23. 12. {[}1901{]}]{Paul Goldmann an Arthur Schnitzler und Olga
               Gussmann, 23. 12. {[}1901{]}}
\nopagebreak\mylabel{v}
\rehead{ }\normalsize\beginnumbering\briefempfaengerindex{Schnitzler, Olga@\textsc{Schnitzler, Olga}!zzzGoldmann, Paul@\emph{von Paul Goldmann}!1901-12-231@{23. 12. {[}1901{]}}|(be}\briefempfaengerindex{Schnitzler, Arthur@\textsc{Schnitzler, Arthur}!zzzGoldmann, Paul@\emph{von Paul Goldmann}!1901-12-231@{23. 12. {[}1901{]}}|(be}
\toendnotes[C]{\smallbreak\pagebreak[2]}\Standort{DLA, A:Schnitzler, HS.NZ85.1.3171.}
\physDesc{Brief, 1 Blatt, 4 Seiten
\newline{}Handschrift: blaue Tinte, deutsche Kurrent
\newline{}Schnitzler: mit rotem Buntstift eine Unterstreichung }\toendnotes[C]{\smallbreak}
\pstart
           \noindent{}\raggedleft{}{\pb}\textcolor{pink}{\textcolor{gray}{\textbf{DESSAUERSTRASSE 19}}}{}\ledrightnote{\textcolor{pink}{Dessauer Straße}}\pend
           
\pstart
           \textcolor{pink}{Berlin}{}\ledrightnote{\textcolor{pink}{Berlin}}, 23. Dezember.\pend
           
\pstart\center{}Mein lieber Freund,\pend
\pstart
           Ich fahre heut{ }Mittag nach \textcolor{pink}{Frankfurt}{}\ledrightnote{\textcolor{pink}{Frankfurt am Main}}. \label{K_L03097-1v}\edtext{Wenn Du gekommen wäreſt}{\lemma{\textnormal{\emph{Wenn Du gekommen wäreſt}}}\Cendnote{\textnormal{siehe Paul Goldmann an Arthur Schnitzler, 19. 12. [1901]}}}\label{K_L03097-1h}, ſo wäre ich erſt morgen gefahren. Ich bedaure
               unendlich, daß ich Dich jetzt nicht ſehen kann.\pend
           
\pstart
           Was Du mir über \label{K_L03097-2v}\edtext{\textsc{Olga}}{\lemma{\textnormal{\emph{Olga}}}\Cendnote{\textnormal{\textcolor{blue}{Olga Gussmann} war erneut schwanger. Am 9. 8. 1902 brachte
                  sie den gemeinsamen Sohn \textcolor{blue}{Heinrich} auf die Welt.}}}\label{K_L03097-2h} ſchreibſt,
               iſt ſehr erfreulich auch für mich, weil es ja, wie ich weiß, Euren Wünſchen
               entſpricht. Ich wünſche von Herzen, daß die kritiſche Zeit vorübergehen möge, ohne
                  \strikeout{daß} allzuviel \label{K_L03097-3v}\edtext{Leiden}{\lemma{\textnormal{\emph{Leiden}}}\Cendnote{\textnormal{siehe A. S.: \emph{Tagebuch}, 23. 12. 1901}}}\label{K_L03097-3h} und Aufregung. Ich \substVorne{}\textsuperscript{h\textcolor{gray}{offe}}\substDazwischen{}denke\substHinten{}, daß ſich in Euer Beider Leben Manches freundlicher {\pb}und ruhiger geſtalten wird, wenn
               dieſe Hoffnung ſich erfüllt haben wird. Gern würde ich \textsc{Olga} noch ein paar Zeilen ſchreiben. Aber ich habe keine Minute und kann gerade noch
               raſch dieſen Brief fertigſtellen, den \textsc{Olga} auch als einen an ſie gerichteten betrachten ſoll. Liebes Fräulein \textsc{Olga}, Ich wünſche Ihnen von ganzem Herzen Glück. Und es wird Alles ſchon gut
               werden.\pend
           
\pstart
           Wenn ich von \label{K_L03097-5v}\edtext{Urtheilsloſigkeit der \textcolor{pink}{Wien}{}\ledrightnote{\textcolor{pink}{Wien}}er Freunde}{\lemma{\textnormal{\emph{Urtheilsloſigkeit … Freunde}}}\Cendnote{\textnormal{siehe Paul Goldmann an Arthur Schnitzler, 19. 12. [1901]}}}\label{K_L03097-5h} geſprochen habe, ſo iſt wieder einmal mein Temperament mit mir durchgegangen.
               Entſchuldige den ſchroffen {\pb}Ausdruck!
               Daß \label{K_L03097-11v}\edtext{\uline{Du} von »\textcolor{green}{Lebendigen
                  Stunden}{}\ledrightnote{\textcolor{green}{Lebendige Stunden}}« mehr hältſt}{\lemma{\textnormal{\emph{Du … hältſt}}}\Cendnote{\textnormal{siehe Arthur Schnitzler an Hermann Bahr, 28. 10. 1901}}}\label{K_L03097-11h}, als von der »\textcolor{green}{Frau mit dem Dolch}{}\ledrightnote{\textcolor{green}{Die Frau mit dem Dolche}}«, kann
               ich begreifen, da das erſte \textcolor{green}{Stück}{}\ledrightnote{{$\rightarrow$}\textcolor{green}{Lebendige Stunden}} Deinem Herzen eben näher ſteht. Ich kann aber nicht verſtehen, wie ein
               objektiv denkender \strikeout{Dritter} Anderer ſich über die
               vorausſichtliche Bühnenwirkung der beiden \textcolor{green}{Stücke}{}\ledrightnote{\textcolor{green}{Die Frau mit dem Dolche}{\newline}\textcolor{green}{Lebendige Stunden}} täuſchen kann. Es iſt klar, daß die »\textcolor{green}{Frau mit dem Dolche}{}\ledrightnote{\textcolor{green}{Die Frau mit dem Dolche}}« der Erfolg des Abends ſein wird und daß die »\textcolor{green}{Lebendigen Stunden}{}\ledrightnote{\textcolor{green}{Lebendige Stunden}}\label{T_L03097-2v}\edtext{«,}{\lemma{\textnormal{\emph{«,}}}\Cendnote{\textnormal{in der Vorlage steht das Komma vor dem schließenden
                  Anführungszeichen}}}\label{T_L03097-2h} wenn nicht die Darſtellung ein Wunder thut, faſt
               wirkungslos bleiben werden. Die »\textcolor{green}{Letzten Masken}{}\ledrightnote{\textcolor{green}{Die letzten Masken}}«
               habe ich auch geleſen – Ich {\pb}konnte es
               nicht ſertigbringen, das \textcolor{green}{Buch}{}\ledrightnote{{$\rightarrow$}\textcolor{green}{Lebendige Stunden. Vier Einakter}}
               auf dem Tiſch liegen zu laſſen und bis zur \label{K_L03097-12v}\edtext{\textsc{\textcolor{green}{Première}{}\ledrightnote{{$\rightarrow$}\textcolor{green}{Lebendige Stunden. Vier Einakter}}}}{\lemma{\textnormal{\emph{Première}}}\Cendnote{\textnormal{am 4. 1. 1902 am \textcolor{pink}{Deutschen Theater Berlin}}}}\label{K_L03097-12h} zu warten. Ich fand darin Geiſtreiches und Feines, hatte aber nicht den
               ſtarken Eindruck, den ich erwartet hatte. Das eigentliche Drama wäre meiner Anſicht
               nach doch geweſen, wenn der \textcolor{green}{Journaliſt}{}\ledrightnote{{$\rightarrow$}\textcolor{green}{Die letzten Masken}} dem \textcolor{green}{Schriftſteller}{}\ledrightnote{{$\rightarrow$}\textcolor{green}{Die letzten Masken}} geſagt hätte, was er ihm zu ſagen hatte. Dann wäre es
               natürlich ein anderes Stück geworden; aber ich weiß nicht, ob \strikeout{\textcolor{gray}{es} nicht d\textcolor{gray}{ram}} nicht ein \uline{Dramatiker} gerade dieſes \strikeout{Stück hätte} andere Stück hätte ſchreiben müſſen. Im
               Übrigen, die \textcolor{green}{Aufführung}{}\ledrightnote{{$\rightarrow$}\textcolor{green}{Lebendige Stunden. Vier Einakter}} wird
                  lehren{\dotsfour}\pend
           
\pstart
           Tauſend Grüße, mein lieber Freund! Und frohe Feiertage! {\\[\baselineskip]}Dein {\\[\baselineskip]}\spacefill\mbox{Paul Goldmann}\pend
           \leftskip=0em{}
\pstart
           \noindent{}{\pb}\label{T_L03097-1v}\edtext{Bitte, ſchreib’ mir nach \textcolor{pink}{Frankfurt}{}\ledrightnote{\textcolor{pink}{Frankfurt am Main}}: \textcolor{pink}{\textsc{Reuterweg} 59}{}\ledrightnote{\textcolor{pink}{Reuterweg}}, bei \textsc{Dr. \textcolor{blue}{Rosengart}{}\ledrightnote{\textcolor{blue}{Josef Rosengart}}}.}{\lemma{\textnormal{\emph{Bitte, … Rosengart.}}}\Cendnote{\textnormal{kopfüber am oberen Rand der
                     ersten Seite}}}\label{T_L03097-1h}\pend
           \endnumbering\briefempfaengerindex{Schnitzler, Olga@\textsc{Schnitzler, Olga}!zzzGoldmann, Paul@\emph{von Paul Goldmann}!1901-12-231@{23. 12. {[}1901{]}}|)be}\briefempfaengerindex{Schnitzler, Arthur@\textsc{Schnitzler, Arthur}!zzzGoldmann, Paul@\emph{von Paul Goldmann}!1901-12-231@{23. 12. {[}1901{]}}|)be}\mylabel{h}
\begin{anhang}
\end{anhang}\normalsize

\doendnotes{C}
\bigskip
\vfill

\clearpage

\footnotesize

\lohead{\textsc{register}}

% Definiere theindex-Environment komplett neu ohne reledmac
\makeatletter
\renewenvironment{theindex}{%
  \section*{\indexname}%
  \setlength{\parindent}{0pt}%
  \setlength{\parskip}{0pt plus 0.3pt}%
  \let\item\@idxitem
}{%
  \clearpage
}
\makeatother

\IfFileExists{\jobname-pw.ind}{\input{\jobname-pw.ind}}{}

\end{document}

      