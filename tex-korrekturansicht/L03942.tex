%% latex-korrekturansicht-vorspann.tex
%% Vorspann für die Korrekturansicht.
%% Lädt die gemeinsame Datei latex-vorspann.tex mit gesetztem Schalter.

\newif\ifkorrekturansicht
\korrekturansichttrue

\input{../tex-inputs/latex-vorspann}


\section[Arthur Schnitzler an Theodor Herzl, 30. 12. 1892]{L03942 Arthur Schnitzler an Theodor Herzl, 30. 12. 1892}
\nopagebreak\mylabel{L03942v}
\rehead{ }\normalsize\beginnumbering\briefempfaengerindex{, @\textsc{, }!zzz, @\emph{von  }!1892-12-301@{30. 12. 1892}|(be}
\toendnotes[C]{\smallbreak\pagebreak[2]}\Standort{Wien, Österreichische Gesellschaft für Literatur, Abschrift Herzl.}
\physDesc{Brief, maschinenschriftliche Abschrift, 1 Blatt, 1 Seite, 2045 Zeichen
\newline{}maschinell}
\buchAbdrucke{\weitereDrucke{Arthur Schnitzler: \emph{Briefe 1875–1912}. Herausgegeben von Therese Nickl und Heinrich Schnitzler. Frankfurt am Main: \emph{S. Fischer} 1981, S. 142–143.} }\toendnotes[C]{\smallbreak}
\pstart
           {\pb}S 4\pend
           
\pstart{}Verehrtester Freund,\pend\vspace{0.5em}
\pstart
           nehmen Sie meine herzlichsten Neujahrsgrüße entgegen! Ich sende Ihnen dieselben mit
               besonderer Freude, denn wenn ich so die Ergebnisse des heurigen Jahres überschaue, so
               finde ich, daß jener \label{K_L03941-1v}\edtext{Brief}{\lemma{\textnormal{\emph{Brief}}}\Cendnote{\textnormal{Theodor Herzl an Arthur Schnitzler, 29. 7. 1892. }}}\label{K_L03941-1}, mit welchem Sie sich als
               einen so liebenswürdigen Betrachter des \textcolor{green}{Märchen}\pwindex{Schnitzler, Arthur 15. 5. 1862 Wien – 21. 10. 1931 ebd.@\textsc{Schnitzler, Arthur} (15. 5. 1862 Wien – 21. 10. 1931 ebd.), \emph{Schriftsteller, Mediziner}!Märchen. Schauspiel in drei Aufzügen@\strich\emph{Das Märchen. Schauspiel in drei Aufzügen}|pw}{}\ledrightnote{\textcolor{green}{Das Märchen. Schauspiel in drei Aufzügen}}
               zu erkennen gaben und zugleich manche Mißverständnisse unserer bisherigen Beziehungen
               lösten, zu den wärmsten und wohltuendsten Erlebnissen meines 92er Jahres gehören. Ich
               stehe in meiner eigenen Anerkennung noch nicht fest genug, um eine Liebenswürdigkeit
               wie die Ihre nicht besonders stark zu empfinden. Es wundert mich umsomehr, daß Sie
               mir noch bis zu einem gewissen Grad zu mißtrauen scheinen. Die Gründe, mit welchen
               Sie mein Ersuchen um einige Ihrer Arbeiten ablehnen, veranlassten mich zu dieser
               Bemerkung. Sie, mein lieber und verehrter Freund, stehen auf meine »reciproke«
               Anerkennung gewiß nicht an, und ich meinerseits glaube vor dem Verdacht sicher zu
               sein, aus dem Bedürfnis Revanchefreundlichkeiten auszutheilen mich für Ihre
               Manuscripte zu interessieren. Daß Sie manches Dramatische geschrieben haben, daß Sie
               auch jetzt für gut halten, geht aus \label{K_L03941-2v}\edtext{einem Ihrer Briefe}{\lemma{\textnormal{\emph{einem Ihrer Briefe}}}\Cendnote{\textnormal{XXXX}}}\label{K_L03941-2} mit Sicherheit hervor, und
               wenn Sie vor zehn oder zwölf Jahren nicht bezweifelt haben, daß ich mich für Ihre
               Stücke interessire, so liegt heute wohl auch kein Grund dafür vor. Es wäre doch
               ganz schön, wenn aus der Formel, welche wir beide über den Anfang unserer Briefe
               setzen, auch ein Inhalt flösse. Einigen wir uns dahin, daß wir durchaus keinen Grund
               haben, in Phrasen miteinander zu correspondiren, und daß jeder Satz, welcher einer
               dem andern schreibt diesen verbindlich macht, jenem Satze zu glauben. Das ist
               natürlich keine Erpressung, als wenn Sie mir nun unbedingt was schicken müßten; aber
               ein Ersuchen ist es, in meinen Worten an Sie mehr als Höflichkeit sehen zu wollen.
               Ich war ja so frei, auch die {\pb}die Ihren als etwas besseres zu nehmen. – Und nun leben Sie wohl und seien
               Sie meiner aufrichtigen und wärmsten Hochschätzung versichert\pend
           
\pstart
           Ihr{\\[\baselineskip]}\spacefill\mbox{Arth. Schnitzler}\pend
           \leftskip=0em{}
\pstart
           30/12 92\pend
           \selectlanguage{ngerman}\endnumbering\briefempfaengerindex{, @\textsc{, }!zzz, @\emph{von  }!1892-12-301@{30. 12. 1892}|)be}\mylabel{L03942h}
\begin{anhang}
\end{anhang}\normalsize

\doendnotes{C}
\bigskip
\vfill

\clearpage

\footnotesize

\lohead{\textsc{register}}

% Definiere theindex-Environment komplett neu ohne reledmac
\makeatletter
\renewenvironment{theindex}{%
  \section*{\indexname}%
  \setlength{\parindent}{0pt}%
  \setlength{\parskip}{0pt plus 0.3pt}%
  \let\item\@idxitem
}{%
  \clearpage
}
\makeatother

\IfFileExists{\jobname-pw.ind}{\input{\jobname-pw.ind}}{}

\end{document}

      