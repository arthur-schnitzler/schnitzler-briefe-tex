%% latex-korrekturansicht-vorspann.tex
%% Vorspann für die Korrekturansicht.
%% Lädt die gemeinsame Datei latex-vorspann.tex mit gesetztem Schalter.

\newif\ifkorrekturansicht
\korrekturansichttrue

\input{../tex-inputs/latex-vorspann}


\renewcommand{\erwaehntePersonen}{Personen: Hermann Bahr, Albert Bassermann, Oskar Blumenthal, Otto Brahm, Ludwig Fulda, Gerhart Hauptmann, Moritz Heimann, Georg Hirschfeld, Alfred Holzbock, Siegfried Jacobsohn, Gustav Kadelburg, Alfred Kerr, Sigmund Lautenburg, Max Reinhardt, Peter Rotenstern, Felix Salten, Paul Schlenther, Franz von Schönthan-Pernwald, Siegfried Trebitsch, Jakob Wassermann}
\renewcommand{\erwaehnteInstitutionen}{Institutionen: Burgtheater, Lessing-Theater}
\renewcommand{\erwaehnteOrte}{Orte: Berlin, Deutsches Theater Berlin, Heiligenstadt, Hradec Králové, Magenta, Solferino, Wien}
\renewcommand{\erwaehnteWerke}{Werke: ?? [Romanfragmente von Gerhart Hauptmann], Der Biberpelz. Eine Diebskomödie, Der Fall Brahm, Der Fall Hauptmann, Der Hervorruf und unsere dramatischen Autoren, Der Schleier der Beatrice. Schauspiel in fünf Akten, Der Tag, Der Tag, Die Jungfern vom Bischofsberg. Lustspiel, Die Schaubühne, Die Weber. Schauspiel aus den vierziger Jahren, Die Zeit, Florian Geyer. Die Tragödie des Bauernkrieges, Geleitsworte, Gesammelte Werke in sechs Bänden, Hanneles Himmelfahrt. Traumdichtung in zwei Teilen, Neues Wiener Journal, Romeo and Juliet, Tartuffe, Vor Sonnenaufgang, »Die Jungfern vom Bischofsberg.« Lustspiel von Gerhart Hauptmann. Erstaufführung im Lessing-Theater}
\section[ Felix Salten an Arthur Schnitzler, 20. 4. 1907]{Felix Salten an Arthur Schnitzler, 20. 4. 1907}
\nopagebreak\mylabel{v}
\rehead{ }\normalsize\beginnumbering\briefempfaengerindex{Schnitzler, Arthur@\textsc{Schnitzler, Arthur}!zzzSalten, Felix@\emph{von Felix Salten}!1907-04-201@{20. 4. 1907}|(be}
\toendnotes[C]{\smallbreak\pagebreak[2]}\Standort{CUL, Schnitzler, B 89, B 1.}
\physDesc{Brief, 1 Blatt, 1 Seite, 15274 Zeichen
\newline{}maschinell
\newline{}Handschrift: schwarze Tinte (\noindent{}Unterschrift)
\newline{}Beilagen: 1) Zeitungsausschnitt, 1 Blatt, 2 Seiten  2) maschineller Durchschlag einer Abschrift eines Briefes von \textcolor{blue}{Moritz
                                 Heimann}, 1 Blatt, 1 Seite 3) maschineller Durchschlag eines Briefes von \textcolor{blue}{Salten} an \textcolor{blue}{Moritz Heimann}, 8 Blatt, 8 Seiten, paginiert:
                                    »2«–»8«, teilweise minimale
                                 Korrekturen mit schwarzer Tinte, die in der Wiedergabe übernommen sind
\newline{}Ordnung: mit Bleistift von unbekannter Hand nummeriert: »228« }\toendnotes[C]{\smallbreak}
\pstart
           \raggedleft{}{\pb}\textcolor{pink}{Wien-Heiligenstadt}{}\ledrightnote{\textcolor{pink}{Heiligenstadt}}, 20. April 1907\pend
           
\pstart{}Lieber,\pend
\pstart
           beigeschlossen sende ich Ihnen den \label{K_L03438-1v}\edtext{Fall \textcolor{blue}{Heimann}{}\ledrightnote{\textcolor{blue}{Moritz Heimann}}}{\lemma{\textnormal{\emph{Fall Heimann}}}\Cendnote{\textnormal{siehe A. S.: \emph{Tagebuch}, 25. 3. 1907}}}\label{K_L03438-1h}, zu dem sich eine weitere Bemerkung ja erübrigt. Mit \textcolor{blue}{Lautenburg}{}\ledrightnote{\textcolor{blue}{Sigmund Lautenburg}} werde ich wegen des Herrn \textcolor{blue}{Rothenstern}{}\ledrightnote{\textcolor{blue}{Peter Rotenstern}} sprechen. Hoffentlich \label{K_L03438-2v}\edtext{sehen wir uns bald}{\lemma{\textnormal{\emph{sehen wir uns bald}}}\Cendnote{\textnormal{Nachweisbar sahen sie sich am 29. 4. 1907 wieder.}}}\label{K_L03438-2h}.\pend
           
\pstart
           Herzlichst Ihr {\\[\baselineskip]}{[}hs.:{]} \spacefill\mbox{Salten}\pend
           \leftskip=0em{}{\bigskip}
\pstart
           \noindent{}\centering{}{\pb}\textcolor{gray}{\textbf{\textbf{\label{K_L03438-3v}\edtext{Feuilleton.}{\lemma{\textnormal{\emph{Feuilleton.}}}\Cendnote{\textnormal{\textcolor{blue}{Felix Salten}: \emph{\textcolor{green}{Der Fall Hauptmann}}. In: \emph{\textcolor{green}{Die Zeit}}, Jg. 6, Nr. 1.576, 12. 2. 1907, Morgenblatt, S. 1–2.}}}\label{K_L03438-3h}}}}\pend
           
\pstart
           \noindent{}\centering{}\textcolor{gray}{\textbf{\textbf{\textcolor{green}{\so{Der Fall Hauptmann}}{}\ledrightnote{\textcolor{green}{Der Fall Hauptmann}}.}}}\pend
           
\pstart
           \noindent{}\textcolor{gray}{\textbf{Im \textcolor{brown}{Leſſing-Theater}{}\ledrightnote{\textcolor{brown}{Lessing-Theater}} iſt das
                  neue \textcolor{green}{Stück}{}\ledrightnote{{$\rightarrow$}\textcolor{green}{Die Jungfern vom Bischofsberg. Lustspiel}} von \textcolor{blue}{Gerhart Hauptmann}{}\ledrightnote{\textcolor{blue}{Gerhart Hauptmann}} durchgefallen. Aber nicht
                  ſo einfach durchgefallen, wie ſonſt wohl andere Stücke, die eben keine Gnade und
                  keinen Applaus finden. Sie haben es ausgelacht, verhöhnt, bebrüllt und bejohlt;
                  haben das Gewebe der Handlung, während es noch vor ihnen abrollte, mit ihren
                  Wutausbrüchen in Fetzen geriſſen, haben mit ihrem Spott bei offener Bühne die
                  Worte, die ſich hervorwagten, abgefangen, ſie verdreht und ihnen das Antlitz
                  entſtellt oder ſie mit ihrem Schimpf kurzweg niedergeſchlagen. Man fragt ſich, wie
                  das geſchehen konnte. Die tauſendköpfige Beſtie hat den Dichter, als er (»\textcolor{green}{Vor Sonnenaufgang}{}\ledrightnote{\textcolor{green}{Vor Sonnenaufgang}}«), ein neuer Mann, vor ſie
                  hintrat, giftig angefaucht. Vor vielen Jahren. Seither hielt er ſie gebändigt und
                  gezähmt, an manchem Abend. Und ſie fraß aus ſeiner Hand. Nun konnte ſie diesmal
                  ſeinem Zwang entſpringen, ſeine Feſſeln ſo völlig abwerfen und ihm die Zähne
                  fletſchen wie einſt? Iſt ihm da unverſehens ein Malheur paſſiert? Oder{\dots} Philiſter über dir, \textcolor{blue}{Gerhart Hauptmann}{}\ledrightnote{\textcolor{blue}{Gerhart Hauptmann}}!{\dots} iſt die Kraft von ihm
                  gewichen?}}\pend
           {\bigskip}
\pstart
           \noindent{}\textcolor{gray}{\textbf{Jetzt liegt auch die Buchausgabe der »\textcolor{green}{Jungfern vom Biſchofsberg}{}\ledrightnote{\textcolor{green}{Die Jungfern vom Bischofsberg. Lustspiel}}« vor. Und lieſt man dies neue
                  Werk von \textcolor{blue}{Gerhart Hauptmann}{}\ledrightnote{\textcolor{blue}{Gerhart Hauptmann}}, ruhig,
                  unbeirrt, überlegſam und mit allem guten Willen, dann zeigt es ſich, daß dem \textcolor{pink}{Berlin}{}\ledrightnote{\textcolor{pink}{Berlin}}er Premierenvolk kein Meiſterwerk zum
                  Opfer fiel. Kultiviertere, an alten, erlauchten Traditionen erzogene
                  Theaterbeſucher hätten wahrſcheinlich gefühlt, daß ſie dem Dichter der »\textcolor{green}{Weber}{}\ledrightnote{\textcolor{green}{Die Weber. Schauspiel aus den vierziger Jahren}}«, des »\textcolor{green}{Hannele}{}\ledrightnote{\textcolor{green}{Hanneles Himmelfahrt. Traumdichtung in zwei Teilen}}« und noch zwölf anderer großer Kunſtwerke Reſpekt ſchulden, und
                  hätten nicht zum Hausſchlüſſel gegriffen. Aber alle hätten dieſes \textcolor{green}{Stück}{}\ledrightnote{{$\rightarrow$}\textcolor{green}{Die Jungfern vom Bischofsberg. Lustspiel}} fallen laſſen. Nach genauer,
                  wohlwollender, pietätvoller Prüfung dieſes Luſtſpiels muß man ein Urteil
                  beſtätigen, das gewiß allzu ſchreiend, allzu unhöflich im Ton, allzu hitzig und
                  turbulent abgegeben wurde. Das aber gerecht iſt. Leider Gottes. Leer und banal in
                  ſeiner Handlung iſt dieſes Stück. Gequält und mühſam in ſeinen Geſtalten. Armſelig
                  und atemlos in ſeinem Dialog. Albern, leider Gottes, albern, wo es ſpaßhaft ſein
                  will. Und ohnmächtig, wo es nach Humor ringt. Irgendein ganz matter, ganz leiſer
                  Schimmer von perſönlich nahe erlebten Dingen, von perſönlich nahe geſchauten
                  Menschen haftet manchmal an dieſen Figuren. Wer dem Kreis, aus dem dies Stück
                  geholt wurde, angehört, wer tiefer hineingeſchaut hat, dem mag dieſer Schimmer
                  heller, vertrauter, aufklärender glänzen. Der mag vielleicht auch erraten, was
                  hier die dichteriſche Abſicht geweſen. Herausgekommen, ſichtbar und deutlich
                  geworden iſt ſie nicht. Leider Gottes.}}\pend
           
\pstart
           \centering{}\textcolor{gray}{\textbf{*}}\pend
           
\pstart
           \noindent{}\textcolor{gray}{\textbf{Und ſo erkennt man: die »\textcolor{green}{Jungfern vom Biſchofsberg}{}\ledrightnote{\textcolor{green}{Die Jungfern vom Bischofsberg. Lustspiel}}«, das iſt keineswegs nur ein mißlungenes Werk:
                  das iſt eine Kriſis. (Hoffentlich keine Katastrophe.) Mißlingen kann jedem
                  Künſtler einmal ein Werk. Was liegt daran? Der Größte verfehlt manchmal den Kern
                  eines Stoffes, erwiſcht ihn nicht, verrennt ſich und ſcheitert mit irgendeinem
                  beſonderen Wollen. Aber er darf nicht unter ſeinem Niveau ſcheitern. \textcolor{blue}{Gerhart Hauptmann}{}\ledrightnote{\textcolor{blue}{Gerhart Hauptmann}} iſt hier auf einmal weit
                  hinter ſich ſelbſt zurück, tief unter ſeinem Rang. Wir ſahen ihn noch nie in
                  ſolcher Niederung. Beiſpielmäßig: es gibt einige sehr ſchwächliche Stücke von \textcolor{blue}{Georg Hirſchfeld}{}\ledrightnote{\textcolor{blue}{Georg Hirschfeld}}, die ſich ausnehmen wie ein
                  ſchwacher Abklatſch von \textcolor{blue}{Gerhart Hauptmann}{}\ledrightnote{\textcolor{blue}{Gerhart Hauptmann}}.
                  Dieſes \textcolor{green}{Luſtſpiel}{}\ledrightnote{{$\rightarrow$}\textcolor{green}{Die Jungfern vom Bischofsberg. Lustspiel}} von \textcolor{blue}{Hauptmann}{}\ledrightnote{\textcolor{blue}{Gerhart Hauptmann}} aber nimmt ſich aus wie ein
                  ſchwacher Abklatſch von \textcolor{blue}{Georg Hirſchfeld}{}\ledrightnote{\textcolor{blue}{Georg Hirschfeld}}.
                  Das eben iſt ſo verwirrend. Er erſcheint hier als der Epigone ſeiner eigenen
                  Epigonen. Man hat den Eindruck: jemand, der vom Seſſel gefallen iſt.}}\pend
           
\pstart
           \centering{}\textcolor{gray}{\textbf{*}}\pend
           
\pstart
           \noindent{}\textcolor{gray}{\textbf{ Man hat den Eindruck: ein Abſturz. Die treueſten kritiſchen
                  Anhänger verlaſſen \textcolor{blue}{Hauptmann}{}\ledrightnote{\textcolor{blue}{Gerhart Hauptmann}} jetzt wie die
                  vielberufenen Ratten das ſinkende Schiff. Seine begeiſterten Schild- und
                  Schwertträger. Und ſeine alten Gegner lächeln triumphierend. Jeder von ihnen fühlt
                  ſich als ein Prophet: »Ich hab’ es ja immer geſagt.« (Was natürlich ekelhaft iſt.)
                  Jetzt ſtellen ſich die Anklagen ein, die Vorwürfe und Ratſchläge. Auch möchte man
                  Erklärungen finden für dieſen merkwürdigen Fall. Ein müder Mann, heißt es, der
                  ausruhen ſollte. Sein Geiſt ſoll brachliegen eine Weile, wie ein Acker, der
                  allzuoft nacheinander hat Ernten tragen müſſen. Natürlich, rufen andere, es war zu
                  viel; jedes Jahr ein Stück. Das geht über ſeine Kraft. Dann wird der Direktor \textcolor{blue}{Brahm}{}\ledrightnote{\textcolor{blue}{Otto Brahm}} hineinverwickelt. Hat denn der nicht
                  geſehen, wie ſchlecht das neue \textcolor{green}{Werk}{}\ledrightnote{{$\rightarrow$}\textcolor{green}{Die Jungfern vom Bischofsberg. Lustspiel}} iſt? Wär’s nicht ſeine Pflicht geweſen, den \textcolor{blue}{Freund}{}\ledrightnote{{$\rightarrow$}\textcolor{blue}{Gerhart Hauptmann}} zu warnen, ihm, wenn’s nicht
                  anders ging, die \textcolor{pink}{Bühne}{}\ledrightnote{{$\rightarrow$}\textcolor{pink}{Deutsches Theater Berlin}} zu
                  verſchließen? Zuletzt gegen \textcolor{blue}{Hauptmann}{}\ledrightnote{\textcolor{blue}{Gerhart Hauptmann}} die
                  Beſchuldigung menſchlicher und künſtleriſcher Leichtfertigkeit.}}\pend
           
\pstart
           \centering{}\textcolor{gray}{\textbf{*}}\pend
           
\pstart
           \noindent{}\textcolor{gray}{\textbf{Ich möchte, in \label{T_L03438-1v}\edtext{Parentheſe}{\lemma{\textnormal{\emph{Parentheſe}}}\Cendnote{\textnormal{im Druck steht
                        »Parantheſe«}}}\label{T_L03438-1h}, ein Wort für \textcolor{blue}{Brahm}{}\ledrightnote{\textcolor{blue}{Otto Brahm}} einlegen. Denn ich glaube, daß ihm doch ein wenig
                  Unrecht geſchieht. Auch dann Unrecht, wenn er, wie ſich’s von ſeinem Urteil
                  erwarten läßt, die »\textcolor{green}{Jungern vom Bischofsberg}{}\ledrightnote{\textcolor{green}{Die Jungfern vom Bischofsberg. Lustspiel}}«
                  von Anfang an für ſchlecht gehalten hat. Durfte er denn wirklich einem Stück von
                     \label{K_L03438-4v}\edtext{\textcolor{blue}{Gerhart Hauptmann}{}\ledrightnote{\textcolor{blue}{Gerhart Hauptmann}} ſein \textcolor{brown}{Theater}{}\ledrightnote{{$\rightarrow$}\textcolor{brown}{Lessing-Theater}}}{\lemma{\textnormal{\emph{Gerhart … Theater}}}\Cendnote{\textnormal{Die
                     Argumentation über den Flop der \emph{\textcolor{green}{Jungern vom
                        Bischofsberg}} hat an dieser Stelle einige Ähnlichkeit mit der Debatte
                     über \emph{\textcolor{green}{Der Schleier der Beatrice}} von \textcolor{blue}{Schnitzler}, das \textcolor{blue}{Paul Schlenther} im Frühjahr 1900 für das
                        \emph{\textcolor{brown}{Burgtheater}} ablehnte. In dem von \textcolor{blue}{Salten} maßgeblich betriebenen öffentlichen
                     Protest der Theaterkritiker heißt es: »Wir stellen die Qualitäten dieses
                        Werkes in dem vorliegenden Falle gänzlich außer Discussion und lassen ebenso
                        die allenfalls naheliegende Frage unerörtert, ob ein Stück von \textcolor{blue}{Arthur Schnitzler} nicht auch dann einen gewissen Anspruch darauf hat, der
                        Oeffentlichkeit und der Kritik im Verlaufe zweier Jahre vorgeführt zu
                        werden, wenn es (\textsc{error possibilis}) der Meinung des Directors zufolge
                        zweifelhafte Erfolgaussichten besitzt.« Bahr/Schnitzler, T030017}}}\label{K_L03438-4h} verweigern? Das
                  ſchlagende Argument des Premierenſkandals, mit dem jetzt alle ſo bequem und ſo
                  unwiderſprechlich hantieren, ſtand ihm doch nicht im Angeſicht des \textcolor{green}{Manuſkript}{}\ledrightnote{{$\rightarrow$}\textcolor{green}{Die Jungfern vom Bischofsberg. Lustspiel}}s zu Gebote. Vielleicht
                  verwarf \textcolor{blue}{Hauptmann}{}\ledrightnote{\textcolor{blue}{Gerhart Hauptmann}} die Prophezeiung, hätte
                  vielleicht Ratgeber gefunden, die ein günſtigeres Horoſkop ſtellten. War denn die
                  Gefahr ausgeſchloſſen, daß \textcolor{blue}{Hauptmann}{}\ledrightnote{\textcolor{blue}{Gerhart Hauptmann}}, den
                  freundlicheren Weisſagern trauend und dem Schwarzſeher, \textcolor{blue}{Brahm}{}\ledrightnote{\textcolor{blue}{Otto Brahm}} zürnend, zu \textcolor{blue}{Reinhardt}{}\ledrightnote{\textcolor{blue}{Max Reinhardt}} ging? Wenn dann das \textcolor{green}{Stück}{}\ledrightnote{{$\rightarrow$}\textcolor{green}{Die Jungfern vom Bischofsberg. Lustspiel}} auch bei \textcolor{blue}{Reinhardt}{}\ledrightnote{\textcolor{blue}{Max Reinhardt}} fiel, blieb noch immer das ärgerliche Räſonnement: Ja, wenn \textcolor{blue}{Brahm}{}\ledrightnote{\textcolor{blue}{Otto Brahm}} gewollt hätte {\dots} im \textcolor{brown}{Leſſing-Theater}{}\ledrightnote{\textcolor{brown}{Lessing-Theater}}, mit \textcolor{blue}{Baſſermann}{}\ledrightnote{\textcolor{blue}{Albert Bassermann}}, wäre nichts Schlimmes paſſiert.
                  Ich glaube, \textcolor{blue}{Brahm}{}\ledrightnote{\textcolor{blue}{Otto Brahm}} war gar nicht in der
                  Lage, hier etwas zu verhindern, hätte ſeinem Hauſe nur dieſen für ihn wichtigſten
                  Dichter verloren, was nicht zu riskieren war. Ganz abgeſehen davon, daß \textcolor{blue}{Hauptmann}{}\ledrightnote{\textcolor{blue}{Gerhart Hauptmann}}, geſtützt auf ſeine Erfolge, den
                  Anſpruch hat, mit jedem Stück einfach angenommen und geſpielt zu werden. Und daß
                  er ſchließlich nicht unter \textcolor{blue}{Brahm}{}\ledrightnote{\textcolor{blue}{Otto Brahm}}s Kuratel
                  ſteht.}}\pend
           
\pstart
           \centering{}\textcolor{gray}{\textbf{*}}\pend
           
\pstart
           \noindent{}\textcolor{gray}{\textbf{Er iſt ganz allein verantwortlich; hat es auch neulich ſelbſt
                  geſagt, daß er »\label{K_L03438-5v}\edtext{jederzeit bereit sei, vor ſein \textcolor{green}{Werk}{}\ledrightnote{{$\rightarrow$}\textcolor{green}{Die Jungfern vom Bischofsberg. Lustspiel}} zu treten}{\lemma{\textnormal{\emph{jederzeit … treten}}}\Cendnote{\textnormal{In Erneuerung einer
                     Umfrage von \textcolor{blue}{Hermann Bahr} (siehe Arthur Schnitzler an Hermann Bahr, Antwort auf eine Umfrage,
               15. 2. 1899) fragte
                     \textcolor{blue}{Alfred Holzbock} im \emph{\textcolor{green}{Tag}}, ob 
                     Autoren sich bei Aufführungen auf die Bühne stellen sollten. \textcolor{blue}{Gerhart Hauptmann} wird 
                     zitiert mit: »Ich bin jederzeit bereit, zwar nicht vor das Publikum, wohl aber vor mein Werk zu treten.«
                     (nach \emph{\textcolor{green}{Der Hervorruf und unsere dramatischen Autoren}}. In: \emph{\textcolor{green}{Neues
                        Wiener Journal}}, Jg. 15, Nr. 4.470, 1. 1. 1907, S. 15.)}}}\label{K_L03438-5h}«. Leichtfertigkeit wird man ihm nicht
                  vorwerfen dürfen. Wer einmal ſein Geſicht geſehen hat, denkt nicht an dergleichen.
                  Die Bilder, die von ihm verbreitet ſind, geben von dieſem Geſicht nur wenig. Geben
                  nur einen falſchen Begriff davon. Keines gibt den edlen Glanz, der auf dieſem
                  Antlitz ruht, keines dieſe leuchtende Unberührtheit ſeiner Mienen. Kein Bild gibt
                  dieſen Ausdruck von knabenhafter, unendlicher Güte, der um ſeine feinen Lippen
                  schwebt. Kein Bild gibt auch die tiefe Heiterkeit ſeiner ſtrahlenden blauen Augen.
                  Ich habe ihn nur hin und wieder einmal, ganz flüchtig, geſehen, aber ich muß
                  ſagen: ich glaube {\pb}an \textcolor{blue}{Gerhart Hauptmann}{}\ledrightnote{\textcolor{blue}{Gerhart Hauptmann}}, um ſeiner ſchönen Augen
                  willen.}}\pend
           
\pstart
           \centering{}\textcolor{gray}{\textbf{*}}\pend
           
\pstart
           \noindent{}\textcolor{gray}{\textbf{Lieber Gott, überhaupt das Perſönliche. Es iſt, namentlich in
                  einem Fall wie dieſem, das einzig Verläßliche. Irgendein Heuchler, der ſich
                  heimlich einmal den Kopf bebutterte, hat das \textcolor{green}{Tartüffe}{}\ledrightnote{{$\rightarrow$}\textcolor{green}{Tartuffe}}-Wort erfunden: Die wahre Kunſtkritik ſoll nie
                  perſönlich werden. Wie jede Lüge, die ſich praktiſch erweiſt und vielen Leuten
                  Vorteil bringt, hat man auch dieſe zum Grundſatz erhoben, hat ſich beeilt, dieſes
                  herrliche Axiom in Sicherheit zu bringen und jeglicher Debatte zu entrücken. In
                  Wirklichkeit aber ſollte die wahre Kunſtkritik gar nichts anderes ſein, als
                  perſönlich, ſo gewiß, als ja auch jede wahre Kunſt etwas rein Perſönliches iſt und
                  nur in perſönlichen Eigenſchaften des Charakters, des Gemüts und im perſönlichen
                  Erleben ihre verborgenſten Quellen hat. Iſt einer tot, dann freilich{\dots}, dann wirft die Kunſtkritik ſchleunigſt dieſen
                  famoſen Grundſatz beiſeite und wird perſönlich. Aber dann iſt es meiſtens ſchon zu
                  ſpät. Erſtens weil dann die Profeſſoren kommen (was immer ein Malheur iſt) und mit
                  toten Dokumenten arbeiten. Und zweitens, weil dann die lebendigen Zeugen, die aus
                  unmittelbarer Anſchauung pſychologiſch Schöpfenden nicht mehr da ſind. Wie viel
                  wichtige Zeitgeſchichte, wie viel rätſellöſendes, unſchätzbares Material geht ſo
                  verloren. Wie aufklärend, wenn man von einem Dichter ſagen dürfte: er iſt ein
                  enger, habſüchtiger, neidiſcher Menſch, voll Beſchränktheit und kleiner Laſter.
                  Oder von einem anderen: er hat eine rein muſikaliſch-formale Begabung, aber er iſt
                  ſo grenzenlos dumm, deshalb kann er euch nur ein paar Verſe, aber nie eine Geſtalt
                  oder ein Weltbild geben. Oder von einem Schauſpieler: er iſt verlogen,
                  hinterliſtig und voll Tücke, deshalb ſpielt er die Biedermänner mit der heißen
                  verſchwiegenen Sehnſucht, für einen ehrlichen Kerl zu gelten, ſo famos. Sein
                  ganzes Spieltalent entſpringt dem Wunſche, ſeinen Charakter zu verbergen, ſich zu
                  verſtellen.}}\pend
           
\pstart
           \centering{}\textcolor{gray}{\textbf{*}}\pend
           
\pstart
           \noindent{}\textcolor{gray}{\textbf{\textcolor{blue}{Gerhart Hauptmann}{}\ledrightnote{\textcolor{blue}{Gerhart Hauptmann}} iſt ſicher durch
                  perſönliche Erlebniſſe, durch Wandlungen und Geſchehnisse perſönlichſter Art zu
                  dieſem \textcolor{green}{Stück}{}\ledrightnote{{$\rightarrow$}\textcolor{green}{Die Jungfern vom Bischofsberg. Lustspiel}}
                  herabgeglitten. Und hat’s vielleicht deshalb gerade nicht bemerkt, daß er
                  herabglitt. Wollte ich in dieſer wichtigen Angelegenheit, in der wir dieſen
                  plötzlichen Kräfteverfall unſeres ſtärkſten Dramatikers betrachten, wollte ich
                  diesmal den lügneriſchen Grundſatz, an den ich ohnehin nicht glaube, beiſeite
                  laſſen, ich könnte nichts Poſitives anführen, weil ich \textcolor{blue}{Hauptmann}{}\ledrightnote{\textcolor{blue}{Gerhart Hauptmann}} nicht nahe genug ſtehe, um Einblick in ſein
                  perſönliches Walten, in ſeinen Charakter zu haben. Aber ich bin felſenfeſt davon
                  überzeugt, daß es irgendwie mit ihm nicht in Ordnung iſt. Nicht mit ſeinem Weſen,
                  denn an dieſes, an dieſe adelige Menſchlichkeit \textcolor{blue}{Hauptmann}{}\ledrightnote{\textcolor{blue}{Gerhart Hauptmann}}s glaube ich. Wohl aber mit ſeinem Schickſal. Ein müder Mann?
                  Das Gerede von ſeiner Müdigkeit halte ich für Unſinn. Wenn man fünfundvierzig
                  Jahre alt iſt, ſteht man in der Fülle der Kraft. Wo hätte ſie \textcolor{blue}{Gerhart Hauptmann}{}\ledrightnote{\textcolor{blue}{Gerhart Hauptmann}} verbraucht? Er hat ohne Amt, ohne
                  Berufsarbeit ſeit zwanzig Jahren nur ſeinem Schaffen gelebt. Auf dem Lande, auf
                  Reiſen. Von überall her Anregung und Erfriſchung empfangend. Dafür ſind ſechzehn
                  Dramen keine Arbeit, die einen Mann umwirft und ermüdet. Ein Jahr iſt lang, und
                  wenn man nichts anderes tut, kann einem produktiven Menſchen in zwölf Monaten doch
                  ein Stück gedeihen. Fertig? Ach, ich weiß, es gibt ſo viele ſchöne Seelen, die
                  immer gern ſchreien: der iſt fertig! Am liebſten hätten ſie, wenn alle
                  ſchöpferiſchen Geiſter »fertig« wären. \textcolor{blue}{Hauptmann}{}\ledrightnote{\textcolor{blue}{Gerhart Hauptmann}} hat ſo viele Gleiſe gelegt. »\textcolor{green}{Die Weber}{}\ledrightnote{\textcolor{green}{Die Weber. Schauspiel aus den vierziger Jahren}}«, »\textcolor{green}{Hannele}{}\ledrightnote{\textcolor{green}{Hanneles Himmelfahrt. Traumdichtung in zwei Teilen}}«, »\textcolor{green}{Florian Geyer}{}\ledrightnote{\textcolor{green}{Florian Geyer. Die Tragödie des Bauernkrieges}}« uſw., daß man nicht annehmen
                  kann, er ſei fertig. Sicher iſt nur, daß er diesmal entgleiſt iſt. Und das
                  erſcheint mir bedenklich genug.}}\pend
           
\pstart
           \centering{}\textcolor{gray}{\textbf{*}}\pend
           
\pstart
           \noindent{}\textcolor{gray}{\textbf{Es gibt noch andere Bedenken. Das leere, banale \textcolor{green}{Vorwort}{}\ledrightnote{{$\rightarrow$}\textcolor{green}{Geleitsworte}}, das er
                  ſeinen \textcolor{green}{geſammelten Werken}{}\ledrightnote{\textcolor{green}{Gesammelte Werke in sechs Bänden}} in dieſem Winter
                  mitgab. Dann das beängſtigend ſchlechte Deutſch, das man in ſeinen kürzlich
                  veröffentlichten \label{K_L03438-6v}\edtext{\textcolor{green}{Romanfragmente}{}\ledrightnote{{$\rightarrow$}\textcolor{green}{?? [Romanfragmente von Gerhart Hauptmann]}}n}{\lemma{\textnormal{\emph{Romanfragmenten}}}\Cendnote{\textnormal{nicht ermittelt}}}\label{K_L03438-6h} bemerkte.
                  Vielleicht muß man trotz all ſeiner hohen Fähigkeit die Stellung, die er einnimmt,
                  jetzt revidieren. Er war ſo lange ein Wahrzeichen, war mehr ein Begriff als eine
                  Perſon. \textcolor{blue}{Hauptmann}{}\ledrightnote{\textcolor{blue}{Gerhart Hauptmann}}. Da mußte man für ihn,
                  für die Sache ſein, die ſeinen Namen trug. \textsc{\label{K_L03438-7v}\edtext{In hoc signo}{\lemma{\textnormal{\emph{In hoc signo}}}\Cendnote{\textnormal{lateinisch: in diesem
                        Zeichen}}}\label{K_L03438-7h}}{ }{\dots} oder gegen ihn. Parteifahne. Er war der große Sieg,
                  der Anno 89 von den Modernen erfochten wurde. Die Schlacht bei \textcolor{blue}{Hauptmann}{}\ledrightnote{\textcolor{blue}{Gerhart Hauptmann}}. Ein hiſtoriſcher Name. \textcolor{pink}{Königgrätz}{}\ledrightnote{\textcolor{pink}{Hradec Králové}}, \textcolor{pink}{Solferino}{}\ledrightnote{\textcolor{pink}{Solferino}},
                     \textcolor{pink}{Magenta}{}\ledrightnote{\textcolor{pink}{Magenta}} ſind ja auch kleine Neſter. Und
                  doch unſterblich. \textcolor{blue}{Hauptmann}{}\ledrightnote{\textcolor{blue}{Gerhart Hauptmann}} iſt nicht
                  klein. Aber die Schlacht bei \textcolor{blue}{Hauptmann}{}\ledrightnote{\textcolor{blue}{Gerhart Hauptmann}} iſt
                  am Ende größer geweſen, und wichtiger. Und jetzt tritt er uns auf einmal als er
                  ſelbſt entgegen. Als ein talentvoller Dichter, dem ein Luſtſpiel jämmerlich
                  verdarb. Das fromme Wort der unentwegt Andächtigen: »O, \textcolor{blue}{Hauptmann}{}\ledrightnote{\textcolor{blue}{Gerhart Hauptmann}}, meine Zuversicht! {\dots}«
                  wird allerdings für immer zunichte. Nehmt ihn, wie er iſt: ein Dramatiker von
                  Genie. Ein Dichter von Intuition, dem aber der feſte Halt eines tiefen
                  künſtleriſchen Intellekts manchmal verſagt iſt. Trifft er’s (von ſelbſt), dann
                  iſt’s herrlich. Trifft er’s nicht, dann iſt es unrettbar. Und da er nirgendwo in
                  ſeiner Seele und in ſeinem Geiſt ehern iſt, da ſeine Selbſterkenntnis nicht kalte
                  Augen, ſein Wille zur Selbſtentwicklung nicht ſtählerne Muskeln hat, brach er uns
                  endlich unter dem Prunkgewand des \textsc{Pontifex maximus}
                  zuſammen. Seine \textcolor{green}{Romanfragmente}{}\ledrightnote{{$\rightarrow$}\textcolor{green}{?? [Romanfragmente von Gerhart Hauptmann]}}, ſein \textcolor{green}{Vorwort}{}\ledrightnote{{$\rightarrow$}\textcolor{green}{Gesammelte Werke in sechs Bänden}}, ſein \textcolor{green}{Luſtſpiel}{}\ledrightnote{{$\rightarrow$}\textcolor{green}{Die Jungfern vom Bischofsberg. Lustspiel}} ſind Symptome, zeigen einen kindlichen Poeten, dem die
                  artiſtiſche Bewußtheit nicht gegeben ward. Nehmt ihn, wie er iſt. Und ihr habt
                  nicht wenig.}}\pend
           
\pstart
           \centering{}\textcolor{gray}{\textbf{*}}\pend
           
\pstart
           \noindent{}\textcolor{gray}{\textbf{Daß er gerade bei einem Luſtſpiel die Partie verlor, gerade hier
                  ſo ganz ohne Trümpfe blieb, iſt am Ende die wichtigſte Seite an der Sache. Das
                  deutſche Drama iſt ſeit dem Kampf, der Anno 89 geführt
                  wurde, befreit und erlöst. Das deutſche Luſtſpiel iſt nicht vorwärts gekommen.
                  Seltſam, daß gerade der Mann, auf den ſich nach der \textcolor{green}{Biberpelz}{}\ledrightnote{\textcolor{green}{Der Biberpelz. Eine Diebskomödie}}-Komödie alle Hoffnung richtete, in ſeinem erſten
                  wirklichen Bemühen ein \textcolor{green}{Luſtſpiel}{}\ledrightnote{{$\rightarrow$}\textcolor{green}{Die Jungfern vom Bischofsberg. Lustspiel}} liefert, bei dem man faſt verſucht wird, \textcolor{blue}{Ludwig Fulda}{}\ledrightnote{\textcolor{blue}{Ludwig Fulda}} all die Herbheit abzubitten, mit der man ſeine
                  ſüßen Nichtigkeiten abwies. Wir ſind im Luſtſpiel heute noch am ſelben Platz wie
                  89, haben \textcolor{blue}{Blumenthal}{}\ledrightnote{\textcolor{blue}{Oskar Blumenthal}}, \textcolor{blue}{Kadelburg}{}\ledrightnote{\textcolor{blue}{Gustav Kadelburg}}, \textcolor{blue}{Schönthan}{}\ledrightnote{\textcolor{blue}{Franz von Schönthan-Pernwald}}
                  noch nicht überwunden. Die Schlacht am \textcolor{green}{Biſchofsberg}{}\ledrightnote{\textcolor{green}{Die Jungfern vom Bischofsberg. Lustspiel}} iſt verloren. Und das moderne deutſche Luſtſpiel noch nicht
                  geſchrieben.}}\pend
           
\pstart
           \raggedleft{}\textcolor{gray}{\textbf{\textbf{Felix Salten.}}}\pend
           {\bigskip}
\pstart
           \noindent{}\centering{}{\pb}\uline{BRIEF DES HERRN \textcolor{blue}{MORITZ
                        HEIMANN}{}\ledrightnote{\textcolor{blue}{Moritz Heimann}} AN MICH.}\pend
           
\pstart
           \noindent{}Auf Ihren Brief hätte ich Ihnen gleich geantwortet, und wohl auch ohne einen solchen
               Ihnen geschrieben, wenn mir das Schreiben eines Briefes zur Zeit nicht so arg
               zusetzte. Man hat Sie nicht falsch berichtet, aber ich nehme an, dass man Ihnen auch
               den Grund dessen gesagt hat, was Sie meinen Groll nennen: es ist Ihr 
               \textcolor{green}{Aufsatz}{}\ledrightnote{{$\rightarrow$}\textcolor{green}{Der Fall Hauptmann}} über \textcolor{blue}{Hauptmann}{}\ledrightnote{\textcolor{blue}{Gerhart Hauptmann}}
                in der »\textcolor{green}{Zeit}{}\ledrightnote{\textcolor{green}{Die Zeit}}«, die \label{K_L03438-8v}\edtext{\textcolor{green}{Ergänzung}{}\ledrightnote{{$\rightarrow$}\textcolor{green}{Der Fall Brahm}}}{\lemma{\textnormal{\emph{Ergänzung}}}\Cendnote{\textnormal{\textcolor{blue}{Felix Salten}: \emph{\textcolor{green}{Der Fall Brahm}}. In: \emph{\textcolor{green}{Die Schaubühne}}, Jg. 3, Nr. 9, 28. 2. 1907, S. 221–225.}}}\label{K_L03438-8h} in der \textcolor{green}{Schaubü{[}h{]}ne}{}\ledrightnote{\textcolor{green}{Die Schaubühne}} bestätigt mir nur den
               Eindruck davon. Dass Sie ihn schrieben und wie Sie ihn schrieben! Diese schlecht
               verhehlte Freude, diese falsche Gerechtigkeit, diese Demaskierung – mit einem Wort –
               des kaltherzig berechnet leidenschaftlichen, des politisierenden Journalismus, –
               alles dies hat mich bis in den Grund empört. Sie fangen damit an, die Ereignisse der
                  \textcolor{green}{Premiere}{}\ledrightnote{{$\rightarrow$}\textcolor{green}{Die Jungfern vom Bischofsberg. Lustspiel}} zu beschreiben,
               und schön, anschaulich, mit aller wünsche{[}ns{]}werten »Poesie« zu
               schreiben und sind gar nicht dabei gewesen, – Ihre Freunde werden Ihnen, gefragt,
               sagen können, wie sich da{[}s{]} macht. Doch ich will und kann mich
               nicht auf die Einzelheiten einlassen und ich hoffe, dass auch Ihnen daran nichts
               liegt. Ich debattiere auch nicht mit Ihnen über \textcolor{blue}{Hauptmann}{}\ledrightnote{\textcolor{blue}{Gerhart Hauptmann}} und sein Werk, ich habe in dem \textcolor{green}{Artikel}{}\ledrightnote{{$\rightarrow$}\textcolor{green}{Der Fall Hauptmann}}{ }\uline{Sie} gelesen und das hat mich erregt; der \label{K_L03438-9v}\edtext{\textcolor{green}{Aufsatz}{}\ledrightnote{{$\rightarrow$}\textcolor{green}{»Die Jungfern vom Bischofsberg.« Lustspiel von Gerhart Hauptmann. Erstaufführung im Lessing-Theater}} von \textcolor{blue}{Kerr}{}\ledrightnote{\textcolor{blue}{Alfred Kerr}}}{\lemma{\textnormal{\emph{Aufsatz von Kerr}}}\Cendnote{\textnormal{\textcolor{blue}{Alfred Kerr}: \emph{\textcolor{green}{»Die Jungfern vom Bischofsberg.« Lustspiel von Gerhart
                        Hauptmann. Erstaufführung im Lessing-Theater}}. In: \emph{\textcolor{green}{Der Tag}}, Nr. 64, 5. 2. 1907, S. [1–2].}}}\label{K_L03438-9h} hat meinen Beifall gehabt (bis
               auf eine Stelle) auch das soll Ihnen sagen, was der Ihre mir gesagt und getan
               hat.\pend
           
\pstart
           Lassen Sie mich glauben, dass Sie nur eine Unklughei\textcolor{gray}{t} getan haben; aber es wäre nicht
               das erstemal, dass eine Unklugheit auch eine Unredlichkeit sein kann. Wenn irgend wer
               Ihnen geraten hat, den \textcolor{green}{Aufsatz}{}\ledrightnote{{$\rightarrow$}\textcolor{green}{Der Fall Hauptmann}} zu publizieren, oder auch nur nicht abgeraten hat, der hat Ihnen
               übel gedient, übler als ich in in diesem Augenblick.\pend
           \pstart \spacefill\mbox{\textcolor{blue}{Moritz Heimann}{}\ledrightnote{\textcolor{blue}{Moritz Heimann}}}\pend{}{\bigskip}
\pstart
           \raggedleft{}{\pb}\textcolor{pink}{Wien-Heiligenstadt}{}\ledrightnote{\textcolor{pink}{Heiligenstadt}}, 1\label{K_L03438-10v}\edtext{6}{\lemma{\textnormal{\emph{6}}}\Cendnote{\textnormal{Die Ziffer »6« wurde mit schwarzer
                        Tinte, wohl von \textcolor{blue}{Salten}, durch
                        Ergänzung eines Oberstrichs aus der getippten »0«
                        gebildet.}}}\label{K_L03438-10h}. April 1907.\pend
           
\pstart{}Lieber Herr \textcolor{blue}{Heimann}{}\ledrightnote{\textcolor{blue}{Moritz Heimann}}\pend
\pstart
           Sie bekommen meine Antwort erst heute, weil ich in
               diesen Tagen viel Wichtiges zu tun hatte. Und weil ich Ihnen nicht unter dem ersten
               Eindruck Ihres Briefes schreiben wollte.\pend
           
\pstart
           Mir wurde kein Grund angegeben, weshalb Sie meinetwegen Ihr Herz ausschütten. Sondern
               Herr \textcolor{blue}{Jacobsohn}{}\ledrightnote{\textcolor{blue}{Siegfried Jacobsohn}} schrieb: »\textcolor{blue}{Heimann}{}\ledrightnote{\textcolor{blue}{Moritz Heimann}} hat mir den Groll ausgeschüttet, den er gegen und
               über Sie auf dem Herzen hat«. Das Wort »Groll« wiederholte ich dann einfach.\pend
           
\pstart
           Herr \textcolor{blue}{Jacobsohn}{}\ledrightnote{\textcolor{blue}{Siegfried Jacobsohn}} fügte hinzu: »Schreiben Sie ihm
               selbst, wenn Sie Wert darauf legen, der Sache auf den Grund zu gehen.«\pend
           
\pstart
           Ich legte Wert darauf, und schrieb Ihnen. An meine \textcolor{green}{\textcolor{blue}{Hauptmann}{}\ledrightnote{\textcolor{blue}{Gerhart Hauptmann}}-Kritiken}{}\ledrightnote{{$\rightarrow$}\textcolor{green}{Der Fall Hauptmann}{\newline}{$\rightarrow$}\textcolor{green}{Der Fall Brahm}} dachte ich dabei
               gar nicht, denn die Wendung \textcolor{blue}{Jacobsohn}{}\ledrightnote{\textcolor{blue}{Siegfried Jacobsohn}}s »der
               Sache auf den Grund gehen« deutete mir nicht darauf hin. Ich setzte auch voraus, dass
               Sie vor einem offenen, mit aller Behutsamkeit, und – wie Sie wissen mussten – ohne
               Leichtsinn ausgesprochenen Urteil einige Achtung haben. Ein Irrtum, der nun
               aufgeklärt ist.\pend
           
\pstart
           Ich schrieb Ihnen, weil ich schon vor Monaten zu mehreren Leuten (darunter auch zu
                  \textcolor{blue}{Wassermann}{}\ledrightnote{\textcolor{blue}{Jakob Wassermann}} und \textcolor{blue}{Trebitsch}{}\ledrightnote{\textcolor{blue}{Siegfried Trebitsch}}) geäussert hatte, Ihr Benehmen gegen mich während
               meiner letzten \textcolor{pink}{Berlin}{}\ledrightnote{\textcolor{pink}{Berlin}}er Zeit und nachher sei mir
               merkwürdig versteckt erschienen. Also schon lange vor den »\textcolor{green}{Jungfern vom Bischofsberg}{}\ledrightnote{\textcolor{green}{Die Jungfern vom Bischofsberg. Lustspiel}}«. Das deutete ich Ihnen auch in
               meinem Briefe ziemlich lesbar an, und glaubte, Sie würden die Ihnen also gegebene
               Gelegenheit, aufrichtig zu sein, benützen. Ein Irrtum, der jetzt gleichfalls
               aufgeklärt ist.\pend
           
\pstart
           Ich antworte Ihnen ausführlich. Einfacher und kürzer {\pb}könnte ich auf Ihren Brief
               entgegnen: »\textcolor{green}{Ich bin kein Schurke,
                  Tybalt, ich seh’ Du kennst mich nicht – somit Lebwohl}{}\ledrightnote{{$\rightarrow$}\textcolor{green}{Romeo and Juliet}}«. Aber es zeigt sich, dass
               solche Milde übel angebracht ist und dass \textcolor{green}{Tybalt}{}\ledrightnote{{$\rightarrow$}\textcolor{green}{Romeo and Juliet}} bald darauf dennoch niedergeschlagen werden muss.
               Deshalb antworte ich Ihnen lieber gleich ausführlich und erspare das Lebwohl für den
               Schluss.\pend
           
\pstart
           Man braucht Ihren Brief nur neben meine \textcolor{green}{\textcolor{blue}{Hauptmann}{}\ledrightnote{\textcolor{blue}{Gerhart Hauptmann}}-Kritiken}{}\ledrightnote{{$\rightarrow$}\textcolor{green}{Der Fall Hauptmann}{\newline}{$\rightarrow$}\textcolor{green}{Der Fall Brahm}} zu legen und Ihre
               ganze Taktik enthüllt sich im Augenblick. Auch für diejenigen, die es nicht wissen
               sollten, dass ich von Anfang an jedes Werk \textcolor{blue}{Hauptmann}{}\ledrightnote{\textcolor{blue}{Gerhart Hauptmann}}s mit Bewunderung aufgenommen habe. Auch für diejenigen, die es
               weder aus meinen Schriften noch aus meiner persönlichen Bekanntschaft zu wissen
               vermögen, dass ich mich niemals gefreut habe, wenn irgendwo einem arbeitenden Manne
               ein Werk misslang. Und dass solche Freude meinem ganzen Wesen fremd ist.\pend
           
\pstart
           Für jeden Unbefangenen sprechen es meine \textcolor{green}{\textcolor{blue}{Hauptmann}{}\ledrightnote{\textcolor{blue}{Gerhart Hauptmann}}-Kritiken}{}\ledrightnote{{$\rightarrow$}\textcolor{green}{Der Fall Hauptmann}{\newline}{$\rightarrow$}\textcolor{green}{Der Fall Brahm}} ohne alle
               Unterstimmen aus, dass ich die »\textcolor{green}{Jungfern vom
                  Bischofsberg}{}\ledrightnote{\textcolor{green}{Die Jungfern vom Bischofsberg. Lustspiel}}« für schlecht halte. Nur diese. Dass ich aus \textcolor{blue}{Hauptmann}{}\ledrightnote{\textcolor{blue}{Gerhart Hauptmann}}s Prosa und aus eben diesem letzten \textcolor{green}{Lustspiel}{}\ledrightnote{{$\rightarrow$}\textcolor{green}{Die Jungfern vom Bischofsberg. Lustspiel}} den Eindruck empfing, er ermangle
               der Selbstkritik und der Fähigkeit des artistischen Arbeitens. Dass ich für das
               Misslingen dieses \textcolor{green}{Lustspiel}{}\ledrightnote{{$\rightarrow$}\textcolor{green}{Die Jungfern vom Bischofsberg. Lustspiel}}es
               Ursachen suche, die mir ausserhalb von \textcolor{blue}{Hauptmann}{}\ledrightnote{\textcolor{blue}{Gerhart Hauptmann}}s Person zu liegen scheinen. Vor allem aber, dass ich über diesen
               Einzelfall hinaus an \textcolor{blue}{Hauptmann}{}\ledrightnote{\textcolor{blue}{Gerhart Hauptmann}}s dichterische
               Bedeutung glaube, und meine Leser auffordere, über diesen Einzelfall hinweg der
               Bedeutung des ganzen \textcolor{blue}{Mann}{}\ledrightnote{{$\rightarrow$}\textcolor{blue}{Gerhart Hauptmann}}es
               eingedenkt zu bleiben.\pend
           
\pstart
           Sie beschuldigen mich dagegen einer »schlecht verhehlten Freude«. Dass heisst, Sie
               zögern keinen Augenblick es auszu{\pb}sprechen, dass Sie eine
               niedrige Gesinnung bei mir annehmen. Darin liegt nicht nur eine Fälschung meiner
               Kritik; (denn Sie werden allen Leuten, die meine »Freude« nicht ausfinden können,
               lächelnd zu verstehen geben, dass \uline{Sie} eben ein
               feineres Gehör haben, als andere Menschen) darin liegt auch eine Treulosigkeit gegen
               unseren persönlichen Verkehr. Denn nur, wenn Sie sich alles dessen entschlagen, was
               Sie im Umgang mit mir an mir kennen gelernt haben, sind Sie imstande einen solchen
               Vorwurf gegen mich zu erheben. Darin liegt aber auch schon die Bereitschaft, diesen
               persönlichen Verkehr künftighin zur Bekräftigung Ihres Briefes umzufärben und zu
               verleumden.\pend
           
\pstart
           Viel deutlicher geht das Verhalten, zu dem Sie sich entschlossen haben, aus dem
               andern Vorwurf hervor, den Sie mir machen, aus der von Ihnen sorgfältig zugefeilten
               Formel vom »kaltherzig, berechnet leidenschaftlichen, politisierenden Journalismus«.
               Was Sie hier begehen, ist weit schlimmer. Gerade Sie kennen mich genug oder sind doch
               – was dasselbe bleibt – verpflichtet, mich hinlänglich zu kennen, um zu wissen, dass
               ich nicht kaltherzig bin und dass, wenn Leidenschaftlichkeit bei mir irgendwo zutage
               tritt, nicht die Spur einer Berechnung mit dabei ist. Gerade Sie wissen, warum ich
               als produktiver Mensch den Journalismus ausübe und wie ich ihn ausübe. Dass ich
               jemals politisierend meine Urteile gedrechselt hätte, ist aus meinem
                  Lebe{[}n{]} kein einziges Mal ersichtlich. Trotzdem werfen Sie mir
               diese Worte zu und vergreifen sich an mir, Sie – an mir, Sie, der den Journalismus
               mit solcher Mühe umwirbt – an mir, der ich von meinem Standpunkt aus mit Ihnen über
               Journalismus gar nicht zu reden brauchte; – Sie – an mir, der Sie sich damit
               begnügen, in gefahrlos verschwiegenen Zimmern ohne alle Verantwortung zu pre{\pb}digen und Klugreden zu halten, –
               an mir, der beständig seine Haut zum Markte trägt.\pend
           
\pstart
           Sie sprechen von einer politisierenden Absicht, und sagen dann: »Wenn irgendwer Ihnen
               geraten hat, diesen Artikel zu publizieren u. s. w.« Sie haben also die Ansicht, dass
               man – ehe man sein Urteil publiziert, – sich dazu raten oder davon abraten
                  lässt{[}.{]} Sie haben die Anschauung, dass man sich
               gemeinschaftlich darüber einigt, etwa gruppenweise oder durch Klüngelinteressen
               zusammengeführt, darüber berät, ob es »klug« oder »unklug« ist, diese oder jene
               Ansicht zu publizieren, kurz, dass man hier nach einer gewissen gemeinsam
               beschlossenen Taktik vorgeht.\pend
           
\pstart
           Ich habe von jeher meine Kritiken veröffentlicht, ohne sie vorher irgend einem
               Menschen zu zeigen, auch ohne zu bedenken, ob mir das, was ich sage, Freunde oder
               Feinde, Nutzen oder Schaden bringt, habe von jeher dieses Verfahren – wenn man nur
               seine aufrichtige Ueberzeugung sagt – für das einzig mögliche gehalten, und stehe nun
               voll Erstaunen vor einer Denkweise, die mir übrigens sehr viel Licht über Ihren
               ganzen Brief verbreitet.\pend
           
\pstart
           Sehen Sie, lieber Herr \textcolor{blue}{Heimann}{}\ledrightnote{\textcolor{blue}{Moritz Heimann}}, aus diesem
               Schluss Ihres Briefes, aus Ihren Worten, die Sie im Vollton bedauernder Wohlmeinung
               aussprechen, raucht mir etwas entgegen, was mir zuwider ist. Hier haben Sie sich ganz unwillkürlich etwas
               entschlüpfen lassen, und Ihr Brief wird dadurch auf einmal zu einem Dokument
               gestempelt.\pend
           
\pstart
           Sie durften sich’s – vielleicht – erlauben und von einer Unklugheit sprechen, wenn
               Sie es nämlich annehmen, dass es ein Ziel des Klugen sein muss, mit seiner Meinung
               einflussreichen Personen zu gefallen. Aber Sie durften nicht – auch nicht
               vermutungsweise – von einer Unredlichkeit sprechen. In meinem ganzen Leben, in mei{\pb}ner ganzen publizistischen
               Tätigkeit ist nichts vorhanden, was Ihnen ein Recht dazu gibt. Wenn Sie es trotzdem
               tun, dann ist es eben Ihre Gesinnung gegen mich, die nach einem schmähenden Ausdruck
               langt, die aber ihrem Schimpf gerne den Anschein einer höheren Gerechtigkeit geben
               möchte. Leider kann ich Ihnen solchen Luxus nicht gestatten. Und ich habe für das
               Wort Unredlichkeit nur die eine Entgegnung: Frechheit.\pend
           
\pstart
           Sie werfen mir vor, ich hätte die \textcolor{green}{Premiere}{}\ledrightnote{{$\rightarrow$}\textcolor{green}{Die Jungfern vom Bischofsberg. Lustspiel}} »beschrieben« ohne dabei gewesen zu sein. Diesem Vorwurf liesse
               sich selbst dann begegnen, wenn ich den Abend \uline{beschrieben} hätte. Ich habe jedoch aus Berichten, die übereinstimmend in
               allen Zeitungen zu lesen waren, wie nach absolut glaubwürdigen Privatnachrichten in
               knapp zehn Zeilen konstatiert, dass dieser Vorfall sich ereignet hat. Mehr nicht.
               Dieser laute und überall besprochene Vorfall bildete den äusserlichen Ausgangspunkt
               meines \textcolor{green}{Artikel}{}\ledrightnote{{$\rightarrow$}\textcolor{green}{Der Fall Hauptmann}}s. Deshalb
               musste dieser Vorfall auch am Anfange des \textcolor{green}{Artikel}{}\ledrightnote{{$\rightarrow$}\textcolor{green}{Der Fall Hauptmann}}s konstatierend erwähnt werden. Das ist eine Sache
               der Technik, von der ich allerdings glaube, dass Sie sie nicht verstehen. Ich bin
               aber gar nicht mehr im Zweifel darüber, dass Sie den Unterschied zwischen Beschreiben
               und Konstatieren diesmal absichtlich verwechseln. Und ich weiss, dass Sie \label{K_L03438-11v}\edtext{mala fide}{\lemma{\textnormal{\emph{mala fide}}}\Cendnote{\textnormal{lateinisch: bösen Glaubens}}}\label{K_L03438-11h}
               handeln, wenn Sie mir zumuten, (Sie mir) ich hätte nach einem \textcolor{pink}{Berlin}{}\ledrightnote{\textcolor{pink}{Berlin}}er Theaterskandal geschnappt, um ihn zum Gegenstand einer
               »Schilderung« zu machen!\pend
           
\pstart
           Damit allein aber geben Sie sich nicht zufrieden. Sie müssen noch sagen, ich hätte
               »schön« »anschaulich« beschrieben, müssen das Wort Poesie unter Anführungszeichen
               setzen und hoffen dabei, das werde mich treffen, weil es gegen Dinge in mir gerichtet
               ist, die mir am wertvollsten sind und von denen im Umkreis meiner Tagesarbeit
               sprechen zu lassen, mir empfindlich sein kann. {\pb}Hier brechen Sie mit Vorbedacht
               und mit Hohn in die Intimität meines Wesens ein, um mich desto sicherer zu verletzen.
               Dieser dreiste Griff an die geistigen Schamteile und Zeugungsorgane eines andern ist
               so widerwärtig, so durch nichts entschuldbar, dass ich i\textcolor{gray}{h}{[}n{]} hier nur feststelle und weiter nichts drauf sage.\pend
           
\pstart
           Ihr ganzer Brief ist lediglich eine Spekulation auf meine Gutmütigkeit. Hätten Sie
               mich nicht für so gutmütig gehalten, Sie hätten es nie versucht, mich mit dieser
               wohlfeilen Literaten-Psychologie zu dupieren.\pend
           
\pstart
           Sie haben irgend ein dumpfes Gefühl gegen mich, das ich bei Ihren Jahren und in Ihrem
               Zustande schliesslich begreife, und das ich bezeichnen könnte, wenn ich wollte. Die
               absolute Wahrheit meiner \textcolor{green}{\textcolor{blue}{Hauptmann}{}\ledrightnote{\textcolor{blue}{Gerhart Hauptmann}}-Kritiken}{}\ledrightnote{{$\rightarrow$}\textcolor{green}{Der Fall Hauptmann}{\newline}{$\rightarrow$}\textcolor{green}{Der Fall Brahm}} reizt gewisse
               Empfindlichkeiten und Instinkte in Ihnen, die ich gleichfalls bezeichnen könnte.\pend
           
\pstart
           Aber Sie schweigen. Trotzdem unser Umgang Ihnen jede Handhabe bietet, offen mit mir
               zu sein und (wenn Sie mich einmal sachlich im Unrecht glauben) sachlich und anständig
               zu mir zu kommen und mit mir zu reden {\dotstwo} trotzdem schweigen
               Sie gegen mich und »schütten anderen Ihr Herz aus«. Erst als ich davon höre und in
               einem erklärlichen Reinlichkeitsbedürfnis Sie gradeaus frage, – erst dann bequemen
               Sie sich zu einer direkten Aeusserung. Dabei jedoch wollen Sie vor mir verheimlichen,
               was in Ihnen vorgeht, möchten aber trotzdem als ein aufrichtiger und freimütiger Mann
               vor mir erscheinen.\pend
           
\pstart
           Und so schreiben Sie diesen Brief, der freimütig aussehen soll, geben sich als den
               Rechtschaffenen und Wackeren: Nicht etwa, dass Sie keine Kritik vertragen {\dotstwo} Gott bewahre! Bis auf eine Stelle (ich könnte diese
               Stelle nennen) hat 
               \textcolor{blue}{Kerr}{}\ledrightnote{\textcolor{blue}{Alfred Kerr}}{ }{\pb}Ihren »Beifall« gehabt. Nicht,
                  \uline{dass} ich etwas gegen \textcolor{blue}{Hauptmann}{}\ledrightnote{\textcolor{blue}{Gerhart Hauptmann}} zu sagen wagte, beanstanden Sie {\dotstwo} behüte! Sie debattieren nicht mit mir über \textcolor{blue}{Hauptmann}{}\ledrightnote{\textcolor{blue}{Gerhart Hauptmann}}. Sie machen es viel geschickter: Sie
               sprechen über \uline{mich}. Weil Sie gegen meine
               künstlerischen Argumente unfähig sind etwas vorzubringen, muss ich es sein, meine
               ganze Person, wogegen Sie sich wenden. Hier können Sie sich die Argumente sparen,
               (meinen Sie), und beweislos den Schreiber beschimpfen, da gegen das Geschriebene
               nicht gut anzukämpfen ist. Gelingt es nur, den menschlichen Wert des Kritikers zu
                  vernichten{[},{]} dann ist auch seine Kritik entwertet und kann aus der \textcolor{blue}{Hauptmann}{}\ledrightnote{\textcolor{blue}{Gerhart Hauptmann}}-Debatte ohne weiteres ausgeschaltet
               werden.\pend
           
\pstart
           Sie verfahren dabei wirklich sehr schlau, gebrauchen »feine« Worte und Wendungen,
               nehmen einen »höheren« Standpunkt ein, damit der meinige tiefer erscheine. Sie geben
               sich eine edle Haltung, indem Sie eine kerzengerade Sache auf eine pfäffische Weise
               verdrehen. Sie sind salbungsvoll, gerecht und fromm, damit \uline{Sie} Recht behalten und ich im Unrecht bleibe.\pend
           
\pstart
           Wenn einer von uns beiden der Politisierende gewesen ist{[},{]} dann
               sind Sie das, mein lieber Herr \textcolor{blue}{Heimann}{}\ledrightnote{\textcolor{blue}{Moritz Heimann}}! Und es
               wäre mir nicht schwer, jetzt die Offensive zu ergreifen, und Ihnen zu beweisen, Ihnen
               Punkt für Punkt nachzurechnen, dass Sie lange schon, immer und überall politisierende Kleinliteratur und
               literarische Politik betreiben und betrieben haben. Denn jetzt ist mir doch über
               viele Dinge, besonders aber über dieses unverantwortliche, behutsam rückversicherte
               Predigertum ein Licht aufgegangen.\pend
           
\pstart
           Sie haben die Sache mit mir sehr klug angefangen, aber es war doch recht töricht von
               Ihnen, gar so klug sein zu wollen. Sie haben mich für gutmütig gehalten und damit
               nicht schlecht ge{\pb}urteilt. Nur
               dass ich jetzt meine Gutmütigkeit doch ein wenig zu zügeln verstehe, was Sie freilich
               nicht voraus wissen konnten. Ihnen war nur bekannt, dass ich in meinem Leben schon
               oft von Gehässigkeit, verletzten Eitelkeiten und geschädigten Cliquen-Interessen
               wütend angefallen worden bin, und niemals so viel Ernst für derlei Dinge aufgebracht
               habe, um sie energisch abzuwehren. Jetzt aber bin ich zu der Ueberzeugung gelangt,
               dass es ein Unrecht war, mir von den Leuten, denen meine Kritik wider den Strich
               ging, Böswilligkeiten bieten zu lassen. Ich habe nachgerade genug von diesem Spiel
               und bin fest entschlossen, es nicht mehr zu dulden, wenn sich Literaten-Schmähsucht
               an mir vergreifen will, es nicht mehr zu dulden, wenn ein Einbruch in mein Wesen
               versucht wird{[}.{]} Sie sind jetzt der erste, den ich wieder einmal
               dabei abfasse.\pend
           
\pstart
           Ich lege den Akt \textcolor{blue}{Heimann}{}\ledrightnote{\textcolor{blue}{Moritz Heimann}} so wie er ist (meine
                  \textcolor{green}{Artikel}{}\ledrightnote{{$\rightarrow$}\textcolor{green}{Der Fall Hauptmann}{\newline}{$\rightarrow$}\textcolor{green}{Der Fall Brahm}}, Ihren
               Brief, meine Antwort) zur Feststellung des Sachverhaltes für künftige Geschehnisse
               und zur persönlichen Aufklärung für diesen jetzigen Vorfall in die Hände einiger mir
               wertvoller Menschen.\pend
           
\pstart
           Mit Ihnen selbst bin ich fertig, und schliesse meine Privatkorrespondenz mit Ihnen
               ein für allemal. Sollten Ihnen weitere Auseinandersetzungen mit mir erwünscht sein,
               dann verweise ich Sie vor die Oeffentlichkeit. Was Sie dort vorbringen, werde ich
               anhören, und Ihnen eben dort entgegnen. Die Bequemlichkeit der Hintertreppe und die
               Gefahrlosigkeit des Literaten-Schwatzes, kurz diesen ganzen Komfort, den sich
               Menschen in Ihrer Lage auf Kosten anderer so gerne gestatten, kann ich Ihnen zu
               meinem Bedauern nicht zubilligen.\pend
           \endnumbering\briefempfaengerindex{Schnitzler, Arthur@\textsc{Schnitzler, Arthur}!zzzSalten, Felix@\emph{von Felix Salten}!1907-04-201@{20. 4. 1907}|)be}\mylabel{h}  \normalsize

\doendnotes{C}
\bigskip
\vfill

\clearpage

\footnotesize

\lohead{\textsc{register}}

% Definiere theindex-Environment komplett neu ohne reledmac
\makeatletter
\renewenvironment{theindex}{%
  \section*{\indexname}%
  \setlength{\parindent}{0pt}%
  \setlength{\parskip}{0pt plus 0.3pt}%
  \let\item\@idxitem
}{%
  \clearpage
}
\makeatother

\IfFileExists{\jobname-pw.ind}{\input{\jobname-pw.ind}}{}

\end{document}

      