%% latex-korrekturansicht-vorspann.tex
%% Vorspann für die Korrekturansicht.
%% Lädt die gemeinsame Datei latex-vorspann.tex mit gesetztem Schalter.

\newif\ifkorrekturansicht
\korrekturansichttrue

\input{../tex-inputs/latex-vorspann}


\renewcommand{\erwaehntePersonen}{Personen: Frieda Pollak, Felix Salten, Ottilie Salten}
\renewcommand{\erwaehnteInstitutionen}{Institutionen: Volkstheater}
\renewcommand{\erwaehnteOrte}{Orte: Cottagegasse, Sternwartestraße 71, Wien, XVIII., Währing}
\renewcommand{\erwaehnteWerke}{Werke: Kinder der Freude. Drei Einakter}
\section[ Felix Salten an Arthur Schnitzler, 27. 12. 1917]{Felix Salten an Arthur Schnitzler, 27. 12. 1917}
\nopagebreak\mylabel{v}
\rehead{ }\normalsize\beginnumbering\briefempfaengerindex{Schnitzler, Arthur@\textsc{Schnitzler, Arthur}!zzzSalten, Felix@\emph{von Felix Salten}!1917-12-271@{27. 12. 1917}|(be}
\toendnotes[C]{\smallbreak\pagebreak[2]}\Standort{CUL, Schnitzler, B 89, B 2.}
\physDesc{Postkarte, 342 Zeichen
\newline{}Handschrift: schwarze Tinte, lateinische Kurrent
\newline{}Versand: Stempel: »\nobreak{}\oindex{XVIII., Waehring@\textbf{XVIII., Währing}, \emph{A.ADM3}|pwk}18/\textsubscript{1} Wien 110, 27. XII. 17, 4\textsuperscript{\textcolor{gray}{2}0}\nobreak{}«.  
\newline{}Ordnung: 1) mit Bleistift von \textcolor{blue}{Frieda Pollak} (?) mit
                                 dem Buchstaben »A« (Abgeschrieben/Abschrift)
                                 gekennzeichnet  2) mit Bleistift von unbekannter Hand nummeriert: »280«}\toendnotes[C]{\smallbreak}\pstart{}{\pb}\textcolor{gray}{\textbf{\textit{FELIX SALTEN}}}\pend{}\pstart{}\textcolor{pink}{\textcolor{gray}{\textbf{\textit{WIEN, XVIII.}}}}{}\ledrightnote{\textcolor{pink}{XVIII., Währing}}\pend{}\pstart{}\textcolor{pink}{\textcolor{gray}{\textbf{\textit{COTTAGEGASSE 37}}}}{}\ledrightnote{\textcolor{pink}{Cottagegasse}}\pend{}
{\bigskip}\pstart{}Herrn\pend{}\pstart{}D\textsuperscript{r} Arthur Schnitzler\pend{}\pstart{}\textcolor{pink}{Wien}{}\ledrightnote{\textcolor{pink}{Wien}}\pend{}\pstart{}\textcolor{pink}{XVIII. Sternwartestrasse 71}{}\ledrightnote{\textcolor{pink}{Sternwartestraße 71}}\pend{}
{\bigskip}
\pstart
           \raggedleft{}{\pb}27. XII. 17\pend
           
\pstart{}Lieber Arthur,\pend
\pstart
           gestern{ }Vormittag war ich bei Ihnen, habe Sie aber nicht zu Hause getroffen; so
               muss ich Ihnen nun auf diesem Weg für Ihre \label{K_L03567-1v}\edtext{freundlichen Zeilen}{\lemma{\textnormal{\emph{freundlichen Zeilen}}}\Cendnote{\textnormal{Am 22. 12. 1917 hatten \textcolor{blue}{Salten}s drei Einakter \emph{\textcolor{green}{Kinder der Freude}} die Uraufführung am \emph{\textcolor{brown}{Deutschen Volkstheater}}. Die Regie verantwortete ebenfalls \textcolor{blue}{Salten}. \textcolor{blue}{Schnitzler} las den Text am 12. 11. 1917 und fand ihn
                  furchtbar. Er besuchte nicht die Premiere, zu der er \textcolor{blue}{Salten} trotzdem gratuliert haben dürfte, sondern die
                  Aufführung am 18. 1. 1918.}}}\label{K_L03567-1h} danken. Ich hätte es gern mündlich getan.\pend
           
\pstart
           Viele Grüße von \textcolor{blue}{uns}{}\ledrightnote{{$\rightarrow$}\textcolor{blue}{Ottilie Salten}} zu
               Ihnen. {\\[\baselineskip]}Ihr {\\[\baselineskip]}\spacefill\mbox{Felix Salten}\pend
           \leftskip=0em{}\endnumbering\briefempfaengerindex{Schnitzler, Arthur@\textsc{Schnitzler, Arthur}!zzzSalten, Felix@\emph{von Felix Salten}!1917-12-271@{27. 12. 1917}|)be}\mylabel{h}  \normalsize

\doendnotes{C}
\bigskip
\vfill

\clearpage

\footnotesize

\lohead{\textsc{register}}

% Definiere theindex-Environment komplett neu ohne reledmac
\makeatletter
\renewenvironment{theindex}{%
  \section*{\indexname}%
  \setlength{\parindent}{0pt}%
  \setlength{\parskip}{0pt plus 0.3pt}%
  \let\item\@idxitem
}{%
  \clearpage
}
\makeatother

\IfFileExists{\jobname-pw.ind}{\input{\jobname-pw.ind}}{}

\end{document}

      