%% latex-korrekturansicht-vorspann.tex
%% Vorspann für die Korrekturansicht.
%% Lädt die gemeinsame Datei latex-vorspann.tex mit gesetztem Schalter.

\newif\ifkorrekturansicht
\korrekturansichttrue

\input{../tex-inputs/latex-vorspann}


\renewcommand{\erwaehntePersonen}{Personen: Frieda Pollak, Felix Salten}
\renewcommand{\erwaehnteOrte}{Orte: Bad Aussee, Berghof, Unterach am Attersee, Weißenbach am Attersee, Wien}
\renewcommand{\erwaehnteWerke}{}
\section[ Felix Salten an Arthur Schnitzler, 19. 7. 1921]{Felix Salten an Arthur Schnitzler, 19. 7. 1921}
\nopagebreak\mylabel{v}
\rehead{ }\normalsize\beginnumbering\briefempfaengerindex{Schnitzler, Arthur@\textsc{Schnitzler, Arthur}!zzzSalten, Felix@\emph{von Felix Salten}!1921-07-191@{19. 7. 1921}|(be}
\toendnotes[C]{\smallbreak\pagebreak[2]}\Standort{CUL, Schnitzler, B 89, B 2.}
\physDesc{Briefkarte, 231 Zeichen
\newline{}Handschrift: schwarze Tinte, lateinische Kurrent
\newline{}Ordnung: 1) mit Bleistift von \textcolor{blue}{Frieda Pollak} (?) mit
                                 dem Buchstaben »A« (Abgeschrieben/Abschrift)
                                 gekennzeichnet  2) mit Bleistift von unbekannter Hand nummeriert: »285«}\toendnotes[C]{\smallbreak}
\pstart
           \raggedleft{}{\pb}\textcolor{pink}{Berghof}{}\ledrightnote{\textcolor{pink}{Berghof}}, 19. 7. 21\pend
           
\pstart{}Lieber,\pend
\pstart
           wie geht es Ihnen und \label{K_L03572-1v}\edtext{wo sind Sie}{\lemma{\textnormal{\emph{wo sind Sie}}}\Cendnote{\textnormal{\textcolor{blue}{Schnitzler} war in \textcolor{pink}{Wien}. Nach \textcolor{pink}{Weißenbach}
                  reiste er nicht, jedoch in das nicht weit entfernte \textcolor{pink}{Bad Aussee}.}}}\label{K_L03572-1h}? Ich wüßte gerne Beides von Ihnen. Auch, ob die
               Möglichkeit, von der ja die Rede war, dass Sie nach \textcolor{pink}{Weissenbach}{}\ledrightnote{\textcolor{pink}{Weißenbach am Attersee}} oder sonst in die Nähe kommen, noch besteht.\pend
           
\pstart
           Herzlichst Ihr {\\[\baselineskip]}\spacefill\mbox{Salten}\pend
           \leftskip=0em{}\endnumbering\briefempfaengerindex{Schnitzler, Arthur@\textsc{Schnitzler, Arthur}!zzzSalten, Felix@\emph{von Felix Salten}!1921-07-191@{19. 7. 1921}|)be}\mylabel{h}  \normalsize

\doendnotes{C}
\bigskip
\vfill

\clearpage

\footnotesize

\lohead{\textsc{register}}

% Definiere theindex-Environment komplett neu ohne reledmac
\makeatletter
\renewenvironment{theindex}{%
  \section*{\indexname}%
  \setlength{\parindent}{0pt}%
  \setlength{\parskip}{0pt plus 0.3pt}%
  \let\item\@idxitem
}{%
  \clearpage
}
\makeatother

\IfFileExists{\jobname-pw.ind}{\input{\jobname-pw.ind}}{}

\end{document}

      