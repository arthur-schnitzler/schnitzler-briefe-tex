%% latex-korrekturansicht-vorspann.tex
%% Vorspann für die Korrekturansicht.
%% Lädt die gemeinsame Datei latex-vorspann.tex mit gesetztem Schalter.

\newif\ifkorrekturansicht
\korrekturansichttrue

\input{../tex-inputs/latex-vorspann}


\renewcommand{\erwaehntePersonen}{Personen:  ?? [Kusine von Felix Salten]}
\renewcommand{\erwaehnteInstitutionen}{Institutionen: Die Zeit}
\renewcommand{\erwaehnteOrte}{Orte: Volkstheater, Wien}
\renewcommand{\erwaehnteWerke}{Werke: Das vierte Gebot. Volksstück in vier Acten, Empfängnis}
\section[ Felix Salten an Arthur Schnitzler, {[}24. 3. 1902{]}]{Felix Salten an Arthur Schnitzler, {[}24. 3. 1902{]}}
\nopagebreak\mylabel{v}
\rehead{ }\normalsize\beginnumbering\briefempfaengerindex{Schnitzler, Arthur@\textsc{Schnitzler, Arthur}!zzzSalten, Felix@\emph{von Felix Salten}!1902-03-241@{{[}24. 3. 1902{]}}|(be}
\toendnotes[C]{\smallbreak\pagebreak[2]}\Standort{CUL, Schnitzler, B 89, A 2.}
\physDesc{Karte, 252 Zeichen
\newline{}Handschrift: Bleistift, lateinische Kurrent
\newline{}Schnitzler: mit Bleistift datiert: »24/3 {[}1{]}902.« 
\newline{}Ordnung: mit Bleistift von unbekannter Hand nummeriert: »151« }\toendnotes[C]{\smallbreak}
\pstart
           \noindent{}{\pb}Lieber, hier der \label{K_L03327-1v}\edtext{Sitz
               zum »\textcolor{green}{IV. Gebot}{}\ledrightnote{\textcolor{green}{Das vierte Gebot. Volksstück in vier Acten}}«}{\lemma{\textnormal{\emph{Sitz
               zum »IV. Gebot«}}}\Cendnote{\textnormal{im \textcolor{pink}{Volkstheater}}}}\label{K_L03327-1h} – ich werde wol spät kommen, weil ich bei der »\textcolor{brown}{Zeit}{}\ledrightnote{\textcolor{brown}{Die Zeit}}« bin.\pend
           
\pstart
           Die »\textcolor{green}{Empfängnis}{}\ledrightnote{\textcolor{green}{Empfängnis}}« bring ich zum \label{K_L03327-2v}\edtext{Vorlesen}{\lemma{\textnormal{\emph{Vorlesen}}}\Cendnote{\textnormal{siehe A. S.: \emph{Tagebuch}, 24. 3. 1902}}}\label{K_L03327-2h} nachher mit.\pend
           
\pstart
           Entschuldigen Sie das »\label{K_L03327-3v}\edtext{Rosa-Brieferl}{\lemma{\textnormal{\emph{Rosa-Brieferl}}}\Cendnote{\textnormal{Bezug auf die
                  Papierfarbe der Karte}}}\label{K_L03327-3h}\textcolor{gray}{«,} aber meine \label{K_L03327-4v}\edtext{\textcolor{blue}{Cousine}{}\ledrightnote{{$\rightarrow$}\textcolor{blue}{?? [Kusine von Felix Salten]}}}{\lemma{\textnormal{\emph{Cousine}}}\Cendnote{\textnormal{\textcolor{blue}{Salten} hatte nur Cousinen väterlicherseits.
                  Welche genau gemeint war, kann nicht mit Bestimmtheit gesagt werden.}}}\label{K_L03327-4h}, bei
               der ich schreibe, ist so poetisch\pend
           
\pstart
           Herzlichst {\\[\baselineskip]}\spacefill\mbox{Salten}\pend
           \leftskip=0em{}\endnumbering\briefempfaengerindex{Schnitzler, Arthur@\textsc{Schnitzler, Arthur}!zzzSalten, Felix@\emph{von Felix Salten}!1902-03-241@{{[}24. 3. 1902{]}}|)be}\mylabel{h}  \normalsize

\doendnotes{C}
\bigskip
\vfill

\clearpage

\footnotesize

\lohead{\textsc{register}}

% Definiere theindex-Environment komplett neu ohne reledmac
\makeatletter
\renewenvironment{theindex}{%
  \section*{\indexname}%
  \setlength{\parindent}{0pt}%
  \setlength{\parskip}{0pt plus 0.3pt}%
  \let\item\@idxitem
}{%
  \clearpage
}
\makeatother

\IfFileExists{\jobname-pw.ind}{\input{\jobname-pw.ind}}{}

\end{document}

      