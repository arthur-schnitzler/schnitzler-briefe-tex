%% latex-korrekturansicht-vorspann.tex
%% Vorspann für die Korrekturansicht.
%% Lädt die gemeinsame Datei latex-vorspann.tex mit gesetztem Schalter.

\newif\ifkorrekturansicht
\korrekturansichttrue

\input{../tex-inputs/latex-vorspann}


\section[Arthur Schnitzler an Stefan Zweig, 9. 11. 1911]{L03789 Arthur Schnitzler an Stefan Zweig, 9. 11. 1911}
\nopagebreak\mylabel{L03789v}
\rehead{ }\normalsize\beginnumbering\briefempfaengerindex{Zweig, Stefan@\textsc{Zweig, Stefan}!zzzSchnitzler, Arthur@\emph{von Arthur Schnitzler}!1911-11-091@{9. 11. 1911}|(be}
\toendnotes[C]{\smallbreak\pagebreak[2]}\Standort{Jerusalem, National Library of Israel, ARC. Ms. Var. 305 1 58 Stefan Zweig Collection.}
\physDesc{Bildpostkarte, 1 Blatt, 2 Seiten, 457 Zeichen
\newline{}Handschrift: schwarze Tinte, lateinische Kurrent
\newline{}Versand: Stempel: »\nobreak{}\oindex{Hamburg@\textbf{Hamburg}|pwk}Hamburg, 9. 11. 11, 10–11M\nobreak{}«.  }\toendnotes[C]{\smallbreak}\pstart{}{\pb}Herrn Dr. Stephan Zweig\pend{}\pstart{}\textcolor{pink}{Wien VIII}\oindex{VIII., Josefstadt@\textbf{VIII., Josefstadt}|pw}{}\ledrightnote{\textcolor{pink}{VIII., Josefstadt}}\pend{}\pstart{}\textcolor{pink}{Kochgasse 8}\oindex{Kochgasse 8@\textbf{Kochgasse 8}|pw}{}\ledrightnote{\textcolor{pink}{Kochgasse 8}}\pend{}{\bigskip}
\pstart
           {\pb}\textcolor{gray}{\textbf{\textcolor{pink}{Hamburg, Hotel Atlantic}\oindex{Hotel Atlantic@\textbf{Hotel Atlantic}|pw}{}\ledrightnote{\textcolor{pink}{Hotel Atlantic}}}}\hfill \textcolor{pink}{\textcolor{gray}{\textbf{Restaurant Pfordte}}}\oindex{Restaurant Pfordte im Hotel Atlantic@\textbf{Restaurant Pfordte im Hotel Atlantic}|pw}{}\ledrightnote{\textcolor{pink}{Restaurant Pfordte im Hotel Atlantic}}\pend
           \vspace{1em}
\pstart
           \centering{}{\pb}\textcolor{pink}{Hamburg}\oindex{Hamburg@\textbf{Hamburg}|pw}{}\ledrightnote{\textcolor{pink}{Hamburg}}, 9. 11. 911\pend
           \vspace{0.5em}
\pstart
           lieber Herr Doctor, außer Herrn \textcolor{blue}{Morisse}\pwindex{Morisse, Paul 1866-03-11 – 1946-09-28@\textsc{Morisse, Paul} (1866-03-11 – 1946-09-28), \emph{Übersetzer/Übersetzerin}|pw}{}\ledrightnote{\textcolor{blue}{Paul Morisse}} haben sich noch andre Übersetzer gemeldet, darunter \textcolor{blue}{Rémon}\pwindex{Remon, Maurice 27.11.1861 – 20.06.1945@\textsc{Rémon, Maurice} (27.11.1861 – 20.06.1945), \emph{Übersetzer/Übersetzerin}|pw}{}\ledrightnote{\textcolor{blue}{Maurice Rémon}}, der schon \textcolor{green}{einiges}\pwindex{femme au poignard@\emph{La femme au poignard}|pwv}\pwindex{Morts se taisent@\emph{Les Morts se taisent}|pwv}\pwindex{Jour de gloire@\emph{Jour de gloire}|pwv}{}\ledrightnote{{$\rightarrow$}\emph{\textcolor{green}{La femme au poignard}}{\newline}{$\rightarrow$}\emph{\textcolor{green}{Les Morts se taisent}}{\newline}{$\rightarrow$}\emph{\textcolor{green}{Jour de gloire}}} von mir übersetzt
               und auch zur Veröffentlichg gebracht hat. Entschieden hab ich mich noch nicht. Hat
               \textcolor{blue}{Morisse}\pwindex{Morisse, Paul 1866-03-11 – 1946-09-28@\textsc{Morisse, Paul} (1866-03-11 – 1946-09-28), \emph{Übersetzer/Übersetzerin}|pw}{}\ledrightnote{\textcolor{blue}{Paul Morisse}} schon \uline{Stücke} übersetzt? Vielen Dank für Ihr
               freundliches Interesse. Gratuliere zu den schönen Annahmen. Lassen Sie sich’s in \textcolor{pink}{Meran}\oindex{Meran@\textbf{Meran}|pw}{}\ledrightnote{\textcolor{pink}{Meran}} wohl ergehen. Herzliche Grüße. Ihr\pend
           \pstart \spacefill\mbox{Arthur Schnitzler}\pend{}\selectlanguage{ngerman}\endnumbering\briefempfaengerindex{Zweig, Stefan@\textsc{Zweig, Stefan}!zzzSchnitzler, Arthur@\emph{von Arthur Schnitzler}!1911-11-091@{9. 11. 1911}|)be}\mylabel{L03789h}  \normalsize

\doendnotes{C}
\bigskip
\vfill

\clearpage

\footnotesize

\lohead{\textsc{register}}

% Definiere theindex-Environment komplett neu ohne reledmac
\makeatletter
\renewenvironment{theindex}{%
  \section*{\indexname}%
  \setlength{\parindent}{0pt}%
  \setlength{\parskip}{0pt plus 0.3pt}%
  \let\item\@idxitem
}{%
  \clearpage
}
\makeatother

\IfFileExists{\jobname-pw.ind}{\input{\jobname-pw.ind}}{}

\end{document}

      