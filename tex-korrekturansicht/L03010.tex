%% latex-korrekturansicht-vorspann.tex
%% Vorspann für die Korrekturansicht.
%% Lädt die gemeinsame Datei latex-vorspann.tex mit gesetztem Schalter.

\newif\ifkorrekturansicht
\korrekturansichttrue

\input{../tex-inputs/latex-vorspann}


\renewcommand{\erwaehntePersonen}{Personen: Richard Beer-Hofmann, Josef Kainz, Margarethe Kainz, Felix Salten, Felix Speidel, Else Speidel-Haeberle}
\renewcommand{\erwaehnteOrte}{Orte: Meissl {\kaufmannsund}  Schadn, Volkstheater, Wien}
\renewcommand{\erwaehnteWerke}{Werke: Die Verschwörung des Fiesko zu Genua, Ein Walzertraum. Operette in drei Akten, Vom andern Ufer. Einakter}
\section[ Arthur Schnitzler an Felix Salten, 7. 11. 1907]{Arthur Schnitzler an Felix Salten, 7. 11. 1907}
\nopagebreak\mylabel{v}
\rehead{ }\normalsize\beginnumbering\briefempfaengerindex{Salten, Felix@\textsc{Salten, Felix}!zzzSchnitzler, Arthur@\emph{von Arthur Schnitzler}!1907-11-071@{7. 11. 1907}|(be}
\toendnotes[C]{\smallbreak\pagebreak[2]}\Standort{Wienbibliothek im Rathaus, ZPH 1681, 2.1.516.}
\physDesc{Brief, 1 Blatt, 1 Seite, 390 Zeichen
\newline{}Handschrift: schwarze Tinte, lateinische Kurrent
\newline{}Ordnung: mit Bleistift von unbekannter Hand nummeriert: »7« }\toendnotes[C]{\smallbreak}
\pstart
           \raggedleft{}{\pb}7. 11. 907\pend
           
\pstart{}lieber,\pend
\pstart
           \textcolor{blue}{Kainz}{}\ledrightnote{\textcolor{blue}{Josef Kainz}} spielt am Samstag den \textcolor{green}{Fiesco}{}\ledrightnote{\textcolor{green}{Die Verschwörung des Fiesko zu Genua}}, Frau \textcolor{blue}{Kainz}{}\ledrightnote{\textcolor{blue}{Margarethe Kainz}} ist bei Ihrer \label{K_L03010-1v}\edtext{\textcolor{green}{Première}{}\ledrightnote{{$\rightarrow$}\textcolor{green}{Vom andern Ufer. Einakter}}}{\lemma{\textnormal{\emph{Première}}}\Cendnote{\textnormal{Die Premiere von \emph{\textcolor{green}{Vom
                  andern Ufer}} fand 9. 11. 1907 am \textcolor{pink}{Volkstheater}
                  statt. \textcolor{blue}{Schnitzler} nahm teil.}}}\label{K_L03010-1h}, geht aber dann zu \textcolor{green}{Fiesco}{}\ledrightnote{\textcolor{green}{Die Verschwörung des Fiesko zu Genua}} hinüber, so
               daß sie wohl beide \introOben{}nachher\introOben{} nicht mit mir sein werden. \textcolor{blue}{Richard}{}\ledrightnote{\textcolor{blue}{Richard Beer-Hofmann}} sagte mir gestern, er wollte zur \uline{zweiten} Vorstellung
               gehen. \textcolor{blue}{Speidels}{}\ledrightnote{\textcolor{blue}{Felix Speidel}{\newline}\textcolor{blue}{Else Speidel-Haeberle}}
                  sin\textcolor{gray}{d} wohl im \textcolor{pink}{Theater}{}\ledrightnote{{$\rightarrow$}\textcolor{pink}{Volkstheater}}. Ich würde vorschlagen: \textcolor{pink}{Meissl {\kaufmannsund} Schadn}{}\ledrightnote{\textcolor{pink}{Meissl {\kaufmannsund} Schadn}} wie \label{K_L03010-2v}\edtext{neulich nach dem \textcolor{green}{Walzertraum}{}\ledrightnote{\textcolor{green}{Ein Walzertraum. Operette in drei Akten}}}{\lemma{\textnormal{\emph{neulich … Walzertraum}}}\Cendnote{\textnormal{siehe A. S.: \emph{Tagebuch}, 23. 10. 1907}}}\label{K_L03010-2h}. Sie vergessen nicht mir die Loge zu schicken?\pend
           
\pstart
           herzlichst Ihr {\\[\baselineskip]}\spacefill\mbox{Arthur}\pend
           \leftskip=0em{}\endnumbering\briefempfaengerindex{Salten, Felix@\textsc{Salten, Felix}!zzzSchnitzler, Arthur@\emph{von Arthur Schnitzler}!1907-11-071@{7. 11. 1907}|)be}\mylabel{h}  \normalsize

\doendnotes{C}
\bigskip
\vfill

\clearpage

\footnotesize

\lohead{\textsc{register}}

% Definiere theindex-Environment komplett neu ohne reledmac
\makeatletter
\renewenvironment{theindex}{%
  \section*{\indexname}%
  \setlength{\parindent}{0pt}%
  \setlength{\parskip}{0pt plus 0.3pt}%
  \let\item\@idxitem
}{%
  \clearpage
}
\makeatother

\IfFileExists{\jobname-pw.ind}{\input{\jobname-pw.ind}}{}

\end{document}

      