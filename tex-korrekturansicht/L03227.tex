%% latex-korrekturansicht-vorspann.tex
%% Vorspann für die Korrekturansicht.
%% Lädt die gemeinsame Datei latex-vorspann.tex mit gesetztem Schalter.

\newif\ifkorrekturansicht
\korrekturansichttrue

\input{../tex-inputs/latex-vorspann}


\renewcommand{\erwaehntePersonen}{Personen: Moritz Coschell, Auguste Glümer, Marie Glümer}
\renewcommand{\erwaehnteOrte}{Orte: Berlin, Café Grunewald, Dessauer Straße, Grunewald, Ivano-Frankivsk}
\renewcommand{\erwaehnteWerke}{Werke: [Gemälde von Moritz Coschell]}
\section[ Paul Goldmann an Arthur Schnitzler, 14. 10. {[}1902{]}]{Paul Goldmann an Arthur Schnitzler, 14. 10. {[}1902{]}}
\nopagebreak\mylabel{v}
\rehead{ }\normalsize\beginnumbering\briefempfaengerindex{Schnitzler, Arthur@\textsc{Schnitzler, Arthur}!zzzGoldmann, Paul@\emph{von Paul Goldmann}!1902-10-141@{14. 10. {[}1902{]}}|(be}
\toendnotes[C]{\smallbreak\pagebreak[2]}\Standort{DLA, A:Schnitzler, HS.NZ85.1.3172.}
\physDesc{Brief, 1 Blatt, 2 Seiten
\newline{}Handschrift: blaue Tinte, deutsche Kurrent
\newline{}Schnitzler: 1) mit Bleistift das Jahr »{[}1{]}902« vermerkt  2) mit rotem Buntstift drei Unterstreichungen}\toendnotes[C]{\smallbreak}
\pstart
           \noindent{}\raggedleft{}{\pb}\textcolor{pink}{\textcolor{gray}{\textbf{DESSAUERSTRASSE 19}}}{}\ledrightnote{\textcolor{pink}{Dessauer Straße}}\pend
           
\pstart
           \textcolor{pink}{Berlin}{}\ledrightnote{\textcolor{pink}{Berlin}}, 14. Okt.\pend
           
\pstart\center{}Mein lieber Freund,\pend
\pstart
           \textsc{\textcolor{blue}{Coschell}{}\ledrightnote{\textcolor{blue}{Moritz Coschell}}} iſt gar nicht in \textcolor{pink}{Berlin}{}\ledrightnote{\textcolor{pink}{Berlin}}. Er macht Studien
               zu ſeinem \label{K_L03227-1v}\edtext{jüdiſchen \textcolor{green}{Gemälde}{}\ledrightnote{{$\rightarrow$}\textcolor{green}{[Gemälde von Moritz Coschell]}}}{\lemma{\textnormal{\emph{jüdiſchen Gemälde}}}\Cendnote{\textnormal{nicht ermittelt}}}\label{K_L03227-1h} in \textsc{\textcolor{pink}{Stanislau}{}\ledrightnote{\textcolor{pink}{Ivano-Frankivsk}}}.\pend
           
\pstart
           \label{K_L03227-2v}\edtext{\textsc{\textcolor{blue}{Gusti}{}\ledrightnote{\textcolor{blue}{Auguste Glümer}}}}{\lemma{\textnormal{\emph{Gusti}}}\Cendnote{\textnormal{Bezug unklar; \textcolor{blue}{Schnitzler} traf \textcolor{blue}{Auguste
                     Glümer} jedenfalls gleich am nächsten Tag (vgl. A. S.: \emph{Tagebuch}, 15. 10. 1902)}}}\label{K_L03227-2h} wird ſich mit Dir in Verbindung
               setzen.\pend
           
\pstart
           \textsc{\textcolor{blue}{Mizzi}{}\ledrightnote{\textcolor{blue}{Marie Glümer}}} iſt krank. Sie {\pb}hat ihre alten Kopfſchmerzen
               u. wohnt im \textsc{\textcolor{pink}{Grunewald}{}\ledrightnote{\textcolor{pink}{Grunewald}}}, \textsc{\textcolor{pink}{Café Grunewald}{}\ledrightnote{\textcolor{pink}{Café Grunewald}}}.\pend
           
\pstart
           Auf Mittwoch{ }Abend, 7 Uhr!\pend
           
\pstart
           Herzlichſt {\\[\baselineskip]}Dein {\\[\baselineskip]}\spacefill\mbox{Paul Goldmn}\pend
           \leftskip=0em{}\endnumbering\briefempfaengerindex{Schnitzler, Arthur@\textsc{Schnitzler, Arthur}!zzzGoldmann, Paul@\emph{von Paul Goldmann}!1902-10-141@{14. 10. {[}1902{]}}|)be}\mylabel{h}
\begin{anhang}
\end{anhang}\normalsize

\doendnotes{C}
\bigskip
\vfill

\clearpage

\footnotesize

\lohead{\textsc{register}}

% Definiere theindex-Environment komplett neu ohne reledmac
\makeatletter
\renewenvironment{theindex}{%
  \section*{\indexname}%
  \setlength{\parindent}{0pt}%
  \setlength{\parskip}{0pt plus 0.3pt}%
  \let\item\@idxitem
}{%
  \clearpage
}
\makeatother

\IfFileExists{\jobname-pw.ind}{\input{\jobname-pw.ind}}{}

\end{document}

      