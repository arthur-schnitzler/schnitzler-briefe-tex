%% latex-korrekturansicht-vorspann.tex
%% Vorspann für die Korrekturansicht.
%% Lädt die gemeinsame Datei latex-vorspann.tex mit gesetztem Schalter.

\newif\ifkorrekturansicht
\korrekturansichttrue

\input{../tex-inputs/latex-vorspann}


\section[Stefan Zweig an Arthur Schnitzler, 20. 3. 1913]{L03644 Stefan Zweig an Arthur Schnitzler, 20. 3. 1913}
\nopagebreak\mylabel{L03644v}
\rehead{ }\normalsize\beginnumbering\briefempfaengerindex{Schnitzler, Arthur@\textsc{Schnitzler, Arthur}!zzzZweig, Stefan@\emph{von Stefan Zweig}!1913-03-201@{20. 3. 1913}|(be}
\toendnotes[C]{\smallbreak\pagebreak[2]}\Standort{CUL, Schnitzler, B 118.}
\physDesc{Bildpostkarte, 309 Zeichen
\newline{}Handschrift: schwarze Tinte, lateinische Kurrent
\newline{}Versand: Stempel: »\nobreak{}\oindex{Paris@\textbf{Paris}, \emph{P.PPLC}|pwk}Paris III \textcolor{gray}{Ste} Anne, 20. 3. \textcolor{gray}{1}3, \textcolor{gray}{×} 30\nobreak{}«.  }
\buchAbdrucke{\weitereDrucke{Stefan Zweig: \emph{Briefwechsel mit Hermann Bahr, Sigmund Freud, Rainer Maria
                        Rilke und Arthur Schnitzler}. Frankfurt am Main: \emph{S. Fischer} 1987, S. 375.} }\toendnotes[C]{\smallbreak}\pstart{}{\pb}D\textsuperscript{r}
                     Arthur Schnitzler\pend{}\pstart{}\textcolor{pink}{Wien (Autriche)}\oindex{Wien@\textbf{Wien}, \emph{A.ADM2}|pw}{}\ledrightnote{\textcolor{pink}{Wien}}\pend{}\pstart{}\textcolor{pink}{\label{K_L03644-1v}\edtext{Sternwartestrasse 72}{\lemma{\textnormal{\emph{Sternwartestrasse 72}}}\Cendnote{\textnormal{\textcolor{blue}{Zweig}\pwindex{Zweig, Stefan 28.11.1881 – 23.02.1942@\textsc{Zweig, Stefan} (28.11.1881 – 23.02.1942), \emph{Schriftsteller/Schriftstellerin}|pwk} wechselt
                        bei der Adressierung seiner Schreiben an \textcolor{blue}{Schnitzler} immer wieder zwischen der falschen Hausnummer
                        »72« und der richtigen
                        »71«.}}}\label{K_L03644-1}}\oindex{Sternwartestrasse 71@\textbf{Sternwartestraße 71}, \emph{Wohngebäude (K.WHS)}|pw}{}\ledrightnote{\textcolor{pink}{Sternwartestraße 71}}\pend{}{\bigskip}
\pstart
           \noindent{}\centering{}{\pb}\textcolor{gray}{\textbf{16 – \textcolor{pink}{PARIS}\oindex{Paris@\textbf{Paris}, \emph{P.PPLC}|pw}{}\ledrightnote{\textcolor{pink}{Paris}}. \textcolor{pink}{La Colonne de Juillet}\oindex{colonne de Juillet@\textbf{colonne de Juillet}, \emph{Denkmal}|pw}{}\ledrightnote{\textcolor{pink}{colonne de Juillet}},}}\pend
           
\pstart
           \centering{}\textcolor{gray}{\textbf{\textcolor{pink}{Place de la Bastille}\oindex{Place de la Bastille@\textbf{Place de la Bastille}, \emph{S.SQR}|pw}{}\ledrightnote{\textcolor{pink}{Place de la Bastille}}.{ }3 ND Phot.}}\pend
           \vspace{1em}
\pstart
           {\pb}\textcolor{pink}{15, rue Beaujolais}\oindex{Hôtel de Beaujolais@\textbf{Hôtel de Beaujolais}, \emph{Hotel (K.HTL)}|pw}{}\ledrightnote{\textcolor{pink}{Hôtel de Beaujolais}}\pend
           
\pstart
           Hotel\pend
           \vspace{0.5em}
\pstart
           Verehrter Herr Doktor, ich freue
               mich von Herzen, dass \label{K_L03644-2v}\edtext{\textcolor{blue}{Heinis}\pwindex{Schnitzler, Heinrich 09.08.1902 – 12.07.1982@\textsc{Schnitzler, Heinrich} (09.08.1902 – 12.07.1982), \emph{Regisseur/Regisseurin, Schauspieler/Schauspielerin}|pw}{}\ledrightnote{\textcolor{blue}{Heinrich Schnitzler}}
               Operation}{\lemma{\textnormal{\emph{Heinis
               Operation}}}\Cendnote{\textnormal{Bei \textcolor{blue}{Heinrich Schnitzler}\pwindex{Schnitzler, Heinrich 09.08.1902 – 12.07.1982@\textsc{Schnitzler, Heinrich} (09.08.1902 – 12.07.1982), \emph{Regisseur/Regisseurin, Schauspieler/Schauspielerin}|pwk} wurde am 
            4. 3. 1913 eine 
            Blinddarmentzündung diagnostiziert. Am 9. 3. 1913
            wurde er von seinem Onkel \textcolor{blue}{Julius Schnitzler}\pwindex{Schnitzler, Julius 13.07.1865 – 29.06.1939@\textsc{Schnitzler, Julius} (13.07.1865 – 29.06.1939), \emph{Chirurg/Chirurgin}|pwk} operiert.}}}\label{K_L03644-2} so gut geglückt ist
      und die Schwere (von dem ich
      nichts wusste) Ihnen und Ihrer
      Frau \textcolor{blue}{Gemahlin}\pwindex{Schnitzler, Olga 17.01.1882 – 13.01.1970@\textsc{Schnitzler, Olga} (17.01.1882 – 13.01.1970), \emph{Schauspieler/Schauspielerin, Sänger/Sängerin}|pwv}{}\ledrightnote{{$\rightarrow$}\emph{\textcolor{blue}{Olga Schnitzler}}} genommen
      ist. Volle Genesung wünscht
      von ferne Ihr ergebener\pend
           \pstart \spacefill\mbox{Stefan Zweig}\pend{}\selectlanguage{ngerman}\endnumbering\briefempfaengerindex{Schnitzler, Arthur@\textsc{Schnitzler, Arthur}!zzzZweig, Stefan@\emph{von Stefan Zweig}!1913-03-201@{20. 3. 1913}|)be}\mylabel{L03644h}  \normalsize

\doendnotes{C}
\bigskip
\vfill

\clearpage

\footnotesize

\lohead{\textsc{register}}

% Definiere theindex-Environment komplett neu ohne reledmac
\makeatletter
\renewenvironment{theindex}{%
  \section*{\indexname}%
  \setlength{\parindent}{0pt}%
  \setlength{\parskip}{0pt plus 0.3pt}%
  \let\item\@idxitem
}{%
  \clearpage
}
\makeatother

\IfFileExists{\jobname-pw.ind}{\input{\jobname-pw.ind}}{}

\end{document}

      