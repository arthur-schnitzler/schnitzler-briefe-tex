%% latex-korrekturansicht-vorspann.tex
%% Vorspann für die Korrekturansicht.
%% Lädt die gemeinsame Datei latex-vorspann.tex mit gesetztem Schalter.

\newif\ifkorrekturansicht
\korrekturansichttrue

\input{../tex-inputs/latex-vorspann}


\section[Elsa Plessner an Arthur Schnitzler, 13. 1. 1897]{L03712 Elsa Plessner an Arthur Schnitzler, 13. 1. 1897}
\nopagebreak\mylabel{L03712v}
\rehead{ }\normalsize\beginnumbering\briefempfaengerindex{Schnitzler, Arthur@\textsc{Schnitzler, Arthur}!zzzPlessner, Elsa@\emph{von Elsa Plessner}!1897-01-131@{13. 1. 1897}|(be}
\toendnotes[C]{\smallbreak\pagebreak[2]}
\correspDesc{Versand  durch Elsa Plessner am 13. 1. 1897 in Meran
\newline{}Erhalt  durch Arthur Schnitzler im Zeitraum [14. 1. 1897
                  – 18. 1. 1897?] in Wien}\toendnotes[C]{\smallbreak}
\Standort{DLA, A:Schnitzler, HS.1985.1.419.}
\physDesc{Brief, 2 Blätter, 3 Seiten, 3617 Zeichen
\newline{}Handschrift: schwarze Tinte, lateinische Kurrent}\toendnotes[C]{\smallbreak}
\pstart
           {\pb}\textcolor{pink}{Meran, Pension Wolf}\oindex{Hotel Meranerhof@\textbf{Hotel Meranerhof}, \emph{Hotel}|pw}{}\ledrightnote{\textcolor{pink}{Hotel Meranerhof}}, den
                     13. 1. 97.\pend
           
\pstart{}Verehrter, lieber, guter Herr Doctor!\pend\vspace{0.5em}
\pstart
           Haben Sie Sünden? – Wenn ja, nun so erkläre ich, dass sie Ihnen in Bausch und Bogen
               verziehen sind – blos um des guten Werkes willen, das Sie an mir thun! – – Ohne
               Scherz – ich kann Ihnen das wirklich nie danken, wüsste auch nicht wie – was Sie mir
               durch Ihren Rath und Ihre beispiellose Theilnahme für meine literarischen Schmerzen
               an Güte und Liebenswürdigkeit zuwenden. – – –  \pend
           
\pstart
           Aber dass ich \introOben{}an\introOben{} mangelnder Einsicht in das Wesen meiner
               eigenen Entwicklung leide ist \uline{leider} nicht richtig –.
               Zum Unglück kenne ich mich so genau, wie ich – z. B. Ihren \textcolor{green}{Anatol}\pwindex{Schnitzler, Arthur 15. 5. 1862 Wien – 21. 10. 1931 ebd.@\textsc{Schnitzler, Arthur} (15. 5. 1862 Wien – 21. 10. 1931 ebd.), \emph{Schriftsteller, Mediziner}!Anatol@\strich\emph{Anatol}|pw}{}\ledrightnote{\textcolor{green}{Anatol}} kenne! – Ich weiß es ganz genau, dass ich an eine
               gewisse Grenze gekommen bin – etwas, das »Halt« ruft –. Ich bin so furchtbar leer –
               wie eine alte Erbsenhülse {\pb}ich habe nichts mehr zu geben – und das ist
               auch gar nicht zu verwundern. Ich – als Künstlerin, kann nur etwas sein, wenn ich
               subjectiv bin – darum kann mir das Sehen und Denken \introOben{}allein\introOben{}
               nicht viel bringen. Sie haben ganz recht – ich müsste was \uline{sein} – aber ich bin eben nichts –. Mein Leben in Ordnung bringen – ? – Das ist
               so hin gesagt – und \uline{Sie} könnens ja auch! – Sie gehen
               und packen mit zwei Händen was sich packen lässt – Sie holen sich, was Sie brauchen!
               Aber ich? – Ich muss warten bis gnädig etwas zu mir kommt – das kann ich \introOben{}dann\introOben{} höchstens kritisieren! – Das hab’ ich auch bis jetzt
               gethan – alle meine Versuche in der Kunst waren nichts als ein »Nein« sagen!
               Vielleicht entsinnen sich? Aber damit kommt man nur kurze Zeit aus. Jetzt endlich hab
               ich das Brünnlein ausgeschöpft, mein bißchen Fond verbraucht – aber \uline{Neues} ist \uline{nicht}
               hinzugekommen! – Und das ist es! – Variationen auf der G Saite fabricieren? – Ich
               nicht! – Auch finde ich nicht mehr zurück. – Bis jetzt hab ich nur gefühlt und gedacht
               – »das muss \substVorne{}\textsuperscript{A}\substDazwischen{}a\substHinten{}nders sein« – {\pb}aber wie muss es sein? – Darum sind die »\textcolor{green}{Orchideen}\pwindex{Plessner, Elsa 22.\,8.\,1875 Wien – 7.\,5.\,1932 Alicante@\textsc{Plessner, Elsa} (22.\,8.\,1875 Wien – 7.\,5.\,1932 Alicante), \emph{Schriftstellerin}!Orchideen [Schauspiel in drei Akten]@\strich\emph{Orchideen [Schauspiel in drei Akten]}|pw}{}\ledrightnote{\textcolor{green}{Orchideen [Schauspiel in drei Akten]}}« verhauen, weil ich zum ersten mal
               versuchte, positiv zu sein – etwas zu gestalten – oder besser – zu construiren, was
               ich eben nicht kenne, nicht \uline{habe}! – Da hat eben das
               Denken und Sehen Schiffbruch gelitten – denn \uline{Denken},
               glaub ich, ist immer mehr oder weniger unpersönlich – und meine ganze Kraft liegt im
               Persönlichen. – Die verschwommene\strikeout{n}, träumerische
               Weichheit, das Knochenlose, das den größten Reiz z. B. des »\textcolor{green}{Heimweh}\pwindex{Plessner, Elsa 22.\,8.\,1875 Wien – 7.\,5.\,1932 Alicante@\textsc{Plessner, Elsa} (22.\,8.\,1875 Wien – 7.\,5.\,1932 Alicante), \emph{Schriftstellerin}!Heimweh [dreiaktige Tragikomödie]@\strich\emph{Heimweh [dreiaktige Tragikomödie]}|pw}{}\ledrightnote{\textcolor{green}{Heimweh [dreiaktige Tragikomödie]}}« bildet – ist auch dessen Schwäche, aber das bin \uline{ich}, das beherrsche ich, da bin ich zu Hause – – –
               gewesen; aber jetzt bin ich delogirt und habe noch keine andre Unterkunft! – Ich bin
               an etwas undefinirbarem angeprallt – und kann nicht weiter! – Gedacht und gesehen
               habe ich so viel – aber, verehrter Herr Doctor, ich komme mir vor wie eine
               Perlmuschel. – Der kostbare Perlstoff ist da – wenn nur das winzige Körnchen auch da
               wäre, das die Perle erst hervorruft! – – Es ist eine furchtbare, tote Zeit!! – Es
               muss etwas Neues kommen. Und täglich und stündlich hofft man – und dann ist es doch
               nicht gekommen. Ich habe so viel – so eine Menge in mir – aber das muss erst frei
               gemacht werden und ich weiß nicht womit – ! Und wenn ich doch so gerne arbeiten
               möchte – – Ich habe immer das Gefühl, dass Alles an mir vorbeigeht, das wichtigste,
               und – ich weiß nicht wie – wie wenn ich auf einer Felsenspitze mitten im Meer säße
               und es wäre immer Ebbe und ich warte auf die Flut. – –\pend
           
\pstart
           Und ich soll etwas sein? – Soll mir mein Leben gestalten? – Das fließt auseinander –
               wie soll ich das machen!? – Und das reibt wahnsinnig auf, dieser Zustand –.\pend
           
\pstart
           Bitte, verehrter Herr Doctor – seien Sie nicht böse über diesen confusen Erguß – aber
               Ihre liebenswürdigen \label{K_L03712-1v}\edtext{Zeilen}{\lemma{\textnormal{\emph{Zeilen}}}\Cendnote{\textnormal{nicht überliefert}}}\label{K_L03712-1} haben der Tonne
               das Spundloch eingeschlagen. Nochmals meinen herzlichsten Dank! –\pend
           \pstart \spacefill\mbox{Elsa Plessner}\pend{}\selectlanguage{ngerman}\endnumbering\briefempfaengerindex{Schnitzler, Arthur@\textsc{Schnitzler, Arthur}!zzzPlessner, Elsa@\emph{von Elsa Plessner}!1897-01-131@{13. 1. 1897}|)be}\mylabel{L03712h}  \normalsize

\doendnotes{C}
\bigskip
\vfill

\clearpage

\footnotesize

\lohead{\textsc{register}}

% Definiere theindex-Environment komplett neu ohne reledmac
\makeatletter
\renewenvironment{theindex}{%
  \section*{\indexname}%
  \setlength{\parindent}{0pt}%
  \setlength{\parskip}{0pt plus 0.3pt}%
  \let\item\@idxitem
}{%
  \clearpage
}
\makeatother

\IfFileExists{\jobname-pw.ind}{\input{\jobname-pw.ind}}{}

\end{document}

      