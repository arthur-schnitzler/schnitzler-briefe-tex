%% latex-korrekturansicht-vorspann.tex
%% Vorspann für die Korrekturansicht.
%% Lädt die gemeinsame Datei latex-vorspann.tex mit gesetztem Schalter.

\newif\ifkorrekturansicht
\korrekturansichttrue

\input{../tex-inputs/latex-vorspann}


\renewcommand{\erwaehntePersonen}{Personen: Richard Beer-Hofmann, Otto Brahm, Julius von Gans-Ludassy, Georg Hirschfeld, Hugo von Hofmannsthal, Wilhelmine Mitterwurzer, Ottilie Salten, Gustav Schwarzkopf}
\renewcommand{\erwaehnteOrte}{Orte: London, Paris, Riva del Garda, Wien}
\renewcommand{\erwaehnteWerke}{Werke: Agnes Jordan. Schauspiel in fünf Akten}
\section[ Felix Salten an Arthur Schnitzler, 5. 5. 1897]{Felix Salten an Arthur Schnitzler, 5. 5. 1897}
\nopagebreak\mylabel{v}
\rehead{ }\normalsize\beginnumbering\briefempfaengerindex{Schnitzler, Arthur@\textsc{Schnitzler, Arthur}!zzzSalten, Felix@\emph{von Felix Salten}!1897-05-051@{5. 5. 1897}|(be}
\toendnotes[C]{\smallbreak\pagebreak[2]}\Standort{CUL, Schnitzler, B 89, A 2.}
\physDesc{Brief, 1 Blatt, 3 Seiten, 2588 Zeichen
\newline{}Handschrift: schwarze Tinte, lateinische Kurrent
\newline{}Ordnung: mit Bleistift von unbekannter Hand nummeriert: »87« }\toendnotes[C]{\smallbreak}
\pstart
           \raggedleft{}{\pb}\textcolor{pink}{Wien}{}\ledrightnote{\textcolor{pink}{Wien}}, am 5. Mai 97.\pend
           
\pstart
           Lieber Arthur, seit ein paar Tagen bin ich wieder in \textcolor{pink}{Wien}{}\ledrightnote{\textcolor{pink}{Wien}}. Ich war in \textcolor{pink}{Riva}{}\ledrightnote{\textcolor{pink}{Riva del Garda}} – sehr schön. Aber es hätte viel schöner sein können, wenn die \textcolor{blue}{Mitterwurzer}{}\ledrightnote{\textcolor{blue}{Wilhelmine Mitterwurzer}} nicht dabei gewesen wäre. \textcolor{pink}{Hier}{}\ledrightnote{{$\rightarrow$}\textcolor{pink}{Wien}} lebe ich in einer
               merkwürdigen Sorglosigkeit. Eigentlich begreife ich es selbst nicht, warum ich mich
               so völlig unbekümmert hintreiben laße. Manchmal sage ich mir, dass irgend eine
               günstige Wendung bevorsteht, dass ich sie in allen Gliedern spüre und dass ich
               deshalb so frei bin. Dabei fällt mir immer ein, was Sie mir gelegentlich sagten: Dass
               man sich bei mir immer eines Glückfalles versieht. Für Ihren Brief dank ich Ihnen
               sehr. Es war ja nicht viel, aber etwas, und ich bin jetzt – Frl. \textcolor{blue}{M.}{}\ledrightnote{\textcolor{blue}{Ottilie Salten}} ausgenommen – sehr einsam. Ich arbeite ordentlich dabei
                  {\pb}und nach allen Seiten. Es
               geht kein Tag hin, an dem mir nicht etwas Erfreuliches einfiele. Dabei bin ich jetzt
               an Büchern und Menschen und an den Erinnerungen vorbei auf einem ziemlich directen
               Weg zu mir selbst. Es ist eine eigenthümlich aufregende Zeit. Frühling kann man nicht
               sagen, – denn es ist etwas \substVorne{}\textsuperscript{z}\substDazwischen{}Z\substHinten{}weites, alles ist dezidirter, kühler und alle Formen sind ohne den
               ahnungsvollen Nebel, und klarer. Es gibt keinen Menschen, kein Buch, nichts in meinem
               Leben, zu dem ich nicht eine \substVorne{}\textsuperscript{v}\substDazwischen{}t\substHinten{}otal veränderte Beziehung hätte, als vorher. Das ist natürlich nicht erst in
               acht Tagen geworden, aber erst auf meiner Reise, und dann jetzt hier, habe ich ein
               wenig Ordnung mit diesen Dingen gemacht, und meine Interessen gesäubert.\pend
           
\pstart
           Von äußeren Umständen weiß ich Ihnen {\pb}nichts Neues zu sagen.
               Vielleicht finde ich eine Stellung – \textcolor{blue}{Ludaßy}{}\ledrightnote{\textcolor{blue}{Julius von Gans-Ludassy}}
               behauptet, \strikeout{\textcolor{gray}{d}} er habe große Dinge vor, – vielleicht wird etwas mit einer \label{K_L03264-1v}\edtext{Direction}{\lemma{\textnormal{\emph{Direction}}}\Cendnote{\textnormal{vgl. Felix Salten an Arthur Schnitzler, [10. 1. 1897]}}}\label{K_L03264-1h}; vielleicht schreibe ich von den Stoffen, mit denen ich mich jetzt
               beschäftige einen zu Ende, – das letztere ist das wahrscheinlichste. \strikeout{\textcolor{gray}{×}\-\textcolor{gray}{×}\-\textcolor{gray}{×}} Das Bicycle und Frl. \textcolor{blue}{M.}{}\ledrightnote{\textcolor{blue}{Ottilie Salten}} füllen meine
               übrige Zeit aus. Seit der Reise ist auch hier eine entscheidende Wendung eingetreten.
               Das macht mich auch besser und ruhiger und gibt meinem Leben wieder einen vollen
               Duft, denn ich habe lange Niemanden lieb gehabt. Sonst leb ich mit keinem Menschen
               und habe Keinen, mit dem ich sprechen möchte.\pend
           
\pstart
           Bei den übrigen ist, glaub ich, alles beim Alten, oder doch nichts wesentliches
               geschehen. \textcolor{blue}{Hugo}{}\ledrightnote{\textcolor{blue}{Hugo von Hofmannsthal}} sehe ich selten, und wenn,
               dann reden wir vom Bicycle. Mein Verkehr mit \textcolor{blue}{Beer-Hofmann}{}\ledrightnote{\textcolor{blue}{Richard Beer-Hofmann}} beschränkt sich aufs Pokerspielen und mit \textcolor{blue}{Schwarzkopf}{}\ledrightnote{\textcolor{blue}{Gustav Schwarzkopf}} kann ich garnichts sprechen. \textcolor{blue}{G. Hirschfeld}{}\ledrightnote{\textcolor{blue}{Georg Hirschfeld}} sehe ich manchmal. Er will mir sein \textcolor{green}{Stück}{}\ledrightnote{{$\rightarrow$}\textcolor{green}{Agnes Jordan. Schauspiel in fünf Akten}} vorlesen. \textcolor{blue}{Brahm}{}\ledrightnote{\textcolor{blue}{Otto Brahm}} ist \textcolor{pink}{hier}{}\ledrightnote{{$\rightarrow$}\textcolor{pink}{Wien}}, glaube ich, – ich habe ihn aber noch nicht gesehen.\pend
           
\pstart
           Sie \label{K_L03264-2v}\edtext{kommen ja wol bald}{\lemma{\textnormal{\emph{kommen ja wol bald}}}\Cendnote{\textnormal{An den Aufenthalt in \textcolor{pink}{Paris} (siehe Arthur Schnitzler an Felix Salten, 26. 4. 1897) hängte \textcolor{blue}{Schnitzler} eine
                  Reise nach \textcolor{pink}{London}. Am 2. 6. 1897 kam er
                  wieder in \textcolor{pink}{Wien} an.}}}\label{K_L03264-2h}? Bis dahin höre ich
               doch noch öfter, wie es Ihnen geht.\pend
           
\pstart
           Herzlich Ihr {\\[\baselineskip]}\spacefill\mbox{Salten}\pend
           \leftskip=0em{}\endnumbering\briefempfaengerindex{Schnitzler, Arthur@\textsc{Schnitzler, Arthur}!zzzSalten, Felix@\emph{von Felix Salten}!1897-05-051@{5. 5. 1897}|)be}\mylabel{h}  \normalsize

\doendnotes{C}
\bigskip
\vfill

\clearpage

\footnotesize

\lohead{\textsc{register}}

% Definiere theindex-Environment komplett neu ohne reledmac
\makeatletter
\renewenvironment{theindex}{%
  \section*{\indexname}%
  \setlength{\parindent}{0pt}%
  \setlength{\parskip}{0pt plus 0.3pt}%
  \let\item\@idxitem
}{%
  \clearpage
}
\makeatother

\IfFileExists{\jobname-pw.ind}{\input{\jobname-pw.ind}}{}

\end{document}

      