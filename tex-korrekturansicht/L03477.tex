%% latex-korrekturansicht-vorspann.tex
%% Vorspann für die Korrekturansicht.
%% Lädt die gemeinsame Datei latex-vorspann.tex mit gesetztem Schalter.

\newif\ifkorrekturansicht
\korrekturansichttrue

\input{../tex-inputs/latex-vorspann}


\renewcommand{\erwaehnteOrte}{Orte: Bendlerstraße, Berlin, Wien}
\renewcommand{\erwaehnteWerke}{}
\section[ Paul Goldmann an Arthur Schnitzler, 10. 2. 1915]{Paul Goldmann an Arthur Schnitzler, 10. 2. 1915}
\nopagebreak\mylabel{v}
\rehead{ }\normalsize\beginnumbering\briefempfaengerindex{Schnitzler, Arthur@\textsc{Schnitzler, Arthur}!zzzGoldmann, Paul@\emph{von Paul Goldmann}!1915-02-102@{10. 2. 1915}|(be}
\toendnotes[C]{\smallbreak\pagebreak[2]}\Standort{DLA, A:Schnitzler, HS.NZ85.1.3176.}
\physDesc{Brief, 1 Blatt, 4 Seiten, 1095 Zeichen
\newline{}Handschrift: , deutsche Kurrent
\newline{}Schnitzler: 1) mit Bleistift Vermerk »\textcolor{blue}{Goldma{\geminationn}}«  2) mit rotem Buntstift eine Unterstreichung}\toendnotes[C]{\smallbreak}
\pstart
           \noindent{}\raggedleft{}{\pb}\textcolor{gray}{\textbf{\textcolor{pink}{W. BENDLERSTRASSE 36}{}\ledrightnote{\textcolor{pink}{Bendlerstraße}}}}\pend
           
\pstart
           10. 2. 15.\pend
           
\pstart{}Lieber Arthur,\pend
\pstart
           Ich Danke Dir herzlich für Dein \label{K_L03477-1v}\edtext{Glückwunſchtelegramm}{\lemma{\textnormal{\emph{Glückwunſchtelegramm}}}\Cendnote{\textnormal{\textcolor{blue}{Goldmann} war am 31. 1. 1915 50 Jahre alt geworden.}}}\label{K_L03477-1h}, das mich aufrichtig erfreut
               hat.\pend
           
\pstart
           Die guten gemeinſamen Stunden, die Du erwähnſt, 
                  –
                auch ich habe ſie nicht vergeſſen. Wie könnte ich auch? Sie ſind ein
               weſentlicher Teil meines Lebens u. gehören {\pb}zum
               Beſten, das es enthält.\pend
           
\pstart
           Zwei Lebenswege, die lange gemeinſam verlaufen ſind, haben ſich \label{K_L03477-2v}\edtext{getrennt}{\lemma{\textnormal{\emph{getrennt}}}\Cendnote{\textnormal{Zum großen Bruch war es Anfang 1911
                  gekommen, siehe Paul Goldmann an Arthur Schnitzler, 13. 1. 1911.}}}\label{K_L03477-2h}, –
               zwei Menſchen, die ſich lange nahegeſtanden, haben ſich von einander entfernt. Wen
               trifft die Schuld? Vielleicht gibt es da überhaupt keine Schuld, ſondern nur ein
               Geſetz der Entwickelung.\pend
           
\pstart
           Aber die Vergangenheit {\pb}bleibt beſtehen. Und ſie
               hat ſoviel verſöhnende Kraft durch die Fülle des Guten, das ſie enthält! Ich danke
               Dir, daß Du ſie angerufen, – danke dem Freunde langer Jahre für alles, das er mir
               gegeben, – u. danke Dir von Herzen, daß Du mir auch heut noch eine freundliche
               Geſinnug bewahrſt. Auch bei mir hat dieſe Geſinnung alles Trennende überdauert; {\pb}u. an der Aufrichtigkeit, mit der ich Dir Gutes
               wünſche, hat ſich bei mir niemals etwas geändert u. wird ſich niemals etwas
               ändern.\pend
           
\pstart
           Mit herzlichem Gruß {\\[\baselineskip]}Dein {\\[\baselineskip]}\spacefill\mbox{Paul Goldmann.}\pend
           \leftskip=0em{}\endnumbering\briefempfaengerindex{Schnitzler, Arthur@\textsc{Schnitzler, Arthur}!zzzGoldmann, Paul@\emph{von Paul Goldmann}!1915-02-102@{10. 2. 1915}|)be}\mylabel{h}  \normalsize

\doendnotes{C}
\bigskip
\vfill

\clearpage

\footnotesize

\lohead{\textsc{register}}

% Definiere theindex-Environment komplett neu ohne reledmac
\makeatletter
\renewenvironment{theindex}{%
  \section*{\indexname}%
  \setlength{\parindent}{0pt}%
  \setlength{\parskip}{0pt plus 0.3pt}%
  \let\item\@idxitem
}{%
  \clearpage
}
\makeatother

\IfFileExists{\jobname-pw.ind}{\input{\jobname-pw.ind}}{}

\end{document}

      