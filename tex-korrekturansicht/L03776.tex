%% latex-korrekturansicht-vorspann.tex
%% Vorspann für die Korrekturansicht.
%% Lädt die gemeinsame Datei latex-vorspann.tex mit gesetztem Schalter.

\newif\ifkorrekturansicht
\korrekturansichttrue

\input{../tex-inputs/latex-vorspann}


\section[Arthur Schnitzler an Stefan Zweig, 9. 11. 1914]{L03776 Arthur Schnitzler an Stefan Zweig, 9. 11. 1914}
\nopagebreak\mylabel{L03776v}
\rehead{ }\normalsize\beginnumbering\briefempfaengerindex{Zweig, Stefan@\textsc{Zweig, Stefan}!zzzSchnitzler, Arthur@\emph{von Arthur Schnitzler}!1914-11-091@{9. 11. 1914}|(be}
\toendnotes[C]{\smallbreak\pagebreak[2]}\Standort{Jerusalem, National Library of Israel, ARC. Ms. Var. 305 1 58 Stefan Zweig Collection.}
\physDesc{Brief, 1 Blatt, 1 Seite, 791 Zeichen
\newline{}Schreibmaschine
\newline{}Handschrift: schwarze Tinte, lateinische Kurrent (\noindent{}Unterschrift)}\toendnotes[C]{\smallbreak}
\pstart
           {\pb}\textcolor{gray}{\textbf{Dr. Arthur Schnitzler}}\hfill 9. 11. 1914. \pend
           
\pstart
           \textcolor{gray}{\textbf{\textcolor{pink}{Wien XVIII. Sternwartestrasse 71}\oindex{Sternwartestrasse 71@\textbf{Sternwartestraße 71}, \emph{Wohngebäude (K.WHS)}|pw}{}\ledrightnote{\textcolor{pink}{Sternwartestraße 71}}}}\pend
           
\pstart\center{}Lieber Herr Doktor Zweig.\pend\vspace{0.5em}
\pstart
           Wie Sie vielleicht schon erfahren haben, soll eine Internationale Revue gegründet
               werden, für deren Zustandekommen sich hier besonders Dr. \textcolor{blue}{Ludo Hartmann}\pwindex{Hartmann, Ludo Moritz 02.03.1865 – 14.11.1924@\textsc{Hartmann, Ludo Moritz} (02.03.1865 – 14.11.1924), \emph{Politiker, Historiker, Volksbildner}|pw}{}\ledrightnote{\textcolor{blue}{Ludo Moritz Hartmann}} einsetzt. Er \label{K_L03776-1v}\edtext{war bei mir}{\lemma{\textnormal{\emph{war bei mir}}}\Cendnote{\textnormal{Vgl. A. S.: \emph{Tagebuch}, 7. 11. 1914. Die
                  Zeitschrift wurde nicht umgesetzt.}}}\label{K_L03776-1} unter anderm um mich zu fragen, ob ich
               eine Verbindung zwischen ihm und \textcolor{blue}{Romain
                  Rolland}\pwindex{Rolland, Romain 29.01.1866 – 30.12.1944@\textsc{Rolland, Romain} (29.01.1866 – 30.12.1944), \emph{Schriftsteller}|pw}{}\ledrightnote{\textcolor{blue}{Romain Rolland}} anbahnen könne. Ich habe mir erlaubt ihn mit dieser Absicht an sie,
               lieber Herr Doktor, zu weisen und er möchte Sie bitten in obengedachtem Sinn, wenn es
               irgend angeht, \label{K_L03776-2v}\edtext{an \textcolor{blue}{Rolland}\pwindex{Rolland, Romain 29.01.1866 – 30.12.1944@\textsc{Rolland, Romain} (29.01.1866 – 30.12.1944), \emph{Schriftsteller}|pw}{}\ledrightnote{\textcolor{blue}{Romain Rolland}} zu schreiben}{\lemma{\textnormal{\emph{an Rolland zu schreiben}}}\Cendnote{\textnormal{Am 11. 11. 1914 (Poststempel) schrieb \textcolor{blue}{Zweig}\pwindex{Zweig, Stefan 28.11.1881 – 23.02.1942@\textsc{Zweig, Stefan} (28.11.1881 – 23.02.1942), \emph{Schriftsteller}|pwk} an \textcolor{blue}{Rolland}\pwindex{Rolland, Romain 29.01.1866 – 30.12.1944@\textsc{Rolland, Romain} (29.01.1866 – 30.12.1944), \emph{Schriftsteller}|pwk}: »Ich verständige Sie gleichzeitig, dass ein Versuch einer
                     neutralen Zeitschrift in der \textcolor{pink}{Schweiz}\oindex{Schweiz@\textbf{Schweiz}, \emph{A.PCLI}|pw}
                     doppelsprachig unternommen werden soll. Professor \textcolor{blue}{Brockhausen}\pwindex{Brockhausen, Carl 1859-05-09 – 1951-09-16@\textsc{Brockhausen, Carl} (1859-05-09 – 1951-09-16), \emph{Ministerialbeamter, Jurist, Hochschullehrer}|pw}, ein bekannter Nationalöconom und
                     Friedensfreund, wird in dieser Sache von \textcolor{pink}{Wien}\oindex{Wien@\textbf{Wien}, \emph{A.ADM2}|pw} aus delegiert, er wird sicherlich in der \textcolor{pink}{Schweiz}\oindex{Schweiz@\textbf{Schweiz}, \emph{A.PCLI}|pw} auch Ihre Mitarbeit zu werben suchen, und ich
                     kann Ihnen nur sagen, dass er als rechtlich und tüchtig gilt, seine
                     vortreffliche Absicht nicht zu bezweifeln ist. Die Organisation kann ich nicht
                     beurteilen – hoffentlich setzt er sie Ihnen auseinander.« (\textcolor{blue}{Romain Rolland}\pwindex{Rolland, Romain 29.01.1866 – 30.12.1944@\textsc{Rolland, Romain} (29.01.1866 – 30.12.1944), \emph{Schriftsteller}|pwk}, \textcolor{blue}{Stefan Zweig}\pwindex{Zweig, Stefan 28.11.1881 – 23.02.1942@\textsc{Zweig, Stefan} (28.11.1881 – 23.02.1942), \emph{Schriftsteller}|pwk}: \emph{Von Welt zu Welt. Briefe
                        einer Freundschaft 1914–1918}. Mit einem Begleitwort von Peter
                     Handke. Aus dem Französischen von Eva und Gerhard Schwewe (Briefe Rollands) und
                     Christel Gersch (Briefe Zweigs). Berlin: \emph{Aufbau
                        Verlag}{ }2014.)}}}\label{K_L03776-2}. Interessieren Sie sich für die ganze Angelegenheit, mit der es
               schon in allernächster Zeit Ernst werden soll, so setzen Sie sich mit \textcolor{blue}{Ludo Hartmann}\pwindex{Hartmann, Ludo Moritz 02.03.1865 – 14.11.1924@\textsc{Hartmann, Ludo Moritz} (02.03.1865 – 14.11.1924), \emph{Politiker, Historiker, Volksbildner}|pw}{}\ledrightnote{\textcolor{blue}{Ludo Moritz Hartmann}} vielleicht telefonisch in
               Verbindung, nicht wahr?\pend
           
\pstart
           Entschuldigen Sie die Bemühung, seien Sie herzlichst gegrüsst und auf baldiges
               Wiedersehen{\\[\baselineskip]}Ihr \spacefill\mbox{{[}hs.:{]} Arthur Schnitzler}\pend
           \leftskip=0em{}\selectlanguage{ngerman}\endnumbering\briefempfaengerindex{Zweig, Stefan@\textsc{Zweig, Stefan}!zzzSchnitzler, Arthur@\emph{von Arthur Schnitzler}!1914-11-091@{9. 11. 1914}|)be}\mylabel{L03776h}  \normalsize

\doendnotes{C}
\bigskip
\vfill

\clearpage

\footnotesize

\lohead{\textsc{register}}

% Definiere theindex-Environment komplett neu ohne reledmac
\makeatletter
\renewenvironment{theindex}{%
  \section*{\indexname}%
  \setlength{\parindent}{0pt}%
  \setlength{\parskip}{0pt plus 0.3pt}%
  \let\item\@idxitem
}{%
  \clearpage
}
\makeatother

\IfFileExists{\jobname-pw.ind}{\input{\jobname-pw.ind}}{}

\end{document}

      