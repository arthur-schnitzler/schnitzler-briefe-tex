%% latex-korrekturansicht-vorspann.tex
%% Vorspann für die Korrekturansicht.
%% Lädt die gemeinsame Datei latex-vorspann.tex mit gesetztem Schalter.

\newif\ifkorrekturansicht
\korrekturansichttrue

\input{../tex-inputs/latex-vorspann}


\renewcommand{\erwaehntePersonen}{Personen: Olga Schnitzler, Elisabeth Steinrück, Max Sternfeld}
\renewcommand{\erwaehnteInstitutionen}{Institutionen: Berliner Tageblatt, Goethe-Bund}
\renewcommand{\erwaehnteOrte}{Orte: Alte Philharmonie (Berlin), Berlin, Deutschland, Palasthotel Berlin, Redaktion des Berliner Tageblatts, Wallnertheaterstraße, Wien}
\renewcommand{\erwaehnteWerke}{Werke: Der Goethebund gegen die Theatercensur. (Telegramm der »Neuen Freien Presse«.), Neue Freie Presse, Tagebuch}
\section[ Paul Goldmann an Arthur Schnitzler, 8. 3. 1903]{Paul Goldmann an Arthur Schnitzler, 8. 3. 1903}
\nopagebreak\mylabel{v}
\rehead{ }\normalsize\beginnumbering\briefempfaengerindex{Schnitzler, Arthur@\textsc{Schnitzler, Arthur}!zzzGoldmann, Paul@\emph{von Paul Goldmann}!1903-03-082@{8. 3. 1903}|(be}
\toendnotes[C]{\smallbreak\pagebreak[2]}\Standort{DLA, A:Schnitzler, HS.NZ85.1.3173.}
\physDesc{Postkarte
\newline{}Handschrift: 1) blaue Tinte, deutsche Kurrent\hspace{1em}2) blaue Tinte, lateinische Kurrent (\noindent{}Adresse)\hspace{1em}
\newline{}Versand: Stempel: »\nobreak{}\oindex{Berlin@\textbf{Berlin}, \emph{https://www.geonames.org/ontologyP.PPLC}|pwk}Berlin, W. 9, 8. 3. 03, 5—N.\nobreak{}«. Stempel: »\nobreak{}\oindex{Berlin@\textbf{Berlin}, \emph{https://www.geonames.org/ontologyP.PPLC}|pwk}Berlin\textcolor{gray}{,} O.
                                       P27 (R15), 8 III 03, 5\textsuperscript{30} N\textcolor{gray}{.}\nobreak{}«.  }\toendnotes[C]{\smallbreak}\pstart{}{\pb}Fräulein Elisabeth Gussmann\pend{}\pstart{}für Herrn Dr. Schnitzler\pend{}\pstart{}\textcolor{pink}{Wallnertheaterstraße 40}{}\ledrightnote{\textcolor{pink}{Wallnertheaterstraße}}\pend{}\pstart{}II. bei \textcolor{blue}{Sternfeld}{}\ledrightnote{\textcolor{blue}{Max Sternfeld}}.\pend{}
{\bigskip}
\pstart
           \noindent{}{\pb}\textcolor{pink}{Berlin}{}\ledrightnote{\textcolor{pink}{Berlin}}, 8. März.\hfill Mein lieber Freund,
                  \pend
           
\pstart
           Ich habe Dich zwei Mal im \textcolor{pink}{\textsc{Hotel}}{}\ledrightnote{{$\rightarrow$}\textcolor{pink}{Palasthotel Berlin}}
               geſucht, um Dir zu ſagen, daß ich heut{ }{ }Abend\strikeout{\textcolor{gray}{×}\-\textcolor{gray}{×}\-\textcolor{gray}{×}} leider \label{K_L03368-1v}\edtext{nicht kommen}{\lemma{\textnormal{\emph{nicht kommen}}}\Cendnote{\textnormal{vermutlich zu \textcolor{blue}{Elisabeth Gussmann} – dafür spricht der \emph{\textcolor{green}{Tagebuch}}-Eintrag zum 8. 3. 1903 und die Adressierung der Postkarte an
                  sie}}}\label{K_L03368-1h} kann. Ich erhielt heut{ }Morgen telegraphiſchen Auftrag aus \textcolor{pink}{Wien}{}\ledrightnote{\textcolor{pink}{Wien}}, den \label{K_L03368-2v}\edtext{\textcolor{green}{Bericht}{}\ledrightnote{{$\rightarrow$}\textcolor{green}{Der Goethebund gegen die Theatercensur. (Telegramm der »Neuen Freien Presse«.)}} über die \textcolor{brown}{Goethebund}{}\ledrightnote{\textcolor{brown}{Goethe-Bund}}-Verſammlung}{\lemma{\textnormal{\emph{Bericht … Goethebund-Verſammlung}}}\Cendnote{\textnormal{[\textcolor{blue}{Paul Goldmann}:] \emph{\textcolor{green}{Der Goethebund gegen die Theatercensur. (Telegramm der
                        »Neuen Freien Presse«.)}}. In: \emph{\textcolor{green}{Neue Freie
                        Presse}}, Nr. 13841, 9. 3. 1903,
                     Abendblatt, S. 3–4. Der \textcolor{pink}{deutsch}e \emph{\textcolor{brown}{Goethe-Bund}} tagte am 8. 3. 1903 in der \textcolor{pink}{Alten Berliner Philharmonie}.}}}\label{K_L03368-2h} noch heut zu ſchicken, muß ihn mir alſo heut{ }Abend auf der \textcolor{pink}{Redaktion}{}\ledrightnote{\textcolor{pink}{Redaktion des Berliner Tageblatts}} des \textcolor{brown}{Berl. Tagebl.}{}\ledrightnote{\textcolor{brown}{Berliner Tageblatt}} beſorgen und von dort abſenden. Das
               dauert mindeſtens bis 10. Wo u. wann kann ich Dich \label{K_L03368-3v}\edtext{morgen}{\lemma{\textnormal{\emph{morgen}}}\Cendnote{\textnormal{Am 9. 3. 1903 holte \textcolor{blue}{Goldmann}{ }\textcolor{blue}{Schnitzler} und \textcolor{blue}{Olga Gussmann} im \textcolor{pink}{Palasthotel} ab und begleitete sie zum Zug Richtung \textcolor{pink}{Wien}.}}}\label{K_L03368-3h} ſehen? Viele herzliche Grüße an Dich und die
               Anderen, namentlich an \textsc{\textcolor{blue}{Olga}{}\ledrightnote{\textcolor{blue}{Olga Schnitzler}}}. Dein \spacefill\mbox{P. G.}\pend
           \endnumbering\briefempfaengerindex{Schnitzler, Arthur@\textsc{Schnitzler, Arthur}!zzzGoldmann, Paul@\emph{von Paul Goldmann}!1903-03-082@{8. 3. 1903}|)be}\mylabel{h}  \normalsize

\doendnotes{C}
\bigskip
\vfill

\clearpage

\footnotesize

\lohead{\textsc{register}}

% Definiere theindex-Environment komplett neu ohne reledmac
\makeatletter
\renewenvironment{theindex}{%
  \section*{\indexname}%
  \setlength{\parindent}{0pt}%
  \setlength{\parskip}{0pt plus 0.3pt}%
  \let\item\@idxitem
}{%
  \clearpage
}
\makeatother

\IfFileExists{\jobname-pw.ind}{\input{\jobname-pw.ind}}{}

\end{document}

      