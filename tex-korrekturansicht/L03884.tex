%% latex-korrekturansicht-vorspann.tex
%% Vorspann für die Korrekturansicht.
%% Lädt die gemeinsame Datei latex-vorspann.tex mit gesetztem Schalter.

\newif\ifkorrekturansicht
\korrekturansichttrue

\input{../tex-inputs/latex-vorspann}


\section[Arthur Schnitzler an Romain Rolland, 14. 12. 1914]{L03884 Arthur Schnitzler an Romain Rolland, 14. 12. 1914}
\nopagebreak\mylabel{L03884v}
\rehead{ }\normalsize\beginnumbering\briefempfaengerindex{, @\textsc{, }!zzz, @\emph{von  }!1914-12-141@{14. 12. 1914}|(be}
\toendnotes[C]{\smallbreak\pagebreak[2]}\Standort{DLA, A:Schnitzler, 85.1.1714.}
\physDesc{BriefDurchschlag, 2352 Zeichen
\newline{}Schreibmaschine}
\buchAbdrucke{\weitereDrucke{Arthur Schnitzler: \emph{Briefe 1913–1931}. Frankfurt am Main: \emph{S. Fischer} 1984, S. 63–64.} }\toendnotes[C]{\smallbreak}
\pstart
           \raggedleft{}{\pb}14. 12. 1914.\pend
           
\pstart{}Verehrter Herr Rolland.\pend\vspace{0.5em}
\pstart
           Sie wollen also wirklich, wie mir \textcolor{blue}{Stefan Zweig}\pwindex{Zweig, Stefan 28.\,11.\,1881 Wien – 23.\,2.\,1942 Petrópolis@\textsc{Zweig, Stefan} (28.\,11.\,1881 Wien – 23.\,2.\,1942 Petrópolis), \emph{Schriftsteller}|pw}{}\ledrightnote{\textcolor{red}{KEY PROBLEM}} sagt,
               die gro8e Freundlichkeit haben meine \textcolor{green}{Erklärung}\pwindex{Schnitzler, Arthur 15. 5. 1862 Wien – 21. 10. 1931 ebd.@\textsc{Schnitzler, Arthur} (15. 5. 1862 Wien – 21. 10. 1931 ebd.), \emph{Schriftsteller, Mediziner}!Brief Artur Schnitzlers@\strich\emph{Ein Brief Artur Schnitzlers}|pwv}{}\ledrightnote{{$\rightarrow$}\emph{\textcolor{green}{Ein Brief Artur Schnitzlers}}} ins Französische zu übersetzen und wünschen überdies, zum Zweck
               der Veröffentlichung in einer deutschen \textcolor{pink}{Schweizer}\oindex{Schweiz@\textbf{Schweiz}|pw}{}\ledrightnote{\textcolor{pink}{Schweiz}}
               Zeitlung ein zweites \textcolor{green}{Exemplar}\pwindex{Schnitzler, Arthur 15. 5. 1862 Wien – 21. 10. 1931 ebd.@\textsc{Schnitzler, Arthur} (15. 5. 1862 Wien – 21. 10. 1931 ebd.), \emph{Schriftsteller, Mediziner}!Brief Artur Schnitzlers@\strich\emph{Ein Brief Artur Schnitzlers}|pwv}{}\ledrightnote{{$\rightarrow$}\emph{\textcolor{green}{Ein Brief Artur Schnitzlers}}}, das ich
               Ihnen hiemit gerne und mit vielem Dank für Ihre besondere Liebenswürdigkeit zusende.
               Auch mir ist bisher nicht bekannt geworden, dass jener \textcolor{green}{russische Artikel}\pwindex{?? [Journalist, der fiktives russisches Interview verantwortet] @\textsc{?? [Journalist, der fiktives russisches Interview verantwortet]}!?? [Fiktives Interview aus der Kriegszeit]@\strich\emph{?? [Fiktives Interview aus der Kriegszeit]}|pwv}{}\ledrightnote{{$\rightarrow$}\emph{\textcolor{green}{?? [Fiktives Interview aus der Kriegszeit]}}} den Weg nach anderen Ländern
               gefunden hätte. Die Existenz jenes \textcolor{green}{Artikels}\pwindex{?? [Journalist, der fiktives russisches Interview verantwortet] @\textsc{?? [Journalist, der fiktives russisches Interview verantwortet]}!?? [Fiktives Interview aus der Kriegszeit]@\strich\emph{?? [Fiktives Interview aus der Kriegszeit]}|pwv}{}\ledrightnote{{$\rightarrow$}\emph{\textcolor{green}{?? [Fiktives Interview aus der Kriegszeit]}}} oder erdichteten \textcolor{green}{Interviews}\pwindex{?? [Journalist, der fiktives russisches Interview verantwortet] @\textsc{?? [Journalist, der fiktives russisches Interview verantwortet]}!?? [Fiktives Interview aus der Kriegszeit]@\strich\emph{?? [Fiktives Interview aus der Kriegszeit]}|pwv}{}\ledrightnote{{$\rightarrow$}\emph{\textcolor{green}{?? [Fiktives Interview aus der Kriegszeit]}}} – ich weiss bis heute nicht, was es war – steht dennoch
               zweifellos fest und die \textcolor{pink}{russischen}\oindex{Russland@\textbf{Russland}|pw}{}\ledrightnote{\textcolor{pink}{Russland}}{ }\textcolor{blue}{Freunde}\pwindex{Vengerova, Isabella 1.\,3.\,1877 Minsk – 7.\,2.\,1956 New York City@\textsc{Vengerova, Isabella} (1.\,3.\,1877 Minsk – 7.\,2.\,1956 New York City), \emph{Musikpädagogin, Pianistin}|pwv}\pwindex{Moller, Alice 24.\,4.\,1871 Wien – Oktober 1962@\textsc{Moller, Alice} (24.\,4.\,1871 Wien – Oktober 1962), \emph{Kassierin}|pwv}{}\ledrightnote{{$\rightarrow$}\emph{\textcolor{blue}{Isabella Vengerova}}{\newline}{$\rightarrow$}\emph{\textcolor{blue}{Alice Moller}}}, die mich auf einem
               komplizierten Umweg davon unterrichtet haben, liessen mir überdies mitteilen, dass
               Versuche in ihren Kreisen die vollkommene Unmöglichkeit einer Authentizität jener mir
               zugeschriebenen Aeusserungen aus meinem bisher unbescholtenen literarischen
               Lebenswandel zu beweisen, an der allgemeinen Verbitterung und Verhetzung gescheitert
               sind. Wie schon in meiner \textcolor{green}{Erklärung}\pwindex{Schnitzler, Arthur 15. 5. 1862 Wien – 21. 10. 1931 ebd.@\textsc{Schnitzler, Arthur} (15. 5. 1862 Wien – 21. 10. 1931 ebd.), \emph{Schriftsteller, Mediziner}!Brief Artur Schnitzlers@\strich\emph{Ein Brief Artur Schnitzlers}|pw}{}\ledrightnote{\textcolor{green}{Ein Brief Artur Schnitzlers}} steht, ist es mir
               bisher nicht gelungen mir den Wortlaut jener gefälschten \textcolor{green}{Aeußerungen}\pwindex{?? [Journalist, der fiktives russisches Interview verantwortet] @\textsc{?? [Journalist, der fiktives russisches Interview verantwortet]}!?? [Fiktives Interview aus der Kriegszeit]@\strich\emph{?? [Fiktives Interview aus der Kriegszeit]}|pwv}{}\ledrightnote{{$\rightarrow$}\emph{\textcolor{green}{?? [Fiktives Interview aus der Kriegszeit]}}} zugänglich zu machen, der Sinn meiner
               Auslassungen sollte ungefähr nach jenem Blatt der folgende gewesen sein: dass ich \textcolor{blue}{Tolstoi}\pwindex{Tolstoi, Leo N. von 9.\,9.\,1828 Yasnaya Polyana – 20.\,11.\,1910 Lev Tolstoy@\textsc{Tolstoi, Leo N. von} (9.\,9.\,1828 Yasnaya Polyana – 20.\,11.\,1910 Lev Tolstoy), \emph{Schriftsteller, Schriftsteller, Krimiautor}|pw}{}\ledrightnote{\textcolor{blue}{Leo N. von Tolstoi}} als einen alten Faselhans bezeichne, von \textcolor{blue}{Maeterlinck}\pwindex{Maeterlinck, Maurice 29.\,8.\,1862 Gent – 6.\,5.\,1949 Nizza@\textsc{Maeterlinck, Maurice} (29.\,8.\,1862 Gent – 6.\,5.\,1949 Nizza), \emph{Schriftsteller}|pw}{}\ledrightnote{\textcolor{blue}{Maurice Maeterlinck}} behaupte, dass er seine Bauern schinde, von
               \textcolor{blue}{Anatole France}\pwindex{France, Anatole 16.\,4.\,1844 Paris – 12.\,10.\,1924 Saint-Cyr-sur-Loire@\textsc{France, Anatole} (16.\,4.\,1844 Paris – 12.\,10.\,1924 Saint-Cyr-sur-Loire), \emph{Schriftsteller}|pw}{}\ledrightnote{\textcolor{blue}{Anatole France}}, dass er mich irgendwie bestohlen
               habe, und dass ich endlich die Behauptung aufstellte, \textcolor{blue}{Hauptmann}\pwindex{Hauptmann, Gerhart 15.\,11.\,1862 Szczawno-Zdrój – 6.\,6.\,1946 Jagniątków@\textsc{Hauptmann, Gerhart} (15.\,11.\,1862 Szczawno-Zdrój – 6.\,6.\,1946 Jagniątków), \emph{Schriftsteller}|pw}{}\ledrightnote{\textcolor{blue}{Gerhart Hauptmann}} sei ein viel gröBerer Dichter als \textcolor{blue}{Shakespeare}\pwindex{Shakespeare, William 23.\,4.\,1564? Stratford-upon-Avon – 3.\,5.\,1616 ebd.@\textsc{Shakespeare, William} (23.\,4.\,1564? Stratford-upon-Avon – 3.\,5.\,1616 ebd.), \emph{Schauspieler, Dramatiker}|pw}{}\ledrightnote{\textcolor{blue}{William Shakespeare}}. Aus \textcolor{pink}{Russland}\oindex{Russland@\textbf{Russland}|pw}{}\ledrightnote{\textcolor{pink}{Russland}} kam auch das
               dringende Ersuchen an mich gegen diese Verleumdungen etwas zu unternehmen.\pend
           
\pstart
           Dass eine so törichte Geschichte mir den ersten Anlass geben würde eine persönliche
               Verbindung mit Ihnen anzuknüpfen hätten wir uns wohl Beide nicht träumen lassen. Aber
               da es sich nun einmal so fügt, will ich diese Gelegenheit gerne benützen, um Ihnen zu
               sagen, wie sehr ich Sie verehre und mit welchem Vergnügen, mit welcher wachsenden
               Freude ich Ihren wunderschönen \textcolor{green}{Jean Christophe}\pwindex{Rolland, Romain 29.\,1.\,1866 Clamecy – 30.\,12.\,1944 Vézelay@\textsc{Rolland, Romain} (29.\,1.\,1866 Clamecy – 30.\,12.\,1944 Vézelay), \emph{Schriftsteller}!Jean Christophe@\strich\emph{Jean Christophe}|pw}{}\ledrightnote{\textcolor{green}{Jean Christophe}} gelesen
               habe. Lassen Sie mich hoffen, dass eine Beziehung, die wenigstens von mir zu Ihnen
               innerlich längst bestanden, so seltsam sie auch in ihrem äusseren Umriss anheben mag,
               in jenen besseren Zeiten, die wir alle ersehnen und vielleicht auch noch früher,
               einen glücklichen Fortgang finde. Für heute aber seien Sie nur nochmals vielmals
               bedankt und herzlich gegrüsst von\pend
           \pstart Ihrem sehr ergebenen\pend{}\selectlanguage{ngerman}\endnumbering\briefempfaengerindex{, @\textsc{, }!zzz, @\emph{von  }!1914-12-141@{14. 12. 1914}|)be}\mylabel{L03884h}
\begin{anhang}
\end{anhang}\normalsize

\doendnotes{C}
\bigskip
\vfill

\clearpage

\footnotesize

\lohead{\textsc{register}}

% Definiere theindex-Environment komplett neu ohne reledmac
\makeatletter
\renewenvironment{theindex}{%
  \section*{\indexname}%
  \setlength{\parindent}{0pt}%
  \setlength{\parskip}{0pt plus 0.3pt}%
  \let\item\@idxitem
}{%
  \clearpage
}
\makeatother

\IfFileExists{\jobname-pw.ind}{\input{\jobname-pw.ind}}{}

\end{document}

      