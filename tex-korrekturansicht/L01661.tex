%% latex-korrekturansicht-vorspann.tex
%% Vorspann für die Korrekturansicht.
%% Lädt die gemeinsame Datei latex-vorspann.tex mit gesetztem Schalter.

\newif\ifkorrekturansicht
\korrekturansichttrue

\input{../tex-inputs/latex-vorspann}


               \section[Arthur Schnitzler: Widmungsexemplar Dämmerseelen für Hugo von Hofmannsthal, {[}7.?{]} 3. 1907]{ Arthur Schnitzler: Widmungsexemplar Dämmerseelen für Hugo von
               Hofmannsthal, {[}7.?{]} 3. 1907}\nopagebreak\mylabel{v}\rehead{ }\normalsize\beginnumbering\briefempfaengerindex{Hofmannsthal, Hugo von@\textsc{Hofmannsthal, Hugo von}!zzzSchnitzler, Arthur@\emph{von Arthur Schnitzler}!1907-03-072@{{[}7.?{]} 3. 1907}|(be} \toendnotes[C]{\smallbreak\pagebreak[2]} \Standort{FDH, FDH 3222.}
\physDesc{Widmung am Vorsatzblatt
\newline{}Handschrift: schwarze Tinte, deutsche Kurrent}\buchAbdrucke{\weitereDrucke{Hugo von Hofmannsthal: \emph{Bibliothek}. Hg. Ellen Ritter † in Zusammenarbeit mit Dalia Bukauskaité und Konrad
                        Heumann. Frankfurt am Main: \emph{S. Fischer} 2011, S. 603 (Sämtliche Werke. Kritische Ausgabe, XL).} }\toendnotes[C]{\smallbreak}\pstart
           \noindent{}{\pb}Meinem lieben Hugo\pend
           \pstart \spacefill\mbox{Arthur}\pend{}\pstart
           \textcolor{pink}{Wien}{}\ledrightnote{\textcolor{pink}{Wien}}{ }\label{K_L01661_1v}\edtext{März 907}{\lemma{\textnormal{\emph{März 907}}}\Cendnote{\textnormal{Am 7. 3. 1907 notiert sich \textcolor{blue}{Schnitzler} das Erscheinen. Am 8. 3. 1907 vom \emph{\textcolor{green}{Börsenblatt für den deutschen
                              Buchhandel}} als Neuerscheinung gemeldet}}}\label{K_L01661_1h}.\pend
           {\bigskip}\pstart
           \noindent{}\centering{}{\pb}\textcolor{gray}{\textbf{\textcolor{green}{Dämmerſeelen}{}\ledrightnote{\textcolor{green}{Dämmerseelen. Novellen}}}}\pend
           \pstart
           \noindent{}\centering{}\textcolor{gray}{\textbf{Novellen}}{\\}\textcolor{gray}{\textbf{von}}{\\}\textcolor{gray}{\textbf{Arthur Schnitzler}}\pend
           {\bigskip}\pstart
           \noindent{}\centering{}\textcolor{gray}{\textbf{\textcolor{brown}{S. Fiſcher, Verlag}{}\ledrightnote{\textcolor{brown}{S. Fischer Verlag}}, \textcolor{pink}{Berlin}{}\ledrightnote{\textcolor{pink}{Berlin}}}}\pend
           \pstart
           \noindent{}\centering{}\textcolor{gray}{\textbf{1907}}\pend
           \endnumbering\briefempfaengerindex{Hofmannsthal, Hugo von@\textsc{Hofmannsthal, Hugo von}!zzzSchnitzler, Arthur@\emph{von Arthur Schnitzler}!1907-03-072@{{[}7.?{]} 3. 1907}|)be}\mylabel{h}  \normalsize

\doendnotes{C}
\bigskip
\vfill

\clearpage

\footnotesize

\lohead{\textsc{register}}

% Definiere theindex-Environment komplett neu ohne reledmac
\makeatletter
\renewenvironment{theindex}{%
  \section*{\indexname}%
  \setlength{\parindent}{0pt}%
  \setlength{\parskip}{0pt plus 0.3pt}%
  \let\item\@idxitem
}{%
  \clearpage
}
\makeatother

\IfFileExists{\jobname-pw.ind}{\input{\jobname-pw.ind}}{}

\end{document}

      