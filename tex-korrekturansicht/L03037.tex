%% latex-korrekturansicht-vorspann.tex
%% Vorspann für die Korrekturansicht.
%% Lädt die gemeinsame Datei latex-vorspann.tex mit gesetztem Schalter.

\newif\ifkorrekturansicht
\korrekturansichttrue

\input{../tex-inputs/latex-vorspann}


\renewcommand{\erwaehntePersonen}{Personen: Felix Salten, Johann Schnitzler}
\renewcommand{\erwaehnteWerke}{Werke: Die Gesellschaft. Monatsschrift für Litteratur, Kunst und Sozialpolitik, Morgenandacht}
\section[Arthur Schnitzler an Felix Salten, {[}26. 10. 1893 – 2. 5. 1894?{]}]{Arthur Schnitzler an Felix Salten, {[}26. 10. 1893 – 2. 5. 1894?{]}}
\nopagebreak\mylabel{v}
\rehead{ }\normalsize\beginnumbering\briefempfaengerindex{Salten, Felix@\textsc{Salten, Felix}!zzzSchnitzler, Arthur@\emph{von Arthur Schnitzler}!1893-10-261@{{[}26. 10. 1893 –
                  2. 5. 1894?{]}}|(be}
\toendnotes[C]{\smallbreak\pagebreak[2]}\Standort{Wienbibliothek im Rathaus, ZPH 1681, 2.1.516.}
\physDesc{Brief, 1 Blatt, 2 Seiten (Briefpapier mit Trauerrand)
\newline{}Handschrift: Bleistift, deutsche Kurrent
\newline{}Ordnung: mit Bleistift von unbekannter Hand Blattzählung:
                                    »19« }\toendnotes[C]{\smallbreak}
\pstart{}{\pb}Lieber!\pend
\pstart
           Was ſind das für \label{K_L03037-2v}\edtext{Lächerlichkeiten}{\lemma{\textnormal{\emph{Lächerlichkeiten}}}\Cendnote{\textnormal{Das Korrespondenzstück ist undatiert. Durch
                  die Verwendung von Briefpapier mit Trauerrand lässt es sich in das Jahr nach dem
                  Tod des \textcolor{blue}{Vaters} am 2. 5. 1893 verorten.
                  Am 25. 10. 1893 trug
                     \textcolor{blue}{Schnitzler} das Gedicht in Gegenwart \textcolor{blue}{Salten}s vor, was zumindest als Indiz genommen
                  werden kann, dass das Schreiben danach abgefasst ist. Aus dem Zeitraum }}}\label{K_L03037-2h}? Bin ich ein grüner
               Oberſchwan? Bin ich ein verlobter Fähnrich, dem der Tiefſinn die Leuchter hinters
               Fenſter geſetzt hat? Oder hab ich gar die Gewohnheit, Sternſchnuppen in Cylinder
               aufzufangen? Beſſer iſt es ſchon, wenn Sie mich morgen zwiſchen {\pb}½ 6 und 6 aufſuchen.– \pend
           
\pstart
           Es wäre möglich, daſs ich Sie morgen im Laufe des Nachmittags aufſuche – kanns aber
               nicht verſprechen. \pend
           
\pstart
           Herzliche Grüße. Was Sie mir ſchrieben, »\label{K_L03037-1v}\edtext{\textcolor{green}{das iſt von einem böſen Wahn der
                  trügevolle Schimmer}{}\ledrightnote{{$\rightarrow$}\textcolor{green}{Morgenandacht}}}{\lemma{\textnormal{\emph{das … Schimmer}}}\Cendnote{\textnormal{In \textcolor{blue}{Schnitzler}s Gedicht \emph{\textcolor{green}{Morgenandacht}}
                  heißt es in der 8. Strophe: »Das war von einem holden Wahn / Der trügevolle
                     Schimmer«. (\emph{\textcolor{green}{Die Gesellschaft}}, Jg. 7, Bd. 1, H. 2,
                        Februar 1891, S. 190.)}}}\label{K_L03037-1h}.« \pend
           \pstart Ihr \spacefill\mbox{ArthSchn}\pend{}\endnumbering\briefempfaengerindex{Salten, Felix@\textsc{Salten, Felix}!zzzSchnitzler, Arthur@\emph{von Arthur Schnitzler}!1893-10-261@{{[}26. 10. 1893 –
                  2. 5. 1894?{]}}|)be}\mylabel{h}
\begin{anhang}
\end{anhang}\normalsize

\doendnotes{C}
\bigskip
\vfill

\clearpage

\footnotesize

\lohead{\textsc{register}}

% Definiere theindex-Environment komplett neu ohne reledmac
\makeatletter
\renewenvironment{theindex}{%
  \section*{\indexname}%
  \setlength{\parindent}{0pt}%
  \setlength{\parskip}{0pt plus 0.3pt}%
  \let\item\@idxitem
}{%
  \clearpage
}
\makeatother

\IfFileExists{\jobname-pw.ind}{\input{\jobname-pw.ind}}{}

\end{document}

      