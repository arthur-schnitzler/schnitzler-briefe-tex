%% latex-korrekturansicht-vorspann.tex
%% Vorspann für die Korrekturansicht.
%% Lädt die gemeinsame Datei latex-vorspann.tex mit gesetztem Schalter.

\newif\ifkorrekturansicht
\korrekturansichttrue

\input{../tex-inputs/latex-vorspann}


\renewcommand{\erwaehntePersonen}{Personen:  A. B., Paul Adam, Lou Andreas-Salomé, Richard Beer-Hofmann, Henri-Gabriel Ibels, Leopold Sonnemann, Oscar Wilde, Adolphe Léon Willette}
\renewcommand{\erwaehnteInstitutionen}{Institutionen: Frankfurter Zeitung}
\renewcommand{\erwaehnteOrte}{Orte: Bad Tölz, Frankfurt am Main, Japan, Kahlenberg, Kopenhagen, Paris, Wien, rue Feydeau}
\renewcommand{\erwaehnteWerke}{Werke: Die Kammer, Eine japanische Kaiserstadt, Firnißtag im Salon de Champs Elysées, Frankfurter Zeitung, Japan, La Revue blanche, Le Courrier français, Les Funérailles, L’amour s’amuse. Saynète, Pariser Malerei. (Der Salon der Champs Elysées.) [I]., Tannhäuser und der Sängerkrieg auf Wartburg, »L’Assaut malicieux«, »Tannhäuser« in Paris}
\section[Paul Goldmann an Arthur Schnitzler, 19. 5. {[}1895{]}]{Paul Goldmann an Arthur Schnitzler, 19. 5. {[}1895{]}}
\nopagebreak\mylabel{v}
\rehead{ }\normalsize\beginnumbering\briefempfaengerindex{Schnitzler, Arthur@\textsc{Schnitzler, Arthur}!zzzGoldmann, Paul@\emph{von Paul Goldmann}!1895-05-191@{19. 5. {[}1895{]}}|(be}
\toendnotes[C]{\smallbreak\pagebreak[2]}\Standort{DLA, A:Schnitzler, HS.NZ85.1.3165.}
\physDesc{Brief, 3 Blätter, 12 Seiten
\newline{}Handschrift: schwarze Tinte, deutsche Kurrent
\newline{}Schnitzler: 1) mit Bleistift das Jahr »95« vermerkt  2) mit rotem Buntstift eine Unterstreichung}\toendnotes[C]{\smallbreak}
\pstart
           \noindent{}{\pb}\textcolor{gray}{\textbf{\textbf{\textcolor{brown}{Frankfurter Zeitung}{}\ledrightnote{\textcolor{brown}{Frankfurter Zeitung}}}}}\pend
           
\pstart
           \textcolor{gray}{\textbf{(\textcolor{brown}{\begin{otherlanguage}{french}Gazette de Francfort\end{otherlanguage}}{}\ledrightnote{\textcolor{brown}{Frankfurter Zeitung}}). }}\pend
           
\pstart
           \textcolor{gray}{\textbf{\textbf{\begin{otherlanguage}{french}Fondateur M. \textcolor{blue}{L.
                              Sonnemann}{}\ledrightnote{\textcolor{blue}{Leopold Sonnemann}}\end{otherlanguage}.}}}\pend
           
\pstart
           \begin{otherlanguage}{french}\textcolor{gray}{\textbf{Journal politique, financier,}}\end{otherlanguage}\hfill \textsc{\textcolor{pink}{Paris}{}\ledrightnote{\textcolor{pink}{Paris}}}, 19. Mai.\pend
           
\pstart
           \begin{otherlanguage}{french}\textcolor{gray}{\textbf{commercial et littéraire.}}\end{otherlanguage}\pend
           
\pstart
           \begin{otherlanguage}{french}\textcolor{gray}{\textbf{\textbf{Paraissant trois fois par jour.}}}\end{otherlanguage}\pend
           
\pstart
           \begin{otherlanguage}{french}\textcolor{gray}{\textbf{\textbf{Bureau à \textcolor{pink}{Paris}{}\ledrightnote{\textcolor{pink}{Paris}}}}}\end{otherlanguage}\pend
           
\pstart
           \begin{otherlanguage}{french}\textcolor{gray}{\textbf{\textbf{\textcolor{pink}{24. Rue Feydeau}{}\ledrightnote{\textcolor{pink}{rue Feydeau}}.}}}\end{otherlanguage}\pend
           
\pstart\center{}Mein lieber Freund,\pend
\pstart
           Gewiß, gewiß – ſeit ich von \textcolor{pink}{Frankfurt}{}\ledrightnote{\textcolor{pink}{Frankfurt am Main}} zurück bin,
               liegt es mir ſchwer auf der Seele. Täglich will ich Dir ſchreiben. Aber ich habe
               unmenſchlich zu thun. \strikeout{Lie} Lieſt Du die »\textcolor{green}{Frankfurter Zeitung}{}\ledrightnote{\textcolor{green}{Frankfurter Zeitung}}« noch? Jeden Tag kannſt Du es
               ſehen: \label{K_L02735-5v}\edtext{\textcolor{green}{\textsc{Salon}}{}\ledrightnote{{$\rightarrow$}\textcolor{green}{Pariser Malerei. (Der Salon der Champs Elysées.) [I].}}}{\lemma{\textnormal{\emph{Salon}}}\Cendnote{\textnormal{\textcolor{blue}{Paul Goldmann}: \emph{\textcolor{green}{Pariser Malerei. (Der Salon der Champs Elysées.)}}. In:
                        \emph{\textcolor{green}{Frankfurter Zeitung}}, Jg. 39, Nr. 135,
                        16. 5. 1895, Erstes Morgenblatt, S. 1–2; Nr. 136,
                        17. 5. 1895, Erstes Morgenblatt, S. 1–2. Bereits am
                  Monatsanfang hatte er zur Ausstellung geschrieben: \textcolor{blue}{G.} [=\textcolor{blue}{Paul Goldmann}]: \emph{\textcolor{green}{Firnißtag im Salon de
                        Champs Elysées}}. In: \emph{\textcolor{green}{Frankfurter
                        Zeitung}}, Jg. 39, Nr. 121, 2. 5. 1895, Zweites Morgenblatt,
                     S. 1.}}}\label{K_L02735-5h}, \label{K_L02735-123v}\edtext{\textcolor{green}{Kammer}{}\ledrightnote{\textcolor{green}{Die Kammer}}}{\lemma{\textnormal{\emph{Kammer}}}\Cendnote{\textnormal{\textcolor{blue}{G.} [=\textcolor{blue}{Paul Goldmann}]: \emph{\textcolor{green}{Die Kammer}}. In:
                        \emph{\textcolor{green}{Frankfurter Zeitung}}, Jg. 39, Nr. 135,
                        16. 5. 1895, Drittes Morgenblatt, S. 1.}}}\label{K_L02735-123h}, \label{K_L02735-234v}\edtext{\textcolor{green}{Tannhäuſer}{}\ledrightnote{\textcolor{green}{Tannhäuser und der Sängerkrieg auf Wartburg}}}{\lemma{\textnormal{\emph{Tannhäuſer}}}\Cendnote{\textnormal{\textcolor{blue}{G.} [=\textcolor{blue}{Paul Goldmann}]: \emph{\textcolor{green}{»Tannhäuser« in
                        Paris}}. In: \emph{\textcolor{green}{Frankfurter Zeitung}},
                     Jg. 39, Nr. 131, 12. 5. 1895, Erstes Morgenblatt,
                  S. 1–2.}}}\label{K_L02735-234h}, \label{K_L02735-456v}\edtext{\textcolor{pink}{Japan}{}\ledrightnote{\textcolor{pink}{Japan}}}{\lemma{\textnormal{\emph{Japan}}}\Cendnote{\textnormal{Worauf sich \textcolor{blue}{Goldmann} hier bezog, ist unklar. Mögliche Erklärungen: Es
                  handelt sich um ein Feuilleton, das länger zurück lag, beispielsweise: \textcolor{blue}{A. B.}: \emph{\textcolor{green}{Eine japanische Kaiserstadt}}. In: \emph{\textcolor{green}{Frankfurter Zeitung}}, Jg. 39, Nr. 111, 22. 4. 1895,
                     Morgenblatt, S. 1–2. (Dagegen spricht das Namenskürzel, für das es bei
                     \textcolor{blue}{Goldmann} keinen Beleg gibt.) Oder es
                  könnte sich um die kleine, nicht namentlich gekennzeichnete \textcolor{green}{Meldung} aus \textcolor{pink}{Japan} handeln, die am 18. 5. 1895 erschien und
                  die möglicherweise ohne Quellenangabe aus einer französischen Zeitung entnommen
                  wurde (Nr. 137, Erstes Morgenblatt, S. 1). Weiters wäre denkbar, dass
                  ein Text nur in einem Teil der \textcolor{green}{Ausgabe} enthalten war.}}}\label{K_L02735-456h}{ }\textsc{etc. etc.} Und dann ſchreibe ich Dir nicht, weil ich endlich
               das Bedürfniß {\pb}fühle, Dir den \uline{großen} Brief zu ſchreiben und Dir gar ſoviel zu ſagen haben:
               Innerliches, nichts äußerlich Neues. Nun muß ich aber doch \strikeout{\textcolor{gray}{m}it} noch einmal den kurzen Brief abſenden. Heut Sonntag{ }Nachmittag wollte ich Dir ausführlich ſchreiben. Ich blieb eigens
               deshalb zu Hauſe. Da kam wieder dieſe verfluchte Tagesarbeit dazwiſchen. Nun iſt es
                  ſieben Uhr, und es bleibt mir nur Zeit zu einem {\pb}raſchen Gruß.\pend
           
\pstart
           Gruß und Dank! Für ſoviel Treues und Liebes habe ich Dir zu danken. \textcolor{blue}{Eure}{}\ledrightnote{{$\rightarrow$}\textcolor{blue}{Lou Andreas-Salomé}{\newline}{$\rightarrow$}\textcolor{blue}{Richard Beer-Hofmann}} Karte vom \label{K_L02735-1v}\edtext{\textsc{\textcolor{pink}{Kahlenberge}{}\ledrightnote{\textcolor{pink}{Kahlenberg}}}}{\lemma{\textnormal{\emph{Kahlenberge}}}\Cendnote{\textnormal{Am 8. 5. 1895 waren \textcolor{blue}{Richard Beer-Hofmann}, \textcolor{blue}{Lou Andreas-Salomé} und \textcolor{blue}{Schnitzler}
                  am \textcolor{pink}{Kahlenberg} und dürften eine Postkarte an
                     \textcolor{blue}{Goldmann} geschickt haben.}}}\label{K_L02735-1h}, die
               Photographie, Deine lieben Briefe haben mich ſo innig erfreut! Es thut mir ſo wohl,
               daß Ihr und Du beſonders an mich denkſt, daß ich mich ein wenig bei Euch weiß. Dieſe
               kleinen Gaben bewegen mich ſehr – ſie rühren mich (wenn das nicht {\pb}ſo ein dummes Wort wäre). Dank, tauſend Dank!\pend
           
\pstart
           Daß \textcolor{blue}{Ihr}{}\ledrightnote{{$\rightarrow$}\textcolor{blue}{Richard Beer-Hofmann}} mit Frau \textsc{\textcolor{blue}{Andreas}{}\ledrightnote{\textcolor{blue}{Lou Andreas-Salomé}}} Freund geworden ſeid, iſt ſo gekommen, wie ich es erwartet. Sie gehört zu uns.
               Denn ſie iſt ein lieber, feiner und ehrlicher \textcolor{blue}{Menſch}{}\ledrightnote{{$\rightarrow$}\textcolor{blue}{Lou Andreas-Salomé}}. Und ich weiß aus Erfahrung, wie wohl der Umgang mit
               dieſer \textcolor{blue}{Frau}{}\ledrightnote{{$\rightarrow$}\textcolor{blue}{Lou Andreas-Salomé}} thut!
               Klimatiſche Wirkung – das ſagſt Du ſehr gut. Aber nun iſt Eines zu beachten: {\pb}Dieſe \textcolor{blue}{Frau}{}\ledrightnote{{$\rightarrow$}\textcolor{blue}{Lou Andreas-Salomé}}, die ſo ganz unperſönlich wirkt – manchmal ſo wie
               abſoluter Verſtand und abſolute Wahrheit – hat eine heiße Sehnſucht, aus dieſer
               Verſtandes-Sphäre herauszukommen. Sie will \uline{Weib} ſein,
               will lieben und geliebt werden. Und wenn ſie aus dem Abſoluten ins Menſchliche
               niederſteigen wollte – in den Tag hinein, wie \strikeout{das} die
               erſte beſte kleine {\pb}\label{K_L02735-55v}\edtext{Nähterin}{\lemma{\textnormal{\emph{Nähterin}}}\Cendnote{\textnormal{veraltet: Näherin}}}\label{K_L02735-55h} – wenn ich Weibliche\substVorne{}\textsuperscript{\textcolor{gray}{r}}\substDazwischen{}s\substHinten{} an ihr merkte – \label{K_L02735-2v}\edtext{\begin{otherlanguage}{french}\textsc{des douceurs, des chatteries}\end{otherlanguage}}{\lemma{\textnormal{\emph{des … chatteries}}}\Cendnote{\textnormal{französisch: Schmeicheleien,
                  Zärtlichkeiten}}}\label{K_L02735-2h} – Weibliches, das ſo gar nicht zu ihr gehört (obwohl ſie
               auch nicht unangenehm männlich ist) – dann war ſie \strikeout{im}
               mir immer verhaßt. Jawohl, ein nervöſer Haß! Gegen dieſe \textcolor{blue}{Frau}{}\ledrightnote{{$\rightarrow$}\textcolor{blue}{Lou Andreas-Salomé}}, die mir ſo viel Gutes gethan, wie
               Wenige auf \strikeout{a} der Welt! Die an mich geglaubt! Die ſich
               die Mühe genommen hat, an {\pb}mich zu glauben! Es iſt
               abſcheulich! Aber zu Zeiten haßte ich ſie, ich muß es Dir ſagen. In einer gewiſſen
               Entfernung \strikeout{war ſ} hatte ich eine große Verehrung für
               ſie. Je näher ſie mir kam, umſo weniger ſympathiſch wurde ſie mir.\pend
           
\pstart
           Nun wohl, die \textcolor{blue}{Frau}{}\ledrightnote{{$\rightarrow$}\textcolor{blue}{Lou Andreas-Salomé}} weiß mit
               ihrem unfehlbaren Verſtande ſehr wohl, daß ſie dieſe unperſönliche Wirkung ausübt.
               »Klimatischer {\pb}Einfluß«, man kann es nicht beſſer
               ſagen. \textcolor{blue}{Sie}{}\ledrightnote{{$\rightarrow$}\textcolor{blue}{Lou Andreas-Salomé}} will aber
               perſönlich wirken – als Weib wirken. Und das iſt nun die Tragödie ihres Lebens.\pend
           
\pstart
           Daß ſie ſich zu \textcolor{blue}{Euch}{}\ledrightnote{{$\rightarrow$}\textcolor{blue}{Richard Beer-Hofmann}}
               hingezogen fühlt, verſtehe ich ſehr gut. Sie hat ſich für mich intereſſirt, weil ich
               ein Typus war, den ſie noch nicht kannte: warm, melancholiſch, weich und \strikeout{\textcolor{pink}{wien
                  }{}\ledrightnote{\textcolor{pink}{Wien}}\textcolor{gray}{e}} überhaupt \textcolor{pink}{wien}{}\ledrightnote{\textcolor{pink}{Wien}}eriſch. Und nun findet ſie
               bei \textcolor{blue}{Euch}{}\ledrightnote{{$\rightarrow$}\textcolor{blue}{Richard Beer-Hofmann}} dieſen {\pb}\strikeout{Tys} Typus in ſeiner Vervollkommung, während ich doch
               nur Anſätze dazu habe. Und gerade das iſt es, wonach \textcolor{blue}{ſie}{}\ledrightnote{{$\rightarrow$}\textcolor{blue}{Lou Andreas-Salomé}} ſich ſehnt: dieſer Gemüthston, in dem
               ſoviel warmes Leben iſt........\pend
           
\pstart
           Nach \label{K_L02735-3v}\edtext{\textsc{\textcolor{pink}{Kopenhagen}{}\ledrightnote{\textcolor{pink}{Kopenhagen}}}}{\lemma{\textnormal{\emph{Kopenhagen}}}\Cendnote{\textnormal{Die Reise fand erst ein Jahr später als
                  geplant, im August 1896, statt. \textcolor{blue}{Goldmann} kam ebenfalls mit.}}}\label{K_L02735-3h} kann ich nicht kommen.
               Ich muß im Auguſt nach \textsc{\textcolor{pink}{Tölz}{}\ledrightnote{\textcolor{pink}{Bad Tölz}}}, zur Kur. Werde ich Dich ſehen? Du wirſt {\pb}Dich
               natürlich in Deinen Plänen durch mich nicht ſtören laſſen. \strikeout{\textcolor{gray}{×}\-\textcolor{gray}{×}\-\textcolor{gray}{×}\-\textcolor{gray}{×}}{ }\textsc{\textcolor{pink}{Kopenhagen}{}\ledrightnote{\textcolor{pink}{Kopenhagen}}} mußt und ſollſt Du ſehen. Aber vielleicht ließe ſich doch eine Vereinbarung
               treffen für die Rückreiſe.\pend
           
\pstart
           Ich ſende Dir anbei wieder einige Artikel. Beſonders in der »\textsc{\begin{otherlanguage}{french}\textcolor{green}{Revue Blanche}{}\ledrightnote{\textcolor{green}{La Revue blanche}}\end{otherlanguage}}« mache ich Dich aufmerkſam auf die \label{K_L02735-4v}\edtext{\textcolor{green}{Vertheidigung}{}\ledrightnote{{$\rightarrow$}\textcolor{green}{»L’Assaut malicieux«}}}{\lemma{\textnormal{\emph{Vertheidigung}}}\Cendnote{\textnormal{\textcolor{blue}{Paul Adam}: \emph{\textcolor{green}{»L’Assaut malicieux«}}. In: \emph{\textcolor{green}{La Revue blanche}}, Jg. 8, Nr. 47, 15. 5. 1895, 15. 5. 1895, S. 458–462.}}}\label{K_L02735-4h} des \textsc{\textcolor{blue}{Oscar Wilde}{}\ledrightnote{\textcolor{blue}{Oscar Wilde}}} durch \textsc{\textcolor{blue}{Paul Adam}{}\ledrightnote{\textcolor{blue}{Paul Adam}}}. Ferner ſende ich Dir ein {\pb}dummes \textcolor{green}{Stück}{}\ledrightnote{{$\rightarrow$}\textcolor{green}{L’amour s’amuse. Saynète}} »\begin{otherlanguage}{french}\textsc{\textcolor{green}{L’amour s’amuse}{}\ledrightnote{\textcolor{green}{L’amour s’amuse. Saynète}}}\end{otherlanguage}«, das \uline{nicht} zu leſen iſt. Aber es iſt von
                  \textsc{\uline{\textcolor{blue}{Ibels}{}\ledrightnote{\textcolor{blue}{Henri-Gabriel Ibels}}}} illustrirt, einem neuen \textcolor{blue}{Künſtler}{}\ledrightnote{{$\rightarrow$}\textcolor{blue}{Henri-Gabriel Ibels}}, deſſen ſeltſame Art Dich intereſſiren wird. Den »\begin{otherlanguage}{french}\textsc{\textcolor{green}{Courrier Francais}{}\ledrightnote{\textcolor{green}{Le Courrier français}}}\end{otherlanguage}« ſende ich Dir nur wegen der \label{K_L02735-6v}\edtext{\textcolor{green}{Zeichnung}{}\ledrightnote{{$\rightarrow$}\textcolor{green}{Les Funérailles}} von \textsc{\textcolor{blue}{Willette}{}\ledrightnote{\textcolor{blue}{Adolphe Léon Willette}}} in der Mitte des \textcolor{green}{Heft}{}\ledrightnote{{$\rightarrow$}\textcolor{green}{Le Courrier français}}es}{\lemma{\textnormal{\emph{Zeichnung … Heftes}}}\Cendnote{\textnormal{Vermutlich
                  handelte es sich um \emph{\textcolor{green}{Les Funérailles}}, auf
                  einer Doppelseite in der Mitte des \textcolor{green}{Heft}es vom 12. 5. 1895
                  erschienen.}}}\label{K_L02735-6h}. Endlich mein \textcolor{green}{\textsc{Salon}-Feuilleton}{}\ledrightnote{{$\rightarrow$}\textcolor{green}{Pariser Malerei. (Der Salon der Champs Elysées.) [I].}}. Ich habe es hauptſächlich für
               Dich geſchrieben und, ſowenig es mir gefällt, möchte {\pb}ich doch daß Du es lieſt.\pend
           
\pstart
           Grüß’ Dich Gott, mein lieber Freund! Grüße \textsc{\textcolor{blue}{Richard}{}\ledrightnote{\textcolor{blue}{Richard Beer-Hofmann}}} und die Frau \textsc{\textcolor{blue}{Andreas}{}\ledrightnote{\textcolor{blue}{Lou Andreas-Salomé}}}.\pend
           
\pstart
           Schreib’ mir bald!\pend
           
\pstart
           Und nächſtens bekommſt Du den großen Brief!\pend
           
\pstart
           Ich umarme {\\[\baselineskip]}Dich von Herzen {\\[\baselineskip]}Dein {\\[\baselineskip]}\spacefill\mbox{Paul Goldmann.}\pend
           \leftskip=0em{}\endnumbering\briefempfaengerindex{Schnitzler, Arthur@\textsc{Schnitzler, Arthur}!zzzGoldmann, Paul@\emph{von Paul Goldmann}!1895-05-191@{19. 5. {[}1895{]}}|)be}\mylabel{h}  \normalsize

\doendnotes{C}
\bigskip
\vfill

\clearpage

\footnotesize

\lohead{\textsc{register}}

% Definiere theindex-Environment komplett neu ohne reledmac
\makeatletter
\renewenvironment{theindex}{%
  \section*{\indexname}%
  \setlength{\parindent}{0pt}%
  \setlength{\parskip}{0pt plus 0.3pt}%
  \let\item\@idxitem
}{%
  \clearpage
}
\makeatother

\IfFileExists{\jobname-pw.ind}{\input{\jobname-pw.ind}}{}

\end{document}

      