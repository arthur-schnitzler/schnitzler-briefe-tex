%% latex-korrekturansicht-vorspann.tex
%% Vorspann für die Korrekturansicht.
%% Lädt die gemeinsame Datei latex-vorspann.tex mit gesetztem Schalter.

\newif\ifkorrekturansicht
\korrekturansichttrue

\input{../tex-inputs/latex-vorspann}


\renewcommand{\erwaehntePersonen}{Personen: Ferdinand Bronner, Ida Falk, Johann Wolfgang von Goethe, Ludwig Grann, Hugo von Hofmannsthal, Guy de Maupassant, Friedrich von Oppeln-Bronikowski, Felix Salten, Siegfried Trebitsch}
\renewcommand{\erwaehnteInstitutionen}{Institutionen: Emil Goldschmidt Verlag}
\renewcommand{\erwaehnteOrte}{Orte: Bad Ischl, Berlin, Frankfurt am Main, Hotel und Pension Rudolfshöhe (Leopold Petter), München, Nürnberg, Wien, Wiesbaden}
\renewcommand{\erwaehnteWerke}{Werke: Aus dem Nachlasse von Maupassant. Eine Leidenschaft, Begräbnis, Das Bergwerk zu Falun, Das Manhard-Zimmer, Der Hinterbliebene, Der Hinterbliebene. Kurze Novellen, Der Schleier der Beatrice. Schauspiel in fünf Akten, Eine Goethe-Enquête, Familie Wawroch. Ein österreichisches Drama in vier Akten, Fernen, Flucht, Heldentod, Lebenszeit, Schöne Seelen. Komödie in einem Akt, Sedan, Vater Milon und andere Erzählungen. Neue Novellen aus dem litterarischen Nachlaß, Wiener Allgemeine Montags-Zeitung, Wiener Allgemeine Rundschau}
\section[ Arthur Schnitzler an Felix Salten, 4. 9. 1899]{Arthur Schnitzler an Felix Salten, 4. 9. 1899}
\nopagebreak\mylabel{v}
\rehead{ }\normalsize\beginnumbering\briefempfaengerindex{Salten, Felix@\textsc{Salten, Felix}!zzzSchnitzler, Arthur@\emph{von Arthur Schnitzler}!1899-09-041@{4. 9. 1899}|(be}
\toendnotes[C]{\smallbreak\pagebreak[2]}\Standort{Wienbibliothek im Rathaus, ZPH 1681, 2.1.516.}
\physDesc{Brief, 2 Blätter, 8 Seiten, 1693 Zeichen
\newline{}Handschrift: Bleistift, deutsche Kurrent
\newline{}Ordnung: mit Bleistift von unbekannter Hand Nummerierung der Doppelseiten des Konvoluts:
                                    »69«–»72« }
\buchAbdrucke{\weitereDrucke{Arthur Schnitzler: \emph{Briefe 1875–1912}. Hg. Therese Nickl und Heinrich Schnitzler. Frankfurt am Main: \emph{S. Fischer} 1981, S. 375–376.} }\toendnotes[C]{\smallbreak}
\pstart
           \raggedleft{}{\pb}\textcolor{pink}{Ischl, Rudolfshöhe}{}\ledrightnote{\textcolor{pink}{Hotel und Pension Rudolfshöhe (Leopold Petter)}}{ }4/9 99.\pend
           
\pstart
           lieber Freund, ich will Ihnen vor allem ſagen, dſs mir
               nicht nur »\textcolor{green}{Flucht}{}\ledrightnote{\textcolor{green}{Flucht}}«, ſondern auch das \textcolor{green}{\textsc{Manhard}zi{\geminationm}er}{}\ledrightnote{\textcolor{green}{Das Manhard-Zimmer}} noch
               beſſer \label{K_L02967-1v}\edtext{gefallen}{\lemma{\textnormal{\emph{gefallen}}}\Cendnote{\textnormal{siehe Felix Salten an Arthur Schnitzler, [29. 8. 1899]}}}\label{K_L02967-1h} haben, als nach dem erſten Leſen. Ich zweifle nicht, dſs Ihre \textcolor{green}{Novelletten}{}\ledrightnote{{$\rightarrow$}\textcolor{green}{Flucht}{\newline}{$\rightarrow$}\textcolor{green}{Fernen}{\newline}{$\rightarrow$}\textcolor{green}{Sedan}{\newline}{$\rightarrow$}\textcolor{green}{Lebenszeit}{\newline}{$\rightarrow$}\textcolor{green}{Der Hinterbliebene}{\newline}{$\rightarrow$}\textcolor{green}{Das Manhard-Zimmer}{\newline}{$\rightarrow$}\textcolor{green}{Begräbnis}{\newline}{$\rightarrow$}\textcolor{green}{Heldentod}} ein hübſches
               { }\textcolor{green}{Buch}{}\ledrightnote{{$\rightarrow$}\textcolor{green}{Der Hinterbliebene. Kurze Novellen}} gäben, möchte aber von
               einem entgiltigen {\pb}Urtheil über die Wirkung
               als ganzes, \uline{alle} Sachen auf einmal, womöglich in der
               von Ihnen gewählten Reihenfolge leſen. Herausgeben unbedingt, ſag ich ſchon heute,
               und womöglich zugleich mit dem \label{K_L02967-2v}\edtext{\textcolor{green}{Stück}{}\ledrightnote{{$\rightarrow$}\textcolor{green}{Schöne Seelen. Komödie in einem Akt}}}{\lemma{\textnormal{\emph{Stück}}}\Cendnote{\textnormal{siehe Felix Salten an Arthur Schnitzler, 9. 10. 1899}}}\label{K_L02967-2h} herauskommen. – In der {\pb}\label{K_L02967-3v}\edtext{\textcolor{green}{Zeitung}{}\ledrightnote{{$\rightarrow$}\textcolor{green}{Wiener Allgemeine Montags-Zeitung}}}{\lemma{\textnormal{\emph{Zeitung}}}\Cendnote{\textnormal{Von der ersten Ausgabe weg, die am 3. 7. 1899 erschienen war, betreute \textcolor{blue}{Salten} die Rubrik »\emph{\textcolor{green}{Wiener
                     Allgemeine Rundschau}}« der wöchentlich erscheinenenden \emph{\textcolor{green}{Wiener Allgemeinen Montags-Zeitung}}. Das \textcolor{green}{Blatt} wurde Mitte Dezember 1899 eingestellt.}}}\label{K_L02967-3h} findet ſich viel leſenswerthes;
               natürlich iſt es Ihnen aus Gründen, die nicht in Ihnen liegen, unmöglich, das
               Anſtrebenswerthe daraus zu machen. Glänzend hab ich Ihre \label{K_L02967-4v}\edtext{\textcolor{green}{\textcolor{blue}{Goethe}{}\ledrightnote{\textcolor{blue}{Johann Wolfgang von Goethe}}späße}{}\ledrightnote{{$\rightarrow$}\textcolor{green}{Eine Goethe-Enquête}}}{\lemma{\textnormal{\emph{Goethespäße}}}\Cendnote{\textnormal{[\textcolor{blue}{Felix Salten}]: \emph{\textcolor{green}{Eine Goethe-Enquête}}. In: \emph{\textcolor{green}{Wiener Allgemeine Montags-Zeitung}}, 28. 8. 1899, S. 3. Die nicht gezeichnete \textcolor{green}{Umfrage} ist deutlich als
                  Satire erkennbar (»indem wir eine Anzahl hervorragender Persönlichkeiten im
                     Geiste um ihre Meinung befragt«) und bringt erfundene Aussagen
                  von 17 Prominenten zu \textcolor{blue}{Goethe}.}}}\label{K_L02967-4h}
               gefunden. Können Sie mir die \label{K_L02967-5v}\edtext{\textcolor{green}{Familie{ }{\pb}\textsc{Wawroch}}{}\ledrightnote{\textcolor{green}{Familie Wawroch. Ein österreichisches Drama in vier Akten}} von \textcolor{blue}{Adamus}{}\ledrightnote{\textcolor{blue}{Ferdinand Bronner}}}{\lemma{\textnormal{\emph{Familie … Adamus}}}\Cendnote{\textnormal{\textcolor{blue}{Ferdinand Bronner}s \emph{\textcolor{green}{Familie Wawroch. Ein österreichisches Drama in vier Akten}}
                  war 1899 unter dem Pseudonym \textcolor{blue}{Franz Adamus} erschienen. Eine Lektüre durch \textcolor{blue}{Schnitzler} ist nicht nachweisbar.}}}\label{K_L02967-5h}
               ſchicken? (Ich glaub mich zu erinnern dſs Sie ſie haben.) – Die Überſetzungen von \textsc{\textcolor{blue}{S. Tr.}{}\ledrightnote{\textcolor{blue}{Siegfried Trebitsch}}} find ich ſchlecht. – Das \label{K_L02967-6v}\edtext{raſche
               Abdrucken des neuen \textcolor{blue}{\textcolor{green}{Maupassant}{}\ledrightnote{{$\rightarrow$}\textcolor{green}{Aus dem Nachlasse von Maupassant. Eine Leidenschaft}}}{}\ledrightnote{\textcolor{blue}{Guy de Maupassant}}}{\lemma{\textnormal{\emph{raſche … Maupassant}}}\Cendnote{\textnormal{\textcolor{blue}{Guy de Maupassant}: \emph{\textcolor{green}{Aus dem Nachlasse von Maupassant. Eine Leidenschaft}}.
                     In: \emph{\textcolor{green}{Wiener Allgemeine Montags-Zeitung}},
                        28. 8. 1899, S. 2–4. Der Text ist der
                     Ende Juli im Verlag \emph{\textcolor{brown}{Emil Goldschmidt}} erschienenen Buchausgabe \emph{\textcolor{green}{Vater Milon und andere Erzählungen. Neue Novellen aus dem
                     litterarischen Nachlaß}} entnommen. Die in der \textcolor{green}{Buchausgabe}, nicht aber im \textcolor{green}{Abdruck} gezeichnete
                  Übersetzung stammt von \textcolor{blue}{Friedrich von
                     Oppeln-Bronikowski}.}}}\label{K_L02967-6h} zeigt den rechten Weg auf dieſem Gebiet. –\pend
           
\pstart
           Ich bleibe noch bis etwa 10. oder 9.{ }\label{K_L02967-7v}\edtext{\textcolor{pink}{hier}{}\ledrightnote{{$\rightarrow$}\textcolor{pink}{Bad Ischl}}}{\lemma{\textnormal{\emph{hier}}}\Cendnote{\textnormal{\textcolor{blue}{Schnitzler} reiste am 12. 9. 1899 von \textcolor{pink}{Ischl} nach \textcolor{pink}{München} ab. Von dort reiste er am 16. 9. 1899 weiter nach \textcolor{pink}{Nürnberg}, dann am 19. 9. 1899 weiter nach \textcolor{pink}{Frankfurt am Main} und am 24. 9. 1899 nach \textcolor{pink}{Wiesbaden}. Zwischen 4. 10. 1899 und 11. 10. 1899 war er in \textcolor{pink}{Berlin}. Am
                     12. 10. 1899
                  kehrte er nach \textcolor{pink}{Wien} zurück.}}}\label{K_L02967-7h}. Dann {\pb}vorerſt \textcolor{pink}{München}{}\ledrightnote{\textcolor{pink}{München}}, dann?– 20, 22. werd ich in \textcolor{pink}{Berlin}{}\ledrightnote{\textcolor{pink}{Berlin}} ſein.
                  Wahrſcheinli{[}ch{]} iſt mein \label{K_L02967-8v}\edtext{\textcolor{green}{Stück}{}\ledrightnote{{$\rightarrow$}\textcolor{green}{Der Schleier der Beatrice. Schauspiel in fünf Akten}}}{\lemma{\textnormal{\emph{Stück}}}\Cendnote{\textnormal{\textcolor{blue}{Schnitzler} schloss \emph{\textcolor{green}{Der Schleier der Beatrice}} am 9. 9. 1899 vorläufig
                  ab.}}}\label{K_L02967-8h} bis dahin fertig. Die Führung und mancherlei ausgeſprochnes dürfte gut
               ſein; doch fühl ich oft, wie die Kraft des Ausdrucks {\pb}aus dem Gehirn (denn da ſcheint ſie mir zu
               ſein) nicht in den Bleiſtift will. –\pend
           
\pstart
           Arbeiten bleibt endlich doch das einzige. Sonſt iſts im Weſentlichen i{\geminationm}er gleich traurig. – Auch \textcolor{blue}{Hugo}{}\ledrightnote{\textcolor{blue}{Hugo von Hofmannsthal}} arbeitet {\pb}\textcolor{pink}{hier}{}\ledrightnote{{$\rightarrow$}\textcolor{pink}{Bad Ischl}} an einem \label{K_L02967-9v}\edtext{neuen Stück}{\lemma{\textnormal{\emph{neuen Stück}}}\Cendnote{\textnormal{Am 31. 8. 1899 hatte \textcolor{blue}{Hugo von
                     Hofmannsthal}{ }\textcolor{blue}{Schnitzler} bereits zwei Akte aus \emph{\textcolor{green}{Das Bergwerk zu Falun}} vorgelesen.}}}\label{K_L02967-9h} (\textcolor{green}{Bergwerk von Falun}{}\ledrightnote{\textcolor{green}{Das Bergwerk zu Falun}} – Sie wiſſens ja ſchon.) Auch
               ihm hat \textcolor{green}{Flucht}{}\ledrightnote{\textcolor{green}{Flucht}} gut gefallen \strikeout{(} (das \textcolor{green}{andre}{}\ledrightnote{{$\rightarrow$}\textcolor{green}{Das Manhard-Zimmer}{\newline}{$\rightarrow$}\textcolor{green}{Sedan}} hat er noch nicht geleſen.) –\pend
           
\pstart
           Heute traf ich Frau \textsc{\textcolor{blue}{Ida {\pb}F.}{}\ledrightnote{\textcolor{blue}{Ida Falk}}} – \label{K_L02967-10v}\edtext{Verlobt}{\lemma{\textnormal{\emph{Verlobt}}}\Cendnote{\textnormal{\textcolor{blue}{Ida Falk}, ehemalige Geliebte sowohl von \textcolor{blue}{Schnitzler} als auch von \textcolor{blue}{Salten}, hatte sich mit \textcolor{blue}{Ludwig Grann} verlobt, vgl. A. S.: \emph{Tagebuch}, 23. 10. 1899.}}}\label{K_L02967-10h}. \substVorne{}\textsuperscript{\textcolor{gray}{×}\-\textcolor{gray}{×}\-\textcolor{gray}{×}\-\textcolor{gray}{×}\-\textcolor{gray}{×}\-\textcolor{gray}{×}{ }\textcolor{gray}{×}\-\textcolor{gray}{×}\-\textcolor{gray}{×}\-\textcolor{gray}{×}}\substDazwischen{}Man ſoll nie Namen ſchreiben\substHinten{}. – Komiſcherweiſe iſt \introOben{}hier\introOben{} eine vorübergehende
               Verbindg zwiſchen mir und einer abſoluten Wiederholung jenes Typus eingetreten. –\pend
           
\pstart
           Herzlichſt Ihr {\\[\baselineskip]}\spacefill\mbox{A. S.}\pend
           \leftskip=0em{}\endnumbering\briefempfaengerindex{Salten, Felix@\textsc{Salten, Felix}!zzzSchnitzler, Arthur@\emph{von Arthur Schnitzler}!1899-09-041@{4. 9. 1899}|)be}\mylabel{h}  \normalsize

\doendnotes{C}
\bigskip
\vfill

\clearpage

\footnotesize

\lohead{\textsc{register}}

% Definiere theindex-Environment komplett neu ohne reledmac
\makeatletter
\renewenvironment{theindex}{%
  \section*{\indexname}%
  \setlength{\parindent}{0pt}%
  \setlength{\parskip}{0pt plus 0.3pt}%
  \let\item\@idxitem
}{%
  \clearpage
}
\makeatother

\IfFileExists{\jobname-pw.ind}{\input{\jobname-pw.ind}}{}

\end{document}

      