%% latex-korrekturansicht-vorspann.tex
%% Vorspann für die Korrekturansicht.
%% Lädt die gemeinsame Datei latex-vorspann.tex mit gesetztem Schalter.

\newif\ifkorrekturansicht
\korrekturansichttrue

\input{../tex-inputs/latex-vorspann}


               \section[Therese Rie-Andro an Arthur Schnitzler, 23. 7. 1923]{ Therese Rie-Andro an Arthur Schnitzler, 23. 7. 1923}\nopagebreak\mylabel{v}\rehead{ }\normalsize\beginnumbering\briefempfaengerindex{Schnitzler, Arthur@\textsc{Schnitzler, Arthur}!zzzRie, Therese@\emph{von Therese Rie}!1923-07-231@{23. 7. 1923}|(be} \toendnotes[C]{\smallbreak\pagebreak[2]} \Standort{CUL, Schnitzler, B 658.}
\physDesc{Brief, 1 Blatt, 2 Seiten
\newline{}Handschrift: blaue Tinte, lateinische Kurrent
\newline{}Schnitzler: 1) mit Bleistift beschriftet: »\textsc{Rie}« 2) mit rotem Buntstift vier Unterstreichungen}\toendnotes[C]{\smallbreak}\pstart
           \raggedleft{}{\pb}\textcolor{pink}{Bernried/Starnbergerseee}{}\ledrightnote{\textcolor{pink}{Bernried}},
                     23. 7. 23. \pend
           \pstart
           \raggedleft{}\textcolor{pink}{Altwirt.}{}\ledrightnote{\textcolor{pink}{Hotel Seeblick}}\hspace*{1.5em}\textcolor{pink}{Oberbayern}{}\ledrightnote{\textcolor{pink}{Oberbayern}}\pend
           \pstart{}Verehrter Herr Doktor,\pend\pstart
           Es iſt wirklich lieb von Ihnen, daſs Sie von meiner Literatur noch immer nicht genug
               haben; aber leider bin ich nun schon zu Ende, es exiſtieren bloß noch ein paar
               Jugendsünden und verſtreute oder ungedruckte Sachen. So schmeichelhaft es iſt – ich
               hab’ nichts mehr! – Aber \uline{nicht} schmeichelhaft, lieber
               Herr Doktor, iſt die Annahme, ich nähme meine eigenen Briefe auf die Reise mit! Das
               läßt auf düſtere Erfahrungen schließen, die Sie mit Schreibweibern gemacht haben
               müssen! Da tun Sie mir sehr leid! – Ist es nicht tausend mal schöner und wichtiger,
               zu schw{\geminationm}en\textcolor{gray}{,} zu rudern und unter alten
               Bäumen zu liegen? Ich meine, der Dichter der \textcolor{green}{Lebendigen
                  Stunden}{}\ledrightnote{\textcolor{green}{Lebendige Stunden}} gibt mir da Recht!\pend
           \pstart
           Aber da fällt mir doch ein, daſs ich noch was \textcolor{green}{Schönes}{}\ledrightnote{→\textcolor{green}{Musikalische Reise ins Land der Vergangenheit}}{ }\introOben{}daheim\introOben{} habe: von \textcolor{blue}{Romain
                  Rolland}{}\ledrightnote{\textcolor{blue}{Romain Rolland}} (von mir übersetzt.) Das beko{\geminationm}en Sie.
               Für die Reise freilich nicht mehr rechtzeitig, da ich vor dem 15. Auguſt
               kaum in \textcolor{pink}{Wien}{}\ledrightnote{\textcolor{pink}{Wien}} bin und Sie wol schon fort. Aber
               hoffentlich gefällt es Ihnen auch später noch. Denn es dreht sich nur um die Muſik
               und das iſt doch das Einzige, was im Leben in der Stadt \strikeout{(auch)} noch \uline{wirklich} iſt.\pend
           \pstart
           Daß Sie mir ein Buch von sich geben wollen, iſt sehr lieb von Ihnen. Ihre \textcolor{green}{gesa{\geminationm}elten Werke}{}\ledrightnote{\textcolor{green}{Gesammelte Werke}}
                  (\label{K_L02575-2v}\edtext{bis zum \textcolor{green}{Weiten Land}{}\ledrightnote{\textcolor{green}{Das weite Land. Tragikomödie in fünf Akten}}}{\lemma{\textnormal{\emph{bis zum Weiten Land}}}\Cendnote{\textnormal{Sie besitzt die Ausgabe von
                     1912 ohne die beiden Ergänzungsbände von
               1922.}}}\label{K_L02575-2h}) besitze ich natürlich; ich gestehe {\pb}Ihnen eine große Zuneigung zu \textcolor{green}{Fink und Fliederbuſch}{}\ledrightnote{\textcolor{green}{Fink und Fliederbusch. Komödie in drei Akten}}, gerade weil dieses Stück alle
               wolgeölten Gemüter einmal in Aufruhr versetzt hat; aber \textcolor{green}{Beate}{}\ledrightnote{\textcolor{green}{Frau Beate und ihr Sohn. Novelle}} oder \textcolor{green}{Casanova}{}\ledrightnote{\textcolor{green}{Casanovas Heimfahrt}} liebe ich nicht minder
               – also bitte, suchen \uline{Sie} mir etwas aus, dann habe ich
               zu der Freude des Empfangens auch noch die Ihrer Auswahl.\pend
           \pstart
           Die beiden \label{K_L02575-1v}\edtext{Ausschnitte}{\lemma{\textnormal{\emph{Ausschnitte}}}\Cendnote{\textnormal{nicht überliefert}}}\label{K_L02575-1h}, die ich einlege,
               sind aus einer \textcolor{pink}{New-York}{}\ledrightnote{\textcolor{pink}{New York City}}er Revue: der eine enthält
               zwei Worte über den \textcolor{green}{Casanova}{}\ledrightnote{\textcolor{green}{Casanovas Heimfahrt}}. Der andre hat mit
               Kunſt überhaupt nichts zu tun, iſt aber menſchlich so packend und traurig, daſs er
               Sie vielleicht intereſſirt; auch ein »\textcolor{green}{Bernhardi}{}\ledrightnote{→\textcolor{green}{Professor Bernhardi. Komödie in fünf Akten}}« hätte drüber nichts zu lachen! \pend
           \pstart
           Und nun wünsche ich Ihnen schöne, helle, frohe So{\geminationm}ertage!\pend
           \pstart
           Ihre{\\[\baselineskip]}\spacefill\mbox{Therese Rie.}\pend
           \leftskip=0em{}\endnumbering\briefempfaengerindex{Schnitzler, Arthur@\textsc{Schnitzler, Arthur}!zzzRie, Therese@\emph{von Therese Rie}!1923-07-231@{23. 7. 1923}|)be}\mylabel{h}  \normalsize

\doendnotes{C}
\bigskip
\vfill

\clearpage

\footnotesize

\lohead{\textsc{register}}

% Definiere theindex-Environment komplett neu ohne reledmac
\makeatletter
\renewenvironment{theindex}{%
  \section*{\indexname}%
  \setlength{\parindent}{0pt}%
  \setlength{\parskip}{0pt plus 0.3pt}%
  \let\item\@idxitem
}{%
  \clearpage
}
\makeatother

\IfFileExists{\jobname-pw.ind}{\input{\jobname-pw.ind}}{}

\end{document}

      