%% latex-korrekturansicht-vorspann.tex
%% Vorspann für die Korrekturansicht.
%% Lädt die gemeinsame Datei latex-vorspann.tex mit gesetztem Schalter.

\newif\ifkorrekturansicht
\korrekturansichttrue

\input{../tex-inputs/latex-vorspann}


               \section[Hermann Bahr an Arthur Schnitzler, 18. 6. 1909]{ Hermann Bahr an Arthur Schnitzler, 18. 6. 1909}\nopagebreak\mylabel{v}\rehead{ }\normalsize\beginnumbering\briefempfaengerindex{Schnitzler, Arthur@\textsc{Schnitzler, Arthur}!zzzBahr, Hermann@\emph{von Hermann Bahr}!1909-06-182@{18. 6. 1909}|(be} \toendnotes[C]{\smallbreak\pagebreak[2]} \Standort{CUL, Schnitzler, B 5b.}
\physDesc{Bildpostkarte
\newline{}Handschrift: schwarze Tinte, deutsche Kurrent\newline{}Versand: 1) Stempel: »\nobreak{}\oindex{Santa Maria Elisabetta@\textbf{Santa Maria Elisabetta}, \emph{Bezirk (A.BZK)}|pwk}S. Elisabetta\nobreak{}«.  2) Stempel: »\nobreak{}\oindex{Santa Maria Elisabetta@\textbf{Santa Maria Elisabetta}, \emph{Bezirk (A.BZK)}|pwk}S. Elisabetta Lido, 18 6 08\nobreak{}«. 
\newline{}Schnitzler: mit Bleistift ergänzt »\textsc{Bahr}« \newline{}Ordnung: mit Bleistift von unbekannter Hand nummeriert:
                              »157« }\buchAbdrucke{\weitereDrucke{Hermann Bahr, Arthur Schnitzler: \emph{Briefwechsel, Aufzeichnungen, Dokumente (1891–1931)}. Hg. Kurt Ifkovits und Martin Anton Müller. Göttingen: \emph{Wallstein} 2018, S. 417.} }\toendnotes[C]{\smallbreak}\pstart{}{\pb}\textsc{D\textsuperscript{r} Artur Schnitzler}\pend{}\pstart{}\textsc{\textcolor{pink}{XVIII Spöttelgasse 7}{}\ledrightnote{\textcolor{pink}{Edmund-Weiß-Gasse}}}\pend{}\pstart{}\textsc{\textcolor{pink}{Vienna}{}\ledrightnote{\textcolor{pink}{Wien}}}\pend{}\pstart{}\textsc{\textcolor{pink}{Austria}{}\ledrightnote{\textcolor{pink}{Österreich}}}\pend{}{\bigskip}\pstart
           \noindent{}\centering{}\textcolor{gray}{\textbf{{\pb}\textcolor{pink}{Venezia. Rio dei Mendicanti e fondamenta}{}\ledrightnote{\textcolor{pink}{Venedig}}}}\pend
           \pstart
           \raggedleft{}{\pb}18. 6. 09\pend
           \pstart
           Lieber Artur! Hoffentlich iſt Dir »\textcolor{green}{Drut}{}\ledrightnote{\textcolor{green}{Drut}}« sowie mein »\label{K_L01846_1v}\edtext{\textcolor{green}{Tagebuch}{}\ledrightnote{\textcolor{green}{Tagebuch [Berlin: Paul Cassirer]}}}{\lemma{\textnormal{\emph{Tagebuch}}}\Cendnote{\textnormal{Das heißt die erste gesammelte Buchausgabe
                  der Kolumnen: \textcolor{blue}{Hermann Bahr}: \emph{\textcolor{green}{Tagebuch}}. Berlin: \emph{\textcolor{brown}{Paul Cassirer}}{ }1909.}}}\label{K_L01846_1h}« richtig zugekommen. – Wir ſind ſeit drei Wochen hier und gehen
                  \textcolor{gray}{nun} nächſte Woche nach \textcolor{pink}{Bayreuth}{}\ledrightnote{\textcolor{pink}{Bayreuth}}. – Grüß Deine verehrte liebe \textcolor{blue}{Frau}{}\ledrightnote{→\textcolor{blue}{Olga Schnitzler}} und habt einen ſchönen Sommer!\pend
           \pstart
           Herzlichſt{\\[\baselineskip]}Dein alter{\\[\baselineskip]}\spacefill\mbox{HermannBahr}\pend
           \leftskip=0em{}\endnumbering\briefempfaengerindex{Schnitzler, Arthur@\textsc{Schnitzler, Arthur}!zzzBahr, Hermann@\emph{von Hermann Bahr}!1909-06-182@{18. 6. 1909}|)be}\mylabel{h}  \normalsize

\doendnotes{C}
\bigskip
\vfill

\clearpage

\footnotesize

\lohead{\textsc{register}}

% Definiere theindex-Environment komplett neu ohne reledmac
\makeatletter
\renewenvironment{theindex}{%
  \section*{\indexname}%
  \setlength{\parindent}{0pt}%
  \setlength{\parskip}{0pt plus 0.3pt}%
  \let\item\@idxitem
}{%
  \clearpage
}
\makeatother

\IfFileExists{\jobname-pw.ind}{\input{\jobname-pw.ind}}{}

\end{document}

      