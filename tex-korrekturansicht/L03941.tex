%% latex-korrekturansicht-vorspann.tex
%% Vorspann für die Korrekturansicht.
%% Lädt die gemeinsame Datei latex-vorspann.tex mit gesetztem Schalter.

\newif\ifkorrekturansicht
\korrekturansichttrue

\input{../tex-inputs/latex-vorspann}


\section[Arthur Schnitzler an Theodor Herzl, 12.–14. 11. 1892]{L03941 Arthur Schnitzler an Theodor Herzl, 12.–14. 11. 1892}
\nopagebreak\mylabel{L03941v}
\rehead{ }\normalsize\beginnumbering\briefempfaengerindex{, @\textsc{, }!zzz, @\emph{von  }!1892-11-142@{12.–14. 11. 1892}|(be}
\toendnotes[C]{\smallbreak\pagebreak[2]}\Standort{Jerusalem, Central Zionist Archives, H1\1924-3.}
\physDesc{Brief, 2 Blätter, 8 Seiten, 3161 Zeichen
\newline{}Handschrift: schwarze Tinte, deutsche Kurrent
\newline{}Ordnung: mit Bleistift von unbekannter Hand innerhalb das Konvoluts paginiert:
                                    »11«–»14« }
\buchAbdrucke{\weitereDrucke{1) \emph{Unveröffentlichtes aus Arthur Schnitzlers Nachlaß.} In: \emph{Neue Zürcher Zeitung. Beilage Literatur und Kunst}, Nr. 91/92, 9. 1. 1966, S. 4–5.} \weitereDrucke{2) Arthur Schnitzler: \emph{Briefe 1875–1912}. Herausgegeben von Therese Nickl und Heinrich Schnitzler. Frankfurt am Main: \emph{S. Fischer} 1981, S. 142–143.} }\toendnotes[C]{\smallbreak}
\pstart
           \raggedleft{}{\pb}\uline{\textcolor{pink}{Wien}\oindex{Wien@\textbf{Wien}, \emph{Verwaltungsgebiet}|pw}{}\ledrightnote{\textcolor{pink}{Wien}}, 12. November 892.}\pend
           
\pstart{}Verehrteſter Freund,\pend\vspace{0.5em}
\pstart
           zuerſt will ich Ihnen für Ihre \label{K_L03941-1v}\edtext{liebenswürdigen Worte}{\lemma{\textnormal{\emph{liebenswürdigen Worte}}}\Cendnote{\textnormal{Theodor Herzl an Arthur Schnitzler, 10. [11.?] 1892.}}}\label{K_L03941-1} herzlich danken, u. da{\geminationn}
               gleich ſagen, wer \textcolor{blue}{\textsc{Loris}}\pwindex{Hofmannsthal, Hugo von 1.\,2.\,1874 Wien – 15.\,7.\,1929 Rodaun@\textsc{Hofmannsthal, Hugo von} (1.\,2.\,1874 Wien – 15.\,7.\,1929 Rodaun), \emph{Schriftsteller}|pw}{}\ledrightnote{\textcolor{blue}{Hugo von Hofmannsthal}} ist. Räthſelhaft, daſs
               Sie es von \textcolor{blue}{Goldma{\geminationn}}\pwindex{Goldmann, Paul 31.\,1.\,1865 Breslau – 25.\,9.\,1935 Wien@\textsc{Goldmann, Paul} (31.\,1.\,1865 Breslau – 25.\,9.\,1935 Wien), \emph{Schriftsteller, Journalist}|pw}{}\ledrightnote{\textcolor{blue}{Paul Goldmann}} nicht wiſſen. Ich ſelber
               bin es leider nicht. Erſtens wäre ich dann um 12 Jahre jünger und zweitens hätte ich
                  »\textcolor{green}{Geſtern}\pwindex{Gestern. Dramatische Studie in einem Akt in Versen@\emph{Gestern. Dramatische Studie in einem Akt in Versen}|pw}{}\ledrightnote{\textcolor{green}{Gestern. Dramatische Studie in einem Akt in Versen}}« geſchrieben, den ſchönſten Einakter
               in Verſen, der ſeit ſehr, ſehr langer Zeit in deutſcher Sprache erſchienen iſt. Von
               dieſem merk{\pb}würdigen Achtzehnjährigen wird noch ſehr viel geſprochen werden. We{\geminationn} Sie
               ſchon die \textcolor{green}{Einleitungsverſe}\pwindex{Prolog [zum Anatol]@\emph{Prolog [zum Anatol]}|pwv}{}\ledrightnote{{$\rightarrow$}\emph{\textcolor{green}{Prolog [zum Anatol]}}} zum
                  \textcolor{green}{Anatol}\pwindex{Schnitzler, Arthur 15. 5. 1862 Wien – 21. 10. 1931 ebd.@\textsc{Schnitzler, Arthur} (15. 5. 1862 Wien – 21. 10. 1931 ebd.), \emph{Schriftsteller, Mediziner}!Anatol@\strich\emph{Anatol}|pw}{}\ledrightnote{\textcolor{green}{Anatol}} »zum küſſen« finden, ſo will ich Sie
               vor den unzüchtigen Gedanken warnen, die in Ihnen beim Genuſs ſeiner andern Sachen
               aufſteigen könnten. In Wirklichkeit heißt der Herr \textcolor{blue}{Hugo von Hofmannsthal}\pwindex{Hofmannsthal, Hugo von 1.\,2.\,1874 Wien – 15.\,7.\,1929 Rodaun@\textsc{Hofmannsthal, Hugo von} (1.\,2.\,1874 Wien – 15.\,7.\,1929 Rodaun), \emph{Schriftsteller}|pw}{}\ledrightnote{\textcolor{blue}{Hugo von Hofmannsthal}}, hat im Juli maturiert und studiert Jus an der \textcolor{brown}{Wr. Univerſität}\orgindex{Universität Wien@Universität Wien|pw}{}\ledrightnote{\textcolor{brown}{Universität Wien}}. Sie wiſſen ja, verehrteſter, wie
               wenig {\pb}wörtlich das zu nehmen iſt. Wenn es geſtattet ist, ſeiner Biographie
               vorzugreifen, ſo will ich Ihnen auch mittheilen, daſs ich heute Abend nach der \textcolor{violet}{\textsc{Première} von \textcolor{green}{Musotte}\pwindex{Maupassant, Guy de 5.\,8.\,1850 Tourville-sur-Arques – 7.\,7.\,1893 Paris@\textsc{Maupassant, Guy de} (5.\,8.\,1850 Tourville-sur-Arques – 7.\,7.\,1893 Paris), \emph{Schriftsteller}!Musotte@\strich\emph{Musotte}|pw}\pwindex{Normand, Jacques 25.\,11.\,1848 Paris – 1931 ebd.@\textsc{Normand, Jacques} (25.\,11.\,1848 Paris – 1931 ebd.), \emph{Schriftsteller}!Musotte@\strich\emph{Musotte}|pw}{}\ledrightnote{\textcolor{green}{Musotte}}}\eventindex{Volkstheater@\textbf{Volkstheater}!Premiere von Musotte, 12.11.1892@Premiere von Musotte, 12.11.1892|pw}{}\ledrightnote{\textcolor{violet}{Premiere von Musotte, 12.11.1892}} mit ihm ſoupiren und ihm von Ihrem freundlichen Intereſſe erzählen will. Im
               übrigen, fragen Sie doch \label{K_L03941-2v}\edtext{\textcolor{blue}{Goldma{\geminationn}}\pwindex{Goldmann, Paul 31.\,1.\,1865 Breslau – 25.\,9.\,1935 Wien@\textsc{Goldmann, Paul} (31.\,1.\,1865 Breslau – 25.\,9.\,1935 Wien), \emph{Schriftsteller, Journalist}|pw}{}\ledrightnote{\textcolor{blue}{Paul Goldmann}} nach ihm; – er hat ihn ja
                  entdeckt}{\lemma{\textnormal{\emph{Goldmann … entdeckt}}}\Cendnote{\textnormal{Vgl. Hugo von Hofmannsthal an Arthur Schnitzler, 17. 5. [1896].}}}\label{K_L03941-2}! –\pend
           
\pstart
           – Von \textcolor{pink}{Wiener}\oindex{Wien@\textbf{Wien}, \emph{Verwaltungsgebiet}|pw}{}\ledrightnote{\textcolor{pink}{Wien}} Kunſt ſoll ich Ihnen was berichten? –
               Nun, die literariſche Bewegung äußert ſich darin, daß im \textcolor{brown}{Wiedener {\pb}Theater}\orgindex{Theater an der Wien@Theater an der Wien|pw}{}\ledrightnote{\textcolor{brown}{Theater an der Wien}} oder \textcolor{brown}{Carltheater}\orgindex{Carl-Theater@Carl-Theater|pw}{}\ledrightnote{\textcolor{brown}{Carl-Theater}}{ }\textsc{Couplets} gegen den Naturalismus geſungen werden (»brutal–!«
               »Skandal!«), daſs es keine Verleger, keine neuen Stücke, dagegen ſehr viele
               Kaffeehäuſer gibt, in denen alle Literaten, denen Vormittags nichts eingefallen iſt,
               Nachmittag ihre Gedanken austauſchen. Sitzen zwei zuſa{\geminationm}en, ſo ne{\geminationn}t man ſie eine
                  \label{K_L03941-3v}\edtext{\textsc{Clique}}{\lemma{\textnormal{\emph{Clique}}}\Cendnote{\textnormal{Vgl. A. S.: \emph{Tagebuch}, 9. 10. 1891.}}}\label{K_L03941-3} – und
               ſitzen gar drei zuſa{\geminationm}en, – ſo ſind ſie es {\pb}wirklich. Man glaubt weder an ſich, noch
               an die andern – und hat großentheils Recht. – Ihr
                  \textcolor{green}{Feuilleton}\pwindex{Herzl, Theodor 2.\,5.\,1860 Budapest – 3.\,7.\,1904 Edlach@\textsc{Herzl, Theodor} (2.\,5.\,1860 Budapest – 3.\,7.\,1904 Edlach), \emph{Schriftsteller, Journalist}!Kaffeehaus der »neuen Richtung«@\strich\emph{Kaffeehaus der »neuen Richtung«}|pwv}{}\ledrightnote{{$\rightarrow$}\emph{\textcolor{green}{Kaffeehaus der »neuen Richtung«}}} von dazumal
               fällt mir ein: \label{K_L03941-4v}\edtext{\textcolor{green}{Kaffeehaus der neuen Richtung}\pwindex{Herzl, Theodor 2.\,5.\,1860 Budapest – 3.\,7.\,1904 Edlach@\textsc{Herzl, Theodor} (2.\,5.\,1860 Budapest – 3.\,7.\,1904 Edlach), \emph{Schriftsteller, Journalist}!Kaffeehaus der »neuen Richtung«@\strich\emph{Kaffeehaus der »neuen Richtung«}|pw}{}\ledrightnote{\textcolor{green}{Kaffeehaus der »neuen Richtung«}}}{\lemma{\textnormal{\emph{Kaffeehaus … Richtung}}}\Cendnote{\textnormal{\textcolor{blue}{Theodor Herzl}\pwindex{Herzl, Theodor 2.\,5.\,1860 Budapest – 3.\,7.\,1904 Edlach@\textsc{Herzl, Theodor} (2.\,5.\,1860 Budapest – 3.\,7.\,1904 Edlach), \emph{Schriftsteller, Journalist}|pwk}: \emph{\textcolor{green}{Das Kaffeehaus der »neuen Richtung«}\pwindex{Herzl, Theodor 2.\,5.\,1860 Budapest – 3.\,7.\,1904 Edlach@\textsc{Herzl, Theodor} (2.\,5.\,1860 Budapest – 3.\,7.\,1904 Edlach), \emph{Schriftsteller, Journalist}!Kaffeehaus der »neuen Richtung«@\strich\emph{Kaffeehaus der »neuen Richtung«}|pwk}}. In: \emph{\textcolor{green}{Wiener Allgemeine Zeitung}\pwindex{Wiener Allgemeine Zeitung@\emph{Wiener Allgemeine Zeitung}|pwk}}, Nr. 783,
                        4. 5. 1882, Morgenblatt, S. 1–4.}}}\label{K_L03941-4} hieß es,
               nicht? – wenn Sie mir gelegentlich dasſelbe ſchicken wollten (Sie haben es doch wohl)
               freute es mich ſehr. Und noch nach einem andern Werk gelüſtet es mich wieder; das
               iſt der \textcolor{green}{Tabarin}\pwindex{Herzl, Theodor 2.\,5.\,1860 Budapest – 3.\,7.\,1904 Edlach@\textsc{Herzl, Theodor} (2.\,5.\,1860 Budapest – 3.\,7.\,1904 Edlach), \emph{Schriftsteller, Journalist}!Tabarin. Schauspiel in einem Act. Frei nach Catulle Mendès@\strich\emph{Tabarin. Schauspiel in einem Act. Frei nach Catulle Mendès}|pw}{}\ledrightnote{\textcolor{green}{Tabarin. Schauspiel in einem Act. Frei nach Catulle Mendès}}. Nun aber will ich noch mit
               einer ganz beſonderen Bitte heraus (die {\pb}einleitenden \substVorne{}\textsuperscript{\textcolor{gray}{F}}\substDazwischen{}P\substHinten{}hraſen ſchenken Sie mir ja) ich
               möchte ſehr gern diejenigen Ihrer Stücke leſen, auf die Sie ſelbſt was halten u. die
               \uline{nicht} aufgeführt worden sind. – Sie würden meinem literariſchen u perſönlichen
               Intereſſe in gleicher Weiſe durch Berückſichtigg dieſes Erſuchens
               entgegenko{\geminationm}en. \textcolor{gray}{–}\pend
           
\pstart
           – Ihre Schlußpointe zu den \textcolor{green}{Weihnachtseinkäufen}\pwindex{Schnitzler, Arthur 15. 5. 1862 Wien – 21. 10. 1931 ebd.@\textsc{Schnitzler, Arthur} (15. 5. 1862 Wien – 21. 10. 1931 ebd.), \emph{Schriftsteller, Mediziner}!Weihnachts-Einkäufe@\strich\emph{Weihnachts-Einkäufe}|pw}{}\ledrightnote{\textcolor{green}{Weihnachts-Einkäufe}}
               gefällt mir vorzüglich; nur glaub’ ich wär ſie aus der einen Scene ſchwierig
               herauszuentwickeln. Es wäre überhaupt {\pb}was andres; in Ihrer Pointe liegt ganz einfach
               ein ſehr reizendes Luſt- oder vielleicht gar Schauſpiel verſteckt, welches zu
               ſchreiben Sie höflichſt gebeten werden. – Neugierig bin ich, ob Sie eins von den
               Dingen bühnenwirkſam finden werden. –\pend
           \selectlanguage{ngerman}\vspace{1em}
\pstart
           \raggedleft{}14. 11.\pend
           \vspace{0.5em}
\pstart
           Ich wurde neulich unterbrochen, u. ko{\geminationm}e erſt heute zum Abschluſs meines Briefes\pend
           
\pstart
           Laſſen Sie mich Ihnen alſo nur noch einmal ſagen, wie ſehr mich Ihre Freundlichkeit
               und Antheil{\pb}nahme ehrt und wie es mich freuen würde, bald wieder was von Ihnen zu
               hören. Sie haben mir nun \label{K_L03941-5v}\edtext{zwei
                  Briefe}{\lemma{\textnormal{\emph{zwei
                  Briefe}}}\Cendnote{\textnormal{29. 7. 1892, 10. [11.?] 1892.}}}\label{K_L03941-5} über mich geſchrieben; ich darf nun wohl
               einen über Sie erwarten?\pend
           \pstart Mit herzlichen Grüßen Ihr ſehr ergebner \spacefill\mbox{Arthur Schnitzler}\pend{}
\pstart
           \noindent{}\textcolor{pink}{I Grillparzerstrasse 7}\oindex{Wien@\textbf{Wien}!I., Innere Stadt@\textbf{I., Innere Stadt}!Wohnung und Ordination Arthur Schnitzler Grillparzerstraße 7/3. Stock@\textbf{Wohnung und Ordination Arthur Schnitzler Grillparzerstraße 7/3. Stock}, \emph{Ordination}|pw}{}\ledrightnote{\textcolor{pink}{Wohnung und Ordination Arthur Schnitzler Grillparzerstraße 7/3. Stock}}.\pend
           \selectlanguage{ngerman}\endnumbering\briefempfaengerindex{, @\textsc{, }!zzz, @\emph{von  }!2@{12.–14. 11. 1892}|)be}\mylabel{L03941h}
\begin{anhang}
\end{anhang}\normalsize

\doendnotes{C}
\bigskip
\vfill

\clearpage

\footnotesize

\lohead{\textsc{register}}

% Definiere theindex-Environment komplett neu ohne reledmac
\makeatletter
\renewenvironment{theindex}{%
  \section*{\indexname}%
  \setlength{\parindent}{0pt}%
  \setlength{\parskip}{0pt plus 0.3pt}%
  \let\item\@idxitem
}{%
  \clearpage
}
\makeatother

\IfFileExists{\jobname-pw.ind}{\input{\jobname-pw.ind}}{}

\end{document}

      