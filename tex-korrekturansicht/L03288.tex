%% latex-korrekturansicht-vorspann.tex
%% Vorspann für die Korrekturansicht.
%% Lädt die gemeinsame Datei latex-vorspann.tex mit gesetztem Schalter.

\newif\ifkorrekturansicht
\korrekturansichttrue

\input{../tex-inputs/latex-vorspann}


\renewcommand{\erwaehntePersonen}{Personen: Georg Hirschfeld, Ottilie Salten}
\renewcommand{\erwaehnteInstitutionen}{Institutionen: Deutsches Theater Berlin}
\renewcommand{\erwaehnteOrte}{Orte: Berlin, Pötzleinsdorf, Wien, XIII., Hietzing}
\renewcommand{\erwaehnteWerke}{Werke: Der grüne Kakadu – Paracelsus – Die Gefährtin. Drei Einakter}
\section[ Felix Salten an Arthur Schnitzler, 28. 4. 1899]{Felix Salten an Arthur Schnitzler, 28. 4. 1899}
\nopagebreak\mylabel{v}
\rehead{ }\normalsize\beginnumbering\briefempfaengerindex{Schnitzler, Arthur@\textsc{Schnitzler, Arthur}!zzzSalten, Felix@\emph{von Felix Salten}!1899-04-282@{28. 4. 1899}|(be}
\toendnotes[C]{\smallbreak\pagebreak[2]}\Standort{CUL, Schnitzler, B 89, A 2.}
\physDesc{Brief, 1 Blatt, 1 Seite, 697 Zeichen
\newline{}Handschrift: schwarze Tinte, lateinische Kurrent
\newline{}Ordnung: mit Bleistift von unbekannter Hand nummeriert: »112« }\toendnotes[C]{\smallbreak}
\pstart
           \raggedleft{}{\pb}\textcolor{pink}{Hietzing}{}\ledrightnote{\textcolor{pink}{XIII., Hietzing}}, 28. April 99\pend
           
\pstart
           Lieber Freund, leider war ich in den letzten Tagen wieder durch
               vielerlei ernste Angelegenheiten so gehetzt, dass ich nicht zu Ihnen konnte. Auch
               meine \textcolor{pink}{Berlin}{}\ledrightnote{\textcolor{pink}{Berlin}}er Reise, die ich so gerne gemacht
               hätte, musste unterbleiben, weil die \label{K_L03288-1v}\edtext{Geschichte mit \textcolor{blue}{Otti}{}\ledrightnote{\textcolor{blue}{Ottilie Salten}}}{\lemma{\textnormal{\emph{Geschichte mit Otti}}}\Cendnote{\textnormal{Bezug unklar}}}\label{K_L03288-1h} noch immer zu keinem
               Abschluß gekommen ist. Sie leidet entsetzlich unter der großen wie unter den vielen
               kleinen Gemeinheiten, welche ihr angethan werden. \textcolor{blue}{Hirschfeld}{}\ledrightnote{\textcolor{blue}{Georg Hirschfeld}} ist, wie Sie wissen werden, in \textcolor{pink}{Hietzing}{}\ledrightnote{\textcolor{pink}{XIII., Hietzing}} und wohnt gleich neben mir. Sonst sehe ich Niemanden. Bitte,
               vielleicht schreiben Sie mir: wie es Ihnen geht, und wie Ihre \label{K_L03288-2v}\edtext{\textcolor{green}{Prémiere}{}\ledrightnote{{$\rightarrow$}\textcolor{green}{Der grüne Kakadu – Paracelsus – Die Gefährtin. Drei Einakter}}}{\lemma{\textnormal{\emph{Prémiere}}}\Cendnote{\textnormal{\textcolor{blue}{Schnitzler} weilte in \textcolor{pink}{Berlin}, um bei den Proben für die Premiere seines
                  Einakterzyklus’ \emph{\textcolor{green}{Der grüne Kakadu – Paracelsus –
                     Die Gefährtin}} am 29. 4. 1899 am \emph{\textcolor{brown}{Deutschen Theater}}
                  teilzunehmen. Er kehrte am 3. 5. 1899 nach \textcolor{pink}{Wien} zurück und
                  sah \textcolor{blue}{Salten} nachweislich am 11. 5. 1899
                  wieder.}}}\label{K_L03288-2h} ausgefallen ist, wann Sie wiederkommen, und wann wir uns sehen.\pend
           
\pstart
           Sehr herzlich Ihr treuer {\\[\baselineskip]}\spacefill\mbox{Felix Salten}\pend
           \leftskip=0em{}\endnumbering\briefempfaengerindex{Schnitzler, Arthur@\textsc{Schnitzler, Arthur}!zzzSalten, Felix@\emph{von Felix Salten}!1899-04-282@{28. 4. 1899}|)be}\mylabel{h}  \normalsize

\doendnotes{C}
\bigskip
\vfill

\clearpage

\footnotesize

\lohead{\textsc{register}}

% Definiere theindex-Environment komplett neu ohne reledmac
\makeatletter
\renewenvironment{theindex}{%
  \section*{\indexname}%
  \setlength{\parindent}{0pt}%
  \setlength{\parskip}{0pt plus 0.3pt}%
  \let\item\@idxitem
}{%
  \clearpage
}
\makeatother

\IfFileExists{\jobname-pw.ind}{\input{\jobname-pw.ind}}{}

\end{document}

      