%% latex-korrekturansicht-vorspann.tex
%% Vorspann für die Korrekturansicht.
%% Lädt die gemeinsame Datei latex-vorspann.tex mit gesetztem Schalter.

\newif\ifkorrekturansicht
\korrekturansichttrue

\input{../tex-inputs/latex-vorspann}


               \section[Richard Beer-Hofmann an Arthur Schnitzler, 14. 2. 1912]{ Richard Beer-Hofmann an Arthur Schnitzler,
               14. 2. 1912}\nopagebreak\mylabel{v}\rehead{ }\normalsize\beginnumbering\briefempfaengerindex{Schnitzler, Arthur@\textsc{Schnitzler, Arthur}!zzzBeer-Hofmann, Richard@\emph{von Richard Beer-Hofmann}!1912-02-141@{14. 2. 1912}|(be} \toendnotes[C]{\smallbreak\pagebreak[2]} \Standort{CUL, Schnitzler, B 8.}
\physDesc{Brief, 1 Blatt, 2 Seiten
\newline{}Handschrift: Bleistift, lateinische Kurrent\newline{}Ordnung: mit Bleistift von unbekannter Hand nummeriert: »243« }\buchAbdrucke{\weitereDrucke{Arthur Schnitzler, Richard Beer-Hofmann: \emph{Briefwechsel 1891–1931}. Hg. Konstanze Fliedl. Wien, Zürich: \emph{Europaverlag} 1992, S. 216.} }\pstart
           \raggedleft{}{\pb}14./II. 12. \pend
           \pstart
           Lieber Arthur! Ich möchte gerne \textcolor{blue}{Stuckens}{}\ledrightnote{\textcolor{blue}{Eduard Stucken}{\newline}\textcolor{blue}{Ania Stucken}} die eben bei mir waren sprechen. Da sie – wie ich
               weiss – bei D\textsuperscript{r}{ }\textcolor{blue}{Rosenbaum}{}\ledrightnote{\textcolor{blue}{Richard Rosenbaum}} – zu Tische sind könnte ich sie dort
               anrufen, – wenn ich die Telephonnu{\geminationm}er wüsste.\pend
           \pstart
           {\pb}Sie steht nicht im Telephonbuch.
               Wissen \uline{Sie}{ }\uline{sie} (Styl des Dichters \textcolor{blue}{G. H.}{}\ledrightnote{\textcolor{blue}{Gerhart Hauptmann}})?\pend
           \pstart
           Oder können \uline{Sie sie} leicht erfahren?\pend
           \pstart
           Herzlichst{\\[\baselineskip]}\spacefill\mbox{Richard}\pend
           \leftskip=0em{}\endnumbering\briefempfaengerindex{Schnitzler, Arthur@\textsc{Schnitzler, Arthur}!zzzBeer-Hofmann, Richard@\emph{von Richard Beer-Hofmann}!1912-02-141@{14. 2. 1912}|)be}\mylabel{h}  \normalsize

\doendnotes{C}
\bigskip
\vfill

\clearpage

\footnotesize

\lohead{\textsc{register}}

% Definiere theindex-Environment komplett neu ohne reledmac
\makeatletter
\renewenvironment{theindex}{%
  \section*{\indexname}%
  \setlength{\parindent}{0pt}%
  \setlength{\parskip}{0pt plus 0.3pt}%
  \let\item\@idxitem
}{%
  \clearpage
}
\makeatother

\IfFileExists{\jobname-pw.ind}{\input{\jobname-pw.ind}}{}

\end{document}

      