%% latex-korrekturansicht-vorspann.tex
%% Vorspann für die Korrekturansicht.
%% Lädt die gemeinsame Datei latex-vorspann.tex mit gesetztem Schalter.

\newif\ifkorrekturansicht
\korrekturansichttrue

\input{../tex-inputs/latex-vorspann}


               \section[Arthur Schnitzler an Richard Beer-Hofmann, 2. 6. 1898]{ Arthur Schnitzler an Richard Beer-Hofmann, 2. 6. 1898}\nopagebreak\mylabel{v}\rehead{ }\normalsize\beginnumbering\briefempfaengerindex{Beer-Hofmann, Richard@\textsc{Beer-Hofmann, Richard}!zzzSchnitzler, Arthur@\emph{von Arthur Schnitzler}!1898-06-021@{2. 6. 1898}|(be} \toendnotes[C]{\smallbreak\pagebreak[2]} \Standort{YCGL, MSS 31.}
\physDesc{Brief, 1 Blatt, 3 Seiten, Umschlag
\newline{}Handschrift: Bleistift, deutsche Kurrent\newline{}Versand: 1) Stempel: »\nobreak{}\oindex{IX., Alsergrund@\textbf{IX., Alsergrund}, \emph{Bezirk (A.BZK)}|pwk}Wien 9/3, 2. 6. 98, 1–2N\nobreak{}«.  2) Stempel: »\nobreak{}\oindex{Steindorf am Ossiacher See@\textbf{Steindorf am Ossiacher See}, \emph{http://www.geonames.org/ontologyA.ADM3}|pwk}Steindorf am Ossiacher See, 3 6 98\nobreak{}«. }\buchAbdrucke{\weitereDrucke{Arthur Schnitzler, Richard Beer-Hofmann: \emph{Briefwechsel 1891–1931}. Hg. Konstanze Fliedl. Wien, Zürich: \emph{Europaverlag} 1992, S. 116.} }\pstart{}{\pb}Herrn \textsc{Dr. Richard
                     Beer-Hofmann}\pend{}\pstart{}\textsc{\textcolor{pink}{Steindorf}{}\ledrightnote{\textcolor{pink}{Steindorf am Ossiacher See}}}\pend{}\pstart{}\textsc{am }\textcolor{pink}{\textsc{Ossiacher}{ }See}{}\ledrightnote{\textcolor{pink}{Ossiacher See}}\pend{}\pstart{}in \textcolor{pink}{\textsc{Kärnthen}}{}\ledrightnote{\textcolor{pink}{Kärnten}}\pend{}{\bigskip}\pstart{}{\pb}Lieber Richard\pend\pstart
           ich habe ganz den Eindruck, als ob ich So{\geminationn}tag{ }früh von hier wegfahren und vielleicht Dinſtag am \textcolor{pink}{\textsc{Ossiacher}{ }See}{}\ledrightnote{\textcolor{pink}{Ossiacher See}} eintreffen
               würde. {\pb}Weiteres und näheres, was ganz dasſelbe
               iſt, merkwürdigerweiſe, weiſs ich noch nicht. Doch ſcheints mir, daſs ich ein paar
               Tage im \textcolor{pink}{Annenheim}{}\ledrightnote{\textcolor{pink}{Annenheim}} wohnen werde; eventuell radle
               ich aber mit \textcolor{blue}{\textsc{Kramer}}{}\ledrightnote{\textcolor{blue}{Leopold Kramer}} weiter ins \textcolor{pink}{\textsc{Lavantthal}}{}\ledrightnote{\textcolor{pink}{Lavanttal}}. {\pb}Vom \textcolor{pink}{\textsc{Semmering}}{}\ledrightnote{\textcolor{pink}{Semmering}} aus will ich die ganze Tour per Rad machen.\pend
           \pstart
           Ich freue mich Sie bald zu ſehen.\pend
           \pstart
           Grüßen Sie \textcolor{blue}{Paula}{}\ledrightnote{\textcolor{blue}{Paula Beer-Hofmann}}, \textcolor{blue}{Mirjam}{}\ledrightnote{\textcolor{blue}{Mirjam Beer-Hofmann}} und ſich ſelbſt.\pend
           \pstart
           Herzlichſt der Ihre{\\[\baselineskip]}\spacefill\mbox{Arthur}\pend
           \leftskip=0em{}\endnumbering\briefempfaengerindex{Beer-Hofmann, Richard@\textsc{Beer-Hofmann, Richard}!zzzSchnitzler, Arthur@\emph{von Arthur Schnitzler}!1898-06-021@{2. 6. 1898}|)be}\mylabel{h}  \normalsize

\doendnotes{C}
\bigskip
\vfill

\clearpage

\footnotesize

\lohead{\textsc{register}}

% Definiere theindex-Environment komplett neu ohne reledmac
\makeatletter
\renewenvironment{theindex}{%
  \section*{\indexname}%
  \setlength{\parindent}{0pt}%
  \setlength{\parskip}{0pt plus 0.3pt}%
  \let\item\@idxitem
}{%
  \clearpage
}
\makeatother

\IfFileExists{\jobname-pw.ind}{\input{\jobname-pw.ind}}{}

\end{document}

      