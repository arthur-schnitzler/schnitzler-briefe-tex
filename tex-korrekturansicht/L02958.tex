%% latex-korrekturansicht-vorspann.tex
%% Vorspann für die Korrekturansicht.
%% Lädt die gemeinsame Datei latex-vorspann.tex mit gesetztem Schalter.

\newif\ifkorrekturansicht
\korrekturansichttrue

\input{../tex-inputs/latex-vorspann}


\renewcommand{\erwaehntePersonen}{Personen: Richard Beer-Hofmann, Paul Horn, Josef Jarno, Felix Salten, Richard Specht, Ignaz Wild, Grethe Wreden}
\renewcommand{\erwaehnteInstitutionen}{Institutionen: Saisontheater Ischl}
\renewcommand{\erwaehnteOrte}{Orte: Anzenau, Bad Aussee, Bad Ischl, Hotel und Pension Rudolfshöhe (Leopold Petter), Salzburg, Salzkammergut, Stadttheater (Bad Ischl), Strobl, Ungarn, Wien}
\renewcommand{\erwaehnteWerke}{Werke: Abschiedssouper, Anatol, Artifex, Das Märchen. Schauspiel in drei Aufzügen, Die Frage an das Schicksal}
\section[Arthur Schnitzler an Felix Salten, 5. 7. 1893]{Arthur Schnitzler an Felix Salten, 5. 7. 1893}
\nopagebreak\mylabel{v}
\rehead{ }\normalsize\beginnumbering\briefempfaengerindex{Salten, Felix@\textsc{Salten, Felix}!zzzSchnitzler, Arthur@\emph{von Arthur Schnitzler}!1893-07-052@{5. 7. 1893}|(be}
\toendnotes[C]{\smallbreak\pagebreak[2]}\Standort{Wienbibliothek im Rathaus, ZPH 1681, 2.1.516.}
\physDesc{Brief, 2 Blätter, 6 Seiten, 1497 Zeichen (Briefpapier mit Trauerrand)
\newline{}Handschrift: schwarze Tinte, deutsche Kurrent
\newline{}Ordnung: mit Bleistift von unbekannter Hand Nummerierung der Doppelseiten des
                                 Konvoluts: »81«–»83« }\toendnotes[C]{\smallbreak}
\pstart
           \raggedleft{}{\pb}\textsc{\textcolor{pink}{Pension Leopold}{}\ledrightnote{\textcolor{pink}{Hotel und Pension Rudolfshöhe (Leopold Petter)}}}, 5/7 93. \pend
           
\pstart{}Mein lieber Salten,\pend
\pstart
           das wichtigſte zuerſt: geſtern{ }\textsc{per}{ }\label{K_L02958-1v}\edtext{\textsc{Bic.}}{\lemma{\textnormal{\emph{Bic.}}}\Cendnote{\textnormal{Bicycle (Fahrrad). Zu den Ausflügen siehe A. S.: \emph{Tagebuch}, 4. 7. 1893 und 5. 7. 1893}}}\label{K_L02958-1h} in \textsc{\textcolor{pink}{Strobl}{}\ledrightnote{\textcolor{pink}{Strobl}}}, heut in \textsc{\textcolor{pink}{Anzenau}{}\ledrightnote{\textcolor{pink}{Anzenau}}} geweſen – geht im ganzen recht gut. Leider i{\geminationm}er
               allein; \textsc{\textcolor{blue}{Richard}{}\ledrightnote{\textcolor{blue}{Richard Beer-Hofmann}}} ko{\geminationm}t nach (wie geſtern) oder auch nicht (wie heute.) –
               Geſchrieben noch nichts; und {\pb}heute{ }früh, einſam, in \textsc{\textcolor{pink}{Anzenau}{}\ledrightnote{\textcolor{pink}{Anzenau}}}, die Verſe meines \textcolor{green}{allegor
                  Gedichts}{}\ledrightnote{{$\rightarrow$}\textcolor{green}{Artifex}} in Ihrem Sinne in regelmäßige Jamben übertragen. –\pend
           
\pstart
           – Meine Sti{\geminationm}ung recht ſchlecht. Leer, traurig. –
                  Heut hab ich ſogar geweint – in \textsc{\textcolor{pink}{Anzenau}{}\ledrightnote{\textcolor{pink}{Anzenau}}}! – Außerdem hab ich durch den ſonderbarſten der Zufälle auch noch \label{K_L02958-11v}\edtext{neue Dinge}{\lemma{\textnormal{\emph{neue Dinge}}}\Cendnote{\textnormal{Über den Aufenthalt von \textcolor{blue}{Marie Glümer} in \textcolor{pink}{Salzburg}, wo sie eine
                  intime Beziehung mit \textcolor{blue}{Rudolf von
                     Cuny-Pierron} hatte, vgl. A. S.: \emph{Tagebuch}, 4. 7. 1893.}}}\label{K_L02958-11h} erfahren – {\pb}aus
                  \textsc{\textcolor{pink}{Salzb.}{}\ledrightnote{\textcolor{pink}{Salzburg}}} – alſo eigentlich ſehr alte Dinge – O Menſch, ahnen Sie etwa, wie geſcheidt ich
               war, als ich das \textcolor{green}{Märchen}{}\ledrightnote{\textcolor{green}{Das Märchen. Schauspiel in drei Aufzügen}} ſchrieb? – Bitte,
               fragen Sie noch nichts in einem eventuellen Brief, den Sie mir ſchreiben – ich wäre
               nervös, we{\geminationn} ich es verraten müßte. –\pend
           
\pstart
           – \label{K_L02958-2v}\edtext{\textsc{\textcolor{blue}{Jarno}{}\ledrightnote{\textcolor{blue}{Josef Jarno}}} hab ich geſprochen; {\pb}der hatte
               natürlich mein \textcolor{green}{Stück}{}\ledrightnote{{$\rightarrow$}\textcolor{green}{Anatol}} überhaupt
               noch nicht geleſen}{\lemma{\textnormal{\emph{Jarno … geleſen}}}\Cendnote{\textnormal{siehe A. S.: \emph{Tagebuch}, 4. 7. 1893}}}\label{K_L02958-2h}; iſt ein Komödiant, aber nebſtbei ein geſcheidter \textcolor{pink}{ungar}{}\ledrightnote{{$\rightarrow$}\textcolor{pink}{Ungarn}}iſcher Jud u wahrſcheinlich ein großes
               Talent. – Jetzt iſt er vom \textcolor{green}{Abſchiedsſouper}{}\ledrightnote{\textcolor{green}{Abschiedssouper}}
               ſehr entzückt, und \textsc{\textcolor{blue}{Wild}{}\ledrightnote{\textcolor{blue}{Ignaz Wild}}} (der \textcolor{brown}{Direktor}{}\ledrightnote{{$\rightarrow$}\textcolor{brown}{Saisontheater Ischl}}) \label{K_L02958-3v}\edtext{führt am Montag{ }{\pb}»\textcolor{green}{Frage}{}\ledrightnote{\textcolor{green}{Die Frage an das Schicksal}}«
               u »\textcolor{green}{Abſchiedſouper}{}\ledrightnote{\textcolor{green}{Abschiedssouper}}« auf}{\lemma{\textnormal{\emph{führt … auf}}}\Cendnote{\textnormal{im \textcolor{pink}{Saisontheater} in \textcolor{pink}{Bad Ischl}}}}\label{K_L02958-3h}, ohne ſie geleſen zu haben, oh nicht wegen \textsc{\textcolor{blue}{Jarno}{}\ledrightnote{\textcolor{blue}{Josef Jarno}}}, ſondern weil er ſich denkt, daſs mein Name (oh nicht als Dichter!!) ihm das
                  \textcolor{pink}{Haus}{}\ledrightnote{\textcolor{pink}{Stadttheater (Bad Ischl)}} füllt. –\pend
           
\pstart
           – Sagen Sie’s aber noch
               niemandem. We{\geminationn} es ſicher iſt, aviſire ich Sie – Wo iſt
                  \textcolor{blue}{Paul Horn}{}\ledrightnote{\textcolor{blue}{Paul Horn}}? Vielleicht {\pb}gibt »ſeine« \label{K_L02958-4v}\edtext{\textcolor{blue}{Grethe}{}\ledrightnote{\textcolor{blue}{Grethe Wreden}} die \textcolor{green}{Cora}{}\ledrightnote{{$\rightarrow$}\textcolor{green}{Die Frage an das Schicksal}}}{\lemma{\textnormal{\emph{Grethe die Cora}}}\Cendnote{\textnormal{siehe Arthur Schnitzler an Felix Salten, 9. 7. 1893}}}\label{K_L02958-4h}. – Wann ko{\geminationm}t \textsc{\textcolor{blue}{Richard Specht}{}\ledrightnote{\textcolor{blue}{Richard Specht}}}? – Einmal will ich mit \textsc{\textcolor{blue}{Rich. BHof}{}\ledrightnote{\textcolor{blue}{Richard Beer-Hofmann}}} nach \textsc{\textcolor{pink}{Salzburg}{}\ledrightnote{\textcolor{pink}{Salzburg}}} mittells der \label{K_L02958-5v}\edtext{neuen Bahn}{\lemma{\textnormal{\emph{neuen Bahn}}}\Cendnote{\textnormal{Gemeint war die im Juni 1893 in Betrieb genommene \textcolor{pink}{Salzkammergut}-Lokalbahn zwischen \textcolor{pink}{Salzburg} und \textcolor{pink}{Bad
               Ischl}.}}}\label{K_L02958-5h}. –\pend
           
\pstart
           – Seien Sie ſo gut und ſchreiben Sie ſofort. –\pend
           
\pstart
           Herzlich der Ihre {\\[\baselineskip]}\spacefill\mbox{Arthur}\pend
           \leftskip=0em{}\endnumbering\briefempfaengerindex{Salten, Felix@\textsc{Salten, Felix}!zzzSchnitzler, Arthur@\emph{von Arthur Schnitzler}!1893-07-052@{5. 7. 1893}|)be}\mylabel{h}  \normalsize

\doendnotes{C}
\bigskip
\vfill

\clearpage

\footnotesize

\lohead{\textsc{register}}

% Definiere theindex-Environment komplett neu ohne reledmac
\makeatletter
\renewenvironment{theindex}{%
  \section*{\indexname}%
  \setlength{\parindent}{0pt}%
  \setlength{\parskip}{0pt plus 0.3pt}%
  \let\item\@idxitem
}{%
  \clearpage
}
\makeatother

\IfFileExists{\jobname-pw.ind}{\input{\jobname-pw.ind}}{}

\end{document}

      