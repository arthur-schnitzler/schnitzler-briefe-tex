%% latex-korrekturansicht-vorspann.tex
%% Vorspann für die Korrekturansicht.
%% Lädt die gemeinsame Datei latex-vorspann.tex mit gesetztem Schalter.

\newif\ifkorrekturansicht
\korrekturansichttrue

\input{../tex-inputs/latex-vorspann}


\renewcommand{\erwaehntePersonen}{Personen: Karl Glossy, Heinrich Kanner, Felix Salten, Ottilie Salten, Olga Schnitzler, Isidor Singer, Jakob Wassermann}
\renewcommand{\erwaehnteInstitutionen}{Institutionen: Österreichische Rundschau}
\renewcommand{\erwaehnteOrte}{Orte: Edmund-Weiß-Gasse 7, Riedhof, Semmering, Theater in der Josefstadt, Wien}
\renewcommand{\erwaehnteWerke}{Werke: Bemerkungen, Clarissa Mirabel, Das Buch der Könige, Immer modern, Österreichische Rundschau}
\section[ Arthur Schnitzler an Felix Salten, 20. 12. 1905]{Arthur Schnitzler an Felix Salten, 20. 12. 1905}
\nopagebreak\mylabel{v}
\rehead{ }\normalsize\beginnumbering\briefempfaengerindex{Salten, Felix@\textsc{Salten, Felix}!zzzSchnitzler, Arthur@\emph{von Arthur Schnitzler}!1905-12-201@{20. 12. 1905}|(be}
\toendnotes[C]{\smallbreak\pagebreak[2]}\Standort{Wienbibliothek im Rathaus, ZPH 1681, 2.1.516.}
\physDesc{Brief, 1 Blatt, 3 Seiten, 1070 Zeichen
\newline{}Handschrift: schwarze Tinte, deutsche Kurrent
\newline{}Ordnung: mit Bleistift von unbekannter Hand Nummerierung der Blätter des Konvoluts:
                                    »15« }
\buchAbdrucke{\weitereDrucke{Arthur Schnitzler: \emph{Briefe 1875–1912}. Hg. Therese Nickl und Heinrich Schnitzler. Frankfurt am Main: \emph{S. Fischer} 1981, S. 522–523.} }\toendnotes[C]{\smallbreak}
\pstart
           \noindent{}{\pb}\textcolor{gray}{\textbf{Dr. Arthur Schnitzler}}\hfill 20. 12. 905\pend
           
\pstart
           \textcolor{gray}{\textbf{\textcolor{pink}{Wien, XVIII. Spoettelgasse 7}{}\ledrightnote{\textcolor{pink}{Edmund-Weiß-Gasse 7}}.}}\pend
           
\pstart
           lieber, herzlichen Dank für das \label{K_L03001-1v}\edtext{\textcolor{green}{Königsbüchel}{}\ledrightnote{\textcolor{green}{Das Buch der Könige}}}{\lemma{\textnormal{\emph{Königsbüchel}}}\Cendnote{\textnormal{siehe Felix Salten: Widmungsexemplar Das Buch der Könige für Arthur
               Schnitzler, [zwischen 1. und 20. 12.] 1905}}}\label{K_L03001-1h}, deſſen Köſtlich- u Koſtbarkeiten wiederzugenießen ich mich ſchon ſehr
               freue.\pend
           
\pstart
           Ferner: eine Anzahl ſogenannter \label{K_L03001-2v}\edtext{\textcolor{green}{Aphorismen}{}\ledrightnote{{$\rightarrow$}\textcolor{green}{Bemerkungen}}}{\lemma{\textnormal{\emph{Aphorismen}}}\Cendnote{\textnormal{\textcolor{blue}{Arthur Schnitzler}: \emph{\textcolor{green}{Bemerkungen}}. In: \emph{\textcolor{green}{Österreichische Rundschau}}. Bd. 5, Nr. 60/61, 21. 12. 1905, S. 395–396.}}}\label{K_L03001-2h} lag ſchon für die
               Weihnachtszeit bereit – da kam ein wahrer Brandbrief von \textsc{\textcolor{blue}{Glossy}{}\ledrightnote{\textcolor{blue}{Karl Glossy}}} (der mich ſchon ſeit Gründg der \textcolor{brown}{Oe. Rdſch.}{}\ledrightnote{\textcolor{brown}{Österreichische Rundschau}}
               heftig um Beiträge angeht aus der (wörtlich) »vor Aufregung phyſiſch {\pb}erkrankt ſei, durch meine neuerliche
               Absage–«) – nun und ich ſandte ihm die paar \textcolor{green}{Nichtigkeiten}{}\ledrightnote{{$\rightarrow$}\textcolor{green}{Bemerkungen}}, in der angenehmen Gewißheit, daſs \textsc{\textcolor{blue}{Singer}{}\ledrightnote{\textcolor{blue}{Isidor Singer}}} und \textsc{\textcolor{blue}{Kanner}{}\ledrightnote{\textcolor{blue}{Heinrich Kanner}}s} Geſundheit durch mein
               Fernbleiben unerſchüttert bleiben. (Und nun hab ich wieder einmal die feſte Abſicht,
               mit nichts mehr in die Oeffentlichkeit zu ko{\geminationm}en, eh ich
               wieder was ganz ordentliches herausgebracht habe.)\pend
           
\pstart
           Drittens. \label{K_L03001-3v}\edtext{Morgen Donnerſtag}{\lemma{\textnormal{\emph{Morgen Donnerſtag}}}\Cendnote{\textnormal{\textcolor{blue}{Arthur} und \textcolor{blue}{Olga Schnitzler} sahen sich \emph{\textcolor{green}{Immer
                     modern}} von Henri Léon Lavedan an, vgl. A. S.: \emph{Tagebuch}, 21. 12. 1905. Ein anschließender Besuch im \textcolor{pink}{Riedhof} ist nicht belegt. Auch ein
                  Zusammentreffen mit \textcolor{blue}{Salten} ist nicht
                  nachweisbar.}}}\label{K_L03001-3h} gehn {\pb}\textcolor{blue}{wir}{}\ledrightnote{{$\rightarrow$}\textcolor{blue}{Olga Schnitzler}} ins \textcolor{pink}{Joſefſtädter Theater}{}\ledrightnote{\textcolor{pink}{Theater in der Josefstadt}}, und wären ſehr erfreut, nachher (im \textcolor{pink}{Riedhof}{}\ledrightnote{\textcolor{pink}{Riedhof}} wie u wo neulich) mit Ihnen \textcolor{blue}{beiden}{}\ledrightnote{{$\rightarrow$}\textcolor{blue}{Ottilie Salten}} zuſa{\geminationm}entreffen zu können. Und we{\geminationn} Sie verhindert ſind, geben Sie ein andres Rendevous oder ko{\geminationm}en zu uns. \label{K_L03001-4v}\edtext{Mittwoch}{\lemma{\textnormal{\emph{Mittwoch}}}\Cendnote{\textnormal{Siehe A. S.: \emph{Tagebuch}, 27. 12. 1905. \textcolor{blue}{Salten} war nicht bei der privaten Lesung von
                     \emph{\textcolor{green}{Clarissa Mirabel}}.}}}\label{K_L03001-4h} ſind Sie wohl
               auch zur \textsc{\textcolor{blue}{Wasserm}{}\ledrightnote{\textcolor{blue}{Jakob Wassermann}}}. \textcolor{green}{Vorleſung}{}\ledrightnote{{$\rightarrow$}\textcolor{green}{Clarissa Mirabel}} geladen? Und
               am \label{K_L03001-5v}\edtext{\textsc{\textcolor{pink}{Se{\geminationm}ering}{}\ledrightnote{\textcolor{pink}{Semmering}}}, Jänner}{\lemma{\textnormal{\emph{Semmering, Jänner}}}\Cendnote{\textnormal{nicht geschehen}}}\label{K_L03001-5h}, halten wir doch
               feſt?\pend
           
\pstart
           Herzlichſt Ihr {\\[\baselineskip]}\spacefill\mbox{A.}\pend
           \leftskip=0em{}\endnumbering\briefempfaengerindex{Salten, Felix@\textsc{Salten, Felix}!zzzSchnitzler, Arthur@\emph{von Arthur Schnitzler}!1905-12-201@{20. 12. 1905}|)be}\mylabel{h}  \normalsize

\doendnotes{C}
\bigskip
\vfill

\clearpage

\footnotesize

\lohead{\textsc{register}}

% Definiere theindex-Environment komplett neu ohne reledmac
\makeatletter
\renewenvironment{theindex}{%
  \section*{\indexname}%
  \setlength{\parindent}{0pt}%
  \setlength{\parskip}{0pt plus 0.3pt}%
  \let\item\@idxitem
}{%
  \clearpage
}
\makeatother

\IfFileExists{\jobname-pw.ind}{\input{\jobname-pw.ind}}{}

\end{document}

      