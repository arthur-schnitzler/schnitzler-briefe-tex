%% latex-korrekturansicht-vorspann.tex
%% Vorspann für die Korrekturansicht.
%% Lädt die gemeinsame Datei latex-vorspann.tex mit gesetztem Schalter.

\newif\ifkorrekturansicht
\korrekturansichttrue

\input{../tex-inputs/latex-vorspann}


\renewcommand{\erwaehntePersonen}{Personen: Felix Dörmann, Guy de Maupassant, Constantin Meunier, Jakob Wassermann}
\renewcommand{\erwaehnteOrte}{Orte: Florenz, Hofkirche, Wien}
\renewcommand{\erwaehnteWerke}{Werke: Die Geschichte der jungen Renate Fuchs, Dämmerseele, Dämmerseelen. Novellen, Frau Bertha Garlan. Roman, Neue Freie Presse}
\section[ Felix Salten an Arthur Schnitzler, 22. 5. 1902]{Felix Salten an Arthur Schnitzler, 22. 5. 1902}
\nopagebreak\mylabel{v}
\rehead{ }\normalsize\beginnumbering\briefempfaengerindex{Schnitzler, Arthur@\textsc{Schnitzler, Arthur}!zzzSalten, Felix@\emph{von Felix Salten}!1902-05-221@{22. 5. 1902}|(be}
\toendnotes[C]{\smallbreak\pagebreak[2]}\Standort{CUL, Schnitzler, B 89, A 2.}
\physDesc{Brief, 1 Blatt, 2 Seiten, 1537 Zeichen
\newline{}Handschrift: Bleistift, lateinische Kurrent
\newline{}Ordnung: mit Bleistift von unbekannter Hand nummeriert: »155« }\toendnotes[C]{\smallbreak}
\pstart
           \raggedleft{}{\pb}\textcolor{pink}{Florenz}{}\ledrightnote{\textcolor{pink}{Florenz}}, 22. Mai 02.\pend
           
\pstart
           Lieber Arthur, eben las ich Ihre kleine \label{K_L03330-1v}\edtext{\textcolor{green}{Novelle}{}\ledrightnote{{$\rightarrow$}\textcolor{green}{Dämmerseele}}}{\lemma{\textnormal{\emph{Novelle}}}\Cendnote{\textnormal{\textcolor{blue}{Arthur Schnitzler}: \emph{\textcolor{green}{Dämmerseele}}. In: \emph{\textcolor{green}{Neue
                        Freie Presse}}, Nr. 13.553, 18. 5. 1902,
                     Morgenblatt, Pfingstbeilage, S. 31–33.}}}\label{K_L03330-1h} in der »\textcolor{green}{N. fr. Pr.}{}\ledrightnote{\textcolor{green}{Neue Freie Presse}}« Ich glaube, das ist nicht blos an sich etwas
               Gutes, sondern auch ein Schritt weiter. Es ist alles Psychologische in eine knappe
               Gegenständlichkeit verlegt, und gut zusa{\geminationm}engefaßt. \textcolor{blue}{Meunier}{}\ledrightnote{\textcolor{blue}{Constantin Meunier}} und \textcolor{blue}{Maupassant}{}\ledrightnote{\textcolor{blue}{Guy de Maupassant}}. Und es ist wirklich »erzählt«. Ich finde neue Spuren \strikeout{\textcolor{gray}{darin},} und täusche mich hoffentlich
               nicht. Nebenbei: die ganze \label{K_L03330-2v}\edtext{\textcolor{green}{Renate}{}\ledrightnote{\textcolor{green}{Die Geschichte der jungen Renate Fuchs}}}{\lemma{\textnormal{\emph{Renate}}}\Cendnote{\textnormal{\textcolor{blue}{Jakob Wassermann}s \emph{\textcolor{green}{Die Geschichte der jungen Renate Fuchs}} (1900/1901)}}}\label{K_L03330-2h} liegt auch drin, im Extract,
               und eigentlich viel plastischer und aufrichtiger, obwol vorn und hinten alles fehlt.
               Der \label{K_L03330-3v}\edtext{Titel »\textcolor{green}{Dämmerseele}{}\ledrightnote{\textcolor{green}{Dämmerseele}}« scheint mir aber ganz verfehlt}{\lemma{\textnormal{\emph{Titel … verfehlt}}}\Cendnote{\textnormal{Da nur der Erstdruck \emph{\textcolor{green}{Dämmerseele}} hieß, dürfte \textcolor{blue}{Schnitzler}{ }\textcolor{blue}{Salten}s Kritik ernst genommen haben. Die
                  erste \textcolor{green}{Buchausgabe} von 1907 verwendete \emph{\textcolor{green}{Dämmerseelen}} als Gesamttitel, die betreffende \textcolor{green}{Novelle} wurde aber zu \emph{\textcolor{green}{Die Fremde}} umbenannt.}}}\label{K_L03330-3h}. – Geschrieben nimmt sich
               alles härter aus \textcolor{gray}{–} bitte – reduziren Sie also das Folgende auf die
               Wirkung des \uline{Gesagten}: Es ist ein \textcolor{blue}{Dörmann}{}\ledrightnote{\textcolor{blue}{Felix Dörmann}} Titel, d. h. ein Versuch eine Gattung abzugrenzen, zu
               benennen, aber die Grenze und die Benennung sind nicht scharf, und dem Wort haftet
               eine leidige, ins Sentimentale gehende Weichlichkeit an. Es liegt auch kaum die
               Notwendigkeit vor, durch den Titel etwas zu erklären; mit ihm selbst das Wort zu
               ergreifen. Und gerade mit diesem Titel ist alles in einer eigentlich hindernden und
               auch irreführenden Art vorweg genommen. Er ist vielleicht aus der \textcolor{pink}{Hofkirche}{}\ledrightnote{\textcolor{pink}{Hofkirche}} besser zu holen. Am besten aus der Einfachheit.
               Ganz außerordentlich ist der Schluß. Das geht in kurzer Wendung {\pb}zu einer beinahe dramatischen
               Höhe, jedesfalls zu einem weiten Ausblick. Nun bedaure ich es, dass ich noch nicht
               dazu kam, über die \label{K_L03330-4v}\edtext{\textcolor{green}{Bertha Garlan}{}\ledrightnote{\textcolor{green}{Frau Bertha Garlan. Roman}}}{\lemma{\textnormal{\emph{Bertha Garlan}}}\Cendnote{\textnormal{Sowohl der \textcolor{green}{Zeitungsabdruck} als auch die \textcolor{green}{Buchausgabe} waren bereits im
                     Jahr davor erschienen.}}}\label{K_L03330-4h} zu schreiben. Das will
               ich \label{K_L03330-5v}\edtext{im Sommer nachholen}{\lemma{\textnormal{\emph{im Sommer nachholen}}}\Cendnote{\textnormal{nicht nachweisbar}}}\label{K_L03330-5h}. Jetzt war und bin
               ich eben sehr mit mir selbst beschäftigt.\pend
           
\pstart
           herzlichst Ihr {\\[\baselineskip]}\spacefill\mbox{Salten}\pend
           \leftskip=0em{}\endnumbering\briefempfaengerindex{Schnitzler, Arthur@\textsc{Schnitzler, Arthur}!zzzSalten, Felix@\emph{von Felix Salten}!1902-05-221@{22. 5. 1902}|)be}\mylabel{h}  \normalsize

\doendnotes{C}
\bigskip
\vfill

\clearpage

\footnotesize

\lohead{\textsc{register}}

% Definiere theindex-Environment komplett neu ohne reledmac
\makeatletter
\renewenvironment{theindex}{%
  \section*{\indexname}%
  \setlength{\parindent}{0pt}%
  \setlength{\parskip}{0pt plus 0.3pt}%
  \let\item\@idxitem
}{%
  \clearpage
}
\makeatother

\IfFileExists{\jobname-pw.ind}{\input{\jobname-pw.ind}}{}

\end{document}

      