%% latex-korrekturansicht-vorspann.tex
%% Vorspann für die Korrekturansicht.
%% Lädt die gemeinsame Datei latex-vorspann.tex mit gesetztem Schalter.

\newif\ifkorrekturansicht
\korrekturansichttrue

\input{../tex-inputs/latex-vorspann}


               \section[Hermann Bahr an Arthur Schnitzler, 4. 2. 1906]{ Hermann Bahr an Arthur Schnitzler, 4. 2. 1906}\nopagebreak\mylabel{v}\rehead{ }\normalsize\beginnumbering\briefempfaengerindex{Schnitzler, Arthur@\textsc{Schnitzler, Arthur}!zzzBahr, Hermann@\emph{von Hermann Bahr}!1906-02-042@{4. 2. 1906}|(be} \toendnotes[C]{\smallbreak\pagebreak[2]} \Standort{CUL, Schnitzler, B 5b.}
\physDesc{Brief, 1 Blatt, 1 Seite
\newline{}Handschrift: blaue Tinte, deutsche Kurrent\newline{}Ordnung: mit Bleistift von unbekannter Hand
                           nummeriert: »136« }\buchAbdrucke{\weitereDrucke{Hermann Bahr, Arthur Schnitzler: \emph{Briefwechsel, Aufzeichnungen, Dokumente (1891–1931)}. Hg. Kurt Ifkovits und Martin Anton Müller. Göttingen: \emph{Wallstein} 2018, S. 373.} }\toendnotes[C]{\smallbreak}\pstart
           \raggedleft{}{\pb}4. 2. 06\pend
           \pstart\center{}Lieber Arthur!\pend\pstart
           \uline{Mir} hat der \textcolor{blue}{Intendant}{}\ledrightnote{→\textcolor{blue}{Albert von Speidel}} die Genehmigung für den »\textcolor{green}{Ruf}{}\ledrightnote{\textcolor{green}{Der Ruf des Lebens. Schauspiel in drei Akten}}« verweigert, was aber nicht ausſchließt (da es offenbar nur
               zu den Chicanen gehört, welche mich hinausekeln ſollen), daß er \textcolor{green}{ihn}{}\ledrightnote{→\textcolor{green}{Der Ruf des Lebens. Schauspiel in drei Akten}}, wenn ich bis dahin meinen Vertrag gelöſt
               haben ſollte, nach einem \textcolor{pink}{Berliner}{}\ledrightnote{\textcolor{pink}{Berlin}} Erfolge ſehr gern
               nehmen wird.\pend
           \pstart
           Grüß \textcolor{blue}{Salten}{}\ledrightnote{\textcolor{blue}{Felix Salten}} und \textcolor{blue}{Brahm}{}\ledrightnote{\textcolor{blue}{Otto Brahm}} herzlichſt.\pend
           \pstart
           Hoffentlich ſehen wir uns dann doch endlich einmal.\pend
           \pstart
           Herzlichſt{\\[\baselineskip]}\spacefill\mbox{Hermann}\pend
           \leftskip=0em{}\endnumbering\briefempfaengerindex{Schnitzler, Arthur@\textsc{Schnitzler, Arthur}!zzzBahr, Hermann@\emph{von Hermann Bahr}!1906-02-042@{4. 2. 1906}|)be}\mylabel{h}  \normalsize

\doendnotes{C}
\bigskip
\vfill

\clearpage

\footnotesize

\lohead{\textsc{register}}

% Definiere theindex-Environment komplett neu ohne reledmac
\makeatletter
\renewenvironment{theindex}{%
  \section*{\indexname}%
  \setlength{\parindent}{0pt}%
  \setlength{\parskip}{0pt plus 0.3pt}%
  \let\item\@idxitem
}{%
  \clearpage
}
\makeatother

\IfFileExists{\jobname-pw.ind}{\input{\jobname-pw.ind}}{}

\end{document}

      