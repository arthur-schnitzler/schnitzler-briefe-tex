%% latex-korrekturansicht-vorspann.tex
%% Vorspann für die Korrekturansicht.
%% Lädt die gemeinsame Datei latex-vorspann.tex mit gesetztem Schalter.

\newif\ifkorrekturansicht
\korrekturansichttrue

\input{../tex-inputs/latex-vorspann}


\renewcommand{\erwaehntePersonen}{Personen: Marie Suin Beausacq, Hugo Felix, Rudolf Gussmann, Sully Prudhomme, Olga Schnitzler, Elisabeth Steinrück, Anton Pavlovič Čechov}
\renewcommand{\erwaehnteInstitutionen}{Institutionen: Die Zeit. Wiener Wochenschrift, Paul Ollendorff, Schiller-Theater}
\renewcommand{\erwaehnteOrte}{Orte: Berlin, Dessauer Straße, Italien, Montreux, Paris, Wien}
\renewcommand{\erwaehnteWerke}{Werke: Der Schleier der Beatrice. Schauspiel in fünf Akten, Die Zeit, Die Zeit. Wiener Wochenschrift, Maximes de la vie. Préface par Sully Prud’homme, Schatten des Todes}
\section[ Paul Goldmann an Arthur Schnitzler, 2. {[}10. 1902{]}]{Paul Goldmann an Arthur Schnitzler, 2. {[}10. 1902{]}}
\nopagebreak\mylabel{v}
\rehead{ }\normalsize\beginnumbering\briefempfaengerindex{Schnitzler, Arthur@\textsc{Schnitzler, Arthur}!zzzGoldmann, Paul@\emph{von Paul Goldmann}!1902-10-021@{2. {[}10. 1902{]}}|(be}
\toendnotes[C]{\smallbreak\pagebreak[2]}\Standort{DLA, A:Schnitzler, HS.NZ85.1.3172.}
\physDesc{Brief, 2 Blätter, 7 Seiten
\newline{}Handschrift: blaue Tinte, deutsche Kurrent
\newline{}Schnitzler: 1) mit Bleistift das Jahr »{[}1{]}902« vermerkt  2) mit rotem Buntstift eine Unterstreichung}\toendnotes[C]{\smallbreak}
\pstart
           \noindent{}\raggedleft{}{\pb}\textcolor{pink}{\textcolor{gray}{\textbf{DESSAUERSTRASSE 19}}}{}\ledrightnote{\textcolor{pink}{Dessauer Straße}}\pend
           
\pstart
           \textcolor{pink}{Berlin}{}\ledrightnote{\textcolor{pink}{Berlin}}, \label{K_L03223-11v}\edtext{2. Sept.}{\lemma{\textnormal{\emph{2. Sept.}}}\Cendnote{\textnormal{Die Datierung ist offensichtlich falsch, da \textcolor{blue}{Goldmann} am 1. 9. [1902] noch in
                           \textcolor{pink}{Montreux} weilte und eine längere
                        Heimreise plante. Da \textcolor{blue}{Goldmann}s Brief
                        vom 6. 10. [1902]
                        Reaktionen auf Antworten enthält, deren Fragen im vorliegenden Brief
                        gestellt werden, ist ein Irrtum um einen Monat von September auf Oktober
                        anzunehmen. }}}\label{K_L03223-11h}\pend
           
\pstart\center{}Mein lieber Freund,\pend
\pstart
           Die \label{K_L03223-1v}\edtext{Paß-Angelegenheit}{\lemma{\textnormal{\emph{Paß-Angelegenheit}}}\Cendnote{\textnormal{Bezug auf einen Pass für \textcolor{blue}{Elisabeth Gussmann}, die wohl ohne entsprechende Dokumente
                  für ihre Anstellung am \emph{\textcolor{brown}{Schiller-Theater}} nach
                     \textcolor{pink}{Berlin} gezogen war, siehe A. S.: \emph{Tagebuch}, 25. 9. 1902. }}}\label{K_L03223-1h} hat
               mich nicht gar ſo viel Zeit gekoſtet, und ich brauche Dir nicht erſt zu ſagen, daß es
               mir eine große Freude macht, meine Zeit auf eine Angelegenheit zu verwenden, die Dich
               (wenn auch nur indirekt) betrifft. Die einwöchentliche Frist müßt \textcolor{blue}{Ihr}{}\ledrightnote{{$\rightarrow$}\textcolor{blue}{Elisabeth Steinrück}{\newline}{$\rightarrow$}\textcolor{blue}{Olga Schnitzler}} benutzen, um wenigſtens die
               Ausſtellung eines Interims-Paſſes zu ermöglichen. Sonſt ſtehe ich {\pb}für nichts. Es muß doch noch Rechtsmittel geben, um
               den \textcolor{blue}{Kerl}{}\ledrightnote{{$\rightarrow$}\textcolor{blue}{Rudolf Gussmann}} zu zwingen.
               Vielleicht iſt, da der \textcolor{blue}{Vater}{}\ledrightnote{{$\rightarrow$}\textcolor{blue}{Rudolf Gussmann}}
               ſo vollſtändig ſeine Pflichten vernachläſſigt, eine frühere Großjährigkeits-Erklärung
               oder die Beſtellung eines Vormunds möglich.\pend
           
\pstart
           Die Ausſicht, Dich \label{K_L03223-2v}\edtext{bald \textcolor{pink}{hier}{}\ledrightnote{{$\rightarrow$}\textcolor{pink}{Berlin}}}{\lemma{\textnormal{\emph{bald hier}}}\Cendnote{\textnormal{\textcolor{blue}{Schnitzler} war von 13. 10. 1902 bis 18. 10. 1902 in \textcolor{pink}{Berlin}. Die beiden trafen sich in dieser Zeit
                  täglich.}}}\label{K_L03223-2h} zu ſehen, bereitet mir große Freude. Freilich werde ich von Deinem
               Aufenthalt wenig haben, {\pb}da gerade Mitte Oktober meine Arbeit ins Ungeheure wachſen
               dürfte.\pend
           
\pstart
           \textsc{Dr. \textcolor{blue}{Hugo Felix}{}\ledrightnote{\textcolor{blue}{Hugo Felix}}} iſt \textcolor{pink}{hier}{}\ledrightnote{{$\rightarrow$}\textcolor{pink}{Berlin}} – ein ſehr
               lieber Menſch, der mir ausgezeichnet gefällt. Er hat mich \strikeout{gebete} erſucht, Dich zu bitten, Du möchteſt ihm doch die Erlaubniß geben,
               aus der »\textsc{\textcolor{green}{Beatrice}{}\ledrightnote{\textcolor{green}{Der Schleier der Beatrice. Schauspiel in fünf Akten}}}«, die er entzückend findet und von der er ſagt, daß ſie ihm herrlich »liegt«,
               für \textcolor{pink}{Italien}{}\ledrightnote{\textcolor{pink}{Italien}} eine \label{K_L03223-3v}\edtext{Oper}{\lemma{\textnormal{\emph{Oper}}}\Cendnote{\textnormal{Obwohl \textcolor{blue}{Schnitzler} wohl zugestimmt hat (vgl. Paul Goldmann an Arthur Schnitzler, 6. 10. [1902]), ist keine
                  entsprechende Oper des Komponisten \textcolor{blue}{Felix
                     Hugo} bekannt.}}}\label{K_L03223-3h} zu machen. Er will ſich nicht direkt an {\pb}Dich wenden, weil \textcolor{blue}{er}{}\ledrightnote{{$\rightarrow$}\textcolor{blue}{Hugo Felix}} fürchtet, Du würdeſt ihm gegenüber, auch
               wenn Dir der Vorſchlag nicht paßte, mit der Sprache nicht heraus wollen, um ihn nicht
                  \strikeout{\textcolor{gray}{kra}} zu kränken, und würdeſt Dich ſo gebunden fühlen, ſeine Bitte bejahend zu
               beantworten. Darum hat er mich um meine Vermittelung gebeten, die ich gern übernehme,
               weil ich überzeugt bin, daß Gutes für beide Theile herauskommen würde, wenn die
               Angelegenheit {\pb}ſich arrangiren ließe. Ich bitte um
               eine möglichſt umgehende Antwort, da ich Montag{ }Abend mit \textsc{\textcolor{blue}{Felix}{}\ledrightnote{\textcolor{blue}{Hugo Felix}}} zuſammenſein ſoll und ihm einen Beſcheid bringen möchte.\pend
           
\pstart
           Ich danke Dir für die Empfehlung der \label{K_L03223-12v}\edtext{\textcolor{green}{Werk}{}\ledrightnote{{$\rightarrow$}\textcolor{green}{Schatten des Todes}}e von \textsc{\textcolor{blue}{Tschechow}{}\ledrightnote{\textcolor{blue}{Anton Pavlovič Čechov}}}}{\lemma{\textnormal{\emph{Werke von Tschechow}}}\Cendnote{\textnormal{vermutlich die Novelle \emph{\textcolor{green}{Schatten des Todes}}, die \textcolor{blue}{Schnitzler} soeben gelesen hatte (vgl. A. S.: \emph{Tagebuch}, 26. 8. 1902), und andere}}}\label{K_L03223-12h}. Ich entdecke dieſer
               Tage ein entzückendes franzöſiſches \textcolor{green}{Aphorismen-{\pb}Buch}{}\ledrightnote{\textcolor{green}{Maximes de la vie. Préface par Sully Prud’homme}}{ }\label{K_L03223-4v}\edtext{»\textsc{\textcolor{green}{Maximes de la vie}{}\ledrightnote{\textcolor{green}{Maximes de la vie. Préface par Sully Prud’homme}}}« von \textsc{\textcolor{blue}{Comtesse Diane}{}\ledrightnote{\textcolor{blue}{Marie Suin Beausacq}}}}{\lemma{\textnormal{\emph{»Maximes … Diane}}}\Cendnote{\textnormal{\textcolor{blue}{Comtesse Diane} [ = \textcolor{blue}{Marie Suin Beausacq}]: \emph{\textcolor{green}{Maximes de la vie. Préface par \textcolor{blue}{Sully
                           Prud’homme}}}. \textcolor{pink}{Paris}: \emph{\textcolor{brown}{P. Ollendorf}}{ }1883. Eine Lektüre durch \textcolor{blue}{Schnitzler} ist
                  nicht bekannt.}}}\label{K_L03223-4h}. Laß’ es Dich die 8 \textsc{MK} nicht
               reuen, die es koſtet; Du wirſt Freude daran haben.\pend
           
\pstart
           Ich hoffe, daß \textsc{\textcolor{blue}{Olga}{}\ledrightnote{\textcolor{blue}{Olga Schnitzler}}} bald \label{K_L03223-5v}\edtext{wiederhergeſtellt}{\lemma{\textnormal{\emph{wiederhergeſtellt}}}\Cendnote{\textnormal{vgl. A. S.: \emph{Tagebuch}, 30. 9. 1902}}}\label{K_L03223-5h} ſein wird\textcolor{gray}{,} bitte, ſie vielmals von mir zu grüßen, {\pb}und begrüße auch Dich auf das Herzlichſte. {\\[\baselineskip]}Dein {\\[\baselineskip]}\spacefill\mbox{Paul Goldmnn}\pend
           \leftskip=0em{}
\pstart
           \noindent{}Ich würde Dir dankbar ſein, wenn Du mir mittheilen wollteſt, welchen Eindruck die
                     \label{K_L03223-6v}\edtext{»\textcolor{green}{\textcolor{brown}{Zeit}{}\ledrightnote{\textcolor{brown}{Die Zeit. Wiener Wochenschrift}}}{}\ledrightnote{\textcolor{green}{Die Zeit. Wiener Wochenschrift}}«}{\lemma{\textnormal{\emph{»Zeit«}}}\Cendnote{\textnormal{Bezug auf die Umstellung der
                        \emph{\textcolor{green}{Zeit}} auf ein \textcolor{green}{Tagesblatt}, siehe Paul Goldmann an Arthur Schnitzler und Olga
               Gussmann, 7. 7. [1901]}}}\label{K_L03223-6h} auf Dich und überhaupt in \textcolor{pink}{Wien}{}\ledrightnote{\textcolor{pink}{Wien}}
                  macht?\pend
           \endnumbering\briefempfaengerindex{Schnitzler, Arthur@\textsc{Schnitzler, Arthur}!zzzGoldmann, Paul@\emph{von Paul Goldmann}!1902-10-021@{2. {[}10. 1902{]}}|)be}\mylabel{h}
\begin{anhang}
\end{anhang}\normalsize

\doendnotes{C}
\bigskip
\vfill

\clearpage

\footnotesize

\lohead{\textsc{register}}

% Definiere theindex-Environment komplett neu ohne reledmac
\makeatletter
\renewenvironment{theindex}{%
  \section*{\indexname}%
  \setlength{\parindent}{0pt}%
  \setlength{\parskip}{0pt plus 0.3pt}%
  \let\item\@idxitem
}{%
  \clearpage
}
\makeatother

\IfFileExists{\jobname-pw.ind}{\input{\jobname-pw.ind}}{}

\end{document}

      