%% latex-korrekturansicht-vorspann.tex
%% Vorspann für die Korrekturansicht.
%% Lädt die gemeinsame Datei latex-vorspann.tex mit gesetztem Schalter.

\newif\ifkorrekturansicht
\korrekturansichttrue

\input{../tex-inputs/latex-vorspann}


\renewcommand{\erwaehntePersonen}{Personen: Hugo Felix, Ludwig Fulda, Paul Lindau, Alfred de Musset, Paul Schlenther, Olga Schnitzler}
\renewcommand{\erwaehnteInstitutionen}{Institutionen: Berliner Theater, Burgtheater, Neue Freie Presse, Rütten {\kaufmannsund}  Loening, Volkstheater}
\renewcommand{\erwaehnteOrte}{Orte: Berlin, Dessauer Straße, Frankfurt am Main, Lessing-Theater, Prag, Ständetheater, Volkstheater, Wien}
\renewcommand{\erwaehnteWerke}{Werke: Der Schleier der Beatrice. Schauspiel in fünf Akten, Der einsame Weg. Schauspiel in fünf Akten, Die Zeit, Il ne faut jurer de rien, Kaltwasser. Lustspiel in drei Aufzügen, Man soll nichts verschwören. Komödie}
\section[ Paul Goldmann an Arthur Schnitzler, 6. 10. {[}1902{]}]{Paul Goldmann an Arthur Schnitzler, 6. 10. {[}1902{]}}
\nopagebreak\mylabel{v}
\rehead{ }\normalsize\beginnumbering\briefempfaengerindex{Schnitzler, Arthur@\textsc{Schnitzler, Arthur}!zzzGoldmann, Paul@\emph{von Paul Goldmann}!1902-10-061@{6. 10. {[}1902{]}}|(be}
\toendnotes[C]{\smallbreak\pagebreak[2]}\Standort{DLA, A:Schnitzler, HS.NZ85.1.3172.}
\physDesc{Brief, 1 Blatt, 2 Seiten
\newline{}Handschrift: blaue Tinte, deutsche Kurrent
\newline{}Schnitzler: 1) mit Bleistift das Jahr »{[}1{]}902« vermerkt  2) mit rotem Buntstift vier Unterstreichungen}\toendnotes[C]{\smallbreak}
\pstart
           \noindent{}\raggedleft{}{\pb}\textcolor{pink}{\textcolor{gray}{\textbf{DESSAUERSTRASSE 19}}}{}\ledrightnote{\textcolor{pink}{Dessauer Straße}}\pend
           
\pstart
           \textcolor{pink}{Berlin}{}\ledrightnote{\textcolor{pink}{Berlin}}, 6. Okt.\pend
           
\pstart\center{}Mein lieber Freund,\pend
\pstart
           Mit \textsc{\textcolor{blue}{Lindau}{}\ledrightnote{\textcolor{blue}{Paul Lindau}}} ſtehe ich gegenwärtig ſehr ſchlecht. Die Gründe erzähle ich Dir mündlich. Ich
               kann ihm alſo das \label{K_L03225-32v}\edtext{\textcolor{green}{Stück}{}\ledrightnote{{$\rightarrow$}\textcolor{green}{Der einsame Weg. Schauspiel in fünf Akten}}}{\lemma{\textnormal{\emph{Stück}}}\Cendnote{\textnormal{Von welchem Stück die Rede ist, ist ungeklärt. Es dürfte
                  sich nicht um \emph{\textcolor{green}{Der Schleier der Beatrice}}
                  handeln, da \textcolor{blue}{Paul Lindau} bereits in einem
                  Brief an \textcolor{blue}{Schnitzler} vom
                     11. 9. 1900 das \textcolor{green}{Stück} für das \emph{\textcolor{brown}{Berliner
                     Theater}} abgelehnt hatte. (\emph{Cambridge University
                        Library}, B 60.) Eventuell handelt es sich um das zum
                  Zeitpunkt noch nicht fertig gestellte nächste Stück, \emph{\textcolor{green}{Der einsame Weg}}, an dessen viertem Akt er zuletzt
                  arbeitete. }}}\label{K_L03225-32h} einſtweilen nicht einreichen. Aber wie \textsc{\textcolor{blue}{Lindau}{}\ledrightnote{\textcolor{blue}{Paul Lindau}}} ſchon iſt, kann ſich die Situation raſch ändern; und dann ſtehe ich
               ſelbſtverſtändlich zu Deiner Verfügung.\pend
           
\pstart
           \label{K_L03225-2v}\edtext{\textsc{\textcolor{blue}{Felix}{}\ledrightnote{\textcolor{blue}{Hugo Felix}}}}{\lemma{\textnormal{\emph{Felix}}}\Cendnote{\textnormal{siehe Paul Goldmann an Arthur Schnitzler, 2. [10. 1902]}}}\label{K_L03225-2h} habe ich Deine Antwort {\pb}übermittelt; er ſandte
               mir ein ganz beglücktes Telegramm.\pend
           
\pstart
           \label{K_L03225-3v}\edtext{\textsc{\textcolor{blue}{\textcolor{green}{Fulda}{}\ledrightnote{{$\rightarrow$}\textcolor{green}{Kaltwasser. Lustspiel in drei Aufzügen}}}{}\ledrightnote{\textcolor{blue}{Ludwig Fulda}}}}{\lemma{\textnormal{\emph{Fulda}}}\Cendnote{\textnormal{\textcolor{blue}{Ludwig
                     Fulda}s dreiaktiges Lustspiel \emph{\textcolor{green}{Kaltwasser}} hatte am 5. 10. 1902
                  Uraufführung am \textcolor{pink}{Berlin}er \textcolor{pink}{Lessing-Theater}.}}}\label{K_L03225-3h} iſt bös durchgefallen.\pend
           
\pstart
           Kann ich die \label{K_L03225-4v}\edtext{\textsc{\textcolor{blue}{\textcolor{green}{Musset}{}\ledrightnote{{$\rightarrow$}\textcolor{green}{Il ne faut jurer de rien}}}{}\ledrightnote{\textcolor{blue}{Alfred de Musset}}}-\textcolor{green}{Überſetzung}{}\ledrightnote{{$\rightarrow$}\textcolor{green}{Man soll nichts verschwören. Komödie}}}{\lemma{\textnormal{\emph{Musset-Überſetzung}}}\Cendnote{\textnormal{\textcolor{blue}{Alfred de Musset}: \emph{\textcolor{green}{Man soll nichts verschwören}}. Aus dem Französischen von
                        \textcolor{blue}{Paul Goldmann}. \textcolor{pink}{Frankfurt am Main}: \emph{\textcolor{brown}{Rütten
                           {\kaufmannsund} Loening}}{ }1902. Die Uraufführung des \textcolor{green}{Stück}s in der \textcolor{green}{Übersetzung}{ }\textcolor{blue}{Goldmann}s fand am 5. 3. 1903 im \textcolor{pink}{Deutschen
                     Landestheater} in \textcolor{pink}{Prag} statt. Eine
                  Aufführung am \textcolor{pink}{Wien}er \textcolor{pink}{Volkstheater} fand nicht statt.}}}\label{K_L03225-4h} dem \textcolor{brown}{Volkstheater}{}\ledrightnote{\textcolor{brown}{Volkstheater}} einreichen? Mit \textsc{\textcolor{brown}{\textcolor{blue}{Schlenther}{}\ledrightnote{\textcolor{blue}{Paul Schlenther}}}{}\ledrightnote{{$\rightarrow$}\textcolor{brown}{Burgtheater}}} will ich nichts zu thun haben.\pend
           
\pstart
           Iſt \label{K_L03225-5v}\edtext{\textsc{\textcolor{blue}{Olga}{}\ledrightnote{\textcolor{blue}{Olga Schnitzler}}} wieder ganz geſund}{\lemma{\textnormal{\emph{Olga wieder ganz geſund}}}\Cendnote{\textnormal{siehe Paul Goldmann an Arthur Schnitzler, 2. [10. 1902]}}}\label{K_L03225-5h}?\pend
           
\pstart
           Ich denke auch, die \label{K_L03225-6v}\edtext{»\textcolor{green}{Zeit}{}\ledrightnote{\textcolor{green}{Die Zeit}}«}{\lemma{\textnormal{\emph{»Zeit«}}}\Cendnote{\textnormal{siehe Paul Goldmann an Arthur Schnitzler, 16. 9. [1902]}}}\label{K_L03225-6h} wird ſich noch ſehr gut machen. Die \label{K_L03225-7v}\edtext{\textcolor{brown}{N. Fr. Pr.}{}\ledrightnote{\textcolor{brown}{Neue Freie Presse}}}{\lemma{\textnormal{\emph{N. Fr. Pr.}}}\Cendnote{\textnormal{In welcher spezifischen Weise bei der
                     \emph{\textcolor{brown}{Neuen Freie Presse}} in den ersten zwei
                  Wochen nach dem ersten Erscheinen der ersten Nummer der Tageszeitung \emph{\textcolor{green}{Die Zeit}} Entspannung eingetreten ist, ließ
                  sich nicht ermitteln. Vgl. Paul Goldmann an Arthur Schnitzler und Olga
               Gussmann, 7. 7. [1901]. }}}\label{K_L03225-7h} frohlockt zu früh. Viele treue Grüße!\pend
           \pstart Dein \spacefill\mbox{Paul Goldm}\pend{}\endnumbering\briefempfaengerindex{Schnitzler, Arthur@\textsc{Schnitzler, Arthur}!zzzGoldmann, Paul@\emph{von Paul Goldmann}!1902-10-061@{6. 10. {[}1902{]}}|)be}\mylabel{h}
\begin{anhang}
\end{anhang}\normalsize

\doendnotes{C}
\bigskip
\vfill

\clearpage

\footnotesize

\lohead{\textsc{register}}

% Definiere theindex-Environment komplett neu ohne reledmac
\makeatletter
\renewenvironment{theindex}{%
  \section*{\indexname}%
  \setlength{\parindent}{0pt}%
  \setlength{\parskip}{0pt plus 0.3pt}%
  \let\item\@idxitem
}{%
  \clearpage
}
\makeatother

\IfFileExists{\jobname-pw.ind}{\input{\jobname-pw.ind}}{}

\end{document}

      