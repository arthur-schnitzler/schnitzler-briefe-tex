%% latex-korrekturansicht-vorspann.tex
%% Vorspann für die Korrekturansicht.
%% Lädt die gemeinsame Datei latex-vorspann.tex mit gesetztem Schalter.

\newif\ifkorrekturansicht
\korrekturansichttrue

\input{../tex-inputs/latex-vorspann}


\section[Stefan Zweig an Arthur Schnitzler, {[}zwischen 7. und 10. 2. 1915?{]}]{L03651 Stefan Zweig an Arthur Schnitzler, {[}zwischen 7. und 10. 2. 1915?{]}}
\nopagebreak\mylabel{L03651v}
\rehead{ }\normalsize\beginnumbering\briefempfaengerindex{, @\textsc{, }!zzz, @\emph{von  }!1915-02-101@{{[}zwischen 7. und 10. 2. 1915?{]}}|(be}
\toendnotes[C]{\smallbreak\pagebreak[2]}\Standort{CUL, Schnitzler, B 118.}
\physDesc{Briefkarte, 1038 Zeichen
\newline{}Handschrift: lila Tinte, lateinische Kurrent
\newline{}Schnitzler: mit rotem Buntstift zwei Unterstreichungen }
\buchAbdrucke{\weitereDrucke{Stefan Zweig: \emph{Briefwechsel mit Hermann Bahr, Sigmund Freud, Rainer Maria
                        Rilke und Arthur Schnitzler}. Herausgegeben von Jeffrey B. Berlin, Hans-Ulrich Lindken und Donald A. Prater. Frankfurt am Main: \emph{S. Fischer} 1987, S. 390–391.} }\toendnotes[C]{\smallbreak}
\pstart
           {\pb}\textcolor{gray}{\textbf{SZ}}\hfill \textcolor{gray}{\textbf{\textcolor{pink}{VIII. KOCHGASSE 8}\oindex{Wien@\textbf{Wien}!VIII., Josefstadt@\textbf{VIII., Josefstadt}!Kochgasse 8@\textbf{Kochgasse 8}, \emph{Wohngebäude}|pw}{}\ledrightnote{\textcolor{pink}{Kochgasse 8}}}}\pend
           \vspace{0.5em}
\pstart
           Verehrter lieber Herr Doktor,{ }\textcolor{blue}{Romain Rolland}\pwindex{Rolland, Romain 29.\,1.\,1866 Clamecy – 30.\,12.\,1944 Vézelay@\textsc{Rolland, Romain} (29.\,1.\,1866 Clamecy – 30.\,12.\,1944 Vézelay), \emph{Schriftsteller}|pw}{}\ledrightnote{\textcolor{blue}{Romain Rolland}} hat mir endlich wieder einen
                  \label{K_L03651-1v}\edtext{Brief}{\lemma{\textnormal{\emph{Brief}}}\Cendnote{\textnormal{Das vorliegende Korrespondenzstück ist undatiert, wird aber
                  durch das erwähnte Schreiben \textcolor{blue}{Rollands}\pwindex{Rolland, Romain 29.\,1.\,1866 Clamecy – 30.\,12.\,1944 Vézelay@\textsc{Rolland, Romain} (29.\,1.\,1866 Clamecy – 30.\,12.\,1944 Vézelay), \emph{Schriftsteller}|pwk}, das
                  mit 5. 2. 1915 datiert ist, zeitlich verortet. Nimmt man an, dass der
                  Postweg in Kriegszeiten etwas länger dauert als gewöhnlich, so ist ein Erhalt des
                  Schreibens durch \textcolor{blue}{Zweig}\pwindex{Zweig, Stefan 28.\,11.\,1881 Wien – 23.\,2.\,1942 Petrópolis@\textsc{Zweig, Stefan} (28.\,11.\,1881 Wien – 23.\,2.\,1942 Petrópolis), \emph{Schriftsteller}|pwk} und in Folge die
                  vorliegende Benachrichtigung von \textcolor{blue}{Schnitzler}
                  für den Zeitraum zwischen 7. 2. 1915 und 10. 2. 1915
                  wahrscheinlich, da bei der Antwort am Folgetag \textcolor{blue}{Schnitzler} bereits an \textcolor{blue}{Rolland}\pwindex{Rolland, Romain 29.\,1.\,1866 Clamecy – 30.\,12.\,1944 Vézelay@\textsc{Rolland, Romain} (29.\,1.\,1866 Clamecy – 30.\,12.\,1944 Vézelay), \emph{Schriftsteller}|pwk}
                  geschrieben hatte (vgl. Arthur Schnitzler an Stefan Zweig, 11. 2. 1915).}}}\label{K_L03651-1} ungehindert schreiben können. Er spricht auch von Ihnen darin – er hat
               offenbar in \textcolor{green}{Berl. Tag.}\pwindex{Berliner Tageblatt@\emph{Berliner Tageblatt}|pw}{}\ledrightnote{\textcolor{green}{Berliner Tageblatt}} jenen \label{K_L03651-2v}\edtext{\textcolor{green}{Artikel}\pwindex{Kultur–?@\emph{Kultur–?}|pwv}{}\ledrightnote{{$\rightarrow$}\emph{\textcolor{green}{Kultur–?}}}}{\lemma{\textnormal{\emph{Artikel}}}\Cendnote{\textnormal{\textcolor{blue}{P. B.}\pwindex{Block, Paul 30.\,5.\,1862 Klaipėda – 15.\,8.\,1934 Bad Harzburg@\textsc{Block, Paul} (30.\,5.\,1862 Klaipėda – 15.\,8.\,1934 Bad Harzburg), \emph{Schriftsteller, Journalist}|pwk} [= \textcolor{blue}{Paul Block}\pwindex{Block, Paul 30.\,5.\,1862 Klaipėda – 15.\,8.\,1934 Bad Harzburg@\textsc{Block, Paul} (30.\,5.\,1862 Klaipėda – 15.\,8.\,1934 Bad Harzburg), \emph{Schriftsteller, Journalist}|pwk}]: \emph{\textcolor{green}{Kultur–?}\pwindex{Kultur–?@\emph{Kultur–?}|pwk}} In: \emph{\textcolor{green}{Berliner Tageblatt}\pwindex{Berliner Tageblatt@\emph{Berliner Tageblatt}|pwk}}, Jg. 44, Nr. 33,
                        21. 1. 1915, Abendausgabe, S. [3]. Ausgangspunkt für \textcolor{blue}{Paul Block}\pwindex{Block, Paul 30.\,5.\,1862 Klaipėda – 15.\,8.\,1934 Bad Harzburg@\textsc{Block, Paul} (30.\,5.\,1862 Klaipėda – 15.\,8.\,1934 Bad Harzburg), \emph{Schriftsteller, Journalist}|pwk} war die ungezeichnete, kritische
                  Meldung \emph{\textcolor{green}{Schnitzler erhebt Einspruch}\pwindex{Schnitzler erhebt Einspruch@\emph{Schnitzler erhebt Einspruch}|pwk}} in der
                     \emph{\textcolor{green}{Deutschen Tageszeitung}\pwindex{Deutsche Tageszeitung@\emph{Deutsche Tageszeitung}|pwk}} (Jg. 22,
                     Nr. 24, 19. 1. 1915, Abend-Ausgabe, S. 5). In ihr wird \textcolor{blue}{Schnitzler} vorgeworfen, er habe sich mit
                  seinem Protest \emph{\textcolor{green}{Une protestation d’Arthur
                     Schnitzler}\pwindex{Schnitzler, Arthur 15. 5. 1862 Wien – 21. 10. 1931 ebd.@\textsc{Schnitzler, Arthur} (15. 5. 1862 Wien – 21. 10. 1931 ebd.), \emph{Schriftsteller, Mediziner}!Une protestation d’Arthur Schnitzler@\strich\emph{Une protestation d’Arthur Schnitzler}|pwk}} von \textcolor{blue}{Rolland}\pwindex{Rolland, Romain 29.\,1.\,1866 Clamecy – 30.\,12.\,1944 Vézelay@\textsc{Rolland, Romain} (29.\,1.\,1866 Clamecy – 30.\,12.\,1944 Vézelay), \emph{Schriftsteller}|pwk}
                  instrumentalisieren lassen, um die »die innere Uneinigkeit der deutschen
                     Geister« international sichtbar zu machen. \textcolor{blue}{Block}\pwindex{Block, Paul 30.\,5.\,1862 Klaipėda – 15.\,8.\,1934 Bad Harzburg@\textsc{Block, Paul} (30.\,5.\,1862 Klaipėda – 15.\,8.\,1934 Bad Harzburg), \emph{Schriftsteller, Journalist}|pwk} wendet das Argument gegen den anonymen Verfasser der
                     \textcolor{green}{Meldung}\pwindex{Schnitzler erhebt Einspruch@\emph{Schnitzler erhebt Einspruch}|pwkv}, indem er ihm
                  vorhält, gerade die Uneinigkeit unter die deutschsprachigen Schriftsteller tragen
                  zu wollen. \textcolor{blue}{Schnitzler} erhielt auch durch
                  seinen Verleger \textcolor{blue}{S. Fischer}\pwindex{Fischer, Samuel 24.\,12.\,1859 Liptovský Mikuláš – 15.\,10.\,1934 Berlin@\textsc{Fischer, Samuel} (24.\,12.\,1859 Liptovský Mikuláš – 15.\,10.\,1934 Berlin), \emph{Verleger}|pwk} am
                     25. 1. 1915 Nachricht von \emph{\textcolor{green}{Kultur–?}\pwindex{Kultur–?@\emph{Kultur–?}|pwk}}: »Lieber Freund,{ / }in der Abendausgabe des \textcolor{green}{Berliner
                           Tageblattes}\pwindex{Berliner Tageblatt@\emph{Berliner Tageblatt}|pw} vom 21. d. M. habe ich die anliegende \textcolor{green}{Notiz}\pwindex{Kultur–?@\emph{Kultur–?}|pwv} gefunden, die
                        wohl inzwischen schon zu Ihrer Kenntnis gelangt ist. Ganz klar ist aus
                        dieser \textcolor{green}{Notiz}\pwindex{Kultur–?@\emph{Kultur–?}|pwv} nicht
                        zu ersehen, ob \textcolor{blue}{Romain Rolland}\pwindex{Rolland, Romain 29.\,1.\,1866 Clamecy – 30.\,12.\,1944 Vézelay@\textsc{Rolland, Romain} (29.\,1.\,1866 Clamecy – 30.\,12.\,1944 Vézelay), \emph{Schriftsteller}|pw} mit der
                        Veröffentlichung Ihres \textcolor{green}{Briefes}\pwindex{Schnitzler, Arthur 15. 5. 1862 Wien – 21. 10. 1931 ebd.@\textsc{Schnitzler, Arthur} (15. 5. 1862 Wien – 21. 10. 1931 ebd.), \emph{Schriftsteller, Mediziner}!Une protestation d’Arthur Schnitzler@\strich\emph{Une protestation d’Arthur Schnitzler}|pwv} nur seinerseits einleitende Bemerkungen verbunden hat oder
                        ob er – was nicht anzunehmen ist – in Ihrem Briefe Aenderungen und
                        Korrekturen vorgenommen hat. Für alle Fälle stelle ich mich Ihnen zur
                        Verfügung, wenn Sie das Bedürfnis haben sollten, in dieser Sache etwas zu
                        veröffentlichen oder zu berichtigen. Auch würde evtl. die ›\textcolor{green}{Neue Rundschau}\pwindex{neue Rundschau@\emph{Die neue Rundschau}|pw}‹ wenn Sie es für zweckmässig halten,
                        von der Angelegenheit Notiz nehmen können.{ / }Mit herzlichen Grüssen{ / }Ihr{ / }SFischer.« Am 27. 1. 1915 schrieb \textcolor{blue}{Schnitzler} an \textcolor{blue}{Block}\pwindex{Block, Paul 30.\,5.\,1862 Klaipėda – 15.\,8.\,1934 Bad Harzburg@\textsc{Block, Paul} (30.\,5.\,1862 Klaipėda – 15.\,8.\,1934 Bad Harzburg), \emph{Schriftsteller, Journalist}|pwk}:
                        »Sehr geehrter Herr Block.{ / }Darf ich Ihnen die Hand drücken für die liebenswürdige und vornehme Art, in
                        der Sie sich meiner gegen einen (mir bisher nicht vor Augen gekommenen) \textcolor{green}{Angriff}\pwindex{Schnitzler erhebt Einspruch@\emph{Schnitzler erhebt Einspruch}|pwv} in der \textcolor{green}{D. T. Z.}\pwindex{Deutsche Tageszeitung@\emph{Deutsche Tageszeitung}|pw}{ }\textcolor{green}{angenommen}\pwindex{Kultur–?@\emph{Kultur–?}|pwv} haben?
                        Wenn ich auch glaube, daß Sie dem Herrn Anonymus mit Ihrer schönen
                        Erwiderung eine unverdiente Ehre erwiesen haben; mich hat sie natürlich
                        trotzdem sehr gefreut. Ich für meinen Teil habe mich begreiflicher Weise
                        niemals entschließen können, auf all die antisemitischen Verdrehungen,
                        Begeiferungen und Verleumdungen, die ich im Laufe einer mehr als
                        zwanzigjährigen schriftstellerischen Tätigkeit erfahren habe, ein Wort zu
                        entgegnen. Doch habe ich mir eine sehr hübsche (freilich, wie sich immer
                        wieder zeigt, nicht vollständige) Sammlung von derlei Zeug angelegt, das
                        vielleicht einmal als ein ganz bescheidenes Dokument von unserer Zeiten
                        Schande neben bedeutenderen wird bestehen können. Auch hier hat es diesmal
                        an Blättchen nicht gefehlt, die Inhalt und Absicht meines (mit gutem Grund
                        in der neutralen \textcolor{pink}{Schweiz}\oindex{Schweiz@\textbf{Schweiz}|pw} sowohl im
                        französischen \textcolor{green}{Journal de Genève}\pwindex{Journal de Genève@\emph{Journal de Genève}|pw}, als in der
                        deutschen \textcolor{green}{Zürcher Zeitung}\pwindex{Neue Zürcher Zeitung@\emph{Neue Zürcher Zeitung}|pw} veröffentlichten) \textcolor{green}{Protestes}\pwindex{Schnitzler, Arthur 15. 5. 1862 Wien – 21. 10. 1931 ebd.@\textsc{Schnitzler, Arthur} (15. 5. 1862 Wien – 21. 10. 1931 ebd.), \emph{Schriftsteller, Mediziner}!Une protestation d’Arthur Schnitzler@\strich\emph{Une protestation d’Arthur Schnitzler}|pwv}\pwindex{Schnitzler, Arthur 15. 5. 1862 Wien – 21. 10. 1931 ebd.@\textsc{Schnitzler, Arthur} (15. 5. 1862 Wien – 21. 10. 1931 ebd.), \emph{Schriftsteller, Mediziner}!Brief Artur Schnitzlers@\strich\emph{Ein Brief Artur Schnitzlers}|pwv} gegen das mir in
                        einer russischen Zeitung angedichtete \textcolor{green}{Interview}\pwindex{?? [Journalist, der fiktives russisches Interview verantwortet] @\textsc{?? [Journalist, der fiktives russisches Interview verantwortet]}!?? [Fiktives Interview aus St. Petersburg, 1914]@\strich\emph{?? [Fiktives Interview aus St. Petersburg, 1914]}|pwv}, tückisch-albern umzudeuten versuchten, ohne daß es mich
                        in Erstaunen gesetzt hätte. Denn ich hatte keinen Moment lang erwartet, daß
                        eine Sorte von Zeitungsschreibern, deren ganze Existenzberechtigung und
                        Existenzmöglichkeit sich in ruhigeren Zeiten nur durch fleißig betriebene
                        Rassen- und Konfessionshetze zu erweisen vermochte, einen Burgfrieden zu
                        halten entschlossen wäre, in dem ihre traurige Eigenart sich naturgemäß gar
                        nicht betätigen könnte. Aber man darf wohl sagen: wenn es heute in \textcolor{pink}{deutschen}\oindex{Deutschland@\textbf{Deutschland}|pw} und \textcolor{pink}{österreichischen}\oindex{Österreich@\textbf{Österreich}|pw} Landen irgendwo noch Verräter gibt, so sind es
                        gewiß vor allen Andern Leute, die in diesen Tagen antisemitische Politik
                        treiben und so in den Einen allmälig doch erlöschende Gefühle der
                        Feindseligkeit oder Fremdheit, in den Andern bittere und erbitternde Zweifel
                        an einer durch Arbeit, Blut und Heimatliebe dreifach besiegelten
                        Dazugehörigkeit immer von Neuem in Schwingung zu bringen versuchen. Ich
                        wollte, man verstünde das bei uns überall so gut, als es bei Ihnen, wie ich
                        aus manchen an maßgebenden Stellen getanen Äußerungen entnehmen darf, zu
                        geschehen scheint. Verzeihen Sie, wenn ich Sie nun doch, verehrtester Herr
                        Block, bitte, diesen Brief als einen privaten zu betrachten. So sehr ich
                        ihn, wie alles, was ich schreibe, nach Sinn und Wort durchaus zu vertreten
                        imstande bin, – ich möchte jetzt nicht schuld sein, daß sich weitere
                        Diskussionen an jene ganz untendenziöse Erklärung knüpften, zu deren \textcolor{green}{Veröffentlichung}\pwindex{Schnitzler, Arthur 15. 5. 1862 Wien – 21. 10. 1931 ebd.@\textsc{Schnitzler, Arthur} (15. 5. 1862 Wien – 21. 10. 1931 ebd.), \emph{Schriftsteller, Mediziner}!Une protestation d’Arthur Schnitzler@\strich\emph{Une protestation d’Arthur Schnitzler}|pwv}\pwindex{Schnitzler, Arthur 15. 5. 1862 Wien – 21. 10. 1931 ebd.@\textsc{Schnitzler, Arthur} (15. 5. 1862 Wien – 21. 10. 1931 ebd.), \emph{Schriftsteller, Mediziner}!Brief Artur Schnitzlers@\strich\emph{Ein Brief Artur Schnitzlers}|pwv} in \textcolor{green}{Schweizer Blättern}\pwindex{Journal de Genève@\emph{Journal de Genève}|pwv}\pwindex{Neue Zürcher Zeitung@\emph{Neue Zürcher Zeitung}|pwv} ich nach der
                        ganzen Sachlage mich gedrungen sah. { / }Nochmals herzlich dankend und grüßend{ / }Ihr sehr ergebener« (\textcolor{blue}{Schnitzler}: \emph{Briefe 1913–1931},
                     S. 76–77). Es sei kurz darauf hingewiesen, dass die \emph{\textcolor{green}{Deutsche Tageszeitung}\pwindex{Deutsche Tageszeitung@\emph{Deutsche Tageszeitung}|pwk}} am 15. 10. 1915 in Form
                  einer \textcolor{green}{Rezension von \emph{\textcolor{green}{Komödie der Worte}\pwindex{Schnitzler, Arthur 15. 5. 1862 Wien – 21. 10. 1931 ebd.@\textsc{Schnitzler, Arthur} (15. 5. 1862 Wien – 21. 10. 1931 ebd.), \emph{Schriftsteller, Mediziner}!Komödie der Worte. Drei Einakter@\strich\emph{Komödie der Worte. Drei Einakter}|pwk}}}\pwindex{Komödie der Worte@\emph{Komödie der Worte}|pwkv} einen weiteren anonymen Angriff auf \textcolor{blue}{Schnitzler} publizierte. }}}\label{K_L03651-2} gegen und für Sie gelesen – und schreibt
                  »\label{K_L03651-3v}\edtext{\begin{otherlanguage}{french}Le voici logé a la même enseigne en \textcolor{pink}{Allemagne}\oindex{Deutschland@\textbf{Deutschland}|pw}{}\ledrightnote{\textcolor{pink}{Deutschland}} que je le suis en \textcolor{pink}{France}\oindex{Frankreich@\textbf{Frankreich}|pw}{}\ledrightnote{\textcolor{pink}{Frankreich}}! Exprimez lui de ma part toute ma sympathie confraternelle – si
                  toutefois elle {\pb}ne le
                     comp{[}r{]}omet pas encore plus. Ah que les gens sont fous!
                  C’est presque comique.\end{otherlanguage}}{\lemma{\textnormal{\emph{Le … comique.}}}\Cendnote{\textnormal{französisch: Und wie finden Sie, was
                  unserem armen \textcolor{blue}{Arthur Schnitzler} widerfahren
                  ist? Da gerät er in \textcolor{pink}{Deutschland}\oindex{Deutschland@\textbf{Deutschland}|pwk} in die gleiche
                  Lage wie ich in \textcolor{pink}{Frankreich}\oindex{Frankreich@\textbf{Frankreich}|pwk}! Drücken Sie ihm
                  meine brüderliche Verbundenheit aus, falls ihn das nicht noch mehr kompromittiert.
                  Ach, wie töricht die Menschen doch sind! Es ist schon fast komisch. (Zitiert
                     nach \textcolor{blue}{Romain Rolland}\pwindex{Rolland, Romain 29.\,1.\,1866 Clamecy – 30.\,12.\,1944 Vézelay@\textsc{Rolland, Romain} (29.\,1.\,1866 Clamecy – 30.\,12.\,1944 Vézelay), \emph{Schriftsteller}|pwk}, \textcolor{blue}{Stefan Zweig}\pwindex{Zweig, Stefan 28.\,11.\,1881 Wien – 23.\,2.\,1942 Petrópolis@\textsc{Zweig, Stefan} (28.\,11.\,1881 Wien – 23.\,2.\,1942 Petrópolis), \emph{Schriftsteller}|pwk}: \emph{Von Welt zu Welt. Briefe
                        einer Freundschaft 1914–1918}. Mit einem Begleitwort von Peter
                     Handke. Aus dem Französischen von Eva und Gerhard Schwewe (Briefe Rollands) und
                     Christel Gersch (Briefe Zweigs). Berlin: \emph{Aufbau
                        Verlag}{ }2014.) }}}\label{K_L03651-3}« Wirklich – dieser Versuch auch Sie jetzt einzubeziehen in den
               grossen deutschen Bann war schon zu ärgerlich! Wird man all dies in zehn Jahren noch
               verstehen können? Ich denke Ihrer oft und in Herzlichkeit; hoffentlich hat die
               tragische Monotonie der andauernden bewegungslosen Kämpfe auch in Ihnen wieder die
               Arbeit als Gegengewalt hochgebracht\substVorne{}\textsuperscript{{ }und}\substDazwischen{}.\substHinten{} Ich habe keinen bessern Wunsch als Sie wieder schaffend und gesammelt zu
               wissen, dass wenigstens hier im Geistigen etwas Fruchtbares bleibe aus diesen
               sinnlosen Tagen der Vernichtung.\pend
           
\pstart
           Ihnen und Ihrer lieben \textcolor{blue}{Frau}\pwindex{Schnitzler, Olga 17.\,1.\,1882 Wien – 13.\,1.\,1970 Lugano@\textsc{Schnitzler, Olga} (17.\,1.\,1882 Wien – 13.\,1.\,1970 Lugano), \emph{Schauspielerin, Sängerin}|pwv}{}\ledrightnote{{$\rightarrow$}\emph{\textcolor{blue}{Olga Schnitzler}}} herzlich getreu{\\[\baselineskip]}\spacefill\mbox{Stefan Zweig}\pend
           \leftskip=0em{}\selectlanguage{ngerman}\endnumbering\briefempfaengerindex{, @\textsc{, }!zzz, @\emph{von  }!1915-02-071@{{[}zwischen 7. und 10. 2. 1915?{]}}|)be}\mylabel{L03651h}  \normalsize

\doendnotes{C}
\bigskip
\vfill

\clearpage

\footnotesize

\lohead{\textsc{register}}

% Definiere theindex-Environment komplett neu ohne reledmac
\makeatletter
\renewenvironment{theindex}{%
  \section*{\indexname}%
  \setlength{\parindent}{0pt}%
  \setlength{\parskip}{0pt plus 0.3pt}%
  \let\item\@idxitem
}{%
  \clearpage
}
\makeatother

\IfFileExists{\jobname-pw.ind}{\input{\jobname-pw.ind}}{}

\end{document}

      