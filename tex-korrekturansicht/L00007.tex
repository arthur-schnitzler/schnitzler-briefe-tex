%% latex-korrekturansicht-vorspann.tex
%% Vorspann für die Korrekturansicht.
%% Lädt die gemeinsame Datei latex-vorspann.tex mit gesetztem Schalter.

\newif\ifkorrekturansicht
\korrekturansichttrue

\input{../tex-inputs/latex-vorspann}


               \section[Wilhelm Bölsche an Arthur Schnitzler, 25. 10. 1890]{ Wilhelm Bölsche an Arthur Schnitzler, 25. 10. 1890}\nopagebreak\mylabel{v}\rehead{ }\normalsize\beginnumbering\briefempfaengerindex{Schnitzler, Arthur@\textsc{Schnitzler, Arthur}!zzzBoelsche, Wilhelm@\emph{von Wilhelm Bölsche}!1890-10-251@{25. 10. 1890}|(be} \toendnotes[C]{\smallbreak\pagebreak[2]} \Standort{DLA, A:Schnitzler, HS.NZ85.1.2577,1.}
\physDesc{Brief, 1 Blatt, 1 Seite
\newline{}Handschrift: schwarze Tinte, deutsche Kurrent
\newline{}Schnitzler: mit rotem Buntstift nummeriert: »3« }\buchAbdrucke{\weitereDrucke{Wilhelm Bölsche: \emph{Briefwechsel. Mit Autoren der Freien Bühne}. Hg. Gerd-Hermann Susen. Berlin: \emph{Weidler} 2010, S. 669 (Werke und Briefe. Wissenschaftliche Ausgabe, Briefe I).} }\toendnotes[C]{\smallbreak}\pstart
           \raggedleft{}{\pb}25. X. 90.\pend
           \pstart\center{}Verehrter Herr Doktor!\pend\pstart
           Leider haben wir »Gedichten« bei der »\textcolor{green}{Freien
                        Bühne}{}\ledrightnote{\textcolor{green}{Freie Bühne für modernes Leben}}« jetzt \label{K_L00007_1v}\edtext{ganz abgeſchworen}{\lemma{\textnormal{\emph{ganz abgeſchworen}}}\Cendnote{\textnormal{Das letzte Gedicht
                        war knapp vier Monate zuvor in der \emph{\textcolor{green}{Freien
                            Bühne}} in Heft 22 vom 2. 7. 1890 erschienen.}}}\label{K_L00007_1h}
                    und bringen \uline{nur} Proſa. So muß ich alſo Ihr \textcolor{green}{Gedicht}{}\ledrightnote{→\textcolor{green}{Morgenandacht}} auch ablehnen, das
                    übrigens (bei etwas ſtarker Länge) ſeines Reizes nicht entbehrt.\pend
           \pstart
           Mit vorzüglicher Hochachtung{\\[\baselineskip]}\spacefill\mbox{Wilhelm Bölsche.}\pend
           \leftskip=0em{}\endnumbering\briefempfaengerindex{Schnitzler, Arthur@\textsc{Schnitzler, Arthur}!zzzBoelsche, Wilhelm@\emph{von Wilhelm Bölsche}!1890-10-251@{25. 10. 1890}|)be}\mylabel{h}  \normalsize

\doendnotes{C}
\bigskip
\vfill

\clearpage

\footnotesize

\lohead{\textsc{register}}

% Definiere theindex-Environment komplett neu ohne reledmac
\makeatletter
\renewenvironment{theindex}{%
  \section*{\indexname}%
  \setlength{\parindent}{0pt}%
  \setlength{\parskip}{0pt plus 0.3pt}%
  \let\item\@idxitem
}{%
  \clearpage
}
\makeatother

\IfFileExists{\jobname-pw.ind}{\input{\jobname-pw.ind}}{}

\end{document}

      