%% latex-korrekturansicht-vorspann.tex
%% Vorspann für die Korrekturansicht.
%% Lädt die gemeinsame Datei latex-vorspann.tex mit gesetztem Schalter.

\newif\ifkorrekturansicht
\korrekturansichttrue

\input{../tex-inputs/latex-vorspann}


               \section[Richard Beer-Hofmann an Arthur Schnitzler, 21. 8. 1891]{ Richard Beer-Hofmann an Arthur Schnitzler, 21. 8. 1891}\nopagebreak\mylabel{v}\rehead{ }\normalsize\beginnumbering\briefempfaengerindex{Schnitzler, Arthur@\textsc{Schnitzler, Arthur}!zzzBeer-Hofmann, Richard@\emph{von Richard Beer-Hofmann}!1891-08-211@{21. 8. 1891}|(be} \toendnotes[C]{\smallbreak\pagebreak[2]} \Standort{CUL, Schnitzler, B 8.}
\physDesc{Briefkarte
\newline{}Handschrift: schwarze Tinte, lateinische Kurrent
\newline{}Schnitzler: mit Bleistift nummeriert: »4.« }\buchAbdrucke{\weitereDrucke{Arthur Schnitzler, Richard Beer-Hofmann: \emph{Briefwechsel 1891–1931}. Hg. Konstanze Fliedl. Wien, Zürich: \emph{Europaverlag} 1992, S. 32.} }\toendnotes[C]{\smallbreak}\pstart\center{}{\pb}Lieber Arthur!\pend\pstart
           Zwei uns befreundete \textcolor{blue}{Damen}{}\ledrightnote{→\textcolor{blue}{?? [Befreundete Frau 1]}{\newline}→\textcolor{blue}{?? [Befreundete Frau 2]}} – nicht aus \textcolor{pink}{Wien}{}\ledrightnote{\textcolor{pink}{Wien}} – wollen nach \textcolor{pink}{Wien}{}\ledrightnote{\textcolor{pink}{Wien}} von hier aus, um Professor \textcolor{blue}{Kraft-Ebing}{}\ledrightnote{\textcolor{blue}{Richard von Krafft-Ebing}} zu consultiren. Ist \textcolor{blue}{Kraft-Ebing}{}\ledrightnote{\textcolor{blue}{Richard von Krafft-Ebing}} aber jetzt in \textcolor{pink}{Wien}{}\ledrightnote{\textcolor{pink}{Wien}}? Wenn nicht,
               ist bekannt, wann er zurückkehrt? Bitte antworten Sie mir bald. Bez. meiner Wenigkeit
               ist noch kein Entschluss gefasst, \textcolor{pink}{Wien}{}\ledrightnote{\textcolor{pink}{Wien}} – \textcolor{pink}{Pörtschach}{}\ledrightnote{\textcolor{pink}{Pörtschach}} – \textcolor{pink}{Aussee}{}\ledrightnote{\textcolor{pink}{Bad Aussee}} – alles noch ungewiss.\pend
           \pstart
           {\pb}Was haben Sie beschlossen?\pend
           \pstart
           Grüßen Sie mir herzlich \textcolor{blue}{Salten}{}\ledrightnote{\textcolor{blue}{Felix Salten}}.\pend
           \pstart
           Ihr{\\[\baselineskip]}treuer{\\[\baselineskip]}\spacefill\mbox{Richard}\pend
           \leftskip=0em{}\pstart
           21. Aug. 91.\pend
           \endnumbering\briefempfaengerindex{Schnitzler, Arthur@\textsc{Schnitzler, Arthur}!zzzBeer-Hofmann, Richard@\emph{von Richard Beer-Hofmann}!1891-08-211@{21. 8. 1891}|)be}\mylabel{h}  \normalsize

\doendnotes{C}
\bigskip
\vfill

\clearpage

\footnotesize

\lohead{\textsc{register}}

% Definiere theindex-Environment komplett neu ohne reledmac
\makeatletter
\renewenvironment{theindex}{%
  \section*{\indexname}%
  \setlength{\parindent}{0pt}%
  \setlength{\parskip}{0pt plus 0.3pt}%
  \let\item\@idxitem
}{%
  \clearpage
}
\makeatother

\IfFileExists{\jobname-pw.ind}{\input{\jobname-pw.ind}}{}

\end{document}

      