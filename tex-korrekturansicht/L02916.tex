%% latex-korrekturansicht-vorspann.tex
%% Vorspann für die Korrekturansicht.
%% Lädt die gemeinsame Datei latex-vorspann.tex mit gesetztem Schalter.

\newif\ifkorrekturansicht
\korrekturansichttrue

\input{../tex-inputs/latex-vorspann}


         
         \renewcommand{\erwaehntePersonen}{Personen:  ?? [in Berlin lebender Wiener Journalist], Hermann Bahr, Richard Beer-Hofmann, Ludwig Fulda, Julius von Gans-Ludassy, Carl Karlweis, Helene Keßler, Isidor Landau, Rudolf Lothar, Fritz Mauthner, Erich von Monbart, Felix Salten, Ottilie Salten}
         \renewcommand{\erwaehnteInstitutionen}{Institutionen: Berliner Börsen-Courier, Deutsches Theater Berlin, Freie Bühne, Neue Freie Presse, Volkstheater}
         \renewcommand{\erwaehnteOrte}{Orte: Berlin, Dessauer Straße, Lessing-Theater, Puchberg am Schneeberg, Volkstheater, Wien}
         \renewcommand{\erwaehnteWerke}{Werke: Berliner Tageblatt, Der Star. Ein Wiener Stück in vier Akten, Der letzte Knopf, Die Familie von Barchwitz, Kleine Chronik. [Theater.] [Onkel Toni], König Harlekin. Maskenspiel in vier Aufzügen, Neue Freie Presse, Neues Wiener Tagblatt, Onkel Toni. Eine Komödie aus der Gesellschaft in vier Aufzügen, Theater, Kunst und Literatur [König Harlekin], Theater- und Kunstnachrichten [König Harlekin], Wiener Deutsches Volkstheater. (Gastspiel im Deutschen Theater.) »König Harlekin«, ein Maskenspiel in vier Aufzügen von Rudolf Lothar}
               \section[ Paul Goldmann an Arthur Schnitzler, 27. 5. {[}1900{]}]{Paul Goldmann an Arthur Schnitzler, 27. 5. {[}1900{]}}\nopagebreak\mylabel{v}\rehead{ }\normalsize\beginnumbering\briefempfaengerindex{Schnitzler, Arthur@\textsc{Schnitzler, Arthur}!zzzGoldmann, Paul@\emph{von Paul Goldmann}!1900-05-271@{27. 5. {[}1900{]}}|(be} \toendnotes[C]{\smallbreak\pagebreak[2]} \Standort{DLA, A:Schnitzler, HS.NZ85.1.3170.}
\physDesc{Brief, 1 Blatt, 4 Seiten
\newline{}Handschrift: blaue Tinte, deutsche Kurrent
\newline{}Schnitzler: 1) mit Bleistift das Jahr »{[}1{]}900« vermerkt  2) mit rotem Buntstift sechs Unterstreichungen}\toendnotes[C]{\smallbreak}\pstart
           \noindent{}{\pb}\textcolor{pink}{\textcolor{gray}{\textbf{DESSAUERSTRASSE 19}}}{}\ledrightnote{\textcolor{pink}{Dessauer Straße}}\hfill \textcolor{pink}{Berlin}{}\ledrightnote{\textcolor{pink}{Berlin}}, 27. Mai.\pend
           \pstart
           \centering{}Mein lieber Freund,\pend
           \pstart
           \noindent{}Du biſt wieder einmal ganz verſtummt. Von Woche zu Woche warte ich auf eine
               Nachricht, aber vergebens.\pend
           \pstart
           Wann alſo wirſt Du anfangen zu \label{K_L02916-1v}\edtext{reiſen}{\lemma{\textnormal{\emph{reiſen}}}\Cendnote{\textnormal{\textcolor{blue}{Schnitzler} war bereits seit 24. 5. 1900 in \textcolor{pink}{Puchberg am Schneeberg}, wo er bis 27. 5. 1900 blieb und
                  Zeit mit \textcolor{blue}{Felix Salten} und \textcolor{blue}{Ottilie Metzeles} (später \textcolor{blue}{Salten}) verbrachte.}}}\label{K_L02916-1h}? Und wohin?
               Intereſſant wäre es auch, die Frage zu ſtellen: mit wem? Aber ich ſtelle ſie lieber
               nicht.\pend
           \pstart
           \textsc{\textcolor{blue}{Rudolf Lothar}{}\ledrightnote{\textcolor{blue}{Rudolf Lothar}}} hat ſich hier hübſch benommen. Er hat ſich einen in \textcolor{pink}{Berlin}{}\ledrightnote{\textcolor{pink}{Berlin}} lebenden \label{K_L02916-2v}\edtext{\textcolor{pink}{Wien}{}\ledrightnote{\textcolor{pink}{Wien}}er \textcolor{blue}{Journaliſten}{}\ledrightnote{{$\rightarrow$}\textcolor{blue}{?? [in Berlin lebender Wiener Journalist]}} engagirt}{\lemma{\textnormal{\emph{Wiener … engagirt}}}\Cendnote{\textnormal{Nicht identifiziert. Siehe auch Paul Goldmann an Arthur Schnitzler, 10. 5. [1900].}}}\label{K_L02916-2h}, der \strikeout{b} von \textcolor{pink}{Berlin}{}\ledrightnote{\textcolor{pink}{Berlin}}er Redaktionen wegen »Inkorrektheiten«
               entlaſſen worden iſt, und hat von dieſem am Abend
               ſeiner 
                  \textsc{\textcolor{green}{Première}{}\ledrightnote{{$\rightarrow$}\textcolor{green}{König Harlekin. Maskenspiel in vier Aufzügen}}}{ }ein gefälſchtes \label{K_L02916-3v}\edtext{\textcolor{green}{Telegramm}{}\ledrightnote{{$\rightarrow$}\textcolor{green}{Theater, Kunst und Literatur [König Harlekin]}}}{\lemma{\textnormal{\emph{Telegramm}}}\Cendnote{\textnormal{Abgedruckt zum Beispiel im \emph{\textcolor{green}{Neuen Wiener Tagblatt}}: \textcolor{blue}{o. V.}: \emph{\textcolor{green}{Theater, Kunst und Literatur}}. In: \emph{\textcolor{green}{Neues Wiener Tagblatt}}, Jg. 34, Nr. 137,
                        20. 5. 1900, Tages-Ausgabe,
                  S. 8.}}}\label{K_L02916-3h} an alle \textcolor{pink}{Wien}{}\ledrightnote{\textcolor{pink}{Wien}}er Blätter
               ſenden laſſen. Für die \textcolor{brown}{N. Fr. Pr.}{}\ledrightnote{\textcolor{brown}{Neue Freie Presse}} hat \textsc{\textcolor{blue}{Landau}{}\ledrightnote{\textcolor{blue}{Isidor Landau}}} vom \textcolor{brown}{Börſencourier}{}\ledrightnote{\textcolor{brown}{Berliner Börsen-Courier}}{\pb}{ }\label{K_L02916-14v}\edtext{\textcolor{green}{telegraphirt}{}\ledrightnote{{$\rightarrow$}\textcolor{green}{Theater- und Kunstnachrichten [König Harlekin]}}}{\lemma{\textnormal{\emph{telegraphirt}}}\Cendnote{\textnormal{o. V. [=\textcolor{blue}{Isidor Landau}]: \emph{\textcolor{green}{Theater- und Kunstnachrichten}}. In: \emph{\textcolor{green}{Neue Freie Presse}}, Nr. 12837, 20. 5. 1900, Morgenblatt, S. 9.}}}\label{K_L02916-14h}, der
               bekanntlich die Spezialität hat, Alles zu loben. Aber ſelbſt deſſen \textcolor{green}{Telegramm}{}\ledrightnote{{$\rightarrow$}\textcolor{green}{Theater- und Kunstnachrichten [König Harlekin]}} genügte noch nicht, und ſo hat
               man in der \textcolor{brown}{Redaktion}{}\ledrightnote{{$\rightarrow$}\textcolor{brown}{Neue Freie Presse}} dieſe
               Fälſchung durch Einfügung einiger lobender Sätze noch \strikeout{\textcolor{gray}{f}} weiter gefälſcht. Dem \textsc{\textcolor{blue}{Fritz Mauthner}{}\ledrightnote{\textcolor{blue}{Fritz Mauthner}}} hat ſich \textsc{\textcolor{blue}{Lothar}{}\ledrightnote{\textcolor{blue}{Rudolf Lothar}}} ſeit dem Tage ſeiner Ankunft an die Rockſchöße gehangen, er hat ihn umwedelt
               und umſchmeichelt. Die Folge davon war, daß \textsc{\textcolor{blue}{Mauthner}{}\ledrightnote{\textcolor{blue}{Fritz Mauthner}}} in ſeinem \label{K_L02916-6v}\edtext{\textcolor{green}{Feuilleton}{}\ledrightnote{{$\rightarrow$}\textcolor{green}{Wiener Deutsches Volkstheater. (Gastspiel im Deutschen Theater.) »König Harlekin«, ein Maskenspiel in vier Aufzügen von Rudolf Lothar}}}{\lemma{\textnormal{\emph{Feuilleton}}}\Cendnote{\textnormal{\textcolor{blue}{F. M.} [=\textcolor{blue}{Fritz Mauthner}]: \emph{\textcolor{green}{Wiener Deutsches Volkstheater. (Gastspiel im Deutschen
                        Theater.) »König Harlekin«, ein Maskenspiel in vier Aufzügen von Rudolf
                        Lothar}}. In: \emph{\textcolor{green}{Berliner Tageblatt}},
                     Jg. 29, Nr. 254, 20. 5. 1900,
                  S. [3].}}}\label{K_L02916-6h} vom »\textcolor{green}{Dichter{ }\textsc{\textcolor{blue}{Lothar}{}\ledrightnote{\textcolor{blue}{Rudolf Lothar}}}}{}\ledrightnote{{$\rightarrow$}\textcolor{green}{Wiener Deutsches Volkstheater. (Gastspiel im Deutschen Theater.) »König Harlekin«, ein Maskenspiel in vier Aufzügen von Rudolf Lothar}}« ſprach. Damit iſt \textsc{\textcolor{blue}{Mauthner}{}\ledrightnote{\textcolor{blue}{Fritz Mauthner}}} als Kritiker allerdings für mich gerichtet.\pend
           \pstart
           Als \textsc{\textcolor{blue}{Karlweiss}{}\ledrightnote{\textcolor{blue}{Carl Karlweis}}}’ »\textcolor{green}{Onkel Toni}{}\ledrightnote{\textcolor{green}{Onkel Toni. Eine Komödie aus der Gesellschaft in vier Aufzügen}}« \textcolor{pink}{hier}{}\ledrightnote{{$\rightarrow$}\textcolor{pink}{Berlin}}{ }\label{K_L02916-11v}\edtext{aufgeführt}{\lemma{\textnormal{\emph{aufgeführt}}}\Cendnote{\textnormal{\textcolor{blue}{Goldmann} bezog sich auf das Gastspiel des
                     \emph{\textcolor{brown}{Volkstheater}}s von \emph{\textcolor{green}{Onkel Toni}} am 11. 5. 1900.}}}\label{K_L02916-11h} wurde, \textcolor{green}{telegraphirte}{}\ledrightnote{{$\rightarrow$}\textcolor{green}{Kleine Chronik. [Theater.] [Onkel Toni]}} ich {\pb}ganz
               ſanft: Die vortreffliche Aufführung habe über die ſchwachen Stellen des \textcolor{green}{Stück}{}\ledrightnote{{$\rightarrow$}\textcolor{green}{Onkel Toni. Eine Komödie aus der Gesellschaft in vier Aufzügen}}es hinweggeholfen. Der
               Satz \strikeout{wurde} wurde \label{K_L02916-7v}\edtext{geſtrichen}{\lemma{\textnormal{\emph{geſtrichen}}}\Cendnote{\textnormal{\textcolor{blue}{o. V.} [=\textcolor{blue}{Paul Goldmann}]: \emph{\textcolor{green}{Kleine Chronik. [Theater.]}}. In: \emph{\textcolor{green}{Neue Freie Presse}}, Nr. 12829, 12. 5. 1900, Abendblatt, S. 1.}}}\label{K_L02916-7h}. Ein \textcolor{green}{Stück}{}\ledrightnote{\textcolor{green}{Onkel Toni. Eine Komödie aus der Gesellschaft in vier Aufzügen}} von \textsc{\textcolor{blue}{Karlweiss}{}\ledrightnote{\textcolor{blue}{Carl Karlweis}}} darf nicht einmal ſchwache Stellen haben!\pend
           \pstart
           Der »\textsc{\textcolor{green}{Star}{}\ledrightnote{\textcolor{green}{Der Star. Ein Wiener Stück in vier Akten}}}« von \textsc{\textcolor{blue}{Bahr}{}\ledrightnote{\textcolor{blue}{Hermann Bahr}}} hat mir hingegen \label{K_L02916-12v}\edtext{gefallen}{\lemma{\textnormal{\emph{gefallen}}}\Cendnote{\textnormal{Das \textcolor{green}{Stück} wurde am \textcolor{pink}{Berlin}er \textcolor{pink}{Lessing-Theater}
               gespielt.}}}\label{K_L02916-12h}. Dieſer widerliche \textcolor{blue}{Burſch}{}\ledrightnote{{$\rightarrow$}\textcolor{blue}{Hermann Bahr}} hat doch – leider! – Humor und Talent.\pend
           \pstart
           Bitte, \label{K_L02916-19v}\edtext{lies’}{\lemma{\textnormal{\emph{lies’}}}\Cendnote{\textnormal{\textcolor{blue}{Schnitzler} las den \textcolor{green}{Roman} (vgl. A. S.: \emph{Lektüren}, Deutschsprachige-Literatur).}}}\label{K_L02916-19h}, wenn Du es noch
               nicht kennſt, »\textcolor{green}{Die Familie von \textsc{Barchwitz}}{}\ledrightnote{\textcolor{green}{Die Familie von Barchwitz}}« von \textsc{\textcolor{blue}{Hans von Kahlenberg}{}\ledrightnote{{$\rightarrow$}\textcolor{blue}{Helene Keßler}}}. Seit Langem hat mich kein Roman ſo intereſſirt. \strikeout{\textcolor{gray}{Verg}}{ }\textcolor{blue}{Verfaſſerin}{}\ledrightnote{{$\rightarrow$}\textcolor{blue}{Helene Keßler}} iſt ein nicht
               mehr {\pb}ganz \strikeout{\textcolor{gray}{hu}} junges, aber \strikeout{\textcolor{gray}{r}} noch \strikeout{recht} recht hübſches \textcolor{blue}{Mädchen}{}\ledrightnote{{$\rightarrow$}\textcolor{blue}{Helene Keßler}}, ein Fräulein von \textsc{\textcolor{blue}{Montbart}{}\ledrightnote{{$\rightarrow$}\textcolor{blue}{Helene Keßler}}}, \textcolor{blue}{Offizier}{}\ledrightnote{{$\rightarrow$}\textcolor{blue}{Erich von Monbart}}s-Tochter.\pend
           \pstart
           Was macht \textsc{\textcolor{blue}{Richard}{}\ledrightnote{\textcolor{blue}{Richard Beer-Hofmann}}}?\pend
           \pstart
           Bitte, ſchreib’ mir bald!\pend
           \pstart
           Viele treue Grüße! {\\[\baselineskip]}Dein {\\[\baselineskip]}\spacefill\mbox{Paul Goldmann}\pend
           \leftskip=0em{}\pstart
           \noindent{}Auch \textsc{\textcolor{blue}{Ludassy}{}\ledrightnote{\textcolor{blue}{Julius von Gans-Ludassy}}} benimmt ſich abſcheulich hier und macht ſich aus dem \label{K_L02916-21v}\edtext{Verbot ſeines ſchlechten \textcolor{green}{Stück}{}\ledrightnote{{$\rightarrow$}\textcolor{green}{Der letzte Knopf}}es}{\lemma{\textnormal{\emph{Verbot … Stückes}}}\Cendnote{\textnormal{\textcolor{blue}{Julius von Gans-Ludassy}s \emph{\textcolor{green}{Der letzte Knopf}} war am 8. 4. 1900 am \textcolor{pink}{Volkstheater}
                     uraufgeführt worden. Das \textcolor{green}{Stück}, das für einen Skandal sorgte, sollte auch in \textcolor{pink}{Berlin} aufgeführt werden. \textcolor{blue}{Ludwig Fulda}, der als Präsident der \emph{\textcolor{brown}{Freien Bühne}} das von der Zensur verbotene \textcolor{green}{Stück} annahm, musste von
                     seiner Funktion zurücktreten. Bei einer Matinée des \emph{\textcolor{brown}{Deutschen Theater}}s am 30. 5. 1900 wurde \emph{\textcolor{green}{Der letzte
                        Knopf}} vor einem geladenen Publikum schließlich doch aufgeführt.}}}\label{K_L02916-21h}
                  eine unerträgliche Reklame.\pend
           \endnumbering\briefempfaengerindex{Schnitzler, Arthur@\textsc{Schnitzler, Arthur}!zzzGoldmann, Paul@\emph{von Paul Goldmann}!1900-05-271@{27. 5. {[}1900{]}}|)be}\mylabel{h}\begin{anhang}\end{anhang}\normalsize

\doendnotes{C}
\bigskip
\vfill

\clearpage

\footnotesize

\lohead{\textsc{register}}

% Definiere theindex-Environment komplett neu ohne reledmac
\makeatletter
\renewenvironment{theindex}{%
  \section*{\indexname}%
  \setlength{\parindent}{0pt}%
  \setlength{\parskip}{0pt plus 0.3pt}%
  \let\item\@idxitem
}{%
  \clearpage
}
\makeatother

\IfFileExists{\jobname-pw.ind}{\input{\jobname-pw.ind}}{}

\end{document}

      