%% latex-korrekturansicht-vorspann.tex
%% Vorspann für die Korrekturansicht.
%% Lädt die gemeinsame Datei latex-vorspann.tex mit gesetztem Schalter.

\newif\ifkorrekturansicht
\korrekturansichttrue

\input{../tex-inputs/latex-vorspann}


\renewcommand{\erwaehntePersonen}{Personen: Olga Schnitzler}
\renewcommand{\erwaehnteInstitutionen}{Institutionen: Lessing-Theater}
\renewcommand{\erwaehnteOrte}{Orte: Bendlerstraße, Berlin, Schöneberger Ufer, Wien}
\renewcommand{\erwaehnteWerke}{Werke: Das weite Land. Tragikomödie in fünf Akten}
\section[ Eva Marie Goldmann an Arthur Schnitzler, 1. 10. 1911]{Eva Marie Goldmann an Arthur Schnitzler, 1. 10. 1911}
\nopagebreak\mylabel{v}
\rehead{ }\normalsize\beginnumbering\briefempfaengerindex{Schnitzler, Arthur@\textsc{Schnitzler, Arthur}!zzzGoldmann, Eva Marie@\emph{von Eva Marie Goldmann}!1911-10-011@{1. 10. 1911}|(be}
\toendnotes[C]{\smallbreak\pagebreak[2]}\Standort{DLA, A:Schnitzler, HS.NZ85.1.3160.}
\physDesc{Brief, 1 Blatt, 2 Seiten, 489 Zeichen
\newline{}Handschrift: , lateinische Kurrent
\newline{}Schnitzler: mit Bleistift Unterstreichung des »G« im vorgedruckten
                                 Briefkopf }\toendnotes[C]{\smallbreak}
\pstart
           \raggedleft{}{\pb}\textcolor{pink}{Berlin}{}\ledrightnote{\textcolor{pink}{Berlin}}, d. 1. X. \uline{1911}\uline{.}\pend
           
\pstart
           \textcolor{gray}{\textbf{EG}}\hfill \textcolor{gray}{\textbf{\textcolor{pink}{W.
                              SCHÖNEBERGER-UFER 34}{}\ledrightnote{\textcolor{pink}{Schöneberger Ufer}}.}}\pend
           
\pstart{}Verehrter Herr Doctor,\pend
\pstart
           ich will Ihnen nur rasch den Empfang Ihres liebenswertigen Briefes bestätigen, u.
               Ihnen für Ihre freundlichen Zeilen herzlichst danken. Beantworten kann ich sie heute nicht – aus irdischem Jammer. {\pb}Ich stecke nämlich mitten in den \label{K_L03541-1v}\edtext{Umzugsvorbereitungen\textcolor{gray}{,}}{\lemma{\textnormal{\emph{Umzugsvorbereitungen,}}}\Cendnote{\textnormal{Sie zogen in die \textcolor{pink}{Bendlerstraße 36}.}}}\label{K_L03541-1h} und was das bedeutet, kann \uline{nur} eine Frau ermessen!\pend
           
\pstart
           Hoffentlich wird Sie in absehbarer Zeit \label{K_L03541-2v}\edtext{»\textcolor{green}{Das Weite Land}{}\ledrightnote{\textcolor{green}{Das weite Land. Tragikomödie in fünf Akten}}« persönlich nach \textcolor{pink}{Berlin}{}\ledrightnote{\textcolor{pink}{Berlin}} führen}{\lemma{\textnormal{\emph{»Das … führen}}}\Cendnote{\textnormal{Am 14. 10. 1911 fanden die Uraufführungen von \emph{\textcolor{green}{Das weite Land}} in neun Städten statt, darunter \textcolor{pink}{Berlin} mit dem \emph{\textcolor{brown}{Lessingtheater}}. \textcolor{blue}{Schnitzler} sah das
                     \textcolor{green}{Stück} dort am 2. 11. 1911.}}}\label{K_L03541-2h}.\pend
           
\pstart
           Mit den besten Grüssen für Frau \textcolor{blue}{Olga}{}\ledrightnote{\textcolor{blue}{Olga Schnitzler}} u.
               Sie {\\[\baselineskip]}Ihre ergebene {\\[\baselineskip]}\spacefill\mbox{EvaMGoldmann.}\pend
           \leftskip=0em{}\endnumbering\briefempfaengerindex{Schnitzler, Arthur@\textsc{Schnitzler, Arthur}!zzzGoldmann, Eva Marie@\emph{von Eva Marie Goldmann}!1911-10-011@{1. 10. 1911}|)be}\mylabel{h}
\begin{anhang}
\end{anhang}\normalsize

\doendnotes{C}
\bigskip
\vfill

\clearpage

\footnotesize

\lohead{\textsc{register}}

% Definiere theindex-Environment komplett neu ohne reledmac
\makeatletter
\renewenvironment{theindex}{%
  \section*{\indexname}%
  \setlength{\parindent}{0pt}%
  \setlength{\parskip}{0pt plus 0.3pt}%
  \let\item\@idxitem
}{%
  \clearpage
}
\makeatother

\IfFileExists{\jobname-pw.ind}{\input{\jobname-pw.ind}}{}

\end{document}

      