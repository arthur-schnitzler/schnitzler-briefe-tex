%% latex-korrekturansicht-vorspann.tex
%% Vorspann für die Korrekturansicht.
%% Lädt die gemeinsame Datei latex-vorspann.tex mit gesetztem Schalter.

\newif\ifkorrekturansicht
\korrekturansichttrue

\input{../tex-inputs/latex-vorspann}


\renewcommand{\erwaehntePersonen}{Personen: Olga Schnitzler, Elisabeth Steinrück}
\renewcommand{\erwaehnteOrte}{Orte: Berlin, Brühl, Dessauer Straße, Edmund-Weiß-Gasse, Wien}
\renewcommand{\erwaehnteWerke}{}
\section[ Paul Goldmann an Arthur Schnitzler, 16. 5. {[}1902{]}]{Paul Goldmann an Arthur Schnitzler, 16. 5. {[}1902{]}}
\nopagebreak\mylabel{v}
\rehead{ }\normalsize\beginnumbering\briefempfaengerindex{Schnitzler, Arthur@\textsc{Schnitzler, Arthur}!zzzGoldmann, Paul@\emph{von Paul Goldmann}!1902-05-162@{16. 5. {[}1902{]}}|(be}
\toendnotes[C]{\smallbreak\pagebreak[2]}\Standort{DLA, A:Schnitzler, HS.NZ85.1.3172.}
\physDesc{Brief, 1 Blatt, 2 Seiten
\newline{}Handschrift: schwarze Tinte, deutsche Kurrent
\newline{}Schnitzler: mit Bleistift das Jahr »1902« vermerkt }\toendnotes[C]{\smallbreak}
\pstart
           \noindent{}\raggedleft{}{\pb}\textcolor{pink}{\textcolor{gray}{\textbf{DESSAUERSTRASSE 19}}}{}\ledrightnote{\textcolor{pink}{Dessauer Straße}}\pend
           
\pstart
           \textcolor{pink}{Berlin}{}\ledrightnote{\textcolor{pink}{Berlin}}, 16. Mai.\pend
           
\pstart\center{}Mein lieber Freund,\pend
\pstart
           Ich habe mich alſo entſchloſſen, zu fahren, nur weiß ich noch nicht, ob Samſtag{ }früh oder Samſtag{ }Abend fahre. Da ich mir auch denke, daß Du jedenfalls ſchon \label{K_L03209-1v}\edtext{Samſtag{ }Abend}{\lemma{\textnormal{\emph{Samſtag Abend}}}\Cendnote{\textnormal{\textcolor{blue}{Schnitzler} fuhr bereits Freitagabend, also
                  am 16. 5. 1902, in
                  die \textcolor{pink}{Brühl}.}}}\label{K_L03209-1h} nach der \textcolor{pink}{Brühl}{}\ledrightnote{\textcolor{pink}{Brühl}} fahren möchteſt, ſo will ich Dich in Deinen
               Diſpoſitionen auf keinen Fall ſtören und werde Dir über {\pb}meine Ankunft nichts Näheres mittheilen. Sonntag{ }früh komme ich in Deine \textcolor{pink}{Wohnung}{}\ledrightnote{{$\rightarrow$}\textcolor{pink}{Edmund-Weiß-Gasse}}. Wenn Du in der \label{K_L03209-4v}\edtext{\textcolor{pink}{Brühl}{}\ledrightnote{\textcolor{pink}{Brühl}}}{\lemma{\textnormal{\emph{Brühl}}}\Cendnote{\textnormal{Es ist nicht gänzlich zu klären, ob \textcolor{blue}{Schnitzler} am 18. 5. 1902 in \textcolor{pink}{Wien} oder in der \textcolor{pink}{Brühl}
                  übernachtete, wo er jedenfalls am 19. 5. 1902 gemeinsam mit \textcolor{blue}{Goldmann} war.}}}\label{K_L03209-4h} biſt, ſo hinterlaſſe mir einen Brief
               mit Angabe der Adreſſe. Grüße die \textcolor{blue}{Mädele}{}\ledrightnote{{$\rightarrow$}\textcolor{blue}{Olga Schnitzler}{\newline}{$\rightarrow$}\textcolor{blue}{Elisabeth Steinrück}} nicht wieder auf das Herzlichſte. Ich komme
               beim beſten Willen nicht mehr dazu, auf ihre lieben Briefe zu antworten.\pend
           
\pstart
           Von Herzen {\\[\baselineskip]}Dein {\\[\baselineskip]}\spacefill\mbox{Paul Goldmann}\pend
           \leftskip=0em{}\endnumbering\briefempfaengerindex{Schnitzler, Arthur@\textsc{Schnitzler, Arthur}!zzzGoldmann, Paul@\emph{von Paul Goldmann}!1902-05-162@{16. 5. {[}1902{]}}|)be}\mylabel{h}
\begin{anhang}
\end{anhang}\normalsize

\doendnotes{C}
\bigskip
\vfill

\clearpage

\footnotesize

\lohead{\textsc{register}}

% Definiere theindex-Environment komplett neu ohne reledmac
\makeatletter
\renewenvironment{theindex}{%
  \section*{\indexname}%
  \setlength{\parindent}{0pt}%
  \setlength{\parskip}{0pt plus 0.3pt}%
  \let\item\@idxitem
}{%
  \clearpage
}
\makeatother

\IfFileExists{\jobname-pw.ind}{\input{\jobname-pw.ind}}{}

\end{document}

      