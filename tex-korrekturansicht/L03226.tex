%% latex-korrekturansicht-vorspann.tex
%% Vorspann für die Korrekturansicht.
%% Lädt die gemeinsame Datei latex-vorspann.tex mit gesetztem Schalter.

\newif\ifkorrekturansicht
\korrekturansichttrue

\input{../tex-inputs/latex-vorspann}


\renewcommand{\erwaehntePersonen}{Personen: August Strindberg}
\renewcommand{\erwaehnteInstitutionen}{Institutionen: Schall und Rauch}
\renewcommand{\erwaehnteOrte}{Orte: Berlin, Breslau, Dessauer Straße, Schall und Rauch, Wien}
\renewcommand{\erwaehnteWerke}{Werke: Rausch}
\section[ Paul Goldmann an Arthur Schnitzler, 12. 10. {[}1902{]}]{Paul Goldmann an Arthur Schnitzler, 12. 10. {[}1902{]}}
\nopagebreak\mylabel{v}
\rehead{ }\normalsize\beginnumbering\briefempfaengerindex{Schnitzler, Arthur@\textsc{Schnitzler, Arthur}!zzzGoldmann, Paul@\emph{von Paul Goldmann}!1902-10-121@{12. 10. {[}1902{]}}|(be}
\toendnotes[C]{\smallbreak\pagebreak[2]}\Standort{DLA, A:Schnitzler, HS.NZ85.1.3172.}
\physDesc{Brief, 1 Blatt, 2 Seiten
\newline{}Handschrift: blaue Tinte, deutsche Kurrent
\newline{}Schnitzler: mit Bleistift das Jahr »{[}1{]}902« vermerkt }\toendnotes[C]{\smallbreak}
\pstart
           \noindent{}\raggedleft{}{\pb}\textcolor{pink}{\textcolor{gray}{\textbf{DESSAUERSTRASSE 19}}}{}\ledrightnote{\textcolor{pink}{Dessauer Straße}}\pend
           
\pstart
           \textcolor{pink}{Berlin}{}\ledrightnote{\textcolor{pink}{Berlin}}, 12.
                  Okt.\pend
           
\pstart\center{}Mein lieber Freund,\pend
\pstart
           Sei herzlichſt willkommen! Ich freue mich unendlich, daß Du \label{K_L03226-1v}\edtext{\textcolor{pink}{da}{}\ledrightnote{{$\rightarrow$}\textcolor{pink}{Berlin}}}{\lemma{\textnormal{\emph{da}}}\Cendnote{\textnormal{\textcolor{blue}{Schnitzler} reiste am 12. 10. 1902 in \textcolor{pink}{Wien} ab und kam am nächsten Tag in \textcolor{pink}{Berlin} an, wo er bis 18. 10. 1902 blieb.
                  Danach reiste er weiter nach \textcolor{pink}{Breslau}.}}}\label{K_L03226-1h}
               biſt!\pend
           
\pstart
           Ich habe wahnſinnig zu thun, daß es mir unmöglich iſt, während des Tages zu Dir zu
               kommen. Komm’ auch nicht zu mir, denn ich habe keine freie Viertelſtunde. Am Beſten
               iſt es wohl, wir \label{K_L03226-2v}\edtext{treffen uns
                  Abends}{\lemma{\textnormal{\emph{treffen uns
                  Abends}}}\Cendnote{\textnormal{siehe A. S.: \emph{Tagebuch}, 13. 10. 1902}}}\label{K_L03226-2h} in der \textsc{\begin{otherlanguage}{french}Première\end{otherlanguage}} von »\textcolor{brown}{Schall und Rauch}{}\ledrightnote{\textcolor{brown}{Schall und Rauch}}«. Ein \strikeout{\textcolor{gray}{Stü}} Drama »\textcolor{green}{Rauſch}{}\ledrightnote{\textcolor{green}{Rausch}}« von \textsc{\textcolor{blue}{Strindberg}{}\ledrightnote{\textcolor{blue}{August Strindberg}}}{ }{\pb}wird geſpielt. Es ſoll ein intereſſanter Abend
               werden. Ich lege ein Billet bei; und wenn Du ganz lieb ſein willſt, ſo kommſt Du
               gegen 7 Uhr zu mir, mich ins \textcolor{pink}{Theater}{}\ledrightnote{{$\rightarrow$}\textcolor{pink}{Schall und Rauch}} abholen.\pend
           
\pstart
           Von Herzen {\\[\baselineskip]}Dein {\\[\baselineskip]}\spacefill\mbox{Paul Goldmn}\pend
           \leftskip=0em{}\endnumbering\briefempfaengerindex{Schnitzler, Arthur@\textsc{Schnitzler, Arthur}!zzzGoldmann, Paul@\emph{von Paul Goldmann}!1902-10-121@{12. 10. {[}1902{]}}|)be}\mylabel{h}
\begin{anhang}
\end{anhang}\normalsize

\doendnotes{C}
\bigskip
\vfill

\clearpage

\footnotesize

\lohead{\textsc{register}}

% Definiere theindex-Environment komplett neu ohne reledmac
\makeatletter
\renewenvironment{theindex}{%
  \section*{\indexname}%
  \setlength{\parindent}{0pt}%
  \setlength{\parskip}{0pt plus 0.3pt}%
  \let\item\@idxitem
}{%
  \clearpage
}
\makeatother

\IfFileExists{\jobname-pw.ind}{\input{\jobname-pw.ind}}{}

\end{document}

      