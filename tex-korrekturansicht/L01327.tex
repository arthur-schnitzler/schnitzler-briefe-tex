%% latex-korrekturansicht-vorspann.tex
%% Vorspann für die Korrekturansicht.
%% Lädt die gemeinsame Datei latex-vorspann.tex mit gesetztem Schalter.

\newif\ifkorrekturansicht
\korrekturansichttrue

\input{../tex-inputs/latex-vorspann}


               \section[Franz Blei an Arthur Schnitzler, 12. 10. 1903]{ Franz Blei an Arthur Schnitzler, 12. 10. 1903}\nopagebreak\mylabel{v}\rehead{ }\normalsize\beginnumbering\briefempfaengerindex{Schnitzler, Arthur@\textsc{Schnitzler, Arthur}!zzzBlei, Franz@\emph{von Franz Blei}!1903-10-121@{12. 10. 1903}|(be} \toendnotes[C]{\smallbreak\pagebreak[2]} \Standort{CUL, Schnitzler, B 14.}
\physDesc{Brief, 1 Blatt, 2 Seiten
\newline{}Handschrift: schwarze Tinte, lateinische Kurrent
\newline{}Schnitzler: 1) mit Bleistift beschriftet: »\textsc{Blei}« und datiert »12/10 903« 2) mit rotem Buntstift zwei Unterstreichungen\newline{}Ordnung: 1) mit Bleistift von unbekannter Hand nummeriert: »\strikeout{7}« 2) mit Bleistift von unbekannter Hand nummeriert: »2«}\toendnotes[C]{\smallbreak}\pstart
           \centering{}{\pb}\textcolor{pink}{München, Arcisstrasse 19}{}\ledrightnote{\textcolor{pink}{Arcisstraße}}\pend
           \pstart{}Sehr geehrter Herr Arthur Schnitzler,\pend\pstart
           Miss \textcolor{blue}{Johnson}{}\ledrightnote{\textcolor{blue}{Fanny Johnson}} kam mit Empfehlungen von sehr
                    guten \textcolor{pink}{Engländern}{}\ledrightnote{\textcolor{pink}{England}}, wie \textcolor{blue}{Yeats}{}\ledrightnote{\textcolor{blue}{William Butler Yeats}} und \textcolor{blue}{A. Symons}{}\ledrightnote{\textcolor{blue}{Arthur Symons}} zu
                    mir und auf die Frage, was sie übersetzen solle, rieth ich ihr zu dem \textcolor{green}{Grünen Kakadu}{}\ledrightnote{\textcolor{green}{Der grüne Kakadu. Groteske in einem Akt}}. Die \textcolor{blue}{Dame}{}\ledrightnote{→\textcolor{blue}{Fanny Johnson}} wird sicher eine sehr gute
                    Übertragung zu stand bringen und dass sie damit bei den \textcolor{pink}{englischen}{}\ledrightnote{\textcolor{pink}{England}} Bühnen mehr Glück haben wird wie mit ihren
                    eigenhändigen Stücken ist keine Frage. Wenn Sie {\pb}daher nicht andere entscheidende
                    Gründe dagegen haben, möchte ich mir erlauben, Ihnen Miss \textcolor{blue}{Johnson}{}\ledrightnote{\textcolor{blue}{Fanny Johnson}} für die \textcolor{green}{Übertragung}{}\ledrightnote{→\textcolor{green}{Der grüne Kakadu. Groteske in einem Akt}} zu empfehlen.\pend
           \pstart
           Ich bin Ihr ganz ergebener{\\[\baselineskip]}\spacefill\mbox{Franz Blei.}\pend
           \leftskip=0em{}\pstart
           12. 10. 1903.\pend
           \endnumbering\briefempfaengerindex{Schnitzler, Arthur@\textsc{Schnitzler, Arthur}!zzzBlei, Franz@\emph{von Franz Blei}!1903-10-121@{12. 10. 1903}|)be}\mylabel{h}  \normalsize

\doendnotes{C}
\bigskip
\vfill

\clearpage

\footnotesize

\lohead{\textsc{register}}

% Definiere theindex-Environment komplett neu ohne reledmac
\makeatletter
\renewenvironment{theindex}{%
  \section*{\indexname}%
  \setlength{\parindent}{0pt}%
  \setlength{\parskip}{0pt plus 0.3pt}%
  \let\item\@idxitem
}{%
  \clearpage
}
\makeatother

\IfFileExists{\jobname-pw.ind}{\input{\jobname-pw.ind}}{}

\end{document}

      