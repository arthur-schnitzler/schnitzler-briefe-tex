%% latex-korrekturansicht-vorspann.tex
%% Vorspann für die Korrekturansicht.
%% Lädt die gemeinsame Datei latex-vorspann.tex mit gesetztem Schalter.

\newif\ifkorrekturansicht
\korrekturansichttrue

\input{../tex-inputs/latex-vorspann}


               \section[Richard Beer-Hofmann an Arthur Schnitzler, 19. 10. 1907]{ Richard Beer-Hofmann an Arthur Schnitzler, 19. 10. 1907}\nopagebreak\mylabel{v}\rehead{ }\normalsize\beginnumbering\briefempfaengerindex{Schnitzler, Arthur@\textsc{Schnitzler, Arthur}!zzzBeer-Hofmann, Richard@\emph{von Richard Beer-Hofmann}!1907-10-191@{19. 10. 1907}|(be} \toendnotes[C]{\smallbreak\pagebreak[2]} \Standort{CUL, Schnitzler, B 8.}
\physDesc{Brief, 1 Blatt (Briefpapier mit Trauerrand), 1 Seite
\newline{}Handschrift: Bleistift, lateinische Kurrent\newline{}Ordnung: mit Bleistift von unbekannter Hand nummeriert:
                                    »214« }\buchAbdrucke{\weitereDrucke{Arthur Schnitzler, Richard Beer-Hofmann: \emph{Briefwechsel 1891–1931}. Hg. Konstanze Fliedl. Wien, Zürich: \emph{Europaverlag} 1992, S. 186.} }\toendnotes[C]{\smallbreak}\pstart
           \raggedleft{}{\pb}19/X 07\pend
           \pstart
           Lieber Arthur! Wollen Sie heute Abends – anstatt \label{KLL01724_AS-1v}\edtext{{[}Zeichnung einer schwarz-weiß-gekachelten \textcolor{brown}{Fledermaus}{}\ledrightnote{\textcolor{brown}{Cabaret Fledermaus}}{]}}{\lemma{\textnormal{\emph{Zeichnung … Fledermaus}}}\Cendnote{\textnormal{Am 19. 10. 1907
                  eröffnete das \emph{\textcolor{brown}{Cabaret Fledermaus}} mit einer
                  Innenausstattung, die mit ihren schwarz-weiß gekachelten Fliesen
                  Aufsehen erregte. \textcolor{blue}{Altenberg} war ein
                  Unterstützer des neuen Etablissements und trug selbst gerne karierte Kleidung.}}}\label{KLL01724_AS-1h} und {[}Zeichung von \textcolor{blue}{Altenberg}{}\ledrightnote{\textcolor{blue}{Peter Altenberg}}
                  mit Sprechblase: {]} das höchste!! {[}–{]} bei uns
               essen?\pend
           \pstart
           (\textcolor{blue}{Leo}{}\ledrightnote{\textcolor{blue}{Leo Van-Jung}} ko{\geminationm}t auch) um
               halbacht?\pend
           \pstart
           Es wäre sehr schön.\pend
           \pstart
           Herzlich{\\[\baselineskip]}\spacefill\mbox{Richard}\pend
           \leftskip=0em{}\endnumbering\briefempfaengerindex{Schnitzler, Arthur@\textsc{Schnitzler, Arthur}!zzzBeer-Hofmann, Richard@\emph{von Richard Beer-Hofmann}!1907-10-191@{19. 10. 1907}|)be}\mylabel{h}  \normalsize

\doendnotes{C}
\bigskip
\vfill

\clearpage

\footnotesize

\lohead{\textsc{register}}

% Definiere theindex-Environment komplett neu ohne reledmac
\makeatletter
\renewenvironment{theindex}{%
  \section*{\indexname}%
  \setlength{\parindent}{0pt}%
  \setlength{\parskip}{0pt plus 0.3pt}%
  \let\item\@idxitem
}{%
  \clearpage
}
\makeatother

\IfFileExists{\jobname-pw.ind}{\input{\jobname-pw.ind}}{}

\end{document}

      