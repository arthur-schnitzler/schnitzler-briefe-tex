%% latex-korrekturansicht-vorspann.tex
%% Vorspann für die Korrekturansicht.
%% Lädt die gemeinsame Datei latex-vorspann.tex mit gesetztem Schalter.

\newif\ifkorrekturansicht
\korrekturansichttrue

\input{../tex-inputs/latex-vorspann}


               \section[Robert Adam an Arthur Schnitzler, 1. 10. 1918]{ Robert Adam an Arthur Schnitzler, 1. 10. 1918}\nopagebreak\mylabel{v}\rehead{ }\normalsize\beginnumbering\briefempfaengerindex{Schnitzler, Arthur@\textsc{Schnitzler, Arthur}!zzzAdam, Robert@\emph{von Robert Adam}!1918-10-011@{1. 10. 1918}|(be} \toendnotes[C]{\smallbreak\pagebreak[2]} \Standort{CUL, Schnitzler, B 1.}
\physDesc{Brief, 1 Blatt, 3 Seiten
\newline{}Handschrift: schwarze Tinte, deutsche Kurrent
\newline{}Schnitzler: 1) mit Bleistift beschriftet: »\textsc{Adam}« 2) mit rotem Buntstift zwei Unterstreichungen\newline{}Ordnung: von unbekannter Hand nummeriert:
                                            »7« }\Standort{Wien, Österreichische Nationalbibliothek, Cod.ser. 52.269, 223 recto.}
\physDesc{Brief, maschinelle Abschrift
\newline{}Schreibmaschine}\toendnotes[C]{\smallbreak}\pstart
           \raggedleft{}{\pb}\textcolor{pink}{Wien}{}\ledrightnote{\textcolor{pink}{Wien}}, am 1. Oktober 1918\pend
           \pstart\center{}Hochverehrter Doktor!\pend\pstart
           Ich vermute Sie von Ihrer Reiſe, die Ihnen hoffentlich Erholung gebracht hat,
                    bereits nach \textcolor{pink}{Wien}{}\ledrightnote{\textcolor{pink}{Wien}} zurückgekehrt und frage mich
                    an, ob und wann Sie ein Beſuch nicht ſtören würde. Es wäre mir ſehr lieb, wenn
                    ich über das Stück »\textcolor{green}{Yppl}{}\ledrightnote{\textcolor{green}{Yppl. Idylle in fünf Akten}}« und über die Frage,
                    ob nicht jetzt Schritte möglich wären, den »\textcolor{green}{Neidhard}{}\ledrightnote{\textcolor{green}{Neidhard}}« dem \textcolor{pink}{Burgtheater}{}\ledrightnote{\textcolor{pink}{Burgtheater}}
                    näherzubringen, mit Ihnen ſprechen könnte. Darf ich Ihnen hiebei eines der \textcolor{green}{Bücher}{}\ledrightnote{→\textcolor{green}{Geistesstörung und Verbrechen im Kindesalter}{\newline}→\textcolor{green}{Minderjährige Verbrecher. (Versuch einer strafgerichtlichen Psychologie) mit Original-Gutachten von Berenini – Brusa – Colajanni – Negri – Nordau – Pierantoni}} über jugend{\pb}liche Verbrecher (und welches?)
                    mitbringen?\pend
           \pstart
           Meine Urlaubswoche verlebte ich, vom Wetter nicht ſehr begünſtigt, in der \textcolor{pink}{Welſ}{}\ledrightnote{\textcolor{pink}{Wels}}er und \textcolor{pink}{Linz}{}\ledrightnote{\textcolor{pink}{Linz}}er Gegend; die Wanderungen waren, da ich zwei Laib Brot im
                    Ruckſack mitſchleppen mußte, einigermaßen beſchwerlich, die Ernährungs- und
                    Unterkunftsfragen nicht immer leicht zu löſen. Immerhin gab es ſchöne Stunden in
                        \textcolor{pink}{Wilhering}{}\ledrightnote{\textcolor{pink}{Wilhering}}, \textcolor{pink}{Ottensheim}{}\ledrightnote{\textcolor{pink}{Ottensheim}}, \textcolor{pink}{Eberſtall-Zell}{}\ledrightnote{\textcolor{pink}{Eberstalzell}}, \textcolor{pink}{Vorchdorf}{}\ledrightnote{\textcolor{pink}{Vorchdorf}}, \textcolor{pink}{St.
                        Florian}{}\ledrightnote{\textcolor{pink}{Sankt Florian}} und auf dem \textcolor{pink}{Pöſtlingberg}{}\ledrightnote{\textcolor{pink}{Pöstlingberg}}.
                    Näheres – falls Sie es intereſſieren ſollte – hoffe ich Ihnen münd{\pb}lich mitteilen zu können.\pend
           \pstart
           Mit den ergebenſten Grüßen Ihr\pend
           \pstart \spacefill\mbox{D\textsuperscript{r}RAdam}\pend{}\endnumbering\briefempfaengerindex{Schnitzler, Arthur@\textsc{Schnitzler, Arthur}!zzzAdam, Robert@\emph{von Robert Adam}!1918-10-011@{1. 10. 1918}|)be}\mylabel{h}  \normalsize

\doendnotes{C}
\bigskip
\vfill

\clearpage

\footnotesize

\lohead{\textsc{register}}

% Definiere theindex-Environment komplett neu ohne reledmac
\makeatletter
\renewenvironment{theindex}{%
  \section*{\indexname}%
  \setlength{\parindent}{0pt}%
  \setlength{\parskip}{0pt plus 0.3pt}%
  \let\item\@idxitem
}{%
  \clearpage
}
\makeatother

\IfFileExists{\jobname-pw.ind}{\input{\jobname-pw.ind}}{}

\end{document}

      