%% latex-korrekturansicht-vorspann.tex
%% Vorspann für die Korrekturansicht.
%% Lädt die gemeinsame Datei latex-vorspann.tex mit gesetztem Schalter.

\newif\ifkorrekturansicht
\korrekturansichttrue

\input{../tex-inputs/latex-vorspann}


\section[Theodor Herzl an Arthur Schnitzler, 29. 12. 1894]{L03841 Theodor Herzl an Arthur Schnitzler, 29. 12. 1894}
\nopagebreak\mylabel{L03841v}
\rehead{ }\normalsize\beginnumbering\briefempfaengerindex{, @\textsc{, }!zzz, @\emph{von  }!1894-12-291@{29. 12. 1894}|(be}
\toendnotes[C]{\smallbreak\pagebreak[2]}\Standort{CUL, Schnitzler, B 39.}
\physDesc{Brief, 1 Blatt, 1 Seite, 573 Zeichen
\newline{}Handschrift: schwarze Tinte, lateinische Kurrent
\newline{}Ordnung: mit Bleistift von unbekannter Hand nummeriert: »20« }\toendnotes[C]{\smallbreak}
\pstart
           {\pb}\textcolor{brown}{\textcolor{gray}{\textbf{NOUVELLE PRESSE LIBRE}}}\orgindex{Neue Freie Presse@Neue Freie Presse|pw}{}\ledrightnote{\textcolor{brown}{Neue Freie Presse}}\hfill \textcolor{pink}{\textcolor{gray}{\textbf{8, RUE DE MONCEAU }}}\oindex{8, Rue de Monceau@\textbf{8, Rue de Monceau}, \emph{Wohngebäude}|pw}{}\ledrightnote{\textcolor{pink}{8, Rue de Monceau}}\pend
           
\pstart
           \textcolor{gray}{\textbf{D\textsuperscript{r}{ }TH. HERZL}}\hfill 29. XII. 94\pend
           
\pstart{}Mein lieber Freund!\pend\vspace{0.5em}
\pstart
           Hier \label{K_L03841-1v}\edtext{III.}{\lemma{\textnormal{\emph{III.}}}\Cendnote{\textnormal{Das Manuskript des dritten Aktes des Schauspiels \emph{\textcolor{green}{Das neue Ghetto}\pwindex{Herzl, Theodor 2.\,5.\,1860 Budapest – 3.\,7.\,1904 Edlach@\textsc{Herzl, Theodor} (2.\,5.\,1860 Budapest – 3.\,7.\,1904 Edlach), \emph{Schriftsteller, Journalist}!neue Ghetto. Schauspiel in vier Acten@\strich\emph{Das neue Ghetto. Schauspiel in vier Acten}|pwk}}, das als Beilage den Brief
                  begleitete, ist nicht überliefert.}}}\label{K_L03841-1}\pend
           
\pstart
           Morgen wahrscheinlich IV und Titelblatt mit Personenverzeichniss,
               vielleicht auch schon der Begleitbrief. \pend
           
\pstart
           Dann werde ich auch Ihren lieben \label{K_L03841-2v}\edtext{vorgestrigen Brief}{\lemma{\textnormal{\emph{vorgestrigen Brief}}}\Cendnote{\textnormal{XXXX26.12.1894}}}\label{K_L03841-2} beantworten. \pend
           
\pstart
           Herzlich Ihr aufrichtiger {\\[\baselineskip]}\spacefill\mbox{Th Herzl}\pend
           \leftskip=0em{}
\pstart
           \noindent{}Seite 9 \label{K_L03841-3v}\edtext{des heutigen Mscpts}{\lemma{\textnormal{\emph{des heutigen Mscpts}}}\Cendnote{\textnormal{dritter Akt des Schauspiels \emph{\textcolor{green}{Das neue Ghetto}\pwindex{Herzl, Theodor 2.\,5.\,1860 Budapest – 3.\,7.\,1904 Edlach@\textsc{Herzl, Theodor} (2.\,5.\,1860 Budapest – 3.\,7.\,1904 Edlach), \emph{Schriftsteller, Journalist}!neue Ghetto. Schauspiel in vier Acten@\strich\emph{Das neue Ghetto. Schauspiel in vier Acten}|pwk}}}}}\label{K_L03841-3} (ich fing die
                  Nummerirung von vorn an weil ich die frühere Zahl nicht mehr wusste) Seite 9 in
                  der \label{K_L03841-4v}\edtext{Erzählung des Rabbiners}{\lemma{\textnormal{\emph{Erzählung des Rabbiners}}}\Cendnote{\textnormal{Die Figur des Rabbiners
                     Dr. Friedheimer referiert in der sechsten Szene des dritten Aktes eine
                     Anekdote aus einer alten Chronik, die in der Druckfassung auf den Monat »Ab des Jahres 5143« des jüdischen Kalenders datiert wird, s. \textcolor{blue}{Theodor Herzl}\pwindex{Herzl, Theodor 2.\,5.\,1860 Budapest – 3.\,7.\,1904 Edlach@\textsc{Herzl, Theodor} (2.\,5.\,1860 Budapest – 3.\,7.\,1904 Edlach), \emph{Schriftsteller, Journalist}|pwk}: \emph{\textcolor{green}{Das neue Ghetto. Schauspiel in 4 Acten}\pwindex{Herzl, Theodor 2.\,5.\,1860 Budapest – 3.\,7.\,1904 Edlach@\textsc{Herzl, Theodor} (2.\,5.\,1860 Budapest – 3.\,7.\,1904 Edlach), \emph{Schriftsteller, Journalist}!neue Ghetto. Schauspiel in vier Acten@\strich\emph{Das neue Ghetto. Schauspiel in vier Acten}|pwk}},
                        Wien: \emph{Buchdruckerei »Industrie« – Selbstverlag}{ }1903, S. 74. Das entspricht Juli oder August des Jahres 1383 nach christlicher Zeitrechnung.}}}\label{K_L03841-4}
                  ist der jüdische Name des Sommermonats (\label{K_L03841-5v}\edtext{\hebraeisch{Ab}}{\lemma{\textnormal{\emph{Ab}}}\Cendnote{\textnormal{hebräisch \hebraeisch{אָב}, av: elfter Monat des jüdischen Kalenders}}}\label{K_L03841-5}? \label{K_L03841-6v}\edtext{\hebraeisch{Nischan}}{\lemma{\textnormal{\emph{Nischan}}}\Cendnote{\textnormal{hebräisch \hebraeisch{ניסן}, nisan: siebter Monat des jüdischen Kalenders}}}\label{K_L03841-6}? ich weiss im Augenblick
                  nicht) und die geeignete Jahreszahl nachzutragen. Ich werde sie Ihnen
                     morgen schicken mit der Bitte sie einzuflicken\pend
           \selectlanguage{ngerman}\endnumbering\briefempfaengerindex{, @\textsc{, }!zzz, @\emph{von  }!1894-12-291@{29. 12. 1894}|)be}\mylabel{L03841h}
\begin{anhang}
\end{anhang}\normalsize

\doendnotes{C}
\bigskip
\vfill

\clearpage

\footnotesize

\lohead{\textsc{register}}

% Definiere theindex-Environment komplett neu ohne reledmac
\makeatletter
\renewenvironment{theindex}{%
  \section*{\indexname}%
  \setlength{\parindent}{0pt}%
  \setlength{\parskip}{0pt plus 0.3pt}%
  \let\item\@idxitem
}{%
  \clearpage
}
\makeatother

\IfFileExists{\jobname-pw.ind}{\input{\jobname-pw.ind}}{}

\end{document}

      