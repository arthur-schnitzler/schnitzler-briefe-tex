%% latex-korrekturansicht-vorspann.tex
%% Vorspann für die Korrekturansicht.
%% Lädt die gemeinsame Datei latex-vorspann.tex mit gesetztem Schalter.

\newif\ifkorrekturansicht
\korrekturansichttrue

\input{../tex-inputs/latex-vorspann}


\renewcommand{\erwaehntePersonen}{Personen: Maria Sandholzer, Leo Van-Jung}
\renewcommand{\erwaehnteOrte}{Orte: Bad Ischl, Hotel und Pension Rudolfshöhe (Leopold Petter), Linzer Gasse, Salzburg, Wien}
\renewcommand{\erwaehnteWerke}{}
\section[ Felix Salten an Arthur Schnitzler, 21. 8. 1897]{Felix Salten an Arthur Schnitzler, 21. 8. 1897}
\nopagebreak\mylabel{v}
\rehead{ }\normalsize\beginnumbering\briefempfaengerindex{Schnitzler, Arthur@\textsc{Schnitzler, Arthur}!zzzSalten, Felix@\emph{von Felix Salten}!1897-08-211@{21. 8. 1897}|(be}
\toendnotes[C]{\smallbreak\pagebreak[2]}\Standort{CUL, Schnitzler, B 89, A 2.}
\physDesc{Postkarte, 338 Zeichen
\newline{}Handschrift: Bleistift, lateinische Kurrent
\newline{}Versand: Stempel: »\nobreak{}\oindex{Salzburg@\textbf{Salzburg}, \emph{A.ADM2}|pwk}Salzburg-Stadt, 21/8 \textcolor{gray}{9}7\nobreak{}«. Stempel: »\nobreak{}\oindex{Bad Ischl@\textbf{Bad Ischl}, \emph{P.PPL}|pwk}Ischl, 21. 8. 97, 10–11 \textcolor{gray}{N}\nobreak{}«.  
\newline{}Ordnung: mit Bleistift von unbekannter Hand nummeriert: »95« }\toendnotes[C]{\smallbreak}\pstart{}{\pb}Herrn D\textsuperscript{r} Arthur Schnitzler \pend{}\pstart{}\textcolor{pink}{Ischl}{}\ledrightnote{\textcolor{pink}{Bad Ischl}}\pend{}\pstart{}\textcolor{pink}{Pension Rudolfshöhe}{}\ledrightnote{\textcolor{pink}{Hotel und Pension Rudolfshöhe (Leopold Petter)}}\pend{}
{\bigskip}
\pstart
           \noindent{}{\pb}lieber Arthur, bin Mittwoch mit \textcolor{blue}{Van Jung}{}\ledrightnote{\textcolor{blue}{Leo Van-Jung}} leider zu spät \label{K_L03272-1v}\edtext{hereingekommen}{\lemma{\textnormal{\emph{hereingekommen}}}\Cendnote{\textnormal{Vermutlich aus \textcolor{pink}{Pressbaum} nach \textcolor{pink}{Wien} (vgl. Felix Salten an Arthur Schnitzler, 13. 7. 1897). Am Folgetag, dem 18. 8. 1897, reiste
                     \textcolor{blue}{Schnitzler} nach \textcolor{pink}{Ischl}, so dass sie sich verpasst hatten.}}}\label{K_L03272-1h} und habe
               sehr bedauert, Sie nicht mehr sehen zu können. Bin seit heute{ }früh hier, \textcolor{pink}{Linzerstraße 74}{}\ledrightnote{\textcolor{pink}{Linzer Gasse}} bei
               Frau \textcolor{blue}{Sandholzer}{}\ledrightnote{\textcolor{blue}{Maria Sandholzer}}.\pend
           
\pstart
           Vielleicht \label{K_L03272-2v}\edtext{kommen Sie einmal her, oder
               ich nach \textcolor{pink}{Ischl}{}\ledrightnote{\textcolor{pink}{Bad Ischl}}}{\lemma{\textnormal{\emph{kommen … Ischl}}}\Cendnote{\textnormal{nicht geschehen, vgl. Felix Salten an Arthur Schnitzler, 31. 8. 1897}}}\label{K_L03272-2h}. Jedenfalls verständigen wir uns vorher davon.\pend
           
\pstart
           Herzlich {\\[\baselineskip]}\spacefill\mbox{Salten}\pend
           \leftskip=0em{}\endnumbering\briefempfaengerindex{Schnitzler, Arthur@\textsc{Schnitzler, Arthur}!zzzSalten, Felix@\emph{von Felix Salten}!1897-08-211@{21. 8. 1897}|)be}\mylabel{h}  \normalsize

\doendnotes{C}
\bigskip
\vfill

\clearpage

\footnotesize

\lohead{\textsc{register}}

% Definiere theindex-Environment komplett neu ohne reledmac
\makeatletter
\renewenvironment{theindex}{%
  \section*{\indexname}%
  \setlength{\parindent}{0pt}%
  \setlength{\parskip}{0pt plus 0.3pt}%
  \let\item\@idxitem
}{%
  \clearpage
}
\makeatother

\IfFileExists{\jobname-pw.ind}{\input{\jobname-pw.ind}}{}

\end{document}

      