%% latex-korrekturansicht-vorspann.tex
%% Vorspann für die Korrekturansicht.
%% Lädt die gemeinsame Datei latex-vorspann.tex mit gesetztem Schalter.

\newif\ifkorrekturansicht
\korrekturansichttrue

\input{../tex-inputs/latex-vorspann}


               \section[Arthur Schnitzler an Richard Beer-Hofmann, 4. 6. 1898]{ Arthur Schnitzler an Richard Beer-Hofmann, 4. 6. 1898}\nopagebreak\mylabel{v}\rehead{ }\normalsize\beginnumbering\briefempfaengerindex{Beer-Hofmann, Richard@\textsc{Beer-Hofmann, Richard}!zzzSchnitzler, Arthur@\emph{von Arthur Schnitzler}!1898-06-041@{4. 6. 1898}|(be} \toendnotes[C]{\smallbreak\pagebreak[2]} \Standort{YCGL, MSS 31.}
\physDesc{Brief, 1 Blatt, 3 Seiten, Umschlag
\newline{}Handschrift: Bleistift, deutsche Kurrent\newline{}Versand: 1) Stempel: »\nobreak{}\oindex{I., Innere Stadt@\textbf{I., Innere Stadt}, \emph{Bezirk (A.BZK)}|pwk}Wien 9/1, 4. 6. 98, 7–8 N\nobreak{}«.  2) Stempel: »\nobreak{}\oindex{Steindorf am Ossiacher See@\textbf{Steindorf am Ossiacher See}, \emph{http://www.geonames.org/ontologyA.ADM3}|pwk}Steindorf am Ossiacher See, 5 6 {[}98{]}\nobreak{}«. }\buchAbdrucke{\weitereDrucke{Arthur Schnitzler, Richard Beer-Hofmann: \emph{Briefwechsel 1891–1931}. Hg. Konstanze Fliedl. Wien, Zürich: \emph{Europaverlag} 1992, S. 117.} }\pstart{}{\pb}Herrn \textsc{Dr. Richard
                     Beer-Hofmann}\pend{}\pstart{}\textsc{\textcolor{pink}{Steindorf}{}\ledrightnote{\textcolor{pink}{Steindorf am Ossiacher See}}}\pend{}\pstart{}\textsc{am \textcolor{pink}{Ossiacher-See}{}\ledrightnote{\textcolor{pink}{Ossiacher See}}}\pend{}\pstart{}\textcolor{pink}{\textsc{Kärnthen}}{}\ledrightnote{\textcolor{pink}{Kärnten}}\pend{}{\bigskip}\pstart
           \raggedleft{}{\pb}Samſtag{ }Nachmitg{\\}4. 6. 98.\pend
           \pstart
           Lieber Richard, ich habe heute einen Postcarton an Ihre Adreſſe
               aufgegeben und komme bald nach. Morgen So{\geminationn}tag{ }früh 7.45 fahre ich auf den \textcolor{pink}{\textsc{Semmering}}{}\ledrightnote{\textcolor{pink}{Semmering}}; dort ſetz ich mich aufs Rad und will ſehn, wie weit ich komme. Von der {\pb}Reiſe aus verſtändige ich Sie.
                  Dinſtag bin ich wohl in \textcolor{pink}{\textsc{Steindorf}}{}\ledrightnote{\textcolor{pink}{Steindorf am Ossiacher See}}. Ob \textcolor{blue}{\textsc{Kramer}}{}\ledrightnote{\textcolor{blue}{Leopold Kramer}} mitfährt, iſt ungewiſs. Ich
               glaub nicht. Eben telephonirt er mir, dſs ihm ſein Rad geſtohlen worden iſt; er will
               ſich gleich ein neues kaufen, aber – zum mindeſtens das letztere {\pb}iſt unſahrscheinlich. –\pend
           \pstart
           Herzlichen Gruſs. Ihren Brief hab ich heute früh beko{\geminationm}en; – »bete und
               arbeite« – d. h. ſchreiben Sie und lernen Sie \textsc{Bicycle}fahren.\pend
           \pstart Ihr \spacefill\mbox{Arthur Sch}\pend{}\endnumbering\briefempfaengerindex{Beer-Hofmann, Richard@\textsc{Beer-Hofmann, Richard}!zzzSchnitzler, Arthur@\emph{von Arthur Schnitzler}!1898-06-041@{4. 6. 1898}|)be}\mylabel{h}  \normalsize

\doendnotes{C}
\bigskip
\vfill

\clearpage

\footnotesize

\lohead{\textsc{register}}

% Definiere theindex-Environment komplett neu ohne reledmac
\makeatletter
\renewenvironment{theindex}{%
  \section*{\indexname}%
  \setlength{\parindent}{0pt}%
  \setlength{\parskip}{0pt plus 0.3pt}%
  \let\item\@idxitem
}{%
  \clearpage
}
\makeatother

\IfFileExists{\jobname-pw.ind}{\input{\jobname-pw.ind}}{}

\end{document}

      