%% latex-korrekturansicht-vorspann.tex
%% Vorspann für die Korrekturansicht.
%% Lädt die gemeinsame Datei latex-vorspann.tex mit gesetztem Schalter.

\newif\ifkorrekturansicht
\korrekturansichttrue

\input{../tex-inputs/latex-vorspann}


               \section[Arthur Schnitzler an Richard Beer-Hofmann, 10. 10. 1907]{ Arthur Schnitzler an Richard Beer-Hofmann, 10. 10. 1907}\nopagebreak\mylabel{v}\rehead{ }\normalsize\beginnumbering\briefempfaengerindex{Beer-Hofmann, Richard@\textsc{Beer-Hofmann, Richard}!zzzSchnitzler, Arthur@\emph{von Arthur Schnitzler}!1907-10-101@{10. 10. 1907}|(be} \toendnotes[C]{\smallbreak\pagebreak[2]} \Standort{YCGL, MSS 31.}
\physDesc{Brief, 1 Blatt, 1 Seite, Umschlag
\newline{}Handschrift: Bleistift, deutsche Kurrent}\buchAbdrucke{\weitereDrucke{1) Arthur Schnitzler, Richard Beer-Hofmann: \emph{Briefwechsel 1891–1931}. Hg. Konstanze Fliedl. Wien, Zürich: \emph{Europaverlag} 1992, S. 185.} \weitereDrucke{2) Hermann Bahr, Arthur Schnitzler: \emph{Briefwechsel, Aufzeichnungen, Dokumente (1891–1931)}. Hg. Kurt Ifkovits und Martin Anton Müller. Göttingen: \emph{Wallstein} 2018, S. 397.} }\toendnotes[C]{\smallbreak}\pstart{}{\pb}\textcolor{gray}{\textbf{\label{KLL01719_Beer-Hofmann-1v}\edtext{Dr. Arthur
                        Schnitzler}{\lemma{\textnormal{\emph{Dr. Arthur
                        Schnitzler}}}\Cendnote{\textnormal{Der hier
                        das Korrespondenzstück ergänzende Umschlag wird unter den von \textcolor{blue}{Olga Schnitzler} geschickten
                        Korrespondenzstücken des Jahres 1907 aufbewahrt. Da bei diesen
                        kein Umschlag fehlt und unter der Annahme, dass die Jahresangabe stimmt, ist
                        es wahrscheinlich, dass der Umschlag zu diesem Brief gehört.}}}\label{KLL01719_Beer-Hofmann-1h}}}\pend{}\pstart{}\textcolor{gray}{\textbf{\textcolor{pink}{Wien XVIII. Spoettelgasse 7}{}\ledrightnote{\textcolor{pink}{Edmund-Weiß-Gasse}}.}}\pend{}{\bigskip}\pstart{}{\pb}\textsc{Dr. Richard Beerhofmann}\pend{}\pstart{}\textsc{\textcolor{pink}{Wien}{}\ledrightnote{\textcolor{pink}{Wien}}}\pend{}{\bigskip}\pstart
           \raggedleft{}{\pb}10. X. 907\pend
           \pstart{}lieber Richard,\pend\pstart
           \textcolor{blue}{Bahr}{}\ledrightnote{\textcolor{blue}{Hermann Bahr}} bittet mich Ihnen ſein \textcolor{green}{Stück}{}\ledrightnote{→\textcolor{green}{Die gelbe Nachtigall}} zu ſchicken. Hier iſt es.\pend
           \pstart
           Herzlichſt{\\[\baselineskip]}Ihr{\\[\baselineskip]}\spacefill\mbox{A.}\pend
           \leftskip=0em{}\pstart
           \noindent{}\textcolor{blue}{\textsc{Burckhardt}}{}\ledrightnote{\textcolor{blue}{Max Eugen Burckhard}} liegt bei \textcolor{pink}{\textsc{Loew}}{}\ledrightnote{\textcolor{pink}{Sanatorium Loew}}, mit einer (durch \textcolor{blue}{\textsc{Hajek}}{}\ledrightnote{\textcolor{blue}{Markus Hajek}} endlich geſtillten) ſchweren Naſenblutung. Ich geh jetzt hin\pend
           \endnumbering\briefempfaengerindex{Beer-Hofmann, Richard@\textsc{Beer-Hofmann, Richard}!zzzSchnitzler, Arthur@\emph{von Arthur Schnitzler}!1907-10-101@{10. 10. 1907}|)be}\mylabel{h}  \normalsize

\doendnotes{C}
\bigskip
\vfill

\clearpage

\footnotesize

\lohead{\textsc{register}}

% Definiere theindex-Environment komplett neu ohne reledmac
\makeatletter
\renewenvironment{theindex}{%
  \section*{\indexname}%
  \setlength{\parindent}{0pt}%
  \setlength{\parskip}{0pt plus 0.3pt}%
  \let\item\@idxitem
}{%
  \clearpage
}
\makeatother

\IfFileExists{\jobname-pw.ind}{\input{\jobname-pw.ind}}{}

\end{document}

      