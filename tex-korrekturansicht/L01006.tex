%% latex-korrekturansicht-vorspann.tex
%% Vorspann für die Korrekturansicht.
%% Lädt die gemeinsame Datei latex-vorspann.tex mit gesetztem Schalter.

\newif\ifkorrekturansicht
\korrekturansichttrue

\input{../tex-inputs/latex-vorspann}


               \section[Arthur Schnitzler an Richard Beer-Hofmann, 19. 12. 1899]{ Arthur Schnitzler an Richard Beer-Hofmann, 19. 12. 1899}\nopagebreak\mylabel{v}\rehead{ }\normalsize\beginnumbering\briefempfaengerindex{Beer-Hofmann, Richard@\textsc{Beer-Hofmann, Richard}!zzzSchnitzler, Arthur@\emph{von Arthur Schnitzler}!1899-12-191@{19. 12. 1899}|(be} \toendnotes[C]{\smallbreak\pagebreak[2]} \Standort{CUL, Schnitzler, B 8.}
\physDesc{Kartenbrief
\newline{}Handschrift: Bleistift, deutsche Kurrent\newline{}Versand: 1) Rohrpost 2) Stempel: »\nobreak{}\oindex{I., Innere Stadt@\textbf{I., Innere Stadt}, \emph{Bezirk (A.BZK)}|pwk}Wien 1/1, 19 XII 99, 4 10N\nobreak{}«. 3) Stempel: »\nobreak{}\oindex{I., Innere Stadt@\textbf{I., Innere Stadt}, \emph{Bezirk (A.BZK)}|pwk}Wien 1/1, 19 XII 99, 4 10N\nobreak{}«. }\toendnotes[C]{\smallbreak}\pstart{}{\pb}Herrn \textsc{Dr. Rich.
                     Beer-Hofmann}\pend{}\pstart{}\textcolor{pink}{Wien}{}\ledrightnote{\textcolor{pink}{Wien}}\pend{}\pstart{}\textsc{\textcolor{pink}{I. Wollzeile 15}{}\ledrightnote{\textcolor{pink}{Wollzeile}}}.\pend{}{\bigskip}\pstart
           \noindent{}{\pb}lieber Richard, bitte kommen Sie in die \textcolor{pink}{\textcolor{green}{\label{K_L01006_1v}\edtext{Loge 6}{\lemma{\textnormal{\emph{Loge 6}}}\Cendnote{\textnormal{Das \emph{\textcolor{brown}{Burgtheater}}
                        veranstaltete eine gemeinsame Aufführung von \textcolor{blue}{Schnitzler}s \emph{\textcolor{green}{Paracelus}} und \emph{\textcolor{green}{Die Gefährtin}} mit dem Dramenfragment \emph{\textcolor{green}{Esther}} von \textcolor{blue}{Franz Grillparzer}.}}}\label{K_L01006_1h}}{}\ledrightnote{\textcolor{green}{Esther}{\newline}\textcolor{green}{Paracelsus. Versspiel in einem Akt}{\newline}\textcolor{green}{Die Gefährtin. Schauspiel in einem Akt}}}{}\ledrightnote{→\textcolor{pink}{Burgtheater}}, rechts, 1. Gallerie!\pend
           \pstart
           Ich ſelbſt bin bei \textcolor{blue}{Schlenther}{}\ledrightnote{\textcolor{blue}{Paul Schlenther}}. Nachtmahlen ka{\geminationn} ich nicht mit Ihnen; \textcolor{blue}{ſchweſterlicher}{}\ledrightnote{→\textcolor{blue}{Gisela Hajek}} Geburtstag.\pend
           \pstart
           Herzlichſt Ihr{\\[\baselineskip]}\spacefill\mbox{Arthur}\pend
           \leftskip=0em{}\endnumbering\briefempfaengerindex{Beer-Hofmann, Richard@\textsc{Beer-Hofmann, Richard}!zzzSchnitzler, Arthur@\emph{von Arthur Schnitzler}!1899-12-191@{19. 12. 1899}|)be}\mylabel{h}  \normalsize

\doendnotes{C}
\bigskip
\vfill

\clearpage

\footnotesize

\lohead{\textsc{register}}

% Definiere theindex-Environment komplett neu ohne reledmac
\makeatletter
\renewenvironment{theindex}{%
  \section*{\indexname}%
  \setlength{\parindent}{0pt}%
  \setlength{\parskip}{0pt plus 0.3pt}%
  \let\item\@idxitem
}{%
  \clearpage
}
\makeatother

\IfFileExists{\jobname-pw.ind}{\input{\jobname-pw.ind}}{}

\end{document}

      