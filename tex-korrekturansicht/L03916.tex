%% latex-korrekturansicht-vorspann.tex
%% Vorspann für die Korrekturansicht.
%% Lädt die gemeinsame Datei latex-vorspann.tex mit gesetztem Schalter.

\newif\ifkorrekturansicht
\korrekturansichttrue

\input{../tex-inputs/latex-vorspann}


\section[Arthur Schnitzler an Theodor Herzl, XXXX]{L03916 Arthur Schnitzler an Theodor Herzl, XXXX}
\nopagebreak\mylabel{L03916v}
\rehead{ }\normalsize\beginnumbering\briefempfaengerindex{, @\textsc{, }!zzz, @\emph{von  }!1901-01-141@{{[}Erste Woche 1901?{]}}|(be}
\toendnotes[C]{\smallbreak\pagebreak[2]}\Standort{Jerusalem, Central Zionist Archives, H1\1926-8.}
\physDesc{Briefkarte, 389 Zeichen
\newline{}Handschrift: schwarze Tinte, deutsche Kurrent}\toendnotes[C]{\smallbreak}
\pstart
           \noindent{}{\pb}lieber Doctor Herzl, ſeien Sie nicht
               ungehalten, dſs ich \uline{Sie} Frage, aber ich wüßte nicht
               wen ſonſt, da ich nur mit Ihnen in dieſer Sache verhandelt habe: an wen hab ich mich \substVorne{}\textsuperscript{wegen}\substDazwischen{}mit\substHinten{} meinen \textcolor{green}{Honoraranſprüchen}\pwindex{Schnitzler, Arthur 15. 5. 1862 Wien – 21. 10. 1931 ebd.@\textsc{Schnitzler, Arthur} (15. 5. 1862 Wien – 21. 10. 1931 ebd.), \emph{Schriftsteller, Mediziner}!Lieutenant Gustl. Novelle@\strich\emph{Lieutenant Gustl. Novelle}|pwv}{}\ledrightnote{{$\rightarrow$}\emph{\textcolor{green}{Lieutenant Gustl. Novelle}}} zu wenden? Man hat nemlich voll{\pb}ko{\geminationm}en vergeſſen, die Länge meiner \textcolor{green}{Novelle}\pwindex{Schnitzler, Arthur 15. 5. 1862 Wien – 21. 10. 1931 ebd.@\textsc{Schnitzler, Arthur} (15. 5. 1862 Wien – 21. 10. 1931 ebd.), \emph{Schriftsteller, Mediziner}!Lieutenant Gustl. Novelle@\strich\emph{Lieutenant Gustl. Novelle}|pwv}{}\ledrightnote{{$\rightarrow$}\emph{\textcolor{green}{Lieutenant Gustl. Novelle}}} in Betracht zu ziehen, was ich bei
               Einſendg ausdrücklich betonte.\pend
           
\pstart
           Verbindlichſt grüßend{\\[\baselineskip]}Ihr ergebn\textcolor{gray}{er}{\\[\baselineskip]}\spacefill\mbox{Arthur Schnitzler}\pend
           \leftskip=0em{}\selectlanguage{ngerman}\endnumbering\briefempfaengerindex{, @\textsc{, }!zzz, @\emph{von  }!1901-01-011@{{[}Erste Woche 1901?{]}}|)be}\mylabel{L03916h}
\begin{anhang}
\end{anhang}\normalsize

\doendnotes{C}
\bigskip
\vfill

\clearpage

\footnotesize

\lohead{\textsc{register}}

% Definiere theindex-Environment komplett neu ohne reledmac
\makeatletter
\renewenvironment{theindex}{%
  \section*{\indexname}%
  \setlength{\parindent}{0pt}%
  \setlength{\parskip}{0pt plus 0.3pt}%
  \let\item\@idxitem
}{%
  \clearpage
}
\makeatother

\IfFileExists{\jobname-pw.ind}{\input{\jobname-pw.ind}}{}

\end{document}

      