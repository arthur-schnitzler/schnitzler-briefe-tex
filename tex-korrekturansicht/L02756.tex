%% latex-korrekturansicht-vorspann.tex
%% Vorspann für die Korrekturansicht.
%% Lädt die gemeinsame Datei latex-vorspann.tex mit gesetztem Schalter.

\newif\ifkorrekturansicht
\korrekturansichttrue

\input{../tex-inputs/latex-vorspann}


               \section[Paul Goldmann an Arthur Schnitzler, 17. 10. {[}1895{]}]{ Paul Goldmann an Arthur Schnitzler, 17. 10. {[}1895{]}}\nopagebreak\mylabel{v}\rehead{ }\normalsize\beginnumbering\briefempfaengerindex{Schnitzler, Arthur@\textsc{Schnitzler, Arthur}!zzzGoldmann, Paul@\emph{von Paul Goldmann}!1895-10-171@{17. 10. {[}1895{]}}|(be} \toendnotes[C]{\smallbreak\pagebreak[2]} \Standort{DLA, A:Schnitzler, HS.NZ85.1.3165.}
\physDesc{Brief, 2 Blätter, 7 Seiten
\newline{}Handschrift: blaue Tinte, deutsche Kurrent
\newline{}Schnitzler: 1) mit Bleistift eine Unterstreichung, eine seitliche Markierung
                                 und das Jahr »95« vermerkt 2) mit rotem Buntstift acht Unterstreichungen}\toendnotes[C]{\smallbreak}\pstart
           \noindent{}{\pb}\textcolor{gray}{\textbf{\textbf{\textcolor{brown}{Frankfurter Zeitung}{}\ledrightnote{\textcolor{brown}{Frankfurter Zeitung}}}}}\pend
           \pstart
           \textcolor{gray}{\textbf{(\textcolor{brown}{\begin{otherlanguage}{french}Gazette de Francfort\end{otherlanguage}}{}\ledrightnote{\textcolor{brown}{Frankfurter Zeitung}}). }}\pend
           \pstart
           \textcolor{gray}{\textbf{\textbf{\begin{otherlanguage}{french}Fondateur M. \textcolor{blue}{L.
                              Sonnemann}{}\ledrightnote{\textcolor{blue}{Leopold Sonnemann}}\end{otherlanguage}.}}}\pend
           \pstart
           \begin{otherlanguage}{french}\textcolor{gray}{\textbf{\textcolor{green}{Journal}{}\ledrightnote{→\textcolor{green}{Frankfurter Zeitung}} politique,
                           financier,}}\end{otherlanguage}\hfill \textsc{\textcolor{pink}{Paris}{}\ledrightnote{\textcolor{pink}{Paris}}}, 17. Oktober.\pend
           \pstart
           \begin{otherlanguage}{french}\textcolor{gray}{\textbf{commercial et littéraire.}}\end{otherlanguage}\pend
           \pstart
           \begin{otherlanguage}{french}\textcolor{gray}{\textbf{\textbf{Paraissant trois fois par jour.}}}\end{otherlanguage}\pend
           \pstart
           \begin{otherlanguage}{french}\textcolor{gray}{\textbf{\textbf{Bureau à \textcolor{pink}{Paris}{}\ledrightnote{\textcolor{pink}{Paris}}:}}}\end{otherlanguage}\pend
           \pstart
           \begin{otherlanguage}{french}\textcolor{gray}{\textbf{\textbf{\textcolor{pink}{24. Rue Feydeau}{}\ledrightnote{\textcolor{pink}{rue Feydeau}}.}}}\end{otherlanguage}\pend
           \pstart\center{}Mein lieber Freund,\pend\pstart
           Herzlichſten Dank für die Kritiken! Das iſt gar eine amüſante Lectüre. Wie \strikeout{D} Dein Bild da aus \strikeout{all}
               all’ den Spiegeln der Öffentlichkeit zurückgeworfen wird! Aber manchmal ſieht es mich
               auch fremd an, ſchmerzlich fremd, und meine trüben Ahnungen kommen wieder. Ja, ja,
               laß’ nur! Es iſt Unſinn, ich weiß{\dotsfive}\pend
           \pstart
           Sehr intereſſant, dieſe Lectüre. Über \textsc{\textcolor{green}{\textcolor{blue}{Speidel}{}\ledrightnote{\textcolor{blue}{Ludwig Speidel}}}{}\ledrightnote{→\textcolor{green}{Theater- und Kunstnachrichten. [Burgtheater] [Liebelei, Rechte der Seele]}}} ſchrieb {\pb}ich Dir ſchon. \label{K_L02756-1v}\edtext{\textsc{\textcolor{blue}{\textcolor{green}{Kalbeck}{}\ledrightnote{→\textcolor{green}{Theater, Kunst und Literatur. Burgtheater [Liebelei, Rechte der Seele]}{\newline}→\textcolor{green}{Burgtheater. »Liebelei«, Schauspiel in drei Acten von Arthur Schnitzler. – »Rechte der Seele«, Schauspiel in einem Acte von Guiseppe Giacosa; deutsch von Otto Eisenschitz}}}{}\ledrightnote{\textcolor{blue}{Max Kalbeck}}}}{\lemma{\textnormal{\emph{Kalbeck}}}\Cendnote{\textnormal{Nachtkritik: \textcolor{blue}{M. K.} [=\textcolor{blue}{Max Kalbeck}]: \emph{\textcolor{green}{Theater,
                        Kunst und Literatur. Burgtheater}}. In: \emph{\textcolor{green}{Neues Wiener Tagblatt}}, Jg. 29, Nr. 278, 10. 10. 1895, S. 7 und Feuilleton: \textcolor{blue}{Max Kalbeck}: \emph{\textcolor{green}{Burgtheater. »Liebelei«, Schauspiel in drei Acten von Arthur
                        Schnitzler. – »Rechte der Seele«, Schauspiel in einem Acte von Guiseppe
                        Giacosa; deutsch von Otto Eisenschitz}}. In: \emph{\textcolor{green}{Neues Wiener Tagblatt}}, Jg. 29, Nr. 279, 11. 10. 1895, S. 1–3.}}}\label{K_L02756-1h} iſt
               unerträglich ſchwülſtig geſchrieben. Gefällt ihm das \textcolor{green}{Stück}{}\ledrightnote{→\textcolor{green}{Liebelei. Schauspiel in drei Akten}} wirklich ſo? Oder hat er nur
               vernommen, daß es \textsc{\textcolor{blue}{Speidel}{}\ledrightnote{\textcolor{blue}{Ludwig Speidel}}} loben würde und ſich darum beeilt, um die Wette zu loben, – auf Seiten der
               Mächtigen, wie immer? Ich glaube, der iſt kein echter, auf den kannſt Du Dich nicht
               verlaſſen, – wohl aber auf \textsc{\textcolor{blue}{Speidel}{}\ledrightnote{\textcolor{blue}{Ludwig Speidel}}}. Schön iſt das Wohlwollen und die Sympathie, die faſt bei \uline{Allen} zutage tritt. Einiges davon iſt wohl auf Rechnung des \textcolor{pink}{Wien}{}\ledrightnote{\textcolor{pink}{Wien}}eriſchen zu ſetzen, die {\pb}Hauptſache aber kommt aus der Achtung und dem
               Reſpect vor dem \uline{Menſchen}{ }\textsc{Schnitzler}. Durch warmen, \textcolor{gray}{×}\-\textcolor{gray}{×}\-\textcolor{gray}{×} herzlichen, neidloſen Ton ragt vor Allem \label{K_L02756-2v}\edtext{\textsc{\textcolor{green}{\textcolor{blue}{Hirschfeld}{}\ledrightnote{\textcolor{blue}{Robert Hirschfeld}}}{}\ledrightnote{→\textcolor{green}{Burgtheater. (»Liebelei« von Arthur Schnitzler. – »Rechte der Seele« von Giacosa.)}}}}{\lemma{\textnormal{\emph{Hirschfeld}}}\Cendnote{\textnormal{\textcolor{blue}{L. A. Terne} [=\textcolor{blue}{Robert Hirschfeld}]: \emph{\textcolor{green}{Burgtheater. (»Liebelei« von Arthur Schnitzler. – »Rechte der Seele« von
                        Giacosa.)}} In: \emph{\textcolor{green}{Wiener Sonn- und
                        Montags-Zeitung}}, Jg. 33, Nr. 41, 14. 10. 1895, S. 1–3.}}}\label{K_L02756-2h} hervor. Das iſt Einer, der ſich
               wirklich mit Deinem Talent und Deinem Erfolge freut. Das Schönſte aber iſt – es iſt
               ſeltſam, daß ich dieſem widerwärtigen \textcolor{blue}{Menſchen}{}\ledrightnote{→\textcolor{blue}{Jakob Julius David}} das Zugeſtändniß machen muß – \label{K_L02756-3v}\edtext{\textsc{\uline{\textcolor{blue}{J. J. David}{}\ledrightnote{\textcolor{blue}{Jakob Julius David}}s}}{ }\textcolor{green}{Feuilleton}{}\ledrightnote{→\textcolor{green}{Arthur Schnitzler}}}{\lemma{\textnormal{\emph{J. J. Davids Feuilleton}}}\Cendnote{\textnormal{\textcolor{blue}{–v–} [=\textcolor{blue}{J.
                        J. David}]: \emph{\textcolor{green}{Arthur Schnitzler}}. In:
                        \emph{\textcolor{green}{Neues Wiener Journal}}, Jg. 3, Nr. 703,
                        9. 10. 1895, S. 1–2. (Am Tag der
                  Uraufführung). Zusätzlich dazu verfasste \textcolor{blue}{David} eine Nachtkritik: \textcolor{blue}{–v–} [=\textcolor{blue}{J.
                        J. David}]: \emph{\textcolor{green}{Theater und Kunst.
                        (Burgtheater.)}} In: \emph{\textcolor{green}{Neues Wiener
                        Journal}}, Jg. 3, Nr. 704, 10. 10. 1895, S. 5.}}}\label{K_L02756-3h}
               über Dich. Das iſt prächtig geſchrieben, das iſt ein klug und wahr gezeichnetes
               Seelenbild von Dir und das ſchlägt {\pb}in meinem Innern
               liebe Saiten an, die lange nicht geklungen. Es hat mich tief berührt, und ich will
               dem \textcolor{blue}{Manne}{}\ledrightnote{→\textcolor{blue}{Jakob Julius David}} Manches um
               deßwillen verzeihen. \label{K_L02756-886v}\edtext{\textsc{\uline{\textcolor{blue}{\textcolor{green}{Bauer}{}\ledrightnote{→\textcolor{green}{Hofburgtheater [Rechte der Seele, Liebelei]}}}{}\ledrightnote{\textcolor{blue}{Julius Bauer}}}}}{\lemma{\textnormal{\emph{Bauer}}}\Cendnote{\textnormal{[\textcolor{blue}{Julius Bauer}]: \emph{\textcolor{green}{Hofburgtheater}}. In: \emph{\textcolor{green}{Illustriertes Wiener Extrablatt}}, Jg. 24, Nr. 278, 10. 10. 1895, S. 5.}}}\label{K_L02756-886h} tadelt den
               Schluß, und hat vielleicht nicht Unrecht. \label{K_L02756-88v}\edtext{\textsc{\textcolor{blue}{\uline{\textcolor{green}{Hevesi}{}\ledrightnote{→\textcolor{green}{Burgtheater. (Herr Mitterwurzer als König Philipp. – »Rechte der Seele«, von Guiseppe Giacosa. – »Liebelei«, von Arthur Schnitzler.)}{\newline}→\textcolor{green}{Burgtheater. (»Rechte der Seele«, Schauspiel in einem Akt von Giuseppe Giacosa. – »Liebelei«, Schauspiel in drei Aufzügen von Arthur Schnitzler.)}}}}{}\ledrightnote{\textcolor{blue}{Ludwig Hevesi}}}}{\lemma{\textnormal{\emph{Hevesi}}}\Cendnote{\textnormal{\textcolor{blue}{L. H–i} [=\textcolor{blue}{Ludwig Hevesi}]: \emph{\textcolor{green}{Burgtheater. (»Rechte der Seele«, Schauspiel in einem Akt von Giuseppe
                        Giacosa. – »Liebelei«, Schauspiel in drei Aufzügen von Arthur
                        Schnitzler.)}} In: \emph{\textcolor{green}{Fremden-Blatt}},
                     Jg. 51, Nr. 279, 11. 10. 1895, S. 13–14.
                  Unter den Zeitungsausschnitten \textcolor{blue}{Schnitzler}s
                  findet sich auch eine zweite Fassung, offenbar für eine Zeitung außerhalb \textcolor{pink}{Wien}s verfasst (\emph{\textcolor{green}{Breslauer Zeitung}}?): \textcolor{blue}{L. H–i} [=\textcolor{blue}{Ludwig Hevesi}]: \emph{\textcolor{green}{Burgtheater. (Herr Mitterwurzer als König Philipp. – »Rechte der Seele«,
                        von Guiseppe Giacosa. – »Liebelei«, von Arthur Schnitzler.)}}.}}}\label{K_L02756-88h}{ }\strikeout{m} iſt vortrefflich und geſcheit, beſonders \strikeout{das}, was er über die Paradoxe ſagt, ſind goldene Worte.
                  \label{K_L02756-5v}\edtext{\textsc{\uline{\textcolor{blue}{\textcolor{green}{Uhl}{}\ledrightnote{→\textcolor{green}{K. k. Hofburgtheater: »Rechte der Seele«, Schauspiel in einem Acte von Giuseppe Giacosa. – »Liebelei«, Schauspiel in drei Acten von Arthur Schnitzler. Zum ersten Male aufgeführt am 9. October}}}{}\ledrightnote{\textcolor{blue}{Friedrich Uhl}}}}}{\lemma{\textnormal{\emph{Uhl}}}\Cendnote{\textnormal{[\textcolor{blue}{Friedrich Uhl}]: \emph{\textcolor{green}{K. k. Hofburgtheater: »Rechte der Seele«, Schauspiel in
                        einem Acte von Giuseppe Giacosa. – »Liebelei«, Schauspiel in drei Acten von
                        Arthur Schnitzler. Zum ersten Male aufgeführt am 9. October}}. In: \emph{\textcolor{green}{Wiener Abendpost}}, Nr. 234, 10. 10. 1895, S. 1–2.}}}\label{K_L02756-5h} iſt merkwürdig
               boshaft, hat \strikeout{ſichtlich} ſichtlich in der Abſicht
               geſchrieben, Dir wehzuthun, packt das \textcolor{green}{Stück}{}\ledrightnote{→\textcolor{green}{Liebelei. Schauspiel in drei Akten}} viel zu ſchwer {\pb}an,
               ſagt aber ſchließlich doch manches Beherzigenswerthe; ſein Tadel gegen die Figur des
               Vaters iſt viel zu \strikeout{hefti} heftig ausgedrückt, aber im
               Grunde ſcheint er Recht zu haben. Durch beſondere Dummheit zeichnet ſich \label{K_L02756-65v}\edtext{\textsc{\uline{\textcolor{blue}{\textcolor{green}{Bunzl}{}\ledrightnote{→\textcolor{green}{Burgtheater. »Rechte der Seele«, Schauspiel in einem Akt von Giuseppe Giacosa. Deutsch von Otto Eisenschütz. – »Liebelei«, Schauspiel in drei Akten von Arthur Schnitzler. Zum erstenmale aufgeführt am 9. Oktober}}}{}\ledrightnote{\textcolor{blue}{Arthur Bunzl}}}}}{\lemma{\textnormal{\emph{Bunzl}}}\Cendnote{\textnormal{\textcolor{blue}{Arthur Bunzl}: \emph{\textcolor{green}{Burgtheater. »Rechte der Seele«, Schauspiel in einem Akt von
                        Giuseppe Giacosa. Deutsch von Otto Eisenschütz. – »Liebelei«, Schauspiel in
                        drei Akten von Arthur Schnitzler. Zum erstenmale aufgeführt am 9. Oktober}}. In: \emph{\textcolor{green}{Österreichische
                        Volks-Zeitung}}, Jg. 41, Nr. 279, 11. 10. 1895, S. 1–2.}}}\label{K_L02756-65h} aus; er war aber immer ein Ochs.
               Köſtlich iſt die künſleriſche Strenge des »\label{K_L02756-22v}\edtext{\textcolor{green}{\textcolor{green}{Neuigkeits-Weltblatts}{}\ledrightnote{\textcolor{green}{Neuigkeits-Welt-Blatt}}}{}\ledrightnote{→\textcolor{green}{Hofburgtheater. (»Rechte der Seele«, Schauspiel in einem Akte von Guiseppe Giacosa. – »Liebelei«, Schauspiel in drei Akten von Arthur Schnitzler. – Erstaufführung am 9. Oktober 1895.)}}}{\lemma{\textnormal{\emph{Neuigkeits-Weltblatts}}}\Cendnote{\textnormal{\textcolor{blue}{Alpha}: \emph{\textcolor{green}{Hofburgtheater. (»Rechte der Seele«, Schauspiel in einem Akte von Guiseppe
                        Giacosa. – »Liebelei«, Schauspiel in drei Akten von Arthur Schnitzler. –
                        Erstaufführung am 9. Oktober 1895.)}} In: \emph{\textcolor{green}{Neuigkeits-Welt-Blatt}}, Jg. 22, Nr. 235, 12. 10. 1895, S. 10.}}}\label{K_L02756-22h}«. Hübſch ſind auch die \label{K_L02756-999v}\edtext{\textcolor{blue}{\textcolor{green}{\uline{Socialiſten}}{}\ledrightnote{→\textcolor{green}{Burgtheater. [Rechte der Seele, Liebelei]}}}{}\ledrightnote{→\textcolor{blue}{Edmund Wengraf}}}{\lemma{\textnormal{\emph{Socialiſten}}}\Cendnote{\textnormal{\textcolor{blue}{e. w.} [=\textcolor{blue}{Edmund Wengraf}]: \emph{\textcolor{green}{Burgtheater}}. In: \emph{\textcolor{green}{Arbeiter-Zeitung}}, Jg. 7, Nr. 279, 11. 10. 1895, Morgenblatt, S. 5.}}}\label{K_L02756-999h}, welche unzufrieden
               ſind, {\pb}weil das \textcolor{green}{Stück}{}\ledrightnote{→\textcolor{green}{Liebelei. Schauspiel in drei Akten}} nicht nach Dreck ſtinkt: »\label{K_L02756-444v}\edtext{\textcolor{green}{Das iſt nicht das wahre Volk}{}\ledrightnote{→\textcolor{green}{Burgtheater. [Rechte der Seele, Liebelei]}}}{\lemma{\textnormal{\emph{Das … Volk}}}\Cendnote{\textnormal{Paraphrase, kein direktes
               Zitat}}}\label{K_L02756-444h}«. Daß ſelbſt die \textcolor{blue}{\uline{Antiſemiten}}{}\ledrightnote{→\textcolor{blue}{r. p.}} über Dich ſympathiſch ſchreiben (»\label{K_L02756-111v}\edtext{\textcolor{green}{\textcolor{green}{Reichspoſt}{}\ledrightnote{\textcolor{green}{Reichspost}}}{}\ledrightnote{→\textcolor{green}{k. k. Hofburgtheater [Rechte der Seele, Liebelei]}}}{\lemma{\textnormal{\emph{Reichspoſt}}}\Cendnote{\textnormal{\textcolor{blue}{r. p.}: \emph{\textcolor{green}{k. k. Hofburgtheater}}. In: \emph{\textcolor{green}{Reichspost}}, Jg. 2, Nr. 235, 12. 10. 1895, S. 1.}}}\label{K_L02756-111h}«), iſt ein
               wahrer Triumph für Dich
               und beweiſt abermals,
               daß der Antiſemitismus ſich nur gegen
               die widerlichen Saujuden richtet und vor dem ehrenhaften und tüchtigen Juden
               entwaffnen muß. \label{K_L02756-87v}\edtext{\textsc{\uline{\textcolor{blue}{\textcolor{green}{Granichstaedten}{}\ledrightnote{→\textcolor{green}{Burgtheater. Zwei Schauspiele: »Rechte der Seele« von Giuseppe Giacosa. – »Liebelei« von Arthur Schnitzler}}}{}\ledrightnote{\textcolor{blue}{Emil Granichstaedten}}}}}{\lemma{\textnormal{\emph{Granichstaedten}}}\Cendnote{\textnormal{\textcolor{blue}{Emil Granichstaedten}: \emph{\textcolor{green}{Burgtheater. Zwei Schauspiele: »Rechte der Seele« von
                        Giuseppe Giacosa. – »Liebelei« von Arthur Schnitzler}}. In: \emph{\textcolor{green}{Die Presse}}, Jg. 48, Nr. 279, 11. 10. 1895, S. 1–2.}}}\label{K_L02756-87h} iſt ſo
               ungeſchickt und offen gemein, daß es {\pb}nicht einmal
               empört; jede Zeile ſagt ſelbſt dem \strikeout{\textcolor{gray}{nic}} nichteingeweihten Leſer im Vertrauen, daß der \textcolor{blue}{Verfaſſer}{}\ledrightnote{→\textcolor{blue}{Emil Granichstaedten}} lügt{\dotsfive}\pend
           \pstart
           Das Geſammtbild iſt glänzend; und der Erfolg iſt ſo groß, wie ich ihn nur irgend für
               Dich wünſchen konnte. Jetzt mach’ Dich bald und frohen Muthes an die neue Arbeit!\pend
           \pstart
           Viele treue Grüße! {\\[\baselineskip]}Dein {\\[\baselineskip]}\spacefill\mbox{Paul Goldmann.}\pend
           \leftskip=0em{}\endnumbering\briefempfaengerindex{Schnitzler, Arthur@\textsc{Schnitzler, Arthur}!zzzGoldmann, Paul@\emph{von Paul Goldmann}!1895-10-171@{17. 10. {[}1895{]}}|)be}\mylabel{h}\begin{anhang}\end{anhang}\normalsize

\doendnotes{C}
\bigskip
\vfill

\clearpage

\footnotesize

\lohead{\textsc{register}}

% Definiere theindex-Environment komplett neu ohne reledmac
\makeatletter
\renewenvironment{theindex}{%
  \section*{\indexname}%
  \setlength{\parindent}{0pt}%
  \setlength{\parskip}{0pt plus 0.3pt}%
  \let\item\@idxitem
}{%
  \clearpage
}
\makeatother

\IfFileExists{\jobname-pw.ind}{\input{\jobname-pw.ind}}{}

\end{document}

      