%% latex-korrekturansicht-vorspann.tex
%% Vorspann für die Korrekturansicht.
%% Lädt die gemeinsame Datei latex-vorspann.tex mit gesetztem Schalter.

\newif\ifkorrekturansicht
\korrekturansichttrue

\input{../tex-inputs/latex-vorspann}


         
         \renewcommand{\erwaehntePersonen}{Personen: Hermann Bahr, Julius Bauer, Jakob Julius David, Robert Hirschfeld, Anton Reitler, Felix Salten, Paul Schlenther, Olga Schnitzler, Ludwig Speidel, Elisabeth Steinrück}
         \renewcommand{\erwaehnteInstitutionen}{Institutionen: Burgtheater, Vossische Zeitung}
         \renewcommand{\erwaehnteOrte}{Orte: Berlin, Dessauer Straße, Rotensterngasse, Wien, Österreich}
         \renewcommand{\erwaehnteWerke}{Werke: Der Schleier der Beatrice. Schauspiel in fünf Akten, Erklärung [Schleier der Beatrice], Vossische Zeitung, Wiener Leben}
               \section[ Paul Goldmann an Arthur Schnitzler, 5. 10. {[}1900{]}]{Paul Goldmann an Arthur Schnitzler, 5. 10. {[}1900{]}}\nopagebreak\mylabel{v}\rehead{ }\normalsize\beginnumbering\briefempfaengerindex{Schnitzler, Arthur@\textsc{Schnitzler, Arthur}!zzzGoldmann, Paul@\emph{von Paul Goldmann}!1900-10-051@{5. 10. {[}1900{]}}|(be} \toendnotes[C]{\smallbreak\pagebreak[2]} \Standort{DLA, A:Schnitzler, HS.NZ85.1.3170.}
\physDesc{Brief, 1 Blatt, 3 Seiten
\newline{}Handschrift: blaue Tinte, deutsche Kurrent\newline{}Beilage: ein aufgeklebter beschnittener Zeitungsausschnitt 
\newline{}Schnitzler: 1) mit Bleistift das Jahr »{[}1{]}900« vermerkt  2) mit rotem Buntstift zwei Unterstreichungen}\toendnotes[C]{\smallbreak}\pstart
           \noindent{}{\pb}\textcolor{pink}{Berlin}{}\ledrightnote{\textcolor{pink}{Berlin}}, 5. Oktober.\hfill \textcolor{pink}{\textcolor{gray}{\textbf{DESSAUERSTRASSE 19}}}{}\ledrightnote{\textcolor{pink}{Dessauer Straße}}\pend
           \pstart\center{}Mein lieber Freund,\pend\pstart
           Ein Herr \textsc{\textcolor{blue}{Anton Reitler}{}\ledrightnote{\textcolor{blue}{Anton Reitler}}} (?) läßt ſich in einem \label{K_L02935-1v}\edtext{\textcolor{green}{Wiener Briefe}{}\ledrightnote{{$\rightarrow$}\textcolor{green}{Wiener Leben}}}{\lemma{\textnormal{\emph{Wiener Briefe}}}\Cendnote{\textnormal{\textcolor{blue}{Anton Reitler}: \emph{\textcolor{green}{Wiener Leben}}. In: \emph{\textcolor{green}{Vossische Zeitung}}, Nr. 466, 5. 10. 1900, Morgen-Ausgabe, S. [16].}}}\label{K_L02935-1h} in der »\textcolor{green}{Voſſiſchen Zeitung}{}\ledrightnote{\textcolor{green}{Vossische Zeitung}}« heut folgendermaßen aus:\pend
           {\bigskip}\pstart
           \noindent{}\textcolor{gray}{\textbf{Ein anderes Ereigniß, das mit dem \textcolor{brown}{Theater}{}\ledrightnote{{$\rightarrow$}\textcolor{brown}{Burgtheater}} in Zuſammenhang ſtand, beginnt
                  bereits dem Gedächtnis der Zeitgenoſſen zu entſchwinden: Die \label{K_L02935-2v}\edtext{Affaire Schnitzler-\textcolor{blue}{Schlenther}{}\ledrightnote{\textcolor{blue}{Paul Schlenther}}}{\lemma{\textnormal{\emph{Affaire Schnitzler-Schlenther}}}\Cendnote{\textnormal{siehe Richard Beer-Hofmann an Arthur Schnitzler, 14. 9. 1900}}}\label{K_L02935-2h}. \textcolor{blue}{Schlenther}{}\ledrightnote{\textcolor{blue}{Paul Schlenther}} ſoll das neue \textcolor{green}{Stück}{}\ledrightnote{{$\rightarrow$}\textcolor{green}{Der Schleier der Beatrice. Schauspiel in fünf Akten}} Schnitzlers »\textcolor{green}{Der Schleier der Beatrice}{}\ledrightnote{\textcolor{green}{Der Schleier der Beatrice. Schauspiel in fünf Akten}}« im Januar für das \textcolor{brown}{Burgtheater}{}\ledrightnote{\textcolor{brown}{Burgtheater}} angenommen, im September
                  abgelehnt haben, was die Vormünder der \textcolor{pink}{öſterreich}{}\ledrightnote{\textcolor{pink}{Österreich}}iſchen dramatiſchen Produktion zu einem flammenden Proteſte
                  gegen das Vorgehen \textcolor{blue}{Schlenther}{}\ledrightnote{\textcolor{blue}{Paul Schlenther}}s veranlaßte.
                  Aus den der Oeffentlichkeit mitgetheilten, gewiß nicht für die Oeffentlichkeit
                  bestimmt geweſenen \label{K_L02935-3v}\edtext{\textcolor{green}{Briefen}{}\ledrightnote{{$\rightarrow$}\textcolor{green}{Erklärung [Schleier der Beatrice]}}}{\lemma{\textnormal{\emph{Briefen}}}\Cendnote{\textnormal{siehe Paul Goldmann an Arthur Schnitzler, 19. 9. [1900]}}}\label{K_L02935-3h} wird der Unbefangene das angebliche \textcolor{blue}{Schlenther}{}\ledrightnote{\textcolor{blue}{Paul Schlenther}}ſche Verſchulden nicht ableiten können; aus den \textcolor{green}{Briefen}{}\ledrightnote{{$\rightarrow$}\textcolor{green}{Erklärung [Schleier der Beatrice]}} geht nichts anderes hervor, als
                  daß \textcolor{blue}{Schlenther}{}\ledrightnote{\textcolor{blue}{Paul Schlenther}} ſich das Recht der
                  Erſtaufführung des \textcolor{green}{Stück}{}\ledrightnote{{$\rightarrow$}\textcolor{green}{Der Schleier der Beatrice. Schauspiel in fünf Akten}}es
                     \so{für den Fall} der Annahme ſichern wollte und
                  ſicherte, keineswegs aber, daß das \textcolor{green}{Stück}{}\ledrightnote{{$\rightarrow$}\textcolor{green}{Der Schleier der Beatrice. Schauspiel in fünf Akten}} ſchon angenommen war. Da man auf Seite \textcolor{blue}{Schlenther}{}\ledrightnote{\textcolor{blue}{Paul Schlenther}}s böſe Abſicht gewiß nicht
                  vermuthet, ſo kann der Auslegung, die die \textcolor{blue}{Schlenther}{}\ledrightnote{\textcolor{blue}{Paul Schlenther}}ſchen \textcolor{green}{Briefe}{}\ledrightnote{{$\rightarrow$}\textcolor{green}{Erklärung [Schleier der Beatrice]}} bei Schnitzler fanden, nichts anderes als ein Mißverſtändniß zu
                  Grunde liegen. Die literariſchen \textcolor{blue}{Freunde}{}\ledrightnote{{$\rightarrow$}\textcolor{blue}{Hermann Bahr}{\newline}{$\rightarrow$}\textcolor{blue}{Felix Salten}{\newline}{$\rightarrow$}\textcolor{blue}{Julius Bauer}{\newline}{$\rightarrow$}\textcolor{blue}{Robert Hirschfeld}{\newline}{$\rightarrow$}\textcolor{blue}{Ludwig Speidel}{\newline}{$\rightarrow$}\textcolor{blue}{Jakob Julius David}}
                  Schnitzlers ließen aber ſofort ſchweres \textcolor{green}{Geſchütz}{}\ledrightnote{{$\rightarrow$}\textcolor{green}{Erklärung [Schleier der Beatrice]}} gegen \textcolor{blue}{Schlenther}{}\ledrightnote{\textcolor{blue}{Paul Schlenther}} auffahren und ſtellten ohne weiteres auf ſeiner Seite die böſe
                  Abſicht feſt.}}\pend
           \pstart
           {\pb}Die Parteilichkeit der Darſtellung darf Dich mit
               Rückſicht auf die Beziehungen \textsc{\textcolor{blue}{Schlenther}{}\ledrightnote{\textcolor{blue}{Paul Schlenther}}s} zur »\textcolor{brown}{Voſſiſchen Zeitung}{}\ledrightnote{\textcolor{brown}{Vossische Zeitung}}« nicht verwundern. Ich theile Dir das nur
               mit, damit Du Dir dieſen Herrn \textsc{\textcolor{blue}{Anton Reitler}{}\ledrightnote{\textcolor{blue}{Anton Reitler}}}\label{K_L02935-11v}\edtext{\textsc{ad notam}}{\lemma{\textnormal{\emph{ad notam}}}\Cendnote{\textnormal{lateinisch: zur Kenntnis}}}\label{K_L02935-11h}
               nimmſt.\pend
           \pstart
           Ich vergaß \label{K_L02935-12v}\edtext{geſtern}{\lemma{\textnormal{\emph{geſtern}}}\Cendnote{\textnormal{Paul Goldmann an Arthur Schnitzler, 4. 10. [1900]}}}\label{K_L02935-12h}, Dir Grüße aufzutragen an die ſtrebſamen \label{K_L02935-15v}\edtext{\textcolor{blue}{Fräulein}{}\ledrightnote{{$\rightarrow$}\textcolor{blue}{Olga Schnitzler}{\newline}{$\rightarrow$}\textcolor{blue}{Elisabeth Steinrück}} aus {\pb}der \textcolor{pink}{Rothen-Stern-Gaſſe}{}\ledrightnote{\textcolor{pink}{Rotensterngasse}}}{\lemma{\textnormal{\emph{Fräulein … Rothen-Stern-Gaſſe}}}\Cendnote{\textnormal{siehe Paul Goldmann an Arthur Schnitzler, 19. 9. [1900]}}}\label{K_L02935-15h}.\pend
           \pstart
           Viele Grüße auch an Dich!\pend
           \pstart
           Dein {\\[\baselineskip]}\spacefill\mbox{Paul Goldmnn}\pend
           \leftskip=0em{}\endnumbering\briefempfaengerindex{Schnitzler, Arthur@\textsc{Schnitzler, Arthur}!zzzGoldmann, Paul@\emph{von Paul Goldmann}!1900-10-051@{5. 10. {[}1900{]}}|)be}\mylabel{h}\begin{anhang}\end{anhang}\normalsize

\doendnotes{C}
\bigskip
\vfill

\clearpage

\footnotesize

\lohead{\textsc{register}}

% Definiere theindex-Environment komplett neu ohne reledmac
\makeatletter
\renewenvironment{theindex}{%
  \section*{\indexname}%
  \setlength{\parindent}{0pt}%
  \setlength{\parskip}{0pt plus 0.3pt}%
  \let\item\@idxitem
}{%
  \clearpage
}
\makeatother

\IfFileExists{\jobname-pw.ind}{\input{\jobname-pw.ind}}{}

\end{document}

      