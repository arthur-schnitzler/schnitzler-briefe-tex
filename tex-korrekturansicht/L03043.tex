%% latex-korrekturansicht-vorspann.tex
%% Vorspann für die Korrekturansicht.
%% Lädt die gemeinsame Datei latex-vorspann.tex mit gesetztem Schalter.

\newif\ifkorrekturansicht
\korrekturansichttrue

\input{../tex-inputs/latex-vorspann}


\renewcommand{\erwaehntePersonen}{Personen: Felix Salten}
\renewcommand{\erwaehnteInstitutionen}{Institutionen: Paul Zsolnay Verlag}
\renewcommand{\erwaehnteOrte}{Orte: Berlin, Leipzig, Wien}
\renewcommand{\erwaehnteWerke}{Werke: Simson. Das Schicksal eines Erwählten}
\section[ Felix Salten: Widmungsexemplar Simson für Arthur Schnitzler, 1. 10. 1928]{Felix Salten: Widmungsexemplar Simson für Arthur
               Schnitzler, 1. 10. 1928}
\nopagebreak\mylabel{v}
\rehead{ }\normalsize\beginnumbering\briefempfaengerindex{Schnitzler, Arthur@\textsc{Schnitzler, Arthur}!zzzSalten, Felix@\emph{von Felix Salten}!1928-10-011@{1. 10. 1928}|(be}
\toendnotes[C]{\smallbreak\pagebreak[2]}\Standort{DLA, G:Schnitzler, Arthur (Sammlung Heinrich Schnitzler).}
\physDesc{Widmung am Schmutztitel, 54 Zeichen
\newline{}Handschrift: schwarze Tinte, lateinische Kurrent}
\pstart
           \noindent{}\centering{}{\pb}\textcolor{gray}{\textbf{\textsc{\so{FELIX SALTEN}}}}\pend
           
\pstart
           \noindent{}\centering{}\textcolor{gray}{\textbf{\so{Geſammelte Werke}}}\pend
           
\pstart
           \noindent{}\centering{}\textcolor{gray}{\textbf{\so{in Einzelausgaben}}}\pend
           
\pstart
           \noindent{}Arthur Schnitzler\pend
           
\pstart
           herzlichst {\\[\baselineskip]}\spacefill\mbox{Felix Salten}\pend
           \leftskip=0em{}
\pstart
           \textcolor{pink}{Wien}{}\ledrightnote{\textcolor{pink}{Wien}}, 1. X. 28\pend
           {\bigskip}
\pstart
           \noindent{}\centering{}{\pb}\textcolor{gray}{\textbf{\textsc{\so{FELIX SALTEN}}}}\pend
           
\pstart
           \noindent{}\centering{}\textcolor{gray}{\textbf{\textcolor{green}{\textbf{\so{Simſon}}}{}\ledrightnote{\textcolor{green}{Simson. Das Schicksal eines Erwählten}}}}\pend
           
\pstart
           \noindent{}\centering{}\textcolor{gray}{\textbf{\so{Das Schickſal eines Erwählten}}}\pend
           
\pstart
           \noindent{}\centering{}\textcolor{gray}{\textbf{\textsc{\so{ROMAN}}}}\pend
           {\bigskip}
\pstart
           \noindent{}\centering{}\textcolor{gray}{\textbf{\so{1928}}}\pend
           
\pstart
           \noindent{}\centering{}\textcolor{gray}{\textbf{\textsc{\textcolor{brown}{\so{PAUL ZSOLNAY VERLAG}}{}\ledrightnote{\textcolor{brown}{Paul Zsolnay Verlag}}}}}\pend
           
\pstart
           \noindent{}\centering{}\textcolor{gray}{\textbf{\textcolor{pink}{\so{BERLIN}}{}\ledrightnote{\textcolor{pink}{Berlin}}{ }\so{/}{ }\textcolor{pink}{\so{WIEN}}{}\ledrightnote{\textcolor{pink}{Wien}}{ }\so{/}{ }\textcolor{pink}{\so{LEIPZIG}}{}\ledrightnote{\textcolor{pink}{Leipzig}}}}\pend
           \endnumbering\briefempfaengerindex{Schnitzler, Arthur@\textsc{Schnitzler, Arthur}!zzzSalten, Felix@\emph{von Felix Salten}!1928-10-011@{1. 10. 1928}|)be}\mylabel{h}  \normalsize

\doendnotes{C}
\bigskip
\vfill

\clearpage

\footnotesize

\lohead{\textsc{register}}

% Definiere theindex-Environment komplett neu ohne reledmac
\makeatletter
\renewenvironment{theindex}{%
  \section*{\indexname}%
  \setlength{\parindent}{0pt}%
  \setlength{\parskip}{0pt plus 0.3pt}%
  \let\item\@idxitem
}{%
  \clearpage
}
\makeatother

\IfFileExists{\jobname-pw.ind}{\input{\jobname-pw.ind}}{}

\end{document}

      