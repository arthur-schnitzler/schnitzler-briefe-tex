%% latex-korrekturansicht-vorspann.tex
%% Vorspann für die Korrekturansicht.
%% Lädt die gemeinsame Datei latex-vorspann.tex mit gesetztem Schalter.

\newif\ifkorrekturansicht
\korrekturansichttrue

\input{../tex-inputs/latex-vorspann}


\renewcommand{\erwaehntePersonen}{Personen:  ?? [Partner von Theodore Rottenberg, Ende 1902/Anfang 1903], Hermann Bahr, Rudolf Bernauer, Heinrich Conrad, Gaspard Gourgaud, Pjotr Alexejewitsch Kropotkin, Carl Meinhard, Max Nordau, Max Pannwitz, Emilie Dorothea Popper, Theodore Rottenberg, Olga Schnitzler, Heinrich Schnitzler, Elisabeth Steinrück}
\renewcommand{\erwaehnteInstitutionen}{Institutionen: Robert Lutz, Schiller-Theater}
\renewcommand{\erwaehnteOrte}{Orte: Berlin, Dessauer Straße, Die bösen Buben, Frankfurt am Main, Paris, Schillertheater, Stuttgart, Théâtre Antoine-Simone Berriau, Wien}
\renewcommand{\erwaehnteWerke}{Werke: Au Perroquet Vert, Der einsame Weg. Schauspiel in fünf Akten, Der grüne Kakadu. Groteske in einem Akt, Deutsche Theaterstücke in Frankreich, Liebelei. Schauspiel in drei Akten, Literatur, Memoiren eines Revolutionärs. 2 Bde., Napoleons Gedanken und Erinnerungen. St. Helena 1815–18, Reigen. Zehn Dialoge, Tagebuch, Vossische Zeitung}
\section[ Paul Goldmann an Arthur Schnitzler, 14. 11. {[}1903{]}]{Paul Goldmann an Arthur Schnitzler, 14. 11. {[}1903{]}}
\nopagebreak\mylabel{v}
\rehead{ }\normalsize\beginnumbering\briefempfaengerindex{Schnitzler, Arthur@\textsc{Schnitzler, Arthur}!zzzGoldmann, Paul@\emph{von Paul Goldmann}!1903-11-141@{14. 11. {[}1903{]}}|(be}
\toendnotes[C]{\smallbreak\pagebreak[2]}\Standort{DLA, A:Schnitzler, HS.NZ85.1.3173.}
\physDesc{Brief, 1 Blatt, 4 Seiten
\newline{}Handschrift: blaue Tinte, deutsche Kurrent
\newline{}Schnitzler: 1) mit Bleistift das Jahr »{[}1{]}903.«   2) mit Bleistift auf einem beigelegten Blatt mutmaßlich eine
                                 Antwortskizze, die nur unzuverlässig zu entziffern ist: »{\pb}{ / }\textcolor{gray}{\textbf{Recept}}{ / }\uline{\textcolor{blue}{Nordau}} – hat falſch\textcolor{gray}{e}
                                       berichte d Kriti – hier an tgl. –
                                       (folgende.) – –{ / }dzt: d Kritik nach nicht anzufangen! –){ / }–{ / }\textcolor{blue}{Dor{[}a{]}
                                          Popper} – \textcolor{blue}{Paul}\textcolor{gray}{!}{ / }–{ / }Nicht beide unſtet iſ\textcolor{gray}{t}« 3) mit rotem Buntstift zwei Unterstreichungen}\toendnotes[C]{\smallbreak}
\pstart
           \noindent{}\raggedleft{}{\pb}\textcolor{gray}{\textbf{\textcolor{pink}{DESSAUERSTRASSE 19}{}\ledrightnote{\textcolor{pink}{Dessauer Straße}}}}\pend
           
\pstart
           \textcolor{pink}{Berlin}{}\ledrightnote{\textcolor{pink}{Berlin}}, 14. November.\pend
           
\pstart\center{}Mein lieber Freund,\pend
\pstart
           Verzeih mir, daß ich ſo lange nicht geſchrieben habe. Ich lebe ſeit meiner \label{K_L03388-1v}\edtext{Rückkehr}{\lemma{\textnormal{\emph{Rückkehr}}}\Cendnote{\textnormal{siehe Paul Goldmann an Arthur Schnitzler, 7. 9. 1903}}}\label{K_L03388-1h} in fortwährend wechſelnden Stimmungen, in vielen Sorgen und Widrigkeiten.
               Eine große Müdigkeit hielt mich vom Schreiben zurück. Im Grunde \strikeout{iſ\textcolor{gray}{t}} bleibt doch immer Alles beim
               Alten. Wozu alſo ſchreiben?\pend
           
\pstart
           Deine lieben Nachrichten haben mir ſehr gefehlt. Warum haſt \uline{Du} mir denn nicht geſchrieben? Sind wir denn ſo formell geworden, daß
               Einer auf des Andern Brief wartet, um ihm Nachricht von ſich zu geben? Geſtern{ }{\pb}habe ich endlich durch \textsc{\textcolor{blue}{Liesl}{}\ledrightnote{\textcolor{blue}{Elisabeth Steinrück}}}, die ich bei den \label{K_L03388-2v}\edtext{»\textcolor{pink}{Böſen Buben}{}\ledrightnote{\textcolor{pink}{Die bösen Buben}}«}{\lemma{\textnormal{\emph{»Böſen Buben«}}}\Cendnote{\textnormal{\textcolor{pink}{Die bösen Buben} war der Name eines \textcolor{pink}{Berlin}er Kabaretts, das 1901 von \textcolor{blue}{Rudolf Bernauer} und \textcolor{blue}{Carl Meinhard} gegründet worden war und bis
                     1905 bestand.}}}\label{K_L03388-2h} ſprach, etwas Näheres über Dich
               erfahren. Ich habe zu meiner großen Freude gehört, daß es Dir, Deiner \textcolor{blue}{Frau}{}\ledrightnote{{$\rightarrow$}\textcolor{blue}{Olga Schnitzler}} und dem \textcolor{blue}{Kinde}{}\ledrightnote{{$\rightarrow$}\textcolor{blue}{Heinrich Schnitzler}} gut geht. Und nicht minder freue ich
               mich über die Ausſicht, Dich bald in \label{K_L03388-3v}\edtext{\textcolor{pink}{Berlin}{}\ledrightnote{\textcolor{pink}{Berlin}}}{\lemma{\textnormal{\emph{Berlin}}}\Cendnote{\textnormal{Das nächste Mal war \textcolor{blue}{Schnitzler} zwischen 5. 2. 1904 und 17. 2. 1904 in \textcolor{pink}{Berlin}. \textcolor{blue}{Goldmann} traf er
                  jedenfalls am 7. 2. 1904, 10. 2. 1904 und 16. 2. 1904.}}}\label{K_L03388-3h} zu ſehen. Zu Deinen Erfolgen in der letzten Zeit
                  (\label{K_L03388-4v}\edtext{\textcolor{brown}{Schillertheater}{}\ledrightnote{\textcolor{brown}{Schiller-Theater}}}{\lemma{\textnormal{\emph{Schillertheater}}}\Cendnote{\textnormal{Am 29. 10. 1903 hatte am \textcolor{pink}{Berlin}er \textcolor{pink}{Schillertheater} ein »\textcolor{blue}{Schnitzler}-Abend« mit einer Aufführung von
                     \emph{\textcolor{green}{Liebelei}} und \emph{\textcolor{green}{Literatur}} stattgefunden.}}}\label{K_L03388-4h}, \label{K_L03388-44v}\edtext{\textcolor{pink}{Paris}{}\ledrightnote{\textcolor{pink}{Paris}}}{\lemma{\textnormal{\emph{Paris}}}\Cendnote{\textnormal{Die Inszenierung von \emph{\textcolor{green}{Au Perroquet vert}} (\emph{\textcolor{green}{Der
                     grüne Kakadu}}) wurde im \textcolor{pink}{Théatre Antoine}
                  zwischen 7. 11. 1903 und 6. 12. 1903 zwölf Mal gegeben.}}}\label{K_L03388-44h}, \label{K_L03388-77v}\edtext{\textcolor{blue}{Bahr}{}\ledrightnote{\textcolor{blue}{Hermann Bahr}}s \textcolor{green}{Vorleſung}{}\ledrightnote{{$\rightarrow$}\textcolor{green}{Reigen. Zehn Dialoge}}}{\lemma{\textnormal{\emph{Bahrs Vorleſung}}}\Cendnote{\textnormal{\textcolor{blue}{Bahr} hatte eine
                  öffentliche Vorlesung von \emph{\textcolor{green}{Reigen}} geplant.
                  Letztlich wurde ihm das behördlich untersagt. Vgl. A. S.: \emph{Tagebuch}, 1. 11. 1903; Bahr/Schnitzler, D041436.}}}\label{K_L03388-77h}) beglückwünſche ich Dich herzlichſt, und ich hoffe,
               daß das neue \textcolor{green}{Stück}{}\ledrightnote{{$\rightarrow$}\textcolor{green}{Der einsame Weg. Schauspiel in fünf Akten}} dieſe
               »ſchöne« Reihe mit Glanz fortſetzen wird. Den \label{K_L03388-6v}\edtext{\textcolor{green}{Artikel}{}\ledrightnote{{$\rightarrow$}\textcolor{green}{Deutsche Theaterstücke in Frankreich}} von \textsc{\textcolor{blue}{Nordau}{}\ledrightnote{\textcolor{blue}{Max Nordau}}}}{\lemma{\textnormal{\emph{Artikel von Nordau}}}\Cendnote{\textnormal{\textcolor{blue}{M. N.} [ =\textcolor{blue}{Max Nordau}]: \emph{\textcolor{green}{Deutsche Theaterstücke in Frankreich}}. In: \emph{\textcolor{green}{Vossische Zeitung}}, Jg. XXXX,
                     Nr. YYYY, 11.? 11. 1903, S. XXXX.}}}\label{K_L03388-6h}
               ſchickte ich Dir, weil ich es bemerkenswerth fand, daß dieſer {\pb}\textcolor{blue}{Menſch}{}\ledrightnote{{$\rightarrow$}\textcolor{blue}{Max Nordau}}, der Alles verreißt,
               ſo freundlich über Dich \textcolor{green}{ſprach}{}\ledrightnote{{$\rightarrow$}\textcolor{green}{Deutsche Theaterstücke in Frankreich}}.\pend
           
\pstart
           Für Fräulein \label{K_L03388-7v}\edtext{\textsc{\textcolor{blue}{Dora Popper}{}\ledrightnote{\textcolor{blue}{Emilie Dorothea Popper}}}}{\lemma{\textnormal{\emph{Dora Popper}}}\Cendnote{\textnormal{\textcolor{blue}{Goldmann} bemühte sich um Presseberichte für die Pianistin, vgl. Paul Goldmann an Arthur Schnitzler, 13. 12. [1903]. }}}\label{K_L03388-7h} habe ich
               leider nicht viel thun können. Was mir möglich war, habe ich gethan.\pend
           
\pstart
           \label{K_L03388-8v}\edtext{\textcolor{green}{\textsc{Gourgauds} Geſpräche mit \textsc{Napoleon}}{}\ledrightnote{\textcolor{green}{Napoleons Gedanken und Erinnerungen. St. Helena 1815–18}}}{\lemma{\textnormal{\emph{Gourgauds … Napoleon}}}\Cendnote{\textnormal{\textcolor{blue}{Gaspard Gourgaud}: \emph{\textcolor{green}{Napoleons Gedanken und Erinnerungen. St. Helena
                        1815–18}}. Übersetzt von \textcolor{blue}{Heinrich
                        Conrad}. \textcolor{pink}{Stuttgart}: \emph{\textcolor{brown}{Robert Lutz}}{ }1901.}}}\label{K_L03388-8h}, die ich Dir verdanke (ich werde Dir das \textcolor{green}{Buch}{}\ledrightnote{{$\rightarrow$}\textcolor{green}{Napoleons Gedanken und Erinnerungen. St. Helena 1815–18}} in \textcolor{pink}{Berlin}{}\ledrightnote{\textcolor{pink}{Berlin}} zurückgeben) haben mir viel Genuß bereitet. Ein herrliches Buch \substVorne{}\textsuperscript{i}\substDazwischen{}ſ\substHinten{}ind \label{K_L03388-9v}\edtext{\textcolor{green}{\textsc{Krapotkins} Memoiren}{}\ledrightnote{\textcolor{green}{Memoiren eines Revolutionärs. 2 Bde.}}}{\lemma{\textnormal{\emph{Krapotkins Memoiren}}}\Cendnote{\textnormal{\textcolor{blue}{Peter Kropotkin}: \emph{\textcolor{green}{Memoiren eines Revolutionärs}}. 2 Bände. Übersetzt von
                        \textcolor{blue}{Max Pannwitz}. \textcolor{pink}{Stuttgart}: \emph{\textcolor{brown}{Robert
                        Lutz}}{ }1900. Eventuell las \textcolor{blue}{Schnitzler} die \textcolor{green}{Memoiren}{ }1923, als er im \emph{\textcolor{green}{Tagebuch}} nur notierte, »\textcolor{blue}{Kropotkin}« zu lesen (7. 5. 1923, 14. 6. 1923).}}}\label{K_L03388-9h}, \strikeout{d\textcolor{gray}{e}} (im ſelben \textcolor{brown}{Verlag}{}\ledrightnote{{$\rightarrow$}\textcolor{brown}{Robert Lutz}}e
               erſchienen), \strikeout{die} deren Lektüre ich Dir dringend
               empfehle.\pend
           
\pstart
           Mit \label{K_L03388-10v}\edtext{\textcolor{blue}{\textcolor{pink}{Frankfurt}{}\ledrightnote{\textcolor{pink}{Frankfurt am Main}}}{}\ledrightnote{{$\rightarrow$}\textcolor{blue}{Theodore Rottenberg}}}{\lemma{\textnormal{\emph{Frankfurt}}}\Cendnote{\textnormal{Bezug auf sein Verhältnis mit und zu
                     \textcolor{blue}{Theodore Rottenberg}, mit der \textcolor{blue}{Goldmann} in einer intimen Beziehung
                  stand.}}}\label{K_L03388-10h} bin ich in reger Correſpondenz. Hier und da fährt ein Sturm
               dazwiſchen. Ich weiß nicht, was werden ſoll. Ich mag mich an dieſe \textcolor{blue}{Frau}{}\ledrightnote{{$\rightarrow$}\textcolor{blue}{Theodore Rottenberg}} nicht durch Heirath binden, weil das
               mein \strikeout{Ru\textcolor{gray}{in}} wirthſchaftlicher {\pb}Ruin wäre und weil auch,
               infolge der \textcolor{blue}{Affaire}{}\ledrightnote{{$\rightarrow$}\textcolor{blue}{?? [Partner von Theodore Rottenberg, Ende 1902/Anfang 1903]}} in
               dieſem Winter, viel \label{K_L03388-11v}\edtext{Schmutz}{\lemma{\textnormal{\emph{Schmutz}}}\Cendnote{\textnormal{siehe Paul Goldmann an Arthur Schnitzler, 7. 9. 1903}}}\label{K_L03388-11h} an \textcolor{blue}{ihr}{}\ledrightnote{\textcolor{blue}{Theodore Rottenberg}} haftet; anderſeits kann ich
               nicht einmal den Gedanken ertragen, auf ſie zu verzichten.\pend
           
\pstart
           Grüße Deine \textcolor{blue}{Frau}{}\ledrightnote{{$\rightarrow$}\textcolor{blue}{Olga Schnitzler}}
               vielmals, ſchreib mir bald und ſei ſelbſt herzlichſt gegrüßt von Deinem getreuen {\\[\baselineskip]}\spacefill\mbox{Paul Goldmann.}\pend
           \leftskip=0em{}\endnumbering\briefempfaengerindex{Schnitzler, Arthur@\textsc{Schnitzler, Arthur}!zzzGoldmann, Paul@\emph{von Paul Goldmann}!1903-11-141@{14. 11. {[}1903{]}}|)be}\mylabel{h}
\begin{anhang}
\end{anhang}\normalsize

\doendnotes{C}
\bigskip
\vfill

\clearpage

\footnotesize

\lohead{\textsc{register}}

% Definiere theindex-Environment komplett neu ohne reledmac
\makeatletter
\renewenvironment{theindex}{%
  \section*{\indexname}%
  \setlength{\parindent}{0pt}%
  \setlength{\parskip}{0pt plus 0.3pt}%
  \let\item\@idxitem
}{%
  \clearpage
}
\makeatother

\IfFileExists{\jobname-pw.ind}{\input{\jobname-pw.ind}}{}

\end{document}

      