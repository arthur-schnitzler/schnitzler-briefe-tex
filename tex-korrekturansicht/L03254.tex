%% latex-korrekturansicht-vorspann.tex
%% Vorspann für die Korrekturansicht.
%% Lädt die gemeinsame Datei latex-vorspann.tex mit gesetztem Schalter.

\newif\ifkorrekturansicht
\korrekturansichttrue

\input{../tex-inputs/latex-vorspann}


\renewcommand{\erwaehntePersonen}{Personen: Clementine Goldmann, Fedor Mamroth, Olga Schnitzler}
\renewcommand{\erwaehnteOrte}{Orte: Berlin, Dessauer Straße, Frankfurt am Main, Marienbad, Wien}
\renewcommand{\erwaehnteWerke}{}
\section[ Paul Goldmann an Arthur Schnitzler, 7. 7. 1907]{Paul Goldmann an Arthur Schnitzler, 7. 7. 1907}
\nopagebreak\mylabel{v}
\rehead{ }\normalsize\beginnumbering\briefempfaengerindex{Schnitzler, Arthur@\textsc{Schnitzler, Arthur}!zzzGoldmann, Paul@\emph{von Paul Goldmann}!1907-07-071@{7. 7. 1907}|(be}
\toendnotes[C]{\smallbreak\pagebreak[2]}\Standort{DLA, A:Schnitzler, HS.NZ85.1.3175.}
\physDesc{Brief, 1 Blatt, 4 Seiten
\newline{}Handschrift: blaue Tinte, deutsche Kurrent}\toendnotes[C]{\smallbreak}
\pstart
           \noindent{}\raggedleft{}{\pb}\textcolor{pink}{\textcolor{gray}{\textbf{DESSAUERSTRASSE 19}}}{}\ledrightnote{\textcolor{pink}{Dessauer Straße}}\pend
           
\pstart
           7. 7. 07.\pend
           
\pstart{}Lieber Freund,\pend
\pstart
           Das traurige \label{K-L03254-1v}\edtext{Ereignis}{\lemma{\textnormal{\emph{Ereignis}}}\Cendnote{\textnormal{\textcolor{blue}{Goldmann}s Onkel \textcolor{blue}{Fedor Mamroth} war
                  am 25. 6. 1907 an den Folgen seines Darmkrebs
                  verstorben.}}}\label{K-L03254-1h} hat in ſeinem Gefolge eine ſolche Fülle von Angelegenheiten
               gehabt, die erledigt werden mußten, daß ich erſt heut
               dazu komme, Deinen lieben Brief zu beantworten u. Dir, auch im Namen der Meinigen,
               für Deine ſchönen, teilnehmenden Worte zu danken, die uns Alle tief berührt haben.
               {\\}Mir iſt der Tod zum erſten Mal ganz in die Nähe gekommen, {\pb}u. ich habe ihn erkannt, als das, was er iſt:
               unſinnig u. ſcheußlich. {\\}Das Schwerſte, das Du mir zu überwinden wünſchſt, waren
               nicht die Tage in \textcolor{pink}{Frankfurt}{}\ledrightnote{\textcolor{pink}{Frankfurt am Main}}. Das Schwerſte
               beginnt jetzt. Es iſt die Leere, die das Hinſcheiden eines geliebten \textcolor{blue}{Menſchen}{}\ledrightnote{{$\rightarrow$}\textcolor{blue}{Fedor Mamroth}} im Leben des
               Zurückgebliebenen läßt – es iſt die Sehnſucht, ein teures Geſicht wiederzuſehen, eine
               vertraute Stimme zu hören, die man niemals wiederſehen u. \strikeout{wiederhoh\textcolor{gray}{re}}{ }\strikeout{wird} wiederhören wird, – {\pb}es iſt die Unmöglichkeit, ſich \textcolor{blue}{Jemanden}{}\ledrightnote{{$\rightarrow$}\textcolor{blue}{Fedor Mamroth}} als todt (todt!) vorzuſtellen, der
               noch vor Kurzem von Geiſt u. Leben ſprühte u. an dem man mit ganzer Seele gehangen
                  hat{\dotsseven}{\\}Dir u. Deiner \textcolor{blue}{Frau}{}\ledrightnote{{$\rightarrow$}\textcolor{blue}{Olga Schnitzler}} (der
               ich für ihre Teilnahme vielmals zu danken bitte) wünſche ich frohe Sommertage.
               Schreib’ mir jedenfalls, wo \textcolor{blue}{Ihr}{}\ledrightnote{{$\rightarrow$}\textcolor{blue}{Olga Schnitzler}} ſeid. Freilich iſt die Hoffnung gering, daß ich \textcolor{blue}{Euch}{}\ledrightnote{{$\rightarrow$}\textcolor{blue}{Olga Schnitzler}}{ }\label{K-L03254-2v}\edtext{in dieſem Sommer ſehen}{\lemma{\textnormal{\emph{in dieſem Sommer ſehen}}}\Cendnote{\textnormal{\textcolor{blue}{Schnitzler} und \textcolor{blue}{Goldmann} trafen sich erst am 8. 10. 1907
                  wieder.}}}\label{K-L03254-2h} werde, da ich diesmal meine \textcolor{blue}{Mutter}{}\ledrightnote{\textcolor{blue}{Clementine Goldmann}} nicht allein {\pb}laſſen u. mit ihr
               keine weiten Reiſen machen kann. Wahrſcheinlich gehen wir im Auguſt zunächſt nach \textcolor{pink}{Marienbad}{}\ledrightnote{\textcolor{pink}{Marienbad}}. {\\}\label{K-03254-3v}\edtext{Mißverſtändniſſe}{\lemma{\textnormal{\emph{Mißverſtändniſſe}}}\Cendnote{\textnormal{Bezug unklar; möglicherweise hatte es beim letzten
                  persönlichen Treffen am 2. 6. 1907 eine Auseinandersetzung gegeben}}}\label{K-03254-3h} ſollen uns gewiß
               nicht mehr trennen. Ich bin wenigſtens diesmal von \textcolor{pink}{Wien}{}\ledrightnote{\textcolor{pink}{Wien}} mit dem feſten Vorſatz fortgefahren, Alles \strikeout{zu}, was an mir liegt, zu tun, um mir \substVorne{}\textsuperscript{\textcolor{gray}{me}}\substDazwischen{}eine\substHinten{} alte Freundſchaft zu erhalten, deren Wert ich gewiß nicht geringer bemeſſe,
               wie einſt\substVorne{}\textsuperscript{,}\substDazwischen{}.\substHinten{}\pend
           
\pstart
           Nimm’ alſo nochmals meinen u. der Meinigen herzlichſten Dank u. ſei, ſammt
               Deiner \textcolor{blue}{Frau}{}\ledrightnote{{$\rightarrow$}\textcolor{blue}{Olga Schnitzler}}, vielmals gegrüßt
               von {\\[\baselineskip]}Deinem {\\[\baselineskip]}\spacefill\mbox{Paul Goldmann.}\pend
           \leftskip=0em{}\endnumbering\briefempfaengerindex{Schnitzler, Arthur@\textsc{Schnitzler, Arthur}!zzzGoldmann, Paul@\emph{von Paul Goldmann}!1907-07-071@{7. 7. 1907}|)be}\mylabel{h}  \normalsize

\doendnotes{C}
\bigskip
\vfill

\clearpage

\footnotesize

\lohead{\textsc{register}}

% Definiere theindex-Environment komplett neu ohne reledmac
\makeatletter
\renewenvironment{theindex}{%
  \section*{\indexname}%
  \setlength{\parindent}{0pt}%
  \setlength{\parskip}{0pt plus 0.3pt}%
  \let\item\@idxitem
}{%
  \clearpage
}
\makeatother

\IfFileExists{\jobname-pw.ind}{\input{\jobname-pw.ind}}{}

\end{document}

      