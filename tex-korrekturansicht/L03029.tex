%% latex-korrekturansicht-vorspann.tex
%% Vorspann für die Korrekturansicht.
%% Lädt die gemeinsame Datei latex-vorspann.tex mit gesetztem Schalter.

\newif\ifkorrekturansicht
\korrekturansichttrue

\input{../tex-inputs/latex-vorspann}


\renewcommand{\erwaehntePersonen}{Personen: Gustav Kadelburg, Heinrich Kadelburg, Felix Salten, Adele Sandrock}
\renewcommand{\erwaehnteOrte}{Orte: Café Central, Wien}
\renewcommand{\erwaehnteWerke}{Werke: Adele Sandrock und das Volkstheater, Das Märchen. Schauspiel in drei Aufzügen, Der Fall Sandrock, Hinter den Coulissen [Sandrock im Volkstheater], Neues Wiener Journal, Theater und Kunst. [Fräulein Adele Sandrock dürfte…]}
\section[Arthur Schnitzler an Felix Salten, {[}2. 4. 1894?{]}]{Arthur Schnitzler an Felix Salten, {[}2. 4. 1894?{]}}
\nopagebreak\mylabel{v}
\rehead{ }\normalsize\beginnumbering\briefempfaengerindex{Salten, Felix@\textsc{Salten, Felix}!zzzSchnitzler, Arthur@\emph{von Arthur Schnitzler}!1894-04-022@{{[}2. 4. 1894?{]}}|(be}
\toendnotes[C]{\smallbreak\pagebreak[2]}\Standort{Wienbibliothek im Rathaus, ZPH 1681, 2.1.516.}
\physDesc{Briefkarte, 279 Zeichen (Karte mit Trauerrand)
\newline{}Handschrift: schwarze Tinte, deutsche Kurrent
\newline{}Ordnung: mit Bleistift von unbekannter Hand Nummerierung der Blätter des Konvoluts:
                                    »32« }\toendnotes[C]{\smallbreak}
\pstart
           \noindent{}{\pb}Lieber Freund; Frl. \textcolor{blue}{S.}{}\ledrightnote{\textcolor{blue}{Adele Sandrock}} telephonirt mir eben, daſs ſie zu nervös iſt, Abends
               u. ſ. w. – Eine mit der \label{K_L03029-1v}\edtext{\textcolor{blue}{Kadelburg}{}\ledrightnote{\textcolor{blue}{Gustav Kadelburg}}affaire}{\lemma{\textnormal{\emph{Kadelburgaffaire}}}\Cendnote{\textnormal{Am 30. 3. 1894 war im \emph{\textcolor{green}{Neuen Wiener Journal}} in der Rubrik »Theater
                  und Kunst« die \textcolor{green}{Meldung}
                  erschienen (Nr. 154, S. 6), dass \textcolor{blue}{Adele Sandrock} von Auftritten ferngehalten werde und durch den Regisseur
                     \textcolor{blue}{Heinrich Kadelburg} gemobbt worden sei.
                  An den Folgetagen waren mehrere Dementi erschienen (\emph{\textcolor{green}{Hinter den Coulissen}}, 31. 3. 1894, Nr. 155, S. 5; \emph{\textcolor{green}{Adele Sandrock und das Volkstheater}}, 1. 4. 1894, Nr. 156, S. 5). Am 4. 4. 1894 folgte eines von \textcolor{blue}{Schnitzler}, worin er meinte, dass er \emph{\textcolor{green}{Das Märchen}} nicht speziell für \textcolor{blue}{Sandrock} geschrieben habe (vgl. \emph{\textcolor{green}{Der Fall Sandrock}}, Nr. 158, S. 5). Das
                  vorliegende Korrespondenzstück ist undatiert, dürfte aber in den Zeitraum des
                  Skandals fallen – und da an diesen Tagen nur für den 2. 4. 1894 ein
                  Treffen mit \textcolor{blue}{Salten} festgehalten ist, das
                  sich noch dazu im \textcolor{pink}{Café Central} zugetragen
                  haben könnte, lässt sich eine – wenngleich unsichere – Datierung erreichen.}}}\label{K_L03029-1h}
                  zuſa{\geminationm}enhängende Klagegeſchichte. – Jeden{\pb}falls treffen wir, Sie, u ich uns
                  Abends um 10 im \textsc{\textcolor{pink}{Central}{}\ledrightnote{\textcolor{pink}{Café Central}}}. –\pend
           
\pstart
           – Ja richtig: Sie möchten nicht böſe ſein. –\pend
           
\pstart
           Herzlichen Gruß {\\[\baselineskip]}Ihr \spacefill\mbox{ArthurSch.}\pend
           \leftskip=0em{}\endnumbering\briefempfaengerindex{Salten, Felix@\textsc{Salten, Felix}!zzzSchnitzler, Arthur@\emph{von Arthur Schnitzler}!1894-04-022@{{[}2. 4. 1894?{]}}|)be}\mylabel{h}  \normalsize

\doendnotes{C}
\bigskip
\vfill

\clearpage

\footnotesize

\lohead{\textsc{register}}

% Definiere theindex-Environment komplett neu ohne reledmac
\makeatletter
\renewenvironment{theindex}{%
  \section*{\indexname}%
  \setlength{\parindent}{0pt}%
  \setlength{\parskip}{0pt plus 0.3pt}%
  \let\item\@idxitem
}{%
  \clearpage
}
\makeatother

\IfFileExists{\jobname-pw.ind}{\input{\jobname-pw.ind}}{}

\end{document}

      