%% latex-korrekturansicht-vorspann.tex
%% Vorspann für die Korrekturansicht.
%% Lädt die gemeinsame Datei latex-vorspann.tex mit gesetztem Schalter.

\newif\ifkorrekturansicht
\korrekturansichttrue

\input{../tex-inputs/latex-vorspann}


               \section[Hugo von Hofmannsthal an Arthur Schnitzler, {[}25. –29. 1. 1892?{]}]{ Hugo von Hofmannsthal an Arthur Schnitzler, {[}25. –29. 1.
                    1892?{]}}\nopagebreak\mylabel{v}\rehead{ }\normalsize\beginnumbering\briefempfaengerindex{Schnitzler, Arthur@\textsc{Schnitzler, Arthur}!zzzHofmannsthal, Hugo von@\emph{von Hugo von Hofmannsthal}!1892-01-251@{{[}25. –29. 1. 1892?{]}}|(be} \toendnotes[C]{\smallbreak\pagebreak[2]} \Standort{CUL, Schnitzler, B 43.}
\physDesc{Briefkarte
\newline{}Handschrift: Bleistift, deutsche Kurrent
\newline{}Schnitzler: mit Bleistift datiert: »Anfg 92.« \newline{}Ordnung: mit Bleistift von unbekannter Hand nummeriert:
                                                »13« }\buchAbdrucke{\weitereDrucke{Hugo von Hofmannsthal, Arthur Schnitzler: \emph{Briefwechsel}. Hg. Therese Nickl und Heinrich Schnitzler. Frankfurt am Main: \emph{S. Fischer} 1964, S. 14.} }\toendnotes[C]{\smallbreak}\pstart
           \noindent{}{\pb}\textcolor{gray}{\textbf{\label{T_L00065-1v}\edtext{AvH}{\lemma{\textnormal{\emph{AvH}}}\Cendnote{\textnormal{Monogramm der Mutter \textcolor{blue}{Anna von Hofmannsthal} mit Krone in
                                Golddruck}}}\label{T_L00065-1h}}}\pend
           \pstart{}Lieber Freund.\pend\pstart
           Bitte ſchreiben Sie ſich auch da hinein. Näheres \label{K_L00065_1v}\edtext{Sonntag}{\lemma{\textnormal{\emph{Sonntag}}}\Cendnote{\textnormal{Das
                        erste Treffen nach dem Erscheinen von \emph{\textcolor{green}{Der
                            Sohn}} lässt sich am 31. 1. 1892 belegen, wodurch sich dieses Korrespondenzstück
                        zeitlich vorne und hinten eingrenzen lässt. Eine kleine Einschränkung gibt
                        auch der Umstand, dass am Vortag nicht mehr von »Sonntag« sondern von
                        »morgen« die Rede gewesen sein dürfte, was den 30.
                        ausschließt.}}}\label{K_L00065_1h}. Die Idee und die \label{K_L00065_2v}\edtext{3 letzten Zeilen}{\lemma{\textnormal{\emph{3 letzten Zeilen}}}\Cendnote{\textnormal{Das stützt die Datierung \textcolor{blue}{Schnitzler}s, da \emph{\textcolor{green}{Der Sohn}} im
                        Januarheft der \emph{\textcolor{green}{Freien Bühne}} erschien. \textcolor{blue}{Schnitzler} vermerkt dies am 24. 1. 1892 im \emph{\textcolor{green}{Tagebuch}}, weswegen anzunehmen ist, dass
                        auch \textcolor{blue}{Hofmannsthal} in etwa zu dieser Zeit
                        die Möglichkeit hatte, die Geschichte zu lesen.}}}\label{K_L00065_2h} vom »\textcolor{green}{Sohn}{}\ledrightnote{\textcolor{green}{Der Sohn. Aus den Papieren eines Arztes}}« ſind ganz 1892; das übrige etwas älter,
                    aber gar nicht {\pb}bös. Ich
                    hoffe, daſs Sie gut aufgelegt ſind\pend
           \pstart
           Herzlichſt{\\[\baselineskip]}\spacefill\mbox{Loris}\pend
           \leftskip=0em{}\endnumbering\briefempfaengerindex{Schnitzler, Arthur@\textsc{Schnitzler, Arthur}!zzzHofmannsthal, Hugo von@\emph{von Hugo von Hofmannsthal}!1892-01-251@{{[}25. –29. 1. 1892?{]}}|)be}\mylabel{h}  \normalsize

\doendnotes{C}
\bigskip
\vfill

\clearpage

\footnotesize

\lohead{\textsc{register}}

% Definiere theindex-Environment komplett neu ohne reledmac
\makeatletter
\renewenvironment{theindex}{%
  \section*{\indexname}%
  \setlength{\parindent}{0pt}%
  \setlength{\parskip}{0pt plus 0.3pt}%
  \let\item\@idxitem
}{%
  \clearpage
}
\makeatother

\IfFileExists{\jobname-pw.ind}{\input{\jobname-pw.ind}}{}

\end{document}

      