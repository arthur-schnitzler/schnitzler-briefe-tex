%% latex-korrekturansicht-vorspann.tex
%% Vorspann für die Korrekturansicht.
%% Lädt die gemeinsame Datei latex-vorspann.tex mit gesetztem Schalter.

\newif\ifkorrekturansicht
\korrekturansichttrue

\input{../tex-inputs/latex-vorspann}


\renewcommand{\erwaehntePersonen}{Personen: Julius von Gans-Ludassy, Hugo Ganz, Theodor Herzl, Heinrich Kanner, Felix Salten}
\renewcommand{\erwaehnteInstitutionen}{Institutionen: Bezirksgericht Wien Josefstadt, Concordia, Die Zeit}
\renewcommand{\erwaehnteOrte}{Orte: Alser Straße, Pötzleinsdorf, Starkfriedgassse, Wien}
\renewcommand{\erwaehnteWerke}{}
\section[ Felix Salten an Arthur Schnitzler, {[}18.? 10. 1906{]}]{Felix Salten an Arthur Schnitzler, {[}18.? 10. 1906{]}}
\nopagebreak\mylabel{v}
\rehead{ }\normalsize\beginnumbering\briefempfaengerindex{Schnitzler, Arthur@\textsc{Schnitzler, Arthur}!zzzSalten, Felix@\emph{von Felix Salten}!1906-10-182@{{[}18.? 10. 1906{]}}|(be}
\toendnotes[C]{\smallbreak\pagebreak[2]}\Standort{CUL, Schnitzler, B 89, B 1.}
\physDesc{Brief, 1 Blatt, 3 Seiten, 3912 Zeichen
\newline{}Handschrift: schwarze Tinte, lateinische Kurrent
\newline{}Schnitzler: mit Bleistift datiert: »October 906« 
\newline{}Ordnung: mit Bleistift von unbekannter Hand nummeriert: »225« }\toendnotes[C]{\smallbreak}
\pstart
           \raggedleft{}{\pb}\label{K_L03434-1v}\edtext{Donnerstag}{\lemma{\textnormal{\emph{Donnerstag}}}\Cendnote{\textnormal{\textcolor{blue}{Schnitzler} datierte das ansonsten nur
                        durch \textcolor{blue}{Salten}s Angabe des Wochentags
                        zeitlich näher bestimmte Korrespondenzstück auf »October 906«. Eine weitere Einschränkung ist dadurch möglich, dass am Dienstag, dem 23. 10. 1906, mehrere Zeitungen
                        den Prozessbeginn für den 24. 11. 1906
                        verkündeten. Das deutet auf eine gerichtliche Festsetzung dieses Termins am
                           22. 10. 1906 hin – eben jenem Montag, von
                        dem in diesem Korrespondenzstück die Rede ist. Dass das folgende Schreiben
                           (Felix Salten an Arthur Schnitzler, [20.? 10. 1906]) von einer Verschiebung
                        des Prozessbeginns zeugt, fügt sich problemlos in diesen zeitlichen Ablauf
                        ein.}}}\label{K_L03434-1h}.\pend
           
\pstart{}Lieber,\pend
\pstart
           Ein Kapitel \textcolor{blue}{Ludassy}{}\ledrightnote{\textcolor{blue}{Julius von Gans-Ludassy}}. Es ist langweilig und
               lästig, aber ich muß ein Stückchen Vorgeschichte erwähnen. Wie ich \label{K_L03434-2v}\edtext{mit ihm
               auseinanderkam}{\lemma{\textnormal{\emph{mit ihm
               auseinanderkam}}}\Cendnote{\textnormal{vgl. Felix Salten an Arthur Schnitzler, 9. 3. 1906}}}\label{K_L03434-2h}, wissen Sie ja. Es war der \label{K_L03434-3v}\edtext{\textcolor{blue}{Hugo Ganz}{}\ledrightnote{\textcolor{blue}{Hugo Ganz}}-Prozess}{\lemma{\textnormal{\emph{Hugo Ganz-Prozess}}}\Cendnote{\textnormal{\textcolor{blue}{Hugo Ganz} hatte 1903{ }\emph{\textcolor{brown}{Die Zeit}} wegen schlechter Behandlung auf
                  Abfertigung geklagt. Die Verhandlungen fanden im Januar 1904 statt. Die zweite Instanz bestätigte Ende März 1904 das Urteil, demnach \textcolor{blue}{Ganz} 18.000 Kronen zustanden (vgl. A. S.: \emph{Tagebuch}, 19. 1. 1904). Der in Folge
                  genannte \textcolor{blue}{Heinrich Kanner} war der
                  Herausgeber der \emph{\textcolor{brown}{Zeit}}, der durch seine
                  schlechten Umgangsformen \textcolor{blue}{Ganz}’ Kündigung
                  bewirkt haben soll.}}}\label{K_L03434-3h} gewesen. Die »\textcolor{brown}{Concordia}{}\ledrightnote{\textcolor{brown}{Concordia}}« ereiferte sich gegen \textcolor{blue}{Kanner}{}\ledrightnote{\textcolor{blue}{Heinrich Kanner}}, den ich verteidigte. In der Versammlung saß \textcolor{blue}{Ludaßy}{}\ledrightnote{\textcolor{blue}{Julius von Gans-Ludassy}} mit mir an einem Tisch. Ich sagte in meiner Rede, \textcolor{blue}{Ludaßy}{}\ledrightnote{\textcolor{blue}{Julius von Gans-Ludassy}} sei als Chef auch heftig gewesen, ohne
               dass die \textcolor{brown}{Concordia}{}\ledrightnote{\textcolor{brown}{Concordia}} u. s. w. Als ich geendigt
               hatte, zischelte mir \textcolor{blue}{Ludaßy}{}\ledrightnote{\textcolor{blue}{Julius von Gans-Ludassy}}, der ganz blaß
               war, zu: »Das war geschmacklos und undankbar{\dotstwo}« Ich: »Wofür
               bin ich Ihnen denn Dank schuldig?« Er: »Ich weiß auch Sachen von Ihnen{\dots}« Worauf ich, der ich einerseits fand, es sei vielleicht
               zu viel von mir gewesen, wenn ich bei Gelegenheit \textcolor{blue}{Kanner}{}\ledrightnote{\textcolor{blue}{Heinrich Kanner}}s auf \textcolor{blue}{Ludaßy}{}\ledrightnote{\textcolor{blue}{Julius von Gans-Ludassy}}’s verjährte
               Brotherren-Grobheit anspielte, andrerseits über die »Sachen«, die er wißen wollte,
               aufgebracht war, ihm sagte: (auch aus versammlungstechnischen Gründen): »Ich werde
               jetzt aussprechen, dass diese Reminiszenz meine Spitze gegen Sie enthielt, und dann
               werden Sie sofort erklären, was Sie von mir wißen.« Er antwortete: »Abgemacht.« Ich
               tat nun meinerseits, wie versprochen. Wie ich ihn aber aufforderte, ja bevor ich ihn
               noch auffordern konnte, nunmehr sein Wort einzulösen, reichte er mir die Hand, mit
               den Worten: »Sei’n wir wieder gut{\dotstwo}« Ich schlug seine Hand
               aus, und begehrte, die »Sachen« zu wißen. Er blieb dabei: »Laßen wir’s gut sein.« Da
               sagte ich ihm, in Erinnerung an manche ähnliche Büberei: »Das ist echt Ihre Art. Wenn
               Sie jetzt nicht sofort mit {\pb}der
               Sprache herausrücken, sind Sie ein feiger Lump{\dots}« oder
                  Kerl{\dotstwo} oder Schuft, oder so was ähnliches. \textcolor{blue}{Ludaßy}{}\ledrightnote{\textcolor{blue}{Julius von Gans-Ludassy}} stand vom Tisch auf und seither grüßen
               wir uns nicht mehr.\pend
           
\pstart
           Sie erinnern sich dieser abscheulichen Geschichte gewiß; erinnern sich ihrer um so
               eher, als ich sie gleich damals, und hernach noch oft bei Ihnen zum Besten gab, wenn
               wir über Freund \textcolor{blue}{Ludaßy}{}\ledrightnote{\textcolor{blue}{Julius von Gans-Ludassy}} und sein Verhältnis zu
               mir, zu Ihnen und zu uns Allen sprachen.\pend
           
\pstart
           Diese Geschichte, als die Entstehungsursache seiner Feindschaft gegen mich, habe ich
               vor dem Ehrenrat zu Protokoll gegeben. Herr \textcolor{blue}{Ludaßy}{}\ledrightnote{\textcolor{blue}{Julius von Gans-Ludassy}}{ }\uline{leugnet} diesen Vorfall, bezichtigt mich der
               Unwahrheit, und erhebt Ehrenbeleidigungsklage gegen mich, weil ich ihn durch
               Erzählung dieser von mir erlogenen Episode vor dem Ehrenrat dem Gespött preisgegeben
               habe. Die Verhandlung findet Montag, \textcolor{brown}{Bezirksgericht Josefstadt}{}\ledrightnote{\textcolor{brown}{Bezirksgericht Wien Josefstadt}}, \textcolor{pink}{Alserstraße}{}\ledrightnote{\textcolor{pink}{Alser Straße}} statt. Herr \textcolor{blue}{Ludaßy}{}\ledrightnote{\textcolor{blue}{Julius von Gans-Ludassy}} will
               damit der Schwurgerichtsverhandlung gegen sich in listiger Weise präludiren.\pend
           
\pstart
           Es kommt nun für mich darauf an, zu beweisen, dass ich diesen Vorfall gleich damals,
               nach der \textcolor{blue}{Kanner}{}\ledrightnote{\textcolor{blue}{Heinrich Kanner}}-Versammlung, dritten Personen
               erzählt habe. Ich weiß nun, dass ich Ihnen gleich damals ausführlich davon Mitteilung
               machte, um Sie in Kenntnis zu setzen, dass ich mit \textcolor{blue}{Ludaßy}{}\ledrightnote{\textcolor{blue}{Julius von Gans-Ludassy}} verfeindet sei. Weiß, dass ich Ihnen im \label{K_L03434-4v}\edtext{Sommer 190\substVorne{}\textsuperscript{5}\substDazwischen{}4\substHinten{}}{\lemma{\textnormal{\emph{Sommer 1904}}}\Cendnote{\textnormal{siehe A. S.: \emph{Tagebuch}, 6. 7. 1904}}}\label{K_L03434-4h} in \textcolor{pink}{Pötzleinsdorf}{}\ledrightnote{\textcolor{pink}{Pötzleinsdorf}}, in der \textcolor{pink}{Starkfriedgaße}{}\ledrightnote{\textcolor{pink}{Starkfriedgassse}}, wo ich damals wohnte, die Sache \uline{wieder} erzählte, worauf Sie mir \textcolor{blue}{Ludaßy}{}\ledrightnote{\textcolor{blue}{Julius von Gans-Ludassy}}’s Schmutzwort über \textcolor{blue}{Herzl}{}\ledrightnote{\textcolor{blue}{Theodor Herzl}}, das er kurz nach \textcolor{blue}{Herzl}{}\ledrightnote{\textcolor{blue}{Theodor Herzl}}’s Tode
               geäußert hatte, gleichsam zur Illustrirung mitteilten.\pend
           
\pstart
           Nun bitte ich Sie, mir das zu bezeugen. Sie sind der Einzige, dem ich so oft von der
               Sache sprach. Es ist \uline{wichtig}, dass mir der Wahrheit
               gemäß bezeugt wird, ich habe diesen Vorfall \uline{lange}{ }\uuline{vor} dem Ehrenratsverfahren, \uline{oftmals} und \uline{immer} in \uline{der{\pb}selben Form} erzählt, und immer als die letzte Ursache der Entzweiung
               bezeichnet.\pend
           
\pstart
           Die Äußerung über \textcolor{blue}{Herzl}{}\ledrightnote{\textcolor{blue}{Theodor Herzl}} wird in der Montag-Verhandlung nicht zur Sprache kommmen. Ich hoffe,
               Sie zögern nicht, mir durch die einfache Constatirung dieser Tatsache in meinem \introOben{}mir\introOben{} aufgedrungenen Abwehrkampf gegen eine der bissigsten
               Canaillen, die es gibt, beizustehen; in einem Kampf, in dem ich ohnehin zu sehr
               allein stehe. Bitte geben Sie mir pneumatisch Nachricht, ob Sie sich dieser Dinge,
               namentlich des Sommers 1904, ec. erinnern, und ob ich Sie
               als Zeugen nennen darf. Das Wesentliche ist, ob Sie – wie ich annehme – Sich
               besinnen, diese Geschichte lange \uline{vor} dem Dezember vo\textcolor{gray}{r.} Jahres und oft vorher von
               mir gehört zu haben.\pend
           
\pstart
           herzlichst Ihr {\\[\baselineskip]}\spacefill\mbox{Salten}\pend
           \leftskip=0em{}\endnumbering\briefempfaengerindex{Schnitzler, Arthur@\textsc{Schnitzler, Arthur}!zzzSalten, Felix@\emph{von Felix Salten}!1906-10-182@{{[}18.? 10. 1906{]}}|)be}\mylabel{h}  \normalsize

\doendnotes{C}
\bigskip
\vfill

\clearpage

\footnotesize

\lohead{\textsc{register}}

% Definiere theindex-Environment komplett neu ohne reledmac
\makeatletter
\renewenvironment{theindex}{%
  \section*{\indexname}%
  \setlength{\parindent}{0pt}%
  \setlength{\parskip}{0pt plus 0.3pt}%
  \let\item\@idxitem
}{%
  \clearpage
}
\makeatother

\IfFileExists{\jobname-pw.ind}{\input{\jobname-pw.ind}}{}

\end{document}

      