%% latex-korrekturansicht-vorspann.tex
%% Vorspann für die Korrekturansicht.
%% Lädt die gemeinsame Datei latex-vorspann.tex mit gesetztem Schalter.

\newif\ifkorrekturansicht
\korrekturansichttrue

\input{../tex-inputs/latex-vorspann}


               \section[Hermann Bahr an Arthur Schnitzler, 17. 9. 1893]{ Hermann Bahr an Arthur Schnitzler, 17. 9. 1893}\nopagebreak\mylabel{v}\rehead{ }\normalsize\beginnumbering\briefempfaengerindex{Schnitzler, Arthur@\textsc{Schnitzler, Arthur}!zzzBahr, Hermann@\emph{von Hermann Bahr}!1893-09-171@{17. 9. 1893}|(be} \toendnotes[C]{\smallbreak\pagebreak[2]} \Standort{CUL, Schnitzler, B 5b.}
\physDesc{Brief, 1 Blatt, 2 Seiten
\newline{}Handschrift: schwarze Tinte, deutsche Kurrent\newline{}Ordnung: 1) mit rotem Buntstift von unbekannter Hand nummeriert: »13« 2) mit Bleistift von unbekannter Hand nummeriert: »13«}\buchAbdrucke{\weitereDrucke{Hermann Bahr, Arthur Schnitzler: \emph{Briefwechsel, Aufzeichnungen, Dokumente (1891–1931)}. Hg. Kurt Ifkovits und Martin Anton Müller. Göttingen: \emph{Wallstein} 2018, S. 37.} }\pstart
           \raggedleft{}{\pb}\textcolor{pink}{\textsc{Salesianerg 12}}{}\ledrightnote{\textcolor{pink}{Salesianergasse}}{\\}17 9 93\pend
           \pstart\center{}Lieber Freund!\pend\pstart
           Hätten Sie Dienſtag oder Mittwoch Abend von 8–10 etwa für mich Zeit? Ich muß Sie
               ſprechen, aus tauſend privaten Gründen u. einem journaliſtiſchen, der es mich
               wünſchen ließe, daß Sie auch \textcolor{blue}{Beer-Hofmann}{}\ledrightnote{\textcolor{blue}{Richard Beer-Hofmann}} (deſſen
               Adreſſe ich leider nicht weiß) {\pb}mitzukommen bitten
               würden. Ich habe ſchon wieder ſo entſetzlich viel zu thun, daß ich durchaus die Zeit
               nicht finde, einmal nachmittag zu Ihnen zu gehen.\pend
           \pstart
           Herzlichſt{\\[\baselineskip]}Ihr{\\[\baselineskip]}\spacefill\mbox{HermannBahr}\pend
           \leftskip=0em{}\pstart
           \noindent{}Ein Rendezvous im \textcolor{pink}{Grienſteidl}{}\ledrightnote{\textcolor{pink}{Café Griensteidl}}, etwa um 8, wäre
                  das bequemſte.\pend
           \endnumbering\briefempfaengerindex{Schnitzler, Arthur@\textsc{Schnitzler, Arthur}!zzzBahr, Hermann@\emph{von Hermann Bahr}!1893-09-171@{17. 9. 1893}|)be}\mylabel{h}  \normalsize

\doendnotes{C}
\bigskip
\vfill

\clearpage

\footnotesize

\lohead{\textsc{register}}

% Definiere theindex-Environment komplett neu ohne reledmac
\makeatletter
\renewenvironment{theindex}{%
  \section*{\indexname}%
  \setlength{\parindent}{0pt}%
  \setlength{\parskip}{0pt plus 0.3pt}%
  \let\item\@idxitem
}{%
  \clearpage
}
\makeatother

\IfFileExists{\jobname-pw.ind}{\input{\jobname-pw.ind}}{}

\end{document}

      