%% latex-korrekturansicht-vorspann.tex
%% Vorspann für die Korrekturansicht.
%% Lädt die gemeinsame Datei latex-vorspann.tex mit gesetztem Schalter.

\newif\ifkorrekturansicht
\korrekturansichttrue

\input{../tex-inputs/latex-vorspann}


               \section[Laura Marholm an Arthur Schnitzler, 16. 4. 1895]{ Laura Marholm an Arthur Schnitzler, 16. 4. 1895}\nopagebreak\mylabel{v}\rehead{ }\normalsize\beginnumbering\briefempfaengerindex{Schnitzler, Arthur@\textsc{Schnitzler, Arthur}!zzzMarholm, Laura@\emph{von Laura Marholm}!1895-04-161@{16. 4. 1895}|(be} \toendnotes[C]{\smallbreak\pagebreak[2]} \Standort{CUL, Schnitzler, B 69.}
\physDesc{Brief, 1 Blatt, 2 Seiten
\newline{}Handschrift: schwarze Tinte, lateinische Kurrent
\newline{}Schnitzler: 1) mit Bleistift beschriftet: »\textsc{Marholm}« 2) mit rotem Buntstift eine Unterstreichung}\toendnotes[C]{\smallbreak}\pstart
           \noindent{}\raggedleft{}{\pb}\textcolor{pink}{Schliersee}{}\ledrightnote{\textcolor{pink}{Schliersee}}, \textcolor{pink}{Oberbaiern}{}\ledrightnote{\textcolor{pink}{Oberbayern}}\pend
           \pstart
           \raggedleft{}16. April 95\pend
           \pstart\center{}Sehr geehrter Herr Professor\pend\pstart
           Ich erlaube mir Ihnen beifolgend mein »\textcolor{green}{Buch der
                        Frauen}{}\ledrightnote{\textcolor{green}{Das Buch der Frauen}}« zu übersenden, das in den \textcolor{pink}{Wien}{}\ledrightnote{\textcolor{pink}{Wien}}er Blättern viel besprochen worden ist und Ihnen daher
                    vielleicht nicht als ganz unbekannter Gast in die Hand kommt. Ich hätte \introOben{}dazu\introOben{} – obgleich ich weiß, das Sie das, was lebendig und
                    Lebensbeitrag in der Litteratur ist, mit aufmerksamen Blick verfolgen – doch
                    nicht den Muth \strikeout{dazu} gehabt, wenn mir nicht ein
                    gelehrter Herr in \textcolor{pink}{Straßburg}{}\ledrightnote{\textcolor{pink}{Straßburg}}, Dr. \textcolor{blue}{Kraft}{}\ledrightnote{\textcolor{blue}{Heinrich Kraft}} von der \textcolor{pink}{Frauenklinik}{}\ledrightnote{\textcolor{pink}{Universitäts-Frauenklinik}}, neulich geschrieben hätte, »\textcolor{green}{Das Buch der Frauen}{}\ledrightnote{\textcolor{green}{Das Buch der Frauen}}« sei ihm durch die
                    Übereinstimmung der intuitiv erfaßten Ausgangspunkte mit den anthropologischen,
                    psychologischen und physiologischen Ausgangspunkten in \textcolor{blue}{Havelock Ellis}{}\ledrightnote{\textcolor{blue}{Havelock Ellis}} »\textcolor{green}{Mann {\kaufmannsund} Weib}{}\ledrightnote{\textcolor{green}{Mann und Weib}}« merkwürdig und verheißner
                    für die Sache {\pb}der Frauenkenntniß
                    selber und das Weitere, was ich zu sagen hätte. Und ich habe ja allerdings noch
                    kaum mit dem Heraussagen angefangen.\pend
           \pstart
           Ich bin ganz u. gar nicht eine gelehrte Frau und halte auch nichts davon für die
                    wirkliche Entwicklung des Weibes. Ich habe das Leben mitgelebt und einen \textcolor{blue}{Mann}{}\ledrightnote{→\textcolor{blue}{Ola Hansson}} gefunden, der alle meine Möglichkeiten
                    als Weib frei macht und zur Entwicklung treibt. Das ist alles und doch etwas
                    Seltenes. Und darum wage ich es, Ihnen dieses \textcolor{green}{Buch}{}\ledrightnote{→\textcolor{green}{Das Buch der Frauen}} zu übersenden mit der Bitte, es gelegentlich anzublättern. Das
                    ist immer alles, worauf es ankommt. Spricht ein Buch nicht zu einem beim ersten
                    Hineinblicken durch die Blutmale in seinem Satzbau, durch die Seelenschwingung
                    in seinem Stil – dann ist nichts rechtes dran.\pend
           \pstart
           Aber spricht es zu Ihnen, verehrter Herr Doktor, dann würden Sie mich durch ein
                    Zeichen der Mittheilung nicht nur sehr froh machen, sondern auch zu weiterer
                    Selbstmittheilung in anderen Büchern ermuthigen.\pend
           \pstart
           Mit ausgezeichneter Hochachtung{\\[\baselineskip]}\spacefill\mbox{Laura Hansson-Marholm}\pend
           \leftskip=0em{}\endnumbering\briefempfaengerindex{Schnitzler, Arthur@\textsc{Schnitzler, Arthur}!zzzMarholm, Laura@\emph{von Laura Marholm}!1895-04-161@{16. 4. 1895}|)be}\mylabel{h}  \normalsize

\doendnotes{C}
\bigskip
\vfill

\clearpage

\footnotesize

\lohead{\textsc{register}}

% Definiere theindex-Environment komplett neu ohne reledmac
\makeatletter
\renewenvironment{theindex}{%
  \section*{\indexname}%
  \setlength{\parindent}{0pt}%
  \setlength{\parskip}{0pt plus 0.3pt}%
  \let\item\@idxitem
}{%
  \clearpage
}
\makeatother

\IfFileExists{\jobname-pw.ind}{\input{\jobname-pw.ind}}{}

\end{document}

      