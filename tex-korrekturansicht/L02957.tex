%% latex-korrekturansicht-vorspann.tex
%% Vorspann für die Korrekturansicht.
%% Lädt die gemeinsame Datei latex-vorspann.tex mit gesetztem Schalter.

\newif\ifkorrekturansicht
\korrekturansichttrue

\input{../tex-inputs/latex-vorspann}


\renewcommand{\erwaehntePersonen}{Personen: Richard Beer-Hofmann, Felix Salten, Carl von Torresani-Lanzenfeld}
\renewcommand{\erwaehnteInstitutionen}{Institutionen: Internationale Ausstellung für Musik und Theaterwesen}
\renewcommand{\erwaehnteOrte}{Orte: Berggasse, I., Innere Stadt, IX., Alsergrund, Prater, Stephansplatz, Wien}
\renewcommand{\erwaehnteWerke}{}
\section[Arthur Schnitzler an Felix Salten, 8. 10. 1892]{Arthur Schnitzler an Felix Salten, 8. 10. 1892}
\nopagebreak\mylabel{v}
\rehead{ }\normalsize\beginnumbering\briefempfaengerindex{Salten, Felix@\textsc{Salten, Felix}!zzzSchnitzler, Arthur@\emph{von Arthur Schnitzler}!1892-10-081@{8. 10. 1892}|(be}
\toendnotes[C]{\smallbreak\pagebreak[2]}\Standort{Wienbibliothek im Rathaus, ZPH 1681, 2.1.516.}
\physDesc{Kartenbrief, 200 Zeichen
\newline{}Handschrift: schwarze Tinte, deutsche Kurrent
\newline{}Versand: 1) Stempel: »\nobreak{}\oindex{I., Innere Stadt@\textbf{I., Innere Stadt}, \emph{A.ADM3}|pwk}Wien 1/1 1, 8. 10. \textcolor{gray}{9}2, 2–3N\nobreak{}«.   2) Stempel: »\nobreak{}\oindex{IX., Alsergrund@\textbf{IX., Alsergrund}, \emph{A.ADM3}|pwk}Wien 9/1 66, 8. 10. 92, 5N, Bestellt\nobreak{}«. 
\newline{}Ordnung: mit Bleistift von unbekannter Hand nummeriert:
                                    »86« }\toendnotes[C]{\smallbreak}\pstart{}{\pb}Herrn \textsc{Felix
                     Salten}\pend{}\pstart{}\textsc{\textcolor{pink}{Wien}{}\ledrightnote{\textcolor{pink}{Wien}}}\pend{}\pstart{}\textsc{\textcolor{pink}{IX Berggasse 13}{}\ledrightnote{\textcolor{pink}{Berggasse}}}\pend{}
{\bigskip}
\pstart{}{\pb}Lieber Salten,\pend
\pstart
           morgen So{\geminationn}tag{ }Nachmittag \uuline{4 Uhr} ſind \textsc{\textcolor{blue}{BHofm}{}\ledrightnote{\textcolor{blue}{Richard Beer-Hofmann}}} u ich \textsc{\textcolor{pink}{Stefansplatz}{}\ledrightnote{\textcolor{pink}{Stephansplatz}}}, wollen in die \label{K_L02957-1v}\edtext{\textcolor{brown}{Ausſtellung}{}\ledrightnote{\textcolor{brown}{Internationale Ausstellung für Musik und Theaterwesen}}}{\lemma{\textnormal{\emph{Ausſtellung}}}\Cendnote{\textnormal{die \emph{\textcolor{brown}{Internationale Ausstellung für Musik und Theaterwesen}} im
                     \textcolor{pink}{Wien}er \textcolor{pink}{Prater}.}}}\label{K_L02957-1h}. – Ich ſchrieb auch an \textsc{\textcolor{blue}{Torresani}{}\ledrightnote{\textcolor{blue}{Carl von Torresani-Lanzenfeld}}}. Ko{\geminationm}en Sie doch auch!
            \pend
           \pstart Herzlich \spacefill\mbox{ArthSchn}\pend{}\endnumbering\briefempfaengerindex{Salten, Felix@\textsc{Salten, Felix}!zzzSchnitzler, Arthur@\emph{von Arthur Schnitzler}!1892-10-081@{8. 10. 1892}|)be}\mylabel{h}  \normalsize

\doendnotes{C}
\bigskip
\vfill

\clearpage

\footnotesize

\lohead{\textsc{register}}

% Definiere theindex-Environment komplett neu ohne reledmac
\makeatletter
\renewenvironment{theindex}{%
  \section*{\indexname}%
  \setlength{\parindent}{0pt}%
  \setlength{\parskip}{0pt plus 0.3pt}%
  \let\item\@idxitem
}{%
  \clearpage
}
\makeatother

\IfFileExists{\jobname-pw.ind}{\input{\jobname-pw.ind}}{}

\end{document}

      