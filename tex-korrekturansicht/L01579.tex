%% latex-korrekturansicht-vorspann.tex
%% Vorspann für die Korrekturansicht.
%% Lädt die gemeinsame Datei latex-vorspann.tex mit gesetztem Schalter.

\newif\ifkorrekturansicht
\korrekturansichttrue

\input{../tex-inputs/latex-vorspann}


               \section[Arthur Schnitzler an Charlotte Ehrenstein, 29. 1. 1906]{ Arthur Schnitzler an Charlotte Ehrenstein,
                    29. 1. 1906}\nopagebreak\mylabel{v}\rehead{ }\normalsize\beginnumbering\briefempfaengerindex{Ehrenstein, Charlotte@\textsc{Ehrenstein, Charlotte}!zzzSchnitzler, Arthur@\emph{von Arthur Schnitzler}!1906-01-293@{29. 1. 1906}|(be} \toendnotes[C]{\smallbreak\pagebreak[2]} \Standort{Jerusalem, The National Library of Israel, ARC. Ms. Var. 306 1 118.}
\physDesc{Briefkarte
\newline{}Handschrift: schwarze Tinte, deutsche Kurrent\newline{}Ordnung: mit Bleistift von unbekannter Hand nummeriert »3« }\toendnotes[C]{\smallbreak}\pstart
           \noindent{}{\pb}\textcolor{gray}{\textbf{Dr. Arthur Schnitzler}}{\\}\textcolor{gray}{\textbf{\textcolor{pink}{Wien, XVIII. Spoettelgasse 7}{}\ledrightnote{\textcolor{pink}{Edmund-Weiß-Gasse}}.}}\hfill 29. 1. 906. \pend
           \pstart
           Sehr geehrte gnäd\textcolor{gray}{i}ge Frau, ich danke Ihnen u
                    Herrn \textcolor{blue}{Treibel}{}\ledrightnote{\textcolor{blue}{Adolf Treibl}} für die freundlichen
                    Nachrichten; hoffentlich erfahr’ ich bald vollko{\geminationm}en
                    günſtiges über das Befinden Ihres \textcolor{blue}{Sohn\textcolor{gray}{e}s}{}\ledrightnote{→\textcolor{blue}{Albert Ehrenstein}}, den ich {\pb}beſtens zu grüßen bitte.\pend
           \pstart
           Mit Empfehlungen an Sie gnädige Frau und Ihren \textcolor{blue}{Gatten}{}\ledrightnote{→\textcolor{blue}{Alexander Ehrenstein}}\pend
           \pstart
           ganz ergebens{\\[\baselineskip]}\spacefill\mbox{ArtSch.}\pend
           \leftskip=0em{}\endnumbering\briefempfaengerindex{Ehrenstein, Charlotte@\textsc{Ehrenstein, Charlotte}!zzzSchnitzler, Arthur@\emph{von Arthur Schnitzler}!1906-01-293@{29. 1. 1906}|)be}\mylabel{h}  \normalsize

\doendnotes{C}
\bigskip
\vfill

\clearpage

\footnotesize

\lohead{\textsc{register}}

% Definiere theindex-Environment komplett neu ohne reledmac
\makeatletter
\renewenvironment{theindex}{%
  \section*{\indexname}%
  \setlength{\parindent}{0pt}%
  \setlength{\parskip}{0pt plus 0.3pt}%
  \let\item\@idxitem
}{%
  \clearpage
}
\makeatother

\IfFileExists{\jobname-pw.ind}{\input{\jobname-pw.ind}}{}

\end{document}

      