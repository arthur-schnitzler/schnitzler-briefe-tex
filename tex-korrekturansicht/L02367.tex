%% latex-korrekturansicht-vorspann.tex
%% Vorspann für die Korrekturansicht.
%% Lädt die gemeinsame Datei latex-vorspann.tex mit gesetztem Schalter.

\newif\ifkorrekturansicht
\korrekturansichttrue

\input{../tex-inputs/latex-vorspann}


               \section[Olga Schnitzler an Anna Bahr-Mildenburg, 11. 5. 1921]{ Olga Schnitzler an Anna Bahr-Mildenburg, 11. 5. 1921}\nopagebreak\mylabel{v}\rehead{ }\normalsize\beginnumbering\briefempfaengerindex{Bahr-Mildenburg, Anna@\textsc{Bahr-Mildenburg, Anna}!zzzSchnitzler, Olga@\emph{von Olga Schnitzler}!1921-05-111@{11. 5. 1921}|(be} \toendnotes[C]{\smallbreak\pagebreak[2]} \Standort{TMW, HS AM 31276 BaM.}
\physDesc{Brief, 1 Blatt, 2 Seiten
\newline{}Handschrift: schwarze Tinte, lateinische Kurrent}\buchAbdrucke{\weitereDrucke{1) Arthur Schnitzler: \emph{The Letters of Arthur Schnitzler to Hermann Bahr}. Edited, annotated, and with an introduction, by Donald G.
                        Daviau. Chapel Hill: \emph{The University of North Carolina Press} 1978, S. 116 (University of North Carolina studies in the Germanic languages
                        and literatures, 89).} \weitereDrucke{2) Hermann Bahr, Arthur Schnitzler: \emph{Briefwechsel, Aufzeichnungen, Dokumente (1891–1931)}. Hg. Kurt Ifkovits und Martin Anton Müller. Göttingen: \emph{Wallstein} 2018, S. 541–542.} }\toendnotes[C]{\smallbreak}\pstart{}{\pb}Meine liebe und hochverehrte gnädige Frau,\pend\pstart
           soeben erst erfahre ich von D\textsuperscript{r}{ }\label{K_L02367_1v}\edtext{\textcolor{blue}{Knappe}{}\ledrightnote{\textcolor{blue}{Heinrich Knappe}}}{\lemma{\textnormal{\emph{Knappe}}}\Cendnote{\textnormal{Korrepetitor von
                     \textcolor{blue}{Anna Bahr-Mildenburg}.}}}\label{K_L02367_1h}, was für
                  \label{K_L02367_2v}\edtext{schreckliche
                  Wochen}{\lemma{\textnormal{\emph{schreckliche
                  Wochen}}}\Cendnote{\textnormal{Am
                     17. 4. 1921 starb ihre Mutter \textcolor{blue}{Anna
                     Bellschan von Mildenburg} in \textcolor{pink}{Klagenfurt}.}}}\label{K_L02367_2h} Sie hatten, – ich hatte ja keine Ahnung! Ich war selbst
               krank und hab mich vor lauter Kummer ganz in meine vier Wände verkrochen,– nun war
                  \textcolor{blue}{Arthur}{}\ledrightnote{} eine Woche bei mir, er ist heute früh
               abgereist, und ich glaube, an freundlichere Zeiten und besseres Verstehen zwischen
               uns.\pend
           \pstart
           Nehmen Sie diese Blumen, liebe gnädige Frau, als ein Zeichen meiner innigsten
               Verehrung für Sie entgegen,– und glauben Sie an die herzlichste Anteilnahme{\pb}\pend
           \pstart
           Ihrer aufrichtig ergebenen{\\[\baselineskip]}\spacefill\mbox{Olga Schnitzler.}\pend
           \leftskip=0em{}\pstart
           \noindent{}11. Mai 21. \pend
           \endnumbering\briefempfaengerindex{Bahr-Mildenburg, Anna@\textsc{Bahr-Mildenburg, Anna}!zzzSchnitzler, Olga@\emph{von Olga Schnitzler}!1921-05-111@{11. 5. 1921}|)be}\mylabel{h}  \normalsize

\doendnotes{C}
\bigskip
\vfill

\clearpage

\footnotesize

\lohead{\textsc{register}}

% Definiere theindex-Environment komplett neu ohne reledmac
\makeatletter
\renewenvironment{theindex}{%
  \section*{\indexname}%
  \setlength{\parindent}{0pt}%
  \setlength{\parskip}{0pt plus 0.3pt}%
  \let\item\@idxitem
}{%
  \clearpage
}
\makeatother

\IfFileExists{\jobname-pw.ind}{\input{\jobname-pw.ind}}{}

\end{document}

      