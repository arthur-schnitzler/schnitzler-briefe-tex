%% latex-korrekturansicht-vorspann.tex
%% Vorspann für die Korrekturansicht.
%% Lädt die gemeinsame Datei latex-vorspann.tex mit gesetztem Schalter.

\newif\ifkorrekturansicht
\korrekturansichttrue

\input{../tex-inputs/latex-vorspann}


\renewcommand{\erwaehntePersonen}{Personen: Felix Dörmann, Adele Sandrock, Johanna Simonetta Sandrock}
\renewcommand{\erwaehnteOrte}{Orte: Café Arkaden, Riedhof, Volkstheater, Wien}
\renewcommand{\erwaehnteWerke}{Werke: Anfang vom Ende, Neue Deutsche Rundschau, Therese Krones. Genrebild mit Gesang und Tanz in drei Akten}
\section[Felix Salten an Arthur Schnitzler, {[}15.? 6. 1894{]}]{Felix Salten an Arthur Schnitzler, {[}15.? 6. 1894{]}}
\nopagebreak\mylabel{v}
\rehead{ }\normalsize\beginnumbering\briefempfaengerindex{Schnitzler, Arthur@\textsc{Schnitzler, Arthur}!zzzSalten, Felix@\emph{von Felix Salten}!1894-06-151@{{[}15.? 6. 1894{]}}|(be}
\toendnotes[C]{\smallbreak\pagebreak[2]}\Standort{CUL, Schnitzler, B 89, A 1.}
\physDesc{Brief, 1 Blatt, 1 Seite, 484 Zeichen
\newline{}Handschrift: schwarze Tinte, lateinische Kurrent
\newline{}Schnitzler: mit Bleistift datiert: »Juni 94« 
\newline{}Ordnung: mit Bleistift von unbekannter Hand nummeriert: »39« }\toendnotes[C]{\smallbreak}
\pstart{}{\pb}Lieber Freund!\pend
\pstart
           a.) \label{K_L03138-1v}\edtext{werde ich sogleich thun}{\lemma{\textnormal{\emph{werde ich sogleich thun}}}\Cendnote{\textnormal{Bezug unklar}}}\label{K_L03138-1h}, und mich bemühen, dass
               die Sache am Ende sich nicht jährt, ehe sie geordnet ist.\pend
           
\pstart
           b.) \label{K_L03138-2v}\edtext{soll in den nächsten Tagen
                  erfolgen}{\lemma{\textnormal{\emph{soll … erfolgen}}}\Cendnote{\textnormal{Bezug unklar}}}\label{K_L03138-2h}, bin
               nicht Schuld, dass es noch nicht geschehen.\pend
           
\pstart
           c.) \label{K_L03138-3v}\edtext{\textcolor{blue}{Dörmann}{}\ledrightnote{\textcolor{blue}{Felix Dörmann}} frägt an}{\lemma{\textnormal{\emph{Dörmann frägt an}}}\Cendnote{\textnormal{\textcolor{blue}{Felix Dörmann} arbeitete von März 1894 bis Ende Juni 1894 als Herausgeber der Zeitschrift \emph{\textcolor{green}{Neue Deutsche Rundschau}}. Darin findet sich in der
                  Zeit jedoch kein Abdruck dieses oder anderer \textcolor{green}{Gedicht}e von \textcolor{blue}{Schnitzler}. In einem Brief vom 20. 6. 1894
                  bat \textcolor{blue}{Dörmann}{ }\textcolor{blue}{Schnitzler}, ihm »ein paar andere
                     ſchöne Verse [zu] ſchicken«, was zumindest darauf hindeutet, dass \textcolor{blue}{Schnitzler} ihm – womöglich auf \textcolor{blue}{Salten}s Aufforderung hin – etwas geschickt
                  hatte. Nachdem \textcolor{blue}{Dörmann}s Engagement bei der
                  Monatsschrift aber nur wenige Wochen dauerte, dürfte das mit ein Grund sein, warum
                  aus der Sache nichts wurde.}}}\label{K_L03138-3h}, ob er Ihr Gedicht »\textcolor{green}{Dass all das Schöne nun längst zu Ende}{}\ledrightnote{{$\rightarrow$}\textcolor{green}{Anfang vom Ende}}«
               bringen darf. Schreiben Sie ihm vielleicht eine Karte.\pend
           
\pstart
           c.) Sind Sie \label{K_L03138-4v}\edtext{morgen bei »\textcolor{green}{Therese
                  Krones}{}\ledrightnote{\textcolor{green}{Therese Krones. Genrebild mit Gesang und Tanz in drei Akten}}?«}{\lemma{\textnormal{\emph{morgen … Krones?«}}}\Cendnote{\textnormal{Das erlaubt die genauere
                  Datierung, da die Premiere von \emph{\textcolor{green}{Therese Krones}}
                  am 16. 6. 1894 am
                     \textcolor{pink}{Deutschen Volkstheater} stattfand. Sowohl \textcolor{blue}{Schnitzler} als auch \textcolor{blue}{Salten} nahmen teil. Danach waren sie gemeinsam mit \textcolor{blue}{Adele Sandrock} und deren Mutter \textcolor{blue}{Johanna Simonetta Sandrock} im \textcolor{pink}{Riedhof}.}}}\label{K_L03138-4h} Ich bin auf alle Fälle da, und
                  \introOben{}wir\introOben{} soupiren dann zusammen? Wenn nicht \textcolor{pink}{Arkaden Café}{}\ledrightnote{\textcolor{pink}{Café Arkaden}}!\pend
           
\pstart
           Herzlichst Ihr {\\[\baselineskip]}\spacefill\mbox{Salten}\pend
           \leftskip=0em{}\endnumbering\briefempfaengerindex{Schnitzler, Arthur@\textsc{Schnitzler, Arthur}!zzzSalten, Felix@\emph{von Felix Salten}!1894-06-151@{{[}15.? 6. 1894{]}}|)be}\mylabel{h}  \normalsize

\doendnotes{C}
\bigskip
\vfill

\clearpage

\footnotesize

\lohead{\textsc{register}}

% Definiere theindex-Environment komplett neu ohne reledmac
\makeatletter
\renewenvironment{theindex}{%
  \section*{\indexname}%
  \setlength{\parindent}{0pt}%
  \setlength{\parskip}{0pt plus 0.3pt}%
  \let\item\@idxitem
}{%
  \clearpage
}
\makeatother

\IfFileExists{\jobname-pw.ind}{\input{\jobname-pw.ind}}{}

\end{document}

      