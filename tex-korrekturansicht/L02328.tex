%% latex-korrekturansicht-vorspann.tex
%% Vorspann für die Korrekturansicht.
%% Lädt die gemeinsame Datei latex-vorspann.tex mit gesetztem Schalter.

\newif\ifkorrekturansicht
\korrekturansichttrue

\input{../tex-inputs/latex-vorspann}


               \section[Robert Adam an Arthur Schnitzler, 8. 10. 1919]{ Robert Adam an Arthur Schnitzler, 8. 10. 1919}\nopagebreak\mylabel{v}\rehead{ }\normalsize\beginnumbering\briefempfaengerindex{Schnitzler, Arthur@\textsc{Schnitzler, Arthur}!zzzAdam, Robert@\emph{von Robert Adam}!1919-10-081@{8. 10. 1919}|(be} \toendnotes[C]{\smallbreak\pagebreak[2]} \Standort{Wien, Österreichische Nationalbibliothek, Cod. ser. 52.268, 23v.}
\physDesc{maschinelle Abschrift
\newline{}Schreibmaschine}\Standort{Wien, Österreichische Nationalbibliothek, Cod. ser. 52.268, 36.}
\physDesc{handschriftliche Abschrift
\newline{}Handschrift: schwarze Tinte, Gabelsberger Kurzschrift}\pstart
           \raggedleft{}{\pb}8. Oktober 1919\pend
           \pstart{}Hochverehrter Herr Doktor!\pend\pstart
           Ich nehme an, dass Sie bereits nach \textcolor{pink}{Wien}{}\ledrightnote{\textcolor{pink}{Wien}}
                    zurückgekehrt sind, und erlaube mir die Anfrage, ob ich Sie in der nächsten Zeit
                    einmal besuchen könnte?\pend
           \pstart
           Ich arbeite sehr fleissig an der \textcolor{green}{Märchenkomödie}{}\ledrightnote{\textcolor{green}{Märchenkomödie}}, selbst vewundert, dass ich noch nicht abgeschreckt bin.
                    Vielleicht wird aus ihr doch noch etwas, was mich zufrieden stellt. Mit den
                    besten Grüssen Ihr \spacefill\mbox{Dr RAP}\pend
           \endnumbering\briefempfaengerindex{Schnitzler, Arthur@\textsc{Schnitzler, Arthur}!zzzAdam, Robert@\emph{von Robert Adam}!1919-10-081@{8. 10. 1919}|)be}\mylabel{h}  \normalsize

\doendnotes{C}
\bigskip
\vfill

\clearpage

\footnotesize

\lohead{\textsc{register}}

% Definiere theindex-Environment komplett neu ohne reledmac
\makeatletter
\renewenvironment{theindex}{%
  \section*{\indexname}%
  \setlength{\parindent}{0pt}%
  \setlength{\parskip}{0pt plus 0.3pt}%
  \let\item\@idxitem
}{%
  \clearpage
}
\makeatother

\IfFileExists{\jobname-pw.ind}{\input{\jobname-pw.ind}}{}

\end{document}

      