%% latex-korrekturansicht-vorspann.tex
%% Vorspann für die Korrekturansicht.
%% Lädt die gemeinsame Datei latex-vorspann.tex mit gesetztem Schalter.

\newif\ifkorrekturansicht
\korrekturansichttrue

\input{../tex-inputs/latex-vorspann}


\renewcommand{\erwaehntePersonen}{Personen: Richard Beer-Hofmann, Theodore Rottenberg, Olga Schnitzler, Heinrich Schnitzler}
\renewcommand{\erwaehnteOrte}{Orte: Berlin, Dessauer Straße, Eppan an der Weinstraße, Marienbad, Südtirol, Tirol, Wien}
\renewcommand{\erwaehnteWerke}{Werke: Dr. Arthur Schnitzler vermählt sich, Prager Tagblatt}
\section[ Paul Goldmann an Arthur Schnitzler, 23. 7. {[}1903{]}]{Paul Goldmann an Arthur Schnitzler, 23. 7. {[}1903{]}}
\nopagebreak\mylabel{v}
\rehead{ }\normalsize\beginnumbering\briefempfaengerindex{Schnitzler, Arthur@\textsc{Schnitzler, Arthur}!zzzGoldmann, Paul@\emph{von Paul Goldmann}!1903-07-231@{23. 7. {[}1903{]}}|(be}
\toendnotes[C]{\smallbreak\pagebreak[2]}\Standort{DLA, A:Schnitzler, HS.NZ85.1.3173.}
\physDesc{Brief, 1 Blatt, 3 Seiten
\newline{}Handschrift: blaue Tinte, deutsche Kurrent
\newline{}Schnitzler: mit Bleistift das Jahr »{[}1{]}903« vermerkt }\toendnotes[C]{\smallbreak}
\pstart
           \noindent{}\raggedleft{}{\pb}\textcolor{gray}{\textbf{\textcolor{pink}{DESSAUERSTRASSE 19}{}\ledrightnote{\textcolor{pink}{Dessauer Straße}}}}\pend
           
\pstart
           \textcolor{pink}{Berlin}{}\ledrightnote{\textcolor{pink}{Berlin}}, 23. Juli.\pend
           
\pstart\center{}Mein lieber Freund,\pend
\pstart
           Unſere Briefe haben ſich gekreuzt. Wenn \label{K_L03378-1v}\edtext{»\textcolor{blue}{ſie}{}\ledrightnote{{$\rightarrow$}\textcolor{blue}{Theodore Rottenberg}}«}{\lemma{\textnormal{\emph{»ſie«}}}\Cendnote{\textnormal{siehe Paul Goldmann an Arthur Schnitzler, 19. 7. [1903]}}}\label{K_L03378-1h} mit mir kommt (was noch ſehr ungewiß iſt), werden ich wohl ſo \label{K_L03378-2v}\edtext{zwiſchen dem 5. und 10. Auguſt in \textcolor{pink}{Wien}{}\ledrightnote{\textcolor{pink}{Wien}} eintreffen, um von da nach \textcolor{pink}{Tirol}{}\ledrightnote{\textcolor{pink}{Tirol}{\newline}\textcolor{pink}{Südtirol}} weiterzufahren}{\lemma{\textnormal{\emph{zwiſchen … weiterzufahren}}}\Cendnote{\textnormal{siehe Paul Goldmann an Arthur Schnitzler, 27. 6. [1903]}}}\label{K_L03378-2h}. Biſt Du dann noch in \textcolor{pink}{Wien}{}\ledrightnote{\textcolor{pink}{Wien}}? Kommt »\textcolor{blue}{ſie}{}\ledrightnote{{$\rightarrow$}\textcolor{blue}{Theodore Rottenberg}}« nicht mit, ſo gehe ich
               vielleicht nach \textcolor{pink}{Marienbad}{}\ledrightnote{\textcolor{pink}{Marienbad}} zur Kur.\pend
           
\pstart
           Bitte nochmals: empfiehl’ mir {\pb}eine ſchön gelegene,
               kühle und billige \textcolor{pink}{Tirol}{}\ledrightnote{\textcolor{pink}{Tirol}{\newline}\textcolor{pink}{Südtirol}}er Sommerſtation,
               wo man nicht allzuſehr \label{K_L03378-4v}\edtext{unter
                  Beobachtung}{\lemma{\textnormal{\emph{unter
                  Beobachtung}}}\Cendnote{\textnormal{siehe Paul Goldmann an Arthur Schnitzler, 19. 7. [1903]}}}\label{K_L03378-4h} ſteht. \textsc{\textcolor{blue}{Richard}{}\ledrightnote{\textcolor{blue}{Richard Beer-Hofmann}}} widerräth \textsc{\textcolor{pink}{Eppan}{}\ledrightnote{\textcolor{pink}{Eppan an der Weinstraße}}} als zu heiß.\pend
           
\pstart
           Warum regſt Du Dich über die \label{K_L03378-6v}\edtext{Indiskretionen der \strikeout{\textcolor{gray}{×}} Zeitungen}{\lemma{\textnormal{\emph{Indiskretionen der Zeitungen}}}\Cendnote{\textnormal{Zeitungsmeldungen hatten
                  die bevorstehende Hochzeit von \textcolor{blue}{Schnitzler}
                  und \textcolor{blue}{Olga Gussmann} gebracht, beispielsweise:
                        »\textcolor{green}{– \so{Dr.{ }}\textcolor{blue}{\so{Arthur Schnitzler}} vermählt
                        ſich in den allernächſten Tagen in aller Stille mit Fräulein \textcolor{blue}{Olga \so{Gußmann}}.}«, \emph{\textcolor{green}{Prager Tagblatt}}, Jg. 27, Nr. 191,
                        15. 7. 1903, Morgen-Ausgabe, S. 8.}}}\label{K_L03378-6h}
               ſo auf? Das ſind doch die natürlichen Begleiterſcheinungen der Berühmtheit. Wenn man
               ſo in der Öffentlichkeit ſteht, wie Du, muß man ſich {\pb}auch gefallen laſſen, daß die Öffentlichkeit ſich mit Einem beſchäftigt. Ich finde
               darum die Zeitungen gar nicht ſo »widerlich«. Und ſchließlich: was ſchadet es auch,
               daß ſie melden, was doch bald wahr ſein wird? Sei nicht ſo nervös, mein lieber, alter
               (entſchuldige!) Freund!\pend
           
\pstart
           Grüße \textsc{\textcolor{blue}{Olga}{}\ledrightnote{\textcolor{blue}{Olga Schnitzler}}} und \textsc{\textcolor{blue}{Heinrich}{}\ledrightnote{\textcolor{blue}{Heinrich Schnitzler}}} und ſei ſelbſt vielmals und herzlichſt gegrüßt von Deinem {\\[\baselineskip]}\spacefill\mbox{Paul Goldmn\textcolor{gray}{.}}\pend
           \leftskip=0em{}\endnumbering\briefempfaengerindex{Schnitzler, Arthur@\textsc{Schnitzler, Arthur}!zzzGoldmann, Paul@\emph{von Paul Goldmann}!1903-07-231@{23. 7. {[}1903{]}}|)be}\mylabel{h}
\begin{anhang}
\end{anhang}\normalsize

\doendnotes{C}
\bigskip
\vfill

\clearpage

\footnotesize

\lohead{\textsc{register}}

% Definiere theindex-Environment komplett neu ohne reledmac
\makeatletter
\renewenvironment{theindex}{%
  \section*{\indexname}%
  \setlength{\parindent}{0pt}%
  \setlength{\parskip}{0pt plus 0.3pt}%
  \let\item\@idxitem
}{%
  \clearpage
}
\makeatother

\IfFileExists{\jobname-pw.ind}{\input{\jobname-pw.ind}}{}

\end{document}

      