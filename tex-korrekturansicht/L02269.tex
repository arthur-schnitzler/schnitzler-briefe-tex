%% latex-korrekturansicht-vorspann.tex
%% Vorspann für die Korrekturansicht.
%% Lädt die gemeinsame Datei latex-vorspann.tex mit gesetztem Schalter.

\newif\ifkorrekturansicht
\korrekturansichttrue

\input{../tex-inputs/latex-vorspann}


               \section[Arthur Schnitzler an Robert Adam, 27. 8. 1917]{ Arthur Schnitzler an Robert Adam, 27. 8. 1917}\nopagebreak\mylabel{v}\rehead{ }\normalsize\beginnumbering\briefempfaengerindex{Adam, Robert@\textsc{Adam, Robert}!zzzSchnitzler, Arthur@\emph{von Arthur Schnitzler}!1917-08-271@{27. 8. 1917}|(be} \toendnotes[C]{\smallbreak\pagebreak[2]} \Standort{DLA, 96.34.2/4.}
\physDesc{Postkarte
\newline{}Handschrift: Bleistift, lateinische Kurrent\newline{}Versand: Stempel: »\nobreak{}\oindex{XVIII., Waehring@\textbf{XVIII., Währing}, \emph{Bezirk (A.BZK)}|pwk}18/1 Wien, 27. VIII. 17\nobreak{}«.  }\toendnotes[C]{\smallbreak}\pstart{}{\pb}Schnitzler \textcolor{pink}{Wien
                            XVIII}{}\ledrightnote{\textcolor{pink}{VIII., Josefstadt}}\pend{}\pstart{}\textcolor{pink}{Sternwartestr 71}{}\ledrightnote{\textcolor{pink}{Sternwartestraße}}\pend{}{\bigskip}\pstart{}Herrn Dr. Robert Adam\pend{}\pstart{}Pollak,\pend{}\pstart{}\textcolor{pink}{Wien XII}{}\ledrightnote{\textcolor{pink}{XII., Meidling}}.\pend{}\pstart{}\textcolor{pink}{Meidlinger Hptstr 52 od. 56}{}\ledrightnote{\textcolor{pink}{Meidlinger Hauptstraße}}.\pend{}{\bigskip}\pstart
           \noindent{}{\pb}Verehrter Herr Doctor, vielen Dank für
                    Ihren lieben Brief. Sollten Sie \label{K_L02269_1v}\edtext{Dinstag}{\lemma{\textnormal{\emph{Dinstag}}}\Cendnote{\textnormal{Es kam zum gewünschten Treffen am
                        Folgetag, dem 28. 8. 1917.}}}\label{K_L02269_1h} gegen 7 nichts
                    bessres vorhaben, so wird es mich freuen, Sie bei mir zu sehen.\pend
           \pstart
           Herzlichst grüßt Sie{\\[\baselineskip]}Ihr erg{\\[\baselineskip]}\spacefill\mbox{A. S.}\pend
           \leftskip=0em{}\endnumbering\briefempfaengerindex{Adam, Robert@\textsc{Adam, Robert}!zzzSchnitzler, Arthur@\emph{von Arthur Schnitzler}!1917-08-271@{27. 8. 1917}|)be}\mylabel{h}  \normalsize

\doendnotes{C}
\bigskip
\vfill

\clearpage

\footnotesize

\lohead{\textsc{register}}

% Definiere theindex-Environment komplett neu ohne reledmac
\makeatletter
\renewenvironment{theindex}{%
  \section*{\indexname}%
  \setlength{\parindent}{0pt}%
  \setlength{\parskip}{0pt plus 0.3pt}%
  \let\item\@idxitem
}{%
  \clearpage
}
\makeatother

\IfFileExists{\jobname-pw.ind}{\input{\jobname-pw.ind}}{}

\end{document}

      