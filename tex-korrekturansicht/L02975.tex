%% latex-korrekturansicht-vorspann.tex
%% Vorspann für die Korrekturansicht.
%% Lädt die gemeinsame Datei latex-vorspann.tex mit gesetztem Schalter.

\newif\ifkorrekturansicht
\korrekturansichttrue

\input{../tex-inputs/latex-vorspann}


\renewcommand{\erwaehntePersonen}{Personen: Karl Kraus, Felix Salten, Gustav Schwarzkopf, Richard Winterstein}
\renewcommand{\erwaehnteOrte}{Orte: Kaltenleutgeben, Wien}
\renewcommand{\erwaehnteWerke}{Werke: Arthur Moeller-Bruck: Die moderne Literatur in Gruppen- und Einzeldarstellungen. Band X. Das junge Wien. Verlag von Schuster und Löffler, Berlin und Leipzig, Der Fall Baumberg, Die Fackel, Die Zeit. Wiener Wochenschrift, Die moderne Literatur in Gruppen- und Einzeldarstellungen. Band X. Das junge Wien, Eine Variété-Komödie, [In dieser Woche wird man wieder einmal Bernhard Baumeister feiern], »Sonnwendtag.« (Drama in fünf Aufzügen von Karl Schönherr. – Erste Aufführung im Burgtheater den 19. April 1902)}
\section[ Arthur Schnitzler an Felix Salten, 14. 6. 1902]{Arthur Schnitzler an Felix Salten, 14. 6. 1902}
\nopagebreak\mylabel{v}
\rehead{ }\normalsize\beginnumbering\briefempfaengerindex{Salten, Felix@\textsc{Salten, Felix}!zzzSchnitzler, Arthur@\emph{von Arthur Schnitzler}!1902-06-141@{14. 6. 1902}|(be}
\toendnotes[C]{\smallbreak\pagebreak[2]}\Standort{Wienbibliothek im Rathaus, ZPH 1681, 2.1.516.}
\physDesc{Brief, 1 Blatt, 2 Seiten, 308 Zeichen
\newline{}Handschrift: Bleistift, deutsche Kurrent
\newline{}Ordnung: mit Bleistift von unbekannter Hand Nummerierung der Blätter des Konvoluts:
                                    »61« }\toendnotes[C]{\smallbreak}
\pstart
           \raggedleft{}{\pb}14. 6. 902.\pend
           
\pstart
           lieber, wie ein Herr Dr \textsc{\textcolor{blue}{Winterstein}{}\ledrightnote{\textcolor{blue}{Richard Winterstein}}} dem Dr. \textsc{\textcolor{blue}{Schwarzkopf}{}\ledrightnote{\textcolor{blue}{Gustav Schwarzkopf}}} erzählte, war \textsc{\textcolor{blue}{Karl Kraus}{}\ledrightnote{\textcolor{blue}{Karl Kraus}}} von \label{K_L02975-1v}\edtext{\textsc{Martin Finder}}{\lemma{\textnormal{\emph{Martin Finder}}}\Cendnote{\textnormal{Unter diesem Pseudonym \textcolor{blue}{Salten}s (siehe Arthur Schnitzler an Felix Salten, 27. 5. 1902) erschienen – mit der Unsicherheit der mit Kürzel publizierten
                  Veröffentlichung – fünf Texte in Band 31 von \emph{\textcolor{green}{Die
                     Zeit. Wiener Wochenschrift}}: \textcolor{blue}{Martin Finder}: \emph{\textcolor{green}{Der Fall Baumberg}}. In: Nr. 394, 19. 4. 1902, S. 42–43; \textcolor{blue}{Martin Finder}: \emph{\textcolor{green}{»Sonnwendtag.« (Drama in fünf Aufzügen von Karl Schönherr.
                        – Erste Aufführung im Burgtheater den 19. April 1902)}}. In: Nr. 395,
                        26. 4. 1902, S. 58–59; \textcolor{blue}{M. F.}: \emph{\textcolor{green}{[In dieser Woche wird man wieder einmal Bernhard Baumeister feiern]}}.
                     In: Nr. 396, 3. 5. 1902, S. 75; \textcolor{blue}{Martin Finder}: \emph{\textcolor{green}{Arthur Moeller-Bruck: Die moderne Literatur in Gruppen- und
                        Einzeldarstellungen. Band X. Das junge Wien. Verlag von Schuster und
                        Löffler, Berlin und Leipzig}}. In: Nr. 399, 24. 5. 1902, S. 127; \textcolor{blue}{Martin Finder}: \emph{\textcolor{green}{Eine Variété-Komödie}} In: Nr. 400, 31. 5. 1902, S. 138–139. Jahre später
                  verwendete \textcolor{blue}{Salten} das Pseudonym gelegentlich
                  immer noch.}}}\label{K_L02975-1h} ſehr entzückt, den er offenbar wegen der \label{K_L02975-2v}\edtext{bekannten \textcolor{green}{Stelle}{}\ledrightnote{{$\rightarrow$}\textcolor{green}{Der Fall Baumberg}} für einen Chriſten, oder gar für einen
                  Antiſemiten}{\lemma{\textnormal{\emph{bekannten … Antiſemiten}}}\Cendnote{\textnormal{Am naheliegendsten ist,
                  dass \textcolor{blue}{Karl Kraus} die Besprechung \emph{\textcolor{green}{Arthur Moeller-Bruck: Die moderne Literatur in
                     Gruppen- und Einzeldarstellungen}} gefiel, da hier, vergleichbar mit seinen
                  Kritiken, der schlechten Sprache des besprochenen \textcolor{green}{Text}s viel Aufmerksamkeit geschenkt
                  wird. Die »bekannte[] Stelle« kann hingegen nicht mit Sicherheit
                  ermittelt werden. Eventuell bezog sich \textcolor{blue}{Schnitzler} gleich auf den ersten Text (\emph{\textcolor{green}{Der Fall Baumberg}}) bzw. folgende Passage: »\textcolor{green}{Von allen Erwerbsarten iſt
                        das Theater heute noch die beſte. Beſſer ſogar als die Börſe, weil man ja
                        nur gewinnen, aber nichts verlieren kann, weshalb wir denn auch ſo manchen
                        unter den Bühnendichtern ſehen, der ſonſt gewiſs nur als Börſeaner ſich
                        fortgebracht hätte.}« \textcolor{blue}{Schnitzler} fand das Lob des
                  unwissenden \textcolor{blue}{Kraus}’ wohl deshalb so
                     »amuſant«, weil \textcolor{blue}{Salten} und
                     \textcolor{blue}{Kraus} zerstritten waren und \textcolor{blue}{Salten} in der \emph{\textcolor{green}{Fackel}} häufig kritisiert wurde.}}}\label{K_L02975-2h}{ }{\pb}hielt.\pend
           
\pstart
           Ich finde dieſe Sachlichkeit wider Willen amuſant genug, um ſie Ihnen
               mitzutheilen {\\[\baselineskip]}Herzlich {\\[\baselineskip]}Ihr {\\[\baselineskip]}\spacefill\mbox{A.}\pend
           \leftskip=0em{}\endnumbering\briefempfaengerindex{Salten, Felix@\textsc{Salten, Felix}!zzzSchnitzler, Arthur@\emph{von Arthur Schnitzler}!1902-06-141@{14. 6. 1902}|)be}\mylabel{h}  \normalsize

\doendnotes{C}
\bigskip
\vfill

\clearpage

\footnotesize

\lohead{\textsc{register}}

% Definiere theindex-Environment komplett neu ohne reledmac
\makeatletter
\renewenvironment{theindex}{%
  \section*{\indexname}%
  \setlength{\parindent}{0pt}%
  \setlength{\parskip}{0pt plus 0.3pt}%
  \let\item\@idxitem
}{%
  \clearpage
}
\makeatother

\IfFileExists{\jobname-pw.ind}{\input{\jobname-pw.ind}}{}

\end{document}

      