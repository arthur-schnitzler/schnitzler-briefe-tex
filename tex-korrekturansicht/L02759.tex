%% latex-korrekturansicht-vorspann.tex
%% Vorspann für die Korrekturansicht.
%% Lädt die gemeinsame Datei latex-vorspann.tex mit gesetztem Schalter.

\newif\ifkorrekturansicht
\korrekturansichttrue

\input{../tex-inputs/latex-vorspann}


               \section[Paul Goldmann an Arthur Schnitzler, 16. 12. {[}1895{]}]{ Paul Goldmann an Arthur Schnitzler, 16. 12. {[}1895{]}}\nopagebreak\mylabel{v}\rehead{ }\normalsize\beginnumbering\briefempfaengerindex{Schnitzler, Arthur@\textsc{Schnitzler, Arthur}!zzzGoldmann, Paul@\emph{von Paul Goldmann}!1895-12-162@{16. 12. {[}1895{]}}|(be} \toendnotes[C]{\smallbreak\pagebreak[2]} \Standort{DLA, A:Schnitzler, HS.NZ85.1.3165.}
\physDesc{Brief, 1 Blatt, 3 Seiten
\newline{}Handschrift: blaue Tinte, deutsche Kurrent
\newline{}Schnitzler: mit Bleistift das Jahr » 95« vermerkt }\toendnotes[C]{\smallbreak}\pstart
           \noindent{}{\pb}\textcolor{gray}{\textbf{\textbf{\textcolor{brown}{Frankfurter Zeitung}{}\ledrightnote{\textcolor{brown}{Frankfurter Zeitung}}}}}\pend
           \pstart
           \textcolor{gray}{\textbf{(\textcolor{brown}{\begin{otherlanguage}{french}Gazette de Francfort\end{otherlanguage}}{}\ledrightnote{\textcolor{brown}{Frankfurter Zeitung}}). }}\pend
           \pstart
           \textcolor{gray}{\textbf{\textbf{\begin{otherlanguage}{french}Fondateur M. \textcolor{blue}{L.
                              Sonnemann}{}\ledrightnote{\textcolor{blue}{Leopold Sonnemann}}\end{otherlanguage}.}}}\pend
           \pstart
           \begin{otherlanguage}{french}\textcolor{gray}{\textbf{\textcolor{green}{Journal}{}\ledrightnote{→\textcolor{green}{Frankfurter Zeitung}} politique, financier,}}\end{otherlanguage}\pend
           \pstart
           \begin{otherlanguage}{french}\textcolor{gray}{\textbf{commercial et littéraire.}}\end{otherlanguage}\pend
           \pstart
           \begin{otherlanguage}{french}\textcolor{gray}{\textbf{\textbf{Paraissant trois fois par jour.}}}\end{otherlanguage}\pend
           \pstart
           \begin{otherlanguage}{french}\textcolor{gray}{\textbf{\textbf{Bureau à \textcolor{pink}{Paris}{}\ledrightnote{\textcolor{pink}{Paris}}:}}}\end{otherlanguage}\pend
           \pstart
           \begin{otherlanguage}{french}\textcolor{gray}{\textbf{\textbf{\textcolor{pink}{24. Rue Feydeau}{}\ledrightnote{\textcolor{pink}{rue Feydeau}}.}}}\end{otherlanguage}\hfill \textsc{\textcolor{pink}{Paris}{}\ledrightnote{\textcolor{pink}{Paris}}}, 16. December.\pend
           \pstart\center{}Mein lieber Freund,\pend\pstart
           die Opernglas-Definitionen Deines letzten lieben Briefes reichen nicht aus. Was
               verſtehſt Du unter »billig«? Ich habe mich umgethan\strikeout{,}
               und habe folgende Preiſe feſtgeſtellt: Ein kleines Damen-Opernglas aus buntfarbigem
               Perlmutter, innen vergoldet, koſtet von 35 \textsc{frcs} aufwärts;
               etwas kleiner iſt es auch zu 25 \textsc{frcs} zu haben. Beifolgendes
               Blatt Papier gibt die Größe der unteren Gläſer an; die {\pb}Tintenſtriche bezeichnen die Längen-Dimenſion, wenn
               es geſchloſſen iſt. Das ſieht ganz niedlich aus, aber die Gläſer ſind nicht gerade
               hervorragend, wie es natürlich iſt bei ſo kleinen Inſtrumenten. Würde das Deinem
               Wunſche entſprechen? Das iſt das billigſte Preis-Niveau; ſonſt natürlich ſind
               Inſtrumente von 100 \textsc{frcs} aufwärts zu haben. Ich habe eines
               für 150 mit zwölf Gläſern geſehen, das ſehr ſchön angibt; aber das iſt {\pb}natürlich zu theuer.\pend
           \pstart
           Laß’ mir umgehend Deine Aufträge zukommen. Nimm ruhig das für \textsc{35 Frcs}. Das Geld darfſt Du mir ſchicken, den \label{K_L02759-1v}\edtext{ich habe keinen \textsc{Sou}}{\lemma{\textnormal{\emph{ich habe keinen Sou}}}\Cendnote{\textnormal{im Sinne von: ich habe kein
               Geld}}}\label{K_L02759-1h}.\pend
           \pstart
           Kann Dir heute nicht mehr ſchreiben. Mein Kopf geht auseinander. Ich erlebe unſagbar
               traurige Dinge.\pend
           \pstart
           Grüß’ Dich Gott, liebſter {\\[\baselineskip]}Freund! Dein {\\[\baselineskip]}\spacefill\mbox{Paul Goldmann}\pend
           \leftskip=0em{}\pstart
           \noindent{}Wenn die Zeit zu kurz wird, telegraphire mir!\pend
           \endnumbering\briefempfaengerindex{Schnitzler, Arthur@\textsc{Schnitzler, Arthur}!zzzGoldmann, Paul@\emph{von Paul Goldmann}!1895-12-162@{16. 12. {[}1895{]}}|)be}\mylabel{h}\begin{anhang}\end{anhang}\normalsize

\doendnotes{C}
\bigskip
\vfill

\clearpage

\footnotesize

\lohead{\textsc{register}}

% Definiere theindex-Environment komplett neu ohne reledmac
\makeatletter
\renewenvironment{theindex}{%
  \section*{\indexname}%
  \setlength{\parindent}{0pt}%
  \setlength{\parskip}{0pt plus 0.3pt}%
  \let\item\@idxitem
}{%
  \clearpage
}
\makeatother

\IfFileExists{\jobname-pw.ind}{\input{\jobname-pw.ind}}{}

\end{document}

      