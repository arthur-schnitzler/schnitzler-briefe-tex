%% latex-korrekturansicht-vorspann.tex
%% Vorspann für die Korrekturansicht.
%% Lädt die gemeinsame Datei latex-vorspann.tex mit gesetztem Schalter.

\newif\ifkorrekturansicht
\korrekturansichttrue

\input{../tex-inputs/latex-vorspann}


\section[Stefan Zweig an Arthur Schnitzler, 2. 4. 1930]{L03693 Stefan Zweig an Arthur Schnitzler, 2. 4. 1930}
\nopagebreak\mylabel{L03693v}
\rehead{ }\normalsize\beginnumbering\briefempfaengerindex{Schnitzler, Arthur@\textsc{Schnitzler, Arthur}!zzzZweig, Stefan@\emph{von Stefan Zweig}!1930-04-021@{2. 4. 1930}|(be}
\toendnotes[C]{\smallbreak\pagebreak[2]}
\correspDesc{Versand  durch Stefan Zweig am 2. 4. 1930 in Salzburg
\newline{}Erhalt  durch Arthur Schnitzler im Zeitraum [3. 4. 1930
                  – 6. 4. 1930?] in Wien}\toendnotes[C]{\smallbreak}
\Standort{CUL, Schnitzler, B 118.}
\physDesc{Brief, 1 Blatt, 1 Seite, 413 Zeichen
\newline{}Schreibmaschine
\newline{}Handschrift: Bleistift, lateinische Kurrent (\noindent{}Unterschrift, Korrektur)
\newline{}Schnitzler: 1) mit rotem Buntstift beschriftet: »\textsc{Zweig}«  2) mit rotem Buntstift eine Unterstreichung}
\buchAbdrucke{\weitereDrucke{Stefan Zweig: \emph{Briefwechsel mit Hermann Bahr, Sigmund Freud, Rainer Maria
                        Rilke und Arthur Schnitzler}. Herausgegeben von Jeffrey B. Berlin, Hans-Ulrich Lindken und Donald A. Prater. Frankfurt am Main: \emph{S. Fischer} 1987, S. 449.} }\toendnotes[C]{\smallbreak}
\pstart
           {\pb}\textcolor{gray}{\textbf{SZ}}\hfill \textcolor{gray}{\textbf{\textcolor{pink}{Kapuzinerberg 5}\oindex{Paschinger Schlössl@\textbf{Paschinger Schlössl}, \emph{Wohngebäude}|pw}{}\ledrightnote{\textcolor{pink}{Paschinger Schlössl}}}}\pend
           
\pstart
           \raggedleft{}\textcolor{gray}{\textbf{\textcolor{pink}{SALZBURG}\oindex{Salzburg@\textbf{Salzburg}, \emph{Verwaltungsgebiet}|pw}{}\ledrightnote{\textcolor{pink}{Salzburg}}, am}}{ }2. April 1930. \pend
           
\pstart{}Sehr verehrter Herr Doktor!\pend\vspace{0.5em}
\pstart
           Ich bin \label{K_L03693-1v}\edtext{nicht Besitzer Ihrer geheimen
                  Telefonnummer}{\lemma{\textnormal{\emph{nicht … Telefonnummer}}}\Cendnote{\textnormal{Im Brief vom 2. 11. 1929 hatte \textcolor{blue}{Zweig}\pwindex{Zweig, Stefan 28.\,11.\,1881 Wien – 23.\,2.\,1942 Petrópolis@\textsc{Zweig, Stefan} (28.\,11.\,1881 Wien – 23.\,2.\,1942 Petrópolis), \emph{Schriftsteller}|pwk} diese schon von \textcolor{blue}{Berta Zuckerkandl}\pwindex{Zuckerkandl, Berta 13.\,4.\,1864 Wien – 16.\,10.\,1945 Paris@\textsc{Zuckerkandl, Berta} (13.\,4.\,1864 Wien – 16.\,10.\,1945 Paris), \emph{Schriftstellerin, Journalistin, Übersetzerin}|pwk} erfragt. Vgl. Arthur Schnitzler an Gerty Hofmannsthal, 17. 2. 1931.}}}\label{K_L03693-1}, vielleicht sind Sie so lieb, sie
               meiner \textcolor{pink}{Wiener}\oindex{Wien@\textbf{Wien}, \emph{Verwaltungsgebiet}|pw}{}\ledrightnote{\textcolor{pink}{Wien}} Adresse (\textcolor{pink}{IX., Garnisongasse 10}\oindex{Wien@\textbf{Wien}!IX., Alsergrund@\textbf{IX., Alsergrund}!Garnisongasse 10@\textbf{Garnisongasse 10}, \emph{Wohngebäude}|pw}{}\ledrightnote{\textcolor{pink}{Garnisongasse 10}}, Tel.Nr. A 26-0-57) anzuvertrauen und
               mir zu sagen, wann ich \substVorne{}\textsuperscript{s}\substDazwischen{}S\substHinten{}ie \label{K_L03693-2v}\edtext{wieder einmal sehen}{\lemma{\textnormal{\emph{wieder einmal sehen}}}\Cendnote{\textnormal{Laut \textcolor{blue}{Schnitzlers}{ }\emph{\textcolor{green}{Tagebuch}\pwindex{Schnitzler, Arthur 15. 5. 1862 Wien – 21. 10. 1931 ebd.@\textsc{Schnitzler, Arthur} (15. 5. 1862 Wien – 21. 10. 1931 ebd.), \emph{Schriftsteller, Mediziner}!Tagebuch@\strich\emph{Tagebuch}|pwk}} kam es am 6. 4. 1930 zu einem
                  Telefonat und am 7. 4. 1930 zu einem Treffen.}}}\label{K_L03693-2} dürfte; ich bin endlich
               wieder einmal eine Woche in \textcolor{pink}{Wien}\oindex{Wien@\textbf{Wien}, \emph{Verwaltungsgebiet}|pw}{}\ledrightnote{\textcolor{pink}{Wien}}, um ein wenig die
                  \label{K_L03693-3v}\edtext{\textcolor{green}{Proben}\pwindex{Zweig, Stefan 28.\,11.\,1881 Wien – 23.\,2.\,1942 Petrópolis@\textsc{Zweig, Stefan} (28.\,11.\,1881 Wien – 23.\,2.\,1942 Petrópolis), \emph{Schriftsteller}!Lamm des Armen. Tragikomödie in drei Akten@\strich\emph{Das Lamm des Armen. Tragikomödie in drei Akten}|pwv}{}\ledrightnote{{$\rightarrow$}\emph{\textcolor{green}{Das Lamm des Armen. Tragikomödie in drei Akten}}}}{\lemma{\textnormal{\emph{Proben}}}\Cendnote{\textnormal{\textcolor{blue}{Stefan Zweigs}\pwindex{Zweig, Stefan 28.\,11.\,1881 Wien – 23.\,2.\,1942 Petrópolis@\textsc{Zweig, Stefan} (28.\,11.\,1881 Wien – 23.\,2.\,1942 Petrópolis), \emph{Schriftsteller}|pwk} Theaterstück \emph{\textcolor{green}{Das Lamm des Armen}\pwindex{Zweig, Stefan 28.\,11.\,1881 Wien – 23.\,2.\,1942 Petrópolis@\textsc{Zweig, Stefan} (28.\,11.\,1881 Wien – 23.\,2.\,1942 Petrópolis), \emph{Schriftsteller}!Lamm des Armen. Tragikomödie in drei Akten@\strich\emph{Das Lamm des Armen. Tragikomödie in drei Akten}|pwk}} wurde am 12. 4. 1930 im
                     \textcolor{pink}{Wiener}\oindex{Wien@\textbf{Wien}, \emph{Verwaltungsgebiet}|pwk}{ }\textcolor{pink}{Burgtheater}\oindex{Wien@\textbf{Wien}!I., Innere Stadt@\textbf{I., Innere Stadt}!Burgtheater@\textbf{Burgtheater}, \emph{Theater}|pwk} erstaufgeführt.}}}\label{K_L03693-3}
               mitanzusehen.\pend
           
\pstart
           Ihr immer getreuer{\\[\baselineskip]}\spacefill\mbox{{[}hs.:{]} Stefan Zweig}\pend
           \leftskip=0em{}
\pstart
           \noindent{}Herrn Dr. Artur Schnitzler{\\}\uline{\textcolor{pink}{\so{Wien}, XVIII}\oindex{XVIII., Währing@\textbf{XVIII., Währing}, \emph{Verwaltungsgebiet}|pw}{}\ledrightnote{\textcolor{pink}{XVIII., Währing}}.}\pend
           \selectlanguage{ngerman}\endnumbering\briefempfaengerindex{Schnitzler, Arthur@\textsc{Schnitzler, Arthur}!zzzZweig, Stefan@\emph{von Stefan Zweig}!1930-04-021@{2. 4. 1930}|)be}\mylabel{L03693h}  \normalsize

\doendnotes{C}
\bigskip
\vfill

\clearpage

\footnotesize

\lohead{\textsc{register}}

% Definiere theindex-Environment komplett neu ohne reledmac
\makeatletter
\renewenvironment{theindex}{%
  \section*{\indexname}%
  \setlength{\parindent}{0pt}%
  \setlength{\parskip}{0pt plus 0.3pt}%
  \let\item\@idxitem
}{%
  \clearpage
}
\makeatother

\IfFileExists{\jobname-pw.ind}{\input{\jobname-pw.ind}}{}

\end{document}

      