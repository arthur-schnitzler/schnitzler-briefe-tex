%% latex-korrekturansicht-vorspann.tex
%% Vorspann für die Korrekturansicht.
%% Lädt die gemeinsame Datei latex-vorspann.tex mit gesetztem Schalter.

\newif\ifkorrekturansicht
\korrekturansichttrue

\input{../tex-inputs/latex-vorspann}


\renewcommand{\erwaehntePersonen}{Personen: Hugo von Hofmannsthal, Ida Nacht, Paul Salten}
\renewcommand{\erwaehnteOrte}{Orte: Berlin, Wien}
\renewcommand{\erwaehnteWerke}{Werke: Der Schrei der Liebe. Novelle}
\section[ Felix Salten an Arthur Schnitzler, {[}16.? 10. 1903{]}]{Felix Salten an Arthur Schnitzler, {[}16.? 10. 1903{]}}
\nopagebreak\mylabel{v}
\rehead{ }\normalsize\beginnumbering\briefempfaengerindex{Schnitzler, Arthur@\textsc{Schnitzler, Arthur}!zzzSalten, Felix@\emph{von Felix Salten}!1@{{[}16.? 10. 1903{]}}|(be}
\toendnotes[C]{\smallbreak\pagebreak[2]}\Standort{CUL, Schnitzler, B 89, A 2.}
\physDesc{Karte, 497 Zeichen
\newline{}Handschrift: schwarze Tinte, lateinische Kurrent
\newline{}Schnitzler: mit Bleistift datiert: »Oct {[}1{]}903« 
\newline{}Ordnung: mit Bleistift von unbekannter Hand nummeriert: »{\pb}176« }\toendnotes[C]{\smallbreak}
\pstart
           \noindent{}{\pb}Lieber, da wir die \textcolor{blue}{Amme}{}\ledrightnote{{$\rightarrow$}\textcolor{blue}{Ida Nacht}} und das \textcolor{blue}{Kleine}{}\ledrightnote{{$\rightarrow$}\textcolor{blue}{Paul Salten}} nicht so lange allein laßen wollen, \label{K_L03350-1v}\edtext{kommen wir Sonntag nicht zum Essen}{\lemma{\textnormal{\emph{kommen … Essen}}}\Cendnote{\textnormal{Hier dürfte es sich
                     um die Antwort auf \textcolor{blue}{Schnitzler}s Brief vom 15. 10. 1903
                        handeln, was eine genauere zeitliche Eingrenzung des undatierten Korrespondenzstücks über \textcolor{blue}{Schnitzler}s
                        Angabe »Oct 903« hinaus in den Zeitraum erlaubt. Zudem dürfte am Vortag des Treffens, dem 17. 10. 1903, bereits
                        von ›morgen‹ die Rede gewesen sein und wäre es eine sehr kurzfristige Absage des Mittagessens gewesen, weswegen dieser
                        Tag ebenfalls nicht in Frage kommt.}}}\label{K_L03350-1h}, sondern um 3 od.
                  ½ 4 zum Kaffee, wenn wir einen kriegen.\pend
           
\pstart
           \textcolor{blue}{Hfthl.}{}\ledrightnote{\textcolor{blue}{Hugo von Hofmannsthal}} bittet mich am \label{K_L03350-2v}\edtext{Dienstag vorzulesen, weil er 
               Mittwoch}{\lemma{\textnormal{\emph{Dienstag … Mittwoch}}}\Cendnote{\textnormal{Es dürfte sich dabei um einen weiteren Schlenker beim Versuch handeln, 
                  die private Lesung von \emph{\textcolor{green}{Der Schrei der Liebe}} zu terminisieren,
                  die dann trotz der Ankündigung im vorliegenden Korrespondenzstück am Mittwoch, dem 21. 10. 1903
                  stattfand. Dass der Termin am Mittwoch hielt, dürfte daran liegen, dass \textcolor{blue}{Hofmannsthal} erst am 26. 10. 1903 nach \textcolor{pink}{Berlin} reiste.}}}\label{K_L03350-2h} abreist. Also Dienstag.
               Ich hoffe sehr, dass Sie nicht verhindert sind, denn ich möchte es jetzt nicht mehr
               verschieben. Sonst müßte die Sache bis November bleiben,
               weil \textcolor{blue}{H.}{}\ledrightnote{\textcolor{blue}{Hugo von Hofmannsthal}} dabei sein will, und ein so langer
               Aufschub wäre mir jetzt mehr als unangenehm.\pend
           
\pstart
           Also zunächst auf Sonntag.\pend
           
\pstart
           herzlichst {\\[\baselineskip]}Ihr {\\[\baselineskip]}\spacefill\mbox{S.}\pend
           \leftskip=0em{}\endnumbering\briefempfaengerindex{Schnitzler, Arthur@\textsc{Schnitzler, Arthur}!zzzSalten, Felix@\emph{von Felix Salten}!1903-10-161@{{[}16.? 10. 1903{]}}|)be}\mylabel{h}  \normalsize

\doendnotes{C}
\bigskip
\vfill

\clearpage

\footnotesize

\lohead{\textsc{register}}

% Definiere theindex-Environment komplett neu ohne reledmac
\makeatletter
\renewenvironment{theindex}{%
  \section*{\indexname}%
  \setlength{\parindent}{0pt}%
  \setlength{\parskip}{0pt plus 0.3pt}%
  \let\item\@idxitem
}{%
  \clearpage
}
\makeatother

\IfFileExists{\jobname-pw.ind}{\input{\jobname-pw.ind}}{}

\end{document}

      