%% latex-korrekturansicht-vorspann.tex
%% Vorspann für die Korrekturansicht.
%% Lädt die gemeinsame Datei latex-vorspann.tex mit gesetztem Schalter.

\newif\ifkorrekturansicht
\korrekturansichttrue

\input{../tex-inputs/latex-vorspann}


\section[Arthur Schnitzler an Gustav Schwarzkopf, 22. 7. 1910]{L04162 Arthur Schnitzler an Gustav Schwarzkopf, 22. 7. 1910}
\nopagebreak\mylabel{L04162v}
\rehead{ }\normalsize\beginnumbering\briefempfaengerindex{Schwarzkopf, Gustav@\textsc{Schwarzkopf, Gustav}!zzzSchnitzler, Arthur@\emph{von Arthur Schnitzler}!1910-07-221@{22. 7. 1910}|(be}
\toendnotes[C]{\smallbreak\pagebreak[2]}
\correspDesc{Versand  durch Arthur Schnitzler am 22. 7. 1910 in Wien
\newline{}Erhalt  durch Gustav Schwarzkopf im Zeitraum [23. 7. 1910 – 27. 7. 1910?] \textbf{Ort fehlend} }\toendnotes[C]{\smallbreak}
\Standort{CUL, Schnitzler, B 96.}
\physDesc{Brief, 1 Blatt, 4 Seiten, 981 Zeichen
\newline{}Handschrift: Bleistift, deutsche Kurrent}\toendnotes[C]{\smallbreak}
\pstart
           {\pb}\textcolor{gray}{\textbf{Dr. Arthur Schnitzler}}\hfill 22. 7. 910\pend
           
\pstart
           \textcolor{gray}{\textbf{\textcolor{pink}{Wien XVIII.
                        Spoettelgasse 7}\oindex{Wien@\textbf{Wien}!XVIII., Währing@\textbf{XVIII., Währing}!Edmund-Weiß-Gasse@\textbf{Edmund-Weiß-Gasse}, \emph{Straße}|pw}{}\ledrightnote{\textcolor{pink}{Edmund-Weiß-Gasse}}.}}\pend
           
\pstart{}lieber Guſtav,\pend\vspace{0.5em}
\pstart
           ich habe Sie \label{K_L04162-1v}\edtext{geſtern aufgeſucht}{\lemma{\textnormal{\emph{geſtern aufgeſucht}}}\Cendnote{\textnormal{Das ist nur implizit im \emph{\textcolor{green}{Tagebuch}\pwindex{Schnitzler, Arthur 15. 5. 1862 Wien – 21. 10. 1931 ebd.@\textsc{Schnitzler, Arthur} (15. 5. 1862 Wien – 21. 10. 1931 ebd.), \emph{Schriftsteller, Mediziner}!Tagebuch@\strich\emph{Tagebuch}|pwk}}-Eintrag zum 21. 7. 1910 erwähnt: »Besorgungen in der
                        \textcolor{pink}{Stadt}\oindex{I., Innere Stadt@\textbf{I., Innere Stadt}, \emph{Verwaltungsgebiet}|pwv}.«}}}\label{K_L04162-1}
               und bei dieſer Gelegenheit erfahren, daß Sie ſchon fort ſind; ich dachte Sie wollten
               erſt Ende dieſer Woche abreiſen. Meinen \textcolor{blue}{Bruder}\pwindex{Schnitzler, Julius 13.\,7.\,1865 Wien – 29.\,6.\,1939 ebd.@\textsc{Schnitzler, Julius} (13.\,7.\,1865 Wien – 29.\,6.\,1939 ebd.), \emph{Chirurg}|pwv}{}\ledrightnote{{$\rightarrow$}\emph{\textcolor{blue}{Julius Schnitzler}}} hatt’ ich brieflich interpellirt, aber keine Antwort
               erhalten, bis er {\pb}\label{K_L04162-2v}\edtext{vorgeſtern Abend perſönlich}{\lemma{\textnormal{\emph{vorgeſtern Abend perſönlich}}}\Cendnote{\textnormal{Vgl. A. S.: \emph{Tagebuch}, 20. 7. 1910.}}}\label{K_L04162-2} erſchien – und ſich in der
               erwarteter Weiſe äußerte: daſs er Krankheitsfälle nicht aus der Ferne beurtheilen
               könne. Soweit ſich theoretiſch-mediziniſches daran knüpfte, entſprach es ungefähr den
               Anſichten, die ich Ihnen gegenüber ausge{\pb}ſprochen. Den \label{K_L04162-3v}\edtext{Chirurgen \textsc{\textcolor{blue}{Wiesinger}\pwindex{Wiesinger @\textsc{Wiesinger}, \emph{Chirurg}|pw}{}\ledrightnote{\textcolor{blue}{Wiesinger}}}}{\lemma{\textnormal{\emph{Chirurgen Wiesinger}}}\Cendnote{\textnormal{In \textcolor{pink}{Wien}\oindex{Wien@\textbf{Wien}, \emph{Verwaltungsgebiet}|pwk} ist zu dieser Zeit kein Arzt dieses Namens nachgewiesen.}}}\label{K_L04162-3} ke{\geminationn}t mein \textcolor{blue}{Bruder}\pwindex{Schnitzler, Julius 13.\,7.\,1865 Wien – 29.\,6.\,1939 ebd.@\textsc{Schnitzler, Julius} (13.\,7.\,1865 Wien – 29.\,6.\,1939 ebd.), \emph{Chirurg}|pwv}{}\ledrightnote{{$\rightarrow$}\emph{\textcolor{blue}{Julius Schnitzler}}} dem Namen nach als ſehr tüchtig. –\pend
           
\pstart
           – \label{K_L04162-4v}\edtext{Montag oder
                  Dinſtag}{\lemma{\textnormal{\emph{Montag oder
                  Dinſtag}}}\Cendnote{\textnormal{Es wurde
                  Dienstag, der 26. 7. 1910.}}}\label{K_L04162-4} will ich, \introOben{}(\introOben{}wahrſcheinlich\introOben{})\introOben{}{ }\strikeout{O}
               allein, da \textcolor{blue}{Olga}\pwindex{Schnitzler, Olga 17.\,1.\,1882 Wien – 13.\,1.\,1970 Lugano@\textsc{Schnitzler, Olga} (17.\,1.\,1882 Wien – 13.\,1.\,1970 Lugano), \emph{Schauspielerin, Sängerin}|pw}{}\ledrightnote{\textcolor{blue}{Olga Schnitzler}} zu viel im \textcolor{pink}{Haus}\oindex{Wien@\textbf{Wien}!XVIII., Währing@\textbf{XVIII., Währing}!Sternwartestraße 71@\textbf{Sternwartestraße 71}, \emph{Wohngebäude}|pwv}{}\ledrightnote{{$\rightarrow$}\emph{\textcolor{pink}{Sternwartestraße 71}}} zu thun hat, aus dem die Handwerker
               noch nicht verſchwunden ſind) auf ein paar {\pb}Tage auf den \textcolor{pink}{Se{\geminationm}ering}\oindex{Semmering@\textbf{Semmering}, \emph{Verwaltungsgebiet}|pw}{}\ledrightnote{\textcolor{pink}{Semmering}} (vielleicht \label{K_L04162-5v}\edtext{über \textcolor{pink}{Mönichkirchen}\oindex{Mönichkirchen [Niederösterreich]@\textbf{Mönichkirchen [Niederösterreich]}, \emph{Verwaltungsgebiet}|pw}{}\ledrightnote{\textcolor{pink}{Mönichkirchen [Niederösterreich]}}}{\lemma{\textnormal{\emph{über Mönichkirchen}}}\Cendnote{\textnormal{Nach \textcolor{pink}{Mönichkirchen}\oindex{Mönichkirchen [Niederösterreich]@\textbf{Mönichkirchen [Niederösterreich]}, \emph{Verwaltungsgebiet}|pwk} kam \textcolor{blue}{Schnitzler} erst am A. S.: \emph{Wiener Schnitzler}, 28. 7. 1910.}}}\label{K_L04162-5}.) Ihr \textcolor{blue}{Bruder}\pwindex{Schwarzkopf, Max 12.\,6.\,1857 Wien – 14.\,4.\,1928 ebd.@\textsc{Schwarzkopf, Max} (12.\,6.\,1857 Wien – 14.\,4.\,1928 ebd.), \emph{Rechtsanwalt}|pwv}{}\ledrightnote{{$\rightarrow$}\emph{\textcolor{blue}{Max Schwarzkopf}}} gab
               mir die Hoffnung, daſs Sie \label{K_L04162-6v}\edtext{auch hinauf}{\lemma{\textnormal{\emph{auch hinauf}}}\Cendnote{\textnormal{Daraus dürfte nichts geworden sein.}}}\label{K_L04162-6} kommen wollen, we{\geminationn} ein Zimmer frei wird. Jedenfalls ſehen wir uns. Meine Adreſſe: \textcolor{pink}{Südbahnhotel}\oindex{Südbahnhotel [Semmering]@\textbf{Südbahnhotel [Semmering]}, \emph{Hotel}|pw}{}\ledrightnote{\textcolor{pink}{Südbahnhotel [Semmering]}}.\pend
           
\pstart
           Herzlichſt mit Grüßen{\\[\baselineskip]} von uns Allen Ihr \spacefill\mbox{A.}\pend
           \leftskip=0em{}\selectlanguage{ngerman}\endnumbering\briefempfaengerindex{Schwarzkopf, Gustav@\textsc{Schwarzkopf, Gustav}!zzzSchnitzler, Arthur@\emph{von Arthur Schnitzler}!1910-07-221@{22. 7. 1910}|)be}\mylabel{L04162h}
\begin{anhang}
\end{anhang}\normalsize

\doendnotes{C}
\bigskip
\vfill

\clearpage

\footnotesize

\lohead{\textsc{register}}

% Definiere theindex-Environment komplett neu ohne reledmac
\makeatletter
\renewenvironment{theindex}{%
  \section*{\indexname}%
  \setlength{\parindent}{0pt}%
  \setlength{\parskip}{0pt plus 0.3pt}%
  \let\item\@idxitem
}{%
  \clearpage
}
\makeatother

\IfFileExists{\jobname-pw.ind}{\input{\jobname-pw.ind}}{}

\end{document}

      