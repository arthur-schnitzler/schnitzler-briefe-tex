%% latex-korrekturansicht-vorspann.tex
%% Vorspann für die Korrekturansicht.
%% Lädt die gemeinsame Datei latex-vorspann.tex mit gesetztem Schalter.

\newif\ifkorrekturansicht
\korrekturansichttrue

\input{../tex-inputs/latex-vorspann}


               \section[Oscar Blumenthal an Arthur Schnitzler, 16. 1. 1894]{ Oscar Blumenthal an Arthur Schnitzler, 16. 1. 1894}\nopagebreak\mylabel{v}\rehead{ }\normalsize\beginnumbering\briefempfaengerindex{Schnitzler, Arthur@\textsc{Schnitzler, Arthur}!zzzBlumenthal, Oskar@\emph{von Oskar Blumenthal}!1894-01-161@{16. 1. 1894}|(be} \toendnotes[C]{\smallbreak\pagebreak[2]} \Standort{CUL, Schnitzler, B 15.}
\physDesc{Brief, 1 Blatt, 1 Seite
\newline{}Handschrift  : schwarze Tinte, deutsche Kurrent\newline{}Handschrift Oskar Blumenthal: schwarze Tinte, deutsche Kurrent (\noindent{}Unterschrift)
\newline{}Schnitzler: 1) mit Bleistift auf der Rückseite beschriftet: »\textsc{Blumenthal}« 2) mit rotem Buntstift eine Unterstreichung und nummeriert:
                                    »5«}\toendnotes[C]{\smallbreak}\pstart
           \noindent{}\centering{}{\pb}\textcolor{gray}{\textbf{\textsc{\textcolor{brown}{Lessing-Theater}{}\ledrightnote{\textcolor{brown}{Lessing-Theater}}}}}\pend
           \pstart
           \noindent{}\centering{}\textcolor{gray}{\textbf{\textsc{Director}:}}{\\}\textcolor{gray}{\textbf{DR. OSCAR BLUMENTHAL.}}\pend
           \pstart
           \noindent{}\raggedleft{}\textcolor{gray}{\textbf{\textcolor{pink}{Berlin N.W.}{}\ledrightnote{\textcolor{pink}{Berlin}}, den}}{ }16. Januar \textcolor{gray}{\textbf{189}}4.{\\}\textcolor{gray}{\textbf{\textcolor{pink}{Friedrich-Carl-Ufer}{}\ledrightnote{\textcolor{pink}{Kapelle-Ufer}}}}.\pend
           \pstart
           \centering{}Werther Herr Doktor!\pend
           \pstart
           \noindent{}Nach dem wenig ermuthigenden Ausgang der \textcolor{pink}{Wien}{}\ledrightnote{\textcolor{pink}{Wien}}er
                  \label{K_L00292_1v}\edtext{Probeaufführung}{\lemma{\textnormal{\emph{Probeaufführung}}}\Cendnote{\textnormal{Die Uraufführung fand am
                     1. 12. 1893 am \textcolor{pink}{Deutschen
                     Volkstheater} statt. Bereits nach der zweiten Vorstellung wurde das Stück
                  abgesetzt.}}}\label{K_L00292_1h} des »\textcolor{green}{Märchens}{}\ledrightnote{\textcolor{green}{Das Märchen. Schauspiel in drei Aufzügen}}« glaube ich, daß
               wir gut thun werden, vorläufig in \textcolor{pink}{Berlin}{}\ledrightnote{\textcolor{pink}{Berlin}} von dem \textcolor{green}{Stücke}{}\ledrightnote{→\textcolor{green}{Das Märchen. Schauspiel in drei Aufzügen}} abzuſehen. Sehr gerne werde
               ich gelegentlich einen Ihrer Einakter bringen; aber da es ſich hier immer darum
               handelt, ein begleitendes Werk zu finden, das für ſich allein den Abend nicht
               ausfüllen würde, ſo läßt ſich hier beim beſten Willen ein Darſtellungstermin nicht
               feſtſetzen.\pend
           \pstart
           Mit beſten Grüßen Ihr ergebener{\\[\baselineskip]}\spacefill\mbox{{[}hs. Blumenthal:{]} Dr. Osc. Blumenthal}\pend
           \leftskip=0em{}\endnumbering\briefempfaengerindex{Schnitzler, Arthur@\textsc{Schnitzler, Arthur}!zzzBlumenthal, Oskar@\emph{von Oskar Blumenthal}!1894-01-161@{16. 1. 1894}|)be}\mylabel{h}  \normalsize

\doendnotes{C}
\bigskip
\vfill

\clearpage

\footnotesize

\lohead{\textsc{register}}

% Definiere theindex-Environment komplett neu ohne reledmac
\makeatletter
\renewenvironment{theindex}{%
  \section*{\indexname}%
  \setlength{\parindent}{0pt}%
  \setlength{\parskip}{0pt plus 0.3pt}%
  \let\item\@idxitem
}{%
  \clearpage
}
\makeatother

\IfFileExists{\jobname-pw.ind}{\input{\jobname-pw.ind}}{}

\end{document}

      