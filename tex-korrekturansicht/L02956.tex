%% latex-korrekturansicht-vorspann.tex
%% Vorspann für die Korrekturansicht.
%% Lädt die gemeinsame Datei latex-vorspann.tex mit gesetztem Schalter.

\newif\ifkorrekturansicht
\korrekturansichttrue

\input{../tex-inputs/latex-vorspann}


\renewcommand{\erwaehntePersonen}{Personen: Hermann Bahr, Felix Salten}
\renewcommand{\erwaehnteOrte}{Orte: Internationales Ausstellungstheater im k.k. Prater, Wien}
\renewcommand{\erwaehnteWerke}{Werke: Allgemeine Theater-Revue für Bühne und Welt, Theater-Briefe. Wien}
\section[Arthur Schnitzler an Felix Salten, {[}21. 5. 1892?{]}]{Arthur Schnitzler an Felix Salten, {[}21. 5. 1892?{]}}
\nopagebreak\mylabel{v}
\rehead{ }\normalsize\beginnumbering\briefempfaengerindex{Salten, Felix@\textsc{Salten, Felix}!zzzSchnitzler, Arthur@\emph{von Arthur Schnitzler}!1892-05-211@{{[}21. 5. 1892?{]}}|(be}
\toendnotes[C]{\smallbreak\pagebreak[2]}\Standort{Wienbibliothek im Rathaus, ZPH 1681, 2.1.516.}
\physDesc{Brief, 1 Blatt, 2 Seiten, 297 Zeichen
\newline{}Handschrift: Bleistift, deutsche Kurrent
\newline{}Ordnung: mit Bleistift von unbekannter Hand Nummerierung der Blätter des Konvoluts:
                                    »24« }\toendnotes[C]{\smallbreak}
\pstart
           \raggedleft{}{\pb}\uline{\label{K_L02956-1v}\edtext{Samſtag}{\lemma{\textnormal{\emph{Samſtag}}}\Cendnote{\textnormal{Das Erscheinen des \textcolor{green}{Artikel}s von \textcolor{blue}{Bahr} gibt eine zeitliche Einordnung.}}}\label{K_L02956-1h}.}\pend
           
\pstart{}Lieber Freund,\pend
\pstart
           es wäre mir ſehr angenehm, Sie beim Schneider{ }heut{ }Abend zu ſehen (ich habe einen \label{K_L02956-3v}\edtext{Sitz ins \textcolor{pink}{Theater}{}\ledrightnote{{$\rightarrow$}\textcolor{pink}{Internationales Ausstellungstheater im k.k. Prater}}}{\lemma{\textnormal{\emph{Sitz ins Theater}}}\Cendnote{\textnormal{siehe A. S.: \emph{Tagebuch}, 21. 5. 1892}}}\label{K_L02956-3h}.)\pend
           
\pstart
           – Ich werde wahrſcheinlich \uline{morgen}{ }Nachmttg frei ſein.\pend
           
\pstart
           {\pb}– Eben den \label{K_L02956-4v}\edtext{\textcolor{green}{Artikel}{}\ledrightnote{{$\rightarrow$}\textcolor{green}{Theater-Briefe. Wien}}}{\lemma{\textnormal{\emph{Artikel}}}\Cendnote{\textnormal{\textcolor{blue}{Hermann Bahr}: \emph{\textcolor{green}{Theater-Briefe. Wien}}. In: \emph{\textcolor{green}{Allgemeine Theater-Revue für Bühne und Welt}}, Jg. 1,
                     Nr. 4, Mitte Mai 1892, S. 40–41.}}}\label{K_L02956-4h}
               von \textsc{\textcolor{blue}{Bahr}{}\ledrightnote{\textcolor{blue}{Hermann Bahr}}} geleſen in der \textsc{\textcolor{green}{Theater revue}{}\ledrightnote{\textcolor{green}{Allgemeine Theater-Revue für Bühne und Welt}}}, den ich ſehr luſtig finde; es iſt wenigſtens echter \textcolor{gray}{\textcolor{blue}{Bahr}{}\ledrightnote{\textcolor{blue}{Hermann Bahr}}}.–\pend
           
\pstart
           Herzlichſt Ihr {\\[\baselineskip]}\spacefill\mbox{Arth}\pend
           \leftskip=0em{}\endnumbering\briefempfaengerindex{Salten, Felix@\textsc{Salten, Felix}!zzzSchnitzler, Arthur@\emph{von Arthur Schnitzler}!1892-05-211@{{[}21. 5. 1892?{]}}|)be}\mylabel{h}  \normalsize

\doendnotes{C}
\bigskip
\vfill

\clearpage

\footnotesize

\lohead{\textsc{register}}

% Definiere theindex-Environment komplett neu ohne reledmac
\makeatletter
\renewenvironment{theindex}{%
  \section*{\indexname}%
  \setlength{\parindent}{0pt}%
  \setlength{\parskip}{0pt plus 0.3pt}%
  \let\item\@idxitem
}{%
  \clearpage
}
\makeatother

\IfFileExists{\jobname-pw.ind}{\input{\jobname-pw.ind}}{}

\end{document}

      