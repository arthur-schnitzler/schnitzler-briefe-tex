%% latex-korrekturansicht-vorspann.tex
%% Vorspann für die Korrekturansicht.
%% Lädt die gemeinsame Datei latex-vorspann.tex mit gesetztem Schalter.

\newif\ifkorrekturansicht
\korrekturansichttrue

\input{../tex-inputs/latex-vorspann}


               \section[Arthur Schnitzler an Georg Brandes, 22. 12. 1915]{ Arthur Schnitzler an Georg Brandes, 22. 12. 1915}\nopagebreak\mylabel{v}\rehead{ }\normalsize\beginnumbering\briefempfaengerindex{Brandes, Georg@\textsc{Brandes, Georg}!zzzSchnitzler, Arthur@\emph{von Arthur Schnitzler}!1915-12-221@{22. 12. 1915}|(be} \toendnotes[C]{\smallbreak\pagebreak[2]} \Standort{Kopenhagen, Det Kongelige Bibliotek, Georg Brandes Arkiv, box 125.}
\physDesc{Brief, 2 Blätter, 3 Seiten (Seite 3 mit Schreibmaschine paginiert)
\newline{}Schreibmaschine
\newline{}Handschrift: schwarze Tinte (\noindent{}Überarbeitung, Unterstreichung, Unterschrift)\newline{}Ordnung: mit Bleistift von unbekannter Hand auf dem ersten Blatt nummeriert: »39.«, das zweite Blatt datiert mit »22/12 15« }\buchAbdrucke{\weitereDrucke{1) Georg Brandes, Arthur Schnitzler: \emph{Ein Briefwechsel}. Hg. Kurt Bergel. Bern: \emph{Francke} 1956, S. 120–121.} \weitereDrucke{2) Arthur Schnitzler: \emph{Briefe 1913–1931}. Hg. Peter Michael Braunwarth, Richard Miklin, Susanne Pertlik und Heinrich Schnitzler. Frankfurt am Main: \emph{S. Fischer} 1984, S. 109–110.} }\toendnotes[C]{\smallbreak}\pstart
           \noindent{}{\pb}\textcolor{gray}{\textbf{Dr. Arthur Schnitzler}}{\\}\textcolor{gray}{\textbf{\textcolor{pink}{Wien XVIII. Sternwartestrasse 71}{}\ledrightnote{\textcolor{pink}{Sternwartestraße}}}}\pend
           \pstart
           \raggedleft{}22. 12. 1915. \pend
           \pstart\center{}Lieber und verehrter Freund.\pend\pstart
           Herzlichsten Dank für Ihre rasche Antwort\introOben{},\introOben{} und zugleich
                    eine Aufklärung. Es ist mir gar nicht eingefallen eine \introOben{}»\introOben{}Anspielung\introOben{}«\introOben{} zu machen, d\substVorne{}\textsuperscript{a}\substDazwischen{}e\substHinten{}nn das, worauf ich Ihrer Meinung nach angespielt habe, ist mir bis zum
                    Eintreffen Ihres Briefes total unbekannt geblieben. Wenn ich diesen richtig
                    verstanden habe, hat man Ihnen offenbar Aeusserungen in den Mund gelegt, die Sie
                    niemals getan haben. Mir ist gleich zu Anfang des Krieges ganz Aehnliches
                    passiert. Von Freunden in \textcolor{pink}{Russland}{}\ledrightnote{\textcolor{pink}{Russland}} wurde ich
                    in Kenntnis gesetzt, es sei in dortigen Zeitungen ein \textcolor{green}{Interview}{}\ledrightnote{→\textcolor{green}{?? [Fiktives Interview aus der Kriegszeit]}} erschienenen, in dem ich irgend
                    einem \textcolor{blue}{Journalisten}{}\ledrightnote{→\textcolor{blue}{?? [Journalist, der fiktives russisches Interview verantwortet]}}
                    gegenüber die albernsten Dinge über \textcolor{blue}{Tolstoi}{}\ledrightnote{\textcolor{blue}{Leo N. von Tolstoi}},
                        \textcolor{blue}{Anatole France}{}\ledrightnote{\textcolor{blue}{Anatole France}}, \textcolor{blue}{Shakespeare}{}\ledrightnote{\textcolor{blue}{William Shakespeare}} und \textcolor{blue}{Maeterlin\introOben{}c\introOben{}k}{}\ledrightnote{\textcolor{blue}{Maurice Maeterlinck}} geäussert hätte. Man riet
                    mir dringend etwas dagegen zu unternehmen (was ich anfangs nicht wollte), weil
                    man in \textcolor{pink}{Russland}{}\ledrightnote{\textcolor{pink}{Russland}} all diesen Unsinn glaubte.
                    Durch Vermittlung \textcolor{blue}{Romain Rollands}{}\ledrightnote{\textcolor{blue}{Romain Rolland}} liess ich
                    nun in \textcolor{pink}{Schweiz}{}\ledrightnote{\textcolor{pink}{Schweiz}}er Blättern eine Entgegnung
                        erschei{\pb}nen, in der ich versicherte, dass
                    ich niemals ein Wort von all dem Widersinn geäussert und bald darauf stellte
                    sich das Ganze auch als die \label{K_L02224_1v}\edtext{Mystifikation}{\lemma{\textnormal{\emph{Mystifikation}}}\Cendnote{\textnormal{nicht
                        ermittelt}}}\label{K_L02224_1h} irgend eines \textcolor{brown}{russischen Winkelblattes}{}\ledrightnote{→\textcolor{brown}{[Russische Zeitschrift, in der 1914 gefälschtes Interview  von Schnitzler erschien]}} heraus. Hingegen wurde ich von gewissen \textcolor{pink}{deutschen}{}\ledrightnote{\textcolor{pink}{Deutschland}} und \textcolor{pink}{österreichischen}{}\ledrightnote{\textcolor{pink}{Österreich}}, selbstverständlich antisemitischen Blättern in der
                    blödesten Weise angegriffen, weil ich es für notwendig gefunden hatte jene
                    erlogenen Aeusserungen über die feindesländischen Dichter richtig zu stellen.
                    Und noch bei Gelegenheit meiner letzten \label{K_L02224_2v}\edtext{\textcolor{green}{Premiere}{}\ledrightnote{→\textcolor{green}{Komödie der Worte. Drei Einakter}}}{\lemma{\textnormal{\emph{Premiere}}}\Cendnote{\textnormal{\emph{\textcolor{green}{Komödie der Worte}}, Uraufführung am
                            12. 10. 1915.}}}\label{K_L02224_2h} bekam ich es in irgend einem solchen,
                    sich patr\introOben{}i\introOben{}otisch gebärdenden Journal zu lesen, dass mir
                    das Organ für diese Zeit fehle, wie ich ja schon zu Beginn des Krieges
                    (wörtlich) »\textcolor{green}{\label{K_L02224_3v}\edtext{Torheiten über unsere
                            Feinde}{\lemma{\textnormal{\emph{Torheiten … Feinde}}}\Cendnote{\textnormal{[O. V.:] \emph{\textcolor{green}{Komödie der Worte}}. In:
                                    \emph{\textcolor{green}{Deutsche Tageszeitung}}, Jg. 22,
                                Nr. 517, 15. 10. 1915, S. 6. Als unmittelbare
                            Quelle bietet sich die – möglicherweise von \textcolor{blue}{Hans Brecka} gestaltete – Zusammenstellung \emph{\textcolor{green}{Kampf gegen den Theaterschund und
                                Bühnenschmutz}} ([O. V.], in: \emph{\textcolor{green}{Reichspost}}, Jg. 22, Nr. 508, 28. 10. 1915,
                                S. 9) an.}}}\label{K_L02224_3h}}{}\ledrightnote{→\textcolor{green}{Komödie der Worte}}« geäussert. Sie können sich also
                    denken, lieber Freund, dass es mir schon a priori näher liegen müsste \strikeout{dergleichen} Zeitungsgeschwätz anzuzweifeln als es
                    auf Treu und Glauben hinzunehmen. Meine von Ihnen missverstandene Bemerkung aber
                    bezog sich nur auf den Umstand, dass unseres Wissens in den {\pb}ersten Monaten des Krieges die Presse aller
                    neutralen Länder ihre Nachrichten – nicht nur über den Krieg selbst, sondern
                    auch über die \uline{inneren} Zustände \textcolor{pink}{Deutschlands}{}\ledrightnote{\textcolor{pink}{Deutschland}} und \textcolor{pink}{Oesterreich-Ungarns}{}\ledrightnote{\textcolor{pink}{Österreich-Ungarn}} in reicherem Mass von der Entente als von den
                    Zentralmächten bezog, sowie ich mich auch gedrängt fühlte \textcolor{blue}{Freunde}{}\ledrightnote{→\textcolor{blue}{Eugen Deimel}} in \textcolor{pink}{Amerika}{}\ledrightnote{\textcolor{pink}{Amerika}} in diesem Sinne nach Möglichkeit aufzuklären (was übrigens zur
                    Folge hatte, dass einer dieser \label{K_L02224_4v}\edtext{Privatbriefe}{\lemma{\textnormal{\emph{Privatbriefe}}}\Cendnote{\textnormal{Die ganze
                        Angelegenheit ausführlicher in \textcolor{blue}{Schnitzlers} Brief an \textcolor{blue}{Eugen
                            Deimel}, 25. 11. 1914 (Heinz P. Adamek: \emph{In die Neue Welt… Arthur Schnitzler – Eugen Deimel
                                Briefwechsel}. Wien: \emph{Holzhausen}{ }2003, S. 210–211).}}}\label{K_L02224_4h} ganz entstellt in ein \textcolor{brown}{New-Yorker Blatt}{}\ledrightnote{→\textcolor{brown}{New Yorker Staats-Zeitung}} und von dort
                    wieder \introOben{}–\introOben{} noch entstellter in deutsche Blätter
                    überging. Also ich denke wir wissen beide wie viel wir von dem zu halten haben,
                    was in den Zeitungen steht\substVorne{}\textsuperscript{.}\substDazwischen{}!\substHinten{}\introOben{})\introOben{}\pend
           \pstart
           Für heute nur so viel; mögen Ihnen die Feiertage lauter Gutes, insbesondere
                    völlige Genesung bringen und uns allen eine gegründetere Hoffnung auf die
                    baldige Wiederkehr schönerer Zeiten, als wir sie nach dem augenblicklichen Stand
                    der Dinge hegen dürfen.\pend
           \pstart
           Mit herzlichen Grüssen{\\[\baselineskip]}Ihr allezeit freundschaftlich ergebener{\\[\baselineskip]}\spacefill\mbox{{[}hs.:{]} Arthur Schnitzler}\pend
           \leftskip=0em{}\endnumbering\briefempfaengerindex{Brandes, Georg@\textsc{Brandes, Georg}!zzzSchnitzler, Arthur@\emph{von Arthur Schnitzler}!1915-12-221@{22. 12. 1915}|)be}\mylabel{h}  \normalsize

\doendnotes{C}
\bigskip
\vfill

\clearpage

\footnotesize

\lohead{\textsc{register}}

% Definiere theindex-Environment komplett neu ohne reledmac
\makeatletter
\renewenvironment{theindex}{%
  \section*{\indexname}%
  \setlength{\parindent}{0pt}%
  \setlength{\parskip}{0pt plus 0.3pt}%
  \let\item\@idxitem
}{%
  \clearpage
}
\makeatother

\IfFileExists{\jobname-pw.ind}{\input{\jobname-pw.ind}}{}

\end{document}

      