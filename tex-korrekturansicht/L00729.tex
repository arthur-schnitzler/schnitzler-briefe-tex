%% latex-korrekturansicht-vorspann.tex
%% Vorspann für die Korrekturansicht.
%% Lädt die gemeinsame Datei latex-vorspann.tex mit gesetztem Schalter.

\newif\ifkorrekturansicht
\korrekturansichttrue

\input{../tex-inputs/latex-vorspann}


               \section[Arthur Schnitzler an Richard Beer-Hofmann, 4. 10. 1897]{ Arthur Schnitzler an Richard Beer-Hofmann, 4. 10. 1897}\nopagebreak\mylabel{v}\rehead{ }\normalsize\beginnumbering\briefempfaengerindex{Beer-Hofmann, Richard@\textsc{Beer-Hofmann, Richard}!zzzSchnitzler, Arthur@\emph{von Arthur Schnitzler}!1897-10-041@{4. 10. 1897}|(be} \toendnotes[C]{\smallbreak\pagebreak[2]} \Standort{CUL, Schnitzler, B 8.1, S. 66.}
\physDesc{maschinelle Abschrift
\newline{}Schreibmaschine\newline{}Ordnung: von unbekannter Hand nummeriert: »105« }\buchAbdrucke{\weitereDrucke{Arthur Schnitzler, Richard Beer-Hofmann: \emph{Briefwechsel 1891–1931}. Hg. Konstanze Fliedl. Wien, Zürich: \emph{Europaverlag} 1992, S. 113.} }\toendnotes[C]{\smallbreak}\pstart
           \raggedleft{}{\pb}\textcolor{pink}{Wien}{}\ledrightnote{\textcolor{pink}{Wien}}, 4. 10. 97.\pend
           \pstart
           Lieber Richard, Sie teleph. mich immer an, wenn ich nicht zu Haus
               bin. Vormittag bin ich nämlich auf dem Land. Schaun Sie doch einmal Nachmittag bevor
               Sie nach \textcolor{pink}{Heiligst.}{}\ledrightnote{\textcolor{pink}{Heiligenstadt}} fahren, zu mir herauf. Ich möchte
               auch gern einmal mit Ihnen hinaus. \textcolor{blue}{Hugo}{}\ledrightnote{\textcolor{blue}{Hugo von Hofmannsthal}} schreibt
               mir, er kommt nächste Woche nach \textcolor{pink}{Wien}{}\ledrightnote{\textcolor{pink}{Wien}} und möchte
               Ihnen und mir viel vorlesen.\pend
           \pstart
           Herzlich Ihr \spacefill\mbox{Arthur.}\pend
           \pstart
           \noindent{}Ich arbeite sozusagen.\pend
           \pstart
           (\label{T_L00729_1v}\edtext{w. o.}{\lemma{\textnormal{\emph{w. o.}}}\Cendnote{\textnormal{»wie oben«: Verweis auf frühere Stelle der
                     Briefabschrift. Der Brief wurde in die \textcolor{pink}{Wollzeile 15} geschickt.}}}\label{T_L00729_1h})\pend
           \endnumbering\briefempfaengerindex{Beer-Hofmann, Richard@\textsc{Beer-Hofmann, Richard}!zzzSchnitzler, Arthur@\emph{von Arthur Schnitzler}!1897-10-041@{4. 10. 1897}|)be}\mylabel{h}  \normalsize

\doendnotes{C}
\bigskip
\vfill

\clearpage

\footnotesize

\lohead{\textsc{register}}

% Definiere theindex-Environment komplett neu ohne reledmac
\makeatletter
\renewenvironment{theindex}{%
  \section*{\indexname}%
  \setlength{\parindent}{0pt}%
  \setlength{\parskip}{0pt plus 0.3pt}%
  \let\item\@idxitem
}{%
  \clearpage
}
\makeatother

\IfFileExists{\jobname-pw.ind}{\input{\jobname-pw.ind}}{}

\end{document}

      