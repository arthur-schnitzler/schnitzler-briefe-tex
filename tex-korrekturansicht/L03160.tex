%% latex-korrekturansicht-vorspann.tex
%% Vorspann für die Korrekturansicht.
%% Lädt die gemeinsame Datei latex-vorspann.tex mit gesetztem Schalter.

\newif\ifkorrekturansicht
\korrekturansichttrue

\input{../tex-inputs/latex-vorspann}


\renewcommand{\erwaehntePersonen}{Personen: Richard Beer-Hofmann, Paul Goldmann, Elisabeth Kotter, Maria Charlotte Lamberg}
\renewcommand{\erwaehnteOrte}{Orte: Bad Ischl, Heringsdorf, München, Pustertal, Rügen, Salzburg, Wien}
\renewcommand{\erwaehnteWerke}{Werke: Der Tod Georgs, Die Münchener Kunstausstellungen. I. Im königl. Glaspalast, Die Münchener Kunstausstellungen. II. Im königl. Glaspalast, Freiwild. Schauspiel in 3 Akten, Münchener Brief. (Orig.-Corr. der »Wiener Allg. Ztg.«), Wiener Allgemeine Zeitung}
\section[ Felix Salten an Arthur Schnitzler, {[}30. 7. 1895{]}]{Felix Salten an Arthur Schnitzler, {[}30. 7. 1895{]}}
\nopagebreak\mylabel{v}
\rehead{ }\normalsize\beginnumbering\briefempfaengerindex{Schnitzler, Arthur@\textsc{Schnitzler, Arthur}!zzzSalten, Felix@\emph{von Felix Salten}!1895-07-302@{{[}30. 7. 1895{]}}|(be}
\toendnotes[C]{\smallbreak\pagebreak[2]}\Standort{CUL, Schnitzler, B 89, A 1.}
\physDesc{Brief, 1 Blatt, 4 Seiten, 1091 Zeichen
\newline{}Handschrift: Bleistift, lateinische Kurrent
\newline{}Schnitzler: mit Bleistift datiert: »30/7 95.« 
\newline{}Ordnung: mit Bleistift von unbekannter Hand nummeriert: »60« }\toendnotes[C]{\smallbreak}
\pstart
           \noindent{}{\pb}Lieber Freund! Dank für den Brief. Ich bin hier so auf
               mich allein gestellt, und durch alle die traurigen Agonie-Stimmungen die ich täglich
               mitmache, so herabgedrückt, dass ich es noch weit angenehmer empfinde, als Sie, wenn
               man mir {\pb}Briefe schreibt. Dass
                  \label{K_L03160-2v}\edtext{\textcolor{green}{Freiwild}{}\ledrightnote{\textcolor{green}{Freiwild. Schauspiel in 3 Akten}} fortschreitet}{\lemma{\textnormal{\emph{Freiwild fortschreitet}}}\Cendnote{\textnormal{Am 15. 6. 1895 hatte \textcolor{blue}{Schnitzler}
                  die Arbeit an \emph{\textcolor{green}{Freiwild}} wiederaufgenommen. Am
                     2. 8. 1895
                  stellte er den ersten \textcolor{green}{Akt}
                  fertig.}}}\label{K_L03160-2h} ist recht. Auch dem \label{K_L03160-3v}\edtext{\textcolor{green}{Götterliebling}{}\ledrightnote{\textcolor{green}{Der Tod Georgs}}}{\lemma{\textnormal{\emph{Götterliebling}}}\Cendnote{\textnormal{\textcolor{blue}{Richard Beer-Hofmann} arbeitete in dieser
                  Zeit intensiv an jener Erzählung, die später den Titel \emph{\textcolor{green}{Der Tod Georgs}} tragen sollte.}}}\label{K_L03160-3h} wär das schon sehr zu
               wünschen. Möchten doch beide \textcolor{green}{Sachen}{}\ledrightnote{{$\rightarrow$}\textcolor{green}{Der Tod Georgs}{\newline}{$\rightarrow$}\textcolor{green}{Freiwild. Schauspiel in 3 Akten}} bis zum Herbste fertig sein. \textcolor{pink}{Pusterthal}{}\ledrightnote{\textcolor{pink}{Pustertal}} wäre sehr schön, ob wir uns nicht aber
               doch lieber ruhig in \textcolor{pink}{Ischl}{}\ledrightnote{\textcolor{pink}{Bad Ischl}} aufhalten und in den
               gewissen behaglichen Parthien die \label{K_L03160-4v}\edtext{Gegend abfahren}{\lemma{\textnormal{\emph{Gegend abfahren}}}\Cendnote{\textnormal{Sie machten letztlich
                  eine Radtour von \textcolor{pink}{Salzburg} nach \textcolor{pink}{München}, siehe Felix Salten an Arthur Schnitzler, 22. 7. 1895.}}}\label{K_L03160-4h} wollen. Dann {\pb}noch Eins. Ich werde sehr
               gequält nach \textcolor{pink}{Rügen}{}\ledrightnote{\textcolor{pink}{Rügen}} zu fahren. \uline{\textcolor{blue}{E.}{}\ledrightnote{\textcolor{blue}{Elisabeth Kotter}}}, die in \textcolor{pink}{Heringsdorf}{}\ledrightnote{\textcolor{pink}{Heringsdorf}} ist, schreibt
               rührende Briefe. Vielleicht finde ich mich also dann doch bestimmt so gegen den 27. od. 28. August dahin
               zu reisen. Aber das wird sich \uline{ja} alles noch
               entscheiden, bis ich nach \textcolor{pink}{Ischl}{}\ledrightnote{\textcolor{pink}{Bad Ischl}}{ }{\pb}komme. Vorerst freue ich
               mich auf den Montag, oder Sonntag. Ich verständige Sie jedenfalls noch vorher. Für heute sende ich die gewünschten \label{K_L03160-5v}\edtext{\textcolor{green}{Feuilletons}{}\ledrightnote{{$\rightarrow$}\textcolor{green}{Die Münchener Kunstausstellungen. I. Im königl. Glaspalast}{\newline}{$\rightarrow$}\textcolor{green}{Die Münchener Kunstausstellungen. II. Im königl. Glaspalast}{\newline}{$\rightarrow$}\textcolor{green}{Münchener Brief. (Orig.-Corr. der »Wiener Allg. Ztg.«)}}}{\lemma{\textnormal{\emph{Feuilletons}}}\Cendnote{\textnormal{\textcolor{blue}{f. s.} [ = \textcolor{blue}{Felix Salten}]: \emph{\textcolor{green}{Münchener Brief. (Orig.-Corr. der »Wiener Allg. Ztg.«)}}. In: \emph{\textcolor{green}{Wiener Allgemeine Zeitung}}, Nr. 5.200, 6. 7. 1895, S. 8; \textcolor{blue}{Felix Salten}: \emph{\textcolor{green}{Die Münchener Kunstausstellungen. I. Im königl.
                        Glaspalast}}. In: \emph{\textcolor{green}{ebd.}}, Nr. 5.215,
                        24. 7. 1895, S. 2; \textcolor{blue}{ders.}: \emph{\textcolor{green}{Die Münchener Kunstausstellungen. II. Im königl. Glaspalast}}. In: \emph{\textcolor{green}{ebd.}}, Nr. 5.216, 25. 7. 1895, S. 2–3. Siehe Felix Salten an Arthur Schnitzler, 22. 7. 1895.}}}\label{K_L03160-5h}. Auch die für \textcolor{blue}{Goldmann}{}\ledrightnote{\textcolor{blue}{Paul Goldmann}} bestimmten, welche Sie absenden werden, falls es
               noch Zeit ist, ja?\pend
           
\pstart
           Also auf baldiges Wiedersehen, herzlichst {\\[\baselineskip]}Ihr \spacefill\mbox{Salten.}\pend
           \leftskip=0em{}\endnumbering\briefempfaengerindex{Schnitzler, Arthur@\textsc{Schnitzler, Arthur}!zzzSalten, Felix@\emph{von Felix Salten}!1895-07-302@{{[}30. 7. 1895{]}}|)be}\mylabel{h}  \normalsize

\doendnotes{C}
\bigskip
\vfill

\clearpage

\footnotesize

\lohead{\textsc{register}}

% Definiere theindex-Environment komplett neu ohne reledmac
\makeatletter
\renewenvironment{theindex}{%
  \section*{\indexname}%
  \setlength{\parindent}{0pt}%
  \setlength{\parskip}{0pt plus 0.3pt}%
  \let\item\@idxitem
}{%
  \clearpage
}
\makeatother

\IfFileExists{\jobname-pw.ind}{\input{\jobname-pw.ind}}{}

\end{document}

      