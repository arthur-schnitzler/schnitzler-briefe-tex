%% latex-korrekturansicht-vorspann.tex
%% Vorspann für die Korrekturansicht.
%% Lädt die gemeinsame Datei latex-vorspann.tex mit gesetztem Schalter.

\newif\ifkorrekturansicht
\korrekturansichttrue

\input{../tex-inputs/latex-vorspann}


               \section[Friedrich M. Fels an Arthur Schnitzler, {[}1. 1. 1893?{]}]{ Friedrich M. Fels an Arthur Schnitzler, {[}1. 1. 1893?{]}}\nopagebreak\mylabel{v}\rehead{ }\normalsize\beginnumbering\briefempfaengerindex{Schnitzler, Arthur@\textsc{Schnitzler, Arthur}!zzzFels, Friedrich Michael@\emph{von Friedrich Michael Fels}!1893-01-013@{{[}1. 1. 1893?{]}}|(be} \toendnotes[C]{\smallbreak\pagebreak[2]} \Standort{DLA, A:Schnitzler, HS.NZ85.1.2956.}
\physDesc{Brief, 1 Blatt, 1 Seite
\newline{}Handschrift: schwarze Tinte, lateinische Kurrent
\newline{}Schnitzler: mit Bleistift datiert: »93« und nummeriert: »6« }\toendnotes[C]{\smallbreak}\pstart
           \noindent{}{\pb}Lieber Doktor Arthur! Das \label{K_L00154_1v}\edtext{Verfehlen}{\lemma{\textnormal{\emph{Verfehlen}}}\Cendnote{\textnormal{Vgl. A. S.: \emph{Tagebuch}, 1. 1. 1893: »Bei
                                \textcolor{blue}{Fels}; verschlossene Thür. (Er
                            krank.)«. Möglicherweise ist dieses undatierte
                        Korrespondenzstück im Anschluss an dieses Ereignis verfasst.}}}\label{K_L00154_1h} heute
                    war mir sehr unangenehm; de{\geminationn} kaum waren Sie in der
                        \label{K_L00154_2v}\edtext{\textcolor{pink}{Reisnerstraſse}{}\ledrightnote{\textcolor{pink}{Reisnerstraße}}}{\lemma{\textnormal{\emph{Reisnerstraſse}}}\Cendnote{\textnormal{Hier befand sich die Redaktion
                        der \emph{\textcolor{brown}{Allgemeinen Kunst-Chronik}}.}}}\label{K_L00154_2h}, als
                    ich hin kam. So ko{\geminationn}te ich den eckelhalften Weg in
                    die \textcolor{pink}{Leopoldstadt}{}\ledrightnote{\textcolor{pink}{II., Leopoldstadt}} nicht verhindern. Natürlich
                    hatte ich gleich eine kleine Freude, als mir der \textcolor{blue}{Alte}{}\ledrightnote{→\textcolor{blue}{Wilhelm Lauser}} eröffnete, we{\geminationn} ich
                    noch ein paar Tage krank und arbeitsunfähig sei, er genötigt sei, die Stelle
                    aufzugeben. Also jetzt \uline{muſs} ich gesund sein.
                        We{\geminationn} ich ich nur eſsen kö{\geminationn}te? Große und wichtige Frage: darf ich baden?\pend
           \pstart
           Künftig werde ich, um bei meinen 70 fl zu bleiben, schon um zehn oder halb elf
                    aufs Bureau ko{\geminationm}en; Sie kö{\geminationn}en also zu früherer Zeit ko{\geminationm}en, vielleicht morgen?\pend
           \pstart
           Herzlichst{\\[\baselineskip]}\spacefill\mbox{Fels}\pend
           \leftskip=0em{}\pstart
           \noindent{}Das muſs ich kriegen: 1. Appetit, 2. die Möglichkeit zu gehen, ohne
                        umzufallen.\pend
           \endnumbering\briefempfaengerindex{Schnitzler, Arthur@\textsc{Schnitzler, Arthur}!zzzFels, Friedrich Michael@\emph{von Friedrich Michael Fels}!1893-01-013@{{[}1. 1. 1893?{]}}|)be}\mylabel{h}  \normalsize

\doendnotes{C}
\bigskip
\vfill

\clearpage

\footnotesize

\lohead{\textsc{register}}

% Definiere theindex-Environment komplett neu ohne reledmac
\makeatletter
\renewenvironment{theindex}{%
  \section*{\indexname}%
  \setlength{\parindent}{0pt}%
  \setlength{\parskip}{0pt plus 0.3pt}%
  \let\item\@idxitem
}{%
  \clearpage
}
\makeatother

\IfFileExists{\jobname-pw.ind}{\input{\jobname-pw.ind}}{}

\end{document}

      