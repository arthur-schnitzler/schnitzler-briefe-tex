%% latex-korrekturansicht-vorspann.tex
%% Vorspann für die Korrekturansicht.
%% Lädt die gemeinsame Datei latex-vorspann.tex mit gesetztem Schalter.

\newif\ifkorrekturansicht
\korrekturansichttrue

\input{../tex-inputs/latex-vorspann}


\section[Sigmund Freud an Arthur Schnitzler, 24. 5. 1926]{L03890 Sigmund Freud an Arthur Schnitzler, 24. 5. 1926}
\nopagebreak\mylabel{L03890v}
\rehead{ }\normalsize\beginnumbering\briefempfaengerindex{, @\textsc{, }!zzz, @\emph{von  }!1926-05-241@{24. 5. 1926}|(be}
\toendnotes[C]{\smallbreak\pagebreak[2]}\Standort{Washington, DC, Library of Congress, Freud Archives, C41F8.}
\physDesc{Brief, Fotokopie, 1 Blatt, 2 Seiten, 1077 Zeichen
\newline{}Handschrift: schwarze Tinte, deutsche Kurrent
\newline{}Zusatz: Der Verbleib des Originals ist ungeklärt. Zum Zeitpunkt der
                                 ersten Edition 1955 befand es sich im Besitz von \textcolor{blue}{Heinrich Schnitzler}\pwindex{Schnitzler, Heinrich 9.\,8.\,1902 Hinterbrühl – 12.\,7.\,1982 Wien@\textsc{Schnitzler, Heinrich} (9.\,8.\,1902 Hinterbrühl – 12.\,7.\,1982 Wien), \emph{Regisseur, Schauspieler}|pw}. }
\buchAbdrucke{\weitereDrucke{1) Sigmund Freud: \emph{Briefe an Arthur Schnitzler.} Herausgegeben von Henry Schnitzler. In: \emph{Neue deutsche Rundschau}, Jg. 66 (Januar 1955) Nr. 1, S. 99–100.} \weitereDrucke{2) Sigmund Freud: \emph{Sigmund Freud Edition. Digitale historisch-kritische
                        Gesamtausgabe}. Herausgegeben von Christine Diercks,  Arkadi Blatow und  Elisabeth Skale. (2014–2025) \url{https://www.freudedition.net/briefe/freud-sigmund/schnitzler-arthur/1926/05/24}.} }\toendnotes[C]{\smallbreak}
\pstart
           \raggedleft{}{\pb}24. 5. 26\pend
           
\pstart
           \textcolor{gray}{\textbf{PROF. D\textsuperscript{R.} FREUD}}\hfill \textcolor{gray}{\textbf{\textcolor{pink}{WIEN IX., BERGGASSE 19}\oindex{Wien@\textbf{Wien}!IX., Alsergrund@\textbf{IX., Alsergrund}!Berggasse 19@\textbf{Berggasse 19}, \emph{Wohngebäude}|pw}{}\ledrightnote{\textcolor{pink}{Berggasse 19}}. }}\pend
           
\pstart\center{}Verehrteſter!\pend\vspace{0.5em}
\pstart
           Ich weiß nicht, ob Sie ſchon zurück ſind. Wenn nicht, werden dieſe Zeilen des Dankes
               für Ihren \label{K_L03890-1v}\edtext{Gruß von der \textcolor{pink}{See}\oindex{Mittelmeer@\textbf{Mittelmeer}|pwv}\oindex{Atlantischer Ozean@\textbf{Atlantischer Ozean}|pwv}{}\ledrightnote{{$\rightarrow$}\emph{\textcolor{pink}{Mittelmeer}}{\newline}{$\rightarrow$}\emph{\textcolor{pink}{Atlantischer Ozean}}}}{\lemma{\textnormal{\emph{Gruß von der See}}}\Cendnote{\textnormal{Die Postkarte ist nicht erhalten. \textcolor{blue}{Schnitzler} unternahm gemeinsam mit seiner
                  Tochter \textcolor{blue}{Lili}\pwindex{Cappellini, Lili 13.\,9.\,1909 Wien – 26.\,7.\,1928 Venedig@\textsc{Cappellini, Lili} (13.\,9.\,1909 Wien – 26.\,7.\,1928 Venedig)|pwk} eine Schiffsreise durch das \textcolor{pink}{westliche Mittelmeer}\oindex{Mittelmeer@\textbf{Mittelmeer}|pwk} nach \textcolor{pink}{Las Palmas}\oindex{Las Palmas de Gran Canaria@\textbf{Las Palmas de Gran Canaria}|pwk} bis nach \textcolor{pink}{Hamburg}\oindex{Hamburg@\textbf{Hamburg}|pwk}. Sie begann mit dem Nachtzug am 15. 4. 1926 und ging am 19. 5. 1926 zu Ende, als
                  er in \textcolor{pink}{Berlin}\oindex{Berlin@\textbf{Berlin}, \emph{Hauptstadt}|pwk} den Nachtzug nach \textcolor{pink}{Wien}\oindex{Wien@\textbf{Wien}, \emph{Verwaltungsgebiet}|pwk} bestieg.}}}\label{K_L03890-1} Ihre Heimkehr erwarten.\pend
           
\pstart
           Das Ereignis gieng beſſer vorüber, als ich erwartet. Viel Herzlichkeit kein Miston,
               dank vor allem der aufrichtigen Enthaltung der offiziellen Kreiſe. (Zu denen ja die
               sozialiſtiſche \textcolor{pink}{Wien}\oindex{Wien@\textbf{Wien}, \emph{Verwaltungsgebiet}|pw}{}\ledrightnote{\textcolor{pink}{Wien}}er Kommune nicht zält). Die
               Juden haben ſich von allen Seiten und aller Orten mit Begeiſterung meiner Person
               bemächtigt, als ob ich ein gottesfürchtiger großer Rabbi wäre. Ich habe nichts
               dagegen, nachdem ich meine Stellung zum {\pb}Glauben
               unzweideutig \label{K_L03890-2v}\edtext{\textcolor{green}{klargelegt}\pwindex{Freud, Sigmund 6.\,5.\,1856 Pribor – 23.\,9.\,1939 London@\textsc{Freud, Sigmund} (6.\,5.\,1856 Pribor – 23.\,9.\,1939 London), \emph{Psychoanalytiker}!Zwangshandlungen und Religionsübungen@\strich\emph{Zwangshandlungen und Religionsübungen}|pwv}{}\ledrightnote{{$\rightarrow$}\emph{\textcolor{green}{Zwangshandlungen und Religionsübungen}}}}{\lemma{\textnormal{\emph{klargelegt}}}\Cendnote{\textnormal{Mehrere Schriften kommen in Frage – vor allem verfasste \textcolor{blue}{Freud}\pwindex{Freud, Sigmund 6.\,5.\,1856 Pribor – 23.\,9.\,1939 London@\textsc{Freud, Sigmund} (6.\,5.\,1856 Pribor – 23.\,9.\,1939 London), \emph{Psychoanalytiker}|pwk} noch einige weitere, in denen er über
                  die Rolle der Religion reflektierte. Er könnte sich hier auf seinen Aufsatz \emph{\textcolor{green}{Zwangshandlungen und Religionsübungen}\pwindex{Freud, Sigmund 6.\,5.\,1856 Pribor – 23.\,9.\,1939 London@\textsc{Freud, Sigmund} (6.\,5.\,1856 Pribor – 23.\,9.\,1939 London), \emph{Psychoanalytiker}!Zwangshandlungen und Religionsübungen@\strich\emph{Zwangshandlungen und Religionsübungen}|pwk}}
                  beziehen, der im Mai 1907 die erste Nummer der \emph{\textcolor{green}{Zeitschrift für Religionspsychologie}\pwindex{Zeitschrift für Religionspsychologie@\emph{Zeitschrift für Religionspsychologie}|pwk}} eröffnete
                     (Bd. 1, H. 1, S. 4–12). Insofern er auf Kenntnis
                  durch \textcolor{blue}{Schnitzler} setzt, dürfte er sich
                  vielleicht auf \emph{\textcolor{green}{Totem und Tabu}\pwindex{Freud, Sigmund 6.\,5.\,1856 Pribor – 23.\,9.\,1939 London@\textsc{Freud, Sigmund} (6.\,5.\,1856 Pribor – 23.\,9.\,1939 London), \emph{Psychoanalytiker}!Totem und Tabu@\strich\emph{Totem und Tabu}|pwk}} beziehen.
               }}}\label{K_L03890-2} habe. Das Judentum bedeutet mir noch ſehr viel affektiv.\pend
           
\pstart
           Mit dem 70ſten Geburtstag iſt doch ein Gefü{[}h{]}l großer Befreiung
               verbunden geweſen: Endlich hat man das Recht zu jenem Ausruf des \textcolor{green}{Steinklopferhanns}\pwindex{Anzengruber, Ludwig 29.\,11.\,1839 Wien – 10.\,12.\,1889 ebd.@\textsc{Anzengruber, Ludwig} (29.\,11.\,1839 Wien – 10.\,12.\,1889 ebd.), \emph{Schriftsteller}!Kreuzelschreiber@\strich\emph{Die Kreuzelschreiber}|pwv}{}\ledrightnote{{$\rightarrow$}\emph{\textcolor{green}{Die Kreuzelschreiber}}}: \label{K_L03890-3v}\edtext{\textcolor{green}{Es kann
                  der nix g’schehen}\pwindex{Anzengruber, Ludwig 29.\,11.\,1839 Wien – 10.\,12.\,1889 ebd.@\textsc{Anzengruber, Ludwig} (29.\,11.\,1839 Wien – 10.\,12.\,1889 ebd.), \emph{Schriftsteller}!Kreuzelschreiber@\strich\emph{Die Kreuzelschreiber}|pw}{}\ledrightnote{\textcolor{green}{Die Kreuzelschreiber}}}{\lemma{\textnormal{\emph{Es … g’schehen}}}\Cendnote{\textnormal{dialektal,
                  eigentlich: Es kann dir nix gschehn. Mehrfach wiederholter Ausspruch der Figur des
                     \textcolor{green}{Steinklopferhans}\pwindex{Anzengruber, Ludwig 29.\,11.\,1839 Wien – 10.\,12.\,1889 ebd.@\textsc{Anzengruber, Ludwig} (29.\,11.\,1839 Wien – 10.\,12.\,1889 ebd.), \emph{Schriftsteller}!Kreuzelschreiber@\strich\emph{Die Kreuzelschreiber}|pwkv} in der
                  Bauernkomödie \emph{\textcolor{green}{Die Kreuzelschreiber}\pwindex{Anzengruber, Ludwig 29.\,11.\,1839 Wien – 10.\,12.\,1889 ebd.@\textsc{Anzengruber, Ludwig} (29.\,11.\,1839 Wien – 10.\,12.\,1889 ebd.), \emph{Schriftsteller}!Kreuzelschreiber@\strich\emph{Die Kreuzelschreiber}|pwk}}
                     (1872) von \textcolor{blue}{Ludwig
                     Anzengruber}\pwindex{Anzengruber, Ludwig 29.\,11.\,1839 Wien – 10.\,12.\,1889 ebd.@\textsc{Anzengruber, Ludwig} (29.\,11.\,1839 Wien – 10.\,12.\,1889 ebd.), \emph{Schriftsteller}|pwk}, der zu einer verbreiteten Redewendung geworden war. Der hier
                  von \textcolor{blue}{Freud}\pwindex{Freud, Sigmund 6.\,5.\,1856 Pribor – 23.\,9.\,1939 London@\textsc{Freud, Sigmund} (6.\,5.\,1856 Pribor – 23.\,9.\,1939 London), \emph{Psychoanalytiker}|pwk} hergestellte Bezug zur
                  Sterblichkeit entspricht der ursprünglichen Verwendung im Stück.}}}\label{K_L03890-3}!
               Sonderbar, denn die Zal iſt doch nur eine Konvention.\pend
           
\pstart
           Am 15 Juni gehen \textcolor{blue}{wir}\pwindex{Bernays, Martha 26.\,7.\,1861 Hamburg – 2.\,11.\,1951 London@\textsc{Bernays, Martha} (26.\,7.\,1861 Hamburg – 2.\,11.\,1951 London)|pwv}{}\ledrightnote{{$\rightarrow$}\emph{\textcolor{blue}{Martha Bernays}}} auf den \textcolor{pink}{Semmering}\oindex{Semmering@\textbf{Semmering}, \emph{Verwaltungsgebiet}|pw}{}\ledrightnote{\textcolor{pink}{Semmering}}. Es ſoll doch
               nicht ein Vorrecht des Kranken bleiben, Sie öfter zu ſehen.\pend
           
\pstart
           In herzl Ergebenheit{\\[\baselineskip]} Ihr \spacefill\mbox{Freud}\pend
           \leftskip=0em{}
\pstart
           \noindent{}P. S. Über Ihre \textcolor{green}{Traumnovelle}\pwindex{Schnitzler, Arthur 15. 5. 1862 Wien – 21. 10. 1931 ebd.@\textsc{Schnitzler, Arthur} (15. 5. 1862 Wien – 21. 10. 1931 ebd.), \emph{Schriftsteller, Mediziner}!Traumnovelle@\strich\emph{Traumnovelle}|pw}{}\ledrightnote{\textcolor{green}{Traumnovelle}} habe ich mir
                  einige Gedanken gemacht.\pend
           \selectlanguage{ngerman}\endnumbering\briefempfaengerindex{, @\textsc{, }!zzz, @\emph{von  }!1926-05-241@{24. 5. 1926}|)be}\mylabel{L03890h}
\begin{anhang}
\end{anhang}\normalsize

\doendnotes{C}
\bigskip
\vfill

\clearpage

\footnotesize

\lohead{\textsc{register}}

% Definiere theindex-Environment komplett neu ohne reledmac
\makeatletter
\renewenvironment{theindex}{%
  \section*{\indexname}%
  \setlength{\parindent}{0pt}%
  \setlength{\parskip}{0pt plus 0.3pt}%
  \let\item\@idxitem
}{%
  \clearpage
}
\makeatother

\IfFileExists{\jobname-pw.ind}{\input{\jobname-pw.ind}}{}

\end{document}

      