%% latex-korrekturansicht-vorspann.tex
%% Vorspann für die Korrekturansicht.
%% Lädt die gemeinsame Datei latex-vorspann.tex mit gesetztem Schalter.

\newif\ifkorrekturansicht
\korrekturansichttrue

\input{../tex-inputs/latex-vorspann}


\renewcommand{\erwaehntePersonen}{Personen: Franz Defregger, Josef Putzenbacher, Gustav Schwarzkopf, Emil Schwarzkopf, Max Schwarzkopf, Rudolf Schwarzkopf}
\renewcommand{\erwaehnteOrte}{Orte: Cortina d'Ampezzo, Dölsach, Lienz, Pfarrkirche Dölsach, Polen, Putzenbacher, Toblach, Wien}
\renewcommand{\erwaehnteWerke}{Werke: Heilige Familie}
\section[ Felix Salten an Arthur Schnitzler, 12. 8. 1893]{Felix Salten an Arthur Schnitzler, 12. 8. 1893}
\nopagebreak\mylabel{v}
\rehead{ }\normalsize\beginnumbering\briefempfaengerindex{Schnitzler, Arthur@\textsc{Schnitzler, Arthur}!zzzSalten, Felix@\emph{von Felix Salten}!1893-08-125@{12. 8. 1893}|(be}
\toendnotes[C]{\smallbreak\pagebreak[2]}\Standort{CUL, Schnitzler, B 89, A 1.}
\physDesc{Brief, 1 Blatt, 2 Seiten, 602 Zeichen
\newline{}Handschrift: Bleistift, lateinische Kurrent
\newline{}Ordnung: mit Bleistift von unbekannter Hand nummeriert: »29« }\toendnotes[C]{\smallbreak}
\pstart
           \noindent{}{\pb}Lieber Freund!{ }\textcolor{pink}{Hier}{}\ledrightnote{{$\rightarrow$}\textcolor{pink}{Dölsach}} ist es einfach herrlich.
                  Gestern mit Rad und Hund in \textcolor{pink}{\strikeout{Dölsach}}{}\ledrightnote{\textcolor{pink}{Dölsach}}{ }\textcolor{pink}{\substVorne{}\textsuperscript{g}\substDazwischen{}L\substHinten{}ienz}{}\ledrightnote{\textcolor{pink}{Lienz}} gewesen, und dort eine Einladung zu einem Radfahrfeste erhalten.
               Im Coupé mit einem \textcolor{pink}{poln}{}\ledrightnote{\textcolor{pink}{Polen}}ischen Juden über’s –
               Bicycle gesprochen. Nächste Woche fahre ich per Bahn nach \textcolor{pink}{Toblach}{}\ledrightnote{\textcolor{pink}{Toblach}}, von da nach \textcolor{pink}{Cortina}{}\ledrightnote{\textcolor{pink}{Cortina d'Ampezzo}}. Dann berichte ich über Alles. Hier in der kleinen \textcolor{pink}{Dorfkirche}{}\ledrightnote{{$\rightarrow$}\textcolor{pink}{Pfarrkirche Dölsach}} ist das Original von \textcolor{blue}{Defregger}{}\ledrightnote{\textcolor{blue}{Franz Defregger}}’s \textcolor{green}{Madonna}{}\ledrightnote{\textcolor{green}{Heilige Familie}}, und viele Jugendskizzen, wie Portraits von ihm zeigt der \label{K_L03126-1v}\edtext{\textcolor{blue}{Wirt}{}\ledrightnote{{$\rightarrow$}\textcolor{blue}{Josef Putzenbacher}} in
               seiner \textcolor{pink}{Stube}{}\ledrightnote{{$\rightarrow$}\textcolor{pink}{Putzenbacher}}}{\lemma{\textnormal{\emph{Wirt in
               seiner Stube}}}\Cendnote{\textnormal{\textcolor{blue}{Josef \textcolor{pink}{Putzenbacher}}?}}}\label{K_L03126-1h}. Wenn Sie schreiben, dann {\pb}bitte \textcolor{pink}{Dölsach}{}\ledrightnote{\textcolor{pink}{Dölsach}}{ }\textsuperscript{b}/\textcolor{pink}{Lienz}{}\ledrightnote{\textcolor{pink}{Lienz}}, poste
               restante.\pend
           
\pstart
           Grüßen Sie \textcolor{blue}{Schwarzkopf’s}{}\ledrightnote{\textcolor{blue}{Gustav Schwarzkopf}{\newline}\textcolor{blue}{Emil Schwarzkopf}{\newline}\textcolor{blue}{Max Schwarzkopf}{\newline}\textcolor{blue}{Rudolf Schwarzkopf}} und seien Sie herzlich gegrüßt\pend
           
\pstart
           Ihr treuer {\\[\baselineskip]}\spacefill\mbox{Salten}\pend
           \leftskip=0em{}
\pstart
           \noindent{}\textcolor{pink}{Dölsach}{}\ledrightnote{\textcolor{pink}{Dölsach}}, 12 Aug. 93.\pend
           \endnumbering\briefempfaengerindex{Schnitzler, Arthur@\textsc{Schnitzler, Arthur}!zzzSalten, Felix@\emph{von Felix Salten}!1893-08-125@{12. 8. 1893}|)be}\mylabel{h}  \normalsize

\doendnotes{C}
\bigskip
\vfill

\clearpage

\footnotesize

\lohead{\textsc{register}}

% Definiere theindex-Environment komplett neu ohne reledmac
\makeatletter
\renewenvironment{theindex}{%
  \section*{\indexname}%
  \setlength{\parindent}{0pt}%
  \setlength{\parskip}{0pt plus 0.3pt}%
  \let\item\@idxitem
}{%
  \clearpage
}
\makeatother

\IfFileExists{\jobname-pw.ind}{\input{\jobname-pw.ind}}{}

\end{document}

      