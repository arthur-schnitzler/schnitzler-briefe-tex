%% latex-korrekturansicht-vorspann.tex
%% Vorspann für die Korrekturansicht.
%% Lädt die gemeinsame Datei latex-vorspann.tex mit gesetztem Schalter.

\newif\ifkorrekturansicht
\korrekturansichttrue

\input{../tex-inputs/latex-vorspann}


\renewcommand{\erwaehntePersonen}{Personen: Felix Salten}
\renewcommand{\erwaehnteInstitutionen}{Institutionen: S. Fischer Verlag}
\renewcommand{\erwaehnteOrte}{Orte: Berlin}
\renewcommand{\erwaehnteWerke}{Werke: Börsenblatt für den Deutschen Buchhandel, Der Ruf des Lebens. Schauspiel in drei Akten}
\section[Arthur Schnitzler: Widmungsexemplar Der Ruf des Lebens für Felix Salten, {[}zwischen 4.–7.?{]} 2. 1906]{Arthur Schnitzler: Widmungsexemplar Der Ruf des Lebens für Felix
               Salten, {[}zwischen 4.–7.?{]} 2. 1906}
\nopagebreak\mylabel{v}
\rehead{ }\normalsize\beginnumbering\briefempfaengerindex{Salten, Felix@\textsc{Salten, Felix}!zzzSchnitzler, Arthur@\emph{von Arthur Schnitzler}!2@{{[}zwischen
                  4.–7.?{]} 2. 1906}|(be}
\toendnotes[C]{\smallbreak\pagebreak[2]}\Standort{Wienbibliothek im Rathaus, A-124219/2.Ex., DS-2019-711.}
\physDesc{Widmung am Vorsatzblatt, 55 Zeichen
\newline{}Handschrift: schwarze Tinte, deutsche Kurrent
\newline{}Ordnung: mit schwarzer Tinte ausgefüllter Stempel: »\noindent{}\textcolor{gray}{\textbf{\textit{Felix Salten}}}{ / }\textcolor{gray}{\textbf{\textit{Inv. Nr.}}}{ }4478{ / }\textcolor{gray}{\textbf{\textit{Werk Nr.}}}{ }2206{ / }\textcolor{gray}{\textbf{\textit{Schrank}}}{ }XIV A. Z. \textcolor{gray}{\textbf{\textit{Fach}}} b« }\toendnotes[C]{\smallbreak}
\pstart
           \noindent{}{\pb}Meinem lieben Felix Salten\pend
           \pstart \spacefill\mbox{Arthur Sch}\pend{}
\pstart
           \textcolor{pink}{Berlin}{}\ledrightnote{\textcolor{pink}{Berlin}}{ }\label{K_L03607-1v}\edtext{Feber 906}{\lemma{\textnormal{\emph{Feber 906}}}\Cendnote{\textnormal{\emph{\textcolor{green}{Der Ruf des Lebens}} wurde am 13. 2. 1906 vom \emph{\textcolor{green}{Börsenblatt für den deutschen Buchhandel}} als Neuerscheinung gemeldet.
                        \textcolor{blue}{Schnitzler} war von 4. 2. 1906 bis
                        7. 2. 1906
                     und von 18. 2. 1906 bis 27. 2. 1906 in \textcolor{pink}{Berlin}. Es ist anzunehmen, dass er
                     die Widmungen während der ersten Reise verfasste.}}}\label{K_L03607-1h}.\pend
           {\bigskip}
\pstart
           \noindent{}\centering{}{\pb}\textcolor{gray}{\textbf{\textbf{\textcolor{green}{\so{Der Ruf des Lebens}}{}\ledrightnote{\textcolor{green}{Der Ruf des Lebens. Schauspiel in drei Akten}}}}}\pend
           
\pstart
           \noindent{}\centering{}\textcolor{gray}{\textbf{\so{Schauſpiel in drei Akten von}}}{\\}\textcolor{gray}{\textbf{\textbf{\so{Arthur Schnitzler}}}}\pend
           {\bigskip}
\pstart
           \noindent{}\centering{}\textcolor{gray}{\textbf{\textcolor{brown}{\so{S. Fiſcher, Verlag}}{}\ledrightnote{\textcolor{brown}{S. Fischer Verlag}}\so{,{ }}\textcolor{pink}{\so{Berlin}}{}\ledrightnote{\textcolor{pink}{Berlin}}}}\pend
           
\pstart
           \noindent{}\centering{}\textcolor{gray}{\textbf{\so{1906}}}\pend
           \endnumbering\briefempfaengerindex{Salten, Felix@\textsc{Salten, Felix}!zzzSchnitzler, Arthur@\emph{von Arthur Schnitzler}!1906-02-042@{{[}zwischen
                  4.–7.?{]} 2. 1906}|)be}\mylabel{h}  \normalsize

\doendnotes{C}
\bigskip
\vfill

\clearpage

\footnotesize

\lohead{\textsc{register}}

% Definiere theindex-Environment komplett neu ohne reledmac
\makeatletter
\renewenvironment{theindex}{%
  \section*{\indexname}%
  \setlength{\parindent}{0pt}%
  \setlength{\parskip}{0pt plus 0.3pt}%
  \let\item\@idxitem
}{%
  \clearpage
}
\makeatother

\IfFileExists{\jobname-pw.ind}{\input{\jobname-pw.ind}}{}

\end{document}

      