%% latex-korrekturansicht-vorspann.tex
%% Vorspann für die Korrekturansicht.
%% Lädt die gemeinsame Datei latex-vorspann.tex mit gesetztem Schalter.

\newif\ifkorrekturansicht
\korrekturansichttrue

\input{../tex-inputs/latex-vorspann}


\renewcommand{\erwaehntePersonen}{Personen: Richard Beer-Hofmann, Robert Hirschfeld, Alfred Kerr, Olga Schnitzler, Elisabeth Steinrück, Oscar Tietz}
\renewcommand{\erwaehnteInstitutionen}{Institutionen: Hermann Tietz und Co., Neue Freie Presse}
\renewcommand{\erwaehnteOrte}{Orte: Berlin, Breslau, Dessauer Straße, Dolomiten, Lago di Garda, Sekirn, Trient, Vahrn, Val Gardena, Wien, Wörthersee}
\renewcommand{\erwaehnteWerke}{Werke: Baedeker-Reiseführer, Insolvenz des großen Berliner Waarenhauses Tietz, Neue Freie Presse}
\section[ Paul Goldmann an Arthur Schnitzler, 19. 7. {[}1901{]}]{Paul Goldmann an Arthur Schnitzler, 19. 7. {[}1901{]}}
\nopagebreak\mylabel{v}
\rehead{ }\normalsize\beginnumbering\briefempfaengerindex{Schnitzler, Arthur@\textsc{Schnitzler, Arthur}!zzzGoldmann, Paul@\emph{von Paul Goldmann}!1901-07-191@{19. 7. {[}1901{]}}|(be}
\toendnotes[C]{\smallbreak\pagebreak[2]}\Standort{DLA, A:Schnitzler, HS.NZ85.1.3171.}
\physDesc{Brief, 2 Blätter, 6 Seiten
\newline{}Handschrift: blaue Tinte, deutsche Kurrent
\newline{}Schnitzler: 1) mit Bleistift das Jahr »1901« vermerkt  2) mit rotem Buntstift fünf Unterstreichungen}\toendnotes[C]{\smallbreak}
\pstart
           \noindent{}\raggedleft{}{\pb}\textcolor{pink}{\textcolor{gray}{\textbf{DESSAUERSTRASSE 19}}}{}\ledrightnote{\textcolor{pink}{Dessauer Straße}}\pend
           
\pstart
           \textcolor{pink}{Berlin}{}\ledrightnote{\textcolor{pink}{Berlin}}, 19. Juli.\pend
           
\pstart\center{}Mein lieber Freund,\pend
\pstart
           Ich wollte morgen fahren, aber dieſe verfluchte \textcolor{brown}{Bande}{}\ledrightnote{{$\rightarrow$}\textcolor{brown}{Neue Freie Presse}} (die \textcolor{brown}{Redaktion}{}\ledrightnote{{$\rightarrow$}\textcolor{brown}{Neue Freie Presse}}) läßt mich nicht fort. Ich führe
               hier Ausgleichs-Verhandlungen mit dem \textcolor{blue}{Beſitzer}{}\ledrightnote{{$\rightarrow$}\textcolor{blue}{Oscar Tietz}} des großen Waarenhauſes \strikeout{T\textcolor{gray}{ie}}{ }\textsc{\textcolor{brown}{Tietz}{}\ledrightnote{\textcolor{brown}{Hermann Tietz und Co.}}}, deſſen Inſolvenz die \textcolor{brown}{N. Fr. Pr.}{}\ledrightnote{\textcolor{brown}{Neue Freie Presse}} fälſchlich
                  \label{K_L03073-1v}\edtext{\textcolor{green}{gemeldet}{}\ledrightnote{{$\rightarrow$}\textcolor{green}{Insolvenz des großen Berliner Waarenhauses Tietz}}}{\lemma{\textnormal{\emph{gemeldet}}}\Cendnote{\textnormal{[O. V.:] \emph{\textcolor{green}{Insolvenz des großen Berliner
                        Waarenhauses Tietz}}. In: \emph{\textcolor{green}{Neue Freie
                        Presse}}, Nr. 13239, 5. 7. 1901,
                     Abendblatt, S. 3.}}}\label{K_L03073-1h} und der das \textcolor{brown}{Blatt}{}\ledrightnote{{$\rightarrow$}\textcolor{brown}{Neue Freie Presse}} klagen will. (Das ſage ich Dir im Vertrauen).
               Nach 14 tägigen Verhandlungen habe ich den Ausgleich hier endlich zuſtande gebracht.
               Da macht auf einmal die \textcolor{brown}{N. Fr. Pr.}{}\ledrightnote{\textcolor{brown}{Neue Freie Presse}} neue
               Schwierigkeiten, und Alles iſt {\pb}wieder in Frage geſtellt.\pend
           
\pstart
           Vielleicht kann ich doch wenigſtens Montag (22. Juli)
               fahren. Dann bleibe ich einen Tag in \textcolor{pink}{Breslau}{}\ledrightnote{\textcolor{pink}{Breslau}},
               zwei oder drei Tage in \textsc{\textcolor{pink}{Wien}{}\ledrightnote{\textcolor{pink}{Wien}}}, gehe hierauf an den \textcolor{pink}{Wörtherſee}{}\ledrightnote{\textcolor{pink}{Wörthersee}} zu \textsc{\textcolor{blue}{Hirschfeld}{}\ledrightnote{\textcolor{blue}{Robert Hirschfeld}}} und werde irgendwo dort wohnen. Das Beſte alſo iſt, Du ſendeſt mir weitere
               Nachricht an die Adreſſe von \textsc{\textcolor{blue}{Hirschfeld}{}\ledrightnote{\textcolor{blue}{Robert Hirschfeld}}} in \textsc{\textcolor{pink}{Seekirn}{}\ledrightnote{\textcolor{pink}{Sekirn}}}. Ich möchte am \textcolor{pink}{Wörtherſee}{}\ledrightnote{\textcolor{pink}{Wörthersee}} nicht allzulange
               bleiben. \textsc{\textcolor{blue}{Richard}{}\ledrightnote{\textcolor{blue}{Richard Beer-Hofmann}}}, der mir während des ganzen Jahres kein Wort geſchrieben und auch jetzt ſich
               nicht einmal zu einer Zeile {\pb}aufgeſchwungen hat, in der er den Wunſch ausſpricht, mich zu ſehen, werde ich
               wahrſcheinlich überhaupt nicht \label{K_L03073-43v}\edtext{aufſuchen}{\lemma{\textnormal{\emph{aufſuchen}}}\Cendnote{\textnormal{siehe Paul Goldmann an Arthur Schnitzler, 29. 7. [1901]}}}\label{K_L03073-43h}.\pend
           
\pstart
           Mir liegt nun daran, in Ruhe irgendwo \uline{möglichſt hoch}
               ein paar Wochen zu verbringen, am Liebſten in den \textcolor{pink}{Dolomiten}{}\ledrightnote{\textcolor{pink}{Dolomiten}}, wenn das \textcolor{pink}{Grödner Thal}{}\ledrightnote{\textcolor{pink}{Val Gardena}} zu
               ſonnig iſt. Die Idee, den Schluß am \textcolor{pink}{Gardaſee}{}\ledrightnote{\textcolor{pink}{Lago di Garda}} zu
               machen, finde ich entzückend. Den Ort, wo wir bis dahin bleiben wollen, magſt Du
                  \label{K_L03073-2v}\edtext{beſtimmen}{\lemma{\textnormal{\emph{beſtimmen}}}\Cendnote{\textnormal{siehe Paul Goldmann an Arthur Schnitzler, 26. 4. [1901]}}}\label{K_L03073-2h}. Nur bitte ich Dich, dabei auch ein klein wenig meine Wünſche zu
               berückſichtigen. So ſehr {\pb}es mir auch
               zur Befriedigung gereichen würde, an einem Orte mich aufzuhalten, wo Du Dich wohl
               befindeſt, ſo wäre es mir doch nicht \strikeout{\textcolor{gray}{×}} unangenehm, wenn an dieſem Orte auch ich mich wohlbefinden könnte. Ich
               brauche, was ein Menſch mit völlig zerrütteten Nerven braucht: Ruhe, Höhenluft,
               Kühle. Und in landſchaftlicher Beziehung habe ich, wie geſagt, ein großes \strikeout{Verlan} Verlangen nach einer \textcolor{pink}{Dolomiten}{}\ledrightnote{\textcolor{pink}{Dolomiten}}-Gegend\substVorne{}\textsuperscript{.}\substDazwischen{} (\substHinten{}vielleicht bei \textcolor{pink}{Trient}{}\ledrightnote{\textcolor{pink}{Trient}}). Aber ich möchte,
               daß dies Alles ſchon vor meiner Ankunft {\pb}feſtgeſetzt wäre. Denn ich möchte nicht wieder, wie im vorigen Jahre, dreiviertel meines Urlaubs mit dem Studium von \textsc{\textcolor{green}{Bädekers}{}\ledrightnote{\textcolor{green}{Baedeker-Reiseführer}}} und Eiſenbahn-Fahrplänen verbringen.\pend
           
\pstart
           \textsc{\textcolor{blue}{Kerr}{}\ledrightnote{\textcolor{blue}{Alfred Kerr}}} kann hier erſt gegen Mitte Auguſt fort. Er will
               dann \label{K_L03073-3v}\edtext{zu uns ſtoßen}{\lemma{\textnormal{\emph{zu uns ſtoßen}}}\Cendnote{\textnormal{nicht geschehen}}}\label{K_L03073-3h} und möchte gern, daß
               wir womöglich eine mehrtägige gemeinſame Fußwanderung im Gebirge machten. Auch \textsc{\textcolor{blue}{Hirschfeld}{}\ledrightnote{\textcolor{blue}{Robert Hirschfeld}}}{ }{\pb}werde ich dazu animiren, bei einer
               ſolchen Parthie \label{K_L03073-5v}\edtext{mitzuhalten}{\lemma{\textnormal{\emph{mitzuhalten}}}\Cendnote{\textnormal{nicht geschehen}}}\label{K_L03073-5h}.\pend
           
\pstart
           Schreib’ mir alſo nach \textsc{\textcolor{pink}{Seekirn}{}\ledrightnote{\textcolor{pink}{Sekirn}}} an \textsc{\textcolor{blue}{Hirschfeld}{}\ledrightnote{\textcolor{blue}{Robert Hirschfeld}}s} Adreſſe. Viele treue Grüße
               Dir und den beiden lieblichen \textcolor{blue}{Schweſtern}{}\ledrightnote{{$\rightarrow$}\textcolor{blue}{Olga Schnitzler}{\newline}{$\rightarrow$}\textcolor{blue}{Elisabeth Steinrück}}! {\\[\baselineskip]}Dein {\\[\baselineskip]}\spacefill\mbox{Paul Goldmann.}\pend
           \leftskip=0em{}
\pstart
           \noindent{}Wie lange ich bei Euch bleibe? Je nachdem Ihr Euch zu mir
                     be\textcolor{gray}{rufe}t: ſehr lange oder ſehr kurz.\pend
           \endnumbering\briefempfaengerindex{Schnitzler, Arthur@\textsc{Schnitzler, Arthur}!zzzGoldmann, Paul@\emph{von Paul Goldmann}!1901-07-191@{19. 7. {[}1901{]}}|)be}\mylabel{h}
\begin{anhang}
\end{anhang}\normalsize

\doendnotes{C}
\bigskip
\vfill

\clearpage

\footnotesize

\lohead{\textsc{register}}

% Definiere theindex-Environment komplett neu ohne reledmac
\makeatletter
\renewenvironment{theindex}{%
  \section*{\indexname}%
  \setlength{\parindent}{0pt}%
  \setlength{\parskip}{0pt plus 0.3pt}%
  \let\item\@idxitem
}{%
  \clearpage
}
\makeatother

\IfFileExists{\jobname-pw.ind}{\input{\jobname-pw.ind}}{}

\end{document}

      