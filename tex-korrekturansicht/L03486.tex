%% latex-korrekturansicht-vorspann.tex
%% Vorspann für die Korrekturansicht.
%% Lädt die gemeinsame Datei latex-vorspann.tex mit gesetztem Schalter.

\newif\ifkorrekturansicht
\korrekturansichttrue

\input{../tex-inputs/latex-vorspann}


\renewcommand{\erwaehntePersonen}{Personen: Richard Beer-Hofmann, Josef Kainz, Felix Salten, Felix Speidel, Else Speidel-Haeberle}
\renewcommand{\erwaehnteOrte}{Orte: Volkstheater, Wien}
\renewcommand{\erwaehnteWerke}{Werke: Vom andern Ufer. Einakter}
\section[ Felix Salten an Arthur Schnitzler, {[}6. 11.? 1907{]}]{Felix Salten an Arthur Schnitzler, {[}6. 11.? 1907{]}}
\nopagebreak\mylabel{v}
\rehead{ }\normalsize\beginnumbering\briefempfaengerindex{Schnitzler, Arthur@\textsc{Schnitzler, Arthur}!zzzSalten, Felix@\emph{von Felix Salten}!1907-11-061@{{[}6. 11.? 1907{]}}|(be}
\toendnotes[C]{\smallbreak\pagebreak[2]}\Standort{CUL, Schnitzler, B 89, B 1.}
\physDesc{Brief, 1 Blatt, 1 Seite, 382 Zeichen
\newline{}Handschrift: schwarze Tinte, lateinische Kurrent
\newline{}Schnitzler: mit Bleistift datiert: »März 07\textcolor{gray}{?}« 
\newline{}Ordnung: mit Bleistift von unbekannter Hand nummeriert: »229« }\toendnotes[C]{\smallbreak}
\pstart
           \raggedleft{}{\pb}\label{K_L03486-1v}\edtext{Mittwoch}{\lemma{\textnormal{\emph{Mittwoch}}}\Cendnote{\textnormal{Die Datierung
                     gelingt über den Umweg, dass der an einem Donnerstag geschriebene Brief vom
                        7. 11. 907 auf alle
                     die im vorliegenden Brief angeschnittenen Themen antwortet. Entsprechend ist
                     dieses Korrespondenzstück am Vortag zu datieren.}}}\label{K_L03486-1h}\pend
           
\pstart{}Lieber,\pend
\pstart
           vielleicht können wir \label{K_L03486-2v}\edtext{Samstag{ }nach dem Theater}{\lemma{\textnormal{\emph{Samstag nach dem Theater}}}\Cendnote{\textnormal{Die Premiere von \textcolor{blue}{Salten}s Stück \emph{\textcolor{green}{Vom andern Ufer}} fand 9. 11. 1907 am \textcolor{pink}{Volkstheater} statt. \textcolor{blue}{Schnitzler} nahm teil.}}}\label{K_L03486-2h} beisammen sein? Mir ist es
               ganz egal wo; ich möchte nur irgendwo hin gehen, wo wenig Leute sind. Wenn Sie \textcolor{blue}{Richard}{}\ledrightnote{\textcolor{blue}{Richard Beer-Hofmann}} sehen, bitte, sagen Sie es ihm auch.
               Ich höre, dass Herr \textcolor{blue}{Kainz}{}\ledrightnote{\textcolor{blue}{Josef Kainz}} ins Theater geht;
               natürlich wär es mir angenehm, wenn er mit käme. Auch \textcolor{blue}{Speidels}{}\ledrightnote{\textcolor{blue}{Felix Speidel}{\newline}\textcolor{blue}{Else Speidel-Haeberle}} werden dann wol mit uns sein. Bitte um eine
               Zeile.\pend
           
\pstart
           Herzlichst Ihr {\\[\baselineskip]}\spacefill\mbox{Salten}\pend
           \leftskip=0em{}\endnumbering\briefempfaengerindex{Schnitzler, Arthur@\textsc{Schnitzler, Arthur}!zzzSalten, Felix@\emph{von Felix Salten}!1907-11-061@{{[}6. 11.? 1907{]}}|)be}\mylabel{h}  \normalsize

\doendnotes{C}
\bigskip
\vfill

\clearpage

\footnotesize

\lohead{\textsc{register}}

% Definiere theindex-Environment komplett neu ohne reledmac
\makeatletter
\renewenvironment{theindex}{%
  \section*{\indexname}%
  \setlength{\parindent}{0pt}%
  \setlength{\parskip}{0pt plus 0.3pt}%
  \let\item\@idxitem
}{%
  \clearpage
}
\makeatother

\IfFileExists{\jobname-pw.ind}{\input{\jobname-pw.ind}}{}

\end{document}

      