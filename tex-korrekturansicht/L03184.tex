%% latex-korrekturansicht-vorspann.tex
%% Vorspann für die Korrekturansicht.
%% Lädt die gemeinsame Datei latex-vorspann.tex mit gesetztem Schalter.

\newif\ifkorrekturansicht
\korrekturansichttrue

\input{../tex-inputs/latex-vorspann}


\renewcommand{\erwaehntePersonen}{Personen: Richard Beer-Hofmann, Hugo von Hofmannsthal, Johann Schnitzler}
\renewcommand{\erwaehnteOrte}{Orte: Burgring, Ordination Dr. Arthur Schnitzler, Wien}
\renewcommand{\erwaehnteWerke}{Werke: Neue Freie Presse}
\section[Felix Salten an Arthur Schnitzler, {[}April 1892{]}]{Felix Salten an Arthur Schnitzler, {[}April 1892{]}}
\nopagebreak\mylabel{v}
\rehead{ }\normalsize\beginnumbering\briefempfaengerindex{Schnitzler, Arthur@\textsc{Schnitzler, Arthur}!zzzSalten, Felix@\emph{von Felix Salten}!1892-04-012@{{[}April 1892{]}}|(be}
\toendnotes[C]{\smallbreak\pagebreak[2]}\Standort{CUL, Schnitzler, B 89, A 1.}
\physDesc{Brief, 1 Blatt, 2 Seiten, 430 Zeichen
\newline{}Handschrift: Bleistift, lateinische Kurrent
\newline{}Schnitzler: mit Bleistift datiert: »April 92« 
\newline{}Ordnung: mit Bleistift von unbekannter Hand nummeriert: »10« }\toendnotes[C]{\smallbreak}
\pstart
           \noindent{}{\pb}Lieber Arthur! Ich ging vorbei, vergaß natürlich, dass
               Sie \label{K_L03184-1v}\edtext{\textcolor{pink}{Burgring 1}{}\ledrightnote{\textcolor{pink}{Burgring}} ordiniren}{\lemma{\textnormal{\emph{Burgring 1 ordiniren}}}\Cendnote{\textnormal{Wohnsitz der Familie und nach dem Tod des \textcolor{blue}{Vater}s die \textcolor{pink}{Adresse} von \textcolor{blue}{Schnitzler}s erster eigener \textcolor{pink}{Praxis}}}}\label{K_L03184-1h}. Ihre Handschuhe brachte ich zurück, u. sagen wollte ich Ihnen, dass ich
                  Abends wahrscheinlich komme, doch erst gegen 11 Uhr.
               Jetzt bin ich müde und ruhe mich {\pb}ein wenig aus und lese die
                  \textcolor{green}{Neue fr Pr.}{}\ledrightnote{\textcolor{green}{Neue Freie Presse}} u. bilde mir ein, ich »bin mein
               mich innig liebender{[}«{]}\pend
           
\pstart
           \centering{}Arthur Schnitzler.\pend
           
\pstart
           \noindent{}Habe heute gearbeitet{[},{]} aber wenig, gehe jetzt nach
               Hause, wieder arbeiten.\pend
           
\pstart
           \textcolor{blue}{Loris}{}\ledrightnote{\textcolor{blue}{Hugo von Hofmannsthal}}, \textcolor{blue}{Beer
                  Hofmann}{}\ledrightnote{\textcolor{blue}{Richard Beer-Hofmann}}?\pend
           \endnumbering\briefempfaengerindex{Schnitzler, Arthur@\textsc{Schnitzler, Arthur}!zzzSalten, Felix@\emph{von Felix Salten}!1892-04-012@{{[}April 1892{]}}|)be}\mylabel{h}  \normalsize

\doendnotes{C}
\bigskip
\vfill

\clearpage

\footnotesize

\lohead{\textsc{register}}

% Definiere theindex-Environment komplett neu ohne reledmac
\makeatletter
\renewenvironment{theindex}{%
  \section*{\indexname}%
  \setlength{\parindent}{0pt}%
  \setlength{\parskip}{0pt plus 0.3pt}%
  \let\item\@idxitem
}{%
  \clearpage
}
\makeatother

\IfFileExists{\jobname-pw.ind}{\input{\jobname-pw.ind}}{}

\end{document}

      