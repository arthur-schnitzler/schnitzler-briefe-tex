%% latex-korrekturansicht-vorspann.tex
%% Vorspann für die Korrekturansicht.
%% Lädt die gemeinsame Datei latex-vorspann.tex mit gesetztem Schalter.

\newif\ifkorrekturansicht
\korrekturansichttrue

\input{../tex-inputs/latex-vorspann}


\renewcommand{\erwaehntePersonen}{Personen: Stefan Zweig}
\renewcommand{\erwaehnteOrte}{Orte: Nordsee, Salzburg, Sternwartestraße 71, Sylt, Wien, Währinger Cottage}
\renewcommand{\erwaehnteWerke}{}
\section[Stefan Zweig an Arthur Schnitzler, 12. 7. 1923]{Stefan Zweig an Arthur Schnitzler, 12. 7. 1923}
\nopagebreak\mylabel{v}
\rehead{ }\normalsize\beginnumbering\briefempfaengerindex{Schnitzler, Arthur@\textsc{Schnitzler, Arthur}!zzzZweig, Stefan@\emph{von Stefan Zweig}!1923-07-121@{12. 7. 1923}|(be}
\toendnotes[C]{\smallbreak\pagebreak[2]}\Standort{CUL, Schnitzler, B 118.}
\physDesc{Bildpostkarte, 1 Blatt, 2 Seiten, 402 Zeichen
\newline{}Handschrift: blaue Tinte, lateinische Kurrent
\newline{}Versand: Stempel: »\nobreak{}\oindex{Salzburg@\textbf{Salzburg}, \emph{A.ADM2}|pwk}Salzburg 2, 12. VII. 23, 9\nobreak{}«.  }\toendnotes[C]{\smallbreak}\pstart{}{\pb}D\textsuperscript{r}
                  Arthur Schnitzler\pend{}\pstart{}\textcolor{pink}{Wien – Cottage}{}\ledrightnote{\textcolor{pink}{Währinger Cottage}}\pend{}\pstart{}\textcolor{pink}{\label{K_L03666-1v}\edtext{Sternwartestrasse 72}{\lemma{\textnormal{\emph{Sternwartestrasse 72}}}\Cendnote{\textnormal{\textcolor{blue}{Zweig} wechselt
                        bei der Adressierung seiner Schreiben an \textcolor{blue}{Schnitzler} immer wieder zwischen der falschen Hausnummer
                        »72« und der richtigen
                        »71«.}}}\label{K_L03666-1h}}{}\ledrightnote{\textcolor{pink}{Sternwartestraße 71}}\pend{}
{\bigskip}
\pstart
           \noindent{}\centering{}{\pb}\textcolor{gray}{\textbf{Durchbrechende Sonne}}\pend
           
\pstart
           \noindent{}{\pb}Lieber verehrter Herr
                  Doktor, diese Karte nahm ich nur von der \textcolor{pink}{Nordsee}{}\ledrightnote{\textcolor{pink}{Nordsee}} mit, wo ich auf \textcolor{pink}{Sylt}{}\ledrightnote{\textcolor{pink}{Sylt}} herrliche Tage
               hatte, Sie sei verwendet, Ihnen auf das innigste für Ihre guten Worte zu danken: es
               beglückt mich noch genau so wie in meinen ersten literarischen Jahren ein Wort von
               Ihnen zu erhalten: fühlen Sie daran die ungewandelte Verehrung Ihres \pend
           \pstart \spacefill\mbox{Stefan Zweig}\pend{}\endnumbering\briefempfaengerindex{Schnitzler, Arthur@\textsc{Schnitzler, Arthur}!zzzZweig, Stefan@\emph{von Stefan Zweig}!1923-07-121@{12. 7. 1923}|)be}\mylabel{h}
\begin{anhang}
\end{anhang}\normalsize

\doendnotes{C}
\bigskip
\vfill

\clearpage

\footnotesize

\lohead{\textsc{register}}

% Definiere theindex-Environment komplett neu ohne reledmac
\makeatletter
\renewenvironment{theindex}{%
  \section*{\indexname}%
  \setlength{\parindent}{0pt}%
  \setlength{\parskip}{0pt plus 0.3pt}%
  \let\item\@idxitem
}{%
  \clearpage
}
\makeatother

\IfFileExists{\jobname-pw.ind}{\input{\jobname-pw.ind}}{}

\end{document}

      