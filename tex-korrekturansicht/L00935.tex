%% latex-korrekturansicht-vorspann.tex
%% Vorspann für die Korrekturansicht.
%% Lädt die gemeinsame Datei latex-vorspann.tex mit gesetztem Schalter.

\newif\ifkorrekturansicht
\korrekturansichttrue

\input{../tex-inputs/latex-vorspann}


               \section[Hugo von Hofmannsthal an Arthur Schnitzler, 6. 7. 1899]{ Hugo von Hofmannsthal an Arthur Schnitzler, 6. 7. 1899}\nopagebreak\mylabel{v}\rehead{ }\normalsize\beginnumbering\briefempfaengerindex{Schnitzler, Arthur@\textsc{Schnitzler, Arthur}!zzzHofmannsthal, Hugo von@\emph{von Hugo von Hofmannsthal}!1899-07-063@{6. 7. 1899}|(be} \toendnotes[C]{\smallbreak\pagebreak[2]} \Standort{CUL, Schnitzler, B 43.}
\physDesc{Postkarte
\newline{}Handschrift: schwarze Tinte, deutsche Kurrent\newline{}Versand: 1) Stempel: »\nobreak{}\oindex{Marienbad@\textbf{Marienbad}, \emph{Besiedelter Ort (A.BSO)}|pwk}Marienbad, 6. 7. 1899, 2\nobreak{}«.  2) Stempel: »\nobreak{}\oindex{IX., Alsergrund@\textbf{IX., Alsergrund}, \emph{Bezirk (A.BZK)}|pwk}Wien 9/3 72, 7. 7. \textcolor{gray}{99}, 9.\textcolor{gray}{V}, Bestellt\nobreak{}«. 
\newline{}Schnitzler: mit Bleistift die Jahreszahl ergänzt: »99« \newline{}Ordnung: 1) mit Bleistift von unbekannter Hand nummeriert:
                                        »149« 2) mit Bleistift von unbekannter Hand nummeriert: »\strikeout{152}«}\buchAbdrucke{\weitereDrucke{Hugo von Hofmannsthal, Arthur Schnitzler: \emph{Briefwechsel}. Hg. Therese Nickl und Heinrich Schnitzler. Frankfurt am Main: \emph{S. Fischer} 1964, S. 123.} }\pstart{}{\pb}\textsc{Herrn D\textsuperscript{r} Arthur
                            Schnitzler}\pend{}\pstart{}\textsc{\textcolor{pink}{IX Wien}{}\ledrightnote{\textcolor{pink}{Wien}}}\pend{}\pstart{}\textcolor{pink}{\textsc{Franckgasse 1}}{}\ledrightnote{\textcolor{pink}{Frankgasse}}\pend{}{\bigskip}\pstart
           \raggedleft{}{\pb}\textcolor{pink}{Marienbad}{}\ledrightnote{\textcolor{pink}{Marienbad}}{\\}\textcolor{pink}{Hotel Klinger}{}\ledrightnote{\textcolor{pink}{Hotel Klinger}}{\\}6\textsuperscript{ten} July\pend
           \pstart
           Sie haben verſprochen, mich nicht zu lange Zeit ohne Kenntnis Ihrer Adreſſe zu
                    laſſen; die \textcolor{pink}{croatiſche}{}\ledrightnote{\textcolor{pink}{Kroatien}} iſt wohl lange nicht
                    mehr gültig?\pend
           \pstart
           Herzlich{\\[\baselineskip]}\spacefill\mbox{Hugo.}\pend
           \leftskip=0em{}\endnumbering\briefempfaengerindex{Schnitzler, Arthur@\textsc{Schnitzler, Arthur}!zzzHofmannsthal, Hugo von@\emph{von Hugo von Hofmannsthal}!1899-07-063@{6. 7. 1899}|)be}\mylabel{h}  \normalsize

\doendnotes{C}
\bigskip
\vfill

\clearpage

\footnotesize

\lohead{\textsc{register}}

% Definiere theindex-Environment komplett neu ohne reledmac
\makeatletter
\renewenvironment{theindex}{%
  \section*{\indexname}%
  \setlength{\parindent}{0pt}%
  \setlength{\parskip}{0pt plus 0.3pt}%
  \let\item\@idxitem
}{%
  \clearpage
}
\makeatother

\IfFileExists{\jobname-pw.ind}{\input{\jobname-pw.ind}}{}

\end{document}

      