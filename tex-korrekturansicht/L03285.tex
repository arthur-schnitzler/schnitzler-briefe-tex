%% latex-korrekturansicht-vorspann.tex
%% Vorspann für die Korrekturansicht.
%% Lädt die gemeinsame Datei latex-vorspann.tex mit gesetztem Schalter.

\newif\ifkorrekturansicht
\korrekturansichttrue

\input{../tex-inputs/latex-vorspann}


\renewcommand{\erwaehntePersonen}{Personen: Julius Szeps}
\renewcommand{\erwaehnteInstitutionen}{Institutionen: Wiener Allgemeine Zeitung}
\renewcommand{\erwaehnteOrte}{Orte: Paris, Schulerstraße, Universitätsstraße, Wien}
\renewcommand{\erwaehnteWerke}{}
\section[ Felix Salten an Arthur Schnitzler, 10. 12. 1898]{Felix Salten an Arthur Schnitzler, 10. 12. 1898}
\nopagebreak\mylabel{v}
\rehead{ }\normalsize\beginnumbering\briefempfaengerindex{Schnitzler, Arthur@\textsc{Schnitzler, Arthur}!zzzSalten, Felix@\emph{von Felix Salten}!1898-12-102@{10. 12. 1898}|(be}
\toendnotes[C]{\smallbreak\pagebreak[2]}\Standort{CUL, Schnitzler, B 89, A 2.}
\physDesc{Brief, 1 Blatt, 1 Seite, 435 Zeichen
\newline{}Handschrift: schwarze Tinte, lateinische Kurrent
\newline{}Ordnung: mit Bleistift von unbekannter Hand nummeriert: »109« }
\pstart
           \noindent{}{\pb}\textcolor{gray}{\textbf{\textbf{»\textcolor{brown}{Wiener Allgemeine
                        Zeitung}{}\ledrightnote{\textcolor{brown}{Wiener Allgemeine Zeitung}}«}}}\pend
           
\pstart
           \textcolor{gray}{\textbf{Redaction:}}\pend
           
\pstart
           \textcolor{gray}{\textbf{\textbf{\textcolor{pink}{IX/3, Univerſitätsſtraße Nr. 6}{}\ledrightnote{\textcolor{pink}{Universitätsstraße}}}}}\pend
           
\pstart
           \textcolor{gray}{\textbf{Adminiſtration:}}\hfill \textcolor{gray}{\textbf{\textcolor{pink}{Wien}{}\ledrightnote{\textcolor{pink}{Wien}},}}{ }10. Dezemb. \textcolor{gray}{\textbf{189}}8\pend
           
\pstart
           \textcolor{gray}{\textbf{\textbf{\textcolor{pink}{I. Schulerſtraße Nr. 20}{}\ledrightnote{\textcolor{pink}{Schulerstraße}}.}}}\pend
           
\pstart
           \textcolor{gray}{\textbf{Telegramm-Adreſſe: »Allgemeine, \textcolor{pink}{Wien}{}\ledrightnote{\textcolor{pink}{Wien}}{[}«{]}.}}\pend
           
\pstart
           \textcolor{gray}{\textbf{Telephon der Redaction: Nr. 805 u. 2180.}}\pend
           
\pstart
           \textcolor{gray}{\textbf{\hspace*{1.5em}„\hspace*{1.5em}„\hspace*{1.5em} Adminiſtration: Nr. 1024.}}\pend
           
\pstart{}Lieber Freund,\pend
\pstart
           während ich unwol war ist D\textsuperscript{r} \textcolor{blue}{Szeps}{}\ledrightnote{\textcolor{blue}{Julius Szeps}} nach \textcolor{pink}{Paris}{}\ledrightnote{\textcolor{pink}{Paris}} gereist,
               und ich erfahre jetzt, dass ein Betrag, welcher heute
               fällig war, nicht ausgezahlt werden kann, weil er nicht angewiesen wurde. Bitte,
               helfen Sie mir nochmals aus der Verlegenheit und senden Sie mir 10f. Ich werde Ihnen
               beide 10f. nächste Woche sicher zurückgeben. Ganz sicher. Ich brauche es wirklich
               (wegen meiner Leute) sehr notwendig.\pend
           
\pstart
           Herzlichst Ihr {\\[\baselineskip]}\spacefill\mbox{Salten}\pend
           \leftskip=0em{}\endnumbering\briefempfaengerindex{Schnitzler, Arthur@\textsc{Schnitzler, Arthur}!zzzSalten, Felix@\emph{von Felix Salten}!1898-12-102@{10. 12. 1898}|)be}\mylabel{h}  \normalsize

\doendnotes{C}
\bigskip
\vfill

\clearpage

\footnotesize

\lohead{\textsc{register}}

% Definiere theindex-Environment komplett neu ohne reledmac
\makeatletter
\renewenvironment{theindex}{%
  \section*{\indexname}%
  \setlength{\parindent}{0pt}%
  \setlength{\parskip}{0pt plus 0.3pt}%
  \let\item\@idxitem
}{%
  \clearpage
}
\makeatother

\IfFileExists{\jobname-pw.ind}{\input{\jobname-pw.ind}}{}

\end{document}

      