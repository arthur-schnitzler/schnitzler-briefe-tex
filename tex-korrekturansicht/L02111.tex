%% latex-korrekturansicht-vorspann.tex
%% Vorspann für die Korrekturansicht.
%% Lädt die gemeinsame Datei latex-vorspann.tex mit gesetztem Schalter.

\newif\ifkorrekturansicht
\korrekturansichttrue

\input{../tex-inputs/latex-vorspann}


               \section[Oscar Blumenthal an Arthur Schnitzler, {[}nach dem 13. 3. 1912{]}]{ Oscar Blumenthal an Arthur Schnitzler, {[}nach dem
                    13. 3. 1912{]}}\nopagebreak\mylabel{v}\rehead{ }\normalsize\beginnumbering\briefempfaengerindex{Schnitzler, Arthur@\textsc{Schnitzler, Arthur}!zzzBlumenthal, Oskar@\emph{von Oskar Blumenthal}!1912-03-141@{{[}nach dem 13. 3. 1912{]}}|(be} \toendnotes[C]{\smallbreak\pagebreak[2]} \Standort{CUL, Schnitzler, B 15.}
\physDesc{Klappkarte, 1 Karte, 2 Seiten
\newline{}Faksimilierte eigenhändige Danksagung
\newline{}Schnitzler: auf der ersten Seite mit Bleistift beschriftet: »{\pb}\textsc{Blumenthal}« \newline{}Ordnung: mit Bleistift von unbekannter Hand in eckiger Klammer
                                            datiert: »1912« }\toendnotes[C]{\smallbreak}\pstart
           \noindent{}\centering{}{\pb}{[}Fotografie Blumenthals von \textcolor{blue}{Erwin Raupp}{}\ledrightnote{\textcolor{blue}{Erwin Raupp}}{]}\pend
           \pstart
           {\pb}Für alle aufrichtenden Worte und
                    tröſtenden Zurufe zu meinem \label{K_L02111_1v}\edtext{sechzigſten Geburtstag}{\lemma{\textnormal{\emph{sechzigſten Geburtstag}}}\Cendnote{\textnormal{am
                            13. 3. 1912}}}\label{K_L02111_1h}{ }ſpricht der nebenſtehende ältere Herr seinen
                    innigſten Dank aus. Denn wenn man sein Alter nicht mehr verbergen kann, so muß
                    man damit coquettieren!{\dots} Mit einem warmen
                    Händedruck\pend
           \pstart \spacefill\mbox{Osc. Blumenthal.}\pend{}\endnumbering\briefempfaengerindex{Schnitzler, Arthur@\textsc{Schnitzler, Arthur}!zzzBlumenthal, Oskar@\emph{von Oskar Blumenthal}!1912-03-141@{{[}nach dem 13. 3. 1912{]}}|)be}\mylabel{h}  \normalsize

\doendnotes{C}
\bigskip
\vfill

\clearpage

\footnotesize

\lohead{\textsc{register}}

% Definiere theindex-Environment komplett neu ohne reledmac
\makeatletter
\renewenvironment{theindex}{%
  \section*{\indexname}%
  \setlength{\parindent}{0pt}%
  \setlength{\parskip}{0pt plus 0.3pt}%
  \let\item\@idxitem
}{%
  \clearpage
}
\makeatother

\IfFileExists{\jobname-pw.ind}{\input{\jobname-pw.ind}}{}

\end{document}

      