%% latex-korrekturansicht-vorspann.tex
%% Vorspann für die Korrekturansicht.
%% Lädt die gemeinsame Datei latex-vorspann.tex mit gesetztem Schalter.

\newif\ifkorrekturansicht
\korrekturansichttrue

\input{../tex-inputs/latex-vorspann}


               \section[Paul Goldmann an Arthur Schnitzler, 22. 3. {[}1896{]}]{ Paul Goldmann an Arthur Schnitzler, 22. 3. {[}1896{]}}\nopagebreak\mylabel{v}\rehead{ }\normalsize\beginnumbering\briefempfaengerindex{Schnitzler, Arthur@\textsc{Schnitzler, Arthur}!zzzGoldmann, Paul@\emph{von Paul Goldmann}!1896-03-221@{22. 3. {[}1896{]}}|(be} \toendnotes[C]{\smallbreak\pagebreak[2]} \Standort{DLA, A:Schnitzler, HS.NZ85.1.3166.}
\physDesc{Brief, 1 Blatt, 3 Seiten
\newline{}Handschrift: blaue Tinte, deutsche Kurrent
\newline{}Schnitzler: 1) mit Bleistift das Jahr »96« vermerkt 2) mit rotem Buntstift eine Unterstreichung}\toendnotes[C]{\smallbreak}\pstart
           \noindent{}{\pb}\textcolor{gray}{\textbf{\textbf{\textcolor{brown}{Frankfurter Zeitung}{}\ledrightnote{\textcolor{brown}{Frankfurter Zeitung}}}}}\pend
           \pstart
           \textcolor{gray}{\textbf{(\textcolor{brown}{\begin{otherlanguage}{french}Gazette de Francfort\end{otherlanguage}}{}\ledrightnote{\textcolor{brown}{Frankfurter Zeitung}}).}}\pend
           \pstart
           \textcolor{gray}{\textbf{\textbf{\begin{otherlanguage}{french}Fondateur M.\end{otherlanguage}{ }\textcolor{blue}{L. Sonnemann}{}\ledrightnote{\textcolor{blue}{Leopold Sonnemann}}.}}}\pend
           \pstart
           \begin{otherlanguage}{french}\textcolor{gray}{\textbf{\textcolor{green}{Journal}{}\ledrightnote{→\textcolor{green}{Frankfurter Zeitung}} politique,
                        financier,}}\end{otherlanguage}\pend
           \pstart
           \begin{otherlanguage}{french}\textcolor{gray}{\textbf{commercial et littéraire.}}\end{otherlanguage}\pend
           \pstart
           \begin{otherlanguage}{french}\textcolor{gray}{\textbf{\textbf{Paraissant trois fois par jour.}}}\end{otherlanguage}\hfill \textsc{\textcolor{pink}{Paris}{}\ledrightnote{\textcolor{pink}{Paris}}}, 22. März.\pend
           \pstart
           \begin{otherlanguage}{french}\textcolor{gray}{\textbf{\textbf{Bureau à \textcolor{pink}{Paris}{}\ledrightnote{\textcolor{pink}{Paris}}}}}\end{otherlanguage}\pend
           \pstart
           \begin{otherlanguage}{french}\textcolor{gray}{\textbf{\textbf{\textcolor{pink}{24. Rue Feydeau}{}\ledrightnote{\textcolor{pink}{rue Feydeau}}.}}}\end{otherlanguage}\pend
           \pstart\center{}Mein lieber Freund,\pend\pstart
           Hab’ Geduld mit mir; Du haſt ſie, und ich bin Dir von Herzen dankbar dafür. Das iſt
               ein toller Arbeits-Monat. Es regnet Arbeit, alle Winde
               wehen Arbeit einher. Ich ſchreibe Artikel jeder Art über Gott und die Welt und
               Sonſtiges. Sonſt komme ich zu nichts. Jede Woche beginne ich mit dem Vorſatz: Nun
               werde ich ihm ſchreiben. Ihm biſt natürlich Du. Und die Woche geht vorüber, und ich
               habe nicht geſchrieben. {\pb}Auch bin ich krank. Mein
               Augenleiden wird ernſt. Die Ärzte ſagen, ich ſolle ausruhen. Haha! Und bei alledem
               denke ich faſt jeden Tag an Dich, mit Beſorgniß, und frage mich: Wie wird er das
               aufnehmen, daß ich ihm nicht ſchreibe? Nun weiß ichs und bin beruhigt. Diese Woche
               denke ich kann ich Dir doch den ausführlichen Brief ſchreiben. Neues weiß ich
               übrigens nicht. Die \textcolor{green}{Überſetzung}{}\ledrightnote{→\textcolor{green}{Amourette. Pièce en trois actes. Adaptée de Arthur Schnitzler}}s-Angelegenheit ſtockt. \textsc{\textcolor{blue}{Thorel}{}\ledrightnote{\textcolor{blue}{Jean Thorel}}} und ich laufen uns nach und können {\pb}uns nicht
               treffen.\pend
           \pstart
           Dank’ für das \label{K_L02768-1v}\edtext{Bulletin}{\lemma{\textnormal{\emph{Bulletin}}}\Cendnote{\textnormal{möglicherweise 
               die »Depesche« des letzten Briefs, Paul Goldmann an Arthur Schnitzler, 22. 3. [1896]}}}\label{K_L02768-1h}. Was macht das neue \label{K_L02768-2v}\edtext{\textcolor{green}{Stück}{}\ledrightnote{→\textcolor{green}{Freiwild. Schauspiel in 3 Akten}}}{\lemma{\textnormal{\emph{Stück}}}\Cendnote{\textnormal{Am 23. 2. 1896 begann \textcolor{blue}{Schnitzler} ein weiteres Mal, \emph{\textcolor{green}{Freiwild}} neu zu schreiben. Er war mit dem \textcolor{green}{Stück} noch immmer nicht
                  zufrieden.}}}\label{K_L02768-2h}? Was ſagſt Du zu \textsc{\textcolor{blue}{Herzl}{}\ledrightnote{\textcolor{blue}{Theodor Herzl}}s}{ }\strikeout{al\textcolor{gray}{b}} albernem \label{K_L02768-5v}\edtext{\textcolor{green}{Buche}{}\ledrightnote{→\textcolor{green}{Der Judenstaat. Versuch einer modernen Lösung der Judenfrage}}}{\lemma{\textnormal{\emph{Buche}}}\Cendnote{\textnormal{\emph{\textcolor{green}{Der Judenstaat. Versuch einer
                     modernen Lösung der Judenfrage}} wurde Mitte Februar 1896
                  ausgeliefert. \textcolor{blue}{Schnitzler} sprach am 8. 3. 1896 mit \textcolor{blue}{Herzl} über das Buch.}}}\label{K_L02768-5h}? Was macht \textsc{\textcolor{blue}{Richard}{}\ledrightnote{\textcolor{blue}{Richard Beer-Hofmann}}}?\pend
           \pstart
           Grüß’ Dich Gott, mein lieber Freund!\pend
           \pstart
           Von Herzen {\\[\baselineskip]}Dein {\\[\baselineskip]}\spacefill\mbox{Paul Goldmann}\pend
           \leftskip=0em{}\endnumbering\briefempfaengerindex{Schnitzler, Arthur@\textsc{Schnitzler, Arthur}!zzzGoldmann, Paul@\emph{von Paul Goldmann}!1896-03-221@{22. 3. {[}1896{]}}|)be}\mylabel{h}\begin{anhang}\end{anhang}\normalsize

\doendnotes{C}
\bigskip
\vfill

\clearpage

\footnotesize

\lohead{\textsc{register}}

% Definiere theindex-Environment komplett neu ohne reledmac
\makeatletter
\renewenvironment{theindex}{%
  \section*{\indexname}%
  \setlength{\parindent}{0pt}%
  \setlength{\parskip}{0pt plus 0.3pt}%
  \let\item\@idxitem
}{%
  \clearpage
}
\makeatother

\IfFileExists{\jobname-pw.ind}{\input{\jobname-pw.ind}}{}

\end{document}

      