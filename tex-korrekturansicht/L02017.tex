%% latex-korrekturansicht-vorspann.tex
%% Vorspann für die Korrekturansicht.
%% Lädt die gemeinsame Datei latex-vorspann.tex mit gesetztem Schalter.

\newif\ifkorrekturansicht
\korrekturansichttrue

\input{../tex-inputs/latex-vorspann}


               \section[Albert Ehrenstein an Arthur Schnitzler, 27. 4. 1911]{ Albert Ehrenstein an Arthur Schnitzler, 27. 4. 1911}\nopagebreak\mylabel{v}\rehead{ }\normalsize\beginnumbering\briefempfaengerindex{Schnitzler, Arthur@\textsc{Schnitzler, Arthur}!zzzEhrenstein, Albert@\emph{von Albert Ehrenstein}!1911-04-271@{27. 4. 1911}|(be} \toendnotes[C]{\smallbreak\pagebreak[2]} \Standort{DLA, A:Schnitzler, HS.NZ85.1.2836.}
\physDesc{Brief, 4 Blätter, 4 Seiten
\newline{}Schreibmaschine, maschinschriftliche Paginierung 2–4
\newline{}Handschrift: schwarze Tinte, lateinische Kurrent (\noindent{}Unterschrift)\newline{}Ordnung: ans Ende der
                                            Abschrift gereiht und dort auch kommentiert: »Brief vom
                                                27. April 1911 (letzter Brief) befindet
                                                sich unter den Abschriften der Briefe, ebenso eine Kopie der
                                                eigenen Antwort« }\Standort{Wienbibliothek im Rathaus, H.I.N.-141856.}
\physDesc{4 Blätter, 4 Seiten, maschineller Durchschlag
\newline{}Schreibmaschine, maschinschriftliche Paginierung 2–4
\newline{}Handschrift: schwarze Tinte, deutsche Kurrent (\noindent{}Unterschrift)}\Standort{Jerusalem, The National Library of Israel, ARC. Ms. Var. 306 1 117.}
\physDesc{Brief, 2 Blätter, 3 Seiten, Umschlag, Entwurf
\newline{}Handschrift: schwarze Tinte, deutsche Kurrent}\buchAbdrucke{\weitereDrucke{Albert Ehrenstein: \emph{Briefe}. Hg. Hanni Mittelmann. München: \emph{Boer} 1989, S. 60–63 (Werke, 1).} }\pstart
           {\pb}Dr. Albert Ehrenstein.\hfill 27. April 1911\pend
           \pstart
           \textcolor{pink}{Universität Wien}{}\ledrightnote{\textcolor{pink}{Universität Wien}}\pend
           \pstart{}Sehr geehrter Herr Doktor!\pend\pstart
           Vor einigen Tagen kam mir wieder Ihr letzter Brief in die Hand und ich finde nun,
                    dass ich die darin behandelte Angelegenheit nicht so auf sich beruhen lassen
                    kann. Auch fühlte ich mich verpflichtet, Herrn \textcolor{blue}{Karl Kraus}{}\ledrightnote{\textcolor{blue}{Karl Kraus}}, dem ich leider bis zum 21. April von dem
                    Inhalt Ihres Briefes, der sich auch auf ihn bezieht, keine Mitteilung gemacht
                    hatte, das Schreiben vorzulegen. Ich bin daraufhin zu dem Entschlusse gelangt,
                    einige unerlässliche Feststellungen vorzunehmen, sowohl für meine Person, wie
                    für Herrn \textcolor{blue}{Kraus}{}\ledrightnote{\textcolor{blue}{Karl Kraus}}, der, wie ich mich eben
                    überzeugt habe, vollkommen unverschuldet mit dieser Sache in Zusammenhang
                    gebracht wurde und der sich durch die Voraussetzung Ihres Briefes: »Es ist
                    jedenfalls total ausgeschlossen, dass sich \textcolor{blue}{Grossmann}{}\ledrightnote{\textcolor{blue}{Stefan Großmann}} und \textcolor{blue}{Kraus}{}\ledrightnote{\textcolor{blue}{Karl Kraus}} diese Fabel
                    einfach aus den Fingern gesogen hätten« einigermassen überrascht fühlte.\pend
           \pstart
           Ich erkläre hiemit ausdrücklich: es tut mir leid, Ihnen gegenüber Aeusserungen
                    gemacht zu haben, die Herrn \textcolor{blue}{Grossmann}{}\ledrightnote{\textcolor{blue}{Stefan Großmann}} zu
                    gravieren schienen. Ich stehe nicht an, sie mit der Kundgebung meines lebhaften
                    Bedauerns vollständig zurückzuziehen. Es wird vielleicht gut sein, wenn ich
                    Ihnen meine Aeusserungen, die ich nicht mehr aufrecht erhalte, in Erinnerung
                    bringe. Mit Rücksicht auf hinhaltende Versprechungen, die Herr \textcolor{blue}{Grossmann}{}\ledrightnote{\textcolor{blue}{Stefan Großmann}} einzelnen {\pb}Schauspielern gemacht haben soll, sagte ich, Herr \textcolor{blue}{Grossmann}{}\ledrightnote{\textcolor{blue}{Stefan Großmann}} scheine eine Art Hochstapler zu sein. Sie
                    antworteten darauf: »Nennen Sie nicht das Wort ›Hoch‹ in Verbindung mit \textcolor{blue}{Stephan Grossmann}{}\ledrightnote{\textcolor{blue}{Stefan Großmann}}!« Ferner sagte ich,
                    fussend auf einem Schauspielergerede, Herr \textcolor{blue}{Grossmann}{}\ledrightnote{\textcolor{blue}{Stefan Großmann}} solle erotisches Entgegenkommen gegen Beweise seiner
                    direktorialen oder kritischen Gunst tauschweise eingehandelt haben. Sie
                    antworteten: »Auch ich habe von Schauspielern gehört, Stephan \textcolor{blue}{Grossmann}{}\ledrightnote{\textcolor{blue}{Stefan Großmann}} ist das grösste Schwein, das in \textcolor{pink}{Wien}{}\ledrightnote{\textcolor{pink}{Wien}} existiert.« Ich bedaure sehr, diese drastischen
                    Worte, die besser nicht über unser Privatgespräch hinaus Wirkung erlangt hätten,
                    weitergetragen zu haben, in Kreise, die Ihnen, wie Sie sagen, Ȋusserlich und
                    innerlich ferne sind und bleiben sollen«.\pend
           \pstart
           Nochmals, ich bedauere also: gestützt auf ein Schauspielergeschwätz, Ihnen
                    Mitteilungen über Herrn \textcolor{blue}{Grossmann}{}\ledrightnote{\textcolor{blue}{Stefan Großmann}} gemacht zu
                    haben, und noch mehr bedauere ich, gestützt auf Ihre Autorität, die mir diesen
                    Tratsch zu bestätigen schien, ihn an drei Leute weitergegeben zu haben. Ich habe
                    damit das Odium auf mich genommen, scheinbar eine private Mitteilung benützt,
                    jedenfalls aber Sie in die Unannehmlichkeit versetzt zu haben, Ihre Bemerkungen
                    eventuell gegen Herrn \textcolor{blue}{Grossmann}{}\ledrightnote{\textcolor{blue}{Stefan Großmann}} vertreten zu
                    müssen. Wiewohl ich keinen Moment zweifelte, dass Sie dies im Stande wären, und
                    ferner nicht zweifelte, dass Ihre Information gegebenenfalls eine Stütze für
                    mich wäre, so sehe ich doch ohne weiteres ein, dass ich Ihnen damit keinen
                    Dienst erwiesen habe. Ich bedauere dies und bitte Sie dafür um Entschuldigung.
                    Trotzdem ist es un{\pb}erlässlich, den Tatbestand zu
                    klären. Was ich bei aller Dankbarkeit, zu der ich Ihnen gegenüber verpflichtet
                    bin, absolut nicht aus der Welt schaffen kann, ist: dass die zitierten Worte
                    wirklich Ihrerseits gefallen sind, also deren Anführung keineswegs, wie Sie die
                    Sache dargestellt haben, auf einer entstellend-erfinderischen Phantasietätigkeit
                    meinerseits beruht. Für meine Erinnerungen bin ich vor jedem Forum
                    verantwortlich. Und ich erinnere mich, Sie haben mit mir in jenem für Herrn \textcolor{blue}{Grossmann}{}\ledrightnote{\textcolor{blue}{Stefan Großmann}} abträglichen Sinne gesprochen. Ich
                    will Ihnen aber gern damit entgegenkommen, dass ich – wie Herr \textcolor{blue}{Kraus}{}\ledrightnote{\textcolor{blue}{Karl Kraus}} mir rät – ebenso wie ich meine Behauptungen über
                    Herrn \textcolor{blue}{Grossmann}{}\ledrightnote{\textcolor{blue}{Stefan Großmann}} nicht aufrecht erhalte, auch
                    die Ihren mit Bedauern zurückziehe.\pend
           \pstart
           Was Herrn \textcolor{blue}{Kraus}{}\ledrightnote{\textcolor{blue}{Karl Kraus}} betrifft, dem ich leider erst
                    jetzt Ihren Brief gezeigt habe, fühle ich mich verpflichtet, das Folgende
                    anzuführen: Herr \textcolor{blue}{Kraus}{}\ledrightnote{\textcolor{blue}{Karl Kraus}} hat von der Affäre
                    zwar durch mich gehört, sie aber nicht weitererzählt. Er äusserte, als ich ihm
                    Ihren Brief zeigte, die Annahme, dass er mit dieser Sache etwas zu tun habe,
                    müssten sich \textcolor{blue}{Grossmann}{}\ledrightnote{\textcolor{blue}{Stefan Großmann}} und Schnitzler aus
                    den Fingern gesogen haben. Nichts vermöge ihn weniger zu interessieren, als ein
                    Beweis, Herr \textcolor{blue}{Grossmann}{}\ledrightnote{\textcolor{blue}{Stefan Großmann}} habe seine
                    direktoriale oder kritische Gewalt Schauspielerinnen gegenüber missbraucht. Im
                    Gegenteil könnte ihn der Nachweis, Herr \textcolor{blue}{Grossmann}{}\ledrightnote{\textcolor{blue}{Stefan Großmann}} habe auf erotischem Wege ein Talent entdeckt, möglicherweise
                    dazu bringen, an die Befähigung des Herrn \textcolor{blue}{Grossmann}{}\ledrightnote{\textcolor{blue}{Stefan Großmann}}, ein Theater zu leiten, fortan zu glauben. Herr \textcolor{blue}{Kraus}{}\ledrightnote{\textcolor{blue}{Karl Kraus}} erklärte ferner, dass ihn ein
                    Einzelfall von Korruption längst nicht mehr beschäftigen könne und tatsäch{\pb}liche Feststellungen auf dem Gebiete der
                    Theatermoral stünden im stärksten Widerspruch zu Allem, was er je zur
                    Psychologie der Schauspielerin geschrieben habe und was ihn vom
                    Freiwildstandpunkt in denkbar weitester Ferne halte. Mir selbst riet er
                    eindringlich und energisch ab, mich mit einem Falle zu befassen, der entschieden
                    so tief unter meinem wie unter seinem Niveau sei und auf Wissenschaft und Erweis
                    in solchen Dingen zu verzichten. Diese Unterredung trug auch mit dazu bei, dass
                    ich es sehr bedauerte und bedauere, den Schauspielerklatsch aufgegriffen zu
                    haben. Wiederholt erkläre ich dies für meine Person, wiederholt muss ich
                    feststellen, dass es Herrn \textcolor{blue}{Kraus}{}\ledrightnote{\textcolor{blue}{Karl Kraus}} überaus
                    peinlich berührt hat, durch rein passive Beteiligung an einer Sache, die so tief
                    unter seinen geistigen Interessen liege, mit Kreisen zu Zusammenhang gebracht
                    worden zu sein, die ihm, wie er sagt, äusserlich und innerlich ferne sind und
                    bleiben sollen.\pend
           \pstart
           Mit diesen Richtigstellungen ist die Angelegenheit für mich erledigt.\pend
           \pstart
           Hochachtungsvoll:{\\[\baselineskip]}\spacefill\mbox{{[}hs.:{]} D\textsuperscript{r} Albert Ehrenstein}\pend
           \leftskip=0em{}\pstart
           \noindent{}{[}ms.:{]} Wohlgeboren Herrn{\\}Dr. Arthur \so{Schnitzler},{\\}\textcolor{pink}{\so{Wien}}{}\ledrightnote{\textcolor{pink}{Wien}}.\pend
           \endnumbering\briefempfaengerindex{Schnitzler, Arthur@\textsc{Schnitzler, Arthur}!zzzEhrenstein, Albert@\emph{von Albert Ehrenstein}!1911-04-271@{27. 4. 1911}|)be}\mylabel{h}  \normalsize

\doendnotes{C}
\bigskip
\vfill

\clearpage

\footnotesize

\lohead{\textsc{register}}

% Definiere theindex-Environment komplett neu ohne reledmac
\makeatletter
\renewenvironment{theindex}{%
  \section*{\indexname}%
  \setlength{\parindent}{0pt}%
  \setlength{\parskip}{0pt plus 0.3pt}%
  \let\item\@idxitem
}{%
  \clearpage
}
\makeatother

\IfFileExists{\jobname-pw.ind}{\input{\jobname-pw.ind}}{}

\end{document}

      