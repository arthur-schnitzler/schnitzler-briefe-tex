%% latex-korrekturansicht-vorspann.tex
%% Vorspann für die Korrekturansicht.
%% Lädt die gemeinsame Datei latex-vorspann.tex mit gesetztem Schalter.

\newif\ifkorrekturansicht
\korrekturansichttrue

\input{../tex-inputs/latex-vorspann}


               \section[Friedrich M. Fels und Jenny Nordegg an Arthur Schnitzler, 15. 10. 1895]{ Friedrich M. Fels und Jenny Nordegg an Arthur Schnitzler,
                    15. 10. 1895}\nopagebreak\mylabel{v}\rehead{ }\normalsize\beginnumbering\briefempfaengerindex{Schnitzler, Arthur@\textsc{Schnitzler, Arthur}!zzzNordegg, Jenny@\emph{von Jenny Nordegg}!1895-10-151@{15. 10. 1895}|(be}\briefempfaengerindex{Schnitzler, Arthur@\textsc{Schnitzler, Arthur}!zzzFels, Friedrich Michael@\emph{von Friedrich Michael Fels}!1895-10-151@{15. 10. 1895}|(be} \toendnotes[C]{\smallbreak\pagebreak[2]} \Standort{DLA, A:Schnitzler, HS.NZ85.1.2956.}
\physDesc{Postkarte
\newline{}Handschrift Friedrich Michael Fels: schwarze Tinte, lateinische Kurrent\newline{}Handschrift Jenny Nordegg: schwarze Tinte\newline{}Versand: 1) Stempel: »\nobreak{}\oindex{Zuerich@\textbf{Zürich}, \emph{Besiedelter Ort (A.BSO)}|pwk}Zürich Bhf. Exp., 15. X. 95, 11\nobreak{}«.  2) Stempel: »\nobreak{}\oindex{IX., Alsergrund@\textbf{IX., Alsergrund}, \emph{Bezirk (A.BZK)}|pwk}Wien 9/3, 17 10. 95, 9.V, Bestellt\nobreak{}«. 
\newline{}Schnitzler: mit Bleistift nummeriert: »27« }\toendnotes[C]{\smallbreak}\pstart{}{\pb}Herrn Dr. med. Arthur
                        Schnitzler\pend{}\pstart{}Schriftsteller\pend{}\pstart{}\textcolor{pink}{Wien}{}\ledrightnote{\textcolor{pink}{Wien}}\pend{}\pstart{}\textcolor{pink}{IX, Frankgaſse 1}{}\ledrightnote{\textcolor{pink}{Frankgasse}}\pend{}\pstart{}\textcolor{pink}{Österreich}{}\ledrightnote{\textcolor{pink}{Österreich}}\pend{}{\bigskip}\pstart
           \noindent{}\centering{}{\pb}\textcolor{gray}{\textbf{\textcolor{pink}{Grand Restaurant et Café Metropol Zurich
                                Auböck {\kaufmannsund} Ziegler Pr.}{}\ledrightnote{\textcolor{pink}{Café Metropol}}}}\pend
           \pstart
           \noindent{}\centering{}\textcolor{gray}{\textbf{\textcolor{pink}{Irrgarten (Labyrinth)}{}\ledrightnote{\textcolor{pink}{Irrgarten (Irrgänge)}} D\textsuperscript{ir}{ }\textcolor{blue}{G. D.’Ouvenou}{}\ledrightnote{\textcolor{blue}{Gésa D’Ouvenou}}.}}\pend
           \pstart{}Lieber Dr. Schnitzler!\pend\pstart
           Soeben lesen wir \textcolor{blue}{Speidel}{}\ledrightnote{\textcolor{blue}{Ludwig Speidel}}s \textcolor{green}{Kritik}{}\ledrightnote{→\textcolor{green}{[Burgtheater.] Liebelei}} und freuen uns riesig über Ihren
                    Erfolg. Fahren Sie so weiter, junger Ma{\geminationn}, und
                    vergeſsen Sie im Glücke nicht »\textcolor{green}{derer, die am Wege sterben}{}\ledrightnote{→\textcolor{green}{Uriel Acosta Trauerspiel in fünf Aufzügen}}«.\pend
           \pstart
           Herzlichst{\\[\baselineskip]}{\\[\baselineskip]}\spacefill\mbox{{[}hs. Nordegg:{]} Jenny Nordegg}{\\[\baselineskip]}{[}hs. Fels:{]} und \spacefill\mbox{Friedr. M. Fels}\pend
           \leftskip=0em{}\endnumbering\briefempfaengerindex{Schnitzler, Arthur@\textsc{Schnitzler, Arthur}!zzzNordegg, Jenny@\emph{von Jenny Nordegg}!1895-10-151@{15. 10. 1895}|)be}\briefempfaengerindex{Schnitzler, Arthur@\textsc{Schnitzler, Arthur}!zzzFels, Friedrich Michael@\emph{von Friedrich Michael Fels}!1895-10-151@{15. 10. 1895}|)be}\mylabel{h}  \normalsize

\doendnotes{C}
\bigskip
\vfill

\clearpage

\footnotesize

\lohead{\textsc{register}}

% Definiere theindex-Environment komplett neu ohne reledmac
\makeatletter
\renewenvironment{theindex}{%
  \section*{\indexname}%
  \setlength{\parindent}{0pt}%
  \setlength{\parskip}{0pt plus 0.3pt}%
  \let\item\@idxitem
}{%
  \clearpage
}
\makeatother

\IfFileExists{\jobname-pw.ind}{\input{\jobname-pw.ind}}{}

\end{document}

      