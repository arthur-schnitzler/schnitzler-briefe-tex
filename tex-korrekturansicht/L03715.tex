%% latex-korrekturansicht-vorspann.tex
%% Vorspann für die Korrekturansicht.
%% Lädt die gemeinsame Datei latex-vorspann.tex mit gesetztem Schalter.

\newif\ifkorrekturansicht
\korrekturansichttrue

\input{../tex-inputs/latex-vorspann}


\section[Elsa Plessner an Arthur Schnitzler, 9. 8. 1897]{L03715 Elsa Plessner an Arthur Schnitzler, 9. 8. 1897}
\nopagebreak\mylabel{L03715v}
\rehead{ }\normalsize\beginnumbering\briefempfaengerindex{Schnitzler, Arthur@\textsc{Schnitzler, Arthur}!zzzPlessner, Elsa@\emph{von Elsa Plessner}!1897-08-092@{9. 8. 1897}|(be}
\toendnotes[C]{\smallbreak\pagebreak[2]}
\correspDesc{Versand  durch Elsa Plessner am 9. 8. 1897 in Wien
\newline{}Erhalt  durch Arthur Schnitzler im Zeitraum [9. 8. 1897
                  – 12. 8. 1897?] in Wien}\toendnotes[C]{\smallbreak}
\Standort{DLA, A:Schnitzler, HS.1985.1.419.}
\physDesc{Brief, 1 Blatt, 4 Seiten, 1896 Zeichen
\newline{}Handschrift: schwarze Tinte, lateinische Kurrent}\toendnotes[C]{\smallbreak}
\pstart
           \raggedleft{}{\pb}\textcolor{pink}{Wien-Sievering, Fröschlgasse N\textsuperscript{o} 6}\oindex{Wien@\textbf{Wien}!XIX., Döbling@\textbf{XIX., Döbling}!Fröschelgasse 6@\textbf{Fröschelgasse 6}, \emph{Wohngebäude}|pw}{}\ledrightnote{\textcolor{pink}{Fröschelgasse 6}}\pend
           
\pstart
           \raggedleft{}den 9. VIII. 97\pend
           
\pstart{}Verehrter, lieber Herr Doctor!\pend\vspace{0.5em}
\pstart
           Ihre lieben und liebenswürdigen \label{K_L03715-1v}\edtext{Zeilen}{\lemma{\textnormal{\emph{Zeilen}}}\Cendnote{\textnormal{nicht überliefert}}}\label{K_L03715-1}
               bestätige \strikeout{m}ich \introOben{}mit\introOben{}
               herzlicher Freude! Inzwischen haben Sie ja auch erfahren, dass ich selbst von der
                  \label{K_L03715-2v}\edtext{absurden Correctur-idee}{\lemma{\textnormal{\emph{absurden Correctur-idee}}}\Cendnote{\textnormal{Vgl. Elsa Plessner an Arthur Schnitzler, 7. 8. [1897].}}}\label{K_L03715-2}{ }\label{K_L03715-3v}\edtext{zurückkam}{\lemma{\textnormal{\emph{zurückkam}}}\Cendnote{\textnormal{Vgl. Elsa Plessner an Arthur Schnitzler, 10. 1. 1900.}}}\label{K_L03715-3}, gleich
               nachdem ich diese Absicht Ihnen mittheilte! Ich kann Ihnen nur sagen, dass ich folgen
               werde; schön still und ruhig sein und mich trösten. Ich bin ja so folg{\pb}sam{\dotstwo}{ }Heute haben Sie sich wieder ein neues Verdienst erworben! Sie haben
               meine Ballfrisur für die kommende Saison gerettet, die ich auf dem besten Wege war,
               zu zerstören durch verzweifeltes Ausrupfen jedes einzelnen Haares! Als ich Ihre
                  \label{K_L03715-4v}\edtext{Trostzeilen}{\lemma{\textnormal{\emph{Trostzeilen}}}\Cendnote{\textnormal{nicht überliefert}}}\label{K_L03715-4} erhielt, beendigte ich sofort diese
               ebenso amusante, als vortheilhafte Procedur. Es ist aber nicht schön von Ihnen, dass
               Sie meinen heiligen Schmerz herabwürdigend, mich — zwar lieb und herzig – so \label{K_L03715-5v}\edtext{frozzeln}{\lemma{\textnormal{\emph{frozzeln}}}\Cendnote{\textnormal{necken}}}\label{K_L03715-5}. Von einer … »\label{K_L03715-6v}\edtext{\textcolor{green}{Parabel}\pwindex{Plessner, Elsa 22.\,8.\,1875 Wien – 7.\,5.\,1932 Alicante@\textsc{Plessner, Elsa} (22.\,8.\,1875 Wien – 7.\,5.\,1932 Alicante), \emph{Schriftstellerin}!gläserne Käfig. Eine Parabel@\strich\emph{Der gläserne Käfig. Eine Parabel}|pw}{}\ledrightnote{\textcolor{green}{Der gläserne Käfig. Eine Parabel}}}{\lemma{\textnormal{\emph{Parabel}}}\Cendnote{\textnormal{\textcolor{blue}{Plessners}\pwindex{Plessner, Elsa 22.\,8.\,1875 Wien – 7.\,5.\,1932 Alicante@\textsc{Plessner, Elsa} (22.\,8.\,1875 Wien – 7.\,5.\,1932 Alicante), \emph{Schriftstellerin}|pwk} Text \emph{\textcolor{green}{Der gläserne Käfig}\pwindex{Plessner, Elsa 22.\,8.\,1875 Wien – 7.\,5.\,1932 Alicante@\textsc{Plessner, Elsa} (22.\,8.\,1875 Wien – 7.\,5.\,1932 Alicante), \emph{Schriftstellerin}!gläserne Käfig. Eine Parabel@\strich\emph{Der gläserne Käfig. Eine Parabel}|pwk}} erschien im Erstdruck (\emph{\textcolor{brown}{Die Zeit}\orgindex{Zeit. Wiener Wochenschrift@Die Zeit. Wiener Wochenschrift|pwk}}, Bd. 12, Nr. 149, 7.\,8.\,1897, S. 95–96) mit unautorisierten Änderungen. Dazu gehörte
                  die nicht von der Autorin vorgesehenen Gattungsbezeichnungen »eine
                     Parabel«.}}}\label{K_L03715-6}{\dotstwo}« 
               dürfen Sie mir eben nur schreiben, aber nicht
                  spre{\pb}chen, sonst hätte ich Ihnen schnell bewiesen,
               dass ich selbst Ihnen, meinem hochmögenden Gönner gegenüber, nicht »wehrlos« bin,
               wenn ich 10 rosige und scharfe Fingernägel nicht ganz vergesse.\pend
           
\pstart
           – Sonst aber bin ich kalt- und \label{K_L03715-7v}\edtext{back-fischblütig}{\lemma{\textnormal{\emph{back-fischblütig}}}\Cendnote{\textnormal{Backfisch, veraltet: Teenagerin, junge Frau}}}\label{K_L03715-7} –, werde mich nicht ins Wasser
               stürzen, umso mehr ich wieder einmal – von 40° Fieber vor 14 Tagen
               aufgestanden – höchst sorgsam auf meine miserable Gesundheit achten muss, welche ein
               anderes als moralisch kaltes Bad jetzt absolut nicht verträgt. –\pend
           
\pstart
           {\pb}Also ich tröste mich{\dotsfour}\pend
           
\pstart
           Jemand, der nicht allzudumm ist, hat \introOben{}mir\introOben{} einmal gesagt –
               sehr drastisch und geradezu – »Publikum ist, wer nichts versteht« – Da Sie derselben
               Ansicht – nur in homöopathischer Verdünnung, zu sein scheinen, wird es wohl so sein.
               – – –\pend
           
\pstart
           Viele, viele herzliche »Danke« für Ihren geschriebenen Samariterdienst – und ebenso
               viel herzliche Grüße!\pend
           
\pstart
           Stets dankbarer und mit unveränderlicher Verehrung{\\[\baselineskip]}\spacefill\mbox{Elsa Plessner}\pend
           \leftskip=0em{}
\pstart
           \noindent{}P. S. Haben Sie \label{K_L03715-8v}\edtext{»\textcolor{green}{Warten}\pwindex{Plessner, Elsa 22.\,8.\,1875 Wien – 7.\,5.\,1932 Alicante@\textsc{Plessner, Elsa} (22.\,8.\,1875 Wien – 7.\,5.\,1932 Alicante), \emph{Schriftstellerin}!Warten. Novelle@\strich\emph{Warten. Novelle}|pw}{}\ledrightnote{\textcolor{green}{Warten. Novelle}}«}{\lemma{\textnormal{\emph{»Warten«}}}\Cendnote{\textnormal{\emph{\textcolor{green}{Magazin für Litteratur}\pwindex{Magazin für die Literatur des Auslandes@\emph{Magazin für die Literatur des Auslandes}|pwk}}, Jg. 66, Nr. 29,
                           24. 7. 1897, Sp. 867–875. }}}\label{K_L03715-8} im »\textcolor{green}{Magazin}\pwindex{Magazin für die Literatur des Auslandes@\emph{Magazin für die Literatur des Auslandes}|pw}{}\ledrightnote{\textcolor{green}{Magazin für die Literatur des Auslandes}}« vom 23. Juli bemerkt?\pend
           \selectlanguage{ngerman}\endnumbering\briefempfaengerindex{Schnitzler, Arthur@\textsc{Schnitzler, Arthur}!zzzPlessner, Elsa@\emph{von Elsa Plessner}!1897-08-092@{9. 8. 1897}|)be}\mylabel{L03715h}  \normalsize

\doendnotes{C}
\bigskip
\vfill

\clearpage

\footnotesize

\lohead{\textsc{register}}

% Definiere theindex-Environment komplett neu ohne reledmac
\makeatletter
\renewenvironment{theindex}{%
  \section*{\indexname}%
  \setlength{\parindent}{0pt}%
  \setlength{\parskip}{0pt plus 0.3pt}%
  \let\item\@idxitem
}{%
  \clearpage
}
\makeatother

\IfFileExists{\jobname-pw.ind}{\input{\jobname-pw.ind}}{}

\end{document}

      