%% latex-korrekturansicht-vorspann.tex
%% Vorspann für die Korrekturansicht.
%% Lädt die gemeinsame Datei latex-vorspann.tex mit gesetztem Schalter.

\newif\ifkorrekturansicht
\korrekturansichttrue

\input{../tex-inputs/latex-vorspann}


               \section[Richard Beer-Hofmann an Arthur Schnitzler, 14. 6. 1895]{ Richard Beer-Hofmann an Arthur Schnitzler,
               14. 6. 1895}\nopagebreak\mylabel{v}\rehead{ }\normalsize\beginnumbering\briefempfaengerindex{Schnitzler, Arthur@\textsc{Schnitzler, Arthur}!zzzBeer-Hofmann, Richard@\emph{von Richard Beer-Hofmann}!1895-06-141@{14. 6. 1895}|(be} \toendnotes[C]{\smallbreak\pagebreak[2]} \Standort{CUL, Schnitzler, B 8.}
\physDesc{Brief, 1 Blatt, 3 Seiten
\newline{}Handschrift: Bleistift, lateinische Kurrent
\newline{}Schnitzler: mit Bleistift datiert: »14/6 95« und nummeriert: »60« }\buchAbdrucke{\weitereDrucke{Arthur Schnitzler, Richard Beer-Hofmann: \emph{Briefwechsel 1891–1931}. Hg. Konstanze Fliedl. Wien, Zürich: \emph{Europaverlag} 1992, S. 74.} }\toendnotes[C]{\smallbreak}\pstart
           \noindent{}{\pb}Lieber Arthur! In
               einer halben Stunde werde ich ins Bett fallen; – vorher nur folgendes: Ich bin gegen
                  \uline{\textcolor{blue}{Zasche}{}\ledrightnote{\textcolor{blue}{Theodor Zasche}}}
                als Illustrator – aber das wird wol nicht viel nützen. \uline{Datiren}
                sie jedenfalls die \textcolor{green}{Novelle}{}\ledrightnote{→\textcolor{green}{Die kleine Komödie}}. Man {\pb}soll wissen daß sie \uline{vor}{ }\textcolor{green}{Sterben}{}\ledrightnote{\textcolor{green}{Sterben. Novelle}} geschrieben ist. Daß sie \textcolor{blue}{Fischer}{}\ledrightnote{\textcolor{blue}{Samuel Fischer}} gefällt ist allerdings sehr unheimlich aber vielleicht lügt er.
               Keinesfalls verdient sie es, denn sie hat wirklich viel Grazie\pend
           \pstart
           {\pb}Heute bin ich seelig – 14 Tage
               sind vorbei. Schreiben Sie mir mehr, und öfter, Sie wissen wie sehr ich mich damit
               freue.\pend
           \pstart
           Gute Nacht\pend
           \pstart Ihr \spacefill\mbox{Richard}\pend{}\pstart
           \textcolor{pink}{Časlau}{}\ledrightnote{\textcolor{pink}{Caslau}}{ }14/VI 95\pend
           \endnumbering\briefempfaengerindex{Schnitzler, Arthur@\textsc{Schnitzler, Arthur}!zzzBeer-Hofmann, Richard@\emph{von Richard Beer-Hofmann}!1895-06-141@{14. 6. 1895}|)be}\mylabel{h}  \normalsize

\doendnotes{C}
\bigskip
\vfill

\clearpage

\footnotesize

\lohead{\textsc{register}}

% Definiere theindex-Environment komplett neu ohne reledmac
\makeatletter
\renewenvironment{theindex}{%
  \section*{\indexname}%
  \setlength{\parindent}{0pt}%
  \setlength{\parskip}{0pt plus 0.3pt}%
  \let\item\@idxitem
}{%
  \clearpage
}
\makeatother

\IfFileExists{\jobname-pw.ind}{\input{\jobname-pw.ind}}{}

\end{document}

      