%% latex-korrekturansicht-vorspann.tex
%% Vorspann für die Korrekturansicht.
%% Lädt die gemeinsame Datei latex-vorspann.tex mit gesetztem Schalter.

\newif\ifkorrekturansicht
\korrekturansichttrue

\input{../tex-inputs/latex-vorspann}


\renewcommand{\erwaehntePersonen}{Personen: Paul Goldmann, Olga Schnitzler}
\renewcommand{\erwaehnteOrte}{Orte: Berlin, Gentzgasse, Salzburg, Wien, XVIII., Währing}
\renewcommand{\erwaehnteWerke}{}
\section[ Paul Goldmann an Olga Gussmann, 21. 1. 1903]{Paul Goldmann an Olga Gussmann, 21. 1. 1903}
\nopagebreak\mylabel{v}
\rehead{ }\normalsize\beginnumbering\briefempfaengerindex{Schnitzler, Olga@\textsc{Schnitzler, Olga}!zzzGoldmann, Paul@\emph{von Paul Goldmann}!1903-01-211@{21. 1. 1903}|(be}
\toendnotes[C]{\smallbreak\pagebreak[2]}\Standort{DLA, A:Schnitzler, HS.NZ85.1.5247.}
\physDesc{Postkarte, 274 Zeichen
\newline{}Handschrift: 1) blaue Tinte, deutsche Kurrent\hspace{1em}2) blaue Tinte, lateinische Kurrent (\noindent{}Adresse)\hspace{1em}
\newline{}Versand: Stempel: »\nobreak{}\oindex{Berlin@\textbf{Berlin}, \emph{P.PPLC}|pwk}Berlin, S. W. 11, 21. 1. 03, 12–1N.\nobreak{}«. Stempel: »\nobreak{}\oindex{XVIII., Waehring@\textbf{XVIII., Währing}, \emph{A.ADM3}|pwk}18/1 Wien 110, 22. 1. 03, 10. V, Bestellt\nobreak{}«.  }\toendnotes[C]{\smallbreak}\pstart{}{\pb}Frau\pend{}\pstart{}Olga Gussmann\pend{}\pstart{}\textcolor{pink}{Wien}{}\ledrightnote{\textcolor{pink}{Wien}}\pend{}\pstart{}\textcolor{pink}{Gentzgaſse 110}{}\ledrightnote{\textcolor{pink}{Gentzgasse}}\pend{}\pstart{}\textcolor{pink}{Währing}{}\ledrightnote{\textcolor{pink}{XVIII., Währing}}\pend{}
{\bigskip}
\pstart
           \noindent{}{\pb}\textcolor{pink}{Berlin}{}\ledrightnote{\textcolor{pink}{Berlin}}, 21. Januar. Liebe Freundin, Ich danke vielm\substVorne{}\textsuperscript{ä}\substDazwischen{}a\substHinten{}ls für die Grüße aus \label{K_L03529-1v}\edtext{\textcolor{pink}{Salzburg}{}\ledrightnote{\textcolor{pink}{Salzburg}}}{\lemma{\textnormal{\emph{Salzburg}}}\Cendnote{\textnormal{Die Postkarte, die 
                  \textcolor{blue}{Arthur Schnitzler} und \textcolor{blue}{Olga Gussmann} 
                  von ihrem Aufenthalt in \textcolor{pink}{Salzburg} zwischen 12. 1. 1903 und 19. 1. 1903 gesandt haben, ist, wie nahezu alle Korrespondenzstücke \textcolor{blue}{Schnitzler}s an \textcolor{blue}{Goldmann} überhaupt,
                  nicht erhalten.}}}\label{K_L03529-1h}, die ich Ihnen und \textcolor{blue}{Arthur}{}\ledrightnote{} herzlichſt erwidere. Ich ſtecke tief in
               der Arbeit. Warum höre ich gar nichts mehr von \textcolor{blue}{Arthur}{}\ledrightnote{}?\pend
           
\pstart
           Herzlichſt Ihr getreuer {\\[\baselineskip]}\spacefill\mbox{Paul Goldmann}\pend
           \leftskip=0em{}\endnumbering\briefempfaengerindex{Schnitzler, Olga@\textsc{Schnitzler, Olga}!zzzGoldmann, Paul@\emph{von Paul Goldmann}!1903-01-211@{21. 1. 1903}|)be}\mylabel{h}  \normalsize

\doendnotes{C}
\bigskip
\vfill

\clearpage

\footnotesize

\lohead{\textsc{register}}

% Definiere theindex-Environment komplett neu ohne reledmac
\makeatletter
\renewenvironment{theindex}{%
  \section*{\indexname}%
  \setlength{\parindent}{0pt}%
  \setlength{\parskip}{0pt plus 0.3pt}%
  \let\item\@idxitem
}{%
  \clearpage
}
\makeatother

\IfFileExists{\jobname-pw.ind}{\input{\jobname-pw.ind}}{}

\end{document}

      