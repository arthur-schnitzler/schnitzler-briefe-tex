%% latex-korrekturansicht-vorspann.tex
%% Vorspann für die Korrekturansicht.
%% Lädt die gemeinsame Datei latex-vorspann.tex mit gesetztem Schalter.

\newif\ifkorrekturansicht
\korrekturansichttrue

\input{../tex-inputs/latex-vorspann}


               \section[Robert Adam an Arthur Schnitzler, 1. 6. 1915]{ Robert Adam an Arthur Schnitzler, 1. 6. 1915}\nopagebreak\mylabel{v}\rehead{ }\normalsize\beginnumbering\briefempfaengerindex{Schnitzler, Arthur@\textsc{Schnitzler, Arthur}!zzzAdam, Robert@\emph{von Robert Adam}!1915-06-011@{1. 6. 1915}|(be} \toendnotes[C]{\smallbreak\pagebreak[2]} \Standort{DLA, A:Schnitzler, HS.NZ85.1.4230,8.}
\physDesc{Brief, 1 Blatt, 3 Seiten
\newline{}Handschrift: schwarze Tinte, deutsche Kurrent
\newline{}Schnitzler: 1) mit Bleistift beschriftet: »\textsc{Adam}« 2) mit rotem Buntstift eine Unterstreichung}\Standort{Wien, Österreichische Nationalbibliothek, Cod.ser. 52.267, 88–89.}
\physDesc{maschinelle Abschrift
\newline{}Schreibmaschine}\pstart
           \raggedleft{}{\pb}\textcolor{pink}{Ziſtersdorf}{}\ledrightnote{\textcolor{pink}{Zistersdorf}}, am 1. Juni 1915\pend
           \pstart{}Hochverehrter Herr Doktor!\pend\pstart
           Sie haben an mehreren meiner literariſchen Produktionen, zuerſt an der »\textcolor{green}{Geſchichte des \textsc{Alî ibn
                            Bekkâr}}{}\ledrightnote{\textcolor{green}{Die Geschichte des Alî ibn Bekkâr mit Schams an-Nahâr}}«, dann am »\textcolor{green}{\textsc{Neidhard}}{}\ledrightnote{\textcolor{green}{Neidhard}}« und zuletzt an der Studie »\textcolor{green}{\textsc{Fatme}}{}\ledrightnote{\textcolor{green}{Fatme}}«, einen mich derart ermutigenden Anteil genommen, daß ich es heute wage,
                    Ihnen die beifolgenden ſechs Szenen, die ich unter dem Titel »\textcolor{green}{\textsc{\uline{Der Fremde}}}{}\ledrightnote{\textcolor{green}{Der Fremde}}« zuſammenfaſſen möchte, mit der ergebenen Bitte zu überſenden, Sie möchten
                    dies von mir ſelbſt nicht allzu geſchickt und ebenmäßig angefertigte Manuſkript
                    einer Durchſicht würdigen und, falls Sie der Inhalt nicht abſtößt, Ihrer
                    Manuſkript-Sammlung einreihen.\pend
           \pstart
           Dieſe ſeltſame Bitte richte ich deswe{\pb}gen an Sie,
                    hochverehrter Herr Doktor, weil ich nicht bloß wegen der Zeitverhältniſſe und
                    wegen des Mißgeſchicks, das mich bei jedem Verſuch, in die Deutſche Literatur
                    einzudringen, beharrlich verfolgt, ſondern wegen des beſonderen ärgerlichen
                    Inhalts der vorliegenden Arbeit kaum hoffen darf, ſie in abſehbarer Zeit in
                    Buchform zu leſen und Ihnen ſenden zu können, anderſeits aber mein ſehnlicher
                    Wunſch dahin geht, eine Produktion, die mir ſelber ſehr am Herzen liegt, dem
                    Manne zur Verfügung zu ſtellen, an deſſen Urteil und Würdigung mir am
                    allermeiſten gelegen iſt.\pend
           \pstart
           Hinzu kommt noch die Erwägung, daß ſich »\textcolor{green}{Der
                        Fremde}{}\ledrightnote{\textcolor{green}{Der Fremde}}« der Idee nach als drittes Stück der »\textcolor{green}{Geſchichte des \textsc{Alî ibn Bekkâr}}{}\ledrightnote{\textcolor{green}{Die Geschichte des Alî ibn Bekkâr mit Schams an-Nahâr}}« und dem »\textcolor{green}{\textsc{Neidhard}}{}\ledrightnote{\textcolor{green}{Neidhard}}« anreiht, die Sie, hochverehrter Herr Doktor, bereits kennen, indem er den
                    Gedankenkreis der beiden Komödien abſchließt, und daß es mir daher angelegen
                    ſein muß, Ihnen auch das letzte Stück, das ſich mit {\pb}dem Problem der Liebe beſchäftigt, mitzuteilen.
                    Daß es eine ſonderbare Art Drama darſtellt, muß ich zugeben: der äußeren
                    Handlung nach – wenn von einer solchen bei ihm überhaupt die Rede ſein darf –
                    mag es ſich wie die Expoſition einer Tragödie ausnehmen, der Idee nach aber iſt
                    die Tragödie in ihm bereits abgeſchloſſen – die Tragödie oder die Komödie, wie
                    man’s nehmen mag. –\pend
           \pstart
           Verzeihen Sie mir, wie nun ſchon ſo oft, auch diesmal meine Zudringlichkeit und
                    genehmigen Sie die Verſicherung meiner Dankbarkeit und Hochachtung.\pend
           \pstart
           Ihr ſehr ergebener{\\[\baselineskip]}\spacefill\mbox{Robert Adam}\pend
           \leftskip=0em{}\pstart
           \noindent{}\raggedleft{}(D\textsuperscript{r} Rob. Ad. Pollak,{\\}kk. Bez. Richter,{\\}\textcolor{pink}{Ziſtersdorf}{}\ledrightnote{\textcolor{pink}{Zistersdorf}})\pend
           \endnumbering\briefempfaengerindex{Schnitzler, Arthur@\textsc{Schnitzler, Arthur}!zzzAdam, Robert@\emph{von Robert Adam}!1915-06-011@{1. 6. 1915}|)be}\mylabel{h}  \normalsize

\doendnotes{C}
\bigskip
\vfill

\clearpage

\footnotesize

\lohead{\textsc{register}}

% Definiere theindex-Environment komplett neu ohne reledmac
\makeatletter
\renewenvironment{theindex}{%
  \section*{\indexname}%
  \setlength{\parindent}{0pt}%
  \setlength{\parskip}{0pt plus 0.3pt}%
  \let\item\@idxitem
}{%
  \clearpage
}
\makeatother

\IfFileExists{\jobname-pw.ind}{\input{\jobname-pw.ind}}{}

\end{document}

      