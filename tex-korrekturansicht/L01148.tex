%% latex-korrekturansicht-vorspann.tex
%% Vorspann für die Korrekturansicht.
%% Lädt die gemeinsame Datei latex-vorspann.tex mit gesetztem Schalter.

\newif\ifkorrekturansicht
\korrekturansichttrue

\input{../tex-inputs/latex-vorspann}


               \section[Arthur Schnitzler an Richard Beer-Hofmann, 1{[}7?{]}. 7. 1901]{ Arthur Schnitzler an Richard Beer-Hofmann, 1{[}7?{]}. 7. 1901}\nopagebreak\mylabel{v}\rehead{ }\normalsize\beginnumbering\briefempfaengerindex{Beer-Hofmann, Richard@\textsc{Beer-Hofmann, Richard}!zzzSchnitzler, Arthur@\emph{von Arthur Schnitzler}!1901-07-171@{1{[}7?{]}. 7. 1901}|(be} \toendnotes[C]{\smallbreak\pagebreak[2]} \Standort{YCGL, MSS 31.}
\physDesc{Bildpostkarte
\newline{}Handschrift: Bleistift, deutsche Kurrent\newline{}Versand: 1) Stempel: »\nobreak{}\oindex{Vahrn@\textbf{Vahrn}, \emph{Besiedelter Ort (A.BSO)}|pwk}Va\textcolor{gray}{hr}n , 1\textcolor{gray}{7} {[}7. 1901{]}\nobreak{}«.  2) Stempel: »\nobreak{}\oindex{Poertschach@\textbf{Pörtschach}, \emph{https://www.geonames.org/ontologyP.PPL}|pwk}Pörtschach am
                                                  See, 18/7 01\nobreak{}«. }\pstart{}{\pb}Herrn Dr. \textsc{Rich. Beer
                            Hofmann}\pend{}\pstart{}\textcolor{pink}{\textsc{Pörtschach}}{}\ledrightnote{\textcolor{pink}{Pörtschach}}\pend{}\pstart{}\textcolor{pink}{\textsc{Villa Arnstein}}{}\ledrightnote{\textcolor{pink}{Villa Arnstein}}\pend{}{\bigskip}\pstart
           \noindent{}{\pb}\textcolor{gray}{\textbf{Gruss aus \textcolor{pink}{Vahrn}{}\ledrightnote{\textcolor{pink}{Vahrn}}.}}\pend
           \pstart
           Vielen Dank für den Brief. Antwort demnächſt.\pend
           \pstart
           Hier iſt es heiſs – aber warm!\footnote{} Ihr Andenken lebt fort, wie das der Ihren.\pend
           \pstart
           Herzlichst Ihr{\\[\baselineskip]}\spacefill\mbox{A.}\pend
           \leftskip=0em{}\endnumbering\briefempfaengerindex{Beer-Hofmann, Richard@\textsc{Beer-Hofmann, Richard}!zzzSchnitzler, Arthur@\emph{von Arthur Schnitzler}!1901-07-171@{1{[}7?{]}. 7. 1901}|)be}\mylabel{h}  \normalsize

\doendnotes{C}
\bigskip
\vfill

\clearpage

\footnotesize

\lohead{\textsc{register}}

% Definiere theindex-Environment komplett neu ohne reledmac
\makeatletter
\renewenvironment{theindex}{%
  \section*{\indexname}%
  \setlength{\parindent}{0pt}%
  \setlength{\parskip}{0pt plus 0.3pt}%
  \let\item\@idxitem
}{%
  \clearpage
}
\makeatother

\IfFileExists{\jobname-pw.ind}{\input{\jobname-pw.ind}}{}

\end{document}

      