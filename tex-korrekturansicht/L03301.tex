%% latex-korrekturansicht-vorspann.tex
%% Vorspann für die Korrekturansicht.
%% Lädt die gemeinsame Datei latex-vorspann.tex mit gesetztem Schalter.

\newif\ifkorrekturansicht
\korrekturansichttrue

\input{../tex-inputs/latex-vorspann}


\renewcommand{\erwaehntePersonen}{Personen: Samuel Fischer, Georg Hirschfeld}
\renewcommand{\erwaehnteInstitutionen}{Institutionen: S. Fischer Verlag, Volkstheater}
\renewcommand{\erwaehnteOrte}{Orte: Berlin, Wien, Österreich}
\renewcommand{\erwaehnteWerke}{Werke: Der Gemeine. Schauspiel in drei Aufzügen, Der Hinterbliebene. Kurze Novellen}
\section[ Felix Salten an Arthur Schnitzler, 9. 10. 1899]{Felix Salten an Arthur Schnitzler, 9. 10. 1899}
\nopagebreak\mylabel{v}
\rehead{ }\normalsize\beginnumbering\briefempfaengerindex{Schnitzler, Arthur@\textsc{Schnitzler, Arthur}!zzzSalten, Felix@\emph{von Felix Salten}!1899-10-091@{9. 10. 1899}|(be}
\toendnotes[C]{\smallbreak\pagebreak[2]}\Standort{CUL, Schnitzler, B 89, A 2.}
\physDesc{Brief, 1 Blatt, 2 Seiten, 969 Zeichen
\newline{}Handschrift: schwarze Tinte, lateinische Kurrent
\newline{}Ordnung: mit Bleistift von unbekannter Hand nummeriert: »125« }\toendnotes[C]{\smallbreak}
\pstart
           \raggedleft{}{\pb}\textcolor{pink}{Wien}{}\ledrightnote{\textcolor{pink}{Wien}}, 9. X. 99.\pend
           
\pstart
           Lieber Freund, von \textcolor{blue}{Hirschfeld}{}\ledrightnote{\textcolor{blue}{Georg Hirschfeld}} erfahre ich, dass Sie jetzt in \label{K_L03301-1v}\edtext{\textcolor{pink}{Berlin}{}\ledrightnote{\textcolor{pink}{Berlin}}}{\lemma{\textnormal{\emph{Berlin}}}\Cendnote{\textnormal{\textcolor{blue}{Schnitzler} war zwischen 4. 10. 1899 und 11. 10. 1899 in \textcolor{pink}{Berlin}.}}}\label{K_L03301-1h} sind, und da fällt es mir ein,
               ob Sie nicht jetzt Gelegenheit hätten, mit \textcolor{blue}{Fischer}{}\ledrightnote{\textcolor{blue}{Samuel Fischer}} ein Wort über meine \textcolor{green}{Novellen}{}\ledrightnote{{$\rightarrow$}\textcolor{green}{Der Hinterbliebene. Kurze Novellen}} zu sprechen d. h. wenn es Ihnen sonst passt, und
               wenn es im Übrigen Ihre Meinung ist, dass die \label{K_L03301-2v}\edtext{Novellen gut}{\lemma{\textnormal{\emph{Novellen gut}}}\Cendnote{\textnormal{siehe Arthur Schnitzler an Felix Salten, 4. 9. 1899}}}\label{K_L03301-2h} sind. Um was ich Sie bitten würde, wäre eben nicht die »Empfehlung«, sondern,
               wenn die übrigen Umstände es erlauben, eine intensivere Intervention. Ich möchte \uline{sehr gerne}{ }\label{K_L03301-3v}\edtext{bei \textcolor{brown}{Fischer}{}\ledrightnote{\textcolor{brown}{S. Fischer Verlag}} verlegt}{\lemma{\textnormal{\emph{bei Fischer verlegt}}}\Cendnote{\textnormal{siehe Felix Salten an Arthur Schnitzler, [29. 8. 1899]}}}\label{K_L03301-3h} werden, möchte aber von \textcolor{blue}{Fischer}{}\ledrightnote{\textcolor{blue}{Samuel Fischer}} keinen
               Korb bekommen. Vielleicht macht es etwas bei ihm aus, wenn Sie ihm sagen, dass in den
                  \label{K_L03301-4v}\edtext{nächsten Monaten ein \textcolor{green}{Stück}{}\ledrightnote{{$\rightarrow$}\textcolor{green}{Der Gemeine. Schauspiel in drei Aufzügen}} von mir am \textcolor{brown}{Volkstheater}{}\ledrightnote{\textcolor{brown}{Volkstheater}}}{\lemma{\textnormal{\emph{nächsten … Volkstheater}}}\Cendnote{\textnormal{Das \emph{\textcolor{brown}{Deutsche Volkstheater}} hatte \emph{\textcolor{green}{Der
                     Gemeine}} angenommen. Aufgrund der antimilitaristischen Haltung des \textcolor{green}{Stück}s wurde es in \textcolor{pink}{Österreich} jedoch erst 1919 aufgeführt.}}}\label{K_L03301-4h} kommt. Bitte, schreiben Sie mir ein Wort, ob
               Sie das thun können, nur bitte, sagen Sie {\pb}es mir ganz ruhig, wenn Sie’s
               aus irgend einem Grund nicht gerne thun möchten.\pend
           
\pstart
           Hoffentlich sind Sie \label{K_L03301-5v}\edtext{bald wieder
                  hier}{\lemma{\textnormal{\emph{bald wieder
                  hier}}}\Cendnote{\textnormal{\textcolor{blue}{Schnitzler} kehrte am 12. 10. 1899 nach \textcolor{pink}{Wien} zurück.}}}\label{K_L03301-5h}. \label{K_L03301-6v}\edtext{\textcolor{blue}{Hirschfeld}{}\ledrightnote{\textcolor{blue}{Georg Hirschfeld}} erzählt nur, dass er Sie ganz
               erfüllt von Arbeit angetroffen}{\lemma{\textnormal{\emph{Hirschfeld … angetroffen}}}\Cendnote{\textnormal{siehe A. S.: \emph{Tagebuch}, 5. 10. 1899}}}\label{K_L03301-6h} hat; ich freue mich sehr darüber.\pend
           
\pstart
           Herzlichst Ihr {\\[\baselineskip]}\spacefill\mbox{Salten}\pend
           \leftskip=0em{}\endnumbering\briefempfaengerindex{Schnitzler, Arthur@\textsc{Schnitzler, Arthur}!zzzSalten, Felix@\emph{von Felix Salten}!1899-10-091@{9. 10. 1899}|)be}\mylabel{h}  \normalsize

\doendnotes{C}
\bigskip
\vfill

\clearpage

\footnotesize

\lohead{\textsc{register}}

% Definiere theindex-Environment komplett neu ohne reledmac
\makeatletter
\renewenvironment{theindex}{%
  \section*{\indexname}%
  \setlength{\parindent}{0pt}%
  \setlength{\parskip}{0pt plus 0.3pt}%
  \let\item\@idxitem
}{%
  \clearpage
}
\makeatother

\IfFileExists{\jobname-pw.ind}{\input{\jobname-pw.ind}}{}

\end{document}

      