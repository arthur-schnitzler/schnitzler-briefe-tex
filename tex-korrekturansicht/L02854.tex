%% latex-korrekturansicht-vorspann.tex
%% Vorspann für die Korrekturansicht.
%% Lädt die gemeinsame Datei latex-vorspann.tex mit gesetztem Schalter.

\newif\ifkorrekturansicht
\korrekturansichttrue

\input{../tex-inputs/latex-vorspann}


               \section[ Paul Goldmann an Arthur Schnitzler, 24. 8. {[}1898{]}]{Paul Goldmann an Arthur Schnitzler, 24. 8. {[}1898{]}}\nopagebreak\mylabel{v}\rehead{ }\normalsize\beginnumbering\briefempfaengerindex{Schnitzler, Arthur@\textsc{Schnitzler, Arthur}!zzzGoldmann, Paul@\emph{von Paul Goldmann}!1898-08-241@{24. 8. {[}1898{]}}|(be} \toendnotes[C]{\smallbreak\pagebreak[2]} \Standort{DLA, A:Schnitzler, HS.NZ85.1.3168.}
\physDesc{Brief, 2 Blätter, 7 Seiten
\newline{}Handschrift: blaue Tinte, deutsche Kurrent
\newline{}Schnitzler: 1) mit Bleistift das Jahr »98« vermerkt 2) mit rotem Buntstift vier Unterstreichungen}\toendnotes[C]{\smallbreak}\pstart
           \raggedleft{}{\pb}\textsc{\textcolor{pink}{Tschifu}{}\ledrightnote{\textcolor{pink}{Yantai}}}, 24. Auguſt.\pend
           \pstart\center{}Mein lieber Freund,\pend\pstart
           Hier erhielt ich Deine lieben Briefe vom 28. Juni u.
               vom 10. Juli. Ich hoffe, daß Deine \label{K_L02854-1v}\edtext{Reiſe}{\lemma{\textnormal{\emph{Reiſe}}}\Cendnote{\textnormal{siehe Paul Goldmann an Arthur Schnitzler, 16. 5. 1898}}}\label{K_L02854-1h} Dir Erfriſchung und Abziehung von Deinen trüben, Deinen ſo unnöthig trüben
               Gedanken gebracht hat. Wie gern wäre ich \strikeout{\textcolor{gray}{mt}} mitgekommen, wie alljährlich! Hoffentlich können wir \label{K_L02854-2v}\edtext{nächſtes Jahr}{\lemma{\textnormal{\emph{nächſtes Jahr}}}\Cendnote{\textnormal{Sie sahen sich bereits Anfang des
                  nächsten Jahres wieder: \textcolor{blue}{Goldmann} überraschte \textcolor{blue}{Schnitzler} am
                     14. 1. 1899 mit
                  einem Besuch in \textcolor{pink}{Wien}.}}}\label{K_L02854-2h} wieder zuſammen
               ſein.\pend
           \pstart
           Mit wahrer Freude habe ich aus Deinen lieben Briefen geſehen, wie reich das
               literariſche Erträgniß dieſes Jahres für Dich ſein wird.
               Wenn Dich Deine Hypochondrie {\pb}ſo arbeitſam macht, ſo
               will ich mich recht gern mit ihr \strikeout{abf\textcolor{gray}{i}nden} abfinden. Dieſer Brief erreicht Dich
               wahrſcheinlich ſchon nach der \label{K_L02854-3v}\edtext{\begin{otherlanguage}{french}\textsc{\textcolor{green}{Première}{}\ledrightnote{→\textcolor{green}{Das Vermächtnis. Schauspiel in drei Akten}}}\end{otherlanguage}{ }in \textcolor{pink}{Berlin}{}\ledrightnote{\textcolor{pink}{Berlin}}}{\lemma{\textnormal{\emph{Première in Berlin}}}\Cendnote{\textnormal{Die Uraufführung von \emph{\textcolor{green}{Das Vermächtnis}} fand am 8. 10. 1898 am \textcolor{pink}{Deutschen Theater} in \textcolor{pink}{Berlin} statt und
                  war ein Erfolg.}}}\label{K_L02854-3h}, und ich bin überzeugt, daß Du \strikeout{\textcolor{gray}{×}\-\textcolor{gray}{×}\-\textcolor{gray}{×}\-\textcolor{gray}{×}{ }\textcolor{gray}{×}\-\textcolor{gray}{×}\-\textcolor{gray}{×}\-\textcolor{gray}{×}\-\textcolor{gray}{×}\-\textcolor{gray}{×}\-\textcolor{gray}{×}\-\textcolor{gray}{×}} einen neuen ſchönen Erfolg erringen wirſt, zu dem ich Dich im Voraus von
               ganzem Herzen beglückwünſche. Der Titel des \textcolor{green}{Stück}{}\ledrightnote{→\textcolor{green}{Das Vermächtnis. Schauspiel in drei Akten}}es iſt vielverſprechend. Aber was ſteht darin? \strikeout{Sob} Sobald Du nur irgend kannſt, ſendeſt Du mir ein
               Exemplar, nicht wahr? Deine Idee, ein \label{K_L02854-12v}\edtext{\textcolor{green}{Renaiſſance-Stück}{}\ledrightnote{→\textcolor{green}{Der Schleier der Beatrice. Schauspiel in fünf Akten}}}{\lemma{\textnormal{\emph{Renaiſſance-Stück}}}\Cendnote{\textnormal{siehe A. S.: \emph{Tagebuch}, 5. 7. 1898}}}\label{K_L02854-12h} zu {\pb}ſchreiben, gefällt mir weniger. Mir
               kommt \substVorne{}\textsuperscript{vor,}\substDazwischen{}vor,\substHinten{} als würde Dir das nicht liegen, und ſeit die \textsc{Renaissance} von den \textsc{\textcolor{blue}{Bahr}{}\ledrightnote{\textcolor{blue}{Hermann Bahr}}} und \textsc{\textcolor{blue}{Hofmannsthal}{}\ledrightnote{\textcolor{blue}{Hugo von Hofmannsthal}}} zum \label{K_L02854-4v}\edtext{Dogma}{\lemma{\textnormal{\emph{Dogma}}}\Cendnote{\textnormal{\textcolor{blue}{Hermann
                     Bahr} hatte seine jüngste Sammlung von Kritiken \emph{\textcolor{green}{Renaissance. Neue Studien zur Kritik
                        der Moderne}} (\textcolor{pink}{Berlin}: \emph{\textcolor{brown}{S. Fischer}}{ }1897) betitelt. \textcolor{blue}{Hofmannsthal}
                  hatte in seinem Essay \emph{\textcolor{green}{Über moderne englische Malerei. Rückblick auf die internationale Ausstellung Wien 1894}} 
                  und seinem Dramenfragment \emph{\textcolor{green}{Der Tod des Tizian}} sein Interesse
                  an der Renaissance kundgetan.}}}\label{K_L02854-4h} erhoben worden iſt, iſt ſie mir verleidet.
               Wenn Dich die \strikeout{alte}{ }\strikeout{alten} alten Zeiten locken, was ich begreife, ſo
               ſchreibe Du ein \label{K_L02854-112v}\edtext{Alt-\textcolor{pink}{Wien}{}\ledrightnote{\textcolor{pink}{Wien}}er-Stück}{\lemma{\textnormal{\emph{Alt-Wiener-Stück}}}\Cendnote{\textnormal{Gemeint ist damit das
                  \textcolor{pink}{Wien} vor der Stadterneuerung durch die \textcolor{pink}{Ringstraße}nbauten. Am ehesten kann \emph{\textcolor{green}{Der junge Medardus}} (1910) als Alt-\textcolor{pink}{Wien}er Stück gelten.}}}\label{K_L02854-112h}. Ich meine, Du
               könnteſt da etwas Entzückendes machen. Folge mir und laſſe Dich von den Zünftlern
               nicht aus Deinem Leben und Deiner Wärme ins »Literariſche« hineinlocken!\pend
           \pstart
           {\pb}Wann ich zurück komme? Ich habe keine Ahnung. Wenn
               ich im ſelben Tempo fortarbeite, kann der nächſte Sommer herankommen. Denn ich
               arbeite qualvoll ſchwer, da ich es ſo gern vermeiden möchte, Banalitäten zu ſagen,
               und ſitze über einem Feuilleton manchmal 14 Tage. Freilich beginne ich die Geſchichte
               ſatt zu bekommen, – die ewige Feuilleton-Schmiererei ebenſo wie den Miſthaufen \textcolor{pink}{China}{}\ledrightnote{\textcolor{pink}{China}}; und da \strikeout{i\textcolor{gray}{c}h} auch meine Familie auf Abkürzung meiner Reiſe {\pb}dringt, ſo könnte es geſchehen, daß ich nach \textsc{\textcolor{pink}{Peking}{}\ledrightnote{\textcolor{pink}{Peking}}} einfach kurz abbreche und heimkehre, ohne \textcolor{pink}{Japan}{}\ledrightnote{\textcolor{pink}{Japan}} geſehen zu haben. Das wäre ein ſchweres Opfer, aber es iſt nicht
               unmöglich, daß ich es bringen muß. In dieſem Falle wäre ich etwa im Februar wieder in \textcolor{pink}{Europa}{}\ledrightnote{\textcolor{pink}{Europa}}. Jedenfalls bitte ich Dich, mir nur noch bis \uline{Ende Oktober} nach \textsc{\textcolor{pink}{Shanghai}{}\ledrightnote{\textcolor{pink}{Shanghai}}} zu ſchreiben. Was \uline{bis zum 20. Oktober} von \textsc{\textcolor{pink}{Wien}{}\ledrightnote{\textcolor{pink}{Wien}}} abgeht, erreicht mich ſicher noch in \textcolor{pink}{China}{}\ledrightnote{\textcolor{pink}{China}}. {\pb}\substVorne{}\textsuperscript{v}\substDazwischen{}V\substHinten{}on da ab bitte ich Dich, alle Deine \strikeout{lieben}
               lieben Briefe meiner \textcolor{blue}{Mutter}{}\ledrightnote{→\textcolor{blue}{Clementine Goldmann}}
               zu ſenden (\textsc{\textcolor{pink}{Frankfurt am Main, \strikeout{Rossert} Rossertstraße 15}{}\ledrightnote{\textcolor{pink}{Rossertstraße}}}), welche \strikeout{all\textcolor{gray}{e}s} immer meine
               Adreſſe kennen und mir Alles nachſenden wird.\pend
           \pstart
           Willſt Du glauben, daß \textsc{\textcolor{blue}{Richard}{}\ledrightnote{\textcolor{blue}{Richard Beer-Hofmann}}} mir mit keiner Sylbe ſeine \label{K_L02854-7v}\edtext{Verheirathung}{\lemma{\textnormal{\emph{Verheirathung}}}\Cendnote{\textnormal{siehe Paul Goldmann an Arthur Schnitzler, 26. 6. [1898]}}}\label{K_L02854-7h} angezeigt hat? Es gibt Fälle, wo man ſchreiben muß, ſelbſt wenn man niemals
               ſchreibt. Und mich kränkt {\pb}beſonders der Gedanke,
               daß er weder Dich noch den jungen Herrn \textcolor{blue}{von \textsc{Hoffmannsthal}}{}\ledrightnote{\textcolor{blue}{Hugo von Hofmannsthal}} in dieſer Weiſe vernachläſſigt haben würde. \label{K_L02854-11v}\edtext{\begin{otherlanguage}{french}\textsc{Avec moi, on en prend à son aise!}\end{otherlanguage}}{\lemma{\textnormal{\emph{Avec … aise!}}}\Cendnote{\textnormal{französisch, etwa: Mit mir muss man es
                  nicht so genau nehmen!}}}\label{K_L02854-11h}\pend
           \pstart
           Das iſt aber nur zwiſchen Dir und mir geſagt, und Du ſollſt ihm, wie \textsc{\textcolor{blue}{Leo}{}\ledrightnote{\textcolor{blue}{Leo Van-Jung}}}
                die herzlichſten Grüße von mir übermitteln.\pend
           \pstart
           Auch Dir, mein lieber Freund, herzlichſte und treueſte Grüße!\pend
           \pstart
           Dein{\\[\baselineskip]}\spacefill\mbox{\strikeout{Paul} Paul Goldmann}\pend
           \leftskip=0em{}\pstart
           \noindent{}Viele Grüße an Deine \textcolor{blue}{Freundin}{}\ledrightnote{→\textcolor{blue}{Marie Reinhard}}!\pend
           \endnumbering\briefempfaengerindex{Schnitzler, Arthur@\textsc{Schnitzler, Arthur}!zzzGoldmann, Paul@\emph{von Paul Goldmann}!1898-08-241@{24. 8. {[}1898{]}}|)be}\mylabel{h}\begin{anhang}\end{anhang}\normalsize

\doendnotes{C}
\bigskip
\vfill

\clearpage

\footnotesize

\lohead{\textsc{register}}

% Definiere theindex-Environment komplett neu ohne reledmac
\makeatletter
\renewenvironment{theindex}{%
  \section*{\indexname}%
  \setlength{\parindent}{0pt}%
  \setlength{\parskip}{0pt plus 0.3pt}%
  \let\item\@idxitem
}{%
  \clearpage
}
\makeatother

\IfFileExists{\jobname-pw.ind}{\input{\jobname-pw.ind}}{}

\end{document}

      