%% latex-korrekturansicht-vorspann.tex
%% Vorspann für die Korrekturansicht.
%% Lädt die gemeinsame Datei latex-vorspann.tex mit gesetztem Schalter.

\newif\ifkorrekturansicht
\korrekturansichttrue

\input{../tex-inputs/latex-vorspann}


               \section[Arthur Schnitzler an Robert Adam, 25. 5. 1921]{ Arthur Schnitzler an Robert Adam, 25. 5. 1921}\nopagebreak\mylabel{v}\rehead{ }\normalsize\beginnumbering\briefempfaengerindex{Adam, Robert@\textsc{Adam, Robert}!zzzSchnitzler, Arthur@\emph{von Arthur Schnitzler}!1921-05-251@{25. 5. 1921}|(be} \toendnotes[C]{\smallbreak\pagebreak[2]} \Standort{DLA, 96.34.2/26.}
\physDesc{Postkarte
\newline{}Handschrift: schwarze Tinte, deutsche Kurrent\newline{}Versand: 1) Stempel: »\nobreak{}\oindex{I., Innere Stadt@\textbf{I., Innere Stadt}, \emph{Bezirk (A.BZK)}|pwk}1/1 Wien 8, 25. V. 21, 5\nobreak{}«.  2) zusätzlicher Stempel: »\noindent{}›HELFT \textcolor{pink}{ÖSTERREICHS} KINDERN!‹{ / }\textcolor{brown}{AMERIK. KINDERHILFSAKTION}{ / }\textcolor{pink}{WIEN I.}{ / }\textcolor{pink}{ELISABETHSTR. 9}«, dieser auch in englischer Sprache gestempelt, aber nur
                                 bruchstückhaft entzifferbar3) die falsche Bezirksangabe in der Empfängeradresse wurde von
                                 unbekannter Hand mit Bleistift zu »XII«
                                 korrigiert.}\pstart{}{\pb}A. S. \textcolor{pink}{Wien XVIII. \textsc{Sternwartestr} 71}{}\ledrightnote{\textcolor{pink}{Sternwartestraße}}\pend{}{\bigskip}\pstart{}Herrn Ob. Landesgerichtsrat\pend{}\pstart{}\textsc{Dr. Robert Adam Pollak}\pend{}\pstart{}\textcolor{pink}{\textsc{Wien XIII}}{}\ledrightnote{\textcolor{pink}{XIII., Hietzing}}\pend{}\pstart{}\textcolor{pink}{58 \textsc{Meidlinger Hauptstr}
                     58}{}\ledrightnote{\textcolor{pink}{Meidlinger Hauptstraße}}\pend{}{\bigskip}\pstart
           \raggedleft{}{\pb}25. 5. 1921\pend
           \pstart{}Verehrtester Herr Doktor,\pend\pstart
           wenn Sie denn am Dienſtag (31. 5.) gegen Abend nach 6 für
               mich Zeit hätten, wäre mir Ihr lieber Beſuch ſehr willko{\geminationm}en. In der Zwiſchenzeit war ich auch verreiſt, \textcolor{pink}{München}{}\ledrightnote{\textcolor{pink}{München}} u. \textcolor{pink}{Salzburg}{}\ledrightnote{\textcolor{pink}{Salzburg}}.\pend
           \pstart
           Herzlichſt grüßend Ihr{\\[\baselineskip]}ſehr ergebener{\\[\baselineskip]}\spacefill\mbox{Arthur Schnitzler}\pend
           \leftskip=0em{}\endnumbering\briefempfaengerindex{Adam, Robert@\textsc{Adam, Robert}!zzzSchnitzler, Arthur@\emph{von Arthur Schnitzler}!1921-05-251@{25. 5. 1921}|)be}\mylabel{h}  \normalsize

\doendnotes{C}
\bigskip
\vfill

\clearpage

\footnotesize

\lohead{\textsc{register}}

% Definiere theindex-Environment komplett neu ohne reledmac
\makeatletter
\renewenvironment{theindex}{%
  \section*{\indexname}%
  \setlength{\parindent}{0pt}%
  \setlength{\parskip}{0pt plus 0.3pt}%
  \let\item\@idxitem
}{%
  \clearpage
}
\makeatother

\IfFileExists{\jobname-pw.ind}{\input{\jobname-pw.ind}}{}

\end{document}

      