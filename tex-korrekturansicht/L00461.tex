%% latex-korrekturansicht-vorspann.tex
%% Vorspann für die Korrekturansicht.
%% Lädt die gemeinsame Datei latex-vorspann.tex mit gesetztem Schalter.

\newif\ifkorrekturansicht
\korrekturansichttrue

\input{../tex-inputs/latex-vorspann}


               \section[Arthur Schnitzler an Richard Beer-Hofmann, {[}10. 7. 1895?{]}]{ Arthur Schnitzler an Richard Beer-Hofmann, {[}10. 7. 1895?{]}}\nopagebreak\mylabel{v}\rehead{ }\normalsize\beginnumbering\briefempfaengerindex{Beer-Hofmann, Richard@\textsc{Beer-Hofmann, Richard}!zzzSchnitzler, Arthur@\emph{von Arthur Schnitzler}!1895-07-101@{{[}10. 7. 1895?{]}}|(be} \toendnotes[C]{\smallbreak\pagebreak[2]} \Standort{YCGL, MSS 31.}
\physDesc{Briefkarte
\newline{}Handschrift: Bleistift, deutsche Kurrent}\buchAbdrucke{\weitereDrucke{Arthur Schnitzler, Richard Beer-Hofmann: \emph{Briefwechsel 1891–1931}. Hg. Konstanze Fliedl. Wien, Zürich: \emph{Europaverlag} 1992, S. 78.} }\pstart
           \noindent{}{\pb}Lieber Richard, ich bin noch in \textcolor{pink}{\textsc{Marienbad}}{}\ledrightnote{\textcolor{pink}{Marienbad}}. Vielleicht ko{\geminationm} ich So{\geminationn}tag nach \textcolor{pink}{Iſchl}{}\ledrightnote{\textcolor{pink}{Bad Ischl}}. Jedenfalls erhalten Sie früher Nachricht, damit Sie nicht
               erſchrecken. In \textcolor{pink}{Prag}{}\ledrightnote{\textcolor{pink}{Prag}}, \textcolor{pink}{\textsc{Karlsbad}}{}\ledrightnote{\textcolor{pink}{Karlsbad}} bin ich geweſen. Wenn Sie mir {\pb}noch heute
               ſchreiben, d. h. nach Erhalten dieſes hier, oder auch morgen, ſo beko{\geminationm} ich Ihren Brief noch da; – was mich herzlich freuen
               würde. Ich hoffe Sie ſind tief im \textcolor{green}{Liebling}{}\ledrightnote{\textcolor{green}{Der Tod Georgs}} und
               befinden ſich ſo wohl als ichs Ihnen wünſche.\pend
           \pstart Viele herzl. Grüße Ihr \spacefill\mbox{Arth}\pend{}\endnumbering\briefempfaengerindex{Beer-Hofmann, Richard@\textsc{Beer-Hofmann, Richard}!zzzSchnitzler, Arthur@\emph{von Arthur Schnitzler}!1895-07-101@{{[}10. 7. 1895?{]}}|)be}\mylabel{h}  \normalsize

\doendnotes{C}
\bigskip
\vfill

\clearpage

\footnotesize

\lohead{\textsc{register}}

% Definiere theindex-Environment komplett neu ohne reledmac
\makeatletter
\renewenvironment{theindex}{%
  \section*{\indexname}%
  \setlength{\parindent}{0pt}%
  \setlength{\parskip}{0pt plus 0.3pt}%
  \let\item\@idxitem
}{%
  \clearpage
}
\makeatother

\IfFileExists{\jobname-pw.ind}{\input{\jobname-pw.ind}}{}

\end{document}

      