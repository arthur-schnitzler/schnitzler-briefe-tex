%% latex-korrekturansicht-vorspann.tex
%% Vorspann für die Korrekturansicht.
%% Lädt die gemeinsame Datei latex-vorspann.tex mit gesetztem Schalter.

\newif\ifkorrekturansicht
\korrekturansichttrue

\input{../tex-inputs/latex-vorspann}


               \section[Richard Beer-Hofmann an Arthur Schnitzler, 10. 9. 1895]{ Richard Beer-Hofmann an Arthur Schnitzler,
               10. 9. 1895}\nopagebreak\mylabel{v}\rehead{ }\normalsize\beginnumbering\briefempfaengerindex{Schnitzler, Arthur@\textsc{Schnitzler, Arthur}!zzzBeer-Hofmann, Richard@\emph{von Richard Beer-Hofmann}!1895-09-101@{10. 9. 1895}|(be} \toendnotes[C]{\smallbreak\pagebreak[2]} \Standort{CUL, Schnitzler, B 8.}
\physDesc{Brief, 1 Blatt, 3 Seiten
\newline{}Handschrift: Bleistift, lateinische Kurrent
\newline{}Schnitzler: mit Bleistift nummeriert: »68« }\buchAbdrucke{\weitereDrucke{Arthur Schnitzler, Richard Beer-Hofmann: \emph{Briefwechsel 1891–1931}. Hg. Konstanze Fliedl. Wien, Zürich: \emph{Europaverlag} 1992, S. 79.} }\pstart
           \raggedleft{}{\pb}\uline{\textcolor{pink}{Schönberg im
                        Stubaithal}{}\ledrightnote{\textcolor{pink}{Schönberg im Stubaital}}}{\\}10 Sept 1895\pend
           \pstart
           Lieber Arthur, ich bin nicht in \textcolor{pink}{Kopenhagen}{}\ledrightnote{\textcolor{pink}{Kopenhagen}}; am Abend vor der Abreise entdeckte ich, daß ich gar nicht nach
                  \textcolor{pink}{Kopenhagen}{}\ledrightnote{\textcolor{pink}{Kopenhagen}} wollte und sagte einfach ab. Ich
               hatte Sehnsucht, wirkliche Sehnsucht, allein zu sein. So einfach gieng es nicht. Ich
               mußte, oder, besser ließ mich bereden, in ein Compromiß zu willigen, \strikeout{nac} nach welchem ich nicht sofort aber doch in 3–4
               Tagen allein sein werde. Vorläufig ist {\pb}Frau \textcolor{blue}{Lou}{}\ledrightnote{\textcolor{blue}{Lou Andreas-Salomé}} mit mir gereist; sie reist aber Ende der Woche ab. \uline{Offiziell ist sie verhindert nach \textcolor{pink}{Kopenhagen}{}\ledrightnote{\textcolor{pink}{Kopenhagen}} jetzt zu reisen und kann es erst im Oktober.} Ich bitte
               das festzuhalten.\pend
           \pstart
           – Auch ihr gegenüber. –\pend
           \pstart
           Für alle Fälle habe ich \introOben{}an\introOben{}{ }\textcolor{blue}{Gusti}{}\ledrightnote{\textcolor{blue}{Auguste Chlum}} telegrafirt, ob sie nicht
               Ende der Woche ko{\geminationm}en kann und warte auf Antwort. So will
               ich allein sein. Aber – übrigens das lässt sich besser besprechen, als beschreiben.
               Hier ist {\pb}{[}es{]} einfach herrlich. Das Dorf liegt über der \textcolor{pink}{Brennerstrasse}{}\ledrightnote{\textcolor{pink}{Brenner}}{ }\strikeout{zirc} über 1000 Meter hoch zwei einviertel Stunden mit
               Wagen von \textcolor{pink}{Innsbruck}{}\ledrightnote{\textcolor{pink}{Innsbruck}}. Absolute Ruhe, ein kleines
               Gasthaus – »\textcolor{pink}{Jagerhof}{}\ledrightnote{\textcolor{pink}{Gasthaus Jagerhof}}« für Fremde eingerichtet, aber
               absolut nicht Hôtel. Heute übernachtete ich in einem Bauernhof, weil mein Zimmer erst
               heute frei wird. Aber Frau \textcolor{blue}{Lou}{}\ledrightnote{\textcolor{blue}{Lou Andreas-Salomé}} ko{\geminationm}t soeben an den Tisch. Adieu.\pend
           \pstart Herzlichst \spacefill\mbox{Richard}\pend{}\endnumbering\briefempfaengerindex{Schnitzler, Arthur@\textsc{Schnitzler, Arthur}!zzzBeer-Hofmann, Richard@\emph{von Richard Beer-Hofmann}!1895-09-101@{10. 9. 1895}|)be}\mylabel{h}  \normalsize

\doendnotes{C}
\bigskip
\vfill

\clearpage

\footnotesize

\lohead{\textsc{register}}

% Definiere theindex-Environment komplett neu ohne reledmac
\makeatletter
\renewenvironment{theindex}{%
  \section*{\indexname}%
  \setlength{\parindent}{0pt}%
  \setlength{\parskip}{0pt plus 0.3pt}%
  \let\item\@idxitem
}{%
  \clearpage
}
\makeatother

\IfFileExists{\jobname-pw.ind}{\input{\jobname-pw.ind}}{}

\end{document}

      