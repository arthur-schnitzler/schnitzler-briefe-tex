%% latex-korrekturansicht-vorspann.tex
%% Vorspann für die Korrekturansicht.
%% Lädt die gemeinsame Datei latex-vorspann.tex mit gesetztem Schalter.

\newif\ifkorrekturansicht
\korrekturansichttrue

\input{../tex-inputs/latex-vorspann}


\renewcommand{\erwaehntePersonen}{Personen:  ?? [Kostfrau von Charlotte Lamberg], Maria Charlotte Lamberg, Charlotte Pohl-Glas}
\renewcommand{\erwaehnteOrte}{Orte: Wien}
\renewcommand{\erwaehnteWerke}{}
\section[ Felix Salten an Arthur Schnitzler, {[}9. 6. 1895{]}]{Felix Salten an Arthur Schnitzler, {[}9. 6. 1895{]}}
\nopagebreak\mylabel{v}
\rehead{ }\normalsize\beginnumbering\briefempfaengerindex{Schnitzler, Arthur@\textsc{Schnitzler, Arthur}!zzzSalten, Felix@\emph{von Felix Salten}!1895-06-091@{{[}9. 6. 1895{]}}|(be}
\toendnotes[C]{\smallbreak\pagebreak[2]}\Standort{CUL, Schnitzler, B 89, A 1.}
\physDesc{Brief, 1 Blatt, 2 Seiten, 520 Zeichen
\newline{}Handschrift: Bleistift, lateinische Kurrent
\newline{}Schnitzler: mit Bleistift datiert: »9/6 95« 
\newline{}Ordnung: mit Bleistift von unbekannter Hand nummeriert: »55« }\toendnotes[C]{\smallbreak}
\pstart
           \noindent{}{\pb}Lieber Freund, Sie sind nicht böse, dass ich nochmals
               zu Ihnen komme, ehe ich Ihnen das Erste zurückgegeben. Aber ich muss Sie jetzt
               bitten, mir noch einmal mit 10 fl zu helfen. Die \label{K_L03156-1v}\edtext{\textcolor{blue}{Kostfrau}{}\ledrightnote{{$\rightarrow$}\textcolor{blue}{?? [Kostfrau von Charlotte Lamberg]}}}{\lemma{\textnormal{\emph{Kostfrau}}}\Cendnote{\textnormal{nicht ermittelt}}}\label{K_L03156-1h} des \label{K_L03156-2v}\edtext{\textcolor{blue}{Kindes}{}\ledrightnote{{$\rightarrow$}\textcolor{blue}{Maria Charlotte Lamberg}}}{\lemma{\textnormal{\emph{Kindes}}}\Cendnote{\textnormal{von \textcolor{blue}{Salten} und \textcolor{blue}{Charlotte Glas}, \textcolor{blue}{Maria Charlotte Lamberg}, im Alter von vier
                  Monaten am 27. 7. 1895 verstorben}}}\label{K_L03156-2h} ist vom
               Land hereingekommen: Das \uline{\textcolor{blue}{K.}{}\ledrightnote{{$\rightarrow$}\textcolor{blue}{Maria Charlotte Lamberg}}} sei krank und sie brauche das Geld für das und für jenes. Ich kann sie nicht
               fortschicken ohne G. Bitte, senden Sie mir noch einmal 10 fl, ich {\pb}werde Ihnen diese \uline{20 fl.} bis Dienstag{ }Vormittag{ }\uuline{ganz positiv} zurückgeben. Sie können sich vollständig
               darauf verlaßen. Ich danke Ihnen \pend
           
\pstart
           herzlich {\\[\baselineskip]}Ihr {\\[\baselineskip]}\spacefill\mbox{Salten}\pend
           \leftskip=0em{}\endnumbering\briefempfaengerindex{Schnitzler, Arthur@\textsc{Schnitzler, Arthur}!zzzSalten, Felix@\emph{von Felix Salten}!1895-06-091@{{[}9. 6. 1895{]}}|)be}\mylabel{h}  \normalsize

\doendnotes{C}
\bigskip
\vfill

\clearpage

\footnotesize

\lohead{\textsc{register}}

% Definiere theindex-Environment komplett neu ohne reledmac
\makeatletter
\renewenvironment{theindex}{%
  \section*{\indexname}%
  \setlength{\parindent}{0pt}%
  \setlength{\parskip}{0pt plus 0.3pt}%
  \let\item\@idxitem
}{%
  \clearpage
}
\makeatother

\IfFileExists{\jobname-pw.ind}{\input{\jobname-pw.ind}}{}

\end{document}

      