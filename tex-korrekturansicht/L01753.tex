%% latex-korrekturansicht-vorspann.tex
%% Vorspann für die Korrekturansicht.
%% Lädt die gemeinsame Datei latex-vorspann.tex mit gesetztem Schalter.

\newif\ifkorrekturansicht
\korrekturansichttrue

\input{../tex-inputs/latex-vorspann}


               \section[Hugo von Hofmannsthal an Arthur Schnitzler, {[}17. 1. 1908{]}]{ Hugo von Hofmannsthal an Arthur Schnitzler, {[}17. 1. 1908{]}}\nopagebreak\mylabel{v}\rehead{ }\normalsize\beginnumbering\briefempfaengerindex{Schnitzler, Arthur@\textsc{Schnitzler, Arthur}!zzzHofmannsthal, Hugo von@\emph{von Hugo von Hofmannsthal}!1908-01-172@{17. 1. 1908}|(be} \toendnotes[C]{\smallbreak\pagebreak[2]} \Standort{CUL, Schnitzler, B 43.}
\physDesc{Brief, 1 Blatt, 4 Seiten
\newline{}Handschrift: schwarze Tinte, deutsche Kurrent
\newline{}Schnitzler: mit Bleistift datiert: »17/1 908« und beschriftet: »Hugo« \newline{}Ordnung: 1) mit Bleistift von unbekannter Hand nummeriert: »290« 2) mit Bleistift von unbekannter Hand nummeriert: »292«}\buchAbdrucke{\weitereDrucke{Hugo von Hofmannsthal, Arthur Schnitzler: \emph{Briefwechsel}. Hg. Therese Nickl und Heinrich Schnitzler. Frankfurt am Main: \emph{S. Fischer} 1964, S. 235.} }\toendnotes[C]{\smallbreak}\pstart
           \raggedleft{}{\pb}Freitag.\pend
           \pstart{}mein lieber Arthur\pend\pstart
           ich freue mich \uuline{ſehr}. (Mehr als ich gedacht hätte daſs
               ich mich freuen würde, wenn man mir vorher geſagt hätte: wird es Sie freuen, wenn A{\dots}?)\pend
           \pstart
           Es iſt beſonders lieb, daſs Sie \textcolor{brown}{ihn}{}\ledrightnote{→\textcolor{brown}{Franz-Grillparzer-Preis}} (durch den \textcolor{blue}{Redacteur}{}\ledrightnote{→\textcolor{blue}{Karl Werkmann}} der \textcolor{green}{Zeit}{}\ledrightnote{→\textcolor{green}{Die Zeit}}) gleich \label{K_L01753_1v}\edtext{\textcolor{green}{mir {\pb}verliehen}{}\ledrightnote{→\textcolor{green}{Verleihung des Grillparzer-Preises an Artur Schnitzler}}}{\lemma{\textnormal{\emph{mir verliehen}}}\Cendnote{\textnormal{\textcolor{blue}{Schnitzler}s erste Reaktion auf die Verleihung des \emph{\textcolor{brown}{Grillparzer-Preises}}: »Ich hätte nicht geglaubt, daß der \textcolor{brown}{Preis} mir
                     verliehen werden würde. Es kamen doch so viele Stücke hierfür in Betracht. Zum
                     Beispiel ›\textcolor{green}{Oedipus und die Sphinx}‹, von \textcolor{blue}{Hofmannsthal}, dann ›\textcolor{green}{Und
                           Pippa tanzt}‹, von \textcolor{blue}{Hauptmann}.«
                           ({[}\textcolor{blue}{Karl
                        Werkmann:}{]}{ }\emph{\textcolor{green}{Verleihung des Grillparzer-Preises an Artur Schnitzler.}}
                     In: \emph{\textcolor{green}{Die Zeit}}, Jg. 7, Nr. 1907,
                     Abendblatt, 15. 1. 1908, S. 2).}}}\label{K_L01753_1h}
                  haben.\hspace*{1.5em}Aber, im Ernſt, hätte ich ihn jemals
                  beko{\geminationm}en, bevor Sie ihn hatten ſo hätte ich ihn mit
               einem ſehr groben Brief zurückgeſchickt, ſo leid es mir um das Geld gethan
                  hätte.\hspace*{1.5em}Komiſch übrigens (gewiß hat {\pb}der \textcolor{blue}{Interviewer}{}\ledrightnote{→\textcolor{blue}{Karl Werkmann}}{ }ſich blöd ausgedrückt) daſs Sie ſich ſollten ſo
               quaſi »beſcheiden« ausgedrückt haben ſtatt zu ſagen: Natürlich muſs ich ihn kriegen,
               ſchon längſt hätten mir die Schweine ihn geben müſſen u. ſ. f.\pend
           \pstart
           Ich ſehne mich ſehr {\pb}nach
                  Ihnen.\hspace*{1.5em}Wie wird uns \textcolor{blue}{Olga}{}\ledrightnote{\textcolor{blue}{Olga Schnitzler}} dafür entſchädigen daſs ſie ſich \uline{wichtig} gemacht hat? Nun übrigens, das arme Ding, ich laſſe ſie ſchön und
               herzlich grüßen.\pend
           \pstart
           Von Herzen Ihr{\\[\baselineskip]}\spacefill\mbox{Hugo.}\pend
           \leftskip=0em{}\endnumbering\briefempfaengerindex{Schnitzler, Arthur@\textsc{Schnitzler, Arthur}!zzzHofmannsthal, Hugo von@\emph{von Hugo von Hofmannsthal}!1908-01-172@{17. 1. 1908}|)be}\mylabel{h}  \normalsize

\doendnotes{C}
\bigskip
\vfill

\clearpage

\footnotesize

\lohead{\textsc{register}}

% Definiere theindex-Environment komplett neu ohne reledmac
\makeatletter
\renewenvironment{theindex}{%
  \section*{\indexname}%
  \setlength{\parindent}{0pt}%
  \setlength{\parskip}{0pt plus 0.3pt}%
  \let\item\@idxitem
}{%
  \clearpage
}
\makeatother

\IfFileExists{\jobname-pw.ind}{\input{\jobname-pw.ind}}{}

\end{document}

      