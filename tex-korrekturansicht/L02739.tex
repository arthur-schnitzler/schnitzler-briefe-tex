%% latex-korrekturansicht-vorspann.tex
%% Vorspann für die Korrekturansicht.
%% Lädt die gemeinsame Datei latex-vorspann.tex mit gesetztem Schalter.

\newif\ifkorrekturansicht
\korrekturansichttrue

\input{../tex-inputs/latex-vorspann}


               \section[Paul Goldmann an Arthur Schnitzler, 6. 7. {[}1895{]}]{ Paul Goldmann an Arthur Schnitzler, 6. 7. {[}1895{]}}\nopagebreak\mylabel{v}\rehead{ }\normalsize\beginnumbering\briefempfaengerindex{Schnitzler, Arthur@\textsc{Schnitzler, Arthur}!zzzGoldmann, Paul@\emph{von Paul Goldmann}!1895-07-061@{6. 7. {[}1895{]}}|(be} \toendnotes[C]{\smallbreak\pagebreak[2]} \Standort{DLA, A:Schnitzler, HS.NZ85.1.3165.}
\physDesc{Brief, 2 Blätter, 8 Seiten
\newline{}Handschrift: schwarze Tinte, deutsche Kurrent
\newline{}Schnitzler: 1) mit Bleistift das Jahr »95« vermerkt 2) mit rotem Buntstift eine Unterstreichung}\toendnotes[C]{\smallbreak}\pstart
           \noindent{}{\pb}\textcolor{gray}{\textbf{\textbf{\textcolor{brown}{Frankfurter Zeitung}{}\ledrightnote{\textcolor{brown}{Frankfurter Zeitung}}}}}\pend
           \pstart
           \textcolor{gray}{\textbf{(\textcolor{brown}{\begin{otherlanguage}{french}Gazette de Francfort\end{otherlanguage}}{}\ledrightnote{\textcolor{brown}{Frankfurter Zeitung}}). }}\pend
           \pstart
           \textcolor{gray}{\textbf{\textbf{\begin{otherlanguage}{french}Fondateur M. \textcolor{blue}{L.
                              Sonnemann}{}\ledrightnote{\textcolor{blue}{Leopold Sonnemann}}\end{otherlanguage}.}}}\pend
           \pstart
           \begin{otherlanguage}{french}\textcolor{gray}{\textbf{\textcolor{green}{Journal}{}\ledrightnote{→\textcolor{green}{Frankfurter Zeitung}} politique,
                        financier,}}\end{otherlanguage}\pend
           \pstart
           \begin{otherlanguage}{french}\textcolor{gray}{\textbf{commercial et littéraire.}}\end{otherlanguage}\pend
           \pstart
           \begin{otherlanguage}{french}\textcolor{gray}{\textbf{\textbf{Paraissant trois fois par jour.}}}\end{otherlanguage}\hfill \textsc{\textcolor{pink}{Paris}{}\ledrightnote{\textcolor{pink}{Paris}}}, 6. Juli.\pend
           \pstart
           \begin{otherlanguage}{french}\textcolor{gray}{\textbf{\textbf{Bureau à \textcolor{pink}{Paris}{}\ledrightnote{\textcolor{pink}{Paris}}:}}}\end{otherlanguage}\pend
           \pstart
           \begin{otherlanguage}{french}\textcolor{gray}{\textbf{\textbf{\textcolor{pink}{24. Rue Feydeau}{}\ledrightnote{\textcolor{pink}{rue Feydeau}}.}}}\end{otherlanguage}\pend
           \pstart\center{}Mein lieber Freund,\pend\pstart
           Ich habe Dir nichts Neues zu ſagen, aber ich ſchreib’ Dir, um dir zu ſagen, daß ich
               mich von Herzen freue, Dich \label{K_L02739-1v}\edtext{unterwegs}{\lemma{\textnormal{\emph{unterwegs}}}\Cendnote{\textnormal{Am 3. 7. 1895 trat \textcolor{blue}{Schnitzler} seinen Sommerurlaub an, der ihn zuerst für vier
                     Tage nach \textcolor{pink}{Prag} führte. Es folgten \textcolor{pink}{Karlsbad}, \textcolor{pink}{Marienbad}, \textcolor{pink}{Franzensbad} und \textcolor{pink}{Nürnberg}.
                     Ab 15. 7. 1895 war
                     bis 10. 8. 1895 in 
                     \textcolor{pink}{Bad Ischl}.}}}\label{K_L02739-1h} zu wiſſen, und daß ich Dich mit meinen
               beſten Wünſchen begleite.\pend
           \pstart
           \textcolor{pink}{Prag}{}\ledrightnote{\textcolor{pink}{Prag}} muß ſchön ſein. Viel alte Steine und blonde
               junge Mädchen dawiſchen {\pb}und ein rauſchender \textcolor{pink}{Fluß}{}\ledrightnote{→\textcolor{pink}{Moldau}} und der dreißigjährige
               Krieg. So ſtell’ ich mirs vor. Was Du von dem alten \label{K_L02739-2v}\edtext{\textcolor{pink}{Friedhof}{}\ledrightnote{→\textcolor{pink}{Alter Jüdischer Friedhof}}}{\lemma{\textnormal{\emph{Friedhof}}}\Cendnote{\textnormal{Am 5. 7. 1895 besuchte \textcolor{blue}{Schnitzler} mit \textcolor{blue}{Marie Reinhard} den \textcolor{pink}{jüdischen Friedhof}, der seit ein paar Jahren nicht mehr in aktiver Verwendung war.}}}\label{K_L02739-2h}
               ſchreibſt, hat mir beinahe Heimweh danach gemacht. So iſt der Tod anheimelnd, – wenn
               man nämlich oben ſteht zwiſchen den verſinkenden Steinen und dem grünen Gras, in
               Sommerluft und Frieden. Nur iſt das nicht der eigentliche Friedhof. Der eigentliche
                  {\pb}Friedhof – das wäre, wenn man ihn von unten
               anſieht. Da muß er ſchauderhaft ſein, aber das \strikeout{i\textcolor{gray}{n}} iſt auch des Todes wahres Geſicht. Hierher gehört ein Capitel über die
               Oberflächlichkeit der menſchlichen Todes-Anſchauung, welche die Friedhöfe von oben
               betrachtet ſtatt von unten, welche ſich unter die \strikeout{g\textcolor{gray}{er}} rauſchenden Bäume der Friedhöfe ſtellt und ſagt: {\pb}»Welch’ ſanfte Ruhe!« Nein, es iſt nicht die Ruhe,
               es iſt das Vermodern.
            \pend
           \pstart
           Dabei vergeſſe ich, daß ich zum Autor von »\textcolor{green}{Sterben}{}\ledrightnote{\textcolor{green}{Sterben. Novelle}}« ſpreche.\pend
           \pstart
           \strikeout{Ac\textcolor{gray}{h}} Oh, ich möchte gern \strikeout{hinunter} hinunter, unter
               die Erde. Ich kann\strikeout{s} wirklich nicht mehr. Seit einigen Tagen ſehe ich wieder mit
               erbarmungsloſer Klarheit, was ich Alles verfehlt, was {\pb}ich nie erreichen werde. Ich ſehe mich mit
               energieloſen Schritten durch die Straßen gehen, und aus den Spiegeln der Läden blickt
               mir mein Geſicht entgegen und ruft: »\label{K_L02739-3v}\edtext{\begin{otherlanguage}{french}\textsc{Un raté}\end{otherlanguage}}{\lemma{\textnormal{\emph{Un raté}}}\Cendnote{\textnormal{französisch:
                  Versager}}}\label{K_L02739-3h}.« Haha, mit 30 Jahren!\pend
           \pstart
           Sterben, oh ja! Aber glücklich leben wäre doch noch viel ſchöner, und {\pb}ich glaube immer noch daran, obwohl ich es mit
               unbeweisbarer Logik darthun kann, daß ich zu ſchwach bin, mir irgend eines der großen
                  Lebensgüter zu erkämpfen.\pend
           \pstart
           Das iſt ſo ehrlich, was ich Dir da ſchreibe, ſo ohne Poſe, weiß Gott!\pend
           \pstart
           \textsc{\textcolor{blue}{Becque}{}\ledrightnote{\textcolor{blue}{Henry Becque}}} hat mir verſprochen, er will »\textsc{\textcolor{green}{Mourir}{}\ledrightnote{→\textcolor{green}{Mourir}}}« leſen. Wird ers thun?{\dots} Ich {\pb}ſchicks ihm Montag.
               Dann könnte man mit ihm die Verleger-Frage berathen.\pend
           \pstart
           Wer die betreffende \label{K_L02739-4v}\edtext{\textcolor{blue}{Frau}{}\ledrightnote{→\textcolor{blue}{Regina Candiani}}}{\lemma{\textnormal{\emph{Frau}}}\Cendnote{\textnormal{Am 2. 7. 1895 notierte \textcolor{blue}{Schnitzler} im Tagebuch: »Uebersetzungsantrag \textcolor{green}{Sterben} und andre Frau \textcolor{blue}{Candiani} ―«. \textcolor{blue}{Regine Candiani} war eine 
                  \textcolor{pink}{russland}stämmige Übersetzerin,
                  die seit 1875 in \textcolor{pink}{Frankreich}
                  lebte und \textcolor{blue}{Tolstoi} und \textcolor{blue}{Turgenjew}
                  übersetzte. Übersetzungen von \textcolor{blue}{Schnitzler}
                  sind nicht nachgewiesen, obwohl zumindest zwischen 1902 und 1903
                  Korrespondenzstücke in seinem Nachlass liegen.}}}\label{K_L02739-4h} iſt, möchte ich Dir gern ſagen, könnt’ ich nur ihren Namen leſen.
               Bitte ſchreib’ mir ihn noch einmal recht deutlich auf. Von was iſt ſie Sekretär? In welcher Stadt
               lebt ſie? Daß Du Dich zu nichts verpflichtet haſt, iſt gut. Unter keinen Umſtänden
                  {\pb}darfſt Du Deine übrigen Werke vergeben.
                  Sieh\textcolor{gray}{’} Dir auch an, ob die Überſetzungen ’was taugen oder ſchick’
               ſie mir. Die Frauenzimmer thun in der Regel das Übersetzen ab, wie das Strümpfe
               flicken.\pend
           \pstart
           Grüß’ Dich Gott, mein lieber Freund. Mit wem immer Du biſt, grüß’ ihn von mir. Ich
               wünſche Dir von Herzen Glück und Sonnenschein auf dem Weg.\pend
           \pstart
           Dein treuer {\\[\baselineskip]}\spacefill\mbox{Paul Goldmann}\pend
           \leftskip=0em{}\endnumbering\briefempfaengerindex{Schnitzler, Arthur@\textsc{Schnitzler, Arthur}!zzzGoldmann, Paul@\emph{von Paul Goldmann}!1895-07-061@{6. 7. {[}1895{]}}|)be}\mylabel{h}\begin{anhang}\end{anhang}\normalsize

\doendnotes{C}
\bigskip
\vfill

\clearpage

\footnotesize

\lohead{\textsc{register}}

% Definiere theindex-Environment komplett neu ohne reledmac
\makeatletter
\renewenvironment{theindex}{%
  \section*{\indexname}%
  \setlength{\parindent}{0pt}%
  \setlength{\parskip}{0pt plus 0.3pt}%
  \let\item\@idxitem
}{%
  \clearpage
}
\makeatother

\IfFileExists{\jobname-pw.ind}{\input{\jobname-pw.ind}}{}

\end{document}

      