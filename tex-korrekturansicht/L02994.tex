%% latex-korrekturansicht-vorspann.tex
%% Vorspann für die Korrekturansicht.
%% Lädt die gemeinsame Datei latex-vorspann.tex mit gesetztem Schalter.

\newif\ifkorrekturansicht
\korrekturansichttrue

\input{../tex-inputs/latex-vorspann}


\renewcommand{\erwaehntePersonen}{Personen: Felix Salten, Ottilie Salten, Richard Wagner}
\renewcommand{\erwaehnteOrte}{Orte: Oper, Riedhof, Wien}
\renewcommand{\erwaehnteWerke}{Werke: Symphonie Nr. 3 D-Moll, Tristan und Isolde}
\section[ Arthur Schnitzler an Felix Salten, {[}23. 12. 1904?{]}]{Arthur Schnitzler an Felix Salten, {[}23. 12. 1904?{]}}
\nopagebreak\mylabel{v}
\rehead{ }\normalsize\beginnumbering\briefempfaengerindex{Salten, Felix@\textsc{Salten, Felix}!zzzSchnitzler, Arthur@\emph{von Arthur Schnitzler}!1904-12-232@{{[}23. 12. 1904?{]}}|(be}
\toendnotes[C]{\smallbreak\pagebreak[2]}\Standort{Wienbibliothek im Rathaus, ZPH 1681, 2.1.516.}
\physDesc{Brief, 1 Blatt, 3 Seiten, 683 Zeichen
\newline{}Handschrift: Bleistift, deutsche Kurrent
\newline{}Ordnung: mit Bleistift von unbekannter Hand Nummerierung der Blätter des Konvoluts:
                                    »11«–»12« }\toendnotes[C]{\smallbreak}
\pstart
           \noindent{}{\pb}lieber, wir haben geſtern{ }Abend ¾ Stunden gewartet, dachten umſoweniger dran, dſs Sie noch kommen
               würden, als Sie mir ja \label{K_L02994-1v}\edtext{geſchriebn}{\lemma{\textnormal{\emph{geſchriebn}}}\Cendnote{\textnormal{siehe Felix Salten an Arthur Schnitzler, [20. 12. 1904]}}}\label{K_L02994-1h} hatten, daſs Sie auch im \textcolor{green}{Concert}{}\ledrightnote{{$\rightarrow$}\textcolor{green}{Symphonie Nr. 3 D-Moll}} wären und vom \textcolor{green}{Concert}{}\ledrightnote{{$\rightarrow$}\textcolor{green}{Symphonie Nr. 3 D-Moll}} aus \substVorne{}\textsuperscript{\textcolor{gray}{kämen}}\substDazwischen{}in den\substHinten{}{ }\textcolor{pink}{Riedhof}{}\ledrightnote{\textcolor{pink}{Riedhof}} gehen {\pb}würden. Ich dachte natürlich an eine
               redactionelle oder ſonſtige Verhinderung Ihrerſeits, und ſo gingen wir, zwar mit
               Bedauern, aber höchſt unſchuldsvoll nach Hause.\pend
           
\pstart
           Ich grüße Sie herzlich und wünſche Ihnen, nebſt allem ſchönen, daſs der Genius Ihrer
                  {\pb}Empfindlichkeit zur Hölle fahre.\pend
           
\pstart
           Ihr {\\[\baselineskip]}\spacefill\mbox{A.}\pend
           \leftskip=0em{}
\pstart
           \noindent{}Heute wollten wir zu \label{K_L02994-2v}\edtext{\textcolor{green}{Triſtan}{}\ledrightnote{\textcolor{green}{Tristan und Isolde}}{[},{]}}{\lemma{\textnormal{\emph{Triſtan,}}}\Cendnote{\textnormal{\textcolor{blue}{Richard Wagner}s \emph{\textcolor{green}{Tristan und Isolde}} wurde in der \textcolor{pink}{Oper} gegeben.}}}\label{K_L02994-2h} haben nichts mehr bekommen, ſind
                  wieder Erwarten heim{[},{]} theilen Sie mir bitte ein Wort \introOben{}\textsc{pneumatisch}\introOben{} ob Sie und \textcolor{blue}{Otti}{}\ledrightnote{\textcolor{blue}{Ottilie Salten}}{ }heute{ }Abend 9 Uhr im \textcolor{pink}{Riedhof}{}\ledrightnote{\textcolor{pink}{Riedhof}} mit uns
                  nachtmahlen wollen.\pend
           
\pstart
           \spacefill\mbox{A.}\pend
           \endnumbering\briefempfaengerindex{Salten, Felix@\textsc{Salten, Felix}!zzzSchnitzler, Arthur@\emph{von Arthur Schnitzler}!1904-12-232@{{[}23. 12. 1904?{]}}|)be}\mylabel{h}  \normalsize

\doendnotes{C}
\bigskip
\vfill

\clearpage

\footnotesize

\lohead{\textsc{register}}

% Definiere theindex-Environment komplett neu ohne reledmac
\makeatletter
\renewenvironment{theindex}{%
  \section*{\indexname}%
  \setlength{\parindent}{0pt}%
  \setlength{\parskip}{0pt plus 0.3pt}%
  \let\item\@idxitem
}{%
  \clearpage
}
\makeatother

\IfFileExists{\jobname-pw.ind}{\input{\jobname-pw.ind}}{}

\end{document}

      