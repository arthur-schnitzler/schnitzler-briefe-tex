%% latex-korrekturansicht-vorspann.tex
%% Vorspann für die Korrekturansicht.
%% Lädt die gemeinsame Datei latex-vorspann.tex mit gesetztem Schalter.

\newif\ifkorrekturansicht
\korrekturansichttrue

\input{../tex-inputs/latex-vorspann}


\renewcommand{\erwaehntePersonen}{Personen: Fritz Freund, Wolfgang Heine}
\renewcommand{\erwaehnteInstitutionen}{Institutionen: Ernst Rowohlt Verlag, Staatsanwaltschaft Berlin, Wiener Verlag}
\renewcommand{\erwaehnteOrte}{Orte: Berlin, Dessauer Straße, Deutschland, Polen, Wien}
\renewcommand{\erwaehnteWerke}{Werke: Der Kampf um den Reigen. Vollständiger Bericht über die sechstägige Verhandlung gegen Direktion und Darsteller des Kleinen Schauspielhauses Berlin, Neue Freie Presse, Reigen. Zehn Dialoge}
\section[ Paul Goldmann an Arthur Schnitzler, 19. 3. {[}1904{]}]{Paul Goldmann an Arthur Schnitzler, 19. 3. {[}1904{]}}
\nopagebreak\mylabel{v}
\rehead{ }\normalsize\beginnumbering\briefempfaengerindex{Schnitzler, Arthur@\textsc{Schnitzler, Arthur}!zzzGoldmann, Paul@\emph{von Paul Goldmann}!1904-03-191@{19. 3. {[}1904{]}}|(be}
\toendnotes[C]{\smallbreak\pagebreak[2]}\Standort{DLA, A:Schnitzler, HS.NZ85.1.3174.}
\physDesc{Brief, 1 Blatt, 3 Seiten
\newline{}Handschrift: blaue Tinte, deutsche Kurrent
\newline{}Schnitzler: mit Bleistift das Jahr »{[}1{]}904« vermerkt }\toendnotes[C]{\smallbreak}
\pstart
           \noindent{}\raggedleft{}{\pb}\textcolor{gray}{\textbf{\textcolor{pink}{DESSAUERSTRASSE 19}{}\ledrightnote{\textcolor{pink}{Dessauer Straße}}}}\pend
           
\pstart
           \textcolor{pink}{Berlin}{}\ledrightnote{\textcolor{pink}{Berlin}}, 19. März.\pend
           
\pstart\center{}Mein lieber Freund,\pend
\pstart
           Das \label{K_L03441-1v}\edtext{Verbot des »\textcolor{green}{Reigen}{}\ledrightnote{\textcolor{green}{Reigen. Zehn Dialoge}}«}{\lemma{\textnormal{\emph{Verbot des »Reigen«}}}\Cendnote{\textnormal{Am 16. 3. 1904 wurde die
                     1903 im \emph{\textcolor{brown}{Wiener
                     Verlag}} erschienene \emph{\textcolor{green}{Buchausgabe des
                     Reigen}} in \textcolor{pink}{Deutschland} durch die \emph{\textcolor{brown}{Berliner Staatsanwaltschaft}} konfisziert. Kurz
                  darauf folgte die Ausweitung des Verbots auf \textcolor{pink}{Polen}.}}}\label{K_L03441-1h} durch die \textcolor{pink}{Berlin}{}\ledrightnote{\textcolor{pink}{Berlin}}er \textcolor{brown}{Staatsanwaltſchaft}{}\ledrightnote{\textcolor{brown}{Staatsanwaltschaft Berlin}} ſcheint ſich nun wohl leider zu
               beſtätigen? Ich bitte Dich, mir mitzutheilen, ob ich Dir in dieſer Angelegenheit
               irgendwie \strikeout{di} dienen kann? Du weißt, daß nach \textcolor{pink}{deutsch}{}\ledrightnote{{$\rightarrow$}\textcolor{pink}{Deutschland}}em Recht, auf jede
               Confiscation ein Prozeß folgen muß. Eſ iſt alſo dringend nöthig, daß Du oder Dein \textcolor{brown}{\textcolor{blue}{Verleger}{}\ledrightnote{\textcolor{blue}{Fritz Freund}}}{}\ledrightnote{\textcolor{brown}{Wiener Verlag}} einen tüchtigen Rechtsanwalt \strikeout{zur} als Berather
               nehmt, – womöglich einen, der durch ein {\pb}Wort
               politiſcher Oppoſition nicht ſcheut. Beiſpielsweiſe würde ich \label{K_L03441-11v}\edtext{\textsc{\textcolor{blue}{Heine}{}\ledrightnote{\textcolor{blue}{Wolfgang Heine}}}}{\lemma{\textnormal{\emph{Heine}}}\Cendnote{\textnormal{Siehe zu \textcolor{blue}{Heine} auch \emph{\textcolor{green}{Der Kampf um den Reigen. Vollständiger Bericht
                        über die sechstägige Verhandlung gegen Direktion und Darsteller des Kleinen
                        Schauspielhauses Berlin}}. Herausgegeben und mit einer Einleitung von \textcolor{blue}{Wolfgang Heine}, Rechtsanwalt,
                     Staatsminister a. D. \textcolor{pink}{Berlin}: \emph{\textcolor{brown}{Rowohlt}}{ }1922.}}}\label{K_L03441-11h} empfehlen.\pend
           
\pstart
           Schreibe mir, ob ich igendwelche Schritte in dieſer Angelegenheit für Dich thun kann,
               – ob Du wünſcheſt, daß irgend Etwas in den \textcolor{pink}{Berlin}{}\ledrightnote{\textcolor{pink}{Berlin}}er Blättern oder in der \textcolor{green}{N. Fr. Pr.}{}\ledrightnote{\textcolor{green}{Neue Freie Presse}}
               veröffentlicht wird?\pend
           
\pstart
           Das Verbot richtet hoffentlich keinen großen \label{K_L03441-4v}\edtext{materiellen Schaden}{\lemma{\textnormal{\emph{materiellen Schaden}}}\Cendnote{\textnormal{Das Verbot des \emph{\textcolor{green}{Reigen}}
                  hatte tatsächlich den gegenteiligen Effekt – das \textcolor{green}{Buch} verkaufte sich gut.}}}\label{K_L03441-4h} mehr an, – im Gegentheil
                  {\pb}wird es wohl, wie immer ſolche Verbote, auf das
                  \textcolor{green}{Buch}{}\ledrightnote{{$\rightarrow$}\textcolor{green}{Reigen. Zehn Dialoge}} erſt recht aufmerkſam
               machen.\pend
           
\pstart
           Viele herzliche Grüße! {\\[\baselineskip]}Dein {\\[\baselineskip]}\spacefill\mbox{Paul Goldm}\pend
           \leftskip=0em{}\endnumbering\briefempfaengerindex{Schnitzler, Arthur@\textsc{Schnitzler, Arthur}!zzzGoldmann, Paul@\emph{von Paul Goldmann}!1904-03-191@{19. 3. {[}1904{]}}|)be}\mylabel{h}
\begin{anhang}
\end{anhang}\normalsize

\doendnotes{C}
\bigskip
\vfill

\clearpage

\footnotesize

\lohead{\textsc{register}}

% Definiere theindex-Environment komplett neu ohne reledmac
\makeatletter
\renewenvironment{theindex}{%
  \section*{\indexname}%
  \setlength{\parindent}{0pt}%
  \setlength{\parskip}{0pt plus 0.3pt}%
  \let\item\@idxitem
}{%
  \clearpage
}
\makeatother

\IfFileExists{\jobname-pw.ind}{\input{\jobname-pw.ind}}{}

\end{document}

      