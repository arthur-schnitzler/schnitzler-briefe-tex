%% latex-korrekturansicht-vorspann.tex
%% Vorspann für die Korrekturansicht.
%% Lädt die gemeinsame Datei latex-vorspann.tex mit gesetztem Schalter.

\newif\ifkorrekturansicht
\korrekturansichttrue

\input{../tex-inputs/latex-vorspann}


               \section[Hugo von Hofmannsthal an Arthur Schnitzler, {[}8.–9. 1. 1904{]}]{ Hugo von Hofmannsthal an Arthur Schnitzler, {[}8.–9. 1. 1904{]}}\nopagebreak\mylabel{v}\rehead{ }\normalsize\beginnumbering\briefempfaengerindex{Schnitzler, Arthur@\textsc{Schnitzler, Arthur}!zzzHofmannsthal, Hugo von@\emph{von Hugo von Hofmannsthal}!1904-01-083@{{[}8.–9. 1. 1904{]}}|(be} \toendnotes[C]{\smallbreak\pagebreak[2]} \Standort{CUL, Schnitzler, B 43b/1.}
\physDesc{Brief, 1 Blatt, 3 Seiten
\newline{}Handschrift Gertrude von Hofmannsthal: schwarze Tinte, lateinische Kurrent
\newline{}Schnitzler: mit Bleistift datiert: »Jänner 904« und beschriftet: »Hugo« \newline{}Ordnung: mit Bleistift von unbekannter Hand nummeriert:
                                                »251 213a« }\buchAbdrucke{\weitereDrucke{1) Hugo von Hofmannsthal, Arthur Schnitzler: \emph{Briefwechsel}. Hg. Therese Nickl und Heinrich Schnitzler. Frankfurt am Main: \emph{S. Fischer} 1964, S. 181–182.} \weitereDrucke{2) Hermann Bahr, Arthur Schnitzler: \emph{Briefwechsel, Aufzeichnungen, Dokumente
                                (1891–1931)}. Hg. Kurt Ifkovits und Martin Anton Müller. Göttingen: \emph{Wallstein} 2018, S. 288–289.} }\toendnotes[C]{\smallbreak}\pstart
           \noindent{}{\pb}Lieber Arthur, ich bin natürlich äusserst bestürzt über die
                    plötzlich so sehr ernsthaft gewordene Situation \textcolor{blue}{Bahrs}{}\ledrightnote{\textcolor{blue}{Hermann Bahr}}. Die Diagnose \textcolor{blue}{Ortner’s}{}\ledrightnote{\textcolor{blue}{Norbert von Ortner-Rodenstätt}}
                    lautete: schwere Erkrankung der Aorta und der Kranzarterien sowie Angina
                    pectoris. Der Frau \textcolor{blue}{Bahr}{}\ledrightnote{\textcolor{blue}{Rosa Bahr}} scheint der \textcolor{blue}{Hausarzt}{}\ledrightnote{→\textcolor{blue}{Norbert von Ortner-Rodenstätt}} den Zustand als
                    schwere Herzmuskelerkrankung {\pb}bezeichnet und wenig Hoffnung gegeben zu haben{[}.{]}\pend
           \pstart
           \textcolor{blue}{Bahr}{}\ledrightnote{\textcolor{blue}{Hermann Bahr}} reist Mittwoch früh nach dem \textcolor{pink}{Sanatorium für Herzkranke in Marbach am
                        Bodensee}{}\ledrightnote{\textcolor{pink}{Sanatorium Schloss Marbach am Bodensee}} für mindestens 3 Monate. Ich schwanke zwischen einer sehr
                    traurigen Auffassung und einer etwas hoffnungsvolleren, die darauf beruht, dass
                    doch Ihr \textcolor{blue}{Bruder}{}\ledrightnote{→\textcolor{blue}{Julius Schnitzler}} ihn {\pb}erst im April
                    untersucht hat ferner die Ärzte im Mai in \textcolor{pink}{Edlach}{}\ledrightnote{\textcolor{pink}{Edlach}} und das so plötzliche Eintreten einer so äusserst
                    schweren Erkrankung in diesem Alter mir ganz räthselhaft erscheint.\pend
           \pstart
           Ich hin sehr bekümmert und wünsche mir sehr mit Ihnen darüber zu reden. Von
                    Herzen Ihr{\\[\baselineskip]}\spacefill\mbox{Hugo.}\pend
           \leftskip=0em{}\endnumbering\briefempfaengerindex{Schnitzler, Arthur@\textsc{Schnitzler, Arthur}!zzzHofmannsthal, Hugo von@\emph{von Hugo von Hofmannsthal}!1904-01-083@{{[}8.–9. 1. 1904{]}}|)be}\mylabel{h}  \normalsize

\doendnotes{C}
\bigskip
\vfill

\clearpage

\footnotesize

\lohead{\textsc{register}}

% Definiere theindex-Environment komplett neu ohne reledmac
\makeatletter
\renewenvironment{theindex}{%
  \section*{\indexname}%
  \setlength{\parindent}{0pt}%
  \setlength{\parskip}{0pt plus 0.3pt}%
  \let\item\@idxitem
}{%
  \clearpage
}
\makeatother

\IfFileExists{\jobname-pw.ind}{\input{\jobname-pw.ind}}{}

\end{document}

      