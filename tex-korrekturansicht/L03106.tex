%% latex-korrekturansicht-vorspann.tex
%% Vorspann für die Korrekturansicht.
%% Lädt die gemeinsame Datei latex-vorspann.tex mit gesetztem Schalter.

\newif\ifkorrekturansicht
\korrekturansichttrue

\input{../tex-inputs/latex-vorspann}


\renewcommand{\erwaehntePersonen}{Personen:  C.}
\renewcommand{\erwaehnteInstitutionen}{Institutionen: »Freie Bühne« Verein für moderne Literatur}
\renewcommand{\erwaehnteOrte}{Orte: Berggasse, Café Kremser, Deutschland, Wien}
\renewcommand{\erwaehnteWerke}{Werke: Tagebuch}
\section[Felix Salten an Arthur Schnitzler, {[}28. 9. 1891?{]}]{Felix Salten an Arthur Schnitzler, {[}28. 9. 1891?{]}}
\nopagebreak\mylabel{v}
\rehead{ }\normalsize\beginnumbering\briefempfaengerindex{Schnitzler, Arthur@\textsc{Schnitzler, Arthur}!zzzSalten, Felix@\emph{von Felix Salten}!1891-09-281@{{[}28. 9. 1891?{]}}|(be}
\toendnotes[C]{\smallbreak\pagebreak[2]}\Standort{CUL, Schnitzler, B 89, A 1.}
\physDesc{Visitenkarte, 166 Zeichen
\newline{}Handschrift: Bleistift, lateinische Kurrent
\newline{}Schnitzler: mit Bleistift datiert: »Sept. 91« 
\newline{}Ordnung: mit Bleistift von unbekannter Hand nummeriert: »7a« }\toendnotes[C]{\smallbreak}
\pstart
           \noindent{}{\pb}Lieber Freund! Verzeihen Sie, dass ich \label{K_L03106-1v}\edtext{heute so ohne Gruss}{\lemma{\textnormal{\emph{heute so ohne Gruss}}}\Cendnote{\textnormal{Die Zahl der Tage, die \textcolor{blue}{Salten} und \textcolor{blue}{Schnitzler} im September 1891 am gleichen Ort sind, ist gering, da der
                  eine frühestens ab 14. 9. 1891 in \textcolor{pink}{Wien} ist, der andere aber zwischen 19. 9. 1891 und 26. 9. 1891 in \textcolor{pink}{Deutschland}. Berücksichtigt man auch, dass es zu einem
                  Treffen am Vormittag in einer größeren Runde gekommen sein muss,
                  bietet sich mit \textcolor{blue}{Schnitzler}s \emph{\textcolor{green}{Tagebuch}} nur ein Treffen im Theaterausschuss der \emph{\textcolor{brown}{Freien Bühne}} an, das am 28. 9. 1891
                  stattfand.}}}\label{K_L03106-1h} verschwunden bin. Das kam wegen der kleinen \label{K_L03106-2v}\edtext{\textcolor{blue}{C.}{}\ledrightnote{\textcolor{blue}{C.}}}{\lemma{\textnormal{\emph{C.}}}\Cendnote{\textnormal{nicht ermittelt}}}\label{K_L03106-2h}\pend
           
\pstart
           Ich bin um 10 im \textcolor{pink}{Kremser}{}\ledrightnote{\textcolor{pink}{Café Kremser}}, wo ich Sie
               gar gerne \label{K_L03106-3v}\edtext{sehen {\pb}möchte}{\lemma{\textnormal{\emph{sehen möchte}}}\Cendnote{\textnormal{nicht geschehen}}}\label{K_L03106-3h}\pend
           \pstart Herzlich Ihr\pend{}
\pstart
           \centering{}\textcolor{gray}{\textbf{FELIX SALTEN}}\pend
           
\pstart
           \noindent{}\raggedleft{}\textcolor{gray}{\textbf{\textcolor{pink}{IX., BERGGASSE 13}{}\ledrightnote{\textcolor{pink}{Berggasse}}.}}\pend
           \endnumbering\briefempfaengerindex{Schnitzler, Arthur@\textsc{Schnitzler, Arthur}!zzzSalten, Felix@\emph{von Felix Salten}!1891-09-281@{{[}28. 9. 1891?{]}}|)be}\mylabel{h}  \normalsize

\doendnotes{C}
\bigskip
\vfill

\clearpage

\footnotesize

\lohead{\textsc{register}}

% Definiere theindex-Environment komplett neu ohne reledmac
\makeatletter
\renewenvironment{theindex}{%
  \section*{\indexname}%
  \setlength{\parindent}{0pt}%
  \setlength{\parskip}{0pt plus 0.3pt}%
  \let\item\@idxitem
}{%
  \clearpage
}
\makeatother

\IfFileExists{\jobname-pw.ind}{\input{\jobname-pw.ind}}{}

\end{document}

      