%% latex-korrekturansicht-vorspann.tex
%% Vorspann für die Korrekturansicht.
%% Lädt die gemeinsame Datei latex-vorspann.tex mit gesetztem Schalter.

\newif\ifkorrekturansicht
\korrekturansichttrue

\input{../tex-inputs/latex-vorspann}


               \section[Paul Goldmann an Arthur Schnitzler, Paul Goldmann an Arthur Schnitzler, 27. 10. {[}1896{]}]{ Paul Goldmann an Arthur Schnitzler, 27. 10. {[}1896{]}}\nopagebreak\mylabel{v}\rehead{ }\normalsize\beginnumbering\briefempfaengerindex{Schnitzler, Arthur@\textsc{Schnitzler, Arthur}!zzzGoldmann, Paul@\emph{von Paul Goldmann}!1896-10-272@{27. 10. {[}1896{]}}|(be} \toendnotes[C]{\smallbreak\pagebreak[2]} \Standort{DLA, A:Schnitzler, HS.NZ85.1.3166.}
\physDesc{Brief, 1 Blatt, 4 Seiten
\newline{}Handschrift: blaue Tinte, deutsche Kurrent
\newline{}Schnitzler: mit Bleistift das Jahr »96« vermerkt }\toendnotes[C]{\smallbreak}\pstart
           \noindent{}{\pb}\textcolor{gray}{\textbf{\textbf{\textcolor{brown}{Frankfurter Zeitung}{}\ledrightnote{\textcolor{brown}{Frankfurter Zeitung}}}}}\pend
           \pstart
           \textcolor{gray}{\textbf{(\textcolor{brown}{\begin{otherlanguage}{french}Gazette de Francfort\end{otherlanguage}}{}\ledrightnote{\textcolor{brown}{Frankfurter Zeitung}}).}}\pend
           \pstart
           \textcolor{gray}{\textbf{\textbf{\begin{otherlanguage}{french}Fondateur M.\end{otherlanguage}{ }\textcolor{blue}{L. Sonnemann}{}\ledrightnote{\textcolor{blue}{Leopold Sonnemann}}.}}}\pend
           \pstart
           \begin{otherlanguage}{french}\textcolor{gray}{\textbf{\textcolor{green}{Journal}{}\ledrightnote{→\textcolor{green}{Frankfurter Zeitung}} politique,
                        financier,}}\end{otherlanguage}\pend
           \pstart
           \begin{otherlanguage}{french}\textcolor{gray}{\textbf{commercial et littéraire.}}\end{otherlanguage}\pend
           \pstart
           \begin{otherlanguage}{french}\textcolor{gray}{\textbf{\textbf{Paraissant trois fois par jour.}}}\end{otherlanguage}\hfill \textsc{\textcolor{pink}{Paris}{}\ledrightnote{\textcolor{pink}{Paris}}}, 27. October.\pend
           \pstart
           \begin{otherlanguage}{french}\textcolor{gray}{\textbf{\textbf{Bureau à \textcolor{pink}{Paris}{}\ledrightnote{\textcolor{pink}{Paris}}}}}\end{otherlanguage}\pend
           \pstart
           \begin{otherlanguage}{french}\textcolor{gray}{\textbf{\textbf{\textcolor{pink}{24. Rue Feydeau}{}\ledrightnote{\textcolor{pink}{rue Feydeau}}.}}}\end{otherlanguage}\pend
           \pstart{}Mein lieber Freund,\pend\pstart
           Deine lieben Briefe treffen mich in einer Zeit größter Arbeit. Ich kann Dir
               einſtweilen nur mit flüchtigen Worten ſagen, wie ſehr ich mich freue, daß \label{K_L02788-1v}\edtext{der große Tag}{\lemma{\textnormal{\emph{der große Tag}}}\Cendnote{\textnormal{die Uraufführung des \emph{\textcolor{green}{Freiwild}}s am 3. 11. 1896 am \textcolor{pink}{Deutschen Theater} in \textcolor{pink}{Berlin}}}}\label{K_L02788-1h} ſo nahe iſt. Ich heiße Dich \label{K_L02788-2v}\edtext{willkommen in \textcolor{pink}{Berlin}{}\ledrightnote{\textcolor{pink}{Berlin}}}{\lemma{\textnormal{\emph{willkommen in Berlin}}}\Cendnote{\textnormal{\textcolor{blue}{Schnitzler} hielt sich von 26. 10. 1896 bis zum
                     9. 11. 1896 in
                     \textcolor{pink}{Berlin} auf.}}}\label{K_L02788-2h} und wünſche Dir einen
               frohen und glücklichen Aufenthalt. Nächſtens antworte ich Dir ausführlicher auf
               Deinen letzten längeren Brief, der mich ſehr erfreut hat. {\pb}Warte jedenfalls nicht auf meine Antwort und
               ſchreibe mir gleich ein kurzes Wort über Deine \textcolor{pink}{Berlin}{}\ledrightnote{\textcolor{pink}{Berlin}}er \strikeout{\textcolor{gray}{A}} Eindrücke und insbeſondere \strikeout{uber} darüber, wie
               Dein \textcolor{green}{Stück}{}\ledrightnote{→\textcolor{green}{Freiwild. Schauspiel in 3 Akten}} Dir \label{K_L02788-3v}\edtext{auf den Proben gefällt}{\lemma{\textnormal{\emph{auf den Proben gefällt}}}\Cendnote{\textnormal{\textcolor{blue}{Schnitzler} notierte sich im \emph{\textcolor{green}{Tagebuch}} zunächst äußerst negative (siehe A. S.: \emph{Tagebuch}, 28. 10. 1896), später aber
                  auch positivere (siehe A. S.: \emph{Tagebuch}, 2. 11. 1896 Eindrücke von den \emph{\textcolor{green}{Freiwild}}-Proben.}}}\label{K_L02788-3h}.
               Einen Rath nur in Kürze: Ganz \textcolor{pink}{Deutſchland}{}\ledrightnote{\textcolor{pink}{Deutschland}} ſteht
               unter dem Banne des Eindruck\textcolor{gray}{e}s, den die \label{K_L02788-4v}\edtext{Affaire \textsc{\textcolor{blue}{Bruesewitz}{}\ledrightnote{\textcolor{blue}{Henning von Brüsewitz}}}}{\lemma{\textnormal{\emph{Affaire Bruesewitz}}}\Cendnote{\textnormal{siehe Paul Goldmann an Arthur Schnitzler, 17. 10. [1896]}}}\label{K_L02788-4h} gemacht hat. Man lechzt nach einem Wort, das dieſe ſchurkiſchen
               Officiers-Feiglinge geißelt. Keiner kann beſſer dieſes Wort aus{\pb}ſprechen, als Du. Leg’ es Deinem anſtändigen
               Officier in den Mund, in der \textcolor{green}{Scene}{}\ledrightnote{→\textcolor{green}{Freiwild. Schauspiel in 3 Akten}}, wo er ſagt: \label{K_L02788-55v}\edtext{Solche
               Leute haben im Frieden eigentlich gar keine Exiſtenz-Berechtigung}{\lemma{\textnormal{\emph{Solche … Exiſtenz-Berechtigung}}}\Cendnote{\textnormal{Aussage des Offiziers \textcolor{green}{Rohnstedt} am Ende des ersten \textcolor{green}{Akt}s}}}\label{K_L02788-55h}. Laß ihn noch
               etwas Allgemeines, Kräftiges, Erlöſendes ſagen. Dieſes Wort allein kann den Erfolg
               des \textcolor{green}{Sück}{}\ledrightnote{→\textcolor{green}{Freiwild. Schauspiel in 3 Akten}}es entſcheiden. Nimm’
               meinen \label{K_L02788-7v}\edtext{Rath}{\lemma{\textnormal{\emph{Rath}}}\Cendnote{\textnormal{Über eine Einarbeitung des Vorschlags nichts bekannt.
                  Zumindest der Bezug zu der Affäre wurde noch Jahre später hergestellt,
                  beispielsweise »\begin{otherlanguage}{english}The most celebrated of these was
                     ›\textcolor{green}{Freiwild}‹, an attack on the duel, that received enormous
                        advertizing from the strange coincidence that, while the play was in
                        rehearsal, Lieut. \textcolor{blue}{von Brüsewitz}, by the brutal killing of a
                     civilian in a \textcolor{pink}{Carlsruhe} restaurant, vindicated his ›military honor‹
                        exactly as the play had foretold an officer would be obliged to do. The
                        excitement over the \textcolor{pink}{Carlsruhe} incident rushed the play to such a
                        huge popularity that one of the German comic papers showed a cartoon of
                        Manager \textcolor{blue}{Brahm}, of the \textcolor{brown}{Deutsches Theater}, paying out
                        royalties to the leading playwrights of the season, when Lieut.
                     \textcolor{blue}{Brüsewitz} enters saying: ›I’ve come for my share of the
                     royalties on ›\textcolor{green}{Freiwild}‹!«\end{otherlanguage}« (\emph{\textcolor{green}{Arthur Schnitzler. Dramatist of the Twilight Soul}}. In:
                        \emph{\textcolor{green}{Current Literature}}, Bd. 51, H. 6,
                     Dezember 1911, S. 670–672, hier: 671}}}\label{K_L02788-7h} an, ich glaube, ich habe Dir ſelten ſo gut gerathen!{\dotsfour}\pend
           \pstart
           Auf ein Telegramm am Tage nach der \textsc{Première}{ }{\pb}rechne ich mit Sicherheit.\pend
           \pstart
           Viele treue Grüße!\pend
           \pstart
           Und ein inniges Glückauf!\pend
           \pstart
           Dein treuer {\\[\baselineskip]}\spacefill\mbox{Paul Goldmann}\pend
           \leftskip=0em{}\pstart
           \noindent{}Schönen Gruß an den \textsc{Dr. \textcolor{blue}{Bie}{}\ledrightnote{\textcolor{blue}{Oskar Bie}}}, wenn Du \label{K_L02788-8v}\edtext{ihn ſiehſt}{\lemma{\textnormal{\emph{ihn ſiehſt}}}\Cendnote{\textnormal{\textcolor{blue}{Schnitzler} traf am 31. 10. 1896, 5. 11. 1896 und
                        7. 11. 1896
                     auf \textcolor{blue}{Oskar Bie}.}}}\label{K_L02788-8h}\pend
           \endnumbering\briefempfaengerindex{Schnitzler, Arthur@\textsc{Schnitzler, Arthur}!zzzGoldmann, Paul@\emph{von Paul Goldmann}!1896-10-272@{27. 10. {[}1896{]}}|)be}\mylabel{h}  \normalsize

\doendnotes{C}
\bigskip
\vfill

\clearpage

\footnotesize

\lohead{\textsc{register}}

% Definiere theindex-Environment komplett neu ohne reledmac
\makeatletter
\renewenvironment{theindex}{%
  \section*{\indexname}%
  \setlength{\parindent}{0pt}%
  \setlength{\parskip}{0pt plus 0.3pt}%
  \let\item\@idxitem
}{%
  \clearpage
}
\makeatother

\IfFileExists{\jobname-pw.ind}{\input{\jobname-pw.ind}}{}

\end{document}

      