%% latex-korrekturansicht-vorspann.tex
%% Vorspann für die Korrekturansicht.
%% Lädt die gemeinsame Datei latex-vorspann.tex mit gesetztem Schalter.

\newif\ifkorrekturansicht
\korrekturansichttrue

\input{../tex-inputs/latex-vorspann}


               \section[Hugo von Hofmannsthal an Arthur Schnitzler, 11. 9. 1903]{ Hugo von Hofmannsthal an Arthur Schnitzler, 11. 9. 1903}\nopagebreak\mylabel{v}\rehead{ }\normalsize\beginnumbering\briefempfaengerindex{Schnitzler, Arthur@\textsc{Schnitzler, Arthur}!zzzHofmannsthal, Hugo von@\emph{von Hugo von Hofmannsthal}!1903-09-111@{11. 9. 1903}|(be} \toendnotes[C]{\smallbreak\pagebreak[2]} \Standort{CUL, Schnitzler, B 43.}
\physDesc{Postkarte
\newline{}Handschrift: schwarze Tinte, deutsche Kurrent\newline{}Versand: 1) mit Bleistift von unbekannter Hand nachgesandt nach »\textsc{\textcolor{pink}{XVIII Spöttelgasse 7}.}« 2) Stempel: »\nobreak{}\oindex{Rodaun@\textbf{Rodaun}, \emph{Teil eines besiedelten Ortes (A.BSOX)}|pwk}Rodaun, 12 9 03, 9–11V\nobreak{}«. 3) Stempel: »\nobreak{}\oindex{IX., Alsergrund@\textbf{IX., Alsergrund}, \emph{Bezirk (A.BZK)}|pwk}Wien 9/3, 13. 9. 03, Bestellt\nobreak{}«. 4) Stempel: »\nobreak{}\oindex{XVIII., Waehring@\textbf{XVIII., Währing}, \emph{Bezirk (A.BZK)}|pwk}18/1 Wien, 14. 09. 03, 10.V, Bestellt\nobreak{}«. 
\newline{}Schnitzler: mit Bleistift auf das Datum des Poststempels – des Samstags –
                                 datiert: »12/9 903.« \newline{}Ordnung: 1) mit Bleistift von unbekannter Hand nummeriert: »\strikeout{219}« 2) mit Bleistift von unbekannter Hand nummeriert: »200«}\buchAbdrucke{\weitereDrucke{Hugo von Hofmannsthal, Arthur Schnitzler: \emph{Briefwechsel}. Hg. Therese Nickl und Heinrich Schnitzler. Frankfurt am Main: \emph{S. Fischer} 1964, S. 173–174.} }\toendnotes[C]{\smallbreak}\pstart{}{\pb}\textsc{Herrn D\textsuperscript{r} Arthur Schnitzler}\pend{}\pstart{}\textcolor{pink}{\textsc{Wien}}{}\ledrightnote{\textcolor{pink}{Wien}}\pend{}\pstart{}\textcolor{pink}{\textsc{IX. Franckgasse 1}.}{}\ledrightnote{\textcolor{pink}{Frankgasse}}\pend{}{\bigskip}\pstart
           {\pb}Freitag, nachmittags.\pend
           \pstart
           Hätte die größte Lust, mich in die Dampftramway zu ſetzen und Sie gegen
                  Abend zu beſuchen, wenn ich eine Ahnung hätte, erſtens ob Sie in \textcolor{pink}{Wien}{}\ledrightnote{\textcolor{pink}{Wien}} ſind und zweitens \label{K_L01318_1v}\edtext{\uline{wo} Sie wohnen}{\lemma{\textnormal{\emph{wo Sie wohnen}}}\Cendnote{\textnormal{Am 11. 9. 1903 war er in eine neue Wohnung eines
                  kurz vorher errichteten Gebäude in der \textcolor{pink}{Spöttelgasse 7} (heute: Edmund-Weiß-Gasse) im \textcolor{pink}{18. Wiener
                     Gemeindebezirk} gezogen.}}}\label{K_L01318_1h}. Da auch \textcolor{blue}{Richard}{}\ledrightnote{\textcolor{blue}{Richard Beer-Hofmann}} nicht mehr hier, kann ich beides nicht erfahren. Leider!\pend
           \pstart
           Bitte gleich um ein paar Zeilen.\hspace*{1.5em}Herzlich
                  \spacefill\mbox{Hugo.}\pend
           \endnumbering\briefempfaengerindex{Schnitzler, Arthur@\textsc{Schnitzler, Arthur}!zzzHofmannsthal, Hugo von@\emph{von Hugo von Hofmannsthal}!1903-09-111@{11. 9. 1903}|)be}\mylabel{h}  \normalsize

\doendnotes{C}
\bigskip
\vfill

\clearpage

\footnotesize

\lohead{\textsc{register}}

% Definiere theindex-Environment komplett neu ohne reledmac
\makeatletter
\renewenvironment{theindex}{%
  \section*{\indexname}%
  \setlength{\parindent}{0pt}%
  \setlength{\parskip}{0pt plus 0.3pt}%
  \let\item\@idxitem
}{%
  \clearpage
}
\makeatother

\IfFileExists{\jobname-pw.ind}{\input{\jobname-pw.ind}}{}

\end{document}

      