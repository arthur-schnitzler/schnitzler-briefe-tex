%% latex-korrekturansicht-vorspann.tex
%% Vorspann für die Korrekturansicht.
%% Lädt die gemeinsame Datei latex-vorspann.tex mit gesetztem Schalter.

\newif\ifkorrekturansicht
\korrekturansichttrue

\input{../tex-inputs/latex-vorspann}


\renewcommand{\erwaehntePersonen}{Personen: Paul Morisse, Olga Schnitzler, Stefan Zweig}
\renewcommand{\erwaehnteOrte}{Orte: Mals, Meran, Sternwartestraße 71, Val Venosta, Wien, Währinger Cottage}
\renewcommand{\erwaehnteWerke}{Werke: Das weite Land. Tragikomödie in fünf Akten}
\section[Stefan Zweig an Arthur Schnitzler, 19. 11. 1911]{Stefan Zweig an Arthur Schnitzler, 19. 11. 1911}
\nopagebreak\mylabel{v}
\rehead{ }\normalsize\beginnumbering\briefempfaengerindex{Schnitzler, Arthur@\textsc{Schnitzler, Arthur}!zzzZweig, Stefan@\emph{von Stefan Zweig}!1911-11-191@{19. 11. 1911}|(be}
\toendnotes[C]{\smallbreak\pagebreak[2]}\Standort{CUL, Schnitzler, B 118.}
\physDesc{Bildpostkarte, 1 Blatt, 2 Seiten, 520 Zeichen
\newline{}Handschrift: schwarze Tinte, lateinische Kurrent
\newline{}Versand: Stempel: »\nobreak{}\oindex{Mals@\textbf{Mals}|pwk}Mals – Bozen, 19. XI. 11\nobreak{}«.  }\toendnotes[C]{\smallbreak}\pstart{}{\pb}D\textsuperscript{r} Artur
                  Schnitzler\pend{}\pstart{}\textcolor{pink}{Wien – Cottage}{}\ledrightnote{\textcolor{pink}{Währinger Cottage}}\pend{}\pstart{}\textcolor{pink}{\label{K_L03629-1v}\edtext{Sternwartestrasse 72}{\lemma{\textnormal{\emph{Sternwartestrasse 72}}}\Cendnote{\textnormal{\textcolor{blue}{Zweig} wechselt bei der Adressierung
                        seiner Schreiben an \textcolor{blue}{Schnitzler} immer
                        wieder zwischen der falschen Hausnummer »72« und der
                        richtigen »71«.}}}\label{K_L03629-1h}}{}\ledrightnote{\textcolor{pink}{Sternwartestraße 71}}\pend{}
{\bigskip}
\pstart
           \noindent{}\centering{}\textcolor{gray}{\textbf{{\pb}\textcolor{pink}{Südtirol, Kurort Meran}{}\ledrightnote{\textcolor{pink}{Meran}} geg \textcolor{pink}{Vinschgau}{}\ledrightnote{\textcolor{pink}{Val Venosta}}}}\pend
           
\pstart
           \noindent{}\centering{}\textcolor{gray}{\textbf{Zielspitze\textcolor{red}{\textsuperscript{\textbf{KEY}}}, 3006 m.\hspace*{2em}Gfallwand\textcolor{red}{\textsuperscript{\textbf{KEY}}}, 3179 m.\hspace*{2em}Blasius-Spitze\textcolor{red}{\textsuperscript{\textbf{KEY}}},\hspace*{2em}Roteck\textcolor{red}{\textsuperscript{\textbf{KEY}}} 3331 m.\hspace*{2em}Tschigat\textcolor{red}{\textsuperscript{\textbf{KEY}}}, 2999 m.}}\pend
           {\vspace{1\baselineskip}}
\pstart
           \noindent{}{\pb}Verehrter Herr Doktor, ich habe \textcolor{blue}{Morisse}{}\ledrightnote{\textcolor{blue}{Paul Morisse}} nochmals geschrieben, er möge \label{K_L03629-2v}\edtext{bei einer Übertragung}{\lemma{\textnormal{\emph{bei einer Übertragung}}}\Cendnote{\textnormal{\textcolor{blue}{Zweig} versuchte \textcolor{blue}{Paul Morisse} als Übersetzer für \textcolor{blue}{Schnitzlers} Tragikomödie \emph{\textcolor{green}{Das weite Land}} zu vermitteln, {XXXX ref}.}}}\label{K_L03629-2h} womöglich gemeinsam mit einem
               Theaterroutinièr vorgehen und zweifle nicht, dass er diesen Rat befolgen würde. Er
               ist sehr tüchtig und hat den Vorteil einige Jahre in \textcolor{pink}{\uline{Wien}}{}\ledrightnote{\textcolor{pink}{Wien}} gelebt zu haben und das Specifisch-\textcolor{pink}{Wienerische}{}\ledrightnote{\textcolor{pink}{Wien}} besser wiederzugeben. Ich hoffe Sie bald darüber nach {\pb}meiner Rückkunft sprechen zu können und
               grüsse Ihre Frau \textcolor{blue}{Gemahlin}{}\ledrightnote{{$\rightarrow$}\textcolor{blue}{Olga Schnitzler}} und
               Sie viele Male als Ihr getreuer\pend
           \pstart \spacefill\mbox{Stefan Zweig}\pend{}\endnumbering\briefempfaengerindex{Schnitzler, Arthur@\textsc{Schnitzler, Arthur}!zzzZweig, Stefan@\emph{von Stefan Zweig}!1911-11-191@{19. 11. 1911}|)be}\mylabel{h}
\begin{anhang}
\end{anhang}\normalsize

\doendnotes{C}
\bigskip
\vfill

\clearpage

\footnotesize

\lohead{\textsc{register}}

% Definiere theindex-Environment komplett neu ohne reledmac
\makeatletter
\renewenvironment{theindex}{%
  \section*{\indexname}%
  \setlength{\parindent}{0pt}%
  \setlength{\parskip}{0pt plus 0.3pt}%
  \let\item\@idxitem
}{%
  \clearpage
}
\makeatother

\IfFileExists{\jobname-pw.ind}{\input{\jobname-pw.ind}}{}

\end{document}

      