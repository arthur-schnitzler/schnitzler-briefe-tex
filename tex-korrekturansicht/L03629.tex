%% latex-korrekturansicht-vorspann.tex
%% Vorspann für die Korrekturansicht.
%% Lädt die gemeinsame Datei latex-vorspann.tex mit gesetztem Schalter.

\newif\ifkorrekturansicht
\korrekturansichttrue

\input{../tex-inputs/latex-vorspann}


\section[Stefan Zweig an Arthur Schnitzler, 19. 11. 1911]{L03629 Stefan Zweig an Arthur Schnitzler, 19. 11. 1911}
\nopagebreak\mylabel{L03629v}
\rehead{ }\normalsize\beginnumbering\briefempfaengerindex{Schnitzler, Arthur@\textsc{Schnitzler, Arthur}!zzzZweig, Stefan@\emph{von Stefan Zweig}!1911-11-191@{19. 11. 1911}|(be}
\toendnotes[C]{\smallbreak\pagebreak[2]}\Standort{CUL, Schnitzler, B 118.}
\physDesc{Bildpostkarte, 517 Zeichen
\newline{}Handschrift: schwarze Tinte, lateinische Kurrent
\newline{}Versand: Stempel: »\nobreak{}\oindex{Mals@\textbf{Mals}|pwk}Mals-Bozen, 19. XI. 11\nobreak{}«.  }
\buchAbdrucke{\weitereDrucke{Stefan Zweig: \emph{Briefwechsel mit Hermann Bahr, Sigmund Freud, Rainer Maria
                        Rilke und Arthur Schnitzler}. Frankfurt am Main: \emph{S. Fischer} 1987, S. 368.} }\toendnotes[C]{\smallbreak}\pstart{}{\pb}D\textsuperscript{r} Artur
                  Schnitzler\pend{}\pstart{}\textcolor{pink}{Wien – Cottage}\oindex{Waehringer Cottage@\textbf{Währinger Cottage}|pw}{}\ledrightnote{\textcolor{pink}{Währinger Cottage}}\pend{}\pstart{}\textcolor{pink}{\label{K_L03629-1v}\edtext{Sternwartestrasse 72}{\lemma{\textnormal{\emph{Sternwartestrasse 72}}}\Cendnote{\textnormal{\textcolor{blue}{Zweig}\pwindex{Zweig, Stefan 28.11.1881 – 23.02.1942@\textsc{Zweig, Stefan} (28.11.1881 – 23.02.1942), \emph{Schriftsteller/Schriftstellerin}|pwk} wechselt bei der Adressierung
                        seiner Schreiben an \textcolor{blue}{Schnitzler} immer
                        wieder zwischen der falschen Hausnummer »72« und der
                        richtigen »71«.}}}\label{K_L03629-1}}\oindex{Sternwartestrasse 71@\textbf{Sternwartestraße 71}|pw}{}\ledrightnote{\textcolor{pink}{Sternwartestraße 71}}\pend{}{\bigskip}
\pstart
           \noindent{}\centering{}\textcolor{gray}{\textbf{{\pb}\textcolor{pink}{Südtirol, Kurort Meran}\oindex{Meran@\textbf{Meran}|pw}{}\ledrightnote{\textcolor{pink}{Meran}} geg \textcolor{pink}{Vinschgau}\oindex{Val Venosta@\textbf{Val Venosta}|pw}{}\ledrightnote{\textcolor{pink}{Val Venosta}}.
               }}\pend
           
\pstart
           \centering{}\textcolor{gray}{\textbf{\textcolor{pink}{Zielspitze}\oindex{Zielspitze@\textbf{Zielspitze}|pw}{}\ledrightnote{\textcolor{pink}{Zielspitze}}, 3006 m.\hspace*{2em}\textcolor{pink}{Gfallwand}\oindex{Gfallwand@\textbf{Gfallwand}|pw}{}\ledrightnote{\textcolor{pink}{Gfallwand}}, 3179 m.\hspace*{2em}\textcolor{pink}{Blasius-Spitze}\oindex{Blasiuszeiger@\textbf{Blasiuszeiger}|pw}{}\ledrightnote{\textcolor{pink}{Blasiuszeiger}}.\hspace*{2em}\textcolor{pink}{Roteck}\oindex{Roteck@\textbf{Roteck}|pw}{}\ledrightnote{\textcolor{pink}{Roteck}} 3331 m.\hspace*{2em}\textcolor{pink}{Tschigat}\oindex{Tschigat@\textbf{Tschigat}|pw}{}\ledrightnote{\textcolor{pink}{Tschigat}}, 2999 m.}}\pend
           {\vspace{1\baselineskip}}\vspace{1em}
\pstart
           \noindent{}{\pb}Verehrter Herr Doktor, ich habe \textcolor{blue}{Morisse}\pwindex{Morisse, Paul 1866-03-11 – 1946-09-28@\textsc{Morisse, Paul} (1866-03-11 – 1946-09-28), \emph{Übersetzer/Übersetzerin}|pw}{}\ledrightnote{\textcolor{blue}{Paul Morisse}} nochmals geschrieben, er möge \label{K_L03629-2v}\edtext{bei einer Übertragung}{\lemma{\textnormal{\emph{bei einer Übertragung}}}\Cendnote{\textnormal{Vgl. Stefan Zweig an Arthur Schnitzler, 6. [11.?] 1911.}}}\label{K_L03629-2} womöglich gemeinsam mit einem \textcolor{blue}{Theaterroutinièr}\pwindex{Charasson, Henriette 1884-01-06 – 1972-12-24@\textsc{Charasson, Henriette} (1884-01-06 – 1972-12-24), \emph{Schriftsteller/Schriftstellerin}|pwv}{}\ledrightnote{{$\rightarrow$}\emph{\textcolor{blue}{Henriette Charasson}}} vorgehen
               und zweifle nicht, dass er diesen Rat befolgen würde. Er ist sehr tüchtig und hat den
               Vorteil einige Jahre in \textcolor{pink}{\uline{Wien}}\oindex{Wien@\textbf{Wien}|pw}{}\ledrightnote{\textcolor{pink}{Wien}} gelebt zu haben und das Specifisch-\textcolor{pink}{Wienerische}\oindex{Wien@\textbf{Wien}|pw}{}\ledrightnote{\textcolor{pink}{Wien}} besser wiederzugeben. Ich hoffe Sie bald darüber nach {\pb}meiner Rückkunft sprechen zu können und
               grüsse Ihre Frau \textcolor{blue}{Gemahlin}\pwindex{Schnitzler, Olga 17.01.1882 – 13.01.1970@\textsc{Schnitzler, Olga} (17.01.1882 – 13.01.1970), \emph{Schauspieler/Schauspielerin, Sänger/Sängerin}|pwv}{}\ledrightnote{{$\rightarrow$}\emph{\textcolor{blue}{Olga Schnitzler}}} und
               Sie viele Male als Ihr getreuer\pend
           \pstart \spacefill\mbox{Stefan Zweig}\pend{}\selectlanguage{ngerman}\endnumbering\briefempfaengerindex{Schnitzler, Arthur@\textsc{Schnitzler, Arthur}!zzzZweig, Stefan@\emph{von Stefan Zweig}!1911-11-191@{19. 11. 1911}|)be}\mylabel{L03629h}  \normalsize

\doendnotes{C}
\bigskip
\vfill

\clearpage

\footnotesize

\lohead{\textsc{register}}

% Definiere theindex-Environment komplett neu ohne reledmac
\makeatletter
\renewenvironment{theindex}{%
  \section*{\indexname}%
  \setlength{\parindent}{0pt}%
  \setlength{\parskip}{0pt plus 0.3pt}%
  \let\item\@idxitem
}{%
  \clearpage
}
\makeatother

\IfFileExists{\jobname-pw.ind}{\input{\jobname-pw.ind}}{}

\end{document}

      