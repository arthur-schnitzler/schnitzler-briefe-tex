%% latex-korrekturansicht-vorspann.tex
%% Vorspann für die Korrekturansicht.
%% Lädt die gemeinsame Datei latex-vorspann.tex mit gesetztem Schalter.

\newif\ifkorrekturansicht
\korrekturansichttrue

\input{../tex-inputs/latex-vorspann}


\renewcommand{\erwaehntePersonen}{Personen: Otto Brahm, Theodor Fontane, Paul Goldmann, Vally Rosengart, Paul Schlenther, Friedrich Stephany}
\renewcommand{\erwaehnteInstitutionen}{Institutionen: Burgtheater, Vossische Zeitung}
\renewcommand{\erwaehnteOrte}{Orte: Frankfurt am Main, Wien}
\renewcommand{\erwaehnteWerke}{}
\section[Vally Rosengart an Arthur Schnitzler, {[}16. 1. 1898{]}]{Vally Rosengart an Arthur Schnitzler, {[}16. 1. 1898{]}}
\nopagebreak\mylabel{v}
\rehead{ }\normalsize\beginnumbering\briefempfaengerindex{Schnitzler, Arthur@\textsc{Schnitzler, Arthur}!zzzRosengart, Vally@\emph{von Vally Rosengart}!1898-01-162@{{[}16. 1. 1898{]}}|(be}
\toendnotes[C]{\smallbreak\pagebreak[2]}\Standort{DLA, A:Schnitzler, HS.NZ85.1.4334.}
\physDesc{Telegramm, 424 Zeichen
\newline{}maschinell
\newline{}Schnitzler: mit rotem Buntstift drei Unterstreichungen 
\newline{}Ordnung: beschnitten }\toendnotes[C]{\smallbreak}
\pstart
           \centering{}{\pb}de \textcolor{pink}{frankfurtm}{}\ledrightnote{\textcolor{pink}{Frankfurt am Main}}
                  854 62 16/1{ }4 58 s=\pend
           
\pstart
           \textcolor{blue}{paul}{}\ledrightnote{\textcolor{blue}{Paul Goldmann}} wuenscht, ohne persoenlich
               hervorzutreten, fuer \label{K_L02797-1v}\edtext{\textcolor{blue}{schlenther}{}\ledrightnote{\textcolor{blue}{Paul Schlenther}}s nachfolge}{\lemma{\textnormal{\emph{schlenthers nachfolge}}}\Cendnote{\textnormal{\textcolor{blue}{Paul Schlenther} war 1886 als Nachfolger von \textcolor{blue}{Theodor
                     Fontane} zur \emph{\textcolor{brown}{Vossischen Zeitung}}
                  gekommen. Als er Mitte Januar 1898 zum neuen \emph{\textcolor{brown}{Burgtheater}}-Direktor ernannt wurde, wurde seine
                  Position vakant. Damit erläutern sich zwei bislang kryptische Stellen in der
                  Korrespondenz zwischen \textcolor{blue}{Schnitzler} und \textcolor{blue}{Otto Brahm}. \textcolor{blue}{Schnitzler} kontaktierte \textcolor{blue}{Brahm} im
                  Sinne \textcolor{blue}{Goldmann}s, woraufhin \textcolor{blue}{Brahm} am 18. 1. 1898 mit einem Telegramm antwortete: »Ihren \textcolor{blue}{Kandidaten}{ }\textcolor{blue}{Schlenther} empfohlen.« Am
                     21. 1. 1898 verfasste \textcolor{blue}{Schlenther} einen Brief an \textcolor{blue}{Schnitzler}, in dem er angibt, er habe die Anfrage an den Chefredakteur
                     \textcolor{blue}{Friedrich Stephany} weitergereicht, doch
                  dürfte die Stelle erst im Herbst nachbesetzt werden. In einem Antwortbrief \textcolor{blue}{Schnitzler}s an \textcolor{blue}{Brahm} vom 22. 1. 1898 wird
                  die Sache zum letzten Mal angesprochen: »Für Ihre Verwendung betreffs \textcolor{blue}{Goldmann} noch einmal herzlichsten Dank.
                     Er selbst wußte nichts davon; nur seine Verwandten; heute weiß er es natürlich.
                     Halten Sie einen Erfolg für möglich?« (\emph{Der Briefwechsel Arthur Schnitzler — Otto Brahm}.
                        Vollständige Ausgabe. Herausgegeben, eingeleitet und erläutert von Oskar
                        Seidlin. Tübingen: \emph{Niemeyer}{ }1975, S. 42–43). Zugleich erlaubt diese Stelle die Datierung zusammen mit der Monats- und Tagesangabe in der
                  Übermittlungszeile des Telegramms.}}}\label{K_L02797-1h} bei \textcolor{brown}{vossischer}{}\ledrightnote{\textcolor{brown}{Vossische Zeitung}} zu candidiren und bittet sie, schnellstens und nachdruecklichst
               in diesem sinne zu wirken. vielleicht machen sie \textcolor{blue}{brahm}{}\ledrightnote{\textcolor{blue}{Otto Brahm}} telegraphisch aufmerksam, dasz \textcolor{blue}{goldmann}{}\ledrightnote{\textcolor{blue}{Paul Goldmann}} zu haben waere, betonen seine glaenzende eignung und ersuchen \textcolor{blue}{brahm}{}\ledrightnote{\textcolor{blue}{Otto Brahm}} zu interveniren. herzlichen dank fuer
               alles, was sie dem \textcolor{blue}{freunde}{}\ledrightnote{{$\rightarrow$}\textcolor{blue}{Paul Goldmann}}
               thun = \spacefill\mbox{rosengart – \textcolor{blue}{goldmann}{}\ledrightnote{\textcolor{blue}{Paul Goldmann}}.}\pend
           \endnumbering\briefempfaengerindex{Schnitzler, Arthur@\textsc{Schnitzler, Arthur}!zzzRosengart, Vally@\emph{von Vally Rosengart}!1898-01-162@{{[}16. 1. 1898{]}}|)be}\mylabel{h}  \normalsize

\doendnotes{C}
\bigskip
\vfill

\clearpage

\footnotesize

\lohead{\textsc{register}}

% Definiere theindex-Environment komplett neu ohne reledmac
\makeatletter
\renewenvironment{theindex}{%
  \section*{\indexname}%
  \setlength{\parindent}{0pt}%
  \setlength{\parskip}{0pt plus 0.3pt}%
  \let\item\@idxitem
}{%
  \clearpage
}
\makeatother

\IfFileExists{\jobname-pw.ind}{\input{\jobname-pw.ind}}{}

\end{document}

      