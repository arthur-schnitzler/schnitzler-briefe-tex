%% latex-korrekturansicht-vorspann.tex
%% Vorspann für die Korrekturansicht.
%% Lädt die gemeinsame Datei latex-vorspann.tex mit gesetztem Schalter.

\newif\ifkorrekturansicht
\korrekturansichttrue

\input{../tex-inputs/latex-vorspann}


               \section[Richard Beer-Hofmann an Arthur Schnitzler, {[}27. 12. 1910{]}]{ Richard Beer-Hofmann an Arthur Schnitzler, {[}27. 12. 1910{]}}\nopagebreak\mylabel{v}\rehead{ }\normalsize\beginnumbering\briefempfaengerindex{Schnitzler, Arthur@\textsc{Schnitzler, Arthur}!zzzBeer-Hofmann, Richard@\emph{von Richard Beer-Hofmann}!1910-12-271@{{[}27. 12. 1910{]}}|(be} \toendnotes[C]{\smallbreak\pagebreak[2]} \Standort{CUL, Schnitzler, B 8.}
\physDesc{Kartenbrief, 1 Blatt, 3 Seiten
\newline{}Handschrift: blauer Buntstift, lateinische Kurrent\newline{}Versand: ohne postalischen Übermittlungsvermerk 
\newline{}Schnitzler: mit Bleistift beschriftet: »\textsc{BH}« und datiert: »27/12« \newline{}Ordnung: mit Bleistift von unbekannter Hand nummeriert:
                                    »226a« und datiert: »1909?« }\toendnotes[C]{\smallbreak}\pstart{}{\pb}Herrn\pend{}\pstart{}Arthur Schnitzler\pend{}{\bigskip}\pstart
           \noindent{}{\pb}Lieber Arthur! \textcolor{blue}{Gerty}{}\ledrightnote{\textcolor{blue}{Gertrude von Hofmannsthal}} teleph.: \textcolor{blue}{Hugo}{}\ledrightnote{\textcolor{blue}{Hugo von Hofmannsthal}}{ }\label{K_L01995_1v}\edtext{reist}{\lemma{\textnormal{\emph{reist}}}\Cendnote{\textnormal{\textcolor{blue}{Hugo} und \textcolor{blue}{Gerty
                     von Hofmannsthal} reisten am 28. 12. 1910  nach \textcolor{pink}{Neubeuern}.}}}\label{K_L01995_1h} morgen ab; da Sie ihn sprechen
               wollen, lässt er sagen es gienge heute{ }5½–6½ im Caffee \textcolor{pink}{Pucher}{}\ledrightnote{\textcolor{pink}{Café Pucher}}, oder – er geht
               zu \textcolor{green}{Anatol}{}\ledrightnote{\textcolor{green}{Anatol}} – in einem Zwischenakt im Foyer – Sie
               müssten {\pb}aber sagen – in welchem?
               Ihre Antwort werde ich zu \textcolor{blue}{Schlesinger}{}\ledrightnote{\textcolor{blue}{Franziska Schlesinger}}s
               telephoniren.\pend
           \pstart
           Herzlichst{\\[\baselineskip]}\spacefill\mbox{Richard}\pend
           \leftskip=0em{}\endnumbering\briefempfaengerindex{Schnitzler, Arthur@\textsc{Schnitzler, Arthur}!zzzBeer-Hofmann, Richard@\emph{von Richard Beer-Hofmann}!1910-12-271@{{[}27. 12. 1910{]}}|)be}\mylabel{h}  \normalsize

\doendnotes{C}
\bigskip
\vfill

\clearpage

\footnotesize

\lohead{\textsc{register}}

% Definiere theindex-Environment komplett neu ohne reledmac
\makeatletter
\renewenvironment{theindex}{%
  \section*{\indexname}%
  \setlength{\parindent}{0pt}%
  \setlength{\parskip}{0pt plus 0.3pt}%
  \let\item\@idxitem
}{%
  \clearpage
}
\makeatother

\IfFileExists{\jobname-pw.ind}{\input{\jobname-pw.ind}}{}

\end{document}

      