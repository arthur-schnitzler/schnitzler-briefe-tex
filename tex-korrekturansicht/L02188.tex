%% latex-korrekturansicht-vorspann.tex
%% Vorspann für die Korrekturansicht.
%% Lädt die gemeinsame Datei latex-vorspann.tex mit gesetztem Schalter.

\newif\ifkorrekturansicht
\korrekturansichttrue

\input{../tex-inputs/latex-vorspann}


               \section[Arthur Schnitzler an Richard Beer-Hofmann, 25. 7. 1914]{ Arthur Schnitzler an Richard Beer-Hofmann, 25. 7. 1914}\nopagebreak\mylabel{v}\rehead{ }\normalsize\beginnumbering\briefempfaengerindex{Beer-Hofmann, Richard@\textsc{Beer-Hofmann, Richard}!zzzSchnitzler, Arthur@\emph{von Arthur Schnitzler}!1914-07-251@{25. 7. 1914}|(be} \toendnotes[C]{\smallbreak\pagebreak[2]} \Standort{YCGL, MSS 31.}
\physDesc{Bildpostkarte
\newline{}Handschrift: Bleistift, deutsche Kurrent\newline{}Versand: Stempel: »\nobreak{}\oindex{Celerina@\textbf{Celerina}, \emph{Besiedelter Ort (A.BSO)}|pwk}Celerin\textcolor{gray}{a}
                                          (\textcolor{gray}{Graubünden}), 25. VII. 14, 5\nobreak{}«.  
\newline{}Beer-Hofmann: mit blauem Buntstift den Erhalt
                                 markiert: »E« }\toendnotes[C]{\smallbreak}\pstart{}{\pb}Hrn \textsc{Dr Richard
                        Beerhofma{\geminationn}}\pend{}\pstart{}\textsc{\textcolor{pink}{Weißenbach}{}\ledrightnote{\textcolor{pink}{Weißenbach am Attersee}}.}\pend{}\pstart{}\textsc{Am \textcolor{pink}{Attersee}{}\ledrightnote{\textcolor{pink}{Attersee}}}\pend{}\pstart{}\textcolor{pink}{\textsc{Oberoesterreic{[}h{]}}}{}\ledrightnote{\textcolor{pink}{Oberösterreich}}\pend{}\pstart{}\textcolor{pink}{\textsc{Austria}}{}\ledrightnote{\textcolor{pink}{Österreich}}\pend{}{\bigskip}\pstart
           \noindent{}\centering{}{\pb}\textcolor{gray}{\textbf{\textcolor{pink}{Celerina}{}\ledrightnote{\textcolor{pink}{Celerina}} gegen \textcolor{pink}{Pontresina}{}\ledrightnote{\textcolor{pink}{Pontresina}} ges.}}\pend
           \pstart
           {\pb}Aus \textcolor{pink}{Pontresina}{}\ledrightnote{\textcolor{pink}{Pontresina}} (zu
                  lärmen\textcolor{gray}{d} – \textsc{Hotel} ſowohl als Ort)
               hieher \label{KLL02188_Beer-Hofmann-1v}\edtext{überſiedelt}{\lemma{\textnormal{\emph{überſiedelt}}}\Cendnote{\textnormal{siehe A. S.: \emph{Tagebuch}, 21. 7. 1914}}}\label{KLL02188_Beer-Hofmann-1h} (\textsc{\textcolor{pink}{Cresta Palace, Celerina}{}\ledrightnote{\textcolor{pink}{Cresta Palace}}}) und höchſt befriedigt, grüßen wir Sie alle {\pb}herzlichſt. Ebenſo \textcolor{blue}{Gustav}{}\ledrightnote{\textcolor{blue}{Gustav Schwarzkopf}}, der wohl
                  \textcolor{gray}{und} bei Ihnen iſt?\pend
           \pstart Ihr \spacefill\mbox{A.}\pend{}\pstart
           25/7 914\pend
           \endnumbering\briefempfaengerindex{Beer-Hofmann, Richard@\textsc{Beer-Hofmann, Richard}!zzzSchnitzler, Arthur@\emph{von Arthur Schnitzler}!1914-07-251@{25. 7. 1914}|)be}\mylabel{h}  \normalsize

\doendnotes{C}
\bigskip
\vfill

\clearpage

\footnotesize

\lohead{\textsc{register}}

% Definiere theindex-Environment komplett neu ohne reledmac
\makeatletter
\renewenvironment{theindex}{%
  \section*{\indexname}%
  \setlength{\parindent}{0pt}%
  \setlength{\parskip}{0pt plus 0.3pt}%
  \let\item\@idxitem
}{%
  \clearpage
}
\makeatother

\IfFileExists{\jobname-pw.ind}{\input{\jobname-pw.ind}}{}

\end{document}

      