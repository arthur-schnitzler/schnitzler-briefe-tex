%% latex-korrekturansicht-vorspann.tex
%% Vorspann für die Korrekturansicht.
%% Lädt die gemeinsame Datei latex-vorspann.tex mit gesetztem Schalter.

\newif\ifkorrekturansicht
\korrekturansichttrue

\input{../tex-inputs/latex-vorspann}


               \section[ Paul Goldmann an Arthur Schnitzler, 10. 12. {[}1897{]}]{Paul Goldmann an Arthur Schnitzler, 10. 12. {[}1897{]}}\nopagebreak\mylabel{v}\rehead{ }\normalsize\beginnumbering\briefempfaengerindex{Schnitzler, Arthur@\textsc{Schnitzler, Arthur}!zzzGoldmann, Paul@\emph{von Paul Goldmann}!1897-12-101@{10. 12. {[}1897{]}}|(be} \toendnotes[C]{\smallbreak\pagebreak[2]} \Standort{DLA, A:Schnitzler, HS.NZ85.1.3167.}
\physDesc{Brief, 2 Blätter, 7 Seiten
\newline{}Handschrift: blaue Tinte, deutsche Kurrent
\newline{}Schnitzler: 1) mit Bleistift das Jahr »97« vermerkt 2) mit rotem Buntstift drei Unterstreichungen}\toendnotes[C]{\smallbreak}\pstart
           \noindent{}{\pb}\textcolor{gray}{\textbf{\textbf{\textcolor{brown}{Frankfurter Zeitung}{}\ledrightnote{\textcolor{brown}{Frankfurter Zeitung}}}}}\pend
           \pstart
           \textcolor{gray}{\textbf{(\textcolor{brown}{\begin{otherlanguage}{french}Gazette de Francfort\end{otherlanguage}}{}\ledrightnote{\textcolor{brown}{Frankfurter Zeitung}}).}}\pend
           \pstart
           \textcolor{gray}{\textbf{\textbf{\begin{otherlanguage}{french}Fondateur M.\end{otherlanguage}{ }\textcolor{blue}{L. Sonnemann}{}\ledrightnote{\textcolor{blue}{Leopold Sonnemann}}.}}}\pend
           \pstart
           \begin{otherlanguage}{french}\textcolor{gray}{\textbf{Journal politique, financier,}}\end{otherlanguage}\pend
           \pstart
           \begin{otherlanguage}{french}\textcolor{gray}{\textbf{commercial et littéraire.}}\end{otherlanguage}\pend
           \pstart
           \begin{otherlanguage}{french}\textcolor{gray}{\textbf{\textbf{Paraissant trois fois par jour.}}}\end{otherlanguage}\hfill \textsc{\textcolor{pink}{Paris}{}\ledrightnote{\textcolor{pink}{Paris}}}, 10. December.\pend
           \pstart
           \begin{otherlanguage}{french}\textcolor{gray}{\textbf{\textbf{Bureau à \textcolor{pink}{Paris}{}\ledrightnote{\textcolor{pink}{Paris}}}}}\end{otherlanguage}\pend
           \pstart
           \begin{otherlanguage}{french}\textcolor{gray}{\textbf{\textbf{\textcolor{pink}{10 Rue de la Bourse}{}\ledrightnote{\textcolor{pink}{rue de la Bourse}}.}}}\end{otherlanguage}\pend
           \pstart\center{}Mein lieber Freund,\pend\pstart
           Endlich ein freier Augenblick! Ich habe eine Reihe furchtbar aufgeregter Tage hinter
               mir. Die Geſchichte fing an mit einem \label{K_L02833-1v}\edtext{\textcolor{green}{Artikel}{}\ledrightnote{→\textcolor{green}{Aux Syndiqués de Francfort}} von \textsc{\textcolor{blue}{Millevoye}{}\ledrightnote{\textcolor{blue}{Lucien Millevoye}}}}{\lemma{\textnormal{\emph{Artikel von Millevoye}}}\Cendnote{\textnormal{Ende November und Anfang Dezember 1897
                  erschienen fast täglich Kommentare zur Affäre \textcolor{blue}{Dreyfus} von \textcolor{blue}{Lucien Millevoye} in
                  der von ihm geleiteten Zeitung \emph{\textcolor{green}{La Patrie}}. \textcolor{blue}{Goldmann} bezog sich auf folgenden \textcolor{green}{Artikel}: \textcolor{blue}{Lucien Millevoye}: \emph{\textcolor{green}{Aux Syndiqués de Francfort}}. In: \emph{\textcolor{green}{La Patrie. Organe de la defense nationale}}, Jg. 57,
                     Nr. 5, 4. 12. 1897, S. 1. Als Beilage ist
                  der \textcolor{green}{Artikel} nicht
                  erhalten.}}}\label{K_L02833-1h}, der mich mit Koth bewarf. Ich lege ihn Dir bei, damit Du ſiehſt,
                  \strikeout{in} welchen Ton die Polemik in dieſen heißen Tagen
               angenommen hat und was man ſich Alles ſagen laſſen muß, wenn man ruhig und beſcheiden
               für ſeine Überzeugung eintritt. Sonntag kam der
                  \label{K_L02833-2v}\edtext{Einbruch}{\lemma{\textnormal{\emph{Einbruch}}}\Cendnote{\textnormal{Darüber wurde auch berichtet: [o. V.:] \emph{\textcolor{green}{À la chambre}}. In: \emph{\textcolor{green}{L’Express du Midi. Organe quotidien de Défense Sociale et Religieuse}},
                     Jg. 7, Nr. 2077, 7. 12. 1897,
                  S. [2].}}}\label{K_L02833-2h}, von dem Du wohl in den Blättern geleſen haſt. Man hat mir
               meine Briefe geſtohlen, {\pb}Briefe von meiner Familie
               und von Dir. Wahrſcheinlich war der Einbruch eine verkleidete Hausſuchung. Irgend ein
               officieller Dummkopf hat vielleicht geglaubt, daß \substVorne{}\textsuperscript{\textcolor{gray}{×}}\substDazwischen{}e\substHinten{}r bei mir Documente zum Fall \textsc{\textcolor{blue}{Dreyfus}{}\ledrightnote{\textcolor{blue}{Alfred Dreyfus}}} finden könnte oder \strikeout{doc} documentariſche Beweiſe
               für die Exiſtenz des famoſen »\label{K_L02833-3v}\edtext{Syndicats}{\lemma{\textnormal{\emph{Syndicats}}}\Cendnote{\textnormal{Bezug auf das
                  vermeintliche »Judensyndikat« hinter der \textcolor{blue}{Dreyfus}-Affäre (vgl. \textcolor{blue}{Emile Zola}: \emph{\textcolor{green}{Le Syndicat}}. In: \emph{\textcolor{green}{Le
                        Figaro}}, Jg. 43, Nr. 335, 1. 12. 1897,
                     S. 1).}}}\label{K_L02833-3h}« (das nie exiſtirt hat). Tagelang hat ſich hier die Preſſe
               mit mir beſchäftigt, und obwohl kein böſes Wort gegen mich gefallen iſt, ſo iſt es
               doch unheimlich, als \textcolor{pink}{Deutſch}{}\ledrightnote{→\textcolor{pink}{Deutschland}}er
               in ſo leidenſchaftlich bewegter Zeit im Mittelpunkt des Intereſſes zu ſtehen.\pend
           \pstart
           {\pb}Endlich alſo kann ich ein wenig aufathmen, und
               endlich kann ich Dir Deinen ſo lieben und ſchönen Brief beantworten. Ich habe mich
               von Herzen über Deine \label{K_L02833-4v}\edtext{\textcolor{pink}{Prag}{}\ledrightnote{\textcolor{pink}{Prag}}er \textcolor{green}{Erfolge}{}\ledrightnote{→\textcolor{green}{Freiwild. Schauspiel in 3 Akten}{\newline}→\textcolor{green}{Die Toten schweigen}{\newline}→\textcolor{green}{Weihnachts-Einkäufe}}}{\lemma{\textnormal{\emph{Prager Erfolge}}}\Cendnote{\textnormal{siehe Paul Goldmann an Arthur Schnitzler, 23. 12. [1897]}}}\label{K_L02833-4h} gefreut. Es iſt gut, daß das Alles noch vor die Zeit des \label{K_L02833-5v}\edtext{Aufruhrs}{\lemma{\textnormal{\emph{Aufruhrs}}}\Cendnote{\textnormal{Auslöser waren gewaltvolle Proteste als Reaktion auf die \textcolor{blue}{Badeni}sche
                  Sprachverordnung, die sich von Ende November bis
                  Anfang Dezember 1897 erstreckten. Auch \textcolor{blue}{Schnitzler} notierte die »Unruhen,
                     politischer Natur« am 28. 11. 1897 – vor seiner Abreise aus Prag – im
                     \emph{\textcolor{green}{Tagebuch}}.}}}\label{K_L02833-5h} gefallen iſt, ſonſt wäre
               es für Dich auch recht ungemüthlich in \textsc{\textcolor{pink}{Prag}{}\ledrightnote{\textcolor{pink}{Prag}}} geworden. Mich erſtaunt nur, daß Du Dich ſonſt nicht wohler dort gefühlt haſt.
               Denn es ſoll eine ſehr ſchöne \textcolor{pink}{Stadt}{}\ledrightnote{→\textcolor{pink}{Prag}} ſein.\pend
           \pstart
           Für Deinen Bericht über das kleine \label{K_L02833-6v}\edtext{\textcolor{blue}{Fräulein}{}\ledrightnote{→\textcolor{blue}{Alice Ziegler}}}{\lemma{\textnormal{\emph{Fräulein}}}\Cendnote{\textnormal{siehe Paul Goldmann an Arthur Schnitzler, 19. 11. [1897]}}}\label{K_L02833-6h} danke ich Dir von ganzem Herzen. Er hat mich ſehr nachdenklich geſtimmt.
               Deine Beobachtungen {\pb}ſind zweifelsohne richtig,
               Deine \strikeout{Schluſ\textcolor{gray}{ſe}} Schlüſſe nicht weniger. Es wäre vielleicht ſehr unklug von mir, wenn ich
               irgend etwas thäte. Ich werde auch wahrſcheinlich nichts thun. Aber andrerſeits übt
               gerade dieſe halbe Kindlichkeit auf mich \strikeout{e\textcolor{gray}{×}} einen ungeheuren Reiz aus. Du meinſt, das ſei \textsc{Perversion}. Ich weiß es nicht, aber der Reiz beſteht. Und er wird
               hundertfach verſtärkt durch das \textcolor{pink}{Pariſ}{}\ledrightnote{\textcolor{pink}{Paris}}er Leben.
               Wenn man ſo Jahre lang mitten unter \label{K_L02833-8v}\edtext{\begin{otherlanguage}{french}Raffinement\end{otherlanguage}}{\lemma{\textnormal{\emph{Raffinement}}}\Cendnote{\textnormal{Fremdwort mit Ursprung im Französischen:
                  Feinheit}}}\label{K_L02833-8h} und Proſtitution gelebt hat (wie es das Loos des Fremden in \textsc{\textcolor{pink}{Paris}{}\ledrightnote{\textcolor{pink}{Paris}}} iſt), ſo bekommt man eine unendliche Sehnſucht nach {\pb}Einfachheit und Reinheit. Und wenn man außerdem noch
               zum poetiſchen Träumen \strikeout{aufgelegt}{ }\strikeout{iſt ſo} angelegt iſt, ſo liebt man die unfertigen
               Dinge. Die Poeſie beſteht darin, daß man den Dingen etwas hinzufügt. Das iſt der Reiz
               des halben Kindes für den Träumer, und darum bleibt \strikeout{\textcolor{gray}{×}\-\textcolor{gray}{×}} ihm die fertige Frau gleichgiltig. Nebenbei geſagt übrigens: Welche Frau iſt
               überhaupt fertig?\pend
           \pstart
           Bitte, liebſter Freund, ſchreib’ mir bald. In dieſer {\pb}Welt voll Feindſeligkeiten ſehne ich mich ſehr nach einem guten Worte von Dir.\pend
           \pstart
           Fragen, die beſonders zu beantworten wären: Was macht Deine \textcolor{blue}{Freundin}{}\ledrightnote{→\textcolor{blue}{Marie Reinhard}}? \strikeout{Was} Wie ſteht es mit Deinem \label{K_L02833-99v}\edtext{neuen \textcolor{green}{Stück}{}\ledrightnote{→\textcolor{green}{Das Vermächtnis. Schauspiel in drei Akten}}}{\lemma{\textnormal{\emph{neuen Stück}}}\Cendnote{\textnormal{\textcolor{blue}{Schnitzler} arbeitete intensiv an dem
                  Schauspiel \emph{\textcolor{green}{Das Vermächtnis}}, hatte dabei
                  jedoch einige Schwierigkeiten, die er immer wieder im \emph{\textcolor{green}{Tagebuch}} festhielt (vgl. z. B. 9. 12. 1897).}}}\label{K_L02833-99h}?
               Und was iſt mit dem \textcolor{green}{Stück}{}\ledrightnote{→\textcolor{green}{’s Katherl. Volksstück in fünf Aufzügen}} von
                  \textsc{\textcolor{blue}{Burckhardt}{}\ledrightnote{\textcolor{blue}{Max Eugen Burckhard}}}, welches der alberne \textsc{\textcolor{blue}{Bahr}{}\ledrightnote{\textcolor{blue}{Hermann Bahr}}} mit \textsc{\textcolor{blue}{Shakespeare}{}\ledrightnote{\textcolor{blue}{William Shakespeare}}}{ }\label{K_L02833-77v}\edtext{\textcolor{green}{vergleicht}{}\ledrightnote{→\textcolor{green}{’s Katherl. (Volksstück in fünf Aufzügen von Max Burckhard. Zum ersten Mal aufgeführt im Raimundtheater am 25. November 1897.)}}}{\lemma{\textnormal{\emph{vergleicht}}}\Cendnote{\textnormal{\textcolor{blue}{Hermann Bahr}: \emph{\textcolor{green}{’s Katherl. (Volksstück in fünf Aufzügen von Max Burckhard.
                        Zum ersten Mal aufgeführt im Raimundtheater am 25. November 1897.)}}. In:
                        \emph{\textcolor{green}{Die Zeit}}, Bd. 13, Nr. 165, 27. 11. 1897, S. 141.}}}\label{K_L02833-77h}?\pend
           \pstart
           Sei von Herzen gegrüßt.\pend
           \pstart
           Dein treuer {\\[\baselineskip]}\spacefill\mbox{Paul Goldmann.}\pend
           \leftskip=0em{}\pstart
           \noindent{}Bitte, grüße doch auch einmal \label{K_L02833-123v}\edtext{\textcolor{blue}{Frau \textsc{Altmann}}{}\ledrightnote{→\textcolor{blue}{Helene Schnitzler}} und deren {\pb}\textcolor{blue}{Sohne}{}\ledrightnote{→\textcolor{blue}{Hans Schnitzler}}}{\lemma{\textnormal{\emph{Frau … Sohne}}}\Cendnote{\textnormal{\textcolor{blue}{Helene Schnitzler}, geborene \textcolor{blue}{Altmann}, Ehefrau von
                        \textcolor{blue}{Schnitzler}s Bruder \textcolor{blue}{Julius}, Mutter von \textcolor{blue}{Hans Schnitzler}, \textcolor{blue}{Schnitzler}s
                     Neffen}}}\label{K_L02833-123h}, wenn Du ſie ſiehſt.\pend
           \endnumbering\briefempfaengerindex{Schnitzler, Arthur@\textsc{Schnitzler, Arthur}!zzzGoldmann, Paul@\emph{von Paul Goldmann}!1897-12-101@{10. 12. {[}1897{]}}|)be}\mylabel{h}  \normalsize

\doendnotes{C}
\bigskip
\vfill

\clearpage

\footnotesize

\lohead{\textsc{register}}

% Definiere theindex-Environment komplett neu ohne reledmac
\makeatletter
\renewenvironment{theindex}{%
  \section*{\indexname}%
  \setlength{\parindent}{0pt}%
  \setlength{\parskip}{0pt plus 0.3pt}%
  \let\item\@idxitem
}{%
  \clearpage
}
\makeatother

\IfFileExists{\jobname-pw.ind}{\input{\jobname-pw.ind}}{}

\end{document}

      