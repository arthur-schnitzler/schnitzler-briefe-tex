%% latex-korrekturansicht-vorspann.tex
%% Vorspann für die Korrekturansicht.
%% Lädt die gemeinsame Datei latex-vorspann.tex mit gesetztem Schalter.

\newif\ifkorrekturansicht
\korrekturansichttrue

\input{../tex-inputs/latex-vorspann}


         
         \renewcommand{\erwaehntePersonen}{Personen: Heinrich Heine, Alfred Kerr}
         \renewcommand{\erwaehnteOrte}{Orte: Frankfurt am Main, Wien}
         \renewcommand{\erwaehnteWerke}{Werke: Frankfurter Zeitung, Heine}
               \section[ Paul Goldmann an Arthur Schnitzler, 13. 12. {[}1899{]}]{Paul Goldmann an Arthur Schnitzler, 13. 12. {[}1899{]}}\nopagebreak\mylabel{v}\rehead{ }\normalsize\beginnumbering\briefempfaengerindex{Schnitzler, Arthur@\textsc{Schnitzler, Arthur}!zzzGoldmann, Paul@\emph{von Paul Goldmann}!1899-12-131@{13. 12. {[}1899{]}}|(be} \toendnotes[C]{\smallbreak\pagebreak[2]} \Standort{DLA, A:Schnitzler, HS.NZ85.1.3169.}
\physDesc{Brief, 1 Blatt, 2 Seiten
\newline{}Handschrift: blaue Tinte, deutsche Kurrent
\newline{}Schnitzler: 1) mit Bleistift das Jahr »99« vermerkt  2) mit rotem Buntstift eine Unterstreichung}\toendnotes[C]{\smallbreak}\pstart
           \raggedleft{}{\pb}\textcolor{pink}{Frankfurt}{}\ledrightnote{\textcolor{pink}{Frankfurt am Main}}, 13. Dezember.\pend
           \pstart{}Mein lieber Freund,\pend\pstart
           da Du wohl nicht die »\textcolor{green}{Frankfurter Zeitung}{}\ledrightnote{\textcolor{green}{Frankfurter Zeitung}}«
               lieſt, \label{K_L02899-1v}\edtext{ſende ich Dir anbei das geſtern erſchienene \textcolor{green}{Feuilleton}{}\ledrightnote{{$\rightarrow$}\textcolor{green}{Heine}} von \textsc{\textcolor{blue}{Kerr}{}\ledrightnote{\textcolor{blue}{Alfred Kerr}}} über \textsc{\textcolor{blue}{Heine}{}\ledrightnote{\textcolor{blue}{Heinrich Heine}}}}{\lemma{\textnormal{\emph{ſende … Heine}}}\Cendnote{\textnormal{\textcolor{blue}{Alfred Kerr}: \emph{\textcolor{green}{Heine}}. In: \emph{\textcolor{green}{Frankfurter
                        Zeitung}}, Jg. XXXX, Nr. 345, 13. 12. 1899, S. XXXX. \textcolor{blue}{Schnitzler} hatte den Brief spätestens am
                     15. 12. 1899 in den Händen, da schrieb er an \textcolor{blue}{Kerr}: »Lieber Herr \textcolor{blue}{Kerr}, ich muss Ihnen diesen Brief meines Freundes \textcolor{blue}{Goldmann} doch senden – Sie werden so
                     freundlich sein, ihm (\textcolor{blue}{G.}!) nie zu
                     verrathen, daß ich es gethan, und senden mir ihn (den Brief) auch bald wieder
                     zurück. Freuen wird es Sie jedenfalls – wie man überhaupt Ehrgeiz hat, – haben
                     soll? haben muss? – das beste bleibt doch zu wünschen, dass andere kluge
                     Menschen gut über uns denken. Der Ansicht \textcolor{blue}{G.}s über Ihr \textcolor{green}{Feuilleton} schließ ich mich vollkommen an – ohne sein Empfinden von
                     ›Zurückgeworfensein in die Mittelmäßigkeit‹ im geringsten berechtigt zu finden.
                     Denn auch er gehört zu den ganz vortrefflichen.« (In: \textcolor{blue}{Kerr}, \textcolor{blue}{Schnitzler}: \emph{»Es ist eine sehr seltsame
                        Gefühlsmischung, die Sie erwecken.« Briefwechsel 1896–1925}. Hg.
                     Elgin Helmstaedt. In: \emph{Sinn und Form}, Jg. 69,
                     H. 5, September/Oktober 2017,
               S. 598–599.)}}}\label{K_L02899-1h}. Ich halte dasſelbe für eines der vollendetſten
               Kunſtwerke, welche die neuere deutſche Journaliſtik hervorgebracht hat. Wenn man
               ſelbſt Zeitungsſchreiber von Beruf iſt, ſo fühlt man ſich tief verſtimmt durch \strikeout{eine} dieſe \strikeout{ſolche}{ }\textcolor{green}{Arbeit}{}\ledrightnote{{$\rightarrow$}\textcolor{green}{Heine}}, die eine ſolche Kunſt
               des Ausdrucks, eine ſolche Kraft der Concentrirung, einen ſo unbedingt perſönlichen
               Styl und ein ſo gründliches Wiſſen bekundet. Es ſteckt thatſächlich etwas Geniales \substVorne{}\textsuperscript{darin}\substDazwischen{}darin\substHinten{} – {\pb}etwas von \textsc{\textcolor{blue}{Heine}{}\ledrightnote{\textcolor{blue}{Heinrich Heine}}’s} Größe (ohne den leiſeſten
               Anklang an \textsc{\textcolor{blue}{Heine}{}\ledrightnote{\textcolor{blue}{Heinrich Heine}}’s} Art), – und, wenn man ſelbſt
               Zeitungsſchreiber von Beruf iſt (ſiehe oben), ſo fühlt man ſich erbarmungslos in die
               Mittelmäßigkeit zurückgeworfen.\pend
           \pstart
           Viele treue Grüße! {\\[\baselineskip]}Dein {\\[\baselineskip]}\spacefill\mbox{Paul Goldmann}\pend
           \leftskip=0em{}\endnumbering\briefempfaengerindex{Schnitzler, Arthur@\textsc{Schnitzler, Arthur}!zzzGoldmann, Paul@\emph{von Paul Goldmann}!1899-12-131@{13. 12. {[}1899{]}}|)be}\mylabel{h}  \normalsize

\doendnotes{C}
\bigskip
\vfill

\clearpage

\footnotesize

\lohead{\textsc{register}}

% Definiere theindex-Environment komplett neu ohne reledmac
\makeatletter
\renewenvironment{theindex}{%
  \section*{\indexname}%
  \setlength{\parindent}{0pt}%
  \setlength{\parskip}{0pt plus 0.3pt}%
  \let\item\@idxitem
}{%
  \clearpage
}
\makeatother

\IfFileExists{\jobname-pw.ind}{\input{\jobname-pw.ind}}{}

\end{document}

      