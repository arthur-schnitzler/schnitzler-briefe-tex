%% latex-korrekturansicht-vorspann.tex
%% Vorspann für die Korrekturansicht.
%% Lädt die gemeinsame Datei latex-vorspann.tex mit gesetztem Schalter.

\newif\ifkorrekturansicht
\korrekturansichttrue

\input{../tex-inputs/latex-vorspann}


\renewcommand{\erwaehntePersonen}{Personen: Julius Bauer, Gotthold Ephraim Lessing, Felix Salten}
\renewcommand{\erwaehnteInstitutionen}{Institutionen: Concordia}
\renewcommand{\erwaehnteOrte}{Orte: Wien}
\renewcommand{\erwaehnteWerke}{Werke: Lessing Almanach, [Um einer Partei anzugehören]}
\section[ Felix Salten an Arthur Schnitzler, {[}26. 1. 1912{]}]{Felix Salten an Arthur Schnitzler, {[}26. 1. 1912{]}}
\nopagebreak\mylabel{v}
\rehead{ }\normalsize\beginnumbering\briefempfaengerindex{Schnitzler, Arthur@\textsc{Schnitzler, Arthur}!zzzSalten, Felix@\emph{von Felix Salten}!1912-01-261@{{[}26. 1. 1912{]}}|(be}
\toendnotes[C]{\smallbreak\pagebreak[2]}\Standort{CUL, Schnitzler, B 89, B 2.}
\physDesc{Briefkarte, 574 Zeichen
\newline{}Handschrift: Bleistift, lateinische Kurrent
\newline{}Schnitzler: mit Bleistift datiert: »26/1 912« 
\newline{}Ordnung: mit Bleistift von unbekannter Hand nummeriert: »270« }\toendnotes[C]{\smallbreak}
\pstart
           \noindent{}\centering{}{\pb}\textcolor{gray}{\textbf{\textsc{Felix Salten}}}\pend
           
\pstart
           \raggedleft{}Freitag.\pend
           
\pstart{}Lieber,\pend
\pstart
           \label{K_L03555-1v}\edtext{\textcolor{blue}{Bauer}{}\ledrightnote{\textcolor{blue}{Julius Bauer}}}{\lemma{\textnormal{\emph{Bauer}}}\Cendnote{\textnormal{\textcolor{blue}{Julius Bauer} war der Herausgeber des \emph{\textcolor{green}{Lessing-Almanach}}s, eine Ballspende des \emph{\textcolor{brown}{Concordia}}balls, für den \textcolor{blue}{Schnitzler} einen \textcolor{green}{Aphorismus} beisteuerte.}}}\label{K_L03555-1h} wendet sich wieder einmal
               an mich. (weil Sie kein Telefon haben) Er bittet mich, Sie aufmerksam zu machen, dass
               Ihr \textcolor{green}{Beitrag}{}\ledrightnote{{$\rightarrow$}\textcolor{green}{[Um einer Partei anzugehören]}} (für den er Ihnen
               bestes dankt) \introOben{}als\introOben{} der einzige, nicht auf \textcolor{blue}{Lessing}{}\ledrightnote{\textcolor{blue}{Gotthold Ephraim Lessing}} zu beziehende da stehen würde in jener fabelhaften
                  \textcolor{green}{Ballspende}{}\ledrightnote{{$\rightarrow$}\textcolor{green}{Lessing Almanach}}, welche durchaus
                  \textcolor{blue}{Lessing}{}\ledrightnote{\textcolor{blue}{Gotthold Ephraim Lessing}} gewidmet ist. Er läßt Sie bitten,
               ihm heute oder morgen –
               weil es schon sehr eilt – irgend etwas \textcolor{blue}{Lessing}{}\ledrightnote{\textcolor{blue}{Gotthold Ephraim Lessing}}-sagendes zu spenden. Und er wird dann, um Ihre Antwort zu hören, bei
               mir anrufen. (Weil Sie kein Telefon u. s. w.)\pend
           
\pstart
           Auf baldiges Wiedersehen u. herzlichste Grüße von Haus zu Haus {\\[\baselineskip]}Ihr {\\[\baselineskip]}\spacefill\mbox{Salten}\pend
           \leftskip=0em{}\endnumbering\briefempfaengerindex{Schnitzler, Arthur@\textsc{Schnitzler, Arthur}!zzzSalten, Felix@\emph{von Felix Salten}!1912-01-261@{{[}26. 1. 1912{]}}|)be}\mylabel{h}  \normalsize

\doendnotes{C}
\bigskip
\vfill

\clearpage

\footnotesize

\lohead{\textsc{register}}

% Definiere theindex-Environment komplett neu ohne reledmac
\makeatletter
\renewenvironment{theindex}{%
  \section*{\indexname}%
  \setlength{\parindent}{0pt}%
  \setlength{\parskip}{0pt plus 0.3pt}%
  \let\item\@idxitem
}{%
  \clearpage
}
\makeatother

\IfFileExists{\jobname-pw.ind}{\input{\jobname-pw.ind}}{}

\end{document}

      