%% latex-korrekturansicht-vorspann.tex
%% Vorspann für die Korrekturansicht.
%% Lädt die gemeinsame Datei latex-vorspann.tex mit gesetztem Schalter.

\newif\ifkorrekturansicht
\korrekturansichttrue

\input{../tex-inputs/latex-vorspann}


               \section[Paul Goldmann an Arthur Schnitzler, 6. 12. {[}1893{]}]{ Paul Goldmann an Arthur Schnitzler, 6. 12. {[}1893{]}}\nopagebreak\mylabel{v}\rehead{ }\normalsize\beginnumbering\briefempfaengerindex{Schnitzler, Arthur@\textsc{Schnitzler, Arthur}!zzzGoldmann, Paul@\emph{von Paul Goldmann}!1893-12-061@{6. 12. {[}1893{]}}|(be} \toendnotes[C]{\smallbreak\pagebreak[2]} \Standort{DLA, A:Schnitzler, HS.NZ85.1.3163.}
\physDesc{Brief, 1 Blatt, 2 Seiten
\newline{}Handschrift: schwarze Tinte, deutsche Kurrent
\newline{}Schnitzler: mit Bleistift das Jahr »93« vermerkt }\toendnotes[C]{\smallbreak}\pstart
           \noindent{}{\pb}\textcolor{gray}{\textbf{\textbf{\textcolor{brown}{Frankfurter Zeitung}{}\ledrightnote{\textcolor{brown}{Frankfurter Zeitung}}.}}}\pend
           \pstart
           \textcolor{gray}{\textbf{\textbf{(\textcolor{brown}{\begin{otherlanguage}{french}Gazette de Francfort\end{otherlanguage}}{}\ledrightnote{\textcolor{brown}{Frankfurter Zeitung}}.)}}}\pend
           \pstart
           \textcolor{gray}{\textbf{\begin{otherlanguage}{french}\textcolor{blue}{Directeur}{}\ledrightnote{→\textcolor{blue}{Leopold Sonnemann}}\end{otherlanguage}{ }\textbf{M. \textcolor{blue}{L. Sonnemann}{}\ledrightnote{\textcolor{blue}{Leopold Sonnemann}}.}}}\hfill \textsc{\textcolor{pink}{Paris}{}\ledrightnote{\textcolor{pink}{Paris}}}, 6. December\textcolor{gray}{.}\pend
           \pstart
           \begin{otherlanguage}{french}\textcolor{gray}{\textbf{\textcolor{green}{Journal}{}\ledrightnote{\textcolor{green}{Frankfurter Zeitung}} politique, financier,}}\end{otherlanguage}\pend
           \pstart
           \begin{otherlanguage}{french}\textcolor{gray}{\textbf{commercial et litteraire.}}\end{otherlanguage}\pend
           \pstart
           \begin{otherlanguage}{french}\textcolor{gray}{\textbf{\textbf{Paraissant trois fois par jour}}}\end{otherlanguage}\pend
           \pstart
           \begin{otherlanguage}{french}\textcolor{gray}{\textbf{\textbf{Bureaux à \textcolor{pink}{Paris}{}\ledrightnote{\textcolor{pink}{Paris}}:}}}\end{otherlanguage}\pend
           \pstart
           \begin{otherlanguage}{french}\textcolor{gray}{\textbf{\textbf{\textcolor{pink}{rue Richelieu 75}{}\ledrightnote{\textcolor{pink}{rue Richelieu}}.}}}\end{otherlanguage}\pend
           \pstart\center{}Mein lieber Freund!\pend\pstart
           Beilegend eine Zuſchrift \textsc{\textcolor{blue}{Uhl}{}\ledrightnote{\textcolor{blue}{Friedrich Uhl}}s}, die ich heut erhielt. Bitte, ſende ſie mir ſofort zurück.\pend
           \pstart
           Und ſchreib’ mir doch endlich einmal zwei Worte.\pend
           \pstart
           Iſt es wahr, daß das \textcolor{brown}{Volkstheater}{}\ledrightnote{\textcolor{brown}{Volkstheater}} Dich gleich
               nach der zweiten Vorſtellung \label{K_L02722-1v}\edtext{abgeſetzt}{\lemma{\textnormal{\emph{abgeſetzt}}}\Cendnote{\textnormal{Bereits bei der zweiten
                  und letzten Vorstellung des \emph{\textcolor{green}{Märchen}}s am 2. 12. 1893 war kaum
                  Publikum vor Ort. Die Absetzung stand zu diesem Zeitpunkt aufgrund der 
                  Schwäche des dritten \textcolor{green}{Akt}s bereits fest. Das Theater hatte zu verstehen gegeben, dass das Stück
                     in einer abgeänderten Fassung wiederaufgenommen würde. \textcolor{blue}{Schnitzler} unternahm
                     es, den Akt umzuschreiben, zu einer Wiederaufnahme kam es trotzdem nicht.}}}\label{K_L02722-1h}? Das ſieht der \label{K_L02722-2v}\edtext{feigen und gemeinen Bande}{\lemma{\textnormal{\emph{feigen … Bande}}}\Cendnote{\textnormal{\textcolor{blue}{Goldmann} hielt
                     wenig von der künstlerischen Zugangsweise des Theaters, 
                  vgl. Paul Goldmann an Arthur Schnitzler, 18. 8. [1893]}}}\label{K_L02722-2h} ganz ähnlich. Wahrſcheinlich haben die Frauen der Actionäre {\pb}proteſtirt. Die Verherrlichung einer Gefallenen!
                  \label{K_L02722-4v}\edtext{\textsc{\begin{otherlanguage}{french}Pensez donc\end{otherlanguage}}!}{\lemma{\textnormal{\emph{Pensez donc!}}}\Cendnote{\textnormal{französisch: man stelle sich vor!}}}\label{K_L02722-4h}\pend
           \pstart
           Weiter ſchreiben, liebſter Freund, weiter ſchreiben!\pend
           \pstart
           Dein {\\[\baselineskip]}treuer {\\[\baselineskip]}\spacefill\mbox{Paul Goldmann}\pend
           \leftskip=0em{}\endnumbering\briefempfaengerindex{Schnitzler, Arthur@\textsc{Schnitzler, Arthur}!zzzGoldmann, Paul@\emph{von Paul Goldmann}!1893-12-061@{6. 12. {[}1893{]}}|)be}\mylabel{h}  \normalsize

\doendnotes{C}
\bigskip
\vfill

\clearpage

\footnotesize

\lohead{\textsc{register}}

% Definiere theindex-Environment komplett neu ohne reledmac
\makeatletter
\renewenvironment{theindex}{%
  \section*{\indexname}%
  \setlength{\parindent}{0pt}%
  \setlength{\parskip}{0pt plus 0.3pt}%
  \let\item\@idxitem
}{%
  \clearpage
}
\makeatother

\IfFileExists{\jobname-pw.ind}{\input{\jobname-pw.ind}}{}

\end{document}

      