%% latex-korrekturansicht-vorspann.tex
%% Vorspann für die Korrekturansicht.
%% Lädt die gemeinsame Datei latex-vorspann.tex mit gesetztem Schalter.

\newif\ifkorrekturansicht
\korrekturansichttrue

\input{../tex-inputs/latex-vorspann}


               \section[Richard Beer-Hofmann an Arthur Schnitzler, {[}3.? 5. 1895{]}]{ Richard Beer-Hofmann an Arthur Schnitzler,
               {[}3.? 5. 1895{]}}\nopagebreak\mylabel{v}\rehead{ }\normalsize\beginnumbering\briefempfaengerindex{Schnitzler, Arthur@\textsc{Schnitzler, Arthur}!zzzBeer-Hofmann, Richard@\emph{von Richard Beer-Hofmann}!1895-05-031@{{[}3.? 5. 1895{]}}|(be} \toendnotes[C]{\smallbreak\pagebreak[2]} \Standort{CUL, Schnitzler, B 8.}
\physDesc{Briefkarte
\newline{}Handschrift: blauer Buntstift, lateinische Kurrent
\newline{}Schnitzler: mit Bleistift datiert: »Mai 95« und nummeriert: »59« }\toendnotes[C]{\smallbreak}\pstart
           \noindent{}{\pb}Lieber Arthur, da
               es regnet bin ich jedenfalls schon vor 7 Uhr zu Hause; ich bleibe
               zu Hause bis 8. Dann gehe ich \uline{voraussichtlich} (nicht sicher) ins Caffée.
               Möglicherweise ist \label{K_L00437_1v}\edtext{\textcolor{blue}{Hugo}{}\ledrightnote{\textcolor{blue}{Hugo von Hofmannsthal}}}{\lemma{\textnormal{\emph{Hugo}}}\Cendnote{\textnormal{Das spricht für den 3. 5. 1895, da an diesem Tag
                     \textcolor{blue}{Schnitzler} und \textcolor{blue}{Lou
                     Andreas-Salomé} zu \textcolor{blue}{Beer-Hofmann}
                  gehen und dann gemeinsam mit \textcolor{blue}{Hofmannsthal} in
                  ein Lokal.}}}\label{K_L00437_1h} um 7 bei mir Herzlichst\pend
           \pstart \spacefill\mbox{Richard}\pend{}\endnumbering\briefempfaengerindex{Schnitzler, Arthur@\textsc{Schnitzler, Arthur}!zzzBeer-Hofmann, Richard@\emph{von Richard Beer-Hofmann}!1895-05-031@{{[}3.? 5. 1895{]}}|)be}\mylabel{h}  \normalsize

\doendnotes{C}
\bigskip
\vfill

\clearpage

\footnotesize

\lohead{\textsc{register}}

% Definiere theindex-Environment komplett neu ohne reledmac
\makeatletter
\renewenvironment{theindex}{%
  \section*{\indexname}%
  \setlength{\parindent}{0pt}%
  \setlength{\parskip}{0pt plus 0.3pt}%
  \let\item\@idxitem
}{%
  \clearpage
}
\makeatother

\IfFileExists{\jobname-pw.ind}{\input{\jobname-pw.ind}}{}

\end{document}

      