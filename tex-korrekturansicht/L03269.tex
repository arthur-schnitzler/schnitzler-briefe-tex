%% latex-korrekturansicht-vorspann.tex
%% Vorspann für die Korrekturansicht.
%% Lädt die gemeinsame Datei latex-vorspann.tex mit gesetztem Schalter.

\newif\ifkorrekturansicht
\korrekturansichttrue

\input{../tex-inputs/latex-vorspann}


\renewcommand{\erwaehntePersonen}{Personen: Georg Hirschfeld}
\renewcommand{\erwaehnteInstitutionen}{Institutionen: Burgtheater, Deutsches Theater Berlin}
\renewcommand{\erwaehnteOrte}{Orte: Bad Ischl, Hotel und Pension Rudolfshöhe (Leopold Petter), I., Innere Stadt, Kaltenbach, Wien}
\renewcommand{\erwaehnteWerke}{Werke: Agnes Jordan. Schauspiel in fünf Akten, Der Hinterbliebene. Kurze Novellen, Neues Wiener Journal, Theater und Kunst [Agnes Jordan angenommen]}
\section[ Felix Salten an Arthur Schnitzler, 17. 7. 1897]{Felix Salten an Arthur Schnitzler, 17. 7. 1897}
\nopagebreak\mylabel{v}
\rehead{ }\normalsize\beginnumbering\briefempfaengerindex{Schnitzler, Arthur@\textsc{Schnitzler, Arthur}!zzzSalten, Felix@\emph{von Felix Salten}!1897-07-171@{17. 7. 1897}|(be}
\toendnotes[C]{\smallbreak\pagebreak[2]}\Standort{CUL, Schnitzler, B 89, A 2.}
\physDesc{Postkarte, 279 Zeichen
\newline{}Handschrift: Bleistift, lateinische Kurrent
\newline{}Versand: Stempel: »\nobreak{}\oindex{I., Innere Stadt@\textbf{I., Innere Stadt}, \emph{A.ADM3}|pwk}Wien 1/1 1, 17. 7. 97, 11–12 N\nobreak{}«. Stempel: »\nobreak{}\oindex{Bad Ischl@\textbf{Bad Ischl}, \emph{P.PPL}|pwk}Ischl, 18\textcolor{gray}{.} 7. 97\nobreak{}«.  
\newline{}Schnitzler: mit Bleistift datiert: »17. 7\textcolor{gray}{. 97}« 
\newline{}Ordnung: mit Bleistift von unbekannter Hand nummeriert: »92« }\toendnotes[C]{\smallbreak}\pstart{}{\pb}Herrn D\textsuperscript{r} Arthur Schnitzler\pend{}\pstart{}\textcolor{pink}{Ischl}{}\ledrightnote{\textcolor{pink}{Bad Ischl}}\pend{}\pstart{}\textcolor{pink}{Kaltenbach}{}\ledrightnote{\textcolor{pink}{Kaltenbach}}, \textcolor{pink}{Pension Rudolfshöhe}{}\ledrightnote{\textcolor{pink}{Hotel und Pension Rudolfshöhe (Leopold Petter)}}.\pend{}
{\bigskip}
\pstart
           \noindent{}{\pb}Lieber Freund, viel Dank für Ihren Brief. Die \label{K_L03269-1v}\edtext{Sache \textcolor{blue}{G. H.}{}\ledrightnote{\textcolor{blue}{Georg Hirschfeld}}}{\lemma{\textnormal{\emph{Sache G. H.}}}\Cendnote{\textnormal{Wenige Tage zuvor wurde die Annahme von
                     \textcolor{blue}{Georg Hirschfeld}s neuem Stück, , am \emph{\textcolor{brown}{Deutschen Theater Berlin}} gemeldet (vgl. [O. V.]: \emph{\textcolor{green}{Theater und Kunst}}. In: \emph{\textcolor{green}{Neues Wiener Journal}}, Nr. 1337, 14. 7. 1897, S. 6). Das \emph{\textcolor{brown}{Burgtheater}} zog in diesen Tagen die Annahme des Stücks zurück, was \textcolor{blue}{Schnitzler} in
                  einem Brief von \textcolor{blue}{Hirschfeld} vom 12. 7. 1897 erfuhr: »Denken Sie, bald nach
                     Ihrem Brief bekam ich endlich \textcolor{blue}{Burckhard}s Brief, in dem er 
                     mir auseinanderſetzte, mit allem Lob, aller Achtung, daß er das \textcolor{green}{Stück}{ }\uline{nicht}
                  nehmen könnte. Zenſurbedenken, und wenn dieſe fortfielen, ›sociale‹ Bedenken, ein
                  Teil des Publikums würde oſtentativ Bravo klatſchen, der andere dadurch – beleidigt ſein.« (\emph{CUL}, B42)
                  Im Hintergrund der Entscheidung dürfte aus Sicht \textcolor{blue}{Salten}s und \textcolor{blue}{Schnitzler}s \textcolor{blue}{Hermann
                     Bahr} gestanden sein, der in engem Austausch mit dem Direktor \textcolor{blue}{Max Burckhard} stand. (Vgl. Felix Salten an Arthur Schnitzler, 22. 7. 1897, Felix Salten an Arthur Schnitzler, 23. 7. 1897)}}}\label{K_L03269-1h} wusste ich schon, da \textcolor{blue}{H.}{}\ledrightnote{\textcolor{blue}{Georg Hirschfeld}} mir schrieb. Auch ich habe die bewussten
               Einflüße sofort erkannt, und mich sehr geärgert. Mein \label{K_L03269-2v}\edtext{\textcolor{green}{Buch}{}\ledrightnote{{$\rightarrow$}\textcolor{green}{Der Hinterbliebene. Kurze Novellen}}}{\lemma{\textnormal{\emph{Buch}}}\Cendnote{\textnormal{der Novellenband \emph{\textcolor{green}{Der Hinterbliebene}}? (vgl. Felix Salten an Arthur Schnitzler, [30. 10. 1896])}}}\label{K_L03269-2h} ist noch nicht fertig. Auf
               Wiedersehen\pend
           \pstart \spacefill\mbox{Salten}\pend{}\endnumbering\briefempfaengerindex{Schnitzler, Arthur@\textsc{Schnitzler, Arthur}!zzzSalten, Felix@\emph{von Felix Salten}!1897-07-171@{17. 7. 1897}|)be}\mylabel{h}  \normalsize

\doendnotes{C}
\bigskip
\vfill

\clearpage

\footnotesize

\lohead{\textsc{register}}

% Definiere theindex-Environment komplett neu ohne reledmac
\makeatletter
\renewenvironment{theindex}{%
  \section*{\indexname}%
  \setlength{\parindent}{0pt}%
  \setlength{\parskip}{0pt plus 0.3pt}%
  \let\item\@idxitem
}{%
  \clearpage
}
\makeatother

\IfFileExists{\jobname-pw.ind}{\input{\jobname-pw.ind}}{}

\end{document}

      