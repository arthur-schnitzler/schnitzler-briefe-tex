%% latex-korrekturansicht-vorspann.tex
%% Vorspann für die Korrekturansicht.
%% Lädt die gemeinsame Datei latex-vorspann.tex mit gesetztem Schalter.

\newif\ifkorrekturansicht
\korrekturansichttrue

\input{../tex-inputs/latex-vorspann}


\renewcommand{\erwaehnteOrte}{Orte: Bozen, Frankgasse, Franzensfeste, Gasthof zum Lamm, Klausen (Südtirol), Wien}
\renewcommand{\erwaehnteWerke}{}
\section[ Paul Goldmann an Arthur Schnitzler, 11. 8. 1903]{Paul Goldmann an Arthur Schnitzler, 11. 8. 1903}
\nopagebreak\mylabel{v}
\rehead{ }\normalsize\beginnumbering\briefempfaengerindex{Schnitzler, Arthur@\textsc{Schnitzler, Arthur}!zzzGoldmann, Paul@\emph{von Paul Goldmann}!1903-08-112@{11. 8. 1903}|(be}
\toendnotes[C]{\smallbreak\pagebreak[2]}\Standort{DLA, A:Schnitzler, HS.NZ85.1.3173.}
\physDesc{Postkarte
\newline{}Handschrift: 1) blaue Tinte, deutsche Kurrent\hspace{1em}2) blaue Tinte, lateinische Kurrent (\noindent{}Adresse)\hspace{1em}
\newline{}Versand: Stempel: »\nobreak{}Wien 1/1 15, 11 VIII 03, 6 30N\nobreak{}«. Stempel: »\nobreak{}Wien 9/2 71 r, 11 VIII 03, 7 20N\nobreak{}«.  
\newline{}Schnitzler: mit Bleistift datiert: »11. 8\textcolor{gray}{.}{ }{[}1{]}903.« }\toendnotes[C]{\smallbreak}\pstart{}{\pb}Herrn\pend{}\pstart{}Dr. Arthur Schnitzler\pend{}\pstart{}\textcolor{pink}{IX. Frankgaſse 1}{}\ledrightnote{\textcolor{pink}{Frankgasse}}{ }\textcolor{pink}{Wien}{}\ledrightnote{\textcolor{pink}{Wien}}\pend{}
{\bigskip}
\pstart
           {\pb}Dienſtag{ }Abend.\pend
           
\pstart{}Mein lieber Freund,\pend
\pstart
           Ich \label{K_L03383-1v}\edtext{fahre}{\lemma{\textnormal{\emph{fahre}}}\Cendnote{\textnormal{siehe Paul Goldmann an Arthur Schnitzler, 27. 6. [1903]}}}\label{K_L03383-1h}{ }heut{ }Abend, habe morgen{ }Nachmittag in \textcolor{pink}{Franzensfeſte}{}\ledrightnote{\textcolor{pink}{Franzensfeste}}
               Rendezvous, dürfte morgenAbend in \textsc{\textcolor{pink}{Klausen}{}\ledrightnote{\textcolor{pink}{Klausen (Südtirol)}}} (\textsc{\textcolor{pink}{Hotel Lamm}{}\ledrightnote{\textcolor{pink}{Gasthof zum Lamm}}}) ſein. Donnerſtag ſende ich Dir Nachricht nach
                  \textsc{\textcolor{pink}{Bozen}{}\ledrightnote{\textcolor{pink}{Bozen}}}. Du kannft mir bis Donnerſtag{ }früh Nachricht nach \textsc{\textcolor{pink}{Klausen}{}\ledrightnote{\textcolor{pink}{Klausen (Südtirol)}}} ſenden, wenn Du willſt.\pend
           
\pstart
           Leb’ wohl! Glückliche Reiſe! Tauſend Dank! Grüße! {\\[\baselineskip]}Dein \spacefill\mbox{Paul
                  Goldm}\pend
           \leftskip=0em{}\endnumbering\briefempfaengerindex{Schnitzler, Arthur@\textsc{Schnitzler, Arthur}!zzzGoldmann, Paul@\emph{von Paul Goldmann}!1903-08-112@{11. 8. 1903}|)be}\mylabel{h}
\begin{anhang}
\end{anhang}\normalsize

\doendnotes{C}
\bigskip
\vfill

\clearpage

\footnotesize

\lohead{\textsc{register}}

% Definiere theindex-Environment komplett neu ohne reledmac
\makeatletter
\renewenvironment{theindex}{%
  \section*{\indexname}%
  \setlength{\parindent}{0pt}%
  \setlength{\parskip}{0pt plus 0.3pt}%
  \let\item\@idxitem
}{%
  \clearpage
}
\makeatother

\IfFileExists{\jobname-pw.ind}{\input{\jobname-pw.ind}}{}

\end{document}

      