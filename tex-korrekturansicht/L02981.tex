%% latex-korrekturansicht-vorspann.tex
%% Vorspann für die Korrekturansicht.
%% Lädt die gemeinsame Datei latex-vorspann.tex mit gesetztem Schalter.

\newif\ifkorrekturansicht
\korrekturansichttrue

\input{../tex-inputs/latex-vorspann}


\renewcommand{\erwaehntePersonen}{Personen: Otto Brahm, Mirjam Horwitz, Adolf Landesmann, Felix Salten}
\renewcommand{\erwaehnteOrte}{Orte: Berlin, Palasthotel Berlin, Wien}
\renewcommand{\erwaehnteWerke}{Werke: Die Gespräche des göttlichen Pietro Aretino, Vom göttlichen Aretino}
\section[ Arthur Schnitzler an Felix Salten, 4. 3. 1903]{Arthur Schnitzler an Felix Salten, 4. 3. 1903}
\nopagebreak\mylabel{v}
\rehead{ }\normalsize\beginnumbering\briefempfaengerindex{Salten, Felix@\textsc{Salten, Felix}!zzzSchnitzler, Arthur@\emph{von Arthur Schnitzler}!1903-03-042@{4. 3. 1903}|(be}
\toendnotes[C]{\smallbreak\pagebreak[2]}\Standort{Wienbibliothek im Rathaus, ZPH 1681, 2.1.516.}
\physDesc{Brief, 1 Blatt, 3 Seiten, 834 Zeichen
\newline{}Handschrift: Bleistift, deutsche Kurrent
\newline{}Ordnung: mit Bleistift von unbekannter Hand Nummerierung der Doppelseiten des Konvoluts:
                                    »57«–»58« }\toendnotes[C]{\smallbreak}
\pstart
           \raggedleft{}{\pb}4. 3. 903\pend
           
\pstart
           \raggedleft{}Abds{ }\textsc{\textcolor{pink}{Berlin}{}\ledrightnote{\textcolor{pink}{Berlin}}}\pend
           
\pstart
           lieber Freund, meinem \label{K_L02981-1v}\edtext{Brief von heute{ }Nachmittg}{\lemma{\textnormal{\emph{Brief … Nachmittg}}}\Cendnote{\textnormal{Arthur Schnitzler an Felix Salten, 4. 3. 1903}}}\label{K_L02981-1h} iſt nachzutragen: als ich das \textcolor{pink}{Hotel}{}\ledrightnote{{$\rightarrow$}\textcolor{pink}{Palasthotel Berlin}} verlieſs, erwartete mich \textcolor{blue}{M. H.}{}\ledrightnote{\textcolor{blue}{Mirjam Horwitz}},
               ſie zeigte mir den Brief, den Sie an den \label{K_L02981-2v}\edtext{\textcolor{blue}{Vertrauten}{}\ledrightnote{{$\rightarrow$}\textcolor{blue}{Adolf Landesmann}} geſchrieben; ich
               hatte ihn (kleine Welt!) geſtern{ }Abend bei \textcolor{blue}{Brahm}{}\ledrightnote{\textcolor{blue}{Otto Brahm}} kennen
                  gelernt}{\lemma{\textnormal{\emph{Vertrauten … gelernt}}}\Cendnote{\textnormal{Die Identifizierung gelingt
                  durch Ausschluss: Von der Abendgesellschaft am 3. 3. 1903 war einzig \textcolor{blue}{Adolf Landesmann}{ }\textcolor{blue}{Schnitzler} zuvor nicht bekannt
                  gewesen.}}}\label{K_L02981-2h}{\dotstwo} ich entledigte mich meines Auftrags ganz geſchickt; ſie
                  {\pb}möchte ihre Briefe zurück haben – ich
               rieth ihr, dem keinerlei Werth beizulegen; theile Ihnen aber, \substVorne{}\textsuperscript{ihrer}\substDazwischen{}\textcolor{blue}{M}{}\ledrightnote{\textcolor{blue}{Mirjam Horwitz}}.s
               \substHinten{} Bitte entſprechend, d\substVorne{}\textsuperscript{\textcolor{gray}{en}}\substDazwischen{}ie\substHinten{}ſen Wunſch mit. Thränen, etwas Kliſche; mehr Zorn als Kränkung wie mir
               ſcheint. Im ganzen kein Anlaſs ſich aufzuregen.\pend
           
\pstart
           – Ich habe hier auch die \label{K_L02981-3v}\edtext{\textcolor{green}{Geſpräche des göttlichen {\pb}\textsc{Aretin}}{}\ledrightnote{\textcolor{green}{Die Gespräche des göttlichen Pietro Aretino}}}{\lemma{\textnormal{\emph{Geſpräche … Aretin}}}\Cendnote{\textnormal{siehe Felix Salten an Arthur Schnitzler, 3. 3. 1903}}}\label{K_L02981-3h} geleſen; nicht ganz ohne Enttäuſchg. Ich hoffe Ihre
               rö\textcolor{gray}{mi}ſche Buhlerin wird intereſſantere Dinge zu \textcolor{green}{erzählen}{}\ledrightnote{{$\rightarrow$}\textcolor{green}{Vom göttlichen Aretino}} wiſſen. Amuſirt hat mich am
               meiſten die kleine \textcolor{green}{Pippa}{}\ledrightnote{{$\rightarrow$}\textcolor{green}{Die Gespräche des göttlichen Pietro Aretino}} mit
               ihrem dummen Hineinreden.\pend
           
\pstart
           Leben Sie wohl. Herzlichſt Ihr {\\[\baselineskip]}\spacefill\mbox{A.}\pend
           \leftskip=0em{}\endnumbering\briefempfaengerindex{Salten, Felix@\textsc{Salten, Felix}!zzzSchnitzler, Arthur@\emph{von Arthur Schnitzler}!1903-03-042@{4. 3. 1903}|)be}\mylabel{h}  \normalsize

\doendnotes{C}
\bigskip
\vfill

\clearpage

\footnotesize

\lohead{\textsc{register}}

% Definiere theindex-Environment komplett neu ohne reledmac
\makeatletter
\renewenvironment{theindex}{%
  \section*{\indexname}%
  \setlength{\parindent}{0pt}%
  \setlength{\parskip}{0pt plus 0.3pt}%
  \let\item\@idxitem
}{%
  \clearpage
}
\makeatother

\IfFileExists{\jobname-pw.ind}{\input{\jobname-pw.ind}}{}

\end{document}

      