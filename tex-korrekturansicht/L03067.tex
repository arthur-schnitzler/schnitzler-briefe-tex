%% latex-korrekturansicht-vorspann.tex
%% Vorspann für die Korrekturansicht.
%% Lädt die gemeinsame Datei latex-vorspann.tex mit gesetztem Schalter.

\newif\ifkorrekturansicht
\korrekturansichttrue

\input{../tex-inputs/latex-vorspann}


\renewcommand{\erwaehntePersonen}{Personen: Felix Salten, Olga Schnitzler, Irene Triesch}
\renewcommand{\erwaehnteInstitutionen}{Institutionen: Akademischer Verein für Kunst und Literatur, Jung-Wiener Theater zum Lieben Augustin}
\renewcommand{\erwaehnteOrte}{Orte: Berlin, Dessauer Straße, Wien}
\renewcommand{\erwaehnteWerke}{Werke: Der Schleier der Beatrice. Schauspiel in fünf Akten}
\section[ Paul Goldmann an Arthur Schnitzler, 16. 5. {[}1901{]}]{Paul Goldmann an Arthur Schnitzler, 16. 5. {[}1901{]}}
\nopagebreak\mylabel{v}
\rehead{ }\normalsize\beginnumbering\briefempfaengerindex{Schnitzler, Arthur@\textsc{Schnitzler, Arthur}!zzzGoldmann, Paul@\emph{von Paul Goldmann}!1901-05-161@{16. 5. {[}1901{]}}|(be}
\toendnotes[C]{\smallbreak\pagebreak[2]}\Standort{DLA, A:Schnitzler, HS.NZ85.1.3171.}
\physDesc{Brief, 1 Blatt, 3 Seiten
\newline{}Handschrift: blaue Tinte, deutsche Kurrent
\newline{}Schnitzler: 1) mit Bleistift das Jahr »1901.« vermerkt  2) mit rotem Buntstift zwei Unterstreichungen}\toendnotes[C]{\smallbreak}
\pstart
           \noindent{}\raggedleft{}{\pb}\textcolor{pink}{\textcolor{gray}{\textbf{DESSAUERSTRASSE 19}}}{}\ledrightnote{\textcolor{pink}{Dessauer Straße}}\pend
           
\pstart
           \textcolor{pink}{Berlin}{}\ledrightnote{\textcolor{pink}{Berlin}}, 16. Mai.\pend
           
\pstart\center{}Mein lieber Freund,\pend
\pstart
           Ich freue mich ſehr, daß es Fräulein \textsc{\textcolor{blue}{Olga}{}\ledrightnote{\textcolor{blue}{Olga Schnitzler}}} gut geht, und bitte, ſie recht herzlich von mir zu grüßen.\pend
           
\pstart
           Dem \textcolor{brown}{akad. literariſchen Verein}{}\ledrightnote{\textcolor{brown}{Akademischer Verein für Kunst und Literatur}} kannſt Du \label{K_L03067-1v}\edtext{die »\textsc{\textcolor{green}{Beatrice}{}\ledrightnote{\textcolor{green}{Der Schleier der Beatrice. Schauspiel in fünf Akten}}}« ruhig geben}{\lemma{\textnormal{\emph{die … geben}}}\Cendnote{\textnormal{Es ist keine
                  Aufführung von \emph{\textcolor{green}{Der Schleier der Beatrice}}
                  durch den \emph{\textcolor{brown}{Akademischen Verein für Kunst und
                     Literatur}} bekannt. Zu \textcolor{blue}{Irene
                  Triesch}s erstem Auftritt als \textcolor{green}{Beatrice}{ }siehe Paul Goldmann an Arthur Schnitzler, 20. 2. 1900. \textcolor{blue}{Olga Gussmann} trat nie als \textcolor{green}{Beatrice} auf.}}}\label{K_L03067-1h}. Den Aufführungen,
               die er veranſtaltet, wird großes Intereſſe entgegengebracht, und der \textcolor{brown}{Verein}{}\ledrightnote{{$\rightarrow$}\textcolor{brown}{Akademischer Verein für Kunst und Literatur}} gibt ſich {\pb}Mühe, gute Aufführungen herauszubringen, wenn er
               auch natürlich nicht über Darſteller erſten Ranges verfügt. Nur müßteſt Du die
               Vorbereitungen etwas überwachen u. Dir das Recht ſichern, bei der Rollenbeſetzung
               mitzuſprechen. Vielleicht iſt die \textsc{\textcolor{blue}{Triesch}{}\ledrightnote{\textcolor{blue}{Irene Triesch}}} zu einer Gaſtrolle als \textsc{\textcolor{green}{Beatrice}{}\ledrightnote{\textcolor{green}{Der Schleier der Beatrice. Schauspiel in fünf Akten}}} zu haben. Oder wie wenn Frl. \textsc{\textcolor{blue}{Olga}{}\ledrightnote{\textcolor{blue}{Olga Schnitzler}}} die Rolle kreirte?\pend
           
\pstart
           {\pb}Was iſt mit dem \label{K_L03067-2v}\edtext{Theater »\textcolor{brown}{zum lieben
                  Auguſtin}{}\ledrightnote{\textcolor{brown}{Jung-Wiener Theater zum Lieben Augustin}}}{\lemma{\textnormal{\emph{Theater … Auguſtin}}}\Cendnote{\textnormal{Das \emph{\textcolor{brown}{Jung-Wiener Theater zum lieben Augustin}} war ein von \textcolor{blue}{Felix Salten} geleitetes literarisches Varieté, das am 16. 11. 1901 eröffnet wurde.}}}\label{K_L03067-2h}«? Ein glücklicher
               Titel und wohl auch eine glückliche Idee. Wer gibt das Geld? Jetzt hat alſo auch \textsc{\textcolor{blue}{Salten}{}\ledrightnote{\textcolor{blue}{Felix Salten}}} ein Mittel gefunden, reich und berühmt zu werden. Ich ſchäme mich ſehr, ſo ganz
               allein zu zurückzubleiben.\pend
           
\pstart
           Viele treue Grüße! {\\[\baselineskip]}Dein {\\[\baselineskip]}\spacefill\mbox{Paul Goldmann.}\pend
           \leftskip=0em{}\endnumbering\briefempfaengerindex{Schnitzler, Arthur@\textsc{Schnitzler, Arthur}!zzzGoldmann, Paul@\emph{von Paul Goldmann}!1901-05-161@{16. 5. {[}1901{]}}|)be}\mylabel{h}
\begin{anhang}
\end{anhang}\normalsize

\doendnotes{C}
\bigskip
\vfill

\clearpage

\footnotesize

\lohead{\textsc{register}}

% Definiere theindex-Environment komplett neu ohne reledmac
\makeatletter
\renewenvironment{theindex}{%
  \section*{\indexname}%
  \setlength{\parindent}{0pt}%
  \setlength{\parskip}{0pt plus 0.3pt}%
  \let\item\@idxitem
}{%
  \clearpage
}
\makeatother

\IfFileExists{\jobname-pw.ind}{\input{\jobname-pw.ind}}{}

\end{document}

      