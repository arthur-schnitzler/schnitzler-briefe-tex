%% latex-korrekturansicht-vorspann.tex
%% Vorspann für die Korrekturansicht.
%% Lädt die gemeinsame Datei latex-vorspann.tex mit gesetztem Schalter.

\newif\ifkorrekturansicht
\korrekturansichttrue

\input{../tex-inputs/latex-vorspann}


\section[Oscar Blumenthal an Arthur Schnitzler, 1. 8. 1892]{L00110 Oscar Blumenthal an Arthur Schnitzler, 1. 8. 1892}
\nopagebreak\mylabel{L00110v}
\rehead{ }\normalsize\beginnumbering\briefempfaengerindex{Schnitzler, Arthur@\textsc{Schnitzler, Arthur}!zzzBlumenthal, Oskar@\emph{von Oskar Blumenthal}!1892-08-011@{1. 8. 1892}|(be}
\toendnotes[C]{\smallbreak\pagebreak[2]}
\correspDesc{Versand  durch Oscar Blumenthal am 1. 8. 1892 in Berlin
\newline{}Erhalt  durch Arthur Schnitzler im Zeitraum [2. 8. 1892
                  – 6. 8. 1892?] in Wien}\toendnotes[C]{\smallbreak}
\Standort{CUL, Schnitzler, B 15.}
\physDesc{Brief, 1 Blatt, 1 Seite, 536 Zeichen
\newline{}Handschrift Schreibkraft: schwarze Tinte, deutsche Kurrent
\newline{}Handschrift Oskar Blumenthal: schwarze Tinte, deutsche Kurrent (\noindent{}Unterschrift)
\newline{}Schnitzler: mit rotem Buntstift eine Unterstreichung und nummeriert:
                                    »3« }
\pstart
           \centering{}{\pb}\textcolor{gray}{\textbf{\textcolor{brown}{LESSING-THEATER}\orgindex{Lessing-Theater@Lessing-Theater|pw}{}\ledrightnote{\textcolor{brown}{Lessing-Theater}}}}\pend
           
\pstart
           \centering{}\textcolor{gray}{\textbf{Director:}}{\\}\textcolor{gray}{\textbf{Dr. Oscar Blumenthal.}}\pend
           
\pstart
           \raggedleft{}\textcolor{gray}{\textbf{\textcolor{pink}{Berlin N.W.}\oindex{Berlin@\textbf{Berlin}, \emph{Hauptstadt}|pw}{}\ledrightnote{\textcolor{pink}{Berlin}}, den}}{ }1. August \textcolor{gray}{\textbf{189}}2.{\\}\textcolor{gray}{\textbf{\textcolor{pink}{Friedrich-Carl-Ufer}\oindex{Kapelle-Ufer@\textbf{Kapelle-Ufer}, \emph{Straße}|pw}{}\ledrightnote{\textcolor{pink}{Kapelle-Ufer}}}}.\pend
           
\pstart\center{}Werther Herr Doktor!\pend\vspace{0.5em}
\pstart
           Ueber den Aufführungstermin von »\textcolor{green}{Das Märchen}\pwindex{Schnitzler, Arthur 15.\,5.\,1862 Wien – 21.\,10.\,1931 ebd.@\textsc{Schnitzler, Arthur} (15.\,5.\,1862 Wien – 21.\,10.\,1931 ebd.), \emph{Schriftsteller, Mediziner}!Märchen. Schauspiel in drei Aufzügen@\strich\emph{Das Märchen. Schauspiel in drei Aufzügen}|pw}{}\ledrightnote{\textcolor{green}{Das Märchen. Schauspiel in drei Aufzügen}}«
               kann ich Ihnen im Augenblick eine beſtimmte Zuſage nicht machen, da ſich die
               Dispoſitionen für die neue Saiſon noch nicht klar genug überblicken laſſen. Doch wird
               jedenfalls erſt im zweiten Quartal die Aufführung ſtattfinden können, da
               ich für die Monate Oktober, November, Dezember
               theils durch die abgeſchloſſenen Verträge, theils durch das Gaſtſpiel der \textcolor{blue}{\textsc{Duse}}\pwindex{Duse, Eleonora 3.\,10.\,1858 Vigevano – 21.\,4.\,1924 Pittsburgh@\textsc{Duse, Eleonora} (3.\,10.\,1858 Vigevano – 21.\,4.\,1924 Pittsburgh), \emph{Schauspielerin}|pw}{}\ledrightnote{\textcolor{blue}{Eleonora Duse}}{ }ſehr eingeengt bin.\pend
           
\pstart
           Mit freundlichen Grüßen\hspace*{2.5em}Ihr{\\[\baselineskip]}\spacefill\mbox{{[}hs. Blumenthal:{]} Dr. Osc. Blumenthal}\pend
           \leftskip=0em{}
\pstart
           \noindent{}{[}hs.:{]} Herrn{\\}\textsc{Dr Arthur Schnitzler}{\\}\textsc{\textcolor{pink}{Wien I.}\oindex{I., Innere Stadt@\textbf{I., Innere Stadt}, \emph{Verwaltungsgebiet}|pw}{}\ledrightnote{\textcolor{pink}{I., Innere Stadt}}}\pend
           
\pstart
           cop.\pend
           \selectlanguage{ngerman}\endnumbering\briefempfaengerindex{Schnitzler, Arthur@\textsc{Schnitzler, Arthur}!zzzBlumenthal, Oskar@\emph{von Oskar Blumenthal}!1892-08-011@{1. 8. 1892}|)be}\mylabel{L00110h}  \normalsize

\doendnotes{C}
\bigskip
\vfill

\clearpage

\footnotesize

\lohead{\textsc{register}}

% Definiere theindex-Environment komplett neu ohne reledmac
\makeatletter
\renewenvironment{theindex}{%
  \section*{\indexname}%
  \setlength{\parindent}{0pt}%
  \setlength{\parskip}{0pt plus 0.3pt}%
  \let\item\@idxitem
}{%
  \clearpage
}
\makeatother

\IfFileExists{\jobname-pw.ind}{\input{\jobname-pw.ind}}{}

\end{document}

      