%% latex-korrekturansicht-vorspann.tex
%% Vorspann für die Korrekturansicht.
%% Lädt die gemeinsame Datei latex-vorspann.tex mit gesetztem Schalter.

\newif\ifkorrekturansicht
\korrekturansichttrue

\input{../tex-inputs/latex-vorspann}


\section[Stefan Zweig an Arthur Schnitzler, {[}26. 11. 1914?{]}]{L03647 Stefan Zweig an Arthur Schnitzler, {[}26. 11. 1914?{]}}
\nopagebreak\mylabel{L03647v}
\rehead{ }\normalsize\beginnumbering\briefempfaengerindex{Schnitzler, Arthur@\textsc{Schnitzler, Arthur}!zzzZweig, Stefan@\emph{von Stefan Zweig}!1914-11-261@{{[}26. 11. 1914?{]}}|(be}
\toendnotes[C]{\smallbreak\pagebreak[2]}\Standort{CUL, Schnitzler, B 118.}
\physDesc{Briefkarte, 898 Zeichen
\newline{}Handschrift: lila Tinte, lateinische Kurrent
\newline{}Schnitzler: mit rotem Buntstift Eine Unterstreichung }
\buchAbdrucke{\weitereDrucke{Stefan Zweig: \emph{Briefwechsel mit Hermann Bahr, Sigmund Freud, Rainer Maria
                        Rilke und Arthur Schnitzler}. Frankfurt am Main: \emph{S. Fischer} 1987, S. 381–382.} }\toendnotes[C]{\smallbreak}
\pstart
           {\pb}\textcolor{gray}{\textbf{SZ}}\hfill \textcolor{gray}{\textbf{\textcolor{pink}{VIII. KOCHGASSE 8.}\oindex{Kochgasse 8@\textbf{Kochgasse 8}, \emph{Wohngebäude (K.WHS)}|pw}{}\ledrightnote{\textcolor{pink}{Kochgasse 8}}}}\pend
           \vspace{0.5em}
\pstart
           Verehrter Herr Doktor, ich bin sehr unglücklich: Sie haben mich
               vergebens angerufen. Aber ich unterschätzte das Militär und meinte, dass wenn man um
               6 Uhr früh ausrückte das \label{K_L03647-1v}\edtext{Salutieren zu
               erlernen}{\lemma{\textnormal{\emph{Salutieren zu
               erlernen}}}\Cendnote{\textnormal{Am 12. 11. 1914 wurde \textcolor{blue}{Zweig}\pwindex{Zweig, Stefan 28.11.1881 – 23.02.1942@\textsc{Zweig, Stefan} (28.11.1881 – 23.02.1942), \emph{Schriftsteller}|pwk} in den Militärdienst aufgenommen, am
                  14. 11. 1914 war er erstmals bei seiner vorläufigen Einsatzstelle in
                  \textcolor{pink}{Klosterneuburg}\oindex{Klosterneuburg@\textbf{Klosterneuburg}, \emph{P.PPLA3}|pwk}, vom 23. bis
                     30. 11. 1914 vermerkte er im Tagebuch eine Woche zeitraubender
                  Exerzierübungen ebendort, vgl. \textcolor{blue}{Stefan Zweig}\pwindex{Zweig, Stefan 28.11.1881 – 23.02.1942@\textsc{Zweig, Stefan} (28.11.1881 – 23.02.1942), \emph{Schriftsteller}|pwk}: \emph{\textcolor{green}{Tagebuch im Kriegsjahr 1914}\pwindex{Tagebuch im Kriegsjahr 1914 vom Tage der deutschen Kriegserklaerung an Russland@\emph{Tagebuch im Kriegsjahr 1914 vom Tage der deutschen Kriegserklärung an Rußland}|pwk}}. In:
                     https://stefanzweig.digital, SZ-AAP/L2.}}}\label{K_L03647-1}, um 12 Uhr schon zu Hause sein könnte. In Wirklichkeit wurde es 4
               Uhr und ich weiss noch nicht bestimmt, ob ich die Materie beherrsche. All das sind
               Vorbereitungen für meinen Dienst: am 1. Dez.{ }{\pb}bin ich ins \textcolor{brown}{Kriegsarchiv}\orgindex{Kriegsarchiv@Kriegsarchiv|pw}{}\ledrightnote{\textcolor{brown}{Kriegsarchiv}} einberufen und werde dort (unter Aufsicht von
                  \textcolor{blue}{Bartsch}\pwindex{Bartsch, Rudolf Hans 11.02.1873 – 07.02.1952@\textsc{Bartsch, Rudolf Hans} (11.02.1873 – 07.02.1952), \emph{Schriftsteller}|pw}{}\ledrightnote{\textcolor{blue}{Rudolf Hans Bartsch}} und \textcolor{blue}{Ginzkey}\pwindex{Ginzkey, Franz Karl 08.09.1871 – 11.04.1963@\textsc{Ginzkey, Franz Karl} (08.09.1871 – 11.04.1963), \emph{Schriftsteller}|pw}{}\ledrightnote{\textcolor{blue}{Franz Karl Ginzkey}}) die vielfach geheimen Documente des Krieges zu ordnen
               und zu gestalten haben, eine Arbeit auf die ich mich so sehr freue wie nur möglich,
               obzwar sie viel fordert. So versäumte ich die Freude, Sie sprechen zu können, auch
               die nächsten Tage exerciere ich in \textcolor{pink}{Klosterneuburg}\oindex{Klosterneuburg@\textbf{Klosterneuburg}, \emph{P.PPLA3}|pw}{}\ledrightnote{\textcolor{pink}{Klosterneuburg}}
               und bitte Sie darum, mir die \label{K_L03647-2v}\edtext{\textcolor{green}{Berichtigung}\pwindex{Brief Artur Schnitzlers@\emph{Ein Brief Artur Schnitzlers}|pwv}{}\ledrightnote{{$\rightarrow$}\emph{\textcolor{green}{Ein Brief Artur Schnitzlers}}}}{\lemma{\textnormal{\emph{Berichtigung}}}\Cendnote{\textnormal{Der Brief ist undatiert.
                  \textcolor{blue}{Schnitzler} kontaktierte \textcolor{blue}{Zweig}\pwindex{Zweig, Stefan 28.11.1881 – 23.02.1942@\textsc{Zweig, Stefan} (28.11.1881 – 23.02.1942), \emph{Schriftsteller}|pwk}, um mit ihm den Text der
                  \textcolor{green}{Berichtigung}\pwindex{Brief Artur Schnitzlers@\emph{Ein Brief Artur Schnitzlers}|pwkv} eines ihn diffamierenden \textcolor{green}{Interviews}\pwindex{?? [Fiktives Interview aus der Kriegszeit]@\emph{?? [Fiktives Interview aus der Kriegszeit]}|pwkv} durchzugehen, 
                  das in \textcolor{pink}{St. Petersburg}\oindex{Sankt Petersburg@\textbf{Sankt Petersburg}, \emph{P.PPLA}|pwk} erschienen war. Bis zum 24. 11. 1914 feilte er nachweislich
                  am Text, was als der früheste Zeitpunkt, an dem dieses Schreiben verfasst sein kann, zu gelten hat. Da aber \textcolor{blue}{Zweig}\pwindex{Zweig, Stefan 28.11.1881 – 23.02.1942@\textsc{Zweig, Stefan} (28.11.1881 – 23.02.1942), \emph{Schriftsteller}|pwk}
                  informiert zu sein scheint, dass der Text fertiggestellt war und um postalische Übermittelung bittet, dürfte
                  das Schreiben \textcolor{blue}{Schnitzlers} vom 27. 11. 1914 die unmittelbare Antwort
                  auf die vorliegende Karte darstellen. Die Dringlichkeit, die aus der Verwendung des Telefons durch \textcolor{blue}{Schnitzler}
                  ablesbar ist, spricht dafür, dass dieser umgehend auf die vorliegende
                  Karte reagierte und nicht ein, zwei Tage zuwartete. \textcolor{blue}{Zweig}\pwindex{Zweig, Stefan 28.11.1881 – 23.02.1942@\textsc{Zweig, Stefan} (28.11.1881 – 23.02.1942), \emph{Schriftsteller}|pwk} verfasste seine Karte
                  nach vier Uhr am Abend, so dass \textcolor{blue}{Schnitzler} am nächsten Tag der Sekretärin
                  seine Antwort diktiert haben dürfte.}}}\label{K_L03647-2} brieflich zu senden – ich bin nicht mehr
               Herr meiner Zeit.\pend
           
\pstart
           Viele viele Grüsse Ihres aufrichtig getreuen{\\[\baselineskip]}\spacefill\mbox{Stefan Zweig}\pend
           \leftskip=0em{}\selectlanguage{ngerman}\endnumbering\briefempfaengerindex{Schnitzler, Arthur@\textsc{Schnitzler, Arthur}!zzzZweig, Stefan@\emph{von Stefan Zweig}!1914-11-261@{{[}26. 11. 1914?{]}}|)be}\mylabel{L03647h}  \normalsize

\doendnotes{C}
\bigskip
\vfill

\clearpage

\footnotesize

\lohead{\textsc{register}}

% Definiere theindex-Environment komplett neu ohne reledmac
\makeatletter
\renewenvironment{theindex}{%
  \section*{\indexname}%
  \setlength{\parindent}{0pt}%
  \setlength{\parskip}{0pt plus 0.3pt}%
  \let\item\@idxitem
}{%
  \clearpage
}
\makeatother

\IfFileExists{\jobname-pw.ind}{\input{\jobname-pw.ind}}{}

\end{document}

      