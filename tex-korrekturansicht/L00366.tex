%% latex-korrekturansicht-vorspann.tex
%% Vorspann für die Korrekturansicht.
%% Lädt die gemeinsame Datei latex-vorspann.tex mit gesetztem Schalter.

\newif\ifkorrekturansicht
\korrekturansichttrue

\input{../tex-inputs/latex-vorspann}


               \section[Richard Beer-Hofmann an Arthur Schnitzler, {[}5./6.? 9. 1894{]}]{ Richard Beer-Hofmann an Arthur Schnitzler,
               {[}5./6.? 9. 1894{]}}\nopagebreak\mylabel{v}\rehead{ }\normalsize\beginnumbering\briefempfaengerindex{Schnitzler, Arthur@\textsc{Schnitzler, Arthur}!zzzBeer-Hofmann, Richard@\emph{von Richard Beer-Hofmann}!1894-09-051@{{[}5./6.? 9. 1894{]}}|(be} \toendnotes[C]{\smallbreak\pagebreak[2]} \Standort{CUL, Schnitzler, B 8.}
\physDesc{Brief, 1 Blatt, 2 Seiten
\newline{}Handschrift: blauer Buntstift, lateinische Kurrent
\newline{}Schnitzler: mit Bleistift datiert: »July Sept. 94.« und nummeriert: »37« }\buchAbdrucke{\weitereDrucke{Arthur Schnitzler, Richard Beer-Hofmann: \emph{Briefwechsel 1891–1931}. Hg. Konstanze Fliedl. Wien, Zürich: \emph{Europaverlag} 1992, S. 59.} }\toendnotes[C]{\smallbreak}\pstart
           \noindent{}{\pb}Lieber Arthur! Ich
               bin nicht hier in \textcolor{pink}{Wien}{}\ledrightnote{\textcolor{pink}{Wien}} – nur Ihr Stock ist hier –
               ich bin hoffentlich auf der \label{K_L00366_1v}\edtext{Route}{\lemma{\textnormal{\emph{Route}}}\Cendnote{\textnormal{Mehrere Hinweise erlauben
                  das Einordnen dieses Korrespondenzstücks. Am 4. 9. 1894
                  reiste \textcolor{blue}{Schnitzler} von \textcolor{pink}{Bad Ischl} nach \textcolor{pink}{Wien}. Zwei Tage
                  zuvor hatten er und \textcolor{blue}{Beer-Hofmann}
                  sich zuletzt
                  gesehen. Offenbar nahm er an, jener wäre ebenfalls in \textcolor{pink}{Wien}. Da \textcolor{blue}{Beer-Hofmann} aber am
                     7. 9. 1894 bereits wieder aus \textcolor{pink}{Bad Ischl}
                   schreibt – was den Inhalt dieses Briefes obsolet werden
                  ließe – dürfte das Schreiben frühestens am 5. und spätestens am
                     6. 9. 1895 verfasst sein.}}}\label{K_L00366_1h} nach \textcolor{pink}{Italien}{}\ledrightnote{\textcolor{pink}{Italien}}, momentan – {\pb}da ich dies schreibe, – friere
               ich in \textcolor{pink}{Ischl}{}\ledrightnote{\textcolor{pink}{Bad Ischl}}, – hier. Dieser Brief ist
               unanständig wegen der vielen »hier«.\pend
           \pstart Herzlichst Ihr \spacefill\mbox{R}\pend{}\endnumbering\briefempfaengerindex{Schnitzler, Arthur@\textsc{Schnitzler, Arthur}!zzzBeer-Hofmann, Richard@\emph{von Richard Beer-Hofmann}!1894-09-051@{{[}5./6.? 9. 1894{]}}|)be}\mylabel{h}  \normalsize

\doendnotes{C}
\bigskip
\vfill

\clearpage

\footnotesize

\lohead{\textsc{register}}

% Definiere theindex-Environment komplett neu ohne reledmac
\makeatletter
\renewenvironment{theindex}{%
  \section*{\indexname}%
  \setlength{\parindent}{0pt}%
  \setlength{\parskip}{0pt plus 0.3pt}%
  \let\item\@idxitem
}{%
  \clearpage
}
\makeatother

\IfFileExists{\jobname-pw.ind}{\input{\jobname-pw.ind}}{}

\end{document}

      