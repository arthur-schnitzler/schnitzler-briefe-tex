%% latex-korrekturansicht-vorspann.tex
%% Vorspann für die Korrekturansicht.
%% Lädt die gemeinsame Datei latex-vorspann.tex mit gesetztem Schalter.

\newif\ifkorrekturansicht
\korrekturansichttrue

\input{../tex-inputs/latex-vorspann}


               \section[Arthur Schnitzler an Richard Beer-Hofmann, 26. 6. 1899]{ Arthur Schnitzler an Richard Beer-Hofmann, 26. 6. 1899}\nopagebreak\mylabel{v}\rehead{ }\normalsize\beginnumbering\briefempfaengerindex{Beer-Hofmann, Richard@\textsc{Beer-Hofmann, Richard}!zzzSchnitzler, Arthur@\emph{von Arthur Schnitzler}!1899-06-261@{26. 6. 1899}|(be} \toendnotes[C]{\smallbreak\pagebreak[2]} \Standort{YCGL, MSS 31.}
\physDesc{Bildpostkarte
\newline{}Handschrift: Bleistift, deutsche Kurrent\newline{}Versand: 1) Stempel: »\nobreak{}\oindex{Orahovica@\textbf{Orahovica}, \emph{http://www.geonames.org/ontologyP.PPLA2}|pwk}Orahovica, 26. \textcolor{gray}{6}. {[}99{]}\nobreak{}«.  2) Stempel: »\nobreak{}\oindex{Seeboden@\textbf{Seeboden}, \emph{http://www.geonames.org/ontologyA.ADM3}|pwk}Seeboden, 28. {[}6{]}. 9\textcolor{gray}{9}\nobreak{}«. }\pstart{}{\pb}\textsc{Dr. Rich. Beer-Hofmann}\pend{}\pstart{}\textcolor{pink}{\textsc{Seeboden am Millstätter}see}{}\ledrightnote{\textcolor{pink}{Seeboden}}\pend{}\pstart{}\textcolor{pink}{\textsc{Villa Platzer}}{}\ledrightnote{\textcolor{pink}{Villa Platzer}}\pend{}\pstart{}\textcolor{pink}{\textsc{Kärnthen}}{}\ledrightnote{\textcolor{pink}{Kärnten}}\pend{}{\bigskip}\pstart
           \noindent{}\centering{}\textcolor{gray}{\textbf{{\pb}\textcolor{pink}{Pozdrav iz Orahovice}{}\ledrightnote{\textcolor{pink}{Orahovica}}.}}\pend
           \pstart
           {\pb}Herzlichſte Grüße. Ich hoffe in \textcolor{pink}{Wien}{}\ledrightnote{\textcolor{pink}{Wien}} einen Brief von Ihnen zu finden. Ihr \spacefill\mbox{Arth}\pend
           \endnumbering\briefempfaengerindex{Beer-Hofmann, Richard@\textsc{Beer-Hofmann, Richard}!zzzSchnitzler, Arthur@\emph{von Arthur Schnitzler}!1899-06-261@{26. 6. 1899}|)be}\mylabel{h}  \normalsize

\doendnotes{C}
\bigskip
\vfill

\clearpage

\footnotesize

\lohead{\textsc{register}}

% Definiere theindex-Environment komplett neu ohne reledmac
\makeatletter
\renewenvironment{theindex}{%
  \section*{\indexname}%
  \setlength{\parindent}{0pt}%
  \setlength{\parskip}{0pt plus 0.3pt}%
  \let\item\@idxitem
}{%
  \clearpage
}
\makeatother

\IfFileExists{\jobname-pw.ind}{\input{\jobname-pw.ind}}{}

\end{document}

      