%% latex-korrekturansicht-vorspann.tex
%% Vorspann für die Korrekturansicht.
%% Lädt die gemeinsame Datei latex-vorspann.tex mit gesetztem Schalter.

\newif\ifkorrekturansicht
\korrekturansichttrue

\input{../tex-inputs/latex-vorspann}


\renewcommand{\erwaehnteOrte}{Orte: Mori, Palast Hotel Lido, Pörtschach, Riva del Garda, Villach, Welsberg-Taisten, Wildbad Waldbrunn}
\renewcommand{\erwaehnteWerke}{}
\section[ Paul Goldmann an Arthur Schnitzler, 15. 8. {[}1901{]}]{Paul Goldmann an Arthur Schnitzler, 15. 8. {[}1901{]}}
\nopagebreak\mylabel{v}
\rehead{ }\normalsize\beginnumbering\briefempfaengerindex{Schnitzler, Arthur@\textsc{Schnitzler, Arthur}!zzzGoldmann, Paul@\emph{von Paul Goldmann}!1901-08-151@{15. 8. {[}1901{]}}|(be}
\toendnotes[C]{\smallbreak\pagebreak[2]}\Standort{DLA, A:Schnitzler, HS.NZ85.1.3171.}
\physDesc{Bildpostkarte
\newline{}Handschrift: 1) schwarze Tinte, deutsche Kurrent\hspace{1em}2) schwarze Tinte, lateinische Kurrent (\noindent{}Adresse)\hspace{1em}
\newline{}Versand: 1) Stempel: »\nobreak{}\oindex{Mori@\textbf{Mori}, \emph{https://www.geonames.org/ontologyP.PPLA3}|pwk}Mori, {[}15{]}. 8\nobreak{}«.   2) Stempel: »\nobreak{}\oindex{Welsberg-Taisten@\textbf{Welsberg-Taisten}, \emph{Besiedelter Ort (A.BSO)}|pwk}{[}Welsbe{]}rg, 17. 8. 01\nobreak{}«. 
\newline{}Schnitzler: mit Bleistift das Jahr »{[}1{]}901« vermerkt }\toendnotes[C]{\smallbreak}\pstart{}{\pb}Herrn\pend{}\pstart{}Dr. Arthur Schnitzler\pend{}\pstart{}\textcolor{pink}{Welsberg im Pusterthal}{}\ledrightnote{\textcolor{pink}{Welsberg-Taisten}}\pend{}\pstart{}\textcolor{pink}{Wildbad-Hôtel}{}\ledrightnote{\textcolor{pink}{Wildbad Waldbrunn}}.\pend{}
{\bigskip}
\pstart
           \noindent{}{\pb}\textcolor{gray}{\textbf{\textcolor{pink}{PALAST HÔTEL LIDO}{}\ledrightnote{\textcolor{pink}{Palast Hotel Lido}}, \textcolor{pink}{RIVA \textsuperscript{A}/GARDASEE}{}\ledrightnote{\textcolor{pink}{Riva del Garda}}.}}\pend
           
\pstart
           \raggedleft{}15. Auguſt.\pend
           
\pstart
           Es iſt kühl und \textcolor{gray}{kö}ſtlich hier. Aber einſam. Bitte, lieber \textsc{Arthur}, hol’ auch einmal \textsc{Poste
                  restante} Briefe von der Poſt u. heb’ ſie mir auf, damit ich ſie \label{K_L03079-1v}\edtext{Sonntag}{\lemma{\textnormal{\emph{Sonntag}}}\Cendnote{\textnormal{Am Sonntag, dem 18. 8. 1901, kam \textcolor{blue}{Goldmann} in
                     \textcolor{pink}{Welsberg} an. \textcolor{blue}{Schnitzler} war bereits seit 15. 8. 1901 dort. Am 26. 8. 1901 reiste er
                  weiter nach \textcolor{pink}{Villach}, \textcolor{blue}{Goldmann} nach \textcolor{pink}{Pörtschach}.}}}\label{K_L03079-1h} finde. Herzliche Grüße an alle! \spacefill\mbox{P. G.}\pend
           \endnumbering\briefempfaengerindex{Schnitzler, Arthur@\textsc{Schnitzler, Arthur}!zzzGoldmann, Paul@\emph{von Paul Goldmann}!1901-08-151@{15. 8. {[}1901{]}}|)be}\mylabel{h}
\begin{anhang}
\end{anhang}\normalsize

\doendnotes{C}
\bigskip
\vfill

\clearpage

\footnotesize

\lohead{\textsc{register}}

% Definiere theindex-Environment komplett neu ohne reledmac
\makeatletter
\renewenvironment{theindex}{%
  \section*{\indexname}%
  \setlength{\parindent}{0pt}%
  \setlength{\parskip}{0pt plus 0.3pt}%
  \let\item\@idxitem
}{%
  \clearpage
}
\makeatother

\IfFileExists{\jobname-pw.ind}{\input{\jobname-pw.ind}}{}

\end{document}

      