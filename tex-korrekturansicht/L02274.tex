%% latex-korrekturansicht-vorspann.tex
%% Vorspann für die Korrekturansicht.
%% Lädt die gemeinsame Datei latex-vorspann.tex mit gesetztem Schalter.

\newif\ifkorrekturansicht
\korrekturansichttrue

\input{../tex-inputs/latex-vorspann}


               \section[Georg Brandes an Arthur Schnitzler, 28. 9. 1917]{ Georg Brandes an Arthur Schnitzler, 28. 9. 1917}\nopagebreak\mylabel{v}\rehead{ }\normalsize\beginnumbering\briefempfaengerindex{Schnitzler, Arthur@\textsc{Schnitzler, Arthur}!zzzBrandes, Georg@\emph{von Georg Brandes}!1917-09-281@{28. 9. 1917}|(be} \toendnotes[C]{\smallbreak\pagebreak[2]} \Standort{CUL, Schnitzler, B 17.}
\physDesc{Postkarte
\newline{}Handschrift: schwarze Tinte, lateinische Kurrent\newline{}Versand: 1) Stempel: »\nobreak{}\oindex{Kopenhagen@\textbf{Kopenhagen}, \emph{Besiedelter Ort (A.BSO)}|pwk}Kj\textcolor{gray}{øbenha}vn, 2{[}8. 9.{]} 17, 2—3 E\nobreak{}«.  2) Stempel: »\nobreak{}Zensuriert \textcolor{brown}{{[}k. u. k.{]} Zensurstelle Wien}\nobreak{}«. 
\newline{}Schnitzler: mit rotem Buntstift vereinzelte Unterstreichungen \newline{}Ordnung: mit Bleistift von unbekannter Hand nummeriert: »47« }\buchAbdrucke{\weitereDrucke{Georg Brandes, Arthur Schnitzler: \emph{Ein Briefwechsel}. Hg. Kurt Bergel. Bern: \emph{Francke} 1956, S. 121.} }\toendnotes[C]{\smallbreak}\pstart{}{\pb}Herrn Dr. Arthur
                        Schnitzler\pend{}\pstart{}\textcolor{pink}{Sternwartestrasse 71}{}\ledrightnote{\textcolor{pink}{Sternwartestraße}}\pend{}\pstart{}\textcolor{pink}{Wien \textsubscript{XVIII}}{}\ledrightnote{\textcolor{pink}{XVIII., Währing}}\pend{}{\bigskip}\pstart
           \raggedleft{}{\pb}\textcolor{pink}{Kopenhagen}{}\ledrightnote{\textcolor{pink}{Kopenhagen}}{ }28 Sept. 17\pend
           \pstart
           Verehrter Freund! \hspace*{3.5em}Es hat mich riesig gefreut, dass Sie auch in
                    dieser traurigen Zeit an mich gedacht haben. Ich habe Ihr lächelnd-wehmütiges
                        \textcolor{green}{Buch}{}\ledrightnote{→\textcolor{green}{Doktor Gräsler, Badearzt}} mit grossem Behagen
                    gelesen und die Bekanntschaft mit zwei reizenden jungen Damen, genannt \textcolor{green}{Sabine}{}\ledrightnote{→\textcolor{green}{Doktor Gräsler, Badearzt}} und \textcolor{green}{Katharina}{}\ledrightnote{→\textcolor{green}{Doktor Gräsler, Badearzt}}, gemacht. Auch eine gewisse
                    Lebensphilosophie, eine überlegene, ist in dem \textcolor{green}{Buch}{}\ledrightnote{→\textcolor{green}{Doktor Gräsler, Badearzt}}. Während Sie erfinderisch schöpfen, muss ich
                    mich begnügen, geschichtliche Gestalten neu zu formen. Ich habe während des
                    Krieges ein \textcolor{green}{Buch über \textcolor{blue}{Goethe}{}\ledrightnote{\textcolor{blue}{Johann Wolfgang von Goethe}}}{}\ledrightnote{→\textcolor{green}{Wolfgang Goethe}} und eins über \textcolor{green}{\textcolor{blue}{Voltaire}{}\ledrightnote{\textcolor{blue}{Voltaire}}}{}\ledrightnote{→\textcolor{green}{Voltaire und sein Jahrhundert}} publiciert, beide in zwei Bänden, ausserdem einen \textcolor{green}{Band über den Weltkrieg}{}\ledrightnote{→\textcolor{green}{Verdenskrigen [The World at War]}}, nur hier und in
                        \textcolor{pink}{Amerika}{}\ledrightnote{\textcolor{pink}{Amerika}} als \textcolor{green}{\uline{The World at War}}{}\ledrightnote{\textcolor{green}{Verdenskrigen [The World at War]}} erschienen. Seit 5 Monaten bin ich so närrisch, an einer grösseren Arbeit
                    über \textcolor{green}{\textcolor{blue}{\uline{Cäsar}}{}\ledrightnote{\textcolor{blue}{Gaius Iulius Caesar}}}{}\ledrightnote{→\textcolor{green}{Gaius Julius Cæsar}} zu pfuschen. Die wird wol mehr {\pb}als ein Jahr noch nehmen.
                    Ich denke oft an \textcolor{pink}{Wien}{}\ledrightnote{\textcolor{pink}{Wien}} und an die Freunde dort.
                    Seit wir uns im November 1912
               sahen, ist Alles verändert. Ich kann
                    kaum verstehen, dass es fast schon 5 Jahre her ist.\pend
           \pstart
           Bitte sehr mich in der Erinnerung Ihrer Frau \textcolor{blue}{Gemahlin}{}\ledrightnote{→\textcolor{blue}{Olga Schnitzler}} und \textcolor{blue}{Beer-Hofmanns}{}\ledrightnote{\textcolor{blue}{Richard Beer-Hofmann}{\newline}\textcolor{blue}{Paula Beer-Hofmann}} zurückzurufen.\pend
           \pstart
           Ihr ganz ergebener{\\[\baselineskip]}\spacefill\mbox{Georg Brandes}\pend
           \leftskip=0em{}\endnumbering\briefempfaengerindex{Schnitzler, Arthur@\textsc{Schnitzler, Arthur}!zzzBrandes, Georg@\emph{von Georg Brandes}!1917-09-281@{28. 9. 1917}|)be}\mylabel{h}  \normalsize

\doendnotes{C}
\bigskip
\vfill

\clearpage

\footnotesize

\lohead{\textsc{register}}

% Definiere theindex-Environment komplett neu ohne reledmac
\makeatletter
\renewenvironment{theindex}{%
  \section*{\indexname}%
  \setlength{\parindent}{0pt}%
  \setlength{\parskip}{0pt plus 0.3pt}%
  \let\item\@idxitem
}{%
  \clearpage
}
\makeatother

\IfFileExists{\jobname-pw.ind}{\input{\jobname-pw.ind}}{}

\end{document}

      