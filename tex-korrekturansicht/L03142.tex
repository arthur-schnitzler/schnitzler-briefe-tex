%% latex-korrekturansicht-vorspann.tex
%% Vorspann für die Korrekturansicht.
%% Lädt die gemeinsame Datei latex-vorspann.tex mit gesetztem Schalter.

\newif\ifkorrekturansicht
\korrekturansichttrue

\input{../tex-inputs/latex-vorspann}


\renewcommand{\erwaehntePersonen}{Personen: Paul Goldmann, Ferdinand von Holzinger, Maria Charlotte Lamberg, Hilda von Mitis, Maximilian von Mitis, Maria Pia Mitis, Charlotte Pohl-Glas}
\renewcommand{\erwaehnteOrte}{Orte: Alser Straße, Bad Ischl, Salzburg, Steyr, Veronikagasse, Wien, XVII., Hernals}
\renewcommand{\erwaehnteWerke}{Werke: Die kleine Veronika, Heldentod, Neue Deutsche Rundschau, Wiener Allgemeine Zeitung}
\section[ Felix Salten an Arthur Schnitzler, 7. 8. 1894]{Felix Salten an Arthur Schnitzler, 7. 8. 1894}
\nopagebreak\mylabel{v}
\rehead{ }\normalsize\beginnumbering\briefempfaengerindex{Schnitzler, Arthur@\textsc{Schnitzler, Arthur}!zzzSalten, Felix@\emph{von Felix Salten}!1894-08-071@{7. 8. 1894}|(be}
\toendnotes[C]{\smallbreak\pagebreak[2]}\Standort{CUL, Schnitzler, B 89, A 1.}
\physDesc{Brief, 1 Blatt, 2 Seiten, 491 Zeichen
\newline{}Handschrift: schwarze Tinte, lateinische Kurrent
\newline{}Ordnung: mit Bleistift von unbekannter Hand nummeriert: »43« }\toendnotes[C]{\smallbreak}
\pstart
           \raggedleft{}{\pb}\textcolor{pink}{Wien}{}\ledrightnote{\textcolor{pink}{Wien}}, 7. VIII. 94\pend
           
\pstart{}Lieber Arthur,\pend
\pstart
           I. \label{K_L03142-1v}\edtext{Process}{\lemma{\textnormal{\emph{Process}}}\Cendnote{\textnormal{Es dürfte sich um den zweiten Prozess gegen \textcolor{blue}{Salten}s Partnerin \textcolor{blue}{Charlotte Glas} handeln. Der erste hatte wenige Tage zuvor, am 25. 7. 1894, stattgefunden. Bei einer Versammlung am
                     1. 5. 1894 hatte sie einen Hochruf auf die
                     »internationale revolutionäre Sozialdemokratie« ausgerufen. Die
                  Verwendung des Wortes ›revolutionär‹ wurde ihr als umstürzlerisch zur Last
                  gelegt worden. Der \textcolor{blue}{Richter} verurteilte sie zu 14 Tagen Arrest, die sie Mitte September 1894 absolvierte, vgl. Felix Salten an Arthur Schnitzler, [11. 9. 1894]. Am 30. 11. 1894 wurde sie dann neuerlich in \textcolor{pink}{Steyr} für ein ähnliches Vergehen zu einem
                  weiteren Monat verurteilt. Diesen Arrest trat sie am 15. 1. 1895 in \textcolor{pink}{Wien} an, vgl. Felix Salten an Arthur Schnitzler, [14?. 1. 1895]. Zu diesem Zeitpunkt
                  war sie bereits mit dem gemeinsamen \textcolor{blue}{Kind} mit \textcolor{blue}{Salten} schwanger.}}}\label{K_L03142-1h}
               ist neuerdings vertagt.\pend
           
\pstart
           II. Wie ich Ihnen auf meiner \label{K_L03142-2v}\edtext{Karte
               nach \textcolor{pink}{Salzburg}{}\ledrightnote{\textcolor{pink}{Salzburg}}}{\lemma{\textnormal{\emph{Karte
               nach Salzburg}}}\Cendnote{\textnormal{Nicht erhalten. \textcolor{blue}{Schnitzler} war zwischen 1. 8. 1894 und 5. 8. 1894 in \textcolor{pink}{Salzburg}. \textcolor{blue}{Schnitzler} hatte sich für \textcolor{blue}{Goldmann}
                  bei
                  \textcolor{blue}{Salten} um Informationen zu \textcolor{blue}{Hilda von Mitis}  erkundigt, vgl. Paul Goldmann an Arthur Schnitzler, 29. 7. [1894].}}}\label{K_L03142-2h} berichtet, lebt
                  \textcolor{blue}{H. M.}{}\ledrightnote{\textcolor{blue}{Hilda von Mitis}} bei ihren \textcolor{blue}{Eltern}{}\ledrightnote{{$\rightarrow$}\textcolor{blue}{Maximilian von Mitis}{\newline}{$\rightarrow$}\textcolor{blue}{Maria Pia Mitis}}, welche bis zum 1. d. M.{ }\textcolor{pink}{Alserstrasse 42}{}\ledrightnote{\textcolor{pink}{Alser Straße}} wohnten, aber übersiedelt sind.
               Ich konnte damals die neue Adresse nicht ermitteln, habe sie jedoch heute erfragt. \textcolor{blue}{\uline{H. M.}}{}\ledrightnote{\textcolor{blue}{Hilda von Mitis}} wohnt: \textcolor{pink}{\uline{Hernals}}{}\ledrightnote{\textcolor{pink}{XVII., Hernals}}, \textcolor{pink}{Veronikagasse 25}{}\ledrightnote{\textcolor{pink}{Veronikagasse}}, II. Stock
               Thür 19.\pend
           
\pstart
           III. \label{K_L03142-3v}\edtext{\textcolor{green}{Heldentod}{}\ledrightnote{\textcolor{green}{Heldentod}} ruht}{\lemma{\textnormal{\emph{Heldentod ruht}}}\Cendnote{\textnormal{\textcolor{blue}{Salten}s Novelle \emph{\textcolor{green}{Heldentod}} wurde im Jahr darauf
                  publiziert: \textcolor{blue}{Felix Salten}: \emph{\textcolor{green}{Heldentod}}. In: \emph{\textcolor{green}{Wiener
                        Allgemeine Zeitung}}, Nr. 504, 1. 1. 1895,
                     S. 2–3.}}}\label{K_L03142-3h}.\pend
           
\pstart
           IV. \label{K_L03142-4v}\edtext{\textcolor{green}{Confirmandin}{}\ledrightnote{\textcolor{green}{Die kleine Veronika}}}{\lemma{\textnormal{\emph{Confirmandin}}}\Cendnote{\textnormal{Die Novelle mit dem Arbeitstitel \emph{\textcolor{green}{Die Confirmandin}} erschien Jahre später unter
                  dem Titel \emph{\textcolor{green}{Die kleine Veronika}}, \emph{\textcolor{green}{Neue Deutsche Rundschau}}, Jg. 13, Nr. 12,
                        Dezember 1902, S. 1285–1333.}}}\label{K_L03142-4h} geht
               langsam vorwärts, doch war ich in diesen {\pb}Tagen durch Besuch
               aufgehalten.\pend
           
\pstart
           V. ..........!!\pend
           
\pstart
           Herzlichst {\\[\baselineskip]}\spacefill\mbox{Salten.}\pend
           \leftskip=0em{}\endnumbering\briefempfaengerindex{Schnitzler, Arthur@\textsc{Schnitzler, Arthur}!zzzSalten, Felix@\emph{von Felix Salten}!1894-08-071@{7. 8. 1894}|)be}\mylabel{h}  \normalsize

\doendnotes{C}
\bigskip
\vfill

\clearpage

\footnotesize

\lohead{\textsc{register}}

% Definiere theindex-Environment komplett neu ohne reledmac
\makeatletter
\renewenvironment{theindex}{%
  \section*{\indexname}%
  \setlength{\parindent}{0pt}%
  \setlength{\parskip}{0pt plus 0.3pt}%
  \let\item\@idxitem
}{%
  \clearpage
}
\makeatother

\IfFileExists{\jobname-pw.ind}{\input{\jobname-pw.ind}}{}

\end{document}

      