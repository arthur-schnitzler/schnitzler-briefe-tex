%% latex-korrekturansicht-vorspann.tex
%% Vorspann für die Korrekturansicht.
%% Lädt die gemeinsame Datei latex-vorspann.tex mit gesetztem Schalter.

\newif\ifkorrekturansicht
\korrekturansichttrue

\input{../tex-inputs/latex-vorspann}


\renewcommand{\erwaehntePersonen}{Personen: Richard Beer-Hofmann, Georges Courteline, Clemens von Franckenstein, Paul Naschauer, Siegfried Trebitsch}
\renewcommand{\erwaehnteInstitutionen}{Institutionen: Wiener Schachclub}
\renewcommand{\erwaehnteOrte}{Orte: Raimund-Theater, Wien}
\renewcommand{\erwaehnteWerke}{Werke: Boubouroche. Lebensbild in 2 Acten, Boubouroche. Pièce en deux actes, Tagebuch}
\section[ Felix Salten an Arthur Schnitzler, {[}18. 11. 1899{]}]{Felix Salten an Arthur Schnitzler, {[}18. 11. 1899{]}}
\nopagebreak\mylabel{v}
\rehead{ }\normalsize\beginnumbering\briefempfaengerindex{Schnitzler, Arthur@\textsc{Schnitzler, Arthur}!zzzSalten, Felix@\emph{von Felix Salten}!1899-11-182@{{[}18. 11. 1899{]}}|(be}
\toendnotes[C]{\smallbreak\pagebreak[2]}\Standort{CUL, Schnitzler, B 89, A 2.}
\physDesc{Brief, 1 Blatt, 2 Seiten, 428 Zeichen
\newline{}Handschrift: Bleistift, lateinische Kurrent
\newline{}Schnitzler: mit Bleistift datiert: »18/1\textcolor{gray}{1} 99.« 
\newline{}Ordnung: mit Bleistift von unbekannter Hand nummeriert: »126« }\toendnotes[C]{\smallbreak}
\pstart{}{\pb}Lieber Freund,\pend
\pstart
           \label{K_L03302-1v}\edtext{Morgen, längstens Dienstag}{\lemma{\textnormal{\emph{Morgen, … Dienstag}}}\Cendnote{\textnormal{Die zweite Ziffer der Monatsangabe von
                     \textcolor{blue}{Schnitzler}s Datierung könnte auch als »2«
                  gelesen werden. Das lässt sich aber durch den Inhalt ausschließen, da der
                     19. 12. 1899 ein Dienstag war und die Unterscheidung zwischen
                  »morgen«/»Dienstag« keinen Sinn ergäbe.}}}\label{K_L03302-1h} bringe ich Ihnen \label{K_L03302-2v}\edtext{»\textcolor{green}{Boubouroch}{}\ledrightnote{\textcolor{green}{Boubouroche. Lebensbild in 2 Acten}}.«}{\lemma{\textnormal{\emph{»Boubouroch.«}}}\Cendnote{\textnormal{Es dürfte sich um
                  eine französischsprachige Ausgabe von \textcolor{blue}{Georges
                     Courteline}s \emph{\textcolor{green}{Boubouroche}} gehandelt
                  haben. Durch die zeitliche Nähe zur deutschsprachigen Premiere am 31. 1. 1900 im \textcolor{pink}{Raimundtheater} wäre auch ein Bühnenmanuskript
                  der \textcolor{green}{Bearbeitung} von \textcolor{blue}{Siegfried Trebitsch} denkbar. \textcolor{blue}{Schnitzler} besuchte wenige Wochen später die
                  Premiere.}}}\label{K_L03302-2h} Ich glaube, die practischen Zwecke zu kennen, u. wenn ich mich
               nicht irre, sind sie sehr gut. In den \label{K_L03302-3v}\edtext{\textcolor{brown}{Club}{}\ledrightnote{{$\rightarrow$}\textcolor{brown}{Wiener Schachclub}}}{\lemma{\textnormal{\emph{Club}}}\Cendnote{\textnormal{Gemeint ist der \emph{\textcolor{brown}{Wiener Schachclub}}, dem \textcolor{blue}{Schnitzler} und \textcolor{blue}{Beer-Hofmann} in
                  diesen Tagen beitraten. \textcolor{blue}{Paul Naschauer} dürfte
                  die Verbindungsperson zum \textcolor{brown}{Club} gewesen sein.}}}\label{K_L03302-3h} trete ich ein, wenn die Anmeldung collectiv
               erfolgt, und wenn die \label{K_L03302-4v}\edtext{60 fl. honoris causa}{\lemma{\textnormal{\emph{60 fl. honoris causa}}}\Cendnote{\textnormal{Die Beitrittsgebühr entspricht 
                  im Jahr 2024 einem Wert von 1.000 €.}}}\label{K_L03302-4h} nachgelaßen {\pb}werden. In diesem Fall, bitte,
               melden Sie mich an, da ich ja doch \textcolor{blue}{Naschauer}{}\ledrightnote{\textcolor{blue}{Paul Naschauer}}
               nicht so bald spreche.\pend
           
\pstart
           Auf \label{K_L03302-5v}\edtext{Wiedersehen also morgen}{\lemma{\textnormal{\emph{Wiedersehen also morgen}}}\Cendnote{\textnormal{Im \emph{\textcolor{green}{Tagebuch}} wird für den 19. 11. 1899 kein Treffen vermerkt.}}}\label{K_L03302-5h}, \label{K_L03302-6v}\edtext{längstens Dienstag}{\lemma{\textnormal{\emph{längstens Dienstag}}}\Cendnote{\textnormal{Ebenfalls nicht im \emph{\textcolor{green}{Tagebuch}} erwähnt. Eventuell sahen sie sich am Dienstag,
                  dem 21. 11. 1899 im
                  Konzert von \textcolor{blue}{Clemens von
                  Franckenstein}.}}}\label{K_L03302-6h}, {\\[\baselineskip]}herzlichst {\\[\baselineskip]}Ihr {\\[\baselineskip]}\spacefill\mbox{Salten}\pend
           \leftskip=0em{}\endnumbering\briefempfaengerindex{Schnitzler, Arthur@\textsc{Schnitzler, Arthur}!zzzSalten, Felix@\emph{von Felix Salten}!1899-11-182@{{[}18. 11. 1899{]}}|)be}\mylabel{h}  \normalsize

\doendnotes{C}
\bigskip
\vfill

\clearpage

\footnotesize

\lohead{\textsc{register}}

% Definiere theindex-Environment komplett neu ohne reledmac
\makeatletter
\renewenvironment{theindex}{%
  \section*{\indexname}%
  \setlength{\parindent}{0pt}%
  \setlength{\parskip}{0pt plus 0.3pt}%
  \let\item\@idxitem
}{%
  \clearpage
}
\makeatother

\IfFileExists{\jobname-pw.ind}{\input{\jobname-pw.ind}}{}

\end{document}

      