%% latex-korrekturansicht-vorspann.tex
%% Vorspann für die Korrekturansicht.
%% Lädt die gemeinsame Datei latex-vorspann.tex mit gesetztem Schalter.

\newif\ifkorrekturansicht
\korrekturansichttrue

\input{../tex-inputs/latex-vorspann}


\section[ Felix Salten an Arthur Schnitzler, {[}18. 11. 1899{]}]{L03302 Felix Salten an Arthur Schnitzler,  {[}18. 11. 1899{]}}
\nopagebreak\mylabel{L03302v}
\rehead{ }\normalsize\beginnumbering\briefempfaengerindex{Schnitzler, Arthur@\textsc{Schnitzler, Arthur}!zzzSalten, Felix@\emph{von Felix Salten}!1899-11-182@{{[}18. 11. 1899{]}}|(be}
\toendnotes[C]{\smallbreak\pagebreak[2]}
\correspDesc{Versand  durch Felix Salten am [18. 11. 1899] in Wien
\newline{}Erhalt  durch Arthur Schnitzler im Zeitraum [18. 11. 1899 – 22. 11. 1899?] in Wien}\toendnotes[C]{\smallbreak}
\Standort{CUL, Schnitzler, B 89, A 2.}
\physDesc{Brief, 1 Blatt, 2 Seiten, 428 Zeichen
\newline{}Handschrift: Bleistift, lateinische Kurrent
\newline{}Schnitzler: mit Bleistift datiert: »18/1\textcolor{gray}{1} 99.« 
\newline{}Ordnung: mit Bleistift von unbekannter Hand nummeriert: »126« }\toendnotes[C]{\smallbreak}
\pstart{}{\pb}Lieber Freund,\pend\vspace{0.5em}
\pstart
           \label{K_L03302-1v}\edtext{Morgen, längstens Dienstag}{\lemma{\textnormal{\emph{Morgen, … Dienstag}}}\Cendnote{\textnormal{Die zweite Ziffer der Monatsangabe von
                     \textcolor{blue}{Schnitzlers} Datierung könnte auch als »2«
                  gelesen werden. Das lässt sich aber durch den Inhalt ausschließen, da 
                  \textcolor{blue}{Schnitzler} ab dem 5. 12. 1899
                  bereits Mitglied des \emph{\textcolor{brown}{Wiener Schachclubs}\orgindex{Wiener Schachclub@Wiener Schachclub|pwk}} gewesen sein dürfte. Auch war der
                     19. 12. 1899 ein Dienstag und somit ergäbe die Unterscheidung zwischen
                  »morgen«/»Dienstag« keinen Sinn.}}}\label{K_L03302-1} bringe ich Ihnen \label{K_L03302-2v}\edtext{»\textcolor{green}{Boubouroch}\pwindex{Courteline, Georges 25.\,6.\,1858 Tours – 25.\,6.\,1929 Paris@\textsc{Courteline, Georges} (25.\,6.\,1858 Tours – 25.\,6.\,1929 Paris), \emph{Schriftsteller}!Boubouroche. Lebensbild in 2 Acten@\strich\emph{Boubouroche. Lebensbild in 2 Acten}|pw}{}\ledrightnote{\textcolor{green}{Boubouroche. Lebensbild in 2 Acten}}.«}{\lemma{\textnormal{\emph{»Boubouroch.«}}}\Cendnote{\textnormal{Es dürfte sich um
                  eine französischsprachige Ausgabe von \textcolor{blue}{Georges
                     Courtelines}\pwindex{Courteline, Georges 25.\,6.\,1858 Tours – 25.\,6.\,1929 Paris@\textsc{Courteline, Georges} (25.\,6.\,1858 Tours – 25.\,6.\,1929 Paris), \emph{Schriftsteller}|pwk}{ }\emph{\textcolor{green}{Boubouroche}\pwindex{Courteline, Georges 25.\,6.\,1858 Tours – 25.\,6.\,1929 Paris@\textsc{Courteline, Georges} (25.\,6.\,1858 Tours – 25.\,6.\,1929 Paris), \emph{Schriftsteller}!Boubouroche. Pièce en deux actes@\strich\emph{Boubouroche. Pièce en deux actes}|pwk}} gehandelt
                  haben. Durch die zeitliche Nähe zur deutschsprachigen Premiere am 31. 1. 1900 im \textcolor{pink}{Raimundtheater}\oindex{Wien@\textbf{Wien}!VI., Mariahilf@\textbf{VI., Mariahilf}!Raimund-Theater@\textbf{Raimund-Theater}, \emph{Theater}|pwk} wäre auch ein Bühnenmanuskript
                  der \textcolor{green}{Bearbeitung}\pwindex{Courteline, Georges 25.\,6.\,1858 Tours – 25.\,6.\,1929 Paris@\textsc{Courteline, Georges} (25.\,6.\,1858 Tours – 25.\,6.\,1929 Paris), \emph{Schriftsteller}!Boubouroche. Lebensbild in 2 Acten@\strich\emph{Boubouroche. Lebensbild in 2 Acten}|pwkv} von \textcolor{blue}{Siegfried Trebitsch}\pwindex{Trebitsch, Siegfried 22.\,12.\,1868 Wien – 3.\,6.\,1956 Zürich@\textsc{Trebitsch, Siegfried} (22.\,12.\,1868 Wien – 3.\,6.\,1956 Zürich), \emph{Schriftsteller, Übersetzer}|pwk} denkbar. \textcolor{blue}{Schnitzler} besuchte wenige Wochen später die
                     \textcolor{violet}{Premiere}\eventindex{Raimund-Theater@\textbf{Raimund-Theater}!Premiere von Das dritte Kind, Boubouroche, Der Herr Gegencandidat, 31.1.1900@Premiere von Das dritte Kind, Boubouroche, Der Herr Gegencandidat, 31.1.1900|pwkv}, vgl. A. S.: \emph{Kulturveranstaltungen}, 31. 1. 1900.}}}\label{K_L03302-2} Ich glaube, die practischen Zwecke zu kennen, u. wenn ich mich
               nicht irre, sind sie sehr gut. In den \label{K_L03302-3v}\edtext{\textcolor{brown}{Club}\orgindex{Wiener Schachclub@Wiener Schachclub|pwv}{}\ledrightnote{{$\rightarrow$}\emph{\textcolor{brown}{Wiener Schachclub}}}}{\lemma{\textnormal{\emph{Club}}}\Cendnote{\textnormal{Gemeint ist der \emph{\textcolor{brown}{Wiener Schachclub}\orgindex{Wiener Schachclub@Wiener Schachclub|pwk}}, dem in den kommenden Wochen neben \textcolor{blue}{Salten}\pwindex{Salten, Felix 6.\,9.\,1869 Budapest – 8.\,10.\,1945 Zürich@\textsc{Salten, Felix} (6.\,9.\,1869 Budapest – 8.\,10.\,1945 Zürich), \emph{Schriftsteller, Journalist, Chefredakteur}|pwk} und \textcolor{blue}{Schnitzler} auch \textcolor{blue}{Beer-Hofmann}\pwindex{Beer-Hofmann, Richard 11.\,7.\,1866 Wien – 26.\,9.\,1945 New York City@\textsc{Beer-Hofmann, Richard} (11.\,7.\,1866 Wien – 26.\,9.\,1945 New York City), \emph{Schriftsteller}|pwk} und
                     \textcolor{blue}{Hofmannsthal}\pwindex{Hofmannsthal, Hugo von 1.\,2.\,1874 Wien – 15.\,7.\,1929 Rodaun@\textsc{Hofmannsthal, Hugo von} (1.\,2.\,1874 Wien – 15.\,7.\,1929 Rodaun), \emph{Schriftsteller}|pwk} beitraten. Am
                     1. 1. 1900 brachte die \emph{\textcolor{green}{Wiener Schachzeitung}\pwindex{Wiener Schachzeitung@\emph{Wiener Schachzeitung}|pwk}} ihre Namen als bei der letzten Sitzung neu
                  aufgenommene Mitglieder. \textcolor{blue}{Paul Naschauer}\pwindex{Naschauer, Paul 6.\,9.\,1866 Baden bei Wien – 20.\,5.\,1900 Wien@\textsc{Naschauer, Paul} (6.\,9.\,1866 Baden bei Wien – 20.\,5.\,1900 Wien)|pwk} und
                     \textcolor{blue}{Leo van Jung}\pwindex{Van-Jung, Leo 15.\,10.\,1866 Odessa – 2.\,7.\,1939 Riga@\textsc{Van-Jung, Leo} (15.\,10.\,1866 Odessa – 2.\,7.\,1939 Riga), \emph{Gesangspädagoge, Mathematiker}|pwk} wurden im gleichen Blatt am
                     12. 12. 1899 als neue Mitglieder verlautbart,
                  dürften also die Verbindungspersonen zum \textcolor{brown}{Club}\orgindex{Wiener Schachclub@Wiener Schachclub|pwkv} gewesen sein.}}}\label{K_L03302-3} trete ich ein, wenn die
               Anmeldung collectiv erfolgt, und wenn die \label{K_L03302-4v}\edtext{60 fl. honoris causa}{\lemma{\textnormal{\emph{60 fl. honoris causa}}}\Cendnote{\textnormal{Die Beitrittsgebühr entspricht im Jahr 2024 einem Wert von
                  1000 €.}}}\label{K_L03302-4} nachgelaßen {\pb}werden. In diesem Fall, bitte, melden Sie mich an, da ich ja doch \textcolor{blue}{Naschauer}\pwindex{Naschauer, Paul 6.\,9.\,1866 Baden bei Wien – 20.\,5.\,1900 Wien@\textsc{Naschauer, Paul} (6.\,9.\,1866 Baden bei Wien – 20.\,5.\,1900 Wien)|pw}{}\ledrightnote{\textcolor{blue}{Paul Naschauer}} nicht so bald spreche.\pend
           
\pstart
           Auf \label{K_L03302-5v}\edtext{Wiedersehen also morgen}{\lemma{\textnormal{\emph{Wiedersehen also morgen}}}\Cendnote{\textnormal{Im \emph{\textcolor{green}{Tagebuch}\pwindex{Schnitzler, Arthur 15. 5. 1862 Wien – 21. 10. 1931 ebd.@\textsc{Schnitzler, Arthur} (15. 5. 1862 Wien – 21. 10. 1931 ebd.), \emph{Schriftsteller, Mediziner}!Tagebuch@\strich\emph{Tagebuch}|pwk}} wird für den 19. 11. 1899 kein Treffen vermerkt. Eventuell aber wollte man sich bei
                     \textcolor{blue}{Beer-Hofmann}\pwindex{Beer-Hofmann, Richard 11.\,7.\,1866 Wien – 26.\,9.\,1945 New York City@\textsc{Beer-Hofmann, Richard} (11.\,7.\,1866 Wien – 26.\,9.\,1945 New York City), \emph{Schriftsteller}|pwk} treffen, vgl. Hugo von Hofmannsthal an Arthur Schnitzler, [19. 11. 1899?].}}}\label{K_L03302-5}, \label{K_L03302-6v}\edtext{längstens Dienstag}{\lemma{\textnormal{\emph{längstens Dienstag}}}\Cendnote{\textnormal{Ebenfalls nicht im \emph{\textcolor{green}{Tagebuch}\pwindex{Schnitzler, Arthur 15. 5. 1862 Wien – 21. 10. 1931 ebd.@\textsc{Schnitzler, Arthur} (15. 5. 1862 Wien – 21. 10. 1931 ebd.), \emph{Schriftsteller, Mediziner}!Tagebuch@\strich\emph{Tagebuch}|pwk}} erwähnt. Eventuell sahen sie sich am Dienstag,
                  dem 21. 11. 1899, im
                  Konzert von \textcolor{blue}{Clemens von Franckenstein}\pwindex{Franckenstein, Clemens von 14.\,7.\,1875 Wiesentheid – 19.\,8.\,1942 Hechendorf am Pilsensee@\textsc{Franckenstein, Clemens von} (14.\,7.\,1875 Wiesentheid – 19.\,8.\,1942 Hechendorf am Pilsensee), \emph{Theaterleiter, Komponist, Dirigent}|pwk}, das
                     \textcolor{blue}{Schnitzler} besuchte.}}}\label{K_L03302-6},
               {\\[\baselineskip]}herzlichst {\\[\baselineskip]}Ihr {\\[\baselineskip]}\spacefill\mbox{Salten}\pend
           \leftskip=0em{}\selectlanguage{ngerman}\endnumbering\briefempfaengerindex{Schnitzler, Arthur@\textsc{Schnitzler, Arthur}!zzzSalten, Felix@\emph{von Felix Salten}!1899-11-182@{{[}18. 11. 1899{]}}|)be}\mylabel{L03302h}  \newcommand{\dateiname}{L03302}\newcommand{\titel}{Felix Salten an Arthur Schnitzler, [18. 11. 1899]}\newcommand{\editorInnen}{Martin Anton Müller und Laura Untner}\normalsize

\doendnotes{C}
\bigskip
\vfill

\clearpage

\footnotesize

\lohead{\textsc{register}}

% Definiere theindex-Environment komplett neu ohne reledmac
\makeatletter
\renewenvironment{theindex}{%
  \section*{\indexname}%
  \setlength{\parindent}{0pt}%
  \setlength{\parskip}{0pt plus 0.3pt}%
  \let\item\@idxitem
}{%
  \clearpage
}
\makeatother

\IfFileExists{\jobname-pw.ind}{\input{\jobname-pw.ind}}{}

\end{document}

