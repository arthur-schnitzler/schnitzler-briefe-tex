%% latex-korrekturansicht-vorspann.tex
%% Vorspann für die Korrekturansicht.
%% Lädt die gemeinsame Datei latex-vorspann.tex mit gesetztem Schalter.

\newif\ifkorrekturansicht
\korrekturansichttrue

\input{../tex-inputs/latex-vorspann}


\renewcommand{\erwaehntePersonen}{Personen: Johannes Nielsen, Olga Schnitzler}
\renewcommand{\erwaehnteOrte}{Orte: Berlin, Dessauer Straße, Det Kongelige Teater, Kopenhagen, Wien}
\renewcommand{\erwaehnteWerke}{Werke: Lebendige Stunden. Vier Einakter, Levende Timer. Skuespil i 1 akt}
\section[ Paul Goldmann an Arthur Schnitzler, 26. 5. {[}1904{]}]{Paul Goldmann an Arthur Schnitzler, 26. 5. {[}1904{]}}
\nopagebreak\mylabel{v}
\rehead{ }\normalsize\beginnumbering\briefempfaengerindex{Schnitzler, Arthur@\textsc{Schnitzler, Arthur}!zzzGoldmann, Paul@\emph{von Paul Goldmann}!1904-05-261@{26. 5. {[}1904{]}}|(be}
\toendnotes[C]{\smallbreak\pagebreak[2]}\Standort{DLA, A:Schnitzler, HS.NZ85.1.3174.}
\physDesc{Brief, 1 Blatt, 2 Seiten
\newline{}Handschrift: blaue Tinte, deutsche Kurrent
\newline{}Schnitzler: mit Bleistift das Jahr »{[}1{]}904« vermerkt }\toendnotes[C]{\smallbreak}
\pstart
           \noindent{}\raggedleft{}{\pb}\textcolor{gray}{\textbf{\textcolor{pink}{DESSAUERSTRASSE 19}{}\ledrightnote{\textcolor{pink}{Dessauer Straße}}}}\pend
           
\pstart
           \textcolor{pink}{Berlin}{}\ledrightnote{\textcolor{pink}{Berlin}}, 26. Mai.\pend
           
\pstart{}Mein lieber Freund,\pend
\pstart
           Deine \label{K_L03444-1v}\edtext{Karten}{\lemma{\textnormal{\emph{Karten}}}\Cendnote{\textnormal{Nachdem er
               zuletzt aus \textcolor{pink}{Rom} eine Karte bekam (vgl. Paul Goldmann an Arthur Schnitzler, 1[6?.] 5. [1904]), 
               dürfte sich der Dank auf eine Karte oder Karten aus \textcolor{pink}{Neapel} oder \textcolor{pink}{Sizilien}
               beziehen.}}}\label{K_L03444-1h} werden immer ſchöner; \strikeout{und} es muß eine
               herrliche Reiſe ſein. Ich danke Dir vielmals, daß Du unterwegs meiner gedenkſt, und
               bedaure nur, daß ich Deine Adreſſe nicht weiß. Hoffentlich erreichen Dich meine nach
                  \textcolor{pink}{Wien}{}\ledrightnote{\textcolor{pink}{Wien}} gerichteten Briefe.\pend
           
\pstart
           Von mir iſt nichts Neues zu berichten. Es geht alles ſeinen alten Gang.\pend
           
\pstart
           Nach Telegrammen aus \textsc{\textcolor{pink}{Kopenhagen}{}\ledrightnote{\textcolor{pink}{Kopenhagen}}}, die ich in \textcolor{pink}{Berlin}{}\ledrightnote{\textcolor{pink}{Berlin}}er Blättern las, ſind die
                  »\textcolor{green}{Lebendigen Stunden}{}\ledrightnote{\textcolor{green}{Lebendige Stunden. Vier Einakter}}« {\pb}\textcolor{pink}{dort}{}\ledrightnote{{$\rightarrow$}\textcolor{pink}{Kopenhagen}} mit großem Erfolg
                  \label{K_L03444-2v}\edtext{aufgeführt}{\lemma{\textnormal{\emph{aufgeführt}}}\Cendnote{\textnormal{\emph{\textcolor{green}{Levende Timer. Skuespil i 1 akt}}
                  (Übersetzung: \textcolor{blue}{Johannes Nielsen}) hatte am
                     19. 5. 1904 am \textcolor{pink}{Kopenhagen}er \textcolor{pink}{Kongelige Teater}
                  Premiere.}}}\label{K_L03444-2h} worden.\pend
           
\pstart
           Viele herzliche Grüße Dir und Deiner \textcolor{blue}{Frau}{}\ledrightnote{{$\rightarrow$}\textcolor{blue}{Olga Schnitzler}} von {\\[\baselineskip]}Deinem getreuen {\\[\baselineskip]}\spacefill\mbox{Paul Goldmann.}\pend
           \leftskip=0em{}\endnumbering\briefempfaengerindex{Schnitzler, Arthur@\textsc{Schnitzler, Arthur}!zzzGoldmann, Paul@\emph{von Paul Goldmann}!1904-05-261@{26. 5. {[}1904{]}}|)be}\mylabel{h}
\begin{anhang}
\end{anhang}\normalsize

\doendnotes{C}
\bigskip
\vfill

\clearpage

\footnotesize

\lohead{\textsc{register}}

% Definiere theindex-Environment komplett neu ohne reledmac
\makeatletter
\renewenvironment{theindex}{%
  \section*{\indexname}%
  \setlength{\parindent}{0pt}%
  \setlength{\parskip}{0pt plus 0.3pt}%
  \let\item\@idxitem
}{%
  \clearpage
}
\makeatother

\IfFileExists{\jobname-pw.ind}{\input{\jobname-pw.ind}}{}

\end{document}

      