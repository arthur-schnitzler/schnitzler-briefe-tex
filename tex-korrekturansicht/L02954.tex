%% latex-korrekturansicht-vorspann.tex
%% Vorspann für die Korrekturansicht.
%% Lädt die gemeinsame Datei latex-vorspann.tex mit gesetztem Schalter.

\newif\ifkorrekturansicht
\korrekturansichttrue

\input{../tex-inputs/latex-vorspann}


\renewcommand{\erwaehntePersonen}{Personen: Richard Beer-Hofmann, Felix Salten}
\renewcommand{\erwaehnteOrte}{Orte: Kärntnerring 12/Bösendorferstraße 11, Wien}
\renewcommand{\erwaehnteWerke}{Werke: Der Schleier der Pierette. Pantomime in drei Bildern, Familie}
\section[Arthur Schnitzler an Felix Salten, {[}13. 6. 1893?{]}]{Arthur Schnitzler an Felix Salten, {[}13. 6. 1893?{]}}
\nopagebreak\mylabel{v}
\rehead{ }\normalsize\beginnumbering\briefempfaengerindex{Salten, Felix@\textsc{Salten, Felix}!zzzSchnitzler, Arthur@\emph{von Arthur Schnitzler}!1893-06-131@{{[}13. 6. 1893?{]}}|(be}
\toendnotes[C]{\smallbreak\pagebreak[2]}\Standort{Wienbibliothek im Rathaus, ZPH 1681, 2.1.516.}
\physDesc{Brief, 1 Blatt, 3 Seiten, 201 Zeichen
\newline{}Handschrift: Bleistift, deutsche Kurrent
\newline{}Ordnung: mit Bleistift von unbekannter Hand Nummerierung der Doppelseiten des
                                 Konvoluts: »29«–»30« }\toendnotes[C]{\smallbreak}
\pstart{}{\pb}Lieber Freund,\pend
\pstart
           das \label{K_L02954-1v}\edtext{\textcolor{green}{Stück}{}\ledrightnote{{$\rightarrow$}\textcolor{green}{Familie}} wird ſchon um
               5 geleſen}{\lemma{\textnormal{\emph{Stück … geleſen}}}\Cendnote{\textnormal{Das Korrespondenzstück ist
                  undatiert. Der Text weist auf eine Lesung eines dramatischen Werks durch \textcolor{blue}{Schnitzler} bei ihm \textcolor{pink}{zuhause} hin. Folgende Annahmen erlauben
                  Einschränkungen vorzunehmen: \textcolor{blue}{Salten} und \textcolor{blue}{Beer-Hofmann} kamen der Einladung nach. Die
                  Lesung fand nicht an einem Abend statt. Die Pantomime, die nachmalig
                  den Titel \emph{\textcolor{green}{Der Schleier der Pierrette}} bekam,
                  war nicht gemeint (vgl. A. S.: \emph{Tagebuch}, 15. 11. 1892). Das grenzt die Datierung auf die Lesung von \emph{\textcolor{green}{Familie}} am 14. 6. 1893 ein. Das Korrespondenzstück lief
                  wahrscheinlich am Vortag.}}}\label{K_L02954-1h}, weil 
                  \textcolor{blue}{\textsc{Beer-Hofma{\geminationn}}}{}\ledrightnote{\textcolor{blue}{Richard Beer-Hofmann}} ins Theater geht. Bitte ſehr, ſeien {\pb}Sie pünktlich bei \textcolor{pink}{mir}{}\ledrightnote{{$\rightarrow$}\textcolor{pink}{Kärntnerring 12/Bösendorferstraße 11}}.
                  We{\geminationn} Sie früher ko{\geminationm}en,
               iſt es mir aber eine ganz ſpecielle Freude.\pend
           
\pstart
           {\pb}Herzlichſt {\\[\baselineskip]}Ihr {\\[\baselineskip]}\spacefill\mbox{ArthSch}\pend
           \leftskip=0em{}\endnumbering\briefempfaengerindex{Salten, Felix@\textsc{Salten, Felix}!zzzSchnitzler, Arthur@\emph{von Arthur Schnitzler}!1893-06-131@{{[}13. 6. 1893?{]}}|)be}\mylabel{h}  \normalsize

\doendnotes{C}
\bigskip
\vfill

\clearpage

\footnotesize

\lohead{\textsc{register}}

% Definiere theindex-Environment komplett neu ohne reledmac
\makeatletter
\renewenvironment{theindex}{%
  \section*{\indexname}%
  \setlength{\parindent}{0pt}%
  \setlength{\parskip}{0pt plus 0.3pt}%
  \let\item\@idxitem
}{%
  \clearpage
}
\makeatother

\IfFileExists{\jobname-pw.ind}{\input{\jobname-pw.ind}}{}

\end{document}

      