%% latex-korrekturansicht-vorspann.tex
%% Vorspann für die Korrekturansicht.
%% Lädt die gemeinsame Datei latex-vorspann.tex mit gesetztem Schalter.

\newif\ifkorrekturansicht
\korrekturansichttrue

\input{../tex-inputs/latex-vorspann}


         
         \renewcommand{\erwaehntePersonen}{Personen: Auguste Chlum}
         \renewcommand{\erwaehnteInstitutionen}{Institutionen: Wiener Verlag}
         \renewcommand{\erwaehnteOrte}{Orte: Berlin, Dessauer Straße, Kleines Schauspielhaus, Wien}
         \renewcommand{\erwaehnteWerke}{Werke: Reigen. Zehn Dialoge}
               \section[ Paul Goldmann an Arthur Schnitzler, 20. 4. {[}1900{]}]{Paul Goldmann an Arthur Schnitzler, 20. 4. {[}1900{]}}\nopagebreak\mylabel{v}\rehead{ }\normalsize\beginnumbering\briefempfaengerindex{Schnitzler, Arthur@\textsc{Schnitzler, Arthur}!zzzGoldmann, Paul@\emph{von Paul Goldmann}!1900-04-202@{20. 4. {[}1900{]}}|(be} \toendnotes[C]{\smallbreak\pagebreak[2]} \Standort{DLA, A:Schnitzler, HS.NZ85.1.3170.}
\physDesc{Brief, 1 Blatt, 2 Seiten
\newline{}Handschrift: blaue Tinte, deutsche Kurrent
\newline{}Schnitzler: 1) mit Bleistift das Jahr »{[}1{]}900« vermerkt  2) mit rotem Buntstift eine Unterstreichung}\toendnotes[C]{\smallbreak}\pstart
           \noindent{}{\pb}\textcolor{pink}{\textcolor{gray}{\textbf{DESSAUERSTRASSE 19}}}{}\ledrightnote{\textcolor{pink}{Dessauer Straße}}\pend
           {\bigskip}\pstart
           \raggedleft{}\textcolor{pink}{Berlin}{}\ledrightnote{\textcolor{pink}{Berlin}}, 20. April.\pend
           \pstart\center{}Mein lieber Freund,\pend\pstart
           Ich danke Dir vielmals für den \label{K_L02912-1v}\edtext{»\textcolor{green}{Reigen}{}\ledrightnote{\textcolor{green}{Reigen. Zehn Dialoge}}«}{\lemma{\textnormal{\emph{»Reigen«}}}\Cendnote{\textnormal{Privatdruck des \emph{\textcolor{green}{Reigen}}
                  in einer Auflage von 200 Stück (\textcolor{green}{Erstausgabe}1903 im \emph{\textcolor{brown}{Wiener
                  Verlag}})}}}\label{K_L02912-1h}. Ich habe es in einem Zuge noch einmal durchgeleſen. Köſtlich,
               köſtlich! Aber die \strikeout{K\textcolor{gray}{ro}} Krone des Ganzen iſt doch die \textcolor{green}{Schauſpielerin}{}\ledrightnote{{$\rightarrow$}\textcolor{green}{Reigen. Zehn Dialoge}}. Eine Figur von unvergleichlicher Komik. Ich
               habe mich geſchüttelt vor Lachen. Wie ſchade, daß dieſes \textcolor{green}{Buch}{}\ledrightnote{{$\rightarrow$}\textcolor{green}{Reigen. Zehn Dialoge}}, das zu Deinen beſten gehört, dem
               Publikum \label{K_L02912-3v}\edtext{nicht bekannt werden
                  ſoll}{\lemma{\textnormal{\emph{nicht … ſoll}}}\Cendnote{\textnormal{vgl. den Skandal rund um die
                  erste vollständige Aufführung (23. 12. 1920, \textcolor{pink}{Kleines
                     Schauspielhaus}, \textcolor{pink}{Berlin}) und den
                  darauffolgenden »\emph{\textcolor{green}{Reigen}}-Prozess«}}}\label{K_L02912-3h}!
               Druck und Ausſtattung ſind vornehm und geſchmackvoll.\pend
           \pstart
           {\pb}Geſtern ſprach ich \textcolor{blue}{\textsc{Gusti Gl}.}{}\ledrightnote{\textcolor{blue}{Auguste Chlum}} und ſagte ihr, daß Du nach ihr gefragt
               haſt. Sie antwortete, ſie ſei jetzt nicht in der Stimmung, aber ſie werde Dir ſchon
               ſchreiben. Sieht übrigens angegriffen und gealtert aus.\pend
           \pstart
           Viele treue Grüße! {\\[\baselineskip]}Dein {\\[\baselineskip]}\spacefill\mbox{Paul Goldmann}\pend
           \leftskip=0em{}\endnumbering\briefempfaengerindex{Schnitzler, Arthur@\textsc{Schnitzler, Arthur}!zzzGoldmann, Paul@\emph{von Paul Goldmann}!1900-04-202@{20. 4. {[}1900{]}}|)be}\mylabel{h}\begin{anhang}\end{anhang}\normalsize

\doendnotes{C}
\bigskip
\vfill

\clearpage

\footnotesize

\lohead{\textsc{register}}

% Definiere theindex-Environment komplett neu ohne reledmac
\makeatletter
\renewenvironment{theindex}{%
  \section*{\indexname}%
  \setlength{\parindent}{0pt}%
  \setlength{\parskip}{0pt plus 0.3pt}%
  \let\item\@idxitem
}{%
  \clearpage
}
\makeatother

\IfFileExists{\jobname-pw.ind}{\input{\jobname-pw.ind}}{}

\end{document}

      