%% latex-korrekturansicht-vorspann.tex
%% Vorspann für die Korrekturansicht.
%% Lädt die gemeinsame Datei latex-vorspann.tex mit gesetztem Schalter.

\newif\ifkorrekturansicht
\korrekturansichttrue

\input{../tex-inputs/latex-vorspann}


\renewcommand{\erwaehntePersonen}{Personen: Samuel Fischer}
\renewcommand{\erwaehnteOrte}{Orte: Wien}
\renewcommand{\erwaehnteWerke}{Werke: Die kleine Veronika, Neue Deutsche Rundschau}
\section[ Felix Salten an Arthur Schnitzler, {[}24. 10. 1902{]}]{Felix Salten an Arthur Schnitzler, {[}24. 10. 1902{]}}
\nopagebreak\mylabel{v}
\rehead{ }\normalsize\beginnumbering\briefempfaengerindex{Schnitzler, Arthur@\textsc{Schnitzler, Arthur}!zzzSalten, Felix@\emph{von Felix Salten}!1902-10-241@{{[}24. 10. 1902{]}}|(be}
\toendnotes[C]{\smallbreak\pagebreak[2]}\Standort{CUL, Schnitzler, B 89, A 2.}
\physDesc{Brief, 1 Blatt, 1 Seite, 157 Zeichen
\newline{}Handschrift: blaue Tinte, lateinische Kurrent
\newline{}Schnitzler: mit Bleistift datiert: »24/X 902« 
\newline{}Ordnung: mit Bleistift von unbekannter Hand nummeriert: »161« }\toendnotes[C]{\smallbreak}
\pstart
           \noindent{}{\pb}Lieber, die \label{K_L03336-1v}\edtext{\textcolor{green}{kl. Veronika}{}\ledrightnote{\textcolor{green}{Die kleine Veronika}}}{\lemma{\textnormal{\emph{kl. Veronika}}}\Cendnote{\textnormal{\textcolor{blue}{Felix Salten}: \emph{\textcolor{green}{Die kleine Veronika}}. In: \emph{\textcolor{green}{Neue Deutsche Rundschau}}, Jg. 13, Nr. 12, Dezember 1902, S. 1285–1333.}}}\label{K_L03336-1h} erscheint
               also am 1. Dezember in der »\textcolor{green}{N. D. R}{}\ledrightnote{\textcolor{green}{Neue Deutsche Rundschau}}« Eben theilt es mir \textcolor{blue}{Fischer}{}\ledrightnote{\textcolor{blue}{Samuel Fischer}} mit. Ich freue mich aufrichtig und \label{K_L03336-2v}\edtext{danke Ihnen}{\lemma{\textnormal{\emph{danke Ihnen}}}\Cendnote{\textnormal{siehe Arthur Schnitzler an Felix Salten, 16. 10. 1902}}}\label{K_L03336-2h} herzlichst.\pend
           
\pstart
           Ihr {\\[\baselineskip]}\spacefill\mbox{Salten}\pend
           \leftskip=0em{}\endnumbering\briefempfaengerindex{Schnitzler, Arthur@\textsc{Schnitzler, Arthur}!zzzSalten, Felix@\emph{von Felix Salten}!1902-10-241@{{[}24. 10. 1902{]}}|)be}\mylabel{h}  \normalsize

\doendnotes{C}
\bigskip
\vfill

\clearpage

\footnotesize

\lohead{\textsc{register}}

% Definiere theindex-Environment komplett neu ohne reledmac
\makeatletter
\renewenvironment{theindex}{%
  \section*{\indexname}%
  \setlength{\parindent}{0pt}%
  \setlength{\parskip}{0pt plus 0.3pt}%
  \let\item\@idxitem
}{%
  \clearpage
}
\makeatother

\IfFileExists{\jobname-pw.ind}{\input{\jobname-pw.ind}}{}

\end{document}

      