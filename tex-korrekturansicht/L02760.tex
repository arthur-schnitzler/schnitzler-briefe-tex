%% latex-korrekturansicht-vorspann.tex
%% Vorspann für die Korrekturansicht.
%% Lädt die gemeinsame Datei latex-vorspann.tex mit gesetztem Schalter.

\newif\ifkorrekturansicht
\korrekturansichttrue

\input{../tex-inputs/latex-vorspann}


               \section[Paul Goldmann an Arthur Schnitzler, 21. 12. {[}1895{]}]{ Paul Goldmann an Arthur Schnitzler, 21. 12. {[}1895{]}}\nopagebreak\mylabel{v}\rehead{ }\normalsize\beginnumbering\briefempfaengerindex{Schnitzler, Arthur@\textsc{Schnitzler, Arthur}!zzzGoldmann, Paul@\emph{von Paul Goldmann}!1895-12-211@{21. 12. {[}1895{]}}|(be} \toendnotes[C]{\smallbreak\pagebreak[2]} \Standort{DLA, A:Schnitzler, HS.NZ85.1.3165.}
\physDesc{Brief, 1 Blatt, 4 Seiten
\newline{}Handschrift: blaue Tinte, deutsche Kurrent
\newline{}Schnitzler: mit Bleistift das Jahr » 95« vermerkt }\toendnotes[C]{\smallbreak}\pstart
           \noindent{}{\pb}\textcolor{gray}{\textbf{\textbf{\textcolor{brown}{Frankfurter Zeitung}{}\ledrightnote{\textcolor{brown}{Frankfurter Zeitung}}}}}\pend
           \pstart
           \textcolor{gray}{\textbf{(\textcolor{brown}{\begin{otherlanguage}{french}Gazette de Francfort\end{otherlanguage}}{}\ledrightnote{\textcolor{brown}{Frankfurter Zeitung}}). }}\pend
           \pstart
           \textcolor{gray}{\textbf{\textbf{\begin{otherlanguage}{french}Fondateur M. \textcolor{blue}{L.
                              Sonnemann}{}\ledrightnote{\textcolor{blue}{Leopold Sonnemann}}\end{otherlanguage}.}}}\pend
           \pstart
           \begin{otherlanguage}{french}\textcolor{gray}{\textbf{\textcolor{green}{Journal}{}\ledrightnote{→\textcolor{green}{Frankfurter Zeitung}} politique, financier,}}\end{otherlanguage}\pend
           \pstart
           \begin{otherlanguage}{french}\textcolor{gray}{\textbf{commercial et littéraire.}}\end{otherlanguage}\pend
           \pstart
           \begin{otherlanguage}{french}\textcolor{gray}{\textbf{\textbf{Paraissant trois fois par jour.}}}\end{otherlanguage}\hfill \textsc{\textcolor{pink}{Paris}{}\ledrightnote{\textcolor{pink}{Paris}}}, 21. December.\pend
           \pstart
           \begin{otherlanguage}{french}\textcolor{gray}{\textbf{\textbf{Bureau à \textcolor{pink}{Paris}{}\ledrightnote{\textcolor{pink}{Paris}}:}}}\end{otherlanguage}\pend
           \pstart
           \begin{otherlanguage}{french}\textcolor{gray}{\textbf{\textbf{\textcolor{pink}{24. Rue Feydeau}{}\ledrightnote{\textcolor{pink}{rue Feydeau}}.}}}\end{otherlanguage}\pend
           {\bigskip}\pstart
           Schöne Geſchichte, mein lieber Freund! Ich bekomme
               eben Deinen Brief, die Viſitkarte iſt darin, das Geld iſt herausgenommen. Auf dem
               Umſchlag iſt ein Vermerk der \textcolor{brown}{franzöſiſchen Poſt}{}\ledrightnote{\textcolor{brown}{Französische Post}}
               zu leſen, daß der Brief mit einer Öffnung von 2 Centimeter angekommen iſt, welche
               Öffnung die \textcolor{brown}{Poſt}{}\ledrightnote{\textcolor{brown}{Französische Post}} gewiſſenhaft \strikeout{verklebt} verklebt hat – nachdem das Geld herausgenommen
               worden. Zu machen iſt da kaum etwas. Ich richte ſofort eine Reclamation an die \textcolor{brown}{franzöſiſche Poſt}{}\ledrightnote{\textcolor{brown}{Französische Post}}, wozu ich das Couvert brauche
               (ſonſt hätte ich dirs geſchickt). {\pb}Du ſebſt haſt
               hoffentlich ſofort auf Grund meiner Depeſche reclamirt. Nützen wird es nichts; Gott
               weiß, wo in \textcolor{pink}{Europa}{}\ledrightnote{\textcolor{pink}{Europa}} das Geld ſich jetzt
               herumtreibt. Die \textcolor{brown}{Poſt}{}\ledrightnote{\textcolor{brown}{Französische Post}} iſt nicht haftbar; denn das
               Geld war nicht declarirt, und der Brief, wofür ſie einzig haftet, iſt angekommen.
               Frage immerhin einen Advokaten, ob man nicht auf Grund der von der \textcolor{brown}{Poſt}{}\ledrightnote{\textcolor{brown}{Französische Post}} ſelbſt conſtatirten \uline{Beſchädigung} des Briefes einen Schadens-Anſpruch erheben kann. {\pb}Aber, Kind, welche Unvorſichtigkeit! 3 Goldſtücke im
               einfachen Couvert! Das \uline{muß} man ja ſtehlen. Ich ſelbſt
               würde es ſtehlen, wenn ich Poſtbeamter wäre. Warum haſt Du mir keine Poſtanweiſung
               geſchickt? Das wäre ſogar noch billiger geweſen.\pend
           \pstart
           Ich ärgere mich furchtbar\strikeout{,} und ich denke nach, ob ich
               nicht irgendwie daran ſchuld bin, – aber nein, ich glaube nicht.\pend
           \pstart
           {\pb}Was nun?\pend
           \pstart
           Viele treue Grüße! {\\[\baselineskip]}Dein{\\[\baselineskip]}\spacefill\mbox{Paul Goldmann.}\pend
           \leftskip=0em{}\endnumbering\briefempfaengerindex{Schnitzler, Arthur@\textsc{Schnitzler, Arthur}!zzzGoldmann, Paul@\emph{von Paul Goldmann}!1895-12-211@{21. 12. {[}1895{]}}|)be}\mylabel{h}  \normalsize

\doendnotes{C}
\bigskip
\vfill

\clearpage

\footnotesize

\lohead{\textsc{register}}

% Definiere theindex-Environment komplett neu ohne reledmac
\makeatletter
\renewenvironment{theindex}{%
  \section*{\indexname}%
  \setlength{\parindent}{0pt}%
  \setlength{\parskip}{0pt plus 0.3pt}%
  \let\item\@idxitem
}{%
  \clearpage
}
\makeatother

\IfFileExists{\jobname-pw.ind}{\input{\jobname-pw.ind}}{}

\end{document}

      