%% latex-korrekturansicht-vorspann.tex
%% Vorspann für die Korrekturansicht.
%% Lädt die gemeinsame Datei latex-vorspann.tex mit gesetztem Schalter.

\newif\ifkorrekturansicht
\korrekturansichttrue

\input{../tex-inputs/latex-vorspann}


\renewcommand{\erwaehntePersonen}{Personen: A. Courth, Guglielmo Ferrero, Hans Kurella, Cesare Lombroso, August Strindberg}
\renewcommand{\erwaehnteInstitutionen}{Institutionen: Philipp Reclam jun.}
\renewcommand{\erwaehnteOrte}{Orte: Leipzig, Wien}
\renewcommand{\erwaehnteWerke}{Werke: Das Weib als Verbrecherin und Prostituierte, Genie und Irrsinn in ihren Beziehungen zum Gesetz, zur Kritik und zur Geschichte, La donna delinquente: La prostituta e la donna normale}
\section[Felix Salten an Arthur Schnitzler, {[}28. 7. 1894{]}]{Felix Salten an Arthur Schnitzler, {[}28. 7. 1894{]}}
\nopagebreak\mylabel{v}
\rehead{ }\normalsize\beginnumbering\briefempfaengerindex{Schnitzler, Arthur@\textsc{Schnitzler, Arthur}!zzzSalten, Felix@\emph{von Felix Salten}!1894-07-281@{{[}28. 7. 1894{]}}|(be}
\toendnotes[C]{\smallbreak\pagebreak[2]}\Standort{CUL, Schnitzler, B 89, A 1.}
\physDesc{Brief, 1 Blatt, 1 Seite, 236 Zeichen
\newline{}Handschrift: Bleistift, lateinische Kurrent
\newline{}Schnitzler: mit Bleistift datiert: »\uline{28/7 94}.« 
\newline{}Ordnung: mit Bleistift von unbekannter Hand nummeriert: »42« }\toendnotes[C]{\smallbreak}
\pstart
           \noindent{}{\pb}Lieber Freund, bitte können Sie mir jenes \label{K_L03141-1v}\edtext{\textcolor{green}{Buch}{}\ledrightnote{{$\rightarrow$}\textcolor{green}{Genie und Irrsinn in ihren Beziehungen zum Gesetz, zur Kritik und zur Geschichte}} von \textcolor{blue}{Lombroso}{}\ledrightnote{\textcolor{blue}{Cesare Lombroso}}, das von Verbrecher {\kaufmannsund}
               Irrsinn handelt}{\lemma{\textnormal{\emph{Buch … handelt}}}\Cendnote{\textnormal{\textcolor{blue}{Cesare Lombroso}: \emph{\textcolor{green}{Genie und Irrsinn in ihren Beziehungen zum Gesetz, zur
                        Kritik und zur Geschichte}}. Übersetzt von \textcolor{blue}{A. Courth}. \textcolor{pink}{Leipzig}: \emph{\textcolor{brown}{Reclam}}{ }1887.}}}\label{K_L03141-1h}, nebst \uline{dem \label{K_L03141-2v}\edtext{\textcolor{green}{neuen Werk}{}\ledrightnote{{$\rightarrow$}\textcolor{green}{Das Weib als Verbrecherin und Prostituierte}}}{\lemma{\textnormal{\emph{neuen Werk}}}\Cendnote{\textnormal{Eben war die Übersetzung des
                     gemeinsam mit \textcolor{blue}{Guglielmo Ferrero}
                     verfassten Werkes \emph{\textcolor{green}{La donna delinquente: La
                        prostituta e la donna normale}} (1893)
                     erschienen: \emph{\textcolor{green}{Das Weib als Verbrecherin
                           und Prostituierte. Anthropologische Studien, gegründet auf eine
                           Darstellung der Biologie und Psychologie des normalen Weibes}}.
                        Autorisierte Übersetzung von \textcolor{blue}{H.
                           Kurella}. Hamburg: \emph{Verlagsanstalt und
                           Druckerei A.G. (Vorm. J. F. Richter)}{ }1894. \textcolor{blue}{Strindberg} wird darin nicht erwähnt. Im Hinblick auf die
                     Vorgeschichte (vgl. Felix Salten an Arthur Schnitzler, 10. 8. 1892)
                     überrascht \textcolor{blue}{Salten}s
                     Unbekümmertheit, \textcolor{blue}{Schnitzler} um
                     Bücher von \textcolor{blue}{Lombroso} zu bitten.}}}\label{K_L03141-2h},
                  in welchem \textcolor{blue}{Strindberg}{}\ledrightnote{\textcolor{blue}{August Strindberg}} als wahnsinnig}
               bezeichnet wird, auf ein paar Tage leihen? Ich habe mich auf beides zu beziehen.\pend
           \pstart Ihr \spacefill\mbox{Salten}\pend{}\endnumbering\briefempfaengerindex{Schnitzler, Arthur@\textsc{Schnitzler, Arthur}!zzzSalten, Felix@\emph{von Felix Salten}!1894-07-281@{{[}28. 7. 1894{]}}|)be}\mylabel{h}  \normalsize

\doendnotes{C}
\bigskip
\vfill

\clearpage

\footnotesize

\lohead{\textsc{register}}

% Definiere theindex-Environment komplett neu ohne reledmac
\makeatletter
\renewenvironment{theindex}{%
  \section*{\indexname}%
  \setlength{\parindent}{0pt}%
  \setlength{\parskip}{0pt plus 0.3pt}%
  \let\item\@idxitem
}{%
  \clearpage
}
\makeatother

\IfFileExists{\jobname-pw.ind}{\input{\jobname-pw.ind}}{}

\end{document}

      