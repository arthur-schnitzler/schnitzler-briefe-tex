%% latex-korrekturansicht-vorspann.tex
%% Vorspann für die Korrekturansicht.
%% Lädt die gemeinsame Datei latex-vorspann.tex mit gesetztem Schalter.

\newif\ifkorrekturansicht
\korrekturansichttrue

\input{../tex-inputs/latex-vorspann}


               \section[Mirjam Beer-Hofmann an Arthur Schnitzler, 8. 6. 1927]{ Mirjam Beer-Hofmann an Arthur Schnitzler,
                    8. 6. 1927}\nopagebreak\mylabel{v}\rehead{ }\normalsize\beginnumbering\briefempfaengerindex{Schnitzler, Arthur@\textsc{Schnitzler, Arthur}!zzzBeer-Hofmann, Mirjam@\emph{von Mirjam Beer-Hofmann}!1927-06-081@{8. 6. 1927}|(be} \toendnotes[C]{\smallbreak\pagebreak[2]} \Standort{CUL, Schnitzler, B 8.}
\physDesc{Brief, 1 Blatt, 3 Seiten
\newline{}Handschrift: schwarze Tinte, lateinische Kurrent
\newline{}Schnitzler: 1) mit Bleistift beschriftet: »\textsc{BH Mirjam}« 2) mit rotem Buntstift mehrere Unterstreichungen\newline{}Ordnung: mit Bleistift von unbekannter Hand
                                    nummeriert: »273« }\buchAbdrucke{\weitereDrucke{Arthur Schnitzler, Richard Beer-Hofmann: \emph{Briefwechsel 1891–1931}. Hg. Konstanze Fliedl. Wien, Zürich: \emph{Europaverlag} 1992, S. 230.} }\toendnotes[C]{\smallbreak}\pstart
           \raggedleft{}{\pb}\textcolor{pink}{Berlin}{}\ledrightnote{\textcolor{pink}{Berlin}}{ }8. 6. 27\pend
           \pstart{}Liebster Arthur!\pend\pstart
           So war ich wieder in \textcolor{pink}{Wien}{}\ledrightnote{\textcolor{pink}{Wien}} und Du warst nicht da
                    und wenn Du in \textcolor{pink}{Berlin}{}\ledrightnote{\textcolor{pink}{Berlin}} bist, hab’ ich Dich auch
                    nur höchstens die Rückfahrt von \textcolor{blue}{Michaelis}{}\ledrightnote{\textcolor{blue}{Dora Michaelis}{\newline}\textcolor{blue}{Karl Michaelis}} und das sind höchstens 15 Minuten. Was soll man da machen?
                    Und ich hätte Dir oft viel zu sagen und will es Dir halt jetzt schreiben. Ich
                    freue {\pb}mich sehr, dass \textcolor{blue}{Lily}{}\ledrightnote{\textcolor{blue}{Lili Schnitzler}} heiratet und Du damit zufrieden bist und
                    ihren \textcolor{blue}{Mann}{}\ledrightnote{→\textcolor{blue}{Arnoldo Cappellini}} gern hast. Das
                    hat man mir erzählt und zwar von glaubwürdiger Stelle, so dass ich es annehme
                    und Dir doch darüber schreiben darf. Weisst Du, es ist sehr gut, wenn man sehr
                    jung heiratet, es bleibt einem unendlich viel erspart. Ich weiss zwar nicht,
                    wann \textcolor{blue}{Lily}{}\ledrightnote{\textcolor{blue}{Lili Schnitzler}} heiratet, jedenfalls {\pb}aber sag’ ihr schon heute
                    viel Liebes von mir. Und Dir wünsch’ ich i{\geminationm}er, auch
                    ohne Gelegenheit nur viel Schönes und Frohes.\pend
           \pstart
           Ko{\geminationm}st Du nicht wieder nach \textcolor{pink}{Berlin}{}\ledrightnote{\textcolor{pink}{Berlin}}?\pend
           \pstart
           Innigst{\\[\baselineskip]}Deine{\\[\baselineskip]}\spacefill\mbox{Mirjam}\pend
           \leftskip=0em{}\pstart
           \noindent{}Viele herzliche Grüsse und Wünsche von meinem \textcolor{blue}{Mann}{}\ledrightnote{→\textcolor{blue}{Ernst Lens}}.\pend
           \pstart
           Der Brief ist nur für Dich, denn Dir gegenüber bin ich doch nie erwachsen \textcolor{gray}{u}
                        geniere mich daher Dir zu sagen, wie lieb ich Dich habe.\pend
           \endnumbering\briefempfaengerindex{Schnitzler, Arthur@\textsc{Schnitzler, Arthur}!zzzBeer-Hofmann, Mirjam@\emph{von Mirjam Beer-Hofmann}!1927-06-081@{8. 6. 1927}|)be}\mylabel{h}  \normalsize

\doendnotes{C}
\bigskip
\vfill

\clearpage

\footnotesize

\lohead{\textsc{register}}

% Definiere theindex-Environment komplett neu ohne reledmac
\makeatletter
\renewenvironment{theindex}{%
  \section*{\indexname}%
  \setlength{\parindent}{0pt}%
  \setlength{\parskip}{0pt plus 0.3pt}%
  \let\item\@idxitem
}{%
  \clearpage
}
\makeatother

\IfFileExists{\jobname-pw.ind}{\input{\jobname-pw.ind}}{}

\end{document}

      