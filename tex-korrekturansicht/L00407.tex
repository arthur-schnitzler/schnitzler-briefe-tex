%% latex-korrekturansicht-vorspann.tex
%% Vorspann für die Korrekturansicht.
%% Lädt die gemeinsame Datei latex-vorspann.tex mit gesetztem Schalter.

\newif\ifkorrekturansicht
\korrekturansichttrue

\input{../tex-inputs/latex-vorspann}


               \section[Friedrich M. Fels an Arthur Schnitzler, {[}26. 11. 1894{]}]{ Friedrich M. Fels an Arthur Schnitzler, {[}26. 11. 1894{]}}\nopagebreak\mylabel{v}\rehead{ }\normalsize\beginnumbering\briefempfaengerindex{Schnitzler, Arthur@\textsc{Schnitzler, Arthur}!zzzFels, Friedrich Michael@\emph{von Friedrich Michael Fels}!1894-11-262@{{[}26. 11. 1894{]}}|(be} \toendnotes[C]{\smallbreak\pagebreak[2]} \Standort{DLA, A:Schnitzler, HS.NZ85.1.2956.}
\physDesc{Brief, 1 Blatt, 1 Seite
\newline{}Handschrift: schwarze Tinte, lateinische Kurrent
\newline{}Schnitzler: 1) mit Bleistift nummeriert: »20« 2) mit schwarzer Tinte datiert: »26. 11. 94«3) mit rotem Buntstift eine Unterstreichung}\pstart{}{\pb}Lieber Dr. Schnitzler!\pend\pstart
           Vielleicht hätten Sie die Freundlichkeit, möglichst bald \uline{\textcolor{blue}{Hugo Gerlach}{}\ledrightnote{\textcolor{blue}{Hugo Gerlach}}} zu besuchen. Er hat vielleicht die Diphteritis. Wohnung: \textcolor{pink}{XVIII (Währing), Sechsschi{\geminationm}elgaſse 4}{}\ledrightnote{\textcolor{pink}{Sechsschimmelgasse}} II. Stock Thür 12. –\pend
           \pstart
           Vielleicht sind \introOben{}Sie\introOben{} auch so gütig, mir \uline{1 fl} zu geben, den Sie bei \textcolor{blue}{Gerlach}{}\ledrightnote{\textcolor{blue}{Hugo Gerlach}}
               zurücklassen. Herzl. Dank. – Vom alten \textcolor{blue}{Mayer}{}\ledrightnote{\textcolor{blue}{Edmund Mayer}} hab
               ich keine Antwort. Die \textcolor{brown}{Kölnische Zeitung}{}\ledrightnote{\textcolor{brown}{Kölnische Zeitung}} hat meinen
               Artikel »\textcolor{green}{Skandinavien in Deutschland}{}\ledrightnote{\textcolor{green}{Skandinavien in Deutschland}}« acceptiert
               unter der Bedingung, daſs ich ihn um ⅓ kürze. Mein Roman wächst, blüht und gedeiht –
               ich habe früher nur den Ton nicht getroffen; jetzt nachdem ich der Kälte und Ironie
               den Abschied gegeben und \introOben{}auf\introOben{} harmlos humoristische Wirkung
               denke, gehts famos.\pend
           \pstart
           Gruſs und Dank{\\[\baselineskip]}\spacefill\mbox{Fels}\pend
           \leftskip=0em{}\endnumbering\briefempfaengerindex{Schnitzler, Arthur@\textsc{Schnitzler, Arthur}!zzzFels, Friedrich Michael@\emph{von Friedrich Michael Fels}!1894-11-262@{{[}26. 11. 1894{]}}|)be}\mylabel{h}  \normalsize

\doendnotes{C}
\bigskip
\vfill

\clearpage

\footnotesize

\lohead{\textsc{register}}

% Definiere theindex-Environment komplett neu ohne reledmac
\makeatletter
\renewenvironment{theindex}{%
  \section*{\indexname}%
  \setlength{\parindent}{0pt}%
  \setlength{\parskip}{0pt plus 0.3pt}%
  \let\item\@idxitem
}{%
  \clearpage
}
\makeatother

\IfFileExists{\jobname-pw.ind}{\input{\jobname-pw.ind}}{}

\end{document}

      