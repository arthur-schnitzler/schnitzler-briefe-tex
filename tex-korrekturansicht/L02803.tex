%% latex-korrekturansicht-vorspann.tex
%% Vorspann für die Korrekturansicht.
%% Lädt die gemeinsame Datei latex-vorspann.tex mit gesetztem Schalter.

\newif\ifkorrekturansicht
\korrekturansichttrue

\input{../tex-inputs/latex-vorspann}


               \section[ Paul Goldmann an Arthur Schnitzler, 16. 2. {[}1897{]}]{Paul Goldmann an Arthur Schnitzler, 16. 2. {[}1897{]}}\nopagebreak\mylabel{v}\rehead{ }\normalsize\beginnumbering\briefempfaengerindex{Schnitzler, Arthur@\textsc{Schnitzler, Arthur}!zzzGoldmann, Paul@\emph{von Paul Goldmann}!1897-02-161@{16. 2. {[}1897{]}}|(be} \toendnotes[C]{\smallbreak\pagebreak[2]} \Standort{DLA, A:Schnitzler, HS.NZ85.1.3167.}
\physDesc{Brief, 1 Blatt, 4 Seiten
\newline{}Handschrift: blaue Tinte, deutsche Kurrent
\newline{}Schnitzler: mit Bleistift das Jahr »97« vermerkt }\toendnotes[C]{\smallbreak}\pstart
           \noindent{}{\pb}\textcolor{gray}{\textbf{\textbf{\textcolor{brown}{Frankfurter Zeitung}{}\ledrightnote{\textcolor{brown}{Frankfurter Zeitung}}}}}\pend
           \pstart
           \textcolor{gray}{\textbf{(\textcolor{brown}{\begin{otherlanguage}{french}Gazette de Francfort\end{otherlanguage}}{}\ledrightnote{\textcolor{brown}{Frankfurter Zeitung}}).}}\pend
           \pstart
           \textcolor{gray}{\textbf{\textbf{\begin{otherlanguage}{french}Fondateur M.\end{otherlanguage}{ }\textcolor{blue}{L. Sonnemann}{}\ledrightnote{\textcolor{blue}{Leopold Sonnemann}}.}}}\pend
           \pstart
           \begin{otherlanguage}{french}\textcolor{gray}{\textbf{Journal politique, financier,}}\end{otherlanguage}\pend
           \pstart
           \begin{otherlanguage}{french}\textcolor{gray}{\textbf{commercial et littéraire.}}\end{otherlanguage}\pend
           \pstart
           \begin{otherlanguage}{french}\textcolor{gray}{\textbf{\textbf{Paraissant trois fois par jour.}}}\end{otherlanguage}\hfill \textsc{\textcolor{pink}{Paris}{}\ledrightnote{\textcolor{pink}{Paris}}}, 16. Februar.\pend
           \pstart
           \begin{otherlanguage}{french}\textcolor{gray}{\textbf{\textbf{Bureau à \textcolor{pink}{Paris}{}\ledrightnote{\textcolor{pink}{Paris}}}}}\end{otherlanguage}\pend
           \pstart
           \begin{otherlanguage}{french}\textcolor{gray}{\textbf{\textbf{\textcolor{pink}{24. Rue Feydeau}{}\ledrightnote{\textcolor{pink}{rue Feydeau}}.}}}\end{otherlanguage}\pend
           \pstart\center{}Mein lieber Freund,\pend\pstart
           Ich ſtecke mitten in den \label{K_L02803-1v}\edtext{\textcolor{green}{\textsc{\textcolor{pink}{Kreta}{}\ledrightnote{\textcolor{pink}{Kreta}}}-Geſchichten}{}\ledrightnote{→\textcolor{green}{[Kreta-Geschichten]}}}{\lemma{\textnormal{\emph{Kreta-Geſchichten}}}\Cendnote{\textnormal{Am
                     6. 2. 1897 waren erste \textcolor{pink}{griechische} Kriegsschiffe auf \textcolor{pink}{Kreta} gelandet, um die unzufriedene Bevölkerung gegen die \textcolor{pink}{türkische} Regierung zu unterstützen. In Folge
                  kam es zwischen 18. 4. und 20. 5. 1897
                  zum \textcolor{pink}{Türkisch}-\textcolor{pink}{Griechischen} Krieg. \textcolor{blue}{Goldmann}
                  schrieb am XXXX.}}}\label{K_L02803-1h} und kann Dir heut nur kurz
               meine Befriedigung über all’ das Beruhigende, das Dein lieber Brief enthält, – und
               mein Entzücken über die Ausſicht melden, Dich hier zu haben. Es iſt vielleicht ſehr
               egoiſtiſch, daß ich in all’ Deinem Kummer nur die große Freude ſehe, die für mich
               herauswächſt. Aber auch Dir wird \textsc{\textcolor{pink}{Paris}{}\ledrightnote{\textcolor{pink}{Paris}}} gut thun, ich bin deſſen ſicher. {\pb}Du wirſt
               hier Alles von fern und von hoch ſehen und wirſt leicht darüber hinwegkommen – im
               Rauſch eines \textcolor{pink}{Pariſ}{}\ledrightnote{\textcolor{pink}{Paris}}er Frühlings.\pend
           \pstart
           Wirſt Du \label{K_L02803-2v}\edtext{bald kommen}{\lemma{\textnormal{\emph{bald kommen}}}\Cendnote{\textnormal{\textcolor{blue}{Schnitzler} kam am 12. 4. 1897 in \textcolor{pink}{Paris} an.}}}\label{K_L02803-2h}? Es kann geſchehen, daß ich
               Anfang März oder Ende Februar
               auf vierzehn Tage nach der \textsc{\textcolor{pink}{Riviera}{}\ledrightnote{\textcolor{pink}{Riviera}}} gehen muß, um \label{K_L02803-3v}\edtext{Saiſon-Feuilletons}{\lemma{\textnormal{\emph{Saiſon-Feuilletons}}}\Cendnote{\textnormal{nicht geschehen,
                     vgl. Paul Goldmann an Arthur Schnitzler, 22. 3. [1897]}}}\label{K_L02803-3h} zu ſchreiben. Wenn ich Dir alſo Wohnung beſorgen ſoll, gib’ mir \uline{umgehend} ſchriftlichen oder telegraphiſchen Auftrag.
               Und laß’ mich nur tüchtig für Dich arbeiten. {\pb}Das
               wird die erſte \textcolor{pink}{Pariſ}{}\ledrightnote{\textcolor{pink}{Paris}}er Wohnung ſein, die ich mit
               Vergnügen ſuchen werde.\pend
           \pstart
           Nun bleib’ aber auch bei dem Plan. Glaub’ mir, nirgends biſt Du ſo aus der Welt, wie
               in \textsc{\textcolor{pink}{Paris}{}\ledrightnote{\textcolor{pink}{Paris}}}. Daß Du zugleich zum Genuſſe der \textcolor{pink}{Stadt}{}\ledrightnote{→\textcolor{pink}{Paris}} kommſt, dafür laß’ mich nur \strikeout{ſ\textcolor{gray}{or}} ſorgen.\pend
           \pstart
           Grüß’ Dich Gott, Liebſter! Laß’ Dich nicht von den äußeren Unannehmlichkeiten
               niederdrücken. »\label{K_L02803-5v}\edtext{\begin{otherlanguage}{french}\textsc{Tout s’arrange}\end{otherlanguage}}{\lemma{\textnormal{\emph{Tout s’arrange}}}\Cendnote{\textnormal{französisch: alles wird sich
                  richten}}}\label{K_L02803-5h}« ſagt einer meiner hieſiegen Freunde, und das iſt wahr. {\pb}Es gibt nur \uline{ein}
               wirkliches Unglück: die Krankheit. Was von Menſchen kommt, iſt nicht gefährlich.\pend
           \pstart
           Dein treuer {\\[\baselineskip]}\spacefill\mbox{Paul Goldmnn}\pend
           \leftskip=0em{}\endnumbering\briefempfaengerindex{Schnitzler, Arthur@\textsc{Schnitzler, Arthur}!zzzGoldmann, Paul@\emph{von Paul Goldmann}!1897-02-161@{16. 2. {[}1897{]}}|)be}\mylabel{h}  \normalsize

\doendnotes{C}
\bigskip
\vfill

\clearpage

\footnotesize

\lohead{\textsc{register}}

% Definiere theindex-Environment komplett neu ohne reledmac
\makeatletter
\renewenvironment{theindex}{%
  \section*{\indexname}%
  \setlength{\parindent}{0pt}%
  \setlength{\parskip}{0pt plus 0.3pt}%
  \let\item\@idxitem
}{%
  \clearpage
}
\makeatother

\IfFileExists{\jobname-pw.ind}{\input{\jobname-pw.ind}}{}

\end{document}

      