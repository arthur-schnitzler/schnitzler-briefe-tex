%% latex-korrekturansicht-vorspann.tex
%% Vorspann für die Korrekturansicht.
%% Lädt die gemeinsame Datei latex-vorspann.tex mit gesetztem Schalter.

\newif\ifkorrekturansicht
\korrekturansichttrue

\input{../tex-inputs/latex-vorspann}


\renewcommand{\erwaehntePersonen}{Personen: Raoul Auernheimer, Felix Salten, Olga Schnitzler}
\renewcommand{\erwaehnteOrte}{Orte: Berlin, Dessauer Straße, Schöneberger Ufer, Wien}
\renewcommand{\erwaehnteWerke}{Werke: Der Weg ins Freie, Der Weg ins Freie. Roman, Die Zeit, Frau Bertha Garlan. Roman, Neue Freie Presse, Schnitzlers Wiener Roman}
\section[ Paul Goldmann an Arthur Schnitzler, 5. 6. {[}1908{]}]{Paul Goldmann an Arthur Schnitzler, 5. 6. {[}1908{]}}
\nopagebreak\mylabel{v}
\rehead{ }\normalsize\beginnumbering\briefempfaengerindex{Schnitzler, Arthur@\textsc{Schnitzler, Arthur}!zzzGoldmann, Paul@\emph{von Paul Goldmann}!1908-06-051@{5. 6. {[}1908{]}}|(be}
\toendnotes[C]{\smallbreak\pagebreak[2]}\Standort{DLA, A:Schnitzler, HS.NZ85.1.3176.}
\physDesc{Brief, 1 Blatt, 3 Seiten, 760 Zeichen
\newline{}Handschrift Schreibkraft: blaue Tinte, lateinische Kurrent
\newline{}Handschrift Paul Goldmann: blaue Tinte, deutsche Kurrent (\noindent{}Schlussformel und Unterschrift)
\newline{}Schnitzler: mit Bleistift »\textcolor{blue}{Goldma{\geminationn}}« vermerkt }\toendnotes[C]{\smallbreak}
\pstart
           \raggedleft{}{\pb}\textcolor{gray}{\textbf{\textcolor{pink}{DESSAUERSTRASSE 19}{}\ledrightnote{\textcolor{pink}{Dessauer Straße}}}}, d. 5. 6.\pend
           
\pstart{}Lieber Freund,\pend
\pstart
           Mit der Uebersendung Deines \label{K_L03484-1v}\edtext{\textcolor{green}{Roman}{}\ledrightnote{{$\rightarrow$}\textcolor{green}{Der Weg ins Freie. Roman}}}{\lemma{\textnormal{\emph{Roman}}}\Cendnote{\textnormal{Die fehlende Datierung auf das Jahr
                     1908 gelingt durch implizite Kriterien. \textcolor{blue}{Goldmann} wohnte ab dem Frühjahr 1900 und höchstens bis Anfang 1909 in der \textcolor{pink}{Dessauerstraße} (ab 1909 wird er in \textcolor{pink}{Berlin}er
                  Adressbüchern als am \textcolor{pink}{Schöneberger Ufer}
                  wohnhaft verzeichnet). In dieser Zeit erschienen nur zwei Romane: \emph{\textcolor{green}{Frau Bertha Garlan}} (1901) und \emph{\textcolor{green}{Der Weg ins Freie}} (1908). Nur für
                  den zweiteren Titel schrieben \textcolor{blue}{Salten} und \textcolor{blue}{Auernheimer} Rezensionen.}}}\label{K_L03484-1h}s hast Du mir
               eine große Freude gemacht. Ich werde sofort die Lektüre beginnen und danke Dir
               einstweilen herzlichst für \textcolor{green}{Buch}{}\ledrightnote{{$\rightarrow$}\textcolor{green}{Der Weg ins Freie. Roman}} und Widmung.\pend
           
\pstart
           Von allen Seiten hoere ich \textcolor{pink}{hier}{}\ledrightnote{{$\rightarrow$}\textcolor{pink}{Berlin}}
               in den waermsten Aus{\pb}drücken von Deinem neuen
                  \textcolor{green}{Werke}{}\ledrightnote{{$\rightarrow$}\textcolor{green}{Der Weg ins Freie. Roman}} sprechen. Die 
               \textcolor{green}{Feuilletons}{}\ledrightnote{{$\rightarrow$}\textcolor{green}{Schnitzlers Wiener Roman}{\newline}{$\rightarrow$}\textcolor{green}{Der Weg ins Freie}} von
               \label{K_L03484-2v}\edtext{\textcolor{blue}{Salten}{}\ledrightnote{\textcolor{blue}{Felix Salten}}}{\lemma{\textnormal{\emph{Salten}}}\Cendnote{\textnormal{\textcolor{blue}{Felix Salten}: \emph{\textcolor{green}{Schnitzlers Wiener Roman}}. In: \emph{\textcolor{green}{Die Zeit}}, Jg. 7, Nr. 2.042, 30. 5. 1908, Morgenblatt, S. 1–2.}}}\label{K_L03484-2h} und \label{K_L03484-3v}\edtext{\textcolor{blue}{Auernheimer}{}\ledrightnote{\textcolor{blue}{Raoul Auernheimer}}}{\lemma{\textnormal{\emph{Auernheimer}}}\Cendnote{\textnormal{\textcolor{blue}{Raoul Auernheimer}: \emph{\textcolor{green}{Der Weg ins Freie}}. In: \emph{\textcolor{green}{Neue Freie Presse}}, Nr. 15.728, 3. 6. 1908, Morgenblatt, S. 1–3.}}}\label{K_L03484-3h} haben das \textcolor{green}{Buch}{}\ledrightnote{{$\rightarrow$}\textcolor{green}{Der Weg ins Freie. Roman}} in \textcolor{pink}{Wien}{}\ledrightnote{\textcolor{pink}{Wien}} aufs beste eingeführt. Du scheinst also diesmal auf einen
               großen Erfolg rechnen zu dürfen und ich wünsche und hoffe, daß diese
               Erfolgs-Aussichten sich glänzend erfüllen moegen.\pend
           
\pstart
           Hoffentlich geht es Dir und {\pb}Deiner \textcolor{blue}{Frau}{}\ledrightnote{{$\rightarrow$}\textcolor{blue}{Olga Schnitzler}} gut. Ich vermute, daß \textcolor{blue}{Ihr}{}\ledrightnote{{$\rightarrow$}\textcolor{blue}{Olga Schnitzler}} von Eurer \label{K_L03484-4v}\edtext{Reise}{\lemma{\textnormal{\emph{Reise}}}\Cendnote{\textnormal{siehe Paul Goldmann an Arthur Schnitzler, 8. 5. 1908}}}\label{K_L03484-4h} schon zurück seid, und denke mir, daß sie sehr interessant gewesen sein
               muß.\pend
           
\pstart
           Ich wünsche \textcolor{blue}{Euch}{}\ledrightnote{{$\rightarrow$}\textcolor{blue}{Olga Schnitzler}} frohe
               Feiertage und bin mit vielen herzlichen Grüßen an Euch \textcolor{blue}{Beide}{}\ledrightnote{{$\rightarrow$}\textcolor{blue}{Olga Schnitzler}}\pend
           
\pstart
           {[}hs. Goldmann:{]} Dein {\\[\baselineskip]}\spacefill\mbox{Paul Goldmann.}\pend
           \leftskip=0em{}\endnumbering\briefempfaengerindex{Schnitzler, Arthur@\textsc{Schnitzler, Arthur}!zzzGoldmann, Paul@\emph{von Paul Goldmann}!1908-06-051@{5. 6. {[}1908{]}}|)be}\mylabel{h}
\begin{anhang}
\end{anhang}\normalsize

\doendnotes{C}
\bigskip
\vfill

\clearpage

\footnotesize

\lohead{\textsc{register}}

% Definiere theindex-Environment komplett neu ohne reledmac
\makeatletter
\renewenvironment{theindex}{%
  \section*{\indexname}%
  \setlength{\parindent}{0pt}%
  \setlength{\parskip}{0pt plus 0.3pt}%
  \let\item\@idxitem
}{%
  \clearpage
}
\makeatother

\IfFileExists{\jobname-pw.ind}{\input{\jobname-pw.ind}}{}

\end{document}

      