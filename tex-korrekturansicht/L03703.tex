%% latex-korrekturansicht-vorspann.tex
%% Vorspann für die Korrekturansicht.
%% Lädt die gemeinsame Datei latex-vorspann.tex mit gesetztem Schalter.

\newif\ifkorrekturansicht
\korrekturansichttrue

\input{../tex-inputs/latex-vorspann}


\section[Elsa Plessner an Arthur Schnitzler, 21. 9. 1896]{L03703 Elsa Plessner an Arthur Schnitzler, 21. 9. 1896}
\nopagebreak\mylabel{L03703v}
\rehead{ }\normalsize\beginnumbering\briefempfaengerindex{Schnitzler, Arthur@\textsc{Schnitzler, Arthur}!zzzPlessner, Elsa@\emph{von Elsa Plessner}!1896-09-213@{21. 9. 1896}|(be}
\toendnotes[C]{\smallbreak\pagebreak[2]}
\correspDesc{Versand  durch Elsa Plessner am 21. 9. 1896 in Wien
\newline{}Erhalt  durch Arthur Schnitzler im Zeitraum [21. 9. 1896
                  – 24. 9. 1896?] in Wien}\toendnotes[C]{\smallbreak}
\Standort{DLA, A:Schnitzler, HS.1985.1.419.}
\physDesc{Brief, 2 Blätter, 3 Seiten, 2789 Zeichen
\newline{}Handschrift: schwarze Tinte, lateinische Kurrent
\newline{}Schnitzler: mit rotem Buntstift eine Unterstreichung }\toendnotes[C]{\smallbreak}
\pstart
           {\pb}\textcolor{pink}{I. Bäckerstrasse N\textsuperscript{o}
                     1}\oindex{Wien@\textbf{Wien}!I., Innere Stadt@\textbf{I., Innere Stadt}!Bäckerstraße 1@\textbf{Bäckerstraße 1}, \emph{Wohngebäude}|pw}{}\ledrightnote{\textcolor{pink}{Bäckerstraße 1}}, den 21. 9. 96. \pend
           
\pstart\center{}Verehrter Herr Doctor!\pend\vspace{0.5em}
\pstart
           Mit dem Tage, der eben schließt, sind Sie zum \uline{Erzengel} avancirt. –\pend
           
\pstart
           Herzlichsten, aufrichtigsten Dank für die Geduld und Aufmerksamkeit die Sie meinen
                  \label{K_L03703-1v}\edtext{\textcolor{green}{Arbeiten}\pwindex{Plessner, Elsa 22.\,8.\,1875 Wien – 7.\,5.\,1932 Alicante@\textsc{Plessner, Elsa} (22.\,8.\,1875 Wien – 7.\,5.\,1932 Alicante), \emph{Schriftstellerin}!gläserne Käfig. Skizzen und Novellen@\strich\emph{Der gläserne Käfig. Skizzen und Novellen}|pwv}{}\ledrightnote{{$\rightarrow$}\emph{\textcolor{green}{Der gläserne Käfig. Skizzen und Novellen}}}}{\lemma{\textnormal{\emph{Arbeiten}}}\Cendnote{\textnormal{\textcolor{blue}{Elsa Plessners}\pwindex{Plessner, Elsa 22.\,8.\,1875 Wien – 7.\,5.\,1932 Alicante@\textsc{Plessner, Elsa} (22.\,8.\,1875 Wien – 7.\,5.\,1932 Alicante), \emph{Schriftstellerin}|pwk} Band \emph{\textcolor{green}{Der gläserne Käfig}\pwindex{Plessner, Elsa 22.\,8.\,1875 Wien – 7.\,5.\,1932 Alicante@\textsc{Plessner, Elsa} (22.\,8.\,1875 Wien – 7.\,5.\,1932 Alicante), \emph{Schriftstellerin}!gläserne Käfig. Skizzen und Novellen@\strich\emph{Der gläserne Käfig. Skizzen und Novellen}|pwk}} mit vierzehn Novellen und Skizzen
                  erschien 1901. Welche zehn Texte daraus sie in welcher Reihenfolge
                     \textcolor{blue}{Schnitzler} mit dem vorangegangenen Brief
                  geschickt hatte, läßt sich nur zum Teil rekonstruieren. Sicher dabei waren die
                  Skizzen \emph{\textcolor{green}{Warten}\pwindex{Plessner, Elsa 22.\,8.\,1875 Wien – 7.\,5.\,1932 Alicante@\textsc{Plessner, Elsa} (22.\,8.\,1875 Wien – 7.\,5.\,1932 Alicante), \emph{Schriftstellerin}!Warten. Novelle@\strich\emph{Warten. Novelle}|pwk}}, \emph{\textcolor{green}{Der Selbstmörder}\pwindex{Plessner, Elsa 22.\,8.\,1875 Wien – 7.\,5.\,1932 Alicante@\textsc{Plessner, Elsa} (22.\,8.\,1875 Wien – 7.\,5.\,1932 Alicante), \emph{Schriftstellerin}!Leiter der Seele@\strich\emph{Die Leiter der Seele}|pwk}}, \emph{\textcolor{green}{Begräbnißtag}\pwindex{Begräbnißtag@\emph{Der Begräbnißtag}|pwk}}, \emph{\textcolor{green}{Im Feuer geprüft}\pwindex{Im Feuer geprüft@\emph{Im Feuer geprüft}|pwk}} und
                     \emph{\textcolor{green}{Im Widerschein}\pwindex{Im Widerschein@\emph{Im Widerschein}|pwk}}.}}}\label{K_L03703-1} zugewendet haben.
               Diese Liebenswürdigkeit, die Sie mir gegenüber so oft schon bethätigten ist so
               beispiellos, dass mir jeder Ausdruck fehlt, sie näher zu characterisieren! Sie werden
               zwar sagen: »\label{K_L03703-2v}\edtext{Schlamperei}{\lemma{\textnormal{\emph{Schlamperei}}}\Cendnote{\textnormal{\textcolor{blue}{Schnitzlers} Antwort ist nicht erhalten; im \emph{\textcolor{green}{Tagebuch}\pwindex{Schnitzler, Arthur 15. 5. 1862 Wien – 21. 10. 1931 ebd.@\textsc{Schnitzler, Arthur} (15. 5. 1862 Wien – 21. 10. 1931 ebd.), \emph{Schriftsteller, Mediziner}!Tagebuch@\strich\emph{Tagebuch}|pwk}}-Eintrag zum 19. 9. 1896 findet sich eine Einordnung seines
                  Urteils: »\textcolor{blue}{Else Plessners}\pwindex{Plessner, Elsa 22.\,8.\,1875 Wien – 7.\,5.\,1932 Alicante@\textsc{Plessner, Elsa} (22.\,8.\,1875 Wien – 7.\,5.\,1932 Alicante), \emph{Schriftstellerin}|pw} schickte mir
                     neulich ihre Skizzen. Schlampert, journalistisch, hie und da originelle
                     Züge. –«}}}\label{K_L03703-2}! Man muß alle Ausdrücke finden!« Ich bin aber
               wieder so empörend faul, nicht lange darüber nachzudenken! »Wie gesagt« – Sie sind
               ein Engel in xter Potenz! – Geradezu fabelhaft finde ich es, daß sie die {\pb}schöne Zeit, die Sie zu so vielem Anderen hätten
               verwenden können, zur Anfertigung der \label{K_L03703-3v}\edtext{graziösen Excerpte}{\lemma{\textnormal{\emph{graziösen Excerpte}}}\Cendnote{\textnormal{\textcolor{blue}{Schnitzlers} Antwortbrief und seine
                  Redaktionsarbeiten an den eingesandten \textcolor{green}{Texten}\pwindex{Plessner, Elsa 22.\,8.\,1875 Wien – 7.\,5.\,1932 Alicante@\textsc{Plessner, Elsa} (22.\,8.\,1875 Wien – 7.\,5.\,1932 Alicante), \emph{Schriftstellerin}!gläserne Käfig. Skizzen und Novellen@\strich\emph{Der gläserne Käfig. Skizzen und Novellen}|pwkv} sind nicht überliefert.}}}\label{K_L03703-3} aus meinen \textcolor{green}{Meisterwerken}\pwindex{Plessner, Elsa 22.\,8.\,1875 Wien – 7.\,5.\,1932 Alicante@\textsc{Plessner, Elsa} (22.\,8.\,1875 Wien – 7.\,5.\,1932 Alicante), \emph{Schriftstellerin}!gläserne Käfig. Skizzen und Novellen@\strich\emph{Der gläserne Käfig. Skizzen und Novellen}|pwv}{}\ledrightnote{{$\rightarrow$}\emph{\textcolor{green}{Der gläserne Käfig. Skizzen und Novellen}}} geopfert haben!
               Wie werde ich das \introOben{}vor\introOben{} der \textcolor{pink}{deutschen}\oindex{Deutschland@\textbf{Deutschland}|pw}{}\ledrightnote{\textcolor{pink}{Deutschland}} Literatur verantworten können? – Übrigens, verehrter Meister \textcolor{green}{Anatol}\pwindex{Schnitzler, Arthur 15. 5. 1862 Wien – 21. 10. 1931 ebd.@\textsc{Schnitzler, Arthur} (15. 5. 1862 Wien – 21. 10. 1931 ebd.), \emph{Schriftsteller, Mediziner}!Anatol@\strich\emph{Anatol}|pwv}{}\ledrightnote{{$\rightarrow$}\emph{\textcolor{green}{Anatol}}} – Sie haben mir so den
               Kopf gewaschen, dass mir alle Haarwurzeln weh thun und, – – mit Recht!!! – Alle die
               Abscheulichkeiten, die ich verbrochen, haben Sie mir in einem so lieblichen Neben-
               und Nacheinander vor mein jetzt gänzlich zerschmettertes literarisches Gewissen
               geführt – – – \label{K_L03703-4v}\edtext{mea culpa}{\lemma{\textnormal{\emph{mea culpa}}}\Cendnote{\textnormal{latein: durch meine Schuld}}}\label{K_L03703-4}!! –
               Eines aber freut mich riesig – dass \textcolor{green}{No 1.}\pwindex{Plessner, Elsa 22.\,8.\,1875 Wien – 7.\,5.\,1932 Alicante@\textsc{Plessner, Elsa} (22.\,8.\,1875 Wien – 7.\,5.\,1932 Alicante), \emph{Schriftstellerin}!Warten. Novelle@\strich\emph{Warten. Novelle}|pwv}{}\ledrightnote{{$\rightarrow$}\emph{\textcolor{green}{Warten. Novelle}}} (jetzt »\textcolor{green}{Warten}\pwindex{Plessner, Elsa 22.\,8.\,1875 Wien – 7.\,5.\,1932 Alicante@\textsc{Plessner, Elsa} (22.\,8.\,1875 Wien – 7.\,5.\,1932 Alicante), \emph{Schriftstellerin}!Warten. Novelle@\strich\emph{Warten. Novelle}|pw}{}\ledrightnote{\textcolor{green}{Warten. Novelle}}« früher »\textcolor{green}{Blätter}\pwindex{Plessner, Elsa 22.\,8.\,1875 Wien – 7.\,5.\,1932 Alicante@\textsc{Plessner, Elsa} (22.\,8.\,1875 Wien – 7.\,5.\,1932 Alicante), \emph{Schriftstellerin}!Warten. Novelle@\strich\emph{Warten. Novelle}|pw}{}\ledrightnote{\textcolor{green}{Warten. Novelle}}«) Ihnen nun doch ein wenig gefällt! Denn
               das ist die einzige Arbeit, an der mir etwas liegt und auch – meine letzte!!
               Überhaupt finde {\pb}ich zu meinem großem Vergnügen, dass Sie
               alle die Arbeiten für die relativ besten erklären, die richtig jüngeres Datum tragen
               als die andern. \label{K_L03703-5v}\edtext{Der \textcolor{green}{Onkel}\pwindex{Plessner, Elsa 22.\,8.\,1875 Wien – 7.\,5.\,1932 Alicante@\textsc{Plessner, Elsa} (22.\,8.\,1875 Wien – 7.\,5.\,1932 Alicante), \emph{Schriftstellerin}!Onkel@\strich\emph{Der Onkel}|pw}{}\ledrightnote{\textcolor{green}{Der Onkel}}}{\lemma{\textnormal{\emph{Der Onkel}}}\Cendnote{\textnormal{Die hier genannten Texte \emph{\textcolor{green}{Der Onkel}\pwindex{Plessner, Elsa 22.\,8.\,1875 Wien – 7.\,5.\,1932 Alicante@\textsc{Plessner, Elsa} (22.\,8.\,1875 Wien – 7.\,5.\,1932 Alicante), \emph{Schriftstellerin}!Onkel@\strich\emph{Der Onkel}|pwk}}, \emph{\textcolor{green}{Sie
                     gähnt}\pwindex{Plessner, Elsa 22.\,8.\,1875 Wien – 7.\,5.\,1932 Alicante@\textsc{Plessner, Elsa} (22.\,8.\,1875 Wien – 7.\,5.\,1932 Alicante), \emph{Schriftstellerin}!Sie gähnt@\strich\emph{Sie gähnt}|pwk}} und \emph{\textcolor{green}{Eile}\pwindex{Plessner, Elsa 22.\,8.\,1875 Wien – 7.\,5.\,1932 Alicante@\textsc{Plessner, Elsa} (22.\,8.\,1875 Wien – 7.\,5.\,1932 Alicante), \emph{Schriftstellerin}!Eile@\strich\emph{Eile}|pwk}} sind nicht unter
                  diesen Titeln in den Band \emph{\textcolor{green}{Der gläserne Käfig}\pwindex{Plessner, Elsa 22.\,8.\,1875 Wien – 7.\,5.\,1932 Alicante@\textsc{Plessner, Elsa} (22.\,8.\,1875 Wien – 7.\,5.\,1932 Alicante), \emph{Schriftstellerin}!gläserne Käfig. Skizzen und Novellen@\strich\emph{Der gläserne Käfig. Skizzen und Novellen}|pwk}}
                  aufgenommen worden. Es ist aber gut möglich, dass es sich um frühe Versionen
                  später umbenannter Texte handelt. \textcolor{blue}{Elsa
                     Plessner}\pwindex{Plessner, Elsa 22.\,8.\,1875 Wien – 7.\,5.\,1932 Alicante@\textsc{Plessner, Elsa} (22.\,8.\,1875 Wien – 7.\,5.\,1932 Alicante), \emph{Schriftstellerin}|pwk} betont in ihren Briefen an \textcolor{blue}{Schnitzler} mehrfach, dass ihr schmales Werk nichts über die später
                  publizierten Texte hinaus enthalte, vgl. Elsa Plessner an Arthur Schnitzler, 12. 10. 1900.}}}\label{K_L03703-5}, das Monstrum von Geschmacklosigkeit, ist aus dem
                  Jahre \uuline{93} – sowie auch »\textcolor{green}{Sie gähnt}\pwindex{Plessner, Elsa 22.\,8.\,1875 Wien – 7.\,5.\,1932 Alicante@\textsc{Plessner, Elsa} (22.\,8.\,1875 Wien – 7.\,5.\,1932 Alicante), \emph{Schriftstellerin}!Sie gähnt@\strich\emph{Sie gähnt}|pw}{}\ledrightnote{\textcolor{green}{Sie gähnt}}« ungefähr so
               alt ist. Was Sie von »\textcolor{green}{Eile}\pwindex{Plessner, Elsa 22.\,8.\,1875 Wien – 7.\,5.\,1932 Alicante@\textsc{Plessner, Elsa} (22.\,8.\,1875 Wien – 7.\,5.\,1932 Alicante), \emph{Schriftstellerin}!Eile@\strich\emph{Eile}|pw}{}\ledrightnote{\textcolor{green}{Eile}}« schreiben, kann ich
               eigentlich nicht begreifen! Die zehn \textcolor{green}{Skizzen}\pwindex{Plessner, Elsa 22.\,8.\,1875 Wien – 7.\,5.\,1932 Alicante@\textsc{Plessner, Elsa} (22.\,8.\,1875 Wien – 7.\,5.\,1932 Alicante), \emph{Schriftstellerin}!gläserne Käfig. Skizzen und Novellen@\strich\emph{Der gläserne Käfig. Skizzen und Novellen}|pwv}{}\ledrightnote{{$\rightarrow$}\emph{\textcolor{green}{Der gläserne Käfig. Skizzen und Novellen}}} und das \textcolor{green}{Stück}\pwindex{Plessner, Elsa 22.\,8.\,1875 Wien – 7.\,5.\,1932 Alicante@\textsc{Plessner, Elsa} (22.\,8.\,1875 Wien – 7.\,5.\,1932 Alicante), \emph{Schriftstellerin}!Heimweh [dreiaktige Tragikomödie]@\strich\emph{Heimweh [dreiaktige Tragikomödie]}|pwv}{}\ledrightnote{{$\rightarrow$}\emph{\textcolor{green}{Heimweh [dreiaktige Tragikomödie]}}}, sowie die »\textcolor{green}{freien
                  Rhythmen}\pwindex{Plessner, Elsa 22.\,8.\,1875 Wien – 7.\,5.\,1932 Alicante@\textsc{Plessner, Elsa} (22.\,8.\,1875 Wien – 7.\,5.\,1932 Alicante), \emph{Schriftstellerin}!Pierettes Tagebuch [19 unveröffentlichte Gedichte]@\strich\emph{Pierettes Tagebuch [19 unveröffentlichte Gedichte]}|pw}{}\ledrightnote{\textcolor{green}{Pierettes Tagebuch [19 unveröffentlichte Gedichte]}}{[}«{]}, die Sie seinerzeit so wüthend gemacht haben, sind meine
               ganze, gesammte Production von – 3 Jahren!! – Das ist doch nicht viel? – Mir sind die
               alten Sachen so in der Seele zuwider, dass ich am liebsten gar nichts davon mehr
               wissen wollte – soll ich da wirklich noch lange in dem alten Kehricht herumstöbern? –
               Wenn ich nicht \uuline{müsste} – so ließe ich sie wirklich
               nicht aus Tageslicht – doch so? – Ich werde die Blößen der armen Kinder nothdürftig
               bedecken, so von oben auf nach Ihren Angaben und dann – fort mit Schaden – ! Für die
               Zukunft verspreche und gelobe ich, nach Ihren Directiven anständig und ehrlich zu
               arbeiten, nichts mehr zu schleudern, und im übrigen auf mein Talent, das Sie ja »mit
               einem heitern, einem nassen Auge« anerkennen, zu bauen. – – –\pend
           
\pstart
           Darf ich mir die Anfrage gestatten, was ich nun \label{K_L03703-6v}\edtext{betreffs Director \textcolor{blue}{Brahm}\pwindex{Brahm, Otto 5.\,2.\,1856 Hamburg – 28.\,11.\,1912 Berlin@\textsc{Brahm, Otto} (5.\,2.\,1856 Hamburg – 28.\,11.\,1912 Berlin), \emph{Theaterleiter, Regisseur}|pw}{}\ledrightnote{\textcolor{blue}{Otto Brahm}}}{\lemma{\textnormal{\emph{betreffs Director Brahm}}}\Cendnote{\textnormal{\textcolor{blue}{Schnitzler} sandte mit einem nicht
                  überlieferten Brief das \emph{\textcolor{green}{Heinweh}\pwindex{Plessner, Elsa 22.\,8.\,1875 Wien – 7.\,5.\,1932 Alicante@\textsc{Plessner, Elsa} (22.\,8.\,1875 Wien – 7.\,5.\,1932 Alicante), \emph{Schriftstellerin}!Heimweh [dreiaktige Tragikomödie]@\strich\emph{Heimweh [dreiaktige Tragikomödie]}|pwk}} an \textcolor{blue}{Otto Brahm}\pwindex{Brahm, Otto 5.\,2.\,1856 Hamburg – 28.\,11.\,1912 Berlin@\textsc{Brahm, Otto} (5.\,2.\,1856 Hamburg – 28.\,11.\,1912 Berlin), \emph{Theaterleiter, Regisseur}|pwk}, der ihm am
                     21. 9. 1896 antwortete: »Frl. \textcolor{blue}{Plessner}\pwindex{Plessner, Elsa 22.\,8.\,1875 Wien – 7.\,5.\,1932 Alicante@\textsc{Plessner, Elsa} (22.\,8.\,1875 Wien – 7.\,5.\,1932 Alicante), \emph{Schriftstellerin}|pw} soll aber darum doch liebevoll gelesen
                     werden.« (\emph{Der Briefwechsel Arthur Schnitzler – Otto Brahm}.
                     Vollständige Ausgabe. Herausgegeben, eingeleitet und erläutert von Oskar
                     Seidlin. Tübingen: \emph{Niemeyer}{ }1975, S. 22.) }}}\label{K_L03703-6} thun soll? – ihm ein Abschrift meines
                  \textcolor{green}{Stückes}\pwindex{Plessner, Elsa 22.\,8.\,1875 Wien – 7.\,5.\,1932 Alicante@\textsc{Plessner, Elsa} (22.\,8.\,1875 Wien – 7.\,5.\,1932 Alicante), \emph{Schriftstellerin}!Heimweh [dreiaktige Tragikomödie]@\strich\emph{Heimweh [dreiaktige Tragikomödie]}|pwv}{}\ledrightnote{{$\rightarrow$}\emph{\textcolor{green}{Heimweh [dreiaktige Tragikomödie]}}}{ }\introOben{}senden\introOben{} mit gleichzeitiger Bezugnahme auf Sie, verehrter
               Meister? – – – Oder erst nach eventueller Antwort diesbezüglich von dort an Sie? – – \pend
           \pstart Mit Dank und Verehrung grüßt \spacefill\mbox{Elsa Plessner.}\pend{}\selectlanguage{ngerman}\endnumbering\briefempfaengerindex{Schnitzler, Arthur@\textsc{Schnitzler, Arthur}!zzzPlessner, Elsa@\emph{von Elsa Plessner}!1896-09-213@{21. 9. 1896}|)be}\mylabel{L03703h}  \normalsize

\doendnotes{C}
\bigskip
\vfill

\clearpage

\footnotesize

\lohead{\textsc{register}}

% Definiere theindex-Environment komplett neu ohne reledmac
\makeatletter
\renewenvironment{theindex}{%
  \section*{\indexname}%
  \setlength{\parindent}{0pt}%
  \setlength{\parskip}{0pt plus 0.3pt}%
  \let\item\@idxitem
}{%
  \clearpage
}
\makeatother

\IfFileExists{\jobname-pw.ind}{\input{\jobname-pw.ind}}{}

\end{document}

      