%% latex-korrekturansicht-vorspann.tex
%% Vorspann für die Korrekturansicht.
%% Lädt die gemeinsame Datei latex-vorspann.tex mit gesetztem Schalter.

\newif\ifkorrekturansicht
\korrekturansichttrue

\input{../tex-inputs/latex-vorspann}


               \section[Arthur Schnitzler an Robert Adam, 14. 8. 1929]{ Arthur Schnitzler an Robert Adam, 14. 8. 1929}\nopagebreak\mylabel{v}\rehead{ }\normalsize\beginnumbering\briefempfaengerindex{Adam, Robert@\textsc{Adam, Robert}!zzzSchnitzler, Arthur@\emph{von Arthur Schnitzler}!1929-08-141@{14. 8. 1929}|(be} \toendnotes[C]{\smallbreak\pagebreak[2]} \Standort{DLA, 96.34.2/32.}
\physDesc{Postkarte
\newline{}Handschrift: schwarze Tinte, lateinische Kurrent\newline{}Versand: 1) nachgesandt nach \textcolor{pink}{\textsc{Bad-Aussee}}, \textcolor{pink}{\textsc{Meraner Haus}}« 2) Stempel: »\nobreak{}\oindex{IX., Alsergrund@\textbf{IX., Alsergrund}, \emph{Bezirk (A.BZK)}|pwk}9/\textsubscript{1} Wien
                                        38, 15. VIII. 29, 18\nobreak{}«. 3) Stempel: »\nobreak{}\oindex{XII., Meidling@\textbf{XII., Meidling}, \emph{Bezirk (A.BZK)}|pwk}12/\textsubscript{1} Wien
                                        82, 16. VIII. 29, 19\nobreak{}«. }\toendnotes[C]{\smallbreak}\pstart{}{\pb}\label{T_L02520-1v}\edtext{\textcolor{gray}{\textbf{A. S.}}}{\lemma{\textnormal{\emph{A. S.}}}\Cendnote{\textnormal{ovaler Absenderkleber}}}\label{T_L02520-1h}\pend{}\pstart{}\textcolor{pink}{\textcolor{gray}{\textbf{WIEN, XVIII.}}}{}\ledrightnote{\textcolor{pink}{XVIII., Währing}}\pend{}\pstart{}\textcolor{pink}{\textcolor{gray}{\textbf{STERNWARTESTR. 71}}}{}\ledrightnote{\textcolor{pink}{Sternwartestraße}}\pend{}{\bigskip}\pstart{}{\pb}Hrn Ober L.\textcolor{gray}{g}r. Rath\pend{}\pstart{}Dr. Robert Adam Pollak\pend{}\pstart{}\textcolor{pink}{Wien XIII}{}\ledrightnote{\textcolor{pink}{XIII., Hietzing}}\pend{}\pstart{}\textcolor{pink}{Meidlinger Hptstr 58}{}\ledrightnote{\textcolor{pink}{Meidlinger Hauptstraße}}.\pend{}{\bigskip}\pstart
           \raggedleft{}{\pb}\textcolor{pink}{Wien}{}\ledrightnote{\textcolor{pink}{Wien}}, 14/8 929\pend
           \pstart{}verehrter Herr Doctor, \pend\pstart
           Ihren \textcolor{green}{Aufsatz}{}\ledrightnote{→\textcolor{green}{Zur Frage des Laienrichtertums beim Handelsgericht}}, so präcise und
                    so klar hab ich mit aufrichtigem Vergnügen gelesen. Ich danke Ihnen sehr, auch
                    für den lieben Brief und grüße Sie herzlichst.\pend
           \pstart
           {\pb}Ihr sehr ergebner{\\[\baselineskip]}\spacefill\mbox{ArthSchnitzler}\pend
           \leftskip=0em{}\endnumbering\briefempfaengerindex{Adam, Robert@\textsc{Adam, Robert}!zzzSchnitzler, Arthur@\emph{von Arthur Schnitzler}!1929-08-141@{14. 8. 1929}|)be}\mylabel{h}  \normalsize

\doendnotes{C}
\bigskip
\vfill

\clearpage

\footnotesize

\lohead{\textsc{register}}

% Definiere theindex-Environment komplett neu ohne reledmac
\makeatletter
\renewenvironment{theindex}{%
  \section*{\indexname}%
  \setlength{\parindent}{0pt}%
  \setlength{\parskip}{0pt plus 0.3pt}%
  \let\item\@idxitem
}{%
  \clearpage
}
\makeatother

\IfFileExists{\jobname-pw.ind}{\input{\jobname-pw.ind}}{}

\end{document}

      