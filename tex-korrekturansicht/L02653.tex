%% latex-korrekturansicht-vorspann.tex
%% Vorspann für die Korrekturansicht.
%% Lädt die gemeinsame Datei latex-vorspann.tex mit gesetztem Schalter.

\newif\ifkorrekturansicht
\korrekturansichttrue

\input{../tex-inputs/latex-vorspann}


               \section[Else Lasker-Schüler an Arthur Schnitzler, 10. 12. 1924]{ Else Lasker-Schüler an Arthur Schnitzler, 10. 12. 1924}\nopagebreak\mylabel{v}\rehead{ }\normalsize\beginnumbering\briefempfaengerindex{Schnitzler, Arthur@\textsc{Schnitzler, Arthur}!zzzLasker-Schueler, Else@\emph{von Else Lasker-Schüler}!1924-12-102@{10. 12. 1924}|(be} \toendnotes[C]{\smallbreak\pagebreak[2]} \Standort{unbekannt, Privatbesitz, \emph{ohne Signatur}.}
\physDesc{Brief, 5 Blätter, 6 Seiten (Paginierung 2–5)
\newline{}Handschrift: schwarze Tinte, lateinische Kurrent
\newline{}Schnitzler: 1) mit Bleistift auf der ersten Seite Vermerk: »\textsc{Laske\textcolor{gray}{r} Schüler}« und »\textcolor{blue}{L ſch}«, auf der zweiten Seite »2/1«, auf den
                                 Seiten zwei bis vier außerdem die Datierung »10/12 24« 2) mit rotem Buntstift Vermerk: »(\textsc{Ihr
                                          \textcolor{blue}{Sohn}})«\newline{}Ordnung: von unbekannter Hand mit rotem Buntstift zwölf
                                 Unterstreichungen \newline{}Zusatz: Der Brief lässt sich 2002 im Besitz des
                                 Antiquariats Eberhard Köstler in Tutzing nachweisen.
                                    2006 wurde er an das Antiquariat Inlibris in Wien
                                 verkauft. Der weitere Verbleib ist ungeklärt. Ebenso ungeklärt
                                 bleibt, warum das Original des Briefes nicht im Nachlass
                                 Schnitzlers überliefert ist. Die im Nachlass befindliche Abschrift
                                 weist handschriftliche Spuren \textcolor{blue}{Schnitzler}s aus, wurde aber nicht mit einer von
                                 Schnitzlers Schreibmaschinen getippt. Eine wahrscheinliche (obzwar
                                 nicht häufiger nachweisbare) Erklärung ist, dass \textcolor{blue}{Schnitzler} selbst den Brief an
                                 eine Autografensammlerin oder einen -sammler schenkte. Grundlage
                                 unserer Transkription stellt eine Kopie dar. }\Standort{DLA, A:Schnitzler, HS.1985.1.3875.}
\physDesc{Brief, 1 Blatt, 4 Seiten, maschinelle Abschrift
\newline{}Schreibmaschine
\newline{}Schnitzler: mit rotem Buntstift Vermerk: »\uline{\textsc{Else Lasker Schule{[}r{]}}}« }\buchAbdrucke{\weitereDrucke{Else Lasker-Schüler: \emph{Werke und Briefe. Kritische Ausgabe. Band 7: Briefe
                        1914–1924}. Hg. Karl Jürgen Skrodzki. Frankfurt am Main: \emph{Jüdischer Verlag Verlag} 2004, S. 315–316.} }\toendnotes[C]{\smallbreak}\pstart
           \noindent{}10. XII. 24\hfill \textcolor{pink}{Berlin W Motztr. 78}{}\ledrightnote{\textcolor{pink}{Motzstraße}}{ }\pend
           \pstart
           \raggedleft{}\textcolor{pink}{Hôtel Koschel}{}\ledrightnote{\textcolor{pink}{Hotel Koschel}}\pend
           \pstart\center{}Hochzuverehrender Herr Doktor und lieber Dichter\pend\pstart
           Ich fühle es mit Bestimmtheit, daß ich diesen Brief nicht nur in \introOben{}den\introOben{} Wind schreibe. Wenn man wenigstens immer in den Wind schriebe, aber man
               schreibt ja nicht an kleinherzige Menschen. Es hat mir kein Mensch geraten an Sie,
               lieber Dichter, zu schreiben, es überkam mich \strikeout{\textcolor{gray}{×}}, Sie um eine große Gefälligkeit zu bitten, nämlich mit {\pb}meinem geliebten
                  \textcolor{blue}{Kinde}{}\ledrightnote{→\textcolor{blue}{Paul Lasker-Schüler}}, meinem \textcolor{blue}{Sohn}{}\ledrightnote{→\textcolor{blue}{Paul Lasker-Schüler}} zu sprechen. Ich bin
               Else Lasker-Schüler; mein \textcolor{blue}{Junge}{}\ledrightnote{→\textcolor{blue}{Paul Lasker-Schüler}} wohnt in \textcolor{pink}{Wien \introOben{}VIII.\introOben{} Florianigasse 47/49 Stiege II. \introOben{}Thüre 25\introOben{}}{}\ledrightnote{\textcolor{pink}{Florianigasse}} in einem grossen Zimmer bei einer netten \textcolor{pink}{\textcolor{blue}{Wirtin}{}\ledrightnote{→\textcolor{blue}{Elise Schiedlbauer}}}{}\ledrightnote{→\textcolor{pink}{Gastwirtschaft Schiedlbauer}}. Wenn Sie ihm schreiben lassen, kommt er zur angegebenen Zeit, Herr Doktor. Ich
               möchte Ihnen so viel sagen; schon wie ich im Januar in
                  \textcolor{pink}{Wien}{}\ledrightnote{\textcolor{pink}{Wien}} war. Ich bekam dort Scharlach und
               Diphteritie, saß dabei vier Wochen in Flanell gehüllt im \textcolor{pink}{Cafe Central}{}\ledrightnote{\textcolor{pink}{Café Central}} am Fenster und ich glaube das herrliche \textcolor{pink}{Wien}{}\ledrightnote{\textcolor{pink}{Wien}}er Trinkwasser heilte mich. Ich habe in \textcolor{pink}{München}{}\ledrightnote{\textcolor{pink}{München}} jetzt Gelegenheit gehabt, meinen \textcolor{blue}{Paul}{}\ledrightnote{\textcolor{blue}{Paul Lasker-Schüler}} zeichnerisch anzubringen {\pb}aber er
               liebt \textcolor{pink}{Wien}{}\ledrightnote{\textcolor{pink}{Wien}} so und bat mich doch dort bleiben zu
               können. Zunächst versuchte er mit einem \label{K_L02653-2v}\edtext{\textcolor{blue}{Freund}{}\ledrightnote{→\textcolor{blue}{[?? Freund 1 von Paul Lasker Schüler]}}}{\lemma{\textnormal{\emph{Freund}}}\Cendnote{\textnormal{nicht identifiziert}}}\label{K_L02653-2h} Plakate zu
               zeichnen für Geschäfte. Einen Monat ging das, aber nun ist grosser Stillstand.
                  Nu\textcolor{gray}{n} möchte ich so gern, hochzuverehrender Herr Doktor, daß Sie
               mein liebes \textcolor{blue}{Kind}{}\ledrightnote{→\textcolor{blue}{Paul Lasker-Schüler}} kennen
               lernen; er ist der liebste \uline{kindlichste}{ }\textcolor{blue}{Junge}{}\ledrightnote{→\textcolor{blue}{Paul Lasker-Schüler}}, den ich fast kenne –
               im Grunde;– aber was man \uline{mir} nicht antut – vielleicht
               aus Feigheit, – muß der arme \textcolor{blue}{Junge}{}\ledrightnote{→\textcolor{blue}{Paul Lasker-Schüler}} erleiden. Ich weiß \uline{wie} unerhört er
                  \label{K_L02653-3v}\edtext{in \textcolor{pink}{Wien}{}\ledrightnote{\textcolor{pink}{Wien}} angeschwärzt}{\lemma{\textnormal{\emph{in Wien angeschwärzt}}}\Cendnote{\textnormal{nicht ermittelt}}}\label{K_L02653-3h}
               wurde; niemand spricht von seiner Bescheidenheit, auch in künstlerischen Dingen. {\pb}\uline{Darum} wird er sich alleine nie durchsetzen, ich meine
               – weiterkommen – äußerlich – was doch \introOben{}hier\introOben{} sein muss. Er
               giebt sich \strikeout{\textcolor{gray}{so}} Mühe, aber es gelingt ihm nicht und ich tue ja alles was in meiner Kraft
               liegt. Danach muß er stets genug zu essen und Anzuziehen haben und wenn er \uline{nicht} charmant seinen \label{K_L02653-5v}\edtext{Besuch}{\lemma{\textnormal{\emph{Besuch}}}\Cendnote{\textnormal{kein
                  Zusammentreffen \textcolor{blue}{Paul Lasker-Schüler}s mit \textcolor{blue}{Schnitzler}
                     ist belegt}}}\label{K_L02653-5h} bei Ihnen machen sollte, so kann ich nichts dafür. Wirklich es
                  \textcolor{gray}{l}eben nicht zwei Menschen mehr, die \label{K_L02653-6v}\edtext{verfolgter}{\lemma{\textnormal{\emph{verfolgter}}}\Cendnote{\textnormal{Womöglich deutete Lasker-Schüler hier antisemitische Anfeindungen an.}}}\label{K_L02653-6h} sind
               wie wir zwei, mein \textcolor{blue}{Junge}{}\ledrightnote{→\textcolor{blue}{Paul Lasker-Schüler}} und
               ich. Herr Doktor, ich bitte Sie herzlich als Mensch und als Dichterin, \introOben{}(\introOben{}und nie werde ich es Ihnen vergessen) meinen \textcolor{blue}{Jungen}{}\ledrightnote{→\textcolor{blue}{Paul Lasker-Schüler}} einmal einzuladen. \textcolor{blue}{Wedekind}{}\ledrightnote{\textcolor{blue}{Frank Wedekind}}{ }\introOben{}war direkt begeistert von ihm in \textcolor{pink}{Zürich}{}\ledrightnote{\textcolor{pink}{Zürich}}\introOben{} und Prof. \label{K_L02653-7v}\edtext{\textcolor{blue}{Einstein}{}\ledrightnote{\textcolor{blue}{Albert Einstein}}}{\lemma{\textnormal{\emph{Einstein}}}\Cendnote{\textnormal{\textcolor{blue}{Albert Einstein} und Else Lasker-Schüler
                  lebten beide in der \textcolor{pink}{Haberlandstraße 5} (heute 3) in \textcolor{pink}{Berlin}.}}}\label{K_L02653-7h} fand ihn prachtvoll{[}.{]}{ }{\pb}Vielleicht können Sie ihm raten, wohin er sich wenden soll, Ihr Wort in \textcolor{pink}{Wien}{}\ledrightnote{\textcolor{pink}{Wien}} gilt ja. Was kann ich für Sie je tun? Kommen
               Sie bald nach \textcolor{pink}{Berlin}{}\ledrightnote{\textcolor{pink}{Berlin}}? Sehe ich Sie? Denken Sie,
               ich kenne nur ein \label{K_L02653-8v}\edtext{Schauspiel}{\lemma{\textnormal{\emph{Schauspiel}}}\Cendnote{\textnormal{nicht ermittelt}}}\label{K_L02653-8h} von Ihnen; ich gehe
               so selten ins Theater, so erschöpft bin ich am Abend. Ich bitte Sie mir die Freude zu
               machen, Herr Doktor, und es wäre so schön mein \textcolor{blue}{Junge}{}\ledrightnote{→\textcolor{blue}{Paul Lasker-Schüler}} und seine \label{K_L02653-9v}\edtext{\textcolor{blue}{Freunde}{}\ledrightnote{→\textcolor{blue}{[?? Freund 1 von Paul Lasker Schüler]}{\newline}→\textcolor{blue}{[?? Freund 2 von Paul Lasker Schüler]}}}{\lemma{\textnormal{\emph{Freunde}}}\Cendnote{\textnormal{nicht identifiziert}}}\label{K_L02653-9h} würden mal wo
               eingeladen in Familie, alle drei, entzückende Bengels. Als wir noch in \textcolor{pink}{Berlin}{}\ledrightnote{\textcolor{pink}{Berlin}} waren, gingen wir oft zusammen ins Kino,
               mein Sanatorium. Ich grüße Sie, hochwerter lieber Dichter, Ihre {\\}\spacefill\mbox{Else Lasker-Schüler}{ }{\\}der Prinz von Theben {[}Mondsichel mit
                  Stern{]}\pend
           \pstart
           \noindent{}Was kann ich je für Sie tun?\pend
           \pstart
           {\pb}\label{T_L02653-1v}\edtext{\textcolor{pink}{Motzstr. 78 Berlin W.}{}\ledrightnote{\textcolor{pink}{Motzstraße}}{\\}\textcolor{pink}{Hôtel Koschel}{}\ledrightnote{\textcolor{pink}{Hotel Koschel}}{\\}{[}Segelschiff auf Wasser{]}{\\}mit lieben Grüßen}{\lemma{\textnormal{\emph{Motzstr. 78 … Grüßen}}}\Cendnote{\textnormal{auf der Rückseite
                  des letzten Blattes, dieses ins Querformat gedreht und in der linken oberen Ecke
                  beschrieben}}}\label{T_L02653-1h}\pend
           \endnumbering\briefempfaengerindex{Schnitzler, Arthur@\textsc{Schnitzler, Arthur}!zzzLasker-Schueler, Else@\emph{von Else Lasker-Schüler}!1924-12-102@{10. 12. 1924}|)be}\mylabel{h}  \normalsize

\doendnotes{C}
\bigskip
\vfill

\clearpage

\footnotesize

\lohead{\textsc{register}}

% Definiere theindex-Environment komplett neu ohne reledmac
\makeatletter
\renewenvironment{theindex}{%
  \section*{\indexname}%
  \setlength{\parindent}{0pt}%
  \setlength{\parskip}{0pt plus 0.3pt}%
  \let\item\@idxitem
}{%
  \clearpage
}
\makeatother

\IfFileExists{\jobname-pw.ind}{\input{\jobname-pw.ind}}{}

\end{document}

      