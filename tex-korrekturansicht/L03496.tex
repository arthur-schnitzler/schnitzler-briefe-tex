%% latex-korrekturansicht-vorspann.tex
%% Vorspann für die Korrekturansicht.
%% Lädt die gemeinsame Datei latex-vorspann.tex mit gesetztem Schalter.

\newif\ifkorrekturansicht
\korrekturansichttrue

\input{../tex-inputs/latex-vorspann}


\renewcommand{\erwaehntePersonen}{Personen: Reinhard Ernst Petermann, Felix Salten, Ottilie Salten}
\renewcommand{\erwaehnteOrte}{Orte: Dalmatien, Pula, Triest, Venedig, Wien}
\renewcommand{\erwaehnteWerke}{Werke: Der Tod des Junggesellen. Novelle, Die Zeit, Die Zeit. Wiener Wochenschrift, Führer durch Dalmatien, Komtesse Mizzi oder: Der Familientag, Neue Freie Presse, Unsichere Reise, Österreichische Rundschau}
\section[Felix Salten an Arthur Schnitzler, {[}zwischen 19. und 21. 4. 1908{]}]{Felix Salten an Arthur Schnitzler,
               {[}zwischen 19. und 21. 4. 1908{]}}
\nopagebreak\mylabel{v}
\rehead{ }\normalsize\beginnumbering\briefempfaengerindex{Schnitzler, Arthur@\textsc{Schnitzler, Arthur}!zzzSalten, Felix@\emph{von Felix Salten}!1@{{[}zwischen 19. und 21. 4. 1908{]}}|(be}
\toendnotes[C]{\smallbreak\pagebreak[2]}\Standort{CUL, Schnitzler, B 89, B 1.}
\physDesc{Brief, 1 Blatt, 1 Seite, 255 Zeichen
\newline{}Handschrift: Bleistift, lateinische Kurrent
\newline{}Schnitzler: mit Bleistift datiert: »Ende April \textcolor{gray}{08}« und Vermerk »\textsc{Salten}« 
\newline{}Ordnung: mit Bleistift von unbekannter Hand nummeriert: »245?« }\toendnotes[C]{\smallbreak}
\pstart{}{\pb}Lieber,\pend
\pstart
           bitte geben Sie dem Boten das \label{K_L03496-1v}\edtext{\textcolor{green}{\textcolor{pink}{dalmati}{}\ledrightnote{{$\rightarrow$}\textcolor{pink}{Dalmatien}}nische Buch}{}\ledrightnote{\textcolor{green}{Führer durch Dalmatien}}}{\lemma{\textnormal{\emph{dalmatinische Buch}}}\Cendnote{\textnormal{nicht identifiziert; möglicherweise der
                     \emph{\textcolor{green}{Führer durch Dalmatien}}
                  (1899) von \textcolor{blue}{Reinhard E.
                     Petermann} für die bevorstehende Reise? }}}\label{K_L03496-1h} und seien Sie bestens dafür
               bedankt. Die »\label{K_L03496-2v}\edtext{\textcolor{green}{Komtesse Mizzi}{}\ledrightnote{\textcolor{green}{Komtesse Mizzi oder: Der Familientag}}}{\lemma{\textnormal{\emph{Komtesse Mizzi}}}\Cendnote{\textnormal{\textcolor{blue}{Arthur Schnitzler}: \emph{\textcolor{green}{Komtesse Mizzi oder: Der Familientag}}. In: \emph{\textcolor{green}{Neue Freie Presse}}, Nr. 15.684, 19. 4. 1908, Osterbeilage, S. 31–35. Durch
                  das Erscheinungsdatum kann \textcolor{blue}{Schnitzler}s
                  Datierung auf »Ende April \textcolor{gray}{08}« eingeschränkt werden. Nach hinten lässt sich ebenfalls eine zeitliche Einschränkung treffen. 
                  \textcolor{blue}{Salten}s erstes Feuilleton von der Reise ist mit »\textcolor{pink}{Venedig}, 23. April«
                  datiert (\textcolor{blue}{Felix Salten}: \emph{\textcolor{green}{Unsichere Reise}}. In: \emph{\textcolor{green}{Die Zeit}}, 
                     Jg. 7, Nr. 2.008, 26. 4. 1908, Morgenblatt, S. 1–3, hier 3.). Geschildert wird, dass der Erzähler/\textcolor{blue}{Salten} von \textcolor{pink}{Triest} aus eine Schifffahrt entlang \textcolor{pink}{dalmatischen Küste}
                  unternehmen wollte, aber das Schiff bereits in \textcolor{pink}{Pula} verlassen hat. Er fuhr mit dem Zug zurück
                  nach \textcolor{pink}{Triest}, wo er in der Nacht den Dampfer nach \textcolor{pink}{Venedig} bestieg. Sofern die Reise akkurat 
                  beschrieben ist, musste er spätestens am 21. 4. 1908 in \textcolor{pink}{Wien} den Nachtzug bestiegen
                  haben.}}}\label{K_L03496-2h}«, die ich eben las, ist reizend. Die andere \label{K_L03496-3v}\edtext{\textcolor{green}{Geschichte}{}\ledrightnote{{$\rightarrow$}\textcolor{green}{Der Tod des Junggesellen. Novelle}}
                in der »\textcolor{green}{Zeit}{}\ledrightnote{\textcolor{green}{Österreichische Rundschau}{\newline}\textcolor{green}{Die Zeit. Wiener Wochenschrift}}}{\lemma{\textnormal{\emph{Geschichte
                in der »Zeit}}}\Cendnote{\textnormal{\textcolor{blue}{Arthur Schnitzler}: \emph{\textcolor{green}{Der Tod des Junggesellen. Novelle}}. In: \emph{\textcolor{green}{Österreichische Rundschau}}, Bd. 15, 1. 4. 1908, S. 19–26. Die \emph{\textcolor{green}{Österreichische Rundschau}} galt als Nachfolger der Wochenschrift \emph{\textcolor{green}{Die Zeit}}.}}}\label{K_L03496-3h}« nehm’ ich mir auf die Reise
               mit.\pend
           
\pstart
           Viele herzliche Grüße von \textcolor{blue}{uns}{}\ledrightnote{{$\rightarrow$}\textcolor{blue}{Ottilie Salten}} zu Ihnen {\\[\baselineskip]}Ihr \spacefill\mbox{Salten}\pend
           \leftskip=0em{}\endnumbering\briefempfaengerindex{Schnitzler, Arthur@\textsc{Schnitzler, Arthur}!zzzSalten, Felix@\emph{von Felix Salten}!1908-04-191@{{[}zwischen 19. und 21. 4. 1908{]}}|)be}\mylabel{h}  \normalsize

\doendnotes{C}
\bigskip
\vfill

\clearpage

\footnotesize

\lohead{\textsc{register}}

% Definiere theindex-Environment komplett neu ohne reledmac
\makeatletter
\renewenvironment{theindex}{%
  \section*{\indexname}%
  \setlength{\parindent}{0pt}%
  \setlength{\parskip}{0pt plus 0.3pt}%
  \let\item\@idxitem
}{%
  \clearpage
}
\makeatother

\IfFileExists{\jobname-pw.ind}{\input{\jobname-pw.ind}}{}

\end{document}

      