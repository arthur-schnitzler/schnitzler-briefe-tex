%% latex-korrekturansicht-vorspann.tex
%% Vorspann für die Korrekturansicht.
%% Lädt die gemeinsame Datei latex-vorspann.tex mit gesetztem Schalter.

\newif\ifkorrekturansicht
\korrekturansichttrue

\input{../tex-inputs/latex-vorspann}


               \section[Richard Beer-Hofmann an Arthur Schnitzler, 2. 8. 1900]{ Richard Beer-Hofmann an Arthur Schnitzler, 2. 8. 1900}\nopagebreak\mylabel{v}\rehead{ }\normalsize\beginnumbering\briefempfaengerindex{Schnitzler, Arthur@\textsc{Schnitzler, Arthur}!zzzBeer-Hofmann, Richard@\emph{von Richard Beer-Hofmann}!1900-08-021@{2. 8. 1900}|(be} \toendnotes[C]{\smallbreak\pagebreak[2]} \Standort{CUL, Schnitzler, B 8.}
\physDesc{Brief, 1 Blatt, 2 Seiten
\newline{}Handschrift: schwarze Tinte, lateinische Kurrent\newline{}Ordnung: mit Bleistift von unbekannter Hand nummeriert: »156« }\buchAbdrucke{\weitereDrucke{Arthur Schnitzler, Richard Beer-Hofmann: \emph{Briefwechsel 1891–1931}. Hg. Konstanze Fliedl. Wien, Zürich: \emph{Europaverlag} 1992, S. 149.} }\toendnotes[C]{\smallbreak}\pstart
           \raggedleft{}{\pb}\textcolor{pink}{Alt-Aussee}{}\ledrightnote{\textcolor{pink}{Altaussee}}{ }2/VIII 1900\pend
           \pstart
           Lieber Arthur! Bei Durchsicht des \textcolor{green}{Reisebuches}{}\ledrightnote{→\textcolor{green}{Deutsche Alpen}} stoße ich auf folgende Tour die ich \textcolor{pink}{Moserboden}{}\ledrightnote{\textcolor{pink}{Mooserboden}}, \textcolor{pink}{Gerlos}{}\ledrightnote{\textcolor{pink}{Gerlos}} etc. vorziehen würde: Von \textcolor{pink}{Jenbach}{}\ledrightnote{\textcolor{pink}{Jenbach}}
               durchs \textcolor{pink}{Zillerthal}{}\ledrightnote{\textcolor{pink}{Zillertal}} nach \textcolor{pink}{\uline{Sterzing}}{}\ledrightnote{\textcolor{pink}{Sterzing}} (11 \introOben{}Bahn-\introOben{}Minuten von \textcolor{pink}{Gossensass}{}\ledrightnote{\textcolor{pink}{Gossensass}})\pend
           \settowidth{\longeste}{Folgende Eintheilung:}\settowidth{\longestz}{über Zell im Zillerth. nach Mayerhofen7.50}\settowidth{\longestd}{}\settowidth{\longestv}{}\settowidth{\longestf}{}\addtolength\longeste{1em}
        \addtolength\longestz{1em}
      \pstart\noindent\makebox[\the\longeste][l]{Folgende Eintheilung:}\makebox[\the\longestz][l]{\strikeout{7.2}{ }\textcolor{pink}{Salzburg}{}\ledrightnote{\textcolor{pink}{Salzburg}}{ }7.28 Früh}
                  \pend\pstart\noindent\makebox[\the\longeste][l]{}\makebox[\the\longestz][l]{\textcolor{pink}{Jenbach}{}\ledrightnote{\textcolor{pink}{Jenbach}}{ }12.28 Mittagessen}
                  \pend\pstart\noindent\makebox[\the\longeste][l]{}\makebox[\the\longestz][l]{Post von \textcolor{pink}{Jenbach}{}\ledrightnote{\textcolor{pink}{Jenbach}} ab um 2.30}
                  \pend\pstart\noindent\makebox[\the\longeste][l]{}\makebox[\the\longestz][l]{über \textcolor{pink}{Zell im Zillerth.}{}\ledrightnote{\textcolor{pink}{Zell am Ziller}} nach \textcolor{pink}{Mayerhofen}{}\ledrightnote{\textcolor{pink}{Mayrhofen}}{ }7.50}
                  \pend\pstart\noindent\makebox[\the\longeste][l]{}\makebox[\the\longestz][l]{Mit Wagen also vielleicht 3 ½–4 Stunden}
                  \pend\pstart
           In \textcolor{pink}{Mayerhofen}{}\ledrightnote{\textcolor{pink}{Mayrhofen}} übernachten – dann über \introOben{}den\introOben{}{ }\textcolor{pink}{Ze{\geminationm}grund}{}\ledrightnote{\textcolor{pink}{Zemmgrund}}, \textcolor{pink}{Schwarzensteingrund}{}\ledrightnote{\textcolor{pink}{Schwarzensteingrund}} und das \textcolor{pink}{Pfitscher Joch}{}\ledrightnote{\textcolor{pink}{Pfitscher Joch}} nach \textcolor{pink}{Sterzing}{}\ledrightnote{\textcolor{pink}{Sterzing}}
               15–16 Stunden. = 2 Tage. Im \textcolor{green}{Mayer}{}\ledrightnote{\textcolor{green}{Deutsche Alpen}} II. Band
               pag. 229. Jedenfalls – nach dem Reisebuch – viel lohnender als \textcolor{pink}{Gerlos}{}\ledrightnote{\textcolor{pink}{Gerlos}} etc. Eventuell\pend
           \settowidth{\longeste}{I Tag.}\settowidth{\longestz}{Salzburg ab 3.12 Nachm.}\settowidth{\longestd}{}\settowidth{\longestv}{}\settowidth{\longestf}{}\addtolength\longeste{1em}
        \addtolength\longestz{1em}
      \pstart\noindent\makebox[\the\longeste][l]{}\makebox[\the\longestz][l]{\textcolor{pink}{Salzburg}{}\ledrightnote{\textcolor{pink}{Salzburg}} ab 3.12 Nachm.}
                  \pend\pstart\noindent\makebox[\the\longeste][l]{I Tag.}\makebox[\the\longestz][l]{\textcolor{pink}{Jenbach}{}\ledrightnote{\textcolor{pink}{Jenbach}}{ }8.45 Abends}
                  \pend\pstart\noindent\makebox[\the\longeste][l]{}\makebox[\the\longestz][l]{übernachten}
                  \pend\settowidth{\longeste}{II Tag}\settowidth{\longestz}{und bequemer Reitweg bis zum Breitlahner (zu Fuß 5 ½ Stunden)}\settowidth{\longestd}{}\settowidth{\longestv}{}\settowidth{\longestf}{}\addtolength\longeste{1em}
        \addtolength\longestz{1em}
      \pstart\noindent\makebox[\the\longeste][l]{}\makebox[\the\longestz][l]{\textcolor{pink}{Jenbach}{}\ledrightnote{\textcolor{pink}{Jenbach}} ab. Post. 8.00 Früh}
                  \pend\pstart\noindent\makebox[\the\longeste][l]{II Tag}\makebox[\the\longestz][l]{\textcolor{pink}{Mayerhofen}{}\ledrightnote{\textcolor{pink}{Mayrhofen}}. 1.30.
                     Mittagessen}
                  \pend\pstart\noindent\makebox[\the\longeste][l]{}\makebox[\the\longestz][l]{und bequemer Reitweg bis zum \textcolor{pink}{Breitlahner}{}\ledrightnote{\textcolor{pink}{Gasthaus Breitlahner}}
                     (zu Fuß 5 ½ Stunden)}
                  \pend\settowidth{\longeste}{III Tag}\settowidth{\longestz}{(nach Sterzing 10-11 Stunden)}\settowidth{\longestd}{}\settowidth{\longestv}{}\settowidth{\longestf}{}\addtolength\longeste{1em}
        \addtolength\longestz{1em}
      \pstart\noindent\makebox[\the\longeste][l]{III Tag }\makebox[\the\longestz][l]{(nach \textcolor{pink}{Sterzing}{}\ledrightnote{\textcolor{pink}{Sterzing}} 10-11 Stunden)}
                  \pend\pstart
           {\pb}Überlegen Sie, und schreiben Sie
               mir noch genau \uline{wann} sie in \textcolor{pink}{Salzburg}{}\ledrightnote{\textcolor{pink}{Salzburg}} sind.\pend
           \pstart
           \textcolor{pink}{Gerlospass}{}\ledrightnote{\textcolor{pink}{Gerlospass}} mißbillige ich. – \textcolor{pink}{Moserboden}{}\ledrightnote{\textcolor{pink}{Mooserboden}} nicht.\pend
           \pstart
           Herzlichst Ihr{\\[\baselineskip]}\spacefill\mbox{Richard}\pend
           \leftskip=0em{}\endnumbering\briefempfaengerindex{Schnitzler, Arthur@\textsc{Schnitzler, Arthur}!zzzBeer-Hofmann, Richard@\emph{von Richard Beer-Hofmann}!1900-08-021@{2. 8. 1900}|)be}\mylabel{h}  \normalsize

\doendnotes{C}
\bigskip
\vfill

\clearpage

\footnotesize

\lohead{\textsc{register}}

% Definiere theindex-Environment komplett neu ohne reledmac
\makeatletter
\renewenvironment{theindex}{%
  \section*{\indexname}%
  \setlength{\parindent}{0pt}%
  \setlength{\parskip}{0pt plus 0.3pt}%
  \let\item\@idxitem
}{%
  \clearpage
}
\makeatother

\IfFileExists{\jobname-pw.ind}{\input{\jobname-pw.ind}}{}

\end{document}

      