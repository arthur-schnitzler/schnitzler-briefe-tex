%% latex-korrekturansicht-vorspann.tex
%% Vorspann für die Korrekturansicht.
%% Lädt die gemeinsame Datei latex-vorspann.tex mit gesetztem Schalter.

\newif\ifkorrekturansicht
\korrekturansichttrue

\input{../tex-inputs/latex-vorspann}


\renewcommand{\erwaehnteOrte}{Orte: Berlin, Skandinavien, Wien}
\renewcommand{\erwaehnteWerke}{Werke: Freiwild. Schauspiel in 3 Akten}
\section[ Felix Salten an Arthur Schnitzler, {[}27. 8. 1896{]}]{Felix Salten an Arthur Schnitzler, {[}27. 8. 1896{]}}
\nopagebreak\mylabel{v}
\rehead{ }\normalsize\beginnumbering\briefempfaengerindex{Schnitzler, Arthur@\textsc{Schnitzler, Arthur}!zzzSalten, Felix@\emph{von Felix Salten}!1896-08-271@{{[}27. 8. 1896{]}}|(be}
\toendnotes[C]{\smallbreak\pagebreak[2]}\Standort{CUL, Schnitzler, B 89, A 1.}
\physDesc{Brief, 1 Blatt, 1 Seite, 348 Zeichen
\newline{}Handschrift: schwarze Tinte, lateinische Kurrent
\newline{}Schnitzler: mit Bleistift datiert: »2\substVorne{}\textsuperscript{9}\substDazwischen{}7\substHinten{}/8 96« 
\newline{}Ordnung: mit Bleistift von unbekannter Hand nummeriert: »78« }\toendnotes[C]{\smallbreak}
\pstart
           \raggedleft{}{\pb}Donnerstag.\pend
           
\pstart
           Lieber Freund, ich bin seit heute{ }\textcolor{pink}{hier}{}\ledrightnote{{$\rightarrow$}\textcolor{pink}{Wien}}, und freue mich sehr, Sie
               recht \label{K_L03179-1v}\edtext{bald wieder zu sehen}{\lemma{\textnormal{\emph{bald wieder zu sehen}}}\Cendnote{\textnormal{Nachweislich sahen sich die beiden am 29. 8. 1896
                  wieder.}}}\label{K_L03179-1h}. Es gibt Vieles zu erzählen. Den \label{K_L03179-2v}\edtext{»\textcolor{green}{Freiwild}{}\ledrightnote{\textcolor{green}{Freiwild. Schauspiel in 3 Akten}}« bekomme ich
               doch zu hören}{\lemma{\textnormal{\emph{»Freiwild« … hören}}}\Cendnote{\textnormal{\textcolor{blue}{Schnitzler} hatte \textcolor{blue}{Salten} bereits am 3. 5. 1896 aus dem \emph{\textcolor{green}{Freiwild}} vorgelesen.}}}\label{K_L03179-2h}, nicht? Ich werde mich dafür
               revanchiren. Nach \label{K_L03179-3v}\edtext{\textcolor{pink}{Berlin}{}\ledrightnote{\textcolor{pink}{Berlin}}}{\lemma{\textnormal{\emph{Berlin}}}\Cendnote{\textnormal{\textcolor{blue}{Schnitzler} war zwischen 22. 8. 1896 und 26. 8. 1896 – auf dem
                  Rückweg von seiner \textcolor{pink}{Skandinavien}reise – in \textcolor{pink}{Berlin} gewesen.}}}\label{K_L03179-3h} konnte ich Ihnen nichts
               mehr schreiben, ich hatte Ihre Karte verlegt\textcolor{gray}{,} und wusste keine
               Adreße.\pend
           
\pstart
           Also auf bald, {\\[\baselineskip]}herzlichst Ihr {\\[\baselineskip]}\spacefill\mbox{Salten}\pend
           \leftskip=0em{}\endnumbering\briefempfaengerindex{Schnitzler, Arthur@\textsc{Schnitzler, Arthur}!zzzSalten, Felix@\emph{von Felix Salten}!1896-08-271@{{[}27. 8. 1896{]}}|)be}\mylabel{h}  \normalsize

\doendnotes{C}
\bigskip
\vfill

\clearpage

\footnotesize

\lohead{\textsc{register}}

% Definiere theindex-Environment komplett neu ohne reledmac
\makeatletter
\renewenvironment{theindex}{%
  \section*{\indexname}%
  \setlength{\parindent}{0pt}%
  \setlength{\parskip}{0pt plus 0.3pt}%
  \let\item\@idxitem
}{%
  \clearpage
}
\makeatother

\IfFileExists{\jobname-pw.ind}{\input{\jobname-pw.ind}}{}

\end{document}

      