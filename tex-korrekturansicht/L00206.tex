%% latex-korrekturansicht-vorspann.tex
%% Vorspann für die Korrekturansicht.
%% Lädt die gemeinsame Datei latex-vorspann.tex mit gesetztem Schalter.

\newif\ifkorrekturansicht
\korrekturansichttrue

\input{../tex-inputs/latex-vorspann}


               \section[Karl Kraus an Arthur Schnitzler, 2. 5. 1893]{ Karl Kraus an Arthur Schnitzler, 2. 5. 1893}\nopagebreak\mylabel{v}\rehead{ }\normalsize\beginnumbering\briefempfaengerindex{Schnitzler, Arthur@\textsc{Schnitzler, Arthur}!zzzKraus, Karl@\emph{von Karl Kraus}!1893-05-021@{2. 5. 1895}|(be} \toendnotes[C]{\smallbreak\pagebreak[2]} \Standort{DLA, A:Schnitzler, HS.NZ85.1.3790, S. 11.}
\physDesc{maschinelle Abschrift
\newline{}Handschrift: Bleistift, deutsche Kurrent (\noindent{}eine Korrektur)}\buchAbdrucke{\weitereDrucke{\emph{Karl Kraus und Arthur Schnitzler. Eine Dokumentation.} Hg. Reinhard Urbach. In: \emph{Literatur und Kritik}, Bd. 49, Oktober 1970, S. 518.} }\toendnotes[C]{\smallbreak}\pstart
           \raggedleft{}{\pb}\textcolor{pink}{Wien}{}\ledrightnote{\textcolor{pink}{Wien}}, 2. Mai 1893. \pend
           \pstart
           \label{K_L00206_1v}\edtext{Eben lese ich}{\lemma{\textnormal{\emph{Eben lese ich}}}\Cendnote{\textnormal{Die \emph{\textcolor{brown}{Wiener
                     Zeitung}} brachte bereits wenige Stunden nach \textcolor{blue}{Johann Schnitzler}s Tod in ihrer Abendausgabe \emph{\textcolor{brown}{Wiener Abendpost}}, Nr. 100 vom 2. 5. 1893, S. 3, eine nicht
                  gezeichnete, kurze Todesmeldung: »\emph{\textcolor{green}{Regierungsrath
                     Professor Schnitzler †}}.«}}}\label{K_L00206_1h}, hochverehrter Herr Doctor,
               von dem schmerzlichen \textcolor{blue}{Ereignisse}{}\ledrightnote{→\textcolor{blue}{Johann Schnitzler}} in Ihrer werten Familie. Nehmen Sie, verehrter, liebster Herr
               Doctor, die Versicherung meiner \uline{herzlichsten, innigsten
                  Antheilnahme}! Ich bin mit hochachtungsvollem Grusse Ihr treuer\pend
           \pstart \spacefill\mbox{K. K.}\pend{}\endnumbering\briefempfaengerindex{Schnitzler, Arthur@\textsc{Schnitzler, Arthur}!zzzKraus, Karl@\emph{von Karl Kraus}!1893-05-021@{2. 5. 1895}|)be}\mylabel{h}  \normalsize

\doendnotes{C}
\bigskip
\vfill

\clearpage

\footnotesize

\lohead{\textsc{register}}

% Definiere theindex-Environment komplett neu ohne reledmac
\makeatletter
\renewenvironment{theindex}{%
  \section*{\indexname}%
  \setlength{\parindent}{0pt}%
  \setlength{\parskip}{0pt plus 0.3pt}%
  \let\item\@idxitem
}{%
  \clearpage
}
\makeatother

\IfFileExists{\jobname-pw.ind}{\input{\jobname-pw.ind}}{}

\end{document}

      