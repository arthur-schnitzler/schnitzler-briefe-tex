%% latex-korrekturansicht-vorspann.tex
%% Vorspann für die Korrekturansicht.
%% Lädt die gemeinsame Datei latex-vorspann.tex mit gesetztem Schalter.

\newif\ifkorrekturansicht
\korrekturansichttrue

\input{../tex-inputs/latex-vorspann}


\renewcommand{\erwaehntePersonen}{Personen: Gisela Hajek, Felix Salten, Olga Schnitzler, Margot Vallo}
\renewcommand{\erwaehnteOrte}{Orte: Pötzleinsdorf, Starkfriedgassse, VIII., Josefstadt, Wien}
\renewcommand{\erwaehnteWerke}{}
\section[ Arthur Schnitzler an Felix Salten, 27. 7. 1904]{Arthur Schnitzler an Felix Salten, 27. 7. 1904}
\nopagebreak\mylabel{v}
\rehead{ }\normalsize\beginnumbering\briefempfaengerindex{Salten, Felix@\textsc{Salten, Felix}!zzzSchnitzler, Arthur@\emph{von Arthur Schnitzler}!1904-07-271@{27. 7. 1904}|(be}
\toendnotes[C]{\smallbreak\pagebreak[2]}\Standort{Wienbibliothek im Rathaus, ZPH 1681, 2.1.516.}
\physDesc{Kartenbrief, 314 Zeichen
\newline{}Handschrift: Bleistift, deutsche Kurrent
\newline{}Versand: 1) Stempel: »\nobreak{}\oindex{VIII., Josefstadt@\textbf{VIII., Josefstadt}, \emph{A.ADM3}|pwk}18/1 Wi{[}en{]}, 27. VII. 04, 6\nobreak{}«.   2) Stempel: »\nobreak{}\oindex{VIII., Josefstadt@\textbf{VIII., Josefstadt}, \emph{A.ADM3}|pwk}{\pb}Wien 18/3 1\textcolor{gray}{44}, 27. 7. 04, \textcolor{gray}{5} N, Bestellt\nobreak{}«. 
\newline{}Ordnung: mit Bleistift von unbekannter Hand Nummerierung der Blätter des Konvoluts:
                                    »20« }\toendnotes[C]{\smallbreak}\pstart{}{\pb}Herrn \textsc{Felix
                     Salten}\pend{}\pstart{}\textcolor{pink}{Wien Pötzleinsdorf}{}\ledrightnote{\textcolor{pink}{Pötzleinsdorf}}\pend{}\pstart{}\textcolor{pink}{Starkfriedgaſſe 12}{}\ledrightnote{\textcolor{pink}{Starkfriedgassse}}.\pend{}
{\bigskip}
\pstart
           \raggedleft{}{\pb}27. 7 904\pend
           
\pstart
           lieber, für morgen müſſen wir leider
               abſagen. Sind mit meiner \textcolor{blue}{Schweſter}{}\ledrightnote{{$\rightarrow$}\textcolor{blue}{Gisela Hajek}} das erſte Mal ſeit vielen Wochen (\textsc{\textcolor{blue}{Margot}{}\ledrightnote{\textcolor{blue}{Margot Vallo}}\strikeout{t}} hatte Scharlach) u das letzte Mal vor ihrer Abreiſe zuſammen.\pend
           
\pstart
           Auf \label{K_L02992-1v}\edtext{nächſte Woche}{\lemma{\textnormal{\emph{nächſte Woche}}}\Cendnote{\textnormal{siehe A. S.: \emph{Tagebuch}, 4. 8. 1904}}}\label{K_L02992-1h}\pend
           
\pstart
           Herzlichen Gruß {\\[\baselineskip]}Ihr {\\[\baselineskip]}\spacefill\mbox{A.}\pend
           \leftskip=0em{}
\pstart
           \noindent{}\label{T_L02992-1v}\edtext{Die \label{K_L02992-2v}\edtext{Bilder}{\lemma{\textnormal{\emph{Bilder}}}\Cendnote{\textnormal{siehe A. S.: \emph{Tagebuch}, 25. 7. 1904}}}\label{K_L02992-2h} ſind da{[}.{]}{ }\textcolor{blue}{Olga}{}\ledrightnote{\textcolor{blue}{Olga Schnitzler}} und andre ſind entzückt.}{\lemma{\textnormal{\emph{Die … entzückt.}}}\Cendnote{\textnormal{seitlich am rechten Rand, quer zum
                     Text}}}\label{T_L02992-1h}\pend
           \endnumbering\briefempfaengerindex{Salten, Felix@\textsc{Salten, Felix}!zzzSchnitzler, Arthur@\emph{von Arthur Schnitzler}!1904-07-271@{27. 7. 1904}|)be}\mylabel{h}  \normalsize

\doendnotes{C}
\bigskip
\vfill

\clearpage

\footnotesize

\lohead{\textsc{register}}

% Definiere theindex-Environment komplett neu ohne reledmac
\makeatletter
\renewenvironment{theindex}{%
  \section*{\indexname}%
  \setlength{\parindent}{0pt}%
  \setlength{\parskip}{0pt plus 0.3pt}%
  \let\item\@idxitem
}{%
  \clearpage
}
\makeatother

\IfFileExists{\jobname-pw.ind}{\input{\jobname-pw.ind}}{}

\end{document}

      