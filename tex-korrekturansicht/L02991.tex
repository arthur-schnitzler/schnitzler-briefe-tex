%% latex-korrekturansicht-vorspann.tex
%% Vorspann für die Korrekturansicht.
%% Lädt die gemeinsame Datei latex-vorspann.tex mit gesetztem Schalter.

\newif\ifkorrekturansicht
\korrekturansichttrue

\input{../tex-inputs/latex-vorspann}


\renewcommand{\erwaehntePersonen}{Personen: Hugo Haberfeld, Johann Klein, Richard Klein, Bertha Klein, Felix Salten, Rosalie Schnitzler, Heinrich Schnitzler}
\renewcommand{\erwaehnteInstitutionen}{Institutionen: Galerie Pisko, [Ausstellung von Josef Beyer, Richard Klein, Lazar Krestin, Paul Reß und Karl Schade]}
\renewcommand{\erwaehnteOrte}{Orte: Wien}
\renewcommand{\erwaehnteWerke}{}
\section[ Arthur Schnitzler an Felix Salten, 13. 4. 1904]{Arthur Schnitzler an Felix Salten, 13. 4. 1904}
\nopagebreak\mylabel{v}
\rehead{ }\normalsize\beginnumbering\briefempfaengerindex{Salten, Felix@\textsc{Salten, Felix}!zzzSchnitzler, Arthur@\emph{von Arthur Schnitzler}!1904-04-131@{13. 4. 1904}|(be}
\toendnotes[C]{\smallbreak\pagebreak[2]}\Standort{Wienbibliothek im Rathaus, ZPH 1681, 2.1.516.}
\physDesc{Brief, 1 Blatt, 3 Seiten, 620 Zeichen
\newline{}Handschrift: Bleistift, deutsche Kurrent
\newline{}Ordnung: mit Bleistift von unbekannter Hand Nummerierung der Doppelseiten des
                                 Konvoluts: »32«–»33« }
\buchAbdrucke{\weitereDrucke{Arthur Schnitzler: \emph{Briefe 1875–1912}. Hg. Therese Nickl und Heinrich Schnitzler. Frankfurt am Main: \emph{S. Fischer} 1981, S. 481.} }\toendnotes[C]{\smallbreak}
\pstart
           \raggedleft{}{\pb}13. 4. 904\pend
           
\pstart
           lieber Freund, ein Vetter, oder wenigſtens \label{K_L02991-1v}\edtext{beinah ein Vetter}{\lemma{\textnormal{\emph{beinah ein Vetter}}}\Cendnote{\textnormal{Der \textcolor{blue}{Vater} von \textcolor{blue}{Richard Klein} war der
                  Bruder von \textcolor{blue}{Rosalie Schnitzler}, \textcolor{blue}{Arthur Schnitzler}s Großmutter
                  väterlicherseits.}}}\label{K_L02991-1h} von mir, \textsc{\textcolor{blue}{Richard Klein}{}\ledrightnote{\textcolor{blue}{Richard Klein}}}{[},{]} ſtellt bei \textcolor{brown}{Pisko}{}\ledrightnote{\textcolor{brown}{Galerie Pisko}} aus,
               ſeine \textcolor{blue}{Mutter}{}\ledrightnote{{$\rightarrow$}\textcolor{blue}{Bertha Klein}} ſchreibt mir,
               ich möchte Sie bitten, dieſe \textcolor{brown}{Ausſtellg}{}\ledrightnote{{$\rightarrow$}\textcolor{brown}{[Ausstellung von Josef Beyer, Richard Klein, Lazar Krestin, Paul Reß und Karl Schade]}} zu beſuchen.– Was hiemit geſchieht. Aber ich denke, nicht Sie
               ſondern \label{K_L02991-2v}\edtext{\textsc{\textcolor{blue}{Haberfeld}{}\ledrightnote{\textcolor{blue}{Hugo Haberfeld}}} ſchrei{\pb}ben über dergleichen}{\lemma{\textnormal{\emph{Haberfeld … dergleichen}}}\Cendnote{\textnormal{siehe Felix Salten an Arthur Schnitzler, [14. 4. 1904]}}}\label{K_L02991-2h}. (Was ich auch meiner \textcolor{blue}{Tante}{}\ledrightnote{{$\rightarrow$}\textcolor{blue}{Bertha Klein}} ſchreibe.)\pend
           
\pstart
           Unser \textcolor{blue}{Bub}{}\ledrightnote{{$\rightarrow$}\textcolor{blue}{Heinrich Schnitzler}} hat die Maſern –
               trotzdem in dieſer Woche die Erkrankungsfälle ſchon ſinken. Was ſchert ſich ſo ein
               Bub um die Statiſtik. Ich denke mir oft, wie gefrozzelt ſich die Leute vorkommen, die
               krank werden, während eine {\pb}Epidemie im
               »Erlöſchen« iſt. (»Der letzte Fall«, Novelle.–)\pend
           
\pstart
           Grüß Sie Gott. {\\[\baselineskip]}Herzlich Ihr {\\[\baselineskip]}\spacefill\mbox{A.}\pend
           \leftskip=0em{}\endnumbering\briefempfaengerindex{Salten, Felix@\textsc{Salten, Felix}!zzzSchnitzler, Arthur@\emph{von Arthur Schnitzler}!1904-04-131@{13. 4. 1904}|)be}\mylabel{h}  \normalsize

\doendnotes{C}
\bigskip
\vfill

\clearpage

\footnotesize

\lohead{\textsc{register}}

% Definiere theindex-Environment komplett neu ohne reledmac
\makeatletter
\renewenvironment{theindex}{%
  \section*{\indexname}%
  \setlength{\parindent}{0pt}%
  \setlength{\parskip}{0pt plus 0.3pt}%
  \let\item\@idxitem
}{%
  \clearpage
}
\makeatother

\IfFileExists{\jobname-pw.ind}{\input{\jobname-pw.ind}}{}

\end{document}

      