%% latex-korrekturansicht-vorspann.tex
%% Vorspann für die Korrekturansicht.
%% Lädt die gemeinsame Datei latex-vorspann.tex mit gesetztem Schalter.

\newif\ifkorrekturansicht
\korrekturansichttrue

\input{../tex-inputs/latex-vorspann}


\section[Olga Schnitzler an Stefan Zweig, 21. 12. 1916]{L03759 Olga Schnitzler an Stefan Zweig, 21. 12. 1916}
\nopagebreak\mylabel{L03759v}
\rehead{ }\normalsize\beginnumbering\briefempfaengerindex{, @\textsc{, }!zzz, @\emph{von  }!1916-12-212@{21. 12. 1916}|(be}
\toendnotes[C]{\smallbreak\pagebreak[2]}\Standort{Jerusalem, National Library of Israel, ARC. Ms. Var. 305 1 58 Stefan Zweig Collection.}
\physDesc{Briefkarte, 833 Zeichen
\newline{}Handschrift: schwarze Tinte, lateinische Kurrent}\toendnotes[C]{\smallbreak}
\pstart
           \raggedleft{}{\pb}21. Dec. 1916. \pend
           \vspace{0.5em}
\pstart
           Lieber Herr Doctor, meine liebe \textcolor{blue}{Hofrätin}\pwindex{Zuckerkandl, Berta 13.\,4.\,1864 Wien – 16.\,10.\,1945 Paris@\textsc{Zuckerkandl, Berta} (13.\,4.\,1864 Wien – 16.\,10.\,1945 Paris), \emph{Journalistin, Übersetzerin}|pwv}{}\ledrightnote{{$\rightarrow$}\emph{\textcolor{blue}{Berta Zuckerkandl}}} erzält mir heute Abend, dass Sie Ihnen von dieser
               übeln \label{K_L03759-1v}\edtext{Klatscherei}{\lemma{\textnormal{\emph{Klatscherei}}}\Cendnote{\textnormal{\textcolor{blue}{Olga
                     Schnitzler}\pwindex{Schnitzler, Olga 17.\,1.\,1882 Wien – 13.\,1.\,1970 Lugano@\textsc{Schnitzler, Olga} (17.\,1.\,1882 Wien – 13.\,1.\,1970 Lugano), \emph{Schauspielerin, Sängerin}|pwk} hatte es schwer und tat sich schwer, mit ihrer Gesangskarriere
                  aus dem Schatten des berühmten \textcolor{blue}{Ehemanns} zu treten. Entsprechend empfindlich reagierte sie auf Gerüchte,
                  die ihr Können in Frage stellten. Vgl. A. S.: \emph{Tagebuch}, 4. 12. 1916. }}}\label{K_L03759-1} berichtet hat und dass Sie dieses
               Gerede richtigstellen wollen. Ich bitte Sie sehr, – tun sie es nicht, – das gibt der
               Sache eine Bedeutung, die sie nicht hat und nicht haben darf. Mein Instinct hat sich
               damals, nach dem ersten Erschrecken, bald gegen den »Warner« {\pb}gewendet, – ich finde, ein \label{K_L03759-2v}\edtext{Mann, der alle paar Jahre auf 2 Stunden in meinem
                  Hause}{\lemma{\textnormal{\emph{Mann, … Hause}}}\Cendnote{\textnormal{Sie scheint zu befürchten, dass
                  jeder sie verteidigende Mann als ihr Liebhaber wahrgenommen werden könnte –
                  oder als jemand, der in sie verliebt wäre.}}}\label{K_L03759-2} ist, darf so etwas gewiss nicht
               tun, – kaum hat ein erprobter Freund das Recht dazu. Die \textcolor{blue}{Hofrätin}\pwindex{Zuckerkandl, Berta 13.\,4.\,1864 Wien – 16.\,10.\,1945 Paris@\textsc{Zuckerkandl, Berta} (13.\,4.\,1864 Wien – 16.\,10.\,1945 Paris), \emph{Journalistin, Übersetzerin}|pwv}{}\ledrightnote{{$\rightarrow$}\emph{\textcolor{blue}{Berta Zuckerkandl}}} hat mir ihren Eindruck von Ihrer aufrichtigen Freude
               an meinem \label{K_L03759-3v}\edtext{\textcolor{violet}{Concert Abend}\eventindex{Wiener Konzerthaus@\textbf{Wiener Konzerthaus}!Gesangskonzert von Olga Schnitzler, 18.11.1916@Gesangskonzert von Olga Schnitzler, 18.11.1916|pwv}{}\ledrightnote{{$\rightarrow$}\emph{\textcolor{violet}{Gesangskonzert von Olga Schnitzler, 18.11.1916}}}}{\lemma{\textnormal{\emph{Concert Abend}}}\Cendnote{\textnormal{Am 18. 11. 1916 war \textcolor{blue}{Olga Schnitzler}\pwindex{Schnitzler, Olga 17.\,1.\,1882 Wien – 13.\,1.\,1970 Lugano@\textsc{Schnitzler, Olga} (17.\,1.\,1882 Wien – 13.\,1.\,1970 Lugano), \emph{Schauspielerin, Sängerin}|pwk} an einem \emph{\textcolor{violet}{Liederkonzert}\eventindex{Wiener Konzerthaus@\textbf{Wiener Konzerthaus}!Gesangskonzert von Olga Schnitzler, 18.11.1916@Gesangskonzert von Olga Schnitzler, 18.11.1916|pwk}} im \textcolor{pink}{Wiener Konzerthaus}\oindex{Wien@\textbf{Wien}!III., Landstraße@\textbf{III., Landstraße}!Wiener Konzerthaus@\textbf{Wiener Konzerthaus}, \emph{Konzertsaal}|pwk}
                  beteiligt gewesen.}}}\label{K_L03759-3} mitgeteilt. Das genügt mir vollkommen, und so lasse ich
               mir Ihre freundlichen Worte auch nicht entstellen. Man hat seinen Weg zu gehen,
               darauf kommt es an. Seien sie herzlich geprüsst!\pend
           \pstart \spacefill\mbox{OlgaSchnitzler.}\pend{}\selectlanguage{ngerman}\endnumbering\briefempfaengerindex{, @\textsc{, }!zzz, @\emph{von  }!1916-12-212@{21. 12. 1916}|)be}\mylabel{L03759h}  \normalsize

\doendnotes{C}
\bigskip
\vfill

\clearpage

\footnotesize

\lohead{\textsc{register}}

% Definiere theindex-Environment komplett neu ohne reledmac
\makeatletter
\renewenvironment{theindex}{%
  \section*{\indexname}%
  \setlength{\parindent}{0pt}%
  \setlength{\parskip}{0pt plus 0.3pt}%
  \let\item\@idxitem
}{%
  \clearpage
}
\makeatother

\IfFileExists{\jobname-pw.ind}{\input{\jobname-pw.ind}}{}

\end{document}

      