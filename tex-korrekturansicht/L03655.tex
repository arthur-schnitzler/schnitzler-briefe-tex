%% latex-korrekturansicht-vorspann.tex
%% Vorspann für die Korrekturansicht.
%% Lädt die gemeinsame Datei latex-vorspann.tex mit gesetztem Schalter.

\newif\ifkorrekturansicht
\korrekturansichttrue

\input{../tex-inputs/latex-vorspann}


\section[Stefan Zweig an Arthur Schnitzler, 13. 10. 1915]{L03655 Stefan Zweig an Arthur Schnitzler, 13. 10. 1915}
\nopagebreak\mylabel{L03655v}
\rehead{ }\normalsize\beginnumbering\briefempfaengerindex{, @\textsc{, }!zzz, @\emph{von  }!1915-10-131@{13. 10. 1915}|(be}
\toendnotes[C]{\smallbreak\pagebreak[2]}\Standort{CUL, Schnitzler, B 118.}
\physDesc{Brief, 1 Blatt, 4 Seiten, 2727 Zeichen
\newline{}Handschrift: lila Tinte, lateinische Kurrent
\newline{}Schnitzler: 1) mit Bleistift »\textsc{Zweig}«  2) mit rotem Buntstift eine Unterstreichung}
\buchAbdrucke{\weitereDrucke{1) Stefan Zweig: \emph{Briefwechsel mit Hermann Bahr, Sigmund Freud, Rainer Maria
                        Rilke und Arthur Schnitzler}. Herausgegeben von Jeffrey B. Berlin,  Hans-Ulrich Lindken und  Donald A. Prater. Frankfurt am Main: \emph{S. Fischer} 1987, S. 395–397.} \weitereDrucke{2) Stefan Zweig: \emph{Briefe. Bd. II: 1914–1919}. Herausgegeben von Knut Beck,  Jeffrey B. Berlin und  Natascha Weschenbach-Feggeler. Frankfurt am Main: \emph{S. Fischer} 1998, S. 89–90.} }\toendnotes[C]{\smallbreak}
\pstart
           \raggedleft{}{\pb}13. October 1915\pend
           
\pstart
           \textcolor{gray}{\textbf{SZ}}\hfill \textcolor{gray}{\textbf{\textcolor{pink}{VIII. KOCHGASSE}\oindex{Wien@\textbf{Wien}!VIII., Josefstadt@\textbf{VIII., Josefstadt}!Kochgasse 8@\textbf{Kochgasse 8}, \emph{Wohngebäude}|pw}{}\ledrightnote{\textcolor{pink}{Kochgasse 8}}}}\pend
           
\pstart
           \raggedleft{}\textcolor{gray}{\textbf{\textcolor{pink}{WIEN}\oindex{Wien@\textbf{Wien}, \emph{Verwaltungsgebiet}|pw}{}\ledrightnote{\textcolor{pink}{Wien}},}}\pend
           \vspace{0.5em}
\pstart
           Verehrter lieber Herr Doktor, ich habe \label{K_L03655-1v}\edtext{\textcolor{violet}{gestern}\eventindex{Burgtheater@\textbf{Burgtheater}!Uraufführung von Komödie der Worte, 12.10.1915@Uraufführung von Komödie der Worte, 12.10.1915|pwv}{}\ledrightnote{{$\rightarrow$}\emph{\textcolor{violet}{Uraufführung von Komödie der Worte, 12.10.1915}}}}{\lemma{\textnormal{\emph{gestern}}}\Cendnote{\textnormal{Am 12. 10. 1915 fand die \emph{\textcolor{violet}{Uraufführung von \emph{\textcolor{green}{Komödie der Worte}\pwindex{Schnitzler, Arthur 15. 5. 1862 Wien – 21. 10. 1931 ebd.@\textsc{Schnitzler, Arthur} (15. 5. 1862 Wien – 21. 10. 1931 ebd.), \emph{Schriftsteller, Mediziner}!Komödie der Worte. Drei Einakter@\strich\emph{Komödie der Worte. Drei Einakter}|pwk}}}\eventindex{Burgtheater@\textbf{Burgtheater}!Uraufführung von Komödie der Worte, 12.10.1915@Uraufführung von Komödie der Worte, 12.10.1915|pwk}} am \emph{\textcolor{brown}{Burgtheater}\orgindex{Burgtheater@Burgtheater|pwk}} statt. }}}\label{K_L03655-1} aus einem versteckten Winkel des \textcolor{pink}{Burgtheaters}\oindex{Wien@\textbf{Wien}!I., Innere Stadt@\textbf{I., Innere Stadt}!Burgtheater@\textbf{Burgtheater}, \emph{Theater}|pw}{}\ledrightnote{\textcolor{pink}{Burgtheater}} die Freude der \label{K_L03655-2v}\edtext{\textcolor{violet}{Wi{[}e{]}derbegegnung}\eventindex{Sternwartestraße 71@\textbf{Sternwartestraße 71}!Private Lesung von Komödie der Worte, 11.4.1915@Private Lesung von Komödie der Worte, 11.4.1915|pwv}{}\ledrightnote{{$\rightarrow$}\emph{\textcolor{violet}{Private Lesung von Komödie der Worte, 11.4.1915}}}}{\lemma{\textnormal{\emph{Wiederbegegnung}}}\Cendnote{\textnormal{\textcolor{blue}{Schnitzler} hatte \textcolor{blue}{Zweig}\pwindex{Zweig, Stefan 28.\,11.\,1881 Wien – 23.\,2.\,1942 Petrópolis@\textsc{Zweig, Stefan} (28.\,11.\,1881 Wien – 23.\,2.\,1942 Petrópolis), \emph{Schriftsteller}|pwk} und \textcolor{blue}{Berta
                     Zuckerkandl}\pwindex{Zuckerkandl, Berta 13.\,4.\,1864 Wien – 16.\,10.\,1945 Paris@\textsc{Zuckerkandl, Berta} (13.\,4.\,1864 Wien – 16.\,10.\,1945 Paris), \emph{Journalistin, Übersetzerin}|pwk} am 11. 4. 1915 die \emph{\textcolor{green}{Komödie der
                     Worte}\pwindex{Schnitzler, Arthur 15. 5. 1862 Wien – 21. 10. 1931 ebd.@\textsc{Schnitzler, Arthur} (15. 5. 1862 Wien – 21. 10. 1931 ebd.), \emph{Schriftsteller, Mediziner}!Komödie der Worte. Drei Einakter@\strich\emph{Komödie der Worte. Drei Einakter}|pwk}} vorgelesen.}}}\label{K_L03655-2} mit Ihren drei \textcolor{green}{Stücken}\pwindex{Schnitzler, Arthur 15. 5. 1862 Wien – 21. 10. 1931 ebd.@\textsc{Schnitzler, Arthur} (15. 5. 1862 Wien – 21. 10. 1931 ebd.), \emph{Schriftsteller, Mediziner}!Komödie der Worte. Drei Einakter@\strich\emph{Komödie der Worte. Drei Einakter}|pwv}{}\ledrightnote{{$\rightarrow$}\emph{\textcolor{green}{Komödie der Worte. Drei Einakter}}} gehabt und war glücklich zu sehen, dass die Andern,
               denen Sie zum erstenmal gegeben waren, so herzlich ihren Dank äusserten. Mir war
               jedes Wort von damals noch gewärtig, manches fehlte mir sogar, nur dass der Interpret
               damals mir lieber war als diesmal manche seiner Darsteller. Für mein Gefühl ist \textcolor{blue}{Walden}\pwindex{Walden, Harry 22.\,10.\,1875 Berlin – 4.\,6.\,1921 ebd.@\textsc{Walden, Harry} (22.\,10.\,1875 Berlin – 4.\,6.\,1921 ebd.), \emph{Schauspieler}|pw}{}\ledrightnote{\textcolor{blue}{Harry Walden}} irgendwie unzulänglich, weil er allen
               Menschen, die er darstellt, etwas Unfreundliches, Antipathisches mitgibt und {\pb}selbst in seiner »\textcolor{green}{Grossen Scene}\pwindex{Schnitzler, Arthur 15. 5. 1862 Wien – 21. 10. 1931 ebd.@\textsc{Schnitzler, Arthur} (15. 5. 1862 Wien – 21. 10. 1931 ebd.), \emph{Schriftsteller, Mediziner}!Große Szene@\strich\emph{Große Szene}|pw}{}\ledrightnote{\textcolor{green}{Große Szene}}« fehlte ihm die Schwungkraft, die
               widerstandslos hinüberreisst, die Selbstberauschtheit – überhaupt, er hatte in beiden
               ersten \textcolor{green}{Stücken}\pwindex{Schnitzler, Arthur 15. 5. 1862 Wien – 21. 10. 1931 ebd.@\textsc{Schnitzler, Arthur} (15. 5. 1862 Wien – 21. 10. 1931 ebd.), \emph{Schriftsteller, Mediziner}!Stunde des Erkennens@\strich\emph{Stunde des Erkennens}|pwv}\pwindex{Schnitzler, Arthur 15. 5. 1862 Wien – 21. 10. 1931 ebd.@\textsc{Schnitzler, Arthur} (15. 5. 1862 Wien – 21. 10. 1931 ebd.), \emph{Schriftsteller, Mediziner}!Große Szene@\strich\emph{Große Szene}|pwv}{}\ledrightnote{{$\rightarrow$}\emph{\textcolor{green}{Stunde des Erkennens}}{\newline}{$\rightarrow$}\emph{\textcolor{green}{Große Szene}}} nicht
               das, was die Menschen \uline{entschuldigt}\strikeout{,}{ }\strikeout{\textcolor{gray}{Fü}\textcolor{gray}{×}} und was Sie doch so sehr in die Rolle mitgegeben hatten, bei dem \textcolor{green}{ersten}\pwindex{Schnitzler, Arthur 15. 5. 1862 Wien – 21. 10. 1931 ebd.@\textsc{Schnitzler, Arthur} (15. 5. 1862 Wien – 21. 10. 1931 ebd.), \emph{Schriftsteller, Mediziner}!Stunde des Erkennens@\strich\emph{Stunde des Erkennens}|pwv}\pwindex{Schnitzler, Arthur 15. 5. 1862 Wien – 21. 10. 1931 ebd.@\textsc{Schnitzler, Arthur} (15. 5. 1862 Wien – 21. 10. 1931 ebd.), \emph{Schriftsteller, Mediziner}!Große Szene@\strich\emph{Große Szene}|pwv}{}\ledrightnote{{$\rightarrow$}\emph{\textcolor{green}{Stunde des Erkennens}}{\newline}{$\rightarrow$}\emph{\textcolor{green}{Große Szene}}} die
               concentrierte Leidenschaft, bei dem \textcolor{green}{zweiten}\pwindex{Schnitzler, Arthur 15. 5. 1862 Wien – 21. 10. 1931 ebd.@\textsc{Schnitzler, Arthur} (15. 5. 1862 Wien – 21. 10. 1931 ebd.), \emph{Schriftsteller, Mediziner}!Große Szene@\strich\emph{Große Szene}|pwv}{}\ledrightnote{{$\rightarrow$}\emph{\textcolor{green}{Große Szene}}} die sprunghafte, aber Leidenschaft, Wärme doch in den \textcolor{green}{beiden}\pwindex{Schnitzler, Arthur 15. 5. 1862 Wien – 21. 10. 1931 ebd.@\textsc{Schnitzler, Arthur} (15. 5. 1862 Wien – 21. 10. 1931 ebd.), \emph{Schriftsteller, Mediziner}!Stunde des Erkennens@\strich\emph{Stunde des Erkennens}|pwv}\pwindex{Schnitzler, Arthur 15. 5. 1862 Wien – 21. 10. 1931 ebd.@\textsc{Schnitzler, Arthur} (15. 5. 1862 Wien – 21. 10. 1931 ebd.), \emph{Schriftsteller, Mediziner}!Große Szene@\strich\emph{Große Szene}|pwv}{}\ledrightnote{{$\rightarrow$}\emph{\textcolor{green}{Stunde des Erkennens}}{\newline}{$\rightarrow$}\emph{\textcolor{green}{Große Szene}}}. \label{K_L03655-3v}\edtext{\textcolor{blue}{Bassermann}\pwindex{Bassermann, Albert 7.\,9.\,1867 Mannheim – 15.\,5.\,1952 Atlantischer Ozean@\textsc{Bassermann, Albert} (7.\,9.\,1867 Mannheim – 15.\,5.\,1952 Atlantischer Ozean), \emph{Schauspieler}|pw}{}\ledrightnote{\textcolor{blue}{Albert Bassermann}}}{\lemma{\textnormal{\emph{Bassermann}}}\Cendnote{\textnormal{\textcolor{blue}{Albert Bassermann}\pwindex{Bassermann, Albert 7.\,9.\,1867 Mannheim – 15.\,5.\,1952 Atlantischer Ozean@\textsc{Bassermann, Albert} (7.\,9.\,1867 Mannheim – 15.\,5.\,1952 Atlantischer Ozean), \emph{Schauspieler}|pwk} spielte die Hauptrolle in
                  der ersten \textcolor{violet}{\textcolor{pink}{Berliner}\oindex{Berlin@\textbf{Berlin}, \emph{Hauptstadt}|pwk} Inszenierung}\eventindex{Lessing-Theater@\textbf{Lessing-Theater}!Premiere von Komödie der Worte, 23.10.1915@Premiere von Komödie der Worte, 23.10.1915|pwkv}, die am 23. 10. 1915 am \emph{\textcolor{brown}{Lessing-Theater}\orgindex{Lessing-Theater@Lessing-Theater|pwk}} Premiere hatte.}}}\label{K_L03655-3} wird
               sicherlich unendlich besser sein und auch besser secundiert werden als in dieser
               sonst recht gelungenen Aufführung, die nur (wie so oft im \textcolor{brown}{Burgtheater}\orgindex{Burgtheater@Burgtheater|pw}{}\ledrightnote{\textcolor{brown}{Burgtheater}}) das Conversationelle nach oben kehrte und das
               Innerliche drückte. Ich glaube, man kennt Sie nicht gut, wenn man Ihre Stücke nur im
               Theater und gerade bei Uns im Theater gesehen hat: irgend ein \strikeout{\textcolor{gray}{Fond}} Geheimnisvolles schwebt da weg, eine Atmosphäre, die sie nicht ganz zu {\pb}halten wissen: die menschliche Wärme
               strömt manchmal zwischen den Worten aus, statt sich mit ihnen chemisch zu binden. Ich
               habe einmal bei \textcolor{blue}{Brahm}\pwindex{Brahm, Otto 5.\,2.\,1856 Hamburg – 28.\,11.\,1912 Berlin@\textsc{Brahm, Otto} (5.\,2.\,1856 Hamburg – 28.\,11.\,1912 Berlin), \emph{Theaterleiter, Regisseur}|pw}{}\ledrightnote{\textcolor{blue}{Otto Brahm}} empfunden, wie man
               gerade in \textcolor{pink}{Wien}\oindex{Wien@\textbf{Wien}, \emph{Verwaltungsgebiet}|pw}{}\ledrightnote{\textcolor{pink}{Wien}} (wo man’s doch am ehesten \introOben{}nicht\introOben{} sollte) immer ein wenig leichter machen will, als sie's
               wirklich specifisch sind: ich spüre selbst im Satyrspiel des gestrigen Abends, im
                  »\textcolor{green}{Bacchusfest}\pwindex{Schnitzler, Arthur 15. 5. 1862 Wien – 21. 10. 1931 ebd.@\textsc{Schnitzler, Arthur} (15. 5. 1862 Wien – 21. 10. 1931 ebd.), \emph{Schriftsteller, Mediziner}!Bacchusfest@\strich\emph{Das Bacchusfest}|pw}{}\ledrightnote{\textcolor{green}{Das Bacchusfest}}« so schöne Dinge, dass ich sie
               ganz geniessen und nicht gerne überspielt sehen wollte. Aber freilich, das Theater
               soll ja nicht den einzelnen Geniessern sondern dem Publicum dienen und so war ich (so
               sehr mir manches schöne Wort fehlte) auch der geschwinderen Form froh, weil ich sah,
               wie sehr die drei \textcolor{green}{Stücke}\pwindex{Schnitzler, Arthur 15. 5. 1862 Wien – 21. 10. 1931 ebd.@\textsc{Schnitzler, Arthur} (15. 5. 1862 Wien – 21. 10. 1931 ebd.), \emph{Schriftsteller, Mediziner}!Komödie der Worte. Drei Einakter@\strich\emph{Komödie der Worte. Drei Einakter}|pw}{}\ledrightnote{\textcolor{green}{Komödie der Worte. Drei Einakter}} gewirkt haben. Gewirkt
               haben gegen eine düstere Zeit, gegen einen Hintergrund, der jedes Echo privaten
               Problemen verweigert und damit \introOben{}haben Sie\introOben{} die siebenfache\strikeout{n} Goldprobe {\pb}bestanden! Nun kann ihnen nirgends und
               nie mehr Ungunst geschehen, sie schreiten weiter und weiter, werden länger dauern als
               das Längste, was wir im Fühlen jetzt als Mass haben, als diese Zeit, die mir wie ein
               halbes Jahrhundert dünkt. Ich danke Ihnen für den schönen Abend, gedenke noch inngst
               jenes andern, da ich zuerst sie hören durfte und mein Glückwunsch zu Werk und Erfolg
               kommt aus aufrichtigem Herzen. Viele Empfehlungen Ihrer verehrten Frau \textcolor{blue}{Gemahlin}\pwindex{Schnitzler, Olga 17.\,1.\,1882 Wien – 13.\,1.\,1970 Lugano@\textsc{Schnitzler, Olga} (17.\,1.\,1882 Wien – 13.\,1.\,1970 Lugano), \emph{Schauspielerin, Sängerin}|pwv}{}\ledrightnote{{$\rightarrow$}\emph{\textcolor{blue}{Olga Schnitzler}}} und getreue Grüsse
               von Ihrem ergebenen\pend
           \pstart \spacefill\mbox{Stefan Zweig}\pend{}\selectlanguage{ngerman}\endnumbering\briefempfaengerindex{, @\textsc{, }!zzz, @\emph{von  }!1915-10-131@{13. 10. 1915}|)be}\mylabel{L03655h}  \normalsize

\doendnotes{C}
\bigskip
\vfill

\clearpage

\footnotesize

\lohead{\textsc{register}}

% Definiere theindex-Environment komplett neu ohne reledmac
\makeatletter
\renewenvironment{theindex}{%
  \section*{\indexname}%
  \setlength{\parindent}{0pt}%
  \setlength{\parskip}{0pt plus 0.3pt}%
  \let\item\@idxitem
}{%
  \clearpage
}
\makeatother

\IfFileExists{\jobname-pw.ind}{\input{\jobname-pw.ind}}{}

\end{document}

      