%% latex-korrekturansicht-vorspann.tex
%% Vorspann für die Korrekturansicht.
%% Lädt die gemeinsame Datei latex-vorspann.tex mit gesetztem Schalter.

\newif\ifkorrekturansicht
\korrekturansichttrue

\input{../tex-inputs/latex-vorspann}


\renewcommand{\erwaehntePersonen}{Personen: Albert Bassermann, Otto Brahm, Olga Schnitzler, Harry Walden, Berta Zuckerkandl, Stefan Zweig}
\renewcommand{\erwaehnteInstitutionen}{Institutionen: Burgtheater, Lessing-Theater}
\renewcommand{\erwaehnteOrte}{Orte: Berlin, Burgtheater, Kochgasse 8, Wien}
\renewcommand{\erwaehnteWerke}{Werke: Das Bacchusfest, Große Szene, Komödie der Worte. Drei Einakter, Stunde des Erkennens}
\section[Stefan Zweig an Arthur Schnitzler, 13. 10. 1915]{Stefan Zweig an Arthur Schnitzler, 13. 10. 1915}
\nopagebreak\mylabel{v}
\rehead{ }\normalsize\beginnumbering\briefempfaengerindex{Schnitzler, Arthur@\textsc{Schnitzler, Arthur}!zzzZweig, Stefan@\emph{von Stefan Zweig}!1915-10-131@{13. 10. 1915}|(be}
\toendnotes[C]{\smallbreak\pagebreak[2]}\Standort{CUL, Schnitzler, B 118.}
\physDesc{Brief, 1 Blatt, 4 Seiten, 2726 Zeichen
\newline{}Handschrift: lila Tinte, lateinische Kurrent
\newline{}Schnitzler: 1) mit Bleistift »\textsc{Zweig}«  2) mit rotem Buntstift eine Unterstreichung}\toendnotes[C]{\smallbreak}
\pstart
           {\pb}\textcolor{gray}{\textbf{SZ}}\hfill 13. October 1915\pend
           
\pstart
           \raggedleft{}\textcolor{gray}{\textbf{\textcolor{pink}{VIII. KOCHGASSE}{}\ledrightnote{\textcolor{pink}{Kochgasse 8}}}}\pend
           
\pstart
           \raggedleft{}\textcolor{gray}{\textbf{\textcolor{pink}{WIEN}{}\ledrightnote{\textcolor{pink}{Wien}},}}\pend
           
\pstart
           Verehrter lieber Herr Doktor, ich habe \label{K_L03655-1v}\edtext{gestern}{\lemma{\textnormal{\emph{gestern}}}\Cendnote{\textnormal{Am 12. 10. 1915 fand die
                  Uraufführung von \emph{\textcolor{green}{Komödie der Worte}} am \emph{\textcolor{brown}{Burgtheater}} statt. }}}\label{K_L03655-1h} aus einem
               versteckten Winkel des \textcolor{pink}{Burgtheaters}{}\ledrightnote{\textcolor{pink}{Burgtheater}} die Freude der
                  \label{K_L03655-2v}\edtext{Wi{[}e{]}derbegegnung}{\lemma{\textnormal{\emph{Wiederbegegnung}}}\Cendnote{\textnormal{\textcolor{blue}{Schnitzler} hatte \textcolor{blue}{Zweig} und \textcolor{blue}{Berta
                     Zuckerkandl} am 11. 4. 1915 die \emph{\textcolor{green}{Komödie der
                     Worte}} vorgelesen.}}}\label{K_L03655-2h} mit Ihren drei \textcolor{green}{Stücken}{}\ledrightnote{{$\rightarrow$}\textcolor{green}{Komödie der Worte. Drei Einakter}} gehabt und war glücklich zu sehen, dass die Andern,
               denen Sie zum erstenmal gegeben waren, so herzlich ihren Dank äusserten. Mir war
               jedes Wort von damals noch gewärtig, manches fehlte mir sogar, nur dass der Interpret
               damals mir lieber war als diesmal manche seiner Darsteller. Für mein Gefühl ist \textcolor{blue}{Walden}{}\ledrightnote{\textcolor{blue}{Harry Walden}} irgendwie unzulänglich, weil er allen
               Menschen, die er darstellt, etwas Unfreundliches, Antipathisches mitgibt und {\pb}selbst in seiner »\textcolor{green}{Grossen Scene}{}\ledrightnote{\textcolor{green}{Große Szene}}« fehlte ihm die Schwungkraft, die
               widerstandslos hinüberreisst, die Selbstberauschtheit – überhaupt, er hatte in beiden
               ersten \textcolor{green}{Stücken}{}\ledrightnote{{$\rightarrow$}\textcolor{green}{Stunde des Erkennens}{\newline}{$\rightarrow$}\textcolor{green}{Große Szene}} nicht
               das, was die Menschen \uline{entschuldigt}\strikeout{, \textcolor{gray}{Fü}\textcolor{gray}{×}} und was Sie doch so sehr in die Rolle mitgegeben hatten,
               bei dem \textcolor{green}{ersten}{}\ledrightnote{{$\rightarrow$}\textcolor{green}{Stunde des Erkennens}{\newline}{$\rightarrow$}\textcolor{green}{Große Szene}} die
               concentrierte Leidenschaft, bei dem \textcolor{green}{zweiten}{}\ledrightnote{{$\rightarrow$}\textcolor{green}{Große Szene}} die sprunghafte, aber Leidenschaft, Wärme doch in den \textcolor{green}{beiden}{}\ledrightnote{{$\rightarrow$}\textcolor{green}{Stunde des Erkennens}{\newline}{$\rightarrow$}\textcolor{green}{Große Szene}}. \label{K_L03655-3v}\edtext{\textcolor{blue}{Bassermann}{}\ledrightnote{\textcolor{blue}{Albert Bassermann}}}{\lemma{\textnormal{\emph{Bassermann}}}\Cendnote{\textnormal{\textcolor{blue}{Albert Bassermann} spielte die Hauptrolle in der ersten \textcolor{pink}{Berliner} Inszenierung, die am 23. 10. 1915 am \emph{\textcolor{brown}{Lessing-Theater}} Premiere hatte.}}}\label{K_L03655-3h} wird
               sicherlich unendlich besser sein und auch besser secundiert werden als in dieser
               sonst recht gelungenen Aufführung, die nur (wie so oft im \textcolor{brown}{Burgtheater}{}\ledrightnote{\textcolor{brown}{Burgtheater}}) das Conversationelle nach oben kehrte und das
               Innerliche drückte. Ich glaube, man kennt Sie nicht gut, wenn man Ihre Stücke nur im
               Theater und gerade bei Uns im Theater gesehen hat: irgend ein \strikeout{\textcolor{gray}{Fond}} Geheimnisvolles schwebt da weg,
               eine Atmosphäre, die sie nicht ganz zu {\pb}halten wissen: die menschliche Wärme strömt manchmal zwischen den Worten aus, statt
               sich mit ihnen chemisch zu binden. Ich habe einmal bei \textcolor{blue}{Brahm}{}\ledrightnote{\textcolor{blue}{Otto Brahm}} empfunden, wie man gerade in \textcolor{pink}{Wien}{}\ledrightnote{\textcolor{pink}{Wien}} (wo man’s doch am ehesten \introOben{}nicht\introOben{}
               sollte) immer ein wenig leichter machen will, als sie's wirklich specifisch sind: ich
               spüre selbst im Satyrspiel des gestrigen Abends, im »\textcolor{green}{Bacchusfest}{}\ledrightnote{\textcolor{green}{Das Bacchusfest}}« so schöne Dinge, dass ich sie ganz geniessen und nicht gerne
               überspielt sehen wollte. Aber freilich, das Theater soll ja nicht den einzelnen
               Geniessern sondern dem Publicum dienen und so war ich (so sehr mir manches schöne
               Wort fehlte) auch der geschwinderen Form froh, weil ich sah, wie sehr die drei \textcolor{green}{Stücke}{}\ledrightnote{\textcolor{green}{Komödie der Worte. Drei Einakter}} gewirkt haben. Gewirkt haben gegen eine
               düstere Zeit, gegen einen Hintergrund, der jedes Echo privaten Problemen verweigert
               und damit \introOben{}haben Sie\introOben{} die siebenfache\strikeout{n} Goldprobe {\pb}bestanden! Nun
               kann ihnen nirgends und nie mehr Ungunst geschehen, sie schreiten weiter und weiter,
               werden länger dauern als das Längste, was wir im Fühlen jetzt als Mass haben, als
               diese Zeit, die mir wie ein halbes Jahrhundert dünkt. Ich danke Ihnen für den schönen
               Abend, gedenke noch inngst jenes andern, da ich zuerst sie hören durfte und mein
               Glückwunsch zu Werk und Erfolg kommt aus aufrichtigem Herzen. Viele Empfehlungen
               Ihrer verehrten Frau \textcolor{blue}{Gemahlin}{}\ledrightnote{{$\rightarrow$}\textcolor{blue}{Olga Schnitzler}}
               und getreue Grüsse von Ihrem ergebenen\pend
           \pstart \spacefill\mbox{Stefan Zweig}\pend{}\endnumbering\briefempfaengerindex{Schnitzler, Arthur@\textsc{Schnitzler, Arthur}!zzzZweig, Stefan@\emph{von Stefan Zweig}!1915-10-131@{13. 10. 1915}|)be}\mylabel{h}
\begin{anhang}
\end{anhang}\normalsize

\doendnotes{C}
\bigskip
\vfill

\clearpage

\footnotesize

\lohead{\textsc{register}}

% Definiere theindex-Environment komplett neu ohne reledmac
\makeatletter
\renewenvironment{theindex}{%
  \section*{\indexname}%
  \setlength{\parindent}{0pt}%
  \setlength{\parskip}{0pt plus 0.3pt}%
  \let\item\@idxitem
}{%
  \clearpage
}
\makeatother

\IfFileExists{\jobname-pw.ind}{\input{\jobname-pw.ind}}{}

\end{document}

      