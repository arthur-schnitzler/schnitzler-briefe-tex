%% latex-korrekturansicht-vorspann.tex
%% Vorspann für die Korrekturansicht.
%% Lädt die gemeinsame Datei latex-vorspann.tex mit gesetztem Schalter.

\newif\ifkorrekturansicht
\korrekturansichttrue

\input{../tex-inputs/latex-vorspann}


               \section[Paul Goldmann an Arthur Schnitzler, Paul Goldmann an Arthur Schnitzler, 17. 5. {[}1896{]}]{ Paul Goldmann an Arthur Schnitzler, 17. 5. {[}1896{]}}\nopagebreak\mylabel{v}\rehead{ }\normalsize\beginnumbering\briefempfaengerindex{Schnitzler, Arthur@\textsc{Schnitzler, Arthur}!zzzGoldmann, Paul@\emph{von Paul Goldmann}!1896-05-172@{17. 5. {[}1896{]}}|(be} \toendnotes[C]{\smallbreak\pagebreak[2]} \Standort{DLA, A:Schnitzler, HS.NZ85.1.3166.}
\physDesc{Brief, 5 Blätter, 19 Seiten
\newline{}Handschrift: blaue Tinte, deutsche Kurrent
\newline{}Schnitzler: 1) mit Bleistift das Jahr »96« vermerkt 2) mit rotem Buntstift elf Unterstreichungen}\toendnotes[C]{\smallbreak}\pstart
           \noindent{}{\pb}\textcolor{gray}{\textbf{\textbf{\textcolor{brown}{Frankfurter Zeitung}{}\ledrightnote{\textcolor{brown}{Frankfurter Zeitung}}}}}\pend
           \pstart
           \textcolor{gray}{\textbf{(\textcolor{brown}{\begin{otherlanguage}{french}Gazette de Francfort\end{otherlanguage}}{}\ledrightnote{\textcolor{brown}{Frankfurter Zeitung}}).}}\pend
           \pstart
           \textcolor{gray}{\textbf{\textbf{\begin{otherlanguage}{french}Fondateur M.\end{otherlanguage}{ }\textcolor{blue}{L. Sonnemann}{}\ledrightnote{\textcolor{blue}{Leopold Sonnemann}}.}}}\pend
           \pstart
           \begin{otherlanguage}{french}\textcolor{gray}{\textbf{\textcolor{green}{Journal}{}\ledrightnote{→\textcolor{green}{Frankfurter Zeitung}} politique,
                        financier,}}\end{otherlanguage}\pend
           \pstart
           \begin{otherlanguage}{french}\textcolor{gray}{\textbf{commercial et littéraire.}}\end{otherlanguage}\pend
           \pstart
           \begin{otherlanguage}{french}\textcolor{gray}{\textbf{\textbf{Paraissant trois fois par jour.}}}\end{otherlanguage}\pend
           \pstart
           \begin{otherlanguage}{french}\textcolor{gray}{\textbf{\textbf{Bureau à \textcolor{pink}{Paris}{}\ledrightnote{\textcolor{pink}{Paris}}}}}\end{otherlanguage}\pend
           \pstart
           \begin{otherlanguage}{french}\textcolor{gray}{\textbf{\textbf{\textcolor{pink}{24. Rue Feydeau}{}\ledrightnote{\textcolor{pink}{rue Feydeau}}.}}}\end{otherlanguage}\hfill \textsc{\textcolor{pink}{Paris}{}\ledrightnote{\textcolor{pink}{Paris}}}, 17. Mai.\pend
           \pstart\center{}Mein lieber Freund,\pend\pstart
           1.) Nach einem flüchtigen Überſchlag von Zeit und Koſten ſehe ich, daß ich mit Dir
               werde kaum zuſammenreiſen können. Denke ſelbſt: Ich bekomme vier Wochen Urlaub und
               habe während desſelben etwa 700 \textsc{Francs} zu verzehren. Die
               Reiſe von hier über \textsc{\textcolor{pink}{Hamburg}{}\ledrightnote{\textcolor{pink}{Hamburg}}} nach \textcolor{pink}{Dänemark}{}\ledrightnote{\textcolor{pink}{Dänemark}}, \textcolor{pink}{Schweden}{}\ledrightnote{\textcolor{pink}{Schweden}} und \textcolor{pink}{Norwegen}{}\ledrightnote{\textcolor{pink}{Norwegen}}{ }\strikeout{\textcolor{gray}{würde}} und von da wieder nach \textsc{\textcolor{pink}{Paris}{}\ledrightnote{\textcolor{pink}{Paris}}} zurück würde allein an 500 \textsc{francs} koſten. Die
               Entfernungen {\pb}ſind außerdem groß, und ich würde
               einen guten Theil meines Urlaubs auf der Eiſenbahn verbringen. Nun ſind bei meiner
               Reiſe andere Rückſichten maßgebend, als bei Deiner. Du gehſt von \textcolor{pink}{Wien}{}\ledrightnote{\textcolor{pink}{Wien}} fort, um Neues zu ſehen, ich entferne mich von \textsc{\textcolor{pink}{Paris}{}\ledrightnote{\textcolor{pink}{Paris}}}, um auszuruhen. Endlich intereſſiren mich die \strikeout{ſkan}{ }\textcolor{pink}{ſkandinaviſchen Länder}{}\ledrightnote{→\textcolor{pink}{Dänemark}{\newline}→\textcolor{pink}{Schweden}{\newline}→\textcolor{pink}{Norwegen}} gar wenig, und eine Reiſe nach der \textcolor{pink}{Schweiz}{}\ledrightnote{\textcolor{pink}{Schweiz}}, mit einem kleinen Abſtecher nach \textcolor{pink}{Florenz}{}\ledrightnote{\textcolor{pink}{Florenz}}, wäre mir weitaus zuträglicher. Um Dich wiederzuſehen,
               bin ich freilich zu allen Conceſſionen {\pb}bereit, aber
               das \textcolor{pink}{ſkandinavi}{}\ledrightnote{→\textcolor{pink}{Dänemark}{\newline}→\textcolor{pink}{Schweden}{\newline}→\textcolor{pink}{Norwegen}}ſche Project erweiſt ſich bei näherer Betrachtung als
               Unmöglichkeit für mich. Mach’ mir alſo, bitte, einen \label{K_L02774-1v}\edtext{anderen Vorſchlag}{\lemma{\textnormal{\emph{anderen Vorſchlag}}}\Cendnote{\textnormal{siehe Paul Goldmann an Arthur Schnitzler, 29. 4. [1896]}}}\label{K_L02774-1h}. Ich gedenke, ſo zwiſchen 5. und 10. Auguſt aufzubrechen und würde meinen Urlaub als
               verfehlt betrachten, wenn ich Dich nicht ſehen könnte, worauf ich mich nun jetzt
               ſchon ſeit meinem letzten Urlaub freue.\pend
           \pstart
           2.) In Sachen von »\textsc{\textcolor{green}{Mourir}{}\ledrightnote{\textcolor{green}{Mourir. Roman}}}« will ich demnächſt etwas thun. Gegenwärtig habe ich ſo Tauſenderlei zu
               erledigen und komme nicht {\pb}dazu, die Leute zu ſehen,
               an die ich denke. Haſt Du an \textsc{\textcolor{blue}{Thorel}{}\ledrightnote{\textcolor{blue}{Jean Thorel}}} ein \textcolor{green}{Exemplar}{}\ledrightnote{→\textcolor{green}{Liebelei. Schauspiel in drei Akten}}
               geſchickt?\pend
           \pstart
           3.) Ich bleibe dabei, daß ich Deine \label{K_L02774-3v}\edtext{Mitarbeiterſchaft}{\lemma{\textnormal{\emph{Mitarbeiterſchaft}}}\Cendnote{\textnormal{siehe Paul Goldmann an Arthur Schnitzler, 29. 4. [1896]}}}\label{K_L02774-3h} bei \textsc{\textcolor{brown}{\textcolor{green}{Albert Langen}{}\ledrightnote{→\textcolor{green}{Simplicissimus}}}{}\ledrightnote{→\textcolor{brown}{Simplicissimus}}} bedaure. \strikeout{Die} Daß Directoren, die über Dich
               ſchimpfen, trotzdem Deine Stücke aufführen, iſt richtig. Aber die Directoren ſind
                  \strikeout{\textcolor{gray}{×}\-\textcolor{gray}{×}} nicht zu umgehen. Hingegen die Sachen, die bei \textsc{\textcolor{blue}{\textcolor{brown}{Langen}{}\ledrightnote{→\textcolor{brown}{Albert Langen}}}{}\ledrightnote{\textcolor{blue}{Albert Langen}}} erſchienen ſind, müßten nicht \strikeout{ged} gedruckt
               werden. \strikeout{A\textcolor{gray}{u}} Auch leiſteſt Du \textsc{\textcolor{blue}{Langen}{}\ledrightnote{\textcolor{blue}{Albert Langen}}}{ }\strikeout{D\textcolor{gray}{e}} einen ganz beſonderen Dienſt, indem Du ihm für ſein neues \textcolor{brown}{Unternehmen}{}\ledrightnote{→\textcolor{brown}{Albert Langen}} die gegenwärtig {\pb}beſonders große Autorität Deines Namens zur
               Verfügung \strikeout{ſtell} ſtellſt. Ferner: Wenn die
               Theater-Directoren über Dich ſchimpfen, weißt Du es nicht. Bei \textsc{\textcolor{blue}{Langen}{}\ledrightnote{\textcolor{blue}{Albert Langen}}} weißt Du es. Und würdeſt Du einem Director Dein Stück geben, der es mit den
               Worten empfinge: »Aufführen muß ichs wohl, aber Sie können nicht deutſch ſchreiben«?
               Endlich und letzlich geht es mir nicht in den Sinn, daß es in der Welt niemals eine
               Strafe für Lausbüberei geben ſoll. \textsc{\textcolor{blue}{Langen}{}\ledrightnote{\textcolor{blue}{Albert Langen}}} hat ſich vor {\pb}Deinen Erfolgen wie ein Lausbube
               über Dich geäußert. Jetzt ſieht er, daß er ſich verhauen hat, und Du ſendeſt ihm
               ſofort liebenswürdig Deine Manuſktripte: »Bitte, mein Herr, wir wollen, den kleinen
               Irrthum berichtigen, der in unſerer gegenſeitigen Schätzung mit untergelaufen
               iſt.«\pend
           \pstart
           4.) Mit \textsc{\textcolor{blue}{Harden}{}\ledrightnote{\textcolor{blue}{Maximilian Harden}}} haſt Du vielleicht Recht; aber hüte Dich vor ihm, er iſt ein falſcher Hund. Mit
               der »\textcolor{green}{Liebelei}{}\ledrightnote{\textcolor{green}{Liebelei. Schauspiel in drei Akten}}« iſt es Dir \uline{nicht} über Gebühr gut gegangen. {\pb}Sie
               nimmt vielleicht einen geringeren Rang in Deiner Schätzung ein, weil Du ſie mit den
               anderen Stücken vergleichſt, die \uline{Du} ſchreiben
               könnteſt und ſchreiben wiſt. Aber verglichen mit den Stücken, welche die \uline{Anderen} ſchreiben, ſteht ſie im erſten Range.\pend
           \pstart
           5.) Nächſte Woche will ich \textsc{\textcolor{blue}{Thorel}{}\ledrightnote{\textcolor{blue}{Jean Thorel}}} aufſuchen, und dann verabreden wir etwas Definitives in der \textcolor{green}{Überſetzung}{}\ledrightnote{→\textcolor{green}{Amourette. Pièce en trois actes}}s-Angelegenheit. Günſtig ſind
               die Chancen für Aufführung ausländiſcher {\pb}Stücke an
               einem anſtändigen Theater gegenwärtig \uline{nicht}.\pend
           \pstart
           6.) Die »\textcolor{green}{Freie Bühne}{}\ledrightnote{\textcolor{green}{Neue Deutsche Rundschau}}« bekomme ich nie zu
               Geſicht. Könnteſt Du mir die \textcolor{green}{Nummer}{}\ledrightnote{→\textcolor{green}{Neue Deutsche Rundschau}} mit dem \label{K_L02774-4v}\edtext{\textcolor{green}{Artikel}{}\ledrightnote{→\textcolor{green}{Arthur Schnitzler}}}{\lemma{\textnormal{\emph{Artikel}}}\Cendnote{\textnormal{\textcolor{blue}{Alfred Kerr}: \emph{\textcolor{green}{Arthur Schnitzler}}. In: \emph{\textcolor{green}{Neue Deutsche Rundschau (Freie Bühne)}}, Jg. 7, H. 3, März 1896, S. 287–292. (Die \emph{\textcolor{green}{Neue Deutsche Rundschau}} wurde als »Freie Bühne« gegründet,
                  hieß aber seit 1894 nicht mehr so.)}}}\label{K_L02774-4h} über Dich nicht
               ſchicken?\pend
           \pstart
           7.) Wenn \textsc{\textcolor{brown}{\textcolor{blue}{Fischer}{}\ledrightnote{\textcolor{blue}{Samuel Fischer}}}{}\ledrightnote{\textcolor{brown}{S. Fischer Verlag}}} Dich \strikeout{\textcolor{gray}{o}} ohne Verpflichtung \label{K_L02774-6v}\edtext{honorirt}{\lemma{\textnormal{\emph{honorirt}}}\Cendnote{\textnormal{Für die erste Auflage der
                     \emph{\textcolor{green}{Liebelei}} erhielt \textcolor{blue}{Schnitzler} vom \emph{\textcolor{brown}{S. Fischer
                     Verlag}} 400 Mark (rund 200 Euro). vgl. A. S.: \emph{Tagebuch}, 30. 4. 1896}}}\label{K_L02774-6h} hat, ſo geht daraus klar hervor, daß er Dich an ſich feſſeln will, um Dich
               bei Deinen ſämmtlichen nächſten Büchern betrügen zu können.\pend
           \pstart
           8.) Ein \textcolor{blue}{Menſch}{}\ledrightnote{→\textcolor{blue}{Peter Altenberg}}, den \textsc{\textcolor{blue}{Bahr}{}\ledrightnote{\textcolor{blue}{Hermann Bahr}}} als \label{K_L02774-7v}\edtext{»neuen Dichter«}{\lemma{\textnormal{\emph{»neuen Dichter«}}}\Cendnote{\textnormal{\textcolor{blue}{Peter Altenberg}, vgl. \textcolor{blue}{Hermann Bahr}: \emph{\textcolor{green}{Ein neuer Dichter}}. In: \emph{\textcolor{green}{Die Zeit}}, Bd. 7, Nr. 83, 2. 5. 1896,
                     S. 75–76.}}}\label{K_L02774-7h} ſignaliſirt, iſt bei mir ſo ſchwer {\pb}compromittirt, daß ich ihn \strikeout{\textcolor{gray}{×}} nicht mehr ohne Vorurtheil leſen kann. Immerhin würde ich gern in das \label{K_L02774-99v}\edtext{\textcolor{green}{Buch}{}\ledrightnote{→\textcolor{green}{Wie ich es sehe}}}{\lemma{\textnormal{\emph{Buch}}}\Cendnote{\textnormal{\textcolor{blue}{Peter Altenberg}: \emph{\textcolor{green}{Wie ich es sehe}}. Berlin: \emph{\textcolor{brown}{S. Fischer Verlag}}{ }1896.}}}\label{K_L02774-99h} hineinſchauen. Aber woher ſoll ichs bekommen? Könnteſt Du mirs
               nicht ſchicken? Nur leihweiſe, natürlich.\pend
           \pstart
           9.) Der kleine \textsc{\textcolor{blue}{Hugo}{}\ledrightnote{\textcolor{blue}{Hugo von Hofmannsthal}}} mag als Menſch charmant ſein, als Schriftſteller iſt er mir aufs Höchſte
               unſympathiſch, und er ſteht mir fern, als hätte ich ihn nie gekannt.\pend
           \pstart
           {\pb}10.) \label{K_L02774-9v}\edtext{\textsc{\textcolor{blue}{Bahr}{}\ledrightnote{\textcolor{blue}{Hermann Bahr}}} erklärt}{\lemma{\textnormal{\emph{Bahr erklärt}}}\Cendnote{\textnormal{siehe A. S.: \emph{Tagebuch}, 17. 4. 1896}}}\label{K_L02774-9h}, Du ſeiſt ein großer Künſtler? – Was haſt Du mir in der letzten Zeit
               Schlechtes geſchrieben?\pend
           \pstart
           11.) Mit dieſer \strikeout{N\textcolor{gray}{×}} Nummer iſt in Deinem Brief die \textcolor{pink}{Köln}{}\ledrightnote{\textcolor{pink}{Köln}}er
               Aufführung der »\textcolor{green}{Liebelei}{}\ledrightnote{\textcolor{green}{Liebelei. Schauspiel in drei Akten}}« bezeichnet. Ich gehe
               zu 12 über:\pend
           \pstart
           12.) Freut mich von Herzen, daß Du mit Deinem neuen \textcolor{green}{Stück}{}\ledrightnote{→\textcolor{green}{Freiwild. Schauspiel in 3 Akten}} auf die rechte Bahn kommſt. Schreib’ mir nur bald, wie
                  \strikeout{es} es vorwarts rückt. Könnteſt {\pb}Du mir nicht das \textcolor{green}{Manuſkript}{}\ledrightnote{→\textcolor{green}{Freiwild. Schauspiel in 3 Akten}} ſchicken, wenn Dus fertig haſt?\pend
           \pstart
           13.) \textsc{\textcolor{blue}{Albert}{}\ledrightnote{\textcolor{blue}{Henri Albert}}} ſehe ich kaum mehr. Er wird ein literariſcher Miſtbube (was er wohl ſtets war).
               Mich braucht er nicht mehr, und darum erklärt er, daß er ein \strikeout{Schriff}{ }Schriftſteller ſei und ich nur ein Journalist. Hat
               ganz Recht, der \textcolor{blue}{Mann}{}\ledrightnote{→\textcolor{blue}{Henri Albert}}, – ich
               meine: das Publicum und auch die Standesgenoſſen denken genau ſo wie er. Was {\pb}Deine Manuſkripte anlangt, ſo reclamire ſie von ihm
               und laß’ ſie vielleicht von einem der jungen Leute, die Dein \strikeout{\textcolor{green}{Stück\textcolor{gray}{e}}{}\ledrightnote{→\textcolor{green}{Liebelei. Schauspiel in drei Akten}}}{ }\textcolor{green}{Stück}{}\ledrightnote{→\textcolor{green}{Liebelei. Schauspiel in drei Akten}} überſetzen
               wollen, zur \uline{Probe} übertragen, \strikeout{da\textcolor{gray}{m}} damit man ſieht, was ſie können.\pend
           \pstart
           14.) Von der \textsc{\textcolor{blue}{Andreas-Salome}{}\ledrightnote{\textcolor{blue}{Lou Andreas-Salomé}}} höre ich nicht eine Zeile, noch ein Wort. Daß ſie \label{K_L02774-11v}\edtext{in \textcolor{pink}{Wien}{}\ledrightnote{\textcolor{pink}{Wien}} war}{\lemma{\textnormal{\emph{in Wien war}}}\Cendnote{\textnormal{Nach einer Reise nach \textcolor{pink}{St. Petersburg} im März 1895
                  lebte \textcolor{blue}{Lou Andreas-Salomé} mehrere Monate in
                     \textcolor{pink}{Wien}. Im Februar 1896 verließ sie die \textcolor{pink}{Stadt} wieder, kehrte aber bereits im Mai zurück. Der »Stimmungswechſel« drückt sich auch
                  dadurch aus, dass sie in \textcolor{blue}{Schnitzler}s \emph{\textcolor{green}{Tagebuch}} zuletzt am 25. 1. 1896 erwähnt
                  wurde und dann für ziemlich genau zehn Jahre nicht mehr.}}}\label{K_L02774-11h}, erfahre ich erſt
               aus Deinem Briefe. Den plötzlichen Stimmungswechſel Euch gegenüber kann ich mir
               ſchwer {\pb}erklären. Oder doch: ſie iſt eine ſehr
               launenhafte Frau. Sie braucht Abwechſlung in \strikeout{\textcolor{gray}{al}} ihrer Menſchen-Nahrung und zehrt nicht gern zweimal von denſelben. Sie hat mit
               Euch Alles gelebt, was ſie mit Euch leben konnte, – hat Euch Alles gegeben, was ſie
               Euch geben konnte. Daher wohl die beiderſeitige Erkältung. Feſthalten aus Moral, aus
               Treue, aus Freundſchaft {\pb}kennt ſie wohl kaum. \strikeout{Sie} Man vergißt bei ihr immer, daß ſie eine Frau iſt,
               und ſie iſt doch eine. Solange ſie mit Einem Freund iſt, iſt ſie beſtändig – inſoweit
               hat ſie männlichen Character. Aber das Weibliche an ihr iſt, daß ſie ihre
               Beſtändigkeiten wechſelt.\pend
           \pstart
           15.) Dein Leben nicht intereſſant? Haha! Ich wünſchte nur, Du könnteſt vier Wochen
               das \strikeout{M\textcolor{gray}{e}} meinige leben. {\pb}Dann würde \strikeout{De} Dir Dein Leben wie ein Roman vorkommen, – wie ein ſchöner Traum. Das Unglück
               iſt nur, daß \strikeout{\textcolor{gray}{m}} wir das, was uns das Leben ſchuldig bleibt, nach den Anſprüchen berechnen, die
                  \uline{wir} an dasſelbe ſtellen, – während wir ſo rechnen
               ſollten: ſoviel gewährt es den Anderen, ſoviel mir. Dann würde faſt immer ein \textsc{Plus} herauskommen, und bei Dir ein ganz gehöriges.\pend
           \pstart
           {\pb}16.) Hier iſt eine »Grabſchrift« mitgetheilt in
               Deinem Briefe, deren Genuß mir leider nicht zugänglich iſt, da ein oder zwei wichtige
               Worte darin infolge einer unerhörten Vertauſchung von I-Punkten und U-Haken vollſtändig unleſerlich ſind – ſelbſt
               für Einen, der \strikeout{es in einem} es, wie ich, nach
               fünfjähriger Lectüre Deiner Briefe, zu einer hübſchen Fertigkeit im
               Hieroglyphen-Entziffern gebracht hat.\pend
           \pstart
           {\pb}17.) »\textsc{\textcolor{brown}{L’Aube}{}\ledrightnote{\textcolor{brown}{L’Aube}}}« zahlt ſicher ſicher nichts, – da kannſt Du beruhigt ſein. Ich habe Deinen
                  \label{K_L02774-34v}\edtext{Namen genannt}{\lemma{\textnormal{\emph{Namen genannt}}}\Cendnote{\textnormal{keine Publikation von oder über \textcolor{blue}{Schnitzler} in \emph{\textcolor{green}{L’Aube}} bekannt}}}\label{K_L02774-34h}, weil ich es mir zum Geſetz \strikeout{\textcolor{gray}{ma}} gemacht, Jedem, der zu mir kommt und mich nach deutſcher Literatur frägt,
               zuerſt von Dir zu ſprechen. Schicke den Leuten irgend etwas Altes, was ſchon gedruckt
               war und wofür Du ſchon gezahlt worden biſt.\pend
           \pstart
           {\pb}18.) \textsc{\textcolor{blue}{Lalo}{}\ledrightnote{\textcolor{blue}{Pierre Lalo}}} will eine Arbeit über \label{K_L02774-35v}\edtext{»\textsc{\textcolor{blue}{Nietzsche}{}\ledrightnote{\textcolor{blue}{Friedrich Nietzsche}}}, Einfluß auf das moderne deutſche Geiſtesleben«}{\lemma{\textnormal{\emph{»Nietzsche, … Geiſtesleben«}}}\Cendnote{\textnormal{nicht bekannt}}}\label{K_L02774-35h} machen. Welches Buch, außer dem der \textsc{\textcolor{green}{\textcolor{blue}{Andreas-Salome}{}\ledrightnote{\textcolor{blue}{Lou Andreas-Salomé}}}{}\ledrightnote{→\textcolor{green}{Friedrich Nietzsche in seinen Werken}}}, kann man ihm zur Lectüre empfehlen? Bitte antworte mir – ausnahmsweiſe einmal
               – auf dieſe Frage.\pend
           \pstart
           19.) Schreib’ bald!\pend
           \pstart
           20.) Sei von ganzem Herzen gegrüßt!\pend
           \pstart
           Dein treuer {\\[\baselineskip]}\spacefill\mbox{Paul Goldmann.}\pend
           \leftskip=0em{}\pstart
           \noindent{}{\pb}\textsc{P. S.}{ }Morgen ſende ich Dir \label{K_L02774-76v}\edtext{»\textsc{\textcolor{green}{Aphrodite}{}\ledrightnote{\textcolor{green}{Aphrodite. Mœurs antiques}}}« von \textsc{\textcolor{blue}{Pierre Louÿs}{}\ledrightnote{\textcolor{blue}{Pierre Louÿs}}}}{\lemma{\textnormal{\emph{»Aphrodite« … Louÿs}}}\Cendnote{\textnormal{siehe A. S.: \emph{Lektüren}, Frankreich}}}\label{K_L02774-76h}. Schreib’ mir, wie Dirs gefällt, Aber zeig’ das \textcolor{green}{Buch}{}\ledrightnote{→\textcolor{green}{Aphrodite. Mœurs antiques}} weder \textsc{\textcolor{blue}{Bahr}{}\ledrightnote{\textcolor{blue}{Hermann Bahr}}} noch einem von den \textsc{\textcolor{blue}{Bahr}{}\ledrightnote{\textcolor{blue}{Hermann Bahr}}ischen}!\pend
           \pstart
           Der \label{K_L02774-32v}\edtext{\textcolor{pink}{Wien}{}\ledrightnote{\textcolor{pink}{Wien}}er »\textsc{\textcolor{green}{Figaro}{}\ledrightnote{\textcolor{green}{Figaro. Humoristisches Wochenblatt}}}«}{\lemma{\textnormal{\emph{Wiener »Figaro«}}}\Cendnote{\textnormal{\textcolor{blue}{Schnitzler} könnte \textcolor{blue}{Goldmann} auf die \textcolor{green}{Zeichnung} »\emph{\textcolor{green}{Unter Wiener
                        Grisetten}}« von \textcolor{blue}{Theodor Zasche}
                     hingewiesen haben. Darauf wird u. a. \textcolor{blue}{Schnitzler} im \textcolor{pink}{Café Griensteidl}
                     sitzend abgebildet. Vor dem Fenster des \textcolor{pink}{Café}s stehen die »\textcolor{green}{Wiener Grisetten}«, die darüber sprechen, dass \textcolor{blue}{Schnitzler} nun berühmt sei und sie abgeschrieben habe. Siehe \textcolor{blue}{Theodor Zasche}: \emph{\textcolor{green}{Unter Wiener Grisetten}}. In: \emph{\textcolor{green}{Wiener Luft. Beiblatt zum »Figaro«}}, Jg. 40, Nr. 17,
                           25. 4. 1896, S. [1].}}}\label{K_L02774-32h} hat
                  mich ſehr gefreut. Wie iſt Einem eigentlich zumuthe, wenn man berühmt iſt?\pend
           \endnumbering\briefempfaengerindex{Schnitzler, Arthur@\textsc{Schnitzler, Arthur}!zzzGoldmann, Paul@\emph{von Paul Goldmann}!1896-05-172@{17. 5. {[}1896{]}}|)be}\mylabel{h}\begin{anhang}\end{anhang}\normalsize

\doendnotes{C}
\bigskip
\vfill

\clearpage

\footnotesize

\lohead{\textsc{register}}

% Definiere theindex-Environment komplett neu ohne reledmac
\makeatletter
\renewenvironment{theindex}{%
  \section*{\indexname}%
  \setlength{\parindent}{0pt}%
  \setlength{\parskip}{0pt plus 0.3pt}%
  \let\item\@idxitem
}{%
  \clearpage
}
\makeatother

\IfFileExists{\jobname-pw.ind}{\input{\jobname-pw.ind}}{}

\end{document}

      