%% latex-korrekturansicht-vorspann.tex
%% Vorspann für die Korrekturansicht.
%% Lädt die gemeinsame Datei latex-vorspann.tex mit gesetztem Schalter.

\newif\ifkorrekturansicht
\korrekturansichttrue

\input{../tex-inputs/latex-vorspann}


\renewcommand{\erwaehntePersonen}{Personen: Heinrich Kanner, Ottilie Salten, Isidor Singer}
\renewcommand{\erwaehnteInstitutionen}{Institutionen: Die Zeit}
\renewcommand{\erwaehnteOrte}{Orte: Riedhof, Wien, Wipplingerstraße}
\renewcommand{\erwaehnteWerke}{}
\section[ Felix Salten an Arthur Schnitzler, {[}23?. 12. 1904{]}]{Felix Salten an Arthur Schnitzler, {[}23?. 12. 1904{]}}
\nopagebreak\mylabel{v}
\rehead{ }\normalsize\beginnumbering\briefempfaengerindex{Schnitzler, Arthur@\textsc{Schnitzler, Arthur}!zzzSalten, Felix@\emph{von Felix Salten}!1904-12-233@{{[}23?. 12. 1904{]}}|(be}
\toendnotes[C]{\smallbreak\pagebreak[2]}\Standort{CUL, Schnitzler, B 89, B 1.}
\physDesc{Briefkarte, 276 Zeichen
\newline{}Handschrift: schwarze Tinte, lateinische Kurrent
\newline{}Schnitzler: mit Bleistift falsch datiert: »21/12 904« 
\newline{}Ordnung: mit Bleistift von unbekannter Hand nummeriert: »{\pb}19\substVorne{}\textsuperscript{8}\substDazwischen{}7\substHinten{}« }\toendnotes[C]{\smallbreak}
\pstart
           \noindent{}{\pb}\textcolor{gray}{\textbf{DIE}}\pend
           
\pstart
           \textcolor{gray}{\textbf{\textcolor{brown}{ZEIT}{}\ledrightnote{\textcolor{brown}{Die Zeit}}}}\hfill \textcolor{gray}{\textbf{\emph{\textcolor{pink}{WIEN}{}\ledrightnote{\textcolor{pink}{Wien}}}}}{ }Freitag\pend
           
\pstart
           \textcolor{gray}{\textbf{\textcolor{pink}{Wien}{}\ledrightnote{\textcolor{pink}{Wien}}er Tageszeitung}}\hfill \textcolor{gray}{\textbf{\emph{\textcolor{pink}{I. Wipplingerstrasse 38}{}\ledrightnote{\textcolor{pink}{Wipplingerstraße}}}}}\pend
           
\pstart
           \textcolor{gray}{\textbf{Herausgeber:}}\pend
           
\pstart
           \textcolor{gray}{\textbf{\textbf{Prof. Dr. \textcolor{blue}{I. Singer}{}\ledrightnote{\textcolor{blue}{Isidor Singer}}}}}\pend
           
\pstart
           \textcolor{gray}{\textbf{\textbf{Dr. \textcolor{blue}{Heinrich Kanner}{}\ledrightnote{\textcolor{blue}{Heinrich Kanner}}}}}\pend
           
\pstart
           \textcolor{gray}{\textbf{\textbf{Feuilleton-Redaktion}}}\pend
           
\pstart
           Lieber, jetzt – \label{K_L03404-1v}\edtext{¾ 6}{\lemma{\textnormal{\emph{¾ 6}}}\Cendnote{\textnormal{17 Uhr 45}}}\label{K_L03404-1h} kommt Ihr \label{K_L03404-2v}\edtext{Brief}{\lemma{\textnormal{\emph{Brief}}}\Cendnote{\textnormal{Arthur Schnitzler an Felix Salten, [23. 12. 1904?]}}}\label{K_L03404-2h} – ich sende Ihnen also den Diener: ob’ \textcolor{blue}{Otti}{}\ledrightnote{\textcolor{blue}{Ottilie Salten}} mitkann weiß ich noch nicht bestimmt, jedenfalls bin ich also gegen
                  9 im \label{K_L03404-3v}\edtext{\textcolor{pink}{Riedhof}{}\ledrightnote{\textcolor{pink}{Riedhof}}}{\lemma{\textnormal{\emph{Riedhof}}}\Cendnote{\textnormal{siehe A. S.: \emph{Tagebuch}, 23. 12. 1904}}}\label{K_L03404-3h}.\pend
           
\pstart
           Meinem Genius thun Sie Unrecht, er heißt anders, und ich möchte auch nicht dass er
               zur Hölle fährt.\pend
           \pstart herzlich Ihr \spacefill\mbox{Salten}\pend{}\endnumbering\briefempfaengerindex{Schnitzler, Arthur@\textsc{Schnitzler, Arthur}!zzzSalten, Felix@\emph{von Felix Salten}!1904-12-233@{{[}23?. 12. 1904{]}}|)be}\mylabel{h}  \normalsize

\doendnotes{C}
\bigskip
\vfill

\clearpage

\footnotesize

\lohead{\textsc{register}}

% Definiere theindex-Environment komplett neu ohne reledmac
\makeatletter
\renewenvironment{theindex}{%
  \section*{\indexname}%
  \setlength{\parindent}{0pt}%
  \setlength{\parskip}{0pt plus 0.3pt}%
  \let\item\@idxitem
}{%
  \clearpage
}
\makeatother

\IfFileExists{\jobname-pw.ind}{\input{\jobname-pw.ind}}{}

\end{document}

      