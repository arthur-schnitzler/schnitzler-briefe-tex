%% latex-korrekturansicht-vorspann.tex
%% Vorspann für die Korrekturansicht.
%% Lädt die gemeinsame Datei latex-vorspann.tex mit gesetztem Schalter.

\newif\ifkorrekturansicht
\korrekturansichttrue

\input{../tex-inputs/latex-vorspann}


               \section[Paul Goldmann an Arthur Schnitzler, Paul Goldmann an Arthur Schnitzler, 4. 6. {[}1896{]}]{ Paul Goldmann an Arthur Schnitzler, 4. 6. {[}1896{]}}\nopagebreak\mylabel{v}\rehead{ }\normalsize\beginnumbering\briefempfaengerindex{Schnitzler, Arthur@\textsc{Schnitzler, Arthur}!zzzGoldmann, Paul@\emph{von Paul Goldmann}!1896-06-042@{4. 6. {[}1896{]}}|(be} \toendnotes[C]{\smallbreak\pagebreak[2]} \Standort{DLA, A:Schnitzler, HS.NZ85.1.3166.}
\physDesc{Brief, 2 Blätter, 8 Seiten
\newline{}Handschrift: blaue Tinte, deutsche Kurrent
\newline{}Schnitzler: 1) mit Bleistift das Jahr »96« vermerkt 2) mit rotem Buntstift vier Unterstreichungen}\toendnotes[C]{\smallbreak}\pstart
           \noindent{}{\pb}\textcolor{gray}{\textbf{\textbf{\textcolor{brown}{Frankfurter Zeitung}{}\ledrightnote{\textcolor{brown}{Frankfurter Zeitung}}}}}\pend
           \pstart
           \textcolor{gray}{\textbf{(\textcolor{brown}{\begin{otherlanguage}{french}Gazette de Francfort\end{otherlanguage}}{}\ledrightnote{\textcolor{brown}{Frankfurter Zeitung}}).}}\pend
           \pstart
           \textcolor{gray}{\textbf{\textbf{\begin{otherlanguage}{french}Fondateur M.\end{otherlanguage}{ }\textcolor{blue}{L. Sonnemann}{}\ledrightnote{\textcolor{blue}{Leopold Sonnemann}}.}}}\pend
           \pstart
           \begin{otherlanguage}{french}\textcolor{gray}{\textbf{\textcolor{green}{Journal}{}\ledrightnote{→\textcolor{green}{Frankfurter Zeitung}} politique,
                        financier,}}\end{otherlanguage}\pend
           \pstart
           \begin{otherlanguage}{french}\textcolor{gray}{\textbf{commercial et littéraire.}}\end{otherlanguage}\pend
           \pstart
           \begin{otherlanguage}{french}\textcolor{gray}{\textbf{\textbf{Paraissant trois fois par jour.}}}\end{otherlanguage}\pend
           \pstart
           \begin{otherlanguage}{french}\textcolor{gray}{\textbf{\textbf{Bureau à \textcolor{pink}{Paris}{}\ledrightnote{\textcolor{pink}{Paris}}}}}\end{otherlanguage}\hfill \textsc{\textcolor{pink}{Paris}{}\ledrightnote{\textcolor{pink}{Paris}}}, 4. Juni.\pend
           \pstart
           \begin{otherlanguage}{french}\textcolor{gray}{\textbf{\textbf{\textcolor{pink}{24. Rue Feydeau}{}\ledrightnote{\textcolor{pink}{rue Feydeau}}.}}}\end{otherlanguage}\pend
           \pstart\center{}Mein lieber Freund,\pend\pstart
           In Eile nur ein Wort des Dankes für Deinen lieben Brief!\pend
           \pstart
           So iſt es alſo abgemacht: Ich komme nach \label{K_L02776-1v}\edtext{\textcolor{pink}{Dänemark}{}\ledrightnote{\textcolor{pink}{Dänemark}}}{\lemma{\textnormal{\emph{Dänemark}}}\Cendnote{\textnormal{siehe Paul Goldmann an Arthur Schnitzler, 29. 4. [1896]}}}\label{K_L02776-1h}, – immer unter \strikeout{Vo\textcolor{gray}{r}} der Vorausſetzung, daß die weite Reiſe nicht über meine Mittel geht. Kannſt Du
               mir mittheilen, was man ungefähr pro Tag in \textsc{\textcolor{pink}{Scottsborg}{}\ledrightnote{→\textcolor{pink}{Skodsborg}}} braucht? {\pb}Ich freue mich unendlich darauf,
               Dich wiederzusehen. Du wirſt mir wohl noch weitere Details angeben. Wann reiſt \textsc{\textcolor{blue}{Richard}{}\ledrightnote{\textcolor{blue}{Richard Beer-Hofmann}}}? Zurück will ich dann über \label{K_L02776-2v}\edtext{\textcolor{pink}{Berlin}{}\ledrightnote{\textcolor{pink}{Berlin}}}{\lemma{\textnormal{\emph{Berlin}}}\Cendnote{\textnormal{siehe A. S.: \emph{Tagebuch}, 26. 8. 1896}}}\label{K_L02776-2h} gehen.\pend
           \pstart
           Die \label{K_L02776-4v}\edtext{Ernennung von \textsc{\textcolor{blue}{Antoine}{}\ledrightnote{\textcolor{blue}{André Antoine}}} zum Director des \textsc{\textcolor{brown}{Odéon}{}\ledrightnote{\textcolor{brown}{Odéon}}}}{\lemma{\textnormal{\emph{Ernennung … Odéon}}}\Cendnote{\textnormal{\textcolor{blue}{André Antoine} wurde 1896 neben \textcolor{blue}{Paul Ginisty} zum \textcolor{blue}{Ko-Direktor} des \emph{\textcolor{brown}{Odéon}} ernannt.}}}\label{K_L02776-4h} eröffnet uns eine
               unverhoffte Ausſicht, Dein \textcolor{green}{Stück}{}\ledrightnote{→\textcolor{green}{Liebelei. Schauspiel in drei Akten}} doch noch hier auf ein großes Theater zu bringen. Nächſtens mehr darüber.\pend
           \pstart
           {\pb}\textsc{M. \textcolor{blue}{Christian Schefer}{}\ledrightnote{\textcolor{blue}{Christian Schefer}}} beſuchte mich dieſer Tage u. ſagte mir, er habe einen \label{K_L02776-3v}\edtext{\textcolor{green}{Artikel}{}\ledrightnote{→\textcolor{green}{Un jeune écrivain viennois: M. Arthur Schnitzler}}}{\lemma{\textnormal{\emph{Artikel}}}\Cendnote{\textnormal{\textcolor{blue}{Christian Schefer}: \emph{\textcolor{green}{Un jeune écrivain viennois: M. Arthur Schnitzler}}. In:
                        \emph{\textcolor{green}{La Nouvelle Revue}}, Jg. 18, Nr. 100,
                        Mai–Juni 1896,
                     S. 855–859.}}}\label{K_L02776-3h} über Dich geſchrieben, und derſelbe werde bereits in
               den nächſten Wochen erſcheinen. Er hat natürlich auch einige Ausſtellungen gemacht,
               und ich habe mich wohl gehütet, \strikeout{zu} ihn daran zu
               verhindern (ſo dumm ich auch ſeine Einwände finde). Die »\textsc{\textcolor{green}{Nouvelle Revue}{}\ledrightnote{\textcolor{green}{La Nouvelle Revue}}}« iſt, wie Du {\pb}weißt, von der Deutſchen-Feindin
               \textsc{Madame \textcolor{blue}{Adam}{}\ledrightnote{\textcolor{blue}{Juliette Adam}}} redigirt. Noch nie iſt darin ein ausführlicher Artikel über einen deutſchen
               Schriftſteller erſchienen; die \textcolor{green}{Beſprechung}{}\ledrightnote{→\textcolor{green}{Un jeune écrivain viennois: M. Arthur Schnitzler}}, die Dir \textsc{M. \textcolor{blue}{Schefer}{}\ledrightnote{\textcolor{blue}{Christian Schefer}}} widmet, iſt darum noch aus dieſem beſonderen Grunde ehrenvoll für Dich.\pend
           \pstart
           Von mir ſoll ich Dir ſchreiben? Was denn, bitte? Ich weiß {\pb}nichts, was Dich intereſſiren könnte. Mein Leben
               ſteht überdies faſt jeden Tag in der \textcolor{green}{Frankfurter
                  Zeitung}{}\ledrightnote{\textcolor{green}{Frankfurter Zeitung}}.\pend
           \pstart
           Die »\textsc{\textcolor{green}{Illustration}{}\ledrightnote{\textcolor{green}{L’Illustration}}}« ſchicke ich Dir dieſer Tage.\pend
           \pstart
           Gewiß, \textsc{\textcolor{blue}{Dehmel}{}\ledrightnote{\textcolor{blue}{Richard Dehmel}}} iſt mir widerwärtig – oh, und wie!\pend
           \pstart
           Gewiß, der kleine \textsc{\textcolor{blue}{Loris}{}\ledrightnote{\textcolor{blue}{Hugo von Hofmannsthal}}} iſt nicht manierirt, {\pb}ſondern ehrlich – oder
               vielmehr ſeine Manier iſt Ehrlichkeit. Aber das iſt eben das Schlimme, das eine ſo
               ungünſtige \strikeout{Prog} Prognoſe rechtfertigt. \strikeout{\textcolor{gray}{W}} Wenns nur in der Haut ſäße! Aber es ſitzt tiefer, im Kern. Man hat dem kleinen
                  \textcolor{blue}{Burſchen}{}\ledrightnote{→\textcolor{blue}{Hugo von Hofmannsthal}} ſolange
               eingeredet, daß er ein Genie iſt, bis er dahin gekommen iſt, jeden Sprung ſeiner
               Gedanken für genial zu nehmen. {\pb}Er hat nicht eine
               der nothwendigſten Eigenſchaften des Talents: Selbſtzucht. Er empfindet drauf
               los und ſchreibt \label{K_L02776-15v}\edtext{\textsc{idem}}{\lemma{\textnormal{\emph{idem}}}\Cendnote{\textnormal{lateinisch: entsprechend}}}\label{K_L02776-15h}. Auch liegt
               Verbildung vor, – Überſtopfung mit Wiſſenskram. Man hat dieſen jungen \textcolor{blue}{Mann}{}\ledrightnote{→\textcolor{blue}{Hugo von Hofmannsthal}} ſyſtematiſch zum Dichter
               ausbilden wollen, und das geht nicht. Die \textsc{\textcolor{blue}{Goethe}{}\ledrightnote{\textcolor{blue}{Johann Wolfgang von Goethe}}s} laſſen ſich nicht züchten. Das
               Beſte in der Entwickelung {\pb}thut der Zufall (oder das
               Leben, wenn man demſelben Ding einen anderen Namen geben will, oder die Natur, was
               auch dasſelbe iſt).\pend
           \pstart
           Grüß’ Dich Gott, mein lieber Freund!\pend
           \pstart
           Dein treuer {\\[\baselineskip]}\spacefill\mbox{Paul Goldmann.}\pend
           \leftskip=0em{}\endnumbering\briefempfaengerindex{Schnitzler, Arthur@\textsc{Schnitzler, Arthur}!zzzGoldmann, Paul@\emph{von Paul Goldmann}!1896-06-042@{4. 6. {[}1896{]}}|)be}\mylabel{h}\begin{anhang}\end{anhang}\normalsize

\doendnotes{C}
\bigskip
\vfill

\clearpage

\footnotesize

\lohead{\textsc{register}}

% Definiere theindex-Environment komplett neu ohne reledmac
\makeatletter
\renewenvironment{theindex}{%
  \section*{\indexname}%
  \setlength{\parindent}{0pt}%
  \setlength{\parskip}{0pt plus 0.3pt}%
  \let\item\@idxitem
}{%
  \clearpage
}
\makeatother

\IfFileExists{\jobname-pw.ind}{\input{\jobname-pw.ind}}{}

\end{document}

      