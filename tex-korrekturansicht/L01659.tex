%% latex-korrekturansicht-vorspann.tex
%% Vorspann für die Korrekturansicht.
%% Lädt die gemeinsame Datei latex-vorspann.tex mit gesetztem Schalter.

\newif\ifkorrekturansicht
\korrekturansichttrue

\input{../tex-inputs/latex-vorspann}


               \section[Hugo von Hofmannsthal an Arthur Schnitzler, 23. 2. 1907]{ Hugo von Hofmannsthal an Arthur Schnitzler, 23. 2. 1907}\nopagebreak\mylabel{v}\rehead{ }\normalsize\beginnumbering\briefempfaengerindex{Schnitzler, Arthur@\textsc{Schnitzler, Arthur}!zzzHofmannsthal, Hugo von@\emph{von Hugo von Hofmannsthal}!1907-02-231@{23. 2. 1907}|(be} \toendnotes[C]{\smallbreak\pagebreak[2]} \Standort{CUL, Schnitzler, B 43.}
\physDesc{Postkarte
\newline{}Handschrift: schwarze Tinte, deutsche Kurrent\newline{}Versand: 1) Rohrpost 2) Stempel: »\nobreak{}\oindex{XVIII., Waehring@\textbf{XVIII., Währing}, \emph{Bezirk (A.BZK)}|pwk}18/1 Wien, 23 II 07, 2\textsuperscript{50}\nobreak{}«. 
\newline{}Schnitzler: mit Bleistift datiert: »Anf 907« \newline{}Ordnung: 1) mit Bleistift von unbekannter Hand nummeriert:
                                    »271« 2) mit Bleistift von unbekannter Hand nummeriert:
                                    »274«\newline{}Zusatz: mit Bleistift von unbekannter Hand ein Frauenportrait quer unten
                                 rechts auf die Karte gezeichnet }\buchAbdrucke{\weitereDrucke{Hugo von Hofmannsthal, Arthur Schnitzler: \emph{Briefwechsel}. Hg. Therese Nickl und Heinrich Schnitzler. Frankfurt am Main: \emph{S. Fischer} 1964, S. 226.} }\pstart{}{\pb}\textsc{Herrn D\textsuperscript{r} Arthur Schnitzler}\pend{}\pstart{}\textcolor{pink}{\textsc{Wien}}{}\ledrightnote{\textcolor{pink}{Wien}}\pend{}\pstart{}\textcolor{pink}{\textsc{XVII Spöttelgasse 7}}{}\ledrightnote{\textcolor{pink}{Edmund-Weiß-Gasse}}\pend{}\pstart{}nächſt \textcolor{pink}{Türkenschanzstrasse}{}\ledrightnote{\textcolor{pink}{Türkenschanzstrasse}}\pend{}\pstart{}\textsc{pneumatisch}\pend{}{\bigskip}\pstart
           \noindent{}{\pb}Wir gehen heute abend zu den \textcolor{blue}{† † † Bären}{}\ledrightnote{\textcolor{blue}{Paula Beer-Hofmann}{\newline}\textcolor{blue}{Richard Beer-Hofmann}} und möchten am Weg bei Euch
               hineinſchauen d. h. circa 5\textsuperscript{h} Uhr. Falls
               läſtig bitte abſagen telephoniſch N. 229.\pend
           \pstart \spacefill\mbox{Hugo.}\pend{}\endnumbering\briefempfaengerindex{Schnitzler, Arthur@\textsc{Schnitzler, Arthur}!zzzHofmannsthal, Hugo von@\emph{von Hugo von Hofmannsthal}!1907-02-231@{23. 2. 1907}|)be}\mylabel{h}  \normalsize

\doendnotes{C}
\bigskip
\vfill

\clearpage

\footnotesize

\lohead{\textsc{register}}

% Definiere theindex-Environment komplett neu ohne reledmac
\makeatletter
\renewenvironment{theindex}{%
  \section*{\indexname}%
  \setlength{\parindent}{0pt}%
  \setlength{\parskip}{0pt plus 0.3pt}%
  \let\item\@idxitem
}{%
  \clearpage
}
\makeatother

\IfFileExists{\jobname-pw.ind}{\input{\jobname-pw.ind}}{}

\end{document}

      