%% latex-korrekturansicht-vorspann.tex
%% Vorspann für die Korrekturansicht.
%% Lädt die gemeinsame Datei latex-vorspann.tex mit gesetztem Schalter.

\newif\ifkorrekturansicht
\korrekturansichttrue

\input{../tex-inputs/latex-vorspann}


         
         \renewcommand{\erwaehntePersonen}{Personen: Richard Beer-Hofmann, Georg Brandes, Max Eugen Burckhard, Auguste Chlum, Isidor Fuchs, Marie Glümer, Alfred Kerr, Karl Kraus, Paul Lindau, Theodor Loewe, Helene Meyer-Cohn, Alexander Meyer-Cohn, Edith Philipp, Felix Salten, Paul Schlenther, Olga Schnitzler, Karl Schönherr, Elisabeth Steinrück, Anna Wendt, Anton Pavlovič Čechov}
         \renewcommand{\erwaehnteInstitutionen}{Institutionen: Burgtheater, Neue Freie Presse, Secessionsbühne}
         \renewcommand{\erwaehnteOrte}{Orte: Baden bei Wien, Berlin, Breslau, Lobe-Theater, Rotensterngasse}
         \renewcommand{\erwaehnteWerke}{Werke: Antworten des Herausgebers. Habitué, Der Bär, Der Gemeine. Schauspiel in drei Aufzügen, Der Schleier der Beatrice. Schauspiel in fünf Akten, Die Bildschnitzer, Die Fackel, Die Zeit. Wiener Wochenschrift, Ein Sommer in China. Reisebilder, Ein Sommer in China. Reisebilder. Zweite, durchgesehene und vermehrte Auflage, Erklärung [Schleier der Beatrice], Neue Freie Presse, [Burckhard über Schnitzler-Schlenther], [Die Affaire Schlenther-Schnitzler]}
               \section[ Paul Goldmann an Arthur Schnitzler, 14. 10. {[}1900{]}]{Paul Goldmann an Arthur Schnitzler, 14. 10. {[}1900{]}}\nopagebreak\mylabel{v}\rehead{ }\normalsize\beginnumbering\briefempfaengerindex{Schnitzler, Arthur@\textsc{Schnitzler, Arthur}!zzzGoldmann, Paul@\emph{von Paul Goldmann}!1900-10-141@{14. 10. {[}1900{]}}|(be} \toendnotes[C]{\smallbreak\pagebreak[2]} \Standort{DLA, A:Schnitzler, HS.NZ85.1.3170.}
\physDesc{Brief, 2 Blätter, 8 Seiten
\newline{}Handschrift: blaue Tinte, deutsche Kurrent
\newline{}Schnitzler: 1) mit Bleistift das Jahr »{[}1{]}900« vermerkt  2) mit rotem Buntstift acht Unterstreichungen und ein
                                    »X«}\toendnotes[C]{\smallbreak}\pstart
           \raggedleft{}{\pb}\textcolor{pink}{Berlin}{}\ledrightnote{\textcolor{pink}{Berlin}}, 14. Oktober.\pend
           \pstart\center{}Mein lieber Freund,\pend\pstart
           Heut am Sonntag habe
               ich endlich ein paar Minuten frei zu einem Briefe an Dich.\pend
           \pstart
           Die \label{K_L02936-1v}\edtext{»\textcolor{green}{\textcolor{green}{Fackel}{}\ledrightnote{{$\rightarrow$}\textcolor{green}{Die Fackel}{\newline}{$\rightarrow$}\textcolor{green}{[Die Affaire Schlenther-Schnitzler]}{\newline}{$\rightarrow$}\textcolor{green}{Antworten des Herausgebers. Habitué}}}{}\ledrightnote{\textcolor{green}{Die Fackel}}«}{\lemma{\textnormal{\emph{»Fackel«}}}\Cendnote{\textnormal{Bezugnahme auf \textcolor{blue}{Karl Kraus}: \emph{\textcolor{green}{[Die Affaire Schlenther-Schnitzler]}}. In: \emph{\textcolor{green}{Die Fackel}}, Jg. 2, Nr. 53, Mitte September 1900, S. 1–6 und auf \textcolor{blue}{Karl Kraus}: \emph{\textcolor{green}{Antworten des Herausgebers. Habitué}}. In: \emph{\textcolor{green}{Die Fackel}}, Jg. 2, Nr. 54, Ende September 1900, S. 25–26. Siehe zum
                  Konflikt zwischen \textcolor{blue}{Schnitzler} und \textcolor{blue}{Paul Schlenther} auch Paul Goldmann an Arthur Schnitzler, 12. 11. [1899].}}}\label{K_L02936-1h}. Was willſt Du von dem
               Lausbuben? Offen geſtanden, ich hätte noch Schlimmeres erwartet. Im Übrigen hat
                  \label{K_L02936-2v}\edtext{\textsc{\textcolor{blue}{Burckhardt}{}\ledrightnote{\textcolor{blue}{Max Eugen Burckhard}}} in der »\textcolor{green}{Zeit}{}\ledrightnote{\textcolor{green}{Die Zeit. Wiener Wochenschrift}}«}{\lemma{\textnormal{\emph{Burckhardt in der »Zeit«}}}\Cendnote{\textnormal{\textcolor{blue}{Max Burckhard}: \emph{\textcolor{green}{XXXX}}. In: \emph{\textcolor{green}{Zeit}},
                     Jg. YY, Nr. YY, YY. 10. 1900, S. YY.}}}\label{K_L02936-2h} das wahre Wort
                  \textcolor{green}{geſchrieben}{}\ledrightnote{{$\rightarrow$}\textcolor{green}{[Burckhard über Schnitzler-Schlenther]}}: die Leute
               rächen ſich jetzt an Dir, weil ſie Dir haben applaudiren müſſen. Auf das Geſindel im
               Allgemeinen war niemals zu rechnen. Ob die \label{K_L02936-3v}\edtext{\textcolor{green}{Aktion}{}\ledrightnote{{$\rightarrow$}\textcolor{green}{Erklärung [Schleier der Beatrice]}}}{\lemma{\textnormal{\emph{Aktion}}}\Cendnote{\textnormal{siehe Bahr/Schnitzler, T030017}}}\label{K_L02936-3h} ſonſt wirkungslos geblieben, wird ſich zeigen. Welche Wirkung hätte {\pb}denn auch kommen ſollen? Die Hauptſache war, daß der
               Herr \textsc{\textcolor{blue}{Schlenther}{}\ledrightnote{\textcolor{blue}{Paul Schlenther}}} eine Antwort auf ſein unerhörtes Benehmen bekam. Und den ſchlechten Ruf, den er
               ohnedies hat, hat dieſe Affaire nur noch vergrößert. Er hat’s geſpürt und wirds noch
               weiter ſpüren. Dieſe Affaire, mag man ſagen, was man will, iſt ein Grund mehr für
               ſeinen Weggang vom \textcolor{brown}{Burgtheater}{}\ledrightnote{\textcolor{brown}{Burgtheater}}. Selbſt \textcolor{pink}{hier}{}\ledrightnote{{$\rightarrow$}\textcolor{pink}{Berlin}}, wo man ihn für einen Gott
               hält, hat ſie ihm geſchadet{\dotsfive}\pend
           \pstart
           Dein \label{K_L02936-4v}\edtext{»Ohrenleiden«}{\lemma{\textnormal{\emph{»Ohrenleiden«}}}\Cendnote{\textnormal{\textcolor{blue}{Schnitzler} litt an Otosklerose
                  (Verknöcherung des Innenohrs mit zunehmender Schwerhörigkeit)}}}\label{K_L02936-4h}: darauf weiß
               ich nur \uline{eine} Antwort: Heirathen. Ich ſchwöre Dir:
               wenn Du Frau {\pb}und Kinder haben wirſt, wirſt Du Dich
               weniger mit Deinem Ohre beſchäftigen; und wenn Du Dich weniger damit beſchäftigen
               wirſt, \strikeout{\textcolor{gray}{r}\textcolor{gray}{×}\-\textcolor{gray}{×}} wirſt Du weniger darunter leiden.\pend
           \pstart
           Mit \textsc{\textcolor{blue}{Lindau}{}\ledrightnote{\textcolor{blue}{Paul Lindau}}} werde ich bei nächſter Gelegenheit \label{K_L02936-5v}\edtext{wegen \textsc{\textcolor{blue}{Salten}{}\ledrightnote{\textcolor{blue}{Felix Salten}}}}{\lemma{\textnormal{\emph{wegen Salten}}}\Cendnote{\textnormal{unklar; womöglich ging es um eine
                  mögliche Uraufführung von \textcolor{blue}{Salten}s Dreiakter
                     \emph{\textcolor{green}{Der Gemeine}}}}}\label{K_L02936-5h} ſprechen.\pend
           \pstart
           \textsc{\textcolor{blue}{Kerr}{}\ledrightnote{\textcolor{blue}{Alfred Kerr}}} ſehe ich ſehr ſelten. Wenn wir uns ſehen, ſprechen wir ſehr freundſchaftlich
               miteinander. Er ſteckt tief in ſeinem \label{K_L02936-6v}\edtext{\textcolor{blue}{Liebeswonnen}{}\ledrightnote{{$\rightarrow$}\textcolor{blue}{Anna Wendt}}}{\lemma{\textnormal{\emph{Liebeswonnen}}}\Cendnote{\textnormal{siehe Paul Goldmann an Arthur Schnitzler, 18. 4. [1900]}}}\label{K_L02936-6h} und ſtrebt der Erfüllung ſeiner Wünſche zu, was mit großen {\pb}Kämpfen verbunden ſcheint. Aber \textcolor{blue}{er}{}\ledrightnote{{$\rightarrow$}\textcolor{blue}{Alfred Kerr}} wird es ſchon durchſetzen. Er und das
                  \textcolor{blue}{Mädel}{}\ledrightnote{{$\rightarrow$}\textcolor{blue}{Anna Wendt}} ſcheinen ſich ſehr
               zu lieben, und das iſt die Hauptſache.\pend
           \pstart
           Ich bin mit dem Hauſe \textsc{\textcolor{blue}{M.-C.}{}\ledrightnote{{$\rightarrow$}\textcolor{blue}{Helene Meyer-Cohn}{\newline}{$\rightarrow$}\textcolor{blue}{Alexander Meyer-Cohn}}} vollkommen auseinander. Dieſe ganze Geſchichte hat für mich mit einem großen
               Ekel geendet, – einem Ekel namentlich vor der »Geſellſchaft«, vor dieſen Leuten, die
               Einen nicht verſtehen und die Einen zur Tafel ziehen als Hanswurſt. Aber wehe, wenn
               man verſuchen will, auch einmal ſein Leben zu leben! {\pb}Im Übrigen hat die \textcolor{blue}{Kleine}{}\ledrightnote{{$\rightarrow$}\textcolor{blue}{Helene Meyer-Cohn}}
               ja ganz recht gehabt, und ich bin fett und grotesk und nicht fähig, Liebe \strikeout{zu \textcolor{gray}{e}i} einzuflößen. Ich habe mich in
               die Arbeit geſtürzt, um das Alles zu vergeſſen.\pend
           \pstart
           \textsc{\textcolor{blue}{Brandes}{}\ledrightnote{\textcolor{blue}{Georg Brandes}}} iſt \textcolor{pink}{hier}{}\ledrightnote{{$\rightarrow$}\textcolor{pink}{Berlin}} und erzählt mir
               viel von ſeinen \label{K_L02936-9v}\edtext{Liebesabenteuern}{\lemma{\textnormal{\emph{Liebesabenteuern}}}\Cendnote{\textnormal{vgl. Paul Goldmann an Arthur Schnitzler, 4. 10. [1900]}}}\label{K_L02936-9h}. Dieſer Tage kommt auch ſeine \textcolor{blue}{Tochter}{}\ledrightnote{{$\rightarrow$}\textcolor{blue}{Edith Philipp}}.\pend
           \pstart
           Nach \textcolor{pink}{Breslau}{}\ledrightnote{\textcolor{pink}{Breslau}} zur \label{K_L02936-11v}\edtext{Aufführung der »\textsc{\textcolor{green}{Beatrice}{}\ledrightnote{\textcolor{green}{Der Schleier der Beatrice. Schauspiel in fünf Akten}}}«}{\lemma{\textnormal{\emph{Aufführung der »Beatrice«}}}\Cendnote{\textnormal{\emph{\textcolor{green}{Der Schleier der Beatrice}} wurde am 1. 12. 1900 am \textcolor{pink}{Lobe-Theater} in \textcolor{pink}{Breslau} uraufgeführt.
                  Zu einem früheren Zeitpunkt war der 17. 11. 1900
                  als Premierentermin geplant.}}}\label{K_L02936-11h} möchte ich unendlich gern fahren. Ich habe das
               hier mit meinem Collegen \textsc{\textcolor{blue}{Fuchs}{}\ledrightnote{{$\rightarrow$}\textcolor{blue}{Isidor Fuchs}}} beſprochen, und {\pb}er ſagte mir: »Ja, fahren Sie
               nur! Aber den Direktor \textsc{\textcolor{blue}{Löwe}{}\ledrightnote{\textcolor{blue}{Theodor Loewe}}} dürfen Sie nicht tadeln; er iſt bei \textcolor{brown}{uns}{}\ledrightnote{{$\rightarrow$}\textcolor{brown}{Neue Freie Presse}}{ }\label{K_L02936-12v}\edtext{\textsc{persona gratissima}}{\lemma{\textnormal{\emph{persona gratissima}}}\Cendnote{\textnormal{lateinisch: willkommene Person, hier im
                  Sinne von ›immun‹}}}\label{K_L02936-12h}.« Alſo, ich ſetze den Fall, die Aufführung könnte den
               Aufgaben des \textcolor{green}{Stück}{}\ledrightnote{{$\rightarrow$}\textcolor{green}{Der Schleier der Beatrice. Schauspiel in fünf Akten}}es nicht
               gerecht werden (was ich befürchte), ſo werde ich das nicht ſagen dürfen, oder man
               wird es mir ſtreichen. Unter dieſen Umſtänden iſt es wirklich beſſer, nicht
               hinzugehen und die Berichterſtattung dem Direktor \textsc{\textcolor{blue}{Löwe}{}\ledrightnote{\textcolor{blue}{Theodor Loewe}}} zu überlaſſen, der ſelbſt an die \textcolor{brown}{N. Fr. Pr.}{}\ledrightnote{\textcolor{brown}{Neue Freie Presse}}{ }{\pb}zu telegraphiren pflegt und unter allen Umſtänden
                  \label{K_L02936-13v}\edtext{das Beſte ſagen wird}{\lemma{\textnormal{\emph{das Beſte ſagen wird}}}\Cendnote{\textnormal{siehe zur Berichterstattung zur
                  Uraufführung von \emph{\textcolor{green}{Der Schleier der Beatrice}} in
                  der \emph{\textcolor{green}{Neuen Freien Presse}}{ }\textcolor{blue}{Goldmann}s Brief vom 3. 12. [1900]}}}\label{K_L02936-13h}.\pend
           \pstart
           Grüße mir die ſtrebſamen \label{K_L02936-17v}\edtext{\textcolor{blue}{Fräulein}{}\ledrightnote{{$\rightarrow$}\textcolor{blue}{Olga Schnitzler}{\newline}{$\rightarrow$}\textcolor{blue}{Elisabeth Steinrück}} aus der \textcolor{pink}{Rothen-Stern-Gaſſe}{}\ledrightnote{\textcolor{pink}{Rotensterngasse}}}{\lemma{\textnormal{\emph{Fräulein … Rothen-Stern-Gaſſe}}}\Cendnote{\textnormal{siehe Paul Goldmann an Arthur Schnitzler, 19. 9. [1900]}}}\label{K_L02936-17h} und theile mir deren genaue Adreſſe mit (Name und Hausnummer), damit ich
               ihnen mein \label{K_L02936-14v}\edtext{\textcolor{green}{Buch}{}\ledrightnote{{$\rightarrow$}\textcolor{green}{Ein Sommer in China. Reisebilder. Zweite, durchgesehene und vermehrte Auflage}}}{\lemma{\textnormal{\emph{Buch}}}\Cendnote{\textnormal{die \textcolor{green}{zweite Auflage} von \emph{\textcolor{green}{Ein
                     Sommer in China}}, siehe Paul Goldmann an Arthur Schnitzler, 4. 10. [1900]}}}\label{K_L02936-14h} ſchicken kann.\pend
           \pstart
           Die \textsc{\textcolor{blue}{Glümerinnen}{}\ledrightnote{{$\rightarrow$}\textcolor{blue}{Marie Glümer}{\newline}{$\rightarrow$}\textcolor{blue}{Auguste Chlum}}} ſind wieder beieinander, und Frl. \textsc{\textcolor{blue}{Mizzi}{}\ledrightnote{\textcolor{blue}{Marie Glümer}}} hat neulich {\pb}einen ſehr \strikeout{ſchöne\textcolor{gray}{nn}} ſchönen und ſehr verdienten \label{K_L02936-20v}\edtext{Erfolg}{\lemma{\textnormal{\emph{Erfolg}}}\Cendnote{\textnormal{als weibliche \textcolor{blue}{Hauptrolle} der \textcolor{pink}{Berlin}er \emph{\textcolor{brown}{Secessionsbühne}} in \emph{\textcolor{green}{Die Bildschnitzer}}
                  von \textcolor{blue}{Karl Schönherr} und in \emph{\textcolor{green}{Der Bär}} von \textcolor{blue}{Anton
                     Čechov}}}}\label{K_L02936-20h} gehabt bei Publikum und Kritik. Auch ſie ſehe ich ſelten,
               und ich lebe, eingeſponnen in Arbeit, ein ödes und nutzloſes Leben.\pend
           \pstart
           Was macht \textsc{\textcolor{blue}{Richard}{}\ledrightnote{\textcolor{blue}{Richard Beer-Hofmann}}}? Keine Möglichkeit, von ihm eine Antwort zu bekommen.\pend
           \pstart
           Schreib’ mir bald und \strikeout{ſei} ſei von Herzen
               gegrüßt!{\\[\baselineskip]} Dein {\\[\baselineskip]}\spacefill\mbox{Paul Goldmnn}\pend
           \leftskip=0em{}\endnumbering\briefempfaengerindex{Schnitzler, Arthur@\textsc{Schnitzler, Arthur}!zzzGoldmann, Paul@\emph{von Paul Goldmann}!1900-10-141@{14. 10. {[}1900{]}}|)be}\mylabel{h}  \normalsize

\doendnotes{C}
\bigskip
\vfill

\clearpage

\footnotesize

\lohead{\textsc{register}}

% Definiere theindex-Environment komplett neu ohne reledmac
\makeatletter
\renewenvironment{theindex}{%
  \section*{\indexname}%
  \setlength{\parindent}{0pt}%
  \setlength{\parskip}{0pt plus 0.3pt}%
  \let\item\@idxitem
}{%
  \clearpage
}
\makeatother

\IfFileExists{\jobname-pw.ind}{\input{\jobname-pw.ind}}{}

\end{document}

      