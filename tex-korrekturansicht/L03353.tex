%% latex-korrekturansicht-vorspann.tex
%% Vorspann für die Korrekturansicht.
%% Lädt die gemeinsame Datei latex-vorspann.tex mit gesetztem Schalter.

\newif\ifkorrekturansicht
\korrekturansichttrue

\input{../tex-inputs/latex-vorspann}


\renewcommand{\erwaehntePersonen}{Personen: Hermann Bahr, Paul Goldmann, Markus Hajek, Theodor Herzl, Gustav Schwarzkopf, Isidor Singer}
\renewcommand{\erwaehnteOrte}{Orte: Berlin, Frankgasse 1, Wien}
\renewcommand{\erwaehnteWerke}{Werke: Anatol, Arthur Schnitzler und sein »Reigen«, Berliner Theater. (»Der Schleier der Beatrice« von Arthur Schnitzler.), Das Märchen. Schauspiel in drei Aufzügen, Das junge Österreich. II, Der Schleier der Beatrice. Schauspiel in fünf Akten, Der grüne Kakadu. Groteske in einem Akt, Deutsche Zeitung, Die Frau mit dem Dolche, Die Gefährtin. Schauspiel in einem Akt, Die letzten Masken, Feuilleton. Carl-Theater. (»Freiwild«, Schauspiel von Arthur Schnitzler.), Liebelei. Schauspiel in drei Akten, Literatur, Neue Freie Presse, Paracelsus. Versspiel in einem Akt, Reigen. Zehn Dialoge, Wiener Allgemeine Zeitung, »Der Schleier der Beatrice«. (Zum erstenmale aufgeführt im Lobe-Theater zu Breslau)}
\section[ Felix Salten an Arthur Schnitzler, {[}9. 11. 1903{]}]{Felix Salten an Arthur Schnitzler, {[}9. 11. 1903{]}}
\nopagebreak\mylabel{v}
\rehead{ }\normalsize\beginnumbering\briefempfaengerindex{Schnitzler, Arthur@\textsc{Schnitzler, Arthur}!zzzSalten, Felix@\emph{von Felix Salten}!1903-11-092@{{[}9. 11. 1903{]}}|(be}
\toendnotes[C]{\smallbreak\pagebreak[2]}\Standort{CUL, Schnitzler, B 89, A 2.}
\physDesc{Brief, 2 Blätter, 5 Seiten, 14228 Zeichen
\newline{}Handschrift: Bleistift, lateinische Kurrent
\newline{}Schnitzler: mit Bleistift datiert: »Nov. 903« 
\newline{}Ordnung: mit Bleistift von unbekannter Hand nummeriert: »179« }\toendnotes[C]{\smallbreak}
\pstart
           \raggedleft{}{\pb}\label{K_L03353-1v}\edtext{Montag}{\lemma{\textnormal{\emph{Montag}}}\Cendnote{\textnormal{Da der Brief \textcolor{blue}{Schnitzler}s, auf den \textcolor{blue}{Salten} hier reagierte, auf den 7. 11. 1903 datiert und \textcolor{blue}{Schnitzler} bereits am 10. 11. 1903 antwortete, ist dieser Brief auf den
                        [9. 11. 1903]
                     datierbar.}}}\label{K_L03353-1h}{ }Abds\pend
           
\pstart
           Lieber! Wenn ein Werk vor einem gutwilligen, unbeeinflußten Hörer
               seine Wirkung verfehlt, dann muß das Werk daran irgendwie schuld sein. So habe ich
               immer gedacht, und so denke ich auch heute. Da ich nun annehme, dass Sie meinem \textcolor{green}{Feuilleton}{}\ledrightnote{{$\rightarrow$}\textcolor{green}{Arthur Schnitzler und sein »Reigen«}} ein gutwilliger,
               unbeeinflußter Leser waren, so ist einfach mein \textcolor{green}{Feuilleton}{}\ledrightnote{{$\rightarrow$}\textcolor{green}{Arthur Schnitzler und sein »Reigen«}} mißlungen. Es kann offenbar nicht anders
               sein.\pend
           
\pstart
           Das Entscheidende ist mir: Sie fühlen sich verletzt, und: Sie haben durch mein \textcolor{green}{Reigen-Feuilleton}{}\ledrightnote{{$\rightarrow$}\textcolor{green}{Arthur Schnitzler und sein »Reigen«}} eine bittere
               Stunde gehabt. Ich werde in meiner Antwort, (die Sie doch erwarten?) auf nichts
               anderes Bedacht nehmen, als auf diese beiden Umstände. Denn es war nicht meine
               Absicht, Sie zu verletzen, und das \textcolor{green}{Feuill.}{}\ledrightnote{{$\rightarrow$}\textcolor{green}{Arthur Schnitzler und sein »Reigen«}} wurde nicht geschrieben, um die Stunde, in der Sie es lesen, zu
               einer bitteren zu machen. Ganz im Gegentheil, wie Sie mir hoffentlich glauben.\pend
           
\pstart
           Wenn meine \textcolor{green}{Arbeit}{}\ledrightnote{{$\rightarrow$}\textcolor{green}{Arthur Schnitzler und sein »Reigen«}} trotzdem so
               auf Sie gewirkt hat, dann ist eben »das« nicht herausgekommen, was ich herausbringen
               wollte. Nachdem ich seit gestern diese Sache ernsthaft
               überlegt habe, nachdem ich alle Empfindlichkeiten, die sich regen wollten, und alle
               sonstigen Unterstimmen zum Schweigen gebracht habe, bin ich zu diesem Resultat
               gelangt. Ich sehe heute zwar selbst noch nicht genau,
               wo der Fehler stecken mag, aber ich zweifle nicht, dass \substVorne{}\textsuperscript{d}\substDazwischen{}e\substHinten{}in Fehler an meiner \textcolor{green}{Arbeit}{}\ledrightnote{{$\rightarrow$}\textcolor{green}{Arthur Schnitzler und sein »Reigen«}} vorhanden ist; ich \uline{will} daran nicht
               zweifeln, und ich muß nun versuchen, das \textcolor{green}{Feuilleton}{}\ledrightnote{{$\rightarrow$}\textcolor{green}{Arthur Schnitzler und sein »Reigen«}} zu erklären, außerdem aber auf eine
               Beschuldigung, die Sie gegen mich erheben, antworten. Zwei schwere und \strikeout{\textcolor{gray}{mißl}ige} mißliche Dinge.\pend
           
\pstart
           Zuerst also die Beschuldigung\textcolor{gray}{,} ich hätte mündlich, und bisher auch
               öffentlich-kritisch eine andere Meinung über Sie zum Ausdruck gebracht, als die in
               meinem \textcolor{green}{Reigen-Feuilleton}{}\ledrightnote{{$\rightarrow$}\textcolor{green}{Arthur Schnitzler und sein »Reigen«}}
               niedergelegte. Das sei unaufrichtig, und habe Sie verletzt.\pend
           
\pstart
           Darauf ließe sich erwidern, dass ich jetzt sehr wol eine andere Meinung haben könn\substVorne{}\textsuperscript{e}\substDazwischen{}te\substHinten{}, ohne dass eine Unaufrichtigkeit mir deshalb vorzuwerfen wäre. Es kommt ja,
               wenn man seine alten, gewohnten Urtheile über einen Künstler nach Jahren wieder
               einmal versammelt \introOben{}vor\introOben{}, dass die eine oder die andere der
               früheren Meinungen Einem inzwischen davongelaufen ist, sich nicht mehr einstellen
               will, indessen andere, neue Anschauungen sich plötzlich einfinden. So entstünde dann
               in der Con\substVorne{}\textsuperscript{c}\substDazwischen{}z\substHinten{}entration kritischen Arbeitens ein verändertes Gesammtbild, und man dürfte
               deswegen von einer Unaufrichtigkeit noch nicht sprechen.\pend
           
\pstart
           Bei mir ist aber nicht einmal \uline{das} zutreffend. Was ich
               im »\textcolor{green}{R-F}{}\ledrightnote{{$\rightarrow$}\textcolor{green}{Arthur Schnitzler und sein »Reigen«}}« schrieb, habe ich
               seit Jahren \uline{gedacht}, und Ihnen mein Denken nicht
               vorenthalten. Sie \uline{müßen} sich erinnern, wie oft ich
               Ihnen sagte, dass der \textcolor{green}{Anatol}{}\ledrightnote{\textcolor{green}{Anatol}} jetzt anders auf
               mich wirke, als vor 12 Jahren, und Sie müßen sich erinnern, dass ich bei diesem
               Thema: \textcolor{green}{Anatol}{}\ledrightnote{\textcolor{green}{Anatol}}, \textcolor{green}{Märchen}{}\ledrightnote{\textcolor{green}{Das Märchen. Schauspiel in drei Aufzügen}} ec. einmal (es war in der \textcolor{pink}{Frankgaße}{}\ledrightnote{\textcolor{pink}{Frankgasse 1}}) so heftig im Ausdruck wurde, dass wir Beide darüber ins Lachen
               geriethen. Sie müßen sich ferner erinnern, dass ich Ihnen in unseren häufigen
               Gesprächen über die »\textcolor{green}{Beatrice}{}\ledrightnote{\textcolor{green}{Der Schleier der Beatrice. Schauspiel in fünf Akten}}« sagte, es müße
               nun etwas anderes kommen! Ich {\pb}rechnete, mit Ihrer Zustimmung, die \textcolor{green}{Beatrice}{}\ledrightnote{\textcolor{green}{Der Schleier der Beatrice. Schauspiel in fünf Akten}}
               als den Abschluß Ihrer \textcolor{green}{Anatol}{}\ledrightnote{\textcolor{green}{Anatol}}-Epoche, fand, dass
               auch der vorher geschriebene »\textcolor{green}{Grüne Kakadu}{}\ledrightnote{\textcolor{green}{Der grüne Kakadu. Groteske in einem Akt}}« ein
               erstes Anzeichen für die neue Entwicklung sei, besprach mit Ihnen die Rückfälligkeit
               der »\textcolor{green}{Gefährtin}{}\ledrightnote{\textcolor{green}{Die Gefährtin. Schauspiel in einem Akt}}« und dass nach meinem Gefühl der
                  »\textcolor{green}{Paracelsus}{}\ledrightnote{\textcolor{green}{Paracelsus. Versspiel in einem Akt}}« mißlungen sei.\pend
           
\pstart
           Am 16. Dez. 1900{ }\label{K_L03353-2v}\edtext{\textcolor{green}{schrieb ich}{}\ledrightnote{{$\rightarrow$}\textcolor{green}{»Der Schleier der Beatrice«. (Zum erstenmale aufgeführt im Lobe-Theater zu Breslau)}} dann in der »\textcolor{green}{W\textsuperscript{r} Allg. Ztg}{}\ledrightnote{\textcolor{green}{Wiener Allgemeine Zeitung}}« über
               die \textcolor{green}{Beatrice}{}\ledrightnote{\textcolor{green}{Der Schleier der Beatrice. Schauspiel in fünf Akten}}}{\lemma{\textnormal{\emph{schrieb … Beatrice}}}\Cendnote{\textnormal{\textcolor{blue}{Felix Salten}: \emph{\textcolor{green}{»Der Schleier der Beatrice«. (Zum erstenmale aufgeführt im
                        Lobe-Theater zu Breslau)}}. In: \emph{\textcolor{green}{Wiener
                        Allgemeine Zeitung. 6 Uhr-Blatt}}, Nr. 6.832, 16. 12. 1900, S. 10.}}}\label{K_L03353-2h}: »\textcolor{green}{Und demnach kann auch der ›\textcolor{green}{Schl. d. B.}{}\ledrightnote{\textcolor{green}{Der Schleier der Beatrice. Schauspiel in fünf Akten}}‹ nach der eingangs erwähnten Formel declinirt
                  werden: ›Schnitzler – Vorstadt – süßes Mädel‹. Der \uline{ganze Ideenkreis}, der \textcolor{green}{Anatol}{}\ledrightnote{{$\rightarrow$}\textcolor{green}{Anatol}} und seine Mädchen, der die \textcolor{green}{Christine}{}\ledrightnote{{$\rightarrow$}\textcolor{green}{Liebelei. Schauspiel in drei Akten}} der \textcolor{green}{Liebelei}{}\ledrightnote{\textcolor{green}{Liebelei. Schauspiel in drei Akten}}, der alle die \uline{kleinen und großen}
                  Dialoge, Novellen und Stücke Schnitzlers erfüllt, \uline{erfüllt auch dieses \textcolor{green}{Drama}{}\ledrightnote{{$\rightarrow$}\textcolor{green}{Der Schleier der Beatrice. Schauspiel in fünf Akten}}}. \textcolor{green}{Anatol}{}\ledrightnote{{$\rightarrow$}\textcolor{green}{Anatol}}\substVorne{}\textsuperscript{s}\substDazwischen{},\substHinten{} der ästhetisirende Liebhaber, bezaubert von der unbewußten Grazie eines
                  Vorstadtmädels, melancholisch durch Eifersucht auf Vergangenheit und Gegenwart,
                  nachdenklich über die Rätsel des Liebesverkehrs, und manchmal im \begin{otherlanguage}{french}chambre Separée\end{otherlanguage}{ }summarisch: ›So ist das Leben‹, – \textcolor{green}{Filippo Loschi}{}\ledrightnote{{$\rightarrow$}\textcolor{green}{Der Schleier der Beatrice. Schauspiel in fünf Akten}} trägt seine
                  Züge.}{}\ledrightnote{{$\rightarrow$}\textcolor{green}{»Der Schleier der Beatrice«. (Zum erstenmale aufgeführt im Lobe-Theater zu Breslau)}}« Und weiter: »\textcolor{green}{\uline{\textcolor{green}{Beatrice}{}\ledrightnote{{$\rightarrow$}\textcolor{green}{Der Schleier der Beatrice. Schauspiel in fünf Akten}}, das
                     Vorstadtmädel}, \uline{süß natürlich}, \uline{sehr süß}, hinreißend in ihrer inneren Naivetät,
                  berauschend in ihrer stets bereiten Weiblichkeit, \uline{und
                     sie geht den Weg der Vorstadtmädel}{\dotstwo}}{}\ledrightnote{{$\rightarrow$}\textcolor{green}{»Der Schleier der Beatrice«. (Zum erstenmale aufgeführt im Lobe-Theater zu Breslau)}}«\pend
           
\pstart
           Dieses \textcolor{green}{Feuilleton}{}\ledrightnote{{$\rightarrow$}\textcolor{green}{»Der Schleier der Beatrice«. (Zum erstenmale aufgeführt im Lobe-Theater zu Breslau)}} haben Sie
               damals in einem sicherlich übertriebenen Lob »ein Meisterwerk« genannt. Immer\substVorne{}\textsuperscript{\textcolor{gray}{g}}\substDazwischen{}h\substHinten{}in, ich durfte glauben, dass Sie mir Recht geben, durfte es umso mehr, als
               ich ja nur geschrieben hatte, was ich so oft mündlich zu Ihnen geäußert habe.\pend
           
\pstart
           Heute schreiben Sie mir, Sie müßten es »bei mir lesen, dass Ihnen erst mit der \textcolor{green}{Beatrice}{}\ledrightnote{\textcolor{green}{Der Schleier der Beatrice. Schauspiel in fünf Akten}} eine \uline{einigermaßen}{ }\uline{neue}{ }\uline{Verkleidung} der alten Figur gelungen ist!«\pend
           
\pstart
           Nein, lieber, \uline{das} haben Sie bei mir \uline{nicht} gelesen. Ich schrieb: »\textcolor{green}{Dem oft variierten süßen Mädel \uline{gab}{ }\uline{er}{ }\uline{in}{ }\uline{der}{ }\uline{\textcolor{green}{Beatrice}{}\ledrightnote{{$\rightarrow$}\textcolor{green}{Der Schleier der Beatrice. Schauspiel in fünf Akten}}}{ }\uuline{endgiltige} Gestalt; \uuline{rückte den von ihm geschaffenen Typus ins Erhabene}!!!}{}\ledrightnote{{$\rightarrow$}\textcolor{green}{Arthur Schnitzler und sein »Reigen«}}{[}«{]}\pend
           
\pstart
           Sie werden im Ernst nicht behaupten können, das heiß\substVorne{}\textsuperscript{t}\substDazwischen{}e\substHinten{} auf Deutsch: »Damit ist Ihnen \uline{eine einigermaßen
                  neue Verkleidung gelungen}!« Das heißt, was es sagt\substVorne{}\textsuperscript{,}\substDazwischen{}:\substHinten{} »\textcolor{green}{rückte den Typus ins
                  Erhabene, gab \uline{endgiltige Gestalt}}{}\ledrightnote{{$\rightarrow$}\textcolor{green}{Arthur Schnitzler und sein »Reigen«}}.« Ich bitte Sie den Unterschied zwischen dem, was Sie mir vorwerfen, was Sie
               aus meinen Zeilen herauslesen, und zwischen dem, was ich geschrieben habe, zu
               beachten.\pend
           
\pstart
           Das süße Mädel ist nun einmal ein Typus. Man bedient sich des Wortes in der
               Literatur, wie im Leben, zur kurzen Verständigung, um eine besti{\geminationm}te Gattung rasch zu bezeichnen. Es gibt garnicht viele
               Dichter, die einen Typus geschaffen, die eine neue Gestalt im Leben sichtbar gemacht
               und \strikeout{\textcolor{gray}{×}} die Literatur mit ihr bereichert haben. Muß ich das hier wirklich anführen, um
               zu erklären, dass es keinen Vorwurf bedeutet, Ihnen vom süßen Mädel zu sprechen\substVorne{}\textsuperscript{,}\substDazwischen{}?\substHinten{}{ }\textcolor{blue}{Bahr}{}\ledrightnote{\textcolor{blue}{Hermann Bahr}} hat \textcolor{green}{geschrieben}{}\ledrightnote{{$\rightarrow$}\textcolor{green}{Das junge Österreich. II}}: \label{K_L03353-3v}\edtext{\textcolor{green}{Schnitzler ist ein Virtuos}{}\ledrightnote{{$\rightarrow$}\textcolor{green}{Das junge Österreich. II}} –
               auf \uline{einer} Saite.}{\lemma{\textnormal{\emph{Schnitzler … Saite.}}}\Cendnote{\textnormal{Wörtlich lautet die Stelle: »\textcolor{green}{Er ist ein großer Virtuose,
                        aber einer kleinen Note.}« \textcolor{blue}{Hermann Bahr}: \emph{\textcolor{green}{Das junge Oesterreich. II}}. In: \emph{\textcolor{green}{Deutsche Zeitung}}, Jg. 23, Nr. 7.813, 27. 9. 1893, Morgenausgabe, S. 1–3. Siehe Bahr/Schnitzler, T030009. \textcolor{blue}{Schnitzler} hatte sich damals sehr wohl darüber geärgert,
                     vgl. A. S.: \emph{Tagebuch}, 27. 9. 1893.}}}\label{K_L03353-3h} Und
                  \label{K_L03353-4v}\edtext{\textcolor{blue}{\textcolor{green}{Herzl}{}\ledrightnote{{$\rightarrow$}\textcolor{green}{Feuilleton. Carl-Theater. (»Freiwild«, Schauspiel von Arthur Schnitzler.)}}}{}\ledrightnote{\textcolor{blue}{Theodor Herzl}}}{\lemma{\textnormal{\emph{Herzl}}}\Cendnote{\textnormal{»\textcolor{green}{Daß es noch größere Fragen
                        gebe, als ob die Mitzi mit dem Rudi vom Ferdl plötzlich verlassen worden
                        sei, scheint er in seinen Werken nicht zu wissen.}« \textcolor{blue}{H.} [ = \textcolor{blue}{Theodor Herzl}]: \emph{\textcolor{green}{Feuilleton.
                        Carl-Theater. (»Freiwild«, Schauspiel von Arthur Schnitzler.)}}. In: \emph{\textcolor{green}{Neue Freie Presse}}, Nr. 12.024, 13. 2. 1898, S. 1–2. \textcolor{blue}{Schnitzler} hatte sich auch über dieses \textcolor{green}{Feuilleton} geärgert, vgl. A. S.: \emph{Tagebuch}, 13. 2. 1898 und Paul Goldmann an Arthur Schnitzler, 7. 3. [1898].}}}\label{K_L03353-4h} und \label{K_L03353-5v}\edtext{\textcolor{blue}{\textcolor{green}{Goldmann}{}\ledrightnote{{$\rightarrow$}\textcolor{green}{Berliner Theater. (»Der Schleier der Beatrice« von Arthur Schnitzler.)}}}{}\ledrightnote{\textcolor{blue}{Paul Goldmann}}}{\lemma{\textnormal{\emph{Goldmann}}}\Cendnote{\textnormal{\textcolor{blue}{Goldmann}s \textcolor{green}{Kritik} an der \textcolor{pink}{Berlin}er Aufführung von \emph{\textcolor{green}{Der Schleier der
                     Beatrice}} endete damit, dass er das \emph{\textcolor{green}{Stück}} als »\textcolor{green}{verfehlt}« bezeichnete und über \textcolor{blue}{Schnitzler}s
                  Zukunft als großer Dichter schrieb: »\textcolor{green}{Und die Frage, ob es ihm
                        gelingen wird, das hohe Ziel zu erreichen, nach dem er mit so schönem
                        Bemühen strebt, hängt ab von der Frage, ob er die Kraft haben wird, aus der
                        kleinen und abgesonderten Welt, in der sein Schaffen sich bisher
                        hauptsächlich bewegt hat und in der die Stimmungen – die Stimmungen, die aus
                        den kleinen Gefühlen hervorgehen – eine allzu wichtige Rolle spielen, den
                        Weg zu finden ins große Leben hinein […].}« \textcolor{blue}{Paul Goldmann}: \emph{\textcolor{green}{Berliner Theater. (»Der Schleier der Beatrice« von Arthur
                        Schnitzler.)}}. In: \emph{\textcolor{green}{Neue Freie
                        Presse}}, Nr. 13.851, 19. 3. 1903,
                     Morgenblatt, S. 1–5. Siehe A. S.: \emph{Tagebuch}, 19. 3. 1903.}}}\label{K_L03353-5h} schrieben, Schnitzler kann nichts als das süße Mädel.
                  \uline{Nichts} davon steht in meinem \textcolor{green}{Feuilleton}{}\ledrightnote{{$\rightarrow$}\textcolor{green}{Arthur Schnitzler und sein »Reigen«}}, wie nichts davon in meinem
               Urtheil über Sie \substVorne{}\textsuperscript{\textcolor{gray}{ste}}\substDazwischen{}zu\substHinten{} finden ist, nicht im Geschriebenen und nicht im Mündlichen.\pend
           
\pstart
           Hätte ich geschrieben: Schnitzler kommt vom süßen Mädel nicht los, dann hätte ich
               mich der Einkastelung schuldig gemacht. Aber ich habe {\pb}geschrieben: »\textcolor{green}{{\dotsfive} gab endgiltige Gestalt, rückte den Typus ins Erhabene
                  und \uuline{entledigte sich}{\dotstwo}}{}\ledrightnote{{$\rightarrow$}\textcolor{green}{Arthur Schnitzler und sein »Reigen«}}« Erlauben Sie, dass ich auf diesen Unterschied aufmerksam mache. Ich schrieb:
                  »\textcolor{green}{In diesem \textcolor{green}{Werke}{}\ledrightnote{{$\rightarrow$}\textcolor{green}{Der Schleier der Beatrice. Schauspiel in fünf Akten}} nahm er \uline{Abschied von dem Vorstadtmotiv}}{}\ledrightnote{{$\rightarrow$}\textcolor{green}{Arthur Schnitzler und sein »Reigen«}}{[}«{]}!!!! Damit glaubte ich, das Kastel, in das andere Sie sperren
               möchten, zerschlagen zu haben, und glaube es noch immer.\pend
           
\pstart
           Es blieben noch die Worte: »\textcolor{green}{niedliche und langwierige Gefährtin der Dichterjugend.}{}\ledrightnote{{$\rightarrow$}\textcolor{green}{Arthur Schnitzler und sein »Reigen«}}« Nicht im
               Entferntesten fiel es mir ein, darin könne etwas Kränkendes für Sie liegen. Es ist in
               meiner Art, mich soweit als möglich in den anderen zu versetzen, wenn ich schreibe,
               und da mag ich über das süße Mädel ein ungeduldigeres Wort gesagt haben. Es thut mir
               leid. Sachlich war es nicht falsch, der anderen Frauengestalten dabei nicht zu
               gedenken. Diese spielen in Ihrem Schaffen bis zum \textcolor{green}{Reigen}{}\ledrightnote{\textcolor{green}{Reigen. Zehn Dialoge}} und zur \textcolor{green}{Beatrice}{}\ledrightnote{\textcolor{green}{Der Schleier der Beatrice. Schauspiel in fünf Akten}} keine so
               wichtige Rolle, dass man sie \substVorne{}\textsuperscript{auf}\substDazwischen{}in\substHinten{} einer geradlinigen und knappen Auseinandersetzung Ihres Entwicklungsganges
               hätte anbringen müßen.\pend
           
\pstart
           Es bliebe noch: Goldschmiedearbeit, Kleinkunst. Ich erkläre ausdrücklich, dass ich es
               bedaure, diese Worte angewendet zu haben, bedauere, weil sie eine von mir nicht
               geahnte und nicht beabsichtigte Wirkung auf Sie hervorbrachten. Trotzdem, ich kann
               sie verantworten. Der Absatz beginnt: »\textcolor{green}{Schnitzler \uuline{hatte} noch andere
                  Eigenschaften, ec.}{}\ledrightnote{{$\rightarrow$}\textcolor{green}{Arthur Schnitzler und sein »Reigen«}}« »\textcolor{green}{hatte}{}\ledrightnote{{$\rightarrow$}\textcolor{green}{Arthur Schnitzler und sein »Reigen«}}«. Darin liegt einfach Alles. Ich nenne Sie keinen Goldschmied, ich
               sage nicht, Sie \uline{sind} ein Kleinkünstler. Ich beziehe
               diese beiden Worte{[},{]} wie aus dem \textcolor{green}{F.}{}\ledrightnote{\textcolor{green}{Arthur Schnitzler und sein »Reigen«}} hervorgeht{[},{]}{ }\uuline{nur} auf Ihre \uline{Anfänge},
               nur auf den \textcolor{green}{Anatol}{}\ledrightnote{\textcolor{green}{Anatol}}\textcolor{gray}{,} als auf d\substVorne{}\textsuperscript{\textcolor{gray}{em}}\substDazwischen{}as\substHinten{} Werk, auf dem Ihr Ruhm wie auf \label{K_L03353-6v}\edtext{einem}{\lemma{\textnormal{\emph{einem}}}\Cendnote{\textnormal{in der Vorlage steht:
                     »einer«}}}\label{K_L03353-6h} Quader ruht. Diese Basis kann sich in späteren
               Zeiten durch Umwertung verschieben. Historisch wird man sie aber doch belaßen müßen.
               Und \uline{gleich}, nachdem die beiden ominösen Worte gesagt
               sind, kommt: »\textcolor{green}{Dann aber fand er die
                  Handgriffe zu einem \uline{stärkeren Material}, zu einer
                     \uline{höheren Plastik}!}{}\ledrightnote{{$\rightarrow$}\textcolor{green}{Arthur Schnitzler und sein »Reigen«}}« Heißt das, Sie zu einem
               Goldschmied stempeln? Dann kommt: »\textcolor{green}{\uline{Umfassendere Kräfte}{ }\uline{werden in ihm frei}, \uline{großzügiger und weniger zierlich}.}{}\ledrightnote{{$\rightarrow$}\textcolor{green}{Arthur Schnitzler und sein »Reigen«}}« Heisst das, Sie sind ein
               Kleinkünstler?\pend
           
\pstart
           Es bliebe noch: »\textcolor{green}{Er darf nicht
                  wiederkommen. So nicht!}{}\ledrightnote{{$\rightarrow$}\textcolor{green}{Arthur Schnitzler und sein »Reigen«}}« Lieber, das habe ich Ihnen oft gesagt, das ist meine
               Überzeugung, und es ist meine Überzeugung, dass Sie »ein neuer Rausch« umfangen wird.
               Sie umschreiben das \introOben{}leider\introOben{} mit den bitteren Worten, »\uline{dass ich noch Besseres \strikeout{\textcolor{gray}{×}} von Ihnen zu erwarten scheine}«. Besseres wol auch, aber was wichtiger
               ist: \uline{Anderes}! Zu diesem Anderen rechne ich die »\textcolor{green}{letzten Masken}{}\ledrightnote{\textcolor{green}{Die letzten Masken}}«. Rechne ich nicht die »\textcolor{green}{Literatur}{}\ledrightnote{\textcolor{green}{Literatur}}« und nur halb die \textcolor{green}{Frau mit dem Dolch}{}\ledrightnote{\textcolor{green}{Die Frau mit dem Dolche}}, deren geniale Erfindung mich so sehr in
               meinem Glauben an Ihre Wandlung bestärkte, dass meine Abneigung gegen \textcolor{blue}{Schwarzkopf}{}\ledrightnote{\textcolor{blue}{Gustav Schwarzkopf}} akut wurde, als er von einem »Tric«
               sprach. Ich zweifle nicht, dass dieser ehrliche \textcolor{blue}{Mann}{}\ledrightnote{{$\rightarrow$}\textcolor{blue}{Gustav Schwarzkopf}}{[},{]} wenn er die Gelegenheit gehabt hätte, auch geschrieben hätte,
               es sei »ein Tric«. Und ich zweifle nicht, dass Sie das geschriebene ebenso ruhig
               angenommen hätten \substVorne{}\textsuperscript{\textcolor{gray}{×}\-\textcolor{gray}{×}}\substDazwischen{}wie\substHinten{} Sie das gesprochene Strohwort hingenommen haben. Gegen mich aber regen sich
               bei Ihnen so heftige Stimmen des Misstrauens, weil ich auf einem höheren Niveau und
               mit größeren Maßstäben \strikeout{\textcolor{gray}{von Ihrem}{ }\textcolor{gray}{×}\-\textcolor{gray}{×}\-\textcolor{gray}{×}\-\textcolor{gray}{×}{ }\textcolor{gray}{×}\-\textcolor{gray}{×}\-\textcolor{gray}{×}\-\textcolor{gray}{×}\-\textcolor{gray}{×},} die Linie Ihres
               Schaffens ziehe.\pend
           
\pstart
           Ich sinne vergebens darüber nach, wie es mir passiren konnte, von Ihnen \uline{so arg} mißverstanden zu werden. Und da ich mich zu der
               Annahme, dass Sie mir irgendwie gereizt und beeinflußt, oder mißtrauisch
               gegenüberstehen nicht entschließen kann, komme ich immer wieder zu dem Resultat: es
               muß an meinem \textcolor{green}{Feuilleton}{}\ledrightnote{{$\rightarrow$}\textcolor{green}{Arthur Schnitzler und sein »Reigen«}}
               irgendwie und irgendwo {\pb}ein
               Fehler stecken.\pend
           
\pstart
           \uline{Nur} deshalb möchte ich Ihnen noch zu bedenken geben,
               was Sie offenbar ganz übersehen haben. Dieses \textcolor{green}{Reigen-Feuilleton}{}\ledrightnote{{$\rightarrow$}\textcolor{green}{Arthur Schnitzler und sein »Reigen«}} erschien in der Absicht, Ihnen und Ihrem
                  \textcolor{green}{Buch}{}\ledrightnote{{$\rightarrow$}\textcolor{green}{Reigen. Zehn Dialoge}} zu Hilfe zu kommen. Es
               erschien in der Verbotswoche, und unter dem Widerstand \uline{aller}{ }\label{K_L03353-7v}\edtext{Faktoren}{\lemma{\textnormal{\emph{Faktoren}}}\Cendnote{\textnormal{Geschäftsführer}}}\label{K_L03353-7h}. Erinnern Sie sich, dass Ihr \label{K_L03353-8v}\edtext{eigener \textcolor{blue}{Schwager}{}\ledrightnote{{$\rightarrow$}\textcolor{blue}{Markus Hajek}}}{\lemma{\textnormal{\emph{eigener Schwager}}}\Cendnote{\textnormal{siehe A. S.: \emph{Tagebuch}, 5. 4. 1903}}}\label{K_L03353-8h} erklärt hat, (was er heute wieder beim
                  Prof. \textcolor{blue}{Singer}{}\ledrightnote{\textcolor{blue}{Isidor Singer}} that) »über so eine
               Schweinerei« schreibt man nicht. Diese Worte waren die Parole in allen \textcolor{pink}{Wien}{}\ledrightnote{\textcolor{pink}{Wien}}er Redactionen\textcolor{gray}{,} und Niemand
               konnte dagegen an, diese Worte wurden ins breiteste Publicum getragen und es sollte
               überall heißen, der \textcolor{green}{Reigen}{}\ledrightnote{\textcolor{green}{Reigen. Zehn Dialoge}} ist kein Kunstwerk
               sondern eine Pornographie. Da ist es mir eine Freude gewesen, dass ich das
               Selbstverständliche und ganz Unverdienstliche aussprechen durfte: der \textcolor{green}{R.}{}\ledrightnote{\textcolor{green}{Reigen. Zehn Dialoge}} ist ein Kunstwerk! Dass ich durch die
               Nebeneinanderstellung mit dem \textcolor{green}{Anatol}{}\ledrightnote{\textcolor{green}{Anatol}} zeigen
               konnte, warum er es ist. Hätte ich, wie ich ohne Mühe und wie ich es lieber gethan
               haben würde, meine Pfeifen höher gestimmt, dann würde ich Niemanden überzeugt haben,
               und ich hätte dem \textcolor{green}{Buch}{}\ledrightnote{{$\rightarrow$}\textcolor{green}{Reigen. Zehn Dialoge}}{ }\uline{nur} geschadet, weil alle Leute gesagt hätten:
               »Natü-ürlich, der Salten!« So aber habe ich, \uline{das weiß ich
                  genau}, aufklärend und nützlich gewirkt! Woran mir sonst \uline{nie} etwas liegt, woran ich sonst \uline{nie}
               denke, diesmal lag mir daran, die Leute zu überzeugen, au\substVorne{}\textsuperscript{ch}\substDazwischen{}f\substHinten{} die Fernerstehenden zu wirken, die Gegner so viel als möglich zu entwaffnen.
                  \uline{Das} hat meinem \textcolor{green}{F.}{}\ledrightnote{{$\rightarrow$}\textcolor{green}{Arthur Schnitzler und sein »Reigen«}} vielleicht bei Ihnen geschadet. Aber die \uline{allerbeste Absicht} müßten Sie mir doch zubilligen.\pend
           
\pstart
           Aus \uline{taktischen} Gründen stehen die Schlußworte da:
                  »\textcolor{green}{wir sind neugierig auf den
                  neuen Schn.}{}\ledrightnote{{$\rightarrow$}\textcolor{green}{Arthur Schnitzler und sein »Reigen«}}« Ich habe mir damit vorsichtsweise eine Stufe gebaut, auf die ich
               steigen und den neuen Schnitzler von da aus demnächst zeigen wollte. Es sind diese
               Worte ein Riesenthor, das ich vor Ihnen aufmache; da kann einfach alles kommen, da
               erwartet man alles. Die Entwicklungsfähigkeit, die Wandlungsmöglichkeit, die heute
               noch nicht zu begrenzende Complexität, (lauter Dinge, die Ihnen oft, und oft von
               nahestehenden Freunden geleugnet wurden) werden Ihnen hier als etwas
               Selbstverständliches zugesprochen; – und – Sie schreiben
                  \textcolor{gray}{m}i\textcolor{gray}{r}, ich hätte Sie in ein Kastel
               gesperrt!\pend
           
\pstart
           Ich frage mich, sehr betroffen, wie ich Ihnen gestehen will, ob denn die zwölf Jahre
               intimer Gemeinschaft nicht bei Ihnen standen, als \label{K_L03353-9v}\edtext{Sie}{\lemma{\textnormal{\emph{Sie}}}\Cendnote{\textnormal{in der
                  Vorlage steht: »sie«}}}\label{K_L03353-9h} diese Zeilen lasen, und ob sie so
               schwach waren, \strikeout{Ihnen} dass \label{K_L03353-10v}\edtext{sie}{\lemma{\textnormal{\emph{sie}}}\Cendnote{\textnormal{in der
                  Vorlage steht: »Sie«}}}\label{K_L03353-10h} Ihnen nicht helfen konnten, de\substVorne{}\textsuperscript{m}\substDazwischen{}n\substHinten{} Sinn dieser Worte zu entziffern, die wahre Meinung, den wahren Sinn, wenn
               schon die Worte allein nicht deutlich genug gewesen sind. Ich frage mich weiter, ob
               diese zwölf Jahre, in denen ich eine Theilnahme für Ihre Schriften gezeigt habe, die
               in ihrer Intensität, in ihrer Aktivität, in ihrer Beständigkeit wie in ihrem
               Verständnis gewiss keine alltägliche gewesen ist, ob diese Jahre so kraftlos sind,
               dass sie beschämt Ihre Vorwürfe hören mußten, ohne sie aus eigenem Vorrath widerlegen
               zu können.\pend
           
\pstart
           Sie werden auch meine Deprimirtheit darüber begreifen, dass ein \textcolor{green}{Feuilleton}{}\ledrightnote{{$\rightarrow$}\textcolor{green}{Arthur Schnitzler und sein »Reigen«}}, in welchem mit dem Absatz »\textcolor{green}{Dass Einer aber lachen kann}{}\ledrightnote{{$\rightarrow$}\textcolor{green}{Arthur Schnitzler und sein »Reigen«}}«, –
               bis zu »\textcolor{green}{der Humor allein ist am
                  Ziel, er ist die Nähe, ist der Gipfel, er ist das En\substVorne{}\textsuperscript{gi}\substDazwischen{}dg\substHinten{}iltige!}{}\ledrightnote{{$\rightarrow$}\textcolor{green}{Arthur Schnitzler und sein »Reigen«}}« so ein Ton absoluter und höchster Anerkennung angeschlagen
               wird, so vollständig umgedeutet werden kann.\pend
           
\pstart
           Neben vielen Anderen Dingen thut es mir am meisten leid, dass Sie, wie es
                  scheint{[},{]} durch mein \textcolor{green}{F.}{}\ledrightnote{{$\rightarrow$}\textcolor{green}{Arthur Schnitzler und sein »Reigen«}} zu starkem Selbstzweifel veranlaßt wurden. Da muß ich
               Ihnen aber doch sagen, dass Sie \uline{dazu} nicht den
               mindesten Anlaß haben, dass ich nicht blos »Besseres von Ihnen zu erwarten scheine«
               sondern daß sich nahezu alle meine Urtheile, die Ihre künstlerische Kraft betreffen,
               in den letzten Jahren nur gefestigt haben! Und ich muß {\pb}doch einmal noch Sie darauf
               aufmerksam machen, dass in meinem \textcolor{green}{Feuilleton}{}\ledrightnote{{$\rightarrow$}\textcolor{green}{Arthur Schnitzler und sein »Reigen«}} überall, wo etwa von Ihren Grenzen die Rede
                  ist{[},{]} ein »\uline{hatte}«, ein »\uline{war}«, kurz ein \uline{Perfectum} steht. Und dass überall, wo von der Gegenwart gesprochen wird, das
               Wort Vo\textcolor{gray}{rn}, Reife, Entwicklung, das Geringste ist, was gesagt wird,
               und dass die Thatkraft als eine hoffnungsreiche bezeichnet wird. Das ist die Linie,
               die ich einhalten wollte, und die ich, wie es scheint, doch nicht straff genug
               gezogen habe.\pend
           
\pstart
           Noch nie habe ich eine kritische Arbeit so gerne geschrieben, und noch nie ist mir
               mein kritisches Amt, das ich ja nicht aus innerster Neigung auf mich genommen habe,
               das ich aber doch immer mit Gewissenhaftigkeit und gutem Willen versehe, so verleidet
               und zum Überdruß gewesen, wie jetzt, seit ich Ihren Brief empfing.\pend
           
\pstart
           Ich weiß nach dem Vorgefallenen nicht, ob ich Sie durch diesen langen Brief auch nur
               in einem Punct überzeugt habe. Ich weiß ja jetzt auch garnichts mehr, und ich
               überlege mir, ob es einen Werth für Sie haben kann, wenn ich jetzt noch Ihrer \label{K_L03353-11v}\edtext{Vorlesung}{\lemma{\textnormal{\emph{Vorlesung}}}\Cendnote{\textnormal{siehe A. S.: \emph{Tagebuch}, 12. 11. 1903}}}\label{K_L03353-11h} beiwohne. Nicht als ob mein Urtheil über Sie befangen oder schwankend gemacht
               werden könnte, aber wie ich Ihnen nun meine Meinung formuliren soll, und wie Sie sie
               aufnehmen, dessen bin ich jetzt nicht mehr sicher, und glaube, wir wollen es diesmal
               lieber unterlaßen.\pend
           \pstart Ihr \spacefill\mbox{F S.}\pend{}\endnumbering\briefempfaengerindex{Schnitzler, Arthur@\textsc{Schnitzler, Arthur}!zzzSalten, Felix@\emph{von Felix Salten}!1903-11-092@{{[}9. 11. 1903{]}}|)be}\mylabel{h}  \normalsize

\doendnotes{C}
\bigskip
\vfill

\clearpage

\footnotesize

\lohead{\textsc{register}}

% Definiere theindex-Environment komplett neu ohne reledmac
\makeatletter
\renewenvironment{theindex}{%
  \section*{\indexname}%
  \setlength{\parindent}{0pt}%
  \setlength{\parskip}{0pt plus 0.3pt}%
  \let\item\@idxitem
}{%
  \clearpage
}
\makeatother

\IfFileExists{\jobname-pw.ind}{\input{\jobname-pw.ind}}{}

\end{document}

      