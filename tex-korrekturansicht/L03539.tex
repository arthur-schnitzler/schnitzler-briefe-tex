%% latex-korrekturansicht-vorspann.tex
%% Vorspann für die Korrekturansicht.
%% Lädt die gemeinsame Datei latex-vorspann.tex mit gesetztem Schalter.

\newif\ifkorrekturansicht
\korrekturansichttrue

\input{../tex-inputs/latex-vorspann}


\renewcommand{\erwaehntePersonen}{Personen: Paul Goldmann}
\renewcommand{\erwaehnteOrte}{Orte: Berlin, Wien}
\renewcommand{\erwaehnteWerke}{Werke: Fräulein Else}
\section[Franziska Goldmann an Arthur Schnitzler, {[}Ende Oktober 1925?{]}]{Franziska Goldmann an Arthur Schnitzler, {[}Ende Oktober 1925?{]}}
\nopagebreak\mylabel{v}
\rehead{ }\normalsize\beginnumbering\briefempfaengerindex{Schnitzler, Arthur@\textsc{Schnitzler, Arthur}!zzzGoldmann, Franziska@\emph{von Franziska Goldmann}!1925-10-311@{{[}Ende Oktober 1925?{]}}|(be}
\toendnotes[C]{\smallbreak\pagebreak[2]}\Standort{DLA, A:Schnitzler, HS.NZ85.1.3161.}
\physDesc{Brief, 1 Blatt, 2 Seiten, 513 Zeichen
\newline{}Handschrift: schwarze Tinte, lateinische Kurrent
\newline{}Schnitzler: 1) mit Bleistift Vermerk »\textcolor{blue}{Franz
                                          Gold\textcolor{gray}{ma}{[}nn{]}}«  2) mit rotem Buntstift drei Unterstreichungen}\toendnotes[C]{\smallbreak}
\pstart\center{}{\pb}Sehr geehrter Herr Dr.\pend
\pstart
           Bitte entschuldigen Sie, daß ich Ihnen erst jetzt für die Mühe danke, die Sie sich
               machten, indem Sie mir Ihr reizendes \label{K_L03539-1v}\edtext{\textcolor{green}{Buch}{}\ledrightnote{{$\rightarrow$}\textcolor{green}{Fräulein Else}}}{\lemma{\textnormal{\emph{Buch}}}\Cendnote{\textnormal{In \textcolor{blue}{Goldmann}s Brief vom 24. 10. 1925 ist zu lesen: »\textcolor{blue}{Franzi} iſt bereits in ›\emph{\textcolor{green}{Fräulein
                  Elſe}}‹ vertieft u. erklärt, es ſei das Schönſte, das ſie je geleſen habe, –
                  dankt Dir auch für die eigenhändige Widmung, mit der ſie in ihrer Klaſſe großen
                  Eindruck zu machen hofft.« Aufgrund der Ähnlichkeit der Schilderungen ist davon
                  auszugehen, dass der Brief von \textcolor{blue}{Franziska
                     Goldmann} ungefähr zur selben Zeit, Ende Oktober 1925, entstand.}}}\label{K_L03539-1h} schickten. Ich war aber sehr neugierig
               darauf und wollte es zuerst auslesen. Es hat mir \substVorne{}\textsuperscript{\textcolor{gray}{f}}\substDazwischen{}v\substHinten{}on Anfang bis Ende den größten Spaß gemacht, besonders der Schluß, den ich
               sehr aufregend und tragisch finde\substVorne{}\textsuperscript{.}\substDazwischen{},\substHinten{} und ist eines der schönsten Bücher, die ich gelesen habe. Über die Widmung
               sind meine \label{T_L03539-1v}\edtext{s{[}ä{]}mtlichen}{\lemma{\textnormal{\emph{sämtlichen}}}\Cendnote{\textnormal{korrigiert aus »samtlichen«}}}\label{T_L03539-1h} Freunde zersprungen.\pend
           
\pstart
           {\pb}Mit nochmals vielem herzlichen Dank {\\[\baselineskip]}Ihre {\\[\baselineskip]}\spacefill\mbox{Franzi Goldmann}\pend
           \leftskip=0em{}\endnumbering\briefempfaengerindex{Schnitzler, Arthur@\textsc{Schnitzler, Arthur}!zzzGoldmann, Franziska@\emph{von Franziska Goldmann}!1925-10-311@{{[}Ende Oktober 1925?{]}}|)be}\mylabel{h}  \normalsize

\doendnotes{C}
\bigskip
\vfill

\clearpage

\footnotesize

\lohead{\textsc{register}}

% Definiere theindex-Environment komplett neu ohne reledmac
\makeatletter
\renewenvironment{theindex}{%
  \section*{\indexname}%
  \setlength{\parindent}{0pt}%
  \setlength{\parskip}{0pt plus 0.3pt}%
  \let\item\@idxitem
}{%
  \clearpage
}
\makeatother

\IfFileExists{\jobname-pw.ind}{\input{\jobname-pw.ind}}{}

\end{document}

      