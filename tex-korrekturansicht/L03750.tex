%% latex-korrekturansicht-vorspann.tex
%% Vorspann für die Korrekturansicht.
%% Lädt die gemeinsame Datei latex-vorspann.tex mit gesetztem Schalter.

\newif\ifkorrekturansicht
\korrekturansichttrue

\input{../tex-inputs/latex-vorspann}


\section[Arthur Schnitzler an Stefan Zweig, 27. 7. 1923]{L03750 Arthur Schnitzler an Stefan Zweig, 27. 7. 1923}
\nopagebreak\mylabel{L03750v}
\rehead{ }\normalsize\beginnumbering\briefempfaengerindex{, @\textsc{, }!zzz, @\emph{von  }!1923-07-271@{27. 7. 1923}|(be}
\toendnotes[C]{\smallbreak\pagebreak[2]}\Standort{Jerusalem, National Library of Israel, ARC. Ms. Var. 305 1 58 Stefan Zweig Collection.}
\physDesc{Postkarte, 824 Zeichen
\newline{}Handschrift: schwarze Tinte, lateinische Kurrent
\newline{}Versand: Stempel: »\nobreak{}\oindex{XVIII., Währing@\textbf{XVIII., Währing}, \emph{Verwaltungsgebiet}|pwk}18/\textsubscript{1} Wien
                                       110, 27. VII. 23, 15\nobreak{}«.  
\newline{}Zweig: mit Bleistift datiert: »27/VII 1923« }\toendnotes[C]{\smallbreak}\pstart{}{\pb}\label{T_L03750-1v}\edtext{\textcolor{gray}{\textbf{A. S.}}}{\lemma{\textnormal{\emph{A. S.}}}\Cendnote{\textnormal{ovaler Absenderkleber}}}\label{T_L03750-1}\pend{}\pstart{}\textcolor{pink}{\textcolor{gray}{\textbf{WIEN, XVIII.}}}\oindex{XVIII., Währing@\textbf{XVIII., Währing}, \emph{Verwaltungsgebiet}|pw}{}\ledrightnote{\textcolor{pink}{XVIII., Währing}}\pend{}\pstart{}\textcolor{pink}{\textcolor{gray}{\textbf{STERNWARTESTR. 71}}}\oindex{Wien@\textbf{Wien}!XVIII., Währing@\textbf{XVIII., Währing}!Sternwartestraße 71@\textbf{Sternwartestraße 71}, \emph{Wohngebäude}|pw}{}\ledrightnote{\textcolor{pink}{Sternwartestraße 71}}\pend{}{\bigskip}\pstart{}An\pend{}\pstart{}Hrn Dr. Stefan Zweig\pend{}\pstart{}\textcolor{pink}{Salzburg}\oindex{Salzburg@\textbf{Salzburg}, \emph{Verwaltungsgebiet}|pw}{}\ledrightnote{\textcolor{pink}{Salzburg}}\pend{}\pstart{}\textcolor{pink}{Kapuzinerberg 5}\oindex{Paschinger Schlössl@\textbf{Paschinger Schlössl}, \emph{Wohngebäude}|pw}{}\ledrightnote{\textcolor{pink}{Paschinger Schlössl}}\pend{}{\bigskip}\vspace{1em}
\pstart
           \raggedleft{}{\pb}27. 7. 923.
               \pend
           \vspace{0.5em}
\pstart
           lieber Herr Doctor, vielen vielen Dank! Wie Sie sehen bin ich noch
               (war wieder) in \textcolor{pink}{Wien}\oindex{Wien@\textbf{Wien}, \emph{Verwaltungsgebiet}|pw}{}\ledrightnote{\textcolor{pink}{Wien}}, fahre voraussichtlich Ende
               nächster Woche nach \textcolor{pink}{Deutschland}\oindex{Deutschland@\textbf{Deutschland}|pw}{}\ledrightnote{\textcolor{pink}{Deutschland}} (\textcolor{pink}{Schwarzwald}\oindex{Schwarzwald@\textbf{Schwarzwald}, \emph{Gebirge}|pw}{}\ledrightnote{\textcolor{pink}{Schwarzwald}}, \label{K_L03750-1v}\edtext{\textcolor{pink}{Baden Baden}\oindex{Baden-Baden@\textbf{Baden-Baden}|pw}{}\ledrightnote{\textcolor{pink}{Baden-Baden}}}{\lemma{\textnormal{\emph{Baden Baden}}}\Cendnote{\textnormal{\textcolor{blue}{Schnitzler} reiste am 3. 8. 1923 von \textcolor{pink}{Wien}\oindex{Wien@\textbf{Wien}, \emph{Verwaltungsgebiet}|pwk} nach \textcolor{pink}{Salzburg}\oindex{Salzburg@\textbf{Salzburg}, \emph{Verwaltungsgebiet}|pwk}, wo er im \textcolor{pink}{Österreichischen
                     Hof}\oindex{Österreichischer Hof@\textbf{Österreichischer Hof}, \emph{Hotel}|pwk} abstieg und zwei Nächte blieb. Über \textcolor{pink}{Stuttgart}\oindex{Stuttgart@\textbf{Stuttgart}|pwk} (eine Übernachtung) reiste er dann nach \textcolor{pink}{Baden-Baden}\oindex{Baden-Baden@\textbf{Baden-Baden}|pwk}. Hier blieb er bis zum 14. 8. 1923, danach
                  folgte die Reise in die \textcolor{pink}{Schweiz}\oindex{Schweiz@\textbf{Schweiz}|pwk}, vor allem
                  nach \textcolor{pink}{Celerina}\oindex{Celerina@\textbf{Celerina}|pwk}. Am 7. 9. 1923 reiste er
                  nach \textcolor{pink}{Vorarlberg}\oindex{Vorarlberg@\textbf{Vorarlberg}|pwk}. Am 15. 9. 1923 kam er wieder
                  in \textcolor{pink}{Wien}\oindex{Wien@\textbf{Wien}, \emph{Verwaltungsgebiet}|pwk} an.}}}\label{K_L03750-1}, wo meine \textcolor{blue}{Kinder}\pwindex{Cappellini, Lili 13.\,9.\,1909 Wien – 26.\,7.\,1928 Venedig@\textsc{Cappellini, Lili} (13.\,9.\,1909 Wien – 26.\,7.\,1928 Venedig)|pwv}\pwindex{Schnitzler, Heinrich 9.\,8.\,1902 Hinterbrühl – 12.\,7.\,1982 Wien@\textsc{Schnitzler, Heinrich} (9.\,8.\,1902 Hinterbrühl – 12.\,7.\,1982 Wien), \emph{Regisseur, Schauspieler}|pwv}{}\ledrightnote{{$\rightarrow$}\emph{\textcolor{blue}{Lili Cappellini}}{\newline}{$\rightarrow$}\emph{\textcolor{blue}{Heinrich Schnitzler}}} bei ihrer \textcolor{blue}{Mutter}\pwindex{Schnitzler, Olga 17.\,1.\,1882 Wien – 13.\,1.\,1970 Lugano@\textsc{Schnitzler, Olga} (17.\,1.\,1882 Wien – 13.\,1.\,1970 Lugano), \emph{Schauspielerin, Sängerin}|pwv}{}\ledrightnote{{$\rightarrow$}\emph{\textcolor{blue}{Olga Schnitzler}}} sind) und in die \textcolor{pink}{Schweiz}\oindex{Schweiz@\textbf{Schweiz}|pw}{}\ledrightnote{\textcolor{pink}{Schweiz}}. Um \textcolor{blue}{R. R.}\pwindex{Rolland, Romain 29.\,1.\,1866 Clamecy – 30.\,12.\,1944 Vézelay@\textsc{Rolland, Romain} (29.\,1.\,1866 Clamecy – 30.\,12.\,1944 Vézelay), \emph{Schriftsteller}|pw}{}\ledrightnote{\textcolor{blue}{Romain Rolland}}
               kennen zu lernen und Sie wiederzusehen, werd ich, we{\geminationn} nicht unvorhergesehene
               Hindernisse obwalten – mich gern auf der Durchreise in \textcolor{pink}{Salzburg}\oindex{Salzburg@\textbf{Salzburg}, \emph{Verwaltungsgebiet}|pw}{}\ledrightnote{\textcolor{pink}{Salzburg}} aufhalten – ich denke, das wäre dann 3., ev.
               4. od 5. August. Wohnen werd ich im \textcolor{pink}{oesterr. Hof.}\oindex{Österreichischer Hof@\textbf{Österreichischer Hof}, \emph{Hotel}|pw}{}\ledrightnote{\textcolor{pink}{Österreichischer Hof}} – und Sie in {\pb}jedem Fall vorher
               verständigen. (Oder raten Sie mir ein andres Hotel? Ist \textcolor{pink}{Europe}\oindex{Grand Hotel de L’Europe, G. Jung@\textbf{Grand Hotel de L’Europe, G. Jung}, \emph{Hotel}|pw}{}\ledrightnote{\textcolor{pink}{Grand Hotel de L’Europe, G. Jung}} erschwinglich – was bei kurzem Aufenthalt durch die \textcolor{pink}{Bahnhofnähe}\oindex{Hauptbahnhof Salzburg@\textbf{Hauptbahnhof Salzburg}, \emph{Bahnhofsgebäude}|pwv}{}\ledrightnote{{$\rightarrow$}\emph{\textcolor{pink}{Hauptbahnhof Salzburg}}} verlockend
               wäre!).\pend
           
\pstart
           Empfehlen Sie mich Ihrer verehrten \textcolor{blue}{Gattin}\pwindex{Zweig, Friderike Maria 4.\,12.\,1882 Wien – 18.\,1.\,1971 Stamford@\textsc{Zweig, Friderike Maria} (4.\,12.\,1882 Wien – 18.\,1.\,1971 Stamford), \emph{Schriftstellerin}|pwv}{}\ledrightnote{{$\rightarrow$}\emph{\textcolor{blue}{Friderike Maria Zweig}}} und seien Sie sehr herzlich gegrüßt, und nochmals allerwärmstens
               bedankt von Ihrem\pend
           \pstart \spacefill\mbox{Arthur Schnitzler}\pend{}\selectlanguage{ngerman}\endnumbering\briefempfaengerindex{, @\textsc{, }!zzz, @\emph{von  }!1923-07-271@{27. 7. 1923}|)be}\mylabel{L03750h}  \normalsize

\doendnotes{C}
\bigskip
\vfill

\clearpage

\footnotesize

\lohead{\textsc{register}}

% Definiere theindex-Environment komplett neu ohne reledmac
\makeatletter
\renewenvironment{theindex}{%
  \section*{\indexname}%
  \setlength{\parindent}{0pt}%
  \setlength{\parskip}{0pt plus 0.3pt}%
  \let\item\@idxitem
}{%
  \clearpage
}
\makeatother

\IfFileExists{\jobname-pw.ind}{\input{\jobname-pw.ind}}{}

\end{document}

      