%% latex-korrekturansicht-vorspann.tex
%% Vorspann für die Korrekturansicht.
%% Lädt die gemeinsame Datei latex-vorspann.tex mit gesetztem Schalter.

\newif\ifkorrekturansicht
\korrekturansichttrue

\input{../tex-inputs/latex-vorspann}


\section[Theodor Herzl an Arthur Schnitzler, 15. 6. 1893]{L03830 Theodor Herzl an Arthur Schnitzler, 15. 6. 1893}
\nopagebreak\mylabel{L03830v}
\rehead{ }\normalsize\beginnumbering\briefempfaengerindex{Schnitzler, Arthur@\textsc{Schnitzler, Arthur}!zzzHerzl, Theodor@\emph{von Theodor Herzl}!1893-06-151@{15. 6. 1893}|(be}
\toendnotes[C]{\smallbreak\pagebreak[2]}\Standort{CUL, Schnitzler, B 39.}
\physDesc{Brief, 3 Blätter, 9 Seiten, 6518 Zeichen
\newline{}Handschrift: schwarze Tinte, lateinische Kurrent
\newline{}Ordnung: 1) mit Bleistift von unbekannter Hand nummeriert: »10«  2) mit Bleistift mutmaßlich von \textcolor{blue}{Leon
                                    Kellner}\pwindex{Kellner, Leon 1859-04-17 – 1928-12-05@\textsc{Kellner, Leon} (1859-04-17 – 1928-12-05), \emph{Zionist, Literaturhistoriker, Anglist}|pw} Markierung interessanter Stellen 3) mit blauem Buntstift von \textcolor{blue}{Leon Kellner}\pwindex{Kellner, Leon 1859-04-17 – 1928-12-05@\textsc{Kellner, Leon} (1859-04-17 – 1928-12-05), \emph{Zionist, Literaturhistoriker, Anglist}|pw}
                                 Markierung von Stellen für die Publikation}\toendnotes[C]{\smallbreak}
\pstart
           {\pb}\textcolor{brown}{\textcolor{gray}{\textbf{NOUVELLE PRESSE LIBRE}}}\orgindex{Neue Freie Presse@Neue Freie Presse|pw}{}\ledrightnote{\textcolor{brown}{Neue Freie Presse}}\hfill \textcolor{pink}{\textcolor{gray}{\textbf{8, Rue de Monceau}}}\oindex{8, Rue de Monceau@\textbf{8, Rue de Monceau}, \emph{Wohngebäude (K.WHS)}|pw}{}\ledrightnote{\textcolor{pink}{8, Rue de Monceau}}\pend
           
\pstart
           \textcolor{gray}{\textbf{D\textsuperscript{r}{ }TH. HERZL}}\hfill 15. Juni 893\pend
           
\pstart{}Lieber Freund!\pend\vspace{0.5em}
\pstart
           Mein drittes liebes \textcolor{blue}{Kind}\pwindex{Neumann, Margarethe 20.05.1893 – 15.03.1943@\textsc{Neumann, Margarethe} (20.05.1893 – 15.03.1943)|pwv}{}\ledrightnote{{$\rightarrow$}\emph{\textcolor{blue}{Margarethe Neumann}}},
               ein Mäderl u. es heisst \textcolor{blue}{Greterl}\pwindex{Neumann, Margarethe 20.05.1893 – 15.03.1943@\textsc{Neumann, Margarethe} (20.05.1893 – 15.03.1943)|pw}{}\ledrightnote{\textcolor{blue}{Margarethe Neumann}}, ist schon
               bald \label{K_L03830-1v}\edtext{vier Wochen alt}{\lemma{\textnormal{\emph{vier Wochen alt}}}\Cendnote{\textnormal{Am 20. 5. 1893 war die dritte
                  Tochter \textcolor{blue}{Margarethe}\pwindex{Neumann, Margarethe 20.05.1893 – 15.03.1943@\textsc{Neumann, Margarethe} (20.05.1893 – 15.03.1943)|pwk}, später genannt Trude,
                  zur Welt gekommen.}}}\label{K_L03830-1}. Meine \textcolor{blue}{Frau}\pwindex{Herzl, Julie 01.02.1868 – 10.11.1907@\textsc{Herzl, Julie} (01.02.1868 – 10.11.1907)|pwv}{}\ledrightnote{{$\rightarrow$}\emph{\textcolor{blue}{Julie Herzl}}} hat sich fast gänzlich erholt, u. wir denken an die
               Reise wenn nur nichts dazwischen kommt. Mein ältestes \textcolor{blue}{Mäderl}\pwindex{Hueft, Pauline 1890-03-29 – 1930-09-08@\textsc{Hüft, Pauline} (1890-03-29 – 1930-09-08)|pwv}{}\ledrightnote{{$\rightarrow$}\emph{\textcolor{blue}{Pauline Hüft}}} ist seit gestern
               krank. Der \textcolor{blue}{Arzt}\pwindex{?? [Arzt der Familie Herzl in Paris] @\textsc{?? [Arzt der Familie Herzl in Paris]}|pwv}{}\ledrightnote{{$\rightarrow$}\emph{\textcolor{blue}{?? [Arzt der Familie Herzl in Paris]}}} sagte uns
                  gestern Abends dass es vielleicht Masern werden. Heute
               meint er, dass es nicht dazu kommen würde. Aber sie fiebert noch stark.\pend
           
\pstart
           Jedenfalls war die Nacht für uns schlaflos. Meine arme \textcolor{blue}{Frau}\pwindex{Herzl, Julie 01.02.1868 – 10.11.1907@\textsc{Herzl, Julie} (01.02.1868 – 10.11.1907)|pwv}{}\ledrightnote{{$\rightarrow$}\emph{\textcolor{blue}{Julie Herzl}}} sass auf einem Sessel. Ich hatte mir
               meinen \textcolor{blue}{Buben}\pwindex{Herzl, Hans 10.06.1891 – 14.09.1930@\textsc{Herzl, Hans} (10.06.1891 – 14.09.1930)|pwv}{}\ledrightnote{{$\rightarrow$}\emph{\textcolor{blue}{Hans Herzl}}} ins
               Schreibzimmer gelegt, um wenns noch nicht zu {\pb}spät wäre die Ansteckung zu verhindern.
               Der \textcolor{blue}{Kerl}\pwindex{Herzl, Hans 10.06.1891 – 14.09.1930@\textsc{Herzl, Hans} (10.06.1891 – 14.09.1930)|pwv}{}\ledrightnote{{$\rightarrow$}\emph{\textcolor{blue}{Hans Herzl}}} hat die ganze Nacht
               krakehlt, erst als ich ihm Prügel fest versprach, wurde er ruhig und sagte: Nit
               weinen, wieder lieb! – Er ist zwei Jahre alt.\pend
           
\pstart
           Kinder sind immerwährend zugleich Freude u. Angst, u. aus beiden Gründen wird Einem
               durch sie das Leben lieb.\pend
           
\pstart
           Auch in der Politik darf nichts dazwischen kommen, dann reisen wir am
                  26 ds. mit dem \textcolor{brown}{O. E.}\orgindex{Orient Express@Orient Express|pw}{}\ledrightnote{\textcolor{brown}{Orient Express}} nach \textcolor{pink}{Wien}\oindex{Wien@\textbf{Wien}, \emph{A.ADM2}|pw}{}\ledrightnote{\textcolor{pink}{Wien}}. Ich möchte bevor wir nach \textcolor{pink}{Baden}\oindex{Baden bei Wien@\textbf{Baden bei Wien}, \emph{P.PPLA3}|pw}{}\ledrightnote{\textcolor{pink}{Baden bei Wien}} gehen einigemal in die \textcolor{pink}{Wiener}\oindex{Wien@\textbf{Wien}, \emph{A.ADM2}|pw}{}\ledrightnote{\textcolor{pink}{Wien}} Theater gehen. Ich weiss nicht mehr, wie sich der Lieutenant in den
               Backfisch verlieben kann u. umgekehrt. Gesteh’ ichs, ich sehne mich wieder nach der
               heimischen \label{K_L03830-2v}\edtext{Imbecillität}{\lemma{\textnormal{\emph{Imbecillität}}}\Cendnote{\textnormal{Schwachsinn}}}\label{K_L03830-2}.\pend
           
\pstart
           Mit dem \textcolor{green}{Flüchtling}\pwindex{Fluechtling. Lustspiel in einem Aufzug@\emph{Der Flüchtling. Lustspiel in einem Aufzug}|pw}{}\ledrightnote{\textcolor{green}{Der Flüchtling. Lustspiel in einem Aufzug}} ists sonderbar. {\pb}Mein erster \label{K_L03830-3v}\edtext{Erfolg in \textcolor{pink}{Berlin}\oindex{Berlin@\textbf{Berlin}, \emph{P.PPLC}|pw}{}\ledrightnote{\textcolor{pink}{Berlin}}}{\lemma{\textnormal{\emph{Erfolg in Berlin}}}\Cendnote{\textnormal{ Die \textcolor{violet}{Berlinpremiere}\eventindex{Berliner Theater@\textbf{Berliner Theater}!Auffuehrung von Der Fluechtling und Die Eine weint, die Andere lacht, 31.5.1893@Aufführung von Der Flüchtling und Die Eine weint, die Andere lacht, 31.5.1893|pwkv} von \emph{\textcolor{green}{Der
                     Flüchtling. Lustspiel in einem Aufzug}\pwindex{Fluechtling. Lustspiel in einem Aufzug@\emph{Der Flüchtling. Lustspiel in einem Aufzug}|pwk}} von \textcolor{blue}{Theodor Herzl}\pwindex{Herzl, Theodor 1860-05-02 – 1904-07-03@\textsc{Herzl, Theodor} (1860-05-02 – 1904-07-03), \emph{Schriftsteller, Journalist}|pwk} fand zusammen mit einer Aufführung von \emph{\textcolor{green}{Die Eine weint, die Andere lacht. Schauspiel in
                     vier Akten}\pwindex{Eine weint, die Andere lacht. Schauspiel in vier Akten@\emph{Die Eine weint, die Andere lacht. Schauspiel in vier Akten}|pwk}} von \textcolor{blue}{Philippe Dumanoir}\pwindex{Dumanoir, Philippe 1806-07-25 – 1865-11-13@\textsc{Dumanoir, Philippe} (1806-07-25 – 1865-11-13), \emph{Schriftsteller/Schriftstellerin, Librettist/Librettistin}|pwk}
                  und \textcolor{blue}{Ange de Kéraniou}\pwindex{Keraniou, Ange de 1829-05-04 – 1872@\textsc{Kéraniou, Ange de} (1829-05-04 – 1872), \emph{Schriftsteller}|pwk} am 31. 5. 1893 am \emph{\textcolor{brown}{Berliner
                     Theater}\orgindex{Berliner Theater@Berliner Theater|pwk}} statt. }}}\label{K_L03830-3}. Als ich einige Tage später davon hörte, nahm ich
               das \textcolor{green}{Buch}\pwindex{Fluechtling. Lustspiel in einem Aufzug@\emph{Der Flüchtling. Lustspiel in einem Aufzug}|pwv}{}\ledrightnote{{$\rightarrow$}\emph{\textcolor{green}{Der Flüchtling. Lustspiel in einem Aufzug}}} vor, las es, war über
               die saloppe Sprache entsetzt; nur das, was die vornehmen \label{K_L03830-4v}\edtext{Kritiker rügten}{\lemma{\textnormal{\emph{Kritiker rügten}}}\Cendnote{\textnormal{Der Kritiker \textcolor{blue}{Ludwig Berthold}\pwindex{Berthold, Ludwig @\textsc{Berthold, Ludwig}, \emph{Journalist, Theaterkritiker}|pwk} bemängelte
                  die Nebenhandlung: »Nebenbei läuft noch eine sehr gleichgiltige Gesellschafterin
                  und deren vom Schnupfen befallener, immerwährend niesender Liebhaber. Diese grobe
                  Geschmacklosigkeit [...] wäre im Stande gewesen gewesen, das Interesse [...]
                  abzuschwächen, wenn nicht Frl. \textcolor{blue}{Nuscha
                     Butze}\pwindex{Butze, Nuscha 1860-02-22 – 1913-12-10@\textsc{Butze, Nuscha} (1860-02-22 – 1913-12-10), \emph{Schauspielerin, Theaterdirektorin}|pwk} und Herr \textcolor{blue}{Ludwig Stahl}\pwindex{Stahl, Ludwig? 04.04.1856 – 25.08.1908@\textsc{Stahl, Ludwig?} (04.04.1856 – 25.08.1908), \emph{Regisseur, Schauspieler}|pwk} den
                  erkälteten Herrn etwas in den Hintergrund gedrängt hätten [...].« (\textcolor{blue}{L. B.}\pwindex{Berthold, Ludwig @\textsc{Berthold, Ludwig}, \emph{Journalist, Theaterkritiker}|pwkv}: \emph{\textcolor{green}{Berliner Theaterbericht}\pwindex{Berliner Theaterbericht@\emph{Berliner Theaterbericht}|pwk}}. In: \emph{\textcolor{green}{Die Gesellschaft. Illustrirtes
                     Wochenblatt}\pwindex{Gesellschaft. Politisches illustriertes Wochenblatt@\emph{Die Gesellschaft. Politisches illustriertes Wochenblatt}|pwk}}, Jg. 7, Nr. 23, 11. 6. 1893,
               S. 10)}}}\label{K_L03830-4}, hat mich ergötzt: das Niesen des Eifersüchtigen. Es ist ein
               prächtiger Bühneneinfall, denn an der Stelle ist nicht mehr Zeit, auch nur mit einem
               Wort zu verhindern, dass die \substVorne{}\textsuperscript{Leute}\substDazwischen{}Zuschauer\substHinten{} den  Streit für ernst halten, u. damit man die Angst der Margarethe, auf die
               es ankommt, beobachten könne, muss die \label{K_L03830-5v}\edtext{Contrahage}{\lemma{\textnormal{\emph{Contrahage}}}\Cendnote{\textnormal{Forderung zum
                  Duell}}}\label{K_L03830-5} spassig sein.\pend
           
\pstart
           Was sagen Sie, mit welcher \label{K_L03830-6v}\edtext{\begin{otherlanguage}{french}désinvolture\end{otherlanguage}}{\lemma{\textnormal{\emph{désinvolture}}}\Cendnote{\textnormal{französisch: Ungeniertheit}}}\label{K_L03830-6} ich
               mich lobe? Deutlicher als alles sagt Ihnen dies, dass ich von einem Abgeschiedenen
               spreche.\pend
           
\pstart
           Beurtheilen Sie die Aufführung des \textcolor{green}{Flüchtlings}\pwindex{Fluechtling. Lustspiel in einem Aufzug@\emph{Der Flüchtling. Lustspiel in einem Aufzug}|pw}{}\ledrightnote{\textcolor{green}{Der Flüchtling. Lustspiel in einem Aufzug}}
               nicht falsch. Sie braucht sie ebensowenig zu demüthigen, wie die Erfolge der
               gewöhnlichen Dutzendscribenten. Verstehen Sie {\pb}das Leben! Hier die Geschichte des \textcolor{green}{Flüchtlings}\pwindex{Fluechtling. Lustspiel in einem Aufzug@\emph{Der Flüchtling. Lustspiel in einem Aufzug}|pw}{}\ledrightnote{\textcolor{green}{Der Flüchtling. Lustspiel in einem Aufzug}}. 1887 wollte ich eine
                  \textcolor{pink}{italienische}\oindex{Italien@\textbf{Italien}, \emph{A.PCLI}|pw}{}\ledrightnote{\textcolor{pink}{Italien}} Reise machen. Reisegeld gabs
               verflucht wenig. \textcolor{blue}{Groller}\pwindex{Groller, Balduin 05.09.1848 – 22.03.1916@\textsc{Groller, Balduin} (05.09.1848 – 22.03.1916), \emph{Schriftsteller, Journalist}|pw}{}\ledrightnote{\textcolor{blue}{Balduin Groller}} (\textcolor{brown}{Illustrirte Zeitung}\orgindex{Neue Illustrierte Zeitung@Neue Illustrierte Zeitung|pw}{}\ledrightnote{\textcolor{brown}{Neue Illustrierte Zeitung}}) war charmant genug, mir \strikeout{den} damals zu sagen, ich solle, wie \label{K_L03830-7v}\edtext{für die \textcolor{green}{blaue Donau}\pwindex{der schoenen blauen Donau@\emph{An der schönen blauen Donau}|pw}{}\ledrightnote{\textcolor{green}{An der schönen blauen Donau}}}{\lemma{\textnormal{\emph{für die blaue Donau}}}\Cendnote{\textnormal{In der Zeitschrift \emph{\textcolor{green}{An der schönen Blauen Donau}\pwindex{der schoenen blauen Donau@\emph{An der schönen blauen Donau}|pwk}} waren von \textcolor{blue}{Herzl}\pwindex{Herzl, Theodor 1860-05-02 – 1904-07-03@\textsc{Herzl, Theodor} (1860-05-02 – 1904-07-03), \emph{Schriftsteller, Journalist}|pwk} im Vorjahr der dramatische Scherz \emph{\textcolor{green}{Schlechte Nachrichten}\pwindex{Schlechte Nachrichten. Ein dramatischer Scherz@\emph{Schlechte Nachrichten. Ein dramatischer Scherz}|pwk}} und die Novelle \emph{\textcolor{green}{Der sechste Welttheil}\pwindex{sechste Welttheil@\emph{Der sechste Welttheil}|pwk}} erschienen (\emph{\textcolor{green}{Schlechte Nachrichten. Ein dramatischer
                        Scherz}\pwindex{Schlechte Nachrichten. Ein dramatischer Scherz@\emph{Schlechte Nachrichten. Ein dramatischer Scherz}|pwk}}. In: \emph{\textcolor{green}{An der schönen blauen
                        Donau}\pwindex{der schoenen blauen Donau@\emph{An der schönen blauen Donau}|pwk}}, Jg. 1, H. 2, 1. 2. 1886, S. 50–52 und \emph{\textcolor{green}{Der sechste Welttheil}\pwindex{sechste Welttheil@\emph{Der sechste Welttheil}|pwk}}. In: \emph{\textcolor{green}{An der schönen blauen Donau}\pwindex{der schoenen blauen Donau@\emph{An der schönen blauen Donau}|pwk}}, Jg. 1, H. 9,
                        15. 5. 1886, S. 257–259). 1887 wurde dort
                  sein Einakter \emph{\textcolor{green}{Die causa Hirschkorn}\pwindex{causa Hirschkorn. Lustspiel in einem Act@\emph{Die causa Hirschkorn. Lustspiel in einem Act}|pwk}} abgedruckt (\emph{\textcolor{green}{Die causa Hirschkorn. Lustspiel in einem
                        Act}\pwindex{causa Hirschkorn. Lustspiel in einem Act@\emph{Die causa Hirschkorn. Lustspiel in einem Act}|pwk}}. In: \emph{\textcolor{green}{An der schönen blauen
                        Donau}\pwindex{der schoenen blauen Donau@\emph{An der schönen blauen Donau}|pwk}}, Jg. 2, H. 11, 1. 6. 1887,
                  S. 254–256.).}}}\label{K_L03830-7} etwas Dramatisches \label{K_L03830-8v}\edtext{für ihn schreiben}{\lemma{\textnormal{\emph{für ihn schreiben}}}\Cendnote{\textnormal{\emph{\textcolor{green}{Der Flüchtling. Lustspiel in einem Act}\pwindex{Fluechtling. Lustspiel in einem Aufzug@\emph{Der Flüchtling. Lustspiel in einem Aufzug}|pwk}}.
                     In: \emph{\textcolor{green}{Neue Illustrirte Zeitung}\pwindex{Neue Illustrierte Zeitung@\emph{Neue Illustrierte Zeitung}|pwk}}, Jg. 15,
                     Bd. 2, Nr. 36, 5. 6. 1887, S. 567–569 und Nr. 37,
                        12. 6. 1887, S. 579–582.}}}\label{K_L03830-8}. Gerade sauste mir der
               Ihnen bekannte Einfall dieses \textcolor{green}{Einakters}\pwindex{Fluechtling. Lustspiel in einem Aufzug@\emph{Der Flüchtling. Lustspiel in einem Aufzug}|pwv}{}\ledrightnote{{$\rightarrow$}\emph{\textcolor{green}{Der Flüchtling. Lustspiel in einem Aufzug}}} durch den Kopf. Hingesetzt u. hingeschleudert. Ich glaube in drei
               Tagen. Ich wollte schon abreisen. Nicht mehr deutlich weiss ich ob ich das Honorar
               vorgeschossen bekam. Ich vermuthe es, denn ich reiste ab u. schrieb mich dann bis \textcolor{pink}{Neapel}\oindex{Neapel@\textbf{Neapel}, \emph{P.PPLA}|pw}{}\ledrightnote{\textcolor{pink}{Neapel}} durch. (Freilich hat mein guter \textcolor{blue}{Vater}\pwindex{Herzl, Jakob 1837-03-14 – 1902-06-09@\textsc{Herzl, Jakob} (1837-03-14 – 1902-06-09), \emph{Bankdirektor, Großkaufmann}|pw}{}\ledrightnote{\textcolor{blue}{Jakob Herzl}} auch was hergegeben.) Dieser \textcolor{green}{Schmarrn}\pwindex{Fluechtling. Lustspiel in einem Aufzug@\emph{Der Flüchtling. Lustspiel in einem Aufzug}|pwv}{}\ledrightnote{{$\rightarrow$}\emph{\textcolor{green}{Der Flüchtling. Lustspiel in einem Aufzug}}}, den ich wie alle
               meine Stücke dem \textcolor{brown}{Burgtheater}\orgindex{Burgtheater@Burgtheater|pw}{}\ledrightnote{\textcolor{brown}{Burgtheater}} einreichte, wurde
               ich weiss nicht mehr warum – gewiss aus keinem literarischen Grunde – angenommen
               u. lag dann zwei Jahre. \textcolor{blue}{Förster}\pwindex{Foerster, August 03.06.1828 – 22.12.1889@\textsc{Förster, August} (03.06.1828 – 22.12.1889), \emph{Theaterleiter, Regisseur, Schauspieler}|pw}{}\ledrightnote{\textcolor{blue}{August Förster}} wurde
               Director. Ich war bei der \textcolor{brown}{Allg. Ztg}\orgindex{Wiener Allgemeine Zeitung@Wiener Allgemeine Zeitung|pw}{}\ledrightnote{\textcolor{brown}{Wiener Allgemeine Zeitung}}. Ich hatte
                  {\pb}in der \textcolor{brown}{Redaction}\orgindex{Wiener Allgemeine Zeitung@Wiener Allgemeine Zeitung|pwv}{}\ledrightnote{{$\rightarrow$}\emph{\textcolor{brown}{Wiener Allgemeine Zeitung}}} einen unangenehmen \textcolor{blue}{Collegen}\pwindex{?? [Kulturredakteur der Wiener Allgemeinen Zeitung] @\textsc{?? [Kulturredakteur der Wiener Allgemeinen Zeitung]}|pwv}{}\ledrightnote{{$\rightarrow$}\emph{\textcolor{blue}{?? [Kulturredakteur der Wiener Allgemeinen Zeitung]}}}, einen boshaften
               Narren, der mich molestirte wo er konnte u. mit dem ich nicht einmal auf dem
               Grussfuss stand. Dieser schrieb \label{K_L03830-9v}\edtext{eine
               hämische \textcolor{green}{Notiz}\pwindex{Theater an der Wien]@\emph{[Theater an der Wien]}|pwv}{}\ledrightnote{{$\rightarrow$}\emph{\textcolor{green}{[Theater an der Wien]}}}}{\lemma{\textnormal{\emph{eine
               hämische Notiz}}}\Cendnote{\textnormal{Am 4. 4. 1889 berichtete
                  die \emph{\textcolor{green}{Wiener Allgemeine Zeitung}\pwindex{Wiener Allgemeine Zeitung@\emph{Wiener Allgemeine Zeitung}|pwk}} über eine neue
                     \textcolor{violet}{Aufführung}\eventindex{Theater an der Wien@\textbf{Theater an der Wien}!Auffuehrung von O, diese Schwiegermutter, 2.4.1889@Aufführung von O, diese Schwiegermutter{\rufezeichen}, 2.4.1889|pwkv} des
                  Schwanks \emph{\textcolor{green}{O, diese Schwiegermutter}\pwindex{O, diese Schwiegermutter@\emph{O, diese Schwiegermutter{\rufezeichen}}|pwk}}. Die
                  Hauptrolle, die seit der \textcolor{violet}{Premiere}\eventindex{Theater an der Wien@\textbf{Theater an der Wien}!Premiere von Die Zaubergeige, O, diese Schwiegermutter, 1.12.1888@Premiere von Die Zaubergeige, O, diese Schwiegermutter{\rufezeichen}, 1.12.1888|pwkv} am 1. 12. 1888 der Schauspieler \textcolor{blue}{Alexander Giradi}\pwindex{Girardi, Alexander 05.12.1850 – 20.04.1918@\textsc{Girardi, Alexander} (05.12.1850 – 20.04.1918), \emph{Schauspieler}|pwk} innegehabt hatte, wurde nun von \textcolor{blue}{Heinrich Förster}\pwindex{Foerster, Heinrich 1859-06-27 – 1897-09-08@\textsc{Förster, Heinrich} (1859-06-27 – 1897-09-08), \emph{Schauspieler, Theaterregisseur}|pwk}, Sohn des \textcolor{blue}{Burgtheaterdirektors}\pwindex{Foerster, August 03.06.1828 – 22.12.1889@\textsc{Förster, August} (03.06.1828 – 22.12.1889), \emph{Theaterleiter, Regisseur, Schauspieler}|pwkv}
                  übernommen. In der \textcolor{green}{Kritik}\pwindex{Theater an der Wien]@\emph{[Theater an der Wien]}|pwkv}
                  heißt es u. a.: »Sagen wir es rund heraus: Herr \textcolor{blue}{Förster}\pwindex{Foerster, Heinrich 1859-06-27 – 1897-09-08@\textsc{Förster, Heinrich} (1859-06-27 – 1897-09-08), \emph{Schauspieler, Theaterregisseur}|pw} hat gestern den Erwartungen
                     keineswegs entsprochen. [...], man kann sich keinen grelleren Kontrast denken,
                     als \textcolor{blue}{Girardi}\pwindex{Girardi, Alexander 05.12.1850 – 20.04.1918@\textsc{Girardi, Alexander} (05.12.1850 – 20.04.1918), \emph{Schauspieler}|pw} und Herrn \textcolor{blue}{Förster}\pwindex{Foerster, Heinrich 1859-06-27 – 1897-09-08@\textsc{Förster, Heinrich} (1859-06-27 – 1897-09-08), \emph{Schauspieler, Theaterregisseur}|pw} in der Rolle des Henri Duval. [...] Zum
                     Bonvivant mangeln ihm alle Eigenschaften.« (\emph{\textcolor{green}{[Theater an der Wien]}\pwindex{Theater an der Wien]@\emph{[Theater an der Wien]}|pwk}}. In: \emph{\textcolor{green}{Wiener Allgemeine Zeitung}\pwindex{Wiener Allgemeine Zeitung@\emph{Wiener Allgemeine Zeitung}|pwk}}, Nr. 3257,
                        4. 4. 1889, S. 4.)}}}\label{K_L03830-9} über \textcolor{blue}{Foersters}\pwindex{Foerster, August 03.06.1828 – 22.12.1889@\textsc{Förster, August} (03.06.1828 – 22.12.1889), \emph{Theaterleiter, Regisseur, Schauspieler}|pw}{}\ledrightnote{\textcolor{blue}{August Förster}}{ }\textcolor{blue}{Sohn}\pwindex{Foerster, Heinrich 1859-06-27 – 1897-09-08@\textsc{Förster, Heinrich} (1859-06-27 – 1897-09-08), \emph{Schauspieler, Theaterregisseur}|pw}{}\ledrightnote{\textcolor{blue}{Heinrich Förster}}. \textcolor{blue}{Förster}\pwindex{Foerster, August 03.06.1828 – 22.12.1889@\textsc{Förster, August} (03.06.1828 – 22.12.1889), \emph{Theaterleiter, Regisseur, Schauspieler}|pw}{}\ledrightnote{\textcolor{blue}{August Förster}} glaubte, dass mein »\textcolor{blue}{Kamerad}\pwindex{?? [Kulturredakteur der Wiener Allgemeinen Zeitung] @\textsc{?? [Kulturredakteur der Wiener Allgemeinen Zeitung]}|pwv}{}\ledrightnote{{$\rightarrow$}\emph{\textcolor{blue}{?? [Kulturredakteur der Wiener Allgemeinen Zeitung]}}}« mich Unaufgeführten rächen wollte u. \label{K_L03830-10v}\edtext{setzte den \textcolor{green}{Flüchtling}\pwindex{Fluechtling. Lustspiel in einem Aufzug@\emph{Der Flüchtling. Lustspiel in einem Aufzug}|pw}{}\ledrightnote{\textcolor{green}{Der Flüchtling. Lustspiel in einem Aufzug}} erschrocken an}{\lemma{\textnormal{\emph{setzte … an}}}\Cendnote{\textnormal{Am
                     4. 5. 1889 erlebte \textcolor{blue}{Herzls}\pwindex{Herzl, Theodor 1860-05-02 – 1904-07-03@\textsc{Herzl, Theodor} (1860-05-02 – 1904-07-03), \emph{Schriftsteller, Journalist}|pwk}
                  Lustspiel \emph{\textcolor{green}{Der Flüchtling}\pwindex{Fluechtling. Lustspiel in einem Aufzug@\emph{Der Flüchtling. Lustspiel in einem Aufzug}|pwk}} seine \textcolor{violet}{Uraufführung}\eventindex{Burgtheater@\textbf{Burgtheater}!Premiere von Der Schierling, Urauffuehrung von Im Bunde der Dritte, Der Fluechtling, 4.5.1889@Premiere von Der Schierling, Uraufführung von Im Bunde der Dritte, Der Flüchtling, 4.5.1889|pwkv} am \emph{\textcolor{brown}{Burgtheater}\orgindex{Burgtheater@Burgtheater|pwk}}.}}}\label{K_L03830-10}. Sind das Komödien,
               was?\pend
           
\pstart
           Noch besser die \textcolor{pink}{Berliner}\oindex{Berlin@\textbf{Berlin}, \emph{P.PPLC}|pw}{}\ledrightnote{\textcolor{pink}{Berlin}} Geschichte des \textcolor{green}{Flüchtlings}\pwindex{Fluechtling. Lustspiel in einem Aufzug@\emph{Der Flüchtling. Lustspiel in einem Aufzug}|pw}{}\ledrightnote{\textcolor{green}{Der Flüchtling. Lustspiel in einem Aufzug}}. Sie wissen dass ich mit meinem Stück
                  »\textcolor{green}{Der Bernhardiner}\pwindex{Was wird man sagen?@\emph{Was wird man sagen?}|pw}{}\ledrightnote{\textcolor{green}{Was wird man sagen?}}« –  nicht mein
               schlechtestes, was freilich nichts sagen will – am \textcolor{pink}{Berliner}\oindex{Berlin@\textbf{Berlin}, \emph{P.PPLC}|pw}{}\ledrightnote{\textcolor{pink}{Berlin}} Theater einen der beschämendsten Durchfälle am »\textcolor{brown}{Berliner Theater}\orgindex{Berliner Theater@Berliner Theater|pw}{}\ledrightnote{\textcolor{brown}{Berliner Theater}}« erlitt. Es war ein Lustspiel, das \textcolor{blue}{Barnay}\pwindex{Barnay, Ludwig 1842-02-11 – 1924-02-01@\textsc{Barnay, Ludwig} (1842-02-11 – 1924-02-01), \emph{Schriftsteller, Schauspieler, Theaterdirektor}|pw}{}\ledrightnote{\textcolor{blue}{Ludwig Barnay}} weil er eine Rolle für sich zurecht
               schneidern wollte als Schauspiel spielte{[}.{]}{\pb}Es war eine zu verstohlene Satire auf
               die Sentimentalität u. jene Halben, die sich vom \label{K_L03830-11v}\edtext{\begin{otherlanguage}{french}qu' en dira-t-on\end{otherlanguage}}{\lemma{\textnormal{\emph{qu' en dira-t-on}}}\Cendnote{\textnormal{französisch: was man sagen wird, Gerede
                  der Leute}}}\label{K_L03830-11} leiten lassen. Alle Absichten wurden ins Gegentheil verkehrt,
               u. zw. unter meinen Augen. Ich war schwach genug zu allem Ja zu sagen, aber
               hauptsächlich war ich wirthschaftlich schwach. Ich brauchte \label{K_L03830-12v}\edtext{die Aufführung}{\lemma{\textnormal{\emph{die Aufführung}}}\Cendnote{\textnormal{
                  Die \emph{\textcolor{violet}{Theateruraufführung von \emph{\textcolor{green}{Was wird man sagen?}\pwindex{Was wird man sagen?@\emph{Was wird man sagen?}|pwk}} alias \emph{\textcolor{green}{Der Bernhardiner}\pwindex{Was wird man sagen?@\emph{Was wird man sagen?}|pwk}} von \textcolor{blue}{Theodor Herzl}\pwindex{Herzl, Theodor 1860-05-02 – 1904-07-03@\textsc{Herzl, Theodor} (1860-05-02 – 1904-07-03), \emph{Schriftsteller, Journalist}|pwk}}\eventindex{Berliner Theater@\textbf{Berliner Theater}!Urauffuehrung von Der Bernhardiner, 30.10.1890@Uraufführung von Der Bernhardiner, 30.10.1890|pwk}} fand am 30. 10. 1890 am \emph{\textcolor{brown}{Berliner Theater}\orgindex{Berliner Theater@Berliner Theater|pwk}} statt.}}}\label{K_L03830-12}. \pend
           
\pstart
           \textcolor{blue}{Barnay}\pwindex{Barnay, Ludwig 1842-02-11 – 1924-02-01@\textsc{Barnay, Ludwig} (1842-02-11 – 1924-02-01), \emph{Schriftsteller, Schauspieler, Theaterdirektor}|pw}{}\ledrightnote{\textcolor{blue}{Ludwig Barnay}} wusste wohl, dass ich ihm geschrieben
               u. gesagt hatte, dass mein Stück »\textcolor{green}{Was wird man
                  sagen?}\pwindex{Was wird man sagen?@\emph{Was wird man sagen?}|pw}{}\ledrightnote{\textcolor{green}{Was wird man sagen?}}« ein Lustspiel sei, u. dass ich es als solches gespielt wünsche. Er
               sah auch, wie tapfer u. schweigsam ich den ganzen Misserfolg allein trug, ohne zu
               maukezen. Ich hätte aus der Verhunzung meines \textcolor{green}{Stückes}\pwindex{Was wird man sagen?@\emph{Was wird man sagen?}|pw}{}\ledrightnote{\textcolor{green}{Was wird man sagen?}} immerhin ein Feuilleton herausfetzen können. Ich hatte aber nach der
               Niederlage die richtige Haltung {\pb}\strikeout{während} so wie ich sie vorher nicht hatte. Es wäre
               geschmacklos u. feig gewesen, die Schuld abzuwälzen. \pend
           
\pstart
           Aber wenn mich ganz \textcolor{pink}{Berlin}\oindex{Berlin@\textbf{Berlin}, \emph{P.PPLC}|pw}{}\ledrightnote{\textcolor{pink}{Berlin}} u. was dahinter steht
               – also alle deutschen Theater – für einen unfähigen Idioten tief unter nehmen wir an
                  \textcolor{blue}{Triesch}\pwindex{Triesch, Friedrich Gustav 16.06.1845 – 24.05.1907@\textsc{Triesch, Friedrich Gustav} (16.06.1845 – 24.05.1907), \emph{Schriftsteller}|pw}{}\ledrightnote{\textcolor{blue}{Friedrich Gustav Triesch}} halten mussten, der eine \textcolor{blue}{Barnay}\pwindex{Barnay, Ludwig 1842-02-11 – 1924-02-01@\textsc{Barnay, Ludwig} (1842-02-11 – 1924-02-01), \emph{Schriftsteller, Schauspieler, Theaterdirektor}|pw}{}\ledrightnote{\textcolor{blue}{Ludwig Barnay}} kannte das Unrecht, das ich ruhig
               aushielt. Nun, er lehnte dennoch ein \textcolor{green}{Stück}\pwindex{Prinzen aus Genieland. Lustspiel in 4 Akten@\emph{Prinzen aus Genieland. Lustspiel in 4 Akten}|pwv}{}\ledrightnote{{$\rightarrow$}\emph{\textcolor{green}{Prinzen aus Genieland. Lustspiel in 4 Akten}}} ab, mit dem ich \introOben{}vielleicht\introOben{}
               meine Revanche hätte nehmen können, obwol seine Regisseure es zur Aufführung
               empfahlen: das unter dem schlechten Titel \textcolor{green}{Prinzen aus
                  Genieland}\pwindex{Prinzen aus Genieland. Lustspiel in 4 Akten@\emph{Prinzen aus Genieland. Lustspiel in 4 Akten}|pw}{}\ledrightnote{\textcolor{green}{Prinzen aus Genieland. Lustspiel in 4 Akten}}{ }\label{K_L03830-13v}\edtext{im \textcolor{brown}{Carltheater}\orgindex{Carl-Theater@Carl-Theater|pw}{}\ledrightnote{\textcolor{brown}{Carl-Theater}}}{\lemma{\textnormal{\emph{im Carltheater}}}\Cendnote{\textnormal{ Die \emph{\textcolor{violet}{Theateruraufführung von \emph{\textcolor{green}{Prinzen aus Genieland.
                        Lustspiel in 4 Akten}\pwindex{Prinzen aus Genieland. Lustspiel in 4 Akten@\emph{Prinzen aus Genieland. Lustspiel in 4 Akten}|pwk}}}\eventindex{Carl-Theater@\textbf{Carl-Theater}!Urauffuehrung von Prinzen in Genieland, 20.11.1891@Uraufführung von Prinzen in Genieland, 20.11.1891|pwk}} fand am 20. 11. 1891 am \emph{\textcolor{brown}{Carl-Theater}\orgindex{Carl-Theater@Carl-Theater|pwk}} statt.}}}\label{K_L03830-13} von den \introOben{}Possen\introOben{}Darstellern wie ich glaube nicht umgebrachte \textcolor{green}{Künstlerlustspiel}\pwindex{Prinzen aus Genieland. Lustspiel in 4 Akten@\emph{Prinzen aus Genieland. Lustspiel in 4 Akten}|pwv}{}\ledrightnote{{$\rightarrow$}\emph{\textcolor{green}{Prinzen aus Genieland. Lustspiel in 4 Akten}}}.\pend
           
\pstart
           \textcolor{blue}{Barnay}\pwindex{Barnay, Ludwig 1842-02-11 – 1924-02-01@\textsc{Barnay, Ludwig} (1842-02-11 – 1924-02-01), \emph{Schriftsteller, Schauspieler, Theaterdirektor}|pw}{}\ledrightnote{\textcolor{blue}{Ludwig Barnay}} gibt jetzt sein \textcolor{brown}{Theater}\orgindex{Berliner Theater@Berliner Theater|pwv}{}\ledrightnote{{$\rightarrow$}\emph{\textcolor{brown}{Berliner Theater}}} auf. Er \label{K_L03830-14v}\edtext{ordnet offenbar sein Haus}{\lemma{\textnormal{\emph{ordnet … Haus}}}\Cendnote{\textnormal{\textcolor{blue}{Ludwig Barnay}\pwindex{Barnay, Ludwig 1842-02-11 – 1924-02-01@\textsc{Barnay, Ludwig} (1842-02-11 – 1924-02-01), \emph{Schriftsteller, Schauspieler, Theaterdirektor}|pwk} verließ \textcolor{pink}{Berlin}\oindex{Berlin@\textbf{Berlin}, \emph{P.PPLC}|pwk}{ }1894 und gab die Leitung des \emph{\textcolor{brown}{Berliner
                     Theaters}\orgindex{Berliner Theater@Berliner Theater|pwk}} ab.}}}\label{K_L03830-14} bevor er wieder auf Reisen geht. Vielleicht findet er,
               dass man den Journalisten {\pb}nicht
               unversöhnt herumgehen lassen darf – u. gibt \label{K_L03830-15v}\edtext{als letzte Novität}{\lemma{\textnormal{\emph{als letzte Novität}}}\Cendnote{\textnormal{So im Wortlaut im \emph{\textcolor{green}{Berliner
                     Theaterbericht}\pwindex{Berliner Theaterbericht@\emph{Berliner Theaterbericht}|pwk}} der Zeitschrift \emph{\textcolor{green}{Die
                     Gesellschaft}\pwindex{Gesellschaft. Politisches illustriertes Wochenblatt@\emph{Die Gesellschaft. Politisches illustriertes Wochenblatt}|pwk}} über die \textcolor{violet}{Premiere}\eventindex{Berliner Theater@\textbf{Berliner Theater}!Auffuehrung von Der Fluechtling und Die Eine weint, die Andere lacht, 31.5.1893@Aufführung von Der Flüchtling und Die Eine weint, die Andere lacht, 31.5.1893|pwkv}: »Das \emph{\textcolor{brown}{Berliner Theater}\orgindex{Berliner Theater@Berliner Theater|pwk}}
                  brachte am 31. v. M. als letzte Novität in dieser Saison ein
                  einactiges Lustspiel ›\emph{\textcolor{green}{Der Flüchtling}\pwindex{Fluechtling. Lustspiel in einem Aufzug@\emph{Der Flüchtling. Lustspiel in einem Aufzug}|pwk}}‹ von \textcolor{blue}{Theodor Herzl}\pwindex{Herzl, Theodor 1860-05-02 – 1904-07-03@\textsc{Herzl, Theodor} (1860-05-02 – 1904-07-03), \emph{Schriftsteller, Journalist}|pwk}.« (\textcolor{blue}{L. B.}\pwindex{Berthold, Ludwig @\textsc{Berthold, Ludwig}, \emph{Journalist, Theaterkritiker}|pwkv}: \emph{\textcolor{green}{Berliner Theaterbericht}\pwindex{Berliner Theaterbericht@\emph{Berliner Theaterbericht}|pwk}}. In: \emph{\textcolor{green}{Die Gesellschaft. Illustrirtes
                     Wochenblatt}\pwindex{Gesellschaft. Politisches illustriertes Wochenblatt@\emph{Die Gesellschaft. Politisches illustriertes Wochenblatt}|pwk}}, Jg. 7, Nr. 23, 11. 6. 1893,
                  S. 10).}}}\label{K_L03830-15} seiner Direction mein \textcolor{green}{Stückchen}\pwindex{Fluechtling. Lustspiel in einem Aufzug@\emph{Der Flüchtling. Lustspiel in einem Aufzug}|pwv}{}\ledrightnote{{$\rightarrow$}\emph{\textcolor{green}{Der Flüchtling. Lustspiel in einem Aufzug}}}, ohne mich zu fragen.\pend
           
\pstart
           Verstehen Sie das Leben, Freund! Ich habe \textcolor{blue}{Barnay}\pwindex{Barnay, Ludwig 1842-02-11 – 1924-02-01@\textsc{Barnay, Ludwig} (1842-02-11 – 1924-02-01), \emph{Schriftsteller, Schauspieler, Theaterdirektor}|pw}{}\ledrightnote{\textcolor{blue}{Ludwig Barnay}} für die Aufmerksamkeit nicht gedankt. Er ist davon wahrscheinlich
               sehr überrascht. Wie überrascht wäre er aber, wenn er wüsste dass ich Alles verziehen
                  \strikeout{was} obschon nicht vergessen hatte. Und dass er er
               gerade durch den Fehler, den er begangen, vor dem Stahl meiner Feder immer sicher
               war. Diese Leute wissen nicht, dass wir Anderen die Zeitung nie für unsere
               Privatangelegenheiten verwenden.\pend
           
\pstart
           Ja, ich könnte Ihnen viel erzählen, auch von der Lustspielconcurrenz und anderen
               Gemeinheiten des \textcolor{brown}{Deutschen Volkstheaters}\orgindex{Volkstheater@Volkstheater|pw}{}\ledrightnote{\textcolor{brown}{Volkstheater}} in \textcolor{pink}{Wien}\oindex{Wien@\textbf{Wien}, \emph{A.ADM2}|pw}{}\ledrightnote{\textcolor{pink}{Wien}}. \strikeout{Ich \textcolor{gray}{×}} Es hat lange gedauert, bis die Miserablen des {\pb}Theaters mich gebrochen haben. Sie hätten
               es nie zuwege gebracht, wenn ich mich nicht um sie gekümmert hätte, sondern
               geschrieben wie ich wollte, wie mirs zu Muthe und im Sinne war. Und ich sage Ihnen
               das, damit Sie aus meinem Falle lernen. Pfeifen Sie auf das Gesindel. Schreiben Sie
                  \uline{nur}, wie es Ihnen gefällt. Bei Ihrem Talent ist
               es, dann eine innere Nothwendigkeit, dass Sie auch eines nicht fernen Tages den
               äusseren Erfolg sehen. Aber werden Sie viel glücklicher sein, wenn man Sie \strikeout{solchen} vor die grosse \label{K_L03830-16v}\edtext{\begin{otherlanguage}{french}Courtine\end{otherlanguage}}{\lemma{\textnormal{\emph{Courtine}}}\Cendnote{\textnormal{französisch: Vorhang}}}\label{K_L03830-16} des \textcolor{pink}{Burg}\oindex{Burgtheater@\textbf{Burgtheater}, \emph{S.THTR}|pw}{}\ledrightnote{\textcolor{pink}{Burgtheater}} oder \textcolor{pink}{Lessingtheaters}\oindex{Lessing-Theater@\textbf{Lessing-Theater}, \emph{Theater (K.THE)}|pw}{}\ledrightnote{\textcolor{pink}{Lessing-Theater}} treten lässt? Das ist ein Glück welches täglich so u. so
               viele Mätzchenmacher haben.\pend
           
\pstart
           Auf Wiedersehen in \textcolor{pink}{Wien}\oindex{Wien@\textbf{Wien}, \emph{A.ADM2}|pw}{}\ledrightnote{\textcolor{pink}{Wien}}{\\[\baselineskip]} Ihr Freund{\\[\baselineskip]}\spacefill\mbox{Th Herzl}\pend
           \leftskip=0em{}
\pstart
           \noindent{}Ich brauche Ihnen nicht zu sagen dass alles, was in diesem Briefe steht nur \uline{für Sie allein} geschrieben ist.\pend
           \selectlanguage{ngerman}\endnumbering\briefempfaengerindex{Schnitzler, Arthur@\textsc{Schnitzler, Arthur}!zzzHerzl, Theodor@\emph{von Theodor Herzl}!1893-06-151@{15. 6. 1893}|)be}\mylabel{L03830h}
\begin{anhang}
\end{anhang}\normalsize

\doendnotes{C}
\bigskip
\vfill

\clearpage

\footnotesize

\lohead{\textsc{register}}

% Definiere theindex-Environment komplett neu ohne reledmac
\makeatletter
\renewenvironment{theindex}{%
  \section*{\indexname}%
  \setlength{\parindent}{0pt}%
  \setlength{\parskip}{0pt plus 0.3pt}%
  \let\item\@idxitem
}{%
  \clearpage
}
\makeatother

\IfFileExists{\jobname-pw.ind}{\input{\jobname-pw.ind}}{}

\end{document}

      