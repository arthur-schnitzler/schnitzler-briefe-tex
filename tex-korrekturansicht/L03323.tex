%% latex-korrekturansicht-vorspann.tex
%% Vorspann für die Korrekturansicht.
%% Lädt die gemeinsame Datei latex-vorspann.tex mit gesetztem Schalter.

\newif\ifkorrekturansicht
\korrekturansichttrue

\input{../tex-inputs/latex-vorspann}


\renewcommand{\erwaehnteOrte}{Orte: Burgtheater, Mödling, Wien}
\renewcommand{\erwaehnteWerke}{Werke: (Burgtheater.) [Das Komplott von Gustav Triesch], Das Komplott. Lustspiel in vier Akten, Tagebuch, Wiener Allgemeine Zeitung}
\section[ Felix Salten an Arthur Schnitzler, {[}19. 2. 1902{]}]{Felix Salten an Arthur Schnitzler, {[}19. 2. 1902{]}}
\nopagebreak\mylabel{v}
\rehead{ }\normalsize\beginnumbering\briefempfaengerindex{Schnitzler, Arthur@\textsc{Schnitzler, Arthur}!zzzSalten, Felix@\emph{von Felix Salten}!1902-02-191@{{[}19. 2. 1902{]}}|(be}
\toendnotes[C]{\smallbreak\pagebreak[2]}\Standort{CUL, Schnitzler, B 89, A 2.}
\physDesc{Brief, 1 Blatt, 2 Seiten, 302 Zeichen
\newline{}Handschrift: Bleistift, lateinische Kurrent
\newline{}Schnitzler: mit Bleistift datiert: »19/2 902« 
\newline{}Ordnung: mit Bleistift von unbekannter Hand nummeriert: »147« }\toendnotes[C]{\smallbreak}
\pstart
           \noindent{}{\pb}Lieber Arthur, entschuldigen Sie, dass ich \label{K_L03323-1v}\edtext{gestern}{\lemma{\textnormal{\emph{gestern}}}\Cendnote{\textnormal{\textcolor{blue}{Schnitzler} pendelte in
                  diesen Tagen zwischen \textcolor{pink}{Wien} und \textcolor{pink}{Mödling}. Für den 18. 2. 1902 gibt es
                  keinen Eintrag im \emph{\textcolor{green}{Tagebuch}}. Dieser Brief kann
                  als Indiz genommen werden, dass sich \textcolor{blue}{Schnitzler} an diesem Tag in \textcolor{pink}{Wien}
                  aufhielt.}}}\label{K_L03323-1h} nicht kam. Ich hatte eine abscheuliche Scene, die eben anfing,
               als ich fortgehen wollte (ohne damit in Zusa{\geminationm}enhang zu
               stehen) und die in aller Lieblichkeit {\pb}bis 12\textsuperscript{h} dauerte.\pend
           
\pstart
           Ich bin \label{K_L03323-2v}\edtext{Morgen{ }nach dem \textcolor{pink}{Burgth.}{}\ledrightnote{\textcolor{pink}{Burgtheater}}}{\lemma{\textnormal{\emph{Morgen nach dem Burgth.}}}\Cendnote{\textnormal{Zumindest
                  partiell erlaubt das die Verifizierung der Datierung. Am
                     18. 2. 1902 hatte \emph{\textcolor{green}{Das
                     Komplott. Lustspiel in vier Akten}} am \textcolor{pink}{Burgtheater} Uraufführung. \textcolor{blue}{Salten}
                  verriss sie (\textcolor{blue}{f. s.}:
                        \emph{\textcolor{green}{(Burgtheater.)}} In: \emph{\textcolor{green}{Wiener Allgemeine Zeitung. 6-Uhr-Blatt}}, Nr. 7.183,
                        20. 2. 1902, S. 2.). Es ist also unwahrscheinlich,
                  dass er sich das Stück ein zweites Mal ansah. Entsprechend ist eine ansonsten in der Korrespondenz durchaus
                  vorkommende Umdatierung des Schreibens um einen Tag früher oder später hier nicht wahrscheinlich, da 
                  er unter diesen Umständen neuerlich \emph{\textcolor{green}{Das
                     Komplott}} gesehen hätte.}}}\label{K_L03323-2h} im Caféhaus.
               Vielleicht sind Sie \label{K_L03323-3v}\edtext{dort}{\lemma{\textnormal{\emph{dort}}}\Cendnote{\textnormal{Einen Kaffeehausbesuch am  20. 2. 1902
                  kann mit \textcolor{blue}{Schnitzler}s \emph{\textcolor{green}{Tagebuch}} nicht belegt werden.}}}\label{K_L03323-3h}?\pend
           
\pstart
           Herzlichst {\\[\baselineskip]}Ihr {\\[\baselineskip]}\spacefill\mbox{Salten}\pend
           \leftskip=0em{}\endnumbering\briefempfaengerindex{Schnitzler, Arthur@\textsc{Schnitzler, Arthur}!zzzSalten, Felix@\emph{von Felix Salten}!1902-02-191@{{[}19. 2. 1902{]}}|)be}\mylabel{h}  \normalsize

\doendnotes{C}
\bigskip
\vfill

\clearpage

\footnotesize

\lohead{\textsc{register}}

% Definiere theindex-Environment komplett neu ohne reledmac
\makeatletter
\renewenvironment{theindex}{%
  \section*{\indexname}%
  \setlength{\parindent}{0pt}%
  \setlength{\parskip}{0pt plus 0.3pt}%
  \let\item\@idxitem
}{%
  \clearpage
}
\makeatother

\IfFileExists{\jobname-pw.ind}{\input{\jobname-pw.ind}}{}

\end{document}

      