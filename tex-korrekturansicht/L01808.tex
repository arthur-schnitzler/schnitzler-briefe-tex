%% latex-korrekturansicht-vorspann.tex
%% Vorspann für die Korrekturansicht.
%% Lädt die gemeinsame Datei latex-vorspann.tex mit gesetztem Schalter.

\newif\ifkorrekturansicht
\korrekturansichttrue

\input{../tex-inputs/latex-vorspann}


               \section[Hugo von Hofmannsthal an Arthur Schnitzler, {[}23. 11. 1908{]}]{ Hugo von Hofmannsthal an Arthur Schnitzler, {[}23. 11. 1908{]}}\nopagebreak\mylabel{v}\rehead{ }\normalsize\beginnumbering\briefempfaengerindex{Schnitzler, Arthur@\textsc{Schnitzler, Arthur}!zzzHofmannsthal, Hugo von@\emph{von Hugo von Hofmannsthal}!1908-11-231@{{[}23. 11. 1908{]}}|(be} \toendnotes[C]{\smallbreak\pagebreak[2]} \Standort{CUL, Schnitzler, B 43.}
\physDesc{Brief, 1 Blatt, 4 Seiten
\newline{}Handschrift: schwarze Tinte, deutsche Kurrent
\newline{}Schnitzler: mit Bleistift datiert: »Früh 909« und beschriftet: »Hugo« \newline{}Ordnung: 1) mit Bleistift von unbekannter Hand nummeriert: »\strikeout{298}« 2) mit Bleistift von unbekannter Hand nummeriert: »306«}\buchAbdrucke{\weitereDrucke{Hugo von Hofmannsthal, Arthur Schnitzler: \emph{Briefwechsel}. Hg. Therese Nickl und Heinrich Schnitzler. Frankfurt am Main: \emph{S. Fischer} 1964, S. 242–243.} }\toendnotes[C]{\smallbreak}\pstart
           \raggedleft{}{\pb}\textcolor{pink}{R.}{}\ledrightnote{\textcolor{pink}{Rodaun}}{\\}Montag.\pend
           \pstart{}mein lieber Arthur\pend\pstart
           ſo nett und gemütlich es \label{K_L01808_1v}\edtext{neulich}{\lemma{\textnormal{\emph{neulich}}}\Cendnote{\textnormal{am
                     26. 10. 1908 und am 15. 11. 1908}}}\label{K_L01808_1h} abends bei Euch war, ſo ſehr wünſche ich mir nach der ungewohnten
               Zufälligkeit, daſs wir \introOben{}2mal\introOben{} Fremde bei Euch trafen, wieder
               die Freude, Sie allein zu ſehen.\hspace*{1.5em}Es gibt Zeiten, in
               welchen man beſonders deutlich fühlt, welche Menſchen {\pb}auf der Welt man ſehr gern hat,
               und für mich ist dieſe jetzige Zeit eine ſolche.\pend
           \pstart
           Vielleicht, da Ihr viel vorhabt, telegrafiert ihr einmal, 1–2 Tage voraus, einen
               Abend wo wir kommen dürfen.\pend
           \pstart
           Die \textcolor{green}{Gedichte}{}\ledrightnote{→\textcolor{green}{[Gedichte]}} von \textcolor{blue}{Winterſtein}{}\ledrightnote{\textcolor{blue}{Alfred von Winterstein}} gefallen mir \uline{sehr}
               gut. Was würde ihm wünſchens{\pb}wert
               ſein daſs man dafür thäte? \pend
           \pstart
           Ich ſage mir manchmal, daſs vermutlich die Anfänge dieſer Erkrankung meiner Nerven
               weit zurück liegen und daſs meine \label{K_L01808_2v}\edtext{Verſtörtheit}{\lemma{\textnormal{\emph{Verſtörtheit}}}\Cendnote{\textnormal{siehe Hugo von Hofmannsthal an Arthur Schnitzler, 24. 7. [1908], 
                  vgl. A. S.: \emph{Tagebuch}, 24. 11. 1908}}}\label{K_L01808_2h}
               über gewiſſe Dinge in Ihrem \textcolor{green}{Roman}{}\ledrightnote{→\textcolor{green}{Der Weg ins Freie. Roman}}
               (menſchliche viel mehr als künſtleriſche, aber \uline{nicht}
               im Bereich des Judenproblems) {\pb}vielleicht ſchon nichts normales mehr war.\pend
           \pstart
           Auf Wiederſehen, mein lieber Arthur.\pend
           \pstart
           Ihr alter{\\[\baselineskip]}\spacefill\mbox{Hugo.}\pend
           \leftskip=0em{}\pstart
           \noindent{}Dem Profeſſor \textcolor{blue}{Seidler}{}\ledrightnote{\textcolor{blue}{Gustav Seidler}} hab ich gedankt.\pend
           \endnumbering\briefempfaengerindex{Schnitzler, Arthur@\textsc{Schnitzler, Arthur}!zzzHofmannsthal, Hugo von@\emph{von Hugo von Hofmannsthal}!1908-11-231@{{[}23. 11. 1908{]}}|)be}\mylabel{h}  \normalsize

\doendnotes{C}
\bigskip
\vfill

\clearpage

\footnotesize

\lohead{\textsc{register}}

% Definiere theindex-Environment komplett neu ohne reledmac
\makeatletter
\renewenvironment{theindex}{%
  \section*{\indexname}%
  \setlength{\parindent}{0pt}%
  \setlength{\parskip}{0pt plus 0.3pt}%
  \let\item\@idxitem
}{%
  \clearpage
}
\makeatother

\IfFileExists{\jobname-pw.ind}{\input{\jobname-pw.ind}}{}

\end{document}

      