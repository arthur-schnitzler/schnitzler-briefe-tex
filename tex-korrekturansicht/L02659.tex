%% latex-korrekturansicht-vorspann.tex
%% Vorspann für die Korrekturansicht.
%% Lädt die gemeinsame Datei latex-vorspann.tex mit gesetztem Schalter.

\newif\ifkorrekturansicht
\korrekturansichttrue

\input{../tex-inputs/latex-vorspann}


               \section[Paul Goldmann an Arthur Schnitzler, 7. 1. 1891]{ Paul Goldmann an Arthur Schnitzler, 7. 1. 1891}\nopagebreak\mylabel{v}\rehead{ }\normalsize\beginnumbering\briefempfaengerindex{Schnitzler, Arthur@\textsc{Schnitzler, Arthur}!zzzGoldmann, Paul@\emph{von Paul Goldmann}!1891-01-072@{7. 1. 1891}|(be} \toendnotes[C]{\smallbreak\pagebreak[2]} \Standort{DLA, A:Schnitzler, HS.NZ85.1.3162.}
\physDesc{Postkarte
\newline{}Handschrift: 1) blaue Tinte, deutsche Kurrent\hspace{1em}2) blaue Tinte, lateinische Kurrent (\noindent{}Adresse)\hspace{1em}\newline{}Versand: 1) Stempel: »\nobreak{}Wien
                                       {[}T{]}elegrafen-Centrale, 8-\textcolor{gray}{1-9}1, 12 V\nobreak{}«.  2) Stempel: »\nobreak{}Wien Kärntnerring, 8/1 91, 12–1 N\nobreak{}«. 
\newline{}Schnitzler: mit Bleistift das Datum »8/1 91« vermerkt }\toendnotes[C]{\smallbreak}\pstart{}{\pb}Herrn\pend{}\pstart{}Dr. Arthur Schnitzler\pend{}\pstart{}\textcolor{pink}{I. Giselastraße II\textsubscript{III}.
                  }{}\ledrightnote{\textcolor{pink}{Bösendorferstraße}}\pend{}\pstart{}\textcolor{pink}{Wien}{}\ledrightnote{\textcolor{pink}{Wien}}\pend{}{\bigskip}\pstart
           {\pb}\textcolor{pink}{Wien}{}\ledrightnote{\textcolor{pink}{Wien}} den \textsuperscript{7}/\textsubscript{1} 1891.
               \pend
           \pstart
           Lieber Arthur! Herzlichſten Dank für Deine
               Liebenswürdigkeit! Das \textcolor{green}{Referat}{}\ledrightnote{→\textcolor{green}{?? [Rezension des Gastspiels von Anna Hochenburger, 7.1.1891]}}
               ſchreib’ ſo groß wie Du willſt, 30, 40, 50 Zeilen; nur – nochmals – darf Niemand
               erfahren, daß Du es geſchrieben. Wenn du \label{K_L02659-2v}\edtext{heut{ }Abend}{\lemma{\textnormal{\emph{heut Abend}}}\Cendnote{\textnormal{\textcolor{blue}{Goldmann} datierte die Postkarte auf den
                     7. 1. 1891, während der Poststempel den 8. 1. 1891 ausweist, was sich auch durch die erwähnte Theateraufführung
                  belegen lässt. Erklärbar wäre das damit, dass die Karte zwar tatsächlich am 7. verfasst wurde, aber zu einer so späten Uhrzeit,
                  dass klar war, dass nicht mehr die Theateraufführungen des gleichen Tages, sondern nur die vom Folgetag gemeint sein konnten.}}}\label{K_L02659-2h} Zeit haſt, würde ich mich ſehr
               freuen, Dich im »\label{K_L02659-3v}\edtext{\textcolor{brown}{Theater an der Wien}{}\ledrightnote{\textcolor{brown}{Theater an der Wien}}}{\lemma{\textnormal{\emph{Theater an der Wien}}}\Cendnote{\textnormal{Tatsächlich sahen sich beide am 8. 1. 1891 die Operette
                     \emph{\textcolor{green}{Boccaccio}} von \textcolor{blue}{Franz von Suppè} an.}}}\label{K_L02659-3h}« Loge N\textsuperscript{o} 6, 1. Stock, zu ſehen \introOben{}Karte brauchſt Du keine.\introOben{}
                  (\textsc{\textcolor{green}{Boccaccio}{}\ledrightnote{\textcolor{green}{Boccaccio. Komische Operette in 3 Acten}}}). Schreib’ mir, ob Du kommen kannſt.\pend
           \pstart Herzl. Gruß Dein \spacefill\mbox{P. G.}\pend{}\endnumbering\briefempfaengerindex{Schnitzler, Arthur@\textsc{Schnitzler, Arthur}!zzzGoldmann, Paul@\emph{von Paul Goldmann}!1891-01-072@{7. 1. 1891}|)be}\mylabel{h}  \normalsize

\doendnotes{C}
\bigskip
\vfill

\clearpage

\footnotesize

\lohead{\textsc{register}}

% Definiere theindex-Environment komplett neu ohne reledmac
\makeatletter
\renewenvironment{theindex}{%
  \section*{\indexname}%
  \setlength{\parindent}{0pt}%
  \setlength{\parskip}{0pt plus 0.3pt}%
  \let\item\@idxitem
}{%
  \clearpage
}
\makeatother

\IfFileExists{\jobname-pw.ind}{\input{\jobname-pw.ind}}{}

\end{document}

      