%% latex-korrekturansicht-vorspann.tex
%% Vorspann für die Korrekturansicht.
%% Lädt die gemeinsame Datei latex-vorspann.tex mit gesetztem Schalter.

\newif\ifkorrekturansicht
\korrekturansichttrue

\input{../tex-inputs/latex-vorspann}


\renewcommand{\erwaehntePersonen}{Personen: Ernest von Koerber, Julius Szeps}
\renewcommand{\erwaehnteInstitutionen}{Institutionen: Ministerium für Inneres, Wiener Verlag}
\renewcommand{\erwaehnteOrte}{Orte: Wien}
\renewcommand{\erwaehnteWerke}{Werke: Reigen. Zehn Dialoge, Wiener Allgemeine Montags-Zeitung}
\section[ Felix Salten an Arthur Schnitzler, {[}27. 6. 1899{]}]{Felix Salten an Arthur Schnitzler, {[}27. 6. 1899{]}}
\nopagebreak\mylabel{v}
\rehead{ }\normalsize\beginnumbering\briefempfaengerindex{Schnitzler, Arthur@\textsc{Schnitzler, Arthur}!zzzSalten, Felix@\emph{von Felix Salten}!1899-06-271@{{[}27. 6. 1899{]}}|(be}
\toendnotes[C]{\smallbreak\pagebreak[2]}\Standort{CUL, Schnitzler, B 89, A 2.}
\physDesc{Brief, 1 Blatt, 1 Seite, 587 Zeichen
\newline{}Handschrift: schwarze Tinte, lateinische Kurrent
\newline{}Schnitzler: mit Bleistift datiert: »27/6 99« 
\newline{}Ordnung: mit Bleistift von unbekannter Hand nummeriert: »118« }\toendnotes[C]{\smallbreak}
\pstart
           \noindent{}{\pb}Lieber Freund, wegen des »\textcolor{green}{Liebesreigen}{}\ledrightnote{\textcolor{green}{Reigen. Zehn Dialoge}}« möchte ich so bald wie möglich mit Ihnen
                  \label{K_L03294-1v}\edtext{sprechen}{\lemma{\textnormal{\emph{sprechen}}}\Cendnote{\textnormal{wegen einer Veröffentlichung in der \emph{\textcolor{green}{Wiener Allgemeinen Montags-Zeitung}}, zu der es aber nicht
                  kam (siehe Felix Salten an Arthur Schnitzler, 21. 6. 1899)}}}\label{K_L03294-1h}. D\textsuperscript{r}{ }\textcolor{blue}{Szeps}{}\ledrightnote{\textcolor{blue}{Julius Szeps}} macht im \textcolor{brown}{Ministerium}{}\ledrightnote{{$\rightarrow$}\textcolor{brown}{Ministerium für Inneres}} Anstrengungen denselben
               durchzusetzen, und an eventuelle Aufregung im Leserkreis kehre ich mich nicht. Ich
               könnte Ihnen ein nicht unbeträchtliches Honorar dafür bieten, und glaube, wenn es
               durch Vermittlung des \textcolor{blue}{Minister}{}\ledrightnote{{$\rightarrow$}\textcolor{blue}{Ernest von Koerber}}s gelingt, die Sache durch die Censur zu drücken, wäre ein wichtiges
               Präjudiz geschaffen, das Ihnen auch für eine \label{K_L03294-2v}\edtext{Buchausgabe}{\lemma{\textnormal{\emph{Buchausgabe}}}\Cendnote{\textnormal{1900 ließ \textcolor{blue}{Schnitzler} einen Privatdruck des \emph{\textcolor{green}{Reigen}} mit einer Auflage von 200 Stück anfertigen. Erst 1903 erschien das \textcolor{green}{Stück} im \emph{\textcolor{brown}{Wiener Verlag}}.}}}\label{K_L03294-2h} sehr
               werthvoll sein könnte. Bitte, theilen Sie mir gleich nach Ihrer \label{K_L03294-3v}\edtext{Rückkunft}{\lemma{\textnormal{\emph{Rückkunft}}}\Cendnote{\textnormal{\textcolor{blue}{Schnitzler} kehrte am 28. 6. 1899 nach \textcolor{pink}{Wien} zurück. Siehe auch Felix Salten an Arthur Schnitzler, 21. 6. 1899.}}}\label{K_L03294-3h} mit, wann ich Sie sprechen
               kann.\pend
           
\pstart
           Herzlichst Ihr {\\[\baselineskip]}\spacefill\mbox{Salten}\pend
           \leftskip=0em{}\endnumbering\briefempfaengerindex{Schnitzler, Arthur@\textsc{Schnitzler, Arthur}!zzzSalten, Felix@\emph{von Felix Salten}!1899-06-271@{{[}27. 6. 1899{]}}|)be}\mylabel{h}  \normalsize

\doendnotes{C}
\bigskip
\vfill

\clearpage

\footnotesize

\lohead{\textsc{register}}

% Definiere theindex-Environment komplett neu ohne reledmac
\makeatletter
\renewenvironment{theindex}{%
  \section*{\indexname}%
  \setlength{\parindent}{0pt}%
  \setlength{\parskip}{0pt plus 0.3pt}%
  \let\item\@idxitem
}{%
  \clearpage
}
\makeatother

\IfFileExists{\jobname-pw.ind}{\input{\jobname-pw.ind}}{}

\end{document}

      