%% latex-korrekturansicht-vorspann.tex
%% Vorspann für die Korrekturansicht.
%% Lädt die gemeinsame Datei latex-vorspann.tex mit gesetztem Schalter.

\newif\ifkorrekturansicht
\korrekturansichttrue

\input{../tex-inputs/latex-vorspann}


               \section[Arthur Schnitzler an Hermann Bahr, 10. 11. 1903]{ Arthur Schnitzler an Hermann Bahr, 10. 11. 1903}\nopagebreak\mylabel{v}\rehead{ }\normalsize\beginnumbering\briefempfaengerindex{Bahr, Hermann@\textsc{Bahr, Hermann}!zzzSchnitzler, Arthur@\emph{von Arthur Schnitzler}!1903-11-101@{10. 11. 1903}|(be} \toendnotes[C]{\smallbreak\pagebreak[2]} \Standort{TMW, HS AM 23359 Ba.}
\physDesc{Brief, 2 Blätter, 6 Seiten
\newline{}Handschrift: schwarze Tinte, deutsche Kurrent\newline{}Ordnung: 1) Lochung 2) mit Bleistift von unbekannter Hand das zweite Blatt datiert »10. 11. 03« und mit »II« versehen}\buchAbdrucke{\weitereDrucke{1) Arthur Schnitzler: \emph{Briefe 1875–1912}. Hg. Therese Nickl und Heinrich Schnitzler. Frankfurt am Main: \emph{S. Fischer} 1981, S. 473–474.} \weitereDrucke{2) \emph{10. 11. 1903.} In: Arthur Schnitzler: \emph{The Letters of Arthur Schnitzler to Hermann Bahr}. Edited, annotated, and with an introduction, by Donald G.
                        Daviau. Chapel Hill: \emph{The University of North Carolina Press} 1978, S. 80–81 (University of North Carolina studies in the Germanic languages
                        and literatures, 89).} \weitereDrucke{3) Hermann Bahr, Arthur Schnitzler: \emph{Briefwechsel, Aufzeichnungen, Dokumente (1891–1931)}. Hg. Kurt Ifkovits und Martin Anton Müller. Göttingen: \emph{Wallstein} 2018, S. 278–279.} }\toendnotes[C]{\smallbreak}\pstart
           \raggedleft{}{\pb}\textcolor{pink}{Wien}{}\ledrightnote{\textcolor{pink}{Wien}}{ }10. 11. 903.\pend
           \pstart{}mein lieber Hermann,\pend\pstart
           ich danke dir herzlich, dſs du die \textsc{\textcolor{green}{Exc.}{}\ledrightnote{\textcolor{green}{Excentric}}} zu \damage{e}inem ſo ſchönen Erfolg gebracht ha{[}ſ{]}t u gratulire dir zu
               dem ganzen Abend. Ich war mit \textcolor{blue}{Olga}{}\ledrightnote{\textcolor{blue}{Olga Schnitzler}} auf d \textcolor{pink}{Semmering}{}\ledrightnote{\textcolor{pink}{Semmering}}; darum haben wir dich nicht um Karten
               gebeten. Ich ſelbſt wäre übrigens keineswegs \substVorne{}\textsuperscript{dort}\substDazwischen{}im \textcolor{pink}{Bös-Saal}{}\ledrightnote{\textcolor{pink}{Bösendorfer-Saal}}\substHinten{} geweſen – denn, du verſtehſt es gewiſs, ich kann mir eigene Sachen vor
               großem Publikum nicht vorleſen laſſen. –\pend
           \pstart
           Der Recurs iſt prachtvoll. Und ich würde ihn mit Freuden vor die nächſte {\pb}Auflage des \textcolor{green}{Reigen}{}\ledrightnote{\textcolor{green}{Reigen. Zehn Dialoge}} drucken laſſen – we{\geminationn} er nicht ſo viel Lob über mich enthielte. Man läßt
               ſich gerne an fremden Höfen mit ſchmetternden Trompetenſtößen empfangen – aber \substVorne{}\textsuperscript{ich}\substDazwischen{}man\substHinten{} ka{\geminationn}\substVorne{}\textsuperscript{m}\substDazwischen{}ſ\substHinten{}ich doch nicht im eigenen Hauſe feiern laſſen{\dotstwo}
               Doch wäre es zu ſchade, wenn dieſes Meiſterſtück der Oeffentlichkeit vorenthalten
               würde. Daſs ſich in \textcolor{pink}{Wien}{}\ledrightnote{\textcolor{pink}{Wien}} nichts würde anfangen
               laſſen, war vorauszuſetzen. Die Kerle ſind ja nicht mehr feig, weil ihnen even{\pb}tuell was geſchehen
               könnte – ſondern aus Liebe zur Sache. Wie wärs denn mit dem Ausland? \textcolor{brown}{Berliner Tageblatt}{}\ledrightnote{\textcolor{brown}{Berliner Tageblatt}} (oder \textcolor{brown}{Voſſiſche}{}\ledrightnote{\textcolor{brown}{Vossische Zeitung}}?) wären vielleicht zu gewinnen? Wenn kein Tagesblatt, eine Wochen
               oder Monatsſchrift? – Wie immer – ich danke dir und \textsc{\textcolor{blue}{Burckhard}{}\ledrightnote{\textcolor{blue}{Max Eugen Burckhard}}\strikeout{t}} vielmals und wärmſtens. Was iſt das übrigens für eine \label{K_L01338_1v}\edtext{Stelle im \textcolor{green}{\textsc{Lamprecht}}{}\ledrightnote{→\textcolor{green}{Die verbotene »Reigen«-Vorlesung}}}{\lemma{\textnormal{\emph{Stelle im Lamprecht}}}\Cendnote{\textnormal{Vgl. [O. V.:] \emph{\textcolor{green}{Die verbotene
                        »Reigen«-Vorlesung}}. In: \emph{\textcolor{green}{Die Zeit}},
                     Jg. 2, Nr. 396, 5. 11. 1903, S. 3: »In den weiteren
                     Darlegungen des Rekurses bespricht \textcolor{blue}{Bahr} die
                     literarische Persönlichkeit \textcolor{blue}{Artur
                        Schnitzlers}. Er führt an, daß \textcolor{blue}{Schnitzler} als \textcolor{pink}{österreichischer}
                     Dichter auch im Ausland stets an erster Stelle genannt werde, daß \textcolor{blue}{Schnitzler}’s Wirken vielfache Auszeichnungen
                     erhielt, daß der Historiker \textcolor{blue}{Lamprecht} über
                     den \textcolor{pink}{Wien}er in anerkennender Weise sich
                     ausgesprochen habe, [{\dots}]«. Das dürfte
                  wiederum auf die allgemeinen Ausführungen über \textcolor{blue}{Schnitzler} in \textcolor{blue}{Karl Lamprecht}s \emph{\textcolor{green}{Deutsche Geschichte. Erster Ergänzungsband}}
                     (Berlin: \emph{\textcolor{brown}{R. Gaertners
                        Verlagsbuchhandlung}}{ }1902, S. 362) Bezug nehmen.}}}\label{K_L01338_1h}, die durch die Blätter
               ging? Ich habe nichts geleſen.\pend
           \pstart
           \textcolor{blue}{Salten}{}\ledrightnote{\textcolor{blue}{Felix Salten}} thu ich gewiſs nicht Unrecht. {\pb}Lies nur – we{\geminationn} es ſo viel Intereſſe für dich hat, – \substVorne{}\textsuperscript{den}\substDazwischen{}meinen\substHinten{} ganzen \label{K_L01338_2v}\edtext{Brief an \textcolor{blue}{Salten}{}\ledrightnote{\textcolor{blue}{Felix Salten}}}{\lemma{\textnormal{\emph{Brief an Salten}}}\Cendnote{\textnormal{vom 7. 11. 1903, abgedruckt
                  in A. S.\emph{Briefe} I,468–470.}}}\label{K_L01338_2h}. Nicht um
               Lob und Tadel handelt es ſich. Das weſentliche für mich bleibt, daſs in dem \label{K_L01338_3v}\edtext{\textcolor{green}{Feuilleton}{}\ledrightnote{→\textcolor{green}{Arthur Schnitzler und sein Reigen}}}{\lemma{\textnormal{\emph{Feuilleton}}}\Cendnote{\textnormal{\textcolor{blue}{Felix Salten}: \emph{\textcolor{green}{Arthur Schnitzler und sein Reigen}}. In: \emph{\textcolor{green}{Die Zeit}}, Jg. 2, Nr. 398, 7. 11. 1903,
                  S. 1–2.}}}\label{K_L01338_3h} genau \uline{die}{ }Sachen \introOben{}zu meinen Ungunſten\introOben{}
               drinſtehen – über deren mangelnde Berechtigung ſich ſein Verfaſſer Dutzendemale mir
               gegenüber ausgeſprochen. Lies den Brief. – Und das ärgerliche – worüber wir auch ſo
               oft geſprochen haben – der Verſuch, einem Dichter Gebiete abzuſtecken – oder zu
               verwehren. Ich, als einziger Menſch auf der bewohnten Erde, ſoll nicht mehr {\pb}das Recht haben,
               erotiſche Beziehungen zu ſchildern, oder unverehelichte junge Damen darzuſtellen? –
               Es werden nach mir noch etwa hunderttauſend Bücher von Liebe und \textcolor{green}{Liebelei}{}\ledrightnote{→\textcolor{green}{Liebelei. Schauspiel in drei Akten}}, ſüßen und ſauren Mädeln, und \textcolor{green}{Anatolen und Mäxen}{}\ledrightnote{→\textcolor{green}{Anatol}} geſchrieben
               werden – wie ſie vor mir geſchrieben worden ſind. Und gerade ich beko{\geminationm} immer ſozuſagen einen Krach in den Schädel, wenn auch
               nur \substVorne{}\textsuperscript{ein}\substDazwischen{}aus\substHinten{} der Ferne ein Hauch von Erotik über meine Geſtalten weht? {\pb}Und der letzte Krach
               geht gerade von \textcolor{blue}{Salten}{}\ledrightnote{\textcolor{blue}{Felix Salten}} aus, mit dem
               gemeinſchaftlich ich mich über diese Kräche \introOben{}ſo oft\introOben{} beluſtigt
               und geärgert habe? – Aber laſſen wir das auf eventuelle mündliche Unterhaltung. – Ich
               darf dich wohl dieſer Tage wieder in \textcolor{pink}{St Veit}{}\ledrightnote{\textcolor{pink}{Ober Sankt Veit}}
               aufſuchen?\pend
           \pstart
           Herzlichſt dein getreuer{\\[\baselineskip]}\spacefill\mbox{Arthur.}\pend
           \leftskip=0em{}\endnumbering\briefempfaengerindex{Bahr, Hermann@\textsc{Bahr, Hermann}!zzzSchnitzler, Arthur@\emph{von Arthur Schnitzler}!1903-11-101@{10. 11. 1903}|)be}\mylabel{h}  \normalsize

\doendnotes{C}
\bigskip
\vfill

\clearpage

\footnotesize

\lohead{\textsc{register}}

% Definiere theindex-Environment komplett neu ohne reledmac
\makeatletter
\renewenvironment{theindex}{%
  \section*{\indexname}%
  \setlength{\parindent}{0pt}%
  \setlength{\parskip}{0pt plus 0.3pt}%
  \let\item\@idxitem
}{%
  \clearpage
}
\makeatother

\IfFileExists{\jobname-pw.ind}{\input{\jobname-pw.ind}}{}

\end{document}

      