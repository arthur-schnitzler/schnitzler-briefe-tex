%% latex-korrekturansicht-vorspann.tex
%% Vorspann für die Korrekturansicht.
%% Lädt die gemeinsame Datei latex-vorspann.tex mit gesetztem Schalter.

\newif\ifkorrekturansicht
\korrekturansichttrue

\input{../tex-inputs/latex-vorspann}


               \section[Olga Schnitzler an Paula Beer-Hofmann, 22. 4. 1913]{ Olga Schnitzler an Paula Beer-Hofmann, 22. 4. 1913}\nopagebreak\mylabel{v}\rehead{ }\normalsize\beginnumbering\briefempfaengerindex{Beer-Hofmann, Paula@\textsc{Beer-Hofmann, Paula}!zzzSchnitzler, Olga@\emph{von Olga Schnitzler}!1913-04-223@{22. 4. 1913}|(be} \toendnotes[C]{\smallbreak\pagebreak[2]} \Standort{YCGL, MSS 31.}
\physDesc{Bildpostkarte
\newline{}Handschrift: schwarze Tinte, lateinische Kurrent\newline{}Versand: Stempel: »\nobreak{}\oindex{Baden bei Wien@\textbf{Baden bei Wien}, \emph{Besiedelter Ort (A.BSO)}|pwk}Baden 1
                                          Nie\textcolor{gray}{d.Ö}s\textcolor{gray}{te}rr., 22 IV 13, 6\nobreak{}«.  }\pstart{}{\pb}Frau Paula Beer-Hofmann\pend{}\pstart{}\textcolor{pink}{Wien XIX}{}\ledrightnote{\textcolor{pink}{XIX., Döbling}}\pend{}\pstart{}\textcolor{pink}{Hasenauerstrasse 59}{}\ledrightnote{\textcolor{pink}{Hasenauerstraße}}.\pend{}{\bigskip}\pstart
           \noindent{}\centering{}{\pb}\textcolor{gray}{\textbf{\textcolor{pink}{Baden b. Wien}{}\ledrightnote{\textcolor{pink}{Baden bei Wien}} – Blick in das
                        \textcolor{pink}{Helenental}{}\ledrightnote{\textcolor{pink}{Helenental}}}}\pend
           \pstart
           \noindent{}\textcolor{pink}{Hôtel Grüner Baum}{}\ledrightnote{\textcolor{pink}{Grüner Baum}}.\pend
           \pstart
           {\pb}Liebe Paula, ich wollte während der letzten Tage immer zu Ihnen
               kommen und für den schönen Kamm danken, aber es ist mir so schäbig gegangen, dass ich
               plötzlich herausgefahren bin. Hier ist es schön und still und die Nerven benehmen
               sich auch besser.\hspace*{1.5em}Auf Wiedersehen, {\pb}und herzliche Grüsse Ihnen Allen!\pend
           \pstart Ihre \spacefill\mbox{Olga.}\pend{}\pstart
           22. April 13. \pend
           \endnumbering\briefempfaengerindex{Beer-Hofmann, Paula@\textsc{Beer-Hofmann, Paula}!zzzSchnitzler, Olga@\emph{von Olga Schnitzler}!1913-04-223@{22. 4. 1913}|)be}\mylabel{h}  \normalsize

\doendnotes{C}
\bigskip
\vfill

\clearpage

\footnotesize

\lohead{\textsc{register}}

% Definiere theindex-Environment komplett neu ohne reledmac
\makeatletter
\renewenvironment{theindex}{%
  \section*{\indexname}%
  \setlength{\parindent}{0pt}%
  \setlength{\parskip}{0pt plus 0.3pt}%
  \let\item\@idxitem
}{%
  \clearpage
}
\makeatother

\IfFileExists{\jobname-pw.ind}{\input{\jobname-pw.ind}}{}

\end{document}

      