%% latex-korrekturansicht-vorspann.tex
%% Vorspann für die Korrekturansicht.
%% Lädt die gemeinsame Datei latex-vorspann.tex mit gesetztem Schalter.

\newif\ifkorrekturansicht
\korrekturansichttrue

\input{../tex-inputs/latex-vorspann}


\renewcommand{\erwaehntePersonen}{Personen: Ferdinand von Fellner-Feldegg, Felix Salten}
\renewcommand{\erwaehnteInstitutionen}{Institutionen: Lessing-Theater}
\renewcommand{\erwaehnteOrte}{Orte: Armbrustergasse, Berlin, Edmund-Weiß-Gasse 7, Heiligenstadt, I., Innere Stadt, Raimund-Theater, Wien, XIX., Döbling, XVIII., Währing}
\renewcommand{\erwaehnteWerke}{Werke: Mit seinem Gotte allein. Volksschauspiel in 4 Aufzügen, Vom andern Ufer. Einakter}
\section[ Felix Salten an Arthur Schnitzler, 1. 10. 1907]{Felix Salten an Arthur Schnitzler, 1. 10. 1907}
\nopagebreak\mylabel{v}
\rehead{ }\normalsize\beginnumbering\briefempfaengerindex{Schnitzler, Arthur@\textsc{Schnitzler, Arthur}!zzzSalten, Felix@\emph{von Felix Salten}!1907-10-011@{1. 10. 1907}|(be}
\toendnotes[C]{\smallbreak\pagebreak[2]}\Standort{CUL, Schnitzler, B 89, B 1.}
\physDesc{Postkarte, 559 Zeichen
\newline{}Handschrift: schwarze Tinte, lateinische Kurrent
\newline{}Versand: Stempel: »\nobreak{}\oindex{I., Innere Stadt@\textbf{I., Innere Stadt}, \emph{A.ADM3}|pwk}1/\textsubscript{1} Wien \textcolor{gray}{5}, 2. X. 07, VII\nobreak{}«. Stempel: »\nobreak{}\oindex{XVIII., Waehring@\textbf{XVIII., Währing}, \emph{A.ADM3}|pwk}18/\textsubscript{1} Wien 110, 2. X. 07, XII\nobreak{}«.  
\newline{}Ordnung: mit Bleistift von unbekannter Hand nummeriert: »235« }\toendnotes[C]{\smallbreak}\pstart{}{\pb}Salten\pend{}\pstart{}\textcolor{pink}{Wien XIX.}{}\ledrightnote{\textcolor{pink}{XIX., Döbling}}\pend{}\pstart{}\textcolor{pink}{Armbrustergasse 6}{}\ledrightnote{\textcolor{pink}{Armbrustergasse}}\pend{}
{\bigskip}\pstart{}Herrn D\textsuperscript{r} Arthur Schnitzler\pend{}\pstart{}\textcolor{pink}{Wien XVIII.}{}\ledrightnote{\textcolor{pink}{XVIII., Währing}}\pend{}\pstart{}\textcolor{pink}{Spöttelgaße 7}{}\ledrightnote{\textcolor{pink}{Edmund-Weiß-Gasse 7}}.\pend{}
{\bigskip}
\pstart
           \raggedleft{}{\pb}\textcolor{pink}{Heiligenstadt}{}\ledrightnote{\textcolor{pink}{Heiligenstadt}}, 1. X. 07\pend
           
\pstart
           Lieber, es geht leider am Freitag
               nicht. Die \label{K_L03512-1v}\edtext{\textcolor{green}{Première}{}\ledrightnote{{$\rightarrow$}\textcolor{green}{Mit seinem Gotte allein. Volksschauspiel in 4 Aufzügen}}}{\lemma{\textnormal{\emph{Première}}}\Cendnote{\textnormal{Uraufführung von \emph{\textcolor{green}{Mit seinem Gotte allein. Volksschauspiel in 4 Aufzügen}}
                  von \textcolor{blue}{Ferdinand von Fellner-Feldegg}}}}\label{K_L03512-1h} im \textcolor{pink}{Raimund-Theater}{}\ledrightnote{\textcolor{pink}{Raimund-Theater}} ist vom Samstag auf den Freitag
               rückverlegt worden, und da muß ich eben hinein. Ich bin aber sehr wahrscheinlich noch
               in der nächsten Woche hier, denn ich höre – indirekt – dass ich in \textcolor{pink}{Berlin}{}\ledrightnote{\textcolor{pink}{Berlin}} erst am \label{K_L03512-2v}\edtext{19. Okt.{ }\textcolor{green}{drankomme}{}\ledrightnote{{$\rightarrow$}\textcolor{green}{Vom andern Ufer. Einakter}}}{\lemma{\textnormal{\emph{19. Okt. drankomme}}}\Cendnote{\textnormal{Die Uraufführung von \textcolor{blue}{Salten}s Einakterreihe \emph{\textcolor{green}{Vom
                     andern Ufer}} fand vier Tage früher, am 15. 10. 1907, am \emph{\textcolor{brown}{Lessing-Theater}}
                  statt.}}}\label{K_L03512-2h}, und erhalte wol morgen od. übermorgen eine direkte Verständigung. Wenn Ihnen der
                  \label{K_L03512-3v}\edtext{Sonntag}{\lemma{\textnormal{\emph{Sonntag}}}\Cendnote{\textnormal{siehe A. S.: \emph{Tagebuch}, 6. 10. 1907}}}\label{K_L03512-3h} nicht passt, machen wir vielleicht Freitag
               beim Tennis einen andern Tag aus.\pend
           
\pstart
           Herzlichst{\\[\baselineskip]} Ihr \spacefill\mbox{Salten}\pend
           \leftskip=0em{}\endnumbering\briefempfaengerindex{Schnitzler, Arthur@\textsc{Schnitzler, Arthur}!zzzSalten, Felix@\emph{von Felix Salten}!1907-10-011@{1. 10. 1907}|)be}\mylabel{h}  \normalsize

\doendnotes{C}
\bigskip
\vfill

\clearpage

\footnotesize

\lohead{\textsc{register}}

% Definiere theindex-Environment komplett neu ohne reledmac
\makeatletter
\renewenvironment{theindex}{%
  \section*{\indexname}%
  \setlength{\parindent}{0pt}%
  \setlength{\parskip}{0pt plus 0.3pt}%
  \let\item\@idxitem
}{%
  \clearpage
}
\makeatother

\IfFileExists{\jobname-pw.ind}{\input{\jobname-pw.ind}}{}

\end{document}

      