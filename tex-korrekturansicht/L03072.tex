%% latex-korrekturansicht-vorspann.tex
%% Vorspann für die Korrekturansicht.
%% Lädt die gemeinsame Datei latex-vorspann.tex mit gesetztem Schalter.

\newif\ifkorrekturansicht
\korrekturansichttrue

\input{../tex-inputs/latex-vorspann}


\renewcommand{\erwaehntePersonen}{Personen: Heinrich Kanner, Alfred Kerr, Olga Schnitzler, Isidor Singer, Elisabeth Steinrück}
\renewcommand{\erwaehnteInstitutionen}{Institutionen: Die Zeit, Die Zeit. Wiener Wochenschrift, Neue Freie Presse}
\renewcommand{\erwaehnteOrte}{Orte: Berlin, Dessauer Straße, Dresden, St. Anton am Arlberg, Urtijëi, Val Gardena, Wien, Wörthersee}
\renewcommand{\erwaehnteWerke}{Werke: Die Zeit, Die Zeit. Wiener Wochenschrift, Neue Freie Presse}
\section[ Paul Goldmann an Arthur Schnitzler und Olga Gussmann, 7. 7. {[}1901{]}]{Paul Goldmann an Arthur Schnitzler und Olga
               Gussmann, 7. 7. {[}1901{]}}
\nopagebreak\mylabel{v}
\rehead{ }\normalsize\beginnumbering\briefempfaengerindex{Schnitzler, Olga@\textsc{Schnitzler, Olga}!zzzGoldmann, Paul@\emph{von Paul Goldmann}!1901-07-071@{7. 7. {[}1901{]}}|(be}\briefempfaengerindex{Schnitzler, Arthur@\textsc{Schnitzler, Arthur}!zzzGoldmann, Paul@\emph{von Paul Goldmann}!1901-07-071@{7. 7. {[}1901{]}}|(be}
\toendnotes[C]{\smallbreak\pagebreak[2]}\Standort{DLA, A:Schnitzler, HS.NZ85.1.3171.}
\physDesc{Brief, 1 Blatt, 4 Seiten
\newline{}Handschrift: blaue Tinte, deutsche Kurrent
\newline{}Schnitzler: mit rotem Buntstift zwei Unterstreichungen }\toendnotes[C]{\smallbreak}
\pstart
           \noindent{}\raggedleft{}{\pb}\textcolor{pink}{\textcolor{gray}{\textbf{DESSAUERSTRASSE 19}}}{}\ledrightnote{\textcolor{pink}{Dessauer Straße}}\pend
           
\pstart
           \textcolor{pink}{Berlin}{}\ledrightnote{\textcolor{pink}{Berlin}}, 7. Juli.\pend
           
\pstart\center{}Mein lieber Freund,\pend
\pstart
           Endlich zieht Vernunft in Eure Reiſepläne ein, und ich freue mich ſehr darüber und
               über die Ausſicht, Euch doch zu \label{K_L03072-1v}\edtext{ſehen}{\lemma{\textnormal{\emph{ſehen}}}\Cendnote{\textnormal{siehe Paul Goldmann an Arthur Schnitzler, 26. 4. [1901]}}}\label{K_L03072-1h}. Ich gehe ſo zwiſchen dem 20. u. 25. von hier fort, bleibe einen oder zwei Tage in \textcolor{pink}{Dresden}{}\ledrightnote{\textcolor{pink}{Dresden}} und \textcolor{pink}{Wien}{}\ledrightnote{\textcolor{pink}{Wien}}, gehe dann meinetwegen nach dem \label{K_L03072-2v}\edtext{\textcolor{pink}{Wörtherſee}{}\ledrightnote{\textcolor{pink}{Wörthersee}}}{\lemma{\textnormal{\emph{Wörtherſee}}}\Cendnote{\textnormal{siehe Paul Goldmann an Arthur Schnitzler, 13. 5. [1901]}}}\label{K_L03072-2h} und komme von da aus ſehr gern zu Euch. 
               \textcolor{pink}{\textsc{St. Ulrich}}{}\ledrightnote{\textcolor{pink}{Urtijëi}}
                im {\pb}\textcolor{pink}{Grödener Thal}{}\ledrightnote{\textcolor{pink}{Val Gardena}} würde mir beſonders gefallen.
               Denn ſeit Jahren wünſche ich, das \textcolor{pink}{Grödener Thal}{}\ledrightnote{\textcolor{pink}{Val Gardena}}
               kennen zu lernen. Bitte, halt’ alſo dieſes Projekt feſt. Vielleicht können wir dann
               auch von dort aus ein paar Tage in die Berge ſteigen.\pend
           
\pstart
           Ich höre, daß die »\textcolor{brown}{Zeit}{}\ledrightnote{\textcolor{brown}{Die Zeit}{\newline}\textcolor{brown}{Die Zeit. Wiener Wochenschrift}}« von 1. Oktober ab \label{K_L03072-3v}\edtext{Tagesblatt}{\lemma{\textnormal{\emph{Tagesblatt}}}\Cendnote{\textnormal{\emph{\textcolor{green}{Die Zeit}} wurde erst ab dem 27. 9. 1902 (bis 31. 8. 1919) als Tageszeitung von \textcolor{blue}{Heinrich Kanner} und \textcolor{blue}{Isidor Singer}
                  herausgegeben. Bis zum 29. 10. 1904 erschien \emph{\textcolor{green}{Die Zeit}} parallel als Wochenschrift. Die \emph{\textcolor{green}{Neue Freie Presse}} ersetzte sie nicht.}}}\label{K_L03072-3h}
               wird mit 1 Million \textsc{Kronen} Capital. Weißt Du etwas davon?
               Kommt es dazu, ſo bedeutet {\pb}das, nach meiner
               Überzeugung, den Anfang vom Ende der \textcolor{brown}{N. F. Pr.}{}\ledrightnote{\textcolor{brown}{Neue Freie Presse}} So
               ſetzt auch \textsc{Dr. \textcolor{blue}{Kanner}{}\ledrightnote{\textcolor{blue}{Heinrich Kanner}}} ſeinen Lebensplan durch. Nur ich, – ich allein bleibe auf der Strecke. Es iſt
               martervoll!\pend
           
\pstart
           Viele treue Grüße! {\\[\baselineskip]}Dein {\\[\baselineskip]}\spacefill\mbox{Paul Goldmann.}\pend
           \leftskip=0em{}{\bigskip}
\pstart
           \noindent{}Liebes Fräulein \textsc{Olga}, Ich
               danke Ihnen für Ihren lieben und guten Brief. Jetzt, bitte, ſetzen Sie noch durch,
               daß wir ins \textcolor{pink}{Grödener Thal}{}\ledrightnote{\textcolor{pink}{Val Gardena}}{ }{\pb}gehen. Ich möchte ſehr gern dorthin, was für \textsc{Arthur} immerhin einen ausreichenden Grund bilden könnte,
                  \strikeout{\textcolor{gray}{in}d\textcolor{gray}{e}s} ſich für einen anderen Ort zu
               entſchließen. Auch ich möchte, gleich Ihnen, ſtillſitzen und Ruhe, Ruhe haben. Über
                  \textsc{\textcolor{blue}{Kerr}{}\ledrightnote{\textcolor{blue}{Alfred Kerr}}} ſprechen wir mündlich. Er wird übrigens nur nachkommen und nicht mitkommen
               können. Ihrem lieben \textcolor{blue}{Schweſterchen}{}\ledrightnote{{$\rightarrow$}\textcolor{blue}{Elisabeth Steinrück}} wünſche ich gute Beſſerung. Haben Sie keine Sorgen! Wenn ſie
                  \textsc{Arthurs} Behandlung bisher ausgehalten hat, wird ſie auch
               davonkommen. Sie iſt eine widerſtandsfähige Natur.\pend
           
\pstart
           Herzlichſt Ihr {\\[\baselineskip]}\spacefill\mbox{Dr. Paul Goldmann.}\pend
           \leftskip=0em{}\endnumbering\briefempfaengerindex{Schnitzler, Olga@\textsc{Schnitzler, Olga}!zzzGoldmann, Paul@\emph{von Paul Goldmann}!1901-07-071@{7. 7. {[}1901{]}}|)be}\briefempfaengerindex{Schnitzler, Arthur@\textsc{Schnitzler, Arthur}!zzzGoldmann, Paul@\emph{von Paul Goldmann}!1901-07-071@{7. 7. {[}1901{]}}|)be}\mylabel{h}
\begin{anhang}
\end{anhang}\normalsize

\doendnotes{C}
\bigskip
\vfill

\clearpage

\footnotesize

\lohead{\textsc{register}}

% Definiere theindex-Environment komplett neu ohne reledmac
\makeatletter
\renewenvironment{theindex}{%
  \section*{\indexname}%
  \setlength{\parindent}{0pt}%
  \setlength{\parskip}{0pt plus 0.3pt}%
  \let\item\@idxitem
}{%
  \clearpage
}
\makeatother

\IfFileExists{\jobname-pw.ind}{\input{\jobname-pw.ind}}{}

\end{document}

      