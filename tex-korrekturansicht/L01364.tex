%% latex-korrekturansicht-vorspann.tex
%% Vorspann für die Korrekturansicht.
%% Lädt die gemeinsame Datei latex-vorspann.tex mit gesetztem Schalter.

\newif\ifkorrekturansicht
\korrekturansichttrue

\input{../tex-inputs/latex-vorspann}


               \section[Hugo von Hofmannsthal an Arthur Schnitzler, 26. 1. 1904]{ Hugo von Hofmannsthal an Arthur Schnitzler, 26. 1. 1904}\nopagebreak\mylabel{v}\rehead{ }\normalsize\beginnumbering\briefempfaengerindex{Schnitzler, Arthur@\textsc{Schnitzler, Arthur}!zzzHofmannsthal, Hugo von@\emph{von Hugo von Hofmannsthal}!1904-01-261@{26. 1. 1904}|(be} \toendnotes[C]{\smallbreak\pagebreak[2]} \Standort{CUL, Schnitzler, B 43.}
\physDesc{Bildpostkarte
\newline{}Handschrift: schwarze Tinte, deutsche Kurrent\newline{}Versand: 1) Stempel: »\nobreak{}\oindex{Bahnhof@\textbf{Bahnhof}, \emph{Bahnhofsgebäude (K.BHF)}|pwk}Venezia Ferrovia, 27{[}-1{]}-04, 8M\nobreak{}«.  2) Stempel: »\nobreak{}\oindex{XVIII., Waehring@\textbf{XVIII., Währing}, \emph{Bezirk (A.BZK)}|pwk}18/1 Wien, 28. 1. 04, 12.V, Bestellt\nobreak{}«. \newline{}Ordnung: mit Bleistift von unbekannter Hand nummeriert:
                                    »212« }\buchAbdrucke{\weitereDrucke{Hugo von Hofmannsthal, Arthur Schnitzler: \emph{Briefwechsel}. Hg. Therese Nickl und Heinrich Schnitzler. Frankfurt am Main: \emph{S. Fischer} 1964, S. 182.} }\toendnotes[C]{\smallbreak}\pstart{}{\pb}\textsc{Herrn D\textsuperscript{r} Arthur Schnitzler}\pend{}\pstart{}\textcolor{pink}{\textsc{Wien}}{}\ledrightnote{\textcolor{pink}{Wien}}\pend{}\pstart{}\textcolor{pink}{\textsc{XVIII Spöttelgasse 7}}{}\ledrightnote{\textcolor{pink}{Edmund-Weiß-Gasse}}\pend{}\pstart{}\textsc{\textcolor{pink}{Austria}{}\ledrightnote{\textcolor{pink}{Österreich}}}\pend{}{\bigskip}\pstart
           \noindent{}\centering{}\textcolor{gray}{\textbf{{\pb}\textcolor{brown}{Venezia – R. Accademia di Belle Arti}{}\ledrightnote{\textcolor{brown}{Accademia di belle arti di Venezia}}}}\pend
           \pstart
           \noindent{}\centering{}\textcolor{gray}{\textbf{\textcolor{green}{L’Arrivo nel Porto di Colonia della nave che
                        conduceva S. Orsola e le Vergini}{}\ledrightnote{\textcolor{green}{Die Ankuft der Pilger in Köln}} (\textcolor{blue}{Carpaccio}{}\ledrightnote{\textcolor{blue}{Vittore Carpaccio}})}}\pend
           \pstart
           \raggedleft{}26. I.\pend
           \pstart
           Hier iſt es ſchön ſtill und i{\geminationm}erfort Sonne. — S. 128 im
                  »\textcolor{green}{einſ. Weg}{}\ledrightnote{\textcolor{green}{Der einsame Weg. Schauspiel in fünf Akten}}« (ein ſchönes Stück!) ſteht noch
               immer die Stelle die überflüſſig an \textcolor{green}{Baumeiſter \textsc{Solness}}{}\ledrightnote{\textcolor{green}{Baumeister Solness}}{ }\label{K_L01364_1v}\edtext{erinnert}{\lemma{\textnormal{\emph{erinnert}}}\Cendnote{\textnormal{In der Erstausgabe von \emph{\textcolor{green}{Der einsame
                     Weg}} (Berlin: \emph{\textcolor{brown}{S. Fischer}}{ }1904) steht auf S. 128: »Dann bist Du vielleicht eine Prinzessin
                     geworden und ich Fürst einer versunkenen Stadt«. Das alludiert an ein mit
                  »Prinzessin« angesprochenes Mädchen, dem vom \textcolor{green}{Baumeister Solness} ein Königreich versprochen
                  wird.}}}\label{K_L01364_1h}.\pend
           \pstart
           Grüße{\\[\baselineskip]}\spacefill\mbox{Hugo.}\pend
           \leftskip=0em{}\endnumbering\briefempfaengerindex{Schnitzler, Arthur@\textsc{Schnitzler, Arthur}!zzzHofmannsthal, Hugo von@\emph{von Hugo von Hofmannsthal}!1904-01-261@{26. 1. 1904}|)be}\mylabel{h}  \normalsize

\doendnotes{C}
\bigskip
\vfill

\clearpage

\footnotesize

\lohead{\textsc{register}}

% Definiere theindex-Environment komplett neu ohne reledmac
\makeatletter
\renewenvironment{theindex}{%
  \section*{\indexname}%
  \setlength{\parindent}{0pt}%
  \setlength{\parskip}{0pt plus 0.3pt}%
  \let\item\@idxitem
}{%
  \clearpage
}
\makeatother

\IfFileExists{\jobname-pw.ind}{\input{\jobname-pw.ind}}{}

\end{document}

      