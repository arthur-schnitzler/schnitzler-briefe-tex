%% latex-korrekturansicht-vorspann.tex
%% Vorspann für die Korrekturansicht.
%% Lädt die gemeinsame Datei latex-vorspann.tex mit gesetztem Schalter.

\newif\ifkorrekturansicht
\korrekturansichttrue

\input{../tex-inputs/latex-vorspann}


               \section[Richard Beer-Hofmann an Arthur Schnitzler, {[}24. 12. 1902{]}]{ Richard Beer-Hofmann an Arthur Schnitzler, {[}24. 12. 1902{]}}\nopagebreak\mylabel{v}\rehead{ }\normalsize\beginnumbering\briefempfaengerindex{Schnitzler, Arthur@\textsc{Schnitzler, Arthur}!zzzBeer-Hofmann, Richard@\emph{von Richard Beer-Hofmann}!1902-12-241@{{[}24. 12. 1902{]}}|(be} \toendnotes[C]{\smallbreak\pagebreak[2]} \Standort{CUL, Schnitzler, B 8.}
\physDesc{Visitenkarte mit Trauerrand
\newline{}Handschrift: schwarze Tinte, lateinische Kurrent\newline{}Ordnung: mit Bleistift von unbekannter Hand nummeriert:
                                    »179a« }\buchAbdrucke{\weitereDrucke{Arthur Schnitzler, Richard Beer-Hofmann: \emph{Briefwechsel 1891–1931}. Hg. Konstanze Fliedl. Wien, Zürich: \emph{Europaverlag} 1992, S. 159.} }\toendnotes[C]{\smallbreak}\pstart
           \noindent{}{\pb}Anstatt \label{K_L01259_1v}\edtext{der aus \textcolor{pink}{Pontresina}{}\ledrightnote{\textcolor{pink}{Pontresina}}}{\lemma{\textnormal{\emph{der aus Pontresina}}}\Cendnote{\textnormal{Es handelt sich möglicherweise um eine
                  beim gemeinsamen Aufenthalt 1900 gekaufte Uhr, die durch das
                  Weihnachtsgeschenk ersetzt werden sollte.}}}\label{K_L01259_1h} zu tragen.\pend
           \pstart Von Herzen Ihr\pend{}\pstart
           \centering{}{\pb}\textcolor{gray}{\textbf{\textsc{Richard \strikeout{\label{T_L01259-1v}\edtext{Beer-Hofmann}{\lemma{\textnormal{\emph{Beer-Hofmann}}}\Cendnote{\textnormal{Streichung mit Tinte}}}\label{T_L01259-1h}}}}}\pend
           \pstart
           \noindent{}\raggedleft{}\textcolor{gray}{\textbf{\textcolor{pink}{RODAUN}{}\ledrightnote{\textcolor{pink}{Rodaun}}, bei \textcolor{pink}{Wien}{}\ledrightnote{\textcolor{pink}{Wien}}.}}\pend
           \endnumbering\briefempfaengerindex{Schnitzler, Arthur@\textsc{Schnitzler, Arthur}!zzzBeer-Hofmann, Richard@\emph{von Richard Beer-Hofmann}!1902-12-241@{{[}24. 12. 1902{]}}|)be}\mylabel{h}  \normalsize

\doendnotes{C}
\bigskip
\vfill

\clearpage

\footnotesize

\lohead{\textsc{register}}

% Definiere theindex-Environment komplett neu ohne reledmac
\makeatletter
\renewenvironment{theindex}{%
  \section*{\indexname}%
  \setlength{\parindent}{0pt}%
  \setlength{\parskip}{0pt plus 0.3pt}%
  \let\item\@idxitem
}{%
  \clearpage
}
\makeatother

\IfFileExists{\jobname-pw.ind}{\input{\jobname-pw.ind}}{}

\end{document}

      