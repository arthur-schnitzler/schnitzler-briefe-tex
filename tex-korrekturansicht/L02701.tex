%% latex-korrekturansicht-vorspann.tex
%% Vorspann für die Korrekturansicht.
%% Lädt die gemeinsame Datei latex-vorspann.tex mit gesetztem Schalter.

\newif\ifkorrekturansicht
\korrekturansichttrue

\input{../tex-inputs/latex-vorspann}


               \section[Paul Goldmann an Arthur Schnitzler, 5. 8. {[}1892{]}]{ Paul Goldmann an Arthur Schnitzler, 5. 8. {[}1892{]}}\nopagebreak\mylabel{v}\rehead{ }\normalsize\beginnumbering\briefempfaengerindex{Schnitzler, Arthur@\textsc{Schnitzler, Arthur}!zzzGoldmann, Paul@\emph{von Paul Goldmann}!1892-08-151@{5. 8. {[}1892{]}}|(be} \toendnotes[C]{\smallbreak\pagebreak[2]} \Standort{DLA, A:Schnitzler, HS.NZ85.1.3163.}
\physDesc{Brief, 2 Blätter, 8 Seiten
\newline{}Handschrift: schwarze Tinte, deutsche Kurrent
\newline{}Schnitzler: mit Bleistift das Jahr »92« vermerkt }\toendnotes[C]{\smallbreak}\pstart
           \noindent{}{\pb}\textcolor{brown}{\textcolor{gray}{\textbf{CASINO}}}{}\ledrightnote{\textcolor{brown}{Casino de Biarritz}}\pend
           \pstart
           \textcolor{brown}{\textcolor{gray}{\textbf{DE}}}{}\ledrightnote{\textcolor{brown}{Casino de Biarritz}}\pend
           \pstart
           \textcolor{brown}{\textcolor{gray}{\textbf{BIARRITZ}}}{}\ledrightnote{\textcolor{brown}{Casino de Biarritz}}\hfill 5. Auguſt. \pend
           \pstart\center{}Mein lieber Arthur!\pend\pstart
           Im Abreiſefieber mußte ich Deinen letzten lieben Brief unbeantwortet laſſen. Erſt
                  heut finde ich die nöthige Zeit und Ruhe zu einer
               Zeile Antwort. Da ſitze ich in halber Schlaftrunkenheit und reibe mir die Augen. Das
               blaue, blaue Meer blinkt zum Fenſter hinein und rauſcht mir in die Ohren (\textcolor{pink}{Atlantiſche\substVorne{}\textsuperscript{s }\substDazwischen{}r\substHinten{}{ }{\pb}Ocean}{}\ledrightnote{\textcolor{pink}{Atlantischer Ozean}}, mein lieber Arthur, \textsc{\textcolor{pink}{Golf von Gascogne}{}\ledrightnote{\textcolor{pink}{Biskaya}}}.) Und ich frage mich: wie \strikeout{ka} komme ich hierher
               in den blauen, blauen Süden, \strikeout{und} an die Grenzmarke
               von \textcolor{pink}{Frankreich}{}\ledrightnote{\textcolor{pink}{Frankreich}} und \textcolor{pink}{Spanien}{}\ledrightnote{\textcolor{pink}{Spanien}}{ }\strikeout{(Su} (Südweſtgrenze, mein lieber Arthur) – ich, der
               ich \label{K_L02701-44v}\edtext{geſtern}{\lemma{\textnormal{\emph{geſtern}}}\Cendnote{\textnormal{im übertragenen Sinn von »es kommt mir vor, als wäre es gestern gewesen« gemeint}}}\label{K_L02701-44h} noch im \textsc{Café \textcolor{pink}{Pfob}{}\ledrightnote{\textcolor{pink}{Café Pfob}}} ſaß und die bekannte \label{K_L02701-1v}\edtext{\textsc{Café}haus-Ecke}{\lemma{\textnormal{\emph{Caféhaus-Ecke}}}\Cendnote{\textnormal{Vgl. \textcolor{blue}{Schnitzler}s Texte \emph{\textcolor{green}{Aus der Kaffeehausecke}} und
                     \emph{\textcolor{green}{Gespräch, welches in der Kaffeehausecke nach
                     Vorlesung der »Elixiere« geführt wird}}. Dass \textcolor{blue}{Goldmann} ebenso den Begriff
                     »Caféhaus-Ecke« benutzte, deutet darauf, dass der Begriff allgemein im
                  Freundeskreis verwendet wurde.}}}\label{K_L02701-1h} mit Aphorismen austapezierte. Und da
               willſt Du noch Lachen über \label{K_L02701-5v}\edtext{»die
                  Fäden«}{\lemma{\textnormal{\emph{»die
                  Fäden«}}}\Cendnote{\textnormal{Möglicherweise schließt hier
                  \textcolor{blue}{Goldmann} an bestimmte Aussagen von \textcolor{blue}{Schnitzler} an. 
                     In seinem \emph{\textcolor{green}{Tagebuch}} schreibt dieser mehrfach von
                     »Fäden«, die ihn mit der Welt und die Welt an sich
                  verknüpfen.}}}\label{K_L02701-5h}?\pend
           \pstart
           Das iſt wunderbar, all’ das. Aber Du {\pb}weißt, daß das
               Wunderbare nicht das Glückliche iſt. Und meine Reiſe, die objectiv wunderſchön iſt,
               iſt es ſubjectiv um ſo weniger. Schlaftrunken laſſe ich mich durch die Welt
               ſchleppen. Und mitten in der himmliſchen Herrlichkeit des Südens ſchwirrt mir der
               Fledermausſchwarm meiner Sorgen unaufhörlich um das Haupt, Tag und Nacht, Tag und
               Nacht. Das Glück? Überall, wo ich hinkomme: »Eine Empfehlung, {\pb}und es iſt geſtern dageweſen«. Ich habe nur ein
               nervöſes Bedürfniß nach \label{K_L02701-2v}\edtext{\textsc{Locomotion}}{\lemma{\textnormal{\emph{Locomotion}}}\Cendnote{\textnormal{Fortbewegung}}}\label{K_L02701-2h} in mir, halte es
               nirgends aus und habe ſtets eine Stimme in mir, die mir ſagt: »Dort drüben iſt es
               ſchöner.« Und ſo geht es weiter und weiter: übermorgen
               nach \textsc{\textcolor{pink}{San Sebastian}{}\ledrightnote{\textcolor{pink}{San Sebastian}}} (Nord\textcolor{pink}{ſpanien}{}\ledrightnote{\textcolor{pink}{Spanien}}, mein lieber Arthur), dann
               nach den \textcolor{pink}{Pyrenäen}{}\ledrightnote{\textcolor{pink}{Pyrenees}}, dann wieder heim. Überall
               unterwegs bin natürlich {\pb}bitterlich allein. Kein
               Menſch zu finden in dieſem verdammten Lande. Mit dem deutſchen Accent ſcheucht man
               die Leute von ſich fort, \strikeout{als} und man ſitzt im \label{K_L02701-3v}\edtext{\textsc{Coupé}}{\lemma{\textnormal{\emph{Coupé}}}\Cendnote{\textnormal{Zugabteil}}}\label{K_L02701-3h} und im
                  \label{T_L02701-3v}\edtext{W\textcolor{gray}{i}thshaus}{\lemma{\textnormal{\emph{Withshaus}}}\Cendnote{\textnormal{ein deutlicher u-Strich macht den
                  Vokal der ersten Silbe zu einem »u«, doch dürfte es sich um ein Versehen gehandelt haben.}}}\label{T_L02701-3h} ſo gemieden, als wäre man der Scharfrichter der zu
               einer Hinrichtung fährt{\dotsfour}\pend
           \pstart
           Mein \textcolor{blue}{Onkel}{}\ledrightnote{→\textcolor{blue}{Fedor Mamroth}} iſt in \textsc{\textcolor{pink}{Salzburg}{}\ledrightnote{\textcolor{pink}{Salzburg}}}{ }\textsc{\textcolor{pink}{(Faberhaus}{}\ledrightnote{\textcolor{pink}{Faberhäuser}}}). {\pb} Wenn Du ihn einmal über den Sonntag
               beſuchen könnteſt, möcht’ er ſich rieſig mit Dir freuen. Bitte, ſahr’ doch einmal
               hinüber. Ich weiß Euch zwei Gerne zuſammen, die Ihr mir die theuerſten \textcolor{blue}{Freunde}{}\ledrightnote{→\textcolor{blue}{Fedor Mamroth}} ſind. Du kannſt all’
               Deine literariſchen Angelegenheiten mit ihm beſprechen, und beſſeren ſachverſtändigen
               Rath kannſts Du Dir {\pb}nicht wünſchen. Mußt Dich aber
               vorher anmelden, damit er nicht etwa auf Ausflug iſt{\dotsfour}\pend
           \pstart
           Dich im September wiederſehen? Schönſte aller Ausſichten!
               Aber glaubſt Du, ich \label{K_L02701-4v}\edtext{glaub’s?}{\lemma{\textnormal{\emph{glaub’s?}}}\Cendnote{\textnormal{Goldmann lag hatte mit seiner Vermutung
                  wohl recht. Es ist kein Treffen zwischen Goldmann und Schnitzler im September 1892 bekannt.}}}\label{K_L02701-4h}{ }{\dotsfour}\pend
           \pstart
           Bitte, ſei brav’ und ſchreib’ mir eine Zeile nach \textsc{\textcolor{pink}{Pau}{}\ledrightnote{\textcolor{pink}{Pau}}}, \textsc{\textcolor{pink}{Pyrénées}{}\ledrightnote{\textcolor{pink}{Pyrenees}}}, \textsc{Poste restante}, wo ich Mittwoch einzutreffen gedenke. Erhältſt Du {\pb}meinen Brief zu ſpät, ſo ſchreib mir, bitte, nach
                  \textsc{\textcolor{pink}{Cauterets}{}\ledrightnote{\textcolor{pink}{Cauterets}}}, \textsc{\strikeout{\textcolor{pink}{Pyree}{}\ledrightnote{\textcolor{pink}{Pyrenees}}}}{ }\textsc{\textcolor{pink}{Pyrénées}{}\ledrightnote{\textcolor{pink}{Pyrenees}}}, \textsc{Post restante}.\pend
           \pstart
           Und, was wird aus \textsc{\textcolor{blue}{Richard}{}\ledrightnote{\textcolor{blue}{Richard Beer-Hofmann}}}? Keine Seite von ihm ſeit dreiviertel Jahren!\pend
           \pstart
           Ich umarme {\\[\baselineskip]}Dich herzlichſt! {\\[\baselineskip]}Dein {\\[\baselineskip]}treuer {\\[\baselineskip]}\spacefill\mbox{Paul Goldmann.}\pend
           \leftskip=0em{}\endnumbering\briefempfaengerindex{Schnitzler, Arthur@\textsc{Schnitzler, Arthur}!zzzGoldmann, Paul@\emph{von Paul Goldmann}!1892-08-151@{5. 8. {[}1892{]}}|)be}\mylabel{h}\begin{anhang}\end{anhang}\normalsize

\doendnotes{C}
\bigskip
\vfill

\clearpage

\footnotesize

\lohead{\textsc{register}}

% Definiere theindex-Environment komplett neu ohne reledmac
\makeatletter
\renewenvironment{theindex}{%
  \section*{\indexname}%
  \setlength{\parindent}{0pt}%
  \setlength{\parskip}{0pt plus 0.3pt}%
  \let\item\@idxitem
}{%
  \clearpage
}
\makeatother

\IfFileExists{\jobname-pw.ind}{\input{\jobname-pw.ind}}{}

\end{document}

      