%% latex-korrekturansicht-vorspann.tex
%% Vorspann für die Korrekturansicht.
%% Lädt die gemeinsame Datei latex-vorspann.tex mit gesetztem Schalter.

\newif\ifkorrekturansicht
\korrekturansichttrue

\input{../tex-inputs/latex-vorspann}


               \section[Arthur Schnitzler an Richard Beer-Hofmann, 11. 8. 1891]{ Arthur Schnitzler an Richard Beer-Hofmann, 11. 8. 1891}\nopagebreak\mylabel{v}\rehead{ }\normalsize\beginnumbering\briefempfaengerindex{Beer-Hofmann, Richard@\textsc{Beer-Hofmann, Richard}!zzzSchnitzler, Arthur@\emph{von Arthur Schnitzler}!1891-08-112@{11. 8. 1891}|(be} \toendnotes[C]{\smallbreak\pagebreak[2]} \Standort{YCGL, MSS 31.}
\physDesc{Brief, 1 Blatt (Briefpapier mit Trauerrand), 2 Seiten, Umschlag
\newline{}Handschrift: schwarze Tinte, deutsche Kurrent\newline{}Versand: 1) Stempel: »\nobreak{}Wien, 11 \textcolor{gray}{8} {[}1891{]}, 4.N\nobreak{}«.  2) Stempel: »\nobreak{}\oindex{Bad Aussee@\textbf{Bad Aussee}, \emph{Besiedelter Ort (A.BSO)}|pwk}{\pb}Aussee in
                                          Stei\textcolor{gray}{ermark}, 12. {[}8.{]} 9\textcolor{gray}{1}\nobreak{}«. }\buchAbdrucke{\weitereDrucke{Arthur Schnitzler, Richard Beer-Hofmann: \emph{Briefwechsel 1891–1931}. Hg. Konstanze Fliedl. Wien, Zürich: \emph{Europaverlag} 1992, S. 31.} }\toendnotes[C]{\smallbreak}\pstart{}{\pb}\textsc{Herrn Dr. Rich. Beer-Hofmann}\pend{}\pstart{}\textcolor{pink}{\textsc{Aussee}}{}\ledrightnote{\textcolor{pink}{Bad Aussee}}\pend{}\pstart{}\textcolor{pink}{\textsc{Steiermark}}{}\ledrightnote{\textcolor{pink}{Steiermark}}\pend{}{\bigskip}\pstart
           \raggedleft{}{\pb}11. Aug 91.\pend
           \pstart
           Daß Sie mir noch nicht eine Zeile geſchrieben haben – na reden wir nicht drüber!
               Alſo, mein lieber, ich bin \uline{wahrscheinlich} die \label{K_L00029_1v}\edtext{2 Feiertage}{\lemma{\textnormal{\emph{2 Feiertage}}}\Cendnote{\textnormal{Der
                     15. 8. 1891 – Mariä Himmelfahrt –, war ein Samstag.
                     Dienstag, der 18. 8. war Geburtstag des Kaisers \textcolor{blue}{Franz Joseph}.}}}\label{K_L00029_1h} in \textcolor{pink}{Iſchl}{}\ledrightnote{\textcolor{pink}{Bad Ischl}}. Es wäre wunderſchön, we{\geminationn} wir uns da
               begegneten. Ich habe auch an \textcolor{blue}{\textsc{Loris}}{}\ledrightnote{\textcolor{blue}{Hugo von Hofmannsthal}} nach {\pb}\textcolor{pink}{\textsc{Strobl}}{}\ledrightnote{\textcolor{pink}{Strobl}} geſchrieben. Theilen Sie mir nur mit, ob Sie überhaupt zu erreichen ſind, ob
               Sie nach \textcolor{pink}{Iſchl}{}\ledrightnote{\textcolor{pink}{Bad Ischl}} kommen wollen \textsc{etc. etc.} –\pend
           \pstart
           Es geht Ihnen doch ſo gut wie ichs Ihnen wünſche?\pend
           \pstart
           Herzlichen Gruſs.{\\[\baselineskip]}Ihr\spacefill\mbox{Arthur.}\pend
           \leftskip=0em{}\endnumbering\briefempfaengerindex{Beer-Hofmann, Richard@\textsc{Beer-Hofmann, Richard}!zzzSchnitzler, Arthur@\emph{von Arthur Schnitzler}!1891-08-112@{11. 8. 1891}|)be}\mylabel{h}  \normalsize

\doendnotes{C}
\bigskip
\vfill

\clearpage

\footnotesize

\lohead{\textsc{register}}

% Definiere theindex-Environment komplett neu ohne reledmac
\makeatletter
\renewenvironment{theindex}{%
  \section*{\indexname}%
  \setlength{\parindent}{0pt}%
  \setlength{\parskip}{0pt plus 0.3pt}%
  \let\item\@idxitem
}{%
  \clearpage
}
\makeatother

\IfFileExists{\jobname-pw.ind}{\input{\jobname-pw.ind}}{}

\end{document}

      