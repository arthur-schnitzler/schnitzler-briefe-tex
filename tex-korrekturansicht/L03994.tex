%% latex-korrekturansicht-vorspann.tex
%% Vorspann für die Korrekturansicht.
%% Lädt die gemeinsame Datei latex-vorspann.tex mit gesetztem Schalter.

\newif\ifkorrekturansicht
\korrekturansichttrue

\input{../tex-inputs/latex-vorspann}


\section[Arthur Schnitzler: Widmungsexemplar von Fräulein Else an Berta Zuckerkandl, {[}28. 11. 1924?{]}]{L03994 Arthur Schnitzler: Widmungsexemplar von Fräulein Else an Berta
               Zuckerkandl, {[}28. 11. 1924?{]}}
\nopagebreak\mylabel{L03994v}
\rehead{ }\normalsize\beginnumbering\briefempfaengerindex{Zuckerkandl, Berta@\textsc{Zuckerkandl, Berta}!zzzSchnitzler, Arthur@\emph{von Arthur Schnitzler}!1924-11-281@{{[}28. 11. 1924?{]}}|(be}
\toendnotes[C]{\smallbreak\pagebreak[2]}
\correspDesc{Versand  durch Arthur Schnitzler am [28. 11. 1924?] in Wien
\newline{}Erhalt  durch Berta Zuckerkandl im Zeitraum [29. 11. 1924 – 3. 12. 1924?] in Paris}\toendnotes[C]{\smallbreak}
\Standort{Wien, Österreichische Nationalbibliothek, ZUC.5.1.SchFin LIT MAG.}
\physDesc{Widmung am Schmutztitel, 77 Zeichen
\newline{}Handschrift: schwarze Tinte, lateinische Kurrent}
\pstart
           \noindent{}{\pb}Frau Hofrätin Berta Zuckerkandl\pend
           
\pstart
           \centering{}\textcolor{gray}{\textbf{ARTHUR
               SCHNITZLER}}\pend
           
\pstart
           \centering{}\textcolor{gray}{\textbf{\textcolor{green}{FRÄULEIN ELSE}\pwindex{Schnitzler, Arthur 15. 5. 1862 Wien – 21. 10. 1931 ebd.@\textsc{Schnitzler, Arthur} (15. 5. 1862 Wien – 21. 10. 1931 ebd.), \emph{Schriftsteller, Mediziner}!Fräulein Else@\strich\emph{Fräulein Else}|pw}{}\ledrightnote{\textcolor{green}{Fräulein Else}}}}\pend
           
\pstart
           freundſchaftlich herzlichſt {\\[\baselineskip]}\spacefill\mbox{Arthur Schnitzler}\pend
           \leftskip=0em{}\selectlanguage{ngerman}\vspace{1em}{\vspace{1\baselineskip}}
\pstart
           \centering{}{\pb}\textcolor{gray}{\textbf{\so{ARTHUR SCHNITZLER}}}\pend
           
\pstart
           \centering{}\textcolor{gray}{\textbf{\textcolor{green}{FRÄULEIN ELSE}\pwindex{Schnitzler, Arthur 15. 5. 1862 Wien – 21. 10. 1931 ebd.@\textsc{Schnitzler, Arthur} (15. 5. 1862 Wien – 21. 10. 1931 ebd.), \emph{Schriftsteller, Mediziner}!Fräulein Else@\strich\emph{Fräulein Else}|pw}{}\ledrightnote{\textcolor{green}{Fräulein Else}}}}\pend
           
\pstart
           \centering{}\textcolor{gray}{\textbf{NOVELLE}}\pend
           {\vspace{1\baselineskip}}
\pstart
           \centering{}\textcolor{gray}{\textbf{1924}}\pend
           
\pstart
           \centering{}\textcolor{gray}{\textbf{\textcolor{brown}{PAUL ZSOLNAY VERLAG}\orgindex{Paul Zsolnay Verlag@Paul Zsolnay Verlag|pw}{}\ledrightnote{\textcolor{brown}{Paul Zsolnay Verlag}}}}\pend
           
\pstart
           \centering{}\textcolor{gray}{\textbf{\textcolor{pink}{BERLIN}\oindex{Berlin@\textbf{Berlin}, \emph{Hauptstadt}|pw}{}\ledrightnote{\textcolor{pink}{Berlin}} ⋅ \textcolor{pink}{WIEN}\oindex{Wien@\textbf{Wien}, \emph{Verwaltungsgebiet}|pw}{}\ledrightnote{\textcolor{pink}{Wien}} ⋅ \textcolor{pink}{LEIPZIG}\oindex{Leipzig@\textbf{Leipzig}, \emph{Hauptstadt}|pw}{}\ledrightnote{\textcolor{pink}{Leipzig}}}}\pend
           \selectlanguage{ngerman}\endnumbering\briefempfaengerindex{Zuckerkandl, Berta@\textsc{Zuckerkandl, Berta}!zzzSchnitzler, Arthur@\emph{von Arthur Schnitzler}!1924-11-281@{{[}28. 11. 1924?{]}}|)be}\mylabel{L03994h}
\begin{anhang}
\end{anhang}\normalsize

\doendnotes{C}
\bigskip
\vfill

\clearpage

\footnotesize

\lohead{\textsc{register}}

% Definiere theindex-Environment komplett neu ohne reledmac
\makeatletter
\renewenvironment{theindex}{%
  \section*{\indexname}%
  \setlength{\parindent}{0pt}%
  \setlength{\parskip}{0pt plus 0.3pt}%
  \let\item\@idxitem
}{%
  \clearpage
}
\makeatother

\IfFileExists{\jobname-pw.ind}{\input{\jobname-pw.ind}}{}

\end{document}

      