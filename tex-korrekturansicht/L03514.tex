%% latex-korrekturansicht-vorspann.tex
%% Vorspann für die Korrekturansicht.
%% Lädt die gemeinsame Datei latex-vorspann.tex mit gesetztem Schalter.

\newif\ifkorrekturansicht
\korrekturansichttrue

\input{../tex-inputs/latex-vorspann}


\renewcommand{\erwaehntePersonen}{Personen: Felix Salten, Ottilie Salten}
\renewcommand{\erwaehnteInstitutionen}{Institutionen: Lessing-Theater}
\renewcommand{\erwaehnteOrte}{Orte: Berlin, Charlottenburg, Edmund-Weiß-Gasse 7, Friedrichstraße, Hotel Victoria und Victoria-Café, Unter den Linden, Wien, XVIII., Währing}
\renewcommand{\erwaehnteWerke}{Werke: Vom andern Ufer. Einakter}
\section[ Felix Salten an Arthur Schnitzler, 22. 10. 1907]{Felix Salten an Arthur Schnitzler, 22. 10. 1907}
\nopagebreak\mylabel{v}
\rehead{ }\normalsize\beginnumbering\briefempfaengerindex{Schnitzler, Arthur@\textsc{Schnitzler, Arthur}!zzzSalten, Felix@\emph{von Felix Salten}!1907-10-221@{22. 10. 1907}|(be}
\toendnotes[C]{\smallbreak\pagebreak[2]}\Standort{CUL, Schnitzler, B 89, B 1.}
\physDesc{Bildpostkarte, 241 Zeichen
\newline{}Handschrift: schwarze Tinte, lateinische Kurrent
\newline{}Versand: Stempel: »\nobreak{}\oindex{Charlottenburg@\textbf{Charlottenburg}, \emph{P.PPLX}|pwk}Charlottenburg 2, 22. 10. 07, 4\textcolor{gray}{–5N.}\nobreak{}«.  
\newline{}Schnitzler: mit Bleistift datiert: »22/\textcolor{gray}{×} 07« 
\newline{}Ordnung: mit Bleistift von unbekannter Hand nummeriert: »237« }\toendnotes[C]{\smallbreak}\pstart{}{\pb}Herrn D\textsuperscript{r} Arthur Schnitzler\pend{}\pstart{}\textcolor{pink}{Wien XVIII.}{}\ledrightnote{\textcolor{pink}{XVIII., Währing}}\pend{}\pstart{}\textcolor{pink}{Spöttelgaße 7}{}\ledrightnote{\textcolor{pink}{Edmund-Weiß-Gasse 7}}\pend{}
{\bigskip}
\pstart
           \noindent{}\centering{}{\pb}\textcolor{pink}{\textcolor{gray}{\textbf{Berlin}}}{}\ledrightnote{\textcolor{pink}{Berlin}}\pend
           
\pstart
           \noindent{}\centering{}\textcolor{gray}{\textbf{\textcolor{pink}{Unter den Linden}{}\ledrightnote{\textcolor{pink}{Unter den Linden}}}}\pend
           
\pstart
           \noindent{}\centering{}\textcolor{gray}{\textbf{\textcolor{pink}{Ecke Friedrichstr.}{}\ledrightnote{\textcolor{pink}{Friedrichstraße}} (\textcolor{pink}{Victoria-Cafe}{}\ledrightnote{\textcolor{pink}{Hotel Victoria und Victoria-Café}})}}\pend
           
\pstart{}{\pb}Lieber,\pend
\pstart
           vielen dank für die \label{K_L03514-1v}\edtext{Depesche}{\lemma{\textnormal{\emph{Depesche}}}\Cendnote{\textnormal{\textcolor{blue}{Schnitzler} dürfte anlässlich der Uraufführung von
                     \emph{\textcolor{green}{Vom andern Ufer}} am 15. 10. 1907 am \emph{\textcolor{brown}{Lessing-Theater}} ein Gratulationstelegramm geschrieben haben.}}}\label{K_L03514-1h}. \textcolor{blue}{Wir}{}\ledrightnote{{$\rightarrow$}\textcolor{blue}{Ottilie Salten}} sind diese Woche in \textcolor{pink}{Wien}{}\ledrightnote{\textcolor{pink}{Wien}}. Wenn’s noch
               schön ist, \label{K_L03514-2v}\edtext{komm’ ich auf den
                  Tennisplatz}{\lemma{\textnormal{\emph{komm’ … Tennisplatz}}}\Cendnote{\textnormal{vgl. A. S.: \emph{Tagebuch}, 23. 10. 1907}}}\label{K_L03514-2h}, hoffe aber jedenfalls, Sie bald zu sehen.\pend
           
\pstart
           Herzlichst von \textcolor{blue}{uns}{}\ledrightnote{{$\rightarrow$}\textcolor{blue}{Ottilie Salten}} zu
               Ihnen Ihr {\\[\baselineskip]}\spacefill\mbox{Salten}\pend
           \leftskip=0em{}\endnumbering\briefempfaengerindex{Schnitzler, Arthur@\textsc{Schnitzler, Arthur}!zzzSalten, Felix@\emph{von Felix Salten}!1907-10-221@{22. 10. 1907}|)be}\mylabel{h}  \normalsize

\doendnotes{C}
\bigskip
\vfill

\clearpage

\footnotesize

\lohead{\textsc{register}}

% Definiere theindex-Environment komplett neu ohne reledmac
\makeatletter
\renewenvironment{theindex}{%
  \section*{\indexname}%
  \setlength{\parindent}{0pt}%
  \setlength{\parskip}{0pt plus 0.3pt}%
  \let\item\@idxitem
}{%
  \clearpage
}
\makeatother

\IfFileExists{\jobname-pw.ind}{\input{\jobname-pw.ind}}{}

\end{document}

      