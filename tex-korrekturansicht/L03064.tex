%% latex-korrekturansicht-vorspann.tex
%% Vorspann für die Korrekturansicht.
%% Lädt die gemeinsame Datei latex-vorspann.tex mit gesetztem Schalter.

\newif\ifkorrekturansicht
\korrekturansichttrue

\input{../tex-inputs/latex-vorspann}


\renewcommand{\erwaehntePersonen}{Personen: Hermann Bahr, Georg Brandes, Alexander N. Ostrowski, Olga Schnitzler, Elisabeth Steinrück}
\renewcommand{\erwaehnteInstitutionen}{Institutionen: S. Fischer Verlag}
\renewcommand{\erwaehnteOrte}{Orte: Berlin, Bozen, Dessauer Straße, Grünentorgasse, Italien, Rom, Südtirol, Welsberg-Taisten, Wien}
\renewcommand{\erwaehnteWerke}{Werke: Der Schleier der Beatrice. Schauspiel in fünf Akten, Frau Bertha Garlan. Roman, Gewitter, Politiken, Skikkelser og Tanker. Arthur Schnitzler}
\section[ Paul Goldmann an Arthur Schnitzler, 26. 4. {[}1901{]}]{Paul Goldmann an Arthur Schnitzler, 26. 4. {[}1901{]}}
\nopagebreak\mylabel{v}
\rehead{ }\normalsize\beginnumbering\briefempfaengerindex{Schnitzler, Arthur@\textsc{Schnitzler, Arthur}!zzzGoldmann, Paul@\emph{von Paul Goldmann}!1901-04-261@{26. 4. {[}1901{]}}|(be}
\toendnotes[C]{\smallbreak\pagebreak[2]}\Standort{DLA, A:Schnitzler, HS.NZ85.1.3171.}
\physDesc{Brief, 1 Blatt, 3 Seiten
\newline{}Handschrift: blaue Tinte, deutsche Kurrent
\newline{}Schnitzler: mit rotem Buntstift zwei Unterstreichungen }\toendnotes[C]{\smallbreak}
\pstart
           \noindent{}\raggedleft{}{\pb}\textcolor{pink}{\textcolor{gray}{\textbf{DESSAUERSTRASSE 19}}}{}\ledrightnote{\textcolor{pink}{Dessauer Straße}}\pend
           
\pstart
           \textcolor{pink}{Berlin}{}\ledrightnote{\textcolor{pink}{Berlin}}, 26. April.\pend
           
\pstart\center{}Mein lieber Freund,\pend
\pstart
           Dank für den lieben Brief! Dank auch für den »\textcolor{green}{Schleier der \textsc{Beatrice}}{}\ledrightnote{\textcolor{green}{Der Schleier der Beatrice. Schauspiel in fünf Akten}}« und »\textsc{\textcolor{green}{Bertha Garlan}{}\ledrightnote{\textcolor{green}{Frau Bertha Garlan. Roman}}}«, die ich in \label{K_L03064-1v}\edtext{ſchön gebundenen
                  Exemplaren}{\lemma{\textnormal{\emph{ſchön … Exemplaren}}}\Cendnote{\textnormal{\emph{\textcolor{green}{Der Schleier der Beatrice}} war am 21. 2. 1901 bei \emph{\textcolor{brown}{S.
                     Fischer}} erschienen, \emph{\textcolor{green}{Frau Bertha
                     Garlan}} am 13. 4. 1901.}}}\label{K_L03064-1h} erhielt! Dank
               endlich für Deine Bemühungen bei \textsc{\textcolor{blue}{Bahr}{}\ledrightnote{\textcolor{blue}{Hermann Bahr}}} in Sachen des Stückes \label{K_L03064-2v}\edtext{»\textcolor{green}{Gewitter}{}\ledrightnote{\textcolor{green}{Gewitter}}«}{\lemma{\textnormal{\emph{»Gewitter«}}}\Cendnote{\textnormal{Bezug unklar; möglicherweise handelte es sich um den
                  Fünfakter \emph{\textcolor{green}{Gewitter}} von \textcolor{blue}{Alexander Ostrowski}}}}\label{K_L03064-2h}!\pend
           
\pstart
           Ich freue mich, daß Du wieder glücklich \label{K_L03064-3v}\edtext{daheim}{\lemma{\textnormal{\emph{daheim}}}\Cendnote{\textnormal{\textcolor{blue}{Schnitzler} war am 19. 4. 1901 von seiner
                     \textcolor{pink}{Italien}reise zurückgekehrt.}}}\label{K_L03064-3h} biſt.
               Auch \label{K_L03064-4v}\edtext{die andere Nachricht}{\lemma{\textnormal{\emph{die andere Nachricht}}}\Cendnote{\textnormal{\textcolor{blue}{Olga}s Schwangerschaft, die jedoch
                  abgebrochen werden musste (vgl. A. S.: \emph{Tagebuch}, 10. 5. 1901)}}}\label{K_L03064-4h} iſt \strikeout{\textcolor{gray}{echt}} eine erfreuliche. Eine Frau und ein Kind, – das iſt wohl die \strikeout{Löſ\textcolor{gray}{un}} Erklärung für das, was die Natur mit uns vorhat; und demjengen, der danach
               handelt, ſpendet ſie Glücksgefühle, wie immer, wenn man ihre geheimen Abſichten
               erräth. Das iſt der Weg zum Gück: die geheimen Abſichten der Natur errathen. Ich
               wünſche Dir einen Sohn. {\pb}Daß man mit
               ſeiner Geliebten nach \textcolor{pink}{Italien}{}\ledrightnote{\textcolor{pink}{Italien}} gehen muß, iſt
               ſelbſtverſtändlich. Ich möchte wiſſen, was \textcolor{pink}{Italien}{}\ledrightnote{\textcolor{pink}{Italien}} ſonſt \strikeout{\textcolor{gray}{×}\-\textcolor{gray}{×}} für einen Sinn hat, als den: eine Umgebung für eine Liebe zu ſein. Darum
               beneide ich Dich nicht um Deine \label{K_L03064-66v}\edtext{\textcolor{pink}{Rom}{}\ledrightnote{\textcolor{pink}{Rom}}fahrt}{\lemma{\textnormal{\emph{Romfahrt}}}\Cendnote{\textnormal{siehe Paul Goldmann an Arthur Schnitzler, 6. 4. [1901]}}}\label{K_L03064-66h}. Wohl aber beneide ich Dich um Deine \label{K_L03064-23v}\edtext{Sehnſucht nach \textsc{\textcolor{blue}{Olga}{}\ledrightnote{\textcolor{blue}{Olga Schnitzler}}}}{\lemma{\textnormal{\emph{Sehnſucht nach Olga}}}\Cendnote{\textnormal{siehe A. S.: \emph{Tagebuch}, 17. 4. 1901}}}\label{K_L03064-23h}. Ich darf mich nach Keiner ſehnen.\pend
           
\pstart
           Der \label{K_L03064-25v}\edtext{\textcolor{green}{Artikel}{}\ledrightnote{{$\rightarrow$}\textcolor{green}{Skikkelser og Tanker. Arthur Schnitzler}}}{\lemma{\textnormal{\emph{Artikel}}}\Cendnote{\textnormal{\textcolor{blue}{Georg Brandes}: \emph{\textcolor{green}{Skikkelser og Tanker.
                        Arthur Schnitzler}}. In: \emph{\textcolor{green}{Politiken}},
                     Nr. 98, 9. 4. 1901, S. 1 Es gibt ein
                  nicht überliefertes Korrespondenzstück \textcolor{blue}{Goldmann}s, in dem er \textcolor{blue}{Schnitzler} den
                     \textcolor{green}{Artikel} übersandte,
                     vgl. Arthur Schnitzler an Georg Brandes, 25. 4. 1901.}}}\label{K_L03064-25h} von \textsc{\textcolor{blue}{Brandes}{}\ledrightnote{\textcolor{blue}{Georg Brandes}}} über Dich war recht ſchleuderhaft geſchrieben. \textsc{\textcolor{blue}{Brandes}{}\ledrightnote{\textcolor{blue}{Georg Brandes}}} war dieſer Tage in \textcolor{pink}{Berlin}{}\ledrightnote{\textcolor{pink}{Berlin}} – in merkwürdiger
               Stimmung: gezwungen heiter, manchmal verſtört. Plötzlich iſt er abgereiſt. Ich habe
               ihn ſehr gern. Er hat etwas ſo Feines und Gütiges\substVorne{}\textsuperscript{!}\substDazwischen{}.\substHinten{}\pend
           
\pstart
           \label{K_L03064-123v}\edtext{Sommerpläne}{\lemma{\textnormal{\emph{Sommerpläne}}}\Cendnote{\textnormal{\textcolor{blue}{Schnitzler} und \textcolor{blue}{Goldmann} trafen sich im August 1901 jedenfalls mehrmals in \textcolor{pink}{Südtirol}, konkret am 7. 8. 1901 in \textcolor{pink}{Welsberg}, am 13. 8. 1901 in \textcolor{pink}{Bozen} und zwischen 18. 8. 1901 und 29. 8. 1901 noch
                  einmal in \textcolor{pink}{Welsberg}. Danach reiste \textcolor{blue}{Goldmann} mit \textcolor{blue}{Schnitzler} nach \textcolor{pink}{Wien}
                  zurück und blieb dort wohl noch ein paar Tage.}}}\label{K_L03064-123h}? Wie Du willſt. Mir {\pb}iſt Alles eins. Ich fahre weg oder
               bleibe auch zu Hauſe. Bin auf dem Tiefpunkt aller menſchlichen Verfaſſung angelangt{\dotsfour}\pend
           
\pstart
           Grüße an die \textcolor{blue}{\textcolor{pink}{Grünethorgaſſe}{}\ledrightnote{\textcolor{pink}{Grünentorgasse}}}{}\ledrightnote{{$\rightarrow$}\textcolor{blue}{Olga Schnitzler}{\newline}{$\rightarrow$}\textcolor{blue}{Elisabeth Steinrück}}, Grüße an Dich! {\\[\baselineskip]}Von Herzen {\\[\baselineskip]}Dein {\\[\baselineskip]}\spacefill\mbox{Paul Goldmnn}\pend
           \leftskip=0em{}\endnumbering\briefempfaengerindex{Schnitzler, Arthur@\textsc{Schnitzler, Arthur}!zzzGoldmann, Paul@\emph{von Paul Goldmann}!1901-04-261@{26. 4. {[}1901{]}}|)be}\mylabel{h}
\begin{anhang}
\end{anhang}\normalsize

\doendnotes{C}
\bigskip
\vfill

\clearpage

\footnotesize

\lohead{\textsc{register}}

% Definiere theindex-Environment komplett neu ohne reledmac
\makeatletter
\renewenvironment{theindex}{%
  \section*{\indexname}%
  \setlength{\parindent}{0pt}%
  \setlength{\parskip}{0pt plus 0.3pt}%
  \let\item\@idxitem
}{%
  \clearpage
}
\makeatother

\IfFileExists{\jobname-pw.ind}{\input{\jobname-pw.ind}}{}

\end{document}

      