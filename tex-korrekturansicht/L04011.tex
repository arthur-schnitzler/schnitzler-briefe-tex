%% latex-korrekturansicht-vorspann.tex
%% Vorspann für die Korrekturansicht.
%% Lädt die gemeinsame Datei latex-vorspann.tex mit gesetztem Schalter.

\newif\ifkorrekturansicht
\korrekturansichttrue

\input{../tex-inputs/latex-vorspann}


\section[Frank Wedekind an Arthur Schnitzler, 11. 6. 1913]{L04011 Frank Wedekind an Arthur Schnitzler, 11. 6. 1913}
\nopagebreak\mylabel{L04011v}
\rehead{ }\normalsize\beginnumbering\briefempfaengerindex{Schnitzler, Arthur@\textsc{Schnitzler, Arthur}!zzzWedekind, Frank@\emph{von Frank Wedekind}!1913-06-111@{11. 6. 1913}|(be}
\toendnotes[C]{\smallbreak\pagebreak[2]}
\correspDesc{Versand  durch Frank Wedekind am 11. 6. 1913 in Wien
\newline{}Erhalt  durch Arthur Schnitzler im Zeitraum [11. 6. 1913
                  – 15. 6. 1913?] in Wien}\toendnotes[C]{\smallbreak}
\Standort{DLA, HS.2006.43.28.}
\physDesc{Brief, 1 Blatt, 2 Seiten, 539 Zeichen
\newline{}Handschrift: schwarze Tinte, deutsche Kurrent
\newline{}Schnitzler: mit rotem Buntstift 1 Unterstreichung }
\buchAbdrucke{\weitereDrucke{\emph{Frank Wedekinds Korrespondenz digital}. (7. 10. 2024) \url{https://briefedition.wedekind.h-da.de/view/document/single.xhtml?contentType=1&documentId=1553}.} }\toendnotes[C]{\smallbreak}
\pstart
           \centering{}{\pb}\textcolor{gray}{\textbf{\textcolor{pink}{\label{K_L04011-1v}\edtext{HOTEL TEGETTHOFF}{\lemma{\textnormal{\emph{Hotel Tegetthoff}}}\Cendnote{\textnormal{\textcolor{blue}{Frank}\pwindex{Wedekind, Frank 24.\,7.\,1864 Hannover – 9.\,3.\,1918 München@\textsc{Wedekind, Frank} (24.\,7.\,1864 Hannover – 9.\,3.\,1918 München), \emph{Schriftsteller, Schauspieler, Schriftsteller}|pwk} und \textcolor{blue}{Tilly Wedekind}\pwindex{Wedekind, Tilly 11.\,4.\,1886 Graz – 20.\,4.\,1970 München@\textsc{Wedekind, Tilly} (11.\,4.\,1886 Graz – 20.\,4.\,1970 München), \emph{Schauspielerin}|pwk} weilten zwischen
                                 4. 6. 1913 und 13. 6. 1913 anlässlich
                              eines Gastspiels in \textcolor{pink}{Wien}\oindex{Wien@\textbf{Wien}, \emph{Verwaltungsgebiet}|pwk}.}}}\label{K_L04011-1},
                           I. JOHANNESGASSE 23, WIEN}\oindex{Wien@\textbf{Wien}!I., Innere Stadt@\textbf{I., Innere Stadt}!Hotel Tegetthoff@\textbf{Hotel Tegetthoff}, \emph{Hotel}|pw}{}\ledrightnote{\textcolor{pink}{Hotel Tegetthoff}}               }}\pend
           
\pstart
           \centering{}\textcolor{gray}{\textbf{TELEGRAMM-ADRESSE: \textcolor{pink}{TEGETTHOFFHOTEL}\oindex{Wien@\textbf{Wien}!I., Innere Stadt@\textbf{I., Innere Stadt}!Hotel Tegetthoff@\textbf{Hotel Tegetthoff}, \emph{Hotel}|pw}{}\ledrightnote{\textcolor{pink}{Hotel Tegetthoff}}, WIEN. INTERNATIONALER HOTEL-TELEGRAPHEN-CODE.}}\pend
           
\pstart\center{}Sehr verehrter Herr Doctor!\pend\vspace{0.5em}
\pstart
           Zu unſerem innigen Bedauern \label{K_L04011-2v}\edtext{hörten wir
                  heute}{\lemma{\textnormal{\emph{hörten wir
                  heute}}}\Cendnote{\textnormal{\textcolor{blue}{Wedekind}\pwindex{Wedekind, Frank 24.\,7.\,1864 Hannover – 9.\,3.\,1918 München@\textsc{Wedekind, Frank} (24.\,7.\,1864 Hannover – 9.\,3.\,1918 München), \emph{Schriftsteller, Schauspieler, Schriftsteller}|pwk} notierte für den
                     11. 6. 1913 in seinem Tagebuch einen kurzen »Besuch bei \textcolor{blue}{Schnitzler}«, dürfte diesen aber nicht angetroffen haben. \textcolor{blue}{Schnitzler} erwähnt an diesem Tag keinen Besuch \textcolor{blue}{Wedekinds}\pwindex{Wedekind, Frank 24.\,7.\,1864 Hannover – 9.\,3.\,1918 München@\textsc{Wedekind, Frank} (24.\,7.\,1864 Hannover – 9.\,3.\,1918 München), \emph{Schriftsteller, Schauspieler, Schriftsteller}|pwk} in seinem \emph{\textcolor{green}{Tagebuch}\pwindex{Schnitzler, Arthur 15. 5. 1862 Wien – 21. 10. 1931 ebd.@\textsc{Schnitzler, Arthur} (15. 5. 1862 Wien – 21. 10. 1931 ebd.), \emph{Schriftsteller, Mediziner}!Tagebuch@\strich\emph{Tagebuch}|pwk}}.}}}\label{K_L04011-2} von der \label{K_L04011-3v}\edtext{Erkrankung Ihres lieben \textcolor{blue}{Kindes}\pwindex{Schnitzler, Heinrich 9.\,8.\,1902 Hinterbrühl – 12.\,7.\,1982 Wien@\textsc{Schnitzler, Heinrich} (9.\,8.\,1902 Hinterbrühl – 12.\,7.\,1982 Wien), \emph{Regisseur, Schauspieler}|pwv}{}\ledrightnote{{$\rightarrow$}\emph{\textcolor{blue}{Heinrich Schnitzler}}}}{\lemma{\textnormal{\emph{Erkrankung … Kindes}}}\Cendnote{\textnormal{Am 9. 6. 1913 waren \textcolor{blue}{Olga}\pwindex{Schnitzler, Olga 17.\,1.\,1882 Wien – 13.\,1.\,1970 Lugano@\textsc{Schnitzler, Olga} (17.\,1.\,1882 Wien – 13.\,1.\,1970 Lugano), \emph{Schauspielerin, Sängerin}|pwk} und \textcolor{blue}{Arthur
                     Schnitzler} zu einer mehrwöchigen Reise in die \textcolor{pink}{Schweiz}\oindex{Schweiz@\textbf{Schweiz}|pwk} aufgebrochen. Bereits am Folgetag erhielten sie
                  ein Telegramm, demnach ihr Sohn \textcolor{blue}{Heinrich}\pwindex{Schnitzler, Heinrich 9.\,8.\,1902 Hinterbrühl – 12.\,7.\,1982 Wien@\textsc{Schnitzler, Heinrich} (9.\,8.\,1902 Hinterbrühl – 12.\,7.\,1982 Wien), \emph{Regisseur, Schauspieler}|pwk} an
                  Scharlach erkrankt war. Die \textcolor{blue}{Eltern}\pwindex{Schnitzler, Olga 17.\,1.\,1882 Wien – 13.\,1.\,1970 Lugano@\textsc{Schnitzler, Olga} (17.\,1.\,1882 Wien – 13.\,1.\,1970 Lugano), \emph{Schauspielerin, Sängerin}|pwkv} verließen das eben erreichte \textcolor{pink}{Chur}\oindex{Chur@\textbf{Chur}|pwk} sofort wieder und waren seit 11. 6. 1913 wieder in \textcolor{pink}{Wien}\oindex{Wien@\textbf{Wien}, \emph{Verwaltungsgebiet}|pwk}.}}}\label{K_L04011-3}. Wollen Sie und Ihre verehrte Frau
                  \textcolor{blue}{Gemahlin}\pwindex{Schnitzler, Olga 17.\,1.\,1882 Wien – 13.\,1.\,1970 Lugano@\textsc{Schnitzler, Olga} (17.\,1.\,1882 Wien – 13.\,1.\,1970 Lugano), \emph{Schauspielerin, Sängerin}|pwv}{}\ledrightnote{{$\rightarrow$}\emph{\textcolor{blue}{Olga Schnitzler}}} meiner \textcolor{blue}{Frau}\pwindex{Wedekind, Tilly 11.\,4.\,1886 Graz – 20.\,4.\,1970 München@\textsc{Wedekind, Tilly} (11.\,4.\,1886 Graz – 20.\,4.\,1970 München), \emph{Schauspielerin}|pwv}{}\ledrightnote{{$\rightarrow$}\emph{\textcolor{blue}{Tilly Wedekind}}} und meine aufrichtigſten
               Wünſche zu möglichst baldiger Geneſung des Kleinen entgegennehmen. Für das ſchöne
                  \label{K_L04011-4v}\edtext{Geſchenk Ihres \textcolor{green}{Novellen{\pb}bandes}\pwindex{Schnitzler, Arthur 15. 5. 1862 Wien – 21. 10. 1931 ebd.@\textsc{Schnitzler, Arthur} (15. 5. 1862 Wien – 21. 10. 1931 ebd.), \emph{Schriftsteller, Mediziner}!Masken und Wunder. Novellen@\strich\emph{Masken und Wunder. Novellen}|pwv}\pwindex{Schnitzler, Arthur 15. 5. 1862 Wien – 21. 10. 1931 ebd.@\textsc{Schnitzler, Arthur} (15. 5. 1862 Wien – 21. 10. 1931 ebd.), \emph{Schriftsteller, Mediziner}!Frau Beate und ihr Sohn. Novelle@\strich\emph{Frau Beate und ihr Sohn. Novelle}|pwv}{}\ledrightnote{{$\rightarrow$}\emph{\textcolor{green}{Masken und Wunder. Novellen}}{\newline}{$\rightarrow$}\emph{\textcolor{green}{Frau Beate und ihr Sohn. Novelle}}}}{\lemma{\textnormal{\emph{Geſchenk Ihres Novellenbandes}}}\Cendnote{\textnormal{Es könnte sich um \emph{\textcolor{green}{Masken und Wunder}\pwindex{Schnitzler, Arthur 15. 5. 1862 Wien – 21. 10. 1931 ebd.@\textsc{Schnitzler, Arthur} (15. 5. 1862 Wien – 21. 10. 1931 ebd.), \emph{Schriftsteller, Mediziner}!Masken und Wunder. Novellen@\strich\emph{Masken und Wunder. Novellen}|pwk}} handeln – die Lektüre der
                     darin enthaltenen Novelle \emph{\textcolor{green}{Der Tod des Junggesellen}\pwindex{Schnitzler, Arthur 15. 5. 1862 Wien – 21. 10. 1931 ebd.@\textsc{Schnitzler, Arthur} (15. 5. 1862 Wien – 21. 10. 1931 ebd.), \emph{Schriftsteller, Mediziner}!Tod des Junggesellen. Novelle@\strich\emph{Der Tod des Junggesellen. Novelle}|pwk}}
                  notierte \textcolor{blue}{Wedekind}\pwindex{Wedekind, Frank 24.\,7.\,1864 Hannover – 9.\,3.\,1918 München@\textsc{Wedekind, Frank} (24.\,7.\,1864 Hannover – 9.\,3.\,1918 München), \emph{Schriftsteller, Schauspieler, Schriftsteller}|pwk} in seinem Tagebuch für
                  den 1. 8. 1913 – oder die erst wenige Wochen zuvor
                  erschienene Novelle \emph{\textcolor{green}{Frau Beate und ihr Sohn}\pwindex{Schnitzler, Arthur 15. 5. 1862 Wien – 21. 10. 1931 ebd.@\textsc{Schnitzler, Arthur} (15. 5. 1862 Wien – 21. 10. 1931 ebd.), \emph{Schriftsteller, Mediziner}!Frau Beate und ihr Sohn. Novelle@\strich\emph{Frau Beate und ihr Sohn. Novelle}|pwk}}.
                  Wie \textcolor{blue}{Schnitzler} das Geschenk überreichte, ist
                  unklar.}}}\label{K_L04011-4} ſage ich Ihnen herzlichen Dank. Ich freue mich außerordentlich auf
               den großen Genuß, den er mir bereiten wird.\pend
           
\pstart
           Nochmals die beſten Wünſche und herzliche Grüße an Sie und Ihre Frau \textcolor{blue}{Gemahlin}\pwindex{Schnitzler, Olga 17.\,1.\,1882 Wien – 13.\,1.\,1970 Lugano@\textsc{Schnitzler, Olga} (17.\,1.\,1882 Wien – 13.\,1.\,1970 Lugano), \emph{Schauspielerin, Sängerin}|pwv}{}\ledrightnote{{$\rightarrow$}\emph{\textcolor{blue}{Olga Schnitzler}}} von meiner \textcolor{blue}{Frau}\pwindex{Wedekind, Tilly 11.\,4.\,1886 Graz – 20.\,4.\,1970 München@\textsc{Wedekind, Tilly} (11.\,4.\,1886 Graz – 20.\,4.\,1970 München), \emph{Schauspielerin}|pwv}{}\ledrightnote{{$\rightarrow$}\emph{\textcolor{blue}{Tilly Wedekind}}} und mir.\pend
           
\pstart
           Ihr ergebener{\\[\baselineskip]}\spacefill\mbox{Frank Wedekind.}\pend
           \leftskip=0em{}
\pstart
           11. 6. 13.\pend
           \selectlanguage{ngerman}\endnumbering\briefempfaengerindex{Schnitzler, Arthur@\textsc{Schnitzler, Arthur}!zzzWedekind, Frank@\emph{von Frank Wedekind}!1913-06-111@{11. 6. 1913}|)be}\mylabel{L04011h}  \normalsize

\doendnotes{C}
\bigskip
\vfill

\clearpage

\footnotesize

\lohead{\textsc{register}}

% Definiere theindex-Environment komplett neu ohne reledmac
\makeatletter
\renewenvironment{theindex}{%
  \section*{\indexname}%
  \setlength{\parindent}{0pt}%
  \setlength{\parskip}{0pt plus 0.3pt}%
  \let\item\@idxitem
}{%
  \clearpage
}
\makeatother

\IfFileExists{\jobname-pw.ind}{\input{\jobname-pw.ind}}{}

\end{document}

      