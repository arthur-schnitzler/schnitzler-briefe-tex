%% latex-korrekturansicht-vorspann.tex
%% Vorspann für die Korrekturansicht.
%% Lädt die gemeinsame Datei latex-vorspann.tex mit gesetztem Schalter.

\newif\ifkorrekturansicht
\korrekturansichttrue

\input{../tex-inputs/latex-vorspann}


               \section[Arthur Schnitzler an Hermann Bahr, 17. 8. 1904]{ Arthur Schnitzler an Hermann Bahr, 17. 8. 1904}\nopagebreak\mylabel{v}\rehead{ }\normalsize\beginnumbering\briefempfaengerindex{Bahr, Hermann@\textsc{Bahr, Hermann}!zzzSchnitzler, Arthur@\emph{von Arthur Schnitzler}!1904-08-171@{17. 8. 1904}|(be} \toendnotes[C]{\smallbreak\pagebreak[2]} \Standort{TMW, HS AM 23366 Ba.}
\physDesc{Kartenbrief
\newline{}Handschrift: Bleistift, deutsche Kurrent\newline{}Versand: 1) Stempel: »\nobreak{}\oindex{XVIII., Waehring@\textbf{XVIII., Währing}, \emph{Bezirk (A.BZK)}|pwk}Wien 18, 17. VIII 04\nobreak{}«.  2) Stempel: »\nobreak{}\oindex{XIII., Hietzing@\textbf{XIII., Hietzing}, \emph{Bezirk (A.BZK)}|pwk}Wien 13/7, 18. 8. 04, 8. V, Bestellt\nobreak{}«. }\buchAbdrucke{\weitereDrucke{1) \emph{17. 8. 1904.} In: Arthur Schnitzler: \emph{The Letters of Arthur Schnitzler to Hermann Bahr}. Edited, annotated, and with an introduction, by Donald G.
                        Daviau. Chapel Hill: \emph{The University of North Carolina Press} 1978, S. 85 (University of North Carolina studies in the Germanic languages
                        and literatures, 89).} \weitereDrucke{2) Hermann Bahr, Arthur Schnitzler: \emph{Briefwechsel, Aufzeichnungen, Dokumente (1891–1931)}. Hg. Kurt Ifkovits und Martin Anton Müller. Göttingen: \emph{Wallstein} 2018, S. 316.} }\toendnotes[C]{\smallbreak}\pstart{}{\pb}Herrn Hermann
                  Bahr\pend{}\pstart{}\textcolor{pink}{Wien Ob. St. Veit}{}\ledrightnote{\textcolor{pink}{Ober Sankt Veit}}\pend{}\pstart{}\textcolor{pink}{Veitliſſengaſſe.}{}\ledrightnote{\textcolor{pink}{Veitlissengasse}}\pend{}{\bigskip}\pstart
           \raggedleft{}{\pb}17. 8.\pend
           \pstart
           lieber Hermann, \textcolor{blue}{wir}{}\ledrightnote{→\textcolor{blue}{Olga Schnitzler}} wollen Freitag um 8 in dem
                  \label{K_L01427_1v}\edtext{\textcolor{pink}{\textsc{Kuffner} Garten}{}\ledrightnote{\textcolor{pink}{Ottakringer Bräu}}}{\lemma{\textnormal{\emph{Kuffner Garten}}}\Cendnote{\textnormal{Gastgarten des \textcolor{pink}{Ottakringer Bräu}; das Lokal gehörte zu der im 16. \textcolor{pink}{Wien}er Gemeindebezirk (\textcolor{pink}{Ottakring}) angesiedelten Brauerei \textcolor{pink}{Kuffner}.}}}\label{K_L01427_1h} in \textcolor{pink}{Hietzing}{}\ledrightnote{\textcolor{pink}{XIII., Hietzing}} nachtmahlen; –
               und hoffen ſehr, wenn du nichts anderes vorhaſt, dich dort zu treffen.\pend
           \pstart
           Ziehſt du ein andres Rendezvous vor, so verſtändige mich.\pend
           \pstart
           Von Herzen{\\[\baselineskip]}dein \spacefill\mbox{Arthur}\pend
           \leftskip=0em{}\endnumbering\briefempfaengerindex{Bahr, Hermann@\textsc{Bahr, Hermann}!zzzSchnitzler, Arthur@\emph{von Arthur Schnitzler}!1904-08-171@{17. 8. 1904}|)be}\mylabel{h}  \normalsize

\doendnotes{C}
\bigskip
\vfill

\clearpage

\footnotesize

\lohead{\textsc{register}}

% Definiere theindex-Environment komplett neu ohne reledmac
\makeatletter
\renewenvironment{theindex}{%
  \section*{\indexname}%
  \setlength{\parindent}{0pt}%
  \setlength{\parskip}{0pt plus 0.3pt}%
  \let\item\@idxitem
}{%
  \clearpage
}
\makeatother

\IfFileExists{\jobname-pw.ind}{\input{\jobname-pw.ind}}{}

\end{document}

      