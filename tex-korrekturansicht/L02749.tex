%% latex-korrekturansicht-vorspann.tex
%% Vorspann für die Korrekturansicht.
%% Lädt die gemeinsame Datei latex-vorspann.tex mit gesetztem Schalter.

\newif\ifkorrekturansicht
\korrekturansichttrue

\input{../tex-inputs/latex-vorspann}


               \section[Paul Goldmann an Arthur Schnitzler, 6. 10. {[}1895{]}]{ Paul Goldmann an Arthur Schnitzler, 6. 10. {[}1895{]}}\nopagebreak\mylabel{v}\rehead{ }\normalsize\beginnumbering\briefempfaengerindex{Schnitzler, Arthur@\textsc{Schnitzler, Arthur}!zzzGoldmann, Paul@\emph{von Paul Goldmann}!1895-10-062@{6. 10. {[}1895{]}}|(be} \toendnotes[C]{\smallbreak\pagebreak[2]} \Standort{DLA, A:Schnitzler, HS.NZ85.1.3165.}
\physDesc{Brief, 1 Blatt, 3 Seiten
\newline{}Handschrift: blaue Tinte, deutsche Kurrent
\newline{}Schnitzler: mit Bleistift das Jahr » 95« vermerkt }\toendnotes[C]{\smallbreak}\pstart
           \noindent{}{\pb}\textcolor{gray}{\textbf{\textbf{\textcolor{brown}{Frankfurter Zeitung}{}\ledrightnote{\textcolor{brown}{Frankfurter Zeitung}}}}}\pend
           \pstart
           \textcolor{gray}{\textbf{(\textcolor{brown}{\begin{otherlanguage}{french}Gazette de Francfort\end{otherlanguage}}{}\ledrightnote{\textcolor{brown}{Frankfurter Zeitung}}). }}\pend
           \pstart
           \textcolor{gray}{\textbf{\textbf{\begin{otherlanguage}{french}Fondateur M. \textcolor{blue}{L. Sonnemann}{}\ledrightnote{\textcolor{blue}{Leopold Sonnemann}}\end{otherlanguage}.}}}\pend
           \pstart
           \begin{otherlanguage}{french}\textcolor{gray}{\textbf{\textcolor{green}{Journal}{}\ledrightnote{→\textcolor{green}{Frankfurter Zeitung}} politique, financier,}}\end{otherlanguage}\pend
           \pstart
           \begin{otherlanguage}{french}\textcolor{gray}{\textbf{commercial et littéraire.}}\end{otherlanguage}\pend
           \pstart
           \begin{otherlanguage}{french}\textcolor{gray}{\textbf{\textbf{Paraissant trois fois par jour.}}}\end{otherlanguage}\hfill \textsc{\textcolor{pink}{Paris}{}\ledrightnote{\textcolor{pink}{Paris}}}, 6. Oktober.\pend
           \pstart
           \begin{otherlanguage}{french}\textcolor{gray}{\textbf{\textbf{Bureau à \textcolor{pink}{Paris}{}\ledrightnote{\textcolor{pink}{Paris}}:}}}\end{otherlanguage}\pend
           \pstart
           \begin{otherlanguage}{french}\textcolor{gray}{\textbf{\textbf{\textcolor{pink}{24. Rue Feydeau}{}\ledrightnote{\textcolor{pink}{rue Feydeau}}.}}}\end{otherlanguage}\pend
           \pstart\center{}Mein lieber Freund,\pend\pstart
           Morgen ſchreibe ich Dir ausführlicher. Heut hab’ ich alle Hände voll zu thun: \label{K_L02749-1v}\edtext{\textsc{\begin{otherlanguage}{french}Grand Prix d’automne\end{otherlanguage}}}{\lemma{\textnormal{\emph{Grand Prix d’automne}}}\Cendnote{\textnormal{Gemeint ist wohl der \emph{\textcolor{brown}{Prix
                        Montgomery}}, ein Hindernisrennen mit Pferden, der zuvor \textcolor{brown}{Grand Prix d’automne} hieß und von
                        6. 11. 1895 bis 10. 11. 1895 in \textcolor{pink}{Auteil} ausgetragen wurde.}}}\label{K_L02749-1h}{ }\textsc{etc}. Einſtweilen
               will ich Dir nur von Herzen danken für Deine treue Berichterſtattung und Dir ſagen,
               daß all’ meine Wünſche mit Dir ſind in dieſen {\pb}ereignißreichen und hoffentlich nicht allzu ſchweren Tagen. Ich habe das Bedürfniß,
               einen Segensſpruch zu thun. Es iſt doch ſchade, daß \strikeout{\textcolor{gray}{w}} wir den alten lieben Gott ſeines \strikeout{Antes} Amtes entſetzt haben. Zum Segnen war er ſo
               bequem, ſo handtlich. So empfehle ich Dich dem Schutze aller guten Mächte. Mit all’
               dieſen Wünſchen wird man ja freilich {\pb}das Schickſal
               nicht vom Wege ablenken können, das ſeinen Lauf nimmt. Aber ich glaube die Richtung
               zu ſehen, in der dieſes Dein Schickſal geht, und ich glaube zu erkennen, ſo ſicher
               als ich je etwas erkannt, daß es die gute Richtung iſt.\pend
           \pstart
           Glück viel, viel, viel Glück, mein theurer Freund!\pend
           \pstart
           Dein {\\[\baselineskip]}\spacefill\mbox{Paul Goldmann}\pend
           \leftskip=0em{}\endnumbering\briefempfaengerindex{Schnitzler, Arthur@\textsc{Schnitzler, Arthur}!zzzGoldmann, Paul@\emph{von Paul Goldmann}!1895-10-062@{6. 10. {[}1895{]}}|)be}\mylabel{h}\begin{anhang}\end{anhang}\normalsize

\doendnotes{C}
\bigskip
\vfill

\clearpage

\footnotesize

\lohead{\textsc{register}}

% Definiere theindex-Environment komplett neu ohne reledmac
\makeatletter
\renewenvironment{theindex}{%
  \section*{\indexname}%
  \setlength{\parindent}{0pt}%
  \setlength{\parskip}{0pt plus 0.3pt}%
  \let\item\@idxitem
}{%
  \clearpage
}
\makeatother

\IfFileExists{\jobname-pw.ind}{\input{\jobname-pw.ind}}{}

\end{document}

      