%% latex-korrekturansicht-vorspann.tex
%% Vorspann für die Korrekturansicht.
%% Lädt die gemeinsame Datei latex-vorspann.tex mit gesetztem Schalter.

\newif\ifkorrekturansicht
\korrekturansichttrue

\input{../tex-inputs/latex-vorspann}


\section[Stefan Zweig an Arthur Schnitzler, 21. 12. {[}1929{]}]{L03679 Stefan Zweig an Arthur Schnitzler, 21. 12. {[}1929{]}}
\nopagebreak\mylabel{L03679v}
\rehead{ }\normalsize\beginnumbering\briefempfaengerindex{Schnitzler, Arthur@\textsc{Schnitzler, Arthur}!zzzZweig, Stefan@\emph{von Stefan Zweig}!1929-12-211@{21. 12. {[}1929{]}}|(be}
\toendnotes[C]{\smallbreak\pagebreak[2]}
\correspDesc{Versand  durch Stefan Zweig am 21. 12. [1929] in Salzburg
\newline{}Erhalt  durch Arthur Schnitzler am 21. 12. [1929] in Wien}\toendnotes[C]{\smallbreak}
\Standort{CUL, Schnitzler, B 118.}
\physDesc{Telegramm, 1 Blatt, 1 Seite, 153 Zeichen
\newline{}maschinell
\newline{}Versand: 1) Stempel: »\nobreak{}21/XII, S 207 \textcolor{blue}{Horak}\pwindex{Horak @\textsc{Horak}, \emph{Telegrafenbeamter/Telegrafenbeamtin}|pw}\nobreak{}«.   2) Stempel: »\nobreak{}21 DEC, 15\textsuperscript{9}, ausgefertigt\nobreak{}«. 
\newline{}Schnitzler: mit rotem Buntstift datiert: »21/12 {[}1{]}929« }\toendnotes[C]{\smallbreak}\pstart{}{\pb}= artur schnitzler \textcolor{pink}{deutsches volkstheater wien}\oindex{Wien@\textbf{Wien}!VII., Neubau@\textbf{VII., Neubau}!Volkstheater@\textbf{Volkstheater}, \emph{Theater}|pw}{}\ledrightnote{\textcolor{pink}{Volkstheater}} = \pend{}{\bigskip}\vspace{1em}
\pstart
           \centering{}\textcolor{pink}{{\pb}salzburg}\oindex{Salzburg@\textbf{Salzburg}, \emph{Verwaltungsgebiet}|pw}{}\ledrightnote{\textcolor{pink}{Salzburg}} 4+720 18
                     21{ }14/10\pend
           \vspace{0.5em}
\pstart
           herzlichst und mit den innigsten wuenschen \label{K_L03679-1v}\edtext{heute}{\lemma{\textnormal{\emph{heute}}}\Cendnote{\textnormal{Am 21. 12. 1929 hatte \emph{\textcolor{green}{Im Spiel der Sommerlüfte}\pwindex{Schnitzler, Arthur 15. 5. 1862 Wien – 21. 10. 1931 ebd.@\textsc{Schnitzler, Arthur} (15. 5. 1862 Wien – 21. 10. 1931 ebd.), \emph{Schriftsteller, Mediziner}!Im Spiel der Sommerlüfte. In drei Aufzügen@\strich\emph{Im Spiel der Sommerlüfte. In drei Aufzügen}|pwk}} Uraufführung am \emph{\textcolor{brown}{Volkstheater}\orgindex{Volkstheater@Volkstheater|pwk}}. }}}\label{K_L03679-1} mit ihnen ihr getreuer
                  \spacefill\mbox{stefan zweig =}\pend
           \selectlanguage{ngerman}\endnumbering\briefempfaengerindex{Schnitzler, Arthur@\textsc{Schnitzler, Arthur}!zzzZweig, Stefan@\emph{von Stefan Zweig}!1929-12-211@{21. 12. {[}1929{]}}|)be}\mylabel{L03679h}
\begin{anhang}
\end{anhang}\normalsize

\doendnotes{C}
\bigskip
\vfill

\clearpage

\footnotesize

\lohead{\textsc{register}}

% Definiere theindex-Environment komplett neu ohne reledmac
\makeatletter
\renewenvironment{theindex}{%
  \section*{\indexname}%
  \setlength{\parindent}{0pt}%
  \setlength{\parskip}{0pt plus 0.3pt}%
  \let\item\@idxitem
}{%
  \clearpage
}
\makeatother

\IfFileExists{\jobname-pw.ind}{\input{\jobname-pw.ind}}{}

\end{document}

      