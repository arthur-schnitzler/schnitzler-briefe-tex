%% latex-korrekturansicht-vorspann.tex
%% Vorspann für die Korrekturansicht.
%% Lädt die gemeinsame Datei latex-vorspann.tex mit gesetztem Schalter.

\newif\ifkorrekturansicht
\korrekturansichttrue

\input{../tex-inputs/latex-vorspann}


\renewcommand{\erwaehntePersonen}{Personen: Peter Dorner, Paul Goldmann, Clementine Goldmann, Hugo von Hofmannsthal, Olga Schnitzler, Heinrich Schnitzler, Elisabeth Steinrück}
\renewcommand{\erwaehnteInstitutionen}{Institutionen: Schiller-Theater}
\renewcommand{\erwaehnteOrte}{Orte: Berlin, Dessauer Straße, Hinterbrühl, Welsberg-Taisten, Wien}
\renewcommand{\erwaehnteWerke}{}
\section[ Paul Goldmann an Olga Gussmann, 9. 7. {[}1902{]}]{Paul Goldmann an Olga Gussmann, 9. 7. {[}1902{]}}
\nopagebreak\mylabel{v}
\rehead{ }\normalsize\beginnumbering\briefempfaengerindex{Schnitzler, Olga@\textsc{Schnitzler, Olga}!zzzGoldmann, Paul@\emph{von Paul Goldmann}!1902-07-092@{9. 7. {[}1902{]}}|(be}
\toendnotes[C]{\smallbreak\pagebreak[2]}\Standort{DLA, A:Schnitzler, HS.NZ85.1.5247.}
\physDesc{Brief, 1 Blatt, 4 Seiten, 1819 Zeichen
\newline{}Handschrift: blaue Tinte, deutsche Kurrent}\toendnotes[C]{\smallbreak}
\pstart
           \noindent{}\raggedleft{}{\pb}\textcolor{gray}{\textbf{\textcolor{pink}{DESSAUERSTRASSE 19}{}\ledrightnote{\textcolor{pink}{Dessauer Straße}}}}\pend
           
\pstart
           \textcolor{pink}{Berlin}{}\ledrightnote{\textcolor{pink}{Berlin}}, 9. Juli.\pend
           
\pstart\center{}Liebe Freundin,\pend
\pstart
           Bitte, laſſen Sie das Danken ſein. Das war doch Alles ſelbſtverſtändlich. Es iſt noch
               die erſte und einfachſte Pflicht der Freundſchaft, in wichtigen Lebensangelegenheiten
                  \label{K_L03532-1v}\edtext{Beiſtand}{\lemma{\textnormal{\emph{Beiſtand}}}\Cendnote{\textnormal{siehe Paul Goldmann an Arthur Schnitzler, 16. 6. [1902]}}}\label{K_L03532-1h} zu leiſten.\pend
           
\pstart
           Ihre lieben \label{K_L03532-2v}\edtext{Mittheilungen über 
               \textcolor{blue}{\textsc{Peter Dorner}}{}\ledrightnote{\textcolor{blue}{Peter Dorner}}}{\lemma{\textnormal{\emph{Mittheilungen … Dorner}}}\Cendnote{\textnormal{\textcolor{blue}{Arthur Schnitzler} hatte
                  den Kunstschmied am 4. 7. 1902 in dessen Atelier aufgesucht.}}}\label{K_L03532-2h}{ }\textsc{etc.} haben
               mich ſehr intereſſirt. Nur hätte ich gern auch etwas Näheres über Ihr Ergehen
               gehört.\pend
           
\pstart
           Daß unſer liebes \label{K_L03532-3v}\edtext{\textsc{\textcolor{pink}{Welsberg}{}\ledrightnote{\textcolor{pink}{Welsberg-Taisten}}} von \textsc{\textcolor{blue}{Hoffmannsthal}{}\ledrightnote{\textcolor{blue}{Hugo von Hofmannsthal}}} »entdeckt«}{\lemma{\textnormal{\emph{Welsberg … »entdeckt«}}}\Cendnote{\textnormal{\textcolor{blue}{Hugo von Hofmannsthal} reiste am 4. 7. 1902 gemeinsam
                  mit \textcolor{blue}{Schnitzler} nach \textcolor{pink}{Welsberg} und blieb nach \textcolor{blue}{Schnitzler}s Abreise ein paar Tage länger (siehe Arthur Schnitzler an Hermann Bahr, [9. 7. 1902]).}}}\label{K_L03532-3h} worden iſt,
               thut mir leid. Es wird jetzt ein literariſcher Ort werden – obwohl \strikeout{es}{ }{\pb}es doch ein beſſeres Schickſal verdient hätte.\pend
           
\pstart
           Meine \textcolor{blue}{Mutter}{}\ledrightnote{{$\rightarrow$}\textcolor{blue}{Clementine Goldmann}} hat ſich ſehr
               über Ihre und \textsc{\textcolor{blue}{Liesl}{}\ledrightnote{\textcolor{blue}{Elisabeth Steinrück}}s} Grüße gefreut und erwidert ſie
               auf das Herzlichſte.\pend
           
\pstart
           Bitte, grüßen Sie meinen lieben \textsc{\textcolor{blue}{Arthur}{}\ledrightnote{}}, wenn er \label{K_L03532-4v}\edtext{morgen}{\lemma{\textnormal{\emph{morgen}}}\Cendnote{\textnormal{\textcolor{blue}{Goldmann} war nicht am aktuellen Stand, \textcolor{blue}{Schnitzler} war bereits seit 8. 7. 1902 wieder in
                     \textcolor{pink}{Wien}.}}}\label{K_L03532-4h} zurückkommt, vielmals von mir.
               Ich \strikeout{\textcolor{gray}{bed}} danke ihm für ſeine Karten von unterwegs und hoffe, bald Ausführlicheres von
               ihm zu hören.\pend
           
\pstart
           Wenn Ihnen der blöde \textcolor{blue}{Fratz}{}\ledrightnote{{$\rightarrow$}\textcolor{blue}{Elisabeth Steinrück}}
               (ich meine natürlich \textsc{\textcolor{blue}{Liesl}{}\ledrightnote{\textcolor{blue}{Elisabeth Steinrück}}}) erzählt hat, daß ich über Sie »geſchimpft« habe, ſo hat ſie wieder einmal {\pb}geſprochen, was \textcolor{blue}{ſie}{}\ledrightnote{{$\rightarrow$}\textcolor{blue}{Elisabeth Steinrück}} nicht verantworten kann. Ich habe ihr nur geſagt (weil ſie
               mir durch Äußerungen und Verhalten dazu Anlaß gegeben hatte), was ich auch Ihnen
               ſchon geſagt habe: wie wenig Sie \textcolor{blue}{Beide}{}\ledrightnote{{$\rightarrow$}\textcolor{blue}{Elisabeth Steinrück}} mich verſtehen und wie ſehr es \strikeout{mich}
               mir leid thut, daß ich gerade i\substVorne{}\textsuperscript{m}\substDazwischen{}n\substHinten{} einem Kreiſe, dem ich ſo nahe ſtehe, ſo wenig Verſtändniß finde. An Ihrer
               freundſchaftlichen Geſinnung für mich zweifle ich keinen Augenblick, ebenſo wie Sie
               hoffentlich nicht an der meinigen zweifeln. Das Wort »Haß« ſollte in einem Briefe,
               den Sie mir ſchreiben, wirklich nicht ſtehen.\pend
           
\pstart
           {\pb}Es thut mir leid, daß ich nicht \label{K_L03532-5v}\edtext{auch Ihnen zu einem Engagement an einem \textcolor{pink}{Berlin}{}\ledrightnote{\textcolor{pink}{Berlin}}er Theater verhelfen}{\lemma{\textnormal{\emph{auch … verhelfen}}}\Cendnote{\textnormal{Bezug auf \textcolor{blue}{Elisabeth
                     Gussmann}s Engagement am \emph{\textcolor{brown}{Schiller-Theater}} ab dem 1. 9. 1902, siehe Paul Goldmann an Arthur Schnitzler, 16. 6. [1902]}}}\label{K_L03532-5h} kann; aber ich \strikeout{\textcolor{gray}{d}} denke mir, daß Sie \label{K_L03532-6v}\edtext{Beſſeres
                  gefunden}{\lemma{\textnormal{\emph{Beſſeres
                  gefunden}}}\Cendnote{\textnormal{Er meint, die Rolle als \textcolor{blue}{Schnitzler}s Partnerin und Mutter des
                  gemeinsamen Sohnes \textcolor{blue}{Heinrich}, dessen Geburt
                  bevorstand, wäre wichtiger, als ihre Karriere.}}}\label{K_L03532-6h} haben, als Ihnen die größte
               Stellung an der größten Bühne jemals hätte bieten können.\pend
           
\pstart
           Mit herzlichen Grüßen an Sie und \textsc{\textcolor{blue}{Liesl}{}\ledrightnote{\textcolor{blue}{Elisabeth Steinrück}}} (der ich für ihren Brief danke) bin ich {\\[\baselineskip]}Ihr ergebener {\\[\baselineskip]}\spacefill\mbox{Dr. Paul Goldmann.}\pend
           \leftskip=0em{}\endnumbering\briefempfaengerindex{Schnitzler, Olga@\textsc{Schnitzler, Olga}!zzzGoldmann, Paul@\emph{von Paul Goldmann}!1902-07-092@{9. 7. {[}1902{]}}|)be}\mylabel{h}  \normalsize

\doendnotes{C}
\bigskip
\vfill

\clearpage

\footnotesize

\lohead{\textsc{register}}

% Definiere theindex-Environment komplett neu ohne reledmac
\makeatletter
\renewenvironment{theindex}{%
  \section*{\indexname}%
  \setlength{\parindent}{0pt}%
  \setlength{\parskip}{0pt plus 0.3pt}%
  \let\item\@idxitem
}{%
  \clearpage
}
\makeatother

\IfFileExists{\jobname-pw.ind}{\input{\jobname-pw.ind}}{}

\end{document}

      