%% latex-korrekturansicht-vorspann.tex
%% Vorspann für die Korrekturansicht.
%% Lädt die gemeinsame Datei latex-vorspann.tex mit gesetztem Schalter.

\newif\ifkorrekturansicht
\korrekturansichttrue

\input{../tex-inputs/latex-vorspann}


\renewcommand{\erwaehntePersonen}{Personen: Karl Emil Franzos}
\renewcommand{\erwaehnteInstitutionen}{Institutionen: Deutsche Dichtung}
\renewcommand{\erwaehnteOrte}{Orte: Berlin, Berlin W, Frankgasse 1, Friedrich Wilhelm-Strasse 6, IX., Alsergrund, Wien}
\renewcommand{\erwaehnteWerke}{}
\section[ Karl Emil Franzos an Arthur Schnitzler, 8. 9. 1900]{Karl Emil Franzos an Arthur Schnitzler, 8. 9. 1900}
\nopagebreak\mylabel{v}
\rehead{ }\normalsize\beginnumbering\briefempfaengerindex{Schnitzler, Arthur@\textsc{Schnitzler, Arthur}!zzzFranzos, Karl Emil@\emph{von Karl Emil Franzos}!1900-09-082@{8. 9. 1900}|(be}
\toendnotes[C]{\smallbreak\pagebreak[2]}\Standort{DLA, A:Schnitzler, HS.1985.1.3025.}
\physDesc{Postkarte, 661 Zeichen
\newline{}Handschrift: schwarze Tinte, deutsche Kurrent
\newline{}Versand: 1) Stempel: »\nobreak{}\oindex{Berlin@\textbf{Berlin}, \emph{P.PPLC}|pwk}Berl{[}lin{]}
                                       10, 8. 9. 00, 8–9\nobreak{}«.   2) Stempel: »\nobreak{}\oindex{IX., Alsergrund@\textbf{IX., Alsergrund}, \emph{A.ADM3}|pwk}Wien 9/3, 10. 9. 00, 8.V, Bestellt\nobreak{}«. }\pstart{}{\pb}Herrn Dr. \textcolor{blue}{A.
                     Schnitzler}{}\ledrightnote{}\pend{}\pstart{}\textcolor{pink}{Wien IX}{}\ledrightnote{\textcolor{pink}{IX., Alsergrund}}\pend{}\pstart{}\textcolor{pink}{Frankgasse 1}{}\ledrightnote{\textcolor{pink}{Frankgasse 1}}.\pend{}
{\bigskip}
\pstart
           \centering{}{\pb}\textcolor{gray}{\textbf{Redaction der »\textcolor{brown}{Deutschen
                        Dichtung}{}\ledrightnote{\textcolor{brown}{Deutsche Dichtung}}«}}\pend
           
\pstart
           \raggedleft{}\textcolor{gray}{\textbf{\textbf{\emph{\textcolor{pink}{Berlin W. 10}{}\ledrightnote{\textcolor{pink}{Berlin W}}}},}}{ }\substVorne{}\textsuperscript{9}\substDazwischen{}8\substHinten{}. IX \textcolor{gray}{\textbf{1\strikeout{8}9}}00\pend
           
\pstart
           \raggedleft{}\textcolor{gray}{\textbf{\emph{\textcolor{pink}{Friedrich Wilhelm-Strasse 6}{}\ledrightnote{\textcolor{pink}{Friedrich Wilhelm-Strasse 6}}}.}}\pend
           
\pstart{}Verehrter Herr Doctor!\pend
\pstart
           Es thut mir ſehr leid, daß zunächſt nichts von Ihnen zu haben ist, doch hoffe ich auf
               Ihre freundliche Zuſage, beim Nächſten an mich zu denken. Wir kö{\geminationn}en Längeres und Kürzeres brauchen; haben Sie was, ſo
               ſchicken Sie s und fügen Sie Ihren Honorar-Anspruch bei; wir ko{\geminationm}en da{\geminationn} schon zu einem
               Gehalt etc{[}.{]} Am liebſten brächte ich ein Drama von Ihnen; da
               Ihnen dadurch weder die Bühnen-Tantième noch das Honorar der Buchgausgabe irgend
               tangirt wär, ſo iſt dies vielleicht auch Ihnen das Genehmſte!\pend
           
\pstart
           Mit beſten Empfehlungen Ihr ſehr ergebner{\\[\baselineskip]}\spacefill\mbox{K. E. Franzos}\pend
           \leftskip=0em{}
\pstart
           \noindent{}\textsc{Herrn Dr. A. Schnitzler, \textcolor{pink}{Wien
                        IX.}{}\ledrightnote{\textcolor{pink}{IX., Alsergrund}}\textcolor{pink}{Frankgasse 1}{}\ledrightnote{\textcolor{pink}{Frankgasse 1}}.}\pend
           \endnumbering\briefempfaengerindex{Schnitzler, Arthur@\textsc{Schnitzler, Arthur}!zzzFranzos, Karl Emil@\emph{von Karl Emil Franzos}!1900-09-082@{8. 9. 1900}|)be}\mylabel{h}  \normalsize

\doendnotes{C}
\bigskip
\vfill

\clearpage

\footnotesize

\lohead{\textsc{register}}

% Definiere theindex-Environment komplett neu ohne reledmac
\makeatletter
\renewenvironment{theindex}{%
  \section*{\indexname}%
  \setlength{\parindent}{0pt}%
  \setlength{\parskip}{0pt plus 0.3pt}%
  \let\item\@idxitem
}{%
  \clearpage
}
\makeatother

\IfFileExists{\jobname-pw.ind}{\input{\jobname-pw.ind}}{}

\end{document}

      