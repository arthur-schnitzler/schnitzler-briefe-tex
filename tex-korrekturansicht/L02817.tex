%% latex-korrekturansicht-vorspann.tex
%% Vorspann für die Korrekturansicht.
%% Lädt die gemeinsame Datei latex-vorspann.tex mit gesetztem Schalter.

\newif\ifkorrekturansicht
\korrekturansichttrue

\input{../tex-inputs/latex-vorspann}


               \section[ Paul Goldmann an Arthur Schnitzler, 13. 7. {[}1897{]}]{Paul Goldmann an Arthur Schnitzler, 13. 7. {[}1897{]}}\nopagebreak\mylabel{v}\rehead{ }\normalsize\beginnumbering\briefempfaengerindex{Schnitzler, Arthur@\textsc{Schnitzler, Arthur}!zzzGoldmann, Paul@\emph{von Paul Goldmann}!1897-07-132@{13. 7. {[}1897{]}}|(be} \toendnotes[C]{\smallbreak\pagebreak[2]} \Standort{DLA, A:Schnitzler, HS.NZ85.1.3167.}
\physDesc{Brief, 1 Blatt, 3 Seiten
\newline{}Handschrift: blaue Tinte, deutsche Kurrent
\newline{}Schnitzler: mit Bleistift das Jahr »97« vermerkt }\toendnotes[C]{\smallbreak}\pstart
           \noindent{}{\pb}\textcolor{gray}{\textbf{\textbf{\textcolor{brown}{Frankfurter Zeitung}{}\ledrightnote{\textcolor{brown}{Frankfurter Zeitung}}}}}\pend
           \pstart
           \textcolor{gray}{\textbf{(\textcolor{brown}{\begin{otherlanguage}{french}Gazette de Francfort\end{otherlanguage}}{}\ledrightnote{\textcolor{brown}{Frankfurter Zeitung}}).}}\pend
           \pstart
           \textcolor{gray}{\textbf{\textbf{\begin{otherlanguage}{french}Fondateur M.\end{otherlanguage}{ }\textcolor{blue}{L. Sonnemann}{}\ledrightnote{\textcolor{blue}{Leopold Sonnemann}}.}}}\pend
           \pstart
           \begin{otherlanguage}{french}\textcolor{gray}{\textbf{Journal politique, financier,}}\end{otherlanguage}\hfill \textsc{\textcolor{pink}{Paris}{}\ledrightnote{\textcolor{pink}{Paris}}}, 13. Juli.\pend
           \pstart
           \begin{otherlanguage}{french}\textcolor{gray}{\textbf{commercial et littéraire.}}\end{otherlanguage}\pend
           \pstart
           \begin{otherlanguage}{french}\textcolor{gray}{\textbf{\textbf{Paraissant trois fois par jour.}}}\end{otherlanguage}\pend
           \pstart
           \begin{otherlanguage}{french}\textcolor{gray}{\textbf{\textbf{Bureau à \textcolor{pink}{Paris}{}\ledrightnote{\textcolor{pink}{Paris}}}}}\end{otherlanguage}\pend
           \pstart
           \begin{otherlanguage}{french}\textcolor{gray}{\textbf{\textbf{\textcolor{pink}{10 Rue de la Bourse}{}\ledrightnote{\textcolor{pink}{rue de la Bourse}}.}}}\end{otherlanguage}\pend
           \pstart\center{}Mein lieber Freund,\pend\pstart
           Eine ausführliche Beantwortung Deiner lieben Briefe behalte ich mir für demnächſt
               vor. Heut nur in aller Eile:\pend
           \pstart
           Ich habe geſtern von der \textcolor{brown}{Redaction}{}\ledrightnote{→\textcolor{brown}{Frankfurter Zeitung}} meinen Urlaub für Anfang Auguſt verlangt. Ob ich ihn bekommen werde und ob man mich nicht zwingen wird, bis Ende Auguſt (während der \label{K_L02817-1v}\edtext{Reiſe des \textcolor{blue}{Präſident}{}\ledrightnote{→\textcolor{blue}{Félix Faure}}en}{\lemma{\textnormal{\emph{Reiſe des Präſidenten}}}\Cendnote{\textnormal{siehe Paul Goldmann an Arthur Schnitzler, 15. 6. [1897]}}}\label{K_L02817-1h} der \textcolor{pink}{Republik}{}\ledrightnote{→\textcolor{pink}{Frankreich}})
               hierzubleiben, weiß ich nicht. Jedenfalls habe ich mir in \label{K_L02817-2v}\edtext{\textsc{\textcolor{brown}{\textcolor{pink}{Bayreuth}{}\ledrightnote{\textcolor{pink}{Bayreuth}}}{}\ledrightnote{→\textcolor{brown}{Bayreuther Festspiele}}}}{\lemma{\textnormal{\emph{Bayreuth}}}\Cendnote{\textnormal{siehe Paul Goldmann an Arthur Schnitzler, 15. 6. [1897]}}}\label{K_L02817-2h} Sitze beſtellt und deren {\pb}zwei für die \textsc{\textcolor{green}{Parsifal}{}\ledrightnote{\textcolor{green}{Parsifal}}}-Aufführung vom 11. Auguſt bekommen. Wenn Du
               nicht mitkommen kannſt, ſo frage doch den \label{K_L02817-3v}\edtext{\textsc{\textcolor{blue}{Richard}{}\ledrightnote{\textcolor{blue}{Richard Beer-Hofmann}}}}{\lemma{\textnormal{\emph{Richard}}}\Cendnote{\textnormal{\textcolor{blue}{Schnitzler} hatte \textcolor{blue}{Richard Beer-Hofmann} bereits wegen einer früheren
                  Vorstellung gefragt, woraufhin sich \textcolor{blue}{Beer-Hofmann} aber nicht festlegen wollte (vgl. Arthur Schnitzler an Richard Beer-Hofmann,
               12. 6. 1897 und Richard Beer-Hofmann an Arthur Schnitzler,
               13. 6. 1897). Aus einem Brief \textcolor{blue}{Goldmann}s 
                  an \textcolor{blue}{Beer-Hofmann} vom 24. 7. {[}1897{]}
                  ist zu entnehmen, dass er hoffte, ihn bereits in \textcolor{pink}{München} zu treffen. Nicht nur schließt das eine Teilnahme in \textcolor{pink}{Bayreuth} aus,
                  sondern auch dieses Treffen dürften nicht stattgefunden haben. \textcolor{blue}{Beer-Hofmann}s
                  »Daten« ist zu entnehmen, dass er 1897 nicht ins Ausland reiste (vgl. Eugene
                     Weber: \emph{Richard Beer-Hofmann: Daten}. In: \emph{Modern Austrian
                        Literature}, Jg. 17, 1984, Nr. 2, S. 13–42, hier:
                     S. 22.}}}\label{K_L02817-3h}, ob er nicht den zweiten Sitz benutzen will? Er müßte mir
               aber \uuline{ſofort} antworten, da ich bis 20. Juli Beſcheid ſagen muß. Ginge ich nun nach \textcolor{pink}{Bayreuth}{}\ledrightnote{\textcolor{pink}{Bayreuth}}, was ſollte ich dann von 11 bis 20. Auguſt
               anfangen, ehe Du nach \label{K_L02817-4v}\edtext{\textsc{\textcolor{pink}{Muenchen}{}\ledrightnote{\textcolor{pink}{München}}}}{\lemma{\textnormal{\emph{Muenchen}}}\Cendnote{\textnormal{\textcolor{blue}{Schnitzler} verreiste im
                     Sommer 1897 nicht nach \textcolor{pink}{München}.}}}\label{K_L02817-4h} kommen kannſt? Auch liegt es mir daran, möglichſt viel
               Zeit in guter Luft, im Gebirge zu verbringen, nicht in der großen Stadt. {\pb}Wäreſt Du nicht für \textcolor{pink}{Süd-Tirol}{}\ledrightnote{\textcolor{pink}{Südtirol}} zu haben? Das iſt doch das herrlichſte \textcolor{pink}{Land}{}\ledrightnote{→\textcolor{pink}{Südtirol}} der Welt, und ich begreife nicht, daß
               Ihr das ſo wenig mögt.\pend
           \pstart
           Sobald ich von meiner \textcolor{brown}{Redaction}{}\ledrightnote{→\textcolor{brown}{Frankfurter Zeitung}}
               Beſcheid habe, ſchreibe ich Dir.\pend
           \pstart
           Viele treue Grüße!\pend
           \pstart
           Dein {\\[\baselineskip]}\spacefill\mbox{Paul Goldmann}\pend
           \leftskip=0em{}\pstart
           \noindent{}Ich habe nicht an \label{K_L02817-8v}\edtext{\textsc{\textcolor{blue}{\textcolor{pink}{Andermatt}{}\ledrightnote{\textcolor{pink}{Andermatt}}}{}\ledrightnote{→\textcolor{blue}{Marie Reinhard}}}}{\lemma{\textnormal{\emph{Andermatt}}}\Cendnote{\textnormal{siehe Paul Goldmann an Arthur Schnitzler, 2. 7. [1897]}}}\label{K_L02817-8h} ſchreiben können, weil ich nicht weiß, wie ich adreſſiren ſoll. Soll ich
                     »\textsc{Madame}« ſchreiben? Und welchen Namen?\pend
           \endnumbering\briefempfaengerindex{Schnitzler, Arthur@\textsc{Schnitzler, Arthur}!zzzGoldmann, Paul@\emph{von Paul Goldmann}!1897-07-132@{13. 7. {[}1897{]}}|)be}\mylabel{h}\begin{anhang}\end{anhang}\normalsize

\doendnotes{C}
\bigskip
\vfill

\clearpage

\footnotesize

\lohead{\textsc{register}}

% Definiere theindex-Environment komplett neu ohne reledmac
\makeatletter
\renewenvironment{theindex}{%
  \section*{\indexname}%
  \setlength{\parindent}{0pt}%
  \setlength{\parskip}{0pt plus 0.3pt}%
  \let\item\@idxitem
}{%
  \clearpage
}
\makeatother

\IfFileExists{\jobname-pw.ind}{\input{\jobname-pw.ind}}{}

\end{document}

      