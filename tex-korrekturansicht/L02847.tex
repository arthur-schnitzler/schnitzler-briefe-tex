%% latex-korrekturansicht-vorspann.tex
%% Vorspann für die Korrekturansicht.
%% Lädt die gemeinsame Datei latex-vorspann.tex mit gesetztem Schalter.

\newif\ifkorrekturansicht
\korrekturansichttrue

\input{../tex-inputs/latex-vorspann}


               \section[ Paul Goldmann an Arthur Schnitzler, 4. 4. 1898]{Paul Goldmann an Arthur Schnitzler, 4. 4. 1898}\nopagebreak\mylabel{v}\rehead{ }\normalsize\beginnumbering\briefempfaengerindex{Schnitzler, Arthur@\textsc{Schnitzler, Arthur}!zzzGoldmann, Paul@\emph{von Paul Goldmann}!1898-04-041@{4. 4. 1898}|(be} \toendnotes[C]{\smallbreak\pagebreak[2]} \Standort{DLA, A:Schnitzler, HS.NZ85.1.3168.}
\physDesc{Brief, 1 Blatt, 4 Seiten
\newline{}Handschrift: schwarze Tinte, lateinische Kurrent}\toendnotes[C]{\smallbreak}\pstart
           \noindent{}{\pb}\textcolor{brown}{\textcolor{gray}{\textbf{\textsc{Frankfurter Zeitung}}}}{}\ledrightnote{\textcolor{brown}{Frankfurter Zeitung}}\hfill \substVorne{}\textsuperscript{\textcolor{gray}{\textbf{\textcolor{pink}{Frankfurt a. M.}{}\ledrightnote{\textcolor{pink}{Frankfurt am Main}},}}}{\allowbreak}\substDazwischen{}\textsc{\textcolor{pink}{Genua}{}\ledrightnote{\textcolor{pink}{Genua}}}\substHinten{}{ }4. April \textcolor{gray}{\textbf{189}}8.\pend
           \pstart
           \textsc{\textcolor{gray}{\textbf{und}}}\pend
           \pstart
           \textcolor{gray}{\textbf{\textsc{Handelsblatt.}}}\pend
           \pstart
           \textcolor{gray}{\textbf{\textsc{\textcolor{brown}{Redaction}{}\ledrightnote{→\textcolor{brown}{Frankfurter Zeitung}}.\footnote{\noindent{}\textcolor{gray}{\textbf{\textsc{Für die \textcolor{brown}{Redaktion} bestimmte Briefe und Sendungen
                                    wolle man \so{nicht} an die Person eines
                                    Redakteurs, sondern stets \textbf{an die Redaktion der
                                          \textcolor{brown}{Frankfurter Zeitung}} adressiren.}}}}}}}\pend
           \pstart
           \textcolor{gray}{\textbf{\textsc{Telegramm-Adresse:}}}\pend
           \pstart
           \textcolor{gray}{\textbf{\textsc{\textcolor{brown}{Zeitung}{}\ledrightnote{→\textcolor{brown}{Frankfurter Zeitung}}{ }\textcolor{pink}{Frankfurt Main}{}\ledrightnote{\textcolor{pink}{Frankfurt am Main}}.}}}\pend
           \pstart{}Mein lieber Freund,\pend\pstart
           Tauſend Dank für Deinen ſo lieben Brief! Es thut wohl, zum Abſchied ſo gute Worte zu
               hören.\pend
           \pstart
           Ich gehe morgen{ }früh aufs Schiff, fahre zuerſt nach \textsc{\textcolor{pink}{Hongkong}{}\ledrightnote{\textcolor{pink}{Hong Kong}}} (5. Mai), von dort den \strikeout{Per}{ }\textcolor{pink}{Perlfluß}{}\ledrightnote{\textcolor{pink}{Perlfluss}} hinauf nach \textsc{\textcolor{pink}{Canton}{}\ledrightnote{\textcolor{pink}{Guangzhou}}}, zurück nach \textsc{\textcolor{pink}{Hongkong}{}\ledrightnote{\textcolor{pink}{Hong Kong}}}, zur See nach \textsc{\textcolor{pink}{Shanghai}{}\ledrightnote{\textcolor{pink}{Shanghai}}}, von da den \textsc{\textcolor{pink}{Jang-tse-kiang}{}\ledrightnote{\textcolor{pink}{Jangtsekiang}}} hinauf, vielleicht bis \textsc{\textcolor{pink}{Hankau}{}\ledrightnote{\textcolor{pink}{Wuhan}}}, zurück nach \textsc{\textcolor{pink}{Shanghai}{}\ledrightnote{\textcolor{pink}{Shanghai}}}, von da nach \textsc{\textcolor{pink}{Kiao-tschau}{}\ledrightnote{\textcolor{pink}{Kiautschou}}}, von da nach \textsc{\textcolor{pink}{Tientsin}{}\ledrightnote{\textcolor{pink}{Tianjin}}}, von da nach \textsc{\strikeout{P\textcolor{gray}{e}}}{ }\textsc{\textcolor{pink}{Peking}{}\ledrightnote{\textcolor{pink}{Peking}}}, zurück nach \strikeout{\textsc{\textcolor{pink}{Peking}{}\ledrightnote{\textcolor{pink}{Peking}}}}{ }{\pb}\textsc{\textcolor{pink}{Tientsin}{}\ledrightnote{\textcolor{pink}{Tianjin}}}, von da zur See nach \textsc{\textcolor{pink}{Chemulpo}{}\ledrightnote{\textcolor{pink}{Incheon}}} (\textsc{\textcolor{pink}{Korea}{}\ledrightnote{→\textcolor{pink}{Südkorea}}}) und landeinwärts bis \textsc{\textcolor{pink}{Säul}{}\ledrightnote{\textcolor{pink}{Seoul}}}, von da nach \textsc{\textcolor{pink}{Japan}{}\ledrightnote{\textcolor{pink}{Japan}}}.\pend
           \pstart
           Das iſt der vorläufige Entwurf. Bitte, ſchreib’ mir nach \textsc{\textcolor{pink}{Shanghai}{}\ledrightnote{\textcolor{pink}{Shanghai}}, \textcolor{brown}{Deutsches Post Amt (\begin{otherlanguage}{english}German Post
                        Office\end{otherlanguage})}{}\ledrightnote{→\textcolor{brown}{Deutsche Post in China}} Poste Restante}. Ich bin dort vorausſichtlich
               Ende Mai aber \strikeout{es wird}
               während der ganzen Dauer meiner Reiſe wird meine Adreſſe ſo lauten, da ich mir von
                  \textsc{\textcolor{pink}{Shanghai}{}\ledrightnote{\textcolor{pink}{Shanghai}}} immer die Briefe nachſchicken laſſen werde.\pend
           \pstart
           Was nach meiner Rückkehr ſein wird, weiß ich nicht. \textsc{\textcolor{pink}{Berlin}{}\ledrightnote{\textcolor{pink}{Berlin}}} wohl kaum. Es ſind noch andere Projecte in der {\pb}Luſt, aber das Alles wird ſich wohl zerſchlagen, und ich werde ins Joch nach \textsc{\textcolor{pink}{Paris}{}\ledrightnote{\textcolor{pink}{Paris}}} zurück müſſen.\pend
           \pstart
           Wie ſchön iſt \textsc{\textcolor{pink}{Genua}{}\ledrightnote{\textcolor{pink}{Genua}}}. Nie in meinem Leben habe ich ſolche Paläſte geſehen. Kennſt Du es? Die \textcolor{pink}{italien}{}\ledrightnote{\textcolor{pink}{Italien}}iſche \textsc{Renaissance} iſt doch unübertroffen, ſelbſt im Großartigen. Die \textcolor{pink}{fran}{}\ledrightnote{→\textcolor{pink}{Frankreich}}zöſiſche und \textcolor{pink}{deutſch}{}\ledrightnote{→\textcolor{pink}{Deutschland}}e Renaiſſance iſt nur
               nachempfunden.\pend
           \pstart
           Und dieſe liebe goldene Sonne! Armer Freund Du in Deinem Winter!\pend
           \pstart
           Ich umarme Dich im Geiſte, mein lieber Arthur, und grüße Dich noch einmal von ganzem
                  {\pb}Herzen! Ich will von unterwegs viel an Dich
               denken. Bleib’ mir gut, liebſter Freund!\pend
           \pstart
           Dein treuer {\\[\baselineskip]}\spacefill\mbox{Paul Goldmann.}\pend
           \leftskip=0em{}\pstart
           \noindent{}Viele herzliche Grüße an Deine \textcolor{blue}{Freundin}{}\ledrightnote{→\textcolor{blue}{Marie Reinhard}}!\pend
           \pstart
           Erhole Dich im Sommer und geh’ auch ein wenig in die Welt hinaus aus Deinem
                  Hypochondrie-Winkel, wo Du Dich mit ſchwarzen Gedanken eingeſponnen haſt! Du wirſt
                  ſehen, wie das Alles in der Sonne zerfliegt! Gerade geht ſie drüben über dem Meere
                  unter. Ich ſage Dir, draußen iſt Licht und Wärme!\pend
           \pstart
           Und nochmals Lebewohl!!!!\pend
           \endnumbering\briefempfaengerindex{Schnitzler, Arthur@\textsc{Schnitzler, Arthur}!zzzGoldmann, Paul@\emph{von Paul Goldmann}!1898-04-041@{4. 4. 1898}|)be}\mylabel{h}\begin{anhang}\end{anhang}\normalsize

\doendnotes{C}
\bigskip
\vfill

\clearpage

\footnotesize

\lohead{\textsc{register}}

% Definiere theindex-Environment komplett neu ohne reledmac
\makeatletter
\renewenvironment{theindex}{%
  \section*{\indexname}%
  \setlength{\parindent}{0pt}%
  \setlength{\parskip}{0pt plus 0.3pt}%
  \let\item\@idxitem
}{%
  \clearpage
}
\makeatother

\IfFileExists{\jobname-pw.ind}{\input{\jobname-pw.ind}}{}

\end{document}

      