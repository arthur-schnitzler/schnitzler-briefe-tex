%% latex-korrekturansicht-vorspann.tex
%% Vorspann für die Korrekturansicht.
%% Lädt die gemeinsame Datei latex-vorspann.tex mit gesetztem Schalter.

\newif\ifkorrekturansicht
\korrekturansichttrue

\input{../tex-inputs/latex-vorspann}


               \section[Paul Goldmann an Arthur Schnitzler, 27. 6. {[}1892{]}]{ Paul Goldmann an Arthur Schnitzler, 27. 6. {[}1892{]}}\nopagebreak\mylabel{v}\rehead{ }\normalsize\beginnumbering\briefempfaengerindex{Schnitzler, Arthur@\textsc{Schnitzler, Arthur}!zzzGoldmann, Paul@\emph{von Paul Goldmann}!1892-06-271@{27. 6. {[}1892{]}}|(be} \toendnotes[C]{\smallbreak\pagebreak[2]} \Standort{DLA, A:Schnitzler, HS.NZ85.1.3163.}
\physDesc{Brief, 2 Blätter, 6 Seiten
\newline{}Handschrift: schwarze Tinte, deutsche Kurrent
\newline{}Schnitzler: mit Bleistift das Jahr »92« vermerkt }\toendnotes[C]{\smallbreak}\pstart
           \noindent{}{\pb}\textcolor{gray}{\textbf{\textcolor{brown}{Frankfurter Zeitung}{}\ledrightnote{\textcolor{brown}{Frankfurter Zeitung}}.}}\pend
           \pstart
           \textcolor{gray}{\textbf{(\textcolor{brown}{Gazette de Francfort}{}\ledrightnote{\textcolor{brown}{Frankfurter Zeitung}}.)}}\pend
           \pstart
           \textcolor{gray}{\textbf{\begin{otherlanguage}{french}Directeur\end{otherlanguage}: \textbf{M. \textcolor{blue}{L. Sonnemann}{}\ledrightnote{\textcolor{blue}{Leopold Sonnemann}}}.}}\hfill \textsc{\textcolor{pink}{Paris}{}\ledrightnote{\textcolor{pink}{Paris}}}, 27. Juni.\pend
           \pstart
           \textcolor{gray}{\textbf{\begin{otherlanguage}{french}Journal politique, financier,\end{otherlanguage}}}\pend
           \pstart
           \textcolor{gray}{\textbf{\begin{otherlanguage}{french}commercial et litteraire.\end{otherlanguage}}}\pend
           \pstart
           \textcolor{gray}{\textbf{\begin{otherlanguage}{french}\textbf{Paraissant trois fois par jour}\end{otherlanguage}}}\pend
           \pstart
           \textcolor{gray}{\textbf{–}}\pend
           \pstart
           \textcolor{gray}{\textbf{\begin{otherlanguage}{french}\textbf{Bureaux à \textcolor{pink}{Paris}{}\ledrightnote{\textcolor{pink}{Paris}}:}\end{otherlanguage}}}\pend
           \pstart
           \textcolor{gray}{\textbf{\begin{otherlanguage}{french}\textbf{\textcolor{pink}{rue Richelieu 75}{}\ledrightnote{\textcolor{pink}{rue Richelieu}}.}\end{otherlanguage}}}\pend
           \pstart
           \centering{}Mein lieber Arthur!,\pend
           \pstart
           \noindent{}Mir ſcheint, wir haben uns im ſelben Moment hingeſetzt, um aneinander zu ſchreiben.
               Auch das ſoll als ein liebes Zeichen genommen werden. Wie unendlich, aus tiefſtem
               Herzen froh Du mich mit Deinem Brief gemacht haſt, kann ich Dir nicht ſagen. Ich bin
               ſo ſtolz, ſo ſtolz auf dieſe treue Freundſchaft, die Du mir entgegenbringſt. Und das
               iſt das einzige wirkliche Gut, das mir das Leben bisher geboten. Ich habe heut wieder einmal {\pb}nach
               langer Zeit ein warmes Aufwallen von Gück im Herzen gehabt und danke das Dir. Oh {\dots} doch laſſen wir die Gefühle. Mein Privatleben verlange
               nicht zu wiſſen. Ich wüßte auch nicht, wie ich es Dir ſchildern ſollte in ſeiner Öde
               und Verlaſſenheit. Ich bin ein armer einſamer Narr, und betrinke mich an Arbeit, um
               das auf Stunden zu vergeſſen – mein bewährtes Recept. Verkehr außer \textsc{\textcolor{blue}{Arthur Klein}{}\ledrightnote{\textcolor{blue}{Arthur Klein}}} nur ein ſeltſamer \label{K_L02699-6v}\edtext{\textcolor{blue}{Burſch}{}\ledrightnote{→\textcolor{blue}{?? [Dänischer Maler in Paris, 1892]}}}{\lemma{\textnormal{\emph{Burſch}}}\Cendnote{\textnormal{nicht identifiziert}}}\label{K_L02699-6h} von einem
               däniſchen Maler, viel mehr Millionärsſohn, der gern großer Künſtler werden möchte und
               an ſeinem Dilettantismus {\pb}und an unglücklicher Liebe
               zugrunde geht. Seltſamer, ſehr lieber Menſch, der ſich zweifellos in den nächſten
               Jahren erſchießen wird. Um ihn herum ein oder zwei \label{K_L02699-8v}\edtext{Freunde}{\lemma{\textnormal{\emph{Freunde}}}\Cendnote{\textnormal{nicht
                  identifiziert}}}\label{K_L02699-8h}, auch deutſche Millionärsſöhne, gutmüthig, mit künſtleriſchen
               Inſpirationen, inoffenſiv. \textsc{Arthur Schnitzler} iſt in dieſem
               Kreiſe ein bekannter Begriff; ich leſe Dich vor, ich ſchildere dich \textsc{etc.}{ }\textsc{etc.} In franzöſiſche Kreiſe {[}ist{]} nicht
               hineinzukommen. Der \label{K_L02699-2v}\edtext{\textsc{\begin{otherlanguage}{french}sale Prussien\end{otherlanguage}}}{\lemma{\textnormal{\emph{sale Prussien}}}\Cendnote{\textnormal{französisch: schmutziger Preuße}}}\label{K_L02699-2h}{ }\strikeout{iſt wie} klebt Einem wie ein Peſthauch an, vor dem
               ſich alle Thüren {\pb}verſperren{\dotsfour}\pend
           \pstart
           Thu’ mir den einzigen Gefallen, laß’ Dich nicht \label{K_L02699-4v}\edtext{in \textsc{\textcolor{pink}{Prag}{}\ledrightnote{\textcolor{pink}{Prag}}}}{\lemma{\textnormal{\emph{in Prag}}}\Cendnote{\textnormal{Über das ganze Jahr 1892 gab es Bemühungen, \emph{\textcolor{green}{Das
                     Märchen}} am \emph{\textcolor{brown}{Neuen Deutschen Theater}} in
                     \textcolor{pink}{Prag} aufzuführen. Am 4. 1. 1892 notierte \textcolor{blue}{Schnitzler} im \emph{\textcolor{green}{Tagebuch}} die Zusage. Das \textcolor{green}{Schauspiel} sollte im Oktober des Jahres aufgeführt werden
                     (vgl. A. S.: \emph{Tagebuch}, 6. 1. 1892, 6. 8. 1892). Letztendlich wurde
                  die Aufführung jedoch untersagt (vgl. A. S.: \emph{Tagebuch}, 9. 1. 1893, 12. 1. 1893). }}}\label{K_L02699-4h} aufführen! In \textsc{\textcolor{pink}{Prag}{}\ledrightnote{\textcolor{pink}{Prag}}} kann man Dich erſtens nicht verſtehen und zweitens nicht ſpielen. Die Sache muß
               Mißerfolg haben, und damit verdirbſt Du Dir dann Deine \textcolor{pink}{Berlin}{}\ledrightnote{\textcolor{pink}{Berlin}}er Aufführung. Warte ruhig ab! Glaube mir, Deine Zeit \uline{muß} kommen. Aber über \textsc{\textcolor{pink}{Prag}{}\ledrightnote{\textcolor{pink}{Prag}}} geht man nicht zur Höhe der Künſtlerſchaft{\dotsfour}\pend
           \pstart
           Es freut mich unſäglich zu hören, daß Du an der Arbeit biſt. Schaffe, liebſter
               Freund, und werde nicht {\pb}müde! Du biſt der Einzige von
               uns, der eine Zukunft hat!\pend
           \pstart
           Und \label{K_L02699-5v}\edtext{\uline{\textcolor{blue}{das}{}\ledrightnote{→\textcolor{blue}{Marie Glümer}}}}{\lemma{\textnormal{\emph{das}}}\Cendnote{\textnormal{\textcolor{blue}{Goldmann} bezieht sich auf die seit
                     1889 andauernde Beziehung zwischen \textcolor{blue}{Schnitzler} und \textcolor{blue}{Marie
                     Glümer}.}}}\label{K_L02699-5h} dauert auch noch fort? Ich kenne mich nicht mehr aus: iſt
               es gut? iſt es ſchlimm? Da gibt es nur Eines: die Dinge zu Ende leben; und \strikeout{iſt} kommt kein Ende, ſo iſt es deshalb, weil es
               vielleicht keines gibt. Obwohl ich glaube, daß, wenn Du Dich einmal losriſſeſt und in
               die Welt hinausgingſt, die herrliche, große, Dir die zwei weißen Arme doch zu eng
               erſcheinen würden, die jetzt Deinen {\pb}Lebenskreis
               begrenzen. Verſuche es! Einen Monat! Komm hierher, oder irgendwohin! Sieh’ Dir die
               Sache von außen an! Ich meine, Du biſt die Probe Dir ſchuldig und denen, die an Dich
               glauben. Geht’s nicht \strikeout{\textcolor{gray}{×}} ohne das verteufelte Glück, ſo kannſt Du ja immer noch heimkehren.\pend
           \pstart
           Sei innigſt umarmt! Tauſend Dank! {\\[\baselineskip]}Dein{\\[\baselineskip]}treuer{\\[\baselineskip]}\spacefill\mbox{Paul Goldmnn.}\pend
           \leftskip=0em{}\endnumbering\briefempfaengerindex{Schnitzler, Arthur@\textsc{Schnitzler, Arthur}!zzzGoldmann, Paul@\emph{von Paul Goldmann}!1892-06-271@{27. 6. {[}1892{]}}|)be}\mylabel{h}\begin{anhang}\end{anhang}\normalsize

\doendnotes{C}
\bigskip
\vfill

\clearpage

\footnotesize

\lohead{\textsc{register}}

% Definiere theindex-Environment komplett neu ohne reledmac
\makeatletter
\renewenvironment{theindex}{%
  \section*{\indexname}%
  \setlength{\parindent}{0pt}%
  \setlength{\parskip}{0pt plus 0.3pt}%
  \let\item\@idxitem
}{%
  \clearpage
}
\makeatother

\IfFileExists{\jobname-pw.ind}{\input{\jobname-pw.ind}}{}

\end{document}

      