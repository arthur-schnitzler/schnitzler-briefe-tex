%% latex-korrekturansicht-vorspann.tex
%% Vorspann für die Korrekturansicht.
%% Lädt die gemeinsame Datei latex-vorspann.tex mit gesetztem Schalter.

\newif\ifkorrekturansicht
\korrekturansichttrue

\input{../tex-inputs/latex-vorspann}


\renewcommand{\erwaehntePersonen}{Personen: Shelomoh ben Mosheh Alḳabets, Olga Schnitzler}
\renewcommand{\erwaehnteOrte}{Orte: Dr. Ludwig Leber-Straße, Edmund-Weiß-Gasse 7, Mariazell, Reichenau an der Rax, Wien, XVIII., Währing}
\renewcommand{\erwaehnteWerke}{Werke: Lecha Dodi, Spätgotische Marienstatue mit Strahlenkranz, Tagebuch}
\section[ Felix Salten und Richard Metzl an Arthur Schnitzler, {[}30. 7. 1905?{]}]{Felix Salten und Richard Metzl an Arthur
               Schnitzler, {[}30. 7. 1905?{]}}
\nopagebreak\mylabel{v}
\rehead{ }\normalsize\beginnumbering\briefempfaengerindex{Schnitzler, Arthur@\textsc{Schnitzler, Arthur}!zzzMetzl, Richard@\emph{von Richard Metzl}!1905-07-303@{{[}30. 7. 1905?{]}}|(be}\briefempfaengerindex{Schnitzler, Arthur@\textsc{Schnitzler, Arthur}!zzzSalten, Felix@\emph{von Felix Salten}!1905-07-303@{{[}30. 7. 1905?{]}}|(be}
\toendnotes[C]{\smallbreak\pagebreak[2]}\Standort{CUL, Schnitzler, B 89, B 1.}
\physDesc{Bildpostkarte, 107 Zeichen
\newline{}Handschrift Felix Salten: Bleistift, lateinische Kurrent
\newline{}Handschrift Richard Metzl: Bleistift, deutsche Kurrent
\newline{}Versand: Stempel: »\nobreak{}\oindex{Mariazell@\textbf{Mariazell}, \emph{P.PPLA3}|pwk}\textcolor{gray}{Mariazell}, 30 7 \textcolor{gray}{05}\nobreak{}«.  
\newline{}Ordnung: mit Bleistift von unbekannter Hand nummeriert: »202« }\toendnotes[C]{\smallbreak}\pstart{}{\pb}Herrn D\textsuperscript{r} Arthur Schnitzler\pend{}\pstart{}\textcolor{pink}{Wien XVIII.}{}\ledrightnote{\textcolor{pink}{XVIII., Währing}}\pend{}\pstart{}\textcolor{pink}{Spöttelgaſse 7}{}\ledrightnote{\textcolor{pink}{Edmund-Weiß-Gasse 7}}\pend{}
{\bigskip}
\pstart
           \noindent{}\centering{}{\pb}\textcolor{gray}{\textbf{\textsc{Gruss aus \label{K_L03410-1v}\edtext{\textcolor{pink}{Mariazell}{}\ledrightnote{\textcolor{pink}{Mariazell}}}{\lemma{\textnormal{\emph{Mariazell}}}\Cendnote{\textnormal{Die am 6. 5. 1905
                           erwähnte »\textcolor{pink}{Maria Zell}er Partie« fand
                           aus nicht überlieferten Gründen letztlich ohne Beteiligung \textcolor{blue}{Schnitzler}s und seiner \textcolor{blue}{Frau} statt und lässt
                           sich auf ein Zeitfenster eingrenzen. Am A. S.: \emph{Tagebuch}, 28. 7. 1905
                           sahen sich \textcolor{blue}{Schnitzler} und \textcolor{blue}{Salten}
                           in \textcolor{pink}{Reichenau an der Rax}, am
                              31. 7. 1931 war \textcolor{blue}{Salten}
                           wieder in \textcolor{pink}{Wien} – »aus \textcolor{pink}{Mariazell}, angeekelt«, wie
                              \textcolor{blue}{Schnitzler} im \emph{\textcolor{green}{Tagebuch}} festhielt.}}}\label{K_L03410-1h}}}}\pend
           
\pstart
           \noindent{}\centering{}\textcolor{gray}{\textbf{\textcolor{green}{MARIENSTATUE}{}\ledrightnote{\textcolor{green}{Spätgotische Marienstatue mit Strahlenkranz}}}}\pend
           
\pstart
           \noindent{}\centering{}\textcolor{gray}{\textbf{\textcolor{pink}{WIENERGASSE}{}\ledrightnote{\textcolor{pink}{Dr. Ludwig Leber-Straße}}}}\pend
           
\pstart
           Das \label{K_L03410-2v}\edtext{\textcolor{green}{Lechodaudi}{}\ledrightnote{\textcolor{green}{Lecha Dodi}}}{\lemma{\textnormal{\emph{Lechodaudi}}}\Cendnote{\textnormal{\textcolor{green}{Lecha Dodi (L’kha Dodi)}
                  sind die ersten beiden Wörter einer Hymne von \textcolor{blue}{Shelomoh ben Mosheh Alḳabets}, mit der der Sabbat eingeläutet wird. \textcolor{blue}{Salten} dürfte hier dem  
                  Vergnügen Ausdruck verleihen, in einem katholischen Wallfahrtsort ein jüdisches Lied 
                  zu singen. Um tatsächlich mit dem Beginn des Sabbats übereinzustimmen, müsste die Karte
                  am Freitag Abend verfasst sein. Der Poststempel
                  weist aber auf Sonntag, den 30., so dass \textcolor{blue}{Salten} hier
                  nicht versuchen dürfte, in der Aussage eine Datums- und Uhrzeitangabe zu verstecken.}}}\label{K_L03410-2h}
                  singend,
            \pend
           \pstart herzlich Ihr \spacefill\mbox{Salten}\pend{}
\pstart
           {[}hs. Metzl:{]} Beſten Gruß {\\[\baselineskip]}\spacefill\mbox{R Metzl}\pend
           \leftskip=0em{}\endnumbering\briefempfaengerindex{Schnitzler, Arthur@\textsc{Schnitzler, Arthur}!zzzMetzl, Richard@\emph{von Richard Metzl}!1905-07-303@{{[}30. 7. 1905?{]}}|)be}\briefempfaengerindex{Schnitzler, Arthur@\textsc{Schnitzler, Arthur}!zzzSalten, Felix@\emph{von Felix Salten}!1905-07-303@{{[}30. 7. 1905?{]}}|)be}\mylabel{h}  \normalsize

\doendnotes{C}
\bigskip
\vfill

\clearpage

\footnotesize

\lohead{\textsc{register}}

% Definiere theindex-Environment komplett neu ohne reledmac
\makeatletter
\renewenvironment{theindex}{%
  \section*{\indexname}%
  \setlength{\parindent}{0pt}%
  \setlength{\parskip}{0pt plus 0.3pt}%
  \let\item\@idxitem
}{%
  \clearpage
}
\makeatother

\IfFileExists{\jobname-pw.ind}{\input{\jobname-pw.ind}}{}

\end{document}

      