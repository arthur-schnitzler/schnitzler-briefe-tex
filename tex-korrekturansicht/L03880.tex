%% latex-korrekturansicht-vorspann.tex
%% Vorspann für die Korrekturansicht.
%% Lädt die gemeinsame Datei latex-vorspann.tex mit gesetztem Schalter.

\newif\ifkorrekturansicht
\korrekturansichttrue

\input{../tex-inputs/latex-vorspann}


\section[Wanda Sacher-Masoch an Arthur Schnitzler, 9. 12. 1908]{L03880 Wanda Sacher-Masoch an Arthur Schnitzler, 9. 12. 1908}
\nopagebreak\mylabel{L03880v}
\rehead{ }\normalsize\beginnumbering\briefempfaengerindex{, @\textsc{, }!zzz, @\emph{von  }!1908-12-091@{9. 12. 1908}|(be}
\toendnotes[C]{\smallbreak\pagebreak[2]}\Standort{DLA, A:Schnitzler, 1985.1.4377.}
\physDesc{Briefkarte, 278 Zeichen
\newline{}Handschrift: schwarze Tinte, deutsche Kurrent
\newline{}Schnitzler: mit rotem Buntstift eine Unterstreichung 
\newline{}Ordnung: mit Bleistift eine Unterstreichung }
\pstart{}{\pb}Sehr geehrter Herr!\pend\vspace{0.5em}
\pstart
           Wären Sie bereit ein Stück mit mir zuſammen zu ſchreiben?\pend
           
\pstart
           Es handelt ſich um ein ſehr \uline{dramatiſches
                  Erlebnis}\textcolor{gray}{,} das ich in \textcolor{pink}{München}\oindex{München@\textbf{München}|pw}{}\ledrightnote{\textcolor{pink}{München}} hatte.\pend
           
\pstart
           {\pb}Durch eine raſche Rückäußerung würden Sie mich ſehr
               verbinden.\pend
           
\pstart
           Hochachtungsvoll{\\[\baselineskip]}\spacefill\mbox{Wanda v. Sacher-Maſoch.}\pend
           \leftskip=0em{}
\pstart
           \textsc{\textcolor{pink}{Morges, Schweiz}\oindex{Morges@\textbf{Morges}, \emph{Verwaltungsgebiet}|pw}{}\ledrightnote{\textcolor{pink}{Morges}}
                        (\textcolor{pink}{Vaud}\oindex{Kanton Waadt@\textbf{Kanton Waadt}|pw}{}\ledrightnote{\textcolor{pink}{Kanton Waadt}})}{ }9/12–08\pend
           \selectlanguage{ngerman}\endnumbering\briefempfaengerindex{, @\textsc{, }!zzz, @\emph{von  }!1908-12-091@{9. 12. 1908}|)be}\mylabel{L03880h}
\begin{anhang}
\end{anhang}\normalsize

\doendnotes{C}
\bigskip
\vfill

\clearpage

\footnotesize

\lohead{\textsc{register}}

% Definiere theindex-Environment komplett neu ohne reledmac
\makeatletter
\renewenvironment{theindex}{%
  \section*{\indexname}%
  \setlength{\parindent}{0pt}%
  \setlength{\parskip}{0pt plus 0.3pt}%
  \let\item\@idxitem
}{%
  \clearpage
}
\makeatother

\IfFileExists{\jobname-pw.ind}{\input{\jobname-pw.ind}}{}

\end{document}

      