%% latex-korrekturansicht-vorspann.tex
%% Vorspann für die Korrekturansicht.
%% Lädt die gemeinsame Datei latex-vorspann.tex mit gesetztem Schalter.

\newif\ifkorrekturansicht
\korrekturansichttrue

\input{../tex-inputs/latex-vorspann}


               \section[Robert Adam an Arthur Schnitzler, 17. 1. 1919]{ Robert Adam an Arthur Schnitzler, 17. 1. 1919}\nopagebreak\mylabel{v}\rehead{ }\normalsize\beginnumbering\briefempfaengerindex{Schnitzler, Arthur@\textsc{Schnitzler, Arthur}!zzzAdam, Robert@\emph{von Robert Adam}!1919-01-171@{17. 1. 1919}|(be} \toendnotes[C]{\smallbreak\pagebreak[2]} \Standort{CUL, Schnitzler, B 1.}
\physDesc{Brief, 1 Blatt, 4 Seiten
\newline{}Handschrift: schwarze Tinte, deutsche Kurrent
\newline{}Schnitzler: 1) mit Bleistift beschriftet: »\textsc{Adam}« 2) mit rotem Buntstift mehrere Unterstreichungen\newline{}Ordnung: mit Bleistift von unbekannter Hand nummeriert:
                                        »12« }\Standort{Wien, Österreichische Nationalbibliothek, Cod.ser. 52.269, 231.}
\physDesc{Brief, maschinelle Abschrift
\newline{}Schreibmaschine}\toendnotes[C]{\smallbreak}\pstart
           \raggedleft{}{\pb}\textcolor{pink}{Wien}{}\ledrightnote{\textcolor{pink}{Wien}}, 17. Januar 1919\pend
           \pstart{}Hochverehrter Herr Doktor!\pend\pstart
           Da ich vom \textcolor{pink}{Deutſchen Volkstheater}{}\ledrightnote{\textcolor{pink}{Volkstheater}} zwei Monate
                    lang nichts zu hören bekam, konnte ich meine verzweifelte Ungeduld nicht mehr
                    bezwingen und ſchrieb an die Direktion, ſie möchte ſo gut ſein, mich vom Stande
                    der Dinge zu verſtändigen (was wir in der Amtsſprache eine »Betreibung« nennen).
                    Heut erhielt ich nun vom Dramaturgen D\textsuperscript{r}{ }\textcolor{blue}{\textsc{Glücksmann}}{}\ledrightnote{\textcolor{blue}{Heinrich Glücksmann}} folgenden Brief:\pend
           \pstart
           »Ihre beiden dramatiſchen Arbeiten ›\textcolor{green}{Der Fremde}{}\ledrightnote{\textcolor{green}{Der Fremde}}‹
                    und ›\textcolor{green}{Yppl}{}\ledrightnote{\textcolor{green}{Der Fremde}}‹ ſind längſt geleſen. Ich geſtehe
                    ſofort: mit lebhafteſtem Genuß. Die \textcolor{blue}{Chriſtus}{}\ledrightnote{\textcolor{blue}{Jesus}}-Szenen ſind nicht alle gleichwertig, aber doch zumeiſt ſchön und
                    tief und nachklingend. Vielleicht wird es möglich ſein, ſie im Rahmen einer {\pb}literariſchen Veranſtaltung zu
                    bringen. Das 3. Bild wird man wohl auslaſſen müſſen, aus inneren Gründen, und
                    vielleicht iſt auch das 4., nur ein undramatiſches Gleichnis, von der Bühne
                    herab nicht wirkſam. 1.{ }2.{ }5. und 6 dürften jedoch ihre Probe beſtehen.\pend
           \pstart
           »Was ›\textcolor{green}{Yppl}{}\ledrightnote{\textcolor{green}{Yppl. Idylle in fünf Akten}}‹ anbelangt, ſo iſt es eine gute
                    Satire auf das kleinſtädtiſche Beamtenleben. Die Löſung des Konfliktes erſcheint
                    mir freilich gewaltſam und nicht überzeugend, die Wiederholung der Probe des
                    Dilettanten-Stückes wäre zu vermeiden, weil ſie ein bischen auf den Gang der
                    Handlung drückt. Jedenfalls habe ich es für meine Pflicht gehalten, Herrn
                    Direktor \textcolor{blue}{Bernau}{}\ledrightnote{\textcolor{blue}{Alfred Bernau}} für die beiden Arbeiten zu
                    intereſſieren. Sobald er dazu kommt, wird er ſie auch leſen.«\pend
           \pstart
           Nach Erhalt dieſes Briefes, des angenehmſten, den ich noch in Theaterdingen
                    bekommen habe, begab ich mich – heut vormittags – in das Theater und ſuchte D\textsuperscript{r}{ }\textcolor{blue}{\textsc{Glücksmann}}{}\ledrightnote{\textcolor{blue}{Heinrich Glücksmann}} auf. Er äußerte ſich ſehr liebenswürdig {\pb}über beide \textcolor{green}{Stücke}{}\ledrightnote{→\textcolor{green}{Yppl. Idylle in fünf Akten}{\newline}→\textcolor{green}{Der Fremde}} und ſagte, er habe ſie dem
                        \textcolor{blue}{Direktor}{}\ledrightnote{→\textcolor{blue}{Alfred Bernau}}{ }ſchon längſt
                    als erſte der von ihm zu leſenden vorbereitet, doch ſei er immer noch abgehalten
                    geweſen, die Lektüre vorzunehmen.\pend
           \pstart
           Daß D\textsuperscript{r}{ }\textcolor{blue}{\textsc{Glücksmann}}{}\ledrightnote{\textcolor{blue}{Heinrich Glücksmann}} gerade das 3. und 4. Bild des »\textcolor{green}{Fremden}{}\ledrightnote{\textcolor{green}{Yppl. Idylle in fünf Akten}}«
                        (»\textcolor{green}{Die Hure}{}\ledrightnote{→\textcolor{green}{Yppl. Idylle in fünf Akten}}« und »\textcolor{green}{Der Hund}{}\ledrightnote{→\textcolor{green}{Yppl. Idylle in fünf Akten}}«) für
                    untheatraliſch hält, iſt mir nicht recht begreiflich, da ich immer gerade dieſe
                    beiden Szenen für die dramatiſch allein wirkſamen gehalten habe, und auch Sie,
                    hochverehrter Herr Doktor, haben eine ähnliche Meinung geäußert.\pend
           \pstart
           Welche Szenen aber zur Aufführung kommen, ſcheint mir von ſekundärer Wichtigkeit;
                    wenn nur überhaupt eine Annahme erfolgte! Denn damit wäre wohl die Möglichkeit
                    gegeben, einen Verleger zu finden, und ich ſehne mich unbändig danach, juſt den
                        »\textcolor{green}{Fremden}{}\ledrightnote{\textcolor{green}{Yppl. Idylle in fünf Akten}}« gedruckt zu ſehen.\pend
           \pstart
           Ich will nun den Verſuch machen, Direktor \textcolor{blue}{Bernau}{}\ledrightnote{\textcolor{blue}{Alfred Bernau}} im Theater anzu{\pb}treffen und ihn zu Beſchleunigung ſeiner Entſcheidung zu veranlaſſen. Sollten
                    Sie, hochverehrter Herr Doktor, in der nächſten Zeit einmal mit ihm
                    zuſammentreffen, ſo bitte ihn bei dieſer Gelegenheit meine Stücke in Erinnerung
                    zu bringen (ſofern es Ihnen nicht unangenehm iſt).\pend
           \pstart
           Außer dieſem Ereignis weiß ich aus der Monotonie meiner Exiſtenz nichts zu
                    berichten: ich arbeite im Amt und leſe daheim, halbſatt und halbwarm und halb im
                    Winterſchlaf.\pend
           \pstart
           Wenn ich Sie nicht ſtöre, möchte ich Sie gerne wieder einmal aufſuchen; ich habe
                    einige kleine Lektüreentdeckungen gemacht, die Sie vielleicht intereſſieren
                    könnten.\pend
           \pstart
           Mit beſten Grüßen Ihr ſehr{\\[\baselineskip]}ergebener{\\[\baselineskip]}\spacefill\mbox{D\textsuperscript{r}RAdam}\pend
           \leftskip=0em{}\endnumbering\briefempfaengerindex{Schnitzler, Arthur@\textsc{Schnitzler, Arthur}!zzzAdam, Robert@\emph{von Robert Adam}!1919-01-171@{17. 1. 1919}|)be}\mylabel{h}  \normalsize

\doendnotes{C}
\bigskip
\vfill

\clearpage

\footnotesize

\lohead{\textsc{register}}

% Definiere theindex-Environment komplett neu ohne reledmac
\makeatletter
\renewenvironment{theindex}{%
  \section*{\indexname}%
  \setlength{\parindent}{0pt}%
  \setlength{\parskip}{0pt plus 0.3pt}%
  \let\item\@idxitem
}{%
  \clearpage
}
\makeatother

\IfFileExists{\jobname-pw.ind}{\input{\jobname-pw.ind}}{}

\end{document}

      