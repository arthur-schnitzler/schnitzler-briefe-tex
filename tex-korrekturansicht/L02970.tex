%% latex-korrekturansicht-vorspann.tex
%% Vorspann für die Korrekturansicht.
%% Lädt die gemeinsame Datei latex-vorspann.tex mit gesetztem Schalter.

\newif\ifkorrekturansicht
\korrekturansichttrue

\input{../tex-inputs/latex-vorspann}


\renewcommand{\erwaehntePersonen}{Personen:  ?? [Hausmeister von Felix Salten in der Kochgasse 1901], Adolf Lantz, Felix Salten}
\renewcommand{\erwaehnteInstitutionen}{Institutionen: ?? [Wiener Club September 1901]}
\renewcommand{\erwaehnteOrte}{Orte: Wien}
\renewcommand{\erwaehnteWerke}{Werke: Der Puppenspieler. Studie in einem Aufzuge, Die Frau mit dem Dolche, Die letzten Masken, Lebendige Stunden, Literatur}
\section[ Arthur Schnitzler an Felix Salten, 16. 9. 1901]{Arthur Schnitzler an Felix Salten, 16. 9. 1901}
\nopagebreak\mylabel{v}
\rehead{ }\normalsize\beginnumbering\briefempfaengerindex{Salten, Felix@\textsc{Salten, Felix}!zzzSchnitzler, Arthur@\emph{von Arthur Schnitzler}!1901-09-161@{16. 9. 1901}|(be}
\toendnotes[C]{\smallbreak\pagebreak[2]}\Standort{Wienbibliothek im Rathaus, ZPH 1681, 2.1.516.}
\physDesc{Brief, 1 Blatt, 2 Seiten, 359 Zeichen
\newline{}Handschrift: Bleistift, deutsche Kurrent
\newline{}Ordnung: mit Bleistift von unbekannter Hand Nummerierung der Blätter des Konvoluts:
                                    »23« }\toendnotes[C]{\smallbreak}
\pstart
           \raggedleft{}{\pb}16. \substVorne{}\textsuperscript{1}\substDazwischen{}9\substHinten{}. 901\pend
           
\pstart
           Lieber Freund, der kleine Herr \label{K_L02970-1v}\edtext{\textcolor{blue}{Lanz}{}\ledrightnote{\textcolor{blue}{Adolf Lantz}}}{\lemma{\textnormal{\emph{Lanz}}}\Cendnote{\textnormal{\textcolor{blue}{Adolf Lantz}?}}}\label{K_L02970-1h}, der Ihnen \label{K_L02970-2v}\edtext{ſ. Z.}{\lemma{\textnormal{\emph{ſ. Z.}}}\Cendnote{\textnormal{seiner Zeit}}}\label{K_L02970-2h} einige Manuſcripte überreicht laßt Sie
               durch mich bitten, diese Manuscripte bei Ihrem \textcolor{blue}{Hausmeiſter}{}\ledrightnote{{$\rightarrow$}\textcolor{blue}{?? [Hausmeister von Felix Salten in der Kochgasse 1901]}} zu \substVorne{}\textsuperscript{über}\substDazwischen{}hinter\substHinten{}legen, wo er ſie ſich abholen möchte.–\pend
           
\pstart
           Warum hab ich Sie a\textcolor{gray}{m}{ }\label{K_L02970-3v}\edtext{Samſtag}{\lemma{\textnormal{\emph{Samſtag}}}\Cendnote{\textnormal{siehe Arthur Schnitzler an Felix Salten, [14. 9. 1901?]}}}\label{K_L02970-3h} nicht geſehen? Sollten ſie ſchon im \textcolor{brown}{Club}{}\ledrightnote{{$\rightarrow$}\textcolor{brown}{?? [Wiener Club September 1901]}}{ }{\pb}geweſen ſein?–\pend
           
\pstart
           Ich ſchreibe \label{K_L02970-4v}\edtext{\textcolor{green}{2 Einakter}{}\ledrightnote{{$\rightarrow$}\textcolor{green}{Der Puppenspieler. Studie in einem Aufzuge}{\newline}{$\rightarrow$}\textcolor{green}{Die letzten Masken}}}{\lemma{\textnormal{\emph{2 Einakter}}}\Cendnote{\textnormal{\emph{\textcolor{green}{Der Puppenspieler}} und \emph{\textcolor{green}{Die letzten Masken}}, vgl. A. S.: \emph{Tagebuch}, 16. 9. 1901}}}\label{K_L02970-4h}, die zu den \textcolor{green}{3 andren}{}\ledrightnote{{$\rightarrow$}\textcolor{green}{Literatur}{\newline}{$\rightarrow$}\textcolor{green}{Die Frau mit dem Dolche}{\newline}{$\rightarrow$}\textcolor{green}{Lebendige Stunden}} gehören.\pend
           
\pstart
           Herzlichſt Ihr {\\[\baselineskip]}\spacefill\mbox{ArthSch}\pend
           \leftskip=0em{}\endnumbering\briefempfaengerindex{Salten, Felix@\textsc{Salten, Felix}!zzzSchnitzler, Arthur@\emph{von Arthur Schnitzler}!1901-09-161@{16. 9. 1901}|)be}\mylabel{h}  \normalsize

\doendnotes{C}
\bigskip
\vfill

\clearpage

\footnotesize

\lohead{\textsc{register}}

% Definiere theindex-Environment komplett neu ohne reledmac
\makeatletter
\renewenvironment{theindex}{%
  \section*{\indexname}%
  \setlength{\parindent}{0pt}%
  \setlength{\parskip}{0pt plus 0.3pt}%
  \let\item\@idxitem
}{%
  \clearpage
}
\makeatother

\IfFileExists{\jobname-pw.ind}{\input{\jobname-pw.ind}}{}

\end{document}

      