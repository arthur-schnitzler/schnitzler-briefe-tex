%% latex-korrekturansicht-vorspann.tex
%% Vorspann für die Korrekturansicht.
%% Lädt die gemeinsame Datei latex-vorspann.tex mit gesetztem Schalter.

\newif\ifkorrekturansicht
\korrekturansichttrue

\input{../tex-inputs/latex-vorspann}


               \section[Arthur Schnitzler an Hugo von Hofmannsthal, {[}11. 8. 1891{]}]{ Arthur Schnitzler an Hugo von Hofmannsthal, {[}11. 8. 1891{]}}\nopagebreak\mylabel{v}\rehead{ }\normalsize\beginnumbering\briefempfaengerindex{Hofmannsthal, Hugo von@\textsc{Hofmannsthal, Hugo von}!zzzSchnitzler, Arthur@\emph{von Arthur Schnitzler}!1891-08-113@{{[}11. 8. 1891{]}}|(be} \toendnotes[C]{\smallbreak\pagebreak[2]} \Standort{FDH, Hs-30885,14.}
\physDesc{Briefkarte
\newline{}Handschrift: schwarze Tinte, deutsche Kurrent\newline{}Ordnung: mit Bleistift von unbekannter Hand datiert: »Aug 91« }\buchAbdrucke{\weitereDrucke{Hugo von Hofmannsthal, Arthur Schnitzler: \emph{Briefwechsel}. Hg. Therese Nickl und Heinrich Schnitzler. Frankfurt am Main: \emph{S. Fischer} 1964, S. 12.} }\toendnotes[C]{\smallbreak}\pstart
           \noindent{}{\pb}Lieber Loris\hspace*{1.5em}\label{K_L00030_1v}\edtext{eben habe ich an \textcolor{blue}{Richard \textsc{Beer-Hofmann}}{}\ledrightnote{\textcolor{blue}{Richard Beer-Hofmann}} geſchrieben}{\lemma{\textnormal{\emph{eben … geſchrieben}}}\Cendnote{\textnormal{vgl. Arthur Schnitzler an Richard Beer-Hofmann, 11. 8. 1891}}}\label{K_L00030_1h}, er möge womöglich So{\geminationn}tag 16. 8.{ }Vormittag nach \textcolor{pink}{Iſchl}{}\ledrightnote{\textcolor{pink}{Bad Ischl}} herüber zu kommen.
               Da ich ſchon am So{\geminationn}tag{ }Abend wieder nach \textcolor{pink}{Wien}{}\ledrightnote{\textcolor{pink}{Wien}} fahre, wäre es
               reizend {\pb}von Ihnen, auch ſchon So{\geminationn}tag{ }Vormittag nach \textcolor{pink}{Iſchl}{}\ledrightnote{\textcolor{pink}{Bad Ischl}} zu ſauſen\introOben{}(\introOben{}, wo ich die Adresse \textcolor{pink}{\textsc{\uline{Pension Leopold}}}{}\ledrightnote{\textcolor{pink}{Hotel und Pension Rudolfshöhe (Leopold Petter)}} habe\introOben{})\introOben{}.\pend
           \pstart
           Mit herzlichem Gruſs und in der angenehmen Erwartung Sie zu ſehen{\\[\baselineskip]}Ihr
                  \spacefill\mbox{Arthur}\pend
           \leftskip=0em{}\endnumbering\briefempfaengerindex{Hofmannsthal, Hugo von@\textsc{Hofmannsthal, Hugo von}!zzzSchnitzler, Arthur@\emph{von Arthur Schnitzler}!1891-08-113@{{[}11. 8. 1891{]}}|)be}\mylabel{h}  \normalsize

\doendnotes{C}
\bigskip
\vfill

\clearpage

\footnotesize

\lohead{\textsc{register}}

% Definiere theindex-Environment komplett neu ohne reledmac
\makeatletter
\renewenvironment{theindex}{%
  \section*{\indexname}%
  \setlength{\parindent}{0pt}%
  \setlength{\parskip}{0pt plus 0.3pt}%
  \let\item\@idxitem
}{%
  \clearpage
}
\makeatother

\IfFileExists{\jobname-pw.ind}{\input{\jobname-pw.ind}}{}

\end{document}

      