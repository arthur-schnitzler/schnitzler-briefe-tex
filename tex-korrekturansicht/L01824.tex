%% latex-korrekturansicht-vorspann.tex
%% Vorspann für die Korrekturansicht.
%% Lädt die gemeinsame Datei latex-vorspann.tex mit gesetztem Schalter.

\newif\ifkorrekturansicht
\korrekturansichttrue

\input{../tex-inputs/latex-vorspann}


               \section[Hermann Bahr an Arthur Schnitzler, 18. 1. 1909]{ Hermann Bahr an Arthur Schnitzler, 18. 1. 1909}\nopagebreak\mylabel{v}\rehead{ }\normalsize\beginnumbering\briefempfaengerindex{Schnitzler, Arthur@\textsc{Schnitzler, Arthur}!zzzBahr, Hermann@\emph{von Hermann Bahr}!1909-01-181@{18. 1. 1909}|(be} \toendnotes[C]{\smallbreak\pagebreak[2]} \Standort{CUL, Schnitzler, B 5b.}
\physDesc{Brief, 1 Blatt, 2 Seiten
\newline{}Handschrift: blaue Tinte, deutsche Kurrent
\newline{}Schnitzler: mit Bleistift beschriftet: »Bahr« \newline{}Ordnung: mit Bleistift von unbekannter Hand
                           nummeriert: »155« }\buchAbdrucke{\weitereDrucke{Hermann Bahr, Arthur Schnitzler: \emph{Briefwechsel, Aufzeichnungen, Dokumente (1891–1931)}. Hg. Kurt Ifkovits und Martin Anton Müller. Göttingen: \emph{Wallstein} 2018, S. 414.} }\toendnotes[C]{\smallbreak}\pstart
           \raggedleft{}{\pb}\textcolor{pink}{Wien XIII/\textsubscript{7}}{}\ledrightnote{\textcolor{pink}{Ober Sankt Veit}}{\\}18. 1. 09\pend
           \pstart\center{}Lieber Arthur!\pend\pstart
           Danke ſchön für Deine ſo liebe Karte. Ich komme eben vom \textcolor{pink}{Semmering}{}\ledrightnote{\textcolor{pink}{Semmering}} (wo ich übrigens Deinen Bruder \textcolor{blue}{Julius}{}\ledrightnote{\textcolor{blue}{Julius Schnitzler}}{ }ſtolz im \textsc{\textcolor{pink}{Nizza}{}\ledrightnote{\textcolor{pink}{Nizza}}{ }Express} vorüber ſauſen ſah), hab einen
               ſcheußlichen Hexenſchuß, ſitz in einem durch Überſchwemmung aus einem geplatzten
               Waſſerrohr faſt demolierten Haus und ſoll in zwei Tagen nach \textcolor{pink}{Dresden}{}\ledrightnote{\textcolor{pink}{Dresden}} zur \label{K_L01824_1v}\edtext{\textcolor{blue}{Strauß}{}\ledrightnote{\textcolor{blue}{Richard Strauss}}-\textcolor{green}{Elektra}{}\ledrightnote{\textcolor{green}{Elektra (op. 58)}{\newline}\textcolor{green}{Elektra. Tragödie in einem Aufzug}}-Première}{\lemma{\textnormal{\emph{Strauß-Elektra-Première}}}\Cendnote{\textnormal{Am
                     25. 1. 1909, \textcolor{blue}{Bahr} war vom
                     23. bis zum 26. in \textcolor{pink}{Dresden}.}}}\label{K_L01824_1h}, weshalb ich, Dir herzlichſt für Deinen guten Willen
               dankend, Dich bitten muß, Deine ſo liebe Abſicht erſt auszuführen, bis ich nächſte
               Woche von \textcolor{pink}{Dresden}{}\ledrightnote{\textcolor{pink}{Dresden}} zurück, halbwegs in Ordnung und
               auch mit den drei letzten Kapiteln meines neuen \label{K_L01824_2v}\edtext{\textcolor{green}{Romans}{}\ledrightnote{→\textcolor{green}{Drut}}}{\lemma{\textnormal{\emph{Romans}}}\Cendnote{\textnormal{\textcolor{blue}{Hermann Bahr}: \emph{\textcolor{green}{Drut. Roman}}. Berlin: \emph{\textcolor{brown}{S. Fischer}}{ }1909.}}}\label{K_L01824_2h} aus dem Roheſten bin, worauf ich anzufangen hoffe, wieder einem
               Menſchen zu gleichen.\pend
           \pstart
           {\pb}Ich freue mich unendlich \substVorne{}\textsuperscript{D}\substDazwischen{}a\substHinten{}uf Dich, ich hab Dir ja ſo viel, ſo viel zu ſagen und manchmal ist mir ſchon
               ordentlich bang nach Dir. Nur hat ſich mein Leben allmälig ſo merkwürdig geſtellt,
               daß ich mir ſchon wirklich \strikeout{nicht} manchmal vorkomme,
               nicht mehr auf der Erde zu ſein, ſondern nur noch ein hinten her, neben bei irgendwo
               mitſauſendes, nachwirbelndes \label{K_L01824_3v}\edtext{Gehängſel}{\lemma{\textnormal{\emph{Gehängſel}}}\Cendnote{\textnormal{Anhängsel}}}\label{K_L01824_3h}!\pend
           \pstart
           Grüß Deine liebe \textcolor{blue}{Frau}{}\ledrightnote{→\textcolor{blue}{Olga Schnitzler}} herzlichſt
               von mir, auch den \textcolor{blue}{Sohn}{}\ledrightnote{→\textcolor{blue}{Heinrich Schnitzler}}, Herrn
               Sohn muß man jetzt wol bald ſchon ſagen.\pend
           \pstart
           Herzlichſt{\\[\baselineskip]}immer Dein{\\[\baselineskip]}\spacefill\mbox{Hermann}\pend
           \leftskip=0em{}\endnumbering\briefempfaengerindex{Schnitzler, Arthur@\textsc{Schnitzler, Arthur}!zzzBahr, Hermann@\emph{von Hermann Bahr}!1909-01-181@{18. 1. 1909}|)be}\mylabel{h}  \normalsize

\doendnotes{C}
\bigskip
\vfill

\clearpage

\footnotesize

\lohead{\textsc{register}}

% Definiere theindex-Environment komplett neu ohne reledmac
\makeatletter
\renewenvironment{theindex}{%
  \section*{\indexname}%
  \setlength{\parindent}{0pt}%
  \setlength{\parskip}{0pt plus 0.3pt}%
  \let\item\@idxitem
}{%
  \clearpage
}
\makeatother

\IfFileExists{\jobname-pw.ind}{\input{\jobname-pw.ind}}{}

\end{document}

      