%% latex-korrekturansicht-vorspann.tex
%% Vorspann für die Korrekturansicht.
%% Lädt die gemeinsame Datei latex-vorspann.tex mit gesetztem Schalter.

\newif\ifkorrekturansicht
\korrekturansichttrue

\input{../tex-inputs/latex-vorspann}


\renewcommand{\erwaehnteOrte}{Orte: Edmund-Weiß-Gasse, Klosterneuburg, Wien}
\renewcommand{\erwaehnteWerke}{}
\section[ Paul Goldmann an Arthur Schnitzler, 12. 4. 1909]{Paul Goldmann an Arthur Schnitzler, 12. 4. 1909}
\nopagebreak\mylabel{v}
\rehead{ }\normalsize\beginnumbering\briefempfaengerindex{Schnitzler, Arthur@\textsc{Schnitzler, Arthur}!zzzGoldmann, Paul@\emph{von Paul Goldmann}!1909-04-121@{12. 4. 1909}|(be}
\toendnotes[C]{\smallbreak\pagebreak[2]}\Standort{DLA, A:Schnitzler, HS.NZ85.1.3175.}
\physDesc{Postkarte, 359 Zeichen
\newline{}Handschrift: 1) schwarze Tinte, deutsche Kurrent\hspace{1em}2) schwarze Tinte, lateinische Kurrent (\noindent{}Adresse)\hspace{1em}
\newline{}Versand: Stempel: »\nobreak{}4/1 Wien 50 P., 12. IV. 09, XI\nobreak{}«. Stempel: »\nobreak{}18/1 Wien 1/1 P., 12. IV. 09, XI\textsuperscript{50}\nobreak{}«.  
\newline{}Schnitzler: mit Bleistift das Datum »12/4« und »{\pb}\textcolor{blue}{Goldmann}« vermerkt }\toendnotes[C]{\smallbreak}\pstart{}{\pb}Herrn\pend{}\pstart{}Dr. Arthur Schnitzler\pend{}\pstart{}\textcolor{pink}{Wien}{}\ledrightnote{\textcolor{pink}{Wien}}\pend{}\pstart{}\textcolor{pink}{XVIII. Spöttelgaſse 7}{}\ledrightnote{\textcolor{pink}{Edmund-Weiß-Gasse}}.\pend{}
{\bigskip}
\pstart
           Montag{ }früh\pend
           
\pstart
           Lieber Freund, Ich fahre jetzt nach \textcolor{pink}{Kloſterneuburg}{}\ledrightnote{\textcolor{pink}{Klosterneuburg}} u. werde, wenn nicht einvorhergeſehene
               Verſpätung eintritt, Nachmittags zwiſchen 6 u. 7{ }\label{K_L03467-1v}\edtext{zu Dir kommen}{\lemma{\textnormal{\emph{zu Dir kommen}}}\Cendnote{\textnormal{siehe A. S.: \emph{Tagebuch}, 12. 4. 1909}}}\label{K_L03467-1h}. Wenn Du aber etwas vorhaſt, ſo laß’ Dich nicht ſtören, ich werde nicht
               gekränkt ſein, wenn ich Dich nicht zu Hauſe finde.\pend
           \pstart Herzl. Gruß! \spacefill\mbox{Paul Goldmann}\pend{}\endnumbering\briefempfaengerindex{Schnitzler, Arthur@\textsc{Schnitzler, Arthur}!zzzGoldmann, Paul@\emph{von Paul Goldmann}!1909-04-121@{12. 4. 1909}|)be}\mylabel{h}
\begin{anhang}
\end{anhang}\normalsize

\doendnotes{C}
\bigskip
\vfill

\clearpage

\footnotesize

\lohead{\textsc{register}}

% Definiere theindex-Environment komplett neu ohne reledmac
\makeatletter
\renewenvironment{theindex}{%
  \section*{\indexname}%
  \setlength{\parindent}{0pt}%
  \setlength{\parskip}{0pt plus 0.3pt}%
  \let\item\@idxitem
}{%
  \clearpage
}
\makeatother

\IfFileExists{\jobname-pw.ind}{\input{\jobname-pw.ind}}{}

\end{document}

      