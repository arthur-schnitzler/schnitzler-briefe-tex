%% latex-korrekturansicht-vorspann.tex
%% Vorspann für die Korrekturansicht.
%% Lädt die gemeinsame Datei latex-vorspann.tex mit gesetztem Schalter.

\newif\ifkorrekturansicht
\korrekturansichttrue

\input{../tex-inputs/latex-vorspann}


\renewcommand{\erwaehntePersonen}{Personen: Gerhart Hauptmann, Olga Schnitzler, Ernst von Wolzogen}
\renewcommand{\erwaehnteInstitutionen}{Institutionen: Überbrettl}
\renewcommand{\erwaehnteOrte}{Orte: Berlin, Dessauer Straße, Deutsches Theater Berlin, Wien}
\renewcommand{\erwaehnteWerke}{Werke: Der rothe Hahn. Tragikomödie in vier Akten}
\section[ Paul Goldmann an Arthur Schnitzler, 29. 11. {[}1901{]}]{Paul Goldmann an Arthur Schnitzler, 29. 11. {[}1901{]}}
\nopagebreak\mylabel{v}
\rehead{ }\normalsize\beginnumbering\briefempfaengerindex{Schnitzler, Arthur@\textsc{Schnitzler, Arthur}!zzzGoldmann, Paul@\emph{von Paul Goldmann}!1901-11-291@{29. 11. {[}1901{]}}|(be}
\toendnotes[C]{\smallbreak\pagebreak[2]}\Standort{DLA, A:Schnitzler, HS.NZ85.1.3171.}
\physDesc{Brief, 1 Blatt, 4 Seiten
\newline{}Handschrift: blaue Tinte, deutsche Kurrent
\newline{}Schnitzler: mit Bleistift das Jahr »{[}1{]}901.« vermerkt }\toendnotes[C]{\smallbreak}
\pstart
           \noindent{}\raggedleft{}{\pb}\textcolor{pink}{\textcolor{gray}{\textbf{DESSAUERSTRASSE 19}}}{}\ledrightnote{\textcolor{pink}{Dessauer Straße}}\pend
           
\pstart
           \textcolor{pink}{Berlin}{}\ledrightnote{\textcolor{pink}{Berlin}}, 29. November.\pend
           
\pstart\center{}Mein lieber Freund,\pend
\pstart
           »Ungütig«! Du greifſt mich an, greifſt mich an der Stelle an, wo ich am
               Verwundbarſten bin, – da, wo mein Lebensnerv ſitzt. Ich wehre mich gegen Deinen
               Angriff. Und das nennſt Du »ungütig aufnehmen«. Das iſt ein glänzender
               Lutſpiel-Einfall, und Du ſollſt dir ihn aufnotiren.\pend
           
\pstart
           »Zurechtweiſen«. Gewiß, \label{K_L03092-1v}\edtext{\textsc{\textcolor{blue}{Olga}{}\ledrightnote{\textcolor{blue}{Olga Schnitzler}}}}{\lemma{\textnormal{\emph{Olga}}}\Cendnote{\textnormal{siehe Paul Goldmann an Arthur Schnitzler, 23. 11. [1901]}}}\label{K_L03092-1h} hat mich nicht zurechtweiſen {\pb}\uline{gewollt}. Aber ſie hat’s \uline{gethan}. Und was mich ſo ſehr erregte, \strikeout{war,}
               war, daß ich plötzlich erkennen mußte, wie dieſes \textcolor{blue}{Mädchen}{}\ledrightnote{{$\rightarrow$}\textcolor{blue}{Olga Schnitzler}}, dem ich in aufrichtigſter Freundſchaft zugethan
               bin, die die \textcolor{blue}{Freundin}{}\ledrightnote{{$\rightarrow$}\textcolor{blue}{Olga Schnitzler}} meines
               liebſten Freundes iſt, weltenweit davon entfernt iſt, mich zu
                  verſtehen\textcolor{gray}{!}\pend
           
\pstart
           Im Übrigen iſt wirklich genug geredet; und es iſt ſehr blöd, daß wir uns da
               gegenſeitig allerlei Grobheiten ſchreiben, wo wir uns doch {\pb}wirklich Wichtigeres zu ſagen hätten.\pend
           
\pstart
           Mein lieber Freund, ich kann Dir heut nicht ſo
               ausführlich ſchreiben, als ich möchte. Ich habe wahnſinnig zu thun. In einigen Tagen
               hoffe ich Zeit zu einem längeren Brief zu finden.\pend
           
\pstart
           Der »\label{K_L03092-34v}\edtext{\textcolor{green}{Rothe Hahn}{}\ledrightnote{\textcolor{green}{Der rothe Hahn. Tragikomödie in vier Akten}}}{\lemma{\textnormal{\emph{Rothe Hahn}}}\Cendnote{\textnormal{\emph{\textcolor{green}{Der rothe Hahn. Tragikomödie in vier Akten}}
                  von \textcolor{blue}{Gerhart Hauptmann} hatte am 27. 11. 1901 am \textcolor{pink}{Deutschen
                     Theater Berlin} die Uraufführung. Siehe Paul Goldmann an Arthur Schnitzler, 6. 12. [1901].}}}\label{K_L03092-34h}« war gräßlich, \label{K_L03092-22v}\edtext{\textsc{\textcolor{blue}{Wolzogen}{}\ledrightnote{\textcolor{blue}{Ernst von Wolzogen}}{[}s{]}}}{\lemma{\textnormal{\emph{Wolzogens}}}\Cendnote{\textnormal{siehe Paul Goldmann an Arthur Schnitzler, 18. 2. [1901]}}}\label{K_L03092-22h} »\textcolor{brown}{Überbrettl}{}\ledrightnote{\textcolor{brown}{Überbrettl}}« fürchterlich.\pend
           
\pstart
           Was Du wir über Dein \label{K_L03092-21v}\edtext{Ohr}{\lemma{\textnormal{\emph{Ohr}}}\Cendnote{\textnormal{Bezug auf \textcolor{blue}{Schnitzler}s Otosklerose – einer Verknöcherung des Innenohrs mit
                  zunehmender Schwerhörigkeit –, an der er seit Herbst 1896 litt.\textcolor{blue}{Goldmann} nahm \textcolor{blue}{Schnitzler} darin zumeist nicht ernst. Siehe etwa Paul Goldmann an Arthur Schnitzler, 22. 3. [1897], Paul Goldmann an Arthur Schnitzler, 13. 9. 1897 und Paul Goldmann an Arthur Schnitzler, 28. 2. [1898].}}}\label{K_L03092-21h}
               ſchreibſt, iſt betrübend. Aber ich {\pb}kann mir nicht
               helfen, ich habe ſo eine Ahnung, daß \strikeout{D\textcolor{gray}{ir
                     das}} Du mit Deinem Ohrenleiden vielleicht viel weniger zu ſchaffen hätteſt, wenn Du
               nicht ſo oft zum Ohrenarzt gingeſt. Verringerung der Hörweite! \strikeout{Ich} Das wechſelt, wie alle Sinnesfunktionen bei allen
               nervöſen Menſchen. Von der Verringerung der Hörweite müßten doch diejenigen etwas
               merken, die mit Dir ſprechen. Ich habe davon auch nicht das leiſeſte Anzeichen
               bemerkt.\pend
           
\pstart
           Tauſend Grüße! {\\[\baselineskip]}Dein \spacefill\mbox{Paul Goldmann.}\pend
           \leftskip=0em{}\endnumbering\briefempfaengerindex{Schnitzler, Arthur@\textsc{Schnitzler, Arthur}!zzzGoldmann, Paul@\emph{von Paul Goldmann}!1901-11-291@{29. 11. {[}1901{]}}|)be}\mylabel{h}
\begin{anhang}
\end{anhang}\normalsize

\doendnotes{C}
\bigskip
\vfill

\clearpage

\footnotesize

\lohead{\textsc{register}}

% Definiere theindex-Environment komplett neu ohne reledmac
\makeatletter
\renewenvironment{theindex}{%
  \section*{\indexname}%
  \setlength{\parindent}{0pt}%
  \setlength{\parskip}{0pt plus 0.3pt}%
  \let\item\@idxitem
}{%
  \clearpage
}
\makeatother

\IfFileExists{\jobname-pw.ind}{\input{\jobname-pw.ind}}{}

\end{document}

      