%% latex-korrekturansicht-vorspann.tex
%% Vorspann für die Korrekturansicht.
%% Lädt die gemeinsame Datei latex-vorspann.tex mit gesetztem Schalter.

\newif\ifkorrekturansicht
\korrekturansichttrue

\input{../tex-inputs/latex-vorspann}


               \section[ Paul Goldmann an Arthur Schnitzler, 4. 9. 1897]{Paul Goldmann an Arthur Schnitzler, 4. 9. 1897}\nopagebreak\mylabel{v}\rehead{ }\normalsize\beginnumbering\briefempfaengerindex{Schnitzler, Arthur@\textsc{Schnitzler, Arthur}!zzzGoldmann, Paul@\emph{von Paul Goldmann}!1897-09-041@{4. 9. 1897}|(be} \toendnotes[C]{\smallbreak\pagebreak[2]} \Standort{DLA, A:Schnitzler, HS.NZ85.1.3167.}
\physDesc{Postkarte
\newline{}Handschrift: 1) schwarze Tinte, deutsche Kurrent\hspace{1em}2) schwarze Tinte, lateinische Kurrent (\noindent{}Adresse)\hspace{1em}\newline{}Versand: 1) Stempel: »\nobreak{}\oindex{Muenchen@\textbf{München}, \emph{https://www.geonames.org/ontologyP.PPLA}|pwk}Muenchen 1, 4{[}.{]} 9{[}. 1897{]}, 6–{[}7{]}\nobreak{}«.  2) Stempel: »\nobreak{}Wien 9/3 72, 5. 9. 1897, 11.V, Bestellt\nobreak{}«. 
\newline{}Schnitzler: mit Bleistift das Jahr »97« vermerkt }\toendnotes[C]{\smallbreak}\pstart{}{\pb}\textcolor{gray}{\textbf{An}}\pend{}\pstart{}Herrn\pend{}\pstart{}Dr. Arthur Schnitzler\pend{}\pstart{}\textcolor{gray}{\textbf{in}}{ }\textcolor{pink}{Wien}{}\ledrightnote{\textcolor{pink}{Wien}}\pend{}\pstart{}\textcolor{pink}{IX. Frankgaſse 1}{}\ledrightnote{\textcolor{pink}{Frankgasse}}. \pend{}{\bigskip}\pstart
           {\pb}\textsc{\textcolor{pink}{Muenchen}{}\ledrightnote{\textcolor{pink}{München}}}, 4. September.\pend
           \pstart
           Mein lieber \strikeout{\textcolor{gray}{F}} Freund, Ich fand hier im \textsc{\textcolor{pink}{Hotel}{}\ledrightnote{→\textcolor{pink}{Hotel Marienbad}}} eine Karte von der \label{K_L02822-11v}\edtext{\textcolor{blue}{Frau}{}\ledrightnote{→\textcolor{blue}{Rosa Freudenthal}} des \textcolor{blue}{Rechtsgelehrten}{}\ledrightnote{→\textcolor{blue}{Hermann Freudenthal}}}{\lemma{\textnormal{\emph{Frau des Rechtsgelehrten}}}\Cendnote{\textnormal{\textcolor{blue}{Rosa Freudenthal}, Ehefrau des Anwalts \textcolor{blue}{Hermann Freudenthal}, mit der \textcolor{blue}{Schnitzler} seit dem 2. 7. 1897 ein
                  Verhältnis hatte}}}\label{K_L02822-11h}. Bitte, danke ihr in meinem Namen, ſage ihr, daß es ſehr
               lieb war, an mich gedacht zu haben, und daß die Karte ſehr herzig geſchrieben war.
               Euch Allen geht es in \textcolor{pink}{Wien}{}\ledrightnote{\textcolor{pink}{Wien}} hoffentlich gut. Mir
               aber iſt das Herz \strikeout{\textcolor{gray}{w}u}{ }\strikeout{\textcolor{gray}{w}u} wund vom Abſchiednehmen. Und ich bin wieder einſam in
               der großen kalten Welt. Und es regnet draußen. Viele treue Grüße Dir, der \label{K_L02822-3v}\edtext{Familie \textcolor{blue}{\textsc{Altmann}}{}\ledrightnote{→\textcolor{blue}{Emma Altmann}}}{\lemma{\textnormal{\emph{Familie Altmann}}}\Cendnote{\textnormal{\textcolor{blue}{Schnitzler} verbrachte Ende August und Anfang
                     September 1897 Zeit mit \textcolor{blue}{Emma
                     Altmann}, der Mutter seiner Schwägerin \textcolor{blue}{Helene}, Ehefrau von \textcolor{blue}{Julius
                     Schnitzler}.}}}\label{K_L02822-3h}, der \textcolor{blue}{Frau}{}\ledrightnote{→\textcolor{blue}{Rosa Freudenthal}} des \textcolor{blue}{Rechtsgelehrten}{}\ledrightnote{→\textcolor{blue}{Hermann Freudenthal}}{ }\textsc{etc.}\pend
           \pstart Dein \spacefill\mbox{Paul Goldm}\pend{}\pstart
           \noindent{}\label{T_L02822-1v}\edtext{In \textcolor{pink}{Frankfurt}{}\ledrightnote{\textcolor{pink}{Frankfurt am Main}} bin ich Dienstag oder Mittwoch, Adreſſe: \textsc{\textcolor{pink}{Rossertstraſse 15}{}\ledrightnote{\textcolor{pink}{Rossertstraße}}}}{\lemma{\textnormal{\emph{In … Rossertstraſse 15}}}\Cendnote{\textnormal{entlang der oberen Kante, verkehrt zum Text}}}\label{T_L02822-1h}\pend
           \endnumbering\briefempfaengerindex{Schnitzler, Arthur@\textsc{Schnitzler, Arthur}!zzzGoldmann, Paul@\emph{von Paul Goldmann}!1897-09-041@{4. 9. 1897}|)be}\mylabel{h}  \normalsize

\doendnotes{C}
\bigskip
\vfill

\clearpage

\footnotesize

\lohead{\textsc{register}}

% Definiere theindex-Environment komplett neu ohne reledmac
\makeatletter
\renewenvironment{theindex}{%
  \section*{\indexname}%
  \setlength{\parindent}{0pt}%
  \setlength{\parskip}{0pt plus 0.3pt}%
  \let\item\@idxitem
}{%
  \clearpage
}
\makeatother

\IfFileExists{\jobname-pw.ind}{\input{\jobname-pw.ind}}{}

\end{document}

      