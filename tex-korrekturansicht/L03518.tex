%% latex-korrekturansicht-vorspann.tex
%% Vorspann für die Korrekturansicht.
%% Lädt die gemeinsame Datei latex-vorspann.tex mit gesetztem Schalter.

\newif\ifkorrekturansicht
\korrekturansichttrue

\input{../tex-inputs/latex-vorspann}


\renewcommand{\erwaehntePersonen}{Personen: Eva Marie Goldmann}
\renewcommand{\erwaehnteInstitutionen}{Institutionen: Neue Freie Presse}
\renewcommand{\erwaehnteOrte}{Orte: Bendlerstraße, Berlin, Velden am Wörthersee, Wien}
\renewcommand{\erwaehnteWerke}{Werke: ?? [Roman mit erotischen Schilderungen]}
\section[ Paul Goldmann an Arthur Schnitzler, 2. 8. 1931]{Paul Goldmann an Arthur Schnitzler, 2. 8. 1931}
\nopagebreak\mylabel{v}
\rehead{ }\normalsize\beginnumbering\briefempfaengerindex{Schnitzler, Arthur@\textsc{Schnitzler, Arthur}!zzzGoldmann, Paul@\emph{von Paul Goldmann}!1931-08-021@{2. 8. 1931}|(be}
\toendnotes[C]{\smallbreak\pagebreak[2]}\Standort{DLA, A:Schnitzler, HS.NZ85.1.3176.}
\physDesc{Brief, 1 Blatt, 1 Seite, 542 Zeichen
\newline{}Schreibmaschine
\newline{}Handschrift: blaue Tinte, lateinische Kurrent (\noindent{}eine Korrektur und Unterschrift)}\toendnotes[C]{\smallbreak}
\pstart
           \noindent{}{\pb}\textcolor{gray}{\textbf{Dr. Paul Goldmann}}\hfill \textcolor{gray}{\textbf{\textcolor{pink}{Berlin W. 10}{}\ledrightnote{\textcolor{pink}{Berlin}}}}\pend
           
\pstart
           \textcolor{gray}{\textbf{Vertreter der »\textcolor{brown}{Neuen Freien
                           Presse}{}\ledrightnote{\textcolor{brown}{Neue Freie Presse}}«}}\hfill \textcolor{gray}{\textbf{\textcolor{pink}{Bendlerſtraße 36}{}\ledrightnote{\textcolor{pink}{Bendlerstraße}}.}}\pend
           
\pstart
           \raggedleft{}\textcolor{gray}{\textbf{Tel. Lützow 9142}}\pend
           
\pstart
           \raggedleft{}2. 8. 31.\pend
           
\pstart\center{}Lieber Freund,\pend
\pstart
           Als rekommandierte Drucksache übersende ich Dir das \label{K_L03518-1v}\edtext{\textcolor{green}{Buch}{}\ledrightnote{{$\rightarrow$}\textcolor{green}{?? [Roman mit erotischen Schilderungen]}}}{\lemma{\textnormal{\emph{Buch}}}\Cendnote{\textnormal{nicht ermittelt, siehe Paul Goldmann an Arthur Schnitzler, 19. 5. 1931}}}\label{K_L03518-1h}, das Du so freundlich warst, mir zu borgen, und ich danke Dir herzlich\introOben{}st\introOben{} dafür. Es hat mich sehr interessiert, und ich finde, dass
               es, auch abgesehen von seinen erotischen Schilderungen, verdient, gelesen zu
               werden.\pend
           
\pstart
           Ich hoffe, dass Dich meine Sendungen nicht in \textcolor{pink}{Wien}{}\ledrightnote{\textcolor{pink}{Wien}}
               erreichen und dass Du an irgendeinem schönen Erholungsort Deinen Sommer verbringst.
               Ich gehe nächster Tage nach \textcolor{pink}{Velden am
               Wörthersee}{}\ledrightnote{\textcolor{pink}{Velden am Wörthersee}}.\pend
           
\pstart
           Mit vielen herzlichen Grüssen, auch von meiner \textcolor{blue}{Frau}{}\ledrightnote{{$\rightarrow$}\textcolor{blue}{Eva Marie Goldmann}}, {\\[\baselineskip]}{[}hs. Goldmann:{]} Dein {\\[\baselineskip]}\spacefill\mbox{Paul Goldmann}\pend
           \leftskip=0em{}\endnumbering\briefempfaengerindex{Schnitzler, Arthur@\textsc{Schnitzler, Arthur}!zzzGoldmann, Paul@\emph{von Paul Goldmann}!1931-08-021@{2. 8. 1931}|)be}\mylabel{h}
\begin{anhang}
\end{anhang}\normalsize

\doendnotes{C}
\bigskip
\vfill

\clearpage

\footnotesize

\lohead{\textsc{register}}

% Definiere theindex-Environment komplett neu ohne reledmac
\makeatletter
\renewenvironment{theindex}{%
  \section*{\indexname}%
  \setlength{\parindent}{0pt}%
  \setlength{\parskip}{0pt plus 0.3pt}%
  \let\item\@idxitem
}{%
  \clearpage
}
\makeatother

\IfFileExists{\jobname-pw.ind}{\input{\jobname-pw.ind}}{}

\end{document}

      