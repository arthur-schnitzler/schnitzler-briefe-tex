%% latex-korrekturansicht-vorspann.tex
%% Vorspann für die Korrekturansicht.
%% Lädt die gemeinsame Datei latex-vorspann.tex mit gesetztem Schalter.

\newif\ifkorrekturansicht
\korrekturansichttrue

\input{../tex-inputs/latex-vorspann}


\renewcommand{\erwaehntePersonen}{Personen: Vasilij Ilʹič Safonov, Ottilie Salten, Olga Schnitzler}
\renewcommand{\erwaehnteOrte}{Orte: Wien}
\renewcommand{\erwaehnteWerke}{}
\section[ Felix Salten an Arthur Schnitzler, {[}19?. 11. 1903{]}]{Felix Salten an Arthur Schnitzler, {[}19?. 11. 1903{]}}
\nopagebreak\mylabel{v}
\rehead{ }\normalsize\beginnumbering\briefempfaengerindex{Schnitzler, Arthur@\textsc{Schnitzler, Arthur}!zzzSalten, Felix@\emph{von Felix Salten}!1903-11-191@{{[}19?. 11. 1903{]}}|(be}
\toendnotes[C]{\smallbreak\pagebreak[2]}\Standort{CUL, Schnitzler, B 89, A 2.}
\physDesc{Brief, 1 Blatt, 1 Seite, 284 Zeichen
\newline{}Handschrift: schwarze Tinte, lateinische Kurrent
\newline{}Schnitzler: mit Bleistift datiert: »18/11 903« 
\newline{}Ordnung: mit Bleistift von unbekannter Hand nummeriert: »177« }\toendnotes[C]{\smallbreak}
\pstart
           \raggedleft{}{\pb}\label{K_L03351-1v}\edtext{Donnerstag}{\lemma{\textnormal{\emph{Donnerstag}}}\Cendnote{\textnormal{Diese Wochentagsangabe und die
                        Datierung \textcolor{blue}{Schnitzler}s auf einen
                        Mittwoch widersprechen sich. Durch die Angabe der Uhrzeit nach
                           Mitternacht dürfte das Schreiben am Donnerstag, dem 19., um 1 Uhr früh verfasst worden
                        sein.}}}\label{K_L03351-1h}.\pend
           
\pstart
           \raggedleft{}1\textsuperscript{h} früh.\pend
           
\pstart
           Lieber Freund, wenn Sie \textcolor{blue}{Beide}{}\ledrightnote{{$\rightarrow$}\textcolor{blue}{Olga Schnitzler}}{ }heute{ }Abend mit \textcolor{blue}{Safonoff}{}\ledrightnote{\textcolor{blue}{Vasilij Ilʹič Safonov}} bei \textcolor{blue}{uns}{}\ledrightnote{{$\rightarrow$}\textcolor{blue}{Ottilie Salten}} essen wollten (8\textsuperscript{h.}) würden wir uns herzlich darüber freuen. \textcolor{blue}{Safonoff}{}\ledrightnote{\textcolor{blue}{Vasilij Ilʹič Safonov}} ist eben angekommen, deshalb bitte ich wegen der knappen Frist um
               Entschuldigung. Sie pneumatisiren mir hoffentlich Ihre \label{K_L03351-2v}\edtext{Zusage}{\lemma{\textnormal{\emph{Zusage}}}\Cendnote{\textnormal{\textcolor{blue}{Schnitzler} war anderweitig verpflichtet,
                     vgl. A. S.: \emph{Tagebuch}, 19. 11. 1903.}}}\label{K_L03351-2h}\textcolor{gray}{.}\pend
           \pstart herzlichst Ihr \spacefill\mbox{S.}\pend{}\endnumbering\briefempfaengerindex{Schnitzler, Arthur@\textsc{Schnitzler, Arthur}!zzzSalten, Felix@\emph{von Felix Salten}!1903-11-191@{{[}19?. 11. 1903{]}}|)be}\mylabel{h}  \normalsize

\doendnotes{C}
\bigskip
\vfill

\clearpage

\footnotesize

\lohead{\textsc{register}}

% Definiere theindex-Environment komplett neu ohne reledmac
\makeatletter
\renewenvironment{theindex}{%
  \section*{\indexname}%
  \setlength{\parindent}{0pt}%
  \setlength{\parskip}{0pt plus 0.3pt}%
  \let\item\@idxitem
}{%
  \clearpage
}
\makeatother

\IfFileExists{\jobname-pw.ind}{\input{\jobname-pw.ind}}{}

\end{document}

      