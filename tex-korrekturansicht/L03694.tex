%% latex-korrekturansicht-vorspann.tex
%% Vorspann für die Korrekturansicht.
%% Lädt die gemeinsame Datei latex-vorspann.tex mit gesetztem Schalter.

\newif\ifkorrekturansicht
\korrekturansichttrue

\input{../tex-inputs/latex-vorspann}


\renewcommand{\erwaehntePersonen}{Personen: Hermann Bahr, Elsa Plessner}
\renewcommand{\erwaehnteInstitutionen}{Institutionen: Die Zeit. Wiener Wochenschrift}
\renewcommand{\erwaehnteOrte}{Orte: Florianigasse 44, Paris, Wien}
\renewcommand{\erwaehnteWerke}{Werke: Warten}
\section[Elsa Plessner an Arthur Schnitzler, {[}Mitte April 1897{]}]{Elsa Plessner an Arthur Schnitzler, {[}Mitte April 1897{]}}
\nopagebreak\mylabel{v}
\rehead{ }\normalsize\beginnumbering\briefempfaengerindex{Schnitzler, Arthur@\textsc{Schnitzler, Arthur}!zzzPlessner, Elsa@\emph{von Elsa Plessner}!1897-04-211@{{[}Mitte April
                     1897{]}}|(be}
\toendnotes[C]{\smallbreak\pagebreak[2]}\Standort{DLA, A:Schnitzler, 85.1.4198.}
\physDesc{Brief, 1 Blatt, 2 Seiten, 775 Zeichen
\newline{}Handschrift: , lateinische Kurrent
\newline{}Schnitzler: mit Bleistift datiert: »22/4 97« }
\buchAbdrucke{\weitereDrucke{Hermann Bahr, Arthur Schnitzler: \emph{Briefwechsel, Aufzeichnungen, Dokumente (1891–1931)}. Hg. Kurt Ifkovits und Martin Anton Müller. Göttingen: \emph{Wallstein} 2018, S. 141.} }\toendnotes[C]{\smallbreak}
\pstart
           \raggedleft{}{\pb}\textcolor{pink}{Wien VIII. Florianigasse N\textsuperscript{o} 44}{}\ledrightnote{\textcolor{pink}{Florianigasse 44}}.\pend
           
\pstart{}Hochverehrter Herr Doctor!\pend\vspace{0.5em}
\pstart
           Schon wieder einmal komme ich Sie um etwas zu bitten!!. Aber Sie sind ja immer so
               gut. Also die Sache ist die, dass ich bei Herrn \textcolor{blue}{H.
                  Bahr}{}\ledrightnote{\textcolor{blue}{Hermann Bahr}} die Novelle, die Sie »\textcolor{green}{Warten}{}\ledrightnote{\textcolor{green}{Warten}}«
               getauft haben, (bei mir hieß sie zuerst »Blätter«) – an die Sie sich hoffentlich noch
               erinnern – für »\textcolor{brown}{die Zeit}{}\ledrightnote{\textcolor{brown}{Die Zeit. Wiener Wochenschrift}}« eingereicht habe, und
               dass ich Sie nun herzlichst bitte, ein – (oder zwei?) \label{K_L03694-11v}\edtext{gute Worte}{\lemma{\textnormal{\emph{gute Worte}}}\Cendnote{\textnormal{siehe Arthur Schnitzler an Hermann Bahr, 22. 4. 1897.}}}\label{} für mich und
               sie bei genanntem Herrn einzulegen.\pend
           
\pstart
           Ich traue mich diesbezüglich nur deshalb an Sie heran, weil Ihnen die Arbeit
               seinerzeit gefiel. Aber – Sie wissen ja, wie das ist, – ein empfehlendes Wort
               Ihrerseits ist doch zehnmal gewichtiger als die beste Arbeit einer \label{K_L03694-22v}\edtext{obscurité}{\lemma{\textnormal{\emph{obscurité}}}\Cendnote{\textnormal{französisch, sinngemäß: Unerkannten}}}\label{}. – Also – besten
               herzlichsten Dank im voraus!\pend
           
\pstart
           In steter Verehrung{\\[\baselineskip]}\spacefill\mbox{ElsaPlessner}\pend
           \leftskip=0em{}\endnumbering\briefempfaengerindex{Schnitzler, Arthur@\textsc{Schnitzler, Arthur}!zzzPlessner, Elsa@\emph{von Elsa Plessner}!1897-04-101@{{[}Mitte April
                     1897{]}}|)be}\mylabel{h}
\begin{anhang}
\end{anhang}\normalsize

\doendnotes{C}
\bigskip
\vfill

\clearpage

\footnotesize

\lohead{\textsc{register}}

% Definiere theindex-Environment komplett neu ohne reledmac
\makeatletter
\renewenvironment{theindex}{%
  \section*{\indexname}%
  \setlength{\parindent}{0pt}%
  \setlength{\parskip}{0pt plus 0.3pt}%
  \let\item\@idxitem
}{%
  \clearpage
}
\makeatother

\IfFileExists{\jobname-pw.ind}{\input{\jobname-pw.ind}}{}

\end{document}

      