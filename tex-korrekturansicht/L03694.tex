%% latex-korrekturansicht-vorspann.tex
%% Vorspann für die Korrekturansicht.
%% Lädt die gemeinsame Datei latex-vorspann.tex mit gesetztem Schalter.

\newif\ifkorrekturansicht
\korrekturansichttrue

\input{../tex-inputs/latex-vorspann}


\section[Elsa Plessner an Arthur Schnitzler, {[}Mitte April 1897{]}]{L03694 Elsa Plessner an Arthur Schnitzler, {[}Mitte April 1897{]}}
\nopagebreak\mylabel{L03694v}
\rehead{ }\normalsize\beginnumbering\briefempfaengerindex{Schnitzler, Arthur@\textsc{Schnitzler, Arthur}!zzzPlessner, Elsa@\emph{von Elsa Plessner}!1897-04-212@{{[}Mitte April 1897{]}}|(be}
\toendnotes[C]{\smallbreak\pagebreak[2]}
\correspDesc{Versand  durch Elsa Plessner im Zeitraum [Mitte April 1897] in Wien
\newline{}Erhalt  durch Arthur Schnitzler am 22. 4. 1897 in Paris}\toendnotes[C]{\smallbreak}
\Standort{DLA, A:Schnitzler, 85.1.4198.}
\physDesc{Brief, 1 Blatt, 2 Seiten, 772 Zeichen
\newline{}Handschrift: schwarze Tinte, lateinische Kurrent
\newline{}Schnitzler: mit Bleistift den Empfang datiert: »\substVorne{}\textsuperscript{1}\substDazwischen{}2\substHinten{}2/4 97« }
\buchAbdrucke{\weitereDrucke{Hermann Bahr, Arthur Schnitzler: \emph{Briefwechsel, Aufzeichnungen, Dokumente (1891–1931)}. Herausgegeben von Kurt Ifkovits und Martin Anton Müller. Göttingen: \emph{Wallstein} 2018, S. 141.} }\toendnotes[C]{\smallbreak}
\pstart
           \raggedleft{}{\pb}\textcolor{pink}{Wien VIII. Florianigasse N\textsuperscript{o} 44}\oindex{Wien@\textbf{Wien}!VIII., Josefstadt@\textbf{VIII., Josefstadt}!Florianigasse 44@\textbf{Florianigasse 44}, \emph{Wohngebäude}|pw}{}\ledrightnote{\textcolor{pink}{Florianigasse 44}}.\pend
           
\pstart{}Hochverehrter Herr Doctor!\pend\vspace{0.5em}
\pstart
           Schon wieder einmal komme ich Sie um etwas zu bitten!!. Aber Sie sind ja immer so
               gut. Also die Sache ist die, dass ich bei Herrn \textcolor{blue}{H.
                  Bahr}\pwindex{Bahr, Hermann 19.\,7.\,1863 Linz – 15.\,1.\,1934 München@\textsc{Bahr, Hermann} (19.\,7.\,1863 Linz – 15.\,1.\,1934 München), \emph{Schriftsteller, Kritiker}|pw}{}\ledrightnote{\textcolor{blue}{Hermann Bahr}} die Novelle, die Sie »\textcolor{green}{Warten}\pwindex{Plessner, Elsa 22.\,8.\,1875 Wien – 7.\,5.\,1932 Alicante@\textsc{Plessner, Elsa} (22.\,8.\,1875 Wien – 7.\,5.\,1932 Alicante), \emph{Schriftstellerin}!Warten. Novelle@\strich\emph{Warten. Novelle}|pw}{}\ledrightnote{\textcolor{green}{Warten. Novelle}}«
               getauft haben, (bei mir hieß sie zuerst »Blätter«) – an die Sie sich hoffentlich
                  \label{K_L03694-1v}\edtext{noch erinnern}{\lemma{\textnormal{\emph{noch erinnern}}}\Cendnote{\textnormal{\textcolor{blue}{Plessner}\pwindex{Plessner, Elsa 22.\,8.\,1875 Wien – 7.\,5.\,1932 Alicante@\textsc{Plessner, Elsa} (22.\,8.\,1875 Wien – 7.\,5.\,1932 Alicante), \emph{Schriftstellerin}|pwk} hatte die \textcolor{green}{Erzählung}\pwindex{Plessner, Elsa 22.\,8.\,1875 Wien – 7.\,5.\,1932 Alicante@\textsc{Plessner, Elsa} (22.\,8.\,1875 Wien – 7.\,5.\,1932 Alicante), \emph{Schriftstellerin}!Warten. Novelle@\strich\emph{Warten. Novelle}|pwkv} am 14. 4. 1896{ }\textcolor{blue}{Schnitzler} in einer ersten Fassung zugesandt
                  und am 15. 9. 1896 in
                     einem Paket mit weiteren Texten erneut. Er äußerte seine Zustimmung zu dem \textcolor{green}{Text}\pwindex{Plessner, Elsa 22.\,8.\,1875 Wien – 7.\,5.\,1932 Alicante@\textsc{Plessner, Elsa} (22.\,8.\,1875 Wien – 7.\,5.\,1932 Alicante), \emph{Schriftstellerin}!Warten. Novelle@\strich\emph{Warten. Novelle}|pwkv},
                     vgl. Elsa Plessner an Arthur Schnitzler, 21. 9. 1896.}}}\label{K_L03694-1} – für »\textcolor{brown}{die Zeit}\orgindex{Zeit. Wiener Wochenschrift@Die Zeit. Wiener Wochenschrift|pw}{}\ledrightnote{\textcolor{brown}{Die Zeit. Wiener Wochenschrift}}« eingereicht habe, und dass ich Sie nun
               herzlichst bitte, ein – (oder zwei?) \label{K_L03694-2v}\edtext{gute Worte}{\lemma{\textnormal{\emph{gute Worte}}}\Cendnote{\textnormal{siehe Arthur Schnitzler an Hermann Bahr, 22. 4. 1897. }}}\label{K_L03694-2} für mich
               und sie bei genanntem Herrn einzulegen.\pend
           
\pstart
           {\pb}Ich traue mich diesbezüglich nur deshalb an Sie heran, weil Ihnen die \textcolor{green}{Arbeit}\pwindex{Plessner, Elsa 22.\,8.\,1875 Wien – 7.\,5.\,1932 Alicante@\textsc{Plessner, Elsa} (22.\,8.\,1875 Wien – 7.\,5.\,1932 Alicante), \emph{Schriftstellerin}!Warten. Novelle@\strich\emph{Warten. Novelle}|pwv}{}\ledrightnote{{$\rightarrow$}\emph{\textcolor{green}{Warten. Novelle}}} seinerzeit gefiel. Aber – Sie wissen
               ja, wie das ist, – ein empfehlendes Wort Ihrerseits ist doch zehnmal gewichtiger als
               die beste Arbeit einer \label{K_L03694-3v}\edtext{\begin{otherlanguage}{french}obscurité\end{otherlanguage}}{\lemma{\textnormal{\emph{obscurité}}}\Cendnote{\textnormal{französisch, sinngemäß:
                  Unerkannten}}}\label{K_L03694-3}. – Also – besten herzlichsten Dank im voraus!\pend
           
\pstart
           In steter Verehrung{\\[\baselineskip]}\spacefill\mbox{Elsa Plessner}\pend
           \leftskip=0em{}\selectlanguage{ngerman}\endnumbering\briefempfaengerindex{Schnitzler, Arthur@\textsc{Schnitzler, Arthur}!zzzPlessner, Elsa@\emph{von Elsa Plessner}!1897-04-152@{{[}Mitte April 1897{]}}|)be}\mylabel{L03694h}  \normalsize

\doendnotes{C}
\bigskip
\vfill

\clearpage

\footnotesize

\lohead{\textsc{register}}

% Definiere theindex-Environment komplett neu ohne reledmac
\makeatletter
\renewenvironment{theindex}{%
  \section*{\indexname}%
  \setlength{\parindent}{0pt}%
  \setlength{\parskip}{0pt plus 0.3pt}%
  \let\item\@idxitem
}{%
  \clearpage
}
\makeatother

\IfFileExists{\jobname-pw.ind}{\input{\jobname-pw.ind}}{}

\end{document}

      