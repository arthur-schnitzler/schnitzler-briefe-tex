%% latex-korrekturansicht-vorspann.tex
%% Vorspann für die Korrekturansicht.
%% Lädt die gemeinsame Datei latex-vorspann.tex mit gesetztem Schalter.

\newif\ifkorrekturansicht
\korrekturansichttrue

\input{../tex-inputs/latex-vorspann}


\renewcommand{\erwaehntePersonen}{Personen: Hans Breuer, Johanna von Hartenau, Isidor Mautner, Jenny Mautner, Felix Salten}
\renewcommand{\erwaehnteOrte}{Orte: Wien}
\renewcommand{\erwaehnteWerke}{}
\section[ Felix Salten an Arthur Schnitzler, 23. 2. 1926]{Felix Salten an Arthur Schnitzler, 23. 2. 1926}
\nopagebreak\mylabel{v}
\rehead{ }\normalsize\beginnumbering\briefempfaengerindex{Schnitzler, Arthur@\textsc{Schnitzler, Arthur}!zzzSalten, Felix@\emph{von Felix Salten}!1926-02-231@{23. 2. 1926}|(be}
\toendnotes[C]{\smallbreak\pagebreak[2]}\Standort{CUL, Schnitzler, B 89, B 2.}
\physDesc{Brief, 1 Blatt, 1 Seite, 607 Zeichen
\newline{}Schreibmaschine
\newline{}Handschrift: schwarze Tinte, lateinische Kurrent (\noindent{}Unterschrift)
\newline{}Ordnung: 1) mit Bleistift von unbekannter Hand beschriftet: »{\pb}Salten«  2) mit Bleistift von unbekannter Hand nummeriert: »297«}\toendnotes[C]{\smallbreak}
\pstart
           \raggedleft{}{\pb}\textcolor{pink}{Wien}{}\ledrightnote{\textcolor{pink}{Wien}}, 23. Februar 1926\pend
           
\pstart{}Lieber Schnitzler,\pend
\pstart
           Sie werden nächster Tage von der Gräfin \textcolor{blue}{Hartenau}{}\ledrightnote{\textcolor{blue}{Johanna von Hartenau}} ein Albumblatt erhalten und dazu das Ersuchen, dieses Blatt mit
               einer Widmung für das Ehepaar \textcolor{blue}{Isidor}{}\ledrightnote{\textcolor{blue}{Isidor Mautner}} und \textcolor{blue}{Jenny Mautner}{}\ledrightnote{\textcolor{blue}{Jenny Mautner}} zu versehen, da die beiden im
                  \label{K_L03585-1v}\edtext{nächsten Monat ihre goldene
                  Hochzeit}{\lemma{\textnormal{\emph{nächsten … Hochzeit}}}\Cendnote{\textnormal{Die Hochzeit hatte am 19. 3. 1876 in \textcolor{pink}{Wien}
                  stattgefunden.}}}\label{K_L03585-1h} feiern. Der \label{K_L03585-2v}\edtext{Tod ihres Schwiegersohnes, des Dr. \textcolor{blue}{Hans
                  Breuer}{}\ledrightnote{\textcolor{blue}{Hans Breuer}}}{\lemma{\textnormal{\emph{Tod … Breuer}}}\Cendnote{\textnormal{\textcolor{blue}{Hans Breuer} war am 27. 1. 1926 verstorben.}}}\label{K_L03585-2h}, hat jedes Fest, das geplant wurde,
               unmöglich gemacht und das Album soll, wie mir Gräfin \textcolor{blue}{Hartenau}{}\ledrightnote{\textcolor{blue}{Johanna von Hartenau}} sagt, die einzige Freude sein, die man \textcolor{blue}{Mautners}{}\ledrightnote{\textcolor{blue}{Isidor Mautner}{\newline}\textcolor{blue}{Jenny Mautner}} bereiten kann. Die \textcolor{blue}{Gräfin}{}\ledrightnote{{$\rightarrow$}\textcolor{blue}{Johanna von Hartenau}} hat mich ersucht, bei
               Ihnen wegen Ausfüllung des Albumblattes vorstellig zu werden.\pend
           
\pstart
           Mit herzlichem Gruss {\\[\baselineskip]}Ihr {\\[\baselineskip]}{[}hs.:{]} \spacefill\mbox{Felix Salten}\pend
           \leftskip=0em{}\endnumbering\briefempfaengerindex{Schnitzler, Arthur@\textsc{Schnitzler, Arthur}!zzzSalten, Felix@\emph{von Felix Salten}!1926-02-231@{23. 2. 1926}|)be}\mylabel{h}  \normalsize

\doendnotes{C}
\bigskip
\vfill

\clearpage

\footnotesize

\lohead{\textsc{register}}

% Definiere theindex-Environment komplett neu ohne reledmac
\makeatletter
\renewenvironment{theindex}{%
  \section*{\indexname}%
  \setlength{\parindent}{0pt}%
  \setlength{\parskip}{0pt plus 0.3pt}%
  \let\item\@idxitem
}{%
  \clearpage
}
\makeatother

\IfFileExists{\jobname-pw.ind}{\input{\jobname-pw.ind}}{}

\end{document}

      