%% latex-korrekturansicht-vorspann.tex
%% Vorspann für die Korrekturansicht.
%% Lädt die gemeinsame Datei latex-vorspann.tex mit gesetztem Schalter.

\newif\ifkorrekturansicht
\korrekturansichttrue

\input{../tex-inputs/latex-vorspann}


               \section[Paul Goldmann an Arthur Schnitzler, 5. 7. 1891]{ Paul Goldmann an Arthur Schnitzler, 5. 7. 1891}\nopagebreak\mylabel{v}\rehead{ }\normalsize\beginnumbering\briefempfaengerindex{Schnitzler, Arthur@\textsc{Schnitzler, Arthur}!zzzGoldmann, Paul@\emph{von Paul Goldmann}!1891-07-051@{5. 7. 1891}|(be} \toendnotes[C]{\smallbreak\pagebreak[2]} \Standort{DLA, A:Schnitzler, HS.NZ85.1.3162.}
\physDesc{Postkarte
\newline{}Handschrift: 1) schwarze Tinte, deutsche Kurrent\hspace{1em}2) schwarze Tinte, lateinische Kurrent (\noindent{}Adresse)\hspace{1em}\newline{}Versand: 1) Stempel: »\nobreak{}\oindex{Bruessel@\textbf{Brüssel}, \emph{Besiedelter Ort (A.BSO)}|pwk}\begin{otherlanguage}{dutch}’Sgravenhage\end{otherlanguage}, \begin{otherlanguage}{dutch}5 Jul 91\end{otherlanguage}, 7–8N\nobreak{}«.  2) Stempel: »\nobreak{}Wien 1/1, 7{[}.{]} 7. 91, 4\textsuperscript{1}/\textsubscript{2}
                                       - 6N, Bestellt\nobreak{}«. 
\newline{}Schnitzler: mit Bleistift das Empfangsdatum »7/ 7 91« und das Jahr »91« vermerkt }\toendnotes[C]{\smallbreak}\pstart{}{\pb}\textcolor{pink}{Autriche}{}\ledrightnote{\textcolor{pink}{Österreich}}! \pend{}\pstart{}Herrn\pend{}\pstart{}Dr. Arthur Schnitzler\pend{}\pstart{}\textcolor{pink}{Wien}{}\ledrightnote{\textcolor{pink}{Wien}}\pend{}\pstart{}\textcolor{pink}{I. Giselastraße 11}{}\ledrightnote{\textcolor{pink}{Bösendorferstraße}}.\pend{}{\bigskip}\pstart
           \noindent{}{\pb}\textcolor{pink}{Haag}{}\ledrightnote{\textcolor{pink}{Den Haag}}, 6. Juli.\hspace*{1.5em}Mein lieber Arthur! Einen herzlichen Gruß von
               unterwegs. Ich bin zur \label{K_L02666-1v}\edtext{\textcolor{brown}{Puppenausſtellung}{}\ledrightnote{\textcolor{brown}{Puppenaustellung}}}{\lemma{\textnormal{\emph{Puppenausſtellung}}}\Cendnote{\textnormal{Die \emph{\textcolor{brown}{Puppenausstellung}} in \textcolor{pink}{Scheveningen}
                  fand von 4. 7. 1891 bis 4. 8. 1891 statt.}}}\label{K_L02666-1h} nach
                  \textsc{\textcolor{pink}{Scheveningen}{}\ledrightnote{\textcolor{pink}{Scheveningen}}} geſchickt worden u. habe bei dieſer Gelegenheit ein Stück \textcolor{pink}{Holland}{}\ledrightnote{\textcolor{pink}{Niederlande}} mit angeſehen. Unvergeßliche u. unvergleichliche
               Eindrücke in \textcolor{pink}{Rotterdam}{}\ledrightnote{\textcolor{pink}{Rotterdam}}, \textcolor{pink}{Haag}{}\ledrightnote{\textcolor{pink}{Den Haag}} und am Meer! Eine neue Welt, in der Alles ſympathiſch
               iſt, \strikeout{ohne} ohne ſchön zu ſein, und wo doch vieles
               ſchön iſt, vieles neu ohne Gleichen u. ſympathiſch iſt. Näheres aus \textcolor{pink}{Brüſſel}{}\ledrightnote{\textcolor{pink}{Brüssel}}. – Gekreuzt? Wann haben ſich 2 Briefe von uns gekreuzt? \substVorne{}\textsuperscript{\textcolor{gray}{Seit}}\substDazwischen{}Vor\substHinten{} Deinem \label{T_L02666-1v}\edtext{letzten habe ich Monate
               lang nichts}{\lemma{\textnormal{\emph{letzten … nichts}}}\Cendnote{\textnormal{seitlich am rechten
                  Rand}}}\label{T_L02666-1h}{ }\label{T_L02666-2v}\edtext{von Dir erhalten?! – Dein treuer
                  \spacefill\mbox{Paul Goldmann.}}{\lemma{\textnormal{\emph{von … Goldmann.}}}\Cendnote{\textnormal{kopfüber am oberen Rand}}}\label{T_L02666-2h}\pend
           \endnumbering\briefempfaengerindex{Schnitzler, Arthur@\textsc{Schnitzler, Arthur}!zzzGoldmann, Paul@\emph{von Paul Goldmann}!1891-07-051@{5. 7. 1891}|)be}\mylabel{h}  \normalsize

\doendnotes{C}
\bigskip
\vfill

\clearpage

\footnotesize

\lohead{\textsc{register}}

% Definiere theindex-Environment komplett neu ohne reledmac
\makeatletter
\renewenvironment{theindex}{%
  \section*{\indexname}%
  \setlength{\parindent}{0pt}%
  \setlength{\parskip}{0pt plus 0.3pt}%
  \let\item\@idxitem
}{%
  \clearpage
}
\makeatother

\IfFileExists{\jobname-pw.ind}{\input{\jobname-pw.ind}}{}

\end{document}

      