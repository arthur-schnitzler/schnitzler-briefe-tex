%% latex-korrekturansicht-vorspann.tex
%% Vorspann für die Korrekturansicht.
%% Lädt die gemeinsame Datei latex-vorspann.tex mit gesetztem Schalter.

\newif\ifkorrekturansicht
\korrekturansichttrue

\input{../tex-inputs/latex-vorspann}


\section[Arthur Schnitzler an Stefan Zweig, 20. 8. 1920]{L03763 Arthur Schnitzler an Stefan Zweig, 20. 8. 1920}
\nopagebreak\mylabel{L03763v}
\rehead{ }\normalsize\beginnumbering\briefempfaengerindex{, @\textsc{, }!zzz, @\emph{von  }!1920-08-201@{20. 8. 1920}|(be}
\toendnotes[C]{\smallbreak\pagebreak[2]}\Standort{Jerusalem, National Library of Israel, ARC. Ms. Var. 305 1 58 Stefan Zweig Collection.}
\physDesc{Brief, 1 Blatt, 1 Seite, 688 Zeichen
\newline{}Schreibmaschine
\newline{}Handschrift: schwarze Tinte (\noindent{}Unterschrift)}\toendnotes[C]{\smallbreak}
\pstart
           {\pb}\textcolor{gray}{\textbf{D\textsuperscript{R} ARTHUR SCHNITZLER}}\hfill 20. 8. 1920. \pend
           
\pstart
           \textcolor{gray}{\textbf{\textcolor{pink}{WIEN, XVIII.
                        STERNWARTESTRASSE 71}\oindex{Wien@\textbf{Wien}!XVIII., Währing@\textbf{XVIII., Währing}!Sternwartestraße 71@\textbf{Sternwartestraße 71}, \emph{Wohngebäude}|pw}{}\ledrightnote{\textcolor{pink}{Sternwartestraße 71}}.}}\pend
           
\pstart{}Lieber Herr Dr. Zweig.\pend\vspace{0.5em}
\pstart
           Vielen Dank für Ihren \label{K_L03763-1v}\edtext{Brief}{\lemma{\textnormal{\emph{Brief}}}\Cendnote{\textnormal{Stefan Zweig an Arthur Schnitzler, 18. 8. 1920.}}}\label{K_L03763-1} und für Ihr \label{K_L03763-2v}\edtext{Telegramm}{\lemma{\textnormal{\emph{Telegramm}}}\Cendnote{\textnormal{Das Telegramm ist nicht erhalten, vgl. Stefan Zweig an Arthur Schnitzler, 18. 8. 1920.}}}\label{K_L03763-2}, das um einen
               Tag später ankam als Ihr Brief. Zu den 10{\%} habe ich mich auch
                  \label{K_L03763-3v}\edtext{entschlossen}{\lemma{\textnormal{\emph{entschlossen}}}\Cendnote{\textnormal{\textcolor{blue}{Schnitzler} schrieb an den Verleger \textcolor{blue}{Thomas Seltzer}\pwindex{Seltzer, Thomas 22.\,2.\,1875 Poltava – 11.\,9.\,1943 New York City@\textsc{Seltzer, Thomas} (22.\,2.\,1875 Poltava – 11.\,9.\,1943 New York City), \emph{Übersetzer, Verleger}|pwk}: »20. 8. 1920.{ / }Sehr geehrter Herr.{ / }Herr \textcolor{blue}{Stefan Zweig}\pwindex{Zweig, Stefan 28.\,11.\,1881 Wien – 23.\,2.\,1942 Petrópolis@\textsc{Zweig, Stefan} (28.\,11.\,1881 Wien – 23.\,2.\,1942 Petrópolis), \emph{Schriftsteller}|pw} macht mir von Ihrem
                        freundlichen Wunsch Mitteilung meine Novelle ›\textcolor{green}{Casanovas Heimfahrt}\pwindex{Schnitzler, Arthur 15. 5. 1862 Wien – 21. 10. 1931 ebd.@\textsc{Schnitzler, Arthur} (15. 5. 1862 Wien – 21. 10. 1931 ebd.), \emph{Schriftsteller, Mediziner}!Casanovas Heimfahrt@\strich\emph{Casanovas Heimfahrt}|pw}‹ in englischer Uebersetzung
                        herauszugeben. Ich würde Ihnen diese Bewilligung gerne zur folgenden
                        Bedingungen erteilen:{ / }10{\%} vom Ladenpreis jedes Exemplars, zahlbar bei
                        Erscheinen der betreffenden Auflagen. Zahlung eines Betrages von fünfhundert
                        Dollars bei Abschluss des Vertrages, die als Vorschuss in Abzug zu bringen
                        sind.{ / }Ihrer freundlichen Antwort sehe ich gerne entgegen und erwarte auch nähere
                        Angaben über den von Ihnen gewählten \textcolor{blue}{Uebersetzer}\pwindex{Paul, Eden 1.\,11.\,1865 Sturminster Marshall – 1.\,12.\,1944@\textsc{Paul, Eden} (1.\,11.\,1865 Sturminster Marshall – 1.\,12.\,1944), \emph{Schriftsteller, Mediziner}|pwv}\pwindex{Paul, Cedar 1880 – 18.\,3.\,1972@\textsc{Paul, Cedar} (1880 – 18.\,3.\,1972), \emph{Schriftstellerin}|pwv}, ferner darüber, wie hoch Sie
                        den Ladenpreis eines Exemplars anzusetzen gedenken und wieviele Auflagen sie
                        drucken wollen.{ / }Mit vorzüglicher Hochachtung{ / }{[}Raum für Unterschrift{]}{ / }Verleger \textcolor{blue}{Thomas Seltzer}\pwindex{Seltzer, Thomas 22.\,2.\,1875 Poltava – 11.\,9.\,1943 New York City@\textsc{Seltzer, Thomas} (22.\,2.\,1875 Poltava – 11.\,9.\,1943 New York City), \emph{Übersetzer, Verleger}|pw}, \textcolor{pink}{New-York}\oindex{New York City@\textbf{New York City}|pw}«. (\emph{Deutsches Literaturarchiv Marbach}, HS.NZ85.1.1911,1.) \emph{\textcolor{green}{Casanova’s Homecoming}\pwindex{Schnitzler, Arthur 15. 5. 1862 Wien – 21. 10. 1931 ebd.@\textsc{Schnitzler, Arthur} (15. 5. 1862 Wien – 21. 10. 1931 ebd.), \emph{Schriftsteller, Mediziner}!Casanova’s Homecoming@\strich\emph{Casanova’s Homecoming}|pwk}} erschien in der
                     Übersetzung von \textcolor{blue}{Eden Paul}\pwindex{Paul, Eden 1.\,11.\,1865 Sturminster Marshall – 1.\,12.\,1944@\textsc{Paul, Eden} (1.\,11.\,1865 Sturminster Marshall – 1.\,12.\,1944), \emph{Schriftsteller, Mediziner}|pwk} und \textcolor{blue}{Cedar Paul}\pwindex{Paul, Cedar 1880 – 18.\,3.\,1972@\textsc{Paul, Cedar} (1880 – 18.\,3.\,1972), \emph{Schriftstellerin}|pwk} 
                  im Jahr 1922, wobei \textcolor{blue}{Schnitzler} über Jahre damit
                     beschäftigt ist, die vollständige Zahlung seines Honorar einzufordern.
                  }}}\label{K_L03763-3}. Mit dem Vorschuss bin ich etwas höher gegangen. Ich glaube, wir
               sollten nicht immer umrechnen. Hundert Dollars sind doch nicht mehr als fünfhundert
               Kronen, nicht zwanzigtausend, wie uns die \textcolor{pink}{Amerikaner}\oindex{Vereinigte Staaten von Amerika [USA]@\textbf{Vereinigte Staaten von Amerika [USA]}|pw}{}\ledrightnote{\textcolor{pink}{Vereinigte Staaten von Amerika [USA]}} jetzt einreden wollen. Und ich stelle meine Honorarforderungen,
               wenn irgend möglich, von diesem Standpunkt aus. Dass ich damit bisher immer reussiert
               hätte, will ich allerdings nicht behaupten. \pend
           
\pstart
           Auf baldiges Wiedersehen entweder in \textcolor{pink}{Salzburg}\oindex{Salzburg@\textbf{Salzburg}, \emph{Verwaltungsgebiet}|pw}{}\ledrightnote{\textcolor{pink}{Salzburg}} oder
               in \textcolor{pink}{Wien}\oindex{Wien@\textbf{Wien}, \emph{Verwaltungsgebiet}|pw}{}\ledrightnote{\textcolor{pink}{Wien}}. \pend
           
\pstart
           Seien Sie herzlichst gegrüsst von Ihrem sehr ergebenen{\\[\baselineskip]}\spacefill\mbox{{[}hs.:{]} Arthur Schnitzler}\pend
           \leftskip=0em{}\selectlanguage{ngerman}\endnumbering\briefempfaengerindex{, @\textsc{, }!zzz, @\emph{von  }!1920-08-201@{20. 8. 1920}|)be}\mylabel{L03763h}
\begin{anhang}
\end{anhang}\normalsize

\doendnotes{C}
\bigskip
\vfill

\clearpage

\footnotesize

\lohead{\textsc{register}}

% Definiere theindex-Environment komplett neu ohne reledmac
\makeatletter
\renewenvironment{theindex}{%
  \section*{\indexname}%
  \setlength{\parindent}{0pt}%
  \setlength{\parskip}{0pt plus 0.3pt}%
  \let\item\@idxitem
}{%
  \clearpage
}
\makeatother

\IfFileExists{\jobname-pw.ind}{\input{\jobname-pw.ind}}{}

\end{document}

      