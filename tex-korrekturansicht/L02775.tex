%% latex-korrekturansicht-vorspann.tex
%% Vorspann für die Korrekturansicht.
%% Lädt die gemeinsame Datei latex-vorspann.tex mit gesetztem Schalter.

\newif\ifkorrekturansicht
\korrekturansichttrue

\input{../tex-inputs/latex-vorspann}


               \section[Paul Goldmann an Arthur Schnitzler, Paul Goldmann an Arthur Schnitzler, 24. 5.{[}1896{]}]{ Paul Goldmann an Arthur Schnitzler, 24. 5.{[}1896{]}}\nopagebreak\mylabel{v}\rehead{ }\normalsize\beginnumbering\briefempfaengerindex{Schnitzler, Arthur@\textsc{Schnitzler, Arthur}!zzzGoldmann, Paul@\emph{von Paul Goldmann}!1896-05-241@{24. 5.{[}1896{]}}|(be} \toendnotes[C]{\smallbreak\pagebreak[2]} \Standort{DLA, A:Schnitzler, HS.NZ85.1.3166.}
\physDesc{Brief, 2 Blätter, 8 Seiten
\newline{}Handschrift: blaue Tinte, deutsche Kurrent
\newline{}Schnitzler: 1) mit Bleistift das Jahr »96« vermerkt 2) mit rotem Buntstift auf der ersten Seite Vermerk: »\textcolor{blue}{Kerr}« und insgesamt drei Unterstreichungen}\toendnotes[C]{\smallbreak}\pstart
           \noindent{}{\pb}\textcolor{gray}{\textbf{\textbf{\textcolor{brown}{Frankfurter Zeitung}{}\ledrightnote{\textcolor{brown}{Frankfurter Zeitung}}}}}\pend
           \pstart
           \textcolor{gray}{\textbf{(\textcolor{brown}{\begin{otherlanguage}{french}Gazette de Francfort\end{otherlanguage}}{}\ledrightnote{\textcolor{brown}{Frankfurter Zeitung}}).}}\pend
           \pstart
           \textcolor{gray}{\textbf{\textbf{\begin{otherlanguage}{french}Fondateur M.\end{otherlanguage}{ }\textcolor{blue}{L. Sonnemann}{}\ledrightnote{\textcolor{blue}{Leopold Sonnemann}}.}}}\pend
           \pstart
           \begin{otherlanguage}{french}\textcolor{gray}{\textbf{\textcolor{green}{Journal}{}\ledrightnote{→\textcolor{green}{Frankfurter Zeitung}} politique,
                        financier,}}\end{otherlanguage}\pend
           \pstart
           \begin{otherlanguage}{french}\textcolor{gray}{\textbf{commercial et littéraire.}}\end{otherlanguage}\pend
           \pstart
           \begin{otherlanguage}{french}\textcolor{gray}{\textbf{\textbf{Paraissant trois fois par jour.}}}\end{otherlanguage}\pend
           \pstart
           \begin{otherlanguage}{french}\textcolor{gray}{\textbf{\textbf{Bureau à \textcolor{pink}{Paris}{}\ledrightnote{\textcolor{pink}{Paris}}}}}\end{otherlanguage}\hfill \textsc{\textcolor{pink}{Paris}{}\ledrightnote{\textcolor{pink}{Paris}}}, 24. Mai.\pend
           \pstart
           \begin{otherlanguage}{french}\textcolor{gray}{\textbf{\textbf{\textcolor{pink}{24. Rue Feydeau}{}\ledrightnote{\textcolor{pink}{rue Feydeau}}.}}}\end{otherlanguage}\pend
           \pstart\center{}Mein lieber Freund,\pend\pstart
           Vielen Dank für die »\textcolor{green}{Freie
                  Bühne}{}\ledrightnote{→\textcolor{green}{Neue Deutsche Rundschau}}«, die ich anbei zurückſende. (Das heißt nicht »anbei«. Ich behalte ſie
               noch bis Dienſtag, um ſie \textsc{M. \textcolor{blue}{Schefer}{}\ledrightnote{\textcolor{blue}{Christian Schefer}}} zu zeigen, der mich an dieſem Tage beſuchen
               kommt). Der \label{K_L02775-88v}\edtext{\textcolor{green}{Artikel}{}\ledrightnote{→\textcolor{green}{Arthur Schnitzler}}}{\lemma{\textnormal{\emph{Artikel}}}\Cendnote{\textnormal{\textcolor{blue}{Alfred Kerr}: \emph{\textcolor{green}{Arthur Schnitzler}}. In: \emph{\textcolor{green}{Neue Deutsche Rundschau (Freie Bühne)}}, Jg. 7, H. 3, März 1896, S. 287–292.}}}\label{K_L02775-88h} iſt höchſt
               intereſſant. Ich freue mich über den ſchönen Enthuſiasmus, den mein lieber \textsc{Arthur} erregt. Auch ſagt der {\pb}\textcolor{blue}{Verfaſſer}{}\ledrightnote{→\textcolor{blue}{Alfred Kerr}} manches Richtige.
               Im Allgemeinen aber ſind \strikeout{\textcolor{gray}{E}} mir ſeine kraftgenialiſche Art und Styl nicht ſehr ſympathiſch.\pend
           \pstart
           \label{K_L02775-24v}\edtext{Beifolgenden Brief}{\lemma{\textnormal{\emph{Beifolgenden Brief}}}\Cendnote{\textnormal{Beilage nicht erhalten, Verfasser nicht
                  identifiziert}}}\label{K_L02775-24h} empfehle ich \strikeout{D\textcolor{gray}{ic}h} Dir aufs Wärmſte zur bejahenden Beantwortung.
               Verfaſſer iſt ein \textcolor{blue}{Vetter}{}\ledrightnote{→\textcolor{blue}{?? [Vetter von Heinrich Kanner]}} von
                  \textsc{\textcolor{blue}{Kanner}{}\ledrightnote{\textcolor{blue}{Heinrich Kanner}}} – kreuzbraver \textcolor{blue}{Menſch}{}\ledrightnote{→\textcolor{blue}{?? [Vetter von Heinrich Kanner]}} –
               ſelbſt ſchwer lungenleidend, der wohl im »\textcolor{green}{Sterben}{}\ledrightnote{\textcolor{green}{Sterben. Novelle}}« ein Stück {\pb}ſeines Schickſals
               gefunden hat.\pend
           \pstart
           Über den \label{K_L02775-13v}\edtext{»\textcolor{green}{Vortrag}{}\ledrightnote{→\textcolor{green}{Poesie und Leben. Aus einem Vortrage}}«}{\lemma{\textnormal{\emph{»Vortrag«}}}\Cendnote{\textnormal{\textcolor{blue}{Hugo von Hofmannsthal}: \emph{\textcolor{green}{Poesie und Leben}}. In: \emph{\textcolor{green}{Die Zeit}}, Bd. 7, Nr. 85, 16. 5. 1896,
                     S. 104–106.}}}\label{K_L02775-13h} von \textsc{\textcolor{blue}{Loris}{}\ledrightnote{\textcolor{blue}{Hugo von Hofmannsthal}}}, den die letzte »\textcolor{green}{Zeit}{}\ledrightnote{\textcolor{green}{Die Zeit. Wiener Wochenschrift}}« gebracht, war ich
               wüthend. Ich verſtehe nicht ein Wort von dem, was er will. Und dann Stellen, wie:
                  »\textcolor{green}{Eine neue und kühne Verbindung
                  von Worten iſt das wundervollſte Geſchenk für die Seelen und nichts geringeres als
                  ein Standbild des Knaben \textsc{\textcolor{blue}{Antinous}{}\ledrightnote{\textcolor{blue}{Antinoos}}} oder eine große gewölbte Pforte}{}\ledrightnote{→\textcolor{green}{Poesie und Leben. Aus einem Vortrage}}«. Das iſt doch unerhört! Was iſt eine
               große gewölbte {\pb}Pforte für die Seelen? Und was hat
               das, zum Teufel, mit dem Standbild des Knaben \textsc{\textcolor{blue}{Antinous}{}\ledrightnote{\textcolor{blue}{Antinoos}}} zu thun? Ich will nicht ausſchließen, daß das wirklich empfunden iſt. Aber wenn
               auch – ſo thut das eine ganz unerhörte Empfindungen-Verwirrung dar. Auch iſt es eine
               verfluchte Schlamperei, ſich ſo gehen zu laſſen und jede \label{K_L02775-22v}\edtext{\textsc{\begin{otherlanguage}{french}incohérence\end{otherlanguage}}}{\lemma{\textnormal{\emph{incohérence}}}\Cendnote{\textnormal{französisch: mangelnder Zusammenhang}}}\label{K_L02775-22h}
               auszuſprechen, die Einem durchs Hirn fährt, \strikeout{die \textcolor{gray}{×}\-\textcolor{gray}{×}\-\textcolor{gray}{×}\-\textcolor{gray}{×}\-\textcolor{gray}{×}\-\textcolor{gray}{×}\-\textcolor{gray}{×}{ }\textcolor{gray}{×}\-\textcolor{gray}{×}\-\textcolor{gray}{×}\-\textcolor{gray}{×}{ }\textcolor{gray}{wird}} in der Überzeugung, das {\pb}ſei genial. Auch
               wird die Literatur auf dieſe Weiſe zu einer Geheim-Sprache, die nur mehr ein paar
               Eingeweihte verſtehen. Dieſer junge \textcolor{blue}{Mann}{}\ledrightnote{→\textcolor{blue}{Hugo von Hofmannsthal}} ſchreibt doch fürs Publicum. Und wenn er ſich nicht mehr
               ſo ausdrücken kann, daß ihn das Publicum versteht – wenn ſeine Gedanken einen Flug
               nehmen, {\pb}wo die Maſſe ihm nicht nach kann und wo er
               ſelbſt kaum noch mit kann – dann ſoll er eben \strikeout{kei\textcolor{gray}{n}}\textcolor{gray}{} nichts mehr drucken laſſen und keine Vorträge halten. Hübſch iſt auch, daß
               es einmal heißt, »\textcolor{green}{bei den neueren deutſchen ſogenannten Dichtern}{}\ledrightnote{→\textcolor{green}{Poesie und Leben. Aus einem Vortrage}}«. Und weiter unten:
               »\textcolor{green}{Sie wundern ſich, daß Ihnen \uline{ein Dichter} die Regeln
               lobt \textsc{etc.}}{}\ledrightnote{→\textcolor{green}{Poesie und Leben. Aus einem Vortrage}}«
                Alſo größenwahnſinnig {\pb}iſt dieſer junge \textcolor{blue}{Mann}{}\ledrightnote{→\textcolor{blue}{Hugo von Hofmannsthal}} auch ſchon. Worauf hin? Mit dem »jungen \textsc{\textcolor{blue}{Goethe}{}\ledrightnote{\textcolor{blue}{Johann Wolfgang von Goethe}}}« iſt es bisher nichts geworden. Bisher hat es eigentlich nur in einem Punkte
               geſtimmt: in der Jugend.\pend
           \pstart
           Nein, iſt dieſer arme kleine \textcolor{blue}{Burſch}{}\ledrightnote{→\textcolor{blue}{Hugo von Hofmannsthal}} verdorben worden\strikeout{!} von \textsc{\textcolor{blue}{Bahr}{}\ledrightnote{\textcolor{blue}{Hermann Bahr}}}, dieſem verfluchten \textcolor{blue}{Pfuſcher}{}\ledrightnote{→\textcolor{blue}{Hermann Bahr}} und \textcolor{blue}{Schurken}{}\ledrightnote{→\textcolor{blue}{Hermann Bahr}}!\pend
           \pstart
           {\pb}Grüß’ Dich Gott, liebſter Freund.\pend
           \pstart
           Auch ſchreibſt Du mir wohl nächſtens einmal.\pend
           \pstart
           Dein {\\[\baselineskip]}treuer {\\[\baselineskip]}\spacefill\mbox{Paul Goldmann}\pend
           \leftskip=0em{}\endnumbering\briefempfaengerindex{Schnitzler, Arthur@\textsc{Schnitzler, Arthur}!zzzGoldmann, Paul@\emph{von Paul Goldmann}!1896-05-241@{24. 5.{[}1896{]}}|)be}\mylabel{h}  \normalsize

\doendnotes{C}
\bigskip
\vfill

\clearpage

\footnotesize

\lohead{\textsc{register}}

% Definiere theindex-Environment komplett neu ohne reledmac
\makeatletter
\renewenvironment{theindex}{%
  \section*{\indexname}%
  \setlength{\parindent}{0pt}%
  \setlength{\parskip}{0pt plus 0.3pt}%
  \let\item\@idxitem
}{%
  \clearpage
}
\makeatother

\IfFileExists{\jobname-pw.ind}{\input{\jobname-pw.ind}}{}

\end{document}

      