%% latex-korrekturansicht-vorspann.tex
%% Vorspann für die Korrekturansicht.
%% Lädt die gemeinsame Datei latex-vorspann.tex mit gesetztem Schalter.

\newif\ifkorrekturansicht
\korrekturansichttrue

\input{../tex-inputs/latex-vorspann}


\renewcommand{\erwaehntePersonen}{Personen: Stefan George, Max Henze, Hugo von Hofmannsthal,  Jesus, Felix Salten, Gustav Schwarzkopf}
\renewcommand{\erwaehnteInstitutionen}{Institutionen: Allgemeine Theater-Revue für Bühne und Welt. Illustrierte Halbmonatsschrift, Heinrich Minden}
\renewcommand{\erwaehnteOrte}{Orte: Berlin, Café Central, Dresden, Leipzig, Wien}
\renewcommand{\erwaehnteWerke}{Werke: Blätter für die Kunst, Der Tod des Tizian. Ein Bruchstück, Die Bilanz der Ehe. Novellistische Studien. 2 Bde., Vielfarbige Distichen V.}
\section[Arthur Schnitzler an Felix Salten, 21. 3. 1892]{Arthur Schnitzler an Felix Salten, 21. 3. 1892}
\nopagebreak\mylabel{v}
\rehead{ }\normalsize\beginnumbering\briefempfaengerindex{Salten, Felix@\textsc{Salten, Felix}!zzzSchnitzler, Arthur@\emph{von Arthur Schnitzler}!1892-03-211@{21. 3. 1892}|(be}
\toendnotes[C]{\smallbreak\pagebreak[2]}\Standort{Wienbibliothek im Rathaus, ZPH 1681, 2.1.516.}
\physDesc{Brief, 1 Blatt, 4 Seiten, 623 Zeichen
\newline{}Handschrift: Bleistift, deutsche Kurrent
\newline{}Ordnung: mit Bleistift von unbekannter Hand Nummerierung der Doppelseiten des
                                 Konvoluts: »84«–»85« }\toendnotes[C]{\smallbreak}
\pstart
           \raggedleft{}{\pb}21/3 92{\\}\textcolor{pink}{Wien}{}\ledrightnote{\textcolor{pink}{Wien}}.\pend
           
\pstart{}Lieber Freund,\pend
\pstart
           \label{K_L02955-1v}\edtext{\textsc{\textcolor{blue}{Loris}{}\ledrightnote{\textcolor{blue}{Hugo von Hofmannsthal}}} war Nachmittg bei mir}{\lemma{\textnormal{\emph{Loris … mir}}}\Cendnote{\textnormal{siehe A. S.: \emph{Tagebuch}, 21. 3. 1892}}}\label{K_L02955-1h}. Hat beiliegenden \label{K_L02955-2v}\edtext{Brief}{\lemma{\textnormal{\emph{Brief}}}\Cendnote{\textnormal{Beilage nicht erhalten}}}\label{K_L02955-2h} erhalten,
               welchen er Sie zu erledigen bittet. – Zugleich erſucht er Sie um ſeine \textsc{\textcolor{green}{\label{K_L02955-3v}\edtext{Distichen}{\lemma{\textnormal{\emph{Distichen}}}\Cendnote{\textnormal{Ende Juli 1891 hatte \textcolor{blue}{Hofmannsthal} an \textcolor{blue}{Salten}{ }\emph{\textcolor{green}{Vielfarbige Distichen V}} gesandt.
                           (Hugo von Hofmannsthal: \emph{Brief-Chronik.
                              Regest-Ausgabe}. Hg. Martin E. Schmid. Band 1: 1874–1911.
                           Heidelberg: \emph{Winter}{ }2003, S. 21.)}}}\label{K_L02955-3h}}{}\ledrightnote{\textcolor{green}{Vielfarbige Distichen V.}}}, von denen er kein \textsc{Duplium} beſitzt. Dann, we{\geminationn} Sie’s {\pb}nicht
               etwa ſelber verliehen haben, die \label{K_L02955-4v}\edtext{\textsc{\textcolor{green}{Bilanz der Ehe}{}\ledrightnote{\textcolor{green}{Die Bilanz der Ehe. Novellistische Studien. 2 Bde.}}}}{\lemma{\textnormal{\emph{Bilanz der Ehe}}}\Cendnote{\textnormal{\textcolor{blue}{Gustav Schwarzkopf}: \emph{\textcolor{green}{Bilanz der Ehe. Novellistische Studien}}. 2 Bde. \textcolor{pink}{Dresden}/\textcolor{pink}{Leipzig}: \emph{\textcolor{brown}{Heinrich Minden}}{ }1885.}}}\label{K_L02955-4h}. –\pend
           
\pstart
           Er ſchickt mit größter Eile den \textcolor{green}{Tod des Tizian}{}\ledrightnote{\textcolor{green}{Der Tod des Tizian. Ein Bruchstück}}
               als Fragment an die neue \label{K_L02955-5v}\edtext{\textcolor{brown}{\textsc{\textcolor{blue}{Henze}{}\ledrightnote{\textcolor{blue}{Max Henze}}}’ſche Zeitung}{}\ledrightnote{{$\rightarrow$}\textcolor{brown}{Allgemeine Theater-Revue für Bühne und Welt. Illustrierte Halbmonatsschrift}}}{\lemma{\textnormal{\emph{Henze’ſche Zeitung}}}\Cendnote{\textnormal{Das \textcolor{green}{Dramenfragment} erschien schließlich in \textcolor{blue}{Stefan George}s \emph{\textcolor{green}{Blätter für
                     die Kunst}}: \textcolor{blue}{Hugo von Hofmannsthal}: \emph{\textcolor{green}{Der Tod des Tizian. Ein Bruchstück}}. In: \emph{\textcolor{green}{Blätter für die Kunst}}, Jg. 1, H. 1, Oktober 1892, S. 12–24.}}}\label{K_L02955-5h}{ }\textsc{\textcolor{pink}{Berlin}{}\ledrightnote{\textcolor{pink}{Berlin}}}, las ihn mir heute{ }Nachmittag vor. – Schön – ! Na, wir {\pb}reden bald drüber, hoffentlich beko{\geminationm}en Sie’s bald zu leſen; ſchade daſs Sie’s heut nicht gehört haben.\pend
           
\pstart
           – Ich ko{\geminationm}e, we{\geminationn} nicht früher, \label{K_L02955-66v}\edtext{\substVorne{}\textsuperscript{Fre}\substDazwischen{}\textsc{Do{\geminationn}}\substHinten{}\textsc{erstag}{ }Abend ins \textsc{\textcolor{pink}{Central}{}\ledrightnote{\textcolor{pink}{Café Central}}}{\lemma{\textnormal{\emph{FreDonnerstag … Feiertag}}}\Cendnote{\textnormal{Nicht im \emph{\textcolor{green}{Tagebuch}}. Zumindest ein
                   Indiz gibt diese Stelle, dass \textcolor{blue}{Schnitzler} seine Kaffeehausbesuche in der
                  Nacht nur dann ansetzte, wenn er am Folgetag keine Ordination hielt.}}}\label{K_L02955-66h}} (Freitg iſt nämlich \label{K_L02955-6v}\edtext{Feiertag}{\lemma{\textnormal{\emph{Feiertag}}}\Cendnote{\textnormal{Mariä
                  Verkündigung / Verkündigung des Herrn}}}\label{K_L02955-6h}.) \pend
           
\pstart
           Herzlichſt {\pb}der Ihre {\\[\baselineskip]}\spacefill\mbox{ArthSch}\pend
           \leftskip=0em{}\endnumbering\briefempfaengerindex{Salten, Felix@\textsc{Salten, Felix}!zzzSchnitzler, Arthur@\emph{von Arthur Schnitzler}!1892-03-211@{21. 3. 1892}|)be}\mylabel{h}  \normalsize

\doendnotes{C}
\bigskip
\vfill

\clearpage

\footnotesize

\lohead{\textsc{register}}

% Definiere theindex-Environment komplett neu ohne reledmac
\makeatletter
\renewenvironment{theindex}{%
  \section*{\indexname}%
  \setlength{\parindent}{0pt}%
  \setlength{\parskip}{0pt plus 0.3pt}%
  \let\item\@idxitem
}{%
  \clearpage
}
\makeatother

\IfFileExists{\jobname-pw.ind}{\input{\jobname-pw.ind}}{}

\end{document}

      