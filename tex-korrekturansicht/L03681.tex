%% latex-korrekturansicht-vorspann.tex
%% Vorspann für die Korrekturansicht.
%% Lädt die gemeinsame Datei latex-vorspann.tex mit gesetztem Schalter.

\newif\ifkorrekturansicht
\korrekturansichttrue

\input{../tex-inputs/latex-vorspann}


\section[Stefan Zweig an Arthur Schnitzler, 12. 10. 1917]{L03681 Stefan Zweig an Arthur Schnitzler, 12. 10. 1917}
\nopagebreak\mylabel{L03681v}
\rehead{ }\normalsize\beginnumbering\briefempfaengerindex{, @\textsc{, }!zzz, @\emph{von  }!1917-10-122@{12. 10. 1917}|(be}
\toendnotes[C]{\smallbreak\pagebreak[2]}\Standort{CUL, Schnitzler, B 118.}
\physDesc{Briefkarte, 1185 Zeichen
\newline{}Handschrift: lila Tinte, lateinische Kurrent
\newline{}Schnitzler: mit Bleistift »\textsc{Zweig}« }
\buchAbdrucke{\weitereDrucke{Stefan Zweig: \emph{Briefwechsel mit Hermann Bahr, Sigmund Freud, Rainer Maria
                        Rilke und Arthur Schnitzler}. Herausgegeben von Jeffrey B. Berlin, Hans-Ulrich Lindken und Donald A. Prater. Frankfurt am Main: \emph{S. Fischer} 1987, S. 409.} }\toendnotes[C]{\smallbreak}
\pstart
           \raggedleft{}{\pb}12. October 1917\pend
           
\pstart
           \textcolor{gray}{\textbf{SZ}}\hfill \textcolor{gray}{\textbf{\textcolor{pink}{VIII. KOCHGASSE 8}\oindex{Wien@\textbf{Wien}!VIII., Josefstadt@\textbf{VIII., Josefstadt}!Kochgasse 8@\textbf{Kochgasse 8}, \emph{Wohngebäude}|pw}{}\ledrightnote{\textcolor{pink}{Kochgasse 8}}.}}\pend
           \vspace{0.5em}
\pstart
           Lieber verehrter Herr Doktor, ich danke Ihnen sehr für Ihre guten
               Worte; dass dieses \textcolor{green}{Werk}\pwindex{Zweig, Stefan 28.\,11.\,1881 Wien – 23.\,2.\,1942 Petrópolis@\textsc{Zweig, Stefan} (28.\,11.\,1881 Wien – 23.\,2.\,1942 Petrópolis), \emph{Schriftsteller}!Jeremias. Ein dramatische Dichtung in neun Bildern@\strich\emph{Jeremias. Ein dramatische Dichtung in neun Bildern}|pwv}{}\ledrightnote{{$\rightarrow$}\emph{\textcolor{green}{Jeremias. Ein dramatische Dichtung in neun Bildern}}}, eigentlich aus
               Zorn und Qual geboren, mir nun Liebe gerade der Besten gewinnt, bezeugt mir die
               Notwendigkeit dieser Erbitterung, die ich lange selbst wie eine sinnlose Verstörung
               empfand. Vielleicht hat die Verwandlung die Leidenschaft gelöst und damit auch das
               Leiden erlöst: ich fühle mich jetzt freier, so sehr ich äusserlich noch gebunden
               bin.\pend
           
\pstart
           {\pb}Es wäre nur ein menschliches Bedürfis,
               Sie und Ihre verehrte Frau \textcolor{blue}{Gemahlin}\pwindex{Schnitzler, Olga 17.\,1.\,1882 Wien – 13.\,1.\,1970 Lugano@\textsc{Schnitzler, Olga} (17.\,1.\,1882 Wien – 13.\,1.\,1970 Lugano), \emph{Schauspielerin, Sängerin}|pwv}{}\ledrightnote{{$\rightarrow$}\emph{\textcolor{blue}{Olga Schnitzler}}} wieder einmal sehen zu dürfen. Aber ich lebe ganz im Ungewissen.
               Vor 6 Wochen hat das \textcolor{brown}{Auswärtige Amt}\orgindex{Außenministerium@Außenministerium|pw}{}\ledrightnote{\textcolor{brown}{Außenministerium}} für mich um einen
               \label{K_L03681-1v}\edtext{Urlaub nach der \textcolor{pink}{Schweiz}\oindex{Schweiz@\textbf{Schweiz}|pw}{}\ledrightnote{\textcolor{pink}{Schweiz}}}{\lemma{\textnormal{\emph{Urlaub nach der Schweiz}}}\Cendnote{\textnormal{\textcolor{blue}{Zweig}\pwindex{Zweig, Stefan 28.\,11.\,1881 Wien – 23.\,2.\,1942 Petrópolis@\textsc{Zweig, Stefan} (28.\,11.\,1881 Wien – 23.\,2.\,1942 Petrópolis), \emph{Schriftsteller}|pwk} bekam 
                  den gewünschten Urlaub und war ab 5. 11. 1917 von der Arbeit im \emph{\textcolor{brown}{Kriegsarchiv}\orgindex{Österreichisches Staatsarchiv@Österreichisches Staatsarchiv|pwk}} freigestellt.
                  Eine Woche später reiste er in die \textcolor{pink}{Schweiz}\oindex{Schweiz@\textbf{Schweiz}|pwk}, wo er – nunmehr als ständiger Mitarbeiter der
                  \emph{\textcolor{brown}{Neuen Freien Presse}\orgindex{Neue Freie Presse@Neue Freie Presse|pwk}} – bis März 1919 blieb.}}}\label{K_L03681-1} gebeten, wo ich einige
               Vorträge halten soll. Das \textcolor{brown}{Kriegsministerium}\orgindex{k. u. k. Kriegsministerium@k. u. k. Kriegsministerium|pw}{}\ledrightnote{\textcolor{brown}{k. u. k. Kriegsministerium}}, das jeden
               Filmschapsel und Operettengaukler willig entliess, hat in sechs Wochen nicht geruht,
               darauf Antwort zu geben, der Vortrag wartet auf mich und ich weiss nicht, ob ich darf
               oder nicht. Freilich: ich rühre nicht einen Finger, weil es mir zu kläglich scheint,
               nach drei Jahren Dienst um so einen Atemzug Freiheit noch bittlich zu werden: aber
               ich hänge jetzt in der Luft und weiss nicht von heute auf morgen. \pend
           \pstart Herzlich ergeben Ihr getreuer \spacefill\mbox{Stefan Zweig}\pend{}\selectlanguage{ngerman}\endnumbering\briefempfaengerindex{, @\textsc{, }!zzz, @\emph{von  }!1917-10-122@{12. 10. 1917}|)be}\mylabel{L03681h}  \normalsize

\doendnotes{C}
\bigskip
\vfill

\clearpage

\footnotesize

\lohead{\textsc{register}}

% Definiere theindex-Environment komplett neu ohne reledmac
\makeatletter
\renewenvironment{theindex}{%
  \section*{\indexname}%
  \setlength{\parindent}{0pt}%
  \setlength{\parskip}{0pt plus 0.3pt}%
  \let\item\@idxitem
}{%
  \clearpage
}
\makeatother

\IfFileExists{\jobname-pw.ind}{\input{\jobname-pw.ind}}{}

\end{document}

      