%% latex-korrekturansicht-vorspann.tex
%% Vorspann für die Korrekturansicht.
%% Lädt die gemeinsame Datei latex-vorspann.tex mit gesetztem Schalter.

\newif\ifkorrekturansicht
\korrekturansichttrue

\input{../tex-inputs/latex-vorspann}


\section[Stefan Zweig an Arthur Schnitzler, {[}26. 4. 1915{]}]{L03654 Stefan Zweig an Arthur Schnitzler, {[}26. 4. 1915{]}}
\nopagebreak\mylabel{L03654v}
\rehead{ }\normalsize\beginnumbering\briefempfaengerindex{, @\textsc{, }!zzz, @\emph{von  }!1915-04-261@{{[}26. 4. 1915{]}}|(be}
\toendnotes[C]{\smallbreak\pagebreak[2]}\Standort{CUL, Schnitzler, B 118.}
\physDesc{Brief, 1 Blatt, 3 Seiten, 1692 Zeichen
\newline{}Handschrift: lila Tinte, lateinische Kurrent
\newline{}Schnitzler: 1) mit Bleistift »\textsc{Zweig}« und datiert »Mai 915«  2) mit rotem Buntstift drei Unterstreichungen}
\buchAbdrucke{\weitereDrucke{Stefan Zweig: \emph{Briefwechsel mit Hermann Bahr, Sigmund Freud, Rainer Maria
                        Rilke und Arthur Schnitzler}. Herausgegeben von Jeffrey B. Berlin,  Hans-Ulrich Lindken und  Donald A. Prater. Frankfurt am Main: \emph{S. Fischer} 1987, S. 393–394.} }\toendnotes[C]{\smallbreak}
\pstart
           {\pb}\textcolor{gray}{\textbf{SZ}}\hfill \textcolor{gray}{\textbf{\textcolor{pink}{VIII. KOCHGASSE}\oindex{Wien@\textbf{Wien}!VIII., Josefstadt@\textbf{VIII., Josefstadt}!Kochgasse 8@\textbf{Kochgasse 8}, \emph{Wohngebäude}|pw}{}\ledrightnote{\textcolor{pink}{Kochgasse 8}}}}\pend
           
\pstart
           \raggedleft{}\textcolor{gray}{\textbf{\textcolor{pink}{WIEN}\oindex{Wien@\textbf{Wien}, \emph{Verwaltungsgebiet}|pw}{}\ledrightnote{\textcolor{pink}{Wien}},}}\pend
           
\pstart{}Sehr verehrter lieber Herr Doktor,\pend\vspace{0.5em}
\pstart
           ich war innerlich noch sehr bedrückt, Ihnen für den \label{K_L03654-1v}\edtext{schönen Abend}{\lemma{\textnormal{\emph{schönen Abend}}}\Cendnote{\textnormal{\textcolor{blue}{Schnitzler} hatte \textcolor{blue}{Zweig}\pwindex{Zweig, Stefan 28.\,11.\,1881 Wien – 23.\,2.\,1942 Petrópolis@\textsc{Zweig, Stefan} (28.\,11.\,1881 Wien – 23.\,2.\,1942 Petrópolis), \emph{Schriftsteller}|pwk} und \textcolor{blue}{Berta
                     Zuckerkandl}\pwindex{Zuckerkandl, Berta 13.\,4.\,1864 Wien – 16.\,10.\,1945 Paris@\textsc{Zuckerkandl, Berta} (13.\,4.\,1864 Wien – 16.\,10.\,1945 Paris), \emph{Journalistin, Übersetzerin}|pwk} am 11. 4. 1915 die \emph{\textcolor{green}{Komödie der
                     Worte}\pwindex{Schnitzler, Arthur 15. 5. 1862 Wien – 21. 10. 1931 ebd.@\textsc{Schnitzler, Arthur} (15. 5. 1862 Wien – 21. 10. 1931 ebd.), \emph{Schriftsteller, Mediziner}!Komödie der Worte. Drei Einakter@\strich\emph{Komödie der Worte. Drei Einakter}|pwk}} vorgelesen und bei der Gelegenheit wurde auch die Frage nach einem
                  passenden Titel für das Werk diskutiert.}}}\label{K_L03654-1} von damals nicht noch besonders
               gedankt zu haben: der Grund für dieses Unterlassen war, dass ich mich innerlich um
               den Titel für das \textcolor{green}{Werk}\pwindex{Schnitzler, Arthur 15. 5. 1862 Wien – 21. 10. 1931 ebd.@\textsc{Schnitzler, Arthur} (15. 5. 1862 Wien – 21. 10. 1931 ebd.), \emph{Schriftsteller, Mediziner}!Komödie der Worte. Drei Einakter@\strich\emph{Komödie der Worte. Drei Einakter}|pwv}{}\ledrightnote{{$\rightarrow$}\emph{\textcolor{green}{Komödie der Worte. Drei Einakter}}} mühte
               und ohne diese bescheidene Gegengabe Ihnen nicht schreiben wollte. Und nun muss ich
               Ihnen für neuerliche Güte danken: glauben Sie mir, bitte, dass ich gerade in dieser
               Zeit, wo sonst alle Menschen das Harte in sich herauskehren, Ihnen dafür besonders
               erkenntlich {\pb}bin.\pend
           
\pstart
           In der \label{K_L03654-2v}\edtext{Sache D\textsuperscript{r}{ }\textcolor{blue}{Rosenbaums}\pwindex{Rosenbaum, Richard 4.\,11.\,1867 Žikov – 25.\,6.\,1942 Konzentrationslager Theresienstadt@\textsc{Rosenbaum, Richard} (4.\,11.\,1867 Žikov – 25.\,6.\,1942 Konzentrationslager Theresienstadt), \emph{Dramaturg, Verleger}|pw}{}\ledrightnote{\textcolor{blue}{Richard Rosenbaum}}}{\lemma{\textnormal{\emph{Sache … Rosenbaums}}}\Cendnote{\textnormal{siehe Stefan Zweig an Arthur Schnitzler, [zwischen
               7. 4. 1915 und 9. 4. 1915?]. }}}\label{K_L03654-2} habe ich
                  \label{K_L03654-3v}\edtext{von \textcolor{blue}{Gerhardt Hauptmann}\pwindex{Hauptmann, Gerhart 15.\,11.\,1862 Szczawno-Zdrój – 6.\,6.\,1946 Jagniątków@\textsc{Hauptmann, Gerhart} (15.\,11.\,1862 Szczawno-Zdrój – 6.\,6.\,1946 Jagniątków), \emph{Schriftsteller}|pw}{}\ledrightnote{\textcolor{blue}{Gerhart Hauptmann}} noch keine Antwort}{\lemma{\textnormal{\emph{von … Antwort}}}\Cendnote{\textnormal{Sowohl \textcolor{blue}{Schnitzler} wie
                  auch \textcolor{blue}{Gerhart Hauptmann}\pwindex{Hauptmann, Gerhart 15.\,11.\,1862 Szczawno-Zdrój – 6.\,6.\,1946 Jagniątków@\textsc{Hauptmann, Gerhart} (15.\,11.\,1862 Szczawno-Zdrój – 6.\,6.\,1946 Jagniątków), \emph{Schriftsteller}|pwk} traten mit einer
                  Erklärung für \textcolor{blue}{Rosenbaum}\pwindex{Rosenbaum, Richard 4.\,11.\,1867 Žikov – 25.\,6.\,1942 Konzentrationslager Theresienstadt@\textsc{Rosenbaum, Richard} (4.\,11.\,1867 Žikov – 25.\,6.\,1942 Konzentrationslager Theresienstadt), \emph{Dramaturg, Verleger}|pwk} öffentlich für
                  diesen ein, siehe A. S.: \emph{»Das Zeitlose ist von kürzester Dauer«}, Der Rücktritt des Burgtheatersekretärs Dr. Rosenbaum, 16. 5. 1915. »\textcolor{blue}{Zweig}\pwindex{Zweig, Stefan 28.\,11.\,1881 Wien – 23.\,2.\,1942 Petrópolis@\textsc{Zweig, Stefan} (28.\,11.\,1881 Wien – 23.\,2.\,1942 Petrópolis), \emph{Schriftsteller}|pw} hatte \textcolor{blue}{Gerhart Hauptmann}\pwindex{Hauptmann, Gerhart 15.\,11.\,1862 Szczawno-Zdrój – 6.\,6.\,1946 Jagniątków@\textsc{Hauptmann, Gerhart} (15.\,11.\,1862 Szczawno-Zdrój – 6.\,6.\,1946 Jagniątków), \emph{Schriftsteller}|pw} in einem (unveröffentlichten) Brief
                     vom 13. 4. 1915 um ›ein Wort zum Abschied‹ \textcolor{blue}{Rosenbaums}\pwindex{Rosenbaum, Richard 4.\,11.\,1867 Žikov – 25.\,6.\,1942 Konzentrationslager Theresienstadt@\textsc{Rosenbaum, Richard} (4.\,11.\,1867 Žikov – 25.\,6.\,1942 Konzentrationslager Theresienstadt), \emph{Dramaturg, Verleger}|pw} vom \textcolor{brown}{Burgtheaters}\orgindex{Burgtheater@Burgtheater|pw} gebeten. \textcolor{blue}{Hauptmann}\pwindex{Hauptmann, Gerhart 15.\,11.\,1862 Szczawno-Zdrój – 6.\,6.\,1946 Jagniątków@\textsc{Hauptmann, Gerhart} (15.\,11.\,1862 Szczawno-Zdrój – 6.\,6.\,1946 Jagniątków), \emph{Schriftsteller}|pw}
                     hatte darauf am 20. 4. kurz geantwortet: ›Der Weggang Dr. \textcolor{blue}{Rosenbaums}\pwindex{Rosenbaum, Richard 4.\,11.\,1867 Žikov – 25.\,6.\,1942 Konzentrationslager Theresienstadt@\textsc{Rosenbaum, Richard} (4.\,11.\,1867 Žikov – 25.\,6.\,1942 Konzentrationslager Theresienstadt), \emph{Dramaturg, Verleger}|pw} vom \textcolor{brown}{Burgtheaters}\orgindex{Burgtheater@Burgtheater|pw} hat mich sehr schmerzlich berührt, weil
                     ich weiß, mit welcher Hingebung er dem Institute verbunden ist. Ich begrüsse
                     Sie herzlich, danke Ihnen wärmstens für Ihre lieben Zeilen und füge ein paar
                     Abschiedsworte {[}\ldots{]} für Dr.
                        \textcolor{blue}{Rosenbaum}\pwindex{Rosenbaum, Richard 4.\,11.\,1867 Žikov – 25.\,6.\,1942 Konzentrationslager Theresienstadt@\textsc{Rosenbaum, Richard} (4.\,11.\,1867 Žikov – 25.\,6.\,1942 Konzentrationslager Theresienstadt), \emph{Dramaturg, Verleger}|pw} hier bei.‹ Am
                        4. 5. bedankte \textcolor{blue}{Zweig}\pwindex{Zweig, Stefan 28.\,11.\,1881 Wien – 23.\,2.\,1942 Petrópolis@\textsc{Zweig, Stefan} (28.\,11.\,1881 Wien – 23.\,2.\,1942 Petrópolis), \emph{Schriftsteller}|pw}
                     sich für \textcolor{blue}{Hauptmanns}\pwindex{Hauptmann, Gerhart 15.\,11.\,1862 Szczawno-Zdrój – 6.\,6.\,1946 Jagniątków@\textsc{Hauptmann, Gerhart} (15.\,11.\,1862 Szczawno-Zdrój – 6.\,6.\,1946 Jagniątków), \emph{Schriftsteller}|pw} Brief und schrieb
                     (in einem ebenfalls unveröffentlichten Brief): ›Erst heute bekam ich Ihren
                     Brief vom 20. April, aber diesmals darf die Post nicht gescholten
                     sein: die ungeheuren Truppentransporte haben die Strecken für sich genommen und
                     für die verzögerte Freude einzelner Briefe haben wir heute die gemeinsame des
                     großen Sieges.‹« (\emph{Briefwechsel mit Hermann Bahr, Sigmund Freud, Rainer Maria
                        Rilke und Arthur Schnitzler}, S. 472.)}}}\label{K_L03654-3}: ist es die
               Post, die den Brief so lange hält oder irgend Etwas in ihm? Jedesfalls bin ich sehr
               erbittert, wie gut \textcolor{blue}{Thimig}\pwindex{Thimig, Hugo 16.\,6.\,1854 Dresden – 24.\,9.\,1944 Wien@\textsc{Thimig, Hugo} (16.\,6.\,1854 Dresden – 24.\,9.\,1944 Wien), \emph{Theaterleiter, Schauspieler}|pw}{}\ledrightnote{\textcolor{blue}{Hugo Thimig}} alles gelungen ist.
               In aller Stille hat man diesen guten \textcolor{blue}{Mann}\pwindex{Rosenbaum, Richard 4.\,11.\,1867 Žikov – 25.\,6.\,1942 Konzentrationslager Theresienstadt@\textsc{Rosenbaum, Richard} (4.\,11.\,1867 Žikov – 25.\,6.\,1942 Konzentrationslager Theresienstadt), \emph{Dramaturg, Verleger}|pwv}{}\ledrightnote{{$\rightarrow$}\emph{\textcolor{blue}{Richard Rosenbaum}}} begraben und in einem Jahr wird niemand mehr von ihm
               wissen. Ich hoffe noch immer, etwas tun zu können: es wäre ja sehr nötig und nicht
               nur im moralischen Sinne, denn D\textsuperscript{r}{ }\textcolor{blue}{R}\pwindex{Rosenbaum, Richard 4.\,11.\,1867 Žikov – 25.\,6.\,1942 Konzentrationslager Theresienstadt@\textsc{Rosenbaum, Richard} (4.\,11.\,1867 Žikov – 25.\,6.\,1942 Konzentrationslager Theresienstadt), \emph{Dramaturg, Verleger}|pw}{}\ledrightnote{\textcolor{blue}{Richard Rosenbaum}}, der jetzt ein Vierteljahrhundert in
               unablässiger Arbeit gelebt hat, braucht Wirksamkeit, um nicht bitter zu werden.
               Hoffentlich findet sich da ein Weg.\pend
           
\pstart
           Ich freue mich sehr, Sie {\pb}und Ihre
               verehrte Frau \textcolor{blue}{Gemahlin}\pwindex{Schnitzler, Olga 17.\,1.\,1882 Wien – 13.\,1.\,1970 Lugano@\textsc{Schnitzler, Olga} (17.\,1.\,1882 Wien – 13.\,1.\,1970 Lugano), \emph{Schauspielerin, Sängerin}|pwv}{}\ledrightnote{{$\rightarrow$}\emph{\textcolor{blue}{Olga Schnitzler}}} bald
               wieder sehen zu dürfen: heute abends habe ich mir den \label{K_L03654-4v}\edtext{\textcolor{violet}{Sonatenabend \textcolor{blue}{Walter}\pwindex{Walter, Bruno 15.\,9.\,1876 Berlin – 17.\,2.\,1962 Beverly Hills@\textsc{Walter, Bruno} (15.\,9.\,1876 Berlin – 17.\,2.\,1962 Beverly Hills), \emph{Theaterleiter, Komponist, Dirigent}|pw}{}\ledrightnote{\textcolor{blue}{Bruno Walter}}{ }\textcolor{blue}{Rosé}\pwindex{Rosé, Arnold 24.\,10.\,1863 Iași – 25.\,8.\,1946 London@\textsc{Rosé, Arnold} (24.\,10.\,1863 Iași – 25.\,8.\,1946 London), \emph{Violinist}|pw}{}\ledrightnote{\textcolor{blue}{Arnold Rosé}}}\eventindex{Wiener Konzerthaus@\textbf{Wiener Konzerthaus}!Sonatenabend von Bruno Walter und Arnold Rosé, 26.4.1915@Sonatenabend von Bruno Walter und Arnold Rosé, 26.4.1915|pw}{}\ledrightnote{\textcolor{violet}{Sonatenabend von Bruno Walter und Arnold Rosé, 26.4.1915}}}{\lemma{\textnormal{\emph{Sonatenabend Walter Rosé}}}\Cendnote{\textnormal{Das gemeinsame \emph{\textcolor{violet}{Konzert von \textcolor{blue}{Bruno
                        Walter}\pwindex{Walter, Bruno 15.\,9.\,1876 Berlin – 17.\,2.\,1962 Beverly Hills@\textsc{Walter, Bruno} (15.\,9.\,1876 Berlin – 17.\,2.\,1962 Beverly Hills), \emph{Theaterleiter, Komponist, Dirigent}|pwk} und \textcolor{blue}{Arnold Rosé}\pwindex{Rosé, Arnold 24.\,10.\,1863 Iași – 25.\,8.\,1946 London@\textsc{Rosé, Arnold} (24.\,10.\,1863 Iași – 25.\,8.\,1946 London), \emph{Violinist}|pwk}}\eventindex{Wiener Konzerthaus@\textbf{Wiener Konzerthaus}!Sonatenabend von Bruno Walter und Arnold Rosé, 26.4.1915@Sonatenabend von Bruno Walter und Arnold Rosé, 26.4.1915|pwk}} fand im \textcolor{pink}{Mittleren Konzerthaussaal}\oindex{Wien@\textbf{Wien}!III., Landstraße@\textbf{III., Landstraße}!Wiener Konzerthaus@\textbf{Wiener Konzerthaus}, \emph{Konzertsaal}|pwk}
                  statt.}}}\label{K_L03654-4}, \label{K_L03654-5v}\edtext{morgen das \textcolor{violet}{\textcolor{green}{Lied von der Erde}\pwindex{Mahler, Gustav 7.\,7.\,1860 Kaliště – 18.\,5.\,1911 Wien@\textsc{Mahler, Gustav} (7.\,7.\,1860 Kaliště – 18.\,5.\,1911 Wien), \emph{Theaterleiter, Komponist, Dirigent}!Lied von der Erde@\strich\emph{Das Lied von der Erde}|pw}{}\ledrightnote{\textcolor{green}{Das Lied von der Erde}}}\eventindex{Musikverein@\textbf{Musikverein}!Aufführung von Das Lied von der Erde, 27.4.1915@Aufführung von Das Lied von der Erde, 27.4.1915|pw}{}\ledrightnote{\textcolor{violet}{Aufführung von Das Lied von der Erde, 27.4.1915}}}{\lemma{\textnormal{\emph{morgen … Erde}}}\Cendnote{\textnormal{Von den hier aufgeführten
                  Veranstaltungen besuchten \textcolor{blue}{Olga}\pwindex{Schnitzler, Olga 17.\,1.\,1882 Wien – 13.\,1.\,1970 Lugano@\textsc{Schnitzler, Olga} (17.\,1.\,1882 Wien – 13.\,1.\,1970 Lugano), \emph{Schauspielerin, Sängerin}|pwk} und \textcolor{blue}{Arthur Schnitzler} nur die \textcolor{violet}{Aufführung von \emph{\textcolor{green}{Das Lied von
                           der Erde}\pwindex{Mahler, Gustav 7.\,7.\,1860 Kaliště – 18.\,5.\,1911 Wien@\textsc{Mahler, Gustav} (7.\,7.\,1860 Kaliště – 18.\,5.\,1911 Wien), \emph{Theaterleiter, Komponist, Dirigent}!Lied von der Erde@\strich\emph{Das Lied von der Erde}|pwk}}}\eventindex{Musikverein@\textbf{Musikverein}!Aufführung von Das Lied von der Erde, 27.4.1915@Aufführung von Das Lied von der Erde, 27.4.1915|pwkv} am 27. 4. 1915 im
                  \textcolor{pink}{Großen Musikvereinssaal}\oindex{Wien@\textbf{Wien}!I., Innere Stadt@\textbf{I., Innere Stadt}!Musikverein@\textbf{Musikverein}, \emph{Konzertsaal}|pwk}.}}}\label{K_L03654-5}, \label{K_L03654-6v}\edtext{Mittwoch{ }\textcolor{violet}{\textcolor{green}{Elektra}\pwindex{Hofmannsthal, Hugo von 1.\,2.\,1874 Wien – 15.\,7.\,1929 Rodaun@\textsc{Hofmannsthal, Hugo von} (1.\,2.\,1874 Wien – 15.\,7.\,1929 Rodaun), \emph{Schriftsteller}!Elektra [op. 58]@\strich\emph{Elektra [op. 58]}|pw}{}\ledrightnote{\textcolor{green}{Elektra [op. 58]}}}\eventindex{Oper@\textbf{Oper}!Aufführung von Elektra, 28.4.1915@Aufführung von Elektra, 28.4.1915|pwv}{}\ledrightnote{{$\rightarrow$}\emph{\textcolor{violet}{Aufführung von Elektra, 28.4.1915}}}}{\lemma{\textnormal{\emph{Mittwoch Elektra}}}\Cendnote{\textnormal{Am 28.\,4.\,1915 wurde \textcolor{violet}{\emph{\textcolor{green}{Elektra}\pwindex{Hofmannsthal, Hugo von 1.\,2.\,1874 Wien – 15.\,7.\,1929 Rodaun@\textsc{Hofmannsthal, Hugo von} (1.\,2.\,1874 Wien – 15.\,7.\,1929 Rodaun), \emph{Schriftsteller}!Elektra [op. 58]@\strich\emph{Elektra [op. 58]}|pwk}} in der \emph{\textcolor{brown}{Wiener Oper}\orgindex{K.K. Hof-Oper@K.K. Hof-Oper|pwk}}}\eventindex{Oper@\textbf{Oper}!Aufführung von Elektra, 28.4.1915@Aufführung von Elektra, 28.4.1915|pwkv} gespielt.}}}\label{K_L03654-6} zugedacht, ich lebe jetzt wirklich von Musik, denn sonst
               wäre es nicht zu ertragen.\pend
           
\pstart
           In dankbarer Verehrung getreu Ihr{\\[\baselineskip]}\spacefill\mbox{Stefan Zweig}\pend
           \leftskip=0em{}
\pstart
           \noindent{}Viele Grüsse Ihrer Frau \textcolor{blue}{Gemahlin}\pwindex{Schnitzler, Olga 17.\,1.\,1882 Wien – 13.\,1.\,1970 Lugano@\textsc{Schnitzler, Olga} (17.\,1.\,1882 Wien – 13.\,1.\,1970 Lugano), \emph{Schauspielerin, Sängerin}|pwv}{}\ledrightnote{{$\rightarrow$}\emph{\textcolor{blue}{Olga Schnitzler}}}! Und noch die \label{K_L03654-7v}\edtext{Erinnerung}{\lemma{\textnormal{\emph{Erinnerung}}}\Cendnote{\textnormal{Siehe Stefan Zweig an Arthur Schnitzler, [7.?] 12. 1914.}}}\label{K_L03654-7}: wenn Sie
                  einmal Zeit und Lust haben gedenken Sie jenes Bildhauers \textcolor{blue}{Gustinus Ambrosi}\pwindex{Ambrosi, Gustinus 24.\,2.\,1893 Eisenstadt – 30.\,6.\,1975 Wien@\textsc{Ambrosi, Gustinus} (24.\,2.\,1893 Eisenstadt – 30.\,6.\,1975 Wien), \emph{Schriftsteller, Bildhauer, Philosoph}|pw}{}\ledrightnote{\textcolor{blue}{Gustinus Ambrosi}}, der so gerne Ihre Büste machte. Ich
                  halte diesen taubstummen Menschen für einen wahrhaft genialen Künstler, \label{T_L03654-1v}\edtext{er ist auch menschlich, ein
                  unvergleichliches Erlebnis.}{\lemma{\textnormal{\emph{er … Erlebnis.}}}\Cendnote{\textnormal{seitlich
                     entlang des linken Blattrandes}}}\label{T_L03654-1}\pend
           \selectlanguage{ngerman}\endnumbering\briefempfaengerindex{, @\textsc{, }!zzz, @\emph{von  }!1915-04-261@{{[}26. 4. 1915{]}}|)be}\mylabel{L03654h}
\begin{anhang}
\end{anhang}\normalsize

\doendnotes{C}
\bigskip
\vfill

\clearpage

\footnotesize

\lohead{\textsc{register}}

% Definiere theindex-Environment komplett neu ohne reledmac
\makeatletter
\renewenvironment{theindex}{%
  \section*{\indexname}%
  \setlength{\parindent}{0pt}%
  \setlength{\parskip}{0pt plus 0.3pt}%
  \let\item\@idxitem
}{%
  \clearpage
}
\makeatother

\IfFileExists{\jobname-pw.ind}{\input{\jobname-pw.ind}}{}

\end{document}

      