%% latex-korrekturansicht-vorspann.tex
%% Vorspann für die Korrekturansicht.
%% Lädt die gemeinsame Datei latex-vorspann.tex mit gesetztem Schalter.

\newif\ifkorrekturansicht
\korrekturansichttrue

\input{../tex-inputs/latex-vorspann}


\renewcommand{\erwaehntePersonen}{Personen: Gustinus Ambrosi, Gerhart Hauptmann, Richard Rosenbaum, Arnold Rosé, Olga Schnitzler, Hugo Thimig, Bruno Walter, Berta Zuckerkandl, Stefan Zweig}
\renewcommand{\erwaehnteInstitutionen}{Institutionen: Burgtheater, K.K. Hof-Oper}
\renewcommand{\erwaehnteOrte}{Orte: Kochgasse 8, Musikverein, Wien, Wiener Konzerthaus}
\renewcommand{\erwaehnteWerke}{Werke: Das Lied von der Erde, Elektra [op. 58], Komödie der Worte. Drei Einakter}
\section[Stefan Zweig an Arthur Schnitzler, {[}26. 4. 1915{]}]{Stefan Zweig an Arthur Schnitzler, {[}26. 4. 1915{]}}
\nopagebreak\mylabel{v}
\rehead{ }\normalsize\beginnumbering\briefempfaengerindex{Schnitzler, Arthur@\textsc{Schnitzler, Arthur}!zzzZweig, Stefan@\emph{von Stefan Zweig}!1915-04-261@{{[}26. 4. 1915{]}}|(be}
\toendnotes[C]{\smallbreak\pagebreak[2]}\Standort{CUL, Schnitzler, B 118.}
\physDesc{Brief, 1 Blatt, 3 Seiten, 1693 Zeichen
\newline{}Handschrift: lila Tinte, lateinische Kurrent
\newline{}Schnitzler: 1) mit Bleistift »\textsc{Zweig}«  2) mit rotem Buntstift drei Unterstreichungen}\toendnotes[C]{\smallbreak}
\pstart
           {\pb}\textcolor{gray}{\textbf{SZ}}\hfill \textcolor{gray}{\textbf{\textcolor{pink}{VIII. KOCHGASSE}{}\ledrightnote{\textcolor{pink}{Kochgasse 8}}}}\pend
           
\pstart
           \raggedleft{}\textcolor{gray}{\textbf{\textcolor{pink}{WIEN}{}\ledrightnote{\textcolor{pink}{Wien}},}}\pend
           
\pstart{}Sehr verehrter lieber Herr Doktor,\pend
\pstart
           ich war innerlich noch sehr bedrückt, Ihnen für den \label{K_L03654-1v}\edtext{schönen Abend}{\lemma{\textnormal{\emph{schönen Abend}}}\Cendnote{\textnormal{\textcolor{blue}{Schnitzler} hatte \textcolor{blue}{Zweig} und \textcolor{blue}{Berta
                     Zuckerkandl} am 11. 4. 1915 die \emph{\textcolor{green}{Komödie der
                     Worte}} vorgelesen.}}}\label{K_L03654-1h} von damals nicht noch besonders gedankt zu haben:
               der Grund für dieses Unterlassen war, dass ich mich innerlich um den Titel für das
                  \textcolor{green}{Werk}{}\ledrightnote{{$\rightarrow$}\textcolor{green}{Komödie der Worte. Drei Einakter}} mühte und ohne diese
               bescheidene Gegengabe Ihnen nicht schreiben wollte. Und nun muss ich Ihnen für
               neuerliche Güte danken: glauben Sie mir, bitte, dass ich gerade in dieser Zeit, wo
               sonst alle Menschen das Harte in sich herauskehren, Ihnen dafür besonders erkenntlich
                  {\pb}bin.\pend
           
\pstart
           In der \label{K_L03654-2v}\edtext{Sache D\textsuperscript{r}{ }\textcolor{blue}{Rosenbaums}{}\ledrightnote{\textcolor{blue}{Richard Rosenbaum}}}{\lemma{\textnormal{\emph{Sache … Rosenbaums}}}\Cendnote{\textnormal{{XXXX ref}. }}}\label{K_L03654-2h} habe ich
                  \label{K_L03654-3v}\edtext{von \textcolor{blue}{Gerhardt Hauptmann}{}\ledrightnote{\textcolor{blue}{Gerhart Hauptmann}} noch keine Antwort}{\lemma{\textnormal{\emph{von … Antwort}}}\Cendnote{\textnormal{Sowohl \textcolor{blue}{Schnitzler} wie
                  auch \textcolor{blue}{Gerhart Hauptmann} traten mit einer
                  Erklärung für \textcolor{blue}{Rosenbaum} öffentlich für
                  diesen ein, siehe A. S.: \emph{»Das Zeitlose ist von kürzester Dauer«}, Der Rücktritt des Burgtheatersekretärs Dr. Rosenbaum, 16. 5. 1915. »\textcolor{blue}{Zweig} hatte \textcolor{blue}{Gerhart Hauptmann} in einem (unveröffentlichten) Brief
                     vom 13. 4. 1915 um ›ein Wort zum Abschied‹ \textcolor{blue}{Rosenbaums} vom \textcolor{brown}{Burgtheaters} gebeten. \textcolor{blue}{Hauptmann}
                     hatte darauf am 20. 4. kurz geantwortet: ›Der Weggang Dr. \textcolor{blue}{Rosenbaums} vom \textcolor{brown}{Burgtheaters} hat mich sehr schmerzlich berührt, weil
                     ich weiß, mit welcher Hingebung er dem Institute verbunden ist. Ich begrüsse
                     Sie herzlich, danke Ihnen wärmstens für Ihre lieben Zeilen und füge ein paar
                     Abschiedsworte {[}\ldots{]} für Dr.
                        \textcolor{blue}{Rosenbaum} hier bei.‹ Am
                        4. 5. bedankte \textcolor{blue}{Zweig}
                     sich für \textcolor{blue}{Hauptmanns} Brief und schrieb
                     (in einem ebenfalls unveröffentlichten Brief): ›Erst heute bekam ich Ihren
                     Brief vom 20. April, aber diesmals darf die Post nicht gescholten
                     sein: die ungeheuren Truppentransporte haben die Strecken für sich genommen und
                     für die verzögerte Freude einzelner Briefe haben wir heute die gemeinsame des
                     großen Sieges.‹« (\emph{Briefwechsel mit Hermann Bahr, Sigmund Freud, Rainer Maria
                        Rilke und Arthur Schnitzler}, S. 472.)}}}\label{K_L03654-3h}: ist es die
               Post, die den Brief so lange hält oder irgend Etwas in ihm? Jedesfalls bin ich sehr
               erbittert, wie gut \textcolor{blue}{Thimig}{}\ledrightnote{\textcolor{blue}{Hugo Thimig}} alles gelungen ist.
               In aller Stille hat man diesen guten \textcolor{blue}{Mann}{}\ledrightnote{{$\rightarrow$}\textcolor{blue}{Richard Rosenbaum}} begraben und in einem Jahr wird niemand mehr von ihm
               wissen. Ich hoffe noch immer, etwas tun zu können: es wäre ja sehr nötig und nicht
               nur im moralischen Sinne, denn D\textsuperscript{r}{ }\textcolor{blue}{R}{}\ledrightnote{\textcolor{blue}{Richard Rosenbaum}}, der jetzt ein Vierteljahrhundert in
               unablässiger Arbeit gelebt hat, braucht Wirksamkeit, um nicht bitter zu werden.
               Hoffentlich findet sich da ein Weg.\pend
           
\pstart
           Ich freue mich sehr, Sie {\pb}und Ihre
               verehrte Frau \textcolor{blue}{Gemahlin}{}\ledrightnote{{$\rightarrow$}\textcolor{blue}{Olga Schnitzler}} bald
               wieder sehen zu dürfen: heute abends habe ich mir den \label{K_L03654-4v}\edtext{Sonatenabend \textcolor{blue}{Walter}{}\ledrightnote{\textcolor{blue}{Bruno Walter}}{ }\textcolor{blue}{Rosé}{}\ledrightnote{\textcolor{blue}{Arnold Rosé}}}{\lemma{\textnormal{\emph{Sonatenabend Walter Rosé}}}\Cendnote{\textnormal{Im \textcolor{pink}{Mittleren Konzerthaussaal}.}}}\label{K_L03654-4h}, \label{K_L03654-5v}\edtext{morgen das \textcolor{green}{Lied von der Erde}{}\ledrightnote{\textcolor{green}{Das Lied von der Erde}}}{\lemma{\textnormal{\emph{morgen … Erde}}}\Cendnote{\textnormal{Von den hier aufgeführten
                  Veranstaltungen besuchten \textcolor{blue}{Olga} und \textcolor{blue}{Arthur Schnitzler} nur die Aufführung von \emph{\textcolor{green}{Das Lied von der Erde}} am 27. 4. 1915 im \textcolor{pink}{Großen Musikvereinssaal}.}}}\label{K_L03654-5h}, \label{K_L03654-6v}\edtext{Mittwoch{ }\textcolor{green}{Elektra}{}\ledrightnote{\textcolor{green}{Elektra [op. 58]}}}{\lemma{\textnormal{\emph{Mittwoch Elektra}}}\Cendnote{\textnormal{Am 28. 4. 1915 wurde \emph{\textcolor{green}{Elektra}} in der \emph{\textcolor{brown}{Wiener Oper}} gespielt.}}}\label{K_L03654-6h} zugedacht, ich lebe jetzt wirklich von
               Musik, denn sonst wäre es nicht zu ertragen.\pend
           
\pstart
           In dankbarer Verehrung getreu Ihr{\\[\baselineskip]}\spacefill\mbox{Stefan Zweig}\pend
           \leftskip=0em{}
\pstart
           \noindent{}Viele Grüsse Ihrer Frau \textcolor{blue}{Gemahlin}{}\ledrightnote{{$\rightarrow$}\textcolor{blue}{Olga Schnitzler}}!\pend
           
\pstart
           \noindent{}Und noch die Erinnerung: wenn Sie einmal Zeit und Lust haben gedenken Sie jenes
                  Bildhauers \textcolor{blue}{Gustinus Ambrosi}{}\ledrightnote{\textcolor{blue}{Gustinus Ambrosi}}, der so gerne
                  Ihre Büste machte. Ich halte diesen taubstummen Menschen für einen wahrhaft
                  genialen Künstler, er ist auch menschlich, ein unvergleichliches Erlebnis.\pend
           \endnumbering\briefempfaengerindex{Schnitzler, Arthur@\textsc{Schnitzler, Arthur}!zzzZweig, Stefan@\emph{von Stefan Zweig}!1915-04-261@{{[}26. 4. 1915{]}}|)be}\mylabel{h}
\begin{anhang}
\end{anhang}\normalsize

\doendnotes{C}
\bigskip
\vfill

\clearpage

\footnotesize

\lohead{\textsc{register}}

% Definiere theindex-Environment komplett neu ohne reledmac
\makeatletter
\renewenvironment{theindex}{%
  \section*{\indexname}%
  \setlength{\parindent}{0pt}%
  \setlength{\parskip}{0pt plus 0.3pt}%
  \let\item\@idxitem
}{%
  \clearpage
}
\makeatother

\IfFileExists{\jobname-pw.ind}{\input{\jobname-pw.ind}}{}

\end{document}

      