%% latex-korrekturansicht-vorspann.tex
%% Vorspann für die Korrekturansicht.
%% Lädt die gemeinsame Datei latex-vorspann.tex mit gesetztem Schalter.

\newif\ifkorrekturansicht
\korrekturansichttrue

\input{../tex-inputs/latex-vorspann}


               \section[Arthur Schnitzler an Richard Beer-Hofmann, 15. 6. 1911]{ Arthur Schnitzler an Richard Beer-Hofmann, 15. 6. 1911}\nopagebreak\mylabel{v}\rehead{ }\normalsize\beginnumbering\briefempfaengerindex{Beer-Hofmann, Richard@\textsc{Beer-Hofmann, Richard}!zzzSchnitzler, Arthur@\emph{von Arthur Schnitzler}!1911-06-151@{15. 6. 1911}|(be} \toendnotes[C]{\smallbreak\pagebreak[2]} \Standort{YCGL, MSS 31.}
\physDesc{Brief, 1 Blatt, 2 Seiten
\newline{}Handschrift: Bleistift, deutsche Kurrent}\buchAbdrucke{\weitereDrucke{Arthur Schnitzler, Richard Beer-Hofmann: \emph{Briefwechsel 1891–1931}. Hg. Konstanze Fliedl. Wien, Zürich: \emph{Europaverlag} 1992, S. 214.} }\toendnotes[C]{\smallbreak}\pstart
           \noindent{}{\pb}\textcolor{gray}{\textbf{Dr. Arthur Schnitzler}}\hfill 15/6 911\pend
           \pstart
           \textcolor{gray}{\textbf{\textcolor{pink}{Wien XVIII. Sternwartestrasse 71}{}\ledrightnote{\textcolor{pink}{Sternwartestraße}}}}\pend
           \pstart{}lieber Richard,\pend\pstart
           wollen Sie heute nach dem Nachtmahl, ſo um 9 etwa mit Ihrer \textcolor{blue}{Frau}{}\ledrightnote{→\textcolor{blue}{Paula Beer-Hofmann}} herüberko{\geminationm}en ſo würde es uns freuen. Sie werden, bereits
               geſättigt, \textcolor{blue}{\textsc{Rosenbaum}}{}\ledrightnote{\textcolor{blue}{Richard Rosenbaum}} (nicht \textcolor{blue}{den}{}\ledrightnote{→\textcolor{blue}{Karl Rostler}} vom Berg
                  \label{TLL02022_Beer-Hofmann-1v}\edtext{(\textcolor{pink}{Semmering}{}\ledrightnote{\textcolor{pink}{Semmering}})}{\lemma{\textnormal{\emph{(Semmering)}}}\Cendnote{\textnormal{\textcolor{blue}{Schnitzler} verwendet eckige Klammern.}}}\label{TLL02022_Beer-Hofmann-1h},
               ſondern \label{KLL02022_Beer-Hofmann-1v}\edtext{den vom \textcolor{blue}{Berger}{}\ledrightnote{\textcolor{blue}{Alfred von Berger}}}{\lemma{\textnormal{\emph{den vom Berger}}}\Cendnote{\textnormal{\textcolor{blue}{Richard Rosenbaum} war beim \emph{\textcolor{brown}{Burgtheater}} angestellt, dessen Direktor \textcolor{blue}{Alfred von Berger} war.}}}\label{KLL02022_Beer-Hofmann-1h}) ſamt \textcolor{blue}{\textsc{Towska}}{}\ledrightnote{\textcolor{blue}{Kory Elisabeth Rosenbaum}} vorfinden; der erſtere ſehr nett, die zweitere {\pb}mir noch wenig bekannt.\pend
           \pstart
           Und wann reiſen Sie? Wir \label{KLL02022_Beer-Hofmann-2v}\edtext{gegen 26.}{\lemma{\textnormal{\emph{gegen 26.}}}\Cendnote{\textnormal{Zu der Reise kam es
                  nicht.}}}\label{KLL02022_Beer-Hofmann-2h} – \textcolor{pink}{\textsc{Seis}}{}\ledrightnote{\textcolor{pink}{Seis am Schlern}}.\pend
           \pstart
           Herzlichſt{\\[\baselineskip]}Ihr{\\[\baselineskip]}\spacefill\mbox{A.}\pend
           \leftskip=0em{}\endnumbering\briefempfaengerindex{Beer-Hofmann, Richard@\textsc{Beer-Hofmann, Richard}!zzzSchnitzler, Arthur@\emph{von Arthur Schnitzler}!1911-06-151@{15. 6. 1911}|)be}\mylabel{h}  \normalsize

\doendnotes{C}
\bigskip
\vfill

\clearpage

\footnotesize

\lohead{\textsc{register}}

% Definiere theindex-Environment komplett neu ohne reledmac
\makeatletter
\renewenvironment{theindex}{%
  \section*{\indexname}%
  \setlength{\parindent}{0pt}%
  \setlength{\parskip}{0pt plus 0.3pt}%
  \let\item\@idxitem
}{%
  \clearpage
}
\makeatother

\IfFileExists{\jobname-pw.ind}{\input{\jobname-pw.ind}}{}

\end{document}

      