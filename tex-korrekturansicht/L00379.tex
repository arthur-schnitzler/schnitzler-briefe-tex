%% latex-korrekturansicht-vorspann.tex
%% Vorspann für die Korrekturansicht.
%% Lädt die gemeinsame Datei latex-vorspann.tex mit gesetztem Schalter.

\newif\ifkorrekturansicht
\korrekturansichttrue

\input{../tex-inputs/latex-vorspann}


               \section[Friedrich M. Fels an Arthur Schnitzler, 7. 10. 1894]{ Friedrich M. Fels an Arthur Schnitzler, 7. 10. 1894}\nopagebreak\mylabel{v}\rehead{ }\normalsize\beginnumbering\briefempfaengerindex{Schnitzler, Arthur@\textsc{Schnitzler, Arthur}!zzzFels, Friedrich Michael@\emph{von Friedrich Michael Fels}!1894-10-072@{7. 10. 1894}|(be} \toendnotes[C]{\smallbreak\pagebreak[2]} \Standort{DLA, A:Schnitzler, HS.NZ85.1.2956.}
\physDesc{Kartenbrief
\newline{}Handschrift: schwarze Tinte, lateinische Kurrent\newline{}Versand: 1) Stempel: »\nobreak{}Wien 18{[}/1{]}, 7 10 {[}1894{]}\nobreak{}«.  2) Stempel: »\nobreak{}\oindex{IX., Alsergrund@\textbf{IX., Alsergrund}, \emph{Bezirk (A.BZK)}|pwk}Wien 9/3, 8. 10. 1894, 8.V, Bestellt\nobreak{}«. 
\newline{}Schnitzler: mit Bleistift falsch datiert: »1/10 94« und nummeriert: »15« }\toendnotes[C]{\smallbreak}\pstart{}{\pb}Herrn Dr. Arthur
                        Schnitzler\pend{}\pstart{}Schriftsteller\pend{}\pstart{}\textcolor{pink}{Wien}{}\ledrightnote{\textcolor{pink}{Wien}}\pend{}\pstart{}\textcolor{pink}{IX, Frankgaße 1}{}\ledrightnote{\textcolor{pink}{Frankgasse}}\pend{}{\bigskip}\pstart
           \noindent{}\raggedleft{}{\pb}\textcolor{pink}{Wien XVIII, Gürtelstraße 90}{}\ledrightnote{\textcolor{pink}{Währinger Gürtel}} parterre Th. 9 \pend
           \pstart{}Lieber Dr. Schnitzler!\pend\pstart
           Entschuldigen Sie, we{\geminationn} ich Sie schon wieder mit
                    einer Bitte belästige. Bei der »\uline{\textcolor{brown}{Wiener Allgemeinen Zeitung}{}\ledrightnote{\textcolor{brown}{Wiener Allgemeine Zeitung}}}« soll eine große Veränderung bevorstehen, wobei \damage{viel}leicht auch für mich etwas abfallen kö{\geminationn}te. Doch hat mein Gewährsma{\geminationn} versprechen müßen, niemanden etwas von der Sache
                    zu verraten; er behauptet aber, \uline{Sie} wüßten ganz
                        besti{\geminationm}t davon, da der neue \textcolor{blue}{Herausgeber}{}\ledrightnote{→\textcolor{blue}{Julius von Gans-Ludassy}} ein \label{K_L00379_1v}\edtext{Freund}{\lemma{\textnormal{\emph{Freund}}}\Cendnote{\textnormal{\textcolor{blue}{Julius von Gans-Ludassy} war mit einer
                        Kusine von \textcolor{blue}{Schnitzler} verheiratet.}}}\label{K_L00379_1h}
                    von Ihnen sei etc. We{\geminationn} dies der Fall ist, sind Sie
                    wohl so freundlich, mir anzugeben, an wen ich mich zu wenden habe, und ein gutes
                    Wort für mich einzulegen.\pend
           \pstart
           Mit bestem Dank und Gruß{\\[\baselineskip]}Ihr{\\[\baselineskip]}\spacefill\mbox{Fels}\pend
           \leftskip=0em{}\endnumbering\briefempfaengerindex{Schnitzler, Arthur@\textsc{Schnitzler, Arthur}!zzzFels, Friedrich Michael@\emph{von Friedrich Michael Fels}!1894-10-072@{7. 10. 1894}|)be}\mylabel{h}  \normalsize

\doendnotes{C}
\bigskip
\vfill

\clearpage

\footnotesize

\lohead{\textsc{register}}

% Definiere theindex-Environment komplett neu ohne reledmac
\makeatletter
\renewenvironment{theindex}{%
  \section*{\indexname}%
  \setlength{\parindent}{0pt}%
  \setlength{\parskip}{0pt plus 0.3pt}%
  \let\item\@idxitem
}{%
  \clearpage
}
\makeatother

\IfFileExists{\jobname-pw.ind}{\input{\jobname-pw.ind}}{}

\end{document}

      