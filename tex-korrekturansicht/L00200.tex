%% latex-korrekturansicht-vorspann.tex
%% Vorspann für die Korrekturansicht.
%% Lädt die gemeinsame Datei latex-vorspann.tex mit gesetztem Schalter.

\newif\ifkorrekturansicht
\korrekturansichttrue

\input{../tex-inputs/latex-vorspann}


               \section[Hugo von Hofmannsthal an Arthur Schnitzler, {[}21. 4. 1893{]}]{ Hugo von Hofmannsthal an Arthur Schnitzler, {[}21. 4. 1893{]}}\nopagebreak\mylabel{v}\rehead{ }\normalsize\beginnumbering\briefempfaengerindex{Schnitzler, Arthur@\textsc{Schnitzler, Arthur}!zzzHofmannsthal, Hugo von@\emph{von Hugo von Hofmannsthal}!1893-04-212@{{[}21. 4. 1893{]}}|(be} \toendnotes[C]{\smallbreak\pagebreak[2]} \Standort{CUL, Schnitzler, B 43.}
\physDesc{Briefkarte mit aufgeprägtem Wappen
\newline{}Handschrift: schwarze Tinte, deutsche Kurrent
\newline{}Schnitzler: mit Bleistift das Datum ergänzt: »21/4 93« und nummeriert: »46« }\buchAbdrucke{\weitereDrucke{1) Hugo von Hofmannsthal, Arthur Schnitzler: \emph{Briefwechsel}. Hg. Therese Nickl und Heinrich Schnitzler. Frankfurt am Main: \emph{S. Fischer} 1964, S. 38.} \weitereDrucke{2) Hermann Bahr, Arthur Schnitzler: \emph{Briefwechsel, Aufzeichnungen, Dokumente
                                (1891–1931)}. Hg. Kurt Ifkovits und Martin Anton Müller. Göttingen: \emph{Wallstein} 2018, S. 35.} }\toendnotes[C]{\smallbreak}\pstart
           \raggedleft{}{\pb}Freitag,
                        abend.\pend
           \pstart{}Lieber Arthur!\pend\pstart
           Ich finde das Benehmen des \textcolor{blue}{Fels}{}\ledrightnote{\textcolor{blue}{Friedrich Michael Fels}} nicht recht
                    verſtändlich. Ich habe die ganze Geſchichte sogleich an \textcolor{blue}{\textsc{J. J. David}}{}\ledrightnote{\textcolor{blue}{Jakob Julius David}} geſchrieben, und von ſeiner größeren Routine in Journalſachen einen Rath
                    erbeten. Er antwortet mir: er kann nichts thuen, iſt übrigens durch das
                    »frevelhafte Stillſchweigen des \textcolor{blue}{Fels}{}\ledrightnote{\textcolor{blue}{Friedrich Michael Fels}}
                    vollkommen disguſtiert«. Heute Nacht ſpreche ich \textcolor{blue}{Bahr}{}\ledrightnote{\textcolor{blue}{Hermann Bahr}} und ſchreibe Ihnen pneumatiſch das Reſultat.\pend
           \pstart
           \numberlinefalse{}\centering{}–\numberlinetrue{}\pend
           \pstart
           \noindent{}Ich werde mit meinem \textcolor{green}{Einacter}{}\ledrightnote{→\textcolor{green}{Der Thor und der Tod}}{ }Sonntag fertig und möchte daß wir den
                        nachmittag 4–9 miteinander verbringen, Land oder Stadt, damit
                    ich ihn vorleſen kann, natürlich nur unter uns \textcolor{blue}{5}{}\ledrightnote{→\textcolor{blue}{Richard Beer-Hofmann}{\newline}→\textcolor{blue}{Felix Salten}{\newline}→\textcolor{blue}{Gustav Schwarzkopf}{\newline}→\textcolor{blue}{Hugo von Hofmannsthal}} (die
                    Hex mitgerechnet). Bei dieſer Gelegenheit beſprechen wir
                    wohl am beſten das unmittelbar {\pb}nötige in der ekelhaften
                    obigen Affaire.\pend
           \pstart Ihr \spacefill\mbox{Hugo}\pend{}\endnumbering\briefempfaengerindex{Schnitzler, Arthur@\textsc{Schnitzler, Arthur}!zzzHofmannsthal, Hugo von@\emph{von Hugo von Hofmannsthal}!1893-04-212@{{[}21. 4. 1893{]}}|)be}\mylabel{h}  \normalsize

\doendnotes{C}
\bigskip
\vfill

\clearpage

\footnotesize

\lohead{\textsc{register}}

% Definiere theindex-Environment komplett neu ohne reledmac
\makeatletter
\renewenvironment{theindex}{%
  \section*{\indexname}%
  \setlength{\parindent}{0pt}%
  \setlength{\parskip}{0pt plus 0.3pt}%
  \let\item\@idxitem
}{%
  \clearpage
}
\makeatother

\IfFileExists{\jobname-pw.ind}{\input{\jobname-pw.ind}}{}

\end{document}

      