%% latex-korrekturansicht-vorspann.tex
%% Vorspann für die Korrekturansicht.
%% Lädt die gemeinsame Datei latex-vorspann.tex mit gesetztem Schalter.

\newif\ifkorrekturansicht
\korrekturansichttrue

\input{../tex-inputs/latex-vorspann}


\renewcommand{\erwaehntePersonen}{Personen: Franz von Aichelburg-Labia, Richard Beer-Hofmann, Johannes Brahms, Bertha Flegmann, Marie Glümer, Marie Griebl, Emil Höfer, Josef Jarno, Felix Salten, Grethe Wreden}
\renewcommand{\erwaehnteInstitutionen}{Institutionen: Saisontheater Ischl, Volkstheater}
\renewcommand{\erwaehnteOrte}{Orte: Bad Ischl, Berggasse, Gmunden, IX., Alsergrund, Wien}
\renewcommand{\erwaehnteWerke}{Werke: Abschiedssouper, Artifex, Die Frage an das Schicksal}
\section[Arthur Schnitzler an Felix Salten, 9. 7. 1893]{Arthur Schnitzler an Felix Salten, 9. 7. 1893}
\nopagebreak\mylabel{v}
\rehead{ }\normalsize\beginnumbering\briefempfaengerindex{Salten, Felix@\textsc{Salten, Felix}!zzzSchnitzler, Arthur@\emph{von Arthur Schnitzler}!1893-07-091@{9. 7. 1893}|(be}
\toendnotes[C]{\smallbreak\pagebreak[2]}\Standort{Wienbibliothek im Rathaus, ZPH 1681, 2.1.516.}
\physDesc{Kartenbrief, 446 Zeichen
\newline{}Handschrift: Bleistift, deutsche Kurrent
\newline{}Versand: Stempel: »\nobreak{}\oindex{Bad Ischl@\textbf{Bad Ischl}, \emph{P.PPL}|pwk}I{[}sch{]}l, 9/7 93, 9 A\nobreak{}«. Stempel: »\nobreak{}\oindex{IX., Alsergrund@\textbf{IX., Alsergrund}, \emph{A.ADM3}|pwk}Wien 9/1 66, 10. 7. 93, 9 V., Bestellt\nobreak{}«.  
\newline{}Ordnung: mit Bleistift von unbekannter Hand nummeriert:
                                    »80« }\toendnotes[C]{\smallbreak}\pstart{}{\pb}Hrn \textsc{Felix
                     Salten}\pend{}\pstart{}\textcolor{pink}{Wien}{}\ledrightnote{\textcolor{pink}{Wien}}\pend{}\pstart{}\textsc{\textcolor{pink}{IX Berggasse 13}{}\ledrightnote{\textcolor{pink}{Berggasse}}}.\pend{}
{\bigskip}
\pstart
           \noindent{}{\pb}Lieber Freund! – Mein \label{K_L02959-1v}\edtext{\textcolor{green}{Stück}{}\ledrightnote{{$\rightarrow$}\textcolor{green}{Abschiedssouper}}{ }\textcolor{brown}{hier}{}\ledrightnote{{$\rightarrow$}\textcolor{brown}{Saisontheater Ischl}}{ }Freitag}{\lemma{\textnormal{\emph{Stück hier Freitag}}}\Cendnote{\textnormal{siehe Arthur Schnitzler an Felix Salten, 5. 7. 1893}}}\label{K_L02959-1h}. \textsc{\textsc{\uline{\textcolor{green}{Anatol}{}\ledrightnote{{$\rightarrow$}\textcolor{green}{Abschiedssouper}}}}}{ }\textsc{\textcolor{blue}{Hoefer}{}\ledrightnote{\textcolor{blue}{Emil Höfer}}}, \textsc{\textsc{\uline{\textcolor{green}{Max}{}\ledrightnote{{$\rightarrow$}\textcolor{green}{Abschiedssouper}}}}}{ }\textsc{\textcolor{blue}{Jarno}{}\ledrightnote{\textcolor{blue}{Josef Jarno}}}. \uline{\textsc{\textcolor{green}{Cora}{}\ledrightnote{{$\rightarrow$}\textcolor{green}{Die Frage an das Schicksal}}}}{ }\textsc{\textcolor{blue}{Wreden}{}\ledrightnote{{$\rightarrow$}\textcolor{blue}{Grethe Wreden}}}{ }\textsc{\uline{\textcolor{green}{Annie}{}\ledrightnote{{$\rightarrow$}\textcolor{green}{Abschiedssouper}}}}{ }\textsc{\textcolor{blue}{\substVorne{}\textsuperscript{M}\substDazwischen{}Gr\substHinten{}iebl}{}\ledrightnote{\textcolor{blue}{Marie Griebl}}} (\textcolor{brown}{Volkstheater}{}\ledrightnote{\textcolor{brown}{Volkstheater}}.) – \label{K_L02959-2v}\edtext{War beim \textcolor{blue}{Bezhauptm.}{}\ledrightnote{{$\rightarrow$}\textcolor{blue}{Franz von Aichelburg-Labia}}}{\lemma{\textnormal{\emph{War beim Bezhauptm.}}}\Cendnote{\textnormal{siehe A. S.: \emph{Tagebuch}, 7. 7. 1893}}}\label{K_L02959-2h} in \textcolor{pink}{Gmunden}{}\ledrightnote{\textcolor{pink}{Gmunden}} von wegen Cenſur.\pend
           
\pstart
           – Aus \textcolor{pink}{Wien}{}\ledrightnote{\textcolor{pink}{Wien}} von \label{K_L02959-3v}\edtext{Frl. \textsc{\textcolor{blue}{G.}{}\ledrightnote{\textcolor{blue}{Marie Glümer}}}}{\lemma{\textnormal{\emph{Frl. G.}}}\Cendnote{\textnormal{\textcolor{blue}{Marie Glümer}, siehe A. S.: \emph{Tagebuch}, 8. 7. 1893 und 9. 7. 1893}}}\label{K_L02959-3h} Verzweiflungsſchreie entſetzlicher Art. Ich habe kein Wort geſchrieben. –\pend
           
\pstart
           – Ein paar Verſe weiter»gedichtet« an de\textcolor{gray}{m}{ }\textcolor{green}{allegor. Gedicht}{}\ledrightnote{{$\rightarrow$}\textcolor{green}{Artifex}}. – – Schreibe
               dieſe Zeilen bei Frau \textsc{\textcolor{blue}{Flegmann}{}\ledrightnote{\textcolor{blue}{Bertha Flegmann}}}. – Eben ging \textsc{\textcolor{blue}{Brahms}{}\ledrightnote{\textcolor{blue}{Johannes Brahms}}} weg. – \textsc{\textcolor{blue}{Richard}{}\ledrightnote{\textcolor{blue}{Richard Beer-Hofmann}}} iſt da, grüßt Sie herzlich. Ihr \spacefill\mbox{Arthur}\pend
           \endnumbering\briefempfaengerindex{Salten, Felix@\textsc{Salten, Felix}!zzzSchnitzler, Arthur@\emph{von Arthur Schnitzler}!1893-07-091@{9. 7. 1893}|)be}\mylabel{h}  \normalsize

\doendnotes{C}
\bigskip
\vfill

\clearpage

\footnotesize

\lohead{\textsc{register}}

% Definiere theindex-Environment komplett neu ohne reledmac
\makeatletter
\renewenvironment{theindex}{%
  \section*{\indexname}%
  \setlength{\parindent}{0pt}%
  \setlength{\parskip}{0pt plus 0.3pt}%
  \let\item\@idxitem
}{%
  \clearpage
}
\makeatother

\IfFileExists{\jobname-pw.ind}{\input{\jobname-pw.ind}}{}

\end{document}

      