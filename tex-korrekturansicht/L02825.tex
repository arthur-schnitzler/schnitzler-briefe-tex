%% latex-korrekturansicht-vorspann.tex
%% Vorspann für die Korrekturansicht.
%% Lädt die gemeinsame Datei latex-vorspann.tex mit gesetztem Schalter.

\newif\ifkorrekturansicht
\korrekturansichttrue

\input{../tex-inputs/latex-vorspann}


               \section[ Paul Goldmann an Arthur Schnitzler, {[}17./18.?{]} 9. 1897]{Paul Goldmann an Arthur Schnitzler, {[}17./18.?{]} 9. 1897}\nopagebreak\mylabel{v}\rehead{ }\normalsize\beginnumbering\briefempfaengerindex{Schnitzler, Arthur@\textsc{Schnitzler, Arthur}!zzzGoldmann, Paul@\emph{von Paul Goldmann}!1897-09-172@{{[}17./18.?{]} 9. 1897}|(be} \toendnotes[C]{\smallbreak\pagebreak[2]} \Standort{DLA, A:Schnitzler, HS.NZ85.1.3167.}
\physDesc{Brief, 1 Blatt, 3 Seiten
\newline{}Handschrift: blaue Tinte, deutsche Kurrent}\toendnotes[C]{\smallbreak}\pstart
           \noindent{}{\pb}\textcolor{brown}{\textcolor{gray}{\textbf{\textsc{Frankfurter Zeitung}}}}{}\ledrightnote{\textcolor{brown}{Frankfurter Zeitung}}\hfill \textcolor{gray}{\textbf{\textcolor{pink}{Frankfurt a. M.}{}\ledrightnote{\textcolor{pink}{Frankfurt am Main}},}}{ }\label{K_L02825-111v}\edtext{17. Sept. \textcolor{gray}{\textbf{189}}7}{\lemma{\textnormal{\emph{17. Sept. 1897}}}\Cendnote{\textnormal{Dieser und der vorangegangene
                        Brief (Paul Goldmann an Arthur Schnitzler, [16./17.?] 9. [1897]) sind auf den
                        gleichen Tag datiert, in diesem Brief wird aber auf den vorangegangenen als
                           »geſtrigen Brief« verwiesen, wodurch entweder der
                        vorliegende auf den 18. 9. 1897 oder
                        andernfalls der frühere auf den 16. 9. 1897
                        zu datieren wäre.}}}\label{K_L02825-111h}.\pend
           \pstart
           \textsc{\textcolor{gray}{\textbf{und}}}\pend
           \pstart
           \textcolor{gray}{\textbf{\textsc{Handelsblatt.}}}\pend
           \pstart
           \textcolor{gray}{\textbf{\textsc{\textcolor{brown}{Redaction}{}\ledrightnote{→\textcolor{brown}{Frankfurter Zeitung}}.\footnote{\noindent{}\textcolor{gray}{\textbf{\textsc{Für die \textcolor{brown}{Redaktion} bestimmte Briefe und Sendungen
                                    wolle man \so{nicht} an die Person eines
                                    Redakteurs, sondern stets \textbf{an die Redaktion der
                                          \textcolor{brown}{Frankfurter Zeitung}} adressiren}}}.}}}}\pend
           \pstart
           \textcolor{gray}{\textbf{\textsc{Telegramm-Adresse:}}}\pend
           \pstart
           \textcolor{gray}{\textbf{\textsc{\textcolor{brown}{Zeitung}{}\ledrightnote{→\textcolor{brown}{Frankfurter Zeitung}}{ }\textcolor{pink}{Frankfurt Main}{}\ledrightnote{\textcolor{pink}{Frankfurt am Main}}.}}}\pend
           \pstart{}Mein lieber Freund,\pend\pstart
           Ich will Dir nur noch raſch für Deinen lieben Brief danken, den ich heut bekam.\pend
           \pstart
           Sieh’ nicht ſo trübſelig in die Zukunft und laß’ die Wolken machen, was ſie wollen.
               Dein Lebensweg liegt klar und ſchön vor meinen Blicken, und ich ſehe beſſer, weil
               Deine augenblicklichen \label{K_L02825-2v}\edtext{Verſtimmungen}{\lemma{\textnormal{\emph{Verſtimmungen}}}\Cendnote{\textnormal{wohl aufgrund von \textcolor{blue}{Schnitzler}s Affäre mit der verheirateten \textcolor{blue}{Rosa Freudenthal} und der noch immer relativ
                  geheim gehaltenen Schwangerschaft \textcolor{blue}{Marie
                     Reinhard}s}}}\label{K_L02825-2h} mir nicht die Ausſicht verdunkeln. Du wirſt wieder Ruhe
               bekommen, wirſt wieder arbeiten und dann wirſt Du ſelbſt wieder {\pb}beſſer und heiterer geſtimmt ſein. Ich meine, das
               Nöthigſte wäre für Dich, daß Du ſo raſch als möglich die Arbeit wieder aufzunehmen
               ſuchteſt.\pend
           \pstart
           Mein \textcolor{blue}{Schwager}{}\ledrightnote{→\textcolor{blue}{Josef Rosengart}} hat ſich über d\substVorne{}\textsuperscript{ie}\substDazwischen{}en\substHinten{} »\label{K_L02825-3v}\edtext{\textcolor{blue}{Bauernfänger}{}\ledrightnote{→\textcolor{blue}{Bernardo Bossato}}}{\lemma{\textnormal{\emph{Bauernfänger}}}\Cendnote{\textnormal{Bezug unklar}}}\label{K_L02825-3h}« ſehr amüſirt, bleibt aber \label{K_L02825-4v}\edtext{betreffs des Ohrenklingens}{\lemma{\textnormal{\emph{betreffs des Ohrenklingens}}}\Cendnote{\textnormal{siehe Paul Goldmann an Arthur Schnitzler, 13. 9. 1897}}}\label{K_L02825-4h} unerſchütterlich bei ſeiner Anſicht.\pend
           \pstart
           Wenn ich Deine Andeutungen bezüglich \label{K_L02825-5v}\edtext{Fräulein \textcolor{blue}{G.}{}\ledrightnote{→\textcolor{blue}{Marie Glümer}}}{\lemma{\textnormal{\emph{Fräulein G.}}}\Cendnote{\textnormal{Nur in Andeutungen im \emph{\textcolor{green}{Tagebuch}}
                  werden Momente einer komischen Geschichte klar: Einerseits erhielt \textcolor{blue}{Schnitzler}
                  am 30. 8. 1897 eine Karte von ihr, die
                  an einen anderen Liebhaber gerichtet gewesen sein dürfte. Am 3. 9. 1897
                  debütierte sie in \textcolor{pink}{Wien} und ihm war es ein Anliegen,
                  dass sie von seiner erfolgten Rückkehr nichts wusste.}}}\label{K_L02825-5h} richtig verſtanden habe, ſo iſt das
               eine vollendet komiſche Geſchichte.\pend
           \pstart
           Die \label{K_L02825-6v}\edtext{nächſte Woche}{\lemma{\textnormal{\emph{nächſte Woche}}}\Cendnote{\textnormal{Eine Woche später,
                  am 24. 9. 1897, kam es zur
                  Totgeburt des \textcolor{blue}{Sohn}s von
                     \textcolor{blue}{Schnitzler} und \textcolor{blue}{Marie Reinhard} in \textcolor{pink}{Mauer bei
                     Wien}.}}}\label{K_L02825-6h} wird alſo, wie ich aus Deinem Briefe entnehme, wichtig und
               ereignißreich werden. Ich wünſche Dir und Deiner {\pb}\textcolor{blue}{Freundin}{}\ledrightnote{→\textcolor{blue}{Marie Reinhard}} von Herzen allen
               guten Muth in den bevorſtehenden ſchweren Stunden.\pend
           \pstart
           Auf meinen \label{K_L02825-8v}\edtext{geſtrigen Brief}{\lemma{\textnormal{\emph{geſtrigen Brief}}}\Cendnote{\textnormal{siehe Paul Goldmann an Arthur Schnitzler, [16./17.?] 9. [1897]}}}\label{K_L02825-8h} antworteſt Du wohl baldmöglichſt.\pend
           \pstart
           Die Meinigen grüßen Dich.\pend
           \pstart
           In Treue {\\[\baselineskip]}Dein {\\[\baselineskip]}\spacefill\mbox{Paul Goldm}\pend
           \leftskip=0em{}\pstart
           \noindent{}Was machen \textsc{\textcolor{blue}{Richard}{}\ledrightnote{\textcolor{blue}{Richard Beer-Hofmann}}} und \label{K_L02825-11v}\edtext{\textsc{\textcolor{blue}{Richard}{}\ledrightnote{\textcolor{blue}{Richard Beer-Hofmann}}s}{ }\textcolor{blue}{Tochter}{}\ledrightnote{→\textcolor{blue}{Mirjam Beer-Hofmann}}}{\lemma{\textnormal{\emph{Richards Tochter}}}\Cendnote{\textnormal{\textcolor{blue}{Mirjam}, die Tochter von \textcolor{blue}{Richard} und \textcolor{blue}{Paula
                        Beer-Hofmann}, kam am 4. 9. 1897 zur Welt.}}}\label{K_L02825-11h}?\pend
           \endnumbering\briefempfaengerindex{Schnitzler, Arthur@\textsc{Schnitzler, Arthur}!zzzGoldmann, Paul@\emph{von Paul Goldmann}!1897-09-172@{{[}17./18.?{]} 9. 1897}|)be}\mylabel{h}  \normalsize

\doendnotes{C}
\bigskip
\vfill

\clearpage

\footnotesize

\lohead{\textsc{register}}

% Definiere theindex-Environment komplett neu ohne reledmac
\makeatletter
\renewenvironment{theindex}{%
  \section*{\indexname}%
  \setlength{\parindent}{0pt}%
  \setlength{\parskip}{0pt plus 0.3pt}%
  \let\item\@idxitem
}{%
  \clearpage
}
\makeatother

\IfFileExists{\jobname-pw.ind}{\input{\jobname-pw.ind}}{}

\end{document}

      