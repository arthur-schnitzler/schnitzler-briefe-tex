%% latex-korrekturansicht-vorspann.tex
%% Vorspann für die Korrekturansicht.
%% Lädt die gemeinsame Datei latex-vorspann.tex mit gesetztem Schalter.

\newif\ifkorrekturansicht
\korrekturansichttrue

\input{../tex-inputs/latex-vorspann}


\renewcommand{\erwaehntePersonen}{Personen: Adolf Brüning, Sébastien Roch Nicolas Chamfort, Peter Dorner, Johann Wolfgang von Goethe, Alfred Kerr, Franz Lenbach, Adolphe Mathurin de Lescure, Felix Salten, Olga Schnitzler, Elisabeth Steinrück, Franz von Stuck, Irene Triesch, H. G. Wells}
\renewcommand{\erwaehnteInstitutionen}{Institutionen: Deutsches Theater Berlin, Flammarion, Hermann Seemann Nachfolger, Verlag von Ernst Wasmuth}
\renewcommand{\erwaehnteOrte}{Orte: Berlin, Dessauer Straße, Deutsches Theater Berlin, Leipzig, München, Paris, Rooseveltplatz, Wien}
\renewcommand{\erwaehnteWerke}{Werke: Das rothe Zimmer, Die Frau mit dem Dolche, Die Schmiedekunst nach Originalen des XV. bis XVIII. Jahrhunderts, Die Schmiedekunst seit dem Ende der Renaissance, Die Zeit. Wiener Wochenschrift, Lebendige Stunden. Vier Einakter, Literatur, Œuvres choisies de N. Chamfort, publiées avec préface, notes et tables}
\section[ Paul Goldmann an Arthur Schnitzler, 23. 9. {[}1901{]}]{Paul Goldmann an Arthur Schnitzler, 23. 9. {[}1901{]}}
\nopagebreak\mylabel{v}
\rehead{ }\normalsize\beginnumbering\briefempfaengerindex{Schnitzler, Arthur@\textsc{Schnitzler, Arthur}!zzzGoldmann, Paul@\emph{von Paul Goldmann}!1901-09-231@{23. 9. {[}1901{]}}|(be}
\toendnotes[C]{\smallbreak\pagebreak[2]}\Standort{DLA, A:Schnitzler, HS.NZ85.1.3171.}
\physDesc{Brief, 1 Blatt, 4 Seiten
\newline{}Handschrift: blaue Tinte, deutsche Kurrent
\newline{}Schnitzler: 1) mit Bleistift das Jahr »{[}1{]}901« vermerkt  2) mit rotem Buntstift vier Unterstreichungen}\toendnotes[C]{\smallbreak}
\pstart
           \noindent{}\raggedleft{}{\pb}\textcolor{pink}{\textcolor{gray}{\textbf{DESSAUERSTRASSE 19}}}{}\ledrightnote{\textcolor{pink}{Dessauer Straße}}\pend
           
\pstart
           \textcolor{pink}{Berlin}{}\ledrightnote{\textcolor{pink}{Berlin}}, 23. Sept.\pend
           
\pstart\center{}Mein lieber Freund,\pend
\pstart
           Die \textsc{\textcolor{blue}{Triesch}{}\ledrightnote{\textcolor{blue}{Irene Triesch}}} hat bereits die \label{K_L03085-1v}\edtext{Rollen in Deinen
                  \textcolor{green}{Stücken}{}\ledrightnote{{$\rightarrow$}\textcolor{green}{Lebendige Stunden. Vier Einakter}}}{\lemma{\textnormal{\emph{Rollen in Deinen
                  Stücken}}}\Cendnote{\textnormal{In \emph{\textcolor{green}{Die
                     Frau mit dem Dolche}} spielte \textcolor{blue}{Irene
                     Triesch} am \textcolor{pink}{Deutschen Theater Berlin} die
                  Rolle der \textcolor{green}{Pauline} und in
                     \emph{\textcolor{green}{Literatur}} jene der \textcolor{green}{Margarete}.}}}\label{K_L03085-1h} bekommen und iſt
               namentlich von de\substVorne{}\textsuperscript{m}\substDazwischen{}r\substHinten{}{ }\textcolor{green}{Frau mit dem Dolch}{}\ledrightnote{\textcolor{green}{Die Frau mit dem Dolche}} entzückt. Hat ſie ſich auch
               bereits recht hübſch zurechtgelegt\strikeout{.} und ſpricht jeden
               Tag \textcolor{blue}{Goethe}{}\ledrightnote{\textcolor{blue}{Johann Wolfgang von Goethe}}ſche Verſe, um ſich im
               Verſe-Recitiren zu üben. Sie will nach \textsc{\textcolor{pink}{München}{}\ledrightnote{\textcolor{pink}{München}}} fahren und \textsc{\textcolor{blue}{Lenbach}{}\ledrightnote{\textcolor{blue}{Franz Lenbach}}} oder \textsc{\textcolor{blue}{Stuck}{}\ledrightnote{\textcolor{blue}{Franz von Stuck}}} bitten, das betreffende \label{K_L03085-2v}\edtext{Bild}{\lemma{\textnormal{\emph{Bild}}}\Cendnote{\textnormal{\textcolor{blue}{Franz Lenbach} und \textcolor{blue}{Franz von Stuck} waren \textcolor{pink}{Münch}ner Maler.}}}\label{K_L03085-2h} zu {\pb}entwerfen, was gar nicht übel
               wäre.\pend
           
\pstart
           \strikeout{Daß} Wann \strikeout{kommſt}{ }\label{K_L03085-3v}\edtext{kommſt}{\lemma{\textnormal{\emph{kommſt}}}\Cendnote{\textnormal{\textcolor{blue}{Schnitzler} kam für die Uraufführung von \emph{\textcolor{green}{Lebendige Stunden}} (4. 1. 1902, \emph{\textcolor{brown}{Deutsches Theater}}) nach \textcolor{pink}{Berlin}. Er blieb von 28. 12. 1901 bis 6. 1. 1902.}}}\label{K_L03085-3h} Du?\pend
           
\pstart
           Daß Du mir \textsc{\textcolor{blue}{Kerr}{}\ledrightnote{\textcolor{blue}{Alfred Kerr}}s} Beſuch in \textcolor{pink}{Berlin}{}\ledrightnote{\textcolor{pink}{Berlin}} verſchwiegen haſt, iſt bedauerlich. Immerhin wirſt Du
               bei unſerem nächſten Beiſammenſein behaupten, es mir geſchrieben zu haben.\pend
           
\pstart
           \textsc{\textcolor{blue}{Salten}{}\ledrightnote{\textcolor{blue}{Felix Salten}}} iſt morgen bei mir zu Tiſch.\pend
           
\pstart
           \textsc{\textcolor{blue}{Peter Dorner}{}\ledrightnote{\textcolor{blue}{Peter Dorner}}}, denke Dir!, ſchickte {\pb}mir
               dieſelben \label{K_L03085-123v}\edtext{Bücher}{\lemma{\textnormal{\emph{Bücher}}}\Cendnote{\textnormal{nicht ermittelt}}}\label{K_L03085-123h}, die er Dir
               geſandt. Ich habe ihm ein ſchönes \label{K_L03085-1111v}\edtext{\textcolor{green}{Werk über Schmiede\substVorne{}\textsuperscript{\textcolor{gray}{r}}\substDazwischen{}a\substHinten{}rbeit}{}\ledrightnote{{$\rightarrow$}\textcolor{green}{Die Schmiedekunst nach Originalen des XV. bis XVIII. Jahrhunderts}} mit Nachbildungen alter Meiſterſtücke}{\lemma{\textnormal{\emph{Werk … Meiſterſtücke}}}\Cendnote{\textnormal{Möglicherweise: \emph{\textcolor{green}{Die Schmiedekunst nach Originalen des XV. bis
                        XVIII. Jahrhunderts}}. \textcolor{pink}{Berlin}: \emph{\textcolor{brown}{Verlag von Ernst Wasmuth}}{ }1887.}}}\label{K_L03085-1111h} im Betrage von 30 \textsc{MK}, als Gegengeſchenk
               geſandt. Dann gibt es ein noch viel ſchöneres \textcolor{green}{Werk}{}\ledrightnote{{$\rightarrow$}\textcolor{green}{Die Schmiedekunst nach Originalen des XV. bis XVIII. Jahrhunderts}} derſelben Art, das 44 \textsc{MK} koſtet, betitelt »\label{K_L03085-13v}\edtext{\textcolor{green}{Die deutſche Schmiedekunſt}{}\ledrightnote{\textcolor{green}{Die Schmiedekunst seit dem Ende der Renaissance}}}{\lemma{\textnormal{\emph{Die … Schmiedekunſt}}}\Cendnote{\textnormal{Vermutlich: \textcolor{blue}{Adolf Brüning}: \emph{\textcolor{green}{Die Schmiedekunst seit dem Ende der Renaissance}}.
                     Mit 150 Abbildungen. \textcolor{pink}{Leipzig}: \emph{\textcolor{brown}{Verlag von Hermann Seemann
                        Nachfolger}} [1901?].}}}\label{K_L03085-13h}«. Mir allein
               iſt es zu theuer. Möchteſt Du Dich mit der Hälfte betheiligen? Davon würde der \textcolor{blue}{Mann}{}\ledrightnote{{$\rightarrow$}\textcolor{blue}{Peter Dorner}} wenigſtens \strikeout{etwas} etwas Ordentliches {\pb}profitiren.\pend
           
\pstart
           Danke den lieben \textcolor{blue}{Mädchen}{}\ledrightnote{{$\rightarrow$}\textcolor{blue}{Olga Schnitzler}{\newline}{$\rightarrow$}\textcolor{blue}{Elisabeth Steinrück}} in meinem Namen für ihre reizenden Briefe, die mich unendlich
               erfreut haben. Sie ſollen nur nicht böſe ſein, daß ich nicht gleich antworte; aber
               ich ſtecke tief in der Arbeit. Nächſter Tage ſchreibe ich ihnen. Iſt die Adreſſe
               immer noch \textsc{\textcolor{pink}{Maximilianplatz}{}\ledrightnote{\textcolor{pink}{Rooseveltplatz}}}?\pend
           
\pstart
           Viele treue Grüße {\\[\baselineskip]}Dein {\\[\baselineskip]}\spacefill\mbox{Paul Goldmnn}\pend
           \leftskip=0em{}
\pstart
           \noindent{}\label{T_L03085-1v}\edtext{{\pb}Lies’ in der letzten »\textcolor{green}{Zeit}{}\ledrightnote{\textcolor{green}{Die Zeit. Wiener Wochenschrift}}« die ſchönen Geſpenſtergeſchichten
                     \label{K_L03085-99v}\edtext{»\textcolor{green}{Das rothe Zimmer}{}\ledrightnote{\textcolor{green}{Das rothe Zimmer}}«}{\lemma{\textnormal{\emph{»Das rothe Zimmer«}}}\Cendnote{\textnormal{\textcolor{blue}{H. G. Wells}: \emph{\textcolor{green}{Das rothe Zimmer}}. In: \emph{\textcolor{green}{Die Zeit. Wiener Wochenschrift}}, Jg. XXXX, Nr. XXXX,
                        XXXX (Datum), S. XXXX.}}}\label{K_L03085-99h}. \label{K_L03085-33v}\edtext{\textsc{\textcolor{blue}{Chamfort}{}\ledrightnote{\textcolor{blue}{Sébastien Roch Nicolas Chamfort}}}}{\lemma{\textnormal{\emph{Chamfort}}}\Cendnote{\textnormal{\emph{\textcolor{green}{Œuvres choisies de N. Chamfort, publiées avec
                        préface, notes et tables}}. 2 Bde. Hg. v. \textcolor{blue}{Adolphe Mathurin de Lescure}. \textcolor{pink}{Paris}: \emph{\textcolor{brown}{Flammarion}}{ }1892.}}}\label{K_L03085-33h} (\textcolor{green}{\begin{otherlanguage}{french}\textsc{Œuvres choisies}\end{otherlanguage}, in 2 Bänden}{}\ledrightnote{\textcolor{green}{Œuvres choisies de N. Chamfort, publiées avec préface, notes et tables}}) iſt bei \textsc{\textcolor{brown}{Flammarion}{}\ledrightnote{\textcolor{brown}{Flammarion}}} erſchienen.}{\lemma{\textnormal{\emph{Lies’ … erſchienen.}}}\Cendnote{\textnormal{kopfüber am oberen
                     Rand der ersten Seite}}}\label{T_L03085-1h}\pend
           \endnumbering\briefempfaengerindex{Schnitzler, Arthur@\textsc{Schnitzler, Arthur}!zzzGoldmann, Paul@\emph{von Paul Goldmann}!1901-09-231@{23. 9. {[}1901{]}}|)be}\mylabel{h}
\begin{anhang}
\end{anhang}\normalsize

\doendnotes{C}
\bigskip
\vfill

\clearpage

\footnotesize

\lohead{\textsc{register}}

% Definiere theindex-Environment komplett neu ohne reledmac
\makeatletter
\renewenvironment{theindex}{%
  \section*{\indexname}%
  \setlength{\parindent}{0pt}%
  \setlength{\parskip}{0pt plus 0.3pt}%
  \let\item\@idxitem
}{%
  \clearpage
}
\makeatother

\IfFileExists{\jobname-pw.ind}{\input{\jobname-pw.ind}}{}

\end{document}

      