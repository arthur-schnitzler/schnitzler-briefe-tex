%% latex-korrekturansicht-vorspann.tex
%% Vorspann für die Korrekturansicht.
%% Lädt die gemeinsame Datei latex-vorspann.tex mit gesetztem Schalter.

\newif\ifkorrekturansicht
\korrekturansichttrue

\input{../tex-inputs/latex-vorspann}


               \section[Arthur Schnitzler an Georg Brandes, 21. 8. 1920]{ Arthur Schnitzler an Georg Brandes, 21. 8. 1920}\nopagebreak\mylabel{v}\rehead{ }\normalsize\beginnumbering\briefempfaengerindex{Brandes, Georg@\textsc{Brandes, Georg}!zzzSchnitzler, Arthur@\emph{von Arthur Schnitzler}!1920-08-211@{21. 8. 1920}|(be} \toendnotes[C]{\smallbreak\pagebreak[2]} \Standort{Kopenhagen, Det Kongelige Bibliotek, Georg Brandes Arkiv, box 125.}
\physDesc{Postkarte
\newline{}Handschrift: Bleistift, lateinische Kurrent\newline{}Versand: Stempel: »\nobreak{}\oindex{XVIII., Waehring@\textbf{XVIII., Währing}, \emph{Bezirk (A.BZK)}|pwk}18/\textcolor{gray}{×} Wien, 21. VIII. 20, 4\nobreak{}«.  \newline{}Ordnung: 1) mit Bleistift von unbekannter Hand links der Briefmarke nummeriert: »43« 2) mit Bleistift von unbekannter Hand auf der Textseite zusätzlich
                                 die Datierung wiederholt: »21/8 20«}\buchAbdrucke{\weitereDrucke{Georg Brandes, Arthur Schnitzler: \emph{Ein Briefwechsel}. Hg. Kurt Bergel. Bern: \emph{Francke} 1956, S. 131.} }\pstart{}{\pb}\textcolor{pink}{Wien XVIII. Sternwartestr 71.}{}\ledrightnote{\textcolor{pink}{Sternwartestraße}},
                        A S\pend{}{\bigskip}\pstart{}Hrn Georg Brandes\pend{}\pstart{}\textcolor{pink}{Kopenhagen}{}\ledrightnote{\textcolor{pink}{Kopenhagen}}\pend{}\pstart{}\textcolor{pink}{Daenemark}{}\ledrightnote{\textcolor{pink}{Dänemark}}\pend{}{\bigskip}\pstart
           \raggedleft{}{\pb}21. 8. 20\pend
           \pstart
           lieber und verehrter Freund, eben trifft Ihre Karte vom
                        17. 8 ein. Ihr Brief vom 13. 6 ist angelangt; vor
                    etwa 4, 5 Tagen schrieb ich Ihnen einen sehr langen Brief\substVorne{}\textsuperscript{–}\substDazwischen{},\substHinten{} und wünschte mir sehr eine Bestätigung zu erhalten, daß Sie ihn in
                    Händen haben, mir fällt ein, dſs ich Ihnen von gemeinsamen Beka{\geminationn}ten kaum etwas geschrieben habe. \textcolor{blue}{Richard Beer Hofm}{}\ledrightnote{\textcolor{blue}{Richard Beer-Hofmann}} mit den Seinigen befindet sich wohl,
                    und ich treffe nächster Tage mit ihm in \textcolor{pink}{Aussee}{}\ledrightnote{\textcolor{pink}{Bad Aussee}}
                        zusa{\geminationm}en. In der gleichen Gegend \textcolor{blue}{Hofma{\geminationn}sthal}{}\ledrightnote{\textcolor{blue}{Hugo von Hofmannsthal}},
                        \textcolor{blue}{Salten}{}\ledrightnote{\textcolor{blue}{Felix Salten}} nicht weit davon am Attersee;– wir alle sind eigentlich, we{\geminationn} mans recht bedenkt – bisher – über die Unbilden
                    dieser Zeit ganz leidlich weggeko{\geminationm}en;– was fingen
                    wir Menschen ohne {\pb}unsre bewunderungswürdige
                    und etwas beschämende Accomodationsfähigkeit an.\pend
           \pstart
           Ich bin wie immer von ganzem Herzen{\\[\baselineskip]}Ihr getreuer{\\[\baselineskip]}\spacefill\mbox{Arthur Schnitzler}\pend
           \leftskip=0em{}\endnumbering\briefempfaengerindex{Brandes, Georg@\textsc{Brandes, Georg}!zzzSchnitzler, Arthur@\emph{von Arthur Schnitzler}!1920-08-211@{21. 8. 1920}|)be}\mylabel{h}  \normalsize

\doendnotes{C}
\bigskip
\vfill

\clearpage

\footnotesize

\lohead{\textsc{register}}

% Definiere theindex-Environment komplett neu ohne reledmac
\makeatletter
\renewenvironment{theindex}{%
  \section*{\indexname}%
  \setlength{\parindent}{0pt}%
  \setlength{\parskip}{0pt plus 0.3pt}%
  \let\item\@idxitem
}{%
  \clearpage
}
\makeatother

\IfFileExists{\jobname-pw.ind}{\input{\jobname-pw.ind}}{}

\end{document}

      