%% latex-korrekturansicht-vorspann.tex
%% Vorspann für die Korrekturansicht.
%% Lädt die gemeinsame Datei latex-vorspann.tex mit gesetztem Schalter.

\newif\ifkorrekturansicht
\korrekturansichttrue

\input{../tex-inputs/latex-vorspann}


         
         \renewcommand{\erwaehntePersonen}{Personen:  ?? [zweiter Feuilletonredakteur der Frankfurter Zeitung], Richard Beer-Hofmann, Alfred Gold, Robert Hirschfeld, Hugo von Hofmannsthal, Fedor Mamroth, Gustav Schwarzkopf, Jakob Wassermann}
         \renewcommand{\erwaehnteInstitutionen}{Institutionen: Frankfurter Zeitung}
         \renewcommand{\erwaehnteOrte}{Orte: Frankfurt am Main, Wien}
         \renewcommand{\erwaehnteWerke}{Werke: Das Bergwerk zu Falun, Der Graf von Charolais. Ein Trauerspiel, Die Frau im Fenster. Die Hochzeit der Sobeide. Der Abenteurer und die Sängerin. Theater in Versen, Ein Sommer in China. Reisebilder}
               \section[ Paul Goldmann an Arthur Schnitzler, 6. 12. {[}1899{]}]{Paul Goldmann an Arthur Schnitzler, 6. 12. {[}1899{]}}\nopagebreak\mylabel{v}\rehead{ }\normalsize\beginnumbering\briefempfaengerindex{Schnitzler, Arthur@\textsc{Schnitzler, Arthur}!zzzGoldmann, Paul@\emph{von Paul Goldmann}!1899-12-061@{6. 12. {[}1899{]}}|(be} \toendnotes[C]{\smallbreak\pagebreak[2]} \Standort{DLA, A:Schnitzler, HS.NZ85.1.3169.}
\physDesc{Brief, 1 Blatt, 4 Seiten
\newline{}Handschrift: blaue Tinte, deutsche Kurrent
\newline{}Schnitzler: 1) mit Bleistift das Jahr »99« vermerkt  2) mit rotem Buntstift fünf Unterstreichungen}\toendnotes[C]{\smallbreak}\pstart
           \centering{}{\pb}\textcolor{pink}{Frankfurt}{}\ledrightnote{\textcolor{pink}{Frankfurt am Main}}, 6. Dezember.\pend
           \pstart\center{}Mein lieber Freund,\pend\pstart
           Eine kleine Anfrage, die ich Dich aber bitten muß, ſtreng vertraulich zu behandeln.
               Die »\textcolor{brown}{Frankfurter Zeitung}{}\ledrightnote{\textcolor{brown}{Frankfurter Zeitung}}« ſucht einen \label{K_L02897-1v}\edtext{zweiten \textcolor{blue}{Feuilleton-Redakteur}{}\ledrightnote{{$\rightarrow$}\textcolor{blue}{?? [zweiter Feuilletonredakteur der Frankfurter Zeitung]}}}{\lemma{\textnormal{\emph{zweiten Feuilleton-Redakteur}}}\Cendnote{\textnormal{nicht ermittelt}}}\label{K_L02897-1h}, eine Hilſskraft
               für \textsc{Dr. \textcolor{blue}{Mamroth}{}\ledrightnote{\textcolor{blue}{Fedor Mamroth}}}; eventuell könnte der Betreffende zugleich das Muſik-Referat übernehmen. Weißt
               Du Jemanden, einen jüngeren oder älteren Mann, der geeignet wäre? Was iſt
               beiſpielsweiſe mit {\pb}\textsc{\textcolor{blue}{Alfred Gold}{}\ledrightnote{\textcolor{blue}{Alfred Gold}}}?\pend
           \pstart
           Weiter, gleichfalls vertraulich: \textsc{\textcolor{blue}{Wassermann}{}\ledrightnote{\textcolor{blue}{Jakob Wassermann}}} iſt nicht mehr zu halten. Er hat die \textcolor{brown}{Berichterſtattung}{}\ledrightnote{{$\rightarrow$}\textcolor{brown}{Frankfurter Zeitung}} gar zu gewiſſenlos geführt. Man wird ihm am
                  1. Januar kündigen. Ich habe bereits Alles gethan,
               um \textsc{\textcolor{blue}{Schwarzkopf}{}\ledrightnote{\textcolor{blue}{Gustav Schwarzkopf}}} die Stelle zu verſchaffen. Mein \textcolor{blue}{Onkel}{}\ledrightnote{{$\rightarrow$}\textcolor{blue}{Fedor Mamroth}} iſt einverſtanden, und wenn mir die \label{K_L02897-2v}\edtext{\begin{otherlanguage}{french}Canaille\end{otherlanguage}}{\lemma{\textnormal{\emph{Canaille}}}\Cendnote{\textnormal{französisch: Schurkin; siehe auch Paul Goldmann an Arthur Schnitzler, 2. [1.? 1897] und Paul Goldmann an Arthur Schnitzler, 12. 11. [1899]}}}\label{K_L02897-2h}, ſeine Frau, nicht dazwiſchen hetzt, wird es wohl werden. \strikeout{M\textcolor{gray}{irt}} Mir hätte, offen geſtanden, \textsc{\textcolor{blue}{Hirschfeld}{}\ledrightnote{\textcolor{blue}{Robert Hirschfeld}}}{ }{\pb}näher gelegen. Aber Dir zuliebe ſoll es \textsc{\textcolor{blue}{Schwarzkopf}{}\ledrightnote{\textcolor{blue}{Gustav Schwarzkopf}}} ſein – wenn eben nichts Unvorhergeſehenes dazwiſchen kommt.\pend
           \pstart
           Viele treue Grüße! {\\[\baselineskip]}Dein {\\[\baselineskip]}\spacefill\mbox{Paul Goldmann}\pend
           \leftskip=0em{}\pstart
           \noindent{}Was macht \label{K_L02897-98v}\edtext{\textsc{\textcolor{blue}{Richard}{}\ledrightnote{\textcolor{blue}{Richard Beer-Hofmann}}s}{ }\textcolor{green}{Drama}{}\ledrightnote{{$\rightarrow$}\textcolor{green}{Der Graf von Charolais. Ein Trauerspiel}}}{\lemma{\textnormal{\emph{Richards Drama}}}\Cendnote{\textnormal{seit Sommer arbeitete \textcolor{blue}{Beer-Hofmann} an dem Trauerspiel \emph{\textcolor{green}{Der Graf von Charolais}} (vgl. Richard Beer-Hofmann an Arthur Schnitzler, 28. 8. 1899)}}}\label{K_L02897-98h}? Und was
                     \label{K_L02897-123v}\edtext{\textcolor{green}{dasjenige}{}\ledrightnote{{$\rightarrow$}\textcolor{green}{Das Bergwerk zu Falun}} von \textsc{\textcolor{blue}{Hoffmannsthal}{}\ledrightnote{\textcolor{blue}{Hugo von Hofmannsthal}}}}{\lemma{\textnormal{\emph{dasjenige von Hoffmannsthal}}}\Cendnote{\textnormal{Bezug auf \emph{\textcolor{green}{Das Bergwerk zu Falun}}, das \textcolor{blue}{Hugo von Hofmannsthal} am 29. 10. 1899 bei \textcolor{blue}{Beer-Hofmann} in Anwesenheit \textcolor{blue}{Schnitzler}s vorlas?}}}\label{K_L02897-123h}? Letzterer hat
                  mir vor einigen Wochen {\pb}ſein \label{K_L02897-75v}\edtext{Buch}{\lemma{\textnormal{\emph{Buch}}}\Cendnote{\textnormal{\emph{\textcolor{green}{Die Frau im Fenster. Die
                        Hochzeit der Sobeide. Der Abenteurer und die Sängerin. Theater in
                        Versen}} erschien zwar bereits im April 1899, aber teilweise
                     versandte es \textcolor{blue}{Hofmannsthal} erst gegen
                     Jahresende.}}}\label{K_L02897-75h} geſchickt mit einer Widmung: »in herzlicher Sympathie«. Ich
                  hatte Luſt, ihm \textcolor{green}{meines}{}\ledrightnote{{$\rightarrow$}\textcolor{green}{Ein Sommer in China. Reisebilder}}
                  zurückzuſchicken mit der Widmung: »in ſympathiſcher Herzlichkeit« – habe es aber
                  unterlaſſen.\pend
           \endnumbering\briefempfaengerindex{Schnitzler, Arthur@\textsc{Schnitzler, Arthur}!zzzGoldmann, Paul@\emph{von Paul Goldmann}!1899-12-061@{6. 12. {[}1899{]}}|)be}\mylabel{h}\begin{anhang}\end{anhang}\normalsize

\doendnotes{C}
\bigskip
\vfill

\clearpage

\footnotesize

\lohead{\textsc{register}}

% Definiere theindex-Environment komplett neu ohne reledmac
\makeatletter
\renewenvironment{theindex}{%
  \section*{\indexname}%
  \setlength{\parindent}{0pt}%
  \setlength{\parskip}{0pt plus 0.3pt}%
  \let\item\@idxitem
}{%
  \clearpage
}
\makeatother

\IfFileExists{\jobname-pw.ind}{\input{\jobname-pw.ind}}{}

\end{document}

      