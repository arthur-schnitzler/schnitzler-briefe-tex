%% latex-korrekturansicht-vorspann.tex
%% Vorspann für die Korrekturansicht.
%% Lädt die gemeinsame Datei latex-vorspann.tex mit gesetztem Schalter.

\newif\ifkorrekturansicht
\korrekturansichttrue

\input{../tex-inputs/latex-vorspann}


\renewcommand{\erwaehnteInstitutionen}{Institutionen: Die Zeit}
\renewcommand{\erwaehnteOrte}{Orte: Exelberg, Greifenstein, Wien}
\renewcommand{\erwaehnteWerke}{}
\section[ Felix Salten an Arthur Schnitzler, {[}1. 8. 1903{]}]{Felix Salten an Arthur Schnitzler, {[}1. 8. 1903{]}}
\nopagebreak\mylabel{v}
\rehead{ }\normalsize\beginnumbering\briefempfaengerindex{Schnitzler, Arthur@\textsc{Schnitzler, Arthur}!zzzSalten, Felix@\emph{von Felix Salten}!1903-08-011@{{[}1. 8. 1903{]}}|(be}
\toendnotes[C]{\smallbreak\pagebreak[2]}\Standort{CUL, Schnitzler, B 89, A 2.}
\physDesc{Brief, 1 Blatt, 1 Seite, 328 Zeichen
\newline{}Handschrift: Bleistift, lateinische Kurrent
\newline{}Schnitzler: mit Bleistift datiert: »1/8 903.« 
\newline{}Ordnung: mit Bleistift von unbekannter Hand nummeriert: »166« }\toendnotes[C]{\smallbreak}
\pstart
           \noindent{}{\pb}Lieber, wollen Sie \label{K_L03341-1v}\edtext{morgen}{\lemma{\textnormal{\emph{morgen}}}\Cendnote{\textnormal{siehe A. S.: \emph{Tagebuch}, 2. 8. 1903}}}\label{K_L03341-1h}{ }früh eine Radparthie mit mir
               machen? Ich fahre über den \textcolor{pink}{Exelberg}{}\ledrightnote{\textcolor{pink}{Exelberg}} nach \textcolor{pink}{Greifenstein}{}\ledrightnote{\textcolor{pink}{Greifenstein}}. Etwa ½ 10 Uhr. Bin
               aber auch bereit, wo anders hin zu fahren. Wenn Sie Lust haben, telefoniren Sie mich
               bei Empfang ds. in der \textcolor{brown}{Redaction}{}\ledrightnote{{$\rightarrow$}\textcolor{brown}{Die Zeit}} an. Bin ich nicht da, dann sagen Sie dem Beamten, bitte, das
               Nothwendige.\pend
           
\pstart
           herzlich {\\[\baselineskip]}Ihr {\\[\baselineskip]}\spacefill\mbox{Salten}\pend
           \leftskip=0em{}\endnumbering\briefempfaengerindex{Schnitzler, Arthur@\textsc{Schnitzler, Arthur}!zzzSalten, Felix@\emph{von Felix Salten}!1903-08-011@{{[}1. 8. 1903{]}}|)be}\mylabel{h}  \normalsize

\doendnotes{C}
\bigskip
\vfill

\clearpage

\footnotesize

\lohead{\textsc{register}}

% Definiere theindex-Environment komplett neu ohne reledmac
\makeatletter
\renewenvironment{theindex}{%
  \section*{\indexname}%
  \setlength{\parindent}{0pt}%
  \setlength{\parskip}{0pt plus 0.3pt}%
  \let\item\@idxitem
}{%
  \clearpage
}
\makeatother

\IfFileExists{\jobname-pw.ind}{\input{\jobname-pw.ind}}{}

\end{document}

      