%% latex-korrekturansicht-vorspann.tex
%% Vorspann für die Korrekturansicht.
%% Lädt die gemeinsame Datei latex-vorspann.tex mit gesetztem Schalter.

\newif\ifkorrekturansicht
\korrekturansichttrue

\input{../tex-inputs/latex-vorspann}


\renewcommand{\erwaehnteOrte}{Orte: Berlin, Dessauer Straße, [Hotel]}
\renewcommand{\erwaehnteWerke}{Werke: Tagebuch}
\section[ Paul Goldmann an Arthur Schnitzler, 2. 3. {[}1901{]}]{Paul Goldmann an Arthur Schnitzler, 2. 3. {[}1901{]}}
\nopagebreak\mylabel{v}
\rehead{ }\normalsize\beginnumbering\briefempfaengerindex{Schnitzler, Arthur@\textsc{Schnitzler, Arthur}!zzzGoldmann, Paul@\emph{von Paul Goldmann}!1901-03-021@{2. 3. {[}1901{]}}|(be}
\toendnotes[C]{\smallbreak\pagebreak[2]}\Standort{DLA, A:Schnitzler, HS.NZ85.1.3171.}
\physDesc{Brief, 1 Blatt, 1 Seite
\newline{}Handschrift: blaue Tinte, deutsche Kurrent
\newline{}Schnitzler: mit Bleistift das Jahr »{[}1{]}901« vermerkt }\toendnotes[C]{\smallbreak}
\pstart
           \noindent{}\raggedleft{}{\pb}\textcolor{pink}{\textcolor{gray}{\textbf{DESSAUERSTRASSE 19}}}{}\ledrightnote{\textcolor{pink}{Dessauer Straße}}\pend
           
\pstart
           \textcolor{pink}{Berlin}{}\ledrightnote{\textcolor{pink}{Berlin}}, 2. März.\pend
           
\pstart\center{}Mein lieber Freund,\pend
\pstart
           Sei herzlichſt \label{K_L03060-1v}\edtext{willkommen}{\lemma{\textnormal{\emph{willkommen}}}\Cendnote{\textnormal{\textcolor{blue}{Schnitzler} war zwischen 3. 3. 1901 und 10. 3. 1901 in \textcolor{pink}{Berlin}.}}}\label{K_L03060-1h}! Ich komme um \label{K_L03060-2v}\edtext{12 Uhr}{\lemma{\textnormal{\emph{12 Uhr}}}\Cendnote{\textnormal{gemeint war der darauffolgende Tag; es
                  ist unklar, ob sich \textcolor{blue}{Schnitzler} und \textcolor{blue}{Goldmann} am 3. 3. 1901 tatsächlich sahen – im \emph{\textcolor{green}{Tagebuch}} gibt es darauf keinerlei
                  Hinweise}}}\label{K_L03060-2h} ins \label{K_L03060-3v}\edtext{\textcolor{pink}{Hotel}{}\ledrightnote{{$\rightarrow$}\textcolor{pink}{[Hotel]}}}{\lemma{\textnormal{\emph{Hotel}}}\Cendnote{\textnormal{nicht ermittelt}}}\label{K_L03060-3h}.\pend
           
\pstart
           Viele Grüße!\pend
           
\pstart
           Dein {\\[\baselineskip]}\spacefill\mbox{P. G.}\pend
           \leftskip=0em{}\endnumbering\briefempfaengerindex{Schnitzler, Arthur@\textsc{Schnitzler, Arthur}!zzzGoldmann, Paul@\emph{von Paul Goldmann}!1901-03-021@{2. 3. {[}1901{]}}|)be}\mylabel{h}
\begin{anhang}
\end{anhang}\normalsize

\doendnotes{C}
\bigskip
\vfill

\clearpage

\footnotesize

\lohead{\textsc{register}}

% Definiere theindex-Environment komplett neu ohne reledmac
\makeatletter
\renewenvironment{theindex}{%
  \section*{\indexname}%
  \setlength{\parindent}{0pt}%
  \setlength{\parskip}{0pt plus 0.3pt}%
  \let\item\@idxitem
}{%
  \clearpage
}
\makeatother

\IfFileExists{\jobname-pw.ind}{\input{\jobname-pw.ind}}{}

\end{document}

      