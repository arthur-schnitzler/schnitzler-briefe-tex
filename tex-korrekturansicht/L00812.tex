%% latex-korrekturansicht-vorspann.tex
%% Vorspann für die Korrekturansicht.
%% Lädt die gemeinsame Datei latex-vorspann.tex mit gesetztem Schalter.

\newif\ifkorrekturansicht
\korrekturansichttrue

\input{../tex-inputs/latex-vorspann}


               \section[Richard Beer-Hofmann an Arthur Schnitzler, 4. 7. 1898]{ Richard Beer-Hofmann an Arthur Schnitzler, 4. 7. 1898}\nopagebreak\mylabel{v}\rehead{ }\normalsize\beginnumbering\briefempfaengerindex{Schnitzler, Arthur@\textsc{Schnitzler, Arthur}!zzzBeer-Hofmann, Richard@\emph{von Richard Beer-Hofmann}!1898-07-041@{4. 7. 1898}|(be} \toendnotes[C]{\smallbreak\pagebreak[2]} \Standort{CUL, Schnitzler, B 8.}
\physDesc{Telegramm
\newline{}maschinell\newline{}Versand: Stempel des Telegrammbeamten \textcolor{blue}{Nikorowicz} }\buchAbdrucke{\weitereDrucke{Arthur Schnitzler, Richard Beer-Hofmann: \emph{Briefwechsel 1891–1931}. Hg. Konstanze Fliedl. Wien, Zürich: \emph{Europaverlag} 1992, S. 122.} }\toendnotes[C]{\smallbreak}\pstart
           {\pb}\textcolor{pink}{win}{}\ledrightnote{\textcolor{pink}{Wien}} fr \textcolor{pink}{steindorfosziachersee}{}\ledrightnote{\textcolor{pink}{Steindorf am Ossiacher See}} 5 22 4/7{ }3 45n = \pend
           \pstart
           schicken sye mir bitte sofort \label{K_L00812_1v}\edtext{nummero
                  sechzehn}{\lemma{\textnormal{\emph{nummero
                  sechzehn}}}\Cendnote{\textnormal{Diese erschien am
                     1. 7. 1898.}}}\label{K_L00812_1h} der \textcolor{brown}{wiener
                  rundschau}{}\ledrightnote{\textcolor{brown}{Wiener Rundschau}} von \label{K_L00812_2v}\edtext{ersten juny}{\lemma{\textnormal{\emph{ersten juny}}}\Cendnote{\textnormal{Wie aus dem Brief vom 3. 7. 1898 hervorgeht, meinte er den
                     1. 7. 1898.}}}\label{K_L00812_2h} bryef unterwegs herzlychst
                  \spacefill\mbox{rychard .+}\pend
           \endnumbering\briefempfaengerindex{Schnitzler, Arthur@\textsc{Schnitzler, Arthur}!zzzBeer-Hofmann, Richard@\emph{von Richard Beer-Hofmann}!1898-07-041@{4. 7. 1898}|)be}\mylabel{h}  \normalsize

\doendnotes{C}
\bigskip
\vfill

\clearpage

\footnotesize

\lohead{\textsc{register}}

% Definiere theindex-Environment komplett neu ohne reledmac
\makeatletter
\renewenvironment{theindex}{%
  \section*{\indexname}%
  \setlength{\parindent}{0pt}%
  \setlength{\parskip}{0pt plus 0.3pt}%
  \let\item\@idxitem
}{%
  \clearpage
}
\makeatother

\IfFileExists{\jobname-pw.ind}{\input{\jobname-pw.ind}}{}

\end{document}

      