%% latex-korrekturansicht-vorspann.tex
%% Vorspann für die Korrekturansicht.
%% Lädt die gemeinsame Datei latex-vorspann.tex mit gesetztem Schalter.

\newif\ifkorrekturansicht
\korrekturansichttrue

\input{../tex-inputs/latex-vorspann}


\section[Arthur Schnitzler an Stefan Zweig, 30. 7. 1923]{L03751 Arthur Schnitzler an Stefan Zweig, 30. 7. 1923}
\nopagebreak\mylabel{L03751v}
\rehead{ }\normalsize\beginnumbering\briefempfaengerindex{, @\textsc{, }!zzz, @\emph{von  }!1923-07-301@{30. 7. 1923}|(be}
\toendnotes[C]{\smallbreak\pagebreak[2]}\Standort{Jerusalem, National Library of Israel, ARC. Ms. Var. 305 1 58 Stefan Zweig Collection.}
\physDesc{Postkarte, 459 Zeichen
\newline{}Handschrift: Bleistift, lateinische Kurrent
\newline{}Versand: Stempel: »\nobreak{}\oindex{XVIII., Währing@\textbf{XVIII., Währing}, \emph{Verwaltungsgebiet}|pwk}{[}18/\textsubscript{1}
                                          Wien{]} 110, \textcolor{gray}{30. VII.} 23, 12\nobreak{}«.  }\toendnotes[C]{\smallbreak}\pstart{}{\pb}\label{T_L03751-1v}\edtext{\textcolor{gray}{\textbf{A. S.}}}{\lemma{\textnormal{\emph{A. S.}}}\Cendnote{\textnormal{ovaler Absenderkleber}}}\label{T_L03751-1}\pend{}\pstart{}\textcolor{pink}{\textcolor{gray}{\textbf{WIEN, XVIII.}}}\oindex{XVIII., Währing@\textbf{XVIII., Währing}, \emph{Verwaltungsgebiet}|pw}{}\ledrightnote{\textcolor{pink}{XVIII., Währing}}\pend{}\pstart{}\textcolor{pink}{\textcolor{gray}{\textbf{STERNWARTESTR. 71}}}\oindex{Wien@\textbf{Wien}!XVIII., Währing@\textbf{XVIII., Währing}!Sternwartestraße 71@\textbf{Sternwartestraße 71}, \emph{Wohngebäude}|pw}{}\ledrightnote{\textcolor{pink}{Sternwartestraße 71}}\pend{}{\bigskip}\pstart{}Hrn.\pend{}\pstart{}Dr. Stefan Zweig \pend{}\pstart{}\textcolor{pink}{Salzburg}\oindex{Salzburg@\textbf{Salzburg}, \emph{Verwaltungsgebiet}|pw}{}\ledrightnote{\textcolor{pink}{Salzburg}}\pend{}\pstart{}\textcolor{pink}{Kapuzinerberg 5}\oindex{Paschinger Schlössl@\textbf{Paschinger Schlössl}, \emph{Wohngebäude}|pw}{}\ledrightnote{\textcolor{pink}{Paschinger Schlössl}}.\pend{}{\bigskip}\vspace{1em}
\pstart
           \raggedleft{}{\pb}\textcolor{pink}{Wien}\oindex{Wien@\textbf{Wien}, \emph{Verwaltungsgebiet}|pw}{}\ledrightnote{\textcolor{pink}{Wien}}, 30. 7. 23\pend
           
\pstart{}lieber Herr Doctor\pend\vspace{0.5em}
\pstart
           we{\geminationn} Sie also so freundlich sein wollen mir ein Zimmer (möglichst oben, und \textcolor{pink}{Salzach}\oindex{Salzach [Stadtgebiet Salzburg]@\textbf{Salzach [Stadtgebiet Salzburg]}, \emph{Fluss}|pw}{}\ledrightnote{\textcolor{pink}{Salzach [Stadtgebiet Salzburg]}} – aber das ist alles nicht wichtig) im
                  \textcolor{pink}{Oest. Hof}\oindex{Österreichischer Hof@\textbf{Österreichischer Hof}, \emph{Hotel}|pw}{}\ledrightnote{\textcolor{pink}{Österreichischer Hof}}{ }\strikeout{zu} reserviren zu lassen, u. zw.
                  \textcolor{gray}{f}ür Freitag{ }Nachm (5 Uhr glaub ich), so wär ich Ihnen sehr
               dankbar. Sollten meine Dispositionen noch eine Aenderung erfahren, so verständige ich
               Sie rechtzeitig. Ich freu mit sehr.\pend
           
\pstart
           Herzlichst mit vielen Grüßen{\\[\baselineskip]}Ihr \spacefill\mbox{Arthur Schnitzler}\pend
           \leftskip=0em{}\selectlanguage{ngerman}\endnumbering\briefempfaengerindex{, @\textsc{, }!zzz, @\emph{von  }!1923-07-301@{30. 7. 1923}|)be}\mylabel{L03751h}  \normalsize

\doendnotes{C}
\bigskip
\vfill

\clearpage

\footnotesize

\lohead{\textsc{register}}

% Definiere theindex-Environment komplett neu ohne reledmac
\makeatletter
\renewenvironment{theindex}{%
  \section*{\indexname}%
  \setlength{\parindent}{0pt}%
  \setlength{\parskip}{0pt plus 0.3pt}%
  \let\item\@idxitem
}{%
  \clearpage
}
\makeatother

\IfFileExists{\jobname-pw.ind}{\input{\jobname-pw.ind}}{}

\end{document}

      