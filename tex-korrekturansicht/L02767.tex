%% latex-korrekturansicht-vorspann.tex
%% Vorspann für die Korrekturansicht.
%% Lädt die gemeinsame Datei latex-vorspann.tex mit gesetztem Schalter.

\newif\ifkorrekturansicht
\korrekturansichttrue

\input{../tex-inputs/latex-vorspann}


               \section[Paul Goldmann an Arthur Schnitzler, 6. 2. {[}1896{]}]{ Paul Goldmann an Arthur Schnitzler, 6. 2. {[}1896{]}}\nopagebreak\mylabel{v}\rehead{ }\normalsize\beginnumbering\briefempfaengerindex{Schnitzler, Arthur@\textsc{Schnitzler, Arthur}!zzzGoldmann, Paul@\emph{von Paul Goldmann}!1896-02-061@{6. 2. {[}1896{]}}|(be} \toendnotes[C]{\smallbreak\pagebreak[2]} \Standort{DLA, A:Schnitzler, HS.NZ85.1.3166.}
\physDesc{Brief, 2 Blätter, 7 Seiten
\newline{}Handschrift: blaue Tinte, deutsche Kurrent
\newline{}Schnitzler: 1) mit Bleistift das Jahr »96« vermerkt 2) mit rotem Buntstift eine Unterstreichung}\toendnotes[C]{\smallbreak}\pstart
           \noindent{}{\pb}\textcolor{gray}{\textbf{\textbf{\textcolor{brown}{Frankfurter Zeitung}{}\ledrightnote{\textcolor{brown}{Frankfurter Zeitung}}}}}\pend
           \pstart
           \textcolor{gray}{\textbf{(\textcolor{brown}{\begin{otherlanguage}{french}Gazette de Francfort\end{otherlanguage}}{}\ledrightnote{\textcolor{brown}{Frankfurter Zeitung}}).}}\pend
           \pstart
           \textcolor{gray}{\textbf{\textbf{\begin{otherlanguage}{french}Fondateur M.\end{otherlanguage}{ }\textcolor{blue}{L. Sonnemann}{}\ledrightnote{\textcolor{blue}{Leopold Sonnemann}}.}}}\pend
           \pstart
           \begin{otherlanguage}{french}\textcolor{gray}{\textbf{\textcolor{green}{Journal}{}\ledrightnote{→\textcolor{green}{Frankfurter Zeitung}} politique,
                        financier,}}\end{otherlanguage}\pend
           \pstart
           \begin{otherlanguage}{french}\textcolor{gray}{\textbf{commercial et littéraire.}}\end{otherlanguage}\pend
           \pstart
           \begin{otherlanguage}{french}\textcolor{gray}{\textbf{\textbf{Paraissant trois fois par jour.}}}\end{otherlanguage}\pend
           \pstart
           \begin{otherlanguage}{french}\textcolor{gray}{\textbf{\textbf{Bureau à \textcolor{pink}{Paris}{}\ledrightnote{\textcolor{pink}{Paris}}}}}\end{otherlanguage}\hfill \textsc{\textcolor{pink}{Paris}{}\ledrightnote{\textcolor{pink}{Paris}}}, 6. Februar.\pend
           \pstart
           \begin{otherlanguage}{french}\textcolor{gray}{\textbf{\textbf{\textcolor{pink}{24. Rue Feydeau}{}\ledrightnote{\textcolor{pink}{rue Feydeau}}.}}}\end{otherlanguage}\pend
           \pstart\center{}Mein lieber Freund,\pend\pstart
           Ich ſchreibe Dir nicht nach \textsc{\textcolor{pink}{Berlin}{}\ledrightnote{\textcolor{pink}{Berlin}}}, weil ich nicht weiß, ob mein Brief Dich noch dort erreicht.\pend
           \pstart
           Alſo nochmals: innigen Glückwünſch! Nun biſt Du ganz und gar ein gemachter Mann.
               Selbſt dem ſkeptiſchen und kalten \textcolor{pink}{Berlin}{}\ledrightnote{\textcolor{pink}{Berlin}} haſt Du
               gefallen. Jetzt wird das \textcolor{green}{Stück}{}\ledrightnote{→\textcolor{green}{Liebelei. Schauspiel in drei Akten}}
               durch ganz \textcolor{pink}{Deutſchland}{}\ledrightnote{\textcolor{pink}{Deutschland}} gehen\strikeout{,} und Du biſt heut, in Deinen jungen Jahren, einer der
               erſten deutſchen Bühnendichter. {\pb}Das war zwar Alles
               vorauszuſehen; aber es iſt doch herrlich, daß man es \strikeout{e\textcolor{gray}{r}} erleb\strikeout{\textcolor{gray}{e}}t. Mach’ Dir keine Sorge über die Zukunft. Dein Talent wird ſich immer ſtärker
               und ſchöner entwickeln. Aber ich ſetze den, wie Du ſelbſt zugeben
                  wirſt{[},{]} etwas unwahrſcheinlichen, Fall, daß \strikeout{Du} Du fortan nur mehr lauter Stücke \textsc{à la}{ }\textsc{\textcolor{blue}{Rudolf Lothar}{}\ledrightnote{\textcolor{blue}{Rudolf Lothar}}} zuſtande bringſt, ſo würde ſelbſt das nichts machen. Du haſt bereits ein Werk
               geſchaffen, das {\pb}bleiben wird, und ſelbſt wenn Du
               gar nichts mehr ſchriebeſt, hätteſt Du Deinen Platz in der deutſchen Literatur
               geſichert. Ich meine alſo, Du kannſt ganz ruhig ſein\strikeout{,}
               und kannſt die Zweifel zum Teufel jagen, wenn ſie kommen. Es war ſehr lieb von Dir,
               mir noch kurz vor der \textsc{\begin{otherlanguage}{french}Première\end{otherlanguage}} zu ſchreiben. Deine \textcolor{pink}{Berlin}{}\ledrightnote{\textcolor{pink}{Berlin}}er \label{K_L02767-88v}\edtext{Perſonal-Eindrücke halte ich nicht für
               ganz zutreffend. \textsc{\textcolor{blue}{Harden}{}\ledrightnote{\textcolor{blue}{Maximilian Harden}}}}{\lemma{\textnormal{\emph{Perſonal-Eindrücke … Harden}}}\Cendnote{\textnormal{vgl. A. S.: \emph{Tagebuch}, 4. 2. 1896}}}\label{K_L02767-88h} mag \strikeout{eine beſ\textcolor{gray}{tr}} ein beſtrickender Menſch ſein aber ein »Freier« {\pb}iſt er nicht, ſondern ein Streber ohne Moral und
               Gewiſſen. Freilich ein großes Talent. Aber vielleicht muß man ſo ſein? Vielleicht iſt
               es Kraft, wenn man ſo iſt? Die Schwachen, die hinten bleiben, kommen dann mit der
               Moral, und das iſt vielleicht ſehr albern.\pend
           \pstart
           Ich habe geſtern, mit Deiner \label{K_L02767-77v}\edtext{Depeſche}{\lemma{\textnormal{\emph{Depeſche}}}\Cendnote{\textnormal{gemeint wohl ein Eilbrief, in dem ein positiver Bericht der \textcolor{pink}{Berliner} Premiere enthalten war}}}\label{K_L02767-77h} in der Hand, einen
               Schritt beim »\textsc{\textcolor{brown}{Figaro}{}\ledrightnote{\textcolor{brown}{Le Figaro}}}« gethan, den ich mir für \strikeout{\textcolor{gray}{einen}} den entſcheidenden Moment {\pb}aufgeſpart hatte.
               Da iſt es nämlich unendlich ſchwer, \strikeout{mit} eine Notiz
               anzubringen, weil die \textcolor{brown}{Leute}{}\ledrightnote{→\textcolor{brown}{Le Figaro}} das
               Bewußtſein ihrer ungeheuren Publicität haben und gewohnt ſind, daß man es ihnen
               zahlt. Nichtsdeſtoweniger iſt es mir gelungen, ein paar \label{K_L02767-1v}\edtext{\textcolor{green}{Zeilen}{}\ledrightnote{→\textcolor{green}{Courrier des Théatres [Liebelei-Premiere Berlin]}}}{\lemma{\textnormal{\emph{Zeilen}}}\Cendnote{\textnormal{o. V. [=\textcolor{blue}{Jules Huret}]: \emph{\textcolor{green}{Courrier des Théatres}}. In: \emph{\textcolor{green}{Le Figaro}}, Jg. 42, Nr. 37, 6. 2. 1896, S. 4: »\begin{otherlanguage}{french}De \textcolor{pink}{Berlin} ›\textcolor{brown}{Le Deutsches Theater}‹ vient de jouer avec
                        in grand succès la comédie \emph{\textcolor{green}{Liebelei}} (\textcolor{green}{Le Badinage amoureux}) de M. \textcolor{blue}{Arthur Schnitzler}, un jeune auteur \textcolor{pink}{viennois}. La comédie, qui raconte, en
                        trois actes tantôt gais, tantôt dramatiques, les amours d’une petite
                        grisette \textcolor{pink}{viennoise} avec un jeune homme du
                        monde, qui vit et meurt pour une autre, a été représentée au \textcolor{brown}{Burgtheater} de \textcolor{pink}{Vienne} au commencement de cette saison et y tient l’affiche depuis.
                        Le public \textcolor{pink}{berlin}ois, qui vient de
                        ratifier le jugement de celui de \textcolor{pink}{Vienne},
                        a fait un accueil chaleureux à l’auteur de \emph{\textcolor{green}{Liebelei}}. La critique \textcolor{pink}{berlin}oise apprécie
                        également la pièce en termes fort élogieux.\end{otherlanguage}« (»Das Deutsche Theater in Berlin hat soeben mit großem Erfolg die
                     Komödie \textcolor{green}{Liebelei} des jungen Wiener Autors
                     Arthur Schnitzler aufgeführt. Die Komödie, die in drei teils heiteren, teils
                     dramatischen Akten von der Liebe eines kleinen Wiener Mädchens zu einem jungen
                     Mann von Welt erzählt, der für eine andere lebt und stirbt, wurde zu Beginn
                     dieser Spielzeit im Wiener Burgtheater aufgeführt und steht seitdem dort auf
                     dem Spielplan. Das Berliner Publikum, das gerade das Urteil des Wiener
                     Publikums bestätigt hat, hat dem Autor der \textcolor{green}{Liebelei} einen herzlichen Empfang bereitet. Auch die Berliner Kritiker
                     bewerteten das Stück sehr lobend.«)}}}\label{K_L02767-1h} über Dich hineinzubringen,
               und das hat für die \textcolor{pink}{Pariſ}{}\ledrightnote{\textcolor{pink}{Paris}}er Aufführungs-Projecte
               den größten Werth. {\pb}Bitte, nimm eine Karte,
               adreſſire ſie an \textsc{\begin{otherlanguage}{french}M. \textcolor{blue}{Jules Huret}{}\ledrightnote{\textcolor{blue}{Jules Huret}} du
                        »\textcolor{brown}{Figaro}{}\ledrightnote{\textcolor{brown}{Le Figaro}}«, \textcolor{pink}{Rue Drouot, Paris}{}\ledrightnote{\textcolor{pink}{Rue Drouot}}\end{otherlanguage}} und ſchreibe darauf etwas wie: \label{K_L02767-8v}\edtext{\begin{otherlanguage}{french}\textsc{remercie bien vivement M. \textcolor{blue}{Huret}{}\ledrightnote{\textcolor{blue}{Jules Huret}} de la \strikeout{\textcolor{gray}{×}}{ }\textcolor{green}{note}{}\ledrightnote{→\textcolor{green}{Courrier des Théatres [Liebelei-Premiere Berlin]}}, qu’il a eu
                     l’amabilité d’insérer au sujet de la représentation de »\textcolor{green}{Liebelei}{}\ledrightnote{\textcolor{green}{Liebelei. Schauspiel in drei Akten}}« à \textcolor{pink}{Berlin}{}\ledrightnote{\textcolor{pink}{Berlin}}.}\end{otherlanguage}}{\lemma{\textnormal{\emph{remercie … Berlin.}}}\Cendnote{\textnormal{französisch: [Arthur Schnitzler]
                     dankt Herrn \textcolor{blue}{Huret} herzlich für die \textcolor{green}{Notiz}, die er
                     freundlicherweise über die Aufführung der \textcolor{green}{Liebelei} in \textcolor{pink}{Berlin} eingerückt
                     hat.}}}\label{K_L02767-8h} Anbei erhältſt Du den »\textcolor{green}{Figaro}{}\ledrightnote{\textcolor{green}{Le Figaro}}«
               (Theater-Rubrik). Ich bin ſehr {\pb}ſtolz auf meinen
               franzöſiſchen Styl.\pend
           \pstart
           Grüß’ Dich Gott, mein lieber Freund!\pend
           \pstart
           In Treue {\\[\baselineskip]}Dein {\\[\baselineskip]}\spacefill\mbox{Paul Goldmann.}\pend
           \leftskip=0em{}\endnumbering\briefempfaengerindex{Schnitzler, Arthur@\textsc{Schnitzler, Arthur}!zzzGoldmann, Paul@\emph{von Paul Goldmann}!1896-02-061@{6. 2. {[}1896{]}}|)be}\mylabel{h}\begin{anhang}\end{anhang}\normalsize

\doendnotes{C}
\bigskip
\vfill

\clearpage

\footnotesize

\lohead{\textsc{register}}

% Definiere theindex-Environment komplett neu ohne reledmac
\makeatletter
\renewenvironment{theindex}{%
  \section*{\indexname}%
  \setlength{\parindent}{0pt}%
  \setlength{\parskip}{0pt plus 0.3pt}%
  \let\item\@idxitem
}{%
  \clearpage
}
\makeatother

\IfFileExists{\jobname-pw.ind}{\input{\jobname-pw.ind}}{}

\end{document}

      