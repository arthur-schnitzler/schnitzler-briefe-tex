%% latex-korrekturansicht-vorspann.tex
%% Vorspann für die Korrekturansicht.
%% Lädt die gemeinsame Datei latex-vorspann.tex mit gesetztem Schalter.

\newif\ifkorrekturansicht
\korrekturansichttrue

\input{../tex-inputs/latex-vorspann}


               \section[Paul Goldmann an Arthur Schnitzler, {[}5.? 2. 1896{]}]{ Paul Goldmann an Arthur Schnitzler, {[}5.? 2. 1896{]}}\nopagebreak\mylabel{v}\rehead{ }\normalsize\beginnumbering\briefempfaengerindex{Schnitzler, Arthur@\textsc{Schnitzler, Arthur}!zzzGoldmann, Paul@\emph{von Paul Goldmann}!1896-02-052@{{[}5.? 2. 1896{]}}|(be} \toendnotes[C]{\smallbreak\pagebreak[2]} \Standort{DLA, A:Schnitzler, HS.NZ85.1.3166.}
\physDesc{Telegramm1 Blatt, 1 Seite
\newline{}maschinell
\newline{}Schnitzler: mit Bleistift datiert: »Feber 99« \newline{}Ordnung: beschnitten }\toendnotes[C]{\smallbreak}\pstart
           \noindent{}{\pb}\label{K_L02688-1v}\edtext{glueckwunsch}{\lemma{\textnormal{\emph{glueckwunsch}}}\Cendnote{\textnormal{Wohl zur Premiere von \emph{\textcolor{green}{Liebelei}} (gemeinsam mit \emph{\textcolor{green}{Der zerbrochene
                     Krug}}) am 4. 2. 1896 im \emph{\textcolor{brown}{Deutschen
                     Theater}} in \textcolor{pink}{Berlin}. \textcolor{blue}{Schnitzler} war anwesend, weswegen dieses Telegramm nach \textcolor{pink}{Berlin} gerichtet gewesen sein dürfte.}}}\label{K_L02688-1h}
               von ganzem herzen. gruss. \spacefill\mbox{goldmann =«}\pend
           \endnumbering\briefempfaengerindex{Schnitzler, Arthur@\textsc{Schnitzler, Arthur}!zzzGoldmann, Paul@\emph{von Paul Goldmann}!1896-02-052@{{[}5.? 2. 1896{]}}|)be}\mylabel{h}\begin{anhang}\end{anhang}\normalsize

\doendnotes{C}
\bigskip
\vfill

\clearpage

\footnotesize

\lohead{\textsc{register}}

% Definiere theindex-Environment komplett neu ohne reledmac
\makeatletter
\renewenvironment{theindex}{%
  \section*{\indexname}%
  \setlength{\parindent}{0pt}%
  \setlength{\parskip}{0pt plus 0.3pt}%
  \let\item\@idxitem
}{%
  \clearpage
}
\makeatother

\IfFileExists{\jobname-pw.ind}{\input{\jobname-pw.ind}}{}

\end{document}

      