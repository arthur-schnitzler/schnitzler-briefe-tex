%% latex-korrekturansicht-vorspann.tex
%% Vorspann für die Korrekturansicht.
%% Lädt die gemeinsame Datei latex-vorspann.tex mit gesetztem Schalter.

\newif\ifkorrekturansicht
\korrekturansichttrue

\input{../tex-inputs/latex-vorspann}


\renewcommand{\erwaehntePersonen}{Personen: Lili Cappellini, Anna Katharina Rehmann, Felix Salten, Ottilie Salten, Paul Salten, Heinrich Schnitzler, Olga Schnitzler}
\renewcommand{\erwaehnteOrte}{Orte: Bregenz, Gütsch, Hotel de l’Europe, Innsbruck, Kaltenleutgeben, München, Salzburg, Wien}
\renewcommand{\erwaehnteWerke}{}
\section[ Felix Salten an Arthur Schnitzler, 23. 8. 1909]{Felix Salten an Arthur Schnitzler, 23. 8. 1909}
\nopagebreak\mylabel{v}
\rehead{ }\normalsize\beginnumbering\briefempfaengerindex{Schnitzler, Arthur@\textsc{Schnitzler, Arthur}!zzzSalten, Felix@\emph{von Felix Salten}!1909-08-231@{23. 8. 1909}|(be}
\toendnotes[C]{\smallbreak\pagebreak[2]}\Standort{CUL, Schnitzler, B 89, B 1.}
\physDesc{Briefkarte, 1294 Zeichen
\newline{}Handschrift: schwarze Tinte, lateinische Kurrent
\newline{}Schnitzler: mit Bleistift Vermerk: »\textsc{Salten}« 
\newline{}Ordnung: mit Bleistift von unbekannter Hand nummeriert: »256« }\toendnotes[C]{\smallbreak}
\pstart
           \raggedleft{}{\pb}\textcolor{pink}{Kaltenleutgeben}{}\ledrightnote{\textcolor{pink}{Kaltenleutgeben}}, 23. VIII. 09\pend
           
\pstart
           Lieber,{ }morgen gehe ich nun nach \textcolor{pink}{Wien}{}\ledrightnote{\textcolor{pink}{Wien}} und Mittwoch{ }Abend nach \textcolor{pink}{Innsbruck}{}\ledrightnote{\textcolor{pink}{Innsbruck}}. Am 30. u. 31. werde ich in
                  \textcolor{pink}{Bregenz}{}\ledrightnote{\textcolor{pink}{Bregenz}} sein. Ich weiß nicht mehr, wer mir
               gesagt hat, Sie hätten die Absicht, \label{K_L03506-1v}\edtext{nach \textcolor{pink}{Bregenz}{}\ledrightnote{\textcolor{pink}{Bregenz}} zu kommen}{\lemma{\textnormal{\emph{nach Bregenz zu kommen}}}\Cendnote{\textnormal{Die vorliegende Karte dürfte nach \textcolor{pink}{Wien} adressiert
                  gewesen sein und \textcolor{blue}{Schnitzler} verpasst haben. Er reiste
                  am Abend des A. S.: \emph{Tagebuch}, 23. 8. 1909 nach \textcolor{pink}{München}
                  und blieb (mit einer kurzen Unterbrechung) bis zum Abend des A. S.: \emph{Tagebuch}, 1. 9. 1909.}}}\label{K_L03506-1h}. Ist das richtig? Ich wohne \textcolor{pink}{Hotel Europe}{}\ledrightnote{\textcolor{pink}{Hotel de l’Europe}}. Am 1. Sept. will ich für \textcolor{gray}{2} Tage nach \textcolor{pink}{Luzern}{}\ledrightnote{\textcolor{pink}{Gütsch}}. Träfe ich Sie \label{K_L03506-2v}\edtext{am 4. od. 5. in \textcolor{pink}{Salzburg}{}\ledrightnote{\textcolor{pink}{Salzburg}}}{\lemma{\textnormal{\emph{am 4. od. 5. in Salzburg}}}\Cendnote{\textnormal{nicht geschehen}}}\label{K_L03506-2h}? Geben Sie mir
               vielleicht nach \textcolor{pink}{Bregenz}{}\ledrightnote{\textcolor{pink}{Bregenz}} Nachricht, falls Sie
               nicht selbst hinkommen, was mich natürlich sehr freuen würde. Von dieser Reise gehe
               ich nicht mehr hierher zurück. \textcolor{blue}{Otti}{}\ledrightnote{\textcolor{blue}{Ottilie Salten}}
               übersiedelt heute in acht Tagen mit den \textcolor{blue}{Kinder}{}\ledrightnote{{$\rightarrow$}\textcolor{blue}{Paul Salten}{\newline}{$\rightarrow$}\textcolor{blue}{Anna Katharina Rehmann}}n nach \textcolor{pink}{Wien}{}\ledrightnote{\textcolor{pink}{Wien}}. Ab
                  6. bin ich da, und freue mich aufs \label{K_L03506-3v}\edtext{Tennis, das wir dann gleich wieder
                  aufnehmen}{\lemma{\textnormal{\emph{Tennis, … aufnehmen}}}\Cendnote{\textnormal{siehe A. S.: \emph{Tagebuch}, 6. 9. 1909}}}\label{K_L03506-3h} wollen. Dass \textcolor{blue}{Heini}{}\ledrightnote{\textcolor{blue}{Heinrich Schnitzler}}’s \label{K_L03506-4v}\edtext{\textcolor{blue}{Schwesterl}{}\ledrightnote{{$\rightarrow$}\textcolor{blue}{Lili Cappellini}} so bald bevor
                  steht}{\lemma{\textnormal{\emph{Schwesterl … steht}}}\Cendnote{\textnormal{\textcolor{blue}{Lili Schnitzler} wurde am 13. 9. 1909
                  geboren. Warum \textcolor{blue}{Salten} sicher scheint, dass es ein Mädchen werden sollte, ist unklar.}}}\label{K_L03506-4h}, wußte ich nicht. Aber – je eher, je besser! (Vorausgesetzt,
               u. s. w.) Wir senden Ihrer \textcolor{blue}{Frau}{}\ledrightnote{{$\rightarrow$}\textcolor{blue}{Olga Schnitzler}} viele herzliche Grüße und wünschen ihr von Herzen, dass alles \uline{sehr} gut und sehr leicht sein möge! Grüßen Sie auch
               den lieben \textcolor{blue}{Heini}{}\ledrightnote{\textcolor{blue}{Heinrich Schnitzler}} von uns allen. Bald wird man
               Ihnen auch schreiben müßen: »Grüßen Sie Ihre Kinder!« Eigentlich kann mans ja schon
               heute. Also: Grüßen Sie Ihre \textcolor{blue}{Kinder}{}\ledrightnote{{$\rightarrow$}\textcolor{blue}{Heinrich Schnitzler}{\newline}{$\rightarrow$}\textcolor{blue}{Lili Cappellini}}. – Frau \textcolor{blue}{Olga}{}\ledrightnote{\textcolor{blue}{Olga Schnitzler}}
               hat \textcolor{blue}{Annerl}{}\ledrightnote{\textcolor{blue}{Anna Katharina Rehmann}} einen entzückenden Brief
               geschrieben, der ihr großen Eindruck macht. Sie will sich selbst {\pb}bedanken, und wird nächstens
               einen Brief diktiren.\pend
           
\pstart
           Auf Wiedersehen in \textcolor{pink}{Salzburg}{}\ledrightnote{\textcolor{pink}{Salzburg}} – \textcolor{pink}{Bregenz}{}\ledrightnote{\textcolor{pink}{Bregenz}} oder \textcolor{pink}{Wien}{}\ledrightnote{\textcolor{pink}{Wien}}. Jedenfalls bald. {\\[\baselineskip]}herzlichst {\\[\baselineskip]}I\textcolor{gray}{h}r {\\[\baselineskip]}\spacefill\mbox{Salten}\pend
           \leftskip=0em{}\endnumbering\briefempfaengerindex{Schnitzler, Arthur@\textsc{Schnitzler, Arthur}!zzzSalten, Felix@\emph{von Felix Salten}!1909-08-231@{23. 8. 1909}|)be}\mylabel{h}  \normalsize

\doendnotes{C}
\bigskip
\vfill

\clearpage

\footnotesize

\lohead{\textsc{register}}

% Definiere theindex-Environment komplett neu ohne reledmac
\makeatletter
\renewenvironment{theindex}{%
  \section*{\indexname}%
  \setlength{\parindent}{0pt}%
  \setlength{\parskip}{0pt plus 0.3pt}%
  \let\item\@idxitem
}{%
  \clearpage
}
\makeatother

\IfFileExists{\jobname-pw.ind}{\input{\jobname-pw.ind}}{}

\end{document}

      