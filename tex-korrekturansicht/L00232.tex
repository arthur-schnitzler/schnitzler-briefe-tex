%% latex-korrekturansicht-vorspann.tex
%% Vorspann für die Korrekturansicht.
%% Lädt die gemeinsame Datei latex-vorspann.tex mit gesetztem Schalter.

\newif\ifkorrekturansicht
\korrekturansichttrue

\input{../tex-inputs/latex-vorspann}


               \section[Arthur Schnitzler an Richard Beer-Hofmann, {[}zwischen 3. und 15. 7. 1893?{]}]{ Arthur Schnitzler an Richard Beer-Hofmann, {[}zwischen 3. und
               15. 7. 1893?{]}}\nopagebreak\mylabel{v}\rehead{ }\normalsize\beginnumbering\briefempfaengerindex{Beer-Hofmann, Richard@\textsc{Beer-Hofmann, Richard}!zzzSchnitzler, Arthur@\emph{von Arthur Schnitzler}!1893-07-032@{{[}zwischen 3. und
                  15. 7. 1893?{]}}|(be} \toendnotes[C]{\smallbreak\pagebreak[2]} \Standort{YCGL, MSS 31.}
\physDesc{Brief, 1 Blatt, 2 Seiten, Umschlag
\newline{}Handschrift: Bleistift, deutsche Kurrent\newline{}Versand: ohne postalischen Übermittlungsvermerk }\toendnotes[C]{\smallbreak}\pstart{}{\pb}Herrn \textsc{Dr. Richard
                     Beer-Hofmann}\pend{}\pstart{}\textsc{bei Hr. \textcolor{blue}{Johann Strauss}{}\ledrightnote{\textcolor{blue}{Johann Strauss}}}\pend{}\pstart{}\textcolor{pink}{\textsc{Villa Erdödy}}{}\ledrightnote{\textcolor{pink}{Villa Erdödy}}.\pend{}{\bigskip}\pstart
           \noindent{}{\pb}Lieber Richard, – ich bleibe Nachmittags zu Hauſe. Ko{\geminationm}en Sie einfach direct von \label{K_L00232_1v}\edtext{\textcolor{blue}{\textsc{Str.}’s}{}\ledrightnote{\textcolor{blue}{Johann Strauss}}}{\lemma{\textnormal{\emph{Str.’s}}}\Cendnote{\textnormal{Das Korrespondenzstück ist undatiert.
                  Einzig für den Aufenthalt \textcolor{blue}{Schnitzler}s von
                     2.–15. 7. 1893 lassen sich Begegnungen mit \textcolor{blue}{Johann Strauss} im seinem \emph{\textcolor{green}{Tagebuch}} ausmachen. Ob \textcolor{blue}{Beer-Hofmann}s Kontakt in der gleichen Zeit stattfand oder länger bestand,
                  ist nicht zu klären.}}}\label{K_L00232_1h} zu mir {\pb}herüber.\pend
           \pstart Herzlich grüßt Ihr \spacefill\mbox{Arthur}\pend{}\endnumbering\briefempfaengerindex{Beer-Hofmann, Richard@\textsc{Beer-Hofmann, Richard}!zzzSchnitzler, Arthur@\emph{von Arthur Schnitzler}!1893-07-032@{{[}zwischen 3. und
                  15. 7. 1893?{]}}|)be}\mylabel{h}  \normalsize

\doendnotes{C}
\bigskip
\vfill

\clearpage

\footnotesize

\lohead{\textsc{register}}

% Definiere theindex-Environment komplett neu ohne reledmac
\makeatletter
\renewenvironment{theindex}{%
  \section*{\indexname}%
  \setlength{\parindent}{0pt}%
  \setlength{\parskip}{0pt plus 0.3pt}%
  \let\item\@idxitem
}{%
  \clearpage
}
\makeatother

\IfFileExists{\jobname-pw.ind}{\input{\jobname-pw.ind}}{}

\end{document}

      