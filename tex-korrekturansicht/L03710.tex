%% latex-korrekturansicht-vorspann.tex
%% Vorspann für die Korrekturansicht.
%% Lädt die gemeinsame Datei latex-vorspann.tex mit gesetztem Schalter.

\newif\ifkorrekturansicht
\korrekturansichttrue

\input{../tex-inputs/latex-vorspann}


\section[Elsa Plessner an Arthur Schnitzler, 29. 12. 1896]{L03710 Elsa Plessner an Arthur Schnitzler, 29. 12. 1896}
\nopagebreak\mylabel{L03710v}
\rehead{ }\normalsize\beginnumbering\briefempfaengerindex{Schnitzler, Arthur@\textsc{Schnitzler, Arthur}!zzzPlessner, Elsa@\emph{von Elsa Plessner}!1896-12-291@{29. 12. 1896}|(be}
\toendnotes[C]{\smallbreak\pagebreak[2]}
\correspDesc{Versand  durch Elsa Plessner am 29. 12. 1896 in Meran
\newline{}Erhalt  durch Arthur Schnitzler im Zeitraum [30. 12. 1896 – 3. 1. 1897?] in Wien}\toendnotes[C]{\smallbreak}
\Standort{DLA, A:Schnitzler, HS.1985.1.419.}
\physDesc{Brief, 1 Blatt, 3 Seiten, 2206 Zeichen
\newline{}Handschrift: schwarze Tinte, lateinische Kurrent
\newline{}Schnitzler: mit rotem Buntstift eine Unterstreichung }\toendnotes[C]{\smallbreak}
\pstart
           {\pb}\textcolor{pink}{Meran, Pension Wolf}\oindex{Hotel Meranerhof@\textbf{Hotel Meranerhof}, \emph{Hotel}|pw}{}\ledrightnote{\textcolor{pink}{Hotel Meranerhof}}, den
                     29. 12. 96.\pend
           
\pstart\center{}Hochverehrter Herr Doctor!\pend\vspace{0.5em}
\pstart
           
               Anbei »\label{K_L03710-1v}\edtext{\textcolor{green}{Orchideen}\pwindex{Plessner, Elsa 22.\,8.\,1875 Wien – 7.\,5.\,1932 Alicante@\textsc{Plessner, Elsa} (22.\,8.\,1875 Wien – 7.\,5.\,1932 Alicante), \emph{Schriftstellerin}!Orchideen [Schauspiel in drei Akten]@\strich\emph{Orchideen [Schauspiel in drei Akten]}|pw}{}\ledrightnote{\textcolor{green}{Orchideen [Schauspiel in drei Akten]}}}{\lemma{\textnormal{\emph{Orchideen}}}\Cendnote{\textnormal{Das Werk ist nicht überliefert.}}}\label{K_L03710-1}«. Erschrecken Sie, bitte, nicht über die
               Dampfgeschwindigkeit, mit der ich Sie überfalle. Nämlich ich dachte so: »Ist das \textcolor{green}{Stück}\pwindex{Plessner, Elsa 22.\,8.\,1875 Wien – 7.\,5.\,1932 Alicante@\textsc{Plessner, Elsa} (22.\,8.\,1875 Wien – 7.\,5.\,1932 Alicante), \emph{Schriftstellerin}!Orchideen [Schauspiel in drei Akten]@\strich\emph{Orchideen [Schauspiel in drei Akten]}|pwv}{}\ledrightnote{{$\rightarrow$}\emph{\textcolor{green}{Orchideen [Schauspiel in drei Akten]}}} in der Anlage verhauen,
               so nützt keine »Feile« was, ist es aber gut, so können Sie sich die Feile {[}»{]}hinzudenken«.
               Also nehme ich keinen Anstand, es Ihnen noch in einem noch wenig verfeinerten, ersten
               Justzustand zu übersenden mit der Bitte um \uline{strenges
                  Gericht}. das Sie vielleicht durch Roth oder Blaustift in den Text hinein
               bemerkbar machen {\pb}zu wollen, so gut sind!! – Erschrecken Sie, bitte
               nicht, wenn Sie den Lieutenant sehen — kein Bösewicht x-ter Auflage – . Die mit
               Bleistift notirte Rollenbesetzung  ist natürlich nur dazu da, Sie ein bisschen im
               vorhinein über die Figuren zu orientiren – – ! – Die Grundidee meines \textcolor{green}{Stückes}\pwindex{Plessner, Elsa 22.\,8.\,1875 Wien – 7.\,5.\,1932 Alicante@\textsc{Plessner, Elsa} (22.\,8.\,1875 Wien – 7.\,5.\,1932 Alicante), \emph{Schriftstellerin}!Orchideen [Schauspiel in drei Akten]@\strich\emph{Orchideen [Schauspiel in drei Akten]}|pw}{}\ledrightnote{\textcolor{green}{Orchideen [Schauspiel in drei Akten]}} ist \introOben{}mir\introOben{} eigentlich gekommen
               durch die \textcolor{blue}{Töchter}\pwindex{Tullia Major @\textsc{Tullia Major}, \emph{Prinzessin}|pwv}\pwindex{Tullia Minor @\textsc{Tullia Minor}, \emph{Prinzessin}|pwv}{}\ledrightnote{{$\rightarrow$}\emph{\textcolor{blue}{Tullia Major}}{\newline}{$\rightarrow$}\emph{\textcolor{blue}{Tullia Minor}}} des \textcolor{blue}{Servius Tullus}\pwindex{Servius Tullius @\textsc{Servius Tullius}, \emph{König}|pw}{}\ledrightnote{\textcolor{blue}{Servius Tullius}} – und das
               sage ich Ihnen nur, weil ich nicht will, daß Sie an etwas \uline{anderes} denken, was Sie auch im Beginn gewiss thun werden. – Aber Sie werden
               ja sehen, wie verschieden es nachher wird!! – Über dem ganzen \textcolor{green}{Stück}\pwindex{Plessner, Elsa 22.\,8.\,1875 Wien – 7.\,5.\,1932 Alicante@\textsc{Plessner, Elsa} (22.\,8.\,1875 Wien – 7.\,5.\,1932 Alicante), \emph{Schriftstellerin}!Orchideen [Schauspiel in drei Akten]@\strich\emph{Orchideen [Schauspiel in drei Akten]}|pwv}{}\ledrightnote{{$\rightarrow$}\emph{\textcolor{green}{Orchideen [Schauspiel in drei Akten]}}} schwebt – als unausgesprochenes
               »Sesam« \uline{ein Wort}, das ich jedoch \uline{nirgends} gebraucht habe! – Ich glaube, es wird auch {\pb}Ihnen
               auf die Lippen treten. – Zum Schluß bitte ich Sie noch um Entschuldigung, wegen der
               mangelhaften äußeren Form des \textcolor{green}{Manuscriptes}\pwindex{Plessner, Elsa 22.\,8.\,1875 Wien – 7.\,5.\,1932 Alicante@\textsc{Plessner, Elsa} (22.\,8.\,1875 Wien – 7.\,5.\,1932 Alicante), \emph{Schriftstellerin}!Orchideen [Schauspiel in drei Akten]@\strich\emph{Orchideen [Schauspiel in drei Akten]}|pwv}{}\ledrightnote{{$\rightarrow$}\emph{\textcolor{green}{Orchideen [Schauspiel in drei Akten]}}} – war in der Schnelligkeit nicht anders möglich – und Geduld
               habe ich keine mehr! – So, jetzt wissen Sie alles, was ich auf dem Herzen habe – (d.
               h. diesbezüglich) und somit empfehle ich die »\textcolor{green}{Orchideen}\pwindex{Plessner, Elsa 22.\,8.\,1875 Wien – 7.\,5.\,1932 Alicante@\textsc{Plessner, Elsa} (22.\,8.\,1875 Wien – 7.\,5.\,1932 Alicante), \emph{Schriftstellerin}!Orchideen [Schauspiel in drei Akten]@\strich\emph{Orchideen [Schauspiel in drei Akten]}|pw}{}\ledrightnote{\textcolor{green}{Orchideen [Schauspiel in drei Akten]}}« allen neun Musen und Ihrer Huld – – bitte! – bitte ! – bitte!!!!
               – lassen Sie mich nicht zu lange zappeln – aus Gesundheitsrücksichten für mich und
               meine »Nerven«, die sich in einem pitoyablen Zustand befinden!! – wirklich! – Ich
               gebe Ihnen die notariell beglaubigte Versicherung, daß ich bis zum Eintreffen Ihrer
               Meinungsabgabe keine geruhsame Nacht mehr erleben werde – und ob das recht viele sein
               werden, hängt von Ihrer Güte ab!! – – Die Sonne scheint jetzt wieder 25 Celsiusgrädig
               auf meinen Schreibtisch – d. h. spazieren gehen – also – – schließt mit
               hochachtungsvoller Ergebenheit und herzlichen Grüßen von der Frau Sonne und besten
               von mir\pend
           \pstart \spacefill\mbox{Elsa Plessner}\pend{}\selectlanguage{ngerman}\endnumbering\briefempfaengerindex{Schnitzler, Arthur@\textsc{Schnitzler, Arthur}!zzzPlessner, Elsa@\emph{von Elsa Plessner}!1896-12-291@{29. 12. 1896}|)be}\mylabel{L03710h}  \normalsize

\doendnotes{C}
\bigskip
\vfill

\clearpage

\footnotesize

\lohead{\textsc{register}}

% Definiere theindex-Environment komplett neu ohne reledmac
\makeatletter
\renewenvironment{theindex}{%
  \section*{\indexname}%
  \setlength{\parindent}{0pt}%
  \setlength{\parskip}{0pt plus 0.3pt}%
  \let\item\@idxitem
}{%
  \clearpage
}
\makeatother

\IfFileExists{\jobname-pw.ind}{\input{\jobname-pw.ind}}{}

\end{document}

      