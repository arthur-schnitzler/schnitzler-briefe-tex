%% latex-korrekturansicht-vorspann.tex
%% Vorspann für die Korrekturansicht.
%% Lädt die gemeinsame Datei latex-vorspann.tex mit gesetztem Schalter.

\newif\ifkorrekturansicht
\korrekturansichttrue

\input{../tex-inputs/latex-vorspann}


\renewcommand{\erwaehntePersonen}{Personen: Max Bernstein, Otto Brahm, Gerhart Hauptmann, Hugo von Hofmannsthal, Alfred Kerr, Olga Schnitzler, Heinrich Schnitzler, Elisabeth Steinrück, Hermann Sudermann}
\renewcommand{\erwaehnteInstitutionen}{Institutionen: Schiller-Theater}
\renewcommand{\erwaehnteOrte}{Orte: Berlin, Dessauer Straße, Deutsches Theater Berlin, Wien}
\renewcommand{\erwaehnteWerke}{Werke: Berliner Tageblatt, Der Schleier der Beatrice. Schauspiel in fünf Akten, Der Tag, Die neue Richtung von Paul Goldman. Wien 1903. Verlag L. Rosner, D’Mali. Schauspiel in vier Akten, Ein Brief, Neues Wiener Tagblatt, Verrohung in der Theaterkritik, Verrohung in der Theaterkritik [Teil I], [Feuilleton von Alfred Kerr]}
\section[ Paul Goldmann an Arthur Schnitzler, 10. 11. {[}1902{]}]{Paul Goldmann an Arthur Schnitzler, 10. 11. {[}1902{]}}
\nopagebreak\mylabel{v}
\rehead{ }\normalsize\beginnumbering\briefempfaengerindex{Schnitzler, Arthur@\textsc{Schnitzler, Arthur}!zzzGoldmann, Paul@\emph{von Paul Goldmann}!1902-11-101@{10. 11. {[}1902{]}}|(be}
\toendnotes[C]{\smallbreak\pagebreak[2]}\Standort{DLA, A:Schnitzler, HS.NZ85.1.3172.}
\physDesc{Brief, 1 Blatt, 4 Seiten
\newline{}Handschrift: blaue Tinte, deutsche Kurrent
\newline{}Schnitzler: 1) mit Bleistift das Jahr »{[}1{]}902« vermerkt  2) mit rotem Buntstift sechs Unterstreichungen}\toendnotes[C]{\smallbreak}
\pstart
           \noindent{}\raggedleft{}{\pb}\textcolor{pink}{\textcolor{gray}{\textbf{DESSAUERSTRASSE 19}}}{}\ledrightnote{\textcolor{pink}{Dessauer Straße}}\pend
           
\pstart
           \textcolor{pink}{Berlin}{}\ledrightnote{\textcolor{pink}{Berlin}}, 10. November.\pend
           
\pstart\center{}Mein lieber Freund,\pend
\pstart
           Ich habe fürchterlich viel zu thun u. komme erſt heut
               dazu, Dir vielmals für den \label{K_L03229-1v}\edtext{\textcolor{green}{Ausſchnitt}{}\ledrightnote{{$\rightarrow$}\textcolor{green}{Die neue Richtung von Paul Goldman. Wien 1903. Verlag L. Rosner}}}{\lemma{\textnormal{\emph{Ausſchnitt}}}\Cendnote{\textnormal{[O. V.:] \emph{\textcolor{green}{Die neue Richtung von Paul
                        Goldman. Wien 1903. Verlag L. Rosner}}. In: \emph{\textcolor{green}{Neues Wiener Tagblatt}}, Jg. 36, Nr. 301, 1. 11. 1902, S. 35.}}}\label{K_L03229-1h} aus dem \textcolor{green}{N. W. T.}{}\ledrightnote{\textcolor{green}{Neues Wiener Tagblatt}} und Deinen lieben Brief zu danken.\pend
           
\pstart
           Die guten Nachrichten von \textsc{\textcolor{blue}{Olga}{}\ledrightnote{\textcolor{blue}{Olga Schnitzler}}} und Deinem \textcolor{blue}{Sohne}{}\ledrightnote{{$\rightarrow$}\textcolor{blue}{Heinrich Schnitzler}} haben
               mich ſehr erfreut. Grüße ſie alle Beide recht herzlich. Wie denkt \textsc{\textcolor{blue}{Heinrich Schnitzler}{}\ledrightnote{\textcolor{blue}{Heinrich Schnitzler}}} über \textsc{\textcolor{blue}{Gerhart Hauptmann}{}\ledrightnote{\textcolor{blue}{Gerhart Hauptmann}}}?\pend
           
\pstart
           Mit \textsc{\textcolor{blue}{Brahm}{}\ledrightnote{\textcolor{blue}{Otto Brahm}}} wirſt Du wohl {\pb}inzwiſchen \label{K_L03229-2v}\edtext{einig}{\lemma{\textnormal{\emph{einig}}}\Cendnote{\textnormal{Bezug auf die Aufführung von \emph{\textcolor{green}{Der Schleier der Beatrice}} am \textcolor{pink}{Deutschen
                     Theater Berlin}}}}\label{K_L03229-2h} geworden ſein. Er hat ſich in der letzten \label{K_L03229-3v}\edtext{Cenſur-Affaire}{\lemma{\textnormal{\emph{Cenſur-Affaire}}}\Cendnote{\textnormal{rund um \textcolor{blue}{Max Bernstein}s vieraktiges
                  Schauspiel \emph{\textcolor{green}{D’Mali}} wenige Tage zuvor}}}\label{K_L03229-3h}
               recht \textcolor{gray}{n}ämlich und ſympathiſch benommen.\pend
           
\pstart
           \textsc{\textcolor{blue}{Sudermann}{}\ledrightnote{\textcolor{blue}{Hermann Sudermann}}} miſcht in ſeinen \label{K_L03229-4v}\edtext{\textcolor{green}{Artikel}{}\ledrightnote{{$\rightarrow$}\textcolor{green}{Verrohung in der Theaterkritik [Teil I]}}}{\lemma{\textnormal{\emph{Artikel}}}\Cendnote{\textnormal{Gemeint war der \textcolor{green}{erste Teil} der fünfteiligen, am 30. 10., 7. 11., 17. 11., 25. 11. und
                     1. 12. 1902 in Abendausgaben des \emph{\textcolor{green}{Berliner Tageblatt}}s erschienenen Feuilletonreihe \emph{\textcolor{green}{Verrohung in der Theaterkritik}}: \textcolor{blue}{Hermann Sudermann}: \emph{\textcolor{green}{Verrohung in der Theaterkritik}}. In: \emph{\textcolor{green}{Berliner Tageblatt und Handels-Zeitung}}, Jg. 31,
                     Nr. 553, 30. 10. 1902, Abend-Ausgabe,
                     S. 1–3.}}}\label{K_L03229-4h} Wahres mit Albernem. Was er über den Gebrauch des Wortes
                  \label{K_L03229-123v}\edtext{»unliterariſch«}{\lemma{\textnormal{\emph{»unliterariſch«}}}\Cendnote{\textnormal{vgl. \textcolor{green}{ebd.}, S. 2}}}\label{K_L03229-123h} ſagte, war ſehr richtig. Auch die
                  \label{K_L03229-5v}\edtext{\textsc{\begin{otherlanguage}{french}gaminerie\end{otherlanguage}}}{\lemma{\textnormal{\emph{gaminerie}}}\Cendnote{\textnormal{französisch: Kinderei}}}\label{K_L03229-5h} unſeres
               Freundes \textsc{\textcolor{blue}{Kerr}{}\ledrightnote{\textcolor{blue}{Alfred Kerr}}}, die er in ſeinem letzten \label{K_L03229-6v}\edtext{\textcolor{green}{Feuilleton}{}\ledrightnote{{$\rightarrow$}\textcolor{green}{[Feuilleton von Alfred Kerr]}}}{\lemma{\textnormal{\emph{Feuilleton}}}\Cendnote{\textnormal{XXXX}}}\label{K_L03229-6h} anführt, war recht garſtig.
               Vieles aber ließe ſich leicht widerlegen.\pend
           
\pstart
           {\pb}Haſt Du den \label{K_L03229-7v}\edtext{»\textcolor{green}{Brief}{}\ledrightnote{\textcolor{green}{Ein Brief}}«}{\lemma{\textnormal{\emph{»Brief«}}}\Cendnote{\textnormal{\textcolor{blue}{Hugo von Hofmannsthal}: \emph{\textcolor{green}{Ein Brief}}. In: \emph{\textcolor{green}{Der Tag.
                        Erster Teil: Illustrierte Zeitung}}, Nr. 489, 18. 10. 1902, S. [1–3] und Nr. 491, 19. 10. 1902, S. [1–3]. Eine Lektüre durch \textcolor{blue}{Schnitzler} ist nicht nachweisbar.}}}\label{K_L03229-7h} von \textsc{\textcolor{blue}{Hoffmannsthal}{}\ledrightnote{\textcolor{blue}{Hugo von Hofmannsthal}}} geleſen, der vor einigen Wochen im »\textcolor{green}{Tag}{}\ledrightnote{\textcolor{green}{Der Tag}}«
               erſchienen iſt?\pend
           
\pstart
           Geſtern{ }Nachmittag kam ich endlich dazu, \textsc{\textcolor{blue}{Liesl}{}\ledrightnote{\textcolor{blue}{Elisabeth Steinrück}}} in ihrem \textsc{Boudoir} zu beſuchen. Sie wohnt recht
               ärmlich, das arme Ding, – aber ſie iſt vergnügt und ſpielt ſogar ſchon größere
                  \label{K_L03229-11v}\edtext{Rollen}{\lemma{\textnormal{\emph{Rollen}}}\Cendnote{\textnormal{am \emph{\textcolor{brown}{Schiller-Theater}}, wo
                     \textcolor{blue}{Elisabeth Gussmann} seit 1. 9. 1902 unter Vertrag stand}}}\label{K_L03229-11h}.\pend
           
\pstart
           Ich bin wieder einmal durch Verſchiedenes (Schlafloſigkeit, nervöſe Störungen) ſehr
                  {\pb}niedergedrückt. Daher für heut nur dieſe wenigen Zeilen.\pend
           
\pstart
           Laß’ bald von Dir hören und ſei vielmals und herzlichſt gegrüßt von {\\[\baselineskip]}Deinem {\\[\baselineskip]}\spacefill\mbox{Paul Goldm}\pend
           \leftskip=0em{}\endnumbering\briefempfaengerindex{Schnitzler, Arthur@\textsc{Schnitzler, Arthur}!zzzGoldmann, Paul@\emph{von Paul Goldmann}!1902-11-101@{10. 11. {[}1902{]}}|)be}\mylabel{h}
\begin{anhang}
\end{anhang}\normalsize

\doendnotes{C}
\bigskip
\vfill

\clearpage

\footnotesize

\lohead{\textsc{register}}

% Definiere theindex-Environment komplett neu ohne reledmac
\makeatletter
\renewenvironment{theindex}{%
  \section*{\indexname}%
  \setlength{\parindent}{0pt}%
  \setlength{\parskip}{0pt plus 0.3pt}%
  \let\item\@idxitem
}{%
  \clearpage
}
\makeatother

\IfFileExists{\jobname-pw.ind}{\input{\jobname-pw.ind}}{}

\end{document}

      