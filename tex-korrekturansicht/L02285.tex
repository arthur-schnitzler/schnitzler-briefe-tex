%% latex-korrekturansicht-vorspann.tex
%% Vorspann für die Korrekturansicht.
%% Lädt die gemeinsame Datei latex-vorspann.tex mit gesetztem Schalter.

\newif\ifkorrekturansicht
\korrekturansichttrue

\input{../tex-inputs/latex-vorspann}


               \section[Felix Braun an Arthur Schnitzler, 21. 4. 1918]{ Felix Braun an Arthur Schnitzler, 21. 4. 1918}\nopagebreak\mylabel{v}\rehead{ }\normalsize\beginnumbering\briefempfaengerindex{Schnitzler, Arthur@\textsc{Schnitzler, Arthur}!zzzBraun, Felix@\emph{von Felix Braun}!1918-04-211@{21. 4. 1918}|(be} \toendnotes[C]{\smallbreak\pagebreak[2]} \Standort{DLA, A:Schnitzler, HS.NZ85.1.2604,1.}
\physDesc{Brief, 1 Blatt, 4 Seiten
\newline{}Handschrift: schwarze Tinte, deutsche Kurrent
\newline{}Schnitzler: 1) auf der ersten Seite mit Bleistift beschriftet: »\textsc{Braun}« 2) mit rotem Buntstift eine Unterstreichung}\toendnotes[C]{\smallbreak}\pstart
           \noindent{}\centering{}{\pb}\textcolor{gray}{\textbf{\textcolor{brown}{GEORG MÜLLER VERLAG}{}\ledrightnote{\textcolor{brown}{Georg Müller}}, \textcolor{pink}{MÜNCHEN}{}\ledrightnote{\textcolor{pink}{München}}{ }UND{ }\textcolor{pink}{BERLIN}{}\ledrightnote{\textcolor{pink}{Berlin}}}}\pend
           \pstart
           \noindent{}\textcolor{gray}{\textbf{TELEPHON 32043 ⋅ GIROKONTO BEI DER \textcolor{brown}{ALLG. ELSÄSSISCHEN BANKGESELLSCHAFT}{}\ledrightnote{\textcolor{brown}{Allgemeine elsässische Bankgesellschaft}}, FILIALE \textcolor{pink}{MAINZ}{}\ledrightnote{\textcolor{pink}{Mainz}}}}\pend
           \pstart
           \raggedleft{}\textcolor{gray}{\textbf{\textcolor{pink}{MÜNCHEN}{}\ledrightnote{\textcolor{pink}{München}}, DEN}}{ }21. IV. \textcolor{gray}{\textbf{191}}8\pend
           \pstart
           \raggedleft{}\textcolor{gray}{\textbf{\textcolor{pink}{ELISABETHSTRASSE 26}{}\ledrightnote{\textcolor{pink}{Elisabethstraße}}}}\pend
           \pstart{}Verehrter Herr Doktor!\pend\pstart
           Ich erhielt heute Ihren Expreß-Brief und habe ſogleich mit dem Chef des Verlags,
                    Herrn Dr. \textcolor{blue}{Kauffmann}{}\ledrightnote{\textcolor{blue}{Arthur I. Kauffmann}}, geſprochen, in deſſen
                    Auftrag ich das folgende mitteilen kann:\pend
           \pstart
           Der \textcolor{brown}{Verlag}{}\ledrightnote{→\textcolor{brown}{Georg Müller}} würde die \textcolor{green}{Novelle}{}\ledrightnote{→\textcolor{green}{Casanovas Heimfahrt}} ſofort drucken und
                    zwar in einer Auflage von 8–10.000 Exemplaren; wenn Papier vorhanden ſein
                    ſollte, eventuell mehr. Was den Prozentſatz anbelangt, ſo möchte man ſich erſt
                    nach einer genauen Kalkulation darüber ausſprechen, da noch niemals 25 {\%} gezahlt wurden. Mit der ſpäteren Aufnahme dieſer
                    Bücher in Ihre \textcolor{green}{Geſammelten Werke}{}\ledrightnote{\textcolor{green}{Gesammelte Werke}} iſt man
                    einverſtanden. Für das \textcolor{green}{Stück}{}\ledrightnote{→\textcolor{green}{Die Schwestern oder Casanova in Spa. Lustspiel in Versen}}
                    gilt das gleiche, nur würde man dieſes in einer geringeren Auflage drucken.\pend
           \pstart
           Daß man ſich hier außerordentlich freuen würde, wenn es gelänge, Ihre neuen
                    Bücher zum \textcolor{brown}{Verlag}{}\ledrightnote{→\textcolor{brown}{Georg Müller}} zu
                    bekommen, muß ich gewiß nicht erſt ſagen. Man iſt ſchon über die Möglichkeit
                    hoch erfreut. Hoffentlich realiſiert ſie ſich auch.\pend
           \pstart
           {\pb}Mir perſönlich erlauben Sie, verehrter Herr
                    Doktor, Ihnen zu ſagen, wie ſehr es mich erfreut hat, Sie an meinem letzten Tag
                    in \textcolor{pink}{Wien}{}\ledrightnote{\textcolor{pink}{Wien}} noch geſehen und geſprochen zu haben.
                    Dies ſchöne Abſchiedsfeſt bei Frau \textcolor{blue}{Waſſermann}{}\ledrightnote{\textcolor{blue}{Julie Wassermann}} hat mir den langgehegten Wunſch, einmal mit Ihnen zuſammen
                    zu treffen, erfüllt. Ich danke Ihnen herzlich, daß Sie gekommen ſind, und bitte
                    Sie, den Ausdruck aufrichtiger Verehrung anzunehmen von Ihrem ergebenen\pend
           \pstart \spacefill\mbox{Felix Braun}\pend{}\pstart
           \noindent{}P.S.{\\}Ihrer \textcolor{blue}{Frau
                            Gemahlin}{}\ledrightnote{→\textcolor{blue}{Olga Schnitzler}}, der ich mich beſtens empfehle, bitte ich zu ſagen, daß
                        ich das Paket beim Hotelportier (\textcolor{pink}{Schottenhamel}{}\ledrightnote{\textcolor{pink}{Hotel Schottenhamel}}) hinterlegt habe.\pend
           \endnumbering\briefempfaengerindex{Schnitzler, Arthur@\textsc{Schnitzler, Arthur}!zzzBraun, Felix@\emph{von Felix Braun}!1918-04-211@{21. 4. 1918}|)be}\mylabel{h}  \normalsize

\doendnotes{C}
\bigskip
\vfill

\clearpage

\footnotesize

\lohead{\textsc{register}}

% Definiere theindex-Environment komplett neu ohne reledmac
\makeatletter
\renewenvironment{theindex}{%
  \section*{\indexname}%
  \setlength{\parindent}{0pt}%
  \setlength{\parskip}{0pt plus 0.3pt}%
  \let\item\@idxitem
}{%
  \clearpage
}
\makeatother

\IfFileExists{\jobname-pw.ind}{\input{\jobname-pw.ind}}{}

\end{document}

      