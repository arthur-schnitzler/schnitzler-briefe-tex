%% latex-korrekturansicht-vorspann.tex
%% Vorspann für die Korrekturansicht.
%% Lädt die gemeinsame Datei latex-vorspann.tex mit gesetztem Schalter.

\newif\ifkorrekturansicht
\korrekturansichttrue

\input{../tex-inputs/latex-vorspann}


\renewcommand{\erwaehntePersonen}{Personen: Richard Beer-Hofmann, Marie Reinhard, Felix Salten}
\renewcommand{\erwaehnteOrte}{Orte: Afrika, Paris, Wien, rue de Maubeuge}
\renewcommand{\erwaehnteWerke}{}
\section[ Arthur Schnitzler an Felix Salten, 26. 4. 1897]{Arthur Schnitzler an Felix Salten, 26. 4. 1897}
\nopagebreak\mylabel{v}
\rehead{ }\normalsize\beginnumbering\briefempfaengerindex{Salten, Felix@\textsc{Salten, Felix}!zzzSchnitzler, Arthur@\emph{von Arthur Schnitzler}!1897-04-263@{26. 4. 1897}|(be}
\toendnotes[C]{\smallbreak\pagebreak[2]}\Standort{Wienbibliothek im Rathaus, ZPH 1681, 2.1.516.}
\physDesc{Brief, 1 Blatt, 3 Seiten, 632 Zeichen
\newline{}Handschrift: schwarze Tinte, deutsche Kurrent
\newline{}Ordnung: mit Bleistift von unbekannter Hand Nummerierung der Blätter des Konvoluts:
                                    »76«–»77« }
\buchAbdrucke{\weitereDrucke{Arthur Schnitzler: \emph{Briefe 1875–1912}. Hg. Therese Nickl und Heinrich Schnitzler. Frankfurt am Main: \emph{S. Fischer} 1981, S. 317.} }\toendnotes[C]{\smallbreak}
\pstart
           \noindent{}\raggedleft{}{\pb}\textcolor{pink}{5 rue de Maubeuge}{}\ledrightnote{\textcolor{pink}{rue de Maubeuge}}\pend
           
\pstart
           \raggedleft{}\textsc{\textcolor{pink}{Paris}{}\ledrightnote{\textcolor{pink}{Paris}}{ }26. 4. 97}. \pend
           
\pstart{}lieber Freund,\pend
\pstart
           \textcolor{blue}{Richard}{}\ledrightnote{\textcolor{blue}{Richard Beer-Hofmann}} ſchreibt mir, Sie ſind wenige Tage
               verreiſt? Wie? wo?–\pend
           
\pstart
           Ich hab mir \textcolor{pink}{hier}{}\ledrightnote{{$\rightarrow$}\textcolor{pink}{Paris}} mein Leben ſo
               gut als möglich eingerichtet und bin trotz \label{K_L02963-1v}\edtext{»Thür an Thür«}{\lemma{\textnormal{\emph{»Thür an Thür«}}}\Cendnote{\textnormal{\textcolor{blue}{Schnitzler} war seit 12. 4. 1897 und noch
                  bis 23. 5. 1897
                  gemeinsam mit \textcolor{blue}{Marie Reinhard} in \textcolor{pink}{Paris}, wo sie im selben Haus in der \textcolor{pink}{rue de Maubeuge} wohnten.}}}\label{K_L02963-1h} leidlich {\pb}ungeſtört. Auch hat es ſogar ſein angenehmes.
               Theater, jeden Abend – wie wird man fertig? – Muſeen – jeden Tag – wie wird man
               fertig? Wohne recht wohl, ſpeiſe nicht übel. – Arbeite nichts; bin aber ſehr
               aufnahmsfähig. – {\pb}Entbehre Pilſner u
               Virginier mit \textcolor{pink}{afrika}{}\ledrightnote{\textcolor{pink}{Afrika}}reiſender Leichtigkeit.
                  Ko{\geminationm}e mir vor wie einer, der Strapazen gewachſsen iſt.
               –\pend
           
\pstart
           Einzelheiten in \textcolor{pink}{Wien}{}\ledrightnote{\textcolor{pink}{Wien}}.\pend
           
\pstart
           Sagen Sie mir, wie es Ihnen geht, in jeder Beziehung. Herzlich {\\}Ihr
                  \spacefill\mbox{Arthur Sch}\pend
           \endnumbering\briefempfaengerindex{Salten, Felix@\textsc{Salten, Felix}!zzzSchnitzler, Arthur@\emph{von Arthur Schnitzler}!1897-04-263@{26. 4. 1897}|)be}\mylabel{h}  \normalsize

\doendnotes{C}
\bigskip
\vfill

\clearpage

\footnotesize

\lohead{\textsc{register}}

% Definiere theindex-Environment komplett neu ohne reledmac
\makeatletter
\renewenvironment{theindex}{%
  \section*{\indexname}%
  \setlength{\parindent}{0pt}%
  \setlength{\parskip}{0pt plus 0.3pt}%
  \let\item\@idxitem
}{%
  \clearpage
}
\makeatother

\IfFileExists{\jobname-pw.ind}{\input{\jobname-pw.ind}}{}

\end{document}

      