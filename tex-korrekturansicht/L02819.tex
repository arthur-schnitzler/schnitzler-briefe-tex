%% latex-korrekturansicht-vorspann.tex
%% Vorspann für die Korrekturansicht.
%% Lädt die gemeinsame Datei latex-vorspann.tex mit gesetztem Schalter.

\newif\ifkorrekturansicht
\korrekturansichttrue

\input{../tex-inputs/latex-vorspann}


               \section[ Paul Goldmann an Arthur Schnitzler, 19. 7. {[}1897{]}]{Paul Goldmann an Arthur Schnitzler, 19. 7. {[}1897{]}}\nopagebreak\mylabel{v}\rehead{ }\normalsize\beginnumbering\briefempfaengerindex{Schnitzler, Arthur@\textsc{Schnitzler, Arthur}!zzzGoldmann, Paul@\emph{von Paul Goldmann}!1897-07-192@{19. 7. {[}1897{]}}|(be} \toendnotes[C]{\smallbreak\pagebreak[2]} \Standort{DLA, A:Schnitzler, HS.NZ85.1.3167.}
\physDesc{Brief, 1 Blatt, 2 Seiten
\newline{}Handschrift: blaue Tinte, deutsche Kurrent
\newline{}Schnitzler: 1) mit Bleistift das Jahr »97« vermerkt 2) mit rotem Buntstift eine Unterstreichung}\toendnotes[C]{\smallbreak}\pstart
           \noindent{}{\pb}\textcolor{gray}{\textbf{\textbf{\textcolor{brown}{Frankfurter Zeitung}{}\ledrightnote{\textcolor{brown}{Frankfurter Zeitung}}}}}\pend
           \pstart
           \textcolor{gray}{\textbf{(\textcolor{brown}{\begin{otherlanguage}{french}Gazette de Francfort\end{otherlanguage}}{}\ledrightnote{\textcolor{brown}{Frankfurter Zeitung}}).}}\pend
           \pstart
           \textcolor{gray}{\textbf{\textbf{\begin{otherlanguage}{french}Fondateur M.\end{otherlanguage}{ }\textcolor{blue}{L. Sonnemann}{}\ledrightnote{\textcolor{blue}{Leopold Sonnemann}}.}}}\pend
           \pstart
           \begin{otherlanguage}{french}\textcolor{gray}{\textbf{Journal politique, financier,}}\end{otherlanguage}\pend
           \pstart
           \begin{otherlanguage}{french}\textcolor{gray}{\textbf{commercial et littéraire.}}\end{otherlanguage}\pend
           \pstart
           \begin{otherlanguage}{french}\textcolor{gray}{\textbf{\textbf{Paraissant trois fois par jour.}}}\end{otherlanguage}\pend
           \pstart
           \begin{otherlanguage}{french}\textcolor{gray}{\textbf{\textbf{Bureau à \textcolor{pink}{Paris}{}\ledrightnote{\textcolor{pink}{Paris}}}}}\end{otherlanguage}\hfill \textsc{\textcolor{pink}{Paris}{}\ledrightnote{\textcolor{pink}{Paris}}}, 19. Juli.\pend
           \pstart
           \begin{otherlanguage}{french}\textcolor{gray}{\textbf{\textbf{\textcolor{pink}{10 Rue de la Bourse}{}\ledrightnote{\textcolor{pink}{rue de la Bourse}}.}}}\end{otherlanguage}\pend
           \pstart\center{}Mein lieber Freund,\pend\pstart
           Ich bekomme eben \textsc{\textcolor{blue}{Richard}{}\ledrightnote{\textcolor{blue}{Richard Beer-Hofmann}}s} Brief. \textsc{\textcolor{blue}{Richard}{}\ledrightnote{\textcolor{blue}{Richard Beer-Hofmann}}} iſt ſehr ſtolz auf die Klarheit dieſes Briefes (und mit Recht). Der einzige
               Fehler dieſes ſonnenklaren Handſchreibens iſt der, daß der Abſender vergeſſen hat,
               mir eine Adreſſe zur Antwort anzugeben. Sei alſo ſo gütig als Mittelsmann zu dienen
               und ſage dem lieben \textsc{\textcolor{blue}{Richard}{}\ledrightnote{\textcolor{blue}{Richard Beer-Hofmann}}}: Er ſoll an irgend einem, noch näher zu beſtimmenden Tage um den 15. Auguſt herum \label{K_L02819-1v}\edtext{nach \textsc{\textcolor{pink}{Muenchen}{}\ledrightnote{\textcolor{pink}{München}}} kommen}{\lemma{\textnormal{\emph{nach Muenchen kommen}}}\Cendnote{\textnormal{siehe Paul Goldmann an Arthur Schnitzler, 13. 7. [1897]}}}\label{K_L02819-1h}. Dort bleiben wir {\pb}ein paar Tage und fahren
               dann zuſammen nach \textsc{\textcolor{pink}{Ischl}{}\ledrightnote{\textcolor{pink}{Bad Ischl}}}, wo wir Dich erwarten.\pend
           \pstart
           Iſts ſo Recht?\pend
           \pstart
           Von Herzen {\\[\baselineskip]}Dein {\\[\baselineskip]}\spacefill\mbox{Paul Goldmann}\pend
           \leftskip=0em{}\endnumbering\briefempfaengerindex{Schnitzler, Arthur@\textsc{Schnitzler, Arthur}!zzzGoldmann, Paul@\emph{von Paul Goldmann}!1897-07-192@{19. 7. {[}1897{]}}|)be}\mylabel{h}\begin{anhang}\end{anhang}\normalsize

\doendnotes{C}
\bigskip
\vfill

\clearpage

\footnotesize

\lohead{\textsc{register}}

% Definiere theindex-Environment komplett neu ohne reledmac
\makeatletter
\renewenvironment{theindex}{%
  \section*{\indexname}%
  \setlength{\parindent}{0pt}%
  \setlength{\parskip}{0pt plus 0.3pt}%
  \let\item\@idxitem
}{%
  \clearpage
}
\makeatother

\IfFileExists{\jobname-pw.ind}{\input{\jobname-pw.ind}}{}

\end{document}

      