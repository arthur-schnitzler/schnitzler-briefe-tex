%% latex-korrekturansicht-vorspann.tex
%% Vorspann für die Korrekturansicht.
%% Lädt die gemeinsame Datei latex-vorspann.tex mit gesetztem Schalter.

\newif\ifkorrekturansicht
\korrekturansichttrue

\input{../tex-inputs/latex-vorspann}


               \section[Richard Beer-Hofmann an Arthur Schnitzler, 30. 7. 1907]{ Richard Beer-Hofmann an Arthur Schnitzler, 30. 7. 1907}\nopagebreak\mylabel{v}\rehead{ }\normalsize\beginnumbering\briefempfaengerindex{Schnitzler, Arthur@\textsc{Schnitzler, Arthur}!zzzBeer-Hofmann, Richard@\emph{von Richard Beer-Hofmann}!1907-07-301@{30. 7. 1907}|(be} \toendnotes[C]{\smallbreak\pagebreak[2]} \Standort{CUL, Schnitzler, B 8.}
\physDesc{Brief, 1 Blatt (Briefpapier mit Trauerrand), 2 Seiten
\newline{}Handschrift: schwarze Tinte, lateinische Kurrent\newline{}Ordnung: mit Bleistift von unbekannter Hand nummeriert: »210« }\buchAbdrucke{\weitereDrucke{Arthur Schnitzler, Richard Beer-Hofmann: \emph{Briefwechsel 1891–1931}. Hg. Konstanze Fliedl. Wien, Zürich: \emph{Europaverlag} 1992, S. 182.} }\toendnotes[C]{\smallbreak}\pstart
           \raggedleft{}{\pb}\textcolor{pink}{Maria Schutz}{}\ledrightnote{\textcolor{pink}{Maria Schutz}}{ }30./VII 07.\pend
           \pstart
           Lieber Arthur! Zwischen 14. u. 19. August,
               wollen wir von \textcolor{pink}{Wien}{}\ledrightnote{\textcolor{pink}{Wien}} abreisen das ergiebt, mit der
               Woche \textcolor{pink}{Kärnten}{}\ledrightnote{\textcolor{pink}{Kärnten}}, ein passiren des \textcolor{pink}{Pustertales}{}\ledrightnote{\textcolor{pink}{Pustertal}} zwischen 23.–28. August.\pend
           \pstart
           Wir sind aber müde, verprügelt, keine übermässig heitere Gesellschaft, und ich glaube
               nur mit Vorsicht zu gebrauchen wenn wir nicht wider unsern Willen andere versti{\geminationm}en sollen.\pend
           \pstart
           {\pb}Freilich hoffe ich auf bessere
               Tage; wenn noch ein wenig Elastisches in uns ist, müssen wir wol nach so vieler
               Depression doch irgendeinmal wieder aufschnellen.\pend
           \pstart
           Einen Brief an \textcolor{blue}{Hugo}{}\ledrightnote{\textcolor{blue}{Hugo von Hofmannsthal}} habe ich dieser Tage nach \textcolor{pink}{Waldbrunn}{}\ledrightnote{\textcolor{pink}{Wildbad Waldbrunn}} geschickt; fragen Sie, bitte, gelegentlich
               nach, \label{K_L01697_1v}\edtext{ob er nachgeschickt}{\lemma{\textnormal{\emph{ob er nachgeschickt}}}\Cendnote{\textnormal{Er wurde es und ist im \emph{Briefwechsel Hofmannsthal/Beer-Hofmann} (S. 130)
                  abgedruckt.}}}\label{K_L01697_1h} wurde.\pend
           \pstart
           Sie verständigen mich von Ihren Reise- oder Abreiseplänen?\pend
           \pstart Herzlichst Ihr \spacefill\mbox{Richard}\pend{}\pstart
           An Frau \textcolor{blue}{Olga}{}\ledrightnote{\textcolor{blue}{Olga Schnitzler}} von uns Beiden herzliche Grüsse.\pend
           \endnumbering\briefempfaengerindex{Schnitzler, Arthur@\textsc{Schnitzler, Arthur}!zzzBeer-Hofmann, Richard@\emph{von Richard Beer-Hofmann}!1907-07-301@{30. 7. 1907}|)be}\mylabel{h}  \normalsize

\doendnotes{C}
\bigskip
\vfill

\clearpage

\footnotesize

\lohead{\textsc{register}}

% Definiere theindex-Environment komplett neu ohne reledmac
\makeatletter
\renewenvironment{theindex}{%
  \section*{\indexname}%
  \setlength{\parindent}{0pt}%
  \setlength{\parskip}{0pt plus 0.3pt}%
  \let\item\@idxitem
}{%
  \clearpage
}
\makeatother

\IfFileExists{\jobname-pw.ind}{\input{\jobname-pw.ind}}{}

\end{document}

      