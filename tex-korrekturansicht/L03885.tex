%% latex-korrekturansicht-vorspann.tex
%% Vorspann für die Korrekturansicht.
%% Lädt die gemeinsame Datei latex-vorspann.tex mit gesetztem Schalter.

\newif\ifkorrekturansicht
\korrekturansichttrue

\input{../tex-inputs/latex-vorspann}


\section[Arthur Schnitzler an Romain Rolland, 7. 1. 1915]{L03885 Arthur Schnitzler an Romain Rolland, 7. 1. 1915}
\nopagebreak\mylabel{L03885v}
\rehead{ }\normalsize\beginnumbering\briefempfaengerindex{, @\textsc{, }!zzz, @\emph{von  }!1915-01-071@{7. 1. 1915}|(be}
\toendnotes[C]{\smallbreak\pagebreak[2]}\Standort{DLA, A:Schnitzler, 85.1.1714.}
\physDesc{BriefDurchschlag, 1261 Zeichen
\newline{}Schreibmaschine}
\buchAbdrucke{\weitereDrucke{Arthur Schnitzler: \emph{Briefe 1913–1931}. Frankfurt am Main: \emph{S. Fischer} 1984, S. 69–70.} }\toendnotes[C]{\smallbreak}
\pstart
           \raggedleft{}{\pb}7. 1. 1915.\pend
           
\pstart{}Verehrter Herr Rolland.\pend\vspace{0.5em}
\pstart
           Das \textcolor{green}{\textcolor{green}{Journal de Genève}\pwindex{Schnitzler, Arthur 15. 5. 1862 Wien – 21. 10. 1931 ebd.@\textsc{Schnitzler, Arthur} (15. 5. 1862 Wien – 21. 10. 1931 ebd.), \emph{Schriftsteller, Mediziner}!Une protestation d’Arthur Schnitzler@\strich\emph{Une protestation d’Arthur Schnitzler}|pwv}\pwindex{Rolland, Romain 29.\,1.\,1866 Clamecy – 30.\,12.\,1944 Vézelay@\textsc{Rolland, Romain} (29.\,1.\,1866 Clamecy – 30.\,12.\,1944 Vézelay), \emph{Schriftsteller}!Une protestation d’Arthur Schnitzler@\strich\emph{Une protestation d’Arthur Schnitzler}|pwv}{}\ledrightnote{{$\rightarrow$}\emph{\textcolor{green}{Une protestation d’Arthur Schnitzler}}}}\pwindex{Journal de Genève@\emph{Journal de Genève}|pw}{}\ledrightnote{\textcolor{green}{Journal de Genève}} ist nicht an mich gelangt, während die \textcolor{green}{\textcolor{green}{Zürcher Zeitung}\pwindex{Schnitzler, Arthur 15. 5. 1862 Wien – 21. 10. 1931 ebd.@\textsc{Schnitzler, Arthur} (15. 5. 1862 Wien – 21. 10. 1931 ebd.), \emph{Schriftsteller, Mediziner}!Brief Artur Schnitzlers@\strich\emph{Ein Brief Artur Schnitzlers}|pwv}{}\ledrightnote{{$\rightarrow$}\emph{\textcolor{green}{Ein Brief Artur Schnitzlers}}}}\pwindex{Neue Zürcher Zeitung@\emph{Neue Zürcher Zeitung}|pw}{}\ledrightnote{\textcolor{green}{Neue Zürcher Zeitung}} gestern von der \textcolor{brown}{Redaktion}\orgindex{Neue Zürcher Zeitung@Neue Zürcher Zeitung|pwv}{}\ledrightnote{{$\rightarrow$}\emph{\textcolor{brown}{Neue Zürcher Zeitung}}} aus mit erheblicher Verspätung bei mir angekommen
               ist. Die Zensur entschliesst sich wahrscheinlich besonders schwer Zeitungen in
               französischer Sprache durchzulassen und so werde ich vorläufig darauf verzichten
               müssen, Ihre \textcolor{green}{Übersetzung}\pwindex{Schnitzler, Arthur 15. 5. 1862 Wien – 21. 10. 1931 ebd.@\textsc{Schnitzler, Arthur} (15. 5. 1862 Wien – 21. 10. 1931 ebd.), \emph{Schriftsteller, Mediziner}!Une protestation d’Arthur Schnitzler@\strich\emph{Une protestation d’Arthur Schnitzler}|pwv}\pwindex{Rolland, Romain 29.\,1.\,1866 Clamecy – 30.\,12.\,1944 Vézelay@\textsc{Rolland, Romain} (29.\,1.\,1866 Clamecy – 30.\,12.\,1944 Vézelay), \emph{Schriftsteller}!Une protestation d’Arthur Schnitzler@\strich\emph{Une protestation d’Arthur Schnitzler}|pwv}{}\ledrightnote{{$\rightarrow$}\emph{\textcolor{green}{Une protestation d’Arthur Schnitzler}}}
               meiner \textcolor{green}{Erklärung}\pwindex{Schnitzler, Arthur 15. 5. 1862 Wien – 21. 10. 1931 ebd.@\textsc{Schnitzler, Arthur} (15. 5. 1862 Wien – 21. 10. 1931 ebd.), \emph{Schriftsteller, Mediziner}!Brief Artur Schnitzlers@\strich\emph{Ein Brief Artur Schnitzlers}|pw}{}\ledrightnote{\textcolor{green}{Ein Brief Artur Schnitzlers}} zu lesen, wenn Sie vielleicht
               nicht doch noch einen Versuch machen wollen, mindestens den betreffenden Ausschnitt
               unter Couvert mir zuzuschicken. Die Zensur wird es hoffentlich als politisch
               gefahrlos erkennen, mir einen von mir selbst verfassten und von Romain Rolland
               übersetzten \textcolor{green}{Protest}\pwindex{Schnitzler, Arthur 15. 5. 1862 Wien – 21. 10. 1931 ebd.@\textsc{Schnitzler, Arthur} (15. 5. 1862 Wien – 21. 10. 1931 ebd.), \emph{Schriftsteller, Mediziner}!Une protestation d’Arthur Schnitzler@\strich\emph{Une protestation d’Arthur Schnitzler}|pwv}\pwindex{Rolland, Romain 29.\,1.\,1866 Clamecy – 30.\,12.\,1944 Vézelay@\textsc{Rolland, Romain} (29.\,1.\,1866 Clamecy – 30.\,12.\,1944 Vézelay), \emph{Schriftsteller}!Une protestation d’Arthur Schnitzler@\strich\emph{Une protestation d’Arthur Schnitzler}|pwv}{}\ledrightnote{{$\rightarrow$}\emph{\textcolor{green}{Une protestation d’Arthur Schnitzler}}} zur
               Lektüre frei zu geben.\pend
           
\pstart
           Lassen Sie mich Ihnen heute nochmals für Ihre freundliche Bemühung, sowie für Ihren
               letzten, so liebenswürdigen \label{K_L03885-1v}\edtext{Brief}{\lemma{\textnormal{\emph{Brief}}}\Cendnote{\textnormal{Romain Rolland an Arthur Schnitzler, 19. 12. 1913?}}}\label{K_L03885-1} herzlich danken. Immer wieder lesen wir in der
               letzten Zeit in Feldpostbriefen, dass die feindlichen Soldaten, die einander in den
               Schützengräben gegenüberliegen, in den Kampfpausen einander Höflichkeiten,
               Rücksichten, Gefälligkeiten, ja achtungsvoll-freundschaftliche Gesinnung erweisen;
               wie denken Sie, mein verehrter Herr Rolland, über die Einführung von Schützengräben
               für Journalisten und Diplomaten?\pend
           
\pstart
           Seien Sie herzlichst gegrüsst{\\[\baselineskip]}Ihr sehr ergebener\pend
           \leftskip=0em{}\selectlanguage{ngerman}\endnumbering\briefempfaengerindex{, @\textsc{, }!zzz, @\emph{von  }!1915-01-071@{7. 1. 1915}|)be}\mylabel{L03885h}
\begin{anhang}
\end{anhang}\normalsize

\doendnotes{C}
\bigskip
\vfill

\clearpage

\footnotesize

\lohead{\textsc{register}}

% Definiere theindex-Environment komplett neu ohne reledmac
\makeatletter
\renewenvironment{theindex}{%
  \section*{\indexname}%
  \setlength{\parindent}{0pt}%
  \setlength{\parskip}{0pt plus 0.3pt}%
  \let\item\@idxitem
}{%
  \clearpage
}
\makeatother

\IfFileExists{\jobname-pw.ind}{\input{\jobname-pw.ind}}{}

\end{document}

      