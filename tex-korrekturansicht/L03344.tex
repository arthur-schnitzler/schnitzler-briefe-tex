%% latex-korrekturansicht-vorspann.tex
%% Vorspann für die Korrekturansicht.
%% Lädt die gemeinsame Datei latex-vorspann.tex mit gesetztem Schalter.

\newif\ifkorrekturansicht
\korrekturansichttrue

\input{../tex-inputs/latex-vorspann}


\renewcommand{\erwaehntePersonen}{Personen: Max Eugen Burckhard, Karl Glossy, Heinrich Kanner, Richard Muther, Ferdinand von Saar, August Sauer, Isidor Singer}
\renewcommand{\erwaehnteInstitutionen}{Institutionen: Die Zeit}
\renewcommand{\erwaehnteOrte}{Orte: Wien, Wipplingerstraße}
\renewcommand{\erwaehnteWerke}{}
\section[ Felix Salten an Arthur Schnitzler, 19. 9. {[}1903{]}]{Felix Salten an Arthur Schnitzler, 19. 9. {[}1903{]}}
\nopagebreak\mylabel{v}
\rehead{ }\normalsize\beginnumbering\briefempfaengerindex{Schnitzler, Arthur@\textsc{Schnitzler, Arthur}!zzzSalten, Felix@\emph{von Felix Salten}!1903-09-191@{19. 9. {[}1903{]}}|(be}
\toendnotes[C]{\smallbreak\pagebreak[2]}\Standort{CUL, Schnitzler, B 89, A 2.}
\physDesc{Brief, 1 Blatt, 2 Seiten, 612 Zeichen
\newline{}Handschrift: Bleistift, lateinische Kurrent
\newline{}Ordnung: mit Bleistift von unbekannter Hand nummeriert: »169« }\toendnotes[C]{\smallbreak}
\pstart
           \noindent{}{\pb}\textcolor{gray}{\textbf{DIE}}\pend
           
\pstart
           \textcolor{gray}{\textbf{\textcolor{brown}{ZEIT}{}\ledrightnote{\textcolor{brown}{Die Zeit}}}}\hfill \textcolor{gray}{\textbf{\textcolor{pink}{\emph{WIEN}}{}\ledrightnote{\textcolor{pink}{Wien}}}}{ }19/9.\pend
           
\pstart
           \textcolor{gray}{\textbf{\textcolor{pink}{Wien}{}\ledrightnote{\textcolor{pink}{Wien}}er Tageszeitung}}\hfill \textcolor{gray}{\textbf{\emph{\textcolor{pink}{I. Wipplingerstrasse 38}{}\ledrightnote{\textcolor{pink}{Wipplingerstraße}}}}}\pend
           
\pstart
           \textcolor{gray}{\textbf{Herausgeber:}}\pend
           
\pstart
           \textcolor{gray}{\textbf{\textbf{Prof. Dr. \textcolor{blue}{I. Singer}{}\ledrightnote{\textcolor{blue}{Isidor Singer}}}}}\pend
           
\pstart
           \textcolor{gray}{\textbf{\textbf{Dr. \textcolor{blue}{Heinrich Kanner}{}\ledrightnote{\textcolor{blue}{Heinrich Kanner}}}}}\pend
           
\pstart
           \textcolor{gray}{\textbf{\textbf{Redaction}}}\pend
           
\pstart
           \textcolor{gray}{\textbf{Telegramm-Adresse: \textcolor{brown}{\so{Zeit}}{}\ledrightnote{\textcolor{brown}{Die Zeit}}\so{,}{ }\textcolor{pink}{\so{Wien}}{}\ledrightnote{\textcolor{pink}{Wien}}}}\pend
           
\pstart
           \textcolor{gray}{\textbf{Interurbanes Telephon Nr. 15.988}}\pend
           
\pstart
           \textcolor{gray}{\textbf{= Telephone Nr. 17.040, 17.041 =}}\pend
           
\pstart
           Lieber, die Sache ist folgende: Die \textcolor{brown}{Zt}{}\ledrightnote{\textcolor{brown}{Die Zeit}} veranstaltet ein \label{K_L03344-1v}\edtext{Preisaus{[}s{]}chreiben}{\lemma{\textnormal{\emph{Preisausschreiben}}}\Cendnote{\textnormal{Das Preisausschreiben wurde am 4. 10. 1903 beworben. \textcolor{blue}{Schnitzler}
                  fand sich nicht in der Jury. Stattdessen waren in dieser – neben den anderen von
                     \textcolor{blue}{Salten} Genannten – \textcolor{blue}{Karl Glossy}, \textcolor{blue}{August
                     Sauer} und \textcolor{blue}{Isidor Singer}
                  vertreten.}}}\label{K_L03344-1h} für Feuilleton, 3 Preise zu 800, 400 {\kaufmannsund}{ }\substVorne{}\textsuperscript{3}\substDazwischen{}2\substHinten{}00 Kronen. Noch Geheimnis. Ich soll Sie nun ersuchen, in die Jury
               einzutreten, die dann aus \textcolor{blue}{Burckhard}{}\ledrightnote{\textcolor{blue}{Max Eugen Burckhard}}, \textcolor{blue}{Muther}{}\ledrightnote{\textcolor{blue}{Richard Muther}}, \textcolor{blue}{Saar}{}\ledrightnote{\textcolor{blue}{Ferdinand von Saar}}, Ihnen und mir bestehen würde. Arbeit hätten Sie nicht besonders viel
               daran, weil die \textcolor{brown}{Feuilleton-Redaction}{}\ledrightnote{{$\rightarrow$}\textcolor{brown}{Die Zeit}} natürlich die Auslese trifft {\kaufmannsund} den Herren nur jene Arbeiten vorlegt, die zur
               Prämirung in Betracht kommen. Vielleicht sind Sie so liebenswürdig und theilen mir
               rasch mit, {\pb}ob Sie ja oder nein
               dazu sagen, weil die Sache in den nächsten Tagen publicirt werden soll.\pend
           
\pstart
           Aufrichtig {\\[\baselineskip]}Ihr {\\[\baselineskip]}\spacefill\mbox{Salten}\pend
           \leftskip=0em{}\endnumbering\briefempfaengerindex{Schnitzler, Arthur@\textsc{Schnitzler, Arthur}!zzzSalten, Felix@\emph{von Felix Salten}!1903-09-191@{19. 9. {[}1903{]}}|)be}\mylabel{h}  \normalsize

\doendnotes{C}
\bigskip
\vfill

\clearpage

\footnotesize

\lohead{\textsc{register}}

% Definiere theindex-Environment komplett neu ohne reledmac
\makeatletter
\renewenvironment{theindex}{%
  \section*{\indexname}%
  \setlength{\parindent}{0pt}%
  \setlength{\parskip}{0pt plus 0.3pt}%
  \let\item\@idxitem
}{%
  \clearpage
}
\makeatother

\IfFileExists{\jobname-pw.ind}{\input{\jobname-pw.ind}}{}

\end{document}

      