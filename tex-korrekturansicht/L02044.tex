%% latex-korrekturansicht-vorspann.tex
%% Vorspann für die Korrekturansicht.
%% Lädt die gemeinsame Datei latex-vorspann.tex mit gesetztem Schalter.

\newif\ifkorrekturansicht
\korrekturansichttrue

\input{../tex-inputs/latex-vorspann}


               \section[Max Burckhard: Widmungsexemplar Scala Santa für Arthur Schnitzler, 8. 11. 1911]{ Max Burckhard: Widmungsexemplar Scala Santa für Arthur Schnitzler, 8. 11. 1911}\nopagebreak\mylabel{v}\rehead{ }\normalsize\beginnumbering\briefempfaengerindex{Schnitzler, Arthur@\textsc{Schnitzler, Arthur}!zzzBurckhard, Max Eugen@\emph{von Max Eugen Burckhard}!1911-11-081@{8. 11. 1911}|(be} \toendnotes[C]{\smallbreak\pagebreak[2]} \Standort{DLA, G:Schnitzler, Arthur (Sammlung Heinrich Schnitzler).}
\physDesc{Widmung am Vortitel
\newline{}Handschrift: schwarze Tinte, deutsche Kurrent}\pstart
           \noindent{}{\pb}Für D\textsuperscript{r}{ }\textsc{Arthur Schnitzler}\pend
           \pstart
           herzlichſt{\\[\baselineskip]}\spacefill\mbox{D\textsuperscript{r}Burckhard}\pend
           \leftskip=0em{}\pstart
           8. 11. 1911\pend
           {\bigskip}\pstart
           \noindent{}\centering{}\textcolor{gray}{\textbf{\textcolor{green}{SCALA SANTA{\\}UND{\\}ZWÖLF ANDERE »NEUE WAHRE GESCHICHTEN«}{}\ledrightnote{\textcolor{green}{Scala Santa und zwölf andere »Neue wahre Geschichten«}}}}\pend
           \pstart
           \noindent{}\centering{}\textcolor{gray}{\textbf{VON}}\pend
           \pstart
           \noindent{}\centering{}\textcolor{gray}{\textbf{MAX BURCKHARD}}\pend
           {\bigskip}\pstart
           \noindent{}\centering{}\textcolor{gray}{\textbf{\textcolor{pink}{WIEN}{}\ledrightnote{\textcolor{pink}{Wien}}}}\pend
           \pstart
           \noindent{}\centering{}\textcolor{gray}{\textbf{\textcolor{brown}{DEUTSCH-ÖSTERREICHISCHER VERLAG}{}\ledrightnote{\textcolor{brown}{Deutsch-Österreichischer Verlag}}}}\pend
           \endnumbering\briefempfaengerindex{Schnitzler, Arthur@\textsc{Schnitzler, Arthur}!zzzBurckhard, Max Eugen@\emph{von Max Eugen Burckhard}!1911-11-081@{8. 11. 1911}|)be}\mylabel{h}  \normalsize

\doendnotes{C}
\bigskip
\vfill

\clearpage

\footnotesize

\lohead{\textsc{register}}

% Definiere theindex-Environment komplett neu ohne reledmac
\makeatletter
\renewenvironment{theindex}{%
  \section*{\indexname}%
  \setlength{\parindent}{0pt}%
  \setlength{\parskip}{0pt plus 0.3pt}%
  \let\item\@idxitem
}{%
  \clearpage
}
\makeatother

\IfFileExists{\jobname-pw.ind}{\input{\jobname-pw.ind}}{}

\end{document}

      