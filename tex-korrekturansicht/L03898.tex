%% latex-korrekturansicht-vorspann.tex
%% Vorspann für die Korrekturansicht.
%% Lädt die gemeinsame Datei latex-vorspann.tex mit gesetztem Schalter.

\newif\ifkorrekturansicht
\korrekturansichttrue

\input{../tex-inputs/latex-vorspann}


\section[Theodor Herzl an Arthur Schnitzler, 3. 7. 1894]{L03898 Theodor Herzl an Arthur Schnitzler, 3. 7. 1894}
\nopagebreak\mylabel{L03898v}
\rehead{ }\normalsize\beginnumbering\briefempfaengerindex{, @\textsc{, }!zzz, @\emph{von  }!1894-07-032@{3. 7. 1894}|(be}
\toendnotes[C]{\smallbreak\pagebreak[2]}\Standort{Wien, Österreichische Gesellschaft für Literatur, Abschrift Herzl.}
\physDesc{Brief, maschinenschriftliche Abschrift, 1 Blatt, 1 Seite, 714 Zeichen
\newline{}maschinell}
\buchAbdrucke{\weitereDrucke{Theodor Herzl: \emph{Briefe und autobiographische Notizen 1866–1895}. Bearbeitet von Johannes Wachten in Zusammenarbeit mit Chaya Harel, Daisy Tycho und Manfred Winkler. Berlin, Frankfurt am Main, Wien: \emph{Propyläen} 1983, S. 546 (Briefe und Tagebücher. Herausgegeben von Alex Bein, Hermann Greive, Moshe Schaerf, Julius H. Schoeps und Johannes Wachten, 1).} }\toendnotes[C]{\smallbreak}
\pstart
           {\pb}H15\pend
           
\pstart
           \raggedleft{}\textcolor{pink}{Paris}\oindex{Paris@\textbf{Paris}, \emph{Hauptstadt}|pw}{}\ledrightnote{\textcolor{pink}{Paris}}, 3. Juli 1894.\pend
           
\pstart{}Lieber Freund!\pend\vspace{0.5em}
\pstart
           Der \textcolor{violet}{Heiratsanzeige}\eventindex{Wien@\textbf{Wien}!Hochzeit von Helene Altmann und Julius Schnitzler, 8.7.1894@Hochzeit von Helene Altmann und Julius Schnitzler, 8.7.1894|pwv}{}\ledrightnote{{$\rightarrow$}\emph{\textcolor{violet}{Hochzeit von Helene Altmann und Julius Schnitzler, 8.7.1894}}} Ihres Herrn \textcolor{blue}{Bruders}\pwindex{Schnitzler, Julius 13.\,7.\,1865 Wien – 29.\,6.\,1939 ebd.@\textsc{Schnitzler, Julius} (13.\,7.\,1865 Wien – 29.\,6.\,1939 ebd.), \emph{Chirurg}|pwv}{}\ledrightnote{{$\rightarrow$}\emph{\textcolor{blue}{Julius Schnitzler}}} kann ich keine Adresse
               entnehmen.\pend
           
\pstart
           Ich wende mich daher an Sie mit der Bitte, meine Glückwünsche Ihrem \textcolor{blue}{Bruder}\pwindex{Schnitzler, Julius 13.\,7.\,1865 Wien – 29.\,6.\,1939 ebd.@\textsc{Schnitzler, Julius} (13.\,7.\,1865 Wien – 29.\,6.\,1939 ebd.), \emph{Chirurg}|pwv}{}\ledrightnote{{$\rightarrow$}\emph{\textcolor{blue}{Julius Schnitzler}}} und vor allem Ihrer hochverehrten Frau \textcolor{blue}{Mutter}\pwindex{Schnitzler, Louise 8.\,7.\,1840 Kőszeg – 9.\,9.\,1911 Wien@\textsc{Schnitzler, Louise} (8.\,7.\,1840 Kőszeg – 9.\,9.\,1911 Wien)|pwv}{}\ledrightnote{{$\rightarrow$}\emph{\textcolor{blue}{Louise Schnitzler}}} zu überbringen. Nach Ihrem
               grossen \label{K_L03898-1v}\edtext{Schmerz}{\lemma{\textnormal{\emph{Schmerz}}}\Cendnote{\textnormal{\textcolor{blue}{Schnitzlers} Vater \textcolor{blue}{Johann}\pwindex{Schnitzler, Johann 10.\,4.\,1835 Nagykanizsa – 2.\,5.\,1893 Wien@\textsc{Schnitzler, Johann} (10.\,4.\,1835 Nagykanizsa – 2.\,5.\,1893 Wien), \emph{Laryngologe}|pwk} war am 2. 5. 1893 gestorben.}}}\label{K_L03898-1} wirds wieder lichter im Haus.
               Ich freue mich mit allen Ihren Freunden darüber.\pend
           
\pstart
           Ihnen mein lieber Poet drücke ich dabei wieder einmal die Hand. Was macht die
               Dichtung? Warum schicken Sie mir nicht, was Sie schreiben? Ich würde es mit Vergnügen
               auf dem Telegraphenamt zwischen zwei blutrünstigen Depeschen lesen. Wahrscheinlich
               gegen Ende Juli gehe ich auf Urlaub. Nach \textcolor{pink}{Aussee}\oindex{Bad Aussee@\textbf{Bad Aussee}, \emph{Hauptstadt}|pw}{}\ledrightnote{\textcolor{pink}{Bad Aussee}}. Kommen
               Sie doch ein bischen vorüber. Plaudern!\pend
           
\pstart
           Herzlich Ihr ergebener{\\[\baselineskip]}\spacefill\mbox{Th. Herzl.}\pend
           \leftskip=0em{}\selectlanguage{ngerman}\endnumbering\briefempfaengerindex{, @\textsc{, }!zzz, @\emph{von  }!1894-07-032@{3. 7. 1894}|)be}\mylabel{L03898h}
\begin{anhang}
\end{anhang}\normalsize

\doendnotes{C}
\bigskip
\vfill

\clearpage

\footnotesize

\lohead{\textsc{register}}

% Definiere theindex-Environment komplett neu ohne reledmac
\makeatletter
\renewenvironment{theindex}{%
  \section*{\indexname}%
  \setlength{\parindent}{0pt}%
  \setlength{\parskip}{0pt plus 0.3pt}%
  \let\item\@idxitem
}{%
  \clearpage
}
\makeatother

\IfFileExists{\jobname-pw.ind}{\input{\jobname-pw.ind}}{}

\end{document}

      