%% latex-korrekturansicht-vorspann.tex
%% Vorspann für die Korrekturansicht.
%% Lädt die gemeinsame Datei latex-vorspann.tex mit gesetztem Schalter.

\newif\ifkorrekturansicht
\korrekturansichttrue

\input{../tex-inputs/latex-vorspann}


               \section[Arthur Schnitzler an Richard Beer-Hofmann, 24. 12. 1899]{ Arthur Schnitzler an Richard Beer-Hofmann, 24. 12. 1899}\nopagebreak\mylabel{v}\rehead{ }\normalsize\beginnumbering\briefempfaengerindex{Beer-Hofmann, Richard@\textsc{Beer-Hofmann, Richard}!zzzSchnitzler, Arthur@\emph{von Arthur Schnitzler}!1899-12-241@{24. 12. 1899}|(be} \toendnotes[C]{\smallbreak\pagebreak[2]} \Standort{YCGL, MSS 31.}
\physDesc{Brief, 1 Blatt, 3 Seiten, Umschlag
\newline{}Handschrift: Bleistift, deutsche Kurrent\newline{}Versand: Stempel: »\nobreak{}\oindex{IX., Alsergrund@\textbf{IX., Alsergrund}, \emph{Bezirk (A.BZK)}|pwk}Wien 9/1, 2{[}4. 12. 1899{]}, 5–6V\nobreak{}«.  }\buchAbdrucke{\weitereDrucke{Arthur Schnitzler, Richard Beer-Hofmann: \emph{Briefwechsel 1891–1931}. Hg. Konstanze Fliedl. Wien, Zürich: \emph{Europaverlag} 1992, S. 140.} }\toendnotes[C]{\smallbreak}\pstart{}{\pb}Herrn \textsc{Dr. Richard
                     Beer-Hofmann}\pend{}\pstart{}\textcolor{pink}{Wien}{}\ledrightnote{\textcolor{pink}{Wien}}\pend{}\pstart{}\textsc{\textcolor{pink}{I. Wollzeile 15}{}\ledrightnote{\textcolor{pink}{Wollzeile}}}.\pend{}{\bigskip}\pstart
           \raggedleft{}{\pb}24. 12. 99\pend
           \pstart{}mein lieber Richard,\pend\pstart
           ich ka{\geminationn} nur ſagen, es iſt geradezu feinſinnig, was
               diesmal keine Beleidigung bedeuten ſoll, und ich bin (wiſſen Sie kein andres Wort?)
               beſchämt, befangen – und verſuche mich mit einem Witz aus der Affaire zu {\pb}ziehen – z. B. daſs ich immer auf einen der \textcolor{green}{3 Einakter}{}\ledrightnote{\textcolor{green}{Der grüne Kakadu – Paracelsus – Die Gefährtin. Drei Einakter}} verzichten muſs – bei Ihrem Geſchenk auf
               die \textcolor{green}{Gefährtin}{}\ledrightnote{\textcolor{green}{Die Gefährtin. Schauspiel in einem Akt}} – aber ich will (was gleich ein
               zweiter Witz iſt) die Schachtel ſelbſt als Gefährtin anſehen da sie (dritter Witz)
               keine alte iſt.\pend
           \pstart
           {\pb}Also herzlichen Dank und Gruſs; auf Wiederſehen
                  \label{K_L01007_1v}\edtext{\textcolor{green}{morgen}{}\ledrightnote{→\textcolor{green}{Gläubiger}}}{\lemma{\textnormal{\emph{morgen}}}\Cendnote{\textnormal{Am \emph{\textcolor{brown}{Theater in der Josefstadt}} wurde am 25. 12. 1899{ }\emph{\textcolor{green}{Gläubiger}} von \textcolor{blue}{August Strindberg} und \emph{\textcolor{green}{Die
                     Mondscheinsonate}} von \textcolor{blue}{Ludwig Wolff}
                  gegeben.}}}\label{K_L01007_1h}, wohl ſchon in der \textcolor{pink}{Joſefſtadt}{}\ledrightnote{\textcolor{pink}{Theater in der Josefstadt}}.\pend
           \pstart Ihr \spacefill\mbox{Arthur}\pend{}\endnumbering\briefempfaengerindex{Beer-Hofmann, Richard@\textsc{Beer-Hofmann, Richard}!zzzSchnitzler, Arthur@\emph{von Arthur Schnitzler}!1899-12-241@{24. 12. 1899}|)be}\mylabel{h}  \normalsize

\doendnotes{C}
\bigskip
\vfill

\clearpage

\footnotesize

\lohead{\textsc{register}}

% Definiere theindex-Environment komplett neu ohne reledmac
\makeatletter
\renewenvironment{theindex}{%
  \section*{\indexname}%
  \setlength{\parindent}{0pt}%
  \setlength{\parskip}{0pt plus 0.3pt}%
  \let\item\@idxitem
}{%
  \clearpage
}
\makeatother

\IfFileExists{\jobname-pw.ind}{\input{\jobname-pw.ind}}{}

\end{document}

      