%% latex-korrekturansicht-vorspann.tex
%% Vorspann für die Korrekturansicht.
%% Lädt die gemeinsame Datei latex-vorspann.tex mit gesetztem Schalter.

\newif\ifkorrekturansicht
\korrekturansichttrue

\input{../tex-inputs/latex-vorspann}


         
         \renewcommand{\erwaehntePersonen}{Personen: Albert Bassermann, Alfred von Berger, Auguste Chlum, Marie Glümer, Georg Hirschfeld, Stella Hohenfels, Paul Lindau, Paul Schlenther, Irene Triesch}
         \renewcommand{\erwaehnteInstitutionen}{Institutionen: Berliner Theater, Burgtheater, Deutsches Schauspielhaus in Hamburg, Frankfurter Stadttheater, Volkstheater}
         \renewcommand{\erwaehnteOrte}{Orte: Alpen, Berlin, Dessauer Straße, Dolomiten, Frankfurt am Main, Hamburg, Schlesien, Vorarlberg, Wien}
         \renewcommand{\erwaehnteWerke}{Werke: Der Schleier der Beatrice. Schauspiel in fünf Akten}
               \section[ Paul Goldmann an Arthur Schnitzler, 21. 6. {[}1900{]}]{Paul Goldmann an Arthur Schnitzler, 21. 6. {[}1900{]}}\nopagebreak\mylabel{v}\rehead{ }\normalsize\beginnumbering\briefempfaengerindex{Schnitzler, Arthur@\textsc{Schnitzler, Arthur}!zzzGoldmann, Paul@\emph{von Paul Goldmann}!1900-06-213@{21. 6. {[}1900{]}}|(be} \toendnotes[C]{\smallbreak\pagebreak[2]} \Standort{DLA, A:Schnitzler, HS.NZ85.1.3170.}
\physDesc{Brief, 1 Blatt, 4 Seiten
\newline{}Handschrift: blaue Tinte, deutsche Kurrent
\newline{}Schnitzler: 1) mit Bleistift das Jahr »{[}1{]}900« vermerkt  2) mit rotem Buntstift sechs Unterstreichungen}\toendnotes[C]{\smallbreak}\pstart
           \noindent{}{\pb}\textcolor{pink}{\textcolor{gray}{\textbf{DESSAUERSTRASSE 19}}}{}\ledrightnote{\textcolor{pink}{Dessauer Straße}}\hfill \textcolor{pink}{Berlin}{}\ledrightnote{\textcolor{pink}{Berlin}}, 21. Juni.\pend
           \pstart
           \centering{}Mein lieber Freund,\pend
           \pstart
           \noindent{}Das iſt ein großes \label{K_L02921-1v}\edtext{Ärgerniß}{\lemma{\textnormal{\emph{Ärgerniß}}}\Cendnote{\textnormal{Bezug auf die Absage \textcolor{blue}{Paul Schlenther}s, \emph{\textcolor{green}{Der
                     Schleier der Beatrice}} am \emph{\textcolor{brown}{Burgtheater}}
                  doch nicht aufzuführen (siehe Paul Goldmann an Arthur Schnitzler, 12. 11. [1899])}}}\label{K_L02921-1h}, und es thut mir unendlich leid, daß es Dir nicht
               erſpart geblieben iſt. Von Herrn \textsc{\textcolor{blue}{Schlenther}{}\ledrightnote{\textcolor{blue}{Paul Schlenther}}} freilich überraſcht es mich nicht, und es iſt eigentlich viel natürlicher, daß
               er Dein \textcolor{green}{Stück}{}\ledrightnote{{$\rightarrow$}\textcolor{green}{Der Schleier der Beatrice. Schauspiel in fünf Akten}} nicht aufführt,
               als daß er es aufführt. Dieſem nüchternen \textcolor{blue}{\textcolor{pink}{Berlin}{}\ledrightnote{\textcolor{pink}{Berlin}}er}{}\ledrightnote{{$\rightarrow$}\textcolor{blue}{Paul Schlenther}} liegt Dein \textcolor{green}{Werk}{}\ledrightnote{{$\rightarrow$}\textcolor{green}{Der Schleier der Beatrice. Schauspiel in fünf Akten}} mit all’ ſeinen poetiſchen Schönheiten
                  \strikeout{\textcolor{gray}{ja}} ſo fern! Ja, wenn es \textcolor{pink}{ſchleſi}{}\ledrightnote{{$\rightarrow$}\textcolor{pink}{Schlesien}}ſche Bauern wären oder eine \textcolor{pink}{Berlin}{}\ledrightnote{\textcolor{pink}{Berlin}}er jüdiſche Familie, wie in den »Milieuſtücken« von \textsc{\textcolor{blue}{Hirschfeld}{}\ledrightnote{\textcolor{blue}{Georg Hirschfeld}}}! Wie ſoll ein \textsc{\textcolor{blue}{Schlenther}{}\ledrightnote{\textcolor{blue}{Paul Schlenther}}} Deine »\textsc{\textcolor{green}{Beatrice}{}\ledrightnote{\textcolor{green}{Der Schleier der Beatrice. Schauspiel in fünf Akten}}}« verſtehen? Wenn Du ruhig nachdenkſt, wirſt Du {\pb}ſelbſt einſehen, daß es nicht möglich iſt. Dabei glaube ich noch nicht einmal, daß
               der Refüs ſich in erſter Linie gegen Dein \textcolor{green}{Werk}{}\ledrightnote{{$\rightarrow$}\textcolor{green}{Der Schleier der Beatrice. Schauspiel in fünf Akten}} richtet. Es mag Mancherlei dabei mitgeſpielt haben: Der
               Herr \textcolor{blue}{Direktor}{}\ledrightnote{{$\rightarrow$}\textcolor{blue}{Paul Schlenther}} war zu faul,
               dieſes große \textcolor{green}{Drama}{}\ledrightnote{{$\rightarrow$}\textcolor{green}{Der Schleier der Beatrice. Schauspiel in fünf Akten}}
               einzuſtudiren, was keine leichte Aufgabe iſt. Dann hat er ſich wohl auch vor den
               Koſten der Ausſtattung gefürchtet. Das darf er dem durch ſeine Wirthſchaft ohnehin
               ſchon ſo ſehr aus dem Gleichgewicht gebrachten Büdget des \textcolor{brown}{Burgtheater}{}\ledrightnote{\textcolor{brown}{Burgtheater}}s nicht mehr zumuthen. Und ſo weiter.\pend
           \pstart
           Du wirſt an Herrn \textsc{\textcolor{blue}{Schlenther}{}\ledrightnote{\textcolor{blue}{Paul Schlenther}}} ſchon alle wünſchenswerthe Genugthuung {\pb}erleben. In dieſer Hinſicht bin ich ohne Sorge. Jetzt handelt es ſich nur darum,
               daß Dein \textcolor{green}{Drama}{}\ledrightnote{{$\rightarrow$}\textcolor{green}{Der Schleier der Beatrice. Schauspiel in fünf Akten}} unter allen
               Umſtänden \label{K_L02921-2v}\edtext{aufgeführt}{\lemma{\textnormal{\emph{aufgeführt}}}\Cendnote{\textnormal{siehe Paul Goldmann an Arthur Schnitzler, 12. 11. [1899]}}}\label{K_L02921-2h} wird. Vom \label{K_L02921-76v}\edtext{\textcolor{pink}{Wien}{}\ledrightnote{\textcolor{pink}{Wien}}er \textcolor{brown}{Volkstheater}{}\ledrightnote{\textcolor{brown}{Volkstheater}}}{\lemma{\textnormal{\emph{Wiener Volkstheater}}}\Cendnote{\textnormal{\emph{\textcolor{green}{Der Schleier der Beatrice}} wurde, trotz
                  mehrmaliger Anläufe in den Jahren 1908 (vgl. A. S.: \emph{Tagebuch}, 25. 2. 1908 und 6. 3. 1900) und 1924 (vgl. A. S.: \emph{Tagebuch}, 29. 6. 1900, nicht am \emph{\textcolor{brown}{Volkstheater}}
                  aufgeführt.}}}\label{K_L02921-76h} möchte ich dringend abrathen. Dort haben ſie zu plumpe Hände
               für das \textcolor{green}{Stück}{}\ledrightnote{{$\rightarrow$}\textcolor{green}{Der Schleier der Beatrice. Schauspiel in fünf Akten}}. Aber, \strikeout{da} ich möchte Dir dringend das »\textcolor{brown}{Berliner Theater}{}\ledrightnote{\textcolor{brown}{Berliner Theater}}« empfehlen. \textsc{\textcolor{blue}{Lindau}{}\ledrightnote{\textcolor{blue}{Paul Lindau}}} wird das \textcolor{green}{Werk}{}\ledrightnote{{$\rightarrow$}\textcolor{green}{Der Schleier der Beatrice. Schauspiel in fünf Akten}} mit Liebe
               einſtudiren. Die Ausſtattung wird zwar dürftig ſein; aber \label{K_L02921-78v}\edtext{\textsc{\textcolor{blue}{Bassermann}{}\ledrightnote{\textcolor{blue}{Albert Bassermann}}}}{\lemma{\textnormal{\emph{Bassermann}}}\Cendnote{\textnormal{siehe Paul Goldmann an Arthur Schnitzler, [6.?] 2. 1903}}}\label{K_L02921-78h} wäre ein glänzender Vertreter für den \textcolor{green}{Herzog}{}\ledrightnote{{$\rightarrow$}\textcolor{green}{Der Schleier der Beatrice. Schauspiel in fünf Akten}}. Auch \label{K_L02921-56v}\edtext{\textsc{\textcolor{blue}{Berger}{}\ledrightnote{\textcolor{blue}{Alfred von Berger}}}}{\lemma{\textnormal{\emph{Berger}}}\Cendnote{\textnormal{\textcolor{blue}{Alfred von Berger} hatte \emph{\textcolor{green}{Der Schleier der Beatrice}} für das \emph{\textcolor{brown}{Deutsche Schauspielhaus in Hamburg}} bereits abgelehnt (vgl. A. S.: \emph{Tagebuch}, 17. 2. 1900).}}}\label{K_L02921-56h} würde
               das \textcolor{green}{Stück}{}\ledrightnote{{$\rightarrow$}\textcolor{green}{Der Schleier der Beatrice. Schauspiel in fünf Akten}} gewiß gern in ſeinem
               neuen \textcolor{pink}{Hamburg}{}\ledrightnote{\textcolor{pink}{Hamburg}}er \textcolor{brown}{Theater}{}\ledrightnote{{$\rightarrow$}\textcolor{brown}{Deutsches Schauspielhaus in Hamburg}} aufführen, und die \textsc{\textcolor{blue}{Hohenfels}{}\ledrightnote{\textcolor{blue}{Stella Hohenfels}}} ſpielt vielleicht die \textsc{\textcolor{green}{Beatrice}{}\ledrightnote{{$\rightarrow$}\textcolor{green}{Der Schleier der Beatrice. Schauspiel in fünf Akten}}}. Wirklich ſpielen kann dieſe Rolle {\pb}allerdings
               nur eine: die \label{K_L02921-31v}\edtext{\textsc{\textcolor{blue}{Triesch}{}\ledrightnote{\textcolor{blue}{Irene Triesch}}}}{\lemma{\textnormal{\emph{Triesch}}}\Cendnote{\textnormal{siehe Paul Goldmann an Arthur Schnitzler, 20. 2. 1900}}}\label{K_L02921-31h} in \textcolor{pink}{Frankfurt}{}\ledrightnote{\textcolor{pink}{Frankfurt am Main}}, und darum wäre es
               vielleicht auch nicht ſchlecht, das \textcolor{green}{Stück}{}\ledrightnote{{$\rightarrow$}\textcolor{green}{Der Schleier der Beatrice. Schauspiel in fünf Akten}} zur Erſtaufführung nach \textcolor{brown}{\textcolor{pink}{Frankfurt}{}\ledrightnote{\textcolor{pink}{Frankfurt am Main}}}{}\ledrightnote{{$\rightarrow$}\textcolor{brown}{Frankfurter Stadttheater}} zu geben.\pend
           \pstart
           Wenn Du willſt, gehe ich hier perſönlich zu \label{K_L02921-45v}\edtext{\textsc{\textcolor{blue}{Lindau}{}\ledrightnote{\textcolor{blue}{Paul Lindau}}}}{\lemma{\textnormal{\emph{Lindau}}}\Cendnote{\textnormal{Es sind keine Bemühungen um eine
                  Aufführung von \emph{\textcolor{green}{Der Schleier der Beatrice}} in
                     \textcolor{blue}{Paul Lindau}s \emph{\textcolor{brown}{Berliner Theater}} bekannt.}}}\label{K_L02921-45h} hin.\pend
           \pstart
           Laß’ mich bald wiſſen, was Du beſchloſſen haſt, und ſchreib’ mir auch, wie \strikeout{es \textcolor{gray}{m}} es mit der \label{K_L02921-4v}\edtext{\textcolor{pink}{Alpen}{}\ledrightnote{\textcolor{pink}{Alpen}}wanderung im Auguſt}{\lemma{\textnormal{\emph{Alpenwanderung im Auguſt}}}\Cendnote{\textnormal{siehe Paul Goldmann an Arthur Schnitzler, 16. 6. [1900]}}}\label{K_L02921-4h} ſteht. Die \textcolor{pink}{Dolomiten}{}\ledrightnote{\textcolor{pink}{Dolomiten}} wären mir
               allerdings lieber als \textcolor{pink}{Vorarlberg}{}\ledrightnote{\textcolor{pink}{Vorarlberg}}.\pend
           \pstart
           Viele treue Grüße! {\\[\baselineskip]}Dein {\\[\baselineskip]}\spacefill\mbox{Paul Goldmnn}\pend
           \leftskip=0em{}\pstart
           \noindent{}Wenn Du die \label{K_L02921-5v}\edtext{\textcolor{blue}{Fräuleins \textsc{Glümer}}{}\ledrightnote{{$\rightarrow$}\textcolor{blue}{Marie Glümer}{\newline}{$\rightarrow$}\textcolor{blue}{Auguste Chlum}} ſiehſt}{\lemma{\textnormal{\emph{Fräuleins Glümer ſiehſt}}}\Cendnote{\textnormal{\textcolor{blue}{Marie Glümer} traf \textcolor{blue}{Schnitzler} am 27. 6. 1900
                     wieder.}}}\label{K_L02921-5h}, ſo ſag’ ihnen, daß ich ihnen herzlichſt für ihre lieben Briefe
                  und Karten danke. Ich weiß leider ihre Adreſſe nicht.\pend
           \endnumbering\briefempfaengerindex{Schnitzler, Arthur@\textsc{Schnitzler, Arthur}!zzzGoldmann, Paul@\emph{von Paul Goldmann}!1900-06-213@{21. 6. {[}1900{]}}|)be}\mylabel{h}\begin{anhang}\end{anhang}\normalsize

\doendnotes{C}
\bigskip
\vfill

\clearpage

\footnotesize

\lohead{\textsc{register}}

% Definiere theindex-Environment komplett neu ohne reledmac
\makeatletter
\renewenvironment{theindex}{%
  \section*{\indexname}%
  \setlength{\parindent}{0pt}%
  \setlength{\parskip}{0pt plus 0.3pt}%
  \let\item\@idxitem
}{%
  \clearpage
}
\makeatother

\IfFileExists{\jobname-pw.ind}{\input{\jobname-pw.ind}}{}

\end{document}

      