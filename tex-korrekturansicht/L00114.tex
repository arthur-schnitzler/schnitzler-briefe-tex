%% latex-korrekturansicht-vorspann.tex
%% Vorspann für die Korrekturansicht.
%% Lädt die gemeinsame Datei latex-vorspann.tex mit gesetztem Schalter.

\newif\ifkorrekturansicht
\korrekturansichttrue

\input{../tex-inputs/latex-vorspann}


               \section[Arthur Schnitzler an Richard Beer-Hofmann, 17. 8. 1892]{ Arthur Schnitzler an Richard Beer-Hofmann, 17. 8. 1892}\nopagebreak\mylabel{v}\rehead{ }\normalsize\beginnumbering\briefempfaengerindex{Beer-Hofmann, Richard@\textsc{Beer-Hofmann, Richard}!zzzSchnitzler, Arthur@\emph{von Arthur Schnitzler}!1892-08-171@{17. 8. 1892}|(be} \toendnotes[C]{\smallbreak\pagebreak[2]} \Standort{YCGL, MSS 31.}
\physDesc{Brief, 1 Blatt (Briefpapier mit Trauerrand), 2 Seiten, Umschlag
\newline{}Handschrift: schwarze Tinte, deutsche Kurrent\newline{}Versand: 1) Stempel: »\nobreak{}Wien 4/1, 17 8 92, 6–7N\nobreak{}«.  2) Stempel: »\nobreak{}\oindex{Bad Ischl@\textbf{Bad Ischl}, \emph{Besiedelter Ort (A.BSO)}|pwk}{\pb}Ischl, 18 8 92, 10 \textcolor{gray}{F}\nobreak{}«. }\buchAbdrucke{\weitereDrucke{Arthur Schnitzler, Richard Beer-Hofmann: \emph{Briefwechsel 1891–1931}. Hg. Konstanze Fliedl. Wien, Zürich: \emph{Europaverlag} 1992, S. 36.} }\pstart{}{\pb}Herrn Doctor \textsc{Richard
                     Beer-Hofmann}\pend{}\pstart{}\textcolor{pink}{\textsc{Ischl}}{}\ledrightnote{\textcolor{pink}{Bad Ischl}}\pend{}\pstart{}\textsc{\textcolor{pink}{Grazerstraße 6}{}\ledrightnote{\textcolor{pink}{Grazer Straße}}}.\pend{}\pstart{}(oder \textcolor{pink}{\textsc{Kreuzplatz}}{}\ledrightnote{\textcolor{pink}{Kreuzplatz}}?)\pend{}{\bigskip}\pstart{}{\pb}Lieber Richard,\pend\pstart
           finden Sie nicht auch, daſs Sie mir hätten antworten können? Ich dürfte erſt ca.
                  4. oder 5. September nach \textcolor{pink}{Iſchl}{}\ledrightnote{\textcolor{pink}{Bad Ischl}} ko{\geminationm}en\strikeout{?}. Wollen Sie ein paar Tage darauf mit mir weiter wandern? Ich möchte eine
               größere Fußpartie (nicht Bergbeſteigungen!!) {\pb}in der
                  \textcolor{pink}{Schweiz}{}\ledrightnote{\textcolor{pink}{Schweiz}} machen. – Oder auch die \textcolor{pink}{oberitalien.}{}\ledrightnote{\textcolor{pink}{Italien}}{ }Seen aufsuchen. Ich frage mich heute auch
               bei \textcolor{blue}{\textsc{Loris}}{}\ledrightnote{\textcolor{blue}{Hugo von Hofmannsthal}} an. Aber, bitte, antworten Sie mir.\pend
           \pstart Herzlich Ihr \spacefill\mbox{Arthur}\pend{}\pstart
           \textcolor{pink}{Wien}{}\ledrightnote{\textcolor{pink}{Wien}}{ }17/8 92.\pend
           \endnumbering\briefempfaengerindex{Beer-Hofmann, Richard@\textsc{Beer-Hofmann, Richard}!zzzSchnitzler, Arthur@\emph{von Arthur Schnitzler}!1892-08-171@{17. 8. 1892}|)be}\mylabel{h}  \normalsize

\doendnotes{C}
\bigskip
\vfill

\clearpage

\footnotesize

\lohead{\textsc{register}}

% Definiere theindex-Environment komplett neu ohne reledmac
\makeatletter
\renewenvironment{theindex}{%
  \section*{\indexname}%
  \setlength{\parindent}{0pt}%
  \setlength{\parskip}{0pt plus 0.3pt}%
  \let\item\@idxitem
}{%
  \clearpage
}
\makeatother

\IfFileExists{\jobname-pw.ind}{\input{\jobname-pw.ind}}{}

\end{document}

      