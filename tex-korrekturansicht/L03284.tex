%% latex-korrekturansicht-vorspann.tex
%% Vorspann für die Korrekturansicht.
%% Lädt die gemeinsame Datei latex-vorspann.tex mit gesetztem Schalter.

\newif\ifkorrekturansicht
\korrekturansichttrue

\input{../tex-inputs/latex-vorspann}


\renewcommand{\erwaehntePersonen}{Personen: Ottilie Salten, Ferdinande Schmittlein}
\renewcommand{\erwaehnteInstitutionen}{Institutionen: Burgtheater}
\renewcommand{\erwaehnteOrte}{Orte: Wien, XIII., Hietzing}
\renewcommand{\erwaehnteWerke}{Werke: Das Vermächtnis. Schauspiel in drei Akten}
\section[ Felix Salten an Arthur Schnitzler, 23. 9. 1898]{Felix Salten an Arthur Schnitzler, 23. 9. 1898}
\nopagebreak\mylabel{v}
\rehead{ }\normalsize\beginnumbering\briefempfaengerindex{Schnitzler, Arthur@\textsc{Schnitzler, Arthur}!zzzSalten, Felix@\emph{von Felix Salten}!1898-09-231@{23. 9. 1898}|(be}
\toendnotes[C]{\smallbreak\pagebreak[2]}\Standort{CUL, Schnitzler, B 89, A 2.}
\physDesc{Brief, 1 Blatt, 1 Seite, 451 Zeichen
\newline{}Handschrift: schwarze Tinte, lateinische Kurrent
\newline{}Ordnung: mit Bleistift von unbekannter Hand nummeriert: »108« }\toendnotes[C]{\smallbreak}
\pstart
           \raggedleft{}{\pb}\textcolor{pink}{Hietzing}{}\ledrightnote{\textcolor{pink}{XIII., Hietzing}}, 23./IX. 98\pend
           
\pstart
           Lieber Arthur, von Frau \textcolor{blue}{Schmittlein}{}\ledrightnote{\textcolor{blue}{Ferdinande Schmittlein}} höre ich, dass die \label{K_L03284-1v}\edtext{Rollen zum »\textcolor{green}{Vermächtnis}{}\ledrightnote{\textcolor{green}{Das Vermächtnis. Schauspiel in drei Akten}}«}{\lemma{\textnormal{\emph{Rollen zum »Vermächtnis«}}}\Cendnote{\textnormal{Die Uraufführung fand am 8. 10. 1898 am \emph{\textcolor{brown}{Burgtheater}} statt.}}}\label{K_L03284-1h} schon eingetheilt
               sind. Vielleicht theilen Sie mir, bitte, mit, ob \label{K_L03284-2v}\edtext{Frl. \textcolor{blue}{Metzl}{}\ledrightnote{\textcolor{blue}{Ottilie Salten}}}{\lemma{\textnormal{\emph{Frl. Metzl}}}\Cendnote{\textnormal{\textcolor{blue}{Ottilie Metzl} bekam die Rolle des \textcolor{green}{Lulu}. Siehe auch Arthur Schnitzler an Felix Salten, 24. 9. 1898.}}}\label{K_L03284-2h} nichts bekommt. Ich möchte
               ihr doch gerne etwas Tröstendes und Beruhigendes sagen, ehe sie’s erfährt. Denn ich
               habe ihr nach Ihrer Zusage sehr viel Hoffnung auf die Rolle gemacht, so dass \strikeout{es} sie es diesmal besonders schmerzlich empfinden wird,
               übergangen zu werden.\pend
           
\pstart
           Herzlichst {\\[\baselineskip]}Ihr {\\[\baselineskip]}\spacefill\mbox{Salten}\pend
           \leftskip=0em{}\endnumbering\briefempfaengerindex{Schnitzler, Arthur@\textsc{Schnitzler, Arthur}!zzzSalten, Felix@\emph{von Felix Salten}!1898-09-231@{23. 9. 1898}|)be}\mylabel{h}  \normalsize

\doendnotes{C}
\bigskip
\vfill

\clearpage

\footnotesize

\lohead{\textsc{register}}

% Definiere theindex-Environment komplett neu ohne reledmac
\makeatletter
\renewenvironment{theindex}{%
  \section*{\indexname}%
  \setlength{\parindent}{0pt}%
  \setlength{\parskip}{0pt plus 0.3pt}%
  \let\item\@idxitem
}{%
  \clearpage
}
\makeatother

\IfFileExists{\jobname-pw.ind}{\input{\jobname-pw.ind}}{}

\end{document}

      