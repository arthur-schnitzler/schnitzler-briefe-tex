%% latex-korrekturansicht-vorspann.tex
%% Vorspann für die Korrekturansicht.
%% Lädt die gemeinsame Datei latex-vorspann.tex mit gesetztem Schalter.

\newif\ifkorrekturansicht
\korrekturansichttrue

\input{../tex-inputs/latex-vorspann}


\renewcommand{\erwaehntePersonen}{Personen:  Forster, Hugo Ganz, Ernst Gettke, Theodor Herzl, Heinrich Kanner, Raphael Löwenfeld, Paul Paschen, Isidor Singer, Siegfried Trebitsch, Jakob Wassermann}
\renewcommand{\erwaehnteInstitutionen}{Institutionen: Burgtheater, Die Zeit, S. Fischer Verlag, Schiller-Theater}
\renewcommand{\erwaehnteOrte}{Orte: Berlin, Raimund-Theater, Wien, Wipplingerstraße}
\renewcommand{\erwaehnteWerke}{Werke: Altneuland. Roman, Der Moloch, Die Zeit, Die kleine Veronika, Fünfkreuzertanz, Liebelei. Schauspiel in drei Akten, [Artikel über Kulissenton], [Aufsatz über volkstümliche Klassikervorstellungen], »Altneuland«}
\section[ Felix Salten an Arthur Schnitzler, 15. 10. 1902]{Felix Salten an Arthur Schnitzler, 15. 10. 1902}
\nopagebreak\mylabel{v}
\rehead{ }\normalsize\beginnumbering\briefempfaengerindex{Schnitzler, Arthur@\textsc{Schnitzler, Arthur}!zzzSalten, Felix@\emph{von Felix Salten}!1902-10-153@{15. 10. 1902}|(be}
\toendnotes[C]{\smallbreak\pagebreak[2]}\Standort{CUL, Schnitzler, B 89, A 2.}
\physDesc{Brief, 1 Blatt, 2 Seiten, 1723 Zeichen
\newline{}Handschrift: blaue Tinte, lateinische Kurrent
\newline{}Ordnung: mit Bleistift von unbekannter Hand nummeriert: »160« }\toendnotes[C]{\smallbreak}
\pstart
           \noindent{}{\pb}\textcolor{gray}{\textbf{DIE}}\pend
           
\pstart
           \textcolor{gray}{\textbf{\textcolor{brown}{ZEIT}{}\ledrightnote{\textcolor{brown}{Die Zeit}}}}\pend
           
\pstart
           \textcolor{gray}{\textbf{\textbf{\textcolor{pink}{Wien}{}\ledrightnote{\textcolor{pink}{Wien}}er Tageszeitung}}}\hfill \textcolor{gray}{\textbf{\emph{\textcolor{pink}{WIEN}{}\ledrightnote{\textcolor{pink}{Wien}}}}}{ }15. Octob. 02\pend
           
\pstart
           \textcolor{gray}{\textbf{Herausgeber:}}\hfill \textcolor{gray}{\textbf{\emph{\textcolor{pink}{I., Wipplingerstrasse 38}{}\ledrightnote{\textcolor{pink}{Wipplingerstraße}}}}}\pend
           
\pstart
           \textcolor{gray}{\textbf{\textbf{Prof. Dr. \textcolor{blue}{I. Singer}{}\ledrightnote{\textcolor{blue}{Isidor Singer}}}}}\pend
           
\pstart
           \textcolor{gray}{\textbf{\textbf{Dr. \textcolor{blue}{Heinrich Kanner}{}\ledrightnote{\textcolor{blue}{Heinrich Kanner}}}}}\pend
           
\pstart
           \textcolor{gray}{\textbf{\textbf{Redaction.}}}\pend
           
\pstart
           \textcolor{gray}{\textbf{Telegramm-Adresse: \textcolor{brown}{\so{Zeit}}{}\ledrightnote{\textcolor{brown}{Die Zeit}}\so{,}{ }\textcolor{pink}{\so{Wien}}{}\ledrightnote{\textcolor{pink}{Wien}}}}\pend
           
\pstart
           \textcolor{gray}{\textbf{Interurbanes Telephon Nr. 15.988}}\pend
           
\pstart
           \textcolor{gray}{\textbf{= Telephone Nr. 17.040, 17.041 =}}\pend
           
\pstart
           Lieber Freund, ich habe sehr bedauert, dass mich die
               Satzcorrectur zum »\label{K_L03335-1v}\edtext{\textcolor{green}{Fünfkreuzertanz}{}\ledrightnote{\textcolor{green}{Fünfkreuzertanz}}}{\lemma{\textnormal{\emph{Fünfkreuzertanz}}}\Cendnote{\textnormal{\textcolor{blue}{Felix Salten}: \emph{\textcolor{green}{Fünfkreuzertanz}}. In: \emph{\textcolor{green}{Die Zeit}}, Jg. 1, Nr. 16, 12. 10. 1902,
                     Morgenblatt, S. 2–3.}}}\label{K_L03335-1h}« Samstag bis
                  2 Uhr in der \textcolor{pink}{\textcolor{brown}{Redaction}{}\ledrightnote{{$\rightarrow$}\textcolor{brown}{Die Zeit}}}{}\ledrightnote{{$\rightarrow$}\textcolor{pink}{Wipplingerstraße}} aufhielt, so dass ich Sie nicht mehr sehen konnte. Ich bitte Sie nun um einige
               Kleinigkeiten, die Sie gelegentlich, ohne Mühe ausrichten, und für die ich Ihnen sehr
               dankbar wäre. Erstens Herrn D\textsuperscript{r}{ }\label{K_L03335-2v}\edtext{\textcolor{blue}{Löwenfeld}{}\ledrightnote{\textcolor{blue}{Raphael Löwenfeld}} bestens von mir zu grüßen}{\lemma{\textnormal{\emph{Löwenfeld … grüßen}}}\Cendnote{\textnormal{\textcolor{blue}{Schnitzler} sah \textcolor{blue}{Raphael Löwenfeld} am 15. 10. 1902 und am 17. 10. 1902.}}}\label{K_L03335-2h},
               und ihm zu sagen, dass ich seinen \label{K_L03335-3v}\edtext{\textcolor{green}{Aufsatz}{}\ledrightnote{{$\rightarrow$}\textcolor{green}{[Aufsatz über volkstümliche Klassikervorstellungen]}} über volksthümliche
               Claßikervorstellungen schon sehnlichst erwarte}{\lemma{\textnormal{\emph{Aufsatz … erwarte}}}\Cendnote{\textnormal{nicht nachgewiesen}}}\label{K_L03335-3h}. Dann erkundigen Sie sich, bitte,
               nach dem Schauspieler \textcolor{blue}{Paul Paschen}{}\ledrightnote{\textcolor{blue}{Paul Paschen}} (\textcolor{brown}{Schillertheater}{}\ledrightnote{\textcolor{brown}{Schiller-Theater}}) was das für ein Mensch ist. Ich
               habe durch \label{K_L03335-4v}\edtext{Geh. Rt.}{\lemma{\textnormal{\emph{Geh. Rt.}}}\Cendnote{\textnormal{Geheimrat}}}\label{K_L03335-4h}{ }\textcolor{blue}{Forster}{}\ledrightnote{\textcolor{blue}{Forster}} einen \label{K_L03335-5v}\edtext{\textcolor{green}{Artikel}{}\ledrightnote{{$\rightarrow$}\textcolor{green}{[Artikel über Kulissenton]}} von ihm bekommen über
               die Schweinerei des Coulissentones}{\lemma{\textnormal{\emph{Artikel … Coulissentones}}}\Cendnote{\textnormal{nicht
                  nachgewiesen}}}\label{K_L03335-5h}. Zuletzt noch – wenn \label{K_L03335-6v}\edtext{bei \textcolor{brown}{Fischer}{}\ledrightnote{\textcolor{brown}{S. Fischer Verlag}} eine
               endgültige Entscheidung}{\lemma{\textnormal{\emph{bei … Entscheidung}}}\Cendnote{\textnormal{Bezug auf die
                  Veröffentlichung von \textcolor{blue}{Salten}s \emph{\textcolor{green}{Die kleine Veronika}} bei \emph{\textcolor{brown}{S.
                     Fischer}}, siehe A. S.: \emph{Tagebuch}, 15. 10. 1902 und Arthur Schnitzler an Felix Salten, 16. 10. 1902}}}\label{K_L03335-6h} getroffen ist, depeschiren Sie mir, bitte. Ich bin sehr neugierig, wie Sie
               sich leicht denken können. Ich muß nun den {\pb}»\textcolor{green}{Moloch}{}\ledrightnote{\textcolor{green}{Der Moloch}}« trotzdem ich ihn das erste Mal refüsirt habe, \label{K_L03335-7v}\edtext{\textcolor{green}{besprechen}{}\ledrightnote{{$\rightarrow$}\textcolor{green}{Die Zeit}}}{\lemma{\textnormal{\emph{besprechen}}}\Cendnote{\textnormal{\textcolor{blue}{Felix Salten}: \emph{\textcolor{green}{Ein Gesellschaftsroman}}. In: \emph{\textcolor{green}{Die Zeit}}, Jg. 1, Nr. 81, 19. 12. 1902, Morgenblatt, S. 1–2.}}}\label{K_L03335-7h}. \textcolor{blue}{Hugo Ganz}{}\ledrightnote{\textcolor{blue}{Hugo Ganz}} hätte ihn übel zugerichtet, und bat mich schließlich
               darum, weil er \label{K_L03335-8v}\edtext{\textcolor{blue}{Herzl}{}\ledrightnote{\textcolor{blue}{Theodor Herzl}}’s \substVorne{}\textsuperscript{r}\substDazwischen{}R\substHinten{}oman »\textcolor{green}{Altneuland}{}\ledrightnote{\textcolor{green}{Altneuland. Roman}}« übernommen}{\lemma{\textnormal{\emph{Herzl’s … übernommen}}}\Cendnote{\textnormal{\textcolor{blue}{Lector} [ = \textcolor{blue}{Hugo Ganz}]: \emph{\textcolor{green}{»Altneuland«}}. In: \emph{\textcolor{green}{Die Zeit}},
                     Jg. 1, Nr. 39, 5. 11. 1902, Morgenblatt,
                     S. 1–2.}}}\label{K_L03335-8h} hat. Ich habe aufmerksam gemacht, dass ich das \textcolor{green}{Buch}{}\ledrightnote{{$\rightarrow$}\textcolor{green}{Altneuland. Roman}} nicht loben kann, und da
               man daran keinen Anstoß nahm, habe ich weiter keine Ursache, mit meiner ganzen
               Meinung über \label{K_L03335-9v}\edtext{\textcolor{blue}{W.}{}\ledrightnote{\textcolor{blue}{Jakob Wassermann}}}{\lemma{\textnormal{\emph{W.}}}\Cendnote{\textnormal{\textcolor{blue}{Jakob Wassermann}}}}\label{K_L03335-9h} zurückzuhalten. Bei alledem hat \textcolor{blue}{W.}{}\ledrightnote{\textcolor{blue}{Jakob Wassermann}}
               noch Glück. Erstens ist er aus \textcolor{blue}{Ganz}{}\ledrightnote{\textcolor{blue}{Hugo Ganz}}’ Händen
               entwischt, zweitens nützt ihm die Raserei \textcolor{blue}{Trebitsch}{}\ledrightnote{\textcolor{blue}{Siegfried Trebitsch}}’s bei mir, der schon glaubt, der Tag der nächsten Woche, an
               welchem mein \textcolor{green}{Moloch-F.}{}\ledrightnote{{$\rightarrow$}\textcolor{green}{Die Zeit}}
               erscheint, sei der Tag des Herrn \textcolor{blue}{Trebitsch}{}\ledrightnote{\textcolor{blue}{Siegfried Trebitsch}}.\pend
           
\pstart
           \label{K_L03335-10v}\edtext{\textcolor{blue}{Gettke}{}\ledrightnote{\textcolor{blue}{Ernst Gettke}} ist seit c\textsuperscript{a} 14 Tagen
               im Besitz Ihres Vertrages}{\lemma{\textnormal{\emph{Gettke … Vertrages}}}\Cendnote{\textnormal{siehe A. S.: \emph{Tagebuch}, 29. 10. 1902}}}\label{K_L03335-10h}. Ich besuche ihn heute, und mache ihm von der
               inzwischen eingetretenen \label{K_L03335-11v}\edtext{Änderung der
                  Dinge}{\lemma{\textnormal{\emph{Änderung der
                  Dinge}}}\Cendnote{\textnormal{Bezug auf eine mögliche
                  Aufführung der \emph{\textcolor{green}{Liebelei}}, für die das \emph{\textcolor{brown}{Burgtheater}} noch das ausschließliche
                  Aufführungsrecht hatte, vgl. Arthur Schnitzler an Felix Salten, 16. 10. 1902. Die Premiere am \textcolor{pink}{Raimundtheater} fand am
                     7. 3. 1903 statt.}}}\label{K_L03335-11h} Mittheilung. Das
               schiebt allerdings die Premiere im \textcolor{pink}{R. Th}{}\ledrightnote{\textcolor{pink}{Raimund-Theater}}. ein
               wenig hinaus!\pend
           
\pstart
           Hoffentlich schreiben Sie mir bald! \pend
           
\pstart
           Herzlichst Ihr {\\[\baselineskip]}\spacefill\mbox{Salten}\pend
           \leftskip=0em{}\endnumbering\briefempfaengerindex{Schnitzler, Arthur@\textsc{Schnitzler, Arthur}!zzzSalten, Felix@\emph{von Felix Salten}!1902-10-153@{15. 10. 1902}|)be}\mylabel{h}  \normalsize

\doendnotes{C}
\bigskip
\vfill

\clearpage

\footnotesize

\lohead{\textsc{register}}

% Definiere theindex-Environment komplett neu ohne reledmac
\makeatletter
\renewenvironment{theindex}{%
  \section*{\indexname}%
  \setlength{\parindent}{0pt}%
  \setlength{\parskip}{0pt plus 0.3pt}%
  \let\item\@idxitem
}{%
  \clearpage
}
\makeatother

\IfFileExists{\jobname-pw.ind}{\input{\jobname-pw.ind}}{}

\end{document}

      