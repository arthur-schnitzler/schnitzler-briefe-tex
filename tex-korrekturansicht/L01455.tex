%% latex-korrekturansicht-vorspann.tex
%% Vorspann für die Korrekturansicht.
%% Lädt die gemeinsame Datei latex-vorspann.tex mit gesetztem Schalter.

\newif\ifkorrekturansicht
\korrekturansichttrue

\input{../tex-inputs/latex-vorspann}


               \section[Arthur Schnitzler an Hugo von Hofmannsthal, 15. 10. 1904]{ Arthur Schnitzler an Hugo von Hofmannsthal, 15. 10. 1904}\nopagebreak\mylabel{v}\rehead{ }\normalsize\beginnumbering\briefempfaengerindex{Hofmannsthal, Hugo von@\textsc{Hofmannsthal, Hugo von}!zzzSchnitzler, Arthur@\emph{von Arthur Schnitzler}!1904-10-151@{15. 10. 1904}|(be} \toendnotes[C]{\smallbreak\pagebreak[2]} \Standort{FDH, Hs-30885,117.}
\physDesc{Brief, 5 Blätter, 5 Seiten
\newline{}Schreibmaschine
\newline{}Handschrift: schwarze Tinte, deutsche Kurrent (\noindent{}Korrekturen, Paginierung und Schluss)\newline{}Ordnung: mit Bleistift von Schnitzler ab dem zweiten Blatt mutmaßlich bei
                                 der Durchsicht der Briefe 1929 jeweils datiert: »15/10 904« }\buchAbdrucke{\weitereDrucke{Hugo von Hofmannsthal, Arthur Schnitzler: \emph{Briefwechsel}. Hg. Therese Nickl und Heinrich Schnitzler. Frankfurt am Main: \emph{S. Fischer} 1964, S. 206.} }\toendnotes[C]{\smallbreak}\pstart
           \raggedleft{}{\pb}\textcolor{pink}{Wien}{}\ledrightnote{\textcolor{pink}{Wien}}, 15. Oktober 1904.\pend
           \pstart{}Lieber Hugo!\pend\pstart
           Dass Sie \textcolor{blue}{Lindemann}{}\ledrightnote{\textcolor{blue}{Gustav Lindemann}} Ihre Stücke verweigerten,
               wundert mich, denn dazu liegt meiner Empfindung nach keine Ursache vor. \textcolor{blue}{Fischer}{}\ledrightnote{\textcolor{blue}{Samuel Fischer}} schrieb mir vor \label{K_L01455_1v}\edtext{Monaten}{\lemma{\textnormal{\emph{Monaten}}}\Cendnote{\textnormal{Im Januar 1904 hatten Verhandlungen über eine
                  Tournee \textcolor{blue}{Lindemann}s, bei der \emph{\textcolor{green}{Der einsame Weg}} gegeben werden sollte, stattgefunden.
                  Ab Mai beherrscht die Frage, ob es eine eine Vorauszahlung hätte
                  geben sollen, die Korrespondenz \textcolor{blue}{Schnitzler}s mit
                     \textcolor{blue}{Fischer}. Am 29. 8. 1904
                  schreibt \textcolor{blue}{Fischer} besagte Aufforderung, dass
                  die Verlagsautoren gemeinsam agieren sollen.}}}\label{K_L01455_1h}, er wolle seinen Autoren das
               Ansinnen stellen, aus Ursache des bewussten Streitfalles zwischen ihm und \textcolor{blue}{L.}{}\ledrightnote{\textcolor{blue}{Gustav Lindemann}}, resp. zwischen mir und \textcolor{blue}{L.}{}\ledrightnote{\textcolor{blue}{Gustav Lindemann}} in Betreff des »\textcolor{green}{Einsamen
                  Wegs}{}\ledrightnote{\textcolor{green}{Der einsame Weg. Schauspiel in fünf Akten}}«, \introOben{}dem \textcolor{blue}{L.}{}\ledrightnote{\textcolor{blue}{Gustav Lindemann}}ſchen
                  Unternehmen\introOben{} ihre dramatischen Arbeiten bis auf Weiteres zu verweigern. Ich
               sprach mich mit Entschiedenheit dagegen aus, da mir jede Art von Solidarität ziemlich
               zuwider ist und ich besonders in dem vorliegenden Fall es auch von jedem andern Autor
               unrichtig gefunden hätte, aus einer rein \introOben{}privat-\introOben{}prozessualen Sache eine \introOben{}öffentliche\introOben{} Affäre zu
               machen und damit vielleicht \strikeout{noch} andere Leute, die
               die ganze Geschichte nicht interessiert, materiell zu schädigen. Damit erledigt sich
               Ihre Frage von selbst, und ich bit{\pb}te Sie nur, ohne jede
               Rücksicht auf mich, auch bei \textcolor{blue}{Lindemann}{}\ledrightnote{\textcolor{blue}{Gustav Lindemann}} Ihre
               Stücke ganz nach Gutdünken zu placieren.\pend
           \pstart
           Aber sonst steht die Sache nicht so einfach, und \textcolor{blue}{Lindemann}{}\ledrightnote{\textcolor{blue}{Gustav Lindemann}} ist gewiss nicht so frei von Schuld, als es im Brief des Fräulein
                  \textcolor{blue}{Dumont}{}\ledrightnote{\textcolor{blue}{Louise Dumont}} an Sie in allerbestem Glauben
               dargestellt wird.\pend
           \pstart
           Insbesondere handelt es sich ja darum, dass \textcolor{blue}{L.}{}\ledrightnote{\textcolor{blue}{Gustav Lindemann}}
               nach der matten Aufnahme des \textcolor{green}{Stück}{}\ledrightnote{→\textcolor{green}{Der einsame Weg. Schauspiel in fünf Akten}}s durch das \textcolor{pink}{Berlin}{}\ledrightnote{\textcolor{pink}{Berlin}}er Publikum weder
               von einer vorher, noch von einer nachher zu zahlenden Garantiesumme etwas wissen
               wollte, trotzdem vor der Aufführung – ich glaube, am Tage der Aufführung – ein
               Telegramm \introOben{}von ihm\introOben{} eingelaufen war, das sich mit den letzten
               Bedingungen \textcolor{blue}{Fischers}{}\ledrightnote{\textcolor{blue}{Samuel Fischer}} einverstanden erklärte, –
               womit nicht nur nach allgemeinem Usus, sondern auch nach dem Urteil juridischer
               Sachverständiger, ein rechtsgiltiger Vertrag zustande gekommen war; \introOben{}–\introOben{} und dass sich \textcolor{blue}{Fischer}{}\ledrightnote{\textcolor{blue}{Samuel Fischer}}
               durchaus nicht hütet, die Angelegenheit auf dem Klageweg zu erledi{\pb}gen, /wie Frl. \textcolor{blue}{Dumont}{}\ledrightnote{\textcolor{blue}{Louise Dumont}} in ihrem Brief sagt/ ersehen Sie am besten aus den zwei Briefen, die
               ich Ihnen hier beilege und um deren Rücksendung ich Sie bitte, und aus denen sie
               erstens ersehen, dass Justizrat \textcolor{blue}{Jonas}{}\ledrightnote{\textcolor{blue}{Paul Jonas}} die
               Forderung der sofortigen Zahlung der 5000 M. für begründet hält, und zweitens dass
                  \textcolor{blue}{Fischer}{}\ledrightnote{\textcolor{blue}{Samuel Fischer}} nur meine Einwilligung abwartet, um
               den Prozess gegen \textcolor{blue}{Lindemann}{}\ledrightnote{\textcolor{blue}{Gustav Lindemann}} einzuleiten. Diese
               Einwilligung werde ich ihm natürlich nicht versagen.\pend
           \pstart
           Worin ich \textcolor{blue}{Fischer}{}\ledrightnote{\textcolor{blue}{Samuel Fischer}} Unrecht gebe, ist eigentlich
               nur, dass er nicht gleich zu Beginn der Verhandlungen – lange vor Aufführung des \textcolor{green}{Stücks}{}\ledrightnote{→\textcolor{green}{Der einsame Weg. Schauspiel in fünf Akten}} in \textcolor{pink}{Berlin}{}\ledrightnote{\textcolor{pink}{Berlin}} – den \textcolor{blue}{Lindemann}{}\ledrightnote{\textcolor{blue}{Gustav Lindemann}}’schen
               Antrag in seinem ganzen Umfang /5000 M. Garantie und Aufführung des \textcolor{green}{Stücks}{}\ledrightnote{→\textcolor{green}{Der einsame Weg. Schauspiel in fünf Akten}} in allen von \textcolor{blue}{L.}{}\ledrightnote{\textcolor{blue}{Gustav Lindemann}} angegebenen Städten/ angenommen hat, obwol ich ihm
               telegraphisch meine entschiedene Zustimmung kundgab, sondern dass er sich dann erst
               in Verhandlungen über einzelne Städte einliess, die von der Tournée ausgeschlossen
               sein sollte\introOben{}n\introOben{}. Aber \introOben{}»\introOben{}unvornehm\introOben{}«\introOben{} kann ich das auch nicht finden.\pend
           \pstart
           {\pb}Was aber nun eine \uline{vorherige} Zahlung der Garantiesumme anlangt, so würde ich zu dieser
               Forderung in einem ähnlichen Fall meinen Vertreter neuerdings autorisieren; denn
               gerade die in dem Brief des Frl. \textcolor{blue}{Dumont}{}\ledrightnote{\textcolor{blue}{Louise Dumont}}
               angeführten Daten beweisen, wie gering die finanzielle Sicherheit ist, die in einem
               Unternehmen in der Art des \textcolor{blue}{Lindemann}{}\ledrightnote{\textcolor{blue}{Gustav Lindemann}}’schen,
               selbst bei den besten Absichten und den reinsten künstlerischen Intentionen, den
               Autoren geboten werden kann.\pend
           \pstart
           Uebrigens hätte ja \textcolor{blue}{Lindemann}{}\ledrightnote{\textcolor{blue}{Gustav Lindemann}} sich mindestens zu
               einer teilweisen vorherigen Zahlung verstehen können; aber, ganz im Gegenteil, – und
               dies ist wol das Wichtigste bei der Betrachtung des ganzen Streitfalls –, nach der
                  \textcolor{pink}{Berlin}{}\ledrightnote{\textcolor{pink}{Berlin}}er Première wollte er, trotz des vor der
               Première eingelangten vertragsgleichen Telegramms, weder von einer vorher, noch von
               einer nachher zu zahlenden Garantie, noch überhaupt von einer Aufführung des \textcolor{green}{Stück}{}\ledrightnote{→\textcolor{green}{Der einsame Weg. Schauspiel in fünf Akten}}es im Verlauf seiner Tournée
               etwas wissen.\pend
           \pstart
           {\pb}Bitte, lieber Hugo, grüssen Sie Frl. \textcolor{blue}{Dumont}{}\ledrightnote{\textcolor{blue}{Louise Dumont}} herzlich und teilen Sie ihr doch in Ihrer Antwort auch
               mit, was ich Ihnen gleich im Beginn dieses Briefs \substVorne{}\textsuperscript{erzählt}{\allowbreak}\substDazwischen{}gesagt\substHinten{} habe: dass es durchaus meinen Intentionen widersprach und widerspricht, wenn
                  \textcolor{blue}{Fischer}{}\ledrightnote{\textcolor{blue}{Samuel Fischer}} aus Anlass des bekannten Streitfalls
               dem neuen Unternehmen auch Stücke seiner anderen Autoren verweigert, dass mir im
               übrigen aber das Vorgehen \textcolor{blue}{Fischers}{}\ledrightnote{\textcolor{blue}{Samuel Fischer}} in meiner
               Sache einwandfrei erscheint.\pend
           \pstart
           {[}hs.:{]} Herzliche Grüße und auf baldigs Wiederſehen\damage{.} Mit meinem »\textsc{burlesken} Abend« bei \textcolor{blue}{Rhardt}{}\ledrightnote{\textcolor{blue}{Max Reinhardt}} iſt’s nichts. Er will die \textcolor{green}{Familienſcene}{}\ledrightnote{→\textcolor{green}{Das Haus Delorme. Eine Familienszene}} allein, die ich aber lieber für beſſere
               Gelegenheit zurückbehalte. Über \textcolor{green}{Kakadu}{}\ledrightnote{\textcolor{green}{Der grüne Kakadu. Groteske in einem Akt}}–\textcolor{green}{Abenteurer}{}\ledrightnote{\textcolor{green}{Der Abenteurer und die Sängerin oder Die Geschenke des Lebens}} iſt noch kein Telegramm eingelangt.\pend
           \pstart Ihr \spacefill\mbox{A.}\pend{}\endnumbering\briefempfaengerindex{Hofmannsthal, Hugo von@\textsc{Hofmannsthal, Hugo von}!zzzSchnitzler, Arthur@\emph{von Arthur Schnitzler}!1904-10-151@{15. 10. 1904}|)be}\mylabel{h}  \normalsize

\doendnotes{C}
\bigskip
\vfill

\clearpage

\footnotesize

\lohead{\textsc{register}}

% Definiere theindex-Environment komplett neu ohne reledmac
\makeatletter
\renewenvironment{theindex}{%
  \section*{\indexname}%
  \setlength{\parindent}{0pt}%
  \setlength{\parskip}{0pt plus 0.3pt}%
  \let\item\@idxitem
}{%
  \clearpage
}
\makeatother

\IfFileExists{\jobname-pw.ind}{\input{\jobname-pw.ind}}{}

\end{document}

      