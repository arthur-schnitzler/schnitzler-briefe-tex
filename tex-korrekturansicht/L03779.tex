%% latex-korrekturansicht-vorspann.tex
%% Vorspann für die Korrekturansicht.
%% Lädt die gemeinsame Datei latex-vorspann.tex mit gesetztem Schalter.

\newif\ifkorrekturansicht
\korrekturansichttrue

\input{../tex-inputs/latex-vorspann}


\section[Arthur Schnitzler an Stefan Zweig, 2. 12. 1914]{L03779 Arthur Schnitzler an Stefan Zweig, 2. 12. 1914}
\nopagebreak\mylabel{L03779v}
\rehead{ }\normalsize\beginnumbering\briefempfaengerindex{, @\textsc{, }!zzz, @\emph{von  }!1914-12-021@{2. 12. 1914}|(be}
\toendnotes[C]{\smallbreak\pagebreak[2]}\Standort{Jerusalem, National Library of Israel, ARC. Ms. Var. 305 1 58 Stefan Zweig Collection.}
\physDesc{Brief, 1 Blatt, 2 Seiten, 1646 Zeichen
\newline{}Schreibmaschine
\newline{}Handschrift: schwarze Tinte, lateinische Kurrent (\noindent{}Korrekturen, Ergänzungen, Unterschrift)}
\buchAbdrucke{\weitereDrucke{Arthur Schnitzler: \emph{Briefe 1913–1931}. Frankfurt am Main: \emph{S. Fischer} 1984, S. 59–62.} }\toendnotes[C]{\smallbreak}
\pstart
           {\pb}\textcolor{gray}{\textbf{Dr. Arthur Schnitzler}}\hfill 2. 12. 1914. \pend
           
\pstart
           \textcolor{gray}{\textbf{\textcolor{pink}{Wien XVIII. Sternwartestrasse 71}\oindex{Wien@\textbf{Wien}!XVIII., Währing@\textbf{XVIII., Währing}!Sternwartestraße 71@\textbf{Sternwartestraße 71}, \emph{Wohngebäude}|pw}{}\ledrightnote{\textcolor{pink}{Sternwartestraße 71}}}}\pend
           
\pstart\center{}Lieber Herr Doktor.\pend\vspace{0.5em}
\pstart
           Hier beigeschlossen ein Exemplar der \textcolor{green}{Erklärung}\pwindex{Schnitzler, Arthur 15. 5. 1862 Wien – 21. 10. 1931 ebd.@\textsc{Schnitzler, Arthur} (15. 5. 1862 Wien – 21. 10. 1931 ebd.), \emph{Schriftsteller, Mediziner}!Brief Artur Schnitzlers@\strich\emph{Ein Brief Artur Schnitzlers}|pwv}{}\ledrightnote{{$\rightarrow$}\emph{\textcolor{green}{Ein Brief Artur Schnitzlers}}} mit den besprochenen Aenderungen. Einen andern,
               einen wahrhaft bekennerischen Ton, vermöchte ich kaum zu finden. Je mehr man über die
               Sache nachdenkt, umso dümmer kommt sie einem vor. Ich wollte Sie noch fragen: Was\introOben{},\introOben{} denken Sie, soll nun \label{K_L03779-1v}\edtext{\textcolor{blue}{Rolland}\pwindex{Rolland, Romain 29.\,1.\,1866 Clamecy – 30.\,12.\,1944 Vézelay@\textsc{Rolland, Romain} (29.\,1.\,1866 Clamecy – 30.\,12.\,1944 Vézelay), \emph{Schriftsteller}|pw}{}\ledrightnote{\textcolor{blue}{Romain Rolland}}}{\lemma{\textnormal{\emph{Rolland}}}\Cendnote{\textnormal{\textcolor{blue}{Zweig}\pwindex{Zweig, Stefan 28.\,11.\,1881 Wien – 23.\,2.\,1942 Petrópolis@\textsc{Zweig, Stefan} (28.\,11.\,1881 Wien – 23.\,2.\,1942 Petrópolis), \emph{Schriftsteller}|pwk} schrieb am 5. 12. 1914 an \textcolor{blue}{Rolland}\pwindex{Rolland, Romain 29.\,1.\,1866 Clamecy – 30.\,12.\,1944 Vézelay@\textsc{Rolland, Romain} (29.\,1.\,1866 Clamecy – 30.\,12.\,1944 Vézelay), \emph{Schriftsteller}|pwk}: »\textcolor{blue}{Arthur Schnitzler} sendet Ihnen bei diesem Anlass seine
                        respectvollen Grüße (er wohnt, wenn Sie sie erwidern wollen, \textcolor{pink}{Wien XVIII, Sternwartestraße 71}\oindex{Wien@\textbf{Wien}!XVIII., Währing@\textbf{XVIII., Währing}!Sternwartestraße 71@\textbf{Sternwartestraße 71}, \emph{Wohngebäude}|pw}).
                        Ich freue mich, dass nun ein neuer Beweis in Ihren Händen ist, wie sehr
                        unsere Besten sich bemühen, gerecht zu bleiben. Lassen Sie sich durch
                        einzelne Manifestationen des Hasses nicht verstimmen: gerade extreme Naturen
                        verlieren in solchen Zeiten am leichtesten das innere Gleichgewicht. Und es
                        bedarf einer großen moralischen Stabilität, um aufrecht zu bleiben in diesen
                        Stürmen!{ / }{[}\ldots{]}{ / }PS: Das \textcolor{green}{Original}\pwindex{Schnitzler, Arthur 15. 5. 1862 Wien – 21. 10. 1931 ebd.@\textsc{Schnitzler, Arthur} (15. 5. 1862 Wien – 21. 10. 1931 ebd.), \emph{Schriftsteller, Mediziner}!Brief Artur Schnitzlers@\strich\emph{Ein Brief Artur Schnitzlers}|pwv}{ }\textcolor{blue}{Schnitzlers} könnte auch in einer \textcolor{green}{deutschen Schweizer Zeitung}\pwindex{Neue Zürcher Zeitung@\emph{Neue Zürcher Zeitung}|pwv}
                        erscheinen! Bitte dann um ein Exemplar!« \textcolor{blue}{Romain Rolland}\pwindex{Rolland, Romain 29.\,1.\,1866 Clamecy – 30.\,12.\,1944 Vézelay@\textsc{Rolland, Romain} (29.\,1.\,1866 Clamecy – 30.\,12.\,1944 Vézelay), \emph{Schriftsteller}|pwk}, \textcolor{blue}{Stefan Zweig}\pwindex{Zweig, Stefan 28.\,11.\,1881 Wien – 23.\,2.\,1942 Petrópolis@\textsc{Zweig, Stefan} (28.\,11.\,1881 Wien – 23.\,2.\,1942 Petrópolis), \emph{Schriftsteller}|pwk}: \emph{Von Welt zu Welt. Briefe
                        einer Freundschaft 1914–1918}. Mit einem Begleitwort von Peter
                     Handke. Aus dem Französischen von Eva und Gerhard Schwewe (Briefe Rollands) und
                     Christel Gersch (Briefe Zweigs). Berlin: \emph{Aufbau
                        Verlag}{ }2014. }}}\label{K_L03779-1} mit unseren \textcolor{green}{Erklärungen}\pwindex{Schnitzler, Arthur 15. 5. 1862 Wien – 21. 10. 1931 ebd.@\textsc{Schnitzler, Arthur} (15. 5. 1862 Wien – 21. 10. 1931 ebd.), \emph{Schriftsteller, Mediziner}!Brief Artur Schnitzlers@\strich\emph{Ein Brief Artur Schnitzlers}|pwv}{}\ledrightnote{{$\rightarrow$}\emph{\textcolor{green}{Ein Brief Artur Schnitzlers}}} tun? Sie ins \textcolor{pink}{Französi\introOben{}s\introOben{}che}\oindex{Frankreich@\textbf{Frankreich}|pw}{}\ledrightnote{\textcolor{pink}{Frankreich}}
               übersetzen und eventuell nicht nur an das \textcolor{brown}{Journal de Gen\substVorne{}\textsuperscript{é}\substDazwischen{}è\substHinten{}ve}\orgindex{Journal de Genève@Journal de Genève|pw}{}\ledrightnote{\textcolor{brown}{Journal de Genève}}, sondern sie auch an \textcolor{pink}{französische}\oindex{Frankreich@\textbf{Frankreich}|pw}{}\ledrightnote{\textcolor{pink}{Frankreich}} Journale weitergeben? Könnte er es auch übernehmen den
               Erklärungen in ein deutsches \label{K_L03779-2v}\edtext{\textcolor{pink}{schweizer}\oindex{Schweiz@\textbf{Schweiz}|pw}{}\ledrightnote{\textcolor{pink}{Schweiz}} Journal}{\lemma{\textnormal{\emph{schweizer Journal}}}\Cendnote{\textnormal{\emph{\textcolor{green}{Ein Brief Artur Schnitzlers}\pwindex{Schnitzler, Arthur 15. 5. 1862 Wien – 21. 10. 1931 ebd.@\textsc{Schnitzler, Arthur} (15. 5. 1862 Wien – 21. 10. 1931 ebd.), \emph{Schriftsteller, Mediziner}!Brief Artur Schnitzlers@\strich\emph{Ein Brief Artur Schnitzlers}|pwk}}. In: \emph{\textcolor{green}{Neue Zürcher Zeitung}\pwindex{Neue Zürcher Zeitung@\emph{Neue Zürcher Zeitung}|pwk}}, Jg. 135, Nr. 1700,
                        22. 12. 1914, 2. Mittagsblatt, S. 2.}}}\label{K_L03779-2} Aufnahme zu
               verschaffen? Mir fällt eben ein, dass wir neulich über Regierungsrat \textcolor{blue}{Winternitz}\pwindex{Winternitz, Jakob von 3.\,3.\,1843 Horažďovice – 26.\,1.\,1921 Wien@\textsc{Winternitz, Jakob von} (3.\,3.\,1843 Horažďovice – 26.\,1.\,1921 Wien), \emph{Ministerialbeamter}|pw}{}\ledrightnote{\textcolor{blue}{Jakob von Winternitz}} nicht gesprochen haben. Bitte um
               eine Zeile, wann ich Sie anrufen dürfte. Den \label{K_L03779-3v}\edtext{Appell an die Blätter}{\lemma{\textnormal{\emph{Appell an die Blätter}}}\Cendnote{\textnormal{Arthur Schnitzler an Stefan Zweig, 27. 11. 1914.}}}\label{K_L03779-3}, mit dem meine vorige \textcolor{green}{Erklärung}\pwindex{Schnitzler, Arthur 15. 5. 1862 Wien – 21. 10. 1931 ebd.@\textsc{Schnitzler, Arthur} (15. 5. 1862 Wien – 21. 10. 1931 ebd.), \emph{Schriftsteller, Mediziner}!Brief Artur Schnitzlers@\strich\emph{Ein Brief Artur Schnitzlers}|pwv}{}\ledrightnote{{$\rightarrow$}\emph{\textcolor{green}{Ein Brief Artur Schnitzlers}}} schloss, {\pb}(\label{K_L03779-4v}\edtext{bitte \introOben{}die\introOben{} beide\introOben{}n\introOben{} Exemplare zu
                  vernichten}{\lemma{\textnormal{\emph{bitte … vernichten}}}\Cendnote{\textnormal{\textcolor{blue}{Zweig}\pwindex{Zweig, Stefan 28.\,11.\,1881 Wien – 23.\,2.\,1942 Petrópolis@\textsc{Zweig, Stefan} (28.\,11.\,1881 Wien – 23.\,2.\,1942 Petrópolis), \emph{Schriftsteller}|pwk} kam der Bitte nicht nach, die erste Fassung 
                     ist als Beilage von Arthur Schnitzler an Stefan Zweig, 27. 11. 1914 überliefert.}}}\label{K_L03779-4}) habe ich diesmal weggelassen. Ich glaube, man bedarf
               ihrer nicht.\pend
           
\pstart
           Ich hatte heute den sonderbaren \label{K_L03779-5v}\edtext{Traum}{\lemma{\textnormal{\emph{Traum}}}\Cendnote{\textnormal{Vgl. A. S.: \emph{Tagebuch}, 2. 12. 1914.}}}\label{K_L03779-5}, dass
               ich mit Ihnen in einem offenen Fiaker auf erhöhter Strasse durch eine irgendwie
               orientalische Stadt fuhr; \substVorne{}\textsuperscript{s }\substDazwischen{}S\substHinten{}ie transportierten mich nämlich nach \textcolor{pink}{Sibirien}\oindex{Sibirien@\textbf{Sibirien}, \emph{Region}|pw}{}\ledrightnote{\textcolor{pink}{Sibirien}}, was ein wenig dadurch gemildert war, dass der Weg zuerst durchs
                  \textcolor{pink}{Helenenthal}\oindex{Helenental@\textbf{Helenental}, \emph{Tal}|pw}{}\ledrightnote{\textcolor{pink}{Helenental}} führen sollte. Ich war nur auf
               sechs Monate verbannt, hatte aber den leisen Verdacht gegen Sie, dass Sie mich für
               immer dort lassen wollten. Im übrigen sahen Sie, was eine allgemein bekannte Tatsache
               war, einem Grafen Schönstein wie einem Zwillingsbruder ähnlich. Dieser Graf wurde
               auch irgendwie sichtbar, sah Ihnen natürlich gar nicht ähnlich, hatte einen offenen
               Ueberzieher mit Pelz, trug einen Zwicker und sah verdrossen drein. Nun deuten Sie\substVorne{}\textsuperscript{.}\substDazwischen{}!\substHinten{}\pend
           
\pstart
           Herzlichst grüssend{\\[\baselineskip]}Ihr{\\[\baselineskip]}\spacefill\mbox{{[}hs.:{]} Arthur Schnitzler}\pend
           \leftskip=0em{}\selectlanguage{ngerman}\endnumbering\briefempfaengerindex{, @\textsc{, }!zzz, @\emph{von  }!1914-12-021@{2. 12. 1914}|)be}\mylabel{L03779h}  \normalsize

\doendnotes{C}
\bigskip
\vfill

\clearpage

\footnotesize

\lohead{\textsc{register}}

% Definiere theindex-Environment komplett neu ohne reledmac
\makeatletter
\renewenvironment{theindex}{%
  \section*{\indexname}%
  \setlength{\parindent}{0pt}%
  \setlength{\parskip}{0pt plus 0.3pt}%
  \let\item\@idxitem
}{%
  \clearpage
}
\makeatother

\IfFileExists{\jobname-pw.ind}{\input{\jobname-pw.ind}}{}

\end{document}

      