%% latex-korrekturansicht-vorspann.tex
%% Vorspann für die Korrekturansicht.
%% Lädt die gemeinsame Datei latex-vorspann.tex mit gesetztem Schalter.

\newif\ifkorrekturansicht
\korrekturansichttrue

\input{../tex-inputs/latex-vorspann}


               \section[Paul Goldmann an Arthur Schnitzler, 25. 10. {[}1894{]}]{ Paul Goldmann an Arthur Schnitzler, 25. 10. {[}1894{]}}\nopagebreak\mylabel{v}\rehead{ }\normalsize\beginnumbering\briefempfaengerindex{Schnitzler, Arthur@\textsc{Schnitzler, Arthur}!zzzGoldmann, Paul@\emph{von Paul Goldmann}!1894-10-251@{25. 10. {[}1894{]}}|(be} \toendnotes[C]{\smallbreak\pagebreak[2]} \Standort{DLA, A:Schnitzler, HS.NZ85.1.3164.}
\physDesc{Brief, 3 Blätter, 12 Seiten
\newline{}Handschrift: schwarze Tinte, deutsche Kurrent
\newline{}Schnitzler: 1) mit Bleistift auf dem ersten Blatt die Jahreszahl »94« vermerkt 2) mit rotem Buntstift fünf Unterstreichungen}\toendnotes[C]{\smallbreak}\pstart
           \noindent{}{\pb}\textcolor{gray}{\textbf{\textcolor{brown}{Frankfurter Zeitung}{}\ledrightnote{\textcolor{brown}{Frankfurter Zeitung}}.}}\hfill \textsc{\textcolor{pink}{Paris}{}\ledrightnote{\textcolor{pink}{Paris}}}, 25. Oktober.\pend
           \pstart
           \textcolor{gray}{\textbf{(\textcolor{brown}{Gazette de
                     Francfort}{}\ledrightnote{\textcolor{brown}{Frankfurter Zeitung}}).}}\pend
           \pstart
           \textcolor{gray}{\textbf{\begin{otherlanguage}{french}Fondateur\end{otherlanguage}{ }\textbf{M. \textcolor{blue}{L. Sonnemann}{}\ledrightnote{\textcolor{blue}{Leopold Sonnemann}}}.}}\pend
           \pstart
           \textcolor{gray}{\textbf{\begin{otherlanguage}{french}Journal politique, financier,\end{otherlanguage}}}\pend
           \pstart
           \textcolor{gray}{\textbf{\begin{otherlanguage}{french}commercial et littéraire.\end{otherlanguage}}}\pend
           \pstart
           \textcolor{gray}{\textbf{\begin{otherlanguage}{french}\textbf{Paraissant trois fois par jour}\end{otherlanguage}}}.\pend
           \pstart
           \textcolor{gray}{\textbf{\begin{otherlanguage}{french}\textbf{Bureaux à \textcolor{pink}{Paris}{}\ledrightnote{\textcolor{pink}{Paris}}:}\end{otherlanguage}}}\pend
           \pstart
           \textcolor{gray}{\textbf{\begin{otherlanguage}{french}\textcolor{pink}{\textbf{24. Rue Feydeau}}{}\ledrightnote{\textcolor{pink}{rue Feydeau}}.\end{otherlanguage}}}\pend
           \pstart\center{}Mein lieber Freund,\pend\pstart
           Ich hatte mich ſehr nach einem ausführlichen Briefe von \strikeout{De} Dir geſehnt. Sein Ausbleiben machte mir Sorge, und ich war in meinen
               Grübeleien ſchon zu allerlei traurigen Maximen gelangt. Da kam er endlich, und er
               brachte mir ſoviel Liebes und Gutes, daß ich ihn mit einer wahren Freude geleſen
               habe. Nun wollte ich gleich antworten. Aber ſchlimme Dinge miſchten ſich dazwiſchen.
               Meine Augen ſind ſeit acht Tagen erkrankt. Der Arzt ſcheint eine \textsc{\label{K_mets_Goldmann_94-partII-222v}\edtext{Iritis}{\lemma{\textnormal{\emph{Iritis}}}\Cendnote{\textnormal{Entzündung der
                     Regenbogenhaut}}}\label{K_mets_Goldmann_94-partII-222h}} zu fürchten. {\pb}Die Sache wird täglich
               ſchlimmer; aber es ſind bisher doch nur Vorſymptome da. So habe ich Dir nicht
               geantwortet, nicht weil meine Sehkraft bereits angegriffen iſt, ſondern weil ich
               tief, tief verzweifelt bin. Heut iſt es mir endlich
               gelungen, meine Depreſſion zu überwinden und den ſeeliſchen Rapport mit Dir
               herzuſtellen.\pend
           \pstart
           So laß’ Dich alſo zunächſt von ganzem Herzen beglückwünſchen, daß das \label{K_L02616-1v}\edtext{\textcolor{green}{Werk}{}\ledrightnote{→\textcolor{green}{Liebelei. Schauspiel in drei Akten}} nun endlich
                  vollendet}{\lemma{\textnormal{\emph{Werk … vollendet}}}\Cendnote{\textnormal{Am 14. 10. 1894 las \textcolor{blue}{Schnitzler} die \emph{\textcolor{green}{Liebelei}}{ }\textcolor{blue}{Hugo von Hofmannsthal} und \textcolor{blue}{Felix Salten} vor. Diese urteilten, das Stück sei bis auf
                  wenige Formulierungen fertig. \textcolor{blue}{Schnitzler}
                  hatte die Fertigstellung des Stückes bereits zehn Tage vorher (am 4. 10. 1894) im \emph{\textcolor{green}{Tagebuch}} notiert.}}}\label{K_L02616-1h} iſt. Als wirs ſo
                  \label{K_L02616-2v}\edtext{zuſammen beſprachen}{\lemma{\textnormal{\emph{zuſammen beſprachen}}}\Cendnote{\textnormal{siehe A. S.: \emph{Tagebuch}, 30. 8. 1894}}}\label{K_L02616-2h}, hatte ich die Empfindung, daß Du es {\pb}gut
               machen müßteſt. Es lag in Deinem Ton ſoviel Sicherheit – trotz allen Suchens. \strikeout{\textcolor{gray}{Un}} Und ich fand Dich auch ganz über dem Stoff ſtehend. Die Idee, die Du
               entworfen, iſt glänzend, in all’ ihrer Einfachheit. Daß Du im Stande ſein würdeſt,
               die Form mit Leben zu füllen, war ſicher. Kurzum, ich fuhr weg und erzählte meinem
                  \textcolor{blue}{Onkel}{}\ledrightnote{→\textcolor{blue}{Fedor Mamroth}}: »Du wirſt ſehen, in
               ein, zwei Jahren wird er ſein Meiſterſtück liefern. Darum überraſcht mich nichts am
               Beifall der \textcolor{blue}{Freunde}{}\ledrightnote{→\textcolor{blue}{Felix Salten}{\newline}→\textcolor{blue}{Hugo von Hofmannsthal}}. Mir iſt, als hätten ſie meine Anſicht beſtätigt. Nur möcht’ ichs gerne
               leſen. Dein \label{K_L02616-666v}\edtext{Original-{\pb}Manuſkript}{\lemma{\textnormal{\emph{Original-Manuſkript}}}\Cendnote{\textnormal{\textcolor{blue}{Goldmann} dürfte hier eine (zutreffende)
                  Annahme äußern, nicht ein Urteil nachdem er das Manuskript der Handschrift H\textsuperscript{2} eingesehen hatte. (Vgl. A. S.: \emph{\textcolor{green}{Liebelei}}. Historisch-kritische Ausgabe. Hg.
                     Peter Michael Braunwarth, Gerhard Hubmann und Isabella Schwentner. Berlin,
                     Boston: \emph{de Gruyter}{ }2014 (Werke in historisch-kritischen Ausgaben, hg. Konstanze
                     Fliedl), S. 333–915.)}}}\label{K_L02616-666h} iſt nicht zu entziffern. Aber Du läßt wohl
               noch eine zweite Abſchrift machen. Ich rathe Dir, es zugleich, in einem \textcolor{pink}{Berlin}{}\ledrightnote{\textcolor{pink}{Berlin}}er Theater (\textsc{\textcolor{brown}{\textcolor{blue}{Brahm}{}\ledrightnote{\textcolor{blue}{Otto Brahm}}}{}\ledrightnote{→\textcolor{brown}{Lessing-Theater}}}) \label{K_L02616-3v}\edtext{einzureichen}{\lemma{\textnormal{\emph{einzureichen}}}\Cendnote{\textnormal{\textcolor{blue}{Brahm} leitete das \emph{\textcolor{brown}{Lessing-Theater}}. \textcolor{blue}{Schnitzler} dürfte dem Rat \textcolor{blue}{Goldmann}s nicht gefolgt sein. Stattdessen legt die Korrespondenz zwischen
                  ihm und \textcolor{blue}{Brahm} nahe, dass der
                  Theaterdirektor, nachdem die \emph{\textcolor{green}{Liebelei}} vom \emph{\textcolor{brown}{Burgtheater}} akzeptiert worden war, selbst aktiv
                  wurde.}}}\label{K_L02616-3h}. Dann ſchickſt Du mirs, bitte, vorher; ich gebe Dir mein Wort: in
               drei Tagen haſt Dus wieder. Ich freue mich für Dich, und ich bin glücklich in dem
               Gedanken, wie es jetzt mit Dir vorwärts gehen wird. Dabei bin ich merkwürdiger Weiſe
               gar nicht neidiſch – wie auf alle Anderen – ſondern nur ſroh. Es iſt, als geſchähe in
               meinem eigenen Leben etwas Gutes.\pend
           \pstart
           {\pb}Selbſtverſtändlich mußt Du das \textcolor{green}{Stück}{}\ledrightnote{→\textcolor{green}{Liebelei. Schauspiel in drei Akten}} dem \textcolor{brown}{Burgtheater}{}\ledrightnote{\textcolor{brown}{Burgtheater}}{ }\label{K_L02616-6v}\edtext{einreichen}{\lemma{\textnormal{\emph{einreichen}}}\Cendnote{\textnormal{Am 27. 10. 1894 erhielt \textcolor{blue}{Schnitzler}
                  eine Abschrift der \emph{\textcolor{green}{Liebelei}}, am [31. 10. 1894] gratulierte \textcolor{blue}{Burckhard} und deutete die Annahme an. Sofern
                  es nicht eine weitere Abschrift gab, hatte er also schnell gelesen.}}}\label{K_L02616-6h}. Wenn
               es \textcolor{pink}{Wien}{}\ledrightnote{\textcolor{pink}{Wien}}eriſch iſt, ſo müßte es doch logiſcher
               Weiſe noch beſſer dafür paſſen, als die \strikeout{\textcolor{gray}{×}\-\textcolor{gray}{×}\-\textcolor{gray}{×}\-\textcolor{gray}{×}s}{ }\textcolor{pink}{Berlin}{}\ledrightnote{\textcolor{pink}{Berlin}}eriſchen Stücke (\textsc{\label{K_L02616-345v}\edtext{\textcolor{green}{\textcolor{blue}{Sudermann}{}\ledrightnote{\textcolor{blue}{Hermann Sudermann}}}{}\ledrightnote{→\textcolor{green}{Die Schmetterlingsschlacht. Komödie in 4 Akten}}}{\lemma{\textnormal{\emph{Sudermann}}}\Cendnote{\textnormal{\emph{\textcolor{green}{Die Schmetterlingsschlacht}} von \textcolor{blue}{Hermann Sudermann} hatte am
                        6. 10. 1894 Uraufführung am \emph{\textcolor{brown}{Burgtheater}}.}}}\label{K_L02616-345h}}, \textsc{\label{K_L02616-4567v}\edtext{\textcolor{green}{\textcolor{blue}{Fulda}{}\ledrightnote{\textcolor{blue}{Ludwig Fulda}}}{}\ledrightnote{→\textcolor{green}{Das verlorene Paradies. Schauspiel in drei Aufzügen}}}{\lemma{\textnormal{\emph{Fulda}}}\Cendnote{\textnormal{\emph{\textcolor{green}{Das verlorene Paradies}} von \textcolor{blue}{Ludwig Fulda} wurde erstmals am
                        25. 1. 1891 am \emph{\textcolor{brown}{Burgtheater}}
                     gegeben und befand sich noch 1894 am Spielplan.}}}\label{K_L02616-4567h}}). Daß \label{K_L02616-4v}\edtext{\textsc{\textcolor{blue}{Bahr}{}\ledrightnote{\textcolor{blue}{Hermann Bahr}}} Dich ins \textcolor{brown}{Raimund-Theater}{}\ledrightnote{\textcolor{brown}{Raimund-Theater}}}{\lemma{\textnormal{\emph{Bahr … Raimund-Theater}}}\Cendnote{\textnormal{siehe A. S.: \emph{Tagebuch}, 16. 10. 1894, vgl. Arthur Schnitzler an Richard Beer-Hofmann, 20. 10. 1894}}}\label{K_L02616-4h} weiſen möchte, iſt mir durchaus erklärlich. Das \textcolor{brown}{Burgtheater}{}\ledrightnote{\textcolor{brown}{Burgtheater}} iſt für die große Literatur da \strikeout{du aber} (\textsc{\textcolor{blue}{Bahr}{}\ledrightnote{\textcolor{blue}{Hermann Bahr}}}, \textcolor{green}{Neue Menſchen}{}\ledrightnote{\textcolor{green}{Die neuen Menschen. Ein Schauspiel}}), Du aber ſollſt zum
               Dichter von Volksſtücken geſtempelt werden. Ich bin auch überzeugt, er wird \textsc{\textcolor{blue}{Burckhardt}{}\ledrightnote{\textcolor{blue}{Max Eugen Burckhard}}} gegen Dich zu beeinfluſſen ſuchen, {\pb}der
               Schuft! So ſehr ich dagegen ankämpfe, mein Haß gegen den Burſchen wächſt beinahe
               täglich. Es iſt ein \strikeout{m}{ }\strikeout{unl} unlauterer Menſch. Man braucht ihn nur \label{K_L02616-8v}\edtext{in der »\textcolor{brown}{Zeit}{}\ledrightnote{\textcolor{brown}{Die Zeit. Wiener Wochenschrift}}«}{\lemma{\textnormal{\emph{in der »Zeit«}}}\Cendnote{\textnormal{Das erste Heft erschien am
                     6. 10. 1894 und wöchentlich, so dass \textcolor{blue}{Goldmann} die ersten drei Hefte gekannt haben
                  dürfte.}}}\label{K_L02616-8h} zu beobachten. Alles, was von \textsc{\textcolor{blue}{Kanner}{}\ledrightnote{\textcolor{blue}{Heinrich Kanner}}} kommt, iſt nämlich, originell und muthig. In \label{K_L02616-5v}\edtext{\textsc{\textcolor{blue}{Bahrs}{}\ledrightnote{\textcolor{blue}{Hermann Bahr}}} Reſſort}{\lemma{\textnormal{\emph{Bahrs Reſſort}}}\Cendnote{\textnormal{Dieser verantwortete den
                  Kulturteil.}}}\label{K_L02616-5h} gibt es nichts als berechnetes Laviren, verbunden mit frechem
               literariſchem Pontificiren. Socialpolitiſch und politiſch iſt die \textcolor{brown}{Revüe}{}\ledrightnote{→\textcolor{brown}{Die Zeit. Wiener Wochenschrift}} vorzüglich; literariſch finde ich ſie
               talent- und \strikeout{mit} intereſſelos redigirt; da gibt es nur
               einen \textsc{\textcolor{blue}{Bahr}{}\ledrightnote{\textcolor{blue}{Hermann Bahr}}}, \strikeout{der} alles Andere iſt als Relief befandelt.
                  \strikeout{D\textcolor{gray}{×}\-\textcolor{gray}{×}\-\textcolor{gray}{×}}{ }{\pb}Er wird das ſchöne Unternehmen ſchon umbringen.\pend
           \pstart
           »\label{K_L02616-111v}\edtext{\textcolor{green}{Sterben}{}\ledrightnote{\textcolor{green}{Sterben. Novelle}}}{\lemma{\textnormal{\emph{Sterben}}}\Cendnote{\textnormal{Er bezieht sich auf den ersten Teil
                  des Erstdrucks, der im Oktober-Heft der \emph{\textcolor{green}{Neuen
                     Deutschen Rundschau}} enthalten war (Jg. 5, H. 10, S. 969–988).
                  Zwei weitere Teile folgten bis Dezember. Die Buchausgabe erschien im
                     November 1894, auf 1895 vordatiert. Die von \textcolor{blue}{Goldmann} vorgeschlagenen Änderungen wurden
                  nicht berücksichtigt.}}}\label{K_L02616-111h}« habe ich geleſen. Es hat mich tief, tief ergriffen.
               Wenn Du wüßteſt, was für einen goldenen Reifeton Deine Kunſt jetzt hat! Dieſe klare
               und noble Einfachheit! Dieſe Gemüthstiefe! Und dieſer ſcharfe Verſtand, der in des
               Lebens dunkelſte Gründe dringt! Soweit ich bisher urtheilen kann, iſt es eine große
               Leiſtung, wohl Deine größte biſher. Nur Eines meine ich – ich weiß nicht, ob der
               Eindruck bis zum Schluß vorhalten wird – Du ſollteſt aus der verfluchten Illegitimtät
               heraus. Das bringt etwas {\pb}Halbes hinein. Wenn das
               Mädl ſeine Frau wäre, ſo \strikeout{\textcolor{gray}{×}} wäre es noch ergreifender, noch allgemein menſchlicher. Ich glaube, daß es
               nichts ſchaden könnte, bis nach Weihnachten mit dem Buche zu warten. Vor
                  Weihnachten kommſt Du in den großen Schwall hinein, nachher tritt es
               beſſer hervor.\pend
           \pstart
           Das \label{K_L02616-777v}\edtext{\textcolor{green}{Stück}{}\ledrightnote{→\textcolor{green}{Ottilie. Schauspiel in vier Akten}} von \textsc{\textcolor{blue}{Triesch}{}\ledrightnote{\textcolor{blue}{Friedrich Gustav Triesch}}}}{\lemma{\textnormal{\emph{Stück von Triesch}}}\Cendnote{\textnormal{Am 16. 10. 1894 fand am \emph{\textcolor{brown}{Raimund-Theater}} die Premiere von \emph{\textcolor{green}{Ottilie. Schauspiel in vier Akten}} statt. \textcolor{blue}{Schnitzler} besuchte die Aufführung und notierte im \emph{\textcolor{green}{Tagebuch}}: »bodenlos«.}}}\label{K_L02616-777h}
               hat \textsc{\textcolor{blue}{Bahr}{}\ledrightnote{\textcolor{blue}{Hermann Bahr}}} in der »\textcolor{brown}{Zeit}{}\ledrightnote{\textcolor{brown}{Die Zeit. Wiener Wochenschrift}}« feſt \label{K_L02616-456v}\edtext{\textcolor{green}{gelobt}{}\ledrightnote{→\textcolor{green}{Kunst und Leben. [Raimundtheater. Ottilie von Triesch]}}}{\lemma{\textnormal{\emph{gelobt}}}\Cendnote{\textnormal{\textcolor{blue}{H. B.}: \emph{\textcolor{green}{Kunst und Leben. [Raimundtheater.]}}. In: \emph{\textcolor{brown}{Die Zeit}}, Jg. 1, H. 3, 20. 10. 1894,
               S. 44.}}}\label{K_L02616-456h}. Verhält ſich eben mit der \label{K_L02616-98v}\edtext{\textsc{Clique}}{\lemma{\textnormal{\emph{Clique}}}\Cendnote{\textnormal{vermutlich abschätzig auf die
                  momentanen Akteure der Theater bezogen, nicht unbedingt auf eine spezifische
                  Gruppe von namentlich bekannten Personen}}}\label{K_L02616-98h}, der Herr. Pfui, pfui!\pend
           \pstart
           Das »\textcolor{brown}{Journal}{}\ledrightnote{\textcolor{brown}{Le Journal}}« iſt, ſeit Du es abonnirt haſt,
               recht ſchwach. Es iſt, als geſchähe es abſichtlich. Vergiß nicht, {\pb}die Humoriſten zu leſen: \textsc{\textcolor{blue}{Allais}{}\ledrightnote{\textcolor{blue}{Alphonse Allais}}}, \textsc{\textcolor{blue}{Bill Sharp}{}\ledrightnote{\textcolor{blue}{Pierre Veber}}}{ }\textsc{etc.} Des Letzteren »\label{K_L02616-123v}\edtext{\textcolor{green}{Briefe
                  an \textsc{\textcolor{blue}{Allais}{}\ledrightnote{\textcolor{blue}{Alphonse Allais}}}{ }
                  über die Zündhölzchen}{}\ledrightnote{\textcolor{green}{Lettre à M. Alphonse Allais sur les omnibus}}}{\lemma{\textnormal{\emph{Briefe … Zündhölzchen}}}\Cendnote{\textnormal{\textcolor{blue}{Bill Sharp [=Pierre Veber]}: \emph{\textcolor{green}{Lettre à M. Alphonse Allais sur les
                        allumettes}}. In: \emph{\textcolor{green}{Le Journal}}, Jg. 3,
                     Nr. 732, 29. 9. 1894, S. 1–2. }}}\label{K_L02616-123h} und \label{K_L02616-77v}\edtext{\textcolor{green}{über die Omnibuſſe}{}\ledrightnote{\textcolor{green}{Lettre à M. Alphonse Allais sur les omnibus}}«}{\lemma{\textnormal{\emph{über die Omnibuſſe«}}}\Cendnote{\textnormal{\textcolor{blue}{Bill Sharp [=Pierre Veber]}: \emph{\textcolor{green}{Lettre à M. Alphonse Allais sur les
                     omnibus}}. In: \emph{\textcolor{green}{Le Journal}}, Jg. 3,
                     Nr. 751, 18. 10. 1894, S. 1–2.}}}\label{K_L02616-77h} waren köſtlich.
               Freilich muß man ein wenig \label{T_L02616-33v}\edtext{Lokalkenntniß}{\lemma{\textnormal{\emph{Lokalkenntniß}}}\Cendnote{\textnormal{\textcolor{blue}{Goldmann} schreibt: »Lokalkenntniß
                     zu«}}}\label{T_L02616-33h} haben, um das in ſeiner ganzen Größe zu würdigen. Du haſt
                  \textsc{30 fr. 40 ct.} bei mir gut. Was ſoll damit geſchehen? Ein
               Paar Sachen habe ich für Dich geſammelt, wie ich Dir verſprochen. Es iſt nicht viel
               Bedeutendes drunter, aber allerlei {\pb}Kurioſes. Es iſt
               natürlich lächerlich, daß ich dir zugemuthet habe, über das Alles mir zu berichten.
               Schreib’ mir nur ein Allgemeines Wort, obs Dir ſo recht iſt. Dann fahre ich fort.\pend
           \pstart
           \label{K_L02616-66v}\edtext{Das mit dem \strikeout{ſeh} ſechzehnjährigen \textcolor{blue}{Mädel}{}\ledrightnote{→\textcolor{blue}{Else Singer}}}{\lemma{\textnormal{\emph{Das … Mädel}}}\Cendnote{\textnormal{\textcolor{blue}{Schnitzler} dürfte von der sechzehnjährigen
                     \textcolor{blue}{Else Singer} geschrieben haben, die ihm zu
                  dieser Zeit viele Briefe schickte. Darin thematisierte sie Gerüchte von einer
                  Beziehung \textcolor{blue}{Schnitzler}s mit \textcolor{blue}{Adele Sandrock}.}}}\label{K_L02616-66h} hat mich gerührt. Liebes, kleines
               Ding!\pend
           \pstart
           Die Frau \textsc{\textcolor{blue}{Andreas}{}\ledrightnote{\textcolor{blue}{Lou Andreas-Salomé}}} ſprach ich hier noch einmal. Ich glaube, ſie hat mich lieb gehabt. Nun iſt ſie
               im Groll von mir geſchieden, weil ich ſie zurückgeſtoßen habe. Und allſogleich ſtellt
                  {\pb}ſich bei mir die Reue ein. Aber ſie hat
               unwideruflich mit mir gebrochen.\pend
           \pstart
           Grüß’ mir \textsc{\textcolor{blue}{Richard}{}\ledrightnote{\textcolor{blue}{Richard Beer-Hofmann}}} und \textsc{\textcolor{blue}{Loris}{}\ledrightnote{\textcolor{blue}{Hugo von Hofmannsthal}}}.\pend
           \pstart
           \textsc{\textcolor{blue}{Herzl}{}\ledrightnote{\textcolor{blue}{Theodor Herzl}}} ſehe ich kaum. Bin wieder ganz mit ihm auseinander. Er war ſeit ſeiner
               Rückkunft einmal bei mir, um mir anzuzeigen, daß »\label{K_L02616-665v}\edtext{\textsc{\textcolor{green}{Tabarin}{}\ledrightnote{\textcolor{green}{Tabarin. Schauspiel in einem Act. Frei nach Catulle Mendès}}}« werde aufgeführt}{\lemma{\textnormal{\emph{Tabarin« werde aufgeführt}}}\Cendnote{\textnormal{Der Einakter \emph{\textcolor{green}{Tabarin}} war seit Anfang Oktober als Novität für die 
                  Saison 1894/1895 am \emph{\textcolor{brown}{Burgtheater}} angekündigt. Die Premiere fand am 2. 5. 1895 statt,
                  \textcolor{blue}{Schnitzler} besucht die Aufführung am 7. 5. 1895.}}}\label{K_L02616-665h} werden, was mich neidiſch machen ſollte. Seitdem verkehrt er
               täglich mit \textsc{\textcolor{blue}{Feldmann}{}\ledrightnote{\textcolor{blue}{Siegmund Feldmann}}} und läßt ſich bei mir nicht mehr ſehen. So habe ich ihn auch links liegen
               laſſen.\pend
           \pstart
           Aber Deinen Gruß und {\pb}Dein Lob habe ich ihm ausgerichtet. Das hat ihn ſehr gefreut.\pend
           \pstart
           Meine Sachen ſammeln? Ich weiß genau, daß ſie es nicht werth ſind. Aber mir thut es
               wohl, wenn Du mir das \textcolor{blue}{Gegentheil}{}\ledrightnote{\textcolor{blue}{Gustav Mahler}} ſchreibſt. Natürlich werde ich ſie nicht
               ſammeln.\pend
           \pstart
           Bitte, mich Frl. \textsc{\textcolor{blue}{Sandrock}{}\ledrightnote{\textcolor{blue}{Adele Sandrock}}} zu empfehlen.\pend
           \pstart
           Bitte, mich Deiner Frau \textcolor{blue}{Mutter}{}\ledrightnote{→\textcolor{blue}{Louise Schnitzler}} recht herzlich zu empfehlen. Bitte, Deinen \textcolor{blue}{Bruder}{}\ledrightnote{→\textcolor{blue}{Julius Schnitzler}} und Deine entzückende kleine \textcolor{blue}{Schwägerin}{}\ledrightnote{→\textcolor{blue}{Helene Schnitzler}} recht herzlich von
               mir zu grüßen.\pend
           \pstart
           Und ſei Du ſelbſt von Herzen gegrüßt! Dein{\\[\baselineskip]}treuer \spacefill\mbox{Paul
                  Goldmann}\pend
           \leftskip=0em{}\pstart
           \noindent{}\label{T_L02616-3v}\edtext{\textsc{\textcolor{blue}{Salten}{}\ledrightnote{\textcolor{blue}{Felix Salten}}} laſſe ich zu ſeiner \label{K_L02616-88v}\edtext{neuen
                     Stellung}{\lemma{\textnormal{\emph{neuen
                     Stellung}}}\Cendnote{\textnormal{Er war seit Oktober
                        1894 bei der \emph{\textcolor{brown}{Wiener Allgemeinen
                        Zeitung}} engagiert.}}}\label{K_L02616-88h} gratuliren}{\lemma{\textnormal{\emph{Salten … gratuliren}}}\Cendnote{\textnormal{entlang des linken Blattrands}}}\label{T_L02616-3h}.\pend
           \pstart
           {\pb}\label{T_mets_Goldmann_94-partII-1v}\edtext{Wenn Du vom \textcolor{brown}{Burgtheater}{}\ledrightnote{\textcolor{brown}{Burgtheater}} Antwort haſt, erbitte ich \uline{umgehende} Mittheilung}{\lemma{\textnormal{\emph{Wenn … Mittheilung}}}\Cendnote{\textnormal{auf der ersten Seite oberhalb, verkehrt
                     zum Text}}}\label{T_mets_Goldmann_94-partII-1h}.\pend
           \endnumbering\briefempfaengerindex{Schnitzler, Arthur@\textsc{Schnitzler, Arthur}!zzzGoldmann, Paul@\emph{von Paul Goldmann}!1894-10-251@{25. 10. {[}1894{]}}|)be}\mylabel{h}  \normalsize

\doendnotes{C}
\bigskip
\vfill

\clearpage

\footnotesize

\lohead{\textsc{register}}

% Definiere theindex-Environment komplett neu ohne reledmac
\makeatletter
\renewenvironment{theindex}{%
  \section*{\indexname}%
  \setlength{\parindent}{0pt}%
  \setlength{\parskip}{0pt plus 0.3pt}%
  \let\item\@idxitem
}{%
  \clearpage
}
\makeatother

\IfFileExists{\jobname-pw.ind}{\input{\jobname-pw.ind}}{}

\end{document}

      