%% latex-korrekturansicht-vorspann.tex
%% Vorspann für die Korrekturansicht.
%% Lädt die gemeinsame Datei latex-vorspann.tex mit gesetztem Schalter.

\newif\ifkorrekturansicht
\korrekturansichttrue

\input{../tex-inputs/latex-vorspann}


               \section[Josef Rosengart an Arthur Schnitzler, 5. 12. 1893]{ Josef Rosengart an Arthur Schnitzler, 5. 12. 1893}\nopagebreak\mylabel{v}\rehead{ }\normalsize\beginnumbering\briefempfaengerindex{Schnitzler, Arthur@\textsc{Schnitzler, Arthur}!zzzRosengart, Josef@\emph{von Josef Rosengart}!1893-12-052@{5. 12. 1893}|(be} \toendnotes[C]{\smallbreak\pagebreak[2]} \Standort{DLA, A:Schnitzler, HS.NZ85.1.4334.}
\physDesc{Karte
\newline{}Handschrift: schwarze Tinte, deutsche Kurrent
\newline{}Schnitzler: mit rotem Buntstift eine Unterstreichung }\toendnotes[C]{\smallbreak}\pstart
           \raggedleft{}{\pb}\textcolor{pink}{Frankfurtm}{}\ledrightnote{\textcolor{pink}{Frankfurt am Main}}, 5. Dezbr
                  1893.\pend
           \pstart{}Sehr geehrter Herr Doctor!\pend\pstart
           Durch meinen Schwager \textcolor{blue}{Paul Goldmann}{}\ledrightnote{\textcolor{blue}{Paul Goldmann}} in \textcolor{pink}{Paris}{}\ledrightnote{\textcolor{pink}{Paris}} erfahre ich, daß ich Ihrer beſonderen
               Liebenswürdigkeit die \label{K_L02798-1v}\edtext{Zuſendung der ſo
               ſehr intereſſanten und wiſſenſchaftlich bedeutenden »\textcolor{green}{Internationalen kliniſchen Rundſchau}{}\ledrightnote{\textcolor{green}{Internationale klinische Rundschau}}}{\lemma{\textnormal{\emph{Zuſendung … Rundſchau}}}\Cendnote{\textnormal{siehe Paul Goldmann an Arthur Schnitzler, 4. 11. [1893]}}}\label{K_L02798-1h}« verdanke. Ich danke Ihnen hierfür ganz beſonders, übertragen Sie
               hierdurch doch ein Stückchen Ihrer Freundſchaft für {\pb}meinen \textcolor{blue}{Schwager}{}\ledrightnote{→\textcolor{blue}{Paul Goldmann}} auf
               mich!\pend
           \pstart
           Ich erlaube mir, Ihnen bei dieſer Gelegenheit – und als nunmehr bei Ihnen eingeführt
               zu dem Erfolge Ihres in \label{K_L02798-2v}\edtext{\textcolor{pink}{\textsc{Wien}}{}\ledrightnote{\textcolor{pink}{Wien}} aufgeführten \textcolor{green}{Stückes}{}\ledrightnote{→\textcolor{green}{Das Märchen. Schauspiel in drei Aufzügen}}}{\lemma{\textnormal{\emph{Wien … Stückes}}}\Cendnote{\textnormal{Die Uraufführung von \emph{\textcolor{green}{Das Märchen}} hatte am 1. 12. 1893 am \textcolor{pink}{Deutschen Volkstheater} stattgefunden.}}}\label{K_L02798-2h} Glück zu
               wünſchen. \textcolor{blue}{Paul}{}\ledrightnote{\textcolor{blue}{Paul Goldmann}} hat uns ſchon i{\geminationm}er von Ihnen und von dem Großen, was er von Ihnen
               erwartet, erzählt, daß wir von Ihren Erfolgen nicht überraſcht waren. Genehmigen Sie,
               ſehr verehrter Herr Doctor, den Ausdruck der Hochachtung Ihres ergebenen\pend
           \pstart \spacefill\mbox{DrRosengart.}\pend{}\endnumbering\briefempfaengerindex{Schnitzler, Arthur@\textsc{Schnitzler, Arthur}!zzzRosengart, Josef@\emph{von Josef Rosengart}!1893-12-052@{5. 12. 1893}|)be}\mylabel{h}\begin{anhang}\end{anhang}\normalsize

\doendnotes{C}
\bigskip
\vfill

\clearpage

\footnotesize

\lohead{\textsc{register}}

% Definiere theindex-Environment komplett neu ohne reledmac
\makeatletter
\renewenvironment{theindex}{%
  \section*{\indexname}%
  \setlength{\parindent}{0pt}%
  \setlength{\parskip}{0pt plus 0.3pt}%
  \let\item\@idxitem
}{%
  \clearpage
}
\makeatother

\IfFileExists{\jobname-pw.ind}{\input{\jobname-pw.ind}}{}

\end{document}

      