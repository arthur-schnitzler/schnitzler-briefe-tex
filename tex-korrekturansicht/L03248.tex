%% latex-korrekturansicht-vorspann.tex
%% Vorspann für die Korrekturansicht.
%% Lädt die gemeinsame Datei latex-vorspann.tex mit gesetztem Schalter.

\newif\ifkorrekturansicht
\korrekturansichttrue

\input{../tex-inputs/latex-vorspann}


\renewcommand{\erwaehntePersonen}{Personen: Olga Schnitzler, Heinrich Schnitzler}
\renewcommand{\erwaehnteOrte}{Orte: Berlin, Dänemark, Helsingør, Marienlyst, Schweiz, Tirol, Wien}
\renewcommand{\erwaehnteWerke}{}
\section[ Paul Goldmann an Arthur Schnitzler, 13. 7. 1906]{Paul Goldmann an Arthur Schnitzler, 13. 7. 1906}
\nopagebreak\mylabel{v}
\rehead{ }\normalsize\beginnumbering\briefempfaengerindex{Schnitzler, Arthur@\textsc{Schnitzler, Arthur}!zzzGoldmann, Paul@\emph{von Paul Goldmann}!1906-07-133@{13. 7. 1906 }|(be}
\toendnotes[C]{\smallbreak\pagebreak[2]}\Standort{DLA, A:Schnitzler, HS.NZ85.1.3175.}
\physDesc{Postkarte
\newline{}Handschrift: 1) blaue Tinte, deutsche Kurrent\hspace{1em}2) blaue Tinte, lateinische Kurrent (\noindent{}Adresse)\hspace{1em}
\newline{}Versand: 1) Stempel: »\nobreak{}\oindex{Berlin@\textbf{Berlin}, \emph{https://www.geonames.org/ontologyP.PPLC}|pwk}Berlin, S.W. 11\textsuperscript{e}, 13. 7. 06, 2–3 N.\nobreak{}«.   2) Stempel: »\nobreak{}\oindex{Helsingør@\textbf{Helsingør}, \emph{Besiedelter Ort (A.BSO)}|pwk}Helsingør, 14. 7. 06, 11–12 F\nobreak{}«. 
\newline{}Schnitzler: mit Bleistift das Jahr »{[}1{]}906« vermerkt }\toendnotes[C]{\smallbreak}\pstart{}{\pb}Welt-\textcolor{gray}{\textbf{Poſtkarte}}\pend{}\pstart{}Herrn\pend{}\pstart{}Dr. Arthur Schnitzler (aus \textcolor{pink}{Wien}{}\ledrightnote{\textcolor{pink}{Wien}})\pend{}\pstart{}\textcolor{pink}{Marienlyst}{}\ledrightnote{\textcolor{pink}{Marienlyst}}\pend{}\pstart{}\textcolor{pink}{Dänemark}{}\ledrightnote{\textcolor{pink}{Dänemark}}. \pend{}
{\bigskip}
\pstart
           {\pb}\textcolor{pink}{Berlin}{}\ledrightnote{\textcolor{pink}{Berlin}}, 13. Juli.\pend
           
\pstart
           Lieber Freund, Nach \textcolor{pink}{Dänemark}{}\ledrightnote{\textcolor{pink}{Dänemark}} komme ich nicht – ich ſuche wieder ſtarke Gebirgsluft auf u.
               ſchwanke noch zwiſchen \textcolor{pink}{Schweiz}{}\ledrightnote{\textcolor{pink}{Schweiz}} u. \textcolor{pink}{Tirol}{}\ledrightnote{\textcolor{pink}{Tirol}}. K\textcolor{gray}{ann}ſt Du nicht
               vielleicht nach \textcolor{pink}{Dänemark}{}\ledrightnote{\textcolor{pink}{Dänemark}} noch ins \label{K-L03248-1v}\edtext{Gebirge}{\lemma{\textnormal{\emph{Gebirge}}}\Cendnote{\textnormal{nicht geschehen}}}\label{K-L03248-1h}? Ich würde mich ſehr freuen, \strikeout{Dich}{ }\strikeout{\textcolor{gray}{da}} irgendwo mit Dir \label{K-L03248-2v}\edtext{zuſammenzutreffen}{\lemma{\textnormal{\emph{zuſammenzutreffen}}}\Cendnote{\textnormal{\textcolor{blue}{Schnitzler} und \textcolor{blue}{Goldmann} trafen sich erst am 24. 5. 1907 in \textcolor{pink}{Wien} wieder.}}}\label{K-L03248-2h}. Herzliche Grüße Dir u.
               Deiner \textcolor{blue}{Frau}{}\ledrightnote{{$\rightarrow$}\textcolor{blue}{Olga Schnitzler}} von Deinem
               getreuen \spacefill\mbox{Paul Goldmann.}\pend
           
\pstart
           \noindent{}Was macht \textsc{\textcolor{blue}{Heinrich Schnitzler}{}\ledrightnote{\textcolor{blue}{Heinrich Schnitzler}}}?\pend
           \endnumbering\briefempfaengerindex{Schnitzler, Arthur@\textsc{Schnitzler, Arthur}!zzzGoldmann, Paul@\emph{von Paul Goldmann}!1906-07-133@{13. 7. 1906 }|)be}\mylabel{h}  \normalsize

\doendnotes{C}
\bigskip
\vfill

\clearpage

\footnotesize

\lohead{\textsc{register}}

% Definiere theindex-Environment komplett neu ohne reledmac
\makeatletter
\renewenvironment{theindex}{%
  \section*{\indexname}%
  \setlength{\parindent}{0pt}%
  \setlength{\parskip}{0pt plus 0.3pt}%
  \let\item\@idxitem
}{%
  \clearpage
}
\makeatother

\IfFileExists{\jobname-pw.ind}{\input{\jobname-pw.ind}}{}

\end{document}

      