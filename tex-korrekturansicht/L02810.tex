%% latex-korrekturansicht-vorspann.tex
%% Vorspann für die Korrekturansicht.
%% Lädt die gemeinsame Datei latex-vorspann.tex mit gesetztem Schalter.

\newif\ifkorrekturansicht
\korrekturansichttrue

\input{../tex-inputs/latex-vorspann}


               \section[ Paul Goldmann an Arthur Schnitzler, 28. 4. 1897]{Paul Goldmann an Arthur Schnitzler, 28. 4. 1897}\nopagebreak\mylabel{v}\rehead{ }\normalsize\beginnumbering\briefempfaengerindex{Schnitzler, Arthur@\textsc{Schnitzler, Arthur}!zzzGoldmann, Paul@\emph{von Paul Goldmann}!1897-04-282@{28. 4. 1897}|(be} \toendnotes[C]{\smallbreak\pagebreak[2]} \Standort{DLA, A:Schnitzler, HS.NZ85.1.3167.}
\physDesc{Brief, 1 Blatt, 2 Seiten
\newline{}Handschrift: schwarze Tinte, deutsche Kurrent}\toendnotes[C]{\smallbreak}\pstart
           \noindent{}{\pb}\textcolor{brown}{\textcolor{gray}{\textbf{\textsc{Frankfurter Zeitung}}}}{}\ledrightnote{\textcolor{brown}{Frankfurter Zeitung}}\hfill \textcolor{gray}{\textbf{\textcolor{pink}{Frankfurt a. M.}{}\ledrightnote{\textcolor{pink}{Frankfurt am Main}},}}{ }28. April \textcolor{gray}{\textbf{189}}7.\pend
           \pstart
           \textsc{\textcolor{gray}{\textbf{und}}}\pend
           \pstart
           \textcolor{gray}{\textbf{\textsc{Handelsblatt.}}}\pend
           \pstart
           \textcolor{gray}{\textbf{\textsc{– \textcolor{brown}{Redaction}{}\ledrightnote{→\textcolor{brown}{Frankfurter Zeitung}}.\footnote{\noindent{}\textcolor{gray}{\textbf{\textsc{Für die \textcolor{brown}{Redaktion} bestimmte Briefe und Sendungen
                                    wolle man \so{nicht} an die Person eines
                                    Redakteurs, sondern stets \textbf{an die Redaktion der
                                          \textcolor{brown}{Frankfurter Zeitung}} adressiren}}}.} –}}}\pend
           \pstart
           \textcolor{gray}{\textbf{\textsc{Telegramm-Adresse:}}}\pend
           \pstart
           \textcolor{gray}{\textbf{\textsc{\textcolor{brown}{Zeitung}{}\ledrightnote{→\textcolor{brown}{Frankfurter Zeitung}}{ }\textcolor{pink}{Frankfurt Main}{}\ledrightnote{\textcolor{pink}{Frankfurt am Main}}.}}}\pend
           \pstart\center{}Mein lieber Freund,\pend\pstart
           Meine Familie wird mich vor Ende der Woche kaum fortlaſſen und ſo werde ich Dich wohl
               vor Montag oder Dienſtag nicht wiederſehen. Auch thut mir die Ruhe wahrlich noth. Ich war
               und bin noch zum Theil in einem ſchlimmen körperlichen Zuſtande. Ich danke Dir für
               Deinen lieben Brief und freue mich, daß Ihr Euch in \textsc{\textcolor{pink}{Paris}{}\ledrightnote{\textcolor{pink}{Paris}}} zurechtfindet. Freitag{ }Abend ſolltet Ihr ins \label{K_L02810-1v}\edtext{\begin{otherlanguage}{french}\textsc{\textcolor{pink}{Concert Parisien}{}\ledrightnote{\textcolor{pink}{Concert Parisien}}}\end{otherlanguage}}{\lemma{\textnormal{\emph{Concert Parisien}}}\Cendnote{\textnormal{das taten \textcolor{blue}{sie}, vgl. A. S.: \emph{Tagebuch}, 30. 4. 1897}}}\label{K_L02810-1h} zum \label{K_L02810-2v}\edtext{\begin{otherlanguage}{french}\textsc{vendredi classique}\end{otherlanguage}}{\lemma{\textnormal{\emph{vendredi classique}}}\Cendnote{\textnormal{Der »\begin{otherlanguage}{french}vendredi
                     classique\end{otherlanguage}« war eine Veranstaltungsreihe des \textcolor{pink}{Concert Parisien}, genauso wie beispielsweise der
                     »\begin{otherlanguage}{french}lundi moderne\end{otherlanguage}«.}}}\label{K_L02810-2h} gehen, um \label{K_L02810-3v}\edtext{\textsc{\textcolor{blue}{Villé}{}\ledrightnote{\textcolor{blue}{Dora Villé}}}}{\lemma{\textnormal{\emph{Villé}}}\Cendnote{\textnormal{\textcolor{blue}{Dora Villé}, Sängerin beim »\textcolor{pink}{vendredi classique}«}}}\label{K_L02810-3h}
               zu hören. {\pb}Sagte ich Dir, daß Du das \label{K_L02810-4v}\edtext{\textsc{\textcolor{pink}{Hotel de Ville}{}\ledrightnote{\textcolor{pink}{Hôtel de Ville}}} und das \textsc{\textcolor{pink}{Panthéon}{}\ledrightnote{\textcolor{pink}{Panthéon}}} beſichtigen}{\lemma{\textnormal{\emph{Hotel … beſichtigen}}}\Cendnote{\textnormal{Das \textcolor{pink}{Panthéon} hatte \textcolor{blue}{Schnitzler} bereits am 17. 4. 1897 besucht. Eine Besichtigung des \textcolor{pink}{Hôtel de Ville} (in dem sich das \textcolor{pink}{Paris}er Rathaus befindet) ist nicht bekannt.}}}\label{K_L02810-4h}
               ſollſt?\pend
           \pstart
           Hier nichts Neues. Aber doch: Ich ſoll als Feuilleton-Correſpondent der \textcolor{brown}{Frankfurter Zeitung}{}\ledrightnote{\textcolor{brown}{Frankfurter Zeitung}} über kurz oder lang nach \textsc{\textcolor{pink}{Berlin}{}\ledrightnote{\textcolor{pink}{Berlin}}} gehen\substVorne{}\textsuperscript{\textcolor{gray}{!}}\substDazwischen{}.\substHinten{} (ganz unter uns, nicht wahr?) Soll ich? \textsc{\textcolor{pink}{Paris}{}\ledrightnote{\textcolor{pink}{Paris}}} iſt ſo ſchön!\pend
           \pstart
           Wenn Du Zeit haſt, ſo ſchreib’ mir noch ein Wort über Euer Ergehen ins \textsc{\textcolor{pink}{Hotel Deutscher Kaiser}{}\ledrightnote{\textcolor{pink}{Hotel Deutscher Kaiser}}}. Wenn Du zu faul biſt, ſo ſchreib’ mir nicht.\pend
           \pstart
           Grüß’ Dich Gott! Viele Grüße an Deine \textcolor{blue}{Freundin}{}\ledrightnote{→\textcolor{blue}{Marie Reinhard}}!\pend
           \pstart
           Dein treuer {\\[\baselineskip]}\spacefill\mbox{Paul Goldmnn}\pend
           \leftskip=0em{}\pstart
           \noindent{}Was macht der \label{K_L02810-34v}\edtext{blonde junge \textcolor{blue}{Muſiker}{}\ledrightnote{→\textcolor{blue}{[?? blonder junger Musiker in Paris]}}}{\lemma{\textnormal{\emph{blonde junge Muſiker}}}\Cendnote{\textnormal{nicht identifiziert}}}\label{K_L02810-34h}?\pend
           \endnumbering\briefempfaengerindex{Schnitzler, Arthur@\textsc{Schnitzler, Arthur}!zzzGoldmann, Paul@\emph{von Paul Goldmann}!1897-04-282@{28. 4. 1897}|)be}\mylabel{h}  \normalsize

\doendnotes{C}
\bigskip
\vfill

\clearpage

\footnotesize

\lohead{\textsc{register}}

% Definiere theindex-Environment komplett neu ohne reledmac
\makeatletter
\renewenvironment{theindex}{%
  \section*{\indexname}%
  \setlength{\parindent}{0pt}%
  \setlength{\parskip}{0pt plus 0.3pt}%
  \let\item\@idxitem
}{%
  \clearpage
}
\makeatother

\IfFileExists{\jobname-pw.ind}{\input{\jobname-pw.ind}}{}

\end{document}

      