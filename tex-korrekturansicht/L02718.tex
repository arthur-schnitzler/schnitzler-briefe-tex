%% latex-korrekturansicht-vorspann.tex
%% Vorspann für die Korrekturansicht.
%% Lädt die gemeinsame Datei latex-vorspann.tex mit gesetztem Schalter.

\newif\ifkorrekturansicht
\korrekturansichttrue

\input{../tex-inputs/latex-vorspann}


               \section[Paul Goldmann an Arthur Schnitzler, {[}zwischen 29. 9. und 2. 10. 1893{]}]{ Paul Goldmann an Arthur Schnitzler, {[}zwischen 29. 9. und
               2. 10. 1893{]}}\nopagebreak\mylabel{v}\rehead{ }\normalsize\beginnumbering\briefempfaengerindex{Schnitzler, Arthur@\textsc{Schnitzler, Arthur}!zzzGoldmann, Paul@\emph{von Paul Goldmann}!1893-09-291@{{[}zwischen 29. 9. und
                  2. 10. 1893{]}}|(be} \toendnotes[C]{\smallbreak\pagebreak[2]} \Standort{DLA, A:Schnitzler, HS.NZ85.1.3163.}
\physDesc{Brief, 1 Blatt, 2 Seiten
\newline{}Handschrift: schwarze Tinte, deutsche Kurrent
\newline{}Schnitzler: 1) mit Bleistift das Datum »Octob 93.« vermerkt 2) mit rotem Buntstift eine Unterstreichung}\toendnotes[C]{\smallbreak}\pstart
           \noindent{}{\pb}Herzlichen Dank, liebſter Freund! Die \label{K_L02718-1v}\edtext{\textcolor{brown}{S. u. M.-Ztg.}{}\ledrightnote{\textcolor{brown}{Wiener Sonn- und Montagszeitung}}}{\lemma{\textnormal{\emph{S. u. M.-Ztg.}}}\Cendnote{\textnormal{unklarer Bezug auf die \emph{\textcolor{brown}{Wiener Sonn- und Montagszeitung}}; sofern es eine Reaktion
                  auf einen Text darstellt, der in der letzten oder vorletzten Nummer enthalten war,
                  dürfte es sich um diesen handeln: \textcolor{blue}{G.
                        Engelsmann}: \emph{\textcolor{green}{Zola über die Anonymität der
                        Presse}}. In: \emph{\textcolor{green}{Wiener Sonn- und
                        Montagszeitung}}, Jg. 31, Nr. 38, 18.9.1893, S. 1–2. (\textcolor{blue}{Gabriel Engelsmann} hatte im Vorjahr auch in der
                  von \textcolor{blue}{Goldmann} herausgegeben \emph{\textcolor{brown}{An der schönen blauen Donau}} einen Beitrag.) –
                  Sofern es sich um eine Aussage über \textcolor{blue}{Hermann
                     Bahr} handelt, so dürfte diese aus der Abrechnung stammen, die am
                     24. 7. 1893 im Blatt stand. (\textcolor{blue}{L. A. Terne. (Dr. Rob. Hirschfeld)}: \emph{\textcolor{green}{Zwei Freunde Burkhards}}. In: \emph{\textcolor{green}{Wiener Sonn- und Montags-Zeitung}}, Jg. 31, Nr. 30,
                     S. 1–3.)}}}\label{K_L02718-1h} iſt ganz hübſch; ehrliche Mühe, zu verſtehen, und
               ehrlicher und gutmüthiger \strikeout{Repſ} Reſpekt vor dem
               Talent. \label{K_L02718-2v}\edtext{\textsc{\textcolor{green}{\textcolor{blue}{Bahr}{}\ledrightnote{\textcolor{blue}{Hermann Bahr}}}{}\ledrightnote{→\textcolor{green}{Das junge Österreich}}}}{\lemma{\textnormal{\emph{Bahr}}}\Cendnote{\textnormal{Gemeint war der \textcolor{green}{zweite Teil} von \textcolor{blue}{Hermann Bahr}s dreiteiliger \textcolor{green}{Feuilleton-Reihe}{ }\emph{\textcolor{green}{Das junge Österreich}}. Über \textcolor{blue}{Schnitzler} steht darin: »\textcolor{blue}{Arthur \so{Schnitzler}} ist anders. Er ist ein großer \textcolor{blue}{Virtuose}, aber einer kleinen Note. \textcolor{blue}{Torresani} streut aus reichen Krügen, ohne die einzelne
                     Gabe zu achten. \textcolor{blue}{Schnitzler} darf nicht
                     verschwenden. Er muß sparen. Er hat wenig. So will er es denn mit der
                     zärtlichsten Sorge, mit erfinderischer Mühe, mit geduldigem Geize schleifen,
                     bis das Geringe durch seine unermüdlichen Künste Adel und Würde verdient. Was
                     er bringt, ist nichtig. Aber wie er es bringt, darf gelten. Die großen Züge der
                     Zeit, Leidenschaften, Stürme, Erschütterungen der Menschen, die ungestüme
                     Pracht der Welt an Farben und an Klängen ist ihm versagt. Er weiß immer nur
                     einen einzigen Menschen, ja nur ein einziges Gefühl zu gestalten. Aber dieser
                     Gestalt gibt er Vollkommenheit, Vollendung. So ist er recht der \textsc{\begin{otherlanguage}{french}artiste\end{otherlanguage}} nach dem Herzen des ›Parnasses‹, jener \textcolor{pink}{Franzosen}, welche um den Werth an Gehalt nicht bekümmert,
                     nur in der Fasung Pflicht und Verdienst der Kunst erkennen und als eitel
                     verachten, was nicht seltene Nuance, malendes Objectiv, gesuchte Metapher
                     ist.« (\emph{\textcolor{green}{Das junge Österreich. II}}. In: \emph{\textcolor{green}{Deutsche Zeitung}}, Jg. 23, Nr. 7.813, 27. 9. 1893, Morgen-Ausgabe, S. 1–3, hier
                     S. 1) \textcolor{blue}{Schnitzler} notierte sich
                  dazu am 27. 9. 1893 im
                     \emph{\textcolor{green}{Tagebuch}}: »Ich sei ein großer
                     Virtuos auf kleinem Ton; jedoch \begin{otherlanguage}{french}apporteur du
                        neuf\end{otherlanguage}, etc.;― ich war ärgerlich.«}}}\label{K_L02718-2h} hingegen iſt
               niederträchtig, neidiſch, gemein, \label{K_L02718-5v}\edtext{perfid}{\lemma{\textnormal{\emph{perfid}}}\Cendnote{\textnormal{Dieser Ausdruck \textcolor{blue}{Goldmann}s ermöglicht letztlich die ungefähre
                  Datierung des undatierten Briefes: \textcolor{blue}{Bahr}s
                     \textcolor{green}{Kritik} erschien am 27. 9. 1893. \textcolor{blue}{Schnitzler} datierte den Brief beziehungsweise
                  das Empfangsdatum desselben auf »Octob 93«. Spätestens am 4. 10. 1893 muss \textcolor{blue}{Schnitzler} den
                  Brief erhalten haben, insofern im \emph{\textcolor{green}{Tagebuch}}-Eintrag des genannten Tages
                  Folgendes zu lesen ist: »\textcolor{blue}{Ludaßy} findet (wie \textcolor{blue}{Paul G.}) die \textcolor{green}{Kritik} von \textcolor{blue}{Bahr}
                     perfid.« Anzunehmen ist, dass \textcolor{blue}{Schnitzler}{ }\textcolor{blue}{Goldmann} die \textcolor{green}{Kritik} am 27. oder 28. 9. 1893 schickte, so dass
                     \textcolor{blue}{Goldmann}s Replik zwischen dem 29. 9. 1893 und dem 2. 10. 1893 verfasst sein dürfte.}}}\label{K_L02718-5h}. Und dieſe unverſchämte
               Schwindelei, was \strikeout{Lit}{ }\textcolor{pink}{franzöſiſch}{}\ledrightnote{→\textcolor{pink}{Frankreich}}e Literatur-Kenntniß
               anlangt. \label{K_L02718-55v}\edtext{\textsc{\textcolor{blue}{Courteline}{}\ledrightnote{\textcolor{blue}{Georges Courteline}}}, den Militär-Humoriſten, in einer Linie mit \textsc{\textcolor{blue}{Lavedan}{}\ledrightnote{\textcolor{blue}{Henri Léon Lavedan}}}}{\lemma{\textnormal{\emph{Courteline, … Lavedan}}}\Cendnote{\textnormal{Die weiteren von \textcolor{blue}{Goldmann} kritisierten Aussagen finden sich im ersten Teil: \textcolor{blue}{Hermann Bahr}: \emph{\textcolor{green}{Das junge Österreich. I}}. In: \emph{\textcolor{green}{Deutsche Zeitung}}, Jg. 23, Nr. 7.806, 20. 9. 1893, Morgen-Ausgabe, S. 1–2. zur
               }}}\label{K_L02718-55h} zu nennen! {\pb}\textsc{\textcolor{blue}{Aurélien Scholl}{}\ledrightnote{\textcolor{blue}{Aurélien Scholl}}}, den geiſtreichen \textsc{Chroniqueur à la \textcolor{blue}{Daniel Spitzer}{}\ledrightnote{\textcolor{blue}{Daniel Spitzer}}}, mit \textsc{\textcolor{blue}{Lavedan}{}\ledrightnote{\textcolor{blue}{Henri Léon Lavedan}}}, dem \textcolor{blue}{Analytiker}{}\ledrightnote{→\textcolor{blue}{Henri Léon Lavedan}},
               zuſammenzuſtellen \textsc{etc}. Wirklich zu frech! Und diefer
               unerträgliche Styl!{\dots}\pend
           \pstart
           Grüß’ Dich Gott! {\\[\baselineskip]}Dein {\\[\baselineskip]}\spacefill\mbox{P. G.}\pend
           \leftskip=0em{}\endnumbering\briefempfaengerindex{Schnitzler, Arthur@\textsc{Schnitzler, Arthur}!zzzGoldmann, Paul@\emph{von Paul Goldmann}!1893-09-291@{{[}zwischen 29. 9. und
                  2. 10. 1893{]}}|)be}\mylabel{h}  \normalsize

\doendnotes{C}
\bigskip
\vfill

\clearpage

\footnotesize

\lohead{\textsc{register}}

% Definiere theindex-Environment komplett neu ohne reledmac
\makeatletter
\renewenvironment{theindex}{%
  \section*{\indexname}%
  \setlength{\parindent}{0pt}%
  \setlength{\parskip}{0pt plus 0.3pt}%
  \let\item\@idxitem
}{%
  \clearpage
}
\makeatother

\IfFileExists{\jobname-pw.ind}{\input{\jobname-pw.ind}}{}

\end{document}

      