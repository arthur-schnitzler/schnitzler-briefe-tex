%% latex-korrekturansicht-vorspann.tex
%% Vorspann für die Korrekturansicht.
%% Lädt die gemeinsame Datei latex-vorspann.tex mit gesetztem Schalter.

\newif\ifkorrekturansicht
\korrekturansichttrue

\input{../tex-inputs/latex-vorspann}


\renewcommand{\erwaehnteOrte}{Orte: Frankgasse 1, IX., Alsergrund, Kaserne Ca’ di Dio, Trient, Wien}
\renewcommand{\erwaehnteWerke}{}
\section[ Felix Salten an Arthur Schnitzler, 13. 5. 1902]{Felix Salten an Arthur Schnitzler, 13. 5. 1902}
\nopagebreak\mylabel{v}
\rehead{ }\normalsize\beginnumbering\briefempfaengerindex{Schnitzler, Arthur@\textsc{Schnitzler, Arthur}!zzzSalten, Felix@\emph{von Felix Salten}!1902-05-131@{13. 5. 1902}|(be}
\toendnotes[C]{\smallbreak\pagebreak[2]}\Standort{CUL, Schnitzler, B 89, A 2.}
\physDesc{Bildpostkarte, 90 Zeichen
\newline{}Handschrift: Bleistift, lateinische Kurrent
\newline{}Versand: 1) Stempel: »\nobreak{}\oindex{Trient@\textbf{Trient}, \emph{P.PPLA}|pwk}Trient 2 Trento 2, 13\textcolor{gray}{/5} 0{[}2{]}, 6\nobreak{}«.   2) Stempel: »\nobreak{}\oindex{IX., Alsergrund@\textbf{IX., Alsergrund}, \emph{A.ADM3}|pwk}9/3 W\textcolor{gray}{ien 72}, 20. {[}5. 02{]}, 8, B{[}estellt{]}\nobreak{}«. 
\newline{}Schnitzler: mit Bleistift datiert: »13/5« 
\newline{}Ordnung: mit Bleistift von unbekannter Hand nummeriert: »153« }\pstart{}{\pb}Herrn D\textsuperscript{r} Arthur Schnitzler\pend{}\pstart{}\textcolor{pink}{Wien IX}{}\ledrightnote{\textcolor{pink}{IX., Alsergrund}}\pend{}\pstart{}\textcolor{pink}{Frankgaße N\textsuperscript{o} 1}{}\ledrightnote{\textcolor{pink}{Frankgasse 1}}\pend{}
{\bigskip}
\pstart
           \noindent{}{\pb}\textcolor{gray}{\textbf{\textcolor{pink}{Trento}{}\ledrightnote{\textcolor{pink}{Trient}}}}\hfill \textcolor{gray}{\textbf{\textcolor{pink}{Caserma alla Ca’ di Dio}{}\ledrightnote{\textcolor{pink}{Kaserne Ca’ di Dio}}}}\pend
           
\pstart
           Bin endlich unterwegs. herzlichst Ihr \spacefill\mbox{Salten\textcolor{gray}{.}}\pend
           \endnumbering\briefempfaengerindex{Schnitzler, Arthur@\textsc{Schnitzler, Arthur}!zzzSalten, Felix@\emph{von Felix Salten}!1902-05-131@{13. 5. 1902}|)be}\mylabel{h}  \normalsize

\doendnotes{C}
\bigskip
\vfill

\clearpage

\footnotesize

\lohead{\textsc{register}}

% Definiere theindex-Environment komplett neu ohne reledmac
\makeatletter
\renewenvironment{theindex}{%
  \section*{\indexname}%
  \setlength{\parindent}{0pt}%
  \setlength{\parskip}{0pt plus 0.3pt}%
  \let\item\@idxitem
}{%
  \clearpage
}
\makeatother

\IfFileExists{\jobname-pw.ind}{\input{\jobname-pw.ind}}{}

\end{document}

      