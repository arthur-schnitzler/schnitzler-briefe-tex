%% latex-korrekturansicht-vorspann.tex
%% Vorspann für die Korrekturansicht.
%% Lädt die gemeinsame Datei latex-vorspann.tex mit gesetztem Schalter.

\newif\ifkorrekturansicht
\korrekturansichttrue

\input{../tex-inputs/latex-vorspann}


\renewcommand{\erwaehntePersonen}{Personen: Michael Emil Salzmann}
\renewcommand{\erwaehnteOrte}{Orte: Cortina d'Ampezzo, Grillparzerstraße, Großglockner, Heiligenblut am Großglockner, I., Innere Stadt, Kärnten, Lienz, Mittewald an der Drau, Robert Bernard’s Gasthof, Wien}
\renewcommand{\erwaehnteWerke}{}
\section[ Felix Salten an Arthur Schnitzler, 17. 8. 1893]{Felix Salten an Arthur Schnitzler, 17. 8. 1893}
\nopagebreak\mylabel{v}
\rehead{ }\normalsize\beginnumbering\briefempfaengerindex{Schnitzler, Arthur@\textsc{Schnitzler, Arthur}!zzzSalten, Felix@\emph{von Felix Salten}!1893-08-172@{17. 8. 1893}|(be}
\toendnotes[C]{\smallbreak\pagebreak[2]}\Standort{CUL, Schnitzler, B 89, A 1.}
\physDesc{Bildpostkarte, 513 Zeichen
\newline{}Handschrift: Bleistift, lateinische Kurrent
\newline{}Versand: 1) Stempel: »\nobreak{}\oindex{Heiligenblut am Grossglockner@\textbf{Heiligenblut am Großglockner}, \emph{P.PPLA3}|pwk}Heilig{[}enbl{]}ut, 18/8 93\nobreak{}«.   2) Stempel: »\nobreak{}\oindex{I., Innere Stadt@\textbf{I., Innere Stadt}, \emph{A.ADM3}|pwk}Wien 1/1 1, 19/8. 93, 11½V–1N, Bestellt\nobreak{}«. 
\newline{}Ordnung: mit Bleistift von unbekannter Hand nummeriert: »28« }\toendnotes[C]{\smallbreak}\pstart{}{\pb}Herrn D\textsuperscript{r} Arthur Schnitzler\pend{}\pstart{}\textcolor{pink}{Wien}{}\ledrightnote{\textcolor{pink}{Wien}}\pend{}\pstart{}\textcolor{pink}{I. Grillparzerstraße 7}{}\ledrightnote{\textcolor{pink}{Grillparzerstraße}}.\pend{}
{\bigskip}
\pstart
           \noindent{}\centering{}{\pb}\textcolor{gray}{\textbf{\textbf{»Gruss aus \textcolor{pink}{Heiligenblut}{}\ledrightnote{\textcolor{pink}{Heiligenblut am Großglockner}}«
                           (\textcolor{pink}{Kärnten}{}\ledrightnote{\textcolor{pink}{Kärnten}}).}}}\pend
           
\pstart
           \noindent{}\centering{}\textcolor{gray}{\textbf{\textcolor{pink}{Robert Bernard’s Gasthof}{}\ledrightnote{\textcolor{pink}{Robert Bernard’s Gasthof}}.}}\pend
           
\pstart
           \raggedleft{}17. VIII. 93\pend
           
\pstart
           Lieber Freund! Von \textcolor{pink}{Cortina}{}\ledrightnote{\textcolor{pink}{Cortina d'Ampezzo}} zurück, befinde ich mich auf einer 2tägigen Tour auf den \textcolor{pink}{Glockner}{}\ledrightnote{\textcolor{pink}{Großglockner}}. Hier mit meinem \textcolor{blue}{Bruder}{}\ledrightnote{{$\rightarrow$}\textcolor{blue}{Michael Emil Salzmann}}. Ich danke herzlich
               für Ihren \label{K_L03125-1v}\edtext{Brief}{\lemma{\textnormal{\emph{Brief}}}\Cendnote{\textnormal{Arthur Schnitzler an Felix Salten, [14. 8. 1893]}}}\label{K_L03125-1h}, den ich nach Rückkehr ausführlich beantworte. Für heute nur die unangenehme Mittheilung, dass mein Rad
               zwischen \textcolor{pink}{Mittewald}{}\ledrightnote{\textcolor{pink}{Mittewald an der Drau}}{ }{\kaufmannsund}{ }\textcolor{pink}{Lienz}{}\ledrightnote{\textcolor{pink}{Lienz}} gebrochen ist, u. sich in \textcolor{pink}{Lienz}{}\ledrightnote{\textcolor{pink}{Lienz}} zur Reparatur befindet. Da das \label{K_L03125-2v}\edtext{Gouvernal}{\lemma{\textnormal{\emph{Gouvernal}}}\Cendnote{\textnormal{Fahrradlenker}}}\label{K_L03125-2h} verletzt ist, dürfte die Sache länger dauern, ich \label{K_L03125-3v}\edtext{schreibe oder telegrafire noch am Samstag}{\lemma{\textnormal{\emph{schreibe … Samstag}}}\Cendnote{\textnormal{nicht erhalten}}}\label{K_L03125-3h}\pend
           \pstart Herzl. Ihr \spacefill\mbox{Salten.}\pend{}\endnumbering\briefempfaengerindex{Schnitzler, Arthur@\textsc{Schnitzler, Arthur}!zzzSalten, Felix@\emph{von Felix Salten}!1893-08-172@{17. 8. 1893}|)be}\mylabel{h}  \normalsize

\doendnotes{C}
\bigskip
\vfill

\clearpage

\footnotesize

\lohead{\textsc{register}}

% Definiere theindex-Environment komplett neu ohne reledmac
\makeatletter
\renewenvironment{theindex}{%
  \section*{\indexname}%
  \setlength{\parindent}{0pt}%
  \setlength{\parskip}{0pt plus 0.3pt}%
  \let\item\@idxitem
}{%
  \clearpage
}
\makeatother

\IfFileExists{\jobname-pw.ind}{\input{\jobname-pw.ind}}{}

\end{document}

      