%% latex-korrekturansicht-vorspann.tex
%% Vorspann für die Korrekturansicht.
%% Lädt die gemeinsame Datei latex-vorspann.tex mit gesetztem Schalter.

\newif\ifkorrekturansicht
\korrekturansichttrue

\input{../tex-inputs/latex-vorspann}


\renewcommand{\erwaehntePersonen}{Personen: Richard Beer-Hofmann, Felix Salten}
\renewcommand{\erwaehnteInstitutionen}{Institutionen: Paul Zsolnay Verlag}
\renewcommand{\erwaehnteOrte}{Orte: Genfer See, Grundlsee (Gemeinde), Marienbad, Wien}
\renewcommand{\erwaehnteWerke}{Werke: [Mein lieber Felix Salten]}
\section[ Felix Salten an Arthur Schnitzler, 21. 9. 1929]{Felix Salten an Arthur Schnitzler, 21. 9. 1929}
\nopagebreak\mylabel{v}
\rehead{ }\normalsize\beginnumbering\briefempfaengerindex{Schnitzler, Arthur@\textsc{Schnitzler, Arthur}!zzzSalten, Felix@\emph{von Felix Salten}!1929-09-211@{21. 9. 1929}|(be}
\toendnotes[C]{\smallbreak\pagebreak[2]}\Standort{CUL, Schnitzler, B 89, B 2.}
\physDesc{Brief, 1 Blatt, 1 Seite, 844 Zeichen
\newline{}Handschrift: schwarze Tinte, lateinische Kurrent
\newline{}Schnitzler: mit rotem Buntstift Vermerke: »\textcolor{blue}{F. S.}« und eine Unterstreichung 
\newline{}Ordnung: mit Bleistift von unbekannter Hand nummeriert: »300« }\toendnotes[C]{\smallbreak}
\pstart
           \raggedleft{}{\pb}\textcolor{pink}{Grundlsee}{}\ledrightnote{\textcolor{pink}{Grundlsee (Gemeinde)}}, 21. 9. 29\pend
           
\pstart
           Lieber, für Ihr \label{K_L03587-1v}\edtext{Telegramm vom \textcolor{pink}{Genfersee}{}\ledrightnote{\textcolor{pink}{Genfer See}}}{\lemma{\textnormal{\emph{Telegramm vom Genfersee}}}\Cendnote{\textnormal{Anlässlich des 60. Geburtstags. siehe A. S.: \emph{Tagebuch}, 5. 9. 1929. Er
                  war also \textcolor{blue}{Beer-Hofmann}s Vorhaben gefolgt, vgl. Richard Beer-Hofmann an Arthur Schnitzler, 28. 8. 1929.
               }}}\label{K_L03587-1h} danke ich Ihnen herzlich! Ebenso für Ihre \label{K_L03587-2v}\edtext{Karte aus \textcolor{pink}{Marienbad}{}\ledrightnote{\textcolor{pink}{Marienbad}}}{\lemma{\textnormal{\emph{Karte aus Marienbad}}}\Cendnote{\textnormal{\textcolor{blue}{Schnitzler} war zwischen 12. 9. 1929 und 21. 9. 1929 in \textcolor{pink}{Marienbad}.}}}\label{K_L03587-2h}, die mich sehr gefreut hat.
               Ganz besonders aber muß ich Ihnen für Ihr sozusagen \label{K_L03587-3v}\edtext{öffentlich geäussertes \textcolor{green}{Wort}{}\ledrightnote{{$\rightarrow$}\textcolor{green}{[Mein lieber Felix Salten]}}}{\lemma{\textnormal{\emph{öffentlich … Wort}}}\Cendnote{\textnormal{siehe Arthur Schnitzler an Felix Salten, 29. 7. 1929 und A. S.: \emph{»Das Zeitlose ist von kürzester Dauer«}, [Mein lieber Felix Salten], [November 1929]}}}\label{K_L03587-3h} sein. Der \textcolor{brown}{Zsolnay Verlag}{}\ledrightnote{\textcolor{brown}{Paul Zsolnay Verlag}} überraschte mich
               damit und ich darf wohl sagen, dass ich nicht viele derartig angenehme Überraschungen
               erlebt habe. Einer der mir wertvollsten und mich am meisten wärmenden Aussprüche ist
               der Ihre! Ach ja – doch wozu stotternd und stammelnd an Dinge rühren, die sich so
               schwer aussprechen lassen. Sie können sich ja ungefähr denken, was man empfindet,
               wenn man so alt werden durfte. Und wenn Sie auch nicht genau alles denken oder
               wissen, was gerade mich bewegt, – ich kann’s doch nicht in Worte bringen. Jedenfalls
               haben Sie innigsten Dank! Sehr herzlich und hoffentlich auf sehr bald!\pend
           
\pstart
           Ihr {\\[\baselineskip]}\spacefill\mbox{Felix Salten}\pend
           \leftskip=0em{}\endnumbering\briefempfaengerindex{Schnitzler, Arthur@\textsc{Schnitzler, Arthur}!zzzSalten, Felix@\emph{von Felix Salten}!1929-09-211@{21. 9. 1929}|)be}\mylabel{h}  \normalsize

\doendnotes{C}
\bigskip
\vfill

\clearpage

\footnotesize

\lohead{\textsc{register}}

% Definiere theindex-Environment komplett neu ohne reledmac
\makeatletter
\renewenvironment{theindex}{%
  \section*{\indexname}%
  \setlength{\parindent}{0pt}%
  \setlength{\parskip}{0pt plus 0.3pt}%
  \let\item\@idxitem
}{%
  \clearpage
}
\makeatother

\IfFileExists{\jobname-pw.ind}{\input{\jobname-pw.ind}}{}

\end{document}

      