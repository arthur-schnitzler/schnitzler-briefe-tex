%% latex-korrekturansicht-vorspann.tex
%% Vorspann für die Korrekturansicht.
%% Lädt die gemeinsame Datei latex-vorspann.tex mit gesetztem Schalter.

\newif\ifkorrekturansicht
\korrekturansichttrue

\input{../tex-inputs/latex-vorspann}


\renewcommand{\erwaehntePersonen}{Personen: Karl Morré}
\renewcommand{\erwaehnteOrte}{Orte: Café Central, Hörlgasse, Raimund-Theater, Wien}
\renewcommand{\erwaehnteWerke}{Werke: Ein Regimentsarzt. Volksstück mit Gesang in 4 Akten}
\section[Felix Salten an Arthur Schnitzler, {[}28. 4. 1894{]}]{Felix Salten an Arthur Schnitzler, {[}28. 4. 1894{]}}
\nopagebreak\mylabel{v}
%:
\rehead{ }\normalsize\beginnumbering\briefempfaengerindex{Schnitzler, Arthur@\textsc{Schnitzler, Arthur}!zzzSalten, Felix@\emph{von Felix Salten}!1894-04-282@{{[}28. 4. 1894{]}}|(be}
\toendnotes[C]{\smallbreak\pagebreak[2]}\Standort{CUL, Schnitzler, B 89, A 1.}
\physDesc{Visitenkarte, 171 Zeichen
\newline{}Handschrift: Bleistift, lateinische Kurrent
\newline{}Schnitzler: mit Bleistift datiert: »28/4 94.« 
\newline{}Ordnung: mit Bleistift von unbekannter Hand nummeriert: »36« }\toendnotes[C]{\smallbreak}
\pstart
           \noindent{}\centering{}{\pb}\textcolor{gray}{\textbf{Felix Salten}}\pend
           
\pstart
           \noindent{}\raggedleft{}\textcolor{gray}{\textbf{\textcolor{pink}{IX. Hörlgasse 16}{}\ledrightnote{\textcolor{pink}{Hörlgasse}}.}}\pend
           
\pstart
           {\pb}Lieber Frd. Es ist \uline{nichts} mit heute{ }Abd.{ }\label{K_L03135-1v}\edtext{\textcolor{green}{Regimentsarzt}{}\ledrightnote{\textcolor{green}{Ein Regimentsarzt. Volksstück mit Gesang in 4 Akten}}}{\lemma{\textnormal{\emph{Regimentsarzt}}}\Cendnote{\textnormal{Das vieraktige Stück \emph{\textcolor{green}{Ein Regimentsarzt}} von \textcolor{blue}{Karl Morré} wurde an diesem Tag im \textcolor{pink}{Raimund-Theater} gespielt.}}}\label{K_L03135-1h}. Sind Sie
                  Abds im \textcolor{pink}{Café Central}{}\ledrightnote{\textcolor{pink}{Café Central}}? Es wäre
               gut, wegen des zweifelhaften Wetters für \label{K_L03135-2v}\edtext{Morgen}{\lemma{\textnormal{\emph{Morgen}}}\Cendnote{\textnormal{siehe A. S.: \emph{Tagebuch}, 29. 4. 1894}}}\label{K_L03135-2h} etwas auszumachen.\pend
           
\pstart
           Herzlichst {\\[\baselineskip]}Ihr {\\[\baselineskip]}\spacefill\mbox{Salten}\pend
           \leftskip=0em{}\endnumbering\briefempfaengerindex{Schnitzler, Arthur@\textsc{Schnitzler, Arthur}!zzzSalten, Felix@\emph{von Felix Salten}!1894-04-282@{{[}28. 4. 1894{]}}|)be}\mylabel{h}  \normalsize

\doendnotes{C}
\bigskip
\vfill

\clearpage

\footnotesize

\lohead{\textsc{register}}

% Definiere theindex-Environment komplett neu ohne reledmac
\makeatletter
\renewenvironment{theindex}{%
  \section*{\indexname}%
  \setlength{\parindent}{0pt}%
  \setlength{\parskip}{0pt plus 0.3pt}%
  \let\item\@idxitem
}{%
  \clearpage
}
\makeatother

\IfFileExists{\jobname-pw.ind}{\input{\jobname-pw.ind}}{}

\end{document}

      