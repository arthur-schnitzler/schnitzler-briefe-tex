%% latex-korrekturansicht-vorspann.tex
%% Vorspann für die Korrekturansicht.
%% Lädt die gemeinsame Datei latex-vorspann.tex mit gesetztem Schalter.

\newif\ifkorrekturansicht
\korrekturansichttrue

\input{../tex-inputs/latex-vorspann}


               \section[Arthur Schnitzler an Richard Beer-Hofmann, 14. 2. 1912]{ Arthur Schnitzler an Richard Beer-Hofmann, 14. 2. 1912}\nopagebreak\mylabel{v}\rehead{ }\normalsize\beginnumbering\briefempfaengerindex{Beer-Hofmann, Richard@\textsc{Beer-Hofmann, Richard}!zzzSchnitzler, Arthur@\emph{von Arthur Schnitzler}!1912-02-142@{14. 2. 1912}|(be} \toendnotes[C]{\smallbreak\pagebreak[2]} \Standort{YCGL, MSS 31.}
\physDesc{Briefkarte mit Trauerrand, Umschlag mit Trauerrand
\newline{}Handschrift: Bleistift, lateinische Kurrent\newline{}Versand: ohne postalischen Übermittlungsvermerk }\buchAbdrucke{\weitereDrucke{Arthur Schnitzler, Richard Beer-Hofmann: \emph{Briefwechsel 1891–1931}. Hg. Konstanze Fliedl. Wien, Zürich: \emph{Europaverlag} 1992, S. 216.} }\toendnotes[C]{\smallbreak}\pstart{}{\pb}\textcolor{gray}{\textbf{\textcolor{pink}{XVIII., STERNWARTESTRASSE 71}{}\ledrightnote{\textcolor{pink}{Sternwartestraße}}.}}\pend{}{\bigskip}\pstart{}{\pb}Herrn Doctor Richard Beer-Hofmann\pend{}\pstart{}\textcolor{pink}{\label{TLL02056_Beer-Hofmann-1v}\edtext{Wien}{\lemma{\textnormal{\emph{Wien}}}\Cendnote{\textnormal{Abweichend vom restlichen
                        Korrespondenzstück ist dies nicht in Lateinschrift geschrieben.}}}\label{TLL02056_Beer-Hofmann-1h}}{}\ledrightnote{\textcolor{pink}{Wien}}\pend{}{\bigskip}\pstart
           \noindent{}{\pb}\textcolor{gray}{\textbf{A. S.}}\hfill \textcolor{gray}{\textbf{\textcolor{pink}{XVIII., STERNWARTESTRASSE 71}{}\ledrightnote{\textcolor{pink}{Sternwartestraße}}.}}\pend
           \pstart
           \raggedleft{}14. 2.\pend
           \pstart
           lieber Richard, \textcolor{blue}{Rosenbaums}{}\ledrightnote{\textcolor{blue}{Richard Rosenbaum}} Privat-Teleph. Nu{\geminationm}er mir unbeka{\geminationn}t, will mich
               auch im \textcolor{pink}{Burg. Th.}{}\ledrightnote{\textcolor{pink}{Burgtheater}} nicht erkundigen, da ich einen
               Refus fürchte – oder feurige Kohlen. \textcolor{blue}{Stucken}{}\ledrightnote{\textcolor{blue}{Eduard Stucken}{\newline}\textcolor{blue}{Ania Stucken}}’s wohnen \textcolor{pink}{Hotel Regina}{}\ledrightnote{\textcolor{pink}{Hotel Regina}}. Sie kommen
                  \label{KLL02056_Beer-Hofmann-1v}\edtext{Samstag}{\lemma{\textnormal{\emph{Samstag}}}\Cendnote{\textnormal{siehe A. S.: \emph{Tagebuch}, 17. 2. 1912}}}\label{KLL02056_Beer-Hofmann-1h} gegen {\pb}5 Uhr zum Thee zu uns und wir bitten Sie mit \textcolor{blue}{Paula}{}\ledrightnote{\textcolor{blue}{Paula Beer-Hofmann}} gleichfalls zu erscheinen.\pend
           \pstart Herzlichst Ihr \spacefill\mbox{A. \strikeout{S.}}\pend{}\endnumbering\briefempfaengerindex{Beer-Hofmann, Richard@\textsc{Beer-Hofmann, Richard}!zzzSchnitzler, Arthur@\emph{von Arthur Schnitzler}!1912-02-142@{14. 2. 1912}|)be}\mylabel{h}  \normalsize

\doendnotes{C}
\bigskip
\vfill

\clearpage

\footnotesize

\lohead{\textsc{register}}

% Definiere theindex-Environment komplett neu ohne reledmac
\makeatletter
\renewenvironment{theindex}{%
  \section*{\indexname}%
  \setlength{\parindent}{0pt}%
  \setlength{\parskip}{0pt plus 0.3pt}%
  \let\item\@idxitem
}{%
  \clearpage
}
\makeatother

\IfFileExists{\jobname-pw.ind}{\input{\jobname-pw.ind}}{}

\end{document}

      