%% latex-korrekturansicht-vorspann.tex
%% Vorspann für die Korrekturansicht.
%% Lädt die gemeinsame Datei latex-vorspann.tex mit gesetztem Schalter.

\newif\ifkorrekturansicht
\korrekturansichttrue

\input{../tex-inputs/latex-vorspann}


               \section[Hermann Bahr an Arthur Schnitzler, 11. 12. 1909]{ Hermann Bahr an Arthur Schnitzler, 11. 12. 1909}\nopagebreak\mylabel{v}\rehead{ }\normalsize\beginnumbering\briefempfaengerindex{Schnitzler, Arthur@\textsc{Schnitzler, Arthur}!zzzBahr, Hermann@\emph{von Hermann Bahr}!1909-12-111@{11. 12. 1909}|(be} \toendnotes[C]{\smallbreak\pagebreak[2]} \Standort{CUL, Schnitzler, B 5b.}
\physDesc{Brief, 1 Blatt, 2 Seiten
\newline{}Handschrift Lisa Clarus: blaue Tinte, lateinische Kurrent\newline{}Handschrift Hermann Bahr: blaue Tinte (\noindent{}Unterschrift)
\newline{}Schnitzler: mit Bleistift ergänzt »Bahr« \newline{}Ordnung: mit Bleistift von unbekannter Hand nummeriert: »163« }\buchAbdrucke{\weitereDrucke{Hermann Bahr, Arthur Schnitzler: \emph{Briefwechsel, Aufzeichnungen, Dokumente (1891–1931)}. Hg. Kurt Ifkovits und Martin Anton Müller. Göttingen: \emph{Wallstein} 2018, S. 428.} }\toendnotes[C]{\smallbreak}\pstart
           \raggedleft{}{\pb}11. 12. 09\pend
           \pstart
           \centering{}\textcolor{pink}{Wien XIII/\textsubscript{7}}{}\ledrightnote{\textcolor{pink}{Ober Sankt Veit}}\pend
           \pstart\center{}Lieber Arthur!\pend\pstart
           In \textcolor{pink}{Halle \textsuperscript{a}/Saale}{}\ledrightnote{\textcolor{pink}{Halle an der Saale}}, wo ich
               auch wieder einmal \textcolor{green}{die Toten schweigen}{}\ledrightnote{\textcolor{green}{Die Toten schweigen}} liess, hat
               man mich angefleht Dir doch zuzureden, dass Du selbst einmal hinkommen sollst. Ein
               Oberingenieur \textcolor{blue}{Bacher}{}\ledrightnote{\textcolor{blue}{Oskar Bacher}}, der schon einmal mit Dir
               correspondiert haben will, beschwört Dich, wenn Du zum \label{T_L01897_1v}\edtext{\textcolor{green}{Anathol}{}\ledrightnote{\textcolor{green}{Anatol}}}{\lemma{\textnormal{\emph{Anathol}}}\Cendnote{\textnormal{Das »h« vermutlich von Schnitzler mit rotem
                  Buntstift gestrichen.}}}\label{T_L01897_1h} nach \textcolor{pink}{Berlin}{}\ledrightnote{\textcolor{pink}{Berlin}} fährst,
               doch den Weg über \textcolor{pink}{Halle}{}\ledrightnote{\textcolor{pink}{Halle an der Saale}} zu nehmen. Ich bitte Dich,
               schreib ihm (\textcolor{pink}{Halle, Waidenplan 13}{}\ledrightnote{\textcolor{pink}{Weidenplan}}) ein Wort, und
               zwar baldigst. Denn der gute Mann {\pb}hat mir ein
               unfehlbares Mittel gegen die Gicht versprochen, das ich dringend brauche und er mir
               sicher nicht schickt, so lang ich mich nicht besonders um ihn verdient gemacht habe.
               Und: hast Du vielleicht eine neue kurze, womöglich lustige Novelle? Ich soll hier
                  \label{K_L01897_1v}\edtext{für die \textcolor{brown}{freie Schule}{}\ledrightnote{\textcolor{brown}{Verein »Freie Schule«}}}{\lemma{\textnormal{\emph{für die freie Schule}}}\Cendnote{\textnormal{Am 9. 1. 1910; Er las nichts von \textcolor{blue}{Schnitzler}.}}}\label{K_L01897_1h} vorlesen und möchte was von Dir. Entschuldige, dass
               ich diktiere: ich bin totmüd, in grosser Hast und eben auf den \textcolor{pink}{Semmering}{}\ledrightnote{\textcolor{pink}{Semmering}} abreisend.\pend
           \pstart
           Herzlichst mit den schönsten Grüssen an \textcolor{blue}{Frau}{}\ledrightnote{→\textcolor{blue}{Olga Schnitzler}} und \textcolor{blue}{Kinder}{}\ledrightnote{→\textcolor{blue}{Heinrich Schnitzler}{\newline}→\textcolor{blue}{Lili Schnitzler}}{\\[\baselineskip]}Dein alter{\\[\baselineskip]}\spacefill\mbox{{[}hs. Bahr:{]} HermannBahr}\pend
           \leftskip=0em{}\endnumbering\briefempfaengerindex{Schnitzler, Arthur@\textsc{Schnitzler, Arthur}!zzzBahr, Hermann@\emph{von Hermann Bahr}!1909-12-111@{11. 12. 1909}|)be}\mylabel{h}  \normalsize

\doendnotes{C}
\bigskip
\vfill

\clearpage

\footnotesize

\lohead{\textsc{register}}

% Definiere theindex-Environment komplett neu ohne reledmac
\makeatletter
\renewenvironment{theindex}{%
  \section*{\indexname}%
  \setlength{\parindent}{0pt}%
  \setlength{\parskip}{0pt plus 0.3pt}%
  \let\item\@idxitem
}{%
  \clearpage
}
\makeatother

\IfFileExists{\jobname-pw.ind}{\input{\jobname-pw.ind}}{}

\end{document}

      