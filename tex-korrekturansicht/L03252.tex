%% latex-korrekturansicht-vorspann.tex
%% Vorspann für die Korrekturansicht.
%% Lädt die gemeinsame Datei latex-vorspann.tex mit gesetztem Schalter.

\newif\ifkorrekturansicht
\korrekturansichttrue

\input{../tex-inputs/latex-vorspann}


\renewcommand{\erwaehntePersonen}{Personen: Hermann Junker, Hermann Junker, Olga Schnitzler}
\renewcommand{\erwaehnteOrte}{Orte: Edmund-Weiß-Gasse, Prinz-Eugen-Reiterdenkmal, Wien}
\renewcommand{\erwaehnteWerke}{}
\section[ Paul Goldmann an Arthur Schnitzler, 27. 9. 1909]{Paul Goldmann an Arthur Schnitzler, 27. 9. 1909}
\nopagebreak\mylabel{v}
\rehead{ }\normalsize\beginnumbering\briefempfaengerindex{Schnitzler, Arthur@\textsc{Schnitzler, Arthur}!zzzGoldmann, Paul@\emph{von Paul Goldmann}!1909-09-271@{27. 9. 1909}|(be}
\toendnotes[C]{\smallbreak\pagebreak[2]}\Standort{DLA, A:Schnitzler, HS.NZ85.1.3175.}
\physDesc{Bildpostkarte
\newline{}Handschrift: 1) schwarze Tinte, deutsche Kurrent\hspace{1em}2) schwarze Tinte, lateinische Kurrent (\noindent{}Adresse)\hspace{1em}
\newline{}Versand: Stempel: »\nobreak{}Wien, {[}2{]}7. IX. 09, –9\nobreak{}«.  }\toendnotes[C]{\smallbreak}\pstart{}{\pb}Herrn\pend{}\pstart{}Dr. Arthur Schnitzler\pend{}\pstart{}\textcolor{pink}{Wien}{}\ledrightnote{\textcolor{pink}{Wien}}\pend{}\pstart{}\textcolor{pink}{XVIII. Spöttelgaſse 7}{}\ledrightnote{\textcolor{pink}{Edmund-Weiß-Gasse}}.\pend{}
{\bigskip}
\pstart
           \noindent{}\centering{}{\pb}\textcolor{gray}{\textbf{\textcolor{pink}{WIEN}{}\ledrightnote{\textcolor{pink}{Wien}}. \textcolor{pink}{DENKMAL DES PRINZEN EUGEN}{}\ledrightnote{\textcolor{pink}{Prinz-Eugen-Reiterdenkmal}}.}}\pend
           
\pstart
           \noindent{}\centering{}\textcolor{gray}{\textbf{\label{K-L03252-1v}\edtext{\textcolor{blue}{H. Junker}{}\ledrightnote{\textcolor{blue}{Hermann Junker}{\newline}\textcolor{blue}{Hermann Junker}}}{\lemma{\textnormal{\emph{H. Junker}}}\Cendnote{\textnormal{Ob es sich bei dem Maler des
                        Postkartenmotivs um \textcolor{blue}{Hermann Junker d.
                           Ä.} (1838–1899)
                        oder \textcolor{blue}{Hermann Junker d. J.} (1867–1938) handelt, ist
                        unklar. Das abgedruckte Bild ist jedenfalls vor 1900 entstanden.}}}\label{K-L03252-1h}.}}\pend
           
\pstart
           {\pb}Montag 27. 9.\pend
           
\pstart
           Lieber Freund, Ich habe die Abſicht, Dich, wenn ich von
               Dir nichts Gegenteiliges höre, morgen, Dienſtag, Nachmittag nach 5 Uhr zu \label{K-L03252-2v}\edtext{beſuchen}{\lemma{\textnormal{\emph{beſuchen}}}\Cendnote{\textnormal{siehe A. S.: \emph{Tagebuch}, 28. 9. 1909}}}\label{K-L03252-2h}. Herzliche Grüße Dir u. Deiner \textcolor{blue}{Frau}{}\ledrightnote{{$\rightarrow$}\textcolor{blue}{Olga Schnitzler}}!\pend
           
\pstart
           Dein {\\[\baselineskip]}\spacefill\mbox{Paul Goldmann.}\pend
           \leftskip=0em{}\endnumbering\briefempfaengerindex{Schnitzler, Arthur@\textsc{Schnitzler, Arthur}!zzzGoldmann, Paul@\emph{von Paul Goldmann}!1909-09-271@{27. 9. 1909}|)be}\mylabel{h}  \normalsize

\doendnotes{C}
\bigskip
\vfill

\clearpage

\footnotesize

\lohead{\textsc{register}}

% Definiere theindex-Environment komplett neu ohne reledmac
\makeatletter
\renewenvironment{theindex}{%
  \section*{\indexname}%
  \setlength{\parindent}{0pt}%
  \setlength{\parskip}{0pt plus 0.3pt}%
  \let\item\@idxitem
}{%
  \clearpage
}
\makeatother

\IfFileExists{\jobname-pw.ind}{\input{\jobname-pw.ind}}{}

\end{document}

      