%% latex-korrekturansicht-vorspann.tex
%% Vorspann für die Korrekturansicht.
%% Lädt die gemeinsame Datei latex-vorspann.tex mit gesetztem Schalter.

\newif\ifkorrekturansicht
\korrekturansichttrue

\input{../tex-inputs/latex-vorspann}


\renewcommand{\erwaehntePersonen}{Personen: Richard Beer-Hofmann, Leo Van-Jung}
\renewcommand{\erwaehnteOrte}{Orte: Frankgasse 1, Hauptbahnhof Salzburg, Hotel Castiglione, IX., Alsergrund, München, Pörtschach, Salzburg, Wien, Zürich}
\renewcommand{\erwaehnteWerke}{Werke: ?? [Text zu Van-Jungs Pfeifertrio], Olga Frohgemuth. Erzählung}
\section[ Felix Salten an Arthur Schnitzler, 2{[}3{]}. 6. 1901]{Felix Salten an Arthur Schnitzler, 2{[}3{]}. 6. 1901}
\nopagebreak\mylabel{v}
\rehead{ }\normalsize\beginnumbering\briefempfaengerindex{Schnitzler, Arthur@\textsc{Schnitzler, Arthur}!zzzSalten, Felix@\emph{von Felix Salten}!1901-06-231@{2{[}3{]}. 6. 1901}|(be}
\toendnotes[C]{\smallbreak\pagebreak[2]}\Standort{CUL, Schnitzler, B 89, A 2.}
\physDesc{Postkarte, 737 Zeichen
\newline{}Handschrift: Bleistift, lateinische Kurrent
\newline{}Versand: Stempel: »\nobreak{}\oindex{Hauptbahnhof Salzburg@\textbf{Hauptbahnhof Salzburg}, \emph{Bahnhofsgebäude (K.BHF)}|pwk}Salzburg-Bahnhof, 23/6 \textcolor{gray}{01}, 3-F.\nobreak{}«. Stempel: »\nobreak{}\oindex{IX., Alsergrund@\textbf{IX., Alsergrund}, \emph{A.ADM3}|pwk}Wien 9/3 72, 24. 6. 01, 8. \textcolor{gray}{V}, Bestellt\nobreak{}«.  
\newline{}Ordnung: mit Bleistift von unbekannter Hand nummeriert: »138« }\toendnotes[C]{\smallbreak}\pstart{}{\pb}Herrn D\textsuperscript{r} Arthur Schnitzler\pend{}\pstart{}\textcolor{pink}{Wien IX.}{}\ledrightnote{\textcolor{pink}{IX., Alsergrund}}\pend{}\pstart{}\textcolor{pink}{Frankgaße 1}{}\ledrightnote{\textcolor{pink}{Frankgasse 1}}\pend{}
{\bigskip}
\pstart
           \raggedleft{}{\pb}\textcolor{pink}{Salzburg, Bahnhof}{}\ledrightnote{\textcolor{pink}{Hauptbahnhof Salzburg}}, \label{K_L03314-1v}\edtext{22. Juni 01}{\lemma{\textnormal{\emph{22. Juni 01}}}\Cendnote{\textnormal{Zwischen der Datums- und der
                     Uhrzeitangabe besteht ein Widerspruch, wenn man den Poststempel vom 23. 6. 1901 hinzuzieht. Es ist davon auszugehen,
                     dass \textcolor{blue}{Salten} sie in der Nacht
                     vom 22. auf den 23. verfasste, korrekterweise also bereits am 23. schrieb. Alternativ wäre die Karte unbearbeitet
                  über 24h liegengeblieben.}}}\label{K_L03314-1h}.\pend
           
\pstart
           \raggedleft{}½ 2. Nachts.
               \pend
           
\pstart
           Lieber Freund, ich komme soeben von \textcolor{pink}{München}{}\ledrightnote{\textcolor{pink}{München}} herüber, warte \textcolor{pink}{hier}{}\ledrightnote{{$\rightarrow$}\textcolor{pink}{Hauptbahnhof Salzburg}} auf den Zug nach \textcolor{pink}{Zürich}{}\ledrightnote{\textcolor{pink}{Zürich}}. Hätte ich
               Ihre \label{K_L03314-2v}\edtext{Adreße}{\lemma{\textnormal{\emph{Adreße}}}\Cendnote{\textnormal{\textcolor{blue}{Schnitzler} war seit dem 12. 6. 1901 in \textcolor{pink}{Salzburg}.}}}\label{K_L03314-2h} hier gewußt, ich hätte Ihnen
               gerne geschrieben\substVorne{}\textsuperscript{,}\substDazwischen{} (\substHinten{}dass Sie auf die \textcolor{pink}{Bahn}{}\ledrightnote{{$\rightarrow$}\textcolor{pink}{Hauptbahnhof Salzburg}}
                  kommen{[}){]}, denn ich bin seit 12 Uhr Nachts hier.
                  Heute{ }früh erhielt ich in \textcolor{pink}{München}{}\ledrightnote{\textcolor{pink}{München}} Ihren
               Brief, der mir, – wie alles – nachgesendet wurde.\pend
           
\pstart
           Meine nächste Adreße ist \textcolor{pink}{Paris, Hotel
                  Castiglione}{}\ledrightnote{\textcolor{pink}{Hotel Castiglione}}. Ich freue mich, dass Sie arbeiten. Ich arbeite hoffentlich auf
               der Reise meinen \label{K_L03314-3v}\edtext{\textcolor{green}{Professor}{}\ledrightnote{{$\rightarrow$}\textcolor{green}{Olga Frohgemuth. Erzählung}}}{\lemma{\textnormal{\emph{Professor}}}\Cendnote{\textnormal{\emph{\textcolor{green}{Olga Frohgemuth}}?}}}\label{K_L03314-3h}, wozu ich viel Lust
               habe.\pend
           
\pstart
           Wissen Sie, wo \label{K_L03314-4v}\edtext{\textcolor{blue}{Beer-Hofmann}{}\ledrightnote{\textcolor{blue}{Richard Beer-Hofmann}}}{\lemma{\textnormal{\emph{Beer-Hofmann}}}\Cendnote{\textnormal{\textcolor{blue}{Richard Beer-Hofmann} hielt sich
                  höchstwahrscheinlich bereits in \textcolor{pink}{Pörtschach am
                     Wörthersee} auf, vgl. Arthur Schnitzler an Richard Beer-Hofmann, 25. 6. 1901.}}}\label{K_L03314-4h} ist? Ich möchte ihm drängen, den \label{K_L03314-5v}\edtext{\textcolor{green}{Text}{}\ledrightnote{{$\rightarrow$}\textcolor{green}{?? [Text zu Van-Jungs Pfeifertrio]}} zu \textcolor{blue}{Van-Jung}{}\ledrightnote{\textcolor{blue}{Leo Van-Jung}}s Pfeifertrio}{\lemma{\textnormal{\emph{Text … Pfeifertrio}}}\Cendnote{\textnormal{nicht ermittelt}}}\label{K_L03314-5h} fertig zu stellen.\pend
           
\pstart
           Leben Sie wol und laßen sich’s gut gehen, und grüßen von mir. {\\[\baselineskip]}Herzlichst
               Ihr \spacefill\mbox{Salten.}\pend
           \leftskip=0em{}\endnumbering\briefempfaengerindex{Schnitzler, Arthur@\textsc{Schnitzler, Arthur}!zzzSalten, Felix@\emph{von Felix Salten}!1901-06-231@{2{[}3{]}. 6. 1901}|)be}\mylabel{h}  \normalsize

\doendnotes{C}
\bigskip
\vfill

\clearpage

\footnotesize

\lohead{\textsc{register}}

% Definiere theindex-Environment komplett neu ohne reledmac
\makeatletter
\renewenvironment{theindex}{%
  \section*{\indexname}%
  \setlength{\parindent}{0pt}%
  \setlength{\parskip}{0pt plus 0.3pt}%
  \let\item\@idxitem
}{%
  \clearpage
}
\makeatother

\IfFileExists{\jobname-pw.ind}{\input{\jobname-pw.ind}}{}

\end{document}

      