%% latex-korrekturansicht-vorspann.tex
%% Vorspann für die Korrekturansicht.
%% Lädt die gemeinsame Datei latex-vorspann.tex mit gesetztem Schalter.

\newif\ifkorrekturansicht
\korrekturansichttrue

\input{../tex-inputs/latex-vorspann}


\renewcommand{\erwaehntePersonen}{Personen: Adolf Grosz}
\renewcommand{\erwaehnteInstitutionen}{Institutionen: Buchhandlung L. Rosner, Wiener Verlag}
\renewcommand{\erwaehnteOrte}{Orte: Wien}
\renewcommand{\erwaehnteWerke}{Werke: Der Hinterbliebene. Kurze Novellen}
\section[ Felix Salten: Widmungsexemplar Der Hinterbliebene für Arthur Schnitzler, 3. 1. 1900]{Felix Salten: Widmungsexemplar Der Hinterbliebene für Arthur
               Schnitzler, 3. 1. 1900}
\nopagebreak\mylabel{v}
\rehead{ }\normalsize\beginnumbering\briefempfaengerindex{Schnitzler, Arthur@\textsc{Schnitzler, Arthur}!zzzSalten, Felix@\emph{von Felix Salten}!1900-01-031@{3. 1. 1900}|(be}
\toendnotes[C]{\smallbreak\pagebreak[2]}\Standort{DLA, G:Schnitzler, Arthur (Sammlung Heinrich Schnitzler).}
\physDesc{Widmung am Titelblatt, 67 Zeichen
\newline{}Handschrift: schwarze Tinte, lateinische Kurrent}\toendnotes[C]{\smallbreak}
\pstart
           \noindent{}{\pb}Meinem lieben Arthur Schnitzler\pend
           
\pstart
           herzlichst{\\[\baselineskip]}\spacefill\mbox{FSalten}\pend
           \leftskip=0em{}
\pstart
           \textcolor{pink}{Wien}{}\ledrightnote{\textcolor{pink}{Wien}}, 3. Jänner 00\pend
           {\bigskip}
\pstart
           \noindent{}\textcolor{gray}{\textbf{\textbf{\uline{Felix Salten.}}}}\pend
           
\pstart
           \centering{}\textcolor{gray}{\textbf{\textbf{\textcolor{green}{Der Hinterbliebene.}{}\ledrightnote{\textcolor{green}{Der Hinterbliebene. Kurze Novellen}}}}}\pend
           
\pstart
           \noindent{}\centering{}\textcolor{gray}{\textbf{KURZE NOVELLEN.}}\pend
           
\pstart
           \noindent{}\centering{}\textcolor{gray}{\textbf{*}}\pend
           
\pstart
           \noindent{}\centering{}\textcolor{gray}{\textbf{Umſchlagbild von \textcolor{blue}{A.
                     Grosz}{}\ledrightnote{\textcolor{blue}{Adolf Grosz}}.}}\pend
           {\bigskip}
\pstart
           \noindent{}\centering{}\textcolor{gray}{\textbf{\textcolor{brown}{WIENER VERLAG}{}\ledrightnote{\textcolor{brown}{Wiener Verlag}}}}\pend
           
\pstart
           \noindent{}\centering{}\textcolor{gray}{\textbf{\textbf{(\textcolor{brown}{Buchhandlung L.
                        Rosner}{}\ledrightnote{\textcolor{brown}{Buchhandlung L. Rosner}}. – \label{K_L03048-1v}\edtext{Sep-.Cto.}{\lemma{\textnormal{\emph{Sep-.Cto.}}}\Cendnote{\textnormal{Abkürzung: separates
                     Konto}}}\label{K_L03048-1h})}}}\pend
           
\pstart
           \noindent{}\centering{}\textcolor{gray}{\textbf{\textbf{1900.}}}\pend
           \endnumbering\briefempfaengerindex{Schnitzler, Arthur@\textsc{Schnitzler, Arthur}!zzzSalten, Felix@\emph{von Felix Salten}!1900-01-031@{3. 1. 1900}|)be}\mylabel{h}  \normalsize

\doendnotes{C}
\bigskip
\vfill

\clearpage

\footnotesize

\lohead{\textsc{register}}

% Definiere theindex-Environment komplett neu ohne reledmac
\makeatletter
\renewenvironment{theindex}{%
  \section*{\indexname}%
  \setlength{\parindent}{0pt}%
  \setlength{\parskip}{0pt plus 0.3pt}%
  \let\item\@idxitem
}{%
  \clearpage
}
\makeatother

\IfFileExists{\jobname-pw.ind}{\input{\jobname-pw.ind}}{}

\end{document}

      