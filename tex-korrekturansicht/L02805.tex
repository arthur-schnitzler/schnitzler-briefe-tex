%% latex-korrekturansicht-vorspann.tex
%% Vorspann für die Korrekturansicht.
%% Lädt die gemeinsame Datei latex-vorspann.tex mit gesetztem Schalter.

\newif\ifkorrekturansicht
\korrekturansichttrue

\input{../tex-inputs/latex-vorspann}


               \section[Paul Goldmann an Arthur Schnitzler, Paul Goldmann an Arthur Schnitzler, 11. 3. {[}1897{]}]{ Paul Goldmann an Arthur Schnitzler, 11. 3. {[}1897{]}}\nopagebreak\mylabel{v}\rehead{ }\normalsize\beginnumbering\briefempfaengerindex{Schnitzler, Arthur@\textsc{Schnitzler, Arthur}!zzzGoldmann, Paul@\emph{von Paul Goldmann}!1897-03-111@{11. 3. {[}1897{]}}|(be} \toendnotes[C]{\smallbreak\pagebreak[2]} \Standort{DLA, A:Schnitzler, HS.NZ85.1.3167.}
\physDesc{Brief, 2 Blätter, 7 Seiten
\newline{}Handschrift: blaue Tinte, deutsche Kurrent
\newline{}Schnitzler: mit Bleistift das Jahr »97« vermerkt }\toendnotes[C]{\smallbreak}\pstart
           \noindent{}{\pb}\textcolor{gray}{\textbf{\textbf{\textcolor{brown}{Frankfurter Zeitung}{}\ledrightnote{\textcolor{brown}{Frankfurter Zeitung}}}}}\pend
           \pstart
           \textcolor{gray}{\textbf{(\textcolor{brown}{\begin{otherlanguage}{french}Gazette de Francfort\end{otherlanguage}}{}\ledrightnote{\textcolor{brown}{Frankfurter Zeitung}}).}}\pend
           \pstart
           \textcolor{gray}{\textbf{\textbf{\begin{otherlanguage}{french}Fondateur M.\end{otherlanguage}{ }\textcolor{blue}{L. Sonnemann}{}\ledrightnote{\textcolor{blue}{Leopold Sonnemann}}.}}}\hfill \textsc{\textcolor{pink}{Paris}{}\ledrightnote{\textcolor{pink}{Paris}}}, 1\substVorne{}\textsuperscript{0}\substDazwischen{}1\substHinten{}. März.\pend
           \pstart
           \begin{otherlanguage}{french}\textcolor{gray}{\textbf{Journal politique, financier,}}\end{otherlanguage}\pend
           \pstart
           \begin{otherlanguage}{french}\textcolor{gray}{\textbf{commercial et littéraire.}}\end{otherlanguage}\pend
           \pstart
           \begin{otherlanguage}{french}\textcolor{gray}{\textbf{\textbf{Paraissant trois fois par jour.}}}\end{otherlanguage}\pend
           \pstart
           \begin{otherlanguage}{french}\textcolor{gray}{\textbf{\textbf{Bureau à \textcolor{pink}{Paris}{}\ledrightnote{\textcolor{pink}{Paris}}}}}\end{otherlanguage}\pend
           \pstart
           \begin{otherlanguage}{french}\textcolor{gray}{\textbf{\textbf{\textcolor{pink}{24. Rue Feydeau}{}\ledrightnote{\textcolor{pink}{rue Feydeau}}.}}}\end{otherlanguage}\pend
           \pstart{}Mein lieber Freund,\pend\pstart
           Ich habe mit der verfluchten \label{K_L02805-1v}\edtext{\textcolor{green}{Orient-Geſchichte}{}\ledrightnote{→\textcolor{green}{[Orient-Geschichte]}}}{\lemma{\textnormal{\emph{Orient-Geſchichte}}}\Cendnote{\textnormal{der sich zunehmend zum
                  (Türkisch-Griechischen) Krieg aufschaukelnde Konflikt auf \textcolor{pink}{Kreta}, vgl. Paul Goldmann an Arthur Schnitzler, 16. 2. [1897]}}}\label{K_L02805-1h} unbändig zu thun. Auch \strikeout{\textcolor{gray}{er}} thut mir mein Auge \strikeout{\textcolor{gray}{f}} unerträglich weh. So kommt es, daß ich Deinen lieben Brief erſt heut beantworte.\pend
           \pstart
           Ich danke Dir von ganzem Herzen für den Beiſtand, den Du mir in der Angelegenheit mit
                  \textsc{\textcolor{blue}{Klein}{}\ledrightnote{\textcolor{blue}{Arthur Klein}}s}{ }\textcolor{blue}{Bruder}{}\ledrightnote{→\textcolor{blue}{Richard Klein}}{ }\label{K_L02805-v}\edtext{geleiſtet}{\lemma{\textnormal{\emph{geleiſtet}}}\Cendnote{\textnormal{er schreibt »geliehen«}}}\label{K_L02805-h}. Ich bin ſelbſt
               wohl auch nicht ohne Schuld an dieſen Unannehmlichkeiten. Ich laſſe mir Leute dieſer
               Art zu nahe kommen, in einer gewiſſen ſchlamperten Liebenswürdigkeit. Auch habe ich
               mich von meiner Heftigkeit zu ſehr hinreißen laſſen. \textsc{\textcolor{blue}{Arthur Klein}{}\ledrightnote{\textcolor{blue}{Arthur Klein}}} hat ſich prachtvoll benommen. {\pb}Wenn Du ihn
               ſiehſt, ſo danke ihm noch beſonders, bitte! Freilich hat es weiterhin noch einige
               Klatſchereien gegeben, und die Unannehmlichkeiten ſind noch nicht zu Ende. \strikeout{Abe\textcolor{gray}{r}} Aber ich mache mir heut große Vorwürfe, Dich
               mit der ganzen Sache behelligt zu haben{\dotsfive}\pend
           \pstart
           Soeben erhalte ich für \strikeout{Euch} Dich und \textsc{\textcolor{blue}{Richard}{}\ledrightnote{\textcolor{blue}{Richard Beer-Hofmann}}} zwei Nummern von »\textsc{\textcolor{green}{Politiken}{}\ledrightnote{\textcolor{green}{Politiken}}}«, wo \textsc{\textcolor{blue}{Peter Nansen}{}\ledrightnote{\textcolor{blue}{Peter Nansen}}} über Dich und zugleich über uns \label{K_L02805-3v}\edtext{\textcolor{green}{geſchrieben}{}\ledrightnote{→\textcolor{green}{Arthur Schnitzler. »Elskovsleg«s Forfatter}}}{\lemma{\textnormal{\emph{geſchrieben}}}\Cendnote{\textnormal{\textcolor{blue}{–n–} [=\textcolor{blue}{Peter Nansen}]: \emph{\textcolor{green}{Arthur
                        Schnitzler. »Elskovsleg«s Forfatter}}. In: \emph{\textcolor{green}{Politiken}}, Nr. 68, 9. 3. 1897, S. 1.}}}\label{K_L02805-3h} hat. Ich verſtehe kein Wort davon,
               aber es ſcheint prächtig zu ſein. \strikeout{Du} Ich ſende beide
                  \textcolor{green}{Nummern}{}\ledrightnote{→\textcolor{green}{Politiken}} an Dich.\pend
           \pstart
           Meine \label{K_L02805-5v}\edtext{Reiſe nach \textsc{\textcolor{pink}{Nizza}{}\ledrightnote{\textcolor{pink}{Nizza}}}}{\lemma{\textnormal{\emph{Reiſe nach Nizza}}}\Cendnote{\textnormal{siehe Paul Goldmann an Arthur Schnitzler, 16. 2. [1897]}}}\label{K_L02805-5h} iſt infolge der Orient-Ereigniſſe auf nächſte Woche verſchoben. {\pb}Ich kann Dir gar nicht ſagen, wie ich mich auf Dein
               Kommen freue! Ein vorheriges Zuſammentreffen in der \label{K_L02805-9v}\edtext{\textcolor{pink}{Schweiz}{}\ledrightnote{\textcolor{pink}{Schweiz}}}{\lemma{\textnormal{\emph{Schweiz}}}\Cendnote{\textnormal{\textcolor{blue}{Schnitzler} war am 10. 4. 1897 und 11. 4. 1897 in \textcolor{pink}{Zürich}. Er kam gerade aus \textcolor{pink}{München} und reiste nach \textcolor{pink}{Paris} weiter.}}}\label{K_L02805-9h} iſt leider unmöglich. Ich darf mich nicht vom Flecke
               rühren; hoffentlich habe ich nur hier während Deiner Anweſenheit wenig zu thun, damit
               ich Dich ordentlich genießen kann. Die Wohnungsfrage wird freilich nicht leicht zu
               erledigen ſein. \strikeout{D} Ich habe nochmals energiſcheſte
               Nachforſchungen angeſtellt. Das Reſultat iſt das, was ich gewußt hatte: Anſtändige
                  \textcolor{pink}{fran}{}\ledrightnote{→\textcolor{pink}{Frankreich}}zöſiſche Familien geben
               keine \textsc{Pension}, und diejenigen Familien, welche \textsc{Pension} geben, ſind nicht anſtändig. Ausnahmen gibt {\pb}es wohl, aber eine ſolche zu finden, iſt reine
               Zufallsſache. Im Übrigen denke auch ich, daß Du irgendwo zwiſchen Stadt und Land
               wohnen ſollſt, am Beſten in \textsc{\textcolor{pink}{Passy}{}\ledrightnote{\textcolor{pink}{16. Arrondissement (Passy)}}}, das beſonders anmuthig und zugleich bequem iſt. Was ich Dir ſage, ſind keine
               definitiven Reſultate. Ich habe einige \textcolor{pink}{fran}{}\ledrightnote{→\textcolor{pink}{Frankreich}}zöſiſche Bekannte mit Umfragen beauftragt, und die
               Nachforſchungen dauern fort. Ein \textsc{Hotel}, wie Du es
               wünſcheſt, wird raſch gefunden ſein, ſobald Du mir das Datum \strikeout{meiner} Deiner Ankunft mittheilſt. Allzuviel \textsc{Comfort} wirſt Du freilich nicht finden. Das \textcolor{pink}{Pariſ}{}\ledrightnote{\textcolor{pink}{Paris}}er {\pb}Hotelweſen iſt ſehr
               zurück. Das hat ſchon \label{K_L02805-888v}\edtext{\textsc{\textcolor{blue}{Balzac}{}\ledrightnote{\textcolor{blue}{Honoré de Balzac}}} conſtatirt}{\lemma{\textnormal{\emph{Balzac conſtatirt}}}\Cendnote{\textnormal{\textcolor{blue}{Balzac} thematisierte die Beherbergungsindustrie in \textcolor{pink}{Paris} in mehreren seiner Bücher. Er beschrieb
                   die Hotels als überfüllt, schmutzig und überteuert, mit schlechtem Service und
                  wenig Privatsphäre. Kritisiert wurde von ihm auch die Eigentümerinnen und
                  Eigentümer dieser Hotels, die die Bedürfnisse der Reisenden ausnutzten und
                  überhöhte Preise für minderwertige Unterkünfte verlangten: »il n’existe pas encore
                  un seul hôtel où tout voyageur riche puisse retrouver son chez soi« (»es gibt
                  bislang kein einziges Hotel, in dem selbst ein reicher Reisende sich zuhause
                  fühlen kann«; \emph{\textcolor{green}{Illusions Perdues}}, 2.
                     Teil)}}}\label{K_L02805-888h}, und ſeit \textsc{\textcolor{blue}{Balzac}{}\ledrightnote{\textcolor{blue}{Honoré de Balzac}}} hat ſich wenig geändert{\dotsseven}\pend
           \pstart
           Was Du mir über Deine \textcolor{blue}{Freundin}{}\ledrightnote{→\textcolor{blue}{Marie Reinhard}} ſchreibſt, iſt ſehr ſchön. Ich habe nie daran gezweifelt, daß ſie
               »auf unſerem \textsc{Niveau}« iſt, ſchon weil ſie Deine \textcolor{blue}{Freundin}{}\ledrightnote{→\textcolor{blue}{Marie Reinhard}} iſt. Du kannſt Dir
               denken, wie ich mich darauf freue, ſie kennen zu lernen. Darf ich Dich einſtweilen
               bitten, mich ihr zu empfehlen?{\dotsfour}\pend
           \pstart
           Nach der ſo gut verlaufenen \label{K_L02805-11v}\edtext{Unterredung mit dem {\pb}\textcolor{blue}{Vater}{}\ledrightnote{→\textcolor{blue}{Carl Reinhard}}}{\lemma{\textnormal{\emph{Unterredung … Vater}}}\Cendnote{\textnormal{siehe A. S.: \emph{Tagebuch}, 23. 2. 1897 und Paul Goldmann an Arthur Schnitzler, 24. 2. [1897]}}}\label{K_L02805-11h} ſind wohl die ſchlimmſten Unannehmlichkeiten vorüber. Ich halte es für ein
               großes Glück, daß ein äußerer Zwang Dich auf einige Zeit von \textcolor{pink}{Wien}{}\ledrightnote{\textcolor{pink}{Wien}} wegtreibt. Ich verſpreche mir viel von der Wirkung, die \textcolor{pink}{Paris}{}\ledrightnote{\textcolor{pink}{Paris}} auf Dich haben wird. Es wird Dich
               elektriſiren, und Dich mit Schaffenskraft und Schaffensluſt erfüllen. Auch wirſt Du
               den \textcolor{pink}{Pariſ}{}\ledrightnote{\textcolor{pink}{Paris}}er Frühling ſehen, welcher eine der
               Gnaden Gottes iſt.\pend
           \pstart
           Freilich könnte es ſich auch ereignen, daß Dir hier Alles ſehr zuwider iſt.\pend
           \pstart
           {\pb}Wir wollen den Himmel bitten, daß es gut
               ausgeht.\pend
           \pstart
           Bald höre ich wohl Näheres?\pend
           \pstart
           Ich begrüße Dich von Herzen\pend
           \pstart
           Dein {\\[\baselineskip]}\spacefill\mbox{Paul Goldmann}\pend
           \leftskip=0em{}\pstart
           \noindent{}Schön habt Ihr wieder in \textsc{\textcolor{pink}{Wien}{}\ledrightnote{\textcolor{pink}{Wien}}}{ }\label{K_L02805-34v}\edtext{gewählt}{\lemma{\textnormal{\emph{gewählt}}}\Cendnote{\textnormal{Am 4. 3. 1897 begannen in
                        \textcolor{pink}{Cisleithanien}, dem
                     nördlichen und westlichen Teils \textcolor{pink}{Österreich-Ungarn}s, die \emph{\textcolor{brown}{Reichsrat}}s-, also \textcolor{brown}{Parlament}swahlen. In \textcolor{pink}{Wien} feierte
                     insbesondere die \emph{\textcolor{brown}{Christlichsoziale Partei}}
                     Erfolge. \textcolor{blue}{Schnitzler} notierte dazu am 12. 3. 1897:
                        »Sehr verstimmt, auch durch den Antisem.–«}}}\label{K_L02805-34h}. Ihr ſeid
                  eine rechte Bagage. Schämt Ihr Euch gar nicht vor \textcolor{pink}{Europa}{}\ledrightnote{\textcolor{pink}{Europa}}?\pend
           \endnumbering\briefempfaengerindex{Schnitzler, Arthur@\textsc{Schnitzler, Arthur}!zzzGoldmann, Paul@\emph{von Paul Goldmann}!1897-03-111@{11. 3. {[}1897{]}}|)be}\mylabel{h}  \normalsize

\doendnotes{C}
\bigskip
\vfill

\clearpage

\footnotesize

\lohead{\textsc{register}}

% Definiere theindex-Environment komplett neu ohne reledmac
\makeatletter
\renewenvironment{theindex}{%
  \section*{\indexname}%
  \setlength{\parindent}{0pt}%
  \setlength{\parskip}{0pt plus 0.3pt}%
  \let\item\@idxitem
}{%
  \clearpage
}
\makeatother

\IfFileExists{\jobname-pw.ind}{\input{\jobname-pw.ind}}{}

\end{document}

      