%% latex-korrekturansicht-vorspann.tex
%% Vorspann für die Korrekturansicht.
%% Lädt die gemeinsame Datei latex-vorspann.tex mit gesetztem Schalter.

\newif\ifkorrekturansicht
\korrekturansichttrue

\input{../tex-inputs/latex-vorspann}


\section[Arthur Schnitzler an Richard Beer-Hofmann, 12. 1. 1892]{L00060 Arthur Schnitzler an Richard Beer-Hofmann, 12. 1. 1892}
\nopagebreak\mylabel{L00060v}
\rehead{ }\normalsize\beginnumbering\briefempfaengerindex{, @\textsc{, }!zzz, @\emph{von  }!1892-01-121@{12. 1. 1892}|(be}
\toendnotes[C]{\smallbreak\pagebreak[2]}\Standort{YCGL, MSS 31.}
\physDesc{Postkarte, 223 Zeichen
\newline{}Handschrift: 1) Bleistift, deutsche Kurrent\hspace{1em}2) Bleistift, lateinische Kurrent (\noindent{}Adresse)\hspace{1em}
\newline{}Versand: 1) Rohrpost  2) Stempel: »\nobreak{}\oindex{\oindex{XXXX Ortsangabe fehlt\oindex{XXXX Ortsangabe fehltWien@\textbf{Wien}, \emph{Verwaltungsgebiet}|pwk}Wien
                                                  Fleischmarkt, 1\textcolor{gray}{2} 1 92, 3–4 N\nobreak{}«.  3) Stempel: »\nobreak{}\oindex{\oindex{XXXX Ortsangabe fehlt\oindex{\oindex{XXXX Ortsangabe fehlt\oindex{XXXX Ortsangabe fehltWien@\textbf{Wien}, \emph{Verwaltungsgebiet}\oindex{XXXX Ortsangabe fehltIII., Landstraße@\textbf{III., Landstraße}, \emph{Verwaltungsgebiet}|pwk}Wien 3/1, 12. 11. 92, 4–5 N\nobreak{}«. }\pstart{}{\pb}Hrn Dr. Richard Beer-Hofmann\pend{}\pstart{}\textcolor{pink}{Wien}\oindex{\oindex{XXXX Ortsangabe fehlt\oindex{XXXX Ortsangabe fehltWien@\textbf{Wien}, \emph{Verwaltungsgebiet}|pw}{}\ledrightnote{\textcolor{pink}{Wien}}\pend{}\pstart{}\textcolor{pink}{III. Seidlgasse 30}\oindex{Wien@\textbf{Wien}!III., Landstraße@\textbf{III., Landstraße}!Seidlgasse@\textbf{Seidlgasse}, \emph{Straße}|pw}{}\ledrightnote{\textcolor{pink}{Seidlgasse}}.\pend{}{\bigskip}\vspace{1em}
\pstart
           \noindent{}{\pb}Lieber Richard,\pend
           \settowidth{\longeste}{P. G.:}\settowidth{\longestz}{Paris, 51, rue Vivienne,}\settowidth{\longestd}{}\settowidth{\longestv}{}\settowidth{\longestf}{}\addtolength\longeste{1em}
        \addtolength\longestz{1em}
      \pstart\noindent\makebox[\the\longeste][l]{\textcolor{blue}{P. G.}\pwindex{Goldmann, Paul 31.\,1.\,1865 Breslau – 25.\,9.\,1935 Wien@\textsc{Goldmann, Paul} (31.\,1.\,1865 Breslau – 25.\,9.\,1935 Wien), \emph{Schriftsteller, Journalist}|pw}{}\ledrightnote{\textcolor{blue}{Paul Goldmann}}:}\makebox[\the\longestz][l]{\textsc{\textcolor{pink}{Paris, 51, rue Vivienne}\oindex{\oindex{XXXX Ortsangabe fehltrue Vivienne@\textbf{rue Vivienne}, \emph{Straße}|pw}{}\ledrightnote{\textcolor{pink}{rue Vivienne}},}}
                  \pend\pstart\noindent\makebox[\the\longeste][l]{}\makebox[\the\longestz][l]{\textsc{\textcolor{brown}{gazette de Francfort}\orgindex{Frankfurter Zeitung@Frankfurter Zeitung|pw}{}\ledrightnote{\textcolor{brown}{Frankfurter Zeitung}}.}}
                  \pend
\pstart
           Ob ich heute im Caffé weiſs ich nicht; hoffentlich aber ſehen wir uns \uline{ſehr} bald.\pend
           \pstart Herzlich Ihr getreuer\spacefill\mbox{Arthur.}\pend{}\selectlanguage{ngerman}\endnumbering\briefempfaengerindex{, @\textsc{, }!zzz, @\emph{von  }!1892-01-121@{12. 1. 1892}|)be}\mylabel{L00060h}  \normalsize

\doendnotes{C}
\bigskip
\vfill

\clearpage

\footnotesize

\lohead{\textsc{register}}

% Definiere theindex-Environment komplett neu ohne reledmac
\makeatletter
\renewenvironment{theindex}{%
  \section*{\indexname}%
  \setlength{\parindent}{0pt}%
  \setlength{\parskip}{0pt plus 0.3pt}%
  \let\item\@idxitem
}{%
  \clearpage
}
\makeatother

\IfFileExists{\jobname-pw.ind}{\input{\jobname-pw.ind}}{}

\end{document}

      