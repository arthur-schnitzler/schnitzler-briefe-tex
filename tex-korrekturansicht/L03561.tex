%% latex-korrekturansicht-vorspann.tex
%% Vorspann für die Korrekturansicht.
%% Lädt die gemeinsame Datei latex-vorspann.tex mit gesetztem Schalter.

\newif\ifkorrekturansicht
\korrekturansichttrue

\input{../tex-inputs/latex-vorspann}


\renewcommand{\erwaehntePersonen}{Personen: Lili Cappellini, Felix Salten, Ottilie Salten, Heinrich Schnitzler, Olga Schnitzler}
\renewcommand{\erwaehnteOrte}{Orte: Berghof, Dresden, Frauenkirche, Sternwartestraße 71, Wien}
\renewcommand{\erwaehnteWerke}{}
\section[ Felix Salten an Arthur Schnitzler, 24. 6. 1913]{Felix Salten an Arthur Schnitzler, 24. 6. 1913}
\nopagebreak\mylabel{v}
\rehead{ }\normalsize\beginnumbering\briefempfaengerindex{Schnitzler, Arthur@\textsc{Schnitzler, Arthur}!zzzSalten, Felix@\emph{von Felix Salten}!1913-06-241@{24. 6. 1913}|(be}
\toendnotes[C]{\smallbreak\pagebreak[2]}\Standort{CUL, Schnitzler, B 89, B 2.}
\physDesc{Bildpostkarte, 364 Zeichen
\newline{}Handschrift: schwarze Tinte, lateinische Kurrent
\newline{}Versand: Stempel: »\nobreak{}\oindex{Dresden@\textbf{Dresden}, \emph{P.PPLA}|pwk}Dr{[}esden{]}
                                          Altst. \textcolor{gray}{2}4 f, 24. 6. 13, 6–7 N.\nobreak{}«.  
\newline{}Ordnung: mit Bleistift von unbekannter Hand datiert: »? 1913?« }\toendnotes[C]{\smallbreak}\pstart{}{\pb}Herrn\pend{}\pstart{}D\textsuperscript{r} Arthur Schnitzler\pend{}\pstart{}\textcolor{pink}{Wien}{}\ledrightnote{\textcolor{pink}{Wien}}\pend{}\pstart{}\textcolor{pink}{XVIII. Sternwartestraße 71}{}\ledrightnote{\textcolor{pink}{Sternwartestraße 71}}\pend{}
{\bigskip}
\pstart
           \noindent{}\centering{}{\pb}\textcolor{gray}{\textbf{Altstadt mit \textcolor{pink}{Frauenkirche}{}\ledrightnote{\textcolor{pink}{Frauenkirche}}\textcolor{gray}{,}}}\pend
           
\pstart
           \noindent{}\centering{}\textcolor{gray}{\textbf{\textcolor{pink}{Dresden}{}\ledrightnote{\textcolor{pink}{Dresden}}.}}\pend
           
\pstart{}{\pb}Lieber,\pend
\pstart
           danke schön für Ihr \label{K_L03561-1v}\edtext{Telegramm}{\lemma{\textnormal{\emph{Telegramm}}}\Cendnote{\textnormal{nicht erhalten}}}\label{K_L03561-1h}. \textcolor{blue}{Otti}{}\ledrightnote{\textcolor{blue}{Ottilie Salten}} hat mir vom \textcolor{pink}{Berghof}{}\ledrightnote{\textcolor{pink}{Berghof}}
               aus bis jetzt nur Depeschen u. keinen Brief geschickt, so wußte ich nichts, und war
               beunruhigt. Gestern kam zugleich mit Ihrer Antwort
               auch \textcolor{blue}{Otti}{}\ledrightnote{\textcolor{blue}{Ottilie Salten}}’s Brief. Ich freue mich \uline{sehr}, dass es \label{K_L03561-2v}\edtext{\textcolor{blue}{Heini}{}\ledrightnote{\textcolor{blue}{Heinrich Schnitzler}} so gut geht}{\lemma{\textnormal{\emph{Heini so gut geht}}}\Cendnote{\textnormal{Am 10. 6. 1913 war \textcolor{blue}{Heinrich
                     Schnitzler} an Scharlach erkrankt.}}}\label{K_L03561-2h}!\pend
           
\pstart
           Viele herzliche Grüße für Sie, \textcolor{blue}{Olga}{}\ledrightnote{\textcolor{blue}{Olga Schnitzler}} und
               die \textcolor{blue}{Kinder}{}\ledrightnote{{$\rightarrow$}\textcolor{blue}{Heinrich Schnitzler}{\newline}{$\rightarrow$}\textcolor{blue}{Lili Cappellini}}.
               {\\[\baselineskip]}Ihr {\\[\baselineskip]}\spacefill\mbox{Salten}\pend
           \leftskip=0em{}\endnumbering\briefempfaengerindex{Schnitzler, Arthur@\textsc{Schnitzler, Arthur}!zzzSalten, Felix@\emph{von Felix Salten}!1913-06-241@{24. 6. 1913}|)be}\mylabel{h}  \normalsize

\doendnotes{C}
\bigskip
\vfill

\clearpage

\footnotesize

\lohead{\textsc{register}}

% Definiere theindex-Environment komplett neu ohne reledmac
\makeatletter
\renewenvironment{theindex}{%
  \section*{\indexname}%
  \setlength{\parindent}{0pt}%
  \setlength{\parskip}{0pt plus 0.3pt}%
  \let\item\@idxitem
}{%
  \clearpage
}
\makeatother

\IfFileExists{\jobname-pw.ind}{\input{\jobname-pw.ind}}{}

\end{document}

      