%% latex-korrekturansicht-vorspann.tex
%% Vorspann für die Korrekturansicht.
%% Lädt die gemeinsame Datei latex-vorspann.tex mit gesetztem Schalter.

\newif\ifkorrekturansicht
\korrekturansichttrue

\input{../tex-inputs/latex-vorspann}


               \section[ Paul Goldmann an Arthur Schnitzler, 28. 2. {[}1898{]}]{Paul Goldmann an Arthur Schnitzler, 28. 2. {[}1898{]}}\nopagebreak\mylabel{v}\rehead{ }\normalsize\beginnumbering\briefempfaengerindex{Schnitzler, Arthur@\textsc{Schnitzler, Arthur}!zzzGoldmann, Paul@\emph{von Paul Goldmann}!1898-02-281@{28. 2. {[}1898{]}}|(be} \toendnotes[C]{\smallbreak\pagebreak[2]} \Standort{DLA, A:Schnitzler, HS.NZ85.1.3168.}
\physDesc{Brief, 1 Blatt, 4 Seiten
\newline{}Handschrift: blaue Tinte, lateinische Kurrent
\newline{}Schnitzler: mit Bleistift das Jahr »98« vermerkt }\toendnotes[C]{\smallbreak}\pstart
           \noindent{}{\pb}\textcolor{gray}{\textbf{\textbf{\textcolor{brown}{Frankfurter Zeitung}{}\ledrightnote{\textcolor{brown}{Frankfurter Zeitung}}}}}\pend
           \pstart
           \textcolor{gray}{\textbf{(\textcolor{brown}{\begin{otherlanguage}{french}Gazette de Francfort\end{otherlanguage}}{}\ledrightnote{\textcolor{brown}{Frankfurter Zeitung}}).}}\pend
           \pstart
           \textcolor{gray}{\textbf{\textbf{\begin{otherlanguage}{french}Fondateur M.\end{otherlanguage}{ }\textcolor{blue}{L. Sonnemann}{}\ledrightnote{\textcolor{blue}{Leopold Sonnemann}}.}}}\pend
           \pstart
           \begin{otherlanguage}{french}\textcolor{gray}{\textbf{Journal politique, financier,}}\end{otherlanguage}\hfill \textsc{\textcolor{pink}{Paris}{}\ledrightnote{\textcolor{pink}{Paris}}}, 28. Februar. \pend
           \pstart
           \begin{otherlanguage}{french}\textcolor{gray}{\textbf{commercial et littéraire.}}\end{otherlanguage}\pend
           \pstart
           \begin{otherlanguage}{french}\textcolor{gray}{\textbf{\textbf{Paraissant trois fois par jour.}}}\end{otherlanguage}\pend
           \pstart
           \begin{otherlanguage}{french}\textcolor{gray}{\textbf{\textbf{Bureau à \textcolor{pink}{Paris}{}\ledrightnote{\textcolor{pink}{Paris}}}}}\end{otherlanguage}\pend
           \pstart
           \begin{otherlanguage}{french}\textcolor{gray}{\textbf{\textbf{\textcolor{pink}{10 Rue de la Bourse}{}\ledrightnote{\textcolor{pink}{rue de la Bourse}}.}}}\end{otherlanguage}\pend
           \pstart\center{}Mein lieber Freund,\pend\pstart
           Dieſe fürchterlichen drei Wochen \label{K_L02839-1v}\edtext{\textsc{\textcolor{blue}{Zola}{}\ledrightnote{\textcolor{blue}{Émile Zola}}}-Prozeß}{\lemma{\textnormal{\emph{Zola-Prozeß}}}\Cendnote{\textnormal{\textcolor{blue}{Émile Zola} wurde am 23. 2. 1898 wegen Verleumdung im Zuge der \textcolor{blue}{Dreyfus}-Affäre schuldig gesprochen und zu einem Jahr
                  Gefängnis sowie einer Geldstrafe verurteilt. Wegen eines Verfahrensfehlers wurde
                  dieses Urteil am 2. 4. 1898 wieder
                  aufgehoben.}}}\label{K_L02839-1h} ſind voüber. Ich komme endlich wieder einmal zu mir und – zu
               Dir.\pend
           \pstart
           Sehr gefreut hat es mich, daß \label{K_L02839-2v}\edtext{Du und
                  \textsc{\textcolor{blue}{Richard}{}\ledrightnote{\textcolor{blue}{Richard Beer-Hofmann}}} in \textcolor{pink}{Salzburg}{}\ledrightnote{\textcolor{pink}{Salzburg}}}{\lemma{\textnormal{\emph{Du … Salzburg}}}\Cendnote{\textnormal{\textcolor{blue}{Richard Beer-Hofmann} und \textcolor{blue}{Schnitzler} waren von 7. 2. 1898 bis 13. 2. 1898 gemeinsam in \textcolor{pink}{Salzburg}.}}}\label{K_L02839-2h} meiner gedacht habt. Ich danke Euch für Eure
               liebe Karte.\pend
           \pstart
           Dein lieber Brief war auch ſehr ſchön, aber er ſollte doch etwas heiterer ſein.
               Lieber Sohn, verbittere \strikeout{d\textcolor{gray}{oc}h} Dir
               doch nicht ſo Deines Lebens ſchönſte Zeit! Laß’ es in Deinem \label{K_L02839-3v}\edtext{Ohre klingen}{\lemma{\textnormal{\emph{Ohre klingen}}}\Cendnote{\textnormal{Bezug auf \textcolor{blue}{Schnitzler}s
                  Otosklerose – einer Verknöcherung des Innenohrs mit zunehmender Schwerhörigkeit –,
                  an der er seit Herbst 1896 litt.}}}\label{K_L02839-3h}, wenn es nun
               ſchon durchaus nicht anders will. Aber iſt denn das {\pb}etwas Ernſtes? \label{K_L02839-4v}\edtext{\begin{otherlanguage}{french}\textsc{C’est embêtant, voilà tout.}\end{otherlanguage}}{\lemma{\textnormal{\emph{C’est … tout.}}}\Cendnote{\textnormal{französisch: Es ist ärgerlich, das ist
                  alles.}}}\label{K_L02839-4h} Und Jeder hat ſein \label{K_L02839-8v}\edtext{\begin{otherlanguage}{french}\textsc{embétement}\end{otherlanguage}}{\lemma{\textnormal{\emph{embétement}}}\Cendnote{\textnormal{französisch: Unannehmlichkeit}}}\label{K_L02839-8h},
               und Du haſt abſolut kein Recht, ein Leben ohne \begin{otherlanguage}{french}\textsc{embétement}\end{otherlanguage} zu beanſpruchen. Sei froh, daß Du nichts Schlimmeres haſt. Hindert Dich
               das an irgend etwas Weſentlichem? Schaffen, Erleben, \label{K_L02839-12v}\edtext{\begin{otherlanguage}{french}\textsc{faire l’amour}\end{otherlanguage}}{\lemma{\textnormal{\emph{faire l’amour}}}\Cendnote{\textnormal{französisch: Liebe machen}}}\label{K_L02839-12h}? Nein;
               alſo laß’ \strikeout{\textcolor{gray}{×}} es klingen! Und wenn Du meinſt, es mache dir das Arbeiten unmöglich, ſo \strikeout{halte} halte ich das für einen Fehlſchluß, und ich
               glaube, Du ſchiebſt auf das Ohrenklingen nur \strikeout{\textcolor{gray}{en}} den \strikeout{Meng} Mangel an Inſpiration, welcher daher
               kommt, daß Du zu feſt und zu warm ſitzeſt in Deinem \label{K_L02839-14v}\edtext{\textsc{Phaeaken}-Neſt}{\lemma{\textnormal{\emph{Phaeaken-Neſt}}}\Cendnote{\textnormal{Die Phaiaken sind ein Volk der \textcolor{pink}{griech}ischen Mythologie. »Phaeaken-Neſt« meint im
                  übertragenen Sinne einen Ort, an dem Menschen faul im Luxus leben.}}}\label{K_L02839-14h}.\pend
           \pstart
           {\pb}Das \label{K_L02839-5v}\edtext{\textcolor{green}{Feuilleton}{}\ledrightnote{→\textcolor{green}{Feuilleton. Carl-Theater. (»Freiwild«, Schauspiel von Arthur Schnitzler.)}} von \textsc{\textcolor{blue}{Herzl}{}\ledrightnote{\textcolor{blue}{Theodor Herzl}}}}{\lemma{\textnormal{\emph{Feuilleton von Herzl}}}\Cendnote{\textnormal{[\textcolor{blue}{Theodor Herzl}]: \emph{\textcolor{green}{Feuilleton. Carl-Theater. (»Freiwild«, Schauspiel von
                        Arthur Schnitzler.)}}. In: \emph{\textcolor{green}{Neue Freie
                        Presse}}, Nr. 12024, 13. 2. 1898,
                     S. 1–2. Siehe auch A. S.: \emph{Tagebuch}, 13. 2. 1898.}}}\label{K_L02839-5h}, von welchem Du ſchreibſt, habe ich nicht geleſen. Könnteſt Du mir es
               nicht ſchicken?\pend
           \pstart
           Mach’ Dich mit der erſten warmen Frühlings-Sonne auf und fahre Deinen Hypochondrien
               davon, weit in die Welt hinaus. Wenn Du erſt einmal draußen biſt, wirſt Du ſelbſt
               erſtaunen, was für ein Kerl Du biſt!\pend
           \pstart
           Der \label{K_L02839-11v}\edtext{\textsc{\textcolor{blue}{Zola}{}\ledrightnote{\textcolor{blue}{Émile Zola}}}-Prozeß}{\lemma{\textnormal{\emph{Zola-Prozeß}}}\Cendnote{\textnormal{siehe Paul Goldmann an Arthur Schnitzler, 6. 2. [1898]}}}\label{K_L02839-11h} hat Dir wohl auch bis zum Ende gut gefallen. Es iſt intereſſant, \strikeout{daß} wenn man plötzlich merkt, daß man wieder mitten im
               Mittelalter lebt. Aber es iſt auch gut ſo, daß \strikeout{\textcolor{gray}{m}} wir wieder die alten Feinde vor uns haben. \strikeout{W\textcolor{gray}{om}} Das gibt einen ſchönen Kampf, und {\pb}man weiß
               doch wenigſtens, \strikeout{\textcolor{gray}{e}} wozu man auf der Welt iſt und verliert ſich nicht mehr ins Bodenloſe, wie beim
                  \label{K_L02839-32v}\edtext{Aufſuchen der »neuen Künſte« und
               der »neuen Wahrheiten«}{\lemma{\textnormal{\emph{Aufſuchen … Wahrheiten«}}}\Cendnote{\textnormal{Anspielung auf
                  diverse Erneuerungsideen zur Zeit des \begin{otherlanguage}{french}Fin-de-siècle\end{otherlanguage}}}}\label{K_L02839-32h}. Es gibt eben in Wirklichkeit nirgends \strikeout{d} und
               niemals etwas Neues, und das Einzige, wozu wir Menſchen fähig ſind, iſt, daß wir
               immer das Alte wiedererleben, als Individuen wie als Völker\textcolor{gray}{:} Wir
               leben ewig in der Vergangenheit, in »Leben, wie es iſt«, und eine Sinnes-Täuſchung
               zeigt uns den Ausblick auf das »Leben, wie es ſein ſollte« (wie es aber niemals ſein
               wird), \strikeout{\textcolor{gray}{da}} auf die Zukunft{\dotsfive}\pend
           \pstart
           Im Sommer? Wie gern möchte ich Dich wiederſehen! Aber ich weiß zur Stunde noch nicht,
               wie ſich gewiſſe Dinge geſtalten werden, welche meine \textcolor{brown}{Redaction}{}\ledrightnote{→\textcolor{brown}{Frankfurter Zeitung}} projectirt. Sei von Herzen
               gegrüßte!\pend
           \pstart
           Viele Grüße an Deine \textcolor{blue}{Freundin}{}\ledrightnote{→\textcolor{blue}{Marie Reinhard}}!\hfill Dein treuer {\\}\spacefill\mbox{Paul Goldmn}\pend
           \endnumbering\briefempfaengerindex{Schnitzler, Arthur@\textsc{Schnitzler, Arthur}!zzzGoldmann, Paul@\emph{von Paul Goldmann}!1898-02-281@{28. 2. {[}1898{]}}|)be}\mylabel{h}\begin{anhang}\end{anhang}\normalsize

\doendnotes{C}
\bigskip
\vfill

\clearpage

\footnotesize

\lohead{\textsc{register}}

% Definiere theindex-Environment komplett neu ohne reledmac
\makeatletter
\renewenvironment{theindex}{%
  \section*{\indexname}%
  \setlength{\parindent}{0pt}%
  \setlength{\parskip}{0pt plus 0.3pt}%
  \let\item\@idxitem
}{%
  \clearpage
}
\makeatother

\IfFileExists{\jobname-pw.ind}{\input{\jobname-pw.ind}}{}

\end{document}

      