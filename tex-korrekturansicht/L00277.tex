%% latex-korrekturansicht-vorspann.tex
%% Vorspann für die Korrekturansicht.
%% Lädt die gemeinsame Datei latex-vorspann.tex mit gesetztem Schalter.

\newif\ifkorrekturansicht
\korrekturansichttrue

\input{../tex-inputs/latex-vorspann}


               \section[Hugo von Hofmannsthal und Felix Salten an Arthur Schnitzler, {[}27. 10. 1893{]}]{ Hugo von Hofmannsthal und Felix Salten an Arthur Schnitzler,
                    {[}27. 10. 1893{]}}\nopagebreak\mylabel{v}\rehead{ }\normalsize\beginnumbering\briefempfaengerindex{Schnitzler, Arthur@\textsc{Schnitzler, Arthur}!zzzSalten, Felix@\emph{von Felix Salten}!1893-10-271@{{[}27. 10. 1893{]}}|(be}\briefempfaengerindex{Schnitzler, Arthur@\textsc{Schnitzler, Arthur}!zzzHofmannsthal, Hugo von@\emph{von Hugo von Hofmannsthal}!1893-10-271@{{[}27. 10. 1893{]}}|(be} \toendnotes[C]{\smallbreak\pagebreak[2]} \Standort{CUL, Schnitzler, B 43.}
\physDesc{Brief, 1 Blatt, 1 Seite
\newline{}Handschrift Felix Salten: Bleistift, lateinische Kurrent\newline{}Handschrift Hugo von Hofmannsthal: Bleistift, deutsche Kurrent
\newline{}Schnitzler: mit Bleistift datiert: »2\substVorne{}\textsuperscript{8}\substDazwischen{}7\substHinten{}/10 93« und nummeriert: »59« }\buchAbdrucke{\weitereDrucke{Hugo von Hofmannsthal, Arthur Schnitzler: \emph{Briefwechsel}. Hg. Therese Nickl und Heinrich Schnitzler. Frankfurt am Main: \emph{S. Fischer} 1964, S. 47.} }\pstart
           \raggedleft{}{\pb}½ 8 Uhr.\pend
           \pstart{}Lieber Arthur!\pend\pstart
           Wir kommen heute ſchon um \uuline{10} ins Cafe!\pend
           \pstart \spacefill\mbox{{[}hs. Salten:{]} Salten\hspace*{3.5em}{[}hs. Hofmannsthal:{]} Loris}\pend{}\pstart
           \noindent{}{[}hs. Salten:{]} Der Loris hat vergessen zu schreiben dass wir
                        jetzt gehen zu \textcolor{blue}{Richard}{}\ledrightnote{\textcolor{blue}{Richard Beer-Hofmann}}.\pend
           \endnumbering\briefempfaengerindex{Schnitzler, Arthur@\textsc{Schnitzler, Arthur}!zzzSalten, Felix@\emph{von Felix Salten}!1893-10-271@{{[}27. 10. 1893{]}}|)be}\briefempfaengerindex{Schnitzler, Arthur@\textsc{Schnitzler, Arthur}!zzzHofmannsthal, Hugo von@\emph{von Hugo von Hofmannsthal}!1893-10-271@{{[}27. 10. 1893{]}}|)be}\mylabel{h}  \normalsize

\doendnotes{C}
\bigskip
\vfill

\clearpage

\footnotesize

\lohead{\textsc{register}}

% Definiere theindex-Environment komplett neu ohne reledmac
\makeatletter
\renewenvironment{theindex}{%
  \section*{\indexname}%
  \setlength{\parindent}{0pt}%
  \setlength{\parskip}{0pt plus 0.3pt}%
  \let\item\@idxitem
}{%
  \clearpage
}
\makeatother

\IfFileExists{\jobname-pw.ind}{\input{\jobname-pw.ind}}{}

\end{document}

      