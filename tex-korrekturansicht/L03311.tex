%% latex-korrekturansicht-vorspann.tex
%% Vorspann für die Korrekturansicht.
%% Lädt die gemeinsame Datei latex-vorspann.tex mit gesetztem Schalter.

\newif\ifkorrekturansicht
\korrekturansichttrue

\input{../tex-inputs/latex-vorspann}


\renewcommand{\erwaehntePersonen}{Personen: Caroline Kotter, Ottmar Peter Kotter, Elisabeth Kotter, Leopoldine Müller, Louise Schnitzler}
\renewcommand{\erwaehnteOrte}{Orte: Bad Ischl, Meran, Schruns, Thusis, Traunkai}
\renewcommand{\erwaehnteWerke}{}
\section[ Felix Salten an Arthur Schnitzler, 18. 8. 1900]{Felix Salten an Arthur Schnitzler, 18. 8. 1900}
\nopagebreak\mylabel{v}
\rehead{ }\normalsize\beginnumbering\briefempfaengerindex{Schnitzler, Arthur@\textsc{Schnitzler, Arthur}!zzzSalten, Felix@\emph{von Felix Salten}!1900-08-181@{18. 8. 1900}|(be}
\toendnotes[C]{\smallbreak\pagebreak[2]}\Standort{CUL, Schnitzler, B 89, A 2.}
\physDesc{Brief, 1 Blatt, 2 Seiten, 744 Zeichen
\newline{}Handschrift: schwarze Tinte, lateinische Kurrent
\newline{}Ordnung: mit Bleistift von unbekannter Hand nummeriert: »135« }\toendnotes[C]{\smallbreak}
\pstart
           \raggedleft{}{\pb}\textcolor{pink}{Ischl, Traunquai 27}{}\ledrightnote{\textcolor{pink}{Traunkai}}, \uline{nicht} 11. {\\}18. VIII. 00.\pend
           
\pstart
           Lieber Freund, Ihre Frau \textcolor{blue}{Mama}{}\ledrightnote{{$\rightarrow$}\textcolor{blue}{Louise Schnitzler}} sagte mir heute, Sie hätten sie gefragt, wann ich nach \textcolor{pink}{Meran}{}\ledrightnote{\textcolor{pink}{Meran}} komme. Da ich daraus entnehme, dass Sie meinen Brief in
                  \label{K_L03311-1v}\edtext{\textcolor{pink}{Schruns}{}\ledrightnote{\textcolor{pink}{Schruns}}}{\lemma{\textnormal{\emph{Schruns}}}\Cendnote{\textnormal{siehe Felix Salten an Arthur Schnitzler, 5. 8. 1900}}}\label{K_L03311-1h} schon erhalten haben, bitte ich Sie nochmals um Nachricht, wann Sie in \textcolor{pink}{Meran}{}\ledrightnote{\textcolor{pink}{Meran}} sind\textcolor{gray}{:} Ich kann vom
                  24. an, (auch früher) jeden Tag. Ich bitte Sie, mir
               genau die Tour zu schreiben, die Sie vorschlagen, weil ich mir von hier aus die
               Eisenbahnkarte danach bestellen muß. Das dauert auch 3–4 Tage und je früher ich’s
               weiß, desto besser ist es. Wie geht es? Es thut mir leid, dass ich nicht mit
               konnte.\pend
           
\pstart
            Herzlichst Ihr {\\[\baselineskip]}\spacefill\mbox{Salten.}\pend
           \leftskip=0em{}
\pstart
           \noindent{}\label{K_L03311-2v}\edtext{\textcolor{blue}{Ellychen}{}\ledrightnote{\textcolor{blue}{Caroline Kotter}} und \textcolor{blue}{Peter}{}\ledrightnote{\textcolor{blue}{Ottmar Peter Kotter}}}{\lemma{\textnormal{\emph{Ellychen und Peter}}}\Cendnote{\textnormal{\textcolor{blue}{Caroline} und \textcolor{blue}{Ottmar Peter Kotter}, \textcolor{blue}{Salten}s Kinder mit \textcolor{blue}{Elisabeth
                        Kotter}}}}\label{K_L03311-2h} befinden sich wol, nur heißt \textcolor{blue}{Peter}{}\ledrightnote{\textcolor{blue}{Ottmar Peter Kotter}}
                  jetzt »Pumpi«.\pend
           
\pstart
           Richtig! vor einer {\pb}halben
                  Stunde hab ich Frl. \label{K_L03311-3v}\edtext{\textcolor{blue}{Poldi}{}\ledrightnote{\textcolor{blue}{Leopoldine Müller}}}{\lemma{\textnormal{\emph{Poldi}}}\Cendnote{\textnormal{\textcolor{blue}{Leopoldine Müller}, eine Geliebte \textcolor{blue}{Schnitzler}s}}}\label{K_L03311-3h} gesehen, sie sah
                  bildhübsch aus!\pend
           \endnumbering\briefempfaengerindex{Schnitzler, Arthur@\textsc{Schnitzler, Arthur}!zzzSalten, Felix@\emph{von Felix Salten}!1900-08-181@{18. 8. 1900}|)be}\mylabel{h}  \normalsize

\doendnotes{C}
\bigskip
\vfill

\clearpage

\footnotesize

\lohead{\textsc{register}}

% Definiere theindex-Environment komplett neu ohne reledmac
\makeatletter
\renewenvironment{theindex}{%
  \section*{\indexname}%
  \setlength{\parindent}{0pt}%
  \setlength{\parskip}{0pt plus 0.3pt}%
  \let\item\@idxitem
}{%
  \clearpage
}
\makeatother

\IfFileExists{\jobname-pw.ind}{\input{\jobname-pw.ind}}{}

\end{document}

      