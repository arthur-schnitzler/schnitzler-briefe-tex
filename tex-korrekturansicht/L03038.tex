%% latex-korrekturansicht-vorspann.tex
%% Vorspann für die Korrekturansicht.
%% Lädt die gemeinsame Datei latex-vorspann.tex mit gesetztem Schalter.

\newif\ifkorrekturansicht
\korrekturansichttrue

\input{../tex-inputs/latex-vorspann}


\renewcommand{\erwaehntePersonen}{Personen: Felix Salten, Olga Schnitzler}
\renewcommand{\erwaehnteInstitutionen}{Institutionen: Wiener Allgemeine Zeitung}
\renewcommand{\erwaehnteOrte}{Orte: Karlsbad, Kochgasse, Salzburg, Wien}
\renewcommand{\erwaehnteWerke}{Werke: Der einsame Weg. Schauspiel in fünf Akten, Die Frau mit dem Dolche, Lebendige Stunden, Tagebuch}
\section[ Arthur Schnitzler an Felix Salten, {[}10. 6. 1901?{]}]{Arthur Schnitzler an Felix Salten, {[}10. 6. 1901?{]}}
\nopagebreak\mylabel{v}
\rehead{ }\normalsize\beginnumbering\briefempfaengerindex{Salten, Felix@\textsc{Salten, Felix}!zzzSchnitzler, Arthur@\emph{von Arthur Schnitzler}!1901-06-101@{{[}10. 6. 1901?{]}}|(be}
\toendnotes[C]{\smallbreak\pagebreak[2]}\Standort{Wienbibliothek im Rathaus, ZPH 1681, 2.1.516.}
\physDesc{Brief, 1 Blatt, 3 Seiten, 454 Zeichen
\newline{}Handschrift: Bleistift, deutsche Kurrent
\newline{}Ordnung: mit Bleistift von unbekannter Hand Nummerierung der Blätter des Konvoluts:
                                    »20«–»21« }\toendnotes[C]{\smallbreak}
\pstart
           \raggedleft{}{\pb}\uline{Montag}\pend
           
\pstart
           lieber Freund, ich erfuhr, dſs Sie nicht in \textcolor{pink}{Karlsbad}{}\ledrightnote{\textcolor{pink}{Karlsbad}} ſondern \textcolor{pink}{hier}{}\ledrightnote{{$\rightarrow$}\textcolor{pink}{Wien}} ſind, ſuchte Sie Vormittg in
               Ihrer \textcolor{pink}{Wohnung}{}\ledrightnote{{$\rightarrow$}\textcolor{pink}{Kochgasse}} und der \textsc{\textcolor{brown}{Redaction}{}\ledrightnote{{$\rightarrow$}\textcolor{brown}{Wiener Allgemeine Zeitung}}}, um Ihnen Adieu zu ſagen\pend
           
\pstart
           {\pb}Ich \introOben{}(\textsc{resp}. \textcolor{blue}{wir}{}\ledrightnote{{$\rightarrow$}\textcolor{blue}{Olga Schnitzler}})\introOben{}{ }\label{K_L03038-1v}\edtext{fahre morgen}{\lemma{\textnormal{\emph{fahre morgen}}}\Cendnote{\textnormal{Die Datierung des Korrespondenzstücks
                  kann dadurch, mit Hilfe des \emph{\textcolor{green}{Tagebuch}}s und den
                  impliziten Hinweisen auf die bevorstehenden literarischen \textcolor{green}{Arbeiten}
                  erfolgen.}}}\label{K_L03038-1h} vorläufg nach \textcolor{pink}{Salzburg}{}\ledrightnote{\textcolor{pink}{Salzburg}}
               (wahrſcheinlich) alles \label{K_L03038-2v}\edtext{weitere}{\lemma{\textnormal{\emph{weitere}}}\Cendnote{\textnormal{zur Reise siehe das \emph{\textcolor{green}{Tagebuch}} bis zum 29. 8. 1901}}}\label{K_L03038-2h} iſt noch unbeſtimmt. Sagen Sie mir ein Wort von Ihren Plänen, Briefe werden
               mir nachgeſchickt.\pend
           
\pstart
           {\pb}Ein ſchönes \label{K_L03038-3v}\edtext{3aktiges modernes \textcolor{green}{Stück}{}\ledrightnote{{$\rightarrow$}\textcolor{green}{Der einsame Weg. Schauspiel in fünf Akten}}}{\lemma{\textnormal{\emph{3aktiges modernes Stück}}}\Cendnote{\textnormal{\emph{\textcolor{green}{Der einsame Weg}}, den \textcolor{blue}{Schnitzler} am 21. 7. 1901 vorläufig abschloss}}}\label{K_L03038-3h}, innerlich
               ganz fertig, hoff ich ſehr im Sommer zu vollenden, überdies \label{K_L03038-4v}\edtext{\textcolor{green}{2 Einakter}{}\ledrightnote{{$\rightarrow$}\textcolor{green}{Lebendige Stunden}{\newline}{$\rightarrow$}\textcolor{green}{Die Frau mit dem Dolche}}}{\lemma{\textnormal{\emph{2 Einakter}}}\Cendnote{\textnormal{\emph{\textcolor{green}{Lebendige Stunden}} (abgeschlossen am 28. 7. 1901) und \emph{\textcolor{green}{Die Frau mit dem Dolche}} (abgeschlossen am 3. 8. 1901)}}}\label{K_L03038-4h}.\pend
           
\pstart
           Herzlichſt Ihr {\\[\baselineskip]}\spacefill\mbox{ArthurSch}\pend
           \leftskip=0em{}\endnumbering\briefempfaengerindex{Salten, Felix@\textsc{Salten, Felix}!zzzSchnitzler, Arthur@\emph{von Arthur Schnitzler}!1901-06-101@{{[}10. 6. 1901?{]}}|)be}\mylabel{h}  \normalsize

\doendnotes{C}
\bigskip
\vfill

\clearpage

\footnotesize

\lohead{\textsc{register}}

% Definiere theindex-Environment komplett neu ohne reledmac
\makeatletter
\renewenvironment{theindex}{%
  \section*{\indexname}%
  \setlength{\parindent}{0pt}%
  \setlength{\parskip}{0pt plus 0.3pt}%
  \let\item\@idxitem
}{%
  \clearpage
}
\makeatother

\IfFileExists{\jobname-pw.ind}{\input{\jobname-pw.ind}}{}

\end{document}

      