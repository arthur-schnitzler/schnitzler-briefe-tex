%% latex-korrekturansicht-vorspann.tex
%% Vorspann für die Korrekturansicht.
%% Lädt die gemeinsame Datei latex-vorspann.tex mit gesetztem Schalter.

\newif\ifkorrekturansicht
\korrekturansichttrue

\input{../tex-inputs/latex-vorspann}


               \section[Hugo von Hofmannsthal an Arthur Schnitzler, 26. {[}3. 1911{]}]{ Hugo von Hofmannsthal an Arthur Schnitzler, 26. {[}3. 1911{]}}\nopagebreak\mylabel{v}\rehead{ }\normalsize\beginnumbering\briefempfaengerindex{Schnitzler, Arthur@\textsc{Schnitzler, Arthur}!zzzHofmannsthal, Hugo von@\emph{von Hugo von Hofmannsthal}!1911-03-261@{26. {[}3. 1911{]}}|(be} \toendnotes[C]{\smallbreak\pagebreak[2]} \Standort{CUL, Schnitzler, B 43.}
\physDesc{Briefkarte
\newline{}Handschrift: schwarze Tinte, deutsche Kurrent
\newline{}Schnitzler: mit Bleistift datiert: »3. 910« und beschriftet: »Hugo« \newline{}Ordnung: 1) mit Bleistift von unbekannter Hand nummeriert: »\strikeout{320}« 2) mit Bleistift von unbekannter Hand nummeriert:
                                    »315«}\buchAbdrucke{\weitereDrucke{Hugo von Hofmannsthal, Arthur Schnitzler: \emph{Briefwechsel}. Hg. Therese Nickl und Heinrich Schnitzler. Frankfurt am Main: \emph{S. Fischer} 1964, S. 261.} }\toendnotes[C]{\smallbreak}\pstart
           \raggedleft{}{\pb}Sonntag 26\textsuperscript{ten}\pend
           \pstart{}mein lieber Arthur, \pend\pstart
           ich habe alſo eine \label{K_L02014_1v}\edtext{Loge}{\lemma{\textnormal{\emph{Loge}}}\Cendnote{\textnormal{für die \textcolor{pink}{Wien}er Erstaufführung von \emph{\textcolor{green}{Der
                     Rosenkavalier}}.}}}\label{K_L02014_1h}{ }\textsc{parterre} oder I\textsuperscript{ter} Rang für Sie
               und \textcolor{blue}{Richard}{}\ledrightnote{\textcolor{blue}{Richard Beer-Hofmann}} beſtellt für den \textcolor{green}{8\textsuperscript{ten} April}{}\ledrightnote{→\textcolor{green}{Der Rosenkavalier}}. Nun höre ich auf einmal daſs die \textcolor{blue}{Bären}{}\ledrightnote{\textcolor{blue}{Richard Beer-Hofmann}{\newline}\textcolor{blue}{Paula Beer-Hofmann}} abreiſen. Ich hoffe aber trotzdem, {\pb}daſs es Ihnen vielleicht recht
               iſt, eine Loge zu haben und ſie mit \textcolor{gray}{Ih}rem \textcolor{blue}{Bruder}{}\ledrightnote{→\textcolor{blue}{Julius Schnitzler}} oder ſonſt jemand zu theilen; denn
               ſonſt müſste ich verſuchen, bei \textcolor{blue}{\textsc{Horsetzky}}{}\ledrightnote{\textcolor{blue}{Viktor von Horsetzky-Hornthal}} die für mich \strikeout{vorgeſetz} vorgemerkte Liſte zu
               ändern. Bitte um ein kleines Wort!\pend
           \pstart Herzlich Ihr\spacefill\mbox{Hugo.}\pend{}\endnumbering\briefempfaengerindex{Schnitzler, Arthur@\textsc{Schnitzler, Arthur}!zzzHofmannsthal, Hugo von@\emph{von Hugo von Hofmannsthal}!1911-03-261@{26. {[}3. 1911{]}}|)be}\mylabel{h}  \normalsize

\doendnotes{C}
\bigskip
\vfill

\clearpage

\footnotesize

\lohead{\textsc{register}}

% Definiere theindex-Environment komplett neu ohne reledmac
\makeatletter
\renewenvironment{theindex}{%
  \section*{\indexname}%
  \setlength{\parindent}{0pt}%
  \setlength{\parskip}{0pt plus 0.3pt}%
  \let\item\@idxitem
}{%
  \clearpage
}
\makeatother

\IfFileExists{\jobname-pw.ind}{\input{\jobname-pw.ind}}{}

\end{document}

      