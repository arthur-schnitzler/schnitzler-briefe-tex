%% latex-korrekturansicht-vorspann.tex
%% Vorspann für die Korrekturansicht.
%% Lädt die gemeinsame Datei latex-vorspann.tex mit gesetztem Schalter.

\newif\ifkorrekturansicht
\korrekturansichttrue

\input{../tex-inputs/latex-vorspann}


               \section[ Paul Goldmann an Arthur Schnitzler, 22. 4. 1897]{Paul Goldmann an Arthur Schnitzler, 22. 4. 1897}\nopagebreak\mylabel{v}\rehead{ }\normalsize\beginnumbering\briefempfaengerindex{Schnitzler, Arthur@\textsc{Schnitzler, Arthur}!zzzGoldmann, Paul@\emph{von Paul Goldmann}!1897-04-222@{22. 4. 1897}|(be} \toendnotes[C]{\smallbreak\pagebreak[2]} \Standort{DLA, A:Schnitzler, HS.NZ85.1.3167.}
\physDesc{Brief, 1 Blatt, 3 Seiten
\newline{}Handschrift: schwarze Tinte, deutsche Kurrent
\newline{}Schnitzler: mit rotem Buntstift zwei Unterstreichungen }\toendnotes[C]{\smallbreak}\pstart
           \noindent{}\centering{}{\pb}\textcolor{gray}{\textbf{\textcolor{pink}{Hotel Deutscher Kaiser}{}\ledrightnote{\textcolor{pink}{Hotel Deutscher Kaiser}}}}\pend
           \pstart
           \noindent{}\centering{}\textcolor{gray}{\textbf{(\textcolor{blue}{W. Gömöri}{}\ledrightnote{\textcolor{blue}{Wilhelm Gömöri}})}}\pend
           \pstart
           \noindent{}\centering{}\textcolor{gray}{\textbf{\textcolor{pink}{Frankfurt a. M.}{}\ledrightnote{\textcolor{pink}{Frankfurt am Main}}}}\pend
           \pstart
           \noindent{}\textcolor{gray}{\textbf{\textcolor{pink}{37 Wiesenhüttenplatz 37}{}\ledrightnote{\textcolor{pink}{Hotel Deutscher Kaiser}}.}}\hfill \textcolor{gray}{\textbf{Nahe dem \textcolor{pink}{Centralbahnhof}{}\ledrightnote{\textcolor{pink}{Frankfurt (Main) Hauptbahnhof}}.}}\pend
           \pstart
           \raggedleft{}\textcolor{gray}{\textbf{\textcolor{pink}{Frankfurt a. M.}{}\ledrightnote{\textcolor{pink}{Frankfurt am Main}}, den}}{ }22. April \textcolor{gray}{\textbf{18}}97.\pend
           \pstart{}Mein lieber Freund,\pend\pstart
           Vielen Dank für Deinen lieben Brief!\pend
           \pstart
           Ich bin ſeit Sonntag hier (nachdem ich Samſtag den Anſchluß verfehlt und in \textsc{\textcolor{pink}{Köln}{}\ledrightnote{\textcolor{pink}{Köln}}} hatte übernachten müſſen). Ich bin noch ganz krank hier angekommen und kann
               mich diesmal gar nicht erholen{[}.{]} Meine Familie iſt ſehr gut mit
               mir. Aber wir ſitzen zuſammen und denken über die ausſichtsloſe Zukunft nach, und das
               iſt nicht heiter. Auf der \textcolor{brown}{Redaction}{}\ledrightnote{→\textcolor{brown}{Frankfurter Zeitung}} machen ſie ſchiefe Geſichter, daß ich während des \label{K_L02809-1v}\edtext{Krieges}{\lemma{\textnormal{\emph{Krieges}}}\Cendnote{\textnormal{\textcolor{pink}{Türk}isch-\textcolor{pink}{Griech}ischer Krieg um \textcolor{pink}{Kreta}}}}\label{K_L02809-1h} nicht auf meinem Poſten bin. Ich werde alſo wohl bald zurück {\pb}müſſen. Aber jetzt im Ruhen ſehe ich erſt, wie
               abgehetzt und müde gearbeitet ich bin.\pend
           \pstart
           Alle die Meinigen grüßen Dich herzlichſt.\pend
           \pstart
           Wenn Du Zeit haſt, ſchreib’ mir noch ein Wort hierher, wie es Dir geht. (Meine
               Adreſſe iſt oben auf den Brief gedruckt).\pend
           \pstart
           Ich vergaß Dir zu ſagen, daß Du einen Abend (mit \textcolor{blue}{ihr}{}\ledrightnote{→\textcolor{blue}{Marie Reinhard}}) in die »\label{K_L02809-45v}\edtext{\textsc{\textcolor{pink}{Scala}{}\ledrightnote{\textcolor{pink}{La Scala}}}}{\lemma{\textnormal{\emph{Scala}}}\Cendnote{\textnormal{\textcolor{pink}{Konzertsaal}}}}\label{K_L02809-45h}« gehen ſollſt.\pend
           \pstart
           Geſtern ſah ich \textsc{\textcolor{green}{John Gabriel Borkmann}{}\ledrightnote{\textcolor{green}{John Gabriel Borkman}}}. \strikeout{E} Das \strikeout{D\textcolor{gray}{a}}{ }\textcolor{green}{Drama}{}\ledrightnote{→\textcolor{green}{John Gabriel Borkman}} hat zu Zeiten einen
               großartigen tragiſchen Schwung. Ich zähle es zum Beſten, was \strikeout{\textcolor{gray}{V}\textcolor{gray}{×}\textcolor{gray}{ß}}{ }\textsc{\textcolor{blue}{Ibsen}{}\ledrightnote{\textcolor{blue}{Henrik Ibsen}}} gemacht hat.\pend
           \pstart
           Mein \textcolor{blue}{Onkel}{}\ledrightnote{→\textcolor{blue}{Fedor Mamroth}} iſt voll des Lobes
               über \textsc{\textcolor{blue}{Bahr}{}\ledrightnote{\textcolor{blue}{Hermann Bahr}}s}{ }\textcolor{green}{Roman}{}\ledrightnote{→\textcolor{green}{Theater. Ein Wiener Roman}}{\pb} »\textcolor{green}{Theater}{}\ledrightnote{\textcolor{green}{Theater. Ein Wiener Roman}}«.
                  \label{K_L02809-89v}\edtext{Kennſt}{\lemma{\textnormal{\emph{Kennſt}}}\Cendnote{\textnormal{\textcolor{blue}{Schnitzler} erhielt von \textcolor{blue}{Bahr} ein \textcolor{green}{Widmungsexemplar} (vgl. Hermann Bahr: Widmungsexemplar Theater. Roman für Arthur Schnitzler, [nach dem
               20. 3. 1897]), las dieses jedoch frühestens im März 1905 (vgl. Bahr/Schnitzler, T030052).}}}\label{K_L02809-89h} Du das \textcolor{green}{Ding}{}\ledrightnote{→\textcolor{green}{Theater. Ein Wiener Roman}}? Es
               wäre ſchrecklich, wenn \strikeout{\textcolor{gray}{man}} dem \textcolor{blue}{Kerl}{}\ledrightnote{→\textcolor{blue}{Hermann Bahr}} wirklich
                  \strikeout{\textcolor{gray}{ei}} einmal etwas Gutes gelungen wäre.\pend
           \pstart
           Es freut mich, daß Du mir wegen \label{K_L02809-10v}\edtext{Freitag{ }Abend}{\lemma{\textnormal{\emph{Freitag Abend}}}\Cendnote{\textnormal{siehe Paul Goldmann an Arthur Schnitzler, 17. 4. [1897]}}}\label{K_L02809-10h} nicht böſe biſt. »\label{K_L02809-77v}\edtext{\textcolor{blue}{Sie}{}\ledrightnote{→\textcolor{blue}{?? [Partnerin? von Paul Goldmann 1897]}}}{\lemma{\textnormal{\emph{Sie}}}\Cendnote{\textnormal{nicht identifiziert; womöglich handelte
                  es sich jedoch um die am 12. 4. 1897 im \emph{\textcolor{green}{Tagebuch}} erwähnte
                        »›\textcolor{blue}{Fanny}‹« oder die am
                     13. 5. 1897
                  erwähnte »\textcolor{blue}{Madeleine}«}}}\label{K_L02809-77h}« hat mich nicht zurückgehalten, ganz im Gegentheil. Auch da gibts
               allerlei \begin{otherlanguage}{french}\textsc{Malheur}\end{otherlanguage}.\pend
           \pstart
           Kaufe Dir die ſoeben erſchienene \label{K_L02809-21v}\edtext{\textcolor{green}{\textsc{\textcolor{blue}{Beaumarchais}{}\ledrightnote{\textcolor{blue}{Pierre Augustin Caron de Beaumarchais}}}-Biographie}{}\ledrightnote{→\textcolor{green}{Beaumarchais}} von \textsc{\textcolor{blue}{André Hallays}{}\ledrightnote{\textcolor{blue}{André Hallays}}}}{\lemma{\textnormal{\emph{Beaumarchais-Biographie … Hallays}}}\Cendnote{\textnormal{\textcolor{green}{Lektüre} belegbar, vgl. A. S.: \emph{Lektüren}, Frankreich}}}\label{K_L02809-21h}. Ein reizendes \textcolor{green}{Buch}{}\ledrightnote{→\textcolor{green}{Beaumarchais}}.\pend
           \pstart
           Grüße mir Deine \textcolor{blue}{Freundin}{}\ledrightnote{→\textcolor{blue}{Marie Reinhard}} und
               ſei ſelbſt von Herzen gegrüßt\pend
           \pstart
           Dein treuer {\\[\baselineskip]}\spacefill\mbox{Paul Goldm}\pend
           \leftskip=0em{}\endnumbering\briefempfaengerindex{Schnitzler, Arthur@\textsc{Schnitzler, Arthur}!zzzGoldmann, Paul@\emph{von Paul Goldmann}!1897-04-222@{22. 4. 1897}|)be}\mylabel{h}\begin{anhang}\end{anhang}\normalsize

\doendnotes{C}
\bigskip
\vfill

\clearpage

\footnotesize

\lohead{\textsc{register}}

% Definiere theindex-Environment komplett neu ohne reledmac
\makeatletter
\renewenvironment{theindex}{%
  \section*{\indexname}%
  \setlength{\parindent}{0pt}%
  \setlength{\parskip}{0pt plus 0.3pt}%
  \let\item\@idxitem
}{%
  \clearpage
}
\makeatother

\IfFileExists{\jobname-pw.ind}{\input{\jobname-pw.ind}}{}

\end{document}

      