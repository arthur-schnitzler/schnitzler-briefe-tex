%% latex-korrekturansicht-vorspann.tex
%% Vorspann für die Korrekturansicht.
%% Lädt die gemeinsame Datei latex-vorspann.tex mit gesetztem Schalter.

\newif\ifkorrekturansicht
\korrekturansichttrue

\input{../tex-inputs/latex-vorspann}


               \section[Arthur Schnitzler an Gerhart Hauptmann, 18. 1. 1905]{ Arthur Schnitzler an Gerhart Hauptmann, 18. 1. 1905}\nopagebreak\mylabel{v}\rehead{ }\normalsize\beginnumbering\briefempfaengerindex{Hauptmann, Gerhart@\textsc{Hauptmann, Gerhart}!zzzSchnitzler, Arthur@\emph{von Arthur Schnitzler}!1905-01-181@{18. 1. 1905}|(be} \toendnotes[C]{\smallbreak\pagebreak[2]} \Standort{Staatsbibliothek Berlin – Preußischer Kulturbesitz, GHBrBl A:Schnitzler (10).}
\physDesc{Brief, 1 Blatt, 1 Seite
\newline{}Handschrift: schwarze Tinte, deutsche Kurrent}\toendnotes[C]{\smallbreak}\pstart
           \raggedleft{}{\pb}\textcolor{pink}{XVIII Spoettelgasse \label{T_L01491_1v}\edtext{7}{\lemma{\textnormal{\emph{7}}}\Cendnote{\textnormal{eine
                        undeutliche »7« durchgestrichen und daneben neuerlich
                        hingeschrieben}}}\label{T_L01491_1h}}{}\ledrightnote{\textcolor{pink}{Edmund-Weiß-Gasse}}{\\}\textcolor{pink}{Wien}{}\ledrightnote{\textcolor{pink}{Wien}}{ }18. 1. 905\pend
           \pstart{}Lieber Herr Hauptmann,\pend\pstart
           ich gratulire Ihnen herzlich zum nächſten \textcolor{brown}{Grillparzer-Preis}{}\ledrightnote{\textcolor{brown}{Franz-Grillparzer-Preis}}. Und zur \textcolor{green}{Elga}{}\ledrightnote{\textcolor{green}{Elga}} nicht
               minder, von der ich hoffe, daſs ſie ſo wie ſie iſt, auf die Bühne kommen möge.\pend
           \pstart
           Mit den ſchönſten Grüßen an Sie und Frau \textcolor{blue}{Grethe}{}\ledrightnote{\textcolor{blue}{Margarete Hauptmann}}{\\[\baselineskip]}Ihr{\\[\baselineskip]}Arthur Schnitzler\pend
           \leftskip=0em{}\endnumbering\briefempfaengerindex{Hauptmann, Gerhart@\textsc{Hauptmann, Gerhart}!zzzSchnitzler, Arthur@\emph{von Arthur Schnitzler}!1905-01-181@{18. 1. 1905}|)be}\mylabel{h}  \normalsize

\doendnotes{C}
\bigskip
\vfill

\clearpage

\footnotesize

\lohead{\textsc{register}}

% Definiere theindex-Environment komplett neu ohne reledmac
\makeatletter
\renewenvironment{theindex}{%
  \section*{\indexname}%
  \setlength{\parindent}{0pt}%
  \setlength{\parskip}{0pt plus 0.3pt}%
  \let\item\@idxitem
}{%
  \clearpage
}
\makeatother

\IfFileExists{\jobname-pw.ind}{\input{\jobname-pw.ind}}{}

\end{document}

      