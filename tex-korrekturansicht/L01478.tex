%% latex-korrekturansicht-vorspann.tex
%% Vorspann für die Korrekturansicht.
%% Lädt die gemeinsame Datei latex-vorspann.tex mit gesetztem Schalter.

\newif\ifkorrekturansicht
\korrekturansichttrue

\input{../tex-inputs/latex-vorspann}


               \section[Hermann Bahr an Arthur Schnitzler, 14. 12. 1904]{ Hermann Bahr an Arthur Schnitzler, 14. 12. 1904}\nopagebreak\mylabel{v}\rehead{ }\normalsize\beginnumbering\briefempfaengerindex{Schnitzler, Arthur@\textsc{Schnitzler, Arthur}!zzzBahr, Hermann@\emph{von Hermann Bahr}!1904-12-142@{14. 12. 1904}|(be} \toendnotes[C]{\smallbreak\pagebreak[2]} \Standort{CUL, Schnitzler, B 5b.}
\physDesc{Brief, 1 Blatt, 4 Seiten
\newline{}Handschrift: schwarze Tinte, deutsche Kurrent\newline{}Ordnung: mit Bleistift von unbekannter Hand nummeriert: »124« }\buchAbdrucke{\weitereDrucke{Hermann Bahr, Arthur Schnitzler: \emph{Briefwechsel, Aufzeichnungen, Dokumente (1891–1931)}. Hg. Kurt Ifkovits und Martin Anton Müller. Göttingen: \emph{Wallstein} 2018, S. 334–335.} }\toendnotes[C]{\smallbreak}\pstart
           \raggedleft{}{\pb}14. 12. 04{\\}Nachts\pend
           \pstart\center{}Lieber Arthur!\pend\pstart
           Ich hab Dich nach der \textcolor{green}{Symphonie}{}\ledrightnote{→\textcolor{green}{Symphonie Nr. 3 D-Moll}}
               heut überall geſucht, aber Du warſt wie in die Erde verſunken. So laß mich Dir
               ſchriftlich geſchwind (denn ich bin todtmüd vor Muſik, geſtern auch nach \label{K_L01478_1v}\edtext{\textcolor{green}{Walküre}{}\ledrightnote{\textcolor{green}{Die Walküre}}}{\lemma{\textnormal{\emph{Walküre}}}\Cendnote{\textnormal{am 3. 12. 1904 in der \textcolor{pink}{Hofoper}, mit \textcolor{blue}{Anna
                     von Mildenburg}}}}\label{K_L01478_1h}, die mich ſo wahnſinnig aufgeregt hat, daß ich heut
               erſt in der Früh gegen fünf einſchlafen konnte) herzlichſt für Deinen lieben Brief
               danken. Es iſt möglich, daß Du recht haſt (mit dem, was Du über Deine Intention
               ſagſt, haſt Du natürlich gewiß recht, fraglich bleibt nur, ob nicht bei der
               Ausführung, Dir ſelbſt unbewußt, etwas von einer Untergrundſtimmung in Dir, die ſich
               nach dem Philiſter ſehnt, eingefloſſen iſt), ich mußte mein Gefühl aber einmal
               ausſprechen, mit einiger Schärfe, die nicht Dir gilt, ſondern mir ſelbſt, einer
               inneren Schwäche in \textcolor{gray}{mir}{ }ſelbſt, {\pb}an der
               ich Jahre lang gelitten habe (Manches, was ich jetzt im »\textcolor{green}{Franzl}{}\ledrightnote{\textcolor{green}{Der Franzl. Fünf Bilder aus dem Leben eines guten Mannes}}« nicht mehr mag und dieſe blödſinnige letzte Scene des »\textcolor{green}{Apoſtels}{}\ledrightnote{\textcolor{green}{Der Apostel}}« iſt aus ihr) und von der ich mich nur
               durch eine erbitterte Anrufung meiner innerſten Inſtinkte \label{LL287-2v}frei gemacht habe – ganz frei freilich erſt, ſeit ich mit dem
                  Tode ſo vertraut bin, ſeit der Tod wirklich mein beſter Freund geworden
                  iſt\label{LL287-2h}, der einzige nemlich, den ich mir noch wirklich verdienen will,
                  \label{LL287-3v}aber über dies alles einmal mündlich in
                  einer guten Stunde, denn es iſt tiefer, als ſich ſo hinſchreiben läßt, viel
                     »\label{K_L01478_2v}\edtext{\textcolor{green}{tiefer als der Tag
                     gedacht}{}\ledrightnote{→\textcolor{green}{Symphonie Nr. 3 D-Moll}{\newline}→\textcolor{green}{Also sprach Zarathustra}}}{\lemma{\textnormal{\emph{tiefer … gedacht}}}\Cendnote{\textnormal{Zitat aus dem Lied »Vor
                     Sonnen-Aufgang« in \textcolor{blue}{Friedrich Nietzsche}: \emph{\textcolor{green}{Also sprach Zarathustra. Ein Buch für Alle und
                        Keinen}} (3. Band. Chemnitz: \emph{Schmeitzner}{ }1884), hier wohl nach der Vertonung durch \textcolor{blue}{Gustav Mahler} im 4. Satz der \emph{\textcolor{green}{3.
                        Sinfonie}}.}}}\label{K_L01478_2h}«, \textcolor{green}{Triſtan}{}\ledrightnote{→\textcolor{green}{Tristan und Isolde}}tief, wo Du es jetzt, im zweiten Akt, viel ſchöner finden {\pb}wirſt, als ichs jemals werd ausſprechen
                  können\label{LL287-3h}.\pend
           \pstart
           Sehr leid tut mir, daß ich Samſtag nicht zu Euch kommen kann, 1) weil ich \textcolor{blue}{Hugo}{}\ledrightnote{\textcolor{blue}{Hugo von Hofmannsthal}} verſprochen habe, nach \textcolor{pink}{Rodaun}{}\ledrightnote{\textcolor{pink}{Rodaun}} zu kommen und 2) weil ich auch dort abſagen muß, weil ich
               3) gerade jetzt, bei froheſter innerer \textcolor{green}{Geneſung}{}\ledrightnote{\textcolor{green}{Genesung. Roman}}
               (der Teufel ſoll den \label{K_L01478_3v}\edtext{\textcolor{blue}{Trebitſch}{}\ledrightnote{\textcolor{blue}{Siegfried Trebitsch}} holen, der die ſchönſten Worte ſo
                  beſchmutzt}{\lemma{\textnormal{\emph{Trebitſch … beſchmutzt}}}\Cendnote{\textnormal{vgl. Hermann Bahr an Arthur Schnitzler, 22. 2. 1903}}}\label{K_L01478_3h}, daß einem grauſt, ſie anzurühren), \label{LL287-1v}äußerlich in einem rechten Durcheinander lebe, \label{K_L01478_4v}\edtext{den ich nicht ändern}{\lemma{\textnormal{\emph{den ich nicht ändern}}}\Cendnote{\textnormal{Durcheinander: dialektal auch als Maskulinum.}}}\label{K_L01478_4h} kann und
                  nicht ändern möchte\label{LL287-1h}, kurz: ſo ſehr ich mich wirklich ſehne, wieder einmal
               ruhig bei Euch zu ſitzen, jetzt gerade gehts in den nächſten Tagen leider nicht.\pend
           \pstart
           {\pb}Herzlichſt danke ich auch für den Gruß Deiner
               lieben \textcolor{blue}{Frau}{}\ledrightnote{→\textcolor{blue}{Olga Schnitzler}} und erwiedere ihn
               herzlichſt.\pend
           \pstart
           Ich wünſche mir ſehr, daß ſichs ſo treffen möchte, daß wir doch zwei drei Tage in \textcolor{pink}{Lueg}{}\ledrightnote{\textcolor{pink}{Lueg am Wolfgangsee}}{ }\label{LL287-4v}beiſammen ſind\label{LL287-4h}.\pend
           \pstart
           Dein alter{\\[\baselineskip]}\spacefill\mbox{H.}\pend
           \leftskip=0em{}\endnumbering\briefempfaengerindex{Schnitzler, Arthur@\textsc{Schnitzler, Arthur}!zzzBahr, Hermann@\emph{von Hermann Bahr}!1904-12-142@{14. 12. 1904}|)be}\mylabel{h}  \normalsize

\doendnotes{C}
\bigskip
\vfill

\clearpage

\footnotesize

\lohead{\textsc{register}}

% Definiere theindex-Environment komplett neu ohne reledmac
\makeatletter
\renewenvironment{theindex}{%
  \section*{\indexname}%
  \setlength{\parindent}{0pt}%
  \setlength{\parskip}{0pt plus 0.3pt}%
  \let\item\@idxitem
}{%
  \clearpage
}
\makeatother

\IfFileExists{\jobname-pw.ind}{\input{\jobname-pw.ind}}{}

\end{document}

      