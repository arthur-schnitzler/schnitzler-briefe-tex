%% latex-korrekturansicht-vorspann.tex
%% Vorspann für die Korrekturansicht.
%% Lädt die gemeinsame Datei latex-vorspann.tex mit gesetztem Schalter.

\newif\ifkorrekturansicht
\korrekturansichttrue

\input{../tex-inputs/latex-vorspann}


               \section[Charlotte Ehrenstein an Arthur Schnitzler, {[}22. 1.? 1906{]}]{ Charlotte Ehrenstein an Arthur Schnitzler, {[}22. 1.? 1906{]}}\nopagebreak\mylabel{v}\rehead{ }\normalsize\beginnumbering\briefempfaengerindex{Schnitzler, Arthur@\textsc{Schnitzler, Arthur}!zzzEhrenstein, Charlotte@\emph{von Charlotte Ehrenstein}!1906-01-222@{{[}22. 1.? 1906{]}}|(be} \toendnotes[C]{\smallbreak\pagebreak[2]} \Standort{DLA, A:Schnitzler, HS.NZ85.1.2837,3.}
\physDesc{Brief, 1 Blatt, 3 Seiten
\newline{}Handschrift: schwarze Tinte, deutsche Kurrent}\toendnotes[C]{\smallbreak}\pstart
           \noindent{}{\pb}\textsc{Sr. Hochwohlgeb. Herrn Dr. Arthur Schnitzler}.\pend
           \pstart\center{}Sehr geehrter Herr Doctor!\pend\pstart
           Von Ihrer gütigen Erlaubnis Gebrauch machend, geſtatte ich mir über den Zuſtand
                    meines l. \textcolor{blue}{Albert}{}\ledrightnote{\textcolor{blue}{Albert Ehrenstein}} zu berichten. Am \label{K_L01576_1v}\edtext{Samstag}{\lemma{\textnormal{\emph{Samstag}}}\Cendnote{\textnormal{Obzwar undatiert, dürfte dieses
                        Korrespondenzstück durch die inhaltliche Übereinstimmung am selben Tag wie
                        das Schreiben von \textcolor{blue}{Adolf Treibl} vom Adolf Treibl an Arthur Schnitzler, [22.? 1. 1906] verfasst
                        sein.}}}\label{K_L01576_1h} war Dr. \textcolor{blue}{Kornfeld}{}\ledrightnote{\textcolor{blue}{Sigmund Kornfeld}} nochmals
                    hier und ſah, daſs \textcolor{blue}{Albert}{}\ledrightnote{\textcolor{blue}{Albert Ehrenstein}}{ }ſich ziemlich beruhigte, daher entſchloſs er
                    ſich ihn in häuſlicher Pflege zu laſſen, womit auch mein l. Patient ganz
                    einverſtanden iſt. Die Beſſerung macht nun, wie H. Dr. \textcolor{blue}{Kornfeld}{}\ledrightnote{\textcolor{blue}{Sigmund Kornfeld}}{ }ſagt, und auch ich bemerken kann,
                    befriedigende Fortſchritte {\pb}und ſind nun mein l. \textcolor{blue}{Mann}{}\ledrightnote{→\textcolor{blue}{Alexander Ehrenstein}} und ich auch beruhigter.\pend
           \pstart
           Und nun geſtatten Sie ſehr geehrter Herr Doctor mir für die vielen Beweiſe von
                    Hochherzigkeit, Güte u. Liebenswürdigkeit, welche Sie meinem l. \textcolor{blue}{Albert}{}\ledrightnote{\textcolor{blue}{Albert Ehrenstein}}, meinem l. \textcolor{blue}{Mann}{}\ledrightnote{→\textcolor{blue}{Alexander Ehrenstein}} u. mir erwieſen recht herzlichſt zu danken, u.
                    mir zu verzeihen, daſs ich durch dieſen traurigen Zwiſchenfall, dieſe ſo ſehr in
                    Anſpruch nahm.\pend
           \pstart
           {\pb}Nochmals Sie ſehr geehrter Herr Doctor unſerer ſteten Dankbarkeit
                    verſichernd, Ihre verehrte Frau \textcolor{blue}{Gemahlin}{}\ledrightnote{→\textcolor{blue}{Olga Schnitzler}} um Verzeihung und Nachſicht bittend bin ich Ihre Sie
                    verehrende\pend
           \pstart \spacefill\mbox{Charlotte Ehrenſtein}\pend{}\endnumbering\briefempfaengerindex{Schnitzler, Arthur@\textsc{Schnitzler, Arthur}!zzzEhrenstein, Charlotte@\emph{von Charlotte Ehrenstein}!1906-01-222@{{[}22. 1.? 1906{]}}|)be}\mylabel{h}  \normalsize

\doendnotes{C}
\bigskip
\vfill

\clearpage

\footnotesize

\lohead{\textsc{register}}

% Definiere theindex-Environment komplett neu ohne reledmac
\makeatletter
\renewenvironment{theindex}{%
  \section*{\indexname}%
  \setlength{\parindent}{0pt}%
  \setlength{\parskip}{0pt plus 0.3pt}%
  \let\item\@idxitem
}{%
  \clearpage
}
\makeatother

\IfFileExists{\jobname-pw.ind}{\input{\jobname-pw.ind}}{}

\end{document}

      