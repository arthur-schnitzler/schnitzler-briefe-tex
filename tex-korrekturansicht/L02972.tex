%% latex-korrekturansicht-vorspann.tex
%% Vorspann für die Korrekturansicht.
%% Lädt die gemeinsame Datei latex-vorspann.tex mit gesetztem Schalter.

\newif\ifkorrekturansicht
\korrekturansichttrue

\input{../tex-inputs/latex-vorspann}


\renewcommand{\erwaehntePersonen}{Personen: Felix Salten, Arthur Strasser, Viktor Oskar Tilgner}
\renewcommand{\erwaehnteOrte}{Orte: Wien}
\renewcommand{\erwaehnteWerke}{Werke: Das Mozartdenkmal, Das österreichische Antlitz. Essays, Künstlerhaus, Moderne Rundschau, Secession. (Arthur Strasser), Victor Tilgner †, Wiener Allgemeine Zeitung}
\section[ Arthur Schnitzler an Felix Salten, 25. 3. {[}1902{]}]{Arthur Schnitzler an Felix Salten, 25. 3. {[}1902{]}}
\nopagebreak\mylabel{v}
\rehead{ }\normalsize\beginnumbering\briefempfaengerindex{Salten, Felix@\textsc{Salten, Felix}!zzzSchnitzler, Arthur@\emph{von Arthur Schnitzler}!1902-03-251@{25. 3. {[}1902{]}}|(be}
\toendnotes[C]{\smallbreak\pagebreak[2]}\Standort{Wienbibliothek im Rathaus, ZPH 1681, 2.1.516.}
\physDesc{Brief, 1 Blatt, 3 Seiten, 801 Zeichen
\newline{}Handschrift: Bleistift, deutsche Kurrent
\newline{}Ordnung: mit Bleistift von unbekannter Hand Nummerierung der Doppelseiten des Konvoluts:
                                    »3«–»4« }\toendnotes[C]{\smallbreak}
\pstart
           \raggedleft{}{\pb}\label{K_L02972-1v}\edtext{Dinſtag 25. 3.}{\lemma{\textnormal{\emph{Dinſtag 25. 3.}}}\Cendnote{\textnormal{Die Datierung auf das Jahr 1902 ist möglich, da im in Frage kommenden Zeitraum nur in diesem Jahr der 25. 3. ein Dienstag war.}}}\label{K_L02972-1h}\pend
           
\pstart
           liebſter Freund, ich habe heut{ }Nachmittg einen Theil der \label{K_L02972-2v}\edtext{Aufſätze}{\lemma{\textnormal{\emph{Aufſätze}}}\Cendnote{\textnormal{\textcolor{blue}{Salten} plante eine Zusammenstellung seiner
                  kritischen Zeitungsarbeiten zu publizieren, vgl. A. S.: \emph{Tagebuch}, 30. 3. 1902. Dazu kam es jedoch nicht. Inwiefern
               das Jahre später verfolgte Projekt »Aus einem Wiener Kreis« (vgl. A. S.: \emph{Tagebuch}, 18. 2. 1909) darauf Bezug
               nimmt, und wie sich dieses wiederum zur im selben Jahr veröffentlichten Textsammlung \emph{\textcolor{green}{Das österreichische Antlitz}} verhält,
               lässt sich nicht bestimmen.}}}\label{K_L02972-2h} geleſen,
               darunter die zwei großen, Sie wiſſen, wie beträchtlich meine Schätzung \introOben{}ſchon\introOben{} bisher geweſen iſt, aber ich ka{\geminationn} Sie verſichern, die Sachen ſtehn auf einem noch höhern
               Niveau als wir geglaubt haben. Nebenbei {\pb}–
               das wird hoffentlich dem äußern Erfolg zu ſtatten kommen, – ſchreiben Sie ſo
               (entſchuldigen Sie das Wort) amuſant, dſs mir beinah die Phraſe vom »Nicht aus der
               Hand legen können« in die Feder gekommen wäre. –\pend
           
\pstart
           Die Aufſätze über \textsc{\label{K_L02972-3v}\edtext{\textcolor{blue}{Strasser}{}\ledrightnote{\textcolor{blue}{Arthur Strasser}}}{\lemma{\textnormal{\emph{Strasser}}}\Cendnote{\textnormal{Gemeint sein dürfte: \textcolor{blue}{Felix Salten}: \emph{\textcolor{green}{Secession. (Arthur Strasser)}}. In: \emph{\textcolor{green}{Wiener Allgemeine Zeitung}}, Nr. 6.313, 18. 3. 1899, S. 2–3.}}}\label{K_L02972-3h}} u \textsc{\label{K_L02972-4v}\edtext{\textcolor{blue}{Tilgner}{}\ledrightnote{\textcolor{blue}{Viktor Oskar Tilgner}}}{\lemma{\textnormal{\emph{Tilgner}}}\Cendnote{\textnormal{\textcolor{blue}{Salten} hatte mehrfach über den Bildhauer
                        \textcolor{blue}{Viktor Tilgner} geschrieben, darunter: \textcolor{blue}{Felix Salten}: \emph{\textcolor{green}{Das Mozartdenkmal}}. In: \emph{\textcolor{green}{Moderne Rundschau}}, Jg. 1, Bd. 3, H. 1, 1. 4. 1891, S. 35–36; \textcolor{blue}{f. s.}: \emph{\textcolor{green}{Victor Tilgner †}}. In: \emph{\textcolor{green}{Wiener Allgemeine Zeitung}}, Nr. 5.441, 17. 4. 1896, S. 3; \textcolor{blue}{ders.}: \emph{\textcolor{green}{Künstlerhaus}}. In: \textcolor{green}{ebd.}, Nr. 5.612, 11. 11. 1896, S. 3.}}}\label{K_L02972-4h}} heben Sie vielleicht {\pb}beſſer für eine
               ſpätere Sa{\geminationm}lung auf, um das »moderne \uline{Theater}« nicht zu ſtören? –\pend
           
\pstart
           Zu überlegen, ob die Aufſätze über Literatur 48–98 und ü Theater 48–98 nicht bis auf
               den heutigen Tag fortzuſetzen wären. (Event. als Anhang?)\pend
           
\pstart
           Auf Wiederſehen. Herzlichſt {\\[\baselineskip]}Ihr \spacefill\mbox{A.}\pend
           \leftskip=0em{}\endnumbering\briefempfaengerindex{Salten, Felix@\textsc{Salten, Felix}!zzzSchnitzler, Arthur@\emph{von Arthur Schnitzler}!1902-03-251@{25. 3. {[}1902{]}}|)be}\mylabel{h}  \normalsize

\doendnotes{C}
\bigskip
\vfill

\clearpage

\footnotesize

\lohead{\textsc{register}}

% Definiere theindex-Environment komplett neu ohne reledmac
\makeatletter
\renewenvironment{theindex}{%
  \section*{\indexname}%
  \setlength{\parindent}{0pt}%
  \setlength{\parskip}{0pt plus 0.3pt}%
  \let\item\@idxitem
}{%
  \clearpage
}
\makeatother

\IfFileExists{\jobname-pw.ind}{\input{\jobname-pw.ind}}{}

\end{document}

      