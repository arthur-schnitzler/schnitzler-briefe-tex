%% latex-korrekturansicht-vorspann.tex
%% Vorspann für die Korrekturansicht.
%% Lädt die gemeinsame Datei latex-vorspann.tex mit gesetztem Schalter.

\newif\ifkorrekturansicht
\korrekturansichttrue

\input{../tex-inputs/latex-vorspann}


\renewcommand{\erwaehntePersonen}{Personen: Felix Salten, Arthur Strasser, Viktor Oskar Tilgner}
\renewcommand{\erwaehnteOrte}{Orte: Wien}
\renewcommand{\erwaehnteWerke}{Werke: Das Mozartdenkmal, Künstlerhaus, Moderne Rundschau, Secession. (Arthur Strasser), Victor Tilgner †, Wiener Allgemeine Zeitung}
\section[Arthur Schnitzler an Felix Salten, 25. 3. {[}1902{]}]{Arthur Schnitzler an Felix Salten, 25. 3. {[}1902{]}}
\nopagebreak\mylabel{v}
\rehead{ }\normalsize\beginnumbering\briefempfaengerindex{Salten, Felix@\textsc{Salten, Felix}!zzzSchnitzler, Arthur@\emph{von Arthur Schnitzler}!1902-03-251@{25. 3. {[}1902{]}}|(be}
\toendnotes[C]{\smallbreak\pagebreak[2]}\Standort{Wienbibliothek im Rathaus, ZPH 1681, 2.1.516.}
\physDesc{Brief, 1 Blatt, 3 Seiten
\newline{}Handschrift: Bleistift, deutsche Kurrent}\toendnotes[C]{\smallbreak}
\pstart
           \raggedleft{}{\pb}Dinſtag 25. 3.\pend
           
\pstart
           liebſter Freund, ich habe heut Nachmittg einen Theil der \label{K_L02972-4v}\edtext{Aufſätze}{\lemma{\textnormal{\emph{Aufſätze}}}\Cendnote{\textnormal{\textcolor{blue}{Salten} plante eine
                  Zusammenstellung seiner kritischen Zeitungsarbeiten zu publizieren, siehe A. S.: \emph{Tagebuch}, 30. 3. 1902. Dazu kam es
                  nicht.}}}\label{K_L02972-4h} geleſen, darunter die zwei großen, Sie wiſſen wie \introOben{}ſchon\introOben{} beträchtlich meine Schätzung bisher geweſen iſt, aber ich ka{\geminationn} Sie verſichern, die Sachen ſtehen auf einem noch
               höhern Niveau als wir geglaubt haben. Nebenbei {\pb}– das wird hoffentlich dem äußern Erfolg
               zuſtatten kommen, – ſchreiben Sie ſo (entſchuldigen Sie das Wort) amuſant, dſs mir
               beinah die Phraſe vom »Nicht aus der Hand legen können« in die Feder gekommen wäre.– \pend
           
\pstart
           Die Aufſätze über \textsc{\label{K_L02972-444v}\edtext{\textcolor{blue}{Strasser}{}\ledrightnote{\textcolor{blue}{Arthur Strasser}}}{\lemma{\textnormal{\emph{Strasser}}}\Cendnote{\textnormal{Gemeint sein dürfte: \textcolor{blue}{Felix Salten}:
                           \emph{\textcolor{green}{Secession. (Arthur Strasser)}}. In:
                           \emph{\textcolor{green}{Wiener Allgemeine Zeitung}},
                        Nr. 6.313, 18. 3. 1899,
                     S. 2–3.}}}\label{K_L02972-444h}} u \textsc{\label{K_L02972-12v}\edtext{\textcolor{blue}{Tilgner}{}\ledrightnote{\textcolor{blue}{Viktor Oskar Tilgner}}}{\lemma{\textnormal{\emph{Tilgner}}}\Cendnote{\textnormal{\textcolor{blue}{Salten} hat mehrfach über den Bildhauer \textcolor{blue}{Viktor Tilgner} geschrieben, darunter: \textcolor{blue}{Felix Salten}: \emph{\textcolor{green}{Das Mozartdenkmal}}. In: \emph{\textcolor{green}{Moderne Rundschau}}, Jg. 1, Bd. 3, H. 1,
                           1. 4. 1891, S. 35–36; \textcolor{blue}{f. s.}: \emph{\textcolor{green}{Victor
                           Tilgner †}}. In: \emph{\textcolor{green}{Wiener Allgemeine
                           Zeitung}}, Nr. 5.441, 17. 4. 1896,
                        S. 3; \textcolor{blue}{f. s.}:
                           \emph{\textcolor{green}{Künstlerhaus}}. In: \emph{\textcolor{green}{Wiener Allgemeine Zeitung}}, Nr. 5.612,
                           11. 11. 1896, S. 3. }}}\label{K_L02972-12h}} heben Sie vielleicht {\pb}beſſer für eine
               ſpätere Sa{\geminationm}lung auf, um das »moderne \uline{Theater}« nicht zu ſtören?– \pend
           
\pstart
           Zu überlegen, ob die Aufſätze über Literatur 48–98 und ü Theater 48–98 nicht bis auf
               den heutigen Tag fortzuſetzen wären. (Event. als Anfang?) \pend
           
\pstart
           Auf Wiederſehen. Herzlichſt {\\[\baselineskip]}Ihr \spacefill\mbox{A.}\pend
           \leftskip=0em{}\endnumbering\briefempfaengerindex{Salten, Felix@\textsc{Salten, Felix}!zzzSchnitzler, Arthur@\emph{von Arthur Schnitzler}!1902-03-251@{25. 3. {[}1902{]}}|)be}\mylabel{h}
\begin{anhang}
\end{anhang}\normalsize

\doendnotes{C}
\bigskip
\vfill

\clearpage

\footnotesize

\lohead{\textsc{register}}

% Definiere theindex-Environment komplett neu ohne reledmac
\makeatletter
\renewenvironment{theindex}{%
  \section*{\indexname}%
  \setlength{\parindent}{0pt}%
  \setlength{\parskip}{0pt plus 0.3pt}%
  \let\item\@idxitem
}{%
  \clearpage
}
\makeatother

\IfFileExists{\jobname-pw.ind}{\input{\jobname-pw.ind}}{}

\end{document}

      