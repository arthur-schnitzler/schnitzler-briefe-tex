%% latex-korrekturansicht-vorspann.tex
%% Vorspann für die Korrekturansicht.
%% Lädt die gemeinsame Datei latex-vorspann.tex mit gesetztem Schalter.

\newif\ifkorrekturansicht
\korrekturansichttrue

\input{../tex-inputs/latex-vorspann}


\renewcommand{\erwaehntePersonen}{Personen: Felix Salten, Ottilie Salten, Olga Schnitzler}
\renewcommand{\erwaehnteOrte}{Orte: Bad Ischl, Bansin, Deutschland, Kopenhagen, Lueg am Wolfgangsee, Marienlyst, München, Salzburg, St. Gilgen, Weimar, Wien}
\renewcommand{\erwaehnteWerke}{}
\section[ Felix Salten an Arthur Schnitzler, 17. 7. 1906]{Felix Salten an Arthur Schnitzler, 17. 7. 1906}
\nopagebreak\mylabel{v}
\rehead{ }\normalsize\beginnumbering\briefempfaengerindex{Schnitzler, Arthur@\textsc{Schnitzler, Arthur}!zzzSalten, Felix@\emph{von Felix Salten}!1906-07-172@{17. 7. 1906}|(be}
\toendnotes[C]{\smallbreak\pagebreak[2]}\Standort{CUL, Schnitzler, B 89, B 1.}
\physDesc{Brief, 1 Blatt, 1 Seite, 820 Zeichen
\newline{}Handschrift: schwarze Tinte, lateinische Kurrent
\newline{}Ordnung: mit Bleistift von unbekannter Hand nummeriert: »222« }\toendnotes[C]{\smallbreak}
\pstart
           \raggedleft{}{\pb}\textcolor{pink}{Bansin}{}\ledrightnote{\textcolor{pink}{Bansin}}, 17. 7. 06.\pend
           
\pstart
           Lieber, wir wollen schon bald – vielleicht schon diesen Freitag – nach \textcolor{pink}{Kopenhagen}{}\ledrightnote{\textcolor{pink}{Kopenhagen}} fahren, und dann \label{K_L03431-1v}\edtext{zu Ihnen nach \textcolor{pink}{Marienlyst}{}\ledrightnote{\textcolor{pink}{Marienlyst}} kommen}{\lemma{\textnormal{\emph{zu … kommen}}}\Cendnote{\textnormal{siehe A. S.: \emph{Tagebuch}, 2. 8. 1906}}}\label{K_L03431-1h}. Aber wol nicht länger als auf einen oder zwei Tage. Denn bis die Millionen,
               deren freilich nur Sie allein so sicher gewärtig sind, bis also die Millionen kommen,
               muß ich mich noch mit Kleinigkeiten abgeben und Verhandlungen führen, kann also nicht
               so lange fortbleiben. Ferner ist das Programm, dass ich nach \textcolor{pink}{Wien}{}\ledrightnote{\textcolor{pink}{Wien}} gehe. Von dort eventuell über \textcolor{pink}{Ischl}{}\ledrightnote{\textcolor{pink}{Bad Ischl}}, \textcolor{pink}{Lueg}{}\ledrightnote{\textcolor{pink}{Lueg am Wolfgangsee}}, \textcolor{pink}{Gilgen}{}\ledrightnote{\textcolor{pink}{St. Gilgen}}{ }\textcolor{pink}{Salzburg}{}\ledrightnote{\textcolor{pink}{Salzburg}}{ }\textcolor{pink}{München}{}\ledrightnote{\textcolor{pink}{München}}{ }\textcolor{pink}{hierher}{}\ledrightnote{{$\rightarrow$}\textcolor{pink}{Bansin}} zurück. Und endlich
               ist es meine Absicht, nach \textcolor{pink}{Weimar}{}\ledrightnote{\textcolor{pink}{Weimar}} zu gehen, weil
               ich es \textcolor{blue}{Otti}{}\ledrightnote{\textcolor{blue}{Ottilie Salten}} unbedingt zeigen möchte, ehe wir
               das \textcolor{pink}{Deutsche Reich}{}\ledrightnote{\textcolor{pink}{Deutschland}} verlaßen. Wenn wir uns also
               nach \textcolor{pink}{Kopenhagen}{}\ledrightnote{\textcolor{pink}{Kopenhagen}} in Bewegung setzen, zeige ich
               es Ihnen telegrafisch an. Inzwischen viele herzliche Grüße von \textcolor{blue}{Otti}{}\ledrightnote{\textcolor{blue}{Ottilie Salten}} und mir an Sie \textcolor{blue}{Beide}{}\ledrightnote{{$\rightarrow$}\textcolor{blue}{Olga Schnitzler}}.\pend
           \pstart Ihr \spacefill\mbox{FSalten}\pend{}\endnumbering\briefempfaengerindex{Schnitzler, Arthur@\textsc{Schnitzler, Arthur}!zzzSalten, Felix@\emph{von Felix Salten}!1906-07-172@{17. 7. 1906}|)be}\mylabel{h}  \normalsize

\doendnotes{C}
\bigskip
\vfill

\clearpage

\footnotesize

\lohead{\textsc{register}}

% Definiere theindex-Environment komplett neu ohne reledmac
\makeatletter
\renewenvironment{theindex}{%
  \section*{\indexname}%
  \setlength{\parindent}{0pt}%
  \setlength{\parskip}{0pt plus 0.3pt}%
  \let\item\@idxitem
}{%
  \clearpage
}
\makeatother

\IfFileExists{\jobname-pw.ind}{\input{\jobname-pw.ind}}{}

\end{document}

      