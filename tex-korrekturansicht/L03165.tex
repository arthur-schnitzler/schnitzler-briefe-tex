%% latex-korrekturansicht-vorspann.tex
%% Vorspann für die Korrekturansicht.
%% Lädt die gemeinsame Datei latex-vorspann.tex mit gesetztem Schalter.

\newif\ifkorrekturansicht
\korrekturansichttrue

\input{../tex-inputs/latex-vorspann}


\renewcommand{\erwaehnteOrte}{Orte: Wien}
\renewcommand{\erwaehnteWerke}{Werke: Liebelei. Schauspiel in drei Akten}
\section[ Felix Salten an Arthur Schnitzler, {[}7. 9. 1895{]}]{Felix Salten an Arthur Schnitzler, {[}7. 9. 1895{]}}
\nopagebreak\mylabel{v}
\rehead{ }\normalsize\beginnumbering\briefempfaengerindex{Schnitzler, Arthur@\textsc{Schnitzler, Arthur}!zzzSalten, Felix@\emph{von Felix Salten}!1895-09-071@{{[}7. 9. 1895{]}}|(be}
\toendnotes[C]{\smallbreak\pagebreak[2]}\Standort{CUL, Schnitzler, B 89, A 1.}
\physDesc{Brief, 1 Blatt, 1 Seite, 90 Zeichen (als Briefpapier wurde das Fragment eines Probenplans
                                 verwendet, der vorgedruckte Text mit Bleistift markiert)
\newline{}Handschrift: Bleistift, lateinische Kurrent
\newline{}Schnitzler: mit Bleistift datiert: »7/9 95« 
\newline{}Ordnung: mit Bleistift von unbekannter Hand nummeriert: »65« }\toendnotes[C]{\smallbreak}
\pstart
           \noindent{}\centering{}{\pb}\textcolor{gray}{\textbf{Reſerven: Alle unlängſt oder oft gegebenen, daher als
                        feſtſtehend zu erachtenden Vorſtellungen.}}\pend
           \settowidth{\longeste}{Liebeleimit einem Pfeil markiert}\settowidth{\longestz}{Neu einſtudirt und in Scene geſetzt:}\settowidth{\longestd}{Repriſen:}\settowidth{\longestv}{}\settowidth{\longestf}{}\addtolength\longeste{1em}
        \addtolength\longestz{1em}
        \addtolength\longestd{1em}
      \pstart\noindent\makebox[\the\longeste][l]{\textcolor{gray}{\textbf{Neu:}}}\makebox[\the\longestz][l]{\textcolor{gray}{\textbf{Neu einſtudirt und in Scene geſetzt:}}}
                  \makebox[\the\longestd][l]{\textcolor{gray}{\textbf{Repriſen:}}}\pend\pstart\noindent\makebox[\the\longeste][l]{\textcolor{gray}{\textbf{\textcolor{green}{\label{K_L03165-1v}\edtext{Liebelei}{\lemma{\textnormal{\emph{Liebelei}}}\Cendnote{\textnormal{mit einem Pfeil
                              markiert}}}\label{K_L03165-1h}}{}\ledrightnote{\textcolor{green}{Liebelei. Schauspiel in drei Akten}}}}}\makebox[\the\longestz][l]{}
                  \makebox[\the\longestd][l]{}\pend
\pstart
           lieber Arthur! Wenn Sie schon \label{K_L03165-2v}\edtext{\textcolor{pink}{hier}{}\ledrightnote{{$\rightarrow$}\textcolor{pink}{Wien}}}{\lemma{\textnormal{\emph{hier}}}\Cendnote{\textnormal{\textcolor{blue}{Schnitzler} kehrte an diesem Tag nach \textcolor{pink}{Wien} zurück. Nachweislich sah er \textcolor{blue}{Salten} erst am 12. 9. 1895
                  wieder.}}}\label{K_L03165-2h} sind, laßen Sie michs für Nachmittag wissen\pend
           
\pstart
           Herzl. Ihr {\\[\baselineskip]}\spacefill\mbox{Salten.}\pend
           \leftskip=0em{}\endnumbering\briefempfaengerindex{Schnitzler, Arthur@\textsc{Schnitzler, Arthur}!zzzSalten, Felix@\emph{von Felix Salten}!1895-09-071@{{[}7. 9. 1895{]}}|)be}\mylabel{h}  \normalsize

\doendnotes{C}
\bigskip
\vfill

\clearpage

\footnotesize

\lohead{\textsc{register}}

% Definiere theindex-Environment komplett neu ohne reledmac
\makeatletter
\renewenvironment{theindex}{%
  \section*{\indexname}%
  \setlength{\parindent}{0pt}%
  \setlength{\parskip}{0pt plus 0.3pt}%
  \let\item\@idxitem
}{%
  \clearpage
}
\makeatother

\IfFileExists{\jobname-pw.ind}{\input{\jobname-pw.ind}}{}

\end{document}

      