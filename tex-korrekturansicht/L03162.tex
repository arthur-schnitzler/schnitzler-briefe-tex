%% latex-korrekturansicht-vorspann.tex
%% Vorspann für die Korrekturansicht.
%% Lädt die gemeinsame Datei latex-vorspann.tex mit gesetztem Schalter.

\newif\ifkorrekturansicht
\korrekturansichttrue

\input{../tex-inputs/latex-vorspann}


\renewcommand{\erwaehnteInstitutionen}{Institutionen: Wiener Allgemeine Zeitung}
\renewcommand{\erwaehnteOrte}{Orte: Bad Ischl, Hörlgasse, Universitätsstraße, Wien, Wollzeile}
\renewcommand{\erwaehnteWerke}{}
\section[ Felix Salten an Arthur Schnitzler, {[}10.? 8. 1895{]}]{Felix Salten an Arthur Schnitzler, {[}10.? 8. 1895{]}}
\nopagebreak\mylabel{v}
\rehead{ }\normalsize\beginnumbering\briefempfaengerindex{Schnitzler, Arthur@\textsc{Schnitzler, Arthur}!zzzSalten, Felix@\emph{von Felix Salten}!1895-08-101@{{[}10.? 8. 1895{]}}|(be}
\toendnotes[C]{\smallbreak\pagebreak[2]}\Standort{CUL, Schnitzler, B 89, A 1.}
\physDesc{Brief, 1 Blatt, 1 Seite, 236 Zeichen
\newline{}Handschrift: Bleistift, lateinische Kurrent
\newline{}Schnitzler: mit Bleistift datiert: »10? 8/95« 
\newline{}Ordnung: mit Bleistift von unbekannter Hand nummeriert: »62« }\toendnotes[C]{\smallbreak}
\pstart
           \noindent{}{\pb}\textcolor{gray}{\textbf{\textbf{»\textcolor{brown}{Wiener Allgemeine
                        Zeitung}{}\ledrightnote{\textcolor{brown}{Wiener Allgemeine Zeitung}}«}}}\pend
           
\pstart
           \textcolor{gray}{\textbf{Redaction:}}\pend
           
\pstart
           \textcolor{gray}{\textbf{\textbf{\textcolor{pink}{IX/3, Univerſitätsſtraße Nr. 6}{}\ledrightnote{\textcolor{pink}{Universitätsstraße}}.}}}\pend
           
\pstart
           \textcolor{gray}{\textbf{Adminiſtration:}}\hfill \textcolor{gray}{\textbf{\textcolor{pink}{Wien}{}\ledrightnote{\textcolor{pink}{Wien}}, am ..........{ }189{\dots}}}\pend
           
\pstart
           \textcolor{gray}{\textbf{\textbf{\textcolor{pink}{I. Wollzeile Nr. 5}{}\ledrightnote{\textcolor{pink}{Wollzeile}}} (im Durchhauſe).}}\pend
           
\pstart
           \textcolor{gray}{\textbf{Telegramm-Adreſſe: »Allgemeine, \textcolor{pink}{Wien}{}\ledrightnote{\textcolor{pink}{Wien}}«.}}\pend
           
\pstart
           \textcolor{gray}{\textbf{Telephon der Redaction: Nr. 805 u. 2180.}}\pend
           
\pstart
           \textcolor{gray}{\textbf{\hspace*{1.5em}„\hspace*{1.5em}„\hspace*{1.5em} Adminiſtration: Nr. 1024.}}\pend
           
\pstart
           Lieber Arthur! Ich denke, es ist nicht nötig \label{K_L03162-1v}\edtext{morgen{ }Nachmittag}{\lemma{\textnormal{\emph{morgen Nachmittag}}}\Cendnote{\textnormal{Die mit Fragezeichen versehene Datierung \textcolor{blue}{Schnitzler}s stimmt damit überein, 
                  dass \textcolor{blue}{Schnitzler} und \textcolor{blue}{Salten} sich unmittelbar am Tag nach \textcolor{blue}{Schnitzler}s Rückkehr aus \textcolor{pink}{Ischl} am 11. 8. 1895 trafen.}}}\label{K_L03162-1h} in das heisse Caféhaus zu gehen. Am besten kommen Sie
               vielleicht \substVorne{}\textsuperscript{\textcolor{gray}{zu}}\substDazwischen{}gl\substHinten{}eich zu mir. Ich bin den ganzen Nachmittag von 2\textsuperscript{h} an zu \textcolor{pink}{Hause}{}\ledrightnote{{$\rightarrow$}\textcolor{pink}{Hörlgasse}}, bis
                  6 Uhr. Übrigens auch Vor{\pb}mittag.\pend
           
\pstart
           Herzlich {\\[\baselineskip]}Ihr {\\[\baselineskip]}\spacefill\mbox{Salten}\pend
           \leftskip=0em{}\endnumbering\briefempfaengerindex{Schnitzler, Arthur@\textsc{Schnitzler, Arthur}!zzzSalten, Felix@\emph{von Felix Salten}!1895-08-101@{{[}10.? 8. 1895{]}}|)be}\mylabel{h}  \normalsize

\doendnotes{C}
\bigskip
\vfill

\clearpage

\footnotesize

\lohead{\textsc{register}}

% Definiere theindex-Environment komplett neu ohne reledmac
\makeatletter
\renewenvironment{theindex}{%
  \section*{\indexname}%
  \setlength{\parindent}{0pt}%
  \setlength{\parskip}{0pt plus 0.3pt}%
  \let\item\@idxitem
}{%
  \clearpage
}
\makeatother

\IfFileExists{\jobname-pw.ind}{\input{\jobname-pw.ind}}{}

\end{document}

      