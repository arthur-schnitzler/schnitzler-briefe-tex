%% latex-korrekturansicht-vorspann.tex
%% Vorspann für die Korrekturansicht.
%% Lädt die gemeinsame Datei latex-vorspann.tex mit gesetztem Schalter.

\newif\ifkorrekturansicht
\korrekturansichttrue

\input{../tex-inputs/latex-vorspann}


               \section[Richard Beer-Hofmann an Arthur Schnitzler, {[}zwischen 1. und 12. 9.? 1908{]}]{ Richard Beer-Hofmann an Arthur Schnitzler, {[}zwischen 1. und
               12. 9.? 1908{]}}\nopagebreak\mylabel{v}\rehead{ }\normalsize\beginnumbering\briefempfaengerindex{Schnitzler, Arthur@\textsc{Schnitzler, Arthur}!zzzBeer-Hofmann, Richard@\emph{von Richard Beer-Hofmann}!1908-09-011@{{[}zwischen 1. und
                  12. 9.? 1908{]}}|(be} \toendnotes[C]{\smallbreak\pagebreak[2]} \Standort{CUL, Schnitzler, B 8.}
\physDesc{Bildpostkarte
\newline{}Handschrift: schwarze Tinte, lateinische Kurrent\newline{}Versand: nachgesandt nach »\textsc{\textcolor{pink}{Spöttlgasse 7 Wien XVIII}}« 
\newline{}Schnitzler: mit Bleistift datiert: »Juli (?) 08« und beschriftet: »\textsc{Beerh}« \newline{}Ordnung: 1) mit Bleistift von unbekannter Hand nummeriert: »\strikeout{216}« 2) mit Bleistift von unbekannter Hand nummeriert:
                                    »217«}\buchAbdrucke{\weitereDrucke{Arthur Schnitzler, Richard Beer-Hofmann: \emph{Briefwechsel 1891–1931}. Hg. Konstanze Fliedl. Wien, Zürich: \emph{Europaverlag} 1992, S. 190.} }\toendnotes[C]{\smallbreak}\pstart{}{\pb}D\textsuperscript{r}
                  Arthur Schnitzler\pend{}\pstart{}\textcolor{pink}{Seis}{}\ledrightnote{\textcolor{pink}{Seis am Schlern}}\pend{}\pstart{}bei \textcolor{pink}{Waidbruck}{}\ledrightnote{\textcolor{pink}{Ponte Gardena}}\pend{}\pstart{}\textcolor{pink}{Süd-Tirol}{}\ledrightnote{\textcolor{pink}{Südtirol}}\pend{}\pstart{}\textcolor{pink}{Austria}{}\ledrightnote{\textcolor{pink}{Österreich}}\pend{}{\bigskip}\pstart
           \noindent{}\centering{}{\pb}\textcolor{gray}{\textbf{\textcolor{pink}{VENEZIA}{}\ledrightnote{\textcolor{pink}{Venedig}}}}\pend
           \pstart
           \noindent{}\centering{}\textcolor{pink}{Lido\hspace*{1.5em}Hôtel des
                     Bains}{}\ledrightnote{\textcolor{pink}{Grand Hotel des Bains}}\pend
           \pstart
           {\pb}Lieber Arthur! Ich war still, da ich nicht ja{\geminationm}ern wollte. \textcolor{blue}{Paula}{}\ledrightnote{\textcolor{blue}{Paula Beer-Hofmann}}
               hatte Hals- u. Rippenfellentzündung. Wir wollen nach 15 hier weg, über
                  \textcolor{pink}{Südtirol}{}\ledrightnote{\textcolor{pink}{Südtirol}} nach Hause. Hoffentlich \label{K_L01788_1v}\edtext{schickt man die Karte Ihnen nach}{\lemma{\textnormal{\emph{schickt … nach}}}\Cendnote{\textnormal{Der Poststempel ist nicht entzifferbar. Die
                  Jahreszahl ist durch die Erkrankung \textcolor{blue}{Paula}s
                  gesichert. \textcolor{blue}{Schnitzler}s mit Fragezeichen
                  versehene Angabe »Juli?« ist jedoch nicht haltbar, da er sich zu
                  dieser Zeit in \textcolor{pink}{Seis} aufhielt, eine Nachsendung
                  also nicht notwendig gewesen wäre. Diese Nachsendung hat auch stattgefunden und \textcolor{blue}{Schnitzler} die Karte zwei Tage vor dem 16. 9. 1908 erhalten. Damit lässt
                  sich der Zeitraum zwischen der Abreise aus \textcolor{pink}{Seis}
                  und der Ankunft in \textcolor{pink}{Wien} am 14. 9. 1908 für den
                  Versand der Karte bestimmen.}}}\label{K_L01788_1h}. \label{T_L01788_1v}\edtext{Wir grüssen Sie \textcolor{blue}{Beide}{}\ledrightnote{→\textcolor{blue}{Olga Schnitzler}}
                  herzlich.}{\lemma{\textnormal{\emph{Wir … herzlich.}}}\Cendnote{\textnormal{weiter quer am linken
                  Rand}}}\label{T_L01788_1h}\pend
           \pstart \spacefill\mbox{Richard}\pend{}\endnumbering\briefempfaengerindex{Schnitzler, Arthur@\textsc{Schnitzler, Arthur}!zzzBeer-Hofmann, Richard@\emph{von Richard Beer-Hofmann}!1908-09-011@{{[}zwischen 1. und
                  12. 9.? 1908{]}}|)be}\mylabel{h}  \normalsize

\doendnotes{C}
\bigskip
\vfill

\clearpage

\footnotesize

\lohead{\textsc{register}}

% Definiere theindex-Environment komplett neu ohne reledmac
\makeatletter
\renewenvironment{theindex}{%
  \section*{\indexname}%
  \setlength{\parindent}{0pt}%
  \setlength{\parskip}{0pt plus 0.3pt}%
  \let\item\@idxitem
}{%
  \clearpage
}
\makeatother

\IfFileExists{\jobname-pw.ind}{\input{\jobname-pw.ind}}{}

\end{document}

      