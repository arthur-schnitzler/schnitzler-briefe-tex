%% latex-korrekturansicht-vorspann.tex
%% Vorspann für die Korrekturansicht.
%% Lädt die gemeinsame Datei latex-vorspann.tex mit gesetztem Schalter.

\newif\ifkorrekturansicht
\korrekturansichttrue

\input{../tex-inputs/latex-vorspann}


               \section[Hugo von Hofmannsthal an Arthur Schnitzler, 6. 10. 1905]{ Hugo von Hofmannsthal an Arthur Schnitzler, 6. 10. 1905}\nopagebreak\mylabel{v}\rehead{ }\normalsize\beginnumbering\briefempfaengerindex{Schnitzler, Arthur@\textsc{Schnitzler, Arthur}!zzzHofmannsthal, Hugo von@\emph{von Hugo von Hofmannsthal}!1905-10-061@{6. 10. 1905}|(be} \toendnotes[C]{\smallbreak\pagebreak[2]} \Standort{CUL, Schnitzler, B 43.}
\physDesc{Postkarte
\newline{}Handschrift: schwarze Tinte, deutsche Kurrent\newline{}Versand: Stempel: »\nobreak{}\oindex{Rodaun@\textbf{Rodaun}, \emph{Teil eines besiedelten Ortes (A.BSOX)}|pwk}Ro{[}da{]}un, 6. 10. \textcolor{gray}{0}5, 9–12V\nobreak{}«.  
\newline{}Schnitzler: mit Bleistift datiert: »2/10 90\textcolor{gray}{5}« \newline{}Ordnung: 1) mit Bleistift von unbekannter Hand nummeriert:
                              »255« 2) mit Bleistift von unbekannter Hand nummeriert: »258b«}\buchAbdrucke{\weitereDrucke{Hugo von Hofmannsthal, Arthur Schnitzler: \emph{Briefwechsel}. Hg. Therese Nickl und Heinrich Schnitzler. Frankfurt am Main: \emph{S. Fischer} 1964, S. 216.} }\toendnotes[C]{\smallbreak}\pstart{}{\pb}\textsc{Herrn D\textsuperscript{r} Arthur Schnitzler}\pend{}\pstart{}\textcolor{pink}{\textsc{Wien}}{}\ledrightnote{\textcolor{pink}{Wien}}\pend{}\pstart{}\textcolor{pink}{\textsc{XVIII Spöttelgasse 7}.}{}\ledrightnote{\textcolor{pink}{Edmund-Weiß-Gasse}}\pend{}{\bigskip}\pstart
           \noindent{}{\pb}lieber, freuen
               uns ja doch trotz der \textcolor{blue}{W.}{}\ledrightnote{\textcolor{blue}{Lotte Witt}}{ }ſehr auf eine \label{K_L01557-1v}\edtext{\textcolor{green}{Première}{}\ledrightnote{→\textcolor{green}{Zwischenspiel. Komödie in drei Akten}}}{\lemma{\textnormal{\emph{Première}}}\Cendnote{\textnormal{von \emph{\textcolor{green}{Zwischenspiel}} am 12. 10. 1905}}}\label{K_L01557-1h} von Ihnen. Bitten um 2 mittlere Parkettſitze oder vordere (nicht
                  rückwärtige){[}.{]} Wegen ſchlechter Poſt ſchicken Sie ſie bitte an
                  \textcolor{blue}{\textsc{Schlesinger}}{}\ledrightnote{\textcolor{blue}{Franziska Schlesinger}} für \textsc{Hofma{\geminationn}sthal}, \textcolor{pink}{I. \textsc{Elisabethstrasse 6}}{}\ledrightnote{\textcolor{pink}{Elisabethstraße}}.
               Bitte bezahlen Sie ſie indeſſen für mich. Herzlich, und auf Wiederſehen nachher!\pend
           \pstart \spacefill\mbox{Hugo.}\pend{}\endnumbering\briefempfaengerindex{Schnitzler, Arthur@\textsc{Schnitzler, Arthur}!zzzHofmannsthal, Hugo von@\emph{von Hugo von Hofmannsthal}!1905-10-061@{6. 10. 1905}|)be}\mylabel{h}  \normalsize

\doendnotes{C}
\bigskip
\vfill

\clearpage

\footnotesize

\lohead{\textsc{register}}

% Definiere theindex-Environment komplett neu ohne reledmac
\makeatletter
\renewenvironment{theindex}{%
  \section*{\indexname}%
  \setlength{\parindent}{0pt}%
  \setlength{\parskip}{0pt plus 0.3pt}%
  \let\item\@idxitem
}{%
  \clearpage
}
\makeatother

\IfFileExists{\jobname-pw.ind}{\input{\jobname-pw.ind}}{}

\end{document}

      