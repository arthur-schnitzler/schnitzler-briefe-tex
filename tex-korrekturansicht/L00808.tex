%% latex-korrekturansicht-vorspann.tex
%% Vorspann für die Korrekturansicht.
%% Lädt die gemeinsame Datei latex-vorspann.tex mit gesetztem Schalter.

\newif\ifkorrekturansicht
\korrekturansichttrue

\input{../tex-inputs/latex-vorspann}


               \section[Hugo von Hofmannsthal an Arthur Schnitzler, {[}21. 6. 1898{]}]{ Hugo von Hofmannsthal an Arthur Schnitzler, {[}21. 6. 1898{]}}\nopagebreak\mylabel{v}\rehead{ }\normalsize\beginnumbering\briefempfaengerindex{Schnitzler, Arthur@\textsc{Schnitzler, Arthur}!zzzHofmannsthal, Hugo von@\emph{von Hugo von Hofmannsthal}!1898-06-211@{21. 6. 1898}|(be} \toendnotes[C]{\smallbreak\pagebreak[2]} \Standort{CUL, Schnitzler, B 43.}
\physDesc{Brief, 1 Blatt, 4 Seiten
\newline{}Handschrift: schwarze Tinte, deutsche Kurrent
\newline{}Schnitzler: mit schwarzer Tinte datiert: »21/6 98« \newline{}Ordnung: mit Bleistift von unbekannter Hand nummeriert:
                                        »115« }\buchAbdrucke{\weitereDrucke{Hugo von Hofmannsthal, Arthur Schnitzler: \emph{Briefwechsel}. Hg. Therese Nickl und Heinrich Schnitzler. Frankfurt am Main: \emph{S. Fischer} 1964, S. 103.} }\toendnotes[C]{\smallbreak}\pstart
           \raggedleft{}{\pb}Dienstag.\pend
           \pstart{}mein lieber Arthur\pend\pstart
           es war mir ſehr leid, daſs Sie ſich für \label{K_L00808_1v}\edtext{einen Tag}{\lemma{\textnormal{\emph{einen Tag}}}\Cendnote{\textnormal{\textcolor{blue}{Schnitzler} wollte am
                            16. 6. 1898 nach \textcolor{pink}{Hinterbrühl} radeln, wurde aber von einem Regenguss
                        abgehalten.}}}\label{K_L00808_1h} angeſagt haben und dann doch nicht an einem andern geko{\geminationm}en ſind, \strikeout{es} ich
                    verlang mir ſehr, mit Ihnen zuſa{\geminationm}enzuſein.\pend
           \pstart
           Jetzt hab ich nur wenige {\pb}Tage
                    mehr und die möcht ich mir ſehr ſparſam einteilen, bitte alſo wenn es geht,
                    theilen Sie ſich’s auch ſo ein, wie ich Sie dann bitten werde.\pend
           \pstart
           Übermorgen Donnerstag iſt meine \label{K_L00808_2v}\edtext{Prüfung}{\lemma{\textnormal{\emph{Prüfung}}}\Cendnote{\textnormal{Am
                            23. 6. 1898 hatte er sein Hauptrigorosum in Romanischer
                        Philologie.}}}\label{K_L00808_2h}, dann werde {\pb}ich Ihnen gleich ſchreiben.
                        Mittwoch den 29\textsuperscript{ten} um mittag muſs ich ſchon abreiſen.\pend
           \pstart
           Vor der Prüfung geh ich abends nicht ins Café weil ich zu müd werd.\pend
           \pstart
           Herzlich Ihr{\\[\baselineskip]}\spacefill\mbox{Hugo.}\pend
           \leftskip=0em{}\pstart
           \noindent{}Bitte lieber Arthur richten Sie {\pb}mir \uline{viele} Bücher die ſchön zum leſen ſind für die Waffenübung ich
                        hab gar nichts. Womöglich wenn Sie’s haben möcht ich auch eine Novellenſa{\geminationm}lung oder ſonſt etwas wo ältere allenfalls
                        phantaſtiſche Stoffe drin ſind.\pend
           \endnumbering\briefempfaengerindex{Schnitzler, Arthur@\textsc{Schnitzler, Arthur}!zzzHofmannsthal, Hugo von@\emph{von Hugo von Hofmannsthal}!1898-06-211@{21. 6. 1898}|)be}\mylabel{h}  \normalsize

\doendnotes{C}
\bigskip
\vfill

\clearpage

\footnotesize

\lohead{\textsc{register}}

% Definiere theindex-Environment komplett neu ohne reledmac
\makeatletter
\renewenvironment{theindex}{%
  \section*{\indexname}%
  \setlength{\parindent}{0pt}%
  \setlength{\parskip}{0pt plus 0.3pt}%
  \let\item\@idxitem
}{%
  \clearpage
}
\makeatother

\IfFileExists{\jobname-pw.ind}{\input{\jobname-pw.ind}}{}

\end{document}

      