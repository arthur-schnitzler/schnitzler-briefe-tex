%% latex-korrekturansicht-vorspann.tex
%% Vorspann für die Korrekturansicht.
%% Lädt die gemeinsame Datei latex-vorspann.tex mit gesetztem Schalter.

\newif\ifkorrekturansicht
\korrekturansichttrue

\input{../tex-inputs/latex-vorspann}


\renewcommand{\erwaehntePersonen}{Personen: Siegfried Jacobsohn}
\renewcommand{\erwaehnteOrte}{Orte: Berlin, Hotel Bristol, Hotel Continental (Berlin), Staatsoper Berlin}
\renewcommand{\erwaehnteWerke}{Werke: Fidelio}
\section[ Paul Goldmann an Arthur Schnitzler, 20. 11. 1905]{Paul Goldmann an Arthur Schnitzler, 20. 11. 1905}
\nopagebreak\mylabel{v}
\rehead{ }\normalsize\beginnumbering\briefempfaengerindex{Schnitzler, Arthur@\textsc{Schnitzler, Arthur}!zzzGoldmann, Paul@\emph{von Paul Goldmann}!1905-11-201@{20. 11. 1905}|(be}
\toendnotes[C]{\smallbreak\pagebreak[2]}\Standort{DLA, A:Schnitzler, HS.NZ85.1.3175.}
\physDesc{Postkarte
\newline{}Handschrift: 1) blaue Tinte, deutsche Kurrent\hspace{1em}2) blaue Tinte, lateinische Kurrent (\noindent{}Adresse)\hspace{1em}
\newline{}Versand: 1) Stempel: »\nobreak{}\oindex{Berlin@\textbf{Berlin}, \emph{https://www.geonames.org/ontologyP.PPLC}|pwk}Berlin S. W. 11, 20. 11. 05, 11\textsuperscript{20} V.\nobreak{}«.   2) Stempel: »\nobreak{}\oindex{Berlin@\textbf{Berlin}, \emph{https://www.geonames.org/ontologyP.PPLC}|pwk}Berlin N. W. 7, 20. 11. 05, 11\textsuperscript{40} V.\nobreak{}«. 
\newline{}Schnitzler: mit Bleistift das Datum » 1{[}9{]}05 20/11« vermerkt }\toendnotes[C]{\smallbreak}\pstart{}{\pb}Rohr-\textcolor{gray}{\textbf{Poſtkarte}}\pend{}\pstart{}Herrn\pend{}\pstart{}Dr. Arthur Schnitzler\pend{}\pstart{}\textcolor{pink}{Berlin}{}\ledrightnote{\textcolor{pink}{Berlin}}\pend{}\pstart{}Hotel \substVorne{}\textsuperscript{\textcolor{pink}{B\textcolor{gray}{ri}st\textcolor{gray}{o}l}{}\ledrightnote{\textcolor{pink}{Hotel Bristol}}}{\allowbreak}\substDazwischen{}\textcolor{pink}{Continental}{}\ledrightnote{\textcolor{pink}{Hotel Continental (Berlin)}}\substHinten{}\pend{}
{\bigskip}
\pstart
           \noindent{}{\pb}Montag. Lieber Freund,
               Es hat mir ſehr leid gethan, Deinen lieben Beſuch geſtern verſäumt zu haben. Ich muß wenige Minuten vorher weggegangen ſein.
               Hätteſt Du mir telephonirt, ſo hätte ich Dich gern erwartet.\pend
           
\pstart
           Willſt Du heut{ }Abend mit mir in die \label{K-L03237-1v}\edtext{\textcolor{pink}{Oper}{}\ledrightnote{{$\rightarrow$}\textcolor{pink}{Staatsoper Berlin}}}{\lemma{\textnormal{\emph{Oper}}}\Cendnote{\textnormal{\textcolor{blue}{Schnitzler} verbrachte den Abend
                  nicht mit \textcolor{blue}{Goldmann}, sondern mit \textcolor{blue}{Siegfried Jacobsohn}. Siehe A. S.: \emph{Tagebuch}, 20. 11. 1905.}}}\label{K-L03237-1h} gehen
                  (\textsc{\textcolor{green}{Fidelio}{}\ledrightnote{\textcolor{green}{Fidelio}}}, Urfaſſung)? Bis 4 Uhr halte ich das Billet zu Deiner Verfügung.
               Erbitte telephoniſche Antwort.\pend
           
\pstart
           Herzlichſt {\\[\baselineskip]}\spacefill\mbox{Dein Paul Goldmann}\pend
           \leftskip=0em{}\endnumbering\briefempfaengerindex{Schnitzler, Arthur@\textsc{Schnitzler, Arthur}!zzzGoldmann, Paul@\emph{von Paul Goldmann}!1905-11-201@{20. 11. 1905}|)be}\mylabel{h}  \normalsize

\doendnotes{C}
\bigskip
\vfill

\clearpage

\footnotesize

\lohead{\textsc{register}}

% Definiere theindex-Environment komplett neu ohne reledmac
\makeatletter
\renewenvironment{theindex}{%
  \section*{\indexname}%
  \setlength{\parindent}{0pt}%
  \setlength{\parskip}{0pt plus 0.3pt}%
  \let\item\@idxitem
}{%
  \clearpage
}
\makeatother

\IfFileExists{\jobname-pw.ind}{\input{\jobname-pw.ind}}{}

\end{document}

      