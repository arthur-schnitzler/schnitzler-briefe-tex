%% latex-korrekturansicht-vorspann.tex
%% Vorspann für die Korrekturansicht.
%% Lädt die gemeinsame Datei latex-vorspann.tex mit gesetztem Schalter.

\newif\ifkorrekturansicht
\korrekturansichttrue

\input{../tex-inputs/latex-vorspann}


\renewcommand{\erwaehntePersonen}{Personen: Paul Goldmann, Fedor Mamroth, J. Schwarz}
\renewcommand{\erwaehnteOrte}{Orte: Paris, Wien}
\renewcommand{\erwaehnteWerke}{Werke: Belletristische Rundschau, Frankfurter Zeitung, Sterben. Novelle}
\section[Paul Goldmann an Arthur Schnitzler, 4. 12. 1894]{Paul Goldmann an Arthur Schnitzler, 4. 12. 1894}
\nopagebreak\mylabel{v}
\rehead{ }\normalsize\beginnumbering\briefempfaengerindex{Schnitzler, Arthur@\textsc{Schnitzler, Arthur}!zzzGoldmann, Paul@\emph{von Paul Goldmann}!1894-12-041@{04. 12. 1894}|(be}
\toendnotes[C]{\smallbreak\pagebreak[2]}\Standort{DLA, A:Schnitzler, HS.NZ85.1.3164.}
\physDesc{Brief, 1 Blatt, 1 Seite, 324 Zeichen
\newline{}Handschrift: schwarze Tinte, deutsche Kurrent
\newline{}Schnitzler: mit Bleistift die Jahreszahl »94« vermerkt }\toendnotes[C]{\smallbreak}
\pstart
           \raggedleft{}{\pb}\textcolor{pink}{Paris}{}\ledrightnote{\textcolor{pink}{Paris}},
                  4. December.\pend
           
\pstart\center{}Mein lieber Freund,\pend
\pstart
           Die »\textcolor{green}{Frkf. Ztg.}{}\ledrightnote{\textcolor{green}{Frankfurter Zeitung}}« worin Dein \label{K_L02624-1v}\edtext{\textcolor{green}{Buch}{}\ledrightnote{{$\rightarrow$}\textcolor{green}{Sterben. Novelle}}{ }\textcolor{green}{beſprochen}{}\ledrightnote{{$\rightarrow$}\textcolor{green}{Belletristische Rundschau}}}{\lemma{\textnormal{\emph{Buch beſprochen}}}\Cendnote{\textnormal{\textcolor{blue}{J. Schwarz}: \emph{\textcolor{green}{Belletristische Rundschau}}. In: \emph{\textcolor{green}{Frankfurter Zeitung}}, Nr. 336, 4. 12. 1894,
                     S. 1–3.}}}\label{K_L02624-1h} worden, haſt Du gewiß ſchon geſehen. Der Sicherheit
               halber ſchicke ich ſie Dir zu. Schreib’, bitte, eine \label{K_L02624-2v}\edtext{Zeile an meinen \textcolor{blue}{Onkel}{}\ledrightnote{{$\rightarrow$}\textcolor{blue}{Fedor Mamroth}}}{\lemma{\textnormal{\emph{Zeile an meinen Onkel}}}\Cendnote{\textnormal{siehe Arthur Schnitzler an Fedor Mamroth, 7. 12. 1894}}}\label{K_L02624-2h}, der diesmal beſonders brav geweſen iſt.\pend
           
\pstart
           Wie gehts Dir? Und wann höre ich wieder etwas von Dir?\pend
           
\pstart
           In Treue{\\[\baselineskip]}Dein{\\[\baselineskip]}\spacefill\mbox{Paul Goldmann\textcolor{gray}{.}}\pend
           \leftskip=0em{}\endnumbering\briefempfaengerindex{Schnitzler, Arthur@\textsc{Schnitzler, Arthur}!zzzGoldmann, Paul@\emph{von Paul Goldmann}!1894-12-041@{04. 12. 1894}|)be}\mylabel{h}  \normalsize

\doendnotes{C}
\bigskip
\vfill

\clearpage

\footnotesize

\lohead{\textsc{register}}

% Definiere theindex-Environment komplett neu ohne reledmac
\makeatletter
\renewenvironment{theindex}{%
  \section*{\indexname}%
  \setlength{\parindent}{0pt}%
  \setlength{\parskip}{0pt plus 0.3pt}%
  \let\item\@idxitem
}{%
  \clearpage
}
\makeatother

\IfFileExists{\jobname-pw.ind}{\input{\jobname-pw.ind}}{}

\end{document}

      