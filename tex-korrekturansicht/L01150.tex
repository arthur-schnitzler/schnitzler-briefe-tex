%% latex-korrekturansicht-vorspann.tex
%% Vorspann für die Korrekturansicht.
%% Lädt die gemeinsame Datei latex-vorspann.tex mit gesetztem Schalter.

\newif\ifkorrekturansicht
\korrekturansichttrue

\input{../tex-inputs/latex-vorspann}


               \section[Arthur Schnitzler an Richard Beer-Hofmann, 22. 7. 1901]{ Arthur Schnitzler an Richard Beer-Hofmann, 22. 7. 1901}\nopagebreak\mylabel{v}\rehead{ }\normalsize\beginnumbering\briefempfaengerindex{Beer-Hofmann, Richard@\textsc{Beer-Hofmann, Richard}!zzzSchnitzler, Arthur@\emph{von Arthur Schnitzler}!1901-07-221@{22. 7. 1901}|(be} \toendnotes[C]{\smallbreak\pagebreak[2]} \Standort{YCGL, MSS 31.}
\physDesc{Brief, 1 Blatt, 3 Seiten, Umschlag
\newline{}Handschrift: 1) Bleistift, deutsche Kurrent\hspace{1em}2) schwarze Tinte, deutsche Kurrent (\noindent{}Umschlag)\hspace{1em}\newline{}Versand: 1) Stempel: »\nobreak{}\oindex{Vahrn@\textbf{Vahrn}, \emph{Besiedelter Ort (A.BSO)}|pwk}Vahrn, 22. 7. 01\nobreak{}«.  2) Stempel: »\nobreak{}\oindex{Poertschach@\textbf{Pörtschach}, \emph{https://www.geonames.org/ontologyP.PPL}|pwk}{\pb}Pörtschach am See, 23 7 01\nobreak{}«. }\buchAbdrucke{\weitereDrucke{Arthur Schnitzler, Richard Beer-Hofmann: \emph{Briefwechsel 1891–1931}. Hg. Konstanze Fliedl. Wien, Zürich: \emph{Europaverlag} 1992, S. 153–154.} }\toendnotes[C]{\smallbreak}\pstart{}{\pb}\textsc{Herrn Dr. Rich. Beer-Hofmann}\pend{}\pstart{}\textsc{\textcolor{pink}{Pörtschach}{}\ledrightnote{\textcolor{pink}{Pörtschach}}}\pend{}\pstart{}\textsc{\textcolor{pink}{Villa Arnstein}{}\ledrightnote{\textcolor{pink}{Villa Arnstein}}.}\pend{}{\bigskip}\pstart
           \raggedleft{}{\pb}\textcolor{pink}{\textsc{Vahrn}}{}\ledrightnote{\textcolor{pink}{Vahrn}}, 22/7 901\pend
           \pstart
           lieber Richard, von dem Tod Ihrer \textcolor{blue}{Stiefmama}{}\ledrightnote{→\textcolor{blue}{Rosa Beer}} hab ich durch \textcolor{blue}{Schw.}{}\ledrightnote{\textcolor{blue}{Gustav Schwarzkopf}} erfahren, noch eh Sie mirs ſchrieben, zu formeller Condolenz wars zu
               ſpät, bitte ſagen Sie Ihrem \textcolor{blue}{Papa}{}\ledrightnote{→\textcolor{blue}{Hermann Beer}} nachträglich, daſs ich ihm meine herzliche Theilnahme alſo lieber durch
               Sie ausdrücken laſſe. – \textcolor{blue}{Paul}{}\ledrightnote{\textcolor{blue}{Paul Goldmann}} dürfte ſchon in den
               nächſten Tagen an den \textcolor{pink}{Wörtherſee}{}\ledrightnote{\textcolor{pink}{Wörthersee}} kommen, iſt
               erbittert über Sie, will Sie gar nicht besuchen u. ſ. w. Schreiben Sie ihm doch noch
               eheſtens ein Wort. Vom \textcolor{pink}{Wörtherſee}{}\ledrightnote{\textcolor{pink}{Wörthersee}} ko{\geminationm}t \textcolor{blue}{G.}{}\ledrightnote{\textcolor{blue}{Paul Goldmann}} herunter, ich
               muſs mir noch irgend was höheres ſuchen {\pb}werde mich
               auf der \textcolor{pink}{Seiſer Alpe}{}\ledrightnote{\textcolor{pink}{Seiser Alm}} u im \textcolor{pink}{Tierſer Thal}{}\ledrightnote{\textcolor{pink}{Tiers}} umſehn. Machen Sie’s doch möglich auch zu kommen.
               Die letzten Sommertage denk’ ich \textcolor{pink}{Gardaſee}{}\ledrightnote{\textcolor{pink}{Lago di Garda}}, \textsc{ev}. \textcolor{pink}{Torbole}{}\ledrightnote{\textcolor{pink}{Torbole sul Garda}}? –\pend
           \pstart
           Ich find es hier ſehr angenehm, die Zimmer offenbar neu hergerichtet ſehr hübſch, das
               Eſſen gut, wenig Leut, und warm. Ich \substVorne{}\textsuperscript{\textcolor{gray}{×}\-\textcolor{gray}{×}\-\textcolor{gray}{×}\-\textcolor{gray}{×}\-\textcolor{gray}{×}\-\textcolor{gray}{×}}\substDazwischen{}ſchreibe\substHinten{} (\textcolor{green}{3a. Stück}{}\ledrightnote{→\textcolor{green}{Der einsame Weg. Schauspiel in fünf Akten}{\newline}→\textcolor{green}{Professor Bernhardi. Komödie in fünf Akten}}). An
               der Zerſtörung der »Grämlichkeit« wird von berufener Seite mit Talent gearbeitet.
                  We{\geminationn} mich etwas ſtört, iſt es nur der Um{\pb}ſtand, daſs man in der betreffenden Familie Sie für
               den weitaus hervorragendſten von {\dots} hm {\dots} Alt-\textcolor{pink}{Wien}{}\ledrightnote{\textcolor{pink}{Wien}} hält, eine Meinung, die Sie
               hoffentlich durch Ihr {\dots} wieder hm {\dots} nächſtes \textcolor{green}{Stück}{}\ledrightnote{→\textcolor{green}{Der Graf von Charolais. Ein Trauerspiel}}
               endgiltig begraben werden.\pend
           \pstart
           – Schreiben Sie \introOben{}– we{\geminationn}\introOben{} bald, da{\geminationn} noch hieher, ſonſt \textcolor{pink}{Wien}{}\ledrightnote{\textcolor{pink}{Wien}}.\pend
           \pstart
           Heute Ausflug \textcolor{pink}{Karerſee}{}\ledrightnote{\textcolor{pink}{Karersee}}, wo \textcolor{blue}{Julius}{}\ledrightnote{\textcolor{blue}{Julius Schnitzler}} u \textcolor{blue}{Frau}{}\ledrightnote{→\textcolor{blue}{Helene Schnitzler}}.\pend
           \pstart
           Gehts den Ihren gut? Baden Sie viel? Sehn Sie die übrigen Rundwohner?\pend
           \pstart Von Herzen Ihr \spacefill\mbox{Arthur}\pend{}\endnumbering\briefempfaengerindex{Beer-Hofmann, Richard@\textsc{Beer-Hofmann, Richard}!zzzSchnitzler, Arthur@\emph{von Arthur Schnitzler}!1901-07-221@{22. 7. 1901}|)be}\mylabel{h}  \normalsize

\doendnotes{C}
\bigskip
\vfill

\clearpage

\footnotesize

\lohead{\textsc{register}}

% Definiere theindex-Environment komplett neu ohne reledmac
\makeatletter
\renewenvironment{theindex}{%
  \section*{\indexname}%
  \setlength{\parindent}{0pt}%
  \setlength{\parskip}{0pt plus 0.3pt}%
  \let\item\@idxitem
}{%
  \clearpage
}
\makeatother

\IfFileExists{\jobname-pw.ind}{\input{\jobname-pw.ind}}{}

\end{document}

      