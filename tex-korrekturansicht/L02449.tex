%% latex-korrekturansicht-vorspann.tex
%% Vorspann für die Korrekturansicht.
%% Lädt die gemeinsame Datei latex-vorspann.tex mit gesetztem Schalter.

\newif\ifkorrekturansicht
\korrekturansichttrue

\input{../tex-inputs/latex-vorspann}


               \section[Stefan Großmann an Arthur Schnitzler, 21. 9. 1925]{ Stefan Großmann an Arthur Schnitzler, 21. 9. 1925}\nopagebreak\mylabel{v}\rehead{ }\normalsize\beginnumbering\briefempfaengerindex{Schnitzler, Arthur@\textsc{Schnitzler, Arthur}!zzzGrossmann, Stefan@\emph{von Stefan Großmann}!1925-09-211@{21. 9. 1925}|(be} \toendnotes[C]{\smallbreak\pagebreak[2]} \Standort{DLA, A:Schnitzler, HS.NZ85.1.3232.}
\physDesc{Brief, 1 Blatt, 2 Seiten
\newline{}Schreibmaschine
\newline{}Handschrift: schwarze Tinte, deutsche Kurrent (\noindent{}Unterschrift)
\newline{}Schnitzler: mit rotem Buntstift vier Unterstreichungen }\toendnotes[C]{\smallbreak}\pstart
           \noindent{}\centering{}{\pb}\textcolor{gray}{\textbf{\textcolor{brown}{Das Tage-Buch}{}\ledrightnote{\textcolor{brown}{Das Tage-Buch}}}}\pend
           \pstart
           \noindent{}\centering{}\textcolor{gray}{\textbf{\emph{Herausgeber: Stefan Großmann und \textcolor{blue}{Leopold Schwarzschild}{}\ledrightnote{\textcolor{blue}{Leopold Schwarzschild}}}}}\pend
           \pstart
           \noindent{}\centering{}\textcolor{gray}{\textbf{Tagebuchverlag m. b. H., \textcolor{pink}{Berlin
                        SW 19}{}\ledrightnote{\textcolor{pink}{Berlin}}}}\pend
           \pstart
           \noindent{}\centering{}\textcolor{gray}{\textbf{\textcolor{pink}{BEUTHSTRASSE 19}{}\ledrightnote{\textcolor{pink}{Beuthstrasse}}}}\pend
           \pstart
           \noindent{}\centering{}\textcolor{gray}{\textbf{\emph{Telegramm-Adresse: Tagebuch \textcolor{pink}{Berlin}{}\ledrightnote{\textcolor{pink}{Berlin}} ⋅ Fernsprecher: Merkur 8790–8792}}}\pend
           \pstart
           \noindent{}\centering{}\textcolor{gray}{\textbf{\emph{\so{Sprechstunde der Redaktion: 12–1 Uhr}}}}\pend
           \pstart
           \noindent{}\centering{}\textcolor{gray}{\textbf{*}}\pend
           \pstart
           \noindent{}Tgb./Gr./Schl.\hfill \textcolor{pink}{Berlin}{}\ledrightnote{\textcolor{pink}{Berlin}}, den 21. September
                     1925.\pend
           \pstart
           \raggedleft{}Herrn\pend
           \pstart
           \noindent{}\raggedleft{}Dr. Arthur \so{Schnitzler}\pend
           \pstart
           \noindent{}\raggedleft{}\textcolor{pink}{\so{Wien } XVIII}{}\ledrightnote{\textcolor{pink}{XVIII., Währing}}\pend
           {\bigskip}\pstart
           \noindent{}\raggedleft{}\textcolor{pink}{Sternwartestr. 71}{}\ledrightnote{\textcolor{pink}{Sternwartestraße}}. \pend
           \pstart\center{}Verehrter Herr Doktor Schnitzler!\pend\pstart
           Ich bemühe mich, meinem \textcolor{brown}{TAGE-BUCH}{}\ledrightnote{\textcolor{brown}{Das Tage-Buch}} einen leichten \textcolor{pink}{österreich}{}\ledrightnote{\textcolor{pink}{Österreich}}ischen Anstrich zu
               geben. Sie würden mir eine sehr grosse Freude machen und mich zu grossem Dank
               verpflichten, wenn Sie mir für eine der nächsten Nummern des \textcolor{brown}{TAGE-BUCH}{}\ledrightnote{\textcolor{brown}{Das Tage-Buch}}ES einen Beitrag schicken würden. Gäbe es nicht in einer Ihrer Mappen irgendwo
               eine kleine Novelle, die Sie mir überlassen könnten? Ich würde mich, da sich das \textcolor{brown}{TAGE-BUCH}{}\ledrightnote{\textcolor{brown}{Das Tage-Buch}} ja jetzt durchgesetzt hat, zu dem höchsten Honorar entschliessen, das ich
               aufbringen kann. Aber selbst wenn Sie mir diese Bitte abschlagen müssen – ich hoffe,
               dass es nicht geschehen muss –, weiss ich aus den \label{K_L02449_1v}\edtext{Veröffentlichungen}{\lemma{\textnormal{\emph{Veröffentlichungen}}}\Cendnote{\textnormal{Gegenwärtig ist kein Abdruck eines Textes
                  im Jahre 1925 bekannt.}}}\label{K_L02449_1h} in der \textcolor{brown}{Frankfurter Zeitung}{}\ledrightnote{\textcolor{brown}{Frankfurter Zeitung}}, dass Sie eine grosse Mappe mit Reflexionen haben. Ich
               bitte Sie sehr, öffnen Sie diese Mappe und schicken Sie mir einige Seiten daraus, die
               ich im {\pb}\textcolor{brown}{TAGE-BUCH}{}\ledrightnote{\textcolor{brown}{Das Tage-Buch}} veröffentlichen kann. Ich weiss, dass Sie viele solche Bitten abschlagen,
               dennoch glaube ich, dass Sie mir in mein \textcolor{pink}{Berlin}{}\ledrightnote{\textcolor{pink}{Berlin}}er
               Exil diesmal keine Absage schicken werden.\pend
           \pstart
           Ich bin mit dankbaren Grüssen{\\[\baselineskip]}Ihr sehr ergebener{\\[\baselineskip]}\spacefill\mbox{{[}hs.:{]} Stefan Großmann}\pend
           \leftskip=0em{}\endnumbering\briefempfaengerindex{Schnitzler, Arthur@\textsc{Schnitzler, Arthur}!zzzGrossmann, Stefan@\emph{von Stefan Großmann}!1925-09-211@{21. 9. 1925}|)be}\mylabel{h}  \normalsize

\doendnotes{C}
\bigskip
\vfill

\clearpage

\footnotesize

\lohead{\textsc{register}}

% Definiere theindex-Environment komplett neu ohne reledmac
\makeatletter
\renewenvironment{theindex}{%
  \section*{\indexname}%
  \setlength{\parindent}{0pt}%
  \setlength{\parskip}{0pt plus 0.3pt}%
  \let\item\@idxitem
}{%
  \clearpage
}
\makeatother

\IfFileExists{\jobname-pw.ind}{\input{\jobname-pw.ind}}{}

\end{document}

      