%% latex-korrekturansicht-vorspann.tex
%% Vorspann für die Korrekturansicht.
%% Lädt die gemeinsame Datei latex-vorspann.tex mit gesetztem Schalter.

\newif\ifkorrekturansicht
\korrekturansichttrue

\input{../tex-inputs/latex-vorspann}


\section[Stefan Zweig an Arthur Schnitzler, 21. 2. 1911]{L03636 Stefan Zweig an Arthur Schnitzler, 21. 2. 1911}
\nopagebreak\mylabel{L03636v}
\rehead{ }\normalsize\beginnumbering\briefempfaengerindex{Schnitzler, Arthur@\textsc{Schnitzler, Arthur}!zzzZweig, Stefan@\emph{von Stefan Zweig}!1911-02-211@{21. 2. 1911}|(be}
\toendnotes[C]{\smallbreak\pagebreak[2]}\Standort{CUL, Schnitzler, B 118.}
\physDesc{Bildpostkarte, 635 Zeichen
\newline{}Handschrift: schwarze Tinte, lateinische Kurrent
\newline{}Versand: Stempel: »\nobreak{}\oindex{Boulevard des Italiens@\textbf{Boulevard des Italiens}, \emph{Straße (K.STR)}|pwk}Paris B\textsuperscript{d} des Itali\textcolor{gray}{ens}, 21 Fevr 11, 12 H\nobreak{}«.  }
\buchAbdrucke{\weitereDrucke{Stefan Zweig: \emph{Briefwechsel mit Hermann Bahr, Sigmund Freud, Rainer Maria
                        Rilke und Arthur Schnitzler}. Frankfurt am Main: \emph{S. Fischer} 1987, S. 363.} }\toendnotes[C]{\smallbreak}\pstart{}{\pb}D\textsuperscript{r} Artur
                  Schnitzler\pend{}\pstart{}\textcolor{pink}{Vienne (Autriche)}\oindex{Wien@\textbf{Wien}, \emph{A.ADM2}|pw}{}\ledrightnote{\textcolor{pink}{Wien}}\pend{}\pstart{}\textcolor{pink}{XVIII. \label{K_L03636-1v}\edtext{Sternwartestrasse 72}{\lemma{\textnormal{\emph{Sternwartestrasse 72}}}\Cendnote{\textnormal{\textcolor{blue}{Zweig}\pwindex{Zweig, Stefan 28.11.1881 Wien – 23.02.1942 Petrópolis@\textsc{Zweig, Stefan} (28.11.1881 Wien – 23.02.1942 Petrópolis), \emph{Schriftsteller}|pwk} wechselt bei der Adressierung
                        seiner Schreiben an \textcolor{blue}{Schnitzler} immer
                        wieder zwischen der falschen Hausnummer »72« und der
                        richtigen »71«.}}}\label{K_L03636-1}}\oindex{Sternwartestraße 71@\textbf{Sternwartestraße 71}, \emph{Wohngebäude (K.WHS)}|pw}{}\ledrightnote{\textcolor{pink}{Sternwartestraße 71}}\pend{}{\bigskip}
\pstart
           \noindent{}\centering{}{\pb}\textcolor{gray}{\textbf{175 \hspace*{1em}\textcolor{pink}{PARIS}\oindex{Paris@\textbf{Paris}, \emph{P.PPLC}|pw}{}\ledrightnote{\textcolor{pink}{Paris}}. – \textcolor{pink}{La Place de la Bastille}\oindex{Bastille@\textbf{Bastille}, \emph{Gebäude (K.GBD)}|pw}{}\ledrightnote{\textcolor{pink}{Bastille}}. – LL}}\pend
           \vspace{1em}
\pstart
           \noindent{}{\pb}Verehrter Herr Doktor, ich sende Ihnen und Ihrer werten Frau \textcolor{blue}{Gemahlin}\pwindex{Schnitzler, Olga 17.01.1882 Wien – 13.01.1970 Lugano@\textsc{Schnitzler, Olga} (17.01.1882 Wien – 13.01.1970 Lugano), \emph{Schauspielerin, Sängerin}|pwv}{}\ledrightnote{{$\rightarrow$}\emph{\textcolor{blue}{Olga Schnitzler}}} von hier aus die
               herzlichsten Abschiedsgrüsse vor meiner \label{K_L03636-2v}\edtext{\textcolor{pink}{Amerika}\oindex{Amerika@\textbf{Amerika}, \emph{kein passender Code gefunden}|pw}{}\ledrightnote{\textcolor{pink}{Amerika}}fahrt}{\lemma{\textnormal{\emph{Amerikafahrt}}}\Cendnote{\textnormal{Vom 22. 2. 1911 bis zum 21. 4. 1911
                  unternahm \textcolor{blue}{Stefan Zweig}\pwindex{Zweig, Stefan 28.11.1881 Wien – 23.02.1942 Petrópolis@\textsc{Zweig, Stefan} (28.11.1881 Wien – 23.02.1942 Petrópolis), \emph{Schriftsteller}|pwk} eine \textcolor{pink}{amerikanische}\oindex{Amerika@\textbf{Amerika}, \emph{kein passender Code gefunden}|pwk} Reise. Die erste Station war \textcolor{pink}{New York}\oindex{New York City@\textbf{New York City}, \emph{P.PPL}|pwk}. Von dort reiste er in mehrere
                  Städte an der \textcolor{pink}{nordamerikanischen}\oindex{Nordamerika@\textbf{Nordamerika}, \emph{L.RGN}|pwk} Ostküste,
                  dann nach \textcolor{pink}{Chicago}\oindex{Chicago@\textbf{Chicago}, \emph{P.PPLA2}|pwk} und \textcolor{pink}{Kanada}\oindex{Kanada@\textbf{Kanada}, \emph{A.PCLI}|pwk}, um über \textcolor{pink}{Bermuda}\oindex{Bermuda@\textbf{Bermuda}, \emph{A.PCLD}|pwk} und \textcolor{pink}{Kuba}\oindex{Cuba@\textbf{Cuba}, \emph{A.PCLI}|pwk} bis nach \textcolor{pink}{Südamerika}\oindex{Südamerika@\textbf{Südamerika}, \emph{Kontinent (A.KNT)}|pwk} zu gelangen.}}}\label{K_L03636-2}: Gestern
               sprach ich \label{K_L03636-3v}\edtext{\textcolor{blue}{Paul Morisse}\pwindex{Morisse, Paul 1866-03-11 Rouen – 1946-09-28 Paris@\textsc{Morisse, Paul} (1866-03-11 Rouen – 1946-09-28 Paris), \emph{Übersetzer}|pw}{}\ledrightnote{\textcolor{blue}{Paul Morisse}}}{\lemma{\textnormal{\emph{Paul Morisse}}}\Cendnote{\textnormal{\textcolor{blue}{Paul Morisse}\pwindex{Morisse, Paul 1866-03-11 Rouen – 1946-09-28 Paris@\textsc{Morisse, Paul} (1866-03-11 Rouen – 1946-09-28 Paris), \emph{Übersetzer}|pwk} war Dichter, Übersetzer und
                  Redaktionsmitglied des \emph{\textcolor{brown}{Mercure de France}\orgindex{Mercure de France@Mercure de France|pwk}}. Er
                  verfasste mehrere Übersetzungen von Werken \textcolor{blue}{Zweigs}\pwindex{Zweig, Stefan 28.11.1881 Wien – 23.02.1942 Petrópolis@\textsc{Zweig, Stefan} (28.11.1881 Wien – 23.02.1942 Petrópolis), \emph{Schriftsteller}|pwk}. Die hier geplante Übersetzung von \emph{\textcolor{green}{Das weite Land}\pwindex{weite Land. Tragikomödie in fünf Akten@\emph{Das weite Land. Tragikomödie in fünf Akten}|pwk}} dürfte nie publiziert oder aufgeführt worden sein. 
                  Auf die vorliegende briefliche Einführung folgte ein Brief von \textcolor{blue}{Morisse}\pwindex{Morisse, Paul 1866-03-11 Rouen – 1946-09-28 Paris@\textsc{Morisse, Paul} (1866-03-11 Rouen – 1946-09-28 Paris), \emph{Übersetzer}|pwk} an \textcolor{blue}{Schnitzler}, datiert mit
                  23. 2. 1911. Er beginnt folgendermaßen: »\begin{otherlanguage}{french}Je crois que mon nom ne vous est pas tout a fait inconnu,
                        puisque M. \textcolor{blue}{Stefan Zweig}\pwindex{Zweig, Stefan 28.11.1881 Wien – 23.02.1942 Petrópolis@\textsc{Zweig, Stefan} (28.11.1881 Wien – 23.02.1942 Petrópolis), \emph{Schriftsteller}|pw}, dont j’ai
                        traduit l’\textcolor{green}{ouvrage}\pwindex{Émile Verhaeren@\emph{Émile Verhaeren}|pwv}\pwindex{Émile Verhaeren. Sa vie, son oeuvre@\emph{Émile Verhaeren. Sa vie, son oeuvre}|pwv} sur le poète \textcolor{blue}{Émile
                              Verhaeren}\pwindex{Verhaeren, Émile 21.05.1855 Sint-Amands – 27.11.1916 Rouen@\textsc{Verhaeren, Émile} (21.05.1855 Sint-Amands – 27.11.1916 Rouen), \emph{Schriftsteller, Schriftsteller, Krimiautor}|pw}, m’a dit vous avois parlé de
                        moi.\end{otherlanguage}« (»Ich glaube, mein Name ist ihnen nicht vollständig unbekannt, da
                           \textcolor{blue}{Stefan Zweig}\pwindex{Zweig, Stefan 28.11.1881 Wien – 23.02.1942 Petrópolis@\textsc{Zweig, Stefan} (28.11.1881 Wien – 23.02.1942 Petrópolis), \emph{Schriftsteller}|pw}, dessen \textcolor{green}{Werk}\pwindex{Émile Verhaeren@\emph{Émile Verhaeren}|pwv}\pwindex{Émile Verhaeren. Sa vie, son oeuvre@\emph{Émile Verhaeren. Sa vie, son oeuvre}|pwv} über den Dichter \textcolor{blue}{Émile
                                 Verhaeren}\pwindex{Verhaeren, Émile 21.05.1855 Sint-Amands – 27.11.1916 Rouen@\textsc{Verhaeren, Émile} (21.05.1855 Sint-Amands – 27.11.1916 Rouen), \emph{Schriftsteller, Schriftsteller, Krimiautor}|pw} ich übersetzt habe, mir sagte, er hätte vor Ihnen von mir gesprochen.«)
                  \textcolor{blue}{Schnitzler} dürfte hinhaltend geantwortet haben und die 
                  Sache wurde erst im Herbst/Winter des Jahres wieder aufgenommen, siehe Stefan Zweig an Arthur Schnitzler, 6. [11.?] 1911.
                  }}}\label{K_L03636-3}, den Secretär des
                  »\textcolor{brown}{Mercure de France}\orgindex{Mercure de France@Mercure de France|pw}{}\ledrightnote{\textcolor{brown}{Mercure de France}}«, der sehr gerne – ich
               erzählte ihm davon – das \textcolor{green}{Weite Land}\pwindex{weite Land. Tragikomödie in fünf Akten@\emph{Das weite Land. Tragikomödie in fünf Akten}|pw}{}\ledrightnote{\textcolor{green}{Das weite Land. Tragikomödie in fünf Akten}} übersetzen
               möchte und sich an Sie wenden will. Ich kann ihn \uline{aufrichtigst} empfehlen{[}:{]}{ }{\pb}er ist sehr tüchtig und hat auch die
               nötigen Verbindungen mit den Theatern. Es  ist mir leid, dass ich über den \label{K_L03636-4v}\edtext{\textcolor{pink}{Berliner}\oindex{Berlin@\textbf{Berlin}, \emph{P.PPLC}|pw}{}\ledrightnote{\textcolor{pink}{Berlin}} Erfolg}{\lemma{\textnormal{\emph{Berliner Erfolg}}}\Cendnote{\textnormal{Am 23. 2. 1911 gab \textcolor{blue}{Olga Schnitzler}\pwindex{Schnitzler, Olga 17.01.1882 Wien – 13.01.1970 Lugano@\textsc{Schnitzler, Olga} (17.01.1882 Wien – 13.01.1970 Lugano), \emph{Schauspielerin, Sängerin}|pwk} ein \textcolor{violet}{Gesangskonzert}\eventindex{Klindworth-Scharwenka-Saal@\textbf{Klindworth-Scharwenka-Saal}!Gesangskonzert von Olga Schnitzler, 23.2.1911@Gesangskonzert von Olga Schnitzler, 23.2.1911|pwkv} im \textcolor{pink}{Klindworth-Scharwenka-Saal}\oindex{Klindworth-Scharwenka-Saal@\textbf{Klindworth-Scharwenka-Saal}, \emph{Veranstaltungsgebäude (K.VSB)}|pwk}.}}}\label{K_L03636-4} Ihrer Frau \textcolor{blue}{Gemahlin}\pwindex{Schnitzler, Olga 17.01.1882 Wien – 13.01.1970 Lugano@\textsc{Schnitzler, Olga} (17.01.1882 Wien – 13.01.1970 Lugano), \emph{Schauspielerin, Sängerin}|pwv}{}\ledrightnote{{$\rightarrow$}\emph{\textcolor{blue}{Olga Schnitzler}}} nichts mehr hören
               kann, hoffentlich dann bald in \textcolor{pink}{Wien}\oindex{Wien@\textbf{Wien}, \emph{A.ADM2}|pw}{}\ledrightnote{\textcolor{pink}{Wien}}!\pend
           
\pstart
           In Treue Ihr ergebener{\[\baselineskip]}\spacefill\mbox{Stefan Zweig}\pend
           \leftskip=0em{}\selectlanguage{ngerman}\endnumbering\briefempfaengerindex{Schnitzler, Arthur@\textsc{Schnitzler, Arthur}!zzzZweig, Stefan@\emph{von Stefan Zweig}!1911-02-211@{21. 2. 1911}|)be}\mylabel{L03636h}  \normalsize

\doendnotes{C}
\bigskip
\vfill

\clearpage

\footnotesize

\lohead{\textsc{register}}

% Definiere theindex-Environment komplett neu ohne reledmac
\makeatletter
\renewenvironment{theindex}{%
  \section*{\indexname}%
  \setlength{\parindent}{0pt}%
  \setlength{\parskip}{0pt plus 0.3pt}%
  \let\item\@idxitem
}{%
  \clearpage
}
\makeatother

\IfFileExists{\jobname-pw.ind}{\input{\jobname-pw.ind}}{}

\end{document}

      