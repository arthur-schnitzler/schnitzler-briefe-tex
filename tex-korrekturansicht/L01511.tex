%% latex-korrekturansicht-vorspann.tex
%% Vorspann für die Korrekturansicht.
%% Lädt die gemeinsame Datei latex-vorspann.tex mit gesetztem Schalter.

\newif\ifkorrekturansicht
\korrekturansichttrue

\input{../tex-inputs/latex-vorspann}


               \section[Max Burckhard: Widmungsexemplar Rat Schrimpf für Arthur Schnitzler, {[}13. 4. 1905?{]}]{ Max Burckhard: Widmungsexemplar Rat Schrimpf für Arthur Schnitzler,
                    {[}13. 4. 1905?{]}}\nopagebreak\mylabel{v}\rehead{ }\normalsize\beginnumbering\briefempfaengerindex{Schnitzler, Arthur@\textsc{Schnitzler, Arthur}!zzzBurckhard, Max Eugen@\emph{von Max Eugen Burckhard}!1905-04-121@{{[}12. 4. 1905?{]}}|(be} \toendnotes[C]{\smallbreak\pagebreak[2]} \Standort{DLA, G:Schnitzler, Arthur (Sammlung Heinrich Schnitzler).}
\physDesc{Widmung am Vorsatzblatt
\newline{}Handschrift: schwarze Tinte, deutsche Kurrent\newline{}Ordnung: bei der Enteignung des Exemplars 1938 von unbekannter Hand mit
                                    Bleistift ergänzte Information: »= 483145-B« }\toendnotes[C]{\smallbreak}\pstart
           \noindent{}{\pb}Herrn Arthur Schnitzler\pend
           \pstart
           in herzlicher Verehrung{\\[\baselineskip]}\spacefill\mbox{D\textsuperscript{r} Burckhard}\pend
           \leftskip=0em{}{\bigskip}\pstart
           \noindent{}\centering{}{\pb}\textcolor{gray}{\textbf{\textcolor{green}{Rat Schrimpf}{}\ledrightnote{\textcolor{green}{Rat Schrimpf. Komödie in fünf Akten}}}}\pend
           \pstart
           \noindent{}\centering{}\textcolor{gray}{\textbf{Komödie in fünf Akten}}{\\}\textcolor{gray}{\textbf{von}}{\\}\textcolor{gray}{\textbf{Max Burckhard}}\pend
           {\bigskip}\pstart
           \noindent{}\centering{}\textcolor{gray}{\textbf{\textcolor{pink}{\so{Berlin}}{}\ledrightnote{\textcolor{pink}{Berlin}}{ }\label{K_L01511_1v}\edtext{1905}{\lemma{\textnormal{\emph{1905}}}\Cendnote{\textnormal{Die Uraufführung fand am
                        13. 4. 1905 statt. Am Vortag vom \emph{\textcolor{green}{Börsenblatt für den deutschen
                           Buchhandel}} als Neuerscheinung gemeldet}}}\label{K_L01511_1h}}}\pend
           \pstart
           \noindent{}\centering{}\textcolor{gray}{\textbf{\textcolor{brown}{\so{S. Fiſcher, Verlag}}{}\ledrightnote{\textcolor{brown}{S. Fischer Verlag}}}}\pend
           \endnumbering\briefempfaengerindex{Schnitzler, Arthur@\textsc{Schnitzler, Arthur}!zzzBurckhard, Max Eugen@\emph{von Max Eugen Burckhard}!1905-04-121@{{[}12. 4. 1905?{]}}|)be}\mylabel{h}  \normalsize

\doendnotes{C}
\bigskip
\vfill

\clearpage

\footnotesize

\lohead{\textsc{register}}

% Definiere theindex-Environment komplett neu ohne reledmac
\makeatletter
\renewenvironment{theindex}{%
  \section*{\indexname}%
  \setlength{\parindent}{0pt}%
  \setlength{\parskip}{0pt plus 0.3pt}%
  \let\item\@idxitem
}{%
  \clearpage
}
\makeatother

\IfFileExists{\jobname-pw.ind}{\input{\jobname-pw.ind}}{}

\end{document}

      