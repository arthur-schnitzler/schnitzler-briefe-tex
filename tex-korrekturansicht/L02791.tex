%% latex-korrekturansicht-vorspann.tex
%% Vorspann für die Korrekturansicht.
%% Lädt die gemeinsame Datei latex-vorspann.tex mit gesetztem Schalter.

\newif\ifkorrekturansicht
\korrekturansichttrue

\input{../tex-inputs/latex-vorspann}


               \section[Paul Goldmann an Arthur Schnitzler, Paul Goldmann an Arthur Schnitzler, 23. 11. {[}1896{]}]{ Paul Goldmann an Arthur Schnitzler, 23. 11. {[}1896{]}}\nopagebreak\mylabel{v}\rehead{ }\normalsize\beginnumbering\briefempfaengerindex{Schnitzler, Arthur@\textsc{Schnitzler, Arthur}!zzzGoldmann, Paul@\emph{von Paul Goldmann}!1896-11-232@{23. 11. {[}1896{]}}|(be} \toendnotes[C]{\smallbreak\pagebreak[2]} \Standort{DLA, A:Schnitzler, HS.NZ85.1.3166.}
\physDesc{Brief, 1 Blatt, 4 Seiten
\newline{}Handschrift: blaue Tinte, deutsche Kurrent
\newline{}Schnitzler: mit Bleistift das Jahr »96« vermerkt }\toendnotes[C]{\smallbreak}\pstart
           \noindent{}{\pb}\textcolor{gray}{\textbf{\textbf{\textcolor{brown}{Frankfurter Zeitung}{}\ledrightnote{\textcolor{brown}{Frankfurter Zeitung}}}}}\pend
           \pstart
           \textcolor{gray}{\textbf{(\textcolor{brown}{\begin{otherlanguage}{french}Gazette de Francfort\end{otherlanguage}}{}\ledrightnote{\textcolor{brown}{Frankfurter Zeitung}}).}}\pend
           \pstart
           \textcolor{gray}{\textbf{\textbf{\begin{otherlanguage}{french}Fondateur M.\end{otherlanguage}{ }\textcolor{blue}{L. Sonnemann}{}\ledrightnote{\textcolor{blue}{Leopold Sonnemann}}.}}}\pend
           \pstart
           \begin{otherlanguage}{french}\textcolor{gray}{\textbf{\textcolor{green}{Journal}{}\ledrightnote{→\textcolor{green}{Frankfurter Zeitung}} politique,
                        financier,}}\end{otherlanguage}\pend
           \pstart
           \begin{otherlanguage}{french}\textcolor{gray}{\textbf{commercial et littéraire.}}\end{otherlanguage}\pend
           \pstart
           \begin{otherlanguage}{french}\textcolor{gray}{\textbf{\textbf{Paraissant trois fois par jour.}}}\end{otherlanguage}\hfill \textsc{\textcolor{pink}{Paris}{}\ledrightnote{\textcolor{pink}{Paris}}}, 23. November.\pend
           \pstart
           \begin{otherlanguage}{french}\textcolor{gray}{\textbf{\textbf{Bureau à \textcolor{pink}{Paris}{}\ledrightnote{\textcolor{pink}{Paris}}}}}\end{otherlanguage}\pend
           \pstart
           \begin{otherlanguage}{french}\textcolor{gray}{\textbf{\textbf{\textcolor{pink}{24. Rue Feydeau}{}\ledrightnote{\textcolor{pink}{rue Feydeau}}.}}}\end{otherlanguage}\pend
           \pstart{}Mein lieber Freund,\pend\pstart
           Zugleich mit der Depeſche an meinen \textcolor{blue}{Onkel}{}\ledrightnote{→\textcolor{blue}{Fedor Mamroth}} ſandte ich am Samſtag
               eine an Dich ab. \strikeout{D} Dein \label{K_L02791-1v}\edtext{Telegramm}{\lemma{\textnormal{\emph{Telegramm}}}\Cendnote{\textnormal{siehe Arthur Schnitzler an Paul Goldmann, 21. 11. 1896}}}\label{K_L02791-1h}, das \strikeout{N\textcolor{gray}{×}} Nachricht verlangte, hat ſich mit dem \label{K_L02791-2v}\edtext{meinen}{\lemma{\textnormal{\emph{meinen}}}\Cendnote{\textnormal{siehe Paul Goldmann an Arthur Schnitzler, 21. 11. 1896}}}\label{K_L02791-2h} gekreuzt. Dies zur Steuer der hiſtoriſchen Wahrheit.\pend
           \pstart
           Und nun \strikeout{t\textcolor{gray}{a}} tauſend Dank für Deine freundſchaftliche Theilnahme und Deine lieben
               Glückwünſche. Aber glaube nur \strikeout{ja} ja nicht, daß ich
               ein \strikeout{Hed} Held geworden bin. Die Sache iſt eigentlich
               eine große Comödie, mit ſehr wenig Gefahr. Und willſt Du {\pb}wiſſen, was Muth iſt? Muth iſt: wenn man \label{K_L02791-3v}\edtext{vorher}{\lemma{\textnormal{\emph{vorher}}}\Cendnote{\textnormal{vor einem Pistolenduell}}}\label{K_L02791-3h} eine halbe Flaſche Rothwein
               getrunken hat. Muth iſt: wenn Leute da ſind und zuſchauen. Muth iſt: wenn man unter
               gar keinen Umſtänden weglaufen darf. Muth iſt: wenn man nicht an die Gefahr denkt.
               Und Muth iſt, vor Allem, wie bekannt: wenn man überzeugt iſt, es wird Einem doch
               nichts paſſiren.\pend
           \pstart
           Ein Gefühl, das »Muth« heißt, gibt es ſicher nicht. Es gibt nur \uline{ein} Gefühl: die Furcht; und der Muth iſt die Negirung dieſes {\pb}Gefühls, oder, um mich franzöſiſch zu citiren:
                  \label{K_L02791-55v}\edtext{\begin{otherlanguage}{french}\textsc{le courage, c’est l’effort qu’on fait contre la peur}\end{otherlanguage}}{\lemma{\textnormal{\emph{le … peur}}}\Cendnote{\textnormal{französisch: Mut ist Aufwand, 
                  den man gegen die Angst aufbringt}}}\label{K_L02791-55h}.\pend
           \pstart
           Das ſind ſo die \substVorne{}\textsuperscript{w\textcolor{gray}{a}hren}{\allowbreak}\substDazwischen{}wahren\substHinten{} inneren Vorgänge geweſen. Alles Äußerliche war Schauſpiel und Schwindel. Ich
               habe nicht auf den \textcolor{blue}{Mann}{}\ledrightnote{→\textcolor{blue}{Lucien Millevoye}}
               gezielt, er aber hat auf mich gezielt, was aber nichts macht, da \strikeout{i\textcolor{gray}{c}h}{ }\strikeout{e\textcolor{gray}{r}} er ein ſchlechter Schütze iſt. Für meine Poſition hier iſt die Sache gut
               geweſen, bei meinem \textcolor{brown}{Blatte}{}\ledrightnote{→\textcolor{brown}{Frankfurter Zeitung}}
               hätte ſie mich beinahe meine Stellung gekoſtet (die großen Demokraten ſind gegen das
               Duell). Schlagen mußte ich mich, um nicht als {\pb}Feigling zu erſcheinen. Aber ich hab’ es ungern gethan. Es iſt eigentlich eine
               Kinderei, und hinterher ſchämt man ſich ſehr darüber, daß man nicht verwundet iſt.
               Die Nacht vorher aber hat man Angſt.\pend
           \pstart
           Hoffentlich kann ich Dir eines Tages mit würdigeren Thaten aufwarten.\pend
           \pstart
           Grüß’ Dich Gott, mein lieber Freund. Schreib’ mir bald!\pend
           \pstart
           Dein treuer {\\[\baselineskip]}\spacefill\mbox{Paul Goldm}\pend
           \leftskip=0em{}\pstart
           \noindent{}Morgen ſende ich ab\substVorne{}\textsuperscript{:.}\substDazwischen{}1.)\substHinten{} Das \textcolor{green}{Manuſkript}{}\ledrightnote{→\textcolor{green}{Amourette. Pièce en trois actes}}
                  der Überſetzung von \textsc{\textcolor{blue}{Thorel}{}\ledrightnote{\textcolor{blue}{Jean Thorel}}} 2.) den \label{K_L02791-41v}\edtext{»\textsc{\textcolor{green}{Mercure}{}\ledrightnote{\textcolor{green}{Mercure de France}}}«}{\lemma{\textnormal{\emph{»Mercure«}}}\Cendnote{\textnormal{Kein zeitnah erschienener Artikel im \emph{\textcolor{green}{Mercure}} bietet sich an, weswegen \textcolor{blue}{Goldmann} das Heft geschickt haben könnte, also dürfte es
                     sich um eine allgemeine Beilage handeln.}}}\label{K_L02791-41h} 3.) »\textsc{\textcolor{green}{Adolphe}{}\ledrightnote{\textcolor{green}{Adolphe. Anecdote trouvée dans les papiers d’un inconnu}}}«. Bitte das \textcolor{green}{Manuſkript}{}\ledrightnote{→\textcolor{green}{Amourette. Pièce en trois actes}}{ }\uline{bald} zurückzuſenden.\pend
           \endnumbering\briefempfaengerindex{Schnitzler, Arthur@\textsc{Schnitzler, Arthur}!zzzGoldmann, Paul@\emph{von Paul Goldmann}!1896-11-232@{23. 11. {[}1896{]}}|)be}\mylabel{h}  \normalsize

\doendnotes{C}
\bigskip
\vfill

\clearpage

\footnotesize

\lohead{\textsc{register}}

% Definiere theindex-Environment komplett neu ohne reledmac
\makeatletter
\renewenvironment{theindex}{%
  \section*{\indexname}%
  \setlength{\parindent}{0pt}%
  \setlength{\parskip}{0pt plus 0.3pt}%
  \let\item\@idxitem
}{%
  \clearpage
}
\makeatother

\IfFileExists{\jobname-pw.ind}{\input{\jobname-pw.ind}}{}

\end{document}

      