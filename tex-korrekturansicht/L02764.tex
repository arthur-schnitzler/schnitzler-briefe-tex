%% latex-korrekturansicht-vorspann.tex
%% Vorspann für die Korrekturansicht.
%% Lädt die gemeinsame Datei latex-vorspann.tex mit gesetztem Schalter.

\newif\ifkorrekturansicht
\korrekturansichttrue

\input{../tex-inputs/latex-vorspann}


               \section[Paul Goldmann an Arthur Schnitzler, Paul Goldmann an Arthur Schnitzler, 16. 1. {[}1896{]}]{ Paul Goldmann an Arthur Schnitzler, 16. 1. {[}1896{]}}\nopagebreak\mylabel{v}\rehead{ }\normalsize\beginnumbering\briefempfaengerindex{Schnitzler, Arthur@\textsc{Schnitzler, Arthur}!zzzGoldmann, Paul@\emph{von Paul Goldmann}!1896-01-162@{16. 1. {[}1896{]}}|(be} \toendnotes[C]{\smallbreak\pagebreak[2]} \Standort{DLA, A:Schnitzler, HS.NZ85.1.3166.}
\physDesc{Brief, 1 Blatt, 3 Seiten
\newline{}Handschrift Paul Goldmann: blaue Tinte, deutsche Kurrent\newline{}Handschrift Jean Thorel: schwarze Tinte, lateinische Kurrent\newline{}Beilage: handschriftlicher Brief: 1 Blatt, 1 Seite 
\newline{}Schnitzler: 1) mit Bleistift das Jahr »96« vermerkt 2) mit rotem Buntstift eine Unterstreichung}\toendnotes[C]{\smallbreak}\pstart
           \noindent{}{\pb}\textcolor{gray}{\textbf{\textbf{\textcolor{brown}{Frankfurter Zeitung}{}\ledrightnote{\textcolor{brown}{Frankfurter Zeitung}}}}}\pend
           \pstart
           \textcolor{gray}{\textbf{(\textcolor{brown}{\begin{otherlanguage}{french}Gazette de Francfort\end{otherlanguage}}{}\ledrightnote{\textcolor{brown}{Frankfurter Zeitung}}).}}\pend
           \pstart
           \textcolor{gray}{\textbf{\textbf{\begin{otherlanguage}{french}Fondateur M.\end{otherlanguage}{ }\textcolor{blue}{L. Sonnemann}{}\ledrightnote{\textcolor{blue}{Leopold Sonnemann}}.}}}\pend
           \pstart
           \begin{otherlanguage}{french}\textcolor{gray}{\textbf{\textcolor{green}{Journal}{}\ledrightnote{→\textcolor{green}{Frankfurter Zeitung}} politique,
                        financier,}}\end{otherlanguage}\pend
           \pstart
           \begin{otherlanguage}{french}\textcolor{gray}{\textbf{commercial et littéraire.}}\end{otherlanguage}\pend
           \pstart
           \begin{otherlanguage}{french}\textcolor{gray}{\textbf{\textbf{Paraissant trois fois par jour.}}}\end{otherlanguage}\pend
           \pstart
           \begin{otherlanguage}{french}\textcolor{gray}{\textbf{\textbf{Bureau à \textcolor{pink}{Paris}{}\ledrightnote{\textcolor{pink}{Paris}}}}}\end{otherlanguage}\hfill \textsc{\textcolor{pink}{Paris}{}\ledrightnote{\textcolor{pink}{Paris}}}, 16. Januar.\pend
           \pstart
           \begin{otherlanguage}{french}\textcolor{gray}{\textbf{\textbf{\textcolor{pink}{24. Rue Feydeau}{}\ledrightnote{\textcolor{pink}{rue Feydeau}}.}}}\end{otherlanguage}\pend
           \pstart\center{}Mein lieber Freund,\pend\pstart
           Ich hatte \textsc{\textcolor{blue}{Thorel}{}\ledrightnote{\textcolor{blue}{Jean Thorel}}} die \textcolor{green}{Frankf Zeit.}{}\ledrightnote{\textcolor{green}{Frankfurter Zeitung}} mit dem \textcolor{green}{Referat}{}\ledrightnote{→\textcolor{green}{Schauspielhaus. [Premiere von Liebelei]}} geſchickt, um ihn zur raſcheren
               Erledigung anzutreiben. Das hat auch gewirkt. Heut
               erhalte ich beifolgenden Brief. Das iſt der erſte kleine Erfolg Deines \textcolor{green}{Stück}{}\ledrightnote{→\textcolor{green}{Liebelei. Schauspiel in drei Akten}}es in \textcolor{pink}{Frankreich}{}\ledrightnote{\textcolor{pink}{Frankreich}}; mögen größere nachkommen! \textsc{\textcolor{blue}{Carré}{}\ledrightnote{\textcolor{blue}{Albert Carré}}} und \textsc{\textcolor{blue}{Thorel}{}\ledrightnote{\textcolor{blue}{Jean Thorel}}} ſind die \textcolor{blue}{Directoren}{}\ledrightnote{→\textcolor{blue}{Albert Carré}{\newline}→\textcolor{blue}{Jean Thorel}} des \textcolor{brown}{Vaudeville}{}\ledrightnote{\textcolor{brown}{Théâtre du Vaudeville}}. Es wäre
               herrlich, wenn an dieſem vornehmen \textcolor{brown}{Theater}{}\ledrightnote{→\textcolor{brown}{Théâtre du Vaudeville}}, wo die \textsc{\textcolor{blue}{Réjane}{}\ledrightnote{\textcolor{blue}{Réjane}}} die Hauperſon iſt, etwas {\pb}zu machen wäre. Ich
               möchte gern \strikeout{üb} die freien \textcolor{brown}{Bühnen}{}\ledrightnote{→\textcolor{brown}{Théâtre de l’Œuvre}{\newline}→\textcolor{brown}{Théâtre Libre}} (\textsc{\textcolor{brown}{Oeuvre}{}\ledrightnote{\textcolor{brown}{Théâtre de l’Œuvre}}}, \textsc{\textcolor{brown}{Théâtre Libre}{}\ledrightnote{\textcolor{brown}{Théâtre Libre}}}) mit ihren Miſt-Aufführungen umgehen. Jedenfalls ſchließe einſtweilen \uline{keinerlei} Überſetzungs-Engagement ab. Könnte ich nicht
               ein paar Exemplare des \textcolor{green}{Stückes}{}\ledrightnote{→\textcolor{green}{Liebelei. Schauspiel in drei Akten}}
               haben?\pend
           \pstart
           Was in \textsc{\textcolor{pink}{Frankfurt}{}\ledrightnote{\textcolor{pink}{Frankfurt am Main}}} vorgegangen iſt, weiß ich nicht. Meine \textcolor{blue}{Mutter}{}\ledrightnote{→\textcolor{blue}{Clementine Goldmann}} die mir ſonſt drei Mal die Woche ſchreibt, um {\pb}mir mitzutheilen, wenn irgend Jemandem dort die Naſe
               weh thut, iſt mir jeden Bericht über Deine Anweſenheit ſchuldig geblieben. Oh, ſie
               können Einen nervös machen, die Herrſchaften von der Familie!\pend
           \pstart
           Hoffenklich biſt Du geſund heimgekehrt.\pend
           \pstart
           Grüß’ Dich Gott, mein lieber Freund!\pend
           \pstart
           Dein treuer {\\[\baselineskip]}\spacefill\mbox{Paul Goldmann}\pend
           \leftskip=0em{}\pstart
           \raggedleft{}{\pb}\textcolor{pink}{12 rue de Milan}{}\ledrightnote{\textcolor{pink}{Rue de Milan}}\pend
           \pstart{}\begin{otherlanguage}{french}Cher Monsieur Goldmann\end{otherlanguage}\pend\pstart
           \label{K_L02764-2v}\edtext{\begin{otherlanguage}{french}Je viens – enfin – de lire »\textcolor{green}{Liebelei}{}\ledrightnote{\textcolor{green}{Liebelei. Schauspiel in drei Akten}}«{[}.{]} C’est un pur bijou\strikeout{x}, d’une délicateſse, d’une fraîcheur, et d’une
                  harmonie parfaite. Il faudra absolument que nous reparlions de cela. Auſsitôt que
                  je vais avoir un instant, je vous demanderai rendez-vous.\end{otherlanguage}}{\lemma{\textnormal{\emph{Je … rendez-vous.}}}\Cendnote{\textnormal{französisch, etwa: XXXX (nachtragen,
                  sobald Transkription finalisiert)}}}\label{K_L02764-2h}\pend
           \pstart
           \begin{otherlanguage}{french}Votre dévoué\end{otherlanguage}{\\[\baselineskip]}\spacefill\mbox{\textcolor{blue}{Jean Thorel}{}\ledrightnote{\textcolor{blue}{Jean Thorel}}}\pend
           \leftskip=0em{}\pstart
           \noindent{}J’écris dès aujourd’hui. – XXXX!\pend
           \endnumbering\briefempfaengerindex{Schnitzler, Arthur@\textsc{Schnitzler, Arthur}!zzzGoldmann, Paul@\emph{von Paul Goldmann}!1896-01-162@{16. 1. {[}1896{]}}|)be}\mylabel{h}\begin{anhang}\end{anhang}\normalsize

\doendnotes{C}
\bigskip
\vfill

\clearpage

\footnotesize

\lohead{\textsc{register}}

% Definiere theindex-Environment komplett neu ohne reledmac
\makeatletter
\renewenvironment{theindex}{%
  \section*{\indexname}%
  \setlength{\parindent}{0pt}%
  \setlength{\parskip}{0pt plus 0.3pt}%
  \let\item\@idxitem
}{%
  \clearpage
}
\makeatother

\IfFileExists{\jobname-pw.ind}{\input{\jobname-pw.ind}}{}

\end{document}

      