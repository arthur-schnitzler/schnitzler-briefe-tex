%% latex-korrekturansicht-vorspann.tex
%% Vorspann für die Korrekturansicht.
%% Lädt die gemeinsame Datei latex-vorspann.tex mit gesetztem Schalter.

\newif\ifkorrekturansicht
\korrekturansichttrue

\input{../tex-inputs/latex-vorspann}


\renewcommand{\erwaehntePersonen}{Personen: Olga Schnitzler}
\renewcommand{\erwaehnteOrte}{Orte: Berlin, Dessauer Straße, Italien, Marienbad, Neapel, Palermo, Petersdom, Pompei, Rom, Taormina}
\renewcommand{\erwaehnteWerke}{}
\section[ Paul Goldmann an Arthur Schnitzler, 1{[}6?.{]} 5. {[}1904{]}]{Paul Goldmann an Arthur Schnitzler, 1{[}6?.{]} 5. {[}1904{]}}
\nopagebreak\mylabel{v}
\rehead{ }\normalsize\beginnumbering\briefempfaengerindex{Schnitzler, Arthur@\textsc{Schnitzler, Arthur}!zzzGoldmann, Paul@\emph{von Paul Goldmann}!1904-05-162@{1{[}6?.{]} 5. {[}1904{]}}|(be}
\toendnotes[C]{\smallbreak\pagebreak[2]}\Standort{DLA, A:Schnitzler, HS.NZ85.1.3174.}
\physDesc{Brief, 1 Blatt, 3 Seiten
\newline{}Handschrift: blaue Tinte, deutsche Kurrent
\newline{}Schnitzler: mit Bleistift das Jahr »{[}1{]}904« vermerkt }\toendnotes[C]{\smallbreak}
\pstart
           \noindent{}\raggedleft{}{\pb}\textcolor{gray}{\textbf{\textcolor{pink}{DESSAUERSTRASSE 19}{}\ledrightnote{\textcolor{pink}{Dessauer Straße}}}}\pend
           
\pstart
           \textcolor{pink}{Berlin}{}\ledrightnote{\textcolor{pink}{Berlin}}, 1\textcolor{gray}{6}. Mai.\pend
           
\pstart{}Mein lieber Freund,\pend
\pstart
           Ich danke Dir und Deiner \textcolor{blue}{Frau}{}\ledrightnote{{$\rightarrow$}\textcolor{blue}{Olga Schnitzler}}
               vielmals für Eure Karten von \label{K_L03443-1v}\edtext{unterwegs}{\lemma{\textnormal{\emph{unterwegs}}}\Cendnote{\textnormal{ Zwischen 1. 5. 1904 und 29. 5. 1904 reisten
                     \textcolor{blue}{Arthur und Olga Schnitzler} nach
                     \textcolor{pink}{Italien}. In \textcolor{pink}{Rom}, wo die von \textcolor{blue}{Goldmann} erwähnte Bildpostkarte abgeschickt worden sein dürfte, waren sie
                  vom 3. 5. 1904 bis
                  zum 8. 5. 1904. In
                  Folge reisten sie weiter nach \textcolor{pink}{Neapel}, \textcolor{pink}{Pompei}, \textcolor{pink}{Palermo} und \textcolor{pink}{Taormina}. }}}\label{K_L03443-1h} und
               freue mich ſehr, daß Eure Reiſe zur Ausführung gekommen iſt. Jetzt im Frühling muß es
               herrlich ſein da unten; und der Anblick des \textcolor{pink}{Petersdom}{}\ledrightnote{\textcolor{pink}{Petersdom}}s auf Deiner Karte, den ich noch nie geſehen habe, hat {\pb}auch in mir \strikeout{g\textcolor{gray}{ro}} eine große Sehnſucht nach \textcolor{pink}{Italien}{}\ledrightnote{\textcolor{pink}{Italien}}
               wachgerufen. Aber ich kann ſie nicht befriedigen. Denn meinen Urlaub muß ich diesmal
               ernſtlich zur Stärkung meiner Geſundheit verwenden; und darum bin ich entſchloſſen,
               nach \textcolor{pink}{Marienbad}{}\ledrightnote{\textcolor{pink}{Marienbad}} zu gehen.\pend
           
\pstart
           \label{K_L03443-11v}\edtext{Grüßt mir alſo \textcolor{pink}{Italien}{}\ledrightnote{\textcolor{pink}{Italien}}}{\lemma{\textnormal{\emph{Grüßt mir alſo Italien}}}\Cendnote{\textnormal{Im Brief
                  vom 26. 5. [1904] schreibt
                     \textcolor{blue}{Goldmann}, dass er mangels Adresse seine
                  Briefe nach \textcolor{pink}{Wien} richte. Ob \textcolor{blue}{Schnitzler} diesen Brief nachgesandt bekam oder erst nach
                  seiner Rückkehr vorfand, ist nicht zu bestimmen.}}}\label{K_L03443-11h} und genießt die ſchönen
               Tage dieſer Reiſe aus vollem Herzen!\pend
           
\pstart
           Neues weiß ich aus {\pb}\textcolor{pink}{Berlin}{}\ledrightnote{\textcolor{pink}{Berlin}} nicht zu melden.\pend
           
\pstart
           Viele herzliche Grüße Dir und Deiner \textcolor{blue}{Frau}{}\ledrightnote{{$\rightarrow$}\textcolor{blue}{Olga Schnitzler}} von {\\[\baselineskip]}Deinem getreuen {\\[\baselineskip]}\spacefill\mbox{Paul Goldmnn}\pend
           \leftskip=0em{}\endnumbering\briefempfaengerindex{Schnitzler, Arthur@\textsc{Schnitzler, Arthur}!zzzGoldmann, Paul@\emph{von Paul Goldmann}!1904-05-162@{1{[}6?.{]} 5. {[}1904{]}}|)be}\mylabel{h}
\begin{anhang}
\end{anhang}\normalsize

\doendnotes{C}
\bigskip
\vfill

\clearpage

\footnotesize

\lohead{\textsc{register}}

% Definiere theindex-Environment komplett neu ohne reledmac
\makeatletter
\renewenvironment{theindex}{%
  \section*{\indexname}%
  \setlength{\parindent}{0pt}%
  \setlength{\parskip}{0pt plus 0.3pt}%
  \let\item\@idxitem
}{%
  \clearpage
}
\makeatother

\IfFileExists{\jobname-pw.ind}{\input{\jobname-pw.ind}}{}

\end{document}

      