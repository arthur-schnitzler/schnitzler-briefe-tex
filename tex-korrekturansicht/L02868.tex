%% latex-korrekturansicht-vorspann.tex
%% Vorspann für die Korrekturansicht.
%% Lädt die gemeinsame Datei latex-vorspann.tex mit gesetztem Schalter.

\newif\ifkorrekturansicht
\korrekturansichttrue

\input{../tex-inputs/latex-vorspann}


               \section[ Paul Goldmann an Arthur Schnitzler, 5. 3. {[}1899{]}]{Paul Goldmann an Arthur Schnitzler, 5. 3. {[}1899{]}}\nopagebreak\mylabel{v}\rehead{ }\normalsize\beginnumbering\briefempfaengerindex{Schnitzler, Arthur@\textsc{Schnitzler, Arthur}!zzzGoldmann, Paul@\emph{von Paul Goldmann}!1899-03-051@{5. 3. {[}1899{]}}|(be} \toendnotes[C]{\smallbreak\pagebreak[2]} \Standort{DLA, A:Schnitzler, HS.NZ85.1.3169.}
\physDesc{Brief, 2 Blätter, 8 Seiten
\newline{}Handschrift: schwarze Tinte, deutsche Kurrent
\newline{}Schnitzler: 1) mit Bleistift das Jahr »99« vermerkt 2) mit rotem Buntstift eine Unterstreichung}\toendnotes[C]{\smallbreak}\pstart
           \raggedleft{}{\pb}\textcolor{pink}{Frankfurt}{}\ledrightnote{\textcolor{pink}{Frankfurt am Main}}, 5. März.\pend
           \pstart\center{}Mein lieber Freund,\pend\pstart
           Ich komme aus \textsc{\textcolor{pink}{Paris}{}\ledrightnote{\textcolor{pink}{Paris}}} zurück und höre hier, daß Du mit Deinen \textcolor{green}{drei Einaktern}{}\ledrightnote{→\textcolor{green}{Der grüne Kakadu – Paracelsus – Die Gefährtin. Drei Einakter}} wieder einen großen und ſchönen \label{K_L02868-1v}\edtext{Erfolg}{\lemma{\textnormal{\emph{Erfolg}}}\Cendnote{\textnormal{Der \textcolor{green}{Einakterzyklus} bestehend aus den Stücken \emph{\textcolor{green}{Der grüne Kakadu}}, \emph{\textcolor{green}{Paracelsus}} und \emph{\textcolor{green}{Die Gefährtin}} wurde am 1. 3. 1899 im \textcolor{pink}{Wien}er \textcolor{pink}{Burgtheater} uraufgeführt.}}}\label{K_L02868-1h} gehabt. Ich freue mich darüber von
               Herzen und beglückwünſche Dich aufs Wärmſte. Geleſen habe ich noch keine Kritik, aber
               ich denke, ich finde die \textcolor{pink}{Wien}{}\ledrightnote{\textcolor{pink}{Wien}}er Blätter morgen hier im \textcolor{brown}{Büreau}{}\ledrightnote{→\textcolor{brown}{Frankfurter Zeitung}}. Den »\label{K_L02868-88v}\edtext{\textcolor{green}{Grünen Kakadu}{}\ledrightnote{\textcolor{green}{Der grüne Kakadu. Groteske in einem Akt}}« las ich}{\lemma{\textnormal{\emph{Grünen Kakadu« las ich}}}\Cendnote{\textnormal{\emph{\textcolor{green}{Der grüne Kakadu}} wurde zuerst in der \emph{\textcolor{green}{Neuen Deutschen Rundschau}} (Jg. 10, H. 3,
                        März 1899, S. 282–308) gedruckt. Da \textcolor{blue}{Goldmann} im \emph{\textcolor{green}{Tagebuch}}{ }\textcolor{blue}{Schnitzler}s am 17. 1. 1899 zuletzt
                  als sich in \textcolor{pink}{Wien} aufhaltend erwähnt wird,
                  dürfte er zu diesem Zeitpunkt Kopie des Manuskripts/Typoskripts erhalten haben. (vgl. Arthur Schnitzler an Georg Brandes, 24. 2. 1899) }}}\label{K_L02868-88h} noch auf der Reiſe von \textcolor{pink}{Wien}{}\ledrightnote{\textcolor{pink}{Wien}} nach \textcolor{pink}{Frankfurt}{}\ledrightnote{\textcolor{pink}{Frankfurt am Main}}. Ein
               vortreffliches \textcolor{green}{Stück}{}\ledrightnote{→\textcolor{green}{Der grüne Kakadu. Groteske in einem Akt}}. Da ich
               aber etwas ganz Vollendetes erwartete, hat es mich doch auch ein wenig enttäuſcht.
               Ich erhoffte Revolution und \textcolor{pink}{Baſtillen}{}\ledrightnote{\textcolor{pink}{Bastille}}ſturm, fand aber
               zuletzt doch nur wieder eine Liebesgeſchichte mit einem Theatermädel. Anderſeits iſt
               es, glaube ich, in der Ausführung eines Deiner beſten Stücke und bedeutet doch \strikeout{\textcolor{gray}{einen}} auch einen gewaltigen Schritt nach vorwärts \strikeout{von
                     \textcolor{gray}{dem} alten T} von Deinem alten Ton und Deinen alten
               Stoffen zu irgend etwas Neuem, das ſehr ſchön werden wird.\pend
           \pstart
           {\pb}Mein lieber Freund, ich komme alſo nicht nach \textcolor{pink}{Wien}{}\ledrightnote{\textcolor{pink}{Wien}}. Es war ein quälendes wochenlanges Ringen und
               ein ſchwerer Entſchluß. Wie alle Entſchlüſſe im Augenblick nachdem man ſie gefaßt
               hat, erſcheint mir auch dieſer jetzt recht tadelnswerth. Aber das war zu
               erwarten.\pend
           \pstart
           Als ich von \textcolor{pink}{Wien}{}\ledrightnote{\textcolor{pink}{Wien}} nach \textcolor{pink}{Frankfurt}{}\ledrightnote{\textcolor{pink}{Frankfurt am Main}} kam und ſich in \textcolor{pink}{Frankfurt}{}\ledrightnote{\textcolor{pink}{Frankfurt am Main}} die \textcolor{pink}{Wien}{}\ledrightnote{\textcolor{pink}{Wien}}er Eindrücke zu klären
               begannen, ſchien es mir zunächſt unmöglich, mich wieder in den \textcolor{pink}{Wien}{}\ledrightnote{\textcolor{pink}{Wien}}er Journalismus zu fügen, nachdem ich Jahre lang unter
               größeren und freieren Verhältniſſen gelebt. Und nachdem ich Jahre lang in der »\textcolor{brown}{Frankfurter Zeitung}{}\ledrightnote{\textcolor{brown}{Frankfurter Zeitung}}« gearbeitet, wo ich
               ungehindert meine Anſichten entfalten konnte und eigentlich nur mein Gewiſſen um Rath
               zu fragen brauchte, erſchien es mir unmöglich, mich in die \label{K_L02868-2v}\edtext{»\textcolor{brown}{Neue Freie Preſſe}{}\ledrightnote{\textcolor{brown}{Neue Freie Presse}}«
               { }\strikeout{\textcolor{gray}{einfügen}} hineinzufinden}{\lemma{\textnormal{\emph{»Neue … hineinzufinden}}}\Cendnote{\textnormal{als Redakteur für ausländische Politik in
                  \textcolor{pink}{Wien}}}}\label{K_L02868-2h} mit ihrer Rückſichtennehmerei und Cliquen-Wirthſchaft, welche
               verlangt, daß man Dieſes beſchönigt und Jenes verſchweigt und daß \label{K_L02868-3v}\edtext{man 
               \textsc{\textcolor{blue}{Herzl}{}\ledrightnote{\textcolor{blue}{Theodor Herzl}}s} durchgefallene Stücke}{\lemma{\textnormal{\emph{man … Stücke}}}\Cendnote{\textnormal{\textcolor{blue}{Theodor Herzl} verantwortete das Feuilleton der
                     \emph{\textcolor{brown}{Neuen Freien Presse}}. \textcolor{blue}{Goldmann}
                  behauptet hier, dass die Berichterstattung über dessen Stücke ungerechtfertigt
                  positiv ausfiel.}}}\label{K_L02868-3h} als die {\pb}Meiſterwerke
               eines genialen Schriftſtellers dem Publicum anpreiſt. \strikeout{M} Mir grauſte ferner vor dem Arbeitsgebiet, das mir zugewieſen werden
               ſollte, der ausländiſchen Politik, während doch mein ganzes Beſtreben dahin geht,
               möglichſt aus der Politik heraus in die Literatur oder wenigſtens in den mit
               Literatur ſich beſchäftigenden Journalismus zu kommen. Und mir grauſte vor der
               Rieſen-Arbeit, die man mir in \textcolor{pink}{Wien}{}\ledrightnote{\textcolor{pink}{Wien}} aufbürden
               wollte, vor der Stellung des Redaktions-\label{K_L02868-4v}\edtext{Culis}{\lemma{\textnormal{\emph{Culis}}}\Cendnote{\textnormal{Tagelöhner, Verrichter minderer Dienste}}}\label{K_L02868-4h}, der alle Laſten trägt,
               vor der rückſichtsloſen Ausbeutung der Sklavenhalter in \textcolor{pink}{Wien}{}\ledrightnote{\textcolor{pink}{Wien}} (während die Sklavenhalter in \textcolor{pink}{Frankfurt}{}\ledrightnote{\textcolor{pink}{Frankfurt am Main}} doch ein wenig \strikeout{rü\textcolor{gray}{c}} rückſichtsvoller ausbeuten). Es iſt wahr, als Compenſation für das Alles hatte
               ich Euch in \textcolor{pink}{Wien}{}\ledrightnote{\textcolor{pink}{Wien}}. \strikeout{E} Gewiß, die ſchönſte aller Compenſationen. Aber \strikeout{an} die Hauptſache im Leben iſt die Arbeit, die man thut. Davon geht alle
               Sonne, alles Behagen aus. Und wenn man in ſeinen Wirkungskreis nicht hineinpaßt, ſo
               iſt das Daſein in ſeinem Wichtigſten verfehlt und man wird tiefunglücklich, trotz
               allen Verkehrs {\pb}mit ſehr lieben Menſchen. Beſſer
               eine Arbeit, die Einem wenigſtens einigermaßen zuſagt, und keine lieben Menſchen,
               als, wenn man ſchon einmal wählen muß, liebe Menſchen und eine widerwärtige Arbeit.
                  \introOben{}Hier muß man Stoiker ſein und darf ſeinem weichen Herzen nicht
                  nachgeben.\introOben{} Auch kommt dazu, daß Jeder von Euch jetzt ſein eigenes Leben lebt
               und daß ich von \strikeout{\textcolor{gray}{K}} Keinem, ſelbſt vom nächſten Freunde nicht, beanſpruchen darf, er ſolle mir
               mein Leben leben helfen. Während dieſer Zeit wurde ich in \textcolor{pink}{Frankfurt}{}\ledrightnote{\textcolor{pink}{Frankfurt am Main}} ſehr zum Bleiben gedrängt. Ich ſah, daß \strikeout{es} man in der \textcolor{brown}{Redaktion}{}\ledrightnote{→\textcolor{brown}{Frankfurter Zeitung}} mich achtete und ſchätzte, merkte auch, daß das
               Publicum auf mich hielt. Und ich dachte mir, daß es eigentlich Wahnſinn wäre, zehn
               Jahre Arbeit, die ich in das \textcolor{green}{Blatt}{}\ledrightnote{→\textcolor{green}{Frankfurter Zeitung}} hier geſteckt, wegzuwerfen\strikeout{,} und nach
                  \textcolor{pink}{Wien}{}\ledrightnote{\textcolor{pink}{Wien}} zu gehen, wo kein Menſch mich kennt, wo
               nicht einmal Ihr mehr etwas von meinen Leiſtungen wißt, wo ich von Anfang anfangen
                  \strikeout{müßte} und mir Schritt für Schritt, unter Gott weiß
               welchen Kämpfen, {\pb}eine Stellung erſt ſchaffen müßte,
               die ich hier bereits beſitze. Zukunft endlich (wenn ich überhaupt Zukunft habe) gibt
               es doch nur in \textcolor{pink}{Deutſchland}{}\ledrightnote{\textcolor{pink}{Deutschland}}, nicht in \textcolor{pink}{Öſterreich}{}\ledrightnote{\textcolor{pink}{Österreich}}. Dazu kam noch Allerlei, was die
               Familie angeht.\pend
           \pstart
           Immerhin wollte ich mit der »\textcolor{brown}{Neuen Freien Preſſe}{}\ledrightnote{\textcolor{brown}{Neue Freie Presse}}«
               nicht gleich \strikeout{\textcolor{gray}{ab}} abbrechen und \strikeout{ſp\textcolor{gray}{a}} ſpann die Sache weiter. Wir waren verblieben (die \label{K_L02868-56v}\edtext{\textcolor{blue}{Chefredacteur}{}\ledrightnote{→\textcolor{blue}{Eduard Bacher}}s}{\lemma{\textnormal{\emph{Chefredacteurs}}}\Cendnote{\textnormal{Seit dem Frühjahr 1879 war \textcolor{blue}{Eduard Bacher}
                  Chefredakteur der \emph{\textcolor{brown}{Neuen Freien Presse}}. Es ist
                  nicht geklärt, mit wem \textcolor{blue}{Goldmann} zusätzlich
                  Kontakt hatte.}}}\label{K_L02868-56h} und ich), daß zur Beſiegelung meines Eintritts in die \textcolor{brown}{Redaktion}{}\ledrightnote{→\textcolor{brown}{Neue Freie Presse}} Vertragsbriefe
               ausgetauſcht werden ſollten. Ich ſandte einen früheren Brief von \textsc{\textcolor{blue}{Bacher}{}\ledrightnote{\textcolor{blue}{Eduard Bacher}}}, den dieſer behufs Aufſetzung des Vertrages gewünſcht hatte, an ihn zurück und
               bat um Überſendung des Vertragsbriefes. \label{K_L02868-5v}\edtext{Wenige Tage darauf ſtarb \textsc{\textcolor{blue}{Schiff}{}\ledrightnote{\textcolor{blue}{Emil Schiff}}}}{\lemma{\textnormal{\emph{Wenige … Schiff}}}\Cendnote{\textnormal{\textcolor{blue}{Emil Schiff} verstarb am 23. 1. 1899.}}}\label{K_L02868-5h}, der \textcolor{pink}{Berlin}{}\ledrightnote{\textcolor{pink}{Berlin}}er Correſpondent der \textcolor{brown}{N.
                  Fr. Pr.}{}\ledrightnote{\textcolor{brown}{Neue Freie Presse}}; ich bekam von der \textcolor{brown}{Redaktion}{}\ledrightnote{→\textcolor{brown}{Neue Freie Presse}} ein Telegramm mit der Aufforderung, den \textcolor{pink}{Berlin}{}\ledrightnote{\textcolor{pink}{Berlin}}er \textcolor{blue}{Correſpondenten}{}\ledrightnote{→\textcolor{blue}{?? [Berliner Korrespondent der Frankfurter Zeitung 1899]}} der \textcolor{brown}{Frankfurter Zeitung}{}\ledrightnote{\textcolor{brown}{Frankfurter Zeitung}}{ }\strikeout{als} als Nachfolger für \textsc{\textcolor{blue}{Schiff}{}\ledrightnote{\textcolor{blue}{Emil Schiff}}} zu engagiren. {\pb}Ich telegraphirte \introOben{}und ſchrieb\introOben{} zurück, das ginge aus dieſem und jenem Grunde
               nicht, bot mich aber zugleich als Nachfolger \textsc{\textcolor{blue}{Schiff}{}\ledrightnote{\textcolor{blue}{Emil Schiff}}s} in \textcolor{pink}{Berlin}{}\ledrightnote{\textcolor{pink}{Berlin}} an. In der That wäre mir die Stellung in \textcolor{pink}{Berlin}{}\ledrightnote{\textcolor{pink}{Berlin}} lieber geweſen, \strikeout{als die} als die in \textcolor{pink}{Wien}{}\ledrightnote{\textcolor{pink}{Wien}}. Ich hätte von
                  \textcolor{pink}{Berlin}{}\ledrightnote{\textcolor{pink}{Berlin}} aus über Theater und Kunſt geſchrieben
               und wäre auch der \textcolor{pink}{Wien}{}\ledrightnote{\textcolor{pink}{Wien}}er Redaktions-Wirthſchaft in
                  \textcolor{pink}{Berlin}{}\ledrightnote{\textcolor{pink}{Berlin}} ſehr \strikeout{entrü\textcolor{gray}{c}kt} entrückt geweſen. Meiner Anſicht nach hätte
               die \textcolor{brown}{N. Fr. Pr.}{}\ledrightnote{\textcolor{brown}{Neue Freie Presse}} in mir einen recht geeigneten
               Correſpondenten für \textcolor{pink}{Berlin}{}\ledrightnote{\textcolor{pink}{Berlin}} gehabt. Seit jenem
               Augenblick nun (Ende Januar) habe ich \strikeout{vo} von der \textcolor{brown}{N. Fr.
                  Pr.}{}\ledrightnote{\textcolor{brown}{Neue Freie Presse}} kein Wort mehr gehört. Mehr als vier Wochen vergingen, \substVorne{}\textsuperscript{ohne{ }\textcolor{gray}{dieſe ich}}{\allowbreak}\substDazwischen{}und ich bekam\substHinten{} nicht nur keinen Beſcheid über mein Anerbieten bezüglich des \strikeout{\textcolor{pink}{Wien}{}\ledrightnote{\textcolor{pink}{Wien}}er Poſt\textcolor{gray}{e}}{ }\textcolor{pink}{Berlin}{}\ledrightnote{\textcolor{pink}{Berlin}}er Poſtens, ſondern auch nicht einmal den
               Vertragsbrief, den die Leute mir ſofort hätten ſchicken müſſen. Ich wartete und
               wartete (dies der Grund, weshalb ich Dir ſo lange nicht geſchrieben), hielt es
               natürlich für unter {\pb}meiner Würde zu drängen, und
               nachdem bis zum Ende Februar immer noch weder Beſcheid
               noch Vertrag aus \textcolor{pink}{Wien}{}\ledrightnote{\textcolor{pink}{Wien}} eingetroffen waren,
               unterzeichnete ich einen neuen Vertrag mit der \textcolor{brown}{Frankfurter Zeitung}{}\ledrightnote{\textcolor{brown}{Frankfurter Zeitung}}. Geſtern aber habe ich
               ein Telegramm von \textsc{\textcolor{blue}{Bacher}{}\ledrightnote{\textcolor{blue}{Eduard Bacher}}} erhalten, der ſehr erzürnt darüber iſt, daß ich nicht am 1. März, wie mündlich\strikeout{,}
               beſprochen, in der \textcolor{brown}{Redaktion}{}\ledrightnote{→\textcolor{brown}{Neue Freie Presse}} in
                  \textcolor{pink}{Wien}{}\ledrightnote{\textcolor{pink}{Wien}} angetreten bin! Ich habe ihm den
               Sachverhalt auseinandergeſetzt, und nach dieſem Telegramm wird mir das Verhalten der
               Leute noch räthſelhafter als zuvor.\pend
           \pstart
           In \textcolor{pink}{Frankfurt}{}\ledrightnote{\textcolor{pink}{Frankfurt am Main}} trete ich in die \textcolor{brown}{Feuilleton-Redaktion}{}\ledrightnote{→\textcolor{brown}{Frankfurter Zeitung}} ein, als \label{K_L02868-11v}\edtext{\textsc{Adlatus}}{\lemma{\textnormal{\emph{Adlatus}}}\Cendnote{\textnormal{Gehilfe}}}\label{K_L02868-11h} von \textsc{\textcolor{blue}{Dr. Mamroth}{}\ledrightnote{\textcolor{blue}{Fedor Mamroth}}}, und ſoll zu Reiſe-Miſſionen verwendet werden (im Herbſt nach \textcolor{pink}{Rußland}{}\ledrightnote{\textcolor{pink}{Russland}}, im nächſten Frühjahr zur \textcolor{pink}{Pariſ}{}\ledrightnote{\textcolor{pink}{Paris}}er Weltausſtellung, zu großen \textsc{\begin{otherlanguage}{french}Premièren\end{otherlanguage}} in \textcolor{pink}{Deutſchland}{}\ledrightnote{\textcolor{pink}{Deutschland}} und zu ähnlichen Anläſſen).
                  \strikeout{S\textcolor{gray}{o}{ }\textcolor{gray}{×}\-\textcolor{gray}{×}} So finde ich mich denn, nach ſo viel Wirrſal und Schwanken, \strikeout{\textcolor{gray}{×}\-\textcolor{gray}{×}h\textcolor{gray}{×}\-\textcolor{gray}{×}} auf einmal in der kleinen \textcolor{pink}{Stadt}{}\ledrightnote{→\textcolor{pink}{Frankfurt am Main}}, einſam, ohne Freunde, unter läſtigen Familien-{\pb}Verhältniſſen. \strikeout{\textcolor{gray}{Fe}} Fern von der großen Welt\substVorne{}\textsuperscript{!}\substDazwischen{}.\substHinten{} Und mir iſt, als ſei eine Thür hinter mir ins Schloß gefallen.\pend
           \pstart
           Habe ich recht gehandelt oder falſch? Wird \strikeout{\textcolor{gray}{×}\-\textcolor{gray}{×}s} dieſe neue Exiſtenz zu
               ertragen ſein? Ich weiß es nicht.\pend
           \pstart
           Bitte, zeig’ dem \textsc{\textcolor{blue}{Richard}{}\ledrightnote{\textcolor{blue}{Richard Beer-Hofmann}}} dieſen Brief (wenn es ihn intereſſirt). Sonſt aber betrachte das Mitgetheilte
               als vertraulich; und wenn man \strikeout{d} Dich fragt, warum ich
               nicht zur \textcolor{brown}{N. Fr. Pr.}{}\ledrightnote{\textcolor{brown}{Neue Freie Presse}} gekommen bin, ſo \strikeout{ſ\textcolor{gray}{pric}h} ſage, daß die Verhandlungen
               ſich in die Länge gezogen haben und daß die Sache noch unentſchieden iſt. Ich möchte
               mir nämlich, wenn es ginge, ein{[}e{]} Hinterthür für die Zukunft
               offen laſſen.\pend
           \pstart
           Bitte, ſchreib’ mir bald, liebſter Freund, und vor Allem: \label{K_L02868-45v}\edtext{komm’ demnächſt nach \textcolor{pink}{Frankfurt}{}\ledrightnote{\textcolor{pink}{Frankfurt am Main}}}{\lemma{\textnormal{\emph{komm’ … Frankfurt}}}\Cendnote{\textnormal{\textcolor{blue}{Schnitzler} war das nächste Mal von 19. 9. 1899 bis 23. 9. 1899 in \textcolor{pink}{Frankfurt am Main}.}}}\label{K_L02868-45h}!\pend
           \pstart
           Viele treue Grüße! {\\[\baselineskip]}Dein {\\[\baselineskip]}\spacefill\mbox{Paul Goldmann}\pend
           \leftskip=0em{}\pstart
           \noindent{}Adreſſe: \textsc{\textcolor{pink}{Hotel Central}{}\ledrightnote{\textcolor{pink}{Central-Hotel}}}, \textcolor{pink}{Frankfurt \textsuperscript{a}/M.}{}\ledrightnote{\textcolor{pink}{Frankfurt am Main}}\pend
           \pstart
           Grüße an Deine \textcolor{blue}{Freundin}{}\ledrightnote{→\textcolor{blue}{Marie Reinhard}}!\pend
           \endnumbering\briefempfaengerindex{Schnitzler, Arthur@\textsc{Schnitzler, Arthur}!zzzGoldmann, Paul@\emph{von Paul Goldmann}!1899-03-051@{5. 3. {[}1899{]}}|)be}\mylabel{h}\begin{anhang}\end{anhang}\normalsize

\doendnotes{C}
\bigskip
\vfill

\clearpage

\footnotesize

\lohead{\textsc{register}}

% Definiere theindex-Environment komplett neu ohne reledmac
\makeatletter
\renewenvironment{theindex}{%
  \section*{\indexname}%
  \setlength{\parindent}{0pt}%
  \setlength{\parskip}{0pt plus 0.3pt}%
  \let\item\@idxitem
}{%
  \clearpage
}
\makeatother

\IfFileExists{\jobname-pw.ind}{\input{\jobname-pw.ind}}{}

\end{document}

      