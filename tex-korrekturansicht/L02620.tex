%% latex-korrekturansicht-vorspann.tex
%% Vorspann für die Korrekturansicht.
%% Lädt die gemeinsame Datei latex-vorspann.tex mit gesetztem Schalter.

\newif\ifkorrekturansicht
\korrekturansichttrue

\input{../tex-inputs/latex-vorspann}


               \section[Paul Goldmann an Arthur Schnitzler, 18. 11. 1894]{ Paul Goldmann an Arthur Schnitzler, 18. 11. 1894}\nopagebreak\mylabel{v}\rehead{ }\normalsize\beginnumbering\briefempfaengerindex{Schnitzler, Arthur@\textsc{Schnitzler, Arthur}!zzzGoldmann, Paul@\emph{von Paul Goldmann}!1894-11-181@{18. 11. 1894}|(be} \toendnotes[C]{\smallbreak\pagebreak[2]} \Standort{DLA, A:Schnitzler, HS.NZ85.1.3164.}
\physDesc{Brief, 2 Blätter, 8 Seiten
\newline{}Handschrift: schwarze Tinte, deutsche Kurrent
\newline{}Schnitzler: 1) mit Bleistift auf dem ersten Blatt die Jahreszahl
                                       »94« vermerkt 2) mit rotem Buntstift sieben Unterstreichungen}\toendnotes[C]{\smallbreak}\pstart
           \noindent{}{\pb}\textcolor{gray}{\textbf{\textcolor{brown}{Frankfurter Zeitung}{}\ledrightnote{\textcolor{brown}{Frankfurter Zeitung}}.}}\hfill \textsc{\textcolor{pink}{Paris}{}\ledrightnote{\textcolor{pink}{Paris}}}, 18. November.\pend
           \pstart
           \textcolor{gray}{\textbf{(\textcolor{brown}{Gazette de
                  Francfort}{}\ledrightnote{\textcolor{brown}{Frankfurter Zeitung}}.)}}\pend
           \pstart
           \textcolor{gray}{\textbf{\begin{otherlanguage}{french}Fondateur\end{otherlanguage}{ }\textbf{M. \textcolor{blue}{L.
                  Sonnemann}{}\ledrightnote{\textcolor{blue}{Leopold Sonnemann}}}.}}\pend
           \pstart
           \textcolor{gray}{\textbf{\begin{otherlanguage}{french}Journal politique,
                        financier,\end{otherlanguage}}}\pend
           \pstart
           \textcolor{gray}{\textbf{\begin{otherlanguage}{french}commercial et
                     littéraire.\end{otherlanguage}}}\pend
           \pstart
           \textcolor{gray}{\textbf{\begin{otherlanguage}{french}\textbf{Paraissant trois fois
                           par jour}\end{otherlanguage}}}.\pend
           \pstart
           \textcolor{gray}{\textbf{–}}\pend
           \pstart
           \textcolor{gray}{\textbf{\begin{otherlanguage}{french}\textbf{Bureaux à \textcolor{pink}{Paris}{}\ledrightnote{\textcolor{pink}{Paris}}:}\end{otherlanguage}}}\pend
           \pstart
           \textcolor{gray}{\textbf{\begin{otherlanguage}{french}\textcolor{pink}{\textbf{24. Rue Feydeau}}{}\ledrightnote{\textcolor{pink}{rue Feydeau}}.\end{otherlanguage}}}\pend
           \pstart{}Mein lieber Freund,\pend\pstart
           Ich will Dir täglich ſchreiben und bringe die Energie dafür nicht zuſammen. Nicht
               einmal dafür! Ich bin in einem ſchlimmen Gemütfszuſtande. Ich ſuche nach einem
               Lebensziel und finde es nicht – ſuche mich ſelbſt zu beſchränken, zu erkennen, zu
               ordnen und kann es nicht – und nach kurzen Anläufen falle ich in Zeitvergeudung,
               Außenleben und Wirrniß zurück. Dabei werde ich alle paar Tage daran erinnert, daß ich
               dreißig Jahre bin, nichts geleiſtet habe, zurückbleibe hinter allen Andern. Es iſt
               ein zerſtörendes Gefühl, und doch finde ich die {\pb}Kraft nicht zum Arbeiten. Die Zeit hätte ich jetzt, – alſo es gibt keine
               Entſchuldigung mehr. Das hindert mich an Allem, ſelbſt am Briefeſchreiben. Du
               begreifſt mich gewiß.\pend
           \pstart
           Ich raffe mich heut ein wenig zuſammen; denn ich möchte gar ſo gern hören, wie es mit
               Deinem \textcolor{green}{Stücke}{}\ledrightnote{→\textcolor{green}{Liebelei. Schauspiel in drei Akten}} weitergeht. Was Du
               mir über Deine \label{K_L02620-1v}\edtext{erſte Unterredung mit
                  \textcolor{blue}{B.}{}\ledrightnote{\textcolor{blue}{Max Eugen Burckhard}}}{\lemma{\textnormal{\emph{erſte Unterredung mit
                  B.}}}\Cendnote{\textnormal{siehe A. S.: \emph{Tagebuch}, 5. 11. 1894}}}\label{K_L02620-1h}
               geſchrieben, erſcheint mir ganz und gar nicht ungünſtig. Daß es nicht ſo glatt gehen
               würde, war ſelbſtverſtändlich. Dabei geht es doch noch relativ glatt. Wenn man in
               einem Theater den Director für ſich hat, ſo iſt das, denke ich, Chance genug. Das {\pb}Übrige iſt Zopf und \textsc{chinoiserie}. Dafür ſind wir ja im guten Lande \textcolor{pink}{Öſterreich}{}\ledrightnote{\textcolor{pink}{Österreich}}. Wüßteſt Du nur, was hier die jungen Leute dulden müſſen, ehe ſie
               aufgeführt werden. An die \textsc{\textcolor{brown}{Comédie
                     Française}{}\ledrightnote{\textcolor{brown}{Comédie-Française}}} kommt überhaupt keiner heran, wenn ihn nicht ein \textcolor{brown}{Akademiker}{}\ledrightnote{→\textcolor{brown}{Académie Française}} oder ein großer
               Komödiant protegirt, und \strikeout{\textsc{Henr}} der alte \textsc{\textcolor{blue}{Henri Becque}{}\ledrightnote{\textcolor{blue}{Henri Becque}}} ſelbſt hat ſeinerzeit die Aufführung von »\textsc{\textcolor{green}{La Parisienne}{}\ledrightnote{\textcolor{green}{La Parisienne}}}« durch ein
               Machtwort des \textcolor{blue}{Miniſters}{}\ledrightnote{→\textcolor{blue}{Léon Bourgeois}}
               erzwingen müſſen. Es gibt keinen Erfolg, zu dem man nicht über Hintertreppen ſteigen
               müßte, beſonders beim Theater. Thut mir nur leid, daß ich nicht gerade jetzt um Dich
               bin, um {\pb}mit Dir über all’ die Trottelhaftigkeiten
               zu lachen, die Dir vorausſichtlich werden geſagt oder angethan werden, und vielleicht
               auch um Dir ein Paar unangenehme Wege zu erſparen. Übrigens meinſt Du es ja ſelbſt
               ironiſch, und das iſt das Beſte. Bitte, ſchreib’ mir nur raſch, wieweit die Sache
               iſt. Und möchteſt Du es nicht doch zugleich in \label{K_L02620-3v}\edtext{\textcolor{pink}{Berlin}{}\ledrightnote{\textcolor{pink}{Berlin}} einreichen}{\lemma{\textnormal{\emph{Berlin einreichen}}}\Cendnote{\textnormal{XXXX}}}\label{K_L02620-3h}?\pend
           \pstart
           Geſtern habe ich die \label{K_L02620-2v}\edtext{Fortſetzung von »\textcolor{green}{Sterben}{}\ledrightnote{\textcolor{green}{Sterben. Novelle}}«}{\lemma{\textnormal{\emph{Fortſetzung von »Sterben«}}}\Cendnote{\textnormal{Der zweite Teil (von
                  drei) erschien Anfang November in der \emph{\textcolor{green}{Neuen deutschen Rundschau}} (H. 11,
                  S. 1073–1101).}}}\label{K_L02620-2h} geleſen. Es iſt dumm, daß man es mit
               Zwiſchenräumen \strikeout{von} von einem Monat leſen muß. Ich bin
               mir über den Eindruck infolgedeſſen jetzt weniger {\pb}klar, als am Anfang. Ich weiß nur, daß ich im Einzelnen Entzückendes und Großes
               finde. Auch iſt der Styl köſtlich in ſeiner Einfachheit, mit all’ den Tiefen
               darunter. \strikeout{Ein \textcolor{gray}{×}} Hier und da iſt es mir aber doch zu einfach. Zum Beiſpiel:
                  \textsc{\textcolor{pink}{Salzburg}{}\ledrightnote{\textcolor{pink}{Salzburg}}}, ich meine das
               Landſchaftliche und Äußerliche, iſt meiner Empfindung nach um eine \textsc{Nuance} zu blaß gerathen. Alles in Allem ein reifes und
               ernſtes Werk. Aber, wie geſagt, ich muß es als Buch im Zuſammenhange leſen. Mir ahnt
               nur, daß ich es ſchön finden werde, {\pb}aber ich habe
               noch kein klares Bewußtſein davon. Dieſe verfluchten Fortſetzungen! Eine kleine
               Äußerlichkeit: bei der Buchausgabe \label{K_mets_Goldmann_94-partII-5v}\edtext{ſtreiche}{\lemma{\textnormal{\emph{ſtreiche}}}\Cendnote{\textnormal{\textcolor{blue}{Schnitzler} veränderte die Stelle für die Buchausgabe nicht.}}}\label{K_mets_Goldmann_94-partII-5h} auf
               Seite 1077 in der 20ten Zeile \uline{von unten} hinter
               »Einwohner« die Worte »der Stadt« weg, es iſt zu viel »Stadt« in dem Abſatz.\pend
           \pstart
           Wann kriege, ich nun wohl das \textcolor{green}{Stück}{}\ledrightnote{→\textcolor{green}{Liebelei. Schauspiel in drei Akten}} zu leſen?\pend
           \pstart
           Mein \strikeout{\textcolor{gray}{Onk}}{ }\textcolor{blue}{Onkel}{}\ledrightnote{→\textcolor{blue}{Fedor Mamroth}} hat mich
               vor vier Wochen nach Deiner Adreſſe gefragt, um Dir Bücher zu ſchicken. Da ich aber
               wieder einmal mit ihm grolle, habe ich nicht geantwortet. Hätteſt Du nicht irgend
               einen Vorwand ihm zu \label{K_mets_Goldmann_94-partII-44v}\edtext{ſchreiben}{\lemma{\textnormal{\emph{ſchreiben}}}\Cendnote{\textnormal{siehe Arthur Schnitzler an Fedor Mamroth, 7. 12. 1894}}}\label{K_mets_Goldmann_94-partII-44h}{ }\strikeout{un}, damit er zugleich {\pb}Deine \textcolor{pink}{Adreſſe}{}\ledrightnote{→\textcolor{pink}{Frankgasse}} erführe?\pend
           \pstart
           Die »\textcolor{brown}{Zeit}{}\ledrightnote{\textcolor{brown}{Die Zeit. Wiener Wochenschrift}}« gefällt mir ganz ausnehmend. Das iſt ein
                  \textcolor{green}{Blatt}{}\ledrightnote{→\textcolor{green}{Die Zeit. Wiener Wochenschrift}}, durchaus nach meinem
               Sinn. \textsc{\textcolor{blue}{Kanner}{}\ledrightnote{\textcolor{blue}{Heinrich Kanner}}} übertrifft
               ſich ſelbſt, \textsc{\textcolor{blue}{Bahr}{}\ledrightnote{\textcolor{blue}{Hermann Bahr}}} iſt
               vorzüglich als Theaterkritiker – ich meine die Art, wie er ſchreibt. Seine \label{K_L02620-11v}\edtext{\textcolor{green}{Kritik}{}\ledrightnote{→\textcolor{green}{Kunst und Leben. [Burgtheater. Minna von Barnhelm]}} über die \textsc{\textcolor{blue}{Schratt}{}\ledrightnote{\textcolor{blue}{Katharina Schratt}}}}{\lemma{\textnormal{\emph{Kritik über die Schratt}}}\Cendnote{\textnormal{\textcolor{blue}{Bahr} schrieb in einer Nachtkritik über die
                  Neueinstudierung von \emph{\textcolor{green}{Minna von Barnhelm}} am \emph{\textcolor{brown}{Burgtheater}} (erstmals
                     22. 10. 1894) unter anderem: »Die Francisca, ein
                     unverwüstliches Geschöpf der \textcolor{blue}{\so{Hartmann}}, gibt Frau \textcolor{blue}{\so{Schratt}}. Man heißt ja jetzt
                     unpatriotisch, wenn man für Frau \textcolor{blue}{Schratt}
                     nicht immer schwärmt, als ob das gleich weiß Gott was für eine Beleidigung
                     wäre. Nun, ich meine, Kritik darf auch vor dem Throne nicht schweigen, den der
                     Verwöhnten Schmeichler bauen. Sie ist keine Francisca. Wenn sie schmollen will,
                     keift sie, statt neckisch wird sie zänkisch und das niedliche
                     ›Frauenzimmerchen‹ bleibt die eben zu majestätische Dame schuldig.«
                        (\textcolor{blue}{H. B. [=Hermann Bahr]}: \emph{\textcolor{green}{Kunst und Leben}}. In: \emph{\textcolor{green}{Die Zeit}}, Bd. 1, H. 4,
                        27. 10. 1894, S. 61.)}}}\label{K_L02620-11h}, ſeine
                  \label{K_L02620-44v}\edtext{\textcolor{green}{Polemik}{}\ledrightnote{→\textcolor{green}{Kunst und Leben. [Claque am Raimundtheater]}} mit \textsc{\textcolor{blue}{Mueller-Guttenbrunn}{}\ledrightnote{\textcolor{blue}{Adam Müller-Guttenbrunn}}}}{\lemma{\textnormal{\emph{Polemik mit Mueller-Guttenbrunn}}}\Cendnote{\textnormal{\emph{\textcolor{green}{Die Zeit}} enthält mehrere
                  Seitenhiebe gegen den Leiter des \emph{\textcolor{brown}{Raimund-Theater}}s, \textcolor{blue}{Adam
                     Müller-Guttenbrunn}. \textcolor{blue}{Goldmann} dürfte
                  sich auf folgende ungezeichnete Meldung beziehen: »In der ›\textcolor{green}{Wiener Allgemeinen}‹ hat neulich auch Herr \textcolor{blue}{\so{Müller-Guttenbrunn}} gespochen
                     und mit der Sicherheit, die er stets seinen Behauptungen gibt, betheuert, dass
                     das \textcolor{brown}{Raimund-Theater} keine Claque hat. Da
                     sollte Herr \textcolor{blue}{\so{Salten}}, von dem die hübsche Idee dieser Antikritik ist, jetzt
                     doch auch Herrn \textcolor{blue}{\so{Wessely}} vernehmen, den sehr intelligenten und erfahrenen Chef
                     der Claque. Er kann seine Adresse von jedem Schauspieler erfahren und ihn
                     übrigens meistens in der Kanzlei des \textcolor{brown}{Raimundtheater}s treffen, wo er sich nach den Proben, die er mit Eifer
                     hört, seine Instructionen holt.« (\textcolor{blue}{[O. V.=Hermann Bahr]}: \emph{\textcolor{green}{Kunst und
                        Leben}}. In: \emph{\textcolor{green}{Die Zeit}}, Bd. 1,
                     H. 6, 10. 11. 1894, S. 94.)}}}\label{K_L02620-44h} und
               deſſen \label{K_L02620-v}\edtext{\textcolor{blue}{Regiſſeur}{}\ledrightnote{→\textcolor{blue}{Karl Langkammer}}}{\lemma{\textnormal{\emph{Regiſſeur}}}\Cendnote{\textnormal{Hier dürfte er sich auf die lobende und positive Nachtkritik
                        (\textcolor{blue}{H. B. [=Hermann Bahr]}: \emph{\textcolor{green}{Kunst und Leben}}. In: \emph{\textcolor{green}{Die Zeit}}, Bd. 1, H. 7,
                        17. 11. 1894, S. 108) zur Uraufführung
                  von \emph{\textcolor{green}{Die Eder-Mitzi. Wiener Volksstück in vier
                     Akten}} am 14. 11. 1894 am \emph{\textcolor{brown}{Raimund-Theater}}
                  beziehen. Ob \textcolor{blue}{Goldmann} das Lob ironisch las,
                  ist nicht festzustellen.}}}\label{K_L02620-h} haben mich ſehr ergötzt. Aber wenn er über Kunſt
               pontificirt, iſt er mir unerträglich. Der \label{K_L02620-23v}\edtext{\textcolor{green}{Artikel über
                  Dekadenz}{}\ledrightnote{→\textcolor{green}{Décadence}}}{\lemma{\textnormal{\emph{Artikel über
                  Dekadenz}}}\Cendnote{\textnormal{\textcolor{blue}{Hermann Bahr}: \emph{\textcolor{green}{Décadence}}. In: \emph{\textcolor{green}{Die Zeit}}, Bd. 1,
                     H. 6, 10. 11. 1894, S. 87–89.}}}\label{K_L02620-23h} im
               vorletzten \textcolor{green}{Heft}{}\ledrightnote{→\textcolor{green}{Die Zeit. Wiener Wochenschrift}} iſt vorzüglich
               gemacht, ſtrotzt aber von falſchen Angaben und Urtheilen. Die \textsc{\textcolor{blue}{Stefan George}{}\ledrightnote{\textcolor{blue}{Stefan George}}}, \textsc{\textcolor{blue}{Hermann Bang}{}\ledrightnote{\textcolor{blue}{Herman Bang}}}{ }\textsc{etc.}, die er citirt, kenne ich als \textsc{\label{K_mets_Goldmann_94-partII-88v}\edtext{Faiseurs}{\lemma{\textnormal{\emph{Faiseurs}}}\Cendnote{\textnormal{französisch:
                     Blender}}}\label{K_mets_Goldmann_94-partII-88h}}{\pb}{ }\strikeout{mit} ohne jede tiefere Begabung. Wie gefällt Dir das
                  \textcolor{green}{Blatt}{}\ledrightnote{→\textcolor{green}{Die Zeit. Wiener Wochenschrift}}? Und wir gehts damit?
               Wird es ſich halten?\pend
           \pstart
           Fräulein \textsc{\textcolor{blue}{Sandrock}{}\ledrightnote{\textcolor{blue}{Adele Sandrock}}} hat
               mir einen langen, ſchönen und lieben Brief geſchrieben. Bitte ſag’ ihr einſtweilen,
               wie ſehr ich mich darüber gefreut habe, und daß ich nur nach einer Stimmung ſuche, um
               nach Gebühr zu antworten. Ich will ihr nicht aus dem erſtbeſten Wochentage heraus
               ſchreiben.\pend
           \pstart
           Und bitte, ſchreib’ mir bald und viel – von Dir, von ſonſt Allem, von \textcolor{pink}{Wien}{}\ledrightnote{\textcolor{pink}{Wien}} und wieder von Dir. Was ſchreibſt und lieſt Du?
               Was ſoll mit den \textsc{30 fr. 30 ct} geſchehen, die Du
               bei mit gut haſt? Viele treue Grüße! Dein\pend
           \pstart \spacefill\mbox{Paul Goldmann}\pend{}\endnumbering\briefempfaengerindex{Schnitzler, Arthur@\textsc{Schnitzler, Arthur}!zzzGoldmann, Paul@\emph{von Paul Goldmann}!1894-11-181@{18. 11. 1894}|)be}\mylabel{h}\begin{anhang}\end{anhang}\normalsize

\doendnotes{C}
\bigskip
\vfill

\clearpage

\footnotesize

\lohead{\textsc{register}}

% Definiere theindex-Environment komplett neu ohne reledmac
\makeatletter
\renewenvironment{theindex}{%
  \section*{\indexname}%
  \setlength{\parindent}{0pt}%
  \setlength{\parskip}{0pt plus 0.3pt}%
  \let\item\@idxitem
}{%
  \clearpage
}
\makeatother

\IfFileExists{\jobname-pw.ind}{\input{\jobname-pw.ind}}{}

\end{document}

      