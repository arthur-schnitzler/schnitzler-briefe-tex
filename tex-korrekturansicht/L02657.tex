%% latex-korrekturansicht-vorspann.tex
%% Vorspann für die Korrekturansicht.
%% Lädt die gemeinsame Datei latex-vorspann.tex mit gesetztem Schalter.

\newif\ifkorrekturansicht
\korrekturansichttrue

\input{../tex-inputs/latex-vorspann}


               \section[Paul Goldmann an Arthur Schnitzler, {[}6.?{]} 2. 1903]{ Paul Goldmann an Arthur Schnitzler, {[}6.?{]} 2. 1903}\nopagebreak\mylabel{v}\rehead{ }\normalsize\beginnumbering\briefempfaengerindex{Schnitzler, Arthur@\textsc{Schnitzler, Arthur}!zzzGoldmann, Paul@\emph{von Paul Goldmann}!1903-02-061@{{[}6.?{]} 2. 1903}|(be} \toendnotes[C]{\smallbreak\pagebreak[2]} \Standort{DLA, A:Schnitzler, HS.NZ85.1.3173.}
\physDesc{Telegramm
\newline{}maschinell\newline{}Versand: »\noindent{}\textcolor{gray}{\textbf{Aufgenommen durch}}{ }\textcolor{gray}{\textbf{\textit{\textcolor{blue}{J. Khom}}}}« \newline{}Ordnung: beschnitten }\toendnotes[C]{\smallbreak}\pstart
           \centering{}{\pb}de \textcolor{pink}{berlin}{}\ledrightnote{\textcolor{pink}{Berlin}}
                  25111 17 6{ }9-35,–m=\pend
           \pstart
           \textcolor{green}{tageblatt}{}\ledrightnote{\textcolor{green}{Berliner Tageblatt}} bringt \label{K_L02657-1v}\edtext{\textcolor{green}{ankuendigung}{}\ledrightnote{→\textcolor{green}{[Im Deutschen Theater soll »Der Schleier der Beatrice«]}}}{\lemma{\textnormal{\emph{ankuendigung}}}\Cendnote{\textnormal{»\textcolor{green}{Im \textcolor{brown}{\so{Deutschen Theater}} soll ›\textcolor{green}{\so{Der Schleier der Beatrice}}‹} von \textcolor{blue}{\so{Arthur Schnitzler}} noch in diesem Monat zur Aufführung kommen. Die
                  Hauptrollen spielen \textcolor{blue}{Irene Triesch} und \textcolor{blue}{Bassermann}. Im \so{Mai}
                  gastiert das \textcolor{brown}{Deutsche Theater} im \textcolor{pink}{\so{Wien}}\so{er}{ }\textcolor{brown}{Karl-Theater}.«
                     In: \emph{\textcolor{green}{Berliner Tageblatt}}, Jg. 32,
                     Nr. 65, 5. 2. 1904, Abendblatt,
                  S. 3.}}}\label{K_L02657-1h} dass \textcolor{green}{beatrice}{}\ledrightnote{\textcolor{green}{Der Schleier der Beatrice. Schauspiel in fünf Akten}} noch
               diesen monat aufgefuehrt wird.\pend
           \pstart gruss = \spacefill\mbox{goldmann .+}\pend{}\endnumbering\briefempfaengerindex{Schnitzler, Arthur@\textsc{Schnitzler, Arthur}!zzzGoldmann, Paul@\emph{von Paul Goldmann}!1903-02-061@{{[}6.?{]} 2. 1903}|)be}\mylabel{h}  \normalsize

\doendnotes{C}
\bigskip
\vfill

\clearpage

\footnotesize

\lohead{\textsc{register}}

% Definiere theindex-Environment komplett neu ohne reledmac
\makeatletter
\renewenvironment{theindex}{%
  \section*{\indexname}%
  \setlength{\parindent}{0pt}%
  \setlength{\parskip}{0pt plus 0.3pt}%
  \let\item\@idxitem
}{%
  \clearpage
}
\makeatother

\IfFileExists{\jobname-pw.ind}{\input{\jobname-pw.ind}}{}

\end{document}

      