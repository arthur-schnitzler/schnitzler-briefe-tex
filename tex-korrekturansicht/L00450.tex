%% latex-korrekturansicht-vorspann.tex
%% Vorspann für die Korrekturansicht.
%% Lädt die gemeinsame Datei latex-vorspann.tex mit gesetztem Schalter.

\newif\ifkorrekturansicht
\korrekturansichttrue

\input{../tex-inputs/latex-vorspann}


               \section[Arthur Schnitzler an Richard Beer-Hofmann, 7. 6. 1895]{ Arthur Schnitzler an Richard Beer-Hofmann, 7. 6. 1895}\nopagebreak\mylabel{v}\rehead{ }\normalsize\beginnumbering\briefempfaengerindex{Beer-Hofmann, Richard@\textsc{Beer-Hofmann, Richard}!zzzSchnitzler, Arthur@\emph{von Arthur Schnitzler}!1895-06-071@{7. 6. 1895}|(be} \toendnotes[C]{\smallbreak\pagebreak[2]} \Standort{YCGL, MSS 31.}
\physDesc{Brief, 1 Blatt (Briefpapier mit Trauerrand), 5 Seiten, Umschlag
\newline{}Handschrift: schwarze Tinte, deutsche Kurrent\newline{}Versand: 1) Stempel: »\nobreak{}\oindex{IX., Alsergrund@\textbf{IX., Alsergrund}, \emph{Bezirk (A.BZK)}|pwk}Wien 9/3, 7. 6. 95, 5–6 N\nobreak{}«.  2) Stempel: »\nobreak{}\oindex{Caslau@\textbf{Caslau}, \emph{Besiedelter Ort (A.BSO)}|pwk}Časlau, 8 6 95\nobreak{}«. }\buchAbdrucke{\weitereDrucke{Arthur Schnitzler, Richard Beer-Hofmann: \emph{Briefwechsel 1891–1931}. Hg. Konstanze Fliedl. Wien, Zürich: \emph{Europaverlag} 1992, S. 73.} }\toendnotes[C]{\smallbreak}\pstart{}Herrn n. a. Lieutenant\pend{}\pstart{}{\pb}\textsc{Dr. Richard Beer-Hofmann}\pend{}\pstart{}im k. k. Landw.-Inf-Regiment\pend{}\pstart{}\textsc{\textcolor{pink}{Caslau}{}\ledrightnote{\textcolor{pink}{Caslau}} Nr 12}.\pend{}\pstart{}\textsc{\textcolor{pink}{Böhmen}{}\ledrightnote{\textcolor{pink}{Böhmen}}}\pend{}{\bigskip}\pstart
           \noindent{}{\pb}Lieber Richard, warum ſchreiben Sie mir denn gar nicht?\pend
           \pstart
           Mit \textcolor{blue}{Fels}{}\ledrightnote{\textcolor{blue}{Friedrich Michael Fels}} gehn einige Dinge vor, die ausführlich zu
               erzählen zu langweilig wäre. Er muſs fort, in die \textcolor{pink}{Schweiz}{}\ledrightnote{\textcolor{pink}{Schweiz}} – \textcolor{pink}{deutſche}{}\ledrightnote{\textcolor{pink}{Deutschland}} Militärgeſchichte. Ich
               erlaube mir ihm in Ihrem Namen wie in dem \textcolor{blue}{Hugo}{}\ledrightnote{\textcolor{blue}{Hugo von Hofmannsthal}}s
               (mit dem ich ſchon geſprochen – er war ein paar Tage da, wieder Catarrh – abſolut
               unbedenklich) wie in dem meinen je zehn Gulden zu geben. Geht nicht anders.\pend
           \pstart
           {\pb}– Warum ſchreiben Sie mir eigentlich nicht? –\pend
           \pstart
           \textcolor{blue}{\textsc{Fischer}}{}\ledrightnote{\textcolor{blue}{Samuel Fischer}} hat mir geſchrieben, mir einen Contract auf 5 Jahre für alle meine Werke,
               angeblich denſelben wie \textcolor{blue}{\textsc{Hauptmann}}{}\ledrightnote{\textcolor{blue}{Gerhart Hauptmann}}{ }\textsc{etc} überſandt (Unterſchr\textcolor{gray}{ieb} noch
               nicht.) Will die \textcolor{green}{\textsc{Kleine Komödie}}{}\ledrightnote{\textcolor{green}{Die kleine Komödie}} (die ihm ſehr gut gefällt was mir unheimlich iſt) in der \textcolor{green}{\textsc{Collect. Fischer}}{}\ledrightnote{\textcolor{green}{Collection Fischer}} mit {\pb}\textsc{\textcolor{blue}{Zasche}{}\ledrightnote{\textcolor{blue}{Theodor Zasche}}}’ſchen Illuſtr. bringen, will sie aber zuerſt in der \textcolor{green}{\textsc{Freien Bühne}}{}\ledrightnote{\textcolor{green}{Neue Deutsche Rundschau}} (Auguſtheft, ohne Illuſtr.) veröffentlichen. Wie denken Sie? –\pend
           \pstart
           An \textcolor{blue}{N.}{}\ledrightnote{\textcolor{blue}{Gabor Nobl}} hab ich die 20 fl. geſandt; ich ſprach ihn
               zufällig am ſelben Tag, und er wollte ſie nicht nehmen, was ich aber {\pb}heftig abwehrte. – Die betreffende \textcolor{blue}{Dame}{}\ledrightnote{→\textcolor{blue}{?? [Sexualpartnerin von Richard Beer-Hofmann]}} – nun ſind Sie ja aus allen Sorgen – hat
               natürlich doch \textsc{Lues} gehabt – ſecundäre; auch im Mund. Wenn
               wir alſo bei dem \textcolor{blue}{Hugo}{}\ledrightnote{\textcolor{blue}{Hugo von Hofmannsthal}}’ſchen \textcolor{green}{Märchen}{}\ledrightnote{\textcolor{green}{Das Märchen der 672. Nacht}} bleiben, ka{\geminationn} man ſagen:
               Alles ist eingetroffen, nur – unberufen – hat das \label{K_L00450_1v}\edtext{Pferd}{\lemma{\textnormal{\emph{Pferd}}}\Cendnote{\textnormal{Der
                  Protagonist von \emph{\textcolor{green}{Das Märchen der 672. Nacht}}
                  stirbt am Hufschlag eines Pferdes.}}}\label{K_L00450_1h} nicht ausgeſchlagen. – Daſs Sie {\pb}mir nicht ſchreiben, ist durchaus nicht ſchön. –\pend
           \pstart Herzlich der Ihre \spacefill\mbox{Arthur}\pend{}\pstart
           Haben Sie die \textcolor{green}{Kritik}{}\ledrightnote{→\textcolor{green}{Sterben}}{ }\textcolor{blue}{\textsc{Sokals}}{}\ledrightnote{\textcolor{blue}{Clemens Sokal}} über \textcolor{green}{Sterben}{}\ledrightnote{\textcolor{green}{Sterben. Novelle}} geleſen? Merkwürdig von \label{K_L00450_2v}\edtext{\textsc{\textcolor{blue}{Osten}{}\ledrightnote{\textcolor{blue}{Heinrich Osten}}-\textcolor{blue}{Wengraf}{}\ledrightnote{\textcolor{blue}{Edmund Wengraf}}}ſcher Animosität}{\lemma{\textnormal{\emph{Osten-Wengrafſcher Animosität}}}\Cendnote{\textnormal{die beiden
                  Herausgeber der \emph{\textcolor{green}{Neuen Revue}}, in der am
                     29. 5. 1895 die \textcolor{green}{Rezension} erschienen war.}}}\label{K_L00450_2h} durchtränkt.\pend
           \pstart
           Ich ſchreib jetzt an einem \textcolor{green}{Stück}{}\ledrightnote{→\textcolor{green}{Freiwild. Schauspiel in 3 Akten}}. –\pend
           \endnumbering\briefempfaengerindex{Beer-Hofmann, Richard@\textsc{Beer-Hofmann, Richard}!zzzSchnitzler, Arthur@\emph{von Arthur Schnitzler}!1895-06-071@{7. 6. 1895}|)be}\mylabel{h}  \normalsize

\doendnotes{C}
\bigskip
\vfill

\clearpage

\footnotesize

\lohead{\textsc{register}}

% Definiere theindex-Environment komplett neu ohne reledmac
\makeatletter
\renewenvironment{theindex}{%
  \section*{\indexname}%
  \setlength{\parindent}{0pt}%
  \setlength{\parskip}{0pt plus 0.3pt}%
  \let\item\@idxitem
}{%
  \clearpage
}
\makeatother

\IfFileExists{\jobname-pw.ind}{\input{\jobname-pw.ind}}{}

\end{document}

      