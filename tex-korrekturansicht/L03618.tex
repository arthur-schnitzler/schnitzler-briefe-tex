%% latex-korrekturansicht-vorspann.tex
%% Vorspann für die Korrekturansicht.
%% Lädt die gemeinsame Datei latex-vorspann.tex mit gesetztem Schalter.

\newif\ifkorrekturansicht
\korrekturansichttrue

\input{../tex-inputs/latex-vorspann}


\renewcommand{\erwaehntePersonen}{Personen: Karl Emil Franzos, Johann Schnitzler}
\renewcommand{\erwaehnteOrte}{Orte: Berlin, Kaiserin-Augusta-Straße 71}
\renewcommand{\erwaehnteWerke}{Werke: Amerika, Deutsche Dichtung, Erbschaft, Jugend in Wien, Mein Freund Ypsilon. Aus den Papieren eines Arztes}
\section[Arthur Schnitzler an Karl Emil Franzos, 29. 4. 1888]{Arthur Schnitzler an Karl Emil Franzos, 29. 4. 1888}
\nopagebreak\mylabel{v}
\rehead{ }\normalsize\beginnumbering\briefempfaengerindex{Franzos, Karl Emil@\textsc{Franzos, Karl Emil}!zzzSchnitzler, Arthur@\emph{von Arthur Schnitzler}!1888-04-291@{29. 4. 1888}|(be}
\toendnotes[C]{\smallbreak\pagebreak[2]}\Standort{Wienbibliothek im Rathaus, H.I.N.-60194.}
\physDesc{Brief, 1 Blatt, 2 Seiten, 639 Zeichen
\newline{}Handschrift: schwarze Tinte, deutsche Kurrent}
\buchAbdrucke{\weitereDrucke{Arthur Schnitzler: \emph{Briefe 1875–1912}. Hg. Therese Nickl und Heinrich Schnitzler. Frankfurt am Main: \emph{S. Fischer} 1981, S. 28.} }\toendnotes[C]{\smallbreak}
\pstart
           \raggedleft{}{\pb}\textsc{\textcolor{pink}{Berlin}{}\ledrightnote{\textcolor{pink}{Berlin}}}{ }29. 4. 88\pend
           
\pstart{}Hochgeehrter Herr!\pend
\pstart
           Ich nehme mir die Freiheit, Ihnen zwei \textcolor{green}{\label{K_L03618-1v}\edtext{Erzählungen}{\lemma{\textnormal{\emph{Erzählungen}}}\Cendnote{\textnormal{Von den erhaltenen Prosaarbeiten, die in diesem Zeitraum
                     entstanden, kommen \emph{\textcolor{green}{Erbschaft}}, \emph{\textcolor{green}{Mein Freund Ypsilon. Aus den Papieren eines
                        Arztes}} und \emph{\textcolor{green}{Amerika}} in Frage, vgl. A. S.: \emph{Tagebuch}, 19. 10. 1887 und \emph{\textcolor{green}{Jugend in Wien}} (\textcolor{blue}{Arthur Schnitzler}: \emph{\textcolor{green}{Jugend in Wien. Eine Autobiographie}}. Mit einem
                        Nachwort von Friedrich Torberg. Wien,
                        München, Zürich,
                        New York: \emph{S. Fischer}{ }1968, S. 320).}}}\label{K_L03618-1h}}{}\ledrightnote{{$\rightarrow$}\textcolor{green}{Amerika}{\newline}{$\rightarrow$}\textcolor{green}{Mein Freund Ypsilon. Aus den Papieren eines Arztes}{\newline}{$\rightarrow$}\textcolor{green}{Erbschaft}} zu überſenden, von denen ich mir ſelbst kaum einbilden will, daſs ſie für Ihre
                  »\textsc{\textcolor{green}{Dtsch. Dichtung}{}\ledrightnote{\textcolor{green}{Deutsche Dichtung}}}« der Vorzüge genug beſitzen. Jedenfalls aber wäre mir ein Urtheil von Ihnen
               höchſt erwünscht, um das Sie hiemit zwar unbeſcheiden aber herzlichſt gebeten ſind.
               Ich unterlieſs \label{K_L03618-2v}\edtext{perſönlich}{\lemma{\textnormal{\emph{perſönlich}}}\Cendnote{\textnormal{Am 15. 4. 1888 und am 28. 4. 1888 war \textcolor{blue}{Schnitzler} bei \textcolor{blue}{Franzos} auf Besuch in der \textcolor{pink}{Kaiserin-Augusta-Straße 71}. Die Einladung zum Souper am Vortag dieses
                  Briefes dürfte eine Folge des Empfehlungsschreibens von \textcolor{blue}{Johann Schnitzler} (Johann Schnitzler an Karl Emil Franzos, 4. 4. 1888) gewesen sein.}}}\label{K_L03618-2h} mit Ihnen über
               diese Sache zu reden, da ich in dem Augenblicke dieser {\pb}Bitte am liebſten ein ganz und gar unbeka{\geminationn}ter, gewiſs aber nicht der gut empfohlene und ſo
               liebenswürdig aufgeno{\geminationm}ene »Sohn meines \textcolor{blue}{Vaters}{}\ledrightnote{{$\rightarrow$}\textcolor{blue}{Johann Schnitzler}}« ſein möchte.\pend
           
\pstart
           Mit beſondrer Hochachtung Ihr{\\[\baselineskip]}ergebener{\\[\baselineskip]}\spacefill\mbox{Dr Arthur Schnitzler}\pend
           \leftskip=0em{}\endnumbering\briefempfaengerindex{Franzos, Karl Emil@\textsc{Franzos, Karl Emil}!zzzSchnitzler, Arthur@\emph{von Arthur Schnitzler}!1888-04-291@{29. 4. 1888}|)be}\mylabel{h}
\begin{anhang}
\end{anhang}\normalsize

\doendnotes{C}
\bigskip
\vfill

\clearpage

\footnotesize

\lohead{\textsc{register}}

% Definiere theindex-Environment komplett neu ohne reledmac
\makeatletter
\renewenvironment{theindex}{%
  \section*{\indexname}%
  \setlength{\parindent}{0pt}%
  \setlength{\parskip}{0pt plus 0.3pt}%
  \let\item\@idxitem
}{%
  \clearpage
}
\makeatother

\IfFileExists{\jobname-pw.ind}{\input{\jobname-pw.ind}}{}

\end{document}

      