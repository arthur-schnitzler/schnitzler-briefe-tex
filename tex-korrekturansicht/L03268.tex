%% latex-korrekturansicht-vorspann.tex
%% Vorspann für die Korrekturansicht.
%% Lädt die gemeinsame Datei latex-vorspann.tex mit gesetztem Schalter.

\newif\ifkorrekturansicht
\korrekturansichttrue

\input{../tex-inputs/latex-vorspann}


\renewcommand{\erwaehnteOrte}{Orte: Bad Ischl, Hauptstraße, Pressbaum, Wien}
\renewcommand{\erwaehnteWerke}{}
\section[ Felix Salten an Arthur Schnitzler, 13. 7. 1897]{Felix Salten an Arthur Schnitzler, 13. 7. 1897}
\nopagebreak\mylabel{v}
\rehead{ }\normalsize\beginnumbering\briefempfaengerindex{Schnitzler, Arthur@\textsc{Schnitzler, Arthur}!zzzSalten, Felix@\emph{von Felix Salten}!1897-07-133@{13. 7. 1897}|(be}
\toendnotes[C]{\smallbreak\pagebreak[2]}\Standort{CUL, Schnitzler, B 89, A 2.}
\physDesc{Karte, 218 Zeichen
\newline{}Handschrift: Bleistift, lateinische Kurrent
\newline{}Ordnung: mit Bleistift von unbekannter Hand nummeriert: »91« }\toendnotes[C]{\smallbreak}
\pstart
           \raggedleft{}{\pb}\textcolor{pink}{Wien}{}\ledrightnote{\textcolor{pink}{Wien}}, 13. Juli 97\pend
           
\pstart
           Mir geht’s leidlich, Der Arbeit auch. Am 25. treffen
               Sie mich wahrscheinlich noch \label{K_L03268-1v}\edtext{\textcolor{pink}{hier}{}\ledrightnote{{$\rightarrow$}\textcolor{pink}{Wien}}}{\lemma{\textnormal{\emph{hier}}}\Cendnote{\textnormal{\textcolor{blue}{Schnitzler} kehrte am 25. 7. 1897 nach \textcolor{pink}{Wien} zurück. \textcolor{blue}{Salten} traf er nachweislich am 30. 7. 1897 wieder.}}}\label{K_L03268-1h}. Sollte ich nicht da
               sein, bin ich einstweilen in \textcolor{pink}{Pressbaum}{}\ledrightnote{\textcolor{pink}{Pressbaum}}, wo mich
               Briefe in der \textcolor{pink}{Hauptstraße 7}{}\ledrightnote{\textcolor{pink}{Hauptstraße}} erreichen.\pend
           \pstart Herzlich \spacefill\mbox{Salten}\pend{}\endnumbering\briefempfaengerindex{Schnitzler, Arthur@\textsc{Schnitzler, Arthur}!zzzSalten, Felix@\emph{von Felix Salten}!1897-07-133@{13. 7. 1897}|)be}\mylabel{h}  \normalsize

\doendnotes{C}
\bigskip
\vfill

\clearpage

\footnotesize

\lohead{\textsc{register}}

% Definiere theindex-Environment komplett neu ohne reledmac
\makeatletter
\renewenvironment{theindex}{%
  \section*{\indexname}%
  \setlength{\parindent}{0pt}%
  \setlength{\parskip}{0pt plus 0.3pt}%
  \let\item\@idxitem
}{%
  \clearpage
}
\makeatother

\IfFileExists{\jobname-pw.ind}{\input{\jobname-pw.ind}}{}

\end{document}

      