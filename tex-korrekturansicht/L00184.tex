%% latex-korrekturansicht-vorspann.tex
%% Vorspann für die Korrekturansicht.
%% Lädt die gemeinsame Datei latex-vorspann.tex mit gesetztem Schalter.

\newif\ifkorrekturansicht
\korrekturansichttrue

\input{../tex-inputs/latex-vorspann}


               \section[Karl Kraus an Arthur Schnitzler, 4. 3. 1893]{ Karl Kraus an Arthur Schnitzler, 4. 3. 1893}\nopagebreak\mylabel{v}\rehead{ }\normalsize\beginnumbering\briefempfaengerindex{Schnitzler, Arthur@\textsc{Schnitzler, Arthur}!zzzKraus, Karl@\emph{von Karl Kraus}!1893-03-041@{4. 3. 1893}|(be} \toendnotes[C]{\smallbreak\pagebreak[2]} \Standort{CUL, Schnitzler, B 55.}
\physDesc{Postkarte
\newline{}Handschrift: schwarze Tinte, deutsche Kurrent\newline{}Versand: 1) Stempel: »\nobreak{}\oindex{Berlin@\textbf{Berlin}, \emph{https://www.geonames.org/ontologyP.PPLC}|pwk}Berlin S. O. 26, 4. 3. 93, 7–8 N\nobreak{}«.  2) Stempel: »\nobreak{}\oindex{Opatija@\textbf{Opatija}, \emph{http://www.geonames.org/ontologyP.PPLA2}|pwk}Abbazia, 6/3 \textcolor{gray}{93}\nobreak{}«. }\buchAbdrucke{\weitereDrucke{1) \emph{Karl Kraus und Arthur Schnitzler. Eine Dokumentation.} Hg. Reinhard Urbach. In: \emph{Literatur und Kritik}, Bd. 49, Oktober 1970, S. 515–516.} \weitereDrucke{2) Hermann Bahr, Arthur Schnitzler: \emph{Briefwechsel, Aufzeichnungen, Dokumente
                                (1891–1931)}. Hg. Kurt Ifkovits und Martin Anton Müller. Göttingen: \emph{Wallstein} 2018, S. 34.} }\toendnotes[C]{\smallbreak}\pstart{}{\pb}Herrn\pend{}\pstart{}D\textsuperscript{r.} Arthur Schnitzler\pend{}\pstart{}\textcolor{pink}{Abbazia}{}\ledrightnote{\textcolor{pink}{Hotel Guarnero}} / (Curort)\pend{}\pstart{}\textcolor{pink}{Quisisina}{}\ledrightnote{\textcolor{pink}{Pension Quisisana}}\pend{}{\bigskip}\pstart
           {\pb}\textcolor{pink}{Berlin}{}\ledrightnote{\textcolor{pink}{Berlin}}, 4/3 93.\pend
           \pstart{}Lieber kleiner Doctor!\pend\pstart
           Ich dank Ihnen ſehr für Ihr liebes Schreiben. Mitte der nächſten Woche bin ich
                    wieder in \textcolor{pink}{Wien}{}\ledrightnote{\textcolor{pink}{Wien}} (über \textcolor{pink}{Leipzig}{}\ledrightnote{\textcolor{pink}{Leipzig}} u \textcolor{pink}{Prag}{}\ledrightnote{\textcolor{pink}{Prag}}).\pend
           \pstart
           Ich vergaß damals \textcolor{blue}{\uline{Loris}}{}\ledrightnote{\textcolor{blue}{Hugo von Hofmannsthal}} zu grüßen. Bitte, tragen Sie das nach, wenn Sie ihm ſchreiben. \label{K_L00184_1v}\edtext{\textcolor{blue}{Duße}{}\ledrightnote{\textcolor{blue}{Eleonora Duse}}}{\lemma{\textnormal{\emph{Duße}}}\Cendnote{\textnormal{Warum der Austausch über die
                        Schauspielerin zu diesem Zeitpunkt stattfindet, ist unklar. \textcolor{blue}{Schnitzler} hatte \textcolor{blue}{Eleonora Duse} bereits zehn Monate zuvor gesehen: »17.5. \textcolor{pink}{Theaterausstellung}? \textcolor{blue}{Sardou}: \textcolor{green}{Fernande}. (\textcolor{blue}{Duse}).« (\emph{Theaterbesuche}, \emph{Cambridge University Library},
                            Schnitzler, A 179a; nicht im \emph{\textcolor{green}{Tagebuch}}). Zwei Tage später sah er sie noch in \textcolor{blue}{Ibsen}s \emph{\textcolor{green}{Nora}}. In \textcolor{pink}{Berlin} hingegen trat sie im Dezember 1892
                        zum ersten Mal auf, ein zweites Gastspiel fand ein Jahr später statt.}}}\label{K_L00184_1h}
                    vor der \textcolor{blue}{\uline{Wolter}}{}\ledrightnote{\textcolor{blue}{Charlotte Wolter}}? Jemine! \label{K_L00184_2v}\edtext{\textcolor{blue}{Wengraf}{}\ledrightnote{\textcolor{blue}{Edmund Wengraf}} verriſs ſie}{\lemma{\textnormal{\emph{Wengraf verriſs ſie}}}\Cendnote{\textnormal{unklar, möglicherweise keine publizierte Aussage}}}\label{K_L00184_2h},
                        \label{K_L00184_3v}\edtext{\textcolor{blue}{Bahr}{}\ledrightnote{\textcolor{blue}{Hermann Bahr}} hob ſie in alle Himmel}{\lemma{\textnormal{\emph{Bahr … Himmel}}}\Cendnote{\textnormal{\textcolor{blue}{Bahr} rezensierte die \textcolor{pink}{Wien}er Gastspiele nicht. Es dürfte sich also um eine
                        Anspielung auf das Feuilleton \emph{\textcolor{green}{Eleonora
                            Duse}} vom 9. 5. 1891 (\emph{\textcolor{green}{Frankfurter Zeitung}}, Jg. 35, Nr. 129,
                            1. Morgenblatt, S. 1–2) oder auf den Abdruck in der \emph{\textcolor{green}{Russischen Reise}} (S. 116–125)
                        handeln, womit die deutschsprachige \textcolor{blue}{Duse}-Rezeption eingeleitet wurde.}}}\label{K_L00184_3h} – beides ſpricht gegen ſie.
                    Aber \uline{Ihre} Worte machen mich ſtutzen. »Wollen mal
                    ſehen, was ſich machen läſst« Ich bin gewiss der Letzte, der der Frau nicht ihr
                    Recht widerfahren läſst. Leben Sie recht wohl, \label{K_L00184_4v}\edtext{ertrinken Sie mir nicht}{\lemma{\textnormal{\emph{ertrinken Sie mir nicht}}}\Cendnote{\textnormal{\textcolor{blue}{Schnitzler} urlaubte vom 1.
                        bis zum 11. 3. an der Adria.}}}\label{K_L00184_4h} u
                    ſeien Sie mir herzlichſt gegrüßt\hspace*{3.5em}Ihr
                        \spacefill\mbox{KarlKraus}\pend
           \pstart
           \noindent{}\label{T_L00184_1v}\edtext{\textcolor{blue}{Buſſe}{}\ledrightnote{\textcolor{blue}{Carl Busse}} dankt u. grüßt herzlichſt.}{\lemma{\textnormal{\emph{Buſſe … herzlichſt.}}}\Cendnote{\textnormal{in der oberen rechten Ecke}}}\label{T_L00184_1h}\pend
           \endnumbering\briefempfaengerindex{Schnitzler, Arthur@\textsc{Schnitzler, Arthur}!zzzKraus, Karl@\emph{von Karl Kraus}!1893-03-041@{4. 3. 1893}|)be}\mylabel{h}  \normalsize

\doendnotes{C}
\bigskip
\vfill

\clearpage

\footnotesize

\lohead{\textsc{register}}

% Definiere theindex-Environment komplett neu ohne reledmac
\makeatletter
\renewenvironment{theindex}{%
  \section*{\indexname}%
  \setlength{\parindent}{0pt}%
  \setlength{\parskip}{0pt plus 0.3pt}%
  \let\item\@idxitem
}{%
  \clearpage
}
\makeatother

\IfFileExists{\jobname-pw.ind}{\input{\jobname-pw.ind}}{}

\end{document}

      