%% latex-korrekturansicht-vorspann.tex
%% Vorspann für die Korrekturansicht.
%% Lädt die gemeinsame Datei latex-vorspann.tex mit gesetztem Schalter.

\newif\ifkorrekturansicht
\korrekturansichttrue

\input{../tex-inputs/latex-vorspann}


               \section[Richard Beer-Hofmann an Arthur Schnitzler, 27. 10. 1919]{ Richard Beer-Hofmann an Arthur Schnitzler, 27. 10. 1919}\nopagebreak\mylabel{v}\rehead{ }\normalsize\beginnumbering\briefempfaengerindex{Schnitzler, Arthur@\textsc{Schnitzler, Arthur}!zzzBeer-Hofmann, Richard@\emph{von Richard Beer-Hofmann}!1919-10-271@{27. 10. 1919}|(be} \toendnotes[C]{\smallbreak\pagebreak[2]} \Standort{CUL, Schnitzler, B 8.}
\physDesc{Bildpostkarte
\newline{}Handschrift: Bleistift, lateinische Kurrent\newline{}Versand: Stempel: »\nobreak{}\oindex{Berlin@\textbf{Berlin}, \emph{https://www.geonames.org/ontologyP.PPLC}|pwk}Berlin W, 27. 10. 19, 11–12V\nobreak{}«.  \newline{}Ordnung: mit Bleistift von unbekannter Hand nummeriert: »268« }\buchAbdrucke{\weitereDrucke{Arthur Schnitzler, Richard Beer-Hofmann: \emph{Briefwechsel 1891–1931}. Hg. Konstanze Fliedl. Wien, Zürich: \emph{Europaverlag} 1992, S. 227.} }\toendnotes[C]{\smallbreak}\pstart{}{\pb}Herrn\pend{}\pstart{}D\textsuperscript{r} Arthur Schnitzler\pend{}\pstart{}\textcolor{pink}{Wien XVIII}{}\ledrightnote{\textcolor{pink}{XVIII., Währing}}\pend{}\pstart{}\textcolor{pink}{Sternwartestrasse}{}\ledrightnote{\textcolor{pink}{Sternwartestraße}}\pend{}{\bigskip}\pstart
           \noindent{}\centering{}{\pb}\textcolor{gray}{\textbf{\textcolor{blue}{SANDRO BOTTICELLI}{}\ledrightnote{\textcolor{blue}{Sandro Botticelli}}{ }\textcolor{green}{MADONNA MIT SINGENDEN
                           ENGELN}{}\ledrightnote{\textcolor{green}{Madonna mit dem Kind und singenden Engeln}}}}\pend
           \pstart
           Lieber Arthur! Bisher habe ich 3 mal versucht mit \textcolor{blue}{Reinhardt}{}\ledrightnote{\textcolor{blue}{Max Reinhardt}} über Ihre Angelegenheit zu sprechen. – Vergeblich! Es
               wird sich alles {\pb}entscheiden, –
               meint er – sobald man weiss – was mit dem Circus ist, dh. wann man ihn eröffnen kann
                  (November). Bis dahin sind die Repertoirfragen in Schwebe. \label{T_L02329-1v}\edtext{In Eile,
               aber von Herzen}{\lemma{\textnormal{\emph{In Eile,
               aber von Herzen}}}\Cendnote{\textnormal{rechts, entlang des Textes}}}\label{T_L02329-1h}\pend
           \pstart \spacefill\mbox{\label{T_L02329-2v}\edtext{Richard}{\lemma{\textnormal{\emph{Richard}}}\Cendnote{\textnormal{oberhalb,
                  verkehrt zum Text}}}\label{T_L02329-2h}}\pend{}\endnumbering\briefempfaengerindex{Schnitzler, Arthur@\textsc{Schnitzler, Arthur}!zzzBeer-Hofmann, Richard@\emph{von Richard Beer-Hofmann}!1919-10-271@{27. 10. 1919}|)be}\mylabel{h}  \normalsize

\doendnotes{C}
\bigskip
\vfill

\clearpage

\footnotesize

\lohead{\textsc{register}}

% Definiere theindex-Environment komplett neu ohne reledmac
\makeatletter
\renewenvironment{theindex}{%
  \section*{\indexname}%
  \setlength{\parindent}{0pt}%
  \setlength{\parskip}{0pt plus 0.3pt}%
  \let\item\@idxitem
}{%
  \clearpage
}
\makeatother

\IfFileExists{\jobname-pw.ind}{\input{\jobname-pw.ind}}{}

\end{document}

      