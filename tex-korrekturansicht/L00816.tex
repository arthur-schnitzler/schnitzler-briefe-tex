%% latex-korrekturansicht-vorspann.tex
%% Vorspann für die Korrekturansicht.
%% Lädt die gemeinsame Datei latex-vorspann.tex mit gesetztem Schalter.

\newif\ifkorrekturansicht
\korrekturansichttrue

\input{../tex-inputs/latex-vorspann}


               \section[Arthur Schnitzler an Hugo von Hofmannsthal, 10. 7. 1898]{ Arthur Schnitzler an Hugo von Hofmannsthal, 10. 7. 1898}\nopagebreak\mylabel{v}\rehead{ }\normalsize\beginnumbering\briefempfaengerindex{Hofmannsthal, Hugo von@\textsc{Hofmannsthal, Hugo von}!zzzSchnitzler, Arthur@\emph{von Arthur Schnitzler}!1898-07-102@{10. 7. 1898}|(be} \toendnotes[C]{\smallbreak\pagebreak[2]} \Standort{FDH, Hs-30885,69.}
\physDesc{Brief, 1 Blatt, 2 Seiten
\newline{}Handschrift: schwarze Tinte, deutsche Kurrent}\buchAbdrucke{\weitereDrucke{Hugo von Hofmannsthal, Arthur Schnitzler: \emph{Briefwechsel}. Hg. Therese Nickl und Heinrich Schnitzler. Frankfurt am Main: \emph{S. Fischer} 1964, S. 105.} }\toendnotes[C]{\smallbreak}\pstart
           \raggedleft{}{\pb}So{\geminationn}tag, 10. 7. 98.\pend
           \pstart{}Mein lieber Hugo,\pend\pstart
           morgen Früh reiſe ich ab. Bis Ende der Woche (16.) treffen mich
                    Nachrichten in \textcolor{pink}{Graz, Hotel zum Elefanten}{}\ledrightnote{\textcolor{pink}{Hotel Elefant}}. Für
                    das neue \textcolor{green}{Stück}{}\ledrightnote{→\textcolor{green}{Der Schleier der Beatrice. Schauspiel in fünf Akten}} iſt mir viel
                    und gutes eingefallen; doch werd ich es vor Auguſt kaum beginnen,
                    da ich ein bischen \textcolor{blue}{\textsc{Burckhard}}{}\ledrightnote{\textcolor{blue}{Jacob Burckhardt}}, \textcolor{blue}{\textsc{Gregorovius}}{}\ledrightnote{\textcolor{blue}{Ferdinand Gregorovius}}, {\pb}\textcolor{blue}{\textsc{Geiger}}{}\ledrightnote{\textcolor{blue}{Ludwig Geiger}} leſen will (dazu.)\pend
           \pstart
           – Meine Sti{\geminationm}ung iſt recht düſter; entko{\geminationm}en werd ich ihr nicht.\pend
           \pstart
           Laſſen Sie doch bald von ſich hören.\pend
           \pstart Von Herzen Ihr \spacefill\mbox{Arthur.}\pend{}\endnumbering\briefempfaengerindex{Hofmannsthal, Hugo von@\textsc{Hofmannsthal, Hugo von}!zzzSchnitzler, Arthur@\emph{von Arthur Schnitzler}!1898-07-102@{10. 7. 1898}|)be}\mylabel{h}  \normalsize

\doendnotes{C}
\bigskip
\vfill

\clearpage

\footnotesize

\lohead{\textsc{register}}

% Definiere theindex-Environment komplett neu ohne reledmac
\makeatletter
\renewenvironment{theindex}{%
  \section*{\indexname}%
  \setlength{\parindent}{0pt}%
  \setlength{\parskip}{0pt plus 0.3pt}%
  \let\item\@idxitem
}{%
  \clearpage
}
\makeatother

\IfFileExists{\jobname-pw.ind}{\input{\jobname-pw.ind}}{}

\end{document}

      