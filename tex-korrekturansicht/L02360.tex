%% latex-korrekturansicht-vorspann.tex
%% Vorspann für die Korrekturansicht.
%% Lädt die gemeinsame Datei latex-vorspann.tex mit gesetztem Schalter.

\newif\ifkorrekturansicht
\korrekturansichttrue

\input{../tex-inputs/latex-vorspann}


               \section[Arthur Schnitzler an Hermann Bahr, 7. 2. 1921]{ Arthur Schnitzler an Hermann Bahr, 7. 2. 1921}\nopagebreak\mylabel{v}\rehead{ }\normalsize\beginnumbering\briefempfaengerindex{Bahr, Hermann@\textsc{Bahr, Hermann}!zzzSchnitzler, Arthur@\emph{von Arthur Schnitzler}!1921-02-071@{7. 2. 1921}|(be} \toendnotes[C]{\smallbreak\pagebreak[2]} \Standort{TMW, HS AM 23396 Ba.}
\physDesc{Brief, 1 Blatt, 1 Seite
\newline{}Schreibmaschine
\newline{}Handschrift: 1) schwarze Tinte, lateinische Kurrent (\noindent{}Unterschrift und Grußformel)\hspace{1em}2) Bleistift, lateinische Kurrent (\noindent{}Korrekturen)\hspace{1em}}\Standort{DLA, A:Schnitzler, 85.1.294/7.}
\physDesc{Brief, maschineller Durchschlag
\newline{}Schreibmaschine}\buchAbdrucke{\weitereDrucke{1) \emph{7. 2. 1921.} In: Arthur Schnitzler: \emph{The Letters of Arthur Schnitzler to Hermann Bahr}. Edited, annotated, and with an introduction, by Donald G.
                        Daviau. Chapel Hill: \emph{The University of North Carolina Press} 1978, S. 115 (University of North Carolina studies in the Germanic languages
                        and literatures, 89).} \weitereDrucke{2) Hermann Bahr, Arthur Schnitzler: \emph{Briefwechsel, Aufzeichnungen, Dokumente (1891–1931)}. Hg. Kurt Ifkovits und Martin Anton Müller. Göttingen: \emph{Wallstein} 2018, S. 540.} }\toendnotes[C]{\smallbreak}\pstart
           \noindent{}{\pb}\textcolor{gray}{\textbf{D\textsuperscript{r } Arthur Schnitzler}}\hfill 7. 2. 1921.\pend
           \pstart
           \textcolor{gray}{\textbf{\textcolor{pink}{Wien. XVIII. Sternwartestrasse 71}{}\ledrightnote{\textcolor{pink}{Sternwartestraße}}.}}\pend
           \pstart{}Lieber Hermann.\pend\pstart
           Am \label{K_L02360_1v}\edtext{20. Feber}{\lemma{\textnormal{\emph{20. Feber}}}\Cendnote{\textnormal{Eigentlich am 21., wobei die Unsicherheit über den genauen Geburtstag in der
                  Presse verbreitet war.}}}\label{K_L02360_1h} feiert \textcolor{blue}{Popper-Lynkeus}{}\ledrightnote{\textcolor{blue}{Josef Popper-Lynkeus}} seinen 83. Geburtstag. \substVorne{}\textsuperscript{Es}\substDazwischen{}Das\substHinten{} fängt wie ein Aufruf an, aber es ist nur eine Bitte. Es wäre von einiger
               Bedeutung, insbesondere mit Rücksicht auf die bevorstehende \label{K_L02360_2v}\edtext{\textcolor{green}{Ausgabe}{}\ledrightnote{→\textcolor{green}{Krieg, Wehrpflicht und Staatsverfassung}}}{\lemma{\textnormal{\emph{Ausgabe}}}\Cendnote{\textnormal{Eine Werkausgabe erschien nicht, nur ein
                  neuer Titel: \textcolor{blue}{Josef Popper-Lynkeus}: \emph{\textcolor{green}{Krieg, Wehrpflicht und Staatsverfassung}}. Wien, Berlin,
                     Leipzig, München: \emph{\textcolor{brown}{Rikola}}{ }1921.}}}\label{K_L02360_2h} der \textcolor{blue}{Popper-Lynkeu’schen}{}\ledrightnote{\textcolor{blue}{Josef Popper-Lynkeus}} Werke
               im \label{K_L02360_3v}\edtext{Verlag \textcolor{brown}{Kola}{}\ledrightnote{\textcolor{brown}{Rikola }}}{\lemma{\textnormal{\emph{Verlag Kola}}}\Cendnote{\textnormal{Gemeint ist der \textcolor{pink}{Wien}er Verlag \emph{\textcolor{brown}{Rikola}}, der
                  von \textcolor{blue}{Richard Kola}{ }Ende 1920 mit Unterstützung \textcolor{blue}{Schnitzlers} gegründet wurde und für den in der Folge auch \textcolor{blue}{Bahr} tätig wurde.}}}\label{K_L02360_3h}, wenn an diesem Tag
               von einigen führenden Geistern die rechten Worte über ihn gesagt würden. Man hat mich
               gebeten Dich zu fragen, ob Du vielleicht in Deinem Tagebuch (der 20.
                  Feber ist gerade ein Sonntag) über \textcolor{blue}{Popper-Lynkeus}{}\ledrightnote{\textcolor{blue}{Josef Popper-Lynkeus}}, den Du ja, wie ich weiss, liebst und verehrst, schreiben
               wolltest. Wäre Dir diesmal irgend eine andere Form, ein anderer Rahmen genehm, so
               steht es natürlich ganz bei Dir. Es wäre von hohem Wert (wie ich glaube auch für den
               Elan des Verlages), wenn Du am 20. Februar unter denen nicht fehltest,
               die ein paar Worte über das Werk und das Wesen von \textcolor{blue}{Popper-Lynkeus}{}\ledrightnote{\textcolor{blue}{Josef Popper-Lynkeus}} sagen wollten.\pend
           \pstart
           Ich höre\introOben{}, –\introOben{} und lese es auch aus Deinem \textcolor{green}{Tagebuch}{}\ledrightnote{\textcolor{green}{Tagebuch [Kolumne im Neuen Wiener Journal]}} heraus, dass Du Dich wohlbefindest. Hoffentlich habe ich
               doch bald wieder Gelegenheit mich auch persönlich davon zu überzeugen.\pend
           \pstart
           {[}hs.:{]} Mit herzlichen Grüßen{\\[\baselineskip]}Dein{\\[\baselineskip]}\spacefill\mbox{Arthur}\pend
           \leftskip=0em{}\endnumbering\briefempfaengerindex{Bahr, Hermann@\textsc{Bahr, Hermann}!zzzSchnitzler, Arthur@\emph{von Arthur Schnitzler}!1921-02-071@{7. 2. 1921}|)be}\mylabel{h}  \normalsize

\doendnotes{C}
\bigskip
\vfill

\clearpage

\footnotesize

\lohead{\textsc{register}}

% Definiere theindex-Environment komplett neu ohne reledmac
\makeatletter
\renewenvironment{theindex}{%
  \section*{\indexname}%
  \setlength{\parindent}{0pt}%
  \setlength{\parskip}{0pt plus 0.3pt}%
  \let\item\@idxitem
}{%
  \clearpage
}
\makeatother

\IfFileExists{\jobname-pw.ind}{\input{\jobname-pw.ind}}{}

\end{document}

      