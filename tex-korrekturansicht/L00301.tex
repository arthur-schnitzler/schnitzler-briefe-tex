%% latex-korrekturansicht-vorspann.tex
%% Vorspann für die Korrekturansicht.
%% Lädt die gemeinsame Datei latex-vorspann.tex mit gesetztem Schalter.

\newif\ifkorrekturansicht
\korrekturansichttrue

\input{../tex-inputs/latex-vorspann}


               \section[Joseph Victor Widmann an Arthur Schnitzler, 26. 2. 1894]{ Joseph Victor Widmann an Arthur Schnitzler,
                    26. 2. 1894}\nopagebreak\mylabel{v}\rehead{ }\normalsize\beginnumbering\briefempfaengerindex{Schnitzler, Arthur@\textsc{Schnitzler, Arthur}!zzzWidmann, Joseph Victor@\emph{von Joseph Victor Widmann}!1894-02-261@{26. 2. 1894}|(be} \toendnotes[C]{\smallbreak\pagebreak[2]} \Standort{CUL, Schnitzler, B 113.}
\physDesc{Postkarte
\newline{}Handschrift: schwarze Tinte, deutsche Kurrent\newline{}Versand: 1) Stempel: »\nobreak{}\oindex{Bern@\textbf{Bern}, \emph{Besiedelter Ort (A.BSO)}|pwk}Bern Brf. Exp., 26. II. 94., 1\nobreak{}«.  2) Stempel: »\nobreak{}\oindex{IX., Alsergrund@\textbf{IX., Alsergrund}, \emph{Bezirk (A.BZK)}|pwk}Wien 9/{[}3{]}, 28. 2. 94, 8.V, Bestellt\nobreak{}«. }\toendnotes[C]{\smallbreak}\pstart{}{\pb}\textsc{Herrn
                                D\textsuperscript{r} Arthur Schnitzler}\pend{}\pstart{}Schriftsteller in\pend{}\pstart{}\textcolor{pink}{\textsc{Wien} IX}{}\ledrightnote{\textcolor{pink}{I., Innere Stadt}}\pend{}\pstart{}\textcolor{pink}{\textsc{Frankenstr 1/?}}{}\ledrightnote{\textcolor{pink}{Frankgasse}}\pend{}{\bigskip}\pstart
           \raggedleft{}{\pb}\textcolor{pink}{Bern}{}\ledrightnote{\textcolor{pink}{Bern}}, d.
                            26. Febr. 1894.\pend
           \pstart{}Sehr geehrter Herr!\pend\pstart
           Selbſtverſtändlicher Weiſe habe ich gar nichts dagegen, we{\geminationn}
               Sie zu meiner \textcolor{green}{Kritik}{}\ledrightnote{→\textcolor{green}{Kunst und Litteratur}} über den prächtigen \textcolor{green}{Anatol}{}\ledrightnote{\textcolor{green}{Anatol}} meinen vollen \label{K_L00301_1v}\edtext{Namen ſetzen}{\lemma{\textnormal{\emph{Namen ſetzen}}}\Cendnote{\textnormal{Am Ende der Buchausgabe von \emph{\textcolor{green}{Das Märchen}} (Schauspiel in drei Aufzügen.
                            Dresden, Leipzig: \emph{\textcolor{brown}{E. Pierson’s
                                Verlag}}{ }1894) wurden, als Verlagswerbung, Auszüge aus
                        Kritiken von \emph{\textcolor{green}{Anatol}} gesetzt. Mit seinem
                        nicht erhaltenen Brief dürfte \textcolor{blue}{Schnitzler}
                        um die Erlaubnis für \textcolor{blue}{Widmann}s \textcolor{green}{Besprechung} angesucht
                        haben.}}}\label{K_L00301_1h}; im Gegentheil, ich beke{\geminationn}e mich
                    ſehr gern dazu.\pend
           \pstart
           Hoffentlich beko{\geminationm}en Sie dieſe Zeilen, obwohl in
                    Ihrem Briefchen juſt Ihre Wohnungsangabe verwiſcht war u. ich ſie daher nur
                    andeutungsweiſe auf dieſe Karte ſetzen ko{\geminationn}te.\pend
           \pstart
           Mit freundl. Gruß{\\[\baselineskip]}\spacefill\mbox{J. V. Widmann}\pend
           \leftskip=0em{}\endnumbering\briefempfaengerindex{Schnitzler, Arthur@\textsc{Schnitzler, Arthur}!zzzWidmann, Joseph Victor@\emph{von Joseph Victor Widmann}!1894-02-261@{26. 2. 1894}|)be}\mylabel{h}  \normalsize

\doendnotes{C}
\bigskip
\vfill

\clearpage

\footnotesize

\lohead{\textsc{register}}

% Definiere theindex-Environment komplett neu ohne reledmac
\makeatletter
\renewenvironment{theindex}{%
  \section*{\indexname}%
  \setlength{\parindent}{0pt}%
  \setlength{\parskip}{0pt plus 0.3pt}%
  \let\item\@idxitem
}{%
  \clearpage
}
\makeatother

\IfFileExists{\jobname-pw.ind}{\input{\jobname-pw.ind}}{}

\end{document}

      