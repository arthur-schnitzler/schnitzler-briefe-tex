%% latex-korrekturansicht-vorspann.tex
%% Vorspann für die Korrekturansicht.
%% Lädt die gemeinsame Datei latex-vorspann.tex mit gesetztem Schalter.

\newif\ifkorrekturansicht
\korrekturansichttrue

\input{../tex-inputs/latex-vorspann}


\renewcommand{\erwaehntePersonen}{Personen: Paul Goldmann, Friedrich Hebbel, Paul Marx, Felix Salten, Olga Schnitzler, Elisabeth Steinrück}
\renewcommand{\erwaehnteInstitutionen}{Institutionen: Jung-Wiener Theater zum Lieben Augustin}
\renewcommand{\erwaehnteOrte}{Orte: Berlin, Dessauer Straße, Konservatorium der Gesellschaft der Musikfreunde, Wien}
\renewcommand{\erwaehnteWerke}{Werke: Maria Magdalena. Ein bürgerliches Trauerspiel in drei Akten, Neue Freie Presse}
\section[ Paul Goldmann an Olga Gussmann, 29. 4. {[}1901{]}]{Paul Goldmann an Olga Gussmann, 29. 4. {[}1901{]}}
\nopagebreak\mylabel{v}
\rehead{ }\normalsize\beginnumbering\briefempfaengerindex{Schnitzler, Olga@\textsc{Schnitzler, Olga}!zzzGoldmann, Paul@\emph{von Paul Goldmann}!1901-04-291@{29. 4. {[}1901{]}}|(be}
\toendnotes[C]{\smallbreak\pagebreak[2]}\Standort{DLA, A:Schnitzler, HS.NZ85.1.5247.}
\physDesc{Brief, 1 Blatt, 3 Seiten, 971 Zeichen
\newline{}Handschrift: blaue Tinte, deutsche Kurrent}\toendnotes[C]{\smallbreak}
\pstart
           \noindent{}\raggedleft{}{\pb}\textcolor{gray}{\textbf{\textcolor{pink}{DESSAUERSTRASSE 19}{}\ledrightnote{\textcolor{pink}{Dessauer Straße}}}}\pend
           
\pstart
           \textcolor{pink}{Berlin}{}\ledrightnote{\textcolor{pink}{Berlin}}, 29. April.\pend
           
\pstart\center{}Liebes Fräulein \textsc{Olga},\pend
\pstart
           Ich habe heut ſehr wenig Zeit und kann Ihnen nur in
               Eile für Ihren Brief danken und Ihnen die Hand drücken. Sicherlich haben Sie einen
               großen \label{K_L03530-1v}\edtext{Erfolg}{\lemma{\textnormal{\emph{Erfolg}}}\Cendnote{\textnormal{Am 28. 4. 1901 trat \textcolor{blue}{Olga Gussmann}
                  in einer Schulvorstellung des \textcolor{pink}{Konservatorium}s in \textcolor{blue}{Friedrich
                  Hebbel}s \emph{\textcolor{green}{Maria Magdalena}} auf. Siehe Arthur Schnitzler an Hermann Bahr, 19. 4. 1901.}}}\label{K_L03530-1h} gehabt. Ich
               erwarte bald Bericht. Schicken Sie mir, bitte, auch einige \label{K_L03530-2v}\edtext{Zeitungsausſchnitte}{\lemma{\textnormal{\emph{Zeitungsausſchnitte}}}\Cendnote{\textnormal{siehe Paul Goldmann an Olga Gussmann, 10. 5. [1901]}}}\label{K_L03530-2h}. Hätte man nicht ein Referat in der \textcolor{green}{N. Fr.
                  Pr.}{}\ledrightnote{\textcolor{green}{Neue Freie Presse}} veranlaſſen können? Warum haben Sie mir nicht \substVorne{}\textsuperscript{vorher}{\allowbreak}\substDazwischen{}vorher\substHinten{} geſchrieben?\pend
           
\pstart
           {\pb}Über \label{K_L03530-3v}\edtext{\textsc{\textcolor{blue}{Salten}{}\ledrightnote{\textcolor{blue}{Felix Salten}}}}{\lemma{\textnormal{\emph{Salten}}}\Cendnote{\textnormal{Hatte dieser eine Besprechung der Aufführung abgelehnt? Überraschend,
                  aber möglich, wäre ein Bezug auf das im Entstehen begriffene 
                   \emph{\textcolor{brown}{Jung-Wiener Theater zum lieben Augustin}}. vgl. Paul Goldmann an Arthur Schnitzler, 16. 5. [1901].
               }}}\label{K_L03530-3h} bin ich ganz Ihrer Anſicht.\pend
           
\pstart
           Ob ich einen \label{K_L03530-4v}\edtext{Theil des Sommers mit
               Ihnen verbringen}{\lemma{\textnormal{\emph{Theil … verbringen}}}\Cendnote{\textnormal{siehe Paul Goldmann an Arthur Schnitzler, 26. 4. [1901]}}}\label{K_L03530-4h} werde, weiß ich noch nicht. Ich hätte Luſt, mich in ein ſehr wildes Land
               ſchicken zu laſſen, weit, weit weg.\pend
           
\pstart
           Daß ihre Schweſter \textsc{\textcolor{blue}{Liesl}{}\ledrightnote{\textcolor{blue}{Elisabeth Steinrück}}} meinen Brief noch immer nicht beantwortet hat, iſt ganz einfach empörend. Sagen
               Sie, bitte, dieſem jungen \textcolor{blue}{Geſchöpf}{}\ledrightnote{{$\rightarrow$}\textcolor{blue}{Elisabeth Steinrück}}, daß ich ſie zur Erbin meines ungeheuren Vermögens eingeſetzt ha\substVorne{}\textsuperscript{\textcolor{gray}{be}}\substDazwischen{}tte\substHinten{}, daß ich ſie aber infolge ihres pietätloſen Verhaltens wieder {\pb}aus meinem Teſtament geſtrichen habe.\pend
           
\pstart
           Herzliche Grüße an Sie \textcolor{blue}{\strikeout{Beide}}{}\ledrightnote{{$\rightarrow$}\textcolor{blue}{Elisabeth Steinrück}}{ }\textcolor{blue}{Beide}{}\ledrightnote{{$\rightarrow$}\textcolor{blue}{Elisabeth Steinrück}} und an Herrn \textsc{\textcolor{blue}{Paul}{}\ledrightnote{\textcolor{blue}{Paul Marx}}} von {\\[\baselineskip]}Ihrem ergebenen {\\[\baselineskip]}\spacefill\mbox{Dr. Paul Goldmann.}\pend
           \leftskip=0em{}\endnumbering\briefempfaengerindex{Schnitzler, Olga@\textsc{Schnitzler, Olga}!zzzGoldmann, Paul@\emph{von Paul Goldmann}!1901-04-291@{29. 4. {[}1901{]}}|)be}\mylabel{h}  \normalsize

\doendnotes{C}
\bigskip
\vfill

\clearpage

\footnotesize

\lohead{\textsc{register}}

% Definiere theindex-Environment komplett neu ohne reledmac
\makeatletter
\renewenvironment{theindex}{%
  \section*{\indexname}%
  \setlength{\parindent}{0pt}%
  \setlength{\parskip}{0pt plus 0.3pt}%
  \let\item\@idxitem
}{%
  \clearpage
}
\makeatother

\IfFileExists{\jobname-pw.ind}{\input{\jobname-pw.ind}}{}

\end{document}

      