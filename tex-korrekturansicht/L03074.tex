%% latex-korrekturansicht-vorspann.tex
%% Vorspann für die Korrekturansicht.
%% Lädt die gemeinsame Datei latex-vorspann.tex mit gesetztem Schalter.

\newif\ifkorrekturansicht
\korrekturansichttrue

\input{../tex-inputs/latex-vorspann}


\renewcommand{\erwaehntePersonen}{Personen: Robert Hirschfeld}
\renewcommand{\erwaehnteOrte}{Orte: Berlin, Frankgasse, Sekirn, Vahrn, Wien}
\renewcommand{\erwaehnteWerke}{}
\section[ Paul Goldmann an Arthur Schnitzler, 21. 7. {[}1901{]}]{Paul Goldmann an Arthur Schnitzler, 21. 7. {[}1901{]}}
\nopagebreak\mylabel{v}
\rehead{ }\normalsize\beginnumbering\briefempfaengerindex{Schnitzler, Arthur@\textsc{Schnitzler, Arthur}!zzzGoldmann, Paul@\emph{von Paul Goldmann}!1901-07-211@{21. 7. {[}1901{]}}|(be}
\toendnotes[C]{\smallbreak\pagebreak[2]}\Standort{DLA, A:Schnitzler, HS.NZ85.1.3171.}
\physDesc{Postkarte
\newline{}Handschrift: 1) blaue Tinte, deutsche Kurrent\hspace{1em}2) blaue Tinte, lateinische Kurrent (\noindent{}Adresse)\hspace{1em}
\newline{}Versand: 1) Stempel: »\nobreak{}\oindex{Berlin@\textbf{Berlin}, \emph{https://www.geonames.org/ontologyP.PPLC}|pwk}Berlin, W 9, 21. 7. 01, 2–3 N\nobreak{}«.   2) Stempel: »\nobreak{}Wien 9/3 72, 22. 7. 01, 9. {[}V{]}, Bes{[}tell{]}t\nobreak{}«. 
\newline{}Schnitzler: mit Bleistift das Datum »21/7 {[}1{]}901« vermerkt }\toendnotes[C]{\smallbreak}\pstart{}{\pb}Herrn\pend{}\pstart{}Dr. Arthur Schnitzler\pend{}\pstart{}\textcolor{pink}{Wien}{}\ledrightnote{\textcolor{pink}{Wien}}\pend{}\pstart{}\textcolor{pink}{IX. Frankgaße 1}{}\ledrightnote{\textcolor{pink}{Frankgasse}}.\pend{}
{\bigskip}
\pstart
           \noindent{}{\pb}\textcolor{pink}{Berlin}{}\ledrightnote{\textcolor{pink}{Berlin}}, 21. Juli.\hfill Mein lieber Freund,\pend
           
\pstart
           Haſt Du meinen nach \label{K_L03074-22v}\edtext{\textsc{\textcolor{pink}{Vahrn}{}\ledrightnote{\textcolor{pink}{Vahrn}}} geſandten Brief}{\lemma{\textnormal{\emph{Vahrn geſandten Brief}}}\Cendnote{\textnormal{siehe Paul Goldmann an Arthur Schnitzler, 19. 7. [1901]}}}\label{K_L03074-22h} erhalten? Du
               hatteſt keine nähere Adreſſe angegeben. Ich ſchreibe alſo der Sicherheit halber noch
               einmal an Deine \textcolor{pink}{Wien}{}\ledrightnote{\textcolor{pink}{Wien}}er Adreſſe, daß ich Montag reiſe u. Ende der Woche in \textsc{\textcolor{pink}{Seekirn}{}\ledrightnote{\textcolor{pink}{Sekirn}}} bei \textsc{\textcolor{blue}{Hirschfeld}{}\ledrightnote{\textcolor{blue}{Robert Hirschfeld}}} ſein dürfte, wo ich Deine lieben Nachrichten zu finden hoffe. 
            \pend
           \pstart Herzlichſt Dein
               \spacefill\mbox{P. G.}\pend{}\endnumbering\briefempfaengerindex{Schnitzler, Arthur@\textsc{Schnitzler, Arthur}!zzzGoldmann, Paul@\emph{von Paul Goldmann}!1901-07-211@{21. 7. {[}1901{]}}|)be}\mylabel{h}
\begin{anhang}
\end{anhang}\normalsize

\doendnotes{C}
\bigskip
\vfill

\clearpage

\footnotesize

\lohead{\textsc{register}}

% Definiere theindex-Environment komplett neu ohne reledmac
\makeatletter
\renewenvironment{theindex}{%
  \section*{\indexname}%
  \setlength{\parindent}{0pt}%
  \setlength{\parskip}{0pt plus 0.3pt}%
  \let\item\@idxitem
}{%
  \clearpage
}
\makeatother

\IfFileExists{\jobname-pw.ind}{\input{\jobname-pw.ind}}{}

\end{document}

      