%% latex-korrekturansicht-vorspann.tex
%% Vorspann für die Korrekturansicht.
%% Lädt die gemeinsame Datei latex-vorspann.tex mit gesetztem Schalter.

\newif\ifkorrekturansicht
\korrekturansichttrue

\input{../tex-inputs/latex-vorspann}


\renewcommand{\erwaehntePersonen}{Personen: Paul Schlenther}
\renewcommand{\erwaehnteInstitutionen}{Institutionen: Burgtheater}
\renewcommand{\erwaehnteOrte}{Orte: Bad Ischl, Kochgasse, Reichenau an der Rax, Wien}
\renewcommand{\erwaehnteWerke}{Werke: Der Schleier der Beatrice. Schauspiel in fünf Akten}
\section[ Felix Salten an Arthur Schnitzler, 12. 7. 1900]{Felix Salten an Arthur Schnitzler, 12. 7. 1900}
\nopagebreak\mylabel{v}
\rehead{ }\normalsize\beginnumbering\briefempfaengerindex{Schnitzler, Arthur@\textsc{Schnitzler, Arthur}!zzzSalten, Felix@\emph{von Felix Salten}!1900-07-121@{12. 7. 1900}|(be}
\toendnotes[C]{\smallbreak\pagebreak[2]}\Standort{CUL, Schnitzler, B 89, A 2.}
\physDesc{Brief, 1 Blatt, 1 Seite, 979 Zeichen
\newline{}Handschrift: schwarze Tinte, lateinische Kurrent
\newline{}Ordnung: mit Bleistift von unbekannter Hand nummeriert: »130« }\toendnotes[C]{\smallbreak}
\pstart
           \raggedleft{}{\pb}den 12. Juli 00.\pend
           
\pstart
           Lieber Freund, danke für das Lebenszeichen nach so viel Tagen.
               Möchten Sie bei dem elenden Wetter nicht vor dem 20.{ }\label{K_L03306-1v}\edtext{nach \textcolor{pink}{Wien}{}\ledrightnote{\textcolor{pink}{Wien}} kommen}{\lemma{\textnormal{\emph{nach Wien kommen}}}\Cendnote{\textnormal{\textcolor{blue}{Schnitzler} hielt sich seit dem 5. 7. 1900 in \textcolor{pink}{Reichenau an der Rax} auf. Nach \textcolor{pink}{Wien} kehrte
                     er am 23. 7. 1900 zurück.}}}\label{K_L03306-1h}? Wenn’s einmal
               schön wäre, führe ich ja gerne nach 
               \textcolor{pink}{R.}{}\ledrightnote{\textcolor{pink}{Reichenau an der Rax}}
                 aber, es wird nicht schön. Ich
               bin leider nicht in richtiger Arbeit. Schreibe nur so, – immer ein
                  bisserl\textcolor{gray}{,} und hab geglaubt, weiß Gott, wie viel ich in diesem
               Sommer ausrichten werde. Vielleicht wird's noch besser. Jedenfalls halte ich mich
               täglich dazu. Am 1. August ziehe ich in die \textcolor{pink}{Kochgasse 32, VIII. Bezirk}{}\ledrightnote{\textcolor{pink}{Kochgasse}}\textcolor{gray}{,} hübsche kleine Wohnung. Dann fahre ich am 4. nach \textcolor{pink}{Ischl}{}\ledrightnote{\textcolor{pink}{Bad Ischl}}. Sie
               wol auch? Haben Sie die verschiedenen \textcolor{brown}{Burgtheater}{}\ledrightnote{\textcolor{brown}{Burgtheater}}-Rückblicke in den Zeitungen gesehen? In einigen wird energisch nach
               der \label{K_L03306-2v}\edtext{»\textcolor{green}{Beatrice}{}\ledrightnote{\textcolor{green}{Der Schleier der Beatrice. Schauspiel in fünf Akten}}«}{\lemma{\textnormal{\emph{»Beatrice«}}}\Cendnote{\textnormal{siehe Felix Salten an Arthur Schnitzler, [20. 6. 1900]}}}\label{K_L03306-2h} gefragt. Für \textcolor{blue}{Schlenth}{}\ledrightnote{\textcolor{blue}{Paul Schlenther}}. ist übrigens
               anzumerken, dass er Ihr \textcolor{green}{Stück}{}\ledrightnote{{$\rightarrow$}\textcolor{green}{Der Schleier der Beatrice. Schauspiel in fünf Akten}}{ }\label{K_L03306-3v}\edtext{s. Z.}{\lemma{\textnormal{\emph{s. Z.}}}\Cendnote{\textnormal{seiner Zeit}}}\label{K_L03306-3h} benützte, um in einer leeren Saison volle
               Schubladen zu zeigen. Ein unsolider Geschäftsmann.\pend
           
\pstart
           Was machen Ihre Sommergastspiele? Hoffentlich höre ich bald mündlich oder schriftlich
               genaueres von Ihnen.\pend
           \pstart Herzlichst Ihr \spacefill\mbox{Salten}\pend{}\endnumbering\briefempfaengerindex{Schnitzler, Arthur@\textsc{Schnitzler, Arthur}!zzzSalten, Felix@\emph{von Felix Salten}!1900-07-121@{12. 7. 1900}|)be}\mylabel{h}  \normalsize

\doendnotes{C}
\bigskip
\vfill

\clearpage

\footnotesize

\lohead{\textsc{register}}

% Definiere theindex-Environment komplett neu ohne reledmac
\makeatletter
\renewenvironment{theindex}{%
  \section*{\indexname}%
  \setlength{\parindent}{0pt}%
  \setlength{\parskip}{0pt plus 0.3pt}%
  \let\item\@idxitem
}{%
  \clearpage
}
\makeatother

\IfFileExists{\jobname-pw.ind}{\input{\jobname-pw.ind}}{}

\end{document}

      