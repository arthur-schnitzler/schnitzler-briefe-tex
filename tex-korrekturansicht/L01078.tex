%% latex-korrekturansicht-vorspann.tex
%% Vorspann für die Korrekturansicht.
%% Lädt die gemeinsame Datei latex-vorspann.tex mit gesetztem Schalter.

\newif\ifkorrekturansicht
\korrekturansichttrue

\input{../tex-inputs/latex-vorspann}


               \section[Arthur Schnitzler an Hermann Bahr, 18. 10. 1900]{ Arthur Schnitzler an Hermann Bahr, 18. 10. 1900}\nopagebreak\mylabel{v}\rehead{ }\normalsize\beginnumbering\briefempfaengerindex{Bahr, Hermann@\textsc{Bahr, Hermann}!zzzSchnitzler, Arthur@\emph{von Arthur Schnitzler}!1900-10-181@{18. 10. 1900}|(be} \toendnotes[C]{\smallbreak\pagebreak[2]} \Standort{TMW, HS AM 23338 Ba.}
\physDesc{Brief, 1 Blatt, 3 Seiten
\newline{}Handschrift: schwarze Tinte, deutsche Kurrent}\buchAbdrucke{\weitereDrucke{1) \emph{18. 10. 1900.} In: Arthur Schnitzler: \emph{The Letters of Arthur Schnitzler to Hermann Bahr}. Edited, annotated, and with an introduction, by Donald G.
                        Daviau. Chapel Hill: \emph{The University of North Carolina Press} 1978, S. 67 (University of North Carolina studies in the Germanic languages
                        and literatures, 89).} \weitereDrucke{2) Hermann Bahr, Arthur Schnitzler: \emph{Briefwechsel, Aufzeichnungen, Dokumente (1891–1931)}. Hg. Kurt Ifkovits und Martin Anton Müller. Göttingen: \emph{Wallstein} 2018, S. 192.} }\toendnotes[C]{\smallbreak}\pstart
           {\pb}\textsc{\textcolor{pink}{Baden b/W.}{}\ledrightnote{\textcolor{pink}{Baden bei Wien}}}{ }18. 10. 900\pend
           \pstart
           lieber Hermann, deine Sympathie für die \textcolor{green}{\textsc{Beatrice}}{}\ledrightnote{\textcolor{green}{Der Schleier der Beatrice. Schauspiel in fünf Akten}} freut mich herzlich. Vielen Dank für die lieben Worte, in denen du mirs geſagt
               haſt. We{\geminationn} du erlaubſt, bring ich dir das \textcolor{green}{\textsc{Mscrpt} der Novelle}{}\ledrightnote{→\textcolor{green}{Lieutenant Gustl. Novelle}} nächſtens, vielleicht Mitte oder
               Ende nächſter Woche, bis ich wieder {\pb}in \textcolor{pink}{Wien}{}\ledrightnote{\textcolor{pink}{Wien}} bin. Mit beſonderem Vergnügen habe ich den \textcolor{green}{Franzl}{}\ledrightnote{\textcolor{green}{Der Franzl. Fünf Bilder aus dem Leben eines guten Mannes}} geleſen, beſonders den erſten, dritten und vierten Akt.
               Aber manchem werden gewiſs die beiden andern Akte mit dem \textcolor{gray}{vielen}
               Gemüth noch beſſer gefallen. Es iſt eine köſtliche Lebendigkeit in den Bauernburſchen
               wie in den Hofräthen, {\pb}der Himmel über dem ganzen echt \textcolor{pink}{oeſterreichiſch}{}\ledrightnote{\textcolor{pink}{Österreich}}
               – nur die Geſtirne ko{\geminationm}en mir \substVorne{}\textsuperscript{ſozuſagen }{\allowbreak}\substDazwischen{}zu weilen\substHinten{} ein biſſel »Theater\textcolor{gray}{«} vor.\pend
           \pstart
           Auf Wiederſehen.{\\[\baselineskip]}Herzlichſt dein{\\[\baselineskip]}\spacefill\mbox{Arth Sch.}\pend
           \leftskip=0em{}\pstart
           18. 10. 900.\pend
           \endnumbering\briefempfaengerindex{Bahr, Hermann@\textsc{Bahr, Hermann}!zzzSchnitzler, Arthur@\emph{von Arthur Schnitzler}!1900-10-181@{18. 10. 1900}|)be}\mylabel{h}  \normalsize

\doendnotes{C}
\bigskip
\vfill

\clearpage

\footnotesize

\lohead{\textsc{register}}

% Definiere theindex-Environment komplett neu ohne reledmac
\makeatletter
\renewenvironment{theindex}{%
  \section*{\indexname}%
  \setlength{\parindent}{0pt}%
  \setlength{\parskip}{0pt plus 0.3pt}%
  \let\item\@idxitem
}{%
  \clearpage
}
\makeatother

\IfFileExists{\jobname-pw.ind}{\input{\jobname-pw.ind}}{}

\end{document}

      