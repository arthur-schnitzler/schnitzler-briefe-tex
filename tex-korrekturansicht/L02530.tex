%% latex-korrekturansicht-vorspann.tex
%% Vorspann für die Korrekturansicht.
%% Lädt die gemeinsame Datei latex-vorspann.tex mit gesetztem Schalter.

\newif\ifkorrekturansicht
\korrekturansichttrue

\input{../tex-inputs/latex-vorspann}


               \section[Robert Adam an Arthur Schnitzler, 5. 1. 1930]{ Robert Adam an Arthur Schnitzler, 5. 1. 1930}\nopagebreak\mylabel{v}\rehead{ }\normalsize\beginnumbering\briefempfaengerindex{Schnitzler, Arthur@\textsc{Schnitzler, Arthur}!zzzAdam, Robert@\emph{von Robert Adam}!1930-01-051@{5. 1. 1930}|(be} \toendnotes[C]{\smallbreak\pagebreak[2]} \Standort{CUL, Schnitzler, B 1.}
\physDesc{Brief, 1 Blatt, 2 Seiten
\newline{}Handschrift: schwarze Tinte, deutsche Kurrent
\newline{}Schnitzler: 1) mit rotem Buntstift beschriftet: »\textsc{\textcolor{green}{Spiel}}« 2) mit rotem Buntstift vereinzelte Unterstreichungen\newline{}Ordnung: mit Bleistift von unbekannter Hand nummeriert:
                                        »24« }\Standort{Wien, Österreichische Nationalbibliothek, Cod.ser. 52.269, 153 recto, 155 recto.}
\physDesc{handschriftliche Abschrift, Entwurf
\newline{}Handschrift: schwarze Tinte, Gabelsberger Kurzschrift}\Standort{Wien, Österreichische Nationalbibliothek, Cod.ser. 52.269, 153 recto, 155 recto.}
\physDesc{maschinelle Abschrift, Entwurf
\newline{}Schreibmaschine}\toendnotes[C]{\smallbreak}\pstart
           \raggedleft{}{\pb}\textcolor{pink}{Wien}{}\ledrightnote{\textcolor{pink}{Wien}}, am 5. Januar 1930\pend
           \pstart{}Hochverehrter Herr Doktor!\pend\pstart
           Nehmen Sie vor allem meinen beſten Dank für Ihren Brief, der mich über Verdienſt
                    erfreute, und zugleich für die liebenswürdige Anweiſung der Sitze zum »\textcolor{green}{Spiel der Sommerlüfte}{}\ledrightnote{\textcolor{green}{Im Spiel der Sommerlüfte. In drei Aufzügen}}«. Ich komme jetzt ſo
                    ſelten in’s Theater, daß ich nicht weiß, ob ich ein Urteil äußern darf; ich
                    möchte aber doch ſagen, daß mir die Aufführung vortrefflich zu ſein ſchien.
                    Selbſt mit dem \textcolor{blue}{Darſteller}{}\ledrightnote{→\textcolor{blue}{Alexander Moissi}}
                    des \textcolor{green}{Kaplans}{}\ledrightnote{→\textcolor{green}{Im Spiel der Sommerlüfte. In drei Aufzügen}}, deſſen Sprache,
                    Stimme und Gehaben mir nie recht behagten, konnte ich mich diesmal befreunden,
                    ſodaß ich in den allgemeinen Beifall auch inſoweit er den Schauspielern galt mit
                    gutem Gewiſſen einſtimmen durfte. Manches Zarte Ihrer \textcolor{green}{Komödie}{}\ledrightnote{→\textcolor{green}{Im Spiel der Sommerlüfte. In drei Aufzügen}} iſt allerdings vergröbert, aber
                    ich möchte meinen, daß dieſes Übel mit jeder Bühnendarſtellung unweigerlich
                        ver{\pb}bunden iſt.\pend
           \pstart
           Mit vielen Grüßen und Empfehlungen Ihr ergebener\pend
           \pstart \spacefill\mbox{D\textsuperscript{r}RAdam}\pend{}\endnumbering\briefempfaengerindex{Schnitzler, Arthur@\textsc{Schnitzler, Arthur}!zzzAdam, Robert@\emph{von Robert Adam}!1930-01-051@{5. 1. 1930}|)be}\mylabel{h}  \normalsize

\doendnotes{C}
\bigskip
\vfill

\clearpage

\footnotesize

\lohead{\textsc{register}}

% Definiere theindex-Environment komplett neu ohne reledmac
\makeatletter
\renewenvironment{theindex}{%
  \section*{\indexname}%
  \setlength{\parindent}{0pt}%
  \setlength{\parskip}{0pt plus 0.3pt}%
  \let\item\@idxitem
}{%
  \clearpage
}
\makeatother

\IfFileExists{\jobname-pw.ind}{\input{\jobname-pw.ind}}{}

\end{document}

      