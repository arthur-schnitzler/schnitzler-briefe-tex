%% latex-korrekturansicht-vorspann.tex
%% Vorspann für die Korrekturansicht.
%% Lädt die gemeinsame Datei latex-vorspann.tex mit gesetztem Schalter.

\newif\ifkorrekturansicht
\korrekturansichttrue

\input{../tex-inputs/latex-vorspann}


               \section[Arthur Schnitzler an Richard Beer-Hofmann, 8. 10. 1899]{ Arthur Schnitzler an Richard Beer-Hofmann, 8. 10. 1899}\nopagebreak\mylabel{v}\rehead{ }\normalsize\beginnumbering\briefempfaengerindex{Beer-Hofmann, Richard@\textsc{Beer-Hofmann, Richard}!zzzSchnitzler, Arthur@\emph{von Arthur Schnitzler}!1899-10-081@{8. 10. 1899}|(be} \toendnotes[C]{\smallbreak\pagebreak[2]} \Standort{YCGL, MSS 31.}
\physDesc{Brief, 1 Blatt (Briefpapier mit Trauerrand), 3 Seiten, Umschlag
\newline{}Handschrift: schwarze Tinte, deutsche Kurrent\newline{}Versand: 1) Stempel: »\nobreak{}\oindex{Berlin@\textbf{Berlin}, \emph{https://www.geonames.org/ontologyP.PPLC}|pwk}Berlin, 8. 10. 99, 5–6N\nobreak{}«.  2) Stempel: »\nobreak{}\oindex{Sankt Michael@\textbf{Sankt Michael}, \emph{Bezirk (A.BZK)}|pwk}St. Mich{[}ae{]}l in Eppan, 10 10 99\nobreak{}«. }\buchAbdrucke{\weitereDrucke{Arthur Schnitzler, Richard Beer-Hofmann: \emph{Briefwechsel 1891–1931}. Hg. Konstanze Fliedl. Wien, Zürich: \emph{Europaverlag} 1992, S. 139.} }\toendnotes[C]{\smallbreak}\pstart{}{\pb}\textsc{Herrn Dr. Richard
                     Beer-Hofmann}\pend{}\pstart{}\textcolor{pink}{\textsc{St.
                     Michael im Eppan}}{}\ledrightnote{\textcolor{pink}{Sankt Michael}}\pend{}\pstart{}\textsc{\textcolor{pink}{Tirol}{}\ledrightnote{\textcolor{pink}{Tirol}}}\pend{}{\bigskip}\pstart
           \raggedleft{}{\pb}\textcolor{pink}{\textsc{Berlin}}{}\ledrightnote{\textcolor{pink}{Berlin}}{ } 8. X. 99.\pend
           \pstart
           mein lieber Richard, das iſt entſetzlich, was dieſer \textcolor{blue}{Leo}{}\ledrightnote{\textcolor{blue}{Leo Feld}} wieder \label{K_L00989_1v}\edtext{durchmachen}{\lemma{\textnormal{\emph{durchmachen}}}\Cendnote{\textnormal{Er hatte sich mit
                     \textcolor{blue}{Olga Wohlbrück} verlobt, die beiden
                  heirateten im März 1900 in \textcolor{pink}{Berlin}.}}}\label{K_L00989_1h} muſs! Da kommen einem immer wieder dieſe alten Phraſen
               in den Mund, aber ich will ſie unterdrücken. Wa{\geminationn} kommen
               Sie nach \textcolor{pink}{Wien}{}\ledrightnote{\textcolor{pink}{Wien}}? \textcolor{blue}{Paul
                  Goldmann}{}\ledrightnote{\textcolor{blue}{Paul Goldmann}} ko{\geminationm}t, ebenſo wie ich, Do{\geminationn}erſtg oder Freitag in
                  \textcolor{pink}{Wien}{}\ledrightnote{\textcolor{pink}{Wien}} an – pardon – \uline{will} ankommen – ebenſo wie \uline{ich will}; er wird
               etwa 8 Tage bei mir wohnen. Ich denke, Sie {\pb}werden auch nicht mehr lang da unten oder da oben bleiben? Nun jedenfalls richten
               Sie ſichs wohl ſo ein, dſs Sie \strikeout{Rich}{ }\textcolor{blue}{Paul}{}\ledrightnote{\textcolor{blue}{Paul Goldmann}} noch in
                  \textcolor{pink}{Wien}{}\ledrightnote{\textcolor{pink}{Wien}} antreffen –?\pend
           \pstart
           Ich habe geſtern dem \textcolor{blue}{Brahm}{}\ledrightnote{\textcolor{blue}{Otto Brahm}} die \textcolor{green}{\textsc{Beatrice}}{}\ledrightnote{\textcolor{green}{Der Schleier der Beatrice. Schauspiel in fünf Akten}}, mit guter Wirkung,
               glaub ich, vorgeleſen. Er hat kaum gemerkt, wie viel ich noch dran zu machen habe.
               Die ungeſtrichene Aufführg würde fünf Stunden {\pb}dauern.\pend
           \pstart
           Ihre Ermahnung kam zu ſpät – ich hatte \textcolor{blue}{Brahm}{}\ledrightnote{\textcolor{blue}{Otto Brahm}}{ }ſchon eine »beſſere Meinung« beigebracht. So grüßt er Sie alſo weiter, \textcolor{blue}{\textsc{Kerr}}{}\ledrightnote{\textcolor{blue}{Alfred Kerr}} desgleichen.\pend
           \pstart
           – Hier friert man bereits un\textcolor{gray}{d} heizt ein und friert trotzdem.\pend
           \pstart
           Leben Sie wohl und erlauben Sie mir mich auf die \textcolor{green}{unſelige Mitgift}{}\ledrightnote{\textcolor{green}{Der Graf von Charolais. Ein Trauerspiel}} zu freuen.\pend
           \pstart
           Herzlichſt Ihr{\\[\baselineskip]}\spacefill\mbox{Arthur}\pend
           \leftskip=0em{}\endnumbering\briefempfaengerindex{Beer-Hofmann, Richard@\textsc{Beer-Hofmann, Richard}!zzzSchnitzler, Arthur@\emph{von Arthur Schnitzler}!1899-10-081@{8. 10. 1899}|)be}\mylabel{h}  \normalsize

\doendnotes{C}
\bigskip
\vfill

\clearpage

\footnotesize

\lohead{\textsc{register}}

% Definiere theindex-Environment komplett neu ohne reledmac
\makeatletter
\renewenvironment{theindex}{%
  \section*{\indexname}%
  \setlength{\parindent}{0pt}%
  \setlength{\parskip}{0pt plus 0.3pt}%
  \let\item\@idxitem
}{%
  \clearpage
}
\makeatother

\IfFileExists{\jobname-pw.ind}{\input{\jobname-pw.ind}}{}

\end{document}

      