%% latex-korrekturansicht-vorspann.tex
%% Vorspann für die Korrekturansicht.
%% Lädt die gemeinsame Datei latex-vorspann.tex mit gesetztem Schalter.

\newif\ifkorrekturansicht
\korrekturansichttrue

\input{../tex-inputs/latex-vorspann}


               \section[Arthur Schnitzler an Frank Wedekind, 19. 7. 1913]{ Arthur Schnitzler an Frank Wedekind, 19. 7. 1913}\nopagebreak\mylabel{v}\rehead{ }\normalsize\beginnumbering\briefempfaengerindex{Wedekind, Frank@\textsc{Wedekind, Frank}!zzzSchnitzler, Arthur@\emph{von Arthur Schnitzler}!1913-07-193@{19. 7. 1913}|(be} \toendnotes[C]{\smallbreak\pagebreak[2]} \Standort{München, Monacensia, FW B 159.}
\physDesc{Briefkarte
\newline{}Handschrift: schwarze Tinte, deutsche Kurrent\newline{}Ordnung: 1) mit blauem Buntstift von unbekannter Hand datiert: »Aug. 13« 2) Lochung}\buchAbdrucke{\weitereDrucke{Peter Michael Braunwarth: \emph{In Reife und Überreife.} In: \emph{Die Presse}, 24. 9. 2004, Sec. Spectrum, S. IV.} }\toendnotes[C]{\smallbreak}\pstart
           \raggedleft{}19/7 91\textcolor{gray}{3}\pend
           \pstart
           {\pb}\textcolor{gray}{\textbf{Dr. Arthur Schnitzler}}{\\}\textcolor{gray}{\textbf{\textcolor{pink}{Wien XVIII. Sternwartestrasse 71}{}\ledrightnote{\textcolor{pink}{Sternwartestraße}}}}\pend
           \pstart{}verehrter Herr Wedekind,\pend\pstart
           erſt heute, da bei uns alles wieder in Ordnung iſt und wir uns zur Abreiſe
                    rüſten, dank ich Ihnen für Ihre lieben theilnahmsvollen Zeilen, die Sie
                    anläßlich der Erkrankung unſeres \textcolor{blue}{Sohnes}{}\ledrightnote{→\textcolor{blue}{Heinrich Schnitzler}} an uns gerich{\pb}tet haben. Glücklicherweiſe iſt die
                    Sache von Anfang an leicht verlaufen, und wir hatten mehr Unannehmlichkeiten als
                    Sorgen.\pend
           \pstart
           Sie, mein ſehr verehrter lieber Herr Wedekind u Ihre \substVorne{}\textsuperscript{li}\substDazwischen{}verehrte\substHinten{}{ }\textcolor{blue}{Gattin}{}\ledrightnote{→\textcolor{blue}{Tilly Wedekind}} bei guter
                    Gelegenheit wiederzuſehen hoffen meine Frau u ich von Herzen. Wie ſchade daſs
                    wir diesmal Sie beide und »\textcolor{green}{Franziska}{}\ledrightnote{\textcolor{green}{Franziska}}« verſäumt
                    haben!\pend
           \pstart Viele Grüße von Ihrem \spacefill\mbox{Arthur Schnitzler}\pend{}\endnumbering\briefempfaengerindex{Wedekind, Frank@\textsc{Wedekind, Frank}!zzzSchnitzler, Arthur@\emph{von Arthur Schnitzler}!1913-07-193@{19. 7. 1913}|)be}\mylabel{h}  \normalsize

\doendnotes{C}
\bigskip
\vfill

\clearpage

\footnotesize

\lohead{\textsc{register}}

% Definiere theindex-Environment komplett neu ohne reledmac
\makeatletter
\renewenvironment{theindex}{%
  \section*{\indexname}%
  \setlength{\parindent}{0pt}%
  \setlength{\parskip}{0pt plus 0.3pt}%
  \let\item\@idxitem
}{%
  \clearpage
}
\makeatother

\IfFileExists{\jobname-pw.ind}{\input{\jobname-pw.ind}}{}

\end{document}

      