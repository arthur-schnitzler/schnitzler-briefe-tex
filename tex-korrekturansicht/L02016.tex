%% latex-korrekturansicht-vorspann.tex
%% Vorspann für die Korrekturansicht.
%% Lädt die gemeinsame Datei latex-vorspann.tex mit gesetztem Schalter.

\newif\ifkorrekturansicht
\korrekturansichttrue

\input{../tex-inputs/latex-vorspann}


               \section[Georg Brandes an Arthur Schnitzler, 11. 4. 1911]{ Georg Brandes an Arthur Schnitzler, 11. 4. 1911}\nopagebreak\mylabel{v}\rehead{ }\normalsize\beginnumbering\briefempfaengerindex{Schnitzler, Arthur@\textsc{Schnitzler, Arthur}!zzzBrandes, Georg@\emph{von Georg Brandes}!1911-04-111@{11. 4. 1911}|(be} \toendnotes[C]{\smallbreak\pagebreak[2]} \Standort{CUL, Schnitzler, B 17.}
\physDesc{Postkarte
\newline{}Handschrift: schwarze Tinte, lateinische Kurrent\newline{}Versand: Stempel: »\nobreak{}\oindex{Kopenhagen@\textbf{Kopenhagen}, \emph{Besiedelter Ort (A.BSO)}|pwk}Kjøbenhavn, 11. 4. 11., 10–11¾E\nobreak{}«.  \newline{}Ordnung: mit Bleistift von unbekannter Hand nummeriert: »35« }\buchAbdrucke{\weitereDrucke{Georg Brandes, Arthur Schnitzler: \emph{Ein Briefwechsel}. Hg. Kurt Bergel. Bern: \emph{Francke} 1956, S. 101.} }\toendnotes[C]{\smallbreak}\pstart{}{\pb}Herrn Dr. Arthur
                        Schnitzler\pend{}\pstart{}\textcolor{pink}{Sternwartestrasse 71}{}\ledrightnote{\textcolor{pink}{Sternwartestraße}}\pend{}\pstart{}\textcolor{pink}{Wien XVIII}{}\ledrightnote{\textcolor{pink}{XVIII., Währing}}\pend{}{\bigskip}\pstart
           \raggedleft{}{\pb}\textcolor{pink}{Kopenhagen}{}\ledrightnote{\textcolor{pink}{Kopenhagen}} (\uline{nicht}{ }\textcolor{pink}{Havnegade}{}\ledrightnote{\textcolor{pink}{Havnegade}})\pend
           \pstart{}Verehrter Herr und Freund.\pend\pstart
           Heute schickte ich Ihnen eine \textcolor{green}{Bagatelle}{}\ledrightnote{→\textcolor{green}{Det kgl. Teater [Der Schleier der Pierrette]}} die ich über Ihr hier aufgeführtes \textcolor{green}{Ballet}{}\ledrightnote{→\textcolor{green}{Der Schleier der Pierrette}} geschrieben habe und legte eine
                        \label{K_L02016_1v}\edtext{andere Bagatelle}{\lemma{\textnormal{\emph{andere Bagatelle}}}\Cendnote{\textnormal{nicht ermittelt}}}\label{K_L02016_1h} anbei. In
                    deutscher Sprache habe ich sonst Nichts. In \textcolor{pink}{Deutschland}{}\ledrightnote{\textcolor{pink}{Deutschland}} habe ich nicht einmal mehr einen Verleger. Ich gab in
                    diesen Tagen eine \textcolor{green}{Broschüre}{}\ledrightnote{→\textcolor{green}{Før og nu. To tragiske Skaebner}}
                    heraus, aber Sie lesen ja leider nicht \textcolor{pink}{Dänisch}{}\ledrightnote{\textcolor{pink}{Dänemark}}.\pend
           \pstart
           Ihr grosser Brief machte mir Freude. Wie schön dass es Ihnen endlich gut geht.
                    Nur die Schwerhörigkeit gefällt mir gar nicht. Es ist lumpig von den höheren
                    Mächten, mit Solchem sich schadlos zu halten.\pend
           \pstart
           Mir geht es nicht eben strahlend, aber ich bin nicht krank. Adresse von jetzt bis
                    weiter \textcolor{pink}{Hotel Lutetia}{}\ledrightnote{\textcolor{pink}{Hôtel Lutetia}}, \textcolor{pink}{Boulevard Raspail, Paris}{}\ledrightnote{\textcolor{pink}{Boulevard Raspail}}.\pend
           \pstart
           Ich drücke Ihre Hand in alter Freundschaft.\pend
           \pstart
           Ihr ergebener{\\[\baselineskip]}\spacefill\mbox{Georg Brandes}\pend
           \leftskip=0em{}\endnumbering\briefempfaengerindex{Schnitzler, Arthur@\textsc{Schnitzler, Arthur}!zzzBrandes, Georg@\emph{von Georg Brandes}!1911-04-111@{11. 4. 1911}|)be}\mylabel{h}  \normalsize

\doendnotes{C}
\bigskip
\vfill

\clearpage

\footnotesize

\lohead{\textsc{register}}

% Definiere theindex-Environment komplett neu ohne reledmac
\makeatletter
\renewenvironment{theindex}{%
  \section*{\indexname}%
  \setlength{\parindent}{0pt}%
  \setlength{\parskip}{0pt plus 0.3pt}%
  \let\item\@idxitem
}{%
  \clearpage
}
\makeatother

\IfFileExists{\jobname-pw.ind}{\input{\jobname-pw.ind}}{}

\end{document}

      