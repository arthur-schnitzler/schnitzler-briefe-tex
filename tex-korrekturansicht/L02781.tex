%% latex-korrekturansicht-vorspann.tex
%% Vorspann für die Korrekturansicht.
%% Lädt die gemeinsame Datei latex-vorspann.tex mit gesetztem Schalter.

\newif\ifkorrekturansicht
\korrekturansichttrue

\input{../tex-inputs/latex-vorspann}


               \section[Paul Goldmann an Arthur Schnitzler, Paul Goldmann an Arthur Schnitzler, 14. 7. {[}1896{]}]{ Paul Goldmann an Arthur Schnitzler, 14. 7. {[}1896{]}}\nopagebreak\mylabel{v}\rehead{ }\normalsize\beginnumbering\briefempfaengerindex{Schnitzler, Arthur@\textsc{Schnitzler, Arthur}!zzzGoldmann, Paul@\emph{von Paul Goldmann}!1896-07-141@{14. 7. {[}1896{]}}|(be} \toendnotes[C]{\smallbreak\pagebreak[2]} \Standort{DLA, A:Schnitzler, HS.NZ85.1.3166.}
\physDesc{Brief, 2 Blätter, 7 Seiten
\newline{}Handschrift: blaue Tinte, deutsche Kurrent
\newline{}Schnitzler: 1) mit Bleistift das Jahr »96« vermerkt 2) mit rotem Buntstift zwei Unterstreichungen}\toendnotes[C]{\smallbreak}\pstart
           \noindent{}{\pb}\textcolor{gray}{\textbf{\textbf{\textcolor{brown}{Frankfurter Zeitung}{}\ledrightnote{\textcolor{brown}{Frankfurter Zeitung}}}}}\pend
           \pstart
           \textcolor{gray}{\textbf{(\textcolor{brown}{\begin{otherlanguage}{french}Gazette de Francfort\end{otherlanguage}}{}\ledrightnote{\textcolor{brown}{Frankfurter Zeitung}}).}}\pend
           \pstart
           \textcolor{gray}{\textbf{\textbf{\begin{otherlanguage}{french}Fondateur M.\end{otherlanguage}{ }\textcolor{blue}{L. Sonnemann}{}\ledrightnote{\textcolor{blue}{Leopold Sonnemann}}.}}}\pend
           \pstart
           \begin{otherlanguage}{french}\textcolor{gray}{\textbf{\textcolor{green}{Journal}{}\ledrightnote{→\textcolor{green}{Frankfurter Zeitung}} politique,
                        financier,}}\end{otherlanguage}\pend
           \pstart
           \begin{otherlanguage}{french}\textcolor{gray}{\textbf{commercial et littéraire.}}\end{otherlanguage}\pend
           \pstart
           \begin{otherlanguage}{french}\textcolor{gray}{\textbf{\textbf{Paraissant trois fois par jour.}}}\end{otherlanguage}\pend
           \pstart
           \begin{otherlanguage}{french}\textcolor{gray}{\textbf{\textbf{Bureau à \textcolor{pink}{Paris}{}\ledrightnote{\textcolor{pink}{Paris}}}}}\end{otherlanguage}\hfill \textsc{\textcolor{pink}{Paris}{}\ledrightnote{\textcolor{pink}{Paris}}}, 14. Juli.\pend
           \pstart
           \begin{otherlanguage}{french}\textcolor{gray}{\textbf{\textbf{\textcolor{pink}{24. Rue Feydeau}{}\ledrightnote{\textcolor{pink}{rue Feydeau}}.}}}\end{otherlanguage}\pend
           \pstart\center{}Mein lieber Freund,\pend\pstart
           Da Du mir ſchreibſt, daß \textcolor{pink}{Norwegen}{}\ledrightnote{\textcolor{pink}{Norwegen}} wirklich
               exiſtirt, muß ichs wohl glauben und ſchreibe Dir nach \label{K_L02781-1v}\edtext{\textsc{\textcolor{pink}{Christiania}{}\ledrightnote{→\textcolor{pink}{Oslo}}}}{\lemma{\textnormal{\emph{Christiania}}}\Cendnote{\textnormal{\textcolor{blue}{Schnitzler} kam erst am 24. 7. 1896 in \textcolor{pink}{Christiania} (d. i. \textcolor{pink}{Oslo}) an, las den Brief also vermutlich erst
                  rund zehn Tage später.}}}\label{K_L02781-1h}, welches ſich hoffentlich an Ort und Stelle auch
               wirklich als die \textcolor{pink}{Hauptſtadt}{}\ledrightnote{→\textcolor{pink}{Oslo}}
               dieſes unwahrſcheinlichen \textcolor{pink}{Land}{}\ledrightnote{→\textcolor{pink}{Norwegen}}es erweiſt.\pend
           \pstart
           Ich danke Dir für Deine lieben Nachrichten. Deine Karten athmen frohe Reiſeſtimmung,
               und ich freue mich deſſen.\pend
           \pstart
           {\pb}Nur möchte ich auch einmal etwas Genaueres über
               unſer \label{K_L02781-2v}\edtext{Zuſammentreffen}{\lemma{\textnormal{\emph{Zuſammentreffen}}}\Cendnote{\textnormal{siehe Paul Goldmann an Arthur Schnitzler, 29. 4. [1896]}}}\label{K_L02781-2h} wiſſen. Werden wir uns ſo zwiſchen erſtem und
                  fünftem Auguſt in \textcolor{pink}{Kopenhagen}{}\ledrightnote{\textcolor{pink}{Kopenhagen}} treffen? Ich weiß zwar noch immer nicht, wann und ob ich von hier
               fortkomme (Geld, Geld, Geld!), – auch kann es in dieſem \textcolor{pink}{Lande}{}\ledrightnote{→\textcolor{pink}{Frankreich}} während vierzehn Tagen ſtets \strikeout{ ſ\textcolor{gray}{p}\textcolor{gray}{×}\-\textcolor{gray}{×}\-\textcolor{gray}{×}\textcolor{gray}{iren}} paſſiren, daß Herr \label{K_L02781-3v}\edtext{\textsc{\textcolor{blue}{Felix Faure}{}\ledrightnote{\textcolor{blue}{Félix Faure}}}}{\lemma{\textnormal{\emph{Felix Faure}}}\Cendnote{\textnormal{\textcolor{pink}{franz}ösischer \textcolor{blue}{Präsident}
                     (1895–1899)}}}\label{K_L02781-3h} den Sonnenſtich bekommt oder der
                  \textcolor{blue}{Herzog von \textsc{\textcolor{pink}{Orleans}{}\ledrightnote{\textcolor{pink}{Orléans}}}}{}\ledrightnote{→\textcolor{blue}{Louis Philippe Robert d’Orléans, duc d’Orléans}} den Thron von \textcolor{pink}{Frankreich}{}\ledrightnote{\textcolor{pink}{Frankreich}}{ }{\pb}beſteigt – aber immerhin, wenn ich doch nach \textcolor{pink}{Dänemark}{}\ledrightnote{\textcolor{pink}{Dänemark}} käme, wäre es doch vielleicht nicht
               übel, \strikeout{f\textcolor{gray}{als}} falls wir uns dort treffen könnten, und zu dieſem Zweck müßte ich zunächſt
               einmal wiſſen, \label{K_L02781-4v}\edtext{wo Ihr ſeid}{\lemma{\textnormal{\emph{wo Ihr ſeid}}}\Cendnote{\textnormal{Zu diesem Zeitpunkt war \textcolor{blue}{Schnitzler} noch auf dem Schiff unterwegs und besuchte
                  diverse \textcolor{pink}{norweg}ische
                  Städte.}}}\label{K_L02781-4h}, was Ihr mir bisher mit anerkennenswerther Beharrlichkeit
               verſchwiegen habt.\pend
           \pstart
           Kürzlich wollte ich noch \textsc{\textcolor{blue}{Thorel}{}\ledrightnote{\textcolor{blue}{Jean Thorel}}} – der gegenwärtig bei \label{K_L02781-98v}\edtext{\textsc{\textcolor{blue}{Pierre Loti}{}\ledrightnote{\textcolor{blue}{Pierre Loti}}} an der \textcolor{pink}{ſpani}{}\ledrightnote{→\textcolor{pink}{Spanien}}ſchen
                  Grenze}{\lemma{\textnormal{\emph{Pierre … Grenze}}}\Cendnote{\textnormal{\textcolor{blue}{Loti} lebte seit 1892 in \textcolor{pink}{Hendaye}}}}\label{K_L02781-98h} iſt – zu \textsc{\textcolor{blue}{Antoine}{}\ledrightnote{\textcolor{blue}{André Antoine}}} ſchicken. Aber er meinte, mit \textsc{\textcolor{blue}{Antoine}{}\ledrightnote{\textcolor{blue}{André Antoine}}} ſei fürs Erſte {\pb}nichts zu machen, derſelbe ſei
               verrückter als je, habe keine Ahnung, was er wolle, und nehme als deutſche Stücke
               zunächſt nur \textsc{\textcolor{green}{Wallenstein}{}\ledrightnote{\textcolor{green}{Wallenstein}}} und \textsc{\textcolor{green}{Don Carlos}{}\ledrightnote{\textcolor{green}{Don Karlos, Infant von Spanien}}} in Ausſicht. Wenn man ihm glauben machen könnte, daß die »\textcolor{green}{Liebelei}{}\ledrightnote{\textcolor{green}{Liebelei. Schauspiel in drei Akten}}« von \textsc{\textcolor{blue}{Schiller}{}\ledrightnote{\textcolor{blue}{Friedrich von Schiller}}} wäre, ſo wäre die Sache ſofort erledigt; aber das wird ſchwer halten. Kurzum,
               wir müſſen bis zur »\label{K_L02781-5v}\edtext{\begin{otherlanguage}{french}\textsc{rentrée}\end{otherlanguage}}{\lemma{\textnormal{\emph{rentrée}}}\Cendnote{\textnormal{französisch: Rückkehr (nach der
                  Sommerpause)}}}\label{K_L02781-5h}« warten, und \textsc{\textcolor{blue}{Thorel}{}\ledrightnote{\textcolor{blue}{Jean Thorel}}} möchte inzwiſchen die \textcolor{green}{Überſetzung}{}\ledrightnote{→\textcolor{green}{Amourette. Pièce en trois actes}} anfertigen (Preis 5-600 \textsc{Francs}, – du
               verſtehſt?). {\pb}Wir reden darüber bald mündlich, ſo
               Gott will.\pend
           \pstart
           \strikeout{Sonſt} Vielen Dank für \textsc{\textcolor{blue}{\textcolor{green}{Altenberg}{}\ledrightnote{→\textcolor{green}{Wie ich es sehe}}}{}\ledrightnote{\textcolor{blue}{Peter Altenberg}}}! Ich habe die erſten \textcolor{green}{Seiten}{}\ledrightnote{→\textcolor{green}{Wie ich es sehe}} geleſen und weiß noch nicht recht, wo und wie? Manchmal \strikeout{\textcolor{gray}{man}} meint man, es ſei ein Dichter, manchmal meint man, es ſei \textsc{\textcolor{blue}{Hermann Bahr}{}\ledrightnote{→\textcolor{blue}{Hermann Bahr}}}. Aber jedenfalls leſe ich das \textcolor{green}{Buch}{}\ledrightnote{→\textcolor{green}{Wie ich es sehe}} zu Ende.\pend
           \pstart
           Auf Deiner \label{K_L02781-8v}\edtext{Karte}{\lemma{\textnormal{\emph{Karte}}}\Cendnote{\textnormal{Es dürfte sich um das gleiche Postkartenmotiv
                  handeln, wie jenes, das \textcolor{blue}{Schnitzler} am 9. 7. 1896 an \textcolor{blue}{Beer-Hofmann} sandte, siehe Arthur Schnitzler an Richard Beer-Hofmann, 9. 7. 1896}}}\label{K_L02781-8h} fand ich ein roth angeſtrichenes {\pb}Schiff,
               über dem ein\strikeout{\textcolor{gray}{e}} blaues Geſtirn ſchwebt, das in erklärender Unterſchrift den Beſchauer als
                  »\label{K_L02781-7v}\edtext{\begin{otherlanguage}{french}\textsc{soleil de minuit}\end{otherlanguage}}{\lemma{\textnormal{\emph{soleil de minuit}}}\Cendnote{\textnormal{französisch: Mitternachtssonne}}}\label{K_L02781-7h}«
               vorgeſtellt wird. Das Schiff iſt vor \strikeout{de\textcolor{gray}{m}} der Mitternachtsſonne vorgefahren, wie ein Hotel-Omnibus vor der Hausthür des
               Gaſthofes. Nicht genug damit, ſteht auch noch das \label{K_L02781-987v}\edtext{\textcolor{pink}{Nordcap}{}\ledrightnote{\textcolor{pink}{Nordkap}}}{\lemma{\textnormal{\emph{Nordcap}}}\Cendnote{\textnormal{\textcolor{blue}{Schnitzler} kam am 19. 7. 1896 an das \textcolor{pink}{Nordkap}}}}\label{K_L02781-987h} dabei. Herrgott, biſt Du ein Protz! {\dotsfour}\pend
           \pstart
           Blonde Kinder mit Märchenhaar! Das weckt {\pb}in meinem
               Herzen die Sehnſucht auf. Nur einmal ſolch’ ein Mädchen in die Arme ſchließen und
               hören, daß ſie mich liebt! Einmal nur, – raſch noch in der letzten Viertelſunde dieſer
               ſo ganz verlorenen Jugend! {\dotsfour}\pend
           \pstart
           Grüß’ Dich Gott, mein theurer Freund, und reiſe froh und glücklich!\pend
           \pstart
           Dein treuer {\\[\baselineskip]}\spacefill\mbox{Paul Goldmann}\pend
           \leftskip=0em{}\endnumbering\briefempfaengerindex{Schnitzler, Arthur@\textsc{Schnitzler, Arthur}!zzzGoldmann, Paul@\emph{von Paul Goldmann}!1896-07-141@{14. 7. {[}1896{]}}|)be}\mylabel{h}  \normalsize

\doendnotes{C}
\bigskip
\vfill

\clearpage

\footnotesize

\lohead{\textsc{register}}

% Definiere theindex-Environment komplett neu ohne reledmac
\makeatletter
\renewenvironment{theindex}{%
  \section*{\indexname}%
  \setlength{\parindent}{0pt}%
  \setlength{\parskip}{0pt plus 0.3pt}%
  \let\item\@idxitem
}{%
  \clearpage
}
\makeatother

\IfFileExists{\jobname-pw.ind}{\input{\jobname-pw.ind}}{}

\end{document}

      