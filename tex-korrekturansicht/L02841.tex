%% latex-korrekturansicht-vorspann.tex
%% Vorspann für die Korrekturansicht.
%% Lädt die gemeinsame Datei latex-vorspann.tex mit gesetztem Schalter.

\newif\ifkorrekturansicht
\korrekturansichttrue

\input{../tex-inputs/latex-vorspann}


               \section[ Paul Goldmann an Arthur Schnitzler, 7. 3. {[}1898{]}]{Paul Goldmann an Arthur Schnitzler, 7. 3. {[}1898{]}}\nopagebreak\mylabel{v}\rehead{ }\normalsize\beginnumbering\briefempfaengerindex{Schnitzler, Arthur@\textsc{Schnitzler, Arthur}!zzzGoldmann, Paul@\emph{von Paul Goldmann}!1898-03-072@{7. 3. {[}1898{]}}|(be} \toendnotes[C]{\smallbreak\pagebreak[2]} \Standort{DLA, A:Schnitzler, HS.NZ85.1.3168.}
\physDesc{Brief, 1 Blatt, 4 Seiten
\newline{}Handschrift: blaue Tinte, lateinische Kurrent
\newline{}Schnitzler: 1) mit Bleistift das Jahr »98« vermerkt 2) mit rotem Buntstift zwei Unterstreichungen}\toendnotes[C]{\smallbreak}\pstart
           \noindent{}{\pb}\textcolor{gray}{\textbf{\textbf{\textcolor{brown}{Frankfurter Zeitung}{}\ledrightnote{\textcolor{brown}{Frankfurter Zeitung}}}}}\pend
           \pstart
           \textcolor{gray}{\textbf{(\textcolor{brown}{\begin{otherlanguage}{french}Gazette de Francfort\end{otherlanguage}}{}\ledrightnote{\textcolor{brown}{Frankfurter Zeitung}}).}}\pend
           \pstart
           \textcolor{gray}{\textbf{\textbf{\begin{otherlanguage}{french}Fondateur M.\end{otherlanguage}{ }\textcolor{blue}{L. Sonnemann}{}\ledrightnote{\textcolor{blue}{Leopold Sonnemann}}.}}}\pend
           \pstart
           \begin{otherlanguage}{french}\textcolor{gray}{\textbf{Journal politique, financier,}}\end{otherlanguage}\pend
           \pstart
           \begin{otherlanguage}{french}\textcolor{gray}{\textbf{commercial et littéraire.}}\end{otherlanguage}\pend
           \pstart
           \begin{otherlanguage}{french}\textcolor{gray}{\textbf{\textbf{Paraissant trois fois par jour.}}}\end{otherlanguage}\hfill \textsc{\textcolor{pink}{Paris}{}\ledrightnote{\textcolor{pink}{Paris}}}, 7. März.\pend
           \pstart
           \begin{otherlanguage}{french}\textcolor{gray}{\textbf{\textbf{Bureau à \textcolor{pink}{Paris}{}\ledrightnote{\textcolor{pink}{Paris}}}}}\end{otherlanguage}\pend
           \pstart
           \begin{otherlanguage}{french}\textcolor{gray}{\textbf{\textbf{\textcolor{pink}{10 Rue de la Bourse}{}\ledrightnote{\textcolor{pink}{rue de la Bourse}}.}}}\end{otherlanguage}\pend
           \pstart\center{}Mein lieber Freund,\pend\pstart
           Ich ſchicke Dir \label{K_L02841-1v}\edtext{\textsc{\textcolor{blue}{Herzl}{}\ledrightnote{\textcolor{blue}{Theodor Herzl}}s}{ }\textcolor{green}{Feuilleton}{}\ledrightnote{→\textcolor{green}{Feuilleton. Carl-Theater. (»Freiwild«, Schauspiel von Arthur Schnitzler.)}}}{\lemma{\textnormal{\emph{Herzls Feuilleton}}}\Cendnote{\textnormal{siehe Paul Goldmann an Arthur Schnitzler, 28. 2. [1898]}}}\label{K_L02841-1h} zurück. Es hat mich \strikeout{recht} recht ſehr amüſirt.
               Mißgunſt, welche von Unverſtändniß ſo glücklich unterſtützt wird, daß ſie beinahe zum
               guten Glauben wird! Die »\textcolor{green}{größeren
                  Fragen}{}\ledrightnote{→\textcolor{green}{Feuilleton. Carl-Theater. (»Freiwild«, Schauspiel von Arthur Schnitzler.)}}« ſind Dir nicht zugänglich, mein armer Freund! Du lebſt und producirſt
               im Kleinen und ahnſt nicht, daß es hoch über dem Allen den \textsc{Zionismus} gibt. Wenn Du aber wiſſen willſt, wie man auf dem Theater etwas
               beweiſt mit »\textcolor{green}{geſchloſſenen und
                  wetterfeſten Gründen}{}\ledrightnote{→\textcolor{green}{Feuilleton. Carl-Theater. (»Freiwild«, Schauspiel von Arthur Schnitzler.)}}«, ſo kannſt Du {\pb}das aus
               dem »\textcolor{green}{neuen \textsc{Ghetto}}{}\ledrightnote{\textcolor{green}{Das neue Ghetto}}« lernen.\pend
           \pstart
           Geh’, kümmere Dich nicht um das, was ſo ein Schafskopf ſchreibt, und geh’ Du nur ruhig weiter Deinen Weg.
               Ich ſehe aus Deinem lieben Briefe, daß Du wieder arbeitsluſtig
               biſt und \textsc{voll} von Plänen ſteckſt. Sehr ſchön! Du kannſt
               Herrn \textsc{\textcolor{blue}{Herzl}{}\ledrightnote{\textcolor{blue}{Theodor Herzl}}} durch nichts einen größeren Schmerz zufügen, als dadurch, daß Du ein neues
               gutes Stück ſchreibſt. Ich fürchte, wir werden ihm dieſen Schmerz nicht erſparen
               können.\pend
           \pstart
           Mein \textcolor{brown}{Schiff}{}\ledrightnote{→\textcolor{brown}{Preussen}}s-Platz iſt
               genommen. Ab \textsc{\textcolor{pink}{Genua}{}\ledrightnote{\textcolor{pink}{Genua}}}, 5. \strikeout{An} April.
               Aber die \textcolor{brown}{Vertretung}{}\ledrightnote{→\textcolor{brown}{Frankfurter Zeitung}}s-Frage iſt
               nicht geregelt, und die {\pb}Sache kann ſich immer noch
               in letzter Stunde zerſchlagen.\pend
           \pstart
           Mir iſt recht unheimlich. Ich glaube, ich komme nicht lebendig zurück. Das wäre aber
               noch nicht ſo ſchlimm, wie die Furcht vor der neuen journaliſtiſchen Aufgabe, der ich \strikeout{\textcolor{gray}{×}}
               wohl kaum gewachſen ſein werde: In der Haſt einer Reiſe, in einem feindlichen Klima,
               unter ganz veränderten Lebens-Verhältniſſen Eindrücke von Ländern zu geben, \strikeout{\textcolor{gray}{×}}{ }\strikeout{f\textcolor{gray}{ü}r} von denen man auch nicht die
               leiſeſte Ahnung hat! Mir grauſt, und ich fürchte, ich werde ſehr enttäuſchen. Im
               Übrigen bin ich ſicher caput zu gehen. Ich komme durch tropiſche {\pb}Gegenden, und dicke Leute ſterben immer am
               Fieber.\pend
           \pstart
           Weißt Du, was ſchön wäre? Wenn Du ſo \strikeout{\textcolor{gray}{An}} Ende März nach \textcolor{pink}{Italien}{}\ledrightnote{\textcolor{pink}{Italien}} gingeſt und ſo um den 5. April herum
               auch \label{K_L02841-4v}\edtext{in 
               \textsc{\textcolor{pink}{Genua}{}\ledrightnote{\textcolor{pink}{Genua}}}
               wäreſt}{\lemma{\textnormal{\emph{in 
               Genua
               wäreſt}}}\Cendnote{\textnormal{nicht geschehen}}}\label{K_L02841-4h}! Ich
               möchte Dich gern noch einmal zum Abſchied umarmen!\pend
           \pstart
           Schreib’ mir bald noch einmal hierher; denn ich fahre vielleicht ſchon nächſte Woche
               nach \textcolor{pink}{Frankfurt}{}\ledrightnote{\textcolor{pink}{Frankfurt am Main}}.\pend
           \pstart
           Viele treue Grüße!\pend
           \pstart
           Dein {\\[\baselineskip]}\spacefill\mbox{Paul Goldm}\pend
           \leftskip=0em{}\pstart
           \noindent{}Schönen Grüß an Deine \textcolor{blue}{Freundin}{}\ledrightnote{→\textcolor{blue}{Marie Reinhard}}!\pend
           \endnumbering\briefempfaengerindex{Schnitzler, Arthur@\textsc{Schnitzler, Arthur}!zzzGoldmann, Paul@\emph{von Paul Goldmann}!1898-03-072@{7. 3. {[}1898{]}}|)be}\mylabel{h}\begin{anhang}\end{anhang}\normalsize

\doendnotes{C}
\bigskip
\vfill

\clearpage

\footnotesize

\lohead{\textsc{register}}

% Definiere theindex-Environment komplett neu ohne reledmac
\makeatletter
\renewenvironment{theindex}{%
  \section*{\indexname}%
  \setlength{\parindent}{0pt}%
  \setlength{\parskip}{0pt plus 0.3pt}%
  \let\item\@idxitem
}{%
  \clearpage
}
\makeatother

\IfFileExists{\jobname-pw.ind}{\input{\jobname-pw.ind}}{}

\end{document}

      