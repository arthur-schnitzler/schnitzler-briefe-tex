%% latex-korrekturansicht-vorspann.tex
%% Vorspann für die Korrekturansicht.
%% Lädt die gemeinsame Datei latex-vorspann.tex mit gesetztem Schalter.

\newif\ifkorrekturansicht
\korrekturansichttrue

\input{../tex-inputs/latex-vorspann}


               \section[Hugo von Hofmannsthal an Arthur Schnitzler, 12. 7. {[}1898{]}]{ Hugo von Hofmannsthal an Arthur Schnitzler, 12. 7. {[}1898{]}}\nopagebreak\mylabel{v}\rehead{ }\normalsize\beginnumbering\briefempfaengerindex{Schnitzler, Arthur@\textsc{Schnitzler, Arthur}!zzzHofmannsthal, Hugo von@\emph{von Hugo von Hofmannsthal}!1898-07-122@{12. 7. {[}1898{]}}|(be} \toendnotes[C]{\smallbreak\pagebreak[2]} \Standort{CUL, Schnitzler, B 43.}
\physDesc{Brief, 1 Blatt, 3 Seiten
\newline{}Handschrift: schwarze Tinte, deutsche Kurrent
\newline{}Schnitzler: mit Bleistift die Jahreszahl ergänzt: »98« \newline{}Ordnung: mit Bleistift von unbekannter Hand nummeriert:
                                    »117« }\buchAbdrucke{\weitereDrucke{Hugo von Hofmannsthal, Arthur Schnitzler: \emph{Briefwechsel}. Hg. Therese Nickl und Heinrich Schnitzler. Frankfurt am Main: \emph{S. Fischer} 1964, S. 105.} }\toendnotes[C]{\smallbreak}\pstart
           \raggedleft{}{\pb}\textsc{\textcolor{pink}{Czortków}{}\ledrightnote{\textcolor{pink}{Tschortkiw}}, 12. July}.\pend
           \pstart{}mein lieber Arthur\pend\pstart
           es thut mir ſo leid, daſs Sie ſchon wieder verſtimmter ſind als früher, ich kann
                    mirs faſt nicht erklären, wenn ich an Ihr Leben denk. Es thut mir ſo leid daſs
                    wir uns jetzt noch nicht ſehen können, vielleicht möcht’s dann ein biſſerl
                    beſſer werden. 
               {\pb}Wenn das die \textcolor{blue}{Glümer}{}\ledrightnote{\textcolor{blue}{Marie Glümer}} leſen möcht!\hspace*{2.5em}Dem \textcolor{blue}{Richard}{}\ledrightnote{\textcolor{blue}{Richard Beer-Hofmann}} hab ich einen
                    ſehr eindringlichen langen \label{K_L00818_1v}\edtext{Brief}{\lemma{\textnormal{\emph{Brief}}}\Cendnote{\textnormal{Brief vom
                            11. 7. 1898, abgedruckt in \textcolor{blue}{Hugo von Hofmannsthal}, \textcolor{blue}{Richard Beer-Hofmann}:
                                \emph{Briefwechsel}. Hg. Eugene Weber. Frankfurt am
                            Main: \emph{S. Fischer}{ }1972, S. 76–77.}}}\label{K_L00818_1h} geſchrieben, daſs er mit uns
                    kommen ſoll. Ich wär unausſprechlich froh, wenn das zuſammengienge.\hspace*{2.5em}Laſſen Sie mich nicht zu lang ohne irgend eine
                    Nachricht. Von {\pb}Herzen
                    Ihr\pend
           \pstart \spacefill\mbox{Hugo}\pend{}\endnumbering\briefempfaengerindex{Schnitzler, Arthur@\textsc{Schnitzler, Arthur}!zzzHofmannsthal, Hugo von@\emph{von Hugo von Hofmannsthal}!1898-07-122@{12. 7. {[}1898{]}}|)be}\mylabel{h}  \normalsize

\doendnotes{C}
\bigskip
\vfill

\clearpage

\footnotesize

\lohead{\textsc{register}}

% Definiere theindex-Environment komplett neu ohne reledmac
\makeatletter
\renewenvironment{theindex}{%
  \section*{\indexname}%
  \setlength{\parindent}{0pt}%
  \setlength{\parskip}{0pt plus 0.3pt}%
  \let\item\@idxitem
}{%
  \clearpage
}
\makeatother

\IfFileExists{\jobname-pw.ind}{\input{\jobname-pw.ind}}{}

\end{document}

      