%% latex-korrekturansicht-vorspann.tex
%% Vorspann für die Korrekturansicht.
%% Lädt die gemeinsame Datei latex-vorspann.tex mit gesetztem Schalter.

\newif\ifkorrekturansicht
\korrekturansichttrue

\input{../tex-inputs/latex-vorspann}


\renewcommand{\erwaehntePersonen}{Personen: Marie Brüll, Frieda Pollak, Felix Salten, Ottilie Salten, Olga Schnitzler, Albert Steinrück}
\renewcommand{\erwaehnteOrte}{Orte: Altaussee, Berghof, Fischerndorf, Salzkammergut, Unterach am Attersee, Wien}
\renewcommand{\erwaehnteWerke}{Werke: Kinder der Freude. Drei Einakter, Tagebuch}
\section[ Felix Salten an Arthur Schnitzler, 31. 7. 1916]{Felix Salten an Arthur Schnitzler, 31. 7. 1916}
\nopagebreak\mylabel{v}
\rehead{ }\normalsize\beginnumbering\briefempfaengerindex{Schnitzler, Arthur@\textsc{Schnitzler, Arthur}!zzzSalten, Felix@\emph{von Felix Salten}!1916-07-311@{31. 7. 1916}|(be}
\toendnotes[C]{\smallbreak\pagebreak[2]}\Standort{CUL, Schnitzler, B 89, B 2.}
\physDesc{Bildpostkarte, 585 Zeichen
\newline{}Handschrift: schwarze Tinte, lateinische Kurrent
\newline{}Versand: Stempel: »\nobreak{}\oindex{Unterach am Attersee@\textbf{Unterach am Attersee}, \emph{P.PPL}|pwk}Unterach am Attersee, 31. VII. 16\nobreak{}«.  
\newline{}Ordnung: 1) mit Bleistift von \textcolor{blue}{Frieda Pollak} (?) mit
                                 dem Buchstaben »A« (Abgeschrieben/Abschrift)
                                 gekennzeichnet  2) mit Bleistift von unbekannter Hand nummeriert: »286«}\toendnotes[C]{\smallbreak}\pstart{}{\pb}Herrn\pend{}\pstart{}D\textsuperscript{r} Arthur Schnitzler\pend{}\pstart{}\textcolor{pink}{Alt-Aussee}{}\ledrightnote{\textcolor{pink}{Altaussee}}\pend{}\pstart{}\textcolor{pink}{Fischerndorf}{}\ledrightnote{\textcolor{pink}{Fischerndorf}}\pend{}
{\bigskip}
\pstart
           \noindent{}\centering{}{\pb}\textcolor{gray}{\textbf{\textcolor{pink}{Salzkammergut}{}\ledrightnote{\textcolor{pink}{Salzkammergut}}. \textcolor{pink}{Berghof}{}\ledrightnote{\textcolor{pink}{Berghof}} bei \textcolor{pink}{Unterach}{}\ledrightnote{\textcolor{pink}{Unterach am Attersee}}.}}\pend
           
\pstart
           {\pb}Vielen Dank für Ihre liebe
                  \label{K_L03573-1v}\edtext{Karte}{\lemma{\textnormal{\emph{Karte}}}\Cendnote{\textnormal{nicht erhalten}}}\label{K_L03573-1h}, die ich hier vorfand. Ich bin erst vor
               wenigen Tagen gekommen und finde es \textcolor{pink}{hier}{}\ledrightnote{{$\rightarrow$}\textcolor{pink}{Unterach am Attersee}} wieder einmal herrlich schön. Sie sollten doch (endlich) \label{K_L03573-2v}\edtext{einmal mit \textcolor{blue}{Olga}{}\ledrightnote{\textcolor{blue}{Olga Schnitzler}} herüberkommen}{\lemma{\textnormal{\emph{einmal … herüberkommen}}}\Cendnote{\textnormal{Aus \textcolor{blue}{Schnitzler}s \emph{\textcolor{green}{Tagebuch}} geht nicht hervor, dass sie der Einladung Folge
                  leisteten.}}}\label{K_L03573-2h}. Zu arbeiten habe ich hier noch nicht begonnen. Meine \label{K_L03573-3v}\edtext{\textcolor{green}{Einakter}{}\ledrightnote{{$\rightarrow$}\textcolor{green}{Kinder der Freude. Drei Einakter}}}{\lemma{\textnormal{\emph{Einakter}}}\Cendnote{\textnormal{der Einakterzyklus \emph{\textcolor{green}{Kinder der Freude}}}}}\label{K_L03573-3h} sind in \textcolor{pink}{Wien}{}\ledrightnote{\textcolor{pink}{Wien}} fertig geworden und schon
               verschickt. Auch an Herrn \textcolor{blue}{Steinrück}{}\ledrightnote{\textcolor{blue}{Albert Steinrück}}, der es
               gewünscht hat. Wir beabsichtigen nächstens einmal \label{K_L03573-4v}\edtext{zur \textcolor{blue}{Marie Brüll}{}\ledrightnote{\textcolor{blue}{Marie Brüll}}
               hinüberzufahren und rechnen natürlich darauf, dabei Sie und Frau \textcolor{blue}{Olga}{}\ledrightnote{\textcolor{blue}{Olga Schnitzler}} zu sehen}{\lemma{\textnormal{\emph{zur … sehen}}}\Cendnote{\textnormal{nicht nachweisbar}}}\label{K_L03573-4h}. 
            \pend
           
\pstart
           Inzwischen viele herzliche Grüße von \textcolor{blue}{uns}{}\ledrightnote{{$\rightarrow$}\textcolor{blue}{Ottilie Salten}} zu Ihnen, {\\[\baselineskip]}Ihr {\\[\baselineskip]}\spacefill\mbox{Felix Salten}\pend
           \leftskip=0em{}\endnumbering\briefempfaengerindex{Schnitzler, Arthur@\textsc{Schnitzler, Arthur}!zzzSalten, Felix@\emph{von Felix Salten}!1916-07-311@{31. 7. 1916}|)be}\mylabel{h}  \normalsize

\doendnotes{C}
\bigskip
\vfill

\clearpage

\footnotesize

\lohead{\textsc{register}}

% Definiere theindex-Environment komplett neu ohne reledmac
\makeatletter
\renewenvironment{theindex}{%
  \section*{\indexname}%
  \setlength{\parindent}{0pt}%
  \setlength{\parskip}{0pt plus 0.3pt}%
  \let\item\@idxitem
}{%
  \clearpage
}
\makeatother

\IfFileExists{\jobname-pw.ind}{\input{\jobname-pw.ind}}{}

\end{document}

      