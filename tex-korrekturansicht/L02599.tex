%% latex-korrekturansicht-vorspann.tex
%% Vorspann für die Korrekturansicht.
%% Lädt die gemeinsame Datei latex-vorspann.tex mit gesetztem Schalter.

\newif\ifkorrekturansicht
\korrekturansichttrue

\input{../tex-inputs/latex-vorspann}


               \section[Arthur Schnitzler an Marie Herzfeld, 24. 1. 1908]{ Arthur Schnitzler an Marie Herzfeld, 24. 1. 1908}\nopagebreak\mylabel{v}\rehead{ }\normalsize\beginnumbering\briefempfaengerindex{Herzfeld, Marie@\textsc{Herzfeld, Marie}!zzzSchnitzler, Arthur@\emph{von Arthur Schnitzler}!1908-01-242@{24. 1. 1908}|(be} \toendnotes[C]{\smallbreak\pagebreak[2]} \Standort{Wien, Privatbesitz Reinhard Urbach, \emph{ohne Signatur}.}
\physDesc{Brief, 1 Blatt, 3 Seiten, Fotokopie
\newline{}Handschrift: schwarze Tinte, lateinische Kurrent\newline{}Zusatz: Das Original des Briefes ist verschollen. Evtl. könnte es sich
                                 beim Schreibmedium auch um blaue Tinte handeln. }\buchAbdrucke{\weitereDrucke{Marie Herzfeld: \emph{Briefe an Hugo von Hofmannsthal.} Mitgeteilt von Reinhard Urbach. In: \emph{Hofmannsthal-Blätter} (1971) Nr. 6, S. 442.} }\toendnotes[C]{\smallbreak}\pstart
           {\pb}\textcolor{gray}{\textbf{Dr. Arthur Schnitzler}}\hfill 24/1 908\pend
           \pstart
           \textcolor{gray}{\textbf{\textcolor{pink}{Wien XVIII. Spoettelgasse 7}{}\ledrightnote{\textcolor{pink}{Edmund-Weiß-Gasse}}.}}\pend
           \pstart{}verehrtes Fräulein,\pend\pstart
           ich danke Ihnen herzlich für Ihren liebenswürdgn Brief. Sie sind aber gewissenhaft!
               Es als Fehler einzubekennen, dass Sie mich nach meinem ersten Buch »verkannt« haben–!
               Dazu ist man ja geradezu verpflichtet. Ich glaube, ich habs selber auch gethan. Und
               thue es auch jetzt noch oft genug, in schlimmen Stunden (die einem in diesen schli{\geminationm}en Stunden selbst als die einsichtsvollen erscheinen.)
               Im übrigen, we{\geminationn} man die Wahl hätte zwischen verka{\geminationn}t und {\pb}»falsch
               gekannt« sein – ? Dies letztere passirt einem allerdings nach dem siebzehnten oder
               achtundzwanzigsten Buche eher als nach dem ersten. Und man erholt sich schwerer. Den
                  \textcolor{green}{Stein der Weisen}{}\ledrightnote{→\textcolor{green}{Die Frau des Weisen. Novelletten}} (den Sie
               schätzen) hab ich nicht gefunden und nicht geschrieben. Sie meinen das
               Novellettenbuch »\textcolor{green}{die Frau des Weisen}{}\ledrightnote{\textcolor{green}{Die Frau des Weisen. Novelletten}}«. Ich bin
               wohl vor dem Verdacht geschützt mich revanchiren zu wollen, we{\geminationn} ich Ihnen sage, verehrtes Fräulein, wie stark Ihr
                  \label{K_L02599-1v}\edtext{\textcolor{green}{Leonardobuch}{}\ledrightnote{→\textcolor{green}{Leonardo da Vinci. Der Denker, Forscher und Poet. Nach den veröffentlichten Handschriften}}}{\lemma{\textnormal{\emph{Leonardobuch}}}\Cendnote{\textnormal{\emph{\textcolor{green}{Leonardo da Vinci. Der Denker, Forscher und
                        Poet}}. Nach den veröffentlichten Handschriften. Auswahl, Übersetzung {\kaufmannsund} Einleitung von \textcolor{blue}{Marie Herzfeld}. \textcolor{pink}{Jena}: \emph{\textcolor{brown}{Eugen Diederichs Verlag}}{ }1904.}}}\label{K_L02599-1h} auf mich gewirkt hat. Ich benütze eben die Gelegenheit. Da wir
               einander leider nie begegnen, sind wir auf Gelegen{\pb}heiten angewiesen, um uns gegenseitig schmeichelhafte Dinge zu sagen. Und da Sie
               sogar meine Lyrik nicht ungelobt lassen (was ich als Originalitätshascherei auffasse)
               so müssen Sie es auch geduldg hinnehmen, dass ich mich Ihrer reizvollen \label{K_L02599-6v}\edtext{\textcolor{green}{\textcolor{blue}{Bang}{}\ledrightnote{\textcolor{blue}{Herman Bang}}{ }Silhouette}{}\ledrightnote{→\textcolor{green}{Hermann Bang. Eine Silhouette}}}{\lemma{\textnormal{\emph{Bang Silhouette}}}\Cendnote{\textnormal{\emph{\textcolor{green}{Hermann Bang. Eine Silhouette}} von \textcolor{blue}{Marie Herzfeld}. In: \emph{\textcolor{green}{Neue Freie
                        Presse}}, Nr. 15.590, 16. 1. 1908,
                     Morgenblatt, S. 1–2.}}}\label{K_L02599-6h} mit Vergnügen erinnere.\pend
           \pstart
           Mit herzlichem Gruß Ihr sehr ergebener{\\[\baselineskip]}\spacefill\mbox{Arthur Schnitzler}\pend
           \leftskip=0em{}\endnumbering\briefempfaengerindex{Herzfeld, Marie@\textsc{Herzfeld, Marie}!zzzSchnitzler, Arthur@\emph{von Arthur Schnitzler}!1908-01-242@{24. 1. 1908}|)be}\mylabel{h}  \normalsize

\doendnotes{C}
\bigskip
\vfill

\clearpage

\footnotesize

\lohead{\textsc{register}}

% Definiere theindex-Environment komplett neu ohne reledmac
\makeatletter
\renewenvironment{theindex}{%
  \section*{\indexname}%
  \setlength{\parindent}{0pt}%
  \setlength{\parskip}{0pt plus 0.3pt}%
  \let\item\@idxitem
}{%
  \clearpage
}
\makeatother

\IfFileExists{\jobname-pw.ind}{\input{\jobname-pw.ind}}{}

\end{document}

      