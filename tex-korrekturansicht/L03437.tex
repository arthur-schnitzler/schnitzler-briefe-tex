%% latex-korrekturansicht-vorspann.tex
%% Vorspann für die Korrekturansicht.
%% Lädt die gemeinsame Datei latex-vorspann.tex mit gesetztem Schalter.

\newif\ifkorrekturansicht
\korrekturansichttrue

\input{../tex-inputs/latex-vorspann}


\renewcommand{\erwaehntePersonen}{Personen: Felix Salten, Olga Schnitzler}
\renewcommand{\erwaehnteOrte}{Orte: Edmund-Weiß-Gasse 7, München, Wien, XVIII., Währing}
\renewcommand{\erwaehnteWerke}{}
\section[ Felix Salten an Arthur Schnitzler, 2{[}1{]}. 1. 1907]{Felix Salten an Arthur Schnitzler, 2{[}1{]}. 1. 1907}
\nopagebreak\mylabel{v}
\rehead{ }\normalsize\beginnumbering\briefempfaengerindex{Schnitzler, Arthur@\textsc{Schnitzler, Arthur}!zzzSalten, Felix@\emph{von Felix Salten}!1907-01-211@{2{[}1{]}. 1. 1907}|(be}
\toendnotes[C]{\smallbreak\pagebreak[2]}\Standort{CUL, Schnitzler, B 89, B 1.}
\physDesc{Bildpostkarte, 84 Zeichen
\newline{}Handschrift: schwarze Tinte, lateinische Kurrent
\newline{}Versand: 1) Stempel: »\nobreak{}\oindex{Muenchen@\textbf{München}, \emph{P.PPLA}|pwk}München 2 B. P., 22. 1. 07, 5–6 V.\nobreak{}«.   2) mit schwarzer Tinte von unbekannter Hand das Zustellrayon ergänzt:
                                    »18/1«
\newline{}Ordnung: mit Bleistift von unbekannter Hand nummeriert: »\substVorne{}\textsuperscript{2\textcolor{gray}{34}}\substDazwischen{}227a\substHinten{}« }\toendnotes[C]{\smallbreak}\pstart{}{\pb}Herrn D\textsuperscript{r} Arthur Schnitzler\pend{}\pstart{}\textcolor{pink}{Wien XVIII}{}\ledrightnote{\textcolor{pink}{XVIII., Währing}}.\pend{}\pstart{}\textcolor{pink}{Spöttelgaße 7}{}\ledrightnote{\textcolor{pink}{Edmund-Weiß-Gasse 7}}\pend{}
{\bigskip}
\pstart
           \noindent{}\centering{}{\pb}\textcolor{gray}{\textbf{\textcolor{pink}{München}{}\ledrightnote{\textcolor{pink}{München}}.}}\pend
           
\pstart
           {\pb}\label{K_L03437-1v}\edtext{herzliche Grüße}{\lemma{\textnormal{\emph{herzliche Grüße}}}\Cendnote{\textnormal{Da der Poststempel die Uhrzeit 5–6 Uhr früh angibt, ist es
                  wahrscheinlich, dass die Karte am Vortag verfasst
                  wurde.}}}\label{K_L03437-1h} Ihnen \textcolor{blue}{Beiden}{}\ledrightnote{{$\rightarrow$}\textcolor{blue}{Olga Schnitzler}}\pend
           
\pstart
           Ihr {\\[\baselineskip]}\spacefill\mbox{Salten}\pend
           \leftskip=0em{}\endnumbering\briefempfaengerindex{Schnitzler, Arthur@\textsc{Schnitzler, Arthur}!zzzSalten, Felix@\emph{von Felix Salten}!1907-01-211@{2{[}1{]}. 1. 1907}|)be}\mylabel{h}  \normalsize

\doendnotes{C}
\bigskip
\vfill

\clearpage

\footnotesize

\lohead{\textsc{register}}

% Definiere theindex-Environment komplett neu ohne reledmac
\makeatletter
\renewenvironment{theindex}{%
  \section*{\indexname}%
  \setlength{\parindent}{0pt}%
  \setlength{\parskip}{0pt plus 0.3pt}%
  \let\item\@idxitem
}{%
  \clearpage
}
\makeatother

\IfFileExists{\jobname-pw.ind}{\input{\jobname-pw.ind}}{}

\end{document}

      