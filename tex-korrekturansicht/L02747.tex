%% latex-korrekturansicht-vorspann.tex
%% Vorspann für die Korrekturansicht.
%% Lädt die gemeinsame Datei latex-vorspann.tex mit gesetztem Schalter.

\newif\ifkorrekturansicht
\korrekturansichttrue

\input{../tex-inputs/latex-vorspann}


               \section[Paul Goldmann an Arthur Schnitzler, 12. 9. {[}1895{]}]{ Paul Goldmann an Arthur Schnitzler, 12. 9. {[}1895{]}}\nopagebreak\mylabel{v}\rehead{ }\normalsize\beginnumbering\briefempfaengerindex{Schnitzler, Arthur@\textsc{Schnitzler, Arthur}!zzzGoldmann, Paul@\emph{von Paul Goldmann}!1895-09-122@{12. 9. {[}1895{]}}|(be} \toendnotes[C]{\smallbreak\pagebreak[2]} \Standort{DLA, A:Schnitzler, HS.NZ85.1.3165.}
\physDesc{Brief, 1 Blatt, 2 Seiten
\newline{}Handschrift: blaue Tinte, deutsche Kurrent
\newline{}Schnitzler: mit Bleistift das Jahr »95« vermerkt }\toendnotes[C]{\smallbreak}\pstart
           \noindent{}{\pb}\textcolor{gray}{\textbf{\textbf{\textcolor{brown}{Frankfurter Zeitung}{}\ledrightnote{\textcolor{brown}{Frankfurter Zeitung}}}}}\pend
           \pstart
           \textcolor{gray}{\textbf{(\textcolor{brown}{\begin{otherlanguage}{french}Gazette de Francfort\end{otherlanguage}}{}\ledrightnote{\textcolor{brown}{Frankfurter Zeitung}}). }}\pend
           \pstart
           \textcolor{gray}{\textbf{\textbf{\begin{otherlanguage}{french}Fondateur M. \textcolor{blue}{L.
                                 Sonnemann}{}\ledrightnote{\textcolor{blue}{Leopold Sonnemann}}\end{otherlanguage}.}}}\hfill \textsc{\textcolor{pink}{Paris}{}\ledrightnote{\textcolor{pink}{Paris}}}, 12. September.\pend
           \pstart
           \begin{otherlanguage}{french}\textcolor{gray}{\textbf{\textcolor{green}{Journal}{}\ledrightnote{→\textcolor{green}{Frankfurter Zeitung}} politique,
                        financier,}}\end{otherlanguage}\pend
           \pstart
           \begin{otherlanguage}{french}\textcolor{gray}{\textbf{commercial et littéraire.}}\end{otherlanguage}\pend
           \pstart
           \begin{otherlanguage}{french}\textcolor{gray}{\textbf{\textbf{Paraissant trois fois par jour.}}}\end{otherlanguage}\pend
           \pstart
           \begin{otherlanguage}{french}\textcolor{gray}{\textbf{\textbf{Bureau à \textcolor{pink}{Paris}{}\ledrightnote{\textcolor{pink}{Paris}}:}}}\end{otherlanguage}\pend
           \pstart
           \begin{otherlanguage}{french}\textcolor{gray}{\textbf{\textbf{\textcolor{pink}{24. Rue Feydeau}{}\ledrightnote{\textcolor{pink}{rue Feydeau}}.}}}\end{otherlanguage}\pend
           \pstart\center{}Mein lieber Freund,\pend\pstart
           Seit geſtern bin ich wieder in \textsc{\textcolor{pink}{Paris}{}\ledrightnote{\textcolor{pink}{Paris}}}, und all’ das Schöne der letzten Wochen iſt nicht mehr wahr. Es waren köſtliche
               Stunden mit Euch zuſammen, und mein Herz iſt noch warm \strikeout{\textcolor{gray}{×}} von all dem Lieben, das Ihr mir gegeben. Tauſend Dank dafür!\pend
           \pstart
           Hier will es gar nicht recht gehen. \strikeout{\textcolor{gray}{×}\-\textcolor{gray}{×}\-\textcolor{gray}{×}} Körper und Seele wollen nicht mehr in das bisherige Leben hinein, und ich muß
               alle Kraft zuſammennehmen, um mich zu überwinden.\pend
           \pstart
           {\pb}Bitte, ſchreib’ mir gleich, wie es mit dem \textcolor{brown}{Burgtheater}{}\ledrightnote{\textcolor{brown}{Burgtheater}} ſteht. Die letzte \label{K_L02747-2v}\edtext{\textcolor{green}{Correſpondenz von \textsc{\textcolor{blue}{Uhl}{}\ledrightnote{\textcolor{blue}{Friedrich Uhl}}}}{}\ledrightnote{→\textcolor{green}{?? [Brief aus Wien]}}}{\lemma{\textnormal{\emph{Correſpondenz von Uhl}}}\Cendnote{\textnormal{\textcolor{blue}{Friedrich Uhl}: \emph{\textcolor{green}{XXXX Brief aus Wien}}. In: \emph{\textcolor{green}{Frankfurter Zeitung}}, Jg. YYYY, Nr. YYYY,
                        YY. YY. 1895, S. YY–YY; vgl. Paul Goldmann an Arthur Schnitzler, 23. 9. [1895]}}}\label{K_L02747-2h} bei uns dürſte wohl günſtigen Einfluß haben durch die Drohung, Rechenſchaft
               am Ende des Jahres zu fordern.\pend
           \pstart
           \textcolor{blue}{\textsc{Wolff}}{}\ledrightnote{\textcolor{blue}{Theodor Wolff}} (vom »\textcolor{brown}{Berliner Tageblatt}{}\ledrightnote{\textcolor{brown}{Berliner Tageblatt}}«) erzählte mir, er
               habe in \textcolor{pink}{Berlin}{}\ledrightnote{\textcolor{pink}{Berlin}} jetzt gehört, daß Dein \textcolor{green}{Stück}{}\ledrightnote{→\textcolor{green}{Liebelei. Schauspiel in drei Akten}} unter den erſten \strikeout{d\textcolor{gray}{a}} darankommen ſolle.\pend
           \pstart
           Und ſchreibe mir, wie es Dir ſonſt geht?\pend
           \pstart
           Viele treue Grüße! {\\[\baselineskip]}Dein {\\[\baselineskip]}\spacefill\mbox{Paul Goldmann}\pend
           \leftskip=0em{}\pstart
           \noindent{}\textsc{\textcolor{blue}{Frischauer}{}\ledrightnote{\textcolor{blue}{Berthold Frischauer}}} kommt wirklich an \textsc{\textcolor{blue}{Herzl}{}\ledrightnote{\textcolor{blue}{Theodor Herzl}}s}{ }\label{K_L02747-1v}\edtext{Stelle}{\lemma{\textnormal{\emph{Stelle}}}\Cendnote{\textnormal{als \textcolor{pink}{Paris}er \textcolor{blue}{Korrespondent} der \emph{\textcolor{brown}{Neuen Freien Presse}}}}}\label{K_L02747-1h}.\pend
           \endnumbering\briefempfaengerindex{Schnitzler, Arthur@\textsc{Schnitzler, Arthur}!zzzGoldmann, Paul@\emph{von Paul Goldmann}!1895-09-122@{12. 9. {[}1895{]}}|)be}\mylabel{h}  \normalsize

\doendnotes{C}
\bigskip
\vfill

\clearpage

\footnotesize

\lohead{\textsc{register}}

% Definiere theindex-Environment komplett neu ohne reledmac
\makeatletter
\renewenvironment{theindex}{%
  \section*{\indexname}%
  \setlength{\parindent}{0pt}%
  \setlength{\parskip}{0pt plus 0.3pt}%
  \let\item\@idxitem
}{%
  \clearpage
}
\makeatother

\IfFileExists{\jobname-pw.ind}{\input{\jobname-pw.ind}}{}

\end{document}

      