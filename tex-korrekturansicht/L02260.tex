%% latex-korrekturansicht-vorspann.tex
%% Vorspann für die Korrekturansicht.
%% Lädt die gemeinsame Datei latex-vorspann.tex mit gesetztem Schalter.

\newif\ifkorrekturansicht
\korrekturansichttrue

\input{../tex-inputs/latex-vorspann}


               \section[Robert Adam an Arthur Schnitzler, 16. 5. 1917]{ Robert Adam an Arthur Schnitzler, 16. 5. 1917}\nopagebreak\mylabel{v}\rehead{ }\normalsize\beginnumbering\briefempfaengerindex{Schnitzler, Arthur@\textsc{Schnitzler, Arthur}!zzzAdam, Robert@\emph{von Robert Adam}!1917-05-161@{16. 5. 1917}|(be} \toendnotes[C]{\smallbreak\pagebreak[2]} \Standort{DLA, A:Schnitzler, HS.NZ85.1.4230,18.}
\physDesc{Brief, 1 Blatt, 3 Seiten
\newline{}Handschrift: schwarze Tinte, deutsche Kurrent
\newline{}Schnitzler: 1) mit Bleistift beschriftet: »\textsc{Adam}« 2) mit rotem Buntstift eine Unterstreichung}\Standort{Wien, Österreichische Nationalbibliothek, Cod.ser. 52.263, 192.}
\physDesc{Brief, maschinelle Abschrift
\newline{}Schreibmaschine}\toendnotes[C]{\smallbreak}\pstart
           \raggedleft{}{\pb}\textcolor{pink}{Wien}{}\ledrightnote{\textcolor{pink}{Wien}}, am 16. Mai 1917\pend
           \pstart\center{}Hochverehrter Herr Doktor!\pend\pstart
           Ich empfinde nachgerade ein gewiſſes Schamgefühl, da jede Mitteilung, die ich Ihnen
               über meine literariſchen Geſchicke zu machen habe, die von einem Mißerfolg iſt. Alſo
               ſeit Jahren und nun alſo auch heute.\pend
           \pstart
           Das \textcolor{brown}{Münchner Hoftheater}{}\ledrightnote{\textcolor{brown}{Königliche Hof- und Nationaltheater München}} hat den »\textcolor{green}{Neidhard}{}\ledrightnote{\textcolor{green}{Neidhard}}« abgelehnt »wegen verſchiedener Mängel im dramatiſchen
               Aufbau« – \strikeout{gegen die} für die ich ſelbſt, bei Gott,
               nicht blind bin – »und wegen allzugroßer Längen« – deren Beteiligung im Wege von
               Strichen ich allerdings vorgeſchlagen hatte. Den {\pb}\textcolor{blue}{Dramaturgen}{}\ledrightnote{→\textcolor{blue}{Gerhard Gutherz}} hat indeß »die an
               vielen Stellen aufleuchtende Poeſie und Lyrik (ein \label{K_L02260_1v}\edtext{\griechisch{ἓν διὰ δυοῖν}}{\lemma{\textnormal{\emph{ἓν διὰ δυοῖν}}}\Cendnote{\textnormal{altgriechisch: eins mit zwei; Ausdruck der
                  Rhetorik, bei dem ein neuer Begriff aus zwei Wörten gebildet wird, wie hier
                  »Poesie und Lyrik«}}}\label{K_L02260_1h}) »ebenſo wie der witzige, fein pointierte Dialog in den
               Zwiſchenſpielen« »ſtark gefeſſelt«. Schade, daß die Zwiſchenſpiele nicht abendfüllend
               sind!\pend
           \pstart
           \textcolor{green}{Da ſteh ich nun, ich armer Tor}{}\ledrightnote{→\textcolor{green}{Faust}},
               und bin entſchloſſen, das Ende des Krieges abzuwarten und damit das Herankommen einer
               Zeit, die der ſcheußlichen deutſchfeindlichen Geſinnung, deren meiner Anſicht nach
               der »\textcolor{green}{Neidhard}{}\ledrightnote{\textcolor{green}{Neidhard}}« voll iſt, verſtändnisvoller
               gegenüberſtehen dürfte als die \textcolor{blue}{Hindenburg}{}\ledrightnote{\textcolor{blue}{Paul von Hindenburg}}iſche.
               Oder ſoll ich das kühne Experiment wagen, den »\textcolor{green}{Neidhard}{}\ledrightnote{\textcolor{green}{Neidhard}}«, ſobald er wieder in meinen Händen iſt, neuerlich zuſammenzupacken
               und dem \textcolor{pink}{Burgtheater}{}\ledrightnote{\textcolor{pink}{Burgtheater}} mit der Verſicherung
               einzureichen, daß er dem chriſtlich-germaniſchen Schönheitsideal entſpricht? Da
               dieſes angefeindet {\pb}durch Nichtverwendung \textcolor{pink}{babylon}{}\ledrightnote{\textcolor{pink}{Babylon}}iſcher Motive negativ determiniert iſt, iſt’s
               ſehr wohl möglich, daß der antichriſtlich-antigermaniſche »\textcolor{green}{Neidhard}{}\ledrightnote{\textcolor{green}{Neidhard}}« ſeine volle Erfüllung bedeutet. Der Spaß wäre nicht ſo
               übel, und hätte ich nicht zu befürchten, daß in Folge des zu erwartenden Anſturms
               aller germaniſchen Chriſten und der dadurch bewirkten Ueberlaſtung des Lektors der
               arme »\textcolor{green}{Neidhard}{}\ledrightnote{\textcolor{green}{Neidhard}}« \strikeout{nie}
               weit über die bevorſtehende Wiedergeburt \textcolor{pink}{Öſterreichs}{}\ledrightnote{\textcolor{pink}{Österreich}} hinaus im Archive lagern bliebe, ich wagte wirklich gerne den
               Verſuch. –\pend
           \pstart
           Nehmen Sie, hochverehrter Herr Doktor, neuerlich meinen Dank für Ihre liebenswürdige
               Bemühung entgegen (wie geſagt, ich ſchäme mich meines unumbringbaren Pechs) und
               empfangen Sie die ergebenſten Grüße von Ihrem\pend
           \pstart \spacefill\mbox{Robert Adam}\pend{}\endnumbering\briefempfaengerindex{Schnitzler, Arthur@\textsc{Schnitzler, Arthur}!zzzAdam, Robert@\emph{von Robert Adam}!1917-05-161@{16. 5. 1917}|)be}\mylabel{h}  \normalsize

\doendnotes{C}
\bigskip
\vfill

\clearpage

\footnotesize

\lohead{\textsc{register}}

% Definiere theindex-Environment komplett neu ohne reledmac
\makeatletter
\renewenvironment{theindex}{%
  \section*{\indexname}%
  \setlength{\parindent}{0pt}%
  \setlength{\parskip}{0pt plus 0.3pt}%
  \let\item\@idxitem
}{%
  \clearpage
}
\makeatother

\IfFileExists{\jobname-pw.ind}{\input{\jobname-pw.ind}}{}

\end{document}

      