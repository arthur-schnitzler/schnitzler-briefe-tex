%% latex-korrekturansicht-vorspann.tex
%% Vorspann für die Korrekturansicht.
%% Lädt die gemeinsame Datei latex-vorspann.tex mit gesetztem Schalter.

\newif\ifkorrekturansicht
\korrekturansichttrue

\input{../tex-inputs/latex-vorspann}


\renewcommand{\erwaehntePersonen}{Personen: Felix Salten, Ottilie Salten, Olga Schnitzler}
\renewcommand{\erwaehnteOrte}{Orte: Semmering, Südbahnhotel, Wien, Österreich}
\renewcommand{\erwaehnteWerke}{}
\section[ Felix Salten an Arthur Schnitzler, 15. 2. 1909]{Felix Salten an Arthur Schnitzler, 15. 2. 1909}
\nopagebreak\mylabel{v}
\rehead{ }\normalsize\beginnumbering\briefempfaengerindex{Schnitzler, Arthur@\textsc{Schnitzler, Arthur}!zzzSalten, Felix@\emph{von Felix Salten}!1909-02-151@{15. 2. 1909}|(be}
\toendnotes[C]{\smallbreak\pagebreak[2]}\Standort{CUL, Schnitzler, B 89, B 1.}
\physDesc{Brief, 1 Blatt, 1 Seite, 462 Zeichen
\newline{}Handschrift: schwarze Tinte, lateinische Kurrent
\newline{}Schnitzler: mit Bleistift Vermerk: »\textsc{Salten}« 
\newline{}Ordnung: mit Bleistift von unbekannter Hand nummeriert: »248« }\toendnotes[C]{\smallbreak}
\pstart
           \noindent{}{\pb}\textcolor{gray}{\textbf{\textcolor{pink}{Südbahn-Hôtel}{}\ledrightnote{\textcolor{pink}{Südbahnhotel}}}}\pend
           
\pstart
           \textcolor{gray}{\textbf{\textcolor{pink}{Semmering}{}\ledrightnote{\textcolor{pink}{Semmering}}}}\pend
           
\pstart
           \textcolor{gray}{\textbf{\textcolor{pink}{Austria}{}\ledrightnote{\textcolor{pink}{Österreich}}}}\pend
           
\pstart
           \textcolor{gray}{\textbf{\textsc{TELEGRAMME:}}}\pend
           
\pstart
           \textcolor{gray}{\textbf{\textsc{\textcolor{pink}{SÜDBAHNHÔTEL SEMMERING}{}\ledrightnote{\textcolor{pink}{Südbahnhotel}}}}}\pend
           
\pstart
           \textcolor{gray}{\textbf{\textsc{TELEPHON:}}}\pend
           
\pstart
           \textcolor{gray}{\textbf{\textsc{HÔTEL {\dotsfour} NR. 5.}}}\pend
           
\pstart
           \textcolor{gray}{\textbf{\textsc{DEPENDANCE . NR. 6.}}}\hfill 15. II. 09\pend
           
\pstart
           Lieber, wir wollen noch etwa acht bis zehn Tage bleiben, falls das
               Wetter weiter so herrlich ist und sonst nichts dazwischen kommt. Wenn ich Samstag ins Theater muß, fahre ich Sonntag früh wieder herauf. Wir wünschen sehr, dass Frau
                  \label{K_L03522-1v}\edtext{\textcolor{blue}{Olga}{}\ledrightnote{\textcolor{blue}{Olga Schnitzler}} recht bald wieder wol}{\lemma{\textnormal{\emph{Olga … wol}}}\Cendnote{\textnormal{siehe A. S.: \emph{Tagebuch}, 14. 2. 1909}}}\label{K_L03522-1h} ist, und dass Sie \textcolor{blue}{Beide}{}\ledrightnote{{$\rightarrow$}\textcolor{blue}{Olga Schnitzler}} noch \label{K_L03522-2v}\edtext{vor dem Sonntag hier sein}{\lemma{\textnormal{\emph{vor … sein}}}\Cendnote{\textnormal{nicht geschehen}}}\label{K_L03522-2h} können. Gestern waren noch Sportspiele da (übrigens sehr
                  schön){[},{]} dafür wird’s jetzt still. Alles Gute Ihrer \textcolor{blue}{Frau}{}\ledrightnote{{$\rightarrow$}\textcolor{blue}{Olga Schnitzler}} und herzliche Grüße von
                  \textcolor{blue}{uns}{}\ledrightnote{{$\rightarrow$}\textcolor{blue}{Ottilie Salten}} zu Ihnen\pend
           \pstart Ihr \spacefill\mbox{Salten}\pend{}\endnumbering\briefempfaengerindex{Schnitzler, Arthur@\textsc{Schnitzler, Arthur}!zzzSalten, Felix@\emph{von Felix Salten}!1909-02-151@{15. 2. 1909}|)be}\mylabel{h}  \normalsize

\doendnotes{C}
\bigskip
\vfill

\clearpage

\footnotesize

\lohead{\textsc{register}}

% Definiere theindex-Environment komplett neu ohne reledmac
\makeatletter
\renewenvironment{theindex}{%
  \section*{\indexname}%
  \setlength{\parindent}{0pt}%
  \setlength{\parskip}{0pt plus 0.3pt}%
  \let\item\@idxitem
}{%
  \clearpage
}
\makeatother

\IfFileExists{\jobname-pw.ind}{\input{\jobname-pw.ind}}{}

\end{document}

      