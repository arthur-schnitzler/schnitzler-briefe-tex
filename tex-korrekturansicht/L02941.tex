%% latex-korrekturansicht-vorspann.tex
%% Vorspann für die Korrekturansicht.
%% Lädt die gemeinsame Datei latex-vorspann.tex mit gesetztem Schalter.

\newif\ifkorrekturansicht
\korrekturansichttrue

\input{../tex-inputs/latex-vorspann}


         
         \renewcommand{\erwaehntePersonen}{Personen: François Bassompierre, Giovanni Boccaccio, Johann Wolfgang von Goethe, Hugo von Hofmannsthal}
         \renewcommand{\erwaehnteOrte}{Orte: Berlin, Dessauer Straße, Frankreich, Wien}
         \renewcommand{\erwaehnteWerke}{Werke: Das Erlebnis des Marschalls von Bassompierre, Decamerone, Der Schleier der Beatrice. Schauspiel in fünf Akten, Die Geschichte des Marschalls von Bassompierre, Die Zeit. Wiener Wochenschrift, Frankfurter Zeitung, Memoires du mareschal de Bassompierre, contenant l'histoire de sa vie et de ce qui s'est fait de plus remarquable à la cour de France pendant quelques années. 2 Bde., Unterhaltungen deutscher Ausgewanderten}
               \section[ Paul Goldmann an Arthur Schnitzler, 1. 12. {[}1900{]}]{Paul Goldmann an Arthur Schnitzler, 1. 12. {[}1900{]}}\nopagebreak\mylabel{v}\rehead{ }\normalsize\beginnumbering\briefempfaengerindex{Schnitzler, Arthur@\textsc{Schnitzler, Arthur}!zzzGoldmann, Paul@\emph{von Paul Goldmann}!1900-12-012@{1. 12. {[}1900{]}}|(be} \toendnotes[C]{\smallbreak\pagebreak[2]} \Standort{DLA, A:Schnitzler, HS.NZ85.1.3170.}
\physDesc{Brief, 1 Blatt, 2 Seiten
\newline{}Handschrift: blaue Tinte, deutsche Kurrent\newline{}Beilage: ein Zeitungsausschnitt, beschnitten und aufgeklebt 
\newline{}Schnitzler: 1) mit Bleistift das Jahr »{[}1{]}900« vermerkt  2) mit rotem Buntstift eine Unterstreichung}\toendnotes[C]{\smallbreak}\pstart
           \noindent{}{\pb}\textcolor{pink}{Berlin}{}\ledrightnote{\textcolor{pink}{Berlin}}, 1. Dezember.\hfill \textcolor{pink}{\textcolor{gray}{\textbf{DESSAUERSTRASSE 19}}}{}\ledrightnote{\textcolor{pink}{Dessauer Straße}}\pend
           \pstart\center{}Mein lieber Freund,\pend\pstart
           Es iſt leider doch nicht gegangen. Ich muß \textcolor{pink}{hier}{}\ledrightnote{{$\rightarrow$}\textcolor{pink}{Berlin}} bleiben und kann Dich \label{K_L02941-1v}\edtext{heut{ }Abend}{\lemma{\textnormal{\emph{heut Abend}}}\Cendnote{\textnormal{zur Uraufführung von \emph{\textcolor{green}{Der Schleier der Beatrice}}}}}\label{K_L02941-1h} nur mit allen guten Wünſchen begleiten. Wenn Du dieſen Brief erhältſt, biſt
               Du hoffentlich wieder um einen \textcolor{green}{Erfolg}{}\ledrightnote{{$\rightarrow$}\textcolor{green}{Der Schleier der Beatrice. Schauspiel in fünf Akten}} reicher.\pend
           \pstart
           Beifolgenden \label{K_L02941-2v}\edtext{\textcolor{green}{Artikel}{}\ledrightnote{{$\rightarrow$}\textcolor{green}{Die Geschichte des Marschalls von Bassompierre}}}{\lemma{\textnormal{\emph{Artikel}}}\Cendnote{\textnormal{XXXX}}}\label{K_L02941-2h}, der Deinen Freund \textsc{\textcolor{blue}{Hoffmannsthal}{}\ledrightnote{\textcolor{blue}{Hugo von Hofmannsthal}}} betrifft, finde ich heut in der {\pb}»\textcolor{green}{Frankfurter
                  Zeitung}{}\ledrightnote{\textcolor{green}{Frankfurter Zeitung}}«.\pend
           \pstart
           Viele treue Grüße! {\\[\baselineskip]}{\pb}Dein {\\[\baselineskip]}\spacefill\mbox{Paul Goldmn.}\pend
           \leftskip=0em{}{\bigskip}\pstart
           \noindent{}{\pb}\textcolor{gray}{\textbf{= \textbf{[\textcolor{green}{Die
                        Geſchichte des Marſchalls von Baſſompierre}{}\ledrightnote{\textcolor{green}{Das Erlebnis des Marschalls von Bassompierre}}.]}}}\pend
           \pstart
           \textcolor{gray}{\textbf{Ein Vorkommniß, das in literariſchen Kreiſen von ſich reden
                  macht, verdient um der Perſonen willen, die daran betheiligt ſind, allgemeinere
                  Beachtung. Die dieswöchentliche \textcolor{pink}{Wien}{}\ledrightnote{\textcolor{pink}{Wien}}er »\textcolor{green}{Zeit}{}\ledrightnote{\textcolor{green}{Die Zeit. Wiener Wochenschrift}}« enthält}}\textcolor{gray}{\textbf{den Anfang einer \textcolor{green}{Erzählung}{}\ledrightnote{{$\rightarrow$}\textcolor{green}{Das Erlebnis des Marschalls von Bassompierre}}, die betitelt iſt: »\textcolor{green}{\so{Erlebniß des Marſchalls von Baſſompierre}}{}\ledrightnote{\textcolor{green}{Das Erlebnis des Marschalls von Bassompierre}}« und als \so{Verfaſſer} nennt ſich der hochſtrebende
                     \textcolor{pink}{Wien}{}\ledrightnote{\textcolor{pink}{Wien}}er Poet \textcolor{blue}{\so{Hugo v. Hofmannsthal}}{}\ledrightnote{\textcolor{blue}{Hugo von Hofmannsthal}}. Dieſe \textcolor{green}{Erzählung}{}\ledrightnote{{$\rightarrow$}\textcolor{green}{Das Erlebnis des Marschalls von Bassompierre}}
                  behandelt nicht nur den nämlichen Vorfall, den in \textcolor{blue}{\so{Goethe}}{}\ledrightnote{\textcolor{blue}{Johann Wolfgang von Goethe}}’s »\textcolor{green}{\so{Unterhaltungen deutſcher Ausgewanderten}}{}\ledrightnote{\textcolor{green}{Unterhaltungen deutscher Ausgewanderten}}« \textcolor{green}{Vetter Karl}{}\ledrightnote{{$\rightarrow$}\textcolor{green}{Unterhaltungen deutscher Ausgewanderten}} auf dem
                     »\textcolor{green}{Gut am rechten Ufer des
                     Rheins}{}\ledrightnote{{$\rightarrow$}\textcolor{green}{Unterhaltungen deutscher Ausgewanderten}}« zum Beſten gibt, ſondern, obgleich ſie weit ausführlicher und
                  zufolge ihres näheren Eingehens ins Einzelne blühender iſt, als bei \textcolor{blue}{\textcolor{green}{Goethe}{}\ledrightnote{{$\rightarrow$}\textcolor{green}{Unterhaltungen deutscher Ausgewanderten}}}{}\ledrightnote{\textcolor{blue}{Johann Wolfgang von Goethe}}, der die Hauptvorgänge ſtraff zuſammenzufaſſen ſich begnügt, kann es keinem
                  Zweifel unterliegen, daß \textcolor{blue}{Beide}{}\ledrightnote{{$\rightarrow$}\textcolor{blue}{Johann Wolfgang von Goethe}{\newline}{$\rightarrow$}\textcolor{blue}{Hugo von Hofmannsthal}}, der \textcolor{blue}{Alte}{}\ledrightnote{{$\rightarrow$}\textcolor{blue}{Johann Wolfgang von Goethe}} wie der \textcolor{blue}{Junge}{}\ledrightnote{{$\rightarrow$}\textcolor{blue}{Hugo von Hofmannsthal}}, aus der gleichen Quellen geſchöpft haben. Und \textcolor{blue}{Beide}{}\ledrightnote{{$\rightarrow$}\textcolor{blue}{Johann Wolfgang von Goethe}{\newline}{$\rightarrow$}\textcolor{blue}{Hugo von Hofmannsthal}} lehnen
                  ſich ſo deutlich an das \label{K_L02941-34v}\edtext{\textcolor{pink}{fran}{}\ledrightnote{{$\rightarrow$}\textcolor{pink}{Frankreich}}zöſiſche \textcolor{green}{Original}{}\ledrightnote{{$\rightarrow$}\textcolor{green}{Memoires du mareschal de Bassompierre, contenant l'histoire de sa vie et de ce qui s'est fait de plus remarquable à la cour de France pendant quelques années. 2 Bde.}}}{\lemma{\textnormal{\emph{franzöſiſche Original}}}\Cendnote{\textnormal{gemeint sind \textcolor{blue}{François Bassompierre}s \emph{\textcolor{green}{Memoires du mareschal de Bassompierre}} (1665, 2 Bde.), wobei \textcolor{blue}{Goethe}s \textcolor{green}{Rahmenhandlung} an \textcolor{blue}{Giovanni
                        Boccaccio}s \emph{\textcolor{green}{Decamerone}} angelehnt
                     ist}}}\label{K_L02941-34h} an, daß ihre Schilderungen in ganzen Sätzen übereinſtimmen, aber
                  ſich auch untereinander im Ton des Vortrags außerordentlich ähneln. Daß \textcolor{blue}{Goethe}{}\ledrightnote{\textcolor{blue}{Johann Wolfgang von Goethe}}, in deſſen \textcolor{green}{Decamerone-Nachbildung}{}\ledrightnote{{$\rightarrow$}\textcolor{green}{Unterhaltungen deutscher Ausgewanderten}} das Abenteuer
                  des Marſchalls eine raſch vorübergehende Epiſode, gewiſſermaßen nur ein
                  nebenſächliches Illuſtrationsfaktum iſt, von {[}dem{]}{ }\textcolor{blue}{Hofmannsthal}{}\ledrightnote{\textcolor{blue}{Hugo von Hofmannsthal}} nichts gewußt hat, darf man
                  dreiſt vorausſetzen. Merkwürdig iſt nur, daß \textcolor{blue}{\so{dieſem}}{}\ledrightnote{{$\rightarrow$}\textcolor{blue}{Hugo von Hofmannsthal}} die Behandlung des Motivs durch \textcolor{blue}{Goethe}{}\ledrightnote{\textcolor{blue}{Johann Wolfgang von Goethe}} unbekannt geblieben iſt, denn wäre dies \so{nicht} der Fall geweſen, ſo hätte er doch ſicher auf die \textcolor{green}{Arbeit}{}\ledrightnote{{$\rightarrow$}\textcolor{green}{Unterhaltungen deutscher Ausgewanderten}} ſeines großen \textcolor{blue}{Vorgänger}{}\ledrightnote{{$\rightarrow$}\textcolor{blue}{Johann Wolfgang von Goethe}}s verwieſen. Noch merkwürdiger
                  iſt, daß ſich \textcolor{blue}{Hofmannsthal}{}\ledrightnote{\textcolor{blue}{Hugo von Hofmannsthal}} als \so{Verfaſſer} dieſer \textcolor{green}{Gedichte}{}\ledrightnote{{$\rightarrow$}\textcolor{green}{Das Erlebnis des Marschalls von Bassompierre}} bezeichnet, da, ſelbſt wenn die allerliebſten
                  Stimmungsſchilderungen der \textcolor{green}{Erzählung}{}\ledrightnote{{$\rightarrow$}\textcolor{green}{Das Erlebnis des Marschalls von Bassompierre}} ſein Eigenthum ſein ſollten, eine Hindeutung auf das \textcolor{green}{Originalwerk}{}\ledrightnote{{$\rightarrow$}\textcolor{green}{Memoires du mareschal de Bassompierre, contenant l'histoire de sa vie et de ce qui s'est fait de plus remarquable à la cour de France pendant quelques années. 2 Bde.}} unter keinen
                  Umſtänden zu vermeiden war. Die Zeit, wo man auf das Titelblatt von Komödien und
                  Proſaſchriften einfach zu ſchreiben pflegte: »\so{Nach dem Franzöſiſchen} von X. X.« ſind vorüber, aber ſelbſt damals benützte man die
                  Phraſe »Nach dem Franzöſiſchen«, um, wenn man ſchon den Autor nicht nennen wollte,
                  wenigſtens zuzugeſtehen, daß es ſich um keine Original-Arbeit handle. Da \textcolor{blue}{Hugo v. Hofmannsthal}{}\ledrightnote{\textcolor{blue}{Hugo von Hofmannsthal}} nicht nöthig hat, bei
                  fremden Autoren zu leihen, wäre eine \so{Aufklärung} des
                  Falles gewiß von Intereſſe.}}\pend
           \endnumbering\briefempfaengerindex{Schnitzler, Arthur@\textsc{Schnitzler, Arthur}!zzzGoldmann, Paul@\emph{von Paul Goldmann}!1900-12-012@{1. 12. {[}1900{]}}|)be}\mylabel{h}\begin{anhang}\end{anhang}\normalsize

\doendnotes{C}
\bigskip
\vfill

\clearpage

\footnotesize

\lohead{\textsc{register}}

% Definiere theindex-Environment komplett neu ohne reledmac
\makeatletter
\renewenvironment{theindex}{%
  \section*{\indexname}%
  \setlength{\parindent}{0pt}%
  \setlength{\parskip}{0pt plus 0.3pt}%
  \let\item\@idxitem
}{%
  \clearpage
}
\makeatother

\IfFileExists{\jobname-pw.ind}{\input{\jobname-pw.ind}}{}

\end{document}

      