%% latex-korrekturansicht-vorspann.tex
%% Vorspann für die Korrekturansicht.
%% Lädt die gemeinsame Datei latex-vorspann.tex mit gesetztem Schalter.

\newif\ifkorrekturansicht
\korrekturansichttrue

\input{../tex-inputs/latex-vorspann}


               \section[Thomas Mann an Arthur Schnitzler, 8. 12. 1912]{ Thomas Mann an Arthur Schnitzler, 8. 12. 1912}\nopagebreak\mylabel{v}\rehead{ }\normalsize\beginnumbering\briefempfaengerindex{Schnitzler, Arthur@\textsc{Schnitzler, Arthur}!zzzMann, Thomas@\emph{von Thomas Mann}!1912-12-081@{8. 12. 1912}|(be} \toendnotes[C]{\smallbreak\pagebreak[2]} \Standort{DLA, A:Schnitzler, HS.NZ85.1.3986, S. 7.}
\physDesc{Brief, maschinelle Abschrift
\newline{}Schreibmaschine\newline{}Zusatz: die Abschrift noch zu Lebzeiten
                                    Schnitzlers hergestellt }\toendnotes[C]{\smallbreak}\pstart
           \raggedleft{}{\pb}\textcolor{pink}{München}{}\ledrightnote{\textcolor{pink}{München}}, 8. 12. 1912\pend
           \pstart
           Verehrter Herr Doctor: Ihre gütigen Worte haben mir sehr wohlgetan. Ich danke
                    Ihnen herzlich, – zugleich auch für den »\textcolor{green}{Professor
                        Bernhardi}{}\ledrightnote{\textcolor{green}{Professor Bernhardi. Komödie in fünf Akten}}«, der mir in Ihrem Auftrage übersandt wurde und mir die
                    packendste unter Ihren dramatischen Gaben zu sein scheint. Ich wünsche nichts
                    eifriger, als dass er auch auf dem hiesigen Theater recht bald erscheinen möge.
                    Wenn Sie erlauben, teile ich Ihnen dann wieder meine Eindrücke mit.\pend
           \pstart
           Mit den besten Empfehlungen an Ihre \textcolor{blue}{Gattin}{}\ledrightnote{→\textcolor{blue}{Olga Schnitzler}}, verehrter Herr Doctor, stets Ihr\hspace*{1.5em}\spacefill\mbox{Thomas Mann.}\pend
           \endnumbering\briefempfaengerindex{Schnitzler, Arthur@\textsc{Schnitzler, Arthur}!zzzMann, Thomas@\emph{von Thomas Mann}!1912-12-081@{8. 12. 1912}|)be}\mylabel{h}  \normalsize

\doendnotes{C}
\bigskip
\vfill

\clearpage

\footnotesize

\lohead{\textsc{register}}

% Definiere theindex-Environment komplett neu ohne reledmac
\makeatletter
\renewenvironment{theindex}{%
  \section*{\indexname}%
  \setlength{\parindent}{0pt}%
  \setlength{\parskip}{0pt plus 0.3pt}%
  \let\item\@idxitem
}{%
  \clearpage
}
\makeatother

\IfFileExists{\jobname-pw.ind}{\input{\jobname-pw.ind}}{}

\end{document}

      