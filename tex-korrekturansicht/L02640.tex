%% latex-korrekturansicht-vorspann.tex
%% Vorspann für die Korrekturansicht.
%% Lädt die gemeinsame Datei latex-vorspann.tex mit gesetztem Schalter.

\newif\ifkorrekturansicht
\korrekturansichttrue

\input{../tex-inputs/latex-vorspann}


               \section[Paul Goldmann an Arthur Schnitzler, 18. 6. 1889]{ Paul Goldmann an Arthur Schnitzler, 18. 6. 1889}\nopagebreak\mylabel{v}\rehead{ }\normalsize\beginnumbering\briefempfaengerindex{Schnitzler, Arthur@\textsc{Schnitzler, Arthur}!zzzGoldmann, Paul@\emph{von Paul Goldmann}!1889-06-181@{18. 6. 1889}|(be} \toendnotes[C]{\smallbreak\pagebreak[2]} \Standort{DLA, A:Schnitzler, HS.NZ85.1.3162.}
\physDesc{Brief, 1 Blatt, 2 Seiten
\newline{}Handschrift: blaue Tinte, deutsche Kurrent}\toendnotes[C]{\smallbreak}\pstart
           \noindent{}\centering{}{\pb}\textcolor{gray}{\textbf{\textbf{Adminiſtration: \textcolor{pink}{VII.
                           Seidengaſſe 7}{}\ledrightnote{\textcolor{pink}{Seidengasse}}} (\textcolor{brown}{Jos. Eberle {\kaufmannsund} Co.}{}\ledrightnote{\textcolor{brown}{Josef Eberle  Stein-, Buch und Musikaliendruckerei}})}}\pend
           \pstart
           \noindent{}\centering{}\textcolor{gray}{\textbf{\textcolor{brown}{An der Schönen Blauen Donau}{}\ledrightnote{\textcolor{brown}{An der schönen blauen Donau}}}}\pend
           \pstart
           \noindent{}\centering{}\textcolor{gray}{\textbf{Chef-Redacteur: Dr. \textcolor{blue}{F. Mamroth}{}\ledrightnote{\textcolor{blue}{Fedor Mamroth}}. – Redaction: \textcolor{pink}{IX., Berggaſſe 31}{}\ledrightnote{\textcolor{pink}{Berggasse}}.}}\pend
           \pstart
           \raggedleft{}\textcolor{gray}{\textbf{\textcolor{pink}{Wien}{}\ledrightnote{\textcolor{pink}{Wien}}, den}}{ }18. Juni \textcolor{gray}{\textbf{18}}89.\pend
           \pstart\center{}Sehr geehrter Herr Doctor!\pend\pstart
           Die zwei vermißten \label{K_L02640-1v}\edtext{\textcolor{green}{Gedichte}{}\ledrightnote{→\textcolor{green}{Lieder eines Nervösen}}}{\lemma{\textnormal{\emph{Gedichte}}}\Cendnote{\textnormal{Unter dem Pseudonym »Anatol« und mit dem
                  Titel \emph{\textcolor{green}{Lieder eines Nervösen}} erschienen im
                  ersten Juli-Heft von \emph{\textcolor{green}{An der schönen blauen
                     Donau}} fünf Gedichte \textcolor{blue}{Schnitzler}s.
                     (Jg. 4, H. 13, S. 297). Welche davon kurzzeitig vermisst
                  waren, ist nicht geklärt.}}}\label{K_L02640-1h} und auch eine Anzahl anderer haben ſich bereits
               gefunden. Ich hatte dieſelben in jenes beſondere Fach unſeres Manuſkripten-Kaſtens
               gelegt, in dem die zum Setzen zu gebenden Beiträge aufbewahrt werden und ſofort,
               nachdem ich dies gethan, daran vergeſſen (wie ich dies mit {\pb}Vorliebe zu thun pflege). Die Sachen
               hätten ſich ohnedies dann bei den Vorabeiten für das nächſte \textcolor{green}{Heft}{}\ledrightnote{→\textcolor{green}{An der schönen blauen Donau}} wieder an’s Tageslicht
               emporgearbeitet. Es thut mir nur leid, daß ich Ihnen durch meine Zerſtreutheit einige
               Stunden der Sorge bereitet habe. Ich bitte Sie alſo, vollſtändig beruhigt \introOben{}zu\introOben{} ſein. Wenn Sie mir das nächſte Mal wieder das Vergnügen
               Ihres Beſuches machen werden, werden Sie die \textcolor{green}{Kinder ihrer Muſe}{}\ledrightnote{→\textcolor{green}{Lieder eines Nervösen}} friſch, geſund und unbeſchädigt von
               Angeſicht zu Angeſicht begrüßen können.\pend
           \pstart
           Hochachtungsvoll {\\[\baselineskip]}Ihr ergebner {\\[\baselineskip]}\spacefill\mbox{Dr. Paul Goldmann}\pend
           \leftskip=0em{}\endnumbering\briefempfaengerindex{Schnitzler, Arthur@\textsc{Schnitzler, Arthur}!zzzGoldmann, Paul@\emph{von Paul Goldmann}!1889-06-181@{18. 6. 1889}|)be}\mylabel{h}  \normalsize

\doendnotes{C}
\bigskip
\vfill

\clearpage

\footnotesize

\lohead{\textsc{register}}

% Definiere theindex-Environment komplett neu ohne reledmac
\makeatletter
\renewenvironment{theindex}{%
  \section*{\indexname}%
  \setlength{\parindent}{0pt}%
  \setlength{\parskip}{0pt plus 0.3pt}%
  \let\item\@idxitem
}{%
  \clearpage
}
\makeatother

\IfFileExists{\jobname-pw.ind}{\input{\jobname-pw.ind}}{}

\end{document}

      