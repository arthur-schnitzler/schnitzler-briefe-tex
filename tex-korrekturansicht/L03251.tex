%% latex-korrekturansicht-vorspann.tex
%% Vorspann für die Korrekturansicht.
%% Lädt die gemeinsame Datei latex-vorspann.tex mit gesetztem Schalter.

\newif\ifkorrekturansicht
\korrekturansichttrue

\input{../tex-inputs/latex-vorspann}


\renewcommand{\erwaehntePersonen}{Personen: Olga Schnitzler, Heinrich Schnitzler}
\renewcommand{\erwaehnteOrte}{Orte: Berlin, Grand Hotel Wien, Kärntnerring, Semmering, Wien}
\renewcommand{\erwaehnteWerke}{}
\section[ Paul Goldmann an Arthur Schnitzler, 18. 9. {[}1906{]}]{Paul Goldmann an Arthur Schnitzler, 18. 9. {[}1906{]}}
\nopagebreak\mylabel{v}
\rehead{ }\normalsize\beginnumbering\briefempfaengerindex{Schnitzler, Arthur@\textsc{Schnitzler, Arthur}!zzzGoldmann, Paul@\emph{von Paul Goldmann}!1906-09-181@{18. 9. {[}1906{]}}|(be}
\toendnotes[C]{\smallbreak\pagebreak[2]}\Standort{DLA, A:Schnitzler, HS.NZ85.1.3175.}
\physDesc{Brief, 1 Blatt, 2 Seiten
\newline{}Handschrift: schwarze Tinte, deutsche Kurrent
\newline{}Schnitzler: mit Bleistift das Jahr »{[}19{]}06« vermerkt }\toendnotes[C]{\smallbreak}
\pstart
           \noindent{}\raggedleft{}{\pb}\textcolor{gray}{\textbf{\textbf{\textcolor{pink}{GRAND HÔTEL}{}\ledrightnote{\textcolor{pink}{Grand Hotel Wien}}, \begin{otherlanguage}{french}\textcolor{pink}{VIENNE}{}\ledrightnote{\textcolor{pink}{Wien}}\end{otherlanguage}}}}\pend
           
\pstart
           \noindent{}\raggedleft{}\textcolor{gray}{\textbf{\textcolor{pink}{I., KÄRNTNERRING 9}{}\ledrightnote{\textcolor{pink}{Kärntnerring}}}}.\pend
           
\pstart
           18. \textsc{Sept}.\pend
           
\pstart{}Mein lieber Freund,\pend
\pstart
           Es thut mir unendlich leid, nicht gewußt zu haben, daß Du auf dem \label{K-L03251-1v}\edtext{\textsc{\textcolor{pink}{Semmering}{}\ledrightnote{\textcolor{pink}{Semmering}}}}{\lemma{\textnormal{\emph{Semmering}}}\Cendnote{\textnormal{\textcolor{blue}{Schnitzler} hielt sich zwischen 10. 9. 1906 und 20. 9. 1906 auf dem
                     \textcolor{pink}{Semmering} auf.}}}\label{K-L03251-1h} biſt. Denn ich bin
               über den \textsc{\textcolor{pink}{Semmering}{}\ledrightnote{\textcolor{pink}{Semmering}}} gefahren u. wäre gern ausgeſtiegen, um einen Tag mit Dir zu verbringen. Auch in
                  \textcolor{pink}{Wien}{}\ledrightnote{\textcolor{pink}{Wien}} werde ich Dich leider nicht ſehen, da ich
               vorausſichtlich übermorgen heimfahre.{\\}Deine liebe
                  \label{K-L03251-2v}\edtext{Karte mit den ſchönen Verſen}{\lemma{\textnormal{\emph{Karte … Verſen}}}\Cendnote{\textnormal{siehe A. S.: \emph{Tagebuch}, 5. 8. 1906}}}\label{K-L03251-2h} (wirklich, welch’ ein Talent!) iſt auch erſt vor Kurzem {\pb}in meinen Beſitz gekommen. Ich hätte manches darauf
               zu antworten – aber wozu? Es hat keinen Sinn, auch noch \textsc{privatim} zu polemiſiren. Ich werde mich lieber darauf beſchränken, Dein
               nächſtes Stück öffentlich ſchlecht zu machen. {\\}Im Ernſt: ich hätte Dir ſehr, ſehr
               gern die Hand gedrückt. Vielleicht gibſt Du mir im Laufe des Winters \label{K-L03251-3v}\edtext{Gelegenheit dazu}{\lemma{\textnormal{\emph{Gelegenheit dazu}}}\Cendnote{\textnormal{\textcolor{blue}{Schnitzler} und \textcolor{blue}{Goldmann} trafen sich erst am 24. 5. 1907 in \textcolor{pink}{Wien} wieder.}}}\label{K-L03251-3h} in \textcolor{pink}{Berlin}{}\ledrightnote{\textcolor{pink}{Berlin}}. \strikeout{\textcolor{gray}{×}\-\textcolor{gray}{×}}\pend
           
\pstart
           Inzwiſchen ſei ſamt \textcolor{blue}{Frau}{}\ledrightnote{{$\rightarrow$}\textcolor{blue}{Olga Schnitzler}}
               u. \textcolor{blue}{Kind}{}\ledrightnote{{$\rightarrow$}\textcolor{blue}{Heinrich Schnitzler}} herzlichſt gegrüßt
               von {\\[\baselineskip]}Deinem getreuen {\\[\baselineskip]}\spacefill\mbox{Paul Goldmann.}\pend
           \leftskip=0em{}\endnumbering\briefempfaengerindex{Schnitzler, Arthur@\textsc{Schnitzler, Arthur}!zzzGoldmann, Paul@\emph{von Paul Goldmann}!1906-09-181@{18. 9. {[}1906{]}}|)be}\mylabel{h}  \normalsize

\doendnotes{C}
\bigskip
\vfill

\clearpage

\footnotesize

\lohead{\textsc{register}}

% Definiere theindex-Environment komplett neu ohne reledmac
\makeatletter
\renewenvironment{theindex}{%
  \section*{\indexname}%
  \setlength{\parindent}{0pt}%
  \setlength{\parskip}{0pt plus 0.3pt}%
  \let\item\@idxitem
}{%
  \clearpage
}
\makeatother

\IfFileExists{\jobname-pw.ind}{\input{\jobname-pw.ind}}{}

\end{document}

      