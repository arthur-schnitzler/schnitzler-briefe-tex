%% latex-korrekturansicht-vorspann.tex
%% Vorspann für die Korrekturansicht.
%% Lädt die gemeinsame Datei latex-vorspann.tex mit gesetztem Schalter.

\newif\ifkorrekturansicht
\korrekturansichttrue

\input{../tex-inputs/latex-vorspann}


               \section[Arthur Schnitzler an Paul Goldmann, 22. 11. 1896]{ Arthur Schnitzler an Paul Goldmann, 22. 11. 1896}\nopagebreak\mylabel{v}\rehead{ }\normalsize\beginnumbering\briefempfaengerindex{Goldmann, Paul@\textsc{Goldmann, Paul}!zzzSchnitzler, Arthur@\emph{von Arthur Schnitzler}!1896-11-221@{22. 11. 1896}|(be} \toendnotes[C]{\smallbreak\pagebreak[2]} \buchAlsQuelle{\pwindex{Ritterlichkeit@\emph{Ritterlichkeit}|pwk}Arthur Schnitzler: \emph{Ritterlichkeit. Fragment aus dem Nachlaß}. Bonn: \emph{Bouvier Verlag Herbert Grundmann} 1975, S. 6 (Abhandlungen zur Kunst-, Musik- und
                        Literaturwissenschaft, 176).}\toendnotes[C]{\smallbreak}\pstart
           \noindent{}{\pb}\label{K_L02685-2v}\edtext{ALSO}{\lemma{\textnormal{\emph{Also}}}\Cendnote{\textnormal{Von den Korrespondenzstücken Schnitzlers an Goldmann fehlt
               weitgehend jede Spur. In der Edition von \emph{\textcolor{green}{Ritterlichkeit}} (1975) schreibt die Herausgeberin
               \textcolor{blue}{Rena R. Schlein}:
               »Zwei Telegramme und ein Brief Schnitzlers an Goldmann
                  wurden mir von Dr. \textcolor{blue}{Leo P.
                     Reckford}, der diese Dokumente von der Familie Goldmanns
                  zum Geschenk bekam, für meine Arbeit zur Verfügung
                  gestellt« (S. 1). \textcolor{blue}{Reckford} starb 1988, seine Nachkommen
               haben keine Kenntnis von diesen (und etwaigen weiteren) Korrespondenzstücken und sie
               sind auch nicht auffindbar. \textcolor{blue}{Rena R. Schlein} wäre,
               wenn sie noch leben sollte, deutlich über 100 Jahre alt. Ein Kontakt konnte nicht
               hergestellt werden. Während von dem anderen Telegramm (Arthur Schnitzler an Paul Goldmann, 22. 11. 1896) eine Fotokopie und von dem
               Brief Teile als Fotokopie (Arthur Schnitzler an Paul Goldmann, 22. 11. 1896) im Nachlass Schnitzlers liegen, gibt es für dieses Telegramm keine
               erhaltene Vorlage.}}}\label{K_L02685-2h} DAZU SCHREIB ICH EXTRA \label{K_L02685-1v}\edtext{\textcolor{green}{STUECKE GEGENS DUELL}{}\ledrightnote{→\textcolor{green}{Liebelei. Schauspiel in drei Akten}{\newline}→\textcolor{green}{Freiwild. Schauspiel in 3 Akten}}}{\lemma{\textnormal{\emph{Stuecke gegens Duell}}}\Cendnote{\textnormal{\emph{\textcolor{green}{Liebelei}} und \emph{\textcolor{green}{Freiwild}}}}}\label{K_L02685-1h}\pend
           \pstart TAUSEND GRUESSE UND GLUECKWUENSCHE\spacefill\mbox{ARTHUR}\pend{}\endnumbering\briefempfaengerindex{Goldmann, Paul@\textsc{Goldmann, Paul}!zzzSchnitzler, Arthur@\emph{von Arthur Schnitzler}!1896-11-221@{22. 11. 1896}|)be}\mylabel{h}\begin{anhang}\end{anhang}\normalsize

\doendnotes{C}
\bigskip
\vfill

\clearpage

\footnotesize

\lohead{\textsc{register}}

% Definiere theindex-Environment komplett neu ohne reledmac
\makeatletter
\renewenvironment{theindex}{%
  \section*{\indexname}%
  \setlength{\parindent}{0pt}%
  \setlength{\parskip}{0pt plus 0.3pt}%
  \let\item\@idxitem
}{%
  \clearpage
}
\makeatother

\IfFileExists{\jobname-pw.ind}{\input{\jobname-pw.ind}}{}

\end{document}

      