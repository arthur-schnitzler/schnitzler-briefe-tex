%% latex-korrekturansicht-vorspann.tex
%% Vorspann für die Korrekturansicht.
%% Lädt die gemeinsame Datei latex-vorspann.tex mit gesetztem Schalter.

\newif\ifkorrekturansicht
\korrekturansichttrue

\input{../tex-inputs/latex-vorspann}


               \section[Richard Beer-Hofmann an Arthur Schnitzler, 18. 6. 1898]{ Richard Beer-Hofmann an Arthur Schnitzler, 18. 6. 1898}\nopagebreak\mylabel{v}\rehead{ }\normalsize\beginnumbering\briefempfaengerindex{Schnitzler, Arthur@\textsc{Schnitzler, Arthur}!zzzBeer-Hofmann, Richard@\emph{von Richard Beer-Hofmann}!1898-06-181@{18. 6. 1898}|(be} \toendnotes[C]{\smallbreak\pagebreak[2]} \Standort{CUL, Schnitzler, B 8.}
\physDesc{Brief, 1 Blatt, 4 Seiten
\newline{}Handschrift: Bleistift, lateinische Kurrent\newline{}Ordnung: mit Bleistift von unbekannter Hand nummeriert: »117« }\buchAbdrucke{\weitereDrucke{Arthur Schnitzler, Richard Beer-Hofmann: \emph{Briefwechsel 1891–1931}. Hg. Konstanze Fliedl. Wien, Zürich: \emph{Europaverlag} 1992, S. 120.} }\toendnotes[C]{\smallbreak}\pstart
           \raggedleft{}{\pb}\textcolor{pink}{Steindorf}{}\ledrightnote{\textcolor{pink}{Steindorf am Ossiacher See}}{ }18/VI 98\pend
           \pstart
           Lieber Arthur, vielen Dank für Ihr »\textcolor{green}{Interpunctationsgefühl}{}\ledrightnote{→\textcolor{green}{Schlaflied für Mirjam}}«. Auch mir waren die
                  \uline{–} anstatt \uline{,} zu
               ausdrucksvoll, zu überquellend von Empfindung – wollte nur nichts sagen, um Ihre
               Unbefangenheit nicht zu stören.\pend
           \pstart
           Da es scheint daß Sie \strikeout{zwisch} nach
                  27 Juli nach \textcolor{pink}{Tegernsee}{}\ledrightnote{\textcolor{pink}{Tegernsee}} per Rad
               fahren, so dürfte wol unsere Zusa{\geminationm}enkunft {\pb}am besten in der I oder
                  II. Augustwoche um \textcolor{pink}{Salzburg}{}\ledrightnote{\textcolor{pink}{Salzburg}} herum
               stattfinden. Das würde auch für \textcolor{blue}{Hugo}{}\ledrightnote{\textcolor{blue}{Hugo von Hofmannsthal}} nach seinem
               letzten Brief die beste Zeit sein.\pend
           \pstart
           Vielleicht auch – wenn ich trainirt bin – im September im \textcolor{pink}{Ampezzo}{}\ledrightnote{\textcolor{pink}{Valle d’Ampezzo}}. 20–27 Juli ist
               unsicher da mein \textcolor{blue}{Papa}{}\ledrightnote{→\textcolor{blue}{Alois Hofmann}} mich
               ungern abseits von \textcolor{blue}{Mirjam}{}\ledrightnote{\textcolor{blue}{Mirjam Beer-Hofmann}} sieht. Ich arbeite {\pb}– nicht genug. Ich hoffe, es wird
               besser. Wetter ist scheusslich; heute regenlos, aber der Regen ko{\geminationm}t noch.\pend
           \pstart
           Bitte schreiben Sie mir so oft als möglich; wenn man – wie der zudringliche \textcolor{blue}{Mime}{}\ledrightnote{→\textcolor{blue}{?? [Schauspieler]}} das
               nennt, keine »Ansprache« hat!\pend
           \pstart
           Grüßen Sie wie {\pb}gewöhnlich nach
               Gutdünken und nuancirt. Ich lese ein gutes Buch von \textcolor{blue}{\uline{Mach}}{}\ledrightnote{\textcolor{blue}{Ernst Mach}} (\textcolor{green}{Populärwissensch. Vorles.}{}\ledrightnote{\textcolor{green}{Populär-Wissenschaftliche Vorlesungen}}).\pend
           \pstart
           Von Herzen{\\[\baselineskip]}Ihr{\\[\baselineskip]}\spacefill\mbox{Richard}\pend
           \leftskip=0em{}\pstart
           \noindent{}\textcolor{blue}{Paula}{}\ledrightnote{\textcolor{blue}{Paula Beer-Hofmann}} erwidert Ihren Gruß – \textcolor{blue}{Mirjam}{}\ledrightnote{\textcolor{blue}{Mirjam Beer-Hofmann}} hab ich ihn mitgeteilt; sie hat mich hierauf in den
                  Finger gebissen.\pend
           \endnumbering\briefempfaengerindex{Schnitzler, Arthur@\textsc{Schnitzler, Arthur}!zzzBeer-Hofmann, Richard@\emph{von Richard Beer-Hofmann}!1898-06-181@{18. 6. 1898}|)be}\mylabel{h}  \normalsize

\doendnotes{C}
\bigskip
\vfill

\clearpage

\footnotesize

\lohead{\textsc{register}}

% Definiere theindex-Environment komplett neu ohne reledmac
\makeatletter
\renewenvironment{theindex}{%
  \section*{\indexname}%
  \setlength{\parindent}{0pt}%
  \setlength{\parskip}{0pt plus 0.3pt}%
  \let\item\@idxitem
}{%
  \clearpage
}
\makeatother

\IfFileExists{\jobname-pw.ind}{\input{\jobname-pw.ind}}{}

\end{document}

      