%% latex-korrekturansicht-vorspann.tex
%% Vorspann für die Korrekturansicht.
%% Lädt die gemeinsame Datei latex-vorspann.tex mit gesetztem Schalter.

\newif\ifkorrekturansicht
\korrekturansichttrue

\input{../tex-inputs/latex-vorspann}


               \section[Richard Beer-Hofmann an Arthur Schnitzler, 23. 6. 1895]{ Richard Beer-Hofmann an Arthur Schnitzler, 23. 6. 1895}\nopagebreak\mylabel{v}\rehead{ }\normalsize\beginnumbering\briefempfaengerindex{Schnitzler, Arthur@\textsc{Schnitzler, Arthur}!zzzBeer-Hofmann, Richard@\emph{von Richard Beer-Hofmann}!1895-06-231@{23. 6. 1895}|(be} \toendnotes[C]{\smallbreak\pagebreak[2]} \Standort{CUL, Schnitzler, B 8.}
\physDesc{Brief, 1 Blatt, 4 Seiten
\newline{}Handschrift: Bleistift, lateinische Kurrent
\newline{}Schnitzler: mit Bleistift nummeriert: »62« }\buchAbdrucke{\weitereDrucke{Arthur Schnitzler, Richard Beer-Hofmann: \emph{Briefwechsel 1891–1931}. Hg. Konstanze Fliedl. Wien, Zürich: \emph{Europaverlag} 1992, S. 75–76.} }\pstart
           {\pb}\textcolor{pink}{\uline{Zleb}}{}\ledrightnote{\textcolor{pink}{Schleb}}{ }23/VI 95\pend
           \pstart
           Lieber Arthur! \textcolor{pink}{Zleb}{}\ledrightnote{\textcolor{pink}{Schleb}} ist mit
                    dem Wagen ¾ Stunden von \textcolor{pink}{Caslau}{}\ledrightnote{\textcolor{pink}{Caslau}} entfernt; ich
                    bin weil man doch am Sonntag nicht in \textcolor{pink}{Caslau}{}\ledrightnote{\textcolor{pink}{Caslau}} bleiben kann nach \textcolor{pink}{Zleb}{}\ledrightnote{\textcolor{pink}{Schleb}} gefahren – Sie begreifen – mit mir am Tische zwei
                    unsägliche Cadetten der Reserve, einer aus \textcolor{pink}{Neu-Bidschow}{}\ledrightnote{\textcolor{pink}{Nový Bydžov}}, der andere {\pb}aus \textcolor{pink}{Benatek}{}\ledrightnote{\textcolor{pink}{Benatek}}. Jetzt lesen sie Gottseidank \textcolor{pink}{böhmische}{}\ledrightnote{\textcolor{pink}{Böhmen}} Zeitungen.\pend
           \pstart
           Ich bin also voraussichtlich am 29ten, unwahrscheinlicher Weise
                    schon am 28ten{ }nachts d. i. 11 Uhr nachts in \textcolor{pink}{Wien}{}\ledrightnote{\textcolor{pink}{Wien}}, und werde gegen 3. od 4. nach
                        \textcolor{pink}{Ischl}{}\ledrightnote{\textcolor{pink}{Bad Ischl}} reisen. Ich bin nervös sehr
                    herunter {\pb}so daß ich trotz
                    Müdigkeit nicht schlafe. Ich sehne mich nach Ruhe und Arbeiten. –\pend
           \pstart
           Vielleicht gebe ich mir telegrafisch ein Rendezvous mit Ihnen, wenn ich ankomme.
                    Wann sind Sie in \textcolor{pink}{Ischl}{}\ledrightnote{\textcolor{pink}{Bad Ischl}}? Das können Sie mir
                    zwar sagen, schreiben Sie es mir {\pb}aber lieber, weil mir jeder
                    Brief woltut.\pend
           \pstart
           \label{OL515-1v}\label{OL515-1h}Ad \textcolor{blue}{Burkhardt}{}\ledrightnote{\textcolor{blue}{Max Eugen Burckhard}}: \uline{\textcolor{blue}{Bahr}{}\ledrightnote{\textcolor{blue}{Hermann Bahr}}, \textcolor{blue}{Burkhardt}{}\ledrightnote{\textcolor{blue}{Max Eugen Burckhard}}, \textcolor{blue}{Lueger}{}\ledrightnote{\textcolor{blue}{Karl Lueger}}}. Aber der Erste ist doch anders. Sie sehen sogar gerecht werde ich hier {\dots}\pend
           \pstart
           Der »\textcolor{green}{alte Dichter}{}\ledrightnote{\textcolor{green}{Später Ruhm}}« ist doch schon
                    zusa{\geminationm}engestrichen? \pend
           \pstart
           Herzlichst Ihr{\\[\baselineskip]}\spacefill\mbox{Richard}\pend
           \leftskip=0em{}\endnumbering\briefempfaengerindex{Schnitzler, Arthur@\textsc{Schnitzler, Arthur}!zzzBeer-Hofmann, Richard@\emph{von Richard Beer-Hofmann}!1895-06-231@{23. 6. 1895}|)be}\mylabel{h}  \normalsize

\doendnotes{C}
\bigskip
\vfill

\clearpage

\footnotesize

\lohead{\textsc{register}}

% Definiere theindex-Environment komplett neu ohne reledmac
\makeatletter
\renewenvironment{theindex}{%
  \section*{\indexname}%
  \setlength{\parindent}{0pt}%
  \setlength{\parskip}{0pt plus 0.3pt}%
  \let\item\@idxitem
}{%
  \clearpage
}
\makeatother

\IfFileExists{\jobname-pw.ind}{\input{\jobname-pw.ind}}{}

\end{document}

      