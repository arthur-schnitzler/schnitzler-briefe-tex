%% latex-korrekturansicht-vorspann.tex
%% Vorspann für die Korrekturansicht.
%% Lädt die gemeinsame Datei latex-vorspann.tex mit gesetztem Schalter.

\newif\ifkorrekturansicht
\korrekturansichttrue

\input{../tex-inputs/latex-vorspann}


               \section[Hugo von Hofmannsthal an Arthur Schnitzler, 26. 8. 1897]{ Hugo von Hofmannsthal an Arthur Schnitzler, 26. 8. 1897}\nopagebreak\mylabel{v}\rehead{ }\normalsize\beginnumbering\briefempfaengerindex{Schnitzler, Arthur@\textsc{Schnitzler, Arthur}!zzzHofmannsthal, Hugo von@\emph{von Hugo von Hofmannsthal}!1897-08-261@{26. 8. 1897}|(be} \toendnotes[C]{\smallbreak\pagebreak[2]} \Standort{CUL, Schnitzler, B 43.}
\physDesc{Postkarte
\newline{}Handschrift: Bleistift, deutsche Kurrent\newline{}Versand: 1) Stempel: »\nobreak{}\oindex{Varese@\textbf{Varese}, \emph{Besiedelter Ort (A.BSO)}|pwk}Varese, 27. 8. 97, 8 M\nobreak{}«.  2) Stempel: »\nobreak{}\oindex{IX., Alsergrund@\textbf{IX., Alsergrund}, \emph{Bezirk (A.BZK)}|pwk}Wien 9/3 72, 29. 8. 97, 8.V, Bestellt\nobreak{}«. \newline{}Ordnung: 1) mit Bleistift von unbekannter Hand nummeriert: »95A« 2) mit Bleistift von unbekannter Hand
                                    nummeriert: »102«}\buchAbdrucke{\weitereDrucke{Hugo von Hofmannsthal, Arthur Schnitzler: \emph{Briefwechsel}. Hg. Therese Nickl und Heinrich Schnitzler. Frankfurt am Main: \emph{S. Fischer} 1964, S. 95.} }\pstart{}{\pb}\textsc{Herrn D\textsuperscript{r} Arthur Schnitzler}\pend{}\pstart{}\textcolor{pink}{\textsc{Wien}}{}\ledrightnote{\textcolor{pink}{Wien}}\pend{}\pstart{}\textcolor{pink}{\textsc{IX
                            Franckgasse 1}}{}\ledrightnote{\textcolor{pink}{Frankgasse}}\pend{}\pstart{}\textcolor{pink}{\textsc{Austria}}{}\ledrightnote{\textcolor{pink}{Österreich}}\pend{}\pstart{}\textsc{per Ala}\pend{}{\bigskip}\pstart
           {\pb}26 VIII\hfill \textsc{\textcolor{pink}{Varese}{}\ledrightnote{\textcolor{pink}{Varese}} per \textcolor{pink}{Milano}{}\ledrightnote{\textcolor{pink}{Mailand}}}{\\}\textsc{\textcolor{pink}{Hôtel d’Italie}{}\ledrightnote{\textcolor{pink}{Grand Hotel Varese}}}\pend
           \pstart{}mein lieber Arthur\pend\pstart
           ich bin ſo zufrieden und glücklich wie glaub ich in meinem Leben nicht, ganz
                        überſchwe{\geminationm}t von Plänen und Halbfertigem.
                    Vielleicht hör ich etwas von Ihnen, ich bleibe bis 10. September
                    hier.\pend
           \pstart
           Ihr{\\[\baselineskip]}\spacefill\mbox{Hugo}\pend
           \leftskip=0em{}\pstart
           \noindent{}ich dank Ihnen herzlich für
                        vieles wegen \textcolor{blue}{Poldy}{}\ledrightnote{\textcolor{blue}{Leopold von Andrian-Werburg}}.\pend
           \endnumbering\briefempfaengerindex{Schnitzler, Arthur@\textsc{Schnitzler, Arthur}!zzzHofmannsthal, Hugo von@\emph{von Hugo von Hofmannsthal}!1897-08-261@{26. 8. 1897}|)be}\mylabel{h}  \normalsize

\doendnotes{C}
\bigskip
\vfill

\clearpage

\footnotesize

\lohead{\textsc{register}}

% Definiere theindex-Environment komplett neu ohne reledmac
\makeatletter
\renewenvironment{theindex}{%
  \section*{\indexname}%
  \setlength{\parindent}{0pt}%
  \setlength{\parskip}{0pt plus 0.3pt}%
  \let\item\@idxitem
}{%
  \clearpage
}
\makeatother

\IfFileExists{\jobname-pw.ind}{\input{\jobname-pw.ind}}{}

\end{document}

      