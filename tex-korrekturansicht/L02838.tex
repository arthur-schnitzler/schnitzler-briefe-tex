%% latex-korrekturansicht-vorspann.tex
%% Vorspann für die Korrekturansicht.
%% Lädt die gemeinsame Datei latex-vorspann.tex mit gesetztem Schalter.

\newif\ifkorrekturansicht
\korrekturansichttrue

\input{../tex-inputs/latex-vorspann}


               \section[ Paul Goldmann an Arthur Schnitzler, 6. 2. {[}1898{]}]{Paul Goldmann an Arthur Schnitzler, 6. 2. {[}1898{]}}\nopagebreak\mylabel{v}\rehead{ }\normalsize\beginnumbering\briefempfaengerindex{Schnitzler, Arthur@\textsc{Schnitzler, Arthur}!zzzGoldmann, Paul@\emph{von Paul Goldmann}!1898-02-062@{6. 2. {[}1898{]}}|(be} \toendnotes[C]{\smallbreak\pagebreak[2]} \Standort{DLA, A:Schnitzler, HS.NZ85.1.3168.}
\physDesc{Brief, 1 Blatt, 2 Seiten
\newline{}Handschrift: blaue Tinte, lateinische Kurrent
\newline{}Schnitzler: 1) mit schwarzer Tinte das Jahr »98« vermerkt 2) mit rotem Buntstift zwei Unterstreichungen}\toendnotes[C]{\smallbreak}\pstart
           \noindent{}{\pb}\textcolor{gray}{\textbf{\textbf{\textcolor{brown}{Frankfurter Zeitung}{}\ledrightnote{\textcolor{brown}{Frankfurter Zeitung}}}}}\pend
           \pstart
           \textcolor{gray}{\textbf{(\textcolor{brown}{\begin{otherlanguage}{french}Gazette de Francfort\end{otherlanguage}}{}\ledrightnote{\textcolor{brown}{Frankfurter Zeitung}}).}}\pend
           \pstart
           \textcolor{gray}{\textbf{\textbf{\begin{otherlanguage}{french}Fondateur M.\end{otherlanguage}{ }\textcolor{blue}{L. Sonnemann}{}\ledrightnote{\textcolor{blue}{Leopold Sonnemann}}.}}}\pend
           \pstart
           \begin{otherlanguage}{french}\textcolor{gray}{\textbf{Journal politique, financier,}}\end{otherlanguage}\pend
           \pstart
           \begin{otherlanguage}{french}\textcolor{gray}{\textbf{commercial et littéraire.}}\end{otherlanguage}\pend
           \pstart
           \begin{otherlanguage}{french}\textcolor{gray}{\textbf{\textbf{Paraissant trois fois par jour.}}}\end{otherlanguage}\hfill \textsc{\textcolor{pink}{Paris}{}\ledrightnote{\textcolor{pink}{Paris}}}, 6. Februar.\pend
           \pstart
           \begin{otherlanguage}{french}\textcolor{gray}{\textbf{\textbf{Bureau à \textcolor{pink}{Paris}{}\ledrightnote{\textcolor{pink}{Paris}}}}}\end{otherlanguage}\pend
           \pstart
           \begin{otherlanguage}{french}\textcolor{gray}{\textbf{\textbf{\textcolor{pink}{10 Rue de la Bourse}{}\ledrightnote{\textcolor{pink}{rue de la Bourse}}.}}}\end{otherlanguage}\pend
           \pstart\center{}Mein lieber Freund,\pend\pstart
           Ich bin in tollſter Arbeit. Morgen beginnt der \label{K_L02838-1v}\edtext{Prozeß \textsc{\textcolor{blue}{Zola}{}\ledrightnote{\textcolor{blue}{Émile Zola}}}}{\lemma{\textnormal{\emph{Prozeß Zola}}}\Cendnote{\textnormal{Am 13. 1. 1898
                  hatte \textcolor{blue}{Émile Zola} seinen offenen Brief
                  \emph{\textcolor{green}{J’accuse…!}} veröffentlicht, in dem er offen
                  für \textcolor{blue}{Dreyfus} Partei ergriff. \textcolor{blue}{Émile Zola}
                  wurde nach einem Verleumdungsprozess, der zwischen 7. 2. 1898 und
                  23. 2. 1898 abgehalten wurde, zu einer einjährigen Haftstrafe
                  verurteilt. Der Verhaftung entkam er durch eine Flucht ins Exil, wo er bis zu seiner
                  Begnadigung nach zwei Jahren blieb. Für \textcolor{blue}{Dreyfus} brachte die öffentliche Anprangerung des
                  Unrechts, das ihm angetan wurde, der Wendepunkt, der letztlich zu seiner
                  Entlassung aus der Gefangenschaft führte.}}}\label{K_L02838-1h}. Ich habe nur eine Minute, um Dich zu
               dem neuen ſchönen \label{K_L02838-2v}\edtext{\textcolor{green}{Erfolge}{}\ledrightnote{→\textcolor{green}{Freiwild. Schauspiel in 3 Akten}}}{\lemma{\textnormal{\emph{Erfolge}}}\Cendnote{\textnormal{am 4. 2. 1898 hatte die \emph{\textcolor{green}{Freiwild}}-Premiere im \textcolor{pink}{Wien}er \textcolor{pink}{Carl-Theater} stattgefunden.
                     \textcolor{blue}{Marie Glümer} spielte die Rolle der \textcolor{green}{Pepi Fischer}. }}}\label{K_L02838-2h} in \textcolor{pink}{Wien}{}\ledrightnote{\textcolor{pink}{Wien}} zu beglückwünſchen. Ich ſchöpfe meine Kenntniß
               des \textcolor{green}{Erfolg}{}\ledrightnote{→\textcolor{green}{Freiwild. Schauspiel in 3 Akten}}es nur aus der
                  \label{K_L02838-8v}\edtext{\textcolor{green}{Kritik}{}\ledrightnote{→\textcolor{green}{Theaterzeitung. Carltheater [Freiwild]}} des \textcolor{green}{Extrablatt}{}\ledrightnote{\textcolor{green}{Illustrirtes Wiener Extrablatt}}}{\lemma{\textnormal{\emph{Kritik des Extrablatt}}}\Cendnote{\textnormal{o. V.: \emph{\textcolor{green}{Theaterzeitung. Carltheater}}.
                     In: \emph{\textcolor{green}{Illustrirtes Wiener Extrablatt}},
                     Jg. XXXX, Nr. XXXX, 5. 2. 1898,
                  S. XXXX.}}}\label{K_L02838-8h}. Aber ich denke mir, wenn ſchon dieſes dumme \textcolor{brown}{Blatt}{}\ledrightnote{→\textcolor{brown}{Illustrirtes Wiener Extrablatt}} ſo freundlich iſt, wie ruhmreich muß
               da in Wirklichkeit der \textcolor{green}{\textsc{Premièren}-Abend}{}\ledrightnote{→\textcolor{green}{Freiwild. Schauspiel in 3 Akten}} geweſen ſein! Ich freue mich von
                  \strikeout{\textcolor{gray}{Herz}} ganzem Herzen, {\pb}daß ich Dich ſo ſtolz und
               ſicher \strikeout{\textcolor{gray}{e}} weiterſchreiten ſehe.\pend
           \pstart
           Ich danke Dir für Deinen letzten lieben Brief. Bitte, ſchreib’ mir bald. Schreib’
               mir, wie die \textcolor{green}{Première}{}\ledrightnote{→\textcolor{green}{Freiwild. Schauspiel in 3 Akten}} war, wie
                  \textsc{Frl. \textcolor{blue}{G.}{}\ledrightnote{\textcolor{blue}{Marie Glümer}}} geſpielt hat und was es ſonſt dabei gab.\pend
           \pstart
           Iſt \label{K_L02838-23v}\edtext{\textsc{Dr. \textcolor{blue}{Brandes}{}\ledrightnote{\textcolor{blue}{Georg Brandes}}} ſehr böſe}{\lemma{\textnormal{\emph{Dr. Brandes ſehr böſe}}}\Cendnote{\textnormal{\textcolor{blue}{Georg Brandes} hielt sich
                  zwischen 25. 1. 1898 und
                  2. 2. 1898
                in \textcolor{pink}{Wien} auf und
               traf sich mehrfach mit \textcolor{blue}{Schnitzler}.}}}\label{K_L02838-23h} auf
               mich, weil ich ihm nicht geſchrieben habe?\pend
           \pstart
           Ich begrüße Dich von {\\[\baselineskip]}Herzen und in Treue {\\[\baselineskip]}Dein {\\[\baselineskip]}\spacefill\mbox{Paul Goldmann.}\pend
           \leftskip=0em{}\endnumbering\briefempfaengerindex{Schnitzler, Arthur@\textsc{Schnitzler, Arthur}!zzzGoldmann, Paul@\emph{von Paul Goldmann}!1898-02-062@{6. 2. {[}1898{]}}|)be}\mylabel{h}  \normalsize

\doendnotes{C}
\bigskip
\vfill

\clearpage

\footnotesize

\lohead{\textsc{register}}

% Definiere theindex-Environment komplett neu ohne reledmac
\makeatletter
\renewenvironment{theindex}{%
  \section*{\indexname}%
  \setlength{\parindent}{0pt}%
  \setlength{\parskip}{0pt plus 0.3pt}%
  \let\item\@idxitem
}{%
  \clearpage
}
\makeatother

\IfFileExists{\jobname-pw.ind}{\input{\jobname-pw.ind}}{}

\end{document}

      