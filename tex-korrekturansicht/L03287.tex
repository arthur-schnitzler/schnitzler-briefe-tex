%% latex-korrekturansicht-vorspann.tex
%% Vorspann für die Korrekturansicht.
%% Lädt die gemeinsame Datei latex-vorspann.tex mit gesetztem Schalter.

\newif\ifkorrekturansicht
\korrekturansichttrue

\input{../tex-inputs/latex-vorspann}


\renewcommand{\erwaehntePersonen}{Personen: Marie Reinhard}
\renewcommand{\erwaehnteOrte}{Orte: Café Glattauer, Wien}
\renewcommand{\erwaehnteWerke}{Werke: Tagebuch}
\section[ Felix Salten an Arthur Schnitzler, {[}20. 3. 1899{]}]{Felix Salten an Arthur Schnitzler, {[}20. 3. 1899{]}}
\nopagebreak\mylabel{v}
\rehead{ }\normalsize\beginnumbering\briefempfaengerindex{Schnitzler, Arthur@\textsc{Schnitzler, Arthur}!zzzSalten, Felix@\emph{von Felix Salten}!1899-03-204@{{[}20. 3. 1899{]}}|(be}
\toendnotes[C]{\smallbreak\pagebreak[2]}\Standort{CUL, Schnitzler, B 89, A 2.}
\physDesc{Brief, 1 Blatt, 1 Seite, 235 Zeichen
\newline{}Handschrift: schwarze Tinte, lateinische Kurrent
\newline{}Schnitzler: mit Bleistift datiert: »20/3 99« 
\newline{}Ordnung: mit Bleistift von unbekannter Hand nummeriert: »111« }\toendnotes[C]{\smallbreak}
\pstart
           \noindent{}{\pb}Lieber Arthur, wenn Sie \label{K_L03287-1v}\edtext{Abends in irgend einem Caféhaus}{\lemma{\textnormal{\emph{Abends … Caféhaus}}}\Cendnote{\textnormal{nicht nachweisbar; nach \textcolor{blue}{Marie Reinhard}s Tod am 18. 3. 1899 gibt es für die darauffolgenden rund
                  zwei Wochen keine Einträge im \emph{\textcolor{green}{Tagebuch}}}}}\label{K_L03287-1h} sind, oder wollen dass ich Sie besuche, dann bitte, laßen Sie mir ins \textcolor{pink}{Café Glattauer}{}\ledrightnote{\textcolor{pink}{Café Glattauer}} ein Wort sagen, wohin ich nach
               dem Theater gehe, nur um etwas von Ihnen zu hören.\pend
           
\pstart
           Herzlichst {\\[\baselineskip]}Ihr {\\[\baselineskip]}\spacefill\mbox{Salten}\pend
           \leftskip=0em{}\endnumbering\briefempfaengerindex{Schnitzler, Arthur@\textsc{Schnitzler, Arthur}!zzzSalten, Felix@\emph{von Felix Salten}!1899-03-204@{{[}20. 3. 1899{]}}|)be}\mylabel{h}  \normalsize

\doendnotes{C}
\bigskip
\vfill

\clearpage

\footnotesize

\lohead{\textsc{register}}

% Definiere theindex-Environment komplett neu ohne reledmac
\makeatletter
\renewenvironment{theindex}{%
  \section*{\indexname}%
  \setlength{\parindent}{0pt}%
  \setlength{\parskip}{0pt plus 0.3pt}%
  \let\item\@idxitem
}{%
  \clearpage
}
\makeatother

\IfFileExists{\jobname-pw.ind}{\input{\jobname-pw.ind}}{}

\end{document}

      