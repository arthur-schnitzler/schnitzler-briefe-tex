%% latex-korrekturansicht-vorspann.tex
%% Vorspann für die Korrekturansicht.
%% Lädt die gemeinsame Datei latex-vorspann.tex mit gesetztem Schalter.

\newif\ifkorrekturansicht
\korrekturansichttrue

\input{../tex-inputs/latex-vorspann}


\renewcommand{\erwaehntePersonen}{Personen: Bertha Hausner, Felix Salten, Adele Sandrock}
\renewcommand{\erwaehnteInstitutionen}{Institutionen: Volkstheater}
\renewcommand{\erwaehnteOrte}{Orte: Café Central, Wien}
\renewcommand{\erwaehnteWerke}{Werke: Francillon. Schauspiel in 3 Aufzügen}
\section[Arthur Schnitzler an Felix Salten, {[}14. 4. 1894?{]}]{Arthur Schnitzler an Felix Salten, {[}14. 4. 1894?{]}}
\nopagebreak\mylabel{v}
\rehead{ }\normalsize\beginnumbering\briefempfaengerindex{Salten, Felix@\textsc{Salten, Felix}!zzzSchnitzler, Arthur@\emph{von Arthur Schnitzler}!1894-04-141@{{[}14. 4. 1894?{]}}|(be}
\toendnotes[C]{\smallbreak\pagebreak[2]}\Standort{Wienbibliothek im Rathaus, ZPH 1681, 2.1.516.}
\physDesc{Briefkarte, 191 Zeichen (Briefkarte mit Trauerrand)
\newline{}Handschrift: Bleistift, deutsche Kurrent
\newline{}Ordnung: mit Bleistift von unbekannter Hand Nummerierung der Blätter des Konvoluts:
                                    »6« }\toendnotes[C]{\smallbreak}
\pstart
           \noindent{}{\pb}Lieber Freund,{ }\label{K_L03028-1v}\edtext{\textsc{\textcolor{green}{Francillon}{}\ledrightnote{\textcolor{green}{Francillon. Schauspiel in 3 Aufzügen}}} iſt \uline{abgeſagt}}{\lemma{\textnormal{\emph{Francillon iſt abgeſagt}}}\Cendnote{\textnormal{Für den 15. 4. 1894 war am \emph{\textcolor{brown}{Deutschen Volkstheater}}{ }\emph{\textcolor{green}{Francillon}} mit \textcolor{blue}{Adele Sandrock} als »\textcolor{green}{Francine}« angesetzt, musste aber wegen Erkrankung von
                     \textcolor{blue}{Bertha Hausner} (»\textcolor{green}{Anette}«) kurzfristig abgesagt werden.
                     \textcolor{blue}{Schnitzler} verbrachte den Abend bei \textcolor{blue}{Sandrock}, jedoch ohne \textcolor{blue}{Salten}. Das erlaubt die Datierung des Korrespondenzstücks,
                  wenngleich nicht ausgeschlossen werden kann, dass auch ein anderer Abend im
                  Zeitraum 1893/1894 in Frage
                  kommt.}}}\label{K_L03028-1h}. – Wir gehen um 9 zu Frl. \textcolor{blue}{S.}{}\ledrightnote{\textcolor{blue}{Adele Sandrock}} – Ich hole Sie um ½ 9{ }\textsc{\textcolor{pink}{Café Centra{[}l{]}}{}\ledrightnote{\textcolor{pink}{Café Central}}} ab. – Herzlichen Gruß. \spacefill\mbox{Arth}\pend
           
\pstart
           \noindent{}{\pb}Vielleicht ſchaun Sie gleich nach Tiſch
                  auf 5 Minuten zu mir herüber?\pend
           \endnumbering\briefempfaengerindex{Salten, Felix@\textsc{Salten, Felix}!zzzSchnitzler, Arthur@\emph{von Arthur Schnitzler}!1894-04-141@{{[}14. 4. 1894?{]}}|)be}\mylabel{h}  \normalsize

\doendnotes{C}
\bigskip
\vfill

\clearpage

\footnotesize

\lohead{\textsc{register}}

% Definiere theindex-Environment komplett neu ohne reledmac
\makeatletter
\renewenvironment{theindex}{%
  \section*{\indexname}%
  \setlength{\parindent}{0pt}%
  \setlength{\parskip}{0pt plus 0.3pt}%
  \let\item\@idxitem
}{%
  \clearpage
}
\makeatother

\IfFileExists{\jobname-pw.ind}{\input{\jobname-pw.ind}}{}

\end{document}

      