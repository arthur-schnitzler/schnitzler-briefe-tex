%% latex-korrekturansicht-vorspann.tex
%% Vorspann für die Korrekturansicht.
%% Lädt die gemeinsame Datei latex-vorspann.tex mit gesetztem Schalter.

\newif\ifkorrekturansicht
\korrekturansichttrue

\input{../tex-inputs/latex-vorspann}


               \section[Arthur Schnitzler an Hugo von Hofmannsthal, {[}18. 1. 1894{]}]{ Arthur Schnitzler an Hugo von Hofmannsthal, {[}18. 1. 1894{]}}\nopagebreak\mylabel{v}\rehead{ }\normalsize\beginnumbering\briefempfaengerindex{Hofmannsthal, Hugo von@\textsc{Hofmannsthal, Hugo von}!zzzSchnitzler, Arthur@\emph{von Arthur Schnitzler}!1894-01-181@{{[}18. 1. 1894{]}}|(be} \toendnotes[C]{\smallbreak\pagebreak[2]} \Standort{FDH, Hs-30885,41.}
\physDesc{Brief, 1 Blatt (Briefpapier mit Trauerrand), 3 Seiten
\newline{}Handschrift: Bleistift, deutsche Kurrent\newline{}Ordnung: von Schnitzler mutmaßlich bei der Durchsicht der Korrespondenz 1929
                                    mit Bleistift datiert: »18/1 94« }\buchAbdrucke{\weitereDrucke{Hugo von Hofmannsthal, Arthur Schnitzler: \emph{Briefwechsel}. Hg. Therese Nickl und Heinrich Schnitzler. Frankfurt am Main: \emph{S. Fischer} 1964, S. 49.} }\pstart
           \raggedleft{}{\pb}\uline{Donnerſtag.}\pend
           \pstart{}Lieber Hugo,\pend\pstart
           vielleicht ko{\geminationm}en die beiliegenden 3 Ka{\geminationm}ermuſikabende Ihrem Muſikbedürfnis entgegen.
                    Iſt’s Ihnen alſo recht, ſo möchte ich Ihnen einen Sitz neben mir, womöglich
                    Gallerie nehmen. – Hier iſt der Sitz für {\pb}\textcolor{blue}{\textsc{Mounet Sully}}{}\ledrightnote{\textcolor{blue}{Jean Mounet-Sully}}; 4 fl. 20; was freilich für einen armen Dichter viel iſt. –\pend
           \pstart
           So{\geminationn}tag werd ich vor dem Theater kaum
                    zu \textcolor{blue}{Richard}{}\ledrightnote{\textcolor{blue}{Richard Beer-Hofmann}} kö{\geminationn}en; (höchſtens Sie \introOben{}von dort\introOben{} abholen), weil ich vorher
                    irgendwo (bei \textcolor{blue}{Wetzler}{}\ledrightnote{\textcolor{blue}{Bernhard Wetzler}}’s) einen Thee trinken
                    muſs. –\pend
           \pstart
           Herentgegen müßte es mit dem Teufel zugehen {\pb}we{\geminationn} ich nicht heute Abends um 10 ins
                        \textcolor{pink}{Café Central}{}\ledrightnote{\textcolor{pink}{Café Central}} käme, wo wir dann immer ein
                    Stündchen plaudern könnten – freilich nur wenn Sie dort ſind. Für alle Fälle
                    pneumatiſiren Sie mir wegen der Ka{\geminationm}ermuſik und
                    behalten mich in freundlicher Erinnerung.\pend
           \pstart Ihr \spacefill\mbox{Arthur}\pend{}\endnumbering\briefempfaengerindex{Hofmannsthal, Hugo von@\textsc{Hofmannsthal, Hugo von}!zzzSchnitzler, Arthur@\emph{von Arthur Schnitzler}!1894-01-181@{{[}18. 1. 1894{]}}|)be}\mylabel{h}  \normalsize

\doendnotes{C}
\bigskip
\vfill

\clearpage

\footnotesize

\lohead{\textsc{register}}

% Definiere theindex-Environment komplett neu ohne reledmac
\makeatletter
\renewenvironment{theindex}{%
  \section*{\indexname}%
  \setlength{\parindent}{0pt}%
  \setlength{\parskip}{0pt plus 0.3pt}%
  \let\item\@idxitem
}{%
  \clearpage
}
\makeatother

\IfFileExists{\jobname-pw.ind}{\input{\jobname-pw.ind}}{}

\end{document}

      