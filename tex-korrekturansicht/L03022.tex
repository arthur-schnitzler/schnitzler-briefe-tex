%% latex-korrekturansicht-vorspann.tex
%% Vorspann für die Korrekturansicht.
%% Lädt die gemeinsame Datei latex-vorspann.tex mit gesetztem Schalter.

\newif\ifkorrekturansicht
\korrekturansichttrue

\input{../tex-inputs/latex-vorspann}


\renewcommand{\erwaehntePersonen}{Personen: Ernst Benedikt, Édouard Bourdet, Stefan Hock, Eduard von Keyserling, Anna Katharina Rehmann, Felix Salten, Theodor Salzmann, Helene Thimig, Adolph von Zsolnay, Amanda von Zsolnay}
\renewcommand{\erwaehnteInstitutionen}{Institutionen: Paul Zsolnay Verlag, Ullstein Verlag}
\renewcommand{\erwaehnteOrte}{Orte: Deutschland, Dresden, Sanatorium am Königspark, Wien}
\renewcommand{\erwaehnteWerke}{Werke: Bambi. Eine Lebensgeschichte aus dem Walde, Die Gefangene. Schauspiel in drei Akten, La Prisonnière, Martin Overbeck. Der Roman eines reichen jungen Mannes, Neue Freie Presse, Theodor}
\section[ Arthur Schnitzler an Felix Salten, 10. 2. 1927]{Arthur Schnitzler an Felix Salten, 10. 2. 1927}
\nopagebreak\mylabel{v}
\rehead{ }\normalsize\beginnumbering\briefempfaengerindex{Salten, Felix@\textsc{Salten, Felix}!zzzSchnitzler, Arthur@\emph{von Arthur Schnitzler}!1927-02-101@{10. 2. 1927}|(be}
\toendnotes[C]{\smallbreak\pagebreak[2]}\Standort{Wienbibliothek im Rathaus, ZPH 1681, 2.1.516.}
\physDesc{Brief, 1 Blatt, 2 Seiten, 1237 Zeichen
\newline{}Handschrift: Bleistift, lateinische Kurrent
\newline{}Ordnung: mit Bleistift von unbekannter Hand nummeriert: »3« }
\buchAbdrucke{\weitereDrucke{Arthur Schnitzler: \emph{Briefe 1913–1931}. Hg. Peter Michael Braunwarth, Richard Miklin, Susanne Pertlik und Heinrich Schnitzler. Frankfurt am Main: \emph{S. Fischer} 1984, S. 470–471.} }\toendnotes[C]{\smallbreak}
\pstart
           \raggedleft{}{\pb}\textcolor{pink}{Wien}{}\ledrightnote{\textcolor{pink}{Wien}}{ }10. 2. 927\pend
           
\pstart
           lieber, ich dank Ihnen sehr für Ihre \label{K_L03022-1v}\edtext{Karte}{\lemma{\textnormal{\emph{Karte}}}\Cendnote{\textnormal{Felix Salten an Arthur Schnitzler, 8. 2. 1927}}}\label{K_L03022-1h}. Glauben Sie nicht, daſs ich weniger und daſ\textcolor{gray}{s} ich anders
               Ihrer denke als in früherer Zeit. Daſs ich so wenig sicht- u hörbar bin liegt zum
               Theil an der etwas complicirten \introOben{}(\introOben{}und zeitraubenden\introOben{})\introOben{} Form\strikeout{)} die meine Existenz
                  angeno{\geminationm}en hat; und gar nicht daran, dſs \strikeout{ich} es mich nicht kü{\geminationm}ern sollte, wie es Ihnen geht.
               Ich wußte, dſs Sie in \textcolor{pink}{Dresden}{}\ledrightnote{\textcolor{pink}{Dresden}} im \textcolor{pink}{Sanatorium}{}\ledrightnote{{$\rightarrow$}\textcolor{pink}{Sanatorium am Königspark}}{ }\strikeout{\textcolor{gray}{×}\-\textcolor{gray}{×}\-\textcolor{gray}{×}\-\textcolor{gray}{×}} sind; \label{K_L03022-2v}\edtext{bei \textcolor{blue}{Zsolnays}{}\ledrightnote{\textcolor{blue}{Adolph von Zsolnay}{\newline}\textcolor{blue}{Amanda von Zsolnay}} (zu \textcolor{blue}{Keyserling}{}\ledrightnote{\textcolor{blue}{Eduard von Keyserling}}s Ehren)}{\lemma{\textnormal{\emph{bei … Ehren)}}}\Cendnote{\textnormal{siehe A. S.: \emph{Tagebuch}, 6. 2. 1927}}}\label{K_L03022-2h} hört ichs zuerst, und eben erst sprach auch \textcolor{blue}{Benedikt}{}\ledrightnote{\textcolor{blue}{Ernst Benedikt}}, bei dem ich heute zufällg zu Mittag
               aſs, davon, von Ihrer Arbeitskraft und allerlei sehr herzliches. Auch von dem weiten
               Wiederhall Ihres schönen \label{K_L03022-3v}\edtext{\textcolor{green}{Bambibuch}{}\ledrightnote{\textcolor{green}{Bambi. Eine Lebensgeschichte aus dem Walde}}}{\lemma{\textnormal{\emph{Bambibuch}}}\Cendnote{\textnormal{\textcolor{blue}{Schnitzler} bezog sich hier nicht auf die
                     1922 bei \emph{\textcolor{brown}{Ullstein}}
                  erschienene \emph{\textcolor{green}{Bambi}}-Ausgabe, sondern jene, die
                     1926 bei \emph{\textcolor{brown}{Paul
                     Zsolnay}} erschienen war.}}}\label{K_L03022-3h}es weiſs ich und dſs Sie einen \label{K_L03022-4v}\edtext{\textcolor{green}{Roman}{}\ledrightnote{{$\rightarrow$}\textcolor{green}{Martin Overbeck. Der Roman eines reichen jungen Mannes}}}{\lemma{\textnormal{\emph{Roman}}}\Cendnote{\textnormal{eventuell \emph{\textcolor{green}{Martin Overbeck. Der Roman eines reichen jungen Mannes}},
                  der aber bereits im April 1927 veröffentlicht wurde
                  und folglich schon fertiggeschrieben war?}}}\label{K_L03022-4h} schreiben\textcolor{gray}{.}{ }{\pb}Un\textcolor{gray}{d} habe neulich mit Ergriffenheit Ihr
                  \label{K_L03022-5v}\edtext{\textcolor{green}{Feu{[}i{]}lleton}{}\ledrightnote{{$\rightarrow$}\textcolor{green}{Theodor}}}{\lemma{\textnormal{\emph{Feuilleton}}}\Cendnote{\textnormal{\textcolor{blue}{Felix Salten}: \emph{\textcolor{green}{Theodor}}. In: \emph{\textcolor{green}{Neue
                        Freie Presse}}, Nr. 22.381, 6. 1. 1927,
                     Morgenblatt, S. 13.}}}\label{K_L03022-5h} (du{\geminationm}es Wort)
               über Ihren \textcolor{blue}{Bruder}{}\ledrightnote{{$\rightarrow$}\textcolor{blue}{Theodor Salzmann}} gelesen.
               Und mit Vergnügen gehört, daſs \textcolor{blue}{Annerl}{}\ledrightnote{\textcolor{blue}{Anna Katharina Rehmann}} (we{\geminationn} man noch so sagen darf) nun auch ein
               schauspielerisches Talent in sich entdeckt hat und als »\label{K_L03022-6v}\edtext{\textcolor{green}{Mitgefangne}{}\ledrightnote{{$\rightarrow$}\textcolor{green}{Die Gefangene. Schauspiel in drei Akten}}« von 
               \textcolor{blue}{Helene Thimig}{}\ledrightnote{\textcolor{blue}{Helene Thimig}}}{\lemma{\textnormal{\emph{Mitgefangne« … Thimig}}}\Cendnote{\textnormal{\textcolor{blue}{Helene Thimig} tourte
                  mit dem »Schauspiel in drei Akten« \emph{\textcolor{green}{Die Gefangene}} (\emph{\textcolor{green}{La Prisonnière}}) von \textcolor{blue}{Édouard Bourdet},
                  deutsch von \textcolor{blue}{Stefan Hock}. Das Stück hatte am 21. 5. 1921 in \textcolor{pink}{Wien}
                  die deutschsprachige Uraufführung gehabt. \textcolor{blue}{Schnitzler} sah die Aufführung am 5. 6. 1926.
                  Für die Tournee waren die meisten Rollen neu besetzt worden, darunter \textcolor{blue}{Anna Katharina Salten}.}}}\label{K_L03022-6h}
               in \textcolor{pink}{Deutschland}{}\ledrightnote{\textcolor{pink}{Deutschland}} herumreist. Bescheidene Stichproben
               von meinem Wissen um Sie. Ich hoffe, Sie ergänzen \substVorne{}\textsuperscript{\textcolor{gray}{m}}\substDazwischen{}es\substHinten{} bald. Wa{\geminationn} ko{\geminationm}en Sie wieder? Ich
               habe vorläufg keine Reise-Absichten. Also »klopfen« oder telefoniren Sie bald. Ich
               freu mich darauf\textcolor{gray}{,} Sie endlich einmal wieder ausführlicher zu sprechen.\pend
           
\pstart
           Von Her\textcolor{gray}{zen} Ihr {\\[\baselineskip]}\spacefill\mbox{Arthur}\pend
           \leftskip=0em{}\endnumbering\briefempfaengerindex{Salten, Felix@\textsc{Salten, Felix}!zzzSchnitzler, Arthur@\emph{von Arthur Schnitzler}!1927-02-101@{10. 2. 1927}|)be}\mylabel{h}  \normalsize

\doendnotes{C}
\bigskip
\vfill

\clearpage

\footnotesize

\lohead{\textsc{register}}

% Definiere theindex-Environment komplett neu ohne reledmac
\makeatletter
\renewenvironment{theindex}{%
  \section*{\indexname}%
  \setlength{\parindent}{0pt}%
  \setlength{\parskip}{0pt plus 0.3pt}%
  \let\item\@idxitem
}{%
  \clearpage
}
\makeatother

\IfFileExists{\jobname-pw.ind}{\input{\jobname-pw.ind}}{}

\end{document}

      