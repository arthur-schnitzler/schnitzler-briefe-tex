%% latex-korrekturansicht-vorspann.tex
%% Vorspann für die Korrekturansicht.
%% Lädt die gemeinsame Datei latex-vorspann.tex mit gesetztem Schalter.

\newif\ifkorrekturansicht
\korrekturansichttrue

\input{../tex-inputs/latex-vorspann}


\renewcommand{\erwaehntePersonen}{Personen: Alfred von Berger, Hedwig Bleibtreu, Felix Salten, Ottilie Salten, Adele Sandrock, Olga Schnitzler}
\renewcommand{\erwaehnteInstitutionen}{Institutionen: Burgtheater}
\renewcommand{\erwaehnteOrte}{Orte: Semmering, Wien}
\renewcommand{\erwaehnteWerke}{}
\section[ Felix Salten an Arthur Schnitzler, 14. 10. 1910]{Felix Salten an Arthur Schnitzler, 14. 10. 1910}
\nopagebreak\mylabel{v}
\rehead{ }\normalsize\beginnumbering\briefempfaengerindex{Schnitzler, Arthur@\textsc{Schnitzler, Arthur}!zzzSalten, Felix@\emph{von Felix Salten}!1910-10-141@{14. 10. 1910}|(be}
\toendnotes[C]{\smallbreak\pagebreak[2]}\Standort{CUL, Schnitzler, B 89, B 2.}
\physDesc{Brief, 1 Blatt, 1 Seite, 1151 Zeichen
\newline{}Handschrift: schwarze Tinte, lateinische Kurrent
\newline{}Ordnung: mit Bleistift von unbekannter Hand nummeriert: »267« }\toendnotes[C]{\smallbreak}
\pstart
           \noindent{}{\pb}\textcolor{gray}{\textbf{\textsc{Felix Salten}}}\pend
           
\pstart
           \raggedleft{}14. X. 10\pend
           
\pstart{}Lieber,\pend
\pstart
           ich möchte Ihnen, eh’ Sie \label{K_L03552-1v}\edtext{auf den \textcolor{pink}{Semmering}{}\ledrightnote{\textcolor{pink}{Semmering}} fahren}{\lemma{\textnormal{\emph{auf den Semmering fahren}}}\Cendnote{\textnormal{\textcolor{blue}{Schnitzler} fuhr am 16. 10. 1910 auf den
                     \textcolor{pink}{Semmering} und blieb dort bis zum 19. 10. 1910.}}}\label{K_L03552-1h},
               rasch noch mitteilen, dass ich gestern{ }Abends mit \textcolor{blue}{Berger}{}\ledrightnote{\textcolor{blue}{Alfred von Berger}} sprach, und die
               Gelegenheit wahrnahm, ein Wort für die \textcolor{blue}{Sandrock}{}\ledrightnote{\textcolor{blue}{Adele Sandrock}} zu sagen. \textcolor{blue}{Berger}{}\ledrightnote{\textcolor{blue}{Alfred von Berger}} ist bereit,
                  \label{K_L03552-2v}\edtext{sie zu \textcolor{brown}{engagiren}{}\ledrightnote{{$\rightarrow$}\textcolor{brown}{Burgtheater}}}{\lemma{\textnormal{\emph{sie zu engagiren}}}\Cendnote{\textnormal{Dazu kam es nicht.}}}\label{K_L03552-2h}. Bedingungen:
               sie darf nicht gleich, darf überhaupt in diesem ersten Jahr keinen Vorschuß
               verlangen, weil dafür kein Geld zu haben ist und sie dem \textcolor{blue}{Direktor}{}\ledrightnote{{$\rightarrow$}\textcolor{blue}{Alfred von Berger}} mit solchen Affairen
               Verlegenheiten bereiten würde. Dann: sie muß sich für den Anfang mit 8 bis
               10.000 Kronen Gage begnügen; muß auch wegen Rollen Geduld haben und darf dabei sicher
               sein, dass sie würdige Aufgaben erhält. \textcolor{blue}{Berger}{}\ledrightnote{\textcolor{blue}{Alfred von Berger}}’s Worte: »Ich werde die \textcolor{blue}{Sandrock}{}\ledrightnote{\textcolor{blue}{Adele Sandrock}}
               nicht untergehen laßen!« Dass sie neben der \textcolor{blue}{Bleibtreu}{}\ledrightnote{\textcolor{blue}{Hedwig Bleibtreu}} Platz haben wird, hält er für sicher. Vielleicht teilen Sie ihr
               das mit. Ich glaube, es wird ihr lieber sein als ein Varieté-Stück. Sie kann sich,
               wenn sie die Sache auf dieser Basis betreiben will, mit mir in Verbindung setzen. \textcolor{blue}{Berger}{}\ledrightnote{\textcolor{blue}{Alfred von Berger}} ist am Sonntag zu Mittag bei mir. Es wäre gut, wenn ich bis dahin
               eine Zeile von der \textcolor{blue}{Sandrock}{}\ledrightnote{\textcolor{blue}{Adele Sandrock}} hätte. Auf den \textcolor{pink}{Semmering}{}\ledrightnote{\textcolor{pink}{Semmering}} kann ich leider nicht. Wir wünschen Frau
                  \label{K_L03552-3v}\edtext{\textcolor{blue}{Olga}{}\ledrightnote{\textcolor{blue}{Olga Schnitzler}} schöne Erholung}{\lemma{\textnormal{\emph{Olga schöne Erholung}}}\Cendnote{\textnormal{siehe A. S.: \emph{Tagebuch}, 7. 10. 1910}}}\label{K_L03552-3h} und Ihnen \textcolor{blue}{Beiden}{}\ledrightnote{{$\rightarrow$}\textcolor{blue}{Olga Schnitzler}}
               gutes Wetter!\pend
           
\pstart
           Herzlich von \textcolor{blue}{uns}{}\ledrightnote{{$\rightarrow$}\textcolor{blue}{Ottilie Salten}} zu
               Ihnen {\\[\baselineskip]}Ihr {\\[\baselineskip]}\spacefill\mbox{Salten}\pend
           \leftskip=0em{}\endnumbering\briefempfaengerindex{Schnitzler, Arthur@\textsc{Schnitzler, Arthur}!zzzSalten, Felix@\emph{von Felix Salten}!1910-10-141@{14. 10. 1910}|)be}\mylabel{h}  \normalsize

\doendnotes{C}
\bigskip
\vfill

\clearpage

\footnotesize

\lohead{\textsc{register}}

% Definiere theindex-Environment komplett neu ohne reledmac
\makeatletter
\renewenvironment{theindex}{%
  \section*{\indexname}%
  \setlength{\parindent}{0pt}%
  \setlength{\parskip}{0pt plus 0.3pt}%
  \let\item\@idxitem
}{%
  \clearpage
}
\makeatother

\IfFileExists{\jobname-pw.ind}{\input{\jobname-pw.ind}}{}

\end{document}

      