%% latex-korrekturansicht-vorspann.tex
%% Vorspann für die Korrekturansicht.
%% Lädt die gemeinsame Datei latex-vorspann.tex mit gesetztem Schalter.

\newif\ifkorrekturansicht
\korrekturansichttrue

\input{../tex-inputs/latex-vorspann}


               \section[Paul Goldmann an Arthur Schnitzler, Paul Goldmann an Arthur Schnitzler, 26. 9. {[}1896{]}]{ Paul Goldmann an Arthur Schnitzler, 26. 9. {[}1896{]}}\nopagebreak\mylabel{v}\rehead{ }\normalsize\beginnumbering\briefempfaengerindex{Schnitzler, Arthur@\textsc{Schnitzler, Arthur}!zzzGoldmann, Paul@\emph{von Paul Goldmann}!1896-09-261@{26. 9. {[}1896{]}}|(be} \toendnotes[C]{\smallbreak\pagebreak[2]} \Standort{DLA, A:Schnitzler, HS.NZ85.1.3166.}
\physDesc{Brief, 2 Blätter, 7 Seiten
\newline{}Handschrift: blaue Tinte, deutsche Kurrent\newline{}Beilagen: 1) Thorel: handschriftlicher Brief: 1 Blatt, 1 Seite, schwarze
                                 Tinte, lateinische Kurrent. Mit Bleistift von Schnitzler datiert:
                                    »Sept 96« 2) Nansen: handschriftlicher Brief: 1 Blatt, 4 Seiten, schwarze
                                 Tinte, lateinische Kurrent.
\newline{}Schnitzler: 1) mit Bleistift das Jahr »96« vermerkt 2) mit rotem Buntstift sechs Unterstreichungen}\toendnotes[C]{\smallbreak}\pstart
           \noindent{}{\pb}\textcolor{gray}{\textbf{\textbf{\textcolor{brown}{Frankfurter Zeitung}{}\ledrightnote{\textcolor{brown}{Frankfurter Zeitung}}}}}\pend
           \pstart
           \textcolor{gray}{\textbf{(\textcolor{brown}{\begin{otherlanguage}{french}Gazette de Francfort\end{otherlanguage}}{}\ledrightnote{\textcolor{brown}{Frankfurter Zeitung}}).}}\pend
           \pstart
           \textcolor{gray}{\textbf{\textbf{\begin{otherlanguage}{french}Fondateur M.\end{otherlanguage}{ }\textcolor{blue}{L. Sonnemann}{}\ledrightnote{\textcolor{blue}{Leopold Sonnemann}}.}}}\pend
           \pstart
           d \begin{otherlanguage}{french}\textcolor{gray}{\textbf{\textcolor{green}{Journal}{}\ledrightnote{→\textcolor{green}{Frankfurter Zeitung}} politique,
                        financier,}}\end{otherlanguage}\pend
           \pstart
           \begin{otherlanguage}{french}\textcolor{gray}{\textbf{commercial et littéraire.}}\end{otherlanguage}\pend
           \pstart
           \begin{otherlanguage}{french}\textcolor{gray}{\textbf{\textbf{Paraissant trois fois par jour.}}}\end{otherlanguage}\hfill \textsc{\textcolor{pink}{Paris}{}\ledrightnote{\textcolor{pink}{Paris}}}, 26. September.\pend
           \pstart
           \begin{otherlanguage}{french}\textcolor{gray}{\textbf{\textbf{Bureau à \textcolor{pink}{Paris}{}\ledrightnote{\textcolor{pink}{Paris}}}}}\end{otherlanguage}\pend
           \pstart
           \begin{otherlanguage}{french}\textcolor{gray}{\textbf{\textbf{\textcolor{pink}{24. Rue Feydeau}{}\ledrightnote{\textcolor{pink}{rue Feydeau}}.}}}\end{otherlanguage}\pend
           \pstart\center{}Mein lieber Freund,\pend\pstart
            Ich beſtätige Dir den Empfang der 500 \textsc{Francs}, die ich
               gleich an \textsc{\textcolor{blue}{Thorel}{}\ledrightnote{\textcolor{blue}{Jean Thorel}}} weitergeben will. Anbei ein Brief von ihm.\pend
           \pstart
           Ich füge ferner einen Brief von \textsc{\textcolor{blue}{Nansen}{}\ledrightnote{\textcolor{blue}{Peter Nansen}}{ }\strikeout{bei}} bei, den ich dieſer Tage erhielt, nachdem ich ſeiner \textcolor{blue}{Frau}{}\ledrightnote{→\textcolor{blue}{Betty Nansen}} franzöſiſche \label{K_L02786-2v}\edtext{\textsc{\begin{otherlanguage}{french}chansons\end{otherlanguage}}}{\lemma{\textnormal{\emph{chansons}}}\Cendnote{\textnormal{französisch: Lieder}}}\label{K_L02786-2h} geſchickt.
               Ihr ſolltet dem \textcolor{blue}{Manne}{}\ledrightnote{→\textcolor{blue}{Peter Nansen}} einen
                  \label{K_L02786-3v}\edtext{Gruß ſchreiben}{\lemma{\textnormal{\emph{Gruß ſchreiben}}}\Cendnote{\textnormal{In Folge schrieb \textcolor{blue}{Schnitzler} am 28. 9. 1896 an
                     \textcolor{blue}{Peter Nansen}. Siehe \emph{Peter Nansen – Arthur Schnitzler. Ein Briefwechsel zweier
                        Geistesverwandter}. Herausgegeben, kommentiert und mit einem Nachwort
                     versehen von Karin Bang. Roskilde: \emph{Zentrum für
                        österreichisch-nordische Kulturstudien}{ }2003, S. 5–6 (Småskrifter fra CØNK / Kleine Schriften
                     von ZÖNK 9).}}}\label{K_L02786-3h}, denke ich.\pend
           \pstart
           Es thut mir von Herzen leid, daß Dich die \textcolor{pink}{Wien}{}\ledrightnote{\textcolor{pink}{Wien}}er
               Nervoſitäten wieder haben. Gibts denn {\pb}gar kein
               Mittel dagegen? Geh’ doch auf ein paar Wochen nach dem Süden!\pend
           \pstart
           \label{K_L02786-4v}\edtext{Was hörſt Du aus \textcolor{pink}{Berlin}{}\ledrightnote{\textcolor{pink}{Berlin}} über Dein \textcolor{green}{Stück}{}\ledrightnote{→\textcolor{green}{Freiwild. Schauspiel in 3 Akten}}}{\lemma{\textnormal{\emph{Was … Stück}}}\Cendnote{\textnormal{\textcolor{blue}{Goldmann} wollte wissen, wie die
                  Vorbereitungen zur Uraufführung von \emph{\textcolor{green}{Freiwild}}
                  vorangingen. Vgl. \emph{Der Briefwechsel Arthur Schnitzler — Otto
                        Brahm}. Vollständige Ausgabe. Herausgegeben, eingeleitet und
                     erläutert von Oskar Seidlin. Tübingen: \emph{Niemeyer}{ }1975, S. 14–28. }}}\label{K_L02786-4h}? Daß es Dir zuwider iſt, verſteht
               ſich von ſelbſt. Das iſt die natürliche Reaction gegen die ungeheure Arbeit, die Du
               darauf verwandt haſt.\pend
           \pstart
           Dieſer Tage war ein \textsc{\textcolor{blue}{Arthur Holitscher}{}\ledrightnote{\textcolor{blue}{Arthur Holitscher}}} bei mir. Was iſt das? Er hat zunächſt gegen ſich, daß er von \textsc{\textcolor{blue}{Bahr}{}\ledrightnote{\textcolor{blue}{Hermann Bahr}}} empfohlen wird. Auch ſonſt ſieht er mehr nach einem Lausbuben aus, als nach
               irgend etwas Anderem.\pend
           \pstart
           Der \textsc{\textcolor{blue}{Schiller}{}\ledrightnote{\textcolor{blue}{Friedrich von Schiller}}-\textcolor{blue}{Goethe}{}\ledrightnote{\textcolor{blue}{Johann Wolfgang von Goethe}}sche}{ }\textcolor{green}{Briefwechſel}{}\ledrightnote{\textcolor{green}{Briefwechsel zwischen Schiller und Goethe}} macht mich ſehr {\pb}nervös. Dieſe Leute, die ſich über nichts als über
               Bücher und ſonſtiges Literariſches ſchreiben! Dieſes unerträglich Gönnerhafte von
               Seiten \textsc{\textcolor{blue}{Goethe}{}\ledrightnote{\textcolor{blue}{Johann Wolfgang von Goethe}}s}, der den vornehmen Herrn
               gegenüber dem Profeſſor ſpielt (»Mein Wertheſter«, »werther Mann«) und gegenüber dem
               Mann in kleinen Verhältniſſen mit ſeinen Reiſen renommirt, \strikeout{ſ} mit ſeinem Reitpferde (»Ein Ritt von \textcolor{pink}{Weimar}{}\ledrightnote{\textcolor{pink}{Weimar}} nach \textsc{\textcolor{pink}{Jena}{}\ledrightnote{\textcolor{pink}{Jena}}} wird mir gut thun«) \textsc{etc}. Und dieſes nicht minder
               unerträgliche Sich-Geehrt-Fühlen von Seiten \textsc{\textcolor{blue}{Schiller}{}\ledrightnote{\textcolor{blue}{Friedrich von Schiller}}s}! Eigentlich drückt ſich nur
                  \textsc{\textcolor{blue}{Goethe}{}\ledrightnote{\textcolor{blue}{Johann Wolfgang von Goethe}}} frei aus in dieſer Correſpondenz, {\pb}bei \textsc{\textcolor{blue}{Schiller}{}\ledrightnote{\textcolor{blue}{Friedrich von Schiller}}} merkt man immer die Gedrücktheit. An ihm ſieht man, was für ein
               kleinbürgerlicher \strikeout{\textcolor{gray}{a}} armer Kerl doch ein deutſcher Dichter iſt! Nein, ein Briefwechſel iſt nur
               erfreulich zwiſchen zwei Gleichſtehenden. Ich finde den unſeren viel intereſſanter,
               als das, was ich bisher von dem zwiſchen \textsc{\textcolor{blue}{Goethe}{}\ledrightnote{\textcolor{blue}{Johann Wolfgang von Goethe}}} und \textsc{\textcolor{blue}{Schiller}{}\ledrightnote{\textcolor{blue}{Friedrich von Schiller}}} kenne.\pend
           \pstart
           Was mit \textsc{\textcolor{blue}{Dreyfus}{}\ledrightnote{\textcolor{blue}{Alfred Dreyfus}}} weiter wird, fragſt Du? Gar nichts. Der \textcolor{blue}{Mann}{}\ledrightnote{→\textcolor{blue}{Alfred Dreyfus}} bleibt, wo er iſt, und wird unſchuldig gemordet, wenn
               nicht ein Wunder geſchieht. Die Enthüllungen der Preſſe, welche den {\pb}unerhörten Blödſinn bewieſen, auf dem die Anklage
               aufgebaut iſt, werden hier als niederſchmetternde Schuldbeweiſe betrachtet. Meine
                  \label{K_L02786-14v}\edtext{\textcolor{green}{Artikel}{}\ledrightnote{→\textcolor{green}{Die Enthüllungen über die Affaire Dreyfus}}}{\lemma{\textnormal{\emph{Artikel}}}\Cendnote{\textnormal{\textcolor{blue}{G.} [=\textcolor{blue}{Paul Goldmann}]: \emph{\textcolor{green}{Die Enthüllungen über
                        die Affaire Dreyfus}}. In: \emph{\textcolor{green}{Frankfurter
                        Zeitung}}, Jg. 41, Nr. 258, 16. 9. 1896, Erstes Morgenblatt,
                     S. 1. Seither war nur eine ungezeichnete \textcolor{green}{Notiz} mit einem Brief der Gattin \textcolor{blue}{Lucie Dreyfus} erschienen (\emph{\textcolor{green}{Frankfurter Zeitung}}, Jg. 41, Nr. 261,
                        19. 9. 1896, Abendblatt, S. 2). Eventuell spielt \textcolor{blue}{Goldmann} auf frühere Artikel an, die er seit
                  dem ersten Urteil gegen \textcolor{blue}{Alfred Dreyfus} im
                     Dezember 1894 publiziert hatte?}}}\label{K_L02786-14h} haben nur den \uline{einen} Erfolg gehabt, daß ſie \uline{mir} geſchadet haben. Nicht nur daß ich in der Preſſe öffentlich
               beſchimpft worden bin – auch meine \textcolor{pink}{franzöſiſch}{}\ledrightnote{→\textcolor{pink}{Frankreich}}en Freunde haben mich mit Vorwürfen überſchüttet: »Was geht Sie
               dieſe Geſchichte an? Niemand wird mehr mit Ihnen verkehren können« \textsc{etc.} Wenn mich ein guter {\pb}\label{K_L02786-15v}\edtext{Bekannter in einer
                  Redactionsſtube}{\lemma{\textnormal{\emph{Bekannter … Redactionsſtube}}}\Cendnote{\textnormal{nicht
                  identifiziert}}}\label{K_L02786-15h} vertheidigen will, ſo wird ihm geantwortet: »Fragen Sie ihn
               nur, welchen Grad er in der \textcolor{pink}{deutſch}{}\ledrightnote{→\textcolor{pink}{Deutschland}}en Reſerve einnimmt« \textsc{etc}. Mangels weiteren
               Materials habe ich natürlich die Campagne einſtellen müſſen. Sobald es aber wieder
               losgeht – und es wird wieder losgehen – fange auch ich wieder an. Es kann mir ſehr
               ſchlecht dabei gehen – aber das iſt \strikeout{j\textcolor{gray}{a}} mir gleichgiltig. Das iſt ja gerade das Schöne in unſerem Metier, daß {\pb}man die Unſchuldigen vertheidigen und die Schwachen
               ſchützen kann. \textsc{\textcolor{green}{Don Quixote}{}\ledrightnote{→\textcolor{green}{Don Quijote}}} iſt ein herrliches Vorbild für einen \strikeout{Jou}
               Journaliſten.\pend
           \pstart
           Wie iſts mit \textsc{\textcolor{blue}{Ebermann}{}\ledrightnote{\textcolor{blue}{Leo Ebermann}}} gegangen? Ich höre, man hat ihn als \label{K_L02786-17v}\edtext{zweiten \textsc{\textcolor{blue}{Grillparzer}{}\ledrightnote{\textcolor{blue}{Franz Grillparzer}}}}{\lemma{\textnormal{\emph{zweiten Grillparzer}}}\Cendnote{\textnormal{wohl wegen der in Werken \textcolor{blue}{Grillparzer}s vergleichbaren Antikisierung in
                  der \emph{\textcolor{green}{Athenerin}}}}}\label{K_L02786-17h} begrüßt. Und was iſt das für ein Schwindel mit dem \label{K_L02786-18v}\edtext{in \textcolor{pink}{Berlin}{}\ledrightnote{\textcolor{pink}{Berlin}} aufgeführten
                  \textcolor{green}{Stücke}{}\ledrightnote{→\textcolor{green}{Juana. Drama}} von \textsc{\textcolor{blue}{Bahr}{}\ledrightnote{\textcolor{blue}{Hermann Bahr}}}}{\lemma{\textnormal{\emph{in … Bahr}}}\Cendnote{\textnormal{\textcolor{blue}{Bahr}s Einakter \emph{\textcolor{green}{Juana}} wurde am 22. 9. 1896
                  am \textcolor{pink}{Neuen Theater} in \textcolor{pink}{Berlin} uraufgeführt. \textcolor{blue}{Goldmann}s Vorwurf des »Schwindel« bezieht sich darauf,
                  dass Bahr nur als Übersetzer am Theaterzettel stand, als Autorname aber \textcolor{blue}{Alejandro Lanza} vermerkt
                  war. Bereits die ersten Besprechungen des Stückes konnten berichten, dass es sich
                  dabei um ein Pseudonym \textcolor{blue}{Bahr}s
                  handelte.}}}\label{K_L02786-18h}?\pend
           \pstart
           Grüß’ Dich Gott!\pend
           \pstart
           Schreib’ bald!\pend
           \pstart
           Dein treuer {\\[\baselineskip]}\spacefill\mbox{Paul Goldmann.}\pend
           \leftskip=0em{}\pstart
           \noindent{}Empfiehl’ mich der \label{K_L02786-777v}\edtext{geheimnißvollen \textcolor{blue}{Dame}{}\ledrightnote{→\textcolor{blue}{Marie Reinhard}}}{\lemma{\textnormal{\emph{geheimnißvollen Dame}}}\Cendnote{\textnormal{Arthur Schnitzler an Paul Goldmann, 21. 11. 1896}}}\label{K_L02786-777h}!\pend
           {\bigskip}\pstart
           \raggedleft{}{\pb}{[}hs. Thorel:{]} \begin{otherlanguage}{french}\textcolor{pink}{12 rue de milan}{}\ledrightnote{\textcolor{pink}{Rue de Milan}}{\\}\label{K_L02786-88v}\edtext{jeudi}{\lemma{\textnormal{\emph{jeudi}}}\Cendnote{\textnormal{französisch: Donnerstag. Der
                        Brief könnte demnach vom Vortag, dem 25. 9. 1896 stammen}}}\label{K_L02786-88h}.\end{otherlanguage}\pend
           \pstart\center{}\begin{otherlanguage}{french}Cher monsieur Goldmann,\end{otherlanguage}\pend\pstart
           \label{K_L02786-13v}\edtext{\begin{otherlanguage}{french}Je suis en plein travail – j’ai déjà presque fini le premier
                     \textcolor{green}{acte}{}\ledrightnote{→\textcolor{green}{Amourette. Pièce en trois actes}} – j’aurai voulu
                  vous le montrer, mais mes dates de voyage en au propagé à \textcolor{pink}{Paris}{}\ledrightnote{\textcolor{pink}{Paris}} ont été un peu brouillées, et je depars tout à l’heure
                  pour \textcolor{pink}{Auxerre}{}\ledrightnote{\textcolor{pink}{Auxerre}} où je resterai une huitaine de
                  jours. Sitôt rentré, je suis verrai, \strikeout{\textcolor{gray}{×}} je terminerai.\end{otherlanguage}}{\lemma{\textnormal{\emph{Je … terminerai.}}}\Cendnote{\textnormal{französisch: Ich bin mitten in der
                  Arbeit – ich habe den ersten \textcolor{green}{Akt} schon fast fertig – ich hätte ihn Ihnen gerne gezeigt, aber meine
                  Reise- und XXXXdaten in \textcolor{pink}{Paris} sind ein wenig
                  durcheinander geraten, und ich fahre umgehend nach \textcolor{pink}{Auxerre}, wo ich etwa acht Tage bleiben werde. Sobald ich
                  zurück bin, werde ich sehen, was ich tun kann, und es beenden.}}}\label{K_L02786-13h}\pend
           \pstart
           \label{K_L02786-123v}\edtext{\begin{otherlanguage}{french}\textcolor{gray}{A mesure que je la pénètre davantage}, je me rends de plus en
                  plus compte combien c’est exquis, cette petite \textcolor{green}{pièce}{}\ledrightnote{→\textcolor{green}{Liebelei. Schauspiel in drei Akten}}; et, avec cela, d’une habileté consommée. Et nous
                  aurons faire là un joli cadeau aux \textcolor{pink}{Paris}{}\ledrightnote{\textcolor{pink}{Paris}}iens.\end{otherlanguage}}{\lemma{\textnormal{\emph{A … Parisiens.}}}\Cendnote{\textnormal{französisch: Je mehr ihr weiter
                  vordringe, umso mehr merke ich, wie besonders dieses kleine \textcolor{green}{Stück} ist; und wie geschickt es gemacht
                  ist. Und wir werden den \textcolor{pink}{Paris}ern ein schönes
                  Geschenk machen.}}}\label{K_L02786-123h}\pend
           \pstart
           \label{K_L02786-2345v}\edtext{\begin{otherlanguage}{french}Bien à vous\end{otherlanguage}}{\lemma{\textnormal{\emph{Bien à vous}}}\Cendnote{\textnormal{französisch: Der Ihre}}}\label{K_L02786-2345h}{ }{\\[\baselineskip]}\spacefill\mbox{\textcolor{blue}{Jean Thorel}{}\ledrightnote{\textcolor{blue}{Jean Thorel}}}\pend
           \leftskip=0em{}{\bigskip}\pstart
           \raggedleft{}{\pb}\textcolor{pink}{Kopenhagen}{}\ledrightnote{\textcolor{pink}{Kopenhagen}}{ }20 Sept. 96\pend
           \pstart{}Lieber Herr Goldmann!\pend\pstart
           Wenn ich nicht eher geschrieben habe, ist der Grund meine Manieristische Furcht für
               die deutsche Sprache. Oft habe ich \introOben{}an\introOben{} Ihnen gedacht, an
               Ihnen und Ihren Freunden. Ja, lieber Herr, Freundschaft und Sympathie kann man sich
               nicht verklaren. Vom ersten Tag’, ich Sie sah, habe ich Sie lieb, und ich hoffe, wie
               Sie, dass unsre Freundschaft in aller Zukunft dauern wird – {\pb}auch wenn ich ein schlechter Briefschreiber bin.\pend
           \pstart
           Ich vergesse aber ganz meinen Dank \strikeout{\textcolor{gray}{z}} und den meiner \textcolor{blue}{Frau}{}\ledrightnote{→\textcolor{blue}{Betty Nansen}} zu
               bringen für die Zusendung der franzoesi{[}s{]}chen Chansons. Meine \textcolor{blue}{Frau}{}\ledrightnote{→\textcolor{blue}{Betty Nansen}} freut sich sehr sie zu
               singen – ich sie zu hören.\pend
           \pstart
           Ich bin jetzt Subscribent der \textcolor{green}{Frankf. Zeitung}{}\ledrightnote{\textcolor{green}{Frankfurter Zeitung}}{ }\strikeout{\textcolor{gray}{g}} und habe neulich da ein ausgezeichnetes \textcolor{green}{\introOben{}\textcolor{blue}{Dreyfus}{}\ledrightnote{→\textcolor{blue}{Alfred Dreyfus}}-\introOben{}Feuilleton}{}\ledrightnote{→\textcolor{green}{Die Enthüllungen über die Affaire Dreyfus}} von Ihnen gelesen. Das ist das beste, was ich
               von dieser merkwürdigen Sache gelesen.\pend
           \pstart
           (Ich schreibe so undeutlich um meine Sprachfehler zu verbergen)\pend
           \pstart
           – – Ich wurde gestern in meinem {\pb}Schreiben
               unterbrochen und setze jetzt fort, d. 21.{ }\label{K_L02786-23v}\edtext{hujus}{\lemma{\textnormal{\emph{hujus}}}\Cendnote{\textnormal{lateinisch: von diesem [Monat]}}}\label{K_L02786-23h}\pend
           \pstart
           Meine \textcolor{blue}{Frau}{}\ledrightnote{→\textcolor{blue}{Betty Nansen}} hat i{[}n{]} diesen
               Tagen im \textcolor{brown}{königlichen Theater}{}\ledrightnote{\textcolor{brown}{Det Kongelige Teater}} ihre Entrée gehabt mit grossem Erfolg. In einer kleinen
                  \textcolor{blue}{Ibsen}{}\ledrightnote{\textcolor{blue}{Henrik Ibsen}}-Rolle. Frl. \textcolor{green}{Bernick}{}\ledrightnote{→\textcolor{green}{Samfundets Støtter. Skuespil i fire Akter}} in \textcolor{green}{Stützen der Gesellschaft}{}\ledrightnote{\textcolor{green}{Samfundets Støtter. Skuespil i fire Akter}}.\pend
           \pstart
           Dieses Jahr werde ich deutscher Journalist. Der
               vortreffliche Herr \textcolor{blue}{Fischer}{}\ledrightnote{\textcolor{blue}{Samuel Fischer}} hat mich engagiert
               vier \textcolor{green}{Briefe vom
                  Norden}{}\ledrightnote{→\textcolor{green}{Brief aus dem Norden}{\newline}→\textcolor{green}{Briefe aus dem Norden. II. Das Kopenhagener Theater}} in »\textcolor{green}{Neue deutsche Rundschau}{}\ledrightnote{\textcolor{green}{Neue Deutsche Rundschau}}« zu
               schreiben. Den \label{K_L02786-45v}\edtext{ersten Brief}{\lemma{\textnormal{\emph{ersten Brief}}}\Cendnote{\textnormal{\textcolor{blue}{Peter Nansen}: \emph{\textcolor{green}{Brief aus dem Norden}}. In: \emph{\textcolor{green}{Neue Deutsche Rundschau}}, Jg. 7 (1896),
                     Oktober, S. 1028–1033. Der nächste \textcolor{green}{Brief} erschien im März-Heft 1897.}}}\label{K_L02786-45h} habe ich schon
               fertig. Der kommt im October-Hefte.\pend
           \pstart
           Sie schreiben natürlich oft an Herrn Schnitzler und \textcolor{blue}{Beer-Hofmann}{}\ledrightnote{\textcolor{blue}{Richard Beer-Hofmann}}. Sagen – bitte – den zwei liebenswertesten Menschen, dass sie
               mir nicht böse sein dürfen, weil sie nichts von mir gehört haben. Sie wissen ja alle
                  {\pb}Drei den legitimen Grund meiner Stummheit.\pend
           \pstart
           Ach –
               könnten Sie nur alle drei recht oft \strikeout{\textcolor{gray}{ein}} Abendvisiten machen und mit uns plaudern und
               lachen und bisweilen – weil es auch gut ist – ein bischen sentimental sein.\pend
           \pstart
           Lieber Freund – ich sende Ihnen allen meine besten Grüsse und meine \textcolor{blue}{Frau}{}\ledrightnote{→\textcolor{blue}{Betty Nansen}} fügt ihre Grüsse zu den meinigen.\pend
           \pstart
           Vergessen Sie uns nicht \substVorne{}\textsuperscript{z}\substDazwischen{}u\substHinten{}nd schreiben Sie bald wieder.{\\[\baselineskip]} Ihr ergebner{\\[\baselineskip]}\spacefill\mbox{Peter Nansen}\pend
           \leftskip=0em{}\endnumbering\briefempfaengerindex{Schnitzler, Arthur@\textsc{Schnitzler, Arthur}!zzzGoldmann, Paul@\emph{von Paul Goldmann}!1896-09-261@{26. 9. {[}1896{]}}|)be}\mylabel{h}  \normalsize

\doendnotes{C}
\bigskip
\vfill

\clearpage

\footnotesize

\lohead{\textsc{register}}

% Definiere theindex-Environment komplett neu ohne reledmac
\makeatletter
\renewenvironment{theindex}{%
  \section*{\indexname}%
  \setlength{\parindent}{0pt}%
  \setlength{\parskip}{0pt plus 0.3pt}%
  \let\item\@idxitem
}{%
  \clearpage
}
\makeatother

\IfFileExists{\jobname-pw.ind}{\input{\jobname-pw.ind}}{}

\end{document}

      