%% latex-korrekturansicht-vorspann.tex
%% Vorspann für die Korrekturansicht.
%% Lädt die gemeinsame Datei latex-vorspann.tex mit gesetztem Schalter.

\newif\ifkorrekturansicht
\korrekturansichttrue

\input{../tex-inputs/latex-vorspann}


\renewcommand{\erwaehntePersonen}{Personen:  ?? [Haushaltshilfe von Felix Salten in der Kochgasse 1902], Ottilie Salten}
\renewcommand{\erwaehnteOrte}{Orte: Wien}
\renewcommand{\erwaehnteWerke}{}
\section[ Felix Salten an Arthur Schnitzler, {[}11. 3. 1902{]}]{Felix Salten an Arthur Schnitzler, {[}11. 3. 1902{]}}
\nopagebreak\mylabel{v}
\rehead{ }\normalsize\beginnumbering\briefempfaengerindex{Schnitzler, Arthur@\textsc{Schnitzler, Arthur}!zzzSalten, Felix@\emph{von Felix Salten}!1902-03-111@{{[}11. 3. 1902{]}}|(be}
\toendnotes[C]{\smallbreak\pagebreak[2]}\Standort{CUL, Schnitzler, B 89, A 2.}
\physDesc{Brief, 1 Blatt, 1 Seite, 281 Zeichen
\newline{}Handschrift: Bleistift, lateinische Kurrent
\newline{}Schnitzler: mit Bleistift datiert: »11/3 902« 
\newline{}Ordnung: mit Bleistift von unbekannter Hand nummeriert: »149« }\toendnotes[C]{\smallbreak}
\pstart
           \noindent{}{\pb}Lieber,{ }\textcolor{blue}{Otti}{}\ledrightnote{\textcolor{blue}{Ottilie Salten}} ist ausgegangen und dem \textcolor{blue}{Mädchen}{}\ledrightnote{{$\rightarrow$}\textcolor{blue}{?? [Haushaltshilfe von Felix Salten in der Kochgasse 1902]}} wurde gesagt, es solle nicht »alle
               Leute« zu mir laßen. Diese Gans hat keine bessere \label{K_L03325-1v}\edtext{Ausrede}{\lemma{\textnormal{\emph{Ausrede}}}\Cendnote{\textnormal{\textcolor{blue}{Schnitzler} dürfte nach dem Schreiben vom
                  Vortag (Felix Salten an Arthur Schnitzler, [10?. 3. 1902]) einen Krankenbesuch
                  unternommen haben. Nachdem dieser erste Versuch nicht geklappt hatte, kam er am
                     14. 3. 1902, als
                  es \textcolor{blue}{Salten} bereits besser ging, wieder und
                  wurde vorgelassen.}}}\label{K_L03325-1h} gewußt, als mich spazieren zu schicken. Ich bin \uline{natürlich} sehr zu Hause, d. h. im Bette, und hätte
               mich sehr gefreut Sie zu sehen.\pend
           
\pstart
           Herzlichst Ihr {\\[\baselineskip]}\spacefill\mbox{Salten}\pend
           \leftskip=0em{}\endnumbering\briefempfaengerindex{Schnitzler, Arthur@\textsc{Schnitzler, Arthur}!zzzSalten, Felix@\emph{von Felix Salten}!1902-03-111@{{[}11. 3. 1902{]}}|)be}\mylabel{h}  \normalsize

\doendnotes{C}
\bigskip
\vfill

\clearpage

\footnotesize

\lohead{\textsc{register}}

% Definiere theindex-Environment komplett neu ohne reledmac
\makeatletter
\renewenvironment{theindex}{%
  \section*{\indexname}%
  \setlength{\parindent}{0pt}%
  \setlength{\parskip}{0pt plus 0.3pt}%
  \let\item\@idxitem
}{%
  \clearpage
}
\makeatother

\IfFileExists{\jobname-pw.ind}{\input{\jobname-pw.ind}}{}

\end{document}

      