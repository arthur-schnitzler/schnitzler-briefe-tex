%% latex-korrekturansicht-vorspann.tex
%% Vorspann für die Korrekturansicht.
%% Lädt die gemeinsame Datei latex-vorspann.tex mit gesetztem Schalter.

\newif\ifkorrekturansicht
\korrekturansichttrue

\input{../tex-inputs/latex-vorspann}


\renewcommand{\erwaehntePersonen}{Personen: Marie Glümer, Auguste Glümer,  Meyer, Rudolf Renvers}
\renewcommand{\erwaehnteOrte}{Orte: Berlin, Dessauer Straße, Wien}
\renewcommand{\erwaehnteWerke}{}
\section[ Paul Goldmann an Arthur Schnitzler, 14. 2. {[}1901{]}]{Paul Goldmann an Arthur Schnitzler, 14. 2. {[}1901{]}}
\nopagebreak\mylabel{v}
\rehead{ }\normalsize\beginnumbering\briefempfaengerindex{Schnitzler, Arthur@\textsc{Schnitzler, Arthur}!zzzGoldmann, Paul@\emph{von Paul Goldmann}!1901-02-141@{14. 2. {[}1901{]}}|(be}
\toendnotes[C]{\smallbreak\pagebreak[2]}\Standort{DLA, A:Schnitzler, HS.NZ85.1.3171.}
\physDesc{Brief, 1 Blatt, 1 Seite
\newline{}Handschrift: blaue Tinte, deutsche Kurrent
\newline{}Beilage: handschriftlicher Brief von Dr. Meyer, 1 Blatt, 2 Seiten, schwarze Tinte,
                                 lateinische Kurrent 
\newline{}Schnitzler: mit Bleistift das Jahr »{[}1{]}901« vermerkt }\toendnotes[C]{\smallbreak}
\pstart
           \noindent{}\raggedleft{}{\pb}\textcolor{pink}{\textcolor{gray}{\textbf{DESSAUERSTRASSE 19}}}{}\ledrightnote{\textcolor{pink}{Dessauer Straße}}\pend
           
\pstart
           \textcolor{pink}{Berlin}{}\ledrightnote{\textcolor{pink}{Berlin}}, 14. Februar.\pend
           
\pstart\center{}Mein lieber Freund,\pend
\pstart
           Ein \label{K_L03058-1v}\edtext{\textsc{Dr. \textcolor{blue}{Meyer}{}\ledrightnote{\textcolor{blue}{Meyer}}}}{\lemma{\textnormal{\emph{Dr. Meyer}}}\Cendnote{\textnormal{nicht ermittelt}}}\label{K_L03058-1h}, der mit den \textcolor{blue}{Glümers}{}\ledrightnote{{$\rightarrow$}\textcolor{blue}{Marie Glümer}{\newline}{$\rightarrow$}\textcolor{blue}{Auguste Glümer}} bekannt
               iſt, hat \textsc{\textcolor{blue}{Mizzi}{}\ledrightnote{\textcolor{blue}{Marie Glümer}}} zu 
               \textsc{Prof. \textcolor{blue}{Renvers}{}\ledrightnote{\textcolor{blue}{Rudolf Renvers}}}
                begleitet. Ich
               bat \textsc{\textcolor{blue}{Gusti}{}\ledrightnote{\textcolor{blue}{Auguste Glümer}}}, mich mit dieſem \textsc{Dr. \textcolor{blue}{Meyer}{}\ledrightnote{\textcolor{blue}{Meyer}}} in Verbindung zu ſetzen. Die Folge\strikeout{n} war
               beiliegender Brief, aus dem ich auch nicht ſehr klug werde. Vielleicht ſagt er Dir
               mehr als mir.\pend
           
\pstart
           Viele Grüße! {\\[\baselineskip]}Dein {\\[\baselineskip]}\spacefill\mbox{Paul Goldmn}\pend
           \leftskip=0em{}
\pstart
           \raggedleft{}{\pb}{[}hs. Meyer:{]} \textcolor{pink}{B.}{}\ledrightnote{\textcolor{pink}{Berlin}}{ }Montag.\pend
           
\pstart\center{}Sehr geehrter Herr Doctor!\pend
\pstart
           Auf Wunſch von \label{K_L03058-3v}\edtext{\textcolor{blue}{Fräulein Glümer}{}\ledrightnote{\textcolor{blue}{Marie Glümer}}}{\lemma{\textnormal{\emph{Fräulein Glümer}}}\Cendnote{\textnormal{siehe Paul Goldmann an Arthur Schnitzler, 22. 1. [1901]}}}\label{K_L03058-3h} erlaube ich mir die ergebene Mitteilung, daß ihre Erkrankung auf einer
               ſchlechten Zuſammenſetzung des Blutes + eitrigen Körperſäfte beruht, deren Schwere
               durch die lange Vernachläſſigung beruhigt iſt. –\pend
           
\pstart
           Das Weſentliche für {\pb}ihre Freunde iſt
               ja die Thatſache, daß ſie in 4 Wochen ca mit Sicherheit völlig geſund ſein wird.\pend
           
\pstart
           Mit vorzüglichſter Hochſchätzung empfiehlt ſich Ihnen {\\[\baselineskip]}ganz
                  ergeb\textcolor{gray}{en}{ }{\\[\baselineskip]}\spacefill\mbox{\textcolor{gray}{Meyer}}\pend
           \leftskip=0em{}\endnumbering\briefempfaengerindex{Schnitzler, Arthur@\textsc{Schnitzler, Arthur}!zzzGoldmann, Paul@\emph{von Paul Goldmann}!1901-02-141@{14. 2. {[}1901{]}}|)be}\mylabel{h}
\begin{anhang}
\end{anhang}\normalsize

\doendnotes{C}
\bigskip
\vfill

\clearpage

\footnotesize

\lohead{\textsc{register}}

% Definiere theindex-Environment komplett neu ohne reledmac
\makeatletter
\renewenvironment{theindex}{%
  \section*{\indexname}%
  \setlength{\parindent}{0pt}%
  \setlength{\parskip}{0pt plus 0.3pt}%
  \let\item\@idxitem
}{%
  \clearpage
}
\makeatother

\IfFileExists{\jobname-pw.ind}{\input{\jobname-pw.ind}}{}

\end{document}

      