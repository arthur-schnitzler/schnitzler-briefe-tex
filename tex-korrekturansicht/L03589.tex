%% latex-korrekturansicht-vorspann.tex
%% Vorspann für die Korrekturansicht.
%% Lädt die gemeinsame Datei latex-vorspann.tex mit gesetztem Schalter.

\newif\ifkorrekturansicht
\korrekturansichttrue

\input{../tex-inputs/latex-vorspann}


\renewcommand{\erwaehntePersonen}{Personen: Felix Salten, Boris Sapiro}
\renewcommand{\erwaehnteOrte}{Orte: Wien}
\renewcommand{\erwaehnteWerke}{}
\section[ Felix Salten an Arthur Schnitzler, {[}16. 11. 1929{]}]{Felix Salten an Arthur Schnitzler, {[}16. 11. 1929{]}}
\nopagebreak\mylabel{v}
\rehead{ }\normalsize\beginnumbering\briefempfaengerindex{Schnitzler, Arthur@\textsc{Schnitzler, Arthur}!zzzSalten, Felix@\emph{von Felix Salten}!1929-11-161@{{[}16. 11. 1929{]}}|(be}
\toendnotes[C]{\smallbreak\pagebreak[2]}\Standort{CUL, Schnitzler, B 89, B 2.}
\physDesc{Briefkarte, 186 Zeichen
\newline{}Handschrift: Bleistift, lateinische Kurrent
\newline{}Schnitzler: 1) mit Bleistift datiert: »16. 11. 1929«  2) mit rotem Buntstift Vermerk »\textsc{\textcolor{blue}{Salten}}«
\newline{}Ordnung: mit Bleistift von unbekannter Hand nummeriert: »302« }\toendnotes[C]{\smallbreak}
\pstart
           \raggedleft{}{\pb}Samstag.\pend
           
\pstart
           Lieber, Sie wissen, ich schicke Niemanden zu Ihnen, aber diesen
               jungen, vielerfahrenen, merkwürdigen \textcolor{blue}{Menschen}{}\ledrightnote{{$\rightarrow$}\textcolor{blue}{Boris Sapiro}} werden Sie mit Interesse und amüsiert \label{K_L03589-1v}\edtext{anhören}{\lemma{\textnormal{\emph{anhören}}}\Cendnote{\textnormal{\textcolor{blue}{Boris Sapiro}, siehe A. S.: \emph{Tagebuch}, 18. 11. 1929}}}\label{K_L03589-1h}.\pend
           
\pstart
           Herzlichst Ihr {\\[\baselineskip]}\spacefill\mbox{Felix Salten}\pend
           \leftskip=0em{}\endnumbering\briefempfaengerindex{Schnitzler, Arthur@\textsc{Schnitzler, Arthur}!zzzSalten, Felix@\emph{von Felix Salten}!1929-11-161@{{[}16. 11. 1929{]}}|)be}\mylabel{h}  \normalsize

\doendnotes{C}
\bigskip
\vfill

\clearpage

\footnotesize

\lohead{\textsc{register}}

% Definiere theindex-Environment komplett neu ohne reledmac
\makeatletter
\renewenvironment{theindex}{%
  \section*{\indexname}%
  \setlength{\parindent}{0pt}%
  \setlength{\parskip}{0pt plus 0.3pt}%
  \let\item\@idxitem
}{%
  \clearpage
}
\makeatother

\IfFileExists{\jobname-pw.ind}{\input{\jobname-pw.ind}}{}

\end{document}

      