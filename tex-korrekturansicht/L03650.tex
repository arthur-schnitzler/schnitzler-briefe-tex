%% latex-korrekturansicht-vorspann.tex
%% Vorspann für die Korrekturansicht.
%% Lädt die gemeinsame Datei latex-vorspann.tex mit gesetztem Schalter.

\newif\ifkorrekturansicht
\korrekturansichttrue

\input{../tex-inputs/latex-vorspann}


\section[Stefan Zweig an Arthur Schnitzler, 16. 1. 1915]{L03650 Stefan Zweig an Arthur Schnitzler, 16. 1. 1915}
\nopagebreak\mylabel{L03650v}
\rehead{ }\normalsize\beginnumbering\briefempfaengerindex{, @\textsc{, }!zzz, @\emph{von  }!1915-01-161@{16. 1. 1915}|(be}
\toendnotes[C]{\smallbreak\pagebreak[2]}\Standort{CUL, Schnitzler, B 118.}
\physDesc{Brief, 1 Blatt, 1 Seite, 557 Zeichen
\newline{}Handschrift: schwarze Tinte, lateinische Kurrent
\newline{}Schnitzler: mit rotem Buntstift eine Unterstreichung }
\buchAbdrucke{\weitereDrucke{1) Stefan Zweig: \emph{Briefwechsel mit Hermann Bahr, Sigmund Freud, Rainer Maria
                        Rilke und Arthur Schnitzler}. Herausgegeben von Jeffrey B. Berlin,  Hans-Ulrich Lindken und  Donald A. Prater. Frankfurt am Main: \emph{S. Fischer} 1987, S. 390.} \weitereDrucke{2) Stefan Zweig: \emph{Briefe. Bd. II: 1914–1919}. Herausgegeben von Knut Beck,  Jeffrey B. Berlin und  Natascha Weschenbach-Feggeler. Frankfurt am Main: \emph{S. Fischer} 1998, S. 50.} }\toendnotes[C]{\smallbreak}
\pstart
           {\pb}\textcolor{pink}{Wien}\oindex{Wien@\textbf{Wien}, \emph{Verwaltungsgebiet}|pw}{}\ledrightnote{\textcolor{pink}{Wien}}{ }16. Januar 1915\pend
           \vspace{0.5em}
\pstart
           Lieber verehrter Herr Doktor, den \textcolor{green}{Ausschnitt}\pwindex{Schnitzler, Arthur 15. 5. 1862 Wien – 21. 10. 1931 ebd.@\textsc{Schnitzler, Arthur} (15. 5. 1862 Wien – 21. 10. 1931 ebd.), \emph{Schriftsteller, Mediziner}!Une protestation d’Arthur Schnitzler@\strich\emph{Une protestation d’Arthur Schnitzler}|pwv}\pwindex{Rolland, Romain 29.\,1.\,1866 Clamecy – 30.\,12.\,1944 Vézelay@\textsc{Rolland, Romain} (29.\,1.\,1866 Clamecy – 30.\,12.\,1944 Vézelay), \emph{Schriftsteller}!Une protestation d’Arthur Schnitzler@\strich\emph{Une protestation d’Arthur Schnitzler}|pwv}{}\ledrightnote{{$\rightarrow$}\emph{\textcolor{green}{Une protestation d’Arthur Schnitzler}}} aus dem »\textcolor{green}{Journal
                  de Genève}\pwindex{Journal de Genève@\emph{Journal de Genève}|pw}{}\ledrightnote{\textcolor{green}{Journal de Genève}}« sandte ich Ihnen schon vor paar Tagen \label{K_L03650-1v}\edtext{durch \textcolor{blue}{Stringa}\pwindex{Stringa, Alberto 12.\,1.\,1880 Caprino Veronese – 9.\,11.\,1931 ebd.@\textsc{Stringa, Alberto} (12.\,1.\,1880 Caprino Veronese – 9.\,11.\,1931 ebd.), \emph{Maler}|pw}{}\ledrightnote{\textcolor{blue}{Alberto Stringa}}}{\lemma{\textnormal{\emph{durch Stringa}}}\Cendnote{\textnormal{\textcolor{blue}{Alberto Stringa}\pwindex{Stringa, Alberto 12.\,1.\,1880 Caprino Veronese – 9.\,11.\,1931 ebd.@\textsc{Stringa, Alberto} (12.\,1.\,1880 Caprino Veronese – 9.\,11.\,1931 ebd.), \emph{Maler}|pwk}
                     überbrachte den \textcolor{green}{Ausschnitt}\pwindex{Schnitzler, Arthur 15. 5. 1862 Wien – 21. 10. 1931 ebd.@\textsc{Schnitzler, Arthur} (15. 5. 1862 Wien – 21. 10. 1931 ebd.), \emph{Schriftsteller, Mediziner}!Une protestation d’Arthur Schnitzler@\strich\emph{Une protestation d’Arthur Schnitzler}|pwkv}\pwindex{Rolland, Romain 29.\,1.\,1866 Clamecy – 30.\,12.\,1944 Vézelay@\textsc{Rolland, Romain} (29.\,1.\,1866 Clamecy – 30.\,12.\,1944 Vézelay), \emph{Schriftsteller}!Une protestation d’Arthur Schnitzler@\strich\emph{Une protestation d’Arthur Schnitzler}|pwkv} erst am 17. 1. 1915.}}}\label{K_L03650-1}. Von \textcolor{blue}{Romain
                  Rolland}\pwindex{Rolland, Romain 29.\,1.\,1866 Clamecy – 30.\,12.\,1944 Vézelay@\textsc{Rolland, Romain} (29.\,1.\,1866 Clamecy – 30.\,12.\,1944 Vézelay), \emph{Schriftsteller}|pw}{}\ledrightnote{\textcolor{blue}{Romain Rolland}} habe ich plötzlich \label{K_L03650-2v}\edtext{keine Briefe mehr}{\lemma{\textnormal{\emph{keine Briefe mehr}}}\Cendnote{\textnormal{In
                     der Briefedition \textcolor{blue}{Rolland}\pwindex{Rolland, Romain 29.\,1.\,1866 Clamecy – 30.\,12.\,1944 Vézelay@\textsc{Rolland, Romain} (29.\,1.\,1866 Clamecy – 30.\,12.\,1944 Vézelay), \emph{Schriftsteller}|pwk}–\textcolor{blue}{Zweig}\pwindex{Zweig, Stefan 28.\,11.\,1881 Wien – 23.\,2.\,1942 Petrópolis@\textsc{Zweig, Stefan} (28.\,11.\,1881 Wien – 23.\,2.\,1942 Petrópolis), \emph{Schriftsteller}|pwk} sind folgende Briefe von \textcolor{blue}{Rolland}\pwindex{Rolland, Romain 29.\,1.\,1866 Clamecy – 30.\,12.\,1944 Vézelay@\textsc{Rolland, Romain} (29.\,1.\,1866 Clamecy – 30.\,12.\,1944 Vézelay), \emph{Schriftsteller}|pwk}
                    aus dem Zeitraum abgedruckt: 22. 12. 1914, 11. 1. 1915, 5. 2. 1915.
                     Am 11. 1. 1915 schrieb er: »In den letzten vierzehn Tagen haben wir Ihnen drei Briefe geschrieben: sie kamen zu uns zurück.« (
                     \textcolor{blue}{Romain Rolland}\pwindex{Rolland, Romain 29.\,1.\,1866 Clamecy – 30.\,12.\,1944 Vézelay@\textsc{Rolland, Romain} (29.\,1.\,1866 Clamecy – 30.\,12.\,1944 Vézelay), \emph{Schriftsteller}|pwk}, \textcolor{blue}{Stefan Zweig}\pwindex{Zweig, Stefan 28.\,11.\,1881 Wien – 23.\,2.\,1942 Petrópolis@\textsc{Zweig, Stefan} (28.\,11.\,1881 Wien – 23.\,2.\,1942 Petrópolis), \emph{Schriftsteller}|pwk}: \emph{Von Welt zu Welt. Briefe
                           einer Freundschaft 1914–1918}. Mit einem Begleitwort von Peter
                        Handke. Aus dem Französischen von Eva und Gerhard Schwewe (Briefe Rollands) und
                        Christel Gersch (Briefe Zweigs). Berlin: \emph{Aufbau
                           Verlag}{ }2014.) \textcolor{blue}{Zweig}\pwindex{Zweig, Stefan 28.\,11.\,1881 Wien – 23.\,2.\,1942 Petrópolis@\textsc{Zweig, Stefan} (28.\,11.\,1881 Wien – 23.\,2.\,1942 Petrópolis), \emph{Schriftsteller}|pwk} sandte seinen nächsten Brief vom 17. 1. 1915, indem er ihn dem erwähnten
                     \textcolor{blue}{Stringa}\pwindex{Stringa, Alberto 12.\,1.\,1880 Caprino Veronese – 9.\,11.\,1931 ebd.@\textsc{Stringa, Alberto} (12.\,1.\,1880 Caprino Veronese – 9.\,11.\,1931 ebd.), \emph{Maler}|pwk} nach  \textcolor{pink}{Italien}\oindex{Italien@\textbf{Italien}|pwk} mitgab, um 
                     so die Zensur zu umgehen.}}}\label{K_L03650-2}, die Censur hat anscheinend
               unsere – doch zweifellos staatsgefährliche und an den Fundamenten \textcolor{pink}{Österreichs}\oindex{Österreich@\textbf{Österreich}|pw}{}\ledrightnote{\textcolor{pink}{Österreich}} rüttelnde — Correspondenz unterbunden und
               abgedrosselt. Ich schreibe ihm über \textcolor{pink}{Italien}\oindex{Italien@\textbf{Italien}|pw}{}\ledrightnote{\textcolor{pink}{Italien}} und
               wende mich übrigens heute noch an die Briefcensur direct, um ihr den Begriff \textcolor{blue}{Romain Rolland}\pwindex{Rolland, Romain 29.\,1.\,1866 Clamecy – 30.\,12.\,1944 Vézelay@\textsc{Rolland, Romain} (29.\,1.\,1866 Clamecy – 30.\,12.\,1944 Vézelay), \emph{Schriftsteller}|pw}{}\ledrightnote{\textcolor{blue}{Romain Rolland}} aufzuklären. Hoffentlich
               ge­lingts! Viele viele Grüsse Ihres getreuen\pend
           \pstart \spacefill\mbox{Stefan Zweig}\pend{}\selectlanguage{ngerman}\endnumbering\briefempfaengerindex{, @\textsc{, }!zzz, @\emph{von  }!1915-01-161@{16. 1. 1915}|)be}\mylabel{L03650h}  \normalsize

\doendnotes{C}
\bigskip
\vfill

\clearpage

\footnotesize

\lohead{\textsc{register}}

% Definiere theindex-Environment komplett neu ohne reledmac
\makeatletter
\renewenvironment{theindex}{%
  \section*{\indexname}%
  \setlength{\parindent}{0pt}%
  \setlength{\parskip}{0pt plus 0.3pt}%
  \let\item\@idxitem
}{%
  \clearpage
}
\makeatother

\IfFileExists{\jobname-pw.ind}{\input{\jobname-pw.ind}}{}

\end{document}

      