%% latex-korrekturansicht-vorspann.tex
%% Vorspann für die Korrekturansicht.
%% Lädt die gemeinsame Datei latex-vorspann.tex mit gesetztem Schalter.

\newif\ifkorrekturansicht
\korrekturansichttrue

\input{../tex-inputs/latex-vorspann}


\renewcommand{\erwaehntePersonen}{Personen: Romain Rolland, Alberto Stringa, Stefan Zweig}
\renewcommand{\erwaehnteOrte}{Orte: Italien, Wien, Österreich}
\renewcommand{\erwaehnteWerke}{Werke: Journal de Genève}
\section[Stefan Zweig an Arthur Schnitzler, 16. 1. 1915]{Stefan Zweig an Arthur Schnitzler, 16. 1. 1915}
\nopagebreak\mylabel{v}
\rehead{ }\normalsize\beginnumbering\briefempfaengerindex{Schnitzler, Arthur@\textsc{Schnitzler, Arthur}!zzzZweig, Stefan@\emph{von Stefan Zweig}!1915-01-161@{16. 1. 1915}|(be}
\toendnotes[C]{\smallbreak\pagebreak[2]}\Standort{CUL, Schnitzler, B 118.}
\physDesc{, 1 Blatt, 1 Seite, 559 Zeichen
\newline{}Handschrift: schwarze Tinte, lateinische Kurrent
\newline{}Schnitzler: mit rotem Buntstift eine Unterstreichung }
\buchAbdrucke{\weitereDrucke{1) Stefan Zweig: \emph{Briefwechsel mit Hermann Bahr, Sigmund Freud, Rainer Maria
                        Rilke und Arthur Schnitzler}. Hg. Jeffrey B. Berlin, Hans-Ulrich Lindken und Donald A. Prater. Frankfurt am Main: \emph{S. Fischer} 1987, S. 390.} \weitereDrucke{2) Stefan Zweig: \emph{Briefe. Bd. II: 1914–1919}. Hg. Knut Beck, Jeffrey B. Berlin und Natascha Weschenbach-Feggeler. Frankfurt am Main: \emph{S. Fischer} 1998, S. 50.} }\toendnotes[C]{\smallbreak}
\pstart
           {\pb}\textcolor{pink}{Wien}{}\ledrightnote{\textcolor{pink}{Wien}}{ }16. Januar
                  1915\pend
           
\pstart
           Lieber verehrter Herr Doktor, den Ausschnitt\textcolor{red}{\textsuperscript{\textbf{KEY}}} aus dem »\textcolor{green}{Journal de
                  Genève}{}\ledrightnote{\textcolor{green}{Journal de Genève}}« sandte ich Ihnen schon vor paar Tagen durch \textcolor{blue}{Stringa}{}\ledrightnote{\textcolor{blue}{Alberto Stringa}}. Von \textcolor{blue}{Romain
                  Rolland}{}\ledrightnote{\textcolor{blue}{Romain Rolland}} habe ich plötzlich keine Briefe mehr, die Censur hat anscheinend
               unsere – doch zweifellos staatsgefährliche und an den Fundamenten \textcolor{pink}{Österreichs}{}\ledrightnote{\textcolor{pink}{Österreich}} rüttelnde — Correspondenz unterbunden und
               abgedrosselt. Ich schreibe ihm über \textcolor{pink}{Italien}{}\ledrightnote{\textcolor{pink}{Italien}} und
               wende mich übrigens heute noch an die Briefcensur direct, um ihr den Begriff\textcolor{blue}{Romain Rolland}{}\ledrightnote{\textcolor{blue}{Romain Rolland}} aufzuklären. Hoffentlich
               ge­lingts! Viele viele Grüsse Ihres getreuen\pend
           \pstart \spacefill\mbox{Stefan Zweig}\pend{}\endnumbering\briefempfaengerindex{Schnitzler, Arthur@\textsc{Schnitzler, Arthur}!zzzZweig, Stefan@\emph{von Stefan Zweig}!1915-01-161@{16. 1. 1915}|)be}\mylabel{h}
\begin{anhang}
\end{anhang}\normalsize

\doendnotes{C}
\bigskip
\vfill

\clearpage

\footnotesize

\lohead{\textsc{register}}

% Definiere theindex-Environment komplett neu ohne reledmac
\makeatletter
\renewenvironment{theindex}{%
  \section*{\indexname}%
  \setlength{\parindent}{0pt}%
  \setlength{\parskip}{0pt plus 0.3pt}%
  \let\item\@idxitem
}{%
  \clearpage
}
\makeatother

\IfFileExists{\jobname-pw.ind}{\input{\jobname-pw.ind}}{}

\end{document}

      