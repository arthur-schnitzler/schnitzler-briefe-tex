%% latex-korrekturansicht-vorspann.tex
%% Vorspann für die Korrekturansicht.
%% Lädt die gemeinsame Datei latex-vorspann.tex mit gesetztem Schalter.

\newif\ifkorrekturansicht
\korrekturansichttrue

\input{../tex-inputs/latex-vorspann}


         
         \renewcommand{\erwaehntePersonen}{Personen: Michael Georg Conrad}
         \renewcommand{\erwaehnteOrte}{Orte: Berlin, Dessauer Straße, Wien}
         \renewcommand{\erwaehnteWerke}{Werke: Arthur Schnitzler [Reigen-Privatdruck], Aus der Theaterwelt. (Der gefährlichste Feind der Theatersaison. – Eine interessante Novität Arthur Schnitzler’s. – Dessous der »Familie Wawroch«. – Der Naturalismus in der Desinfektionsanstalt. – Der Claquechef des Deutschen Volkstheaters in …, Die Gesellschaft. Monatsschrift für Litteratur, Kunst und Sozialpolitik, Fremden-Blatt, Reigen. Zehn Dialoge}
               \section[ Paul Goldmann an Arthur Schnitzler, 27. 4. {[}1900{]}]{Paul Goldmann an Arthur Schnitzler, 27. 4. {[}1900{]}}\nopagebreak\mylabel{v}\rehead{ }\normalsize\beginnumbering\briefempfaengerindex{Schnitzler, Arthur@\textsc{Schnitzler, Arthur}!zzzGoldmann, Paul@\emph{von Paul Goldmann}!1900-04-271@{27. 4. {[}1900{]}}|(be} \toendnotes[C]{\smallbreak\pagebreak[2]} \Standort{DLA, A:Schnitzler, HS.NZ85.1.3170.}
\physDesc{Brief, 1 Blatt, 2 Seiten
\newline{}Handschrift: blaue Tinte, deutsche Kurrent
\newline{}Schnitzler: mit Bleistift das Jahr »{[}1{]}900« vermerkt }\toendnotes[C]{\smallbreak}\pstart
           \noindent{}{\pb}\textcolor{pink}{\textcolor{gray}{\textbf{DESSAUERSTRASSE 19}}}{}\ledrightnote{\textcolor{pink}{Dessauer Straße}}\hfill \textcolor{pink}{Berlin}{}\ledrightnote{\textcolor{pink}{Berlin}}, 27. April.\pend
           \pstart\center{}Mein lieber Freund,\pend\pstart
           Ich war ſehr erſtaunt, als ich ſah, daß die Sache mit dem »\textcolor{green}{Reigen}{}\ledrightnote{\textcolor{green}{Reigen. Zehn Dialoge}}« in die \label{K_L02913-1v}\edtext{Zeitungen}{\lemma{\textnormal{\emph{Zeitungen}}}\Cendnote{\textnormal{Am 22. 4. 1900
                     brachte das \emph{\textcolor{green}{Fremdenblatt}} folgende \textcolor{green}{Meldung}
                     in ihrer Kolumne über Ereignisse in Theaterkreisen: \textcolor{blue}{Schnitzler} »hat ein 
                        neues Buch geſchrieben, aber kein dramatiſches. Es nennt ſich ›\textcolor{green}{\so{Reigen}}‹ und ſchildet – wie ſagt
                     man nur, was? – die verſchiedenartigen Geſtalten, welche Liebe annimmt, wenn ſie in der ärmſten
                     Volksſchichte oder bei armen Leuten, beim Kleinbürger oder beim wohlhabenden
                     Bourgeois bis hinauf in den vornehmen Geſellſchaftskreiſen erſcheint. Damen, welche
                     das Buch kaufen wollen, würden aber vergeblich vor dem Buchhandlungsgehilfen
                     erröthen. Denn der Verfaſſer hat das Buch nur in zweihundert Exemplaren als
                     Manuſkript durcken laſſen, um dieſe an einen ausgewählten Kreis von Herren
                     zu verſenden. Die geringe Auflage des Buches geſtattete dem Verfaſſer, die Vorrede
                     in jedem Exemplare mit ſeiner eigenhändigen Unterſchrift zu verſehen – eine 
                     Aufmerkſamkeit, die das Buch jedem Beſitzer umſo intereſſanter erſcheinen
                     läßt.« Ähnlich lautende Meldungen wurden in Folge außerhalb \textcolor{pink}{Wien}s
                     abgedruckt,  beispielsweise: \textcolor{blue}{M. G. C.} [=\textcolor{blue}{Michael Georg Conrad}]: \emph{\textcolor{green}{Arthur Schnitzler}}. In: \emph{\textcolor{green}{Die Gesellschaft. Halbmonatschrift für Litteratur, Kunst und
                        Sozialpolitik}}, Jg. 16, Bd. 3, H. 4, 1900,
                     S. 251.}}}\label{K_L02913-1h} gekommen iſt, und die betreffenden Notizen in den \textcolor{pink}{Wien}{}\ledrightnote{\textcolor{pink}{Wien}}er Blättern ſind eine Albernheit oder eine
               Perfidie. Gefahr könnte erſt entſtehen, wenn Du von irgendwelchem Lumpenhunde beim
                  \strikeout{der} Staatsanwalt denuncirt würdeſt. Und da man
               immer mit ſolchen Lumpenhunden rechnen muß, und da Vorſicht niemals ſchaden kann, {\pb}möchte ich Dir rathen, einen verläßlichen Advokaten
               zu conſultiren, ob man Dir irgend Etwas anhaben kann. Ich glaube zwar nicht, aber es
               iſt immer gut, für alle Fälle \strikeout{be} vorbereitet zu ſein.
               Du aber mußt dafür ſorgen (und haſt jedenfalls ſchon dafür geſorgt), daß das \textcolor{green}{Buch}{}\ledrightnote{{$\rightarrow$}\textcolor{green}{Reigen. Zehn Dialoge}} nur in die Hände ſicherer
               Leute kommt. Vor allen \strikeout{Di} Dingen nicht in weibliche
               Hände! Was man einer Frau gibt, trägt man auf den offenen Markt. Ich weiß ein Lied
               davon zu ſingen. \pend
           \pstart
           Viele treue Grüße! {\\[\baselineskip]}Dein {\\[\baselineskip]}\spacefill\mbox{Paul Goldmann.}\pend
           \leftskip=0em{}\endnumbering\briefempfaengerindex{Schnitzler, Arthur@\textsc{Schnitzler, Arthur}!zzzGoldmann, Paul@\emph{von Paul Goldmann}!1900-04-271@{27. 4. {[}1900{]}}|)be}\mylabel{h}\begin{anhang}\end{anhang}\normalsize

\doendnotes{C}
\bigskip
\vfill

\clearpage

\footnotesize

\lohead{\textsc{register}}

% Definiere theindex-Environment komplett neu ohne reledmac
\makeatletter
\renewenvironment{theindex}{%
  \section*{\indexname}%
  \setlength{\parindent}{0pt}%
  \setlength{\parskip}{0pt plus 0.3pt}%
  \let\item\@idxitem
}{%
  \clearpage
}
\makeatother

\IfFileExists{\jobname-pw.ind}{\input{\jobname-pw.ind}}{}

\end{document}

      