%% latex-korrekturansicht-vorspann.tex
%% Vorspann für die Korrekturansicht.
%% Lädt die gemeinsame Datei latex-vorspann.tex mit gesetztem Schalter.

\newif\ifkorrekturansicht
\korrekturansichttrue

\input{../tex-inputs/latex-vorspann}


               \section[Arthur Schnitzler an Georg Brandes, 7. 8. 1896]{ Arthur Schnitzler an Georg Brandes, 7. 8. 1896}\nopagebreak\mylabel{v}\rehead{ }\normalsize\beginnumbering\briefempfaengerindex{Brandes, Georg@\textsc{Brandes, Georg}!zzzSchnitzler, Arthur@\emph{von Arthur Schnitzler}!1896-08-071@{7. 8. 1896}|(be} \toendnotes[C]{\smallbreak\pagebreak[2]} \Standort{Kopenhagen, Det Kongelige Bibliotek, Georg Brandes Arkiv, box 125.}
\physDesc{Brief, 1 Blatt, 2 Seiten
\newline{}Handschrift: schwarze Tinte, deutsche Kurrent\newline{}Ordnung: mit Bleistift von unbekannter Hand auf der ersten Seite datiert: »7. 7. 1896 (?)« und nummeriert: »4« }\buchAbdrucke{\weitereDrucke{Georg Brandes, Arthur Schnitzler: \emph{Ein Briefwechsel}. Hg. Kurt Bergel. Bern: \emph{Francke} 1956, S. 57.} }\pstart{}{\pb}Sehr geehrter Herr,\pend\pstart
           ſeit ein paar Tagen bin ich hier, in \textcolor{pink}{Skodsborg,
                        Badehotel}{}\ledrightnote{\textcolor{pink}{Badehotel}}, in Gesellſchaft von Dr \textcolor{blue}{\textsc{Richard Beer-Hofmann}}{}\ledrightnote{\textcolor{blue}{Richard Beer-Hofmann}}, und bleibe wohl noch bis gegen den 20. da. Ich wäre
                    höchſt erfreut, wenn mir im Laufe dieſer Zeit einmal Gelegenheit geboten würde,
                    Sie zu ſprechen, und, wie ich aus ihrem Brief an Dr. \textcolor{blue}{B. H.}{}\ledrightnote{\textcolor{blue}{Richard Beer-Hofmann}} entnehmen {\pb}möchte,
                    liegt das im Bereiche der Wahrſcheinlichkeit. Somit darf ich Sie heute in der
                    angenehmen Hoffnung verbindlichſt grüßen, Ihnen bald perſönlich die Hand drücken
                    zu können.\pend
           \pstart
           Ihr dankbar ergebener{\\[\baselineskip]}\spacefill\mbox{Arthur Schnitzler}\pend
           \leftskip=0em{}\pstart
           \textcolor{pink}{Skodsborg}{}\ledrightnote{\textcolor{pink}{Skodsborg}}{ }7/8. 96.\pend
           \endnumbering\briefempfaengerindex{Brandes, Georg@\textsc{Brandes, Georg}!zzzSchnitzler, Arthur@\emph{von Arthur Schnitzler}!1896-08-071@{7. 8. 1896}|)be}\mylabel{h}  \normalsize

\doendnotes{C}
\bigskip
\vfill

\clearpage

\footnotesize

\lohead{\textsc{register}}

% Definiere theindex-Environment komplett neu ohne reledmac
\makeatletter
\renewenvironment{theindex}{%
  \section*{\indexname}%
  \setlength{\parindent}{0pt}%
  \setlength{\parskip}{0pt plus 0.3pt}%
  \let\item\@idxitem
}{%
  \clearpage
}
\makeatother

\IfFileExists{\jobname-pw.ind}{\input{\jobname-pw.ind}}{}

\end{document}

      