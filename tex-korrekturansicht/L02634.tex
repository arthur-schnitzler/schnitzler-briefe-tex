%% latex-korrekturansicht-vorspann.tex
%% Vorspann für die Korrekturansicht.
%% Lädt die gemeinsame Datei latex-vorspann.tex mit gesetztem Schalter.

\newif\ifkorrekturansicht
\korrekturansichttrue

\input{../tex-inputs/latex-vorspann}


               \section[Paul Goldmann an Arthur Schnitzler, 26. 4. 1902]{ Paul Goldmann an Arthur Schnitzler, 26. 4. 1902}\nopagebreak\mylabel{v}\rehead{ }\normalsize\beginnumbering\briefempfaengerindex{Schnitzler, Arthur@\textsc{Schnitzler, Arthur}!zzzGoldmann, Paul@\emph{von Paul Goldmann}!1902-04-261@{26. 4. 1902}|(be} \toendnotes[C]{\smallbreak\pagebreak[2]} \Standort{DLA, A:Schnitzler, HS.NZ85.1.3172.}
\physDesc{Telegramm, 2 Blätter, 2 Seiten
\newline{}maschinell\newline{}Versand: 1) Stempel: »\nobreak{}26. April 1902, \textcolor{blue}{Kvasnicka}\nobreak{}«.  2) Stempel: »\nobreak{}12 40\nobreak{}«. \newline{}Zusatz: mit Bleistift von unbekannter Hand Vermerk:
                                 »71« }\toendnotes[C]{\smallbreak}\pstart{}{\pb}arthur schnitzler \textcolor{pink}{wien}{}\ledrightnote{\textcolor{pink}{Wien}}\pend{}\pstart{}\textcolor{pink}{frankgasze 1}{}\ledrightnote{\textcolor{pink}{Frankgasse}}=\pend{}{\bigskip}\pstart
           \noindent{}\centering{}{\pb}de \textcolor{pink}{berlin}{}\ledrightnote{\textcolor{pink}{Berlin}} 99946 196 26/4{ }10 20 m =\pend
           \pstart
           \noindent{}in ›\textcolor{brown}{taeglichen rundschau}{}\ledrightnote{\textcolor{brown}{Tägliche Rundschau}}‹ veroeffentlicht kritiker
                  \textcolor{blue}{karl strecker}{}\ledrightnote{\textcolor{blue}{Karl Strecker}} folgenden \textcolor{green}{artikel}{}\ledrightnote{→\textcolor{green}{Ein litterarisch-dramatisches Hochstapler-Stücklein}} mit fragenden ueberschrift »ein
               literarisch dramatisches hochstaplerstuecklein«? am donnerstag{ }mittag erhielt
               ich aus \textcolor{pink}{wien}{}\ledrightnote{\textcolor{pink}{Wien}} ein an meine persoenliche adresze
               gerichtetes telegramm, das also lautete: »frejtag{ }\textcolor{brown}{karl
                  wejsz-theater}{}\ledrightnote{\textcolor{brown}{Rose-Theater}} urpremi{[}ere{]} von ›\label{K_L02634-1v}\edtext{\textcolor{green}{kinder der armen}{}\ledrightnote{\textcolor{green}{Die Kinder der Armen}}}{\lemma{\textnormal{\emph{kinder der armen}}}\Cendnote{\textnormal{der Empfänger duplizierte bei der
                  Transkription: »kinder des kinder der
                  armen«}}}\label{K_L02634-1h}{[}‹{]} empfiehlt genejgter aufmerksamkejt
               ergebenst arthur schnitzler.{[}«{]} von diesem telegramm wuerde ich
               selbstverstaendlich niemals oeffentlich notiz genommen haben, wenn ich
                  annehm{[}en{]} koennte, dasz es wirklich von schnitzler aus {\pb}litterarischem interesze abgesandt worden sej haette.
               lejder liegt aber fuer mich nach betrachtung dieses ›\textcolor{green}{volksstueckes}{}\ledrightnote{→\textcolor{green}{Die Kinder der Armen}}‹ der handgrejfliche verdacht
               nahe, dasz hier ein arger miszbrauch mit dem namen eines feinfuehligen poeten
               getrieben worden ist. (ein kollege vom »\textcolor{brown}{berliner
                  tageblatt}{}\ledrightnote{\textcolor{brown}{Berliner Tageblatt}}« hat uebrigens genau daszelbe telegramm zur selbigen
                  stunde erhalten). unter diesen umstaenden sehe ich mich genoetigt,
               die offene frage an schnitzler zu richten, ob er diese seltsame aufmunterung wirklich
               abgefaszt hat? wenn nicht (und das nehme ich an), so liegt es ebenso in seinem
               interesze wie in dem der ehre unserer deutschen dramatisch{[}e{]}n
               litteratur, dasz dieser herr \textcolor{blue}{verfaszer}{}\ledrightnote{→\textcolor{blue}{Ernest von Gréger-Jurco}}, \textcolor{blue}{ernest von jurco}{}\ledrightnote{\textcolor{blue}{Ernest von Gréger-Jurco}} nennt sich
               die \textcolor{blue}{kapazitaet}{}\ledrightnote{→\textcolor{blue}{Ernest von Gréger-Jurco}}, entlarvt
                  wird{[}.{]} sowejt artikel. telegraphire dementi an \textcolor{blue}{strecker}{}\ledrightnote{\textcolor{blue}{Karl Strecker}} redaktion \textcolor{brown}{taeglichen rundschau}{}\ledrightnote{\textcolor{brown}{Tägliche Rundschau}}{ }\textcolor{pink}{berlin zimmerstrasze 7 und 8}{}\ledrightnote{\textcolor{pink}{Zimmerstraße}}. grusz \spacefill\mbox{=
                  goldmann. +}\pend
           \endnumbering\briefempfaengerindex{Schnitzler, Arthur@\textsc{Schnitzler, Arthur}!zzzGoldmann, Paul@\emph{von Paul Goldmann}!1902-04-261@{26. 4. 1902}|)be}\mylabel{h}  \normalsize

\doendnotes{C}
\bigskip
\vfill

\clearpage

\footnotesize

\lohead{\textsc{register}}

% Definiere theindex-Environment komplett neu ohne reledmac
\makeatletter
\renewenvironment{theindex}{%
  \section*{\indexname}%
  \setlength{\parindent}{0pt}%
  \setlength{\parskip}{0pt plus 0.3pt}%
  \let\item\@idxitem
}{%
  \clearpage
}
\makeatother

\IfFileExists{\jobname-pw.ind}{\input{\jobname-pw.ind}}{}

\end{document}

      