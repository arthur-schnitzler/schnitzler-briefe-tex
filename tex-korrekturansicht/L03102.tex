%% latex-korrekturansicht-vorspann.tex
%% Vorspann für die Korrekturansicht.
%% Lädt die gemeinsame Datei latex-vorspann.tex mit gesetztem Schalter.

\newif\ifkorrekturansicht
\korrekturansichttrue

\input{../tex-inputs/latex-vorspann}


\renewcommand{\erwaehntePersonen}{Personen: Bertha Karlsburg}
\renewcommand{\erwaehnteOrte}{Orte: Reisnerstraße, Wien}
\renewcommand{\erwaehnteWerke}{}
\section[Felix Salten an Arthur Schnitzler, {[}23. 5. 1891{]}]{Felix Salten an Arthur Schnitzler, {[}23. 5. 1891{]}}
\nopagebreak\mylabel{v}
\rehead{ }\normalsize\beginnumbering\briefempfaengerindex{Schnitzler, Arthur@\textsc{Schnitzler, Arthur}!zzzSalten, Felix@\emph{von Felix Salten}!1891-05-231@{{[}23. 5. 1891{]}}|(be}
\toendnotes[C]{\smallbreak\pagebreak[2]}\Standort{CUL, Schnitzler, B 89, A 1.}
\physDesc{Visitenkarte, 181 Zeichen
\newline{}Handschrift: Bleistift, lateinische Kurrent
\newline{}Schnitzler: mit Bleistift datiert »23/5 91.« 
\newline{}Ordnung: mit Bleistift von unbekannter Hand nummeriert: »3.« }\toendnotes[C]{\smallbreak}
\pstart
           \noindent{}\centering{}{\pb}\textcolor{gray}{\textbf{Felix Salten}}\pend
           
\pstart
           \noindent{}\textcolor{gray}{\textbf{\textcolor{pink}{Wien}{}\ledrightnote{\textcolor{pink}{Wien}}.}}\pend
           
\pstart
           {\pb}Lieber! Ich bin in einer \label{K_L03102-1v}\edtext{Lage}{\lemma{\textnormal{\emph{Lage}}}\Cendnote{\textnormal{Er war von
                  seiner Partnerin \textcolor{blue}{Bertha Karlsburg} betrogen
                  worden, vgl. A. S.: \emph{Tagebuch}, 23. 5. 1891.}}}\label{K_L03102-1h}, in der ich mit Jemanden, d. h. mit \uline{Einem}
               reden muss. \uuline{Bitte} kommen Sie, lieber Einer, sobald
               Sie diese Zeilen lesen, zu mir.\pend
           \pstart Ihr \spacefill\mbox{Salten.}\pend{}
\pstart
           \noindent{}\textcolor{pink}{III. Reisnerstraße 113}{}\ledrightnote{\textcolor{pink}{Reisnerstraße}}.\pend
           \endnumbering\briefempfaengerindex{Schnitzler, Arthur@\textsc{Schnitzler, Arthur}!zzzSalten, Felix@\emph{von Felix Salten}!1891-05-231@{{[}23. 5. 1891{]}}|)be}\mylabel{h}  \normalsize

\doendnotes{C}
\bigskip
\vfill

\clearpage

\footnotesize

\lohead{\textsc{register}}

% Definiere theindex-Environment komplett neu ohne reledmac
\makeatletter
\renewenvironment{theindex}{%
  \section*{\indexname}%
  \setlength{\parindent}{0pt}%
  \setlength{\parskip}{0pt plus 0.3pt}%
  \let\item\@idxitem
}{%
  \clearpage
}
\makeatother

\IfFileExists{\jobname-pw.ind}{\input{\jobname-pw.ind}}{}

\end{document}

      