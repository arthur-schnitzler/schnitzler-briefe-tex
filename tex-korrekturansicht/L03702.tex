%% latex-korrekturansicht-vorspann.tex
%% Vorspann für die Korrekturansicht.
%% Lädt die gemeinsame Datei latex-vorspann.tex mit gesetztem Schalter.

\newif\ifkorrekturansicht
\korrekturansichttrue

\input{../tex-inputs/latex-vorspann}


\renewcommand{\erwaehntePersonen}{Personen: Maria Janitschek, Albert Langen, Elsa Plessner, Louis Plessner}
\renewcommand{\erwaehnteOrte}{Orte: Bäckerstraße 1, Wien}
\renewcommand{\erwaehnteWerke}{Werke: Anatol, Baby, Der Begräbnißtag, Der gläserne Käfig. Skizzen und Novellen, Die Leiter der Seele, Im Feuer geprüft, Im Widerschein, Neues Wiener Journal, Simplicissimus, Warten}
\section[Elsa Plessner an Arthur Schnitzler, 15. 9. 1896]{Elsa Plessner an Arthur Schnitzler, 15. 9. 1896}
\nopagebreak\mylabel{v}
\rehead{ }\normalsize\beginnumbering\briefempfaengerindex{Schnitzler, Arthur@\textsc{Schnitzler, Arthur}!zzzPlessner, Elsa@\emph{von Elsa Plessner}!1896-09-151@{15. 9. 1896}|(be}
\toendnotes[C]{\smallbreak\pagebreak[2]}\Standort{DLA, A:Schnitzler, HS.1985.1.419.}
\physDesc{Brief,  Blätter, 3 Seiten, 1638 Zeichen
\newline{}Handschrift: , lateinische Kurrent
\newline{}Schnitzler: drei Unterstreichungen }\toendnotes[C]{\smallbreak}
\pstart
           {\pb}\textcolor{pink}{I. Bäckerstrasse N\textsuperscript{o}
                     1}{}\ledrightnote{\textcolor{pink}{Bäckerstraße 1}}, den 15. 9. 96. \pend
           
\pstart{}Verehrter Meister \label{K_L03702-1v}\edtext{Anatol}{\lemma{\textnormal{\emph{Anatol}}}\Cendnote{\textnormal{Bezugnahme auf \textcolor{blue}{Arthur Schnitzlers} Einakter-Zyklus \emph{\textcolor{green}{Anatol}} und den gleichnamigen
                  Protagonisten}}}\label{}!\pend\vspace{0.5em}
\pstart
           Hiemit übersende, Ihrem Wunsch gemäß, den \label{K_L03702-2v}\edtext{Brief}{\lemma{\textnormal{\emph{Brief}}}\Cendnote{\textnormal{nicht
                  überliefert}}}\label{}, den Sie so gütig waren, an mich zu richten sowie einen \label{K_L03702-3v}\edtext{andern}{\lemma{\textnormal{\emph{andern}}}\Cendnote{\textnormal{nicht überliefert}}}\label{} der mir heute Früh
               zukam.\pend
           
\pstart
           Diese liebenswürdigen Zeilen von Frau \textcolor{blue}{Janitschek}{}\ledrightnote{\textcolor{blue}{Maria Janitschek}} haben mich aufrichtig erfreut und dürften auch Sie einigermaßen
               interessieren! Nicht wahr? –\pend
           
\pstart
           Sodann bringt dies umfangreiche Paket meinen \label{K_L03702-4v}\edtext{zukünftigen \textcolor{green}{Band}{}\ledrightnote{{$\rightarrow$}\textcolor{green}{Der gläserne Käfig. Skizzen und Novellen}} Skizzen}{\lemma{\textnormal{\emph{zukünftigen Band Skizzen}}}\Cendnote{\textnormal{\textcolor{blue}{Elsa Plessners} Band \emph{\textcolor{green}{Der gläserne Käfig}} mit vierzehn Novellen und Skizzen
                  erschien 1901. Welche der Texte daraus sie in welcher Reihenfolge mit
                  diesem Brief schickte, läßt sich nur zum Teil rekonstruieren.}}}\label{}, von dem ich
               mir das Schlimmste, was Sie mir darüber sagen können selber schon gesagt habe. Allein
               wie bemerkt zwingen mich rein \label{K_L03702-5v}\edtext{äußerliche Gründe}{\lemma{\textnormal{\emph{äußerliche Gründe}}}\Cendnote{\textnormal{Am
                     19. 9. 1895 war ihr Vater \textcolor{blue}{Louis
                     Plessner} gestorben, woraus finanzielle Schwierigkeiten entstanden sein
                  dürften.}}}\label{} ein »\textcolor{green}{Buch}{}\ledrightnote{{$\rightarrow$}\textcolor{green}{Der gläserne Käfig. Skizzen und Novellen}}«
               vom Stapel {\pb}zu lassen – beklagen sollt Ihr mich, doch nimmer richten!! –
               Doch bitte ich Sie herzlich \label{K_L03702-6v}\edtext{\textcolor{green}{N\textsuperscript{o}
                  1}{}\ledrightnote{{$\rightarrow$}\textcolor{green}{Warten}}}{\lemma{\textnormal{\emph{N\textsuperscript{o}
                  1}}}\Cendnote{\textnormal{Dass es sich um den Text \emph{\textcolor{green}{Warten}} (zunächst unter dem Titel »Blätter« geplant)
                  handelt, ergibt sich aus \textcolor{blue}{Plessners} folgendem
                  Brief vom 21. 9. 1896.}}}\label{}, das \textcolor{green}{Fragment}{}\ledrightnote{{$\rightarrow$}\textcolor{green}{Warten}}
               oder quasi-\label{K_L03702-7v}\edtext{\begin{otherlanguage}{french}croquis\end{otherlanguage}}{\lemma{\textnormal{\emph{croquis}}}\Cendnote{\textnormal{französisch: Entwurf}}}\label{} nochmals zu
               lesen und dabei zu vergessen, dass ich je beabsichtigte, es \introOben{}weiter\introOben{} auszuführen. Vielleicht ändern Sie dann ein wenig Ihre Meinung
               umsomehr, als ich ja stark daran gefeilt und geändert habe! – \textcolor{green}{N\textsuperscript{o} 2}{}\ledrightnote{{$\rightarrow$}\textcolor{green}{Die Leiter der Seele}} ist
                  \label{K_L03702-8v}\edtext{aus dem \textcolor{green}{Simplicissimus}{}\ledrightnote{\textcolor{green}{Simplicissimus}}}{\lemma{\textnormal{\emph{aus dem Simplicissimus}}}\Cendnote{\textnormal{Der Text, der im Band \emph{\textcolor{green}{Der gläserne Käfig}} unter dem Titel \emph{\textcolor{green}{Der Selbstmörder}} publiziert wurde, erschien im ersten
                  Jahrgang des \emph{\textcolor{green}{Simplicissimus}} unter dem Titel
                     \emph{\textcolor{green}{Die Leiter der Seele}} (\textcolor{blue}{E. Pleßner}: \emph{\textcolor{green}{Die Leiter der Seele}}. In: \emph{\textcolor{green}{Simplicissimus}}, Jg. 1, Nr. 10, 6. 6. 1896,
                     S. 6).}}}\label{}, sowie 3, 6 u 7 von \textcolor{blue}{Langen}{}\ledrightnote{\textcolor{blue}{Albert Langen}} für \textcolor{green}{Simpl.}{}\ledrightnote{\textcolor{green}{Simplicissimus}} aus 10 Skizzen
                  \label{K_L03702-9v}\edtext{ausgewählt}{\lemma{\textnormal{\emph{ausgewählt}}}\Cendnote{\textnormal{Neben \emph{\textcolor{green}{Die Leiter der
                     Seele}} lassen sich keine weitere Texte \textcolor{blue}{Plessners} im \emph{\textcolor{green}{Simplicissimus}}
                  nachweisen.}}}\label{} wurden. (? – !) 3 und 6 ganz \label{K_L03702-10v}\edtext{\uuline{alte} Arbeiten }{\lemma{\textnormal{\emph{alte Arbeiten }}}\Cendnote{\textnormal{Vermutlich \emph{\textcolor{green}{Baby}} und
                     \emph{\textcolor{green}{Begräbnißtag}}, die \textcolor{blue}{Plessner} im Brief vom XXXX Auszeichnungsfehler: Dokument L03728 nicht gefunden als ihre frühesten Arbeiten
                  benennt.}}}\label{}{[}.{]} Als beste von Alle\substVorne{}\textsuperscript{n}\substDazwischen{}m\substHinten{}, wenn man so sagen darf, gilt mir N\textsuperscript{o} 8 – »\textcolor{green}{Im Widerschein}{}\ledrightnote{\textcolor{green}{Im Widerschein}}«. – Doch wir werden ja sehen!\pend
           
\pstart
           Seien Sie immer so grob, als Sie nur können, und glauben Sie mir, verehrter Herr
               Doctor, dass mich eine solide, ehrliche Grobheit von Ihnen mehr freut, als alle {\pb}Complimente sämmtlicher Esel\strikeout{-} von \textcolor{pink}{Wien}{}\ledrightnote{\textcolor{pink}{Wien}} zusammengenommen! Die \label{K_L03702-11v}\edtext{\textcolor{green}{Abdrücke}{}\ledrightnote{{$\rightarrow$}\textcolor{green}{Der Begräbnißtag}{\newline}{$\rightarrow$}\textcolor{green}{Im Feuer geprüft}{\newline}{$\rightarrow$}\textcolor{green}{Im Widerschein}}}{\lemma{\textnormal{\emph{Abdrücke}}}\Cendnote{\textnormal{\textcolor{blue}{E. Pleßner}: \emph{\textcolor{green}{Der Begräbnißtag}}. In: \emph{\textcolor{green}{Neues Wiener Journal}}, Nr. 951, 17. 6. 1896,
                  S. 1–2. \textcolor{blue}{E. Pleßner}: \emph{\textcolor{green}{Im Feuer geprüft}}. In: \emph{\textcolor{green}{Neues Wiener Journal}}, Nr. 1008, 14. 8. 1896,
                     S. 1–2. \textcolor{blue}{E. P.}: \emph{\textcolor{green}{Im Widerschein}}. In: \emph{\textcolor{green}{Neues Wiener
                        Journal}}, Nr. 1028, 4. 9. 1896, S. 1.}}}\label{} sind –
               verdammen Sie mich nicht – aus dem \textcolor{green}{N. W.-
                  Journal}{}\ledrightnote{\textcolor{green}{Neues Wiener Journal}}! – – –\pend
           
\pstart
           Und somit überliefre ich mich Ihrer Gnade – ich glaub an sie und hoff' auf sie, wobei
               ich schließlich noch \introOben{}soeben\introOben{} bemerke dass meine Handschrift
               ein wenig der Ihrigen ähnlich ist.\pend
           
\pstart
           Mit Verehrung und Dankbarkeit{\\[\baselineskip]}\spacefill\mbox{Elsa Plessner}\pend
           \leftskip=0em{}
\pstart
           \noindent{}\label{K_L03702-12v}\edtext{\uline{mit 3 Beilagen}}{\lemma{\textnormal{\emph{mit 3 Beilagen}}}\Cendnote{\textnormal{die drei »\textcolor{green}{Abdrücke}« aus dem \emph{\textcolor{green}{Neuen Wiener Journal}}}}}\label{}\pend
           \endnumbering\briefempfaengerindex{Schnitzler, Arthur@\textsc{Schnitzler, Arthur}!zzzPlessner, Elsa@\emph{von Elsa Plessner}!1896-09-151@{15. 9. 1896}|)be}\mylabel{h}
\begin{anhang}
\end{anhang}\normalsize

\doendnotes{C}
\bigskip
\vfill

\clearpage

\footnotesize

\lohead{\textsc{register}}

% Definiere theindex-Environment komplett neu ohne reledmac
\makeatletter
\renewenvironment{theindex}{%
  \section*{\indexname}%
  \setlength{\parindent}{0pt}%
  \setlength{\parskip}{0pt plus 0.3pt}%
  \let\item\@idxitem
}{%
  \clearpage
}
\makeatother

\IfFileExists{\jobname-pw.ind}{\input{\jobname-pw.ind}}{}

\end{document}

      