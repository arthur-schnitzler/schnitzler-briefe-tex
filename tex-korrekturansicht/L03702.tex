%% latex-korrekturansicht-vorspann.tex
%% Vorspann für die Korrekturansicht.
%% Lädt die gemeinsame Datei latex-vorspann.tex mit gesetztem Schalter.

\newif\ifkorrekturansicht
\korrekturansichttrue

\input{../tex-inputs/latex-vorspann}


\section[Elsa Plessner an Arthur Schnitzler, 15. 9. 1896]{L03702 Elsa Plessner an Arthur Schnitzler, 15. 9. 1896}
\nopagebreak\mylabel{L03702v}
\rehead{ }\normalsize\beginnumbering\briefempfaengerindex{Schnitzler, Arthur@\textsc{Schnitzler, Arthur}!zzzPlessner, Elsa@\emph{von Elsa Plessner}!1896-09-153@{15. 9. 1896}|(be}
\toendnotes[C]{\smallbreak\pagebreak[2]}
\correspDesc{Versand  durch Elsa Plessner am 15. 9. 1896 in Wien
\newline{}Erhalt  durch Arthur Schnitzler im Zeitraum [15. 9. 1896
                  – 17. 9. 1896?] in Wien}\toendnotes[C]{\smallbreak}
\Standort{DLA, A:Schnitzler, HS.1985.1.419.}
\physDesc{Brief, 1 Blatt, 3 Seiten, 1643 Zeichen
\newline{}Handschrift: schwarze Tinte, lateinische Kurrent
\newline{}Schnitzler: mit rotem Buntstift drei Unterstreichungen }\toendnotes[C]{\smallbreak}
\pstart
           {\pb}\textcolor{pink}{I. Bäckerstrasse N\textsuperscript{o}
                     1}\oindex{Wien@\textbf{Wien}!I., Innere Stadt@\textbf{I., Innere Stadt}!Bäckerstraße 1@\textbf{Bäckerstraße 1}, \emph{Wohngebäude}|pw}{}\ledrightnote{\textcolor{pink}{Bäckerstraße 1}}, den 15. 9. 96. \pend
           
\pstart{}Verehrter Meister \label{K_L03702-1v}\edtext{\textcolor{green}{Anatol}\pwindex{Schnitzler, Arthur 15. 5. 1862 Wien – 21. 10. 1931 ebd.@\textsc{Schnitzler, Arthur} (15. 5. 1862 Wien – 21. 10. 1931 ebd.), \emph{Schriftsteller, Mediziner}!Anatol@\strich\emph{Anatol}|pwv}{}\ledrightnote{{$\rightarrow$}\emph{\textcolor{green}{Anatol}}}}{\lemma{\textnormal{\emph{Anatol}}}\Cendnote{\textnormal{Die Identifikation \textcolor{blue}{Schnitzlers} mit dem Protagonisten seines gleichnamigen Einakter-Zyklus \emph{\textcolor{green}{Anatol}\pwindex{Schnitzler, Arthur 15. 5. 1862 Wien – 21. 10. 1931 ebd.@\textsc{Schnitzler, Arthur} (15. 5. 1862 Wien – 21. 10. 1931 ebd.), \emph{Schriftsteller, Mediziner}!Anatol@\strich\emph{Anatol}|pwk}} dürfte
               \textcolor{blue}{Schnitzler} zu diesem Zeitpunkt eher unangenehm gewesen sein.}}}\label{K_L03702-1}!\pend\vspace{0.5em}
\pstart
           Hiemit übersende, Ihrem Wunsch gemäß, den \label{K_L03702-2v}\edtext{Brief}{\lemma{\textnormal{\emph{Brief}}}\Cendnote{\textnormal{nicht
                  überliefert. Er dürfte die Einladung zu einem Treffen beinhaltet haben, das am A. S.: \emph{Tagebuch}, 15. 9. 1896 bei
            \textcolor{blue}{Schnitzler} stattfand.}}}\label{K_L03702-2}, den Sie so gütig waren, an mich zu richten sowie einen andern
               der mir heute Früh zukam. Diese liebenswürdigen \label{K_L03702-3v}\edtext{Zeilen von Frau
                  \textcolor{blue}{Janitschek}\pwindex{Janitschek, Maria 23.\,7.\,1859 Mödling – 28.\,4.\,1927@\textsc{Janitschek, Maria} (23.\,7.\,1859 Mödling – 28.\,4.\,1927), \emph{Schriftstellerin}|pw}{}\ledrightnote{\textcolor{blue}{Maria Janitschek}}}{\lemma{\textnormal{\emph{Zeilen … Janitschek}}}\Cendnote{\textnormal{nicht überliefert; \textcolor{blue}{Plessner}\pwindex{Plessner, Elsa 22.\,8.\,1875 Wien – 7.\,5.\,1932 Alicante@\textsc{Plessner, Elsa} (22.\,8.\,1875 Wien – 7.\,5.\,1932 Alicante), \emph{Schriftstellerin}|pwk} hatte \textcolor{blue}{Maria
                     Janitscheks}\pwindex{Janitschek, Maria 23.\,7.\,1859 Mödling – 28.\,4.\,1927@\textsc{Janitschek, Maria} (23.\,7.\,1859 Mödling – 28.\,4.\,1927), \emph{Schriftstellerin}|pwk} Buch \emph{\textcolor{green}{Vom Weibe}\pwindex{Janitschek, Maria 23.\,7.\,1859 Mödling – 28.\,4.\,1927@\textsc{Janitschek, Maria} (23.\,7.\,1859 Mödling – 28.\,4.\,1927), \emph{Schriftstellerin}!Vom Weibe. Charakterzeichnungen@\strich\emph{Vom Weibe. Charakterzeichnungen}|pwk}} eine
                  ausführliche Rezension gewidmet: \emph{\textcolor{green}{Vom Weibe}\pwindex{Plessner, Elsa 22.\,8.\,1875 Wien – 7.\,5.\,1932 Alicante@\textsc{Plessner, Elsa} (22.\,8.\,1875 Wien – 7.\,5.\,1932 Alicante), \emph{Schriftstellerin}!Vom Weibe@\strich\emph{Vom Weibe}|pwk}}. In: \emph{\textcolor{green}{Morgen-Presse}\pwindex{Presse@\emph{Die Presse}|pwk}}, Jg. 49, Nr. 167,
                        18. 6. 1896, S. [1]–2.
               }}}\label{K_L03702-3} haben mich aufrichtig erfreut und dürften auch Sie einigermaßen
               interessieren! Nicht wahr? –\pend
           
\pstart
           Sodann bringt dies umfangreiche Paket meinen \label{K_L03702-4v}\edtext{zukünftigen \textcolor{green}{Band}\pwindex{Plessner, Elsa 22.\,8.\,1875 Wien – 7.\,5.\,1932 Alicante@\textsc{Plessner, Elsa} (22.\,8.\,1875 Wien – 7.\,5.\,1932 Alicante), \emph{Schriftstellerin}!gläserne Käfig. Skizzen und Novellen@\strich\emph{Der gläserne Käfig. Skizzen und Novellen}|pwv}{}\ledrightnote{{$\rightarrow$}\emph{\textcolor{green}{Der gläserne Käfig. Skizzen und Novellen}}} Skizzen}{\lemma{\textnormal{\emph{zukünftigen Band Skizzen}}}\Cendnote{\textnormal{\textcolor{blue}{Elsa Plessners}\pwindex{Plessner, Elsa 22.\,8.\,1875 Wien – 7.\,5.\,1932 Alicante@\textsc{Plessner, Elsa} (22.\,8.\,1875 Wien – 7.\,5.\,1932 Alicante), \emph{Schriftstellerin}|pwk} Band \emph{\textcolor{green}{Der gläserne Käfig}\pwindex{Plessner, Elsa 22.\,8.\,1875 Wien – 7.\,5.\,1932 Alicante@\textsc{Plessner, Elsa} (22.\,8.\,1875 Wien – 7.\,5.\,1932 Alicante), \emph{Schriftstellerin}!gläserne Käfig. Skizzen und Novellen@\strich\emph{Der gläserne Käfig. Skizzen und Novellen}|pwk}} mit vierzehn Novellen und Skizzen
                  erschien 1901. Welche der Texte daraus sie in welcher Reihenfolge mit
                  diesem Brief schickte, läßt sich nur zum Teil rekonstruieren.}}}\label{K_L03702-4}, von dem ich
               mir das Schlimmste, was Sie mir darüber sagen können, selber schon gesagt habe. Allein,
               wie bemerkt{[},{]} zwingen mich rein \label{K_L03702-5v}\edtext{äußerliche Gründe}{\lemma{\textnormal{\emph{äußerliche Gründe}}}\Cendnote{\textnormal{Am
                     19. 9. 1895 war ihr Vater \textcolor{blue}{Louis
                     Plessner}\pwindex{Plessner, Louis 3.\,12.\,1847 Bielsko-Biała – 19.\,9.\,1895 Wien@\textsc{Plessner, Louis} (3.\,12.\,1847 Bielsko-Biała – 19.\,9.\,1895 Wien), \emph{Journalist, Kaufmann}|pwk} gestorben, woraus finanzielle Schwierigkeiten entstanden sein
                  dürften.}}}\label{K_L03702-5}, ein »\textcolor{green}{Buch}\pwindex{Plessner, Elsa 22.\,8.\,1875 Wien – 7.\,5.\,1932 Alicante@\textsc{Plessner, Elsa} (22.\,8.\,1875 Wien – 7.\,5.\,1932 Alicante), \emph{Schriftstellerin}!gläserne Käfig. Skizzen und Novellen@\strich\emph{Der gläserne Käfig. Skizzen und Novellen}|pwv}{}\ledrightnote{{$\rightarrow$}\emph{\textcolor{green}{Der gläserne Käfig. Skizzen und Novellen}}}«
               vom Stapel {\pb}zu lassen – beklagen sollt Ihr mich, doch nimmer richten!! –
               Doch bitte ich Sie herzlich \label{K_L03702-6v}\edtext{\textcolor{green}{N\textsuperscript{o}
                  1}\pwindex{Plessner, Elsa 22.\,8.\,1875 Wien – 7.\,5.\,1932 Alicante@\textsc{Plessner, Elsa} (22.\,8.\,1875 Wien – 7.\,5.\,1932 Alicante), \emph{Schriftstellerin}!Warten. Novelle@\strich\emph{Warten. Novelle}|pwv}{}\ledrightnote{{$\rightarrow$}\emph{\textcolor{green}{Warten. Novelle}}}}{\lemma{\textnormal{\emph{N\textsuperscript{o}
                  1}}}\Cendnote{\textnormal{Dass es sich um den Text \emph{\textcolor{green}{Warten}\pwindex{Plessner, Elsa 22.\,8.\,1875 Wien – 7.\,5.\,1932 Alicante@\textsc{Plessner, Elsa} (22.\,8.\,1875 Wien – 7.\,5.\,1932 Alicante), \emph{Schriftstellerin}!Warten. Novelle@\strich\emph{Warten. Novelle}|pwk}} (zunächst unter dem Titel »Blätter« geplant)
                  handelt, ergibt sich aus \textcolor{blue}{Plessners}\pwindex{Plessner, Elsa 22.\,8.\,1875 Wien – 7.\,5.\,1932 Alicante@\textsc{Plessner, Elsa} (22.\,8.\,1875 Wien – 7.\,5.\,1932 Alicante), \emph{Schriftstellerin}|pwk} folgendem
                  Brief vom 21. 9. 1896.}}}\label{K_L03702-6}, das \textcolor{green}{Fragment}\pwindex{Plessner, Elsa 22.\,8.\,1875 Wien – 7.\,5.\,1932 Alicante@\textsc{Plessner, Elsa} (22.\,8.\,1875 Wien – 7.\,5.\,1932 Alicante), \emph{Schriftstellerin}!Warten. Novelle@\strich\emph{Warten. Novelle}|pwv}{}\ledrightnote{{$\rightarrow$}\emph{\textcolor{green}{Warten. Novelle}}}
               oder quasi-\label{K_L03702-7v}\edtext{\begin{otherlanguage}{french}croquis\end{otherlanguage}}{\lemma{\textnormal{\emph{croquis}}}\Cendnote{\textnormal{französisch: Entwurf}}}\label{K_L03702-7} nochmals zu
               lesen und dabei zu vergessen, dass ich je beabsichtigte, es \introOben{}weiter\introOben{} auszuführen. Vielleicht ändern Sie dann ein wenig Ihre Meinung
               umsomehr, als ich ja stark daran gefeilt und geändert habe! – \textcolor{green}{N\textsuperscript{o} 2}\pwindex{Plessner, Elsa 22.\,8.\,1875 Wien – 7.\,5.\,1932 Alicante@\textsc{Plessner, Elsa} (22.\,8.\,1875 Wien – 7.\,5.\,1932 Alicante), \emph{Schriftstellerin}!Leiter der Seele@\strich\emph{Die Leiter der Seele}|pwv}{}\ledrightnote{{$\rightarrow$}\emph{\textcolor{green}{Die Leiter der Seele}}} ist
                  \label{K_L03702-8v}\edtext{aus dem \textcolor{green}{Simplicissimus}\pwindex{Simplicissimus@\emph{Simplicissimus}|pw}{}\ledrightnote{\textcolor{green}{Simplicissimus}}}{\lemma{\textnormal{\emph{aus dem Simplicissimus}}}\Cendnote{\textnormal{Der Text, der im Band \emph{\textcolor{green}{Der gläserne Käfig}\pwindex{Plessner, Elsa 22.\,8.\,1875 Wien – 7.\,5.\,1932 Alicante@\textsc{Plessner, Elsa} (22.\,8.\,1875 Wien – 7.\,5.\,1932 Alicante), \emph{Schriftstellerin}!gläserne Käfig. Skizzen und Novellen@\strich\emph{Der gläserne Käfig. Skizzen und Novellen}|pwk}} unter dem Titel \emph{\textcolor{green}{Der Selbstmörder}\pwindex{Plessner, Elsa 22.\,8.\,1875 Wien – 7.\,5.\,1932 Alicante@\textsc{Plessner, Elsa} (22.\,8.\,1875 Wien – 7.\,5.\,1932 Alicante), \emph{Schriftstellerin}!Leiter der Seele@\strich\emph{Die Leiter der Seele}|pwk}} publiziert wurde, erschien im ersten
                  Jahrgang des \emph{\textcolor{green}{Simplicissimus}\pwindex{Simplicissimus@\emph{Simplicissimus}|pwk}} unter dem Titel
                     \emph{\textcolor{green}{Die Leiter der Seele}\pwindex{Plessner, Elsa 22.\,8.\,1875 Wien – 7.\,5.\,1932 Alicante@\textsc{Plessner, Elsa} (22.\,8.\,1875 Wien – 7.\,5.\,1932 Alicante), \emph{Schriftstellerin}!Leiter der Seele@\strich\emph{Die Leiter der Seele}|pwk}} (\textcolor{blue}{E. Pleßner}\pwindex{Plessner, Elsa 22.\,8.\,1875 Wien – 7.\,5.\,1932 Alicante@\textsc{Plessner, Elsa} (22.\,8.\,1875 Wien – 7.\,5.\,1932 Alicante), \emph{Schriftstellerin}|pwk}: \emph{\textcolor{green}{Die Leiter der Seele}\pwindex{Plessner, Elsa 22.\,8.\,1875 Wien – 7.\,5.\,1932 Alicante@\textsc{Plessner, Elsa} (22.\,8.\,1875 Wien – 7.\,5.\,1932 Alicante), \emph{Schriftstellerin}!Leiter der Seele@\strich\emph{Die Leiter der Seele}|pwk}}. In: \emph{\textcolor{green}{Simplicissimus}\pwindex{Simplicissimus@\emph{Simplicissimus}|pwk}}, Jg. 1, Nr. 10, 6. 6. 1896,
                        S. 6).}}}\label{K_L03702-8}, sowie \label{K_L03702-9v}\edtext{3, 6 u 7}{\lemma{\textnormal{\emph{3, 6 u 7}}}\Cendnote{\textnormal{Die Textanordnung
                        stimmt noch nicht mit der veröffentlichten Reihung überein, so dass diese weitgehend nicht rekonstruiert werden kann.}}}\label{K_L03702-9} von \textcolor{blue}{Langen}\pwindex{Langen, Albert 8.\,7.\,1869 Antwerpen – 30.\,4.\,1909 München@\textsc{Langen, Albert} (8.\,7.\,1869 Antwerpen – 30.\,4.\,1909 München), \emph{Verleger}|pw}{}\ledrightnote{\textcolor{blue}{Albert Langen}} für \label{K_L03702-10v}\edtext{\textcolor{green}{Simpl.}\pwindex{Simplicissimus@\emph{Simplicissimus}|pw}{}\ledrightnote{\textcolor{green}{Simplicissimus}} aus 10 Skizzen
                  ausgewählt}{\lemma{\textnormal{\emph{Simpl. … ausgewählt}}}\Cendnote{\textnormal{Nach \emph{\textcolor{green}{Die Leiter der
                     Seele}\pwindex{Plessner, Elsa 22.\,8.\,1875 Wien – 7.\,5.\,1932 Alicante@\textsc{Plessner, Elsa} (22.\,8.\,1875 Wien – 7.\,5.\,1932 Alicante), \emph{Schriftstellerin}!Leiter der Seele@\strich\emph{Die Leiter der Seele}|pwk}} erschienen keine weiteren Texte \textcolor{blue}{Plessners}\pwindex{Plessner, Elsa 22.\,8.\,1875 Wien – 7.\,5.\,1932 Alicante@\textsc{Plessner, Elsa} (22.\,8.\,1875 Wien – 7.\,5.\,1932 Alicante), \emph{Schriftstellerin}|pwk} im \emph{\textcolor{green}{Simplicissimus}\pwindex{Simplicissimus@\emph{Simplicissimus}|pwk}}.}}}\label{K_L03702-10} wurden. (? – !) 3 und 6 ganz \label{K_L03702-11v}\edtext{\uuline{alte} Arbeiten}{\lemma{\textnormal{\emph{alte Arbeiten}}}\Cendnote{\textnormal{Möglicherweise \emph{\textcolor{green}{Baby}\pwindex{Plessner, Elsa 22.\,8.\,1875 Wien – 7.\,5.\,1932 Alicante@\textsc{Plessner, Elsa} (22.\,8.\,1875 Wien – 7.\,5.\,1932 Alicante), \emph{Schriftstellerin}!Baby@\strich\emph{Baby}|pwk}} und
                     \emph{\textcolor{green}{Begräbnißtag}\pwindex{Begräbnißtag@\emph{Der Begräbnißtag}|pwk}}, die \textcolor{blue}{Plessner}\pwindex{Plessner, Elsa 22.\,8.\,1875 Wien – 7.\,5.\,1932 Alicante@\textsc{Plessner, Elsa} (22.\,8.\,1875 Wien – 7.\,5.\,1932 Alicante), \emph{Schriftstellerin}|pwk} im Brief vom 12. 10. 1900 als ihre frühesten Arbeiten
                  benennt.}}}\label{K_L03702-11}{[}.{]} Als beste von Alle\substVorne{}\textsuperscript{n}\substDazwischen{}m\substHinten{}, wenn man so sagen darf, gilt mir N\textsuperscript{o} 8 – »\textcolor{green}{Im Widerschein}\pwindex{Im Widerschein@\emph{Im Widerschein}|pw}{}\ledrightnote{\textcolor{green}{Im Widerschein}}«. – Doch wir werden ja sehen!\pend
           
\pstart
           Seien Sie immer so grob, als Sie nur können, und glauben Sie mir, verehrter Herr
               Doctor, dass mich eine solide, ehrliche Grobheit von Ihnen mehr freut, als alle {\pb}Complimente sämmtlicher Esel\strikeout{-} von \textcolor{pink}{Wien}\oindex{Wien@\textbf{Wien}, \emph{Verwaltungsgebiet}|pw}{}\ledrightnote{\textcolor{pink}{Wien}} zusammengenommen! Die \label{K_L03702-12v}\edtext{\textcolor{green}{Abdrücke}\pwindex{Begräbnißtag@\emph{Der Begräbnißtag}|pwv}\pwindex{Im Feuer geprüft@\emph{Im Feuer geprüft}|pwv}\pwindex{Im Widerschein@\emph{Im Widerschein}|pwv}{}\ledrightnote{{$\rightarrow$}\emph{\textcolor{green}{Der Begräbnißtag}}{\newline}{$\rightarrow$}\emph{\textcolor{green}{Im Feuer geprüft}}{\newline}{$\rightarrow$}\emph{\textcolor{green}{Im Widerschein}}}}{\lemma{\textnormal{\emph{Abdrücke}}}\Cendnote{\textnormal{\textcolor{blue}{E. Pleßner}\pwindex{Plessner, Elsa 22.\,8.\,1875 Wien – 7.\,5.\,1932 Alicante@\textsc{Plessner, Elsa} (22.\,8.\,1875 Wien – 7.\,5.\,1932 Alicante), \emph{Schriftstellerin}|pwk}: \emph{\textcolor{green}{Der Begräbnißtag}\pwindex{Begräbnißtag@\emph{Der Begräbnißtag}|pwk}}. In: \emph{\textcolor{green}{Neues Wiener Journal}\pwindex{Neues Wiener Journal@\emph{Neues Wiener Journal}|pwk}}, Nr. 951, 17. 6. 1896,
                  S. 1–2. \textcolor{blue}{E. Pleßner}\pwindex{Plessner, Elsa 22.\,8.\,1875 Wien – 7.\,5.\,1932 Alicante@\textsc{Plessner, Elsa} (22.\,8.\,1875 Wien – 7.\,5.\,1932 Alicante), \emph{Schriftstellerin}|pwk}: \emph{\textcolor{green}{Im Feuer geprüft}\pwindex{Im Feuer geprüft@\emph{Im Feuer geprüft}|pwk}}. In: \emph{\textcolor{green}{Neues Wiener Journal}\pwindex{Neues Wiener Journal@\emph{Neues Wiener Journal}|pwk}}, Nr. 1008, 14. 8. 1896,
                     S. 1–2. \textcolor{blue}{E. P.}\pwindex{Plessner, Elsa 22.\,8.\,1875 Wien – 7.\,5.\,1932 Alicante@\textsc{Plessner, Elsa} (22.\,8.\,1875 Wien – 7.\,5.\,1932 Alicante), \emph{Schriftstellerin}|pwk}: \emph{\textcolor{green}{Im Widerschein}\pwindex{Im Widerschein@\emph{Im Widerschein}|pwk}}. In: \emph{\textcolor{green}{Neues Wiener
                        Journal}\pwindex{Neues Wiener Journal@\emph{Neues Wiener Journal}|pwk}}, Nr. 1028, 4. 9. 1896, S. 1.}}}\label{K_L03702-12} sind –
               verdammen Sie mich nicht – aus dem \textcolor{green}{N. W.-
                  Journal}\pwindex{Neues Wiener Journal@\emph{Neues Wiener Journal}|pw}{}\ledrightnote{\textcolor{green}{Neues Wiener Journal}}! – – –\pend
           
\pstart
           Und somit überliefre ich mich Ihrer Gnade – ich glaub an sie und hoff’ auf sie, wobei
               ich schließlich noch \introOben{}soeben\introOben{} bemerke dass meine Handschrift
               ein wenig der Ihrigen ähnlich ist.\pend
           
\pstart
           Mit Verehrung und Dankbarkeit{\\[\baselineskip]}\spacefill\mbox{Elsa Plessner}\pend
           \leftskip=0em{}
\pstart
           \noindent{}\label{K_L03702-13v}\edtext{\uline{mit 3 Beilagen}}{\lemma{\textnormal{\emph{mit 3 Beilagen}}}\Cendnote{\textnormal{Die Beilagen sind nicht überliefert.
                     Wie aus dem vorliegenden Brief hervorgeht, handelte es sich um ein
                     Korrespondenzstück \textcolor{blue}{Schnitzlers}, ein Brief
                     der Schriftstellerin \textcolor{blue}{Maria Janitschek}\pwindex{Janitschek, Maria 23.\,7.\,1859 Mödling – 28.\,4.\,1927@\textsc{Janitschek, Maria} (23.\,7.\,1859 Mödling – 28.\,4.\,1927), \emph{Schriftstellerin}|pwk},
                     ein Konvolut mit Novellen und Skizzen, die später Eingang in den Band \emph{\textcolor{green}{Der gläserne Käfig}\pwindex{Plessner, Elsa 22.\,8.\,1875 Wien – 7.\,5.\,1932 Alicante@\textsc{Plessner, Elsa} (22.\,8.\,1875 Wien – 7.\,5.\,1932 Alicante), \emph{Schriftstellerin}!gläserne Käfig. Skizzen und Novellen@\strich\emph{Der gläserne Käfig. Skizzen und Novellen}|pwk}} fanden, darunter
                     Abdrucke von Texten \textcolor{blue}{Plessners}\pwindex{Plessner, Elsa 22.\,8.\,1875 Wien – 7.\,5.\,1932 Alicante@\textsc{Plessner, Elsa} (22.\,8.\,1875 Wien – 7.\,5.\,1932 Alicante), \emph{Schriftstellerin}|pwk} aus dem \emph{\textcolor{green}{Neuen Wiener Journal}\pwindex{Neues Wiener Journal@\emph{Neues Wiener Journal}|pwk}}.}}}\label{K_L03702-13}\pend
           \selectlanguage{ngerman}\endnumbering\briefempfaengerindex{Schnitzler, Arthur@\textsc{Schnitzler, Arthur}!zzzPlessner, Elsa@\emph{von Elsa Plessner}!1896-09-153@{15. 9. 1896}|)be}\mylabel{L03702h}  \normalsize

\doendnotes{C}
\bigskip
\vfill

\clearpage

\footnotesize

\lohead{\textsc{register}}

% Definiere theindex-Environment komplett neu ohne reledmac
\makeatletter
\renewenvironment{theindex}{%
  \section*{\indexname}%
  \setlength{\parindent}{0pt}%
  \setlength{\parskip}{0pt plus 0.3pt}%
  \let\item\@idxitem
}{%
  \clearpage
}
\makeatother

\IfFileExists{\jobname-pw.ind}{\input{\jobname-pw.ind}}{}

\end{document}

      