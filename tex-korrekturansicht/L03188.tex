%% latex-korrekturansicht-vorspann.tex
%% Vorspann für die Korrekturansicht.
%% Lädt die gemeinsame Datei latex-vorspann.tex mit gesetztem Schalter.

\newif\ifkorrekturansicht
\korrekturansichttrue

\input{../tex-inputs/latex-vorspann}


\renewcommand{\erwaehntePersonen}{Personen: Richard Beer-Hofmann, Paul Goldmann, Alfred Kerr, Leo Van-Jung}
\renewcommand{\erwaehnteOrte}{Orte: Admont, Altaussee, Berlin, Dessauer Straße, Grundlsee (Gemeinde), Innsbruck, Salzburg}
\renewcommand{\erwaehnteWerke}{}
\section[ Arthur Schnitzler an Paul Goldmann, 1. 7. 1900]{Arthur Schnitzler an Paul Goldmann, 1. 7. 1900}
\nopagebreak\mylabel{v}
\rehead{ }\normalsize\beginnumbering\briefempfaengerindex{Goldmann, Paul@\textsc{Goldmann, Paul}!zzzSchnitzler, Arthur@\emph{von Arthur Schnitzler}!1900-07-011@{1. 7. 1900}|(be}
\toendnotes[C]{\smallbreak\pagebreak[2]}\Standort{Berlin, Akademie der Künste, Alfred Kerr-Archiv, 2487.}
\physDesc{Bildpostkarte, 327 Zeichen
\newline{}Handschrift: 1) Bleistift, deutsche Kurrent\hspace{1em}2) Bleistift, lateinische Kurrent (\noindent{}Adresse)\hspace{1em}
\newline{}Versand: 1) Stempel: »\nobreak{}\oindex{Grundlsee (Gemeinde)@\textbf{Grundlsee (Gemeinde)}, \emph{Besiedelter Ort (A.BSO)}|pwk}G\textcolor{gray}{r}{[}undlse{]}e, 1. 7. 00\nobreak{}«.   2) Stempel: »\nobreak{}8/7 00, 7¼–8½ V., Bestellt vom Postamt\nobreak{}«. }\toendnotes[C]{\smallbreak}\pstart{}{\pb}Dr. Paul Goldmann\pend{}\pstart{}\textcolor{pink}{Berlin W\textcolor{gray}{.}}{}\ledrightnote{\textcolor{pink}{Berlin}}\pend{}\pstart{}\textcolor{pink}{Dessauerstraße 19}{}\ledrightnote{\textcolor{pink}{Dessauer Straße}}.\pend{}
{\bigskip}
\pstart
           \noindent{}{\pb}\textcolor{gray}{\textbf{\textbf{Gruss} aus \textcolor{pink}{Grundlsee}{}\ledrightnote{\textcolor{pink}{Grundlsee (Gemeinde)}}.}}\hfill \textcolor{gray}{\textbf{Totalansicht.}}\pend
           
\pstart
           Ich bin hier per Rad, \label{K_L03188-1v}\edtext{aus \textcolor{pink}{\textsc{Altaussee}}{}\ledrightnote{\textcolor{pink}{Altaussee}}}{\lemma{\textnormal{\emph{aus Altaussee}}}\Cendnote{\textnormal{\textcolor{blue}{Schnitzler} war am 29. 6. 1900 in \textcolor{pink}{Altaussee} angekommen. Am 3. 7. 1900 fuhr er
                  weiter nach \textcolor{pink}{Admont}.}}}\label{K_L03188-1h}, (\textcolor{blue}{Richard}{}\ledrightnote{\textcolor{blue}{Richard Beer-Hofmann}} nicht.) – Die berühmte \label{K_L03188-2v}\edtext{Fußpartie}{\lemma{\textnormal{\emph{Fußpartie}}}\Cendnote{\textnormal{siehe Paul Goldmann an Arthur Schnitzler, 16. 6. [1900]}}}\label{K_L03188-2h} wäre am erwünſchtesten we{\geminationn} ſie etwa Mitte Auguſt begänne. Rendezvous \textcolor{pink}{Salzburg}{}\ledrightnote{\textcolor{pink}{Salzburg}} oder \textcolor{pink}{Innsbruck}{}\ledrightnote{\textcolor{pink}{Innsbruck}}.–
               Nach \textcolor{pink}{Salzburg}{}\ledrightnote{\textcolor{pink}{Salzburg}} auf ein paar Tage könnte \textcolor{blue}{Richard}{}\ledrightnote{\textcolor{blue}{Richard Beer-Hofmann}} (u ich) (und \textcolor{blue}{Leo}{}\ledrightnote{\textcolor{blue}{Leo Van-Jung}}) auch Anfang Auguſt.\pend
           \pstart Herzlichſt \spacefill\mbox{Arthur}\pend{}
\pstart
           \noindent{}Bitte das auch \label{K_L03188-3v}\edtext{\textsc{\textcolor{blue}{Kerr}{}\ledrightnote{\textcolor{blue}{Alfred Kerr}}} zu ſagen}{\lemma{\textnormal{\emph{Kerr zu ſagen}}}\Cendnote{\textnormal{\textcolor{blue}{Goldmann} sandte die Karte direkt weiter, sie ist im Nachlass \textcolor{blue}{Kerr}s überliefert.}}}\label{K_L03188-3h}\pend
           \endnumbering\briefempfaengerindex{Goldmann, Paul@\textsc{Goldmann, Paul}!zzzSchnitzler, Arthur@\emph{von Arthur Schnitzler}!1900-07-011@{1. 7. 1900}|)be}\mylabel{h}  \normalsize

\doendnotes{C}
\bigskip
\vfill

\clearpage

\footnotesize

\lohead{\textsc{register}}

% Definiere theindex-Environment komplett neu ohne reledmac
\makeatletter
\renewenvironment{theindex}{%
  \section*{\indexname}%
  \setlength{\parindent}{0pt}%
  \setlength{\parskip}{0pt plus 0.3pt}%
  \let\item\@idxitem
}{%
  \clearpage
}
\makeatother

\IfFileExists{\jobname-pw.ind}{\input{\jobname-pw.ind}}{}

\end{document}

      