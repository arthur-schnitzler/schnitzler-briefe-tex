%% latex-korrekturansicht-vorspann.tex
%% Vorspann für die Korrekturansicht.
%% Lädt die gemeinsame Datei latex-vorspann.tex mit gesetztem Schalter.

\newif\ifkorrekturansicht
\korrekturansichttrue

\input{../tex-inputs/latex-vorspann}


\renewcommand{\erwaehntePersonen}{Personen: Irene Auernheimer, Hermann Bahr, Richard Beer-Hofmann, Paula Beer-Hofmann, Josef Böhm, Hans von Bülow, Marie Epstein, Anna Epstein, Ella Frankfurter, Albert Frankfurter, Leonie Guttmann, Heinrich Mann, Eduard Pötzl, Felix Salten, Ottilie Salten, Paul Salten, Louise Schnitzler, Heinrich Schnitzler, Olga Schnitzler}
\renewcommand{\erwaehnteInstitutionen}{Institutionen: Österreichischer Lloyd}
\renewcommand{\erwaehnteOrte}{Orte: Berlin, Bozen, Dölsach, England, Große Dolomitenstraße, Lago di Garda, Lienz, Meran, Pustertal, Tirol, Welsberg-Taisten, Wien, Wildbad Waldbrunn}
\renewcommand{\erwaehnteWerke}{Werke: B.Z. am Mittag, Briefe und Schriften, Das gelobte Wien, Der Ruf des Lebens. Schauspiel in drei Akten, Der Weg ins Freie. Roman, Der Wiener Korrespondent, Morgen. Wochenschrift für deutsche Kultur, Zwischen den Rassen}
\section[ Arthur Schnitzler an Felix Salten, 5. 8. 1907]{Arthur Schnitzler an Felix Salten, 5. 8. 1907}
\nopagebreak\mylabel{v}
\rehead{ }\normalsize\beginnumbering\briefempfaengerindex{Salten, Felix@\textsc{Salten, Felix}!zzzSchnitzler, Arthur@\emph{von Arthur Schnitzler}!1907-08-051@{5. 8. 1907}|(be}
\toendnotes[C]{\smallbreak\pagebreak[2]}\Standort{Wienbibliothek im Rathaus, ZPH 1681, 2.1.516.}
\physDesc{Brief, 3 Blätter, 6 Seiten, 2900 Zeichen (Paginiert: »1«–»3«)
\newline{}Handschrift: Bleistift, deutsche Kurrent
\newline{}Ordnung: mit Bleistift von unbekannter Hand Nummerierung der Doppelseiten des
                                 Konvoluts: »8«–»10« }
\buchAbdrucke{\weitereDrucke{1) Arthur Schnitzler: \emph{Briefe 1875–1912}. Hg. Therese Nickl und Heinrich Schnitzler. Frankfurt am Main: \emph{S. Fischer} 1981, S. 560–561.} \weitereDrucke{2) Hermann Bahr, Arthur Schnitzler: \emph{Briefwechsel, Aufzeichnungen, Dokumente (1891–1931)}. Hg. Kurt Ifkovits und Martin Anton Müller. Göttingen: \emph{Wallstein} 2018, S. 395.} }\toendnotes[C]{\smallbreak}
\pstart
           \noindent{}{\pb}\textcolor{gray}{\textbf{Telegramm-Adresse: \textbf{\textcolor{blue}{Böhm}{}\ledrightnote{\textcolor{blue}{Josef Böhm}} – \textcolor{pink}{Welsberg}{}\ledrightnote{\textcolor{pink}{Welsberg-Taisten}}.}}}\pend
           
\pstart
           \textcolor{gray}{\textbf{\textbf{Hôtel {\kaufmannsund} Pension \textcolor{pink}{Wildbad Waldbrunn}{}\ledrightnote{\textcolor{pink}{Wildbad Waldbrunn}}}}}\pend
           
\pstart
           \textcolor{gray}{\textbf{bei \textbf{\textcolor{pink}{Welsberg}{}\ledrightnote{\textcolor{pink}{Welsberg-Taisten}}} (Eilzughaltestelle) }}\pend
           
\pstart
           \textcolor{gray}{\textbf{1150 M. \textsuperscript{ü}/Meer. \hspace*{1.5em}\textcolor{pink}{Hochpusterthal}{}\ledrightnote{\textcolor{pink}{Pustertal}} (\textcolor{pink}{Tirol}{}\ledrightnote{\textcolor{pink}{Tirol}})}}\pend
           
\pstart
           \textcolor{gray}{\textbf{Heilkräftiges altbekanntes Bad in prachtvoller Lage.}}\pend
           
\pstart
           \textcolor{gray}{\textbf{\textbf{Ausgezeichnete Trinkquelle}.}}\pend
           
\pstart
           \textcolor{gray}{\textbf{70 mit allem Comfort eingerichtete Zimmer.}}\pend
           
\pstart
           \raggedleft{}\textcolor{gray}{\textbf{\emph{\textcolor{pink}{Waldbrunn}{}\ledrightnote{\textcolor{pink}{Welsberg-Taisten}}, den}}}{ }5. 8. \textcolor{gray}{\textbf{\emph{190}}}7\pend
           
\pstart
           lieber, ich danke Ihnen für Ihre Nachrichten, laſſen Sie uns jetzt
               nur bald hören, dſs Ihre \textcolor{blue}{Frau}{}\ledrightnote{{$\rightarrow$}\textcolor{blue}{Ottilie Salten}} ſich vollko{\geminationm}en erholt hat. Dem \textcolor{blue}{Buben}{}\ledrightnote{{$\rightarrow$}\textcolor{blue}{Paul Salten}} geht’s wohl ſchon wieder
               ganz gut? Wir ſind nun einen vollen Monat \textcolor{pink}{da}{}\ledrightnote{{$\rightarrow$}\textcolor{pink}{Wildbad Waldbrunn}} und werden wahrſcheinlich \label{K_L03009-1v}\edtext{bis nach dem 20. bleiben}{\lemma{\textnormal{\emph{bis nach dem 20. bleiben}}}\Cendnote{\textnormal{Sie blieben bis zum 26. 8. 1907.}}}\label{K_L03009-1h}.
                  \label{K_L03009-2v}\edtext{Heute ko{\geminationm}t meine \textcolor{blue}{Mama}{}\ledrightnote{{$\rightarrow$}\textcolor{blue}{Louise Schnitzler}} an, vielleicht ni{\geminationm}t ſie \textcolor{blue}{Heini}{}\ledrightnote{\textcolor{blue}{Heinrich Schnitzler}} mit
               nach \textcolor{pink}{Wien}{}\ledrightnote{\textcolor{pink}{Wien}}}{\lemma{\textnormal{\emph{Heute … Wien}}}\Cendnote{\textnormal{\textcolor{blue}{Louise Schnitzler} war zwischen 5. 8. 1907 und 24. 8. 1907 in \textcolor{pink}{Welsberg}. \textcolor{blue}{Heinrich Schnitzler} reiste erst am 26. 8. 1907 ab.}}}\label{K_L03009-2h}; da{\geminationn} wollen wir, \textcolor{blue}{Olga}{}\ledrightnote{\textcolor{blue}{Olga Schnitzler}}
               u ich{[},{]} noch \label{K_L03009-3v}\edtext{ſüdlicher}{\lemma{\textnormal{\emph{ſüdlicher}}}\Cendnote{\textnormal{siehe Felix und Ottilie Salten an Arthur Schnitzler, 3. 8. 1907}}}\label{K_L03009-3h}, vielleicht, u theilweiſe zu Fuſs, über die neue \textcolor{pink}{Dolomitenstraße}{}\ledrightnote{\textcolor{pink}{Große Dolomitenstraße}}; nach \textcolor{pink}{Bozen}{}\ledrightnote{\textcolor{pink}{Bozen}}. In \textcolor{pink}{Meran}{}\ledrightnote{\textcolor{pink}{Meran}} oder am \textcolor{pink}{Gardaſee}{}\ledrightnote{\textcolor{pink}{Lago di Garda}} denken wir eine Woche zu raſten und
                  da{\geminationn}, in den erſten Septembertagen, in \textcolor{pink}{Wien}{}\ledrightnote{\textcolor{pink}{Wien}} einzutreffen.
               Möglich, daſs wir irgendwo \label{K_L03009-4v}\edtext{mit {\pb}\textcolor{blue}{Richard}{}\ledrightnote{\textcolor{blue}{Richard Beer-Hofmann}} u \textcolor{blue}{Paula}{}\ledrightnote{\textcolor{blue}{Paula Beer-Hofmann}} zuſa{\geminationm}entreffen}{\lemma{\textnormal{\emph{mit … zuſammentreffen}}}\Cendnote{\textnormal{nicht geschehen, vgl. Richard Beer-Hofmann an Arthur Schnitzler, 29. 8. 1907 und Arthur Schnitzler an Richard Beer-Hofmann, 9. 9. 1907}}}\label{K_L03009-4h}. Sie wollen im September eine Meerfahrt
               unternehmen? Thäts der \textcolor{pink}{Gardaſee}{}\ledrightnote{\textcolor{pink}{Lago di Garda}} nicht auch?
               Mein Rad hab ich nicht mit, bedaure es auch nicht ſehr, da meine Zeit reichlich
               ausgefüllt iſt. Vormittg Waldwanderungen, allein, oder mit \textcolor{blue}{Olga}{}\ledrightnote{\textcolor{blue}{Olga Schnitzler}}; Nachmittg 2–6 etwa arbeit
               ich; da{\geminationn} ſpaziren; da{\geminationn}
               Nachtmahl und \label{K_L03009-5v}\edtext{Platformwandelei}{\lemma{\textnormal{\emph{Platformwandelei}}}\Cendnote{\textnormal{Die Schreibweise deutet auf eine
                  englischsprachige Aussprache durch \textcolor{blue}{Schnitzler} hin.}}}\label{K_L03009-5h}. Tennis haben wir erſt einmal geſpielt – der Platz
               lächerlich; unſre Partnerin ware eine ſehr charmante junge Frau \textsc{\textcolor{blue}{Epstein}{}\ledrightnote{\textcolor{blue}{Marie Epstein}}} (geboren \textsc{Miss \textcolor{blue}{Hudetz}{}\ledrightnote{\textcolor{blue}{Marie Epstein}}}), Schwägerin der \textsc{\textcolor{blue}{Anna – Epstein Loeb}{}\ledrightnote{\textcolor{blue}{Anna Epstein}}}. Ferner befinden ſich hier die \textcolor{blue}{Schweſtern}{}\ledrightnote{{$\rightarrow$}\textcolor{blue}{Leonie Guttmann}{\newline}{$\rightarrow$}\textcolor{blue}{Ella Frankfurter}} der Frau \textsc{\textcolor{blue}{Auernheimer}{}\ledrightnote{\textcolor{blue}{Irene Auernheimer}}}, und allerlei \label{K_L03009-6v}\edtext{\textsc{Ascenden\textcolor{gray}{z}} u \textsc{Descendenz}}{\lemma{\textnormal{\emph{Ascendenz u Descendenz}}}\Cendnote{\textnormal{Auf- und Absteigendes}}}\label{K_L03009-6h}; zum Theil
               gutes u. vorzügliches Menſchenmaterial. Der Mann der verheirateten \textcolor{blue}{Schweſter}{}\ledrightnote{{$\rightarrow$}\textcolor{blue}{Ella Frankfurter}}, \textcolor{blue}{Frankfurter}{}\ledrightnote{\textcolor{blue}{Albert Frankfurter}} mit Namen, Direktor {\pb}des \textcolor{brown}{oeſterr.
                  Lloyd}{}\ledrightnote{\textcolor{brown}{Österreichischer Lloyd}}, ſcheint was nicht gewöhnliches zu ſein. – Daſs \textcolor{blue}{Bahr}{}\ledrightnote{\textcolor{blue}{Hermann Bahr}} Sie gegen \textcolor{blue}{Pötzl}{}\ledrightnote{\textcolor{blue}{Eduard Pötzl}} –
               wie ſoll man da ſagen – in Schmutz nehmen? – mußte, hat uns ſehr amusirt. We{\geminationn} ich ſowohl Ihren \textcolor{green}{\textcolor{green}{Morgen}{}\ledrightnote{\textcolor{green}{Morgen. Wochenschrift für deutsche Kultur}}ruf}{}\ledrightnote{{$\rightarrow$}\textcolor{green}{Der Wiener Korrespondent}} als \textcolor{blue}{Pötzl}{}\ledrightnote{\textcolor{blue}{Eduard Pötzl}}’s \textcolor{green}{Lobeshymne}{}\ledrightnote{{$\rightarrow$}\textcolor{green}{Das gelobte Wien}} zu leſen beko{\geminationm}en könnte, wär ich
               Ihnen herzlich verbunden. (Daſs Sie mir die berühmte \label{K_L03009-7v}\edtext{Sa{\geminationm}lung der 12 \textcolor{pink}{Berl}{}\ledrightnote{\textcolor{pink}{Berlin}}. Feu{[}i{]}lletons}{\lemma{\textnormal{\emph{Sammlung … Feuilletons}}}\Cendnote{\textnormal{Es dürfte sich um \textcolor{blue}{Salten}s Beiträge für die \emph{\textcolor{green}{B. Z. am
                     Mittag}} handeln. Dass diese, abgesehen von einer Ausnahme, vollständig in
                     \textcolor{blue}{Salten}s Zusammenstellungen seiner
                  journalistischen Arbeiten in seinem Nachlass fehlen, dürfte als Indiz genommen
                  werden, dass \textcolor{blue}{Salten} mit den Texten eine
                  Publikation plante oder sie zumindest als zusammengehörig betrachtete. \textcolor{blue}{Salten}s Brief vom 15. 8. 1907 lässt zudem
                  vermuten, dass es sich um Beiträge zu seiner \textcolor{pink}{England}-Reise im Juni 1906 handelte, vgl. Felix Salten an Arthur Schnitzler, 19. 6. 1906.}}}\label{K_L03009-7h} noch immer
               nicht gegeben haben, nur nebenbei.) Wie ſtehts im übrigen mit Ihren Arbeiten? In
               welcher ſtecken Sie am liebſten? – Ich ſchreibe hier nur an dem \textcolor{green}{Roman}{}\ledrightnote{{$\rightarrow$}\textcolor{green}{Der Weg ins Freie. Roman}}; letzte, zum Theil wohl {\pb}vorletzte Feile; habe ein wunderſchönes
               Zimmer, in das vom Hoteltrubel nichts dringt, mit einem guten Blick über Wieſen und
               Wald ins Thal; vorgebauter Balkon; oberſter Stock. – (Das idealſte Arbeitszimmer –
               ohne dieſes, glaub ich, hielt es mich doch nicht ſo lang hier). An \label{K_L03009-8v}\edtext{\textcolor{pink}{Lienz}{}\ledrightnote{\textcolor{pink}{Lienz}} vorüberfahrend und an \textsc{\textcolor{pink}{Dölsach}{}\ledrightnote{\textcolor{pink}{Dölsach}}}}{\lemma{\textnormal{\emph{Lienz … Dölsach}}}\Cendnote{\textnormal{vgl. Felix Salten an Arthur Schnitzler, 14. 8. 1893}}}\label{K_L03009-8h}
                (ſo heißts doch) blieb ich nicht ungerührt – – \label{K_L03009-9v}\edtext{»\textcolor{green}{wie war
                  ich jung}{}\ledrightnote{{$\rightarrow$}\textcolor{green}{Der Ruf des Lebens. Schauspiel in drei Akten}}« heißt es in der ſchönſten \textcolor{green}{Scene}{}\ledrightnote{\textcolor{green}{Der Ruf des Lebens. Schauspiel in drei Akten}}}{\lemma{\textnormal{\emph{»wie … Scene}}}\Cendnote{\textnormal{Das ist zu finden in der siebten Szene
                  des ersten Akts von \emph{\textcolor{green}{Der Ruf des Lebens}}.
               }}}\label{K_L03009-9h} die ich je geſchrieben habe (aber es ſtehen auch originellere Sachen drin.) –
               Leſe hauptſächlich \textsc{\textcolor{blue}{Bülow (Hans v.}{}\ledrightnote{\textcolor{blue}{Hans von Bülow}}}) \textcolor{green}{Briefe}{}\ledrightnote{\textcolor{green}{Briefe und Schriften}}, jetzt den letzten, 5. Band. Die
                  \textsc{\textcolor{blue}{Mann}{}\ledrightnote{\textcolor{blue}{Heinrich Mann}}}ſchen \textcolor{green}{Zwei Racen}{}\ledrightnote{{$\rightarrow$}\textcolor{green}{Zwischen den Rassen}} mit
               Bewunderung und mit \strikeout{allerlei} leiſem Widerſtand gegen
               allerlei menſchliches in \textsc{\textcolor{blue}{Heinrich}{}\ledrightnote{\textcolor{blue}{Heinrich Mann}}s} Seele\pend
           
\pstart
           {\pb}Es wäre lieb von Ihnen, we{\geminationn} Sie nächſtens etwas mehr von ſich vernehmen ließen;
               insbeſonders wünſcht’ ich zu wiſſen, welchen Ihrer Stoffe ſie jetzt am ſtärkſten
               bewegt und welchen Sie »zunächſt« (ein ſcheußliches \textcolor{pink}{Berlin}{}\ledrightnote{\textcolor{pink}{Berlin}}er Wort) in Bewegung zu ſetzen gedenken. Da{\geminationn} Ihr Befinden, kurz u gut, was Sie mir \introOben{}zu\introOben{} ſagen haben\textcolor{gray}{.} Schöner wärs natürlich,
                  we{\geminationn}{ }{\pb}man an irgd einem Ufer gemeinſam wandelte,
               wo ſich »denn« u. ſ. w.\pend
           
\pstart
           \textcolor{blue}{Wir}{}\ledrightnote{{$\rightarrow$}\textcolor{blue}{Olga Schnitzler}} grüßen Sie vielmals
               {\\[\baselineskip]}Von Herzen {\\[\baselineskip]}Ihr {\\[\baselineskip]}\spacefill\mbox{Arthur}\pend
           \leftskip=0em{}\endnumbering\briefempfaengerindex{Salten, Felix@\textsc{Salten, Felix}!zzzSchnitzler, Arthur@\emph{von Arthur Schnitzler}!1907-08-051@{5. 8. 1907}|)be}\mylabel{h}  \normalsize

\doendnotes{C}
\bigskip
\vfill

\clearpage

\footnotesize

\lohead{\textsc{register}}

% Definiere theindex-Environment komplett neu ohne reledmac
\makeatletter
\renewenvironment{theindex}{%
  \section*{\indexname}%
  \setlength{\parindent}{0pt}%
  \setlength{\parskip}{0pt plus 0.3pt}%
  \let\item\@idxitem
}{%
  \clearpage
}
\makeatother

\IfFileExists{\jobname-pw.ind}{\input{\jobname-pw.ind}}{}

\end{document}

      