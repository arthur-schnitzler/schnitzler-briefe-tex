%% latex-korrekturansicht-vorspann.tex
%% Vorspann für die Korrekturansicht.
%% Lädt die gemeinsame Datei latex-vorspann.tex mit gesetztem Schalter.

\newif\ifkorrekturansicht
\korrekturansichttrue

\input{../tex-inputs/latex-vorspann}


               \section[Gerhart Hauptmann an Arthur Schnitzler, 18. 12. 1906]{ Gerhart Hauptmann an Arthur Schnitzler, 18. 12. 1906}\nopagebreak\mylabel{v}\rehead{ }\normalsize\beginnumbering\briefempfaengerindex{Schnitzler, Arthur@\textsc{Schnitzler, Arthur}!zzzHauptmann, Gerhart@\emph{von Gerhart Hauptmann}!1906-12-181@{18. 12. 1906}|(be} \toendnotes[C]{\smallbreak\pagebreak[2]} \Standort{DLA, A:Schnitzler, 61.587.}
\physDesc{Brief, 1 Blatt (Briefpapier mit Trauerrand), 1 Seite
\newline{}Handschrift: schwarze Tinte, lateinische Kurrent\newline{}Ordnung: mit blauer Tinte von unbekannter Hand »19274« }\buchAbdrucke{\weitereDrucke{Hans-Ulrich Lindken: \emph{Arthur Schnitzler. Aspekte und Akzente. Materialien zu Leben
                        und Werk}. Frankfurt am Main, Bern, New York: \emph{Peter Lang} 1984, S. 184.} }\toendnotes[C]{\smallbreak}\pstart{}{\pb}Lieber Herr Schnitzler.\pend\pstart
           Ich habe herzlich dankbar Ihre \label{K_L01643_1v}\edtext{\textcolor{blue}{Theilnahme}{}\ledrightnote{→\textcolor{blue}{Marie Hauptmann}}}{\lemma{\textnormal{\emph{Theilnahme}}}\Cendnote{\textnormal{Am
                            10. 12. 1906 war die Mutter \textcolor{blue}{Marie Hauptmann} gestorben.}}}\label{K_L01643_1h} empfunden. Lassen Sie mich Ihnen
                    die Hand drücken und seien Sie gegrüsst.\pend
           \pstart
           Ihr{\\[\baselineskip]}\spacefill\mbox{Gerhart Hauptmann}\pend
           \leftskip=0em{}\pstart
           \textcolor{pink}{Agnetendorf}{}\ledrightnote{\textcolor{pink}{Agnetendorf}}{\\}d. 18 Dec 1906\pend
           \endnumbering\briefempfaengerindex{Schnitzler, Arthur@\textsc{Schnitzler, Arthur}!zzzHauptmann, Gerhart@\emph{von Gerhart Hauptmann}!1906-12-181@{18. 12. 1906}|)be}\mylabel{h}  \normalsize

\doendnotes{C}
\bigskip
\vfill

\clearpage

\footnotesize

\lohead{\textsc{register}}

% Definiere theindex-Environment komplett neu ohne reledmac
\makeatletter
\renewenvironment{theindex}{%
  \section*{\indexname}%
  \setlength{\parindent}{0pt}%
  \setlength{\parskip}{0pt plus 0.3pt}%
  \let\item\@idxitem
}{%
  \clearpage
}
\makeatother

\IfFileExists{\jobname-pw.ind}{\input{\jobname-pw.ind}}{}

\end{document}

      