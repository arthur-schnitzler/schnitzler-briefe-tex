%% latex-korrekturansicht-vorspann.tex
%% Vorspann für die Korrekturansicht.
%% Lädt die gemeinsame Datei latex-vorspann.tex mit gesetztem Schalter.

\newif\ifkorrekturansicht
\korrekturansichttrue

\input{../tex-inputs/latex-vorspann}


               \section[Richard Beer-Hofmann an Arthur Schnitzler, {[}18. 7. 1894{]}]{ Richard Beer-Hofmann an Arthur Schnitzler, {[}18. 7. 1894{]}}\nopagebreak\mylabel{v}\rehead{ }\normalsize\beginnumbering\briefempfaengerindex{Schnitzler, Arthur@\textsc{Schnitzler, Arthur}!zzzBeer-Hofmann, Richard@\emph{von Richard Beer-Hofmann}!1894-07-182@{{[}18. 7. 1894{]}}|(be} \toendnotes[C]{\smallbreak\pagebreak[2]} \Standort{CUL, Schnitzler, B 8.}
\physDesc{Brief, 1 Blatt, 1 Seite
\newline{}Handschrift: Bleistift, lateinische Kurrent
\newline{}Schnitzler: mit Bleistift datiert: »18/7 94« und nummeriert: »34« }\buchAbdrucke{\weitereDrucke{Hermann Bahr, Arthur Schnitzler: \emph{Briefwechsel, Aufzeichnungen, Dokumente (1891–1931)}. Hg. Kurt Ifkovits und Martin Anton Müller. Göttingen: \emph{Wallstein} 2018.} }\toendnotes[C]{\smallbreak}\pstart
           \noindent{}{\pb}Lieber Arthur! Habe
               den Brief irrthümlich geöffnet \strikeout{B} sie \strikeout{A} antworten wol \textcolor{blue}{Bahr}{}\ledrightnote{\textcolor{blue}{Hermann Bahr}} dass er Samstag hieher kommen soll? Mit \textcolor{pink}{Salzburg}{}\ledrightnote{\textcolor{pink}{Salzburg}} wird es vorläufig nichts sein: \textcolor{blue}{Hugo}{}\ledrightnote{\textcolor{blue}{Hugo von Hofmannsthal}} wird \label{K_L00355_1v}\edtext{auch
                  nicht}{\lemma{\textnormal{\emph{auch
                  nicht}}}\Cendnote{\textnormal{Er trauerte um \textcolor{blue}{Josefine Wertheimstein}, die am 16. 7. 1894 gestorben
                  war.}}}\label{K_L00355_1h} von \textcolor{pink}{Fusch}{}\ledrightnote{\textcolor{pink}{Bad Fusch}} wo er seit ein paar Tagen
               ist kommen wollen. Verschieben wir also die Sache\pend
           \pstart
           Was ist Nachmittag? Ich bin jedenfalls bis circa ½ 5 zu Hause\pend
           \pstart Herzlich \spacefill\mbox{Richard}\pend{}\endnumbering\briefempfaengerindex{Schnitzler, Arthur@\textsc{Schnitzler, Arthur}!zzzBeer-Hofmann, Richard@\emph{von Richard Beer-Hofmann}!1894-07-182@{{[}18. 7. 1894{]}}|)be}\mylabel{h}  \normalsize

\doendnotes{C}
\bigskip
\vfill

\clearpage

\footnotesize

\lohead{\textsc{register}}

% Definiere theindex-Environment komplett neu ohne reledmac
\makeatletter
\renewenvironment{theindex}{%
  \section*{\indexname}%
  \setlength{\parindent}{0pt}%
  \setlength{\parskip}{0pt plus 0.3pt}%
  \let\item\@idxitem
}{%
  \clearpage
}
\makeatother

\IfFileExists{\jobname-pw.ind}{\input{\jobname-pw.ind}}{}

\end{document}

      