%% latex-korrekturansicht-vorspann.tex
%% Vorspann für die Korrekturansicht.
%% Lädt die gemeinsame Datei latex-vorspann.tex mit gesetztem Schalter.

\newif\ifkorrekturansicht
\korrekturansichttrue

\input{../tex-inputs/latex-vorspann}


               \section[Albert Ehrenstein an Arthur Schnitzler, 3. 12. 1905]{ Albert Ehrenstein an Arthur Schnitzler,
                    3. 12. 1905}\nopagebreak\mylabel{v}\rehead{ }\normalsize\beginnumbering\briefempfaengerindex{Schnitzler, Arthur@\textsc{Schnitzler, Arthur}!zzzEhrenstein, Albert@\emph{von Albert Ehrenstein}!1905-12-031@{3. 12. 1905}|(be} \toendnotes[C]{\smallbreak\pagebreak[2]} \Standort{CUL, Schnitzler, B 30.}
\physDesc{Brief, 1 Blatt, 1 Seite
\newline{}Handschrift: schwarze Tinte, deutsche Kurrent
\newline{}Schnitzler: mit Bleistift beschriftet: »\textsc{Ehrenst}« und die Adresse ergänzt: »\textsc{\textcolor{pink}{Ottakr.str. 114}}« }\buchAbdrucke{\weitereDrucke{Albert Ehrenstein: \emph{Briefe}. Hg. Hanni Mittelmann. München: \emph{Boer} 1989, S. 18 (Werke, 1).} }\pstart
           \raggedleft{}{\pb}\textcolor{pink}{XVI. Wien}{}\ledrightnote{\textcolor{pink}{XVI., Ottakring}},
                            3. XII. 1905\pend
           \pstart{}\textsc{Sehr geehrter Herr Doktor!}\pend\pstart
           Ermuntert durch Herrn Doktors liebenswürdiges Entgegenkommen erlaube ich mir
                    anbei meinen Dialog »\textcolor{green}{Amok}{}\ledrightnote{\textcolor{green}{Amok}}« zu unterbreiten
                    und hoffe ich in einiger Zeit das für mich maßgebende Urteil über dieſen
                    Trauerſchwank von Herrn Doktor hören zu können.\pend
           \pstart
           Ergebenſt{\\[\baselineskip]}\spacefill\mbox{Albert Ehrenstein.}\pend
           \leftskip=0em{}\endnumbering\briefempfaengerindex{Schnitzler, Arthur@\textsc{Schnitzler, Arthur}!zzzEhrenstein, Albert@\emph{von Albert Ehrenstein}!1905-12-031@{3. 12. 1905}|)be}\mylabel{h}  \normalsize

\doendnotes{C}
\bigskip
\vfill

\clearpage

\footnotesize

\lohead{\textsc{register}}

% Definiere theindex-Environment komplett neu ohne reledmac
\makeatletter
\renewenvironment{theindex}{%
  \section*{\indexname}%
  \setlength{\parindent}{0pt}%
  \setlength{\parskip}{0pt plus 0.3pt}%
  \let\item\@idxitem
}{%
  \clearpage
}
\makeatother

\IfFileExists{\jobname-pw.ind}{\input{\jobname-pw.ind}}{}

\end{document}

      