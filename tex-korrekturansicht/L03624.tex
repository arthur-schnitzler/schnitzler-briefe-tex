%% latex-korrekturansicht-vorspann.tex
%% Vorspann für die Korrekturansicht.
%% Lädt die gemeinsame Datei latex-vorspann.tex mit gesetztem Schalter.

\newif\ifkorrekturansicht
\korrekturansichttrue

\input{../tex-inputs/latex-vorspann}


\section[Stefan Zweig an Arthur Schnitzler, 13. 12. 1909]{L03624 Stefan Zweig an Arthur Schnitzler, 13. 12. 1909}
\nopagebreak\mylabel{L03624v}
\rehead{ }\normalsize\beginnumbering\briefempfaengerindex{Schnitzler, Arthur@\textsc{Schnitzler, Arthur}!zzzZweig, Stefan@\emph{von Stefan Zweig}!1909-12-133@{13. 12. 1909}|(be}
\toendnotes[C]{\smallbreak\pagebreak[2]}
\correspDesc{Versand  durch Stefan Zweig am 13. 12. 1909 in Wien
\newline{}Erhalt  durch Arthur Schnitzler im Zeitraum [13. 12. 1909 – 16. 12. 1909?] in Wien}\toendnotes[C]{\smallbreak}
\Standort{CUL, Schnitzler, B 118.}
\physDesc{Brief, 1 Blatt, 3 Seiten, 1907 Zeichen
\newline{}Handschrift: lila Tinte, lateinische Kurrent
\newline{}Schnitzler: mit Bleistift »\textsc{Zweig}« }
\buchAbdrucke{\weitereDrucke{1) Stefan Zweig: \emph{Briefwechsel mit Hermann Bahr, Sigmund Freud, Rainer Maria
                        Rilke und Arthur Schnitzler}. Herausgegeben von Jeffrey B. Berlin, Hans-Ulrich Lindken und Donald A. Prater. Frankfurt am Main: \emph{S. Fischer} 1987, S. 357–358.} \weitereDrucke{2) Stefan Zweig: \emph{Briefe. Bd. I: 1897–1914}. Herausgegeben von Knut Beck, Jeffrey B. Berlin und Natascha Weschenbach-Feggeler. Frankfurt am Main: \emph{S. Fischer} 1995, S. 201.} }\toendnotes[C]{\smallbreak}
\pstart
           {\pb}\textcolor{gray}{\textbf{SZ}}\hfill \textcolor{gray}{\textbf{\textcolor{pink}{VIII. KOCHGASSE 8}\oindex{Wien@\textbf{Wien}!VIII., Josefstadt@\textbf{VIII., Josefstadt}!Kochgasse 8@\textbf{Kochgasse 8}, \emph{Wohngebäude}|pw}{}\ledrightnote{\textcolor{pink}{Kochgasse 8}}}}\pend
           
\pstart
           \raggedleft{}\textcolor{gray}{\textbf{\textcolor{pink}{WIEN}\oindex{Wien@\textbf{Wien}, \emph{Verwaltungsgebiet}|pw}{}\ledrightnote{\textcolor{pink}{Wien}},}}{ }13. Dez 09\pend
           {\vspace{1\baselineskip}}
\pstart{}Sehr verehrter Herr Doktor,\pend\vspace{0.5em}
\pstart
           ich hatte \label{K_L03624-1v}\edtext{gestern}{\lemma{\textnormal{\emph{gestern}}}\Cendnote{\textnormal{\emph{\textcolor{green}{Der Ruf des Lebens}\pwindex{Schnitzler, Arthur 15. 5. 1862 Wien – 21. 10. 1931 ebd.@\textsc{Schnitzler, Arthur} (15. 5. 1862 Wien – 21. 10. 1931 ebd.), \emph{Schriftsteller, Mediziner}!Ruf des Lebens. Schauspiel in drei Akten@\strich\emph{Der Ruf des Lebens. Schauspiel in drei Akten}|pwk}} von \textcolor{blue}{Schnitzler} erlebte am 11. 12. 1909 am \textcolor{pink}{Deutschen Volkstheater}\oindex{Wien@\textbf{Wien}!VII., Neubau@\textbf{VII., Neubau}!Volkstheater@\textbf{Volkstheater}, \emph{Theater}|pwk} seine \textcolor{pink}{Wiener}\oindex{Wien@\textbf{Wien}, \emph{Verwaltungsgebiet}|pwk}{ }\textcolor{violet}{Erstaufführung}\eventindex{Volkstheater@\textbf{Volkstheater}!Premiere von Der Ruf des Lebens, 11.12.1909@Premiere von Der Ruf des Lebens, 11.12.1909|pwkv}. Am 12. 12. 1909 fand die
                  zweite \textcolor{violet}{Vorstellung}\eventindex{Volkstheater@\textbf{Volkstheater}!Aufführung von Der Ruf des Lebens, 12.12.1909@Aufführung von Der Ruf des Lebens, 12.12.1909|pwkv} statt.
                     \textcolor{blue}{Schnitzler} wohnte beiden Aufführungen
                  bei, vgl. A. S.: \emph{Tagebuch}, 11. 12. 1909 und 12. 12. 1909.}}}\label{K_L03624-1}
               die Freude, der erfolgreichen \textcolor{violet}{Aufführung}\eventindex{Volkstheater@\textbf{Volkstheater}!Aufführung von Der Ruf des Lebens, 12.12.1909@Aufführung von Der Ruf des Lebens, 12.12.1909|pwv}{}\ledrightnote{{$\rightarrow$}\emph{\textcolor{violet}{Aufführung von Der Ruf des Lebens, 12.12.1909}}} Ihres »\textcolor{green}{Ruf des Lebens}\pwindex{Schnitzler, Arthur 15. 5. 1862 Wien – 21. 10. 1931 ebd.@\textsc{Schnitzler, Arthur} (15. 5. 1862 Wien – 21. 10. 1931 ebd.), \emph{Schriftsteller, Mediziner}!Ruf des Lebens. Schauspiel in drei Akten@\strich\emph{Der Ruf des Lebens. Schauspiel in drei Akten}|pw}{}\ledrightnote{\textcolor{green}{Der Ruf des Lebens. Schauspiel in drei Akten}}«
               beizuwohnen. Es wäre ungeziemend wollte ich mir eine Bemerkung über das Wesen und den
               Wert des \textcolor{green}{Stückes}\pwindex{Schnitzler, Arthur 15. 5. 1862 Wien – 21. 10. 1931 ebd.@\textsc{Schnitzler, Arthur} (15. 5. 1862 Wien – 21. 10. 1931 ebd.), \emph{Schriftsteller, Mediziner}!Ruf des Lebens. Schauspiel in drei Akten@\strich\emph{Der Ruf des Lebens. Schauspiel in drei Akten}|pwv}{}\ledrightnote{{$\rightarrow$}\emph{\textcolor{green}{Der Ruf des Lebens. Schauspiel in drei Akten}}}{ }\introOben{}zu\introOben{} Ihnen ungefragt gestatten, aber das darf ich Ihnen wohl
               sagen, dass ich vielleicht niemals von einem Ihrer Werke im Theater einen so
               gewaltigen und wirklich die letzten Erschütterungen aufwühlenden Eindruck empfunden
               habe. Sie bedürfen heute längst nicht mehr einer Zustimmung – am wenigsten von uns,
               die wir alle an Ihnen zu lernen haben – aber eben, weil diesem \textcolor{green}{Stück}\pwindex{Schnitzler, Arthur 15. 5. 1862 Wien – 21. 10. 1931 ebd.@\textsc{Schnitzler, Arthur} (15. 5. 1862 Wien – 21. 10. 1931 ebd.), \emph{Schriftsteller, Mediziner}!Ruf des Lebens. Schauspiel in drei Akten@\strich\emph{Der Ruf des Lebens. Schauspiel in drei Akten}|pwv}{}\ledrightnote{{$\rightarrow$}\emph{\textcolor{green}{Der Ruf des Lebens. Schauspiel in drei Akten}}}{ }\label{K_L03624-2v}\edtext{soviel Missverständnis}{\lemma{\textnormal{\emph{soviel Missverständnis}}}\Cendnote{\textnormal{\textcolor{blue}{Schnitzler} vermerkt im \emph{\textcolor{green}{Tagebuch}\pwindex{Schnitzler, Arthur 15. 5. 1862 Wien – 21. 10. 1931 ebd.@\textsc{Schnitzler, Arthur} (15. 5. 1862 Wien – 21. 10. 1931 ebd.), \emph{Schriftsteller, Mediziner}!Tagebuch@\strich\emph{Tagebuch}|pwk}} am 17. 12. 1909 nach einem Gespräch mit \textcolor{blue}{Stefan Zweig}\pwindex{Zweig, Stefan 28.\,11.\,1881 Wien – 23.\,2.\,1942 Petrópolis@\textsc{Zweig, Stefan} (28.\,11.\,1881 Wien – 23.\,2.\,1942 Petrópolis), \emph{Schriftsteller}|pwk}, dass dieser »mit einem
                     Vorurtheil nach den \textcolor{pink}{Berliner}\oindex{Berlin@\textbf{Berlin}, \emph{Hauptstadt}|pw} Kritiken
                     gekommen und ganz gewonnen« worden sei. Die \textcolor{pink}{Berliner}\oindex{Berlin@\textbf{Berlin}, \emph{Hauptstadt}|pwk}{ }\textcolor{violet}{Premiere}\eventindex{Lessing-Theater@\textbf{Lessing-Theater}!Premiere von Der Ruf des Lebens, 24.2.1906@Premiere von Der Ruf des Lebens, 24.2.1906|pwkv} am 24. 2. 1906 war ambivalent
                  besprochen worden, vgl. etwa: \textcolor{blue}{Rudolf
                        Herzog}\pwindex{Herzog, Rudolf 6.\,12.\,1869 Barmen – 3.\,3.\,1943 Rheinbreitbach@\textsc{Herzog, Rudolf} (6.\,12.\,1869 Barmen – 3.\,3.\,1943 Rheinbreitbach), \emph{Schriftsteller}|pwk}: \emph{\textcolor{green}{Lessing Theater. Zum ersten
                        Male: »Der Ruf des Lebens«, Schauspiel in drei Akten von Arthur
                        Schnitzler}\pwindex{Herzog, Rudolf 6.\,12.\,1869 Barmen – 3.\,3.\,1943 Rheinbreitbach@\textsc{Herzog, Rudolf} (6.\,12.\,1869 Barmen – 3.\,3.\,1943 Rheinbreitbach), \emph{Schriftsteller}!Lessing Theater. Zum ersten Male: »Der Ruf des Lebens«, Schauspiel in drei Akten von Arthur Schnitzler@\strich\emph{Lessing Theater. Zum ersten Male: »Der Ruf des Lebens«, Schauspiel in drei Akten von Arthur Schnitzler}|pwk}}. In: \emph{\textcolor{green}{Berliner Neueste
                        Nachrichten}\pwindex{Berliner Neueste Nachrichten@\emph{Berliner Neueste Nachrichten}|pwk}}, Jg. 26, Nr. 94, 25. 2. 1906, S. 3. \textcolor{blue}{M. J. [=Monty Jacobs]}\pwindex{Jacobs, Monty 5.\,1.\,1875 Szczecin – 29.\,12.\,1945 London@\textsc{Jacobs, Monty} (5.\,1.\,1875 Szczecin – 29.\,12.\,1945 London), \emph{Schriftsteller, Journalist, Kritiker}|pwk}: \emph{\textcolor{green}{Lessing Theater. Zum ersten Mal: »Der Ruf des Lebens«,
                        Schauspiel in drei Akten von Arthur Schnitzler}\pwindex{Jacobs, Monty 5.\,1.\,1875 Szczecin – 29.\,12.\,1945 London@\textsc{Jacobs, Monty} (5.\,1.\,1875 Szczecin – 29.\,12.\,1945 London), \emph{Schriftsteller, Journalist, Kritiker}!Lessing Theater. Zum ersten Mal: »Der Ruf des Lebens«, Schauspiel in drei Akten von Arthur Schnitzler@\strich\emph{Lessing Theater. Zum ersten Mal: »Der Ruf des Lebens«, Schauspiel in drei Akten von Arthur Schnitzler}|pwk}}. In: \emph{\textcolor{green}{Berliner Tageblatt}\pwindex{Berliner Tageblatt@\emph{Berliner Tageblatt}|pwk}}, Jg. 35, Nr. 102,
                        25. 2. 1906, S. 2–3. \textcolor{blue}{Alfred Kerr}\pwindex{Kerr, Alfred 25.\,12.\,1867 Breslau – 12.\,10.\,1948 Hamburg@\textsc{Kerr, Alfred} (25.\,12.\,1867 Breslau – 12.\,10.\,1948 Hamburg), \emph{Schriftsteller, Kritiker}|pwk}: \emph{\textcolor{green}{Ödipus und der Ruf des Lebens}\pwindex{Kerr, Alfred 25.\,12.\,1867 Breslau – 12.\,10.\,1948 Hamburg@\textsc{Kerr, Alfred} (25.\,12.\,1867 Breslau – 12.\,10.\,1948 Hamburg), \emph{Schriftsteller, Kritiker}!Ödipus und der Ruf des Lebens@\strich\emph{Ödipus und der Ruf des Lebens}|pwk}}, in: \emph{\textcolor{green}{Neue Rundschau}\pwindex{neue Rundschau@\emph{Die neue Rundschau}|pwk}}, Jg. 17, H. 5, Mai 1906,
                     S. 492–498. \textcolor{blue}{[Siegfried Jacobsohn]}\pwindex{Jacobsohn, Siegfried 28.\,1.\,1881 Berlin – 3.\,12.\,1926 ebd.@\textsc{Jacobsohn, Siegfried} (28.\,1.\,1881 Berlin – 3.\,12.\,1926 ebd.), \emph{Journalist, Kritiker, Publizist}|pwk}: \emph{\textcolor{green}{Der Ruf des Lebens}\pwindex{Ruf des Lebens@\emph{Der Ruf des Lebens}|pwk}}. In: \emph{\textcolor{green}{Die Schaubühne}\pwindex{Schaubühne@\emph{Die Schaubühne}|pwk}}, Jg. 2, Nr. 9, März 1906,
                     S. 246–250. Auch \textcolor{blue}{Schnitzler} selbst
                  war, besonders vom zweiten Akt, nicht überzeugt und versuchte zeitlebens immer
                  wieder, die Schwächen des Stückes zu beheben, jedoch ohne eine neue Fassung
                  fertigzustellen.}}}\label{K_L03624-2} – feind{\pb}lich
               oder auch freundlich – gegenüber stand, möchte ich Ihnen sagen, dass ich das Gefühl
               gänzlichen Einverständnis hatte. Ich habe wie selten hier die Gefühle in einer
                  nah\strikeout{t}en und doch nicht schamlosen menschlichen
               Körperlichkeit gefühlt und den ungeheuren Raum wirklich mit einem süssen und
               bezwingenden Schrecken aufgerissen gesehen, der zwischen dem intensivesten Leben und
               dem Nichts plötzlich aufspringen kann. Nie, soweit ich Ihr Werk überschaue, haben Sie
               eine ähnliche Gewalt über das Schicksal gezeigt und ich wäre froh, wenn Sie \label{T_L03624-1v}\edtext{sich}{\lemma{\textnormal{\emph{sich}}}\Cendnote{\textnormal{Er schreibt: »Sich«.}}}\label{T_L03624-1} dieses \textcolor{green}{Stück}\pwindex{Schnitzler, Arthur 15. 5. 1862 Wien – 21. 10. 1931 ebd.@\textsc{Schnitzler, Arthur} (15. 5. 1862 Wien – 21. 10. 1931 ebd.), \emph{Schriftsteller, Mediziner}!Ruf des Lebens. Schauspiel in drei Akten@\strich\emph{Der Ruf des Lebens. Schauspiel in drei Akten}|pwv}{}\ledrightnote{{$\rightarrow$}\emph{\textcolor{green}{Der Ruf des Lebens. Schauspiel in drei Akten}}} nicht um ein paar
               theatralischer Dinge willen jemals verärgern oder minder lieb haben liessen. Ich
               werde Ihnen immer dafür dankbar sein und ich glaube, immer mehr werden sich finden,
               die es so fühlen werden: nicht um des Gesagten willen, der Worte und der Menschen
               sosehr, sondern um der ungeheuren Vitalität willen, die aus jedem \substVorne{}\textsuperscript{A}\substDazwischen{}W\substHinten{}esen darin atmet. Diese feindliche Um{\pb}schlingung von Leben und Tod, die
               feurige Secunde ihres Einswerdens in der Leidenschaft wird mir unvergesslich eine der
               schönsten Erinnerungen an d\substVorne{}\textsuperscript{ie}\substDazwischen{}en\substHinten{}{ }\textcolor{violet}{Abend}\eventindex{Volkstheater@\textbf{Volkstheater}!Aufführung von Der Ruf des Lebens, 12.12.1909@Aufführung von Der Ruf des Lebens, 12.12.1909|pwv}{}\ledrightnote{{$\rightarrow$}\emph{\textcolor{violet}{Aufführung von Der Ruf des Lebens, 12.12.1909}}} sein.\pend
           
\pstart
           Nehmen Sie also innigen Dank für dieses \textcolor{green}{Werk}\pwindex{Schnitzler, Arthur 15. 5. 1862 Wien – 21. 10. 1931 ebd.@\textsc{Schnitzler, Arthur} (15. 5. 1862 Wien – 21. 10. 1931 ebd.), \emph{Schriftsteller, Mediziner}!Ruf des Lebens. Schauspiel in drei Akten@\strich\emph{Der Ruf des Lebens. Schauspiel in drei Akten}|pwv}{}\ledrightnote{{$\rightarrow$}\emph{\textcolor{green}{Der Ruf des Lebens. Schauspiel in drei Akten}}}, das alte Liebe und Verehrung bei mir nur vermehrt,
               bekräftigt und vertieft hat. Wie freue ich mich Ihrem nächsten entgegen!\pend
           
\pstart
           In herzlicher Ergebenheit{\\[\baselineskip]}\spacefill\mbox{Stefan Zweig}\pend
           \leftskip=0em{}\selectlanguage{ngerman}\endnumbering\briefempfaengerindex{Schnitzler, Arthur@\textsc{Schnitzler, Arthur}!zzzZweig, Stefan@\emph{von Stefan Zweig}!1909-12-133@{13. 12. 1909}|)be}\mylabel{L03624h}  \normalsize

\doendnotes{C}
\bigskip
\vfill

\clearpage

\footnotesize

\lohead{\textsc{register}}

% Definiere theindex-Environment komplett neu ohne reledmac
\makeatletter
\renewenvironment{theindex}{%
  \section*{\indexname}%
  \setlength{\parindent}{0pt}%
  \setlength{\parskip}{0pt plus 0.3pt}%
  \let\item\@idxitem
}{%
  \clearpage
}
\makeatother

\IfFileExists{\jobname-pw.ind}{\input{\jobname-pw.ind}}{}

\end{document}

      