%% latex-korrekturansicht-vorspann.tex
%% Vorspann für die Korrekturansicht.
%% Lädt die gemeinsame Datei latex-vorspann.tex mit gesetztem Schalter.

\newif\ifkorrekturansicht
\korrekturansichttrue

\input{../tex-inputs/latex-vorspann}


\renewcommand{\erwaehntePersonen}{Personen: André Antoine, Henriette Charasson, Samuel Fischer, Paul Morisse, Maurice Rémon, Olga Schnitzler, Stefan Zweig}
\renewcommand{\erwaehnteOrte}{Orte: Justizpalast Wien, Meran, Sternwartestraße 71, VIII., Josefstadt, Wien, Währinger Cottage}
\renewcommand{\erwaehnteWerke}{Werke: Das Haus am Meer. Ein Schauspiel in zwei Teilen (drei Aufzügen), Das weite Land. Tragikomödie in fünf Akten, Le Pays Inconnu}
\section[Stefan Zweig an Arthur Schnitzler, 6. {[}11.?{]} 1911]{Stefan Zweig an Arthur Schnitzler, 6. {[}11.?{]} 1911}
\nopagebreak\mylabel{v}
\rehead{ }\normalsize\beginnumbering\briefempfaengerindex{Schnitzler, Arthur@\textsc{Schnitzler, Arthur}!zzzZweig, Stefan@\emph{von Stefan Zweig}!1911-11-061@{6. {[}11.?{]} 1911}|(be}
\toendnotes[C]{\smallbreak\pagebreak[2]}\Standort{CUL, Schnitzler, B 118.}
\physDesc{Bildpostkarte, 623 Zeichen
\newline{}Handschrift: grüne Tinte, lateinische Kurrent
\newline{}Versand: Stempel: »\nobreak{}\oindex{VIII., Josefstadt@\textbf{VIII., Josefstadt}, \emph{A.ADM3}|pwk}8/\textcolor{gray}{×} Wien, 6. \textcolor{gray}{X}I. 11, 5\nobreak{}«.  }
\buchAbdrucke{\weitereDrucke{Stefan Zweig: \emph{Briefwechsel mit Hermann Bahr, Sigmund Freud, Rainer Maria
                        Rilke und Arthur Schnitzler}. Hg. Jeffrey B. Berlin, Hans-Ulrich Lindken und Donald A. Prater. Frankfurt am Main: \emph{S. Fischer} 1987, S. 367–368.} }\toendnotes[C]{\smallbreak}\pstart{}{\pb}D\textsuperscript{r} Artur
                  Schnitzler\pend{}\pstart{}\textcolor{pink}{Wien – Cottage}{}\ledrightnote{\textcolor{pink}{Währinger Cottage}}\pend{}\pstart{}\strikeout{Cott}{ }\textcolor{pink}{\label{K_L03630-1v}\edtext{Sternwartestrasse 72}{\lemma{\textnormal{\emph{Sternwartestrasse 72}}}\Cendnote{\textnormal{\textcolor{blue}{Zweig} wechselt bei der Adressierung
                        seiner Schreiben an \textcolor{blue}{Schnitzler} immer
                        wieder zwischen der falschen Hausnummer »72« und der
                        richtigen »71«.}}}\label{}}{}\ledrightnote{\textcolor{pink}{Sternwartestraße 71}}\pend{}
{\bigskip}
\pstart
           {\pb}\textcolor{pink}{\textcolor{gray}{\textbf{WIEN}}}{}\ledrightnote{\textcolor{pink}{Wien}}\hfill \textcolor{pink}{\textcolor{gray}{\textbf{Justiz-Palast}}}{}\ledrightnote{\textcolor{pink}{Justizpalast Wien}}\pend
           {\vspace{1\baselineskip}}\vspace{1em}
\pstart
           \noindent{}{\pb}Verehrter Herr Doktor,{ }\label{K_L03630-11v}\edtext{\textcolor{blue}{Paul Morisse}{}\ledrightnote{\textcolor{blue}{Paul Morisse}}}{\lemma{\textnormal{\emph{Paul Morisse}}}\Cendnote{\textnormal{Nach der ersten
                     Kontaktaufnahme im Februar 1911 (siehe Stefan Zweig an Arthur Schnitzler, 21. 2. 1911) betrieb \textcolor{blue}{Morisse} den Plan der Übersetzung von \emph{\textcolor{green}{Das weite Land}} in den folgenden Monaten ernsthafter. Er
                     nahm Kontakt mit \textcolor{blue}{S. Fischer} auf und bekam die Erlaubnis
                  für die Übersetzung von \textcolor{blue}{Schnitzler}. (Für die Entscheidung holte sich \textcolor{blue}{Schnitzler} die Meinung von \textcolor{blue}{André Antoine} 
                  ein, weil auch
                      \textcolor{blue}{Maurice Rémon} die Übersetzungsrechte erbeten hatte.)
                     Zugleich
                     versuchte \textcolor{blue}{Morisse}, ein Theater für die Inszenierung zu finden. Für die Übersetzungsarbeit sicherte er sich
                     eine Mitarbeiterin, \textcolor{blue}{Henriette Charasson}. Außer
                     einer Zeitungsmeldung, in der die Übersetzung unter dem Titel »\emph{\textcolor{green}{le Pays mystérieux}}«
                     angekündigt wurde, scheint sich die Sache schnell zerschlagen zu haben. Im Nachlass \textcolor{blue}{Schnitzlers} in der \emph{Cambridge
                              University Library} finden sich in der Mappe 244
                     mehrere Durchschläge einer französischen Übersetzung, bei der kein finaler Titel,
                     sondern nur handschriftliche Titelangaben angebracht wurden: »\textcolor{green}{Le Pays Inconnu}«, »\textcolor{green}{Le Pays de l’Ame}« und »\textcolor{green}{Le Pays Lontain}«. Ob es sich dabei
                     um die Übersetzung von \textcolor{blue}{Morisse}/\textcolor{blue}{Charasson} handelt,
                     ist unklar.}}}\label{}, dem ich seinerzeit das »\textcolor{green}{Weite Land}{}\ledrightnote{\textcolor{green}{Das weite Land. Tragikomödie in fünf Akten}}« zur Übertragung empfahl, möchte gern an das \textcolor{green}{Werk}{}\ledrightnote{{$\rightarrow$}\textcolor{green}{Das weite Land. Tragikomödie in fünf Akten}} gehen. Ich will Ihnen
               heute nur wiederholen, dass \textcolor{blue}{M.}{}\ledrightnote{\textcolor{blue}{Paul Morisse}} sowohl deutsch
               wie französisch glänzend beherrscht und ein ernster tüchtiger Übersetzer mit vielen
               literarischen Beziehungen ist, den ich Ihnen auf das wärmste empfehlen kann. Ich
               reise heute nach \textcolor{pink}{Meran}{}\ledrightnote{\textcolor{pink}{Meran}}, obwohl es mir gar nicht
               schlecht geht. \textcolor{green}{Das Haus am Meer}{}\ledrightnote{\textcolor{green}{Das Haus am Meer. Ein Schauspiel in zwei Teilen (drei Aufzügen)}} ist von einem
               halben Dutzend erster Bühnen bereits erworben.\pend
           \pstart {\pb}Mit vielen Grüssen an Ihre Frau \textcolor{blue}{Gemahlin}{}\ledrightnote{{$\rightarrow$}\textcolor{blue}{Olga Schnitzler}} und Sie Ihr stets
               getreuer \spacefill\mbox{Stefan Zweig}\pend{}\endnumbering\briefempfaengerindex{Schnitzler, Arthur@\textsc{Schnitzler, Arthur}!zzzZweig, Stefan@\emph{von Stefan Zweig}!1911-11-061@{6. {[}11.?{]} 1911}|)be}\mylabel{h}
\begin{anhang}
\end{anhang}\normalsize

\doendnotes{C}
\bigskip
\vfill

\clearpage

\footnotesize

\lohead{\textsc{register}}

% Definiere theindex-Environment komplett neu ohne reledmac
\makeatletter
\renewenvironment{theindex}{%
  \section*{\indexname}%
  \setlength{\parindent}{0pt}%
  \setlength{\parskip}{0pt plus 0.3pt}%
  \let\item\@idxitem
}{%
  \clearpage
}
\makeatother

\IfFileExists{\jobname-pw.ind}{\input{\jobname-pw.ind}}{}

\end{document}

      