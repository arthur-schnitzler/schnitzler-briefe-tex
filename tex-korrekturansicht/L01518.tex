%% latex-korrekturansicht-vorspann.tex
%% Vorspann für die Korrekturansicht.
%% Lädt die gemeinsame Datei latex-vorspann.tex mit gesetztem Schalter.

\newif\ifkorrekturansicht
\korrekturansichttrue

\input{../tex-inputs/latex-vorspann}


               \section[Arthur Schnitzler an Richard Beer-Hofmann, 23. 5. 1905]{ Arthur Schnitzler an Richard Beer-Hofmann, 23. 5. 1905}\nopagebreak\mylabel{v}\rehead{ }\normalsize\beginnumbering\briefempfaengerindex{Beer-Hofmann, Richard@\textsc{Beer-Hofmann, Richard}!zzzSchnitzler, Arthur@\emph{von Arthur Schnitzler}!1905-05-231@{23. 5. 1905}|(be} \toendnotes[C]{\smallbreak\pagebreak[2]} \Standort{YCGL, MSS 31.}
\physDesc{Kartenbrief
\newline{}Handschrift: schwarze Tinte, deutsche Kurrent\newline{}Versand: 1) Stempel: »\nobreak{}\oindex{XVIII., Waehring@\textbf{XVIII., Währing}, \emph{Bezirk (A.BZK)}|pwk}18/1 Wien 110, 2\textcolor{gray}{3}. V. 05, X\nobreak{}«.  2) Stempel: »\nobreak{}\oindex{Rodaun@\textbf{Rodaun}, \emph{Teil eines besiedelten Ortes (A.BSOX)}|pwk}R{[}odaun{]}, 2\textcolor{gray}{3. 5. 05}, 2–4\textcolor{gray}{N}\nobreak{}«. }\buchAbdrucke{\weitereDrucke{Arthur Schnitzler, Richard Beer-Hofmann: \emph{Briefwechsel 1891–1931}. Hg. Konstanze Fliedl. Wien, Zürich: \emph{Europaverlag} 1992, S. 172.} }\toendnotes[C]{\smallbreak}\pstart{}{\pb}Herrn \textsc{Dr. Richard
                     Beer-Hofmann}\pend{}\pstart{}\textcolor{pink}{Rodaun}{}\ledrightnote{\textcolor{pink}{Rodaun}}\pend{}\pstart{}\textsc{bei \textcolor{pink}{Liesing}{}\ledrightnote{\textcolor{pink}{XXIII., Liesing}}}\pend{}\pstart{}\textcolor{pink}{\textsc{Liesingerstraße 1}}{}\ledrightnote{\textcolor{pink}{Liesingerstraße}}.\pend{}{\bigskip}\pstart
           \raggedleft{}{\pb}23. 5. 905\pend
           \pstart
           lieber Richard, ich beſtätige den unerwarteten Empfang des \textcolor{blue}{\textsc{Frisch}}{}\ledrightnote{\textcolor{blue}{Efraim Frisch}}ſchen \textcolor{green}{Buches}{}\ledrightnote{→\textcolor{green}{Das Verlöbnis. Geschichte eines Knaben}}; – bedeutet das vielleicht den \substVorne{}\textsuperscript{Empfang}{\allowbreak}\substDazwischen{}Anfang\substHinten{} der Überſiedlung? Haben Sie den Grund ſchon gekauft? Könnte man ſich nicht
               wieder einmal, in Ruhe, ſehen? Sprechen? Ihre So{\geminationm}erpläne? Wir auf 3-4 Wochen \textcolor{pink}{Reichenau}{}\ledrightnote{\textcolor{pink}{Reichenau an der Rax}}; mehr dürfte
               nicht herausko{\geminationm}en. –\pend
           \pstart
           – Zum \textcolor{green}{\textsc{Charolais}}{}\ledrightnote{\textcolor{green}{Der Graf von Charolais. Ein Trauerspiel}} (nicht gerade zur Aufführung, in der ich nur \textcolor{blue}{\textsc{Kayssler}}{}\ledrightnote{\textcolor{blue}{Friedrich Kayssler}} und \textcolor{blue}{\textsc{Reinhardt}}{}\ledrightnote{\textcolor{blue}{Max Reinhardt}} hervorragend fand, – zunächſt: \textcolor{blue}{\textsc{Hartau}}{}\ledrightnote{\textcolor{blue}{Ludwig Hartau}}) ka{\geminationn} ich Sie immer wied\textcolor{gray}{er} nur beglückwünſchen.
               Gewiſſe Einwendungen bleiben beſtehen; meine Liebe zu dem Werk erhöht und vertieft
                  sich.\pend
           \pstart
           Herzlichst Ihr{\\[\baselineskip]}\spacefill\mbox{A.}\pend
           \leftskip=0em{}\endnumbering\briefempfaengerindex{Beer-Hofmann, Richard@\textsc{Beer-Hofmann, Richard}!zzzSchnitzler, Arthur@\emph{von Arthur Schnitzler}!1905-05-231@{23. 5. 1905}|)be}\mylabel{h}  \normalsize

\doendnotes{C}
\bigskip
\vfill

\clearpage

\footnotesize

\lohead{\textsc{register}}

% Definiere theindex-Environment komplett neu ohne reledmac
\makeatletter
\renewenvironment{theindex}{%
  \section*{\indexname}%
  \setlength{\parindent}{0pt}%
  \setlength{\parskip}{0pt plus 0.3pt}%
  \let\item\@idxitem
}{%
  \clearpage
}
\makeatother

\IfFileExists{\jobname-pw.ind}{\input{\jobname-pw.ind}}{}

\end{document}

      