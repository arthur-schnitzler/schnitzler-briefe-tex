%% latex-korrekturansicht-vorspann.tex
%% Vorspann für die Korrekturansicht.
%% Lädt die gemeinsame Datei latex-vorspann.tex mit gesetztem Schalter.

\newif\ifkorrekturansicht
\korrekturansichttrue

\input{../tex-inputs/latex-vorspann}


\renewcommand{\erwaehntePersonen}{Personen: Otto Brahm, Marie Reinhard, Olga Schnitzler, Heinrich Schnitzler, Christine Schönberger, Elisabeth Steinrück}
\renewcommand{\erwaehnteInstitutionen}{Institutionen: Deutsches Theater Berlin}
\renewcommand{\erwaehnteOrte}{Orte: Berlin, Carl-Theater, Dessauer Straße, Hinterbrühl, Kurhaus Mödling, Mödling, Villa in der Hinterbrühl, Wien, Zum goldenen Stern}
\renewcommand{\erwaehnteWerke}{Werke: Berliner Theater. (»Lebendige Stunden« von Arthur Schnitzler.), Kleine Chronik. [Das Wiener Gastspiel des Berliner Deutschen Theaters.], Lebendige Stunden. Vier Einakter, Neue Freie Presse}
\section[ Paul Goldmann an Arthur Schnitzler, 14. 1. {[}1902{]}]{Paul Goldmann an Arthur Schnitzler, 14. 1. {[}1902{]}}
\nopagebreak\mylabel{v}
\rehead{ }\normalsize\beginnumbering\briefempfaengerindex{Schnitzler, Arthur@\textsc{Schnitzler, Arthur}!zzzGoldmann, Paul@\emph{von Paul Goldmann}!1902-01-141@{14. 1. {[}1902{]}}|(be}
\toendnotes[C]{\smallbreak\pagebreak[2]}\Standort{DLA, A:Schnitzler, HS.NZ85.1.3172.}
\physDesc{Brief, 1 Blatt, 3 Seiten
\newline{}Handschrift: blaue Tinte, deutsche Kurrent
\newline{}Schnitzler: 1) mit Bleistift das Jahr »{[}1{]}902« vermerkt  2) mit rotem Buntstift vier Unterstreichungen}\toendnotes[C]{\smallbreak}
\pstart
           \noindent{}\raggedleft{}{\pb}\textcolor{pink}{\textcolor{gray}{\textbf{DESSAUERSTRASSE 19}}}{}\ledrightnote{\textcolor{pink}{Dessauer Straße}}\pend
           
\pstart
           \textcolor{pink}{Berlin}{}\ledrightnote{\textcolor{pink}{Berlin}}, 14. Januar.\pend
           
\pstart\center{}Mein lieber Freund,\pend
\pstart
           In Eile – denn ich habe unbeſchreiblich viel zu thun – Dank für Deine lieben Briefe.
               Es freut mich, daß es \textsc{\textcolor{blue}{Olga}{}\ledrightnote{\textcolor{blue}{Olga Schnitzler}}} gut geht und daß Ihr demnächſt \label{K_L03192-1v}\edtext{aufs Land ziehen}{\lemma{\textnormal{\emph{aufs Land ziehen}}}\Cendnote{\textnormal{\textcolor{blue}{Olga Gussmann} war erneut schwanger. Auch sie
                  sollte, wie bereits \textcolor{blue}{Marie Reinhard} im Jahre
                     1897, außerhalb \textcolor{pink}{Wien}s gebären. Dafür suchte \textcolor{blue}{Schnitzler} eine geeignete Unterkunft. Am 3. 2. 1902 zogen \textcolor{blue}{Olga} und ihre Schwester \textcolor{blue}{Elisabeth} also vorübergehend in ein \textcolor{pink}{Mödling}er \textcolor{pink}{Kurhaus}, dann zu \textcolor{blue}{Christine Schönberger} in das Wirtshaus \textcolor{pink}{Zum goldenen Stern} (vgl. A. S.: \emph{Tagebuch}, 1. 3. 1902 und Arthur Schnitzler an Richard Beer-Hofmann, 26. 2. 1902). Im März fand \textcolor{blue}{Schnitzler} schließlich
                  eine \textcolor{pink}{Villa} in der \textcolor{pink}{Hinterbrühl} (vgl. A. S.: \emph{Tagebuch}, 21. 3. 1902), wo \textcolor{blue}{Olga} am 9. 8. 1902{ }\textcolor{blue}{Heinrich Schnitzler} zur Welt
                  brachte.}}}\label{K_L03192-1h} wollt. Wird Euch Beiden wohlthun. Mit \textsc{\textcolor{blue}{Liesl}{}\ledrightnote{\textcolor{blue}{Elisabeth Steinrück}}} iſt es ein Krux. Wäre ſie nur ſchon \label{K_L03192-2v}\edtext{fertig}{\lemma{\textnormal{\emph{fertig}}}\Cendnote{\textnormal{vermutlich
                  Bezug auf \textcolor{blue}{Elisabeth Gussmann}s Ausbildung,
                     siehe Paul Goldmann an Arthur Schnitzler, 18. 2. [1901] und A. S.: \emph{Tagebuch}, 11. 1. 1902}}}\label{K_L03192-2h}! Setzt \strikeout{Ihr do} ihr doch einmal ordentlich den
               Kopf zurecht!\pend
           
\pstart
           Daß \label{K_L03192-3v}\edtext{\textsc{\textcolor{blue}{Brahm}{}\ledrightnote{\textcolor{blue}{Otto Brahm}}} nach \textcolor{pink}{Wien}{}\ledrightnote{\textcolor{pink}{Wien}} kommt, \strikeout{will ich} um Deine \textcolor{green}{Stücke}{}\ledrightnote{{$\rightarrow$}\textcolor{green}{Lebendige Stunden. Vier Einakter}} aufzuführen}{\lemma{\textnormal{\emph{Brahm … aufzuführen}}}\Cendnote{\textnormal{Das \emph{\textcolor{brown}{Deutsche Theater Berlin}} gastierte 1902 am \textcolor{pink}{Wien}er \textcolor{pink}{Carl-Theater}. Die Premiere von \emph{\textcolor{green}{Lebendige Stunden}} fand dort am 6. 5. 1902
                  statt.}}}\label{K_L03192-3h}, {\pb}will ich nur \label{K_L03192-4v}\edtext{\textcolor{green}{melden}{}\ledrightnote{{$\rightarrow$}\textcolor{green}{Kleine Chronik. [Das Wiener Gastspiel des Berliner Deutschen Theaters.]}}}{\lemma{\textnormal{\emph{melden}}}\Cendnote{\textnormal{[O. V.] [=\textcolor{blue}{Paul Goldmann}]: \emph{\textcolor{green}{Kleine Chronik. [Das Wiener Gastspiel des
                        Berliner Deutschen Theaters.]}}. In: \emph{\textcolor{green}{Neue
                        Freie Presse}}, Nr. 13433, 17. 1. 1902,
                     Abendblatt, S. 1.}}}\label{K_L03192-4h}, wenn Du meinſt, es könnte für Dich irgendwie
               von Nutzen ſein. Eine »Nachricht« will ich von Dir nicht haben; Du haſt mich \strikeout{\textcolor{gray}{faß}} mißverſtanden. Wenn ich alſo bis Donnerſtag von
               Dir nichts höre, werde ich \strikeout{nach \textcolor{pink}{Wien}{}\ledrightnote{\textcolor{pink}{Wien}}} annehmen, daß es Dir angemeſſen erſcheint, wenn ich die \textcolor{green}{Meldung}{}\ledrightnote{{$\rightarrow$}\textcolor{green}{Kleine Chronik. [Das Wiener Gastspiel des Berliner Deutschen Theaters.]}} nach \textcolor{pink}{Wien}{}\ledrightnote{\textcolor{pink}{Wien}} ſende, und werde ſie abtelegraphiren.\pend
           
\pstart
           Ich hab\textcolor{gray}{e} bereits angefangen, das \label{K_L03192-6v}\edtext{\textcolor{green}{Feuilleton}{}\ledrightnote{{$\rightarrow$}\textcolor{green}{Berliner Theater. (»Lebendige Stunden« von Arthur Schnitzler.)}}}{\lemma{\textnormal{\emph{Feuilleton}}}\Cendnote{\textnormal{\textcolor{blue}{Paul Goldmann}: \emph{\textcolor{green}{Berliner Theater. (»Lebendige Stunden« von Arthur
                        Schnitzler.)}}. In: \emph{\textcolor{green}{Neue Freie
                        Presse}}, Nr. 13438, 22. 1. 1902,
                     Morgenblatt, S. 1–4. \textcolor{blue}{Schnitzler}
                  ärgerte sich über das kritische \textcolor{green}{Feuilleton} (vgl. A. S.: \emph{Tagebuch}, 22. 1. 1902 und 28. 1. 1902), das die Beziehung der beiden über Jahre hinweg – noch
                  bis zum großen Streit 1910 – belasten sollte.}}}\label{K_L03192-6h} über
               Deine \textcolor{green}{Stücke}{}\ledrightnote{{$\rightarrow$}\textcolor{green}{Lebendige Stunden. Vier Einakter}} zu ſchreiben, bin
               aber nicht über die erſten Zeilen herausgekommen. Unabläſſig wird mir die Feder {\pb}aus der Hand geriſſen. Die Arbeit ſelbſt iſt die
               ſchwerſte, die ich je gemacht. Ich muß mich zwingen (und das iſt ein harter Zwang),
               mit eiſiger Kälte zu erwägen und mich auszudrücken und muß mir einreden, daß ich über
               die \textcolor{green}{Stücke}{}\ledrightnote{{$\rightarrow$}\textcolor{green}{Lebendige Stunden. Vier Einakter}} eines mir
               unbekannten Herrn \textsc{Arthur Schnitzler} ſchreibe. Wenn die
               Parlamentsſeſſion ſo weiter geht, – dann weiß Gott, wann ich fertig werde.\pend
           
\pstart
           Grüße mir \textsc{\textcolor{blue}{Olga}{}\ledrightnote{\textcolor{blue}{Olga Schnitzler}}} und ſei ſelbſt von Herzen gegrüßt! {\\[\baselineskip]}Dein \spacefill\mbox{Paul Goldmnn}\pend
           \leftskip=0em{}\endnumbering\briefempfaengerindex{Schnitzler, Arthur@\textsc{Schnitzler, Arthur}!zzzGoldmann, Paul@\emph{von Paul Goldmann}!1902-01-141@{14. 1. {[}1902{]}}|)be}\mylabel{h}
\begin{anhang}
\end{anhang}\normalsize

\doendnotes{C}
\bigskip
\vfill

\clearpage

\footnotesize

\lohead{\textsc{register}}

% Definiere theindex-Environment komplett neu ohne reledmac
\makeatletter
\renewenvironment{theindex}{%
  \section*{\indexname}%
  \setlength{\parindent}{0pt}%
  \setlength{\parskip}{0pt plus 0.3pt}%
  \let\item\@idxitem
}{%
  \clearpage
}
\makeatother

\IfFileExists{\jobname-pw.ind}{\input{\jobname-pw.ind}}{}

\end{document}

      