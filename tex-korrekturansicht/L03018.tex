%% latex-korrekturansicht-vorspann.tex
%% Vorspann für die Korrekturansicht.
%% Lädt die gemeinsame Datei latex-vorspann.tex mit gesetztem Schalter.

\newif\ifkorrekturansicht
\korrekturansichttrue

\input{../tex-inputs/latex-vorspann}


\renewcommand{\erwaehntePersonen}{Personen:  ?? [Arzt von Elisabeth Steinrück], Samuel Fischer, Hedwig Fischer, Felix Salten, Olga Schnitzler, Elisabeth Steinrück}
\renewcommand{\erwaehnteOrte}{Orte: Bad Ischl, Edmund-Weiß-Gasse 7, Partenkirchen, Salzkammergut, Sternwartestraße 71, Unterach am Attersee, Wien}
\renewcommand{\erwaehnteWerke}{Werke: Olga Frohgemuth. Erzählung}
\section[ Arthur Schnitzler an Felix Salten, 8. 8. 191{[}0{]}]{Arthur Schnitzler an Felix Salten, 8. 8. 191{[}0{]}}
\nopagebreak\mylabel{v}
\rehead{ }\normalsize\beginnumbering\briefempfaengerindex{Salten, Felix@\textsc{Salten, Felix}!zzzSchnitzler, Arthur@\emph{von Arthur Schnitzler}!1910-08-081@{8. 8. 191{[}0{]}}|(be}
\toendnotes[C]{\smallbreak\pagebreak[2]}\Standort{Wienbibliothek im Rathaus, ZPH 1681, 2.1.516.}
\physDesc{Brief, 1 Blatt, 3 Seiten, 864 Zeichen
\newline{}Handschrift: Bleistift, deutsche Kurrent
\newline{}Ordnung: mit Bleistift von unbekannter Hand Nummerierung der Doppelseiten des
                                 Konvoluts: »6«–»7« }\toendnotes[C]{\smallbreak}
\pstart
           \noindent{}{\pb}\textcolor{gray}{\textbf{Dr. Arthur Schnitzler}}\hfill \label{K_L03018-1v}\edtext{8. 8. 1911}{\lemma{\textnormal{\emph{8. 8. 1911}}}\Cendnote{\textnormal{\textcolor{blue}{Schnitzler}s Datierung im Jahr 1911 ist falsch. Mindestens vier Argumente lassen sich
                        finden: die handschriftliche Angabe der neuen Adresse neben dem gedruckten
                        Briefkopf mit der alten Adresse (vgl. Arthur Schnitzler an Hugo von Hofmannsthal, 30. 7. 1910); der Brief bezieht sich auf ein Glückwünschtelegramm,
                        womit wohl jenes zum Einzug in der \textcolor{pink}{Sternwartestraße 71} gemeint ist (Ottilie und Felix Salten an Arthur und Olga
               Schnitzler, [24. 7. 1910]); die inhaltliche Übereinstimmung mit dem
                        (Antwort-)Brief \textcolor{blue}{Salten}s (Felix Salten an Arthur Schnitzler, 17. 8. 1910), worin auch auf die Anwesenheit von \textcolor{blue}{Samuel} und \textcolor{blue}{Hedwig
                                 Fischer} in \textcolor{pink}{Unterach} eingegangen wird; die Erwähung von \textcolor{blue}{Elisabeth Steinrück}s
                        Rippenfellentzündung (vgl. A. S.: \emph{Tagebuch}, 2. 8. 1910). Die Bezugnahme
                        auf \emph{\textcolor{green}{Olga Frohgemuth}} weist zudem auf die
                        bevorstehende Buchpublikation (vgl. Felix Salten: Widmungsexemplar Olga Frohgemuth für Olga und Arthur
               Schnitzler, 26. 9. 1910).}}}\label{K_L03018-1h}\pend
           
\pstart
           \textcolor{gray}{\textbf{\textcolor{pink}{Wien XVIII. Spoettelgasse 7}{}\ledrightnote{\textcolor{pink}{Edmund-Weiß-Gasse 7}}.}}\hfill \textsc{\textcolor{pink}{XVIII. Sternwartestr 71}{}\ledrightnote{\textcolor{pink}{Sternwartestraße 71}}}\pend
           
\pstart
           lieber,{ }\textcolor{blue}{wir}{}\ledrightnote{{$\rightarrow$}\textcolor{blue}{Olga Schnitzler}} danken herzlich für das
               liebe Glückwunſchtelegramm. Nun ſind wir in leidlicher Ordnung; und dieſer Tage
                  \label{K_L03018-2v}\edtext{fahren wir nach \textcolor{pink}{Partenkirchen}{}\ledrightnote{\textcolor{pink}{Partenkirchen}}}{\lemma{\textnormal{\emph{fahren … Partenkirchen}}}\Cendnote{\textnormal{\textcolor{blue}{Schnitzler} war zwischen 20. 8. 1910 und 26. 8. 1910 in \textcolor{pink}{Partenkirchen}.}}}\label{K_L03018-2h}, wo \textsc{\textcolor{blue}{Liesl}{}\ledrightnote{\textcolor{blue}{Elisabeth Steinrück}}} an einer Rippenfellentzündg erkrankt liegt. Wir waren ſchon vor 3 Tagen daran
               hinzufahren, {\pb}da bat uns der \textcolor{blue}{Arzt}{}\ledrightnote{{$\rightarrow$}\textcolor{blue}{?? [Arzt von Elisabeth Steinrück]}} telegraphiſch die Reiſe
               aufzuſchieben, da unſer Erſcheinen bei dem augenblicklich\textcolor{gray}{en}
               Zuſtand der Kranken einen nicht ungefährlichen \textsc{Chok}
               bedeuten müßte. Nun ſcheint es etwas beſſer zu gehen. Ob wir von \textcolor{pink}{P.}{}\ledrightnote{\textcolor{pink}{Partenkirchen}} aus noch \label{K_L03018-3v}\edtext{ins
                  \textsc{\textcolor{pink}{Salzkgut}{}\ledrightnote{\textcolor{pink}{Salzkammergut}}} gelangen}{\lemma{\textnormal{\emph{ins
                  Salzkgut gelangen}}}\Cendnote{\textnormal{Zwischen 29. 8. 1910 und 5. 9. 1910 war \textcolor{blue}{Schnitzler} in \textcolor{pink}{Bad Ischl}.}}}\label{K_L03018-3h}, wie es unſere Abſicht war, läßt ſich heute noch nicht voraus{\pb}ſehen; wollen Sie mir gelegentlich ſagen, wie
               lange Sie un\textcolor{gray}{d} wie lange \textsc{\textcolor{blue}{Fischers}{}\ledrightnote{\textcolor{blue}{Samuel Fischer}{\newline}\textcolor{blue}{Hedwig Fischer}}} noch in \textsc{\textcolor{pink}{Unterach}{}\ledrightnote{\textcolor{pink}{Unterach am Attersee}}} bleiben?\pend
           
\pstart
           Ihren Nachrichten und dem weiteren Schickſale Ihres reizumfloſſenen \textcolor{green}{Frohgemuth}{}\ledrightnote{\textcolor{green}{Olga Frohgemuth. Erzählung}} ſeh ich mit Spa{\geminationn}ung entgegen und hoffe Sie ſind alle wohl u vergnügt.
               Herzlichſt mit Grüßen von uns Allen {\\[\baselineskip]}Ihr {\\[\baselineskip]}\spacefill\mbox{A.}\pend
           \leftskip=0em{}\endnumbering\briefempfaengerindex{Salten, Felix@\textsc{Salten, Felix}!zzzSchnitzler, Arthur@\emph{von Arthur Schnitzler}!1910-08-081@{8. 8. 191{[}0{]}}|)be}\mylabel{h}  \normalsize

\doendnotes{C}
\bigskip
\vfill

\clearpage

\footnotesize

\lohead{\textsc{register}}

% Definiere theindex-Environment komplett neu ohne reledmac
\makeatletter
\renewenvironment{theindex}{%
  \section*{\indexname}%
  \setlength{\parindent}{0pt}%
  \setlength{\parskip}{0pt plus 0.3pt}%
  \let\item\@idxitem
}{%
  \clearpage
}
\makeatother

\IfFileExists{\jobname-pw.ind}{\input{\jobname-pw.ind}}{}

\end{document}

      