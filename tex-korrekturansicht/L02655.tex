%% latex-korrekturansicht-vorspann.tex
%% Vorspann für die Korrekturansicht.
%% Lädt die gemeinsame Datei latex-vorspann.tex mit gesetztem Schalter.

\newif\ifkorrekturansicht
\korrekturansichttrue

\input{../tex-inputs/latex-vorspann}


               \section[Paul Goldmann an Arthur Schnitzler, {[}20. 6. 1901{]}]{ Paul Goldmann an Arthur Schnitzler, {[}20. 6. 1901{]}}\nopagebreak\mylabel{v}\rehead{ }\normalsize\beginnumbering\briefempfaengerindex{Schnitzler, Arthur@\textsc{Schnitzler, Arthur}!zzzGoldmann, Paul@\emph{von Paul Goldmann}!1901-06-201@{{[}20. 6. 1901{]}}|(be} \toendnotes[C]{\smallbreak\pagebreak[2]} \Standort{DLA, A:Schnitzler, HS.NZ85.1.3171.}
\physDesc{Telegramm
\newline{}maschinell
\newline{}Schnitzler: mit Bleistift datiert: »20/6 901« \newline{}Ordnung: beschnitten }\toendnotes[C]{\smallbreak}\pstart
           {\pb}de \textcolor{pink}{berlin}{}\ledrightnote{\textcolor{pink}{Berlin}}
                  98611i19 21:21, 35s=\pend
           \pstart
           mein lieber \label{K_L02655-1v}\edtext{ausgestossener}{\lemma{\textnormal{\emph{ausgestossener}}}\Cendnote{\textnormal{Von einem »militärischen Ehrenrat« wurde
                     \textcolor{blue}{Schnitzler}, der als »Regimentsarzt in der
                  Reserve« beim Heer gemeldet war, der damit einhergehende Rang als Offizier
                  aberkannt. Anlass war die Veröffentlichung von \emph{\textcolor{green}{Lieutenant Gustl}}, die als Verspottung des Heeres empfunden wurde.}}}\label{K_L02655-1h},
               ich druecke dir treu und herzlich die hand = \spacefill\mbox{paul goldmann .+}\pend
           \endnumbering\briefempfaengerindex{Schnitzler, Arthur@\textsc{Schnitzler, Arthur}!zzzGoldmann, Paul@\emph{von Paul Goldmann}!1901-06-201@{{[}20. 6. 1901{]}}|)be}\mylabel{h}  \normalsize

\doendnotes{C}
\bigskip
\vfill

\clearpage

\footnotesize

\lohead{\textsc{register}}

% Definiere theindex-Environment komplett neu ohne reledmac
\makeatletter
\renewenvironment{theindex}{%
  \section*{\indexname}%
  \setlength{\parindent}{0pt}%
  \setlength{\parskip}{0pt plus 0.3pt}%
  \let\item\@idxitem
}{%
  \clearpage
}
\makeatother

\IfFileExists{\jobname-pw.ind}{\input{\jobname-pw.ind}}{}

\end{document}

      