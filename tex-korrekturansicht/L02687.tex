%% latex-korrekturansicht-vorspann.tex
%% Vorspann für die Korrekturansicht.
%% Lädt die gemeinsame Datei latex-vorspann.tex mit gesetztem Schalter.

\newif\ifkorrekturansicht
\korrekturansichttrue

\input{../tex-inputs/latex-vorspann}


               \section[Arthur Schnitzler an Paul Goldmann, 25. 4. 1927]{ Arthur Schnitzler an Paul Goldmann, 25. 4. 1927}\nopagebreak\mylabel{v}\rehead{ }\normalsize\beginnumbering\briefempfaengerindex{Goldmann, Paul@\textsc{Goldmann, Paul}!zzzSchnitzler, Arthur@\emph{von Arthur Schnitzler}!1927-04-251@{25. 4. 1927}|(be} \toendnotes[C]{\smallbreak\pagebreak[2]} \Standort{DLA, A:Schnitzler, HS85.1.5681.}
\physDesc{Bildpostkarte, Fotokopie
\newline{}Handschrift: Bleistift, lateinische Kurrent\newline{}Versand: Stempel: »\nobreak{}\oindex{Bahnhof@\textbf{Bahnhof}, \emph{Bahnhofsgebäude (K.BHF)}|pwk}Venezia Ferrovia, 25. IV 1927, 22–23\nobreak{}«.  \newline{}Zusatz: Von den Korrespondenzstücken \textcolor{blue}{Schnitzler}s an \textcolor{blue}{Goldmann} fehlt weitgehend jede Spur. In der Edition von
                                    \textcolor{green}{Ritterlichkeit}
                                    (1975) schreibt die Herausgeberin \textcolor{blue}{Rena R. Schlein}: »Zwei Telegramme
                                    und ein Brief \textcolor{blue}{Schnitzler}s
                                    an \textcolor{blue}{Goldmann} wurden mir
                                    von Dr. \textcolor{blue}{Leo P. Reckford},
                                    der diese Dokumente von der Familie \textcolor{blue}{Goldmann}s zum Geschenk bekam, für meine
                                    Arbeit zur Verfügung gestellt« (S. 1). \textcolor{blue}{Reckford} starb 1988, seine
                                 Nachkommen haben keine Kenntnis von diesen (und etwaigen weiteren)
                                 Korrespondenzstücken und sie sind auch nicht auffindbar. \textcolor{blue}{Rena R. Schlein} wäre, wenn
                                 sie noch leben sollte, deutlich über 100 Jahre alt. Ein Kontakt
                                 konnte nicht hergestellt werden. Die vorliegende
                                 Schwarz-Weiß-Fotokopie wird im Nachlass \textcolor{blue}{Schnitzler}s zusammen mit Kopien zwei der
                                 drei in \textcolor{green}{Ritterlichkeit}
                                 abgedruckten Korrespondenzstücken aufbewahrt, was darauf hindeutet,
                                 dass auch diese Postkarte zu einem bestimmten Zeitpunkt im Besitz
                                    \textcolor{blue}{Reckford}s gewesen
                                 ist. }\toendnotes[C]{\smallbreak}\pstart{}{\pb}\label{T_L02404-1v}\edtext{\textcolor{gray}{\textbf{A. S.}}}{\lemma{\textnormal{\emph{A. S.}}}\Cendnote{\textnormal{ovaler Absenderkleber}}}\label{T_L02404-1h}\pend{}\pstart{}\textcolor{pink}{\textcolor{gray}{\textbf{WIEN, XVIII.}}}{}\ledrightnote{\textcolor{pink}{XVIII., Währing}}\pend{}\pstart{}\textcolor{pink}{\textcolor{gray}{\textbf{STERNWARTESTR. 71}}}{}\ledrightnote{\textcolor{pink}{Sternwartestraße}}\pend{}{\bigskip}\pstart{}{\pb}\textcolor{pink}{Germania}{}\ledrightnote{\textcolor{pink}{Deutschland}}\pend{}\pstart{}Hn Dr Paul Goldmann\pend{}\pstart{}\textcolor{pink}{Berlin W}{}\ledrightnote{\textcolor{pink}{Berlin}}\pend{}\pstart{}\textcolor{pink}{Bendlerstr 36}{}\ledrightnote{\textcolor{pink}{Stauffenbergstraße}}\pend{}{\bigskip}\pstart
           \noindent{}\centering{}{\pb}\textcolor{gray}{\textbf{\textcolor{pink}{VENEZIA}{}\ledrightnote{\textcolor{pink}{Venedig}} – \textcolor{pink}{Piazzetta S. Marco}{}\ledrightnote{\textcolor{pink}{San Marco}} dalla
                     Laguna.}}\pend
           \pstart
           \raggedleft{}{\pb}\textcolor{pink}{Venedg}{}\ledrightnote{\textcolor{pink}{Venedig}}{ }25/4\pend
           \pstart
           mein lieber Paul, ich bedaure sehr Euern Besuch versäumt zu haben,
               und grüße Dich, die mir verehrte \textcolor{blue}{Gattin}{}\ledrightnote{→\textcolor{blue}{Eva Marie Goldmann}} und die liebe \textcolor{blue}{Tochter}{}\ledrightnote{→\textcolor{blue}{Franziska Goldmann}} aufs herzlichste.\pend
           \pstart
           Auf ein gutes Wiedersehen, sei’s in \textcolor{pink}{Berlin}{}\ledrightnote{\textcolor{pink}{Berlin}}, in
                  \textcolor{pink}{Wien}{}\ledrightnote{\textcolor{pink}{Wien}} oder vielleicht einmal im \label{K_L02687-1v}\edtext{So\damage{\textcolor{gray}{mmer}}}{\lemma{\textnormal{\emph{Sommer}}}\Cendnote{\textnormal{\textcolor{blue}{Goldmann} und \textcolor{blue}{Schnitzler} sahen sich erst am 7. 10. 1927 wieder.}}}\label{K_L02687-1h}?\pend
           \pstart
           Ich dürfte bis Anfang August zu Hause bleiben.\pend
           \pstart Dein \spacefill\mbox{Arthur}\pend{}\endnumbering\briefempfaengerindex{Goldmann, Paul@\textsc{Goldmann, Paul}!zzzSchnitzler, Arthur@\emph{von Arthur Schnitzler}!1927-04-251@{25. 4. 1927}|)be}\mylabel{h}\begin{anhang}\end{anhang}\normalsize

\doendnotes{C}
\bigskip
\vfill

\clearpage

\footnotesize

\lohead{\textsc{register}}

% Definiere theindex-Environment komplett neu ohne reledmac
\makeatletter
\renewenvironment{theindex}{%
  \section*{\indexname}%
  \setlength{\parindent}{0pt}%
  \setlength{\parskip}{0pt plus 0.3pt}%
  \let\item\@idxitem
}{%
  \clearpage
}
\makeatother

\IfFileExists{\jobname-pw.ind}{\input{\jobname-pw.ind}}{}

\end{document}

      