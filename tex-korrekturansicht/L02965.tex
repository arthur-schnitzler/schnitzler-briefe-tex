%% latex-korrekturansicht-vorspann.tex
%% Vorspann für die Korrekturansicht.
%% Lädt die gemeinsame Datei latex-vorspann.tex mit gesetztem Schalter.

\newif\ifkorrekturansicht
\korrekturansichttrue

\input{../tex-inputs/latex-vorspann}


\renewcommand{\erwaehntePersonen}{Personen:  ?? [Totgeborener Sohn von Arthur Schnitzler und Marie Reinhard], Marie Reinhard, Felix Salten}
\renewcommand{\erwaehnteOrte}{Orte: Wien}
\renewcommand{\erwaehnteWerke}{}
\section[ Arthur Schnitzler an Felix Salten, 2{[}5.?{]} 9. 1897]{Arthur Schnitzler an Felix Salten, 2{[}5.?{]} 9. 1897}
\nopagebreak\mylabel{v}
\rehead{ }\normalsize\beginnumbering\briefempfaengerindex{Salten, Felix@\textsc{Salten, Felix}!zzzSchnitzler, Arthur@\emph{von Arthur Schnitzler}!1897-09-252@{2{[}5.?{]} 9. 1897}|(be}
\toendnotes[C]{\smallbreak\pagebreak[2]}\Standort{Wienbibliothek im Rathaus, ZPH 1681, 2.1.516.}
\physDesc{Brief, 1 Blatt, 1 Seite, 145 Zeichen
\newline{}Handschrift: Bleistift, deutsche Kurrent
\newline{}Ordnung: mit Bleistift von unbekannter Hand Nummerierung der Blätter des Konvoluts:
                                    »5« }\toendnotes[C]{\smallbreak}
\pstart
           \noindent{}{\pb}Lieber Freund,\pend
           
\pstart
           ich dachte Sie ko{\geminationm}en heute vielleicht zu mir. Nun ſage ich Ihnen ſchriftlich das traurige was
               zu ſagen iſt. Das \label{K_L02965-1v}\edtext{\textcolor{blue}{Kind}{}\ledrightnote{{$\rightarrow$}\textcolor{blue}{?? [Totgeborener Sohn von Arthur Schnitzler und Marie Reinhard]}}}{\lemma{\textnormal{\emph{Kind}}}\Cendnote{\textnormal{Das \textcolor{blue}{Kind} von \textcolor{blue}{Schnitzler} und \textcolor{blue}{Marie Reinhard} war am 24. 9. 1897 tot
                  geboren worden.}}}\label{K_L02965-1h} iſt todt.\pend
           \pstart Ihr \spacefill\mbox{Arthur}\pend{}\endnumbering\briefempfaengerindex{Salten, Felix@\textsc{Salten, Felix}!zzzSchnitzler, Arthur@\emph{von Arthur Schnitzler}!1897-09-252@{2{[}5.?{]} 9. 1897}|)be}\mylabel{h}  \normalsize

\doendnotes{C}
\bigskip
\vfill

\clearpage

\footnotesize

\lohead{\textsc{register}}

% Definiere theindex-Environment komplett neu ohne reledmac
\makeatletter
\renewenvironment{theindex}{%
  \section*{\indexname}%
  \setlength{\parindent}{0pt}%
  \setlength{\parskip}{0pt plus 0.3pt}%
  \let\item\@idxitem
}{%
  \clearpage
}
\makeatother

\IfFileExists{\jobname-pw.ind}{\input{\jobname-pw.ind}}{}

\end{document}

      