%% latex-korrekturansicht-vorspann.tex
%% Vorspann für die Korrekturansicht.
%% Lädt die gemeinsame Datei latex-vorspann.tex mit gesetztem Schalter.

\newif\ifkorrekturansicht
\korrekturansichttrue

\input{../tex-inputs/latex-vorspann}


\renewcommand{\erwaehntePersonen}{Personen: Heinrich Kanner, Isidor Singer}
\renewcommand{\erwaehnteInstitutionen}{Institutionen: Die Zeit, Die Zeit. Wiener Wochenschrift}
\renewcommand{\erwaehnteOrte}{Orte: Burgtheater, Wien}
\renewcommand{\erwaehnteWerke}{Werke: Andreas Thameyers letzter Brief, Dämmerseele, Neue Freie Presse}
\section[ Felix Salten an Arthur Schnitzler, 12. 3. 1902]{Felix Salten an Arthur Schnitzler, 12. 3. 1902}
\nopagebreak\mylabel{v}
\rehead{ }\normalsize\beginnumbering\briefempfaengerindex{Schnitzler, Arthur@\textsc{Schnitzler, Arthur}!zzzSalten, Felix@\emph{von Felix Salten}!1902-03-121@{12. 3. 1902}|(be}
\toendnotes[C]{\smallbreak\pagebreak[2]}\Standort{CUL, Schnitzler, B 89, A 2.}
\physDesc{Brief, 1 Blatt, 2 Seiten, 819 Zeichen
\newline{}Handschrift: blaue Tinte, lateinische Kurrent
\newline{}Ordnung: mit Bleistift von unbekannter Hand nummeriert: »150« }\toendnotes[C]{\smallbreak}
\pstart
           \raggedleft{}{\pb}den 12. März 02.\pend
           
\pstart
           Lieber – mit der \label{K_L03326-1v}\edtext{»\textcolor{brown}{Zeit}{}\ledrightnote{\textcolor{brown}{Die Zeit}}«}{\lemma{\textnormal{\emph{»Zeit«}}}\Cendnote{\textnormal{Die Herausgeber \textcolor{blue}{Heinrich Kanner} und \textcolor{blue}{Isidor Singer} planten, die Wochenschrift mit
                  diesem Titel um eine gleichnamige Tageszeitung zu erweitern. Diese erschien ab
                     27. 9. 1902. Bis dahin verfasste \textcolor{blue}{Salten} noch unter dem Pseudonym »Martin Finder« Beiträge für die \textcolor{brown}{Wochenschrift}. Im Hinblick
                  auf \textcolor{blue}{Schnitzler} könnte sich \textcolor{blue}{Salten} auf eine mögliche Publikation bezogen haben. Am
                     26. 7. 1902 erschien in der Zeit \emph{\textcolor{green}{Andreas Thameyers letzter Brief}} (Jg. 32, Nr. 408,
                     S. 63–64).}}}\label{K_L03326-1h} bin ich noch lange nicht fertig, und in ernsten
               Verhandlungen eben wegen der Feuilletonredaction. Diese Unterhandlungen werden
               voraussichtlich, – da sie ein negatives Resultat während der ersten Unterredungen
               nicht hatten – bis gegen Ende April dauern, und läßt
               sich heute trotz alledem ihr Ausgang nicht einmal annähernd voraussagen. Sollte aber
               irgend ein Ergebnis früher eintreten, dann theile ich es Ihnen gewiss sogleich mit.
               Im Übrigen – ich brauche das wol nicht zu sagen – soll diese Mittheilung Sie in
               keiner Weise \label{K_L03326-2v}\edtext{beeinflußen}{\lemma{\textnormal{\emph{beeinflußen}}}\Cendnote{\textnormal{Am 6. 3. 1902
                  schrieb \textcolor{blue}{Schnitzler} an einer ersten Fassung von 
                  \emph{\textcolor{green}{Dämmerseele}}. 
                  Möglicherweise pberlegte er, \textcolor{blue}{Salten} die Publikation anzuvertrauen, wenn dieser bereits in
                  einem festen Verhältnis mit der \emph{\textcolor{brown}{Zeit}} gestanden
                  wäre. So erschien der Text am 18. 5. 1902 in der \emph{\textcolor{green}{Neuen Freien
                     Presse}} (\textcolor{blue}{Arthur Schnitzler}: \emph{\textcolor{green}{Dämmerseele}}. In: \emph{\textcolor{green}{Neue
                        Freie Presse}}, Nr. 13.553, 18. 5. 1902,
                     Morgenblatt, Pfingstbeilage, S. 31–33 ). }}}\label{K_L03326-2h}.\pend
           
\pstart
           Ich bin seit heute außer Bett, gehe morgen ins \textcolor{pink}{Burgtheater}{}\ledrightnote{\textcolor{pink}{Burgtheater}}
               und möchte Sie jedenfalls bald gerne sprechen. Kann aber Abends nicht
               ausgehen. Vielleicht entschließen Sie sich, dieser Tage nach {\pb}dem \label{K_L03326-3v}\edtext{Nachtmahl zu
               mir zu kommen? Samstag? od. Freitag}{\lemma{\textnormal{\emph{Nachtmahl … Freitag}}}\Cendnote{\textnormal{\textcolor{blue}{Schnitzler} kam am Freitag, dem 14. 3. 1902.}}}\label{K_L03326-3h}?\pend
           
\pstart
           herzlichst {\\[\baselineskip]}Ihr {\\[\baselineskip]}\spacefill\mbox{Salten}\pend
           \leftskip=0em{}\endnumbering\briefempfaengerindex{Schnitzler, Arthur@\textsc{Schnitzler, Arthur}!zzzSalten, Felix@\emph{von Felix Salten}!1902-03-121@{12. 3. 1902}|)be}\mylabel{h}  \normalsize

\doendnotes{C}
\bigskip
\vfill

\clearpage

\footnotesize

\lohead{\textsc{register}}

% Definiere theindex-Environment komplett neu ohne reledmac
\makeatletter
\renewenvironment{theindex}{%
  \section*{\indexname}%
  \setlength{\parindent}{0pt}%
  \setlength{\parskip}{0pt plus 0.3pt}%
  \let\item\@idxitem
}{%
  \clearpage
}
\makeatother

\IfFileExists{\jobname-pw.ind}{\input{\jobname-pw.ind}}{}

\end{document}

      