%% latex-korrekturansicht-vorspann.tex
%% Vorspann für die Korrekturansicht.
%% Lädt die gemeinsame Datei latex-vorspann.tex mit gesetztem Schalter.

\newif\ifkorrekturansicht
\korrekturansichttrue

\input{../tex-inputs/latex-vorspann}


\section[Frank Wedekind: Widmungsexemplar von Franziska für Arthur Schnitzler, {[}zwischen 1. und. 5.{]} 12. 1911]{L04012 Frank Wedekind: Widmungsexemplar von Franziska für Arthur Schnitzler, {[}zwischen 1. und. 5.{]} 12. 1911}
\nopagebreak\mylabel{L04012v}
\rehead{ }\normalsize\beginnumbering\briefempfaengerindex{Schnitzler, Arthur@\textsc{Schnitzler, Arthur}!zzzWedekind, Frank@\emph{von Frank Wedekind}!1911-12-051@{{[}zwischen 1. und. 5.{]} 12. 1911}|(be}
\toendnotes[C]{\smallbreak\pagebreak[2]}
\correspDesc{Versand  durch Frank Wedekind im Zeitraum [zwischen 1.
                  und. 5.] 12. 1911 in München
\newline{}Erhalt  durch Arthur Schnitzler im Zeitraum [zwischen 2.
                  und. 6.] 12. 1911 in Wien}\toendnotes[C]{\smallbreak}
\buchAlsQuelle{\emph{J. A. Stargardt: Katalog 574}. (1965).}
\buchAbdrucke{\weitereDrucke{\emph{Frank Wedekinds Korrespondenz digital}. (7. 10. 2024) \url{https://briefedition.wedekind.h-da.de/view/document/single.xhtml?contentType=1&documentId=1634}.} }\toendnotes[C]{\smallbreak}
\pstart
           \noindent{}{\pb}An Arthur \so{Schnitzler} mit herzlichem \label{K_L04011-1v}\edtext{Dank}{\lemma{\textnormal{\emph{Dank}}}\Cendnote{\textnormal{Das Widmungsexemplar ist nur durch einen Auktionskatalog
                  überliefert. Der Verbleib unbekannt.}}}\label{K_L04011-1} für das prachtvolle \textcolor{green}{Weite
                  Land}\pwindex{Schnitzler, Arthur 15. 5. 1862 Wien – 21. 10. 1931 ebd.@\textsc{Schnitzler, Arthur} (15. 5. 1862 Wien – 21. 10. 1931 ebd.), \emph{Schriftsteller, Mediziner}!weite Land. Tragikomödie in fünf Akten@\strich\emph{Das weite Land. Tragikomödie in fünf Akten}|pw}{}\ledrightnote{\textcolor{green}{Das weite Land. Tragikomödie in fünf Akten}}\pend
           \pstart \spacefill\mbox{Frank Wedekind}\pend{}
\pstart
           \textcolor{pink}{München}\oindex{München@\textbf{München}|pw}{}\ledrightnote{\textcolor{pink}{München}} im \label{K_L04011-2v}\edtext{Dezember 1911}{\lemma{\textnormal{\emph{Dezember 1911}}}\Cendnote{\textnormal{\textcolor{blue}{Schnitzler}
                        vermerkt die (wenig begeisterte) Lektüre für den 6. 12. 1911 in
                        seinem \emph{\textcolor{green}{Tagebuch}\pwindex{Schnitzler, Arthur 15. 5. 1862 Wien – 21. 10. 1931 ebd.@\textsc{Schnitzler, Arthur} (15. 5. 1862 Wien – 21. 10. 1931 ebd.), \emph{Schriftsteller, Mediziner}!Tagebuch@\strich\emph{Tagebuch}|pwk}}.}}}\label{K_L04011-2}.\pend
           \selectlanguage{ngerman}\vspace{1em}
\pstart
           \noindent{}\centering{}{\pb}\textcolor{gray}{\textbf{\textcolor{green}{Franziska}\pwindex{Wedekind, Frank 24.\,7.\,1864 Hannover – 9.\,3.\,1918 München@\textsc{Wedekind, Frank} (24.\,7.\,1864 Hannover – 9.\,3.\,1918 München), \emph{Schriftsteller, Schauspieler, Schriftsteller}!Franziska. Ein Modernes Mysterium in fünf Akten@\strich\emph{Franziska. Ein Modernes Mysterium in fünf Akten}|pw}{}\ledrightnote{\textcolor{green}{Franziska. Ein Modernes Mysterium in fünf Akten}}}}\pend
           
\pstart
           \centering{}\textcolor{gray}{\textbf{\textcolor{green}{Ein modernes
                     Myſterium}\pwindex{Wedekind, Frank 24.\,7.\,1864 Hannover – 9.\,3.\,1918 München@\textsc{Wedekind, Frank} (24.\,7.\,1864 Hannover – 9.\,3.\,1918 München), \emph{Schriftsteller, Schauspieler, Schriftsteller}!Franziska. Ein Modernes Mysterium in fünf Akten@\strich\emph{Franziska. Ein Modernes Mysterium in fünf Akten}|pw}{}\ledrightnote{\textcolor{green}{Franziska. Ein Modernes Mysterium in fünf Akten}}}}\pend
           
\pstart
           \centering{}\textcolor{gray}{\textbf{\textcolor{green}{in fünf
                  Akten}\pwindex{Wedekind, Frank 24.\,7.\,1864 Hannover – 9.\,3.\,1918 München@\textsc{Wedekind, Frank} (24.\,7.\,1864 Hannover – 9.\,3.\,1918 München), \emph{Schriftsteller, Schauspieler, Schriftsteller}!Franziska. Ein Modernes Mysterium in fünf Akten@\strich\emph{Franziska. Ein Modernes Mysterium in fünf Akten}|pw}{}\ledrightnote{\textcolor{green}{Franziska. Ein Modernes Mysterium in fünf Akten}}}}\pend
           
\pstart
           \centering{}\textcolor{gray}{\textbf{von}}\pend
           
\pstart
           \centering{}\textcolor{gray}{\textbf{Frank Wedekind}}\pend
           {\vspace{1\baselineskip}}
\pstart
           \centering{}\textcolor{gray}{\textbf{\textcolor{brown}{\so{Verlag von Georg Müller}}\orgindex{Georg Müller Verlag@Georg Müller Verlag|pw}{}\ledrightnote{\textcolor{brown}{Georg Müller Verlag}}}}\pend
           
\pstart
           \centering{}\textcolor{gray}{\textbf{\textcolor{pink}{München}\oindex{München@\textbf{München}|pw}{}\ledrightnote{\textcolor{pink}{München}}{ }1913}}\pend
           \selectlanguage{ngerman}\endnumbering\briefempfaengerindex{Schnitzler, Arthur@\textsc{Schnitzler, Arthur}!zzzWedekind, Frank@\emph{von Frank Wedekind}!1911-12-011@{{[}zwischen 1. und. 5.{]} 12. 1911}|)be}\mylabel{L04012h}  \normalsize

\doendnotes{C}
\bigskip
\vfill

\clearpage

\footnotesize

\lohead{\textsc{register}}

% Definiere theindex-Environment komplett neu ohne reledmac
\makeatletter
\renewenvironment{theindex}{%
  \section*{\indexname}%
  \setlength{\parindent}{0pt}%
  \setlength{\parskip}{0pt plus 0.3pt}%
  \let\item\@idxitem
}{%
  \clearpage
}
\makeatother

\IfFileExists{\jobname-pw.ind}{\input{\jobname-pw.ind}}{}

\end{document}

      