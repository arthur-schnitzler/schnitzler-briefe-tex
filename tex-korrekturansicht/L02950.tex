%% latex-korrekturansicht-vorspann.tex
%% Vorspann für die Korrekturansicht.
%% Lädt die gemeinsame Datei latex-vorspann.tex mit gesetztem Schalter.

\newif\ifkorrekturansicht
\korrekturansichttrue

\input{../tex-inputs/latex-vorspann}


\renewcommand{\erwaehntePersonen}{Personen: Felix Salten}
\renewcommand{\erwaehnteOrte}{Orte: Wien}
\renewcommand{\erwaehnteWerke}{Werke: Jahrbuch Paul Zsolnay Verlag}
\section[Arthur Schnitzler an Felix Salten, 29. 7. 1929]{Arthur Schnitzler an Felix Salten, 29. 7. 1929}
\nopagebreak\mylabel{v}
\rehead{ }\normalsize\beginnumbering\briefempfaengerindex{Salten, Felix@\textsc{Salten, Felix}!zzzSchnitzler, Arthur@\emph{von Arthur Schnitzler}!1929-07-291@{29. 7. 1929}|(be}
\toendnotes[C]{\smallbreak\pagebreak[2]}\Standort{Wienbibliothek im Rathaus, ZPH 1681/19, 4.1.2.14.}
\physDesc{Brief, 3 Blätter, 3 Seiten, 2362 Zeichen
\newline{}Schreibmaschine
\newline{}Handschrift: 1) schwarze Tinte (\noindent{}Schlussformel und Unterschrift)\hspace{1em}2) Bleistift, lateinische Kurrent (\noindent{}Korrekturen mit Bleistift)\hspace{1em}
\newline{}Ordnung: 1) mit Bleistift von unbekannter Hand in Blockbuchstaben über dem Text Vermerk:
                                    »Arthur Schnitzler«  2) mit Bleistift von unbekannter Hand in lateinischer Kurrentschrift seitlich
                                 neben der Unterschrift Vermerk: »\noindent{}NB: bleibt!{ / }in normaler Schrift, nicht gesperrt« 3) mit Bleistift von unbekannter Hand in deutscher Kurrentschrift unterhalb der
                                 Unterschrift Vermerk: »\noindent{}Arthur Schnitzler«}\Standort{DLA, A:Schnitzler, XXXX.}
\physDesc{Brief, 3 Blätter, 3 Seiten, 2362 Zeichen
\newline{}maschinell
\newline{}Handschrift: Bleistift (\noindent{}zwei marginale Korrekturen)}\Standort{DLA, A:Schnitzler, HS.NZ85.1.1751.}
\physDesc{Brief, Maschinenschriftliche Abschrift, 2 Blätter, 2 Seiten, 2362 Zeichen
\newline{}maschinell
\newline{}Ordnung: 1) mit schwarzer Tinte Vermerk »Salten«  2) mit Bleistift Vermerk »6. 9. 1929«}
\buchAbdrucke{\weitereDrucke{1) \emph{Arthur Schnitzler.} In: \emph{Felix Salten, dem Freund und verehrten Autor zu seinem
                        60. Geburtstag mit herzlichen Glückwünschen überreicht vom Paul Zsolnay
                        Verlag}. Berlin, Wien, Leipzig: \emph{Zsolnay} 6. September 1929, S. 12–13.} \weitereDrucke{2) \pwindex{Jahrbuch Paul Zsolnay Verlag@\emph{Jahrbuch Paul Zsolnay Verlag}|pwk}\emph{Jahrbuch Paul Zsolnay Verlag – 1930}. Berlin, Wien, Leipzig: \emph{Zsolnay} [November] 1929, S. 12–14.} \weitereDrucke{3) Arthur Schnitzler: \emph{Briefe 1913–1931}. Hg. Peter Michael Braunwarth, Richard Miklin, Susanne Pertlik und Heinrich Schnitzler. Frankfurt am Main: \emph{S. Fischer} 1984, S. 619–620.} }\toendnotes[C]{\smallbreak}
\pstart{}{\pb}Mein lieber Felix Salten.\pend
\pstart
           Am liebsten hätte ich Ihnen zu Ihrem \label{K_L02950-1v}\edtext{sechzigsten Geburtstag}{\lemma{\textnormal{\emph{sechzigsten Geburtstag}}}\Cendnote{\textnormal{\textcolor{blue}{Salten} feierte am 6. 9. 1929
                  seinen 60. Geburtstag. \textcolor{blue}{Schnitzler}
                  finalisierte den Text am 29. 7. 1929. Der »Brief« erschien zuerst in einem Sonderdruck für den
                  Jubilar in Großformat und auf Büttenpapier, dann wenige Wochen später im \emph{\textcolor{green}{Jahrbuch Paul Zsolnay Verlag}} für das Jahr
                     1930, das ab 8. 11. 1929 lieferbar war. Die
                  Druckfassung weicht an mehreren Stellen von der hier präsentierten Fassung ab, die
                  eindeutig die frühere Form darstellt. Ob \textcolor{blue}{Salten} bereits diese oder erst die gedruckte Fassung zu sehen bekam, muss
                  offen bleiben.}}}\label{K_L02950-1h} ganz privat und sehr herzlich die Hand gedrückt; Sie hätten
               dann ohneweiters gewusst und empfunden, was ich hier niederzuschreiben vergeblich
               versuchen werde – und etwas mehr. Denn bei einem solchen Anlass und gar vor mehr oder
               minder fremden Leuten die rechten Worte zu finden, ist nicht ganz leicht, zumal für
               Einen, der weder zum Essayisten noch zum Festredner geboren ist. \pend
           
\pstart
           Ueber das, was man gemeiniglich Leistungen zu nennen pflegt, werden Ihnen in diesen
               Tagen Berufene nach Verdienst viel Ehrenvolles zu sagen wissen; mir persönlich ist
                  \strikeout{noch} jenseits \strikeout{all}
               des Ausserordentlichen, was Sie als Dichter, Journalist und Schriftsteller gewirkt
               haben (dies ist eine alphabetische Reihenfolge und keine Klassifikation) {\pb}\introOben{}vor allem\introOben{} das Gesamtbild Ihres Wesens wert und
               bedeutungsvoll, dessen Entwicklung seit frühesten Anfängen ich mit Spannung,
               Sympathie und Teilnahme nachbarlich mitangesehen und bis zum heutigen Tage als Freund
               begleitet habe. Einem Manne, wie Sie, der, erfüllt von der fruchtbarsten Neugier und
               von der dankbarsten Empfänglichkeit, angeregt von überallher, anregend in die Nähe
               und in die Ferne, Einfühler und Eindenker im besten Sinn, und dabei eigenwillig und
               selbstständig wie Wenige, sich so viele Schätzer und Bewunderer erwarb, konnte es
               natürlich auch nicht an Widersachern fehlen;– welche Genugtuung muss es für Sie sein,
               wenn Sie heute an der Schwelle Ihrer dritten Jugend, in diesem Land der Missgunst und
               der Vorbehalte sich sagen dürfen, dass Ihre reiche, vielfältige und in jedem
               Augenblick lebendige Begabung {\pb}gegen manches nicht immer unabsichtliche
               Missverstehen sich von Jahr zu Jahr in stets höherem Maasse durchzusetzen vermocht\substVorne{}\textsuperscript{e}\substDazwischen{}{ }hat\substHinten{}.
               Sie stehen am Ziele – würde ich sagen, wenn ich nicht, durch Ihre eigene Schuld
                     verwöhnt, gerade nach den Arbeits- und Lebensleistungen Ihrer
               letztvergangenen Jahre ein
                     immer Weiter- und Höherschreiten mit froher Gewissheit von
               Ihnen erwartete. Ich will nichts prophezeien, so wenig diese bescheidenen
               Worte als Rückblick gelten dürfen,– aber
                     freuen will ich mich, dass man Ihnen, mein lieber Freund, an
                  diesem
                     festlichen Tage in doppelter Hinsicht, den Blick sowohl in die
               Vergangenheit als der Zukunft \substVorne{}\textsuperscript{zugewendet}{\allowbreak}\substDazwischen{}zu gewandt\substHinten{}, so vertrauensvoll und so von
               ganzem Herzen Glück wünschen kann. \pend
           
\pstart
           {[}hs.:{]} Ihr getreuer{\\[\baselineskip]}\spacefill\mbox{ArthurSchnitzler}\pend
           \leftskip=0em{}\endnumbering\briefempfaengerindex{Salten, Felix@\textsc{Salten, Felix}!zzzSchnitzler, Arthur@\emph{von Arthur Schnitzler}!1929-07-291@{29. 7. 1929}|)be}\mylabel{h}  \normalsize

\doendnotes{C}
\bigskip
\vfill

\clearpage

\footnotesize

\lohead{\textsc{register}}

% Definiere theindex-Environment komplett neu ohne reledmac
\makeatletter
\renewenvironment{theindex}{%
  \section*{\indexname}%
  \setlength{\parindent}{0pt}%
  \setlength{\parskip}{0pt plus 0.3pt}%
  \let\item\@idxitem
}{%
  \clearpage
}
\makeatother

\IfFileExists{\jobname-pw.ind}{\input{\jobname-pw.ind}}{}

\end{document}

      