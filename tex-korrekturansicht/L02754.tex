%% latex-korrekturansicht-vorspann.tex
%% Vorspann für die Korrekturansicht.
%% Lädt die gemeinsame Datei latex-vorspann.tex mit gesetztem Schalter.

\newif\ifkorrekturansicht
\korrekturansichttrue

\input{../tex-inputs/latex-vorspann}


               \section[Paul Goldmann an Arthur Schnitzler, 6. 11. {[}1895{]}]{ Paul Goldmann an Arthur Schnitzler, 6. 11. {[}1895{]}}\nopagebreak\mylabel{v}\rehead{ }\normalsize\beginnumbering\briefempfaengerindex{Schnitzler, Arthur@\textsc{Schnitzler, Arthur}!zzzGoldmann, Paul@\emph{von Paul Goldmann}!1895-11-061@{6. 11. {[}1895{]}}|(be} \toendnotes[C]{\smallbreak\pagebreak[2]} \Standort{DLA, A:Schnitzler, HS.NZ85.1.3165.}
\physDesc{Brief, 1 Blatt, 3 Seiten
\newline{}Handschrift: blaue Tinte, deutsche Kurrent
\newline{}Schnitzler: mit Bleistift das Jahr » 95« vermerkt }\toendnotes[C]{\smallbreak}\pstart
           \noindent{}{\pb}\textcolor{gray}{\textbf{\textbf{\textcolor{brown}{Frankfurter Zeitung}{}\ledrightnote{\textcolor{brown}{Frankfurter Zeitung}}}}}\pend
           \pstart
           \textcolor{gray}{\textbf{(\textcolor{brown}{\begin{otherlanguage}{french}Gazette de Francfort\end{otherlanguage}}{}\ledrightnote{\textcolor{brown}{Frankfurter Zeitung}}). }}\pend
           \pstart
           \textcolor{gray}{\textbf{\textbf{\begin{otherlanguage}{french}Fondateur M. \textcolor{blue}{L.
                              Sonnemann}{}\ledrightnote{\textcolor{blue}{Leopold Sonnemann}}\end{otherlanguage}.}}}\pend
           \pstart
           \begin{otherlanguage}{french}\textcolor{gray}{\textbf{\textcolor{green}{Journal}{}\ledrightnote{→\textcolor{green}{Frankfurter Zeitung}} politique,
                           financier,}}\end{otherlanguage}\hfill \textsc{\textcolor{pink}{Paris}{}\ledrightnote{\textcolor{pink}{Paris}}}, 6. November.\pend
           \pstart
           \begin{otherlanguage}{french}\textcolor{gray}{\textbf{commercial et littéraire.}}\end{otherlanguage}\pend
           \pstart
           \begin{otherlanguage}{french}\textcolor{gray}{\textbf{\textbf{Paraissant trois fois par jour.}}}\end{otherlanguage}\pend
           \pstart
           \begin{otherlanguage}{french}\textcolor{gray}{\textbf{\textbf{Bureau à \textcolor{pink}{Paris}{}\ledrightnote{\textcolor{pink}{Paris}}:}}}\end{otherlanguage}\pend
           \pstart
           \begin{otherlanguage}{french}\textcolor{gray}{\textbf{\textbf{\textcolor{pink}{24. Rue Feydeau}{}\ledrightnote{\textcolor{pink}{rue Feydeau}}.}}}\end{otherlanguage}\pend
           \pstart\center{}Mein lieber Freund,\pend\pstart
           Seit 14 Tagen warte ich auf jeden neuen Tag, in der Hoffnung, er werde mir eine \strikeout{\textcolor{gray}{en}} freie Stunde bringen, um Dir antworten zu können, aber die freie Stunde will
               nicht kommen. Endloſe \textcolor{brown}{Kammer}{}\ledrightnote{\textcolor{brown}{Französische Abgeordnetenkammer}}-Debatten, \label{K_L02754-77v}\edtext{\textcolor{blue}{Miniſter}{}\ledrightnote{→\textcolor{blue}{Alexandre Ribot}}ſturz, Kriſis, neues \textcolor{brown}{Cabinet}{}\ledrightnote{→\textcolor{brown}{Französische Regierung}}}{\lemma{\textnormal{\emph{Miniſterſturz, … Cabinet}}}\Cendnote{\textnormal{Die Regierung \textcolor{blue}{Alexandre Ribot}s wurde am 28. 10. 1895
                           gestürzt. \textcolor{blue}{Premierminister}{ }\textcolor{blue}{Léon Bourgeois} bildete ein neues \textcolor{brown}{Kabinett}.}}}\label{K_L02754-77h}, \label{K_L02754-22v}\edtext{Strike von \textsc{\textcolor{pink}{Carmaux}{}\ledrightnote{\textcolor{pink}{Carmaux}}}}{\lemma{\textnormal{\emph{Strike von Carmaux}}}\Cendnote{\textnormal{In \textcolor{pink}{Carmaux} streikten Glasarbeiterinnen und Glasarbeiter gegen
                     soziale Missstände.}}}\label{K_L02754-22h}, \label{K_L02754-999v}\edtext{Prozeß \textsc{\textcolor{blue}{de Nayve}{}\ledrightnote{\textcolor{blue}{Baptistin Lucien de Combles de Nayve}}}}{\lemma{\textnormal{\emph{Prozeß de Nayve}}}\Cendnote{\textnormal{\textcolor{blue}{Baptistin de Combles de
                     Nayves} wurde der Prozess gemacht, weil ihm seine \textcolor{blue}{Gattin} vorgeworfen hatte, er hätte
                  absichtlich ihr leibliches \textcolor{blue}{Kind} aus einer früheren Beziehung einen Felsen hinunter in den Tod
                  gestoßen. Letztlich wurde er im Zweifel freigesprochen.}}}\label{K_L02754-999h}, dazwiſchen Theater und ſonſt
               allerhand – es bleibt gerade Zeit zum Eſſen und zum Schlafen, und auch dieſe nicht
               immer. Ich \strikeout{hätte}{ }{\pb}hätte Dir ſoviel zu ſagen, möchte Dir für Deine
               letzten ſo lieben Briefe danken, – aber dieſe Arbeits-Woge iſt ſtärker, als mein
               guter Wille, und ich kann nichts machen, als warten, bis ſie vorüber iſt. Dieſer Tage
               hoffe ich endlich Dir ausführlicher ſchreiben zu können. Einſtweilen ſollen dieſe
               wenigen Zeilen mich nur bei Dir entſchuldigen. Wenn ich nach der \textcolor{pink}{Kammer}{}\ledrightnote{→\textcolor{pink}{Nationalversammlung}} gehe, kaufe ich mir hier und da ein
                  \textcolor{pink}{Wien}{}\ledrightnote{\textcolor{pink}{Wien}}er Blatt auf dem \begin{otherlanguage}{french}\textsc{Boulevard}\end{otherlanguage} und ſehe mit Freude, daß die »\textcolor{green}{Liebelei}{}\ledrightnote{\textcolor{green}{Liebelei. Schauspiel in drei Akten}}« \strikeout{ſ\textcolor{gray}{ei}} ihren {\pb}Platz im Repertoire behält. \strikeout{\textcolor{gray}{×}\-\textcolor{gray}{×}\-\textcolor{gray}{×}\-\textcolor{gray}{×}\-\textcolor{gray}{×}} Das iſt ſchön.\pend
           \pstart
           Viele treue Grüße! {\\[\baselineskip]}Dein {\\[\baselineskip]}\spacefill\mbox{Paul Goldmann.}\pend
           \leftskip=0em{}\endnumbering\briefempfaengerindex{Schnitzler, Arthur@\textsc{Schnitzler, Arthur}!zzzGoldmann, Paul@\emph{von Paul Goldmann}!1895-11-061@{6. 11. {[}1895{]}}|)be}\mylabel{h}  \normalsize

\doendnotes{C}
\bigskip
\vfill

\clearpage

\footnotesize

\lohead{\textsc{register}}

% Definiere theindex-Environment komplett neu ohne reledmac
\makeatletter
\renewenvironment{theindex}{%
  \section*{\indexname}%
  \setlength{\parindent}{0pt}%
  \setlength{\parskip}{0pt plus 0.3pt}%
  \let\item\@idxitem
}{%
  \clearpage
}
\makeatother

\IfFileExists{\jobname-pw.ind}{\input{\jobname-pw.ind}}{}

\end{document}

      