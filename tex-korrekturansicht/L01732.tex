%% latex-korrekturansicht-vorspann.tex
%% Vorspann für die Korrekturansicht.
%% Lädt die gemeinsame Datei latex-vorspann.tex mit gesetztem Schalter.

\newif\ifkorrekturansicht
\korrekturansichttrue

\input{../tex-inputs/latex-vorspann}


               \section[Hugo von Hofmannsthal an Arthur Schnitzler, 16. 11. 1907]{ Hugo von Hofmannsthal an Arthur Schnitzler, 16. 11. 1907}\nopagebreak\mylabel{v}\rehead{ }\normalsize\beginnumbering\briefempfaengerindex{Schnitzler, Arthur@\textsc{Schnitzler, Arthur}!zzzHofmannsthal, Hugo von@\emph{von Hugo von Hofmannsthal}!1907-11-161@{16. 11. 1907}|(be} \toendnotes[C]{\smallbreak\pagebreak[2]} \Standort{CUL, Schnitzler, B 43.}
\physDesc{Brief, 1 Blatt, 1 Seite
\newline{}Schreibmaschine
\newline{}Handschrift: schwarze Tinte (\noindent{}Unterschrift)\newline{}Ordnung: 1) mit Bleistift von unbekannter Hand nummeriert: »\strikeout{286}« 2) mit Bleistift von unbekannter Hand nummeriert: »289«}\buchAbdrucke{\weitereDrucke{Hugo von Hofmannsthal, Arthur Schnitzler: \emph{Briefwechsel}. Hg. Therese Nickl und Heinrich Schnitzler. Frankfurt am Main: \emph{S. Fischer} 1964, S. 234.} }\toendnotes[C]{\smallbreak}\pstart
           \raggedleft{}{\pb}\textcolor{pink}{Rodaun}{}\ledrightnote{\textcolor{pink}{Rodaun}}, den 16. November
                  1907.\pend
           \pstart{}Mein lieber Arthur!\pend\pstart
           Ich danke Ihnen herzlich für den lieben Gedanken, \textcolor{blue}{Papa}{}\ledrightnote{→\textcolor{blue}{Hugo August von Hofmannsthal}} einzuladen. Bitte, tun Sie es. Er wohnt \textcolor{pink}{I. Himmelpfortgasse 17}{}\ledrightnote{\textcolor{pink}{Himmelpfortgasse}}. Er wird erst zum Nachtmahl
               kommen und wir sind dann also vorher ja doch allein, umsomehr als ich Sie durch
               diese Zeilen vielmals bitte, mir zu erlauben, dass ich für meine Person schon um
                  ½ 6 kommen darf, um Ihnen das Vorhandene von meinem \textcolor{green}{Stück}{}\ledrightnote{→\textcolor{green}{Silvia im »Stern«}} vorzulesen. Ich stehe dieser Sache so
               unbeschreiblich ratlos und verworren gegenüber und weiss, dass Sie mir helfen
               können. Also erlauben Sie mir das. Es bedarf weiter keiner Antwort, und ich
               komme.\pend
           \pstart Herzlich Ihr\spacefill\mbox{{[}hs.:{]} \strikeout{Hofm} Hugo.}\pend{}\pstart
           \noindent{}{[}ms.:{]} P. S. \textcolor{blue}{S.}{}\ledrightnote{\textcolor{blue}{Gustav Schwarzkopf}} wäre mir bei so schlechter eigener
                  Verfassung eine Qual.\pend
           \endnumbering\briefempfaengerindex{Schnitzler, Arthur@\textsc{Schnitzler, Arthur}!zzzHofmannsthal, Hugo von@\emph{von Hugo von Hofmannsthal}!1907-11-161@{16. 11. 1907}|)be}\mylabel{h}  \normalsize

\doendnotes{C}
\bigskip
\vfill

\clearpage

\footnotesize

\lohead{\textsc{register}}

% Definiere theindex-Environment komplett neu ohne reledmac
\makeatletter
\renewenvironment{theindex}{%
  \section*{\indexname}%
  \setlength{\parindent}{0pt}%
  \setlength{\parskip}{0pt plus 0.3pt}%
  \let\item\@idxitem
}{%
  \clearpage
}
\makeatother

\IfFileExists{\jobname-pw.ind}{\input{\jobname-pw.ind}}{}

\end{document}

      