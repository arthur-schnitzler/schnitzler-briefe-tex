%% latex-korrekturansicht-vorspann.tex
%% Vorspann für die Korrekturansicht.
%% Lädt die gemeinsame Datei latex-vorspann.tex mit gesetztem Schalter.

\newif\ifkorrekturansicht
\korrekturansichttrue

\input{../tex-inputs/latex-vorspann}


\renewcommand{\erwaehntePersonen}{Personen: Felix Salten}
\renewcommand{\erwaehnteInstitutionen}{Institutionen: S. Fischer Verlag}
\renewcommand{\erwaehnteOrte}{Orte: Berlin, Wien}
\renewcommand{\erwaehnteWerke}{Werke: Börsenblatt für den Deutschen Buchhandel, Zwischenspiel. Komödie in drei Akten}
\section[ Arthur Schnitzler: Widmungsexemplar Zwischenspiel für Felix Salten, 24. 11. 1905]{Arthur Schnitzler: Widmungsexemplar Zwischenspiel für Felix
               Salten, 24. 11. 1905}
\nopagebreak\mylabel{v}
\rehead{ }\normalsize\beginnumbering\briefempfaengerindex{Salten, Felix@\textsc{Salten, Felix}!zzzSchnitzler, Arthur@\emph{von Arthur Schnitzler}!1905-11-242@{24. 11. 1905}|(be}
\toendnotes[C]{\smallbreak\pagebreak[2]}\Standort{Wienbibliothek im Rathaus, A-355800, DS-2019-34.}
\physDesc{Widmung am Schmutztitel, 50 Zeichen
\newline{}Handschrift: schwarze Tinte, deutsche Kurrent}\toendnotes[C]{\smallbreak}
\pstart
           \noindent{}{\pb}Meinem lieben Felix Salten\pend
           \pstart \spacefill\mbox{Arth}\pend{}
\pstart
           \textcolor{pink}{Berlin}{}\ledrightnote{\textcolor{pink}{Berlin}}{ }24. 11. 905\pend
           {\bigskip}
\pstart
           \noindent{}\centering{}{\pb}\textcolor{gray}{\textbf{\textcolor{green}{\so{Zwiſchenſpiel}}{}\ledrightnote{\textcolor{green}{Zwischenspiel. Komödie in drei Akten}}}}\pend
           
\pstart
           \noindent{}\centering{}\textcolor{gray}{\textbf{\so{Komödie in drei Akten von}}}{\\}\textcolor{gray}{\textbf{\textbf{\so{Arthur Schnitzler}}}}\pend
           {\bigskip}
\pstart
           \noindent{}\centering{}\textcolor{gray}{\textbf{\textcolor{pink}{\so{Berlin}}{}\ledrightnote{\textcolor{pink}{Berlin}}{ }\label{K_L03603-1v}\edtext{\so{1906}}{\lemma{\textnormal{\emph{1906}}}\Cendnote{\textnormal{am 28. 10. 1905 vom \emph{\textcolor{green}{Börsenblatt für
                           den deutschen Buchhandel}} als Neuerscheinung gemeldet}}}\label{K_L03603-1h}}}\pend
           
\pstart
           \noindent{}\centering{}\textcolor{gray}{\textbf{\textcolor{brown}{\so{S. Fiſcher, Verlag}}{}\ledrightnote{\textcolor{brown}{S. Fischer Verlag}}}}\pend
           \endnumbering\briefempfaengerindex{Salten, Felix@\textsc{Salten, Felix}!zzzSchnitzler, Arthur@\emph{von Arthur Schnitzler}!1905-11-242@{24. 11. 1905}|)be}\mylabel{h}  \normalsize

\doendnotes{C}
\bigskip
\vfill

\clearpage

\footnotesize

\lohead{\textsc{register}}

% Definiere theindex-Environment komplett neu ohne reledmac
\makeatletter
\renewenvironment{theindex}{%
  \section*{\indexname}%
  \setlength{\parindent}{0pt}%
  \setlength{\parskip}{0pt plus 0.3pt}%
  \let\item\@idxitem
}{%
  \clearpage
}
\makeatother

\IfFileExists{\jobname-pw.ind}{\input{\jobname-pw.ind}}{}

\end{document}

      