%% latex-korrekturansicht-vorspann.tex
%% Vorspann für die Korrekturansicht.
%% Lädt die gemeinsame Datei latex-vorspann.tex mit gesetztem Schalter.

\newif\ifkorrekturansicht
\korrekturansichttrue

\input{../tex-inputs/latex-vorspann}


\section[Arthur Schnitzler an Stefan Zweig, 4. 11. 1929]{L03734 Arthur Schnitzler an Stefan Zweig, 4. 11. 1929}
\nopagebreak\mylabel{L03734v}
\rehead{ }\normalsize\beginnumbering\briefempfaengerindex{Zweig, Stefan@\textsc{Zweig, Stefan}!zzzSchnitzler, Arthur@\emph{von Arthur Schnitzler}!1929-11-041@{4. 11. 1929}|(be}
\toendnotes[C]{\smallbreak\pagebreak[2]}
\correspDesc{Versand  durch Arthur Schnitzler am 4. 11. 1929 in Wien
\newline{}Erhalt  durch Stefan Zweig im Zeitraum [5. 11. 1929 –
            9. 11. 1929?] in Salzburg}\toendnotes[C]{\smallbreak}
\Standort{Jerusalem, National Library of Israel, ARC. Ms. Var. 305 1 58 Stefan Zweig Collection.}
\physDesc{Brief, 1 Blatt, 1 Seite, 1322 Zeichen
\newline{}Schreibmaschine
\newline{}Handschrift: Bleistift (\noindent{}Unterschrift, Ergänzung eines Buchstabens und eine Streichung)}\toendnotes[C]{\smallbreak}
\pstart
           {\pb}\textcolor{gray}{\textbf{D\textsuperscript{R} ARTHUR SCHNITZLER}}\hfill 4. 11. 1929.\pend
           
\pstart
           \textcolor{gray}{\textbf{\textcolor{pink}{WIEN, XVIII. STERNWARTESTRASSE 71}\oindex{Wien@\textbf{Wien}!XVIII., Währing@\textbf{XVIII., Währing}!Sternwartestraße 71@\textbf{Sternwartestraße 71}, \emph{Wohngebäude}|pw}{}\ledrightnote{\textcolor{pink}{Sternwartestraße 71}}.}}\pend
           
\pstart{}Lieber und verehrter Stefan Zweig.\pend\vspace{0.5em}
\pstart
           Besten Dank für Ihre Mitteilung Herrn \textcolor{blue}{A. del Vayo}\pwindex{Álvarez del Vayo, Julio 9.\,2.\,1891 Villaviciosa de Odón – 3.\,5.\,1975 Genf@\textsc{Álvarez del Vayo, Julio} (9.\,2.\,1891 Villaviciosa de Odón – 3.\,5.\,1975 Genf), \emph{Schriftsteller, Politiker, Journalist}|pw}{}\ledrightnote{\textcolor{blue}{Julio Álvarez del Vayo}}
          betreffend. Er möge sich direkt an mich wenden. Können Sie mir vielleicht sagen, was für
          Honorare er zahlt? Bei \textcolor{blue}{Fischer}\pwindex{Fischer, Samuel 24.\,12.\,1859 Liptovský Mikuláš – 15.\,10.\,1934 Berlin@\textsc{Fischer, Samuel} (24.\,12.\,1859 Liptovský Mikuláš – 15.\,10.\,1934 Berlin), \emph{Verleger}|pw}{}\ledrightnote{\textcolor{blue}{Samuel Fischer}} werde ich
          reklamieren. In \textcolor{pink}{Spanien}\oindex{Spanien@\textbf{Spanien}|pw}{}\ledrightnote{\textcolor{pink}{Spanien}} ist ja verhältnismässig recht
          wenig von mir erschienen – so weit ich darüber informiert bin. \pend
           
\pstart
           Ich freue mich auf das versprochene neue \textcolor{green}{Buch}\pwindex{Zweig, Stefan 28.\,11.\,1881 Wien – 23.\,2.\,1942 Petrópolis@\textsc{Zweig, Stefan} (28.\,11.\,1881 Wien – 23.\,2.\,1942 Petrópolis), \emph{Schriftsteller}!Kleine Chronik@\strich\emph{Kleine Chronik}|pwv}{}\ledrightnote{{$\rightarrow$}\emph{\textcolor{green}{Kleine Chronik}}} und beglückwünsche Sie noch einmal zu dem ausserordentlichen
            »\textcolor{green}{Fouché}\pwindex{Zweig, Stefan 28.\,11.\,1881 Wien – 23.\,2.\,1942 Petrópolis@\textsc{Zweig, Stefan} (28.\,11.\,1881 Wien – 23.\,2.\,1942 Petrópolis), \emph{Schriftsteller}!Joseph Fouché. Bildnis eines politischen Menschen@\strich\emph{Joseph Fouché. Bildnis eines politischen Menschen}|pw}{}\ledrightnote{\textcolor{green}{Joseph Fouché. Bildnis eines politischen Menschen}}«, dessen Erfolg sich, wie ich mit Vergnügen
          höre und lese, in Nähe und Ferne immer glänzender bestätigt. \pend
           
\pstart
           Neulich hat man mir aus \textcolor{pink}{Paris}\oindex{Paris@\textbf{Paris}, \emph{Hauptstadt}|pw}{}\ledrightnote{\textcolor{pink}{Paris}} einen \label{K_L03734-1v}\edtext{\textcolor{green}{Ausschnitt}\pwindex{?? [Peur, Film von Arthur Schnitzler]@\emph{?? [Peur, Film von Arthur Schnitzler]}|pwv}{}\ledrightnote{{$\rightarrow$}\emph{\textcolor{green}{?? [Peur, Film von Arthur Schnitzler]}}}}{\lemma{\textnormal{\emph{Ausschnitt}}}\Cendnote{\textnormal{Obzwar \textcolor{blue}{Schnitzler} in Folge die 
              Zeitschrift, in der die \textcolor{green}{Notiz}\pwindex{?? [Peur, Film von Arthur Schnitzler]@\emph{?? [Peur, Film von Arthur Schnitzler]}|pwkv} stand, als \emph{\textcolor{green}{Gringoire}\pwindex{Gringoire@\emph{Gringoire}|pwk}} spezifiziert, konnte die betreffende Stelle bislang nicht nachgewiesen werden.}}}\label{K_L03734-1} geschickt, in dem eine
          Kinovorstellung besprochen war »\textcolor{green}{Peur}\pwindex{Steinhoff, Hans 10.\,3.\,1882 Marienberg – 20.\,4.\,1945 Glienig@\textsc{Steinhoff, Hans} (10.\,3.\,1882 Marienberg – 20.\,4.\,1945 Glienig)!Angst@\strich\emph{Angst}|pw}{}\ledrightnote{\textcolor{green}{Angst}}« d’apres la \textcolor{green}{nouvelle}\pwindex{Zweig, Stefan 28.\,11.\,1881 Wien – 23.\,2.\,1942 Petrópolis@\textsc{Zweig, Stefan} (28.\,11.\,1881 Wien – 23.\,2.\,1942 Petrópolis), \emph{Schriftsteller}!Angst@\strich\emph{Angst}|pwv}{}\ledrightnote{{$\rightarrow$}\emph{\textcolor{green}{Angst}}} de M. Arthur Schnitzler.
          Nach dem Inhalt muss es sich um die »\textcolor{green}{Angst}\pwindex{Steinhoff, Hans 10.\,3.\,1882 Marienberg – 20.\,4.\,1945 Glienig@\textsc{Steinhoff, Hans} (10.\,3.\,1882 Marienberg – 20.\,4.\,1945 Glienig)!Angst@\strich\emph{Angst}|pw}{}\ledrightnote{\textcolor{green}{Angst}}« gehandelt
          haben, die ich selbst hier in einem \label{K_L03734-2v}\edtext{\textcolor{pink}{Kino}\oindex{Wien@\textbf{Wien}!I., Innere Stadt@\textbf{I., Innere Stadt}!Gartenbaukino@\textbf{Gartenbaukino}, \emph{Kino}|pwuv}\oindex{Wien@\textbf{Wien}!I., Innere Stadt@\textbf{I., Innere Stadt}!Imperialkino@\textbf{Imperialkino}, \emph{Kino}|pwuv}{}\ledrightnote{{$\rightarrow$}\emph{\textcolor{pink}{Gartenbaukino}}{\newline}{$\rightarrow$}\emph{\textcolor{pink}{Imperialkino}}}}{\lemma{\textnormal{\emph{Kino}}}\Cendnote{\textnormal{\textcolor{blue}{Schnitzler} und \textcolor{blue}{Clara Katharina Pollaczek}\pwindex{Pollaczek, Clara Katharina 15.\,1.\,1875 Wien – 22.\,7.\,1951 ebd.@\textsc{Pollaczek, Clara Katharina} (15.\,1.\,1875 Wien – 22.\,7.\,1951 ebd.), \emph{Schriftstellerin}|pwk} sahen den \textcolor{green}{Film}\pwindex{Steinhoff, Hans 10.\,3.\,1882 Marienberg – 20.\,4.\,1945 Glienig@\textsc{Steinhoff, Hans} (10.\,3.\,1882 Marienberg – 20.\,4.\,1945 Glienig)!Angst@\strich\emph{Angst}|pwkv} am 14. 6. 1929 entweder im \textcolor{pink}{Imperialkino}\oindex{Wien@\textbf{Wien}!I., Innere Stadt@\textbf{I., Innere Stadt}!Imperialkino@\textbf{Imperialkino}, \emph{Kino}|pwk}
            oder im \textcolor{pink}{Gartenbaukino}\oindex{Wien@\textbf{Wien}!I., Innere Stadt@\textbf{I., Innere Stadt}!Gartenbaukino@\textbf{Gartenbaukino}, \emph{Kino}|pwk}.}}}\label{K_L03734-2} gesehen habe. Die
            \textcolor{green}{Notiz}\pwindex{?? [Peur, Film von Arthur Schnitzler]@\emph{?? [Peur, Film von Arthur Schnitzler]}|pwv}{}\ledrightnote{{$\rightarrow$}\emph{\textcolor{green}{?? [Peur, Film von Arthur Schnitzler]}}} stand im »\textcolor{green}{Gringoir{[}e{]}}\pwindex{Gringoire@\emph{Gringoire}|pw}{}\ledrightnote{\textcolor{green}{Gringoire}}«; meine weiteren Recherchen sind
          noch ohne Erfolg geblieben.\pend
           
\pstart
           Es wäre schön, wenn ich Sie wieder einmal sprechen könnte. Dass Sie das letzte Mal in \textcolor{pink}{Wien}\oindex{Wien@\textbf{Wien}, \emph{Verwaltungsgebiet}|pw}{}\ledrightnote{\textcolor{pink}{Wien}} keine Zeit hatten ist \strikeout{ja} natürlich und Sie, lieber Stefan Zweig, haben mir sicher verziehen, dass ich
          bei der \label{K_L03734-3v}\edtext{\textcolor{violet}{Trauerfeier für \textcolor{blue}{Hofmannsthal}\pwindex{Hofmannsthal, Hugo von 1.\,2.\,1874 Wien – 15.\,7.\,1929 Rodaun@\textsc{Hofmannsthal, Hugo von} (1.\,2.\,1874 Wien – 15.\,7.\,1929 Rodaun), \emph{Schriftsteller}|pw}{}\ledrightnote{\textcolor{blue}{Hugo von Hofmannsthal}}}\eventindex{Burgtheater@\textbf{Burgtheater}!Gedenkfeier für Hugo von Hofmannsthal, 13.10.1929@Gedenkfeier für Hugo von Hofmannsthal, 13.10.1929|pw}{}\ledrightnote{\textcolor{violet}{Gedenkfeier für Hugo von Hofmannsthal, 13.10.1929}}}{\lemma{\textnormal{\emph{Trauerfeier für Hofmannsthal}}}\Cendnote{\textnormal{Am 13. 10. 1929 fand im \textcolor{pink}{Burgtheater}\oindex{Wien@\textbf{Wien}!I., Innere Stadt@\textbf{I., Innere Stadt}!Burgtheater@\textbf{Burgtheater}, \emph{Theater}|pwk} eine \textcolor{violet}{Gedenkfeier}\eventindex{Burgtheater@\textbf{Burgtheater}!Gedenkfeier für Hugo von Hofmannsthal, 13.10.1929@Gedenkfeier für Hugo von Hofmannsthal, 13.10.1929|pwk} für \textcolor{blue}{Hugo von Hofmannsthal}\pwindex{Hofmannsthal, Hugo von 1.\,2.\,1874 Wien – 15.\,7.\,1929 Rodaun@\textsc{Hofmannsthal, Hugo von} (1.\,2.\,1874 Wien – 15.\,7.\,1929 Rodaun), \emph{Schriftsteller}|pwk} statt, bei der \emph{\textcolor{green}{Der Thor
              und der Tod}\pwindex{Hofmannsthal, Hugo von 1.\,2.\,1874 Wien – 15.\,7.\,1929 Rodaun@\textsc{Hofmannsthal, Hugo von} (1.\,2.\,1874 Wien – 15.\,7.\,1929 Rodaun), \emph{Schriftsteller}!Thor und der Tod@\strich\emph{Der Thor und der Tod}|pwk}} gespielt und von \textcolor{blue}{Stefan Zweig}\pwindex{Zweig, Stefan 28.\,11.\,1881 Wien – 23.\,2.\,1942 Petrópolis@\textsc{Zweig, Stefan} (28.\,11.\,1881 Wien – 23.\,2.\,1942 Petrópolis), \emph{Schriftsteller}|pwk}
            eine \emph{\textcolor{green}{Gedächtnisrede}\pwindex{Zweig, Stefan 28.\,11.\,1881 Wien – 23.\,2.\,1942 Petrópolis@\textsc{Zweig, Stefan} (28.\,11.\,1881 Wien – 23.\,2.\,1942 Petrópolis), \emph{Schriftsteller}!Hugo von Hofmannsthal. Gedächtnisrede zur Trauerfeier im Wiener Burgtheater@\strich\emph{Hugo von Hofmannsthal. Gedächtnisrede zur Trauerfeier im Wiener Burgtheater}|pwk}} gehalten wurde.}}}\label{K_L03734-3} nicht
          im \textcolor{pink}{Theater}\oindex{Wien@\textbf{Wien}!I., Innere Stadt@\textbf{I., Innere Stadt}!Burgtheater@\textbf{Burgtheater}, \emph{Theater}|pwv}{}\ledrightnote{{$\rightarrow$}\emph{\textcolor{pink}{Burgtheater}}} war und so Ihre \textcolor{green}{Rede}\pwindex{Zweig, Stefan 28.\,11.\,1881 Wien – 23.\,2.\,1942 Petrópolis@\textsc{Zweig, Stefan} (28.\,11.\,1881 Wien – 23.\,2.\,1942 Petrópolis), \emph{Schriftsteller}!Hugo von Hofmannsthal. Gedächtnisrede zur Trauerfeier im Wiener Burgtheater@\strich\emph{Hugo von Hofmannsthal. Gedächtnisrede zur Trauerfeier im Wiener Burgtheater}|pwv}{}\ledrightnote{{$\rightarrow$}\emph{\textcolor{green}{Hugo von Hofmannsthal. Gedächtnisrede zur Trauerfeier im Wiener Burgtheater}}} nicht gehört habe. Man hat mir
          erzählt, wie schön Sie gesprochen haben. \pend
           
\pstart
           Mit den herzlichsten Grüssen{\\[\baselineskip]}Ihr freundschaftlich ergebener{\\[\baselineskip]}\spacefill\mbox{{[}hs.:{]} ArthSchnitzler}\pend
           \leftskip=0em{}
\pstart
           \noindent{}Herrn Stefan Zweig\pend
           
\pstart
           \noindent{}\textcolor{pink}{Salzburg}\oindex{Salzburg@\textbf{Salzburg}, \emph{Verwaltungsgebiet}|pw}{}\ledrightnote{\textcolor{pink}{Salzburg}}.\pend
           \selectlanguage{ngerman}\endnumbering\briefempfaengerindex{Zweig, Stefan@\textsc{Zweig, Stefan}!zzzSchnitzler, Arthur@\emph{von Arthur Schnitzler}!1929-11-041@{4. 11. 1929}|)be}\mylabel{L03734h}
\begin{anhang}
\end{anhang}\normalsize

\doendnotes{C}
\bigskip
\vfill

\clearpage

\footnotesize

\lohead{\textsc{register}}

% Definiere theindex-Environment komplett neu ohne reledmac
\makeatletter
\renewenvironment{theindex}{%
  \section*{\indexname}%
  \setlength{\parindent}{0pt}%
  \setlength{\parskip}{0pt plus 0.3pt}%
  \let\item\@idxitem
}{%
  \clearpage
}
\makeatother

\IfFileExists{\jobname-pw.ind}{\input{\jobname-pw.ind}}{}

\end{document}

      