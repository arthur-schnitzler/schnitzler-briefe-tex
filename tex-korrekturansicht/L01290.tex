%% latex-korrekturansicht-vorspann.tex
%% Vorspann für die Korrekturansicht.
%% Lädt die gemeinsame Datei latex-vorspann.tex mit gesetztem Schalter.

\newif\ifkorrekturansicht
\korrekturansichttrue

\input{../tex-inputs/latex-vorspann}


               \section[Arthur Schnitzler an Hermann Bahr, 18. 5. 1903]{ Arthur Schnitzler an Hermann Bahr, 18. 5. 1903}\nopagebreak\mylabel{v}\rehead{ }\normalsize\beginnumbering\briefempfaengerindex{Bahr, Hermann@\textsc{Bahr, Hermann}!zzzSchnitzler, Arthur@\emph{von Arthur Schnitzler}!1903-05-181@{18. 5. 1903}|(be} \toendnotes[C]{\smallbreak\pagebreak[2]} \Standort{TMW, HS AM 23355 Ba.}
\physDesc{Brief, 1 Blatt, 2 Seiten
\newline{}Handschrift: schwarze Tinte, deutsche Kurrent\newline{}Ordnung: Lochung }\buchAbdrucke{\weitereDrucke{1) \emph{18. 5. 1903.} In: Arthur Schnitzler: \emph{The Letters of Arthur Schnitzler to Hermann Bahr}. Edited, annotated, and with an introduction, by Donald G.
                        Daviau. Chapel Hill: \emph{The University of North Carolina Press} 1978, S. 78 (University of North Carolina studies in the Germanic languages
                        and literatures, 89).} \weitereDrucke{2) Hermann Bahr, Arthur Schnitzler: \emph{Briefwechsel, Aufzeichnungen, Dokumente (1891–1931)}. Hg. Kurt Ifkovits und Martin Anton Müller. Göttingen: \emph{Wallstein} 2018, S. 265.} }\pstart
           \raggedleft{}{\pb}18. 5. 903.\pend
           \pstart{}lieber Hermann,\pend\pstart
           du haſt jedenfalls auch den Aufruf der \textcolor{brown}{Penſionsanſtalt deutſcher Journaliſten u Schriftſteller}{}\ledrightnote{\textcolor{brown}{Pensionsanstalt deutscher Journalisten und Schriftsteller}} erhalten sowie den
               Zeichnungsſchein für jährlichen \textsc{resp.} für einmaligen
               Beitrag. Da wir nun beide unter {\pb}dieſem Aufruf
               unterſchrieben ſind, möcht ich dich fragen, wieviel \textsc{resp.}
               ob du »einmalig« oder »jährlich« zeichneſt. Ich habe keine rechte Vorstellung, zu wie
               viel man da ungefähr verpflichtet iſt.\pend
           \pstart
           Entschuldg die Beläſtigung{\\[\baselineskip]}Herzlichſt dein{\\[\baselineskip]}\spacefill\mbox{Arthur Sch}\pend
           \leftskip=0em{}\endnumbering\briefempfaengerindex{Bahr, Hermann@\textsc{Bahr, Hermann}!zzzSchnitzler, Arthur@\emph{von Arthur Schnitzler}!1903-05-181@{18. 5. 1903}|)be}\mylabel{h}  \normalsize

\doendnotes{C}
\bigskip
\vfill

\clearpage

\footnotesize

\lohead{\textsc{register}}

% Definiere theindex-Environment komplett neu ohne reledmac
\makeatletter
\renewenvironment{theindex}{%
  \section*{\indexname}%
  \setlength{\parindent}{0pt}%
  \setlength{\parskip}{0pt plus 0.3pt}%
  \let\item\@idxitem
}{%
  \clearpage
}
\makeatother

\IfFileExists{\jobname-pw.ind}{\input{\jobname-pw.ind}}{}

\end{document}

      