%% latex-korrekturansicht-vorspann.tex
%% Vorspann für die Korrekturansicht.
%% Lädt die gemeinsame Datei latex-vorspann.tex mit gesetztem Schalter.

\newif\ifkorrekturansicht
\korrekturansichttrue

\input{../tex-inputs/latex-vorspann}


\renewcommand{\erwaehnteInstitutionen}{Institutionen: Neue Freie Presse, Palasthotel Berlin}
\renewcommand{\erwaehnteOrte}{Orte: Berlin, Dessauer Straße, Wien}
\renewcommand{\erwaehnteWerke}{Werke: Der Schleier der Beatrice. Schauspiel in fünf Akten}
\section[ Paul Goldmann an Arthur Schnitzler, 27. 2. {[}1903?{]}]{Paul Goldmann an Arthur Schnitzler, 27. 2. {[}1903?{]}}
\nopagebreak\mylabel{v}
\rehead{ }\normalsize\beginnumbering\briefempfaengerindex{Schnitzler, Arthur@\textsc{Schnitzler, Arthur}!zzzGoldmann, Paul@\emph{von Paul Goldmann}!1903-02-272@{27. 2. {[}1903?{]}}|(be}
\toendnotes[C]{\smallbreak\pagebreak[2]}\Standort{DLA, A:Schnitzler, HS.NZ85.1.3173.}
\physDesc{Brief, 1 Blatt, 3 Seiten
\newline{}Handschrift: blaue Tinte, deutsche Kurrent}\toendnotes[C]{\smallbreak}
\pstart
           \noindent{}\raggedleft{}{\pb}\textcolor{gray}{\textbf{\textcolor{pink}{DESSAUERSTRASSE 19}{}\ledrightnote{\textcolor{pink}{Dessauer Straße}}}}\pend
           
\pstart
           \textcolor{pink}{Berlin}{}\ledrightnote{\textcolor{pink}{Berlin}}, 27. Februar.\pend
           
\pstart\center{}Liebſter Freund,\pend
\pstart
           Bis \label{K_L03365-11v}\edtext{½ 8}{\lemma{\textnormal{\emph{½ 8}}}\Cendnote{\textnormal{Es dürfte sich, wie aus dem Folgenden hervorgeht, um 7:30 morgens handeln. Wo der Treffpunkt
               angesetzt war, ist nicht zu bestimmen. \textcolor{blue}{Schnitzler} dürfte danach zur Probe von \emph{\textcolor{green}{Der Schleier der Beatrice}} gegangen sein.}}}\label{K_L03365-11h} habe ich auf Dich gewartet. Dann mußte ich fort, um allerlei
                  Informations-Wünſche der \textcolor{pink}{Wien}{}\ledrightnote{\textcolor{pink}{Wien}}er \textcolor{brown}{Redaktion}{}\ledrightnote{{$\rightarrow$}\textcolor{brown}{Neue Freie Presse}} zu
               befriedigen, glaubte auch, Du würdeſt nicht mehr kommen. Um 10 Uhr komme
               ich zurück und höre, daß Du da {\pb}warſt. Es thut mir
               unendlich leid, daß wir uns verfehlt haben. Ich habe um 10 Uhr noch in
               Dein \textsc{\textcolor{brown}{Hotel}{}\ledrightnote{\textcolor{brown}{Palasthotel Berlin}}} telephonirt, höre aber, daß Du nicht mehr da zu finden biſt. Kann ich Dich
                  \label{K_L03365-1v}\edtext{morgen, Samſtag}{\lemma{\textnormal{\emph{morgen, Samſtag}}}\Cendnote{\textnormal{Ein Treffen am Samstag, dem 28. 2. 1903, kam
                  zustande. Am Sonntag, dem 1. 3. 1903, sahen sie sich nicht.}}}\label{K_L03365-1h}, Abend
                  nach 10 Uhr ſehen? Wenn Du kannſt, ſo komme doch, bitte, \substVorne{}\textsuperscript{um}\substDazwischen{}gegen\substHinten{}{ }1\substVorne{}\textsuperscript{\textcolor{gray}{×}}\substDazwischen{}7\substHinten{} Uhr zu mir hinauf. Wenn {\pb}nicht, ſo
               laſſe mir Nachricht zukommen, ob ich Dich Sonntag{ }Nachmittag oder Abend ſprechen kann.\pend
           
\pstart
           Herzlichſt {\\[\baselineskip]}Dein {\\[\baselineskip]}\spacefill\mbox{Paul Goldm}\pend
           \leftskip=0em{}\endnumbering\briefempfaengerindex{Schnitzler, Arthur@\textsc{Schnitzler, Arthur}!zzzGoldmann, Paul@\emph{von Paul Goldmann}!1903-02-272@{27. 2. {[}1903?{]}}|)be}\mylabel{h}  \normalsize

\doendnotes{C}
\bigskip
\vfill

\clearpage

\footnotesize

\lohead{\textsc{register}}

% Definiere theindex-Environment komplett neu ohne reledmac
\makeatletter
\renewenvironment{theindex}{%
  \section*{\indexname}%
  \setlength{\parindent}{0pt}%
  \setlength{\parskip}{0pt plus 0.3pt}%
  \let\item\@idxitem
}{%
  \clearpage
}
\makeatother

\IfFileExists{\jobname-pw.ind}{\input{\jobname-pw.ind}}{}

\end{document}

      