%% latex-korrekturansicht-vorspann.tex
%% Vorspann für die Korrekturansicht.
%% Lädt die gemeinsame Datei latex-vorspann.tex mit gesetztem Schalter.

\newif\ifkorrekturansicht
\korrekturansichttrue

\input{../tex-inputs/latex-vorspann}


               \section[Arthur Schnitzler an Gerhart Hauptmann, 6. 6. 1922]{ Arthur Schnitzler an Gerhart Hauptmann,
                    6. 6. 1922}\nopagebreak\mylabel{v}\rehead{ }\normalsize\beginnumbering\briefempfaengerindex{Hauptmann, Gerhart@\textsc{Hauptmann, Gerhart}!zzzSchnitzler, Arthur@\emph{von Arthur Schnitzler}!1922-06-062@{6. 6. 1922}|(be} \toendnotes[C]{\smallbreak\pagebreak[2]} \Standort{Staatsbibliothek Berlin – Preußischer Kulturbesitz, GHBrBl A:Schnitzler (14).}
\physDesc{Postkarte
\newline{}Handschrift: schwarze Tinte, deutsche Kurrent\newline{}Versand: Stempel: »\nobreak{}1\textcolor{gray}{8}/1 Wien
                                        110, 6. VI. 22, XII\nobreak{}«.  }\toendnotes[C]{\smallbreak}\pstart{}{\pb}Herrn \textsc{Gerhart Hauptmann}\pend{}\pstart{}\textsc{\textcolor{pink}{Agnetendorf}{}\ledrightnote{\textcolor{pink}{Agnetendorf}}}\pend{}\pstart{}\textsc{in \textcolor{pink}{Schlesien}{}\ledrightnote{\textcolor{pink}{Schlesien}}}\pend{}{\bigskip}\pstart
           \raggedleft{}{\pb}\textcolor{pink}{Wien}{}\ledrightnote{\textcolor{pink}{Wien}}. 6. 6. 22\pend
           \pstart
           mein verehrter und lieber Herr Gerhart Hauptmann, ſeien Sie und
                    Ihre verehrte \textcolor{blue}{Gattin}{}\ledrightnote{→\textcolor{blue}{Margarete Hauptmann}} für
                    das herzliche Glückwunſchtelegramm aufs allerwärmſte bedankt\pend
           \pstart Ihr bewundernd\textcolor{gray}{-}getreuer \spacefill\mbox{Arthur
                        Schnitzler}\pend{}\endnumbering\briefempfaengerindex{Hauptmann, Gerhart@\textsc{Hauptmann, Gerhart}!zzzSchnitzler, Arthur@\emph{von Arthur Schnitzler}!1922-06-062@{6. 6. 1922}|)be}\mylabel{h}  \normalsize

\doendnotes{C}
\bigskip
\vfill

\clearpage

\footnotesize

\lohead{\textsc{register}}

% Definiere theindex-Environment komplett neu ohne reledmac
\makeatletter
\renewenvironment{theindex}{%
  \section*{\indexname}%
  \setlength{\parindent}{0pt}%
  \setlength{\parskip}{0pt plus 0.3pt}%
  \let\item\@idxitem
}{%
  \clearpage
}
\makeatother

\IfFileExists{\jobname-pw.ind}{\input{\jobname-pw.ind}}{}

\end{document}

      