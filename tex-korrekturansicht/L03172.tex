%% latex-korrekturansicht-vorspann.tex
%% Vorspann für die Korrekturansicht.
%% Lädt die gemeinsame Datei latex-vorspann.tex mit gesetztem Schalter.

\newif\ifkorrekturansicht
\korrekturansichttrue

\input{../tex-inputs/latex-vorspann}


\renewcommand{\erwaehntePersonen}{Personen: Alexander Zeitlin}
\renewcommand{\erwaehnteInstitutionen}{Institutionen: Akademie der Bildenden Künste Wien}
\renewcommand{\erwaehnteOrte}{Orte: Burgtheater, Wien}
\renewcommand{\erwaehnteWerke}{}
\section[ Felix Salten an Arthur Schnitzler, {[}9. 6. 1896?{]}]{Felix Salten an Arthur Schnitzler, {[}9. 6. 1896?{]}}
\nopagebreak\mylabel{v}
\rehead{ }\normalsize\beginnumbering\briefempfaengerindex{Schnitzler, Arthur@\textsc{Schnitzler, Arthur}!zzzSalten, Felix@\emph{von Felix Salten}!1896-06-093@{{[}9. 6. 1896?{]}}|(be}
\toendnotes[C]{\smallbreak\pagebreak[2]}\Standort{CUL, Schnitzler, B 89, A 1.}
\physDesc{Brief, 1 Blatt, 1 Seite, 342 Zeichen
\newline{}Handschrift: schwarze Tinte, lateinische Kurrent
\newline{}Schnitzler: mit Bleistift datiert: »9/6 9\textcolor{gray}{6}« 
\newline{}Ordnung: mit Bleistift von unbekannter Hand nummeriert: »71« }\toendnotes[C]{\smallbreak}
\pstart
           \noindent{}{\pb}Lieber Arthur, ich hatte bis jetzt zu thun, und konnte
               auch nicht auf die \textcolor{brown}{Akademie}{}\ledrightnote{\textcolor{brown}{Akademie der Bildenden Künste Wien}} kommen. Vielleicht
               empfängt mich Herr \textcolor{blue}{Zeitlin}{}\ledrightnote{\textcolor{blue}{Alexander Zeitlin}}{ }Donnerstag oder Freitag
               um ½ 3 — 3. Das ist eine Stunde, zu welcher ich in keinem Falle
               verhindert bin. Ist’s \label{K_L03172-1v}\edtext{heute etwas mit dem \textcolor{pink}{Burgtheater}{}\ledrightnote{\textcolor{pink}{Burgtheater}}}{\lemma{\textnormal{\emph{heute … Burgtheater}}}\Cendnote{\textnormal{siehe Arthur Schnitzler an Felix Salten, [9. 6. 1896?]}}}\label{K_L03172-1h}? Ich weiss noch nicht, ob ich kann, aber es ist immerhin möglich\pend
           
\pstart
           herzlich Ihr {\\[\baselineskip]}\spacefill\mbox{Salten}\pend
           \leftskip=0em{}\endnumbering\briefempfaengerindex{Schnitzler, Arthur@\textsc{Schnitzler, Arthur}!zzzSalten, Felix@\emph{von Felix Salten}!1896-06-093@{{[}9. 6. 1896?{]}}|)be}\mylabel{h}  \normalsize

\doendnotes{C}
\bigskip
\vfill

\clearpage

\footnotesize

\lohead{\textsc{register}}

% Definiere theindex-Environment komplett neu ohne reledmac
\makeatletter
\renewenvironment{theindex}{%
  \section*{\indexname}%
  \setlength{\parindent}{0pt}%
  \setlength{\parskip}{0pt plus 0.3pt}%
  \let\item\@idxitem
}{%
  \clearpage
}
\makeatother

\IfFileExists{\jobname-pw.ind}{\input{\jobname-pw.ind}}{}

\end{document}

      