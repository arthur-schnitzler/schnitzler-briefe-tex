%% latex-korrekturansicht-vorspann.tex
%% Vorspann für die Korrekturansicht.
%% Lädt die gemeinsame Datei latex-vorspann.tex mit gesetztem Schalter.

\newif\ifkorrekturansicht
\korrekturansichttrue

\input{../tex-inputs/latex-vorspann}


\renewcommand{\erwaehntePersonen}{Personen: Else Berger, Oskar Blumenthal, Antonie Cuny-Pierron, Rudolf Eduard von Cuny-Pierron, Gisela Fischer, Marie Glümer, Paul Goldmann, Felix Salten, Heinrich Schnitzler, Josefine Lydia von Weisswasser}
\renewcommand{\erwaehnteOrte}{Orte: Baden bei Wien, Brühl, Diglas’ Restaurant »Zur schönen Aussicht«, Dölsach, Heiligenstadt, Klosterneuburg, Lienz, Linz, Salzburg, Wien, XIX., Döbling}
\renewcommand{\erwaehnteWerke}{Werke: Abschiedssouper, Eine Partie Klabrias im Café Spitzer, Faust. Eine Tragödie}
\section[Arthur Schnitzler an Felix Salten, {[}14. 8. 1893{]}]{Arthur Schnitzler an Felix Salten, {[}14. 8. 1893{]}}
\nopagebreak\mylabel{v}
\rehead{ }\normalsize\beginnumbering\briefempfaengerindex{Salten, Felix@\textsc{Salten, Felix}!zzzSchnitzler, Arthur@\emph{von Arthur Schnitzler}!1893-08-141@{{[}14. 8. 1893{]}}|(be}
\toendnotes[C]{\smallbreak\pagebreak[2]}\Standort{Wienbibliothek im Rathaus, ZPH 1681, 2.1.516.}
\physDesc{Brief, 2 Blätter, 8 Seiten, 2099 Zeichen (Briefpapier mit Trauerrand)
\newline{}Handschrift: Bleistift, deutsche Kurrent
\newline{}Ordnung: mit Bleistift von unbekannter Hand Nummerierung der Doppelseiten des Konvoluts:
                                    »7«–»10« }
\buchAbdrucke{\weitereDrucke{Arthur Schnitzler: \emph{Briefe 1875–1912}. Hg. Therese Nickl und Heinrich Schnitzler. Frankfurt am Main: \emph{S. Fischer} 1981, S. 211–212.} }\toendnotes[C]{\smallbreak}
\pstart
           \noindent{}{\pb}\label{K_L02960-1v}\edtext{Bei der »\textcolor{pink}{ſchönen Ausſicht}{}\ledrightnote{\textcolor{pink}{Diglas’ Restaurant »Zur schönen Aussicht«}}«}{\lemma{\textnormal{\emph{Bei … Ausſicht«}}}\Cendnote{\textnormal{Der Brief ist ungewöhnlich, da er weder eine Andrede, noch
                  eine Unterschrift aufweist. Das ließe sich damit erklären, dass \textcolor{blue}{Schnitzler} das Schreiben nicht auf dem üblichen Postweg
                  versandte, sondern als offenes Schreiben jemandem mitgab. Ob das der Fall war,
                  lässt sich wegen des fehlenden Umschlags nicht bestimmen.}}}\label{K_L02960-1h} – in \textcolor{pink}{Döbling}{}\ledrightnote{\textcolor{pink}{XIX., Döbling}} – dort, bei der Buche, lehnt mein Rad. –
               Sehr, ſehr, ſehr allein. – Unten die dunkle \textcolor{pink}{Stadt}{}\ledrightnote{{$\rightarrow$}\textcolor{pink}{Wien}} und die Lichter von den fernen Landſtraßen. Um mich
               nachtmahlende recht vergnügte Bürger, ſpärlich eigentlich. – Es iſt gegen
                  neun, u ich \label{K_L02960-2v}\edtext{halte bei
               der Virginier}{\lemma{\textnormal{\emph{halte bei
               der Virginier}}}\Cendnote{\textnormal{er unterbricht das Rauchen
                  seiner Zigarre}}}\label{K_L02960-2h}. Da ich beim Schein der Gartenlaterne {\pb}einen Brief ſchreibe, dürfte ich für einen
               begabten Selbſtmörder gehalten werden. – Hergeko{\geminationm}en über
               einige unwahrſcheinliche Ortſchaften – mit einem Wort: \textcolor{pink}{Heiligenſtadt}{}\ledrightnote{\textcolor{pink}{Heiligenstadt}}. War in \textcolor{pink}{Kloſterneuburg}{}\ledrightnote{\textcolor{pink}{Klosterneuburg}};
               Bei Gelegenheit meines verbogenen Pedales eine herrliche jüdiſche Schloſſerfamilie
                  {\pb}ſtudirt. »Wunderſchön«\footnote{\noindent{}Salten.–}, wie plötzlich zwei ältere jüdiſche \textcolor{pink}{Kloſterneuburg}{}\ledrightnote{\textcolor{pink}{Klosterneuburg}}. »\label{K_L02960-3v}\edtext{Gig\textcolor{gray}{o}hl}{\lemma{\textnormal{\emph{Gigohl}}}\Cendnote{\textnormal{womöglich
                  ein Dialektausdruck für ›Gigerl‹ (Modenarr, Dandy)}}}\label{K_L02960-3h}« bei
                  d\textcolor{gray}{er} Thür erſcheinen {\kaufmannsund}
                  de\textcolor{gray}{m}
               barfußen Schloſſer ſagten, »Nü,
                  \textcolor{gray}{M}äxel, was is mit ä Tarotpartie?« und die 16jährige \label{K_L02960-4v}\edtext{Tochter, die mich offenbar ſofort richtig
                  taxirte, bemerkte
                  »\textcolor{green}{Klabriaspartie}{}\ledrightnote{\textcolor{green}{Eine Partie Klabrias im Café Spitzer}}}{\lemma{\textnormal{\emph{Tochter, … »Klabriaspartie}}}\Cendnote{\textnormal{Sie dürfte Einvernehmen
                     herstellen, dass es sich hier um eine Anspielung auf die (jüdische) Erfolgsposse
                     \emph{\textcolor{green}{Eine Partie Klabrias}} handelte. \textcolor{blue}{Heinrich Schnitzler} kommentierte im Erstdruck diese Stelle
                     mit einem beliebten Ausspruch seines Vaters »Zitate sind entweder aus \textcolor{green}{Faust} oder
                        aus der \textcolor{green}{Klabriaspartie}.«}}}\label{K_L02960-4h}!«\pend
           
\pstart
           {\pb}– Eben \substVorne{}\textsuperscript{machte}{\allowbreak}\substDazwischen{}trank\substHinten{} ich wieder einen Schluck Bier {\kaufmannsund} bemerke meine
               Einſamkeit. Ich lüge mir ſoeben vor, daſs ich begi{\geminationn}e,
               philoſophiſch und gleichgilt\textcolor{gray}{i}g zu werden – gegen »\label{K_L02960-5v}\edtext{\textcolor{green}{all d\textcolor{gray}{en}{ }Tand, der uns von draußen ko{\geminationm}t}{}\ledrightnote{{$\rightarrow$}\textcolor{green}{Abschiedssouper}} –}{\lemma{\textnormal{\emph{all … –}}}\Cendnote{\textnormal{Selbstzitat aus \emph{\textcolor{green}{Abschiedssouper}}, »Als
                     wenn es keine Feierlichkeiten der Seele gäbe, die mit all’ dieſem Tand, der uns
                     von dem Draußen kommt, gar nichts zu thun haben –«}}}\label{K_L02960-5h}« Frl. \textcolor{blue}{G.}{}\ledrightnote{\textcolor{blue}{Marie Glümer}} war 2 oder 3 mal da; und es war wie i{\geminationm}er; – ich hab nie geahnt, daſs Weiber wegen ein u
               derſelben Sache \uline{ſo}{ }{\pb}viel Thränen haben! – Von \textsc{\textcolor{blue}{Blumenthal}{}\ledrightnote{\textcolor{blue}{Oskar Blumenthal}}} kam geſtern ein \label{K_L02960-6v}\edtext{Brief}{\lemma{\textnormal{\emph{Brief}}}\Cendnote{\textnormal{Oscar Blumenthal an Arthur Schnitzler, 12. 8. 1893}}}\label{K_L02960-6h} mit vertröſtenden Phraſen. – Merken Sie, Goldchnittpapier? Ich glaube, Frl.
                  \textsc{\textcolor{blue}{Diglas}{}\ledrightnote{\textcolor{blue}{Antonie Cuny-Pierron}}} hat es dem Kellner zur Verfügung geſtellt.– \pend
           
\pstart
           – \textcolor{blue}{Goldma{\geminationn}}{}\ledrightnote{\textcolor{blue}{Paul Goldmann}} ko{\geminationm}t wahrſcheinlich \label{K_L02960-7v}\edtext{Anfang September nach \textsc{\textcolor{pink}{Salzburg}{}\ledrightnote{\textcolor{pink}{Salzburg}}}}{\lemma{\textnormal{\emph{Anfang … Salzburg}}}\Cendnote{\textnormal{siehe Paul Goldmann an Arthur Schnitzler, 18. 8. [1893]}}}\label{K_L02960-7h}, ich ſchreib ihm – Ende {\pb}Auguſt. Bitte ſa{\geminationm}eln Sie
               nähere Daten über unſre Partie u. entſchließen Sie ſich zu einem ausführlichen
               Schreiben. –\pend
           
\pstart
           – Nun fahr ich hinein, \label{K_L02960-8v}\edtext{morgen in die \textcolor{pink}{Brühl}{}\ledrightnote{\textcolor{pink}{Brühl}}, übermorgen zur »\textcolor{blue}{Liebſten}{}\ledrightnote{\textcolor{blue}{Josefine Lydia von Weisswasser}}«}{\lemma{\textnormal{\emph{morgen … »Liebſten«}}}\Cendnote{\textnormal{siehe A. S.: \emph{Tagebuch}, 15. 8. 1893 und 16. 8. 1893}}}\label{K_L02960-8h}, hihihihihihihihihihi!\pend
           
\pstart
           {\pb}Geſtern war ich \textsc{per} Bic
                  (\label{K_L02960-9v}\edtext{Reichſtraße}{\lemma{\textnormal{\emph{Reichſtraße}}}\Cendnote{\textnormal{Fernstraße}}}\label{K_L02960-9h}) \textcolor{pink}{Baden}{}\ledrightnote{\textcolor{pink}{Baden bei Wien}};
               wurde ſehr ſehnſüchtig u jung \label{K_L02960-10v}\edtext{\textcolor{blue}{geliebt}{}\ledrightnote{{$\rightarrow$}\textcolor{blue}{Else Berger}}}{\lemma{\textnormal{\emph{geliebt}}}\Cendnote{\textnormal{siehe A. S.: \emph{Tagebuch}, 13. 8. 1893}}}\label{K_L02960-10h}. Sonderbar!x in demſelben Garten, in dem ich vor etwa 7 Jahren ein junges
                  \label{K_L02960-11v}\edtext{\textcolor{blue}{Mädel}{}\ledrightnote{{$\rightarrow$}\textcolor{blue}{Gisela Fischer}}}{\lemma{\textnormal{\emph{Mädel}}}\Cendnote{\textnormal{siehe A. S.: \emph{Tagebuch}, 12. 8. 1886}}}\label{K_L02960-11h} wahnſi{\geminationn}ig »herzte« u küſſte, das jetzt längſt
               verheiratet iſt – bis hundert Jahr.\pend
           
\pstart
           {\pb}Wa{\geminationn} ich
               wegfahre, weiſs ich noch nicht. Wohl \label{K_L02960-12v}\edtext{So{\geminationn}tag}{\lemma{\textnormal{\emph{Sonntag}}}\Cendnote{\textnormal{\textcolor{blue}{Schnitzler} reiste am Dienstag, 22. 8. 1893, aus \textcolor{pink}{Wien} ab.}}}\label{K_L02960-12h}. –\pend
           
\pstart
           Leben Sie wohl, ſchreiben Sie was ſchönes und grüßen Sie mir die »wackern« \label{K_L02960-13v}\edtext{\textcolor{pink}{Linz}{}\ledrightnote{\textcolor{pink}{Linz}}er Radfahrer}{\lemma{\textnormal{\emph{Linzer Radfahrer}}}\Cendnote{\textnormal{Er dürfte wohl eher die \textcolor{pink}{Lienzer} Radfahrer meinen, vgl. Felix Salten an Arthur Schnitzler, 12. 8. 1893.}}}\label{K_L02960-13h}.\pend
           
\pstart
           All heil! –\pend
           
\pstart
           \noindent{}\label{T_L02960-1v}\edtext{Nach Schluſs – Eben ging Hr \textcolor{blue}{P.}{}\ledrightnote{\textcolor{blue}{Rudolf Eduard von Cuny-Pierron}}{ }\textsc{\label{K_L02960-14v}\edtext{\begin{otherlanguage}{french}l’amant de\end{otherlanguage}}{\lemma{\textnormal{\emph{l’amant de}}}\Cendnote{\textnormal{französisch: Liebhaber von}}}\label{K_L02960-14h}
                     M \textcolor{blue}{A. D.}{}\ledrightnote{\textcolor{blue}{Antonie Cuny-Pierron}}} an mir vorbei, \label{K_L02960-15v}\edtext{\begin{otherlanguage}{french}Cretin\end{otherlanguage}}{\lemma{\textnormal{\emph{Cretin}}}\Cendnote{\textnormal{französisch: Dummkopf,
                     Idiot}}}\label{K_L02960-15h}!}{\lemma{\textnormal{\emph{Nach … Cretin!}}}\Cendnote{\textnormal{in einem gezeichneten
                     Kasten quer zum Text}}}\label{T_L02960-1h}\pend
           \endnumbering\briefempfaengerindex{Salten, Felix@\textsc{Salten, Felix}!zzzSchnitzler, Arthur@\emph{von Arthur Schnitzler}!1893-08-141@{{[}14. 8. 1893{]}}|)be}\mylabel{h}  \normalsize

\doendnotes{C}
\bigskip
\vfill

\clearpage

\footnotesize

\lohead{\textsc{register}}

% Definiere theindex-Environment komplett neu ohne reledmac
\makeatletter
\renewenvironment{theindex}{%
  \section*{\indexname}%
  \setlength{\parindent}{0pt}%
  \setlength{\parskip}{0pt plus 0.3pt}%
  \let\item\@idxitem
}{%
  \clearpage
}
\makeatother

\IfFileExists{\jobname-pw.ind}{\input{\jobname-pw.ind}}{}

\end{document}

      