%% latex-korrekturansicht-vorspann.tex
%% Vorspann für die Korrekturansicht.
%% Lädt die gemeinsame Datei latex-vorspann.tex mit gesetztem Schalter.

\newif\ifkorrekturansicht
\korrekturansichttrue

\input{../tex-inputs/latex-vorspann}


               \section[Paul Goldmann an Arthur Schnitzler, 18. 8. {[}1893{]}]{ Paul Goldmann an Arthur Schnitzler, 18. 8. {[}1893{]}}\nopagebreak\mylabel{v}\rehead{ }\normalsize\beginnumbering\briefempfaengerindex{Schnitzler, Arthur@\textsc{Schnitzler, Arthur}!zzzGoldmann, Paul@\emph{von Paul Goldmann}!1893-08-182@{18. 8. {[}1893{]}}|(be} \toendnotes[C]{\smallbreak\pagebreak[2]} \Standort{DLA, A:Schnitzler, HS.NZ85.1.3163.}
\physDesc{Brief, 2 Blätter, 8 Seiten
\newline{}Handschrift: schwarze Tinte, deutsche Kurrent
\newline{}Schnitzler: 1) mit Bleistift das Jahr »93« vermerkt 2) mit rotem Buntstift drei Unterstreichungen}\toendnotes[C]{\smallbreak}\pstart
           \noindent{}{\pb}\textcolor{gray}{\textbf{\textbf{\textcolor{brown}{Frankfurter Zeitung}{}\ledrightnote{\textcolor{brown}{Frankfurter Zeitung}}.}}}\pend
           \pstart
           \textcolor{gray}{\textbf{\textbf{(\textcolor{brown}{\begin{otherlanguage}{french}Gazette de Francfort\end{otherlanguage}}{}\ledrightnote{\textcolor{brown}{Frankfurter Zeitung}}.)}}}\pend
           \pstart
           \textcolor{gray}{\textbf{\begin{otherlanguage}{french}\textcolor{blue}{Directeur}{}\ledrightnote{→\textcolor{blue}{Leopold Sonnemann}}\end{otherlanguage}{ }\textbf{M. \textcolor{blue}{L. Sonnemann}{}\ledrightnote{\textcolor{blue}{Leopold Sonnemann}}.}}}\hfill \textsc{\textcolor{pink}{Paris}{}\ledrightnote{\textcolor{pink}{Paris}}}, 18. August.\pend
           \pstart
           \begin{otherlanguage}{french}\textcolor{gray}{\textbf{\textcolor{green}{Journal}{}\ledrightnote{\textcolor{green}{Frankfurter Zeitung}} politique, financier,}}\end{otherlanguage}\pend
           \pstart
           \begin{otherlanguage}{french}\textcolor{gray}{\textbf{commercial et litteraire.}}\end{otherlanguage}\pend
           \pstart
           \begin{otherlanguage}{french}\textcolor{gray}{\textbf{\textbf{Paraissant trois fois par jour}}}\end{otherlanguage}\pend
           \pstart
           \begin{otherlanguage}{french}\textcolor{gray}{\textbf{\textbf{Bureaux à \textcolor{pink}{Paris}{}\ledrightnote{\textcolor{pink}{Paris}}:}}}\end{otherlanguage}\pend
           \pstart
           \begin{otherlanguage}{french}\textcolor{gray}{\textbf{\textbf{\textcolor{pink}{rue Richelieu 75}{}\ledrightnote{\textcolor{pink}{rue Richelieu}}.}}}\end{otherlanguage}\pend
           \pstart\center{}Mein lieber Arthur!\pend\pstart
           Ich habe Dir nicht ſofort geantwortet, weil ich erſt die Antwort des \textcolor{blue}{H. \textsc{Sonnemann}}{}\ledrightnote{\textcolor{blue}{Leopold Sonnemann}}, meines \textcolor{blue}{Chef}{}\ledrightnote{→\textcolor{blue}{Leopold Sonnemann}}s,
               betreffend meinen Urlaub abwarten und Dir Beſtimmtes über meine Reiſepläne mittheilen
               wollte. Bis jetzt iſt noch nichts gekommen, und ich will nun die Antwort auf Deine
               lieben Zeilen nicht länger verſchieben. Aus der Verzögerung der Antwort des \textcolor{blue}{Chef}{}\ledrightnote{→\textcolor{blue}{Leopold Sonnemann}}s ſchließe ich, daß meine
               Bitte um ſofortige Beurlaubung nicht bewilligt werden und daß ich genöthigt werden
               dürfte, bis nach den \label{K_L02712-1v}\edtext{Stichwahlen}{\lemma{\textnormal{\emph{Stichwahlen}}}\Cendnote{\textnormal{In \textcolor{pink}{Frankreich} wurde am 20. 8. 1893 ein neues
                  Parlament gewählt. Am 3. 9. 1893 gewann \textcolor{blue}{Jean Casimir-Perier} die Stichwahl gegen \textcolor{blue}{Georges Clemenceau}.}}}\label{K_L02712-1h}{ }{\pb} – 3. September – zu
               bleiben. Dann komme ich höchſtwahrſcheinlich im Lauf des September nach \textsc{\textcolor{pink}{Salzburg}{}\ledrightnote{\textcolor{pink}{Salzburg}}}, und falls Du \label{K_L02712-2v}\edtext{verreiſt}{\lemma{\textnormal{\emph{verreiſt}}}\Cendnote{\textnormal{Im Sommer, nach dem 18. 8. 1893, verreiste \textcolor{blue}{Schnitzler}
                  vom 22. 8. 1893 bis zum
                     31. 8. 1893 nach \textcolor{pink}{Tirol}, \textcolor{pink}{Südtirol}, \textcolor{pink}{Italien}, \textcolor{pink}{Kärnten}, \textcolor{pink}{Niederösterreich} und in die \textcolor{pink}{Steiermark}. Am 5. 9. 1893 und von 9. 9. 1893 bis 11. 9. 1893 war \textcolor{blue}{Schnitzler}
                  außerdem in \textcolor{pink}{Reichenau an der Rax}, von 16. 9. 1893 bis 19. 9. 1893 in \textcolor{pink}{Salzburg}, wo er jedenfalls am 17. 9. 1893 und 18. 9. 1893{ }\textcolor{blue}{Goldmann} traf. Ein damit einhergehendes
                  Zusammentreffen mit \textcolor{blue}{Hugo von Hofmannsthal}
                  und \textcolor{blue}{Richard Beer-Hofmann} ist nicht
                  bekannt.}}}\label{K_L02712-2h}, bitte ich Dich, mir jetzt noch raſch eine Adreſſe mitzutheilen,
               wo Dich ein Telegramm oder ein Brief von mir erreichen kann. Ich kann Dir gar nicht
               ſagen, wie unendlich ich mich auf ein Wiederſehen mit Dir freue. Aber ich bitte Dich
               nochmals dringend, Dich auf Enttäuſchungen vorzubereiten. Ich habe mich nicht zu
               meinem Vortheil verändert.\pend
           \pstart
           Was Du ſonſt über die Beziehungen zwiſchen Dir und mir ſchreibſt, iſt lieb und gut
               und hat mir aufrichtig wohlgethan. Aber wenn Du einen Ton des Zweifels bei {\pb}mir bemerkſt – ich glaube allerdings, Du haſt
               Unrecht, – trägſt Du nicht auch eine Schuld? Denk’ Dir nur, was Du mir während dieſer
               Jahre geſchrieben haſt und was nicht. Du haſt mich einzig und allein an Deinem
               literariſchen Leben theilnehmen laſſen. Aber von Deinem Perſönlichen, was mir doch
               bei allem Intereſſe für das Erſte das unendlich Werthvollere iſt, weiß ich rein gar
               nichts mehr. Höchſtens hier und da eine Andeutung, es ſei Dir unmöglich, über ſolche
               Dinge zu ſchreiben. Und da ich weiß, daß Du mir ähnlich biſt, und da ich mich kenne,
               wie ich das Wort »unmöglich« gebrauche, weil es ſchöner klingt als »unbequem«, {\pb}wie es doch eigentlich heißen ſollte, – ſo habe ich
               manchmal Reflexionen darüber gemacht – nicht bittere, aber ſchmerzliche. Nun, das
               ſoll ſich wohl Alles jetzt wieder ausgleichen. Auch Deine Bitterkeit gegen mich. Denn
               bei aller Feinheit des Taktes, bei alle\substVorne{}\textsuperscript{\textcolor{gray}{n}}\substDazwischen{}m\substHinten{} noblen Wunſch, ſie zurückzudrängen, klingt ſie in Deinen Briefen durch, und
               ich glaube, immer zu leſen: Nicht einmal \label{K_L02712-33v}\edtext{eine Beſprechung}{\lemma{\textnormal{\emph{eine Beſprechung}}}\Cendnote{\textnormal{von \emph{\textcolor{green}{Anatol}}}}}\label{K_L02712-33h} in der \textcolor{green}{Frankfurter Zeitung}{}\ledrightnote{\textcolor{green}{Frankfurter Zeitung}} hat er mir
               geliefert! Da habe ich wirklich große Schuld. Ich weiß wohl, daß ich nicht gekonnt
               habe. Aber wenn ich ſo zurückdenke, habe ich keine Ahnung, wie das ſo eigentlich {\pb}gekommen iſt. Ich meine, es war doch viel
               Willensſchwäche von meiner Seite dabei. Aber auch darüber wollen wir reden. Über
               Deine ſonſtigen Autoren-Leiden, mein liebſter Arthur, \strikeout{f\textcolor{gray}{×}\-\textcolor{gray}{×}} haſt Du keinen Grund, Dich beſonders traurig zu fühlen. Das gehört dazu, ich
               ſchwöre es Dir, und iſt nur eine zurückzulegende Etape. In \textsc{\textcolor{pink}{Paris}{}\ledrightnote{\textcolor{pink}{Paris}}} iſt doch das geiſtige Leben noch ganz anders entwickelt als in \textcolor{pink}{Deutſchland}{}\ledrightnote{\textcolor{pink}{Deutschland}} und \textcolor{pink}{Öſterreich}{}\ledrightnote{\textcolor{pink}{Österreich}}, ich meine in Bezug auf die \label{K_L02712-3v}\edtext{Zahl der jährlich geſchriebenen {\pb}und gedruckten Werke}{\lemma{\textnormal{\emph{Zahl … Werke}}}\Cendnote{\textnormal{Was die jährlichen Drucke im internationalen Vergleich
                  anbelangt, gibt eine \textcolor{green}{Statistik} aus dem Jahr 1895 Aufschluss:
                        »{[}Es{]} existieren zur Zeit 3985 Papierfabriken auf
                     der Erde, deren Gesammtproduktion sich auf 7904 Millionen Buch im Jahre
                     beläuft. Die Hälfte dieses riesigen Papiermaterials verbraucht die
                     Buchdruckerei, während 600 Millionen Buch auf die Zeitungen entfallen. Per Kopf
                     berechnet verbraucht der \textcolor{pink}{Engländ}er von allen Nationen am meisten Papier, nämlich 11 ½ Buch im
                     Durchschnitt pro Jahr. Nach ihm kommt der \textcolor{pink}{Amerika}ner mit  10¼ Buch pro Jahr und Kopf. Hierauf der \textcolor{pink}{Deutſche} mit 8 und der \textcolor{pink}{Franzose} mit 7 ½  Buch. Weitaus weniger
                     konsumiren \textcolor{pink}{Oesterreich} und \textcolor{pink}{Italien} an Papier, da bei beiden Nationen
                     die durchschnittliche Ziffer pro Jahr und Kopf nur 3 ½ Buch beträgt. Zum Schluß
                     kommt der \textcolor{pink}{Mexikaner}
                     mit 2, der \textcolor{pink}{Spanier}
                     mit 1 ½ und als letzter der \textcolor{pink}{Russe} mit gar nur 1 ⅝ Buch Papier, welches pro Jahr auf den Einwohner
                     entfällt.« ([O. V.]: \emph{\textcolor{green}{Vermischtes}}. In: \emph{\textcolor{green}{Vorwärts}},
                     Jg. 12, Nr. 191, 17. 8. 1895,
                  S. 7.)}}}\label{K_L02712-3h}. Und was ich da ſo über Dummheit und Gemeinheit von Verlegern
               erzählen höre. Ein anderes Beiſpiel: Hier lebt \label{K_L02712-4v}\edtext{\textsc{\textcolor{blue}{Knut Hamsun}{}\ledrightnote{\textcolor{blue}{Knut Hamsun}}}}{\lemma{\textnormal{\emph{Knut Hamsun}}}\Cendnote{\textnormal{Durch seinen Roman \textcolor{green}{Hunger} (norweg. \emph{\textcolor{green}{Sult}}, 1890) berühmt geworden, lebte \textcolor{blue}{Knut Hamsun} zwischen 1893 und
                     1895 an der Adresse \textcolor{pink}{8 rue de
                     Vaurigard} in Paris.}}}\label{K_L02712-4h}, deſſen glänzendes Talent Du doch kennſt. Seit
               Jahresfriſt muß er mit zwei neuen \textcolor{green}{Romanen}{}\ledrightnote{→\textcolor{green}{Neue Erde. Roman}{\newline}→\textcolor{green}{Mysterien. Roman}}, deren \strikeout{Ein\textcolor{gray}{e}}{ }\label{K_L02712-10v}\edtext{einen}{\lemma{\textnormal{\emph{einen}}}\Cendnote{\textnormal{nicht rekonstruierbar}}}\label{K_L02712-10h} mein \textcolor{blue}{Onkel}{}\ledrightnote{→\textcolor{blue}{Fedor Mamroth}} geſehen hat und auch als höchſt
               bedeutend bezeichnet – er hat ihn aus demſelben Grunde nicht drucken können wie den
                  \label{K_L02712-11v}\edtext{\textcolor{green}{Deinen}{}\ledrightnote{→\textcolor{green}{Sterben. Novelle}}}{\lemma{\textnormal{\emph{Deinen}}}\Cendnote{\textnormal{siehe Fedor Mamroth an Arthur Schnitzler, 4. 6. 1893}}}\label{K_L02712-11h} – muß alſo bei allen deutſchen Verlegern hauſiren gehen, findet nicht \uline{einen}, lebt nur durch die Wohlthat zweier \label{K_L02712-12v}\edtext{\textsc{\textcolor{blue}{Mäcene}{}\ledrightnote{→\textcolor{blue}{Albert Langen}}}}{\lemma{\textnormal{\emph{Mäcene}}}\Cendnote{\textnormal{Es dürfte sich um \textcolor{blue}{Willy Gretor} und \textcolor{blue}{Albert
                     Langen} handeln. \textcolor{blue}{Langen} hatte zuerst
                  dem \emph{\textcolor{brown}{S. Fischer-Verlag}} eine Kostenbeteiligung
                  für den Abdruck von \textcolor{blue}{Hamsun}s \emph{\textcolor{green}{Mysterien}} angeboten und, nach der Ablehnung, dafür
                     1894 einen eigenen \emph{\textcolor{brown}{Verlag}}
                  gegründet. Hier erschien im selben Jahr auch der Roman \emph{\textcolor{green}{Neue Erde}}.}}}\label{K_L02712-12h} und wird ſeine \textcolor{green}{Bücher}{}\ledrightnote{→\textcolor{green}{Neue Erde. Roman}{\newline}→\textcolor{green}{Mysterien. Roman}} nur
               publiciren können, wenn ihm die \textcolor{blue}{Letzteren}{}\ledrightnote{→\textcolor{blue}{Albert Langen}}{ }{\pb}Geld leihen, um ſie im \textcolor{brown}{Selbſtverlag}{}\ledrightnote{→\textcolor{brown}{Albert Langen}} erſcheinen zu laſſen. Dein \textsc{\textcolor{green}{Anatol}{}\ledrightnote{\textcolor{green}{Anatol}}} wird meiner Anſicht nach ſehr gekauft werden, wenn Du erſt einen \label{K_L02712-5v}\edtext{Bühnenerfolg haben wirſt}{\lemma{\textnormal{\emph{Bühnenerfolg haben wirſt}}}\Cendnote{\textnormal{Die erste vollständige Aufführung des \emph{\textcolor{green}{Anatol-Zyklus}} fand erst am 3. 12. 1910 statt
                  (doppelte Uraufführung am \emph{\textcolor{brown}{Lessing-Theater}} in
                     \textcolor{pink}{Berlin} und am \emph{\textcolor{brown}{Deutschen Volkstheater}} in \textcolor{pink}{Wien}). Neue Auflagen des \textcolor{green}{Zyklus} gab es jedoch schon ab 1895 bei \emph{\textcolor{brown}{S. Fischer}}.}}}\label{K_L02712-5h}.
                  \label{K_L02712-6v}\edtext{\textsc{\textcolor{blue}{Sudermann}{}\ledrightnote{\textcolor{blue}{Hermann Sudermann}}s} Romane}{\lemma{\textnormal{\emph{Sudermanns Romane}}}\Cendnote{\textnormal{\textcolor{blue}{Hermann Sudermann} wagte bereits in den
                     1870er-Jahren erste literarische Versuche,
                  veröffentlichte jedoch erst 1886 die Novellensammlung \emph{\textcolor{green}{Im Zwielicht}} und 1887
                  seinen ersten \textcolor{green}{Roman}{ }\emph{\textcolor{green}{Frau Sorge}}. Einen großen Erfolg feierte dann
                  das am 29. 11. 1889 am \emph{\textcolor{brown}{Lessing-Theater}} uraufgeführte \textcolor{green}{Stück}{ }\emph{\textcolor{green}{Die Ehre}}.}}}\label{K_L02712-6h} haben ſich Jahre lang
               unbeachtet herumgeſeilt, und jetzt kann man nicht genug davon kriegen. Alſo nur ein
               wenig Geduld, liebſter Freund, und Alles wird gehen. Eine Aufführung im \textcolor{brown}{Volkstheater}{}\ledrightnote{\textcolor{brown}{Volkstheater}} würd\textcolor{gray}{e} ich an
               Deiner Stelle nur annehmen, wenn das \textcolor{green}{Stück}{}\ledrightnote{→\textcolor{green}{Anatol}} bereits in \textcolor{pink}{Deutſchland}{}\ledrightnote{\textcolor{pink}{Deutschland}} geſpielt wäre. Denn in \textsc{\textcolor{pink}{Wien}{}\ledrightnote{\textcolor{pink}{Wien}}} zum überhaupt erſten Mal geſpielt zu werden, bei dieſer irrſinnig dummen Kritik
                  {\pb}und noch dazu in dieſem vollſtändig \label{K_L02712-8v}\edtext{unkünſtleriſch geleiteten \textcolor{brown}{Theater}{}\ledrightnote{→\textcolor{brown}{Volkstheater}}}{\lemma{\textnormal{\emph{unkünſtleriſch … Theater}}}\Cendnote{\textnormal{Von 1889 bis
                     1905 war \textcolor{blue}{Emerich von
                     Bukovics}{ }\textcolor{blue}{Leiter} des \emph{\textcolor{brown}{Volkstheater}}s. Die Uraufführung von \emph{\textcolor{green}{Das Märchen}} an diesem Theater stand unmittelbar bevor
                     (1. 12. 1893), aber
                  erst jahre später wurden zuerst einzelne Szenen aus \emph{\textcolor{green}{Anatol}} gegeben.}}}\label{K_L02712-8h}, würde ich nicht für zuträglich
               halten. Die Hauptſache iſt, die \textcolor{pink}{Berlin}{}\ledrightnote{\textcolor{pink}{Berlin}}er
               Aufführung zu beſchleunigen, und auch darüber wollen wir gemeinſam Rath halten.\pend
           \pstart
           Grüß’ Dich Gott, mein lieber Arthur! Auf hoffentlich baldiges Wiederſehen!\pend
           \pstart
           Dein treuer {\\[\baselineskip]}\spacefill\mbox{Paul Goldm}\pend
           \leftskip=0em{}\pstart
           Wenn Du es ſo machen könnteſt, daß ich auch \textsc{\textcolor{blue}{Loris}{}\ledrightnote{\textcolor{blue}{Hugo von Hofmannsthal}}} und \textsc{\textcolor{blue}{Richard}{}\ledrightnote{\textcolor{blue}{Richard Beer-Hofmann}}} ſehe, ſo wäre das ganz beſonders herrlich. \textsc{\textcolor{blue}{Loris}{}\ledrightnote{\textcolor{blue}{Hugo von Hofmannsthal}}} hat in der \textcolor{green}{Frkf. Ztg.}{}\ledrightnote{\textcolor{green}{Frankfurter Zeitung}} ein \label{K_L02712-9v}\edtext{stupendes \textcolor{green}{Feuilleton}{}\ledrightnote{→\textcolor{green}{Gabriele d’Annunzio}}}{\lemma{\textnormal{\emph{stupendes Feuilleton}}}\Cendnote{\textnormal{\textcolor{blue}{Loris}: \emph{\textcolor{green}{Gabriele d’Annunzio}}. In: \textcolor{green}{Frankfurter Zeitung}, Jg. 37, Nr. 219,
                        9. 8. 1893, Erstes Morgenblatt, S. 1–3. Darin erörterte
                     \textcolor{blue}{Hugo von Hofmannsthal} den Begriff der
                  (literarischen) »Moderne« am Beispiel von \textcolor{blue}{Gabriele d’Annunzio}. \textcolor{blue}{Goldmann}
                  dürfte den \textcolor{green}{Aufsatz} vor
                  allem aufgrund der darin enthaltenen kontra-naturalistischen Ausführungen gefallen
                  haben.}}}\label{K_L02712-9h} gehabt.\pend
           \endnumbering\briefempfaengerindex{Schnitzler, Arthur@\textsc{Schnitzler, Arthur}!zzzGoldmann, Paul@\emph{von Paul Goldmann}!1893-08-182@{18. 8. {[}1893{]}}|)be}\mylabel{h}  \normalsize

\doendnotes{C}
\bigskip
\vfill

\clearpage

\footnotesize

\lohead{\textsc{register}}

% Definiere theindex-Environment komplett neu ohne reledmac
\makeatletter
\renewenvironment{theindex}{%
  \section*{\indexname}%
  \setlength{\parindent}{0pt}%
  \setlength{\parskip}{0pt plus 0.3pt}%
  \let\item\@idxitem
}{%
  \clearpage
}
\makeatother

\IfFileExists{\jobname-pw.ind}{\input{\jobname-pw.ind}}{}

\end{document}

      