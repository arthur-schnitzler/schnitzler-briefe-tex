%% latex-korrekturansicht-vorspann.tex
%% Vorspann für die Korrekturansicht.
%% Lädt die gemeinsame Datei latex-vorspann.tex mit gesetztem Schalter.

\newif\ifkorrekturansicht
\korrekturansichttrue

\input{../tex-inputs/latex-vorspann}


\section[Arthur Schnitzler an Stefan Zweig, 27. 10. 1912]{L03783 Arthur Schnitzler an Stefan Zweig, 27. 10. 1912}
\nopagebreak\mylabel{L03783v}
\rehead{ }\normalsize\beginnumbering\briefempfaengerindex{Zweig, Stefan@\textsc{Zweig, Stefan}!zzzSchnitzler, Arthur@\emph{von Arthur Schnitzler}!1912-10-271@{27. 10. 1912}|(be}
\toendnotes[C]{\smallbreak\pagebreak[2]}\Standort{Jerusalem, National Library of Israel, ARC. Ms. Var. 305 1 58 Stefan Zweig Collection.}
\physDesc{Briefkarte, 1 Blatt, 2 Seiten, 314 Zeichen
\newline{}Handschrift: schwarze Tinte, deutsche Kurrent}\toendnotes[C]{\smallbreak}
\pstart
           {\pb}\textcolor{gray}{\textbf{Dr. Arthur Schnitzler}}\hfill 27. X. 912\pend
           
\pstart
           \textcolor{gray}{\textbf{\textcolor{pink}{Wien XVIII. Sternwartestrasse 71}\oindex{Sternwartestrasse 71@\textbf{Sternwartestraße 71}|pw}{}\ledrightnote{\textcolor{pink}{Sternwartestraße 71}}}}\pend
           \vspace{0.5em}
\pstart
           lieber Doctor Zweig, ich freue mich über Ihren \label{K_L03783-1v}\edtext{\textcolor{green}{\textcolor{violet}{Erfolg}\eventindex{Burgtheater@\textbf{Burgtheater}!Generalprobe von Das Haus am Meer, 25.10.1912@Generalprobe von Das Haus am Meer, 25.10.1912|pwv}{}\ledrightnote{{$\rightarrow$}\emph{\textcolor{violet}{Generalprobe von Das Haus am Meer, 25.10.1912}}}}\pwindex{Haus am Meer. Ein Schauspiel in zwei Teilen (drei Aufzuegen)@\emph{Das Haus am Meer. Ein Schauspiel in zwei Teilen (drei Aufzügen)}|pwv}{}\ledrightnote{{$\rightarrow$}\emph{\textcolor{green}{Das Haus am Meer. Ein Schauspiel in zwei Teilen (drei Aufzügen)}}}}{\lemma{\textnormal{\emph{Erfolg}}}\Cendnote{\textnormal{Am 25. 10. 1912 besuchte \textcolor{blue}{Schnitzler} die \textcolor{violet}{Generalprobe}\eventindex{Burgtheater@\textbf{Burgtheater}!Generalprobe von Das Haus am Meer, 25.10.1912@Generalprobe von Das Haus am Meer, 25.10.1912|pwkv} von \emph{\textcolor{green}{Das
                     Haus am Meer}\pwindex{Haus am Meer. Ein Schauspiel in zwei Teilen (drei Aufzuegen)@\emph{Das Haus am Meer. Ein Schauspiel in zwei Teilen (drei Aufzügen)}|pwk}} im \textcolor{pink}{Burgtheater}\oindex{Burgtheater@\textbf{Burgtheater}|pwk}. Am
                  Folgetag fand die \textcolor{violet}{Uraufführung}\eventindex{Burgtheater@\textbf{Burgtheater}!Urauffuehrung von Das Haus am Meer, 26.10.1912@Uraufführung von Das Haus am Meer, 26.10.1912|pwkv} statt, auf deren Rezeption \textcolor{blue}{Schnitzler} hier reagiert.}}}\label{K_L03783-1} und beglückwünſche Sie herzlich. Der \textcolor{green}{erſte}\pwindex{Haus am Meer. Ein Schauspiel in zwei Teilen (drei Aufzuegen)@\emph{Das Haus am Meer. Ein Schauspiel in zwei Teilen (drei Aufzügen)}|pwv}{}\ledrightnote{{$\rightarrow$}\emph{\textcolor{green}{Das Haus am Meer. Ein Schauspiel in zwei Teilen (drei Aufzügen)}}} und zweite \textcolor{green}{Akt}\pwindex{Haus am Meer. Ein Schauspiel in zwei Teilen (drei Aufzuegen)@\emph{Das Haus am Meer. Ein Schauspiel in zwei Teilen (drei Aufzügen)}|pwv}{}\ledrightnote{{$\rightarrow$}\emph{\textcolor{green}{Das Haus am Meer. Ein Schauspiel in zwei Teilen (drei Aufzügen)}}} haben auch von der Bühne
               her ſtark auf mich {\pb}gewirkt; was ich etwa hinſichtlich des
                  \textcolor{green}{dritten}\pwindex{Haus am Meer. Ein Schauspiel in zwei Teilen (drei Aufzuegen)@\emph{Das Haus am Meer. Ein Schauspiel in zwei Teilen (drei Aufzügen)}|pwv}{}\ledrightnote{{$\rightarrow$}\emph{\textcolor{green}{Das Haus am Meer. Ein Schauspiel in zwei Teilen (drei Aufzügen)}}} auf dem Herzen
               hätte, darf ich Ihnen vielleicht bei \label{K_L03783-2v}\edtext{Gelegenheit}{\lemma{\textnormal{\emph{Gelegenheit}}}\Cendnote{\textnormal{Das nächste
                  nachgewiesene Zusammentreffen fand beim \emph{\textcolor{violet}{Hauptmann-Bankett}\eventindex{Oesterreichischer Ingenieur- und Architektenverein@\textbf{Österreichischer Ingenieur- und Architektenverein}!Hauptmann-Bankett der Concordia, 17.11.1912@Hauptmann-Bankett der Concordia, 17.11.1912|pwk}} am 17. 11. 1912 statt.}}}\label{K_L03783-2} ſagen.\pend
           
\pstart
           Auf baldgs Wiederſehen.{\\[\baselineskip]}Ihr{\\[\baselineskip]}\spacefill\mbox{Arthur Schnitzler}\pend
           \leftskip=0em{}\selectlanguage{ngerman}\endnumbering\briefempfaengerindex{Zweig, Stefan@\textsc{Zweig, Stefan}!zzzSchnitzler, Arthur@\emph{von Arthur Schnitzler}!1912-10-271@{27. 10. 1912}|)be}\mylabel{L03783h}
\begin{anhang}
\end{anhang}\normalsize

\doendnotes{C}
\bigskip
\vfill

\clearpage

\footnotesize

\lohead{\textsc{register}}

% Definiere theindex-Environment komplett neu ohne reledmac
\makeatletter
\renewenvironment{theindex}{%
  \section*{\indexname}%
  \setlength{\parindent}{0pt}%
  \setlength{\parskip}{0pt plus 0.3pt}%
  \let\item\@idxitem
}{%
  \clearpage
}
\makeatother

\IfFileExists{\jobname-pw.ind}{\input{\jobname-pw.ind}}{}

\end{document}

      