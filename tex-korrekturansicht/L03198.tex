%% latex-korrekturansicht-vorspann.tex
%% Vorspann für die Korrekturansicht.
%% Lädt die gemeinsame Datei latex-vorspann.tex mit gesetztem Schalter.

\newif\ifkorrekturansicht
\korrekturansichttrue

\input{../tex-inputs/latex-vorspann}


\renewcommand{\erwaehntePersonen}{Personen: Richard Beer-Hofmann, Ignacy Daszyński, Felix Dörmann, Hans Jæger, Max Koch, Fedor Mamroth, Géza von Mattachich, Heinrich Meyer-Benfey, Olga Schnitzler, Elisabeth Steinrück}
\renewcommand{\erwaehnteInstitutionen}{Institutionen: Die Zeit, Die Zeit. Wiener Wochenschrift, Houghton Library, Neue Freie Presse, Wiener Verlag}
\renewcommand{\erwaehnteOrte}{Orte: Berlin, Dessauer Straße, Deutschland, Hinterbrühl, Salzburg, Sächsische Schweiz, Südtirol, Villa in der Hinterbrühl, Wien, Wiesbaden}
\renewcommand{\erwaehnteWerke}{Werke: ?? [Buch über den Talmud], ?? [Kritik zu Lebendige Stunden], Berliner Theater. (»Heilmar« von Wilhelm Kienzl im königlichen Opernhause.), Berliner Theater. »Der Herr von Abadessa« von Felix Dörmann im Königlichen Schauspielhause, Christiania-Bohême, Der Herr von Abadessa. Ein Abenteurerstreich in drei Akten, Die Zeit. Wiener Wochenschrift, Fra Kristiania-Bohêmen, Frankfurter Zeitung, Lebendige Stunden. Vier Einakter, Lieutenant Gustl. Novelle, Moderne Religion, Neue Freie Presse, [Rede über die Mattachich-Affaire]}
\section[ Paul Goldmann an Arthur Schnitzler, 25. 2. {[}1902{]}]{Paul Goldmann an Arthur Schnitzler, 25. 2. {[}1902{]}}
\nopagebreak\mylabel{v}
\rehead{ }\normalsize\beginnumbering\briefempfaengerindex{Schnitzler, Arthur@\textsc{Schnitzler, Arthur}!zzzGoldmann, Paul@\emph{von Paul Goldmann}!1902-02-251@{25. 2. {[}1902{]}}|(be}
\toendnotes[C]{\smallbreak\pagebreak[2]}\Standort{DLA, A:Schnitzler, HS.NZ85.1.3172.}
\physDesc{Brief, 2 Blätter, 8 Seiten
\newline{}Handschrift: blaue Tinte, deutsche Kurrent
\newline{}Schnitzler: 1) mit Bleistift das Jahr »{[}1{]}902« vermerkt  2) mit rotem Buntstift fünf Unterstreichungen}\toendnotes[C]{\smallbreak}
\pstart
           \noindent{}\raggedleft{}{\pb}\textcolor{pink}{\textcolor{gray}{\textbf{DESSAUERSTRASSE 19}}}{}\ledrightnote{\textcolor{pink}{Dessauer Straße}}\pend
           
\pstart
           \textcolor{pink}{Berlin}{}\ledrightnote{\textcolor{pink}{Berlin}}, 25. Februar.\pend
           
\pstart{}Mein lieber Freund,\pend
\pstart
           Ich komme leider erſt heut dazu, Deinen lieben Brief
               zu beantworten, der mir große Freude bereitet hat, weil er mir wieder einmal
               eingehenderen Bericht über Dein Ergehen gab. Ich habe eine ganze Woche lang an einem
                  \label{K_L03198-1v}\edtext{\textcolor{green}{Feuilleton}{}\ledrightnote{{$\rightarrow$}\textcolor{green}{Berliner Theater. »Der Herr von Abadessa« von Felix Dörmann im Königlichen Schauspielhause}} über den »\textcolor{green}{Herrn von \textsc{Abadessa}}{}\ledrightnote{\textcolor{green}{Der Herr von Abadessa. Ein Abenteurerstreich in drei Akten}}«}{\lemma{\textnormal{\emph{Feuilleton … Abadessa«}}}\Cendnote{\textnormal{\textcolor{blue}{Paul Goldmann}: \emph{\textcolor{green}{Berliner Theater. »Der Herr von Abadessa« von Felix Dörmann
                        im Königlichen Schauspielhause}}. In: \emph{\textcolor{green}{Neue Freie Presse}}, Nr. 13.472, 25. 2. 1902, Morgenblatt, S. 1–4.}}}\label{K_L03198-1h} (bezüglich deſſen
               ich \label{K_L03198-55v}\edtext{Deine Anſicht}{\lemma{\textnormal{\emph{Deine Anſicht}}}\Cendnote{\textnormal{\textcolor{blue}{Schnitzler} fand es schlecht, vgl. A. S.: \emph{Tagebuch}, 17. 12. 1901}}}\label{K_L03198-55h} vollſtändig theile) geſchrieben und zu nichts Anderem Zeit gefunden. Jetzt
               fürchte ich, daß die Rieſenarbeit vergeblich geweſen iſt, weil ich ſehr ſcharf über
                  \textsc{\textcolor{blue}{Dörmann}{}\ledrightnote{\textcolor{blue}{Felix Dörmann}}} abgeurtheilt habe und weil man mir kaum erlauben {\pb}wird, über einen früheren Mitarbeiter der \textcolor{brown}{N. Fr. Pr.}{}\ledrightnote{\textcolor{brown}{Neue Freie Presse}} ſcharf zu urtheilen.\pend
           
\pstart
           Es freut mich ſehr, zu hören, daß es \label{K_L03198-2v}\edtext{\textsc{\textcolor{blue}{Olga}{}\ledrightnote{\textcolor{blue}{Olga Schnitzler}}} beſſer geht}{\lemma{\textnormal{\emph{Olga beſſer geht}}}\Cendnote{\textnormal{siehe Paul Goldmann an Arthur Schnitzler, 16. 1. [1902]}}}\label{K_L03198-2h}. Nächſtens ſchreibe ich ihr wirklich. Ich zweifle nicht, daß dieſe Ausſicht
               die Beſſerung im Befinden der verehrten \textcolor{blue}{Freundin}{}\ledrightnote{{$\rightarrow$}\textcolor{blue}{Olga Schnitzler}} beſchleunigen wird. Wie unendlich gern ich im März mit \textcolor{blue}{Euch}{}\ledrightnote{{$\rightarrow$}\textcolor{blue}{Olga Schnitzler}} in die \label{K_L03198-3v}\edtext{\textcolor{pink}{Berge}{}\ledrightnote{{$\rightarrow$}\textcolor{pink}{Hinterbrühl}}}{\lemma{\textnormal{\emph{Berge}}}\Cendnote{\textnormal{Mit einigen Unterbrechungen hielten sich
                     \textcolor{blue}{Schnitzler}, die schwangere \textcolor{blue}{Olga Gussmann} und womöglich auch deren
                  Schwester \textcolor{blue}{Elisabeth Gussmann} zwischen 21. 3. 1902 und 31. 3. 1902 in der
                  neuen \textcolor{pink}{Unterkunft} in der
                     \textcolor{pink}{Hinterbrühl} auf. Siehe Paul Goldmann an Arthur Schnitzler, 14. 1. [1902].}}}\label{K_L03198-3h} gehen möchte,
               brauche ich nicht erſt zu ſagen. Ich habe die ganze Reiſe bereits in der Phantaſie
               gemacht und dabei ſehr ſchöne Stunden mit \textcolor{blue}{Euch}{}\ledrightnote{{$\rightarrow$}\textcolor{blue}{Olga Schnitzler}} verlebt. In der Wirklichkeit werde ich ſie nicht machen
               können. Ich könnte höchſtens zu Oſtern ein paar Tage fort. Und der Weg von hier nach
                  \textcolor{pink}{Salzburg}{}\ledrightnote{\textcolor{pink}{Salzburg}} oder gar {\pb}nach \textcolor{pink}{Südtirol}{}\ledrightnote{\textcolor{pink}{Südtirol}}
               iſt für die drei oder vier Tage Urlaub, die ich mir nehmen könnte, allzu weit. Etwas
               Anderes wäre \substVorne{}\textsuperscript{ich}\substDazwischen{}es\substHinten{}, wenn \textcolor{blue}{Ihr}{}\ledrightnote{{$\rightarrow$}\textcolor{blue}{Olga Schnitzler}} nach \textcolor{pink}{Deutſchland}{}\ledrightnote{\textcolor{pink}{Deutschland}} kommen könntet (\textcolor{pink}{Sächſiſche Schweiz}{}\ledrightnote{\textcolor{pink}{Sächsische Schweiz}}\strikeout{,} oder \textcolor{pink}{Wiesbaden}{}\ledrightnote{\textcolor{pink}{Wiesbaden}}). Da könnte ich um Oſtern herum ein paar Tage mit \textcolor{blue}{Euch}{}\ledrightnote{\textcolor{blue}{Olga Schnitzler}} ſein. Aber daran iſt ja wohl kaum zu denken. Ich
               wenigſtens würde ſicher nicht nach \textsc{\textcolor{pink}{Wiesbaden}{}\ledrightnote{\textcolor{pink}{Wiesbaden}}} kommen, wenn ich nach \textcolor{pink}{Südtirol}{}\ledrightnote{\textcolor{pink}{Südtirol}} gehen
               könnte.\pend
           
\pstart
           In der \label{K_L03198-4v}\edtext{Affaire \textsc{\textcolor{blue}{Matassich}{}\ledrightnote{\textcolor{blue}{Géza von Mattachich}}}}{\lemma{\textnormal{\emph{Affaire Matassich}}}\Cendnote{\textnormal{siehe Paul Goldmann an Arthur Schnitzler, 10. 2. [1902]}}}\label{K_L03198-4h} haſt Du vollkommen Recht. Es war bei mir nur ſo eine Regung, als ich die \textcolor{green}{Rede}{}\ledrightnote{{$\rightarrow$}\textcolor{green}{[Rede über die Mattachich-Affaire]}}{ }\textsc{\textcolor{blue}{Daszinsky}{}\ledrightnote{\textcolor{blue}{Ignacy Daszyński}}s} las. {\pb}Namentlich ſchien es mir, es ſei für Dich eine
               ſchöne Gelegenheit, Dich bei den Herrn für die \label{K_L03198-143v}\edtext{Entziehung der Charge}{\lemma{\textnormal{\emph{Entziehung der Charge}}}\Cendnote{\textnormal{Bezug auf die \emph{\textcolor{green}{Lieutenant
                     Gustl}}-Affaire, siehe Paul Goldmann an Arthur Schnitzler, [20. 6. 1901]}}}\label{K_L03198-143h} zu revanchiren. Du weißt, ich bin rachſüchtig. Jetzt bin ich ſehr zufrieden,
               daß Du von der gefährlichen Geſchichte die Hände wegläßt.\pend
           
\pstart
           Die »\textcolor{green}{Lebendigen Stunden}{}\ledrightnote{\textcolor{green}{Lebendige Stunden. Vier Einakter}}« werden ſich hoffentlich
               in der nächſten Saiſon über die deutſchen Bühnen bewegen. Vielleicht iſt die ſchon
               vorgerückte Saiſon daran ſchuld, daß es einſtweilen nicht recht vorwärts geht. In der
                  \textcolor{pink}{Berlin}{}\ledrightnote{\textcolor{pink}{Berlin}}er Geſellſchaft höre ich überall mit
               Entzücken davon ſprechen. {\pb}\label{K_L03198-7v}\edtext{\textsc{\textcolor{blue}{Koch}{}\ledrightnote{\textcolor{blue}{Max Koch}}s}{ }\textcolor{green}{Kritik}{}\ledrightnote{{$\rightarrow$}\textcolor{green}{?? [Kritik zu Lebendige Stunden]}}}{\lemma{\textnormal{\emph{Kochs Kritik}}}\Cendnote{\textnormal{XXXX (höchstwahrscheinlich in der \emph{\textcolor{green}{Zeit}})}}}\label{K_L03198-7h} ſende ich Dir anbei zurück. Es
               freut mich, daß ſie ſo günſtig ausgefallen iſt. \strikeout{\textcolor{gray}{×}\-\textcolor{gray}{×}\-\textcolor{gray}{×}\-\textcolor{gray}{×}\-\textcolor{gray}{×}\-\textcolor{gray}{×}\-\textcolor{gray}{×}\-\textcolor{gray}{×}\-\textcolor{gray}{×}\-\textcolor{gray}{×}\-\textcolor{gray}{×}\-\textcolor{gray}{×}\-\textcolor{gray}{×}\-\textcolor{gray}{×}\-\textcolor{gray}{×}\-\textcolor{gray}{×}} Sonſt ſcheint mir dieſer \textcolor{blue}{Kritiker}{}\ledrightnote{{$\rightarrow$}\textcolor{blue}{Max Koch}} ein recht unbedeutender Kopf zu ſein.\pend
           
\pstart
           Ich danke Dir für Deine freundlichen Worte über mein \label{K_L03198-8v}\edtext{\textcolor{green}{Opern-Feuilleton}{}\ledrightnote{{$\rightarrow$}\textcolor{green}{Berliner Theater. (»Heilmar« von Wilhelm Kienzl im königlichen Opernhause.)}}}{\lemma{\textnormal{\emph{Opern-Feuilleton}}}\Cendnote{\textnormal{\textcolor{blue}{Paul Goldmann}: \emph{\textcolor{green}{Berliner Theater. (»Heilmar« von Wilhelm Kienzl im
                        königlichen Opernhause.)}}. In: \emph{\textcolor{green}{Neue
                        Freie Presse}}, Nr. 13.458, 11. 2. 1902,
                     Morgenblatt, S. 1–4.}}}\label{K_L03198-8h} und halte Deine Ausſtellung bezüglich der
               allzu großen Länge einzelner Abſätze für nur zu berechtigt. Ich fühle es ſelber, daß
               es mein ſchwerſter ſchriftſtelleriſcher Fehler iſt, nicht kurz ſein zu können. Aber
               beim Schreiben werde ich von einem beinahe krankhaften Drang befallen, Alles bis auf
               den Grund auszuſchöpfen. {\pb}Daher kommen die Längen,
               über die ich dann erſchreckt bin, wenn ich die Arbeit gedruckt ſehe. Wie lernt man,
               kurz zu ſein? Kannſt Du mir nicht ein Mittel ſagen?\pend
           
\pstart
           Mein \textcolor{blue}{Onkel}{}\ledrightnote{{$\rightarrow$}\textcolor{blue}{Fedor Mamroth}} ſchreibt mir mit
               höchſtem Enthuſiasmus von einem im \textcolor{brown}{Wiener Verlag}{}\ledrightnote{\textcolor{brown}{Wiener Verlag}}
               erſchienenen Buch \label{K_L03198-9v}\edtext{»\textcolor{green}{Chriſtiania-\textsc{Bohême}}{}\ledrightnote{{$\rightarrow$}\textcolor{green}{Christiania-Bohême}}«}{\lemma{\textnormal{\emph{»Chriſtiania-Bohême«}}}\Cendnote{\textnormal{\textcolor{blue}{Hans Jæger}: \emph{\textcolor{green}{Christiania-Bohême}}. \textcolor{pink}{Wien}: \emph{\textcolor{brown}{Wiener Verlag}}{ }1902 (zuerst 1885, \emph{\textcolor{green}{Fra Kristiania-Bohêmen}}).}}}\label{K_L03198-9h} von
                  \textsc{\textcolor{blue}{Hans Jaeger}{}\ledrightnote{\textcolor{blue}{Hans Jæger}}}.\pend
           
\pstart
           Hörſt Du etwas von dem neuen Blatt, der \label{K_L03198-123v}\edtext{»\textcolor{brown}{Zeit}{}\ledrightnote{\textcolor{brown}{Die Zeit}{\newline}\textcolor{brown}{Die Zeit. Wiener Wochenschrift}}«}{\lemma{\textnormal{\emph{»Zeit«}}}\Cendnote{\textnormal{siehe Paul Goldmann an Arthur Schnitzler, 16. 1. [1902]}}}\label{K_L03198-123h}?\pend
           
\pstart
           Im Sommer haſt Du mir ein \label{K_L03198-445v}\edtext{\textcolor{green}{Buch}{}\ledrightnote{{$\rightarrow$}\textcolor{green}{?? [Buch über den Talmud]}}}{\lemma{\textnormal{\emph{Buch}}}\Cendnote{\textnormal{nicht ermittelt}}}\label{K_L03198-445h} geſtohlen: das
               über den {\pb}\textcolor{green}{Talmud}{}\ledrightnote{\textcolor{green}{?? [Buch über den Talmud]}}. Ich brauche es und ſchreibe \introOben{}heut\introOben{} an \label{K_L03198-332v}\edtext{\textcolor{blue}{\textsc{Richard}}{}\ledrightnote{\textcolor{blue}{Richard Beer-Hofmann}}}{\lemma{\textnormal{\emph{Richard}}}\Cendnote{\textnormal{\textcolor{blue}{Goldmann} schrieb \textcolor{blue}{Beer-Hofmannn} noch am selben Tag, vgl. \emph{\textcolor{brown}{Houghton Library}},
                     Harvard (Signatur 825.978). Dem Brief ist zu
                  entnehmen, dass \textcolor{blue}{Goldmann} das \textcolor{green}{Buch} von \textcolor{blue}{Beer-Hofmann} im Sommer 1901
                  geschenkt bekommen hatte, nicht aber der Titel.}}}\label{K_L03198-332h}, er möge mir doch Titel und Verlag angeben, damit
               ich es mir kommen laſſen kann. Da ich aber dieſe Anfrage an \textcolor{blue}{\textsc{Richard}}{}\ledrightnote{\textcolor{blue}{Richard Beer-Hofmann}} für ein völlig ausſichtsloſes Unternehmen halte,
               bitte ich Dich (wenn Du das \textcolor{green}{Buch}{}\ledrightnote{{$\rightarrow$}\textcolor{green}{?? [Buch über den Talmud]}} nicht ſelber brauchſt), mir es gelegentlich zu ſchicken. Iſt \textsc{\textcolor{blue}{Richard}{}\ledrightnote{\textcolor{blue}{Richard Beer-Hofmann}}} wieder ganz \label{K_L03198-56v}\edtext{geſund}{\lemma{\textnormal{\emph{geſund}}}\Cendnote{\textnormal{siehe Paul Goldmann an Arthur Schnitzler, 25. 1. [1902]}}}\label{K_L03198-56h}?\pend
           
\pstart
           Ich ſende Dir anbei zwei \label{K_L03198-545v}\edtext{\textcolor{green}{Feuilletons}{}\ledrightnote{{$\rightarrow$}\textcolor{green}{Moderne Religion}}}{\lemma{\textnormal{\emph{Feuilletons}}}\Cendnote{\textnormal{Beilage nicht erhalten. Es handelte
                  sich um folgendes zweiteiliges \textcolor{green}{Feuilleton} von \textcolor{blue}{Heinrich Meyer-Benfey}: \emph{\textcolor{green}{Moderne Religion}}. In: \emph{\textcolor{green}{Frankfurter Zeitung und Handelsblatt}}, Jg. 46, Nr. 50, 19. 2. 1902, Erstes Morgenblatt, S. 1–3 und Nr. 51, 20. 2. 1902, Erstes Morgenblatt, S. 1–3.}}}\label{K_L03198-545h}{ }\strikeout{de\textcolor{gray}{r}} der \textcolor{green}{Frankfurter Ztg.}{}\ledrightnote{\textcolor{green}{Frankfurter Zeitung}} über »\textcolor{green}{Moderne Religion}{}\ledrightnote{{$\rightarrow$}\textcolor{green}{Moderne Religion}}«, die mich
               zum Nachdenken {\pb}ſehr angeregt haben.\pend
           
\pstart
           Schreib’ mir bald, grüße die \textcolor{blue}{Mädels}{}\ledrightnote{{$\rightarrow$}\textcolor{blue}{Olga Schnitzler}{\newline}{$\rightarrow$}\textcolor{blue}{Elisabeth Steinrück}} und ſei ſelbſt vielmals und von Herzen
               gegrüßt! {\\[\baselineskip]}Dein {\\[\baselineskip]}\spacefill\mbox{Paul Goldmnn}\pend
           \leftskip=0em{}\endnumbering\briefempfaengerindex{Schnitzler, Arthur@\textsc{Schnitzler, Arthur}!zzzGoldmann, Paul@\emph{von Paul Goldmann}!1902-02-251@{25. 2. {[}1902{]}}|)be}\mylabel{h}
\begin{anhang}
\end{anhang}\normalsize

\doendnotes{C}
\bigskip
\vfill

\clearpage

\footnotesize

\lohead{\textsc{register}}

% Definiere theindex-Environment komplett neu ohne reledmac
\makeatletter
\renewenvironment{theindex}{%
  \section*{\indexname}%
  \setlength{\parindent}{0pt}%
  \setlength{\parskip}{0pt plus 0.3pt}%
  \let\item\@idxitem
}{%
  \clearpage
}
\makeatother

\IfFileExists{\jobname-pw.ind}{\input{\jobname-pw.ind}}{}

\end{document}

      