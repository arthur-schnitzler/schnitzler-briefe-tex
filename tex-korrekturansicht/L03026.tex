%% latex-korrekturansicht-vorspann.tex
%% Vorspann für die Korrekturansicht.
%% Lädt die gemeinsame Datei latex-vorspann.tex mit gesetztem Schalter.

\newif\ifkorrekturansicht
\korrekturansichttrue

\input{../tex-inputs/latex-vorspann}


\renewcommand{\erwaehntePersonen}{Personen: Felix Salten}
\renewcommand{\erwaehnteOrte}{Orte: Amerika, Wien}
\renewcommand{\erwaehnteWerke}{Werke: Fünf Minuten Amerika}
\section[Arthur Schnitzler an Felix Salten, 30. 5. 1931]{Arthur Schnitzler an Felix Salten, 30. 5. 1931}
\nopagebreak\mylabel{v}
\rehead{ }\normalsize\beginnumbering\briefempfaengerindex{Salten, Felix@\textsc{Salten, Felix}!zzzSchnitzler, Arthur@\emph{von Arthur Schnitzler}!1931-05-301@{30. 5. 1931}|(be}
\toendnotes[C]{\smallbreak\pagebreak[2]}\Standort{Wienbibliothek im Rathaus, ZPH 1681, 2.1.516.}
\physDesc{Brief, 1 Blatt, 2 Seiten, 532 Zeichen
\newline{}Handschrift: schwarze Tinte, lateinische Kurrent
\newline{}Ordnung: mit Bleistift von unbekannter Hand Nummerierung der Blätter des
                                 Konvoluts: »1« }
\buchAbdrucke{\weitereDrucke{Arthur Schnitzler: \emph{Briefe 1913–1931}. Hg. Peter Michael Braunwarth, Richard Miklin, Susanne Pertlik und Heinrich Schnitzler. Frankfurt am Main: \emph{S. Fischer} 1984, S. 792.} }\toendnotes[C]{\smallbreak}
\pstart
           \raggedleft{}{\pb}\textcolor{pink}{Wien}{}\ledrightnote{\textcolor{pink}{Wien}}, 30. 5. 931\pend
           
\pstart
           lieber, ich danke Ihnen sehr herzlich für die freundliche Uebersendg
               Ihres \label{K_L03026-1v}\edtext{\textcolor{green}{\textcolor{pink}{Amerika}{}\ledrightnote{\textcolor{pink}{Amerika}} Buchs}{}\ledrightnote{{$\rightarrow$}\textcolor{green}{Fünf Minuten Amerika}} und der persönlichen
                  Widmung}{\lemma{\textnormal{\emph{Amerika … Widmung}}}\Cendnote{\textnormal{siehe Felix Salten: Widmungsexemplar Fünf Minuten Amerika für Arthur
               Schnitzler, [zwischen 1. und 28.?] 5. 1931}}}\label{K_L03026-1h}. Daſs ich im übrigen so wenig von mir sehen und hören lasse bitte ich Sie
               damit zu entschuldig\textcolor{gray}{en}, daſs ich mich, sowohl seelisch als
               körperlich, aber sagen wir der Einfachheit halber mit den »Nerven« nicht übermäßg
               wohl und insbesondre höchst ungesellig befinde. Ich nehme an dſs wieder {\pb}eine bessere Periode ko{\geminationm}en wird und dann meld ich mich. \pend
           
\pstart
           Sein Sie bis dahin herzlich {\\[\baselineskip]}und freundschaftlich gegrüßt {\\[\baselineskip]}Ihr {\\[\baselineskip]}\spacefill\mbox{Arth}\pend
           \leftskip=0em{}\endnumbering\briefempfaengerindex{Salten, Felix@\textsc{Salten, Felix}!zzzSchnitzler, Arthur@\emph{von Arthur Schnitzler}!1931-05-301@{30. 5. 1931}|)be}\mylabel{h}
\begin{anhang}
\end{anhang}\normalsize

\doendnotes{C}
\bigskip
\vfill

\clearpage

\footnotesize

\lohead{\textsc{register}}

% Definiere theindex-Environment komplett neu ohne reledmac
\makeatletter
\renewenvironment{theindex}{%
  \section*{\indexname}%
  \setlength{\parindent}{0pt}%
  \setlength{\parskip}{0pt plus 0.3pt}%
  \let\item\@idxitem
}{%
  \clearpage
}
\makeatother

\IfFileExists{\jobname-pw.ind}{\input{\jobname-pw.ind}}{}

\end{document}

      