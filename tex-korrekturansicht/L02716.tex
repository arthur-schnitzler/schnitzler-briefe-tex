%% latex-korrekturansicht-vorspann.tex
%% Vorspann für die Korrekturansicht.
%% Lädt die gemeinsame Datei latex-vorspann.tex mit gesetztem Schalter.

\newif\ifkorrekturansicht
\korrekturansichttrue

\input{../tex-inputs/latex-vorspann}


               \section[Paul Goldmann an Arthur Schnitzler, 14. 9. {[}1893{]}]{ Paul Goldmann an Arthur Schnitzler, 14. 9. {[}1893{]}}\nopagebreak\mylabel{v}\rehead{ }\normalsize\beginnumbering\briefempfaengerindex{Schnitzler, Arthur@\textsc{Schnitzler, Arthur}!zzzGoldmann, Paul@\emph{von Paul Goldmann}!1893-09-141@{14. 9. {[}1893{]}}|(be} \toendnotes[C]{\smallbreak\pagebreak[2]} \Standort{DLA, A:Schnitzler, HS.NZ85.1.3163.}
\physDesc{Brief, 1 Blatt, 2 Seiten
\newline{}Handschrift: schwarze Tinte, deutsche Kurrent
\newline{}Schnitzler: mit Bleistift das Jahr »93« vermerkt }\toendnotes[C]{\smallbreak}\pstart
           \noindent{}{\pb}\textcolor{gray}{\textbf{\textbf{\textcolor{brown}{Frankfurter Zeitung}{}\ledrightnote{\textcolor{brown}{Frankfurter Zeitung}}.}}}\pend
           \pstart
           \textcolor{gray}{\textbf{\textbf{(\textcolor{brown}{\begin{otherlanguage}{french}Gazette de Francfort\end{otherlanguage}}{}\ledrightnote{\textcolor{brown}{Frankfurter Zeitung}}.)}}}\pend
           \pstart
           \textcolor{gray}{\textbf{\begin{otherlanguage}{french}\textcolor{blue}{Directeur}{}\ledrightnote{→\textcolor{blue}{Leopold Sonnemann}}\end{otherlanguage}{ }\textbf{M. \textcolor{blue}{L. Sonnemann}{}\ledrightnote{\textcolor{blue}{Leopold Sonnemann}}.}}}\hfill \textsc{\textcolor{pink}{Salzburg}{}\ledrightnote{\textcolor{pink}{Salzburg}}}, 14. September.\pend
           \pstart
           \begin{otherlanguage}{french}\textcolor{gray}{\textbf{\textcolor{green}{Journal}{}\ledrightnote{\textcolor{green}{Frankfurter Zeitung}} politique, financier,}}\end{otherlanguage}\pend
           \pstart
           \begin{otherlanguage}{french}\textcolor{gray}{\textbf{commercial et litteraire.}}\end{otherlanguage}\pend
           \pstart
           \begin{otherlanguage}{french}\textcolor{gray}{\textbf{\textbf{Paraissant trois fois par jour}}}\end{otherlanguage}\pend
           \pstart
           \begin{otherlanguage}{french}\textcolor{gray}{\textbf{\textbf{Bureaux à \textcolor{pink}{Paris}{}\ledrightnote{\textcolor{pink}{Paris}}:}}}\end{otherlanguage}\pend
           \pstart
           \begin{otherlanguage}{french}\textcolor{gray}{\textbf{\textbf{\textcolor{pink}{rue Richelieu 75}{}\ledrightnote{\textcolor{pink}{rue Richelieu}}.}}}\end{otherlanguage}\pend
           \pstart\center{}Mein lieber Arthur!\pend\pstart
           Ich würdige das Opfer, das Du mir bringſt, in ſeinem vollen Werth und danke es Dir
               von Herzen. Die zwei Tage bis zu Deiner Ankunft werden recht lang werden. Aber noch
               ein letztes Mal: geringe Erwartung, bitte, in Bezug auf mich. Ich bin ſo \label{K_L02716-1v}\edtext{\textsc{\begin{otherlanguage}{french}par terre\end{otherlanguage}}}{\lemma{\textnormal{\emph{par terre}}}\Cendnote{\textnormal{französisch: am Boden}}}\label{K_L02716-1h} durch all’
               das Unheil.\pend
           \pstart
           Mein \textcolor{blue}{Onkel}{}\ledrightnote{→\textcolor{blue}{Fedor Mamroth}} iſt hier. Ob er
               noch \label{K_L02716-2v}\edtext{zur Zeit Deiner Ankunft hier}{\lemma{\textnormal{\emph{zur … hier}}}\Cendnote{\textnormal{\textcolor{blue}{Fedor Mamroth} war noch in
                     \textcolor{pink}{Salzburg}. Am 17. 9. 1893 besuchte er gemeinsam mit \textcolor{blue}{Goldmann} und \textcolor{blue}{Schnitzler}{ }\textcolor{pink}{Hellbrunn}.}}}\label{K_L02716-2h} ſein wird, iſt nicht ſicher,
               aber wahrſcheinlich. Ob das {\pb}\textcolor{pink}{Hotel}{}\ledrightnote{→\textcolor{pink}{Hotel Goldenes Horn}} düſter iſt oder nicht,
               weiß ich eigentlich nicht recht zu ſagen. Aber billige Wohnung, gute Koſt, angenehme
               Bedienung. Bitte, telegraphire noch Samſtag: Abgereiſt
                  \strikeout{.}– ein Wort. Dann beſtelle ich Dir ein Zimmer.\pend
           \pstart
           \label{K_L02716-3v}\edtext{\textcolor{brown}{Volkstheater}{}\ledrightnote{\textcolor{brown}{Volkstheater}}}{\lemma{\textnormal{\emph{Volkstheater}}}\Cendnote{\textnormal{\emph{\textcolor{green}{Das Märchen}} wurde am 1. 9. 1893 von \textcolor{blue}{Emerich von Bukovics}, dem \textcolor{blue}{Leiter} des \emph{\textcolor{brown}{Volkstheater}}s, angenommen. Am 1. 12. 1893 kam es dort zur
                  Uraufführung.}}}\label{K_L02716-3h}: Ich bin nicht einverſtanden, wünſche aber natürlich, daß es
               zum Guten ſein möge. Nun, wir reden ja darüber. Reden! Es iſt ſo ſchön, daß ich feſt
               überzeugt bin, es wird nichts daraus.\pend
           \pstart
           Grüß’ Dich Gott, Lieber und Treuer! {\\[\baselineskip]}Dein {\\[\baselineskip]}\spacefill\mbox{Paul Goldmann.}\pend
           \leftskip=0em{}\pstart
           \noindent{}\textsc{\textcolor{pink}{Getreidegasse}{}\ledrightnote{\textcolor{pink}{Getreidegasse}}}, nicht -\textsc{markt}.\pend
           \endnumbering\briefempfaengerindex{Schnitzler, Arthur@\textsc{Schnitzler, Arthur}!zzzGoldmann, Paul@\emph{von Paul Goldmann}!1893-09-141@{14. 9. {[}1893{]}}|)be}\mylabel{h}  \normalsize

\doendnotes{C}
\bigskip
\vfill

\clearpage

\footnotesize

\lohead{\textsc{register}}

% Definiere theindex-Environment komplett neu ohne reledmac
\makeatletter
\renewenvironment{theindex}{%
  \section*{\indexname}%
  \setlength{\parindent}{0pt}%
  \setlength{\parskip}{0pt plus 0.3pt}%
  \let\item\@idxitem
}{%
  \clearpage
}
\makeatother

\IfFileExists{\jobname-pw.ind}{\input{\jobname-pw.ind}}{}

\end{document}

      