%% latex-korrekturansicht-vorspann.tex
%% Vorspann für die Korrekturansicht.
%% Lädt die gemeinsame Datei latex-vorspann.tex mit gesetztem Schalter.

\newif\ifkorrekturansicht
\korrekturansichttrue

\input{../tex-inputs/latex-vorspann}


\section[Stefan Zweig an Arthur Schnitzler, 1. 8. 1923]{L03668 Stefan Zweig an Arthur Schnitzler, 1. 8. 1923}
\nopagebreak\mylabel{L03668v}
\rehead{ }\normalsize\beginnumbering\briefempfaengerindex{, @\textsc{, }!zzz, @\emph{von  }!1923-08-012@{1. 8. 1923}|(be}
\toendnotes[C]{\smallbreak\pagebreak[2]}\Standort{CUL, Schnitzler, B 118.}
\physDesc{Postkarte, 506 Zeichen
\newline{}Handschrift: lila Tinte, lateinische Kurrent
\newline{}Versand: 1) Aufkleber: »Express«  2) Stempel: »\nobreak{}\oindex{Salzburg@\textbf{Salzburg}, \emph{Verwaltungsgebiet}|pwk}Salzburg 1, 1. VIII. 23\nobreak{}«.  3) Stempel: »\nobreak{}\oindex{XVIII., Währing@\textbf{XVIII., Währing}, \emph{Verwaltungsgebiet}|pwk}Wien 111, \textcolor{gray}{2. 8. 23}, 11\textsuperscript{10}\nobreak{}«. }
\buchAbdrucke{\weitereDrucke{Stefan Zweig: \emph{Briefwechsel mit Hermann Bahr, Sigmund Freud, Rainer Maria
                        Rilke und Arthur Schnitzler}. Herausgegeben von Jeffrey B. Berlin, Hans-Ulrich Lindken und Donald A. Prater. Frankfurt am Main: \emph{S. Fischer} 1987, S. 418.} }\toendnotes[C]{\smallbreak}\pstart{}{\pb}D\textsuperscript{r}
                  Arthur Schnitzler\pend{}\pstart{}\textcolor{pink}{Wien – Cottage}\oindex{Wien@\textbf{Wien}!XVIII., Währing@\textbf{XVIII., Währing}!Währinger Cottage@\textbf{Währinger Cottage}, \emph{Teil eines besiedelten Ortes}|pw}{}\ledrightnote{\textcolor{pink}{Währinger Cottage}}\pend{}\pstart{}\textcolor{pink}{\label{K_L03668-1v}\edtext{Sternwartestrasse 71 oder 72}{\lemma{\textnormal{\emph{Sternwartestrasse … 72}}}\Cendnote{\textnormal{\textcolor{blue}{Zweig}\pwindex{Zweig, Stefan 28.\,11.\,1881 Wien – 23.\,2.\,1942 Petrópolis@\textsc{Zweig, Stefan} (28.\,11.\,1881 Wien – 23.\,2.\,1942 Petrópolis), \emph{Schriftsteller}|pwk} wechselt
                        bei der Adressierung seiner Schreiben an \textcolor{blue}{Schnitzler} immer wieder zwischen der richtigen Hausnummer
                           »71« und der falschen
                  »72«.}}}\label{K_L03668-1}}\oindex{Wien@\textbf{Wien}!XVIII., Währing@\textbf{XVIII., Währing}!Sternwartestraße 71@\textbf{Sternwartestraße 71}, \emph{Wohngebäude}|pw}{}\ledrightnote{\textcolor{pink}{Sternwartestraße 71}}\pend{}{\bigskip}\vspace{1em}
\pstart
           \noindent{}{\pb}Lieber verehrter Herr
                  Doktor, das Zimmer im \textcolor{pink}{Österr. Hof}\oindex{Österreichischer Hof@\textbf{Österreichischer Hof}, \emph{Hotel}|pw}{}\ledrightnote{\textcolor{pink}{Österreichischer Hof}} ist für
                  Freitag reserviert. In der Annahme, dass Sie um
                  5 Uhr ankommen werden wir um ½ 6 im \textcolor{pink}{Österr Hof}\oindex{Österreichischer Hof@\textbf{Österreichischer Hof}, \emph{Hotel}|pw}{}\ledrightnote{\textcolor{pink}{Österreichischer Hof}} den Thee nehmen \introOben{}und dort
                  auf Sie warten\introOben{}. Ich hätte sie natürlich zu uns gebeten, aber \textcolor{blue}{R.}\pwindex{Rolland, Romain 29.\,1.\,1866 Clamecy – 30.\,12.\,1944 Vézelay@\textsc{Rolland, Romain} (29.\,1.\,1866 Clamecy – 30.\,12.\,1944 Vézelay), \emph{Schriftsteller}|pw}{}\ledrightnote{\textcolor{blue}{Romain Rolland}} will nachher um ¾ 8 zu dem Concert
               der Kammermusik und wir speisen dann gleich unten. Sind Sie aber Samstag
               noch da, so bitten wir Sie, \uline{herzlich} Mittags bei uns
               mit \textcolor{blue}{R.}\pwindex{Rolland, Romain 29.\,1.\,1866 Clamecy – 30.\,12.\,1944 Vézelay@\textsc{Rolland, Romain} (29.\,1.\,1866 Clamecy – 30.\,12.\,1944 Vézelay), \emph{Schriftsteller}|pw}{}\ledrightnote{\textcolor{blue}{Romain Rolland}} zu speisen. In Herzlichkeit ergeben Ihr\pend
           \pstart \spacefill\mbox{Stefan Zweig}\pend{}\selectlanguage{ngerman}\endnumbering\briefempfaengerindex{, @\textsc{, }!zzz, @\emph{von  }!1923-08-012@{1. 8. 1923}|)be}\mylabel{L03668h}
\begin{anhang}
\end{anhang}\normalsize

\doendnotes{C}
\bigskip
\vfill

\clearpage

\footnotesize

\lohead{\textsc{register}}

% Definiere theindex-Environment komplett neu ohne reledmac
\makeatletter
\renewenvironment{theindex}{%
  \section*{\indexname}%
  \setlength{\parindent}{0pt}%
  \setlength{\parskip}{0pt plus 0.3pt}%
  \let\item\@idxitem
}{%
  \clearpage
}
\makeatother

\IfFileExists{\jobname-pw.ind}{\input{\jobname-pw.ind}}{}

\end{document}

      