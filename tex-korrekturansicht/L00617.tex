%% latex-korrekturansicht-vorspann.tex
%% Vorspann für die Korrekturansicht.
%% Lädt die gemeinsame Datei latex-vorspann.tex mit gesetztem Schalter.

\newif\ifkorrekturansicht
\korrekturansichttrue

\input{../tex-inputs/latex-vorspann}


               \section[Hermann Bahr an Arthur Schnitzler, 5. 11. 1896]{ Hermann Bahr an Arthur Schnitzler, 5. 11. 1896}\nopagebreak\mylabel{v}\rehead{ }\normalsize\beginnumbering\briefempfaengerindex{Schnitzler, Arthur@\textsc{Schnitzler, Arthur}!zzzBahr, Hermann@\emph{von Hermann Bahr}!1896-11-051@{5. 11. 1896}|(be} \toendnotes[C]{\smallbreak\pagebreak[2]} \Standort{CUL, Schnitzler, B 5b.}
\physDesc{Brief, 1 Blatt, 2 Seiten
\newline{}Handschrift: schwarze Tinte, deutsche Kurrent\newline{}Ordnung: mit Bleistift von unbekannter Hand nummeriert:
                                    »45« }\buchAbdrucke{\weitereDrucke{Hermann Bahr, Arthur Schnitzler: \emph{Briefwechsel, Aufzeichnungen, Dokumente (1891–1931)}. Hg. Kurt Ifkovits und Martin Anton Müller. Göttingen: \emph{Wallstein} 2018, S. 129.} }\toendnotes[C]{\smallbreak}\pstart
           \noindent{}{\pb}\textcolor{gray}{\textbf{»\textcolor{brown}{Die Zeit}{}\ledrightnote{\textcolor{brown}{Die Zeit. Wiener Wochenschrift}}«}}\hfill \textcolor{gray}{\textbf{\textbf{\textcolor{pink}{Wien}{}\ledrightnote{\textcolor{pink}{Wien}}}, den }}5. November \textcolor{gray}{\textbf{189}}6\pend
           \pstart
           \textcolor{gray}{\textbf{Wiener Wochenſchrift}}\hfill \textcolor{gray}{\textbf{\textcolor{pink}{IX/3, Günthergaſſe 1}{}\ledrightnote{\textcolor{pink}{Günthergasse}}.}}\pend
           \pstart
           \textcolor{gray}{\textbf{\textbf{Herausgeber}:}}{\\}\textcolor{gray}{\textbf{Profeſſor Dr. \textcolor{blue}{I. Singer}{}\ledrightnote{\textcolor{blue}{Isidor Singer}},
                        \textcolor{blue}{Hermann Bahr}{}\ledrightnote{\textcolor{blue}{Hermann Bahr}}, Dr. \textcolor{blue}{Heinrich Kanner}{}\ledrightnote{\textcolor{blue}{Heinrich Kanner}}.}}\pend
           \pstart
           \textcolor{gray}{\textbf{Telephon Nr. 6415.}}\pend
           \pstart\center{}Lieber Arthur!\pend\pstart
           Von ganzem Herzen gratuliere ich Dir zu dem großen \label{K_L00617_1v}\edtext{Erfolge von »\textcolor{green}{Freiwild}{}\ledrightnote{\textcolor{green}{Freiwild. Schauspiel in 3 Akten}}«}{\lemma{\textnormal{\emph{Erfolge von »Freiwild«}}}\Cendnote{\textnormal{Uraufführung von \emph{\textcolor{green}{Freiwild}} am 3. 11. 1896 im \textcolor{pink}{Berliner Deutschen Theater}}}}\label{K_L00617_1h}, der mir eine
               außerordentliche Freude gemacht hat. Nun möchte ich, ſobald Du zurück biſt, mit Dir
               ſprechen, was man denn thun kann und ſoll, um eine \textcolor{pink}{Wiener}{}\ledrightnote{\textcolor{pink}{Wien}} Aufführung durchzuſetzen. Ich glaube, mit einiger Schlauheit wird das
               möglich ſein. Bitte, telephoniere mir alſo, wann ich Dich {\pb}treffen kann.\pend
           \pstart
           Dann möchte ich aber auch wiſſen, was mit Deiner \textcolor{green}{Novelle}{}\ledrightnote{→\textcolor{green}{Die Frau des Weisen. Erzählung}} iſt. Es wäre mir ſehr wichtig, Sie so zu bekommen,
               daß ich mit ihr im Januar beginnen kann. Das iſt die beſte Zeit und es ſoll auch
               ſonſt alles geſchehen, um Dir den »Aufenthalt« in \textcolor{brown}{meinem Blatte}{}\ledrightnote{→\textcolor{brown}{Die Zeit. Wiener Wochenschrift}} angenehm und behaglich zu machen.\pend
           \pstart
           Über das alles möchte ich recht bald mit Dir ſprechen.\pend
           \pstart
           Herzlichſt{\\[\baselineskip]}Dein{\\[\baselineskip]}\spacefill\mbox{Hermann}\pend
           \leftskip=0em{}\pstart
           \noindent{}Herrn \textsc{D\textsuperscript{r} Arthur Schnitzler}{\\}\textcolor{pink}{Wien}{}\ledrightnote{\textcolor{pink}{Wien}}{ }\textcolor{pink}{IX \textsc{Frankgasse} 1}{}\ledrightnote{\textcolor{pink}{Frankgasse}}.\pend
           \pstart
           \textcolor{gray}{\textbf{\label{T_L00617_1v}\edtext{Alle für »\textcolor{brown}{Die Zeit}{}\ledrightnote{\textcolor{brown}{Die Zeit. Wiener Wochenschrift}}« beſtimmten Zuſchriften und Sendungen ſind an die
                  Redaction der »\textcolor{brown}{Zeit}{}\ledrightnote{\textcolor{brown}{Die Zeit. Wiener Wochenschrift}}« und \textbf{nicht} an die Perſon eines der Herausgeber zu richten.}{\lemma{\textnormal{\emph{Alle … richten.}}}\Cendnote{\textnormal{am unteren Rand der ersten Seite}}}\label{T_L00617_1h}}}\pend
           \endnumbering\briefempfaengerindex{Schnitzler, Arthur@\textsc{Schnitzler, Arthur}!zzzBahr, Hermann@\emph{von Hermann Bahr}!1896-11-051@{5. 11. 1896}|)be}\mylabel{h}  \normalsize

\doendnotes{C}
\bigskip
\vfill

\clearpage

\footnotesize

\lohead{\textsc{register}}

% Definiere theindex-Environment komplett neu ohne reledmac
\makeatletter
\renewenvironment{theindex}{%
  \section*{\indexname}%
  \setlength{\parindent}{0pt}%
  \setlength{\parskip}{0pt plus 0.3pt}%
  \let\item\@idxitem
}{%
  \clearpage
}
\makeatother

\IfFileExists{\jobname-pw.ind}{\input{\jobname-pw.ind}}{}

\end{document}

      