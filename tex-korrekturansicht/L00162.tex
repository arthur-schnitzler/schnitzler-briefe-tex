%% latex-korrekturansicht-vorspann.tex
%% Vorspann für die Korrekturansicht.
%% Lädt die gemeinsame Datei latex-vorspann.tex mit gesetztem Schalter.

\newif\ifkorrekturansicht
\korrekturansichttrue

\input{../tex-inputs/latex-vorspann}


               \section[Eduard Michael Kafka an Arthur Schnitzler, 24. 1. 1893]{ Eduard Michael Kafka an Arthur Schnitzler, 24. 1. 1893}\nopagebreak\mylabel{v}\rehead{ }\normalsize\beginnumbering\briefempfaengerindex{Schnitzler, Arthur@\textsc{Schnitzler, Arthur}!zzzKafka, Eduard Michael@\emph{von Eduard Michael Kafka}!1893-01-241@{24. 1. 1893}|(be} \toendnotes[C]{\smallbreak\pagebreak[2]} \Standort{DLA, A:Schnitzler, HS.NZ85.1.3604.}
\physDesc{Brief, 1 Blatt, 4 Seiten
\newline{}Handschrift: schwarze Tinte, deutsche Kurrent
\newline{}Schnitzler: mit rotem Buntstift eine Unterstreichung }\toendnotes[C]{\smallbreak}\pstart
           \raggedleft{}{\pb}\textcolor{pink}{Prag}{}\ledrightnote{\textcolor{pink}{Prag}}{ }24/I 93\pend
           \pstart{}Lieber Schnitzler,\pend\pstart
           ich bin in \textcolor{pink}{Prag}{}\ledrightnote{\textcolor{pink}{Prag}}; wenn Sie mir was mitzuteilen haben:
               meine Adreſſe ist \textcolor{pink}{\textsc{Grand Hotel}}{}\ledrightnote{\textcolor{pink}{Grand Hotel Prag}}. Ich bleibe noch mehrere Tage. –\pend
           \pstart
           \textcolor{blue}{Reicher}{}\ledrightnote{\textcolor{blue}{Emanuel Reicher}} bat mich, Ihnen zu ſchreiben, daß er von
                  \textcolor{blue}{Blumenthal}{}\ledrightnote{\textcolor{blue}{Oskar Blumenthal}} die beſtimmte Zuſicherung erhalten,
               daß Ihr \textcolor{green}{Stück}{}\ledrightnote{→\textcolor{green}{Das Märchen. Schauspiel in drei Aufzügen}} bis längſtens {\pb}im April in \textcolor{pink}{Berlin}{}\ledrightnote{\textcolor{pink}{Berlin}} zur Aufführung ko{\geminationm}t.\pend
           \pstart
           Ferner kann ich Ihnen mittheilen, daſs Ihre »\textcolor{green}{Frage an
                  das Schickſal}{}\ledrightnote{\textcolor{green}{Die Frage an das Schicksal}}« nächsten Tage \introOben{}(2 Februar)\introOben{} in \textcolor{pink}{Hamburg}{}\ledrightnote{\textcolor{pink}{Hamburg}} (in der \textcolor{brown}{Freien
                     \textsc{literarischen} Geſellschaft}{}\ledrightnote{\textcolor{brown}{Freie literarische Gesellschaft Hamburg}}) u. Mitte \introOben{}(16.)\introOben{}{ }Februar in \textcolor{pink}{Königsberg}{}\ledrightnote{\textcolor{pink}{Kaliningrad}} zum Vortrag
               gelangt: beidemale durch \textcolor{blue}{Reicher}{}\ledrightnote{\textcolor{blue}{Emanuel Reicher}}.\pend
           \pstart
           Sonntag habe ich die »\label{K_L00162_1v}\edtext{\textcolor{green}{Gläubiger}{}\ledrightnote{\textcolor{green}{Gläubiger}}-\textsc{Pre{\pb}mière}}{\lemma{\textnormal{\emph{Gläubiger-Première}}}\Cendnote{\textnormal{Zusammen mit zwei anderen \textcolor{green}{Einaktern} von \textcolor{blue}{Strindberg} am 22. 1. 1893 im \textcolor{pink}{Residenztheater} in \textcolor{pink}{Berlin}.}}}\label{K_L00162_1h} mitgemacht: ein gewaltiger Eindruck.\pend
           \pstart
           Auch die \label{K_L00162_2v}\edtext{\textcolor{green}{Baumeister \textsc{Solneß}}{}\ledrightnote{\textcolor{green}{Baumeister Solness}}-\textsc{Première}}{\lemma{\textnormal{\emph{Baumeister Solneß-Première}}}\Cendnote{\textnormal{am 19. 1. 1893 am \textcolor{pink}{Deutschen Theater} in \textcolor{pink}{Berlin}}}}\label{K_L00162_2h} war ein bedeutſames Erlebnis.\pend
           \pstart
           Was ich in \textcolor{pink}{Berlin}{}\ledrightnote{\textcolor{pink}{Berlin}}{ }\introOben{}machte oder\introOben{} mache? Ein gütiges Schickſal, in Geſtalt eines
                  \uline{lieben \textcolor{blue}{Mannes}{}\ledrightnote{→\textcolor{blue}{?? [Bekannter von E. M. Kafka]}}}, hat mich dahin \strikeout{ge} entführt. Nächſtens {\pb}übrigens können Sie auch aus einer \uline{anderen Welt} auf ein Lebenszeichen von mir rechnen.
               Vorher \substVorne{}\textsuperscript{aber}\substDazwischen{}allerdings\substHinten{} will ich Sie \introOben{}aber\introOben{} noch vom \textcolor{pink}{\textsc{Nordcap}}{}\ledrightnote{\textcolor{pink}{Nordkap}} grüßen. Nächſtens!\pend
           \pstart
           \textsc{Servus}! Mit herzlichen Grüßen{\\[\baselineskip]}Ihr Sie hochſchätzender{\\[\baselineskip]}\spacefill\mbox{Kafka}\pend
           \leftskip=0em{}\endnumbering\briefempfaengerindex{Schnitzler, Arthur@\textsc{Schnitzler, Arthur}!zzzKafka, Eduard Michael@\emph{von Eduard Michael Kafka}!1893-01-241@{24. 1. 1893}|)be}\mylabel{h}  \normalsize

\doendnotes{C}
\bigskip
\vfill

\clearpage

\footnotesize

\lohead{\textsc{register}}

% Definiere theindex-Environment komplett neu ohne reledmac
\makeatletter
\renewenvironment{theindex}{%
  \section*{\indexname}%
  \setlength{\parindent}{0pt}%
  \setlength{\parskip}{0pt plus 0.3pt}%
  \let\item\@idxitem
}{%
  \clearpage
}
\makeatother

\IfFileExists{\jobname-pw.ind}{\input{\jobname-pw.ind}}{}

\end{document}

      