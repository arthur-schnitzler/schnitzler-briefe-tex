%% latex-korrekturansicht-vorspann.tex
%% Vorspann für die Korrekturansicht.
%% Lädt die gemeinsame Datei latex-vorspann.tex mit gesetztem Schalter.

\newif\ifkorrekturansicht
\korrekturansichttrue

\input{../tex-inputs/latex-vorspann}


               \section[Arthur Schnitzler an Hermann Bahr, 30. 3. 1914]{ Arthur Schnitzler an Hermann Bahr, 30. 3. 1914}\nopagebreak\mylabel{v}\rehead{ }\normalsize\beginnumbering\briefempfaengerindex{Bahr, Hermann@\textsc{Bahr, Hermann}!zzzSchnitzler, Arthur@\emph{von Arthur Schnitzler}!1914-03-301@{30. 3. 1914}|(be} \toendnotes[C]{\smallbreak\pagebreak[2]} \Standort{TMW, HS AM 60140 Ba.}
\physDesc{Briefkarte
\newline{}Handschrift: schwarze Tinte, deutsche Kurrent
\newline{}Bahr: das Urteil über \textcolor{blue}{Anna
                                    Bahr-Mildenburg} seitlich mit rotem Buntstift
                                 hervorgehoben }\buchAbdrucke{\weitereDrucke{1) \emph{30. 3. 1914, Abschrift.} In: Arthur Schnitzler: \emph{The Letters of Arthur Schnitzler to Hermann Bahr}. Edited, annotated, and with an introduction, by Donald G.
                        Daviau. Chapel Hill: \emph{The University of North Carolina Press} 1978, S. 113 (University of North Carolina studies in the Germanic languages
                        and literatures, 89).} \weitereDrucke{2) Hermann Bahr, Arthur Schnitzler: \emph{Briefwechsel, Aufzeichnungen, Dokumente (1891–1931)}. Hg. Kurt Ifkovits und Martin Anton Müller. Göttingen: \emph{Wallstein} 2018, S. 493.} }\toendnotes[C]{\smallbreak}\pstart
           \noindent{}{\pb}\textcolor{gray}{\textbf{Dr. Arthur Schnitzler}}\hfill 30. 3. 914\pend
           \pstart
           \textcolor{gray}{\textbf{\textcolor{pink}{Wien XVIII. Sternwartestrasse 71}{}\ledrightnote{\textcolor{pink}{Sternwartestraße}}}}\pend
           \pstart{}mein lieber Hermann, \pend\pstart
           deine Reiſe- u Aufenthaltspläne laſſen wenig Hoffnung übrig, daſs man einander
               wenigſtens im Laufe des So{\geminationm}ers begegnete – nachdem unſer
               Winterverſuch leider misglückt war. Wir wollen Anfang Mai nach \textcolor{pink}{Florenz}{}\ledrightnote{\textcolor{pink}{Florenz}}; ſpäter (13.) von \textcolor{pink}{\textsc{Genua}}{}\ledrightnote{\textcolor{pink}{Genua}} aus zu Schiff nach \textcolor{pink}{Antwerpen}{}\ledrightnote{\textcolor{pink}{Antwerpen}}, {\pb}über \textcolor{pink}{Holland}{}\ledrightnote{\textcolor{pink}{Niederlande}} zurück. Juni u Juli großentheils \textcolor{pink}{Wien}{}\ledrightnote{\textcolor{pink}{Wien}}. Dann Gebirge. (\textcolor{pink}{Engadin}{}\ledrightnote{\textcolor{pink}{Engadin}}?)
               – \pend
           \pstart
           Am \label{K_L02169_1v}\edtext{Freitag}{\lemma{\textnormal{\emph{Freitag}}}\Cendnote{\textnormal{27. 3. 1914}}}\label{K_L02169_1h} haben \textcolor{blue}{wir}{}\ledrightnote{→\textcolor{blue}{Olga Schnitzler}}, nach ziemlich
               langer Zeit, deine \textcolor{blue}{Frau}{}\ledrightnote{→\textcolor{blue}{Anna Bahr-Mildenburg}} wieder
               ſingen gehört. \label{K_L02169_2v}\edtext{\textcolor{green}{Gurrelieder}{}\ledrightnote{\textcolor{green}{Gurre-Lieder}}}{\lemma{\textnormal{\emph{Gurrelieder}}}\Cendnote{\textnormal{von \textcolor{blue}{Arnold Schönberg}, am 27. 3. 1914 mit \textcolor{blue}{Anna Bahr-Mildenburg}}}}\label{K_L02169_2h}. Was \textcolor{blue}{ſie}{}\ledrightnote{→\textcolor{blue}{Anna Bahr-Mildenburg}} geboten hat, gehört einfach zu dem \uline{größten}, was man \uline{je} im
               Conzertſaal \substVorne{}\textsuperscript{gehört}{\allowbreak}\substDazwischen{}erlebt\substHinten{} hat. Schade daſs du nicht dabei warſt.\pend
           \pstart \textcolor{blue}{Wir}{}\ledrightnote{→\textcolor{blue}{Olga Schnitzler}} grüßen dich herzlichſt! Und
               ſage deiner \textcolor{blue}{Gattin}{}\ledrightnote{→\textcolor{blue}{Anna Bahr-Mildenburg}} daſs wir ſie
               bewundern. Auf Wiederſehen doch hoffentlich einmal! Dein \spacefill\mbox{Arthur}\pend{}\endnumbering\briefempfaengerindex{Bahr, Hermann@\textsc{Bahr, Hermann}!zzzSchnitzler, Arthur@\emph{von Arthur Schnitzler}!1914-03-301@{30. 3. 1914}|)be}\mylabel{h}  \normalsize

\doendnotes{C}
\bigskip
\vfill

\clearpage

\footnotesize

\lohead{\textsc{register}}

% Definiere theindex-Environment komplett neu ohne reledmac
\makeatletter
\renewenvironment{theindex}{%
  \section*{\indexname}%
  \setlength{\parindent}{0pt}%
  \setlength{\parskip}{0pt plus 0.3pt}%
  \let\item\@idxitem
}{%
  \clearpage
}
\makeatother

\IfFileExists{\jobname-pw.ind}{\input{\jobname-pw.ind}}{}

\end{document}

      