%% latex-korrekturansicht-vorspann.tex
%% Vorspann für die Korrekturansicht.
%% Lädt die gemeinsame Datei latex-vorspann.tex mit gesetztem Schalter.

\newif\ifkorrekturansicht
\korrekturansichttrue

\input{../tex-inputs/latex-vorspann}


\section[Theodor Herzl an Arthur Schnitzler, {[}Anfang August 1895{]}]{L03894 Theodor Herzl an Arthur Schnitzler, {[}Anfang August 1895{]}}
\nopagebreak\mylabel{L03894v}
\rehead{ }\normalsize\beginnumbering\briefempfaengerindex{, @\textsc{, }!zzz, @\emph{von  }!1895-08-071@{{[}Anfang August 1895{]}}|(be}
\toendnotes[C]{\smallbreak\pagebreak[2]}\Standort{CUL, Schnitzler, B 39.}
\physDesc{Brief,  Blätter,  Seiten, 417 Zeichen
\newline{}Handschrift: schwarze Tinte, lateinische Kurrent
\newline{}Schnitzler: mit Bleistift datiert: »Anf. Aug 95« }
\buchAbdrucke{\weitereDrucke{Theodor Herzl: \emph{Briefe Anfang Mai 1895 – Anfang Dezember 1898}. Bearbeitet von Barbara Schäfer in Zusammenarbeit mit Sofia Gelmann, Chaya Harel, Ines Rubin und Daisy Ticho. Berlin, Frankfurt am Main, Wien: \emph{Propyläen} 1990, S. 61 (Briefe und Tagebücher. Herausgegeben von Alex Bein, Hermann Greive, Moshe Schaerf, Julius H. Schoeps und Johannes Wachten, 4).} }
\pstart
           \raggedleft{}{\pb}\textcolor{pink}{Aussee Villa Fuchs}\oindex{Villa Fuchs@\textbf{Villa Fuchs}, \emph{Wohngebäude}|pw}{}\ledrightnote{\textcolor{pink}{Villa Fuchs}}\pend
           
\pstart{}Lieber Freund!\pend\vspace{0.5em}
\pstart
           Hier bin ich und werde mich sehr freuen, Sie bald zu sehen.\pend
           
\pstart
           Zeigen Sie mir bitte einen Tag früher Ihre Ankunft an. Ich weiss noch nicht, ob ich
               in den nächsten Tagen nicht werde auf 24 Stunden wegfahren müssen. Sollten Sie sich
               gerade für \uline{den} Tag anmelden, so telegraphire ich Ihnen
               ab.\pend
           
\pstart
           Ich werde Ihnen, wenn ich Ihren Zug erfahre Ihnen zur Bahn entgegenkommen.\pend
           
\pstart
           Herzlich Ihr Freund{\\[\baselineskip]}\spacefill\mbox{Th. Herzl}\pend
           \leftskip=0em{}\selectlanguage{ngerman}\endnumbering\briefempfaengerindex{, @\textsc{, }!zzz, @\emph{von  }!1895-08-011@{{[}Anfang August 1895{]}}|)be}\mylabel{L03894h}
\begin{anhang}
\end{anhang}\normalsize

\doendnotes{C}
\bigskip
\vfill

\clearpage

\footnotesize

\lohead{\textsc{register}}

% Definiere theindex-Environment komplett neu ohne reledmac
\makeatletter
\renewenvironment{theindex}{%
  \section*{\indexname}%
  \setlength{\parindent}{0pt}%
  \setlength{\parskip}{0pt plus 0.3pt}%
  \let\item\@idxitem
}{%
  \clearpage
}
\makeatother

\IfFileExists{\jobname-pw.ind}{\input{\jobname-pw.ind}}{}

\end{document}

      