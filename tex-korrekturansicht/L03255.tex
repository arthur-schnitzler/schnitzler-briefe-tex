%% latex-korrekturansicht-vorspann.tex
%% Vorspann für die Korrekturansicht.
%% Lädt die gemeinsame Datei latex-vorspann.tex mit gesetztem Schalter.

\newif\ifkorrekturansicht
\korrekturansichttrue

\input{../tex-inputs/latex-vorspann}


\renewcommand{\erwaehntePersonen}{Personen:  Nikolaus II. von Russland, Olga Schnitzler,  Wilhelm II. von Preußen}
\renewcommand{\erwaehnteOrte}{Orte: Tirol, Warszòw plaża, Welsberg-Taisten, Österreich-Ungarn, Świnoujście}
\renewcommand{\erwaehnteWerke}{Werke: [Bericht über das Treffen von Kaiser Wilhelm II. und Zar Nikolaus II. in Swinemünde, August 1907]}
\section[ Paul Goldmann an Arthur Schnitzler, 7. 8. 1907]{Paul Goldmann an Arthur Schnitzler, 7. 8. 1907}
\nopagebreak\mylabel{v}
\rehead{ }\normalsize\beginnumbering\briefempfaengerindex{Schnitzler, Arthur@\textsc{Schnitzler, Arthur}!zzzGoldmann, Paul@\emph{von Paul Goldmann}!1907-08-071@{7. 8. 1907}|(be}
\toendnotes[C]{\smallbreak\pagebreak[2]}\Standort{DLA, A:Schnitzler, HS.NZ85.1.3175.}
\physDesc{Bildpostkarte
\newline{}Handschrift: 1) schwarze Tinte, deutsche Kurrent\hspace{1em}2) schwarze Tinte, lateinische Kurrent (\noindent{}Adresse)\hspace{1em}
\newline{}Versand: Stempel: »\nobreak{}\oindex{Swinoujście@\textbf{Świnoujście}, \emph{https://www.geonames.org/ontologyP.PPLA2}|pwk}Swinemünde, 7. 8. 07, 10–11 \textcolor{gray}{V}\nobreak{}«.  }\toendnotes[C]{\smallbreak}\pstart{}{\pb}\textcolor{pink}{Österreich}{}\ledrightnote{\textcolor{pink}{Österreich-Ungarn}}.\pend{}\pstart{}Herrn\pend{}\pstart{}Dr. Arthur Schnitzler\pend{}\pstart{}\textcolor{pink}{Welsberg im Pustertal}{}\ledrightnote{\textcolor{pink}{Welsberg-Taisten}}\pend{}\pstart{}\textcolor{pink}{Tirol}{}\ledrightnote{\textcolor{pink}{Tirol}}.\pend{}
{\bigskip}
\pstart
           \noindent{}{\pb}\textcolor{gray}{\textbf{\textbf{\textcolor{pink}{Swinemünde}{}\ledrightnote{\textcolor{pink}{Świnoujście}}}}}\hfill \textcolor{gray}{\textbf{\textbf{\textcolor{pink}{Strand}{}\ledrightnote{\textcolor{pink}{Warszòw plaża}}}}}\pend
           
\pstart
           7. 8. 07.\pend
           
\pstart
           Ich war \textcolor{pink}{hier}{}\ledrightnote{{$\rightarrow$}\textcolor{pink}{Świnoujście}}, um über die
                  \label{K-L03255-1v}\edtext{Kriſen-\textsc{Entrevue}}{\lemma{\textnormal{\emph{Kriſen-Entrevue}}}\Cendnote{\textnormal{Gemeint war das Treffen zwischen \textcolor{blue}{Kaiser Wilhelm II.} und \textcolor{blue}{Zar Nikolaus II.} am 4. 8. 1907 in \textcolor{pink}{Swinemünde}.
                     XXXX (Bericht in Frankfurter Zeitung)}}}\label{K-L03255-1h} zu \textcolor{green}{berichten}{}\ledrightnote{{$\rightarrow$}\textcolor{green}{[Bericht über das Treffen von Kaiser Wilhelm II. und Zar Nikolaus II. in Swinemünde, August 1907]}}. Herzliche Grüße Dir, lieber Freund, u. Deiner \textcolor{blue}{Frau}{}\ledrightnote{{$\rightarrow$}\textcolor{blue}{Olga Schnitzler}}. Nächſte Woche gehe ich auf Urlaub,
               aber ich weiß noch immer nicht, wohin. Dein \spacefill\mbox{Paul Goldmann}.\pend
           \endnumbering\briefempfaengerindex{Schnitzler, Arthur@\textsc{Schnitzler, Arthur}!zzzGoldmann, Paul@\emph{von Paul Goldmann}!1907-08-071@{7. 8. 1907}|)be}\mylabel{h}  \normalsize

\doendnotes{C}
\bigskip
\vfill

\clearpage

\footnotesize

\lohead{\textsc{register}}

% Definiere theindex-Environment komplett neu ohne reledmac
\makeatletter
\renewenvironment{theindex}{%
  \section*{\indexname}%
  \setlength{\parindent}{0pt}%
  \setlength{\parskip}{0pt plus 0.3pt}%
  \let\item\@idxitem
}{%
  \clearpage
}
\makeatother

\IfFileExists{\jobname-pw.ind}{\input{\jobname-pw.ind}}{}

\end{document}

      