%% latex-korrekturansicht-vorspann.tex
%% Vorspann für die Korrekturansicht.
%% Lädt die gemeinsame Datei latex-vorspann.tex mit gesetztem Schalter.

\newif\ifkorrekturansicht
\korrekturansichttrue

\input{../tex-inputs/latex-vorspann}


\renewcommand{\erwaehntePersonen}{Personen: Oskar Mayer}
\renewcommand{\erwaehnteOrte}{Orte: Ala, Bad Ischl, Bludenz, Bozen, Innsbruck, Karlsbad, Meran, Schruns, Schweiz, Traunkai, Triest, Venedig, Verona, Wien}
\renewcommand{\erwaehnteWerke}{}
\section[ Felix Salten an Arthur Schnitzler, 14. 8. 1900]{Felix Salten an Arthur Schnitzler, 14. 8. 1900}
\nopagebreak\mylabel{v}
\rehead{ }\normalsize\beginnumbering\briefempfaengerindex{Schnitzler, Arthur@\textsc{Schnitzler, Arthur}!zzzSalten, Felix@\emph{von Felix Salten}!1900-08-141@{14. 8. 1900}|(be}
\toendnotes[C]{\smallbreak\pagebreak[2]}\Standort{CUL, Schnitzler, B 89, A 2.}
\physDesc{Brief, 1 Blatt, 3 Seiten, 1285 Zeichen
\newline{}Handschrift: schwarze Tinte, lateinische Kurrent
\newline{}Ordnung: mit Bleistift von unbekannter Hand nummeriert: »134« }\toendnotes[C]{\smallbreak}
\pstart
           \raggedleft{}{\pb}\textcolor{pink}{Ischl, Traunquai 11}{}\ledrightnote{\textcolor{pink}{Traunkai}}. {\\}14. Aug. 00.\pend
           
\pstart
           Lieber Freund, leider mußte ich von \textcolor{pink}{Wien}{}\ledrightnote{\textcolor{pink}{Wien}} aus zuerst nach \textcolor{pink}{Karlsbad}{}\ledrightnote{\textcolor{pink}{Karlsbad}}, wie Sie wissen, u. bin erst heute
               hierhergekommen. Ich muss nun wenigstens 7–8 Tage still sitzen und arbeiten. Außerdem
               bin ich auch nicht besonders wol. Es ist für mich garnicht dran zu denken, dass ich
               nach \label{K_L03310-1v}\edtext{\textcolor{pink}{Schruns}{}\ledrightnote{\textcolor{pink}{Schruns}}}{\lemma{\textnormal{\emph{Schruns}}}\Cendnote{\textnormal{siehe Felix Salten an Arthur Schnitzler, 5. 8. 1900}}}\label{K_L03310-1h} komme. Aber einen Vorschlag: Möchten Sie vom Endpunkt Ihrer Tour aus mit mir
               eine mehrtägige Radparthie machen? Wenn Sie, wie Sie mir
                  schreiben{[},{]} nach \textcolor{pink}{Meran}{}\ledrightnote{\textcolor{pink}{Meran}}
               kommen, dann schlage ich vor, dass wir uns in \textcolor{pink}{Bozen}{}\ledrightnote{\textcolor{pink}{Bozen}} treffen, und überlasse dann Ihnen die Bestimmung der Route. (Gerne {\pb}würde ich über \textcolor{pink}{Verona}{}\ledrightnote{\textcolor{pink}{Verona}} nach \textcolor{pink}{Venedig}{}\ledrightnote{\textcolor{pink}{Venedig}})
               Jedenfalls bitte ich Sie, mir gleich Nachricht darüber zu geben und mir besonders Ort
               der Zusammenkunft und Ziel der Radtour anzugeben, möglichst genau, weil ich mir
               danach meine Eisenbahnkarte bestellen muß. Ich habe die Karte bis \textcolor{pink}{Bludenz}{}\ledrightnote{\textcolor{pink}{Bludenz}} bei mir, aber ich muß jedenfalls noch andere Karten
               aus \textcolor{pink}{Wien}{}\ledrightnote{\textcolor{pink}{Wien}} verschreiben, ich denke: \strikeout{(}\textcolor{pink}{Innsbruck}{}\ledrightnote{\textcolor{pink}{Innsbruck}} – \textcolor{pink}{Ala}{}\ledrightnote{\textcolor{pink}{Ala}}, \textcolor{pink}{Triest}{}\ledrightnote{\textcolor{pink}{Triest}} – \textcolor{pink}{Wien}{}\ledrightnote{\textcolor{pink}{Wien}}, \substVorne{}\textsuperscript{)}\substDazwischen{}o\substHinten{}der auch anders. Das hängt dann eben ganz von der Tour ab. {\pb}Ich möchte noch sagen, dass ich
               jeden Vorschlag acceptire, (es sei denn \textcolor{pink}{Schweiz}{}\ledrightnote{\textcolor{pink}{Schweiz}}, was mir vielleicht zu theuer wäre) und dass ich voraussichtlich keine
               Abhaltung mehr haben werde.\pend
           
\pstart
           Hat mein \label{K_L03310-2v}\edtext{Brief mit
                  \textcolor{gray}{I}nschluß}{\lemma{\textnormal{\emph{Brief mit
                  Inschluß}}}\Cendnote{\textnormal{Beilage,
                     siehe Felix Salten an Arthur Schnitzler, 8. 8. 1900}}}\label{K_L03310-2h} an \textcolor{blue}{Mayer}{}\ledrightnote{\textcolor{blue}{Oskar Mayer}} Sie erreicht?\pend
           
\pstart
           Bitte, schreiben Sie bald.\pend
           
\pstart
           Herzlichst {\\[\baselineskip]}Ihr {\\[\baselineskip]}\spacefill\mbox{Salten}\pend
           \leftskip=0em{}\endnumbering\briefempfaengerindex{Schnitzler, Arthur@\textsc{Schnitzler, Arthur}!zzzSalten, Felix@\emph{von Felix Salten}!1900-08-141@{14. 8. 1900}|)be}\mylabel{h}  \normalsize

\doendnotes{C}
\bigskip
\vfill

\clearpage

\footnotesize

\lohead{\textsc{register}}

% Definiere theindex-Environment komplett neu ohne reledmac
\makeatletter
\renewenvironment{theindex}{%
  \section*{\indexname}%
  \setlength{\parindent}{0pt}%
  \setlength{\parskip}{0pt plus 0.3pt}%
  \let\item\@idxitem
}{%
  \clearpage
}
\makeatother

\IfFileExists{\jobname-pw.ind}{\input{\jobname-pw.ind}}{}

\end{document}

      