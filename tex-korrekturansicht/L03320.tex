%% latex-korrekturansicht-vorspann.tex
%% Vorspann für die Korrekturansicht.
%% Lädt die gemeinsame Datei latex-vorspann.tex mit gesetztem Schalter.

\newif\ifkorrekturansicht
\korrekturansichttrue

\input{../tex-inputs/latex-vorspann}


\renewcommand{\erwaehntePersonen}{Personen: Carl Kohlis, Olga Schnitzler}
\renewcommand{\erwaehnteInstitutionen}{Institutionen: Jung-Wiener Theater zum Lieben Augustin}
\renewcommand{\erwaehnteOrte}{Orte: Berlin, Hotel Kronprinz, Jung-Wiener Theater zum Lieben Augustin, Luisenstraße, Marschallbrücke, Reichstag, Schiffbauerdamm, Wien}
\renewcommand{\erwaehnteWerke}{Werke: Die Insel. Monatsschrift mit Buchschmuck und Illustrationen, Schlange}
\section[ Felix Salten an Arthur Schnitzler, 9. 10. 1901]{Felix Salten an Arthur Schnitzler, 9. 10. 1901}
\nopagebreak\mylabel{v}
\rehead{ }\normalsize\beginnumbering\briefempfaengerindex{Schnitzler, Arthur@\textsc{Schnitzler, Arthur}!zzzSalten, Felix@\emph{von Felix Salten}!1901-10-091@{9. 10. 1901}|(be}
\toendnotes[C]{\smallbreak\pagebreak[2]}\Standort{CUL, Schnitzler, B 89, A 2.}
\physDesc{Brief, 1 Blatt, 1 Seite, 320 Zeichen
\newline{}Handschrift: schwarze Tinte, lateinische Kurrent
\newline{}Ordnung: mit Bleistift von unbekannter Hand nummeriert: »144« }\toendnotes[C]{\smallbreak}
\pstart
           \noindent{}\centering{}{\pb}\textcolor{gray}{\textbf{\textcolor{pink}{Hôtel Kronprinz}{}\ledrightnote{\textcolor{pink}{Hotel Kronprinz}}}}\pend
           
\pstart
           \noindent{}\centering{}\textcolor{gray}{\textbf{\textcolor{pink}{\textsc{Berlin} N.W. 6}{}\ledrightnote{\textcolor{pink}{Berlin}}.}}\pend
           
\pstart
           \noindent{}\raggedleft{}\textcolor{gray}{\textbf{\emph{\textcolor{pink}{Luisen-Str. 30}{}\ledrightnote{\textcolor{pink}{Luisenstraße}},}}}\pend
           
\pstart
           \noindent{}\raggedleft{}\textcolor{gray}{\textbf{\emph{nahe dem \textcolor{pink}{Reichstagspalast}{}\ledrightnote{\textcolor{pink}{Reichstag}},}}}\pend
           
\pstart
           \noindent{}\textcolor{gray}{\textbf{\emph{Direktion: \textcolor{blue}{C.
                              Kohlis}{}\ledrightnote{\textcolor{blue}{Carl Kohlis}}.}}}\hfill \textcolor{gray}{\textbf{\emph{Ecke \textcolor{pink}{Schiffbauerdamm}{}\ledrightnote{\textcolor{pink}{Schiffbauerdamm}} (a. d. \textcolor{pink}{Marschall-Brücke}{}\ledrightnote{\textcolor{pink}{Marschallbrücke}}).}}}\pend
           
\pstart
           \textcolor{gray}{\textbf{Telegr. Adr.: \textsc{\textcolor{pink}{Kronprinzhôtel}{}\ledrightnote{\textcolor{pink}{Hotel Kronprinz}}, \textcolor{pink}{Berlin}{}\ledrightnote{\textcolor{pink}{Berlin}}.}}}\pend
           
\pstart
           \textcolor{gray}{\textbf{Fernsprech-Anschluss: Amt III. N\textsuperscript{o} 8871.}}\pend
           
\pstart
           \raggedleft{}\textcolor{gray}{\textbf{\textcolor{pink}{Berlin}{}\ledrightnote{\textcolor{pink}{Berlin}}, den}}{ }9 October 01\pend
           
\pstart
           Lieber Arthur, herzlichen Dank für die Besorgung der \textcolor{green}{Schlange}{}\ledrightnote{\textcolor{green}{Schlange}}\textcolor{gray}{,}{ }{\kaufmannsund} für die \textcolor{green}{Insel}{}\ledrightnote{\textcolor{green}{Die Insel. Monatsschrift mit Buchschmuck und Illustrationen}} .
               Da ich erst Samstag zurückkomme, (früh)
               können Sie’s vielleicht so einrichten, dass ich Sie Mittag verständigen
               kann, ob {\kaufmannsund} um wie viel Uhr wir Nachmittag
               die \label{K_L03320-1v}\edtext{\textcolor{pink}{Bühne}{}\ledrightnote{{$\rightarrow$}\textcolor{pink}{Jung-Wiener Theater zum Lieben Augustin}} haben, und dass
               Sie dann es gleich dem \textcolor{blue}{Fräulein}{}\ledrightnote{{$\rightarrow$}\textcolor{blue}{Olga Schnitzler}}}{\lemma{\textnormal{\emph{Bühne … Fräulein}}}\Cendnote{\textnormal{Probe für den angedachten Auftritt von
                     \textcolor{blue}{Olga Gussmann} (nachmalige Schnitzler)
                  beim \emph{\textcolor{brown}{Jung-Wiener Theater zum Lieben Augustin}}?
                     (vgl. Paul Goldmann an Arthur Schnitzler, 7. 10. [1901])}}}\label{K_L03320-1h}
               mittheilen.\pend
           
\pstart
           herzlichst Ihr {\\[\baselineskip]}\spacefill\mbox{Salten}\pend
           \leftskip=0em{}\endnumbering\briefempfaengerindex{Schnitzler, Arthur@\textsc{Schnitzler, Arthur}!zzzSalten, Felix@\emph{von Felix Salten}!1901-10-091@{9. 10. 1901}|)be}\mylabel{h}  \normalsize

\doendnotes{C}
\bigskip
\vfill

\clearpage

\footnotesize

\lohead{\textsc{register}}

% Definiere theindex-Environment komplett neu ohne reledmac
\makeatletter
\renewenvironment{theindex}{%
  \section*{\indexname}%
  \setlength{\parindent}{0pt}%
  \setlength{\parskip}{0pt plus 0.3pt}%
  \let\item\@idxitem
}{%
  \clearpage
}
\makeatother

\IfFileExists{\jobname-pw.ind}{\input{\jobname-pw.ind}}{}

\end{document}

      