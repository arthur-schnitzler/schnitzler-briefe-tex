%% latex-korrekturansicht-vorspann.tex
%% Vorspann für die Korrekturansicht.
%% Lädt die gemeinsame Datei latex-vorspann.tex mit gesetztem Schalter.

\newif\ifkorrekturansicht
\korrekturansichttrue

\input{../tex-inputs/latex-vorspann}


\renewcommand{\erwaehntePersonen}{Personen: Hermann Bahr, Felix Salten}
\renewcommand{\erwaehnteOrte}{Orte: Gießhübl, Mödling, Rodaun, Wien}
\renewcommand{\erwaehnteWerke}{}
\section[ Arthur Schnitzler an Felix Salten, {[}2. 5. 1895?{]}]{Arthur Schnitzler an Felix Salten, {[}2. 5. 1895?{]}}
\nopagebreak\mylabel{v}
\rehead{ }\normalsize\beginnumbering\briefempfaengerindex{Salten, Felix@\textsc{Salten, Felix}!zzzSchnitzler, Arthur@\emph{von Arthur Schnitzler}!1895-05-022@{{[}2. 5. 1895?{]}}|(be}
\toendnotes[C]{\smallbreak\pagebreak[2]}\Standort{Wienbibliothek im Rathaus, ZPH 1681, 2.1.516.}
\physDesc{Brief, 1 Blatt, 2 Seiten, 265 Zeichen (Briefpapier mit Trauerrand)
\newline{}Handschrift: Bleistift, deutsche Kurrent
\newline{}Ordnung: mit Bleistift von unbekannter Hand Nummerierung der Blätter des Konvoluts:
                                    »31« }\toendnotes[C]{\smallbreak}
\pstart{}{\pb}Lieber Salten,\pend
\pstart
           \textsc{\textcolor{blue}{Bahr}{}\ledrightnote{\textcolor{blue}{Hermann Bahr}}} hat uns \label{K_L03040-1v}\edtext{abgeſchrieben}{\lemma{\textnormal{\emph{abgeſchrieben}}}\Cendnote{\textnormal{\textcolor{blue}{Schnitzler} dürfte sich auf dieses
                  Korrespondenzstück bezogen haben: Hermann Bahr an Arthur Schnitzler, 2. 5. 1894.
                  Dadurch wird die Datierung des vorliegenden Korrespondenzstücks möglich. Am 3. 5. 1895 machten
                     \textcolor{blue}{Salten} und \textcolor{blue}{Schnitzler} einen gemeinsamen Ausflug nach \textcolor{pink}{Mödling}, \textcolor{pink}{Gießhübl} und
                     \textcolor{pink}{Rodaun}.}}}\label{K_L03040-1h}, alſo ſind wahrſcheinlich
               wir zwei, allein. Bitte holen Sie mich alſo entweder \introOben{}früh\introOben{} um
                  \label{K_L03040-2v}\edtext{¾ 9}{\lemma{\textnormal{\emph{¾ 9}}}\Cendnote{\textnormal{8 Uhr 45}}}\label{K_L03040-2h} von Hauſe ab – oder
               ſorgen {\pb}Sie dafür, daſs eine Abſage bereits
               um ½ 8 Morgens bei mir iſt, was ich übrigens nicht hoffe.\pend
           
\pstart
           Herzliche Grüße {\\[\baselineskip]}\spacefill\mbox{Arthur.}\pend
           \leftskip=0em{}\endnumbering\briefempfaengerindex{Salten, Felix@\textsc{Salten, Felix}!zzzSchnitzler, Arthur@\emph{von Arthur Schnitzler}!1895-05-022@{{[}2. 5. 1895?{]}}|)be}\mylabel{h}  \normalsize

\doendnotes{C}
\bigskip
\vfill

\clearpage

\footnotesize

\lohead{\textsc{register}}

% Definiere theindex-Environment komplett neu ohne reledmac
\makeatletter
\renewenvironment{theindex}{%
  \section*{\indexname}%
  \setlength{\parindent}{0pt}%
  \setlength{\parskip}{0pt plus 0.3pt}%
  \let\item\@idxitem
}{%
  \clearpage
}
\makeatother

\IfFileExists{\jobname-pw.ind}{\input{\jobname-pw.ind}}{}

\end{document}

      