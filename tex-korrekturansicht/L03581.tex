%% latex-korrekturansicht-vorspann.tex
%% Vorspann für die Korrekturansicht.
%% Lädt die gemeinsame Datei latex-vorspann.tex mit gesetztem Schalter.

\newif\ifkorrekturansicht
\korrekturansichttrue

\input{../tex-inputs/latex-vorspann}


\renewcommand{\erwaehntePersonen}{Personen: Felix Salten, Ottilie Salten}
\renewcommand{\erwaehnteOrte}{Orte: Cottagegasse, Wien}
\renewcommand{\erwaehnteWerke}{}
\section[Ottilie und Felix Salten an Arthur Schnitzler, {[}Anfang April 1927?{]}]{Ottilie und Felix Salten an Arthur Schnitzler, {[}Anfang
               April 1927?{]}}
\nopagebreak\mylabel{v}
\rehead{ }\normalsize\beginnumbering\briefempfaengerindex{Schnitzler, Arthur@\textsc{Schnitzler, Arthur}!zzzSalten, Felix@\emph{von Felix Salten}!1927-04-101@{{[}Anfang April 1927?{]}}|(be}\briefempfaengerindex{Schnitzler, Arthur@\textsc{Schnitzler, Arthur}!zzzSalten, Ottilie@\emph{von Ottilie Salten}!1927-04-101@{{[}Anfang April 1927?{]}}|(be}
\toendnotes[C]{\smallbreak\pagebreak[2]}\Standort{CUL, Schnitzler, B 89, B 2.}
\physDesc{Briefkarte, 213 Zeichen
\newline{}Handschrift Ottilie Salten: schwarze Tinte, lateinische Kurrent
\newline{}Handschrift Felix Salten: schwarze Tinte, lateinische Kurrent}\toendnotes[C]{\smallbreak}
\pstart
           \noindent{}\centering{}{\pb}\textcolor{gray}{\textbf{HERR UND FRAU FELIX
                        SALTEN}}\pend
           
\pstart
           \noindent{}\centering{}\textcolor{gray}{\textbf{BITTEN}} Herrn D\textsuperscript{r}\pend
           
\pstart
           \noindent{}\centering{}Arthur Schnitzler\pend
           
\pstart
           \noindent{}\centering{}\textcolor{gray}{\textbf{FÜR}}{ }Mittwoch \textcolor{gray}{\textbf{DEN}} 13. April{ }\textcolor{gray}{\textbf{192}}\pend
           
\pstart
           \noindent{}\centering{}zum Thee\pend
           
\pstart
           \noindent{}5–8 \textcolor{gray}{\textbf{UHR}}.\pend
           
\pstart
           \textcolor{gray}{\textbf{\textcolor{pink}{XVIII., COTTAGEGASSE 37}{}\ledrightnote{\textcolor{pink}{Cottagegasse}}}}\hfill \textcolor{gray}{\textbf{\label{K_L03581-1v}\edtext{U. A. W. G.}{\lemma{\textnormal{\emph{u. A. w. g.}}}\Cendnote{\textnormal{um Antwort wird gebeten}}}\label{K_L03581-1h}}}\pend
           
\pstart
           {[}hs. Felix Salten:{]} Lieber, wir werden uns freuen, Sie
               an \label{K_L03581-2v}\edtext{diesem Tag}{\lemma{\textnormal{\emph{diesem Tag}}}\Cendnote{\textnormal{Es handelte
                  sich um die Feier des 
                  25. Hochzeitstags von \textcolor{blue}{Ottilie} und \textcolor{blue}{Felix Salten}. \textcolor{blue}{Schnitzler}
                  nahm teil, 
                  vgl. A. S.: \emph{Tagebuch}, 13. 4. 1927.
               }}}\label{K_L03581-2h} bei uns zu haben – vielleicht, wenn Sie das vorziehen, kommen Sie gegen 7\textsuperscript{h} und bleiben zum Nachtmahl!!\pend
           
\pstart
           Herzlich {\\[\baselineskip]}\spacefill\mbox{F. S.}\pend
           \leftskip=0em{}\endnumbering\briefempfaengerindex{Schnitzler, Arthur@\textsc{Schnitzler, Arthur}!zzzSalten, Felix@\emph{von Felix Salten}!1927-04-011@{{[}Anfang April 1927?{]}}|)be}\briefempfaengerindex{Schnitzler, Arthur@\textsc{Schnitzler, Arthur}!zzzSalten, Ottilie@\emph{von Ottilie Salten}!1927-04-011@{{[}Anfang April 1927?{]}}|)be}\mylabel{h}  \normalsize

\doendnotes{C}
\bigskip
\vfill

\clearpage

\footnotesize

\lohead{\textsc{register}}

% Definiere theindex-Environment komplett neu ohne reledmac
\makeatletter
\renewenvironment{theindex}{%
  \section*{\indexname}%
  \setlength{\parindent}{0pt}%
  \setlength{\parskip}{0pt plus 0.3pt}%
  \let\item\@idxitem
}{%
  \clearpage
}
\makeatother

\IfFileExists{\jobname-pw.ind}{\input{\jobname-pw.ind}}{}

\end{document}

      