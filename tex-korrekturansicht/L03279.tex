%% latex-korrekturansicht-vorspann.tex
%% Vorspann für die Korrekturansicht.
%% Lädt die gemeinsame Datei latex-vorspann.tex mit gesetztem Schalter.

\newif\ifkorrekturansicht
\korrekturansichttrue

\input{../tex-inputs/latex-vorspann}


\renewcommand{\erwaehntePersonen}{Personen: Ottilie Salten, Leopold Ferdinand Salvator Wölfling}
\renewcommand{\erwaehnteOrte}{Orte: Graz, Sensengasse, Wien}
\renewcommand{\erwaehnteWerke}{Werke: Tagebuch}
\section[ Felix Salten an Arthur Schnitzler, {[}10. 7. 1898{]}]{Felix Salten an Arthur Schnitzler, {[}10. 7. 1898{]}}
\nopagebreak\mylabel{v}
\rehead{ }\normalsize\beginnumbering\briefempfaengerindex{Schnitzler, Arthur@\textsc{Schnitzler, Arthur}!zzzSalten, Felix@\emph{von Felix Salten}!1898-07-103@{{[}10. 7. 1898{]}}|(be}
\toendnotes[C]{\smallbreak\pagebreak[2]}\Standort{CUL, Schnitzler, B 89, A 2.}
\physDesc{Brief, 1 Blatt, 1 Seite, 286 Zeichen
\newline{}Handschrift: schwarze Tinte, lateinische Kurrent
\newline{}Schnitzler: mit Bleistift datiert: »10/7 98« 
\newline{}Ordnung: mit Bleistift von unbekannter Hand nummeriert: »103« }\toendnotes[C]{\smallbreak}
\pstart
           \raggedleft{}{\pb}Sonntag{\\}Mittag.\pend
           
\pstart
           Lieber Arthur, soeben erhalte ich die Nachricht, dass der \label{K_L03279-1v}\edtext{\textcolor{blue}{Erzh.}{}\ledrightnote{{$\rightarrow$}\textcolor{blue}{Leopold Ferdinand Salvator Wölfling}}{ }morgen{ }Abend eintrifft}{\lemma{\textnormal{\emph{Erzh. … eintrifft}}}\Cendnote{\textnormal{vermutlich
                     \textcolor{blue}{Leopold
                     Ferdinand von Österreich-Toskana}, der auch in \textcolor{blue}{Schnitzler}s \emph{\textcolor{green}{Tagebuch}}
                  mit Bezug zu \textcolor{blue}{Salten} nur
                     »Erzherzog« genannt wird, vgl. A. S.: \emph{Tagebuch}, 22. 6. 1898}}}\label{K_L03279-1h} – also nichts mit \label{K_L03279-2v}\edtext{\textcolor{pink}{Graz}{}\ledrightnote{\textcolor{pink}{Graz}}}{\lemma{\textnormal{\emph{Graz}}}\Cendnote{\textnormal{siehe A. S.: \emph{Tagebuch}, 11. 7. 1898}}}\label{K_L03279-2h}, was \textcolor{blue}{uns}{}\ledrightnote{{$\rightarrow$}\textcolor{blue}{Ottilie Salten}} sehr leid
               thut. Leben Sie wol und verbringen einen angenehmen Sommer. Briefe in die \label{K_L03279-3v}\edtext{\textcolor{pink}{Sensengasse}{}\ledrightnote{\textcolor{pink}{Sensengasse}}}{\lemma{\textnormal{\emph{Sensengasse}}}\Cendnote{\textnormal{Bezug unklar; \textcolor{blue}{Salten} war nicht in der \textcolor{pink}{Sensengasse} gemeldet}}}\label{K_L03279-3h} adressirt, erreichen mich immer.\pend
           
\pstart
           Auf Wiedersehen {\\[\baselineskip]}herzlichst {\\[\baselineskip]}Ihr {\\[\baselineskip]}\spacefill\mbox{Salten}\pend
           \leftskip=0em{}\endnumbering\briefempfaengerindex{Schnitzler, Arthur@\textsc{Schnitzler, Arthur}!zzzSalten, Felix@\emph{von Felix Salten}!1898-07-103@{{[}10. 7. 1898{]}}|)be}\mylabel{h}  \normalsize

\doendnotes{C}
\bigskip
\vfill

\clearpage

\footnotesize

\lohead{\textsc{register}}

% Definiere theindex-Environment komplett neu ohne reledmac
\makeatletter
\renewenvironment{theindex}{%
  \section*{\indexname}%
  \setlength{\parindent}{0pt}%
  \setlength{\parskip}{0pt plus 0.3pt}%
  \let\item\@idxitem
}{%
  \clearpage
}
\makeatother

\IfFileExists{\jobname-pw.ind}{\input{\jobname-pw.ind}}{}

\end{document}

      