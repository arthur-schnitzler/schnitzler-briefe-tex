%% latex-korrekturansicht-vorspann.tex
%% Vorspann für die Korrekturansicht.
%% Lädt die gemeinsame Datei latex-vorspann.tex mit gesetztem Schalter.

\newif\ifkorrekturansicht
\korrekturansichttrue

\input{../tex-inputs/latex-vorspann}


               \section[Paul Goldmann an Arthur Schnitzler, 19. 7. {[}1892{]}]{ Paul Goldmann an Arthur Schnitzler, 19. 7. {[}1892{]}}\nopagebreak\mylabel{v}\rehead{ }\normalsize\beginnumbering\briefempfaengerindex{Schnitzler, Arthur@\textsc{Schnitzler, Arthur}!zzzGoldmann, Paul@\emph{von Paul Goldmann}!1892-07-192@{19. 7. {[}1892{]}}|(be} \toendnotes[C]{\smallbreak\pagebreak[2]} \Standort{DLA, A:Schnitzler, HS.NZ85.1.3163.}
\physDesc{Briefkarte
\newline{}Handschrift: schwarze Tinte, deutsche Kurrent
\newline{}Schnitzler: mit Bleistift das Jahr »92« vermerkt }\toendnotes[C]{\smallbreak}\pstart
           \noindent{}\raggedleft{}{\pb}\textcolor{gray}{\textbf{\textcolor{pink}{75, Rue de Richelieu}{}\ledrightnote{\textcolor{pink}{rue Richelieu}}.}}\pend
           \pstart
           \textsc{\textcolor{pink}{Paris}{}\ledrightnote{\textcolor{pink}{Paris}}}, 19. Juli.\hfill Mein lieber Arthur!\pend
           \pstart
           Soeben antwortet mir mein \textcolor{blue}{Onkel}{}\ledrightnote{→\textcolor{blue}{Fedor Mamroth}}, daß er ſich mit ſeinem \textcolor{blue}{Verleger}{}\ledrightnote{→\textcolor{blue}{Salo Schottlaender}} zerſtritten, weil er ihn betrogen (der \textcolor{blue}{Verleger}{}\ledrightnote{→\textcolor{blue}{Salo Schottlaender}} meinen \textcolor{blue}{Onkel}{}\ledrightnote{→\textcolor{blue}{Fedor Mamroth}} nämlich) und daß er
               ſonſt keine \label{K_L02700-1v}\edtext{Beziehungen zu
                  Verlegern}{\lemma{\textnormal{\emph{Beziehungen zu
                  Verlegern}}}\Cendnote{\textnormal{\textcolor{blue}{Schnitzler} war auf der Suche nach einem Verlag für \emph{\textcolor{green}{Anatol}}, nachdem ihm die meisten Verlage absagten
                  ohne das Manuskript eingesehen zu haben. Aus \textcolor{blue}{Goldmann}s Vermittlungen wurde nichts, das Buch erschien im Herbst mit
                  Kostenbeteiligung \textcolor{blue}{Schnitzler}s im \emph{\textcolor{brown}{Bibliographischen Bureau}}.}}}\label{K_L02700-1h} habe. Ich
               verſuche jetzt noch einen andern Weg über den ich Dir ſeinerzeit berichten werde. Ich
               ſchick {\pb}Dir nur dieſe eiligen Zeilen, damit Du nicht
               glaubſt, ich ſei in der Sache \strikeout{unthath} unthätig. – \textsc{\textcolor{blue}{Herzl}{}\ledrightnote{\textcolor{blue}{Theodor Herzl}}} läßt Dich erſuchen, Du möchteſt ihm noch etwas von Deinen Sachen ſchicken (\textsc{\textcolor{pink}{8. Rue}{}\ledrightnote{\textcolor{pink}{rue Monceau}}}{ }\substVorne{}\textsuperscript{\textcolor{pink}{Monc}{}\ledrightnote{\textcolor{pink}{rue Monceau}}}\substDazwischen{}\label{T_L02700-1v}\edtext{\textcolor{pink}{Monceau}{}\ledrightnote{\textcolor{pink}{rue Monceau}}}{\lemma{\textnormal{\emph{Monceau}}}\Cendnote{\textnormal{Zur Verdeutlichung des undeutlich
                        geschriebenen »o« wurde von Goldmann »\textcolor{pink}{Monceau}« ein zweites Mal direkt darunter geschrieben.}}}\label{T_L02700-1h}\substHinten{}). Auch meine Adreſſe iſt nicht mehr \textsc{\textcolor{pink}{R. Vivienne}{}\ledrightnote{\textcolor{pink}{rue Vivienne}}}, ſondern \textcolor{pink}{die oben
                  gedruckte}{}\ledrightnote{→\textcolor{pink}{rue Richelieu}}. \pend
           \pstart
           Grüß’ Dich Gott! {\\[\baselineskip]}Dein {\\[\baselineskip]}\spacefill\mbox{Paul Goldm}\pend
           \leftskip=0em{}\endnumbering\briefempfaengerindex{Schnitzler, Arthur@\textsc{Schnitzler, Arthur}!zzzGoldmann, Paul@\emph{von Paul Goldmann}!1892-07-192@{19. 7. {[}1892{]}}|)be}\mylabel{h}  \normalsize

\doendnotes{C}
\bigskip
\vfill

\clearpage

\footnotesize

\lohead{\textsc{register}}

% Definiere theindex-Environment komplett neu ohne reledmac
\makeatletter
\renewenvironment{theindex}{%
  \section*{\indexname}%
  \setlength{\parindent}{0pt}%
  \setlength{\parskip}{0pt plus 0.3pt}%
  \let\item\@idxitem
}{%
  \clearpage
}
\makeatother

\IfFileExists{\jobname-pw.ind}{\input{\jobname-pw.ind}}{}

\end{document}

      