%% latex-korrekturansicht-vorspann.tex
%% Vorspann für die Korrekturansicht.
%% Lädt die gemeinsame Datei latex-vorspann.tex mit gesetztem Schalter.

\newif\ifkorrekturansicht
\korrekturansichttrue

\input{../tex-inputs/latex-vorspann}


               \section[Arthur Schnitzler an Richard Beer-Hofmann, 2. 10. 1897]{ Arthur Schnitzler an Richard Beer-Hofmann, 2. 10. 1897}\nopagebreak\mylabel{v}\rehead{ }\normalsize\beginnumbering\briefempfaengerindex{Beer-Hofmann, Richard@\textsc{Beer-Hofmann, Richard}!zzzSchnitzler, Arthur@\emph{von Arthur Schnitzler}!1897-10-021@{2. 10. 1897}|(be} \toendnotes[C]{\smallbreak\pagebreak[2]} \Standort{YCGL, MSS 31.}
\physDesc{Briefkarte, Umschlag
\newline{}Handschrift: Bleistift, deutsche Kurrent\newline{}Versand: 1) Stempel: »\nobreak{}\oindex{IX., Alsergrund@\textbf{IX., Alsergrund}, \emph{Bezirk (A.BZK)}|pwk}Wien 9/3, 2. 10. {[}9{]}7, 7–\textcolor{gray}{8} N\nobreak{}«.  2) Stempel: »\nobreak{}\oindex{I., Innere Stadt@\textbf{I., Innere Stadt}, \emph{Bezirk (A.BZK)}|pwk}{\pb}Wien 1/1, \textcolor{gray}{3.} 10. 97, 10 ½ V., Bestellt\nobreak{}«. }\toendnotes[C]{\smallbreak}\pstart{}{\pb}Herrn \textsc{Dr. Richard
                     Beer-Hofmann}\pend{}\pstart{}\textcolor{pink}{Wien}{}\ledrightnote{\textcolor{pink}{Wien}}\pend{}\pstart{}\textcolor{pink}{\textsc{I. Wollzeile 15}}{}\ledrightnote{\textcolor{pink}{Wollzeile}}.\pend{}{\bigskip}\pstart
           \noindent{}{\pb}Lieber Richard, Ich vergaſs, daſs in
               jenem \label{K_L00726_1v}\edtext{Brief}{\lemma{\textnormal{\emph{Brief}}}\Cendnote{\textnormal{nicht überliefert}}}\label{K_L00726_1h} von \textcolor{blue}{Andrian}{}\ledrightnote{\textcolor{blue}{Leopold von Andrian-Werburg}} auch ſteht, Sie mögen ihn \uline{jedenfalls}{ }\textcolor{green}{\textsc{en route}}{}\ledrightnote{\textcolor{green}{En route}} von \textcolor{blue}{\textsc{Huysmans}}{}\ledrightnote{\textcolor{blue}{Joris-Karl Huysmans}} u. etwas über den \textcolor{blue}{Milton}{}\ledrightnote{\textcolor{blue}{John Milton}} (? unleſerlich)
               von \textcolor{blue}{\textsc{Stendhal}}{}\ledrightnote{\textcolor{blue}{Stendhal}}{ }ſchicken.\pend
           \pstart
           Seine Adreſſe iſt \textcolor{pink}{\textsc{Baden Baden}}{}\ledrightnote{\textcolor{pink}{Baden-Baden}}, {\pb}\textcolor{pink}{\textsc{Sanatorium Frey}}{}\ledrightnote{\textcolor{pink}{Sanatorium Frey-Dengler}}. –\pend
           \pstart
           Ich gehe vielleicht morgen (So{\geminationn}tag)
                  Abend ins \label{K_L00726_2v}\edtext{\textcolor{pink}{Carltheater}{}\ledrightnote{\textcolor{pink}{Carl-Theater}}}{\lemma{\textnormal{\emph{Carltheater}}}\Cendnote{\textnormal{Er besuchte die Aufführung von \emph{\textcolor{green}{Der Stellvertreter}} von \textcolor{blue}{William Busnach} und \textcolor{blue}{Georges
                     Duval}.}}}\label{K_L00726_2h}.\pend
           \pstart Herzlich Ihr\spacefill\mbox{Arthur.}\pend{}\endnumbering\briefempfaengerindex{Beer-Hofmann, Richard@\textsc{Beer-Hofmann, Richard}!zzzSchnitzler, Arthur@\emph{von Arthur Schnitzler}!1897-10-021@{2. 10. 1897}|)be}\mylabel{h}  \normalsize

\doendnotes{C}
\bigskip
\vfill

\clearpage

\footnotesize

\lohead{\textsc{register}}

% Definiere theindex-Environment komplett neu ohne reledmac
\makeatletter
\renewenvironment{theindex}{%
  \section*{\indexname}%
  \setlength{\parindent}{0pt}%
  \setlength{\parskip}{0pt plus 0.3pt}%
  \let\item\@idxitem
}{%
  \clearpage
}
\makeatother

\IfFileExists{\jobname-pw.ind}{\input{\jobname-pw.ind}}{}

\end{document}

      