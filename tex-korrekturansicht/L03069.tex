%% latex-korrekturansicht-vorspann.tex
%% Vorspann für die Korrekturansicht.
%% Lädt die gemeinsame Datei latex-vorspann.tex mit gesetztem Schalter.

\newif\ifkorrekturansicht
\korrekturansichttrue

\input{../tex-inputs/latex-vorspann}


\renewcommand{\erwaehntePersonen}{Personen:  Friedrich II. von Preußen, Marie Glümer, Hugo von Hofmannsthal, Gertrude von Hofmannsthal, Monty Jacobs, Dora Jacobs, Ulrich Levysohn, Olga Schnitzler, Paul-Charles-François de Thiébault, Dieudonné Thiébault}
\renewcommand{\erwaehnteInstitutionen}{Institutionen: Berliner Börsen-Courier, C. H. F. Hartmann}
\renewcommand{\erwaehnteOrte}{Orte: Berlin, Dessauer Straße, Leipzig, Salzburg, Südtirol, Tirol, Wörthersee}
\renewcommand{\erwaehnteWerke}{Werke: Berliner Börsen-Courier, Friedrich der Große, seine Familie, seine Freunde und sein Hof; oder Zwanzig Jahre meines Aufenthaltes in Berlin, 2 Bde., Lebendige Stunden. Vier Einakter, [Monty Jacobs über Arthur Schnitzler]}
\section[ Paul Goldmann an Arthur Schnitzler, 11. 6. {[}1901{]}]{Paul Goldmann an Arthur Schnitzler, 11. 6. {[}1901{]}}
\nopagebreak\mylabel{v}
\rehead{ }\normalsize\beginnumbering\briefempfaengerindex{Schnitzler, Arthur@\textsc{Schnitzler, Arthur}!zzzGoldmann, Paul@\emph{von Paul Goldmann}!1901-06-112@{11. 6. {[}1901{]}}|(be}
\toendnotes[C]{\smallbreak\pagebreak[2]}\Standort{DLA, A:Schnitzler, HS.NZ85.1.3171.}
\physDesc{Brief, 1 Blatt, 4 Seiten
\newline{}Handschrift: blaue Tinte, deutsche Kurrent
\newline{}Schnitzler: mit rotem Buntstift drei Unterstreichungen }\toendnotes[C]{\smallbreak}
\pstart
           \noindent{}\raggedleft{}{\pb}\textcolor{pink}{\textcolor{gray}{\textbf{DESSAUERSTRASSE 19}}}{}\ledrightnote{\textcolor{pink}{Dessauer Straße}}\pend
           
\pstart
           \textcolor{pink}{Berlin}{}\ledrightnote{\textcolor{pink}{Berlin}}, 11. Juni.\pend
           
\pstart\center{}Mein lieber Freund,\pend
\pstart
           Endlich ein Brief! Ich war ſchon in Sorge. Jetzt alſo kann ich Dir glückliche Reiſe
               wünſchen, – eine frohe \label{K_L03069-3v}\edtext{Sommerfahrt}{\lemma{\textnormal{\emph{Sommerfahrt}}}\Cendnote{\textnormal{Bezug auf die
                  gemeinsamen Reisen nach \textcolor{pink}{Salzburg}, \textcolor{pink}{Tirol} und \textcolor{pink}{Südtirol} zwischen 12. 6. 1901 und 27. 8. 1901}}}\label{K_L03069-3h} Dir und der lieben \textcolor{blue}{Gefährtin}{}\ledrightnote{{$\rightarrow$}\textcolor{blue}{Olga Schnitzler}}. Eine oder die andere \label{K_L03069-2v}\edtext{Andeutung}{\lemma{\textnormal{\emph{Andeutung}}}\Cendnote{\textnormal{Bezug
                  unklar}}}\label{K_L03069-2h} in Deinem Briefe verſtehe ich nicht. Du wirſt mir ſie wohl mündlich
               aufklären. \label{K_L03069-1v}\edtext{Schlimme Nachricht von \textsc{\textcolor{blue}{Mizzi Gl.}{}\ledrightnote{\textcolor{blue}{Marie Glümer}}}}{\lemma{\textnormal{\emph{Schlimme … Gl.}}}\Cendnote{\textnormal{\textcolor{blue}{Marie Glümer} war neuerdings erkrankt, vgl. A. S.: \emph{Tagebuch}, 6. 6. 1901.}}}\label{K_L03069-1h} Die
               Ärmste!\pend
           
\pstart
           Hoffentlich \label{K_L03069-5v}\edtext{ſehen wir uns}{\lemma{\textnormal{\emph{ſehen wir uns}}}\Cendnote{\textnormal{siehe Paul Goldmann an Arthur Schnitzler, 26. 4. [1901]}}}\label{K_L03069-5h} in einigen {\pb}Wochen. Ich
               möchte diesmal ſchon Ende Juli fort, – mit Rückſicht
               darauf, daß ich kaput bin, wie ſchon lange nicht. Zur Stärkung der erſchlafften
               Nerven brauchte ich allerdings Höhenluft. Darum bin ich wieder unſchlüßig geworden
               bezüglich des \label{K_L03069-11v}\edtext{\textcolor{pink}{Wörther See}{}\ledrightnote{\textcolor{pink}{Wörthersee}}}{\lemma{\textnormal{\emph{Wörther See}}}\Cendnote{\textnormal{siehe Paul Goldmann an Arthur Schnitzler, 13. 5. [1901]}}}\label{K_L03069-11h}s. An hohen Orten anderſeits fürchte ich die Einſamkeit. Weiß alſo nicht, was
               werden wird.\pend
           
\pstart
           Nun wirſt Du wohl auch zum Arbeiten kommen, und ich freue mich, daß \label{K_L03069-12v}\edtext{der dramatiſche Stoff vom {\pb}vorigen Jahr}{\lemma{\textnormal{\emph{der … Jahr}}}\Cendnote{\textnormal{vermutlich Bezug auf den Einakterzyklus \emph{\textcolor{green}{Lebendige Stunden}}}}}\label{K_L03069-12h} ausgereift iſt und zum Greifen fertig daliegt. Ich denke, es wird \textcolor{green}{eines}{}\ledrightnote{{$\rightarrow$}\textcolor{green}{Lebendige Stunden. Vier Einakter}} Deiner beſten Stücke
               werden.\pend
           
\pstart
           Viele treue Grüße an Dich und Fräulein \textsc{\textcolor{blue}{Olga}{}\ledrightnote{\textcolor{blue}{Olga Schnitzler}}}! {\\[\baselineskip]}Dein {\\[\baselineskip]}\spacefill\mbox{Paul Goldmann}\pend
           \leftskip=0em{}
\pstart
           \noindent{}\textsc{Dr. \textcolor{blue}{Montij Jacobs}{}\ledrightnote{\textcolor{blue}{Monty Jacobs}}}, der im \textcolor{green}{Börſencouri\textcolor{gray}{e}r}{}\ledrightnote{\textcolor{green}{Berliner Börsen-Courier}} über Dich \label{K_L03069-14v}\edtext{\textcolor{green}{geſchrieben}{}\ledrightnote{{$\rightarrow$}\textcolor{green}{[Monty Jacobs über Arthur Schnitzler]}}}{\lemma{\textnormal{\emph{geſchrieben}}}\Cendnote{\textnormal{XXXX}}}\label{K_L03069-14h}, iſt ein junger Germaniſt, der \label{K_L03069-123v}\edtext{in wenigen Wochen}{\lemma{\textnormal{\emph{in wenigen Wochen}}}\Cendnote{\textnormal{\textcolor{blue}{Monty Jacobs} und \textcolor{blue}{Dora Levysohn} (dann Jacobs) heirateten am 25. 6. 1901. }}}\label{K_L03069-123h} die \textcolor{blue}{Tochter}{}\ledrightnote{{$\rightarrow$}\textcolor{blue}{Dora Jacobs}} des Herrn {\pb}\textsc{\textcolor{blue}{Levysohn}{}\ledrightnote{\textcolor{blue}{Ulrich Levysohn}}}, des Direktors des »\textcolor{brown}{Börſencourier}{}\ledrightnote{\textcolor{brown}{Berliner Börsen-Courier}}«
                  heirathen wird.\pend
           
\pstart
           Lies die reizenden \label{K_L03069-32v}\edtext{\textcolor{green}{Memoiren}{}\ledrightnote{{$\rightarrow$}\textcolor{green}{Friedrich der Große, seine Familie, seine Freunde und sein Hof; oder Zwanzig Jahre meines Aufenthaltes in Berlin, 2 Bde.}}{ }\textsc{\textcolor{blue}{Thielbaut}{}\ledrightnote{\textcolor{blue}{Paul-Charles-François de Thiébault}}s}}{\lemma{\textnormal{\emph{Memoiren Thielbauts}}}\Cendnote{\textnormal{\textcolor{blue}{Dieudonné Thiébault}: \emph{\textcolor{green}{Friedrich der Große, seine Familie, seine Freunde und
                           sein Hof; oder Zwanzig Jahre meines Aufenthaltes in Berlin}}, 2 Bde.
                           \textcolor{pink}{Leipzig}: \emph{\textcolor{brown}{C. H. F. Hartmann}}{ }1828. \textcolor{blue}{Schnitzler} las \textcolor{blue}{Thiébault}s \textcolor{green}{Memoiren} am 15. 4. 1909 zu Ende.}}}\label{K_L03069-32h} vom Hofe \textcolor{blue}{Friedrichs des Großen}{}\ledrightnote{\textcolor{blue}{Friedrich II. von Preußen}}, die ſoeben in guter
                  deutſcher Ausgabe erſchienen ſind.\pend
           
\pstart
           Über die \label{K_L03069-43v}\edtext{Hochzeit}{\lemma{\textnormal{\emph{Hochzeit}}}\Cendnote{\textnormal{\textcolor{blue}{Hugo von Hofmannsthal} und \textcolor{blue}{Getrude Schlesinger} (dann von
                     Hofmannsthal) heirateten am 1. 6. 1901.}}}\label{K_L03069-43h}
                  Deines Freundes \textsc{\textcolor{blue}{Hoffmannsthal}{}\ledrightnote{\textcolor{blue}{Hugo von Hofmannsthal}}} hätteſt Du mir auch ein Wort ſchreiben können.\pend
           \endnumbering\briefempfaengerindex{Schnitzler, Arthur@\textsc{Schnitzler, Arthur}!zzzGoldmann, Paul@\emph{von Paul Goldmann}!1901-06-112@{11. 6. {[}1901{]}}|)be}\mylabel{h}
\begin{anhang}
\end{anhang}\normalsize

\doendnotes{C}
\bigskip
\vfill

\clearpage

\footnotesize

\lohead{\textsc{register}}

% Definiere theindex-Environment komplett neu ohne reledmac
\makeatletter
\renewenvironment{theindex}{%
  \section*{\indexname}%
  \setlength{\parindent}{0pt}%
  \setlength{\parskip}{0pt plus 0.3pt}%
  \let\item\@idxitem
}{%
  \clearpage
}
\makeatother

\IfFileExists{\jobname-pw.ind}{\input{\jobname-pw.ind}}{}

\end{document}

      