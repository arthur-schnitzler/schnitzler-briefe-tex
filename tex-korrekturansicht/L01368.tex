%% latex-korrekturansicht-vorspann.tex
%% Vorspann für die Korrekturansicht.
%% Lädt die gemeinsame Datei latex-vorspann.tex mit gesetztem Schalter.

\newif\ifkorrekturansicht
\korrekturansichttrue

\input{../tex-inputs/latex-vorspann}


               \section[Arthur Schnitzler an Hermann Bahr, 1. 2. 1904]{ Arthur Schnitzler an Hermann Bahr, 1. 2. 1904}\nopagebreak\mylabel{v}\rehead{ }\normalsize\beginnumbering\briefempfaengerindex{Bahr, Hermann@\textsc{Bahr, Hermann}!zzzSchnitzler, Arthur@\emph{von Arthur Schnitzler}!1904-02-011@{1. 2. 1904}|(be} \toendnotes[C]{\smallbreak\pagebreak[2]} \Standort{TMW, HS AM 23365 Ba.}
\physDesc{Brief, 1 Blatt, 4 Seiten
\newline{}Handschrift: schwarze Tinte, deutsche Kurrent}\buchAbdrucke{\weitereDrucke{1) \emph{1. 2. 1904.} In: Arthur Schnitzler: \emph{The Letters of Arthur Schnitzler to Hermann Bahr}. Edited, annotated, and with an introduction, by Donald G.
                        Daviau. Chapel Hill: \emph{The University of North Carolina Press} 1978, S. 83–84 (University of North Carolina studies in the Germanic languages
                        and literatures, 89).} \weitereDrucke{2) Hermann Bahr, Arthur Schnitzler: \emph{Briefwechsel, Aufzeichnungen, Dokumente (1891–1931)}. Hg. Kurt Ifkovits und Martin Anton Müller. Göttingen: \emph{Wallstein} 2018, S. 294.} }\toendnotes[C]{\smallbreak}\pstart
           \raggedleft{}{\pb}\textcolor{pink}{Wien}{}\ledrightnote{\textcolor{pink}{Wien}}{ }1. 2. 904.\pend
           \pstart
           lieber Hermann, aus deinen Worten ſcheint mir eher eine üble Sti{\geminationm}ung als ein übles Befinden hervorzugehen – was für den
               Betroffenen allerdings aufs gleiche herauskommt. Immerhin – ohne Ratſchlägen u
               Entſchlüſſen vorgreifen zu wollen, deine Idee mit \textcolor{pink}{Taormina}{}\ledrightnote{\textcolor{pink}{Taormina}} iſt mir ſehr ſympathiſch – beſonders weil ich große Luſt hätte, im
                  \label{K_L01368_1v}\edtext{April}{\lemma{\textnormal{\emph{April}}}\Cendnote{\textnormal{Die Reise fand erst im Mai statt.}}}\label{K_L01368_1h} nach \textcolor{pink}{Sicilien}{}\ledrightnote{\textcolor{pink}{Sizilien}}{ }{\pb}zu fahren und es mir
               natürlich höchſt erfreulich wäre, dich dort zu finden. Wir (meine \textcolor{blue}{Frau}{}\ledrightnote{→\textcolor{blue}{Olga Schnitzler}} u ich) möchten gern zu Schiff von \textsc{\textcolor{pink}{Fiume}{}\ledrightnote{\textcolor{pink}{Rijeka}}} nach \textsc{\textcolor{pink}{Palermo}{}\ledrightnote{\textcolor{pink}{Palermo}}}.\pend
           \pstart
           – Donnerſtag reiſe ich nach \textcolor{pink}{Berlin}{}\ledrightnote{\textcolor{pink}{Berlin}}, wo es ſich
               zeigen ſoll, wie der \textcolor{green}{Einſame Weg}{}\ledrightnote{\textcolor{green}{Der einsame Weg. Schauspiel in fünf Akten}} auf der Bühne
               wirkt. Daſs im Gang des Stücks etwas nicht in Ordnung iſt, hat mir während der – oft
               unterbrochenen und ganz neu aufgeno{\geminationm}enen – Arbeit oft
               geſchienen. Die gute Wirkung {\pb}die das Stück im
               Vorleſen machte, hat mich einigermaßen beruhigt; – von den eigentlichen Theaterleuten
               scheint aber keiner ernſtlich an einen äußern Erfolg zu glauben (bei aller möglichen
               Hochachtung \textsc{etc.}). Mir perſönlich ſind an dem Stücke werth:
               die Geſtalten des \textsc{Sala} und der \textsc{Johanna}; ferner der Lauf des 4. u besonders des 5. Aktes. –\pend
           \pstart
           Deine Grüße werden beſtellt, meine \textcolor{blue}{Frau}{}\ledrightnote{→\textcolor{blue}{Olga Schnitzler}} dankt dir herzlich {\pb}für deine Grüße und
               wünſcht dir gleich mir, alles mögliche gute.\pend
           \pstart
           Gelegentlich ein Wort von dir zu hören wäre mir höchſt erwünſcht und ſehr
               erbeten.\pend
           \pstart
           Dein getreuer{\\[\baselineskip]}\spacefill\mbox{Arthur.}\pend
           \leftskip=0em{}\endnumbering\briefempfaengerindex{Bahr, Hermann@\textsc{Bahr, Hermann}!zzzSchnitzler, Arthur@\emph{von Arthur Schnitzler}!1904-02-011@{1. 2. 1904}|)be}\mylabel{h}  \normalsize

\doendnotes{C}
\bigskip
\vfill

\clearpage

\footnotesize

\lohead{\textsc{register}}

% Definiere theindex-Environment komplett neu ohne reledmac
\makeatletter
\renewenvironment{theindex}{%
  \section*{\indexname}%
  \setlength{\parindent}{0pt}%
  \setlength{\parskip}{0pt plus 0.3pt}%
  \let\item\@idxitem
}{%
  \clearpage
}
\makeatother

\IfFileExists{\jobname-pw.ind}{\input{\jobname-pw.ind}}{}

\end{document}

      