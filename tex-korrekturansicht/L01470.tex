%% latex-korrekturansicht-vorspann.tex
%% Vorspann für die Korrekturansicht.
%% Lädt die gemeinsame Datei latex-vorspann.tex mit gesetztem Schalter.

\newif\ifkorrekturansicht
\korrekturansichttrue

\input{../tex-inputs/latex-vorspann}


               \section[Richard Beer-Hofmann an Arthur Schnitzler, 12. 11. 1904]{ Richard Beer-Hofmann an Arthur Schnitzler, 12. 11. 1904}\nopagebreak\mylabel{v}\rehead{ }\normalsize\beginnumbering\briefempfaengerindex{Schnitzler, Arthur@\textsc{Schnitzler, Arthur}!zzzBeer-Hofmann, Richard@\emph{von Richard Beer-Hofmann}!1904-11-121@{12. 11. 1904}|(be} \toendnotes[C]{\smallbreak\pagebreak[2]} \Standort{CUL, Schnitzler, B 8.}
\physDesc{Brief, 1 Blatt, 2 Seiten
\newline{}Handschrift: rote Tinte, lateinische Kurrent\newline{}Ordnung: mit Bleistift von unbekannter Hand nummeriert:
                                    »196« }\buchAbdrucke{\weitereDrucke{Arthur Schnitzler, Richard Beer-Hofmann: \emph{Briefwechsel 1891–1931}. Hg. Konstanze Fliedl. Wien, Zürich: \emph{Europaverlag} 1992, S. 169–170.} }\toendnotes[C]{\smallbreak}\pstart
           \raggedleft{}{\pb}\textcolor{pink}{Rodaun}{}\ledrightnote{\textcolor{pink}{Rodaun}}{ }12/XI 04\pend
           \pstart
           Lieber Arthur! Nach einer \textcolor{pink}{Berlin}{}\ledrightnote{\textcolor{pink}{Berlin}}er
               Zeitungsnotiz ist die Première von \textcolor{green}{\textcolor{blue}{Rüderer}{}\ledrightnote{\textcolor{blue}{Josef Ruederer}}}{}\ledrightnote{→\textcolor{green}{Die Morgenröthe. Komödie aus dem Jahre 1848}} am 15 Nov. – dann ko{\geminationm}e ich daran. \textcolor{blue}{Reinhardt}{}\ledrightnote{\textcolor{blue}{Max Reinhardt}} grüssen Sie von mir und sagen Sie ihm
               daß ich ein Telegra{\geminationm} von ihm erwarte – es kann auch ein
               Brief sein – um abzureisen. Vielleicht auch die Nachricht ob ich »\textcolor{pink}{Bristol}{}\ledrightnote{\textcolor{pink}{Hotel Bristol}}« oder »\textcolor{pink}{Carleton}{}\ledrightnote{\textcolor{pink}{Carlton Hotel}}«
               (schreibt man das so?) wohnen soll. »\textcolor{pink}{Carleton}{}\ledrightnote{\textcolor{pink}{Carlton Hotel}}«
               soll ganz neu, sehr gut, u. noch näher v. Theater gelegen sein, u. \textcolor{blue}{Reinhardt}{}\ledrightnote{\textcolor{blue}{Max Reinhardt}} sagte er würde es dieser Tage mit »\textcolor{pink}{Carleton}{}\ledrightnote{\textcolor{pink}{Carlton Hotel}}« versuchen. \textcolor{blue}{\uline{Moissi}}{}\ledrightnote{\textcolor{blue}{Alexander Moissi}} behandeln Sie möglichst streng, arbeiten Sie persönlich – mit ihm – was Sie
               Ihrem »\textcolor{green}{Henri}{}\ledrightnote{→\textcolor{green}{Der grüne Kakadu. Groteske in einem Akt}}« tun, tun Sie meinem
                  »\textcolor{green}{Philipp}{}\ledrightnote{→\textcolor{green}{Der Graf von Charolais. Ein Trauerspiel}}«. \textcolor{blue}{Kerr}{}\ledrightnote{\textcolor{blue}{Alfred Kerr}}, \textcolor{blue}{Bie}{}\ledrightnote{\textcolor{blue}{Oskar Bie}}, \textcolor{blue}{Heimann}{}\ledrightnote{\textcolor{blue}{Moritz Heimann}} – ausdrückliche Grüße – außerdem Grüsse {\pb}à discretion – zum verteilen. Und
               schreiben Sie – zwei Zeilen – 2 – aus \textcolor{pink}{Berlin}{}\ledrightnote{\textcolor{pink}{Berlin}}.\pend
           \pstart
           Herzlichst Ihr{\\[\baselineskip]}\spacefill\mbox{Richard}\pend
           \leftskip=0em{}\endnumbering\briefempfaengerindex{Schnitzler, Arthur@\textsc{Schnitzler, Arthur}!zzzBeer-Hofmann, Richard@\emph{von Richard Beer-Hofmann}!1904-11-121@{12. 11. 1904}|)be}\mylabel{h}  \normalsize

\doendnotes{C}
\bigskip
\vfill

\clearpage

\footnotesize

\lohead{\textsc{register}}

% Definiere theindex-Environment komplett neu ohne reledmac
\makeatletter
\renewenvironment{theindex}{%
  \section*{\indexname}%
  \setlength{\parindent}{0pt}%
  \setlength{\parskip}{0pt plus 0.3pt}%
  \let\item\@idxitem
}{%
  \clearpage
}
\makeatother

\IfFileExists{\jobname-pw.ind}{\input{\jobname-pw.ind}}{}

\end{document}

      