%% latex-korrekturansicht-vorspann.tex
%% Vorspann für die Korrekturansicht.
%% Lädt die gemeinsame Datei latex-vorspann.tex mit gesetztem Schalter.

\newif\ifkorrekturansicht
\korrekturansichttrue

\input{../tex-inputs/latex-vorspann}


               \section[Arthur Schnitzler an Richard Beer-Hofmann, 15. 11. 1904]{ Arthur Schnitzler an Richard Beer-Hofmann, 15. 11. 1904}\nopagebreak\mylabel{v}\rehead{ }\normalsize\beginnumbering\briefempfaengerindex{Beer-Hofmann, Richard@\textsc{Beer-Hofmann, Richard}!zzzSchnitzler, Arthur@\emph{von Arthur Schnitzler}!1904-11-151@{15. 11. 1904}|(be} \toendnotes[C]{\smallbreak\pagebreak[2]} \Standort{YCGL, MSS 31.}
\physDesc{2 Briefkarten, Umschlag
\newline{}Handschrift: Bleistift, deutsche Kurrent\newline{}Versand: 1) Stempel: »\nobreak{}\oindex{Berlin@\textbf{Berlin}, \emph{https://www.geonames.org/ontologyP.PPLC}|pwk}Berlin W 64, 15. 11. 04, 11–12V\nobreak{}«.  2) Stempel: »\nobreak{}\oindex{Rodaun@\textbf{Rodaun}, \emph{Teil eines besiedelten Ortes (A.BSOX)}|pwk}{\pb}Ro\textcolor{gray}{d}aun, 16 \textcolor{gray}{11} 04\nobreak{}«. }\buchAbdrucke{\weitereDrucke{1) Arthur Schnitzler: \emph{Briefe.} In: \emph{Die Neue Rundschau}, Bd. 68 (1957) Nr. 1, S. 93.} \weitereDrucke{2) Arthur Schnitzler: \emph{Briefe 1875–1912}. Hg. Therese Nickl und Heinrich Schnitzler. Frankfurt am Main: \emph{S. Fischer} 1981, S. 493.} \weitereDrucke{3) Arthur Schnitzler, Richard Beer-Hofmann: \emph{Briefwechsel 1891–1931}. Hg. Konstanze Fliedl. Wien, Zürich: \emph{Europaverlag} 1992, S. 170.} }\toendnotes[C]{\smallbreak}\pstart{}{\pb}\textcolor{gray}{\textbf{ICH
                        WACH!}}\pend{}\pstart{}\textcolor{gray}{\textbf{CONRAD UHL’S \textcolor{pink}{HOTEL BRISTOL}{}\ledrightnote{\textcolor{pink}{Hotel Bristol}}}}\pend{}\pstart{}\textcolor{gray}{\textbf{\textcolor{pink}{BERLIN U. D. LINDEN}{}\ledrightnote{\textcolor{pink}{Unter den Linden}} 5 u. 6}}\pend{}{\bigskip}\pstart{}\textsc{{\pb}Herrn Dr. Richard Beer-Hofmann}\pend{}\pstart{}\textsc{\textcolor{pink}{Rodaun}{}\ledrightnote{\textcolor{pink}{Rodaun}}}\pend{}\pstart{}\textsc{bei \textcolor{pink}{Wien}{}\ledrightnote{\textcolor{pink}{Wien}}}\pend{}\pstart{}\textcolor{pink}{\textsc{Liesingerstraße 1}}{}\ledrightnote{\textcolor{pink}{Liesingerstraße}}\pend{}{\bigskip}\pstart
           \raggedleft{}{\pb}15/11 904\pend
           \pstart
           \textcolor{gray}{\textbf{ICH WACH!}}\hfill \textcolor{gray}{\textbf{CONRAD UHL’S \textcolor{pink}{HOTEL
                              BRISTOL}{}\ledrightnote{\textcolor{pink}{Hotel Bristol}}}}\pend
           \pstart
           \raggedleft{}\textcolor{gray}{\textbf{\textcolor{pink}{BERLIN U. D. LINDEN}{}\ledrightnote{\textcolor{pink}{Unter den Linden}} 5 u. 6}}\pend
           \pstart
           lieber Richard, telegram haben Sie wohl vom Theater aus bekommen:
                  Freitag{ }Samſtag Arrangirprobe. Meine \textcolor{green}{\textsc{Premiere}}{}\ledrightnote{→\textcolor{green}{Der tapfere Cassian. Puppenspiel in einem Akt}}{ }Dinſtag; ich lieſs es Ihnen auch telegraphiren weil Sie am Ende, wenn es
               bei Freitag geblieben wäre, um einen Tag früher gekommen wären. –\pend
           \pstart
           \textcolor{pink}{\textsc{Carlton Hotel}}{}\ledrightnote{\textcolor{pink}{Carlton Hotel}} ſoll, wie mir {\pb}\textcolor{blue}{\textsc{Reinhardt}}{}\ledrightnote{\textcolor{blue}{Max Reinhardt}}, der dort wohnt, ſagt, nichts rechtes ſein; räth es Ihnen nicht.\pend
           \pstart
           Ich wohne \textcolor{pink}{\textsc{Bristol}}{}\ledrightnote{\textcolor{pink}{Hotel Bristol}}, es befriedigt mich von allen \textcolor{pink}{Berlin}{}\ledrightnote{\textcolor{pink}{Berlin}}er Hotels
               doch am meiſten. Hoffentlich auf Wiederſehen.\pend
           \pstart
           \textcolor{blue}{\textsc{\uline{Moissi}}}{}\ledrightnote{\textcolor{blue}{Alexander Moissi}}, den ich geſtern zum erſten Mal im \textcolor{green}{Kakadu}{}\ledrightnote{\textcolor{green}{Der grüne Kakadu. Groteske in einem Akt}}
               proben ſah, \uline{eins der augenfälligſten Talente}, das mir
               in der {\pb}letzten Zeit untergeko{\geminationm}en iſt\substVorne{}\textsuperscript{dſs}\substDazwischen{}. Als\substHinten{}{ }\textcolor{green}{\textsc{Henri}}{}\ledrightnote{→\textcolor{green}{Der grüne Kakadu. Groteske in einem Akt}} ka{\geminationn} er übrigens ſeine Fehler zu Tugenden ausnützen
               (was übrigens auch ein Talent iſt.). Für den \textcolor{green}{\textsc{Filipp}}{}\ledrightnote{\textcolor{green}{Der grüne Kakadu. Groteske in einem Akt}} dürfte ihm wohl das wie ſoll ich ſagen Höfiſche fehlen; aber er iſt ſehr
               lenkſam, und das abſolute ſeiner Begabung innerhalb {\pb}des hier (und anderswo) graſſirenden Mittelmaßes \substVorne{}\textsuperscript{\textcolor{gray}{thut}}\substDazwischen{}müßte\substHinten{} jedem Vernünftigen wohlthun. Seine Ausſprache iſt ja ſehr fremdartig – aber
               ſobald man ſie gewöhnt, wirkt ſie (auf mich wenigſtens) beinah als ein Reiz mehr.
               Natürlich iſt es denkbar, daſs ihn das Publikum anfangs auslacht. Mit dieſem Troſt
               will ich ſchließen. \pend
           \pstart Ihr \spacefill\mbox{A.}\pend{}\endnumbering\briefempfaengerindex{Beer-Hofmann, Richard@\textsc{Beer-Hofmann, Richard}!zzzSchnitzler, Arthur@\emph{von Arthur Schnitzler}!1904-11-151@{15. 11. 1904}|)be}\mylabel{h}  \normalsize

\doendnotes{C}
\bigskip
\vfill

\clearpage

\footnotesize

\lohead{\textsc{register}}

% Definiere theindex-Environment komplett neu ohne reledmac
\makeatletter
\renewenvironment{theindex}{%
  \section*{\indexname}%
  \setlength{\parindent}{0pt}%
  \setlength{\parskip}{0pt plus 0.3pt}%
  \let\item\@idxitem
}{%
  \clearpage
}
\makeatother

\IfFileExists{\jobname-pw.ind}{\input{\jobname-pw.ind}}{}

\end{document}

      