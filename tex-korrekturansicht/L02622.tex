%% latex-korrekturansicht-vorspann.tex
%% Vorspann für die Korrekturansicht.
%% Lädt die gemeinsame Datei latex-vorspann.tex mit gesetztem Schalter.

\newif\ifkorrekturansicht
\korrekturansichttrue

\input{../tex-inputs/latex-vorspann}


\renewcommand{\erwaehntePersonen}{Personen: Hermann Bahr, Richard Beer-Hofmann, Moriz Benedikt, Max Eugen Burckhard, Gabriele D’Annunzio, Paul Goldmann, Theodor Herzl, Pierre Lalo, Hugo Lubliner, Adele Sandrock, Louise Schnitzler, Julius Schnitzler, Helene Schnitzler, Clemens Sokal, Leopold Sonnemann, Ludwig Speidel, Friedrich Uhl}
\renewcommand{\erwaehnteInstitutionen}{Institutionen: Burgtheater, Die Zeit. Wiener Wochenschrift, Frankfurter Zeitung, Journal des débats, Volkstheater}
\renewcommand{\erwaehnteOrte}{Orte: Boulevard de Courcelles, Paris, Wien, rue Feydeau}
\renewcommand{\erwaehnteWerke}{Werke: Au jour le jour. M. Arthur Schnitzler, Das neue Stück. Lustspiel in 4 Acten, Frankfurter Zeitung, Journal des débats. Politiques et littéraires, Liebelei. Schauspiel in drei Akten, Novellen, Sterben. Novelle, Urfaust, Wiener Brief}
\section[Paul Goldmann an Arthur Schnitzler, 28. 11. 1894]{Paul Goldmann an Arthur Schnitzler, 28. 11. 1894}
\nopagebreak\mylabel{v}
\rehead{ }\normalsize\beginnumbering\briefempfaengerindex{Schnitzler, Arthur@\textsc{Schnitzler, Arthur}!zzzGoldmann, Paul@\emph{von Paul Goldmann}!1894-11-282@{28. 11. 1894}|(be}
\toendnotes[C]{\smallbreak\pagebreak[2]}\Standort{DLA, A:Schnitzler, HS.NZ85.1.3164.}
\physDesc{Brief, 3 Blätter, 12 Seiten, 5564 Zeichen
\newline{}Handschrift: schwarze Tinte, deutsche Kurrent
\newline{}Schnitzler: 1) mit Bleistift auf dem ersten Blatt die Jahreszahl »94« vermerkt  2) mit rotem Buntstift acht Unterstreichungen}\toendnotes[C]{\smallbreak}
\pstart
           \noindent{}{\pb}\textcolor{gray}{\textbf{\textcolor{brown}{Frankfurter Zeitung}{}\ledrightnote{\textcolor{brown}{Frankfurter Zeitung}}.}}\hfill \textsc{\textcolor{pink}{Paris}{}\ledrightnote{\textcolor{pink}{Paris}}}, 28. November.\pend
           
\pstart
           \textcolor{gray}{\textbf{(\textcolor{brown}{Gazette de
                     Francfort}{}\ledrightnote{\textcolor{brown}{Frankfurter Zeitung}}).}}\pend
           
\pstart
           \textcolor{gray}{\textbf{\begin{otherlanguage}{french}Fondateur\end{otherlanguage}{ }\textbf{M. \textcolor{blue}{L. Sonnemann}{}\ledrightnote{\textcolor{blue}{Leopold Sonnemann}}}.}}\pend
           
\pstart
           \textcolor{gray}{\textbf{\begin{otherlanguage}{french}Journal politique, financier,\end{otherlanguage}}}\pend
           
\pstart
           \textcolor{gray}{\textbf{\begin{otherlanguage}{french}commercial et littéraire.\end{otherlanguage}}}\pend
           
\pstart
           \textcolor{gray}{\textbf{\begin{otherlanguage}{french}\textbf{Paraissant trois fois par jour}\end{otherlanguage}}}.\pend
           
\pstart
           \textcolor{gray}{\textbf{\begin{otherlanguage}{french}\textbf{Bureaux à \textcolor{pink}{Paris}{}\ledrightnote{\textcolor{pink}{Paris}}:}\end{otherlanguage}}}\pend
           
\pstart
           \textcolor{gray}{\textbf{\begin{otherlanguage}{french}\textcolor{pink}{\textbf{24. Rue Feydeau}}{}\ledrightnote{\textcolor{pink}{rue Feydeau}}.\end{otherlanguage}}}\pend
           
\pstart\center{}Mein lieber Freund,\pend
\pstart
           Ich danke Dir von Herzen für die Überſendung von »\textcolor{green}{Sterben}{}\ledrightnote{\textcolor{green}{Sterben. Novelle}}«. Als ich den \textcolor{green}{Schluß}{}\ledrightnote{{$\rightarrow$}\textcolor{green}{Sterben. Novelle}} las, hatte ich das Gefühl, daß ſich der durch die verfluchten
               Fortſetzungen unterbrochene Strom wieder herſtellte. Der große Schauer kam –
               Ergriffenheit und Entzücken. Das Sterben iſt meiſterhaft geſchildert. Mich ſtört nur
               das Erwürgen\substVorne{}\textsuperscript{.}\substDazwischen{},\substHinten{} – dieſes plötzliche Verfallen in die kriminaliſtiſche Brutalität, nachdem
                  \strikeout{es} vorher \strikeout{Alles}
               Alles eitel Freiheit, Seele, Stimmung geweſen. Ich glaube, das {\pb}hätte zweiſelhaft bleiben müſſen. Vielleicht ſtellte
               ſich das die überhitzte Phantaſie des Mädchens \introOben{}nur\introOben{} ſo vor?
               Vielleicht wollte er ſie umarmen? Mir ſtört das noch rückwärts etwas das Bild des
               Unglücklichen. Er ſoll Einer ſein, der \uline{leidet}, bis
               zum Schluß. Das Handeln iſt ſo unheimlich, ſo gegen ſeine Natur. Der erwürgt nicht,
               glaub’ mirs. Er weint nur, weil ſie nicht mit ihm ſterben will, das Sterben ſelbſt
               wird ihm dadurch zur noch größeren Qual, er wird noch mehr \uline{leidend} zum Schluß. So denke ichs mir. Und {\pb}das Alles könnte erreicht werden, wenn nur ein einziger kleiner Satz am Schluſſe
               geſtrichen würde, wo das Mädel es klar ſagt: »\label{K_L02622-1v}\edtext{Er hatte ſie erwürgen wollen.}{\lemma{\textnormal{\emph{Er … wollen.}}}\Cendnote{\textnormal{\textcolor{blue}{Schnitzler} änderte den Satz in späteren \textcolor{green}{Auflagen} nicht.}}}\label{K_L02622-1h}«\pend
           
\pstart
           Vielleicht habe ich übrigens Unrecht. Denn ich habe das \textcolor{green}{Buch}{}\ledrightnote{{$\rightarrow$}\textcolor{green}{Sterben. Novelle}} mit überſcharfer Kritik geleſen, weil ich
                  \strikeout{\textcolor{gray}{mir}} Dir ſelbſt gegenüber ein unparteiiſches zu fällen mich verpflichtet fühlte und
               ſtets auf der Lauer war, um nicht von meiner Freundſchaft überrumpelt zu werden.
               Sonſt iſt es wohl gelungen, das \textcolor{green}{Buch}{}\ledrightnote{{$\rightarrow$}\textcolor{green}{Sterben. Novelle}} – ſchön und reich. In der Literatur {\pb}weiſt es Dir, meiner Anſicht nach, einen Platz neben \textsc{\textcolor{blue}{d’Annunzio}{}\ledrightnote{\textcolor{blue}{Gabriele D’Annunzio}}} an\substVorne{}\textsuperscript{.}\substDazwischen{};\substHinten{} nur iſt Deine Art etwas blaſſer, weniger raffinirt, ſanfter, als die ſeine.
               Laß’ Dich von Herzen beglückwünſchen.\pend
           
\pstart
           Ich habe ſofort Schritte gethan, um Dir eine Beſprechung in der \textcolor{pink}{Pariſ}{}\ledrightnote{\textcolor{pink}{Paris}}er Preſſe, und zwar in der großen, zu verſchaffen. Ich bin
               zum »\textsc{\textcolor{brown}{Journal des Débats}{}\ledrightnote{\textcolor{brown}{Journal des débats}}}« gegangen und habe Sturm geläutet über die \textcolor{pink}{Wien}{}\ledrightnote{\textcolor{pink}{Wien}}er Literatur. \textsc{\textcolor{blue}{Pierre Lalo}{}\ledrightnote{\textcolor{blue}{Pierre Lalo}}}, ein charmanter und feinſinnger College, hat mir \label{K_L02622-2v}\edtext{Beſprechungen}{\lemma{\textnormal{\emph{Beſprechungen}}}\Cendnote{\textnormal{\textcolor{blue}{Pierre Lalo} schrieb selbst: P. L.: \emph{\textcolor{green}{Au jour le jour. M. \textcolor{blue}{Arthur Schnitzler}}}. In: \emph{\textcolor{green}{Journal des débats}}, Jg. 107,
                        21. 3. 1895, S. 1.}}}\label{K_L02622-2h} verſprochen.
               Ob ers halten {\pb}wird, weiß ich nicht. Jedenfalls
               ſchicke ihm ein \textcolor{green}{Buch}{}\ledrightnote{{$\rightarrow$}\textcolor{green}{Sterben. Novelle}} und
               ſchreibe hinein: \textsc{À Monsieur \textcolor{blue}{Pierre Lalo}{}\ledrightnote{\textcolor{blue}{Pierre Lalo}}}, \textsc{\label{K_L02622-3v}\edtext{hommage de l’auteur}{\lemma{\textnormal{\emph{hommage de l’auteur}}}\Cendnote{\textnormal{französisch: Widmung des
                     Verfassers}}}\label{K_L02622-3h}}, mit Deiner Unterſchrift. Ebenſo ſoll \textsc{\textcolor{blue}{Richard}{}\ledrightnote{\textcolor{blue}{Richard Beer-Hofmann}}} ihm ſein \textcolor{green}{Buch}{}\ledrightnote{{$\rightarrow$}\textcolor{green}{Novellen}} ſchicken.
               Er wohnt \textsc{\textcolor{pink}{19. Boulevard de Courcelles, Paris}{}\ledrightnote{\textcolor{pink}{Boulevard de Courcelles}}}. Unter keinen Umſtänden aber bitte ich \textsc{\textcolor{blue}{Bahr}{}\ledrightnote{\textcolor{blue}{Hermann Bahr}}} die \textcolor{pink}{Adreſſe}{}\ledrightnote{{$\rightarrow$}\textcolor{pink}{Boulevard de Courcelles}} zu geben.
               Ich will nicht, daß er ſich durch meine Vermittelung in der \textcolor{pink}{Pariſ}{}\ledrightnote{\textcolor{pink}{Paris}}er Preſſe lancirt. Sei mir nicht böſe: »\label{K_L02622-4v}\edtext{\textcolor{green}{Ich weiß es wohl, es iſt ein
                  Vorurtheil}{}\ledrightnote{{$\rightarrow$}\textcolor{green}{Urfaust}}}{\lemma{\textnormal{\emph{Ich … Vorurtheil}}}\Cendnote{\textnormal{Mephistopheles im \emph{\textcolor{green}{Urfaust}}: »Ich weis es wohl, es ist ein Vorurtheil /
                     allein genung mir ists einmal zuwieder«.}}}\label{K_L02622-4h}{ }\textsc{etc.}«.\pend
           
\pstart
           {\pb}Bei der »\textcolor{brown}{Frankfurter
                  Zeitung}{}\ledrightnote{\textcolor{brown}{Frankfurter Zeitung}}« habe ich geſtern Schritte gethan. Ich
               hoffe, diesmal wird Alles glatt gehen. Haſt Du die liebenswürdige \label{K_L02622-5v}\edtext{Erwähnung Deines Namens durch \textsc{\textcolor{blue}{Uhl}{}\ledrightnote{\textcolor{blue}{Friedrich Uhl}}} in ſeinem \textcolor{green}{Briefe}{}\ledrightnote{{$\rightarrow$}\textcolor{green}{Wiener Brief}} über
               das \textcolor{green}{Stück}{}\ledrightnote{{$\rightarrow$}\textcolor{green}{Das neue Stück. Lustspiel in 4 Acten}} von \textsc{\textcolor{blue}{Lubliner}{}\ledrightnote{\textcolor{blue}{Hugo Lubliner}}}}{\lemma{\textnormal{\emph{Erwähnung … Lubliner}}}\Cendnote{\textnormal{Am 17. 11. 1894 hatte die Uraufführung von \emph{\textcolor{green}{Das neue Stück}} von \textcolor{blue}{Hugo Lubliner} am \emph{\textcolor{brown}{Deutschen
                     Volkstheater}} stattgefunden. \textcolor{blue}{Uhl}
                  schrieb: »Am lautesten lachten die dienstfreien Schauspieler des \textcolor{brown}{Volkstheaters} im Zuschauerraum, besonders
                     die stets Aufschauen erregende Schwärmerin Frl. \textcolor{blue}{\so{Sandrock}} in einer Loge des ersten Ranges und der geistreiche Lustspieldichter Dr.
                        \textcolor{blue}{\so{Schnitzler}}, der über das Kapitel ›Dichter im \textcolor{brown}{Wiener
                        Volkstheater}‹ eine Leidensgeschichte erzählen könnte. Aber Frl. \textcolor{blue}{Sandrock} konnte auch dieses \textcolor{green}{Stück} nicht
                     retten.« [\textcolor{blue}{Friedrich Uhl}]:
                        \emph{\textcolor{green}{Wiener Brief}}. In: \emph{\textcolor{green}{Frankfurter Zeitung}}, Jg. 39, Nr. 322,
                        20. 11. 1894, Abendblatt, S. 1. }}}\label{K_L02622-5h} geleſen?\pend
           
\pstart
           Ich wünſchte nur, daß ich Dir auch in den Schritten für Dein \textcolor{green}{Stück}{}\ledrightnote{{$\rightarrow$}\textcolor{green}{Liebelei. Schauspiel in drei Akten}} behilflich ſein könnte\substVorne{}\textsuperscript{.}\substDazwischen{},\substHinten{} um Dir ein wenig von dem Paſſionswege zu erſparen. Ich habe mir den Kopf
               zerbrochen, wie ich eingreifen könnte, finde aber nichts. Oder glaubſt Du vielleicht,
               daß {\pb}\textsc{\textcolor{blue}{Uhl}{}\ledrightnote{\textcolor{blue}{Friedrich Uhl}}} etwas in der Sache thun könnte? Dann ſchreib’ mir darüber, und ich wills
               unternehmen. Jedenfalls wiederhole ich Dir von Neuem: laß’ Dich nicht niederdrücken
               und entmuthigen. Die Schwierigkeiten waren vorauszuſehen. Wenn man ein Stück nur zu
               ſchreiben und einzureichen brauchte, um es aufgeführt zu ſehen, ſo wäre es ein
               Vergnügen, Theaterdichter zu ſein. Außerdem bringſt Du Neues, das heißt etwas
               Anti-Dummes, folglich haſt Du die Dummheit gegen Dich. Das iſt doch ganz natürlich.
               Aber man findet ſchon Mittel, {\pb}um mit der Dummheit
               fertig zu werden. Nur Zeit, Geduld und Geſchick gehört dazu. Mit dieſen drei
               Kampfmitteln \strikeout{\textcolor{gray}{we}} mußt Du Dich unter allen Umſtänden ausrüſten. Ich bin \uline{überzeugt}, Du wirſt am Ende durchdringen, und zwar gerade beim \textcolor{brown}{Burgtheater}{}\ledrightnote{\textcolor{brown}{Burgtheater}}. Laß’ Dich alſo nicht verſtimmen.
               Denk’ auch an den ſchönen Haß und Hohn, den dieſe Erfahrungen in Dir aufhäufen und
               der befruchtend wirken wird für \strikeout{ſch} ſpätere Werke.
               Und, bitte, mach’ mir nach wie vor von jedem weiteren Vorkomniß Mittheilung. \label{K_L02622-6v}\edtext{\textsc{\textcolor{blue}{Speidel}{}\ledrightnote{\textcolor{blue}{Ludwig Speidel}}}}{\lemma{\textnormal{\emph{Speidel}}}\Cendnote{\textnormal{\textcolor{blue}{Speidel} war ein enger Berater des \emph{\textcolor{brown}{Burgtheater}}-Direktors \textcolor{blue}{Burckhard}. Vgl. A. S.: \emph{Tagebuch}, 20. 11. 1894, 14. 12. 1894. }}}\label{K_L02622-6h}? {\pb}Vielleicht. \label{K_L02622-7v}\edtext{Wenn Gott will, ſchießt ein Beſen}{\lemma{\textnormal{\emph{Wenn … Beſen}}}\Cendnote{\textnormal{jüdisches Sprichwort}}}\label{K_L02622-7h}. Und die Erfahrung lehrt, daß
               hier und da ein Beſen ſchon geſchoſſen hat. Man \strikeout{ve}
               verleumdet den lieben Gott, wenn man ſo ganz ſeine Exiſtenz leugnet. Ein wenig
               exiſtirt er doch, auch für junge Poeten.\pend
           
\pstart
           Dringend bitte ich Dich, mich bei Frl. \textsc{\textcolor{blue}{Sandrock}{}\ledrightnote{\textcolor{blue}{Adele Sandrock}}} zu entſchuldigen. Ich ſchreibe ihr, ſobald ich einen ſreien Augenblick
               habe.\pend
           
\pstart
           Herr \textsc{\textcolor{blue}{Sokal}{}\ledrightnote{\textcolor{blue}{Clemens Sokal}}} ſoll \label{K_L02622-8v}\edtext{gut aufgenommen}{\lemma{\textnormal{\emph{gut aufgenommen}}}\Cendnote{\textnormal{Bezug unklar}}}\label{K_L02622-8h} werden, {\pb}um deſſentwillen, von dem er kommt, und, wenn er
               will, auch ſeinetwegen.\pend
           
\pstart
           Wie geht die »\textcolor{brown}{Zeit}{}\ledrightnote{\textcolor{brown}{Die Zeit. Wiener Wochenschrift}}«? Und was ſagſt Du dazu?\pend
           
\pstart
           Unter Discretion: Ich höre, daß \textsc{\textcolor{blue}{Benedict}{}\ledrightnote{\textcolor{blue}{Moriz Benedikt}}} Erkundigungen über mich einzieht. Natürlich werde ich nie an \label{K_L02622-9v}\edtext{\textsc{\textcolor{blue}{Herzl}{}\ledrightnote{\textcolor{blue}{Theodor Herzl}}s} Stelle}{\lemma{\textnormal{\emph{Herzls Stelle}}}\Cendnote{\textnormal{siehe Paul Goldmann an Arthur Schnitzler, 1. 5. [1894]}}}\label{K_L02622-9h} kommen, ſchon weil \textsc{\textcolor{blue}{Herzl}{}\ledrightnote{\textcolor{blue}{Theodor Herzl}}} dagegen iſt, und aus andern Gründen. Aber kennſt Du zufällig Jemanden, der dem
               hochmögenden \textcolor{blue}{Herrn}{}\ledrightnote{\textcolor{blue}{Moriz Benedikt}}, natürlich mit unendlicher
               Vorſicht, in einem Geſpräche gelegentlich mittheilen könnte, {\pb}daß ich ein großer Mann bin? Um nicht Alles
               unverſucht zu laſſen!\pend
           
\pstart
           Die gütigen Worte, die Du über mich ſchreibſt, haben mich tief bewegt. Was ich an \uline{Dir} habe, weiß ich längſt; aber es thut wohl, es
               wieder einmal zu fühlen. Wie ſich mein Bild bei Andern malt, ſehe ich täglich und
               ſtündlich, und dieſe Erfahrungen ſprechen ſchreienden, brüllenden Hohn zu Deinen
               lieben Zeilen. Wenn ich \strikeout{dann} Dein \textcolor{green}{Buch}{}\ledrightnote{{$\rightarrow$}\textcolor{green}{Sterben. Novelle}} leſe und dann an meine Thätigkeit denke –
                  {\pb}es iſt beinahe komiſch. Nein, ehrlich geſagt,
               das iſt es nicht: es iſt traurig{\dotsfour}\pend
           
\pstart
           Du erhälſt anbei ein paar \label{K_L02622-10v}\edtext{kurioſe
                  Artikel}{\lemma{\textnormal{\emph{kurioſe
                  Artikel}}}\Cendnote{\textnormal{Beilage nicht erhalten}}}\label{K_L02622-10h}
               aller Art.\pend
           
\pstart
           Was ſoll ich mit den \textsc{30 Francs 30 ct.} machen, die ich Dir
               ſchulde? Du ſetzeſt mich einer ſtarken Verſuchung aus. Ein Anderer hätte ſie längſt
               unterſchlagen. Ich ſehe mit Befriedigung, wie \strikeout{\textcolor{gray}{e}hrlich} ehrlich ich bin.\pend
           
\pstart
           Grüße, bitte, \textcolor{blue}{Mutter}{}\ledrightnote{{$\rightarrow$}\textcolor{blue}{Louise Schnitzler}}, \textcolor{blue}{Bruder}{}\ledrightnote{{$\rightarrow$}\textcolor{blue}{Julius Schnitzler}} und \textcolor{blue}{Schwägerin}{}\ledrightnote{{$\rightarrow$}\textcolor{blue}{Helene Schnitzler}}.\pend
           
\pstart
           In alter Treue{\\[\baselineskip]}Dein{\\[\baselineskip]}\spacefill\mbox{Paul Goldmann.}\pend
           \leftskip=0em{}\endnumbering\briefempfaengerindex{Schnitzler, Arthur@\textsc{Schnitzler, Arthur}!zzzGoldmann, Paul@\emph{von Paul Goldmann}!1894-11-282@{28. 11. 1894}|)be}\mylabel{h}  \normalsize

\doendnotes{C}
\bigskip
\vfill

\clearpage

\footnotesize

\lohead{\textsc{register}}

% Definiere theindex-Environment komplett neu ohne reledmac
\makeatletter
\renewenvironment{theindex}{%
  \section*{\indexname}%
  \setlength{\parindent}{0pt}%
  \setlength{\parskip}{0pt plus 0.3pt}%
  \let\item\@idxitem
}{%
  \clearpage
}
\makeatother

\IfFileExists{\jobname-pw.ind}{\input{\jobname-pw.ind}}{}

\end{document}

      