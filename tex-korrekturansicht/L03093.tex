%% latex-korrekturansicht-vorspann.tex
%% Vorspann für die Korrekturansicht.
%% Lädt die gemeinsame Datei latex-vorspann.tex mit gesetztem Schalter.

\newif\ifkorrekturansicht
\korrekturansichttrue

\input{../tex-inputs/latex-vorspann}


\renewcommand{\erwaehntePersonen}{Personen: Leo Ebermann, Gerhart Hauptmann, Robert Hirschfeld, Moriz Neuda, Karl von Perfall}
\renewcommand{\erwaehnteInstitutionen}{Institutionen: Jung-Wiener Theater zum Lieben Augustin, Neue Freie Presse, Reichstag}
\renewcommand{\erwaehnteOrte}{Orte: Berlin, Dessauer Straße, Deutschland, Wien}
\renewcommand{\erwaehnteWerke}{Werke: Der rothe Hahn. Tragikomödie in vier Akten, Frankfurter Zeitung, Gerhart Hauptmanns Tragikomödie »Der rothe Hahn«, Hanneles Himmelfahrt. Traumdichtung in zwei Teilen, Hauptmanns Niedergang und die Berliner Litteratur-Tyrannei, Kölnische Zeitung, Lebendige Stunden. Vier Einakter, Michael Kramer. Drama, Theater- und Kunstnachrichten. Jung-Wiener-Theater »Zum lieben Augustin«, Wiener Leben}
\section[ Paul Goldmann an Arthur Schnitzler, 4. 12. {[}1901{]}]{Paul Goldmann an Arthur Schnitzler, 4. 12. {[}1901{]}}
\nopagebreak\mylabel{v}
\rehead{ }\normalsize\beginnumbering\briefempfaengerindex{Schnitzler, Arthur@\textsc{Schnitzler, Arthur}!zzzGoldmann, Paul@\emph{von Paul Goldmann}!1901-12-041@{4. 12. {[}1901{]}}|(be}
\toendnotes[C]{\smallbreak\pagebreak[2]}\Standort{DLA, A:Schnitzler, HS.NZ85.1.3171.}
\physDesc{Brief, 1 Blatt, 4 Seiten
\newline{}Handschrift: blaue Tinte, deutsche Kurrent
\newline{}Beilage: zwei Zeitungsausschnitte, beschnitten und eingeklebt 
\newline{}Schnitzler: 1) mit Bleistift das Jahr »{[}1{]}901« vermerkt  2) mit rotem Buntstift zwei Unterstreichungen}\toendnotes[C]{\smallbreak}
\pstart
           \noindent{}\raggedleft{}{\pb}\textcolor{pink}{\textcolor{gray}{\textbf{DESSAUERSTRASSE 19}}}{}\ledrightnote{\textcolor{pink}{Dessauer Straße}}\pend
           
\pstart
           \textcolor{pink}{Berlin}{}\ledrightnote{\textcolor{pink}{Berlin}}, 4. Dezember.\pend
           
\pstart\center{}Mein lieber Freund,\pend
\pstart
           Zolltarif im \textcolor{brown}{Reichſtag}{}\ledrightnote{\textcolor{brown}{Reichstag}}. Ich habe keine freie
               Minute.\pend
           
\pstart
           Tauſend Dank für Deinen lieben Brief.\pend
           
\pstart
           Über Deine Auslegung, daß \textsc{\textcolor{blue}{Hauptmann}{}\ledrightnote{\textcolor{blue}{Gerhart Hauptmann}}} eine geiſtige Krankheit durchmacht, habe ich den Kopf geſchüttelt. Warum eine
               Erklärung {\pb}an den Haaren herbeiziehen? Warum das
               Eigentliche nicht ſehen wollen? Wenn Einer geiſtig leer iſt, ſo iſt er immer geiſtig
               leer geweſen. Man kann ein Stück verfehlen, man kann aber nicht auf einmal weder
               Geiſt noch Talent haben. Und was Deine Anſicht betrifft, \textsc{\textcolor{green}{Hannele}{}\ledrightnote{\textcolor{green}{Hanneles Himmelfahrt. Traumdichtung in zwei Teilen}}} ſei für »alle Zeiten« ein ſchönes Stück, ſo ſprichſt Du im Namen von »allen
               Zeiten« ein künſtleriſches Urtheil aus, zu dem »alle Zeiten« Dich gewiß {\pb}nicht ermächtigt haben.\pend
           
\pstart
           \label{K_L03093-1v}\edtext{Wann kommſt Du?}{\lemma{\textnormal{\emph{Wann kommſt Du?}}}\Cendnote{\textnormal{\textcolor{blue}{Schnitzler} war von 28. 12. 1901 bis 6. 1. 1902 in \textcolor{pink}{Berlin}. Er und \textcolor{blue}{Goldmann} sahen sich jedenfalls am 5. 1. 1902 und 6. 1. 1902,
                  höchstwahrscheinlich auch am 4. 1. 1902 bei der Uraufführung von \emph{\textcolor{green}{Lebendige Stunden}}.}}}\label{K_L03093-1h} Ich freue mich ſehr darauf, Dich
               wiederzuſehen.\pend
           
\pstart
           Haſt Du \label{K_L03093-2v}\edtext{\textsc{\textcolor{blue}{Hirschfeld}{}\ledrightnote{\textcolor{blue}{Robert Hirschfeld}}s}{ }\textcolor{green}{Feuilleton}{}\ledrightnote{{$\rightarrow$}\textcolor{green}{Wiener Leben}}}{\lemma{\textnormal{\emph{Hirschfelds Feuilleton}}}\Cendnote{\textnormal{\textcolor{blue}{Robert Hirschfeld}: \emph{\textcolor{green}{Wiener Leben}}. In: \emph{\textcolor{green}{Frankfurter Zeitung}}, Jg. 46, Nr. 333,
                        1. 12. 1901, Erstes Morgenblatt, S. 1–2.
               }}}\label{K_L03093-2h} in der \textcolor{green}{Frkf. Ztg.}{}\ledrightnote{\textcolor{green}{Frankfurter Zeitung}} geleſen? Wenn das \textcolor{brown}{Jung-Wiener Theater}{}\ledrightnote{\textcolor{brown}{Jung-Wiener Theater zum Lieben Augustin}} ſo erbärmlich war, wie es
               darin geſchildert wird, ſo kann ich auch der \textcolor{brown}{N. Fr.
                  Pr.}{}\ledrightnote{\textcolor{brown}{Neue Freie Presse}} und dem alten \label{K_L03093-3v}\edtext{\textsc{\textcolor{green}{\textcolor{blue}{Neuda}{}\ledrightnote{\textcolor{blue}{Moriz Neuda}}}{}\ledrightnote{{$\rightarrow$}\textcolor{green}{Theater- und Kunstnachrichten. Jung-Wiener-Theater »Zum lieben Augustin«}}}}{\lemma{\textnormal{\emph{Neuda}}}\Cendnote{\textnormal{siehe Paul Goldmann an Arthur Schnitzler, 23. 11. [1901]}}}\label{K_L03093-3h} nicht Unrecht geben.\pend
           
\pstart
           Ich ſende Dir einen \label{K_L03093-44v}\edtext{\textcolor{green}{Ausſchnitt}{}\ledrightnote{{$\rightarrow$}\textcolor{green}{Hauptmanns Niedergang und die Berliner Litteratur-Tyrannei}}}{\lemma{\textnormal{\emph{Ausſchnitt}}}\Cendnote{\textnormal{nicht identifiziert}}}\label{K_L03093-44h}{ }{\pb}aus einem \label{K_L03093-88v}\edtext{\textcolor{green}{Referat}{}\ledrightnote{{$\rightarrow$}\textcolor{green}{Gerhart Hauptmanns Tragikomödie »Der rothe Hahn«}}}{\lemma{\textnormal{\emph{Referat}}}\Cendnote{\textnormal{[O. V.]: \emph{\textcolor{green}{Gerhart Hauptmanns Tragikomödie
                        »Der rothe Hahn«}}. In: \emph{\textcolor{green}{Kölnische
                        Zeitung}}, Nr. 931, 28. 11. 1901,
                     Abend-Ausgabe, S. [2].}}}\label{K_L03093-88h}{ }\textsc{\textcolor{blue}{Perfall}{}\ledrightnote{\textcolor{blue}{Karl von Perfall}}s} in der \textcolor{green}{Kölniſchen Zeitung}{}\ledrightnote{\textcolor{green}{Kölnische Zeitung}}, nur damit Du ſiehſt, daß es außer Herrn
                  \label{K_L03093-55v}\edtext{\textsc{\textcolor{blue}{Ebermann}{}\ledrightnote{\textcolor{blue}{Leo Ebermann}}}}{\lemma{\textnormal{\emph{Ebermann}}}\Cendnote{\textnormal{siehe Paul Goldmann an Arthur Schnitzler, 9. 11. [1901]}}}\label{K_L03093-55h} auch noch andere Leute gibt, die meine Anſicht theilen.\pend
           
\pstart
           Viele treue Grüße! {\\[\baselineskip]}Dein \spacefill\mbox{Paul Goldmann.}\pend
           \leftskip=0em{}{\bigskip}
\pstart
           \noindent{}\textcolor{gray}{\textbf{\textcolor{green}{\textbf{\textcolor{blue}{Hauptmann}{}\ledrightnote{\textcolor{blue}{Gerhart Hauptmann}}s Niedergang und die \textcolor{pink}{Berlin}{}\ledrightnote{\textcolor{pink}{Berlin}}er Litteratur-Tyrannei.} In der
                        »\textcolor{green}{Kölniſchen Zeitung}{}\ledrightnote{\textcolor{green}{Kölnische Zeitung}}« leſen wir: »{\dots} Der Mißerfolg des »\textcolor{green}{roten Hahns}{}\ledrightnote{\textcolor{green}{Der rothe Hahn. Tragikomödie in vier Akten}}«, der dem Mißerfolge des »\textcolor{green}{Michael Kramer}{}\ledrightnote{\textcolor{green}{Michael Kramer. Drama}}« folgt, läßt kaum noch die Hoffnung
                     übrig, daß \textcolor{blue}{Hauptmann}{}\ledrightnote{\textcolor{blue}{Gerhart Hauptmann}} über ſeine früheren
                     Werke zu einer großen Dramatik aufſteigen wird. Es iſt vielmehr ziemlich
                     ſicher, daß er beſtenfalls ſich noch einmal auf halber Höhe aufrichtet, aber
                     der \textcolor{blue}{Hauptmann}{}\ledrightnote{\textcolor{blue}{Gerhart Hauptmann}}, über den eine ganze
                     Litteratur entſtanden iſt, der \textcolor{blue}{Hauptmann}{}\ledrightnote{\textcolor{blue}{Gerhart Hauptmann}}, in dem man die Zukunft des Deutſchen Dramas ahnen wollte,
                     dieſer \textcolor{blue}{Hauptmann}{}\ledrightnote{\textcolor{blue}{Gerhart Hauptmann}} iſt geweſen, und die
                     deutſche Litteratur geht über ihn hinweg, weil ſie ſchon über manchen
                     kurzlebigen Stern, der an dem Theaterhimmel glänzte, hinweggegangen iſt. Aber
                        \textcolor{blue}{Hauptmann}{}\ledrightnote{\textcolor{blue}{Gerhart Hauptmann}}s Niedergang bedeutet, wie
                     die Dinge einmal liegen, noch mehr. \textcolor{blue}{Hauptmann}{}\ledrightnote{\textcolor{blue}{Gerhart Hauptmann}} war ohne ſeinen Willen der große Neuerer, um den ſich ein
                     ganzes Programm, eine ganze Bewegung gebildet hat; er war der heimliche
                     Diktator der \textcolor{pink}{deutſch}{}\ledrightnote{{$\rightarrow$}\textcolor{pink}{Deutschland}}en
                     Theaterlitteratur. Das alles hat ein Ende. \label{T_L03093-1v}\edtext{Und}{\lemma{\textnormal{\emph{Und}}}\Cendnote{\textnormal{In der
                        Vorlage steht »und«.}}}\label{T_L03093-1h} mit ihm bricht ein Gebäude
                     zuſammen, in dem eine ganze Schar schwächerer, aber ſehr lauter Geiſter Obdach
                     gefunden hat. Der Durchfall des »\textcolor{green}{roten
                        Hahns}{}\ledrightnote{\textcolor{green}{Der rothe Hahn. Tragikomödie in vier Akten}}« iſt ſo etwas wie ein litterariſcher Börſenſturz, wie eine
                     Kataſtrophe, die ihre Wirkung ausüben muß, wenn auch noch frecher als nach dem
                        »\textcolor{green}{Michael Kramer}{}\ledrightnote{\textcolor{green}{Michael Kramer. Drama}}« der Verſuch gemacht
                     werden ſollte, das \textcolor{pink}{deutſch}{}\ledrightnote{{$\rightarrow$}\textcolor{pink}{Deutschland}}e Publikum über die Wahrheit zu täuſchen. Die \textcolor{pink}{Berlin}{}\ledrightnote{\textcolor{pink}{Berlin}}er Litteratur-Tyrannei hat am 27. November ihr \label{T_L03093-2v}\edtext{Ende gefunden}{\lemma{\textnormal{\emph{Ende gefunden}}}\Cendnote{\textnormal{In der Vorlage steht
                        »Endegefunden«.}}}\label{T_L03093-2h}.« –}{}\ledrightnote{{$\rightarrow$}\textcolor{green}{Hauptmanns Niedergang und die Berliner Litteratur-Tyrannei}}}}\pend
           \endnumbering\briefempfaengerindex{Schnitzler, Arthur@\textsc{Schnitzler, Arthur}!zzzGoldmann, Paul@\emph{von Paul Goldmann}!1901-12-041@{4. 12. {[}1901{]}}|)be}\mylabel{h}
\begin{anhang}
\end{anhang}\normalsize

\doendnotes{C}
\bigskip
\vfill

\clearpage

\footnotesize

\lohead{\textsc{register}}

% Definiere theindex-Environment komplett neu ohne reledmac
\makeatletter
\renewenvironment{theindex}{%
  \section*{\indexname}%
  \setlength{\parindent}{0pt}%
  \setlength{\parskip}{0pt plus 0.3pt}%
  \let\item\@idxitem
}{%
  \clearpage
}
\makeatother

\IfFileExists{\jobname-pw.ind}{\input{\jobname-pw.ind}}{}

\end{document}

      