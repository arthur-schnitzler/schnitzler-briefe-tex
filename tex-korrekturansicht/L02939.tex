%% latex-korrekturansicht-vorspann.tex
%% Vorspann für die Korrekturansicht.
%% Lädt die gemeinsame Datei latex-vorspann.tex mit gesetztem Schalter.

\newif\ifkorrekturansicht
\korrekturansichttrue

\input{../tex-inputs/latex-vorspann}


         
         \renewcommand{\erwaehntePersonen}{Personen: Richard Beer-Hofmann}
         \renewcommand{\erwaehnteOrte}{Orte: Berlin, Breslau, Dessauer Straße, Hotel Continental (Berlin), Palasthotel Berlin, Wien}
         \renewcommand{\erwaehnteWerke}{Werke: Der Schleier der Beatrice. Schauspiel in fünf Akten}
               \section[ Paul Goldmann an Arthur Schnitzler, 20. 11. {[}1900{]}]{Paul Goldmann an Arthur Schnitzler, 20. 11. {[}1900{]}}\nopagebreak\mylabel{v}\rehead{ }\normalsize\beginnumbering\briefempfaengerindex{Schnitzler, Arthur@\textsc{Schnitzler, Arthur}!zzzGoldmann, Paul@\emph{von Paul Goldmann}!1900-11-201@{20. 11. {[}1900{]}}|(be} \toendnotes[C]{\smallbreak\pagebreak[2]} \Standort{DLA, A:Schnitzler, HS.NZ85.1.3170.}
\physDesc{Brief, 1 Blatt, 2 Seiten
\newline{}Handschrift: blaue Tinte, deutsche Kurrent
\newline{}Schnitzler: 1) mit Bleistift das Jahr »{[}1{]}900« vermerkt  2) mit rotem Buntstift eine Unterstreichung}\toendnotes[C]{\smallbreak}\pstart
           \noindent{}{\pb}\textcolor{pink}{Berlin}{}\ledrightnote{\textcolor{pink}{Berlin}}, 20. November.\hfill \textcolor{pink}{\textcolor{gray}{\textbf{DESSAUERSTRASSE 19}}}{}\ledrightnote{\textcolor{pink}{Dessauer Straße}}\pend
           \pstart{}Mein lieber Freund,\pend\pstart
           Deine \textcolor{pink}{Breslau}{}\ledrightnote{\textcolor{pink}{Breslau}}er 
                  \textsc{\textcolor{green}{Première}{}\ledrightnote{{$\rightarrow$}\textcolor{green}{Der Schleier der Beatrice. Schauspiel in fünf Akten}}}
                iſt, wie ich höre, \label{K_L02939-1v}\edtext{verſchoben}{\lemma{\textnormal{\emph{verſchoben}}}\Cendnote{\textnormal{siehe Paul Goldmann an Arthur Schnitzler, 12. 11. [1900]}}}\label{K_L02939-1h}, und ich kann Dir daher nochmals Glück auf den Weg wünſchen. Vergiß nicht,
               wenn es irgend geht, mir am Sonntag ein paar Worte zu
               telegraphiren! Dann kommſt Du hoffentlich nach \textcolor{pink}{Berlin}{}\ledrightnote{\textcolor{pink}{Berlin}}. Ich hatte eigentlich gehofft, Du würdeſt ſchon voher auf einige Tage
               herkommen. Bitte, ſteige {\pb}doch diesmal nicht in dem
               ungünſtig \strikeout{gel} und entfernt gelegenen \textsc{\textcolor{pink}{Hôtel Continental}{}\ledrightnote{\textcolor{pink}{Hotel Continental (Berlin)}}} ab, ſondern in dem auch ſonſt weit angenehmeren und auch vornehmeren \label{K_L02939-4v}\edtext{\textsc{\textcolor{pink}{Palast-Hotel}{}\ledrightnote{\textcolor{pink}{Palasthotel Berlin}}}}{\lemma{\textnormal{\emph{Palast-Hotel}}}\Cendnote{\textnormal{Am 28. 11. 1900 speiste
                  \textcolor{blue}{Schnitzler} unmittelbar vor seiner Abreise aus \textcolor{pink}{Berlin} im
                  \textcolor{pink}{Hôtel Continental}, was als Indiz genommen werden kann, dass er 
                  sich nicht an \textcolor{blue}{Goldmann}s Rat hielt und die ganze Zeit über in diesem
                  Hotel wohnte.}}}\label{K_L02939-4h}, das fünf Minuten von m\substVorne{}\textsuperscript{ir}\substDazwischen{}einer\substHinten{} Wohnung entfernt liegt.\pend
           \pstart
           Viele treue Grüße! {\\[\baselineskip]}Dein {\\[\baselineskip]}\spacefill\mbox{Paul Goldmn}\pend
           \leftskip=0em{}\pstart
           \noindent{}Sage doch dieſem \textcolor{blue}{Schurken}{}\ledrightnote{{$\rightarrow$}\textcolor{blue}{Richard Beer-Hofmann}}, dem \textsc{\textcolor{blue}{Richard}{}\ledrightnote{\textcolor{blue}{Richard Beer-Hofmann}}}, er ſoll mir die \label{K_L02939-7v}\edtext{\uline{Photographien} von unſerer Reiſe}{\lemma{\textnormal{\emph{Photographien … Reiſe}}}\Cendnote{\textnormal{siehe Paul Goldmann an Arthur Schnitzler, 16. 6. [1900]}}}\label{K_L02939-7h} ſchicken!\pend
           \endnumbering\briefempfaengerindex{Schnitzler, Arthur@\textsc{Schnitzler, Arthur}!zzzGoldmann, Paul@\emph{von Paul Goldmann}!1900-11-201@{20. 11. {[}1900{]}}|)be}\mylabel{h}\begin{anhang}\end{anhang}\normalsize

\doendnotes{C}
\bigskip
\vfill

\clearpage

\footnotesize

\lohead{\textsc{register}}

% Definiere theindex-Environment komplett neu ohne reledmac
\makeatletter
\renewenvironment{theindex}{%
  \section*{\indexname}%
  \setlength{\parindent}{0pt}%
  \setlength{\parskip}{0pt plus 0.3pt}%
  \let\item\@idxitem
}{%
  \clearpage
}
\makeatother

\IfFileExists{\jobname-pw.ind}{\input{\jobname-pw.ind}}{}

\end{document}

      