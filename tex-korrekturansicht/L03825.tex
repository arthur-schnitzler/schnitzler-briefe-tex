%% latex-korrekturansicht-vorspann.tex
%% Vorspann für die Korrekturansicht.
%% Lädt die gemeinsame Datei latex-vorspann.tex mit gesetztem Schalter.

\newif\ifkorrekturansicht
\korrekturansichttrue

\input{../tex-inputs/latex-vorspann}


\section[Theodor Herzl an Arthur Schnitzler, 10. {[}11.?{]} 1892]{L03825 Theodor Herzl an Arthur Schnitzler, 10. {[}11.?{]} 1892}
\nopagebreak\mylabel{L03825v}
\rehead{ }\normalsize\beginnumbering\briefempfaengerindex{Schnitzler, Arthur@\textsc{Schnitzler, Arthur}!zzzHerzl, Theodor@\emph{von Theodor Herzl}!1892-11-101@{10. {[}11.?{]} 1892}|(be}
\toendnotes[C]{\smallbreak\pagebreak[2]}
\correspDesc{Versand  durch Theodor Herzl am 10. [11.?] 1892 in Paris
\newline{}Erhalt  durch Arthur Schnitzler am [12. 11. 1892] in Wien}\toendnotes[C]{\smallbreak}
\Standort{CUL, Schnitzler, B 39.}
\physDesc{Brief, 1 Blatt, 2 Seiten, 1404 Zeichen
\newline{}Handschrift: schwarze Tinte, lateinische Kurrent
\newline{}Schnitzler: mit Bleistift beschriftet »\noindent{}92, Ende{ / }od Anf 9\textcolor{gray}{3}«  
\newline{}Ordnung: mit Bleistift von unbekannter Hand nummeriert: »6« }
\buchAbdrucke{\weitereDrucke{\emph{10. 10. 1892.} In: Theodor Herzl: \emph{Briefe und
                        autobiographische Notizen 1866–1895}. Bearbeitet von Johannes Wachten in Zusammenarbeit mit Chaya Harel, Daisy Tycho und Manfred Winkler. Berlin, Frankfurt am Main, Wien: \emph{Propyläen} 1983, S. 502 (Briefe und Tagebücher. Herausgegeben von Alex Bein, Hermann Greive, Moshe Schaerf, Julius H. Schoeps und Johannes Wachten, 1).} }\toendnotes[C]{\smallbreak}
\pstart
           {\pb}\textcolor{brown}{\textcolor{gray}{\textbf{NOUVELLE PRESSE LIBRE}}}\orgindex{Neue Freie Presse@Neue Freie Presse|pw}{}\ledrightnote{\textcolor{brown}{Neue Freie Presse}}\hfill \textcolor{pink}{\textcolor{gray}{\textbf{8, Rue de Monceau }}}\oindex{8, rue de Monceau@\textbf{8, rue de Monceau}, \emph{Wohngebäude}|pw}{}\ledrightnote{\textcolor{pink}{8, rue de Monceau}}\pend
           
\pstart
           \textcolor{gray}{\textbf{D\textsuperscript{r}{ }TH. HERZL}}\pend
           
\pstart{}Verehrtester Freund,\pend\vspace{0.5em}
\pstart
           Sie müssen mich schon für sehr ungezogen gehalten haben, und ich war nur beschäftigt. \pend
           
\pstart
           Besten Dank für die \label{K_L03825-1v}\edtext{Uebersendung Ihres \textcolor{green}{Buches}\pwindex{Schnitzler, Arthur 15. 5. 1862 Wien – 21. 10. 1931 ebd.@\textsc{Schnitzler, Arthur} (15. 5. 1862 Wien – 21. 10. 1931 ebd.), \emph{Schriftsteller, Mediziner}!Anatol@\strich\emph{Anatol}|pwv}{}\ledrightnote{{$\rightarrow$}\emph{\textcolor{green}{Anatol}}}}{\lemma{\textnormal{\emph{Uebersendung Ihres Buches}}}\Cendnote{\textnormal{Vgl. A. S.: \emph{Tagebuch}, 29. 10. 1892. Die Datierung dieses Briefes
                  durch \textcolor{blue}{Herzl}\pwindex{Herzl, Theodor 2.\,5.\,1860 Budapest – 3.\,7.\,1904 Edlach@\textsc{Herzl, Theodor} (2.\,5.\,1860 Budapest – 3.\,7.\,1904 Edlach), \emph{Schriftsteller, Journalist}|pwk} dürfte um einen Monat falsch liegen. Der Erhalt des Briefes durch \textcolor{blue}{Schnitzler} am 12. 11. 1892 stützt das.}}}\label{K_L03825-1}. Ich wollte Ihnen erst schreiben, nachdem ich es
               ausgelesen hätte. Gestern hab ich es angefangen und bisher drei \textcolor{green}{Stücke}\pwindex{Schnitzler, Arthur 15. 5. 1862 Wien – 21. 10. 1931 ebd.@\textsc{Schnitzler, Arthur} (15. 5. 1862 Wien – 21. 10. 1931 ebd.), \emph{Schriftsteller, Mediziner}!Frage an das Schicksal@\strich\emph{Die Frage an das Schicksal}|pwv}\pwindex{Schnitzler, Arthur 15. 5. 1862 Wien – 21. 10. 1931 ebd.@\textsc{Schnitzler, Arthur} (15. 5. 1862 Wien – 21. 10. 1931 ebd.), \emph{Schriftsteller, Mediziner}!Weihnachts-Einkäufe@\strich\emph{Weihnachts-Einkäufe}|pwv}\pwindex{Schnitzler, Arthur 15. 5. 1862 Wien – 21. 10. 1931 ebd.@\textsc{Schnitzler, Arthur} (15. 5. 1862 Wien – 21. 10. 1931 ebd.), \emph{Schriftsteller, Mediziner}!Episode@\strich\emph{Episode}|pwv}{}\ledrightnote{{$\rightarrow$}\emph{\textcolor{green}{Die Frage an das Schicksal}}{\newline}{$\rightarrow$}\emph{\textcolor{green}{Weihnachts-Einkäufe}}{\newline}{$\rightarrow$}\emph{\textcolor{green}{Episode}}}
               gelesen. Inzwischen ist mir der \label{K_L03825-2v}\edtext{sybaritische}{\lemma{\textnormal{\emph{sybaritische}}}\Cendnote{\textnormal{schwelgerisch
                  (zurückgehend auf die antike Stadt Sybaris, die mit legendärem Reichtum gesegnet
                  war)}}}\label{K_L03825-2} Einfall gekommen, in den seltenen halben Stunden, wo ich zum Träumeln
               Zeit habe, immer nur eins Ihrer \textcolor{green}{Stückchen}\pwindex{Schnitzler, Arthur 15. 5. 1862 Wien – 21. 10. 1931 ebd.@\textsc{Schnitzler, Arthur} (15. 5. 1862 Wien – 21. 10. 1931 ebd.), \emph{Schriftsteller, Mediziner}!Denksteine@\strich\emph{Denksteine}|pwv}\pwindex{Schnitzler, Arthur 15. 5. 1862 Wien – 21. 10. 1931 ebd.@\textsc{Schnitzler, Arthur} (15. 5. 1862 Wien – 21. 10. 1931 ebd.), \emph{Schriftsteller, Mediziner}!Agonie@\strich\emph{Agonie}|pwv}\pwindex{Schnitzler, Arthur 15. 5. 1862 Wien – 21. 10. 1931 ebd.@\textsc{Schnitzler, Arthur} (15. 5. 1862 Wien – 21. 10. 1931 ebd.), \emph{Schriftsteller, Mediziner}!Anatols Hochzeitsmorgen@\strich\emph{Anatols Hochzeitsmorgen}|pwv}\pwindex{Schnitzler, Arthur 15. 5. 1862 Wien – 21. 10. 1931 ebd.@\textsc{Schnitzler, Arthur} (15. 5. 1862 Wien – 21. 10. 1931 ebd.), \emph{Schriftsteller, Mediziner}!Abschiedssouper@\strich\emph{Abschiedssouper}|pwv}{}\ledrightnote{{$\rightarrow$}\emph{\textcolor{green}{Denksteine}}{\newline}{$\rightarrow$}\emph{\textcolor{green}{Agonie}}{\newline}{$\rightarrow$}\emph{\textcolor{green}{Anatols Hochzeitsmorgen}}{\newline}{$\rightarrow$}\emph{\textcolor{green}{Abschiedssouper}}} zu
               lesen. So wart ich also nicht, bis ich zu Ende bin, um Ihnen zu danken. \pend
           
\pstart
           Die erste \textcolor{green}{Geschichte}\pwindex{Schnitzler, Arthur 15. 5. 1862 Wien – 21. 10. 1931 ebd.@\textsc{Schnitzler, Arthur} (15. 5. 1862 Wien – 21. 10. 1931 ebd.), \emph{Schriftsteller, Mediziner}!Frage an das Schicksal@\strich\emph{Die Frage an das Schicksal}|pwv}{}\ledrightnote{{$\rightarrow$}\emph{\textcolor{green}{Die Frage an das Schicksal}}} (\textcolor{green}{Frage an das {\pb}Schicksal}\pwindex{Schnitzler, Arthur 15. 5. 1862 Wien – 21. 10. 1931 ebd.@\textsc{Schnitzler, Arthur} (15. 5. 1862 Wien – 21. 10. 1931 ebd.), \emph{Schriftsteller, Mediziner}!Frage an das Schicksal@\strich\emph{Die Frage an das Schicksal}|pw}{}\ledrightnote{\textcolor{green}{Die Frage an das Schicksal}}) finde ich sehr
               gelungen. Ich kannte sie schon – woher nur? Aus der \label{K_L03825-3v}\edtext{\textcolor{pink}{Brünner}\oindex{Brünn@\textbf{Brünn}|pw}{}\ledrightnote{\textcolor{pink}{Brünn}}{ }\textcolor{green}{Monatsschrift}\pwindex{Moderne Dichtung. Monatsschrift für Literatur und Kritik@\emph{Moderne Dichtung. Monatsschrift für Literatur und Kritik}|pwv}{}\ledrightnote{{$\rightarrow$}\emph{\textcolor{green}{Moderne Dichtung. Monatsschrift für Literatur und Kritik}}}}{\lemma{\textnormal{\emph{Brünner Monatsschrift}}}\Cendnote{\textnormal{Der Einakter \emph{\textcolor{green}{Die Frage an das Schicksal}\pwindex{Schnitzler, Arthur 15. 5. 1862 Wien – 21. 10. 1931 ebd.@\textsc{Schnitzler, Arthur} (15. 5. 1862 Wien – 21. 10. 1931 ebd.), \emph{Schriftsteller, Mediziner}!Frage an das Schicksal@\strich\emph{Die Frage an das Schicksal}|pwk}} erschien zuerst in: \emph{\textcolor{green}{Moderne Dichtung}\pwindex{Moderne Dichtung. Monatsschrift für Literatur und Kritik@\emph{Moderne Dichtung. Monatsschrift für Literatur und Kritik}|pwk}}, Bd. 1, H. 5, 1.\,5.\,1890, S. 299–306.}}}\label{K_L03825-3} (die man Leichtes \uline{Tuch} nennen könnte) oder anderswoher?\pend
           
\pstart
           In der \textcolor{green}{zweiten}\pwindex{Schnitzler, Arthur 15. 5. 1862 Wien – 21. 10. 1931 ebd.@\textsc{Schnitzler, Arthur} (15. 5. 1862 Wien – 21. 10. 1931 ebd.), \emph{Schriftsteller, Mediziner}!Weihnachts-Einkäufe@\strich\emph{Weihnachts-Einkäufe}|pwv}{}\ledrightnote{{$\rightarrow$}\emph{\textcolor{green}{Weihnachts-Einkäufe}}}, die mir zu
               lang ausgesponnen scheint, erwartete ich eine andere weiberkundigere Pointe. Die Frau
               erfährt, dass das »Mädl« nichts Anderes auf der Welt hat, als den Anatol – \uline{darum} nimmt sie ihr ihn weg. Heh?\pend
           
\pstart
           Ich habe nicht Zeit genug, Ihnen alles Gute zu sagen, was ich über die dritte \textcolor{green}{Episode}\pwindex{Schnitzler, Arthur 15. 5. 1862 Wien – 21. 10. 1931 ebd.@\textsc{Schnitzler, Arthur} (15. 5. 1862 Wien – 21. 10. 1931 ebd.), \emph{Schriftsteller, Mediziner}!Episode@\strich\emph{Episode}|pwv}{}\ledrightnote{{$\rightarrow$}\emph{\textcolor{green}{Episode}}} denke.\pend
           
\pstart
           Wer ist \textcolor{blue}{Loris}\pwindex{Hofmannsthal, Hugo von 1.\,2.\,1874 Wien – 15.\,7.\,1929 Rodaun@\textsc{Hofmannsthal, Hugo von} (1.\,2.\,1874 Wien – 15.\,7.\,1929 Rodaun), \emph{Schriftsteller}|pwv}{}\ledrightnote{{$\rightarrow$}\emph{\textcolor{blue}{Hugo von Hofmannsthal}}}? Auch Sie?
               Jedenfalls sind diese paar \textcolor{green}{Verse}\pwindex{Prolog [zum Anatol]@\emph{Prolog [zum Anatol]}|pwv}{}\ledrightnote{{$\rightarrow$}\emph{\textcolor{green}{Prolog [zum Anatol]}}}{ }\label{K_L03825-4v}\edtext{zum Küssen}{\lemma{\textnormal{\emph{zum Küssen}}}\Cendnote{\textnormal{\textcolor{blue}{Schnitzler} vermerkt den Erhalt dieses Briefes am 12. 11. 1892 im \emph{\textcolor{green}{Tagebuch}\pwindex{Schnitzler, Arthur 15. 5. 1862 Wien – 21. 10. 1931 ebd.@\textsc{Schnitzler, Arthur} (15. 5. 1862 Wien – 21. 10. 1931 ebd.), \emph{Schriftsteller, Mediziner}!Tagebuch@\strich\emph{Tagebuch}|pwk}}: »Brief von \textcolor{blue}{Theodor Herzl}\pwindex{Herzl, Theodor 2.\,5.\,1860 Budapest – 3.\,7.\,1904 Edlach@\textsc{Herzl, Theodor} (2.\,5.\,1860 Budapest – 3.\,7.\,1904 Edlach), \emph{Schriftsteller, Journalist}|pw} über \textcolor{green}{Anatol}\pwindex{Schnitzler, Arthur 15. 5. 1862 Wien – 21. 10. 1931 ebd.@\textsc{Schnitzler, Arthur} (15. 5. 1862 Wien – 21. 10. 1931 ebd.), \emph{Schriftsteller, Mediziner}!Anatol@\strich\emph{Anatol}|pw}; er findet \textcolor{blue}{Loris}\pwindex{Hofmannsthal, Hugo von 1.\,2.\,1874 Wien – 15.\,7.\,1929 Rodaun@\textsc{Hofmannsthal, Hugo von} (1.\,2.\,1874 Wien – 15.\,7.\,1929 Rodaun), \emph{Schriftsteller}|pw}’ \textcolor{green}{Verse}\pwindex{Prolog [zum Anatol]@\emph{Prolog [zum Anatol]}|pwv} zum
                     küssen.―« }}}\label{K_L03825-4}. Schreiben Sie mir, wer \textcolor{blue}{Loris}\pwindex{Hofmannsthal, Hugo von 1.\,2.\,1874 Wien – 15.\,7.\,1929 Rodaun@\textsc{Hofmannsthal, Hugo von} (1.\,2.\,1874 Wien – 15.\,7.\,1929 Rodaun), \emph{Schriftsteller}|pwv}{}\ledrightnote{{$\rightarrow$}\emph{\textcolor{blue}{Hugo von Hofmannsthal}}} ist. Ergreifen Sie überhaupt Ihre gute \label{K_L03825-5v}\edtext{Feder von \textcolor{pink}{Toledo}\oindex{Toledo@\textbf{Toledo}, \emph{Verwaltungsgebiet}|pw}{}\ledrightnote{\textcolor{pink}{Toledo}}}{\lemma{\textnormal{\emph{Feder von Toledo}}}\Cendnote{\textnormal{Toledostahl war seit der Antike berühmt für seine Härte und Eignung zum Schmieden von Waffen}}}\label{K_L03825-5} und erzählen Sie mir, was \textcolor{pink}{Wien}\oindex{Wien@\textbf{Wien}, \emph{Verwaltungsgebiet}|pw}{}\ledrightnote{\textcolor{pink}{Wien}}{ }\label{K_L03825-6v}\edtext{\begin{otherlanguage}{french}en l’an de grâce\end{otherlanguage}}{\lemma{\textnormal{\emph{en l’an de grâce}}}\Cendnote{\textnormal{französisch: im Jahr der Gnade}}}\label{K_L03825-6}{ }1892 ist. Recht ausführlich, denn Sie haben Zeit, Sie vielleicht
               Glücklicher.\pend
           
\pstart
           Ich grüsse Sie recht herzlich und ergeben Ihr{\\[\baselineskip]}\spacefill\mbox{Herzl}\pend
           \leftskip=0em{}
\pstart
           \label{K_L03825-7v}\edtext{10/X 92}{\lemma{\textnormal{\emph{10/X 92}}}\Cendnote{\textnormal{Zur fehlerhaften Datierung der Monatsangabe siehe oben.}}}\label{K_L03825-7}\pend
           
\pstart
           \label{T_L03825-1v}\edtext{Erzählen Sie mir was es in Kunst u.
                  Zeitung in \textcolor{pink}{Wien}\oindex{Wien@\textbf{Wien}, \emph{Verwaltungsgebiet}|pw}{}\ledrightnote{\textcolor{pink}{Wien}} gibt. Ich kenne Alles nur aus
                  den Journalen}{\lemma{\textnormal{\emph{Erzählen … Journalen}}}\Cendnote{\textnormal{seitlich entlang des
                     Falzes geschrieben}}}\label{T_L03825-1}\pend
           \selectlanguage{ngerman}\endnumbering\briefempfaengerindex{Schnitzler, Arthur@\textsc{Schnitzler, Arthur}!zzzHerzl, Theodor@\emph{von Theodor Herzl}!1892-11-101@{10. {[}11.?{]} 1892}|)be}\mylabel{L03825h}
\begin{anhang}
\end{anhang}\normalsize

\doendnotes{C}
\bigskip
\vfill

\clearpage

\footnotesize

\lohead{\textsc{register}}

% Definiere theindex-Environment komplett neu ohne reledmac
\makeatletter
\renewenvironment{theindex}{%
  \section*{\indexname}%
  \setlength{\parindent}{0pt}%
  \setlength{\parskip}{0pt plus 0.3pt}%
  \let\item\@idxitem
}{%
  \clearpage
}
\makeatother

\IfFileExists{\jobname-pw.ind}{\input{\jobname-pw.ind}}{}

\end{document}

      