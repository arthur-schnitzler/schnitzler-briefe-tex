%% latex-korrekturansicht-vorspann.tex
%% Vorspann für die Korrekturansicht.
%% Lädt die gemeinsame Datei latex-vorspann.tex mit gesetztem Schalter.

\newif\ifkorrekturansicht
\korrekturansichttrue

\input{../tex-inputs/latex-vorspann}


\renewcommand{\erwaehntePersonen}{Personen: Richard Beer-Hofmann, Paula Beer-Hofmann, Else Berger, Paul Goldmann, Moriz Metzl, Friedrich Mitterwurzer, Wilhelmine Mitterwurzer, Ottilie Salten, Franziska Schlesinger, Emil Schlesinger, Margherita Schlesinger}
\renewcommand{\erwaehnteOrte}{Orte: Bad Aussee, Bad Ischl, Kopenhagen, Skodsborg, Wien}
\renewcommand{\erwaehnteWerke}{Werke: Anatol, Das Märchen. Schauspiel in drei Aufzügen, Ein Engagement, Herodes und Mariamne. Eine Tragödie in fünf Aufzügen, Judith. Eine Tragödie in fünf Aufzügen}
\section[ Felix Salten an Arthur Schnitzler, 8. 8. 1896]{Felix Salten an Arthur Schnitzler, 8. 8. 1896}
\nopagebreak\mylabel{v}
\rehead{ }\normalsize\beginnumbering\briefempfaengerindex{Schnitzler, Arthur@\textsc{Schnitzler, Arthur}!zzzSalten, Felix@\emph{von Felix Salten}!1896-08-081@{8. 8. 1896}|(be}
\toendnotes[C]{\smallbreak\pagebreak[2]}\Standort{CUL, Schnitzler, B 89, A 1.}
\physDesc{Brief, 1 Blatt, 1 Seite, 928 Zeichen
\newline{}Handschrift: blaue Tinte, lateinische Kurrent
\newline{}Ordnung: mit Bleistift von unbekannter Hand nummeriert: »77« }\toendnotes[C]{\smallbreak}
\pstart
           \raggedleft{}{\pb}\textcolor{pink}{Ischl}{}\ledrightnote{\textcolor{pink}{Bad Ischl}}, 8. Aug. 96.\pend
           
\pstart
           Lieber Arthur, die \label{K_L03178-1v}\edtext{Tischkarte}{\lemma{\textnormal{\emph{Tischkarte}}}\Cendnote{\textnormal{vgl. Felix Salten u. a. an Arthur Schnitzler, 6. 8. 1896}}}\label{K_L03178-1h}, welche Ihnen von \textcolor{blue}{Schlesinger}{}\ledrightnote{\textcolor{blue}{Franziska Schlesinger}{\newline}\textcolor{blue}{Emil Schlesinger}}s aus zukam, kann auch als Dokument für die Langeweile gelten,
               mit der man hier seine Zeit hin bringt. Ich wohne mit den \textcolor{blue}{Mädeln}{}\ledrightnote{{$\rightarrow$}\textcolor{blue}{Else Berger}{\newline}{$\rightarrow$}\textcolor{blue}{Margherita Schlesinger}} auf einem
               Gang, was einige Annäherung unvermeidlich mit sich gebracht hat. Frl. \textcolor{blue}{M.}{}\ledrightnote{\textcolor{blue}{Ottilie Salten}} und ich stehen geradeso zu einander, wie in
                  \textcolor{pink}{Wien}{}\ledrightnote{\textcolor{pink}{Wien}}. Die \label{K_L03178-2v}\edtext{Radtour}{\lemma{\textnormal{\emph{Radtour}}}\Cendnote{\textnormal{siehe Felix Salten an Arthur Schnitzler, 21. 7. 1896}}}\label{K_L03178-2h} konnte noch nicht unternommen werden, weil ihr 83 jähriger \label{K_L03178-3v}\edtext{\textcolor{blue}{Vater}{}\ledrightnote{{$\rightarrow$}\textcolor{blue}{Moriz Metzl}} krank}{\lemma{\textnormal{\emph{Vater krank}}}\Cendnote{\textnormal{\textcolor{blue}{Moriz Metzl} verstarb noch im selben Jahr,
                  am 21. 12. 1896.}}}\label{K_L03178-3h} ist, und außerdem noch,
               weil es beständig schüttet.\pend
           
\pstart
           Neulich war ich bei \label{K_L03178-4v}\edtext{\textcolor{blue}{Mitterwurzer}{}\ledrightnote{\textcolor{blue}{Friedrich Mitterwurzer}{\newline}\textcolor{blue}{Wilhelmine Mitterwurzer}} zu Tisch in \textcolor{pink}{Aussee}{}\ledrightnote{\textcolor{pink}{Bad Aussee}}. \textcolor{blue}{Er}{}\ledrightnote{{$\rightarrow$}\textcolor{blue}{Friedrich Mitterwurzer}} war auch da, und fand Ihren \textcolor{green}{Anatol}{}\ledrightnote{\textcolor{green}{Anatol}}, wie auch das \textcolor{green}{Märchen}{}\ledrightnote{\textcolor{green}{Das Märchen. Schauspiel in drei Aufzügen}} »frivol«}{\lemma{\textnormal{\emph{Mitterwurzer … »frivol«}}}\Cendnote{\textnormal{siehe A. S.: \emph{Tagebuch}, 5. 9. 1896}}}\label{K_L03178-4h}. Er studirt den \textcolor{green}{Holofernes}{}\ledrightnote{{$\rightarrow$}\textcolor{green}{Judith. Eine Tragödie in fünf Aufzügen}} und wird auf meine Veranlaßung auch den \textcolor{green}{Herodes}{}\ledrightnote{{$\rightarrow$}\textcolor{green}{Herodes und Mariamne. Eine Tragödie in fünf Aufzügen}} ansehen. Mein \label{K_L03178-5v}\edtext{\textcolor{green}{Stück}{}\ledrightnote{{$\rightarrow$}\textcolor{green}{Ein Engagement}}}{\lemma{\textnormal{\emph{Stück}}}\Cendnote{\textnormal{Es könnte sich um das kurze Stück \emph{\textcolor{green}{Ein Engagement}} gehandelt haben, das \textcolor{blue}{Salten} am 11. 12. 1899 (S. 5–6) in der \emph{\textcolor{green}{Wiener Allgemeinen Montags-Zeitung}} veröffentlicht
                  hatte.}}}\label{K_L03178-5h} (den Einacter) hab ich ihm erzählt, und es gefiel ihm ganz
               besonders. Man braucht Einacter dieses Jahr und so hab’ ich vielleicht einige Chance,
               wenn ich nur damit zustande komme. Grüßen Sie \textcolor{blue}{Richard}{}\ledrightnote{\textcolor{blue}{Richard Beer-Hofmann}} und \textcolor{blue}{Paula}{}\ledrightnote{\textcolor{blue}{Paula Beer-Hofmann}}, und – \label{K_L03178-6v}\edtext{wenn er schon da ist}{\lemma{\textnormal{\emph{wenn er schon da ist}}}\Cendnote{\textnormal{\textcolor{blue}{Paul Goldmann} war am 5. 8. 1896 in \textcolor{pink}{Kopenhagen} angekommen. Zu diesem Zeitpunkt war er bereits in \textcolor{pink}{Skodsborg}.}}}\label{K_L03178-6h} – D\textsuperscript{r}{ }\textcolor{blue}{Goldmann}{}\ledrightnote{\textcolor{blue}{Paul Goldmann}}.\pend
           
\pstart
           Herzlichst Ihr {\\[\baselineskip]}\spacefill\mbox{Salten}\pend
           \leftskip=0em{}\endnumbering\briefempfaengerindex{Schnitzler, Arthur@\textsc{Schnitzler, Arthur}!zzzSalten, Felix@\emph{von Felix Salten}!1896-08-081@{8. 8. 1896}|)be}\mylabel{h}  \normalsize

\doendnotes{C}
\bigskip
\vfill

\clearpage

\footnotesize

\lohead{\textsc{register}}

% Definiere theindex-Environment komplett neu ohne reledmac
\makeatletter
\renewenvironment{theindex}{%
  \section*{\indexname}%
  \setlength{\parindent}{0pt}%
  \setlength{\parskip}{0pt plus 0.3pt}%
  \let\item\@idxitem
}{%
  \clearpage
}
\makeatother

\IfFileExists{\jobname-pw.ind}{\input{\jobname-pw.ind}}{}

\end{document}

      