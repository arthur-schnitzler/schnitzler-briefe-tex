%% latex-korrekturansicht-vorspann.tex
%% Vorspann für die Korrekturansicht.
%% Lädt die gemeinsame Datei latex-vorspann.tex mit gesetztem Schalter.

\newif\ifkorrekturansicht
\korrekturansichttrue

\input{../tex-inputs/latex-vorspann}


\renewcommand{\erwaehntePersonen}{Personen: Richard Lux}
\renewcommand{\erwaehnteInstitutionen}{Institutionen: Wiener Verlag}
\renewcommand{\erwaehnteOrte}{Orte: Leipzig, Wien}
\renewcommand{\erwaehnteWerke}{Werke: Bibliothek moderner deutscher Autoren, Der Schrei der Liebe. Novelle}
\section[ Felix Salten: Widmungsexemplar Der Schrei der Liebe für Arthur Schnitzler, 22. 10. 1904]{Felix Salten: Widmungsexemplar Der Schrei der Liebe für Arthur
               Schnitzler, 22. 10. 1904}
\nopagebreak\mylabel{v}
\rehead{ }\normalsize\beginnumbering\briefempfaengerindex{Schnitzler, Arthur@\textsc{Schnitzler, Arthur}!zzzSalten, Felix@\emph{von Felix Salten}!1904-10-221@{22. 10. 1904}|(be}
\toendnotes[C]{\smallbreak\pagebreak[2]}\Standort{DLA, G:Schnitzler, Arthur (Sammlung Heinrich Schnitzler).}
\physDesc{Widmung am Titelblatt, 67 Zeichen
\newline{}Handschrift: schwarze Tinte, lateinische Kurrent}
\pstart
           \noindent{}\centering{}{\pb}\textcolor{gray}{\textbf{\textcolor{green}{Der {\\}Schrei der Liebe}{}\ledrightnote{\textcolor{green}{Der Schrei der Liebe. Novelle}}}}\pend
           {\bigskip}
\pstart
           \noindent{}Meinem lieben Arthur Schnitzler\pend
           \pstart herzl. \spacefill\mbox{Felix Salten}\pend{}
\pstart
           \textcolor{pink}{Wien}{}\ledrightnote{\textcolor{pink}{Wien}}, 22. X. 04.\pend
           {\bigskip}
\pstart
           \noindent{}\textcolor{gray}{\textbf{\textcolor{green}{Bibl. mod. deutſcher Autoren}{}\ledrightnote{\textcolor{green}{Bibliothek moderner deutscher Autoren}}. Band 5.}}\pend
           {\bigskip}
\pstart
           \noindent{}\centering{}{\pb}\textcolor{gray}{\textbf{Felix}}\pend
           
\pstart
           \noindent{}\centering{}\textcolor{gray}{\textbf{Salten}}\pend
           
\pstart
           \noindent{}\centering{}{\pb}\textcolor{gray}{\textbf{\textcolor{green}{Der Schrei {\\}der Liebe}{}\ledrightnote{\textcolor{green}{Der Schrei der Liebe. Novelle}}}}\pend
           
\pstart
           \noindent{}\centering{}\textcolor{gray}{\textbf{Novelle}}\pend
           
\pstart
           \noindent{}\centering{}\textcolor{gray}{\textbf{Umſchlag von \textcolor{blue}{Richard
                  Lux}{}\ledrightnote{\textcolor{blue}{Richard Lux}}}}\pend
           
\pstart
           \noindent{}\centering{}\textcolor{gray}{\textbf{\textbf{\textcolor{brown}{Wiener Verlag}{}\ledrightnote{\textcolor{brown}{Wiener Verlag}}}}}\pend
           
\pstart
           \noindent{}\centering{}\textcolor{gray}{\textbf{\textcolor{pink}{Wien}{}\ledrightnote{\textcolor{pink}{Wien}} und \textcolor{pink}{Leipzig}{}\ledrightnote{\textcolor{pink}{Leipzig}}}}\pend
           
\pstart
           \noindent{}\centering{}\textcolor{gray}{\textbf{1905}}\pend
           \endnumbering\briefempfaengerindex{Schnitzler, Arthur@\textsc{Schnitzler, Arthur}!zzzSalten, Felix@\emph{von Felix Salten}!1904-10-221@{22. 10. 1904}|)be}\mylabel{h}  \normalsize

\doendnotes{C}
\bigskip
\vfill

\clearpage

\footnotesize

\lohead{\textsc{register}}

% Definiere theindex-Environment komplett neu ohne reledmac
\makeatletter
\renewenvironment{theindex}{%
  \section*{\indexname}%
  \setlength{\parindent}{0pt}%
  \setlength{\parskip}{0pt plus 0.3pt}%
  \let\item\@idxitem
}{%
  \clearpage
}
\makeatother

\IfFileExists{\jobname-pw.ind}{\input{\jobname-pw.ind}}{}

\end{document}

      