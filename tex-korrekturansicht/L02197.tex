%% latex-korrekturansicht-vorspann.tex
%% Vorspann für die Korrekturansicht.
%% Lädt die gemeinsame Datei latex-vorspann.tex mit gesetztem Schalter.

\newif\ifkorrekturansicht
\korrekturansichttrue

\input{../tex-inputs/latex-vorspann}


               \section[Hugo von Hofmannsthal an Arthur Schnitzler, 24. 9. {[}1914{]}]{ Hugo von Hofmannsthal an Arthur Schnitzler, 24. 9. {[}1914{]}}\nopagebreak\mylabel{v}\rehead{ }\normalsize\beginnumbering\briefempfaengerindex{Schnitzler, Arthur@\textsc{Schnitzler, Arthur}!zzzHofmannsthal, Hugo von@\emph{von Hugo von Hofmannsthal}!1914-09-241@{24. 9. {[}1914{]}}|(be} \toendnotes[C]{\smallbreak\pagebreak[2]} \Standort{CUL, Schnitzler, B 43.}
\physDesc{Brief, 1 Blatt1 Blatt
\newline{}Handschrift: schwarze Tinte, deutsche Kurrent\newline{}Beilage: \textcolor{blue}{Alexander Hoyos}: Brief, 1 Blatt, 3 Seiten, schwarze Tinte,
                                 Lateinschrift 
\newline{}Schnitzler: 1) mit Bleistift beschriftet: »Hugo«
                                  2) mit rotem Buntstift eine Unterstreichung\newline{}Ordnung: 1) mit Bleistift von unbekannter Hand nummeriert: »\strikeout{329}« 2) mit Bleistift von unbekannter Hand nummeriert: »352«}\buchAbdrucke{\weitereDrucke{Hugo von Hofmannsthal, Arthur Schnitzler: \emph{Briefwechsel}. Hg. Therese Nickl und Heinrich Schnitzler. Frankfurt am Main: \emph{S. Fischer} 1964, S. 277.} }\toendnotes[C]{\smallbreak}\pstart
           \raggedleft{}{\pb}\textcolor{pink}{R.}{}\ledrightnote{\textcolor{pink}{Rodaun}}{ }24. IX.{\\}abends. \pend
           \pstart{}mein lieber Arthur\pend\pstart
           hier iſt die Antwort von \textcolor{blue}{Alexander Hoyos}{}\ledrightnote{\textcolor{blue}{Sándor Hoyos}}
               (Cabinetschef) bezüglich der \textcolor{pink}{rumäniſchen}{}\ledrightnote{\textcolor{pink}{Rumänien}} Zeitungen.
               Das schwer leſerliche Wort heißt \uline{Erpreſſer}.\hspace*{1.5em}Ich bin noch ziemlich unwohl und ſchwach, muſs viel
               erledigen, daher die Kürze.\pend
           \pstart
           Alles Liebe an \textcolor{blue}{Olga}{}\ledrightnote{\textcolor{blue}{Olga Schnitzler}}.\pend
           \pstart
           Ihr{\\[\baselineskip]}\spacefill\mbox{Hugo.}\pend
           \leftskip=0em{}{\bigskip}\pstart
           \noindent{}{\pb}\textcolor{gray}{\textbf{\textcolor{brown}{Ministère Imperial et Royal}{}\ledrightnote{\textcolor{brown}{Ministerium für Äußeres}}}}\hfill {[}hs. Hoyos:{]} 22/9 1914\pend
           \pstart
           \textcolor{gray}{\textbf{\textcolor{brown}{des affaires étrangères}{}\ledrightnote{\textcolor{brown}{Ministerium für Äußeres}}.}}\pend
           \pstart
           \textcolor{gray}{\textbf{\textsc{Cabinet du \textcolor{blue}{Ministre}{}\ledrightnote{→\textcolor{blue}{Leopold von Berchtold}}.}}}\pend
           \pstart{}Lieber Freund\pend\pstart
           Bitte verzeihe dass ich Dir erst heute für Deine freundliche Anregung vom
                  15. d. Mts. danke, ich war auf 2 Tage verreist und nach meiner
               Rückkehr sehr beschäftigt. Wir haben schon seit einiger Zeit eine Aktion im Sinne
               Deines Briefs {\pb}eingeleitet,
               hoffentlich wird sie von Erfolg begleitet sein{[},{]} leider sind
               unsere Feinde auch sehr auch sehr freigebig und wissen unsere Bemühungen in
               geschickter Weise auszugleichen. So werden die Erpresser immer reicher ohne ihre
               Haltung ändern zu müssen.\pend
           \pstart
           Mit besten Grüßen bin ich {\pb}Dein sehr ergebener {\\[\baselineskip]}\spacefill\mbox{A. Hoyos.}\pend
           \leftskip=0em{}\endnumbering\briefempfaengerindex{Schnitzler, Arthur@\textsc{Schnitzler, Arthur}!zzzHofmannsthal, Hugo von@\emph{von Hugo von Hofmannsthal}!1914-09-241@{24. 9. {[}1914{]}}|)be}\mylabel{h}  \normalsize

\doendnotes{C}
\bigskip
\vfill

\clearpage

\footnotesize

\lohead{\textsc{register}}

% Definiere theindex-Environment komplett neu ohne reledmac
\makeatletter
\renewenvironment{theindex}{%
  \section*{\indexname}%
  \setlength{\parindent}{0pt}%
  \setlength{\parskip}{0pt plus 0.3pt}%
  \let\item\@idxitem
}{%
  \clearpage
}
\makeatother

\IfFileExists{\jobname-pw.ind}{\input{\jobname-pw.ind}}{}

\end{document}

      