%% latex-korrekturansicht-vorspann.tex
%% Vorspann für die Korrekturansicht.
%% Lädt die gemeinsame Datei latex-vorspann.tex mit gesetztem Schalter.

\newif\ifkorrekturansicht
\korrekturansichttrue

\input{../tex-inputs/latex-vorspann}


\renewcommand{\erwaehntePersonen}{Personen: Frieda Pollak, Felix Salten}
\renewcommand{\erwaehnteOrte}{Orte: San Marco, Stazione di Venezia Santa Lucia, Sternwartestraße 71, Torre dell’Orologio, Venedig, Wien, Österreich}
\renewcommand{\erwaehnteWerke}{}
\section[ Felix Salten an Arthur Schnitzler, 29. 3. 1922]{Felix Salten an Arthur Schnitzler, 29. 3. 1922}
\nopagebreak\mylabel{v}
\rehead{ }\normalsize\beginnumbering\briefempfaengerindex{Schnitzler, Arthur@\textsc{Schnitzler, Arthur}!zzzSalten, Felix@\emph{von Felix Salten}!1922-03-291@{29. 3. 1922}|(be}
\toendnotes[C]{\smallbreak\pagebreak[2]}\Standort{CUL, Schnitzler, B 89, B 2.}
\physDesc{Bildpostkarte, 397 Zeichen
\newline{}Handschrift: schwarze Tinte, lateinische Kurrent
\newline{}Versand: 1) Stempel: »\nobreak{}Pregate i vostri corrispondenti di aggiungere
                                       all’indirizzo il numero del quartiere postale\nobreak{}«.   2) Stempel: »\nobreak{}\oindex{Stazione di Venezia Santa Lucia@\textbf{Stazione di Venezia Santa Lucia}, \emph{Bahnhofsgebäude (K.BHF)}|pwk}Venezia Ferrovia, 29. III 1922, 23–24\nobreak{}«. 
\newline{}Ordnung: 1) mit Bleistift von \textcolor{blue}{Frieda Pollak} (?) mit
                                 dem Buchstaben »A« (Abgeschrieben/Abschrift)
                                 gekennzeichnet  2) mit Bleistift von unbekannter Hand nummeriert: »289«}\toendnotes[C]{\smallbreak}\pstart{}{\pb}\textcolor{pink}{Aust\textcolor{gray}{ria}}{}\ledrightnote{\textcolor{pink}{Österreich}}\pend{}\pstart{}Herrn D\textsuperscript{r} Arthur Schnitzler\pend{}\pstart{}\textcolor{pink}{XVIII. Sternwartestrasse 71}{}\ledrightnote{\textcolor{pink}{Sternwartestraße 71}}\pend{}\pstart{}\textcolor{pink}{Wien}{}\ledrightnote{\textcolor{pink}{Wien}}\pend{}
{\bigskip}
\pstart
           \noindent{}\centering{}{\pb}\textcolor{gray}{\textbf{\textcolor{pink}{VENEZIA}{}\ledrightnote{\textcolor{pink}{Venedig}} – \textcolor{pink}{Chiesa S. Marco}{}\ledrightnote{\textcolor{pink}{San Marco}} e \textcolor{pink}{Torre dell’ Orologio}{}\ledrightnote{\textcolor{pink}{Torre dell’Orologio}}}}\pend
           
\pstart
           \noindent{}{\pb}Lieber, es ist schon sehr schön, wieder \textcolor{pink}{hier}{}\ledrightnote{{$\rightarrow$}\textcolor{pink}{Venedig}} zu sein. Bin heute vier Stunden spazieren gegangen. Die Leute sind so freudig, als
               wären auch sie des Wiedersehens froh. Es sind fast gar keine Fremden da. Ich glaube,
               man kann hier mit 50–60 Lire im Tag gut auskommen. Das ist, an unseren Preisen
               gemessen, nicht viel.\pend
           \pstart Herzlichst Ihr \spacefill\mbox{Felix Salten}\pend{}\endnumbering\briefempfaengerindex{Schnitzler, Arthur@\textsc{Schnitzler, Arthur}!zzzSalten, Felix@\emph{von Felix Salten}!1922-03-291@{29. 3. 1922}|)be}\mylabel{h}  \normalsize

\doendnotes{C}
\bigskip
\vfill

\clearpage

\footnotesize

\lohead{\textsc{register}}

% Definiere theindex-Environment komplett neu ohne reledmac
\makeatletter
\renewenvironment{theindex}{%
  \section*{\indexname}%
  \setlength{\parindent}{0pt}%
  \setlength{\parskip}{0pt plus 0.3pt}%
  \let\item\@idxitem
}{%
  \clearpage
}
\makeatother

\IfFileExists{\jobname-pw.ind}{\input{\jobname-pw.ind}}{}

\end{document}

      