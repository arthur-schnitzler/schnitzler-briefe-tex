%% latex-korrekturansicht-vorspann.tex
%% Vorspann für die Korrekturansicht.
%% Lädt die gemeinsame Datei latex-vorspann.tex mit gesetztem Schalter.

\newif\ifkorrekturansicht
\korrekturansichttrue

\input{../tex-inputs/latex-vorspann}


\renewcommand{\erwaehntePersonen}{Personen: Felix Salten, Heinrich Schnitzler, Lilly Schnitzler}
\renewcommand{\erwaehnteOrte}{Orte: Assuan, Karnak Tempelanlage, Old Cataract Hotel, Sternwartestraße 71, Wien, Österreich}
\renewcommand{\erwaehnteWerke}{}
\section[ Felix Salten an Arthur Schnitzler, 1. 3. {[}1924{]}]{Felix Salten an Arthur Schnitzler, 1. 3. {[}1924{]}}
\nopagebreak\mylabel{v}
\rehead{ }\normalsize\beginnumbering\briefempfaengerindex{Schnitzler, Arthur@\textsc{Schnitzler, Arthur}!zzzSalten, Felix@\emph{von Felix Salten}!1924-03-011@{1. 3. {[}1924{]}}|(be}
\toendnotes[C]{\smallbreak\pagebreak[2]}\Standort{CUL, Schnitzler, B 89, B 2.}
\physDesc{Bildpostkarte, 244 Zeichen
\newline{}Handschrift: schwarze Tinte, lateinische Kurrent
\newline{}Versand: Stempel: »\nobreak{}\oindex{Old Cataract Hotel@\textbf{Old Cataract Hotel}, \emph{Hotel (K.HTL)}|pwk}Cata\textcolor{gray}{ract Hote}{[}l{]}, 3. \textcolor{gray}{Mar}{[}ch 1924{]}\nobreak{}«.  
\newline{}Ordnung: mit Bleistift von unbekannter Hand nummeriert: »295« }\pstart{}{\pb}\textcolor{pink}{Austria}{}\ledrightnote{\textcolor{pink}{Österreich}}\pend{}\pstart{}Herrn D\textsuperscript{r} Arthur Schnitzler\pend{}\pstart{}\begin{otherlanguage}{english}\textcolor{pink}{Vienna}{}\ledrightnote{\textcolor{pink}{Wien}}\end{otherlanguage}\pend{}\pstart{}\textcolor{pink}{XVIII. Sternwartestrasse 71}{}\ledrightnote{\textcolor{pink}{Sternwartestraße 71}}\pend{}
{\bigskip}
\pstart
           \noindent{}{\pb}\textcolor{gray}{\textbf{\textsc{No. 36 – \textcolor{pink}{Karnak. Ptolomey Gateways and the Temple of Khonsu, God
                           of the Moon}{}\ledrightnote{\textcolor{pink}{Karnak Tempelanlage}}}.}}\pend
           
\pstart
           \raggedleft{}{\pb}\textcolor{pink}{Assuan}{}\ledrightnote{\textcolor{pink}{Assuan}}, 1. III\pend
           
\pstart
           Es ist halt doch sehr schön, schon um diese Zeit 30° Hitze zu haben – aber leben
               möchte man hier trotzdem nicht. Herzliche Grüße, auch an \textcolor{blue}{Heini}{}\ledrightnote{\textcolor{blue}{Heinrich Schnitzler}} und \textcolor{blue}{Lilli}{}\ledrightnote{\textcolor{blue}{Lilly Schnitzler}}.
               {\\}Ihr {\\}\spacefill\mbox{Felix Salten}\pend
           \endnumbering\briefempfaengerindex{Schnitzler, Arthur@\textsc{Schnitzler, Arthur}!zzzSalten, Felix@\emph{von Felix Salten}!1924-03-011@{1. 3. {[}1924{]}}|)be}\mylabel{h}  \normalsize

\doendnotes{C}
\bigskip
\vfill

\clearpage

\footnotesize

\lohead{\textsc{register}}

% Definiere theindex-Environment komplett neu ohne reledmac
\makeatletter
\renewenvironment{theindex}{%
  \section*{\indexname}%
  \setlength{\parindent}{0pt}%
  \setlength{\parskip}{0pt plus 0.3pt}%
  \let\item\@idxitem
}{%
  \clearpage
}
\makeatother

\IfFileExists{\jobname-pw.ind}{\input{\jobname-pw.ind}}{}

\end{document}

      