%% latex-korrekturansicht-vorspann.tex
%% Vorspann für die Korrekturansicht.
%% Lädt die gemeinsame Datei latex-vorspann.tex mit gesetztem Schalter.

\newif\ifkorrekturansicht
\korrekturansichttrue

\input{../tex-inputs/latex-vorspann}


\section[Stefan Zweig an Arthur Schnitzler, 6. 11. 1929]{L03692 Stefan Zweig an Arthur Schnitzler, 6. 11. 1929}
\nopagebreak\mylabel{L03692v}
\rehead{ }\normalsize\beginnumbering\briefempfaengerindex{Schnitzler, Arthur@\textsc{Schnitzler, Arthur}!zzzZweig, Stefan@\emph{von Stefan Zweig}!1929-11-062@{6. 11. 1929}|(be}
\toendnotes[C]{\smallbreak\pagebreak[2]}
\correspDesc{Versand  durch Stefan Zweig am 6. 11. 1929 in Salzburg
\newline{}Erhalt  durch Arthur Schnitzler im Zeitraum [7. 11. 1929
                  – 11. 11. 1929?] in Wien}\toendnotes[C]{\smallbreak}
\Standort{CUL, Schnitzler, B 118.}
\physDesc{Brief, 1 Blatt, 1 Seite, 1207 Zeichen
\newline{}Schreibmaschine
\newline{}Handschrift: blauer Buntstift, lateinische Kurrent (\noindent{}Unterschrift)
\newline{}Schnitzler: 1) mit rotem Buntstift zehn Unterstreichungen  2) mit rotem Buntstift eine seitliche Anstreichung}
\buchAbdrucke{\weitereDrucke{Stefan Zweig: \emph{Briefwechsel mit Hermann Bahr, Sigmund Freud, Rainer Maria
                        Rilke und Arthur Schnitzler}. Herausgegeben von Jeffrey B. Berlin, Hans-Ulrich Lindken und Donald A. Prater. Frankfurt am Main: \emph{S. Fischer} 1987, S. 447–448.} }\toendnotes[C]{\smallbreak}
\pstart
           {\pb}\textcolor{gray}{\textbf{SZ}}\hfill \textcolor{gray}{\textbf{\textcolor{pink}{SALZBURG}\oindex{Salzburg@\textbf{Salzburg}, \emph{Verwaltungsgebiet}|pw}{}\ledrightnote{\textcolor{pink}{Salzburg}}}}\pend
           
\pstart
           \raggedleft{}\textcolor{gray}{\textbf{\textcolor{pink}{KAPUZINERBERG 5}\oindex{Paschinger Schlössl@\textbf{Paschinger Schlössl}, \emph{Wohngebäude}|pw}{}\ledrightnote{\textcolor{pink}{Paschinger Schlössl}}}}\pend
           
\pstart
           \raggedleft{}6. November 1929.\pend
           
\pstart\center{}Lieber, verehrter Herr Doktor!\pend\vspace{0.5em}
\pstart
           Ueber meine Vereinbarungen mit \textcolor{pink}{Spanien}\oindex{Spanien@\textbf{Spanien}|pw}{}\ledrightnote{\textcolor{pink}{Spanien}} kann ich
               Sie genau informieren: ich habe meinen »\textcolor{green}{Fouché}\pwindex{Zweig, Stefan 28.\,11.\,1881 Wien – 23.\,2.\,1942 Petrópolis@\textsc{Zweig, Stefan} (28.\,11.\,1881 Wien – 23.\,2.\,1942 Petrópolis), \emph{Schriftsteller}!Fouché. Retrato di un Político@\strich\emph{Fouché. Retrato di un Político}|pw}{}\ledrightnote{\textcolor{green}{Fouché. Retrato di un Político}}«
               an \textcolor{blue}{A. del Vayos}\pwindex{Álvarez del Vayo, Julio 9.\,2.\,1891 Villaviciosa de Odón – 3.\,5.\,1975 Genf@\textsc{Álvarez del Vayo, Julio} (9.\,2.\,1891 Villaviciosa de Odón – 3.\,5.\,1975 Genf), \emph{Schriftsteller, Politiker, Journalist}|pwu}{}\ledrightnote{\textcolor{blue}{Julio Álvarez del Vayo}}{ }\textcolor{brown}{Verlag}\orgindex{Espasa-Calpe@Espasa-Calpe|pwuv}{}\ledrightnote{{$\rightarrow$}\emph{\textcolor{brown}{Espasa-Calpe}}} zu 7 {\%} vergeben mit einem \label{K_L03692-1v}\edtext{\begin{otherlanguage}{french}à valoir\end{otherlanguage}}{\lemma{\textnormal{\emph{à valoir}}}\Cendnote{\textnormal{französisch: Vorschuss}}}\label{K_L03692-1} von 1000
                  \textcolor{pink}{frz}\oindex{Frankreich@\textbf{Frankreich}|pw}{}\ledrightnote{\textcolor{pink}{Frankreich}}. Frs., die sie sofort ausbezahlten, und
               Sie werden sicherlich zumindest dieselben Bedingungen bekommen. \pend
           
\pstart
           Dass man in \textcolor{pink}{Paris}\oindex{Paris@\textbf{Paris}, \emph{Hauptstadt}|pw}{}\ledrightnote{\textcolor{pink}{Paris}}{ }\label{K_L03692-2v}\edtext{im Kino eine \textcolor{green}{Novelle}\pwindex{Steinhoff, Hans 10.\,3.\,1882 Marienberg – 20.\,4.\,1945 Glienig@\textsc{Steinhoff, Hans} (10.\,3.\,1882 Marienberg – 20.\,4.\,1945 Glienig)!Angst@\strich\emph{Angst}|pwuv}{}\ledrightnote{{$\rightarrow$}\emph{\textcolor{green}{Angst}}}}{\lemma{\textnormal{\emph{im Kino eine Novelle}}}\Cendnote{\textnormal{1928 wurde \textcolor{blue}{Zweigs}\pwindex{Zweig, Stefan 28.\,11.\,1881 Wien – 23.\,2.\,1942 Petrópolis@\textsc{Zweig, Stefan} (28.\,11.\,1881 Wien – 23.\,2.\,1942 Petrópolis), \emph{Schriftsteller}|pwk} Novelle \emph{\textcolor{green}{Angst}\pwindex{Zweig, Stefan 28.\,11.\,1881 Wien – 23.\,2.\,1942 Petrópolis@\textsc{Zweig, Stefan} (28.\,11.\,1881 Wien – 23.\,2.\,1942 Petrópolis), \emph{Schriftsteller}!Angst@\strich\emph{Angst}|pwk}} verfilmt. Vermutlich ist dieser \emph{\textcolor{green}{Film}\pwindex{Steinhoff, Hans 10.\,3.\,1882 Marienberg – 20.\,4.\,1945 Glienig@\textsc{Steinhoff, Hans} (10.\,3.\,1882 Marienberg – 20.\,4.\,1945 Glienig)!Angst@\strich\emph{Angst}|pwk}} gemeint, vgl. Arthur Schnitzler an Stefan Zweig, 4. 11. 1929. 1929 erschien außerdem die
                  Verfilmung von \emph{\textcolor{green}{Brief einer Unbekannten}\pwindex{Zweig, Stefan 28.\,11.\,1881 Wien – 23.\,2.\,1942 Petrópolis@\textsc{Zweig, Stefan} (28.\,11.\,1881 Wien – 23.\,2.\,1942 Petrópolis), \emph{Schriftsteller}!Brief einer Unbekannten@\strich\emph{Brief einer Unbekannten}|pwk}} unter
                  dem Titel \emph{\textcolor{green}{Narkose}\pwindex{Zweig, Stefan 28.\,11.\,1881 Wien – 23.\,2.\,1942 Petrópolis@\textsc{Zweig, Stefan} (28.\,11.\,1881 Wien – 23.\,2.\,1942 Petrópolis), \emph{Schriftsteller}!Narkose@\strich\emph{Narkose}|pwk}\pwindex{Abel, Alfred 12.\,3.\,1879 Leipzig – 12.\,12.\,1937 Berlin@\textsc{Abel, Alfred} (12.\,3.\,1879 Leipzig – 12.\,12.\,1937 Berlin), \emph{Schauspieler}!Narkose@\strich\emph{Narkose}|pwk}\pwindex{Héribel, Renée @\textsc{Héribel, Renée}, \emph{Schauspieler/Schauspielerin}!Narkose@\strich\emph{Narkose}|pwk}}.}}}\label{K_L03692-2} von mir Ihnen
               zugeschrieben hat, betrachte ich als eine hohe Ehre. Die Leute werfen dort alles auf
               das rührendste durcheinander. Uebrigens ist »\textcolor{green}{Fräulein Else}\pwindex{Schnitzler, Arthur 15. 5. 1862 Wien – 21. 10. 1931 ebd.@\textsc{Schnitzler, Arthur} (15. 5. 1862 Wien – 21. 10. 1931 ebd.), \emph{Schriftsteller, Mediziner}!Madmoiselle Else@\strich\emph{Madmoiselle Else}|pw}{}\ledrightnote{\textcolor{green}{Madmoiselle Else}}« dort ein grosser Erfolg, \textcolor{brown}{Stock}\orgindex{Éditions Stock@Éditions Stock|pw}{}\ledrightnote{\textcolor{brown}{Éditions Stock}} bringt, wie ich höre, eine neue Auflage und erwartet sich sehr viel,
               wenn der \textcolor{green}{Film}\pwindex{Czinner, Paul 30.\,5.\,1890 Budapest – 22.\,6.\,1972 London@\textsc{Czinner, Paul} (30.\,5.\,1890 Budapest – 22.\,6.\,1972 London), \emph{Schriftsteller, Filmregisseur}!Fräulein Else@\strich\emph{Fräulein Else}|pw}{}\ledrightnote{\textcolor{green}{Fräulein Else}} abrollt. Wichtig ist nur, einmal
               in \textcolor{pink}{Paris}\oindex{Paris@\textbf{Paris}, \emph{Hauptstadt}|pw}{}\ledrightnote{\textcolor{pink}{Paris}} ein Theaterstück durchzusetzen. Man ist
               jetzt in \textcolor{pink}{Frankreich}\oindex{Frankreich@\textbf{Frankreich}|pw}{}\ledrightnote{\textcolor{pink}{Frankreich}} dem Ausländer viel offener
               als vordem und, während \textcolor{pink}{Oesterreich}\oindex{Österreich-Ungarn@\textbf{Österreich-Ungarn}|pw}{}\ledrightnote{\textcolor{pink}{Österreich-Ungarn}} herrlich in
               die Alpenländerei hineinmarschiert, beginnt dort der \textcolor{pink}{europäische}\oindex{Europa@\textbf{Europa}|pw}{}\ledrightnote{\textcolor{pink}{Europa}} Gedanke immer selbstverständlicher zu werden. Ich habe mich in
                  \textcolor{pink}{Paris}\oindex{Paris@\textbf{Paris}, \emph{Hauptstadt}|pw}{}\ledrightnote{\textcolor{pink}{Paris}} ungemein wohl gefühlt und wundere mich
               eigentlich, dass Sie sich niemals entschlossen haben, einmal dort einen Wintermonat
               zu verbringen. Viele Freunde Ihrer Bücher erwarten Sie und besonders \textcolor{blue}{Frédéric Lefèvre}\pwindex{Lefèvre, Frédéric 7.\,5.\,1889 Izé – 11.\,9.\,1949 Paris@\textsc{Lefèvre, Frédéric} (7.\,5.\,1889 Izé – 11.\,9.\,1949 Paris), \emph{Schriftsteller, Journalist, Literaturkritiker}|pw}{}\ledrightnote{\textcolor{blue}{Frédéric Lefèvre}} mit seinen \label{K_L03692-3v}\edtext{»heures avec {\dotsfour}«}{\lemma{\textnormal{\emph{»heures avec «}}}\Cendnote{\textnormal{Der Literaturkritiker \textcolor{blue}{Frédéric Lefèvre}\pwindex{Lefèvre, Frédéric 7.\,5.\,1889 Izé – 11.\,9.\,1949 Paris@\textsc{Lefèvre, Frédéric} (7.\,5.\,1889 Izé – 11.\,9.\,1949 Paris), \emph{Schriftsteller, Journalist, Literaturkritiker}|pwk} begründete
                     1922 in der Zeitschrift \emph{\textcolor{green}{Les
                     nouvelles littéraires}\pwindex{Nouvelles littéraires@\emph{Les Nouvelles littéraires}|pwk}} mit der Serie »Une Heure avec {\dots}« ein neuartiges literaturkritisches Interviewformat, das er bis
                     1938 fortsetzte. Erst 1932{ }\textcolor{green}{befragte}\pwindex{Lefèvre, Frédéric 7.\,5.\,1889 Izé – 11.\,9.\,1949 Paris@\textsc{Lefèvre, Frédéric} (7.\,5.\,1889 Izé – 11.\,9.\,1949 Paris), \emph{Schriftsteller, Journalist, Literaturkritiker}!Une heure avec Stefan Zweig. Le rôle de l’intellectuel dans la crise actuelle@\strich\emph{Une heure avec Stefan Zweig. Le rôle de l’intellectuel dans la crise actuelle}|pwkv}
                  er \textcolor{blue}{Zweig}\pwindex{Zweig, Stefan 28.\,11.\,1881 Wien – 23.\,2.\,1942 Petrópolis@\textsc{Zweig, Stefan} (28.\,11.\,1881 Wien – 23.\,2.\,1942 Petrópolis), \emph{Schriftsteller}|pwk}.}}}\label{K_L03692-3}\pend
           
\pstart
           Getreulichst{\\[\baselineskip]} Ihr {\\[\baselineskip]}\spacefill\mbox{{[}hs.:{]} Stefan Zweig}\pend
           \leftskip=0em{}
\pstart
           \noindent{}{[}ms.:{]} Herrn Dr. Arthur Schnitzler\pend
           
\pstart
           \uline{\so{Wien}}\pend
           \selectlanguage{ngerman}\endnumbering\briefempfaengerindex{Schnitzler, Arthur@\textsc{Schnitzler, Arthur}!zzzZweig, Stefan@\emph{von Stefan Zweig}!1929-11-062@{6. 11. 1929}|)be}\mylabel{L03692h}  \normalsize

\doendnotes{C}
\bigskip
\vfill

\clearpage

\footnotesize

\lohead{\textsc{register}}

% Definiere theindex-Environment komplett neu ohne reledmac
\makeatletter
\renewenvironment{theindex}{%
  \section*{\indexname}%
  \setlength{\parindent}{0pt}%
  \setlength{\parskip}{0pt plus 0.3pt}%
  \let\item\@idxitem
}{%
  \clearpage
}
\makeatother

\IfFileExists{\jobname-pw.ind}{\input{\jobname-pw.ind}}{}

\end{document}

      