%% latex-korrekturansicht-vorspann.tex
%% Vorspann für die Korrekturansicht.
%% Lädt die gemeinsame Datei latex-vorspann.tex mit gesetztem Schalter.

\newif\ifkorrekturansicht
\korrekturansichttrue

\input{../tex-inputs/latex-vorspann}


\renewcommand{\erwaehntePersonen}{Personen: Charlotte Pohl-Glas}
\renewcommand{\erwaehnteOrte}{Orte: Wien}
\renewcommand{\erwaehnteWerke}{Werke: Tagebuch}
\section[ Felix Salten an Arthur Schnitzler, {[}14?. 1. 1895{]}]{Felix Salten an Arthur Schnitzler, {[}14?. 1. 1895{]}}
\nopagebreak\mylabel{v}
\rehead{ }\normalsize\beginnumbering\briefempfaengerindex{Schnitzler, Arthur@\textsc{Schnitzler, Arthur}!zzzSalten, Felix@\emph{von Felix Salten}!1895-01-141@{{[}14?. 1. 1895{]}}|(be}
\toendnotes[C]{\smallbreak\pagebreak[2]}\Standort{CUL, Schnitzler, B 89, A 1.}
\physDesc{Brief, 1 Blatt, 2 Seiten, 432 Zeichen
\newline{}Handschrift: Bleistift, lateinische Kurrent
\newline{}Schnitzler: mit Bleistift datiert: »1\substVorne{}\textsuperscript{2}\substDazwischen{}3\substHinten{}/1 95« 
\newline{}Ordnung: mit Bleistift von unbekannter Hand nummeriert: »49« }\toendnotes[C]{\smallbreak}
\pstart
           \noindent{}{\pb}Lieber Freund,{ }\label{K_L03148-1v}\edtext{\textcolor{blue}{Lotte}{}\ledrightnote{\textcolor{blue}{Charlotte Pohl-Glas}} geht morgen in Haft}{\lemma{\textnormal{\emph{Lotte … Haft}}}\Cendnote{\textnormal{siehe Felix Salten an Arthur Schnitzler, 7. 8. 1894}}}\label{K_L03148-1h} und ich habe heute für sie einiges zu kaufen.
               Sie schreibt mir eben um Geld, und bittet mich, da ihre Leute nichts für sie thun
               wollen. Nun ist erst \label{K_L03148-2v}\edtext{morgen{ }der 15\textsuperscript{te}}{\lemma{\textnormal{\emph{morgen der 15\textsuperscript{te}}}}\Cendnote{\textnormal{\textcolor{blue}{Schnitzler} datierte den Brief auf den 13. 1. 1895, doch nahm er dabei eine Überschreibung
                  vor und änderte den 12. ab, den er zuerst
                  geschrieben haben dürfte. Diese Unsicherheit und \textcolor{blue}{Salten}s Aussage, dass »morgen{ }der 15\textsuperscript{te}« sei, sind Gründe für die Datierung des Briefes auf den 14. In jedem Fall dürfte \textcolor{blue}{Schnitzler} am 14. 1. 1895 den Brief erhalten haben, da eine
                  Aussage zu diesem Tag im \emph{\textcolor{green}{Tagebuch}} dadurch motiviert scheint: »\textcolor{blue}{Salten}s \textcolor{blue}{Gel.} wird morgen (wegen social. Geschichten) eingesperrt. Der
                     Glückliche.«}}}\label{K_L03148-2h}, und ich bitte Sie deswegen \uline{recht sehr}, mir \uline{bis morgen} mit fl. 10.– zu helfen. Ich erhalte \uline{morgen{ }3 Uhr Gage}, und gebe Ihnen {\pb}\uline{mein Wort}, dass ich Ihnen das Geld \uline{morgen{ }Nachmittag} sofort hinüberbringe.\pend
           
\pstart
           Besten Dank im Voraus. Herzlichst Ihr {\\[\baselineskip]}\spacefill\mbox{Salten}\pend
           \leftskip=0em{}\endnumbering\briefempfaengerindex{Schnitzler, Arthur@\textsc{Schnitzler, Arthur}!zzzSalten, Felix@\emph{von Felix Salten}!1895-01-141@{{[}14?. 1. 1895{]}}|)be}\mylabel{h}  \normalsize

\doendnotes{C}
\bigskip
\vfill

\clearpage

\footnotesize

\lohead{\textsc{register}}

% Definiere theindex-Environment komplett neu ohne reledmac
\makeatletter
\renewenvironment{theindex}{%
  \section*{\indexname}%
  \setlength{\parindent}{0pt}%
  \setlength{\parskip}{0pt plus 0.3pt}%
  \let\item\@idxitem
}{%
  \clearpage
}
\makeatother

\IfFileExists{\jobname-pw.ind}{\input{\jobname-pw.ind}}{}

\end{document}

      