%% latex-korrekturansicht-vorspann.tex
%% Vorspann für die Korrekturansicht.
%% Lädt die gemeinsame Datei latex-vorspann.tex mit gesetztem Schalter.

\newif\ifkorrekturansicht
\korrekturansichttrue

\input{../tex-inputs/latex-vorspann}


\section[Romain Rolland an Arthur Schnitzler, 6. 2. 1916]{L03883 Romain Rolland an Arthur Schnitzler, 6. 2. 1916}
\nopagebreak\mylabel{L03883v}
\rehead{ }\normalsize\beginnumbering\briefempfaengerindex{, @\textsc{, }!zzz, @\emph{von  }!1916-02-061@{6. 2. 1916}|(be}
\toendnotes[C]{\smallbreak\pagebreak[2]}\Standort{CUL, Schnitzler, B 86.}
\physDesc{Brief, 1 Blatt, 2 Seiten, 627 Zeichen
\newline{}Handschrift: schwarze Tinte, lateinische Kurrent}\toendnotes[C]{\smallbreak}
\pstart
           \raggedleft{}{\pb}\begin{otherlanguage}{french}\label{K_L03883-1v}\edtext{Dimanche 6 fév. 1916}{\lemma{\textnormal{\emph{Dimanche 6 fév. 1916}}}\Cendnote{\textnormal{französisch: »Sonntag, 6. Februar 1916{ / }Sehr geehrter Herr,{ / }ich danke Ihnen herzlich für die Freundlichkeit, mir dieses Telegramm
                              zu senden (das mich in \textcolor{pink}{Genf}\oindex{Genf@\textbf{Genf}|pw} erst
                              heute, am 6. Februar, erreicht hat). Es rührt mich sehr, dass man
                              trotz so vieler Sorgen an meinen armen Geburtstag denken konnte; und
                              ich empfinde eine besondere Dankbarkeit gegenüber der kleinen Gruppe
                              von Wiener Freunden, die mir ihre Sympathie bewahren.{ / }Man spürt, dass es in der Welt so viele zurückgedrängte Freundschaften
                              gibt, die morgen zutage treten werden! Und es ist unsere Aufgabe, wir,
                              die wir die \strikeout{vorangehen} Vorboten unserer Völker sind, schon heute das Morgen zu
                              leben.{ / }Empfangen Sie, sehr geehrter Herr, den Ausdruck meiner ganzen
                              Ergebenheit{ / }Romain Rolland« }}}\label{K_L03883-1}\end{otherlanguage}\pend
           
\pstart{}\begin{otherlanguage}{french}Cher Monsieur\end{otherlanguage}\pend\vspace{0.5em}
\pstart
           \begin{otherlanguage}{french}Je vous remercie cordialement de la bonté que vous avez eue de
                  m’envoyer cette \label{K_L03883-2v}\edtext{dépêche}{\lemma{\textnormal{\emph{dépêche}}}\Cendnote{\textnormal{Am
                     29. 1. 1866 war \textcolor{blue}{Rolland}\pwindex{Rolland, Romain 29.\,1.\,1866 Clamecy – 30.\,12.\,1944 Vézelay@\textsc{Rolland, Romain} (29.\,1.\,1866 Clamecy – 30.\,12.\,1944 Vézelay), \emph{Schriftsteller}|pwk} 50 Jahre alt geworden.}}}\label{K_L03883-2} (qui m’est arrivée à \textcolor{pink}{Genève}\oindex{Genf@\textbf{Genf}|pw}{}\ledrightnote{\textcolor{pink}{Genf}} qu’aujourd’hui, 6 février). Je suis bien touché que
                  l’on puisse penser à mon pauvre anniversaire, au milieu de tant de soucis; et j’ai
                  une particulière gratitude au petit groupe d’amis \textcolor{pink}{viennois}\oindex{Wien@\textbf{Wien}, \emph{Verwaltungsgebiet}|pw}{}\ledrightnote{\textcolor{pink}{Wien}}, qui me gardent leur sympathie.\end{otherlanguage}\pend
           
\pstart
           \begin{otherlanguage}{french}On sent qu’il y a dans {\pb}le monde tant d’amitiés comprimées, qui
                  se feront jour, demain ! Et c’est notre rôle à nous, qui \strikeout{devançons} sommes les fourriers de nos peuples, de vivre déjà
                  demain.\end{otherlanguage}\pend
           
\pstart
           \begin{otherlanguage}{french}Veuillez croire, cher Monsieur, à mes sentiments tout
                  dévoués\end{otherlanguage}{\\[\baselineskip]}\spacefill\mbox{Romain Rolland}\pend
           \leftskip=0em{}\selectlanguage{ngerman}\endnumbering\briefempfaengerindex{, @\textsc{, }!zzz, @\emph{von  }!1916-02-061@{6. 2. 1916}|)be}\mylabel{L03883h}
\begin{anhang}
\end{anhang}\normalsize

\doendnotes{C}
\bigskip
\vfill

\clearpage

\footnotesize

\lohead{\textsc{register}}

% Definiere theindex-Environment komplett neu ohne reledmac
\makeatletter
\renewenvironment{theindex}{%
  \section*{\indexname}%
  \setlength{\parindent}{0pt}%
  \setlength{\parskip}{0pt plus 0.3pt}%
  \let\item\@idxitem
}{%
  \clearpage
}
\makeatother

\IfFileExists{\jobname-pw.ind}{\input{\jobname-pw.ind}}{}

\end{document}

      