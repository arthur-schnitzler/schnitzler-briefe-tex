%% latex-korrekturansicht-vorspann.tex
%% Vorspann für die Korrekturansicht.
%% Lädt die gemeinsame Datei latex-vorspann.tex mit gesetztem Schalter.

\newif\ifkorrekturansicht
\korrekturansichttrue

\input{../tex-inputs/latex-vorspann}


\renewcommand{\erwaehntePersonen}{Personen: Hermann Bahr, Heinrich Kanner, Anna Katharina Rehmann, Paul Salten, Olga Schnitzler, Isidor Singer}
\renewcommand{\erwaehnteInstitutionen}{Institutionen: Die Zeit}
\renewcommand{\erwaehnteOrte}{Orte: Eisenerz, Mariazell, Reichenau an der Rax, Wien, Wipplingerstraße}
\renewcommand{\erwaehnteWerke}{Werke: Die Andere}
\section[ Felix Salten an Arthur Schnitzler, 18. 7. 1905]{Felix Salten an Arthur Schnitzler, 18. 7. 1905}
\nopagebreak\mylabel{v}
\rehead{ }\normalsize\beginnumbering\briefempfaengerindex{Schnitzler, Arthur@\textsc{Schnitzler, Arthur}!zzzSalten, Felix@\emph{von Felix Salten}!1905-07-181@{18. 7. 1905}|(be}
\toendnotes[C]{\smallbreak\pagebreak[2]}\Standort{CUL, Schnitzler, B 89, B 1.}
\physDesc{Briefkarte, 553 Zeichen
\newline{}Handschrift: schwarze Tinte, lateinische Kurrent
\newline{}Ordnung: mit Bleistift von unbekannter Hand nummeriert: »204« }
\buchAbdrucke{\weitereDrucke{Hermann Bahr, Arthur Schnitzler: \emph{Briefwechsel, Aufzeichnungen, Dokumente (1891–1931)}. Hg. Kurt Ifkovits und Martin Anton Müller. Göttingen: \emph{Wallstein} 2018, S. 346–347.} }\toendnotes[C]{\smallbreak}
\pstart
           \noindent{}{\pb}\textcolor{gray}{\textbf{DIE}}\pend
           
\pstart
           \textcolor{gray}{\textbf{\textcolor{brown}{ZEIT}{}\ledrightnote{\textcolor{brown}{Die Zeit}}}}\hfill \textcolor{gray}{\textbf{\textcolor{pink}{\emph{WIEN}}{}\ledrightnote{\textcolor{pink}{Wien}}}}{ }18. 7. 05\pend
           
\pstart
           \textcolor{gray}{\textbf{\textcolor{pink}{Wien}{}\ledrightnote{\textcolor{pink}{Wien}}er Tageszeitung}}\hfill \textcolor{gray}{\textbf{\emph{\textcolor{pink}{I. Wipplingerstrasse 38}{}\ledrightnote{\textcolor{pink}{Wipplingerstraße}}}}}\pend
           
\pstart
           \textcolor{gray}{\textbf{Herausgeber:}}\pend
           
\pstart
           \textcolor{gray}{\textbf{\textbf{Prof. Dr. \textcolor{blue}{I. Singer}{}\ledrightnote{\textcolor{blue}{Isidor Singer}}}}}\pend
           
\pstart
           \textcolor{gray}{\textbf{\textbf{Dr. \textcolor{blue}{Heinrich Kanner}{}\ledrightnote{\textcolor{blue}{Heinrich Kanner}}}}}\pend
           
\pstart
           \textcolor{gray}{\textbf{\textbf{Feuilleton-Redaktion}}}\pend
           
\pstart
           Lieber, bis jetzt waren die \textcolor{blue}{Kinder}{}\ledrightnote{\textcolor{blue}{Paul Salten}{\newline}\textcolor{blue}{Anna Katharina Rehmann}} krank und \textcolor{blue}{Paul}{}\ledrightnote{\textcolor{blue}{Paul Salten}} hat uns wieder viele Sorgen gemacht. Deshalb sind wir nicht abgeko{\geminationm}en. Schreiben Sie mir, ob es Ihnen passt, wenn wir
                  \label{K_L03412-1v}\edtext{Samstag nach \textcolor{pink}{Reichenau}{}\ledrightnote{\textcolor{pink}{Reichenau an der Rax}}}{\lemma{\textnormal{\emph{Samstag nach Reichenau}}}\Cendnote{\textnormal{nicht geschehen}}}\label{K_L03412-1h} kommen, und ob
               Sie dann Lust haben (nur für diesen Fall kämen wir) am Sonntag oder Montag die \label{K_L03412-2v}\edtext{\textcolor{pink}{Maria Zell}{}\ledrightnote{\textcolor{pink}{Mariazell}}er Partie}{\lemma{\textnormal{\emph{Maria Zeller Partie}}}\Cendnote{\textnormal{Diese fand erst am Monatsende und ohne \textcolor{blue}{Schnitzler} statt, vgl. Felix Salten und Richard Metzl an Arthur
               Schnitzler, [30. 7. 1905?].}}}\label{K_L03412-2h} mitzumachen. Ich habe auch \textcolor{pink}{Eisenerz}{}\ledrightnote{\textcolor{pink}{Eisenerz}} u. s. w. vor, worüber wir aber noch
               sprechen könnten. Ich denke mir: Samstag Tennis, Sonntag Tennis. Montag{ }früh od. Sonntag{ }Abds. Abfahrt nach \textcolor{pink}{Mzll.}{}\ledrightnote{\textcolor{pink}{Mariazell}}\pend
           
\pstart
           herzliche Grüße von uns an Sie \textcolor{blue}{Beide}{}\ledrightnote{{$\rightarrow$}\textcolor{blue}{Olga Schnitzler}}{ }{\\[\baselineskip]}Ihr {\\[\baselineskip]}\spacefill\mbox{Salten}\pend
           \leftskip=0em{}
\pstart
           \noindent{}Das \label{K_L03412-3v}\edtext{\textcolor{green}{Stück}{}\ledrightnote{{$\rightarrow$}\textcolor{green}{Die Andere}} von \textcolor{blue}{Bahr}{}\ledrightnote{\textcolor{blue}{Hermann Bahr}}}{\lemma{\textnormal{\emph{Stück von Bahr}}}\Cendnote{\textnormal{\textcolor{green}{Die Andere}, siehe Arthur Schnitzler an Hermann Bahr, 30. 7. 1905 und A. S.: \emph{Tagebuch}, 26. 7. 1905}}}\label{K_L03412-3h} haben Sie erhalten?\pend
           \endnumbering\briefempfaengerindex{Schnitzler, Arthur@\textsc{Schnitzler, Arthur}!zzzSalten, Felix@\emph{von Felix Salten}!1905-07-181@{18. 7. 1905}|)be}\mylabel{h}  \normalsize

\doendnotes{C}
\bigskip
\vfill

\clearpage

\footnotesize

\lohead{\textsc{register}}

% Definiere theindex-Environment komplett neu ohne reledmac
\makeatletter
\renewenvironment{theindex}{%
  \section*{\indexname}%
  \setlength{\parindent}{0pt}%
  \setlength{\parskip}{0pt plus 0.3pt}%
  \let\item\@idxitem
}{%
  \clearpage
}
\makeatother

\IfFileExists{\jobname-pw.ind}{\input{\jobname-pw.ind}}{}

\end{document}

      