%% latex-korrekturansicht-vorspann.tex
%% Vorspann für die Korrekturansicht.
%% Lädt die gemeinsame Datei latex-vorspann.tex mit gesetztem Schalter.

\newif\ifkorrekturansicht
\korrekturansichttrue

\input{../tex-inputs/latex-vorspann}


               \section[Paul Goldmann an Arthur Schnitzler, {[}25.–28.? 2. 1897{]}]{ Paul Goldmann an Arthur Schnitzler, {[}25.–28.? 2. 1897{]}}\nopagebreak\mylabel{v}\rehead{ }\normalsize\beginnumbering\briefempfaengerindex{Schnitzler, Arthur@\textsc{Schnitzler, Arthur}!zzzGoldmann, Paul@\emph{von Paul Goldmann}!1897-02-251@{{[}25.–28.? 2. 1897{]}}|(be} \toendnotes[C]{\smallbreak\pagebreak[2]} \Standort{DLA, A:Schnitzler, HS.NZ85.1.3167.}
\physDesc{Telegramm
\newline{}maschinell
\newline{}Schnitzler: mit Bleistift datiert auf
                                 »Febr 97« \newline{}Ordnung: beschnitten }\toendnotes[C]{\smallbreak}\pstart
           \noindent{}{\pb}\label{K_L02638-5v}\edtext{hier befindlicher \textcolor{blue}{bruder}{}\ledrightnote{→\textcolor{blue}{Richard Klein}}{ }\textcolor{blue}{arthur klein}{}\ledrightnote{\textcolor{blue}{Arthur Klein}}s}{\lemma{\textnormal{\emph{hier … kleins}}}\Cendnote{\textnormal{Eine näherungsweise Datierung des Telegramms geht durch
               Berücksichtigung einerseits der handschriftlichen Datierung \textcolor{blue}{Schnitzler}s, andererseits dass
                  im Brief {XXXX ref} noch nicht, im {XXXX ref} bereits
               von der erledigten Sache die Rede ist.}}}\label{K_L02638-5h} wird mir unerhoert laestig nachdem ich auf seine
               aufforderung mein \label{T_L02638-3v}\edtext{urtheil}{\lemma{\textnormal{\emph{urtheil}}}\Cendnote{\textnormal{im Text steht:
                  »urtheit«}}}\label{T_L02638-3h} ueber sein bild
               abgegeben schrieb er mir unverschaemten brief, ich antwortete dass ich mit unreifen
               burschen nicht \label{T_L02638-2v}\edtext{discutire}{\lemma{\textnormal{\emph{discutire}}}\Cendnote{\textnormal{im Text steht:
                  »discutive«}}}\label{T_L02638-2h} und sandte brief an \textcolor{blue}{arthur klein}{}\ledrightnote{\textcolor{blue}{Arthur Klein}}{ }heut{ }\label{T_L02638-1v}\edtext{erhielt}{\lemma{\textnormal{\emph{erhielt}}}\Cendnote{\textnormal{im Text steht: »orhielt«}}}\label{T_L02638-1h} ich
               herausforderung deren annahme ich natuerlich ablehnte \label{T_L02638-4v}\edtext{der}{\lemma{\textnormal{\emph{der}}}\Cendnote{\textnormal{im Text steht:
                     »ber«}}}\label{T_L02638-4h} bursch stoesst jetzt drohungen gegen mich aus
               kannst du ihn mir nicht vom halse schaffen\pend
           \pstart gruss \spacefill\mbox{paul goldmann =}\pend{}\endnumbering\briefempfaengerindex{Schnitzler, Arthur@\textsc{Schnitzler, Arthur}!zzzGoldmann, Paul@\emph{von Paul Goldmann}!1897-02-251@{{[}25.–28.? 2. 1897{]}}|)be}\mylabel{h}  \normalsize

\doendnotes{C}
\bigskip
\vfill

\clearpage

\footnotesize

\lohead{\textsc{register}}

% Definiere theindex-Environment komplett neu ohne reledmac
\makeatletter
\renewenvironment{theindex}{%
  \section*{\indexname}%
  \setlength{\parindent}{0pt}%
  \setlength{\parskip}{0pt plus 0.3pt}%
  \let\item\@idxitem
}{%
  \clearpage
}
\makeatother

\IfFileExists{\jobname-pw.ind}{\input{\jobname-pw.ind}}{}

\end{document}

      