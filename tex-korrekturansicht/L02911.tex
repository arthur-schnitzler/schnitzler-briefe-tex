%% latex-korrekturansicht-vorspann.tex
%% Vorspann für die Korrekturansicht.
%% Lädt die gemeinsame Datei latex-vorspann.tex mit gesetztem Schalter.

\newif\ifkorrekturansicht
\korrekturansichttrue

\input{../tex-inputs/latex-vorspann}


         
         \renewcommand{\erwaehntePersonen}{Personen: Richard Beer-Hofmann, Auguste Chlum, Gustave Flaubert, Marie Glümer, Johann Wolfgang von Goethe, Clementine Goldmann, Alfred Kerr, Ludwig Speidel, Anna Wendt}
         \renewcommand{\erwaehnteInstitutionen}{Institutionen: Charpentier, Neue Freie Presse}
         \renewcommand{\erwaehnteOrte}{Orte: Berlin, Dessauer Straße, Frankreich, Paris, Wien}
         \renewcommand{\erwaehnteWerke}{Werke: Correspondance, Correspondance. 4 Bde., Der Graf von Charolais. Ein Trauerspiel, Der Schleier der Beatrice. Schauspiel in fünf Akten, Frau Bertha Garlan. Roman, Goethes Unterhaltungen mit dem Kanzler Friedrich von Müller, Tagebuch}
               \section[ Paul Goldmann an Arthur Schnitzler, 18. 4. {[}1900{]}]{Paul Goldmann an Arthur Schnitzler, 18. 4. {[}1900{]}}\nopagebreak\mylabel{v}\rehead{ }\normalsize\beginnumbering\briefempfaengerindex{Schnitzler, Arthur@\textsc{Schnitzler, Arthur}!zzzGoldmann, Paul@\emph{von Paul Goldmann}!1900-04-181@{18. 4. {[}1900{]}}|(be} \toendnotes[C]{\smallbreak\pagebreak[2]} \Standort{DLA, A:Schnitzler, HS.NZ85.1.3170.}
\physDesc{Brief, 2 Blätter, 8 Seiten
\newline{}Handschrift: blaue Tinte, deutsche Kurrent
\newline{}Schnitzler: 1) mit Bleistift das Jahr »{[}1{]}900« vermerkt  2) mit rotem Buntstift fünf Unterstreichungen}\toendnotes[C]{\smallbreak}\pstart
           \noindent{}{\pb}\textcolor{pink}{\textcolor{gray}{\textbf{DESSAUERSTRASSE 19}}}{}\ledrightnote{\textcolor{pink}{Dessauer Straße}}\pend
           \pstart
           \raggedleft{}\textcolor{pink}{Berlin}{}\ledrightnote{\textcolor{pink}{Berlin}}, 18. April.\pend
           \pstart{}Mein lieber Freund,\pend\pstart
           Ich habe mich ſehr mit Deinem lieben Briefe gefreut. Lange habe ich ihn erwartet und
               wußte mir gar nicht zu erklären, warum ich ſo ganz ohne Nachricht blieb. Ich war \strikeout{\textcolor{gray}{nac}h} zum \label{K_L02911-1v}\edtext{\textsc{\textcolor{blue}{Speidel}{}\ledrightnote{\textcolor{blue}{Ludwig Speidel}}}-Banket}{\lemma{\textnormal{\emph{Speidel-Banket}}}\Cendnote{\textnormal{\textcolor{blue}{Schnitzler} nahm an dem Bankett von \textcolor{blue}{Ludwig Speidel} am 15. 4. 1900 teil.
                     »Widerwärtig«, notierte er sich dazu im \emph{\textcolor{green}{Tagebuch}}.}}}\label{K_L02911-1h} geladen und hätte darum ſehr gut nach \textcolor{pink}{Wien}{}\ledrightnote{\textcolor{pink}{Wien}} kommen können und die \textcolor{brown}{N. Fr. Pr.}{}\ledrightnote{\textcolor{brown}{Neue Freie Presse}} hätte mir überdies die Reiſe bezahlen müſſen. Aber
               wenn ich nach \textcolor{pink}{Wien}{}\ledrightnote{\textcolor{pink}{Wien}} komme, ſo komme ich
               Deinetwegen. Und da ich ſo gar nichts von Dir hörte, ........ Aber laſſen wir das! Mir hat meine Hypochondrie wieder einmal \strikeout{\textcolor{gray}{×}\-\textcolor{gray}{×}} einen Streich geſpielt, und es thut mir nun doppelt leid, um die ſchönen
               Oſtertage gekommen zu ſein, {\pb}die ich mit Dir hätte
               verleben können.\pend
           \pstart
           Was Deine \label{K_L02911-2v}\edtext{Furcht vor dem
                  Altwerden}{\lemma{\textnormal{\emph{Furcht vor dem
                  Altwerden}}}\Cendnote{\textnormal{vermutlich Bezug auf \textcolor{blue}{Schnitzler}s bevorstehenden 38. Geburtstag am
                     15. 5. 1900}}}\label{K_L02911-2h} anlangt, – nein, wirklich, mit 38 Jahren iſt man noch nicht alt. Und wenn Du
               Dir das früher einmal als das Ende aller Dinge vorgeſtellt haſt, ſo haſt Du eben
               früher das Leben nicht gekannt, wie man ja ſo Manches ſich unrichtig vorſtellt, wenn
               man gar zu jung iſt. Früher haben Dich die Frauen geliebt, weil Du 20 Jahre alt
               warſt; jetzt haben ſie viel mehr Gründe, Dich zu lieben, und dabei biſt Du immer noch
               jung genug, daß es ihnen Vergnügen macht. Die Geliebten, die Dich ſeinerzeit durch
                  \introOben{}den\introOben{} Hinweis \strikeout{auf ihre}
               beruhigt haben, daß ihre anderen Anbeter Ende der Dreißig ſeien, haben dieſen Anderen
               wahrſcheinlich mit Hinweis auf Dich geſagt: »Das iſt {\pb}ein unreifer Junge. Lieben aber kann man nur einen wirklichen Mann.« Wie alt,
               glaubſt Du, war \textsc{Don Juan}? Jedenfalls nicht zwanzig Jahre.
               Meiner Anſicht nach hatte er zwiſchen 35 und 40, wenn nicht darüber{\dotsseven}\pend
           \pstart
           Auf Deine \label{K_L02911-3v}\edtext{\textcolor{green}{Novelle}{}\ledrightnote{{$\rightarrow$}\textcolor{green}{Frau Bertha Garlan. Roman}}}{\lemma{\textnormal{\emph{Novelle}}}\Cendnote{\textnormal{\textcolor{blue}{Schnitzler} hatte \emph{\textcolor{green}{Frau Bertha Garlan}} am 1. 1. 1900 begonnen und am 16. 4. 1900
                  fertiggestellt.}}}\label{K_L02911-3h} freue ich mich ſehr. Was wird eigentlich aus der \textsc{\textcolor{green}{Beatrice}{}\ledrightnote{\textcolor{green}{Der Schleier der Beatrice. Schauspiel in fünf Akten}}}? Wann beginnen die \label{K_L02911-4v}\edtext{Proben}{\lemma{\textnormal{\emph{Proben}}}\Cendnote{\textnormal{\textcolor{blue}{Schnitzler} war das erste Mal am 23. 11. 1900 bei Proben
                  für die Uraufführung von \emph{\textcolor{green}{Der Schleier der
                     Beatrice}} anwesend.}}}\label{K_L02911-4h}?\pend
           \pstart
           Wie beneide ich Dich um Dein Arbeiten! Ich ſelbſt bringe es nicht zu Stande. Ich habe
               jetzt, nach Wochen angeſpannteſter Arbeit, auch wieder Wochen faſt vollkommener Ruhe.
               Das wäre die Zeit, etwas zu ſchaffen. Ich zermartere mir den Kopf, will heut ein
               Drama ſchreiben, morgen eine Novelle. Aber Alles {\pb}zerrinnt wieder im Nebel. Und ich vergeude meine Zeit mit Beſuchen, mit
               überflüſſiger Reporter-Arbeit und Anderem, wie ja überhaupt der Journalismus eine
               große Zeitvertrödelung iſt. Dabei habe ich das Gefühl, es ſteckt doch noch etwas mehr
               in mir. Aber ich weiß nicht, was ich will. Ich würde Denjengen\strikeout{,} wie einen Erlöſer begrüßen, der mir einen Rath geben,
               mich auf eine größere Arbeit hinweiſen würde, die \strikeout{\textcolor{gray}{me}i\textcolor{gray}{n}} meinen Fähigkeiten entſpräche. Aber, ich weiß, dieſen Rath kann man ſich nur
               ſelbſt geben. Und bei mir finde ich keinen. Ich habe mich ſelten innerlich ſo elend
               gefühlt, mich ſelten ſo verachtet. Große Prätentionen, und innerlich {\pb}Alles leer, le{[}e{]}r! Meine einzige
               Leiſtung iſt, daß ich täglich fetter werde{\dots}\pend
           \pstart
           Im Sommer werde ich wohl meinen Urlaub bekommen. Aber ich werde ihn in \textcolor{pink}{Berlin}{}\ledrightnote{\textcolor{pink}{Berlin}} verbringen müſſen, weil ich diesmal keine
               fünf Mark übrig haben werde, um zu reiſen. Der Hausſtand, den ich hier mit meiner \textcolor{blue}{Mutter}{}\ledrightnote{{$\rightarrow$}\textcolor{blue}{Clementine Goldmann}} führe, \strikeout{ver} nimmt faſt mein ganzes Gehalt in Anſpruch. Der Reſt
               geht für Schulden-Abzahlungen aller Art drauf; und Nebenverdienſt iſt ausgeſchloſſen.
               Nach \label{K_L02911-5v}\edtext{\textsc{\textcolor{pink}{Paris}{}\ledrightnote{\textcolor{pink}{Paris}}}}{\lemma{\textnormal{\emph{Paris}}}\Cendnote{\textnormal{Womöglich erkundigte sich \textcolor{blue}{Schnitzler}, ob \textcolor{blue}{Goldmann} zur Weltausstellung nach \textcolor{pink}{Paris} (15. 4. 1900–12. 11. 1900) fahre.}}}\label{K_L02911-5h} fahre ich unter dieſen
               Umſtänden natürlich nicht.\pend
           \pstart
           {\pb}Kenſt Du \label{K_L02911-42v}\edtext{\textsc{\textcolor{blue}{Flaubert}{}\ledrightnote{\textcolor{blue}{Gustave Flaubert}}s}{ }\textcolor{green}{Briefe}{}\ledrightnote{{$\rightarrow$}\textcolor{green}{Correspondance. 4 Bde.}}}{\lemma{\textnormal{\emph{Flauberts Briefe}}}\Cendnote{\textnormal{\textcolor{blue}{Gustave Flaubert}: \emph{\textcolor{green}{Correspondance}}. 4 Bde. \textcolor{pink}{Paris}: \emph{\textcolor{brown}{Charpentier {\kaufmannsund} Cie}}{ }1887–1893. \textcolor{blue}{Schnitzler} kannte eine spätere \textcolor{green}{Ausgabe} (vgl. A. S.: \emph{Lektüren}, Frankreich).}}}\label{K_L02911-42h}? Wenn
               nicht, ſo mußt Du ſie gleich leſen, und zwar gleich den dritten und vierten \textcolor{green}{Band}{}\ledrightnote{{$\rightarrow$}\textcolor{green}{Correspondance. 4 Bde.}}; die \textcolor{green}{Jugendbriefe}{}\ledrightnote{{$\rightarrow$}\textcolor{green}{Correspondance. 4 Bde.}} in den erſten beiden ſind
               nicht intereſſant. Ich habe ſie jetzt wieder vorgeholt. Jeder Menſch, der ſchreibt,
                  \strikeout{muß} findet darin Troſt, Befreiung und Belehrung.
               Auf dem ſpeciell ſchriftſtelleriſchen Gebiete geben ſie Einem faſt ſo viel, wie
                  \label{K_L02911-7v}\edtext{\textcolor{green}{\textcolor{blue}{Goethe}{}\ledrightnote{\textcolor{blue}{Johann Wolfgang von Goethe}}s Geſpräche}{}\ledrightnote{{$\rightarrow$}\textcolor{green}{Goethes Unterhaltungen mit dem Kanzler Friedrich von Müller}}}{\lemma{\textnormal{\emph{Goethes Geſpräche}}}\Cendnote{\textnormal{siehe Paul Goldmann an Arthur Schnitzler, 25. 9. [1899]}}}\label{K_L02911-7h}; nur ſind ſie nicht ſo univerſell menſchlich, wie dieſe. \textsc{\textcolor{blue}{Flaubert}{}\ledrightnote{\textcolor{blue}{Gustave Flaubert}}} iſt eben doch kein Menſch, ſondern \strikeout{nur} nur ein
                  \textcolor{pink}{Fran}{}\ledrightnote{{$\rightarrow$}\textcolor{pink}{Frankreich}}zoſe{\dotsfour}\pend
           \pstart
           Von \textsc{\textcolor{blue}{Gusti}{}\ledrightnote{\textcolor{blue}{Auguste Chlum}}} weiß\strikeout{’} ich Dir nichts {\pb}zu berichten. Das eigentliche Leben der beiden \textcolor{blue}{Mädels}{}\ledrightnote{{$\rightarrow$}\textcolor{blue}{Auguste Chlum}{\newline}{$\rightarrow$}\textcolor{blue}{Marie Glümer}} bleibt mir
               verſchloſſen. Trotz aller Herzlichkeit der Beziehungen beſteht zwiſchen uns doch
               keine rechte Sympathie, und innerlich ſtehen wir uns fremd gegenüber.\pend
           \pstart
           Was macht \label{K_L02911-8v}\edtext{\textsc{\textcolor{blue}{Richard}{}\ledrightnote{\textcolor{blue}{Richard Beer-Hofmann}}}}{\lemma{\textnormal{\emph{Richard}}}\Cendnote{\textnormal{\textcolor{blue}{Goldmann} bezog sich vermutlich auf \textcolor{blue}{Beer-Hofmann}s Trauerspiel \emph{\textcolor{green}{Der Graf von Charolais}}, an dem er bereits seit 1899 arbeitete. Zu \textcolor{blue}{Beer-Hofmann}s Reisen im Sommer 1900 vgl.
                     Eugene Weber: \emph{Richard Beer-Hofmann: Daten mitgeteilt von
                        Eugene Weber}. In: \emph{Modern Austrian
                        Literature} 17/2 (1984), S. 13–42, hier:
                     S. 23.}}}\label{K_L02911-8h}? Arbeitet er an ſeinem \textcolor{green}{Drama}{}\ledrightnote{{$\rightarrow$}\textcolor{green}{Der Graf von Charolais. Ein Trauerspiel}}? Und was wird er im Sommer machen? Wirſt
               Du mit ihm zuſammen ſein?\pend
           \pstart
           Geſtern ſprach ich wieder einmal \textsc{\textcolor{blue}{Kerr}{}\ledrightnote{\textcolor{blue}{Alfred Kerr}}} nach langer Pauſe. Er ſcheint \textcolor{gray}{nun}{ }\label{K_L02911-11v}\edtext{große \textcolor{blue}{Liebe}{}\ledrightnote{{$\rightarrow$}\textcolor{blue}{Anna Wendt}}}{\lemma{\textnormal{\emph{große Liebe}}}\Cendnote{\textnormal{Bezug auf \textcolor{blue}{Anna Wendt}, die \textcolor{blue}{Alfred
                     Kerr} im April 1900 kennengelernt hatte (vgl.
                     Deborah Vietor-Engländer: \emph{Alfred Kerr. Die
                        Biographie}. Reinbek bei Hamburg:
                        \emph{Rowohlt}{ }2016, S. 229 [E-Book-Ausgabe])}}}\label{K_L02911-11h} zu haben. Ich mag ihn ſehr gern trotz mancher Geſchmack-Defekte; aber er
               ſchließt ſich mir nicht auf. {\pb}Und wir bleiben
               fremd.\pend
           \pstart
           Wann ſehe ich Dich wieder? Wann kommſt Du nach \label{K_L02911-12v}\edtext{\textcolor{pink}{Berlin}{}\ledrightnote{\textcolor{pink}{Berlin}}}{\lemma{\textnormal{\emph{Berlin}}}\Cendnote{\textnormal{siehe Paul Goldmann an Arthur Schnitzler, 13. 4. [1900]}}}\label{K_L02911-12h}?\pend
           \pstart
           Viele treue Grüße! {\\[\baselineskip]}Dein {\\[\baselineskip]}\spacefill\mbox{Paul Goldmann}\pend
           \leftskip=0em{}\pstart
           \noindent{}Meine \textcolor{blue}{Mutter}{}\ledrightnote{{$\rightarrow$}\textcolor{blue}{Clementine Goldmann}} dankt für
                  Deine Grüße und erwidert ſie herzlichſt.\pend
           \endnumbering\briefempfaengerindex{Schnitzler, Arthur@\textsc{Schnitzler, Arthur}!zzzGoldmann, Paul@\emph{von Paul Goldmann}!1900-04-181@{18. 4. {[}1900{]}}|)be}\mylabel{h}\begin{anhang}\end{anhang}\normalsize

\doendnotes{C}
\bigskip
\vfill

\clearpage

\footnotesize

\lohead{\textsc{register}}

% Definiere theindex-Environment komplett neu ohne reledmac
\makeatletter
\renewenvironment{theindex}{%
  \section*{\indexname}%
  \setlength{\parindent}{0pt}%
  \setlength{\parskip}{0pt plus 0.3pt}%
  \let\item\@idxitem
}{%
  \clearpage
}
\makeatother

\IfFileExists{\jobname-pw.ind}{\input{\jobname-pw.ind}}{}

\end{document}

      