%% latex-korrekturansicht-vorspann.tex
%% Vorspann für die Korrekturansicht.
%% Lädt die gemeinsame Datei latex-vorspann.tex mit gesetztem Schalter.

\newif\ifkorrekturansicht
\korrekturansichttrue

\input{../tex-inputs/latex-vorspann}


\renewcommand{\erwaehntePersonen}{Personen: Richard Beer-Hofmann, Ludwig Brefeld, Rosa Freudenthal, Clementine Goldmann, Ernst von Hammerstein-Loxten, Johannes von Miquel, Emil Rechert, Olga Schnitzler, Elisabeth Steinrück}
\renewcommand{\erwaehnteInstitutionen}{Institutionen: Neue Freie Presse}
\renewcommand{\erwaehnteOrte}{Orte: Berlin, Dessauer Straße, Elbe, Engadin, Hannover, Makedonien, Mittellandkanal, Nordmazedonien, Preußen, Welsberg-Taisten, Wien, Wörthersee}
\renewcommand{\erwaehnteWerke}{Werke: Frankfurter Zeitung, Frau Bertha Garlan. Roman, [Novellette]}
\section[ Paul Goldmann an Arthur Schnitzler, 7. 5. {[}1901{]}]{Paul Goldmann an Arthur Schnitzler, 7. 5. {[}1901{]}}
\nopagebreak\mylabel{v}
\rehead{ }\normalsize\beginnumbering\briefempfaengerindex{Schnitzler, Arthur@\textsc{Schnitzler, Arthur}!zzzGoldmann, Paul@\emph{von Paul Goldmann}!1901-05-071@{7. 5. {[}1901{]}}|(be}
\toendnotes[C]{\smallbreak\pagebreak[2]}\Standort{DLA, A:Schnitzler, HS.NZ85.1.3171.}
\physDesc{Brief, 1 Blatt, 4 Seiten
\newline{}Handschrift: blaue Tinte, deutsche Kurrent
\newline{}Schnitzler: 1) mit Bleistift das Jahr »{[}1{]}901.« vermerkt  2) mit rotem Buntstift vier Unterstreichungen}\toendnotes[C]{\smallbreak}
\pstart
           \noindent{}\raggedleft{}{\pb}\textcolor{pink}{\textcolor{gray}{\textbf{DESSAUERSTRASSE 19}}}{}\ledrightnote{\textcolor{pink}{Dessauer Straße}}\pend
           
\pstart
           \textcolor{pink}{Berlin}{}\ledrightnote{\textcolor{pink}{Berlin}}, 7. Mai\pend
           
\pstart\center{}Mein lieber Freund,\pend
\pstart
           Ich habe bei der \textcolor{brown}{N. F. Pr.}{}\ledrightnote{\textcolor{brown}{Neue Freie Presse}} angeregt, mich \label{K_L03065-1v}\edtext{nach \textcolor{pink}{Macedonien}{}\ledrightnote{\textcolor{pink}{Nordmazedonien}{\newline}\textcolor{pink}{Makedonien}} zu ſchicken}{\lemma{\textnormal{\emph{nach … ſchicken}}}\Cendnote{\textnormal{nicht geschehen}}}\label{K_L03065-1h}. Denn ich fühle immer unabweisbarer
               das Bedürfniß, die Kraft, die ich in mir ſpüre, wieder einmal in eine ſchwere Aufgabe
               zu ſetzen, und meinem Schickſal, das mir hart und höhniſch alle Wünſche verſagt,
               wieder einmal davonzugehen. Da ich verflucht bin, nicht geliebt zu werden, will ich
               mich \strikeout{\textcolor{gray}{×}\-\textcolor{gray}{×}\-\textcolor{gray}{×}\-\textcolor{gray}{×}\-\textcolor{gray}{×}\-\textcolor{gray}{×}\-\textcolor{gray}{×}\-\textcolor{gray}{×}{ }\textcolor{gray}{×}\-\textcolor{gray}{×}\-\textcolor{gray}{×}\-\textcolor{gray}{×}\-\textcolor{gray}{×}\-\textcolor{gray}{×}\-\textcolor{gray}{×}} durch neue Eindrücke, harte Arbeit und hoffentlich auch ein wenig Gefahr
               betäuben. \strikeout{\textcolor{gray}{Ob}} Ob man meiner Anregung Folge geben wird, iſt fraglich. Die \label{K_L03065-2v}\edtext{Herren}{\lemma{\textnormal{\emph{Herren}}}\Cendnote{\textnormal{nicht ermittelt}}}\label{K_L03065-2h}, die mein Talent verwalten, benutzen
               dasſelbe lieber zu {\pb}\strikeout{\textcolor{gray}{Ber}} Depeſchen über die \label{K_L03065-3v}\edtext{\textcolor{pink}{preuß}{}\ledrightnote{\textcolor{pink}{Preußen}}iſche Miniſterkriſis}{\lemma{\textnormal{\emph{preußiſche Miniſterkriſis}}}\Cendnote{\textnormal{Bezug auf den von konservativer Seite kritisierten Bau des
                     \textcolor{pink}{Mittellandkanal}s (zwischen \textcolor{pink}{Hannover} und der \textcolor{pink}{Elbe}); Anfang Mai 1901 hatte
                  dieser Konflikt zum Rücktritt des Finanzministers \textcolor{blue}{Johannes von Miquel}, des Landwirtschaftsministers \textcolor{blue}{Ernst von Hammerstein-Loxten} und des
                  Handelsministers \textcolor{blue}{Brefeld} geführt}}}\label{K_L03065-3h}
               und Berichten über die Tage des \textcolor{pink}{Berlin}{}\ledrightnote{\textcolor{pink}{Berlin}}er \label{K_L03065-4v}\edtext{Effektenmarkt}{\lemma{\textnormal{\emph{Effektenmarkt}}}\Cendnote{\textnormal{Wertpapiermarkt}}}\label{K_L03065-4h}es.\pend
           
\pstart
           Mache ich alſo nicht die Reiſe, die ich der \textcolor{brown}{Redaktion}{}\ledrightnote{{$\rightarrow$}\textcolor{brown}{Neue Freie Presse}} vorgeſchlagen habe, ſo werde ich Anfangs Auguſt meinen Urlaub antreten. Diesmal kann es ſich für mich
               nur um den Aufenthalt an einem Ort handeln. Es iſt wieder die leidige Geldfrage.
               Sparen habe ich während des ganzen Jahres nicht gekonnt, dann muß ich meine \textcolor{blue}{Mutter}{}\ledrightnote{{$\rightarrow$}\textcolor{blue}{Clementine Goldmann}} ins \label{K_L03065-5v}\edtext{Bad}{\lemma{\textnormal{\emph{Bad}}}\Cendnote{\textnormal{gemeint war eine Kur}}}\label{K_L03065-5h} ſchicken; und iſt dies gethan, ſo
               bleiben mir im \strikeout{S} beſten Falle etwa 400 \textsc{MK}. Damit kann ich nicht ins \label{K_L03065-7v}\edtext{\textcolor{pink}{Engadin}{}\ledrightnote{\textcolor{pink}{Engadin}}}{\lemma{\textnormal{\emph{Engadin}}}\Cendnote{\textnormal{Bezug unklar, womöglich schlug \textcolor{blue}{Schnitzler} eine Reise in das \textcolor{pink}{Engadin} vor}}}\label{K_L03065-7h} reiſen; ich hätte auch keine Luſt {\pb}dazu. Suche es alſo, bitte, ſo einzurichten, daß wir
               im \uline{ Auguſt} uns \label{K_L03065-8v}\edtext{am \textcolor{pink}{Wörther See}{}\ledrightnote{\textcolor{pink}{Wörthersee}}}{\lemma{\textnormal{\emph{am Wörther See}}}\Cendnote{\textnormal{nicht geschehen, vgl. Paul Goldmann an Arthur Schnitzler, 26. 4. [1901]}}}\label{K_L03065-8h} treffen. \textsc{\textcolor{blue}{Olga}{}\ledrightnote{\textcolor{blue}{Olga Schnitzler}}} und \textsc{\textcolor{blue}{Liesl}{}\ledrightnote{\textcolor{blue}{Elisabeth Steinrück}}} ſollen auch \label{K_L03065-9v}\edtext{hinkommen}{\lemma{\textnormal{\emph{hinkommen}}}\Cendnote{\textnormal{\textcolor{blue}{Olga} und \textcolor{blue}{Elisabeth Gussmann} waren jedenfalls am 7. 8. 1901 gemeinsam mit \textcolor{blue}{Schnitzler} in \textcolor{pink}{Welsberg},
                  wo sich auch \textcolor{blue}{Goldmann} aufhielt.}}}\label{K_L03065-9h}. Mit
                  \label{K_L03065-11v}\edtext{\textsc{\textcolor{blue}{Richard}{}\ledrightnote{\textcolor{blue}{Richard Beer-Hofmann}}}}{\lemma{\textnormal{\emph{Richard}}}\Cendnote{\textnormal{Siehe Paul Goldmann an Arthur Schnitzler, 29. 7. [1901] Möglicherweise
                  trafen sich \textcolor{blue}{Goldmann} und Beer-Hofmann auch
                  am 22. 8. 1901 in
                     \textcolor{pink}{Welsberg}.}}}\label{K_L03065-11h} treffe ich nicht gern
               zuſammen, weil ich wirklich erbittert darüber bin, daß er mir nicht eine Zeile
               geſchrieben hat, ſeit wir uns im letzten Sommer getrennt haben.\pend
           
\pstart
           Was Du mir über Deinen Seelenzuſtand ſchreibſt, iſt wunderſchön, Du haſt zur
               richtigen Zeit offenbar die richtige \textcolor{blue}{Frau}{}\ledrightnote{{$\rightarrow$}\textcolor{blue}{Olga Schnitzler}} getrofſen, und ich hoffe, dieſe Liebe ſoll reiche
               Frucht tragen an dichteriſchen Werken und an Lebensglück.\pend
           
\pstart
           In der \textcolor{green}{Frankf. Zeit.}{}\ledrightnote{\textcolor{green}{Frankfurter Zeitung}} fand ich \label{K_L03065-13v}\edtext{beifolgende {\pb}\textcolor{green}{Novellette}{}\ledrightnote{{$\rightarrow$}\textcolor{green}{[Novellette]}}}{\lemma{\textnormal{\emph{beifolgende Novellette}}}\Cendnote{\textnormal{Beilage nicht erhalten;
                     XXXX}}}\label{K_L03065-13h}. Ich finde, daß ſie feine Beobachtungen und echte \textcolor{pink}{Wien}{}\ledrightnote{\textcolor{pink}{Wien}}er Stimmung enthält. Wer iſt dieſer \textsc{Dr.
                     \textcolor{blue}{Rechert}{}\ledrightnote{\textcolor{blue}{Emil Rechert}}}?\pend
           
\pstart
           Grüße mir die Damen \textsc{\textcolor{blue}{Olga}{}\ledrightnote{\textcolor{blue}{Olga Schnitzler}}} und \textsc{\textcolor{blue}{Liesl}{}\ledrightnote{\textcolor{blue}{Elisabeth Steinrück}}} und ſei Du ſelbſt herzlichſt gegrüßt! {\\[\baselineskip]}Dein treuer {\\[\baselineskip]}\spacefill\mbox{Paul Goldmann.}\pend
           \leftskip=0em{}
\pstart
           \noindent{}Bei der blödſinnigen Arbeitsmenge, die ich zu verrichten habe, konnte ich »\textcolor{green}{Bertha Garlan}{}\ledrightnote{\textcolor{green}{Frau Bertha Garlan. Roman}}« noch nicht leſen. \strikeout{Inzwiſchen} Meine \textcolor{blue}{Mutter}{}\ledrightnote{{$\rightarrow$}\textcolor{blue}{Clementine Goldmann}} iſt ſehr entzückt davon. Inzwiſchen habe ich das
                     \textcolor{green}{Buch}{}\ledrightnote{{$\rightarrow$}\textcolor{green}{Frau Bertha Garlan. Roman}} der \label{K_L03065-14v}\edtext{\textcolor{blue}{Frau Rechtsanwalt}{}\ledrightnote{{$\rightarrow$}\textcolor{blue}{Rosa Freudenthal}}}{\lemma{\textnormal{\emph{Frau Rechtsanwalt}}}\Cendnote{\textnormal{siehe Paul Goldmann an Arthur Schnitzler, 20. 2. 1900}}}\label{K_L03065-14h} borgen müſſen, die an Gelenkrheumatismus erkrankt iſt.\pend
           \endnumbering\briefempfaengerindex{Schnitzler, Arthur@\textsc{Schnitzler, Arthur}!zzzGoldmann, Paul@\emph{von Paul Goldmann}!1901-05-071@{7. 5. {[}1901{]}}|)be}\mylabel{h}
\begin{anhang}
\end{anhang}\normalsize

\doendnotes{C}
\bigskip
\vfill

\clearpage

\footnotesize

\lohead{\textsc{register}}

% Definiere theindex-Environment komplett neu ohne reledmac
\makeatletter
\renewenvironment{theindex}{%
  \section*{\indexname}%
  \setlength{\parindent}{0pt}%
  \setlength{\parskip}{0pt plus 0.3pt}%
  \let\item\@idxitem
}{%
  \clearpage
}
\makeatother

\IfFileExists{\jobname-pw.ind}{\input{\jobname-pw.ind}}{}

\end{document}

      