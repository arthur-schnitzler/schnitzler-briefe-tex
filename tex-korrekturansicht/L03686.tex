%% latex-korrekturansicht-vorspann.tex
%% Vorspann für die Korrekturansicht.
%% Lädt die gemeinsame Datei latex-vorspann.tex mit gesetztem Schalter.

\newif\ifkorrekturansicht
\korrekturansichttrue

\input{../tex-inputs/latex-vorspann}


\section[Stefan Zweig an Arthur Schnitzler, 28. 7. 1923]{L03686 Stefan Zweig an Arthur Schnitzler, 28. 7. 1923}
\nopagebreak\mylabel{L03686v}
\rehead{ }\normalsize\beginnumbering\briefempfaengerindex{, @\textsc{, }!zzz, @\emph{von  }!1923-07-281@{28. 7. 1923}|(be}
\toendnotes[C]{\smallbreak\pagebreak[2]}\Standort{CUL, Schnitzler, B 118.}
\physDesc{Brief, 1 Blatt, 1 Seite, 732 Zeichen
\newline{}Schreibmaschine
\newline{}Handschrift: blaue Tinte, lateinische Kurrent (\noindent{}Unterschrift und Postskriptum)
\newline{}Schnitzler: mit rotem Buntstift zwei Unterstreichungen }
\buchAbdrucke{\weitereDrucke{1) Stefan Zweig: \emph{Briefwechsel mit Hermann Bahr, Sigmund Freud, Rainer Maria
                        Rilke und Arthur Schnitzler}. Herausgegeben von Jeffrey B. Berlin, Hans-Ulrich Lindken und Donald A. Prater. Frankfurt am Main: \emph{S. Fischer} 1987, S. 417.} \weitereDrucke{2) Hermann Bahr, Arthur Schnitzler: \emph{Briefwechsel, Aufzeichnungen, Dokumente (1891–1931)}. Herausgegeben von Kurt Ifkovits und Martin Anton Müller. Göttingen: \emph{Wallstein} 2018, S. 578–579.} }
\pstart
           {\pb}\textcolor{gray}{\textbf{SZ}}\hfill \textcolor{gray}{\textbf{\textcolor{pink}{KAPUZINERBERG 5}\oindex{Paschinger Schlössl@\textbf{Paschinger Schlössl}, \emph{Wohngebäude}|pw}{}\ledrightnote{\textcolor{pink}{Paschinger Schlössl}}}}\pend
           
\pstart
           \raggedleft{}\textcolor{gray}{\textbf{\textcolor{pink}{SALZBURG}\oindex{Salzburg@\textbf{Salzburg}, \emph{Verwaltungsgebiet}|pw}{}\ledrightnote{\textcolor{pink}{Salzburg}},}}\pend
           
\pstart
           \raggedleft{}am 28. Juli 1923.\pend
           
\pstart{}Verehrter Herr Doktor!\pend\vspace{0.5em}
\pstart
            Ich empfange freudig Ihre Nachricht und brauche nicht zu sagen, dass ich Ihnen gern,
               wenn Sie mich rechtzeitig verständigen, im »\textcolor{pink}{Oesterreichischen Hof}\oindex{Österreichischer Hof@\textbf{Österreichischer Hof}, \emph{Hotel}|pw}{}\ledrightnote{\textcolor{pink}{Österreichischer Hof}}« ein Zimmer reserviere. Ich hätte Sie lieber zu uns
               gebeten, aber wir sind durch die Gegenwart \textcolor{blue}{Rollands}\pwindex{Rolland, Romain 29.\,1.\,1866 Clamecy – 30.\,12.\,1944 Vézelay@\textsc{Rolland, Romain} (29.\,1.\,1866 Clamecy – 30.\,12.\,1944 Vézelay), \emph{Schriftsteller}|pw}{}\ledrightnote{\textcolor{blue}{Romain Rolland}} besetzt. Das \textcolor{pink}{Hotel Europe}\oindex{Grand Hotel de L’Europe, G. Jung@\textbf{Grand Hotel de L’Europe, G. Jung}, \emph{Hotel}|pw}{}\ledrightnote{\textcolor{pink}{Grand Hotel de L’Europe, G. Jung}}
               ist aber momentan wirklich etwas kostspielig und Sie werden im »\textcolor{pink}{Oesterreichischen Hof}\oindex{Österreichischer Hof@\textbf{Österreichischer Hof}, \emph{Hotel}|pw}{}\ledrightnote{\textcolor{pink}{Österreichischer Hof}}{[}«{]} ebenso zufrieden sein.\pend
           
\pstart
           Gestern und heute waren wir mit \textcolor{blue}{Bahr}\pwindex{Bahr, Hermann 19.\,7.\,1863 Linz – 15.\,1.\,1934 München@\textsc{Bahr, Hermann} (19.\,7.\,1863 Linz – 15.\,1.\,1934 München), \emph{Schriftsteller, Kritiker}|pw}{}\ledrightnote{\textcolor{blue}{Hermann Bahr}} und heute ging er in einem Zuge zur \textcolor{pink}{Gaissbergspitze}\oindex{Gaisberg@\textbf{Gaisberg}, \emph{Berg}|pw}{}\ledrightnote{\textcolor{pink}{Gaisberg}} hinauf. Es war ein rechtes
               Vergnügen, ihn so heiter und wohlgelaunt, wie seit Jahren nicht, zu sehen.\pend
           
\pstart
           In herzlicher Erwartung Ihnen entgegen und aufrichtig ergeben Ihr{\\[\baselineskip]}\spacefill\mbox{{[}hs.:{]} Stefan Zweig}\pend
           \leftskip=0em{}
\pstart
           \noindent{}{[}hs.:{]} \uline{P.S.} Auch \textcolor{blue}{Bahr}\pwindex{Bahr, Hermann 1858/1859 – 1939 Wien@\textsc{Bahr, Hermann} (1858/1859 – 1939 Wien), \emph{Privatbeamter}|pw}{}\ledrightnote{\textcolor{blue}{Hermann Bahr}} kommt in jenen Tagen aus \textcolor{pink}{München}\oindex{München@\textbf{München}|pw}{}\ledrightnote{\textcolor{pink}{München}} herüber.\pend
           \selectlanguage{ngerman}\endnumbering\briefempfaengerindex{, @\textsc{, }!zzz, @\emph{von  }!1923-07-281@{28. 7. 1923}|)be}\mylabel{L03686h}
\begin{anhang}
\end{anhang}\normalsize

\doendnotes{C}
\bigskip
\vfill

\clearpage

\footnotesize

\lohead{\textsc{register}}

% Definiere theindex-Environment komplett neu ohne reledmac
\makeatletter
\renewenvironment{theindex}{%
  \section*{\indexname}%
  \setlength{\parindent}{0pt}%
  \setlength{\parskip}{0pt plus 0.3pt}%
  \let\item\@idxitem
}{%
  \clearpage
}
\makeatother

\IfFileExists{\jobname-pw.ind}{\input{\jobname-pw.ind}}{}

\end{document}

      