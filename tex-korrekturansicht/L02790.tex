%% latex-korrekturansicht-vorspann.tex
%% Vorspann für die Korrekturansicht.
%% Lädt die gemeinsame Datei latex-vorspann.tex mit gesetztem Schalter.

\newif\ifkorrekturansicht
\korrekturansichttrue

\input{../tex-inputs/latex-vorspann}


               \section[Paul Goldmann an Arthur Schnitzler, Paul Goldmann an Arthur Schnitzler, 13. 11. {[}1896{]}]{ Paul Goldmann an Arthur Schnitzler, 13. 11. {[}1896{]}}\nopagebreak\mylabel{v}\rehead{ }\normalsize\beginnumbering\briefempfaengerindex{Schnitzler, Arthur@\textsc{Schnitzler, Arthur}!zzzGoldmann, Paul@\emph{von Paul Goldmann}!1896-11-131@{13. 11. {[}1896{]}}|(be} \toendnotes[C]{\smallbreak\pagebreak[2]} \Standort{DLA, A:Schnitzler, HS.NZ85.1.3166.}
\physDesc{Brief, 3 Blätter, 10 Seiten
\newline{}Handschrift: blaue Tinte, deutsche Kurrent\newline{}Beilage: zwei beschnittene und zusammengeklebte Zeitungsausschnitte auf
                                 der ersten Seite, der eine aus der Kopfzeile bestehend 
\newline{}Schnitzler: 1) mit Bleistift das Jahr »96« vermerkt 2) mit rotem Buntstift eine Unterstreichung}\toendnotes[C]{\smallbreak}\pstart
           \noindent{}{\pb}\textcolor{gray}{\textbf{\textbf{\textcolor{brown}{Frankfurter Zeitung}{}\ledrightnote{\textcolor{brown}{Frankfurter Zeitung}}}}}\pend
           \pstart
           \textcolor{gray}{\textbf{(\textcolor{brown}{\begin{otherlanguage}{french}Gazette de Francfort\end{otherlanguage}}{}\ledrightnote{\textcolor{brown}{Frankfurter Zeitung}}).}}\pend
           \pstart
           \textcolor{gray}{\textbf{\textbf{\begin{otherlanguage}{french}Fondateur M.\end{otherlanguage}{ }\textcolor{blue}{L. Sonnemann}{}\ledrightnote{\textcolor{blue}{Leopold Sonnemann}}.}}}\pend
           \pstart
           \begin{otherlanguage}{french}\textcolor{gray}{\textbf{\textcolor{green}{Journal}{}\ledrightnote{→\textcolor{green}{Frankfurter Zeitung}} politique,
                        financier,}}\end{otherlanguage}\pend
           \pstart
           \begin{otherlanguage}{french}\textcolor{gray}{\textbf{commercial et littéraire.}}\end{otherlanguage}\pend
           \pstart
           \begin{otherlanguage}{french}\textcolor{gray}{\textbf{\textbf{Paraissant trois fois par jour.}}}\end{otherlanguage}\pend
           \pstart
           \begin{otherlanguage}{french}\textcolor{gray}{\textbf{\textbf{Bureau à \textcolor{pink}{Paris}{}\ledrightnote{\textcolor{pink}{Paris}}}}}\end{otherlanguage}\pend
           \pstart
           \begin{otherlanguage}{french}\textcolor{gray}{\textbf{\textbf{\textcolor{pink}{24. Rue Feydeau}{}\ledrightnote{\textcolor{pink}{rue Feydeau}}.}}}\end{otherlanguage}\pend
           {\bigskip}\pstart
           \noindent{}\centering{}\label{K_L02790-88v}\edtext{\begin{otherlanguage}{french}\textcolor{gray}{\textbf{\textcolor{brown}{LE FIGARO}{}\ledrightnote{\textcolor{brown}{Le Figaro}}{ }MARDI 10 NOVEMBRE}}\end{otherlanguage}}{\lemma{\textnormal{\emph{Le … Mardi 10 Novembre}}}\Cendnote{\textnormal{französisch: \emph{\textcolor{brown}{Le Figaro}}{ }Dienstag, 10. November}}}\label{K_L02790-88h}\pend
           \leftskip=3em{}\pstart
           \noindent{}\label{K_L02790-77v}\edtext{\begin{otherlanguage}{french}\textcolor{gray}{\textbf{Mon cher \textcolor{blue}{Huret}{}\ledrightnote{\textcolor{blue}{Jules Huret}},
                     }}\end{otherlanguage}}{\lemma{\textnormal{\emph{Mon cher Huret,
                     }}}\Cendnote{\textnormal{französisch: Mein lieber
                     Huret,}}}\label{K_L02790-77h}\pend
           \leftskip=0em{}\pstart
           \noindent{}\label{K_L02790-16v}\edtext{\begin{otherlanguage}{french}\textcolor{gray}{\textbf{Pour compléter vos \label{K_L02790-111v}\edtext{\textcolor{green}{renseignements}{}\ledrightnote{→\textcolor{green}{Courrier des Théatres [Freiwild in Berlin und Liebelei]}}}{\lemma{\textnormal{\emph{renseignements}}}\Cendnote{\textnormal{\textcolor{blue}{Jules Huret} leitete die
                           Theaterrubrik des \emph{\textcolor{green}{Figaro}}. Das \textcolor{green}{Telegramm} des \textcolor{blue}{Berliner
                              Korrespondenten} wurde abgedruckt: \emph{\textcolor{green}{Le Figaro}}, Jg. 42, Nr. 312, 7. 11. 1896, S. 4.}}}\label{K_L02790-111h} sur
                        Arthur Schnitzler, laissez-moi vous dire que je viens de terminer la \textcolor{green}{traduction}{}\ledrightnote{→\textcolor{green}{Amourette. Pièce en trois actes. Adaptée de Arthur Schnitzler}} en français
                        de cette \emph{\textcolor{green}{\label{T_L02790-56v}\edtext{Liebelei}{\lemma{\textnormal{\emph{Liebelei}}}\Cendnote{\textnormal{im gedruckten Text steht:
                                    »Liebelci«}}}\label{T_L02790-56h}}{}\ledrightnote{\textcolor{green}{Liebelei. Schauspiel in drei Akten}}} dont vous rappelez le grand succès, l’hiver dernier, à \textcolor{pink}{Vienne}{}\ledrightnote{\textcolor{pink}{Wien}}.}}\end{otherlanguage}}{\lemma{\textnormal{\emph{Pour … Vienne.}}}\Cendnote{\textnormal{französisch: Um Ihre \textcolor{green}{Auskünfte} über \textcolor{blue}{Arthur Schnitzler} zu vervollständigen, möchte ich
                     kundtun, dass ich gerade die französische \textcolor{green}{Übersetzung} der \emph{\textcolor{green}{Liebelei}} abgeschlossen habe, an deren großen Erfolg in \textcolor{pink}{Wien} im letzten Winter Sie sich
                     erinnern.}}}\label{K_L02790-16h}\pend
           \pstart
           \label{K_L02790-12v}\edtext{\begin{otherlanguage}{french}\textcolor{gray}{\textbf{Déjà \label{K_L02790-14v}\edtext{\textcolor{blue}{deux de nos
                           directeurs de théâtre}{}\ledrightnote{→\textcolor{blue}{Paul Ginisty}{\newline}→\textcolor{blue}{Albert Carré}}}{\lemma{\textnormal{\emph{deux … théâtre}}}\Cendnote{\textnormal{vgl. Paul Goldmann an Arthur Schnitzler, 2. [1.? 1897]}}}\label{K_L02790-14h} m’ont promis{\dots} de lire cette \textcolor{green}{traduction}{}\ledrightnote{→\textcolor{green}{Amourette. Pièce en trois actes. Adaptée de Arthur Schnitzler}}. Ai-je
                        besoin d’ajouter qu’ils se proposent même de faire cette lecture »avec le
                        plus vif intérêt«.}}\end{otherlanguage}}{\lemma{\textnormal{\emph{Déjà … intérêt«.}}}\Cendnote{\textnormal{französisch: Zwei unserer
                     Theaterdirektoren haben mir bereits versprochen, die \textcolor{green}{Übersetzung} zu lesen. Muss ich noch
                     hinzufügen, dass sie diese Lektüre »mit dem lebhaftesten Interesse«
                     unternehmen?}}}\label{K_L02790-12h}\pend
           \pstart
           \raggedleft{}\label{K_L02790-889v}\edtext{\begin{otherlanguage}{french}\textcolor{gray}{\textbf{Votre bien dévoué,}}\end{otherlanguage}}{\lemma{\textnormal{\emph{Votre bien dévoué,}}}\Cendnote{\textnormal{französisch: Ihr sehr
                     ergeber}}}\label{K_L02790-889h}\pend
           \pstart
           \noindent{}\raggedleft{}\textcolor{gray}{\textbf{\textcolor{blue}{Jean THOREL}{}\ledrightnote{\textcolor{blue}{Jean Thorel}}.}}\pend
           {\bigskip}\pstart
           \raggedleft{}\textsc{\textcolor{pink}{Paris}{}\ledrightnote{\textcolor{pink}{Paris}}}, 13. November.\pend
           \pstart\center{}Mein lieber Freund,\pend\pstart
           Oben ſiehſt Du einen \label{K_L02790-1v}\edtext{\textcolor{green}{Ausſchnitt}{}\ledrightnote{→\textcolor{green}{Courrier des Théatres [Mon cher Huret; Thorel zur Liebelei-Übersetzung]}} aus dem »\textsc{\textcolor{green}{Figaro}{}\ledrightnote{\textcolor{green}{Le Figaro}}}«}{\lemma{\textnormal{\emph{Ausſchnitt … »Figaro«}}}\Cendnote{\textnormal{\textcolor{blue}{Jean Thorel}: \emph{\textcolor{green}{[Mon cher Huret]}}. In: \emph{\textcolor{green}{Le Figaro}}, Jg. 42, Nr. 315, 10. 11. 1896, S. 4.}}}\label{K_L02790-1h}. Die \textcolor{green}{Überſetzung}{}\ledrightnote{→\textcolor{green}{Amourette. Pièce en trois actes. Adaptée de Arthur Schnitzler}} von \textsc{\textcolor{blue}{Thorel}{}\ledrightnote{\textcolor{blue}{Jean Thorel}}} iſt – unter uns geſagt – leider recht ſchlecht, noch ſchlechter, als ich
               geglaubt. Er hat ſich gar keine Mühe gegeben, \strikeout{\textcolor{gray}{die}} das natürliche und lebendige Deutſch des \textcolor{green}{Dialog}{}\ledrightnote{→\textcolor{green}{Liebelei. Schauspiel in drei Akten}}es in natürliches und lebendiges Franzöſiſch
               umzuſetzen. Ich tröſte mich damit, daß es ein Anderer noch ſchlechter gemacht hätte.
                  {\pb}Auch rechne ich auf die dem \textcolor{green}{Stücke}{}\ledrightnote{→\textcolor{green}{Liebelei. Schauspiel in drei Akten}} innewohnende Poeſie, die ſich beim
               beſten Willen nicht umbringen läßt{\dotsfive}\pend
           \pstart
           Mit Deinem lieben Briefe habe ich mich ſehr gefreut. Ich begreife Deine Stimmung, und
               da Du Dir gewiß über die Gründe klar biſt, wird auch dieſes zweite \textcolor{green}{Stück}{}\ledrightnote{→\textcolor{green}{Freiwild. Schauspiel in 3 Akten}} für Deine Entwickelung nützlich ſein.
                  \label{K_L02790-3v}\edtext{Das \textcolor{green}{Stück}{}\ledrightnote{→\textcolor{green}{Freiwild. Schauspiel in 3 Akten}} iſt Dir unſympathiſch}{\lemma{\textnormal{\emph{Das … unſympathiſch}}}\Cendnote{\textnormal{siehe A. S.: \emph{Tagebuch}, 5. 11. 1896}}}\label{K_L02790-3h}, weil es nicht Deiner Natur und Deiner Schaffensart entſpricht. Es iſt nicht
               aus dem Leben herausgewachſen, ſondern aus einer Idee, zu der hinterdrein die Figuren
               geſucht wurden. Beſonders {\pb}ſieht man das an dem \textcolor{green}{Helden}{}\ledrightnote{→\textcolor{green}{Freiwild. Schauspiel in 3 Akten}}. Den haſt Du nie
               geſehen. Du haſt ihn Dir künſtlich zuſammenzimmern müſſen, damit er zu Deiner Idee
               paßt. Darum biſt Du ſo unſicher bei ſeiner Geſtaltung geweſen, darum iſt er Dir ſo
               ſchwer gefallen, darum iſt er auch heut nicht recht gelungen. Und der Hauptfehler
               war: Es war ein Tendenzſtück, und Du haſt Dir das nicht eingeſtehen wollen und haſt
               es nicht als Tendenzſtück ſchreiben wollen. Es war ein Tendenzſtück, das ſo ausſehen
               ſollte, als ſei es natürlich {\pb}und erlebt. Das iſt
               unmöglich. Die \label{K_L02790-777v}\edtext{\textsc{procédés}}{\lemma{\textnormal{\emph{procédés}}}\Cendnote{\textnormal{französisch: das Prozedere}}}\label{K_L02790-777h}
               Deiner Kunſt, die Natürliches und Erlebtes ausdrücken will und kann, waren hier im
               Zwieſpalt mit den Anforderungen des \textsc{Sujets}. Gerade die
               Unparteilichkeit halte ich für einen Fehler des \textcolor{green}{Stück}{}\ledrightnote{→\textcolor{green}{Freiwild. Schauspiel in 3 Akten}}es. Es mußte parteilich ſein. Es mußte ein \textcolor{green}{Stück}{}\ledrightnote{→\textcolor{green}{Freiwild. Schauspiel in 3 Akten}} werden gegen das Duell.
               Für dieſes \textcolor{green}{Stück}{}\ledrightnote{→\textcolor{green}{Freiwild. Schauspiel in 3 Akten}} mußteſt Du
               Deine bisherige Productions-Art beiſeite laſſen und \introOben{}Du\introOben{}
               mußteſt es mit Haß und Leidenſchaft ſchreiben, \strikeout{g} ganz
               ohne Rückſicht darauf, ob es unwahrſcheinlich und {\pb}ungerecht wurde. Ich meine, Du ſollſt fürs Erſte von allen Stoffen dieſer Art, von
               allen »großen Zeitfragen« \textsc{etc.} laſſen. Ich möchte Dir jetzt
               gerade einen \strikeout{\textcolor{gray}{×}\-\textcolor{gray}{×}\-\textcolor{gray}{×}\-\textcolor{gray}{×}\-\textcolor{gray}{×}\-\textcolor{gray}{×}\-\textcolor{gray}{×}\-\textcolor{gray}{×}} Wanderzug in die Vergangenheit und in die reine Poeſie empfehlen. \uline{Das hiſtoriſche \textcolor{pink}{Wien}{}\ledrightnote{\textcolor{pink}{Wien}}er
                  Stück!} Jetzt mußt Du es ſchreiben, und ich bin überzeugt, es wird Dir
               köſtlich gelingen. Nimm’ Dir zwei oder drei Jahre Zeit und ruhe Dich ein wenig auf
               den zwei ſtarken Erfolgen aus, durch welche Du mit einem {\pb}Male in die allererſte Reihe unter den deutſchen
               Bühnen-Dichtern gerückt biſt. Ich möchte Dir einen ſchönen Stoff vorſchlagen: \uline{\textsc{\textcolor{blue}{Mozart}{}\ledrightnote{\textcolor{blue}{Wolfgang Amadeus Mozart}}}}, ein \textcolor{pink}{Wien}{}\ledrightnote{\textcolor{pink}{Wien}}er Volksſtück mit \textsc{\textcolor{blue}{Mozart}{}\ledrightnote{\textcolor{blue}{Wolfgang Amadeus Mozart}}}’ſcher Muſik. Ich hatte neulich Gelegenheit, \textsc{\textcolor{blue}{Otto Jahn}{}\ledrightnote{\textcolor{blue}{Otto Jahn}}s}{ }\textcolor{green}{\textsc{\textcolor{blue}{Mozart}{}\ledrightnote{\textcolor{blue}{Wolfgang Amadeus Mozart}}}-Biographie}{}\ledrightnote{→\textcolor{green}{W. A. Mozart}} einzuſehen. Natürlich hatte ich keine Zeit, die beiden
               dicken \textcolor{green}{Bände}{}\ledrightnote{→\textcolor{green}{W. A. Mozart}} ganz zu leſen.
               Aber aus dem, was ich geleſen, habe ich den Eindruck gewonnen, daß es ganz einfach
               eine der beſten Biographien iſt, die es gibt. Lies’ das \textcolor{green}{Werk}{}\ledrightnote{→\textcolor{green}{W. A. Mozart}}. Du wirſt \textsc{\textcolor{blue}{Mozart}{}\ledrightnote{\textcolor{blue}{Wolfgang Amadeus Mozart}}}{ }{\pb}lieb gewinnen, er wird Dir nahe treten als \textcolor{pink}{Wien}{}\ledrightnote{\textcolor{pink}{Wien}}er\strikeout{,}{ }\strikeout{als} und als Künſtler. Es iſt ein erſchütterndes
               Ringen in dieſem Leben, das nach dem Dramatiker ruft. Es laſſen ſich ſchöne Dinge
               ſagen über Kunſt und Dummheit und Infamie der Kritik und des Publicums – Dinge, die
               wir oft erlebt haben. Und am Schluß ein großartiges, ergreifendes Sterben, in welches
               das Übernatürliche hineingreift durch die ſo unendlich ſeltſame Geſchichte mit dem
                  \textsc{\textcolor{green}{Requiem}{}\ledrightnote{\textcolor{green}{Requiem d-Moll KV 626}}}. Alles, was Du vom Tode weißt, {\pb}kannſt Du da
               ſagen, und das Publicum \strikeout{dürfte \textcolor{gray}{m}} müßte im Unklaren darüben bleiben, ob der \label{K_L02790-123v}\edtext{geheimnißvolle \textcolor{blue}{Mann}{}\ledrightnote{\textcolor{blue}{Franz Walsegg-Stuppach}}, der das \textsc{Requiem} beſtellt}{\lemma{\textnormal{\emph{geheimnißvolle … beſtellt}}}\Cendnote{\textnormal{Das \emph{\textcolor{green}{Requiem d-Moll}} (KV
                  626) wurde von \textcolor{blue}{Franz von Walsegg} über
                  Mittelsmänner beauftragt. Dass \textcolor{blue}{Mozart}
                  während der Komposition einer Seelenmesse starb, wurde als Hinweis genommen, bei
                  dem zu dieser Zeit noch verborgene Auftraggeber hätte es sich um ein
                  übernatürliches Wesen gehandelt.}}}\label{K_L02790-123h}, nicht wirklich aus dem Übernatürlichen
               herkommt. Und \strikeout{d} um das Alles herum das alte liebe \textcolor{pink}{Wien}{}\ledrightnote{\textcolor{pink}{Wien}} und ſogar, bitte, der Kaiſer \textsc{\textcolor{blue}{Josef}{}\ledrightnote{\textcolor{blue}{Josef II.}}} (der ſich allerdings in der Sache ſehr dumm benommen hat).\pend
           \pstart
           Dieſer Tage ſende ich Dir auch \strikeout{ein} das erſte
               franzöſiſche \textcolor{green}{Buch}{}\ledrightnote{→\textcolor{green}{Adolphe. Anecdote trouvée dans les papiers d’un inconnu}}, das ich
               ſeit Langem mit Genuß geleſen habe (dieſer Satz iſt {\pb}grammatikaliſch ſehr falſch). Es ſtammt natürlich aus dem Jahre 1820 und iſt ganz einfach der größte pſychologiſche Roman, den es gibt:
                  \label{K_L02790-32v}\edtext{»\textsc{\textcolor{green}{Adolphe}{}\ledrightnote{\textcolor{green}{Adolphe. Anecdote trouvée dans les papiers d’un inconnu}}}« von \textsc{\textcolor{blue}{Benjamin Constant}{}\ledrightnote{\textcolor{blue}{Benjamin Constant}}}}{\lemma{\textnormal{\emph{»Adolphe« … Constant}}}\Cendnote{\textnormal{Eine zeitnahe Rezeption durch \textcolor{blue}{Schnitzler} ist nicht belegt. Dieser beendete
                  die Lektüre von \emph{\textcolor{green}{Adolphe}} am 7. 2. 1906.}}}\label{K_L02790-32h}.
               Freilich ein Buch ohne Wärme, aber wie aus Erz gegoſſen, – nicht ein Wort zu viel,
               nicht eines zu wenig – die unerbittlichſte Analyſe eines ſchwachen Characters, die je
               ausgeführt worden. Und wenn man bedenkt, daß \strikeout{\textcolor{gray}{m}} wir \strikeout{\textcolor{gray}{hinterher}}{ }{\pb}\textsc{\textcolor{blue}{Paul Bourget}{}\ledrightnote{\textcolor{blue}{Paul Bourget}}} bewundert haben, nachdem es einen »\textsc{\textcolor{green}{Adolphe}{}\ledrightnote{→\textcolor{green}{Adolphe. Anecdote trouvée dans les papiers d’un inconnu}}}« gegeben hat!\pend
           \pstart
           Grüß’ Dich Gott, mein lieber Freund!\pend
           \pstart
           Schreib’ mir bald!\pend
           \pstart
           In Treue {\\[\baselineskip]}Dein {\\[\baselineskip]}\spacefill\mbox{Paul Goldmann.}\pend
           \leftskip=0em{}\pstart
           \noindent{}Wenn Du den \strikeout{L\textcolor{gray}{eo}}{ }\label{K_L02790-24v}\edtext{\textcolor{blue}{\textsc{Leo Fanjung}}{}\ledrightnote{\textcolor{blue}{Leo Van-Jung}} ſiehſt}{\lemma{\textnormal{\emph{Leo Fanjung ſiehſt}}}\Cendnote{\textnormal{Das nächste belegte
                     Zusammentreffen von \textcolor{blue}{Schnitzler} und \textcolor{blue}{Leo Van-Jung} fand am 22. 11. 1896
                     statt.}}}\label{K_L02790-24h}, ſo grüß’ ihn, bitte.\pend
           \endnumbering\briefempfaengerindex{Schnitzler, Arthur@\textsc{Schnitzler, Arthur}!zzzGoldmann, Paul@\emph{von Paul Goldmann}!1896-11-131@{13. 11. {[}1896{]}}|)be}\mylabel{h}  \normalsize

\doendnotes{C}
\bigskip
\vfill

\clearpage

\footnotesize

\lohead{\textsc{register}}

% Definiere theindex-Environment komplett neu ohne reledmac
\makeatletter
\renewenvironment{theindex}{%
  \section*{\indexname}%
  \setlength{\parindent}{0pt}%
  \setlength{\parskip}{0pt plus 0.3pt}%
  \let\item\@idxitem
}{%
  \clearpage
}
\makeatother

\IfFileExists{\jobname-pw.ind}{\input{\jobname-pw.ind}}{}

\end{document}

      