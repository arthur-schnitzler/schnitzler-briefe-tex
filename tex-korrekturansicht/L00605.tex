%% latex-korrekturansicht-vorspann.tex
%% Vorspann für die Korrekturansicht.
%% Lädt die gemeinsame Datei latex-vorspann.tex mit gesetztem Schalter.

\newif\ifkorrekturansicht
\korrekturansichttrue

\input{../tex-inputs/latex-vorspann}


               \section[Hugo von Hofmannsthal an Arthur Schnitzler, {[}15.? 10. 1896{]}]{ Hugo von Hofmannsthal an Arthur Schnitzler, {[}15.? 10. 1896{]}}\nopagebreak\mylabel{v}\rehead{ }\normalsize\beginnumbering\briefempfaengerindex{Schnitzler, Arthur@\textsc{Schnitzler, Arthur}!zzzHofmannsthal, Hugo von@\emph{von Hugo von Hofmannsthal}!1896-10-151@{{[}15.? 10. 1896{]}}|(be} \toendnotes[C]{\smallbreak\pagebreak[2]} \Standort{CUL, Schnitzler, B 43.}
\physDesc{Brief, 1 Blatt, 1 Seite
\newline{}Handschrift: schwarze Tinte, deutsche Kurrent
\newline{}Schnitzler: mit Bleistift datiert: »16/X 96« \newline{}Ordnung: mit Bleistift von unbekannter Hand nummeriert:
                                    »81« }\buchAbdrucke{\weitereDrucke{Hugo von Hofmannsthal, Arthur Schnitzler: \emph{Briefwechsel}. Hg. Therese Nickl und Heinrich Schnitzler. Frankfurt am Main: \emph{S. Fischer} 1964, S. 75.} }\toendnotes[C]{\smallbreak}\pstart{}{\pb}lieber Arthur\pend\pstart
           Sehr gern will ich wenn mir nichts dazwiſchen kommt, \label{K_L00605_1v}\edtext{übermorgen}{\lemma{\textnormal{\emph{übermorgen}}}\Cendnote{\textnormal{Die
                  Datierung \textcolor{blue}{Schnitzler}s dürfte sich auf den
                  Zeitpunkt des Posterhalts beziehen, vgl. A. S.: \emph{Tagebuch}, 17. 10. 1896. Am Folgetag macht \textcolor{blue}{Schnitzler} am Vormittag einen Ausflug.}}}\label{K_L00605_1h} um 11\textsuperscript{h} v. m. im \textcolor{pink}{Central}{}\ledrightnote{\textcolor{pink}{Café Central}}{ }ſein. Herzlich\pend
           \pstart \spacefill\mbox{Hugo.}\pend{}\pstart
           Donnerstag.\pend
           \endnumbering\briefempfaengerindex{Schnitzler, Arthur@\textsc{Schnitzler, Arthur}!zzzHofmannsthal, Hugo von@\emph{von Hugo von Hofmannsthal}!1896-10-151@{{[}15.? 10. 1896{]}}|)be}\mylabel{h}  \normalsize

\doendnotes{C}
\bigskip
\vfill

\clearpage

\footnotesize

\lohead{\textsc{register}}

% Definiere theindex-Environment komplett neu ohne reledmac
\makeatletter
\renewenvironment{theindex}{%
  \section*{\indexname}%
  \setlength{\parindent}{0pt}%
  \setlength{\parskip}{0pt plus 0.3pt}%
  \let\item\@idxitem
}{%
  \clearpage
}
\makeatother

\IfFileExists{\jobname-pw.ind}{\input{\jobname-pw.ind}}{}

\end{document}

      