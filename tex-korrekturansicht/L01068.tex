%% latex-korrekturansicht-vorspann.tex
%% Vorspann für die Korrekturansicht.
%% Lädt die gemeinsame Datei latex-vorspann.tex mit gesetztem Schalter.

\newif\ifkorrekturansicht
\korrekturansichttrue

\input{../tex-inputs/latex-vorspann}


               \section[Arthur Schnitzler und Paul Goldmann an Richard Beer-Hofmann, 26. 8. 1900]{ Arthur Schnitzler und Paul Goldmann an Richard Beer-Hofmann,
                    26. 8. 1900}\nopagebreak\mylabel{v}\rehead{ }\normalsize\beginnumbering\briefempfaengerindex{Beer-Hofmann, Richard@\textsc{Beer-Hofmann, Richard}!zzzGoldmann, Paul@\emph{von Paul Goldmann}!1900-08-261@{26. 8. 1900}|(be}\briefempfaengerindex{Beer-Hofmann, Richard@\textsc{Beer-Hofmann, Richard}!zzzSchnitzler, Arthur@\emph{von Arthur Schnitzler}!1900-08-261@{26. 8. 1900}|(be} \toendnotes[C]{\smallbreak\pagebreak[2]} \Standort{YCGL, MSS 31.}
\physDesc{Postkarte
\newline{}Handschrift Paul Goldmann: Bleistift, deutsche Kurrent\newline{}Handschrift Arthur Schnitzler: Bleistift, deutsche Kurrent\newline{}Versand: 1) Stempel: »\nobreak{}\oindex{Bormio@\textbf{Bormio}, \emph{https://www.geonames.org/ontologyP.PPLA3}|pwk}Bagni nuovi di Bormio, 27 \textcolor{gray}{8} 00, 6–7V\nobreak{}«.  2) Stempel: »\nobreak{}\oindex{Altaussee@\textbf{Altaussee}, \emph{http://www.geonames.org/ontologyA.ADM3}|pwk}\textcolor{gray}{Alt-Aussee}, 30 8 00\nobreak{}«. \newline{}Ordnung: mit Bleistift von unbekannter Hand datiert: »26. 8.« }\toendnotes[C]{\smallbreak}\pstart{}\textsc{{\pb}\textcolor{pink}{Steiermark}{}\ledrightnote{\textcolor{pink}{Steiermark}}.}\pend{}\pstart{}{\pb}Dr. \textsc{Richard
                            Beer-Hofmann}\pend{}\pstart{}\textcolor{pink}{\textsc{Altaussee}}{}\ledrightnote{\textcolor{pink}{Altaussee}}\pend{}{\bigskip}\pstart
           \noindent{}\centering{}{\pb}\textcolor{gray}{\textbf{\textcolor{pink}{Bormio}{}\ledrightnote{\textcolor{pink}{Bormio}}}}\pend
           \pstart
           \noindent{}\centering{}\textcolor{gray}{\textbf{\textcolor{pink}{I VECCHI BAGNI}{}\ledrightnote{\textcolor{pink}{Alte Therme}}.}}\pend
           \pstart
           26. 8. 900.\pend
           \pstart
           Schönes Wetter, zerriſſene Straßen, \label{K_L01068_1v}\edtext{\textcolor{pink}{Tirol}{}\ledrightnote{\textcolor{pink}{Tirol}{\newline}\textcolor{pink}{Südtirol}}er Sänger}{\lemma{\textnormal{\emph{Tiroler Sänger}}}\Cendnote{\textnormal{Das Erlebnis der Sänger
                        dürfte für die Entwicklung der Novelle \emph{\textcolor{green}{Der
                            blinde Geronimo und sein Bruder}} bedeutsam geworden sein.}}}\label{K_L01068_1h},
                    herzliche Grüße.\pend
           \pstart \spacefill\mbox{Arthur}\pend{}\pstart
           \noindent{}{[}hs. Goldmann:{]} Ich wünſche Du wäreſt auch da.\pend
           \pstart
           Herzlichſt{\\[\baselineskip]}\spacefill\mbox{Paul Goldmann.}\pend
           \leftskip=0em{}\endnumbering\briefempfaengerindex{Beer-Hofmann, Richard@\textsc{Beer-Hofmann, Richard}!zzzGoldmann, Paul@\emph{von Paul Goldmann}!1900-08-261@{26. 8. 1900}|)be}\briefempfaengerindex{Beer-Hofmann, Richard@\textsc{Beer-Hofmann, Richard}!zzzSchnitzler, Arthur@\emph{von Arthur Schnitzler}!1900-08-261@{26. 8. 1900}|)be}\mylabel{h}  \normalsize

\doendnotes{C}
\bigskip
\vfill

\clearpage

\footnotesize

\lohead{\textsc{register}}

% Definiere theindex-Environment komplett neu ohne reledmac
\makeatletter
\renewenvironment{theindex}{%
  \section*{\indexname}%
  \setlength{\parindent}{0pt}%
  \setlength{\parskip}{0pt plus 0.3pt}%
  \let\item\@idxitem
}{%
  \clearpage
}
\makeatother

\IfFileExists{\jobname-pw.ind}{\input{\jobname-pw.ind}}{}

\end{document}

      