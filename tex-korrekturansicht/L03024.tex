%% latex-korrekturansicht-vorspann.tex
%% Vorspann für die Korrekturansicht.
%% Lädt die gemeinsame Datei latex-vorspann.tex mit gesetztem Schalter.

\newif\ifkorrekturansicht
\korrekturansichttrue

\input{../tex-inputs/latex-vorspann}


\renewcommand{\erwaehntePersonen}{Personen: Lili Cappellini, Arnoldo Cappellini, Felix Salten}
\renewcommand{\erwaehnteInstitutionen}{Institutionen: Stella d’Italia}
\renewcommand{\erwaehnteOrte}{Orte: Athen, Istanbul, Rhodos, Triest, Wien}
\renewcommand{\erwaehnteWerke}{Werke: Quer durch den Wurstelprater}
\section[ Arthur Schnitzler an Felix Salten, 11. 4. 1928]{Arthur Schnitzler an Felix Salten, 11. 4. 1928}
\nopagebreak\mylabel{v}
\rehead{ }\normalsize\beginnumbering\briefempfaengerindex{Salten, Felix@\textsc{Salten, Felix}!zzzSchnitzler, Arthur@\emph{von Arthur Schnitzler}!1928-04-112@{11. 4. 1928}|(be}
\toendnotes[C]{\smallbreak\pagebreak[2]}\Standort{Wienbibliothek im Rathaus, ZPH 1681, 2.1.516.}
\physDesc{Brief, 1 Blatt, 2 Seiten, 702 Zeichen
\newline{}Handschrift: Bleistift, lateinische Kurrent
\newline{}Ordnung: mit Bleistift von unbekannter Hand nummeriert: »2« }
\buchAbdrucke{\weitereDrucke{Arthur Schnitzler: \emph{Briefe 1913–1931}. Hg. Peter Michael Braunwarth, Richard Miklin, Susanne Pertlik und Heinrich Schnitzler. Frankfurt am Main: \emph{S. Fischer} 1984, S. 541–542.} }\toendnotes[C]{\smallbreak}
\pstart
           \raggedleft{}{\pb}\textcolor{pink}{Wien}{}\ledrightnote{\textcolor{pink}{Wien}}{ }11. 4. 928\pend
           
\pstart
           lieber, der \label{K_L03024-1v}\edtext{\textcolor{green}{Schrei der Liebe}{}\ledrightnote{\textcolor{green}{Quer durch den Wurstelprater}}}{\lemma{\textnormal{\emph{Schrei der Liebe}}}\Cendnote{\textnormal{vgl. Felix Salten: Widmungsexemplar Der Schrei der Liebe für Arthur
               Schnitzler, Juli 1928}}}\label{K_L03024-1h} ist vorläufg unauffindbar – (ich merke eben, dſs mir auch der \label{K_L03024-2v}\edtext{\textcolor{green}{Wurstlprater}{}\ledrightnote{\textcolor{green}{Quer durch den Wurstelprater}}}{\lemma{\textnormal{\emph{Wurstlprater}}}\Cendnote{\textnormal{vgl. Felix Salten: Widmungsexemplar Wurstelprater für Arthur
               Schnitzler, 12. 12. 1911}}}\label{K_L03024-2h} verschwunden ist) – doch steht ein großes Reinmachen und Bücherklopfen bevor
               – da wird er sich hoffentlich finden. Und we{\geminationn} da nicht,
               im Mai, wo neue Regale kommen und ich überhaupt eine
               »ordentliche Ordnung« machen will. Ich zweifle nicht, daſs die \textcolor{green}{Bücher}{}\ledrightnote{\textcolor{green}{Quer durch den Wurstelprater}{\newline}\textcolor{green}{Quer durch den Wurstelprater}} in meiner Bibliothek vorhanden sind, de{\geminationn} Widmungsexemplare, und gar von Ihnen, leih ich nicht
               her.\pend
           
\pstart
           Morgen fahr ich nach \textcolor{pink}{Triest}{}\ledrightnote{\textcolor{pink}{Triest}}, und Samstag mit der \textcolor{brown}{Stella d’Italia}{}\ledrightnote{\textcolor{brown}{Stella d’Italia}}{ }{\pb}in Begleitung von \textcolor{blue}{Lili}{}\ledrightnote{\textcolor{blue}{Lili Cappellini}} und ihrem \textcolor{blue}{Gatten}{}\ledrightnote{{$\rightarrow$}\textcolor{blue}{Arnoldo Cappellini}} über \textcolor{pink}{Athen}{}\ledrightnote{\textcolor{pink}{Athen}} – \textcolor{pink}{Konstantinopel}{}\ledrightnote{\textcolor{pink}{Istanbul}} und zurück (über \textcolor{pink}{Rhodus}{}\ledrightnote{\textcolor{pink}{Rhodos}}, das es also zu geben scheint.)\pend
           
\pstart
           Auf ein gutes \label{K_L03024-3v}\edtext{Wiedersehn im Mai}{\lemma{\textnormal{\emph{Wiedersehn im Mai}}}\Cendnote{\textnormal{Das nächste nachweisliche Treffen fand
                  am 18. 5. 1928
                  statt.}}}\label{K_L03024-3h}, u alles herzliche bis dahin {\\[\baselineskip]}Ihr {\\[\baselineskip]}\spacefill\mbox{Arth}\pend
           \leftskip=0em{}\endnumbering\briefempfaengerindex{Salten, Felix@\textsc{Salten, Felix}!zzzSchnitzler, Arthur@\emph{von Arthur Schnitzler}!1928-04-112@{11. 4. 1928}|)be}\mylabel{h}  \normalsize

\doendnotes{C}
\bigskip
\vfill

\clearpage

\footnotesize

\lohead{\textsc{register}}

% Definiere theindex-Environment komplett neu ohne reledmac
\makeatletter
\renewenvironment{theindex}{%
  \section*{\indexname}%
  \setlength{\parindent}{0pt}%
  \setlength{\parskip}{0pt plus 0.3pt}%
  \let\item\@idxitem
}{%
  \clearpage
}
\makeatother

\IfFileExists{\jobname-pw.ind}{\input{\jobname-pw.ind}}{}

\end{document}

      