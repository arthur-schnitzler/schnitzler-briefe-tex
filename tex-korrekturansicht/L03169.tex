%% latex-korrekturansicht-vorspann.tex
%% Vorspann für die Korrekturansicht.
%% Lädt die gemeinsame Datei latex-vorspann.tex mit gesetztem Schalter.

\newif\ifkorrekturansicht
\korrekturansichttrue

\input{../tex-inputs/latex-vorspann}


\renewcommand{\erwaehntePersonen}{Personen: Peter Altenberg, Richard Beer-Hofmann, Paula Beer-Hofmann, Julius von Gans-Ludassy, Sigmund Hahn, Rudolf Lothar, Fritz Mauthner, Wilhelmine Mitterwurzer, Friedrich Mitterwurzer, Ottilie Salten, Adele Sandrock, Paul von Schönthan-Pernwald}
\renewcommand{\erwaehnteInstitutionen}{Institutionen: Burgtheater, Deutsches Theater Berlin, Frankfurter Städtisches Schauspielhaus, Lessing-Theater, Mährisches Theater Olmütz, Neues Wiener Tagblatt, Wiener Allgemeine Zeitung}
\renewcommand{\erwaehnteOrte}{Orte: Berlin, Brünn, Burgtheater, Frankfurt am Main, Olomouc, Prag, Theater an der Wien, Theater in der Josefstadt, Wien}
\renewcommand{\erwaehnteWerke}{Werke: Berliner Tageblatt, Der Dornenweg, Der zerbrochene Krug, Der zerbrochene Krug im Deutschen Theater, Deutsches Theater, Gelegenheitskauf, Liebelei. Schauspiel in drei Akten, Matinée, Mährisches Tagblatt, Neues Wiener Tagblatt, Tagebuch, Theater und Kunst [Liebelei am Deutschen Theater], Wiener Allgemeine Zeitung, Wilhelmine Mitterwurzer, »Liebelei«. Schauspiel in 3 Acten von Arthur Schnitzler}
\section[ Felix Salten an Arthur Schnitzler, {[}8. 2. 1896{]}]{Felix Salten an Arthur Schnitzler, {[}8. 2. 1896{]}}
\nopagebreak\mylabel{v}
\rehead{ }\normalsize\beginnumbering\briefempfaengerindex{Schnitzler, Arthur@\textsc{Schnitzler, Arthur}!zzzSalten, Felix@\emph{von Felix Salten}!1896-02-081@{{[}8. 2. 1896{]}}|(be}
\toendnotes[C]{\smallbreak\pagebreak[2]}\Standort{CUL, Schnitzler, B 89, A 1.}
\physDesc{Brief, 1 Blatt, 4 Seiten, 2241 Zeichen
\newline{}Handschrift: Bleistift, lateinische Kurrent
\newline{}Schnitzler: mit Bleistift datiert: »8/2 96« 
\newline{}Ordnung: mit Bleistift von unbekannter Hand nummeriert: »68.« }\toendnotes[C]{\smallbreak}
\pstart
           \raggedleft{}{\pb}Samstag.\pend
           
\pstart
           Lieber Freund, Nachtredacteur beim \textcolor{brown}{Neuen Wiener Tagblatt}{}\ledrightnote{\textcolor{brown}{Neues Wiener Tagblatt}} ist ein Herr \label{K_L03169-1v}\edtext{\textcolor{blue}{Sigmund Hahn}{}\ledrightnote{\textcolor{blue}{Sigmund Hahn}}}{\lemma{\textnormal{\emph{Sigmund Hahn}}}\Cendnote{\textnormal{\textcolor{blue}{Schnitzler} hielt sich in \textcolor{pink}{Berlin} auf, wo am 4. 2. 1896 am \emph{\textcolor{brown}{Deutschen Theater}} die gemeinsame Premieren von \emph{\textcolor{green}{Liebelei}} und \emph{\textcolor{green}{Der
                     zerbrochene Krug}} stattfanden. \textcolor{blue}{Schnitzler} erwähnt sowohl das Studium der Nachtkritiken (5. 2. 1896) wie auch
                  die Feuilletons (6. 2. 1896) in seinem \emph{\textcolor{green}{Tagebuch}}.
                  Hier dürfte er der \textcolor{green}{Notiz}
                  im Abendblatt des \emph{\textcolor{green}{Neuen Wiener Tagblatt}}s:
                     [O. V.]: \emph{\textcolor{green}{Theater und Kunst}}. In: \emph{\textcolor{green}{Neues Wiener Abendblatt. Abend-Ausgabe des
                        »Neuen Wiener Tagblatt«}}, Jg. 30, Nr. 35, 5. 2 1896, S. 3 nachgeforscht haben.}}}\label{K_L03169-1h}, von dem ich aber
               garnichts weiss. \label{K_L03169-2v}\edtext{\textcolor{pink}{Berlin}{}\ledrightnote{\textcolor{pink}{Berlin}} hat mir viele Freude}{\lemma{\textnormal{\emph{Berlin … Freude}}}\Cendnote{\textnormal{\textcolor{blue}{Salten} zeigt sich erfreut darüber, dass die
                     \textcolor{pink}{Berlin}er Inszenierung der \emph{\textcolor{green}{Liebelei}} in der (\textcolor{pink}{Wien}er) Presse viel und positiv besprochen wurde.}}}\label{K_L03169-2h} gemacht,
                  \textcolor{gray}{–} das war sehr hübsch und hat \textcolor{pink}{hier}{}\ledrightnote{{$\rightarrow$}\textcolor{pink}{Wien}} gut gewirkt. \textcolor{blue}{Ludaßy}{}\ledrightnote{\textcolor{blue}{Julius von Gans-Ludassy}} verhält mich zu einer Revue über Ihre \textcolor{pink}{Berlin}{}\ledrightnote{\textcolor{pink}{Berlin}}er u. \label{K_L03169-3v}\edtext{\textcolor{pink}{Frankfurt}{}\ledrightnote{\textcolor{pink}{Frankfurt am Main}}er Erfolge}{\lemma{\textnormal{\emph{Frankfurter Erfolge}}}\Cendnote{\textnormal{Die \emph{\textcolor{green}{Liebelei}} wurde seit
                     11. 1. 1896 auch
                  in \textcolor{pink}{Frankfurt am Main} am \emph{\textcolor{brown}{Städtischen Schauspielhaus}} gegeben.}}}\label{K_L03169-3h}, – wenn die
               Leute was reden, schieb ich es ihm auch zu. Trotzdem sind wir eine Clique. \label{K_L03169-4v}\edtext{Glauben Sie bei \textcolor{green}{\textcolor{blue}{Fritz Mauthner}{}\ledrightnote{\textcolor{blue}{Fritz Mauthner}}}{}\ledrightnote{{$\rightarrow$}\textcolor{green}{Deutsches Theater}} wirklich an \textcolor{blue}{Lothar}{}\ledrightnote{\textcolor{blue}{Rudolf Lothar}}}{\lemma{\textnormal{\emph{Glauben … Lothar}}}\Cendnote{\textnormal{Also ob \textcolor{blue}{Rudolf Lothar}{ }\textcolor{blue}{Fritz Mauthner} mit Stichworten versorgt
                  hatte. Von \textcolor{blue}{Mauthner} erschienen zwei Texte im
                     \emph{\textcolor{green}{Berliner Tageblatt}}: \textcolor{blue}{Fr. M.} [ = \textcolor{blue}{Fritz Mauthner}]: \emph{\textcolor{green}{Deutsches Theater}}. In: \emph{\textcolor{green}{Berliner Tageblatt}}, Jg. 25, Nr. 64, 5. 2. 1896, Morgen-Ausgabe, S. 2–3; \textcolor{blue}{Fr. M.} [ = \textcolor{blue}{Fritz Mauthner}]: \emph{\textcolor{green}{Der zerbrochene Krug im Deutschen Theater}}. In: \emph{\textcolor{green}{Berliner Tageblatt}}, Jg. 25, Nr. 65, 5. 2. 1896, Abend-Ausgabe, S. 1–2. }}}\label{K_L03169-4h}?
               In \label{K_L03169-5v}\edtext{\textcolor{pink}{Olmütz}{}\ledrightnote{\textcolor{pink}{Olomouc}} haben Sie einen großen Erfolg}{\lemma{\textnormal{\emph{Olmütz … Erfolg}}}\Cendnote{\textnormal{Am 30. 1. 1896 hatte am \emph{\textcolor{brown}{Königlich-Städtischem Theater zu Olmütz}} die Premiere von \emph{\textcolor{green}{Liebelei}} stattgefunden.}}}\label{K_L03169-5h} gehabt, – sonst sind Sie
               weder in \textcolor{pink}{Brünn}{}\ledrightnote{\textcolor{pink}{Brünn}} noch in \textcolor{pink}{Prag}{}\ledrightnote{\textcolor{pink}{Prag}} gewesen, das \textcolor{green}{Mährische
                  Tagblatt}{}\ledrightnote{\textcolor{green}{Mährisches Tagblatt}} heb’ ich Ihnen auf, – die \label{K_L03169-6v}\edtext{\textcolor{green}{Kritik}{}\ledrightnote{{$\rightarrow$}\textcolor{green}{»Liebelei«. Schauspiel in 3 Acten von Arthur Schnitzler}}}{\lemma{\textnormal{\emph{Kritik}}}\Cendnote{\textnormal{[O. V.]: \emph{\textcolor{green}{»Liebelei«. Schauspiel in 3 Acten
                        von Arthur Schnitzler}}. In: \emph{\textcolor{green}{Mährisches
                        Tagblatt}}, Jg. 17, Nr. 25, 31. 1. 1896,
                     S. 5–6.}}}\label{K_L03169-6h} ist köstlich.\pend
           
\pstart
           \textcolor{pink}{Hier}{}\ledrightnote{{$\rightarrow$}\textcolor{pink}{Wien}} ist ein wunderschönes
               Frühlingswetter, das alle guten Vorsätze hervor{\pb}treibt und gute Laune
               schafft. Zudem habe ich noch \label{K_L03169-7v}\edtext{Frl. \textcolor{blue}{M.}{}\ledrightnote{\textcolor{blue}{Ottilie Salten}}}{\lemma{\textnormal{\emph{Frl. M.}}}\Cendnote{\textnormal{\textcolor{blue}{Ottilie Metzl}, \textcolor{blue}{Salten}s spätere Ehefrau}}}\label{K_L03169-7h} – Neulich, es war Dienstag, erzählt sie mir, sie habe Alles der Frau \textcolor{blue}{Mitterwurzer}{}\ledrightnote{\textcolor{blue}{Wilhelmine Mitterwurzer}} gesagt. Diese sei sehr erschrocken
               und habe ihr dringend gerathen, den Verkehr mit mir aufzugeben. Darauf entgegnete
               Frl. \textcolor{blue}{M.}{}\ledrightnote{\textcolor{blue}{Ottilie Salten}} sie könne das nicht, und Frau \textcolor{blue}{Mitterw.}{}\ledrightnote{\textcolor{blue}{Wilhelmine Mitterwurzer}} wünschte dann mich wenigstens kennen
               zu lernen. »\uline{Sie} wird mich gleich durch und durch
               schauen?« Natürlich. Sie will mich auch einladen und wir wollen uns bei ihr oben
               sehen. Tags darauf komme ich in die \textcolor{brown}{Redaction}{}\ledrightnote{{$\rightarrow$}\textcolor{brown}{Wiener Allgemeine Zeitung}} und erfahre, dass ich sogleich
               ein \label{K_L03169-8v}\edtext{\textcolor{green}{Feuilleton}{}\ledrightnote{{$\rightarrow$}\textcolor{green}{Wilhelmine Mitterwurzer}}}{\lemma{\textnormal{\emph{Feuilleton}}}\Cendnote{\textnormal{\textcolor{blue}{f. s.} [ = \textcolor{blue}{Felix Salten}]: \emph{\textcolor{green}{Wilhelmine Mitterwurzer}}. In: \emph{\textcolor{green}{Wiener
                        Allgemeine Zeitung}}, Nr. 5.382, 6. 2. 1896, S. 3.}}}\label{K_L03169-8h} schreiben muss – über Frau \textcolor{blue}{Mitterwurzer}{}\ledrightnote{\textcolor{blue}{Wilhelmine Mitterwurzer}} – das Leben, – \substVorne{}\textsuperscript{s}\substDazwischen{}S\substHinten{}ie wissen schon.\pend
           
\pstart
           \textcolor{blue}{Richard}{}\ledrightnote{\textcolor{blue}{Richard Beer-Hofmann}} ist sehr lieb, war neu{\pb}lich mit seinem \label{K_L03169-9v}\edtext{\textcolor{blue}{Mäderl}{}\ledrightnote{{$\rightarrow$}\textcolor{blue}{Paula Beer-Hofmann}}}{\lemma{\textnormal{\emph{Mäderl}}}\Cendnote{\textnormal{\textcolor{blue}{Paula Lissy}, \textcolor{blue}{Beer-Hofmann}s spätere Ehefrau}}}\label{K_L03169-9h} im \textcolor{pink}{Josefstädter Theater}{}\ledrightnote{\textcolor{pink}{Theater in der Josefstadt}}, und ist stolz darauf. \textcolor{blue}{Engländer}{}\ledrightnote{\textcolor{blue}{Peter Altenberg}} war dabei, und erklärt sie natürlich für das
               Höchste.\pend
           
\pstart
           Sonntag war ich bei der \label{K_L03169-10v}\edtext{Matinée}{\lemma{\textnormal{\emph{Matinée}}}\Cendnote{\textnormal{\textcolor{blue}{Salten} hatte eine kurze Rezension verfasst: \textcolor{blue}{f.} [ = \textcolor{blue}{Felix Salten}]: \emph{\textcolor{green}{Matinée}}. In: \emph{\textcolor{green}{Wiener Allgemeine Zeitung}}, Nr. 5.380, 4. 2. 1896, S. 4.}}}\label{K_L03169-10h} im \textcolor{pink}{Theater auf der Wien}{}\ledrightnote{\textcolor{pink}{Theater an der Wien}} fortwährend auf der Bühne. \textcolor{blue}{Mitterwurzer}{}\ledrightnote{\textcolor{blue}{Friedrich Mitterwurzer}} rief nach \textcolor{green}{Akt}{}\ledrightnote{{$\rightarrow$}\textcolor{green}{Gelegenheitskauf}}schluss das Frl. \textcolor{blue}{M.}{}\ledrightnote{\textcolor{blue}{Ottilie Salten}} sie solle mit ihm herauskommen, sich verbeugen, – sie
               wollte nicht, der schrie ihr nach: »Frl. \textcolor{blue}{Sandrock}{}\ledrightnote{\textcolor{blue}{Adele Sandrock}}{ }Frl. \textcolor{blue}{Sandrock}{}\ledrightnote{\textcolor{blue}{Adele Sandrock}}!« und als \label{K_L03169-11v}\edtext{sie}{\lemma{\textnormal{\emph{sie}}}\Cendnote{\textnormal{korrigiert aus »Sie«}}}\label{K_L03169-11h}
               ihn darauf aufmerksam machte, wurde er tobsüchtig. Von Frl. \textcolor{blue}{S.}{}\ledrightnote{\textcolor{blue}{Adele Sandrock}} sind Kleinigkeiten zu berichten{\dotstwo} Ich befand mich ungeheuer wol und daheim auf der Bühne, und hab an Sie gedacht.
                  \textcolor{blue}{P. v. Schönthan}{}\ledrightnote{\textcolor{blue}{Paul von Schönthan-Pernwald}} ging umher, und erzählte
               den Schauspielern, dass er dieses \textcolor{green}{Stück}{}\ledrightnote{{$\rightarrow$}\textcolor{green}{Gelegenheitskauf}} mit seinem Herzblut geschrieben, – man überschätzt die Leute noch
               immer. Der \textcolor{green}{Gelegenheits{\pb}kauf}{}\ledrightnote{\textcolor{green}{Gelegenheitskauf}}{ }ist übrigens im \textcolor{brown}{Burgtheater}{}\ledrightnote{\textcolor{brown}{Burgtheater}} und im \textcolor{brown}{Lessingtheater}{}\ledrightnote{\textcolor{brown}{Lessing-Theater}} angenommen.\pend
           
\pstart
           Eben kommt das Repertoire. Sie sind in dieser Woche nicht drauf, was auch erklärlich
                  ist{[}.{]}{ }Dienstag kommt der \textcolor{green}{Dornenweg}{}\ledrightnote{\textcolor{green}{Der Dornenweg}}. Da sind Sie ja bis Abends da, und \label{K_L03169-12v}\edtext{im \textcolor{pink}{Theater}{}\ledrightnote{{$\rightarrow$}\textcolor{pink}{Burgtheater}}}{\lemma{\textnormal{\emph{im Theater}}}\Cendnote{\textnormal{siehe A. S.: \emph{Tagebuch}, 11. 2. 1896}}}\label{K_L03169-12h}.\pend
           
\pstart
           Herzlichst Ihr {\\[\baselineskip]}\spacefill\mbox{Salten}\pend
           \leftskip=0em{}\endnumbering\briefempfaengerindex{Schnitzler, Arthur@\textsc{Schnitzler, Arthur}!zzzSalten, Felix@\emph{von Felix Salten}!1896-02-081@{{[}8. 2. 1896{]}}|)be}\mylabel{h}  \normalsize

\doendnotes{C}
\bigskip
\vfill

\clearpage

\footnotesize

\lohead{\textsc{register}}

% Definiere theindex-Environment komplett neu ohne reledmac
\makeatletter
\renewenvironment{theindex}{%
  \section*{\indexname}%
  \setlength{\parindent}{0pt}%
  \setlength{\parskip}{0pt plus 0.3pt}%
  \let\item\@idxitem
}{%
  \clearpage
}
\makeatother

\IfFileExists{\jobname-pw.ind}{\input{\jobname-pw.ind}}{}

\end{document}

      