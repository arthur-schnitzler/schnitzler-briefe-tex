%% latex-korrekturansicht-vorspann.tex
%% Vorspann für die Korrekturansicht.
%% Lädt die gemeinsame Datei latex-vorspann.tex mit gesetztem Schalter.

\newif\ifkorrekturansicht
\korrekturansichttrue

\input{../tex-inputs/latex-vorspann}


               \section[Arthur Schnitzler an Hugo von Hofmannsthal, 24. 3. 1899]{ Arthur Schnitzler an Hugo von Hofmannsthal, 24. 3. 1899}\nopagebreak\mylabel{v}\rehead{ }\normalsize\beginnumbering\briefempfaengerindex{Hofmannsthal, Hugo von@\textsc{Hofmannsthal, Hugo von}!zzzSchnitzler, Arthur@\emph{von Arthur Schnitzler}!1899-03-241@{{[}24. 3. 1899{]}}|(be} \toendnotes[C]{\smallbreak\pagebreak[2]} \Standort{FDH, Hs-30885,81.}
\physDesc{Brief, 1 Blatt, 3 Seiten
\newline{}Handschrift: Bleistift, deutsche Kurrent}\buchAbdrucke{\weitereDrucke{Hugo von Hofmannsthal, Arthur Schnitzler: \emph{Briefwechsel}. Hg. Therese Nickl und Heinrich Schnitzler. Frankfurt am Main: \emph{S. Fischer} 1964, S. 121.} }\toendnotes[C]{\smallbreak}\pstart
           \raggedleft{}{\pb}24/3 99\pend
           \pstart
           mein lieber Hugo, we{\geminationn} ich früher
                    nach \textcolor{pink}{Berlin}{}\ledrightnote{\textcolor{pink}{Berlin}} fahre, ſo doch erſt
                        Oſtern, mit meinem \textcolor{blue}{Bruder}{}\ledrightnote{→\textcolor{blue}{Julius Schnitzler}} (\uline{\textcolor{brown}{Chirurgencongreſs}{}\ledrightnote{\textcolor{brown}{28. Congress der deutschen Gesellschaft für Chirurgie}}}). Sagen Sie mir, wa{\geminationn} Sie wieder nach \textcolor{pink}{Wien}{}\ledrightnote{\textcolor{pink}{Wien}} kommen. Vielleicht fahr ich morgen nach \textcolor{pink}{Graz}{}\ledrightnote{\textcolor{pink}{Graz}}, dort ſind jetzt \textcolor{blue}{ihre}{}\ledrightnote{→\textcolor{blue}{Marie Reinhard}}{ }\textcolor{blue}{Eltern}{}\ledrightnote{→\textcolor{blue}{Carl Reinhard}{\newline}→\textcolor{blue}{Therese Reinhard}}. Es brennt
                    in mir weiter, ganz wie we{\geminationn} alles von dem {\pb}tobenden Schmerz aufgefreſſen werden ſollte. Nie nie
                    verſteht man es.\pend
           \pstart
           Sie machen ſich doch nichts daraus, dſs Ihre \textcolor{green}{Stücke}{}\ledrightnote{→\textcolor{green}{Die Hochzeit der Sobeide}{\newline}→\textcolor{green}{Der Abenteurer und die Sängerin oder Die Geschenke des Lebens}} in \textcolor{pink}{B.}{}\ledrightnote{\textcolor{pink}{Berlin}} nicht
                    gegangen ſind; hoff ich.\pend
           \pstart
           Wie ſoll das mit \textcolor{green}{meinen}{}\ledrightnote{→\textcolor{green}{Der grüne Kakadu – Paracelsus – Die Gefährtin. Drei Einakter}} in
                        \textcolor{pink}{B.}{}\ledrightnote{\textcolor{pink}{Berlin}} werden. Jeder Satz iſt beinah eine
                    gemeinſchaftliche Erinnerung – wie jeder Gedanke dieſer vier {\pb}Jahre, wie jedes Haus, jeder Stein, jeder Menſch in
                        \textcolor{pink}{Wien}{}\ledrightnote{\textcolor{pink}{Wien}}; wie meine ganze Existenz. –\pend
           \pstart
           Schreiben Sie mir bitte wie Sie leben, wen Sie ſehen.\pend
           \pstart
           Ihr \textcolor{blue}{Vater}{}\ledrightnote{→\textcolor{blue}{Hugo August von Hofmannsthal}} war bei mir, ich
                    aber nicht zu Haus. Viel bin ich mit \textcolor{blue}{Guſt.
                        Schw.}{}\ledrightnote{\textcolor{blue}{Gustav Schwarzkopf}} zuſa{\geminationm}en, auch mit \textcolor{blue}{Richard}{}\ledrightnote{\textcolor{blue}{Richard Beer-Hofmann}}, \textcolor{blue}{Salten}{}\ledrightnote{\textcolor{blue}{Felix Salten}}.\pend
           \pstart
           Von Herzen Ihr{\\[\baselineskip]}\spacefill\mbox{Arth}\pend
           \leftskip=0em{}\endnumbering\briefempfaengerindex{Hofmannsthal, Hugo von@\textsc{Hofmannsthal, Hugo von}!zzzSchnitzler, Arthur@\emph{von Arthur Schnitzler}!1899-03-241@{{[}24. 3. 1899{]}}|)be}\mylabel{h}  \normalsize

\doendnotes{C}
\bigskip
\vfill

\clearpage

\footnotesize

\lohead{\textsc{register}}

% Definiere theindex-Environment komplett neu ohne reledmac
\makeatletter
\renewenvironment{theindex}{%
  \section*{\indexname}%
  \setlength{\parindent}{0pt}%
  \setlength{\parskip}{0pt plus 0.3pt}%
  \let\item\@idxitem
}{%
  \clearpage
}
\makeatother

\IfFileExists{\jobname-pw.ind}{\input{\jobname-pw.ind}}{}

\end{document}

      