%% latex-korrekturansicht-vorspann.tex
%% Vorspann für die Korrekturansicht.
%% Lädt die gemeinsame Datei latex-vorspann.tex mit gesetztem Schalter.

\newif\ifkorrekturansicht
\korrekturansichttrue

\input{../tex-inputs/latex-vorspann}


\renewcommand{\erwaehnteOrte}{Orte: Frankgasse, Gemmipass, Leukerbad, Schweiz, Tirol, Wien}
\renewcommand{\erwaehnteWerke}{}
\section[ Paul Goldmann an Arthur Schnitzler, 16. 8. 1902]{Paul Goldmann an Arthur Schnitzler, 16. 8. 1902}
\nopagebreak\mylabel{v}
\rehead{ }\normalsize\beginnumbering\briefempfaengerindex{Schnitzler, Arthur@\textsc{Schnitzler, Arthur}!zzzGoldmann, Paul@\emph{von Paul Goldmann}!1902-08-162@{16. 8. 1902}|(be}
\toendnotes[C]{\smallbreak\pagebreak[2]}\Standort{DLA, A:Schnitzler, HS.NZ85.1.3172.}
\physDesc{Bildpostkarte
\newline{}Handschrift: 1) schwarze Tinte, deutsche Kurrent\hspace{1em}2) schwarze Tinte, lateinische Kurrent (\noindent{}Adresse)\hspace{1em}
\newline{}Versand: 1) Stempel: »\nobreak{}\oindex{Leukerbad@\textbf{Leukerbad}, \emph{https://www.geonames.org/ontologyP.PPL}|pwk}Leuk-Bad Loëche-Bains, 16. VIII. 02\nobreak{}«.   2) Stempel: »\nobreak{}9/3 Wien 72, 18. 8. 02, 11. V, Bestellt\nobreak{}«. }\toendnotes[C]{\smallbreak}\pstart{}{\pb}Herrn\pend{}\pstart{}Dr. Arthur Schnitzler\pend{}\pstart{}\textcolor{pink}{Wien}{}\ledrightnote{\textcolor{pink}{Wien}}\pend{}\pstart{}\textcolor{pink}{IX. Frankgaſse 1}{}\ledrightnote{\textcolor{pink}{Frankgasse}}.\pend{}
{\bigskip}
\pstart
           \noindent{}{\pb}\textcolor{gray}{\textbf{\textbf{\textcolor{pink}{Gemmi}{}\ledrightnote{\textcolor{pink}{Gemmipass}}}. Passhöhe 2329 m.}}\pend
           
\pstart
           16. Auguſt.\pend
           
\pstart
           Herzlichſte Grüße! Nächſtes Jahr mußt Du auch nach der \label{K_L03220-1v}\edtext{\textcolor{pink}{Schweiz}{}\ledrightnote{\textcolor{pink}{Schweiz}}}{\lemma{\textnormal{\emph{Schweiz}}}\Cendnote{\textnormal{nicht geschehen}}}\label{K_L03220-1h}. Das iſt viel
               großartiger a\textcolor{gray}{l}s \textcolor{pink}{Tirol}{}\ledrightnote{\textcolor{pink}{Tirol}}.\pend
           \pstart Dein \spacefill\mbox{Paul Goldmnn }\pend{}\endnumbering\briefempfaengerindex{Schnitzler, Arthur@\textsc{Schnitzler, Arthur}!zzzGoldmann, Paul@\emph{von Paul Goldmann}!1902-08-162@{16. 8. 1902}|)be}\mylabel{h}
\begin{anhang}
\end{anhang}\normalsize

\doendnotes{C}
\bigskip
\vfill

\clearpage

\footnotesize

\lohead{\textsc{register}}

% Definiere theindex-Environment komplett neu ohne reledmac
\makeatletter
\renewenvironment{theindex}{%
  \section*{\indexname}%
  \setlength{\parindent}{0pt}%
  \setlength{\parskip}{0pt plus 0.3pt}%
  \let\item\@idxitem
}{%
  \clearpage
}
\makeatother

\IfFileExists{\jobname-pw.ind}{\input{\jobname-pw.ind}}{}

\end{document}

      