%% latex-korrekturansicht-vorspann.tex
%% Vorspann für die Korrekturansicht.
%% Lädt die gemeinsame Datei latex-vorspann.tex mit gesetztem Schalter.

\newif\ifkorrekturansicht
\korrekturansichttrue

\input{../tex-inputs/latex-vorspann}


               \section[Arthur Schnitzler an Georg Brandes, 11. 2. 1925]{ Arthur Schnitzler an Georg Brandes, 11. 2. 1925}\nopagebreak\mylabel{v}\rehead{ }\normalsize\beginnumbering\briefempfaengerindex{Brandes, Georg@\textsc{Brandes, Georg}!zzzSchnitzler, Arthur@\emph{von Arthur Schnitzler}!1925-02-111@{11. 2. 1925}|(be} \toendnotes[C]{\smallbreak\pagebreak[2]} \Standort{Kopenhagen, Det Kongelige Bibliotek, Georg Brandes Arkiv, box 125.}
\physDesc{Brief, 1 Blatt, 2 Seiten
\newline{}Handschrift: schwarze Tinte, lateinische Kurrent\newline{}Ordnung: 1) mit Bleistift von unbekannter Hand nummeriert: »\strikeout{49}« 2) mit Bleistift von unbekannter Hand nummeriert: »50.«}\buchAbdrucke{\weitereDrucke{Georg Brandes, Arthur Schnitzler: \emph{Ein Briefwechsel}. Hg. Kurt Bergel. Bern: \emph{Francke} 1956, S. 143–144.} }\toendnotes[C]{\smallbreak}\pstart
           \raggedleft{}{\pb}\textcolor{pink}{Wien}{}\ledrightnote{\textcolor{pink}{Wien}}, 11. 2. 1925\pend
           \pstart{}lieber und verehrter Freund,\pend\pstart
           ich lese, und mein \textcolor{blue}{Sohn}{}\ledrightnote{→\textcolor{blue}{Heinrich Schnitzler}}
                    schreibt mir, dſs Sie im Laufe des März nach \textcolor{pink}{Berlin}{}\ledrightnote{\textcolor{pink}{Berlin}} kommen wollen. Ich hatte die gleiche Absicht; und
                    wäre nun sehr froh, we{\geminationn} ich Ihnen dort begegnen
                    dürfte. Sind Sie sich über den Termin Ihrer Reise schon klar? Wollten Sie mir
                    darüber so \uline{bald als möglich} ein Wort schreiben,
                    wär ich Ihnen von Herzen dankbar.\pend
           \pstart
           In der \textcolor{pink}{Schweiz}{}\ledrightnote{\textcolor{pink}{Schweiz}} (Vortragsreise und nachheriger
                    Aufenthalt im \textcolor{pink}{Engadin}{}\ledrightnote{\textcolor{pink}{Engadin}}) {\pb}hatte ich einen kurzen Bericht über Sie
                    durch Dr. \textcolor{blue}{Zbinden}{}\ledrightnote{\textcolor{blue}{Hans Zbinden}}, der Sie damals in \textcolor{pink}{Kopenhagen}{}\ledrightnote{\textcolor{pink}{Kopenhagen}} etwas leidend angetroffen hatte.
                    Nun gehts Ihnen hoffentlich wieder ganz gut. Mir auch ganz leidlich. Manche \strikeout{g} schöne Abendstunde verbring ich mit Ihren
                    Büchern, den neuen und den alten. Jetzt bin ich wieder einmal in der »Romantik«
                    der \textcolor{green}{Hauptströmungen}{}\ledrightnote{\textcolor{green}{Hauptströmungen der Literatur des neunzehnten Jahrhunderts}}.\pend
           \pstart
           Also bitte, schreiben Sie mir gleich ein Wort.\pend
           \pstart
           Sie von Herzen grüßend{\\[\baselineskip]}Ihr{\\[\baselineskip]}\spacefill\mbox{Arthur Schnitzler}\pend
           \leftskip=0em{}\endnumbering\briefempfaengerindex{Brandes, Georg@\textsc{Brandes, Georg}!zzzSchnitzler, Arthur@\emph{von Arthur Schnitzler}!1925-02-111@{11. 2. 1925}|)be}\mylabel{h}  \normalsize

\doendnotes{C}
\bigskip
\vfill

\clearpage

\footnotesize

\lohead{\textsc{register}}

% Definiere theindex-Environment komplett neu ohne reledmac
\makeatletter
\renewenvironment{theindex}{%
  \section*{\indexname}%
  \setlength{\parindent}{0pt}%
  \setlength{\parskip}{0pt plus 0.3pt}%
  \let\item\@idxitem
}{%
  \clearpage
}
\makeatother

\IfFileExists{\jobname-pw.ind}{\input{\jobname-pw.ind}}{}

\end{document}

      