%% latex-korrekturansicht-vorspann.tex
%% Vorspann für die Korrekturansicht.
%% Lädt die gemeinsame Datei latex-vorspann.tex mit gesetztem Schalter.

\newif\ifkorrekturansicht
\korrekturansichttrue

\input{../tex-inputs/latex-vorspann}


         \renewcommand{\erwaehnteOrte}{Orte: Berlin, Breslau, Wien}
         \renewcommand{\erwaehnteWerke}{Werke: Der Schleier der Beatrice. Schauspiel in fünf Akten}
               \section[ Paul Goldmann an Arthur Schnitzler, 2. 12. {[}1900{]}]{Paul Goldmann an Arthur Schnitzler, 2. 12. {[}1900{]}}\nopagebreak\mylabel{v}\rehead{ }\normalsize\beginnumbering\briefempfaengerindex{Schnitzler, Arthur@\textsc{Schnitzler, Arthur}!zzzGoldmann, Paul@\emph{von Paul Goldmann}!1900-12-021@{2. 12. {[}1900{]}}|(be} \toendnotes[C]{\smallbreak\pagebreak[2]} \Standort{DLA, A:Schnitzler, HS.NZ85.1.3170.}
\physDesc{Brief, 1 Blatt, 2 Seiten
\newline{}Handschrift: blaue Tinte, deutsche Kurrent
\newline{}Schnitzler: 1) mit Bleistift das Jahr »{[}1{]}900« vermerkt  2) mit rotem Buntstift eine seitliche Markierung}\toendnotes[C]{\smallbreak}\pstart
           \raggedleft{}{\pb}\textcolor{pink}{Berlin}{}\ledrightnote{\textcolor{pink}{Berlin}}, 2. December.\pend
           \pstart\center{}Mein lieber Freund,\pend\pstart
           Soweit aus den Referaten der \textcolor{pink}{Berlin}{}\ledrightnote{\textcolor{pink}{Berlin}}er Blätter
               klug zu werden iſt, hat die \textcolor{pink}{Breslau}{}\ledrightnote{\textcolor{pink}{Breslau}}er 
                  \textsc{\textcolor{green}{Première}{}\ledrightnote{{$\rightarrow$}\textcolor{green}{Der Schleier der Beatrice. Schauspiel in fünf Akten}}}
                das Refultat gehabt, daß durch die \label{K_L02942-1v}\edtext{ſchlechte Aufführung}{\lemma{\textnormal{\emph{ſchlechte Aufführung}}}\Cendnote{\textnormal{Bezug auf die Uraufführung von \emph{\textcolor{green}{Der Schleier der Beatrice}}, siehe A. S.: \emph{Tagebuch}, 1. 12. 1900}}}\label{K_L02942-1h} hindurch der Werth des \textcolor{green}{Stück}{}\ledrightnote{{$\rightarrow$}\textcolor{green}{Der Schleier der Beatrice. Schauspiel in fünf Akten}}es \strikeout{klar geword} offenbar geworden iſt.
               Somit hat \textcolor{pink}{Breslau}{}\ledrightnote{\textcolor{pink}{Breslau}} ſeine Schuldigkeit gethan\substVorne{}\textsuperscript{. U\textcolor{gray}{n}}\substDazwischen{},\substHinten{} und wir werden das \textcolor{green}{Stück}{}\ledrightnote{{$\rightarrow$}\textcolor{green}{Der Schleier der Beatrice. Schauspiel in fünf Akten}} jetzt wohl \label{K_L02942-3v}\edtext{bald}{\lemma{\textnormal{\emph{bald}}}\Cendnote{\textnormal{siehe Paul Goldmann an Arthur Schnitzler, 23. 12. [1899] und 21. 6. [1900]}}}\label{K_L02942-3h} auf einer großen \textcolor{pink}{Berlin}{}\ledrightnote{\textcolor{pink}{Berlin}}er oder \textcolor{pink}{Wien}{}\ledrightnote{\textcolor{pink}{Wien}}er Bühne ſehen. Ich habe geſtern{ }Abend viel an Dich {\pb}gedacht, und es
               that mir unendlich leid, daß ich nicht bei Dir ſein konnte.\pend
           \pstart
           Viele treue Grüße! {\\[\baselineskip]}Dein {\\[\baselineskip]}\spacefill\mbox{Paul Goldmann.}\pend
           \leftskip=0em{}\endnumbering\briefempfaengerindex{Schnitzler, Arthur@\textsc{Schnitzler, Arthur}!zzzGoldmann, Paul@\emph{von Paul Goldmann}!1900-12-021@{2. 12. {[}1900{]}}|)be}\mylabel{h}  \normalsize

\doendnotes{C}
\bigskip
\vfill

\clearpage

\footnotesize

\lohead{\textsc{register}}

% Definiere theindex-Environment komplett neu ohne reledmac
\makeatletter
\renewenvironment{theindex}{%
  \section*{\indexname}%
  \setlength{\parindent}{0pt}%
  \setlength{\parskip}{0pt plus 0.3pt}%
  \let\item\@idxitem
}{%
  \clearpage
}
\makeatother

\IfFileExists{\jobname-pw.ind}{\input{\jobname-pw.ind}}{}

\end{document}

      