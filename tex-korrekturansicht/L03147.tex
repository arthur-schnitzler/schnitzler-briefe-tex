%% latex-korrekturansicht-vorspann.tex
%% Vorspann für die Korrekturansicht.
%% Lädt die gemeinsame Datei latex-vorspann.tex mit gesetztem Schalter.

\newif\ifkorrekturansicht
\korrekturansichttrue

\input{../tex-inputs/latex-vorspann}


\renewcommand{\erwaehntePersonen}{Personen:  ?? [Medizinstudent], Hermann Bahr, M. J. Mayer}
\renewcommand{\erwaehnteInstitutionen}{Institutionen: Berliner Neueste Nachrichten, Münchener General-Anzeiger}
\renewcommand{\erwaehnteOrte}{Orte: Hörlgasse, Sechsschimmelgasse, Wien}
\renewcommand{\erwaehnteWerke}{}
\section[ Felix Salten an Arthur Schnitzler, 15. 9. 189{[}4?{]}]{Felix Salten an Arthur Schnitzler, 15. 9. 189{[}4?{]}}
\nopagebreak\mylabel{v}
\rehead{ }\normalsize\beginnumbering\briefempfaengerindex{Schnitzler, Arthur@\textsc{Schnitzler, Arthur}!zzzSalten, Felix@\emph{von Felix Salten}!1894-09-152@{15. 9. 189{[}4?{]}}|(be}
\toendnotes[C]{\smallbreak\pagebreak[2]}\Standort{CUL, Schnitzler, B 89, A 1.}
\physDesc{Visitenkarte, 296 Zeichen
\newline{}Handschrift: schwarze Tinte, lateinische Kurrent
\newline{}Schnitzler: 1) mit Bleistift beschriftet: »\noindent{}\textsc{Herr}{ }\textcolor{blue}{\textsc{M. J. Mayer}}.{ / }\textsc{\textcolor{pink}{Währ. Sechsschg. 4}\hspace*{1em}3. St.
                                             Th. 1\textcolor{gray}{4}}«  2) mit Bleistift datiert: »15/9 9\textcolor{gray}{4}«
\newline{}Ordnung: mit Bleistift von unbekannter Hand nummeriert: »48« }
\buchAbdrucke{\weitereDrucke{Hermann Bahr, Arthur Schnitzler: \emph{Briefwechsel, Aufzeichnungen, Dokumente (1891–1931)}. Hg. Kurt Ifkovits und Martin Anton Müller. Göttingen: \emph{Wallstein} 2018, S. 81.} }\toendnotes[C]{\smallbreak}
\pstart
           \noindent{}\centering{}{\pb}\textcolor{gray}{\textbf{FELIX SALTEN}}\pend
           
\pstart
           \noindent{}\textcolor{gray}{\textbf{\textcolor{pink}{WIEN}{}\ledrightnote{\textcolor{pink}{Wien}},}}\hfill \textcolor{gray}{\textbf{»\textcolor{brown}{Berliner Neueste
                           Nachrichten}{}\ledrightnote{\textcolor{brown}{Berliner Neueste Nachrichten}}.«}}\pend
           
\pstart
           \textcolor{gray}{\textbf{\textcolor{pink}{IX., Hörlgasse 16}{}\ledrightnote{\textcolor{pink}{Hörlgasse}}.}}\hfill \textcolor{gray}{\textbf{»\textcolor{brown}{Münchener
                           General-Anzeiger}{}\ledrightnote{\textcolor{brown}{Münchener General-Anzeiger}}.«}}\pend
           
\pstart
           {\pb}Lieber Freund,  wenn Sie dem \textcolor{blue}{Überbringer}{}\ledrightnote{{$\rightarrow$}\textcolor{blue}{?? [Medizinstudent]}} dieses irgend
               eine Abschreibearbeit geben können, so thun Sie's, bitte, wenn nicht, schicken Sie ihn
               vielleicht zu \textcolor{blue}{Bahr}{}\ledrightnote{\textcolor{blue}{Hermann Bahr}}, der ja jetzt manches haben
               dürfte.\pend
           
\pstart
           Er ist \label{K_L03147-1v}\edtext{\textcolor{blue}{Mediziner im letzten
                  Jahrgang}{}\ledrightnote{{$\rightarrow$}\textcolor{blue}{?? [Medizinstudent]}}}{\lemma{\textnormal{\emph{Mediziner … Jahrgang}}}\Cendnote{\textnormal{Obwohl naheliegend, dürfte es
                  sich nicht um den von \textcolor{blue}{Schnitzler} handschriftlich auf der Karte vermerkten »\textcolor{blue}{M. J. Mayer}« gehandelt haben –
                  zumindest hat niemand mit diesem Namen zu der Zeit in \textcolor{pink}{Wien} Medizin studiert.}}}\label{K_L03147-1h} und es geht ihm sehr
               schlecht.\pend
           
\pstart
           Herzlichst {\\[\baselineskip]}\spacefill\mbox{Salten.}\pend
           \leftskip=0em{}
\pstart
           \noindent{}Vielleicht \label{K_L03147-2v}\edtext{Abends im Cafe}{\lemma{\textnormal{\emph{Abends im Cafe}}}\Cendnote{\textnormal{kein Kaffeehausbesuch belegt}}}\label{K_L03147-2h}?\pend
           \endnumbering\briefempfaengerindex{Schnitzler, Arthur@\textsc{Schnitzler, Arthur}!zzzSalten, Felix@\emph{von Felix Salten}!1894-09-152@{15. 9. 189{[}4?{]}}|)be}\mylabel{h}  \normalsize

\doendnotes{C}
\bigskip
\vfill

\clearpage

\footnotesize

\lohead{\textsc{register}}

% Definiere theindex-Environment komplett neu ohne reledmac
\makeatletter
\renewenvironment{theindex}{%
  \section*{\indexname}%
  \setlength{\parindent}{0pt}%
  \setlength{\parskip}{0pt plus 0.3pt}%
  \let\item\@idxitem
}{%
  \clearpage
}
\makeatother

\IfFileExists{\jobname-pw.ind}{\input{\jobname-pw.ind}}{}

\end{document}

      