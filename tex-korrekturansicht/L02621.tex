%% latex-korrekturansicht-vorspann.tex
%% Vorspann für die Korrekturansicht.
%% Lädt die gemeinsame Datei latex-vorspann.tex mit gesetztem Schalter.

\newif\ifkorrekturansicht
\korrekturansichttrue

\input{../tex-inputs/latex-vorspann}


               \section[Paul Goldmann an Arthur Schnitzler, 29. 5. {[}1894{]}]{ Paul Goldmann an Arthur Schnitzler, 29. 5. {[}1894{]}}\nopagebreak\mylabel{v}\rehead{ }\normalsize\beginnumbering\briefempfaengerindex{Schnitzler, Arthur@\textsc{Schnitzler, Arthur}!zzzGoldmann, Paul@\emph{von Paul Goldmann}!1894-05-292@{29. 5. {[}1894{]}}|(be} \toendnotes[C]{\smallbreak\pagebreak[2]} \Standort{DLA, A:Schnitzler, HS.NZ85.1.3164.}
\physDesc{Brief, 3 Blätter, 12 Seiten
\newline{}Handschrift: schwarze Tinte, deutsche Kurrent
\newline{}Schnitzler: 1) mit Bleistift auf dem ersten Blatt die Jahreszahl »94« vermerkt 2) mit rotem Buntstift fünf Unterstreichungen}\toendnotes[C]{\smallbreak}\pstart
           \noindent{}{\pb}\textcolor{gray}{\textbf{\textcolor{brown}{Frankfurter Zeitung}{}\ledrightnote{\textcolor{brown}{Frankfurter Zeitung}}.}}\hfill \textsc{\textcolor{pink}{Paris}{}\ledrightnote{\textcolor{pink}{Paris}}}, 29. Mai.\pend
           \pstart
           \textcolor{gray}{\textbf{(\textcolor{brown}{Gazette de
                     Francfort}{}\ledrightnote{\textcolor{brown}{Frankfurter Zeitung}}.)}}\pend
           \pstart
           \textcolor{gray}{\textbf{\begin{otherlanguage}{french}Fondateur\end{otherlanguage}{ }\textbf{M. \textcolor{blue}{L. Sonnemann}{}\ledrightnote{\textcolor{blue}{Leopold Sonnemann}}}.}}\pend
           \pstart
           \textcolor{gray}{\textbf{\begin{otherlanguage}{french}Journal politique, financier,\end{otherlanguage}}}\pend
           \pstart
           \textcolor{gray}{\textbf{\begin{otherlanguage}{french}commercial et littéraire.\end{otherlanguage}}}\pend
           \pstart
           \textcolor{gray}{\textbf{\begin{otherlanguage}{french}\textbf{Paraissant trois fois par jour}\end{otherlanguage}}}.\pend
           \pstart
           \textcolor{gray}{\textbf{–}}\pend
           \pstart
           \textcolor{gray}{\textbf{\begin{otherlanguage}{french}\textbf{Bureaux à \textcolor{pink}{Paris}{}\ledrightnote{\textcolor{pink}{Paris}}:}\end{otherlanguage}}}\pend
           \pstart
           \textcolor{gray}{\textbf{\begin{otherlanguage}{french}\textcolor{pink}{\textbf{24. Rue Feydeau}}{}\ledrightnote{\textcolor{pink}{rue Feydeau}}.\end{otherlanguage}}}\pend
           \pstart\center{}Mein lieber Freund,\pend\pstart
           Ich war acht Tage in \textcolor{pink}{Frankfurt}{}\ledrightnote{\textcolor{pink}{Frankfurt am Main}}; Krankheit meines
                  \textcolor{blue}{Onkels}{}\ledrightnote{→\textcolor{blue}{Fedor Mamroth}} und
               meiner \textcolor{blue}{Mutter}{}\ledrightnote{→\textcolor{blue}{Clementine Goldmann}}. Bei meiner
               Rückkehr fand ich Deine Briefe. \label{K_L02621-1v}\edtext{Miniſterſturz und Miniſter-Kriſis}{\lemma{\textnormal{\emph{Miniſterſturz und Miniſter-Kriſis}}}\Cendnote{\textnormal{Gemeint war der am 22. 5. 1894 vollzogene (erzwungene) Rücktritt des
                  Kabinetts von \textcolor{blue}{Jean Casimir-Perier}.}}}\label{K_L02621-1h}
               geben tauſenderlei zu thun. So komme ich erſt heut dazu, Dir zu antworten.\pend
           \pstart
           Ich habe das \label{K_L02621-2v}\edtext{Geld}{\lemma{\textnormal{\emph{Geld}}}\Cendnote{\textnormal{siehe Paul Goldmann an Arthur Schnitzler, 1. 5. [1894]}}}\label{K_L02621-2h} ſofort an \textsc{\textcolor{blue}{Albert}{}\ledrightnote{\textcolor{blue}{Henri Albert}}} übergeben. Es iſt blödſinnig: aber ich kam mir vor, als wenn ich einen Raub an
               Dir beginge. Trotzdem geht Alles ehrlich zu. Aber das iſt mein Wahn, und noch heut
                  {\pb}iſt es mir unangenehm, davon zu ſprechen. \textsc{\textcolor{blue}{Albert}{}\ledrightnote{\textcolor{blue}{Henri Albert}}} bewährt ſich ſehr als mein Freund, folglich auch als Deiner. Gutes, feines,
               anſchmiegendes, liebes Naturell! Wir machen große Schlachtpläne für Dich. Ich glaube,
               er hat Dir \label{K_L02621-33v}\edtext{darüber geſchrieben}{\lemma{\textnormal{\emph{darüber geſchrieben}}}\Cendnote{\textnormal{\textcolor{blue}{Albert}s Brief vom 23. 5. 1894
                  enthält neben dem Vorhaben, das ›\emph{\textcolor{green}{Abschiedsouper}}‹ bei einer Freien Bühne aufführen zu lassen, auch mehrere
                  Textvorhaben: \emph{\textcolor{green}{Denksteine}} und von ihm noch
                  nicht gelesene Textmanuskripte (\emph{\textcolor{green}{>Die überspannte Person}} und \textcolor{green}{Halb Zwei}, vgl. Paul Goldmann an Arthur Schnitzler, 29. 5. [1894]) möchte er gegen Ende des Sommers im \emph{\textcolor{green}{Mercure de France}} gedruckt sehen. Neben seiner
                  bevorstehenden \textcolor{green}{Rezension}
                  von \emph{\textcolor{green}{Das Märchen}}s in der \emph{\textcolor{green}{Revue Blanche}} plante er, in derselben Zeitschrift über die
                     »Jungen \textcolor{pink}{Wien}er« zu
                  schreiben.}}}\label{K_L02621-33h}. Vielleicht gelingt es gar, Dich \label{K_L02621-4v}\edtext{aufführen}{\lemma{\textnormal{\emph{aufführen}}}\Cendnote{\textnormal{Aus
                  dieser Zeit sind keine Aufführungen in \textcolor{pink}{Paris}
                  bekannt.}}}\label{K_L02621-4h} zu laſſen. Ich denke, im nächſten Heft des »\textsc{\textcolor{green}{Mercure}{}\ledrightnote{\textcolor{green}{Mercure de France}}}« wird \label{K_L02621-5v}\edtext{\textsc{\textcolor{blue}{Albert}{}\ledrightnote{\textcolor{blue}{Henri Albert}}} Dein »\textcolor{green}{Märchen}{}\ledrightnote{\textcolor{green}{Das Märchen. Schauspiel in drei Aufzügen}}« beſprechen}{\lemma{\textnormal{\emph{Albert … beſprechen}}}\Cendnote{\textnormal{\textcolor{blue}{Albert}s \textcolor{green}{Rezension} erschien nicht im \emph{\textcolor{green}{Mercure de France}}, sondern in der \emph{\textcolor{green}{Revue Blanche}}: \textcolor{blue}{Henri Albert}: \emph{\textcolor{green}{Les Lettres allemandes. Drames Nouveaux}}. In: \emph{\textcolor{green}{La Revue Blanche}}, Jg. 6, Nr. 32,
                        Juni 1894, S. 556–560, hier: S. 560.}}}\label{K_L02621-5h}. Von den
               zwei \textcolor{green}{Manuſkripten}{}\ledrightnote{→\textcolor{green}{Die überspannte Person}{\newline}→\textcolor{green}{Halb Zwei}},
               insbeſondere von der »\textcolor{green}{Überſpannten
                  Perſon}{}\ledrightnote{→\textcolor{green}{Die überspannte Person}}« ſind wir Alle hoch entzückt. Unterſchied zwiſchen Dir und \textsc{\textcolor{blue}{Lavedan}{}\ledrightnote{\textcolor{blue}{Henri Léon Lavedan}}} und den \textsc{Lavedanisirenden} Franzoſen: In \textcolor{pink}{Frankreich}{}\ledrightnote{\textcolor{pink}{Frankreich}} Geiſt, Oberflächlichkeit,
               Dekadenz-Koketterie. Bei Dir: Na{\pb}türlichkeit, Tiefe,
                  \uline{Sittlichkeit und Geſundheit} (Thut Dir
               wahrſcheinlich ſehr weh?). \strikeout{Geiſt} Geiſt natürlich
               auch. Das \label{K_L02621-333v}\edtext{Rin\textcolor{gray}{dv}ieh}{\lemma{\textnormal{\emph{Rindvieh}}}\Cendnote{\textnormal{unklare Anspielung}}}\label{K_L02621-333h}, das Dich in der Geſellſchaft zum dekadenten Häuptling
               macht, hat uns eine vergnügte Viertelſtunde bereitet.\pend
           \pstart
           Kennſt Du Frau \textsc{\textcolor{blue}{Andreas-\strikeout{Sa\textcolor{gray}{l}} Salome}{}\ledrightnote{\textcolor{blue}{Lou Andreas-Salomé}}}? Seltſame Frau. Nicht ſchön, ich weiß nicht einmal, ob ſympathiſch, aber
               derzeit unſere gute Freundin. Intime Freundin von \textsc{\textcolor{blue}{Nietzsche}{}\ledrightnote{\textcolor{blue}{Friedrich Nietzsche}}}. \label{K_L02621-6v}\edtext{Geſchlechtsloſe
                  Freundſchaft}{\lemma{\textnormal{\emph{Geſchlechtsloſe Freundſchaft}}}\Cendnote{\textnormal{Rein freundschaftlich war
                  die Beziehung zwischen \textcolor{blue}{Nietzsche} und \textcolor{blue}{Andreas-Salomé} wahrscheinlich nicht. Wie \textcolor{blue}{Andreas-Salomé}s \emph{\textcolor{green}{Lebensrückblick}} zu entnehmen ist, soll ihr \textcolor{blue}{Nietzsche}{ }1892 vergeblich einen Heiratsantrag gemacht haben. Es ist umstritten,
                  ob dieser Bericht wahr ist.}}}\label{K_L02621-6h}, wie ich glaube. Hat vier Jahre lang mit ihm
               gelebt und gearbeitet. Ungeheures Wiſſen, Philo{\pb}ſophin vom Fach. Hat ein merkwürdiges \textcolor{green}{Buch}{}\ledrightnote{\textcolor{green}{Friedrich Nietzsche in seinen Werken}}
               über \textsc{\textcolor{blue}{Nietzsche}{}\ledrightnote{\textcolor{blue}{Friedrich Nietzsche}}} veröffentlicht. Specialität: Religions-Philosophie. Nun gut: Sie weilt ſeit
               einigen Wochen in \textsc{\textcolor{pink}{Paris}{}\ledrightnote{\textcolor{pink}{Paris}}}, und ſie ſchickt Dir dieſen \label{K_L02621-9v}\edtext{Brief}{\lemma{\textnormal{\emph{Brief}}}\Cendnote{\textnormal{Womöglich handelte es sich um
                  den Brief \textcolor{blue}{Andreas-Salomé}s an \textcolor{blue}{Schnitzler} vom 15. 5. 1894.}}}\label{K_L02621-9h}. Willſt Du ihr antworten, ſo thus
               durch mich.\pend
           \pstart
           Alſo es \strikeout{\textcolor{gray}{wa}} wird in \textcolor{pink}{Wien}{}\ledrightnote{\textcolor{pink}{Wien}} dieſe neue \textcolor{green}{Revüe}{}\ledrightnote{→\textcolor{green}{Die Zeit. Wiener Wochenschrift}} begründet. Bitte ſchreib’ mir, was Du
               davon weißt und glaubſt (Zukunft). Ich habe die Empfindung, daß man ſich bei dieſer
               Gründung infam gegen mich benimmt. \textsc{\textcolor{blue}{Kanner}{}\ledrightnote{\textcolor{blue}{Heinrich Kanner}}} – Du weißt, wie hoch ich ſein Talent ſchätze, in welchem {\pb}wahrhaft geniale Züge ſind – iſt der intime \textcolor{blue}{Freund}{}\ledrightnote{→\textcolor{blue}{Heinrich Kanner}} meines \textcolor{blue}{Onkels}{}\ledrightnote{→\textcolor{blue}{Fedor Mamroth}} und meiner Familie. Mit
               mir ſteht er ſchlecht. Dieſer überlegen geſcheite \textcolor{blue}{Menſch}{}\ledrightnote{→\textcolor{blue}{Heinrich Kanner}} begeht die Dummheit, mir die Jahre hindurch
               nachzutragen, daß ich mich einmal in einem Geſpräch \strikeout{über} ihm gegenüber ironiſch-neckend über einige ſeiner Artikel ausgedrückt,
               die ich ſtets ehrlich bewundert habe. Und nun: Iſt es Haß? Iſt es Neid? Iſt es
               Verachtung? – bei dieſer Neugründung ignorirt er mich vollſtändig. Es hätte {\pb}ſich unbedingt gehört, daß man mich aufforderte, von
                  \textsc{\textcolor{pink}{Paris}{}\ledrightnote{\textcolor{pink}{Paris}}} aus für das \textcolor{green}{Blatt}{}\ledrightnote{→\textcolor{green}{Die Zeit. Wiener Wochenschrift}} thätig
               zu ſein. Ich hätte es kaum je annehmen können, aber eine Einladung hätte erfolgen
               müſſen. Statt deſſen iſt \textsc{\textcolor{blue}{Bahr}{}\ledrightnote{\textcolor{blue}{Hermann Bahr}}} ſeit geſtern in \textsc{\textcolor{pink}{Paris}{}\ledrightnote{\textcolor{pink}{Paris}}}, um \textsc{\textcolor{blue}{Albert}{}\ledrightnote{\textcolor{blue}{Henri Albert}}} die \textcolor{pink}{Pariſ}{}\ledrightnote{\textcolor{pink}{Paris}}er Vertretung zu übertragen. Ich
               habe ſelbſtverſtändlich \textsc{\textcolor{blue}{Albert}{}\ledrightnote{\textcolor{blue}{Henri Albert}}} zur Annahme gedrängt, da das in ſeinem Intereſſe iſt. Aber die Kränkung iſt
               nichtsdeſtoweniger ſehr bitter. Da ſiehſt Du einmal in einem praktiſchen Falle, wie
               falſch Deine freundſchaftlichen Anſichten über meine Geltung ſind.\pend
           \pstart
           {\pb}Ich habe gethan, was ich thun konnte, um eine
               Beſprechung des »\textsc{\textcolor{green}{Anatol}{}\ledrightnote{\textcolor{green}{Anatol}}}« in der \textcolor{green}{Frkf. Ztg.}{}\ledrightnote{\textcolor{green}{Frankfurter Zeitung}} durchzuſetzen.
               Vorgebens der wahre Grund ſind gewiſſe \strikeout{inne} innere
               Vorgänge zwiſchen meinem \textcolor{blue}{Onkel}{}\ledrightnote{→\textcolor{blue}{Fedor Mamroth}} und mir, die ich Dir einmal mündlich erklären werde. Hingegen habe ich
               eine \textcolor{green}{Beſprechung}{}\ledrightnote{→\textcolor{green}{[?? Rezension von Beer-Hofmann: Novellen]}} für \textsc{\textcolor{blue}{Richard}{}\ledrightnote{\textcolor{blue}{Richard Beer-Hofmann}}} erwirkt. Nun haben aber die Referenten das Recht ungehindert ſeiner
               Weisungs-Äußerung bei uns, und das dumme \textcolor{blue}{Frauenzimmer}{}\ledrightnote{→\textcolor{blue}{?? [Leitung der dt. Lit. bei der Frankfurter Zeitung]}}, das bei uns die deutſche Literatur voranleitet,
               hat \textsc{\textcolor{blue}{Richard}{}\ledrightnote{\textcolor{blue}{Richard Beer-Hofmann}}}s \strikeout{\textcolor{gray}{B}}{ }\textcolor{green}{Buch}{}\ledrightnote{→\textcolor{green}{Novellen}} abſolut nicht \label{K_L02621-44v}\edtext{\textcolor{green}{ver{\pb}ſtanden}{}\ledrightnote{→\textcolor{green}{[?? Rezension von Beer-Hofmann: Novellen]}}}{\lemma{\textnormal{\emph{verſtanden}}}\Cendnote{\textnormal{XXXX}}}\label{K_L02621-44h}. Dafür kann ich nichts, und
               ich kann es nur bedauern. Ich habe das Ehrenwort meines \textcolor{blue}{Onkels}{}\ledrightnote{→\textcolor{blue}{Fedor Mamroth}}, daß Dein neuer \label{K_L02621-10v}\edtext{Roman}{\lemma{\textnormal{\emph{Roman}}}\Cendnote{\textnormal{Nicht identifiziert. Möglicherweise ging es um Schnitzlers
                  Erzählung \emph{\textcolor{green}{Blumen}}, deren Abdruck in der \emph{\textcolor{green}{Frankfurter Zeitung}}{ }\textcolor{blue}{Mamroth} jedenfalls am 4. 4. 1894 freundlich ablehnte.}}}\label{K_L02621-10h} beſprochen
               wird, ſobald er in Buchform erſchienen iſt.\pend
           \pstart
           Wenn ich keinen ſchweren Krankheitsanfall bekomme, will ich von meinem
               vierwöchentlichen Urlaub drei auf eine Reiſe verwenden. Ich habe keinen höheren
               Wunſch, als dieſe drei Wochen mit Dir zu verbringen. Aber das muß im \label{K_L02621-8v}\edtext{Auguſt}{\lemma{\textnormal{\emph{Auguſt}}}\Cendnote{\textnormal{Von 23. 8. 1894 bis 3. 9. 1894 verbrachten Schnitzler und
                  Goldmann einige Zeit gemeinsam in \textcolor{pink}{Bad Ischl}
                  und \textcolor{pink}{Bad Aussee}.}}}\label{K_L02621-8h} ſein. Kannſt du fort?
               Und wohin? Bitte, ſchreib’ mir bald darüber.\pend
           \pstart
           {\pb}Oh dieſe Hypochondrie in Deinem letzten Briefe!
               Gewiß, es iſt wünſchenswerth frei zu ſein. Aber ich habe oft über die Freiheit
               nachgedacht, und ich fürchte beinahe, daß ſie doch nicht das Gut iſt, \strikeout{daß}{ }\introOben{}das\introOben{} wir glauben. Man würde glücklich auf allen Seiten Wege
               vor ſich ſehen. Und ich wenigſtens gehöre nicht zu den Leuten, die raſch entſchloſſen
               einen von den hundert Wegen einſchlagen, ſondern zu denen, die all’ ihr Leben lang
               damit vertändeln würden, davor zu ſtehen {\pb}und zu
               überlegen: ſoll ich dahin gehen oder dorthin? Und würde ich einen Weg wählen, welchen
               immer, ſo würde mich bis an meinen Tod die Reue verfolgen, daß ich nicht den andern
               eingeſchlagen. Biſt Du nicht auch ein wenig ſo? Gewiß, der Zwang iſt drückend. Aber
               es hat auch ſein gutes: es erſpart einem die Weiche der Wahl und die Verantwortung
               dafür. Der Zwang, \label{K_L02621-3v}\edtext{\textsc{\begin{otherlanguage}{french}c’est une destinée toute faite\end{otherlanguage}}}{\lemma{\textnormal{\emph{c’est … faite}}}\Cendnote{\textnormal{französisch, etwa: das Schicksal ist
                  vorbestimmt}}}\label{K_L02621-3h}. Und wenn er, wie bei dir, nicht mit Infamie verbunden iſt (wie
               bei mir), ſo ſollte man ihn {\pb}ruhig tragen, zumal
               wenn man dabei auch noch graduieren kann. Wer weiß, ob nicht gerade in Deiner Abſcheu
               davor, ein ärztlicher \strikeout{ban} Banauſe zu werden, ein
               gutes Theil Deiner Productionskraft liegt. Und wer weiß, ob dieſe, die vielleicht zum
               großen Theil eine Reaktionserſcheinung iſt, nicht ſehr abnehmen würde, wenn auf der
               andern Seite die Aktion des Zwanges aufhörte. Dabei fällt mir ein, daß es im Obigen
               nicht Productions-Kunſt heißen darf, ſondern »Wille zur Produktion«. Auch ſonſt habe
               ich es mir ganz {\pb}anders gedacht, als es da
               ausgedrückt iſt. Das macht aber nichts.\pend
           \pstart
           Die von Dir erwähnte \textcolor{green}{Erwiderung}{}\ledrightnote{→\textcolor{green}{[Erwiderung]}} von \textsc{\textcolor{blue}{Christensen}{}\ledrightnote{\textcolor{blue}{Hjalmar Christensen}}} habe ich nirgends entdecken können. Könnteſt Du mir nicht die Nummer oder nur
               die ungefähre Erſcheinungszeit angeben?\pend
           \pstart
           Und \textsc{\textcolor{blue}{Richard}{}\ledrightnote{\textcolor{blue}{Richard Beer-Hofmann}}}? Und \textsc{\textcolor{blue}{Loris}{}\ledrightnote{\textcolor{blue}{Hugo von Hofmannsthal}}}?\pend
           \pstart
           Bitte, lies: \textsc{\textcolor{blue}{Bernard Lazare}{}\ledrightnote{\textcolor{blue}{Bernard Lazare}}}: \textsc{\textcolor{green}{L’Antisémitisme}{}\ledrightnote{\textcolor{green}{L’antisémitisme. Son histoire et ses causes}}}. Soeben erſchienen bei \textsc{\textcolor{blue}{Léon Challey}{}\ledrightnote{\textcolor{blue}{Léon Chailley}}}, \textsc{\textcolor{pink}{8. Rue Saint-Joseph}{}\ledrightnote{\textcolor{pink}{Rue Saint-Joseph}}}. Der \textcolor{blue}{Verfaſſer}{}\ledrightnote{→\textcolor{blue}{Bernard Lazare}}, in
               unſerem Alter, iſt ſelbst Jude.\pend
           \pstart
           Mein \textcolor{blue}{Schwager}{}\ledrightnote{→\textcolor{blue}{Josef Rosengart}} iſt
               hochbeglückt mit Deiner \label{K_L02621-7v}\edtext{Zeitſchrift}{\lemma{\textnormal{\emph{Zeitſchrift}}}\Cendnote{\textnormal{nicht ermittelt}}}\label{K_L02621-7h}
               und dankt Dir noch vielmals.\pend
           \pstart
           Viele treue Grüße! {\\[\baselineskip]}Dein {\\[\baselineskip]}\spacefill\mbox{Paul Goldmann}\pend
           \leftskip=0em{}\pstart
           \noindent{}Schreib’ bald!!\pend
           \endnumbering\briefempfaengerindex{Schnitzler, Arthur@\textsc{Schnitzler, Arthur}!zzzGoldmann, Paul@\emph{von Paul Goldmann}!1894-05-292@{29. 5. {[}1894{]}}|)be}\mylabel{h}\begin{anhang}\end{anhang}\normalsize

\doendnotes{C}
\bigskip
\vfill

\clearpage

\footnotesize

\lohead{\textsc{register}}

% Definiere theindex-Environment komplett neu ohne reledmac
\makeatletter
\renewenvironment{theindex}{%
  \section*{\indexname}%
  \setlength{\parindent}{0pt}%
  \setlength{\parskip}{0pt plus 0.3pt}%
  \let\item\@idxitem
}{%
  \clearpage
}
\makeatother

\IfFileExists{\jobname-pw.ind}{\input{\jobname-pw.ind}}{}

\end{document}

      