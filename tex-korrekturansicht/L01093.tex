%% latex-korrekturansicht-vorspann.tex
%% Vorspann für die Korrekturansicht.
%% Lädt die gemeinsame Datei latex-vorspann.tex mit gesetztem Schalter.

\newif\ifkorrekturansicht
\korrekturansichttrue

\input{../tex-inputs/latex-vorspann}


               \section[Hermann Bahr an Arthur Schnitzler, 23. 1. {[}1901{]}]{ Hermann Bahr an Arthur Schnitzler, 23. 1. {[}1901{]}}\nopagebreak\mylabel{v}\rehead{ }\normalsize\beginnumbering\briefempfaengerindex{Schnitzler, Arthur@\textsc{Schnitzler, Arthur}!zzzBahr, Hermann@\emph{von Hermann Bahr}!1901-01-232@{23. 1. {[}1901{]}}|(be} \toendnotes[C]{\smallbreak\pagebreak[2]} \Standort{CUL, Schnitzler, B 5b.}
\physDesc{Brief, 1 Blatt, 1 Seite
\newline{}Handschrift: schwarze Tinte, deutsche Kurrent
\newline{}Schnitzler: mit Bleistift die Jahreszahl »901« ergänzt \newline{}Ordnung: mit Bleistift von unbekannter Hand nummeriert: »72« }\buchAbdrucke{\weitereDrucke{Hermann Bahr, Arthur Schnitzler: \emph{Briefwechsel, Aufzeichnungen, Dokumente (1891–1931)}. Hg. Kurt Ifkovits und Martin Anton Müller. Göttingen: \emph{Wallstein} 2018, S. 192.} }\toendnotes[C]{\smallbreak}\pstart
           \noindent{}\centering{}{\pb}\textcolor{gray}{\textbf{\textcolor{brown}{Redaktion des Neuen Wiener Tagblatt}{}\ledrightnote{\textcolor{brown}{Neues Wiener Tagblatt}}}}\pend
           \pstart
           \noindent{}\centering{}\textcolor{gray}{\textbf{\textsc{\textcolor{pink}{Wien, I., Rothenturmstrasse,
                        Steyrerhof}{}\ledrightnote{\textcolor{pink}{Steyrerhof}}.}}}\pend
           \pstart
           \noindent{}\centering{}\textcolor{gray}{\textbf{Telegramm-Adresse: \textcolor{brown}{Tagblatt}{}\ledrightnote{\textcolor{brown}{Neues Wiener Tagblatt}},
                        \textcolor{pink}{Steyrerhof, Wien}{}\ledrightnote{\textcolor{pink}{Steyrerhof}}. – Telephon Nr. 384.
                     Staats-Telephon Nr. 36.}}\pend
           \pstart
           \raggedleft{}23/1\pend
           \pstart\center{}Lieber Arthur!\pend\pstart
           Ich habe die »\label{K_L01093_1v}\edtext{\textcolor{green}{Marionetten}{}\ledrightnote{\textcolor{green}{Zum großen Wurstel}}}{\lemma{\textnormal{\emph{Marionetten}}}\Cendnote{\textnormal{Erste Fassung von \emph{\textcolor{green}{Zum großen Wurstel}}, die am 8. 3. 1901 von \textcolor{blue}{Wolzogen}s \textcolor{pink}{Überbrettl} aufgeführt wurde. Erst in die Umarbeitung von
                     1905, die vor allem eine Erweiterung der illusionsbrechenden
                  Figuren vornahm, wurde die Hauptfigur von \textcolor{blue}{Bahr}s \emph{\textcolor{green}{Der Meister}} eingearbeitet.}}}\label{K_L01093_1h}«
               gestern nachts ſogleich geleſen und mich diebiſch amüſiert. Sie ſind einfach
               großartig. Bei einer Vorleſung oder in einem kleinen Theater bürge ich für einen ſehr
               ſtarken Erfolg. Im \textcolor{pink}{Volkstheater}{}\ledrightnote{\textcolor{pink}{Volkstheater}} iſt allerdings der
               Raum dafür ſehr ekelhaft und noch ekelhafter ja unſere \label{K_L01093_2v}\edtext{Premièrenjuden}{\lemma{\textnormal{\emph{Premièrenjuden}}}\Cendnote{\textnormal{Vgl. \emph{Briefwechsel}
                     Bahr/Schnitzler 367}}}\label{K_L01093_2h} – aber man muß es halt wagen. \textsc{Manuscript} in ein paar Tagen.\pend
           \pstart
           Herzlichſt{\\[\baselineskip]}Dein{\\[\baselineskip]}\spacefill\mbox{Hermann}\pend
           \leftskip=0em{}\endnumbering\briefempfaengerindex{Schnitzler, Arthur@\textsc{Schnitzler, Arthur}!zzzBahr, Hermann@\emph{von Hermann Bahr}!1901-01-232@{23. 1. {[}1901{]}}|)be}\mylabel{h}  \normalsize

\doendnotes{C}
\bigskip
\vfill

\clearpage

\footnotesize

\lohead{\textsc{register}}

% Definiere theindex-Environment komplett neu ohne reledmac
\makeatletter
\renewenvironment{theindex}{%
  \section*{\indexname}%
  \setlength{\parindent}{0pt}%
  \setlength{\parskip}{0pt plus 0.3pt}%
  \let\item\@idxitem
}{%
  \clearpage
}
\makeatother

\IfFileExists{\jobname-pw.ind}{\input{\jobname-pw.ind}}{}

\end{document}

      