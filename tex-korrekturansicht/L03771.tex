%% latex-korrekturansicht-vorspann.tex
%% Vorspann für die Korrekturansicht.
%% Lädt die gemeinsame Datei latex-vorspann.tex mit gesetztem Schalter.

\newif\ifkorrekturansicht
\korrekturansichttrue

\input{../tex-inputs/latex-vorspann}


\section[Arthur Schnitzler an Stefan Zweig, 15. 1. 1915]{L03771 Arthur Schnitzler an Stefan Zweig, 15. 1. 1915}
\nopagebreak\mylabel{L03771v}
\rehead{ }\normalsize\beginnumbering\briefempfaengerindex{, @\textsc{, }!zzz, @\emph{von  }!1915-01-151@{15. 1. 1915}|(be}
\toendnotes[C]{\smallbreak\pagebreak[2]}\Standort{Jerusalem, National Library of Israel, ARC. Ms. Var. 305 1 58 Stefan Zweig Collection.}
\physDesc{Briefkarte, 534 Zeichen
\newline{}Schreibmaschine
\newline{}Handschrift: schwarze Tinte (\noindent{}Korrekturen, Unterschrift)}\toendnotes[C]{\smallbreak}
\pstart
           {\pb}\textcolor{gray}{\textbf{Dr. Arthur Schnitzler}}\hfill 15. 1. 1915. \pend
           
\pstart
           \textcolor{gray}{\textbf{\textcolor{pink}{Wien XVIII. Sternwartestrasse 71}\oindex{Wien@\textbf{Wien}!XVIII., Währing@\textbf{XVIII., Währing}!Sternwartestraße 71@\textbf{Sternwartestraße 71}, \emph{Wohngebäude}|pw}{}\ledrightnote{\textcolor{pink}{Sternwartestraße 71}}}}\pend
           
\pstart\center{}Lieber Herr Doktor Zweig.\pend\vspace{0.5em}
\pstart
           \textcolor{blue}{Rolland}\pwindex{Rolland, Romain 29.\,1.\,1866 Clamecy – 30.\,12.\,1944 Vézelay@\textsc{Rolland, Romain} (29.\,1.\,1866 Clamecy – 30.\,12.\,1944 Vézelay), \emph{Schriftsteller}|pw}{}\ledrightnote{\textcolor{blue}{Romain Rolland}}{ }\label{K_L03771-1v}\edtext{schreibt mir (am
                  11. d.)}{\lemma{\textnormal{\emph{schreibt mir (am
                  11. d.)}}}\Cendnote{\textnormal{Das
                  Korrespondenzstück ist nicht überliefert.}}}\label{K_L03771-1}, dass er ihnen dreimal geschrieben
               und Ihnen einen Brief von \textcolor{blue}{Richard Bloch}\pwindex{Bloch, Richard 3.\,3.\,1856 Berlin – 1928 ebd.@\textsc{Bloch, Richard} (3.\,3.\,1856 Berlin – 1928 ebd.), \emph{Theaterverleger}|pw}{}\ledrightnote{\textcolor{blue}{Richard Bloch}} für
                  \textcolor{blue}{Paul Amann}\pwindex{Amann, Paul 6.\,3.\,1884 Prag – 24.\,2.\,1958 Fairfield County@\textsc{Amann, Paul} (6.\,3.\,1884 Prag – 24.\,2.\,1958 Fairfield County), \emph{Übersetzer, Philologe, Lehrer}|pw}{}\ledrightnote{\textcolor{blue}{Paul Amann}} geschickt hätte. Das \textcolor{green}{Journal de Genève}\pwindex{Journal de Genève@\emph{Journal de Genève}|pw}{}\ledrightnote{\textcolor{green}{Journal de Genève}}\strikeout{,} ist mir nicht zugekommen, ja bisher nicht einmal
               der betreffende \label{K_L03771-2v}\edtext{\textcolor{green}{Ausschnitt}\pwindex{Schnitzler, Arthur 15. 5. 1862 Wien – 21. 10. 1931 ebd.@\textsc{Schnitzler, Arthur} (15. 5. 1862 Wien – 21. 10. 1931 ebd.), \emph{Schriftsteller, Mediziner}!Une protestation d’Arthur Schnitzler@\strich\emph{Une protestation d’Arthur Schnitzler}|pwv}\pwindex{Rolland, Romain 29.\,1.\,1866 Clamecy – 30.\,12.\,1944 Vézelay@\textsc{Rolland, Romain} (29.\,1.\,1866 Clamecy – 30.\,12.\,1944 Vézelay), \emph{Schriftsteller}!Une protestation d’Arthur Schnitzler@\strich\emph{Une protestation d’Arthur Schnitzler}|pwv}{}\ledrightnote{{$\rightarrow$}\emph{\textcolor{green}{Une protestation d’Arthur Schnitzler}}} mit der \textcolor{blue}{Rolland}\pwindex{Rolland, Romain 29.\,1.\,1866 Clamecy – 30.\,12.\,1944 Vézelay@\textsc{Rolland, Romain} (29.\,1.\,1866 Clamecy – 30.\,12.\,1944 Vézelay), \emph{Schriftsteller}|pw}{}\ledrightnote{\textcolor{blue}{Romain Rolland}}’schen \textcolor{green}{Uebersetzung}\pwindex{Schnitzler, Arthur 15. 5. 1862 Wien – 21. 10. 1931 ebd.@\textsc{Schnitzler, Arthur} (15. 5. 1862 Wien – 21. 10. 1931 ebd.), \emph{Schriftsteller, Mediziner}!Une protestation d’Arthur Schnitzler@\strich\emph{Une protestation d’Arthur Schnitzler}|pwv}\pwindex{Rolland, Romain 29.\,1.\,1866 Clamecy – 30.\,12.\,1944 Vézelay@\textsc{Rolland, Romain} (29.\,1.\,1866 Clamecy – 30.\,12.\,1944 Vézelay), \emph{Schriftsteller}!Une protestation d’Arthur Schnitzler@\strich\emph{Une protestation d’Arthur Schnitzler}|pwv}{}\ledrightnote{{$\rightarrow$}\emph{\textcolor{green}{Une protestation d’Arthur Schnitzler}}}}{\lemma{\textnormal{\emph{Ausschnitt … Uebersetzung}}}\Cendnote{\textnormal{\textcolor{blue}{Arthur Schnitzler}, \textcolor{blue}{Romain Rolland}\pwindex{Rolland, Romain 29.\,1.\,1866 Clamecy – 30.\,12.\,1944 Vézelay@\textsc{Rolland, Romain} (29.\,1.\,1866 Clamecy – 30.\,12.\,1944 Vézelay), \emph{Schriftsteller}|pwk} [Einleitung und Übersetzung]: \emph{\textcolor{green}{Une protestation d’Arthur Schnitzler}\pwindex{Schnitzler, Arthur 15. 5. 1862 Wien – 21. 10. 1931 ebd.@\textsc{Schnitzler, Arthur} (15. 5. 1862 Wien – 21. 10. 1931 ebd.), \emph{Schriftsteller, Mediziner}!Une protestation d’Arthur Schnitzler@\strich\emph{Une protestation d’Arthur Schnitzler}|pwk}\pwindex{Rolland, Romain 29.\,1.\,1866 Clamecy – 30.\,12.\,1944 Vézelay@\textsc{Rolland, Romain} (29.\,1.\,1866 Clamecy – 30.\,12.\,1944 Vézelay), \emph{Schriftsteller}!Une protestation d’Arthur Schnitzler@\strich\emph{Une protestation d’Arthur Schnitzler}|pwk}}. In:
                        \emph{\textcolor{green}{Journal de Genève}\pwindex{Journal de Genève@\emph{Journal de Genève}|pwk}}, Jg. 85, 21. 12. 1914, 3. Ausgabe,
                  S. [1].}}}\label{K_L03771-2} meines \label{K_L03771-3v}\edtext{\textcolor{green}{Protestes}\pwindex{Schnitzler, Arthur 15. 5. 1862 Wien – 21. 10. 1931 ebd.@\textsc{Schnitzler, Arthur} (15. 5. 1862 Wien – 21. 10. 1931 ebd.), \emph{Schriftsteller, Mediziner}!Brief Artur Schnitzlers@\strich\emph{Ein Brief Artur Schnitzlers}|pwv}{}\ledrightnote{{$\rightarrow$}\emph{\textcolor{green}{Ein Brief Artur Schnitzlers}}}}{\lemma{\textnormal{\emph{Protestes}}}\Cendnote{\textnormal{Die deutschsprachige Veröffentlichung des
                  Protestes erschien am Tag nach der französischen: \emph{\textcolor{green}{Ein Brief Artur Schnitzlers}\pwindex{Schnitzler, Arthur 15. 5. 1862 Wien – 21. 10. 1931 ebd.@\textsc{Schnitzler, Arthur} (15. 5. 1862 Wien – 21. 10. 1931 ebd.), \emph{Schriftsteller, Mediziner}!Brief Artur Schnitzlers@\strich\emph{Ein Brief Artur Schnitzlers}|pwk}}. In: \emph{\textcolor{green}{Neue Zürcher Zeitung}\pwindex{Neue Zürcher Zeitung@\emph{Neue Zürcher Zeitung}|pwk}}, Jg. 135, Nr. 1700,
                        22. 12. 1914, 2. Mittagsblatt,
                  S. 2.}}}\label{K_L03771-3}, den er, \textcolor{blue}{Rolland}\pwindex{Rolland, Romain 29.\,1.\,1866 Clamecy – 30.\,12.\,1944 Vézelay@\textsc{Rolland, Romain} (29.\,1.\,1866 Clamecy – 30.\,12.\,1944 Vézelay), \emph{Schriftsteller}|pw}{}\ledrightnote{\textcolor{blue}{Romain Rolland}},
               nach Nichteinlangen des \textcolor{green}{J. d. G.}\pwindex{Journal de Genève@\emph{Journal de Genève}|pw}{}\ledrightnote{\textcolor{green}{Journal de Genève}}, \introOben{}mir\introOben{} zuzusenden versucht hat. So nehme ich an, dass auch einige
               von den \textcolor{blue}{Rolland}\pwindex{Rolland, Romain 29.\,1.\,1866 Clamecy – 30.\,12.\,1944 Vézelay@\textsc{Rolland, Romain} (29.\,1.\,1866 Clamecy – 30.\,12.\,1944 Vézelay), \emph{Schriftsteller}|pw}{}\ledrightnote{\textcolor{blue}{Romain Rolland}}’schen Briefen an Sie von der
               Zensur zurückgehalten wurden.\pend
           
\pstart
           Herzlichen Gruss{\\[\baselineskip]}Ihr{\\[\baselineskip]}\spacefill\mbox{{[}hs.:{]} ArthurSchnitzler}\pend
           \leftskip=0em{}\selectlanguage{ngerman}\endnumbering\briefempfaengerindex{, @\textsc{, }!zzz, @\emph{von  }!1915-01-151@{15. 1. 1915}|)be}\mylabel{L03771h}
\begin{anhang}
\end{anhang}\normalsize

\doendnotes{C}
\bigskip
\vfill

\clearpage

\footnotesize

\lohead{\textsc{register}}

% Definiere theindex-Environment komplett neu ohne reledmac
\makeatletter
\renewenvironment{theindex}{%
  \section*{\indexname}%
  \setlength{\parindent}{0pt}%
  \setlength{\parskip}{0pt plus 0.3pt}%
  \let\item\@idxitem
}{%
  \clearpage
}
\makeatother

\IfFileExists{\jobname-pw.ind}{\input{\jobname-pw.ind}}{}

\end{document}

      