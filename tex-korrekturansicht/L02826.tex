%% latex-korrekturansicht-vorspann.tex
%% Vorspann für die Korrekturansicht.
%% Lädt die gemeinsame Datei latex-vorspann.tex mit gesetztem Schalter.

\newif\ifkorrekturansicht
\korrekturansichttrue

\input{../tex-inputs/latex-vorspann}


               \section[ Paul Goldmann an Arthur Schnitzler, 25. 9. {[}1897{]}]{Paul Goldmann an Arthur Schnitzler, 25. 9. {[}1897{]}}\nopagebreak\mylabel{v}\rehead{ }\normalsize\beginnumbering\briefempfaengerindex{Schnitzler, Arthur@\textsc{Schnitzler, Arthur}!zzzGoldmann, Paul@\emph{von Paul Goldmann}!1897-09-252@{25. 9. {[}1897{]}}|(be} \toendnotes[C]{\smallbreak\pagebreak[2]} \Standort{DLA, A:Schnitzler, HS.NZ85.1.3167.}
\physDesc{Brief, 1 Blatt, 4 Seiten
\newline{}Handschrift: blaue Tinte, deutsche Kurrent
\newline{}Schnitzler: mit Bleistift das Jahr »97« vermerkt }\toendnotes[C]{\smallbreak}\pstart
           \noindent{}{\pb}\textcolor{gray}{\textbf{\textbf{\textcolor{brown}{Frankfurter Zeitung}{}\ledrightnote{\textcolor{brown}{Frankfurter Zeitung}}}}}\pend
           \pstart
           \textcolor{gray}{\textbf{(\textcolor{brown}{\begin{otherlanguage}{french}Gazette de Francfort\end{otherlanguage}}{}\ledrightnote{\textcolor{brown}{Frankfurter Zeitung}}).}}\pend
           \pstart
           \textcolor{gray}{\textbf{\textbf{\begin{otherlanguage}{french}Fondateur M.\end{otherlanguage}{ }\textcolor{blue}{L. Sonnemann}{}\ledrightnote{\textcolor{blue}{Leopold Sonnemann}}.}}}\pend
           \pstart
           \begin{otherlanguage}{french}\textcolor{gray}{\textbf{Journal politique, financier,}}\end{otherlanguage}\pend
           \pstart
           \begin{otherlanguage}{french}\textcolor{gray}{\textbf{commercial et littéraire.}}\end{otherlanguage}\pend
           \pstart
           \begin{otherlanguage}{french}\textcolor{gray}{\textbf{\textbf{Paraissant trois fois par jour.}}}\end{otherlanguage}\pend
           \pstart
           \begin{otherlanguage}{french}\textcolor{gray}{\textbf{\textbf{Bureau à \textcolor{pink}{Paris}{}\ledrightnote{\textcolor{pink}{Paris}}}}}\end{otherlanguage}\hfill \textsc{\textcolor{pink}{Paris}{}\ledrightnote{\textcolor{pink}{Paris}}}, \substVorne{}\textsuperscript{3}\substDazwischen{}2\substHinten{}5. September.\pend
           \pstart
           \begin{otherlanguage}{french}\textcolor{gray}{\textbf{\textbf{\textcolor{pink}{10 Rue de la Bourse}{}\ledrightnote{\textcolor{pink}{rue de la Bourse}}.}}}\end{otherlanguage}\pend
           \pstart\center{}Mein lieber Freund,\pend\pstart
           Es iſt ſehr, ſehr \label{K_L02826-1v}\edtext{traurig}{\lemma{\textnormal{\emph{traurig}}}\Cendnote{\textnormal{Die Totgeburt des \textcolor{blue}{Sohn}s von \textcolor{blue}{Schnitzler} und \textcolor{blue}{Marie
                     Reinhard} am 24. 9. 1897. \textcolor{blue}{Schnitzler} gab
                  sich selbst Schuld am Tod des \textcolor{blue}{Kind}es (vgl. A. S.: \emph{Tagebuch}, 30. 9. 1897).}}}\label{K_L02826-1h}, und
               mich hat es tief ergriffen. Eines muß Dich tröſten: Du haſt keine Schuld. Alles, was
               Du thun konnteſt, haſt Du gethan. Das Schickſal hat es ſo gewollt, und \strikeout{d\textcolor{gray}{×}} da ſtand es nicht mehr in Deiner Macht, zu hindern. Warum das gerade Dich
               treffen mußte? Man muß ſich eben abgewöhnen, nach Gründen zu fragen; es gibt
               keine.\pend
           \pstart
           Das arme \textcolor{blue}{Kind}{}\ledrightnote{→\textcolor{blue}{?? [Totgeborener Sohn von Arthur Schnitzler und Marie Reinhard]}} wollen wir
               nicht beklagen. Es iſt ihm eben nur das Leben erſpart geblieben. Es iſt nach kurzer
                  {\pb}Reiſe an das Ziel gelangt, dem wir alle zugehen
               auf dieſem langen, ſchweren Wege. All’ die Thränen braucht es nicht zu weinen, und
               das Bischen Süßigkeit wird es nicht vermiſſen, weil es ſie nie gekannt hat{\dotssix}\pend
           \pstart
           Was für bittere Stunden Du durchgemacht haben mußt, armer Freund! \strikeout{\textcolor{gray}{Was gilt}{ }\textcolor{gray}{×}\-\textcolor{gray}{×}\-\textcolor{gray}{×}u\textcolor{gray}{×}}{ }\strikeout{\textcolor{gray}{×}\-\textcolor{gray}{×}\-\textcolor{gray}{×}\-\textcolor{gray}{×}\-\textcolor{gray}{×}{ }\textcolor{gray}{×}\-\textcolor{gray}{×}\-\textcolor{gray}{×}\-\textcolor{gray}{×}\-\textcolor{gray}{×}} Könnte ich nur wenigſtens einen Tag bei Dir ſein! Ich würde Dir immerfort
               ſagen: »\label{K_L02826-3v}\edtext{\textcolor{green}{Du biſt jung, und nichts iſt
                  verloren.}{}\ledrightnote{→\textcolor{green}{Erzählungen des Küsters von Dandery}}}{\lemma{\textnormal{\emph{Du … verloren.}}}\Cendnote{\textnormal{Möglicherweise ein nahezu wörtliches
                  Zitat (S. 100) aus \textcolor{blue}{August Blanche}s \emph{\textcolor{green}{Erzählungen des Küsters von Dandery}} (dt.
                  Übers. 1876, \textcolor{pink}{dän}. Original 1856 unter dem Titel \textcolor{green}{Berättelser af Klockaren i
                     Danderyd}).}}}\label{K_L02826-3h}«\pend
           \pstart
           Am Meiſten aber dauert mich die arme \textcolor{blue}{Frau}{}\ledrightnote{→\textcolor{blue}{Marie Reinhard}}. Du biſt {\pb}einfach um
               eine ſchöne Hoffnung ärmer (und auch das nur für den Augenblick). Sie muß es aber als
               einen wahren \label{K_L02826-5v}\edtext{Zuſammenbruch}{\lemma{\textnormal{\emph{Zuſammenbruch}}}\Cendnote{\textnormal{\textcolor{blue}{Marie Reinhard} war zumindest \textcolor{blue}{Schnitzler}s \emph{\textcolor{green}{Tagebuch}} zufolge »gefasst und brav« (A. S.: \emph{Tagebuch}, 25. 9. 1897).}}}\label{K_L02826-5h} empfinden. Sei nur
               recht gut und lieb zu ihr. In der Erfüllung dieſer Pflicht wirſt Du auch für Dich den
               beſten Troſt finden. Und ſag’ ihr, daß ich ihr von ganzem Herzen die Hand drücke.\pend
           \pstart
           Bitte, bitte: ſchreib’ mir bald, und wenn es auch nur ein paar Zeilen ſind.\pend
           \pstart
           Du ſollteſt jetzt ſo bald als möglich eine \label{K_L02826-6v}\edtext{Reiſe machen}{\lemma{\textnormal{\emph{Reiſe machen}}}\Cendnote{\textnormal{\textcolor{blue}{Schnitzler} verreiste erst im November 1897 wieder – nach \textcolor{pink}{Prag}, wo am 27. 11. 1897 die Premiere von \emph{\textcolor{green}{Freiwild}} im \textcolor{pink}{Neuen Deutschen
                     Theater} stattfand.}}}\label{K_L02826-6h}. Komm zu mir nach \textcolor{pink}{Paris}{}\ledrightnote{\textcolor{pink}{Paris}}! {\dots}\pend
           \pstart
           Armer Freund! Es thut mir innig leid, daß Du, gerade Du dieſen Schmerz {\pb}haben mußteſt! Es iſt auch für mich ein recht
               trauriger Tag.\pend
           \pstart
           Ich umarme Dich {\\[\baselineskip]}von Herzen und in Treue {\\[\baselineskip]}Dein {\\[\baselineskip]}\spacefill\mbox{Paul Goldmann}\pend
           \leftskip=0em{}\pstart
           \noindent{}Die \label{K_L02826-8v}\edtext{Briefe}{\lemma{\textnormal{\emph{Briefe}}}\Cendnote{\textnormal{Bezug unklar}}}\label{K_L02826-8h} ſind alle beſorgt. Auf Deinen Brief
                  antworte ich Dir nächſtens.\pend
           \endnumbering\briefempfaengerindex{Schnitzler, Arthur@\textsc{Schnitzler, Arthur}!zzzGoldmann, Paul@\emph{von Paul Goldmann}!1897-09-252@{25. 9. {[}1897{]}}|)be}\mylabel{h}\begin{anhang}\end{anhang}\normalsize

\doendnotes{C}
\bigskip
\vfill

\clearpage

\footnotesize

\lohead{\textsc{register}}

% Definiere theindex-Environment komplett neu ohne reledmac
\makeatletter
\renewenvironment{theindex}{%
  \section*{\indexname}%
  \setlength{\parindent}{0pt}%
  \setlength{\parskip}{0pt plus 0.3pt}%
  \let\item\@idxitem
}{%
  \clearpage
}
\makeatother

\IfFileExists{\jobname-pw.ind}{\input{\jobname-pw.ind}}{}

\end{document}

      