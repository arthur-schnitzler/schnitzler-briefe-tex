%% latex-korrekturansicht-vorspann.tex
%% Vorspann für die Korrekturansicht.
%% Lädt die gemeinsame Datei latex-vorspann.tex mit gesetztem Schalter.

\newif\ifkorrekturansicht
\korrekturansichttrue

\input{../tex-inputs/latex-vorspann}


               \section[Arthur Schnitzler an Robert Adam, 11. 7. 1915]{ Arthur Schnitzler an Robert Adam, 11. 7. 1915}\nopagebreak\mylabel{v}\rehead{ }\normalsize\beginnumbering\briefempfaengerindex{Adam, Robert@\textsc{Adam, Robert}!zzzSchnitzler, Arthur@\emph{von Arthur Schnitzler}!1915-07-111@{11. 7. 1915}|(be} \toendnotes[C]{\smallbreak\pagebreak[2]} \Standort{DLA, 96.34.1/14.}
\physDesc{Briefkarte, Umschlag
\newline{}Handschrift: schwarze Tinte, lateinische Kurrent\newline{}Versand: Stempel: »\nobreak{}Wien, 1\textcolor{gray}{2}. VII. 15\nobreak{}«.  }\toendnotes[C]{\smallbreak}\pstart{}{\pb}\textcolor{gray}{\textbf{Dr. Arthur Schnitzler}}\pend{}\pstart{}\textcolor{gray}{\textbf{\textcolor{pink}{Wien XVIII. Sternwartestrasse 71}{}\ledrightnote{\textcolor{pink}{Sternwartestraße}}}}\pend{}{\bigskip}\pstart{}{\pb}Herrn Dr. Robert Adam
                        Pollak,\pend{}\pstart{}Bezirksrichter in \textcolor{pink}{Zistersdorf}{}\ledrightnote{\textcolor{pink}{Zistersdorf}}\pend{}\pstart{}\textcolor{pink}{N. Oe.}{}\ledrightnote{\textcolor{pink}{Niederösterreich}} –
                    \pend{}{\bigskip}\pstart
           \noindent{}{\pb}\textcolor{gray}{\textbf{Dr. Arthur Schnitzler}}\hfill 11/7 1915\pend
           \pstart
           \textcolor{gray}{\textbf{\textcolor{pink}{Wien XVIII. Sternwartestrasse 71}{}\ledrightnote{\textcolor{pink}{Sternwartestraße}}}}\pend
           \pstart
           Verehrter Herr Doctor, erst gestern Abend bin ich dazu geko{\geminationm}en Ihre \textcolor{green}{Komoedie}{}\ledrightnote{→\textcolor{green}{Gesellschaft [Eine Gaunerkomödie]}} zu lesen – in einem Zug, da sie mich
                    amusiert hat; technisch ist sie auch nicht übel – aber im ganzen ist es dann
                    eine etwas grobe und in ihrer \textcolor{gray}{Accentu}iertheit
                    unwahrscheinliche und recht willkürlich wirkende Sache, mit der nicht übermäßig
                        \introOben{}viel\introOben{} dichterische Ehren aufzuheben sind. I{\geminationm}erhin ist sie spielbar und ich denke, \textcolor{pink}{Residenzbühne}{}\ledrightnote{\textcolor{pink}{Kammerspiele Wien}} oder \textcolor{pink}{Neue Bühne}{}\ledrightnote{\textcolor{pink}{Neue Wiener Bühne}} würden sich gegen den Versuch nicht wehren. Daß
                    Sie jede einzelne Figur persönlich kennen, {\pb}will ich
                    gerne glauben – und jede einzelne wirkte am Ende, in irgend ein andres Stück
                    gestellt, lebendig wirken; – so auf einen Fleck zusa{\geminationm}engebracht, in theatralische Beziehun\textcolor{gray}{ge}n \substVorne{}\textsuperscript{auf}\substDazwischen{}zu\substHinten{}einander, zweifelt man gelegentlich auch an ihrer Lebenswahrheit. De{\geminationn} nichts ist rachsüchtiger als die Kunst – bis zur
                    Ungerechtigkeit! –\pend
           \pstart
           Seien Sie herzlich gegrüßt von Ihrem Sie sehr hochschätzenden{\\[\baselineskip]}\spacefill\mbox{Arthur Schnitzler}\pend
           \leftskip=0em{}\endnumbering\briefempfaengerindex{Adam, Robert@\textsc{Adam, Robert}!zzzSchnitzler, Arthur@\emph{von Arthur Schnitzler}!1915-07-111@{11. 7. 1915}|)be}\mylabel{h}  \normalsize

\doendnotes{C}
\bigskip
\vfill

\clearpage

\footnotesize

\lohead{\textsc{register}}

% Definiere theindex-Environment komplett neu ohne reledmac
\makeatletter
\renewenvironment{theindex}{%
  \section*{\indexname}%
  \setlength{\parindent}{0pt}%
  \setlength{\parskip}{0pt plus 0.3pt}%
  \let\item\@idxitem
}{%
  \clearpage
}
\makeatother

\IfFileExists{\jobname-pw.ind}{\input{\jobname-pw.ind}}{}

\end{document}

      