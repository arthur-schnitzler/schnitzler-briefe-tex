%% latex-korrekturansicht-vorspann.tex
%% Vorspann für die Korrekturansicht.
%% Lädt die gemeinsame Datei latex-vorspann.tex mit gesetztem Schalter.

\newif\ifkorrekturansicht
\korrekturansichttrue

\input{../tex-inputs/latex-vorspann}


               \section[Georg Brandes an Arthur Schnitzler, 3. 4. {[}1900{]}]{ Georg Brandes an Arthur Schnitzler, 3. 4. {[}1900{]}}\nopagebreak\mylabel{v}\rehead{ }\normalsize\beginnumbering\briefempfaengerindex{Schnitzler, Arthur@\textsc{Schnitzler, Arthur}!zzzBrandes, Georg@\emph{von Georg Brandes}!1900-04-034@{3. 4. {[}1900{]}}|(be} \toendnotes[C]{\smallbreak\pagebreak[2]} \Standort{CUL, Schnitzler, B 17.}
\physDesc{Brief, 1 Blatt, 1 Seite
\newline{}Handschrift: schwarze Tinte, lateinische Kurrent\newline{}Ordnung: von Schnitzler mit Bleistift die Jahreszahl hinzugefügt: »900«, von unbekannter Hand mit Bleistift nummeriert:
                                        »19« }\buchAbdrucke{\weitereDrucke{Georg Brandes, Arthur Schnitzler: \emph{Ein Briefwechsel}. Hg. Kurt Bergel. Bern: \emph{Francke} 1956, S. 80.} }\toendnotes[C]{\smallbreak}\pstart
           \raggedleft{}{\pb}\textcolor{pink}{Budapest Hotel Royal}{}\ledrightnote{\textcolor{pink}{Hotel Royal}}{\\}3 April\pend
           \pstart
           Liebster Freund Schnitzler\hspace*{2.5em}So gern ich Sie auf der Reise treffen möchte,
                    es wird mir nicht möglich. Ich habe nach längerem Sträuben die Einladung der
                    hiesigen Minister (\textcolor{blue}{Handels-}{}\ledrightnote{→\textcolor{blue}{Sandór Hegedüs}} und \textcolor{blue}{Ackerbau-Minister}{}\ledrightnote{→\textcolor{blue}{Ignác Darányi}}) angenommen, Donnerstag bis Sonntag auf Staatskosten
                        \textcolor{pink}{Ungarn}{}\ledrightnote{\textcolor{pink}{Ungarn}} zu bereisen und mir die
                    Provinzen zeigen zu lassen. Ob das amusant wird, weiss ich nicht, zweifle, aber
                    ich kann mir die Gelegenheit nicht entgehen \strikeout{z}
                    lassen, etwas zu lernen, das sich mir sonst nicht darbietet.\pend
           \pstart
           Wir sehen uns vielleicht noch auf meiner Rückreise durch \textcolor{pink}{Wien}{}\ledrightnote{\textcolor{pink}{Wien}}.\pend
           \pstart Ihr ergebener Freund \spacefill\mbox{Georg B}\pend{}\endnumbering\briefempfaengerindex{Schnitzler, Arthur@\textsc{Schnitzler, Arthur}!zzzBrandes, Georg@\emph{von Georg Brandes}!1900-04-034@{3. 4. {[}1900{]}}|)be}\mylabel{h}  \normalsize

\doendnotes{C}
\bigskip
\vfill

\clearpage

\footnotesize

\lohead{\textsc{register}}

% Definiere theindex-Environment komplett neu ohne reledmac
\makeatletter
\renewenvironment{theindex}{%
  \section*{\indexname}%
  \setlength{\parindent}{0pt}%
  \setlength{\parskip}{0pt plus 0.3pt}%
  \let\item\@idxitem
}{%
  \clearpage
}
\makeatother

\IfFileExists{\jobname-pw.ind}{\input{\jobname-pw.ind}}{}

\end{document}

      