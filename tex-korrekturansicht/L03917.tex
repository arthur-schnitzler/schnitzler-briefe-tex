%% latex-korrekturansicht-vorspann.tex
%% Vorspann für die Korrekturansicht.
%% Lädt die gemeinsame Datei latex-vorspann.tex mit gesetztem Schalter.

\newif\ifkorrekturansicht
\korrekturansichttrue

\input{../tex-inputs/latex-vorspann}


\section[Arthur Schnitzler an Theodor Herzl, {[}23. oder 24. 4. 1889?{]}]{L03917 Arthur Schnitzler an Theodor Herzl, {[}23. oder 24. 4. 1889?{]}}
\nopagebreak\mylabel{L03917v}
\rehead{ }\normalsize\beginnumbering\briefempfaengerindex{, @\textsc{, }!zzz, @\emph{von  }!1889-04-241@{{[}23. oder 24. 4. 1889?{]}}|(be}
\toendnotes[C]{\smallbreak\pagebreak[2]}\Standort{Jerusalem, Central Zionist Archives, H1\1926-7.}
\physDesc{Briefkarte, 665 Zeichen
\newline{}Handschrift: schwarze Tinte, deutsche Kurrent
\newline{}Ordnung: mit Bleistift von unbekannter Hand nummeriert: »(44) 1« }\toendnotes[C]{\smallbreak}
\pstart
           \noindent{}{\pb}Verehrteſter Herr Doktor! Ich erlaube mir heute eine Bitte an Sie!
               Sind Sie in der Lage, \label{K_L03917-1v}\edtext{Karten zur \textcolor{blue}{Judic}\pwindex{Judic, Anna 19.\,7.\,1849 Semur-en-Auxois – 15.\,4.\,1911 Golfe-Juan@\textsc{Judic, Anna} (19.\,7.\,1849 Semur-en-Auxois – 15.\,4.\,1911 Golfe-Juan), \emph{Schauspielerin, Sängerin}|pw}{}\ledrightnote{\textcolor{blue}{Anna Judic}}}{\lemma{\textnormal{\emph{Karten zur Judic}}}\Cendnote{\textnormal{Das
               Korrespondenzstück ist undatiert. Die förmliche Anrede und Schlussformel verorten es eindeutig in der frühen Zeit der Bekanntschaft. In
               dieser Phase ist \textcolor{blue}{Anna Judic}\pwindex{Judic, Anna 19.\,7.\,1849 Semur-en-Auxois – 15.\,4.\,1911 Golfe-Juan@\textsc{Judic, Anna} (19.\,7.\,1849 Semur-en-Auxois – 15.\,4.\,1911 Golfe-Juan), \emph{Schauspielerin, Sängerin}|pwk} nur zwischen 21. 4. 1889 und 
                  Sonntag, 5. 5. 1889 zu einem Gastspiel in \textcolor{pink}{Wien}\oindex{Wien@\textbf{Wien}, \emph{Verwaltungsgebiet}|pwk}. \textcolor{blue}{Schnitzler}
                  selbst besuchte am Mittwoch, den 1. 5. 1889 eine ihrer Aufführungen von 
                  \emph{\textcolor{green}{Fiacre 117}\pwindex{Millaud, Albert 13.\,1.\,1844 Paris – 23.\,10.\,1892 ebd.@\textsc{Millaud, Albert} (13.\,1.\,1844 Paris – 23.\,10.\,1892 ebd.), \emph{Schriftsteller}!Fiaker 117. Schwank in 3 Acten@\strich\emph{Fiaker 117. Schwank in 3 Acten}|pwk}\pwindex{Najac, Émile de 14.\,12.\,1828 Lorient – 11.\,4.\,1889 Paris@\textsc{Najac, Émile de} (14.\,12.\,1828 Lorient – 11.\,4.\,1889 Paris), \emph{Schriftsteller, Musiker, Dramatiker}!Fiaker 117. Schwank in 3 Acten@\strich\emph{Fiaker 117. Schwank in 3 Acten}|pwk}} und \emph{\textcolor{green}{Les Charbonniers}\pwindex{Gille, Philippe 18.\,12.\,1830 Paris – 19.\,3.\,1901 ebd.@\textsc{Gille, Philippe} (18.\,12.\,1830 Paris – 19.\,3.\,1901 ebd.), \emph{Schriftsteller, Publizist, Dramatiker}!Charbonniers. Opérette en un acte@\strich\emph{Les Charbonniers. Opérette en un acte}|pwk}}. Das \emph{\textcolor{green}{Tagebuch}\pwindex{Schnitzler, Arthur 15. 5. 1862 Wien – 21. 10. 1931 ebd.@\textsc{Schnitzler, Arthur} (15. 5. 1862 Wien – 21. 10. 1931 ebd.), \emph{Schriftsteller, Mediziner}!Tagebuch@\strich\emph{Tagebuch}|pwk}}
                  enthält für diese Tage keine Einträge. Am Samstag, dem 4. 5. 1889
                  wurde am \emph{\textcolor{brown}{Burgtheater}\orgindex{Burgtheater@Burgtheater|pwk}} (unter anderem) \emph{\textcolor{green}{Der Flüchtling}\pwindex{Herzl, Theodor 2.\,5.\,1860 Budapest – 3.\,7.\,1904 Edlach@\textsc{Herzl, Theodor} (2.\,5.\,1860 Budapest – 3.\,7.\,1904 Edlach), \emph{Schriftsteller, Journalist}!Flüchtling. Lustspiel in einem Aufzug@\strich\emph{Der Flüchtling. Lustspiel in einem Aufzug}|pwk}}
                  von \textcolor{blue}{Herzl}\pwindex{Herzl, Theodor 2.\,5.\,1860 Budapest – 3.\,7.\,1904 Edlach@\textsc{Herzl, Theodor} (2.\,5.\,1860 Budapest – 3.\,7.\,1904 Edlach), \emph{Schriftsteller, Journalist}|pwk} uraufgeführt. \textcolor{blue}{Schnitzler} besuchte die 
                  Vorstellung und es scheint schwer vorstellbar, dass er sich von \textcolor{blue}{Herzl}\pwindex{Herzl, Theodor 2.\,5.\,1860 Budapest – 3.\,7.\,1904 Edlach@\textsc{Herzl, Theodor} (2.\,5.\,1860 Budapest – 3.\,7.\,1904 Edlach), \emph{Schriftsteller, Journalist}|pwk} just für den Tag
                  eine Theaterkarte erbitten würde, für den \textcolor{blue}{Herzls}\pwindex{Herzl, Theodor 2.\,5.\,1860 Budapest – 3.\,7.\,1904 Edlach@\textsc{Herzl, Theodor} (2.\,5.\,1860 Budapest – 3.\,7.\,1904 Edlach), \emph{Schriftsteller, Journalist}|pwk}{ }\textcolor{green}{Stück}\pwindex{Herzl, Theodor 2.\,5.\,1860 Budapest – 3.\,7.\,1904 Edlach@\textsc{Herzl, Theodor} (2.\,5.\,1860 Budapest – 3.\,7.\,1904 Edlach), \emph{Schriftsteller, Journalist}!Flüchtling. Lustspiel in einem Aufzug@\strich\emph{Der Flüchtling. Lustspiel in einem Aufzug}|pwkv} angesetzt war. Folglich dürfte \textcolor{blue}{Schnitzler} 
                  um Karten für das vorletzte Wochenende (Donnerstag, 25. bis Sonntag, 28.) bitten. Da wiederum die ersten Rezensionen
                  des Gastspiels bereits erschienen waren, dürfte das Korrespondenzstück auf den 23. 4. 1889 oder
                  24. 4. 1889 zu datieren sein.}}}\label{K_L03917-1} zu
               verſchenken? Ich habe den Muth zu dieſer Frage, indem ich leſe, daſs es
               leer ſein ſoll. Im Falle Sie alſo eine gewiſſe Verfügungsmöglichkeit haben, verbinden
                  {\pb}Sie mich außerordentlich, we{\geminationn} Sie mir für einen oder
               den anderen Abend ein oder zwei \introOben{}\textsc{Parquet-}\introOben{}Sitze verſchaffen können. Beſonders ſympathiſch wäre mir \uline{Do{\geminationn}erstag}{ }\uline{Samstag} oder \uline{So{\geminationn}tag}. Nicht wahr Sie ſagen mir gefälligſt telephoniſch zwiſchen 2 und 3 Uhr \introOben{}–
                  (\textsc{event} bis 5 od 6)\introOben{} oder durch eine Karte Antwort? Und ſind nicht ungehalten
               über mein Erſuchen?\pend
           \pstart Mit herzlichem Gruße und Dank Ihr ergebner
               \spacefill\mbox{Dr. Arthur Schnitzler}\pend{}\selectlanguage{ngerman}\endnumbering\briefempfaengerindex{, @\textsc{, }!zzz, @\emph{von  }!1889-04-231@{{[}23. oder 24. 4. 1889?{]}}|)be}\mylabel{L03917h}
\begin{anhang}
\end{anhang}\normalsize

\doendnotes{C}
\bigskip
\vfill

\clearpage

\footnotesize

\lohead{\textsc{register}}

% Definiere theindex-Environment komplett neu ohne reledmac
\makeatletter
\renewenvironment{theindex}{%
  \section*{\indexname}%
  \setlength{\parindent}{0pt}%
  \setlength{\parskip}{0pt plus 0.3pt}%
  \let\item\@idxitem
}{%
  \clearpage
}
\makeatother

\IfFileExists{\jobname-pw.ind}{\input{\jobname-pw.ind}}{}

\end{document}

      