%% latex-korrekturansicht-vorspann.tex
%% Vorspann für die Korrekturansicht.
%% Lädt die gemeinsame Datei latex-vorspann.tex mit gesetztem Schalter.

\newif\ifkorrekturansicht
\korrekturansichttrue

\input{../tex-inputs/latex-vorspann}


\renewcommand{\erwaehntePersonen}{Personen: Caroline Kotter, Siegfried Trebitsch}
\renewcommand{\erwaehnteOrte}{Orte: Wien}
\renewcommand{\erwaehnteWerke}{}
\section[Felix Salten an Arthur Schnitzler, {[}zwischen 27. und 31. 10. 1903{]}]{Felix Salten an Arthur Schnitzler,
               {[}zwischen 27. und 31. 10. 1903{]}}
\nopagebreak\mylabel{v}
\rehead{ }\normalsize\beginnumbering\briefempfaengerindex{Schnitzler, Arthur@\textsc{Schnitzler, Arthur}!zzzSalten, Felix@\emph{von Felix Salten}!1@{{[}zwischen 27. und 31. 10. 1903{]}}|(be}
\toendnotes[C]{\smallbreak\pagebreak[2]}\Standort{CUL, Schnitzler, B 89, A 2.}
\physDesc{Brief, 1 Blatt, 1 Seite, 111 Zeichen
\newline{}Handschrift: Bleistift, lateinische Kurrent
\newline{}Schnitzler: mit Bleistift datiert: »Oct. {[}1{]}90\textcolor{gray}{3}.« 
\newline{}Ordnung: mit Bleistift von unbekannter Hand nummeriert: »175« }\toendnotes[C]{\smallbreak}
\pstart
           \noindent{}{\pb}Lieber,{ }\label{K_L03349-1v}\edtext{\textcolor{blue}{Trebitsch}{}\ledrightnote{\textcolor{blue}{Siegfried Trebitsch}} ist mir natürlich recht}{\lemma{\textnormal{\emph{Trebitsch … recht}}}\Cendnote{\textnormal{Das Korrespondenzstück ist undatiert. \textcolor{blue}{Schnitzler}
                  datiert es auf den Zeitraum »Oct. 90\textcolor{gray}{3}.«. Es
                  dürfte um das gemeinsame Treffen mit \textcolor{blue}{Trebitsch} am 1. 11. 1903 handeln. Damit wäre der Brief in der vorangehenden
                  Woche verfasst. Der ebenfalls undatierte Brief aus der Zeit [zwischen 26. und 30. 10. 1903] dürfte
                  sich 
                  ebenfalls auf
                  dieses Treffen beziehen und muss vorher gelaufen sein, weil eine dort fehlende
                  Auskunft über die Teilnahme der Tochter \textcolor{blue}{Caroline} nachgereicht wird. Damit lässt sich das Zeitfenster noch etwas
                  verkleinern.}}}\label{K_L03349-1h}. \textcolor{blue}{Lintscherl}{}\ledrightnote{\textcolor{blue}{Caroline Kotter}} bleibt zu
               Hause, denn sie muß schlafen gehen.\pend
           
\pstart
           Herzlichst {\\[\baselineskip]}Ihr {\\[\baselineskip]}\spacefill\mbox{S.}\pend
           \leftskip=0em{}\endnumbering\briefempfaengerindex{Schnitzler, Arthur@\textsc{Schnitzler, Arthur}!zzzSalten, Felix@\emph{von Felix Salten}!1903-10-271@{{[}zwischen 27. und 31. 10. 1903{]}}|)be}\mylabel{h}  \normalsize

\doendnotes{C}
\bigskip
\vfill

\clearpage

\footnotesize

\lohead{\textsc{register}}

% Definiere theindex-Environment komplett neu ohne reledmac
\makeatletter
\renewenvironment{theindex}{%
  \section*{\indexname}%
  \setlength{\parindent}{0pt}%
  \setlength{\parskip}{0pt plus 0.3pt}%
  \let\item\@idxitem
}{%
  \clearpage
}
\makeatother

\IfFileExists{\jobname-pw.ind}{\input{\jobname-pw.ind}}{}

\end{document}

      