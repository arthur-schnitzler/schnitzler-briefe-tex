%% latex-korrekturansicht-vorspann.tex
%% Vorspann für die Korrekturansicht.
%% Lädt die gemeinsame Datei latex-vorspann.tex mit gesetztem Schalter.

\newif\ifkorrekturansicht
\korrekturansichttrue

\input{../tex-inputs/latex-vorspann}


               \section[Thomas Mann an Arthur Schnitzler, 16. 10. 1911]{ Thomas Mann an Arthur Schnitzler, 16. 10. 1911}\nopagebreak\mylabel{v}\rehead{ }\normalsize\beginnumbering\briefempfaengerindex{Schnitzler, Arthur@\textsc{Schnitzler, Arthur}!zzzMann, Thomas@\emph{von Thomas Mann}!1911-10-162@{16. 10. 1911}|(be} \toendnotes[C]{\smallbreak\pagebreak[2]} \Standort{CUL, Schnitzler, B 67.}
\physDesc{Brief, 1 Blatt, 2 Seiten
\newline{}Handschrift: schwarze Tinte, deutsche Kurrent
\newline{}Schnitzler: 1) mit Bleistift beschriftet: »\textsc{Mann}« 2) mit rotem Buntstift eine Unterstreichung}\buchAbdrucke{\weitereDrucke{Hertha Krotkoff: \emph{Arthur Schnitzler – Thomas Mann: Briefe.} In: \emph{Modern Austrian Literature}, Jg. 7 (1974) Nr. 1/2, S. 15–16.} }\toendnotes[C]{\smallbreak}\pstart
           \raggedleft{}{\pb}\textcolor{pink}{München}{}\ledrightnote{\textcolor{pink}{München}} den
                            16. X. 1911.\pend
           \pstart{}Sehr verehrter Herr:\pend\pstart
           Es war mir eine beſondere Freude, am Morgen nach der \label{K_L02039_1v}\edtext{Première}{\lemma{\textnormal{\emph{Première}}}\Cendnote{\textnormal{Diese hatte am
                            14. 10. 1911 gleichzeitig in mehreren Städten
                        stattgefunden.}}}\label{K_L02039_1h}, noch ganz erfüllt von Ihrer Kunſt, das Buch des »\textcolor{green}{Weiten Landes}{}\ledrightnote{\textcolor{green}{Das weite Land. Tragikomödie in fünf Akten}}« von Ihrer eigenen Hand zu
                    empfangen. Ich danke Ihnen von Herzen. Ihr Stück hat hier tiefen Eindruck
                    gemacht, der Beifall am Schluſſe ruhte nicht, bis der \textcolor{blue}{Regiſſeur}{}\ledrightnote{→\textcolor{blue}{Friedrich Basil}} in Ihrem Namen gedankt hatte.
                    Die Aufführung war recht leidlich, \textcolor{blue}{Steinrück}{}\ledrightnote{\textcolor{blue}{Albert Steinrück}}
                    in ſeiner Art meiſter{\pb}haft, wenn auch
                    wohl nicht der Menſch, den Sie geſehen haben. Es fehlte die aeußere Weichheit,
                    die zu der gefährlichen Energie des Mannes ſo lebensvoll kontraſtieren müßte.
                    Dieſer letztere, der erotiſche Ernſt, war deſto eindrucksvoller betont. Mein \textcolor{blue}{Bruder}{}\ledrightnote{→\textcolor{blue}{Heinrich Mann}} und ich verbrachten
                    den Reſt des Abends \introOben{}mit\introOben{} den Hauptdarſtellern. Das
                    Telegramm »an Arthur« war allgemeines Herzensbedürfnis.\pend
           \pstart
           Ihr ergebener{\\[\baselineskip]}\spacefill\mbox{Thomas Mann.}\pend
           \leftskip=0em{}\endnumbering\briefempfaengerindex{Schnitzler, Arthur@\textsc{Schnitzler, Arthur}!zzzMann, Thomas@\emph{von Thomas Mann}!1911-10-162@{16. 10. 1911}|)be}\mylabel{h}  \normalsize

\doendnotes{C}
\bigskip
\vfill

\clearpage

\footnotesize

\lohead{\textsc{register}}

% Definiere theindex-Environment komplett neu ohne reledmac
\makeatletter
\renewenvironment{theindex}{%
  \section*{\indexname}%
  \setlength{\parindent}{0pt}%
  \setlength{\parskip}{0pt plus 0.3pt}%
  \let\item\@idxitem
}{%
  \clearpage
}
\makeatother

\IfFileExists{\jobname-pw.ind}{\input{\jobname-pw.ind}}{}

\end{document}

      