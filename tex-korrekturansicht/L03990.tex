%% latex-korrekturansicht-vorspann.tex
%% Vorspann für die Korrekturansicht.
%% Lädt die gemeinsame Datei latex-vorspann.tex mit gesetztem Schalter.

\newif\ifkorrekturansicht
\korrekturansichttrue

\input{../tex-inputs/latex-vorspann}


\section[Arthur Schnitzler: Widmungsexemplar von La Pénombre des ames an Berta Zuckerkandl, 30. 1. 1930]{L03990 Arthur Schnitzler: Widmungsexemplar von La Pénombre des ames an Berta Zuckerkandl, 30. 1. 1930}
\nopagebreak\mylabel{L03990v}
\rehead{ }\normalsize\beginnumbering\briefempfaengerindex{Zuckerkandl, Berta@\textsc{Zuckerkandl, Berta}!zzzSchnitzler, Arthur@\emph{von Arthur Schnitzler}!1930-01-301@{30. 1. 1930}|(be}
\toendnotes[C]{\smallbreak\pagebreak[2]}
\correspDesc{Versand  durch Arthur Schnitzler am 30. 1. 1930 in Wien
\newline{}Erhalt  durch Berta Zuckerkandl im Zeitraum [31. 1. 1930 – 4. 2. 1930?] \textbf{Ort fehlend} }\toendnotes[C]{\smallbreak}
\Standort{Wien, Österreichische Nationalbibliothek, ZUC.5.1.SchPén LIT MAG.}
\physDesc{Widmung am Schmutztitel, 78 Zeichen
\newline{}Handschrift: schwarze Tinte, lateinische Kurrent}
\pstart
           \noindent{}{\pb}Meiner verehrten Freundin\pend
           
\pstart
           Hofrätin Berta Zuckerkandl\pend
           \pstart Herzlichst
            \spacefill\mbox{ArthSchnitzler}\pend{}{\vspace{1\baselineskip}}
\pstart
           \centering{}\textcolor{gray}{\textbf{\begin{otherlanguage}{french}\textcolor{green}{LA PÉNOMBRE DES AMES}\pwindex{Schnitzler, Arthur 15. 5. 1862 Wien – 21. 10. 1931 ebd.@\textsc{Schnitzler, Arthur} (15. 5. 1862 Wien – 21. 10. 1931 ebd.), \emph{Schriftsteller, Mediziner}!pénombre des âmes@\strich\emph{La pénombre des âmes}|pw}{}\ledrightnote{\textcolor{green}{La pénombre des âmes}}\end{otherlanguage}}}\pend
           \selectlanguage{ngerman}\vspace{1em}{\vspace{1\baselineskip}}
\pstart
           \centering{}{\pb}\textcolor{gray}{\textbf{ARTHUR SCHNITZLER}}\pend
           
\pstart
           \centering{}\textcolor{gray}{\textbf{\begin{otherlanguage}{french}\textcolor{green}{LA{\\}PÉNOMBRE{\\}DES AMES}\pwindex{Schnitzler, Arthur 15. 5. 1862 Wien – 21. 10. 1931 ebd.@\textsc{Schnitzler, Arthur} (15. 5. 1862 Wien – 21. 10. 1931 ebd.), \emph{Schriftsteller, Mediziner}!pénombre des âmes@\strich\emph{La pénombre des âmes}|pw}{}\ledrightnote{\textcolor{green}{La pénombre des âmes}}\end{otherlanguage}}}\pend
           
\pstart
           \centering{}\textcolor{gray}{\textbf{\emph{\begin{otherlanguage}{french}Nouvelles traduites de l’allemand par\end{otherlanguage}}}}\pend
           
\pstart
           \centering{}\textcolor{gray}{\textbf{\textcolor{blue}{Suzanne CLAUSER}\pwindex{Clauser, Suzanne 16.\,5.\,1898 Wien – 11.\,9.\,1981 Paris@\textsc{Clauser, Suzanne} (16.\,5.\,1898 Wien – 11.\,9.\,1981 Paris), \emph{Schriftstellerin, Übersetzerin}|pw}{}\ledrightnote{\textcolor{blue}{Suzanne Clauser}}}}\pend
           {\vspace{1\baselineskip}}
\pstart
           \centering{}\textcolor{gray}{\textbf{\begin{otherlanguage}{french}LE CABINET COSMOPOLITE\end{otherlanguage}}}\pend
           
\pstart
           \centering{}\textcolor{gray}{\textbf{\begin{otherlanguage}{french}Tous droits réservés.\end{otherlanguage}}}\pend
           
\pstart
           \centering{}\textcolor{gray}{\textbf{\begin{otherlanguage}{french}\textcolor{brown}{LIBRAIRIE STOCK}\orgindex{Éditions Stock@Éditions Stock|pw}{}\ledrightnote{\textcolor{brown}{Éditions Stock}}\end{otherlanguage}}}\pend
           
\pstart
           \centering{}\textcolor{gray}{\textbf{\begin{otherlanguage}{french}\textcolor{blue}{DELAMAIN}\pwindex{Delamain, Maurice 28.\,4.\,1883 Jarnac – 2.\,5.\,1974 Paris@\textsc{Delamain, Maurice} (28.\,4.\,1883 Jarnac – 2.\,5.\,1974 Paris), \emph{Kritiker, Rechtsanwalt, Verleger}|pw}\pwindex{Delamain, Jacques 20.\,12.\,1874 Jarnac – 5.\,2.\,1953 Saint-Brice@\textsc{Delamain, Jacques} (20.\,12.\,1874 Jarnac – 5.\,2.\,1953 Saint-Brice), \emph{Verleger}|pw}{}\ledrightnote{\textcolor{blue}{Maurice Delamain}{\newline}\textcolor{blue}{Jacques Delamain}} ET \textcolor{blue}{BOUTELLEAU}\pwindex{Chardonne, Jacques 2.\,1.\,1884 Barbezieux-Saint-Hilaire – 29.\,5.\,1968 La Frette-sur-Seine@\textsc{Chardonne, Jacques} (2.\,1.\,1884 Barbezieux-Saint-Hilaire – 29.\,5.\,1968 La Frette-sur-Seine), \emph{Schriftsteller, Verleger}|pw}{}\ledrightnote{\textcolor{blue}{Jacques Chardonne}}\end{otherlanguage}}}\pend
           \selectlanguage{ngerman}\endnumbering\briefempfaengerindex{Zuckerkandl, Berta@\textsc{Zuckerkandl, Berta}!zzzSchnitzler, Arthur@\emph{von Arthur Schnitzler}!1930-01-301@{30. 1. 1930}|)be}\mylabel{L03990h}
\begin{anhang}
\end{anhang}\normalsize

\doendnotes{C}
\bigskip
\vfill

\clearpage

\footnotesize

\lohead{\textsc{register}}

% Definiere theindex-Environment komplett neu ohne reledmac
\makeatletter
\renewenvironment{theindex}{%
  \section*{\indexname}%
  \setlength{\parindent}{0pt}%
  \setlength{\parskip}{0pt plus 0.3pt}%
  \let\item\@idxitem
}{%
  \clearpage
}
\makeatother

\IfFileExists{\jobname-pw.ind}{\input{\jobname-pw.ind}}{}

\end{document}

      