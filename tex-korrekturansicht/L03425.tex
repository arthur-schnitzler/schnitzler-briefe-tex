%% latex-korrekturansicht-vorspann.tex
%% Vorspann für die Korrekturansicht.
%% Lädt die gemeinsame Datei latex-vorspann.tex mit gesetztem Schalter.

\newif\ifkorrekturansicht
\korrekturansichttrue

\input{../tex-inputs/latex-vorspann}


\renewcommand{\erwaehntePersonen}{Personen: Julius von Gans-Ludassy, Gustav Harpner, Felix Salten}
\renewcommand{\erwaehnteInstitutionen}{Institutionen: Ullstein Verlag}
\renewcommand{\erwaehnteOrte}{Orte: Berlin, Palasthotel Berlin, Wien}
\renewcommand{\erwaehnteWerke}{Werke: ?? [Ludassy will von Salten erpresst worden sein]}
\section[ Felix Salten an Arthur Schnitzler, 17. 5. 1906]{Felix Salten an Arthur Schnitzler, 17. 5. 1906}
\nopagebreak\mylabel{v}
\rehead{ }\normalsize\beginnumbering\briefempfaengerindex{Schnitzler, Arthur@\textsc{Schnitzler, Arthur}!zzzSalten, Felix@\emph{von Felix Salten}!1906-05-171@{17. 5. 1906}|(be}
\toendnotes[C]{\smallbreak\pagebreak[2]}\Standort{CUL, Schnitzler, B 89, B 1.}
\physDesc{Brief, 1 Blatt, 2 Seiten, 1454 Zeichen
\newline{}Handschrift: schwarze Tinte, lateinische Kurrent
\newline{}Ordnung: mit Bleistift von unbekannter Hand nummeriert: »216« }\toendnotes[C]{\smallbreak}
\pstart
           \raggedleft{}{\pb}\textcolor{pink}{Berlin}{}\ledrightnote{\textcolor{pink}{Berlin}}, 17. V. 06.\pend
           
\pstart
           Lieber, –
               da Sie rasche Auskunft wünschen (warum?) in aller Kürze: ich
               höre von meinem \label{K_L03425-1v}\edtext{\textcolor{blue}{Anwalt}{}\ledrightnote{{$\rightarrow$}\textcolor{blue}{Gustav Harpner}}}{\lemma{\textnormal{\emph{Anwalt}}}\Cendnote{\textnormal{Es dürfte sich um \textcolor{blue}{Gustav Harpner} handeln, vgl. Felix Salten an Arthur Schnitzler, [20.? 10. 1906]. Siehe zum Prozess Felix Salten an Arthur Schnitzler, 9. 3. 1906.}}}\label{K_L03425-1h}, dass Herr D\textsuperscript{r}{ }\textcolor{blue}{v. Ludaßy}{}\ledrightnote{\textcolor{blue}{Julius von Gans-Ludassy}} sich jetzt hinter die subjective
               Verjährung verkriechen will; d. h. er macht geltend: der bewußte \textcolor{green}{Angriff}{}\ledrightnote{\textcolor{green}{?? [Ludassy will von Salten erpresst worden sein]}} sei wol innerhalb der gesetzlichen Frist nach seinem
                  \uline{Erscheinen} geklagt worden, sei aber sechs Monate
                  \uline{vor} seinem Erscheinen \uline{geschrieben} worden. Er verlangt, dass man die Zeit so misst, dass man von
               dem Tag an rechnet, an welchem die Tat \uline{begangen}
               wurde! Da käme ihm dann der Schutz der Verjährung zu gute, und er hätte mich straflos
               der Bestechlichkeit beschuldigt, weil ich ihn erst verklagte, als ich seinen \textcolor{green}{Artikel}{}\ledrightnote{{$\rightarrow$}\textcolor{green}{?? [Ludassy will von Salten erpresst worden sein]}} gedruckt las, und
               nicht schon, als er ihn aufgeschrieben hatte. »Es wär’ not« – man müßt’ alle 14 Tag
               zu \textcolor{blue}{Ludaßy}{}\ledrightnote{\textcolor{blue}{Julius von Gans-Ludassy}} fragen schicken: »Haben Sie nicht
               eine Gemeinheit gegen mich begangen?« Ob er mit dieser Bemühung durchdringt, weiß ich
               nicht.\pend
           
\pstart
           Hier hat Herr D\textsuperscript{r}{ }\textcolor{blue}{v. Ludaßy}{}\ledrightnote{\textcolor{blue}{Julius von Gans-Ludassy}} an \textcolor{brown}{Ullstein}{}\ledrightnote{\textcolor{brown}{Ullstein Verlag}}s telegrafirt: »Habe Ihnen Verlagsproject vorzuschlagen. Bitte mir
               unter \uline{Vermeidung Saltens} mitzuteilen, wann ich Sie
               sprechen kann. Wohne \textcolor{pink}{Palasthotel}{}\ledrightnote{\textcolor{pink}{Palasthotel Berlin}}. \textcolor{blue}{L.}{}\ledrightnote{\textcolor{blue}{Julius von Gans-Ludassy}}« \textcolor{brown}{Ullstein}{}\ledrightnote{\textcolor{brown}{Ullstein Verlag}}s haben mir die Depesche sofort gezeigt.\pend
           
\pstart
           Zu diesen Dingen kann ich mich wol jeder Bemerkung enthalten.\pend
           
\pstart
           Nun aber genug. Ich will auch nichts von anderen Dingen {\pb}schreiben, die mir wie Ihnen
               näher u. lieber sind. Es widerstrebt mir aufrichtig, sie in einem Zug mit \textcolor{blue}{Ludaßy}{}\ledrightnote{\textcolor{blue}{Julius von Gans-Ludassy}} zu erörtern. Ohnehin störts mich genug,
               dass dieses Schwein sich immer durch unsere Briefe wälzt.\pend
           
\pstart
           herzlichst {\\[\baselineskip]}Ihr \spacefill\mbox{Salten}\pend
           \leftskip=0em{}\endnumbering\briefempfaengerindex{Schnitzler, Arthur@\textsc{Schnitzler, Arthur}!zzzSalten, Felix@\emph{von Felix Salten}!1906-05-171@{17. 5. 1906}|)be}\mylabel{h}  \normalsize

\doendnotes{C}
\bigskip
\vfill

\clearpage

\footnotesize

\lohead{\textsc{register}}

% Definiere theindex-Environment komplett neu ohne reledmac
\makeatletter
\renewenvironment{theindex}{%
  \section*{\indexname}%
  \setlength{\parindent}{0pt}%
  \setlength{\parskip}{0pt plus 0.3pt}%
  \let\item\@idxitem
}{%
  \clearpage
}
\makeatother

\IfFileExists{\jobname-pw.ind}{\input{\jobname-pw.ind}}{}

\end{document}

      