%% latex-korrekturansicht-vorspann.tex
%% Vorspann für die Korrekturansicht.
%% Lädt die gemeinsame Datei latex-vorspann.tex mit gesetztem Schalter.

\newif\ifkorrekturansicht
\korrekturansichttrue

\input{../tex-inputs/latex-vorspann}


\renewcommand{\erwaehntePersonen}{Personen: Felix Salten}
\renewcommand{\erwaehnteOrte}{Orte: Attersee, Berghof, Semmering, Sternwartestraße 71, Unterach am Attersee, Wien, XVIII., Währing}
\renewcommand{\erwaehnteWerke}{Werke: Die Zeit, Künstler sollen reden}
\section[ Arthur Schnitzler an Felix Salten, 27. 6. 1910]{Arthur Schnitzler an Felix Salten, 27. 6. 1910}
\nopagebreak\mylabel{v}
\rehead{ }\normalsize\beginnumbering\briefempfaengerindex{Salten, Felix@\textsc{Salten, Felix}!zzzSchnitzler, Arthur@\emph{von Arthur Schnitzler}!1910-06-271@{27. 6. 1910}|(be}
\toendnotes[C]{\smallbreak\pagebreak[2]}\Standort{Wienbibliothek im Rathaus, ZPH 1681, 2.1.516.}
\physDesc{Postkarte, 374 Zeichen
\newline{}Handschrift: 1) Bleistift, deutsche Kurrent\hspace{1em}2) Bleistift, lateinische Kurrent (\noindent{}Adresse)\hspace{1em}
\newline{}Versand: Stempel: »\nobreak{}\oindex{XVIII., Waehring@\textbf{XVIII., Währing}, \emph{A.ADM3}|pwk}1\textcolor{gray}{8}/\textsubscript{1} Wien 110, 27. VI. 10, 9 \textcolor{gray}{V}\nobreak{}«.  
\newline{}Ordnung: mit Bleistift von unbekannter Hand nummeriert: »3« }\toendnotes[C]{\smallbreak}\pstart{}{\pb}Hrn Felix Salten\pend{}\pstart{}\textcolor{pink}{Unterach}{}\ledrightnote{\textcolor{pink}{Unterach am Attersee}}\pend{}\pstart{}am \textcolor{pink}{Attersee}{}\ledrightnote{\textcolor{pink}{Attersee}}\pend{}\pstart{}\textcolor{pink}{Berghof}{}\ledrightnote{\textcolor{pink}{Berghof}}.\pend{}
{\bigskip}
\pstart
           \noindent{}{\pb}lieber, ich glaube nicht, dſs wir vor Ende
                  Juli werden \label{K_L03016-1v}\edtext{\textcolor{gray}{ü}berſiedeln}{\lemma{\textnormal{\emph{überſiedeln}}}\Cendnote{\textnormal{Der
                  Umzug in die \textcolor{pink}{Sternwartestraße 71} begann am
                     13. 7. 1910.}}}\label{K_L03016-1h} kö{\geminationn}en, \label{K_L03016-2v}\edtext{Anfang Juli gehn wir für ein paar Tage auf den \textcolor{pink}{Se{\geminationm}ering}{}\ledrightnote{\textcolor{pink}{Semmering}}}{\lemma{\textnormal{\emph{Anfang … Semmering}}}\Cendnote{\textnormal{\textcolor{blue}{Schnitzler} hielt sich zwischen 6. 7. 1910 und 10. 7. 1910 am \textcolor{pink}{Semmering} auf.}}}\label{K_L03016-2h}. – \pend
           
\pstart
           Ich \label{K_L03016-3v}\edtext{geſtriges \textsc{\textcolor{green}{Feu{[}i{]}lleton}{}\ledrightnote{\textcolor{green}{Künstler sollen reden}}}}{\lemma{\textnormal{\emph{geſtriges Feuilleton}}}\Cendnote{\textnormal{\textcolor{blue}{Felix Salten}: \emph{\textcolor{green}{Künstler sollen reden}}. In: \emph{\textcolor{green}{Die Zeit}}, Jg. 9, Nr. 2.784, 26. 6. 1910, Morgenblatt, S. 1–2.}}}\label{K_L03016-3h} – köſtlich! – Eins
               von denen, aus deren Tiefe es noch ſchöner glitzerte als auf der Fläche oben, die
               wahrhaftig auch nicht ohne iſt.\pend
           
\pstart
           Viele Grüße von uns zu Ihnen. {\\[\baselineskip]}Herzlichſt Ihr {\\[\baselineskip]}\spacefill\mbox{A.}\pend
           \leftskip=0em{}
\pstart
           27. 6. 10\pend
           \endnumbering\briefempfaengerindex{Salten, Felix@\textsc{Salten, Felix}!zzzSchnitzler, Arthur@\emph{von Arthur Schnitzler}!1910-06-271@{27. 6. 1910}|)be}\mylabel{h}  \normalsize

\doendnotes{C}
\bigskip
\vfill

\clearpage

\footnotesize

\lohead{\textsc{register}}

% Definiere theindex-Environment komplett neu ohne reledmac
\makeatletter
\renewenvironment{theindex}{%
  \section*{\indexname}%
  \setlength{\parindent}{0pt}%
  \setlength{\parskip}{0pt plus 0.3pt}%
  \let\item\@idxitem
}{%
  \clearpage
}
\makeatother

\IfFileExists{\jobname-pw.ind}{\input{\jobname-pw.ind}}{}

\end{document}

      