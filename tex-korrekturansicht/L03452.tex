%% latex-korrekturansicht-vorspann.tex
%% Vorspann für die Korrekturansicht.
%% Lädt die gemeinsame Datei latex-vorspann.tex mit gesetztem Schalter.

\newif\ifkorrekturansicht
\korrekturansichttrue

\input{../tex-inputs/latex-vorspann}


\renewcommand{\erwaehntePersonen}{Personen: Paul Goldmann, Olga Schnitzler}
\renewcommand{\erwaehnteOrte}{Orte: Alpen, Campo Carlo Magno, Edmund-Weiß-Gasse 7, Madonna di Campiglio, Wien}
\renewcommand{\erwaehnteWerke}{}
\section[ Paul Goldmann an Arthur Schnitzler, 25. 8. {[}1904{]}]{Paul Goldmann an Arthur Schnitzler, 25. 8. {[}1904{]}}
\nopagebreak\mylabel{v}
\rehead{ }\normalsize\beginnumbering\briefempfaengerindex{Schnitzler, Arthur@\textsc{Schnitzler, Arthur}!zzzGoldmann, Paul@\emph{von Paul Goldmann}!1904-08-251@{25. 8. {[}1904{]}}|(be}
\toendnotes[C]{\smallbreak\pagebreak[2]}\Standort{DLA, A:Schnitzler, HS.NZ85.1.3174.}
\physDesc{Bildpostkarte, 247 Zeichen
\newline{}Handschrift: 1) schwarze Tinte, deutsche Kurrent\hspace{1em}2) schwarze Tinte, lateinische Kurrent (\noindent{}Adresse)\hspace{1em}
\newline{}Versand: Stempel: »\nobreak{}\oindex{Madonna di Campiglio@\textbf{Madonna di Campiglio}, \emph{P.PPL}|pwk}\textcolor{gray}{Madonna di Campiglio}\nobreak{}«.  
\newline{}Schnitzler: mit Bleistift das Jahr »904« vermerkt }\toendnotes[C]{\smallbreak}\pstart{}{\pb}Herrn\pend{}\pstart{}Dr. Arthur Schnitzler\pend{}\pstart{}\textcolor{pink}{Wien}{}\ledrightnote{\textcolor{pink}{Wien}}\pend{}\pstart{}\textcolor{pink}{XVIII. Spöttelgaſse 7}{}\ledrightnote{\textcolor{pink}{Edmund-Weiß-Gasse 7}}.\pend{}
{\bigskip}
\pstart
           \noindent{}\centering{}{\pb}\textcolor{gray}{\textbf{\textcolor{pink}{Campo di Carlo Magno}{}\ledrightnote{\textcolor{pink}{Campo Carlo Magno}}}}\pend
           
\pstart
           \raggedleft{}25. Auguſt. \pend
           
\pstart
           \label{K_L03452-1v}\edtext{M. l.}{\lemma{\textnormal{\emph{M. l.}}}\Cendnote{\textnormal{Mein lieber}}}\label{K_L03452-1h} Freund, \label{K_L03452-2v}\edtext{kennſt Du \textsc{\textcolor{pink}{Madonna di Campiglio}{}\ledrightnote{\textcolor{pink}{Madonna di Campiglio}}}}{\lemma{\textnormal{\emph{kennſt … Campiglio}}}\Cendnote{\textnormal{\textcolor{blue}{Schnitzler} hat im Vorjahr vom 17. 8. 1903 auf den
                        18. 8. 1903 in
                     dem Ort übernachtet und war von da aus zu einem Treffen mit \textcolor{blue}{Goldmann} geradelt.}}}\label{K_L03452-2h}? Ein herrlicher Ort. Köſtlichſte Wald- und \textcolor{pink}{Alpen}{}\ledrightnote{\textcolor{pink}{Alpen}}luft. Hätteſt Du nicht Luſt \label{K_L03452-3v}\edtext{hierherzukommen}{\lemma{\textnormal{\emph{hierherzukommen}}}\Cendnote{\textnormal{nicht geschehen}}}\label{K_L03452-3h}? Herzliche Grüße Dir und Deiner \textcolor{blue}{Frau}{}\ledrightnote{{$\rightarrow$}\textcolor{blue}{Olga Schnitzler}}\pend
           \pstart Dein \spacefill\mbox{Paul Goldmann}\pend{}\endnumbering\briefempfaengerindex{Schnitzler, Arthur@\textsc{Schnitzler, Arthur}!zzzGoldmann, Paul@\emph{von Paul Goldmann}!1904-08-251@{25. 8. {[}1904{]}}|)be}\mylabel{h}  \normalsize

\doendnotes{C}
\bigskip
\vfill

\clearpage

\footnotesize

\lohead{\textsc{register}}

% Definiere theindex-Environment komplett neu ohne reledmac
\makeatletter
\renewenvironment{theindex}{%
  \section*{\indexname}%
  \setlength{\parindent}{0pt}%
  \setlength{\parskip}{0pt plus 0.3pt}%
  \let\item\@idxitem
}{%
  \clearpage
}
\makeatother

\IfFileExists{\jobname-pw.ind}{\input{\jobname-pw.ind}}{}

\end{document}

      