%% latex-korrekturansicht-vorspann.tex
%% Vorspann für die Korrekturansicht.
%% Lädt die gemeinsame Datei latex-vorspann.tex mit gesetztem Schalter.

\newif\ifkorrekturansicht
\korrekturansichttrue

\input{../tex-inputs/latex-vorspann}


\renewcommand{\erwaehntePersonen}{Personen: Felix Salten, Olga Schnitzler}
\renewcommand{\erwaehnteOrte}{Orte: Apollo-Theater, Armbrustergasse, Colosseum, Edmund-Weiß-Gasse 7, I., Innere Stadt, Ronacher, Wien, XIX., Döbling, XVIII., Währing}
\renewcommand{\erwaehnteWerke}{}
\section[ Felix Salten an Arthur Schnitzler, 24. 3. 1908]{Felix Salten an Arthur Schnitzler, 24. 3. 1908}
\nopagebreak\mylabel{v}
\rehead{ }\normalsize\beginnumbering\briefempfaengerindex{Schnitzler, Arthur@\textsc{Schnitzler, Arthur}!zzzSalten, Felix@\emph{von Felix Salten}!1908-03-241@{24. 3. 1908}|(be}
\toendnotes[C]{\smallbreak\pagebreak[2]}\Standort{CUL, Schnitzler, B 89, B 1.}
\physDesc{Postkarte, 360 Zeichen
\newline{}Handschrift: schwarze Tinte, lateinische Kurrent
\newline{}Versand: Stempel: »\nobreak{}\oindex{I., Innere Stadt@\textbf{I., Innere Stadt}, \emph{A.ADM3}|pwk}1/\textsubscript{1} Wien 6, 24. III. {[}0{]}8, 6\nobreak{}«.  
\newline{}Schnitzler: mit Bleistift datiert: »26/3 0\textcolor{gray}{8}« und Vermerk: »\textsc{S}{[}alten{]}.
                                 « 
\newline{}Ordnung: mit Bleistift von unbekannter Hand nummeriert: »243« }\toendnotes[C]{\smallbreak}\pstart{}{\pb}Salten, \textcolor{pink}{Wien XIX.}{}\ledrightnote{\textcolor{pink}{XIX., Döbling}}\pend{}\pstart{}\textcolor{pink}{Armbrustergaße 6}{}\ledrightnote{\textcolor{pink}{Armbrustergasse}}\pend{}
{\bigskip}\pstart{}Herrn D\textsuperscript{r} Arthur Schnitzler\pend{}\pstart{}\textcolor{pink}{Wien XVIII. Währing}{}\ledrightnote{\textcolor{pink}{XVIII., Währing}}\pend{}\pstart{}\textcolor{pink}{Spöttelgaße 7}{}\ledrightnote{\textcolor{pink}{Edmund-Weiß-Gasse 7}}\pend{}
{\bigskip}
\pstart
           \raggedleft{}{\pb}Dienstag.\pend
           
\pstart{}Lieber,\pend
\pstart
           wollen wir nicht \label{K_L03493-1v}\edtext{dieser Tage einmal
                  beisa{\geminationm}en sein}{\lemma{\textnormal{\emph{dieser … sein}}}\Cendnote{\textnormal{Siehe A. S.: \emph{Tagebuch}, 30. 3. 1908. Gemeinsam im \textcolor{pink}{Theater} waren sie das nächste Mal am
                     2. 4. 1908.}}}\label{K_L03493-1h}? Vielleicht benachrichtigen Sie mich, wenn Sie mit Ihrer \textcolor{blue}{Frau}{}\ledrightnote{{$\rightarrow$}\textcolor{blue}{Olga Schnitzler}} einmal im Konzert oder im
               Theater sind, und wir essen dann zusammen. Oder wir gehen einmal alle in’s \textcolor{pink}{Apollo}{}\ledrightnote{\textcolor{pink}{Apollo-Theater}}, \textcolor{pink}{Kolosseum}{}\ledrightnote{\textcolor{pink}{Colosseum}} od. dergl.?\pend
           
\pstart
           Herzlichst{\\[\baselineskip]}Ihr \spacefill\mbox{Salten}\pend
           \leftskip=0em{}\endnumbering\briefempfaengerindex{Schnitzler, Arthur@\textsc{Schnitzler, Arthur}!zzzSalten, Felix@\emph{von Felix Salten}!1908-03-241@{24. 3. 1908}|)be}\mylabel{h}  \normalsize

\doendnotes{C}
\bigskip
\vfill

\clearpage

\footnotesize

\lohead{\textsc{register}}

% Definiere theindex-Environment komplett neu ohne reledmac
\makeatletter
\renewenvironment{theindex}{%
  \section*{\indexname}%
  \setlength{\parindent}{0pt}%
  \setlength{\parskip}{0pt plus 0.3pt}%
  \let\item\@idxitem
}{%
  \clearpage
}
\makeatother

\IfFileExists{\jobname-pw.ind}{\input{\jobname-pw.ind}}{}

\end{document}

      