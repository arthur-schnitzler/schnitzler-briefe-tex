%% latex-korrekturansicht-vorspann.tex
%% Vorspann für die Korrekturansicht.
%% Lädt die gemeinsame Datei latex-vorspann.tex mit gesetztem Schalter.

\newif\ifkorrekturansicht
\korrekturansichttrue

\input{../tex-inputs/latex-vorspann}


               \section[Arthur Schnitzler: Widmungsexemplar Der Schleier der Beatrice für Hugo von Hofmannsthal, 28. 2. 1901]{ Arthur Schnitzler: Widmungsexemplar Der Schleier der Beatrice für
                    Hugo von Hofmannsthal, 28. 2. 1901}\nopagebreak\mylabel{v}\rehead{ }\normalsize\beginnumbering\briefempfaengerindex{Hofmannsthal, Hugo von@\textsc{Hofmannsthal, Hugo von}!zzzSchnitzler, Arthur@\emph{von Arthur Schnitzler}!1901-02-281@{28. 2. 1901}|(be} \toendnotes[C]{\smallbreak\pagebreak[2]} \Standort{FDH, FDH 3240.}
\physDesc{Widmung am Vortitel
\newline{}Handschrift: schwarze Tinte, deutsche Kurrent}\buchAbdrucke{\weitereDrucke{Hugo von Hofmannsthal: \emph{Bibliothek}. Hg. Ellen Ritter † in Zusammenarbeit mit Dalia Bukauskaité und
                                Konrad Heumann. Frankfurt am Main: \emph{S. Fischer} 2011, S. 606 (Sämtliche Werke. Kritische Ausgabe, XL).} }\toendnotes[C]{\smallbreak}\pstart
           \noindent{}{\pb}meinem lieben
                    Hugo\pend
           \pstart \spacefill\mbox{Arthur}\pend{}\pstart
           \textcolor{pink}{Wien}{}\ledrightnote{\textcolor{pink}{Wien}}{ }28. 2. 901.\pend
           {\bigskip}\pstart
           \noindent{}\centering{}{\pb}\textcolor{gray}{\textbf{\textcolor{green}{Der Schleier der Beatrice}{}\ledrightnote{\textcolor{green}{Der Schleier der Beatrice. Schauspiel in fünf Akten}}}}\pend
           \pstart
           \noindent{}\centering{}\textcolor{gray}{\textbf{Schauſpiel in fünf Akten}}{\\}\textcolor{gray}{\textbf{von}}{\\}\textcolor{gray}{\textbf{\textbf{Arthur Schnitzler}}}\pend
           {\bigskip}\pstart
           \noindent{}\centering{}\textcolor{gray}{\textbf{\textcolor{pink}{\so{Berlin}}{}\ledrightnote{\textcolor{pink}{Berlin}}}}\pend
           \pstart
           \noindent{}\centering{}\textcolor{gray}{\textbf{\textcolor{brown}{S. Fiſcher, Verlag}{}\ledrightnote{\textcolor{brown}{S. Fischer Verlag}}}}\pend
           \pstart
           \noindent{}\centering{}\textcolor{gray}{\textbf{\label{K_L01100_1v}\edtext{1901}{\lemma{\textnormal{\emph{1901}}}\Cendnote{\textnormal{am 30. 1. 1901
                                vom \emph{\textcolor{green}{Börsenblatt für den deutschen
                                    Buchhandel}} als Neuerscheinung gemeldet}}}\label{K_L01100_1h}}}\pend
           \endnumbering\briefempfaengerindex{Hofmannsthal, Hugo von@\textsc{Hofmannsthal, Hugo von}!zzzSchnitzler, Arthur@\emph{von Arthur Schnitzler}!1901-02-281@{28. 2. 1901}|)be}\mylabel{h}  \normalsize

\doendnotes{C}
\bigskip
\vfill

\clearpage

\footnotesize

\lohead{\textsc{register}}

% Definiere theindex-Environment komplett neu ohne reledmac
\makeatletter
\renewenvironment{theindex}{%
  \section*{\indexname}%
  \setlength{\parindent}{0pt}%
  \setlength{\parskip}{0pt plus 0.3pt}%
  \let\item\@idxitem
}{%
  \clearpage
}
\makeatother

\IfFileExists{\jobname-pw.ind}{\input{\jobname-pw.ind}}{}

\end{document}

      