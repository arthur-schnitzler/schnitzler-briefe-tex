%% latex-korrekturansicht-vorspann.tex
%% Vorspann für die Korrekturansicht.
%% Lädt die gemeinsame Datei latex-vorspann.tex mit gesetztem Schalter.

\newif\ifkorrekturansicht
\korrekturansichttrue

\input{../tex-inputs/latex-vorspann}


               \section[Arthur Schnitzler an Fedor Mamroth, 7. 12. 1894]{ Arthur Schnitzler an Fedor Mamroth, 7. 12. 1894}\nopagebreak\mylabel{v}\rehead{ }\normalsize\beginnumbering\briefempfaengerindex{Mamroth, Fedor@\textsc{Mamroth, Fedor}!zzzSchnitzler, Arthur@\emph{von Arthur Schnitzler}!1894-12-071@{7. 12. 1894}|(be} \toendnotes[C]{\smallbreak\pagebreak[2]} \Standort{YCGL, MSS 31.}
\physDesc{Brief, 1 Blatt, 2 Seiten
\newline{}Handschrift: schwarze Tinte, deutsche Kurrent\newline{}Ordnung: 1) mit blauer Tinte von unbekannter Hand wurde die Unterschrift
                                 ›entziffert‹: »Schnitzler« 2) mit Bleistift von unbekannter Hand wurde bei der Entzifferung
                                 des Nachnamens der Vorname »Arthur« ergänzt.3) mit Bleistift von unbekannter Hand
                                    nummeriert: »421« und
                                 Vermerk: »1K«\newline{}Zusatz: Als Empfänger ist Fedor Mamroth anzunehmen, den Schnitzler
                                 bereits vor dessen Engagement für die \textcolor{brown}{Frankfurter Zeitung} kennengelernt hatte. Der
                                 Brief wird unter jenen Schnitzlers an \textcolor{blue}{Richard Beer-Hofmann} aufbewahrt. Erklären
                                 ließe sich dies etwa damit, dass es sich um einen Briefentwurf und
                                 nicht den tatsächlich gesandten Brief handeln könnte, oder dass \textcolor{blue}{Beer-Hofmann} für die
                                 Übermittlung zuständig war und hier etwas schief lief. }\toendnotes[C]{\smallbreak}\pstart{}{\pb}Verehrteſter Herr Doktor,\pend\pstart
           es iſt mir ein Bedürfnis Ihnen für die liebenswürdige Raſchheit, mit welcher Sie die
                  \label{K_L00409-1v}\edtext{\textcolor{green}{Beſprechung}{}\ledrightnote{→\textcolor{green}{Belletristische Rundschau}}}{\lemma{\textnormal{\emph{Beſprechung}}}\Cendnote{\textnormal{\textcolor{blue}{J. Schwarz}: \emph{\textcolor{green}{Belletristische Rundschau}}. In: \emph{\textcolor{green}{Frankfurter Zeitung}}, Nr. 336, 4. 12. 1894,
                     S. 1–3.}}}\label{K_L00409-1h} meines letzten \textcolor{green}{Buches}{}\ledrightnote{→\textcolor{green}{Sterben. Novelle}} in der \textcolor{green}{Frkf.
                  Ztg.}{}\ledrightnote{\textcolor{green}{Frankfurter Zeitung}} erſcheinen ließen, aufs wärmſte zu danken. Darf ich Sie auch bitten, dem
                  \textcolor{blue}{Autor}{}\ledrightnote{→\textcolor{blue}{J. Schwarz}} des \textcolor{green}{Feuilletons}{}\ledrightnote{→\textcolor{green}{Belletristische Rundschau}} gütigſt mitzutheilen, wie ſehr
               mich {\pb}die ſo erſtaunlich tiefen und warmen Worte
               gefreut haben, die er dem \textcolor{green}{Buch}{}\ledrightnote{→\textcolor{green}{Sterben. Novelle}}
               gewidmet hat? –\pend
           \pstart
           Seien Sie, verehrteſter Herr Doktor, meiner herzlichen Ergebenheit jederzeit
               verſichert!\pend
           \pstart Ihr \spacefill\mbox{DrArthur Schnitzler}\pend{}\pstart
           \textcolor{pink}{Wien}{}\ledrightnote{\textcolor{pink}{Wien}}, 7. 12. 94.\pend
           \endnumbering\briefempfaengerindex{Mamroth, Fedor@\textsc{Mamroth, Fedor}!zzzSchnitzler, Arthur@\emph{von Arthur Schnitzler}!1894-12-071@{7. 12. 1894}|)be}\mylabel{h}  \normalsize

\doendnotes{C}
\bigskip
\vfill

\clearpage

\footnotesize

\lohead{\textsc{register}}

% Definiere theindex-Environment komplett neu ohne reledmac
\makeatletter
\renewenvironment{theindex}{%
  \section*{\indexname}%
  \setlength{\parindent}{0pt}%
  \setlength{\parskip}{0pt plus 0.3pt}%
  \let\item\@idxitem
}{%
  \clearpage
}
\makeatother

\IfFileExists{\jobname-pw.ind}{\input{\jobname-pw.ind}}{}

\end{document}

      