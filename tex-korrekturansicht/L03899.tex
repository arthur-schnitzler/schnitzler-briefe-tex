%% latex-korrekturansicht-vorspann.tex
%% Vorspann für die Korrekturansicht.
%% Lädt die gemeinsame Datei latex-vorspann.tex mit gesetztem Schalter.

\newif\ifkorrekturansicht
\korrekturansichttrue

\input{../tex-inputs/latex-vorspann}


\section[Theodor Herzl an Arthur Schnitzler, 28. 2. 1902]{L03899 Theodor Herzl an Arthur Schnitzler, 28. 2. 1902}
\nopagebreak\mylabel{L03899v}
\rehead{ }\normalsize\beginnumbering\briefempfaengerindex{Schnitzler, Arthur@\textsc{Schnitzler, Arthur}!zzzHerzl, Theodor@\emph{von Theodor Herzl}!1902-02-282@{28. 2. 1902}|(be}
\toendnotes[C]{\smallbreak\pagebreak[2]}
\correspDesc{Versand  durch Theodor Herzl am 28. 2. 1902 in Wien
\newline{}Erhalt  durch Arthur Schnitzler im Zeitraum [28. 2. 1902 – 3. 3. 1902?] in Wien}\toendnotes[C]{\smallbreak}
\Standort{Jerusalem, Central Zionist Archives, H1:2539-3.}
\physDesc{Brief, maschinenschriftliche Abschrift, 1 Blatt, 1 Seite, 322 Zeichen
\newline{}Schreibmaschine}
\buchAbdrucke{\weitereDrucke{Theodor Herzl: \emph{Briefe Ende August 1900 – Ende Dezember 1902}. Bearbeitet von Barbara Schäfer in Zusammenarbeit mit Sofia Gelmann, Chaya Harel und Ines Rubin. Berlin, Frankfurt am Main, Wien: \emph{Propyläen} 1993, S. 444 (Briefe und Tagebücher. Herausgegeben von Alex Bein, Hermann Greive, Moshe Schaerf, Julius H. Schoeps und Johannes Wachten, 6).} }\toendnotes[C]{\smallbreak}
\pstart
           {\pb}\textcolor{gray}{\textbf{\textcolor{brown}{N. FR. PR.}\orgindex{Neue Freie Presse@Neue Freie Presse|pw}{}\ledrightnote{\textcolor{brown}{Neue Freie Presse}}.
                        }}\hfill 28. II. 1902\pend
           
\pstart{}Lieber Dr. Schnitzler,\pend\vspace{0.5em}
\pstart
           \label{K_L03899-1v}\edtext{Ostern}{\lemma{\textnormal{\emph{Ostern}}}\Cendnote{\textnormal{In diesem Jahr fiel der Ostersonntag auf den 7. 4. 1901.}}}\label{K_L03899-1} ist vor der Thür u. ich lade Sie ein. Wenn mich mein ermüdetes
               Feuilletonredacteursgedächtnis nicht täuscht, haben Sie mir Weihnachten etwas für
               Ostern \label{K_L03899-2v}\edtext{versprochen}{\lemma{\textnormal{\emph{versprochen}}}\Cendnote{\textnormal{XXXX}}}\label{K_L03899-2}. Es kann eine Geschichte, Plauderei oder einactiges Stück
               sein.\pend
           
\pstart
           Ihr Jawort bald erwartend mit den besten Grüssen{\\[\baselineskip]} Ihr
               \spacefill\mbox{Herzl}\pend
           \leftskip=0em{}\selectlanguage{ngerman}\endnumbering\briefempfaengerindex{Schnitzler, Arthur@\textsc{Schnitzler, Arthur}!zzzHerzl, Theodor@\emph{von Theodor Herzl}!1902-02-282@{28. 2. 1902}|)be}\mylabel{L03899h}
\begin{anhang}
\end{anhang}\normalsize

\doendnotes{C}
\bigskip
\vfill

\clearpage

\footnotesize

\lohead{\textsc{register}}

% Definiere theindex-Environment komplett neu ohne reledmac
\makeatletter
\renewenvironment{theindex}{%
  \section*{\indexname}%
  \setlength{\parindent}{0pt}%
  \setlength{\parskip}{0pt plus 0.3pt}%
  \let\item\@idxitem
}{%
  \clearpage
}
\makeatother

\IfFileExists{\jobname-pw.ind}{\input{\jobname-pw.ind}}{}

\end{document}

      