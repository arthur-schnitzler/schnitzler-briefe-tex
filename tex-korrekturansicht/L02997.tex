%% latex-korrekturansicht-vorspann.tex
%% Vorspann für die Korrekturansicht.
%% Lädt die gemeinsame Datei latex-vorspann.tex mit gesetztem Schalter.

\newif\ifkorrekturansicht
\korrekturansichttrue

\input{../tex-inputs/latex-vorspann}


\renewcommand{\erwaehntePersonen}{Personen: Berta Czegka, Felix Salten, Ottilie Salten}
\renewcommand{\erwaehnteInstitutionen}{Institutionen: Überbrettl}
\renewcommand{\erwaehnteOrte}{Orte: Wien}
\renewcommand{\erwaehnteWerke}{Werke: Die Zeit, Marionetten. Drei Einakter, Zum großen Wurstel. Burleske in einem Akt}
\section[ Arthur Schnitzler an Felix Salten, 8. 2. 1905]{Arthur Schnitzler an Felix Salten, 8. 2. 1905}
\nopagebreak\mylabel{v}
\rehead{ }\normalsize\beginnumbering\briefempfaengerindex{Salten, Felix@\textsc{Salten, Felix}!zzzSchnitzler, Arthur@\emph{von Arthur Schnitzler}!1905-02-081@{8. 2. 1905}|(be}
\toendnotes[C]{\smallbreak\pagebreak[2]}\Standort{Wienbibliothek im Rathaus, ZPH 1681, 2.1.516.}
\physDesc{Brief, 1 Blatt, 2 Seiten, 786 Zeichen
\newline{}Handschrift: schwarze Tinte, deutsche Kurrent
\newline{}Ordnung: mit Bleistift von unbekannter Hand nummeriert: »28« }\toendnotes[C]{\smallbreak}
\pstart
           \raggedleft{}\textsc{{\pb}\textcolor{pink}{Wien}{}\ledrightnote{\textcolor{pink}{Wien}}}, 8. 2. 905\pend
           
\pstart{}lieber,\pend
\pstart
           erſtens frage ich Sie, ob Sie am \label{K_L02997-1v}\edtext{Sonntag{ }Abend mit Ihrer \textcolor{blue}{Frau}{}\ledrightnote{{$\rightarrow$}\textcolor{blue}{Ottilie Salten}} bei uns nachtmahlen}{\lemma{\textnormal{\emph{Sonntag … nachtmahlen}}}\Cendnote{\textnormal{siehe A. S.: \emph{Tagebuch}, 12. 2. 1905}}}\label{K_L02997-1h} wollen, was uns ſehr freuen würde.\pend
           
\pstart
           Zweitens ſchicke ich Ihnen hier ein \textsc{\textcolor{green}{Manuscript}{}\ledrightnote{{$\rightarrow$}\textcolor{green}{Zum großen Wurstel. Burleske in einem Akt}}}. Es ſind die \label{K_L02997-2v}\edtext{einſtigen \textcolor{green}{Marionetten}{}\ledrightnote{\textcolor{green}{Zum großen Wurstel. Burleske in einem Akt}}}{\lemma{\textnormal{\emph{einſtigen Marionetten}}}\Cendnote{\textnormal{Am 8. 3. 1901
                  führte das \emph{\textcolor{brown}{Überbrettl}} unter dem Titel \emph{\textcolor{green}{Marionetten}} die Burleske auf, 
                  die durchfiel. \textcolor{blue}{Schnitzler} hatte seither die Szene unter dem neuen Titel \emph{\textcolor{green}{Zum großen Wurstel. Burleske in einem Akt}}
                  überarbeitet. Den Titel \emph{\textcolor{green}{Marionetten}} verwendete er 1906 für die Buchausgabe, 
                  die diese Szene und zwei andere vereinte.}}}\label{K_L02997-2h} (die
               natürlich auch noch niemals gedruckt waren) höchſt umgearbeitet; und ich frage Sie,
               ob Sie das \textcolor{green}{Stückerl}{}\ledrightnote{{$\rightarrow$}\textcolor{green}{Zum großen Wurstel. Burleske in einem Akt}} für die
                  \label{K_L02997-3v}\edtext{\textcolor{green}{Oſternu{\geminationm}er}{}\ledrightnote{{$\rightarrow$}\textcolor{green}{Die Zeit}}}{\lemma{\textnormal{\emph{Oſternummer}}}\Cendnote{\textnormal{\textcolor{blue}{Arthur Schnitzler}: \emph{\textcolor{green}{Zum großen Wurstel. Burleske in einem Akt}}. In: \emph{\textcolor{green}{Die Zeit}}, Jg. 4, Nr. 926, 23. 4. 1905, Beilage: Oster-Zeit, S. 3–7.
                  Vor und nach dem Text findet sich jeweils eine Illustration von \textcolor{blue}{Berta Czegka}.}}}\label{K_L02997-3h} haben wollen. Ich ſchicke es Ihnen deshalb
               ſo früh, weil ich Ihnen, für {\pb}den Fall der
               Annahme, vorſchlagen möchte, es illuſtriren zu laſſen, wofür es ſich \introOben{}mir\introOben{} ſehr zu eignen ſcheint – natürlich bin ich da{\geminationn} ſehr gern bereit, \strikeout{den}
               mich mit dem \textcolor{blue}{Illuſtrator}{}\ledrightnote{{$\rightarrow$}\textcolor{blue}{Berta Czegka}},
               den Sie wählen würden, über die Details zu beſprechen. (Eventuell wäre mit dieſem
               Scherz die ganze \textcolor{green}{Oſterbeilage}{}\ledrightnote{{$\rightarrow$}\textcolor{green}{Die Zeit}}
               ausgefüllt.) Als Honorar würd ich 600 Kronen beanſpruchen.\pend
           
\pstart
           Seien Sie herzlich gegrüßt. {\\[\baselineskip]}Ihr {\\[\baselineskip]}\spacefill\mbox{Arth Sch}\pend
           \leftskip=0em{}\endnumbering\briefempfaengerindex{Salten, Felix@\textsc{Salten, Felix}!zzzSchnitzler, Arthur@\emph{von Arthur Schnitzler}!1905-02-081@{8. 2. 1905}|)be}\mylabel{h}  \normalsize

\doendnotes{C}
\bigskip
\vfill

\clearpage

\footnotesize

\lohead{\textsc{register}}

% Definiere theindex-Environment komplett neu ohne reledmac
\makeatletter
\renewenvironment{theindex}{%
  \section*{\indexname}%
  \setlength{\parindent}{0pt}%
  \setlength{\parskip}{0pt plus 0.3pt}%
  \let\item\@idxitem
}{%
  \clearpage
}
\makeatother

\IfFileExists{\jobname-pw.ind}{\input{\jobname-pw.ind}}{}

\end{document}

      