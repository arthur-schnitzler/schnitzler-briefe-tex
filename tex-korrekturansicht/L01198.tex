%% latex-korrekturansicht-vorspann.tex
%% Vorspann für die Korrekturansicht.
%% Lädt die gemeinsame Datei latex-vorspann.tex mit gesetztem Schalter.

\newif\ifkorrekturansicht
\korrekturansichttrue

\input{../tex-inputs/latex-vorspann}


               \section[Arthur Schnitzler an Richard Dehmel, 13. 1. 1902]{ Arthur Schnitzler an Richard Dehmel, 13. 1. 1902}\nopagebreak\mylabel{v}\rehead{ }\normalsize\beginnumbering\briefempfaengerindex{Dehmel, Richard@\textsc{Dehmel, Richard}!zzzSchnitzler, Arthur@\emph{von Arthur Schnitzler}!1902-01-131@{13. 1. 1902}|(be} \toendnotes[C]{\smallbreak\pagebreak[2]} \Standort{Hamburg, Staats- und Universitätsbibliothek, DA:Br:S:616/20.}
\physDesc{Brief, 1 Blatt, 2 Seiten
\newline{}Handschrift: schwarze Tinte, deutsche Kurrent}\pstart{}{\pb}Verehrteſter Herr Dehmel,\pend\pstart
           ich danke verbindlichſt für Ihre freundliche Aufforderung zur Mitarbeiterſchaft
                    am »\textcolor{green}{Buntſcheck}{}\ledrightnote{\textcolor{green}{Der Buntscheck. Ein Sammelbuch herzhafter Kunst für Ohr und Auge deutscher Kinder}}«. Eine beſtimmte Zuſage ka{\geminationn} ich aber erſt machen, we{\geminationn} ich etwas für Ihr Buch geeignetes {\pb}geſchrieben haben werde. Sollte das bis zum
                        September dJ. geſchehen ſein, ſo ſende ich Ihnen den Beitrag
                    ſelbſtverſtändlich mit beſonderm Vergnügen zu.\pend
           \pstart
           Mit beſondrer Hochſchätzung{\\[\baselineskip]}Ihr aufrichtg ergebner{\\[\baselineskip]}\spacefill\mbox{Arthur Schnitzler}\pend
           \leftskip=0em{}\pstart
           \textcolor{pink}{Wien IX. Frankg. 1}{}\ledrightnote{\textcolor{pink}{Frankgasse}}.{\\}13. 1. 902.\pend
           \endnumbering\briefempfaengerindex{Dehmel, Richard@\textsc{Dehmel, Richard}!zzzSchnitzler, Arthur@\emph{von Arthur Schnitzler}!1902-01-131@{13. 1. 1902}|)be}\mylabel{h}  \normalsize

\doendnotes{C}
\bigskip
\vfill

\clearpage

\footnotesize

\lohead{\textsc{register}}

% Definiere theindex-Environment komplett neu ohne reledmac
\makeatletter
\renewenvironment{theindex}{%
  \section*{\indexname}%
  \setlength{\parindent}{0pt}%
  \setlength{\parskip}{0pt plus 0.3pt}%
  \let\item\@idxitem
}{%
  \clearpage
}
\makeatother

\IfFileExists{\jobname-pw.ind}{\input{\jobname-pw.ind}}{}

\end{document}

      