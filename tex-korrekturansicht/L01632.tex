%% latex-korrekturansicht-vorspann.tex
%% Vorspann für die Korrekturansicht.
%% Lädt die gemeinsame Datei latex-vorspann.tex mit gesetztem Schalter.

\newif\ifkorrekturansicht
\korrekturansichttrue

\input{../tex-inputs/latex-vorspann}


               \section[Hugo von Hofmannsthal an Arthur Schnitzler, 14. 10. 1906]{ Hugo von Hofmannsthal an Arthur Schnitzler, 14. 10. 1906}\nopagebreak\mylabel{v}\rehead{ }\normalsize\beginnumbering\briefempfaengerindex{Schnitzler, Arthur@\textsc{Schnitzler, Arthur}!zzzHofmannsthal, Hugo von@\emph{von Hugo von Hofmannsthal}!1906-10-141@{14. 10. 1906}|(be} \toendnotes[C]{\smallbreak\pagebreak[2]} \Standort{CUL, Schnitzler, B 43.}
\physDesc{Postkarte
\newline{}Handschrift: schwarze Tinte, deutsche Kurrent\newline{}Versand: 1) Stempel: »\nobreak{}Wien, 15. 10. 06, V\nobreak{}«.  2) Stempel: »\nobreak{}\oindex{XVIII., Waehring@\textbf{XVIII., Währing}, \emph{Bezirk (A.BZK)}|pwk}18/1 Wien 110, 15. X. 06, XII, Bestellt\nobreak{}«. \newline{}Ordnung: 1) mit Bleistift von unbekannter Hand nummeriert: »\strikeout{266}« 2) mit Bleistift von unbekannter Hand  nummeriert: »267«}\buchAbdrucke{\weitereDrucke{Hugo von Hofmannsthal, Arthur Schnitzler: \emph{Briefwechsel}. Hg. Therese Nickl und Heinrich Schnitzler. Frankfurt am Main: \emph{S. Fischer} 1964, S. 223.} }\pstart{}{\pb}\textsc{Herrn D\textsuperscript{r}}{ }\substVorne{}\textsuperscript{\textsc{Julius}}{\allowbreak}\substDazwischen{}\textsc{Arthur}\substHinten{}{ }\textsc{Schnitzler}\pend{}\pstart{}\textcolor{pink}{\textsc{Wien}}{}\ledrightnote{\textcolor{pink}{Wien}}\pend{}\pstart{}\textcolor{pink}{\textsc{XVIII Spöttelgasse 7}.}{}\ledrightnote{\textcolor{pink}{Edmund-Weiß-Gasse}}\pend{}{\bigskip}\pstart
           \noindent{}{\pb}\textsc{Wenn Wetter nicht zum Schlechten umschlägt (oder Sturm), kommt
                     \textcolor{blue}{Gerty}{}\ledrightnote{\textcolor{blue}{Gertrude von Hofmannsthal}} morgen zum Tennys.}\pend
           \pstart
           So{\geminationn}tg.\pend
           \pstart
           Nachher bei Euch eſſen und gleich nach dem Eſſen weg, wie Sie geſagt haben.\pend
           \endnumbering\briefempfaengerindex{Schnitzler, Arthur@\textsc{Schnitzler, Arthur}!zzzHofmannsthal, Hugo von@\emph{von Hugo von Hofmannsthal}!1906-10-141@{14. 10. 1906}|)be}\mylabel{h}  \normalsize

\doendnotes{C}
\bigskip
\vfill

\clearpage

\footnotesize

\lohead{\textsc{register}}

% Definiere theindex-Environment komplett neu ohne reledmac
\makeatletter
\renewenvironment{theindex}{%
  \section*{\indexname}%
  \setlength{\parindent}{0pt}%
  \setlength{\parskip}{0pt plus 0.3pt}%
  \let\item\@idxitem
}{%
  \clearpage
}
\makeatother

\IfFileExists{\jobname-pw.ind}{\input{\jobname-pw.ind}}{}

\end{document}

      