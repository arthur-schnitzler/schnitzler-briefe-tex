%% latex-korrekturansicht-vorspann.tex
%% Vorspann für die Korrekturansicht.
%% Lädt die gemeinsame Datei latex-vorspann.tex mit gesetztem Schalter.

\newif\ifkorrekturansicht
\korrekturansichttrue

\input{../tex-inputs/latex-vorspann}


\section[Stefan Zweig an Arthur Schnitzler, 2. 1. 1925]{L03670 Stefan Zweig an Arthur Schnitzler, 2. 1. 1925}
\nopagebreak\mylabel{L03670v}
\rehead{ }\normalsize\beginnumbering\briefempfaengerindex{Schnitzler, Arthur@\textsc{Schnitzler, Arthur}!zzzZweig, Stefan@\emph{von Stefan Zweig}!1925-01-021@{2. 1. 1925}|(be}
\toendnotes[C]{\smallbreak\pagebreak[2]}
\correspDesc{Versand  durch Stefan Zweig am 2. 1. 1925 in Salzburg
\newline{}Erhalt  durch Arthur Schnitzler im Zeitraum [3. 1. 1925 – 7. 1. 1925?] in Wien}\toendnotes[C]{\smallbreak}
\Standort{CUL, Schnitzler, B 118.}
\physDesc{Briefkarte, 725 Zeichen
\newline{}Handschrift: blaue Tinte, lateinische Kurrent
\newline{}Schnitzler: 1) mit Bleistift »\textsc{Zweig}«  2) mit rotem Buntstift eine Unterstreichung}
\buchAbdrucke{\weitereDrucke{Stefan Zweig: \emph{Briefwechsel mit Hermann Bahr, Sigmund Freud, Rainer Maria
                        Rilke und Arthur Schnitzler}. Herausgegeben von Jeffrey B. Berlin, Hans-Ulrich Lindken und Donald A. Prater. Frankfurt am Main: \emph{S. Fischer} 1987, S. 421.} }\toendnotes[C]{\smallbreak}
\pstart
           {\pb}\textcolor{gray}{\textbf{SZ}}\hfill 2. Januar 1925\pend
           
\pstart
           \raggedleft{}\strikeout{\textcolor{gray}{\textbf{\textcolor{pink}{VIII.
                        KOCHGASSE 8}\oindex{Wien@\textbf{Wien}!VIII., Josefstadt@\textbf{VIII., Josefstadt}!Kochgasse 8@\textbf{Kochgasse 8}, \emph{Wohngebäude}|pw}{}\ledrightnote{\textcolor{pink}{Kochgasse 8}}}}}\pend
           \vspace{0.5em}
\pstart
           Lieber Herr Doktor, jetzt erst, von \textcolor{pink}{Paris}\oindex{Paris@\textbf{Paris}, \emph{Hauptstadt}|pw}{}\ledrightnote{\textcolor{pink}{Paris}} heimgekehrt und kaum eingewohnt, danke ich ihnen innigst für das
               dreifach kostbare Buch \textcolor{green}{Fräulein Else}\pwindex{Schnitzler, Arthur 15. 5. 1862 Wien – 21. 10. 1931 ebd.@\textsc{Schnitzler, Arthur} (15. 5. 1862 Wien – 21. 10. 1931 ebd.), \emph{Schriftsteller, Mediziner}!Fräulein Else@\strich\emph{Fräulein Else}|pw}{}\ledrightnote{\textcolor{green}{Fräulein Else}}. Dreifach
               kostbar: erstens als meisterliches Werk, zweitens Dank Ihre Widmung, drittens als
               Erstausgabe. Denn dieses Buch wird (wenn ich nur irgendwie Talent zum Profeten habe)
               in so gewaltigen Auflagen bald verbreitet sein, dass die erste ein Sammelobject {\pb}für Bibliophilen darstellen muss. Mir wird
               es aber nicht um d\substVorne{}\textsuperscript{es}\substDazwischen{}en\substHinten{} materiellen Wertes kostbar sein, sondern als geistiger Genuss und als
               Zeichen Ihrer mir so wertvollen Sympathie, – die hoffentlich eine Gegengabe zu den
                  \label{K_L03670-1v}\edtext{Iden den März}{\lemma{\textnormal{\emph{Iden den März}}}\Cendnote{\textnormal{14. 3. 1925}}}\label{K_L03670-1},
               mein neues \textcolor{green}{Essaywerk}\pwindex{Zweig, Stefan 28.\,11.\,1881 Wien – 23.\,2.\,1942 Petrópolis@\textsc{Zweig, Stefan} (28.\,11.\,1881 Wien – 23.\,2.\,1942 Petrópolis), \emph{Schriftsteller}!Kampf mit dem Dämon. Hölderlin – Kleist – Nietzsche@\strich\emph{Der Kampf mit dem Dämon. Hölderlin – Kleist – Nietzsche}|pwv}{}\ledrightnote{{$\rightarrow$}\emph{\textcolor{green}{Der Kampf mit dem Dämon. Hölderlin – Kleist – Nietzsche}}}, mir
               nicht entziehen wird. \pend
           
\pstart
           Freulichst, dankbarst Ihr{\\[\baselineskip]}\spacefill\mbox{Stefan Zweig}\pend
           \leftskip=0em{}\selectlanguage{ngerman}\endnumbering\briefempfaengerindex{Schnitzler, Arthur@\textsc{Schnitzler, Arthur}!zzzZweig, Stefan@\emph{von Stefan Zweig}!1925-01-021@{2. 1. 1925}|)be}\mylabel{L03670h}
\begin{anhang}
\end{anhang}\normalsize

\doendnotes{C}
\bigskip
\vfill

\clearpage

\footnotesize

\lohead{\textsc{register}}

% Definiere theindex-Environment komplett neu ohne reledmac
\makeatletter
\renewenvironment{theindex}{%
  \section*{\indexname}%
  \setlength{\parindent}{0pt}%
  \setlength{\parskip}{0pt plus 0.3pt}%
  \let\item\@idxitem
}{%
  \clearpage
}
\makeatother

\IfFileExists{\jobname-pw.ind}{\input{\jobname-pw.ind}}{}

\end{document}

      