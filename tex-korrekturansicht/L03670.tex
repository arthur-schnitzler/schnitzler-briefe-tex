%% latex-korrekturansicht-vorspann.tex
%% Vorspann für die Korrekturansicht.
%% Lädt die gemeinsame Datei latex-vorspann.tex mit gesetztem Schalter.

\newif\ifkorrekturansicht
\korrekturansichttrue

\input{../tex-inputs/latex-vorspann}


\renewcommand{\erwaehntePersonen}{Personen: Stefan Zweig}
\renewcommand{\erwaehnteOrte}{Orte: Kochgasse 8, Paris, Salzburg, Wien}
\renewcommand{\erwaehnteWerke}{Werke: Der Kampf mit dem Dämon. Hölderlin – Kleist – Nietzsche, Fräulein Else}
\section[Stefan Zweig an Arthur Schnitzler, 2. 1. 1925]{Stefan Zweig an Arthur Schnitzler, 2. 1. 1925}
\nopagebreak\mylabel{v}
\rehead{ }\normalsize\beginnumbering\briefempfaengerindex{Schnitzler, Arthur@\textsc{Schnitzler, Arthur}!zzzZweig, Stefan@\emph{von Stefan Zweig}!1925-01-021@{2. 1. 1925}|(be}
\toendnotes[C]{\smallbreak\pagebreak[2]}\Standort{CUL, Schnitzler, B 118.}
\physDesc{, 1 Blatt, 2 Seiten, 725 Zeichen
\newline{}Handschrift: blaue Tinte, lateinische Kurrent
\newline{}Schnitzler: 1) mit Bleistift »\textsc{Zweig}«  2) mit rotem Buntstift eine Unterstreichung}\toendnotes[C]{\smallbreak}
\pstart
           {\pb}\textcolor{gray}{\textbf{SZ}}\hfill 2. Januar 1925\pend
           
\pstart
           \raggedleft{}\strikeout{\textcolor{gray}{\textbf{\textcolor{pink}{VIII.
                        KOCHGASSE 8}{}\ledrightnote{\textcolor{pink}{Kochgasse 8}}}}}\pend
           
\pstart
           Lieber Herr Doktor, jetzt erst, von \textcolor{pink}{Paris}{}\ledrightnote{\textcolor{pink}{Paris}} heimgekehrt und kaum eingewohnt, danke ich ihnen innigst für das
               dreifach kostbare Buch \textcolor{green}{Fräulein Else}{}\ledrightnote{\textcolor{green}{Fräulein Else}}. Dreifach
               kostbar: erstens als meisterliches Werk, zweitens Dank Ihre Widmung, drittens als
               Erstausgabe. Denn dieses Buch wird (wenn ich nur irgendwie Talent zum Profeten habe)
               in so gewaltigen Auflagen bald verbreitet sein, dass die erste ein Sammelobject {\pb}für Bibliophilen darstellen muss. Mir wird
               es aber nicht um d\substVorne{}\textsuperscript{es}\substDazwischen{}en\substHinten{} materiellen Wertes kostbar sein, sondern als geistiger Genuss und als
               Zeichen Ihrer mir so wertvollen Sympathie, – die hoffentlich eine Gegengabe zu den
                  \label{K_L03670-1v}\edtext{Iden den März}{\lemma{\textnormal{\emph{Iden den März}}}\Cendnote{\textnormal{14. 3. 1925}}}\label{K_L03670-1h},
               mein neues \textcolor{green}{Essaywerk}{}\ledrightnote{{$\rightarrow$}\textcolor{green}{Der Kampf mit dem Dämon. Hölderlin – Kleist – Nietzsche}}, mir
               nicht entziehen wird. \pend
           
\pstart
           Freulichst, dankbarst Ihr{\\[\baselineskip]}\spacefill\mbox{Stefan Zweig}\pend
           \leftskip=0em{}\endnumbering\briefempfaengerindex{Schnitzler, Arthur@\textsc{Schnitzler, Arthur}!zzzZweig, Stefan@\emph{von Stefan Zweig}!1925-01-021@{2. 1. 1925}|)be}\mylabel{h}
\begin{anhang}
\end{anhang}\normalsize

\doendnotes{C}
\bigskip
\vfill

\clearpage

\footnotesize

\lohead{\textsc{register}}

% Definiere theindex-Environment komplett neu ohne reledmac
\makeatletter
\renewenvironment{theindex}{%
  \section*{\indexname}%
  \setlength{\parindent}{0pt}%
  \setlength{\parskip}{0pt plus 0.3pt}%
  \let\item\@idxitem
}{%
  \clearpage
}
\makeatother

\IfFileExists{\jobname-pw.ind}{\input{\jobname-pw.ind}}{}

\end{document}

      