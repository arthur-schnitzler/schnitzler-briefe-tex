%% latex-korrekturansicht-vorspann.tex
%% Vorspann für die Korrekturansicht.
%% Lädt die gemeinsame Datei latex-vorspann.tex mit gesetztem Schalter.

\newif\ifkorrekturansicht
\korrekturansichttrue

\input{../tex-inputs/latex-vorspann}


\renewcommand{\erwaehntePersonen}{Personen: André Antoine, Rudolf Christians, Hanns Heinz Ewers, Alfred Kerr, Paul Martin Marton, Dora Michaelis, Karl Michaelis, Ludwig Stein, Elisabeth Steinrück, Frida Strindberg, Ernst von Wolzogen}
\renewcommand{\erwaehnteInstitutionen}{Institutionen: Secessionsbühne, Überbrettl}
\renewcommand{\erwaehnteOrte}{Orte: Berlin, Bern, Dessauer Straße, Köpenicker Straße, Paris, Rom, Théâtre Antoine-Simone Berriau}
\renewcommand{\erwaehnteWerke}{Werke: Die Gefährtin. Schauspiel in einem Akt, La Compagne, Neue Freie Presse}
\section[ Paul Goldmann an Arthur Schnitzler, 6. 4. {[}1901{]}]{Paul Goldmann an Arthur Schnitzler, 6. 4. {[}1901{]}}
\nopagebreak\mylabel{v}
\rehead{ }\normalsize\beginnumbering\briefempfaengerindex{Schnitzler, Arthur@\textsc{Schnitzler, Arthur}!zzzGoldmann, Paul@\emph{von Paul Goldmann}!1901-04-062@{6. 4. {[}1901{]}}|(be}
\toendnotes[C]{\smallbreak\pagebreak[2]}\Standort{DLA, A:Schnitzler, HS.NZ85.1.3171.}
\physDesc{Brief, 1 Blatt, 4 Seiten
\newline{}Handschrift: blaue Tinte, deutsche Kurrent
\newline{}Schnitzler: 1) mit Bleistift das Jahr »{[}1{]}901« vermerkt  2) mit rotem Buntstift elf Unterstreichungen}\toendnotes[C]{\smallbreak}
\pstart
           \noindent{}\raggedleft{}{\pb}\textcolor{pink}{\textcolor{gray}{\textbf{DESSAUERSTRASSE 19}}}{}\ledrightnote{\textcolor{pink}{Dessauer Straße}}\pend
           
\pstart
           \textcolor{pink}{Berlin}{}\ledrightnote{\textcolor{pink}{Berlin}}, 6. April.\pend
           
\pstart\center{}Mein lieber Freund,\pend
\pstart
           Alſo Du biſt jetzt in \label{K_L03063-1v}\edtext{\textcolor{pink}{Rom}{}\ledrightnote{\textcolor{pink}{Rom}}}{\lemma{\textnormal{\emph{Rom}}}\Cendnote{\textnormal{\textcolor{blue}{Schnitzler} hielt sich von 31. 3. 1901 bis 11. 4. 1901 in \textcolor{pink}{Rom} auf.}}}\label{K_L03063-1h}, und es iſt gewiß ſehr
               herrlich.\pend
           
\pstart
           Daß \label{K_L03063-2v}\edtext{\textsc{\textcolor{blue}{Antoine}{}\ledrightnote{\textcolor{blue}{André Antoine}}} die »\textcolor{green}{Gefährtin}{}\ledrightnote{\textcolor{green}{Die Gefährtin. Schauspiel in einem Akt}}« aufführt}{\lemma{\textnormal{\emph{Antoine … aufführt}}}\Cendnote{\textnormal{\textcolor{blue}{Schnitzler}s Einakter \emph{\textcolor{green}{Die Gefährtin}} wurde als \emph{\textcolor{green}{La
                     Compagne}} zwischen 29. 4. 1902 und 4. 5. 1902 vier Mal im \textcolor{pink}{Théatre Antoine} aufgeführt. Schon im Jahr zuvor wurde die Annahme des \textcolor{green}{Stück}s in Zeitungen
                  gemeldet.}}}\label{K_L03063-2h}, haſt Du wohl geleſen.\pend
           
\pstart
           Die kleine \textsc{\textcolor{blue}{Dora Speyer}{}\ledrightnote{\textcolor{blue}{Dora Michaelis}}} ſprach mit mir über ihre Liebe zu Dir. Ich ſagte ihr, Du würdeſt wohl kaum
               heirathen, wenigtens jetzt nicht ſo bald, und ſie ſolle mit der {\pb}\label{K_L03063-3v}\edtext{Geſchichte}{\lemma{\textnormal{\emph{Geſchichte}}}\Cendnote{\textnormal{siehe Paul Goldmann an Arthur Schnitzler, 21. 3. [1901]}}}\label{K_L03063-3h} fertigzuwerden ſuchen. Das war wohl auch in Deinem Sinne? Hier hat ſich ein
                  \textcolor{blue}{Cousin}{}\ledrightnote{{$\rightarrow$}\textcolor{blue}{Karl Michaelis}}, ein \label{K_L03063-444v}\edtext{\textsc{Dr. \textcolor{blue}{Michaelis}{}\ledrightnote{\textcolor{blue}{Karl Michaelis}}}}{\lemma{\textnormal{\emph{Dr. Michaelis}}}\Cendnote{\textnormal{\textcolor{blue}{Karl Michaelis}, der spätere Ehemann}}}\label{K_L03063-444h},
               wohlhabender Chemiker, in die \textcolor{blue}{Kleine}{}\ledrightnote{{$\rightarrow$}\textcolor{blue}{Dora Michaelis}} verliebt. Sie findet ihn auch ſympathiſch. Ich denke, die
               Conſequenzen w\substVorne{}\textsuperscript{u}\substDazwischen{}e\substHinten{}rden \strikeout{\textcolor{gray}{end}} gezogen werden.\pend
           
\pstart
           Frau \textsc{\textcolor{blue}{Frida Strindberg}{}\ledrightnote{\textcolor{blue}{Frida Strindberg}}} hat thatſächlich ein Verhältniß mit dem jungen \textsc{Dr. \textcolor{blue}{Evers}{}\ledrightnote{\textcolor{blue}{Hanns Heinz Ewers}}} und wird wohl deswegen\strikeout{in} in \textcolor{pink}{Berlin}{}\ledrightnote{\textcolor{pink}{Berlin}} bleiben.\pend
           
\pstart
           Der Direktor \textsc{\textcolor{blue}{Martin}{}\ledrightnote{\textcolor{blue}{Paul Martin Marton}}} von der {\pb}\textcolor{brown}{Seceſſionsbühne}{}\ledrightnote{\textcolor{brown}{Secessionsbühne}}, den wir Beide für einen ſo
               braven Menſchen hielten, ſcheint ein Lump zu ſein. \textsc{\textcolor{blue}{Christians}{}\ledrightnote{\textcolor{blue}{Rudolf Christians}}} erzählte mir einige \label{K_L03063-5v}\edtext{Schweinereien}{\lemma{\textnormal{\emph{Schweinereien}}}\Cendnote{\textnormal{Bezug unklar}}}\label{K_L03063-5h},
               die er gemacht, und ſprach von ihm in Ausdrücken, von denen »Zuchthäusler« noch der
               gelindeſte war.\pend
           
\pstart
           \textcolor{blue}{Wolzogen}{}\ledrightnote{\textcolor{blue}{Ernst von Wolzogen}} bekommt nächſte Saiſon ein \label{K_L03063-33v}\edtext{eigenes \textcolor{brown}{Theater}{}\ledrightnote{{$\rightarrow$}\textcolor{brown}{Überbrettl}}}{\lemma{\textnormal{\emph{eigenes Theater}}}\Cendnote{\textnormal{Gemeint war der Umzug des seit
                  Jahresbeginn 1901 aktiven \emph{\textcolor{brown}{Überbrettl}} in ein Gebäude in der \textcolor{pink}{Köpenicker Straße 68}.}}}\label{K_L03063-33h}. Geldgeber war der \textsc{Prof. \textcolor{blue}{Stein}{}\ledrightnote{\textcolor{blue}{Ludwig Stein}}} aus \textsc{\textcolor{pink}{Bern}{}\ledrightnote{\textcolor{pink}{Bern}}}, jener ſeichte philoſophiſche Schwätzer, den Du wohl in {\pb}der \textcolor{green}{N.
                  Fr. Pr.}{}\ledrightnote{\textcolor{green}{Neue Freie Presse}} häufig – nicht geleſen haſt. Ich bin gegenwärtig ſehr bemüht, das
                  \label{K_L03063-7v}\edtext{Engagement von Frl. \textsc{\textcolor{blue}{Liesl}{}\ledrightnote{\textcolor{blue}{Elisabeth Steinrück}}}}{\lemma{\textnormal{\emph{Engagement … Liesl}}}\Cendnote{\textnormal{siehe Paul Goldmann an Arthur Schnitzler, 18. 2. [1901]}}}\label{K_L03063-7h} durchzuſetzen, weiß aber nicht, ob es mir gelingen wird.\pend
           
\pstart
           \textsc{\textcolor{blue}{Kerr}{}\ledrightnote{\textcolor{blue}{Alfred Kerr}}} geht Dienſtag nach \textcolor{pink}{Paris}{}\ledrightnote{\textcolor{pink}{Paris}}, auf einige Monate. Er möchte rieſig gern \label{K_L03063-11v}\edtext{im Sommer mit uns ſein}{\lemma{\textnormal{\emph{im Sommer mit uns ſein}}}\Cendnote{\textnormal{Es ist keine gemeinsame Reise im Sommer
                     1901 bekannt.}}}\label{K_L03063-11h}. Das wird ſich ja wohl machen
               laſſen.\pend
           
\pstart
           Glückliche Oſtern! Viele treue Grüße! {\\[\baselineskip]}Dein \spacefill\mbox{Paul Goldmann.}\pend
           \leftskip=0em{}\endnumbering\briefempfaengerindex{Schnitzler, Arthur@\textsc{Schnitzler, Arthur}!zzzGoldmann, Paul@\emph{von Paul Goldmann}!1901-04-062@{6. 4. {[}1901{]}}|)be}\mylabel{h}
\begin{anhang}
\end{anhang}\normalsize

\doendnotes{C}
\bigskip
\vfill

\clearpage

\footnotesize

\lohead{\textsc{register}}

% Definiere theindex-Environment komplett neu ohne reledmac
\makeatletter
\renewenvironment{theindex}{%
  \section*{\indexname}%
  \setlength{\parindent}{0pt}%
  \setlength{\parskip}{0pt plus 0.3pt}%
  \let\item\@idxitem
}{%
  \clearpage
}
\makeatother

\IfFileExists{\jobname-pw.ind}{\input{\jobname-pw.ind}}{}

\end{document}

      