%% latex-korrekturansicht-vorspann.tex
%% Vorspann für die Korrekturansicht.
%% Lädt die gemeinsame Datei latex-vorspann.tex mit gesetztem Schalter.

\newif\ifkorrekturansicht
\korrekturansichttrue

\input{../tex-inputs/latex-vorspann}


\renewcommand{\erwaehnteInstitutionen}{Institutionen: Kleines Theater}
\renewcommand{\erwaehnteOrte}{Orte: Berlin, Café Kaiserhof (Inh. Johann Wortner), Wien}
\renewcommand{\erwaehnteWerke}{Werke: Der Gemeine. Schauspiel in drei Aufzügen, Tagebuch}
\section[ Felix Salten an Arthur Schnitzler, {[}3. 12.? 1902{]}]{Felix Salten an Arthur Schnitzler, {[}3. 12.? 1902{]}}
\nopagebreak\mylabel{v}
\rehead{ }\normalsize\beginnumbering\briefempfaengerindex{Schnitzler, Arthur@\textsc{Schnitzler, Arthur}!zzzSalten, Felix@\emph{von Felix Salten}!1902-12-031@{{[}3. 12.? 1902{]}}|(be}
\toendnotes[C]{\smallbreak\pagebreak[2]}\Standort{CUL, Schnitzler, B 89, A 2.}
\physDesc{Brief, 1 Blatt, 1 Seite, 527 Zeichen
\newline{}Handschrift: Bleistift, lateinische Kurrent
\newline{}Schnitzler: mit Bleistift auf »Nov 902« datiert und dem Vermerk »\textsc{Salten}« versehen 
\newline{}Ordnung: mit Bleistift von unbekannter Hand nummeriert: »162« }\toendnotes[C]{\smallbreak}
\pstart
           \centering{}{\pb}\uline{\label{K_L03337-1v}\edtext{Mittwoch}{\lemma{\textnormal{\emph{Mittwoch}}}\Cendnote{\textnormal{Folgt man der Datierung \textcolor{blue}{Schnitzler}s, würde das
                        Korrespondenzstück an einem der vier Mittwoche im November 1902 verfasst worden sein. Im \emph{\textcolor{green}{Tagebuch}} wird \textcolor{blue}{Salten} im November 1902 nicht erwähnt.
                        Für den 4. 12. 1902 – einen Donnerstag – ist hingegen ein Treffen mit
                           \textcolor{blue}{Salten} in einem Kaffeehaus
                        vermerkt, bei der es um die Reise \textcolor{blue}{Salten}s zur Uraufführung von \emph{\textcolor{green}{Der
                           Gemeine}} nach \textcolor{pink}{Berlin} ging. Diese
                        hatte am 25. 11. 1902 am \emph{\textcolor{brown}{Kleinen Theater}} stattgefunden, einem Dienstag. Da
                        eine Rückkehr nach \textcolor{pink}{Wien} am Folgetag
                        unwahrscheinlich ist, kann das Korrespondenzstück auf den nächsten Mittwoch nach der Uraufführung datiert
                        werden.}}}\label{K_L03337-1h}.}\pend
           
\pstart
           Lieber Freund, seit gestern bin ich wieder da, und möchte Sie sehr gerne bald sehen. Hätten
               Sie morgen, Donnerstag{ }Abds, um 10\textcolor{gray}{,} Lust in den \textcolor{pink}{Kaiserhof}{}\ledrightnote{\textcolor{pink}{Café Kaiserhof (Inh. Johann Wortner)}} zu
               kommen? Mir ist es über Erwarten, wie über Verdienst gut gegangen, nur war ich durch
               die verschiedensten Dinge so gehetzt und absorbirt, dass ich außer Depeschen nichts
               schrieb.\pend
           
\pstart
           Entschuldigen Sie mein\strikeout{e} Schweigen, – Sie werden es
               gewiß {[}verstehen{]}, wenn ich Ihnen einiges erzähle. Wenn Sie mir
               nicht abschreiben, bin ich Donnerstag{ }Abds d. i. also morgen im \textcolor{pink}{Café}{}\ledrightnote{{$\rightarrow$}\textcolor{pink}{Café Kaiserhof (Inh. Johann Wortner)}}\textcolor{gray}{.}\pend
           
\pstart
           herzlichst Ihr {\\[\baselineskip]}\spacefill\mbox{Salten}\pend
           \leftskip=0em{}\endnumbering\briefempfaengerindex{Schnitzler, Arthur@\textsc{Schnitzler, Arthur}!zzzSalten, Felix@\emph{von Felix Salten}!1902-12-031@{{[}3. 12.? 1902{]}}|)be}\mylabel{h}  \normalsize

\doendnotes{C}
\bigskip
\vfill

\clearpage

\footnotesize

\lohead{\textsc{register}}

% Definiere theindex-Environment komplett neu ohne reledmac
\makeatletter
\renewenvironment{theindex}{%
  \section*{\indexname}%
  \setlength{\parindent}{0pt}%
  \setlength{\parskip}{0pt plus 0.3pt}%
  \let\item\@idxitem
}{%
  \clearpage
}
\makeatother

\IfFileExists{\jobname-pw.ind}{\input{\jobname-pw.ind}}{}

\end{document}

      