%% latex-korrekturansicht-vorspann.tex
%% Vorspann für die Korrekturansicht.
%% Lädt die gemeinsame Datei latex-vorspann.tex mit gesetztem Schalter.

\newif\ifkorrekturansicht
\korrekturansichttrue

\input{../tex-inputs/latex-vorspann}


               \section[ Paul Goldmann an Arthur Schnitzler, 26. 1. {[}1898{]}]{Paul Goldmann an Arthur Schnitzler, 26. 1. {[}1898{]}}\nopagebreak\mylabel{v}\rehead{ }\normalsize\beginnumbering\briefempfaengerindex{Schnitzler, Arthur@\textsc{Schnitzler, Arthur}!zzzGoldmann, Paul@\emph{von Paul Goldmann}!1898-01-261@{26. 1. {[}1898{]}}|(be} \toendnotes[C]{\smallbreak\pagebreak[2]} \Standort{DLA, A:Schnitzler, HS.NZ85.1.3168.}
\physDesc{Brief, 1 Blatt, 3 Seiten
\newline{}Handschrift: blaue Tinte, lateinische Kurrent
\newline{}Schnitzler: 1) mit Bleistift das Jahr »98« vermerkt 2) mit rotem Buntstift drei Unterstreichungen}\toendnotes[C]{\smallbreak}\pstart
           \noindent{}{\pb}\textcolor{gray}{\textbf{\textbf{\textcolor{brown}{Frankfurter Zeitung}{}\ledrightnote{\textcolor{brown}{Frankfurter Zeitung}}}}}\pend
           \pstart
           \textcolor{gray}{\textbf{(\textcolor{brown}{\begin{otherlanguage}{french}Gazette de Francfort\end{otherlanguage}}{}\ledrightnote{\textcolor{brown}{Frankfurter Zeitung}}).}}\pend
           \pstart
           \textcolor{gray}{\textbf{\textbf{\begin{otherlanguage}{french}Fondateur M.\end{otherlanguage}{ }\textcolor{blue}{L. Sonnemann}{}\ledrightnote{\textcolor{blue}{Leopold Sonnemann}}.}}}\pend
           \pstart
           \begin{otherlanguage}{french}\textcolor{gray}{\textbf{Journal politique, financier,}}\end{otherlanguage}\pend
           \pstart
           \begin{otherlanguage}{french}\textcolor{gray}{\textbf{commercial et littéraire.}}\end{otherlanguage}\pend
           \pstart
           \begin{otherlanguage}{french}\textcolor{gray}{\textbf{\textbf{Paraissant trois fois par jour.}}}\end{otherlanguage}\hfill \textsc{\textcolor{pink}{Paris}{}\ledrightnote{\textcolor{pink}{Paris}}}, 26. Januar. \pend
           \pstart
           \begin{otherlanguage}{french}\textcolor{gray}{\textbf{\textbf{Bureau à \textcolor{pink}{Paris}{}\ledrightnote{\textcolor{pink}{Paris}}}}}\end{otherlanguage}\pend
           \pstart
           \begin{otherlanguage}{french}\textcolor{gray}{\textbf{\textbf{\textcolor{pink}{10 Rue de la Bourse}{}\ledrightnote{\textcolor{pink}{rue de la Bourse}}.}}}\end{otherlanguage}\pend
           \pstart
           Tauſend Dank, liebſter Freund, für Deinen \label{K_L02837-1v}\edtext{Schritt bei \textsc{\textcolor{blue}{Brahm}{}\ledrightnote{\textcolor{blue}{Otto Brahm}}}}{\lemma{\textnormal{\emph{Schritt bei Brahm}}}\Cendnote{\textnormal{siehe Josef Rosengart an Arthur Schnitzler, [16. 1. 1898]}}}\label{K_L02837-1h}. Natürlich iſt Alles vergeblich. Nie bekomme ich dieſe \textcolor{brown}{Stelle}{}\ledrightnote{→\textcolor{brown}{Vossische Zeitung}}. Erſtens paſſe ich nicht in
                  dieſe{[}s{]} temperamentsloſe \textcolor{brown}{Spießbürger-Blatt}{}\ledrightnote{→\textcolor{brown}{Vossische Zeitung}} hinein. Zweitens nehmen die Leute keinen
               Juden. Drittens: Wer bin ich? Wer kennt mich? Bin ich eine literariſche
               Perſönlichkeit? Ich bin ein Journaliſt! Frag’ nur Deinen Freund \textsc{\textcolor{blue}{Hugo}{}\ledrightnote{\textcolor{blue}{Hugo von Hofmannsthal}}}!\pend
           \pstart
           Aber tauſend Dank trotzdem! Es thut mir furchtbar leid, daß meine Leute Dich doch {\pb}mit der Angelegenheit beläſtigt haben.\pend
           \pstart
           \label{K_L02837-2v}\edtext{\textsc{\textcolor{blue}{Bahr}{}\ledrightnote{\textcolor{blue}{Hermann Bahr}}s}{ }\textcolor{green}{Artikel}{}\ledrightnote{→\textcolor{green}{Burgtheater. [Demission von Max Burckhard]}} über die \textcolor{brown}{Burgtheater}{}\ledrightnote{\textcolor{brown}{Burgtheater}}-Kriſis}{\lemma{\textnormal{\emph{Bahrs … Burgtheater-Kriſis}}}\Cendnote{\textnormal{\textcolor{blue}{Hermann Bahr}: \emph{\textcolor{green}{Burgtheater}}. In: \emph{\textcolor{green}{Die
                        Zeit. Wiener Wochenschrift}}, Jg. 14, Nr. 173, 22. 1. 1898, S. 59–60.}}}\label{K_L02837-2h} iſt glänzend. Wie ſchade, daß
               dieſes \textcolor{blue}{Schwein}{}\ledrightnote{→\textcolor{blue}{Hermann Bahr}} Talent hat!
               Wenn man dem \textsc{\textcolor{blue}{Prof. Singer}{}\ledrightnote{\textcolor{blue}{Isidor Singer}}} die Meinung über \textsc{\textcolor{blue}{Bahr}{}\ledrightnote{\textcolor{blue}{Hermann Bahr}}} ſagt, ſo wird er beleidigt. Oder er ſagt: »Schön; aber er wird geleſen!«
               Hübſche Äußerung für den \textcolor{blue}{Herausgeber}{}\ledrightnote{→\textcolor{blue}{Isidor Singer}} eines \textcolor{green}{Blatt}{}\ledrightnote{→\textcolor{green}{Die Zeit. Wiener Wochenschrift}}es, das für Recht und Wahrheit kämpft.\pend
           \pstart
           {\pb}Was macht Dein \label{K_L02837-5v}\edtext{\textcolor{green}{Stück}{}\ledrightnote{→\textcolor{green}{Das Vermächtnis. Schauspiel in drei Akten}}}{\lemma{\textnormal{\emph{Stück}}}\Cendnote{\textnormal{siehe Paul Goldmann an Arthur Schnitzler, 19. 1. [1898]}}}\label{K_L02837-5h}? Iſts fertig? Wann wirds geſpielt?\pend
           \pstart
           Bitte, bitte, ſchreib’ mir bald! Ich fühle mich ſo einſam!\pend
           \pstart
           Sei von Herzen gegrüßt!\pend
           \pstart
           Dein treuer {\\[\baselineskip]}\spacefill\mbox{Paul Goldm}\pend
           \leftskip=0em{}\pstart
           \noindent{}Und was ſagſt \strikeout{z} Du zu \label{K_L02837-11v}\edtext{\textcolor{pink}{Frankreich}{}\ledrightnote{\textcolor{pink}{Frankreich}}}{\lemma{\textnormal{\emph{Frankreich}}}\Cendnote{\textnormal{vermutlich Bezug auf die \textcolor{blue}{Dreyfus}-Affäre}}}\label{K_L02837-11h}?\pend
           \endnumbering\briefempfaengerindex{Schnitzler, Arthur@\textsc{Schnitzler, Arthur}!zzzGoldmann, Paul@\emph{von Paul Goldmann}!1898-01-261@{26. 1. {[}1898{]}}|)be}\mylabel{h}\begin{anhang}\end{anhang}\normalsize

\doendnotes{C}
\bigskip
\vfill

\clearpage

\footnotesize

\lohead{\textsc{register}}

% Definiere theindex-Environment komplett neu ohne reledmac
\makeatletter
\renewenvironment{theindex}{%
  \section*{\indexname}%
  \setlength{\parindent}{0pt}%
  \setlength{\parskip}{0pt plus 0.3pt}%
  \let\item\@idxitem
}{%
  \clearpage
}
\makeatother

\IfFileExists{\jobname-pw.ind}{\input{\jobname-pw.ind}}{}

\end{document}

      