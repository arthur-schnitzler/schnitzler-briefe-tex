%% latex-korrekturansicht-vorspann.tex
%% Vorspann für die Korrekturansicht.
%% Lädt die gemeinsame Datei latex-vorspann.tex mit gesetztem Schalter.

\newif\ifkorrekturansicht
\korrekturansichttrue

\input{../tex-inputs/latex-vorspann}


               \section[Georg Brandes an Arthur Schnitzler, 22. 4. 1896]{ Georg Brandes an Arthur Schnitzler, 22. 4. 1896}\nopagebreak\mylabel{v}\rehead{ }\normalsize\beginnumbering\briefempfaengerindex{Schnitzler, Arthur@\textsc{Schnitzler, Arthur}!zzzBrandes, Georg@\emph{von Georg Brandes}!1896-04-221@{22. 4. 1896}|(be} \toendnotes[C]{\smallbreak\pagebreak[2]} \Standort{CUL, Schnitzler, B 17.}
\physDesc{Brief, 1 Blatt, 1 Seite
\newline{}Handschrift: blaue Tinte, deutsche Kurrent\newline{}Ordnung: von unbekannter Hand nummeriert: »2« }\buchAbdrucke{\weitereDrucke{Georg Brandes, Arthur Schnitzler: \emph{Ein Briefwechsel}. Hg. Kurt Bergel. Bern: \emph{Francke} 1956, S. 56.} }\pstart
           \raggedleft{}{\pb}\textcolor{pink}{Kopenhagen}{}\ledrightnote{\textcolor{pink}{Kopenhagen}}{ }22 April 96\pend
           \pstart{}Verehrter Herr\pend\pstart
           Kürzlich sah ich in \textcolor{pink}{Berlin}{}\ledrightnote{\textcolor{pink}{Österreich}} im \textcolor{pink}{Deutschen Theater}{}\ledrightnote{\textcolor{pink}{Deutsches Theater Berlin}} das Schauspiel \textcolor{green}{Liebelei}{}\ledrightnote{\textcolor{green}{Liebelei. Schauspiel in drei Akten}} und es drängt mich, Ihnen zu sagen, welch
                    starken Eindruck es auf mich gemacht hat. Seit langer Zeit habe ich bei einem
                    deutschen Stück nicht so viel gefühlt. Es hat mich ganz ergriffen. Die
                    Aufführung, die Sie vermutlich kennen, war ganz auf der Höhe der dramatischen
                    Arbeit, \textcolor{blue}{Reicher}{}\ledrightnote{\textcolor{blue}{Emanuel Reicher}}, \textcolor{blue}{Rittner}{}\ledrightnote{\textcolor{blue}{Rudolf Rittner}}, \textcolor{blue}{Jarno}{}\ledrightnote{\textcolor{blue}{Josef Jarno}}, besonders die
                        \textcolor{blue}{Sorma}{}\ledrightnote{\textcolor{blue}{Agnes Sorma}} alle vorzüglich.\pend
           \pstart
           Es ist nicht hübsch von Ihnen, dass Sie mich vergessen haben in Ihrem Erfolg und
                        mi\strikeout{c}r\strikeout{h} das Ding
                    nicht sandten, da ich doch alles Andere von Ihnen habe\pend
           \pstart Mit herzlichstem Gruss\hspace*{3.5em}Ihr\hspace*{3.5em}\spacefill\mbox{Georg Brandes}\pend{}\endnumbering\briefempfaengerindex{Schnitzler, Arthur@\textsc{Schnitzler, Arthur}!zzzBrandes, Georg@\emph{von Georg Brandes}!1896-04-221@{22. 4. 1896}|)be}\mylabel{h}  \normalsize

\doendnotes{C}
\bigskip
\vfill

\clearpage

\footnotesize

\lohead{\textsc{register}}

% Definiere theindex-Environment komplett neu ohne reledmac
\makeatletter
\renewenvironment{theindex}{%
  \section*{\indexname}%
  \setlength{\parindent}{0pt}%
  \setlength{\parskip}{0pt plus 0.3pt}%
  \let\item\@idxitem
}{%
  \clearpage
}
\makeatother

\IfFileExists{\jobname-pw.ind}{\input{\jobname-pw.ind}}{}

\end{document}

      