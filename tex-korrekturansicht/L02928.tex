%% latex-korrekturansicht-vorspann.tex
%% Vorspann für die Korrekturansicht.
%% Lädt die gemeinsame Datei latex-vorspann.tex mit gesetztem Schalter.

\newif\ifkorrekturansicht
\korrekturansichttrue

\input{../tex-inputs/latex-vorspann}


         
         \renewcommand{\erwaehntePersonen}{Personen: Richard Beer-Hofmann}
         \renewcommand{\erwaehnteOrte}{Orte: Hotel Trafoi, Tirol, Trafoi, Wien}
         \renewcommand{\erwaehnteWerke}{Werke: Der blinde Geronimo und sein Bruder, Der blinde Musikant}
               \section[ Paul Goldmann an Arthur Schnitzler, 28. 8. {[}1900{]}]{Paul Goldmann an Arthur Schnitzler, 28. 8. {[}1900{]}}\nopagebreak\mylabel{v}\rehead{ }\normalsize\beginnumbering\briefempfaengerindex{Schnitzler, Arthur@\textsc{Schnitzler, Arthur}!zzzGoldmann, Paul@\emph{von Paul Goldmann}!1900-08-281@{28. 8. {[}1900{]}}|(be} \toendnotes[C]{\smallbreak\pagebreak[2]} \Standort{DLA, A:Schnitzler, HS.NZ85.1.3170.}
\physDesc{Brief, 1 Blatt, 2 Seiten
\newline{}Handschrift: schwarze Tinte, deutsche Kurrent
\newline{}Schnitzler: mit Bleistift das Jahr »{[}1{]}900.« vermerkt }\toendnotes[C]{\smallbreak}\pstart
           \noindent{}\raggedleft{}{\pb}\textcolor{gray}{\textbf{\textcolor{pink}{HOTEL TRAFOI}{}\ledrightnote{{$\rightarrow$}\textcolor{pink}{Hotel Trafoi}}}}\pend
           \pstart
           \noindent{}\raggedleft{}\textcolor{gray}{\textbf{\textcolor{pink}{TIROL}{}\ledrightnote{\textcolor{pink}{Tirol}}.}}\pend
           \pstart
           \raggedleft{}28. Auguſt.\pend
           \stanza{}\label{K_L02928-1v}\edtext{\uline{Der blinde Muſikant.}}{\lemma{\textnormal{\emph{Der blinde Muſikant.}}}\Cendnote{\textnormal{Es ist davon auszugehen, dass eine
                        wahre Begegnung mit einem (blinden?) Sänger dieses \textcolor{green}{Gedicht} inspiriert hatte. \textcolor{blue}{Schnitzler} und \textcolor{blue}{Goldmann} hatten von einem »\textcolor{pink}{Tirol}er Sänger« bereits zwei
                        Tage zuvor an \textcolor{blue}{Richard Beer-Hofmann}
                        geschrieben (Arthur Schnitzler und Paul Goldmann an Richard Beer-Hofmann,
                    26. 8. 1900). Da in diesem
                           \textcolor{green}{Gedicht} explizit
                        von einem blinden Sänger die Rede ist, kann noch einmal mehr vermutet
                        werden, dass die Begegnung mit dem »\textcolor{pink}{Tirol}er Sänger« die Novelle \emph{\textcolor{green}{Der
                           blinde Geronimo}} inspirierte. XXXX sobald 1901 angelegt: auf S. 21 in
                        Transkribus bzw. den entsprechenden Brief verweisen, dort ist der ›Beweis‹
                        zur Vorlage}}}\label{K_L02928-1h}\stanzaend{}\stanza{}Ein altes Haus auf \textcolor{gray}{Pa}ſſes Höh’n\newverse{}Beſchloß die erſte Strecke;\newverse{}Da klang Harmonika-Getön\newverse{}Hervor aus dunkler Ecke.\stanzaend{}\stanza{}Gelehnt an regenfeuchte Wand,\newverse{}Von Kälte ſtarr die Glieder,\newverse{}Stand dort ein blinder Muſikant\newverse{}Und ſpielte ſeine Lieder.\stanzaend{}\stanza{}Er ſpielte und ſein Auge war\newverse{}Gerichtet ſtarr nach oben\newverse{}Und wurde doch kein Licht gewahr,\newverse{}So hoch es auch erhoben.\stanzaend{}\stanza{}{\pb}Er ſpielte luſt’ge Melodie’n\newverse{}Und ſang dazu ganz ſachte;\newverse{}Das Singen faſt ein Weinen ſchien,\newverse{}Nur daß er dazu lachte.\stanzaend{}\stanza{}Wie thut mir Deine bitt’re Noth,\newverse{}Du armer Mann, ſo wehe!\newverse{}Du mit den Augen leer und todt,\newverse{}Verzeih’ mir, daß ich ſehe!\stanzaend{}\stanza{}Bin ich gleich ſehend, ſeh’ ich \strikeout{ih\textcolor{gray}{n}} nicht,\newverse{}Du kannſt mir leicht vergeben.\newverse{}Das Licht, das heißgeliebte Licht,\newverse{}Ich ſuch’s im dunklen Leben.\stanzaend{}\stanza{}Und ſuch’ es heut und immerzu\newverse{}Und ſeh’ es nimmer gleißen.\newverse{}Oh armer blinder Bettler Du,\newverse{}Du ſollſt mich Bruder heißen! {\dotssix}\stanzaend{}\stanza{}Der Wagen rollet aus dem Thor,\newverse{}Klimmt dann auf ſteilem Pfade,\newverse{}Und lange klingt mir noch im Ohr\newverse{}Die Jammer-Serenade.\stanzaend{}\pstart
           Gruß! {\\[\baselineskip]}\spacefill\mbox{P. G.}\pend
           \leftskip=0em{}\endnumbering\briefempfaengerindex{Schnitzler, Arthur@\textsc{Schnitzler, Arthur}!zzzGoldmann, Paul@\emph{von Paul Goldmann}!1900-08-281@{28. 8. {[}1900{]}}|)be}\mylabel{h}  \normalsize

\doendnotes{C}
\bigskip
\vfill

\clearpage

\footnotesize

\lohead{\textsc{register}}

% Definiere theindex-Environment komplett neu ohne reledmac
\makeatletter
\renewenvironment{theindex}{%
  \section*{\indexname}%
  \setlength{\parindent}{0pt}%
  \setlength{\parskip}{0pt plus 0.3pt}%
  \let\item\@idxitem
}{%
  \clearpage
}
\makeatother

\IfFileExists{\jobname-pw.ind}{\input{\jobname-pw.ind}}{}

\end{document}

      