%% latex-korrekturansicht-vorspann.tex
%% Vorspann für die Korrekturansicht.
%% Lädt die gemeinsame Datei latex-vorspann.tex mit gesetztem Schalter.

\newif\ifkorrekturansicht
\korrekturansichttrue

\input{../tex-inputs/latex-vorspann}


\renewcommand{\erwaehntePersonen}{Personen: Olga Schnitzler, Jakob Wassermann, Stefan Zweig}
\renewcommand{\erwaehnteOrte}{Orte: Burgtheater, Kochgasse 8, Semmering, Wien}
\renewcommand{\erwaehnteWerke}{Werke: Das weite Land. Tragikomödie in fünf Akten, Der junge Medardus. Dramatische Historie in einem Vorspiel und fünf Aufzügen}
\renewcommand{\erwaehnteEvents}{Ereignisse: Uraufführung von Der junge Medardus, 24.11.1910}
\section[Stefan Zweig an Arthur Schnitzler, {[}zwischen 18. und 24. 10.? 1910{]}]{Stefan Zweig an Arthur Schnitzler, {[}zwischen 18. und
               24. 10.? 1910{]}}
\nopagebreak\mylabel{v}
\rehead{ }\normalsize\beginnumbering\briefempfaengerindex{Schnitzler, Arthur@\textsc{Schnitzler, Arthur}!zzzZweig, Stefan@\emph{von Stefan Zweig}!1910-10-243@{{[}zwischen 18. und
                  24. 10.? 1910{]}}|(be}
\toendnotes[C]{\smallbreak\pagebreak[2]}\Standort{CUL, Schnitzler, B 118.}
\physDesc{Brief, 1 Blatt, 2 Seiten, 788 Zeichen
\newline{}Handschrift: blaue Tinte, lateinische Kurrent
\newline{}Schnitzler: mit Bleistift »\textsc{Zweig}« }
\buchAbdrucke{\weitereDrucke{Stefan Zweig: \emph{Briefwechsel mit Hermann Bahr, Sigmund Freud, Rainer Maria
                        Rilke und Arthur Schnitzler}. Hg. Jeffrey B. Berlin, Hans-Ulrich Lindken und Donald A. Prater. Frankfurt am Main: \emph{S. Fischer} 1987, S. 361.} }\toendnotes[C]{\smallbreak}
\pstart
           {\pb}\textcolor{pink}{Wien VIII. Kochgasse 8}{}\ledrightnote{\textcolor{pink}{Kochgasse 8}}\pend
           {\vspace{1\baselineskip}}
\pstart{}Sehr verehrter Herr Doktor,\pend\vspace{0.5em}
\pstart
           ich weiss nicht, ob Sie schon \label{K_L03625-1v}\edtext{in \textcolor{pink}{Wien}{}\ledrightnote{\textcolor{pink}{Wien}}{ }}{\lemma{\textnormal{\emph{in Wien }}}\Cendnote{\textnormal{\textcolor{blue}{Schnitzler} und seine \textcolor{blue}{Frau} verbrachten die Tage vom 16. 10. 1910 bis zum
                     19. 10. 1910 am
                     \textcolor{pink}{Semmering}. \textcolor{blue}{Stefan Zweig}, der \textcolor{blue}{Schnitzler} am 11. 10. 1910 bei einer Lesung \textcolor{blue}{Jakob
                     Wassermanns} getroffen hatte, wusste vermutlich von diesem Ausflug. Nach
                  hinten zeitlich begrenzt ist der Brief durch die Erwähnung der \textcolor{violet}{Uraufführung von \emph{\textcolor{green}{Der junge
                     Medardus}} am 24. 11. 1910}. \textcolor{blue}{Schnitzler}
                  unternahm in dieser Zeit keine weiteren Reisen, es muss also von dieser die Rede
                  sein. Entsprechend dürfte dieses Korrespondenzstück in den Zeitraum fallen, der
                  zwei Tage nach Beginn der Reise und fünf Tage nach ihrem Ende anzusetzen
                  ist.}}}\label{} sind und ob ich meine Bitte an Sie richten darf. Sie will nicht viel:
               ich möchte Sie gerne wieder einmal besuchen dürfen, das ist Alles. Man erzählt mir
               viel von Ihrem neuen \textcolor{green}{Stück}{}\ledrightnote{{$\rightarrow$}\emph{\textcolor{green}{Das weite Land. Tragikomödie in fünf Akten}}} und
               so viel Gutes dass ich ganz ungeduldig werde und das Frühjahr kaum erwarten kann:
               freilich ist zuvor noch die Freude der \label{K_L03625-2v}\edtext{»\textcolor{violet}{\textcolor{green}{Medardus}{}\ledrightnote{\textcolor{green}{Der junge Medardus. Dramatische Historie in einem Vorspiel und fünf Aufzügen}}«-Première}{}\ledrightnote{\textcolor{violet}{Uraufführung von Der junge Medardus, 24.11.1910}}}{\lemma{\textnormal{\emph{»Medardus«-Première}}}\Cendnote{\textnormal{\textcolor{blue}{Schnitzlers} Schauspiel \emph{\textcolor{green}{Der junge Medardus}} wurde am 24. 11. 1910 am \textcolor{pink}{Burgtheater} uraufgeführt.}}}\label{}! Wie viel Ihnen
               doch in den letzten Jahren gelungen ist, wir, die wir Ihr Werk lieben, sind immer
                  \strikeout{noch} ungeduldig und wollen noch immer {\pb}mehr – das müssen Sie uns verzeihen,
               dass wir bei aller Liebe am Gegebenen noch nicht genug haben und uns doppelt auf das
               Werdende freuen.\pend
           
\pstart
           Mit den besten Empfehlungen an Ihre Frau \textcolor{blue}{Gemahlin}{}\ledrightnote{{$\rightarrow$}\emph{\textcolor{blue}{Olga Schnitzler}}} und den ergebensten Grüssen{\\[\baselineskip]}in Verehrung
               getreu{\\[\baselineskip]} Ihr{\\[\baselineskip]}\spacefill\mbox{Stefan Zweig}\pend
           \leftskip=0em{}\endnumbering\briefempfaengerindex{Schnitzler, Arthur@\textsc{Schnitzler, Arthur}!zzzZweig, Stefan@\emph{von Stefan Zweig}!1910-10-183@{{[}zwischen 18. und
                  24. 10.? 1910{]}}|)be}\mylabel{h}
\begin{anhang}
\end{anhang}\normalsize

\doendnotes{C}
\bigskip
\vfill

\clearpage

\footnotesize

\lohead{\textsc{register}}

% Definiere theindex-Environment komplett neu ohne reledmac
\makeatletter
\renewenvironment{theindex}{%
  \section*{\indexname}%
  \setlength{\parindent}{0pt}%
  \setlength{\parskip}{0pt plus 0.3pt}%
  \let\item\@idxitem
}{%
  \clearpage
}
\makeatother

\IfFileExists{\jobname-pw.ind}{\input{\jobname-pw.ind}}{}

\end{document}

      