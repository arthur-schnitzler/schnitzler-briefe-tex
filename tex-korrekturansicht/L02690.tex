%% latex-korrekturansicht-vorspann.tex
%% Vorspann für die Korrekturansicht.
%% Lädt die gemeinsame Datei latex-vorspann.tex mit gesetztem Schalter.

\newif\ifkorrekturansicht
\korrekturansichttrue

\input{../tex-inputs/latex-vorspann}


               \section[Paul Goldmann an Arthur Schnitzler, {[}30.? 1. 1896{]}]{ Paul Goldmann an Arthur Schnitzler, {[}30.? 1. 1896{]}}\nopagebreak\mylabel{v}\rehead{ }\normalsize\beginnumbering\briefempfaengerindex{Schnitzler, Arthur@\textsc{Schnitzler, Arthur}!zzzGoldmann, Paul@\emph{von Paul Goldmann}!1896-01-301@{{[}30.? 1. 1896{]}}|(be} \toendnotes[C]{\smallbreak\pagebreak[2]} \Standort{DLA, A:Schnitzler, HS.NZ85.1.3166.}
\physDesc{Telegramm
\newline{}maschinell
\newline{}Schnitzler: mit Bleistift datiert: »Jan\textcolor{gray}{u} 96« \newline{}Ordnung: beschnitten }\toendnotes[C]{\smallbreak}\pstart
           \centering{}{\pb}\textcolor{pink}{w}{}\ledrightnote{\textcolor{pink}{Wien}} de { }\textcolor{pink}{paris}{}\ledrightnote{\textcolor{pink}{Paris}} 18798. \label{K_L02690-1v}\edtext{30.}{\lemma{\textnormal{\emph{30.}}}\Cendnote{\textnormal{Vermutlich der Kalendertag, an dem
                     das Telegramm versandt wurde.}}}\label{K_L02690-1h}{ }12\textcolor{gray}{.} =\pend
           \pstart
           vielen dank fuer liebes \label{K_L02690-11v}\edtext{anerbieten}{\lemma{\textnormal{\emph{anerbieten}}}\Cendnote{\textnormal{Unter der Voraussetzung,
                  dass die Datierung stimmt, könnte es sich um eine Einladung nach \textcolor{pink}{Berlin} gehandelt haben, wo am 4. 2. 1896 die Premiere von \emph{\textcolor{green}{Liebelei}} am \emph{\textcolor{brown}{Deutschen Theater}} bevorstand. Da \textcolor{blue}{Schnitzler} an diesem Tag bereits in \textcolor{pink}{Berlin} ankam, bleibt unklar, ob das Telegramm dahin gesandt wurde, von
                     \textcolor{pink}{Wien} nachgesandt wurde oder (am unwahrscheinlichsten) bis zur Rückkehr
                  nach \textcolor{pink}{Wien} am 11. 2. 1896 liegen blieb.}}}\label{K_L02690-11h} aber leider
               unmoeglich aus zahlreichen gruenden hauptsaechlich geldmangel und schwierigkeit
               inmitten saison ohne zwingendsten grund urlaub zu bekommen\pend
           \pstart gruss = \spacefill\mbox{goldmann}\pend{}\endnumbering\briefempfaengerindex{Schnitzler, Arthur@\textsc{Schnitzler, Arthur}!zzzGoldmann, Paul@\emph{von Paul Goldmann}!1896-01-301@{{[}30.? 1. 1896{]}}|)be}\mylabel{h}  \normalsize

\doendnotes{C}
\bigskip
\vfill

\clearpage

\footnotesize

\lohead{\textsc{register}}

% Definiere theindex-Environment komplett neu ohne reledmac
\makeatletter
\renewenvironment{theindex}{%
  \section*{\indexname}%
  \setlength{\parindent}{0pt}%
  \setlength{\parskip}{0pt plus 0.3pt}%
  \let\item\@idxitem
}{%
  \clearpage
}
\makeatother

\IfFileExists{\jobname-pw.ind}{\input{\jobname-pw.ind}}{}

\end{document}

      