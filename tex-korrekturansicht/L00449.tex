%% latex-korrekturansicht-vorspann.tex
%% Vorspann für die Korrekturansicht.
%% Lädt die gemeinsame Datei latex-vorspann.tex mit gesetztem Schalter.

\newif\ifkorrekturansicht
\korrekturansichttrue

\input{../tex-inputs/latex-vorspann}


               \section[Hugo von Hofmannsthal an Arthur Schnitzler, 5. 6. 1895]{ Hugo von Hofmannsthal an Arthur Schnitzler, 5. 6. 1895}\nopagebreak\mylabel{v}\rehead{ }\normalsize\beginnumbering\briefempfaengerindex{Schnitzler, Arthur@\textsc{Schnitzler, Arthur}!zzzHofmannsthal, Hugo von@\emph{von Hugo von Hofmannsthal}!1895-06-051@{5. 6. 1895}|(be} \toendnotes[C]{\smallbreak\pagebreak[2]} \Standort{CUL, Schnitzler, B 43.}
\physDesc{Postkarte
\newline{}Handschrift: blaue Tinte, deutsche Kurrent\newline{}Versand: 1) Rohrpost 2) Stempel: »\nobreak{}\oindex{III., Landstrasse@\textbf{III., Landstraße}, \emph{Bezirk (A.BZK)}|pwk}Wien 3/1, 5 VI 95, 1120V\nobreak{}«. 3) Stempel: »\nobreak{}\oindex{IX., Alsergrund@\textbf{IX., Alsergrund}, \emph{Bezirk (A.BZK)}|pwk}Wien 9/\textcolor{gray}{3}, 5 VI 95, 1150V\nobreak{}«. 
\newline{}Schnitzler: mit Bleistift datiert: »5/6 95« und nummeriert: »71« }\buchAbdrucke{\weitereDrucke{Hugo von Hofmannsthal, Arthur Schnitzler: \emph{Briefwechsel}. Hg. Therese Nickl und Heinrich Schnitzler. Frankfurt am Main: \emph{S. Fischer} 1964, S. 53–54.} }\pstart{}{\pb}\textsc{Herrn D\textsuperscript{r} Arthur
                            Schnitzler}\pend{}\pstart{}\textsc{\textcolor{pink}{Franckgasse 1}{}\ledrightnote{\textcolor{pink}{Frankgasse}}}\pend{}\pstart{}\textsc{\textcolor{pink}{IX Wien}{}\ledrightnote{\textcolor{pink}{IX., Alsergrund}}}\pend{}{\bigskip}\pstart{}{\pb}lieber,\pend\pstart
           ich fahre morgen für den ganzen Tag in die \textcolor{pink}{Brühl}{}\ledrightnote{\textcolor{pink}{Brühl}}. Ko{\geminationm}en Sie nach? Jedenfalls
                    zwiſchen 4 und 6 werd ich Sie bei der \textcolor{blue}{Tini}{}\ledrightnote{\textcolor{blue}{Christine Schönberger}} erwarten oder genaue Poſt hinterlaſſen, ja?\hspace*{3.5em}Adieu.\pend
           \pstart \spacefill\mbox{Hugo.}\pend{}\endnumbering\briefempfaengerindex{Schnitzler, Arthur@\textsc{Schnitzler, Arthur}!zzzHofmannsthal, Hugo von@\emph{von Hugo von Hofmannsthal}!1895-06-051@{5. 6. 1895}|)be}\mylabel{h}  \normalsize

\doendnotes{C}
\bigskip
\vfill

\clearpage

\footnotesize

\lohead{\textsc{register}}

% Definiere theindex-Environment komplett neu ohne reledmac
\makeatletter
\renewenvironment{theindex}{%
  \section*{\indexname}%
  \setlength{\parindent}{0pt}%
  \setlength{\parskip}{0pt plus 0.3pt}%
  \let\item\@idxitem
}{%
  \clearpage
}
\makeatother

\IfFileExists{\jobname-pw.ind}{\input{\jobname-pw.ind}}{}

\end{document}

      