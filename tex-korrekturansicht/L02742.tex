%% latex-korrekturansicht-vorspann.tex
%% Vorspann für die Korrekturansicht.
%% Lädt die gemeinsame Datei latex-vorspann.tex mit gesetztem Schalter.

\newif\ifkorrekturansicht
\korrekturansichttrue

\input{../tex-inputs/latex-vorspann}


               \section[Paul Goldmann an Arthur Schnitzler, 29. 7. {[}1895{]}]{ Paul Goldmann an Arthur Schnitzler, 29. 7. {[}1895{]}}\nopagebreak\mylabel{v}\rehead{ }\normalsize\beginnumbering\briefempfaengerindex{Schnitzler, Arthur@\textsc{Schnitzler, Arthur}!zzzGoldmann, Paul@\emph{von Paul Goldmann}!1895-07-291@{29. 7. {[}1895{]}}|(be} \toendnotes[C]{\smallbreak\pagebreak[2]} \Standort{DLA, A:Schnitzler, HS.NZ85.1.3165.}
\physDesc{Brief, 2 Blätter, 6 Seiten
\newline{}Handschrift: schwarze Tinte, deutsche Kurrent
\newline{}Schnitzler: 1) mit Bleistift das Jahr »95« vermerkt 2) mit rotem Buntstift drei Unterstreichungen}\toendnotes[C]{\smallbreak}\pstart
           \noindent{}{\pb}\textcolor{gray}{\textbf{\textbf{\textcolor{brown}{Frankfurter Zeitung}{}\ledrightnote{\textcolor{brown}{Frankfurter Zeitung}}}}}\pend
           \pstart
           \textcolor{gray}{\textbf{(\textcolor{brown}{\begin{otherlanguage}{french}Gazette de Francfort\end{otherlanguage}}{}\ledrightnote{\textcolor{brown}{Frankfurter Zeitung}}). }}\pend
           \pstart
           \textcolor{gray}{\textbf{\textbf{\begin{otherlanguage}{french}Fondateur M. \textcolor{blue}{L.
                              Sonnemann}{}\ledrightnote{\textcolor{blue}{Leopold Sonnemann}}\end{otherlanguage}.}}}\pend
           \pstart
           \begin{otherlanguage}{french}\textcolor{gray}{\textbf{\textcolor{green}{Journal}{}\ledrightnote{→\textcolor{green}{Frankfurter Zeitung}} politique,
                        financier,}}\end{otherlanguage}\pend
           \pstart
           \begin{otherlanguage}{french}\textcolor{gray}{\textbf{commercial et littéraire.}}\end{otherlanguage}\pend
           \pstart
           \begin{otherlanguage}{french}\textcolor{gray}{\textbf{\textbf{Paraissant trois fois par jour.}}}\end{otherlanguage}\pend
           \pstart
           \begin{otherlanguage}{french}\textcolor{gray}{\textbf{\textbf{Bureau à \textcolor{pink}{Paris}{}\ledrightnote{\textcolor{pink}{Paris}}:}}}\end{otherlanguage}\hfill \textsc{\textcolor{pink}{Paris}{}\ledrightnote{\textcolor{pink}{Paris}}}, 29. Juli.\pend
           \pstart
           \begin{otherlanguage}{french}\textcolor{gray}{\textbf{\textbf{\textcolor{pink}{24. Rue Feydeau}{}\ledrightnote{\textcolor{pink}{rue Feydeau}}.}}}\end{otherlanguage}\pend
           \pstart\center{}Mein lieber Freund,\pend\pstart
           Vielen Dank für Deinen lieben Brief! \pend
           \pstart
           Mittwoch od. Donnerſtag
               fahre ich von hier fort, gedenke einen Tag in \strikeout{\textsc{\textcolor{pink}{S\textcolor{gray}{t}ras}{}\ledrightnote{→\textcolor{pink}{Straßburg}}}}{ }\textsc{\textcolor{pink}{Strassburg}{}\ledrightnote{\textcolor{pink}{Straßburg}}} mich aufzuhalten, dann zwei oder drei Tage in \textsc{\textcolor{pink}{Muenchen}{}\ledrightnote{\textcolor{pink}{München}}}, wo ich im »\textsc{\textcolor{pink}{Hotel Marienbad}{}\ledrightnote{\textcolor{pink}{Hotel Marienbad}}}« wohnen werde (dies für etwaige Nachrichten). Dann nach \textsc{\textcolor{pink}{Toelz}{}\ledrightnote{\textcolor{pink}{Bad Tölz}}}. Ich habe diesmal {\pb}fünf bis ſechs Wochen
               Urlaub. Wenns der Arzt verlangt, ſo muß ich ſie natürlich ganz auf die Kur verwenden.
               Sollten vier Wochen genügen, ſo möchte ich gern – falls ich noch Geld habe – ſo etwa
               acht Tage irgendwo in der Welt mit Euch zuſammenſein. Jedenfalls ſehe ich mit Freude,
               daß ich Ausſicht habe, Dich ſchon vorher zu ſehen. Mein Wunſch iſt nur, daß es
               möglichſt lange wäre. Nachrichten erreichen mich {\pb}\uline{nach}{ }\textsc{\textcolor{pink}{Muenchen}{}\ledrightnote{\textcolor{pink}{München}}} zunächſt \textsc{\textcolor{pink}{Toelz}{}\ledrightnote{\textcolor{pink}{Bad Tölz}}} (\textsc{\textcolor{pink}{Baiern}{}\ledrightnote{\textcolor{pink}{Bayern}}}) \textsc{Poste-restante}. Kommt die Frau \label{K_L02742-1v}\edtext{\textsc{\textcolor{blue}{Andreas}{}\ledrightnote{\textcolor{blue}{Lou Andreas-Salomé}}} nach \textsc{\textcolor{pink}{Salzburg}{}\ledrightnote{\textcolor{pink}{Salzburg}}}}{\lemma{\textnormal{\emph{Andreas nach Salzburg}}}\Cendnote{\textnormal{Siehe die \emph{\textcolor{green}{Tagebuch}}-Einträge zwischen 20. 8. 1895 und 6. 9. 1895.}}}\label{K_L02742-1h}, ſo gehe ich vielleicht auch
               hinüber. Was Du \textsc{\textcolor{blue}{Richard}{}\ledrightnote{\textcolor{blue}{Richard Beer-Hofmann}}}{ }\label{K_L02742-2v}\edtext{ſagen ſollſt}{\lemma{\textnormal{\emph{ſagen ſollſt}}}\Cendnote{\textnormal{wohl im Hinblick auf die frühere Beziehung \textcolor{blue}{Paul Goldmann}s zu \textcolor{blue}{Lou
                     Andreas-Salomé} zu verstehen, mit der \textcolor{blue}{Richard Beer-Hofmann} seit wenigen Wochen intim war.}}}\label{K_L02742-2h}, weiß ich nicht. Ich gebe Dir
               Vollmacht, zu ſagen, was Du willſt. Mir widerſtrebt es, ihn anzulügen. Ich danke Dir
               für die Mittheilung deſſen, was \label{K_L02742-3v}\edtext{\textsc{\textcolor{blue}{Loris}{}\ledrightnote{\textcolor{blue}{Hugo von Hofmannsthal}}}}{\lemma{\textnormal{\emph{Loris}}}\Cendnote{\textnormal{\textcolor{blue}{Schnitzler} dürfte \textcolor{blue}{Goldmann} aus \textcolor{blue}{Hugo von
                     Hofmannsthal}s Brief vom 17. [7. 1895] zitiert haben, in dem dieser schrieb: »Als ein
                     besonders merkwürdiger Tag erscheint mir der, wo wir mit \textcolor{blue}{Goldmann}{ }{[}{\dots}{]} waren und dann
                     ein großes Gewitter gekommen ist. Ich kann aber nicht finden,
                  warum.«}}}\label{K_L02742-3h} geſchrieben. Es iſt ſehr hübſch – nur weiß man nicht recht,
               was eigentlich an der Sache merkwürdig war, {\pb}\textsc{Goldmann} oder das \strikeout{Gew\textcolor{gray}{i}tter} Gewitter?{\dotsfour}\pend
           \pstart
           \textsc{\textcolor{blue}{Herzl}{}\ledrightnote{\textcolor{blue}{Theodor Herzl}}} iſt vorgeſtern nach \textsc{\textcolor{pink}{Aussee}{}\ledrightnote{\textcolor{pink}{Bad Aussee}}} abgereiſt. Ich bin innnerlich ganz fertig mit ihm. Äußerlich hält es nur noch
               durch ein paar recht lockere Fäden zuſammen. Der \label{K_L02742-55v}\edtext{\textcolor{pink}{ungar}{}\ledrightnote{→\textcolor{pink}{Ungarn}}iſche \textcolor{blue}{Saujud}{}\ledrightnote{→\textcolor{blue}{Theodor Herzl}}}{\lemma{\textnormal{\emph{ungariſche Saujud}}}\Cendnote{\textnormal{Die auf \textcolor{blue}{Herzl}s
                        zunehmende Neuorientierung vom literarischen Schriftsteller zum Zionisten wird hier durch \textcolor{blue}{Goldmann}
                        mit einer überraschend groben Ausdrucksweise kommentiert. Dies dürfte als Hinweis zu lesen sein,
                        dass \textcolor{blue}{Goldmann} den richtigen Umgang mit der jüdischen Kultur in
                        der Assimilation sah, während er bei \textcolor{blue}{Herzl} wahrnehmen wollte,
                        dass dieser das verarmte Judentum aus dem Osten der k. k. Monarchie nicht per se 
                        ablehnte.}}}\label{K_L02742-55h} kommt immer deutlicher \strikeout{\textcolor{gray}{aus}} unter dem \textcolor{blue}{Literaten}{}\ledrightnote{→\textcolor{blue}{Theodor Herzl}}
               hervor, und das wird unerträglich. Ich glaube es wächſt ein \strikeout{ſold} ſolider Haß heran zwiſchen ihm u. mir.\pend
           \pstart
           Was geht mit Deinem \textcolor{green}{Stücke}{}\ledrightnote{→\textcolor{green}{Liebelei. Schauspiel in drei Akten}} vor, daß Du ſo reſignirt über das {\pb}Warten auf Erfolg ſprichſt? Nun, ich höre es ja nächſtens wohl mündlich. Gewiß, Du
               ſollſt den Erfolg nicht erwarten. Laß’ \strikeout{d} das nur
               gehn, das thue ich ſchon für Dich.\pend
           \pstart
           Daß Du »\textcolor{green}{Freiwild}{}\ledrightnote{\textcolor{green}{Freiwild. Schauspiel in 3 Akten}}« ſchreibſt, freut mich ſehr. Du
               haſt Recht: die Arbeit iſt bei dem Allem das Schönſte. Oh, wer arbeiten könnte, \strikeout{\textcolor{gray}{×}} wie Du! Alles gute Glück {\pb}zum \textcolor{green}{Werke}{}\ledrightnote{→\textcolor{green}{Freiwild. Schauspiel in 3 Akten}}!{\dotsfour}\pend
           \pstart
           Grüß’ Dich Gott, mein lieber Freund. Nun wird man ſich bald ſehen. Wie ich mich
               freue!!{\dotstwo}\pend
           \pstart
           Dein treuer {\\[\baselineskip]}\spacefill\mbox{Paul Goldmann {\dotstwo}}\pend
           \leftskip=0em{}\pstart
           \noindent{}Ich weiß \textsc{\textcolor{blue}{Richard}{}\ledrightnote{\textcolor{blue}{Richard Beer-Hofmann}}s} Adreſſe nicht. Bitte,
                     gib\textcolor{gray}{’} ihm inliegenden \label{K_L02742-4v}\edtext{Brief}{\lemma{\textnormal{\emph{Brief}}}\Cendnote{\textnormal{Der Brief, datiert vom 29. 7. {[}1895{]} ist
                        im Nachlass \textcolor{blue}{Beer-Hofmann}s in der \emph{Houghton Library}, Harvard (825978) überliefert.
                     \textcolor{blue}{Goldmann} bedankt sich für Fotografien,
                        eine von \textcolor{blue}{Beer-Hofmann}, die ander von dessen
                        Hund »Flirt«. \textcolor{blue}{Goldmann} berichtet von seinem
                        eigenen Pudel und freut sich auf das bevorstehende Wiedersehen.}}}\label{K_L02742-4h}.\pend
           \endnumbering\briefempfaengerindex{Schnitzler, Arthur@\textsc{Schnitzler, Arthur}!zzzGoldmann, Paul@\emph{von Paul Goldmann}!1895-07-291@{29. 7. {[}1895{]}}|)be}\mylabel{h}  \normalsize

\doendnotes{C}
\bigskip
\vfill

\clearpage

\footnotesize

\lohead{\textsc{register}}

% Definiere theindex-Environment komplett neu ohne reledmac
\makeatletter
\renewenvironment{theindex}{%
  \section*{\indexname}%
  \setlength{\parindent}{0pt}%
  \setlength{\parskip}{0pt plus 0.3pt}%
  \let\item\@idxitem
}{%
  \clearpage
}
\makeatother

\IfFileExists{\jobname-pw.ind}{\input{\jobname-pw.ind}}{}

\end{document}

      