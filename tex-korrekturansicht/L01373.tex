%% latex-korrekturansicht-vorspann.tex
%% Vorspann für die Korrekturansicht.
%% Lädt die gemeinsame Datei latex-vorspann.tex mit gesetztem Schalter.

\newif\ifkorrekturansicht
\korrekturansichttrue

\input{../tex-inputs/latex-vorspann}


               \section[Hermann Bahr an Arthur Schnitzler, 15. 2. 1904]{ Hermann Bahr an Arthur Schnitzler, 15. 2. 1904}\nopagebreak\mylabel{v}\rehead{ }\normalsize\beginnumbering\briefempfaengerindex{Schnitzler, Arthur@\textsc{Schnitzler, Arthur}!zzzBahr, Hermann@\emph{von Hermann Bahr}!1904-02-151@{15. 2. 1904}|(be} \toendnotes[C]{\smallbreak\pagebreak[2]} \Standort{CUL, Schnitzler, B 5b.}
\physDesc{Kartenbrief
\newline{}Handschrift: schwarze Tinte, deutsche Kurrent\newline{}Versand: 1) Stempel: »\nobreak{}\oindex{Opatija@\textbf{Opatija}, \emph{http://www.geonames.org/ontologyP.PPLA2}|pwk}Abbazia, 15. 2. 04\nobreak{}«.  2) Stempel: »\nobreak{}\oindex{XVIII., Waehring@\textbf{XVIII., Währing}, \emph{Bezirk (A.BZK)}|pwk}18/1 Wien, 17. 2. 04, 8.V, Bestellt\nobreak{}«. 
\newline{}Schnitzler: mit rotem Buntstift eine Unterstreichung \newline{}Ordnung: mit Bleistift von unbekannter Hand nummeriert:
                           »111« }\buchAbdrucke{\weitereDrucke{Hermann Bahr, Arthur Schnitzler: \emph{Briefwechsel, Aufzeichnungen, Dokumente (1891–1931)}. Hg. Kurt Ifkovits und Martin Anton Müller. Göttingen: \emph{Wallstein} 2018, S. 300.} }\toendnotes[C]{\smallbreak}\pstart{}{\pb}Herrn \textsc{D\textsuperscript{r} Arthur Schnitzler}\pend{}\pstart{}\textcolor{pink}{\textsc{Wien XVIII}}{}\ledrightnote{\textcolor{pink}{XVIII., Währing}}\pend{}\pstart{}\textcolor{pink}{Spöttelgaſſe 7}{}\ledrightnote{\textcolor{pink}{Edmund-Weiß-Gasse}}\pend{}{\bigskip}\pstart
           \raggedleft{}{\pb}15. 2. 04{\\}\textcolor{pink}{Abbazia}{}\ledrightnote{\textcolor{pink}{Opatija}}{ }\textcolor{pink}{Hot. \textsc{Guarnero}}{}\ledrightnote{\textcolor{pink}{Hotel Guarnero}}\pend
           \pstart{}Lieber Arthur!\pend\pstart
           Ich kam heut hier an und weil der \textcolor{blue}{Trebitſch}{}\ledrightnote{\textcolor{blue}{Siegfried Trebitsch}}, der
               mir ein Telegra{\geminationm} verſprochen, es verbummelt hat, ließ
               ich mich verleiten, in den \textcolor{pink}{Wiener}{}\ledrightnote{\textcolor{pink}{Wien}} Zeitungen
               nachzuſehen, deren Ton aber ſo hundsgemein iſt, daß ich ihn phyſiſch nicht mehr
               vertrage. Und nun nachdem ich mich unſinnig geärgert hab, weiß ich zudem natürlich
               gar nichts: wars ein Erfolg, wars keiner? Ich weiß aber, daß das \textcolor{green}{Stück}{}\ledrightnote{→\textcolor{green}{Der einsame Weg. Schauspiel in fünf Akten}} zu Deinen ſchönſten und reinſten
               Arbeiten gehört, und ich mein, wir ſollten uns überhaupt nicht mehr zu Erfolgen,
               sondern zu den Werken, die uns etwas ſind, gratulieren. Mir iſt der »\textcolor{green}{einſame Weg}{}\ledrightnote{\textcolor{green}{Der einsame Weg. Schauspiel in fünf Akten}}« in ſeinen Hauptgeſtalten und ihrem Erleben sehr
               viel.\pend
           \pstart Herzlichſt \hspace*{1.5em}Dein \spacefill\mbox{Hermann}\pend{}\endnumbering\briefempfaengerindex{Schnitzler, Arthur@\textsc{Schnitzler, Arthur}!zzzBahr, Hermann@\emph{von Hermann Bahr}!1904-02-151@{15. 2. 1904}|)be}\mylabel{h}  \normalsize

\doendnotes{C}
\bigskip
\vfill

\clearpage

\footnotesize

\lohead{\textsc{register}}

% Definiere theindex-Environment komplett neu ohne reledmac
\makeatletter
\renewenvironment{theindex}{%
  \section*{\indexname}%
  \setlength{\parindent}{0pt}%
  \setlength{\parskip}{0pt plus 0.3pt}%
  \let\item\@idxitem
}{%
  \clearpage
}
\makeatother

\IfFileExists{\jobname-pw.ind}{\input{\jobname-pw.ind}}{}

\end{document}

      