%% latex-korrekturansicht-vorspann.tex
%% Vorspann für die Korrekturansicht.
%% Lädt die gemeinsame Datei latex-vorspann.tex mit gesetztem Schalter.

\newif\ifkorrekturansicht
\korrekturansichttrue

\input{../tex-inputs/latex-vorspann}


               \section[Hermann Bahr an Arthur Schnitzler, 21. 10. {[}1901{]}]{ Hermann Bahr an Arthur Schnitzler, 21. 10. {[}1901{]}}\nopagebreak\mylabel{v}\rehead{ }\normalsize\beginnumbering\briefempfaengerindex{Schnitzler, Arthur@\textsc{Schnitzler, Arthur}!zzzBahr, Hermann@\emph{von Hermann Bahr}!1901-10-211@{21. 10. {[}1901{]}}|(be} \toendnotes[C]{\smallbreak\pagebreak[2]} \Standort{CUL, Schnitzler, B 5b.}
\physDesc{Brief, 1 Blatt, 1 Seite
\newline{}Handschrift: blaue Tinte, deutsche Kurrent
\newline{}Schnitzler: mit Bleistift die Jahreszahl »901.« ergänzt \newline{}Ordnung: mit Bleistift von unbekannter Hand nummeriert: »81« }\buchAbdrucke{\weitereDrucke{Hermann Bahr, Arthur Schnitzler: \emph{Briefwechsel, Aufzeichnungen, Dokumente (1891–1931)}. Hg. Kurt Ifkovits und Martin Anton Müller. Göttingen: \emph{Wallstein} 2018, S. 216.} }\toendnotes[C]{\smallbreak}\pstart
           \raggedleft{}{\pb}21. 10.\pend
           \pstart\center{}Lieber Arthur!\pend\pstart
           Ich theile Deine Bedenken betr. »\textcolor{green}{Puppenſpieler}{}\ledrightnote{\textcolor{green}{Der Puppenspieler}}«
               – da wir leider \textcolor{blue}{Mitterwurzer}{}\ledrightnote{\textcolor{blue}{Friedrich Mitterwurzer}} nicht haben. Die
                  \textcolor{green}{Stücke}{}\ledrightnote{→\textcolor{green}{Die Frau mit dem Dolche}{\newline}→\textcolor{green}{Literatur}{\newline}→\textcolor{green}{Lebendige Stunden}}{ }ſind ſeit acht Tagen bei \textcolor{blue}{\textsc{Bukovics}}{}\ledrightnote{\textcolor{blue}{Emerich von Bukovics}}, der mir geſtern ſagte, mit der Lectüre fertig zu ſein und
               Dich \label{K_L01182_1v}\edtext{heute}{\lemma{\textnormal{\emph{heute}}}\Cendnote{\textnormal{Am Folgetag suchte \textcolor{blue}{Schnitzler}{ }\textcolor{blue}{Bukovics} auf.}}}\label{K_L01182_1h} aufſuchen zu
               wollen.\pend
           \pstart
           Herzlichſt{\\[\baselineskip]}Dein{\\[\baselineskip]}\spacefill\mbox{Herm}\pend
           \leftskip=0em{}\endnumbering\briefempfaengerindex{Schnitzler, Arthur@\textsc{Schnitzler, Arthur}!zzzBahr, Hermann@\emph{von Hermann Bahr}!1901-10-211@{21. 10. {[}1901{]}}|)be}\mylabel{h}  \normalsize

\doendnotes{C}
\bigskip
\vfill

\clearpage

\footnotesize

\lohead{\textsc{register}}

% Definiere theindex-Environment komplett neu ohne reledmac
\makeatletter
\renewenvironment{theindex}{%
  \section*{\indexname}%
  \setlength{\parindent}{0pt}%
  \setlength{\parskip}{0pt plus 0.3pt}%
  \let\item\@idxitem
}{%
  \clearpage
}
\makeatother

\IfFileExists{\jobname-pw.ind}{\input{\jobname-pw.ind}}{}

\end{document}

      