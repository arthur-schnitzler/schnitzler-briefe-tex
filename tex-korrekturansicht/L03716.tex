%% latex-korrekturansicht-vorspann.tex
%% Vorspann für die Korrekturansicht.
%% Lädt die gemeinsame Datei latex-vorspann.tex mit gesetztem Schalter.

\newif\ifkorrekturansicht
\korrekturansichttrue

\input{../tex-inputs/latex-vorspann}


\section[Elsa Plessner an Arthur Schnitzler, 23. 10. 1897]{L03716 Elsa Plessner an Arthur Schnitzler, 23. 10. 1897}
\nopagebreak\mylabel{L03716v}
\rehead{ }\normalsize\beginnumbering\briefempfaengerindex{Schnitzler, Arthur@\textsc{Schnitzler, Arthur}!zzzPlessner, Elsa@\emph{von Elsa Plessner}!1897-10-231@{23. 10. 1897}|(be}
\toendnotes[C]{\smallbreak\pagebreak[2]}
\correspDesc{Versand  durch Elsa Plessner am 23. 10. 1897 in Wien
\newline{}Erhalt  durch Arthur Schnitzler im Zeitraum [24. 10. 1897 – 28. 10. 1897?] in Wien}\toendnotes[C]{\smallbreak}
\Standort{DLA, A:Schnitzler, HS.1985.1.419.}
\physDesc{Brief, 1 Blatt, 2 Seiten, 1080 Zeichen
\newline{}Handschrift: schwarze Tinte, lateinische Kurrent}\toendnotes[C]{\smallbreak}
\pstart
           \raggedleft{}{\pb}\textcolor{pink}{Wien, I. Spiegelgasse N\textsuperscript{o} 2}\oindex{Wien@\textbf{Wien}!I., Innere Stadt@\textbf{I., Innere Stadt}!Spiegelgasse 2@\textbf{Spiegelgasse 2}, \emph{Wohngebäude}|pw}{}\ledrightnote{\textcolor{pink}{Spiegelgasse 2}}, den 23. October 1897.\pend
           
\pstart
           \raggedleft{}Telef. N\textsuperscript{r} 7819\pend
           
\pstart\center{}Verehrter Herr Doctor!\pend\vspace{0.5em}
\pstart
           Nachdem ich Sie einige Zeit in Ruhe gelassen habe, übermittle ich heute wieder einmal
               eine neue kleine \textcolor{green}{Arbeit}\pwindex{Plessner, Elsa 22.\,8.\,1875 Wien – 7.\,5.\,1932 Alicante@\textsc{Plessner, Elsa} (22.\,8.\,1875 Wien – 7.\,5.\,1932 Alicante), \emph{Schriftstellerin}!Irmedals Kummer@\strich\emph{Irmedals Kummer}|pwv}{}\ledrightnote{{$\rightarrow$}\emph{\textcolor{green}{Irmedals Kummer}}} Ihrem
               Urtheil.\pend
           
\pstart
           Wenn sie Ihnen gefiele, würde ich mich sehr, sehr freuen. Ich glaube, relativ
               anständig, das heißt ohne stylistische Schlampereien gearbeitet zu haben. Vielleicht
               ist dieser Märchenstyl ein wenig »cliché« allein ich habe ihn mit Vorbedacht benutzt,
               und zwar gerade die gebräuchlichsten Wendungen, blos \strikeout{von} der besseren, satirischen Wirkung halber.
               Natürlich nehme ich das \textcolor{green}{Ding}\pwindex{Plessner, Elsa 22.\,8.\,1875 Wien – 7.\,5.\,1932 Alicante@\textsc{Plessner, Elsa} (22.\,8.\,1875 Wien – 7.\,5.\,1932 Alicante), \emph{Schriftstellerin}!Irmedals Kummer@\strich\emph{Irmedals Kummer}|pwv}{}\ledrightnote{{$\rightarrow$}\emph{\textcolor{green}{Irmedals Kummer}}}
               nicht als »\begin{otherlanguage}{french}\label{K_L03716-1v}\edtext{grande chose}{\lemma{\textnormal{\emph{grande chose}}}\Cendnote{\textnormal{französisch: große Sache}}}\label{K_L03716-1}{ }\end{otherlanguage}«, allein ich habe seit \uuline{¾} Jahren meine
               Feder überhaupt nur zu Briefen spazieren geführt. Darum ist mir »\textcolor{green}{Irmedals Kum{\pb}mer}\pwindex{Plessner, Elsa 22.\,8.\,1875 Wien – 7.\,5.\,1932 Alicante@\textsc{Plessner, Elsa} (22.\,8.\,1875 Wien – 7.\,5.\,1932 Alicante), \emph{Schriftstellerin}!Irmedals Kummer@\strich\emph{Irmedals Kummer}|pw}{}\ledrightnote{\textcolor{green}{Irmedals Kummer}}« sehr werth.– – –\pend
           
\pstart
           \label{K_L03716-2v}\edtext{Ein neues \textcolor{green}{Stück}\pwindex{Plessner, Elsa 22.\,8.\,1875 Wien – 7.\,5.\,1932 Alicante@\textsc{Plessner, Elsa} (22.\,8.\,1875 Wien – 7.\,5.\,1932 Alicante), \emph{Schriftstellerin}!erste Kapitel. Schauspiel in drei Akten@\strich\emph{Das erste Kapitel. Schauspiel in drei Akten}|pwuv}{}\ledrightnote{{$\rightarrow$}\emph{\textcolor{green}{Das erste Kapitel. Schauspiel in drei Akten}}}}{\lemma{\textnormal{\emph{Ein neues Stück}}}\Cendnote{\textnormal{Möglicherweise handelt es sich um erste Entwürfe zum
                  Schaupiel \emph{\textcolor{green}{Die Ehrlosen}\pwindex{Plessner, Elsa 22.\,8.\,1875 Wien – 7.\,5.\,1932 Alicante@\textsc{Plessner, Elsa} (22.\,8.\,1875 Wien – 7.\,5.\,1932 Alicante), \emph{Schriftstellerin}!Ehrlosen. Schauspiel in drei Acten@\strich\emph{Die Ehrlosen. Schauspiel in drei Acten}|pwk}}, dessen Entstehung
                  Plessner im Brief vom 19. 1. 1899 allerdings auf den Herbst 1898
                  datiert.}}}\label{K_L03716-2} liegt auf der Pfanne. Ende Dezember dürften Sie davon
               ereilt werden, Sie, verehrter Herr, der Sie so liebenswürdig der Puffer meines
               künstlerischen Zuges sind. – – Wenn es mir endlich einmal was werden möchte. Weiß
               wirklich nicht, wie es ausfallen wird.\pend
           
\pstart
           Abwarten!\pend
           
\pstart
           Viele, viele Grüße in aufrichtiger, waschechter Verehrung{\\[\baselineskip]}\spacefill\mbox{Elsa Plessner}\pend
           \leftskip=0em{}\selectlanguage{ngerman}\endnumbering\briefempfaengerindex{Schnitzler, Arthur@\textsc{Schnitzler, Arthur}!zzzPlessner, Elsa@\emph{von Elsa Plessner}!1897-10-231@{23. 10. 1897}|)be}\mylabel{L03716h}  \normalsize

\doendnotes{C}
\bigskip
\vfill

\clearpage

\footnotesize

\lohead{\textsc{register}}

% Definiere theindex-Environment komplett neu ohne reledmac
\makeatletter
\renewenvironment{theindex}{%
  \section*{\indexname}%
  \setlength{\parindent}{0pt}%
  \setlength{\parskip}{0pt plus 0.3pt}%
  \let\item\@idxitem
}{%
  \clearpage
}
\makeatother

\IfFileExists{\jobname-pw.ind}{\input{\jobname-pw.ind}}{}

\end{document}

      