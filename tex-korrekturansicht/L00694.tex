%% latex-korrekturansicht-vorspann.tex
%% Vorspann für die Korrekturansicht.
%% Lädt die gemeinsame Datei latex-vorspann.tex mit gesetztem Schalter.

\newif\ifkorrekturansicht
\korrekturansichttrue

\input{../tex-inputs/latex-vorspann}


               \section[Arthur Schnitzler an Hugo von Hofmannsthal, 8. 7. 1897]{ Arthur Schnitzler an Hugo von Hofmannsthal,
                    8. 7. 1897}\nopagebreak\mylabel{v}\rehead{ }\normalsize\beginnumbering\briefempfaengerindex{Hofmannsthal, Hugo von@\textsc{Hofmannsthal, Hugo von}!zzzSchnitzler, Arthur@\emph{von Arthur Schnitzler}!1897-07-081@{8. 7. 1897}|(be} \toendnotes[C]{\smallbreak\pagebreak[2]} \Standort{FDH, Hs-30885,59.}
\physDesc{Brief, 1 Blatt, 4 Seiten
\newline{}Handschrift: schwarze Tinte, deutsche Kurrent}\buchAbdrucke{\weitereDrucke{1) Hugo von Hofmannsthal, Arthur Schnitzler: \emph{Briefwechsel}. Hg. Therese Nickl und Heinrich Schnitzler. Frankfurt am Main: \emph{S. Fischer} 1964, S. 88–89.} \weitereDrucke{2) Arthur Schnitzler: \emph{Briefe 1875–1912}. Hg. Therese Nickl und Heinrich Schnitzler. Frankfurt am Main: \emph{S. Fischer} 1981, S. 334–335.} }\toendnotes[C]{\smallbreak}\pstart
           {\pb}\textcolor{pink}{\textsc{Ischl}}{}\ledrightnote{\textcolor{pink}{Bad Ischl}}{ }8. 7. 97\pend
           \pstart
           Mein lieber Hugo, geſtern iſt Ihr Brief aus der \textcolor{pink}{Fuſch}{}\ledrightnote{\textcolor{pink}{Bad Fusch}} geko{\geminationm}en. Ich freue mich
                    ſehr, dſs es Ihnen gut geht und weiſs dſs manche von den Verſen die Sie
                    »verſuchen«, Ihnen gelingen werden. Glauben Sie das nicht ſelbſt? Ich ſelbſt
                    ſchreibe an einem \textcolor{green}{Stück}{}\ledrightnote{→\textcolor{green}{Das Vermächtnis. Schauspiel in drei Akten}},
                    deſſen zweiten Akt ich heute bego{\geminationn}en habe. Es iſt
                    nicht das, was ich mir vorgeno{\geminationm}en habe, ſondern ein
                    andres, das mir als Einfall bereits vor ein paar Monaten in \textcolor{pink}{Wien}{}\ledrightnote{\textcolor{pink}{Wien}} geko{\geminationm}en und mir
                    plötzlich, in den zwei erſten Tagen meines \textcolor{pink}{Iſchl}{}\ledrightnote{\textcolor{pink}{Bad Ischl}}er {\pb}Aufenthalts mit großer
                    Lebendigkeit, Scene für Scene klar geworden iſt. Ich habe den erſten \textcolor{green}{Akt}{}\ledrightnote{→\textcolor{green}{Das Vermächtnis. Schauspiel in drei Akten}} mit viel Liebe
                    geſchrieben, bin gegen den Schluſs mistrauiſch geworden und fand ihn beim
                    Durchleſen vorgeſtern blaſs. Aus verschiedenen Gründen iſt die ganze Sti{\geminationm}ung wieder ins dunklere hineingerathen, aber die
                    Hoffnung, dſs es wieder beſſer wird, darf beſtehn. Ich werde weiter arbeiten,
                    wie man unter drohenden Wolken weiterfährt; (was doch eigentlich ein recht
                    ſtupider Vergleich iſt.) ((Ich hätt ihn doch ausſtreichen können, ganz
                    einfach?)) \pend
           \pstart
           {\pb}Ich muſs vielleicht bald nach \textcolor{pink}{Wien}{}\ledrightnote{\textcolor{pink}{Wien}}, da ich in der Wohnungsfrage in der beka{\geminationn}ten, noch mancherlei oder vielmehr alles zu
                    ordnen habe. Das urſprünglich geplante Häuschen im Gebirg ist mir weggeſchnappt
                    worden. Es iſt ſehr ärgerlich. Natürlich bleibt es trotzdem bei unſerm \textcolor{pink}{Salzburg}{}\ledrightnote{\textcolor{pink}{Salzburg}}, und ich freu mich ſehr darauf. Sagen
                    Sie mir nur gleich das genaue Datum, da ich mit den Tagen haushalten muſs.\pend
           \pstart
           Morgen ſchicke ich Ihnen den \textcolor{green}{2. Band \textcolor{blue}{Mozart}{}\ledrightnote{\textcolor{blue}{Wolfgang Amadeus Mozart}}}{}\ledrightnote{→\textcolor{green}{W. A. Mozart}}. – \textcolor{blue}{Richard}{}\ledrightnote{\textcolor{blue}{Richard Beer-Hofmann}} arbeitet wirklich; er
                    ſcheint im dritten \textcolor{green}{Capitel}{}\ledrightnote{→\textcolor{green}{Der Tod Georgs}}
                    zu ſein. {\pb}Wenigſtens hat er kaum zu was anderm
                    Zeit und ist eine Radelraunzen wie ein kleines Kind.\pend
           \pstart
           Neulich bin ich nach \textcolor{pink}{Unterach}{}\ledrightnote{\textcolor{pink}{Unterach am Attersee}} zu \textcolor{blue}{Stri}{}\ledrightnote{\textcolor{blue}{Bernhard Strisower}{\newline}\textcolor{blue}{Friederike Strisower}}’s geradelt; ſonſt mach ich nur
                    ganz kleine Spazierfahrten, und plaudre mit einer merkwürdig geſcheiten \textcolor{blue}{Frau}{}\ledrightnote{→\textcolor{blue}{Rosa Freudenthal}}{ }ſehr viel, die Humor
                    hat, und ich verſuche mich zu erinnern, ob ich ſchon je eine Frau mit Humor
                    gekannt habe. –\pend
           \pstart
           Schreiben Sie mir bald.\pend
           \pstart
           Ich leſe noch immer \textcolor{green}{\textcolor{blue}{\textsc{Tolstoi}}{}\ledrightnote{\textcolor{blue}{Leo N. von Tolstoi}}}{}\ledrightnote{→\textcolor{green}{Krieg und Frieden}} u \textcolor{green}{\textcolor{blue}{\textsc{Brandes}}{}\ledrightnote{\textcolor{blue}{Georg Brandes}}}{}\ledrightnote{→\textcolor{green}{William Shakespeare}}.\pend
           \pstart
           Herzlich der Ihre{\\[\baselineskip]}\spacefill\mbox{Arthur.}\pend
           \leftskip=0em{}\endnumbering\briefempfaengerindex{Hofmannsthal, Hugo von@\textsc{Hofmannsthal, Hugo von}!zzzSchnitzler, Arthur@\emph{von Arthur Schnitzler}!1897-07-081@{8. 7. 1897}|)be}\mylabel{h}  \normalsize

\doendnotes{C}
\bigskip
\vfill

\clearpage

\footnotesize

\lohead{\textsc{register}}

% Definiere theindex-Environment komplett neu ohne reledmac
\makeatletter
\renewenvironment{theindex}{%
  \section*{\indexname}%
  \setlength{\parindent}{0pt}%
  \setlength{\parskip}{0pt plus 0.3pt}%
  \let\item\@idxitem
}{%
  \clearpage
}
\makeatother

\IfFileExists{\jobname-pw.ind}{\input{\jobname-pw.ind}}{}

\end{document}

      