%% latex-korrekturansicht-vorspann.tex
%% Vorspann für die Korrekturansicht.
%% Lädt die gemeinsame Datei latex-vorspann.tex mit gesetztem Schalter.

\newif\ifkorrekturansicht
\korrekturansichttrue

\input{../tex-inputs/latex-vorspann}


\renewcommand{\erwaehntePersonen}{Personen: Heinrich Kanner, Isidor Singer}
\renewcommand{\erwaehnteInstitutionen}{Institutionen: Die Zeit}
\renewcommand{\erwaehnteOrte}{Orte: Leoben, Niederösterreich, Steiermark, Wien, Wipplingerstraße}
\renewcommand{\erwaehnteWerke}{Werke: Die Zeit, Die griechische Tänzerin. Novellette}
\section[ Felix Salten an Arthur Schnitzler, 2. 9. 1902]{Felix Salten an Arthur Schnitzler, 2. 9. 1902}
\nopagebreak\mylabel{v}
\rehead{ }\normalsize\beginnumbering\briefempfaengerindex{Schnitzler, Arthur@\textsc{Schnitzler, Arthur}!zzzSalten, Felix@\emph{von Felix Salten}!1902-09-022@{2. 9. 1902}|(be}
\toendnotes[C]{\smallbreak\pagebreak[2]}\Standort{CUL, Schnitzler, B 89, A 2.}
\physDesc{Brief, 1 Blatt, 1 Seite, 1175 Zeichen
\newline{}Handschrift: schwarze Tinte, lateinische Kurrent
\newline{}Ordnung: mit Bleistift von unbekannter Hand nummeriert: »158« }\toendnotes[C]{\smallbreak}
\pstart
           \noindent{}{\pb}\textcolor{gray}{\textbf{DIE}}\pend
           
\pstart
           \textcolor{gray}{\textbf{\textcolor{brown}{ZEIT}{}\ledrightnote{\textcolor{brown}{Die Zeit}}}}\hfill \textcolor{gray}{\textbf{\emph{\textcolor{pink}{WIEN}{}\ledrightnote{\textcolor{pink}{Wien}}},}}{ }2. Septemb. \textcolor{gray}{\textbf{\emph{190}}}2.\pend
           
\pstart
           \textcolor{gray}{\textbf{\textsc{\textbf{\so{Wiener Tagblatt}}}}}\pend
           
\pstart
           \textcolor{gray}{\textbf{HERAUSGEBER:}}\pend
           
\pstart
           \textcolor{gray}{\textbf{\textbf{PROF. DR. \textcolor{blue}{I. SINGER}{}\ledrightnote{\textcolor{blue}{Isidor Singer}}}}}\pend
           
\pstart
           \textcolor{gray}{\textbf{\textbf{DR. \textcolor{blue}{HEINRICH KANNER}{}\ledrightnote{\textcolor{blue}{Heinrich Kanner}}}}}\pend
           
\pstart
           \textcolor{gray}{\textbf{\textbf{\so{REDACTION:}}}}\pend
           
\pstart
           \textcolor{gray}{\textbf{\textcolor{pink}{I/\textsubscript{21},
                           WIPPLINGERSTRASSE 38}{}\ledrightnote{\textcolor{pink}{Wipplingerstraße}}}}\pend
           
\pstart
           Lieber – telefonisch konnte ich Sie nicht mehr erreichen, als heute{ }Mittag Ihr Brief kam. Das Ganze ist selbstverständlich ein Irrthum. D\textsuperscript{r}{ }\textcolor{blue}{Kanner}{}\ledrightnote{\textcolor{blue}{Heinrich Kanner}} acceptirte \label{K_L03333-1v}\edtext{s. Z.}{\lemma{\textnormal{\emph{s. Z.}}}\Cendnote{\textnormal{seiner
                  Zeit}}}\label{K_L03333-1h} Ihre Honorarforderung sofort u. willig und hat nur vergessen die Su{\geminationm}e dem Prof. \textcolor{blue}{Singer}{}\ledrightnote{\textcolor{blue}{Isidor Singer}}, der die Caße führt, bekannt zu geben. Dieser wieder dachte bei
               Absendung des Honorares nicht an ein besonderes Übereinkommen und hat auch nicht
               danach gefragt. In dem jetzt herrschenden Arbeits-Trubel hat ein derartiger Irrthum
               wol nichts \substVorne{}\textsuperscript{v}\substDazwischen{}V\substHinten{}erletzendes an sich und darf wol als entschuldbar gelten. Die
               fehlenden 120 Kronen gehen natürlich gleich an Sie ab. Ich hoffe, Sie nehmen diesen
               Zwischenfall nicht zum Anlaß, mich mit der \label{K_L03333-2v}\edtext{\textcolor{green}{Novelle}{}\ledrightnote{{$\rightarrow$}\textcolor{green}{Die griechische Tänzerin. Novellette}}}{\lemma{\textnormal{\emph{Novelle}}}\Cendnote{\textnormal{\textcolor{blue}{Arthur Schnitzler}: \emph{\textcolor{green}{Die griechische Tänzerin}}. In: \emph{\textcolor{green}{Die Zeit}}, Jg. 1, Nr. 2, 28. 9. 1902, Morgenblatt, Beilage: Sonntags-Zeit,
                  S. 4–7.}}}\label{K_L03333-2h} sitzen zu laßen, und hoffe weiter, Sie haben das \textcolor{green}{Mscpt}{}\ledrightnote{{$\rightarrow$}\textcolor{green}{Die griechische Tänzerin. Novellette}}, wie besprochen, auf Ihre
                  \label{K_L03333-3v}\edtext{Reise}{\lemma{\textnormal{\emph{Reise}}}\Cendnote{\textnormal{\textcolor{blue}{Schnitzler} war von 2. 9. 1902 bis 7. 9. 1902 mit dem
                  Fahrrad in \textcolor{pink}{Niederösterreich} und der \textcolor{pink}{Steiermark} unterwegs.}}}\label{K_L03333-3h} mitgenommen, denn
               es wäre mir doch äußerst unangenehm, wenn Sie, ohne weitere Aufklärung abzuwarten
               (die ja auch durch telef. Anrufen sofort zu erhalten war) die Sache beiseite gelegt
               hätten. Mir ist der Vorfall doppelt unangenehm, weil er mit einem anderen fast auf
               die Stunde zusammentrifft, und ich jetzt mit dem von mir angeworbenen Mitarbeitern
               ziemlich beschämt dastehe.\pend
           
\pstart
           herzlichst Ihr {\\[\baselineskip]}\spacefill\mbox{Salten.}\pend
           \leftskip=0em{}\endnumbering\briefempfaengerindex{Schnitzler, Arthur@\textsc{Schnitzler, Arthur}!zzzSalten, Felix@\emph{von Felix Salten}!1902-09-022@{2. 9. 1902}|)be}\mylabel{h}  \normalsize

\doendnotes{C}
\bigskip
\vfill

\clearpage

\footnotesize

\lohead{\textsc{register}}

% Definiere theindex-Environment komplett neu ohne reledmac
\makeatletter
\renewenvironment{theindex}{%
  \section*{\indexname}%
  \setlength{\parindent}{0pt}%
  \setlength{\parskip}{0pt plus 0.3pt}%
  \let\item\@idxitem
}{%
  \clearpage
}
\makeatother

\IfFileExists{\jobname-pw.ind}{\input{\jobname-pw.ind}}{}

\end{document}

      