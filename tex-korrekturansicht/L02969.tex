%% latex-korrekturansicht-vorspann.tex
%% Vorspann für die Korrekturansicht.
%% Lädt die gemeinsame Datei latex-vorspann.tex mit gesetztem Schalter.

\newif\ifkorrekturansicht
\korrekturansichttrue

\input{../tex-inputs/latex-vorspann}


\renewcommand{\erwaehntePersonen}{Personen: Anton Bettelheim, Eduard Devrient, Paul Goldmann, Felix Salten}
\renewcommand{\erwaehnteInstitutionen}{Institutionen: Jung-Wiener Theater zum Lieben Augustin, Überbrettl}
\renewcommand{\erwaehnteOrte}{Orte: Berlin, Bozen, Lago di Garda, Pustertal, Pörtschach, Trient, Vahrn, Venedig, Verona, Welsberg-Taisten, Wien, Wildbad Waldbrunn}
\renewcommand{\erwaehnteWerke}{Werke: Allgemeine Zeitung, Der einsame Weg. Schauspiel in fünf Akten, Die Frau mit dem Dolche, Die Gedenktafel der Prinzessin Anna, Die Insel. Monatsschrift mit Buchschmuck und Illustrationen, Lebendige Stunden, Literatur, Zum Säkulartag Eduard Devrients}
\section[ Arthur Schnitzler an Felix Salten, 10. 8. 1901]{Arthur Schnitzler an Felix Salten, 10. 8. 1901}
\nopagebreak\mylabel{v}
\rehead{ }\normalsize\beginnumbering\briefempfaengerindex{Salten, Felix@\textsc{Salten, Felix}!zzzSchnitzler, Arthur@\emph{von Arthur Schnitzler}!1901-08-103@{10. 8. 1901}|(be}
\toendnotes[C]{\smallbreak\pagebreak[2]}\Standort{Wienbibliothek im Rathaus, ZPH 1681, 2.1.516.}
\physDesc{Brief, 1 Blatt, 4 Seiten, 919 Zeichen
\newline{}Handschrift: Bleistift, deutsche Kurrent
\newline{}Ordnung: mit Bleistift von unbekannter Hand Nummerierung der Doppelseiten des
                                 Konvoluts: »24«–»25« }\toendnotes[C]{\smallbreak}
\pstart
           \raggedleft{}{\pb}\textsc{\textcolor{pink}{Vahrn}{}\ledrightnote{\textcolor{pink}{Vahrn}}}, 10. 8. 901\pend
           
\pstart
           Mein lieber Freund,{ }heut ſinds 4 Wochen, dſs ich \textcolor{pink}{hier}{}\ledrightnote{{$\rightarrow$}\textcolor{pink}{Vahrn}} bin, habe mich ſehr wohlgefühlt; Montag nach \textcolor{pink}{Bozen}{}\ledrightnote{\textcolor{pink}{Bozen}},
               woſelbſt \textcolor{blue}{Paul Goldma{\geminationn}}{}\ledrightnote{\textcolor{blue}{Paul Goldmann}}, dann \textcolor{pink}{Trient}{}\ledrightnote{\textcolor{pink}{Trient}}, aber wir haben uns nicht zum
                  \textcolor{pink}{Gardaſee}{}\ledrightnote{\textcolor{pink}{Lago di Garda}}, ſondern zu einem ſehr schönen Ort
               im \textcolor{pink}{Puſterthal}{}\ledrightnote{\textcolor{pink}{Pustertal}} entſchloſſen, \textcolor{pink}{Welsberg}{}\ledrightnote{\textcolor{pink}{Welsberg-Taisten}}, \textcolor{pink}{Penſion {\pb}Waldbrunn}{}\ledrightnote{\textcolor{pink}{Wildbad Waldbrunn}}; woſelbſt wir etwa bis
                  Ende Auguſt verbleiben um da{\geminationn} direct \label{K_L02969-1v}\edtext{nach \textcolor{pink}{Wien}{}\ledrightnote{\textcolor{pink}{Wien}} zurückzukehren}{\lemma{\textnormal{\emph{nach Wien zurückzukehren}}}\Cendnote{\textnormal{Nach einem kurzen Aufenthalt in \textcolor{pink}{Pörtschach am Wörthersee} (27. 8. 1901 bis 29. 8. 1901) kehrte \textcolor{blue}{Schnitzler} am 30. 8. 1901 nach \textcolor{pink}{Wien} zurück. Nachweislich sahen sich \textcolor{blue}{Salten} und \textcolor{blue}{Schnitzler} dort am 1. 9. 1901
                  wieder.}}}\label{K_L02969-1h}. So treff’ ich Sie wahrſcheinlich dort noch an, bevor Sie nach \textsc{\textcolor{pink}{Verona}{}\ledrightnote{\textcolor{pink}{Verona}}} oder \textsc{\textcolor{pink}{Venedig}{}\ledrightnote{\textcolor{pink}{Venedig}}} fahren. Wollen Sie mir das \textcolor{green}{\textcolor{green}{Inſelheft}{}\ledrightnote{{$\rightarrow$}\textcolor{green}{Die Insel. Monatsschrift mit Buchschmuck und Illustrationen}}}{}\ledrightnote{{$\rightarrow$}\textcolor{green}{Die Gedenktafel der Prinzessin Anna}} nach \textsc{\textcolor{pink}{Welsberg}{}\ledrightnote{\textcolor{pink}{Welsberg-Taisten}}} ſchicken? wäre Ihnen ſehr dankbar. – Das \label{K_L02969-2v}\edtext{\textcolor{brown}{Brettl}{}\ledrightnote{{$\rightarrow$}\textcolor{brown}{Jung-Wiener Theater zum Lieben Augustin}}}{\lemma{\textnormal{\emph{Brettl}}}\Cendnote{\textnormal{hier als Synonym für ›Kabarett‹. Das \emph{\textcolor{brown}{Jung-Wiener Theater zum Lieben Augustin}} hatte das \textcolor{pink}{Berlin}er \emph{\textcolor{brown}{Überbrettl}} als Vorbild.}}}\label{K_L02969-2h} macht Ihnen natürlich viel
                  Mühe\textcolor{gray}{;} –{ }{\pb}– daſs der Erfolg nicht von \textcolor{pink}{Wien}{}\ledrightnote{\textcolor{pink}{Wien}} beſtritten werden kann, war vom erſten Moment an klar.
               Könnten Sie mir die Nummer der \textcolor{green}{Allg. (Münchner)}{}\ledrightnote{\textcolor{green}{Allgemeine Zeitung}}
               verſchaffen, wo dieſer \label{K_L02969-3v}\edtext{\textcolor{blue}{Bettelheim}{}\ledrightnote{\textcolor{blue}{Anton Bettelheim}} uns \textcolor{green}{beflegelt}{}\ledrightnote{{$\rightarrow$}\textcolor{green}{Zum Säkulartag Eduard Devrients}}}{\lemma{\textnormal{\emph{Bettelheim uns beflegelt}}}\Cendnote{\textnormal{Am Tag des Briefes erschien in der
                  Beilage ein längerer Text über \textcolor{blue}{Eduard
                     Devrient}, der mehrere Seitenhiebe auf populäres Theater enthält. Ob \textcolor{blue}{Schnitzler} davon schon Kenntnis gehabt und
                  sich angesprochen gefühlt hätte, ist zweifelhaft. Vgl. \textcolor{blue}{Anton Bettelheim}: \emph{\textcolor{green}{Zum
                        Säkulartag Eduard Devrients}}. In: \emph{\textcolor{green}{Allgemeine Zeitung}}, Beilage, Nr. 182, 10. 8. 1901, S. 1–6.}}}\label{K_L02969-3h} haben ſoll? –\pend
           
\pstart
           Leben Sie wohl und ſeien Sie herzlich gegrüßt.\pend
           
\pstart
           Das neue \label{K_L02969-4v}\edtext{\textcolor{green}{Stück}{}\ledrightnote{{$\rightarrow$}\textcolor{green}{Der einsame Weg. Schauspiel in fünf Akten}}}{\lemma{\textnormal{\emph{Stück}}}\Cendnote{\textnormal{\emph{\textcolor{green}{Der einsame Weg}}, den \textcolor{blue}{Schnitzler} am 21. 7. 1901 vorläufig abgeschlossen hatte und am
                     20. 11. 1901 neu
                  zu bearbeiten begann}}}\label{K_L02969-4h} iſt doch nicht fertig, ka{\geminationn} es aber bald ſein. {\pb}Dafür \label{K_L02969-5v}\edtext{\textcolor{green}{2 Einakter}{}\ledrightnote{{$\rightarrow$}\textcolor{green}{Lebendige Stunden}{\newline}{$\rightarrow$}\textcolor{green}{Die Frau mit dem Dolche}}}{\lemma{\textnormal{\emph{2 Einakter}}}\Cendnote{\textnormal{\emph{\textcolor{green}{Lebendige Stunden}} (fertiggestellt am 28. 7. 1901) und \emph{\textcolor{green}{Die Frau mit dem Dolche}} (3. 8. 1901)}}}\label{K_L02969-5h},
               die zu »\textcolor{green}{Literatur}{}\ledrightnote{\textcolor{green}{Literatur}}« dazu gegeben werden
               ſollen.\pend
           \pstart Ihr \spacefill\mbox{A.}\pend{}\endnumbering\briefempfaengerindex{Salten, Felix@\textsc{Salten, Felix}!zzzSchnitzler, Arthur@\emph{von Arthur Schnitzler}!1901-08-103@{10. 8. 1901}|)be}\mylabel{h}  \normalsize

\doendnotes{C}
\bigskip
\vfill

\clearpage

\footnotesize

\lohead{\textsc{register}}

% Definiere theindex-Environment komplett neu ohne reledmac
\makeatletter
\renewenvironment{theindex}{%
  \section*{\indexname}%
  \setlength{\parindent}{0pt}%
  \setlength{\parskip}{0pt plus 0.3pt}%
  \let\item\@idxitem
}{%
  \clearpage
}
\makeatother

\IfFileExists{\jobname-pw.ind}{\input{\jobname-pw.ind}}{}

\end{document}

      