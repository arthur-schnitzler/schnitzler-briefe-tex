%% latex-korrekturansicht-vorspann.tex
%% Vorspann für die Korrekturansicht.
%% Lädt die gemeinsame Datei latex-vorspann.tex mit gesetztem Schalter.

\newif\ifkorrekturansicht
\korrekturansichttrue

\input{../tex-inputs/latex-vorspann}


\renewcommand{\erwaehntePersonen}{Personen: Hanns Heinz Ewers, Hugo von Hofmannsthal, Dora Michaelis, Olga Schnitzler, Elisabeth Steinrück, Frida Strindberg, Irene Triesch, Jakob Wassermann}
\renewcommand{\erwaehnteOrte}{Orte: Berlin, Dessauer Straße, Grünentorgasse, Italien, Wien}
\renewcommand{\erwaehnteWerke}{Werke: Tagebuch}
\section[ Paul Goldmann an Arthur Schnitzler, 21. 3. {[}1901{]}]{Paul Goldmann an Arthur Schnitzler, 21. 3. {[}1901{]}}
\nopagebreak\mylabel{v}
\rehead{ }\normalsize\beginnumbering\briefempfaengerindex{Schnitzler, Arthur@\textsc{Schnitzler, Arthur}!zzzGoldmann, Paul@\emph{von Paul Goldmann}!1901-03-211@{21. 3. {[}1901{]}}|(be}
\toendnotes[C]{\smallbreak\pagebreak[2]}\Standort{DLA, A:Schnitzler, HS.NZ85.1.3171.}
\physDesc{Brief, 1 Blatt, 3 Seiten
\newline{}Handschrift: blaue Tinte, deutsche Kurrent
\newline{}Schnitzler: 1) mit Bleistift das Jahr »{[}1{]}901« vermerkt  2) mit rotem Buntstift fünf Unterstreichungen}\toendnotes[C]{\smallbreak}
\pstart
           \noindent{}\raggedleft{}{\pb}\textcolor{pink}{\textcolor{gray}{\textbf{DESSAUERSTRASSE 19}}}{}\ledrightnote{\textcolor{pink}{Dessauer Straße}}\pend
           
\pstart
           \textcolor{pink}{Berlin}{}\ledrightnote{\textcolor{pink}{Berlin}}, 21. März.\pend
           
\pstart\center{}Mein lieber Freund,\pend
\pstart
           Reiſe glücklich! Komm geſund wieder! Und grüße mir das \label{K_L03062-1v}\edtext{\textcolor{pink}{Land der Sehnſucht}{}\ledrightnote{{$\rightarrow$}\textcolor{pink}{Italien}}}{\lemma{\textnormal{\emph{Land der Sehnſucht}}}\Cendnote{\textnormal{Bezug auf \textcolor{blue}{Schnitzler}s \textcolor{pink}{Italien}reise zwischen 26. 3. 1901 und 18. 4. 1901}}}\label{K_L03062-1h}! Ich wollte, ich könnte mit.\pend
           
\pstart
           Hier nichts Neues. Wenn ich nicht irre, hat Frau \textsc{\textcolor{blue}{Frida Strindberg}{}\ledrightnote{\textcolor{blue}{Frida Strindberg}}}{ }\textcolor{pink}{hier}{}\ledrightnote{{$\rightarrow$}\textcolor{pink}{Berlin}} mit dem jungen \textsc{\textcolor{blue}{Hans Heinz Evers}{}\ledrightnote{\textcolor{blue}{Hanns Heinz Ewers}}} ſchleunigſt ein Verhältniß angefangen.\pend
           
\pstart
           Daß die \textsc{\textcolor{blue}{Triesch}{}\ledrightnote{\textcolor{blue}{Irene Triesch}}} im Sommer \label{K_L03062-2v}\edtext{mit uns kommen}{\lemma{\textnormal{\emph{mit uns kommen}}}\Cendnote{\textnormal{Es ist keine gemeinsame Reise im Sommer
                     1901 bekannt.}}}\label{K_L03062-2h} ſoll, iſt mir gar nicht recht.
               Sie hat einfach dekretirt, daß {\pb}ſie
               mitkommen wird, ohne viel zu fragen. Wenn Du willſt, daß ſie kommt, – meinetwegen!
               Einſtweilen kann man immerhin »Ja« ſagen. Im letzten Moment gibt es Ausreden
               genug.\pend
           
\pstart
           Grüße an die \textcolor{blue}{\textcolor{pink}{Grünethorgaſſe}{}\ledrightnote{\textcolor{pink}{Grünentorgasse}}}{}\ledrightnote{{$\rightarrow$}\textcolor{blue}{Elisabeth Steinrück}{\newline}{$\rightarrow$}\textcolor{blue}{Olga Schnitzler}}! Ich ſchreibe nächſtens an dieſe Adreſſe. Habe einſtweilen wenig Zeit.\pend
           
\pstart
           Darum auch für Dich nur dieſe eiligen Zeilen. Ich {\pb}drücke Dir herzlichſt die Hand.
               {\\[\baselineskip]}Dein {\\[\baselineskip]}\spacefill\mbox{Paul Goldmann}\pend
           \leftskip=0em{}
\pstart
           \noindent{}\textsc{\textcolor{blue}{Dora Speyer}{}\ledrightnote{\textcolor{blue}{Dora Michaelis}}} kennen gelernt. Iſt \label{K_L03062-3v}\edtext{noch
                  immer ſehr in Dich verliebt}{\lemma{\textnormal{\emph{noch … verliebt}}}\Cendnote{\textnormal{vgl. \textcolor{blue}{Schnitzler}s \emph{\textcolor{green}{Tagebuch}} ab dem 28. 2. 1900}}}\label{K_L03062-3h}. Mein Herz \strikeout{zu} hat ſie zu gewinnen
                  verſucht, indem ſie von \textsc{\textcolor{blue}{Hoffmannsthal}{}\ledrightnote{\textcolor{blue}{Hugo von Hofmannsthal}}} und \textsc{\textcolor{blue}{Wassermann}{}\ledrightnote{\textcolor{blue}{Jakob Wassermann}}} ſchwärmte. Das iſt nicht ganz der richtige Weg.\pend
           \endnumbering\briefempfaengerindex{Schnitzler, Arthur@\textsc{Schnitzler, Arthur}!zzzGoldmann, Paul@\emph{von Paul Goldmann}!1901-03-211@{21. 3. {[}1901{]}}|)be}\mylabel{h}
\begin{anhang}
\end{anhang}\normalsize

\doendnotes{C}
\bigskip
\vfill

\clearpage

\footnotesize

\lohead{\textsc{register}}

% Definiere theindex-Environment komplett neu ohne reledmac
\makeatletter
\renewenvironment{theindex}{%
  \section*{\indexname}%
  \setlength{\parindent}{0pt}%
  \setlength{\parskip}{0pt plus 0.3pt}%
  \let\item\@idxitem
}{%
  \clearpage
}
\makeatother

\IfFileExists{\jobname-pw.ind}{\input{\jobname-pw.ind}}{}

\end{document}

      