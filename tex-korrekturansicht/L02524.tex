%% latex-korrekturansicht-vorspann.tex
%% Vorspann für die Korrekturansicht.
%% Lädt die gemeinsame Datei latex-vorspann.tex mit gesetztem Schalter.

\newif\ifkorrekturansicht
\korrekturansichttrue

\input{../tex-inputs/latex-vorspann}


               \section[Thomas Mann an Arthur Schnitzler, {[}nach dem 18. 11. 1929{]}]{ Thomas Mann an Arthur Schnitzler, {[}nach dem 18. 11. 1929{]}}\nopagebreak\mylabel{v}\rehead{ }\normalsize\beginnumbering\briefempfaengerindex{Schnitzler, Arthur@\textsc{Schnitzler, Arthur}!zzzMann, Thomas@\emph{von Thomas Mann}!1929-11-191@{{[}nach dem 18. 11. 1929{]}}|(be} \toendnotes[C]{\smallbreak\pagebreak[2]} \Standort{CUL, Schnitzler, B 67.}
\physDesc{Briefkarte
\newline{}Handschrift: schwarze Tinte, deutsche Kurrent (\noindent{}Unterschrift und Nachschrift)}\buchAbdrucke{\weitereDrucke{Hertha Krotkoff: \emph{Arthur Schnitzler – Thomas Mann: Briefe.} In: \emph{Modern Austrian Literature}, Jg. 7 (1974) Nr. 1/2, S. 26.} }\toendnotes[C]{\smallbreak}\pstart
           \noindent{}{\pb}\textcolor{gray}{\textbf{Dr. Thomas Mann}}\hfill \textcolor{gray}{\textbf{\textcolor{pink}{München}{}\ledrightnote{\textcolor{pink}{München}}, den \label{K_L02524_1v}\edtext{14. November 1929}{\lemma{\textnormal{\emph{14. November 1929}}}\Cendnote{\textnormal{Entgegen der gedruckten
                                Datierung ist anzunehmen, dass \textcolor{blue}{Mann} auf die Gratulation vom 18. 11. 1929 antwortet.}}}\label{K_L02524_1h}}}\pend
           \pstart
           \centering{}\textcolor{gray}{\textbf{Herzlich danke ich für die mir anläßlich der Verleihung des
                            \textcolor{brown}{Nobelpreises}{}\ledrightnote{\textcolor{brown}{Nobelpreis}} gewidmeten
                        Glückwünsche.}}\pend
           \pstart \spacefill\mbox{Thomas Mann}\pend{}\pstart
           \noindent{}In treuer Verehrung!\pend
           \endnumbering\briefempfaengerindex{Schnitzler, Arthur@\textsc{Schnitzler, Arthur}!zzzMann, Thomas@\emph{von Thomas Mann}!1929-11-191@{{[}nach dem 18. 11. 1929{]}}|)be}\mylabel{h}  \normalsize

\doendnotes{C}
\bigskip
\vfill

\clearpage

\footnotesize

\lohead{\textsc{register}}

% Definiere theindex-Environment komplett neu ohne reledmac
\makeatletter
\renewenvironment{theindex}{%
  \section*{\indexname}%
  \setlength{\parindent}{0pt}%
  \setlength{\parskip}{0pt plus 0.3pt}%
  \let\item\@idxitem
}{%
  \clearpage
}
\makeatother

\IfFileExists{\jobname-pw.ind}{\input{\jobname-pw.ind}}{}

\end{document}

      